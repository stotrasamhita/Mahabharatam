\part{शल्यपर्व}
\chapter{अध्यायः १}
% Check verse!
श्रीवेदव्यासाय नमः
\threelineshloka
{नारायणं नमस्कृत्य नरं चैव नरोत्तमम्}
{देवीं सरस्वतीं (व्यासं) चैव ततो जयमुदीरयेत् ॥जनमेजय उवाच}
{}


\twolineshloka
{एवं निपातिते कर्णे समरे सव्यसाचिना}
{अल्पावशिष्टाः कुरवः किमकुर्वत वै द्विज}


\twolineshloka
{विदीर्यमाणं च बलं दृष्ट्वा राजा सुयोधनः}
{पाण्डवैः प्राप्तकालं च किं प्रापद्यत कौरवः}


\threelineshloka
{एतदिच्छाम्यहं श्रोतुं तदाचक्ष्व द्विजोत्तम}
{न हि तृप्यामि पूर्वेषां शृण्वानश्चरितं महत् ॥वैशम्पायन उवाच}
{}


\twolineshloka
{ततः कर्णे हते राजन्धार्तराष्ट्रः सुयोधनः}
{भृशं शोकार्णवे मग्नो निराशः सर्वतोऽभवत्}


\twolineshloka
{हाकर्ण हाकर्ण इति शोचमानो मुहुर्मुहुः}
{कृच्छ्रात्स्वशिबिरं प्रायाद्वतशिष्टैर्नृपैः सह}


\twolineshloka
{स समाश्वास्यमानोऽपि हेतुभिः शास्त्रनिश्चितैः}
{राजन्विभूतिमिच्छद्भिः सूतपुत्रमनुस्मरन्}


\twolineshloka
{स दैवं बलवन्मत्वा प्रभाते विमले सति}
{सङ्ग्रामे निश्चयं कृत्वा पुनर्युद्वाय निर्ययौ}


\twolineshloka
{शल्यं सेनापतिं कृत्वा विधिवद्राजसत्तमम्}
{रणाय निर्ययौ राजा हतशिष्टैर्नृपैः सह}


\twolineshloka
{ततः सुतुमुलं युद्धं कुरुपाण्डवसेनयोः}
{बभूव भरतश्रेष्ठ देवासुररणोपमम्}


\twolineshloka
{ततः शल्यो महाराज कृत्वा कदनमाहवे}
{पाण्डुसैन्येऽथ मध्याह्ने धर्मराजेन पातितः}


\twolineshloka
{ततो दुर्योधनो राजा हतबन्धू रणाजिरात्}
{अपसृत्य हदं घोरं विवेश रिपुजाद्भुयात्}


\twolineshloka
{अथापराह्णे तस्याह्नः परिवार्य महारथैः}
{हदादाहूय युद्वाय भीमसेनेन पातितः}


\twolineshloka
{तस्मिंस्तु निहते वीरे महेष्वासास्त्रयो रणे}
{कृतवर्मा कृपो द्रौणिर्जघ्नुः पाण्डवसैनिकान्}


\twolineshloka
{ततः पूर्वाह्णसमये शिबिरादेत्य सञ्जयः}
{प्रविवेश पुरीं दीनो दुःखशोकसमन्वितः}


\twolineshloka
{स प्रविश्य पुरीं सूतो भूजावुच्छ्रित्य दुःखितः}
{वेपमानस्ततो राज्ञः प्रविप्रेश निवेशनम्}


\twolineshloka
{धावतश्चाप्यपश्यxxx तत्रत्यान्पुरुषर्षभान्}
{नष्टचित्तानिवोन्मत्ताञ्शोकेन भृशदुःखितान्}


\twolineshloka
{दृष्ट्वैव च नराञ्शीघ्रं व्याजहारातिदुःखितः}
{अहो बत विपन्नोऽस्मि निधनेन महात्मनः}


\twolineshloka
{अहो सुबलवान्कालो गतिश्च परमा तथा}
{शुक्रतुल्यबलाः सर्वे यत्राहन्यन्त पार्थिवाः}


\twolineshloka
{तं दृष्ट्वैव पुरे राजञ्जनः सर्वः स्म सञ्जयम्}
{प्ररुरोद भयोद्विग्नो हा राजन्निति सुस्वरम्}


\twolineshloka
{आकुमारं नरव्याघ्र तत्पुरं वै समन्ततः}
{आर्तनादं महच्चक्रे श्रुत्वा विनिहतं नृपम्}


\twolineshloka
{तथा स विह्वलः सूतः प्रविश्य नृपतिक्षयम्}
{ददर्श नृपतिश्रेष्ठं प्रज्ञाचक्षुषमीश्वरम्}


\threelineshloka
{दृष्ट्वा चासीनमनघं समन्तात्परिवारितम्}
{स्नुषामिर्भरतश्रेष्ठ गान्धार्या विदुरेण च}
{तथाऽन्यैश्च सुहृद्भिश्च ज्ञातिमिश्च हितैषिभिः}


\twolineshloka
{तमेव चार्थं ध्यायन्तं कर्णस्य निधनं प्रति}
{रुदन्नेवाब्रवीद्वाक्यं राजानं जनमेजय}


\twolineshloka
{नातिहृष्टमनाः सूतो बाष्पसन्दिग्धया गिरा}
{सञ्जयोऽहं नरव्याघ्र नमस्ते भरतर्षभ}


\twolineshloka
{मद्राधिपो हतः शल्यः शकुनिः सौबलस्तथा}
{उलूकः पुरुषव्याघ्र कैतव्यो दृढविक्रमः}


\twolineshloka
{संशप्तका हताः सर्वे काम्भोजाश्च शकैः सह}
{म्लेच्छाश्च पार्वतीयाश्च यवनाश्च निपातिताः}


\threelineshloka
{प्राच्या हता महाराज दाक्षिणात्याश्च सर्वशः}
{उदीच्याश्च हताः सर्वे प्रतीच्याश्च नरोत्तमाः}
{राजानो राजपुत्राश्च सर्वे विनिहता नृप}


% Check verse!
कर्णपुत्रो हतः शूरः सत्यसेनो महाबलः
\twolineshloka
{दुर्योधनो हतो राजा ययोक्तं पाण्डवेन ह}
{xxxxxx महाराज शेते पांसुषु रूषितः}


\twolineshloka
{xxxxxx हतो xxx जञ्शिखण्डी चापराजितः}
{उत्तमौज युधामन्युस्तथा सर्वे प्रभद्रकाः}


\twolineshloka
{पाञ्चालश्च नरव्याघ्र चेदयश्च निषूदिताः}
{तव पुत्रा हताः सर्वे द्रौपदेयाश्च भारत}


\twolineshloka
{नरा विनिहताः सर्वे गजाश्च विनिपातिताः}
{रथिनत्र नरव्याघ्र हयाश्च निहता युधि}


\twolineshloka
{निxxxx शिबिरं रावंजावकानां कृतं प्रभो}
{पाण्डवानां च शूराणां समासाद्य परस्परम्}


\twolineshloka
{xxxx स्रीशेषमभवज्जगत्कालेन मोहितम्}
{सप्त पाण़्डवतः शिष्टा धार्तराष्ट्रास्त्रयोरथाः}


\twolineshloka
{ते चैव भ्रातरः पञ्च वासुदेवोऽथ सात्यकिः}
{कृपश्च कृतवर्मा च द्रौणिश्च जयतां वरः}


\twolineshloka
{एते शेषा महाराज रथिनो नृपसत्तम}
{अक्षौहिणीनामष्टानां दशानां च न संशयः}


\fourlineindentedshloka
{एते शेषा महाराज सर्वेऽन्ये निधनं गताः}
{कालेन निहतं सर्वं जगद्वै भरतर्षभ}
{दुर्योधनं वै पुरतः कृत्वा सर्वे नरा हताः ॥वैशम्पायन उवाच}
{}


\twolineshloka
{एतच्छ्रुत्वा वचः क्रूरं धृतराष्ट्रो जनेश्वरः}
{निपपात स राजेन्द्रो गतसत्वो महीतले}


\twolineshloka
{तस्मिन्निपतिते वीरे विदुरोऽपि महायशाः}
{निपपात महाराज शोकव्यसनकर्शितः}


\twolineshloka
{गान्धारी च महाभागा सर्वाश्च कुरुयोषितः}
{पतिताः सहसा भूमौ श्रुत्वा घोरतरं वचः}


\twolineshloka
{निःसंज्ञं पतितं भूमौ तदाऽऽसीद्राजमण्डलम्}
{विलापमुक्तोपहतं चित्रं न्यस्तं पटे यथा}


\twolineshloka
{कृच्छ्रेण तु ततो राजा धृतराष्ट्रो महीपतिः}
{शनैरलभत प्राणान्पुत्रव्यसनकर्शितः}


\twolineshloka
{लब्ध्वा तु स नृपः प्राणान्वेपमानः सुदुःखितः}
{निरीक्ष्य च दिशः सर्वाः क्षत्तारं वाक्यमब्रवीत्}


\threelineshloka
{विद्धि क्षत्तर्महाप्राज्ञ त्वं गतिर्भरतर्षभ}
{ममानाथस्य सुभृशं पुत्रैर्हीनस्य सर्वशः}
{एवमुक्त्वा ततो भूयो विसंज्ञो निपपात ह}


\twolineshloka
{तं तथा पतितं दृष्ट्वा बान्धवा येऽस्य केचन}
{शीतैस्ते सिषिचुस्तोयैर्विव्यजुर्व्यजनैरपि}


\twolineshloka
{स तु दीर्घेण कालेन प्रत्याश्वस्तो नराधिपः}
{तूष्णीमासीन्महीपालः पुत्रव्यसनकर्शितः}


% Check verse!
निःश्वसञ्जिह्मग इव कुम्भक्षिप्तोऽभवन्नृपः
\twolineshloka
{सञ्जयोऽप्यरुदत्तत्र दृष्ट्वा राजानमातुरम्}
{तथा सर्वाः स्त्रियश्चैव गान्धारी च यशस्विनी}


\twolineshloka
{ततो दीर्घेण कालेन विदुरं वाक्यमब्रवीत्}
{धृतराष्ट्रो नरश्रेष्ठ मुह्यमानो मुहुर्मुहुः}


\twolineshloka
{गच्छन्तु योषितः सर्वा गान्धारी च यशस्विनी}
{तथेमे सुहृदः सर्वे मुह्यते मे मनो भृशम्}


\twolineshloka
{एवमुक्तस्ततः क्षत्ता ताः स्त्रियो भरतर्षभ}
{विसर्जयामास शनैर्मुह्यमानः पुनःपुनः}


\twolineshloka
{निश्चक्रमुस्ततः सर्वाः स्त्रियो भरतसत्तम}
{सुहृदश्च तथा सर्वे दृष्ट्वा राजानमातुरम्}


\threelineshloka
{ततो नरपतिस्तत्र लब्ध्वा संज्ञां परन्तपः}
{गतिर्मे को भवेदद्य इति चिन्तासमाकुलः}
{अपृच्छत्सञ्जयं तत्र रोदमानं भृशातुरम्}


\twolineshloka
{प्राञ्जलिं सञ्जयं दृष्ट्वा रोदमानं मुहुर्मुहुः}
{ज्ञातीन्स्त्रियोऽथ निर्याप्य प्रविश्य विदुरःपुनः}


\twolineshloka
{राजानं शोचमानस्तु तं शोचन्तं मुहुर्मुहुः}
{समाश्वासयत क्षत्ता वचसा मधुरेण च}


\chapter{अध्यायः २}
\twolineshloka
{वैशम्पायन उवाच}
{}


\twolineshloka
{विसृष्टास्वथ नारीषु धृतराष्ट्रोऽम्बिकासुतः}
{विललाप महाराज दुःखाद्दुःखतरं गतः}


\threelineshloka
{सधूममिव निःश्वस्य करौ धुन्वन्पुनःपुनः}
{बहु सञ्चिन्तयित्वा तु सञ्जयं वाक्यमब्रवीत् ॥धृतराष्ट्र उवाच}
{}


\twolineshloka
{अहो बत महद्दुःखं यदहं पाण्डवान्रणे}
{क्षेमिणश्चाव्ययांश्चैव त्वत्तः सूत शृणोमि वै}


\twolineshloka
{वज्रसारमयं नूनं हृदयं सुदृढं मम}
{यच्छ्रुत्वा निहतान्पुत्रान्दीर्यते न सहस्रधा}


\twolineshloka
{चिन्तयित्वा वचस्तेषां बालक्रीडां च सञ्जय}
{अद्य चैव हताञ्श्रुत्वा दीर्यते मे भृशं मनः}


\twolineshloka
{अन्धत्वाद्यदि पुत्राणां न मे रूपिनिदर्शनम्}
{पुत्रस्नेहकृता प्रीतिर्नित्यमेतेषु धारिता}


\twolineshloka
{बालभावमतिक्रम्य यौवनस्थांश्च तानहम्}
{श्रियं प्राप्तांश्च ताञ्श्रुत्वा हृष्ट आसं तदाऽनघ}


\twolineshloka
{तानद्य निहताञ्श्रुत्वा हतैश्वर्यान्हतौजसः}
{न लभेयं क्वचिच्छान्तिं पुत्राधिभिरभिप्लुतः}


\twolineshloka
{एह्येहि वत्स राजेन्द्र ममानाथस्य पुत्रक}
{त्वया हीनो महाबाहो कां नु यास्याम्यहं गतिम्}


\twolineshloka
{कथं त्वं पृथिवीपालांस्त्यक्त्वा तात समागतान्}
{शेषे विनिहतो भूमौ प्राकृतः कुनृपो यथा}


\twolineshloka
{गतिर्भूत्वा महाराज ज्ञातीनां सुहृदां तथा}
{अन्धं वृद्वं च मां वीर विहाय क्व नु यास्यसि}


\twolineshloka
{सा कृपा सा च ते प्रीतिः सा च राजसु मानिता}
{कथं त्वं निहतः पार्थैः संयुगेष्वपराजितः}


\twolineshloka
{को नु मामुत्थितः काले ताततातेति वक्ष्यति}
{महाराजेति सततं लोकनाथेति चासकृत्}


\twolineshloka
{परिष्वज्य च कं कण्ठे स्नेहेन क्लिन्नलोचनः}
{अनुशास्ताऽस्मि कौरव्य तत्साधु वदमे वचः}


\twolineshloka
{ननु नामाहमश्रौषं वचनं तव पुत्रक}
{भूयसी मम पृथ्वीयं तात पार्थस्य नो तथा}


\twolineshloka
{भगदत्तः कृपः शल्य आवन्त्योऽथ जयद्रथः}
{भूरिश्रवाः सोमदत्तो महाराजश्च बाह्लिकः}


\twolineshloka
{अश्वत्थामा च भोजश्च मागधश्च महाबलः}
{बृहद्बलश्च क्राथश्च शकुनिश्चापि सौबलः}


\twolineshloka
{म्लेच्छाश्च शतसाहस्राः शकाश्च यवनैः सह}
{सुदक्षिणश्च काम्भोजस्त्रिगर्ताधिपतिस्तथा}


\twolineshloka
{भीष्मः पितामहश्चैव भारद्वाजोऽथ गौतमः}
{श्रुतायुश्चाश्रुतायुश्च शतायुश्चापि वीर्यवान्}


\twolineshloka
{जलसन्धोऽथार्ष्यशृङ्गी राक्षसश्चाप्यलायुधः}
{अलम्बुसो वीरबाहुः सुबाहुश्च महारथः}


\twolineshloka
{एते चान्ये च बहवो राजानो राजसत्तम}
{मदर्थं प्रहरिष्यन्ति प्राणांस्त्यक्त्वा धनानि च}


\twolineshloka
{तेषां मध्ये स्थितो युद्धे भ्रातृभिः परिवारितः}
{योधयिष्याम्यहं पार्थान्पाञ्चालांश्चैव सर्वशः}


\twolineshloka
{चेदींश्च नृपशार्दूल द्रौपदेयांश्च संयुगे}
{सात्यकिं कुन्तिभोजं च राक्षसं च घटोत्कचम्}


\twolineshloka
{एकोऽप्येषां महाराज समर्थः सन्निवारणे}
{समरे पाण्डवेयानां सङ्क्रुद्धो ह्यभिधावताम्}


\twolineshloka
{किं पुनः सहिता वीराः कृतवैराश्च पाण्डवैः}
{अथवा सर्व एवैते पाण्डवस्यानुयायिभिः}


\twolineshloka
{योत्स्यन्ते सह राजेन्द्र हनिष्यन्ति च तान्मृधे}
{कर्ण एको मया सार्धं निहनिष्यति पाण्डवान्}


% Check verse!
ते वै नृपतयो वीराः स्थास्यन्ति मम शासने
\twolineshloka
{यश्च तेषां प्रणेता वै वासुदेवो महाबलः}
{न स सन्नह्यते राजन्निति मामब्रवीद्वचः}


\twolineshloka
{एवं च वदतः सूत बहुशो मम सन्निधौ}
{युक्तितो ह्यनुपश्यामि निहतान्पाण़्डवान्रणे}


\twolineshloka
{तेषां मध्ये स्थिता यत्र हन्यन्ते मम पुत्रकाः}
{व्यायच्छमानाः समरे किमन्यद्भागधेयतः}


\twolineshloka
{भीष्मश्च निहतो यत्र लोकनाथः प्रतापवान्}
{शिखण़्डिनं समासाद्य मृगेन्द्र इव जम्बुकम्}


\twolineshloka
{द्रोणश्च ब्राह्मणो यत्र सर्वशस्त्रास्त्रपारगः}
{निहतः पाण्डवैः सङ्ख्ये किमन्यद्भागधेयतः}


\threelineshloka
{कर्णश्च निहतः सङ्ख्ये दिव्यास्त्रज्ञो महाबलः}
{भूरिश्रवा हतो पत्र सोमदत्तश्च संयुगे}
{बाह्लिकश्च महाराज किमन्यद्भागधेयतः}


\twolineshloka
{भगदत्तो हतो यत्र गजयुद्धविशारदः}
{जयद्रथश्च निहतः किमन्यद्भागधेयतः}


\twolineshloka
{सुदक्षिणो हतो यत्र जलसन्धश्च पौरवः}
{श्रुतायुश्चाश्रुतायुश्च किमन्यद्भागधेयतः}


\twolineshloka
{महाबलस्तथा पाण्ड्यः सर्वशस्त्रभृतां वरः}
{निहतः पाण्डवैः सङ्ख्ये किमन्यद्भागधेयतः}


\twolineshloka
{बृहद्बलो हतो यत्र मागधश्च महाबलः}
{उग्रायुधश्च विक्रान्तः प्रतिमानं धनुष्मताम्}


\twolineshloka
{आवन्त्यो निहतो यत्र त्रैगर्तश्च जनाधिपः}
{संशप्तकाश्च निहताः किमन्यद्भागधेयतः}


\twolineshloka
{अलम्बुसस्तथा राजन्राक्षसश्चाप्यलायुधः}
{आर्ष्यशृङ्गिश्च निहतः किमन्यद्भागधेयतः}


\twolineshloka
{नारायणा हता यत्र गोपाला युद्धदुर्मदाः}
{म्लेच्छाश्च बहुसाहस्राः किमन्यद्भागधेयतः}


\twolineshloka
{शकुनिः सौबलो यत्र कैतव्यश्च महाबलः}
{निहतः सबलो वीरः किमन्यद्भागधेयतः}


\threelineshloka
{एते चान्ये च बहवः कृतास्त्रा युद्धदुर्मदाः}
{राजानो राजपुत्राश्च शूराः परिघबाहवः}
{निहता बहवो यत्र किमन्यद्भागधेयतः}


\twolineshloka
{यत्र शूरा महेष्वासाः कृतास्त्रा युद्वदुर्मदाः}
{बहवो निहताः सूत महेन्द्रसमविक्रमाः}


\twolineshloka
{नानादेशसमावृत्ताः क्षत्रिया यत्र सञ्जय}
{निहताः समरे सर्वे किमन्यद्भागधेयतः}


\twolineshloka
{पुत्राश्च मे विनिहताः पौत्राश्चैव महाबलाः}
{वयस्या भ्रातरश्चैव किमन्यद्भागधेयतः}


\twolineshloka
{भागधेयसमायुक्तो ध्रुवमुत्पद्यते नरः}
{यस्तु भाग्यसमायुक्तः स शुभं प्राप्नुयान्नरः}


\twolineshloka
{अहं वियुक्तस्तैर्भाग्यैः पुत्रैश्चैवेह सञ्जय}
{कथमद्य भविष्यामि वृद्धः शत्रुवशं गतः}


\twolineshloka
{नान्यदत्र परं मन्ये वनवासादृते प्रभो}
{सोऽहं वनं गमिष्यामि निर्बन्धुर्ज्ञातिसंक्षये}


\twolineshloka
{न हि मेऽन्यद्भवेच्छ्रेयो वनाभ्युपगमादृते}
{इमामवस्थां प्राप्तस्य लूनपक्षस्य सञ्जय}


\twolineshloka
{दुर्योधनो हतो यत्र शल्यश्च निहतो युधि}
{दुःशासनो विविंशश्च विकर्णश्च महाबलः}


\twolineshloka
{कथं हि भीमसेनस्य श्रोष्येऽहं शब्दमुत्तमम्}
{एकेन समरे येन हतं पुत्रशतं मम}


\threelineshloka
{असकृद्वदतस्तस्य दुर्योधनवधेन च}
{दुःखशोकाभिसन्तप्तो न श्रोष्ये परुषा गिरः ॥वैशम्पायन उवाच}
{}


\twolineshloka
{एवं स शोकसन्तप्तः पार्थिवो हतबान्धवः}
{मुहुर्मुहुर्मुह्यमानः पुत्राधिभिरभिप्लुतः}


\twolineshloka
{विलप्य सुचिरं कालं धृतराष्ट्रोऽम्बिकासुतः}
{दीर्घमुष्णं स निःश्वस्य चिन्तयित्वा पराभवम्}


\threelineshloka
{दुःखेन महता राजन्सन्तप्तो भरतर्षभः}
{पुनर्गावल्गणिं सूतं पर्यपृच्छद्यथातथम् ॥धृतराष्ट्र उवाच}
{}


\twolineshloka
{भीष्मद्रोणौ हतौ श्रुत्वा सूतपुत्रं च पातितम्}
{सेनापतिं प्रणेतारं कमकुर्वत मामकाः}


\twolineshloka
{यं यं सेनाप्रणेतारं युधि कुर्वन्ति मामकाः}
{अचिरेणैव कालेन तं तं निघ्नन्ति पाण्डवाः}


\twolineshloka
{रणमूर्ध्नि हतो भीष्मः पश्यतां वः किरीटिना}
{एवमेव हतो द्रोणः सर्वेषामेव पश्यताम्}


\twolineshloka
{एवमेव हतः कर्णः सूतपुत्रः प्रतापवान्}
{सराजकानां सर्वेषां पश्यतां वः किरीटिना}


\twolineshloka
{पूर्वमेवाहमुक्तो वै विदुरेण महात्मना}
{दुर्योधनापराधेन प्रजेयं विनशिष्यति}


\twolineshloka
{केचिन्न सम्यक्पश्यन्ति मूढाः सम्यगवेक्ष्य च}
{तदिदं मम मूढस्य तथाभूतं वचः स्म तत्}


\twolineshloka
{यदब्रवीत्स धर्मात्मा विदुरो दीर्घदर्शिवान्}
{तत्तथा समनुप्राप्तं वचनं सत्यवादिनाः}


\twolineshloka
{दैवोपहतचित्तेन यन्मयाऽनुष्ठितं पुरा}
{अनयस्य फलं तस्य ब्रूहि गावल्गणे पुनः}


\twolineshloka
{को वा मुखमनीकानामासीत्कर्णो निपातिते}
{अर्जुनं वासुदेवं च को वा प्रत्युद्ययौ रथी}


\twolineshloka
{केऽरक्षन्दक्षिणं चक्रं मद्रराजस्य संयुगे}
{वामं च योद्वुकामस्य के वा वीरस्य पृष्ठतः}


\twolineshloka
{कथं च वः समेतानां मद्रराजो महारथः}
{निहतः पाण्डवैः सङ्ख्ये पुत्रो वा मम सञ्जय}


\twolineshloka
{ब्रूहि सर्वं यथातत्त्वं भरतानां महाक्षयम्}
{यथा च निहतः सङ्ख्ये पुत्रो दुर्योधनो मम}


\twolineshloka
{पाञ्चालाश्च यथा सर्वे निहताः सपदानुगाः}
{धृष्टद्युम्नः शिखण्डी च द्रौपद्याः पञ्च चात्मजाः}


\twolineshloka
{पाण्डवाश्च यथा मुक्तास्तथोभौ माधवौ युधि}
{कृपश्च कृतवर्मा च भारद्वाजस्य चात्मजः}


\twolineshloka
{यद्यथा यादृशं चैव युद्धं वृत्तं च साम्प्रतम्}
{अखिलं श्रोतुमिच्छामि कुशलो ह्यसि सञ्जय}


\twolineshloka
{[सञ्जय उवाच}
{}


\twolineshloka
{शृणु राजन्नवहितो यथावृत्तो महान्क्षयः}
{कुरूणां पाण्डवानां च समासाद्य परस्परम्}


\twolineshloka
{निहते सूतपुत्रे तु पाण्डवेन महात्मना}
{विद्रुतेषु च सैन्येषु समानीतेषु चासकृत्}


\twolineshloka
{घोरे मनुष्यदेहानामाजौ नरवरक्षये}
{यत्तत्कर्णे हते पार्थः कसिंहनादमथाकरोत्}


% Check verse!
तदा तव सुतान्राजन्प्राविशत्सुमहद्भयम्
\twolineshloka
{न सन्धातुमनीकानि न चैवाथ पराक्रमे}
{आसीद्बुद्धिर्हते कर्णे तव योधस्य कस्यचित्}


\twolineshloka
{वणिजो नावि भिन्नायामगाधे विप्लुवा इव}
{अपारे पारमिच्छन्तो हते द्वीपे किरीटिना}


\twolineshloka
{सूतपुत्रो हते राजन्वित्रस्ताः शरविक्षताः}
{अनाथा नाथमिच्छन्तो मृगाः सिंहार्दिता इव}


\twolineshloka
{भग्नशृङ्गा इव वृषा शीर्णदंष्ट्रा इवोरगाः}
{प्रत्युपायाम सायाह्ने निर्जिताः सव्यसाचिना}


\twolineshloka
{हतप्रवीरा विध्वस्ता निकृत्ता निशितैः शरैः}
{सूतपुत्रे हते राजन्पुत्रास्ते प्राद्रवंस्ततः}


\twolineshloka
{विध्वस्तकवचनाः सर्वे कान्दिशीका विचेतसः}
{अन्योन्यमभिनिघ्नन्तो वीक्षमाणा भयाद्दिशः}


\twolineshloka
{मामेव नूनं बीभत्सुर्मामेव च वृकोदरः}
{अभियातीति मन्वानाः पेतुर्मम्लुश्च भारत}


\twolineshloka
{अश्वानन्ये गजानन्ये रथानन्ये महारथाः}
{आरुह्य जवसम्पन्नाः पादातान्प्रजहुर्भयात्}


\twolineshloka
{कुञ्जरैः स्यन्दना भग्नाः सादिनश्च महारथैः}
{पदातिसङ्घाश्चाश्वौघैः पलायद्भिर्भृशं हताः}


\twolineshloka
{व्यालतस्करसङ्कीर्णे सार्थहीना यथा वने}
{तथा त्वदीया निहते सूतपुत्रे पदाऽभवन्}


\twolineshloka
{हतारोहास्तथा नागाश्छिन्नहस्तास्तथाऽपरे}
{सर्वं पार्थमयं लोकमपश्यन्वै भयार्दिताः}


\twolineshloka
{तान्प्रेक्ष्य द्रवतः सर्वान्भीमसेनभयार्दितान्}
{दुर्योधनोऽथ स्वं सूतं हाहाकृत्वैवमब्रवीत्}


\twolineshloka
{नातिक्रमिष्यते पार्थो धनुष्पाणिमवस्थितम्}
{जघने युध्यमानं मां तूर्णमश्वान्प्रचोदय}


\twolineshloka
{समरे युध्यमानं हि कौन्तेयो मां धनञ्जयः}
{नोत्सहेताप्यतिक्रान्तुं वेलामिव महार्णवः}


\twolineshloka
{अद्यार्जुनं सगोविन्दं मानिनं च वृकोदरम्}
{निहत्य शिष्टाञ्शत्रूंश्च कर्णस्यानृण्यमाप्नुयाम्}


\twolineshloka
{तच्छ्रुत्वा कुरुराजस्य शूरार्यसदृशं वचः}
{सूतो हेमपरिच्छन्नाञ्शनैरश्वानचोदयत्}


\twolineshloka
{गजाश्वरथहीनास्तु पादाताश्चैव मारिष}
{पञ्चविंशतिसाहस्राः प्राद्रवञ्शनकैरिव}


\twolineshloka
{तान्भीमसेनः सङ्क्रुद्धो धृष्टद्युम्नश्च पार्षतः}
{बलेन चतुरङ्गेण परिक्षिप्याहनच्छरैः}


\twolineshloka
{प्रत्ययुध्यंस्तु ते सर्वे भीमसेनं सपार्षतम्}
{पार्थपार्षतयोश्चान्ये जगृहुस्तत्र नामनी}


\twolineshloka
{अक्रुध्यत रणे भीमस्तैर्मृधे प्रत्यवस्थितैः}
{सोऽवतीर्य रथात्तूर्णं गदापाणिरयुध्यत}


\twolineshloka
{न तान्रथस्थो भूमिष्ठान्धर्मापेक्षी वृकोदरः}
{योधयामास कौन्तेयो भुजवीर्यमुपाश्रितः}


\twolineshloka
{जातरूपपरिच्छन्नां प्रगृह्य महतीं गदाम्}
{न्यवधीत्तावकान्सर्वान्दण्डपाणिरिवान्तकः}


\twolineshloka
{पदातयो हि संरब्धास्त्यक्तजीवितबान्धवाः}
{भीममभ्यद्रवन्सङ्ख्ये पतङ्गा इव पावकम्}


\twolineshloka
{आसाद्य भीमसेनं ते संरब्धा युद्धदुर्मदाः}
{विनेदुः सहसा दृष्ट्वा भूतग्रामा इवान्तकम्}


\twolineshloka
{श्येनवद्व्यचरद्भीमः खङ्गेन गदया तथा}
{पञ्चविंशतिसाहस्रांस्तावकानां व्यपोथयत्}


\twolineshloka
{हत्वा तत्पुरुषानीकं भीमः सत्यपराक्रमः}
{धृष्टद्युम्नं पुरस्कृत्य पुनस्तस्थौ महाबलः}


% Check verse!
धनञ्जयो रथानीकमन्वपद्यत वीर्यवान्
\twolineshloka
{माद्रीपुत्रौ च शकुनिं सात्यकिश्च महाबलः}
{जवेनाभ्यपतन्हृष्टा घ्नन्तो दौर्योधनं बलम्}


\twolineshloka
{तस्याश्ववाहान्सुबहूंस्ते निहत्य शितैः शरैः}
{तमन्वधावंस्त्वरितास्तत्र युद्धमवर्तत}


\twolineshloka
{ततो धनञ्जयो राजन्रथानीकमगाहत}
{विश्रुतं त्रिषु लोकेषु गाण्डीवं व्याक्षिपन्धनुः}


\twolineshloka
{कृष्णसारथिमायान्तं दृष्ट्वा श्वेतहयं रथम्}
{अर्जुनं चापि योद्वारं त्वदीयाः प्राद्रावन्भयात्}


\twolineshloka
{विप्रहीनरथाश्वाश्च शरैश्च परिवारिताः}
{पञ्चविंशतिसाहस्राः पार्थमार्च्छन्पदातयः}


\twolineshloka
{हत्वा तत्पुरुषानीकं पाञ्चालानां महारथः}
{भीमसेनं पुरस्कृत्य न चिरात्प्रत्यदृश्यत}


\twolineshloka
{महाधनुर्धरः श्रीमानमित्रगणमर्दनः}
{पुत्रः पाञ्चालराजस्य धृष्टद्युम्नो महायशाः}


\twolineshloka
{पारावतसवर्णाश्वं कोविदारवरध्वजम्}
{धृष्टद्युम्नं रणे दृष्ट्वा त्वदीयाः प्राद्रवन्भयात्}


\twolineshloka
{गान्धारराजं श्रीघ्रास्त्रमनुसृत्य यशस्विनौ}
{अचिरात्प्रत्यदृश्येतां माद्रीपुत्रौ ससात्यकी}


\twolineshloka
{चेकितानः शिखण्डी च द्रौपदेयाश्च मारिष}
{हत्वा त्वदीयं सुमहत्सैन्यं शङ्खानथाधमन्}


\twolineshloka
{ते सर्वे तावकान्प्रेक्ष्य द्रवतो वै पराङ्मुखान्}
{अभ्यधावन्त निघ्नन्तो वृषाञ्चित्वा वृषा इव}


\twolineshloka
{सेनावशेषं तं दृष्ट्वा तव पुत्रस्य पाण्डवः}
{अवस्थितं सव्यसाची चुक्रोध बलवन्नृप}


\twolineshloka
{तत एनं शरै राजन्सहसा समवाकिरत्}
{रजसा चोद्गतेनाथ न स्म किञ्चन दृश्यते}


\twolineshloka
{अन्धकारीकृते लोके शरीभूते महीतले}
{दिशः सर्वा महाराज तावकाः प्राद्रवन्भयात्}


\twolineshloka
{भज्यमानेषु सर्वेषु कुरुराजो विशाम्पते}
{परेषामात्मनश्चैव सैन्ये ते समुपाद्रवत्}


\twolineshloka
{ततो दुर्योधनः सर्वानाजुहावाथ पाण्डवान्}
{युद्धाय भरतश्रेष्ठ देवानिव पुरा बलिः}


\twolineshloka
{त एनमभिगर्जन्तं सहिताः समुपाद्रवन्}
{नानाशस्त्रसृजः क्रुद्धा भर्त्सयन्तो मुहुर्मुहुः}


% Check verse!
दुर्योधनोऽप्यसम्भ्रान्तस्तानरीन्व्यधमच्छरैः
\twolineshloka
{तत्राद्भुतमपरश्याम तव पुत्रस्य पौरुषम्}
{यदेनं पाण्डवाः सर्वे न शेकुरतिवर्तितुम्}


\twolineshloka
{नातिदूरापयातं च कृतबुद्धि पलायने}
{दुर्योधनः स्वकं सैन्यमपश्यकद्भृशविक्षतम्}


\twolineshloka
{ततोऽवस्थाप्य राजेन्द्र कृतबुद्धिस्तवात्मजः}
{हर्षयन्निव तान्योधांस्ततो वचनमब्रवीत्}


\twolineshloka
{न तं देशं प्रपश्यामि पृथिव्यां पर्वतेषु च}
{यत्र यातान्न वो हन्युः पाण्डवाः किं सृतेन वः}


\twolineshloka
{स्वल्पं चैव बलं तेषां कृष्णौ च भृशविक्षतौ}
{यदि सर्वेऽपि तिष्ठामो ध्रुवं नो विजयो भवेत्}


\twolineshloka
{विप्लयातांस्तु वो भिन्नान्पाण्डवाः कृतकिल्पिषान्}
{अनुसृत्य हनिष्यन्ति श्रेयोः न समरे वधः}


\twolineshloka
{सुखः साङ्ग्रामिको मृत्युः क्षत्रधर्मेण युध्यताम्}
{मृतो दुःखं न जानीते प्रेत्य चानन्त्यमश्नुते}


\twolineshloka
{शृण्वन्तु क्षत्रियाः सर्वे यावन्तोऽत्र समागताः}
{द्विषतो भीमसेनस्य वशमेष्यथ विद्रुताः}


\twolineshloka
{पितामहैराचरितं न धर्मं हातुमर्हथ}
{नान्यत्कर्मास्ति पापीयः क्षत्रियस्य पलायनात्}


\twolineshloka
{न युद्वधर्माच्छ्रेयान्हि पन्थाः स्वर्गस्य कौरवाः}
{सुचिरेणार्जिताँलोकान्सद्यो युद्धात्समश्नुते}


\threelineshloka
{तस्य तद्वचनं राज्ञः पूजयित्वा महारथाः}
{पुनरेवाभ्यवर्तन्त क्षत्रियाः पाण्डवान्प्रति}
{पराजयममृष्यन्त कृतचित्ताश्च विक्रमे}


\twolineshloka
{ततः प्रववृते युद्धं पुनरेव सुदारुणम्}
{तावकानां परेषां च देवासुररणोपमम्}


\twolineshloka
{युधिष्ठिरपुरोगांश्च सर्वसैन्येन पाण्डवान्}
{अन्वधावन्महाराज पुत्रो दुर्योधनस्तव ॥]}


\chapter{अध्यायः ३}
\twolineshloka
{[सञ्जय उवाच}
{}


\twolineshloka
{शृणु राजन्नवहितो यथावृत्तो महान्क्षयः}
{कुरूणां पाण्डवानां च समासाद्य परस्परम्}


\twolineshloka
{निहते सूतपुत्रे तु पाण्डवेन महात्मना}
{विद्रुतेषु च सैन्येषु समानीतेषु चासकृत्}


\twolineshloka
{घोरे मनुष्यदेहानामाजौ नरवरक्षये}
{यत्तत्कर्णे हते पार्थः कसिंहनादमथाकरोत्}


% Check verse!
तदा तव सुतान्राजन्प्राविशत्सुमहद्भयम्
\twolineshloka
{न सन्धातुमनीकानि न चैवाथ पराक्रमे}
{आसीद्बुद्धिर्हते कर्णे तव योधस्य कस्यचित्}


\twolineshloka
{वणिजो नावि भिन्नायामगाधे विप्लुवा इव}
{अपारे पारमिच्छन्तो हते द्वीपे किरीटिना}


\twolineshloka
{सूतपुत्रो हते राजन्वित्रस्ताः शरविक्षताः}
{अनाथा नाथमिच्छन्तो मृगाः सिंहार्दिता इव}


\twolineshloka
{भग्नशृङ्गा इव वृषा शीर्णदंष्ट्रा इवोरगाः}
{प्रत्युपायाम सायाह्ने निर्जिताः सव्यसाचिना}


\twolineshloka
{हतप्रवीरा विध्वस्ता निकृत्ता निशितैः शरैः}
{सूतपुत्रे हते राजन्पुत्रास्ते प्राद्रवंस्ततः}


\twolineshloka
{विध्वस्तकवचनाः सर्वे कान्दिशीका विचेतसः}
{अन्योन्यमभिनिघ्नन्तो वीक्षमाणा भयाद्दिशः}


\twolineshloka
{मामेव नूनं बीभत्सुर्मामेव च वृकोदरः}
{अभियातीति मन्वानाः पेतुर्मम्लुश्च भारत}


\twolineshloka
{अश्वानन्ये गजानन्ये रथानन्ये महारथाः}
{आरुह्य जवसम्पन्नाः पादातान्प्रजहुर्भयात्}


\twolineshloka
{कुञ्जरैः स्यन्दना भग्नाः सादिनश्च महारथैः}
{पदातिसङ्घाश्चाश्वौघैः पलायद्भिर्भृशं हताः}


\twolineshloka
{व्यालतस्करसङ्कीर्णे सार्थहीना यथा वने}
{तथा त्वदीया निहते सूतपुत्रे पदाऽभवन्}


\twolineshloka
{हतारोहास्तथा नागाश्छिन्नहस्तास्तथाऽपरे}
{सर्वं पार्थमयं लोकमपश्यन्वै भयार्दिताः}


\twolineshloka
{तान्प्रेक्ष्य द्रवतः सर्वान्भीमसेनभयार्दितान्}
{दुर्योधनोऽथ स्वं सूतं हाहाकृत्वैवमब्रवीत्}


\twolineshloka
{नातिक्रमिष्यते पार्थो धनुष्पाणिमवस्थितम्}
{जघने युध्यमानं मां तूर्णमश्वान्प्रचोदय}


\twolineshloka
{समरे युध्यमानं हि कौन्तेयो मां धनञ्जयः}
{नोत्सहेताप्यतिक्रान्तुं वेलामिव महार्णवः}


\twolineshloka
{अद्यार्जुनं सगोविन्दं मानिनं च वृकोदरम्}
{निहत्य शिष्टाञ्शत्रूंश्च कर्णस्यानृण्यमाप्नुयाम्}


\twolineshloka
{तच्छ्रुत्वा कुरुराजस्य शूरार्यसदृशं वचः}
{सूतो हेमपरिच्छन्नाञ्शनैरश्वानचोदयत्}


\twolineshloka
{गजाश्वरथहीनास्तु पादाताश्चैव मारिष}
{पञ्चविंशतिसाहस्राः प्राद्रवञ्शनकैरिव}


\twolineshloka
{तान्भीमसेनः सङ्क्रुद्धो धृष्टद्युम्नश्च पार्षतः}
{बलेन चतुरङ्गेण परिक्षिप्याहनच्छरैः}


\twolineshloka
{प्रत्ययुध्यंस्तु ते सर्वे भीमसेनं सपार्षतम्}
{पार्थपार्षतयोश्चान्ये जगृहुस्तत्र नामनी}


\twolineshloka
{अक्रुध्यत रणे भीमस्तैर्मृधे प्रत्यवस्थितैः}
{सोऽवतीर्य रथात्तूर्णं गदापाणिरयुध्यत}


\twolineshloka
{न तान्रथस्थो भूमिष्ठान्धर्मापेक्षी वृकोदरः}
{योधयामास कौन्तेयो भुजवीर्यमुपाश्रितः}


\twolineshloka
{जातरूपपरिच्छन्नां प्रगृह्य महतीं गदाम्}
{न्यवधीत्तावकान्सर्वान्दण्डपाणिरिवान्तकः}


\twolineshloka
{पदातयो हि संरब्धास्त्यक्तजीवितबान्धवाः}
{भीममभ्यद्रवन्सङ्ख्ये पतङ्गा इव पावकम्}


\twolineshloka
{आसाद्य भीमसेनं ते संरब्धा युद्धदुर्मदाः}
{विनेदुः सहसा दृष्ट्वा भूतग्रामा इवान्तकम्}


\twolineshloka
{श्येनवद्व्यचरद्भीमः खङ्गेन गदया तथा}
{पञ्चविंशतिसाहस्रांस्तावकानां व्यपोथयत्}


\twolineshloka
{हत्वा तत्पुरुषानीकं भीमः सत्यपराक्रमः}
{धृष्टद्युम्नं पुरस्कृत्य पुनस्तस्थौ महाबलः}


% Check verse!
धनञ्जयो रथानीकमन्वपद्यत वीर्यवान्
\twolineshloka
{माद्रीपुत्रौ च शकुनिं सात्यकिश्च महाबलः}
{जवेनाभ्यपतन्हृष्टा घ्नन्तो दौर्योधनं बलम्}


\twolineshloka
{तस्याश्ववाहान्सुबहूंस्ते निहत्य शितैः शरैः}
{तमन्वधावंस्त्वरितास्तत्र युद्धमवर्तत}


\twolineshloka
{ततो धनञ्जयो राजन्रथानीकमगाहत}
{विश्रुतं त्रिषु लोकेषु गाण्डीवं व्याक्षिपन्धनुः}


\twolineshloka
{कृष्णसारथिमायान्तं दृष्ट्वा श्वेतहयं रथम्}
{अर्जुनं चापि योद्वारं त्वदीयाः प्राद्रावन्भयात्}


\twolineshloka
{विप्रहीनरथाश्वाश्च शरैश्च परिवारिताः}
{पञ्चविंशतिसाहस्राः पार्थमार्च्छन्पदातयः}


\twolineshloka
{हत्वा तत्पुरुषानीकं पाञ्चालानां महारथः}
{भीमसेनं पुरस्कृत्य न चिरात्प्रत्यदृश्यत}


\twolineshloka
{महाधनुर्धरः श्रीमानमित्रगणमर्दनः}
{पुत्रः पाञ्चालराजस्य धृष्टद्युम्नो महायशाः}


\twolineshloka
{पारावतसवर्णाश्वं कोविदारवरध्वजम्}
{धृष्टद्युम्नं रणे दृष्ट्वा त्वदीयाः प्राद्रवन्भयात्}


\twolineshloka
{गान्धारराजं श्रीघ्रास्त्रमनुसृत्य यशस्विनौ}
{अचिरात्प्रत्यदृश्येतां माद्रीपुत्रौ ससात्यकी}


\twolineshloka
{चेकितानः शिखण्डी च द्रौपदेयाश्च मारिष}
{हत्वा त्वदीयं सुमहत्सैन्यं शङ्खानथाधमन्}


\twolineshloka
{ते सर्वे तावकान्प्रेक्ष्य द्रवतो वै पराङ्मुखान्}
{अभ्यधावन्त निघ्नन्तो वृषाञ्चित्वा वृषा इव}


\twolineshloka
{सेनावशेषं तं दृष्ट्वा तव पुत्रस्य पाण्डवः}
{अवस्थितं सव्यसाची चुक्रोध बलवन्नृप}


\twolineshloka
{तत एनं शरै राजन्सहसा समवाकिरत्}
{रजसा चोद्गतेनाथ न स्म किञ्चन दृश्यते}


\twolineshloka
{अन्धकारीकृते लोके शरीभूते महीतले}
{दिशः सर्वा महाराज तावकाः प्राद्रवन्भयात्}


\twolineshloka
{भज्यमानेषु सर्वेषु कुरुराजो विशाम्पते}
{परेषामात्मनश्चैव सैन्ये ते समुपाद्रवत्}


\twolineshloka
{ततो दुर्योधनः सर्वानाजुहावाथ पाण्डवान्}
{युद्धाय भरतश्रेष्ठ देवानिव पुरा बलिः}


\twolineshloka
{त एनमभिगर्जन्तं सहिताः समुपाद्रवन्}
{नानाशस्त्रसृजः क्रुद्धा भर्त्सयन्तो मुहुर्मुहुः}


% Check verse!
दुर्योधनोऽप्यसम्भ्रान्तस्तानरीन्व्यधमच्छरैः
\twolineshloka
{तत्राद्भुतमपरश्याम तव पुत्रस्य पौरुषम्}
{यदेनं पाण्डवाः सर्वे न शेकुरतिवर्तितुम्}


\twolineshloka
{नातिदूरापयातं च कृतबुद्धि पलायने}
{दुर्योधनः स्वकं सैन्यमपश्यकद्भृशविक्षतम्}


\twolineshloka
{ततोऽवस्थाप्य राजेन्द्र कृतबुद्धिस्तवात्मजः}
{हर्षयन्निव तान्योधांस्ततो वचनमब्रवीत्}


\twolineshloka
{न तं देशं प्रपश्यामि पृथिव्यां पर्वतेषु च}
{यत्र यातान्न वो हन्युः पाण्डवाः किं सृतेन वः}


\twolineshloka
{स्वल्पं चैव बलं तेषां कृष्णौ च भृशविक्षतौ}
{यदि सर्वेऽपि तिष्ठामो ध्रुवं नो विजयो भवेत्}


\twolineshloka
{विप्लयातांस्तु वो भिन्नान्पाण्डवाः कृतकिल्पिषान्}
{अनुसृत्य हनिष्यन्ति श्रेयोः न समरे वधः}


\twolineshloka
{सुखः साङ्ग्रामिको मृत्युः क्षत्रधर्मेण युध्यताम्}
{मृतो दुःखं न जानीते प्रेत्य चानन्त्यमश्नुते}


\twolineshloka
{शृण्वन्तु क्षत्रियाः सर्वे यावन्तोऽत्र समागताः}
{द्विषतो भीमसेनस्य वशमेष्यथ विद्रुताः}


\twolineshloka
{पितामहैराचरितं न धर्मं हातुमर्हथ}
{नान्यत्कर्मास्ति पापीयः क्षत्रियस्य पलायनात्}


\twolineshloka
{न युद्वधर्माच्छ्रेयान्हि पन्थाः स्वर्गस्य कौरवाः}
{सुचिरेणार्जिताँलोकान्सद्यो युद्धात्समश्नुते}


\threelineshloka
{तस्य तद्वचनं राज्ञः पूजयित्वा महारथाः}
{पुनरेवाभ्यवर्तन्त क्षत्रियाः पाण्डवान्प्रति}
{पराजयममृष्यन्त कृतचित्ताश्च विक्रमे}


\twolineshloka
{ततः प्रववृते युद्धं पुनरेव सुदारुणम्}
{तावकानां परेषां च देवासुररणोपमम्}


\twolineshloka
{युधिष्ठिरपुरोगांश्च सर्वसैन्येन पाण्डवान्}
{अन्वधावन्महाराज पुत्रो दुर्योधनस्तव ॥]}


\chapter{अध्यायः ४}
\twolineshloka
{सञ्जय उवाच}
{}


\twolineshloka
{शृणु राजन्नवहितो यथावृत्तो महान्क्षयः}
{कुरूणां पाण्डवानां च समासाद्य परस्परम्}


\twolineshloka
{निहते सूतपुत्रे च फल्गुनेन महात्मना}
{विद्रुतेषु च सैन्येषु समानीतेषु चासकृत्}


\twolineshloka
{विमुखे तव पुत्रे च शोकोपहतचेतसि}
{भृशोद्विग्नेषु सैन्येषु दृष्ट्वा पार्थस्य विक्रमम्}


\twolineshloka
{ध्यायमानेषु योधेषु दुऋखं प्राप्तेषु भारत}
{बलानां मत्यमानानां श्रुत्वा निनदमुत्तमम्}


\threelineshloka
{अभिज्ञानं नरेन्द्राणां विकृतं प्रेक्ष्य संयुगे}
{पतितानवनीपालान्धवजांश्चैव महात्मनाम्}
{रणे विनिहतान्नागान्दृष्ट्वा पत्तींश्च भारत}


\twolineshloka
{आयोधनं महाघोरं रुद्रस्याक्रीडसन्निभम्}
{अप्रख्यातिं गतानां तु राज्ञां शतसहस्रशः}


\threelineshloka
{कृपाविष्टः कृपो दृष्ट्वा वयः शीलसमन्वितः}
{अब्रवीत्तत्र तेजस्वी सोऽभिसृत्य जनाधिपम्}
{दुर्योधनमनुक्रोशाद्वाक्यं वाक्यविशारदः}


\twolineshloka
{दुर्योधन निबोधेयं यत्त्वां वक्ष्यामि कौरव}
{श्रुत्वा कुरु महाराज यदि ते रोचतेऽनघ}


\twolineshloka
{न युद्धधर्माच्छ्रेयान्वै पन्था राजेन्द्र विद्यते}
{यं समाश्रित्य युध्यन्ते क्षत्रियाः क्षत्रियर्षभ}


\twolineshloka
{पुत्रो भ्राता पिता चैव स्वस्रीयो मातुलस्तथा}
{सम्बन्धिबान्धवाश्चैव योद्वव्याः क्षत्रजीविन}


\threelineshloka
{वधे चैव परो धर्मस्तथाऽधर्मः पलायने}
{ते स्म घोरां समापन्ना जीविकां जीवितार्थिनः}
{तदत्र प्रतिवक्ष्यामि किञ्चिदेव हितं वचः}


\threelineshloka
{हते भीष्मे च द्रोणे च कर्णे चैव महारथे}
{जयद्रथे च निहते तव भ्रातृषु चानघ}
{लक्ष्मणे तव पुत्रे च किं शेषं पर्युपास्महे}


\twolineshloka
{येषु भारं समासज्य राज्ये मतिमकुर्महि}
{ते सन्त्यज्य तनूर्याताः शूरा ब्रह्मविदां गतिम्}


\twolineshloka
{वयं त्विह विनाभूता गुणवद्भिर्महारथैः}
{कृपणं वर्तयिष्यामः पातयित्वा नृपान्बहून्}


\twolineshloka
{सर्वैरथ च जीवद्भिर्बीभत्सुरपराजितः}
{कृष्णनेत्रो महाबाहुर्देवैरपि दुरासदः}


\twolineshloka
{इन्द्रकार्मुकवज्राभमिन्द्रकेतुमिवोच्छ्रितम्}
{वानरं केतुमासाद्य सञ्चचाल महाचमूः}


\twolineshloka
{सिंहनादेन भीमस्य पाञ्चजन्यस्वनेन च}
{गाण्डीवस्य च निर्घोषात्सम्मुह्यन्ते मनांसि नः}


\twolineshloka
{स्फुरन्तीव महाविद्युन्मुष्णन्ती नयनप्रभाम्}
{अलातमिव चाविद्वं गाण्डवीं समदृश्यत}


\twolineshloka
{जाम्बूनदविचित्रं च धूयमानं महद्धनुः}
{दृश्यते दिक्षु स्रवासु विद्युदभ्रघनेष्विव}


\threelineshloka
{उह्यमानश्च कृष्णेन वायुनेव बलाहकः}
{तावकं तद्बलं राजन्नर्जुनोऽस्त्रविशारदः}
{गहनं शिशिरापाये ददाहाग्निरिवोल्बणः}


\threelineshloka
{गाहमानमनीकानि महेन्द्रसदृशप्रभम्}
{विक्षोभयन्तं सेनां वै त्रासयन्तं च पार्थिवान्}
{धनञ्जयमपश्याम नलिनीमिव कुञ्जरम्}


\twolineshloka
{त्रासयन्तं तथा योधान्धनुर्घोषेण पांण्डवम्}
{भूय एनमपश्याम सिंहं मृगगणानिव}


\twolineshloka
{सर्वलोकमहेष्वासो वृषभौ सर्वधन्विनाम्}
{आमुक्तकवचौ कृष्णौ लोकमध्ये विरेजतुः}


\twolineshloka
{अद्य सप्तदशाहानि वर्तमानस्य भारत}
{सङ्ग्रामस्यातिघोरस्य युध्यतां चाभितो युधि}


\twolineshloka
{वायुनेव विधूतानि एव सैन्यानि गच्छता}
{शरदम्भोदजालानि विशीर्यन्ते समन्ततः}


\twolineshloka
{तां नावमिव पर्यस्तां मज्जमानां महार्णवे}
{तव सेनां महाराज सव्यसाची व्यकम्पयत्}


\twolineshloka
{क्वनु ते सूतपुत्रोऽभूत्क्वनु द्रोणः सहात्मजः}
{अहं क्व च क्व चात्मा ते हार्दिक्यश्च तथा क्वनु}


% Check verse!
दुःशासनश्च ते भ्राता भ्रातृभिः सहितः क्वनु
\threelineshloka
{वाणगोचरसम्प्राप्तं युध्यमानं जयद्रथम्}
{सम्बन्धिनस्ते भ्रातॄंश्च साहयान्मातुलांस्तथा}
{सर्वान्विक्रम्य मिपतो लोकमाक्रम्य मूर्धनि}


\twolineshloka
{जयद्रथो हतो राजन्किन्नु शेषमुपास्महे}
{को वेह स पुमानास्ते यो विजेष्यति पाण्डवम्}


\twolineshloka
{तस्य चास्त्राणि दिव्यानि विदितानि महात्मनः}
{गाण्डीवस्य च निर्घोषो धैर्याणि हरते हि नः}


\twolineshloka
{नष्टचन्द्रा यथा रात्रिः सेनेयं हतनायका}
{नागभग्नद्रुमा शुष्का नदीव प्रतिभाति मे}


\twolineshloka
{ध्वजिन्यां हतनेत्रायां यथेष्टं श्वेतवाहनः}
{चरिष्यति महाराजः कक्षेष्वग्निरिव ज्वलन्}


\twolineshloka
{सात्यकेश्चैव यो वेगो भीमसेनस्य चोभयोः}
{दारयेत गिरीन्सर्वाञ्शोषयेच्चैव सागरान्}


\twolineshloka
{उवाच वाक्यं यद्भीमः सभामध्ये विशाम्पते}
{कृतं तत्सफलं सर्वं भूयश्चैव करिष्यति}


\twolineshloka
{प्रमुखस्थे तदा कर्णे बलं पाण्डवरक्षितम्}
{दुरासदं तदा गुप्तं व्यूढं गाण्डीवधन्वना}


\twolineshloka
{युष्माभिस्तानि चीर्णानि यान्यसाधूनि साधुषु}
{अकारणकृतान्येव तेषां वः फलमागतम्}


\twolineshloka
{आत्मनोऽर्थे त्वया लोके यत्नतः सर्व आहृतः}
{स ते संशयितस्तात आत्मा च भरतर्षभ}


\twolineshloka
{रक्ष दुर्योधनात्मानमात्मा सर्वस्य भाजनम्}
{भिन्ने हि भाजने तात दिशो गच्छति तद्गतम्}


\twolineshloka
{हीयमानेन वै सन्धिः पर्येष्टव्यः समेन वा}
{विग्रहो वर्धमानेन नीतिरेषा बृहस्पतेः}


\twolineshloka
{ते वयं पाण्डुपुत्रेभ्यो हीनाः स्म बलशक्तितः}
{अत्र ते पाण्डवैः सार्धं सन्धिं मन्ये क्षमं प्रभो}


\twolineshloka
{न जानीते हि यः श्रेयः श्रेयसश्चावमन्यते}
{स क्षिप्रं भ्रश्यते राज्यान्न च श्रेयोऽनुविन्दति}


\twolineshloka
{अणिपत्य हि राजानं राज्यं यदि लभेमहि}
{श्रेयः स्यान्न तु मौढ्येन राजन्गन्तुं पराभवम्}


\twolineshloka
{वैचित्रवीर्यवचनात्कृपाशीलो युधिष्ठिरः}
{विनियुञ्जीत राज्ये त्वां गोविन्दवचनेन च}


\twolineshloka
{`अजातशत्रुः कौरव्यो गुरुशुश्रूषणे रतः}
{धृतराष्ट्रस्य वचनं नावमंस्यति धार्मिकः}


% Check verse!
कुर्वन्ति भ्रातरश्चास्य वचनं नात्र संशयः
\twolineshloka
{यद्ब्रूयाद्धि हृषीकेशो राजानमपराजितम्}
{अर्जुनो भीमसेनश्च सर्वे कुर्युरसंशयम्}


\twolineshloka
{नातिक्रमिष्यते कृष्णो वचनं पाण्डवस्य तु}
{धृतराष्ट्रस्य मन्येऽहं नापि कृष्णस्य पाण्डवः}


\twolineshloka
{एतत्क्षममहं मन्ये तव पार्थैरविग्रहम्}
{न त्वां ब्रवीमि कार्पण्यान्न प्राणपरिरक्षणात्}


% Check verse!
पथ्यं राजन्ब्रवीमि त्वां तत्परासुः स्मरिष्यसि
\twolineshloka
{इति वृद्धो विलप्यैतत्कृपः शारद्वतो वचः}
{दीर्घमुष्णं च निःश्वस्यशुशोच च मुमोह च}


\chapter{अध्यायः ५}
\twolineshloka
{सञ्जय उवाच}
{}


\twolineshloka
{एवमुक्तस्ततो राजा गौतमेन तपस्विना}
{निःश्वस्य दीर्घमुष्णं च तूष्णीमासीद्विशाम्पते}


\twolineshloka
{ततो मुहूर्तं स ध्यात्वा तव पुत्रो महामनाः}
{कृपं शारद्वतं वाक्यमित्युवाच परन्तपः}


\twolineshloka
{यत्किञ्चित्सुहृदा वाच्यं तत्सर्वं श्रावितो ह्यहम्}
{कृतं च भवता सर्वं प्राणान्सन्त्यज्य युध्यता}


\twolineshloka
{गाहमानमनीकानि युध्यमानं महारथैः}
{पाण्डवैरतितेजोभिर्लोकस्त्वामनुदृष्टवान्}


\twolineshloka
{सुहृदा यदिदं वाक्यं भवता श्रावितो ह्यहम्}
{न मां प्रीणाति तत्सर्वं मुमूर्षोरिव भेषजम्}


\twolineshloka
{हेतुकारणसंयुक्तं हितं वचनमुत्तमम्}
{उच्यमानं महाबाहो न मे विप्राग्र्य रोचने}


% Check verse!
राज्याद्विनिकृतोऽस्माभिःकथं सोस्मासु विश्वसेत्
\twolineshloka
{अक्षद्यूते च नृपतिर्जितोऽस्माभिर्महाधनः}
{स कथं मम वाक्यानि श्रद्दध्याद्भूय एव तु}


\threelineshloka
{तथा दूत्येन सम्प्राप्तः कृष्णः पार्थहिते रतः}
{प्रलब्धश्च हृषीकेशस्तच्च कर्माविचारितम्}
{स च मे वचनं ब्रह्मन्कथमेवाभिमन्यते}


\twolineshloka
{विललाप च यत्कृष्णा सभामध्ये समेयुषी}
{न तन्मर्षयते कृष्णो न राज्यहरणं तथा}


\twolineshloka
{एकप्राणावुभौ कृष्णावन्योन्यमभिसंश्रितौ}
{पुरा यच्छ्रुतमेवासीदद्य पश्यामि तत्प्रभो}


\twolineshloka
{स्वस्रीयं निहतं दृष्ट्वा दुःखं स्वपिति केशवः}
{कृतागसो व यं तस्य हितं मे स कथं चरेत्}


\twolineshloka
{अभिमन्योर्विनाशेन न शर्म लभतेऽर्जुनः}
{स कथं मद्धिते यत्नं प्रकरिष्यति याचितः}


\twolineshloka
{मध्यमः पाण्डवस्तीक्ष्णो भीमसेनो महाबलः}
{प्रतिज्ञातं च तेनोग्रं भज्येतापि न सन्नमेत्}


\twolineshloka
{उभौ तौ बद्धनिस्त्रिंशावुभौ चाबद्धकङ्कटौ}
{कृतवैरावुभौ वीरौ यमावपि यमोपमौ}


\twolineshloka
{धृष्टद्युम्नः शिखण्डी च कृतवैरौ मया सह}
{तौ कथं मद्विते यत्नं कुर्यातां द्विजसत्तम}


\twolineshloka
{दुःशासनेन यत्कृष्णा एकवस्त्रा रजस्वला}
{परिक्लिष्टा सभामध्ये सर्वलोकस्य पश्यतः}


\twolineshloka
{तथा विवसनां दीनां स्मरन्त्यद्यापि पाण्डवाः}
{न निवारयितुं शक्याः सङ्ग्रामात्ते परन्तपाः}


\threelineshloka
{यदा च द्रौपदी क्लिष्टा मद्विनाशाय दुःखिता}
{उग्रं तेपे तपः कृष्णा भर्तॄणामर्थसिद्धये}
{स्थण्डिले नित्यदा शेते यावद्वैरस्य यातनम्}


\twolineshloka
{निक्षिप्य मानं दर्पं च वासुदेवसहोदरा}
{कृष्णायाः प्रेष्यवद्भूत्वा शुश्रूषां कुरुते सदा}


\twolineshloka
{इति सर्वं समुन्नद्धं न निर्वाति कथञ्चन}
{अभिमन्योर्विनाशेन स सन्धेयः कथं मया}


\twolineshloka
{कथं च राजा भुक्त्वेमां पृथिवीं सागराम्बराम्}
{पाण्डवानां प्रसादेन भोक्ष्ये राज्यमहं कथम्}


\twolineshloka
{उपर्युपरि राज्ञां वै ज्वलित्वा भास्करो यथा}
{युधिष्ठिरं कथं पञ्चादनुयास्यामि दासवत्}


\twolineshloka
{कथं भुक्त्वा स्वयं भोगान्दत्त्वा दायांश्च पुष्कलान्}
{कृपणं वर्तयिष्यामि कृपणैः सह जीविकाम्}


\twolineshloka
{नाभ्यसूयामि ते वाक्यमुक्तं स्निग्धं हितं त्वया ॥न तु सन्धिमहं मन्ये प्राप्तकालं कथञ्चन}
{}


% Check verse!
सुनीतमनुपश्यामि सुयुद्धेन परन्तप
% Check verse!
नायं क्लीबायितुं कालः संयोद्वुं काल एव नः
\twolineshloka
{इष्टं मे बहुभिर्जज्ञैर्दत्ता विप्रेषु दक्षिणाः}
{प्राप्ताः कामाः श्रुता वेदाः शत्रूणां मूर्ध्नि च स्थितम्}


\twolineshloka
{भृत्या मे सुभृतास्तात दीनश्चाभ्युद्वृतो जनः}
{नोत्साहेऽद्य द्विजश्रेष्ठ पाण्डवान्वक्तुमीदृशम्}


\twolineshloka
{जितानि परराष्ट्राणि स्वराष्ट्रमनुपालितम्}
{भुक्ताश्च विविधा भोगास्त्रिवर्गः सेवितो मया}


\threelineshloka
{पितॄणां गतमानृण्यं क्षत्रधर्मस्य चोभयोः}
{न ध्रुवं सुखमस्तीह कुतो राष्ट्रं कुतो यशः}
{इह कीर्तिर्विचेतव्या सा च युद्वेन नान्यथा}


\twolineshloka
{वृथा च यत्क्षत्रियस्य निधनं तद्विगर्हितम्}
{अधर्मः सुमहानेष यच्छय्यामरणं गृहे}


\twolineshloka
{अरण्ये यो विमुच्येत सङ्ग्रामे वा तनुं नृपः}
{क्रतूनाहृत्य महतो महिमानं स गच्छति}


\twolineshloka
{कृपणं विलपन्नार्तो जरयाऽभिपरिप्लुतः}
{म्रियते रुदतां मध्ये ज्ञातीनां न स पूरुषः}


\twolineshloka
{त्यक्त्वा तु विविधान्भोगान्प्राप्तानां परमां गतिम्}
{अपीदानीं सुयुद्धेन गच्छेयं यत्सलोकताम्}


\threelineshloka
{शूराणामार्यवृत्तानां सङ्क्रामेष्वनिवर्तिनाम्}
{धीमतां सत्यसन्धानां सर्वेषां क्रतुयाजिनाम्}
{शस्त्रावभृथपूतानां ध्रुवो वासस्त्रिविष्टपे}


% Check verse!
मुदा नूनं प्रपश्यन्ति युद्वे ह्यप्सरसां गणाः
\twolineshloka
{पश्यन्ति नूनं पितरः पूजितान्सुरसंसदि}
{अप्सरोभिः परिवृतान्मोदमानांस्त्रिविष्टपे}


\twolineshloka
{पन्थानममरैर्यान्तं शूरैश्चैवानिवर्तिभिः}
{अपि तत्सङ्गतं मार्गं वयमध्यारुहेमहि}


\twolineshloka
{पितामहेन वृद्वेन तथाऽचार्येण धीमता}
{जयद्रथेन कर्णेन तथा दुःशासनेन च}


\twolineshloka
{घटमाना मदर्थेऽस्मिन्हताः शूरा जनाधिपाः}
{शेरते लोहिताक्ताङ्गाः पृथिव्यां शरविक्षताः}


\twolineshloka
{उत्तमास्त्रविदः शूरा यथोक्तक्रतुयाजिनः}
{त्यक्त्वा प्राणान्यथान्यायमिन्द्रसद्मसु धिष्ठितः}


\twolineshloka
{तैः स्वयं रचितो मार्गो दुर्गमो हि पुनर्भवेत्}
{सम्पतद्भिर्महावेगैरितो यास्यामि सद्गतिम्}


\twolineshloka
{ये मदर्थे हताः शूरास्तेषां कृतमनुस्मरन्}
{ऋणं तत्प्रतियुञ्जानो न राज्ये मन आदधे}


\twolineshloka
{पातयित्वा वयस्यांश्च भ्रातृनथ पितामहान्}
{जीवितं यदि रक्षेयं लोको मां गर्हयेद्व्रुवम्}


\twolineshloka
{कीदृशं च भवेद्राज्यं मम हीनस्य बन्धुभिः}
{सखिभिश्च विशेषेण प्रणिपत्य च पाण्डवम्}


\threelineshloka
{सोऽहमेतादृशं कृत्वा जगतोऽस्य पराभवम्}
{सुयुद्धेन हतः स्वर्गं प्राप्सामि न तदन्यथा ॥सञ्जय उवाच}
{}


\twolineshloka
{एं दुर्योधनेनोक्ते सर्वे सम्पूज्य तद्वचः}
{साधुसाध्विति राजानं क्षत्रियाः सम्बभाषिरे}


\twolineshloka
{पराजयमशोचन्तः कृतचित्ताश्च विक्रमे}
{सर्वे सुनिश्चिता योद्धुमुदग्रमनसोऽभवन्}


\twolineshloka
{ततो वाहान्समाश्वास्य सर्वे युद्धाभिनन्दिनः}
{ऊने द्वियोजने गत्वा प्रत्यतिष्ठन्त कौरवाः}


\twolineshloka
{आकाशे विद्रुमे पुण्ये प्रस्थे हिमवतः शुभे}
{अरुणां सरस्वतीं प्राप्य पपुः सस्नुश्च तेजलम्}


\threelineshloka
{तव पुत्रकृतोत्साहाः पर्यवर्तन्त ते ततः}
{पर्यवस्थाप्य चात्मानमन्योन्येन पुनस्तदा}
{सर्वे राजन्न्यवर्तन्त क्षत्रियाः कालचोदिताः}


\chapter{अध्यायः ६}
\twolineshloka
{सञ्जय उवाच}
{}


\twolineshloka
{एवमुक्तस्ततो राजा गौतमेन तपस्विना}
{निःश्वस्य दीर्घमुष्णं च तूष्णीमासीद्विशाम्पते}


\twolineshloka
{ततो मुहूर्तं स ध्यात्वा तव पुत्रो महामनाः}
{कृपं शारद्वतं वाक्यमित्युवाच परन्तपः}


\twolineshloka
{यत्किञ्चित्सुहृदा वाच्यं तत्सर्वं श्रावितो ह्यहम्}
{कृतं च भवता सर्वं प्राणान्सन्त्यज्य युध्यता}


\twolineshloka
{गाहमानमनीकानि युध्यमानं महारथैः}
{पाण्डवैरतितेजोभिर्लोकस्त्वामनुदृष्टवान्}


\twolineshloka
{सुहृदा यदिदं वाक्यं भवता श्रावितो ह्यहम्}
{न मां प्रीणाति तत्सर्वं मुमूर्षोरिव भेषजम्}


\twolineshloka
{हेतुकारणसंयुक्तं हितं वचनमुत्तमम्}
{उच्यमानं महाबाहो न मे विप्राग्र्य रोचने}


% Check verse!
राज्याद्विनिकृतोऽस्माभिः कथं सोस्मासु विश्वसेत्
\twolineshloka
{अक्षद्यूते च नृपतिर्जितोऽस्माभिर्महाधनः}
{स कथं मम वाक्यानि श्रद्दध्याद्भूय एव तु}


\threelineshloka
{तथा दूत्येन सम्प्राप्तः कृष्णः पार्थहिते रतः}
{प्रलब्धश्च हृषीकेशस्तच्च कर्माविचारितम्}
{स च मे वचनं ब्रह्मन्कथमेवाभिमन्यते}


\twolineshloka
{विललाप च यत्कृष्णा सभामध्ये समेयुषी}
{न तन्मर्षयते कृष्णो न राज्यहरणं तथा}


\twolineshloka
{एकप्राणावुभौ कृष्णावन्योन्यमभिसंश्रितौ}
{पुरा यच्छ्रुतमेवासीदद्य पश्यामि तत्प्रभो}


\twolineshloka
{स्वस्रीयं निहतं दृष्ट्वा दुःखं स्वपिति केशवः}
{कृतागसो वयं तस्य हितं मे स कथं चरेत्}


\twolineshloka
{अभिमन्योर्विनाशेन न शर्म लभतेऽर्जुनः}
{स कथं मद्धिते यत्नं प्रकरिष्यति याचितः}


\twolineshloka
{मध्यमः पाण्डवस्तीक्ष्णो भीमसेनो महाबलः}
{प्रतिज्ञातं च तेनोग्रं भज्येतापि न सन्नमेत्}


\twolineshloka
{उभौ तौ बद्धनिस्त्रिंशावुभौ चाबद्धकङ्कटौ}
{कृतवैरावुभौ वीरौ यमावपि यमोपमौ}


\twolineshloka
{धृष्टद्युम्नः शिखण्डी च कृतवैरौ मया सह}
{तौ कथं मद्धिते यत्नं कुर्यातां द्विजसत्तम}


\twolineshloka
{दुःशासनेन यत्कृष्णा एकवस्त्रा रजस्वला}
{परिक्लिष्टा सभामध्ये सर्वलोकस्य पश्यतः}


\twolineshloka
{कथा विवसनां दीनां स्मरन्त्यद्यापि पाण्डवाः}
{न निवारयितुं शक्याः सङ्ग्रामत्ते परन्तपाः}


\threelineshloka
{यदा च द्रौपदी क्लिष्टा मद्विनाशाय दुःखिता}
{उग्रं तेपे तपः कृष्णा भर्तॄणामर्थसिद्धये}
{स्थण्डिले नित्यदा शेते यावद्वैरस्य यातनम्}


\twolineshloka
{निक्षिप्य मानं दर्पं च वासुदेवसहोदरा}
{कृष्णायाः प्रेष्यवद्भूत्वा शुश्रूषां कुरुते सदा}


\twolineshloka
{इति सर्वं समुन्नद्धं न निर्वाति कथञ्चन}
{अभिमन्योर्विनाशेन स सन्धेयः कथं मया}


\twolineshloka
{कथं च राजा भुक्त्वेमां पृथिवीं सागराम्बराम्}
{पाण्डवानां प्रसादेन भोक्ष्ये राज्यमहं कथम्}


\twolineshloka
{उपर्युपरि राज्ञां वै ज्वलित्वा भास्करो यथा}
{युधिष्ठिरं कथं पञ्चादनुयास्यामि दासवत्}


\twolineshloka
{कथं भुक्त्वा स्वयं भोगान्दत्त्वा दायांश्च पुष्कलान्}
{कृपणं वर्तयिष्यामि कृपणैः सह जीविकाम्}


% Check verse!
नाभ्यसूयामि ते वाक्यमुक्तं स्निग्धं हितं त्वया
\twolineshloka
{न तु सन्धिमहं मन्ये प्राप्तकालं कथञ्चन}
{सुनीतमनुपश्यामि सुयुद्धेन परन्तप}


% Check verse!
नायं क्लीबायितुं कालः संयोद्धुं काल एव नः
\twolineshloka
{इष्टं मे बहुभिर्यज्ञैर्दत्ता विप्रेषु दक्षिणाः}
{प्राप्ताः कामाः श्रुता वेदाः शत्रूणां मूर्ध्निं च स्थितम्}


\twolineshloka
{भृत्या मे सुभृतास्तात दीनश्चाभ्यृद्भृतो जनः}
{नोत्सहेऽद्य द्विजश्रेष्ठ पाण्डवान्वक्तुमीदृशम्}


\twolineshloka
{जितानि परराष्ट्राणि स्वराष्ट्रमनुपालितम्}
{भुक्ताश्च विविधा भोगास्त्रिवर्गः सेवितो मया ॥ ==}


\chapter{अध्यायः ७}
\twolineshloka
{सञ्जय उवाच}
{}


\twolineshloka
{एतच्छ्रुत्वा वचो राज्ञो मद्रराजः प्रतापवान्}
{दुर्योधनं तदा राजन्वाक्यमेतदुवाच ह}


\threelineshloka
{दुर्योधन महाबाहो शृणु वाक्यविदां वर}
{यावेतौ मन्यसे कृष्णौ रथस्थौ रथिनांवरौ}
{न मे तुल्यावुभावेतौ बाहुवीर्ये कथञ्चन}


\twolineshloka
{उद्यतां पृथिवीं सर्वां ससुरासुरमानवाम्}
{योधयेयं रणमुखे सङ्क्रुद्धः किमु पाण्डवान्}


\twolineshloka
{विजेष्यामि रणे पार्थान्सोमकांश्च समागतान्}
{अहं सेनाप्रमेता ते भविष्यामि न संशयः}


\twolineshloka
{तं च व्यूहं विधास्यामि न क (त) रिष्यन्ति यं परे}
{इति सत्यं ब्रवीम्येष दुर्योधन न संशयः}


\twolineshloka
{`अद्यैवाहं रणे सर्वान्पाञ्चालान्सह पाण्डवैः}
{निहनिष्यामि वा राजन्स्वर्गं यास्यामि वा हतः}


% Check verse!
अद्य पश्यन्तु मां लोका विचरन्तमभीतवत्
\threelineshloka
{अद्य पाण्डुसुताः सर्वे वासुदेवः ससात्यकिः}
{पाञ्चालाश्चेदयश्चैव द्रौपदेयाश्च सर्वशः}
{धृष्टद्युम्नः शिखण्डी च सर्वे चापि प्रभद्रकाः}


\twolineshloka
{विक्रमं मम पश्यन्तु धनुषश्च महद्बलम्}
{लाघवं चास्त्रवीर्यं च भुजयोश्च बलं युधि}


\twolineshloka
{अद्य पश्यन्तु मे पार्थाः सिद्धाश्च सह चारणैः}
{यादृशं मे बलं बाह्वोः सम्पदस्त्रेषु या च मे}


\twolineshloka
{अद्य मे विक्रमं दृष्ट्वा पाण्डवानां महारथाः}
{प्रतीकारपरा भूत्वा चेष्टन्ते विविधाः क्रियाः}


% Check verse!
अद्य सैन्यानि पाण्डूनां द्रावयिष्ये समन्ततः
\threelineshloka
{द्रोणभीष्मावति विमो सूतपुत्रं च संयुगे}
{विचरिष्ये रणमुखे प्रियार्थं तव कौरव ॥सञ्जय उवाच}
{}


\threelineshloka
{एवमुक्तस्ततो राजा मद्राधिपतिमञ्जसा}
{अभ्यषिञ्चत सेनाया मध्ये भरतसत्तम}
{विधिना शास्त्रदृष्टेन क्लिष्टरूपो विशाम्पते}


\twolineshloka
{अभिषिक्ते ततस्तस्मिन्सिंहनादो महानभूत्}
{तव सैन्येऽभ्यवाद्यन्त वादित्राणि च भारत}


\twolineshloka
{हृष्टाश्चासंस्तथा योधा मद्रकाश्च महारथाः}
{तुष्टुवुश्चैव राजानं शल्यमाहवशोभिनम्}


\threelineshloka
{जय राजंश्चिरं जीव जहि शत्रून्समागतान्}
{तव बाहुबलं प्राप्य धार्तराष्ट्रो महाबलः}
{निखिलां पृथिवीं सर्वां प्रशास्तु निहतद्विषम्}


\twolineshloka
{त्वं हि शक्तो रणे जेतुं ससुरासुरमानवान्}
{मर्त्यधर्माण इह तु किमु सृञ्जयसोमकान्}


\threelineshloka
{एवं सम्पूज्यमानस्तु मद्राणामधिपो बली}
{हर्षं प्राप तदा वीरो दुरापमकृतात्मभिः ॥सञ्जय उवाच}
{}


\twolineshloka
{अभिषिक्ते तथा शल्ये तव सैन्येषु मानद}
{न कर्णव्यसनं किञ्चिन्मेनिरे तत्र भारत}


\twolineshloka
{हृष्टाः सुमनसश्चैव बभूवुस्तत्र सैनिकाः}
{मेनिरे निहतान्पार्थान्मद्रराजवशङ्गतान्}


\twolineshloka
{प्रहर्षं प्राप्य सेना तु तावकी भरतर्षभ}
{तां रात्रिं सुखिता सुप्ता हर्षचित्ता च साभवत्}


\twolineshloka
{सैन्यस्य तव तं शब्दं श्रुत्वा राजा युधिष्ठिरः}
{वार्ष्णेयमब्रवीद्वाक्यं सर्वक्षत्रस्य पश्यतः}


\twolineshloka
{मद्रराजः कृतः शल्यो धार्तराष्ट्रेण माघव}
{सेनापतिर्महेष्वासः सर्वसैन्येषु पूजितः}


\twolineshloka
{एतज्ज्ञात्वा यथाभूतं कुरु माधव यत्क्षमम्}
{भवान्नेता च गोप्ता च विधत्स्व यदनन्तरम्}


\twolineshloka
{तमब्रवीन्महाराज वासुदेवो जनाधिपम्}
{आर्तायनिमहं जाने यथातत्त्वेन भारत}


\twolineshloka
{वीर्यवांश्च महातेजा महात्मा च विशेषतः}
{कृती च चित्रयोधी च संयुक्तो लाघवेन च}


\twolineshloka
{यादृग्भीष्मस्तथा द्रोणो यादृक्कर्णश्च संयुगे}
{तादृशृस्तद्विशिष्टो वा मद्रराजो मतो मम}


\twolineshloka
{युध्यमानस्य तस्याहं चिन्तयानश्च भारत}
{योद्धारं नाधिगच्छामि तुल्यरूपं जनाधिप}


\twolineshloka
{शिखण़्ड्यर्जुनभीमानां सात्वतस्य च भारत}
{धृष्टद्युम्नस्य च तथा बलेनाभ्यधिको रणे}


\twolineshloka
{मद्रराजो महाराजः सिंहद्विरदविक्रमः}
{विचरिष्यत्यभीः काले कालः क्रुद्धः प्रजास्विव}


\twolineshloka
{तस्याद्य न प्रपश्यामि प्रतियोद्धारमाहने}
{त्वामृते पुरुषव्याघ्र शार्दूलसमविक्रमम्}


\twolineshloka
{स त्वमेको हि लोकेऽस्मिन्नान्यस्त्वत्तः पुमान्भवेत्}
{मद्रराजं रणे क्रुद्धं यो हन्यात्कुरुनन्दन}


\twolineshloka
{अहन्यहनि युध्यन्तं क्षोभयन्तं बलं तव}
{तस्माज्जहि रणे शल्यं मघवानिव शम्बरम्}


\threelineshloka
{सौतेः पश्चादसौ वीरो धार्तराष्ट्रेण सत्कृतः}
{तवैव हि जयो नूनं हते मद्रेश्वरे युधि}
{}


\twolineshloka
{तस्मिन्हते हतं सर्वं धार्तराष्ट्रबलं महत् ॥एतच्छ्रुत्वा महाराज वचनं मम साम्प्रतम्}
{}


% Check verse!
प्रत्युद्याहि रणे पार्थ मद्रराजं महारथम् ॥जहि चैनं महाबाहो वासवो नमुचिं यथा
\twolineshloka
{न चैवात्र दया कार्या मातुलोऽयं ममेति वै}
{क्षत्रवर्म पुरस्कृत्य जहि मद्रजनेश्वरम्}


\threelineshloka
{द्रोणभीष्मार्णवं तीर्त्वा कर्णपातालसम्भवम्}
{मा निमज्जस्व सगणः शल्यमासाद्य गोष्पदम्}
{}


\threelineshloka
{यच्च ते तपसो वीर्यं यच्च क्षात्रं बलं तव}
{तद्दर्शय रणे सर्वं जहि चैनं महारथम् ॥सञ्जय उवाच}
{}


\twolineshloka
{एतावदुक्त्वा वचनं केशवः परवीरहा}
{जगाम शिबिरं सायं पूज्यमानोऽथ पाण्डवैः}


\threelineshloka
{केशवे तु तदा याते धर्मपुत्रो युधिष्ठिरः}
{विसृज्य सर्वान्भ्रातॄंश्च पाञ्चालानथ सोमकान्}
{सुष्वाप रजनीं तां तु विशल्य इव कुञ्जरः}


\twolineshloka
{ते च सर्वे महेष्वासाः पाञ्चालाः पाण्डवास्तथा}
{कर्णस्य निधने हृष्टाः सुषुपुस्तां निशां तदा}


\threelineshloka
{गतज्वरं महेष्वासं तीर्णपारं महारथम्}
{बभूव पाण्डवेयानां सैन्यं च मुदितं निशि}
{सूतपुत्रस्य निधनाज्जयं लब्ध्वा च मारिष}


\chapter{अध्यायः ८}
\twolineshloka
{सञ्जय उवाच}
{}


\twolineshloka
{व्यतीतायां रजन्यां तु राजा दुर्योधनस्तदा}
{अब्रवीत्तावकान्सर्वान्सन्नह्यन्तां महारथाः}


\twolineshloka
{राज्ञश्च मतमाज्ञाय समनह्यत सा चमूः}
{अयोजयन्रथांस्तूर्णं पर्यधावंस्तथा परे}


\twolineshloka
{अकल्प्यन्त च मातङ्गाः समनह्यन्त पत्तयः}
{हयानास्तरणोपेतांश्चक्रुरन्ये सहस्रशः}


\twolineshloka
{वादित्राणां च निनदः प्रादुरासीद्विशाम्पते}
{योधानां सैन्यमुख्यानामन्योन्यं प्रतिगर्जताम्}


\twolineshloka
{ततो बलानि सर्वाणि हतशिष्टानि भारत}
{सन्नद्धानि व्यदृश्यन्त मृत्युं कृत्वा निवर्तनम्}


\twolineshloka
{शल्यं सेनापतिं कृत्वा मद्रराजं महारथाः}
{प्रविभज्य बलं सर्वमनीकेषु व्यवस्थिताः}


\threelineshloka
{ततः सर्वे समागम्य पुत्रेण तव सैनिकाः}
{कृपश्च कृतवर्मा च द्रौणिः शल्योऽथ सौबलः}
{अन्ये च पार्थिवाः शेषाः समयं चक्रुरादृताः}


\twolineshloka
{`अद्याचार्यसुतो द्रौणिर्नैको युध्येत शत्रुभिः'}
{न न एकेन योद्धव्यं कथञ्चिदपि पाण्डवैः}


\threelineshloka
{यो ह्येकः पाण्डवैर्युध्येद्यो वा युध्यन्तमुत्सृजेत्}
{स पञ्चभिर्भवेद्युक्तः पातकैश्चोपपातकैः}
{अन्योन्यं परिरक्षद्भिर्योद्धव्यं सहितैश्च नः}


\twolineshloka
{एवं ते समयं कृत्वा सर्वे तत्र महारथाः}
{मद्रराजं पुरस्कृत्य तूर्णमभ्यद्रवन्परान्}


\twolineshloka
{तथैव पाण्डवा राजन्व्यूह्य सैन्यं महारणे}
{अभ्ययुः कौरवान्युद्धे योत्स्यमानाः समन्ततः}


\threelineshloka
{ततो बलं समभवत्क्षुब्धार्णवसमस्वनम्}
{समुद्भूतार्णवाकारमुदीर्णरथकुञ्जरम् ॥धृतराष्ट्र उवाच}
{}


\twolineshloka
{द्रोणस्य चैव भीष्मस्य राधेयस्य च मे श्रुतम्}
{पातनं शंस मे भूयः शल्यस्याथ सुतस्य मे}


\threelineshloka
{कथं रणे हतः शल्यो धर्मराजेन सञ्जय}
{भीमेन च महाबाहुः पुत्रो दुर्योधनो मम ॥सञ्जय उवाच}
{}


\twolineshloka
{क्षयं मनुष्यदेहानां तथा नागाश्वसङ्क्षयम्}
{शृणु राजन्स्थिरो भूत्वा सङ्ग्रामं शंसतो मम}


% Check verse!
आशा बलवती राजन्पुत्राणां तेऽभवत्तदा
\twolineshloka
{हते द्रोणे च भीष्मे च सूतपुत्रे च पातिते}
{शल्यः पार्थान्रणे सर्वान्निहनिष्यति मारिष}


\threelineshloka
{तामाशां हृदये कृत्वा समाश्वस्य च भारत}
{मद्रराजं च समरे समाश्रित्य महारथम्}
{नाथवन्तं तदाऽऽत्मानममन्यत सुतस्तव}


\twolineshloka
{यदा कर्णे हते पार्थाः सिंहनादं प्रचक्रिरे}
{तदा राजन्धार्तराष्ट्रान्प्रविवेश महद्भयम्}


\twolineshloka
{तान्समाश्वास्य तु तदा मद्रराजः प्रतापवान्}
{व्यूह्य व्यूहं महाराज सर्वतोभद्रमृद्धिमत्}


\threelineshloka
{प्रत्युद्ययौ रणे पार्थान्मद्रराजः प्रतापवान्}
{विधून्वन्कार्मुकं चित्रं भारघ्नं वेगवत्तरम्}
{रथप्रवरमास्थाय सैन्धवाश्वं महारथः}


\threelineshloka
{तस्य सूतो महाराज रथस्थोऽशोभयद्रथम्}
{स तेन संवृतो वीरो रथेनामित्रकर्शनः}
{तस्थौ शूरो महाराज पुत्राणां ते भयप्रणुत्}


\twolineshloka
{प्रयाणे मद्रराजोऽभून्मुखं व्यूहस्य दंशितः}
{मद्रकैः सहितो वीरैः कर्णपुत्रैश्च दुर्जयैः}


\twolineshloka
{सव्येऽभूत्कृतवर्मा च त्रिगर्तैः परिवारितः}
{गौतमो दक्षिणे पार्श्वे शकैश्च यवनैः सह}


\twolineshloka
{अश्वत्थामा पृष्ठतोऽभूत्काम्भोजैः परिवारितः}
{दुर्योधनोऽभवन्मध्ये रक्षितः कुरुपुङ्गवैः}


\twolineshloka
{हयानीकेन महता सौबलश्चापि संवृतः}
{प्रययौ सर्वसैन्येन कैतव्यश्च महारथः}


\twolineshloka
{पाण्डवाश्च महेष्वासा व्यूह्य सैन्यमरिन्दमाः}
{त्रिधाभूता महाराज तव सैन्यमुपाद्रवन्}


\twolineshloka
{धृष्टद्युम्नः शिखण्डी च सात्यकिस्च महारथः}
{शल्यस्य वाहिनीं हन्तुमभिदुद्रुवुराहवे}


\twolineshloka
{ततो युधिष्ठिरो राजा स्वेनानीकेन संवृतः}
{शल्यमेवाभिदुद्राव जिघांसुर्भरतर्षभः}


\twolineshloka
{हार्दिक्यं च महेष्वासमर्जुनः शत्रुपूगहा}
{संशप्तकगणांश्चैव वेगितोऽभिविदुद्रुवे}


\twolineshloka
{गौतमं भीमसेनो वै सोमकाश्च महारथाः}
{अभ्यद्रवन्त राजेनद््र जिघांसन्तः परान्युधि}


\twolineshloka
{माद्रीपुत्रौ तु शकुनिमुलूकं च महारथम्}
{ससैन्यौ सहसैन्यौ तावुपतस्थतुराहवे}


\threelineshloka
{तथैवायुतशो योधास्तावकाः पाण्डवान्रणे}
{अभ्यवर्तन्त सङ्क्रुद्धा विविधायुधपाणयः ॥धृतराष्ट्र उवाच}
{}


\twolineshloka
{हते भीष्मे हमेष्वासे द्रोणे कर्णे जयद्रथे}
{कुरुष्वल्पावशिष्टेषु पाण्डवेषु च संयुगे}


\threelineshloka
{संरब्धेषु च पार्थेषु पराक्रान्तेषु सञ्जय}
{मामकानां परेषां च किं शिष्टमभवद्बलम् ॥सञ्जय उवाच}
{}


\twolineshloka
{यथा वयं परे राजन्युद्धाय समुपस्थिताः}
{यावच्चासीद्बलं शिष्टं सङ्ग्रामे तन्निबोध मे}


\twolineshloka
{एकादश सहस्राणि रथानां भरतर्षभ}
{दश दन्तिसहस्राणि सप्त चैव शतानि च}


\twolineshloka
{पूर्णे शतसहस्रे द्वे हयानां तत्र भारत}
{पत्तिकोट्यस्तथा तिस्रो बलमेतत्तवाभवत्}


\twolineshloka
{रथानां षट््सहस्राणि षट्सहस्राश्च कुञ्जराः}
{दश चाश्वसहस्राणि पत्तिकोटी च भारत}


\twolineshloka
{एतद्बलं पाण्डवानामभवच्छेषमाहवे}
{एत एव समाजग्मुर्युद्वाय भरतर्षभ}


\twolineshloka
{एवं विभज्य राजेन्द्र मद्रराजमते स्थिताः}
{पाण्डवान्प्रत्युदीयाम जयगृद्धाः प्रमन्यवः}


\twolineshloka
{तथैव पाण़्डवाः शूराः समरे जितकाशिनः}
{उपयाता नरव्याघ्राः पाञ्चालाश्च यशस्विनः}


\twolineshloka
{एवमेते बलौघेन परस्परवधैषिणः}
{उपयाता नरव्याघ्राः पूर्वां सन्ध्यां प्रति प्रभो}


\twolineshloka
{ततः प्रववृते युद्धं घोररूपं भयानकम्}
{तावकानां परेषां च निघ्नतामितरेतरम्}


\chapter{अध्यायः ९}
\twolineshloka
{सञ्जय उवाच}
{}


\twolineshloka
{ततः प्रववृते युद्धं कुरूणां भयवर्धनम्}
{सृञ्जयैः सह राजेन्द्र घोरं देवासुरोपमम्}


\twolineshloka
{नरा रथा गजौघाश्च वाजिनश्च सहस्रशः}
{वाजिनश्च पराक्रान्ताः समाजग्मुः परस्परम्}


\twolineshloka
{गजानां भीमरूपाणां द्रवतां निःस्वनो महान्}
{अश्रूयत यथा काले जलदानां नभस्तले}


\twolineshloka
{नागैरभ्याहताः केचित्सरथा रथिनोऽपतन्}
{व्यद्रवन्त रणे भीता द्राव्यमाणा मदोत्कटैः}


\twolineshloka
{हयौघान्पादरक्षांश्च रथिनस्तत्र शिक्षिताः}
{शरैः सम्प्रेषयामासुः परलोकाय भारत}


\twolineshloka
{सादिनः शिक्षिता राजन्परिवार्य महारथान्}
{विचरन्तो रणेऽभ्यघ्नन्प्रासशक्त्यृष्टिभिस्तथा}


\twolineshloka
{धन्विनः पुरुषाः केचित्परिवार्य महारथान्}
{एकं बहव आसाद्य प्रैषयन्यमसादनम्}


\twolineshloka
{नागान्रथवरांश्चान्ये परिवार्य महारथाः}
{सोत्तरा युधि निर्जघ्नुर्द्रवमाणं महारथम्}


\twolineshloka
{तथा च रथिनं क्रुद्धं विकिरन्तं शरान्बहून्}
{नागा जघ्नुर्महाराज परिवार्य समन्ततः}


\twolineshloka
{नागा नागमभिद्रुत्य रथी च रथिनं रणे}
{शक्तितोमरनाराचैर्निजघ्नुस्तत्र भारत}


\twolineshloka
{पादातानवमृद्गन्तो रथवारणवाजिनः}
{रणमध्ये व्यदृश्यन्त कुर्वन्तो महदाकुलम्}


\twolineshloka
{हयाश्च पर्यधावन्त चामरैरुपशोभिताः}
{हंसा हिमवतः प्रस्थे पिबन्त इव मेदिनीम्}


\twolineshloka
{तेषां तु वाजिनां भूमिः खुरैश्चित्रा विशाम्पते}
{अशोभत यथा नारी करजैः क्षतविक्षता}


\twolineshloka
{वाजिनां खुरशब्देन रथनेमिस्वनेन च}
{पत्तीनां चापि शब्देन नागानां बृंहितेन च}


\twolineshloka
{वादित्राणां च घोषेण शङ्खानां निनदेन च}
{अभवन्नादिता भूमिर्निर्घातैरिव भारत}


\twolineshloka
{धनुषां कूजमानानां शस्त्रौघानां च पात्यताम्}
{कवचानां प्रभाभिश्च न प्राज्ञायत किञ्चन}


\twolineshloka
{बहवो बाहवश्छिन्ना नागराजकरोपमाः}
{उद्वेष्टन्ते विचेष्टन्ते वेगं कुर्वन्ति दारुणम्}


\twolineshloka
{शिरसां च महाराज पततां धरणीतले}
{च्युतानामिव तालेभ्यः फलानां श्रूयते स्वनः}


\twolineshloka
{शिरोभिः पतितैर्भाति रुधिरार्द्रैर्वसुन्धरा}
{तपनीयनिभैः काले नलिनैरिव भारत}


\twolineshloka
{उद्वृत्तनयनैस्तैस्तु गतसत्वैः सुविक्षतैः}
{व्यभ्राजत मही राजन्पुण्डरीकैरिवावृता}


\twolineshloka
{बाहुभिश्चन्दनादिग्धैः सकेयूरैर्महारधनैः}
{पतितैर्भाति राजेन्द्र महाशक्रध्वजैरिव}


\twolineshloka
{ऊरुभिश्च नरेन्द्राणां विनिकृत्तैर्महाहवे}
{हस्तिहस्तोपमैर्जज्ञे संवृतं तद्रणाङ्कणम्}


\twolineshloka
{कबन्धंशतसङ्कीर्णं छत्रचामरसङ्कुलम्}
{सेनावनं तच्छुशुभे वनं पुष्पाचितं यथा}


\twolineshloka
{तत्र योधा महाराज विचरन्तो ह्यभीतवत्}
{दृश्यन्ते रुधिराक्ताङ्गाः पुष्पिता इव किंशुकाः}


\twolineshloka
{मातङ्गाश्चाप्यदृश्यन्त शरतोमरपीडिताः}
{पतन्तस्तत्र तवैव छिन्नाभ्रासदृशा रणे}


\twolineshloka
{गजानीकं महाराज वध्यमानं महात्मभिः}
{व्यदीर्यत दिशः सर्वा वातनुन्ना धना इव}


\twolineshloka
{ते गजा धनसङ्काशाः पेतुरुर्व्यां समन्ततः}
{वज्रनुन्ना इव बभुः पर्वता युगसङ्क्षये}


\twolineshloka
{हयानां सादिभिः सार्धं पतितानां महीतले}
{राशयः स्म प्रदृश्यन्ते गिरिमात्रांस्ततस्ततः}


\twolineshloka
{सञ्जज्ञे रणभूमौ तु परलोकवहा नदी}
{शोणितोदा रथावर्ता ध्वजवृक्षास्थिशर्करा}


\twolineshloka
{भुजनक्रा धनुःस्रोता हस्तिशैला हयोपला}
{मेदोमज्जाकर्दमिनी छत्रहंसा गदोडुपा}


\twolineshloka
{कवचोष्णीषसञ्छन्ना पताकारुचिरद्रुमा}
{चक्रचक्रावलीजुष्टा त्रिवेणूरगसंवृता}


\twolineshloka
{शूराणां हर्षजननी भीरूणां भयवर्धनी}
{प्रावर्तत नदी रौद्रा कुरुसृञ्जयसङ्गमे}


\twolineshloka
{तां नदीं परलोकाय वहन्तीमतिभैरवाम्}
{तेरुर्वाहननौभिस्ते शूराः परिघबाहवः}


\twolineshloka
{वर्तमाने तदा युद्धे निर्मर्यादे विशाम्पते}
{चतुरङ्गक्षये घोरे युद्धे देवासुरोपमे}


\twolineshloka
{व्याक्रोशन्बान्धवानन्ये तत्र तत्र परन्तप}
{क्रोशद्भिर्दयितैरन्ये भयार्ता न निवर्तिरे}


\twolineshloka
{निर्मर्यादे तथा युद्धे वर्तमाने भयानके}
{अर्जुनो भीमसेनश्च मोहयाञ्चक्रतुः परान्}


\twolineshloka
{सा वध्यमाना महती सेना तव नराधिप}
{अमुह्यत्तत्र तत्रैव योषिन्मदवशादिव}


\twolineshloka
{मोहयित्वा च तां सेनां भीमसेनधनञ्जयौ}
{दध्मतुर्वारिजौ तत्र सिंहनादांश्च चक्रतुः}


\twolineshloka
{श्रुत्वैव तु महाशब्दं धृष्टद्युम्नशिखण्डिनौ}
{धर्मराजं पुरस्कृत्य मद्रराजमभिद्रुतौ}


\twolineshloka
{तत्राश्चर्यमपश्याम घोररूपं महद्भयम्}
{शल्येन सङ्गताः शूरा यदयुध्यन्त भागशः}


\twolineshloka
{माद्रीपुत्रौ तु रभसौ कृतास्त्रौ युद्धदुर्मदौ}
{अभ्ययातां त्वरायुक्तौ जिगीषन्तौ बलं तव}


\twolineshloka
{ततोऽभ्यावर्तत बलं तावकं भरतर्षभ}
{शरैः प्रणुन्नं बहुधा पाण्डवैर्जितकाशिभिः}


\twolineshloka
{वध्यमाना चमूः सा तु पुत्राणां प्रेक्षतां तव}
{भेजे दश दिशो राजन्प्रणुन्ना शरवृष्टिभिः}


\twolineshloka
{हाहाकारो महाञ्जज्ञे योधानां तत्र भारत}
{तिष्ठतिष्ठेति चाप्यासीद्द्रावितानां महात्मनाम्}


\twolineshloka
{क्षत्रियाणां तथाऽन्योन्यं संयुगे जयमिच्छताम्}
{प्राद्रवन्नेव सम्भग्नाः पाण्डवैस्तव सैनिकाः}


\twolineshloka
{त्यक्त्वा युद्धे प्रियान्पुत्रान्भ्रातॄनथ पितामहान्}
{मातुलान्भागिनेयांश्च वयस्यानपि भारत}


\twolineshloka
{हयान्द्विपांस्त्वरयन्तो योधा जग्मुः समन्ततः}
{आत्मत्राणकृतोत्साहास्तावका भरतर्षभ}


\chapter{अध्यायः १०}
\twolineshloka
{सञ्जय उवाच}
{}


\twolineshloka
{तत्प्रभग्नं बलं दृष्ट्वा मद्रराजः प्रतापवान्}
{उवाच सारथिं तूर्णं चोदयाश्वान्महाजवान्}


\twolineshloka
{एष तिष्ठति वै राजा पाण्डुपुत्रो युधिष्ठिरः}
{छत्रेण ध्रियमाणेन पाण्डुरेण विराजता}


\twolineshloka
{अत्र मां प्रापय क्षिप्रं पश्य मे सारथे बलम्}
{न समर्था हि मे पार्थाः स्थातुमद्य पुरो युधि}


\twolineshloka
{एवमुक्तस्ततः प्रायान्मद्रराजस्य सारथिः}
{यत्रा राजा सत्यसन्धो धर्मपुत्रो युधिष्ठिरः}


\twolineshloka
{आपतन्तं च सहसा पाण्डवानां महद्बलम्}
{दधारैको रणे शल्यो वेलोद्वृत्तमिवार्णवम्}


\twolineshloka
{पाण्डवानां बलौघस्तु शल्यमासाद्य मारिष}
{व्यतिष्ठत तदा युद्वे सिन्धोर्वेग इवाचलम्}


\twolineshloka
{मद्रराजं तु समरे दृष्ट्वा युद्धाय धिष्ठितम्}
{कुरवः सन्न्यवर्तन्त मृत्युं कृत्वा निवर्तनम्}


\twolineshloka
{तेषु राजन्निवृत्तेषु व्यूढानीकेषु सर्वशः}
{प्रावर्तत महारौद्रः सङ्ग्रामः शोणितोदकः}


\twolineshloka
{समार्च्छच्चित्रसेनं तु नकुलो युद्धदुर्मदः}
{तौ परस्परमासाद्य चित्रकार्मुकधारिणौ}


\twolineshloka
{मेघाविव यथोद्वृत्तौ दक्षिणोत्तरवर्षिणौ}
{शरतोयैः सिषिचतुस्तौ परस्परमाहवे}


\threelineshloka
{नान्तरं तत्र पश्यामः पाण्डवस्येतरस्य च}
{उभौ कृतास्त्रौ बलिनौ रथचर्याविशारदौ}
{परस्परवधे यत्तौ छिद्रान्वेषणतत्परौ}


\twolineshloka
{चित्रसेनस्तु भल्लेन पीतेन निशितेन च}
{नकुलस्य महाराज मुष्टिदेशेऽच्छिनद्धनुः}


\twolineshloka
{अथैनं छिन्नधन्वानं रुक्मपुङ्खैः शिलाशितैः}
{त्रिभिः शरैरसम्भ्रान्तो ललाटे वै समार्पयत्}


\twolineshloka
{हयांश्चास्य शरैस्तीक्ष्णैः प्रेषयामास मृत्यवे}
{तथा ध्वजं सारथिं च त्रिभिस्त्रिभिरपातयत्}


\twolineshloka
{स शत्रुभुजनिर्मुक्तैर्ललाटस्थैस्त्रिभिः शरैः}
{नकुलः शुशुभे राजंस्त्रिशृङ्ग इव पर्वतः}


\twolineshloka
{स च्छिन्नधन्वा विरथः खङ्गमादाय चर्म च}
{रथादवातरद्वीरः शैलाग्रादिव केसरी}


\twolineshloka
{पद्मामापततस्तस्य शस्वृष्टिं समासृजत्}
{नकुलोऽप्यग्रसत्तां वै चर्मणा लघुविक्रमः}


\twolineshloka
{चित्रसेनरथं प्राप्य चित्रयोधी जितश्रमः}
{आरुरोह महाबाहुः सर्वसैन्यस्य पश्यतः}


\twolineshloka
{सकुण्डलं समुकुटं सुनसं स्वायतेक्षणम्}
{चित्रसेनशिरः कायादपाहरत पाण्डवः}


% Check verse!
स पपात रथात्तस्माद्दिवाकरसमद्युतिः
\twolineshloka
{चित्रसेनशिरस्तत्तु दृष्ट्वा तत्र महारथाः}
{साधुवादस्वनांश्चक्रुः सिंहनादांश्च पुष्कलान्}


\twolineshloka
{विशस्तं भ्रातरं दृष्ट्वा कर्णपुत्रौ महारथौ}
{सुशर्मा सत्यसेनश्च मुञ्चन्तौ विविधाञ्शरान्}


\twolineshloka
{ततोऽभ्यधावतां तूर्णं पाण्डवं रथिनां वरम्}
{जिघांसन्तौ यथा नागं व्याघ्रौ राजन्महावने}


\twolineshloka
{तावभ्येत्य महाबाहू द्वावप्यतिमहारथौ}
{शरौषान्सम्यगस्यन्तौ जीमूतौ सलिलं यथा}


\threelineshloka
{स शरैः सर्वतो विद्धः प्रहृष्ट इव पाण्डवः}
{अन्यत्कार्मुकमादाय रथमारुह्य वेगवान्}
{अतिष्ठत रणे वीरः क्रुद्धरूप इवान्तकः}


\twolineshloka
{तस्य तौ भ्रातरौ राजञ्शरैः सन्नतपर्वभिः}
{रथं विशकलीकर्तुं समारब्धौ विशाम्पते}


\twolineshloka
{ततः प्रहस्य नकुलश्चतुर्भिश्चतुरो रणे}
{जघान निशितैर्बाणैः सत्यसेनस्य वाजिनः}


\twolineshloka
{ततः सन्धाय नारचं रुक्मपुङ्खं शिलाशितम्}
{धनुश्चिच्छेद राजेन्द्र सत्यसेनस्य पाण्डवः}


\twolineshloka
{अथान्यं रथमास्थाय धनुरादाय चापरम्}
{सत्यसेनः सुशर्मा च पाण्डवं पर्यधावताम्}


\twolineshloka
{अविध्यत्तावसम्भ्रान्तौ माद्रीपुत्रः प्रतापवान्}
{द्वाभ्यां द्वाभ्यां महाराज शराभ्यां रणमूर्धनि}


\twolineshloka
{सुशर्मा तु ततः क्रुद्धः पाण्डवस्य महद्धनुः}
{चिच्छेद प्रहसन्युद्धे क्षुरप्रेण महारथः}


\twolineshloka
{अथान्यद्धनुरादाय नकुलः क्रोधमूच्छितः}
{सुशर्माणं पञ्चभिर्विद्ध्वा ध्वजमेकेन चिच्छिदे}


\twolineshloka
{सत्यसेनस्य स धनुर्हस्तावपं च मारिष}
{चिच्छेद तरसा युद्धे तत उच्चुक्रुशुर्जनाः}


\twolineshloka
{अथान्यद्धनुरादाय वेगघ्नं भारसाधनम्}
{शरैः सञ्छादयामास समन्तात्पाण्डुनन्दनम्}


\twolineshloka
{सन्निवार्य तु तान्बाणान्नकुलः परवीरहा}
{सत्यसेनसुशर्माणौ द्वाभ्यां द्वाभ्यामविध्यत}


\twolineshloka
{तावेनं प्रत्यविध्येतां पृथक्पृथगजिह्मगैः}
{सारथिं चास्य राजेन्द्र शितैर्विव्यधतुः शरैः}


\twolineshloka
{सत्यसेनो रथेषां तु नकुलस्य धनुस्तथा}
{पृथक् शराभ्यां चिच्छेद कृतहस्तः प्रतापवान्}


\twolineshloka
{स रथेऽतिरथस्तिष्ठन्रथशक्तिं परामृशत् ॥स्वर्णदण्डामकुण्ठाग्रां तैलधौतां सुनिर्मलाम्}
{}


% Check verse!
लेलिहानामिव विमो नागकन्यां महाविषाम्
% Check verse!
समुद्यम्य च चिक्षेप सत्यसेनस्य संयुगे
\twolineshloka
{सा तस्य हृदयं गत्या विभेद शतधा नृप}
{स पवात रथाद्भूमिं यतसत्वोऽल्पत्तेतनः}


\twolineshloka
{भ्रातरं निहतं दृष्ट्वा सुशर्मा क्रोधमूर्च्छितः}
{अभ्यवर्षच्छरैस्तूर्णं पादातं पाण्डुनन्दनम्}


\twolineshloka
{चतुर्भिश्चतुरो वाहान्ध्वजं छित्त्वा च पञ्चभिः}
{त्रिभिर्वै सारथिं हत्वा कर्णपुत्रो ननाद ह}


\twolineshloka
{नकुलं विरथं दृष्ट्वा द्रौपदेयो महारथम्}
{सुतसोमोऽभिदुद्राव परीप्सन्पितरं रणे}


\twolineshloka
{ततोऽधिरुह्य नकुलः सुतसोमस्य तं रथम्}
{शुशुभे भरतश्रेष्ठो गिरिस्य इव केसरी}


% Check verse!
अन्यत्कार्मुकमादाय सुशर्माणमयोधयत्
\twolineshloka
{तत्र तौ शरवर्षाभ्यां समासाद्य परस्परम्}
{परस्परवधे यत्नं चक्रतुः सुमहारथौ}


\twolineshloka
{सुशर्मा तु तः क्रुद्धः पाण्डवं विशिखैस्त्रिभिः}
{सुतसोमं तु विंशत्या वाह्वोरुरसि चार्पयत्}


\twolineshloka
{ततः क्रुद्धो महाराज नकुलः परवीरहा}
{शरैस्तस्य दिशः सर्वाश्छादयामास वीर्यवान्}


\twolineshloka
{ततो गृहीत्वा तीक्ष्णाग्रमर्धचन्द्रं सुतेजनम्}
{आकर्णपूर्णं चिक्षेप कर्णपुत्राय संयुगे}


\twolineshloka
{तस्य तेन शिरः कायाज्जहार नृपसत्तम}
{पश्यतां सर्वसैन्यानां तदद्भुतमिवाभवत्}


\twolineshloka
{स हतः प्रापतद्राजन्नकुलेन महात्मना}
{नदीपेxxxxxxxणस्तीरजः पादपो महान्}


\twolineshloka
{कर्णपुत्रवचं दृष्ट्वा नकुलस्य च विक्रमम्}
{प्रदुद्राव भवात्सेना तावकी भरतर्वभ}


\twolineshloka
{तां तु सेनां महाराज्ञ मद्रराजः प्रतापवान्}
{xxxxxx शूरः सेनापतिररिन्दमः}


\twolineshloka
{xxxxxx व्यवस्याप्य च वाहिनीम्}
{xxxxxxx भूशं कृत्वा धनुःशब्दं च दारुणम्}


\twolineshloka
{xxxxxx सगरे राजन्रांक्षेता दृढधन्वना}
{प्रत्युद्ययुश्च तांस्ते तु समन्ताद्विगतव्यथाः}


\twolineshloka
{मद्रराजं महेष्वासं परिवार्य समन्ततः}
{स्थिता राजन्महासेना योद्वुकामा समन्ततः}


\twolineshloka
{सात्यकिर्भीमसेनश्च माद्रीपुत्रौ च पाण्डवौ}
{युधिष्ठिरं पुरस्कृत्य हीनिषेवमरिन्दमम्}


\twolineshloka
{परिवार्य रणे वीराः सिंहनादं प्रचक्रिरे}
{बाणशङ्खरवांस्तीव्रान्क्ष्वेलाश्च विविधा दधुः}


\twolineshloka
{तथैव तावकाः सर्वे मद्राधिपतिमञ्जसा}
{परिवार्य सुसंरब्धाः पुनर्युद्धमरोचयन्}


\twolineshloka
{ततः प्रववृते युद्धं भीरूणां भयवर्धनम्}
{तावकानां परेषां च मृत्युं कृत्वा निबर्तनम्}


\twolineshloka
{यथा देवासुरं युद्धं पूर्वमासीद्विशाम्पते}
{अभीतानां तथाऽऽसीत्तद्यमराष्ट्रविवर्धनम्}


\twolineshloka
{ततः कपिध्वजो राजन्हत्वा संशप्तकान्रणे}
{अभ्यद्रवत तां सेनां कौरवीं पाण्डुनन्दनः}


\twolineshloka
{तथैव पाण्डवाः सर्वे धृष्टद्युम्नपुरोगमाः}
{अभ्यधावन्ततां सेनां विसृजन्तः शिताञ्शरान्}


\twolineshloka
{पाण्डवैरवकीर्णानां सम्मोहः समजायत}
{न च जज्ञुस्त्वनीकानि दिशो वा विदिशस्तथा}


\threelineshloka
{आपूर्यमाणा निशितैः शरैः पाण्डवचोदितैः}
{हतप्रवीरा विध्वस्ता वार्यमाणा समन्ततः}
{कौरव्यवध्यत चमूः पाण्डुपुत्रैर्महारथैः}


\twolineshloka
{तथैव पाण्डवं सैन्यं शरै राजन्समन्ततः}
{रणेऽहन्यत पुत्रैस्ते शतशोऽथ सहस्रशः}


\twolineshloka
{ते सेने भृशसन्तप्ते वध्यमाने परस्परम्}
{व्याकुले समपद्येतां वर्षासु सरिताविव}


\twolineshloka
{आविवेश ततस्तीव्रं तावकानां महद्भयम्}
{पाण्डवानां च राजेन्द्र तथाभूते महाहवे}


\chapter{अध्यायः ११}
\twolineshloka
{सञ्जय उवाच}
{}


\twolineshloka
{तस्मिन्विलुलिते सैन्ये वध्यमाने परस्परम्}
{द्रवमाणेषु योधेषु विद्रवत्सु च दन्तिषु}


\twolineshloka
{कूजतां स्तनतां चैव पदातीनां महाहवे}
{निहतेषु महाराज हयेषु बहुधा तदा}


\twolineshloka
{प्रक्षये दारुणे घोरे संहारे सर्वदेहिनाम्}
{नानाशस्त्रसमावापे व्यतिषक्तरथद्विपे}


\twolineshloka
{हर्षणे युद्धशौण्डानां भीरूणां भयवर्धने}
{गाहमानेषु योधेषु परस्परवधैषिषु}


\twolineshloka
{प्राणादाने महाघोरे वर्तमाने दुरोदरे}
{सङ्ग्रामे घोररूपे तु यमराष्ट्रविवर्धने}


\twolineshloka
{पाण्डवास्तावकं सैन्यं व्यधमन्निशितैः शरैः}
{तथैव तावका योधा जघ्नुः पाण्डवसैनिकान्}


\twolineshloka
{तस्मिंस्तथा वर्तमाने युद्धे भीरुभयावहे}
{पूर्वाह्णे चापि सम्प्राप्ते भास्करोदयनं प्रति}


\twolineshloka
{लब्धलक्षाः परे राजन्रक्षितास्तु महात्मना}
{अयोधयंस्तव बलं मृत्युं कृत्वा निवर्तनम्}


\twolineshloka
{बलिभिः पाण्डवैर्दृप्तैर्लैब्धलक्षैः प्रहारिभिः}
{कौरव्यसीदत्पृतना मृगीवाग्निबयाकुला}


\twolineshloka
{तां दृष्ट्वा सीदतीं सेनां पङ्के गामिव दुर्बलाम्}
{उज्जिहीर्षुस्तदा शल्यः प्रायात्पाण्डुसुतान्प्रति}


\twolineshloka
{मद्रराजः सुसङ्क्रुद्धो गृहीत्वा धनुरुत्तमम्}
{अभ्यद्रवत सङ्ग्रामे पाण्डवानाततायिनः}


\twolineshloka
{पाण्डवा अपि भूपाल समरे जितकाशिनः}
{मद्रराजं समासाद्य बिभिदुर्निशितैः शरैः}


\twolineshloka
{ततः शरशतैस्तीक्ष्णैर्मद्रराजो महारथः}
{अर्दयामास तां सेनां धर्मराजस्य पश्यतः}


\twolineshloka
{प्रादुरासन्निमित्तानि नानारूपाण्यनेकशः}
{चचाल शब्दं कुर्वाणा मही चापि सपर्वता}


\twolineshloka
{[सदण्डशूला दीप्ताग्रा दीर्यमाणाः समन्ततः}
{]उल्का भूमिं दिवः पेतुराहत्य रविमण्डलम्}


\twolineshloka
{मृगाश्च महिषाश्चापि पक्षिणश्च विशाम्पते}
{अपसव्यं तदा चक्रुः सेनां ते बहुशो नृप}


\twolineshloka
{[भृगुसूनुधरापुत्रौ शशिजेन समन्वितौ}
{चरमं पाण्डुपुत्राणां पुरस्तात्सर्वभूभुजाम्}


\twolineshloka
{शस्त्राग्रेष्वभवज्ज्वाला नेत्राण्याहत्य वर्षती}
{शिरः स्वलीयन्त भृशं काकोलूकाश्च केतुषु]}


\twolineshloka
{ततस्तद्युद्धमत्युग्रमभवत्सहचारिणाम्}
{तथा सर्वाण्यनीकानि सन्निपत्य जनाधिप}


\twolineshloka
{अभ्यघ्नत्कौरवो राजा पाण्डवानामनीकिनीम्}
{शल्यस्तु शरवर्षेण वर्षन्निव सहस्रदृक्}


\twolineshloka
{अभ्यवर्षत धर्मात्मा कुन्तीपुत्रं युधिष्ठिरम्}
{भीमसेनं शरैश्चापि रुक्मपुङ्खैः शिलाशितैः}


\twolineshloka
{द्रौपदेयांस्तथा सर्वान्माद्रीपुत्रौ च पाण्डवौ}
{धृष्टद्युम्नं च शैनेयं शिखण्डिनमथापि च}


\twolineshloka
{एकैकं दशभिर्बाणैर्विव्याध स महाबलः}
{ततोऽसृजद्बाणवर्षं घर्मान्ते मघवानिव}


\twolineshloka
{ततः प्रभद्रका राजन्सोमकाश्च सहस्रशः}
{पतिताः पात्यमानाश्च दृश्यन्ते शल्यसायकैः}


\twolineshloka
{भ्रमराणामिव व्राताः शलभानामिव व्रजाः}
{हादिन्य इव मेघेभ्यः शल्यस्य न्यपतञ्शराः}


\twolineshloka
{द्विरदास्तुरगाश्चार्ताः पत्तयो रथिनस्तथा}
{शल्यस्य बाणैरपतन्बभ्रमुर्व्यनदंस्तथा}


\threelineshloka
{आविष्ट इव मद्रेशो मन्युना पौरुषेण च}
{प्राच्छादयदरीन्सङ्ख्ये कालसृष्ट इवान्तकः}
{विनर्दमानो मद्रेशो मेघहादो महाबलः}


\twolineshloka
{सा वध्यमाना शल्येन पाण्डवानामनीकिनी}
{अजातशत्रुं कौन्तेयमभ्यधावद्युधिष्ठिरम्}


\twolineshloka
{तां सम्मर्द्य शतैः सङ्ख्ये लघुहस्तः शितैः शरैः}
{बाणवर्षेण महता युधिष्ठिरमताडयत्}


\twolineshloka
{तमापतन्तं जात्यश्वैः क्रुद्धो राजा युधिष्ठिरः}
{अवारयच्छरैस्तीक्ष्णैर्महाद्विपमिवाङ्कुशैः}


\twolineshloka
{तस्य शल्यः शरं घोरं मुमोचाशीविषोपमम्}
{सोऽभ्यविध्यन्महात्मानं वेगेनाभ्यपतच्च गाम्}


\twolineshloka
{ततो वृकोदरः क्रुद्धः शल्यं विव्याध सप्तभिः}
{पञ्चभिः सहदेवस्तु नकुलो दशभिः शरैः}


\twolineshloka
{द्रौपदेयाश्च शत्रुघ्नं शूरमार्तायनिं शरैः}
{अभ्यवर्षन्महाराज मेघा इव महीधस्म्}


\twolineshloka
{ततो दृष्ट्वा वार्यमाणं शल्यं पार्थैः समन्ततः}
{कृतवर्मा कृपश्चैव सङ्क्रुद्धावभ्यधावताम्}


\threelineshloka
{उलूकश्च महावीर्यः शकुनिंश्चापि सौबलः}
{समागम्याथ शनकैरश्वत्थामा महाबलः}
{तव पुत्राश्च कार्त्स्न्येन जुगुपुः शल्यमाहवे}


\threelineshloka
{भीमसेनं त्रिभिर्विद्धा कृतवर्मा शिलीमुखैः}
{बाणवर्षेण महता क्रुद्धरूपमवारयत्}
{धृष्टद्युम्नं ततः क्रुद्धो बाणवर्षैरपीडयत्}


% Check verse!
द्रौपदेयांश्च शकुनिर्यमौ च द्रौणिरभ्ययात्
\twolineshloka
{दुर्योधनो युधांश्रेष्ठ आहवे केशवार्जुनौ}
{समभ्ययादुग्रतेजाः शरैश्चाप्यहनद्बली}


\twolineshloka
{एवं द्वन्द्वशतान्यासंस्त्वदीयानां परैः सह}
{घोररूपाणि चित्राणि तत्रतत्र विशाम्पते}


\threelineshloka
{ऋक्षवर्णाञ्जघानाश्वान्भोजो भीमस्य संयुगे}
{सोऽवतीर्य रथोपस्थाद्धताश्वात्पाण्डुनन्दनः}
{कालो दण्डमिवोद्यम्य गदापाणिरयुध्यत}


\twolineshloka
{प्रमुखे सहदेवस्य जघानाश्वान्स मद्रराट्}
{ततः शल्यस्य तनयं सहदेवोऽसिनावधीत्}


\twolineshloka
{गौतमः पुनराचार्यो धृष्टद्युम्नमयोधयत्}
{असम्भ्रान्तमसम्भ्रान्तो यत्नवान्यत्नवत्तरम्}


\twolineshloka
{द्रौपदेयांस्तथा वीरानेकैकं दशभिः शरैः}
{अविद्ध्यदाचार्यसुतो नातिक्रुद्धो हसन्निव}


\twolineshloka
{[पुनश्च भीमसेनस्य जघानाश्वांस्तथाऽऽहवे}
{सोऽवतीर्य रथात्तूर्णं हताश्वः पाण्डुनन्दनः}


\threelineshloka
{कालो दण़्डमिवोद्यम्य गदां क्रुद्धो महाबलः}
{पोथयामास तुरगान्रथं च कृतवर्मणः}
{कृतवर्मा त्ववप्लुत्य रथात्तस्मादपाक्रमत् ॥]}


\twolineshloka
{शल्योऽपि राजन्सङ्क्रुद्धो निघ्नन्सोमकपाण्डवान्}
{पुनरेव शितैर्बाणैर्युधिष्ठिरमपीडयत्}


\twolineshloka
{तस्य भीमो रणे क्रुद्धः सन्दश्य दशनच्छदम्}
{विनाशायाभिसन्धाय गदामादाय वीर्यवान्}


\twolineshloka
{यमदण्डप्रतीकाशां कालरात्रिमिवोद्यताम्}
{गजवाजिमनुष्याणां देहान्तकरणीमति}


\twolineshloka
{हेमपट्टपरिक्षिप्तामुल्कां प्रज्वलितामिव}
{शैक्यां व्यालीमिवात्युग्रां वज्रकल्पामयोमयीम्}


\twolineshloka
{चन्दनागुरुपङ्काक्तां प्रमदामीप्सितामिव}
{वसामेदोपदिग्धाङ्गीं जिह्वां वैवस्वतीमिव}


\twolineshloka
{पटुघण्टाशतरवां वासवीमशनीमिव}
{निर्मुक्ताशीविषाकारां पृक्तां गजमदैरपि}


\twolineshloka
{त्रासनीं सर्वभूतानां स्वसैन्यपरिहर्षिणीम्}
{मनुष्यलोके विख्यातां गिरिशृङ्गविदारणीम्}


\twolineshloka
{यया कैलासभवने महेश्वरसखं बली}
{आह्वयामास कौन्तेयः सङ्क्रुद्धमलकाधिपम्}


\threelineshloka
{यया मायामयान्दृप्तान्सुबहून्धनदालये}
{जघान गुह्यकान्क्रुद्धो मन्दारार्थे महाबलः}
{निवार्यमाणो बहुभिर्द्रौपद्याः प्रियमास्थितः}


\twolineshloka
{तां वज्रमणिरत्नौघकल्माषां वज्रगौरवाम्}
{समुद्यम्य महाबाहुः शल्यमभ्यपतद्रणे}


\twolineshloka
{गदया युद्धकुशलस्तया दारुणनादया}
{पोथयामास शल्यस्य चतुरोऽश्वान्महाजवान्}


\twolineshloka
{ततः शल्यो रणे क्रुद्धः पीने वक्षसि तोमरम्}
{निचखान नदन्वीरोवर्म भित्त्वा च सोभ्ययात्}


\twolineshloka
{वृकोदरस्त्वसम्भ्रान्तस्तमेवोद्धृत्य तोमरम्}
{यन्तारं मद्रराजस्य निर्बिभेद तदा हृदि}


\twolineshloka
{स भिन्नवर्मा रुधिरं वमन्वित्रस्तमानसः}
{पपाताभिमुखो भीमं मद्रराजस्त्वपाक्रमत्}


\twolineshloka
{कृतप्रतिकृतं दृष्ट्वा शल्यो विस्मितमानसः}
{गदामाश्रित्य धर्मात्मा प्रत्यमित्रमवैक्षत}


\twolineshloka
{ततः सुमनसः पार्था भीमसेनमपूजयन्}
{ते दृष्ट्वा कर्म सङ्ग्रामे घोरमक्लिष्टकर्मणः}


\chapter{अध्यायः १२}
\twolineshloka
{सञ्जय उवाच}
{}


\twolineshloka
{पतितं प्रेक्ष्य यन्तारं शल्यः शैक्यायसीं गदाम्}
{आदाय तरसा राजंस्तस्थौ गिरिरिवाचलः}


\twolineshloka
{तं दीप्तमिव कालाग्निं पाशहस्तमिवान्तकम्}
{सशृङ्गमिव कैलासं सवज्रमिव वासवम्}


\threelineshloka
{सशूलमिव हर्यक्षं सचक्रमिव चक्रिणम्}
{सशक्तिमिव सेनान्यं वने मत्तमिव द्विपम्}
{जवेनाभ्यपतद्भीमः प्रगृह्य महतीं गदाम्}


\twolineshloka
{ततः शङ्खप्रणादश्च तूर्याणां च सहस्रशः}
{सिंहनादश्च सञ्जज्ञे शूराणां हर्षवर्धनः}


\twolineshloka
{प्रैक्षन्त सर्वतस्तौ हि योधा मत्ताविव द्विपौ}
{तावकाश्चापरे चैव साधुसाध्वित्यपूजयन्}


\twolineshloka
{न हि मद्राधिपादन्यो रामाद्वा यदुनन्दनात्}
{सोढुमुत्सहते वेगं भीमसेनस्य संयुगे}


\twolineshloka
{तथा मद्राधिपस्यापि गदावेगं महात्मनः}
{सोढुमुत्सहते नान्यः पुमान्युधि वृकोदरात्}


\twolineshloka
{तौ वृषाविव नर्दन्तौ मण्डलानि विचेरतुः}
{आवर्तितौ गदाहस्तौ मद्रराजवृकोदरौ}


\twolineshloka
{मण्डलावर्तमार्गेषु गदाविहरणेषु च}
{निर्विशेषमभूद्युद्धं तयोः पुरुषसिंहयोः}


\twolineshloka
{तप्तहेममयैः शुभ्रैर्बभूव भयवर्धिनी}
{अग्निज्वालैरिवाबद्धा पट्टैः शल्यस्य सा गदा}


\twolineshloka
{तथैव चरतो मार्गान्मण्डलेषु महात्मनः}
{विद्युदभ्रग्रतीकाशा भीमस्य शुशुभे गदा}


\twolineshloka
{ताडिता मद्रराजेन भीमस्य गदया गदा}
{दह्यमानेव खे राजन्साऽसृजत्पावकार्चिषः}


\twolineshloka
{तथा भीमेन शल्यस्य ताडिता गदया गदा}
{अङ्गारवर्षं मुमुचे तदद्भुतमिवाभवत्}


\twolineshloka
{दन्तैरिव महानागौ शृङ्गैरिव महर्षभौ}
{तौ विरेजतुरन्योन्यं गदाग्राभ्यां परिक्षतौ}


\twolineshloka
{तौ गदाभिहतैर्गात्रैः क्षणेन रुधिरोक्षितौ}
{प्रेक्षणीयतरावास्तां पुष्पिताविव किंशुकौ}


\twolineshloka
{गदया मद्रराजस्य सव्यदक्षिणमाहतः}
{भीमसेनो महाबाहुर्न चचालाचलो यथा}


\twolineshloka
{तथा भीमगदावेगैस्ताड्यमानो मुहुर्मुहुः}
{शल्यो न विव्यथे राजन्दन्तिनेव महागिरिः}


\twolineshloka
{शुश्रुवे दिक्षु सर्वासु तयोः पुरुषसिंहयोः}
{गदानिपातसंहादो वज्रयोरिव निःस्वनः}


\twolineshloka
{निवृत्य तु महावीर्यौ समुच्छ्रितमहागदौ}
{पुनरन्तरमार्गस्थौ पण्डलानि विचेरतुः}


\twolineshloka
{अथाभ्येत्य पदान्यष्टौ सन्निपातोऽभवत्तयोः}
{उद्यम्य लोहदण़्डाभ्यामतिमानुषकर्मणोः}


\twolineshloka
{पोथयन्तौ तदाऽन्योन्यं मण्डलानि विचेरतुः}
{क्रियाविशेषं कृतिनौ दर्शयामासतुस्तदा}


\twolineshloka
{अभ्युद्यतगदौ वीरौ सशृङ्गाविव पर्वतौ}
{तावाजघ्नतुरन्योन्यं मण्डलानि विचेरतुः}


\twolineshloka
{क्रियाविशेषकृतिनौ रणभूमितलेऽचलौ}
{तौ परस्परसंरम्भाद्गदाभ्यां सुभृशाहतौ}


\twolineshloka
{युगपत्पेततुर्वीरावुभाविन्द्रबली इव}
{उभयः सेनयोर्योधास्तदा हाहाकृताभवन्}


% Check verse!
भृशं मर्मस्वभिहतावुभावास्तां सुविह्वलौ
\twolineshloka
{ततः स्वरथमारोप्य मद्राणामृषभं रणे}
{अपोवाह कृपः शल्यं तूर्णमायोधनादथ}


\twolineshloka
{क्षीबवद्विह्वलत्वात्तु निमेषात्पुनरुत्थितः}
{भीमसेनो गदापाणिः समाह्वयत मद्रपम्}


\twolineshloka
{ततस्तु तावकाः शूरा नानाशस्त्रसमायुताः}
{नानावादित्रशब्देन पाण्डुसेनामयोधयन्}


\twolineshloka
{भुजावुच्छ्रित्य शस्त्रं च शब्देन महता ततः}
{अभ्यद्रवन्महाराज दुर्योधनपुरोगमाः}


\twolineshloka
{तदनीकमभिप्रेक्ष्य ततस्ते पाण्डुनन्दनाः}
{प्रययुः सिंहनादेन दुर्योधनवधेप्सया}


\twolineshloka
{तेषामापततां तूर्णं पुत्रस्ते भरतर्षभ}
{प्राप्तेन चेकितानं वै विव्याध हृदये भृशम्}


\twolineshloka
{पपात रथोपस्थे तव पुत्रेण पातितः}
{रुधिरौघपरिक्लिन्नः प्रविश्य विपुलं तमः}


\twolineshloka
{चेकितानं हतं दृष्ट्वा पाण्डवेया महारथाः}
{असक्तमभ्यवर्षन्त शरवर्षाणि भागशः}


\twolineshloka
{तावकानामनीकेषु पाण्डवा जितकाशिनः}
{व्यचरन्त महाराज प्रेक्षणीयाः समन्ततः}


\twolineshloka
{कृपश्च कृतवर्मा च सौबलश्च महारथः}
{अयोधयन्धर्मराजं मद्रराजपुरस्कृताः}


\twolineshloka
{भारद्वाजस्य हन्तारं भूरिवीर्यपराक्रमम्}
{दुर्योधनो महाराज धृष्टद्युम्नमयोधयत्}


\twolineshloka
{त्रिसाहस्रास्तथा राजंस्तव पुत्रेण चोदिताः}
{अयोधयन्त विजयं जैगर्तानां महारथाः}


\twolineshloka
{विजये धृतसङ्कल्पाः समरे त्यक्तजीविताः}
{प्राविशंस्तावका राजन्हंसा इव महत्सरः}


\twolineshloka
{ततो युद्धमभूद्धोरं परस्परवधैषिणाम्}
{अन्योन्यवधसंयुक्तमन्योन्यप्रीतिवर्धनम्}


\twolineshloka
{तस्मिन्प्रवृत्ते सङ्ग्रामे राजन्वीरवरक्षये}
{अनिलेनेरितं घोरमुत्तस्थौ पार्थिवं रजः}


\twolineshloka
{श्रवणान्नामधेयानां पार्थिवानां च कीर्तनात्}
{परस्परं विजानीमस्तदायुध्यन्नभीतवत्}


\twolineshloka
{तद्रजः पुरुषव्याघ्र शोणितेन प्रशामितम्}
{दिशश्च विमला जातास्तस्मिन्रजसि नाशिते}


\twolineshloka
{तथा प्रवृत्ते सङ्ग्रामे घोररूपे भयानके}
{तावकानां परेषां च नासीत्कश्चित्पराङ्मुखः}


\twolineshloka
{ब्रह्मलोकपरा भूत्वा प्रार्थयन्तो जयं युधि}
{सुयुद्धेन पराक्रान्ता नराः स्वर्गमभीप्सवः}


\twolineshloka
{भर्तृपिण्डविमोक्षार्थं भर्तृकार्यविनिश्चिताः}
{स्वर्गसंसक्तमनसो योधा युयुधिरे तदा}


\twolineshloka
{नानारूपाणि शस्त्राणि विसृजन्तो महारथाः}
{अन्योन्यमभिगर्जन्तः प्रहरन्तः परस्परम्}


\twolineshloka
{हत विध्यत गृह्णीत प्रहरध्वं निकृन्तत}
{इति स्म वाचः श्रूयन्ते तवतेषां च वै बले}


\twolineshloka
{ततः शल्यो महाराज धर्मपुत्रं युधिष्ठिरम्}
{विव्याध निशितैर्बाणैर्हन्तुकामो महारथम्}


\twolineshloka
{तस्य पार्थो महाराज नाराचान्वै चतुर्दश}
{मर्माण्युद्दिश्य मर्मज्ञो निचखान हसन्निव}


\twolineshloka
{आवार्य पाण्डवं बाणैर्हन्तुकामो महाबलः}
{विव्याध समरे क्रुद्धो बहुभिः कङ्कपत्रिभिः}


\twolineshloka
{अथ मद्रो महाराज शरेणानतपर्वणा}
{युधिष्ठिरं समाजघ्ने सर्वसैन्यस्य पश्यतः}


\twolineshloka
{धर्मराजोऽपि सङ्क्रुद्धो मद्रराजं महाबलः}
{विव्याध निशितैर्बाणैः कङ्कबर्हिणवाजितैः}


\twolineshloka
{चन्द्रसेनं च सप्तत्या सूतं च नवभिः शरैः}
{द्रुमसेनं चतुःषष्ट्या निजघान महारथः}


\twolineshloka
{चक्ररक्षे हते शल्यः पाण्डवेन महात्मना}
{निजघान ततो राजंश्चेदीन्वै प़ञ्चविंशतिम्}


\twolineshloka
{सात्यकिं पञ्चविंशत्या भीमसेनं च पञ्चभिः}
{माद्रीपुत्रौ शतेनाजौ विव्याध निशितैः शरैः}


\twolineshloka
{एवं विचरतस्तस्य सङ्ग्रामे राजसत्तम}
{सम्प्रैषयच्छितान्पार्थः शरानाशीविषोपमान्}


\twolineshloka
{ध्वजाघ्रं चास्य समरे कुन्तीपुत्रो युधिष्ठिरः}
{प्रमुखे वर्तमानस्य भल्लेनापाहरद्रथात्}


\twolineshloka
{पाण्डुपुत्रेण वै तस्य केतुं छिन्नं महात्मना}
{निपतन्तमपश्याम गिरिशृङ्गमिवाहतम्}


\twolineshloka
{ध्वजं निपतितं दृष्ट्वा पाण़्डवं च व्यवस्थितम्}
{सङ्क्रुद्धो मद्रराजोऽभूच्छरवर्षं मुमोच ह}


\twolineshloka
{शल्यः सायकवर्षेण पर्जन्य इव वृष्टिमान्}
{अभ्यवर्षदमेयात्मा क्षत्रियान्क्षत्रियर्षभः}


\twolineshloka
{सात्यकिं भीमसेनं च माद्रीपुत्रौ च पाण्डवौ}
{एकैकं प़ञ्चभिर्विद्ध्वा युधिष्ठिरमपीडयत्}


\twolineshloka
{ततो बाणमयं जालं विततं पाण्डवोरसि}
{अपश्याम महाराज मेघजालमिवोद्गतम्}


\twolineshloka
{तस्य शल्यो रणे क्रुद्धः शरैः सन्नतपर्वभिः}
{दिशः सञ्छादयामास प्रदिशश्च महारथः}


\twolineshloka
{ततो युधिष्ठिरो राजा बाणजालेन पीडितः}
{बभूव हृतविक्रान्तो जम्भो वृत्रहणा यथा}


\chapter{अध्यायः १३}
\twolineshloka
{सञ्जय उवाच}
{}


\threelineshloka
{पीडिते धर्मराजे तु मद्रराजेन मारिष}
{सात्यकिर्भीमसेनश्च माद्रीपुत्रौ च पाण्डवौ}
{परिवार्य रथैः शल्यं पीडयामासुराहवे}


\twolineshloka
{तमेकं बहुभिर्दृष्ट्वा पीड्यमानं महारथैः}
{साधुवादो महाञ्जज्ञे सिद्धाश्चासन्प्रहर्षिताः}


% Check verse!
आश्चर्यमित्यभाषन्त मुनयश्चापि सङ्गताः
\twolineshloka
{भीमसेनो रणे शल्यं शल्यभूतं पराक्रमे}
{एकेन विद्ध्वा बाणेन पुनर्विव्याध सप्तभिः}


\twolineshloka
{सात्यकिश्च शतेनैनं धर्मपुत्रपरीप्सया}
{मद्रेश्वरमवाकीर्य सिंहनादमथानदत्}


\twolineshloka
{नकुलः पञ्चभिश्चैनं सहदेवश्च प़ञ्चभिः}
{विद्ध्वा तं तु पुनस्तूर्णं ततो विव्याध सप्तभिः}


\twolineshloka
{स तु शूरो रणे यत्तः पीडितस्तैर्महारथैः}
{विकृष्य कार्मुकं घोरं भारघ्नं वेगवत्तरम्}


\twolineshloka
{सात्यकिं पञ्चविंशत्या शल्यो विव्याध मारिष}
{भीमसेनं तु सप्तत्या नकुलं सप्तभिस्तथा}


\twolineshloka
{ततः स विशिखं चापं सहदेवस्य धन्विनः}
{छित्त्वा भल्लेन समरे विव्याधैनं त्रिसप्तभिः}


\threelineshloka
{सहदेवस्तु समरे मातुलं भूरिवर्चसम्}
{सज्यमन्यद्धनुः कृत्वा पञ्चभिः समताडयत्}
{शरैराशीविषाकारैर्ज्वलज्ज्वलनसन्निभैः}


\twolineshloka
{सारथिं चास्य समरे शरेणानतपर्वणा}
{विव्याध भृशसङ्क्रुद्धस्तं वै भूयस्त्रिभिः शरैः}


\twolineshloka
{भीमसेनस्तु सप्तत्या सात्यकिर्नवभिः शरैः}
{धर्मराजस्तथा षष्ट्या गात्रे शल्यं समार्पयत्}


\twolineshloka
{ततः शल्यो महाराज निर्विद्धस्तैर्महारथैः}
{सुस्राव रुधिरं गात्रैर्गैरिकं पर्वतो यथा}


\twolineshloka
{तांश्च सर्वान्महेष्वासान्पञ्चभिः पञ्चभिः शरैः}
{विव्याध तरसा राजंस्तदद्भुतमिवाभवत्}


\twolineshloka
{ततोऽपरेण भल्लेन धर्मपुत्रस्य मारिष}
{धनुश्चिच्छेद समरे सज्जयं स सुमहारथः}


\twolineshloka
{अथान्यद्धनुरादाय धर्मपुत्रो युधिष्ठिरः}
{साश्वसूतध्वजरथं शल्यं प्राच्छादयच्छरैः}


\twolineshloka
{स च्छाद्यमानः समरे धर्मपुत्रस्य सायकैः}
{युधिष्ठिरमथाविध्यद्दशभिर्निशितैः शरैः}


\twolineshloka
{सात्यकिस्तु ततः क्रुद्धो धर्मपुत्रे शरार्दिते}
{मद्राणामधिपं शूरं शरैर्विव्याध पञ्चभिः}


\twolineshloka
{स सात्यकेः प्रचिच्छेद क्षुरप्रेण महद्धनुः}
{भीमसेनमुखांस्तांश्च त्रिभिस्त्रिभिरताडयत्}


\twolineshloka
{तस्य क्रुद्धो महाराज सात्यकिः सत्यविक्रमः}
{तोमरं प्रेषयामास स्वर्णदण्डं महारणे}


\threelineshloka
{भीमसेनोऽथ नाराचं ज्वलन्तमिव पन्नगम्}
{नकुलः समरे शक्तिं सहदेवो गदां शुभाम्}
{धर्मराजः शतघ्नीं च जिघांसुः शल्यमाहवे}


\twolineshloka
{तानापतत एवाशु पञ्चानां वै भुजच्युतान्}
{वारयामास समरे शस्त्रसङ्घैः स मद्रराट्}


\twolineshloka
{सात्यकिप्रहितं शल्यो भल्लैश्चिच्छेद तोमरम्}
{प्रहितं भीमसेनेन शरं कनकभूषणम्}


\twolineshloka
{द्विधा चिच्छेद समरे कृतहस्तः प्रतापवान्}
{नकुलप्रेषितां शक्तिं हेमदण्डां भयावहाम्}


\twolineshloka
{गदां च सहदेवेन शरौघैः समवारयत्}
{शराभ्यां च शतघ्नीं तां राज्ञश्चिच्छेद भारत}


\twolineshloka
{पश्यतां पाण्डुपुत्राणां सिंहनादं ननाद च}
{नामृष्यत्तत्र शैनेयः शत्रोर्विजयमाहवे}


\twolineshloka
{अथान्यद्धनुरादाय सात्यकिः क्रोधमूर्च्छितः}
{द्वाभ्यां मद्रेश्वरं विद्वा सारथिं च त्रिभिः शरैः}


\twolineshloka
{ततः शल्यो रणे राजन्सर्वांस्तान्दशभिः शरैः}
{विव्याध भृशसङ्क्रुद्धस्तोत्रैरिव महाद्विपान्}


\twolineshloka
{ते वार्यमाणाः समरे मद्रराज्ञा महारथाः}
{न शेकुः सम्मुखे स्थातुं तस्य शत्रुनिषूदनाः}


\threelineshloka
{ततो दुर्योधनो राजा दृष्ट्वा शल्यस्य विक्रमम्}
{निहतान्पाण्डवान्मेने पाञ्चालानथ सृञ्जयान्}
{}


\twolineshloka
{`तथाविधं महाराज मद्रराजस्य विक्रमम्}
{असह्यं मानवैर्युद्धे तद्बभूव नरर्षभ ॥'}


\twolineshloka
{ततो राजन्महाबाहुर्भीमसेनः प्रतापवान्}
{सन्त्यज्य मनसा प्राणान्मद्राधिपमयोधयत्}


\twolineshloka
{नकुलः सहदेवश्च सात्यकिश्च महारथः}
{परिवार्य तदा शल्यं समन्ताद्व्यकिरञ्शरैः}


\twolineshloka
{स चतुर्भिर्महेष्वासैः पाण्डवानां महारथः}
{वृतस्तान्योधयामास मद्रराजः प्रतापवान्}


\twolineshloka
{तस्य धर्मसुतो राजन्क्षुरप्रेण महाहवे}
{चक्ररक्षं जघानाशु मद्रराजस्य पार्थिवः}


\twolineshloka
{तस्मिंस्तु निहते शूरे चक्ररक्षे महारथे}
{मद्रराजोऽपि बलवान्सैनिकानावृणोच्छरैः}


\twolineshloka
{समावृतांस्ततस्तांस्तु राजन्वीक्ष्य स्वसैनिकान्}
{चिन्तयामास समरे धर्मपुत्रो युधिष्ठिरः}


\twolineshloka
{कथं नु न भवेत्सत्यं तन्माधववचो महत्}
{अपि क्रुद्धो रणे राजन्क्षपयेत बलं मम}


\twolineshloka
{`अहं मद्धातरश्चैव सात्यकिश्च महारथः}
{पाञ्चालाः सृञ्जयाश्चैव न शक्तास्म हि मद्रपम्}


\twolineshloka
{निहनिष्यति चैवाद्य मातुलोऽस्मान्महाबलः}
{गोविन्दवचनं सत्यं कथं भवति किन्त्विदम्}


\twolineshloka
{ततः सरथनागाश्वा पाण्डवाः पाण्डुपूर्वज}
{मद्रराजं समासेदुः पीडयन्तः समन्ततः}


\twolineshloka
{नानाशस्त्रौघबहुलां शस्त्रवृष्टिं समुद्यताम्}
{व्यधमत्समरे राजा महाभ्राणीव मारुतः}


\twolineshloka
{ततः कनकपुङ्खां तां शल्यक्षिप्तां वियद्गताम्}
{शरवृष्टिमपश्याम शलभानामिवायतिम्}


\twolineshloka
{ते शरा मद्रराजेन प्रेषिता रणमूर्धनि}
{सम्पतन्तः स्म दृश्यन्ते शलभानां व्रजा इव}


\twolineshloka
{मद्रराजधनुर्मुक्तैः शरैः कनकभूषणैः}
{निरन्तरमिवाकाशं सम्बभूव जनाधिप}


\twolineshloka
{न पाण्डवानां नास्माकं तत्र किञ्चिद्व्यदृश्यत}
{बाणान्धकारे महति कृते तत्र महाहवे}


\threelineshloka
{मद्रराजेन बलिना लाघवाच्छरवृष्टिभिः}
{चाल्यमानं तु तं दृष्ट्वा पाण्डवानां बलार्णवम्}
{विस्मयं परमं जग्मुर्देवगन्धर्वदानवाः}


\twolineshloka
{स तु तान्सर्वतो यत्ताञ्शरैः सञ्छाद्य मारिष}
{धर्मराजमवच्छाद्य सिंहवद्व्यनदन्मुहुः}


\twolineshloka
{ते च्छन्नाः समरे तेन पाण्डवानां महारथाः}
{नाशक्नुवंस्तदा युद्धे प्रत्युद्यातुं महारथम्}


\twolineshloka
{धर्मराजपुरोगास्तु भीमसेनमुखा रथाः}
{निजघ्नुः समरे शूरंशल्यमाहवशोभिनम्}


\chapter{अध्यायः १४}
\twolineshloka
{सञ्जय उवाच}
{}


\twolineshloka
{अर्जुनो द्रौणिना विद्धो युद्धे बहुभिरायसैः}
{तस्य चानुचरैः शूरैस्त्रिगर्तानां महारथैः}


\threelineshloka
{द्रौणिं विव्याध समरे त्रिभिरेव शिलीमुखैः}
{तथेतरान्महेष्वासान्द्वाभ्यां द्वाभ्यां धनञ्जयः}
{भूयश्चैव महाराज शरवर्षैरवाकिरत्}


\twolineshloka
{शरकण्टकितास्ते तु तावका भरतर्षभ}
{न जहुः पार्थमासाद्य ताड्यमानाः शितैः शरैः}


\twolineshloka
{अर्जुनं रथवंशेन द्रोणपुत्रपुरोगमाः}
{परिवार्य मुदा युक्ता योधयन्तश्चकाशिरे}


\twolineshloka
{तैस्तु क्षिप्ताः शरा राजन्कार्तस्वरविभूषिताः}
{अर्जुनस्य रथोपस्थं पूरयामासुरञ्जसा}


\twolineshloka
{तथा कृष्णौ महेष्वासौ वृषभौ सर्वधन्विनाम्}
{विव्यधुश्च शरैर्घोरैः प्रहृष्टा युद्धदुर्मदाः}


\twolineshloka
{कूबरं रथचक्राणि ईषा योक्त्राणि वा विभो}
{युगं चैवानुकर्षं च शरभूतमभूत्तदा}


\twolineshloka
{नैतादृशं दृष्टपूर्वं राजन्नेव च नः श्रुतम्}
{यादृशं तत्र पार्थस्य तावकाः सम्प्रचक्रिरे}


\twolineshloka
{स रथः सर्वतो भाति चित्रपुङ्खैः शितैः शरैः}
{उल्काशतैः सम्प्रदीप्तं विमानमिव भूतले}


\twolineshloka
{ततोऽर्जुनो महाराज शरैः सन्नतपर्वभिः}
{अवाकिरत्तां पृतनां मेघो धृष्ट्येव पर्वतम्}


\twolineshloka
{ते वध्यमानाः समरे पार्थनामाङ्कितैः शरैः}
{पार्थभूतममन्यन्त प्रेक्षमाणास्तथाविधम्}


\twolineshloka
{कोपोद्धूतशरज्वालो धनुःशब्दानिलो महान्}
{सैन्येन्धनं ददाहाशु तावकं पार्थपावकः}


\twolineshloka
{चक्राणां पततां चापि युगानां च धरातले}
{तूणीराणां पताकानां ध्वजानां च रथैः सह}


\twolineshloka
{ईषाणामनुकर्षाणां त्रिवेणूनां च भारत}
{अक्षाणामथ योक्त्राणां प्रतोदानां च राशयः}


\twolineshloka
{शिरसां पततां चापि कुण्डलोष्णीषधारिणाम्}
{भुजानां च महाभाग स्कन्धानां च समन्ततः}


\twolineshloka
{छत्राणां व्यजनैः सार्धं मकुटानां च राशयः}
{समदृश्यन्त पार्थस्य रथमार्गेषु भारत}


\threelineshloka
{अगम्यरूपा पृथिवी मांसशोणितकर्दमा}
{भीरूणां त्रासजननी शूराणां हर्षवर्धिनी}
{बभूव भरतश्रेष्ठ रुद्रस्याक्रीडनं यथा}


\twolineshloka
{हत्वा तु समरे पार्थः सहस्रे द्वे परन्तपः}
{रथानां सवरूथानं विधूमोऽग्निरिव ज्वलन्}


\twolineshloka
{यथा हि भगवानग्निर्जगद्दग्ध्वा चराचरम्}
{विधूमो दृश्यते राजंस्तथा पार्थो धऩञ्जयः}


\twolineshloka
{द्रौणिस्तु समरे दृष्ट्वा पाण्डवस्य पराक्रमम्}
{रथेनातिपताकेन पाण्डवं प्रत्यवारयत्}


\twolineshloka
{तावुभौ पुरुषव्याघ्रौ श्वेताश्वौ धन्विनां वरौ}
{समीयतुस्तदाऽन्योन्यं परस्परवधैषिणौ}


\twolineshloka
{तयोरासीन्महाराज बाणवर्षं सुदारुणम्}
{जीमूतयोर्यथा वृष्टिस्तपान्ते भरतर्षभ}


\twolineshloka
{अन्योन्यस्पर्धिनौ तौ तु शरैः सन्नतपर्वभिः}
{ततक्षतुस्तदाऽन्योन्यं शृङ्गाभ्यां वृषभाविव}


\twolineshloka
{तयोर्युद्धं महाराज चिरं सममिवाभवत्}
{शस्त्राणां सङ्गमश्चैव घोरस्तत्राभवत्पुनः}


\twolineshloka
{ततोऽर्जुनं द्वादशभी रुक्मपुङ्खैः सुतेजनैः}
{वासुदेवं च दशभिर्द्रौणिर्विव्याध भारत}


\twolineshloka
{ततः प्रहस्य बीभत्सुर्व्याक्षिपद्गाण्डिवं धनुः}
{मानयित्वा मुहूर्तं तु गुरुपुत्रं महाहवे}


\twolineshloka
{व्यश्वसूतरथं चक्रे सव्यसाची परन्तपः}
{मृदुपूर्वं ततश्चैनं पुनःपुनरताडयत्}


\twolineshloka
{हताश्वे तु रथे तिष्ठन्द्रोणपुत्रस्त्वयस्मयम्}
{मुसलं पाण्डुपुत्राय चिक्षेप परिघोपमम्}


\twolineshloka
{तमापतन्तं सहसा हेमपट्टविभूषितम्}
{चिच्छेद सप्तधा वीरः पार्थः शत्रुनिबर्हणः}


\threelineshloka
{स च्छिन्नं मुसलं दृष्ट्वा द्रौणिः परमकोपनः}
{आददे परिघं घोरं नगेन्द्रशिखरोपमम्}
{चिक्षेप चैव पार्थाय द्रौणिर्युद्धविशारदः}


\twolineshloka
{तमन्तकमिव क्रुद्दं परिघं प्रेक्ष्य पाण्डवः}
{अर्जुनस्त्वरितो जघ्ने पञ्चभिः सायकोत्तमैः}


\twolineshloka
{स च्छिन्नः पतितो भूमौ पार्थबाणैर्महाहवे}
{दारयन्पृथिवीन्द्राणां मनः सब्देन भारत}


% Check verse!
ततोऽपरैस्त्रिभिर्भल्लैर्द्रौणिं विव्याध पाण़्डवः
\twolineshloka
{सोऽतिविद्धो बलवता पार्थेन सुमहात्मना}
{नाकम्पत तदा द्रौणिः पौरुषेषु व्यवस्थितः}


\twolineshloka
{सुरथं च ततो राजन्भारद्वाजो महारथम्}
{अवाकिरच्छरव्रातैः सर्वक्षत्रस्य पश्यतः}


\twolineshloka
{ततस्तु सुरथोऽप्याजौ पाञ्चालानां महारथः}
{रथेन मेघघोषेण द्रौणिमेवाभ्यधावत}


\twolineshloka
{विकर्षन्वै धनुःश्रेष्ठं सर्वभारसहं दृढम्}
{ज्वलनाशीविषनिभैः शरैश्चैनमवाकिरत्}


\twolineshloka
{सुरथं तं ततः क्रुद्धमापतन्तं महारथम्}
{चुकोप समरे द्रौणिर्दण्डाहत इवोरगः}


\threelineshloka
{त्रिशिखां भ्रुकुटीं कृत्वा सृक्विणी परिसंलिहन्}
{उद्वीक्ष्य सुरथं रोषाद्धनुर्ज्यामवमृज्य च}
{मुमोच तीक्ष्णं नाराचं यमदण्डोपमद्युतिम्}


\twolineshloka
{स तस्य हृदयं भित्त्वा प्रविवेशातिवेगितः}
{शक्ताशनिरिवोत्सृष्टो विदार्य धरणीतलम्}


\twolineshloka
{ततः स पतितो भूमौ नाराचेन समाहतः}
{वज्रेण च यथा शृङ्गं पर्वतस्येव दीर्यतः}


\twolineshloka
{तस्मिन्विनिहते वीरे द्रोणपुत्रः प्रतापवान्}
{आरुरोह रथं तूर्णं तमेव रथिनां वरः}


\twolineshloka
{ततः सज्जो महाराज द्रौणिराहवदुर्मदः}
{अर्जुनं योधयामास संशप्तकवृतो रणे}


\twolineshloka
{तत्र युद्धं महच्चासीदर्जुनस्य परैः सह}
{मध्यन्दिनगते सूर्ये यमराष्ट्रविवर्धनम्}


\twolineshloka
{तत्राश्चर्यमपश्याम दृष्ट्वा तेषां पराक्रमम्}
{यदेको युगपद्वीरान्समयोधयदर्जुनः}


\twolineshloka
{विमर्दः सुमहानासीदर्जुनस्य परैः सह}
{शतक्रतोर्यथापूर्वं महत्या दैत्यसेनया}


\chapter{अध्यायः १५}
\twolineshloka
{सञ्जय उवाच}
{}


\twolineshloka
{दुर्योधनो महाराज धृष्टद्युम्नश्च पार्षतः}
{चक्रतुः सुमहद्युद्धं शरशक्तिसमाकुलम्}


\twolineshloka
{ततो राजन्समापेतुः शरधाराः सहस्रशः}
{अम्बुदानां यथा काले जलधाराः समन्ततः}


\twolineshloka
{राजा च पार्षतं विद्ध्वा शरैः पञ्चभिराशुगैः}
{द्रोणहन्तारमुग्रेषुं पुनर्विव्याध सप्तभिः}


\twolineshloka
{धृष्टद्युम्नस्तु समरे बलवान्दृढविक्रमः}
{सप्तत्या विशिखानां वै दुर्योधनमपीडयत्}


\twolineshloka
{पीडितं वीक्ष्य राजानं सोदर्या भरतर्षभ}
{महत्या सेनया सार्धं परिवव्रुः स्म पार्षतम्}


\twolineshloka
{स तैः परिवृतः शूरः सर्वतोऽतिरथै र्भृशम्}
{व्यचरत्समरे राजन्दर्शयन्नस्त्रलाघवम्}


% Check verse!
शिखण्डी कृतवर्माणं गौतमं च महारथम् ॥प्रभद्रकैः समायुक्तो योधयामास धन्विनौ
\twolineshloka
{तत्रापि सुमहद्युद्धं घोररूपं विशाम्पते}
{प्राणान्सन्तजतां युद्धे प्राणद्यूताभिदेवने}


\twolineshloka
{शल्यः सायकवर्षाणि विमुञ्जन्सर्वतोदिशम्}
{पाण्डवान्पीडयामास ससात्यकिवृकोदरान्}


\twolineshloka
{तथा तौ तु यमौ यमतुल्यपराक्रमौ}
{योधयामास राजेन्द्र वीर्येणास्त्रबलेन च}


\twolineshloka
{शल्यसायकनुन्नानां पाण्डवानां महामृधे}
{त्रातारं नाध्यगच्छन्त केचित्तत्र महारथाः}


\twolineshloka
{ततस्तु नकुलः शूरो धर्मराजे प्रपीडिते}
{अभिदुद्राव वेगेन मातुलं माद्रिनन्दनः}


\twolineshloka
{सञ्छाद्य समरे वीरं नकुलः परवीरहा}
{विव्याध चैनं दशभिः स्मयमानः स्तनान्तरे}


\twolineshloka
{सर्वपारसवैर्बाणैः कर्मारपरिमार्जितैः}
{स्वर्णपुङ्खैः शिलाधौतैर्धनुर्यन्त्रप्रचोदितैः}


\twolineshloka
{शल्यस्तु पीडितस्तेन स्वस्रीयेण हि मातुलः}
{नकुलं पीडयामास पत्रिभिर्नतपर्वभिः}


\twolineshloka
{ततो युधिष्ठिरो राजा भीमसेनोऽथ सात्यकिः}
{सहदेवश्च माद्रेणो मद्रराजमुपाद्रवन्}


\threelineshloka
{तानापतत एवाशु पूरयाणान्रथस्वनैः}
{दिशश्च विदिशश्चैव कम्पयानांश्च मेदिनीम्}
{प्रतिजग्राह समरे सेनापतिरमित्रजित्}


\twolineshloka
{युधिष्ठिरं त्रिभिर्विद्ध्वा भीमसेनं च पञ्चभिः}
{सात्यकिं च शतेनाजौ सहदेवं त्रिभिः शरैः}


\twolineshloka
{ततस्तु सशरं चापं नकुलस्य महात्मनः}
{मद्रेश्वरः क्षुरप्रेण मध्ये चिच्छेद मारिष}


\threelineshloka
{तदपास्य धनुश्छिन्नं ततः शल्यस्य सायकैः}
{अथान्यद्धनुरादाय माद्रीपुत्रो महारथः}
{मद्रराजरथं तूर्णं पूरयामास पत्रिभिः}


\twolineshloka
{युधिष्ठिरस्तु मद्रेशं सहदेवश्च मारिष}
{दशभिर्दशभिर्बाणैरुरस्येनमविध्यताम्}


\twolineshloka
{भीमसेनस्तु तं षष्ट्या सात्यकिर्दशभिः शरैः}
{मद्रराजमभिद्रुत्य जघ्नतुः xxङ्कपत्रिभिः}


\twolineshloka
{मद्रराजस्ततः क्रुद्धः सात्यकिं नवभिः शरैः}
{विव्याध भूयः सप्तत्या शराणां नतपर्वणाम्}


\twolineshloka
{अथास्य सशरं चापं मुष्टौ चिच्छेद मारिष}
{हयांश्च चतुरः सङ्ख्ये प्रेषयामास मृत्यवे}


\twolineshloka
{विरथं सात्यकिं कृत्वा मद्रराजो महारथः}
{विशिखानां शतेनैनमाजघान समन्ततः}


\twolineshloka
{माद्रीपुत्रं च संरब्धो भीमसेनं च पाण्डवम्}
{युधिष्ठिरं च कौरव्य विव्याध दशभिः शरैः}


\twolineshloka
{तत्राद्भुतमपश्याम मद्रराजस्य पौरुषम्}
{यदेनं सहिताः पार्था नाभ्यवर्तन्त संयुगे}


\threelineshloka
{अथान्यं रथमास्थाय सात्यकिः सत्यविक्रमः}
{पीडितान्पाण्डवान्दृष्ट्वा मद्रराजवशं गतान्}
{अभिदुद्राव वेगेन मद्राणामधिपं बलात्}


\twolineshloka
{आपतन्तं रथं तस्य शल्यः समितिशोभनः}
{प्रत्युद्ययौ रथेनैव मत्तो मत्तमिव द्विपम्}


\threelineshloka
{स सन्निपातस्तुमुलो बभूवाद्भुतदर्शनः}
{सात्यकेश्चैव शूरस्य मद्राणामधिपस्य च}
{यादृशो वै पुरा वृत्तः शम्बरामरराजयोः}


\twolineshloka
{सात्यकिः प्रेक्ष्य समरे मद्रराजमवस्थितम्}
{विव्याध दशभिर्बाणैस्तिष्ठतिष्ठेति चाब्रवीत्}


\twolineshloka
{मद्रराजस्तु सुभृशं विद्धस्तेन महात्मना}
{सात्यकिं प्रतिविव्याध चित्रपुङ्खैः शितैः शरैः}


\twolineshloka
{ततः पार्था महेष्वासाः सात्वताऽभिसृतं नृपम्}
{अभ्यवर्तन्रथैस्तूर्णं मातुलं वधकाङ्क्षया}


\twolineshloka
{तत आसीत्परामर्दस्तुमुलः शोणितोदकः}
{शूराणां युध्यमानानां सिंहानामिव नर्दताम्}


\twolineshloka
{तेषामासीन्महाराज व्यधिक्षेपः परस्परम्}
{सिंहानामामिषेप्सूनां कूजतामिव संयुगे}


\twolineshloka
{तेषां बाणसहस्रौघैराकीर्णा वसुधाऽभवत्}
{अन्तरिक्षं च सहसा बाणभूतमभूत्तदा}


\twolineshloka
{शरान्धकारं सहसा कृतं तेन समन्ततः}
{अभ्रच्छायेव सञ्जज्ञे शरैर्मुक्तैर्महात्मभिः}


\twolineshloka
{तत्र राजञ्शरैर्मुक्तैर्निर्मुक्तैरिव पन्नैगः}
{स्वर्णपुङ्खैः प्रकाशद्भिर्व्यरोचन्त दिशस्तदा}


\twolineshloka
{तत्राद्भुतं परं चक्रे शल्यः शत्रुनिबर्हणः}
{यदेकः समरे शूरो योधयामास वै बहून्}


\twolineshloka
{मद्रराजभुजोत्सृष्टैः कङ्कबर्हिणवाजितैः}
{सम्पतद्भिः शरैर्घोरैरवाकीर्यत मेदिनी}


\twolineshloka
{तत्र शल्यरथं राजन्विचरन्तं महाहवे}
{अपश्याम यथापूर्वं शक्रस्यासुरसंक्षये}


\chapter{अध्यायः १६}
\twolineshloka
{सञ्जय उवाच}
{}


\twolineshloka
{ततः सेना तव विभो मद्रराजपुरस्कृता}
{पुनरभ्यद्रवत्पार्थान्वेगेन महता रणे}


\twolineshloka
{पीडितास्तावकाः सर्वे प्रधावन्तो रणोत्कटाः}
{क्षणेन चैव पार्थांस्ते बहुत्वात्समलोडयन्}


\twolineshloka
{ते वध्यमानाः समरे पाण्डवा नावतस्थिरे}
{क्षणेनैव महाराज पश्यतोः कृष्णपार्थयोः}


\twolineshloka
{ततो धनञ्जयः क्रुद्धः कृपं सह पदानुगैः}
{अवाकिरच्छरौघेण कृतवर्माणमेव च}


\twolineshloka
{शकुनिं सहदेवस्तु सहसैन्यमवाकिरत्}
{नकुलः पार्श्वतः स्थित्वा मद्रराजमयोधयत्}


\twolineshloka
{द्रौपदेया नरेन्द्राश्च भूयिष्ठान्समवारयन्}
{द्रोणपुत्रं च पाञ्चाल्यः शिखण्डी समवारयत्}


\twolineshloka
{भीमसेनस्तु राजानं गदापाणिरवारयत्}
{शल्यं तु सह सैन्येन कुन्तीपुत्रो युधिष्ठिरः}


\twolineshloka
{ततः समभवद्युद्धं संसक्तं तत्रतत्र ह}
{तावकानां परेषां च सङ्ग्रामेष्वनिवर्तिनाम्}


\twolineshloka
{तत्र पश्यामहे कर्म शल्यस्यातिमहद्रणे}
{यदेकः सर्वसैन्यानि पाण्डवानामयोधयत्}


\twolineshloka
{व्यदृश्यत तदा शल्यो युधिष्ठिरसमीपतः}
{क्रूरश्चन्द्रमसोऽभ्याशे शनैश्चर इव ग्रहः}


\twolineshloka
{पीडयित्वा तु राजानं शरैराशीविषोपमैः}
{अभ्यधावत्पुनर्भीमं शरवर्षैरवाकिरत्}


\twolineshloka
{तस्य तल्लाघवं दृष्ट्वा तथैव च कृतास्त्रताम्}
{अपूजयन्ननीकानि परेषां तावकानि च}


\twolineshloka
{पीड्यमानास्तु शल्येन पाण्डवा भृशाविक्षताः}
{प्राद्रवन्त रणं हित्वा क्रोशमाने युधिष्ठिरे}


\twolineshloka
{वध्यमानेष्वनीकेषु मद्रराजेन पाण्डवः}
{अमर्षवशमापन्नो धर्मराजो युधिष्ठिरः}


\twolineshloka
{ततः पौरुषमास्थाय मद्रराजमताडयत्}
{जयो वाऽस्तु वधो वेति कृतबुद्धिर्महारथः}


% Check verse!
समाहूयाब्रवीत्सर्वान्भ्रातॄन्कृष्णं च पाण्डवः
\threelineshloka
{भीष्मो द्रोणश्च कर्णश्च ये चान्ये पृथिवीक्षितः}
{कौरवार्थे पराक्रान्ताः सङ्ग्रामे निधनं गताः}
{यथाभागं यथोत्साहं भवन्तः कृतपौरुषाः}


\twolineshloka
{भागोऽवशिष्ट एकोऽयं मम शल्यो महारथः}
{सोऽहमद्य युधा जेतुमाशंसे मद्रकाधिपम्}


\twolineshloka
{तत्र यन्मानसं मह्यं तत्सर्वं निगदामि वः}
{चक्ररक्षाविमौ वीरौ मम माद्रवतीसुतौ}


\twolineshloka
{अजेयौ वासवेनापि समरे शूरसम्मतौ}
{साध्विमौ मातुलं युद्धे क्षत्रधर्मपुरस्कृतौ}


\threelineshloka
{मदर्थे प्रतियुध्येतां मानार्हौ सत्यसङ्गरौ}
{मां वा शल्यो रणे हन्ता तं वाऽहं भद्रमस्तु वः}
{इति सत्यामिमां वाणीं लोकवीरा निबोधत}


\twolineshloka
{योत्स्येऽहं मातुलेनाद्य क्षात्रधर्मेण पार्थिवाः}
{स्वमंशमभिसन्धाय विजयायेतराय वा}


\twolineshloka
{तस्य मेऽप्यधिकं शस्त्रं सर्वोपकरणानि च}
{संयुज्यन्तां रथे क्षिप्रं शास्त्रवद्रथयोजकैः}


\twolineshloka
{शैनेयो दक्षिणं चक्रं धृष्टद्युम्नस्तथोत्तरम्}
{पृष्ठगोपो भवत्वद्य मम पार्थो धनञ्जयः}


\twolineshloka
{पुरःसरो ममाद्यास्तु भीमः शस्त्रभृतां वरः}
{एवमभ्यधिकः शल्याद्भविष्यामि महामृधे}


% Check verse!
एवमुक्तास्तथा चक्रुस्तदा राज्ञः प्रियैषिणः
\twolineshloka
{ततः प्रहर्षः सैव्यानां पुनरासीत्तदाः मृधे}
{पाञ्चालानां सोमकानां मात्स्यानां च विशेषतः}


\threelineshloka
{प्रतिज्ञातं च सङ्ग्रामे धर्मराजस्य पूजयन्}
{ततः शङ्खांश्च भेरीश्च शतशश्चैव पुष्करान्}
{अवादयन्त पाञ्चालाः सिंहनादांश्च नेदिरे}


\twolineshloka
{तेऽभ्यधावन्त संरब्धा मद्रराजं तरस्विनम्}
{महता हर्षजेनाथ नादेन कुरुपुङ्गवाः}


\twolineshloka
{हादेन गजघण्टानां शङ्खानां निनदेन च}
{तूर्यशब्देन महता नादयन्तश्च मेदिनीम्}


\twolineshloka
{तान्प्रत्यगृह्णात्पुत्रस्ते मद्रराजश्च वीर्यवान्}
{महामेघानिव बहूञ्शैलावस्तोदयावुभौ}


\twolineshloka
{शल्यस्तु समरश्लाघी धर्मराजमरिन्दमम्}
{ववर्ष शरवर्षेण शम्बरं मघवा इव}


\threelineshloka
{तथैव कुरुराजोऽपि प्रगृह्य रुचिरं धनुः}
{द्रोणोपदेशान्विविधान्दर्शयानो महामनाः}
{ववर्ष शरवर्षाणि चित्रं लघु च सुष्ठु च}


% Check verse!
न चास्य विवरं कश्चिद्ददर्श चरतो रणे
\twolineshloka
{तावुभौ विविधैर्बाणैस्ततक्षाते परस्परम्}
{शार्दूलावामिषप्रेप्सू पराक्रान्ताविवाहवे}


% Check verse!
भीमस्तु तव पुत्रेण युद्धशौण्डेन सङ्गतः
\twolineshloka
{पाञ्चाल्यः सात्यकिश्चैव माद्रीपुत्रौ च पाण्डवौ}
{शकुनिप्रमुखान्वीरान्प्रत्यगृह्णन्समन्ततः}


\twolineshloka
{तदासीत्तुमुलं युद्धं पुनरेव जयैषिणाम्}
{तावकानां परेषां च राजन्दुर्मन्त्रिते तव}


\twolineshloka
{दुर्योधनस्तु भीमस्य शरेणानतपर्वणा}
{चिच्छेदादिश्य सङ्ग्रामे ध्वजं हेमपरिष्कृतम्}


\twolineshloka
{स किङ्किणीकजालेन महता चारुदर्शनः}
{पपात रुचिरः सिंहो भीमसेनस्य पश्यतः}


\twolineshloka
{पुनश्चास्य धनुश्चित्रं गजराजकरोपमम्}
{क्षुरेण शितधारेण प्रचकर्त नराधिपः}


\twolineshloka
{स च्छिन्नधन्वा तेजस्वी रथशक्त्या सुतं तव}
{बिभेदोरसि विक्रम्य स रथोपस्थ आविशत्}


\twolineshloka
{तस्मिन्मोहमनुप्राप्ते पुनरेव वृकोदरः}
{यन्तुरेव शिरः कायात्क्षुरप्रेणाहरत्तदा}


\twolineshloka
{हतसूता हयास्तस्य रथमादाय भारत}
{व्यद्रवन्त दिशो राजन्हाहाकारस्तदाऽभवत्}


\twolineshloka
{तमभ्यधावंस्त्राणार्थं द्रोणपुत्रो महारथः}
{कृपश्च कृतवर्मां च पुत्रं ते हि परीप्सवः}


\twolineshloka
{तस्मिन्विलुलिते सैन्ये त्रस्तांस्तस्य पदानुगान्}
{गाण्डीवधन्वा विस्फार्य धनुस्तानहनच्छरैः}


\twolineshloka
{युधिष्ठिरस्तु मद्रेशमभ्यधावदमर्षितः}
{स्वयं सन्नोदयन्नश्वान्दन्तवर्णान्मनोजवान्}


\twolineshloka
{तत्राश्चर्यपमश्याम कुन्तीपुत्रे युधिष्ठिरे}
{पुरा भूत्वा मृदुर्दान्तो यत्तदा दारुणोऽभवत्}


\twolineshloka
{विवृताक्षश्च कौन्तेयो वेपमानश्च मन्युना}
{चिच्छेद योधान्निशितैः शरैः शतसहस्रशः}


\twolineshloka
{यांयां प्रत्युद्ययौ सेनां तांतां ज्येष्ठः स पाण्डवः}
{शरैरपातयद्राजन्गिरीन्वज्रैरिवोत्तमैः}


\twolineshloka
{साश्वसूतध्वजरथान्रथिनः पातयन्बहून्}
{अक्रीडदेको बलवान्पवनस्तोयदानिव}


\twolineshloka
{साश्वारोहांश्च तुरगान्पत्तींश्चैव सहस्रधा}
{व्यपोथयत सङ्ग्रामे क्रुद्धो रुद्रः पशूनिव}


\twolineshloka
{शून्यमायोधनं कृत्वा शरवर्षैः समन्ततः}
{अभ्यद्रवत मद्रेशं तिष्ठतिष्ठेति चाब्रवीत्}


\twolineshloka
{तस्य तच्चरितं दृष्ट्वा संग्रामे भीमकर्मणः}
{वित्रेसुस्तावकाः सर्वे शल्यस्त्वेनं समभ्ययात्}


\twolineshloka
{ततस्तौ भृशसङ्क्रुद्धौ प्रध्माय सलिलोद्भवौ}
{समाहूय तदाऽन्योन्यं भर्त्सयन्तौ समीयतुः}


\twolineshloka
{शल्यस्तु शरवर्षेण पीडयामास पाण्डवम्}
{मद्रराजं तु कौन्तेयः शरवर्षैरवाकिरत्}


\threelineshloka
{अदृश्येतां तदा राजन्कङ्कपत्रिभिराचितौ}
{उद्भिन्नरुधिरौ शूरौ मद्रराजयुधिष्ठिरौ}
{पुष्पिताविव रेजाते वने शाल्मलिकिंशुकौ}


\twolineshloka
{दीव्यमानौ महात्मानौ प्राणद्यूतेन दुर्मदौ}
{दृष्ट्वा सर्वाणि सैन्यानि नाध्यवस्यंस्तयोर्जयम्}


\twolineshloka
{हत्वा मद्राधिपं पार्थो भोक्ष्यतेऽद्य वसुन्धराम्}
{शल्यो वा पाण्डवं हत्वा दद्याद्दुर्योधनाय गाम्}


\twolineshloka
{इतीव निश्चयो नाभूद्योधानां तत्र भारत}
{प्रदक्षिणमभूत्सर्वं धर्मराजस्य युध्यतः}


\twolineshloka
{ततः शरशतं शल्यो मुमोचाशु युधिष्ठिरे}
{धनुश्चास्य शिताग्रेण बाणेन निरुकृन्तत}


\twolineshloka
{सोऽन्यत्कार्मुकमादाय शल्यं शरशतैस्त्रिभिः}
{अविध्यत्कार्मुकं चास्य क्षुरेण निरकृन्तत}


\twolineshloka
{अथास्य निजघानाश्वांश्चतुरो नतपर्वभिः}
{द्वाभ्यामतिशिताग्राभ्यामुभौ तत्पार्ष्णिसारथी}


\twolineshloka
{ततोऽस्य दीप्यमानेन पीतेन निशितेन च}
{प्रमुखे वर्तमानस्य भल्लेनापाहरद्वृजम्}


% Check verse!
ततः प्रभग्नं तत्सैन्यं दौर्योधनमरिन्दम
\twolineshloka
{ततो मद्राधिपं द्रौणिरभ्यधावत्तथा कृतम्}
{आरोप्य चैनं स्वरथं त्वरमाणः प्रदुद्रुवे}


\twolineshloka
{मुहूर्तमिव तौ गत्वा नर्दमाने युधिष्ठिरे}
{गत्वा ततो मद्रपतिरन्यं स्यन्दनमास्थितः}


\twolineshloka
{विधिवत्कल्पितं शुभ्रं महाम्बुदनिनादिनम्}
{सज्जयन्त्रोपकरणं द्विषतां रोमहर्षणम्}


% Check verse!
तथान्यद्धनुरादाय बलवान्वेगवत्तरम् ॥युधिष्ठिरं मद्रपतिर्भित्त्वा सिंह इवानदत्
\twolineshloka
{ततः स शरवर्षेण पर्जन्य इव वृष्टिमान्}
{अभ्यवर्षदमेयात्मा क्षत्रियान्क्षत्रियर्षभः}


\twolineshloka
{सात्यकिं दशभिर्विद्ध्वा भीमसेनं त्रिभिः शरैः}
{सहदेवं त्रिभिर्विद्धा युधिष्ठिरमपीडयत्}


\twolineshloka
{तांस्तानन्यान्महेष्वासान्साश्वान्सरथकूबरान्}
{अर्दयामास विशिखैरुल्काभिरिव कुञ्जरान्}


\twolineshloka
{कुञ्जरान्कुञ्जरारोहानश्वप्रयायिनः}
{रथांश्च रथिभिः सार्धं जघान रथिनां वरः}


\twolineshloka
{बाहूंश्चिच्छेद तरसा सायुधान्केतनानि च}
{चकार च महीं योधैः स्तीर्णां वेदीं कुशैरिव}


\twolineshloka
{तथा तमरिसैन्यानि घ्नन्तं मृत्युमिवान्तकम्}
{परिवव्रुर्भृशं क्रुद्धाः पाण्डुपाञ्चालसोमकाः}


\chapter{अध्यायः १७}
\twolineshloka
{सञ्जय उवाच}
{}


\twolineshloka
{तं भीमसेनश्च शिनेश्च नप्तामाद्याश्च पुत्रौ पुरुषप्रवीरौ}
{समागतं भीमबलेन राज्ञापर्यापुरन्योन्यमथाह्वयन्त}


\twolineshloka
{ततस्तु शूराः समरे नरेन्द्रनरेश्वरं प्राप्य युधां वरिष्ठम्}
{आवार्य चैनं समरे नृवीराजघ्नुः शितैः पत्रिभिरुग्रवेगैः}


\twolineshloka
{संरक्षितो भीमसेनेन राजामाद्रीसुताभ्यामथ माधवेन}
{मद्राधिपं पत्रिभिरुग्रवेगैःस्तनान्तरे धर्मसुतो निजघ्ने}


\twolineshloka
{ततो रणे तावकानां रथौघाःसमीक्ष्य मद्राधिपतिं शरार्तम्}
{पर्यावव्रुः प्रवराः सर्वयोधादुर्योधनस्यानुमते पुरस्तात्}


\twolineshloka
{ततो द्रुतं मद्रजनाधिपो रणेयुधिष्ठिरं सप्तभिरभ्यविद्व्यत्}
{तं चापि पार्थौ नवभिः पृषत्कै--र्विव्याध राजंस्तुमुले महात्मा}


\twolineshloka
{आकर्णपूर्णायतसम्प्रयुक्तैःशरैस्तदा संयति तैलधौतैः}
{अन्योन्यमाच्छादयतां महारथौमद्राधिपश्चापि युधिष्ठिरश्च}


\twolineshloka
{ततस्तु तूर्णं समरे महारथौपरस्परस्यान्तरमीक्षमाणौ}
{शरैर्भृशं विव्यधतुर्नृपोत्तमौमहबलौ शत्रुभिरप्रधृष्यौ}


\twolineshloka
{तयोर्धनुर्ज्यातलनिःस्वनो महा--न्महेन्द्रवज्राशनितुल्यनिः स्वनः}
{परस्परं बाणगणैर्महात्मनोःप्रवर्षतोर्मद्रपपाण्डुवीरयोः}


\twolineshloka
{तौ चेरतुर्व्याघ्रशिशुप्रकाशौमहावनेष्वामिषगृद्विनाविव}
{विषाणिनौ नागवराविवोभौततक्षतुः संयति जातदर्पौ}


\twolineshloka
{ततस्तु मद्राधिपतिर्महात्मायुधिष्ठिरं भीमबलं प्रसह्य}
{विव्याध वीरं हृदयेऽतिवेगंशरेण सूर्याग्निसमप्रभेण}


\twolineshloka
{ततोऽतिविद्धोऽथ युधिष्ठिरोऽपिसुसम्प्रयुक्तेन शरेण राजन्}
{जघान मद्राधिपतिं महात्मामुदं च लेभे वृषभः कुरूणाम्}


\twolineshloka
{ततो मुहूर्तादिव पार्थिवेन्द्रोलब्ध्वा संज्ञां क्रोधसंरक्तनेत्रः}
{शरेण पार्थं त्वरितो जघानसहस्रनेत्रप्रतिमप्रभावः}


\twolineshloka
{त्वरंस्ततो धर्मसुतो महात्माशल्यस्य कोपान्नवभिः पृषत्कैः}
{भित्त्वा ह्युरस्तपनीयं च वर्मजघान षड्भिस्त्वपरैः पृषत्कैः}


\twolineshloka
{ततस्तु मद्राधिपतिः प्रकृष्टंधनुर्विकृष्य व्यसृजत्पृषत्कान्}
{द्वाभ्यां शराभ्यां च तथैव राज्ञ--श्चिच्छेद चापं कुरुपुङ्गवस्य}


\twolineshloka
{नवं ततोऽन्यत्समरे प्रगृह्यराजा धनुर्धोरतरं महात्मा}
{शल्यं तु विव्याध शरैः समन्ता--द्यथा महेन्द्रो नमुचिं शिताग्रैः}


\twolineshloka
{ततस्तु शल्यो नवभिः पृषत्कै--र्भीमस्य राज्ञश्च युधिष्ठिरस्य}
{निकृत्य रौक्मे पटुवर्मणी तयो--र्विदास्यामास भुजौ महात्मा}


\twolineshloka
{ततोऽपरेण ज्वलनार्कतेजसाक्षुरेण राज्ञो धनुरुन्ममाथ}
{कृपश्च तस्यैव जघान सूतंषड्भिः शरैः सोऽभिमुखः पपात}


\twolineshloka
{मद्राधिपश्चापि युधिष्ठिरस्यशरैश्चतुर्भिर्निजघान वाहान्}
{वाहांश्च हत्वा व्यकरोन्महात्मायोधक्षयं धर्मसुतस्य राज्ञः}


\twolineshloka
{`यदद्भुतं कर्म न शक्यमन्यैःसुदुःसहं तत्कृतवन्तमेकम्}
{शल्यो नरेन्द्रः सविषण्णभावा--द्विचिन्तयामास मृदङ्गकेतुः}


\twolineshloka
{किमेतदिन्द्रावरजस्य वाक्यंमोघं भवत्यद्य विधेर्बलेन}
{जहीति शल्यं ह्यवदत्तदाऽऽजौन लोकनाथस्य वचोऽन्यथा स्यात् ॥'}


\twolineshloka
{तथा कृते राजनि भीमसेनोमद्राधिपस्याथ ततो महात्मा}
{छित्त्वा धनुर्वेगवता शरेणद्वाभ्यामविध्यत्सुभृशं नरेन्द्रम्}


\twolineshloka
{तथापरेणास्य जहार यन्तुःकायाच्छिरः संहननीयमध्यात्}
{जघान चाश्वांश्चतुरः सुशीघ्रंतथा भृशं कुपितो भीमसेनः}


\twolineshloka
{तमग्रणीः सर्वधनुर्धराणा--मेकं चरन्तं समरेऽतिवेगम्}
{भीमः शतेन व्यकिरच्छराणांमाद्रीपुत्रः सहदेवस्तथैव}


\twolineshloka
{तैः सायकैर्मोहितं वीक्ष्य शल्यंभीमः शरैरस्य चकर्त वर्म}
{स भीमसेनेन निकृत्तवर्मामद्राधिपश्चर्म सहस्रतारम्}


\twolineshloka
{प्रगृह्य खङ्गं च रथान्महात्माप्रस्कन्द्य कुन्तीसुतमभ्यधावत्}
{छित्त्वा रथेषां नकुलस्य सोऽथयुधिष्ठिरं भीमबलोऽभ्यधावत्}


\twolineshloka
{तं चापि राजानमथोत्पतन्तंक्रुद्धं यथैवान्तकमापतन्तम्}
{धृष्टद्युम्नो द्रौपदेयाः शिखण्डीशिनेश्च नप्ता सहसा परीयुः}


\twolineshloka
{अथास्य चर्माप्रतिमं न्यकृन्त--द्भीमो महात्मा नवभिः पृषत्कैः}
{खङ्गं च भल्लैर्निचकर्त मुष्टौनदन्प्रहृष्टस्तव सैन्यमध्ये}


\twolineshloka
{तत्कर्म भीमस्य समीक्ष्य हृष्टा--स्ते पाण्डवानां प्रवरा रथौघाः}
{नादं च चक्रुर्भृशमुत्स्मयन्तःशङ्खांश्च दध्मुः शशिसन्निकाशान्}


\twolineshloka
{तेनाथ शब्देन विभीषणेनतथाऽभितप्तं बलमप्रधृष्यम्}
{स्वेदाभिभूतं रुधिरोक्षिताङ्गंविसञ्ज्ञकल्पं च तदा विषण्णम्}


\twolineshloka
{स मद्रराजः सहसा विकीर्णोभीमाग्रगैः पाण्डवयोधमुख्यैः}
{युधिष्ठिरस्याभिमुखं जवेनसिंहो यथा मृगहेतोः प्रयातः}


\twolineshloka
{स धर्मराजो निहताश्वसूतःक्रोधेन दीप्तो ज्वलनप्रकाशः}
{दृष्ट्वा च मद्राधिपतिं स्म तूर्णंसमभ्यधावत्तमरिं बलेन}


\twolineshloka
{गोविन्दवाक्यं त्वरितं विचिन्त्यदध्रे मतिं शल्यविनाशनाय}
{स धर्मराजो निहताश्वसूतोरथे तिष्ठञ्शक्तिमथान्वकर्षत्}


\twolineshloka
{तच्चxx शल्यस्य निशाम्य कर्मतमात्मनो भागमथावशिष्टम्}
{कृत्वा मनः शल्यवधे महात्मायथोक्तमिन्द्रावरजस्य चक्रे}


\twolineshloka
{स धर्मराजो मणिहेमदण्डांजग्राह शक्तिं कनकप्रकाशाम्}
{नेत्रे च दीप्ते सहसा विवृत्यमद्राधिपं क्रुद्धमना निरैक्षत्}


\twolineshloka
{निरीक्षितो धर्मसुतेन राज्ञाषूतात्मना निर्हृतकल्मषेण}
{आसीन्न यद्भस्मसान्मद्रराज--स्तदद्भुतं मे प्रतिभाति राजन्}


\twolineshloka
{ततस्तु शक्तिं रचिरोग्रदण्डांमणिप्रवेकोज्ज्वलितां प्रदीप्ताम्}
{चिक्षेप वेगात्सुभृशं महात्मामद्राधिपाय प्रवरः कुरूणाम्}


\twolineshloka
{दीप्तामथैनां प्रहितां बलेनसविस्फुलिङ्गां सहसापतन्तीम्}
{प्रैक्षन्त सर्वे कुरवः समेतादिवो युगान्ते महतीमिवोल्काम्}


\twolineshloka
{तां कालरात्रीमिव पाशहस्तांयमस्य धात्रीमिव चोग्ररूपाम्}
{स ब्रह्मदण्डप्रतिमाममोधांससर्ज यत्तो युधि धर्मराजः}


\twolineshloka
{गन्धस्रगग्र्यासनपानभोजनै--रभ्यर्चितां पाण्डुसुतैः प्रयत्नात्}
{सांवर्तकाग्निप्रतिमां ज्वलन्तींकृत्यामथर्वाङ्गिरसीमिवोग्राम्}


\twolineshloka
{ईशानहेतोः प्रतिनिर्मितां तांत्वष्ट्रा रिपूणामसुदेहभक्ष्याम्}
{भूम्यन्तरिक्षद्युजलाश्रयाणिप्रसह्य भूतानि निहन्तुमीशाम्}


\twolineshloka
{घण़्टापताकां मणिवज्रभूषांवैदूर्यचित्रां तपनीयदण्डाम्}
{त्वष्ट्रा प्रयत्नान्नियमेन क्लृप्तां ब्रह्मद्विषामन्तकरीममाघाम्}


\twolineshloka
{बलप्रयत्नादनिरोधवेगांमन्त्रैश्च घोरैरपि पूरयित्वा}
{ससर्ज मार्गेण च तां खगानांवधाय मद्राधिपतेस्तदानीम्}


\twolineshloka
{हतो ह्यसावित्यभिगर्जमानोरुद्रोऽन्धकायान्तकरं यथेषुम्}
{प्रसार्य बाहुं सुदृढं सुपाणिंक्रोधेन नृत्यन्निव धर्मराजः}


\twolineshloka
{`स्फुरत्प्रभामण्डलिनोंशुजालै--र्धर्मात्मनो मद्रविनाशकाले}
{पुरत्रयप्रोत्सरणे पुरस्ता--न्माहेश्वरं रूपमभूत्तदानीम्}


\twolineshloka
{आवर्तनाकुञ्चितबाहुदण्डःसन्ध्याविहारी तनुवृत्तमध्यः}
{विशालवक्षा भगवान्हरो यथासुदुर्निरीक्ष्योऽभवदर्जुनाग्रजः'}


\twolineshloka
{तां सर्वशक्त्या प्रहितां सुशक्तिंयुधिष्ठिरेणाप्रतिवार्यवीर्याम्}
{प्रतिग्रहायाभिननन्द शल्यःसम्यग्घुतामग्निरिवाज्यधाराम्}


\twolineshloka
{सा तस्य वर्माभिविदार्य शुभ्र--मुरो विशालं च तथैव भित्त्वा}
{विवेश गां तोयमिवाप्रसक्तायशो विशालं नृपतेर्हरन्ती}


\twolineshloka
{नासाक्षिकर्णास्यविनिः सृतेनप्रस्यन्दता च व्रणसम्भवेन}
{संसिक्तगात्रो रुधिरेण सोऽभू--त्क्रौञ्चो यथा स्कन्दहतो महाद्रिः}


\twolineshloka
{प्रसार्य बाहू च रथाद्गतो गांसञ्छिन्नवर्मा कुरुनन्दनेन}
{महेन्द्रवाहप्रतिमो महात्मावज्राहतं शृङ्गमिवाचलस्य}


\threelineshloka
{ततो निपतितः सोऽभूदिन्द्रध्वज इवोच्छ्रितः}
{बाहू प्रसार्याभिमुखो धर्मराजस्य मद्रराट्}
{स तथा भिन्नसर्वाङ्गो रुधिरेण समुक्षितः}


\twolineshloka
{प्रत्युद्गत इव प्रेम्णा भूम्या स नरपुङ्गवः}
{प्रियया कान्तया कान्तः पतमान इवोरसि}


\twolineshloka
{चिरं भुक्त्वा वसुमतीं प्रियां कान्तामिव प्रभुः}
{सर्वैरङ्गैः समाश्लिष्य प्रसुप्त इव चाभवत्}


\twolineshloka
{धर्म्ये धर्मात्मना युद्वे निहतो धर्मसूनुना}
{सम्यक्स्फीत इवोत्सृष्टः प्रशान्तोऽग्निरिवाध्वरे}


\twolineshloka
{शक्त्या विभिन्नहृदयं विप्रविद्धायुधध्वजम्}
{संशान्तमपि मद्रेशं लक्ष्मीर्नैव विमुञ्चति}


\threelineshloka
{ततो युधिष्ठिरश्चापमादायेन्द्रधनुष्प्रभम्}
{व्यधमद्द्विषतः सङ्ख्ये खगराडिव पन्नगान्}
{देहान्सुनिशितैर्भल्लै रिपूणां नाशयन्क्षणात्}


\threelineshloka
{ततः पार्थस्य बाणौघैरावृताः सैनिकास्तव}
{निमीलिताक्षाः क्षिण्वन्ति भृशमन्योन्यकर्शिताः}
{क्षरन्तो रुधिरं देहैर्विशस्त्रायुधजीविताः}


\twolineshloka
{ततः शल्ये निपतिते मद्रराजानुजो युवा}
{भ्रातुस्तुल्यो गुणैः सर्वै रथी पाण्डवमभ्ययात्}


\twolineshloka
{विव्याध च नरश्रेष्ठो नाराचैर्बहुभिस्त्वरन्}
{हतस्यापचितिं भ्रातुश्चिकीषुर्युद्धदुर्मदः}


\twolineshloka
{तं विव्याधाशुगैः ष़ड््भिर्धर्मराजस्त्वरन्निव}
{कार्मुकं चास्य चिच्छेद क्षुराभ्यां ध्वजमेव च}


\twolineshloka
{ततोऽस्य दीप्यमानेन सुदृढेन शितेन च}
{प्रमुखे वर्तमानस्य भल्लेनापाहरच्छिरः}


\twolineshloka
{सकुण्डलं तद्ददृशे पतमानं शिरो रथात्}
{पुण्यक्षयमनुप्राप्य पतन्स्वर्गादिव च्युतः}


\twolineshloka
{तस्यापकृत्तशीर्षं तु शरीरं पतितं रथात्}
{रुधिरेणावसिक्ताङ्गं दृष्ट्वा सैन्यमभज्यत}


\twolineshloka
{विचित्रकवचे तस्मिन्हते मद्रनृपानुजे}
{हाहाकारं प्रकुर्वाणाः कुरवोऽभिप्रदुद्रुवुः}


\twolineshloka
{शल्यानुजं हतं दृष्ट्वा तावकास्त्यक्तजीविताः}
{वित्रेसुः पाण्डवभयात्प्रविध्वस्तास्तदा भृशम्}


\twolineshloka
{तांस्तथा भज्यमानांस्तु कौरवान्भरतर्षभ}
{शिनेर्नप्ताऽकिरन्बाणैरभ्यवर्तत सात्यकिः}


\twolineshloka
{तमायान्तं महेष्वासं दुष्प्रसह्यं दुरासदम्}
{हार्दिक्यस्त्वरितो राजन्प्रत्यगृह्णादभीतवत्}


\twolineshloka
{तौ समेतौ महात्मानौ वार्ष्णेयौ वरवाजिनौ}
{हार्दिक्यः सात्यकिश्चैव सिंहाविव बलोत्कटौ}


\twolineshloka
{इषुभिर्विमलाभासैश्छादयन्तौ परस्परम्}
{अर्चिर्भिरिव सूर्यस्य दिवाकरसमप्रभौ}


\twolineshloka
{चापमार्गबलोद्वूतान्मार्गणान्वृष्णिसिंहयोः}
{आकाशगानपश्याम पतङ्गानिव शीघ्रगान्}


\twolineshloka
{सात्यकिं दशभिर्विद्ध्वा हयांश्चास्य त्रिभिः शरैः}
{चापमेकेन चिच्छेद हार्दिक्यो नतपर्वणा}


\twolineshloka
{तन्निकृत्तं धनुःश्रेष्ठमपास्य शिनिपुङ्गवः}
{अन्यदादत्त वेगेन धनुर्जलदनिःस्वनम्}


\twolineshloka
{तदादाय धनुःश्रेष्ठं वरिष्ठः सर्वधन्विनाम्}
{हार्दिक्यं दशभिर्बाणैः प्रत्यविध्यत्स्तनान्तरे}


\twolineshloka
{ततो युगं रथेषां च च्छित्त्वा भल्लैः सुसंयतैः}
{अश्वांस्तस्यावधीत्तूर्णमुभौ च पार्ष्णिसारथी}


\twolineshloka
{हार्दिक्यं विरथं दृष्ट्वा कृपः शारद्वतः प्रभो}
{अपोवाह ततः क्षिप्रं रथमारोप्य वीर्यवान्}


\twolineshloka
{मद्रराजे हते राजन्विरथे कृतवर्मणि}
{दुर्योधनबलं सर्वं पुनरासीत्पराङ्मुखम्}


\twolineshloka
{स्व परे नान्वबुध्यन्त सैन्येन रजसा वृते}
{बलं तु हतभूयिष्ठं त्रस्तमासीत्पराङ्मुखम्}


\twolineshloka
{ततो मुहूर्तात्तेऽपश्यन्रजो भौमं समुत्थितम्}
{विविधैः शोणितस्रावैः प्रशान्तं पुरुषर्षभ}


\twolineshloka
{ततो दुर्योधनो दृष्ट्वा भग्नं स्वबलमन्तिकात्}
{जवेनापततः पार्थानेकः सर्वानवारयत्}


\twolineshloka
{पाण्डवान्सरथान्दृष्ट्वा धृष्टद्युम्नं च पार्षतम्}
{आनर्त्तं च दुराधर्षं शितैर्बाणैरवारयत्}


\twolineshloka
{तं परे नाभ्यवर्तन्त मर्त्या मृत्युमिवागतम्}
{अथान्यं रथमास्थाय हार्दिक्योऽपि न्यवर्तत}


\threelineshloka
{ततो युधिष्ठिरो राजा त्वरमाणो महारथः}
{चतुर्भिर्निजघानाश्वान्पत्रिभिः कृतवर्मणः}
{विव्याध गौतमं चापि ष़ड््भिर्भल्लैः सुतेजनैः}


\twolineshloka
{अश्वत्थामाः ततो राज्ञा हताश्वं विरथीकृतम्}
{तमपोवाह हार्दिक्यं स्वरथेन युधिष्ठिरात्}


\twolineshloka
{ततः शारद्वतः षड्भिः प्रत्यविध्यद्युधिष्ठिरम्}
{विव्याध चाश्वान्निशितैस्तस्याष्टाभिः शिलीमुखैः}


\twolineshloka
{एवमेतन्महाराज युद्धशेषमवर्तत}
{तव दुर्मन्त्रिते राजन्सहपुत्रस्य भारत}


\twolineshloka
{तस्मिन्महेष्वासधरे विशस्तेसङ्ग्राममध्ये कुरुपुङ्गवेन}
{पार्थाः समेताः परमप्रहृष्टाःशङ्खान्प्रदध्मुर्हतमीक्ष्य शल्यम्}


\twolineshloka
{युधिष्ठिरं च प्रशशंसुराजौपुरा सुरा वृत्रवधे यथेन्द्रम्}
{चक्रुश्च नानाविधवाद्यशब्दा--न्निनादयन्तो वसुधां समन्तात्}


\chapter{अध्यायः १८}
\twolineshloka
{सञ्जय उवाच}
{}


\twolineshloka
{शल्येऽथ निहते राजन्मद्रराजपदानुगाः}
{रथाः सप्तशता वीरा दुद्रुवुर्महतो बलात्}


\threelineshloka
{दुर्योधनस्तु द्विरदमारुह्याचलसन्निभम्}
{छत्रेण ध्रियमाणेन वीज्यमानश्च चामरैः}
{न गन्तव्यं न गन्तव्यमिति मद्रानवारयत्}


\twolineshloka
{दुर्योधनेन ते वीरा वार्यमाणाः पुनः पुनः}
{युधिष्ठिरं जिघांसन्तः पाण्डूनां प्राविशन्बलम्}


\twolineshloka
{ते तु शूरा महाराज कृतचित्ताश्च योधने}
{धनुःशब्दं महत्कृत्वा सहायुध्यन्त पाण्डवैः}


\twolineshloka
{श्रुत्वा च निहतं शल्यं धर्मपुत्रं च पीडितम्}
{मद्रराजप्रिये युक्तैर्मद्रकाणां महारथैः}


\twolineshloka
{आजगाम ततः पार्थो गाण्डीवं विक्षिपन्धनुः}
{पूरयन्रथघोषेण दिशः सर्वा महारथः}


\twolineshloka
{ततोऽर्जुनश्च भीमश्च माद्रीपुत्रौ च पाण्डवौ}
{सात्यकिश्च नरव्याघ्रो द्रौपदेयाश्च सर्वशः}


\twolineshloka
{धृष्टद्युम्नः शिखण्डी च पाञ्चालाः सह सोमकैः}
{युधिष्ठिरं परीप्सन्तः समन्तात्पर्यवारयन्}


\threelineshloka
{ते समन्तात्परिवृताः पाण्डवैः पुरुषर्षभैः}
{क्षोभयन्ति स्म तां सेनां मकराः सागरं यथा}
{वृक्षानिव महावाताः कम्पयन्ति स्म तावकाः}


\twolineshloka
{पुरोवातेन गङ्गेव--क्षोभ्यमाणा महानदी}
{अक्षोभ्यत तदा राजन्पाण्डूनां ध्वजिनी ततः}


\threelineshloka
{प्रस्कन्द्य सेनां महतीं महात्मानो महारथाः}
{बहवश्चुक्रुशुस्तत्र क्व स राजा युधिष्ठिरः}
{भ्रातरो वाऽस्य ते शूरा दृश्यन्ते नेह केचन}


\twolineshloka
{धृष्टद्युम्नोऽथ शैनेयो द्रौपदेयाश्च सर्वशः}
{पाञ्चालाश्च महावीर्याः शिखण्डी च महारथः}


\twolineshloka
{एवं तान्वादिनः शूरान्द्रौपदेया महारथाः}
{अभ्यघ्नन्युयुधानश्च मद्रराजपदानुगान्}


\twolineshloka
{चक्रैर्विमथितैः केचित्केचिच्छिन्नैर्महाध्वजैः}
{सम्प्रदृश्यन्त समरे तावका निहताः परैः}


\twolineshloka
{आलोक्य पाण्डवान्युद्धे योधा राजन्समन्ततः}
{वार्यमाणा ययुर्वेगात्पुत्रेण तव भारत}


\twolineshloka
{दुर्योधनश्च तान्वीरान्वारयामास सान्त्वयन्}
{न चास्य शासनं केचित्तत्र चक्रुर्महारथाः}


\twolineshloka
{ततो गान्धारराजस्य पुत्रः शकुनिरब्रवीत्}
{दुर्योधनं महाराज वचनं वचनक्षमम्}


\twolineshloka
{किं नः सम्प्रेक्षमाणानां मद्राणां हन्यते बलम्}
{न युक्तमेतत्समरे त्वयि तिष्ठति भारत}


\threelineshloka
{सहितैर्नाम योद्धव्यमित्येष समयः कृतः}
{अथ कस्मात्स्मरन्नेव घ्नतो मर्षयसे नृप ॥दुर्योधन उवाच}
{}


\threelineshloka
{वार्यमाणा मया पूर्वं नैते चक्रुर्वचो मम}
{एते विनिहताः सर्वे प्रस्कन्नाः पाण्डुवाहिनीम् ॥शकुनिरुवाच}
{}


\twolineshloka
{न भर्तुः शासनं वीरा रणे कुर्वन्त्यमर्षिताः}
{अलं क्रोद्वुमथैतेषां नायं काल उपेक्षितुम्}


\twolineshloka
{यामः सर्वेऽत्र सम्भूय सवाजिरथकुञ्जराः}
{परित्रातुं महेष्वासान्मद्रराजपदानुगान्}


\threelineshloka
{अन्योन्यं परिरक्षामो यत्नेन महता नृप}
{एवं सर्वेऽनुसञ्चिन्त्य प्रययुर्यत्र सैनिकाः ॥सञ्जय उवाच}
{}


\twolineshloka
{एवमुक्तस्तदा राजा बलेन महता वृतः}
{प्रययौ सिंहनादेन कम्पयन्निव मेदिनीम्}


\twolineshloka
{हत विध्यत गृह्णीत प्रहरध्वं निकृन्तत}
{इत्यासीत्तुमुलः शब्दस्तव सैन्यस्य भारत}


\twolineshloka
{पाण्डवास्तु रणे दृष्ट्वा मद्रराजपदानुगान्}
{सहितानभ्यवर्तन्तं गुल्ममास्थाय मध्यमम्}


\twolineshloka
{ते मुहूर्ताद्रणे वीरा हस्ताहस्ति विशाम्पते}
{निहताः प्रत्यदृश्यन्त मद्रराजपदानुगाः}


\twolineshloka
{ततो नः सम्प्रयातानां हता मद्रास्तरस्विनः}
{हृष्टाः किलकिलाशब्दमकुर्वन्सहिताः परे}


\twolineshloka
{अथोत्थितानि रुण्डानि समदृश्यन्त सर्वशः}
{पपात महती चोल्का मध्येनादित्यमण्डलम्}


\twolineshloka
{रथैर्भग्नयुगाक्षैश्च निहतैश्च महारथैः}
{अश्वैर्निपातितैश्चैव सञ्छन्नाऽभूद्वसुन्धरा}


\twolineshloka
{वातायमानैस्तुरगैर्युगासक्तैस्ततस्ततः}
{अकृष्यन्त महाराज योधास्तत्र रणाजिरे}


\threelineshloka
{भग्नचक्रान्रथान्केचिदहरंस्तुरगा रणे}
{रथार्धं केचिदादाय दिशो दश विबभ्रमुः}
{तत्रतत्र व्यदृश्यन्त योक्त्रैः श्लिष्टाः स्म वाजिनः}


\twolineshloka
{रथिनः पतमानाश्च दृश्यन्ते स्म नरोत्तमाः}
{गगनात्प्रच्युताः सिद्धाः पुण्यानामिव सङ्क्षये}


\twolineshloka
{निहतेषु च शूरेषु मद्रराजानुगेषु वै}
{अस्मानापततश्चापि दृष्ट्वा पार्था महारथाः}


\twolineshloka
{अभ्यवर्तन्त वेगेन जयगृद्धाः प्रहारिणः}
{बाणशब्दरवान्कृत्वा विमिश्राञ्शङ्खनिःस्वनैः}


\twolineshloka
{अस्मांस्तु पुनरासाद्य लब्धलक्षाः प्रहारिणः}
{शरासनानि धुन्वानाः सिंहनादान्प्रचुक्रुशुः}


\threelineshloka
{ततो हतमभिप्रेक्ष्य मद्रराजबलं महत्}
{मद्रराजं च समरे दृष्ट्वा शूरं निपातितम्}
{दुर्योधनबलं सर्वं पुनरासीत्पराङ्मुखम्}


\twolineshloka
{वध्यमानं महाराज पाण्डवैर्जितकाशिभिः}
{दिशो भेजेऽथ सम्भ्रान्तं त्रासितं दृढधन्विभिः}


\chapter{अध्यायः १९}
\twolineshloka
{सञ्जय उवाच}
{}


\twolineshloka
{पातिते युधि दुर्धर्षे मद्रराजे महारथे}
{तावकास्तव पुत्राश्च प्रायशो विमुखाऽभवन्}


\twolineshloka
{वणिजो नावि भिन्नायां यथागाधे महार्णवे}
{अपारे पारमिच्छन्तो हते शूरे महात्मनि}


\twolineshloka
{मद्रराजे महाराज वित्रस्ताः शरविक्षताः}
{अनाथा नाथमिच्छन्तो मृगाः सिंहार्दिता इव}


\twolineshloka
{वृषा यथा भग्नशृङ्गाः शीर्णदन्ता यथा गजाः}
{मध्याह्ने प्रत्यपायाम निर्जिताऽजातशत्रुणा}


\twolineshloka
{न संस्थातुमनीकानि न च राजन्पराक्रमे}
{आसीद्बुद्धिर्हते शल्ये तव योधस्य कस्यचित्}


\twolineshloka
{भीष्मे द्रोणे च निहते सूतपुत्रे च पातिते}
{यद्दुःखं तव योधानां भयं चासीद्विशाम्पते}


\twolineshloka
{तद्भयं स च नः शोको भूय एवाभ्यवर्तत}
{निराशाश्च जये राजन्हते शल्ये महारथे}


\twolineshloka
{हतप्रवीरा विध्वस्ता निकृत्ताश्च शितैः शरैः}
{मद्रराजे हते राजन्योधास्ते प्राद्रवन्भयात्}


\twolineshloka
{अश्वानन्ये गजानन्ये रथानन्ये महारथाः}
{आरुह्य जवसम्पन्ना पादाताः प्राद्रवंस्तथा}


\twolineshloka
{द्विसाहस्राश्च महातङ्गा गिरिरूपाः प्रहारिणः}
{xxप्राद्रवन्हते शल्ये अङ्कुशाङ्गुष्ठनोदिताः}


\twolineshloka
{ते रणाद्भरतश्रेष्ठ तावकाः प्राद्रवन्दिशः}
{धावतश्चाप्यपश्याम श्वसमानाञ्शराहतान्}


\twolineshloka
{तान्प्रभग्नान्हतान्दृष्ट्वा हतोत्साहान्पराजितान्}
{अभ्यवर्तन्त पाञ्चालाः पाण्डवाश्च जयैषिणः}


\twolineshloka
{बाणशब्दरवाश्चापि सिंहनादाश्च पुष्कलाः}
{शङ्खशब्दश्च शूराणां दारुणः समपद्यत}


\twolineshloka
{दृष्ट्वा तु कौरवं सैन्यं भयत्रस्तं प्रविद्रुतम्}
{अन्योन्यं समभाषन्त पाञ्चालाः पाण्डवैः सह}


\twolineshloka
{अद्य राजा सत्यधृतिर्हतामित्रो युधिष्ठिरः}
{अद्य दुर्योधनो हीनो दीप्तया नृपतिश्रिया}


\twolineshloka
{अद्य श्रुत्वा हतं पुत्रं धृतराष्ट्रो जनेश्वरः}
{निःसंज्ञः पतितो भूमौ किल्बिषं प्रतिपत्स्यते}


\twolineshloka
{अद्य जानाति बीभत्सुं समर्थं स्वधन्विनाम्}
{अद्यात्मानं च दुर्मेधा गर्हयिष्यति पापकृत्}


\threelineshloka
{अद्य क्षत्तुर्वचः सत्यं स्मरतां ब्रुवतो हितम्}
{अद्यप्रभृति पार्थं च प्रेष्यभूत इवाचरन्}
{विजानातु नृपो दुःखं यत्प्राप्तं पाण्डुनन्दनैः}


\twolineshloka
{अद्य कृष्णस्य माहात्म्यं विजानातु महीपतिः}
{अद्यार्जुनधनुर्घोषं घोरं जानातु संयुगे}


\twolineshloka
{अत्त्राणां च बलं सर्वं बाह्वोश्च बलमाहवे}
{`पुत्राणां च वधं घोरं भीमेन श्रोष्यते नृपः'}


\twolineshloka
{अद्य ज्ञास्यति भीमस्य बलं घोरं महात्मनः}
{हते दुर्योधने युद्धे शक्रेणेव तु शम्बरे}


\threelineshloka
{यत्कृतं भीमसेनेन दुःशासनवधे तदा}
{कोऽन्यः कर्ताऽस्ति तल्लोके ऋते भीमान्महाबलात्}
{}


\twolineshloka
{अद्य ज्येष्ठस्य जानीतां पाण्डवस्य पराक्रमम्}
{मद्रराजं हतं श्रुत्वा देवैरपि सुदुःसहम्}


\twolineshloka
{अद्य ज्ञास्यति सङ्ग्रामे माद्रीपुत्रौ सुदुःसहौ}
{निहते सौबले वीरे गान्धारेषु च सर्वशः}


\twolineshloka
{xxxx न तेषां स्याद्येषां योद्धा धनञ्जयः}
{xxxxx धृष्टद्युम्नश्च पार्षतः}


\twolineshloka
{xxxx पञ्च माद्रीपुत्रौ च पाण्डवौ}
{शिखण्डी च महेष्वासो राजा चैव युधिष्ठिरः}


\twolineshloka
{येषां च जगतो नाथो नाथः कृष्णो जनार्दनः}
{कथं तेषां जयो न स्याद्येषां धर्मो व्यपाश्रयः}


\twolineshloka
{`लाभस्तेषां जयस्तेषां कुतस्तेषां पराभवः}
{येषां नाथो हृषीकेशः सर्वलोकविभुर्हरिः'}


\twolineshloka
{भीष्मं द्रोणं च कर्णं च मद्रराजानमेव च}
{तथाऽन्यान्नृपतीन्वीराञ्शतशोऽथ सहस्रशः}


\twolineshloka
{कोऽन्यः शक्तो रणे जेतुमृते पार्थाद्युधिष्ठिरात्}
{यस्य नाथो हृषीकेशः सदा सत्ययशोनिधिः}


\twolineshloka
{इत्येवं वदमानास्ते हर्षेण महता युताः}
{प्रभग्नांस्तावकान्योधान्संहृष्टाः पृष्ठतोऽन्वयुः}


\twolineshloka
{धनञ्जयो रथानीकमभ्यवर्तत वीर्यवान्}
{माद्रीपुत्रौ च शकुनिं सात्यकिश्च महारथः}


\twolineshloka
{तान्प्रेक्ष्य द्रवतः सर्वान्भीमसेनभयार्दितान्}
{दुर्योधनस्तदा सूतमब्रवीद्विस्मयन्निव}


\twolineshloka
{मामतिक्रमते पार्थो धनुष्पाणिमवस्थितम्}
{जघने सर्वसैन्यानां शनैरश्वान्प्रचोदय}


\twolineshloka
{जघने युध्यमानं हि कौन्तेयो मां न संशयः}
{नोत्सहेदभ्यतिक्रान्तुं वेलामिव महोदधिः}


\twolineshloka
{पश्य सैन्यं महत्सूत पाण्डवैः समभिद्रुतम्}
{सैन्यरेणुं समुद्भूतं पश्य चैनं समन्ततः}


% Check verse!
सिंहनादांश्च समुद्भूतं पश्य चैनं समन्ततः ॥तस्माद्याहि शनैः सूत जघनं परिपालय
\twolineshloka
{मयि स्थिते च समरे निरुद्धेषु च पाण्डुषु}
{पुनरावर्तते तूर्णं मामकं बलमोजसा}


\twolineshloka
{तच्छ्रुत्वा तव पुत्रस्य शूरस्य सदृशं वचः}
{सारथिर्हेमसञ्छन्नाञ्शरैरश्वानचोदयत्}


\twolineshloka
{गजाश्वरथिभिर्हीनास्त्यक्तात्मानः पदातयः}
{एकविंशतिसाहस्राः संयुगायावतस्थिरे}


\twolineshloka
{नानादेशसमुद्भूता नानारञ्जितवाससः}
{अवस्थितास्तदा योधाः प्रार्थयन्तो महद्यशः}


\twolineshloka
{तेषामापततां तत्र संहृष्टानां परस्परम्}
{सम्मर्दः सुमहाञ्चज्ञे घोररूपो भयानकः}


\twolineshloka
{भीमसेनं तदा राजन्धृष्टद्युम्नं च पार्षतम्}
{बलेन चतुरङ्गेण नानादेश्यानवारयत्}


\twolineshloka
{भीममेवाभ्यवर्तन्त रणेऽन्ये तु पदातयः}
{प्रक्ष्वेड्यास्फोट्य संहृष्टा वीरलोकं यियासवः}


\threelineshloka
{आसाद्य भीमसेनं तु संरब्धा युद्धदुर्मदाः}
{धार्तराष्ट्रा विनेदुर्हि नान्यामकथनयन्कथाम्}
{परिवार्य रणे भीमं निजघ्नुस्ते समन्ततः}


\twolineshloka
{स वध्यमानः समरे पदातिगणसंवृतः}
{न चचाल ततः स्थानान्मैनाक इव पर्वतः}


\twolineshloka
{ते तु क्रुद्धा महाराज पाण्डवस्य महारथम्}
{निग्रहीतुं प्रवृत्ता हि योधांश्चान्यानवारयन्}


\twolineshloka
{अक्रुध्यत रणे भीमस्तैस्तदा पर्यवस्थितैः}
{सोऽवतीर्य रथात्तूर्णं पदातिः समवस्थितः}


\twolineshloka
{जातरूपप्रतिच्छन्नां प्रगृह्य महतीं गदाम्}
{अवधीत्तावकान्योधान्दण्डपाणिरिवान्तकः}


\twolineshloka
{रथाश्वविप्रहीणांस्तु तान्भीमो गदया बली}
{एकविंशतिसाहस्रान्पदातीन्समपोथयत्}


\twolineshloka
{हत्वा तत्पुरुषानीकं भीमः सत्यपराक्रमः}
{धृष्टद्युम्नं पुरस्कृत्य न चिरात्प्रत्यदृश्यत}


\twolineshloka
{पादाता निहता भूमौ शिश्यिरे रुधिरोक्षितः}
{सम्भग्ना इव वातेन कर्णिकाराः सुपुष्पिताः}


\twolineshloka
{नानाशस्त्रसमायुक्ता नानाकुण्डलधारिणः}
{नानाजात्या हतास्तत्र नादेशसमागताः}


\twolineshloka
{पताकाध्वजसञ्छन्नं पदातीनां महद्बलम्}
{निकृत्तं विबभौ रौद्रं घोररूपं भयावहम्}


\twolineshloka
{युधिष्ठिरपुरोगाश्च सहसैन्या महारथाः}
{अभ्यधावन्महात्मानं पुत्रं दुर्योधनं तव}


\twolineshloka
{ते सर्वे तावकान्दृष्ट्वा महेष्वासान्पराङ्मुखान्}
{नाभ्यवर्तन्त ते पुत्रं वेलामिव महोर्मयः}


\twolineshloka
{तदद्भुतमपश्याम तव पुत्रस्य पौरुषम्}
{यदेकं सहिताः पार्था न शेकुरतिवर्तितुम्}


\twolineshloka
{नातिदूरापयातं तु कृतबुद्वि पलायने}
{दुर्योधनः स्वकं सैन्यमब्रवीद्भृशविक्षतम्}


\twolineshloka
{न तं देशं प्रपश्यामि पृथिव्यां पर्वतेषु च}
{यत्र यातान्न वा हन्युः पाण्डवाः किं सृतेन वः}


\twolineshloka
{अल्पं च बलमेतेषां कृष्णौ च भृशविक्षतौ}
{यदि सर्वेऽत्र तिष्ठामो ध्रुवं नो विजयो भवेत्}


\twolineshloka
{विप्रयातांस्तु वो भिन्नान्पाण्डवाः कृतविप्रियाः}
{अनुयाय हनिष्यन्ति श्रेयान्नः समरे वधः}


\twolineshloka
{शृण्वन्तु क्षत्रियाः सर्वे यावन्तोऽत्र समागताः}
{यदा शूरं च भीरुं च मारयत्यन्तकः सदा}


\twolineshloka
{को नु मूढो न युध्येत पुरुषः क्षत्रियो ध्रुवम्}
{श्रेयान्नो भीमसेनस्य क्रुद्धस्याभिमुखे स्थितः}


\twolineshloka
{सुखः साङ्ग्रामिको मृत्युर्दुःखो व्याधिजरादिभिः}
{मर्त्येनावश्यमर्तव्यं गृहेष्वपि कदाचन}


\twolineshloka
{युध्यतः क्षत्रधर्मेण मृत्युरेष सनातनः}
{हत्वेह सुखमाप्नोति हतः प्रेत्य महत्फलम्}


\twolineshloka
{न युद्धधर्माच्छ्रेयान्वै पन्थाः स्वर्गस्य कौरवाः}
{अचिरेणैव ताँल्लोकान्हतो युद्धे समश्नुते}


\twolineshloka
{श्रुत्वा तद्वचनं तस्य पूजयित्वा च पार्थिवाः}
{पुनरेवाभ्यवर्तन्त पाण्डवानाततायिनः}


\twolineshloka
{तानापतत एवाशु व्यूढानीकाः प्रहारिणः}
{प्रत्युद्ययुस्तदा पार्था जयगृद्धाः प्रमन्यवः}


\twolineshloka
{धनञ्जयो रथेनाजावभ्यवर्तत वीर्यवान्}
{विश्रुतं त्रिषु लोकेषु व्याक्षिपन्गाण्डिवं धनुः}


\twolineshloka
{माद्रीपुत्रौ च शकुनिं सात्यकिश्च महाबलः}
{जवेनाभ्यपतन्हृष्टा यत्ता वै तावकं बलम्}


\chapter{अध्यायः २०}
\twolineshloka
{सञ्जरा उवाच}
{}


\twolineshloka
{सन्निवृत्तेषु सैन्येषु साल्वो म्लेच्छगणाधिपः}
{अभ्यद्रवत्सुसङ्क्रुद्धः पाण्डवानां महद्बलम्}


\twolineshloka
{आस्थाय सुमहानागं प्रभिन्नं पर्वतोपमम्}
{सप्तमैरावतप्रख्यममित्रगणमर्दनम्}


\twolineshloka
{योऽसौ महान्भद्रकुलप्रसूतःसुपूजितो धार्तराष्ट्रेण नित्यम्}
{सुकल्पितः शास्त्रविनिश्चयज्ञैःसदौपवाह्यः समरेषु राजन्}


\twolineshloka
{ऐरावतं दैत्यगणान्विमृद्र--ञ्शक्रो यथा सञ्जनयन्भयानि}
{तमास्थितो राजवरो बभूवयथोदयस्थः सविता क्षपान्ते}


\twolineshloka
{स तेन नागप्रवरेण राज--न्नभ्युद्ययौ पाण्डुसुतान्समेतान्}
{शितैः पृषत्कैर्विददार वेगै--र्महेन्द्रवज्रप्रतिमैः सुघोरैः}


\twolineshloka
{ततः शरान्वै सृजतो महारणेयोधांश्च राजन्नयतो यमालयम्}
{नास्यानत्रं ददृशुः स्वे परे वायथा पुरा वज्रधरस्य दैत्याः}


\twolineshloka
{ते पाण्डवाः सोमकाः सृञ्जयाश्चतमेकनागं ददृशुः समन्तात्}
{सहस्रशो वै विचरन्तमेकंयथा महेन्द्रस्य गजं समीपे}


\twolineshloka
{सन्द्राव्यमाणं तु बलं परेषांपरेतकल्पं विबभौ समन्ततः}
{नैवावतस्थे समरे भृशं भया--द्विमृद्यमानं तु परस्परं तदा}


\twolineshloka
{ततः प्रभग्ना सहसा महाचमूःसा पाण्डवी तेन नराधिपेन}
{दिशश्चतस्रः सहसा विधावितागजेन्द्रवेगं तमपारयन्ती}


\twolineshloka
{दृष्ट्वा च तां वेगवता प्रभग्नांसर्वे त्वदीया युधि योधमुख्याः}
{अपूजयंस्ते तु नराधिपं तंदध्मुश्च शङ्खाञ्शशिसन्निकाशान्}


\twolineshloka
{श्रुत्वा निनादं त्वथ कौरवाणांहर्षाद्विमुक्तं सह शङ्खशब्दैः}
{सेनापतिः पाण्डवसृञ्जयानांपाञ्चालपुत्रो ममृषे न कोपात्}


\twolineshloka
{ततस्तु तं वै द्विरदं महात्माप्रत्युद्ययौ त्वरमाणो जयाय}
{जम्भो यथा शक्रसमागमे वैनागेन्द्रमैरावणमिन्द्रवाह्यम्}


\twolineshloka
{तमापतन्तं सहसा तु दृष्ट्वापाञ्चालपुत्रं युधि राजसिंहः}
{तं वै द्विपं प्रेषयामास तूर्णंवधाय राजन्द्रुपदात्मजस्य}


\twolineshloka
{स तं द्विपेन्द्रं सहसापतन्त--मविध्यदग्निप्रतिमैः पृषत्कैः}
{कर्मारधौतैर्निशितैर्ज्वलद्भि--र्नाराचमुख्यैस्त्रिभिरुग्रवेगैः}


\twolineshloka
{ततोऽपरान्पञ्चशतान्महात्मानाराचमुख्यान्विससर्ज कुम्भे}
{स तैस्तु विद्धः परमद्विपो रणेतदा परावृत्य भृशं प्रदुद्रुवे}


\twolineshloka
{तं नागराजं सहसा प्रणुन्नंविद्राव्यमाणं विनिवर्त्य साल्वः}
{तोत्राङ्कुशैः प्रेषयामास तूर्णंपाञ्चालराजस्य सुतं प्रदिश्य}


\twolineshloka
{दृष्ट्वा पतन्तं सहसा तु नागंधृष्टद्युम्नः स्वरथाच्छीघ्रमेव}
{गदां प्रगृह्योग्रजवेन वीरोभूमिं प्रपन्नो भयविह्वलाङ्गः}


\twolineshloka
{स तं रथं हेमविभूषिताङ्गंसाश्वं ससूतं सहसा विमृद्य}
{उत्क्षिप्य हस्तेन नदन्महाद्विपोविपोथयामास वसुन्धरातले}


\twolineshloka
{पाञ्चालराजस्य सुतं च दृष्ट्वातदार्दितं नागवरेण तेन}
{तमभ्यधावत्सहसा जवेनभीमः शिखण्डी च शिनेश्च नप्ता}


\twolineshloka
{शरैश्च वेगं सहसा निगृह्यतस्याभितो व्यापततो गजस्य}
{स सङ्गृहीतो रथिभिर्गजो वैचचाल तैर्वार्यमाणश्च सङ्ख्ये}


\twolineshloka
{ततः पृषत्कान्प्रववर्ष राजासूर्यो यथा रश्मिजालं समन्तात्}
{तैराशुगैर्वध्यमाना रथौघाःप्रदुद्रुवुः सहितास्तत्रतत्र}


\twolineshloka
{तत्कर्म साल्वस्य समीक्ष्य सर्वेपाञ्चालपुत्रा नृप सृञ्जयाश्च}
{हाहाकारैर्नादयन्ति स्म युद्धेद्विपं समन्ताद्रुरुधुर्नराग्र्याः}


\twolineshloka
{पाञ्चालपुत्रस्त्वरितस्तु शूरोगदां प्रगृह्याचलशृङ्गकल्पाम्}
{ससम्भ्रमं भारत शत्रुघातीजवेन वीरोऽनुससार नागम्}


\twolineshloka
{ततस्तु नागं धरणीधराभंमदं स्रवन्तं जलदप्रकाशम्}
{गदां समातिध्य भृशं जघानपाञ्चालराजस्य सुतस्तरस्वी}


\twolineshloka
{स भिन्नकुम्भः सहसा विनद्यमुखात्प्रभूतं क्षतजं विमुञ्चन्}
{पपात नागो धरमीधराभःक्षितिप्रकम्पाच्चलितो यथाऽद्रिः}


\twolineshloka
{निपात्यमाने तु तदा गजेन्द्रेहाहाकृते तव पुत्रस्य सैन्ये}
{स साल्वराजस्य शिनिप्रवीरोजहार भल्लेन शिरः शितेन}


\twolineshloka
{हृतोत्तमाङ्गो युधि सात्वतेनपपात भूमौ सह नागराज्ञा}
{यथाऽद्रिशृङ्गं सुमहत्प्रणुन्नंवज्रेण देवाधिपचोदितेन}


\chapter{अध्यायः २१}
\twolineshloka
{सञ्जय उवाच}
{}


\twolineshloka
{तस्मिंस्तु निहते शूरे साल्वे समितिशोभने}
{तवाभज्यद्बलं वेगाद्वातेनेव महाद्रुमः}


\twolineshloka
{तत्प्रभग्नं बलं दृष्ट्वा कृतवर्मा महारथः}
{दधार समरे शूरः शत्रुसैन्यं महाबलः}


\twolineshloka
{सन्निवृत्तास्तु ते शूरा दृष्ट्वा सात्वतमाहवे}
{शैलोपमं स्थिरं राजन्कीर्यमाणं शरैर्युधि}


\twolineshloka
{ततः प्रववृते युद्धं कुरूणां पाण्डवैः सह}
{निवृत्तानां महाराज मृत्युं कृत्वा निवर्तनम्}


\twolineshloka
{तत्राश्चर्यमभूद्युद्वं सात्वतस्य परैः सह}
{यदेको वारयामास पाण्डुसेनां दुरासदाम्}


\twolineshloka
{तेषामन्योन्यसुहृदां कृते कर्मणि दुष्करे}
{सिंहनादः प्रहृष्टानां दिविस्पृक्सुमहानभूत्}


\twolineshloka
{तेन शब्देन वित्रस्तान्पाञ्चालान्भरतर्षभ}
{शिनेर्नप्ता महाबाहुरन्वपद्यत सात्यकिः}


\twolineshloka
{स समासाद्य राजानं क्षेमधूर्तिं महाबलम्}
{सप्तभिर्निशितैर्बाणैरनयद्यमसादनम्}


\twolineshloka
{तमायान्तं महाबाहुं प्रवपन्तं शिताञ्शरान्}
{जवेनाभ्यपतद्वीमान्हार्दिक्यः शिनिपुङ्गवम्}


\twolineshloka
{तौ सिंहाविव नर्दन्तौ धन्विनौ रथिनां वरौ}
{अन्योन्यमभिधावन्तौ शस्त्रप्रवरधारिणौ}


\twolineshloka
{पाण्डवाः सहपाञ्चाला योधाश्चान्ये नृपोत्तमाः}
{प्रेक्षकाः समपद्यन्त तयोः पुरुषसिंहयोः}


\twolineshloka
{नाराचैर्वत्सदन्तैश्च वृष्ण्यन्धकमहारथौ}
{अभिजघ्नतुरन्योन्यं प्रहृष्टाविव कुञ्चरौ}


\twolineshloka
{चरन्तौ विविधान्मार्गान्हार्दिक्यशिनिपुङ्गवौ}
{मुहुरन्तर्दधाते तौ बाणवृष्ट्या परस्परम्}


\twolineshloka
{चापवेगबलोद्वूतान्मार्गणान्वृष्णिसिंहयोः}
{आकाशे समपश्याम पतङ्गानिव शीघ्रगान्}


\twolineshloka
{तमेकं सत्यकर्माणमासाद्य हृदिकात्मजः}
{अविध्यन्निशितैर्बाणैश्चतुर्भिश्चतुरो हयान्}


\twolineshloka
{स दीर्घबाहुः सङ्क्रुद्धस्तोत्रार्दित इव द्विपः}
{अष्टभिः कृतवर्माणमविध्यत्परमेषुभिः}


\twolineshloka
{ततः पूर्णायतोत्सृष्टैः कृतवर्मा शिलाशितैः}
{सात्यकिं त्रिभिराहत्य धनुरेकेन चिच्छिदे}


\twolineshloka
{निकृत्तं तद्धनुःश्रेष्ठमपास्य शिनिपुङ्गवः}
{अन्यदादत्त वेगेन शैनेयः सशरं धनुः}


\twolineshloka
{तदादाय धनुःश्रेष्ठं वरिष्ठः सर्वधन्विनाम्}
{आरोप्य च धनुः शीघ्रं महावीर्यो महाबलः}


\twolineshloka
{अमृष्यमाणो धनुषश्छेदनं कृतवर्मणा}
{कुपितोऽतिरथः शीघ्रं कृतवर्माणमभ्ययात्}


\twolineshloka
{ततः सुनिशितैर्बाणैर्दशभिः शिनिपुङ्गवः}
{जघान सूतं चाश्वांश्च ध्वजं च कृतवर्मणः}


\twolineshloka
{ततो राजन्महेष्वासः कृतवर्मा महारथः}
{हताश्वसूतं सम्प्रेक्ष्य रथं हेमपरिष्कृतम्}


\twolineshloka
{रोषेण महताऽऽविष्टः शूलमुद्यम्य मारिष}
{चिक्षेप भुजवेगेन जिघांसुः शिनि पुङ्गवम्}


\threelineshloka
{तच्छूलं सात्वतो ह्याजौ निर्भिद्य निशितैः शरैः}
{चूर्णितं पातयामास मोहयन्निव माधवम्}
{ततोऽपरेण भल्लेन हृद्येनं समताडयत्}


\twolineshloka
{सुयुद्वे युयुधानेन हताश्वो हतसारथिः}
{कृतवर्मा कृतास्त्रेण धरमीमन्वपद्यत}


\twolineshloka
{तस्मिन्सात्यकिना वीरे द्वैरथे विरथीकृते}
{समपद्यत सर्वेषां सैन्यानां सुमहद्भयम्}


\twolineshloka
{पुत्राणां तव चात्यर्थं विषादः समजायत}
{हतसूते हताश्वे तु विरथे कृतवर्मणि}


\twolineshloka
{हताश्वं च समालक्ष्य हतसूतमरिन्दम}
{अभ्यधावत्कृपो राजञ्जिघांसुः शिनिपुङ्गवम्}


\twolineshloka
{तमारोप्य रथोपस्थे मिषतां सर्वधन्विनाम्}
{अपोवाह महाबाहुं तूर्णमायोधनादपि}


\twolineshloka
{शैनेयेऽधिष्ठिते राजन्विरथे कृतवर्मणि}
{दुर्योधनबलं सर्वं पुनरासीत्पराङ्मुखम्}


\twolineshloka
{तत्परे नान्वबुध्यन्त सैन्येन रजसा वृताः}
{तावकाः प्रद्रुता राजन्दुर्योधनमृते नृपम्}


\twolineshloka
{दुर्योधनस्तु सम्प्रेक्ष्य भग्नं स्वबलमन्तिकात्}
{जवेनाभ्यपतत्तूर्णं सर्वांश्चैको न्यवारयत्}


\twolineshloka
{पाण्डूंश्च सर्वान्सङ्क्रुद्धो धृष्टद्युम्नं च पार्षतम्}
{शिखण्डिनं द्रौपदेयान्पाञ्चालानां च ये गणाः}


\twolineshloka
{केकयान्सोमकांश्चैव सृञ्जयांश्चैव मारिष}
{असम्भ्रमं दुराधर्षः शितैर्बाणैरवाकिरत्}


\threelineshloka
{अतिष्ठदाहवे यत्तः पुत्रस्तव महाबलः}
{यथा यज्ञे महानग्निर्मन्त्रपूतः प्रकाशते}
{तथा दुर्योधनो राजा सङ्ग्रामे सर्वतोऽभवत्}


\twolineshloka
{तं परे नाभ्यवर्तन्त मर्त्या मृत्युमिवाहवे}
{अथान्यं रथमास्थाय हार्दिक्यः समपद्यत}


\chapter{अध्यायः २२}
\twolineshloka
{सञ्जय उवाच}
{}


\twolineshloka
{पुत्रस्तु ते महाराज रथस्थो रथिनां वरः}
{दुरुत्सहो बभौ युद्धे यथा रुद्रः प्रतापवान्}


\twolineshloka
{तस्य बाणसहस्रैस्तु प्रच्छन्ना ह्यभवन्मही}
{परांश्च सिषिचे बाणैर्धाराभिरिव पर्वतान्}


\twolineshloka
{न च सोऽस्ति पुमान्कश्चित्पाण्डवानां बलार्णवे}
{हयो गजो रथो वाऽपि यः स्याद्बाणैरविक्षतः}


\twolineshloka
{यं यं हि समरे योधं प्रपश्यामि विशाम्पते}
{स स बाणैश्चितोऽभूद्वै पुत्रेण तव भारत}


\twolineshloka
{यथा सैन्येन रजसा समुद्भूतेन वाहिनी}
{प्रत्यदृश्यत सञ्छन्ना तथा बाणैर्महात्मनः}


\twolineshloka
{बाणभूतामपश्याम पृथिवीं पृथिवीपते}
{दुर्योधनेन प्रकृतां क्षिप्रहस्तेन धन्विना}


\twolineshloka
{तेषु योधसहस्रेषु तावकेषु परेषु च}
{नास्ति दुर्योधनसमः पुमानिति मतिर्मम}


\twolineshloka
{तत्राद्भुतमपश्याम तव पुत्रस्य विक्रमम्}
{यदेकं सहिताः पार्था नाभ्यवर्तन्त भारत}


\twolineshloka
{युधिष्ठिरं शतेनाजौ विव्याध भरतर्षभ}
{भीमसेनं च सप्तत्या सहदेवं च पञ्चभिः}


\twolineshloka
{नकुलं च चतुःषष्ट्या धृष्टद्युम्नं च पञ्चभिः}
{पञ्चभिर्द्रौपदेयांश्च त्रिभिर्विव्याध सात्यकिम्}


\twolineshloka
{धनुश्चिच्छेद भल्लेन सहदेवस्य मारिष}
{तदपास्य धनुश्छिन्नं माद्रीपुत्रः प्रतापवान्}


\twolineshloka
{अभ्यद्रवत राजानं प्रगृह्यान्यन्महद्धनुः}
{ततो दुर्योधनं सङ्ख्ये विव्याध दशभिः शरैः}


\twolineshloka
{नकुलस्तु ततो वीरो राजानं नवभिः शरैः}
{घोररूपैर्महेष्वासो विव्याध च ननाद च}


\threelineshloka
{सात्यकिश्चैव राजानं शरेणानतपर्वणा}
{द्रौपदेयास्त्रिसप्तत्या धर्मराजश्च पञ्चभिः}
{अशीत्या भीमसेनश्च शरै राजानमार्पयन्}


\twolineshloka
{समन्तात्कीर्यमाणस्तु बाणसङ्घैर्महात्मभिः}
{न चचाल महाराज सर्वसैन्यस्य पश्यतः}


\twolineshloka
{लाघवात्सौष्ठवाच्चापि वीर्याच्चापि महात्मनः}
{अति सर्वाणि भूतानि ददृशुः सर्वपार्थिवाः}


\twolineshloka
{धार्तराष्ट्रा हि राजेन्द्र योधास्तु स्वल्पमन्तरम्}
{अपश्यमाना राजानं पर्यवर्तन्त दंशिताः}


\twolineshloka
{तेषामापततां घोरस्तुमुलः समपद्यत}
{क्षुब्धस्य हि समुद्रस्य प्रावृट््काले यथा स्वनः}


\twolineshloka
{समासाद्य रणे ते तु राजानमपराजितम्}
{प्रत्युद्ययुर्महेष्वासाः पाण्डवानाततायिनः}


\threelineshloka
{भीमसेनं रणे क्रुद्धो द्रोणपुत्रो न्यवारयत्}
{तयोर्बाणैर्महाराज प्रमुक्तैः सर्वतोदिशम्}
{नाज्ञायन्त रणे वीरा न दिशः प्रदिशस्तथा}


\twolineshloka
{तावुभौ क्रूरकर्माणावुभौ भारतदुःसहौ}
{घोररूपमयुध्येतां कृतप्रतिकृतैषिणौ}


\twolineshloka
{त्रासयन्तौ दिशः सर्वा ज्याक्षेपकठिनत्वचौ}
{शकुनिस्तु रणे वीरो युधिष्ठिरमपीडयत्}


\twolineshloka
{तस्याश्वांश्चतुरो हत्वा सुबलस्य सुतो विभो}
{नादं चकार बलवत्सर्वसैन्यानि कम्पयन्}


\twolineshloka
{एतस्मिन्नन्तरे वीरं राजानमपराजितम्}
{अपोवाह रथेनाजौ सहदेवः प्रतापवान्}


\threelineshloka
{अथान्यं रथमास्थाय धर्मपुत्रो युधिष्ठिरः}
{शकुनिं नवभिर्विद्ध्वा पुनर्विव्याध पञ्चभिः}
{ननाद च महानादं प्रवरः सर्वधन्विनाम्}


\twolineshloka
{तद्युद्धमभवच्चित्रं घोररूपं च मारिष}
{प्रेक्षतां प्रीतिजननं सिद्धचारणसेवितम्}


\twolineshloka
{उलूकस्तु महेष्वासं नकुलं युद्धदुर्मदम्}
{अभ्यवर्षदमेयात्मा शरवर्षैः समन्ततः}


\twolineshloka
{तथैव नकुलः शूरः सौबलस्य सुतं रणे}
{शरवर्षेण महता समन्तात्पर्यवारयत्}


\twolineshloka
{तौ तत्र समरे वीरौ कुलपुत्रौ महारथौ}
{योधयन्तावपश्येतां कृतप्रतिकृतैषिणौ}


\twolineshloka
{तथैव कृतवर्माणं शैनेयः शत्रुतापनः}
{योधयञ्शुशुभे राजन्बलिं शक्र इवाहवे}


\twolineshloka
{दुर्योधनो धनुश्छित्त्वा धृष्टद्युम्नस्य संयुगे}
{अथैनं छिन्नधन्वानं विव्याध निशितैः शरैः}


\twolineshloka
{धृष्टद्युम्नोऽपि समरे प्रगृह्य परमायुधम्}
{राजानं योधयामास पश्यतां सर्वधन्विनाम्}


\twolineshloka
{तयोर्युद्धं महच्चासीत्सङ्ग्रामे भरतर्षभ}
{प्रभिन्नयोर्यथा सक्तं मत्तयोर्वनहस्तिनोः}


\twolineshloka
{गौतमस्तु रणे क्रुद्धो द्रौपदेयान्महाबालान्}
{विव्याध बहुभिः शूरः शरैः सन्नतपर्वभिः}


\twolineshloka
{तस्य तैरभवद्युद्धमिन्द्रियैरिव देहिनः}
{घोररूपमसंवार्यं निर्मर्यादमवर्तत}


\twolineshloka
{ते च सम्पीडयामासुरिन्द्रियाणीव बालिशम्}
{स च तान्प्रतिसंरब्धः प्रत्ययोधयदाहवे}


\twolineshloka
{एवं चित्रमभूद्युद्धं तस्य तैः सह भारत}
{उत्थायोत्थाय हि यथा देहिनामिन्द्रियैर्विभो}


\twolineshloka
{नराश्चैव नरैः सार्धं दन्तिनो दन्तिभिस्तथा}
{हया हयैः समासक्ता रथिनो रथिभिः सह}


% Check verse!
सङ्कुलं चाभवद्भूयो घोररूपं विशाम्पते
\twolineshloka
{इदं चित्रमिदं घोरमिदं रौद्रमिति प्रभो}
{युद्धान्यासन्महाराज घोरांणि च बहूनि च}


\twolineshloka
{ते समासाद्य समरे परस्परमरिन्दमाः}
{व्यनदंश्चैव जघ्नुश्च समाताद्य महाहवे}


\twolineshloka
{तेषां पत्रसमुद्भूतं रजस्तीव्रमदृश्यत}
{वातेन चोद्धतं राजन्धावद्भिश्चाश्वसादिभिः}


\twolineshloka
{रथनेमिसमुद्भूतं निःश्वासैश्चापि दन्तिनाम्}
{रजः सन्ध्याभ्रकलिलं दिवाकरपथं ययौ}


\twolineshloka
{रजसा तेन सम्पृक्तो भास्करो निष्प्रभः कृतः}
{सञ्छादिताऽभवद्भूमिस्ते च शूरा महारथाः}


\twolineshloka
{मुहूर्तादिव संवृत्तं नीरजस्कं समन्ततः}
{वीरशोणितसिक्तायां भूमौ भरतसत्तम}


% Check verse!
उपाशाम्यत्ततस्तीव्रं तद्रजो घोरदर्शनम्
\twolineshloka
{ततोऽपश्यमहं भूयो द्वन्द्वयुद्धानि भारत}
{यथाप्राणं यथाश्रेष्ठं मध्याह्ने वै सुदारुणम्}


\twolineshloka
{वर्मणां तत्र राजेन्द्र व्यदृश्यन्तोज्ज्वलाः प्रभाः}
{शब्दश्च तुमुलः सङ्ख्ये शराणां पततामभूत्महावेणुवनस्येव दह्यमानस्य पर्वते}


\chapter{अध्यायः २३}
\twolineshloka
{सञ्जय उवाच}
{}


\twolineshloka
{वर्तमाने तदा युद्धे घोररूपे भयानके}
{अभज्यत बलं तत्र तव पुत्रस्य पाण्डवैः}


\twolineshloka
{तांस्तु सर्वानयन्तेन सन्निवार्य महारथाः}
{पुत्रास्ते योधयामासुः पाण्डवानामनीकिनीम्}


\threelineshloka
{निवृत्ताः सहसा योधास्तव पुत्रजयैपिणः}
{सन्निवृत्तेषु तेष्वेवं युद्धमासीत्सुदारुणम्}
{तावकानां परेषां च देवासुररणोपभम्}


% Check verse!
परेषां तावकानां च नासीत्कश्चित्पराङ्मुखः
\twolineshloka
{अनुमानेन युध्यन्ते सञ्ज्ञाभिश्च परस्परम्}
{तेषां क्षयो महानासीद्युध्यतामितरेतरम्}


\twolineshloka
{ततो युधिष्ठिरो राजा क्रोधेन महता युतः}
{जिगीषमाणः सङ्ग्रामे धार्तराष्ट्रान्सराजकान्}


\twolineshloka
{त्रिभिः शारद्वतं विद्व्वा रुक्मपुङ्खैः शिलाशितैः}
{चतुर्भिर्निजघानाश्वान्कल्याणान्कृतवर्मणः}


\twolineshloka
{अश्वत्थामा तु हार्दिक्यमपोवाह यशस्विनम्}
{अथ शारद्वतोऽष्टाभिः प्रत्यविध्यद्युधिष्ठिरम्}


\twolineshloka
{ततो दुर्योधनो राजा रथान्सप्तशतान्रणे}
{प्रैषयद्यत्र राजाऽसौ धर्मपुत्रो युधिष्ठिरः}


\twolineshloka
{ते रथा रथिभिर्युक्ता मनोमारुतरंहसः}
{अभ्यद्रवन्त सङ्ग्रामे कौन्तेयस्य रथं प्रति}


\twolineshloka
{ते समन्तान्महाराज परिवार्य युधिष्ठिरम्}
{अदृश्यं सायकैश्चक्रुर्मेघा इव दिवाकरम्}


\twolineshloka
{तं दृष्ट्वा धर्मराजानं कौरवेयैस्तथावृतम्}
{नामृष्यन्त सुसंरब्धाः शिखण्डिप्रमुखा रथाः}


\twolineshloka
{रथैरश्ववरैर्युक्तैः किङ्किणीजालसंवृतैः}
{आजग्मुरथ रक्षन्तः कुन्तीपुत्रं युधिष्ठिरम्}


\twolineshloka
{ततः प्रववृते रौद्रः सङ्ग्रामः शोणितोदकः}
{पाण्डवानां कुरूणां च यमराष्ट्रविवर्धनः}


\twolineshloka
{रथान्सप्तशतान्हत्वा कुरूणामाततायिनाम्}
{पाण्डवाः सह पाञ्चालैः पुनरेवाभ्यवारयन्}


\twolineshloka
{तत्र युद्धं महच्चासीत्तव पुत्रस्य पाण्डवैः}
{न च तत्तादृशं दृष्टं नैव चापि परिश्रुतम्}


\twolineshloka
{वर्तमाने तदा युद्धे निर्मर्यादे समन्ततः}
{वध्यमानेषु योधेषु तावकेष्वितरेषु च}


\twolineshloka
{विनदत्सु च योधेषु शङ्खवर्यैश्च पूरितैः}
{उत्क्रुष्टैः सिंहनादैश्च गर्जितैश्चैव धन्विनाम्}


\twolineshloka
{अतिप्रवृत्ते युद्धे च छिद्यमानेषु मर्मसु}
{धावमानेषु योधेषु जयगृद्धिषु मारिष}


\twolineshloka
{संहारे सर्वतो जाते पृथिव्यां शोकसम्भवे}
{वह्नीनामुत्तमस्त्रीणां सीमन्तोद्धरणे कृते}


\twolineshloka
{निर्मर्यादे महायुद्धे वर्तमाने सुदारुणे}
{प्रादुरासन्विनाशाय तदोत्पाताः सुदारुणाः}


\twolineshloka
{चचाल शब्दं कुर्वाणा सपर्वतवना मही}
{सदण्डाः सोल्मुका राजन्कीर्यमाणाः समन्ततः}


\twolineshloka
{उल्काः पेतुर्दिवो भूमावाहत्य रविमण्डलम्}
{विष्वग्वाताः प्रादुरासन्नीचैः शर्करवर्षिणः}


\twolineshloka
{अश्रूणि मुमुचुर्नागा वेपथुं चास्पृशन्भृशम्}
{एतान्धोराननादृत्य समुत्पातान्सुदारुणान्}


\twolineshloka
{पुनर्युद्धाय संयत्ताः क्षत्रियास्तस्थुरव्यथाः}
{रमणीये कुरुक्षेत्रे पुण्ये स्वर्गं यियासवः}


\twolineshloka
{ततो गान्धारराजस्य पुत्रः शकुनिरब्रवीत्}
{युध्यध्वमग्रतो यावत्पृष्ठतो हन्मि पाण्डवान्}


\twolineshloka
{ततो नः सम्प्रायातानां मद्रयोधास्तरस्विनः}
{हृष्टाः किलाकिलाशब्दमकुर्वत परे तथा}


\twolineshloka
{अस्मांस्तु पुनरासाद्य लब्धलक्षा दुरासदाः}
{शरासनानि धुन्वन्तः शरवर्षैरवाकिरन्}


\twolineshloka
{ततो हतं परैस्तत्र मद्रराजबलं तदा}
{दुर्योधनबलं दृष्ट्वा पुनरासीत्पराङ्मुखम्}


\twolineshloka
{गान्धारराजस्तु पुनर्वाक्यमाह ततो बली}
{निवर्तध्वमधर्मज्ञा युध्यध्वं किं सृतेन वः}


\twolineshloka
{अनीकं दशसाहस्रमश्वानां भरतर्षभ}
{आसीद्गान्धारराजस्य विशालप्रासयोधिनाम्}


\twolineshloka
{बलेन तेन विक्रम्य वर्तमाने जनक्षये}
{पृष्ठतः पाण्डवानीकमभ्यघ्नन्निशितैः शरैः}


\twolineshloka
{तदभ्रमिव वातेन क्षिप्यमाणं समन्ततः}
{अभज्यत महाराज पाण्डूनां सुमहद्बलम्}


\twolineshloka
{ततो युधिष्ठिरः प्रेक्ष्य भग्नं स्वबलमन्तिकात्}
{अभ्यनोदयदव्यग्रः सहदेवं महाबलम्}


\twolineshloka
{असौ सुबलपुत्रो नो जघनं पीड्य दंशितः}
{सैन्यानि सूदयत्येष पश्य पाण्डव दुर्मतिः}


\twolineshloka
{गच्छ त्वं द्रौपदेयैश्च शकुनिं सौबलं जहि}
{रथानीकमहं धक्ष्ये पाञ्चालसहितोऽनघ}


\twolineshloka
{गच्छन्तु कुञ्जराः सर्वे वाजिनश्च सह त्वया}
{पादाताश्च त्रिसाहस्राः शकुनिं तैर्वृतो जहि}


\twolineshloka
{ततो गजाः सप्तशताश्चापपाणिभिरास्थिताः}
{पञ्च चाश्वसहस्राणि सहदेवश्च वीर्यवान्}


\twolineshloka
{पादाताश्च त्रिसाहस्रा द्रौपदेयाश्च सर्वशः}
{रणे ह्यभ्यद्रवंस्ते तु शकुनिं युद्धदुर्मदम्}


\twolineshloka
{ततस्तु सौबलो राजन्नभ्यतिक्रम्य पाण्डवान्}
{जघान पृष्ठतः सेनां जयगृद्वः प्रतापवान्}


\twolineshloka
{अश्वारोहास्तु संरब्धाः पाण्डवानां तरस्विनाम्}
{प्राविशन्सौबलानीकमभ्यतिक्रम्य तान्रथान्}


\twolineshloka
{ते तत्र सादिनः शूराः सौबलस्य महद्बलम्}
{रणमध्ये व्यतिष्ठन्त शरवर्षैरवाकिरन्}


\twolineshloka
{तदुद्यतगदाप्रासमकापुरुषसेवितम्}
{प्रावर्तत महद्युद्धं राजन्दुर्मन्त्रिते तव}


\twolineshloka
{उपारमन्त ज्याशब्दाः प्रेक्षका रथिनोऽभवन्}
{न हि स्वेषां परेषां वा विशेषः प्रत्यदृश्यत}


\twolineshloka
{शूरबाहुविसृष्टानां शक्तीनां भरतर्षभ}
{ज्योतिषामिव सम्पातमपश्यन्कुरुपाण्डवाः}


\twolineshloka
{ऋष्टिभिर्विमलाभिश्च तत्रतत्र विशाम्पते}
{सम्पतन्तीभिराकाशमावृतं बह्वशोभत}


\twolineshloka
{प्रासानां पततां राजन्रूपमासीत्समन्ततः}
{शलभानामिवाकाशे तदा भरतसत्तम}


\twolineshloka
{रुधिरोक्षितसर्वाङ्गा विप्रविद्धैर्नियन्तृभिः}
{हयाः परिपतन्तिस्म शतशोऽथ सहस्रशः}


\twolineshloka
{अन्योन्यं परिपिष्टाश्च समासाद्य परस्परम्}
{सुविक्षताः स्म दृश्यन्ते वमन्तो रुधिरं मुखैः}


\twolineshloka
{ततोऽभवत्तमो घोरं सैन्येन रजसा वृतम्}
{तानपाक्रमतोऽद्राक्षं तस्माद्देशादरिन्दम}


\twolineshloka
{अश्वान्राजन्मनुष्यांश्च रजसा संवृते सति}
{भूमौ निपतिताश्चान्ये वमन्तो रुधिरं बहु}


\twolineshloka
{केशाकेशि समालग्ना न शेकुश्चेष्टितुं नराः}
{अन्योन्यमश्वपृष्ठेभ्यो विकर्षन्तो महाबलाः}


\twolineshloka
{मल्ला इव समासाद्य निजघ्नुरितरेतरम्}
{अश्वैश्च व्यपकृष्यन्त बहवोऽत्र गतासवः}


\twolineshloka
{भूमौ निपतिताश्चान्ये बहवो विजयैषिणः}
{तत्रतत्र व्यदृश्यन्त पुरुषाः शूरमानिनः}


\twolineshloka
{रक्तोक्षितैश्छिन्नभुजैरवकृष्टशिरोरुहैः}
{व्यदृश्यत मही कीर्णा शतशोऽथ सहस्रशः}


\twolineshloka
{दूरं न शक्यं तत्रासीद्गन्तुमश्वेन केनचित्}
{साश्वारोहैर्हतैरश्वैरावृते वसुधातले}


\threelineshloka
{रुधिरोक्षितसन्नाहैरात्तशस्त्रैरुदायुधैः}
{नानाप्रहरणैर्घोरैः परस्परवधैषिभिः}
{सुसन्निकृष्टे सङ्ग्रामे हतभूयिष्ठसैनिके}


\twolineshloka
{स मुहूर्तं ततो युद्ध्वा सौबलोऽथ विशाम्पते}
{षट््साहस्रैर्हयैः शिष्टेरपायाच्छकुनिस्ततः}


\twolineshloka
{तथैव पाण्डवानीकं रुधिरेण समुक्षितम्}
{षट््साहस्रैर्हयैः शिष्टेरपायाच्छ्रान्तवाहनम्}


\twolineshloka
{अश्वारोहाश्च पाण्डूनामब्रुवन्रुधिरोक्षिताः}
{सुसन्निकृष्टे सङ्ग्रामे भूयिष्ठे त्यक्तजीविताः}


\twolineshloka
{न हि शक्यं रथैर्योद्धुं कुत एव महागजैः}
{रथानेव रथा यान्तु कुञ्जराः कुञ्जरानपि}


\twolineshloka
{प्रतियातो हि शकुनिः स्वमनीकमवस्थितः}
{न पुनः सौबलो राजा योद्धुमभ्यागमिष्यति}


\twolineshloka
{ततस्तु द्रौपदेयाश्च ते च मत्ता महाद्विपाः}
{प्रययुर्यत्र पाञ्चाल्यो धृष्टद्युम्नो महारथः}


\twolineshloka
{सहदेवोऽपि कौरव्य रजोमेघे समुत्थिते}
{एकाकी प्रययौ तत्र यत्र राजा युधिष्ठिरः}


\twolineshloka
{ततस्तेषु प्रयातेषु शकुनिः सौबलः पुनः}
{पार्श्वतोऽभ्यहनत्क्रुद्धो धृष्टद्युम्नस्य वाहिनीम्}


\twolineshloka
{तत्पुनस्तुमुलं युद्धं प्राणांस्त्यक्त्वाऽभ्यवर्तत}
{तावकानां परेषां च परस्परवधैषिणाम्}


\twolineshloka
{ते चान्योन्यमवैक्षन्त तस्मिन्वीरसमागमे}
{योधाः पर्यपतन्राजञ्शतशोऽथ सहस्रशः}


\twolineshloka
{असिभिश्छिद्यमानानां शिरसां लोकसंक्षये}
{प्रादुरासीन्महाञ्शब्दस्तालानां पततामिव}


\threelineshloka
{विमुक्तानां शरीराणां छिन्नानां पततां भुवि}
{सायुधानां च बाहूनामूरूणां च विशाम्पते}
{आसीत्कटकटाशब्दः सुमहान्रोमहर्षणः}


\twolineshloka
{निघ्नन्तो निशितैः शस्त्रैर्भ्रातॄन्पुत्रान्सखीनपि}
{योधाः परिपतन्ति स्म यथाऽऽमिषकृते खगाः}


\twolineshloka
{अन्योन्यं प्रतिसंरब्धाः समासाद्य परस्परम्}
{अहम्पूर्वमहम्पूर्वमिति निघ्नन्सहस्रशः}


\twolineshloka
{संयातेनासनभ्रष्टैरश्वारोहैर्गतासुभिः}
{हयाः परिपतन्ति स्म शतशोऽथ सहस्रशः}


\twolineshloka
{स्फुरतां प्रतिपिष्टानामश्वानां शीघ्रगामिनाम्}
{स्तनतां च मनुष्याणां सन्नद्धानां विशाम्पते}


\twolineshloka
{शक्त्यृष्टिप्रासशब्दश्च तुमुलः समपद्यत}
{भिन्दतां परमर्माणि राजन्दुर्मन्त्रिते तव}


\twolineshloka
{श्रमाभिभूताः संरब्धा श्रान्तवाहाः पिपासवः}
{विक्षताश्च शितैः शस्त्रैरभ्यवर्न्तत तावकाः}


\twolineshloka
{मत्ता रुधिरगन्धेन बहवोऽत्र विचेतसः}
{जघ्नुः परान्स्वकांश्चैव प्राप्तान्प्राप्ताननन्तरान्}


\twolineshloka
{बहवश्च गतप्राणाः क्षत्रिया जयगृद्विनः}
{भूमावभ्यपतन्राजञ्शरवृष्टिभिरावृताः}


\twolineshloka
{वृकगृध्रशृगालानां तुमुले मोदनेऽहनि}
{आसीद्बलक्षयो घोरस्तव पुत्रस्य पश्यतः}


\twolineshloka
{नराश्वकायैः सञ्छन्ना भूमिरासीद्विशाम्पते}
{रुधिरोदकचित्रा च भीरूणां भयवर्धिनी}


\twolineshloka
{असिभिः पट्टसैः शूलैस्तक्षमाणाः पुनःपुनः}
{तावकाः पाण़्वेयाश्च न न्यवर्तन्त भारत}


\twolineshloka
{प्रहरन्तो यथाशक्ति यावत्प्राणस्य धारणम्}
{योधाः परिपतन्ति स्म वमन्तो रुधिरं मुखैः}


\twolineshloka
{शिरो गृहीत्वा केशेषु कबन्धः स प्रदृश्यते}
{उद्यम्य च शितं खङ्गं रुधिरेण परिप्लुतम्}


\twolineshloka
{तथोत्थितेषु बहुषु कबन्धेषु नराधिप}
{तथा रुधिरगन्धेन योधाः कश्मलमाविशन्}


\twolineshloka
{मन्दीभूते ततः शब्दे पाण्डवानां महद्बलम्}
{अल्पावशिष्टैस्तुरगैरभ्यवर्तत सौबलः}


\twolineshloka
{ततोऽभ्यधावंस्त्वरिताः पाण्डवा जयगृद्विनः}
{पदातयश्च नागाश्च सादिनश्चोद्यतायुधाः}


\twolineshloka
{कोष्ठकीकृत्य चाप्येनं परिक्षिप्य च सर्वशः}
{शस्त्रैर्नानाविधैर्जघ्नुर्युद्वपारं तितीर्षवः}


\twolineshloka
{त्वदीयास्तांस्तु सम्प्रेक्ष्य सर्वतः समभिद्रुतान्}
{रथाश्वपत्तिद्विरदाः पाण्डवानभिदुद्रुवुः}


\twolineshloka
{केचित्पदातयः पद्भिर्मुष्टिभिश्च परस्परम्}
{निजघ्नुः समरे शूराः क्षीणशस्त्रास्ततोऽपतन्}


\twolineshloka
{रथेभ्यो रथिनः पेतुर्द्विपेभ्यो हस्तिसादिनः}
{विमानेभ्यो दिवो भ्रष्टाः सिद्वाः पुण्यक्षयादिव}


\twolineshloka
{एवमन्योन्यमायत्ता योधा जघ्नुर्महाहवे}
{पितॄन्भ्रातॄन्वयस्यांश्च पुत्रानपि तथा परे}


\twolineshloka
{एवमासीदमर्यादं युद्वं भरतसत्तम}
{प्रासासिबाणकलिलं वर्तमाने सुदारुणे}


\chapter{अध्यायः २४}
\twolineshloka
{सञ्जय उवाच}
{}


\twolineshloka
{अल्पावशिष्टे सैन्ये तु पाण्डवैर्निहते बले}
{अश्वैः सप्तसहस्रैस्तु उपावर्तत सौबलः}


\twolineshloka
{स यात्वा वाहिनीं तूर्णं चोदयानः स्वकान्युधि}
{युध्यध्वमिति संहृष्टाः पुनःपुनररिन्दमाः}


\twolineshloka
{अपृच्छत्क्षत्रियांस्तत्र क्व नु राजा महाबलः}
{शकुनेस्तद्वचः श्रुत्वा तमूचुर्भरतर्षभ}


\twolineshloka
{असौ तिष्ठति कौरव्यो रणमध्ये महाबलः}
{यत्रैतत्सुमहच्छत्रं पूर्णचन्द्रसमप्रभम्}


\twolineshloka
{यत्र ते सतनुत्राणा रथास्तिष्ठन्ति दंशिताः}
{यत्रैष तुमुलः शब्दः पर्जन्यनिनदोपमः}


\twolineshloka
{तत्र गच्छ द्रुतं राजंस्ततो द्रक्ष्यसि कौरवम्}
{एवमुक्तस्तु तैर्योधैः शकुनिः सौबलस्तदा}


\twolineshloka
{प्रययौ तत्र यत्रास्ते पुत्रस्तव नराधिप}
{सर्वतः संवृतो वीरैः समरे चित्रयोधिभिः}


\twolineshloka
{ततो दुर्योधनं दृष्ट्वा रथानीके व्यवस्थितम्}
{स रथांस्तावकान्सर्वान्हर्षयञ्शकुनिस्ततः}


\twolineshloka
{दुर्योधनमिदं वाक्यं हृष्टरूपो विशाम्पते}
{कृतकार्यमिवात्मानं मन्यमानोऽब्रवीन्नृपम्}


\twolineshloka
{जहि राजन्रथानीकमश्वाः सर्वे जिता मया}
{नात्यक्त्वा जीवितं सङ्ख्ये शक्यो जेतुं युधिष्ठिरः}


\twolineshloka
{हते तस्मिन्रथानीके पाण्डवेनाभिपालिते}
{गजानेतान्हनिष्यामः पदातींश्चेतरांस्तथा}


\twolineshloka
{श्रुत्वा तु वचनं तस्य तावका जयगृद्विनः}
{जवेनाभ्यपतन्हृष्टाः पाण्डवानामनीकिनीम्}


\twolineshloka
{बद्वनिस्त्रिंशहस्ताश्च प्रगृहीतशरासनाः}
{शरासनानि धून्वानाः सिंहनादान्प्रचक्रिरे}


\twolineshloka
{ततो ज्यातलनिर्धोषः पुनरासीद्विशाम्पते}
{प्रादुरासीच्छराणां च सुमुक्तानां सुदारुणः}


\twolineshloka
{तान्समीपगतान्दृष्ट्वा जनानुद्यतकार्मुकान्}
{उवाच देवकीपुत्रं कुन्तीपुत्रो धनञ्जयः}


\twolineshloka
{चोदयाश्वानसम्भ्रान्तः प्रविशैतद्बलार्णवम्}
{अन्तमद्य गमिष्यामि शत्रूणां निशितैः शरैः}


\twolineshloka
{अष्टादश दिनान्यद्य युद्धस्यास्य जनार्दन}
{वर्तमानस्य महतः समासाद्य परस्परम्}


\twolineshloka
{अनन्तकल्पा ध्वजिनी भूत्वा ह्येषां महात्मनाम्}
{क्षयमद्य गता युद्धे पश्य दैवं यथाविधम्}


\twolineshloka
{समुद्रकल्पं च बलं धार्तराष्ट्रस्य माधव}
{अस्मनासाद्य सञ्जातं गोष्पदोपममच्युत}


\twolineshloka
{हते भीष्मे धीर्ममासीच्छमः स्यादिति माधव}
{न च तत्कृतवान्मूढो धार्तराष्ट्रः सुबालिशः}


\twolineshloka
{उक्तं भीष्मेण यद्वाक्यं हितं तथ्यं च माधव}
{तच्चापि नासौ कृतवान्वीतबुद्धिः सुयोधनः}


\twolineshloka
{तस्मिंस्तु निहते भीष्मे प्रच्युते पृथिवीपतौ}
{न जाने कारणं किन्तु येन युद्धमवर्तत}


\twolineshloka
{मूढांस्तु सर्वथा मन्ये धार्तराष्ट्रान्सुबालिशान्}
{पतिते शन्तनोः पुत्रे येऽकार्युः संयुगं पुनः}


\twolineshloka
{अनन्तरं च निहते द्रोणे ब्रह्मविदां वरे}
{राधेये च विकर्णे च नैव शाम्यति वैशसम्}


\twolineshloka
{अल्पावशिष्टे सैन्येऽस्मिन्सूतपुत्रे च पातिते}
{सपुत्रे वै नरव्याघ्रे नैव शाम्यति वैशसम्}


\twolineshloka
{श्रुतायुषि हते वीरे जलसन्धे च मागधे}
{श्रुतायुधे च नृपतौ नैव शाम्यति वैशसम्}


\twolineshloka
{भूरिश्रवसि शल्ये च साल्ये चैव जनार्दन}
{आवन्त्येषु च वीरेषु नैव शाम्यति वैशसम्}


\twolineshloka
{जयद्रथे च निहते राक्षसे चाप्यलायुधे}
{बाह्लिके सोमदत्ते च नैव शाम्यति वैशसम्}


\twolineshloka
{भगदत्ते हते शूरे काम्भोजे च सुदारुणे}
{दुःशासने च निहते नैव शाम्यति वैशसम्}


\twolineshloka
{दृष्ट्वा विनिहताञ्शूरान्पृथङ्माण़्डलिकान्नृपान्}
{बलिनश्च रणे कृष्ण नैव शाम्यति वैशसम्}


\twolineshloka
{अक्षौहिणीपतीन्दृष्ट्वा भीमसेननिपातितान्}
{मोहाद्वा यदि वा लोभान्नैव शाम्यति वैशसम्}


\twolineshloka
{`हतप्रवीरां विध्वस्तां दृष्ट्वा चेमां चमूं रणे}
{अलम्बले च निहते नैव शाम्यति वैशसम्}


\twolineshloka
{भ्रातॄन्विनिहतान्दृष्ट्वा वयस्यान्मातुलानपि}
{पुत्रान्विनिहतान्दृष्ट्वा नैव शाम्यति वैशसम्'}


\twolineshloka
{को नु राजकुले जातः कौरवेषु विशेषतः}
{निरर्थकं महद्वैरं कुर्यादन्यः सुयोधनात्}


\twolineshloka
{गुणतोऽभ्यधिकाञ्ज्ञात्वा बलतः शौर्यतोपि वा}
{अमूढः को नु युध्येत जानन्प्राज्ञो हिताहितम्}


\twolineshloka
{किन्नु तस्य मनो ह्यासीत्त्वयोक्तस्य हितं वचः}
{प्रशमे पाण्डवैः सार्धं सोन्यस्य शृणुयात्कथम्}


\twolineshloka
{येन शान्तनवो भीष्मो द्रोणो विदुर एव च}
{प्रत्याख्याताः शंमस्यार्थे किन्नु तस्याद्य भेषजम्}


\twolineshloka
{मौर्ख्याद्येन पिता वृद्वः प्रत्याख्यातो जनार्दन}
{तथा माता हितं वाक्यं भाषमाणा हितैषिणी}


\twolineshloka
{प्रत्याख्याता ह्यसत्कृत्य स कस्मै रोचयेद्वचः}
{कुलान्तकरणो व्यक्तं जात एष जनार्दन}


\twolineshloka
{तथास्य दृश्यते चेष्टा नीतिश्चैव विशाम्पते}
{नैष दास्यति नो राज्यमिति मे मतिरच्युत}


\twolineshloka
{उक्तोऽहं बहुशस्तात विदुरेण महात्मना}
{न जीवन्दास्यते भागं धार्तराष्ट्रः सुयोधनः}


\twolineshloka
{यावत्प्राणा धरिष्यन्ति धार्तराष्ट्रस्य दुर्मतेः}
{तावद्युष्मास्वपापेषु प्रचरिष्यति पापकम्}


\twolineshloka
{न च युक्तोऽन्यथा जेतुमृते युद्धेन माधव}
{इत्यब्रवीत्सदा मां हि विदुरः सत्यदर्शनः}


\twolineshloka
{तत्सर्वमद्य जानामि व्यवसायं दुरात्मनः}
{यदुक्तं वचनं तेन विदुरेण महात्मना}


\twolineshloka
{यो हि श्रुत्वा वचः पथ्यं जामदग्न्याद्यथातथम्}
{अवामन्यत दुर्बुद्धिर्ध्रुवं नाशमुखे स्थितः}


\twolineshloka
{उक्तं हि बहुभिः सिद्धैर्जातमात्रे सुयोधने}
{एनं प्राप्य दुरात्मानं क्षयं क्षत्रं गमिष्यति}


\twolineshloka
{तदिदं वचनं तेषां निरुक्तं वै जनार्दन}
{क्षयं याता हि राजानो दुर्योधनकृते भृशम्}


% Check verse!
सोऽद्य सर्वान्रणे योधान्निहनिष्यामि माधव
\twolineshloka
{क्षत्रियेषु हतेष्वाशु शून्ये च शिबिरे कृते}
{वधाय चात्मनोऽस्माभिः संयुगं रोचयिष्यति}


\threelineshloka
{तदन्तं हि भवेद्वैरमनुमानेन माधव}
{एवं पश्यामि वार्ष्णेय चिन्तयन्प्रज्ञया स्वया}
{विदुरस्य च वाक्येन चेष्टया च दुरात्मनः}


\twolineshloka
{तस्माद्याहि चमूं वीर यावद्वन्मि शितैः शरैः}
{दुर्योधनं महाबाहो वाहिनीं चास्य संयुगे}


\threelineshloka
{क्षेममद्य करिष्यामि धर्मराजस्य माधव}
{हत्वैतद्दुर्बलं सैन्यं धार्तराष्ट्रस्य पश्यतः ॥सञ्जय उवाच}
{}


\twolineshloka
{अभीशुहस्तो दाशार्हस्तथोक्तः सव्यसाचिना}
{तद्बलौघममित्राणामभीतः प्राविशद्बलात्}


\twolineshloka
{शरासनवनं घोरं शक्तिकण्टकसङ्कुलम्}
{गदापरिघपाषाणं रथनागमहाद्रुमम्}


\twolineshloka
{हयपत्तिलताकीर्णं गाहमानो महायशाः}
{व्यचरत्तत्र गोविन्दो रथेनातिपताकिना}


\twolineshloka
{ते हयाः पाण्डुरा राजन्वहन्तोऽर्जुनमाहवे}
{दिक्षु सर्वास्वदृश्यन्त दाशार्हेण प्रचोदिताः}


\twolineshloka
{ततः प्रायाद्रथेनाजौ सव्यसाची परन्तपः}
{किरञ्शरशतांस्तीक्ष्णान्वारिधारा घनो यथा}


\twolineshloka
{प्रादुरासीन्महाञ्शब्दः शराणां नतपर्वणाम्}
{इषुभिश्छाद्यमानानां समरे सव्यसाचिना}


\twolineshloka
{असज्जन्तस्तनुत्रेषु शरौघाः प्रापतन्भुवि}
{इन्द्राशनिसमस्पर्शा गाण्डीवप्रेषिताः शराः}


\twolineshloka
{नरान्नागान्समाहत्य हयांश्चापि विशाम्पते}
{अपतन्त रणे बाणाः पतङ्गा इव घोषिणः}


\twolineshloka
{आसीत्सर्वमवच्छन्नं गाण्डीवप्रेषितैः शरैः}
{न प्राज्ञायन्त समरे दिशो वा प्रदिशोपि वा}


\twolineshloka
{सर्वमासीज्जगत्पूर्णं पार्थनामाङ्कितैः शरैः}
{रुक्मपुङ्खैस्तैलधौतैः कर्मारपरिमार्जितैः}


\twolineshloka
{ते दह्यमानाः पार्थेन पावकेनेव कुञ्जराः}
{पार्थं न प्राजहुर्घारा वध्यमानाः शितैः शरैः}


\twolineshloka
{शरचापधरः पार्थः प्रज्वलन्निव भास्करः}
{ददाह समरे योधान्कक्षमग्निरिव ज्वलन्}


\twolineshloka
{यथा वनान्ते वनपैर्विसृष्टःकक्षं दहेत्कृष्णगतिः सुघोषः}
{भूरिद्रुमं शुष्कलतावितानंभृशं समृद्धो ज्वलनः प्रतापी}


\twolineshloka
{एवं स नाराचगणम्प्रतापीशरार्चिरुच्चावचतिग्मतेजाः}
{ददाह सर्वां तव पुत्रसेना--ममृष्यमाणस्तरसा तरस्वी}


\twolineshloka
{तस्येषवः प्राणहराः सुमुक्तानासज्जन्वै वर्मसु रुक्मपुङ्खाः}
{न च द्वितीयं प्रमुमोच बाणंनरे हये वा परमद्विपे वा}


\twolineshloka
{अनेकरूपाकृतिभिर्हि बाणै--र्महारथानीकमनुप्रविश्य}
{स एव एकस्तव पुत्रस्य सेनांजघान दैत्यानिव वज्रपाणिः}


\chapter{अध्यायः २५}
\twolineshloka
{सञ्जय उवाच}
{}


\twolineshloka
{युध्यतां यतमानानां शूराणामनिवर्तनाम्}
{सङ्कल्पमकरोन्मोघं गाण्डीवेन धनञ्जयः}


\twolineshloka
{इन्द्राशनिसमस्पर्शानविषह्यान्महौजसः}
{विसृजन्दृश्यते बाणान्धारा मुञ्चन्निवाम्बुदः}


\twolineshloka
{तत्सैन्यं भरतश्रेष्ठ वध्यमानं किरीटिना}
{सम्प्रदुद्राव सङ्ग्रामात्तव पुत्रस्य पश्यतः}


% Check verse!
पितॄन्भ्रातॄन्परित्यज्य वयस्यानपि चापरे
\twolineshloka
{हतधुर्या रथाः केचिद्धतसूतास्तथा परे}
{भग्नाक्षयुगचक्रेषाः केचिदासन्विशाम्पते}


\twolineshloka
{अन्येषां सायकाः क्षीणास्तथाऽन्ये बाणपीडिताः}
{अक्षता युगपत्केचित्प्राद्रवन्भयपीडिताः}


\twolineshloka
{केचित्पुत्रानुपादाय हतभूयिष्ठबान्धवाः}
{विचुक्रुशुः पितॄंस्त्वन्ये सहायानपरे पुनः}


\twolineshloka
{बान्धवांश्च नरव्याघ्र भ्रातॄन्सम्बन्धिनस्तथा}
{दुद्रुवुः केचिदुत्सृज्य तत्रतत्र विशाम्पते}


\twolineshloka
{बहवोऽत्र भृशं विद्धा मुह्यमाना महारथाः}
{निःश्वसन्ति स्म दृश्यन्ते पार्थबाणहता नराः}


\twolineshloka
{तानन्ये रथमारोप्य ह्याश्वास्य च मुहूर्तकम्}
{विश्रान्ताश्च वितृष्णाश्च पुनर्युद्धाय जग्मिरे}


\twolineshloka
{तानपास्य गताः केचित्पुनरेव युयुत्सवः}
{कुर्वन्तस्तव पुत्रस्य शासनं युद्धदुर्मदाः}


\twolineshloka
{पानीयमपरे पीत्वा पर्याश्वास्य च वाहनम्}
{वर्माणि च समारोप्य केचिद्भरतसत्तम}


\twolineshloka
{समाश्वास्यापरे भ्रातॄन्निक्षिप्य शिबिरेऽपि च}
{पुत्रानन्ये पितॄनन्ये पुनर्युद्धमरोचयन्}


\twolineshloka
{सज्जयित्वा रथान्केचिद्यथामुख्यं विशाम्पते}
{आवृत्य पाण्डवानीकं पुनर्युद्धमरोचयन्}


\twolineshloka
{ते शूराः किङ्किणीजालैः समाच्छन्ना बभासिरे}
{त्रैलोक्यविजये युक्ता यथा दैतेयदानवाः}


\twolineshloka
{आगम्य सहसा केचिद्रथैः स्वर्णविभूषितैः}
{पाण्डवानामनीकेषु धृष्टद्युम्नमयोधयन्}


\twolineshloka
{धृष्टद्युम्नोऽपि पाञ्चाल्यः शिखण्डी च महारथः}
{नाकुलिस्तु शतानीको रथानीकमयोधयन्}


\twolineshloka
{पाञ्चाल्यस्तु ततः क्रुद्धः सैन्येन महता वृतः}
{अभ्यद्रवत्सुसङ्क्रुद्धस्तावकान्हन्तुमुद्यतः}


\twolineshloka
{ततस्त्वापततस्तस्य तव पुत्रो जनाधिप}
{बाणसङ्घाननेकान्वै प्रेषयामास भारत}


\twolineshloka
{धृष्टद्युम्नस्ततो राजंस्तव पुत्रेण धन्विना}
{नाराचैरर्धनाराचैर्बहुभिः क्षिप्रकारिभिः}


\twolineshloka
{वत्सदन्तैश्च बाणैश्च कर्मारपरिमार्जितैः}
{अश्वांश्च चतुरो हत्वा बाह्वोरुरसि चार्पितः}


\threelineshloka
{सोऽतिविद्धो महेष्वासस्तोत्रार्दित इव द्विपः}
{तस्याश्वांश्चतुरो बाणैः प्रेषयामास मृत्यवे}
{सारथेश्चास्य भल्लेन शिरः कायादपाहरत्}


\twolineshloka
{ततो दुर्योधनो राजा पृष्ठमारुह्य वाजिनः}
{अपाक्रामद्वतरथो नातिदूरमरिन्दमः}


\twolineshloka
{दृष्टा तु हतविक्रान्तं स्वमनीकं महाबलः}
{तव पुत्रो महाराज प्रययौ यत्र सौबलः}


\twolineshloka
{ततो रथेषु भग्नेषु त्रिसाहस्रा महाद्विपाः}
{पाण्डवान्रथिनः सर्वान्समन्तात्पर्यवारयन्}


\twolineshloka
{ते वृताः समरे पञ्च गजानीकेन भारत}
{अशोभन्त महाराज ग्रहास्तारागणैरिव}


\twolineshloka
{ततोऽर्जुनो महाराज लब्धलक्षौ महाभुजः}
{विनिर्ययौ रथेनैव श्वेताश्वः कृष्णसारथिः}


\twolineshloka
{तैः समन्तात्परिवृतः कुञ्जरैः पर्वतोपमैः}
{नाराचैर्विमलैस्तीक्ष्णैर्गजानीकमयोधयत्}


\twolineshloka
{तत्रैकबाणनिहतानपश्याम महागजान्}
{पतितान्पात्यमानांश्च निर्भिन्नान्सव्यसाचिना}


\threelineshloka
{भीमसेनस्तु तान्दृष्ट्वा नागान्मत्तगजोपमः}
{करेणादाय महतीं गदामभ्यपतद्बली}
{अथाप्लुत्य रथात्तूर्णं दण्डपाणिरिवान्तकः}


\twolineshloka
{तमुद्यतगदं दृष्ट्वा पाण्डवानां महारथम्}
{वित्रेसुस्तावकाः सैन्याः शकृन्मूत्रं च सुस्रुवुः}


% Check verse!
आविग्नं च बलं सर्वं गदाहस्ते वृकोदरे
\twolineshloka
{गदया भीमसेनेन भिन्नकुम्भान्निपातितान्}
{धावमानानपश्याम कुञ्जरान्पर्वतोपमान्}


\twolineshloka
{प्राद्रवन्कुञ्जरास्ते तु भीमसेनगदाहताः}
{पेतुरार्तस्वरं कृत्वा छिन्नपक्षा इवाद्रयः}


\twolineshloka
{प्रभिन्नकुम्भांस्तु बहून्द्रवमाणानितस्ततः}
{पतमानांश्च सम्प्रेक्ष्य वित्रेसुस्तव सैनिकाः}


\twolineshloka
{युधिष्ठिरोऽपि सङ्क्रुद्धो माद्रीपुत्रौ च पाण्डवौ}
{गार्ध्रपत्रैः शितैर्बाणैर्जघ्नुर्वै गजयोधिनः}


\twolineshloka
{धृष्टद्युम्नस्तु समरे पारजित्य नराधिपम्}
{अपक्रान्ते तव सुते हयपृष्ठे समाश्रिते}


\twolineshloka
{दृष्ट्वा च पाण्डवान्सर्वान्कुञ्चरैः परिवारितान्}
{धृष्टद्युम्नो महाराज सहसा समुपाद्रवत्}


\twolineshloka
{पुत्रः पाञ्चालराजस्य जिघांसुः कुञ्जरान्ययौ}
{अदृष्ट्वा तु रथानीके दुर्योधनमरिन्दमम्}


\twolineshloka
{अश्वत्थामा कृपश्चैव कृतवर्मा च सात्वतः}
{अपृच्छन्क्षत्रियांस्तत्र क्व नु दुर्योधनो गतः}


\threelineshloka
{तेऽपश्यमाना राजानं वर्तमाने जनक्षये}
{मन्वाना निहतं तत्र तव पुत्रं महारथाः}
{विवर्णवदना भूत्वा पर्यपृच्छन्त ते सुतम्}


\twolineshloka
{आहुः केचिद्वते सूते प्रयातो यत्र सौबलः}
{हित्वा पाञ्चालराजस्य तदनीकं दुरुत्सहम्}


\threelineshloka
{अपरे त्वब्रुवंस्तत्र क्षत्रिया भृशविक्षताः}
{दुर्योधनेन किं कार्यं द्रक्ष्यध्वं यदि जीवति}
{युध्यध्वं सहिताः सर्वे किं वो राजा करिष्यति}


\twolineshloka
{ते क्षत्रियाः क्षतैर्गात्रैर्हतभूयिष्ठवाहनाः}
{शरैः सम्पीड्यमानास्तु नातिव्यक्तमथाब्रुवन्}


\twolineshloka
{इदं सर्वं बलं हन्मो येन स्म परिवारिताः}
{एते सर्वे गजान्हत्वा उपयान्ति स्म पाण्डवाः}


\twolineshloka
{श्रुत्वा तु वचनं तेषामश्वत्थामा महाबलः}
{भित्त्वा पाञ्चालराजस्य तदनीकं दुरुत्सहम्}


\twolineshloka
{कृपश्च कृतवर्मा च प्रययौ यत्र सौबलः}
{रथानीकं परित्यज्य शूराः सुदृढधन्विनः}


\twolineshloka
{ततस्तेषु प्रयातेषु धृष्टद्युम्नपुरस्कृताः}
{आययुः पाण्डवा राजन्विनिघ्नन्तः स्म तावकान्}


\twolineshloka
{दृष्ट्वा तु तानापततः सम्प्रहृष्टान्महारथान्}
{पराक्रान्तास्ततो वीरा निराशा जीविते तदा}


\twolineshloka
{विवर्णमुखभूयिष्ठमभवत्तावकं बलम्}
{परिक्षीणायुधान्दृष्ट्वा तानहं परिवारितान्}


\twolineshloka
{राजन्बलेन त्र्यङ्गेन त्यक्त्वा जीवितमात्मनः}
{आत्मना पञ्चमोऽयुध्यं पाञ्चालस्य बलेन ह}


\twolineshloka
{तस्मिन्देशे व्यवस्थाय यत्र शारद्वतः स्थितः}
{सम्प्रद्रुता वयं पञ्च किरीटिशरपीडिताः}


\twolineshloka
{धृष्टद्युम्नं महारौद्रं तत्र नाभूद्रणो महान्}
{जितास्तेन वयं सर्वे व्यपयाम रणात्ततः}


\twolineshloka
{अथापश्यं सात्यकिं तमुपायातं महारथम्}
{रथैश्चतुः शतैर्वीरो मामभ्यद्रवदाहवे}


\twolineshloka
{धृष्टद्युम्नादहं मुक्तः कञ्छिछ्रान्तवाहनात्}
{पतितो माधवानीकं दुष्कृती नरकं यथा}


% Check verse!
तत्र युद्धमभूद्धोरं मुहूर्तमतिदारुणम्
\twolineshloka
{सात्यकिस्तु महाबाहुर्मम हत्वा परिच्छदम्}
{जीवग्राहमगृह्णान्मां मूर्च्छितं पतितं भुवि}


\twolineshloka
{ततो मुहूर्तादिव तद्गजानीकमविध्यत}
{गदया भीमसेनेन नाराचैरर्जुनेन च}


\twolineshloka
{अभिपिष्टैर्महानागैः समन्तात्पर्वतोपमैः}
{नातिप्रसिद्धैव गतिः पाण्डवानामजायत}


\twolineshloka
{रथमार्गं ततश्चक्रे भीमसेनो महाबलः}
{पाण्डवानां महाराज व्यपाकर्षन्महागजान्}


\threelineshloka
{अश्वत्थामा कृपश्चैव कृतवर्मा च सात्वतः}
{अपश्यन्तो रथानीके दुर्योधनसमरिन्दमम्}
{राजानं मृगयामासुस्तव पुत्रं महारथम्}


\twolineshloka
{परित्यज्य च पाञ्चाल्यं प्रयाता यत्र सौबलः}
{राज्ञोऽदर्शनसंविग्ना वर्तमानो जनक्षये}


\chapter{अध्यायः २६}
\twolineshloka
{सञ्जय उवाच}
{}


\twolineshloka
{गजानीके हते तस्मिन्पाण्डुपुत्रेण भारत}
{वध्यमाने बले चैव भीमसेनेन संयुगे}


\twolineshloka
{चरन्तं च प्रपश्यामो भीमसेनमरिन्दमम्}
{दण्डहस्तं यथा क्रुद्वमन्तकं प्राणहारिणम्}


\threelineshloka
{समेत्य समरे राजन्हतशेषाः सुतास्तव}
{अदृश्यमाने कौरव्ये पुत्रे दुर्योधने तव}
{सोदर्याः सहिता भूत्वा भीमसेनमुपाद्रवन्}


\twolineshloka
{श्रुतर्वा सञ्जयश्चैव जयत्सेनः श्रुतान्तकः}
{दुर्विमोचनकश्चैव तथा दुर्विषहो बली}


\threelineshloka
{दुर्मर्षणः सुजातश्च जैत्रो भूरिबलो रविः}
{इत्येते सहिता भूत्वा तव पुत्राः समन्ततः}
{भीमसेनमभिद्रुत्य रुरुधुः सर्वतो दिशम्}


\twolineshloka
{ततो भीमो महाराज स्वरथं पुनरास्थितः}
{मुमोच निशितान्बाणान्पुत्राणां तव मर्मसु}


\twolineshloka
{ते कीर्यमाणा भीमेन पुत्रास्तव महारणे}
{भीमसेनमुपासेदुः प्रवाता इव कुञ्जरम्}


\twolineshloka
{ततः क्रुद्धो रणे भीमः शिरो दुर्मर्षणस्य ह}
{क्षुरप्रेण प्रमथ्याशु पातयामास भूतले}


\twolineshloka
{ततोऽपरेण भल्लेन सर्वावरणभेदिना}
{श्रुतान्तमवधीद्बीमस्तव पुत्रं महारथः}


\threelineshloka
{जयत्सेनं ततो विद्वा नाराचेन हसन्निव}
{पातयामास कौरव्यं रथोपस्थादरिन्दमः}
{स पपात रथाद्राजन्भूमौ तूर्णं ममार च}


\twolineshloka
{श्रुतर्वा तु ततो भीमं क्रुद्धो विव्याध मारिष}
{शतेन गृघ्रवाजानां शराणां नतपर्वणाम्}


\twolineshloka
{ततः क्रुद्धो रणे भीमो जैत्रं भूरिबलं रविम्}
{त्रीनेतांस्त्रिभिरानर्च्छद्विषाग्निप्रतिमैः शरैः}


\twolineshloka
{ते हता न्यपतन्भूमौ स्यन्दनेभ्यो महारथाः}
{वसन्ते पुष्पशबला निकृत्ता इव किंशुकाः}


\twolineshloka
{ततोऽपरेण भल्लेन तीक्ष्णेन च परन्तपः}
{दुर्विमोचनमाहत्य प्रेषयामास मृत्यवे}


\twolineshloka
{स हतः प्रापतद्भूमौ स्वरथाद्रथिनां वरः}
{गिरेस्तु कूटजो भग्नो मारुतेनेव पादपः}


\threelineshloka
{दुष्प्रधर्षं ततश्चैव सुजातं च सुतं तव}
{एकैकं न्यहनत्सङ्ख्ये द्वाभ्यांद्वाभ्यां चमूमुखे}
{तौ शिलीमुखविद्धाङ्गौ पेततू रथसत्तमौ}


\twolineshloka
{ततः पतन्तं समरे अभिवीक्ष्य सुतं तव}
{भल्लेन पातयामास भीमो दुर्विपहं रणे}


% Check verse!
स पपात हतो बाहात्पश्यतां सर्वधन्विनाम्
\twolineshloka
{दृष्ट्वा तु निहतान्भ्रातॄन्वहूनेकेन संयुगे}
{अमर्षधशमापसः श्रुतर्वा भीममभ्ययात्}


\twolineshloka
{विक्षिपन्सुमहच्चापं कार्तखरविभूषितम्}
{विमृजन्सायकांश्चैव विषाग्निप्रतिमान्वहून्}


\twolineshloka
{स तु राजन्धनुश्छित्त्वा पाण्डवस्य महामृधे}
{अथैनं छिन्नधन्वानं विंशत्या समवाकिरत्}


\twolineshloka
{ततोऽन्यद्धनुरादाय भीमसेनो महाबलः}
{अवाकिरत्तव सुतं तिष्ठतिष्ठेति चाब्रवीत्}


\twolineshloka
{महदासीत्तयोर्युद्धं चित्ररूपं भयानकम्}
{बादृशं समरे पूर्वं तम्भवासवयोर्युधि}


\twolineshloka
{तयोस्तत्र शितैर्गुक्तैर्यमदण्डनिभैः शरैः}
{समाच्छकाधारा सर्वा त्वं दिशो विदिशस्तथा}


\twolineshloka
{अतः श्रुतर्वा सङ्क्रुद्धो धनुरादाय सायकैः}
{भीमसेनं रणे राजन्बाह्वोरुरसि चार्पयत्}


\twolineshloka
{सोऽतिविद्धो महाराज तव पुत्रेण धन्विना}
{भीमः सञ्चुक्षुभे क्रुद्धः पर्वणीव महोदधिः}


\twolineshloka
{ततो भीमो रुषाविष्टः पुत्रस्य तव मारिष}
{सारथिं चतुरश्चाश्वाञ्शरैर्निन्ये यमक्षयम्}


\twolineshloka
{विरथं तं समालक्ष्य विशिखैर्लोमवाहिभिः}
{अवाकिरदमेयात्मा दर्शयन्पाणिलाघवम्}


\threelineshloka
{श्रुतर्वा विरथो राजन्नाददे खङ्गचर्मणी}
{अथास्याददतः खह्गं शतचन्द्रं च भानुमत्}
{क्षुरप्रेणं शिरः कायात्पातयामास पाण्डवः}


\twolineshloka
{छिन्नोत्तमाङ्गस्य ततः क्षुरप्रेण महात्मना}
{पपात कायः स्वरथाद्वसुधामनुनादयन्}


\twolineshloka
{तस्मिन्निपतिते वीरे तावका भयमोहिताः}
{अभ्यद्रवन्त सङ्ग्रामे भीमसेनं युयुत्सवः}


\twolineshloka
{तानापतत एवाशु हतशेषाद्बलार्णवात्}
{दंशितान्प्रतिजग्राह भीमसेनः प्रतापवान्}


% Check verse!
ते तु तं वै समासाद्य परिवव्रुः समन्ततः
\twolineshloka
{ततस्तु संवृतो भीमस्तावकान्निशितैः शरैः}
{पीडयामास तान्सर्वान्सहस्राक्ष इवासुरान्}


\twolineshloka
{ततः पञ्चशतान्हत्वा सवरूथान्महारथान्}
{जघान कुञ्जरानीकं पुनः सप्तशतं युधि}


\twolineshloka
{हत्वा शतसहस्राणि पत्तीनां परमेषुभिः}
{वाजिनां च शतान्यष्टौ पाण्डवः स्म विराजते}


\twolineshloka
{भीमसेनस्तु कौन्तेयो हत्वा युद्धे सुतांस्तव}
{मेने कृतार्थमात्मानं सफलं जन्म च प्रभो}


\twolineshloka
{तं तथा युध्यमानं च विनिघ्नन्तं च तावकान्}
{ईक्षितुं नोत्सहन्ते स्म तव सैन्या नराधिप}


\twolineshloka
{विद्राव्य च कुरून्सर्वांस्तांश्च हत्वा पदानुगान्}
{दोर्भ्यां शब्दं ततश्चक्रे त्रासयानो महाद्विपान्}


\twolineshloka
{हतभूयिष्ठयोधा तु तव सेना विशामम्पते}
{किञ्चिच्छेषा महाराज कृपणं समपद्यत}


\chapter{अध्यायः २७}
\twolineshloka
{सञ्जय उवाच}
{}


\twolineshloka
{दुर्योधनो महाराज सुदर्शश्चापि ते सुतः}
{हतशेषौ तदा सङ्ख्ये वाजिमध्ये व्यवस्थितौ}


\twolineshloka
{ततो दुर्योधनं दृष्ट्वा वाजिमध्ये व्यवस्थितम्}
{उवाच देवकीपुत्रः कुन्तीपुत्रं धनञ्जयम्}


\twolineshloka
{शत्रवो हतभूयिष्ठा ज्ञातयः परिपालिताः}
{गृहीत्वा सञ्जयं चासौ निवृत्तः शिनिपुङ्गवः}


\twolineshloka
{परिश्रान्तश्च नकुलः सहदेवश्च भारत}
{योधयित्वा रणे पापान्धार्तराष्ट्रान्सहानुगान्}


\twolineshloka
{दुर्योधनमतिक्रम्य त्रय एते व्यवस्थिन्ताः}
{कृपश्च कृतवर्मा च द्रौणिश्चैव महारथः}


\twolineshloka
{असौ तिष्ठति पाञ्चाल्यः श्रिया परमया युतः}
{दुर्योधनबलं हत्वा सह सर्वैः प्रभद्रकैः}


\twolineshloka
{असौ दुर्योधनः पार्थ वाजिमध्ये व्यवस्थितः}
{छत्रेण ध्रियामणेन प्रेक्षमाणो मुहुर्मुहुः}


\twolineshloka
{प्रतिव्यूह्य बलं सर्वं रणमध्ये व्यवस्थितः}
{एनं हत्वा शितैर्बाणैः कृतकृत्यो भविष्यसि}


\twolineshloka
{गजानीकं हतं दृष्ट्वा त्वां च प्राप्तमरिन्दम}
{यावन्न विद्रवन्त्येते तावज्जहि सुयोधन्सम्}


\twolineshloka
{यातु कश्चित्तुं पाञ्चाल्यं क्षिप्रमागम्यतामिति}
{परिश्रान्तबलस्तात नैष मुच्येत किल्बिषी}


\twolineshloka
{हत्वा तव बलं सर्वं सङ्ग्रमे धृतराष्ट्रजः}
{जितान्पाण्डुसुतान्प्रत्वा रूपं धारयते महत्}


\twolineshloka
{निहतं स्वबलं दृष्ट्वा पीडितं चापि पाण्डवैः}
{ध्रुवमेष्यति सङ्ग्रमे वधायैवात्मनो नृपः}


\threelineshloka
{एवमुक्तः फल्गुनस्तु कृष्णं वचनमब्रवीत्}
{धृतराष्ट्रसुताः सर्वे हता भीमेन माधव}
{यावेतावास्थितौ कृष्ण तावद्य नभविष्यतः}


\twolineshloka
{हतो भीष्मो हतो द्रोणः कर्णो वैकर्तनो हतः}
{मद्रराजो हतः शल्यो हतः कृष्ण जयद्रथः}


\threelineshloka
{हयाः पञ्चशताः शिष्टाः शकुनेः सौबलस्य च}
{रथानां तु शते शिष्टे द्वे एव तु जनार्दन}
{दन्तिनां च शतं साग्रं त्रिसाहस्राः पदातयः}


\twolineshloka
{अश्वत्थामा कृपश्चैव त्रिगर्ताधिपतिस्तथा}
{उलूकः शकुनिश्चैव कृतवर्मा च सात्वतः}


\twolineshloka
{एतद्बलमभूच्छेषं धार्तराष्ट्रस्य माधव}
{मोक्षो न नूनं कालात्तु विद्यते भुवि कस्यचित्}


\threelineshloka
{तथा विनिहते सैन्ये पश्य दुर्योधनं स्थितम्}
{अद्याह्ना हि महाराजो हतामित्रो भविष्यति}
{न हि मे मोक्ष्यते कश्चित्परेषामिह चिन्तये}


\twolineshloka
{ये त्वद्य समरं कृष्ण न हास्यन्ति मदोत्कटाः}
{तान्वै सर्वान्हनिष्यामि यद्यपि स्युरमानुषाः}


\twolineshloka
{अद्य युद्धे समुत्पन्नं दीर्घं राज्ञः प्रजागरम्}
{अपनेष्यामि गान्धारिं घातयित्वा शितैः शरैः}


\twolineshloka
{निकृत्या वै दुराचारो यानि रत्नानि सौबलः}
{सभायामहरद्द्यूते पुनस्तान्याहराम्यहम्}


\twolineshloka
{अद्य वेत्स्यन्ति मच्छक्तिं सर्वा नागपुरे स्त्रियः}
{श्रुत्वा पतींश्च पुत्रांश्च पाण्डवैर्निहतान्युधि}


\twolineshloka
{समाप्तमद्य वै कर्म सर्वं कृष्ण भविष्यति}
{अद्य दुर्योधनो दीप्तां श्रियं प्राणांश्च मोक्ष्यति}


\twolineshloka
{नापयाति भयात्कृष्ण सङ्ग्रामाद्यदि चेन्मम}
{निहतं विद्धि वार्ष्णेय धार्तराष्ट्रं सुबालिशम्}


\threelineshloka
{मम ह्येतदपर्याप्तं वाजिबृन्दमरिन्दम}
{सोढुं ज्यातलनिर्घोषं याहि यावन्निहन्म्यहम् ॥सञ्जय उवाच}
{}


\twolineshloka
{एवमुक्तस्तु दाशार्हः पाण्डवेन यशस्विना}
{अचोदयद्धायन्राजन्दुर्योधनबलं प्रति}


\threelineshloka
{तदनीकमभिप्रेक्ष्य त्रयः सज्जा महारथाः}
{भीमसेनोऽकर्जुनश्चैव सहदेवश्च मारिष}
{प्रययुः सिंहनादेन दुर्योधनजिघांसया}


\twolineshloka
{तान्प्रेक्ष्य सहितान्सर्वाञ्जवेनोद्यतकार्मुकान्}
{सौबलोऽभ्यद्रवद्युद्धे पाण्डवानाततायिनः}


\twolineshloka
{सुदर्शनस्त्व मतो भीमसेनं समभ्ययात्}
{सुशर्मा शकुनिश्चैव युयुधाते किरीटिना}


% Check verse!
सहदेवं तव सुतो हयपृष्ठगतोऽभ्ययात्
\twolineshloka
{ततो हि यत्नतः क्षिप्रं तवं पुत्रो जनाधिप}
{प्रासेन सहदेवस्य शिरसि प्राहरद्भृशम्}


\twolineshloka
{सोपाविशद्रथोपस्थे तव सुत्रेण ताडितः}
{रुधिराप्लुतसर्वाङ्ग आशीविष इव श्वसन्}


\twolineshloka
{प्रतिलभ्य ततः सञ्ज्ञां सहदेवो विशाम्पते}
{दुर्योधनं शरैस्तीक्ष्णैः सङ्क्रुद्धः समवाकिरत्}


\twolineshloka
{पार्थोऽपि युधि विक्रम्य कुन्तीपुत्रो धनञ्जयः}
{शूराणामश्वपृष्ठेभ्यः शिरांसि निचकर्त ह}


\twolineshloka
{तदनीकं तदा पार्थो व्यधमद्बहुभिः शरैः}
{पातयित्वा हयान्सर्वांस्त्रिगर्तानां रथान्ययौ}


\twolineshloka
{ततस्ते सहिता भूत्वा त्रिगर्तानां महारथाः}
{अर्जुनं वासुदेवं च शरवर्षैरवाकिरन्}


\twolineshloka
{सत्यकर्माणमाक्षिप्य क्षुरप्रेण महायशाः}
{ततोऽस्य स्यन्दनस्येषां चिच्छिदे पाण्डुनन्दनः}


\twolineshloka
{शिलाशितेन च विभो क्षुरप्रेण महायशाः}
{शिरश्चिच्छेद सहसा तप्तकुण्डलभूषणम्}


\twolineshloka
{सत्येषुमथ चादत्त योधानां मिषतां ततः}
{यथा सिंहो वने राजन्मृगं परि बुभुक्षितः}


\twolineshloka
{तं निहत्य ततः पार्थः सुशर्माणं त्रिभिः शरैः}
{विद्ध्वा तानहनत्सर्वान्रथान्रुक्मविभूओषितान्}


\twolineshloka
{ततः प्रायात्त्वरन्पार्थो दीर्घकालं सुसंवृतम्}
{मुञ्जन्क्रोधविषं तीक्ष्णं प्रस्थलाधिपतिं प्रति}


\twolineshloka
{तमर्जुनः पृषत्कानां शतेन भरतर्षभ}
{पूरयित्वा ततो वाहान्प्राहरत्तस्य धन्विनः}


\twolineshloka
{ततः शरं समादाय यमदण्डोपमं तदा}
{सुशर्माणं समुद्दिश्य चिक्षेपाशु हसन्निव}


\twolineshloka
{स शरः प्रेषितस्तेन क्रोधदीप्तेन धन्विना}
{सुशर्माणं समासाद्य बिभेद हृदयं रणे}


\twolineshloka
{स गतासुर्महाराज पपात धरणीतले}
{नन्दयन्पाण्डवान्सर्वान्व्यथयंश्चापि तावकान्}


\twolineshloka
{सुशर्माणं रणे हत्वा पुत्रानस्य महारथान्}
{सप्त चाष्टौ च त्रिंशच्च सायकैरनयत्क्षयम्}


\twolineshloka
{ततोऽस्य निशितैर्बाणैः सर्वान्हत्वा पदानुगान्}
{अभ्यगाद्भारतीं सेनां हतशेषां महारथः}


\twolineshloka
{भीमस्तु समरे क्रुद्धः पुत्रं तव जनाधिप}
{सुदर्शनमदृश्यन्तं शरैश्चक्रे हसन्निव}


\twolineshloka
{ततोऽस्य प्रहसन्क्रुद्धः शिरः कायादपाहरत्}
{क्षुरप्रेण सुतीक्ष्णेन स हतः प्रापतद्भुवि}


\twolineshloka
{तस्मिंस्तु निहते वीरे ततस्तस्य पदानुगाः}
{परिवव्रू रमे भीमं किरन्तो विविधाञ्शरान्}


\twolineshloka
{ततस्तु निशितैर्बाणैस्तवानीकं वृकोदरः}
{इन्द्राशनिसमस्पर्शैः समन्तात्पर्यवाकिरत्}


% Check verse!
ततः क्षणेन तद्भीमो न्यहनद्भरतर्षभ
\twolineshloka
{तेषु तूत्साद्यमानेषु सेनाध्यक्षा महारथाः}
{भीमसेनं समासाद्य ततोऽयुध्यन्त भारत}


\threelineshloka
{स तान्सर्वाञ्शरैर्घोरैरवाकिरत पाण्डवः}
{तथैव तावका राजन्पाण्डवेयान्महारथान्}
{शरवर्षेण महता समन्तात्पर्यवारयन्}


\twolineshloka
{व्याकुलं तदभूत्सर्वं पाण्डवानां परैः सह}
{तावकानां च समरे पाण्डवेंयैर्युयुत्सताम्}


\twolineshloka
{तत्र योधास्तदा पेतुः परस्परसमाहताः}
{उभयोः सेनयो राजन्संशोचन्तः स्म बान्धवान्}


\chapter{अध्यायः २८}
\twolineshloka
{सञ्जय उवाच}
{}


% Check verse!
तस्मिन्प्रवृत्ते सङ्ग्रामे गजवाजिनरश्रोशकुनिः सौबलो राजन्सहदेवं xxxxxxयात्
\threelineshloka
{ततोऽस्यापततस्तूर्णं सहदेवः पतापवान्}
{शरौघान्प्रेषयामास पतङ्गानिव शीघ्रगान्}
{उलूकं च रणे राजन्विव्याध दशभिः शरैः}


\twolineshloka
{शकृनिश्च महाराज भीमं विद्ध्वा त्रिभिः शरैः}
{नवत्या निशितैर्बाणैः सहदेवमवाकिरत्}


\threelineshloka
{ते शूराः समरे राजन्समासाद्य परस्परम्}
{विव्यधुर्निशितैर्बाणैः कङ्कबर्हिणवाजितैः}
{स्वर्णपुङ्खैः शिलधौतैराकर्णप्रहितैः शरैः}


\twolineshloka
{तेषां चापगुणोत्सृष्टा शरवृष्टिर्विशाम्पते}
{आच्छादयद्दिशः सर्वा धाराभिरिव तोयदः}


\twolineshloka
{ततः क्रुद्धो रणे भीमः सहदेवश्च भारत}
{चेरतुः कदनं सङ्ख्ये कुर्वन्तौ सुमहाबलौ}


\twolineshloka
{ताभ्यां शरशतैश्छनं तद्बलं तव भारत}
{सान्धकारमिवाकाशमभवत्तत्रतत्र ह}


\twolineshloka
{अश्वैर्विपरिधावद्भिः शरच्छन्नैर्विशाम्पते}
{तत्रतत्र कृतो मार्गो विकर्षद्भिर्हतान्बहून्}


\twolineshloka
{निहतानां हयानां च सहैव हयसादिभिः}
{वर्मभिर्विनिकृत्तैश्च प्रासैश्छिन्नैश्च मारिष}


\twolineshloka
{ऋष्टिभिः शक्तिभिश्चैव सासिप्रासपरश्वथैः}
{सञ्छन्ना पृथिवी जज्ञे कुसुमैः शबला इव}


\twolineshloka
{योधास्तत्र महाराज समासाद्य परस्परम्}
{व्यचरन्त रणे क्रुद्धा विनिघ्नन्तः परस्परम्}


\twolineshloka
{उद्वृत्तनयनै रोषात्सन्दष्टौष्ठपुटैर्मुखैः}
{सकुण्डलैर्मही च्छन्ना पद्मकिञ्चल्कसन्निभैः}


\twolineshloka
{भुजैश्छिन्नैर्महाराज नागराजकरोपमैः}
{साङ्गदैः सतनुत्रैश्च सासिप्रासपरश्वथैः}


\twolineshloka
{कबन्धैरुत्थितैश्छिन्नैर्नृत्यद्भिश्चापरैर्युधि}
{क्रव्यादगणसञ्छन्ना घोराऽभूत्पृथिवी विभो}


\twolineshloka
{अल्पावशिष्टे सैन्ये तु कौरवेयान्महाहवे}
{प्रहृष्टाः पाण्डवा भूत्वा निन्यिरे यमसादनम्}


\twolineshloka
{एतस्मिन्नन्तरे शूरः सौबलेयः प्रतापवान्}
{प्रासेन सहदेवस्य शिरसि प्राहरद्भृशम्}


% Check verse!
स विह्वलो महाराज रथोपस्थ उपाविशत्
\twolineshloka
{सहदेवं तथा दृष्ट्वा भीमसेनः प्रतापवान्}
{सर्वसैन्यानि सङ्क्रुद्धो वारयामास भारत}


\twolineshloka
{निर्बिभेद च नाराचैः शतशोऽथ सहस्रशः}
{स निर्भिद्याकरोच्चैव सिंहनादमरिन्दमः}


\twolineshloka
{तेन शभ्देन वित्रस्ताः सर्वे सहयवारणाः}
{प्राद्रवन्सहसा भीताः शकुनेश्च पदानुगाः}


\twolineshloka
{प्रभग्नानथ तान्दृष्ट्वा राजा दुर्योधनोऽब्रवीत्}
{निवर्तध्वमधर्मज्ञा युध्यध्वं किं सृतेन वः}


\twolineshloka
{इह कीर्ति समाधाय प्रेत्य लोकान्समश्नुते}
{प्राणाञ्जहाति यो धीरो युद्धे पृष्ठमदर्शयन्}


\twolineshloka
{एवमुक्तास्तु ते राज्ञा सौबलस्य पदानुगाः}
{पाण्डवानभ्यवर्तन्त मृत्युं कृत्वा निवर्तनम्}


\twolineshloka
{द्रवद्भिस्तत्र राजेन्द्र कृतः शब्दोऽतिदारुणः}
{क्षुब्धसागरसङ्काशः क्षुभितैः सर्वतो दिशम्}


\twolineshloka
{तांस्ततः पुरतो दृष्ट्वा सौबलस्य पदानुगान्}
{प्रत्युद्ययुर्महाराज पाण्डवा विजयोद्यताः}


\threelineshloka
{प्रत्याश्वस्य च दुर्धर्षः सहदेवो विशाम्पते}
{शकुनिं दशभिर्विद्ध्वा हयांश्चास्य त्रिभिः शरैः}
{धनुश्चिच्छेद च शरैः सौबलस्य हसन्निव}


\twolineshloka
{अथान्यद्धनुरादाय शकुनिर्युद्धदुर्मदः}
{विव्याध नकुलं षष्ट्या भीमसेनं च सप्ताभिः}


\twolineshloka
{उलूकोऽपि महाराज भीमं विव्याध सप्तभिः}
{सहदेवं च सप्तत्या परीप्सन्पितरं रणे}


\twolineshloka
{तं भीमसेनः समरे विव्याध नवभिः शरैः}
{शकुनिं च चतुःषष्ट्या पार्श्वस्थांश्च त्रिभिस्त्रिभिः}


\threelineshloka
{ते हन्यमाना भीमेन नाराचैस्तैलपायितैः}
{सहदेवं रणे क्रुद्धाश्छादयञ्शरवृष्टिभिः}
{पर्वतं वारिधाराभिः सविद्युत इवाम्बुदाः}


\twolineshloka
{ततोऽस्यापततः शूरः सहदेवः प्रतापवान्}
{उलूकस्य महाराज भल्लेनापाहरच्छिरः}


\twolineshloka
{स जगाम रथाद्भूमिं सहदेवेन पातितः}
{रुधिराप्लुतसर्वाङ्गो नन्दयन्पाण्डवान्युधि}


\twolineshloka
{पुत्रं तु निहतं दृष्ट्वा शकुनिस्तत्र भारत}
{साश्रुकण्ठो विनिःश्वस्य क्षत्तुर्वाक्यमनुस्मरन्}


\twolineshloka
{चिन्तयित्वा मुहूर्तं स बाष्पपूर्णेक्षणः श्वसन्}
{सहदेवं समासाद्य त्रिभिर्विव्याध सायकैः}


\twolineshloka
{तानपास्य शरान्मुक्ताञ्शरसङ्घैः प्रताम्पवान्}
{सहदेवो महाराज धनुश्चिच्छेद संयुगे}


\twolineshloka
{छिन्ने धनुषि राजेन्द्र शकुनिः सौबलस्तदा}
{प्रगृह्य विपुलं खङ्गं सहदेवाय प्राहिणोत्}


\twolineshloka
{तमापतन्तं सहसा घोररूपं विशाम्पते}
{द्विधा चिच्छेद समरे सौबलस्य हसन्निव}


\twolineshloka
{असिं दृष्ट्वा द्विधा च्छिन्नं प्रगृह्य महतीं गदाम्}
{प्राहिणोत्सहदेवाय सा मोघा न्यपतद्भुवि}


\twolineshloka
{ततः शक्तिं महाघोरां कालरात्रीमिवोद्यताम्}
{प्रेषयामास सक्रुद्धः पाण्डवं प्रति सौबलः}


\twolineshloka
{तामापतन्तीं सहसा शरैः कनकभूषणैः}
{त्रिधा चिच्छेद समरे सहदेवो हसन्निव}


\twolineshloka
{सा पपात त्रिधा च्छिन्ना भूमौ कनकभूषणा}
{शीर्यमाणा यथा दीप्ता गगनाद्वै शतहदा}


\twolineshloka
{शक्तिं विनिहतां दृष्ट्वा सौबलं च भयार्दितम्}
{दुद्रुवुस्तावकाः सर्वे भये जाते ससौबलाः}


\twolineshloka
{अथोत्क्रुष्टं महच्चासीत्पाण्डवैर्जितकाशिभिः}
{धार्तराष्ट्रास्ततः सर्वे प्रायशो विमुखाऽभवन्}


\twolineshloka
{तान्वै विमनसो दृष्ट्वा माद्रीपुत्रः प्रतापवान्}
{शरैरनेकसाहस्रैर्वारयामास संयुगे}


\twolineshloka
{ततो गान्धारकैर्गुप्तं पुष्टैरश्वैर्जये धृतम्}
{आससाद रमे यान्तं सहदेवोऽथ सौबलम्}


\twolineshloka
{स्वमंशमवशिष्टं तं संस्मृत्य शकुनिं नृप}
{रथेन काञ्चनाङ्गेन सहदेवः समभ्ययात्}


\twolineshloka
{अधिज्यं बलवत्कृत्वा व्याक्षिपन्सुमहद्धनुः}
{स सौबलमभिद्रुत्य गार्ध्रपत्रैः शिलाशितैः}


\twolineshloka
{भृशमभ्यहन्त्कुद्धस्तोत्रैरिव महाद्विपम्}
{उवाच चैनं मेधावी विगृह्य स्मारयन्निव}


\threelineshloka
{क्षत्रधर्मे स्थिरो भूत्वा युध्यस्व पुरुषो भव}
{यत्तदा भाषसे मूढ गृह्णन्नक्षान्सभातले}
{फलमद्य प्रपद्यस्व कर्मणस्तस्य दुर्मते}


\twolineshloka
{निहतास्ते दुरात्मानो येऽस्मानवहसन्पुरा}
{दुर्योधनः कुलाङ्गारः शिष्टस्त्वं चास्य मातुलः}


\twolineshloka
{अद्य ते निहनिष्यामि क्षुरेणोन्मथितं शिरः}
{वृक्षात्फलमिवाविद्वं लगुडेन प्रमाथिना}


\twolineshloka
{एवमुक्त्वा महाराज सहदेवो महाबलः}
{सङ्क्रुद्धो रणशार्दूलो वेगेनाभिजगाम तम्}


\twolineshloka
{अभिगम्य सुदुर्धर्षः सहदेवो युधां पतिः}
{विकृष्य बलवच्चापं क्रोधेन प्रज्वलन्निव}


\twolineshloka
{शकुनिं दशभिर्विद्ध्वा चतुर्भिश्चास्य वाजिनः}
{छत्रं ध्वजं धनुश्चास्य च्छित्त्वा सिंह इवानदत्}


\twolineshloka
{छिन्नध्वजधनुश्छत्रः सहदेवेन सौबलः}
{कृतो विद्धश्च बहुभिः सर्वमर्मसु सायकैः}


\twolineshloka
{ततो भूयो महाराज सहदेवः प्रतापवान्}
{शकुनेः प्रेषयान्मास शरवृष्टिं दुरासदाम्}


\twolineshloka
{ततस्तु क्रुद्धः सुबलस्य पुत्रोमाद्रीसुतं सहदेवं विमर्दे}
{प्रासेन जाम्बूनदभूषणेनजिघांसुरेकोऽभिपपात शीघ्रम्}


\twolineshloka
{माद्रीसुतस्तस्य समुद्यतं तंप्रासं सुवृत्तौ च भुजौ रणाग्रे}
{भल्लैस्त्रिभिर्युगपत्सञ्चकर्तननाद चोच्चैस्तरसाऽऽजिमध्ये}


\twolineshloka
{तस्याशुकारी सुसमाहितेनसुवर्णपुङ्खेन दृढायसेन}
{भल्लेन सर्वावरणातिगेनशिरः शरीरात्प्रममाथ भूयः}


\twolineshloka
{शरेण कार्तस्वरभूषितेनदिवाकराभेण सुसंहितेन}
{हृतोत्तमाङ्गो युधि पाण्डवेनपपात भूमौ सुबलस्य पुत्रः}


\twolineshloka
{स तच्छिरो वेगवता शरेणसुवर्णपुङ्खेन शिलाशितेन}
{प्रावेरयत्कुपितः पाण्डुपुत्रोयत्तत्कुरूणामनयस्य मूलम्}


\twolineshloka
{भुजौ सुवृत्तौ प्रचकर्त वीरःपश्चात्कबन्धं रुधिरावसिक्तम्}
{विस्पन्दमानं निपपात घोरंरथोत्तमात्पार्थिव पार्थिवस्य}


\twolineshloka
{हृतोत्तमाङ्गं शकुनिं समीक्ष्यभूमौ शयानं रुधिरार्द्रगात्रम्}
{योधास्त्वदीया भयनष्टसत्वादिशः प्रजग्मुः प्रगृहीतशस्त्राः}


\twolineshloka
{प्रविद्रुताः शुष्कमुखा विसञ्ज्ञागाण्डीवघोषेण समाहताश्च}
{भयार्दिता भग्नरथाश्वनागाःपदातयश्चैव सधार्तराष्ट्राः}


% Check verse!
ततो रथाच्छकुनिं पातयित्वामुदान्विता भारत पाण्डवेयाःशङ्कान्प्रदध्युः समरेऽतिहृष्टाःसकेशवाः सैनिकान्हर्षयन्तः
\twolineshloka
{तं चापि सर्वे प्रतिपूजयन्तोदृष्ट्वा ब्रुवाणाः सहदेवमाजौ}
{दिष्ट्या हतो नैकृतिको महात्मासहात्मजो वीर रणे त्वयेति}


\chapter{अध्यायः २९}
\twolineshloka
{स़ञ्जय उवाच}
{}


\twolineshloka
{ततः क्रुद्धा महाराज सौबलस्य पदानुगाः}
{त्यक्त्वा जीवितमाक्रन्दे पाण्डवान्पर्यवारयन्}


\twolineshloka
{तानर्जुनः प्रत्यगृह्णात्सहदेवजये धृतः}
{भीमसेनश्च तेजस्वी क्रुद्धाशीविषदर्शनः}


\twolineshloka
{शक्त्यृ-ष्टिप्रासहस्तानां सहदेवं जिघांसताम्}
{सङ्कल्पमकरोन्मोघं गाण्डीवेन धनञ्जयः}


\twolineshloka
{सङ्गृहीतायुधान्बाहून्योधानामभिधावताम्}
{भल्लैश्चिच्छेद बीभत्सुः शिरांस्यपि हयानपि}


\twolineshloka
{ते हयाः प्रत्यपद्यन्त वसुधां विगतासवः}
{चरता लोकवीरेण प्रहताः सव्यसाचिना}


\twolineshloka
{ततो दुर्योधनो राजा दृष्ट्वा स्वबलसङ्क्षयम्}
{हतशेषान्समानीय क्रुद्धो रथगणान्बहून्}


\twolineshloka
{कुञ्जरांश्च हयांश्चैव पादातांश्च समन्ततः}
{उवाच दुःखितान्सर्वान्धार्तराष्ट्र इदं वचः}


\twolineshloka
{समासाद्य रणे सर्वान्पाण्डवान्ससुहृद्गणान्}
{पाञ्चाल्यं चापि सबलं हत्वा शीघ्रं न्यवर्तत}


\twolineshloka
{तस्य ते शिरसा गृह्य वचनं युद्धदुर्मदाः}
{अभ्युद्ययू रणे पार्थांस्तव पुत्रस्य शासनात्}


\twolineshloka
{तानभ्यापततः शीघ्रं हतशेषान्महारणे}
{शरैराशीविषाकारैः पाण्डवाः समवाकिरन्}


\twolineshloka
{तत्सैन्यं भरतश्रेष्ठ मुहूर्तेन महात्मभिः}
{अवध्यत रणं प्राप्य त्रातारं नाभ्यविन्दत}


\twolineshloka
{पलायमानं तु भयान्नावतिष्ठति दंशितम्}
{अश्वैर्विपरिधावद्भिः सैन्येन रजसा वृते}


\twolineshloka
{न प्राज्ञायन्त समरे दिशः सप्रदिशस्तथा}
{ततस्तु पाण्डवानीकान्निःसृत्य बहवो जनाः}


\twolineshloka
{अभ्यघ्नंस्तावकान्युद्धे मुहूर्तादिव भारत}
{ततो निःशेषमभवत्तत्सन्यं तव भारत}


\twolineshloka
{अक्षौहिण्यः समेतास्तु तव पुत्रस्य भारत}
{एकादश हता युद्धे ताः प्रभो पाण्डुसृञ्जयैः}


\twolineshloka
{तेषु राजसहस्रेषु तावकेषु महात्मसु}
{एको दुर्योधनो राजन्नदृश्यत भृशं क्षतः}


\twolineshloka
{ततो वीक्ष्य दिशः सर्वा दृष्ट्वा शून्यां च मेदिनीम्}
{विहीनः सर्वयोधैश्च पाण्डवान्वीक्ष्यं संयुगे}


\twolineshloka
{मुदितान्सर्वतः सिद्धान्नर्दमानान्समन्ततः}
{बाणशब्दरवांश्चैव श्रुत्वा तेषां महात्मनाम्}


\twolineshloka
{दुर्योधनो महाराज कश्मलेनाभिसंवृतः}
{अपयो मनश्चक्रे विहीनबलवाहनः}


\chapter{अध्यायः ३०}
\twolineshloka
{धृतराष्ट्र उवाच}
{}


\twolineshloka
{निहते मामके सैन्ये निःशेषे शिबिरे कृते}
{पाण्डवानां बले सूत किन्नु शेषमभूत्तदा}


\fourlineindentedshloka
{एतन्मे पृच्छतो ब्रूहि कुशलो ह्यसि सञ्जय}
{यच्च दुर्योधनो मानी कृतवांस्तनयो मम}
{वलक्षयं तथा दृष्ट्वा स एकः पृथिवीपतिः ॥सञ्जय उवाच}
{}


\twolineshloka
{रथानां द्वे सहस्रे तु सप्त नागशतानि च}
{पञ्च चाश्वसहस्राणि पत्तीनामयुतानि च}


\twolineshloka
{एतच्छेषमभूद्राजन्पाण्डवानां महद्बलम्}
{परिगृह्य हि यद्युद्धे धृष्टद्युम्नो व्यवस्थितः}


\twolineshloka
{एकाकी भरतश्रेष्ठ ततो दुर्योधनो नृपः}
{नापश्यत्समरे कञ्चित्सहायं रथिनां वरः}


\threelineshloka
{नर्दमानान्परान्दृष्ट्वा स्वबलस्य च सङ्क्षयम्}
{दृष्ट्वा भरतशार्दूलः कश्मलेनाभिसंवृतः}
{हतं स्वहयमुत्सृज्य प्राङ्मुखः प्राद्रावद्भयात्}


\twolineshloka
{एकादशचमूभर्ता पुत्रो दुर्योधनस्तव}
{गदामादाय तेजस्वी पदातिः प्रस्थितो हदम्}


\twolineshloka
{नातिदूरं ततो गत्वा पद्भ्यामेव नराधिपः}
{सस्मार वचनं क्षत्रुर्धर्मशीलस्य धीमतः}


\twolineshloka
{इदं नूनं महाप्राज्ञो विदुरो दृष्ट्वान्पुरा}
{महद्व्सनमस्माकं क्षत्रियाणां च सर्वशः}


\twolineshloka
{एवं विचिन्तयानस्तु प्रविविक्षुर्हदं नृपः}
{दुःखसन्तप्तहृदयो दृष्ट्वा राजन्बलक्षयम्}


\twolineshloka
{`दशैकाक्षौहिणीभर्ता तदा दुर्योधनोऽपि सन्}
{प्राप्तवान्व्यसनं तीव्रं दैवं हि बलवत्तरम्'}


\twolineshloka
{पाण्डवास्तु महाराज धृष्टद्युम्नपुरोगमाः}
{अभ्यद्रवन्त सङ्क्रुद्धास्तव राजन्बलं प्रति}


\twolineshloka
{शक्त्यृष्टिप्रासहस्तानां बलानामभिगर्जताम्}
{सङ्कल्पमकरोन्मोघं गाण्डीवेन धनञ्जयः}


\twolineshloka
{तान्हत्वा निशितैर्वाणैः सामात्यान्सह बन्धुभिः}
{रथे श्वेतहये तिष्ठन्नर्जुनो बह्वशोभत}


\twolineshloka
{सुबलस्य हते पुत्रे सवाजिरथकुञ्जरे}
{महावनमिव च्छिन्नमभवत्तावकं बलम्}


\twolineshloka
{अनेकशतसाहस्रे बले दुर्योधनस्य ह}
{नान्यो महारथो राजञ्जीवमानो व्यदृश्यत}


\twolineshloka
{द्रोणपुत्रादृते वीरात्तथैव कृतवर्मणः}
{कृपाच्च गौतमाद्राजन्पार्थिवाच्च तवात्मजात्}


\twolineshloka
{धृष्टद्युम्नस्तु मां दृष्ट्वा हसन्सात्यकिमब्रवीत्}
{किमनेन गृहीतेन नानेनार्थोऽस्ति जीवता}


\twolineshloka
{धृष्टद्युम्नवचः श्रुत्वा शिनेर्नप्ता महारथः}
{उद्यम्य निशितं खङ्गं हन्तुं मामुद्यतस्तदा}


\twolineshloka
{तमागम्य महाप्राज्ञः कृष्णद्वैपायनोऽब्रवीत्}
{मुच्यतां सञ्जयो जीवन्न हन्तव्यः कथञ्चन}


\twolineshloka
{द्वैपायनवचः श्रुत्वा शिनेर्नप्ता कृताञ्जलिः}
{ततो मामब्रवीन्मुक्त्वा स्वस्ति सञ्जय साधय}


\twolineshloka
{अनुज्ञातस्त्वहं तेन न्यस्तवर्मा निरायुधः}
{प्रातिष्ठं येन नगरं सायाह्ने रुधिरोक्षितः}


\twolineshloka
{क्रोशमात्रमपक्रान्तं गदापाणिमवस्थितम्}
{एकं दुर्योधनं राजन्नपश्यं भृशविक्षतम्}


\twolineshloka
{स तु मामश्रुपूर्णाक्षो नाशक्नोदभिवीक्षितुम्}
{उपप्रैक्षत मां दृष्ट्वा तथा दीनमवस्थितम्}


\twolineshloka
{तं चाहमपि शोचन्तं दृष्ट्वैकाकिनमाहवे}
{मुहूर्तं नाशकं वक्तुमतिदुःखपरिप्लुतः}


\twolineshloka
{`यस्य मूर्धाभिषिक्तानां सहस्रमणिमौलिनाम्}
{आहृत्य च करं सर्वं स्वस्य वेश्म समागतम्}


\twolineshloka
{चतुःसागरपर्यन्ता पृथिवी रत्नभूषिता}
{कर्णेनैकेन यस्यार्थे करमाहारिता पुरा}


\twolineshloka
{यस्याज्ञा परराष्ट्रेषु कर्णेनैव प्रसारिता}
{नाभवद्यस्य शस्त्रेषु खेदो राज्ञः प्रशासतः}


\twolineshloka
{आसीनो हास्तिनपुरे क्षेमं राज्यमकण्टकम्}
{अन्वपालयदैश्वर्यात्कुबेरमपि नास्मरत्}


\twolineshloka
{भवनाद्भवनं राजन्प्रयातुं पृथिवीपते}
{देवालयप्रदेशे च पन्था यस्य हिरण्मयः}


\twolineshloka
{पताकावृतसूर्यांशुतोरणोच्छ्रितशोभिताः}
{प्रयाणे पृथिवीभर्तुर्धन्यानामभवन्गृहाः}


\twolineshloka
{आरुह्यैरावतप्रख्यं नागमिन्द्रसमो बली}
{विभूत्या सुमहत्या यः प्रयाति पृथिवीपते}


\twolineshloka
{तं भृशक्षतसर्वाङ्गं पद्ध्यामेव धरातले}
{तिष्ठन्तमेकं दृष्ट्वा तु ममाभूत्क्लेश उत्तमः}


\twolineshloka
{तस्य चैवंविधस्याद्य जगन्नाथस्य भूपते}
{आपदप्रतिमैवाभूद्बलीयान्विधिरेव हि'}


\twolineshloka
{ततोऽस्मै तदहं सर्वमुक्तवान्ग्रहणं तदा}
{द्वैपायनप्रसादाच्च जीवतो मोक्षमाहवे}


\twolineshloka
{स मुहूर्तमिव ध्यात्वा प्रतिलभ्य च चेतनाम्}
{भ्रातॄंश्च सर्वसैन्यानि पर्यपृच्छत मां ततः}


\twolineshloka
{तस्मै तदहमाचक्षे सर्वं प्रत्यक्षदर्शिवान्}
{भ्रातॄश्च निहतान्सर्वान्सैन्यं च विनिपातितम्}


\twolineshloka
{त्रयः किल रथाः शिष्टास्तावकानां नराधिप}
{इति प्रस्थानकाले मां कृष्णद्वैपायनोऽब्रवीत्}


\twolineshloka
{स दीर्घमिव निःश्वस्य प्रत्यवेक्ष्य पुनः पुनः}
{असौ मां पाणिना स्पृष्ट्वा पुत्रस्ते पर्यभाषत}


\twolineshloka
{त्वदन्यो नेह सङ्ग्रामे कश्चिज्जीवति सञ्जय}
{द्वितीयं नेह पश्यामि ससहायाश्च पाण़्डवाः}


\twolineshloka
{ब्रूयाः सञ्जय राजानं पज्ञाचक्षुषमीश्वरम्}
{दुर्योधनस्तव सुतः प्रविष्टो हदमित्युत}


\twolineshloka
{सुहृद्भिस्तादृशैर्हीनः पुत्रैर्भ्रातृभिरेव च}
{पाण्डवैश्च हृते राज्ये को नु जीवेत मादृशः}


\twolineshloka
{आचक्षीथाः सर्वमिदं मां च मुक्तं महाहवात्}
{आस्मिंस्तोयहदे गुप्तं जीवन्तं भृशविक्षतम्}


\twolineshloka
{एवमुक्त्वा महाराज प्राविशत्तं महाहदम्}
{अस्तम्भयत तोयं च मायया मनुजाधिपः}


\twolineshloka
{तस्मिन्हदं प्रविष्टे तु त्रीन्रथाञ्श्रान्तवाहनान्}
{अपश्यं सहितानेकस्तं देशं समुपेयुषः}


\twolineshloka
{कृपं शारद्वतं वीरं द्रौणिं च रथिनां वरम्}
{भोजं च कृतवर्माणं सहिताञ्शरविक्षतान्}


\twolineshloka
{ते सर्वं मामभिप्रेक्ष्य तूर्णमश्वाननोदयन्}
{उपयाय तु मामूचुर्दिष्ट्या जीवसि सञ्जय}


\twolineshloka
{अपृच्छंश्चैव मां सर्वे पुत्रं तव जनाधिपम्}
{कच्चिद्दुर्योधनो राजा स नो जीवति सञ्जय}


\twolineshloka
{आख्यातवानहं तेभ्यस्तदा कुशलिनं नृपम्}
{तच्चैव सर्वमाचक्षं यन्मां दुर्योधनोऽब्रवीत्}


\twolineshloka
{हदं चैवाहमाचक्षं यं प्रविष्टो नराधिपः}
{अश्वत्थामा तु तद्राजन्निशम्य वचनं मम}


\threelineshloka
{तं हदं विपुलं प्रेक्ष्य करुणं पर्यदेवयत्}
{अहो धिक्स न जानाति जीवतोऽस्मान्नराधिप}
{पर्याप्ता हि वयं तेन सह योधयितुं परान्}


\twolineshloka
{ते तु तत्र चिरं कालं विलप्य च महारथाः}
{प्राद्रवन्रथिनां श्रेष्ठा दृष्ट्वा पाण्डुसुतान्रणे}


\twolineshloka
{ते तु मां रथमारोप्य कृपस्य सुपरिष्कृतम्}
{सेनानिवेशमाजग्मुर्हतशेषास्त्रयो रथाः}


\twolineshloka
{तत्र गुल्माः परिक्षिप्ताः सूर्ये चास्तमिते सति}
{सर्वे विचुक्रुशुः श्रुत्वा पुत्राणां तव संक्षयम्}


\twolineshloka
{ततो वृद्धा महाराज योषितां रक्षिणो नराः}
{राजदारानुपादाय पययुर्नगरं प्रति}


\twolineshloka
{तत्र विक्रोशमानानां रुदतीनां च सर्वशः}
{प्रादुरासीन्महाञ्शब्दः श्रुत्वा तद्बलसङ्क्षयम्}


\twolineshloka
{ततस्ता योषितो राजन्रुदन्त्यो वै मुहुर्मुहुः}
{कुरर्य इव शब्देन नादयन्त्यो महीतलम्}


\twolineshloka
{आजघ्नुः करजैश्चापि पाणिभिश्च शिरांस्युत}
{लुलुचुश्च तदा केशान्क्रोशन्त्यस्तत्रतत्र ह}


\twolineshloka
{हाहाकारविनादिन्यो विनिघ्नन्त्य उरांसि च}
{शोचन्त्यस्तत्र रुरुदुः क्रन्दमाना विशाम्पते}


\twolineshloka
{ततो दुर्योधनामात्याः साश्रुकण्ठा भृशातुराः}
{राजदारानुपामन्त्र्य प्रययुर्नगरं प्रति}


\threelineshloka
{वेत्रव्यासक्तहस्ताश्च द्वाराध्यक्षा विशाम्पते}
{शयनीयानि शुभ्राणि स्पर्ध्यास्तरणवन्ति च}
{समादाय ययुस्तूर्णं नगरं दाररक्षिणः}


\twolineshloka
{आस्थायाश्वतरीयुक्तान्स्यन्दनानपरे पुनः}
{स्वान्स्वान्दारानुपादाय प्रययुर्नगरं प्रति}


\twolineshloka
{अदृष्टपूर्वा या नार्यो भास्करेणापि वेश्मसु}
{ददृशुस्ता महाराज जना याताः पुरं प्रति}


\twolineshloka
{ताः स्त्रियो भरतश्रेष्ठ सौकुमार्यसमन्विताः}
{प्रययुर्नगरं तूर्णं हतस्वपतिबान्धवाः}


\twolineshloka
{आगोपालाविपालेभ्यो द्रवन्तो नगरं प्रति}
{ययुर्मनुष्याः सम्भ्रान्ता भीमसेनभयार्दिताः}


\twolineshloka
{अपिचैषां भयं तीव्रं पार्थेभ्योऽभूत्सुदारुणम्}
{प्रेक्षमाणास्तदाऽन्योन्यमाधावन्नगरं प्रति}


\twolineshloka
{तस्मिंस्तथा वर्तमाने विद्रवे भृशदारुणे}
{युयुत्सुः शोकसम्मूढः प्राप्तकालमचिन्तयत्}


\twolineshloka
{जितो दुर्योधनः सङ्ख्ये पाण्डवैर्भीमविक्रमैः}
{एकादशचमूभर्ता भ्रातरश्चास्य सूदिताः}


\twolineshloka
{हताश्च कुरवः सर्वे भीष्मद्रोणपुरः सराः}
{अहमेको विमुक्तस्तु भाग्ययोगाद्यदृच्छया}


\twolineshloka
{विद्रुतानि च सर्वाणि शिबिराद्वै समन्ततः}
{[इतस्ततः पलायन्ते हतनाथा हतौजसः}


\twolineshloka
{अदृष्टपूर्वा दुःखार्ता भयव्याकुललोचनाः}
{हरिणा इव वित्रस्ता वीक्षमाणा दिशो दश]}


\threelineshloka
{दुर्योधनस्य सचिवा ये केचिदवशेषिताः}
{राजदारानुपादाय प्रययुर्नगरं प्रति}
{प्राप्तकालमहं मन्ये प्रवेशं तैः सह प्रभो}


\twolineshloka
{युधिष्ठिरमनुज्ञाय वासूदेवं तथैव च}
{एतमर्थं महाबाहुरुभयोः सन्न्यवेदयत्}


\twolineshloka
{तस्य प्रीतोऽभवद्राजा नित्यं करुणवेदिता}
{परिष्वज्य महाबाहुर्वैश्यापुत्रं व्यसर्जयत्}


\twolineshloka
{ततः स रथमास्थाय द्रुतमश्वानचोदयत्}
{संवाहयितवांश्चापि राजदारान्पुरं प्रति}


\twolineshloka
{तैश्चैव सहितः क्षिप्रमस्तं गच्छति भास्करे}
{प्रविष्टो हास्तिनपुरं बाष्पकण्ठोऽश्रुलोचनः}


\twolineshloka
{अपश्यत महाप्राज्ञं विदुरं साश्रुलोचनम्}
{राज्ञः समीपान्निष्क्रान्तं शोकोपहतचेतसम्}


\twolineshloka
{तमब्रवीत्सत्यधृतिः प्रणतं त्वग्रतः स्थितम् ॥विदुर उवाच}
{}


\fourlineindentedshloka
{दिष्ट्या कुरुक्षये वृत्ते अस्मिंस्त्वं पुत्र जीवसि}
{विना राज्ञः पर्वेशार्द्वै किमसि त्वमिहागतः}
{एतद्वै कारणं सर्वं विस्तरेण निवेदय ॥युयुत्सुरुवाच}
{}


\threelineshloka
{निहते शकुनौ तत्र सज्ञातिसुतबान्धवे}
{हतशेषपरीवारो राजा दुर्योधनस्ततः}
{स्वकं स हयमुत्सृज्य प्राङ्मुखः प्राद्रवद्भयात्}


\twolineshloka
{अपक्रान्ते तु नृपतौ स्कन्धावारनिवेशनात्}
{भयव्याकुलितं सर्वं प्राद्रावन्नगरं प्रति}


\twolineshloka
{ततो राज्ञः कलत्राणि भ्रातॄणां चास्य सर्वतः}
{वाहनेषु समारोप्य अध्यक्षाः प्राद्रावन्भयात्}


\threelineshloka
{ततोऽहं समनुज्ञाप्य राजानं सहकेशवम्}
{प्रविष्टो हास्तिनपुरं रक्षन्लोकस्य वाच्यताम् ॥सञ्जय उवाच}
{}


\threelineshloka
{एतच्छ्रुत्वा तु वचनं वैश्यापुत्रेण भाषितम्}
{प्राप्तकालमिति ज्ञात्वा विदुरः सर्वधर्मवित्}
{अपूजयदमेयात्मा युयुत्सुं वाक्यमब्रवीत्}


\twolineshloka
{प्राप्तकालमिदं सर्वं ब्रुवता भरतक्षये}
{[रक्षितः कुलधर्मश्च सानुक्रोशतया त्वया}


\twolineshloka
{दिष्ठ्या त्वामिह सङ्ग्राम दस्माद्वीरक्षयात्पुरम्}
{समागतमपश्याम ह्यंशुमन्तमिव प्रजाः}


\threelineshloka
{अन्धस्यं नृपतेर्यष्टिर्लुब्धस्यादीर्घदर्शिनः}
{बहुशो याच्यमानस्य दैवोपहतचेतसः}
{त्वमेको व्यसनार्तस्य ध्रियसे पुत्र सर्वथा]}


\threelineshloka
{अद्य त्वमिह विश्रान्तः श्वोऽभिगन्ता युधिष्ठिरम्}
{एतावदुक्त्वा वचनं विदुरः साश्रुलोचनः}
{युयुत्सुं समनुज्ञाप्य प्रविवेश नृपक्षयम्}


\threelineshloka
{[पौरजानपदैर्दुःखाद्धाहेति भृशनादितम्}
{निरानन्दं गतश्रीकं हृताराममिवाशयम्}
{शून्यरूपमपध्वस्तं दुःखाद्दुःखतरोऽभवत्}


\twolineshloka
{विदुरः सर्वधर्मज्ञो विक्लवेनान्तरात्मना}
{विवेश नगरे राजन्निशश्वास शनैः शनैः}


\threelineshloka
{युयुत्सुरपि तां रात्रिं स्वगृहे न्यवसत्तदा}
{वन्द्यमानः स्वकैश्चापि नाभ्यनन्दत्सुदुःखितः}
{चिन्तयानः क्षयं तीव्रं भरतानां परस्परम्}


\chapter{अध्यायः ३१}
\twolineshloka
{`सञ्जय उवाच}
{}


\twolineshloka
{मुहूर्तादिव राजेन्द्र सर्वं शून्यमदृश्यत}
{मत्तवारणसंघुष्टं शिबिरं विद्रुते बले}


\threelineshloka
{यत्र शब्देन महता नान्वबुध्यन्महारथाः}
{तत्र शब्दं न शृणुमो मनुष्यस्यापि कस्यचित्' ॥धृतराष्ट्र उवाच}
{}


\twolineshloka
{हतेषु सर्वसैन्येषु पाण्डुपुत्रै रणाजिरे}
{मामकाश्चावशिष्टास्ते किमकुर्वत सञ्जय}


\threelineshloka
{कृतवर्मा कृपश्चैव द्रोणपुत्रश्च वीर्यवान्}
{दुर्योधनश्च मन्दात्मा राजा किमकरोत्तदा ॥सञ्जय उवाच}
{}


\twolineshloka
{सम्प्राद्रुवत्सु दारेषु क्षत्रियाणां महात्मनाम्}
{विद्रुते शिबिरे शून्ये भृशोद्विग्नास्त्रयो रथाः}


\threelineshloka
{निशम्य पाण्डुपुत्राणां तदा वैजयिनां स्वनम्}
{विद्रुतं शिबिरं दृष्ट्वा सायाह्ने राजगृद्विनः}
{स्थानं नारोचयंस्तत्र ततस्ते हदमभ्ययुः}


\twolineshloka
{युधिष्ठिरोऽपि धर्मात्मा भ्रातृभिः सहितो रणे}
{हृष्टः पर्यपतद्राजन्दुर्योधनवधेप्सया}


\twolineshloka
{मार्गमाणास्तु सङ्क्रुद्धास्तव पुत्रं जयैषिणः}
{यत्नतोऽन्वेषमाणास्ते नैवापश्यञ्जनाधिपम्}


\twolineshloka
{यदा दुर्योधनो युद्धं त्यक्त्वा पद्भ्यां पराक्रममत्}
{तं हदं प्राविशच्चापि विष्टभ्यापः स्वमायया}


\twolineshloka
{यदा तु पाण्डवाः सर्वे सुपरिश्रान्तवाहनाः}
{ततः स्वशिबिरं प्राप्य व्यतिष्ठन्त ससैनिकाः}


\twolineshloka
{ततः कृपश्च द्रौणिश्च कृतवर्मा च सात्वतः}
{सन्निविष्टेषु पार्थेषु प्रययुस्तं हदं शनैः}


\twolineshloka
{ते तं हदं समासाद्य यत्र शेते जनाधिपः}
{अभ्यभाषन्त दुर्धर्षं राजानं सुप्तमम्भसि}


\twolineshloka
{राजन्नुत्तिष्ठ युध्यस्व सहास्माभिर्युधिष्ठिरम्}
{जित्वा वा पृथिवीं भुङ्क्ष्व हतो वा स्वर्गमाप्नुहि}


\twolineshloka
{तेषामपि बलं सर्वं हतं दुर्योधन त्वया}
{प्रतिविद्धाश्च भूयिष्ठं ये शिष्टास्तत्र सैनिकाः}


\threelineshloka
{न ते वेगं विषहितुं शक्तास्तव विशाम्पते}
{अस्माभिरपि गुप्तस्य तस्मादुत्तिष्ठ भारत ॥दुर्योधन उवाच}
{}


\twolineshloka
{दिष्ट्या पश्यामि वो मुक्तानीदृशात्पुरुषक्षयात्}
{पाण्डुकौरवसम्मर्दाज्जीवमानान्नरर्षभान्}


\threelineshloka
{विजेष्यामो वयं सर्वे विश्रान्ता विगतक्लमाः}
{भवन्तश्च परिश्रान्ता वयं च भृशविक्षताः}
{उदीर्णं च बलं तेषां तेन युद्धं न रोचये}


\twolineshloka
{न त्वेतदद्भुतं वीरा यद्वो महदिदं मनः}
{अस्मासु च परा शक्तिर्न तु कालः पराक्रमे}


\threelineshloka
{विश्रम्यैकां निशामद्य भवद्भिः सहितो रणे}
{प्रतियोत्स्याम्यहं शत्रूञ्श्वो न स्याच्च श्रमो मम ॥सञ्जय उवाच}
{}


\twolineshloka
{एवमुक्तोऽब्रवीद्द्रौणी राजानं युद्धदुर्मदम्}
{उत्तिष्ठ राजन्भद्रं ते विजेष्यामो वयं परान्}


\twolineshloka
{इष्टापूर्तेन दानेन सत्येन च जपेन च}
{शपे राजन्यथा ह्यद्य निहनिष्यामि सोमकान्}


\twolineshloka
{मा स्म यज्ञकृतां प्रीतिमाप्नुयां सज्जनोचिताम्}
{यदीमां रजनीं व्युष्टां न हि हन्मि परान्रणे}


\twolineshloka
{नाहत्वा सर्वपाञ्चालान्विमोक्ष्ये कवचं विभो}
{उत्तिष्ठ त्वं ब्रवीम्येतत्तन्मे शृणु जनाधिप}


\twolineshloka
{तेषु सम्बाषमाणेषु व्याधास्तं देशमाययुः}
{मांसभारपरिश्रान्ताः पानीयार्थं यदृच्छया}


\twolineshloka
{ते हि नित्यं महाराज भीमसेनस्य लुब्धकाः}
{मांसभारानुपाजह्नुर्भक्त्या परमया विभो}


\twolineshloka
{ते तत्र धिष्ठितास्तेषां सर्वं तद्वचनं रहः}
{दुर्योधनवचश्चैव शुश्रुवुः सङ्गता मिथः}


\twolineshloka
{तेऽपि सर्वे महेष्वासा अयुद्धार्थिनि कौरवे}
{निर्बन्धं परमं चक्रुस्तदा वै युद्धकाङ्क्षिणः}


\twolineshloka
{तांस्तथा समुदीक्ष्याथ कौरवाणां महारथान्}
{अयुद्धमनसं चैव राजानं स्थितमम्भसि}


\twolineshloka
{तेषां श्रुत्वा च संवादं राज्ञश्च सलिले सतः}
{व्याधा ह्यजानन्राजेन्द्र सलिलस्थं सुयोधनम्}


\twolineshloka
{ते पूर्वं पाण्डुपुत्रेण पृष्टा ह्यासन्सुतं तव}
{यदृच्छोपगतास्तत्र राजानं परिमार्गता}


\twolineshloka
{ततस्ते पाण्डुपुत्रस्य स्मृत्वा तद्भाषितं तदा}
{अन्योन्यमब्रुवन्राजन्मृगव्याधाः शनैरिव}


\twolineshloka
{दुर्योधनं ख्यापयामो धनं दास्यति पाण्डवः}
{सुव्यक्तमिह नः ख्यातो हदे दुर्योधनोनृपः}


\twolineshloka
{तस्माद्गच्छामहे सर्वे यत्र राजा युधिष्ठिरः}
{आख्यातुं सलिले सुप्तं दुर्योधनममर्षणम्}


\twolineshloka
{धृतराष्ट्रात्मजं तस्मै भीमसेनाय धीमते}
{शयानं सलिले सर्वे कथयामो धनुर्भृते}


\twolineshloka
{स नो दास्यति सुप्रीतो धनानि बहुलान्युत}
{किं नो मांसेन शुष्केण परिक्लिष्टेन शोषिणा}


\twolineshloka
{एवमुक्त्वा तु ते व्याधाः सम्प्रहृष्टा धनार्थिनः}
{मांसभारानुपादाय प्रययुः शिबिरं प्रति}


\twolineshloka
{पाण़्डवाश्च महाराज लब्धलक्षाः प्रहारिणः}
{अपश्यमानाः समरे दुर्योधनमवस्थितम्}


\twolineshloka
{निकृतिज्ञस्य पापस्य तस्याभिगमनेप्सया}
{चारान्सम्प्रेषयामासुः समन्तात्तद्रणाजिरे}


\twolineshloka
{आगम्य तु ततः सर्वे नष्टं दुर्योधनं नृपम्}
{न्यवेदयन्त सहिता धर्मराजस्य सैनिकाः}


\twolineshloka
{तेषां तद्वचनं श्रुत्वा चाराणां भरतर्षभ}
{चिन्तामभ्यगमत्तीव्रां निशश्वास च पार्थिवः}


\twolineshloka
{`अरिशेषे जीवति तु सन्दिग्धो विजयो भवेत्}
{राज्यं लभे कथं तद्वि पूजितं विजयादिभिः'}


\twolineshloka
{अथ स्थितानां पाण्डूनां दीनानां भरतर्षभ}
{तस्माद्देशादपक्रम्य त्वरिता लुब्धका विभो}


\twolineshloka
{आजग्मुः शिबिरं हृष्टा दृष्ट्वा दुर्योधनं नृपम्}
{वार्यमाणाः प्रविष्टाश्च भीमसेनस्य पश्यतः}


\twolineshloka
{ते तु पाण्डवमासाद्य भीमसेनं महाबलम्}
{तस्मै तत्सर्वमाचख्युर्यद्वृत्तं यच्च वै श्रुतम्}


\twolineshloka
{ततो वृकोदरो राजन्दत्त्वा तेषां धनं बहु}
{धर्मराजाय तत्सर्वमाचचक्षे परन्तपः}


\twolineshloka
{असौ दुर्योधनो राजन्विज्ञातो मम लुब्धकैः}
{संस्तभ्य सलिलं शेते यस्यार्थे परितप्यसे}


\twolineshloka
{तद्वचो भीमसेनस्य प्रियं श्रुत्वा विशाम्पते}
{अजातशत्रुः कौन्तेयो हृष्टोऽभूत्सह सोदरैः}


\twolineshloka
{तं च श्रुत्वा महेष्वासं प्रविष्टं सलिलहदे}
{क्षिप्रमेव ततोऽगच्छन्पुरस्कृत्य जनार्दनम्}


\twolineshloka
{ततः किलकिलाशब्दः प्रादुरासीद्विशाम्पते}
{पाण्डवानां प्रहृष्टानां पाञ्चालानां च सर्वशः}


\twolineshloka
{सिंहनादांस्ततश्चक्रुः क्ष्वेडाश्च भरतर्षभ}
{त्वरिताः क्षत्रिया राजन्नुदक्रोशन्परस्परम्}


\twolineshloka
{ज्ञातः पापो धार्तराष्ट्रो दृष्टश्चेत्यसकृद्रणे}
{प्राक्रोशन्सोमकास्तत्र हृष्टरूपाः समन्ततः}


\twolineshloka
{तेषामाशु प्रयातानां रथानां तत्र वेगिनाम्}
{वभूव तुमुलः शब्दो दिवस्पृक् पृथिवीपते}


\twolineshloka
{दुर्योधनं परीप्सन्तस्तत्रतत्र युधिष्ठिरम्}
{अन्वयुस्त्वरितास्ते वै राजानं श्रान्तवाहनाः}


\twolineshloka
{अर्जुनो भीमसेनश्च माद्रीपुत्रौ च पाण्डवौ}
{धृष्टद्युम्नश्च पाञ्चाल्यः शिखण्डी चापराजितः}


\threelineshloka
{उत्तमौजा युधामन्युः सात्यकिश्च महारथः}
{पाञ्चालानां च ये शिष्टा द्रौपदेयाश्च भारत}
{हयाश्च सर्वे नागाश्च शतशश्च पदातयः}


\twolineshloka
{ततः प्राप्तो महाराज धर्मराजः प्रतापवान्}
{द्वैपायनहदं घोरं यत्र दुर्योधनोऽभवत्}


\twolineshloka
{शीतामलजलं हृद्यं द्वितीयमिव सागरम्}
{मायया सलिलं स्तभ्य यत्राभूत्ते स्थितः सुतः}


\threelineshloka
{अत्यद्भुतेन विधिना दैवयोगेन भारत}
{सलिलान्तर्गतः शेते दुर्दर्शः कस्यचित्प्रभो}
{मानुषस्य मनुष्येन्द्र गदाहस्तो जनाधिपः}


\twolineshloka
{ततो दुर्योधनो राजा सलिलान्तर्गतो वसन्}
{शुश्रुवे तुमुलं शब्दं जलदोपमनिःस्वनम्}


\twolineshloka
{युधिष्ठिरश्च राजेन्द्र तं हदं सह सोदरैः}
{आजगाम महाराज तव पुत्रवधाय वै}


\twolineshloka
{महता शङ्खनादेन रथनेमिस्वनेन च}
{ऊर्ध्वं धुन्वन्महारेणुं कम्पयंश्चापि मेदिनीम्}


\twolineshloka
{यौधिष्ठिरस्य सैन्यस्य श्रुत्वा शब्दं महारथाः}
{कृतवर्मा कृपो द्रौणी राजानमिदमब्रुवन्}


\twolineshloka
{इमे ह्यायान्ति संहृष्टाः पाण्डवा जितकाशिनः}
{अपयास्यामहे तावदनुजानातु नो भवान्}


\twolineshloka
{दुर्योधनस्तु तच्छ्रुत्वा तेषां तत्र तरस्विनाम्}
{तथेत्युक्त्वा हदं तं वै माययाऽस्तम्भयत्प्रभो}


\twolineshloka
{ते त्वनुज्ञाप्य राजानं भृशं शोकपरायणाः}
{जग्मुर्दूरे महाराज कृपप्रभृतयो रथाः}


\twolineshloka
{ते गत्वा दूरमध्वानं न्यग्रोधं प्रेक्ष्य मारिष}
{न्यविशन्त भृशं श्रान्ताश्चिन्तयन्तो नृपं प्रति}


\twolineshloka
{विष्टभ्य सलिलं सुप्तो धार्तराष्ट्रो महाबलः}
{पाण्डवाश्चापि सम्प्राप्तास्तं देशं युद्धमीप्सवः}


\twolineshloka
{कथं नु युद्धं भविता कथं राजा भविष्यति}
{कथं नु पाण्डवा राजन्प्रतिपत्स्यन्ति कौरवम्}


\twolineshloka
{इत्येवं चिन्तयानास्तु रथेभ्योऽश्वान्विमुच्य ते}
{तत्रासाञ्चक्रिरे राजन्कृपप्रभृतयो रथाः}


\chapter{अध्यायः ३२}
\twolineshloka
{सञ्जय उवाच}
{}


\twolineshloka
{ततस्तेष्वपयातेषु रथेषु त्रिषु पाण्डवाः}
{तं हदं प्रत्यपद्यन्त यत्र दुर्योधनोऽभवत्}


\twolineshloka
{आसाद्य च कुरुश्रेष्ठ तदा द्वैपायनं हदम्}
{स्तम्भितं धार्तराष्ट्रेण दृष्ट्वा तं सलिलाशयम्}


\twolineshloka
{वासुदेवमिदं वाक्यमब्रवीत्कुरुनन्दनः}
{पश्येमां धार्तराष्ट्रेण मायामप्सु प्रयोजिताम्}


\twolineshloka
{विष्टभ्य सलिलं शेते नास्य मानुषतो भयम्}
{दैवीं मायामिमां कृत्वा सलिलान्तर्गतो ह्ययम्}


\fourlineindentedshloka
{निकृत्या निकृतिप्रज्ञो न मे जीवन्विमोक्ष्यते}
{यद्यस्य समरे साह्यं कुरुते वज्रभृत्स्वयम्}
{तथाप्येनं हतं युद्धे लोका द्रक्ष्यन्ति माधव ॥वासुदेव उवाच}
{}


\twolineshloka
{मायाविन इमां मायां मायया जहि भारत}
{मायावी मायया वध्यः सत्यमेतद्युधिष्ठिर}


\twolineshloka
{क्रियाभ्युपायैर्बहुभिर्भायामप्सु प्रयोज्य च}
{जहि त्वं भरतश्रेष्ठ मायात्मानं सुयोधनम्}


\twolineshloka
{क्रियाभ्युपायैरिन्द्रेण निहता दैत्यदानवाः}
{क्रियाभ्युपायैर्बलिभिर्बलिर्बद्धो महात्मना}


\twolineshloka
{क्रियाभ्युपायपूर्वं वै हिरण्याक्षो महासुरः}
{हिरण्यकशिपुश्चैव क्रिययैव निषूदितौ}


\threelineshloka
{वृत्रश्च निहतो राजन्क्रिययैव महाबलः}
{तथा पौलस्त्यतनयो रावणो नाम राक्षसः}
{रामेण निहतो राजन्सानुबन्धः सहानुगः}


% Check verse!
क्रियया योगमास्थाय तथा त्वमपि विक्रम
\twolineshloka
{क्रियाभ्युपायैर्निहतौ मया राजन्पुरातनौ}
{तारकश्च महादैत्यो विप्रचित्तिश्च वीर्यवान्}


\twolineshloka
{वातापिरिल्वलश्चैव त्रिशिराश्च तथा विभो}
{सुन्दोपसुन्दावसुरौ क्रिययैव निषूदितौ}


\twolineshloka
{क्रियाभ्युपायैरिन्द्रेण त्रिदिवं भुज्यते विभो}
{क्रिया बलवती राजन्नान्यत्किंचिद्युधिष्ठिर}


\threelineshloka
{दैत्याश्च दानवाश्चैव राक्षसाः पार्थिवास्तथा}
{क्रियाभ्युपायैर्निहताः क्रियां तस्मात्समाचर ॥सञ्जय उवाच}
{}


\threelineshloka
{इत्युक्तो वासुदेवेन पाण्डवः संशितव्रतः}
{जलस्थं तं महाराज तव पुत्रं महाबलम्}
{अभ्यभाषत कौन्तेयः प्रहसन्निव भारत}


\twolineshloka
{सुयोधन किमर्थोऽयमारम्भोऽप्सु कृतस्त्वया}
{सर्वं क्षत्रं घातयित्वा स्वकुलं च विशाम्पते}


\twolineshloka
{जलाशयं प्रविष्टोऽद्य वाञ्छञ्जीवितमात्मनः}
{उत्तिष्ठ राजन्युध्यस्व सहास्माभिः सुयोधन}


\twolineshloka
{स ते दर्पो नरश्रेष्ठ स च मानः क्व ते गतः}
{यस्त्वं संस्तभ्य सलिलं भीतो राजन्व्यवस्थितः}


\twolineshloka
{सर्वे त्वां शूर इत्येवं जना जल्पन्ति संसदि}
{व्यर्थं तद्भवतो मन्ये शौर्यं सलिलशायिनः}


\twolineshloka
{उत्तिष्ठ राजन्युध्यस्व क्षत्रियोऽसि कुलोद्भवः}
{कौरवेयो विशेषेण कुलं जन्म च संस्मर}


\twolineshloka
{स कथं कौरवे वंशे प्रशंसञ्जन्म चात्मनः}
{युद्वात्त्रस्ततरस्तोयं प्रविश्य प्रतितिष्ठसि}


% Check verse!
अयुद्धेन व्यवस्थानं नैष धर्मः सनातनः
\twolineshloka
{अनार्यजुष्टमस्वर्ग्यं रणे राजन्पलायनम्}
{कथं पारमगत्वा हि युद्धे त्वं वै जिजीविषुः}


\threelineshloka
{इमान्निपतितान्दृष्ट्वा पुत्रान्भ्रातॄन्पितॄंस्तथा}
{सम्बन्धिनो वयस्यांश्च मातुलान्बान्धवांस्तथा}
{घातयित्वा कथं तात हदे तिष्ठति साम्प्रतम्}


\twolineshloka
{शूरमानी न शूरस्त्वं मृषा वदसि भारत}
{शूरोऽहमिति दुर्बुद्धे सर्वलोकस्य शृण्वतः}


\twolineshloka
{न हि शूराः पलायन्ते शत्रून्दृष्ट्वा कथञ्चन}
{ब्रूहि वा त्वं यया वृत्त्या शूर त्यजसि सङ्गरम्}


\twolineshloka
{स त्वमुत्तिष्ठ युध्यस्व विहाय भयमात्मनः}
{घातयित्वा सर्वसैन्यं भ्रातॄंश्चैव सुयोधन}


\twolineshloka
{नेदानीं जीविते बुद्धिः कार्या धर्मचिकीर्षया}
{क्षत्रधर्ममुपाश्रित्य त्वद्विधेन सुयोधन}


\twolineshloka
{यत्तु कर्णमुपाश्रित्य शकुनिं चापि सौबलम्}
{दुःशासनं च मोहात्त्वमात्मानं नावबुद्धवान्}


\twolineshloka
{तत्पापं सुमहत्कृत्वा प्रतियुध्यस्व भारत}
{कथं हि त्वद्विधो मोहाद्रोचयेत पलायनम्}


\twolineshloka
{क्व ते तत्पौरुषं यातं क्व च मानः सुयोधन}
{क्व च विक्रान्तता याता क्व च विस्फूर्जितं महत्}


\twolineshloka
{क्व ते कृतास्त्रता याता किं नु शेषे जलाशये}
{स त्वमुत्तिष्ठ युध्यस्व क्षत्रधर्मेण भारत}


\twolineshloka
{अस्मांस्तु वा पराजित्य प्रशाधि पृथिवीमिमाम्}
{अथवा निहतोस्माभिर्भूमौ स्वप्स्यसि भारत}


\threelineshloka
{एष ते परमो धर्मः सृष्टो धात्रा महात्मना}
{तं कुरुष्व यथातथ्यं पौरुषे स्व व्यवस्थितः ॥सञ्जय उवाच}
{}


\twolineshloka
{एवमुक्तो महाराज धर्मपुत्रेण धीमता}
{सलिलस्थस्तव सुत इदं वचनमब्रवीत्}


\twolineshloka
{नैतच्चित्रं महाराज यद्भीः प्राणिनमाविशेत्}
{न च प्राणभयाद्भीतो व्यपयातोऽस्मि भारत}


\twolineshloka
{अरथश्चानिषङ्गी च निहतः पार्ष्णिसारथिः}
{एकश्चाप्यगणः सङ्ख्ये प्रत्याश्वासमरोचयम्}


\twolineshloka
{न प्राणहेतोर्न भयाश्च विषादाद्विशाम्पते}
{इदमम्भः प्रविष्टोऽस्मि श्रमात्त्विदमनुष्ठितम्}


\threelineshloka
{त्वं चाश्वसिहि कौन्तेय ये चाप्यनुगतास्तव}
{अहमुत्थाय वः सर्वान्प्रतियोत्स्यामि संयुगे ॥युधिष्ठिर उवाच}
{}


\twolineshloka
{आश्वस्ता एव सर्वे स्म चिरं त्वां मृगयामहे}
{तदिदानीं समुत्तिष्ठ युध्यस्वेह सुयोधन}


\threelineshloka
{हत्वा वा समरे पार्थान्स्फीतं राज्यमवाप्नुहि}
{निहतो वा रणेऽस्माभिर्वीरलोकमवाप्स्यसि ॥दुर्योधन उवाच}
{}


\twolineshloka
{यदर्थं राज्यमिच्छामि कुरूणां कुरुनन्दन}
{त इमे निहताः सर्वे भ्रातरो मे जनेश्वर}


\twolineshloka
{क्षीणरत्नां च पृथिवीं हतक्षत्रियपुङ्गवाम्}
{न ह्युत्सहाम्यहं भोक्तुं विधवामिव योषितम्}


\twolineshloka
{अद्यापि त्वहमाशंसे त्वां विजेतुं युधिष्ठिर}
{भङ््क्त्वा पाञ्चालपाण्डूनामुत्साहं भरतर्षभ}


\twolineshloka
{न त्विदानीमहं मन्ये कार्यं युद्धेन कर्हिचित्}
{द्रोणे कर्णे च संशान्ते निहते च पितामहे}


\twolineshloka
{अस्त्विदानीमियं राजन्केवला पृथिवी तव}
{असहायो हि को राजा राज्यमिच्छेत्प्रशासितुम्}


\twolineshloka
{सुहृदस्तादृशान्हत्वा पुत्रान्भ्रातॄन्पितॄनपि}
{भवद्भिश्च हृते राज्ये को नु जीवेत मादृशः}


\twolineshloka
{अहं वनं गमिष्यामि ह्यजिनैः प्रतिवासितः}
{रतिर्हि नास्ति मे राज्ये हतपक्षस्य भारत}


\twolineshloka
{हतबान्धवभूयिष्ठा हताश्वा हतकुञ्जरा}
{एषा ते पृथिवी राजन्भुङ्क्षैनां विगतज्वरः}


\twolineshloka
{वनमेव गमिष्यामि वसानो मृगचर्मणी}
{न हि मे निर्जनस्यास्ति जीवितेऽद्य स्पृहा विभो}


\threelineshloka
{गच्छ त्वं भुङ्क्ष राजेन्द्र पृथिवीं निहतेश्वराम्}
{हतयोधां नष्टरत्नां शीर्णक्षत्रां यथासुखम् ॥[सञ्जय उवाच}
{}


\threelineshloka
{दुर्योधनं तव सुतं सलिलस्थं महायशाः}
{श्रुत्वा तु करुणं वाक्यमभाषत युधिष्ठिरः ॥]युधिष्ठिर उवाच}
{}


\twolineshloka
{आर्तप्रलापान्मा तात सलिलस्थः प्रभाषथाः}
{नैतन्मनसि मे राजन्वाशितं शकुनेरिव}


\twolineshloka
{यदि वापि समर्थः स्यास्त्वं दानाय सुयोधन}
{नाहमिच्छेयमवनिं त्वया दत्तां प्रशासितुम्}


\twolineshloka
{अधर्मेण न गृह्णीयां त्वया दत्तां महीमिमाम्}
{न हि धर्मः स्मृतो राजन्क्षत्रियस्य प्रतिग्रहः}


\twolineshloka
{त्वया दत्तां न चेच्छेयं पृथिवीमखिलामहम्}
{त्वां तु युद्धे विनिर्जित्य भोक्ताऽस्मि वसुधामिमाम्}


\twolineshloka
{अनीश्वरश्च पृथिवीं कथं त्वं दातुमिच्छसि}
{त्वयेयं पृथिवी राजन्किं न दत्ता तदैव हि}


\twolineshloka
{धर्मतो याचमानानां प्रशमार्थं कुलस्य नः}
{वार्ष्णेयं प्रथमं राजन्प्रत्याख्याय महाबालम्}


\twolineshloka
{किमिदानीं ददासि त्वं को हि ते जित्तविभ्रमः}
{अभियुक्तस्तु को राजा दातुमिच्छेद्धि मेदिनीं}


\twolineshloka
{न त्वमद्य महीं दातुमीशः कौरवनन्दन}
{आच्छेत्तुं वा बलाद्राजन्स कथं दातुमिच्छसि}


\twolineshloka
{मां तु निर्जित्य सङ्ग्रामे पालयेमां वसुन्धराम्}
{सूच्यग्रेणापि यद्भूमेरपि भिद्येत भारत}


\twolineshloka
{तन्मात्रमपि तन्मह्यं न ददाति पुरा भवान्}
{स कथं पृथिवीमेतां प्रददासि विशाम्पते}


% Check verse!
सूच्यग्रं नात्यजः पूर्वं स कथं त्यजसि क्षितिम्
\twolineshloka
{एवमैश्वर्यमासाद्य प्रशास्य पृथिवीमिमाम्}
{को हि मूढो व्यवस्येत शत्रोर्दातुं वसुन्धराम्}


\twolineshloka
{त्वं तु केवलमौर्ख्येण विमूढो नावबुध्यसे}
{पृथिवीं दातुकामोऽपि जीवंस्त्वं नैव मोक्ष्यसे}


\twolineshloka
{अस्मान्वा त्वं पराजित्य प्रशाधि पृथिवीमिमाम्}
{अथवा निहतोऽस्माभिर्व्रज लोकाननुत्तमान्}


\twolineshloka
{आवयोर्जीवतो राजन्मयि च त्वयि च ध्रुवम्}
{संशयः सर्वभूतानां विजये नौ भविष्यति}


\twolineshloka
{जीवितं त्वयि दुष्प्रापं मयि यत्परिवर्तते}
{जीवयेयमहं कामं न तु त्वं जीवितुं क्षमः}


\threelineshloka
{दहने हि कृतो यत्नस्त्वयाऽस्मासु विशेषतः}
{आशीविषैर्विषैश्चापि जले जापि प्रवेशनैः}
{त्वया विनिकृता राजन्राज्यस्य हरणेन च}


\twolineshloka
{अप्रियाणां च वचनैर्द्रौपद्याः कर्षणेन च}
{एतस्मात्कारणात्पाप जीवितं न विद्यते}


% Check verse!
उत्तिष्ठोत्तिष्ठ युध्यस्व युद्धे श्रेयो भविष्यति
\twolineshloka
{एवं तु विविधा वाचो जययुक्ताः पुनःपुनः}
{कीर्तयन्ति स्म ते वीरास्तत्रतत्र जनाधिप}


\chapter{अध्यायः ३३}
\twolineshloka
{धृतराष्ट्र उवाच}
{}


\twolineshloka
{एवं सन्तर्ज्यमानस्तु मम पुत्रो महीपतिः}
{प्रकृत्या मन्युमान्वीरः कथमासीत्परन्तपः}


\twolineshloka
{न हि सन्तर्जना तेन श्रुतपूर्वा कथञ्चन}
{राजभावेन मान्यश्च सर्वलोकस्य सोऽभवत्}


\twolineshloka
{[यस्यातपत्रच्छायापि स्वका भानोस्तथा प्रभा}
{खेदायैवाभिमानित्वात्सहेत्सैवं कथं गरिः ॥]}


\twolineshloka
{इयं च पृथिवी सर्वा सम्लेच्छाटविका भृशम्}
{प्रसादाद्भ्रियते यस्य प्रत्यक्षं तव सञ्जय}


\twolineshloka
{स तथा तर्ज्यमानस्तु पाण्डुपुत्रैर्विशेषतः}
{विहीनश्च स्वकैर्भृत्यैर्निर्जिते चावृतो भृशम्}


\threelineshloka
{स श्रुत्वा कटुका वाचो जययुक्ताः पुनःपुनः}
{किमब्रवीत्पाण्डवेयांस्तन्ममाचक्ष्व सञ्जय ॥सञ्जय उवाच}
{}


\twolineshloka
{तर्ज्यमानस्तदा राजन्नुदकस्थस्तवात्मजः}
{युधिष्ठिरेण राजेन्द्र भ्रातृभिः सहितेन ह}


\twolineshloka
{श्रुत्वा स कटुका वाचो विषमस्थो नराधिपः}
{दीर्घमुष्णं च निःश्वस्य सलिलस्थः पुनःपुनः}


\twolineshloka
{सलिलान्तर्गतो राजा धुन्वन्हस्तौ पुनःपुनः}
{मनश्चकार युद्धाय राजानं चाभ्यभाषत}


\twolineshloka
{यूयं ससुहृदः पार्थाः सर्वे सरथवाहनाः}
{अहमेकः परिद्यूनो विरथो हतवाहनः}


\twolineshloka
{आत्तशस्त्रै रथोपेतैर्बहुभिः परिवास्तिः}
{कथमेकः पदातिः सन्नशस्त्रो योद्धुमुत्सह}


\twolineshloka
{एकैकशश्च मां यूयं योधयध्वं युधिष्ठिर}
{न ह्येको बहुभिर्वीरैर्न्याय्यो योधयितुं युधि}


\twolineshloka
{विशेषतो विकवचः श्रान्तश्चापत्समाश्रितः}
{भृशं विक्षतगात्रश्च श्रान्तवाहनसैनिकः}


\twolineshloka
{न मे त्वत्तो भयं राजन्न च पार्थाद्वृकोदरात्}
{फल्गुनाद्वासुदेवाद्वा पाञ्चालेभ्योऽथवा पुनः}


\twolineshloka
{यमाभ्यां युयुधानाद्वा ये चान्ये तव सैनिकाः}
{एकः सर्वानहं क्रुद्धो वारयिष्ये युधि स्थितः}


\twolineshloka
{धर्ममूला सतां कीर्तिर्मनुष्याणां जनाधिप}
{धर्मं चैवेह कीर्तिं च पालयन्प्रब्रवीम्यहम्}


\twolineshloka
{अहमुत्थाय सर्वान्वै प्रतियोत्स्यामि संयुगे}
{अन्वभ्याशं गतान्सर्वान्निहनिष्यामि भारत}


\twolineshloka
{अद्य वः सरथान्साश्वानशस्त्रो विरथोऽपि सन्}
{नक्षत्राणीव सर्वाणि सविता रात्रिसंक्षये}


\twolineshloka
{तेजसा नाशयिष्यामि स्थिरीभवत पाण़्डवाः}
{अद्यानृण्यं गमिष्यामि क्षत्रियाणां यशस्विनां}


\twolineshloka
{बाह्लीकद्रोणभीष्माणां कर्णस्य च महात्मनः}
{जयद्रथस्य शूरस्य भगदत्तस्य चोभयोः}


\twolineshloka
{मद्रराजस्य शल्यस्य भूरिश्रवस एव च}
{पुत्राणां भरतश्रेष्ठ शकुनेः सौबलस्य च}


\threelineshloka
{मेत्राणां सुहृदां चैव बान्धवानां तथैव च}
{प्रानृण्यमद्य गच्छामि हत्वा त्वां भ्रातृभिः सह ॥सञ्जय उवाच}
{}


\threelineshloka
{एतावदुक्त्वा वचनं विरराम जनाधिपः}
{`सलिलान्तर्गतः श्रीमान्पुत्रो दुर्योधनस्तव ॥'युधिष्ठिर उवाच}
{}


\twolineshloka
{दिष्ट्या त्वमपि जानीषे क्षत्रधर्मं सुयोधन}
{दिष्ट्या ते वर्तते बुद्धिर्युद्धायैव महाभुज}


\twolineshloka
{दिष्ट्या शूरोऽसि गान्धारे दिष्ट्या जानासि सङ्गरम्}
{यस्त्वमेको हि नः सर्वान्सङ्गरे योद्भुमिच्छसि}


\twolineshloka
{एक एकेन सङ्गम्य यत्ते सम्मतमायुधम्}
{तत्त्वमादाय युध्यस्व प्रेक्षकास्ते वयं स्थिताः}


\threelineshloka
{अयमिष्टं च ते कामं वीर भूयो ददाम्यहम्}
{हत्वैकं भव नो राजा हतो वा स्वर्गमाप्नुहि ॥दुर्योधन उवाच}
{}


\twolineshloka
{एकश्चेद्योद्भुमाक्रन्दे वरोऽद्य मम दीयताम्}
{आयुधानामियं चापि मता मे सतं गदा}


\twolineshloka
{भ्रातणां भवतामेकः शक्यं मां योऽभिमन्यते}
{पदातिर्गदया सङ्ख्ये स युध्यतु मया सह}


\twolineshloka
{वृत्तानि रथयुद्धानि विचित्रामि पदेपदे}
{इदमेकं गदायुद्धं भवत्वद्याद्भुतं महत्}


\twolineshloka
{अन्नानामपि पर्यायं कर्तुमिच्छन्ति मानवाः}
{युद्धानामपि पर्यायो भवत्वनुमते तव}


\fourlineindentedshloka
{गदया त्वां महाबाहो विजेष्यामि सहानुजम्}
{पाञ्चालान्सृञ्जयांश्चैव ये चान्ये तव सैनिकाः}
{न हि मे सम्भ्रामो जातु शक्रादपि युधिष्ठिर ॥युधिष्ठिर उवाच}
{}


\twolineshloka
{उत्तिष्ठोत्तिष्ठ गान्धारे मां योधय सुयोधन}
{एक एकेन सङ्गम्य संयुगे गदया बली}


\threelineshloka
{पुरुषो भव गान्धारे युध्यस्व सुसमाहितः}
{अद्य ते जीवितं नास्ति यदीन्द्रोपि तवाश्रयः ॥सञ्जय उवाच}
{}


\twolineshloka
{एतत्स नरशार्दूलो नामृष्यत तवात्मजः}
{सलिलान्तर्गतः श्वभ्रे महानाग इव श्वसन्}


\twolineshloka
{तथाऽसौ वाक्प्रतोदेन तुद्यमानः पुनःपुनः}
{वचो न ममृषे राजन्नुत्तमाश्वः कशामिव}


\threelineshloka
{सङ्क्षोभ्य सलिलं वेगाद्गदामादाय वीर्यवान्}
{अद्रिसारमयीं गुर्वीं काञ्चनाङ्गदभूषणाम्}
{अन्तर्जलात्समुत्तस्थौ नागेन्द्र इव निःश्वसन्}


\twolineshloka
{स भित्त्वा स्तम्भितं तोयं स्कन्धे कृत्वायसीं गदाम्}
{उदतिष्ठत पुत्रस्ते प्रतपन्रश्मिवानिव}


\twolineshloka
{ततः शैक्यायसीं गुर्वी जातरूपपरिष्कृताम्}
{गदां परामृशद्धीमान्धार्तराष्ट्रो महाबलः}


\twolineshloka
{गदाहस्तं तु तं दृष्ट्वा सशृङ्गमिव पर्वतम्}
{प्रजानामिव सङ्क्रुद्धं शूलपाणिमिव स्थितम्}


% Check verse!
*सगदो भारतो भाति प्रतपन्भास्करो यथा*
\twolineshloka
{तमुत्तीर्णं महाबाहुं गदाहस्तमरिन्दमम्}
{मेनिरे सर्वभूतानि दण्डपाणिमिवान्तकम्}


\twolineshloka
{वज्रहस्तं यथा शक्रं शूलहस्तं यथा हरम्}
{ददृशुः शर्वपाञ्चालाः पुत्रं तव जनाधिप}


\twolineshloka
{तमुत्तीर्णं तु सम्प्रेक्ष्य समहृष्यन्त सर्वशः}
{पाञ्चालाः पाण्डवेयाश्च तेऽन्योन्यस्य तलान्ददुः}


\twolineshloka
{अवहासं तु तं मत्वा पुत्रो दुर्योधनस्तव}
{उद्वृत्य नयने क्रुद्धो दिधक्षुरिव पाण्डवान्}


\threelineshloka
{त्रिशिखां भ्रुकुटीं कृत्वा सन्दष्टदशनच्छदः}
{प्रत्युवाच ततस्तान्वै पाण्डवान्सह केशवान् ॥दुर्योधन उवाच}
{}


\twolineshloka
{अस्यावहासस्य फलं प्रतिमोक्ष्यथ पाण्डवाः}
{गमिष्यथ हताः सद्यः सपाञ्चाला यमक्षयम्}


\twolineshloka
{उत्थिन्तश्च जलात्तस्मात्पुत्रो दुर्योधनस्तव}
{अतिष्ठत गदापाणी रुधिरेण समुक्षितः}


\twolineshloka
{तस्य शोणितदिग्धस्य सलिलेन समुक्षितम्}
{शरीरं स्म तदा भाति स्रवन्निव महीधारः}


\twolineshloka
{तमुद्यतगदं वीरं मेनिरे तत्र पाण्डवाः}
{वैवस्वतमिव क्रुद्धं शूलपाणिमिव स्थितम्}


\threelineshloka
{स मेघनिनदो हर्षान्नर्दन्निव च गोवृषः}
{आजुहाव ततः पार्थान्गदया युधि वीर्यवान् ॥दुर्योधन उवाच}
{}


\twolineshloka
{एकैकेन च मां यूयमासीदत युधिष्ठिर}
{न ह्येको बहुभिर्न्याय्यो वीरो योधयितुं युधि}


\twolineshloka
{न्यस्तवर्मा विशेषेण श्रान्तश्चाप्सु परिप्लुतः}
{भृशं विक्षतगात्रश्च हतवाहनसैनिकः}


\threelineshloka
{[अवश्यमेव योद्वव्यं सर्वैरेव मया सह}
{युक्तं त्वयुक्तमित्येतद्वेत्सि त्वं चैव सर्वदा] ॥युधिष्ठिर उवाच}
{}


\twolineshloka
{मा भूदियं तव प्रज्ञा कथमेकं सुयोधन}
{यदाऽभिमन्युं बहवो जघ्नुर्युधि महारथाः}


\twolineshloka
{[क्षत्रधर्मं भृशं क्रूरं निरपेक्षं सुनिर्घृणम्}
{अन्यथा तु कथं हन्युरभिमन्युं तथागतम्}


\twolineshloka
{सर्वे भवन्तो धर्मज्ञाः सर्वे शूरास्तनुत्यजः}
{न्यायेन युध्यतां प्रोक्ता शक्रलोकगतिः परा}


\twolineshloka
{यद्येकस्तु न हन्तव्यो बहुभिर्धर्म एव तु}
{तदाऽभिमन्युं बहवो निजघ्नुस्त्वन्मते कथम्}


\twolineshloka
{सर्वो विमृशते जन्तुः कृच्छ्रस्थो धर्मदर्शनम्}
{पदस्थः पिहितं द्वारं परलोकस्य पश्यति ॥]}


\twolineshloka
{आमुञ्च कवचं वीर मूर्धजान्यमयस्व च}
{यच्चान्यदपि ते नास्ति तदप्यादत्स्व भारत}


\twolineshloka
{इममेकं च ते कामं वीर भूयो ददाम्यहम्}
{पञ्चानां पाण़्डवेयानां येन त्वं योद्धुमिच्छसि}


\threelineshloka
{तं हत्वा वै भवाराजा हतो वा स्वर्गमाप्नुहि}
{ऋते च जीविताद्वीर युद्धे किं कुर्म ते प्रियम् ॥सञ्जय उवाच}
{}


\twolineshloka
{ततस्तव सुतो राजन्वर्म जग्राह काञ्चनम्}
{विचित्रं च शिरस्त्राणं जाम्बूनदपरिष्कृतम्}


\twolineshloka
{सोऽवबद्वशिरस्त्राणः शुभकाञ्चनवर्मभृत्}
{रराज राजन्पुत्रस्ते काञ्चनः शैलराडिव}


\twolineshloka
{सन्नद्धः सगतो राजन्सज्जः सङ्ग्राममूर्धनि}
{अब्रवीत्पाण़्डवान्सर्वान्पुत्रो दुर्योधनस्तव}


\twolineshloka
{भ्रातॄणां भवतामेको युध्यतां गदया मया}
{सहदेवेन वा योत्स्ये भीमेन नकुलेन वा}


\twolineshloka
{अथवा फल्गुनेनाद्य त्वया वा भरतर्षभ}
{योत्स्येऽहं सङ्गरं प्राप्य विजेष्ये च रणाजिरे}


\twolineshloka
{अहमद्य गमिष्यामि वैरस्यान्तं सुदुर्गमम्}
{गदया पुरुषव्याघ्र हेमपट्टनिबद्धया}


\twolineshloka
{गदायुद्धे न मे कश्चित्सदृशोऽस्तीति चिन्तये}
{गदया वो हनिष्यामि सर्वानेव समागतान्}


\threelineshloka
{न मे समर्थाः सर्वे वै योद्धुं न्यायेन केचन}
{न युक्तमात्मना वक्तुमेवं गर्वोद्धतं वचः}
{अथवा सफलं ह्येतत्करिष्ये भवतां पुरः}


\twolineshloka
{अस्मिन्महूर्ते सत्यं वा मिथ्या वै तद्भविष्यति}
{]गृह्णातु च गदां यो वै योत्स्यतेऽद्य मया सह}


\chapter{अध्यायः ३४}
\twolineshloka
{सञ्जय उवाच}
{}


\twolineshloka
{एवं दुर्योधने राजन्गर्जमाने मुहुर्मुहुः}
{युधिष्ठिरस्य सङ्क्रुद्धो वासुदेवोऽब्रवीदिदम्}


\twolineshloka
{यदी नाम ह्ययं यूद्धे वरयेत्त्वां युधिष्ठिर}
{अर्जुनं नकुलं चैव सहदेवमथापि वा}


\twolineshloka
{किमिदं साहसं राजंस्त्वया व्याहृतमीदृशम्}
{एकमेव निहत्याजौ भव राजा कुरुष्विति}


\twolineshloka
{एतेन हि कृता योग्या वर्षाणीह त्रयोदश}
{आयसे पुरुषे राजन्भीमसेनजिघांसया}


\twolineshloka
{कथं नाम भवेत्कार्यमस्माद्वि भरतर्षभ}
{साहसं कृतवांस्त्वं तु ह्यनुक्रोशान्नृपोत्तम}


\twolineshloka
{नान्यमस्यानुपश्यामि प्रतियोद्धारमाहवे}
{ऋते वृकोदरात्पार्थात्स च नातिकृतश्रमः}


\twolineshloka
{तदिदं द्यूतमारब्धं पुनरेव यथा पुरा}
{विषमं शकुनेश्चैव तव चैव विशेषतः}


\twolineshloka
{बली भीमः समर्थश्च कृती राजा सुयोधनः}
{बलवान्वा कृती वेति कृती राजन्विशिष्यते}


\twolineshloka
{सोऽयं राजंस्त्वया शत्रुः समे पथि निवेशितः}
{न्यस्तश्चात्मा सुविषमे कृच्छ्रमापादिता वयम्}


\threelineshloka
{कोऽनु सर्वान्विनिर्जित्य शत्रूनेकेन वैरिणा}
{[कृच्छ्रप्राप्तेन च तथा हारयेद्राज्यमागतम्}
{]पातितश्चैकबाणेन शोचयेदेवमाहवे}


\twolineshloka
{न हि पश्यामि तं लोके योऽद्य दुर्योधनं रणे}
{गदाहस्तं विजेतुं वै शक्तः स्यादमरोपि हि}


\twolineshloka
{फल्गुनो वा भवान्वाथ माद्रीपुत्रावथापि वा}
{न समर्थानहं मन्ये गदाहस्तस्य संयुगे}


\twolineshloka
{न त्वं भीमो न नकुलः सहदेवोऽथ फल्गुनः}
{जेतुं न्यायेन शक्तो वै कृती राजा सुयोधनः}


\twolineshloka
{स कथं वदसे शत्रुं युध्यस्व गदयेति हि}
{एकं च नो निहत्याजौ भव राजेति भारत}


\twolineshloka
{वृकोदरं समासाद्य संशयो वै जये हि नः}
{न्यायतो युध्यमानानां कृती ह्येष महाबलः}


\threelineshloka
{[नूनं न राज्यभागेषा पाण्डोः कुन्त्याश्च सन्ततिः}
{अत्यन्तवनवासाय सृष्टा भैक्ष्याय वा पुनः ॥]भीमसेन उवाच}
{}


\twolineshloka
{मधुसूदन माकार्षीविषादं यदुनन्दन}
{अद्य पारं गमिष्यामि वैरस्य भृशदुर्गमम्}


\twolineshloka
{अहं सुयोधनं सङ्ख्ये हनिष्यामि न संशयः}
{विजयो वै ध्रुवः कृष्ण धर्मराजस्य दृश्यते}


\twolineshloka
{अध्यर्धेन गुणेनेयं गदा गुरुतरी मम}
{न तथा धार्तराष्ट्रस्य माकार्षीर्माधव व्यथाम्}


\twolineshloka
{अनया गदयानाहं संयुगे योद्धुमुत्सहे}
{भवन्तः प्रेक्षकाः सर्वे मम सन्तु जनार्दन}


\threelineshloka
{सामरानपि लोकांस्त्रीन्नानाशस्त्रधरान्युधि}
{योधयेयं रणे कृष्ण किमुताद्य सुयोधनम् ॥सञ्जय उवाच}
{}


\twolineshloka
{तथा सम्भाषमाणं तु वासुदेवो वृकोदरम्}
{हृष्टः सम्पूजयामास वचनं चेदमब्रवीत्}


\twolineshloka
{त्वामाश्रित्य महाबाहो धर्मराजो युधिष्ठिरः}
{निहतारिः स्वकां दीप्तां श्रियं प्राप्नोत्यसंशयम्}


\twolineshloka
{त्वया विनिहताः सर्वे धृतराष्ट्रसुता रणे}
{राजानो राजपुत्राश्च नागाश्च विनिपातिताः}


\twolineshloka
{कलिङ्गा मागधाः प्राच्या गान्धाराः कुरवस्तथा}
{त्वामासाद्य महायुद्धे निहताः पाण्डुनन्दन}


\twolineshloka
{हत्वा दुर्योधनं चापि प्रयच्छोर्वीं ससागराम्}
{धर्मराजाय कौन्तेय यथा विष्णुः शचीपतेः}


\twolineshloka
{त्वां च प्राप्य रणे पापो धार्तराष्ट्रो विनङ्क्ष्यति}
{त्वमस्य सक्थिनी भंक्त्वा प्रतिज्ञां पालयिष्यसि}


\twolineshloka
{यत्नेन ह त्वया पापो योद्धव्यो धृतराष्ट्रजः}
{कृती च बलवांश्चैव युद्धशौण्डश्च नित्यदा}


\twolineshloka
{ततस्तु सात्यकी राजन्पूजयामास पाण्डवम्}
{विविधाभिश्च तं वाग्भिर्भिमसेनं जनेश्वर}


\twolineshloka
{पाञ्चालाः पाण्डवेयाश्च धर्मराजपुरोगमाः}
{तद्वचो भीमसेनस्य सर्व एवाभ्यपूजयन्}


\twolineshloka
{ततो भीमबलो भीमो युधिष्ठिरमथाब्रवीत्}
{सृञ्जयैः सह तिष्ठन्तं तपन्तमिव भास्करम्}


\twolineshloka
{अहमेतेन सङ्गम्य संयुगे योद्धुमुत्सहे}
{न हि शक्तो रणे जेतुं मामेष पुरुषाधमः}


\twolineshloka
{अद्य क्रोधं विमोक्ष्यामि निहितं हृदये भृशम्}
{सुयोधने धार्तराष्ट्रे खाण्डवेऽग्निमिवार्जुनः}


\twolineshloka
{शल्यमद्योद्धरिष्यामि तव पाण्डव हृच्छयम्}
{निहते गदया पापे अद्य राजन्सुखी भव}


\twolineshloka
{अद्य कीर्तिमयीं मालां प्रितमोक्ष्ये तवानघ}
{प्राणाञ्श्रियं च राज्यं च मोक्ष्यतेऽद्य सुयोधनः}


\twolineshloka
{राजा च धृतराष्ट्रोऽद्य श्रुत्वा पुत्रं मया हतम्}
{स्मरिष्यत्यशुभं कर्म यत्तच्छकुनिबुद्धिजम्}


\twolineshloka
{इत्युक्त्वा भरतश्रेष्ठो गदामुद्यम्य वीर्यवान्}
{उदतिष्ठत युद्धाय शक्रो वृत्रमिवाह्वयन्}


\twolineshloka
{[तदाह्वानममृष्यन्वै तव पुत्रोऽतिवीर्यवान्}
{प्रत्युपस्थित एवाशु मत्तो मत्तमिव द्विपम्}


\twolineshloka
{गदाहस्तं तव सुतं युद्धाय समुपस्थितम्}
{ददृशुः पाण्डवाः सर्वे कैलासमिव शृङ्गिणम्]}


\twolineshloka
{तमेकाकिनमासाद्य धार्तराष्ट्रं महाबलम्}
{वियूथमिव मातङ्गं समहृष्यन्त पाण्डवाः}


\twolineshloka
{[न सम्भ्रमो न च भयं न च ग्लानिर्न च व्यथा}
{आसीद्दुर्योधनस्यापि स्थितः सिंह इवाहवे]}


\twolineshloka
{समुद्यतगदं दृष्ट्वा कैलासमिव शृङ्गियाम्}
{भीमसेनस्तदा राजन्दुर्योधनमथाब्रवीत्}


\threelineshloka
{राज्ञापि धृतराष्ट्रेण त्वया चास्मासु यत्कृतम्}
{स्मर तद्दुष्कृतं कर्मं यद्भूतं वारणावते}
{}


\twolineshloka
{द्रौपदी च परामृष्टा समामध्ये रजस्वला}
{द्यूते च वञ्चितो राजा शकूनेर्बुद्धिलाघवात्}


\twolineshloka
{यानि चान्यानि दुष्टात्मन्पापानि कृतवानसि}
{अनागासु च पार्थेषु तस्य पश्य महत्फलम्}


\twolineshloka
{त्वत्कृते निहतः शेते शरतल्पे महायशाः}
{याङ्गेयो भरतश्रेष्ठः सर्वेषां नः पितामहः}


\twolineshloka
{हतो द्रोणश्च कर्णश्च इतः शल्यः प्रतापवान्}
{वैराग्नेरादिकर्ता च शकुनिर्निहतो रणे}


\twolineshloka
{भ्रातरस्ते हताः शूराः पुत्राश्च सहसैनिकाः}
{राजानश्च हताः शूराः समरेष्वनिवर्तिनः}


\twolineshloka
{एते चान्ये च निहता बहवः क्षत्रियर्षभाः}
{प्रातिकामी तथा पापो द्रौपद्याः क्लेशकृद्धतः}


\twolineshloka
{अवशिष्टस्त्वमेवैकः कुलघ्नोऽधमपूरुषः}
{त्वामप्यद्य हनिष्यामि गदया नात्र संशयः}


\threelineshloka
{अद्य तेऽहं रणे दर्पं सर्वं नाशयिता नृप}
{राज्याशां विपुलां चापि पाण्डवेषु च दुष्कृतम् ॥दुर्योधन उवाच}
{}


\twolineshloka
{किं कत्थितेन बहुना युध्यस्वाद्य मया सह}
{अद्य तेऽहं विनेष्यामि युद्धश्रद्धां वृकोदर}


\twolineshloka
{किं न पश्यसि मां पाप गदायुद्धे व्यवस्थितम्}
{हिमवच्छिखराकारां प्रगृह्य महतीं गदाम्}


\twolineshloka
{गदिनं कोऽद्य मां पाप हन्तुमुत्सहते रिपुः}
{न्यायतो युध्यमानश्च देवेष्वपि पुरन्दरः}


\twolineshloka
{मा वृथा गर्ज कौन्तेय शारदाभ्रमिवाजलम्}
{दर्शयस्व बलं युद्धे यावन्नासून्निहन्मि ते}


\twolineshloka
{तस्य तद्वचनं श्रुत्वा पाण्डवाः सहसृञ्जयाः}
{सर्वे सम्पूजयामासुस्तद्वचो विजिगीषवः}


\twolineshloka
{उन्मत्तमिव मातङ्गं तलशब्देन मानवाः}
{भूयः संहर्षयामासू राजन्दुर्योधनं द्विषः}


\twolineshloka
{बृंहन्ति कुञ्जरास्तत्र हया हेषन्ति चासकृत्}
{शस्त्राणि सम्प्रदीप्यन्ते पाण्डवानां जयैषिणाम्}


\chapter{अध्यायः ३५}
\twolineshloka
{सञ्जय उवाच}
{}


\twolineshloka
{xxxxxxxxxxxसुसंवृत्ते सुदारुणे}
{xxxxxxxxxxx पाण्डवेषु महात्मसु}


\twolineshloka
{ततस्तालध्वजके रामस्तयोर्युद्ध उपस्थिते}
{श्रुत्वा तच्छिष्ययो राजन्नाजगाम हलायुधः}


\twolineshloka
{तं दृष्ट्वा परमप्रीताः पाण्डवाः सहकेशवाः}
{उपगम्योपसङ्गृह्य विधिवत्प्रत्यपूजयन्}


\twolineshloka
{पूजयित्वा ततः पश्चादिदं वचनमब्रुवन्}
{शिष्ययोः कौशलं युद्धे पश्य रामेति पार्थिव}


\threelineshloka
{अब्रवीच्च तदा रामो दृष्ट्वा कृष्णं सपाण्डवम्}
{दुर्योधनं च कौरव्यं गदापाणिमवस्थितम् ॥राम उवाच}
{}


\threelineshloka
{चत्वारिंशदहान्यद्य द्वे च मे निःसृतस्य वै}
{पुष्येण सम्प्रयातोऽस्मि श्रवणे पुनरागतः}
{शिष्ययोर्वै गदायुद्धं द्रुष्टुकामोऽस्मि माधव}


\twolineshloka
{ततो युधिष्ठिरो राजा परिष्वज्य हलायुधम्}
{स्वागतं कुशलं चास्मै पर्यपृच्छद्यथातथम्}


\twolineshloka
{कृष्णौ चापि महेष्वासावभिवाद्य हलायुधम्}
{सस्वजाते परिप्रीतौ प्रीयमाणौ यशस्विनौ}


\twolineshloka
{माद्रीपुत्रौ तथा शूरौ द्रौपद्याः पञ्च चात्मजाः}
{अभिवाद्य स्थिता राजन्रौहिणेयं महाबलम्}


\twolineshloka
{भीमसेनोऽथ बलवान्पुत्रस्तव जनाधिप}
{तथैव चोद्यतगदौ पूजयामासतुर्बलम्}


\twolineshloka
{स्वागतेन च ते तत्र प्रतिपूज्य पुनःपुनः}
{पश्य युद्धं महाबाहो इति ते राममब्रुवन्}


% Check verse!
एवमुचुर्महात्मानं रौहिणेयं नराधिपाः
\threelineshloka
{परिष्वज्य तदा रामः पाण्डवान्सह सृञ्जयान्}
{अपृच्छत्कुशलं सर्वान्पार्थिवांश्चामितौजसः}
{तथैव ते समासाद्य पप्रच्छुस्तमनामयम्}


\twolineshloka
{प्रत्यभ्यर्च्य हली सर्वान्क्षत्रियांश्च महात्मनः}
{कृत्वा कुशलसम्प्रश्नं संविदं च यथावयः}


\twolineshloka
{जनार्दनं सात्यकिं च प्रेम्णा सम्परिषस्वजे}
{मूर्ध्नि चैतावुपाघ्राय कुशलं पर्यपृच्छत}


\twolineshloka
{तौ च तं विधिवद्राजन्पूजयामासतुर्गुरुम्}
{ब्रह्माणामिव देवेशमिन्द्रोपेन्द्रौ मुदान्वितौ}


\twolineshloka
{ततोऽब्रवीद्धर्मसुतो रौहिणेयमरिन्दमम्}
{इदं भ्रात्रोर्महायुद्धं पश्य रामेति भारत}


\twolineshloka
{तेषां मध्ये महाबाहुः श्रीमान्केशवपूर्वजः}
{न्यविशत्परमप्रीतः पूज्यमानो महारथैः}


\twolineshloka
{स बभौ राजमध्यस्थो नीलवासाः सितप्रभः}
{दिवीव नक्षत्रगणैः परिवीतो निशाकरः}


\twolineshloka
{ततस्तयोः सन्निपातस्तुमुलो रोमहर्षणः}
{आसीदन्तकरो राजन्वैरस्यान्तं विधित्सतोः}


\chapter{अध्यायः ३६}
\twolineshloka
{जनमेजय उवाच}
{}


\twolineshloka
{पूर्वमेव यदा रामस्तस्मिन्युद्ध उपस्थिते}
{आमन्त्र्य केशवं यातो वृष्णिभिः सहितः प्रभुः}


\twolineshloka
{साहाय्यं धार्तराष्ट्रस्य न च कर्ताऽस्मि केशव}
{न चैव पाण्डुपुत्राणां गमिष्यामि यथागतम्}


\twolineshloka
{एवमुक्त्वा तदा रामो यातः शत्रुनिबर्हणः}
{तस्य चागमनं भूयो ब्रह्मञ्शं सितुमर्हसि}


\threelineshloka
{आख्याहि मे विस्तरशः कथं राम उपस्थितः}
{कथं च दृष्टवान्युद्धं कुशलो ह्यसि सत्तम ॥वैशम्पायन उवाच}
{}


\twolineshloka
{उपप्लाव्ये निविष्टेषु पाण्डवेषु महात्मसु}
{प्रेषितो धृतराष्ट्रस्य समीपं मधुसूदनः}


\twolineshloka
{शमं प्रति महाबाहो हितार्थं सर्वदेहिनाम् ॥स गत्वा हास्तिनपुरं धृतराष्ट्रं समेत्य च}
{}


% Check verse!
उक्तवान्वचनं पथ्यं हितं चैव विशेषतः
% Check verse!
न च तत्कृतवान्राजा यथाऽख्यातं हि तत्पुरा
\twolineshloka
{अनवाप्य शमं तत्र कृष्णः पुरुषसत्तमः}
{आगच्छत महाबाहुरुपप्लाव्यं जनाधिप}


\threelineshloka
{ततः प्रत्यागतः कृष्णो धार्तराष्ट्रविसर्जितः}
{आजगाम नरव्याघ्रः पाण्डवानामनीकिनीम्}
{यथोक्तं च यथावृत्तं गत्वा पाण्डवमब्रवीत्}


\twolineshloka
{न कुर्वन्ति वचो मह्यं कुरवः कालनोदिताः}
{निर्गच्छध्व पाण्डवेयाः पुष्येण सहिता मया}


\twolineshloka
{ततो विभज्यमानेषु बलेषु बलिनां वरः}
{प्रोवाच भ्रातरं कृष्णं रौहिणेयो महामनाः}


\twolineshloka
{तेषामपि महाबाहो साहाय्यं मधुसूदन}
{क्रियतामिति तत्कृष्णो नास्य चक्रे वचस्तदा}


\threelineshloka
{ततो मन्युपरीतात्मा जगाम यदुनन्दनः}
{तीर्थयात्रां हलधरः सरस्वत्यां महायशाः}
{मैत्रनक्षत्रयोगे स्म सहितः सर्वयादवैः}


\twolineshloka
{आश्रयामास भोजस्तु दुर्योधनमरिन्दमः}
{युयुधानेन सहितो वासुदेवस्तु पाण्डवान्}


\twolineshloka
{रौहिणेये गते शूरे पुष्येण मधुसूदनः}
{पाण्डवेयान्पुरस्कृत्य ययावभिमुखः कूरून्}


\twolineshloka
{गच्छन्नेव पथिस्थस्तु रामः प्रेष्यानुवाच ह}
{सम्भारांस्तीर्थयात्रायां सर्वोपकरणानि च}


\twolineshloka
{आनयध्वं द्वारकाया अग्नीन्वै याजकांस्तथा}
{सुवर्णरजनं चैव धैनूर्वासांसि वाजिनः}


\twolineshloka
{कुञ्जरांश्च रथांश्चैव खरोष्ट्रं वाहनानि च}
{क्षिप्रमानीयतां सर्वं तीर्थहेतोः परिच्छदम्}


\twolineshloka
{प्रतिस्रोतः सरस्वत्या गच्छध्वं शीघ्रगामिनः}
{ऋत्विजश्चानयध्वं वै शतशश्च द्विजर्षभान्}


\twolineshloka
{एवं सन्दिश्य तु प्रेष्यान्बलदेवो महाबलः}
{तीर्थयात्रां ययौ राजन्कुरूणां वैशसे तदा}


\twolineshloka
{सरस्वत्याः प्रतिस्रोतः समुद्रादभिजग्मिवान}
{रामो यदूत्तमः श्रीमांस्तीर्थयात्रामनुस्मरन्}


\threelineshloka
{ऋत्विग्भिश्च सुहृद्भिश्च यथाऽन्यैर्द्विजसत्तमैः}
{रथैर्गजैस्तथाऽश्वैश्च प्रेष्यैश्च भरतर्षभ}
{गोखरोष्ट्रप्रयुक्तैश्च यानैश्च बहुभिर्वृतः}


\twolineshloka
{श्रान्तानां क्लान्तबपुषां शिशूनां विपुलायुषाम्}
{देशेदेशे तु देयानि दानानि विविधानि च}


\twolineshloka
{अर्चायै चार्थिनां राजन्क्लृप्तानि बहुशस्तथा}
{तानि यानीह देशेषु प्रतीक्षन्ति स्म भारत}


% Check verse!
बुभुक्षितानामर्थाय क्लृप्तमन्नं समन्ततः
\twolineshloka
{योयो यत्र द्विजो भोज्यं भोक्तुं कामयते तदा}
{तस्यतस्य तु तत्रैवमुपजह्रुस्तदा नृष}


\twolineshloka
{तत्रतत्र स्थिता राजन्रौहिणेयस्य शासनात्}
{भक्ष्यपेयस्य कुर्वन्ति राशींस्तत्र समन्ततः}


\twolineshloka
{वासांसि च महार्हाणि पर्यङ्कास्तरणानि च}
{पूजार्थं तत्र क्लृप्तानि विप्राणां सुखमिच्छताम्}


\twolineshloka
{यत्र यः स्वपते विप्रो यो वा जागर्ति भारत}
{तत्रतत्रैव सर्वस्य क्लृप्तं सर्वमपश्यत}


\twolineshloka
{यथासुखं जनः सर्वो याति तिष्ठति वै तदा}
{यातुकामस्य यानानि पानानि तृषितस्य च}


\twolineshloka
{बुभुक्षितस्य चान्नानि स्वादूनि भरतर्षभ}
{उपजह्रुर्नरास्तत्र वस्त्राण्याभरणानि च}


\twolineshloka
{स पन्थाः प्रबभौ राजन्सर्वस्यैव सुखावहः}
{स्वर्गोपमस्तदा वीर नराणां तत्र गच्छताम्}


\threelineshloka
{नित्यप्रमुदितोपेतः स्वादुभक्ष्यो जलान्वितः}
{विपण्यापणपण्यानं नानाजनशतैर्वृतः}
{नानाद्रुमलतोपेतो नानारत्नविभूषितः}


\twolineshloka
{ततो महात्मा नियतो मनस्विनांपुण्येषु तीर्थेषु वसूनि राजन्}
{ददौ द्विजेभ्यः क्रतुदक्षिणाश्चयदुप्रवीरो हलभृत्प्रतीतः}


\twolineshloka
{दोग्ध्रीश्च धेनूश्च सहस्रशो वैसुवाससः काञ्चनबद्धशृङ्गीः}
{हयांश्च नानाविधदेशजातान्यानानि दासांश्च शुभान्द्विजेभ्यः}


\twolineshloka
{रत्नानि मुक्तामणिविद्रुमं चा--प्यग्र्यं सुवर्णं रजतं सुशुन्द्वम्}
{अयस्मयं ताम्रमयं च भाण्डंददौ द्विजातिप्रवरेषु रामः}


\threelineshloka
{एवं स वित्तं प्रददौ महात्मासरस्वतीतीर्थगतेषु भूरि}
{ययौ क्रमेणाप्रतिमप्रभाव--स्ततः कुरुक्षेत्रमुदारवृत्तः ॥जनमेजय उवाच}
{}


\twolineshloka
{सारस्वतानां तीर्थानां गुणोत्पत्तिं वदस्व मे}
{फलं च द्विपदां श्रेष्ठ कर्म निर्वृत्तिमेव च}


\threelineshloka
{यथाक्रमेण भगवंस्तीर्थानामनुपूर्वशः}
{ब्रूहि ब्रह्मविदां श्रेष्ठ परं कौतूहलं हि मे ॥वैशम्पायन उवाच}
{}


\twolineshloka
{तीर्थानां च फलं राजन्गुणोत्पत्तिं चं सर्वशः}
{मयोच्यमानं वै पुण्यं शृणु राजेन्द्र कृत्स्नशः}


\twolineshloka
{पूर्वं महाराज यदुप्रवीरऋत्विक्सुहृद्विप्रगणैश्च सार्धम्}
{पुण्यं प्रभासं समुपाजगामयत्रोडुराढ्यक्ष्मणा क्षीयमाणः}


\twolineshloka
{विमुक्तशापः पुनराप्य तेजःसर्वं जगद्भासयते नरेन्द्र}
{एतत्तु तीर्थप्रवरं पृथिव्यांप्रभासनात्तस्य ततः प्रभासः}


\chapter{अध्यायः ३७}
\twolineshloka
{जनमेजय उवाच}
{}


\twolineshloka
{किमर्थं भगवान्सोमो यक्ष्मणा समगृह्यत}
{कथं च तीर्थप्रवरे तस्मिंश्चन्द्रो न्यमज्जत}


\threelineshloka
{कथमाप्लुत्य तस्मिंस्तु पुनराप्यायितः शशी}
{एतन्मे सर्वमाचक्ष्व विस्तरेण महामुने ॥वैशम्पायन उवाच}
{}


\twolineshloka
{दक्षस्य तनयास्तात प्रादुरासन्विशाम्पते}
{स सप्तविंशतिं कन्या दक्ष सोमाय वै ददौ}


\twolineshloka
{नक्षत्रयोगनिरताः सङ्ख्यानार्थं च ताभवन्}
{पत्न्यो वै तस्य राजेन्द्र सोमस्य शुभकर्मणः}


\twolineshloka
{तास्तु सर्वा विशालाक्ष्यो रूपेणाप्रतिमाऽभवन्}
{अत्यरिच्यत तासां तु रोहिणी रुपसम्पदा}


\twolineshloka
{ततस्तस्यां स भगवान्प्रीतिं चक्रे निशाकरः}
{साऽस्य हृद्या बभूवाथ तस्मात्तां बुभुजे सदा}


\twolineshloka
{ततस्ताः कुपिताः सर्वा नक्षत्राख्या महात्मनः}
{ता गत्वा पितरं प्राहुः प्रजापतिमतन्द्रिताः}


\threelineshloka
{सोमो वसति नास्मासु रोहिणीं भजते सदा}
{ता वयं सहिताः सर्वास्त्वत्सकाशे प्रजेश्वर}
{वत्स्यामो नियताहारास्तपश्चरणतत्पराः}


\twolineshloka
{श्रुत्वा तासां तु वचनं दक्षः सोममथाब्रवीत्}
{समं वर्तस्व सर्वासु मा त्वाऽधर्मो महान्स्पृशेत्}


\twolineshloka
{तास्तु सर्वाऽब्रवीद्दक्षो गच्छध्वं शशिनोन्तिकम्}
{समं वत्स्यति सर्वासु चन्द्रमा मम शासनात्}


\threelineshloka
{विसृष्टास्तास्तथा जग्मुः शीताशोर्भवनं शुभाः}
{तथाऽपि सोमो भगवान्पुनरेव महीपते}
{रोहिण्या सार्धमवसत्प्रीयमाणो मुहुर्मुहुः}


% Check verse!
ततस्ताः सहिताः सर्वा भूयः पितरमब्रुवन्
\twolineshloka
{सोमो वसति नास्मासु नाकरोद्वचनं तव}
{तव शुश्रूषणे युक्ता वत्स्यामो हि तवान्तिके}


\twolineshloka
{तासां तद्वचनं श्रुत्वा दक्षः सोममथाब्रवीत्}
{समं वर्तस्व सर्वासु मा त्वां शप्स्ये विरोचन}


\twolineshloka
{अनादृत्य तु तद्वाक्यं दक्षस्य भगवाञ्शशी}
{रोहिण्या सार्धमवसत्ततस्ताः कुपिताः पुनः}


\twolineshloka
{गत्वा च पितरं प्राहुः प्रणम्य शिरसा तदा}
{सोमो वसति नास्मासु तस्मान्नः शरणं भव}


\twolineshloka
{रोहिण्यामेव भगवान्सदा वसति चन्द्रमाः}
{न त्वद्वचो गणयति नास्मासु स्नेहमिच्छति}


\twolineshloka
{तस्मान्नस्त्राहि सर्वा वै यथा नः सोम आविशेत्}
{तच्छ्रुत्वा भगवान्क्रुद्धो यक्ष्माणं पृथिवीपते}


\twolineshloka
{ससर्ज रोषात्सोमाय स चोडुपतिमाविशत्}
{स यक्ष्मणाऽभिभूतात्माक्षीयताहरहः शशी}


\twolineshloka
{यत्नं चाप्यकरोद्राजन्मोक्षार्थं तस्य यक्ष्मणः}
{इष्ट्वेष्टिभिर्महाराज विविधाभिर्निशाकरः}


\twolineshloka
{न चामुच्यत शापाद्वै क्षयं चैवाधिगच्छति}
{क्षीयमाणे ततः सोमे ओषध्यो न प्रजज्ञिरे}


\twolineshloka
{निरास्वादरसाः सर्वा हतवीर्याश्च सर्वशः}
{ओषधीनां क्षये जाते प्राणिनामपि सङ्क्षयः}


\twolineshloka
{कृशाश्चासन्प्रजाः सर्वाः क्षीयमाणे निशाकरे}
{ततो देवाः समागम्य सोममूचुर्महीपते}


\twolineshloka
{किमिदं भवतो रूपमीदृशं सम्प्रकाशते}
{कारणं ब्रूहि नः सर्वं येनेदं ते महद्भयम्}


\twolineshloka
{श्रुत्वा तु कारणं त्वत्तो विधास्यामस्ततो वयम्}
{एवमुक्तः प्रत्युवाच सर्वांस्ताञ्शशलक्षणः}


\twolineshloka
{शापस्य कारमं चैव यक्ष्माणं च तथाऽऽत्मनः}
{देवास्तस्य वचः श्रुत्वा गत्वा दक्षमथाऽब्रुवन्}


\twolineshloka
{प्रसीद भगवन्सोमे शापोऽयं विनिवर्त्यताम्}
{असौ हि चन्द्रमाः क्षीणः किञ्चिच्छेषो हि लक्ष्यते}


\threelineshloka
{क्षयाच्चैवास्य देवेश प्रजाश्चैव गताः क्षयम्}
{वीरुदोषधयश्चैव बीजानि विविधानि च}
{तथा नरा लोकगुरो प्रसादं कर्तुमर्हसि}


\twolineshloka
{एवमुक्तस्ततो देवान्प्राह वाक्यं प्रजापतिः}
{नैतच्छक्यं मम वचो व्यावर्तयितुमञ्जसा}


\twolineshloka
{हेतुना तु महाभागा निवर्तिष्यति केनचित्}
{समं वर्ततु सर्वासु शशी भार्यासु नित्यशः}


\twolineshloka
{सरस्वत्या वरे तीर्थे निमज्जञ्शशलक्षणः}
{पुनर्वर्धिष्यते देवास्तद्वै सत्यं वचो मम}


\twolineshloka
{मासार्धं च क्षयं सोमो नित्यमेव गमिष्यति}
{मासार्धं तु पुनर्वृद्विं सत्यमेतद्वचो मम}


\twolineshloka
{समुद्रं पश्चिमं गत्वा सरस्वत्यब्धिसङ्गमम्}
{आराधयतु देवेशं ततः कान्तिमवाप्स्यति}


\twolineshloka
{सरस्वतीं ततः सोमः स जगामर्षिशासनात्}
{प्रभासं प्रथमं तीर्थं सरस्वत्या जगाम ह}


\twolineshloka
{अमावास्यां महातेजास्तत्रामज्जन्महाद्युतिः}
{लोकान्प्रभासयामास शीतांशुत्वमवाप च}


\twolineshloka
{देवास्तु सर्वे राजेन्द्र प्रसादं प्राप्य पुष्कलम्}
{सोमेन सहिता भूत्वा दक्षस्य प्रमुखेऽभवन्}


\twolineshloka
{ततः प्रजापतिः सर्वा विससर्जाथ देवताः}
{सोमं च भगवान्प्रीतो भूयो वचनमब्रवीत्}


\twolineshloka
{मावमंस्थाः स्त्रियः पुत्र मा च विप्रान्कदाचन}
{गच्छ युक्तः सदा भूत्वा कुरु वै शासनं मम}


\twolineshloka
{स विसृष्टो महाराज जगामाथ स्वमालयम्}
{प्रजाश्च मुदिता भूत्वा ईजिरे च यथा पुरा}


\twolineshloka
{एवं ते सर्वमाख्यातं यथा शप्तो निशाकरः}
{प्रभासं च यथा तीर्थं तीर्थानां प्रवरं महत्}


\twolineshloka
{अमावास्यां महाराज नित्यशः शशलक्षणः}
{स्नात्वाह्याप्यायते श्रीमान्प्रभासे तीर्थउत्तमे}


\twolineshloka
{अतश्चैतत्प्रजानन्ति प्रभासमिति भूमिप}
{प्रभां हि परमां लेभे तस्मिंश्चामज्ज्य चन्द्रमाः}


\twolineshloka
{`लोकान्प्रभासयामास पुपोष च वपुस्तथा}
{तत्र स्नात्वा हलीरामो दत्वा प्रीतोऽथ दक्षिणाः'}


\twolineshloka
{ततस्तु चमसोद्भेदमभीतस्त्वगमद्बली}
{चमसोद्भेद इत्येवं यं जनाः कथयन्त्युत}


\twolineshloka
{तत्र दत्त्वा च दानानि विशिष्टानि हलायुधः}
{उषित्वा रजनीमेकां स्नात्वा च विधिवत्तदा}


\twolineshloka
{उदपानमथागच्छत्त्वरावान्केशवाग्रजः}
{आद्यं स्वस्त्ययनं चैव तत्रावाप्य महाबलः}


\twolineshloka
{स्निग्धत्वादोषधीनां च भूमेश्च जनमेजय}
{जानन्ति सिद्धा राजेन्द्र निगूढां तां सरस्वतीम्}


\chapter{अध्यायः ३८}
\twolineshloka
{वैशम्पायन उवाच}
{}


\twolineshloka
{तस्मान्नदीगतं चापि ह्युदपानं यशस्विनः}
{त्रितस्य च महाराज जगामाथ हलायुधः}


\twolineshloka
{तत्र दत्त्वा बहुद्रव्यं पूजयित्वा तथा द्विजान्}
{उपस्पृश्य च तत्रैव प्रहृष्टो मुसलायुधः}


\twolineshloka
{तत्र धर्मपरो ह्यासीत्त्रितः स सुमहातपाः}
{कूपे च वसता तेन सोमः पीतो महात्मना}


\threelineshloka
{तत्र चैनं समुत्सृज्य भ्रातरौ जग्मतुर्गृहान्}
{ततस्तौ वै शशापाथ त्रितो ब्राह्मणसत्तमः ॥जनमेजय उवाच}
{}


\twolineshloka
{उदपानं कथं ब्रह्मन्कथं च सुमहातपाः}
{पतितः किं च सन्त्यक्तो भ्रातृभ्यां द्विजसत्तम}


\fourlineindentedshloka
{कूपे कथं च हित्वैनं भ्रातरौ जग्मतुर्गृहान्}
{कथं च याजयामास पपौ सोमं च वै कथम्}
{एतदाचक्ष्व मे ब्रह्मन्श्रोतव्यं यदि मन्यसे ॥वैशम्पायन उवाच}
{}


\twolineshloka
{आसन्पूर्वयुगे राजन्मुनयो भ्रातरस्त्रयः}
{एकतश्च द्वितश्चैव त्रितश्चादित्यसन्निभाः}


\twolineshloka
{सर्वे प्रजापतिसमाः प्रजावन्तस्तथैव च}
{ब्रह्मलोकजिताः सर्वे तपसा ब्रह्मवादिनः}


\twolineshloka
{तेषां तु तपसा प्रीतो नियमेन दमेन च}
{अभवद्गौतमो नित्यं पिता धर्मरतः सदा}


\twolineshloka
{स तु दीर्घेण कालेन तेषां प्रीतिमवाप्य च}
{जगाम भगवान्स्थानमनुरूपमिवात्मनः}


\twolineshloka
{राजानस्तस्य ये ह्यासन्याज्या राजन्महात्मनः}
{ते सर्वे स्वर्गते तस्मिंस्तस्य पुत्रानपूजयन्}


\twolineshloka
{तेषां तु कर्मणा राजंस्तथा चाध्ययनेन च}
{त्रितः स श्रेष्ठतां प्राप यथैवास्य पिता तथा}


\twolineshloka
{तथा सर्वे महाभागा मुनयः पुण्यलक्षणाः}
{अपूजयन्महाभागं यथास्य पितरं तथा}


\twolineshloka
{कदाचिद्वि ततो राजन्भ्रातरावेकतद्वितौ}
{यज्ञार्थं चक्रतुश्चिन्तां तथा वित्तार्थमेव च}


\threelineshloka
{तयोर्बुद्धिः समभवत्त्रितं गृह्य परन्तप}
{याज्यान्सर्वानुपादाय प्रतिगृह्य पशूंस्ततः}
{सोमं पास्यामहे हृष्टाः प्राप्य यज्ञं महाफलम्}


\threelineshloka
{चक्रुश्चैवं तथा राजन्भ्रातरस्त्रय एव च}
{यथा ते तु परिक्रम्य याज्यान्सर्वान्पशून्प्रति}
{याजयित्वा ततो याज्याँल्लब्ध्वा तु सुबहून्पशून्}


\twolineshloka
{याज्येन कर्मणा तेन प्रतिगृह्य विधानतः}
{प्राचीं दिशं महात्मान आजग्मुस्ते महर्षयः}


\twolineshloka
{त्रितस्तेषां महाराज पुरस्ताद्याति हृष्टवत्}
{एकतश्च द्वितश्चैव पृष्ठतः कालयन्पशून्}


\twolineshloka
{तयोश्चिन्ता समभवदेकतस्य द्वितस्य च}
{कथं च स्युरिमा गाव आवाभ्यां हि विना त्रितम्}


\twolineshloka
{तावन्योन्यं समाभाष्य एकतश्च द्वितश्च ह}
{यदूचतुर्मिथः पापौ तन्निबोध जनेश्वर}


\twolineshloka
{त्रितो यज्ञेषु कुशलस्तथा वेदेषु निष्ठितः}
{ततस्त्रितो बहुतरं गावः समुपलप्स्यते}


\threelineshloka
{तदावां सहितौ भूत्वा गाः प्रकाल्य व्रजावहे}
{त्रितोऽपि गच्छतां काममावाभ्यां वै विनाकृताः ॥वैशम्पायन उवाच}
{}


\twolineshloka
{तेषामागच्छतां रात्रौ पथिस्थानां वृकोऽभवत्}
{तत्र कूपो विदूरेऽभूत्सरस्वत्यास्तटे महान्}


\threelineshloka
{अथं त्रितो वृकं दृष्ट्वा पथि तिष्ठन्तमग्रतः}
{तद्भ्यादपसर्पन्वै तस्मिन्कूपे पपात ह}
{अगाधे सुमहाघोरे सर्वभूतभयङ्करे}


\twolineshloka
{त्रितस्ततो महाराज कूपस्थो मुनिसत्तमः}
{आर्तनादं ततश्चक्रे तौ तु शुश्रुवतुर्मुनी}


\twolineshloka
{त ज्ञात्वा पतितं कूपे भ्रातरावेकतद्वितौ}
{वृकत्रासाच्च लोभाच्च समुत्सृज्य प्रजग्मतुः}


\twolineshloka
{भ्रातृभ्यां पशुलुब्धाभ्यामुत्सृष्टः स महातपाः}
{उदपाने तदा राजन्निर्जले पांसुपंवृते}


\twolineshloka
{त्रित आत्मानमालक्ष्य कूपे वीरुत्तृणावृते}
{निमग्नं भरतश्रेष्ठ नरके दुष्कृती यथा}


\twolineshloka
{ततो ह्यचिन्तयत्प्राज्ञो मृतभूतो ह्यसोमपः}
{सोमः कथं तु पातव्य इहस्थेन मया भवेत्}


\twolineshloka
{स एवमभिसञ्चिन्त्य तस्मिन्कूपे महातपाः}
{ददर्श वीरुधं तत्र लम्बमानां यदृच्छया}


\twolineshloka
{पांसुग्रस्ते ततः कूपे न्यखनत्सलिलं मुनिः}
{अग्नीन्सङ्कल्पयामास होतॄनात्मानमेव च}


\twolineshloka
{ततस्तां वीरुधं सोमं सङ्कल्प सुमहातपाः}
{ऋचो जयूंषि सामानि मनसा चिन्तयन्मुनिः}


\twolineshloka
{ग्रावाणः शर्कराः कृत्वा प्रचक्रेऽभिषवं नृप}
{आज्यं च सलिलं चक्रे भागांश्च त्रिदिवौकसाम्}


\twolineshloka
{सोमस्याभिषवस्याग्रे प्रवृत्तस्तुमुलो ध्वनिः}
{स चाविशद्दिवं राजन्स्वरश्चैव त्रितस्त वै}


% Check verse!
समवाप च तं यज्ञां यथोक्तं ब्रह्मवादिभिः
\twolineshloka
{वर्तमाने महायज्ञे त्रितस्य सुमहात्मनः}
{आविग्नं त्रिदिवं सर्वं कारमं च न बुध्यते}


\twolineshloka
{ततः सुतुमुलं शब्दं शुश्रावाथ बृहस्पतिः}
{श्रुत्वा चैवाब्रवीत्सर्वान्देवान्देवपुरोहितः}


\twolineshloka
{त्रितस्य वर्तते यज्ञस्तत्र गच्छामहे सुराः}
{स हि क्रुद्धः सृजेदन्यान्देवानपि महातपाः}


\twolineshloka
{तच्छ्रुत्वा वचनं तस्य सहिताः सर्वदेवताः}
{प्रययुस्तत्र यत्रास्ते त्रितयज्ञश्च वर्तते}


\twolineshloka
{ते तत्र गत्वा विबुधास्तं कूपं यत्र स त्रितः}
{ददृशुस्तं महात्मानं दीक्षितं यज्ञकर्मसु}


\twolineshloka
{दृष्ट्वा चैनं महात्मानं श्रिया परमया युतम्}
{ऊचुश्चैनं महाभागं प्राप्ता भागार्थिनो वयम्}


\twolineshloka
{अथाब्रवीदृषिर्देवान्पश्यध्वं मां दिवौकसः}
{अस्मिन्प्रतिभये कूपे निमग्नं नष्टचेतसम्}


\twolineshloka
{ततस्त्रितो महाराज भागांस्तेषां यथाविधि}
{मन्त्रयुक्तान्समददात्तेन प्रीतास्तदाऽभवन्}


\twolineshloka
{ततो यथाविधि प्राप्तान्भागान्प्राप्य दिवौकसः}
{प्रीतात्मानो ददुस्तस्मै वरान्यान्मनसेच्छति}


\twolineshloka
{स तु वव्रे लवरं देवांस्त्रातुमर्हथ मामितः}
{यश्चाम्भोपस्पृशेत्कूपे स सोमपगतिं लभेत्}


\twolineshloka
{तत्र चोर्मिमती राजन्नुत्पपात सरस्वती}
{तयोत्क्षिप्तस्त्रितस्तस्थौ पूजयंस्त्रिदिवौकसः}


\twolineshloka
{तथेति चोक्त्वा विबुधा जग्मू राजन्यथाऽऽगताः}
{त्रितश्चाभ्यागमत्प्रीतः स्वमेव निलयं तदा}


\twolineshloka
{क्रुद्धस्तु स समासाद्य तावृषी भ्रातरौ तदा}
{उवाच परुषं वाक्यं शशाप च महातपाः}


\twolineshloka
{पशुलुब्धौ युवां यस्मान्मामुत्सृज्य प्रधावितौ}
{तस्माद्वृकाकृती रौद्रौ दंष्ट्रिणावभितश्चरौ}


\twolineshloka
{भवितारौ मया शप्तौ पापेनानेन कर्मणा}
{प्रसवश्चैव युवयोर्गोलाङ्गूलर्क्षवानराः}


\twolineshloka
{इत्युक्तेन तदा तेन क्षणादेव विशाम्पते}
{तथाभूतावदृश्येतां वचनात्सत्यवादिनः}


\twolineshloka
{तत्राप्यमितविक्रान्तः स्पृष्ट्वा तोयं हलायुधः}
{दत्त्वाच विविधान्देयान्पूजयित्वा च वै द्विजान्}


\twolineshloka
{उदपानं च तं वीक्ष्य प्रशस्य च पुनःपुनः}
{नदीगतमदीनात्मा प्राप्तो विनशनं तदा}


\chapter{अध्यायः ३९}
\twolineshloka
{वैशम्पायन उवाच}
{}


\threelineshloka
{ततो विनशनं राजन्नाजगाम हलायुधः}
{शूद्राभीरान्प्रति द्वेषाद्यत्र नष्टा सरस्वती}
{तस्मात्तामृषयो नित्यं प्राहुर्विनशनेति च}


\twolineshloka
{तत्राप्युपस्पृश्य बलः सरस्वत्यां महाबलः}
{सुभूमिकं ततोऽगच्छत्सरस्वत्यास्तटे वरे}


\twolineshloka
{तत्र चाप्सरसः शुभ्रा नित्यकालमतन्द्रिताः}
{क्रीडाभिर्विमलाभिश्च क्रीडन्ति विमलाननाः}


\twolineshloka
{तत्र देवाः सगन्धर्वा मासिमासि जनेश्वर}
{अभिगच्छन्ति तत्तीर्थं पुण्यं ब्राह्मणसेवितम्}


\twolineshloka
{तत्र नृत्यन्ति गन्धर्वास्तथैवाप्सरसां गणाः}
{समेत्य सहिता राजन्यथाप्राप्तं यथासुखम्}


\twolineshloka
{तत्र मोदन्ति देवाश्च पितरश्च सवीरुधः}
{पुण्यैः पुष्पैः सदा दिव्यैः कीर्यमाणाः पुनः पुनः}


\twolineshloka
{आक्रीडभूमिः सा राजंस्तासामप्सरसां शुभा}
{सुभूमिकेति विख्याता सरस्वत्यास्तटे वरे}


\twolineshloka
{तत्र स्नात्वा च दत्त्वा च वसु विप्रेषु माधवः}
{श्रुत्वा गीतं च तद्दिव्यं वादित्राणां च निःस्वनम्}


\twolineshloka
{शय्याश्च विपुला दृष्ट्वा देवगन्धर्वरक्षसाम्}
{गन्धर्वाणां ततस्तीर्थमागच्छद्रोहिणीसुतः}


\twolineshloka
{विश्वावसुमुखास्तत्र गन्धर्वाप्सरसां गणाः}
{नृत्तवादित्रगीतं च कुर्न्वति सुमनोरमम्}


\twolineshloka
{तत्र दत्त्वा हलधरो विप्रेभ्यो विविधं वसु}
{अजाविकं गोखरोष्ट्रं सुवर्णं रजतं तथा}


\twolineshloka
{भोजयित्वा द्विजान्कामैः सन्तर्प्य च महाधनैः}
{प्रययौ सहितो विप्रैः स्तूयमानश्च माधवः}


\twolineshloka
{तस्माद्गन्धर्वतीर्थाच्च महाबाहुररिन्दमः}
{गर्गस्रोतो महातीर्थमाजगामैककुण्डली}


\twolineshloka
{तत्र गर्गेण वृद्धेन तपसा भावितात्मना}
{कालज्ञानगतिश्चैव ज्योतिषां च व्यतिक्रमः}


\twolineshloka
{उत्पाता दारुणाश्चैक शुभाश्च जनमेजय}
{सरस्वत्याः शुभे तीर्थे विदिता वै महात्मना}


% Check verse!
तस्य नाम्ना च तत्तीर्थं गर्गस्रोत इति स्मृतम्
\twolineshloka
{तत्र गर्गं महाभागमृषयः सुव्रता नृप}
{उपासाञ्चक्रिरे नित्यं कालज्ञानं प्रति प्रभो}


\twolineshloka
{तत्र गत्वा महाराज बलः श्वेतानुलेपनः}
{विधिवद्वि धनं दत्त्वा मुनीनां भावितात्मनाम्}


\twolineshloka
{उच्चावचांस्तथा भक्ष्यान्विप्रेभ्यो विप्रदाय सः}
{नीलवासास्तदाऽगच्छच्छङ्घतीर्थं महायशाः}


\threelineshloka
{तत्रापश्यन्महाशङ्खं महामेरुमिवोच्छ्रितम्}
{श्वेतपर्वतसङ्काशमृषिसङ्घैर्निषेवितम्}
{सरस्वत्यास्तटे जातं नगं तालध्वजो बली}


\twolineshloka
{यक्षा विद्याधराश्चैव राक्षसाश्चामितौजसः}
{पिशाचाश्चामितबला यत्र सिद्धाः सहस्रशः}


\twolineshloka
{ते सर्वे ह्यशनं त्यक्त्वा फलं तस्य वनस्पतेः}
{व्रतैश्च नियमैश्चैव कालेकाले स्म भुञ्जते}


\twolineshloka
{प्राप्तैश्च नियमैस्तैस्तैर्विचरन्तः पृथक्पृथक्}
{अदृश्यमाना मनुजैर्व्यचरन्पुरुषर्षभ}


\twolineshloka
{एवं ख्यातो नरव्याघ्र लोकेऽस्मिन्स वनस्पतिः}
{तत्र तीर्थं सरस्वत्याः पावनं लोकविश्रुतम्}


\twolineshloka
{तस्मिंश्च यदुशार्दूलो दत्त्वा तीर्थे पयस्विनीः}
{ताम्रायसानि भाण्डानि वस्त्राणि विविधानि च}


\twolineshloka
{पूजयित्वा द्विजांश्चैव पूजितश्च तपोधनैः}
{पुण्यं नैसर्गिकं राजन्नाजगाम हलायुधः}


\twolineshloka
{तत्र गत्वा मुनीन्दृष्ट्वा नानावेषधरान्बलः}
{आप्लुत्य सलिले चापि पूजयामास वै द्विजान्}


\twolineshloka
{तथैव दत्त्वा विप्रेभ्यः परिभोगान्सुपुष्कलान्}
{ततः प्रायाद्बलो राजन्दक्षिणेन सरस्वतीम्}


\twolineshloka
{गत्वा चैवं महाबाहुर्नातिदूरे महायशाः}
{धर्मात्मा नागधन्वानं तीर्थमागमदच्युतः}


\twolineshloka
{यत्र पन्नगराजस्य वासुकेः सन्निवेशनम्}
{महाद्युतेर्महाराज बहुभिः पन्नगैर्वृतम्}


\threelineshloka
{ऋषिणां हि सहस्राणि तत्र नित्यं चतुर्दश}
{यत्र देवाः समागम्य वासुकिं पन्नगोत्तमम्}
{सर्वपन्नगराजानमभ्यषिञ्चन्यथाविधि}


% Check verse!
पन्नगेभ्यो भयं तत्र विद्यते न स्म पौरव
\twolineshloka
{तत्रापि विधिवद्दत्वा विप्रेभ्यो रत्नसञ्चयान्}
{प्रायात्प्राचीं दिशं राजंस्तत्र तीर्थान्यनेकशः}


\twolineshloka
{सहस्रशतसङ्ख्यानि प्रथितानि पदेपदे}
{आप्लुत्य तत्र तीर्थेषु यथोक्तं तत्र चर्षिभिः}


\twolineshloka
{दत्त्वा वसु द्विजाग्र्येभ्यो निर्जगाम महाबलः}
{तत्रस्थानृषिसङ्घांस्तानभिवाद्य हलायुधः}


\twolineshloka
{ततो रामोऽगमत्तीर्थमृषिभिः सेक्तिं महत्}
{यत्र भूयो निववृते प्राङ्मुखा वै सरस्वती}


\fourlineindentedshloka
{ऋषीणां नैमिषेयाणावमेक्षार्थं महात्मनाम्}
{निवृत्तां तां सरिच्छ्रेष्ठां तत्र दृष्ट्वा तु लाङ्गली}
{वभूव विस्मितो राजन्बलः श्वेतानुलेपनः ॥जनमेजय उवाच}
{}


\twolineshloka
{कस्मात्सरस्वती ब्रह्मन्निवृत्ता कप्राङ्मुखी भवत्}
{व्याख्यातमेतदिच्छामि सर्वमध्वर्युसत्तम}


\threelineshloka
{कस्मिंश्चित्कारणे तत्र विस्मितो यदुनन्दनः}
{निवृत्ता हेतुना केन कथमेव सरिद्वरा ॥वैशम्पायन उवाच}
{}


\twolineshloka
{पूर्वं कृतयुगे राजन्नैमिषेयास्तपस्विनः}
{वर्तमाने सुविपुले सत्रे द्वादशवार्षिके}


\twolineshloka
{ऋषयो बहवो राजंस्तत्सत्रमभिपेदिरे}
{उषित्वा च महाभागास्तस्मिन्सत्रे यथाविधि}


\twolineshloka
{निवृत्ते नैमिषे ये वै सत्रे द्वादशवार्षिके}
{आजग्मुर्ऋषयस्तत्र बहवस्तीर्थकारणात्}


\twolineshloka
{ऋषीणां बहुलत्वात्तु सरस्वत्या विशाम्पते}
{तीर्थानि नगरायन्ते कूले वै दक्षिणोत्तरे}


\twolineshloka
{समन्तपञ्चकं यावत्तावत्ते द्विजसत्तमाः}
{तीर्थलोभान्नरव्याघ्र नद्यास्तीरं समाश्रिताः}


\twolineshloka
{जुह्वतां तत्र तेषां तु मुनीनां भावितात्मनाम्}
{स्वाध्यायेनातिमहता बभूवुः पूरिता दिशः}


\twolineshloka
{अग्निहोत्रैस्ततस्तेषां हूयमानैर्महात्मनाम्}
{अशोभत सरिच्छ्रेष्ठा दीप्यमानैः समन्ततः}


\twolineshloka
{वालखिल्या महाराज अश्मकुट्टाश्च तापसाः}
{दन्तोलूखलिनश्चान्ये सम्प्रक्षालास्तथा परे}


\twolineshloka
{वायुभक्षा जलाहाराः पर्णभक्षाश्च तापसाः}
{नानानियमयुक्ताश्च तथा स्थण्डिलशायिनः}


\twolineshloka
{आसन्वै मुनयस्तवत्र सरस्वत्याः समीपतः}
{शोभयन्तः सरिच्छ्रेष्ठां गङ्गामिव दिवौकसः}


\threelineshloka
{शतशश्च समापेतुर्ऋषयः सत्रयाजिनः}
{तेऽवकाशं न ददृशुः सरस्वत्या महाव्रताः}
{तेऽवकाशं च ददृशुः कुरुक्षेत्रं (त्रे) महाव्रताः}


\twolineshloka
{ततो यज्ञोपवीतैः स्वैस्तत्र कृत्वा सरस्वतीम्}
{जुहुवुश्चाग्निहोत्रांश्च चक्रुश्च विविधाः क्रियाः}


\twolineshloka
{ततस्तमृपिसङ्घातं निराशं चिन्तयान्वितम्}
{दर्शयामास राजेन्द्र तेषामर्थे सरस्वती}


\twolineshloka
{ततः कुञ्जान्बहून्कृत्वा सन्निवृत्ता सरस्वती}
{ऋषीणां पुण्यतपसां कारुण्याज्जनमेजय}


\twolineshloka
{ततो निवृत्त्य राजेन्द्र तेषामर्थे सरस्वती}
{भूयः प्रतीच्यभिमुखी प्रसुस्राव सरिद्वरा}


\twolineshloka
{अमोघा गमनं कृत्वा तेषां भूयो जगाम ह}
{अत्यद्भुतं महच्चक्रे तदा राजन्महानदी}


\twolineshloka
{एवं स कुञ्जो राजन्वै नैमिषीय इति स्मृतः}
{कुरुश्रेष्ठ कुरुक्षेत्रे कुरुष्व महतीं क्रियाम्}


\twolineshloka
{तत्र कुञ्जान्बहून्दृष्ट्वा निवृत्तां च सरस्वतीम्}
{बभूव विस्मयस्तत्र रामस्याथ महात्मनः}


\twolineshloka
{उपस्पृश्य तु तत्रापि विधिवद्यदुनन्दनः}
{दत्त्वा देयान्द्विजातिभ्यो भाण्डानि विविधानि च}


\twolineshloka
{भक्ष्यं भोज्यं कच विविधं ब्राह्मणेभ्यः प्रदाय च}
{ततः प्रायाद्बलो राजन्पूज्यमानो द्विजातिभिः}


\twolineshloka
{सरस्वतीतीर्थवरं नानाद्विजगणायुतम्}
{बदरेङ्गुदश्यामाकाप्लुक्षाश्वत्थबिभीतकैः}


\twolineshloka
{कङ्कोलैश्च पलाश्चैश्च करीरैः पीलुभिस्तथा}
{सरस्वतीतीर्थरुहैस्तरुभिर्विविधैस्तथा}


\twolineshloka
{करुषकवरैश्चैव बिल्वैराम्रातकैस्तथा}
{अतिमुक्तकषण्डैश्च पारिजातैश्च शोभितम्}


\twolineshloka
{कदलीवनभूयिष्ठं दृष्टिकान्तं मनोहरम्}
{वाय्वम्बुफलपर्णादैर्दन्तोलूखलिकैरपि}


\twolineshloka
{तथाऽश्मकुट्टैर्वातेयैर्मुनिभिर्बहुभिर्वृतम्}
{स्वाध्यायघोषसङ्घुष्टं मृगयूथशताकुलम्}


\threelineshloka
{अहिंस्रैर्धर्मपरमैर्नृभिरत्यर्थरोवितम्}
{सप्तसारस्वतं तीर्थमाजगाम हलायुधः}
{यत्र मङ्कणकः सिद्धस्तपस्तेपे महामुनिः}


\chapter{अध्यायः ४०}
\twolineshloka
{जनमेजय उवाच}
{}


\twolineshloka
{सप्तसारस्वतं कस्मात्कश्च मङ्कणको मुनिः}
{कथं सिद्धः स भगवान्कश्चास्य नियमोऽभवत्}


\threelineshloka
{कस्य वंशे समुत्पन्नः किं चाधीतं द्विजोत्तम}
{एतन्मे सर्वमाचक्ष्व यथातत्त्वं महामुने ॥वैशम्पायन उवाच}
{}


\twolineshloka
{सप्तनद्यः सरस्वत्या याभिर्व्याप्तमिदं जगत्}
{आहूता बलवद्भिर्हि तत्रतत्र सरस्वती}


\twolineshloka
{सुप्रभा काञ्चनाक्षी च विशाला च मनोरमा}
{सरस्वती चौघवती सुरेणुर्विमलोदका}


\twolineshloka
{पितामहस्य महतो वर्तमाने महामखे}
{वितते यज्ञवाटे च संसिद्धेषु द्विजातिषु}


\twolineshloka
{पुण्याहघोषैर्विमलैर्वेदानां निनदैस्तथा}
{देवेषु चैव व्यग्रेषु तस्मिन्यज्ञविधौ तदा}


\twolineshloka
{तत्र चैव महाराज दीक्षिते प्रपितामहे}
{यजतस्तस्य सत्रेण सर्वकामसमृद्विना}


\twolineshloka
{मनसा चिन्तिता ह्यर्था धर्मार्थकुशलैस्तदा}
{उपतिष्ठन्ति राजेन्द्र द्विजातींस्तत्रतत्र ह}


\twolineshloka
{जगुश्च तत्र गन्धर्वा ननृतुश्चाप्सरोगणाः}
{वादित्राणि च दिव्यानि वादयामासुरञ्जसा}


\twolineshloka
{तस्य यज्ञस्य सम्पत्त्या तुतुषुर्देवतागणाः}
{विस्मयं परमं जग्मुः किमु मानुषयोनयः}


\twolineshloka
{वर्तमाने यथा यज्ञे पुष्करस्थे पितामहे}
{अब्रुवन्नृषयो राजन्नायं यज्ञो महागुणः}


\twolineshloka
{न दृश्यते सरिच्छ्रेष्ठा यस्मादिह सरस्वती}
{तच्छ्रुत्वा भगवान्प्रीतः सस्माराथ सरस्वतीम्}


\twolineshloka
{पितामहेन यजता आहूता पुष्करेषु वै}
{सुप्रभा नाम राजेन्द्र नाम्ना तत्र सरस्वती}


\twolineshloka
{तां दृष्ट्वा मुनयस्तुष्टास्त्वरायुक्तां सरस्वतीम्}
{पितामहं मानयन्तीं क्रतुं ते बहुमेनिरे}


\twolineshloka
{एवमेषा सरिच्छ्रेष्ठा पुष्करेषु सरस्वती}
{पितामहार्थं सम्भूता तुष्ट्यर्थं च मनीषिणाम्}


\twolineshloka
{नैमिषे मुनयो राजन्समागम्य समासते}
{तत्र चित्राः कथा ह्यासन्वेदं प्रति जनेश्वर}


\twolineshloka
{यत्र ते मुनयो ह्यासन्नानास्वाध्यायवेदिनः}
{ते समागम्य मुनयः सस्मारुर्वै सरस्वतीम्}


\threelineshloka
{सा तु ध्याता महाराज मुनिभिः सत्रयाजिभिः}
{समागतानां राजेन्द्र साहाय्यार्थं महात्मनाम्}
{आजगाम महाभागा तत्र पुण्या सरस्वती}


\twolineshloka
{नैमिषे काञ्चनाक्षी तु मुनीनां सत्रायाजिxxxम्}
{आगता सरितां श्रेष्ठा तत्र भारत पूजित}


\twolineshloka
{गयस्य यजमानस्य गयेष्वेव महाक्रतुम्}
{आहूता सरितां श्रेष्ठा गययज्ञे सरस्वती}


\twolineshloka
{गयस्य यजमानस्य गयेष्वेव महाक्रतुम्}
{विशालां तु गयस्याहुर्ऋषयः संशितव्रता}


\twolineshloka
{सरित्सा हिमवत्पार्श्वात्प्रस्रुता शीघ्रगामिनी}
{औद्दालकेस्तथा यज्ञे यजतस्तस्य भारत}


\twolineshloka
{समेते सर्वतः स्फीते मुनीनां मण्डले तदा}
{उत्तरे कोसलाभागे पुण्ये राजन्महात्मनः}


\twolineshloka
{औद्दालकेन यजता पूर्वं ध्याता सरस्वती}
{आजगाम सरिच्छेष्ठा तं देशं मुनिकारणात्}


\twolineshloka
{पूज्यमाना मुनिगणैर्वल्कलाजिनसंवृतैः}
{मनोहरेति विख्याता सा हि तैर्मनसा वृता}


\threelineshloka
{[सुरणुऋषभे द्वीपे पुण्ये राजर्षिसेविते}
{]कुरोश्च यजमानस्य कुरुक्षेत्रे महात्मनः}
{आजगाम महाभागा सरिच्छ्रेष्ठा सरस्वती}


\twolineshloka
{ओघवत्यपि राजेन्द्र वसिष्ठेन महात्मना}
{समाहूता कुरुक्षेत्रे दिव्यतोया सरस्वती}


\twolineshloka
{दक्षेण यजता चापि गङ्गाद्वारे सरस्वती}
{सुवेणिरिति विख्याता प्रस्रुता शीघ्रगामिनी}


\twolineshloka
{विमलोदा भगवती ब्रह्मणा यजता पुनः}
{समाहूता ययौ तत्र पुण्ये हैमवते गिरौ}


\twolineshloka
{एकीभूतास्ततस्तास्तु तस्मिंस्तीर्थे समागताः}
{सप्तसारस्वतं तीर्थं ततस्तु प्रथितं भुवि}


\twolineshloka
{इति सप्तसरस्वत्यो नामतः परिकीर्तिताः}
{सप्तसारस्वतं चैव तीर्थं पुण्यं तथा स्मृतम्}


\twolineshloka
{शृणु मङ्कणकस्यापि कौमारब्रह्मचारिणः}
{आपगामवगाढस्य राजन्प्रक्रीडितं महत्}


\threelineshloka
{दृष्ट्वा यदृच्छया तत्र स्त्रियमम्भसि भारत}
{स्नायन्तीं रुचिरापाङ्गीं दिग्वाससमनिन्दिताम्}
{}


\twolineshloka
{सरस्वत्यां महाराज चस्कन्दे वीर्यमम्भसि ॥तद्रेतः स तु जग्राह कलशे वै महातपाः}
{}


\twolineshloka
{`ऋषिः परमधर्मात्मा तदा पुरुषसत्तम' ॥सप्तधा प्रविभागं तु कलशस्थं जगाम ह}
{}


\twolineshloka
{तत्रर्षयः सप्त जाता जज्ञिरे मरुतां गणाः ॥वायुवेगो वायुबलो वायुहा वायुमण्डलः}
{}


\twolineshloka
{वायुज्वालो वायुरेता वायुचक्रश्च वीर्यवान्}
{महर्षेश्चरितं यादृक् त्रिषु लोकेषु विश्रुतम्}


\twolineshloka
{पुरा मङ्कणकः सिद्भः कुशाग्रेणेति नः श्रुतम्}
{क्षतः किल करे राजंस्तस्य शाकरसोऽस्रवत्}


% Check verse!
स वै शाकरसं दृष्ट्वा हर्षाविष्टः प्रनृत्तवान्
\twolineshloka
{ततस्तस्मिन्प्रनृत्ते वै स्थावरं जङ्गमं च यत्}
{प्रनृत्तमुभयं वीर सेजसा तस्य मोहितम्}


\threelineshloka
{ब्रह्मादिभिः सुरै राजन्नृषिभिश्च तपोधनैः}
{विज्ञप्तो वै महादेव ऋषेरर्थे नराधिप}
{नायं नृत्येद्यथा देव तथा त्वं कर्तुमर्हसि}


\twolineshloka
{ततो देवो मुनिं दृष्ट्वा हर्षाविष्टमतीव ह}
{सुगणां हितकामार्थं महादेवोऽभ्यभाषत}


\fourlineindentedshloka
{भोभो ब्राह्मण धर्मज्ञ किमर्थं नृत्यते भवान्}
{हर्षस्थानं किमर्थं च तवेदमधिकं मुने}
{तपस्विनो धर्मपथे स्थितस्य द्विजसत्तम ॥ऋषिरुवाच}
{}


\twolineshloka
{किं न पश्यसि मे ब्रह्मन्कराच्छाकरसं स्रुतम्}
{यं दृष्ट्वा सम्प्रनृत्तो वै हर्षेण महता विभो}


\twolineshloka
{तं प्रहस्याब्रवीद्देवो मुनिं रागेण मोहितम्}
{अहं न विस्मयं विप्र गच्छामीति प्रपश्य माम्}


\twolineshloka
{एवमुक्त्वा मुनिश्रेष्ठं महादेवेन धीमता}
{अङ्गुल्यग्रेण राजेन्द्रस्वाङ्गुष्ठस्ताडितोऽभवत्}


% Check verse!
ततो भस्म क्षताद्राजन्निर्गतं हिमसन्निभम्
\twolineshloka
{तद्दृष्ट्वा व्रीडितो राजन्स मुनिः पादयोर्गतः}
{मेने देवं महादेवमिदं चोवाच विस्मितः}


\twolineshloka
{नान्यं देवादहं मन्ये रुद्रात्परतरं महत्}
{सुरासुरस्य जगतो गतिस्त्वमसि शूलधृक्}


\twolineshloka
{त्वया सृष्टमिदं विश्वं वदन्तीह मनीषिणः}
{त्वामेव सर्वं विशति पुनरेव युगक्षये}


\twolineshloka
{देवैरपि न शक्यस्त्वं परिज्ञातुं कुतो मया}
{त्वयि सर्वे स्म दृश्यन्ते भावा ये जगति स्थिताः}


\twolineshloka
{त्वामुपासन्त वरदं देवा ब्रह्मादयोऽनघ}
{सर्वस्त्वमसि देवानां कर्ता कारयिता च ह}


\twolineshloka
{त्वत्प्रसादात्सुराः सर्वे मोदन्तीहाकुतोभयाः}
{`त्वं प्रभुः परमैश्वर्यादधिकं भासि शङ्करः}


\twolineshloka
{त्वयि ब्रह्मा च विष्णुश्च लोकान्सन्धाय तिष्ठतः}
{त्वन्मूलं च जगत्सर्वं भूतस्थावरजङ्गमम्}


\twolineshloka
{स्वर्गं च परमं स्थानं नृणामभ्युदयार्थिनाम्}
{ददासि च प्रसन्नस्त्वं भक्तानां परमेश्वर}


\twolineshloka
{अनावृत्तिपदं नॄणां नित्यं निश्रेयसार्थिनाम्}
{ददासि कर्मिणां कर्म भावयन्ध्यानयोगतः}


\twolineshloka
{न वृथाऽस्ति महादेव प्रसादस्ते महेश्वर}
{यस्मात्त्वयोपकरणात्करोमि कमलेक्षण}


\threelineshloka
{प्रपद्ये शरणं शंभुं सर्वदा सर्वतः स्थितम्}
{कर्मणा मनसा वाचा तमेवाभिभजाम्यहम् ॥जनमेजय उवाच}
{'}


% Check verse!
एवं स्तुत्वा महादेवं स ऋषिः प्रणतोऽभवत्
\twolineshloka
{यदिदं चापलं देव कृतमेतत्स्मयादिकम्}
{ततः प्रसादयमि त्वां तपो मे न क्षरेदिति}


\twolineshloka
{ततो देवः प्रीतमनास्तमृषिं पुनरब्रवीत्}
{तपस्ते वर्धतां विप्र मत्प्रसादात्सहस्रधा}


\twolineshloka
{आश्रमे चेह वत्स्यामि त्वया सार्धमहं सदा}
{सप्तसारस्वते चास्मिन्यो मामर्चिष्यते नरः}


\twolineshloka
{न तस्य दुर्लभं किञ्चिद्धवितेह परत्र वा}
{सारस्वतं च ते लोकं गमिष्यन्ति न संशयः}


\twolineshloka
{एतन्मङ्कणकस्यापि चरितं भूरितेजसः}
{स हि पुत्रः सुकन्यायामुत्पन्नो मातरिश्वना}


\chapter{अध्यायः ४१}
\twolineshloka
{वैशम्पायन उवाच}
{}


\twolineshloka
{उषित्वा तत्र रामस्तु सम्पूज्याश्रमवासिनः}
{तथा मङ्कणके प्रीतिं शुभां चक्रे हलायुधः}


\twolineshloka
{दत्त्वा दानं द्विजातिभ्यो रजनीं तामुपोष्य च}
{पूजितो मुनिसङ्खैश्च प्रातरुत्थाय लाङ्गली}


\twolineshloka
{अनुज्ञाप्य मुनीन्सर्वान्स्पृष्ट्वा तोयं च भारत}
{प्रययौ त्वरितो रामस्तीर्थहेतोर्महाबलः}


\twolineshloka
{ततस्त्वौशनसं तीर्थमाजगाम हलायुधः}
{कपालमोचनं नाम यत्र मुक्तो महामुनिः}


\twolineshloka
{महता शिरसा राजन्ग्रस्तजङ्घो महोदरः}
{राक्षसस्य महाराज रामक्षिप्तस्य वै पुरा}


\threelineshloka
{तत्र पूर्वं तपस्तप्तं काव्येन सुमहात्मना}
{यत्रास्य नीतिरखिला प्रादुर्भूता महात्मनः}
{यत्रस्थश्चिन्तयामास दैत्यदानवविग्रहम्}


\threelineshloka
{तत्प्राप्य च बलो राजंस्तीर्थप्रवरमुत्तमम्}
{विधिवद्वै ददौ वित्तं ब्राह्मणानां महात्मनाम् ॥जनमेजय उवाच}
{}


\threelineshloka
{कपालमोचनं ब्रह्मन्कथं यत्र महामुनिः}
{मुक्तः कथं चास्य शिरो लग्नं केन च हेतुना ॥वैशम्पायन उवाच}
{}


\twolineshloka
{पुरा वै दण्डकारण्ये राघवेण महात्मना}
{वसता राजशार्दूल राक्षसाञ्शमयिष्यता}


\twolineshloka
{जनस्थाने शिरश्छिन्नं राक्षसस्य दुरात्मनः}
{क्षुरेण शितधारेण तत्पणात महावने}


\twolineshloka
{महोदरस्य तल्लग्नं जङ्घायां वै यदृच्छया}
{वने विचरतो राजन्नस्थि भित्त्वाऽस्फुरत्तदा}


\twolineshloka
{स तेन लग्नेन तदा द्विजातिर्न शशाक ह}
{अभिगन्तुं महाप्राज्ञस्तीर्थान्यायतनानि च}


\twolineshloka
{स पूतिना विस्रवता वेदनार्तो महामुनिः}
{जगाम् सर्वतीर्थानि पृथिव्यां चेति नः श्रुतम्}


\twolineshloka
{स गत्वा सरितः सर्वाः समुद्रांश्च महातपाः}
{कथयामास तत्सर्वमृषीणां भावितात्मनाम्}


\twolineshloka
{आप्लुत्य सर्वतीर्थेषु न च मोक्षमवाप्तवान्}
{स तु शुश्राव विप्रेन्द्रो मुनीनां वचनं महत्}


\twolineshloka
{सरस्वत्यास्तीर्थवरं ख्यातमौशनसं तदा}
{सर्वपापप्रशमनं सिद्धिक्षेत्रमनुत्तमम्}


% Check verse!
स तु गत्वा ततस्तत्र तीर्थमौशनसं द्विजः
\twolineshloka
{तत औशनसे तीर्थे तस्योपस्पृशतस्तदा}
{xxx च्छरश्चरणं मुक्त्वा पपातान्तर्जले तदा}


\twolineshloka
{विमुक्तस्तेन शिरसा परं सुखमवाप ह}
{स चाप्यन्तर्जले मूर्धा जगामादर्शनं विभो}


\twolineshloka
{ततः स विरुजो राजन्पूतात्मा वीतकल्मषः}
{आजगामाश्रमं प्रीतः कृतकृत्यो महोदरः}


\twolineshloka
{सोऽथ गत्वाऽऽश्रमं पुण्यं विप्रमुक्तो महातपाः}
{कथयामास तत्सर्वमृषीणां भावितात्मनाम्}


\twolineshloka
{ते श्रुत्वा वचनं तस्य ततस्तीर्थस्य मानद}
{कपालमोचनमिति नाम चक्रुः समागताः}


\twolineshloka
{स चापि तीर्थप्रवरं पुनर्गत्वा महानृषिः}
{पीत्वा पयःसुविपुलं सिद्धिमायात्तदा मुनिः}


\twolineshloka
{तत्र दत्त्वा बहून्देयान्विप्रान्सम्पूज्य माधवः}
{जगाम तत्र राजेन्द्र उशङ्गोराश्रमं तदा}


\twolineshloka
{यत्र तप्तं तपो घोरमार्ष्टिषेणेन भारत}
{ब्राह्मण्यं लब्धवान्यत्र विश्वामित्रो महामुनिः}


\twolineshloka
{सर्वकामसमृद्धं च तदाश्रमपदं महत्}
{मुनिभिर्ब्राह्मणैश्चैव सेवितं सर्वदा विभो}


\twolineshloka
{ततो हलधरः श्रीमान्ब्राह्मणैः परिवारितः}
{जगाम तत्र राजेन्द्र उशङ्गुस्तनुमत्यजत्}


\twolineshloka
{उशङ्गुर्ब्राह्मणो वृद्धस्तपोनित्यश्च भारत}
{देहन्यासे कृतमना विचिन्त्य बहुधा तदा}


\twolineshloka
{ततः सर्वानुपादाय तनयान्वै महातपाः}
{उशङ्गुरब्रवीत्तत्र नयध्वं मां पृथूदकम्}


\twolineshloka
{विज्ञायातीतवयसमुशङ्गुं ते तपोधनाः}
{तं च तीर्थमपानिन्युः सरस्वत्यास्तपोधनम्}


\twolineshloka
{स तैः पुत्रैस्तदा धीमानानीतो वै सरस्वतीम्}
{पुण्यां तीर्थशतोपेतां विप्रसङ्घैर्निषेविताम्}


\threelineshloka
{स तत्र विधिना राजन्नाप्लुत्य सुमहातपाः}
{ज्ञात्वा तीर्थगुणांश्चैव प्राहेदमृषिसत्तमः}
{सुप्रीतः पुरुषव्याघ्र सर्वान्पुत्रानुपासतः}


\twolineshloka
{सरस्वत्युत्तरे तीरे यस्त्यजेदात्मनस्तनुम्}
{पृथूदके जप्यपरो नैनं श्वो मरणं तपेत्}


\threelineshloka
{`इत्युक्त्वा स्वां तनुं त्यक्त्वा प्रपेदे वैष्णवं पदम्'}
{तत्राप्लुत्य स धर्मात्मा उपस्पृश्य हलायुधः}
{दत्त्वा चैव बूहून्देयान्विप्राणां विप्रवत्सलः}


% Check verse!
ससर्ज यत्र भगवाँल्लोकाँल्लोकपितामहः
\twolineshloka
{यत्रार्ष्टिषेणः कौरव्य ब्राह्मण्यं संशितव्रतः}
{तपसा महता राजन्प्राप्तवानृषिसत्तमः}


\twolineshloka
{सिन्धुद्वीपश्च राजर्षिर्देवापिश्च महातपाः}
{ब्राह्मण्यं लब्धवान्यत्र विश्वामित्रस्तथा मुनिः}


\twolineshloka
{महातपस्वीं भगवानुग्रतेजा महातपाः}
{तत्राजगाम बलवान्बलभद्रः प्रतापवान्}


\chapter{अध्यायः ४२}
\twolineshloka
{जनमेजय उवाच}
{}


\twolineshloka
{आर्ष्टिषेणस्तथा ब्रह्मन्विपुलं तप्तवांस्तपः}
{सिन्धुद्वीपः कथं चापि ब्राह्मण्यं लब्धवांस्तदा}


\threelineshloka
{देवापिश्च कथं ब्रह्मन्विश्वामित्रश्च सत्तम}
{तन्ममाचक्ष्व भगवन्परं कौतूहलं हि मे ॥वैशम्पायन उवाच}
{}


\twolineshloka
{पुरा कृतयुगे राजन्नार्ष्टिषेणो द्विजोत्तमः}
{वसन्गुरुकुले नित्यं नित्यमध्ययने रतः}


\twolineshloka
{तस्य राजन्गुरुकुले वसतो नित्यमेव च}
{समाप्तिं नागमद्विद्या नापि वेदा विशाम्पते}


\twolineshloka
{स निर्विण्णस्ततो राजंस्तपस्तेपे महातपाः}
{ततो वै तपसा तेन प्राप्य वेदाननुत्तमान्}


\twolineshloka
{स विद्वान्वेदयुक्तश्च सिद्धश्चाप्यृषिसत्तमः}
{तत्र तीर्थे वरान्प्रादात्त्रीनेव सुमहातपाः}


\twolineshloka
{अस्मिंस्तीर्थे महानद्या अद्यप्रभृति मानवः}
{आप्लुतो वाजिमेधस्य फलं प्राप्स्यति पुष्कलम्}


\twolineshloka
{अद्यप्रभृति नैवात्र भयं व्यालाद्भविष्यति}
{अपि चाल्पेन कालेन फलं प्राप्स्यति पुष्कलम्}


\twolineshloka
{एवमुक्त्वा महातेजा जगाम त्रिदिवं मुनिः}
{एवं सिद्वः स भगवानार्ष्टिषेणः प्रतापवान्}


\twolineshloka
{तस्मिन्नेव तदा तीर्थे सिन्धुद्वीपः प्रतापवान्}
{देवापिश्च महाराज ब्राह्मण्यं प्रापतुर्महत्}


\twolineshloka
{तथाच कौशिकस्तात तपोनित्यो जितेन्द्रियः}
{तपसा वै सुतप्तेन ब्राह्मणत्वमवाप्तवान्}


\twolineshloka
{गाधिर्नाम महानासीत्क्षत्रियः प्रथितो भुवि}
{तस्य पुत्रोऽभवद्राजन्विश्वामित्रः प्रतापवान्}


\twolineshloka
{स राजा कौशिकस्तात महायोग्यभवत्किल}
{स पुत्रमभिषिच्याथ विश्वामित्रं महातपाः}


\twolineshloka
{देहन्यासे मनश्चक्रे तमूचुः प्रणताः प्रजाः}
{न गन्तव्यं महाप्राज्ञ त्राहि चास्मान्महाभयात्}


\twolineshloka
{एवमुक्तः प्रत्युवाच ततो गाधिः प्रजास्ततः}
{विश्वस्य जगतो गोप्ता भविष्यति सुतो मम}


\twolineshloka
{इत्युक्त्वा तु ततो गाधिर्विश्वामित्रं निवेश्य च}
{जगाम त्रिदिवं राजन्विश्वामित्रोऽभवन्नृपः}


\twolineshloka
{न स शक्नोति पृथिवीं यत्नवानपि रक्षितुम्}
{ततः शुश्राव राजा स राक्षसेभ्यो महाभयम्}


\twolineshloka
{निर्ययौ नगराच्चापि चतुरङ्गबलान्वितः}
{स गत्वा दूरमध्वानं वसिष्ठाश्रममभ्ययात्}


\threelineshloka
{तस्य ते सैनिका राजंश्चक्रुस्तत्रानयान्बहून्}
{ततस्तु भगवान्विप्रो वसिष्ठो श्रममभ्ययात्}
{ददृशेऽथ ततः सर्वं भज्यमानं महाव्रनम्}


\twolineshloka
{तस्य क्रुद्धो महाराज वसिष्ठो मुनिसत्तमः}
{सृजस्व शबरान्घोरानिति स्वां गामुवाच ह}


\twolineshloka
{तथोक्ता साऽसृजद्धेनुः पुरुषान्घोरदर्शनान्}
{ते तु तद्बलमासाद्य बभञ्जुः सर्वतोदिशम्}


\twolineshloka
{तच्छ्रुत्वा विद्रुतं सैन्यं विश्वामित्रस्तु गाधिजः}
{तपः परं मन्यमानस्तपस्येव मनो दधे}


\twolineshloka
{सोऽस्मिंस्तीर्थवरे राजन्सरस्वत्याः समाहितः}
{नियमैश्चोपवासैश्च कर्शयन्देहमात्मनः}


\twolineshloka
{जलाहारो वायुभक्षः पर्णाहारश्च सोऽभवत्}
{तथा स्थण्डिलशायी च ये चान्ये नियमाःपृथक्}


\twolineshloka
{असकृत्तस्य देवास्तु व्रतविघ्नं प्रचक्रिरे}
{न चास्य नियमाद्बुद्धिरपयाति महात्मनः}


\twolineshloka
{ततः परेण यत्नेन तप्त्वा बहुविधं तपः}
{तेजसा भास्कराकारो गाधिजः समपद्यत}


\twolineshloka
{तपसा तु तथायुक्तं विश्वामित्रं पितामहः}
{अमन्यत महातेजा वरदोऽदर्शयत्तदा}


\twolineshloka
{स तु वव्रे वरं राजन्स्यामहं ब्राह्मणस्त्विति}
{तथेति चाब्रवीद्ब्रह्मा सर्वलोकपितामहः}


\twolineshloka
{स लब्ध्वा तपसोग्रेण ब्राह्मणत्वं महायशाः}
{विचचार महीं कृत्स्नां कृतकामः सुरोपमः}


\twolineshloka
{तस्मिंस्तीर्थवरे रामः प्रदाय विविधं वसु}
{पयस्विनीस्तथा धेनूर्यानानि शयनानि च}


\twolineshloka
{अथ वस्त्राण्यलङ्कारं भक्ष्यं पेयं च शोभनम्}
{अददन्मुदितो राजन्पूजयित्वा द्विजोत्तमान्}


\twolineshloka
{ययौ राजंस्ततो रामो बकस्याश्रममन्तिकात्}
{यत्र तेपे तपस्तीव्रं दाल्भ्यो बक इति श्रुतिः}


\chapter{अध्यायः ४३}
\twolineshloka
{वैशम्पायन उवाच}
{}


\threelineshloka
{ब्रह्मयोनिभिराकीर्णं जगाम यद्वुनन्दनः}
{यत्र दाल्भ्यो बको राजन्पश्वर्थ सुमहातपाः}
{जुहाव धृतराष्ट्रस्य राष्ट्रं कोपसमन्वितः}


\twolineshloka
{तपसा घोररूपेण कर्शयन्देहमात्मनः}
{क्रोधेन महताऽऽविष्टो धर्मात्मा वै प्रतापवान्}


\twolineshloka
{पुरा हि नैमिशीयानां सत्रे द्वादशवार्षिके}
{वृत्ते विश्वजितोऽन्ते वै पाञ्चालानृषयोऽगमन्}


\threelineshloka
{तत्रेश्वरमयाचन्त दक्षिणार्थं मनस्विनः}
{`तत्र ते लेभिरे राजन्पाञ्चालेभ्यो महर्षयः}
{'बलान्वितान्वत्सतरान्निर्व्याधीन्सप्तविंशतिम्}


\twolineshloka
{तानब्रवीद्बलो दाल्भ्यो विभजध्वं पशूनिति}
{पशूनेतानहं त्यक्त्वा भिक्षिष्ये राजसत्तमम्}


\twolineshloka
{एवमुक्त्वा वको राजन्नृषीन्सर्वांन्प्रतापवान्}
{जगाम धृतराष्ट्रस्य भवनं ब्राह्मणोत्तमः}


\twolineshloka
{स समीपगतो भूत्वा धृतराष्ट्रं जनेश्वरम्}
{अयाचत पशून्दाल्भ्यः स चैनं रुषितोऽब्रवीत्}


\twolineshloka
{यदृच्छया मृता दृष्ट्वा गास्तदा नृपसत्तमः}
{एतान्पशून्नय क्षिप्रं ब्रह्मबन्धो यदीच्छसि}


\twolineshloka
{ऋषिस्त्वथ बकः क्रुद्धश्चिन्तयामास धर्मवित्}
{अहो वत नृशंसं वै वाक्यमुक्तोऽस्मि संसदि}


\twolineshloka
{चिन्तयित्वा मुहूर्तेन रोषाविष्टो द्विजोत्तमः}
{मतिं चक्रे विनाशाय धृतराष्ट्रस्य भूपतेः}


\twolineshloka
{स तूत्कृत्य मृतानां वै मांसानि मुनिसत्तमः}
{जुहाव धृतराष्ट्रस्य राष्ट्रं नरपतेः पुरा}


\twolineshloka
{अवाकर्णे सरस्वत्यास्तीर्थे प्रज्वाल्य पावकम्}
{xxxx दाल्भ्यो महाराज नियमं परमं स्थितः}


% Check verse!
स तैरेव जुहावाग्नौ राष्ट्रं मांसैर्महातपः
\twolineshloka
{तस्मिंस्तु विधिवत्सत्रे सम्प्रवृत्ते सुदारुणे}
{अक्षीयत ततो राष्ट्रं धृतराष्ट्रस्य पार्थिव}


% Check verse!
छिxxxमानं xxxxxx परशुना विभोxxxxxxxxxxxx व्यवकीर्थमचेतनम्
\twolineshloka
{दृष्ट्वा तथाऽवकीर्णं तु राष्ट्रं च मनुजाधिपः}
{बभूव दुर्मना राजा चिन्तयामास च प्रभुः}


\twolineshloka
{xxxxxxxxxरद्यत्रं ब्राह्मणैः सहितः पुरा}
{xxxxxगच्छत्तु क्षीयते राष्ट्रमेव च}


% Check verse!
यदा स पार्थिवः खिन्नस्ते च विप्रास्तदाऽनघ
\twolineshloka
{यदा चापि न शक्नोति राष्ट्रं मोक्षयितुं नृपः}
{अथ विप्रादिकांस्तत्र पप्रच्छ जनमेजय}


\twolineshloka
{ततो विप्रादिकाः प्राहुः पशुविप्रकृतस्त्वया}
{मांसैरभिजुहोतीदं तव राष्ट्रं मुनिर्बकः}


\twolineshloka
{तेन ते हूयमानस्य राष्ट्रस्यास्य क्षयो महान्}
{तस्यैतत्तपसः कर्म येन तेऽद्य लयो महान्}


\threelineshloka
{`यदीच्छसि महाबाहो शान्तिं राष्ट्रस्य भूमिप}
{'अपां कुञ्जे सरस्वत्यास्तं प्रसादय पार्थिव ॥वैशम्पायन उवाच}
{}


\twolineshloka
{सरस्वतीं ततो गत्वा स राजा बकमब्रवीत्}
{निपत्य शिरसा भूमौ प्राञ्जलिर्भरतर्षभ}


\threelineshloka
{प्रसादये त्वां भगवन्नपराधं क्षमस्व मे}
{मम दीनस्य लुब्धस्य मौर्ख्येण हतचेतसः}
{त्वं गतिस्त्वं च मे नाथः प्रसादं कर्तुमर्हसि}


\twolineshloka
{तं तथा विलपन्तं तु शोकोपहतचेतसम्}
{दृष्ट्वा तस्य कृपा जज्ञे राष्ट्रं तस्य व्यमोचयत्}


\twolineshloka
{ऋषिः प्रसन्नस्तस्याभूत्संरम्भं च विहाय सः}
{मोक्षार्थं तस्य राज्यस्य जुहाव पुनराहुतिम्}


\twolineshloka
{मोक्षयित्वा ततो र्ष्ट्रं प्रतिगृह्य पशून्बहून्}
{हृष्टात्मा नैमिशारण्यं जगाम पुनरेव सः}


\twolineshloka
{धृतराष्ट्रोऽपि धर्मात्मा स्वस्थचेता महामनाः}
{स्वमेव नगरं राजन्प्रतिपेदे महर्द्धिमत्}


\twolineshloka
{तत्र तीर्थे महाराज बृहस्पतिरुदारधीः}
{असुराणामभावाय भवाय च दिवौकसाम्}


\twolineshloka
{मांसैरभिजुहावेष्टिमक्षीयन्त ततोऽसुराः}
{दैवतैरपि सम्भग्ना जितकाशिभिराहवे}


\twolineshloka
{तत्रापि विधिवद्दत्त्वा ब्राह्मणेभ्यो महायशाः}
{वाजिनः कुञ्जरांश्चैव रथांश्चश्वतरीयुतान्}


\twolineshloka
{रत्नानि च महार्हाणि धनं धान्यं च पुष्कलम्}
{ययौ तीर्थं महाबाहुर्यायातं पृथिवीपते}


\twolineshloka
{तत्र यज्ञे ययातेश्च महाराज सरस्वती}
{सर्पिः पयश्च सुस्राव नाहुषस्य महात्मनः}


\twolineshloka
{तत्रेष्ट्वा पुरुषव्याघ्रो ययातिः पृथिवीपतिः}
{आक्रामदूर्ध्वं पुदितो लेभे लोकांश्च पुष्कलान्}


\threelineshloka
{पुनस्तत्र च राज्ञस्तु ययातेर्यजतः प्रभोः}
{औदार्यं परमं कृत्वा भक्तिं चात्मनि शाश्वतीम्}
{ददौ कामान्ब्राह्मणेभ्यो वान्यान्यो मनसेच्छति}


\threelineshloka
{यो यत्र स्थित एवेह आहूतो यज्ञसंस्तरे}
{तस्यतस्य सरिच्छ्रेष्ठा गृहादि शयनादिकम्}
{षड्रसं भोजनं चैव दानं नानाविधं तथा}


\twolineshloka
{ते मन्यमाना राज्ञस्तु सम्प्रदानमनुत्तमम्}
{राजानं तुष्टुवुः प्रीता दत्त्वा चैवाशिषः शुभाः}


\twolineshloka
{तत्र देवाः सगन्धर्वाः प्रीता यज्ञस्य सम्पदा}
{विस्मिता मानुषाश्चासन्दृष्ट्वा तां यज्ञसम्पदम्}


\twolineshloka
{ततस्तालकेतुर्महाधर्मकेतु--र्महात्मा कृतात्मा महादाननित्यः}
{वसिष्ठापवाहं महाभीमवेगंधृतात्मा जितात्मा समभ्याजगाम}


\chapter{अध्यायः ४४}
\twolineshloka
{जनमेजय उवाच}
{}


\twolineshloka
{वसिष्ठापवाहो ब्रह्मन्वै भीमवेगः कथं नु सः}
{किमर्थं च सरिच्छ्रेष्ठा तमृषिं प्रत्यवाहयत्}


\threelineshloka
{कथमस्याभवद्वैरं कारणं किं च तत्प्रभो}
{शंस पृष्टो महाप्राज्ञ न हि तृप्यामि कथ्यति ॥वैशम्पायन उवाच}
{}


\twolineshloka
{विश्वामित्रस्य विप्रर्षेर्वसिष्ठस्य च भारत}
{भृशं वैरमभूद्राजंस्तपःस्पर्धाकृतं महत्}


\twolineshloka
{आश्रमो वै वसिष्ठस्य स्थाणुतीर्थेऽभवन्महान्}
{पूर्वतः पार्श्वतश्चासीद्विश्वामित्रस्य धीमतः}


\twolineshloka
{यत्र स्थाणुर्महाराज तप्तवान्परमं तपः}
{तत्रास्य कर्म तद्धोरं प्रवदन्ति मनीषिणः}


\twolineshloka
{यत्रेष्ट्वा भगवान्स्थाणुः पूजयित्वा सरस्वतीम्}
{स्थापयामास तत्तीर्थं स्थाणुतीर्थमिति प्रभो}


\twolineshloka
{तत्र तीर्थे सुराः स्कन्दमभ्यषिञ्चन्नराधिप}
{सैनापत्येन महता सुरारिविनिबर्हणम्}


\twolineshloka
{तस्मिन्सारस्वते तीर्थे विश्वामित्रो महामुनिः}
{वसिष्ठं चालयामास तपसोग्रेण तच्छृणु}


\twolineshloka
{विश्वामित्रवसिष्ठौ तावहन्यहनि भारत}
{स्पर्धां तपः कृतां तीव्रां चक्रतुस्तौ वतपोधनौ}


\twolineshloka
{तत्राप्यधिकसन्तप्तो विश्वामित्रो महामुनिः}
{दृष्ट्वा तेजो वसिष्ठस्य चिन्तामभिजगाम ह}


\twolineshloka
{तस्य बुद्धिरियं ह्यासीद्धर्मनित्यस्य भारत}
{इदं सरस्वती तूर्णं मत्समीपं तपोधनम्}


\twolineshloka
{आनयिष्यति वेगेन वसिष्ठं जपतां वरम्}
{इहागतं द्विजश्रेष्ठं हनिष्यामि न संशयः}


\twolineshloka
{एवं निश्चित्य भगवान्विश्वामित्रो महामुनिः}
{सस्मार सरितां श्रेष्ठां क्रोधसंरक्तलोचनः}


\twolineshloka
{सा ध्याता मुनिना तेन व्याकुलत्वं जगाम ह}
{गत्वा चैनं महावीर्यं महाकोपं च भामिनी}


\twolineshloka
{तदा तं वेपमानाङ्गी विवर्णा प्राञ्जलिर्भृशम्}
{उपतस्थे मुनिवरं विश्वामित्रं सरस्वती}


\twolineshloka
{हतवीरा यथा नारी साऽभवद्दुःखिता भृशम्}
{ब्रूहि किं करवाणीति प्रोवाच मुनिसत्तमम्}


\twolineshloka
{तामुवाच मुनिः क्रुद्धो वसिष्ठं शीघ्रमानय}
{यावदेनं निहन्म्यद्य तच्छ्रुत्वा व्यथिता नदी}


\twolineshloka
{साञ्जलिं तु ततः कृत्वा पुण्डरीकनिभेक्षणा}
{प्राकम्पत भृशं भीता वायुनेवाहता लता}


\twolineshloka
{तथारूपां तु तां दृष्ट्वा मुनिराह महानदीम्}
{अविचारं वसिष्ठं त्वमानयस्त्वान्तिकं मम}


\twolineshloka
{सा तस्य वचनं श्रुत्वा ज्ञात्वा पापं चिकीर्षितम्}
{वसिष्ठस्य प्रभावं च जानन्त्यप्रतिमं भुवि}


\twolineshloka
{साऽभिगम्य वसिष्ठं च इदमर्थमचोदयत्}
{यदुक्ता सरितांश्रेष्ठा विश्वामित्रेण धीमता}


\twolineshloka
{उभयोः शापयोर्भीता वेपमाना पुनःपुनः}
{चिन्तयित्वा महाशापमृषिविप्रासिता भृशम्}


\twolineshloka
{तां कृशां च विवर्णां च दृष्ट्वा चिन्तासमन्विताम्}
{उवाच राजन्धर्मात्मा वसिष्ठो द्विपदां वरः}


\twolineshloka
{पाह्यात्मानं सरिच्छेष्ठे वह मां शीघ्रगामिनी}
{विश्वामित्रः शपेद्वि त्वां मा कृथास्त्वं विचारणाम्}


\twolineshloka
{तस्य तद्वचनं श्रुत्वा कृपाशीलस्य सा सरित्}
{चिन्तयामास कौरव्य किं कृत्वा सुकृतं भवेत्}


\twolineshloka
{तस्याश्चिन्ता समुत्पन्ना वसिष्ठो मय्यतीव हि}
{कृतवान्हि दयां नित्यं तस्या कार्यं हितं मया}


\twolineshloka
{अथ कूले स्वके राजञ्जपन्तमृषिसत्तमम्}
{जुह्वानं कौशिकं प्रेक्ष्य सरस्वत्यभ्यचिन्तयत्}


\twolineshloka
{इदमन्तरमित्येवं ततः सा सरितां वरा}
{कूलापहारमकरोत्स्वेन वेगेन सा सरित्}


\twolineshloka
{तेन कूलापहारेण मैत्रावरुणिरौह्यत}
{उह्यमानः स तुष्टाव तदा राजन्सरस्वतीम्}


\twolineshloka
{पितामहस्य सरसः प्रवृत्ताऽसि सरस्वति}
{व्याप्तं चेदं जगत्सर्वं तवैवाम्भोभिरुत्तमैः}


\twolineshloka
{त्वमेवाकाशगा देवि मेघेषु सृजसे पयः}
{सर्वाश्चापस्त्वमेवेति यथा वयमधीमहि}


\fourlineindentedshloka
{पुष्टिर्द्युतिस्तथा कीर्तिः सिद्धिर्बुद्धिरमा तथा}
{त्वमेव वाणी स्वाहा त्वं तवायत्तमिदं जगत्}
{त्वमेव सर्वभूतेषु वससीह चतुर्विधा ॥वैशम्पायन उवाच}
{}


\twolineshloka
{एवं सरस्वती राजन्स्तूयमाना महर्षिणा}
{वेगेनोवाह तं विप्रं विश्वामित्राश्रमं प्रति}


% Check verse!
न्यवेदयत चाक्षीक्ष्णं विश्वामित्राय तं मुनिम्
\twolineshloka
{तमानीतं सरस्वत्या दृष्ट्वा कोपसमन्वितः}
{अथान्वेषत्प्रहरणं वसिष्ठान्तकरं तदा}


\threelineshloka
{तं तु क्रुद्धमभिप्रेक्ष्य ब्रह्मवध्याभयान्नदी}
{अपोवाह वसिष्ठं तु प्राचीं दिशमतन्द्रिता}
{उभयोः कुर्वती वाक्यं वञ्चयित्वा च गाधिजम्}


\twolineshloka
{ततोपवाहितं दृष्ट्वा वसिष्ठमृषिसत्तमम्}
{अब्रवीत्त्वथ सङ्क्रुद्धो विश्वामित्रः सरस्वतीम्}


\twolineshloka
{यस्मान्मां त्वं सरिच्छ्रेष्ठे वञ्चयित्वा पुनर्गता}
{शोणितं वह कल्याणि राक्षसानां च सम्मतम्}


\twolineshloka
{ततः सरस्वती शप्ता विश्वामित्रेण धीमता}
{अवहच्छोणितोन्मिश्रं तोयं संवत्सरं तदा}


\twolineshloka
{अथर्षयश्च देवाश्च गन्धर्वाप्सरसस्तदा}
{सरस्वतीं तथा दृष्ट्वा बभूवुर्भृशदुःखिताः}


\twolineshloka
{एवं वसिष्ठापवाहो लोके ख्यातो जनाधिप}
{आगच्छच्च पुनर्मार्गं स्वमेव सरितां वरा}


\chapter{अध्यायः ४५}
\twolineshloka
{वैशम्पायन उवाच}
{}


\twolineshloka
{स शप्ता तेन क्रुद्धेन विश्वामित्रेण धीमता}
{तस्मिंस्तीर्थवरे शुभ्रं शोणितं समुपावहत्}


\twolineshloka
{अथाजग्मुस्ततो राजन्राक्षसास्तत्र भारत}
{तत्र ते शोणितं सर्वे पिबन्तः सुखमासते}


\twolineshloka
{तृप्ताश्च सुभृशं तेन सुखिता विगतज्वराः}
{नृत्यन्तश्च हसन्तश्च यथा स्वर्गजितस्तथा}


\twolineshloka
{कस्यचित्त्वथ कालस्य ऋषयः सुतपोधनाः}
{तीर्थयात्रां समाजग्मुः सरस्वत्यां महीपते}


\twolineshloka
{तेषु सर्वेषु तीर्थेषु स्वाप्लुत्य मुनिपुङ्गवाः}
{प्राप्य प्रीतिं परां चापि तपोलुप्धा विशारदाः}


\twolineshloka
{प्रययुर्हि ततो राजन्येन तीर्थमसृग्वहम्}
{अथागम्य महाभागास्तत्तीर्थं दारुणं तदा}


\twolineshloka
{दृष्ट्वा तोयं सरस्वत्याः शोणितेन परिप्लुतम्}
{पीयमानं च रक्षोभिर्बहुभिर्नृपसत्तम}


\twolineshloka
{तान्दृष्ट्वा राक्षसान्राजन्मुनयः संशितव्रताः}
{परित्राणे सरस्वत्याः परं यत्नं प्रचक्रिरे}


\twolineshloka
{ते तु सर्वे महाभागाः समागम्य महाव्रताः}
{आहूय सरितां श्रेष्ठामिदं वचनमब्रुवन्}


\twolineshloka
{कारणं ब्रूहि कल्याणि किमर्थं ते हृदो ह्ययम्}
{एवमग्राह्यतां यातः श्रुत्वाऽध्यास्यामहे वयम्}


\twolineshloka
{ततः सा सर्वमाचष्ट यथावृत्तं प्रवेपती}
{दुःखितामथ तां दृष्ट्वा ऊचुस्ते वै तपोधनाः}


\twolineshloka
{कारणं श्रुतमस्माभिः शापश्चैव श्रुतोऽनघे}
{करिष्यन्ति तु यत्प्राप्तं सर्व एव तपोधनाः}


\twolineshloka
{एवमुक्त्वा सरिच्छ्रेष्ठामूचुस्तेऽथ परस्परम्}
{विमोचयामहे सर्वे शापादेतां सरस्वतीम्}


\twolineshloka
{ते सर्वे ब्राह्मणा राजंस्तपोभिर्नियमैस्तथा}
{उपवासैश्च विविधैर्यमैः कष्टव्रतैस्तथा}


\twolineshloka
{आराध्य पशुभर्तारं महादेवं जगत्पतिम्}
{मोक्षयामासुस्तां देवीं सरिच्छ्रेष्ठां सरस्वतीम्}


\threelineshloka
{तेषां तु सा प्रभावेण प्रकृतिस्था सरस्वती}
{प्रसन्नसलिला जज्ञे यथा पूर्वं तथैव हि}
{निर्मुक्ता च सरिच्छ्रेष्ठा विबभौ सा यथा पुरा}


\twolineshloka
{दृष्ट्वा तोयं सरस्वत्या मुनिभिस्तैस्तथा कृतम्}
{तानेव शरणं जग्मू राक्षसाः क्षुधितास्तथा}


\twolineshloka
{कृत्वाञ्जलिं ततो राजन्राक्षसाः क्षुधयाऽर्दिताः}
{ऊचुस्तान्वै मुनीन्सर्वान्कृपायुक्तान्पुनःपुनः}


\twolineshloka
{वयं च क्षुधिताश्चैव धर्माद्धीनाश्च शाश्वतात्}
{न च नः कामकारोऽयं यद्वयं पापकारिणा}


\threelineshloka
{युष्माकं चाप्रसादेन दुष्कृतेन च कर्मणा}
{पक्षोऽयं वर्धतेऽस्माकं यतः स्मो ब्रह्मराक्षसाः}
{योषितां चैव पापेन योनिदोषकृतेन च}


\twolineshloka
{एवं हि वैश्यशूद्राणां क्षत्रियाणां तथैव च}
{ये ब्राह्मणान्प्रद्विषन्ति ते भवन्तीह राक्षसाः}


\twolineshloka
{`शक्तिमन्तोऽपि ये केचिदाश्रितानां च रक्षणम्}
{न कुर्वन्ति मनुष्यास्ते सम्भवन्तीह राक्षसाः'}


\twolineshloka
{आचार्यमृत्विजं चैव गुरुं वृद्वजनं तथा}
{प्रणिनो येऽवमन्यन्ते ते भवन्तीह राक्षसाः}


\twolineshloka
{तत्कुरुध्वमिहास्माकं तारणं द्विजसत्तमाः}
{शक्ता भवन्तः सर्वेषां लोकानामपि तारणे}


\threelineshloka
{तेषां तु वचनं श्रुत्वा तुष्टुवुस्तां महानदीम्}
{मोक्षार्थं रक्षसां तेषामूचुः प्रयतमानसाः ॥ऋषय ऊचुः}
{}


\twolineshloka
{क्षुतं कीटावपन्नं च यच्चोच्छिष्टाचितं भवेत्}
{सकेशमवधूतं च रुदितोपहतं च यत्}


\threelineshloka
{श्वभिः संसृष्टमन्नं च भागोऽसौ रक्षसामिह}
{तस्माज्ज्ञात्वा सदा विद्वानेतान्यत्नाद्विवर्जयेत्}
{राक्षसान्नमसौ भुङ्क्ते यो भुङ्क्ते ह्यन्नमीदृशम्}


\twolineshloka
{शोधयित्वा ततस्तीर्थमृषयस्ते तपोधनाः}
{मोक्षार्थं राक्षसानां च नदीं तां प्रत्यचोदयन्}


\twolineshloka
{महर्षीणां मतं ज्ञात्वा ततः सा सरितां वरा}
{अरुणामानयामास स्वां तनुं पुरुषर्षभ}


\twolineshloka
{तस्यान्तेराक्षसाः स्नात्वा तनूस्त्यक्त्वा दिवं गताः}
{अरुणायां महाराज ब्रह्मवध्यापहा हि सा}


\threelineshloka
{एतमर्थमभिज्ञाय देवराजः शतक्रतुः}
{तस्मिंस्तीर्थे वरे स्नात्वा विमुक्तः पाप्मना किल ॥जनमेजय उवाच}
{}


\threelineshloka
{किमर्थं भगवाञ्शक्रो ब्रह्मवध्यामवाप्तवान्}
{कथमस्मिंश्च तीर्थे वै आप्लुत्याकल्मषोऽभवत् ॥वैशम्पायन उवाच}
{}


\twolineshloka
{शृणुष्वैतदुपाख्यानं यथावृत्तं जनेश्वर}
{यथा बिभेद समयं नमुचेर्वासवः पुरा}


\twolineshloka
{जमुचिर्वासवाद्भीतः सूर्यरश्मिं समाविशत्}
{तेनेन्द्रः सख्यमकरोत्समयं चेदमब्रवीत्}


\twolineshloka
{न चार्द्रेण न शुष्केण न रात्रौ नापि चाहनि}
{वधिष्याम्यसुरश्रेष्ठ सखे सत्येन ते शपे}


\twolineshloka
{एवं स कृत्वा समयं दृष्ट्वा नीहास्मीश्वरः}
{चिच्छेदास्य शिरो राजन्नपां फेनेन वासवः}


\twolineshloka
{तच्छिरो नमुचेश्छिन्नं पृष्ठतः शक्रमन्वियात्}
{भोभो मित्रह पापेति ब्रुवाणं शक्रमन्तिकात्}


\twolineshloka
{एवं स शिरसा तेन चोद्यमानः पुनः पुनः}
{पितामहाय सन्तप्त एतमर्थं न्यवेदयत्}


\twolineshloka
{तमब्रवील्लोकगुरुररुणायां यथाविधि}
{इष्ट्वोपस्पृश देवेन्द्र तीर्थे पापभयापहे}


\twolineshloka
{[एषा पुण्यजला शक्र कृता मुनिभिरेव तु}
{निगूढमस्यागमनमिहासीत्पूर्वमेव तु}


\twolineshloka
{ततोऽभ्येत्यारुणां देवीं प्लावयामास वारिणा}
{सरस्वत्यारुणायाश्च पुण्योऽयं सङ्गमो महान्}


\twolineshloka
{इह त्वं यज देवेन्द्र दद दानान्यनेकशः}
{अत्राप्लुत्य सुघोरात्त्वं पातकाद्विप्रमोक्ष्यसे ॥]}


\twolineshloka
{इत्युक्तः स सरस्वत्याः कुञ्जे वै जनमेजय}
{इष्ट्वा यथावद्बलभिदरुणायामुपास्पृशत्}


\twolineshloka
{स मुक्तः पाप्मना तेन ब्रह्मवध्याकृतेन च}
{जगाम संहृष्टमनास्त्रिदिवं त्रिदशेश्वरः}


\threelineshloka
{शिरस्तच्चापि नमुचेस्तत्रैवाप्लुत्य भारत}
{लोकान्कामदुधात्प्राप्तमक्षयान्राजसत्तम ॥वैशपायन उवाच}
{}


\twolineshloka
{तत्राप्युपस्पृश्य बलो महात्मादत्त्वा च दानानि पृथग्विधानि}
{अवाप्य धर्मं परमार्यकर्माजगाम सोमस्य महत्सुतीर्थम्}


\twolineshloka
{यत्रायजद्राजसूयेन सोमःसाक्षात्पुरा विधिवत्पार्थिवेन्द्र}
{अत्रिर्धीमान्ब्रह्मपुत्रो बभूवहोता यस्मिन्क्रतुमुख्ये महात्मा}


\twolineshloka
{यस्यान्तेऽभूत्सुमहद्दानवानांदैतेयानां राक्षसानां च देवैः}
{सङ्ग्रामो वै तारकाख्यः सुतीव्रोयत्र स्कन्दस्तारकाख्यं जघान}


\twolineshloka
{सैनापत्यं लब्धवान्दैवतानांमहासेनो यत्र दैत्यान्तकर्ता}
{साक्षाच्चैवं न्यवसत्कार्तिकेयःसदा कुमारो यत्र स प्लुक्षराजः}


\chapter{अध्यायः ४६}
\twolineshloka
{जनमेजय उवाच}
{}


\twolineshloka
{सरस्वत्याः प्रभावोऽयमुक्तस्ते द्विजसत्तम}
{कुमारस्याभिषेकं तु ब्रह्मन्नाख्यातुमर्हसि}


\twolineshloka
{यस्मिन्देशे च काले च यथा च वदतां वर}
{यैश्चाभिषिक्तो भगवान्विधिना येन च प्रभुः}


\threelineshloka
{स्कन्दो यथा च दैत्यानामकरोत्कदनं महत्}
{तथा मे सर्वमाचक्ष्व परं कौतूहलं हि मे ॥वैशम्पायन उवाच}
{}


\twolineshloka
{कुरुवंशस्य सदृशं कौतूहलमिदं तव}
{हर्षमुत्पादयत्वेव वचो मे जनमेजय}


\twolineshloka
{हन्त ते कथयिष्यामि शृण्वानस्य नराधिप}
{अभिषेकं कुमारस्य प्रभावं च महात्मनः}


\twolineshloka
{तेजो माहेश्वरं स्कन्नमग्नौ प्रपतितं पुरा}
{तत्सर्वं भगवानग्निर्नाशकद्धर्तुमक्षयम्}


\twolineshloka
{तेन सीदति तेजस्वी दीप्तिमान्हव्यवाहनः}
{न चैवं धारयामास ब्रह्मणे उक्तवान्प्रभुः}


\twolineshloka
{स गङ्गामुपसगम्य नियोगाद्ब्रह्मणः प्रभुः}
{गर्भमाहितवान्दिव्यं भास्करोपमतेजसम्}


% Check verse!
अथ गङ्गापि तं गर्भमसहन्ती विधारणेउत्ससर्ज गिरौ रम्ये हिमवत्यमरार्चिते
\twolineshloka
{स तत्र ववृधे लोकानावृत्य ज्वलनात्मजः}
{ददृशुर्ज्वलनाकारं तं गर्भमथ कृत्तिकाः}


\twolineshloka
{शरस्तम्बे महात्मानमनलात्मजमीश्वरम्}
{ममायमिति ताः सर्वाः पुत्रार्थिन्योऽभिचुक्रुशुः}


\twolineshloka
{तासां विदित्वा भावं तं मातॄणां भगवान्प्रभुः}
{प्रस्नुतानां पयः षड्भिर्वदनैरपिबत्तदा}


\twolineshloka
{तं प्रभावं समालक्ष्य तस्य बालस्य कृत्तिकाः}
{परं विस्मयमापन्ना देव्यो दिव्यवपुर्धराः}


\twolineshloka
{यत्रोत्सृष्टश्च गर्भः स गङ्गया निरिमूर्धनि}
{स शैलः काञ्चनः सर्वः सम्बभौ मेरुवत्तदा}


\twolineshloka
{वर्धता चैव गर्भेण पृथिवी तेन रञ्जिता}
{अतश्च सर्वे संवृत्ता गिरयः काञ्चनात्मकाः}


\twolineshloka
{कुमारः सुमहावीर्यः कार्तिकेय इति स्मृतः}
{गाङ्गेयः पूर्वमभवन्महाकायो बलान्वितः}


\threelineshloka
{शमेन तपसा चैव वीर्येण च समन्वितः}
{ववृधेऽतीव राजेन्द्र चन्द्रवत्प्रियदर्शनः}
{}


\twolineshloka
{स तस्मिन्काञ्चने दिव्ये शरस्तम्बे श्रिया वृतः}
{स्तयमानः सदा शेते गन्धर्वैर्मुनिभिस्तथा}


\twolineshloka
{तथैनमन्वनृत्यन्त देवकन्याः सहस्रशः}
{दिव्यवादित्रनृत्यज्ञाः स्तुवन्त्यश्चारुदर्शनाः}


\threelineshloka
{अन्वयुश्चाग्नयः सर्वे गङ्गा च सरितां वरा}
{दधार पृथिवी चैनं बिभ्रती रूपमुत्तमम्}
{}


\twolineshloka
{जातकर्मादिकास्तस्य क्रियाश्चक्रे बृहस्पतिः}
{वेदश्चैनं चतुर्मूर्तिरपतस्थे कृताञ्जलिः}


\twolineshloka
{धनुर्वेदश्चतुष्पादः सास्त्रग्रामः ससङ्ग्रहः}
{तत्रैनं समुपातिष्ठत्साक्षाद्वाणी च केवला}


\twolineshloka
{स ददर्श महात्मानं देवदेवमुमापतिः}
{शैलपुत्र्या समागम्यभूतसङ्घशतैर्वृतः}


\twolineshloka
{निकाया भूतसङ्घानां परमाद्भुतदर्शनाः}
{विकृता विकृताकारा विकृताभरणध्वजाः}


\twolineshloka
{व्याघ्रसिंहर्क्षवदना बिडालमकराननाः}
{वृषदंशमुखाश्चान्ये गजोष्ट्रवदनास्तथा}


\twolineshloka
{उलूकवदनाः केचिद्गृध्रगोमायुदर्शनाः}
{क्रौञ्चपारावतनिभैर्वदनै राङ्कवैरपि}


\twolineshloka
{श्वाविच्छल्यकगोधानामजैडकगवां तथा}
{सदृशानि वपूंष्यन्ये तत्रतत्र व्यधारयन्}


\twolineshloka
{केचिच्छेलाम्बुदप्रख्याश्चक्रालातगदायुधाः}
{केचिदञ्जनपुञ्जाभाः केचिच्छ्वेताचलप्रभाः}


\twolineshloka
{सप्तमातृगणाश्चैव समाजग्मुर्विशाम्पते}
{साध्या विश्वेऽथ मरुतो वसवः पितरस्तथा}


\twolineshloka
{रुद्रादित्यास्तथा सिद्धा भुजगा दानवाः खगाः}
{ब्रह्मा स्वयम्भूर्भगवान्सपुत्रः सहविष्णुना}


\twolineshloka
{शक्रस्तथाऽभ्ययाद्द्रष्टुं कुमारममितप्रभम्}
{नारदप्रमुखाश्चापि देवगन्धर्वसत्तमाः}


\threelineshloka
{देवर्षयश्च सिद्धाश्च बृहस्पतिपुरोगमाः}
{पितरो जगतः श्रेष्ठा देवानामपि देवताः}
{तेऽपि तत्र समाजग्मुर्यामा धामाश्च सर्वशः}


\twolineshloka
{स तु बालोऽपि बलवान्महायोगबलान्वितः}
{अभ्याजगाम देवेशं शूलहस्तं पिनाकिनम्}


\twolineshloka
{तमाव्रजन्तमालक्ष्य शिवस्यासीन्मनोगतम्}
{युगपच्छैलपुत्र्याश्च गङ्गायाः पावकस्य च}


\twolineshloka
{कं नु पूर्वमयं बालो गौरवादभ्युपैष्यति}
{अपि मामिति सर्वेषां तेषामासीन्मनोगतम्}


\twolineshloka
{तेषामेतमभिप्रायं चतुर्णामुपलक्ष्य सः}
{युगपद्योगमास्थाय ससर्ज विविधास्तनूः}


\twolineshloka
{ततोऽभवच्चतुर्मूर्तिः क्षणेन भगवान्प्रभुः}
{तस्य शाखो विशाखश्च नैगमेयश्च पृष्ठतः}


\twolineshloka
{एवं स कृत्वा ह्यात्मानं चतुर्धा भगवान्प्रभुः}
{यतो रुद्रस्ततः स्कन्दो जगामाद्भुतदर्शनः}


\twolineshloka
{विशाखस्तु ययौ देवीं ततो गिरिवरात्मजाम्}
{शाखो ययौ स भगवान्दिव्यमूर्तिर्विभावसुम्}


% Check verse!
नैगमेयोऽगमद्गङ्गां कुमारः पावकप्रभः
\twolineshloka
{सर्वे भासुरदेहास्ते चत्वारः समरूपिणः}
{तान्समभ्ययुरव्यग्रास्तदद्भुतमिवाभवत्}


\twolineshloka
{हाहाकारो महानासीद्देवदानवरक्षसाम्}
{तद्दृष्ट्वा महदाश्चर्यमद्भुतं रोमहर्षणम्}


\twolineshloka
{ततो रुद्रश्च देवी च पावकश्च पितामहम्}
{गङ्गया सहिताः सर्वे प्रणिपेतुर्जगत्पतिम्}


\twolineshloka
{प्रणिपत्य ततस्ते तु विधिवद्राजपुङ्गव}
{इदमूचुर्वचो राजन्कार्तिकेयप्रियेप्सया}


\twolineshloka
{अस्य बालस्य भगवन्नाधिपत्य यथेप्सितम्}
{अस्मत्प्रियार्थं देवेश सदृशं दातुमर्हसि}


\twolineshloka
{ततः स भगवान्धीमान्सर्वलोकपितामहः}
{मनसा चिन्तयामास किमयं लभतामिति}


\twolineshloka
{ऐश्वर्याणि च सर्वाणि देवगन्धर्वरक्षसाम्}
{भूतयक्षविहङ्गानां पन्नगानां च सर्वशः}


\twolineshloka
{सर्वमेवादिदेशासौ कौरवेय महात्मनः}
{समर्थं च तमैश्वर्ये महामतिरमन्यत}


\twolineshloka
{ततो मुहूर्तं स ध्यात्वा देवानां स्रेयसि स्थितः}
{सैनापत्यं ददौ तस्मै सर्व भूतेषु भारत}


\twolineshloka
{सर्वदेवनिकायानां ये राजानः परिश्रुताः}
{तान्सर्वान्व्यादिदेशास्मै सर्वभूतपितामहः}


\twolineshloka
{ततः कुमारमादाय देवा ब्रह्मपुरोगमाः}
{अभिषेकार्थमाजग्मुः शैलेन्द्रं सहितास्ततः}


\twolineshloka
{पुण्यां हैमवतीं देवीं सरिच्छ्रेष्ठां सरस्वतीम्}
{समन्तपञ्चके या वै त्रिषु लोकेषु विश्रुता}


\twolineshloka
{तत्र तीरे सरस्वत्याः पुण्ये सर्वगुणान्विते}
{निषेदुर्देवगन्धर्वाः सर्वे सम्पूर्णमानसाः}


\chapter{अध्यायः ४७}
\twolineshloka
{वैशम्पायन उवाच}
{}


\twolineshloka
{ततोऽभिषेकसम्भारान्सर्वान्सम्भृत्य शास्त्रतः}
{बृहस्पतिः समिद्धेऽग्नौ जुहावाग्निं यथाविधि}


\twolineshloka
{ततो हिमवता दत्ते मणिप्रवरशोभिते}
{दिव्यरत्नाचिते पुण्ये निषण्णं परमासने}


\twolineshloka
{सर्वमङ्गलसम्भारैर्विधिमन्त्रपुरस्कृतम्}
{आभिषेचनिकं द्रव्यं गृहीत्वा देवतागणाः}


\twolineshloka
{इन्द्राविष्णू महावीर्यौ सूर्योचन्द्रमसौ तथा}
{धाता चैव विधाता च तथा चैवानिलानलौ}


\twolineshloka
{पूष्णा भगेनार्यम्णा च अंशेन च विवस्वतां}
{रुद्रश्च सहितो धीमान्मित्रेण वरुणेन च}


\twolineshloka
{रुद्रैर्वसुभिरादित्यैरश्विभ्यां च वृतः प्रभुः}
{विश्वैर्देवैर्मरुद्भिश्च साध्यैश्च पितृभिः सह}


\twolineshloka
{गन्धर्वैरप्सरोभिश्च यक्षराक्षसपन्नगैः}
{देवर्षिभिरसङ्ख्यातैस्तथा ब्रह्मर्षिभिस्तथा}


\twolineshloka
{वैखानसैर्वालखिल्यैर्वाय्वाहारैर्मरीचिपैः}
{भृगुभिश्चाङ्गिरोभिश्च यतिभिश्च महात्मभिः}


\twolineshloka
{सर्वैर्विद्याधरैः पुण्यैर्योगसिद्धैस्तथा वृतः}
{पितामहः पुलस्त्यश्च पुलहश्च महातपाः}


\twolineshloka
{अङ्गिराः कश्यपोऽत्रिश्च मरीचिर्भृगुरेव च}
{क्रतुर्हरिः प्रचेताश्च मनुर्दक्षस्तथैव च}


\twolineshloka
{ऋतवश्च ग्रहाश्चैव ज्योतींषि च विशाम्पते}
{मूर्तिमत्यश्च सरितो वेदाश्चैव सनातनाः}


\twolineshloka
{समुद्राश्च हदाश्चैव तीर्थानि विविधानि च}
{पृथिवी द्यौर्दिशश्चैव पादपाश्च जनाधिप}


\twolineshloka
{अदितिर्देवमाता च हीः श्रीः स्वाहा सरस्वती}
{उमा शची सिनीवाली तथैवानुमतिः कुहूः}


\twolineshloka
{राका च धिषणा चैव पत्न्यश्चान्या दिवौकसाम्}
{हिमवांश्चैव विन्ध्यश्च मेरुश्चानेकशृङ्गवान्}


\twolineshloka
{ऐरावतः सानुचरः कलाः काष्ठास्तथैव च}
{मासार्धमासा ऋतवस्तथा रात्र्यहनी नृप}


\twolineshloka
{उच्चैः श्रवा हयश्रेष्ठो नागराजश्च वासुकिः}
{अरुणो गरुडश्चैव वृक्षाश्चौषधिभिः सह}


\twolineshloka
{धर्मश्च भगवान्देवः समाजग्मुर्हि सङ्गताः}
{कालो यमश्च मृत्युश्च यमस्यानुचराश्च ये}


\twolineshloka
{बहुलत्वाच्च नोक्ता ये विविधा देवतागणाः}
{ते कुमाराभिषेकार्षं समाजग्मुस्ततस्ततः}


\twolineshloka
{जगृहुस्ते तदा राजन्सर्वं एव दिवौकसः}
{आभिषेचनिकं भाण्डं मङ्गलानि च सर्वशः}


\twolineshloka
{दिव्यसम्भारसंयुक्तैः कलशैः काञ्चनैर्नृप}
{सारस्वताभिः पुण्याभिरद्भिस्ताभिरलङ्कृतम्}


\twolineshloka
{अभ्यषिञ्चन्कुमारं वै सम्प्रहृष्टा दिवौकसः}
{सैनापत्ये महात्मानमसुराणां भयङ्करम्}


\twolineshloka
{पुरा यथा महाराज वरुणं वै जलेश्वरम्}
{तथाऽभ्यषिञ्चद्भगवान्सर्वलोकपितामहः}


\twolineshloka
{कश्यपश्च महातेजा ये चान्ये सोककीर्तिताः}
{तस्मै ब्रह्मा ददौ प्रीतो बलिनो वातरंहसः}


\twolineshloka
{कामवीर्यधरान्सिद्धान्महापारिषदान्प्रभुः}
{नन्दिसेनं लोहिताक्षं घण्टाकार्णं च सम्मतम्}


\twolineshloka
{चतुर्थमस्यानुचरं ख्यातं कुमुदमालिनम्}
{तत्र स्थाणुर्महातेजा महापारिषदं प्रभुः}


\twolineshloka
{मायाशतधरं कामं कामवीर्यबलान्वितम्}
{ददौ स्कन्दाय राजेन्द्र सुरारिविनिबर्हणम्}


\twolineshloka
{सहि देवासुरे युद्धे दैत्यानां भीमकर्मणाम्}
{जघान दोर्भ्यां सङ्क्रुद्धः प्रयुतानि चतुर्दश}


\twolineshloka
{तथा देवा ददुस्तस्मै सेनां नैर्ऋतसङ्कुलाम्}
{देवशत्रुक्षयकरीमजय्यां विश्वरूपिणीम्}


\twolineshloka
{जयशब्दं तथा चक्रुर्देवाः सर्वे सवासवाः}
{गन्धर्वा यक्षरक्षांसि मुनयः पितरस्तथा}


\twolineshloka
{ततः प्रादादनुचरौ यमः कालोपमावुभौ}
{उन्माथं च प्रमाथं च महावीर्यौ महाद्युती}


\twolineshloka
{सुभ्राजो भास्वरश्चैव यौ तौ सूर्यानुयायिनौ}
{तौ सूर्यः कार्तिकेयाय ददौ प्रीतः प्रतापवान्}


\twolineshloka
{कैलासशृङ्गसङ्काशौ श्वेतमाल्यानुलेपनौ}
{सोमोऽप्यनुचरौ प्रादान्मणिं सुमणिमेव च}


\twolineshloka
{ज्वालाजिह्वं तथा ज्योतिरात्मजाय हुताशनः}
{ददावनुचरौ शूरौ परसैन्यप्रमाथिनौ}


\twolineshloka
{परिधं च वटं चैव भीमं च सुमहाबलम्}
{दहतिं दहनं चैव प्रचण्डौ वीर्यसम्मतौ}


\twolineshloka
{अंशोऽप्यनुचरान्पञ्च ददौ स्कन्दाय धीमते}
{उत्क्रोशं सत्करं चैव वज्रदण्डधरावुभौ}


\twolineshloka
{ददावनलपुत्राय वासवः परवीरहा}
{तौ हि शत्रून्महेन्द्रास्य जघ्नतुः समरे बहून्}


\twolineshloka
{चक्रं विक्रमकं चैव सङ्क्रमं च महाबलम्}
{स्कन्दाय त्रीननुचरान्ददौ विष्णुर्महायशाः}


\twolineshloka
{वर्धनं नन्दनं चैव सर्वविद्याविशारदौ}
{स्कन्दाय ददतुः प्रीतावश्विनौ भिषजां वरौ}


\twolineshloka
{किन्दुं च कुसुमं चैव कुमुदं च महायशाः}
{डम्बराडम्बरौ चैव ददौ धाता महात्मने}


\twolineshloka
{वक्रानुवक्रौ बलिनौ मेषवक्त्रौ बलोत्कटौ}
{ददौ त्वष्टा महामायौ स्कन्दायानुचरावुभौ}


\twolineshloka
{सुव्रतं सत्यसन्धं च ददौ मित्रो महात्मने}
{कुमाराय महात्मानौ तपोविद्याधरौ प्रभुः}


\twolineshloka
{सुदर्शनीयौ वरदौ त्रिषु लोकेषु विश्रुतौ}
{सुव्रतं च महात्मानं शुभकर्माणमेव च}


\twolineshloka
{कार्तिकेयाय सम्प्रादाद्विधाता लोकविश्रुतौ}
{पाणीतकं कालिकं च महामायाविनावुभौ}


\twolineshloka
{पूषा च पार्षदौ प्रादात्कार्तिकेयाय भारत}
{बलं चातिबलं चैव महावक्त्रौ महाबलौ}


\twolineshloka
{प्रददौ कार्तिकेयाय वायुर्भरतसत्तम}
{यमं चातियमं चैव तिमिवक्त्रौ महाबलौ}


\twolineshloka
{प्रददौ कार्तिकेयाय वरुणः सत्यसङ्गरः}
{सुवर्चसं महात्मानं तथैवाप्यतिवर्चसम्}


\twolineshloka
{हिमवान्प्रददौ राजन्हुताशनसुताय वै}
{काञ्चनं च महात्मानं मेघमालिनमेव च}


\twolineshloka
{ददावनुचरौ मेरुरग्निपुत्राय भारत}
{स्थिरं चातिस्थिरं चैव मेरुरेवापरौ ददौ}


\twolineshloka
{महात्मा त्वग्निपुत्राय महाबलपराक्रमौ}
{उच्छृङ्गं चातिशृङ्गं च महापाषाणयोधिनौ}


\twolineshloka
{प्रददावग्निपुत्राय विन्ध्यः पारिषदावुभौ}
{सङ्ग्रहं विग्रहं चैव समुद्रोऽमि गदाधरौ}


\twolineshloka
{प्रददावग्निपुत्राय महापारिषदावुभौ}
{उन्मादं शङ्कुकर्णं च पुष्पदन्तं तथैव च}


\twolineshloka
{प्रददावग्निपुत्राय पार्वती शुभदर्शना}
{जयं महाजयं चैव गङ्गा ज्वलनसूनवे}


\twolineshloka
{प्रददौ पुरुषव्याघ्र वासुकिः पन्नगेश्वरः}
{एवं साध्याश्च रुद्राश्च वसवः पितरस्तथा}


\twolineshloka
{सागराः सरितश्चैव गिरयश्च महाबलाः}
{ददुः सेनागणाध्यक्षाञ्शूलपट्टसधारिणः}


\twolineshloka
{दिव्यप्रहरणोपेतान्नानावेषविभूषितान्}
{शृणु नामानि चाप्येषां येऽन्ये स्कन्दस्य सैनिकाः}


\twolineshloka
{विविधायुधसम्पन्नाश्चित्राभरणभूषिताः}
{शङ्कुकर्णो निकुम्भश्च पद्मः कुमुद एव च}


\twolineshloka
{अनन्तो द्वादशभुजस्तथा कृष्णोपकृष्णकौ}
{घ्राणश्रवाः कपिस्कन्धः काञ्चनाक्षो जलन्धमः}


\twolineshloka
{अक्षः सन्तर्जनो राजन्कुनदीकस्तमोन्तकृत्}
{एकाक्षो द्वाशाक्षश्च तथैवैकजटः प्रभुः}


\twolineshloka
{सहस्रबाहुर्विकटो व्याघ्राक्षः क्षितिकम्पनः}
{पुण्यनामा सुनामा च सुचक्रः प्रियदर्शनः}


\twolineshloka
{परिश्रुतः कोकनदः प्रियमाल्यानुलेपनः}
{अजोदरो गजशिराः स्कन्धाक्षः शतलोचनः}


\twolineshloka
{ज्वालाजिह्वः करालाक्षः शितिकेशो जटी हरिः}
{परिश्रुतः कोकनदः कृष्णकेशो जटाधरः}


\twolineshloka
{चतुर्दंष्ट्रोऽष्टजिह्वश्च मेघनादः पृथुश्रवाः}
{विद्युताक्षो धनुर्वक्त्रो जाठरो मारुताशनः}


\twolineshloka
{उदाराक्षो रथाक्षश्च वज्रनाभो वसुप्रभः}
{समुद्रवेगो राजेन्द्र शैलकम्पी तथैव च}


\twolineshloka
{वृषो मेषः प्रवाहश्च तथा नन्दोपनन्दकौ}
{धूम्रः श्वेतः कलिङ्गश्च सिद्धार्थो वरदस्तथा}


\twolineshloka
{प्रियकश्चैव नन्दश्च गोनन्दश्च प्रतपवान्}
{आनन्दश्च प्रमोदश्च स्वस्तिको ध्रुवकस्तथा}


\twolineshloka
{क्षेमवाहः सुवाहश्च सिद्धपात्रश्च भारत}
{गोव्रजः कनकापीडो महापारिषदेश्वरः}


\twolineshloka
{गायनो हसनश्चैव बाणः खङ्गश्च वीर्यवान्}
{वैताली गतिताली च तथा कथकवातिकौ}


\twolineshloka
{हंसजः पङ्कदिग्धाङ्गः समुद्रोन्मादनश्च ह}
{रणोत्कटः प्रहासश्च श्वेतसिद्धश्च नन्दनः}


\twolineshloka
{कालकण्ठः प्रभासश्च तथा कुम्भाण्डकोदरः}
{कालकक्षः सितश्चैव भूतानां मथनस्तथा}


\twolineshloka
{यज्ञवाहः सुवाहश्च देवयाजी च सोमपः}
{मज्जानश्च महातेजाः क्रथक्राथौ च भारत}


\twolineshloka
{तुहरश्च तुहारश्च चित्रदेवश्च वीर्यवान्}
{मधुरः सुप्रसादश्च किरीटी च महाबलः}


\twolineshloka
{वत्सलो मधुवर्णश्च कलशोदर एव च}
{धर्मदो मन्मथकरः सूचीवक्त्रश्च वीर्यवान्}


\twolineshloka
{श्वेतवक्त्रः सुवक्त्रश्च चारुवक्त्रश्च पाण्डुरः}
{दण्डबाहुः सुबाहुश्च रजः कोकिलकस्तथा}


\twolineshloka
{अचलः कनकाक्षश्च बालानामपि यः प्रभुः}
{सञ्चारकः कोकनदो गृध्रपत्रश्च जम्बुकः}


\twolineshloka
{लोहाजवक्त्रो जवनः कुम्भवक्त्रश्च कुम्भकः}
{स्वर्णग्रीवश्च कृष्णौजा हंसवक्त्रश्च चन्द्रभः}


\twolineshloka
{पाणिकूर्चाश्च शम्बूकः पञ्चवक्त्रश्च शिक्षकः}
{चाषवक्त्रक्ष जम्बूकः शाकवक्त्रश्च कुञ्जलः}


\twolineshloka
{योगयोक्ता महात्मानः सततं ब्राह्मणप्रियाः}
{पैतामहा महात्मानो महापारिषदाश्च ये}


\twolineshloka
{यौवनस्थाश्च बालाश्च वृद्धाश्च जनमेजय}
{सहस्रशः पारिषदाः कुमारमुपतस्थिरे}


\twolineshloka
{वक्त्रैर्नानाविधैर्ये तु शृणु ताञ्चनमेजय}
{कूर्मकुक्कुटवक्त्राश्च शशोलूकमुखास्तथा}


\twolineshloka
{खरोष्ट्रवदनाश्चान्ये वराहवदनास्तथा}
{मार्जारशशवक्ताश्च दीर्घवक्त्राश्च भारत}


\twolineshloka
{नकुलोलूकवक्त्राश्च काकवक्त्रास्तथा परे}
{आखुबभ्रुकवक्त्राश्च मयूरवदनास्तथा}


\twolineshloka
{मत्स्यमेषाननाश्चान्ये अजाविमहिषाननाः}
{ऋक्षशार्दूलवक्त्राश्च द्वीतिसिंहाननास्तथा}


\twolineshloka
{भीमा गजाननाश्चैव तथा नक्रमुखाश्च ये}
{गरुडाननाः कङ्कुमुखा वृककाकमुखास्तथा}


\twolineshloka
{गोखरोष्ट्रमुखाश्चान्ये वृषदंशमुखास्तथा}
{महाजठरपादाङ्गास्तारकाक्षाश्च भारत}


\twolineshloka
{पारावतमुखाश्चान्ये तथा वृषमुखाः परे}
{कोकिलाभाननाश्चान्ये श्येनतित्तिरिकाननाः}


\twolineshloka
{कृकलासमुखाश्चैव विरजोम्बरधारिणः}
{व्यालवक्त्राः शूलमुखाश्चण्डवक्त्राः शुभाननाः}


\twolineshloka
{आशीविषाश्चीरधरा गोनासावदनास्तथा}
{स्थूलोदराः कृशाङ्गाश्च स्थूलाङ्गाश्च कृशोदराः}


\twolineshloka
{हस्वग्रीवा महाकर्णा नानाव्यालविभूषणाः}
{गजेन्द्रचर्मवसनास्तथा कृष्णाजिनाम्बराः}


\twolineshloka
{स्कन्धेमुखा महाराज तथाप्युदरतोमुखाः}
{पृष्ठेमुखा हनुमुखास्तथा जङ्घामुखा अपि}


\twolineshloka
{पार्श्वाननाश्च बहवो नानादेशमुखास्तथा}
{तथा कीटपतङ्गानां सदृशास्या गणेश्वराः}


\twolineshloka
{नानाव्यालमुखाश्चान्ये बहुबाहुशिरोधराः}
{नानावृक्षभुजाः केचित्कटिशीर्षास्तथा परे}


\twolineshloka
{भुजङ्गभोगवदना नानागुल्मनिवासिनः}
{चीरसंवृतगात्राश्च नानाकनकवाससः}


\twolineshloka
{नानावेषधराश्चैव नानामाल्यानुलेपनाः}
{नानावस्त्रधराश्चैव चर्मवासस एव च}


\twolineshloka
{उष्णीषिणो मुकुटिनः सुग्रीवाश्च सुवर्चसः}
{किरीटिनः पञ्चशिखास्तथा काञ्चनमूर्धजाः}


\twolineshloka
{त्रिशिखा द्विशिखाश्चैव तथा सप्तशिखाः परे}
{शिखण्डिनो मुकुटिनो मुण्डाश्च जटिलास्तथा}


\twolineshloka
{चित्रमालाधराः केचित्केचिद्रोमाननास्तथा}
{विग्रहैकरसा नित्यमजेताः सुरसत्तमैः}


\twolineshloka
{कृष्णा निर्मांसवक्त्राश्च दीर्घपृष्ठास्तनूदराः}
{स्थूलपृष्ठा हस्वपृष्ठाः प्रलम्बोदरमेहनाः}


\twolineshloka
{महाभुजा हस्वभुजा हस्वगात्राश्च वामनाः}
{कुब्जाश्च हस्वजङ्घाश्च हस्तिकर्णशिरोधराः}


\twolineshloka
{हस्तिनासाः कूर्मनासा वृकनासास्तथा परे}
{दीर्घोच्छ्वासा दीर्घजङ्घा विकराला ह्यधोमुखाः}


\twolineshloka
{महादंष्ट्रा हस्वदंष्ट्राश्चतुर्दंष्ट्रास्तथा परे}
{वारमेन्द्रनिभाश्चान्ये भीमा राजन्सहस्रशः}


\twolineshloka
{सुविभक्तशरीराश्च दीप्तिमन्तः स्वलङ्कृताः}
{पिङ्गाक्षाः शङ्कुकर्णाश्च रक्तनासाश्च भारत}


\twolineshloka
{पृथुदंष्ट्रा महादंष्ट्राः स्थूलौष्ठा हरिमूर्धजाः}
{नानापादौष्ठदंष्ट्राश्च नानाहस्तशिरोधराः}


\twolineshloka
{नानाचर्मभिराच्छन्ना नानाभाषाश्च भारत}
{कुशला देशभाषासु जल्पन्तोऽन्योन्यमीश्वराः}


\twolineshloka
{हृष्टाः परिपतन्ति स्म महापारिषदास्तथा}
{दीर्घग्रीवा दीर्घनखा दीर्घपादशिरोभुजाः}


\twolineshloka
{पिङ्गाक्षा नीलकण्ठाश्च लम्बकर्णाश्च भारत}
{वृकोदरनिभाश्चैव केचिदञ्जनसन्निभाः}


\twolineshloka
{श्वेताक्षा लोहितग्रीवाः पिङ्गाक्षाश्च तथा परे}
{कल्माषा बहवो राजंश्चित्रवर्णाश्च भारत}


\twolineshloka
{चामरापीडकनिभाः श्वेतलोहितराजयः}
{नानावर्णाः सवर्णाश्च मयूरसदृशप्रभाः}


\twolineshloka
{पुनः प्रहरणान्येषां कीर्त्यमानानि मे शृणु}
{शेषैः कृतः पारिषदैरायुधानां परिग्रहः}


\twolineshloka
{पाशोद्यतकराः केचिद्व्यादितास्याः खराननाः}
{पृष्ठाक्षा नीलकण्ठाश्च तथा परिघबाहवः}


\twolineshloka
{शतघ्नीचक्रहस्ताश्च तथा मुसलपाणयः}
{असिमुद्ग्ररहस्ताश्च दण्डहस्ताश्च भारत}


\twolineshloka
{गदाभुशुण्डिहस्ताश्च तथा तोमरपाणयः}
{आयुधैर्विविधैर्घोरैर्महात्मानो महाजवाः}


\twolineshloka
{महाबला महावेगा महापारिषदास्तथा}
{अभिषेकं कुमारस्य दृष्ट्वा हृष्टा रणप्रियाः}


\twolineshloka
{घण्टाजालपिनद्धाङ्गा ननृतुस्ते महौजसः}
{एते चान्ये च बहवो महापारिषदा नृप}


\twolineshloka
{उपतस्थुर्महात्मानं कार्तिकेयं यशस्विनम्}
{दिव्याश्चाप्यान्तरिक्षाश्च पार्थिवाश्चानिलोपमाः}


\threelineshloka
{व्यादिष्टा दैवतैः शूराः स्कन्दस्यानुचराऽभवन्}
{तादृशानां सहस्राणि प्रयुतान्यर्बुदानि च}
{अभिषिक्तं महात्मानं परिवार्योपतस्थिरे}


\chapter{अध्यायः ४८}
\twolineshloka
{वैशम्पायन उवाच}
{}


\twolineshloka
{शृणु मातृगणान्राजन्कुमारानुचरानिमान्}
{कीर्त्यमानान्मया वीर सपत्नगणसूदनान्}


% Check verse!
यशस्विनीनां मातॄणां शृणु नामानि भारतयाभिर्वायप्तास्त्रयोलोकाः कल्याणीभिश्च भागशः
\twolineshloka
{प्रभावती विशालाक्षी पालिता गोस्तानी तथा}
{श्रीमती बहुला चैव तथैव बहुपुत्रिका}


\twolineshloka
{अप्सुजाता च गोपाली बृहदम्बालिका तथा}
{जयावती मालतिका ध्रुवरत्ना भयङ्करी}


\twolineshloka
{वसुदामा च दामा च विशोका नन्दिनी तथा}
{एकचूडा महाचूडा चक्रनेमिश्च भारत}


\twolineshloka
{उत्तेजनी जयत्सेना कमलाक्ष्यथ शोभना}
{शत्रुञ्जया तथा चैव क्रोधना शलभी खरी}


\twolineshloka
{माधवी शुभवक्त्रा च तीर्थसेनिश्च भारत}
{गीतप्रिया च कल्याणी रुद्ररोमाऽमिताशना}


\twolineshloka
{मेघस्वना भोगवती सुभ्रूश्च कनकावती}
{अलाताक्षी वीर्यवती विद्युज्जिह्वा च भारत}


\twolineshloka
{पुन्नावती सुनक्षत्रा कन्दरा बहुयोजना}
{सन्तानिका च कौरव्य कमला च महाबला}


\twolineshloka
{सुदामा बहुदामा च सुप्रभा च यशस्विनी}
{नृत्यप्रिया च राजेन्द्र शतोलूखलमेखला}


\twolineshloka
{शतघण्टा शतानन्दा भगनन्दा च भाविनी}
{वपुष्मती चन्द्रशीता भद्रकाली च भारत}


\twolineshloka
{ऋक्षाम्बिका निष्कुटिका वामा चत्वरवासिनी}
{सुमङ्गला स्वस्तिमती बुद्धिकामा जयप्रिया}


\twolineshloka
{धनदा सुप्रसादा च भवदा च जलेश्वरी}
{एडी भेडी समेडी च वेतालजननी तथा}


\twolineshloka
{कण्डूतिः कालिका चैव देवमित्रा च भारत}
{वसुश्रीः कोटरा चैव चित्रसेना तथाऽचला}


\twolineshloka
{कुक्कुटिका शङ्खलिका तथा शकुनिका नृप}
{कुण्डारिका कौकुलिका कुम्भिकाऽथ शतोदरी}


\twolineshloka
{उत्क्राथिनी जलेला च महावेगा च कङ्कणा}
{मनोजवा कण्टकिनी प्रघसा पूतना तथा}


\twolineshloka
{केशयन्त्री त्रुटिर्वामा क्रोशनाऽथ तडित्प्रभा}
{मन्दोदरी च मुण्डी च कोटरा मेघवाहिनी}


\twolineshloka
{सुभगा लम्बिनी लम्बा ताम्रचूडा विकाशिनी}
{ऊर्ध्ववेणीधरा चैव पिङ्गाक्षी लोहमेखला}


\twolineshloka
{पृथुवस्त्रा मधुलिका मधुकुम्भा तथैव च}
{पक्षालिका मत्कुलिका जरायुर्जर्जरानना}


\twolineshloka
{ख्याता दहदहा चैव तथा धमधमा नृप}
{खण्डखण्डा च राजेन्द्र पूषणा मणिकुट्टिका}


\twolineshloka
{अमोघा चैव कौरव्य तथा लम्बपयोधरा}
{वेणुवीणाधरा चैव पिङ्गाक्षी लोहमेखला}


\twolineshloka
{शशोकूलमुखी कृष्णा खरजङ्घा महाजवा}
{शिशुमारमुखी श्वेता लोहिताक्षी विभीषणा}


\twolineshloka
{जटालिका कामचरी दीर्घजिह्वा बलोत्कटा}
{कालेहिका वामनिका मुकुटा चैव भारत}


\twolineshloka
{लोहिताक्षी महाकाया हरिपिण्डा च भूमिप}
{एकत्वचा सुकुसुमा कृष्णकर्णी च भारत}


\twolineshloka
{क्षुरकर्णी चतुष्कर्णी कर्णप्रावरणा तथा}
{चतुष्पथनिकेता च गोकर्णी महिषानना}


\twolineshloka
{खरकर्णी महाकर्णी भेरीस्वनमहास्वना}
{शङ्खकुम्भश्रवाश्चैव भगदा च महाबला}


\twolineshloka
{गणा च सुगणा चैव तथा भीत्यथ कामदा}
{चतुष्पथरता चैव भूतितीर्थान्यगोचरी}


\twolineshloka
{पशुदा वित्तदा चैव सुखदा च महायशाः}
{पयोदा गोमहिषदा सविशाला च भारत}


\twolineshloka
{प्रतिष्ठा सुप्रतिष्ठा च रोचमाना सुरोचना}
{नौकर्णी मुखकर्णी च विशिरा मन्थिनी तथा}


\twolineshloka
{एकचन्द्रा मेघकर्णा मेघमाला विरोचना}
{एताश्चान्याश्च बहवो मातरो भरतर्षभ}


\twolineshloka
{कार्तिकेयानुयायिन्यो नानारूपाः सहस्रशः}
{दीर्घनख्यो दीर्घदन्त्यो दीर्घतुण्ड्श्च भारत}


\twolineshloka
{सबला मधुराश्चैव यौवनस्थाः स्वलङ्कृताः}
{माहात्म्येन च संयुक्ताः कामरूपधरास्तथा}


\twolineshloka
{निर्मासगात्र्यः श्वेताश्च तथा काञ्चनसन्निभाः}
{कृष्णमेघनिभाश्चान्या धूम्नाश्च भरतर्षभ}


\twolineshloka
{अरुणाभा महाभोगा दीर्घकेश्यः सिताम्बराः}
{ऊर्ध्ववेणीधराश्चैव पिङ्गाक्ष्यो लम्बमेखलाः}


\twolineshloka
{लम्बोदर्यो लम्बकर्णास्तथा लम्बपयोधराः}
{ताम्राक्ष्यस्ताम्रवर्णाश्च हर्यक्ष्यश्च तथाऽपराः}


\twolineshloka
{वरदाः कामचारिण्यो नित्यं प्रमुदितास्तथा}
{याम्या रौद्रास्तथा सौम्याः कौबेर्योऽथ महाबलाः}


\twolineshloka
{वारुण्योऽथ च माहेन्द्यस्तथाऽऽग्नेय्यः परन्तप}
{वायव्यश्चाथ कौमार्यो ब्राह्मश्च भरतर्षभ}


\twolineshloka
{वैष्णव्यश्च तथा सौर्यो वाराह्यश्च महाबलाः}
{रूपेणाप्सरसां तुल्या मनोहार्यो मनोरमाः}


\twolineshloka
{परपुष्टोपमा वाक्ये तथर्द्ध्या धनदोपमाः}
{शक्रवीर्योपमा युद्धे दीप्ता वह्निसमास्तथा}


\twolineshloka
{शत्रुणां विग्रहे नित्यं भयदास्ता भवन्त्युत}
{कामरूपधराश्चैव जवे वायुसमास्तथा}


\twolineshloka
{अचिन्त्यबलवीर्याश्च तथाऽचिन्त्यपराक्रमाः}
{वृक्षचत्वरवासिन्यश्चतुष्पथनिकेतनाः}


\twolineshloka
{गुहाश्मशानवासिन्यः शैलप्रस्रवणालयाः}
{नानाभरणधारिण्यो नानामाल्याम्बरास्तथा}


\twolineshloka
{नानाविचित्रवेषाश्च नानाभाषास्तथैव च}
{एते चान्ये च मातॄणां गणाः शत्रुभयङ्कराः}


\twolineshloka
{अनुजग्मुर्महात्मानं त्रिदशेन्द्रस्य सम्मते}
{ततः शक्त्यस्त्रमददद्भगवान्पाकशासनः}


\twolineshloka
{गुहाय राजशार्दूल विनाशाय सुरद्विषाम्}
{महास्वनां महाघण्टां द्योतमानां सितप्रभाम्}


\twolineshloka
{अरुणादित्यवर्णां च पताकां भरतर्षभ}
{ददौ पशुपतिस्तस्मै सर्वभूतमहाचमूम्}


\twolineshloka
{उग्रां नानाप्रहरणां तपोवीर्यबलान्विताम्}
{अजेयां स्वगणैर्युक्तां नाम्ना सेनां धनञ्जयाम्}


\twolineshloka
{रुद्रतुल्यबलैर्युक्तां योधानामयुतैस्त्रिभिः}
{न सा विजानाति रणात्कदाचिद्विनिवर्तितुम्}


\twolineshloka
{विष्णुर्ददौ वैजयन्तीं मालां बलविवर्धिनीम्}
{उमा ददौ विरजसी वाससी रविसप्रभे}


\twolineshloka
{गङ्गा कमण्डलुं दिव्यममृतोद्भवमुत्तमम्}
{ददौ प्रीत्या कुमाराय दण्डं चैव बृहस्पतिः}


\twolineshloka
{गरुडो दयितं पुत्रं मयरं चित्रबर्हिणम्}
{अरुणस्ताम्रचूडं च प्रददौ चरणायुधम्}


\threelineshloka
{छागं तु वरुणो राजा बलवीर्यसमन्वितम्}
{कृष्णाजिनं ततो ब्रह्मा ब्रह्मण्याय ददौ प्रभुः}
{समरेषु जयं चैव प्रददौ लोकभावनः}


\twolineshloka
{सैनापत्यमनुप्राप्य स्कन्दो देवगणस्य ह}
{शुशुभे ज्वलितोर्चिष्मान्द्वितीय इव पावकः}


\twolineshloka
{ततः पारिषदैश्चैव मातृभिश्च समन्वितः}
{ययौ देत्यविनाशाय ह्लादयन्सुरपुङ्गवान्}


\twolineshloka
{सा सेना नैर्ऋती भीमा सघण्टोच्छ्रितकेतना}
{सभेरीशङ्खमुरजा सायुधा सपताकिनी}


% Check verse!
शारदी द्यौरिवाभाति ज्योतिर्भिरिव शोभिता
\twolineshloka
{ततो देवनिकायास्ते नानाभूतगणास्तथा}
{वादयामासुरव्याग्रा भेरीः शङ्खांश्च पुष्कलान्}


\twolineshloka
{पटहान्झर्झरांश्चैव क्रकचान्गोविषाणिकान्}
{आडम्बरान्गोमुखांश्च डिण्डिमांश्च महास्वनान्}


\twolineshloka
{तुष्टुवुस्ते कुमारं तु सर्वे देवाः सवासवाः}
{जगुश्च देवगन्धर्वा ननृतुश्चाप्सरोगणाः}


\twolineshloka
{ततः प्रीतो महासेनस्त्रिदशेभ्यो वरं ददौ}
{रिपून्दन्ताऽस्मि समरे ये वो वधचिकीर्षवः}


\twolineshloka
{प्रतिगृह्य वरं देवास्तस्माद्विबुधसत्तमात्}
{प्रीतात्मानो महात्मानो मेनिरे निहतान्रिपून्}


\twolineshloka
{सर्वेषां भूतसङ्घानां हर्षान्नादः समुत्थितः}
{अपूरयत लोकांस्त्रीन्वरे दत्ते महात्मना}


\twolineshloka
{स निर्ययौ महासेनो महत्या सेनया वृतः}
{वधाय युधि दैत्यानां रक्षार्थं च दिवोकसाम्}


\twolineshloka
{व्यवसायो जयो धर्मः सिद्धिर्लक्ष्मीर्धृतिः स्मृतिः}
{महासेनस्य सैन्यानामग्रे जग्मुर्नराधिप}


\twolineshloka
{स तया भीमया देवः शुलमुद्गरहस्तया}
{ज्वलितालातधारिण्या चित्राभरणवर्मया}


\twolineshloka
{गदामुसलनाराचशक्तितोमरहरतया}
{दृप्तसिंहनिनादिन्या विनद्य प्रययौ गुहः}


\twolineshloka
{`तं दष्ट्वा सर्वदैतेया राक्षसा दानवास्तथा}
{व्यद्रवन्त दिशः सर्वा भयोद्विग्नाः समन्ततः}


\twolineshloka
{अभ्यद्रवन्त देवास्तान्विविधायुधपाणयः}
{दृष्ट्वा च स ततः क्रुद्धः स्कन्दस्तेजोबलान्वितः}


\twolineshloka
{शक्त्यस्त्रं भगवान्भीमं पुनःपुनरवाकिरत्}
{आदधच्चात्मनस्तेजो हविषेद्ध इवानलः}


\twolineshloka
{अभ्यस्यमाने शक्त्यस्त्रे स्कन्देनामिततेजसा}
{उल्काज्वाला महाराज पपात वसुधातले}


\twolineshloka
{संहादयन्तश्च तथा निर्घाताश्चापतन्क्षिपौ}
{यथान्तकालसमये सुघोराः स्युस्तथा नृप}


\twolineshloka
{क्षिप्ता ह्येका यदा शक्तिः सुघोराऽनलसूनुना}
{ततः कोट्यो विनिष्पेतुः शक्तीनां भरतर्षभ}


\twolineshloka
{ततः प्रीतो महासेनो जघान भगवान्प्रभुः}
{दैत्येन्द्रं तारकं नाम महाबलपराक्रमम्}


\twolineshloka
{वृतं दैत्यायुतैर्वीरैर्बलिभिर्दशभिर्नृप}
{महिषं चाष्टभिः पद्मैर्वृतं सङ्ख्ये निजघ्निवान्}


\twolineshloka
{त्रिपादं चायुतशतैर्जघान दशभिर्वृतम्}
{हदोदरं निखर्वैश्च वृतं दशभिरीश्वरः}


\twolineshloka
{जघानानुचरैः सार्धं विविधायुधपाणिभिः}
{तथाऽकुर्वन्त विपुलं नादं वध्यत्सु शत्रुषु}


\twolineshloka
{कुमारानुचरा राजन्पूरयन्तो दिशो दश}
{ननृतुश्च ववल्गुश्च जहसुश्च मुदान्विताः}


\twolineshloka
{शक्त्यस्त्रस्य तु राजेन्द्र ततोऽर्चिर्भिः समन्ततः}
{त्रैलोक्यं त्रासितं सर्वं जृम्भमाणाभिरेव च}


\twolineshloka
{दग्धाः सहस्रशो दैत्या नादैः स्कन्दस्य चापरे}
{पताकयावधूताश्च हताः केचित्सुरद्विषः}


\twolineshloka
{केचिद्धण्टारवत्रस्ता निषेदुर्वसुधातले}
{केचित्प्रहरणैश्छिन्ना विनिष्पेतुर्गतायुषः}


\twolineshloka
{एवं सुरद्विषोऽनेकान्बलवानाततायिनः}
{जघान समरे वीरः कार्तिकेयो महाबलः}


\twolineshloka
{बाणो नामाथ दैतेयो बलेः पुत्रो महाबलः}
{क्रौञ्चं पर्वतमाश्रित्य देवसङ्घानबाधत}


\twolineshloka
{तमभ्ययान्महासेनः सुरशत्रुमुदारधीः}
{स कार्तिकेयस्य भयात्क्रौञ्चं शरणमीयिवान्}


\twolineshloka
{ततः क्रौञ्चं महामन्युः क्रौञ्चनादनिनादितम्}
{शक्त्या बिभेद भगवान्कार्तिकेयोऽग्निदत्तया}


\twolineshloka
{ससालस्कन्धशबलं त्रस्तवानरवारणम्}
{प्रोङ्कीनोद्धान्तविहगं विनिष्पतितपन्नगम्}


\twolineshloka
{गोलाङ्गूलर्क्षसङ्घैश्च द्रवद्भिरनुनादितम्}
{कुरङ्गमविनिर्घोषनिनादितवनान्तरम्}


\twolineshloka
{विनिष्पतद्भिः शरभैः सिंहैश्च सहसा द्रुतैः}
{शोच्यामपि दशां प्राप्तो रराजेव सपर्वतः}


\twolineshloka
{विद्याधराः समुत्पेतुस्तस्य शृङ्गनिवासिनः}
{किन्नराश्च समुद्विग्नाः शक्तिपातरवोद्धताः}


\twolineshloka
{ततो दैत्या विनिष्पेतुः शतशोऽथ सहस्रशः}
{प्रदीप्तात्पर्वतश्रेष्ठाद्विचित्राभरणस्रजः}


\twolineshloka
{तान्निजघ्नुरतिक्रम्य कुमाराजुचरा मृधे}
{स चैव भगवान्क्रुद्धो दैत्येन्द्रस्य सुतं तदा}


\twolineshloka
{सहानुजं जघानाशु वृत्रं देवपतिर्यथा}
{बिभेद क्रौञ्चं शक्त्या च पावकिः परवीरहा}


\twolineshloka
{बहुधा चैकधा चैव कृत्वाऽऽत्मानं महाबलः}
{शक्तिः क्षिप्ता रणे तस्य पाणिमेति पुनः पुनः}


\twolineshloka
{एवम्प्रभावो भगवांस्ते भूयश्च पावकिः}
{शौर्याद्द्विगुणयोगेन तेजसा यशसा श्रिया}


% Check verse!
क्रौञ्चस्ते विनिर्भिन्नो दैत्याश्च शतशो हताः
\twolineshloka
{ततः स भगवान्देवो निहत्य विबुधद्विषः}
{सभाज्यमानो विबुधैः परं हर्षमवाप ह}


\threelineshloka
{ततो दुन्दुभयो राजन्नेदुः शङ्खाश्च भारत}
{मुमुचुर्देवयोषाश्च पुष्पवर्षमनुत्तमम्}
{योगिनामीश्वरं देवं शतशोऽथ सहस्रशः}


\twolineshloka
{दिव्यगन्धमुपादाय ववौ पुण्यश्च मारुतः}
{गन्धर्वास्तुष्टुवुश्चैनं यज्वानश्च महर्षयः}


\twolineshloka
{केचिदनं व्यवस्यान्ति पितामहसुतं प्रभुम्}
{सनत्कुमारं सर्वेषां ब्रह्मयोनिं तमग्रजम्}


\twolineshloka
{केचिन्महेश्वरसुतं केचित्पुत्रं विभावसोः}
{उमायाः कृत्तिकानां च गङ्गायाश्च वदन्त्युत}


\twolineshloka
{एकधा च द्विधा चैव चतुर्धा च महाबलम्}
{योगिनामीश्वरं देवं शतशोऽथ सहस्रशः}


\twolineshloka
{एतत्ते कथितं राजन्कार्तिकेयाभिषेचनम्}
{शृणु चैव सरस्वत्यास्तीर्थवंशस्य पुण्यताम्}


\twolineshloka
{बभूव तीर्थप्रवरं हतेषु सुरशत्रुषु}
{कुमारेण महाराज त्रिविष्टपमिवापरम्}


\twolineshloka
{ऐश्वर्याणि च तत्रस्थो ददावीशः पृथक्पृथक्}
{ददौ नैर्ऋतमुख्येभ्यस्त्रैलोक्यं पावकात्मजः}


\twolineshloka
{एवं स भगवांस्तस्मिंस्तीर्थे दैत्यकुलान्तकः}
{अभिषिक्तो महाराज देवसेनापतिः सुरैः}


\twolineshloka
{औशनं नाम तत्तीर्थं यत्र पूर्वमपाम्पतिः}
{अभिषिक्तः सुरगणैर्वरुणो भरतर्षभ}


\twolineshloka
{अस्मिंस्तीर्थवरे स्नात्वा स्कन्दं चाभ्यर्च्य लाङ्गली}
{ब्राह्मणेभ्योददौ रुक्मं वासांस्याभरणानि च}


\twolineshloka
{उषित्वा रजनीं तत्र माधवः परवीरहा}
{पूज्यतीर्थवरं तच्च स्पृष्ट्वा तोयं च लाङ्गली}


\twolineshloka
{हृष्टः प्रीतमनाश्चैव ह्यभवन्माधवोत्तमः}
{एतत्ते सर्वमाख्यातं यन्मां त्वं परिपृच्छसि}


\twolineshloka
{यथाऽभिषिक्तो भगवान्स्कन्दो देवैः समागतैः}
{`सेनानीश्च कृतो राजन्बाल एव महाबलः'}


\chapter{अध्यायः ४९}
\twolineshloka
{जनमेजय उवाच}
{}


\twolineshloka
{अत्यद्भुतमिदं ब्रह्मञ्श्रुतवानस्मि तत्त्वतः}
{अभिषेकं कुमारस्य विस्तरेण यथाविधि}


\twolineshloka
{यच्छ्रुत्वा पूतमात्मानं विजानामि तपोधन}
{प्रहृष्टानि च रोमाणि प्रसन्नं च मनो मम}


\twolineshloka
{अभिषेकं कुमारस्य दैत्यानां च वधं तथा}
{श्रुत्वा मे परमा प्रीतिर्भूयः कौतूहलं हि मे}


\threelineshloka
{अपाम्पतिः कथं ह्यस्मिन्नभिषिक्तः पुरा सुरैः}
{तन्मे ब्रूहि महाप्रज्ञ कुशलो ह्यसि सत्तम ॥वैशम्पायन उवाच}
{}


% Check verse!
शृणु राजन्निदं चित्रं पूर्वकाले यथाऽभवत्
\twolineshloka
{आदौ कृतयुगे राजन्वर्तमाने यथाविधि}
{वरुणं देवताः सर्वाः समेत्येदमथाब्रुवन्}


\twolineshloka
{यथाऽस्मान्सुरराट् शक्रो भयेभ्यः पाति सर्वदा}
{तथा त्वमपि सर्वासां सरितां वै पतिर्भव}


\twolineshloka
{वासश्च ते सदा देव सागरे मकरालये}
{समुद्रोऽयं तव वशे भविष्यति नदीपतिः}


\twolineshloka
{सोमेन सार्धं च तव हानिवृद्वी भविष्यतः}
{एवमस्त्विति तान्देवान्वरुणो वाक्यमब्रवीत्}


\twolineshloka
{समागम्य ततः सर्वे वरुणं सागरालयम्}
{अपाम्पतिं प्रचक्रुर्हि विधिदृष्टेन कर्मणा}


\twolineshloka
{अभिषिच्य ततो देवा वरुणं यादसाम्पतिम्}
{जग्मुः स्वान्येव स्थानानि पूजयित्वा जलेश्वरम्}


\threelineshloka
{अभिषिक्तस्ततो देवैर्वरुणोऽपि महायशाः}
{सरितः सागरांश्चैव नदांश्चापि सरांसि च}
{पालयामास विधिना यथा देवाञ्शतक्रतुः}


\threelineshloka
{ततस्त्रत्राप्युपस्पृश्य दत्त्वा च विविधं वसु}
{अग्नितीर्थं महाप्राज्ञो जगामाथ प्रलम्बहा}
{नष्टो न दृश्यते यत्र शमीगर्भे दुताशनः}


\twolineshloka
{लोकालोकविनाशे च प्रादुर्भूते तदाऽनघ}
{उपतस्थुः सुरा यत्र सर्वलोकपितामहम्}


\threelineshloka
{अग्निः प्रनष्टो भगवान्कारणं च न विद्महे}
{सर्वभूतक्षयो राजन्सम्पादय विभोऽनलम् ॥जनमेजय उवाच}
{}


\threelineshloka
{किमर्थं भगवानग्निः प्रनष्टो लोकभावनाः}
{विज्ञातश्च कथं देवैस्तन्ममाचक्ष्व तत्त्वतः ॥वैशम्पायन उवाच}
{}


\twolineshloka
{भृगोः शापाद्भृशं भीतो जातवेदाः प्रतापवान्}
{शमीगर्भमथासाद्य ननाश भगवांस्ततः}


\twolineshloka
{प्रनष्टे तु तदा वह्नौ देवाः सर्वे सवासवाः}
{अन्वैषन्त तदा नष्टं ज्वलनं भृशदुःखिताः}


\twolineshloka
{ततोऽग्नितीर्थमासाद्य शमीगर्भस्थमेव हि}
{ददृशुर्ज्वलनं तत्र वसमानं यथाविधि}


\twolineshloka
{देवाः सर्वे नरव्याघ्र बृहस्पतिपुरोगमाः}
{ज्वलनं तं समासाद्य प्रीताऽभूवन्सवासवाः}


\twolineshloka
{पुनर्यथागतं जग्मुः सर्वभक्षश्च सोऽभवत्}
{भृगोः शापान्महाभाग यदुक्तं ब्रह्मवादिना}


% Check verse!
तत्राप्याप्लुत्य मतिमान्ब्रह्मशापान्मुमोच ह
\twolineshloka
{तत्राप्लुत्य ततो ब्रह्मा सह देवैः प्रभुः पुरा}
{ससर्ज चान्नानि तथा देवतानां यथाविधि}


\twolineshloka
{तत्र स्नात्वा च दत्त्वा च वसूनि विविधानि च}
{कौबेरं प्रययौ तीर्थं यत्र तप्त्वा महत्तपः}


% Check verse!
धनाधिपत्यं सम्प्राप्तो राजन्नैलबिलः प्रभुः
\twolineshloka
{तत्रस्थमेव तं राजन्धनानि निधयस्तथा}
{उपतद्धतुर्नरश्रेष्ठ तत्तीर्थं लाङ्गली बलः}


\twolineshloka
{गत्वा स्नात्वा च विधिवद्ब्राह्मणेभ्यो धनं ददौ}
{ददृशे तत्र तत्स्थानं कौबेरे काननोत्तमे}


\twolineshloka
{पुरा यत्र तपस्तप्तं विपुलं सुमहात्मना}
{यक्षराज्ञा कुबेरेण वरा लब्धाश्च पुष्कलाः}


\twolineshloka
{धनाधिपत्यं सख्यं च रुद्रेणामिततेजसा}
{सुरत्वं लोकपालत्वं पुत्रं च नलकूबरम्}


\twolineshloka
{यत्र लेभे महाबाहो धनाधिपतिरञ्जसा}
{अभिषिक्तश्च तत्रैव समागम्य मरुद्गणैः}


\twolineshloka
{वाहनं चास्य तद्दत्तं हंसयुक्तं मनोजवम्}
{विमानं पुष्पकं दिव्यं नैर्ऋतैश्वर्यमेव च}


\twolineshloka
{तत्राप्लुत्य बलो राजन्दत्त्वा दायांश्च पुष्कलान्}
{जगाम त्वरितो रामस्तीर्थं स्वेतानुलेपनः}


\twolineshloka
{निषेवितं सर्वसत्वैर्नाम्ना बदरपाचनम्}
{नानर्तुकवनोपेतं सदा पुष्पफलं शुभम्}


\chapter{अध्यायः ५०}
\twolineshloka
{वैशम्पायन उवाच}
{}


\twolineshloka
{ततस्तीर्थवरं रामो ययौ बदरपाचनाम्}
{तपस्विसिद्धचरितं यत्र कन्या धृतव्रता}


\twolineshloka
{भरद्वाजस्य दुहिता रुपेणाप्रतिमा भुवि}
{श्रुतावती नाम विभो कुमारी ब्रह्मचारिणी}


\twolineshloka
{तपश्चचार सात्युग्रं नियमैर्बहुभिर्वृता}
{भर्ता मे देवराजः स्यादिति निश्चित्य भामिनी}


\twolineshloka
{समास्तस्या व्यतिक्रान्ता बह्व्यः कुरुकुलोद्वह}
{चरन्त्या नियमांस्तांस्तांस्त्रीभिस्तीव्रान्सुदुश्चरान्}


\twolineshloka
{तस्यास्तु तेन वृत्तेन तपसा च विशाम्पते}
{भक्त्या च भगवान्प्रीतः परया पाकशासनः}


\twolineshloka
{आजगामाश्रमं तस्यास्त्रिदशाधिपतिः प्रभुः}
{आस्थाय रूपं विप्रर्षेर्वसिष्ठस्य महात्मनः}


\twolineshloka
{सा तं दृष्ट्वोग्रतपसं वसिष्ठं तपतां वरम्}
{आचारैर्मुनिभिर्दृष्टैः पूजयामास भारत}


\twolineshloka
{उवाच नियमज्ञा च कल्याणी सा प्रियंवदा}
{भगवन्मुनिशार्दूल किमाज्ञापयसि प्रभो}


\twolineshloka
{सर्वमद्य यथाशक्ति तव दास्यामि सुव्रत}
{शक्त्रभक्त्या च ते पाणिं न दास्यामि कथञ्चन}


\twolineshloka
{व्रतैश्च नियमैश्चैव तपसा च तपोधन}
{शक्रस्तोषयितव्यो वै मया त्रिभुवनेश्वरः}


\twolineshloka
{इत्युक्तो भगवान्देवः स्मयन्निव निरीक्ष्य ताम्}
{उवाच नियमं ज्ञात्वा सांत्वयन्निव भारत}


\twolineshloka
{उग्रं तपश्चरसि वै विदिता मेऽसि सुव्रते}
{यदर्थमयमारम्भस्तव कल्याणि हृद्गतः}


\twolineshloka
{तच्च सर्वं यथाभूतं भविष्यति वरानने}
{तपसा लभ्यते सर्वं यथाभूतं भविष्यति}


\twolineshloka
{यथा स्थानानि दिव्यानि विबुधानां शुभानने}
{तपसा तानि प्राप्याणि तपोमूलं महात्सुखम्}


\twolineshloka
{इति कृत्वा तपो घोरं देहं सन्न्यस्य मानवाः}
{देवत्वं यान्ति कल्याणि शृणुष्वैकं वचो मम}


\twolineshloka
{प़ञ्च चैतानि सुभगे बदराणि शुभव्रते}
{पचेत्युक्त्वा तु भगवाञ्जगाम बलसूदनः}


\twolineshloka
{आमन्त्र्य तां तु कल्याणीं ततो जप्यं जजाप सः}
{अविदूरे ततस्तस्मादाश्रमात्तीर्थमुत्तमम्}


\twolineshloka
{तच्च तीर्थं महाराज यत्र जप्यं जजाप सः}
{इन्द्रतीर्थेतिविख्यातं त्रिषु लोकेषु मानद}


\twolineshloka
{तस्य जिज्ञासनार्थं स भगवान्पाकशासनः}
{बदराणामपचनं चकार विबुधाधिपः}


\threelineshloka
{ततः प्रतप्ता सा राजन्वाग्यता विगतक्लमा}
{तत्परा शुचिसंवीता पावके समधिश्रयत्}
{अपचद्राजशार्दूल बदराणि महाव्रता}


\twolineshloka
{तस्याः पचन्त्याः सुमहान्कालोऽगात्पुरुषर्षभ}
{न च स्म तान्यपच्यन्त दिनं च क्षयमभ्यगात्}


\twolineshloka
{हुताशनेन दग्धश्च यस्तस्याः काष्ठसञ्चयः}
{अकाष्ठमग्निं सा दृष्ट्वा स्वशरीरमथादहत्}


\twolineshloka
{पादौ प्रक्षिप्य सा पूर्वं पावके चारुदर्शना}
{दग्धौ दग्धौ पुनः पादावुपावर्तयतानघ}


\twolineshloka
{चरणे दह्यमाने च नाचिन्तयदनिन्दिता}
{दुःखं कमलपत्राक्षी महर्षिप्रियकाम्यया}


\twolineshloka
{न वैमनस्यं तस्यास्तु मुखभेदोऽथवाऽभवत्}
{शरीरमग्निना दीप्य जलमध्ये यथा स्थिता}


\twolineshloka
{तच्चास्याः पचने यत्नं न न्यवर्तत भारत}
{सर्वथा बदराण्येव पक्तव्यानीति कन्यका}


\twolineshloka
{सा तन्मनसि कृत्वैव महर्षेर्वचनं शुभा}
{अपचद्बदराण्येव न चापच्यन्त भारत}


\twolineshloka
{तस्यास्तु चरणौ वह्निर्ददाह भगवान्स्वयम्}
{न च तस्या मनोदुःखं स्वल्पमप्यभवत्तदा}


\twolineshloka
{अथ तत्कर्म दृष्ट्वाऽस्याः प्रीतस्त्रिभुवनेश्वरः}
{ततः सन्दर्शयामास कन्यायै रूपमात्मनः}


\twolineshloka
{उवाच च सुरश्रेष्ठस्तां कन्यां सुदृढव्रताम्}
{प्रीतोऽस्मि ते शुभे भक्त्या तपसा नियमेन च}


\twolineshloka
{तस्माद्योऽभिमतः कामः स ते सम्पत्स्यते शुभे}
{देहं त्यक्त्वा महाभागे त्रिदिवे मयि वत्स्यसि}


\threelineshloka
{इदं च ते तीर्थवरं स्थिरं लोके भविष्यति}
{सर्वपापापहं सुभ्रु नाम्ना बदरपाचनम्}
{विख्यातं त्रिषु लोकेषु ब्रह्मर्षिभिरभिप्लुतम्}


\twolineshloka
{अस्मिन्खलु महाभागे शुभे तीर्थवरेऽनघे}
{त्यक्त्वा सप्तर्षयो जग्मुर्हिमवन्तमरुन्धतीम्}


\twolineshloka
{ततस्ते वै महाभागा गत्वा तत्र सुसंशिताः}
{वृत्त्यर्थं फलमूलानि समाहर्तुं ययुः किल}


\twolineshloka
{तेषां वृत्त्यर्थिनां तत्र वसतां हिमवद्वने}
{अनावृष्टिरनुप्राप्ता तदा द्वादशवार्षिकी}


\twolineshloka
{ते कृत्वा चाश्रमं तत्र न्यवसन्त तपस्विनः}
{अरुन्धत्यपि कल्याणी तपोनित्याऽभवत्तदा}


\twolineshloka
{अरुन्धतीं ततो दृष्ट्वा तीव्रं नियममास्थिताम्}
{अथागमत्त्रिनयनः सुप्रीतो वरदस्तदा}


\twolineshloka
{ब्राह्मं रूपं ततः कृत्वा महादेवो महायशाः}
{तामभ्येत्याब्रवीद्देवो भिक्षामिच्छाम्यहं शुभे}


\twolineshloka
{प्रत्युवाच ततः सा तं ब्राह्मणं चारुदर्शना}
{क्षीणोऽन्नसञ्चयो विप्र बदराणीह भक्षय}


\threelineshloka
{ततोऽब्रवीन्महादेवः पचस्वैतानि सुव्रते}
{इत्युक्ता साऽपचत्तानि ब्राह्मणप्रियकाम्यया}
{अधिश्रित्य समिद्धेऽग्नौ बदराणि यशस्विनी}


\twolineshloka
{दिव्या मनोरमाः पुण्याः कथाः शुश्राव सा तदा}
{अतीता सा त्वनावृष्टिर्घोरा द्वादशवार्षिकी}


\twolineshloka
{अनश्नन्त्याः पचन्त्याश्च शृण्वन्त्याश्च कथाः शुभाः}
{दिनोपमः स तस्याथ कालोऽतीतः सुदारुणः}


\twolineshloka
{ततस्तु मुनयः प्राप्ताः फलान्यादाय पर्वतात्}
{ततः स भगवान्प्रीतः प्रोवाचारुन्धतीं ततः}


\twolineshloka
{उपसर्पस्व धर्मज्ञे यथापूर्वमिमानृषीन्}
{प्रीतोऽस्मि तव धर्मज्ञे तपसा नियमेन च}


\twolineshloka
{ततः सन्दर्शयामास स्वरूपं भगवान्हरः}
{ततोऽब्रवीत्तदा तेभ्यस्तस्याश्च चरितं महत्}


\twolineshloka
{भवद्भिर्हिमवत्पृष्ठे यत्तपः समुपार्जितम्}
{अस्यास्च यत्तपो विप्रा न समं तन्मतं मम}


\twolineshloka
{अनया हि तपस्विन्या तपस्तप्तं सुदुश्चरम्}
{अनश्नन्त्या पचन्त्या च समा द्वादश पारिताः}


\twolineshloka
{ततः प्रोवाच भगवांस्तामेवारुन्धतीं पुनः}
{वरं वृणीष्व कल्याणि यत्तेऽभिलषितं हृदि}


\twolineshloka
{साऽब्रवीत्पृथुताम्राक्षी देवं सप्तर्षिसंसदि}
{भगवान्यदि मे प्रीतस्तीर्थं स्यादिदमद्भुतम्}


\twolineshloka
{सिद्धदेवर्षिदयितं नाम्ना बदरपाचनम्}
{तथाऽस्मिन्देवदेवेश त्रिरात्रमुषितः शुचिः}


\twolineshloka
{प्राप्नुयादुपवासेन फलं द्वादशवार्षिकम्}
{एवमस्त्विति तां देवः प्रत्युवाच तपस्विनीम्}


\threelineshloka
{सप्तर्षिभिः स्तुतो देवस्ततो लोकं ययौ तदा}
{ऋषयो विस्मयं जग्मुस्तां दृष्ट्वा चाप्यरुन्धतीम्}
{अश्रान्तां चाविवर्णां च क्षुत्पिपासाऽसमायुताम्}


\twolineshloka
{एवं सिद्धिः परा प्राप्ता अरुन्धत्या विशुद्धया}
{यथा त्वया महाभागे मदर्थे संशितव्रते}


\threelineshloka
{विशेषो हि त्वया भद्रे व्रते ह्यस्मिन्समर्पितः}
{तथा चेदं ददाम्यद्य नियमेन सुतोषितः}
{विशेषं तव कल्याणि प्रयच्छामि वरं वरे}


\threelineshloka
{अरुन्धत्या वरस्तस्या यो दत्तो वै महात्मना}
{तस्य चाहं प्रभावेन तव कल्याणि तेजसा}
{प्रवक्ष्यामि परं भूयो वरमत्र यथाविधि}


\twolineshloka
{यस्त्वेकां रजनीं तीर्थे वत्स्यते सुसमाहितः}
{सस्नात्वा प्राप्स्यते लोकान्देहन्यासात्सुदुर्लभान्}


\twolineshloka
{इत्युक्त्वा भगवान्देवः सहस्राक्षः प्रतापवान्}
{श्रुतावतीं ततः पुण्यां जगाम त्रिदिवं पुनः}


\twolineshloka
{गते वज्रधरे राजंस्तत्र वर्षं पपात ह}
{पुष्पाणां भरतश्रेष्ठ दिव्यानां पुण्यगन्धिनाम्}


\twolineshloka
{दवदुन्दुभयश्चापि नेदुस्तत्र महास्वनाः}
{मारुतश्च ववौ पुण्यः पुण्यगन्धो विशाम्पते}


\threelineshloka
{उत्सृज्य तु शुभा देहं जगामास्य च भार्यताम्}
{तपसोग्रेण तं लब्ध्वा तेन रेमे सहाच्युत ॥जनमेजय उवाच}
{}


\threelineshloka
{का तस्या भगवन्माता क्व संवृद्धा च शोभना}
{श्रोतुमिच्छाम्यहं विप्र परं कौतूहलं हि मे ॥वैशम्पायन उवाच}
{}


\twolineshloka
{भरद्वाजस्य विप्रर्षेः स्कन्नं रेतो महात्मनः}
{दृष्ट्वाऽप्सरसमायान्ती घृताचीं पृथलोचनाम्}


\twolineshloka
{स तु जग्राह तद्रेतः करेण जपतां वरः}
{तदाऽपतत्पर्णपुटे तत्र सा सम्भवत्सुता}


\twolineshloka
{तस्यास्तु जातकर्मादि कृत्वा सर्वं तपोधनः}
{नाम चास्याः स कृतवान्भरद्वाजो महामुनिः}


\twolineshloka
{श्रुतावतीति धर्मात्मा देवर्षिगणसंसदि}
{स्वे च तामाश्रमे न्यस्य जगाम हिमवद्वनम्}


\twolineshloka
{तत्राप्युपस्पृश्य महानुभावोवसूनि दत्त्वा च महाद्विजेभ्यः}
{जगाम तीर्थं सुसमाहितात्माशक्रस्य वृष्णिप्रवरस्तदानीम्}


\chapter{अध्यायः ५१}
\twolineshloka
{वैशम्पायन उवाच}
{}


\twolineshloka
{इन्द्रतीर्थं ततो गत्वा यदूनां प्रवरो बली}
{विप्रेभ्यो धनरत्नानि ददौ स्नात्वा यथाविधि}


\twolineshloka
{तत्र ह्यमरराजो वै ईजे क्रतुशतेन ह}
{बृहस्पतेश्च देवेशः प्रददौ विपुलं धनम्}


\twolineshloka
{अनर्गलान्सजारूथ्यान्सर्वान्विविधदक्षिणान्}
{आजहर क्रतूंस्तत्र यथोक्तं वेदपारगैः}


\twolineshloka
{तान्क्रतून्भरतश्रेष्ठ शतकृत्वो महाद्युतिः}
{पूरयामास विधिवत्ततः ख्यातः शतक्रतुः}


\twolineshloka
{तस्य नाम्ना तु तत्तीर्थं शिवं पुण्यं सनातनम्}
{इन्द्रतीर्थमिति ख्यातं सर्वपापप्रमोचनम्}


\twolineshloka
{उपस्पृश्य च तत्रापि विधिवन्मुसलायुधः}
{ब्राह्मणान्पूजयित्वा च पानाच्छादुनभोजनैः}


\twolineshloka
{शुभं तीर्थवरं तस्माद्रामतीर्थं जगाम ह}
{यत्र रामो महाभागो भार्गवः सुमहातपाः}


\twolineshloka
{असकृत्पृथिवीं कृत्वा हतक्षत्रियपुङ्गवाम्}
{उपाध्यायं पुरस्कृत्य काश्यपं मुनिसत्तमम्}


\twolineshloka
{अयजद्वाजपेयेन सोऽश्वमेधशतेन च}
{प्रददौ दक्षिणार्थं च पृथिवीं सागराम्बराम्}


\twolineshloka
{रामो दत्त्वा धनं तत्र द्विजेभ्यो जनमेजय}
{उपस्पृश्य यथान्यायं पूजयित्वा तथा द्विजान्}


\threelineshloka
{पुण्यतीर्थे शुभे देशे वसु दत्त्वा हलायुधः}
{मुनींश्चैवाभिवाद्याथ यमुनातीर्थमागमत्}
{यत्रानयामास तदा राजसूयमपाम्पतिः}


\threelineshloka
{दितेः सुतान्महाभागो वरुणो वै सितप्रभः}
{यत्र निर्जित्य सङ्ग्रामे मानुषान्दानवांस्तथा}
{गन्धर्वान्राक्षसांश्चैव वरुणः परवीरहा}


\twolineshloka
{तस्मिन्क्रतुवरे वृत्ते सङ्ग्रामः समजायत}
{देवानां दानवानां च त्रैलोक्यस्य भयावहः}


\twolineshloka
{राजसूये क्रतुश्रेष्ठे निवृत्ते जनमेजय}
{जायते सुमहाघोरः संक्षयः क्षत्रियान्प्रति}


\twolineshloka
{हलायुधस्तदा रामस्तस्मिंस्तीर्थवरे शुभे}
{तत्रस्नात्वा च दत्त्वा च द्विजेभ्यो वसु माधवः}


\twolineshloka
{वनमाली ततो हृष्टः स्तूयमानो द्विजातिभिः}
{तस्मादादित्यतीर्थं च जगाम कमलेक्षणः}


\twolineshloka
{यत्रेष्ट्वा भगवाञ्ज्योतिर्भास्करो राजसत्तम}
{ज्योतिषामाधिपत्यं च प्रभावं चाभ्यपद्यत}


\twolineshloka
{तस्या नद्यास्तु तीरे वै सर्वे देवाः सवासवाः}
{विश्वेदेवाः समरुतो गन्धर्वाप्सरसश्च ह}


\twolineshloka
{द्वैपायनः शुकश्चैव कृष्णश्च मधुसूदनः}
{यक्षाश्च राक्षसाश्चैव पिशाचाश्च विशाम्पते}


\twolineshloka
{एते चान्ये च बहवो योगसिद्धाः सहस्रशः}
{तस्मिंस्तीर्थे सरस्वत्याः शिवे पुण्ये परन्तप}


\twolineshloka
{तत्र हत्वा पुरा विष्णुरसुरौ मधुकैटभौ}
{आप्लुत्य भरतश्रेष्ठ तीर्थप्रवर उत्तमे}


\twolineshloka
{द्वैपायनश्च धर्मात्मा तत्रैवाप्लुत्य भारत}
{सम्प्राप्तः परमं योगं सिद्धिं च परमां गतः}


\twolineshloka
{असितो देवलश्चैव तस्मिन्नेव महातपाः}
{परमं योगमास्थाय ऋषिर्योगमवाप्तवान्}


\chapter{अध्यायः ५२}
\twolineshloka
{वैशम्पायन उवाच}
{}


\twolineshloka
{तस्मिन्नेव तु धर्मात्मा वसति स्म तपोधनः}
{गार्हिस्थ्यं धर्ममास्थाय ह्यसितो देवलः पुरा}


\twolineshloka
{धर्मनित्यः शुचिर्दान्तो यज्ञशीलो महातपाः}
{कर्मणा मनसा वाचा समः सर्वेषु जन्तुषु}


\twolineshloka
{अक्रोधनो महाराज तुल्यनिन्दात्मसंस्तुतिः}
{प्रियाप्रिये तुल्यवृत्तिर्यमवत्समदर्शनः}


\threelineshloka
{काञ्चने लोष्ठके चैव समदर्शी महातपाः}
{देवानपूजयन्नित्यमतिथींश्च द्विजैः सह}
{ब्रह्मचर्यरतो नित्यं सदा धर्मपरायणः}


\twolineshloka
{ततोऽभ्येत्य महाभाग योगमास्थाय भिक्षुकः}
{जैगीषव्यो मुनिर्धीमांस्तस्मिंस्तीर्थे समाहितः}


\twolineshloka
{देवलस्याश्रमे राजन्न्यवसत्स महाद्युतिः}
{योगनित्यो महाराज सिद्धिं प्राप्तो महातपाः}


\twolineshloka
{तं तत्र वसमानं तु जैगीषव्यं महामुनिम्}
{देवलो दर्शयन्नेव नैवायुञ्जत धर्मतः}


% Check verse!
एवं तयोर्महाराज दीर्घकालो व्यतिक्रमत्
\twolineshloka
{जैगीषव्यो मुनिस्तं तु ददर्शाथ स केवलम्}
{आहारकाले मतिमान्परिव्राड् जमेजय}


\twolineshloka
{उपातिष्ठत धर्मज्ञो भैक्षकाले स देवलम्}
{गौरवं परमं चक्रे प्रीतिं च विपुलां तथा}


\twolineshloka
{देवलस्तु यथाशक्ति पूजयामास भारत}
{ऋषिदृष्टेन विधिना समा बहीः समाहितः}


\twolineshloka
{कदाचित्तस्य नृयते देवलस्य महात्मनः}
{चिन्ता सुमहती जाता मुनिं दृष्ट्वा महाद्युतिम्}


\twolineshloka
{समास्तु समतिक्रान्ता बह्व्यः पूजयतो मम}
{न चायमलसो भिक्षुरभ्यभाषत किञ्चन}


\twolineshloka
{एवं विगणयन्नेव स जगाम महोदधिम्}
{अन्तरिक्षचरः श्रीमान्कलशं गृह्य देवलं}


\twolineshloka
{गच्छन्नेव स धर्मात्मा समुद्रं सरितां परिम्}
{जैगीषव्यं ततोऽपश्यद्रतं प्रागेव भारत}


\twolineshloka
{ततः सविस्मयश्चिन्तां जगामाथामितप्रभः}
{कथं सिक्षुरचं प्राप्तः समुद्रे स्नात एव च}


\twolineshloka
{इत्येवं चिन्तयामास महर्षिरसितस्तदा}
{स्नात्वा सखुद्रे विधिवच्छुचिर्जप्यं जजाप सः}


\twolineshloka
{कृतजxxxxx श्रीमानाश्रयं च जगाम ह}
{xxxxxxxxxx गृहीत्वा जनमेजय}


\twolineshloka
{ततः स प्रविशन्नेव स्वमाश्रमपदं मुनिः}
{xxxxxxxxx तत्र जैगीषव्यमपश्यत}


\twolineshloka
{न व्याहरति चैवेनं जैगीषव्यः कथञ्चन}
{xxxxxxxx वसति स्म महातपाः}


\twolineshloka
{तं दृष्ट्वा चाप्लुतं तोये सामरे सागरोपमम्}
{प्रविष्टxxxxx चापि पूर्वमेव ददर्श सः}


\twolineshloka
{xxxxx देवलो राजंश्चिन्तयामास बुद्धिमान्}
{xxxxxxx तपसो जैगीषव्यास्व योगजम्}


\twolineshloka
{चिन्तयामास राजेन्द्र तदा स मुनिसत्तमः}
{मया दृष्टः समुद्रे च आश्रमे च कथंन्वयम्}


\threelineshloka
{एवं विगणयन्नेव स मुनिर्मन्त्रपारगः}
{उत्पपाताश्रमात्तस्मादन्तरिक्षं विशाम्पते}
{जिज्ञासार्थं तदा भिक्षोर्जैगीषव्यस्य देवलः}


\twolineshloka
{सोऽन्तरिक्षचरान्सिद्धान्समपश्यत्समाहितान्}
{जैगीषव्यं च तैः सिद्धैः पूज्यमानमपश्यत}


\twolineshloka
{ततोऽसितः सुसंरब्धो व्यवसायी दृढव्रतः}
{अपश्यद्वै दिवं यातं जैगीषव्यं स देवलः}


\twolineshloka
{तस्माच्च पितृलोकं तं व्रजन्तं सोऽन्वपश्यत}
{पितृलोकाच्च तं यातं याम्यं लोकमपश्यत}


\threelineshloka
{`तस्मादादित्यलोकं च व्रजन्तं सोऽन्वपश्यत}
{'तस्मादपि समुत्पत्य सोमलोकमभिष्टुतम्}
{व्रजन्तमन्वपश्यत्स जैगीषव्यं स देवलः}


\twolineshloka
{लोकान्समुत्पतन्तं तु शुभानेकान्तयाजिनाम्}
{ततोऽग्निहोत्रिणां लोकांस्ततश्चाप्युत्पपात ह}


\threelineshloka
{दर्शं च पौर्णमासं च ये यजन्ति तपोधनाः}
{तेभ्यः स ददृशे धीमाँल्लोकेभ्यः पशुयाजिनाम्}
{व्रजन्तं लोकममलमपश्यद्देवपूजितम्}


\twolineshloka
{चातुर्मास्यैर्बहुविधैर्यजन्ते ये तपोधनाः}
{तेषां स्थानं ततो यतां तथाऽग्निष्टोमयाजिनाम्}


\twolineshloka
{अग्निष्टुतेन च तथा ये यजन्ति तपोधनः}
{तत्स्थानमनुसम्प्राप्तमन्वपश्यत देवलः}


\twolineshloka
{वाजपेयं क्रतुवरं तथा बहुसुवर्णकम्}
{आहरन्ति महाप्राज्ञास्तेषां लोकेष्वपश्यत}


\twolineshloka
{यजन्ते राजसूयेन पुण्डरीकेण चैव ये}
{तेषां लोकेष्वपश्यच्च जैगीषव्यं स देवलः}


\twolineshloka
{अश्वमेधं क्रतुवरं नरमेधं तथैव च}
{आहरन्ति नरश्रेष्ठास्तेषां लोकेष्वपश्यत}


\twolineshloka
{सर्वमेधं च दुष्प्रापं तथा सौत्रामणिं च ये}
{तेषां लोकेष्वपश्यच्च जैगीषव्यं स देवलः}


\twolineshloka
{द्वादशाहैश्च सत्रैश्च यजन्ते विविधैर्नृप}
{तेषां लोकेष्वपश्यच्च जैगीषव्यं स देवलः}


\twolineshloka
{मैत्रावरुणयोर्लोकानादित्यानां तथैव च}
{सलोकतामनुप्राप्तमपश्यत ततोऽसितः}


\twolineshloka
{रुद्राणां च वसूनां च स्थानं यच्च बृहस्पतेः}
{तानिं सर्वाण्यतीतं च समपश्यत्ततोऽसितः}


\twolineshloka
{आरुह्य च गवां लोकं प्रयान्तं ब्रह्मसत्रिणाम्}
{लोकानपश्यद्गच्छन्तं जैगीषव्यं ततोऽसितः}


\twolineshloka
{त्रील्लोँकान्प्रवरान्विप्रमुत्पतन्तं स्वतेजसा}
{पतिव्रतानां लोकांश्च व्रजन्तं सोऽन्वपश्यत}


\twolineshloka
{ततो मुनिवरं भूयो जैगीषव्यमथासितः}
{नान्वपश्यत लोकस्थमन्तर्हितमरिन्दम}


\twolineshloka
{सोऽचिन्तयन्महाभागो जैगीषव्यस्य देवलः}
{प्रभावं सुव्रतत्वं च सिद्धिं योगस्य चातुलाम्}


\twolineshloka
{असितोऽपृच्छत तदा सिद्धाँल्लोकेषु सत्तमान्}
{प्रयतः प्राञ्जलिर्भूत्वा धीरस्तान्ब्रह्मचारिणः}


\threelineshloka
{जैगीषव्यं न पश्यामि तं शंसन्तु तपोधनाः}
{एतदिच्छाम्यहं श्रोतुं परं कौतूहलं हि मे ॥सिद्धा ऊचुः}
{}


\threelineshloka
{शृणु देवल भूतार्थं शंसतां नो दृढव्रत}
{जैगीषव्यो गतो लोकं शाश्वतं ब्रह्मणोक्षयम् ॥वैशम्पायन उवाच}
{}


\twolineshloka
{स श्रुत्वा वचनं तेषां सिद्धानां ब्रह्मचारिणाम्}
{असिता देवलस्तूर्णमुत्पपात पपात च}


\fourlineindentedshloka
{ततः सिद्धास्त ऊचुर्हि देवलं पुनरेव ह}
{न देवलगतिस्तत्र तव गन्तुं तपोधन}
{ब्रह्मणः सदने विप्र जैगीषव्यो यदाप्तवान् ॥वैशम्पायन उवाच}
{}


\twolineshloka
{तेषां तद्वचनं श्रुत्वा सिद्धानां देवलः पुनः}
{आनुपूर्व्येण लोकांस्तान्सर्वानवततार ह}


\twolineshloka
{स्वमाश्रमपदं पुण्यमाजगाम पतङ्गवत्}
{प्रविशन्नेव चापश्यज्जैगीषव्यं स देवलः}


\twolineshloka
{ततो बुद्ध्या व्यगणयद्देवलो धर्मयुक्तया}
{दृष्ट्वा प्रभावं तपसो जैगीषव्यस्य योगजम्}


\threelineshloka
{ततोऽब्रविन्महात्मानं जैगीषव्यं स देवलः}
{विनयावनतो राजन्नुपसर्प्य महामुनिम्}
{मोक्षधर्मं समास्थातुमिच्छेयं भगवन्नहम्}


\twolineshloka
{तस्य तद्वचनं श्रुत्वा उपदेशं चकार सः}
{विधिं योगस्य परमं कार्याकार्यस्य शास्त्रतः}


\twolineshloka
{सन्नाये कृतबुद्धिं तं ततो दृष्ट्वा महातपाः}
{सर्वाश्चास्य क्रियाश्चक्रे विधिदृष्टेन कर्मणा}


\twolineshloka
{सन्न्यासे कृतबुद्धिं तं भूतानि पितृभिः सह}
{ततो दृष्ट्वा प्ररुरुदुः कोऽस्मान्संविभजिष्यति}


\twolineshloka
{देवलस्तु वचः श्रुत्वा भूतानां करुणं तथा}
{दिशो दश व्याहरतां मोक्षं त्यक्तुं मनो दधे}


\twolineshloka
{ततस्तु फलमूलानि पवित्राणि च भारत}
{पुष्पाण्योषधयश्चैव रोरूयन्ते सहस्रशः}


\twolineshloka
{पुनर्नो देवलः क्षुद्रो नूनं छेत्स्यति दुर्मतिः}
{अभयं सर्वभूतेभ्यो यो दत्त्वा नावबुध्यते}


\twolineshloka
{ततो भूयो व्यगणयत्स्वबुद्ध्या मुनिसत्तमः}
{मोक्षे गार्हस्थ्यधर्मे वा किन्नु श्रेयस्करं भवेत्}


\twolineshloka
{इति निश्चित्य मनसा देवलो राजसत्तम}
{त्यक्त्वा गार्हस्थ्यधर्मं स मोक्षधर्ममरोचयत्}


\twolineshloka
{एवमादीनि सञ्चिन्त्य देवलो निश्चयान्वितः}
{प्राप्तवान्परमां सिद्धिं परं योगं च भारत}


\twolineshloka
{ततो देवाः समागम्य बृहस्पतिपुरोगमाः}
{जैगीषव्यं तपश्चास्य प्रशंसन्ति तपस्विनः}


\twolineshloka
{अथाब्रवीदृषिवरो देवान्वै नारदस्तथा}
{जैगीषव्ये तपो नास्ति विस्मापयति योऽसितम्}


\twolineshloka
{तमेवंवादिनं धीरं प्रत्यूचुस्ते दिवौकसः}
{नैवमित्येव शंसन्तो जैगीषव्यं महामुनिम्}


\twolineshloka
{नातः परतरं किञ्चित्तुल्यमस्ति प्रभावतः}
{तेजसस्तपसश्चास्य योगस्य च महात्मनः}


\twolineshloka
{एवं प्रभावो धर्मात्मा जैगीषव्यस्तथाऽसितः}
{तयोरिदं स्थानवरं तीर्थं चैव महात्मनोः}


\threelineshloka
{तत्राप्युपस्पृश्य ततो महात्मादत्त्वा च वित्तं हलभृद्द्विजेभ्यः}
{अवाप्य धर्मं परमार्थकर्माजगाम सोमस्य महाxxxxxxर्थम्}
{}


\chapter{अध्यायः ५३}
\twolineshloka
{वैशम्पायन उवाच}
{}


\twolineshloka
{यत्रेजिवानुडुपती राजसूयेन भारत}
{यस्मिन्वृत्ते महानासीत्सङ्ग्रामस्तारकामयः}


\twolineshloka
{तत्राप्युपस्पृश्य बलो दत्त्वा दानानि चात्मवान्}
{सारस्वतस्य धर्मात्मा मुनेस्तीर्थं जगाम ह}


\threelineshloka
{यत्र द्वादशवार्षिक्यामनावृष्ट्यां द्विजोत्तमान्}
{वेदानध्यापयामास पुरा सारस्वतो मुनिः ॥जनमेजय उवाच}
{}


\threelineshloka
{कथं द्वादशवार्षिक्यामनावृष्ट्यां द्विजोत्तमान्}
{वेदानध्यापयामास पुरा सारस्वतो मुनिः ॥वैशम्पायन उवाच}
{}


\twolineshloka
{आसीत्पूर्वं महाराज मुनिर्धीमान्महातपाः}
{दधीचिरिति विख्यातो ब्रह्मचारी जितेन्द्रियः}


\twolineshloka
{तस्यातितपसः शक्रो बिभेति सततं विभो}
{न स लोभयितुं शक्यः फलैर्बहुविधैरपि}


\twolineshloka
{प्रलोभनार्थं तस्याथ प्राहिणोत्पाकशासनः}
{दिव्यामप्सरसं पुण्यां दर्शनीयामलम्बुसाम्}


\twolineshloka
{तस्य तर्पयतो देवान्सरस्वत्यां महात्मनः}
{समीपतो महाराज सोपातिष्ठत भामिनी}


\twolineshloka
{तां दिव्यवपुषं दृष्ट्वा तस्यर्षेर्भावितात्मनः}
{रेतः स्कन्नं सरस्वत्यां तस्मा जग्राह निम्नगा}


\twolineshloka
{कुक्षौ चाप्यदधद्वृष्टा तद्रेतः पुरुषर्षभ}
{सा दधार च तं गर्भं पुत्रहेतोर्महात्मनः ॥थ}


\twolineshloka
{सुषुवे चापि समये पुत्रं सारस्वतं वरम्}
{जगाम पुत्रमादाय तमृषिं प्रति च प्रभो}


\threelineshloka
{ऋषिसंसदि तं दृष्ट्वा सा नदी मुनिसत्तमम्}
{ततः प्रोवाच राजेन्द्र ददती पुत्रमस्य तम्}
{ब्रह्मर्षे तव पुत्रोऽयं त्वद्भक्त्या धारितो मया}


\twolineshloka
{दृष्ट्वा तेऽप्सरसं रेतो यत्स्कन्नं प्रागलम्बुसाम्}
{तत्कुक्षिणाऽहं ब्रह्मर्षे त्वद्भक्त्या धृतवत्यहम्}


\twolineshloka
{न विनाशमिदं गच्छेत्त्वत्तेज इति निश्चयात्}
{प्रतिगृह्णीष्व पुत्रं स्वं मया दत्तमनिन्दितम्}


\twolineshloka
{इत्युक्तः प्रतिजग्राह प्रीतिं चावाप पुष्कलाम्}
{पितृवच्चोपजिघ्रत्तं मूर्ध्नि प्रेम्णा द्विजोत्तमः}


\twolineshloka
{परिष्वज्य चिरं कालं तदा भरतसत्तम}
{सरस्वत्यै वरं प्रादात्प्रीयमाणो महामुनिः}


\twolineshloka
{विश्वेदेवाः सपितरो गन्धर्वाप्सरसां गणाः}
{तृप्तिं यास्यन्ति सुभगे तर्प्यमाणास्तवाम्भसा}


\threelineshloka
{इत्युक्त्वा स तु तुष्टाव वचोभिर्वै महानदीम्}
{प्रीतः परमहृष्टात्मा यथावच्छृणु पार्थिव ॥दधीचिरुवाच}
{}


\twolineshloka
{प्रस्रुताऽसि महाभागे सरसो ब्रह्मणः पुरा}
{जानन्ति त्वां सरिच्छ्रेष्ठे मुनयः संशितव्रताः}


\twolineshloka
{मम प्रियकरी चापि सततं प्रियदर्शने}
{तस्मात्सारस्वतं पुत्रमदधा वरवर्णिनि}


\twolineshloka
{तवैव नाम्ना प्रथितः पुत्रस्ते लोकभावनः}
{सारस्वत इति ख्यातो भविष्यति महातपाः}


\twolineshloka
{एष द्वादशवार्षिक्यामनावृष्ट्यां द्विजर्षभान्}
{सारस्वतो महाभागे वेदानध्यापयिष्यति}


\twolineshloka
{पुण्याभ्यश्च सरिद्भ्यस्त्वं सदा पुण्यतमा शुभे}
{भविष्यसि महाभागे मत्प्रसादात्सरस्वति}


\twolineshloka
{एवं स्त संस्तुता तेन वरं लब्ध्वा महानदी}
{पुत्रमादाय मुदिता जगाम भरतर्षभ}


\twolineshloka
{एतस्मिन्नेव काले तु विरोधे देवदानवैः}
{शक्रः प्रहरणान्विषी लोकांस्त्रीन्विचचार ह}


\twolineshloka
{न चोपलेभे भगवाञ्शक्रः प्रहरणं तदा}
{यद्वै तेषां भवेद्योग्यं वधाय विबुधद्विषाम्}


\twolineshloka
{ततोऽब्रवीत्सुराञ्शक्रो न मे शक्याः सुरारयः}
{ऋतेऽस्थिभिर्दधीचस्य निहन्तुं त्रिदशद्विषः}


\twolineshloka
{तस्माद्यत्नादृषिश्रेष्ठो याच्यतां कार्यसिद्धये}
{दधीचास्थीनि देहीति तैर्वधिष्यामहे रिपून्}


\twolineshloka
{स च तैर्याचितोऽस्थीनि यत्नादृषिवरस्तदा}
{साहाय्यं नः कुरुष्वेति चकारैवाविचारयन्}


\twolineshloka
{स लोकानक्षयान्प्राप्तो देवप्रियकरस्तदा}
{तस्यास्थिभिरथो शक्रः सम्प्रहृष्टमनास्तदा}


\twolineshloka
{कारयामास दिव्यानि नानाप्रहरणान्युत}
{गदा वज्राणि चक्राणि गुरून्दण्डांश्च पुष्कलान्}


\twolineshloka
{स हि तीव्रेण तपसा सम्भृतः परमर्षिणा}
{प्रजापतिसुतेनाथ भृगुणा लोकभावनः}


\twolineshloka
{अतिकायः स तेजस्वी लोकसारो विनिर्मितः}
{जज्ञे शैलगुरुः प्रांशुर्महिम्ना प्रथितः प्रभुः}


% Check verse!
नित्यमुद्विजते चास्य तेजसः पाकशासनः
\threelineshloka
{तेन वज्रेण भगवान्मन्त्रयुक्तेन भारत}
{भृशं क्रोधविसृष्टेन ब्रह्मतेजोद्भवेन च}
{दैत्यदानववीराणां जघान नवतीर्नव}


\twolineshloka
{अथ काले व्यतिक्रान्ते महत्यतिभयङ्करी}
{अनावृष्टिरनुप्राप्ता राजन्द्वादशवार्षिकी}


\twolineshloka
{तस्यां द्वादशवार्षिक्यामनावृष्ट्यां महर्षयः}
{वृत्त्यर्थं प्राद्रवन्राजन्क्षुधार्ताः सर्वतोदिशम्}


\twolineshloka
{दिग्भ्यस्तान्प्रद्रुतान्दृष्ट्वा मुनिः सारस्वतस्तदा}
{गमनाय मतिं चक्रे तं प्रोवाच सरस्वती}


\twolineshloka
{न गन्तव्यमितः पुत्र तवाहारमहं सदा}
{दास्यामि मत्स्यप्रवरानुष्यतामिह भारत}


\twolineshloka
{इत्युक्तस्तर्पयामास स पितॄन्देवतास्तथा}
{आहारमकरोन्नित्यं प्राणान्वेदांश्च धारयन्}


\twolineshloka
{अथ तस्यामनावृष्ट्यामतीतायां महर्षयः}
{अन्योन्यं परिपप्रच्छुः पुनः स्वाध्यायकारणात्}


\twolineshloka
{तेषां क्षुधापरीतानां नष्टा देवा विधावताम्}
{सर्वेषामेव राजेन्द्र न किचित्प्रतिभाति ह}


\twolineshloka
{अथ कश्चिदृषिस्तेषां सारस्वतमुपेयिवान्}
{कुर्वाणं संशितात्मानं स्वाध्यायमृषिसत्तमम्}


\twolineshloka
{स गत्वाऽचष्ट तेभ्यश्च सारस्वतमृषिं प्रभुम्}
{स्वाध्यायममरप्रख्यं कुर्वाणं विजने वने}


\twolineshloka
{ततः सर्वे समाजग्मुस्तत्र राजन्महर्षयः}
{सारस्वतं मुनिश्रेष्ठमिदमूचुः समागताः}


\twolineshloka
{अस्मानध्यापयस्वेति तानुवाच ततो मुनिः}
{शिष्यत्वमुपागच्छध्वं विधिना च ममेत्युत}


\twolineshloka
{तत्राब्रुवन्मुनिगणा बालस्त्वमसि पुत्रक}
{स तानाह न मे धर्मो नश्येदिति पुनर्मुनीन्}


\twolineshloka
{यो ह्यधर्मेण वै ब्रूयाद्गृह्णीयाद्योऽप्यधर्मतः}
{म्रियेतां तावुभौ क्षिप्रं स्यातां वा वैरिणावुभौ}


\twolineshloka
{न हायनैर्न पलितैर्न वित्तेन न बन्धुभिः}
{ऋषयश्चक्रिरे धर्मं योऽनूचानः स नो महान्}


\twolineshloka
{एतच्छ्रुत्वा वचस्तस्य मनुयस्ते विधानतः}
{तस्माद्वेदाननुप्राप्य पुनर्धर्मं प्रचक्रिरे}


\twolineshloka
{षष्टिर्मुनिसहस्राणि शिष्यत्वं प्रतिपेदिरे}
{सारस्वतस्य विप्रर्षेर्वेदस्वाध्यायकारणात्}


\twolineshloka
{मुष्टिं मुष्टिं ततः सर्वे दर्भाणां ते ह्युपाहरन्}
{तस्यासनार्थं विप्रर्षेर्बालस्यापि वशे स्थिताः}


% Check verse!
तत्रापि दत्त्वा वसु रौहिणेयोमहाबलः केशवपूर्वजोऽथजगाम तीर्थं मुदितः क्रमेणतं वृद्धकन्याश्रममे व वीरः
\chapter{अध्यायः ५४}
\twolineshloka
{जनमेजय उवाच}
{}


\twolineshloka
{कथं कुमारी भगवंस्तपोयुक्ता ह्यभूत्पुरा}
{किमर्थं च तपस्तेपे को वाऽस्या नियमोऽभवत्}


\threelineshloka
{सुदुष्करमिदं ब्रह्मंस्त्वत्तः श्रुतमनुत्तमम्}
{आख्याहि तत्त्वमखिलं यथा तपसि सा स्थिता ॥वैशम्पायन उवाच}
{}


\threelineshloka
{ऋषिरासीन्महावीर्यः कुणिर्गार्ग्यो महायशाः}
{स तप्त्वा विपुलं राजंस्तपो वै तपतां वरः}
{तपसाऽथ सुतां सुभ्रूं समुत्पादितवान्विभुः}


\twolineshloka
{तां च दृष्ट्वा मुनिः प्रीतः कुणिर्गार्ग्यो महायशाः}
{जगाम त्रिदिवं राजन्संत्यज्येह कलेवरम्}


\twolineshloka
{सुभ्रूः सा ह्यथ कल्याणी पुण्डरीकनिभेक्षणा}
{महता तपसोग्रेण कृत्वाश्रममनिन्दिता}


\twolineshloka
{उपवासैः पूजयन्ती पितॄन्देवांश्च सा पुरा}
{तस्यास्तु तपसोग्रेण महान्कालोऽत्यगान्नृप}


\twolineshloka
{सा पित्रा दीयमानापि पतिं नैच्छदनिन्दिता}
{आत्मनः सदृशं सा तु भर्तारं नान्वपश्यत}


\twolineshloka
{ततः सा तपसोग्रेण पीडयित्वाऽऽत्मनस्तनुम्}
{पितृदेवार्चनरता बभूव विजने वने}


\twolineshloka
{साऽऽत्मानं मन्यमाना तु कृतकृत्यं श्रमान्विता}
{`जगाम वृद्धभावं तु कौमारब्रह्मचारिणी'}


\twolineshloka
{वार्धकेन च राजेन्द्र तपसा चैव कर्शिता}
{सा नाशकद्यदा गन्तुं पदात्पदमपि स्वयम्}


\twolineshloka
{चकार गमने बुद्धिं परलोकाय वै तदा}
{मोक्तुकामां तु तां दृष्ट्वा शरीरं नारदोऽब्रवीत्}


\twolineshloka
{असंस्कृतायाः कन्यायाः कुतो लोकास्तवानघे}
{एवं तु श्रुतमस्माभिर्देवलोके महाव्रते}


\twolineshloka
{तपः परमकं प्राप्तं न तु लोकास्त्वया जिताः}
{तन्नारदवचः श्रुत्वा साऽब्रवीदृषिसंसदि}


\twolineshloka
{तपसोऽर्धं प्रयच्छामि पाणिग्राहस्य सत्तमाः}
{इत्युक्तेऽस्यास्तु जग्राह पाणिं गालवसम्भवः}


\twolineshloka
{ऋषिः प्राक् शृङ्गवान्नाम समयं चेममब्रवीत्}
{समयेन तवाद्याहं पाणिं स्प्रक्ष्यामि शोभने}


\twolineshloka
{यद्येकरात्रं वस्तव्यं त्वया सह मयेति ह}
{तथेति सा प्रतिश्रुत्य तस्मै पाणिं ददौ तदा}


\twolineshloka
{यथादृष्टेन विधिना हुत्वा चाग्निं विधानतः}
{चक्रे च पाणिग्रहणं तस्योद्वाहं च गालविः}


\twolineshloka
{सा रात्रावभवद्राजंस्तरुणी वरवर्णिनी}
{दिव्याभरणवस्त्रा च दिव्यगन्धानुलेपना}


\twolineshloka
{तां दृष्ट्वा गालविः प्रीतो दीपयन्तीमिव श्रिया}
{उवास च क्षपामेकां प्रभाते साऽब्रवीच्च तम्}


\twolineshloka
{यस्त्वया समयो विप्र कृतो मे तपतां वर}
{तेनोषिताऽस्मि भद्रं ते स्वस्ति तेऽस्तु व्रजाम्यहम्}


\twolineshloka
{सा तु ध्यात्वाऽब्रवीद्भूयो योऽस्मिंस्तीर्थे समाहितः}
{वसते रजनीमेकां तर्पयित्वा दिवौकसः}


\twolineshloka
{चत्वारिंशतमष्टौ च द्वौ चाष्टौ सम्यगाचरेत्}
{यो ब्रह्मचर्यं वर्षाणि फलं तस्य लभेत सः}


\twolineshloka
{एवमुक्त्वा ततः साध्वी देहं त्यक्त्वा दिवं गता}
{ऋषिरप्यभवद्दीनस्तस्या रूपं विचिन्तयन्}


\threelineshloka
{समयेन तपोऽर्धं च कृच्छ्रात्प्रतिगृहीतवान्}
{साधयित्वा तदाऽऽत्मानं तस्याः स गतिमाप्तवान्}
{दुःखितो भरतश्रेष्ठ तस्या रूपबलात्कृतः}


\twolineshloka
{एतत्ते वृद्वकन्याया व्याख्यातं चरितं महत्}
{तथैव ब्रह्मचर्यं च स्वर्गस्य च गतिः शुभा}


% Check verse!
तत्रस्थश्चापि शुश्राव हतं शल्यं हलायुधः
\twolineshloka
{तत्रापि दत्त्वा दानानि द्विजातिभ्यः परन्तपः}
{शुशोच शल्यं सङ्ग्रामे निहतं पाण्डवैस्तदा}


\twolineshloka
{समन्तपञ्चकद्वारात्ततो निष्क्रम्य माधवः}
{पप्रच्छर्षिगणारामः कुरुक्षेत्रस्य यत्फलम्}


% Check verse!
ते पृष्टा यदुसिंहेन कुरुक्षेत्रफलं विभोसमाचख्युर्महात्मानस्तस्मै सर्वं यथातथम्
\chapter{अध्यायः ५५}
\twolineshloka
{ऋषय ऊचुः}
{}


\twolineshloka
{प्रजापतेरुत्तरवेदिरुच्यतेसनातनं राम समन्तपञ्चकम्}
{समीजिरे यत्र पुरा दिवौकसोवरेण सत्रेण महावरप्रदाः}


\threelineshloka
{पुरा च राजर्षिवरेण धीमताबहूनि वर्षाण्यमितेन तेजसा}
{प्रकृष्टमेतत्कुरुणा महात्मनाततः कुरुक्षेत्रमितीह पप्रथे ॥राम उवाच}
{}


\threelineshloka
{किमर्थं कुरुणा कृष्टं क्षेत्रमेतन्महात्मना}
{एतदिच्छाम्यहं श्रोतुं कथ्यमानं तपोधनाः ॥ऋषय ऊचुः}
{}


\threelineshloka
{पुरा किल कुरुं राम कर्षन्तं सततोत्थितम्}
{अभ्येत्य शक्रस्त्रिदिवात्पर्यपृच्छत कारणम् ॥इन्द्र उवाच}
{}


\threelineshloka
{किमिदं वर्तते कर्म प्रयत्नेन परेण च}
{राजर्षे किमभिप्रेत्य येनेयं कृष्यते क्षितिः ॥कुरुरुवाच}
{}


\twolineshloka
{इह ये पुरुषाः क्षेत्रे जनिष्यन्ति शतक्रतो}
{ते गमिष्यन्ति सुकृतां लोकान्पापविवर्जितान्}


\twolineshloka
{अपहास्य तु तं शक्रो जगाम त्रिदिवं पुनः}
{राजर्षिरप्यनिर्विण्णः कर्षत्येव वसुन्धराम्}


\twolineshloka
{आगम्यागम्य चैवैनं भूयोभूयोऽपहास्य च}
{शत्रक्रतुरनिर्विण्णं पृष्ट्वापृष्ट्वा जगाम ह}


\twolineshloka
{यदा तु तपसोग्रेण चकर्ष वसुधां नृपः}
{ततः शक्रोऽब्रवीद्देवान्राजर्षेर्यच्चिकीर्षितम्}


\twolineshloka
{एतच्छ्रुत्वाऽब्रुवन्देवाः सहस्राक्षमिदं वचः}
{वरेण च्छन्द्यतां शक्र राजर्षिर्यदि शक्यते}


\twolineshloka
{यदि ह्यत्र प्रसूता वै स्वर्गं गच्छन्ति मानवाः}
{अस्माननिष्ट्वा क्रतुभिर्भागो नो न भविष्यति}


\twolineshloka
{आगम्य च ततः शक्रस्तदा राजर्षिमब्रवीत्}
{अलं खेदेन भवतः क्रियतां वचनं मम}


\twolineshloka
{मानवा ये निराहारा देहं त्यक्ष्यन्त्यतन्द्रिताः}
{युधि वा निहताः सम्यगपि तिर्यग्गता नृप}


\twolineshloka
{ते स्वर्गभाजो राजेन्द्र भवन्त्विह हतास्तु ये}
{तथास्त्विति ततो राजा कुरुः शक्रमुवाच ह}


\twolineshloka
{ततस्तमभ्यनुज्ञाप्य प्रहृष्टेनान्तरात्मना}
{जगाम त्रिदिवं भूयः क्षिप्रं बलनिषूदनः}


\twolineshloka
{एवमेतद्यदुश्रेष्ठ कृष्टं राजर्षिणा पुरा}
{शक्रेण चाभ्यनुज्ञातः पुण्ये प्राणान्मुमोच ह}


\twolineshloka
{[शक्रेण चाभ्यनुज्ञातं ब्राह्मद्यैश्च सुरैस्तथा}
{नातः परतरं पुण्यं भूमेः स्थानं भविष्यति}


\twolineshloka
{इह तप्स्यन्ति ये केचित्तपः परमकं नराः}
{देहत्यागेन ते सर्वे यास्यन्ति ब्रह्मणः क्षयम्}


\twolineshloka
{ये पुनः पुण्यभाजो वै दानं दास्यन्ति मानवाः}
{तेषां सहस्रगुणितं भविष्यत्यचिरेण वै}


\twolineshloka
{ये चेह नित्यं मनुजा निवत्स्यन्ति शुभैषिणः}
{यमस्य विषयं ते तु न द्रक्ष्यन्ति कदाचन}


\twolineshloka
{यक्ष्यन्ति ये च क्रतुभिर्महद्भिर्मनुजेश्वराः}
{तेषां त्रिविष्टपे वासो यावद्भूमिर्धरिष्यति ॥]}


\twolineshloka
{अपि चात्र स्वयं शक्रो जगौ गाथां सुराधिपः}
{कुरुक्षेत्रनिबद्धां वै तां शृणुष्व हलायुध}


\twolineshloka
{पांसवोऽपि कुरुक्षेत्राद्वायुना समुदीरिताः}
{अपि दुष्कृतकर्माणं नयन्ति परमां गतिम्}


\twolineshloka
{`कुरुक्षेत्रं गमिष्यामि कुरुक्षेत्रे वसाम्यहम्}
{इत्येवं निश्चितो भूत्वा तेन स्वर्गं गमिष्यति}


\twolineshloka
{कुरुक्षेत्रं गमिष्यामि कुरुक्षेत्रे वसाम्यहम्}
{तथा स्थानं च मौनं च वीरासनमुपास्महे}


\twolineshloka
{एवं प्रलपमानोऽपि चिन्तयंश्च मुहुर्मुहुः}
{दूरस्थो यदि वा तिष्ठँल्लभेत्स्वर्गं सुनिश्चितम्'}


\twolineshloka
{सुरर्षभा ब्राह्मणसत्तमाश्चतथा नृगाद्या नरदेवमुख्याः}
{इष्ट्वा महार्हैः क्रतुभिर्नृसिंहसन्त्यज्य देहान्सुगतिं प्रपन्नाः}


\twolineshloka
{तंरन्तुकारन्तुकयोर्यदन्तरंरामहदानां च मचक्रुकस्य च}
{एतत्कुरुक्षेत्रसमन्तपञ्चकंप्रजापतेरुत्तरवेदिरुच्यते}


\twolineshloka
{शिवं महापुण्यमिदं दिवौकसांसुसम्मतं स्वर्गगुणैः समन्वितम्}
{अतश्च सर्वे निहता नृपा रणेयास्यन्ति पुण्यां गतिमक्षयां सदा}


\twolineshloka
{[इत्युवाच स्वयं शक्रः सह ब्रह्मादिभिस्तदा}
{तच्चानुमोदितं सर्वं ब्रह्मविष्णुमहेष्वरैः]}


\chapter{अध्यायः ५६}
\twolineshloka
{वैशम्पायन उवाच}
{}


\twolineshloka
{कुरुक्षेत्रं ततो दृष्ट्वा दत्त्वा देयांश्च सात्वतः}
{आश्रमं सुमहत्पुण्यमगमज्जनमेजय}


\twolineshloka
{मधूकाम्रवणोपेतं प्लुक्षन्यग्रोधसङ्कुलम्}
{चिरबिल्वयुतं पुण्यं पनसार्जुनसङ्कुलम्}


\twolineshloka
{तं दृष्ट्वा यादवश्रेष्ठः प्रवरं पुण्यलक्षणम्}
{पप्रच्छ तानृषीन्सर्वान्कस्याश्रमवरस्त्वयम्}


\twolineshloka
{ते तु सर्वे महात्मानमूचू राजन्हलायुधम्}
{शृणु विस्तरशो राम यस्यायं पूर्वमाश्रमः}


\twolineshloka
{अत्र विष्णुः पुरा देवस्तप्तवांस्तप उत्तमम्}
{अत्रास्य विधिवद्यज्ञाः सर्वे वृत्ताः सनातनाः}


\twolineshloka
{अत्रैव ब्राह्मणी वृद्वा कौमारब्रह्मचारिणी}
{योगयुक्ता दिवं याता तपोयुक्ता विशाम्पते}


\twolineshloka
{बभूव श्रीमती राजञ्शाण्डिल्यस्य महात्मनः}
{सुता धृतव्रता साध्वी नियता ब्रह्मचारिणी}


\twolineshloka
{साऽपि प्राप्य परं योगं गता स्वर्गमनुत्तमम्}
{भुक्त्वाऽश्रमेऽश्वमेधस्य फलं फलवतः शुभम्}


\twolineshloka
{गता स्वर्गं महाराज पूजिता च महात्मभिः}
{अभिगम्याश्रमं पुण्यं स दृष्ट्वा यदुनन्दनः}


\twolineshloka
{ऋषींस्तानभिवाद्याथ पार्श्वे हिमवतोऽच्युतः}
{सन्ध्याकार्याणि सर्वाणि निर्वर्त्यारुरुहेऽचलम्}


\twolineshloka
{नातिदूरं ततो गत्वा नगं तालध्वजो बली}
{पुण्ये तीर्थवरे स्नात्वा विस्मयं परमं गतः}


\twolineshloka
{प्रभवं च सरस्वत्याः प्लक्षप्रस्रवणं बलः}
{सम्प्राप्तः कारपचनं तीर्थप्रवरमुत्तमम्}


\threelineshloka
{हलायुधस्तु तत्रापि दत्त्वा दानं महाबलः}
{आप्लुतः सलिले पुण्ये सुशीते विमले शुचौ}
{सन्तर्पयामास पितॄन्देवांश्च रणदुर्मदः}


\twolineshloka
{तत्रोष्यैकां तु रजनीं यतिभिर्ब्राह्मणैः सह}
{मित्रावरुणयोः पुण्यं जगामाश्रममच्युतः}


\twolineshloka
{इन्द्रोऽग्निरर्यमा चैव यत्र प्राक् प्रीतिमाप्नुवम्}
{तं देशं कारपचनात्स तस्मादाजगामह}


\threelineshloka
{स्नात्वा तत्र च धर्मात्मा परां प्रीतिमवाप्य च}
{ऋषिभिश्चैव सिद्धैश्च सहितो वै महाबलः}
{उपविष्टः कथाः शुभ्राः शुश्राव यदुपुङ्गवः}


\twolineshloka
{तथा तु तिष्ठतां तेषां नारदो भगवानृषिः}
{आजगामाथ तं देशं यत्र रामो व्यवस्थितः}


\twolineshloka
{जटामण्डलसंवीतः कुशचीरी महातपाः}
{हेमदण्डधरो राजन्कमण्डलुधरस्तथा}


\twolineshloka
{महतीं सुखशब्दां तां गृह्य वीणां मनोरमाम्}
{नृत्ये गीते च कुशलो देवब्राह्मणपूजितः}


\twolineshloka
{प्रभवः कलहानं च नित्यं च कलहप्रियः}
{तं देशमगमद्यत्र श्रीमान्रामो व्यवस्थितः}


\twolineshloka
{प्रत्युत्थाय च तं रामः पूजयित्वा यतव्रतम्}
{देवर्षिं पर्यपृच्छत्स यथावृत्तं कुरून्प्रति}


\twolineshloka
{तदाऽस्याकथयद्राजन्नारदः सर्ववेदवित्}
{सर्वमेतद्यथावृत्तमतीतं कुरुसंक्षयम्}


\twolineshloka
{ततोऽब्रवीद्रौहिणेयो नारदं दीनया गिरा}
{किमवस्थं तु तत्क्षत्रं ये तु तत्राभवन्नृपाः}


\threelineshloka
{श्रुतमेतन्मया पूर्वं सर्वमेव तपोधन}
{विस्तरश्रवणे जातं कौतूहलमतीव मे ॥नारद उवाच}
{}


\twolineshloka
{पूर्वमेव हतो भीष्मो द्रोणः सिन्धुपतिस्तथा}
{हतो वैकर्तनः कर्णः पुत्राश्चास्य महारथाः}


\twolineshloka
{भूरिश्रवा रौहिणेय मद्रराजश्च वीर्यवान्}
{एते चान्ये च बहवो हतास्तत्र महाबलाः}


\twolineshloka
{प्रियान्प्राणान्परित्यज्य प्रियार्थं कौरवस्य वै}
{राजानो राजपुत्राश्च समरेष्वनिवर्तिनः}


% Check verse!
अहतांस्तु महाबाहो शृणु मे तत्र माधव
\threelineshloka
{धार्तराष्ट्रबले शेषास्त्रयः समितिमर्दनाः}
{कृपश्च कृतवर्मा च द्रोणपुत्रश्च वीर्यवान्}
{तेऽपि वै विद्रुता राम दिशो दश भयात्तदा}


\twolineshloka
{दुर्योधनो हते सैन्ये विद्रुतेषु पदातिषु}
{हदं द्वैपायनं नाम विवेश भृशदुःखितः}


\twolineshloka
{शयानं धार्तराष्ट्रं तु सलिले स्तम्भिते तदा}
{पाण्डवाः सह कृष्णेन वाग्भिरुग्राभिरार्दयन्}


\twolineshloka
{स तुद्यमानो बलवान्वाग्भी राम समन्ततः}
{उत्थितः स हदाद्वीरः प्रगृह्य महतीं गदाम्}


\twolineshloka
{स चाप्युपागतो योद्धुं भीमेन सह साम्प्रतम्}
{भविष्यति तयोरद्य युद्धं राम सुदारुणम्}


\threelineshloka
{यदि कौतूहलं तेऽस्ति व्रज माधव मा चिरम्}
{पश्य युद्धं महाघोरं शिष्ययोर्यदि मन्यसे ॥वैशम्पायन उवाच}
{}


\threelineshloka
{नारदस्य वचः श्रुत्वा तानभ्यर्च्य द्विजर्षभान्}
{सर्वान्विसर्जयामास ये तेनाभ्यागताः सह}
{गम्यतां द्वारकां चेति सोऽन्वशादनुयायिनः}


\threelineshloka
{सोऽवतीर्याचलश्रेष्ठात्प्लक्षप्रस्रवणाच्छुभात्}
{ततः प्रीतमनाः रामः श्रुता तीर्थफलं महत्}
{विप्राणां सन्निधौ श्लोकमगायदिममच्युतः}


\twolineshloka
{सरस्वतीवाससमा कुतो रतिःसरस्वतीवाससमाः कुतो गुणाः}
{सरस्वतीं प्राप्य दिवं गता जनाःसदा स्मरिष्यन्ति नदीं सरस्वतीम्}


\twolineshloka
{सरस्वती सर्वनदीषु पुण्यासरस्वती लोकशुभावहा सदा}
{सरस्वतीं प्राप्य जनाः सुदुष्कृतंसदा न शोचन्ति परत्र चेह च}


\twolineshloka
{ततो मुहुर्मुहुः प्रीत्या प्रेक्षमाणः सरस्वतीम्}
{हयैर्युक्तं रथं शुभ्रमारुरोह परन्तपः}


\twolineshloka
{स शीघ्रगामिना तेन रथेन यदुपुङ्गवः}
{दिदृक्षुरभिसम्प्राप्तः शिष्ययुद्वमुपस्थितम्}


\threelineshloka
{एवं तदभवद्युद्वं तुमुलं जनमेजयं}
{यत्र दुःखान्वितो राजा धृतराष्ट्रोऽब्रवीदिदम् ॥धृतराष्ट्र उवाच}
{}


\twolineshloka
{राम सन्निहितं श्रुत्वा गदायुद्ध उपस्थिते}
{मम पुत्रः कथं भीमं प्रत्ययुध्यत सञ्जय}


\chapter{अध्यायः ५७}
\twolineshloka
{सञ्जय उवाच}
{}


\twolineshloka
{रामसान्निध्यमागम्य पुत्रो दुर्योधनस्तव}
{योद्धुकामो महाबाहुः समहृष्यत वीर्यवान्}


% Check verse!
दृष्ट्वा लाङ्गलिनं राजा प्रत्युत्थाय च भारत

प्रीत्या*एतच्छ्लोकस्य स्थाने झ.पुस्तके अधोलिखिताः पञ्च श्लोका वर्तन्ते

ते च

प्रीत्या परमया युक्ताः समभ्यर्च्य यथाविधि

आसनं च ददौ तस्मै पर्यपृच्छदनामयम् ॥ततो युधिष्ठिरं रामो वाक्यमेतदुवाच ह

मधुरं धर्मसंयुक्तं शूराणां हितमेव च ॥मया श्रुतं कथयतामृषीणां राजसत्तम

कुरुक्षेत्रं परं पुण्यं पावनं स्वर्ग्यमेव च ॥देवतैर्ऋषिभिर्जुष्टं ब्राह्मणैश्च महात्मभिः

तत्र वै योत्स्यमाना ये देहं त्यक्ष्यन्ति मानवाः ॥तेषां स्वर्गे ध्रुवो वासः शक्रेण सह मारिष

तस्मात्समन्तपञ्चकमितो याम द्रुतं नृप ॥ परमया युक्तो युधिष्ठिरमथाब्रवीत् ॥दुर्योधन उवाच


\twolineshloka
{प्रीत्या परमया युक्ताः समभ्यर्च्य यथाविधि}
{आसनं च ददौ तस्मै पर्यपृच्छदनामयम्}


\twolineshloka
{ततो युधिष्ठिरं रामो वाक्यमेतदुवाच ह}
{मधुरं धर्मसंयुक्तं शूराणां हितमेव च}


\twolineshloka
{मया श्रुतं कथयतामृषीणां राजसत्तम}
{कुरुक्षेत्रं परं पुण्यं पावनं स्वर्ग्यमेव च}


\twolineshloka
{देवतैर्ऋषिभिर्जुष्टं ब्राह्मणैश्च महात्मभिः}
{तत्र वै योत्स्यमाना ये देहं त्यक्ष्यन्ति मानवाः}


\twolineshloka
{तेषां स्वर्गे ध्रुवो वासः शक्रेण सह मारिष}
{तस्मात्समन्तपञ्चकमितो याम द्रुतं नृप}


\twolineshloka
{समन्तपञ्चकं पुण्यमितो याम विशाम्पते}
{प्रथितोत्तरवेदी सा देवलोके प्रजापतेः}


\twolineshloka
{तस्मिन्महापुण्यतमे त्रैलोक्यस्य सनातने}
{सङ्ग्रामे निधनं प्राप्य ध्रुवं स्वर्गं गमिष्यसि}


\twolineshloka
{तथेत्युक्त्वा महाराज कुन्तीपुत्रो युधिष्ठिरः}
{समन्तपञ्चकं वीरः प्रायादभिमुखः प्रभुः}


\twolineshloka
{ततो दुर्योधनो राजा प्रगृह्य महतीं गदाम्}
{पद्ध्याममर्षी द्युतिमानगच्छत्पाण्डवैः सह}


\twolineshloka
{तथा यान्तं गदाहस्तं वर्मणा चापि दंशितम्}
{अन्तरिक्षचरा देवाः साधुसाध्वित्यपूजयन्}


% Check verse!
वादकाश्च नरास्तत्र दृष्ट्वा ते हर्षमागताः
\twolineshloka
{स पाण्डवैः परिवृतः कुरुराजस्तवात्मजः}
{मत्तस्येव गजेन्द्रस्य गतिमास्थाय सोऽव्रजत्}


\twolineshloka
{ततः शङ्खनिनादैश्च भेरीणां च महास्वनैः}
{सिंहनादैश्च शूराणां दिशः सर्वाः प्रपूरिताः}


\threelineshloka
{ततस्ते तु कुरुक्षेत्रं प्राप्ता नरवरोत्तमाः}
{प्रतीच्यभिमुखं देशं यथोद्दिष्टं सुतेन ते}
{दक्षिणेन सरस्वत्याः स्वयनं तीर्थमुत्तमम्}


% Check verse!
तस्मिन्देशे त्वनिरिणे ते तु युद्धमरोचयन्
\twolineshloka
{ततो भीमो महाकोटिं गदां गृह्याथ वर्मभृत्}
{बिभ्रद्रूपं महाराज सदृशं हि गरुत्मतः}


\twolineshloka
{अवबद्धशिरस्त्राणः शुद्धकाञ्चनवर्मभृत्}
{रराज राजन्पुत्रस्ते काञ्चनः शैलराडिव}


\twolineshloka
{तौ तथा सङ्गतौ वीरौ भीमदुर्योधनावुभौ}
{संयुगे सम्प्रकाशेते संरब्धाविव कुञ्जरौ}


\twolineshloka
{रथमण्डलमध्यस्थौ भ्रातरौ तौ नरर्षभौ}
{अशोभेतां महाराज चन्द्रसूर्याविवोदितौ}


\twolineshloka
{तावन्योन्यं निरीक्षेतां क्रुद्धाविव महोरगौ}
{दहन्तौ लोचनै राजन्परस्परवधैषिणौ}


\twolineshloka
{संप्रहृष्टमना राजन्गदामादाय कौरवः}
{सृक्विणी संलिहन्राजन्क्रोधरक्तेक्षणः श्वसन्}


\twolineshloka
{ततो दुर्योधनो राजन्गदामादाय वीर्यवान्}
{भीमसेनमभिप्रेक्ष्य गजो गजमिवाह्वयत्}


\twolineshloka
{अद्रिसारमयीं भीमस्तथैवादाय वीर्यवान्}
{आह्वयामास नृपतिं सिंहः सिंहं यथा वने}


\twolineshloka
{तावुद्यतगदापाणी दुर्योधनवृकोदरौ}
{संयुगे सम्प्रकाशेतां गिरी सशिखराविव}


\twolineshloka
{तावुभौ समतिक्रुद्धावुभौ भीमपराक्रमौ}
{उभौ शिष्यौ गदायुद्धे रौहिणेयस्य धीमतः}


\twolineshloka
{उभौ सदृशकर्माणौ वरुणस्य महाबलौ}
{वासुदेवस्य रामस्य तथा वैश्रवणस्य च}


\twolineshloka
{सदृशौ तौ महाराज मधुकैटभयोरपि}
{उभौ सदृशकर्माणौ तथा सुन्दोपसुन्दयोः}


\twolineshloka
{[रामरावणयोश्चैव वालिसुग्रीवयोस्तथा]}
{तथैव कालस्य समौ मृत्योश्चैव परन्तपौ}


\twolineshloka
{अन्योन्यमभिधावन्तौ मत्ताविव महाद्विपौ}
{वासितासङ्गमे दृप्तौ शरदीव मदोत्कटौ}


\twolineshloka
{उभौ क्रोधविषं दीप्तं वमन्तावुरगाविव}
{अन्योन्यमभिसंरब्धौ प्रेक्षमाणावरिन्दमौ}


\twolineshloka
{उभौ भरतशार्दूलौ विक्रमेण समन्वितौ}
{सिंहाविव दुराधर्षौ गदायुद्धविशारदौ}


\twolineshloka
{नखदंष्ट्रायुधौ वीरौ व्याघ्राविव दुरुत्सहौ}
{प्रजासंहरणे क्षुब्धौ समुद्राविव दुस्तरौ}


\twolineshloka
{लोहिताङ्गाविव क्रुद्धौ प्रतपन्तौ महारथौ}
{[पूर्वपश्चिमजौ मेघौ प्रेक्षमाणावरिन्दमौ}


\twolineshloka
{गर्जमानौ सुविषमं क्षरन्तौ प्रावृषीव हि}
{रश्मियुक्तौ महात्मानौ दीप्तिमन्तौ महाबलौ}


\twolineshloka
{ददृशाते कुरुश्रेष्ठौ कालसूर्याविवोदितौ}
{व्याघ्राविव सुसंरब्धौ गर्जन्ताविवतोयदौ}


\twolineshloka
{जहृषाते महाबाहु सिंहकेसरिणाविव}
{गजाविव सुसंरब्धौ ज्वलिताविव पावकौ}


\twolineshloka
{ददृशाते महात्मानौ सशृङ्गाविव पर्वतौ}
{रोषात्प्रस्फुरमाणोष्ठौ निरीक्षन्तौ परस्परम्}


\twolineshloka
{तौ समेतौ महात्मानौ गदाहस्तौ नरोत्तमौ}
{]उभौ परमसंहृष्टावुभौ परमसम्मतौ}


\threelineshloka
{सदश्वाविव हेषन्तौ बृंहन्ताविव कुञ्जरौ}
{वृषभाविव गर्जन्तौ दुर्योधनवृकोदरौ}
{दैत्याविव बलोन्मत्तौ रेजतुस्तौ नरोत्तमौ}


\twolineshloka
{ततो दुर्योधनो राजन्निदमाह युधिष्ठिरम्}
{सृञ्जयैः सह तिष्ठन्तं तपन्तमिव भास्करम्}


\twolineshloka
{इदं व्यवसितं युद्धं मम भीमस्य चोभयोः}
{उपोपविष्टाः पश्यध्वं विमर्दं नृपसत्तमाः}


\twolineshloka
{ततः समुपविष्टं तत्सुमहद्राजमण्डलम्}
{विराजमानं ददृशे दिवीवादित्यमण्डलम्}


\twolineshloka
{तेषां मध्ये महाबाहुः श्रीमान्केशवपूर्वजः}
{उपविष्टो महाराज पूज्यमानः समन्ततः}


\twolineshloka
{शुशुभे राजमध्यस्थो नीलवासाः सितप्रभः}
{नक्षत्रैरिव सम्पूर्णो वृतो निशि निशाकरः}


\twolineshloka
{तौ तथा तु महाराज गदाहस्तौ सुदुःसहौ}
{अन्योन्यं वाग्भिरुग्राभिस्तक्षमाणौ व्यवस्थितौ}


\twolineshloka
{अप्रियाणि ततोऽन्योन्यमुक्त्वा तौ कुरुसत्तमौ}
{उदीक्षन्तौ स्थितौ तत्र वृत्रशक्रौ यथाऽऽहवे}


\chapter{अध्यायः ५८}
\twolineshloka
{वैशम्पायन उवाच}
{}


\twolineshloka
{ततो वाग्युद्धमभवत्तुमुलं जनमेजय}
{यत्र दुःखान्वितो राजा धृतराष्ट्रोऽब्रवीदिदम्}


\twolineshloka
{धिगस्तु खलु मानुष्यं यस्य निष्ठेयमीदृशी}
{एकादशचमूभर्ता यत्र पुत्रो ममानघ}


\twolineshloka
{आज्ञाप्य सर्वान्नृपतीन्भुक्त्वा चेमां वसुन्धराम्}
{गदामादाय चैकाकी पदातिः प्रस्थितो रणम्}


\twolineshloka
{भूत्वा हि जगतो नाथो ह्यनाथ इव मे सुतः}
{गदामुद्यम्य यो याति किमन्यद्भागधेयतः}


\threelineshloka
{अहो दुःखं महत्प्राप्तं पुत्रेण मम स़ञ्जय}
{एवमुक्त्वा स दुःखार्तो विरराम जनाधिपः ॥सञ्जय उवाच}
{}


\twolineshloka
{स मेघनिनदो हर्षान्निनदन्निव गोवृषः}
{आजुहाव तदा पार्थं युद्धाय युधि वीर्यवान्}


\twolineshloka
{भीममाह्वयमाने तु कुरुराजे महात्मनि}
{प्रादुरासन्सुघोराणि रूपाणि विविधान्युत}


\twolineshloka
{ववुर्वाताः सनिर्घाताः पांसुवर्षं पपात च}
{बभूवुश्च दिशः सर्वास्तिमिरेण समावृताः}


\twolineshloka
{महास्वनाः सुनिर्वातास्तुमुला रोमहर्षणाः}
{पेतुस्तथोल्काः शतशः स्फोटयन्त्यो नभस्तलात्}


\twolineshloka
{राहुश्चाग्रसदादित्यमपर्वणि विशाम्पते}
{चकम्पे च महाकम्पं पृथिवी सवनद्रुमा}


\twolineshloka
{रूक्षाश्च वाताः प्रववुर्नीचैः शर्करकर्षिणः}
{गिरीणां शिखराण्येव न्यपतन्त महीतले}


\twolineshloka
{मृगा बहुविधाकाराः सम्पतन्ति दीशो दश}
{दीप्ताः शिवाश्चाप्यनदन्घोररूपाः सुदारुणाः}


\twolineshloka
{निर्घाताश्च महाघोरा बभूवू रोमहर्षणाः}
{दीप्तायां दिशि राजेन्द्र मृगाश्चाशुभवादिनः}


\twolineshloka
{उदपानगताश्चापो व्यवर्धन्त समन्ततः}
{अशरीरा महानादाः श्रूयन्ते स्म भयानकाः}


\twolineshloka
{एवमादीनि दृष्ट्वा स निमित्तानि वृकोदरः}
{उवाच भ्रातरं ज्येष्ठं धर्मराजं युधिष्ठिरम्}


% Check verse!
एष शक्यो मया जेतुं मन्दात्मा च सुयोधनः
\twolineshloka
{अद्य क्रोधं विमोक्ष्यामि निगूढं हृदयेशयम्}
{सुयोधने कौरवेन्द्रे खाण्डवेऽग्निमिवार्जुनः}


\twolineshloka
{शल्यमद्योद्धरिष्यामि तव पाण्डव हृच्छयम्}
{निहत्य गदया पापमिमं कुरुकुलान्तकम्}


\twolineshloka
{अद्य कीर्तिमयीं मालां प्रतिमोक्ष्याम्यहं त्वयि}
{हत्वेमं पापकर्माणं गदया रणमूर्धनि}


\twolineshloka
{अद्योरुगदया राजन्भेत्ताऽस्मि समरेऽनया}
{नायं प्रवेष्टा नगरं पुनर्वारणसाह्वयम्}


\twolineshloka
{सर्पोत्सर्गस्य शयने विषदानस्य भोजने}
{प्रमाणकोट्यां पातस्य दाहस्य जतुवेश्मनि}


\twolineshloka
{सभायामवहासस्य सर्वस्वहरणस्य च}
{वर्षमज्ञातवासस्य वनवासस्य चानघ}


\twolineshloka
{अद्यान्तमेषां दुःखानां गन्ताऽहं भरतर्षभ}
{एकाह्ना विनिहत्येमं भविष्याम्यात्मनोऽनृणः}


\twolineshloka
{अद्यायुर्धार्तराष्ट्रस्य दुर्मतेरकृतात्मनः}
{समाप्तं भरतश्रेष्ठ मातापित्रोश्च दर्शनम्}


\twolineshloka
{अद्य सौख्यं तु राजेन्द्र कुरुराजस्य दुर्मतेः}
{समाप्तं च महाराज नारीणां दर्शनं पुनः}


\twolineshloka
{अद्यायं कुरुराजस्य शन्तनोः कुलपांसनः}
{प्राणाञ्श्रियं च राज्यं च त्यक्त्वा शेष्यति भूतले}


\twolineshloka
{राजा च धृतराष्ट्रोऽद्य श्रुत्वा पुत्रं निपातितम्}
{स्मरिष्यत्यशुभं कर्म यत्तच्छकुनिबुद्धिजम्}


\twolineshloka
{इत्युक्त्वा राजशार्दूल गदामादाय वीर्यवान्}
{अभ्यतिष्ठत युद्धाय शक्रो वृत्रमिवाह्वयन्}


\twolineshloka
{तमुद्यतगदं दृष्ट्वा कैलासमिव शृङ्गिणम्}
{भीमसेनः पुनः क्रुद्धो दुर्योधनमुवाच ह}


\twolineshloka
{राज्ञश्च धृतराष्ट्रस्य तथा त्वमपि चात्मनः}
{स्मर तद्दुष्कृतं कर्म यद्वृत्तं वारणावते}


\twolineshloka
{द्रौपदी च परिक्लिष्टा सभामध्ये रजस्वला}
{द्यूते च वञ्चितो राजा यत्त्वया सौबलेन च}


\threelineshloka
{वने दुःखं च यत्प्राप्तमस्माभिस्त्वत्कृतं महत्}
{विराटनगरे चैव योन्यन्तरगतैरिव}
{तत्सर्वं पातयाम्यद्य दिष्ट्या दृष्टोऽसि दुर्मते}


\twolineshloka
{त्वत्कृतेऽसौ हतः शेते शरतल्पे पितामहः}
{गाङ्गेयो रथिनां श्रेष्ठो रथिना याज्ञसेनिना}


\twolineshloka
{हतो द्रोणश्च कर्णश्च तथा शल्यः प्रतापवान्}
{वैराग्नेरादिकर्तासौ शकुनिः सौबलो हतः}


\twolineshloka
{प्रातिकामी तथा पापो द्रौपद्याः क्लेशकृद्धतः}
{भ्रातरस्ते हताः सर्वे शूरा विक्रान्तयोधिनः}


\twolineshloka
{एते चान्ये च बहवो निहतास्त्वत्कृते नृपाः}
{त्वामद्य निहनिष्यामि गदया नात्र संशयः}


\twolineshloka
{इत्येवमुच्चै राजानं भाषमाणं वृकोदरम्}
{उवाच गतभी राजन्पुत्रस्ते सत्यविक्रमः}


\twolineshloka
{किं कत्थनेन बहुना युध्यस्वेह वृकोदर}
{अद्य तेऽहं विनेष्यामि युद्धश्रद्धां कुलाधम}


\twolineshloka
{इति दुर्यार्धेनोऽक्षुद्रस्त्वया क्षुद्रबलेन वै}
{शक्यस्रासयितुं वाचा न चान्यः प्राकृतो यथा}


\twolineshloka
{चिरकालेप्सितं दिष्ट्या हृदयस्थमिदं मम}
{त्वया सह गदायुद्धं त्रिदशैरुपपादितम्}


\twolineshloka
{किं वाचा बहुनोक्तेन कत्थितेन च दुर्मते}
{वाणी सङ्कोच्यतामेषा कर्मणा मा चिरं कृथाः}


\twolineshloka
{तस्य तद्वचनं श्रुत्वा सर्व एवाभ्यपूजयन्}
{राजानः सोमकाश्चैव ये तत्रासन्समागताः}


\twolineshloka
{ततः सम्पूजितः सर्वैः सम्प्रहृष्टतनूरुहः}
{भूयो धीरां मतिं चक्रे युद्धाय कुरुनन्दनः}


\twolineshloka
{तं मत्तमिव मातङ्गं तलशब्दैर्नराधिपाः}
{भूयः संहर्षयांचक्रुर्दुर्योधनममर्षणम्}


\twolineshloka
{तं महात्मा महात्मानं गदामुद्यम्य पाण्डवः}
{अभिदुद्राव वेगेन धार्तराष्ट्रं वृकोदरः}


\twolineshloka
{बृंहन्ति कुञ्चरास्तत्र हया हेषन्ति चासकृत्}
{शस्त्राणि चाप्यदीप्यन्त पाण्डवानां जयैषिणां}


\chapter{अध्यायः ५९}
\twolineshloka
{सञ्जय उवाच}
{}


\twolineshloka
{ततो दुर्योधनो दृष्ट्वा भीमसेनं तथाऽऽगतम्}
{प्रत्युद्ययावदीनात्मा वेगेन महता नदन्}


\twolineshloka
{समापेततुरन्योन्यं शृङ्गिणौ वृषभाविव}
{महानिर्घातघोषश्च प्रहाराणामजायत}


\twolineshloka
{अभवच्च तयोर्युद्वं तुमुलं रोमहर्षणम्}
{जिघृक्षतोर्यथाऽन्योन्यमिन्द्रप्रह्लादयोरिव}


\twolineshloka
{रुधिरोक्षितसर्वाङ्गौ गदाहस्तौ मनस्विनौ}
{ददृशाते महात्मानौ पुष्पिताविव किंशुकौ}


\twolineshloka
{तथा तस्मिन्महायुद्धे वर्तमाने सुदारुणे}
{खद्योतसङ्खैरिव खं दर्शनीयं व्यरोचत}


\twolineshloka
{तथा तस्मिन्वर्तमाने सङ्कुले तुमुले भृशम्}
{उभावपि परिश्रान्तौ युध्यमानावरिन्दमौ}


\twolineshloka
{तौ मुहूर्तं समाश्वस्य पुनरेव परन्तपौ}
{सम्प्रहारयतां चित्रे सम्प्रगृह्य गदे शुभे}


\twolineshloka
{तौ तु दृष्ट्वा महावीर्यौ समाश्वस्तौ नरर्षभौ}
{बलिनौ वारणौ यद्वद्वासितार्थे मदोत्कटौ}


\twolineshloka
{समानवीर्यौ सम्प्रेक्ष्य प्रगृहीतगदावुभौ}
{प्रहर्षं परमं जग्मुर्देवगन्धर्वदानवाः}


\twolineshloka
{प्रगृहीतगदौ दृष्ट्वा दुर्योधनवृकोदरौ}
{संशयः सर्वभूतानां विजये समपद्यत}


\twolineshloka
{समागम्य ततो भूयो भ्रातरौ बलिनां वरौ}
{अन्योन्यस्यान्तरप्रेप्सू प्रचक्रातेऽन्तरं प्रति}


\twolineshloka
{यमदण्डोपमां गुर्वीमिन्द्राशनिमिवोद्यताम्}
{ददृशुः प्रेक्षका राजन्रौद्रीं विशसनीं गदाम्}


\twolineshloka
{आविद्ध्यतो गतां तस्य भीमसेनस्य संयुगे}
{शब्दः सुतुमुलो घोरो मुहूर्तं समपद्यत}


\twolineshloka
{आविद्ध्यन्तमरिं प्रेक्ष्य धार्तराष्ट्रोऽथ पाण्डवम्}
{गदामतुलवेगां तां विस्मितः सम्बभूव ह}


\twolineshloka
{चरंश्च विविधान्मार्गान्मण्डलानि च भारत}
{अशोभत तदा वीरो भूय एव वृकोदरः}


\twolineshloka
{तौ परस्परमासाद्य यत्तावन्योन्यसूदने}
{मार्जाराविव भक्षार्थे ततक्षाते मुहुर्मुहुः}


\twolineshloka
{अचरद्भीमसेनस्तु मार्गान्बहुविधांस्तथा}
{मण्डलानि विचित्राणि गतप्रत्यागतानि च}


\twolineshloka
{गोमूत्रिकाणि चित्राणि स्थानानि विविधानि च}
{परिमोक्षं प्रहाराणां वर्जनं परिधावनम्}


\twolineshloka
{अभिद्रवणमाक्षेपमवस्थानं सविग्रहम्}
{मत्स्योद्वृत्तं सोरुवृत्तमवप्लुतमुपप्लुतम्}


\twolineshloka
{उपन्यस्तमपन्यस्तं गदायुद्धविशारदौ}
{एवं तौ विचरन्तौ तु न्यघ्नतां वै परस्परम्}


\twolineshloka
{वञ्चयानौ पुनश्चैव चेरतुः कुरुसत्तमौ}
{विक्रीडन्तौ सुबलिनौ मण्डलानि विचेरतुः}


\twolineshloka
{[तौ दर्शयन्तौ समरे युद्धक्रीडां समन्ततः}
{गदाभ्यां सहसाऽन्योन्यमाजघ्नतुररिन्दमौ}


\twolineshloka
{परस्परं समासाद्य दंष्ट्राभ्यां द्विरदौ यथा}
{अशोभेतां महाराज शोणितेन परिप्लुतौ}


\twolineshloka
{एवं तदभवद्युद्धं घोररूपं परन्तप}
{परिवृत्तेऽहनि क्रूरं वृत्रवासवयोरिव ॥]}


\threelineshloka
{गदाहस्तौ ततस्तौ तु मण्डलावस्थितौ बली}
{दक्षिणं मण्डलं राजन्धार्तराष्ट्रोऽभ्यवर्तत}
{सव्यं तु मण्डलं तत्र भीमसेनोऽभ्यवर्तत}


\twolineshloka
{तथा तु चरतस्तस्य भीमस्य रणमूर्धनि}
{दुर्योधनो महाराज पार्श्वदेशेऽभ्यताडयत्}


\twolineshloka
{आहतस्तु ततो भीमः पुत्रेण तव भारत}
{आविध्यत गदां गुवीं प्रहारं तमचिन्तयन्}


\twolineshloka
{इन्द्राशनिसमां घोरां यमदण्डमिवोद्यताम्}
{ददृशुस्ते महाराज भीमसेनस्य तां गदाम्}


\twolineshloka
{आविध्यन्तं गदां दृष्ट्वा भीमसेनं तवात्मजः}
{समुद्यम्य गदां घोरां प्रत्यविध्यत्परन्तपः}


\twolineshloka
{गदामारुतवेगेन तव पुत्रस्य भारत}
{शब्द आसीत्सुतुमुलस्तेजश्च समजायत}


\twolineshloka
{स चरन्विविधान्मार्गान्मण्डलानि च भागशः}
{समशोभत तेजस्वी भूयो भीमात्सुयोधनः}


\twolineshloka
{आविद्धा सर्ववेगेन भीमेन महती गदा}
{सधूमं सार्चिषं चाग्निं मुमोचोग्रमहास्वना}


\twolineshloka
{आधूतां भीमसेनेन गदां दृष्ट्वा सुयोधनः}
{अद्रिसारमयीं गुर्वीमाविध्यन्बह्वशोभत}


\twolineshloka
{गदामारुतवेगं हि दृष्ट्वा तस्य महात्मनः}
{भयं विवेश पाण्डूंस्तु सर्वानेव ससात्यकीन्}


\twolineshloka
{तौ दर्शयन्तौ समरे युद्धक्रीडां समन्ततः}
{गदाभ्यां सहसाऽन्योन्यमाजघ्नतुररिन्दमौ}


\twolineshloka
{तौ परस्परमासाद्य दंष्ट्राभ्यां द्विरदौ यथा}
{अशोभेतां महाराज शोणितेन परिप्लुतौ}


\twolineshloka
{एवं तदभवद्युद्धं घोररूपमसंवृतम्}
{परिवृत्तेऽहनि क्रूरं वृत्रवासवयोरिव}


\twolineshloka
{दृष्ट्वा व्यवस्थितं भीमं तव पुत्रो महाबलः}
{चरंश्चित्रतरान्मा र्गान्कौन्तेयमभिदुद्रुवे}


\twolineshloka
{तस्य भीमो महावेगां जाम्बूनदपरिष्कृताम्}
{अतिक्रुद्धस्य क्रुद्धस्तु ताडयामास तां गदाम्}


\twolineshloka
{सविस्फुलिङ्गो निर्हादस्तयोस्तत्राभिघातजः}
{प्रादुरासीन्महाराज घृष्टयोर्वज्रयोरिव}


\twolineshloka
{वेगवत्या तया तत्र भीमसेनप्रमुक्तया}
{निपतन्त्या महाराज पृथिवी समकम्पत}


\twolineshloka
{तां नामृष्यत कौरव्यो गदां प्रतिहतां रणे}
{मत्तो द्विप इव क्रुद्धः प्रतिकुञ्जरदर्शनात्}


\twolineshloka
{स सव्यं मण्डलं राजा उद्धाम्य कृतनिश्चयः}
{आजघ्ने मूर्ध्नि कौन्तेयं गदया भीमवेगया}


\twolineshloka
{तया त्वभिहतो भीमः पुत्रेण तव पाण्डवः}
{नाकम्पत महाराज तदद्भुतमिवाभवत्}


\twolineshloka
{आश्चर्यं चापि तद्राजन्सर्वसैन्यान्यपूजयन्}
{यद्गदाभिहतो भीमो नाकम्पत पदात्पदम्}


\twolineshloka
{ततो गुरुतरां दीप्तां गदां हेमपरिष्कृताम्}
{दुर्योधनाय व्यसृजद्भीमो भीमपराक्रमः}


\twolineshloka
{तं प्रहारमसम्भ्रान्तो लाघवेन महाबलः}
{मोघं दुर्योधनश्चक्रे तत्राभूद्विस्मयो महान्}


\twolineshloka
{सा तु मोघा गदा राजन्पतन्ती भीमचोदिता}
{चालयामास पृथिवीं महानिर्घातनिःस्वनाः}


\twolineshloka
{आस्थाय कौशिकान्मार्गानुत्पतन्स पुनः पुनः}
{गदानिपातं प्रज्ञाय भीमसेनं च वञ्चितम्}


\twolineshloka
{वञ्चयित्वा तदा भीमं गदया कुरुसत्तमः}
{ताडयामास सङ्क्रुद्धो वक्षोदेशे महाबलः}


\twolineshloka
{गदया निहतो भीमो मुह्यमानो महारणे}
{नाभ्यमन्यत कर्तव्यं पुत्रेणाभ्याहतस्तव}


\twolineshloka
{तस्मिंस्तथा वर्तमाने राजन्सोमकपाण़्डवाः}
{भृशोपहतसङ्कल्पा नहृष्टमनसोऽभवन्}


\twolineshloka
{स तु तेन प्रहारेण मातङ्ग इव रोषितः}
{हस्तिवद्वस्तिसंकाशमभिदुद्राव ते सुतम्}


\twolineshloka
{ततस्तु तरसा भीमो गदया तनयं तव}
{अभिदुद्राव वेगेन सिंहो वनगजं यथा}


\twolineshloka
{उपसृत्य तु राजानं गदामोक्षविशारदः}
{आविध्यत गदां राजन्समुद्दिश्य सुतं तव}


\twolineshloka
{अताडयद्भीमसेनः पार्श्वे दुर्योधनं तदा}
{स विह्वलः प्रहारेण जानुभ्यामगमन्महीम्}


\twolineshloka
{तस्मिन्कुरुकुलश्रेष्ठे जानुभ्यामवनीं गते}
{उदतिष्ठत्ततो नादः सृञ्जयानां जगत्पते}


\twolineshloka
{तेषां तु निनदं श्रुत्वा शृञ्जयानां नरर्षभः}
{अमर्षाद्भरतश्रेष्ठ पुत्रस्ते समकुप्यत}


\twolineshloka
{उत्थाय तु महाबाहुर्महानाग इव श्वसन्}
{दिधक्षन्निव नेत्राभ्यां भीमसेनमवैक्षत}


\twolineshloka
{ततः स भरतश्रेष्ठो गदापाणिरभिद्रवन्}
{प्रमथिष्यन्निव शिरो भीमसेनस्य संयुगे}


\twolineshloka
{स महात्मा महात्मानं भीमं भीमपराक्रमः}
{अताडयच्छङ्खदेशे न चचालाचलोपमः}


\twolineshloka
{स भूयः शुशुभे पार्थस्ताडितो गदया रणे}
{उद्भिन्नरुधिरो राजन्प्रभिन्न इव कुञ्चरः}


\twolineshloka
{ततो गदां वीरहणीमयोमयींप्रगृह्य वज्राशनितुल्यनिःस्वनाम्}
{अताडयच्छत्रुममित्रकर्शनोबलेन विक्रम्य धनञ्जयाग्रजः}


\twolineshloka
{स भीमसेनाभिहतस्तवात्मजःपपात सङ्कम्पितदेहबन्धनः}
{सुपुष्पितो मारुतवेगताडितोवने महासाल इवावघूर्णितः}


\twolineshloka
{ततः प्रणेदुर्जहृषुश्च पाण्डवाःसमीक्ष्य पुत्रं पतितं क्षितौ तव}
{ततः सुतस्ते प्रतिलभ्य चेतनांसमुत्पपात द्विरदो यथा हदात्}


\twolineshloka
{स पार्थिवो नित्यममर्षितस्तदामहारथः शिक्षितवत्परिभ्रमन्}
{अताडयत्पाण्डवमग्रतः स्थितंस विह्वलाङ्गो जगतीमुपास्पृशत्}


\twolineshloka
{स सिंहनादं विननाद कौरवोनिपात्य भूमौ युधि भीममोजसा}
{बिभेद चैवाशनितुल्यमोजसागदानिपातेन शरीररक्षणम्}


\twolineshloka
{ततोऽन्तरिक्षे निनदो महानभू--द्दिवौकसामप्सरसां च नेदुषाम्}
{पपात चोच्चैरमरप्रवेरितंविचित्रपुष्पोत्करवर्षमुत्तमम्}


\twolineshloka
{ततः परानाविशदुत्तमं भयंसमीक्ष्य भूमौ पतितं नरोत्तमम्}
{अहीयमानं च बलेन कौरवंनिशाम्य भेदं सुदृढस्य वर्मणः}


\twolineshloka
{ततो मुहूर्तादुपलभ्य चेतनांप्रमृज्य वक्त्रं रुधिराक्तमात्मनः}
{धृतिं समालम्ब्य विवृत्य लोचनेबलेन संस्तभ्य वृकोदरः स्थितः}


\twolineshloka
{`ततो यमौ यमसदृशौ पराक्रमेसपार्षतः शिनितनयश्च वीर्यवान्}
{समाह्वयन्नहमहमित्यभित्वरं--स्तवात्मजं समभिययुर्वधैषिणः}


\twolineshloka
{निवर्त्य तान्पुनरपि पाण्डवो बलीतवात्मजं स्वयमभिगम्य कालवत्}
{चचार चाप्यपगतखेदवेपथुःसुरेश्वरो नमुचिमिवोत्तमं रणे'}


\chapter{अध्यायः ६०}
\twolineshloka
{सञ्चय उवाच}
{}


\twolineshloka
{समुदीर्णं ततो दृष्ट्वा सङ्ग्रामं कुरुमुख्ययोः}
{अब्रवीदर्जुनस्तत्र वासुदेवं यशस्विनम्}


\threelineshloka
{अनयोर्वीरयोर्युद्धे को ज्यायान्भवतो मतः}
{कस्य वा को गुणो ज्यायानेतद्वद जनार्दन ॥वासुदेव उवाच}
{}


\twolineshloka
{उपदेशोऽनयोस्तुल्यो भीमस्तु बलवत्तरः}
{कृती यत्नपरस्त्वेष धार्तराष्ट्रो वृकोदरात्}


\twolineshloka
{भीमसेनस्तु धर्मेण युध्यमानो न जेष्यति}
{अन्यायेन तु युध्यन्वै हन्यादेव सुयोधनम्}


\twolineshloka
{मायया निर्जिता देवैरसुरा इति नः श्रुतम्}
{विरोचनस्तु शक्रेण मायया निर्जितः स वै}


\twolineshloka
{मायया चाक्षिपत्तेजो वृत्रस्य बलसूदनः}
{तस्मान्मायामयं वीर आतिष्ठतु वृकोदरः}


\twolineshloka
{प्रतिज्ञातं च भीमेन द्यूतकाले धनञ्जय}
{ऊरू भेत्स्यामि ते सङ्ख्ये गदयेति सुयोधनम्}


\twolineshloka
{सोऽयं प्रतिज्ञां तां चापि पारयत्वरिकर्शनः}
{मायाविनं तु राजानं माययैव निकृन्ततु}


\twolineshloka
{यद्येष बलमास्थाय न्यायेन प्रहरिष्यति}
{विषमस्थस्ततो राजा भविष्यति युधिष्ठिरः}


\twolineshloka
{पुनरेव तु वक्ष्यामि पाण्डवेय निबोध मे}
{धर्मराजापराधेन भयं नः पुनरागतम्}


\threelineshloka
{कृत्वा हि सुमहत्कर्म हत्वा भीष्ममुखान्कुरून्}
{जयः प्राप्तो यशः प्राग्र्यं वैरं च प्रतियातितम्}
{तदेवं विजयः प्राप्तः पुनः संशयितः कृतः}


\twolineshloka
{अबुद्धिरेषां महती धर्मराजस्य पाण्डव}
{यदेकविजये वीर पणितं कृतमीदृशम्}


% Check verse!
सुयोधनः कृती वीर एकायनगतस्तथा
\twolineshloka
{अपि चोशनसा गीतः श्रूयतेऽयं पुरातनः}
{श्लोकस्तत्त्वार्थसहितस्तन्मे निगदतः शृणु}


\twolineshloka
{पुनरावर्तमानानां भग्नानां जीवितैषिणाम्}
{भेतव्यमरिशेषाणामेकायनगता हि ते}


\twolineshloka
{[साहसोत्पतितानां च निराशानां च जीविते}
{न शक्यमग्रतः स्थातुं शक्रेणापि धनञ्जय]}


\threelineshloka
{सुयोधनमिमं भग्नं हतसैन्यं हदं गतम्}
{पराजितं वनप्रेप्सुं निराशं राज्यलम्भने}
{कोऽन्विष्य संयुगे प्राज्ञः पुनर्द्वन्द्वे समाह्वयेत्}


\threelineshloka
{अपि नो निर्जितं राज्यं न हरेत सुयोधनः}
{यस्त्रयोदशवर्षाणि गदया कृतनिश्रमः}
{चरत्यूर्ध्वं च तिर्यक्व भीमसेनिघांसया}


\threelineshloka
{एनं चेन्न महाबाहुरन्यायेन हनिष्यति}
{एष वः कौरवो राजा धार्तराष्ट्रो भविष्यति ॥सञ्जय उवाच}
{}


\twolineshloka
{धनञ्जयस्तु श्रुत्वैतत्केशवस्य महात्मनः}
{प्रेक्षतो भीमसेनस्य सव्यमूरुमताडयत्}


\twolineshloka
{गृह्य संज्ञां ततो भीमो गदया व्यचरद्रणे}
{मण्डलानि विचित्राणि यमकानीतराणि च}


\twolineshloka
{दक्षिणं मण्डलं सव्यं गोमूत्रिकमथापि च}
{व्यचरत्पाण्डवो राजन्नरिं सम्मोहयन्निव}


\twolineshloka
{तथैव तव पुत्रोऽपि गदामार्गविशारदः}
{व्यचरल्लघु चित्रं च भीमसेनजिघांसया}


\twolineshloka
{आधुन्वन्तौ गदे घोरे चन्दनागरुरूषिते}
{वैरस्यान्तं परीप्सन्तौ रणे क्रुद्धाविवान्तकौ}


\twolineshloka
{अन्योन्यं तौ जिघांसन्तौ प्रवीरौ पुरुषर्षभौ}
{युयुधाते गरुत्मन्तौ यथा नागामिषैषिणौ}


\twolineshloka
{मण्डलानि बिचित्राणि चरतोर्नृपभीमयोः}
{गदासम्पातजास्तत्र प्रजज्ञुः पावकार्चिषः}


\twolineshloka
{समं प्रहरतोस्तत्र शूरयोर्बलिनोर्मृधे}
{क्षुब्धयोर्वायुना राजन्द्वयोरिव समुद्रयोः}


\twolineshloka
{तयोः प्रहरतोस्तुल्यं मत्तकुञ्जरयोरिव}
{गदानिर्घातसं हादः प्रहाराणामजायत}


\twolineshloka
{तस्मिंस्तदा सम्प्रहारे दारुणे सङ्कुले भृशम्}
{उभावपि परिश्रान्तौ युध्यमानावरिन्दमौ}


\twolineshloka
{तौ मुहूर्ते समाश्वस्य पुनरेव परन्तप}
{अभ्यहारयतां क्रुद्धौ प्रगृह्य महती गदे}


\twolineshloka
{तयोः समभवद्युद्धं घोररूपमसंवृतम्}
{गदानिपातै राजेन्द्र तक्षतोर्वै परस्परम्}


\twolineshloka
{समरे प्रद्रुतौ तौ तु वृषभाक्षौ तरस्विनौ}
{अन्योन्यं जघ्नतुर्वीरौ पङ्कस्थौ महिषाविव}


\twolineshloka
{जर्झरीकृतसर्वाङ्गौ रुधिरेणाभिसंप्लुतौ}
{ददृशाते हिमवति पुष्पिताविव किंशुकौ}


\twolineshloka
{दुर्योधनस्तु पार्थेन विवरे सम्प्रदर्शिते}
{ईषदुत्स्मयमानस्तु सहसा प्रससार ह}


\twolineshloka
{तमभ्याशगतं प्राज्ञः क्षणे प्रेक्ष्य वृकोदरः}
{अवाक्षिपद्गदां तस्मिन्वेगेन महता बती}


\twolineshloka
{अवक्षेपं तु तं दृष्ट्वा पुत्रस्तव विशाम्पते}
{अपासर्पत्ततः स्थानात्सा मोघा न्यपतद्भुवि}


\twolineshloka
{मोक्षयित्वा प्रहारं तं सुतस्तव सुसम्भ्रमात्}
{भीमसेनं च गदया प्राहरत्कुरुसत्तम}


\twolineshloka
{तस्य विस्यन्दमानेन रुधिरेणामितौजसः}
{प्रहारगुरुपाताच मूर्छेव समजायत}


\twolineshloka
{तन्नावुध्यत पुत्रस्ते पीडितं पाण्डवं रणे}
{धारयामास भीमोऽपि शरीरमतिपीडितम्}


\twolineshloka
{अमन्यत स्थितं ह्येनं प्रहरिष्यन्तमाहवे}
{अतो न प्राहरत्तस्मै पुनरेव तवात्मजः}


\twolineshloka
{ततो मुहूर्तनाश्वस्य दुर्योधनमुपस्थितम्}
{देगेनाभ्ययतद्राजन्गदामादाय पाण्डवः}


\twolineshloka
{तमापतन्तं सम्प्रेक्ष्य संरब्धगमितौजसम्}
{मोधमस्य प्रहारं तं चिकीर्षुर्भरतर्षभ}


\twolineshloka
{अवस्याने मतिं कृत्वा पुत्रन्तव महामनाः}
{इयेषोत्पतितुं राजञ्छलयिष्यन्वृकोदरम्}


\twolineshloka
{अबुध्यद्भीमसेनस्तु राज्ञस्तस्य चिकीर्षितम्}
{अचास्य सममिद्रुत्य समुत्पत्य च सिंहवत्}


\twolineshloka
{सत्या वञ्चवतो राजन्पुनरेवोत्पतिष्यतः}
{ऊरुभ्यां प्राहिणोद्राजन्गदां वेगेन पाण्डवः}


\twolineshloka
{सा वज्रनिष्पेषसमा प्रहिता भीमकर्मणा}
{ऊरू दुर्योधनस्याथ बभञ्ज प्रियदर्शनौ}


\twolineshloka
{स पपात नरव्याघ्रो वसुधामनुनादयन्}
{भग्नोरुर्भीमसेनेन पुत्रस्तव महीपते}


\twolineshloka
{ववुर्वाताः सनिर्घाताः पांसुवर्षं पपात च}
{चचाल पृथिवी चापि सवृक्षक्षुपपर्वता}


\threelineshloka
{तस्मिन्निपतिते वीरे पत्यौ सर्वमहीक्षिताम्}
{महास्वना पुनर्दीप्ता सनिर्घाता भयङ्करी}
{पपात चोल्का महती पतिते पृथिवीपतौ}


\twolineshloka
{तथा शोणितवर्षं च पांसुवर्षं च भारत}
{ववर्ष मघवांस्तत्र तव पुत्रे निपातिते}


\twolineshloka
{यक्षाणां राक्षसानां च पिशाचानां तथैव च}
{अन्तरिक्षे महान्नादस्तत्र भारत शुश्रुवे}


\twolineshloka
{तेन शब्देन घोरेण मृगाणामथ पक्षिणाम्}
{जज्ञे घोरतरः शब्दो बहूनां सर्वतोदिशम्}


\twolineshloka
{ये तत्र वाजिनः शेषा गजाश्च मनुजैः सह}
{मुमुचुस्ते महानादं तव पुत्रे निपातिते}


\twolineshloka
{बेरीशङ्खमृदङ्गानामभवच्च स्वनो महान्}
{अन्तर्भूमिगतश्चैव तव पुत्रे निपातिते}


\twolineshloka
{[बहुपादैर्बहुभुजैः कबन्धैर्घोरदर्शनैः}
{नृत्यद्भिर्भयदैर्व्याप्ता दिशस्तत्राभवन्नृप}


\twolineshloka
{ध्वजवन्तोऽस्त्रवन्तश्च शस्त्रवन्तस्तथैव च}
{प्राकम्पन्त ततो राजंस्तव पुत्रे निपातिते]}


\twolineshloka
{हदाः कूपाश्च रुधिरमुद्वेमुर्नृपसत्तम}
{नद्यश्च सुमहावेगाः प्रतिस्रोतोवहाऽभवन्}


\twolineshloka
{पुल्लिङ्गा इव नार्यस्तु स्त्रीलिङ्गाः पुरुषाऽभवन्}
{दुर्योधने तदा राजन्पतिते तनये तव}


\twolineshloka
{दृष्ट्वा तानद्भुतोत्पातान्पाञ्चालाः पाण्डवैः सह}
{आविग्नमनसः सर्वे बभूवुर्भरतर्षभ}


\twolineshloka
{ययुर्देवा यथाकामं गन्धर्वाप्सरसस्तथा}
{कथयन्तोऽद्भुतं युद्धं सुतयोस्तव भारत}


\twolineshloka
{तथैव सिद्धा राजेन्द्र तथा वातिकचारणाः}
{नरसिंहौ प्रशंसन्तौ विप्रजग्मुर्यथागतम्}


\chapter{अध्यायः ६१}
\twolineshloka
{सञ्जय उवाच}
{}


\twolineshloka
{तं पातितं ततो दृष्ट्वा महासालमिवोद्गतम्}
{प्रहृष्टमनसः सर्वे बभूवुस्तत्र पाण्डवाः}


\twolineshloka
{उन्मत्तमिव मातङ्गं सिंहेन विनिपातितम्}
{ददृशुर्हृष्टरोमाणः सर्वे ते चापि सोमकाः}


\twolineshloka
{ततो दुर्योधनं हत्वा भीमसेनः प्रतापवान्}
{पातितं कौरवेन्द्रं तमुपगम्येदमब्रवीत्}


\threelineshloka
{गौर्गौरिति पुरा मन्द द्रौपदीमेकवाससम्}
{यत्सभायां हसन्नस्मांस्तदा वदसि दुर्मते}
{तस्यावहासस्य फलमद्य त्वं समवाप्नुहि}


\twolineshloka
{एवमुक्त्वा स वामेन पदा मौलिमुपास्पृशत्}
{शिरश्च राजसिंहस्य पादेन समलोडयत्}


\twolineshloka
{तथैव क्रोधसंरक्तो भीमः परबलार्दनः}
{पुनरेवाब्रवीद्वाक्यं यत्तच्छृणु नराधिप}


\twolineshloka
{येऽस्मान्पुरा प्रनृत्यन्ति मूढा गौरिति गौरिति}
{तान्वयं प्रतिनृत्यामः पुनर्गौरिति गौरिति}


\twolineshloka
{नास्माकं निकृतिर्वह्निर्नाक्षद्यूतं न वञ्चना}
{स्वबाहुबलमाश्रित्य प्रबाधामो वयं रिपून्}


\twolineshloka
{सोऽवाप्य वैरस्य चिरस्य पारंवृकोदरः प्राह शनैः प्रहस्य}
{युधिष्ठिरं केशवसृञ्जयांश्चधनञ्जयं माद्रवतीसुतौ च}


\twolineshloka
{रजस्वला द्रौपदीमानयन्तेये चाप्यकुर्वन्त सदस्यवस्त्राम्}
{9-60-10cतान्पश्यध्वंपाण्डवैर्धार्तराष्ट्रान्रणे हतांस्तपसा याज्ञसेन्याः}


\twolineshloka
{ये नः पुरा षण्ढतिलानवोचन्क्रूरा राज्ञो धृतराष्ट्रस्य पुत्राः}
{ते नो हताः सगणाः सानुबन्धाःकामं स्वर्गं नरकं वा व्रजामः}


\twolineshloka
{पुनश्च राज्ञः पतितस्य भूमौस तां गदां स्कन्धगतां प्रगृह्य}
{वामेन पादेन शिरः प्रमृद्यदुर्योधनं नैकृतिकेत्यवोचत्}


\twolineshloka
{हृष्टेन राजन्कुरुपार्थिवस्यक्षुद्रात्मना भीमसेनेन पादम्}
{दृष्ट्वा कृतं मूर्धनि नाभ्यनन्दन्धर्मात्मानः सोमकानां प्रबर्हाः}


\twolineshloka
{तव पुत्रं तथा हत्वा कत्थमानं वृकोदरम्}
{नृत्यमानं च बहुशो धर्मराजोऽब्रवीदिदम्}


\twolineshloka
{गतोऽसि वैरस्यानृण्यं प्रतिज्ञा पूरिता त्वया}
{शुभेनाथाशुभेनैव कर्मणा विरमाधुना}


\twolineshloka
{मा शिरोऽस्य पदाऽमर्दीर्मा धर्मस्तेऽतिगो भवेत्}
{राजा ज्ञातिर्हतश्चायं नैतन्न्याय्यं तवानघ}


\twolineshloka
{एकादशचमूनाथं कुरूणामधिपं तथा}
{मास्प्राक्षीर्भीम पादेन राजानं ज्ञातिमेव च}


\twolineshloka
{हतबन्धुर्हतामात्यो भ्रष्टसैन्यो हतो मृधे}
{सर्वाकारेण शोच्योऽयं नावहास्योऽयमीश्वरः}


\twolineshloka
{विध्वस्तोऽयं हतामात्यो हतबन्धुर्हतात्मजः}
{उत्सन्नपिण्डो भ्राता च नैतन्न्याय्यं कृतं त्वया}


\twolineshloka
{धार्मिको भीमसेनोऽसावित्याहुस्त्वां पुरा जनाः}
{स कस्माद्भीमसेन त्वं राजानमधितिष्ठसि}


\twolineshloka
{इत्युक्त्वा भीमसेनं तु साश्रुकण्ठो युधिष्ठिरः}
{उपसृत्याब्रवीद्दीनो दुर्योधनमरिन्दमम्}


\twolineshloka
{तात मन्युर्न ते कार्यो नात्मा शोच्यस्त्वया तथा}
{नूनं पूर्वकृतं कर्म सुघोरमनुभूयते}


\twolineshloka
{धात्रोपदिष्टं विषमं नूनं फलमसंस्कृतम्}
{यद्वयं त्वां जिघांसामस्त्वं चास्मान्कुरुसत्तम}


\twolineshloka
{आत्मनो ह्यपराधेन महद्व्यसनमीहशम्}
{प्राप्तवानसि यल्लोभान्मदाद्बाल्याच्च भारत}


\twolineshloka
{घातयित्वा वयस्यांश्च भ्रातॄनथ पितॄंस्तथा}
{पुत्रान्पौत्रांस्तथा चान्यांस्ततोसि निधनं गतः}


\twolineshloka
{तवापराधादस्माभिर्भ्रातरस्ते निपातिताः}
{निहता ज्ञातयश्चापि दिष्टं मन्ये दुरत्ययम्}


\twolineshloka
{[आत्मा न शोचनीयस्ते श्लाघ्यो मृत्युस्तवानघ}
{वयमेवाधुना शोच्याः सर्वावस्थासु कौरव}


\twolineshloka
{कृपणं वर्तयिष्यामस्तैर्हीना बन्धुभिः प्रियैः}
{भ्रातॄणां चैव पुत्राणां तथा वै शोकविह्वलाः}


\threelineshloka
{कथं द्रक्ष्यामि विधवा वधूः शोकपरिप्लुताः}
{त्वमेकः सुस्थितो राजन्स्वर्गे ते निलयो ध्रुवः}
{वयं नरकसंज्ञं वै दुःखं प्राप्स्याम दारुणम् ॥]}


\threelineshloka
{स्नुषाश्च प्रस्नुषाश्चैव धृतराष्ट्रस्य विह्वलाः}
{गर्हयिष्यन्ति नो नूनं विधवाः शोककर्शिताः ॥सञ्जय उवाच}
{}


\twolineshloka
{एवमुक्त्वा सुदुःखार्तो निशश्वास स पार्थिवः}
{विललाप चिरं चापि धर्मपुत्रो युधिष्ठिरः}


\chapter{अध्यायः ६२}
\twolineshloka
{धृतराष्ट्रा उवाच}
{}


\twolineshloka
{अधर्मेण हतं दृष्ट्वा राजानं माधवोत्तमः}
{किमब्रवीत्तदा सूत बलदेवो महाबलः}


\threelineshloka
{गदायुद्धविशेषज्ञो गदायुद्धविशारदः}
{कृतवान्रौहिणेयो यत्तन्ममाचक्ष्व सञ्जय ॥सञ्जय उवाच}
{}


\twolineshloka
{शिरस्यभिहतं दृष्ट्वा भीमसेनेन ते सुतम्}
{रामः प्रहरतां श्रेष्ठश्चुक्रोध बलवद्बली}


\twolineshloka
{ततो मध्ये नरेन्द्राणामूर्ध्वबाहुर्हलायुधः}
{कुर्वन्नार्तस्वरं घोरं धिग्धिग्भीमेत्युवाच ह}


\twolineshloka
{अहो धिग्यदधो नाभेः प्रहृतं धर्मविग्रहे}
{नैतद्दृष्टं गदायुद्धे कृतवान्यद्वृकोदरः}


\twolineshloka
{अधो नाभ्या न हन्तव्यमिति शास्त्रस्य निश्चयः}
{अयं तु शास्त्रमुत्सृज्य स्वच्छन्दात्सम्प्रवर्तते}


\threelineshloka
{तस्य तत्तद्ब्रुवाणस्य रोषः समभवन्महान्}
{[ततो राजानमालोक्य रोषसंरक्तलोचनः}
{बलदेवो महाराज ततो वचनमब्रवीत्}


\twolineshloka
{न चैष पतितः कृष्ण केवलं मत्समोऽसमः}
{आश्रितस्य तु दौर्बल्यादाश्रयः परिभर्त्स्यते ॥]}


% Check verse!
ततो लाङ्गलमुद्यम्य भीममभ्यद्रवद्बली
\twolineshloka
{तस्योर्ध्वबाहोः सदृशं रूपमासीन्महात्मनः}
{बहुधातुविचित्रस्य श्वेतस्येव महागिरेः}


\twolineshloka
{`भ्रातृभिः सहितो वीरैः सार्जुनैरस्त्रकोविदैः}
{न विव्यथे महाराज दृष्ट्वा हलधरं बली'}


\twolineshloka
{अथ रामं निजग्राह केशवो विनयान्वितः}
{बाहुभ्यां पीनवृत्ताभ्यां प्रयत्नाद्बलवद्बली}


\threelineshloka
{सितासितौ यदुवरौ शुशुभातेऽधिकं तदा}
{`सङ्गताविव राजेन्द्र कैलासाञ्जनपर्वतौ}
{नभोगतौ यथा राजंश्चन्द्रसूर्यौ दिनक्षये}


% Check verse!
उवाच चैनं संरब्धं शमयन्निव केशवः
\twolineshloka
{आत्मवृद्धिर्मित्रवृद्धिर्मित्रमित्रोदयस्तथा}
{विपरीतं द्विषत्स्वेतत्षड्विधा वृद्धिरात्मनः}


\twolineshloka
{आत्मन्यपि च मित्रे च विपरीतं यदा भवेत्}
{तदा विद्यान्मनोग्लानिमाशु शान्तिकरो भवेत्}


\twolineshloka
{अस्माकं सहजं मित्रं पाण्डवाः शुद्धपौरुषाः}
{स्वकाः पितृष्वसुः पुत्रास्ते परैर्निकृता भृशम्}


% Check verse!
प्रतिज्ञापालनं धर्मः क्षत्रियस्येति वेत्थ तत्
\twolineshloka
{सुयोधनस्य गदया भक्ताऽस्म्यूरू महाहवे}
{इति पूर्वं प्रतिज्ञातं भीमेन हि सभातले}


\threelineshloka
{मैत्रेयेणाभिशप्तश्च पूर्वमेव महार्षिणा}
{ऊरू ते भेत्स्यते भीमो गदयेति परन्तप}
{अतो दोषं न पश्यामि मा क्रुध्यस्व प्रलम्बहन्}


\twolineshloka
{यौनः स्वैः सुखहार्दैश्च सम्बन्धः सह पाण्डवैः}
{तेषां वृद्ध्या हि वृद्धिर्नो मा क्रुधः पुरुषर्षभ}


% Check verse!
वासुदेववचः श्रुत्वा सीरभृत्प्राह धर्मवित्
\twolineshloka
{धर्मश्च धारितः सद्भिः स च द्वाभ्यां नियच्छति}
{अर्थश्चाप्यतिलुब्धस्य कामश्चातिप्रसङ्गिणः}


\twolineshloka
{धर्मार्थौ धर्मकामौ च कामार्थौ चाप्यपीडयन्}
{धर्मार्थकामान्योऽभ्येति सोत्यन्तं सुखमश्नुते}


\threelineshloka
{तदिदं व्याकुलं सर्वं कृतं धर्मस्य पीडनात्}
{भीमसेनेन गोविन्द कामं त्वं तु यथाऽऽत्थ माम् ॥कृष्ण उवाच}
{}


\twolineshloka
{अरोषणो हि धर्मात्मा सततं धर्मवत्सलः}
{भवान्प्रख्यायते लोके तस्मात्संशाम्य मा क्रुधः}


\threelineshloka
{प्राप्तं कलियुगं विद्वि प्रतिज्ञां पाण्डवस्य च}
{आनृण्यं यातु वैरस्य प्रतिज्ञायाश्च पाण्डवः}
{`ततः पुरुषशार्दूलो हत्वा नैकृतिकं रणे}


\twolineshloka
{निकृत्या निकृतिप्रज्ञं यो हन्याद्वैरिणं रणे}
{अधर्मो विद्यते नात्र यद्भीमो हतवान्रिपुम्}


\twolineshloka
{युध्यमानं रणे वीरं कुरुवृष्णियशस्करम्}
{अनेन कर्णः सन्दिष्टः पृष्ठतो धनुरच्छिनत्}


\twolineshloka
{ततः सञ्छिन्नधन्वानं विरथं पौरुषे स्थितम्}
{व्यायुधीकृत्य बहवः सौभद्रमपलायिनम्}


\twolineshloka
{जन्मप्रभृति लुब्धश्च पापात्मा चैष दुर्मतिः}
{निहतो भीमसेनेन दुर्बुद्धिः कुलपांसनः}


\twolineshloka
{प्रतिज्ञां भीमसेनेन त्रयोदशसमार्जिताम्}
{किमर्थं नाभिजानाति युध्यमानो हि शत्रुभिः}


\threelineshloka
{ऊर्ध्वमाक्रम्य वेगेन जिघांसन्तं वृकोदरः}
{बिभेद गदया चोरू न स्थाने न च मण्डले' ॥सञ्जय उवाच}
{}


\twolineshloka
{धर्मच्छलमिमं श्रुत्वा केशवात्स विशाम्पते}
{नैव प्रीतमाना रामो वचनं प्राह संसदि}


\twolineshloka
{हत्वाऽधर्मेण राजानं धर्मात्मानं सुयोधनम्}
{जिह्मयोधीति लोकेऽस्मिन्ख्यातिं यास्यति पाण्डवः}


\twolineshloka
{दुर्योधनोऽपि धर्मात्मा गतिं यास्यति शाश्वतीम्}
{ऋजुयोधी हतो राजा धर्मराष्ट्रो नराधिपः}


\threelineshloka
{युद्धदीक्षां प्रविश्याजौ रणयज्ञं वितत्य च}
{हुत्वाऽऽत्मानममित्राग्नौ प्राप चावभृथं यशः}
{`स्वर्गं गन्ता महाराजः ससुहृज्ज्ञातिबान्धवः'}


\twolineshloka
{इत्युक्त्वा रथमास्थाय रौहिणेयः प्रतापवान्}
{श्वेताभ्रशिखराकारः प्रययौ द्वारकां प्रति}


\twolineshloka
{पाञ्चालाश्च सवार्ष्णेयाः पाण्डवाश्च विशाम्पते}
{रामे द्वारवतीं याते नातिप्रमनसोऽभवन्}


\threelineshloka
{ततो युधिष्ठिरं दीनं चिन्तापरमधोमुखम्}
{शोकोपहतसङ्कल्पं वासुदेवोऽब्रवीदिदम् ॥वासुदेव उवाच}
{}


\twolineshloka
{धर्मराज किमर्थं त्वमधर्ममनुमन्यसे}
{हतबन्धोर्यदेतस्य पतितस्य विचेतसः}


\threelineshloka
{दुर्योधनस्य भीमेन मृद्यमानं शिरः पदा}
{उपप्रेक्षसि कस्मात्त्वं धर्मज्ञः सन्नराधिप ॥युधिष्ठिर उवाच}
{}


\twolineshloka
{न ममैतत्प्रियं कृष्ण यद्राजानं वृकोदरः}
{पदा मूर्ध्न्यस्पृशत्क्रोधान्न च हृष्ये कुलक्षये}


\twolineshloka
{निकृत्या निकृता नित्यं धृतराष्ट्रसुतैर्वयम्}
{बहूनि परुषाण्युक्त्वा वनं प्रस्थापिता वयम्}


\twolineshloka
{भीमसेनस्य तद्दुःखमतीव हृदि वर्तते}
{इति सञ्चिन्त्य वार्ष्णेय मयैतत्समुपेक्षितम्}


\threelineshloka
{तस्माद्धत्वाऽकृतप्रज्ञं लुब्धं कामवशानुगम्}
{लभतां पाण़्डवः कामं धर्मोऽधर्मोऽथवा कृतः ॥सञ्जय उवाच}
{}


\threelineshloka
{इत्युक्तवति कौन्तेये धर्मराजे युधिष्ठिरे}
{वासुदेवो महाबाहुर्युधिष्ठिरमभाषत}
{काममस्त्वेतदिति वै कृच्छ्राद्यदुकुलोद्वहः}


\twolineshloka
{इत्युक्त्वा वासुदेवोऽपि वायुपुत्रप्रियेप्सया}
{अन्वमोदत तत्सर्वं यद्भीमेन कृतं युधि}


\twolineshloka
{`अर्जुनोऽपि महाबाहुरप्रीतेनान्तरात्मना}
{नोवाच किञ्चिद्वचनं भ्रातरं साध्वसाधु वा'}


\twolineshloka
{भीमसेनोऽपि हत्वाऽऽजौ तव पुत्रममर्षणः}
{अभिवाद्याग्रतः स्थित्वा सम्प्रहृष्टः कृताञ्जलिः}


\twolineshloka
{प्रोवाच सुमहातेजा धर्मराजं युधिष्ठिरम्}
{हर्षादुत्फुल्लनयनो जितकाशी विशाम्पते}


\twolineshloka
{तवाद्य पृथिवी सर्वा क्षेमा निहतकण्टका}
{तां प्रशाधि महाराज स्वधर्ममनुपालय}


\twolineshloka
{यस्तु कर्ताऽस्य वैरस्य निकृत्या निकृतिप्रियः}
{सोऽयं विनिहतः शेते पृथिव्यं पृथिवीपते}


\twolineshloka
{दुःशासनप्रभृतयः सर्वे ते चोग्रवादिनः}
{राधेयः शकुनिश्चैव हताश्च तव शत्रवः}


\threelineshloka
{सेयं रत्नसमाकीर्णा मही सवनपर्वता}
{उपावृत्ता महाराज त्वामद्य निहतद्विषम् ॥युधिष्ठिर उवाच}
{}


\twolineshloka
{गतो वैरस्य निधनं हतो राजा सुयोधनः}
{कृष्णस्य मतमास्थाय विजितेयं वसुन्धरा}


\twolineshloka
{दिष्ट्या गतस्त्वमानृण्यं मातुः कोपस्य चोभयोः}
{दिष्ट्या जयसि दुर्धर्ष दिष्ट्या शत्रुर्निपातितः}


\chapter{अध्यायः ६३}
\twolineshloka
{धृतराष्ट्र उवाच}
{}


\threelineshloka
{हतं दुर्योधनं दृष्ट्वा भीमसेनेन संयुगे}
{पाण्डवाः सृञ्जयाश्चैव किमकुर्वत सञ्जय ॥स़ञ्जय उवाच}
{}


\twolineshloka
{हतं दुर्योधनं दृष्ट्वा भीमसेनेन संयुगे}
{सिंहेनेव महाराज मत्तं वनगजं यथा}


\threelineshloka
{प्रहृष्टमनसस्तत्र कृष्णेन सह पाण्डवाः}
{पाञ्चालाः सृञ्जयाश्चैव निहते कुरुनन्दने}
{आविध्यन्नुत्तरीयाणि सिंहनादांश्च नेदिरे}


% Check verse!
नैतान्हर्षसमाविष्टानियं सेहे वसुन्धरा
\twolineshloka
{धनूंष्येके व्याक्षिपन्त ज्याश्चाप्यन्ये तथाऽक्षिपन्}
{दध्मुरन्ये महाशङ्खानन्ये जघ्नुश्च दुन्दुभीन्}


\twolineshloka
{चिकीडुश्च तथैवान्ये जहसुश्च तवाहिताः}
{अब्रुवंश्चासकृद्वीरा भीमसेनमिदं वचः}


\twolineshloka
{दुष्करं भवता कर्म रणेऽद्य सुमहत्कृतम्}
{कौरवेयं रणे हत्वा गदयाऽतिकृतश्रमम्}


\twolineshloka
{इन्द्रेणेव हि वृत्रस्य वधं परमसंयुगे}
{त्वया कृतममन्यन्त शत्रोर्वधमिमं जनाः}


\twolineshloka
{चरन्तं विविधान्मार्गान्मण्डलानि च सर्वशः}
{दुर्योधनमिमं शूरं कोऽन्यो हन्याद्वृकोदरात्}


\twolineshloka
{वैरस्य च गतः पारं त्वमिहान्यैः सुदुर्गमम्}
{अशक्यमेतदन्येन सम्पादयितुमीदृशम्}


\twolineshloka
{कुञ्जरेणेव मत्तेन वीरसङ्ग्राममूर्धनि}
{दुर्योधपशिरो दिष्ट्या पादेन मृदितं त्वया}


\twolineshloka
{सिंहेन महिषस्येव कृत्वा सङ्गरमुत्तमम्}
{दुःशासनस्य रुधिरं दिष्ट्या पीतं त्वयाऽनघ}


\twolineshloka
{ये विप्रकुर्वन्राजानं धर्मात्मानं युधिष्ठिरम्}
{मूर्ध्नि तेषां कृतः पादो दिष्ट्या ते स्वेन कर्मणा}


\twolineshloka
{अमित्राणामधिष्ठानाद्वधाद्दुर्योधनस्य च}
{भीम दिष्ट्या पृथिव्यां ते प्रथितं सुमहद्यशः}


\twolineshloka
{एवं नूनं हते वृत्रे शक्रं नन्दन्ति बन्दिनः}
{तथा त्वां निहतामित्रं वयं नन्दाम भारत}


\threelineshloka
{दुर्योधनवधे यानि रोमाणि हृषितानि नः}
{अद्यापि न विकृष्यन्ते तानि तद्विद्धि भारत}
{इत्यब्रुन्भीमसेनं वातिकास्तत्र सङ्गताः}


\twolineshloka
{तान्हृष्टान्पुरुषव्याघ्रान्पाञ्चालान्पाण्डवैः सह}
{ब्रुवतोऽसदृशं तत्र प्रोवाच मधुसूदनः}


\twolineshloka
{न न्याय्यं निहतं शत्रुं भूयो हन्तुं नराधिपाः}
{असकृद्वाग्भिरुग्राभिर्निहतो ह्येष मन्दधीः}


\twolineshloka
{तदैवैष हतः पापो यदैव निरपत्रपः}
{लुब्धः पापसहायश्च सुहृदां शासनातिगः}


\twolineshloka
{बहुशो विदुरद्रोणकृपगाङ्गेयसृञ्जयैः}
{पाण्डुभ्यः प्रार्थ्यमानोऽपि पित्र्यमंशं न दत्तवान्}


\twolineshloka
{नैव तेभ्योऽद्य मित्रं वा शत्रुर्वा पुरुषाधमः}
{किमनेन वितुन्नेन वाग्भिः काष्ठसधर्मणा}


\twolineshloka
{रथेष्वारोहत क्षिप्रं गच्छामो वसुधाधिपाः}
{दिष्ट्या हतोऽयं पापात्मा सामात्यज्ञातिबान्धवः}


\twolineshloka
{इति श्रुत्वा त्वधिक्षेपं कृष्णाद्दुर्योधनो नृपः}
{अमर्षवशमापन्न उदतिष्ठद्विशाम्पते}


\twolineshloka
{स्फिग्देशेनोपविष्टः स दोर्भ्यां विष्टभ्य मेदिनीम्}
{दृष्टिं भ्रूसङ्कटां कृत्वा वासुदेवे न्यपातयत्}


\twolineshloka
{अर्धोन्नतशरीरस्य रूपमासीन्नृपस्य तु}
{क्रुद्धस्याशीविषस्येव च्छिन्नपुच्छस्य भोगिनः}


\twolineshloka
{प्राणान्तकरिणीं घोरां वेदनामप्यचिन्तयन्}
{दुर्योधनो वासुदेवं वाग्भिरुग्राभिरार्दयत्}


\threelineshloka
{कंसदासस्य दायाद न ते लज्जाऽस्त्यनेन वै}
{अधर्मेण गदायुद्धे यदर्हं विनिपातितः}
{ऊरूभिन्धीति भीमस्य स्मृतिं मिथ्याप्रयच्छता}


\threelineshloka
{किं न विज्ञातमेतन्मे यदर्जुनमवोचथाः}
{घातयित्वा महीपालानृजुयुद्धान्सहस्रशः}
{जिह्मैरुपायैर्बहुभिर्न ते लज्जा न ते घृणा}


\twolineshloka
{अहन्यहनि शूराणां कुर्वाणः कदनं महत्}
{शिखण्डिनं पुरस्कृत्य घातितस्ते पितामहः}


\twolineshloka
{अश्वत्थाम्नः सनामानं हत्वा नागं सुदुर्मते}
{आचार्यो न्यासितः शस्त्रं किं तन्न विदितं मया}


\twolineshloka
{स चानेन नृशंसेन धृष्टद्युम्नेन वीर्यवान्}
{पात्यमानस्त्वया दृष्टो न चैनं त्वमवारयः}


\twolineshloka
{वधार्थं पाण्डुपुत्रस्य याचितां शक्तिमेव च}
{घटोत्कचे व्यंसयतः कस्त्वत्तः पापकृत्तमः}


\twolineshloka
{छिन्नहस्तः प्रायगतस्तथा भूरिश्रवा बली}
{त्वयाऽभिसृष्टेन हतः शैनेयेन दुरात्मना}


\twolineshloka
{कुर्वाणश्चोत्तमं कर्म कर्णः पार्थजिगीषया}
{व्यंसनेनाश्वसेनस्य पन्नगेन्द्रस्य वै पुनः}


\twolineshloka
{पुनश्च पतिते चक्रे व्यसनार्तः पराजितः}
{पापितः समरे कर्णश्चक्रव्याग्रोऽग्रणीर्नृणाम्}


\twolineshloka
{यदि मां चापि कर्णं च भीष्मद्रोणौ च संयुगे}
{ऋजुना प्रतियुध्येथा न ते स्याद्विजयो ध्रुवम्}


\twolineshloka
{त्वया पुनरनार्येण जिह्ममार्गेण पार्थिवाः}
{स्वधर्ममनुतिष्ठन्तो वयं चान्ये च घातिताः}


\threelineshloka
{`त्वया मायाविना कृष्ण मायां कर्मप्रमोषिणीम्}
{कृत्वा हतः सिन्धुपतिः किं तन्न विदितं मम' ॥वासुदेव उवाच}
{}


\twolineshloka
{हतस्त्वमसि गान्धारे सभ्रातृसुतबान्धवः}
{सगणः ससुहृच्चैव पापं मार्गमनुष्ठितः}


\twolineshloka
{तवैव दुष्कृतैर्वीरौ भीष्मद्रोणौ निपातितौ}
{कर्णश्च निहतः सङ्ख्ये तव शीलानुवर्तकः}


\twolineshloka
{याच्यमानं मया मूढ पित्र्यमंशं न दित्ससि}
{पाण्डवेभ्यः स्वराज्यं च लोभाच्छकुनिनिश्चयात्}


\twolineshloka
{विषं ते भीमसेनाय दत्तं सर्वे च पाण्डवाः}
{प्रदीपिता जतुगृहे मात्रा सह सुदुर्मते}


\twolineshloka
{सभायां याज्ञसेनी च कृष्टा द्यूते रजस्वला}
{तदैव तावद्दुष्टात्मन्वध्यस्त्वं निरपत्रप}


\twolineshloka
{अनक्षज्ञं च धर्मज्ञं सौबलेनाक्षवेदिना}
{निकृत्या यत्पराजैषीस्तस्मादसि हतो रणे}


\twolineshloka
{जयद्रथेन पापेन यत्कृष्णा क्लेशिता वने}
{यातेषु मृगयां चैव तृणबिन्दोरथाश्रमम्}


\twolineshloka
{अभिमन्युश्च यद्बाल एको बहुभिराहवे}
{त्वद्दोषैर्निहतः पाप तस्मादसि हतो रणे}


\twolineshloka
{`कुर्वाणं कर्म समरे पाण्डवानर्थकाङ्क्षिणम्}
{यच्छिखण्ड्यवधीद्भीष्मं मित्रार्थे न व्यतिक्रमः}


\twolineshloka
{स्वधर्मं पृष्ठतः कृत्वा आचार्यस्त्वत्प्रियेप्सया}
{पार्षतेन हतः सङ्ख्ये वर्तमानोऽसतां पथि}


\twolineshloka
{प्रतिज्ञामात्मनः सत्यां चिकीर्षन्समरे निपुम्}
{हतवान्सात्वतो विद्वान्सौमदत्तिं महारथम्}


\twolineshloka
{अर्जुनः समरे राजन्युध्यमानः कदाचन}
{निन्दितं पुरुषव्याघ्रः करोति न कथञ्चन}


\twolineshloka
{लब्ध्वाऽपि बहुधा छिद्रं वीरवृत्तमनुस्मरन्}
{निजघान रणे कर्णं मैवं वोचः सुदुर्मते}


\twolineshloka
{देवानां मतमाज्ञाय तेषां प्रियहितेप्सया}
{अर्जुनस्य महानागं मया व्यंसितमस्त्रजम्}


\twolineshloka
{त्वं च भीष्मश्च कर्णश्च द्रोणो द्रौणायनिस्तथा}
{विराटनगरे तस्य ह्यानृशंस्येन जीविताः}


\twolineshloka
{स्मर पार्थस्य विक्रान्तं गन्धर्वेषु कृतं तथा}
{अधर्मं नात्र गान्धारे पाण्डवैर्यत्कृतं त्वयि}


\twolineshloka
{स्वबाहुबलमास्थाय स्वधर्मेण परन्तपाः}
{जितवन्तो रणे वीराः पापोसि निधनं गतः'}


\twolineshloka
{[यान्यकार्याणि चास्माकं कृतानिति प्रभाषसे}
{वैगुण्येन तवात्यर्थं सर्वं हि तदनुष्ठितम्}


\twolineshloka
{बृहस्पतेरुशनसो नोपदेशः श्रुतस्त्वया}
{वृद्धा नोपासिताश्चैव हितं वाक्यं न ते श्रुतभ्}


\threelineshloka
{लोभेनातिबलेन त्वं तृष्णया च वशीकृतः}
{कृतवानस्यकार्याणि विपाकस्तस्य भुज्यताम् ॥]दुर्योधन उवाच}
{}


\twolineshloka
{अधीतं विधिवद्दत्तं भूः प्रभुक्ता ससागरा}
{मूर्ध्नि स्थिममित्राणां को नु स्वन्ततरो मया}


\twolineshloka
{यदिष्टं क्षत्रबन्धूनां स्वधर्ममनुपश्यताम्}
{तदिदं निधनं प्राप्तं को नु स्वन्ततरो मया}


\twolineshloka
{देवार्हा मानुषा भोगाः प्राप्ता असुलभा नृपैः}
{ऐश्वर्यं चोत्तमं प्राप्तं को नु स्वन्ततरो मया}


\twolineshloka
{ससुहृत्सानुबन्धस्च स्वर्गं गन्ताऽहमच्युत}
{यूयं गर्हितसङ्कल्पाः शोचन्तो वर्तयिष्यथ}


\threelineshloka
{`न मे विषादो भीमेन पादेन शिर आहतम्}
{काका वा कङ्कगृध्रा वा निधास्यन्ति पदं क्षणात्' ॥सञ्जय उवाच}
{}


\twolineshloka
{अस्य वाक्यस्य निधने कुरुराजस्य धीमतः}
{अपतत्सुमहद्वर्षं पुष्पाणां पुण्यगन्धिनाम्}


% Check verse!
अवादयन्त गन्धर्वा वादित्रं सुमनोहरम्
\twolineshloka
{जगुश्चाप्सरसो राज्ञो यशः सम्बद्धमेव च}
{सिद्धाश्च मुमुचुर्वाचः साधुसाध्विति पार्थिव}


\twolineshloka
{ववौ च सुरभिर्वायुः पुण्यगन्धो मृदुःसुखः}
{व्यराजंश्च दिशः सर्वा नभो वैदूर्यसन्निभम्}


\twolineshloka
{अत्यद्भुतानि ते दृष्ट्वा वासुदेवपुरोगमाः}
{दुर्योधनस्य पूजां तु दृष्ट्वा व्रीडामुपागमन्}


\twolineshloka
{हतांश्चाधर्मतः श्रुत्वा शोकार्ताः शुशुचुर्हि ते}
{भीष्मं द्रोणं तथा कर्णं भूरिश्रवसमेव च}


\twolineshloka
{तांस्तु चिन्तापरान्दृष्ट्वा पाण्डवान्दीनचेतसः}
{प्रोवाचेदं वचः कृष्णो मेघदुन्दुभिनिस्वनः}


\twolineshloka
{नैष शक्योऽतिशीघ्रास्त्रस्ते च सर्वे महारथाः}
{ऋजुयुद्धेन विक्रान्ता हन्तुं युष्माभिराहवे}


\twolineshloka
{उपायान्निहता ह्येते मया तस्मान्नराधिपाः}
{अन्यथा पाण्डवेयानां नाभविष्यज्जयः क्वचित्}


\twolineshloka
{ते हि सर्वे महेष्वासाश्चत्वारोऽतिरथा मुवि}
{अशक्या धर्मतो हन्तुं लोकपालैरपि स्वयम्}


\twolineshloka
{तथैवायं गदापाणिर्धार्तराष्ट्रो गतश्रमः}
{अशक्यो धर्मतो हन्तुं कालेनापि दि दण्डिना}


\twolineshloka
{नैतन्मनसि कर्तव्यं यदयं निहतो नृपः}
{मिथ्याचर्याच्छलोपायैर्बहवः शत्रवो हताः}


\twolineshloka
{पूर्वैरनुगतो मार्गो देवैरसुरघातिभिः}
{सद्भिश्चानुगतः पन्थाः सर्वैरनुगमिष्यते}


\twolineshloka
{`एवं विधात्रा विहितं स्वयमेषां महात्मनाम्}
{दैवं पुरुषकारेण न शक्यमतिवर्तितुम्}


\twolineshloka
{भूतं भव्यं भविष्यच्च निमेषाद्यो हनिष्यति}
{कृतान्तमन्यथा कर्तुं नेच्छेत्सोऽयं धनञ्जय ॥'}


\twolineshloka
{कृतकृत्याः स्म सायाह्ने निवासं रोचयामहे}
{साश्वनागरथाः सर्वे विश्राम्यन्तु नराधिपाः}


\twolineshloka
{वासुदेववचः श्रुत्वा तदानीं पाण्डवैः सह}
{पाञ्चाला भृशसंहृष्टा विनेदुः सिंहसङ्घवत्}


\twolineshloka
{ततः प्राध्मापयच्छङ्खं पाञ्चजन्यं जनार्दनः}
{`देवदत्तं प्रहृष्टात्मा शङ्खप्रवरमर्जुनः}


\twolineshloka
{अनन्तविजयं राजा कुन्तीपुत्रो युधिष्ठिरः}
{पोण्ड्रं दध्मौ महाशङ्खं भीमकर्मा वृकोदरः}


\twolineshloka
{नकुलः सहदेवश्च सुघोषमणिपुष्पकौ}
{धृष्टद्युम्नस्तथा जैत्रं सात्यकिर्नन्दिवर्धनम्}


\twolineshloka
{तेषां नादेन महता शङ्खानां भरतर्षभ}
{आपुपूरे नभः सर्वं पृथिवी च चचाल ह}


\twolineshloka
{ततः शङ्खाश्च भेर्यश्च पणवानकगोमुखाः}
{पाण्डुसैन्येष्ववाद्यन्त स शब्दस्तुमुलोऽभवत्}


\twolineshloka
{अस्तुवन्पाण्डवानन्ये निर्भीश्च स्तुतिमङ्गलैः'}
{हृष्टा दुर्योधनं दृष्ट्वा निहतं पुरुषर्षभाः}


\chapter{अध्यायः ६४}
\twolineshloka
{सञ्जय उवाच}
{}


\twolineshloka
{ततस्ते प्रययुः सर्वे निवासाय महीक्षितः}
{शङ्खान्प्रध्मापयन्तो वै हृष्टाः परिघबाहवः}


\twolineshloka
{पाण्डवान्गच्छतश्चापि शिबिरं नो विशाम्पते}
{महेष्वासोऽन्वगात्पश्चाद्युयुत्सुः सात्यकिस्तथा}


\twolineshloka
{धृष्टद्युम्नः शिखण्डी च द्रौपदेयाश्च सर्वशः}
{सर्वे चान्ये महेष्वासा ययुः स्वशिबिराण्युत}


\twolineshloka
{ततस्ते प्राविशन्पार्था हतत्विट्कं हतेश्वरम्}
{दुर्योधनस्य शिबिरं रङ्गवन्निःसृते जने}


\twolineshloka
{गतोत्सवं पुरमिव हृतनागमिव हदम्}
{स्त्रीवर्षवरभूयिष्ठं वृद्धामात्यैरधिष्ठितम्}


\twolineshloka
{तत्रैतान्पर्युपातिष्ठन्दुर्योधन पुरःसराः}
{कृताञ्जलिपुटा राजन्काषायमलिनाम्बराः}


\twolineshloka
{शिबिरं समनुप्राप्य कुरुराजस्य पाण्डवाः}
{अवतेरुर्महाराज रथेभ्यो रथसत्तमाः}


\twolineshloka
{ततो गाण्डीवधन्वानमभ्यभाषत केशवः}
{स्थितः प्रियहिते नित्यमतीव भरतर्षभ}


\threelineshloka
{अवरोपय गाण्डीवमक्षयौ च महेषुधी}
{अथाहमवरोक्ष्यामि पश्चाद्भरतसत्तम}
{स्वयं चैवावरोह त्वमेतच्छ्रेयस्तवानघ}


% Check verse!
तच्चाकरोत्तथा वीरः पाण्डुपुत्रो धनञ्जयः
\twolineshloka
{अथ पश्चात्ततः कृष्णो रश्मीनुत्सृज्य वाजिनाम्}
{अवारोहत मेधावी रथाद्गाण्डीवधन्वनः}


\twolineshloka
{अथावतीर्णे भूतानामीश्वरे सुमहात्मनि}
{कपिरप्याश्वपाक्रामत्सहदेवैर्ध्वजालयैः}


\twolineshloka
{स दग्धो द्रोणकर्णाभ्यां दिव्यैरस्त्रैर्महारथः}
{अथ दीप्तोऽग्निना ह्याशु प्रजज्वाल महीपते}


\twolineshloka
{सोपासङ्गः सरश्मिश्च साश्वः सयुगबन्धनः}
{भस्मीभूतोऽपतद्भूमौ रथो गाण्डीवधन्वनः}


\twolineshloka
{तं तथा भस्मभूतं तु दृष्ट्वा पाण्डुसुताः प्रभो}
{अभवन्विस्मिता राजन्नर्जुनश्चेदमब्रवीत्}


\twolineshloka
{कृताञ्जलिः सप्रणयं प्रणिपत्याभिवाद्य ह}
{गोवन्द कस्माद्भगवन्रथो दग्धोऽयमग्निना}


\threelineshloka
{किमेतन्महदाश्चर्यमभवद्यदुनन्दन}
{तन्मे ब्रूहि महाबाहो श्रोतव्यं यदि मन्यसे ॥वासुदेव उवाच}
{}


\twolineshloka
{द्रोणकर्णास्त्रनिर्दग्धः पूर्वमेवायमर्जुन}
{मदास्थितत्वात्समरे न विशीर्णः परन्तप}


\threelineshloka
{इदानीं तु विशीर्णोऽयं दग्धो ब्रह्मास्त्रतेजसा}
{मया विमुक्तः कौन्तेय त्वय्यद्य कृतकर्मणि ॥सञ्जय उवाच}
{}


\twolineshloka
{ईषदुत्स्मयमानस्तु भगवान्केशवोऽरिहा}
{परिष्वज्य च राजानं युधिष्ठिरमभाषत}


\twolineshloka
{दिष्ट्या जयसि कौन्तेय दिष्ट्या ते शत्रवो जिताः}
{दिष्ट्या गाण्डीवधन्वा च भीमसेनश्च पाण्डवः}


\threelineshloka
{त्वं चापि कुशली राजन्माद्रीपुत्रौ च पाण्डवौ}
{मुक्ता वीरक्षयादिस्मात्सङ्ग्रामान्निहतद्विषः}
{क्षिप्रमुत्तरकालानि कुरु कार्याणि भारत}


\twolineshloka
{उपयातमुपप्लाव्यं सह गाण्डीवधन्वना}
{आनीय मधुपर्कं मां यत्पुरा त्वमवोचथाः}


\twolineshloka
{एष भ्राता सखा चैव तव कृष्ण धनञ्जयः}
{रक्षितव्यो महाबाहो सर्वास्वापत्स्विति प्रभो}


% Check verse!
तव चैव ब्रुवाणस्य तथेत्येवाहमब्रुवम्
\threelineshloka
{स सव्यसाची गुप्तस्ते विजयी च जनेश्वर}
{भ्रातृभिः सह राजेन्द्र शूरः सत्यपराक्रमः}
{मुक्तो वीरक्षयादस्मात्सङ्ग्रामाद्रोमहर्षणात्}


\threelineshloka
{एवमुक्तस्तु कृष्णेन धर्मराजो युधिष्ठिरः}
{हृष्टरोमा महाराज प्रत्युवाच जनार्दनम् ॥युधिष्ठिर उवाच}
{}


\twolineshloka
{प्रमुक्तं द्रोणकर्णाभ्यां ब्रह्मास्त्रमरिमर्दन}
{कस्त्वदन्यः सहेत्साक्षादपि वज्री पुरन्दरः}


\twolineshloka
{भवतस्तु प्रसादेन संग्रामे बहवो हताः}
{महारणगतः पार्थो यच्च नासीत्पराङ्मुखः}


\twolineshloka
{तथा तव महाबाहो पर्यायैर्बहुभिर्मया}
{कर्मणआमनुसन्धानात्तेजस्वी जगति श्रुता}


\twolineshloka
{उपप्लाव्ये महर्षिर्मे कृष्णद्वैपायनोऽब्रवीत्}
{यतो धर्मस्ततः कृष्णो यतः कृष्णस्ततो जयः}


\twolineshloka
{`एवमुक्तस्ततः कृष्णः प्रत्युवाच युधिष्ठिरम्}
{न तुल्याश्चार्जुनस्येह बलेन कुरुनन्दन}


\threelineshloka
{स एष सर्वाण्यस्त्राणि दिव्यानि प्राप्य शङ्करात्}
{सत्समो वा विशिष्टो वा रणे त्वमिति पाण्डवः}
{अनुज्ञातः पाण्डुसुतः पुनः प्रत्यागमन्महीम्}


\twolineshloka
{भूतं भव्यं भविष्यच्च अनुज्ञातस्त्वया विभो}
{निमेषार्धान्नरव्याघ्रो नयेदिति मतिर्मम}


\twolineshloka
{द्रोणं भीष्मं कृपं कर्णं द्रोणपुत्रं जयद्रथम्}
{निहन्तुं शक्नुयात्क्रुद्धो निमेषार्धाद्धनञ्जयः}


\twolineshloka
{सदेवासुरगन्धर्वान्सयक्षोरगराक्षसान्}
{त्रीन्वा लोकान्विजेतुं स शक्तः किमिह मानुषान्}


\twolineshloka
{विधिना विहितं चासौ मया सञ्चोदितोऽपि सन्}
{न चकार मतिं हन्तुं ततस्ते बलवत्तराः}


\twolineshloka
{अत्र गीता मया सुष्ठु गिरः सत्या महीपते}
{दर्सितं मयि सर्वं च तेनासौ जितवान्रिपून्}


\twolineshloka
{अर्जुनोऽपि महाबाहुर्मया तुल्यो महीपते}
{स महेश्वरलब्धास्त्रः किं न कुर्याद्विभुः प्रभो ॥'}


\twolineshloka
{इत्येवमुक्ते ते वीराः शिबिरं तव भारत}
{प्रविश्य प्रत्यपद्यन्त कोशरत्नर्धिसञ्चयान्}


\twolineshloka
{रजतं जातरूपं च मणीनथ च मौक्तिकान्}
{भूषणान्यथ मुख्यानि कम्बलान्यजिनानि च}


\twolineshloka
{`गजानश्वान्रथांश्चैव महान्ति शयनानि च'}
{दासीदासमसंख्येयं राज्योपकरणानि च}


\twolineshloka
{ते प्राप्य धनमक्षय्यं त्वदीयं भरतर्षभ}
{उदक्रोशन्महाभागा नरेन्द्र विजितारयः}


\twolineshloka
{ते तु वीराः समाश्वस्य वाहनान्यवमुच्य च}
{अतिष्ठन्त मुहुः सर्वे पाण्डवा विगतज्वराः}


\twolineshloka
{अथाब्रवीन्महाराज वासुदेवो महायशाः}
{अस्माभिर्मङ्गलार्थाय वस्तव्यं शिबिराद्बहिः}


\twolineshloka
{तथेत्युक्त्वा हि ते सर्वे पाण्डवाः सात्यकिस्तथा}
{वासुदेवेन सहिता मङ्गलार्थं बहिर्ययुः}


\twolineshloka
{ते समासाद्य सरितं पुण्यामोघवतीं नृप}
{न्यवसन्नथ तां रात्रिं पाण्डवा हतशत्रवः}


% Check verse!
युधिष्ठिरस्ततो राजा प्राप्तकालमचिन्तयन्
\twolineshloka
{तत्र ते गमनं माप्तं रोचते तव माधव}
{गान्धार्याः क्रोधदीप्तायाः प्रशमार्थमरिन्दम}


\threelineshloka
{हेतुकारणयुक्तैश्च वाक्यैः कालसमीरितैः}
{क्षिप्रमेव महाभाग गान्धारीं प्रशमिष्यसि}
{पितामहश्च भगवान्व्यासस्तत्र भविष्यति}


% Check verse!
ततः सम्प्रेषयामासुर्यादवं नागसाह्वयम्
\twolineshloka
{स च प्रायाज्जवेनाशु वासुदेवः प्रतापवान्}
{दारुकं रथमारोप्य येन राजाऽम्बिकासुतः}


\twolineshloka
{तमूचुः सम्प्रयास्यन्तं शैब्यसुग्रीववाहनम्}
{प्रत्याश्वासय गान्धारीं हतपुत्रां यशस्विनीम्}


\twolineshloka
{स प्रायात्पाण्डवैरुक्तस्तत्पुरं सात्वतां वरः}
{आससाद ततः क्षिप्रं गान्धारीं निहतात्मजाम्}


\chapter{अध्यायः ६५}
\twolineshloka
{जनमेजय उवाच}
{}


\twolineshloka
{किमर्थं द्विजशार्दूल धर्मराजो युधिष्ठिरः}
{गान्धार्याः प्रेषयामास वासुदेवं परन्तपम्}


\twolineshloka
{यदा पूर्वं गतः कृष्णः शमार्थं कौरवान्प्रति}
{न च तं लब्धवान्कामं ततो युद्धमभूदिदम्}


\twolineshloka
{निहतेषु तु योधेषु हते दुर्योधने तदा}
{पृथिव्यां पाण्डवेयस्य निःसपत्ने कृते युधि}


\twolineshloka
{विद्रुते शिबिरे शून्ये प्राप्ते यशसि चोत्तमे}
{किं तु तत्कारणं ब्रह्मन्येन कृष्णो गतः पुनः}


\twolineshloka
{न चैतत्कारणं ब्रह्मन्नल्पं विप्रतिभाति मे}
{यत्रागमदमेयात्मा स्वयमेव जनार्दनः}


\threelineshloka
{तत्त्वतो वै समाचक्ष्व सर्वमध्वर्युसत्तम}
{यच्चात्र कारणं ब्रह्मन्कार्यस्यास्य विनिश्चये ॥वैशम्पायन उवाच}
{}


\twolineshloka
{त्वद्युक्तोऽयमनुप्रश्नो यन्मां पृच्छसि पार्थिव}
{तत्तेऽहं संप्रवक्ष्यामि यथावद्भरतर्षभ}


\twolineshloka
{हतं दुर्योधनं दृष्ट्वा भीमसेनेन संयुगे}
{व्युत्क्रम्य समयं राजन्धार्तराष्ट्रं महाबलम्}


\twolineshloka
{अन्यायेन हतं दृष्ट्वा गदायुद्धेन भारत}
{युधिष्ठिरं महाराज महद्भयमथाविशत्}


\twolineshloka
{सोऽचिन्तयन्महाभागां गान्धारीं तपसान्विताम्}
{घोरेण तपसा युक्तां त्रैलोक्यमपि सा दहेत्}


\twolineshloka
{तस्य चिन्तयमानस्य बुद्धिः समभवत्तदा}
{गान्धार्याः क्रोधदीप्तायाः पूर्वं प्रशमनं भवेत्}


\twolineshloka
{सा हि पुत्रवधं श्रुत्वा कृतमस्माभिरीदृशम्}
{मानसेनाग्निना क्रुद्धा भस्मसान्नः करिष्यति}


\twolineshloka
{कथं दुःखमिदं तीव्रं गान्धारी सम्प्रशक्ष्यति}
{श्रुत्वा विनिहतं पुत्रं छलेनाजिह्मयोधिनम्}


\twolineshloka
{एवं विचिन्त्य बहुधा भयशोकसमन्वितः}
{वासुदेवमिदं वाक्यं धर्मराजोऽभ्यभाषत}


\twolineshloka
{तव प्रसादाद्गोविन्द राज्यं निहतकण्टकम्}
{अप्राप्यं मनसाऽपीदं प्राप्तमस्माभिरच्युत}


\twolineshloka
{प्रत्यक्षं मे महाबाहो सङ्ग्रामे रोमहर्षणे}
{विमर्दः सुमहान्प्राप्तस्त्वया यादवनन्दन}


\twolineshloka
{त्वया देवासुरे युद्धे वधार्थममरद्विषाम्}
{यथा साह्यं पुरा दत्तं हताश्च विबुधद्विषः}


\twolineshloka
{साह्यं तथा महाबाहो दत्तमस्माकमच्युत}
{सारथ्येन च वार्ष्णेय भवता हि धृता वयम्}


\twolineshloka
{यदि न त्वं भवेन्नाथः फल्गुनस्य महारणे}
{कथं शक्यो रणे जेतुं भवेदेष बलार्णवः}


\twolineshloka
{गदाप्रहारा विपुलाः परिघैश्चापि ताडनम्}
{शक्तिभिर्भिण्डिपालैश्च तोमरैः सपरश्वथैः}


\twolineshloka
{अस्मत्कृते त्वया कृष्ण वाचः सुपरुषाः श्रुताः}
{शस्त्राणां च निपाता वै वज्रस्पर्शोपमा रणे}


\twolineshloka
{ते च ते सफला जाता हते दुर्योधनेऽच्युत}
{तत्सर्वं न यथा नश्येत्पुनः कृष्ण तथा कुरु}


\twolineshloka
{सन्देहडोलां प्राप्तं नश्चेतः कृष्ण जये सति}
{गान्धार्या हि महाबाहो क्रोधं शमय माधव}


\threelineshloka
{सा हि नित्यं महाभागा तपसोग्रेण कर्शिता}
{पुत्रपौत्रवधं श्रुत्वा ध्रुवं नः सम्प्रधक्ष्यति}
{तस्या प्रसादनं वीर प्राप्तकालं मतं मम}


\twolineshloka
{कश्च तां क्रोधसन्दीप्तां पुत्रव्यसनकर्शिताम्}
{वीक्षितुं पुरुषः शक्तस्त्वामृते पुरुषोत्तम}


\twolineshloka
{तत्र मे गमनं प्राप्तं रोचते तव माधव}
{गान्धार्याः क्रोधदीप्तायाः प्रशमार्थमरिन्दम}


\twolineshloka
{त्वं हि कर्ता विकर्ता च लोकानां प्रभवाव्ययः}
{हेतुकारणसंयुक्तैर्वाक्यैः कालसमीरितैः}


\twolineshloka
{क्षिप्रमेव महाबाहो गान्धारीं शमयिष्यसि}
{पितामहश्च भगवान्कृष्णस्तत्र भविष्यति}


\twolineshloka
{सर्वथा ते महाबाहो गान्धार्याः क्रोधनाशनम्}
{कर्तव्यं सात्वतां श्रेष्ठ पाण्डवानां हितार्थिना}


\twolineshloka
{धर्मराजस्य वचनं श्रुत्वा यदुकुलोद्वहः}
{आमन्त्र्य दारुकं प्राह रथः सञ्जो विधीयताम्}


\twolineshloka
{केशवस्य वचः श्रुत्वा त्वरमाणोऽथ दारुकः}
{न्यवेदयद्रथं सज्जं केशवाय महात्मने}


\twolineshloka
{तं रथं यादवश्रेष्ठः समारुह्य परन्तपः}
{जगाम हास्तिनपुरं त्वरितः केशवो विभुः}


\twolineshloka
{ततः प्रायान्महाराज माधवो भगवान्रथी}
{नागसाह्वयमासाद्य प्रविवेश च वीर्यवान्}


\threelineshloka
{प्रविश्य नगरं वीरो रथघोषेण नादयन्}
{विदितो धृतराष्ट्रस्य सोऽवतीर्य रथोत्तमात्}
{अभ्यगच्छददीनात्मा धृतराष्ट्रनिवेशनम्}


% Check verse!
पूर्वं चाभिगतं तत्र सोऽपश्यदृषिसत्तमम्
\twolineshloka
{पादौ प्रपीड्य कृष्णस्य राज्ञश्चापि जनार्दनः}
{अभ्यवादयदव्यग्रो गान्धारीं चापि केशवः}


\twolineshloka
{ततस्तु यादवश्रेष्ठो धृतराष्ट्रमधोक्षजः}
{पाणिमालम्ब्य राजेन्द्र सुस्वरं प्ररुरोद ह}


\threelineshloka
{स मुहूर्तादिवोत्सृज्य बाष्पं शोकसमुद्भवम्}
{प्रक्षाल्य वारिणा नेत्रे ह्याचम्य च यथाविधि}
{उवाच प्रश्रितं वाक्यं धृतराष्ट्रमरिन्दमः}


\twolineshloka
{न तेऽस्त्यविदितं किञ्चिद्भूतं भव्यं च भारत}
{कालस्य च यथावृत्तं तत्ते सुविदितं प्रभो}


\twolineshloka
{यदिदं पाण्डवैः सर्वैस्तव चित्तानुरोधिभिः}
{कथं कुलक्षयो न स्यात्तथा क्षत्रस्य भारत}


\twolineshloka
{भ्रातृभिः समयं कृत्वा क्षान्तवान्धर्मवत्सलः}
{द्यूतच्छलजितैः शुद्वैर्वनवासो ह्युपागतः}


\twolineshloka
{अज्ञातवासचर्या च नानावेषसमावृतैः}
{अन्ये च बहवः क्लेशास्त्वशक्तैरिव सर्वदा}


\twolineshloka
{मया च स्वयमागम्य युद्धकाल उपस्थिते}
{सर्वलोकस्य सान्निध्येग्रामांस्त्वं पञ्चयाचितः}


\twolineshloka
{त्वया कालोपसृष्टेन लोभतो नापवर्जिताः}
{तवापराधान्नृपते सर्वं क्षत्रं क्षयं गतम्}


\threelineshloka
{भीष्मेण सोमदत्तेन बाह्लीकेन कृपेण च}
{द्रोणेन च सपुत्रेण विदुरेण च धीमता}
{याचितस्त्वं शमं नित्यं न च तत्कृतवानसि}


\twolineshloka
{कालोपहतचित्ता हि सर्वे मुह्यन्ति भारत}
{यथा मूढो भवान्पूर्वमस्मिन्नर्थे समुद्यते}


\twolineshloka
{किमन्यत्कालयोगाद्धि द्विष्टमेव परायणम्}
{मा च दोषान्महाप्राज्ञ पाण्डवेषु निवेशय}


\twolineshloka
{अल्पोप्यतिक्रमो नास्ति पाण्डवानां महात्मनाम्}
{धर्मतो न्यायतश्चैव स्नेहतश्च परन्तप}


\twolineshloka
{एतत्सर्वं तु विज्ञाय ह्यात्मदोषकृतं फलम्}
{तन्मन्युं पाण्डुपुत्रेषु न भवान्कर्तुमर्हति}


\twolineshloka
{कुलं वंशश्च पिण्डाश्च यच्च पुत्रकृत फलम्}
{गान्धार्यास्तव वै नाथ पाण्डवेषु प्रतिष्ठितम्}


\twolineshloka
{त्वं चैव कुरुशार्दूल गान्धारी च यशस्विनी}
{मा शुचो नरशार्दूल पाण्डवान्प्रतिकिल्बिम्}


\twolineshloka
{एतत्सर्वमनुध्याय आत्मनश्च व्यतिक्रमम्}
{शिवेन पाण्डवान्ध्याहि नमस्ते भरतर्षभ}


\twolineshloka
{जानासि च महाबाहो धर्मराजस्य या त्वयि}
{भक्तिर्भरतशार्दूल स्नेहश्चापि स्वभावतः}


\twolineshloka
{एतच्च कदनं कृत्वा शत्रूणामपकारिणाम्}
{दह्यते स दिवारात्रौ न च शर्माधिगच्छति}


\twolineshloka
{त्वां चैव नरशार्दूल गान्धारीं च यशस्विनीम्}
{स शोचन्नरशार्दूलः शान्तिं नैवाधिगच्छति}


\twolineshloka
{हिया च परयाऽविष्टो भवन्तं नाधिगच्छति}
{पुत्रशोकाभिसन्तप्तं बुद्धिव्याकुलितेन्द्रियम्}


\twolineshloka
{एवमुक्त्वा महाराज धृतराष्ट्रं यदूत्तमः}
{उवाच परमं वाक्यं गान्धारीं शोककर्शिताम्}


\twolineshloka
{सौबलेयि निबोध त्वं यत्त्वां वक्ष्यामि सुव्रते}
{त्वत्समा नास्ति लोकेऽस्मिन्नद्य सीमन्तिनी शुभे}


\twolineshloka
{जानासि च यथा राज्ञि सभायां मम सन्निधौ}
{धर्मार्थसहितं वाक्यमुभयोः पक्षयोर्हितम्}


\twolineshloka
{उक्तवत्यसि कल्याणि न च ते तनयैः कृतम्}
{दुर्योधनस्त्वया चोक्तो जयार्थी परुषं वचः}


\twolineshloka
{शृणु मूढ वचो मह्यं यतो धर्मस्ततो जयः}
{तदिदं समनुप्राप्तं तव वाक्यं नृपात्मजे}


\twolineshloka
{एवं विदित्वा कल्याणि मा स्म शोके मनः कृथाः}
{पाण्डवानां विनाशाय मा ते बुद्धिः कदाचन}


\twolineshloka
{शक्ता चासि महाभागे पृथिवीं सचराचराम्}
{चक्षुषा क्रोधदीप्तेन निर्दग्धुं तपसो बलात्}


\twolineshloka
{वासुदेववचः श्रुत्वा गान्धारी वाक्यमब्रवीत्}
{एवमेतन्महाबाहो यथा वदसि केशव}


\twolineshloka
{आधिभिर्दह्यमानाया मतिः सञ्चलिता मम}
{सा मे व्यवस्थिता श्रुत्वा तव वाक्यं जनार्दन}


\twolineshloka
{राज्ञस्त्वन्धस्य वृद्धस्य हतपुत्रस्य केशव}
{त्वं गतिः सहितैर्वीरैः पाण्डवैर्दिपदा वर}


\twolineshloka
{एतावदुक्त्वा वचनं मुखं प्रच्छाद्य वाससा}
{पुत्रशोकाभिसन्तप्ता गान्धारी प्ररुरोद ह}


\twolineshloka
{तत एनां महाबाहुः केश्वः शोककर्शिताम्}
{हेतुकारणसंयुक्तैर्वाक्यैराश्वासयत्प्रभुः}


\twolineshloka
{समाश्वास्य च गान्धारीं धृतराष्ट्रं च माधवः}
{द्रौणिसङ्कल्पितं भावमन्वबुध्यत केशवः}


\twolineshloka
{ततस्त्वरित उत्थाय पादौ मूर्ध्ना प्रणम्य च}
{द्वैपायनस्य राजेन्द्र ततः कौरवमब्रवीत्}


\twolineshloka
{आपृच्छे त्वां कुरुश्रेष्ठ मा च शोके मनः कृथाः}
{द्रौणेः पापोस्त्यभिप्रायस्तेनास्मि सहसोत्थितः}


% Check verse!
पाण्डवानां वधे रात्रौ बुद्धिस्तेन प्रदर्शिता
\twolineshloka
{एतच्छ्रुत्वा तु वचनं गान्धार्या सहितोऽब्रवीत्}
{धृतराष्ट्रो महाबाहुः केशवं केशिसूदनम्}


\twolineshloka
{शीघ्रं गच्छ महाबाहो पाण्डवान्परिपालय}
{भूयस्त्वया समेष्यामि क्षिप्रमेव जनार्दन}


% Check verse!
प्रायात्ततस्तु त्वरितो दारुकेण सहाच्युतः
\twolineshloka
{वासुदेवे गते राजन्धृतराष्ट्रं जनेश्वरम्}
{आश्वासयदमेयात्मा व्यासो लोकनमस्कृतः}


\twolineshloka
{वासुदेवोऽपि धर्मात्मा कृतकृत्यो जगाम ह}
{शिबिरं हास्तिनपुराद्दिदृक्षुः पाण्डवान्नृप}


\twolineshloka
{आगम्य शिबिरं रात्रौ सोऽभ्यगच्छत पाण्डवान्}
{तच्च तेभ्यः समाख्याय सहितस्तैः समाहितः}


\chapter{अध्यायः ६६}
\twolineshloka
{धृतराष्ट्र उवाच}
{}


\twolineshloka
{अधिष्ठितः पदा मूर्ध्नि भग्नसक्थो महीं गतः}
{शौटीर्यमानी पुत्रो मे किमभाषत सञ्जय}


\threelineshloka
{अत्यर्थं कोपनो राजा जावैरश्च पाण्डुषु}
{व्यसनं परमं प्राप्तः किमाह परमाहवे ॥सञ्जय उवाच}
{}


\twolineshloka
{शृणु राजन्प्रवक्ष्यामि यथावृत्तं नराधिप}
{राज्ञा यदुक्तं मग्नेन तस्मिन्व्यसनसागरे}


\twolineshloka
{भग्नसक्थो नृपो राजन्पांसुना सोऽवकुण्ठितः}
{यमयन्मूर्धजांस्तत्र वीक्ष्य चैव दिशो दश}


\twolineshloka
{केशान्नियस्य यत्नेन निःश्वसन्नुरगो यथा}
{संरम्भाश्रुपरीताब्यां नेत्राभ्यामभिवीक्ष्य माम्}


\threelineshloka
{बाहू धरण्यां निष्पिष्य सुदुर्मत्त इव द्विपः}
{प्रकीर्णान्मूर्धजान्धुन्वन्दन्तैर्दन्तानुपस्पृशन्}
{गर्हयन्पाण्डवं ज्येष्ठं निःश्वस्येदमथाब्रवीत्}


\twolineshloka
{भीष्मे शान्तनवे नाथे कर्णे शस्त्रभृतां वरे}
{गौतमे शकुनौ चापि द्रोणे चास्त्रभृतां वरे}


\threelineshloka
{अश्वत्थाम्नि यथा शल्ये शूरे च कृतवर्मणि}
{अन्येष्वपि च शूरेषु न्यस्तभारो महात्मसु}
{इमामवस्थां प्राप्तोऽस्मि कालो हि दुरतिक्रमः}


\twolineshloka
{एकादशचमूभर्ता सोऽहमेतां दशां गतः}
{कालं प्राप्य महाबाहो न कश्चिदतिवर्तते}


\twolineshloka
{आख्यातव्यं मदीयानां येऽस्मिञ्जीवन्ति संयुगे}
{यथाऽहं भीमसेनेन व्युत्क्रम्य समयं हतः}


\twolineshloka
{बहूनि सुनृशंसानि कृतानि खलु पाण्डवैः}
{भूरिश्रवसि कर्णे च भीष्मे द्रोणे च धीमति}


\twolineshloka
{इदं च गर्हितं कर्म नृशंसैः पाण्डवैः कृतम्}
{येन ते वाच्यतां सत्सु गमिष्यन्तीति मे मतिः}


\twolineshloka
{का प्रीतिः सत्वयुक्तस्य कृत्वोपाधिकृतं जयम्}
{को वा समयभेत्तारं बुधः सम्मन्तुमर्हति}


\twolineshloka
{अधर्मेण जयं लब्ध्वा को नु हृष्येत पण्डितः}
{यथा संहृष्यते पापः पाण्डुपुत्रो वृकोदरः}


\twolineshloka
{किन्नु चित्रमितस्त्वद्य भग्नसक्थस्य यन्मम}
{क्रुद्धेन भीमसेनेन पादेन मृदितं शिरः}


\twolineshloka
{प्रतपन्तं श्रिया जुष्टं वर्तमानं च बन्धुषु}
{एवं कुर्यान्नरो यो हि स वै स़ञ्जय पूरुषः}


\twolineshloka
{अभिज्ञौ युद्धधर्मस्य मम माता पिता च यौ}
{तौ हि सञ्जय दुःखार्तौ विज्ञाप्यौ वचनाद्वि मे}


\twolineshloka
{इष्टं भृत्या भृताः सम्यग्भूः प्रशास्ता ससागरा}
{मूर्ध्नि स्थितममित्राणां जीवतामेव स़ञ्जय}


\twolineshloka
{दत्ता दाया यथाशक्ति मित्राणां च प्रियं कृतम्}
{अमित्रा बाधिताः सर्वे को नु स्वन्ततरो मया}


\twolineshloka
{मानिता बान्धवाः सर्वे मान्यः सम्पूजितो जनः}
{त्रितयं सेवितं सर्वं को नु स्वन्ततरो मया}


\twolineshloka
{आज्ञप्तं नृपमुख्येषु मानः प्राप्तः सुदुर्लभः}
{आजानेयैस्तथा यातं को नु स्वन्ततरो मया}


\twolineshloka
{यातानि परराष्ट्राणि नृपा भुक्ताश्च दासवत्}
{प्रियेभ्यः प्रकृतं साधु को नु स्वन्ततरो मया}


\twolineshloka
{अधीतं विधिवद्दत्तं प्राप्तमायुर्निरामयम्}
{स्वधर्मेण जिता लोकाः को नु स्वन्ततरो मया}


\twolineshloka
{दिष्ट्या नाहं जितः सङ्ख्ये परान्प्रेष्यवदाश्रितः}
{दिष्ट्या मे विपुला लक्ष्मीर्मृते त्वन्यगता विभो}


\twolineshloka
{यदिष्टं क्षत्रबन्धूनां स्वधर्ममनुतिष्ठताम्}
{निधनं तन्मया प्राप्तं को नु स्वन्ततरो मया}


\twolineshloka
{दिष्ट्या नाहं परावृत्तो वैरात्प्राकृतवञ्जितः}
{दिष्ट्या न विमतिं काञ्चिद्भजित्वा तु पराजितः}


\twolineshloka
{सुप्तं वाऽथ प्रमत्तं वा यथा हन्याद्विषेण वा}
{एवं व्युत्क्रान्तधर्मेण व्युत्क्रम्य समयं हतः}


\twolineshloka
{अश्चत्थामा महाभागः कृतवर्मा च सात्वतः}
{कृपः शारद्वतश्चैव वक्तव्या वचनान्मम}


\twolineshloka
{अधर्मेण प्रवृत्तानां पाण्डवानामनेकशः}
{विश्वासं समयघ्नानां न यूयं गन्तुमर्हथ}


\twolineshloka
{वार्तिकांश्चाब्रवीद्राजा पुत्रस्ते सत्यविक्रमः}
{अधर्माद्भीमसेनेन निहतोऽहं यथा रणे}


\twolineshloka
{सोऽहं द्रोणं स्वर्गगतं कर्णशल्यावुभौ तथा}
{वृषसेनं महावीर्यं शकुनिं चापि सौबलम्}


\twolineshloka
{जलसन्धं महावीर्यं भगदत्तं च पार्थिवम्}
{सोमदत्तं महेष्वासं सैन्धवं च जयद्रथम्}


\twolineshloka
{दुऋशासनपुरोगांश्च भ्रातॄनात्मसमांस्तथा}
{दौःशासनिं च विक्रान्तं लक्ष्मणं चात्मजावुभौ}


\twolineshloka
{एतांश्चान्यांश्च सुबहून्मदीयांश्च सहस्रशः}
{पृष्ठतोऽनुगमिष्यामि सार्थहीनो यथाऽध्वगः}


\twolineshloka
{कथं भ्रातॄन्हताञ्श्रुत्वा भर्तारं च स्वसा मम}
{रोरूयामाणा दुःखार्ता दुःशला सा भविष्यति}


\twolineshloka
{स्नुषाभिः प्रस्नुषाभिश्च वृद्धो राजा पिता मम}
{गान्धारीसहितश्चैव का गतिं प्रतिपत्स्यति}


\twolineshloka
{नूनं लक्ष्मणमाताऽपि हतपुत्रा हतेश्वरा}
{विनाशं यास्यति क्षिप्रं कल्याणी पृथुलोचना}


\twolineshloka
{यदि जानाति चार्वाकः परिव्राद्वाग्विशारदः}
{करिष्यति महाभागो ध्रुवं चापचितिं मम}


\twolineshloka
{समन्तप़ञ्चके पुण्ये त्रिषु लोकेषु विश्रुते}
{अहं निधनमासाद्य लोकान्प्राप्स्यामि शाश्वतान्}


\twolineshloka
{ततो जनसहस्राणि बाष्पपूर्णानि दिशो दश ॥ससागरवना घोरा पृथिवी सचराचरा}
{}


\twolineshloka
{चचालाथ सनिर्हादा दिशश्चैवाविलाऽभवन् ॥ते द्रोणपुत्रमासाद्य यथावृत्तं न्यवेदयन्}
{}


\twolineshloka
{व्यवहारं गदायुद्धे पार्थिवस्य च पातनम् ॥तदाख्याय ततः सर्वे द्रोणपुत्रस्य भारत}
{}


\twolineshloka
{`वादिका दुःखसन्तप्ताः शोकोपहतचेतसः'}
{ध्यात्वा च सुचिरं कालं जग्मुरार्ता यथागतम्}


\chapter{अध्यायः ६७}
\threelineshloka
{सञ्जय उवाच}
{ततस्ते सहिताः सर्वे प्रयाता दक्षिणामुखाः}
{उपास्तमयवेलायां शिबिराभ्याशमागताः}


