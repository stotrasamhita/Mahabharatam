\part{अनुशासनपर्व}
\chapter{अध्यायः १}
% Check verse!
श्रीवेदव्यासाय नमः
\fourlineindentedshloka
{नारायणं नमस्कृत्य नरं चैव नरोत्तमम्}
{देवीं सरस्वतीं चैव(व्यासं)ततो जयमुदीरयेत् ॥वैशंपायन उवाच}
{`शरतल्पे महात्मानं शयानमपराजितम्}
{युधिष्ठिर उपागम्य प्रणिपत्येदमब्रवीत् ॥'}


\twolineshloka
{शमो बहुविधाकारः सूक्ष्म उक्तः पितामह}
{न च मे हृदये शान्तिरस्ति श्रुत्वेदमीदृशम्}


\twolineshloka
{अस्मिन्नर्थे बहुविधा शान्तिरुक्ता पितामह}
{स्वकृतात्का नु शान्ति स्याच्छमाद्बहुविधादपि}


\twolineshloka
{शराचितं शरीरं हि तीव्रव्रणमुदीक्ष्य ते}
{शमं नोपलभे वीर दुष्कृतान्येव चिन्तयन्}


\twolineshloka
{रुधिरेणावसिक्ताङ्गं प्रस्रवन्तं यथाचलम्}
{त्वां दृष्ट्वा पुरुषव्याघ्र सीदे वर्षास्विवाम्बुजम्}


\twolineshloka
{अतः कष्टतरं किन्नु मत्कृते यत्पितामहः}
{इमामवस्थां `गमितः प्रत्यमित्रै रणाजिरे}


\twolineshloka
{तथा चान्ये नृपतयः सहपुत्राः सबान्धवाः}
{मत्कृते निधनं प्राप्ताः किन्नु कष्टतरं ततः}


\twolineshloka
{वयं हि धार्तराष्ट्राश्च काममन्युवशं गताः}
{कृत्वेदं निन्दितं कर्म प्राप्स्यामः कां गतिं नृप}


\twolineshloka
{इदं तु धार्तराष्ट्रास्य श्रेयो मन्ये जनाधिप}
{इमामवस्थां संप्राप्तं यदसौ त्वां न पश्यति}


\twolineshloka
{सोऽहमार्तिकरो राजन्सुहृद्वधकरस्तथा}
{न शान्तिमधिगच्छामि पश्यंस्त्वां दुःखितं क्षितौ}


\twolineshloka
{दुर्योधनो हि समरे सहसैन्यः सहानुजः}
{निहतः क्षत्रधर्मेऽस्मिन्दुसत्मा कुलपांसनः}


\twolineshloka
{न स पश्यति दुष्टात्मा त्वामद्य पतितं क्षितौ}
{अतः श्रेयो मृतं मन्ये नेह जीवितमात्मनः}


\threelineshloka
{अहं हि समरे वीर गमितः शत्रुभिः क्षयम्}
{अभविष्यं यदि पुरा सह भ्रातृभिरच्युत}
{न त्वामेवं सुदुःखार्तमद्राक्षं सायकार्दितम्}


% Check verse!
नूनं हि पापकर्माणो धात्रा सृष्टाः स्म हे नृप
\threelineshloka
{अन्यस्मिन्नपि लोके वै यथा मुच्येम किल्बिषात्}
{तथा प्रशाधि मां राजन्मम चेदिच्छसि प्रियम् ॥भीष्म उवाच}
{}


\twolineshloka
{परतन्त्रं कथं हेतुमात्मानमनु पश्यसि}
{कर्मणां हि महाभाग सूक्ष्मं ह्येतदतीन्द्रियम्}


\twolineshloka
{अत्राप्युदाहरन्तीममितिहासं पुरातनम्}
{संवादं मृत्युगौतम्योः काललुब्धकपन्नगैः}


\twolineshloka
{गौतमी नाम काऽप्यासीत्स्थविरा शमसंयुता}
{सर्पेण दष्टं स्वं पुत्रमपश्यद्गतचेतनम्}


\twolineshloka
{अथ तं स्नायुपाशेन बद्ध्वा सर्पममर्षितः}
{लुब्धकोऽर्जुनको नाम नाम गौतम्याः समुपानयत्}


\twolineshloka
{तां चाब्रवीदयं ते स पुत्रहा पन्नगाधमः}
{ब्रहि क्षिप्रं महाभागे वध्यतां केन हेतुना}


\threelineshloka
{अग्नौ प्रक्षिप्यतामेष च्छिद्यतां खण्डशोपि वा}
{न ह्यं बालहा पापश्चिरं जीवितुमर्हति ॥गौतम्युवाच}
{}


\twolineshloka
{विसृजैनमबुद्धिस्त्वमवध्योऽर्जुनक त्वया}
{को ह्यात्मानं गुरुं कुर्यात्प्राप्तव्ये सति चिन्तयन्}


\twolineshloka
{प्लवन्ते धर्मलघवो लोकेऽम्भसि यथा प्लवाः}
{मज्जन्ति पापगुरवः शस्त्रं स्कन्नमिवोदके}


\threelineshloka
{नास्यामृतत्वं भवितैवं हतेऽस्मि-ञ्जीवत्यस्मिन्कोऽत्ययः स्यादयं ते}
{अस्योत्सर्गे प्राणयुक्तस्य जन्तो-र्मृत्युं लोके को न गच्छेदनन्ते ॥लुब्धक उवाच}
{}


\twolineshloka
{जानाम्यहं नेह गुणागुणज्ञाःसदायुक्ता गुरवो वै भवन्ति}
{स्वर्गस्य ते सूपदेशा भवन्तितस्मात्क्षुद्रं सर्पमेनं हनिष्ये}


\threelineshloka
{शममीप्सन्तः कालयोगं त्यजन्तिसद्यः शुचं त्वर्थविदस्त्यजन्ति}
{श्रियः क्षयः शोचतां नित्यशो हितस्माच्छुचं मुञ्च हते भुजङ्गे ॥गौतम्युवाच}
{}


\twolineshloka
{न चैवार्तिर्विद्यतेऽस्मद्विधानांधर्मात्मानः सर्वदा सञ्जना हि}
{नित्यायस्तो बालजनो न चाहंधर्मोपैति प्रभवाम्यस्य नाहम्}


\threelineshloka
{न ब्राह्मणानां कोपोऽस्ति कुतः कोपाच्च यातना}
{मार्दवात्क्षम्यतां साधो मुच्यतामेष पन्नगः ॥लुब्धक उवाच}
{}


\threelineshloka
{हत्वा लाभः श्रेय एवाव्ययः स्या-त्सद्योलाभःस्याद्बलिभ्यः प्रशस्तः}
{कालाल्लाभो यस्तु सद्यो भवेतश्रेयोलाभः कुत्सिते त्वीदृशि स्यात् ॥गौतम्युवाच}
{}


\threelineshloka
{काऽर्थप्राप्तिर्गृह्य शत्रुं निहत्यका कामाप्तिः प्राप्य शत्रुं न मुक्त्वा}
{कस्मात्सौम्याहं न क्षमे नो भुजङ्गेमोक्षार्थं वा कस्य हेतोर्न कुर्याम् ॥लुब्धक उवाच}
{}


\twolineshloka
{अस्मादेकाद्बहवो रक्षितव्यानैको बहुभ्यो गौतमि रक्षितव्यः}
{कृतागसं धर्महेतोस्त्यजन्तिसरीसृपं पापमिमं त्यज त्वम् ॥गौतम्युवाय}


\threelineshloka
{नास्मिन्हते पन्नगे पुत्रको मेसम्प्राप्स्यते लुब्धक जीवितं वै}
{गुणं चान्यं नास्य वधे प्रपश्येतस्मात्सर्पं लुब्धक मुञ्च जीवम् ॥लुब्धक उवाच}
{}


\threelineshloka
{वृत्रं हत्वा देवराट् श्रेष्ठभाग्वैयज्ञं हत्वा भागमवाप चैव}
{शूली देवो देववृत्तं चर त्वंक्षिप्रं सर्पं जहि मा भूत्ते विशङ्का ॥भीष्म उवाच}
{}


\twolineshloka
{असकृत्प्रोच्यमानाऽपि गौतमी भुजगं प्रति}
{लुब्धकेन महाभागा सा पापे नाकरोन्मतिम्}


\threelineshloka
{ईषदुच्छ्वसमानस्तु कृच्छ्रात्संस्तभ्य पन्नगः}
{उत्ससर्ज गिरं मन्दां मानुषीं पाशपीडितः ॥सर्प उवाच}
{}


\twolineshloka
{को न्वर्जुनक दोषोऽत्र विद्यते मम बालिश}
{अस्वतन्त्रं हि मां मृत्युर्विवशं यदचूचुदत्}


\fourlineindentedshloka
{तस्यायं वचनाद्दष्टो न कोपेन न काम्यया}
{तस्य तत्किल्बिषं लुब्ध विद्यते यदि किञ्चन}
{लुब्धक उवाच}
{}


\twolineshloka
{यद्यन्यवशगेनेदं कृतं ते पन्नगाशुभम्}
{कारणं वै त्वमप्यत्र तस्मात्त्वमपि किल्बिषी}


\twolineshloka
{मृत्पात्रस्य क्रियायां हि दण्डचक्रादयो यथा}
{कारणत्वे प्रकल्प्यन्ते तथा त्वमपि पन्नग}


\threelineshloka
{किल्बिषी चापि मे वध्यः किल्बिषी चासि पन्नग}
{आत्मानं कारणं ह्यत्र त्वमाख्यासि भुजङ्गम ॥सर्प उवाच}
{}


\twolineshloka
{सर्व एते ह्यस्ववशा दण्डचक्रादयो यथा}
{तथाहमपि तस्मान्मे नैष दोषो मतस्तव}


\twolineshloka
{अथवा मतमेतत्ते तेप्यन्योऽन्यप्रयोजकाः}
{कार्यकारणसन्देहो भवत्यन्योन्यचोदनात्}


\threelineshloka
{एवं सति न दोषो मे नास्मि वध्यो न किल्बिषी}
{किल्बिषं समवाये स्यान्मन्यसे यदि किल्बिषम् ॥लुब्धक उवाच}
{}


\twolineshloka
{कारणं यदि स्याद्वै न कर्ता स्यास्त्वमप्युत}
{विनाशे कारणं त्वं च तस्माद्वध्योऽसि मे मतः}


\threelineshloka
{असत्यपि कृते कार्ये नेह पन्नग लिप्यते}
{तस्मान्नात्रैव हेतुः स्याद्वध्यः किं बहु भाषसे ॥सर्प उवाच}
{}


\twolineshloka
{कार्याभावे क्रिया न स्यात्सत्यसत्यपि कारणे}
{तस्मात्समेऽस्मिन्हेतौ मे वाच्यो हेतुर्विशेषतः}


\threelineshloka
{यद्यहं कारणत्वेन मतो लुब्धक तत्त्वतः}
{अन्यः प्रयोक्तास्यादत्र किन्नु जन्तुविनाशने ॥लुब्धक उवाच}
{}


\threelineshloka
{वध्यस्त्वं मम दुर्बुद्धे बालघाती नृशंसकृत्}
{भाषसे किं बहु पुनर्वध्यः सन्पन्नगाधम ॥सर्प उवाच}
{}


\threelineshloka
{यथा हवींषि जुह्वाना मखे वै लुब्धकर्त्विजः}
{न फलं प्राप्नुवन्त्यत्र फलयोगे तथा ह्यहम् ॥भीष्म उवाच}
{}


\twolineshloka
{तथा ब्रुवति तस्मिंस्तु पन्नगे मृत्युचोदिते}
{आजगाम ततो मृत्युः पन्नगं चाब्रवीदिदम्}


\twolineshloka
{प्रचोदितोऽहं कालेन पन्नग त्वामचूचुदम्}
{विनाशहेतुर्नास्य त्वमहं न प्राणिनः शिशोः}


\twolineshloka
{यथा वायुर्जलधरान्विकर्षति ततस्ततः}
{तद्वज्जलदवत्सर्प कालस्याहं वशानुगः}


\twolineshloka
{सात्विका राजसाश्चैव तामसा ये च केचन}
{भावाः कालात्मकाः सर्वे प्रवर्तन्तेह जन्तुषु}


\twolineshloka
{जङ्गमाः स्थावराश्चैव दिवि वा यदि वा भुवि}
{सर्वे कालात्मकाः सर्प कालात्मकमिदं जगत्}


\twolineshloka
{प्रवृत्तयश्च लोके या तथैव च निवृत्तयः}
{तासां विकृतयो याश्च सर्वं कालात्मकं स्मृतम्}


\twolineshloka
{आदित्यश्चन्द्रमा विष्णुरापो वायुः शतक्रतुः}
{अग्निः खं पृथिवी मित्रः पर्जन्यो वसवोऽदितिः}


\twolineshloka
{सरितः सागराश्चैव भावाभावौ च पन्नग}
{सर्वे कालेन सृज्यन्ते ह्रियन्ते च पुनःपुनः}


\threelineshloka
{एवं ज्ञात्वा कथं मां सदोषं सर्प मन्यसे}
{अथ चैवं गते दोषे मयि त्वमपि दोषवान् ॥सर्प उवाच}
{}


\twolineshloka
{निर्दोषं दोषवन्तं वा न त्वां मृत्यो ब्रवीम्यहम्}
{त्वयाऽहं चोदित इति ब्रवीम्येतावदेव तु}


\twolineshloka
{यदि काले तु दोषोऽस्ति यदि तत्रापि नेष्यते}
{दोषो नैव परीक्ष्यो मे न ह्यत्राधिकृता वयम्}


\threelineshloka
{निर्मोक्षस्त्वस्य दोषस्य मया कार्यो यथा तथा}
{मृत्योरपि न दोषः स्यादिति मेऽत्र प्रयोजनम् ॥भीष्म उवाच}
{}


\fourlineindentedshloka
{सर्पोऽथार्जुनकं प्राह श्रुतं ते मृत्युभाषितम्}
{नानागसं मां पाशेन सन्तापयितुमर्हसि}
{लुब्धक उवाच}
{}


\twolineshloka
{मृत्योः श्रुतं मे वचनं तव चैव भुजङ्गम}
{नैव तावददोषत्वं भवति त्वयि पन्नग}


\twolineshloka
{मृत्युस्त्वं चैव हेतुर्हि बालस्यास्य विनाशने}
{उभयं कारणं मन्ये न कारणमकारणम्}


\threelineshloka
{धिङ्भृत्युं च दुरात्मानं क्रूरं दुःखकरं सताम्}
{त्वां चैवाहं वधिष्यामि परपापस्य कारणम् ॥मृत्युरुवाच}
{}


\threelineshloka
{विवशौ कालवशगावावां निर्दिष्टकारिणौ}
{नावां दोषेण गन्तव्यौ यदि सम्यक्प्रपश्यसि ॥लुब्धक उवाच}
{}


\threelineshloka
{युवामुभौ कालवशौ यदि मे मृत्युपन्नगौ}
{हर्षक्रोधौ यथा स्यातामेतदिच्छामि वेदितुम् ॥मृत्युरुवाच}
{}


\twolineshloka
{या काचिदेव चेष्टा स्यान्सर्वा कालप्रचोदिता}
{पूर्वमेवैतदुक्तं हि मया लुब्धक तत्वतः}


\threelineshloka
{तस्मादुभौ कालवशावावां निर्दिष्टकारिणौ}
{नावां दोपेण गन्तव्यौ त्वया लुब्धक कर्हिचित् ॥भीष्म उवाच}
{}


\twolineshloka
{अथोपगम्य कालस्तु तस्मिन्धर्मार्थसंशये}
{अब्रवीत्पन्नगं मृत्युं लुब्धं चार्जुनकं तथा}


\twolineshloka
{न ह्यहं नाप्ययं मृत्युर्नायं लुब्धक पन्नगः}
{किल्बिषी जन्तुमरणे न वयं हि प्रयोजकाः}


\twolineshloka
{अकरोद्यदयं कर्म तन्नोऽर्जुनक चोदकम्}
{विनाशहेतुर्नान्योऽस्य वध्यतेऽयं स्वकर्मणा}


\twolineshloka
{यदनेन कृतं कर्म तेनायं निधनं गतः}
{विनाशहेतुः कर्मास्य सर्वे कर्मवशा वयम्}


\twolineshloka
{कर्मदायादवाँल्लोकः कर्मसम्बन्धलक्षणः}
{कर्माणि चोदयन्तीह यथान्योन्यं तथावयम्}


\twolineshloka
{यथा मृत्पिण्डतः कर्ता कुरुते यद्यदिच्छति}
{एवमात्मकृतं कर्म मानवः प्रतिपद्यते}


\twolineshloka
{यथा च्छायातपौ नित्यं सुसम्बद्धौ निरन्तरम्}
{तथा कर्म च कर्ता च सम्बद्धावात्मकर्मभिः}


\twolineshloka
{एवं नाहं न वै मृत्युर्न सर्पो न तथा भव न्}
{न चेयं ब्राह्मणी वृद्धा शिशुरेवात्र कारणम्}


\twolineshloka
{तस्मिंस्तथा ब्रुवाणे तु ब्राह्मणी गौतमी नृप}
{स्वकर्मप्रत्ययाँल्लोकान्मत्वाऽर्जुनकमब्रवीत्}


\twolineshloka
{नैव कालो न भुजगो न मृत्युरिह कारणम्}
{स्वकर्मभिरयं बालः कालेन निधं गतः}


\threelineshloka
{मया च तत्कृतं कर्म येनायं मे मृतः सुतः}
{यातु कालस्तथा मृत्युर्मुञ्चार्जुनक पन्नगम् ॥भीष्म उवाच}
{}


\twolineshloka
{ततो यथागतं जग्मुर्मृत्युः कालोऽथ पन्नगः}
{अभूद्विशोकोऽर्जुनको विशोका चैव गौतमी}


\twolineshloka
{एतच्छ्रुत्वा शमं गच्छ मा भूः शोकपरो नृप}
{स्वकर्मप्रत्ययाँल्लोकांस्त्रीन्विद्धि समितिंजय}


\threelineshloka
{नैव त्वया कृतं कर्म नापि दुर्योधनेन वै}
{कालेनैतत्कृतं विद्धि निहता येन पार्थिवाः ॥वैशम्पायन उवाच}
{}


\twolineshloka
{इत्येतद्वचनं श्रुत्वा बभूव विगतज्वरः}
{युधिष्ठरो महातेजाः पप्रच्छेदं च धर्मवित्}


\chapter{अध्यायः २}
\twolineshloka
{युधिष्ठिर उवाच}
{}


\twolineshloka
{पितामह महाप्राज्ञ सर्वशास्त्रविशारद}
{श्रुतं मे महदाख्यानमिदं मतिमतांवर}


\twolineshloka
{भूयस्तु श्रोतुमिच्छामि धर्मार्थसहितं नृप}
{कथ्यमानं त्वया किञ्चित्तन्मे व्याख्यातुमर्हसि}


\threelineshloka
{केन मृत्युर्गृहस्थेन धर्ममाश्रित्य निर्जितः}
{इत्येतद्धर्ममाचक्ष्व तत्त्वेनापि च पार्थिव ॥भीष्म उवाच}
{}


\twolineshloka
{अत्राप्युदाहरन्तीममितिहासं पुरातनम्}
{यथा मृत्यर्गृहस्थेन धर्ममाश्रित्य निर्जितः}


\twolineshloka
{मनोः प्रजापते राजन्निक्ष्वाकुरभवत्सुतः}
{तस्य पुत्रशतं जज्ञे नृपतेः सूर्यवर्चसः}


\twolineshloka
{दशमस्तस्य पुत्रस्तु दशाश्वो नाम भारत}
{माहिष्मत्यामभूद्राजा धर्मात्मा सत्यविक्रमः}


\twolineshloka
{दशाश्वस्य सुतस्त्वासीद्राजा परमधार्मिकः}
{सत्ये तपसि दाने च यस्य नित्यं रतं मनः}


\twolineshloka
{मदिराश्व इति ख्यातः पृथिव्यां पृथिवीपतिः}
{धनुर्वेदे च वेदे च निरतो योऽभवत्सदा}


\twolineshloka
{मदिराश्वस्य पुत्रस्तु द्युतिमान्नाम पार्थिवः}
{महाभागो महातेजा महासत्वो महाबलः}


\threelineshloka
{पुत्रो द्युतिमतस्त्वासीद्राजा परमधार्मिकः}
{सर्वलोकेषु विख्यातः सुवीरो नाम नामतः}
{धर्मात्मा कोशवांश्चापि देवराज इवापरः}


\twolineshloka
{सुवीरस्य तु पुत्रोऽभूत्सर्वसङ्ग्रामदुर्जयः}
{दुर्जयो नाम विख्यातः सर्वशस्त्रभृतांवर}


\twolineshloka
{दुर्जयस्येन्द्रवपुषः पुत्रोऽग्निसदृशद्युतिः}
{दुर्योधनो नाम महान्राजा राजर्षिसत्तमः}


\twolineshloka
{तस्येन्द्रसमवीर्यस्य सङ्ग्रामेष्वनिवर्तिनः}
{विषये वासवस्तस्य सम्यगेव प्रवर्षति}


\twolineshloka
{रत्नैर्धनैश्च पशुभिः सस्यैश्चापि पृथग्विधैः}
{नगरं विषयश्चास्य प्रतिपूर्णस्तदाऽभवत्}


\twolineshloka
{न तस्य विषये चाभूत्कृपणो नापि दुर्गतः}
{व्याधितो वा कृशो वाऽपरि तस्मिन्नाभून्नरः क्वचित्}


\twolineshloka
{सुदक्षिणो मधुरवागनसूयुर्जितेन्द्रियः}
{धर्मात्मा चानृशंसश्च विक्रान्तोऽथाविकत्थनः}


\twolineshloka
{यज्वा च दान्तो मेधावी ब्रह्मण्यः सत्यसङ्गरः}
{न चावमन्ता दाता च वेदवेदाङ्गपारगः}


\twolineshloka
{तं नर्मदा देवनदी पुण्या शीतजला शिवा}
{चकमे पुरुषव्याघ्रं स्वेन भावेन भारत}


\twolineshloka
{तस्यां जज्ञे तदा नद्यां कन्या राजीवलोचना}
{नाम्ना सुदर्शना राजन्रूपेण च सुदर्शना}


\twolineshloka
{तादृग्रूपा न नारीषु भूतपूर्वा युधिष्ठिर}
{दुर्योधनसुता यादृगभवद्वरवर्णिनी}


\twolineshloka
{तामग्निश्चकमे साक्षाद्राजकन्यां सुदर्शनाम्}
{स्वरूपं दीप्तिमत्कृत्वा शरदर्कसमद्युति}


\twolineshloka
{सा चाग्निशरणे राज्ञः शुश्रूषाकृतनिश्चया}
{नियुक्ता पितृसन्देशादारिराधयिषुः शिखिम्}


\twolineshloka
{तस्या मनोहरं रूपं दृष्ट्वा देवो हुताशनः}
{मन्मथेन समादिष्टः पत्नीत्वे यतते मिथः}


\twolineshloka
{भज मामनवद्याङ्गि कामात्कमललोचने}
{रम्भोरु मृगशावाक्षि पूर्णचन्द्रनिभानने}


\twolineshloka
{तवेदं पद्मपत्राक्षं मुखं दृष्ट्वा मनोहरम्}
{भ्रूलताललितं कान्तमनङ्गो बाधते हि माम्}


\twolineshloka
{ललाटं चन्द्ररेखाभं शिरोरुहविभूषितम्}
{दृष्ट्वा ते पत्रलेखान्तमनङ्गो बाधते भृशम्}


\threelineshloka
{बालातपेन तुलितं सस्वेदपुलकोद्गमम्}
{बिम्बाधरोष्ठवदनं विबुद्धमिव पङ्कजम्}
{अतीव चारुविभ्रान्तं मदमावहते मम}


\twolineshloka
{दन्तप्रकाशनियता वाणी तव सुरक्षिता}
{ताम्रपल्लवसंकाशा जिह्वेयं मे मनोहरा}


\twolineshloka
{समाः स्निग्धाः सुजाताश्च सहिताश्च द्विजास्तव}
{द्विजप्रिये प्रसीदस्व भज मां सुभगा ह्यसि}


\twolineshloka
{मनोज्ञं सुकृतापाङ्गं मुखं तव मनोहरम्}
{स्तनौ ते संहतौ भीरु हाराभरणभूषितौ}


\threelineshloka
{पक्वबिल्वप्रतीकाशौ कर्कशौ सङ्गमक्षमौ}
{गम्भीरनाभिसुभगे भज मां वरवर्णिनि ॥भीष्म उवाच}
{}


\twolineshloka
{सैवमुक्ता विरहिते पावकेन महात्मना}
{ईषदाकम्पहृदया व्रीडिता वाक्यमब्रवीत्}


\twolineshloka
{नलु नाम कुलीनानां कन्यकानां विशेषतः}
{माता पिता प्रभवतः प्रदाने बान्धवाश्च ये}


\twolineshloka
{पाणिग्रहणमन्त्रैश्च हुते चैव विभावसौ}
{सतां मध्ये निविष्टायाः कन्यायाः शरणं पतिः}


\twolineshloka
{साऽहं नात्मवशा देव पितरं वरयस्व मे}
{}


\threelineshloka
{अथ नातिचिराद्राजा कालाद्दुर्योधनस्तदा}
{यज्ञसम्भारनिपुणान्मन्त्रीनाहूय चोक्तवान्}
{यज्ञं यक्ष्येऽहमिति वै सम्भारान्सम्भ्रियन्तु मे}


\twolineshloka
{ततः समाहिते तस्य यज्ञे ब्राह्मणसत्तमैः}
{विप्ररूपी हुतवहो नृपं कन्यामयाचत}


\threelineshloka
{न तु राजा प्रदानाय तस्मै भावमकल्पयत्}
{दरिद्रश्चासवर्णश्च ममायमिति पार्थिवः}
{इति तस्मै न वै कन्यां दित्सां चक्रे नराधिपः}


\twolineshloka
{अथ दीक्षामुपेतस्य यज्ञे तस्य महात्मनः}
{आहितो हवनार्थाय वेद्यामग्निः प्रणश्यत}


\threelineshloka
{ततः स भीतो नृपतिर्भृशं प्रव्यथितेन्द्रियः}
{मन्त्रिणो ब्राह्मणांश्चैव पप्रच्छ किमिदं भवेत्}
{यज्ञे समिद्धो भगवान्नष्टो मे हव्यवाहनः}


\twolineshloka
{सम्मन्त्रकुशलैस्तैस्तैर्ब्राह्मणैर्वेदपारगैः}
{ऋत्विग्भिर्मन्त्रकुशलैर्यज्यतां वा हुताशनः}


\twolineshloka
{अथ ऋक्सामयजुषां ब्राह्मणैर्वेदपारगैः}
{वेदतत्वार्थकुशलैस्ततः सोमपुरस्करैः}


\twolineshloka
{गुह्यैश्च नामभिः सर्वैराहूतो हव्यवाहनः}
{स्वरूपं दीप्तिमत्कृत्वा शरदर्कसमद्युतिः}


\twolineshloka
{ततो महात्मा तानाह दहनो ब्राह्मणर्षभान्}
{वरयाम्यात्मनोऽर्थाय दुर्योधनसुतामिति}


\twolineshloka
{ततस्ते कल्यमुत्थाय तस्मै राज्ञे न्यवेदयन्}
{ब्राह्मणा विस्मिताः सर्वे यदुक्तं चित्रभानुना}


\twolineshloka
{ततः स राजा तच्छ्रुत्वा वचनं ब्रह्मवादिनाम्}
{अवाप्य परमं हर्षं तथेति प्राह बुद्धिमान्}


\twolineshloka
{अयाचत च तं शुल्कं भगवन्तं विभावसुम्}
{नित्यं सान्निध्यमिह ते चित्रभानो भवेदिति}


\threelineshloka
{तमाह भगवानग्निरेवमस्त्विति पार्थिवम्}
{ततः सान्निध्यमद्यापि माहिष्मत्यां विभावसोः ॥दृष्टं हि सहदेवेन दिशं विजयता च तत्}
{}


\twolineshloka
{ततस्तां समलङ्कृत्य कन्यामहतवाससम्}
{ददौ दुर्योधनो राजा पावकाय महात्मने}


\twolineshloka
{प्रतिजग्राह चाग्निस्तु राजकन्यां सुदर्शनाम्}
{विधिना वेददृष्टेन वसोर्धारामिवाध्वरे}


\twolineshloka
{तस्या रूपेण शीलेन कुलेन वपुषा श्रिया}
{अभवत्प्रीतिमानग्निर्गर्भं चास्यां समादधे}


% Check verse!
तस्याः समभवत्पुत्रो नाम्नाऽग्नेयः सुदर्शनः
\twolineshloka
{सुदर्शनस्तु रूपेण पूर्णेन्दुसदृशोपमः}
{शिशुरेवाध्यगात्सर्वं परं ब्रह्म सनातनम्}


\twolineshloka
{अथौघवान्नाम नृपो नृगस्यासीत्पितामहः}
{तस्याप्योघवती कन्या पुत्रश्चौघरथोऽभवत्}


\twolineshloka
{तामोघवान्ददौ तस्मै स्वयमोघवतीं सुताम्}
{सुदर्शनाय विदुषे भार्यार्थे देवरूपिणीम्}


\twolineshloka
{स गृहस्थाश्रमरतस्तया सह सुदर्शनः}
{कुरुक्षेत्रेऽवसद्राजन्नोघवत्या समन्वितः}


\twolineshloka
{गृहस्थश्चावजेष्यामि मृत्युमित्येव स प्रभो}
{प्रतिज्ञामकरोद्धीमान्दीप्ततेजा विशाम्पते}


\twolineshloka
{तामेवौघवतीं राजन्स पावकसुतोऽब्रवीत्}
{अतिथेः प्रतिकूलं ते न कर्तव्यं कथञ्चन}


\twolineshloka
{येनयेन च तुष्येत नित्यमेव त्वयाऽतिथिः}
{अप्यात्मनः प्रदानेन न ते कार्या विचारणा}


\twolineshloka
{एतद्व्रतं मम सदा हृदि सम्परिवर्तते}
{गृहस्थानां च सुश्रोणि नातिथेर्विद्यते परम्}


\twolineshloka
{प्रमाणं यदि वामोरु वचस्ते मम शोभने}
{इदं वचनमव्यग्रा हृदि त्वं धारयेः सदा}


\twolineshloka
{निष्क्रान्ते मयि कल्याणि तथा सन्निहितेऽनघे}
{नातिथिस्तेऽवमन्तव्यः प्रमाणं यद्यहं तव}


\twolineshloka
{तमब्रवीदोघवती तथा मूर्ध्नि कृताञ्जलिः}
{न मे त्वद्वचनात्किञ्चिन्न कर्तव्यं कथञ्चन}


\twolineshloka
{जिगीषमाणस्तु गृहे तदा मृत्युः सुदर्शनम्}
{पृष्ठतोऽन्वगमद्राजन्रन्ध्रान्वेषी तदा सदा}


\twolineshloka
{इध्मार्थं तु गते तस्मिन्नग्रिपुत्रे सुदर्शने}
{अतिधिर्ब्राह्मणः श्रीमांस्तामाहौघवतीं तदा}


\twolineshloka
{आतिथ्यं कृतमिच्छामि त्वयाऽद्य वरवर्णिनि}
{प्रमाणं यदि धर्मस्ते गृहस्थाश्रमसम्मतः}


\twolineshloka
{इत्युक्ता तेन विप्रेणि राजपुत्री यशस्विनी}
{विधिना प्रतिजग्राह वेदोक्तेन विशांपते}


\twolineshloka
{आसनं चैव पाद्यं च तस्मै दत्त्वा द्विजातये}
{प्रोवाचौघवती विप्रं केनार्थः किं ददामि ते}


\twolineshloka
{तामब्रवीत्ततो विप्रो राजपुत्रीं सुदर्शनाम्}
{त्वया ममार्थः कल्याणि निर्विशङ्कैतदाचर}


\twolineshloka
{यदि प्रमाणं धर्मस्ते गृहस्थाश्रमसम्मतः}
{प्रदानेनात्मनो राज्ञि कर्तुमर्हसि मे प्रियम्}


\twolineshloka
{स तया छन्द्यमानोऽन्यैरीप्सितैर्नृपकन्यया}
{नान्यमात्मप्रदानात्स तस्या वव्रे परं द्विजः}


\twolineshloka
{सा तु राजसुता स्मृत्वा भर्तुर्वचनमादितः}
{तथेति लज्जमाना सा तमुवाच द्विजर्षभम्}


\twolineshloka
{ततो विहस्य विप्रर्षिः सा चैवोपविवेश ह}
{संस्मृत्य भर्तुर्वचनं गृहस्थाश्रमकाङ्क्षिणः}


\twolineshloka
{अथेध्मान्समुपादाय स पावकिरुपागमत्}
{मृत्युना रौद्रभावेन नित्यं बन्धुरिवान्वितः}


\twolineshloka
{ततस्त्वाश्रममागम्य स पावकसुतस्तदा}
{तां व्याजहारौघवतीं क्वासि यातेति चासकृत्}


\twolineshloka
{तस्मै प्रतिवचः सा तु भर्त्रे न प्रददौ तदा}
{कराभ्यां तेन विप्रेण स्पृष्टा भर्तृव्रता सती}


\twolineshloka
{उच्छिष्टाऽस्मीति मन्वाना लज्जिता भर्तुरेव च}
{तूष्णींभूताऽभवत्साध्वी विप्रकोपाच्च शङ्किता}


\twolineshloka
{अथ तां किन्विति पुनः प्रोवाच स सुदर्शनः}
{क्व सा साध्वी क्व सायाता गरीय किमतो मम}


\twolineshloka
{पतिव्रता सत्यशीला नित्यं चैवार्जवे रता}
{कथं न प्रत्युदेत्यद्य स्मयमाना यथापुरम्}


\twolineshloka
{उटजस्थस्तु तं विप्रः प्रत्युवाच सुदर्शनम्}
{अतिथिं विद्धि सम्प्राप्तं ब्राह्मणं पावके च माम्}


\twolineshloka
{अनया छन्द्यमानोऽहं भार्यया तव सत्तम}
{तैस्तैरतिथिसत्कारैर्ब्रह्मन्नेषा वृता मया}


\twolineshloka
{आत्मप्रदानविधिना मामर्चति शुभानना}
{अनूरूपं यदत्रान्यत्तद्भवान्कर्तुमर्हति}


\twolineshloka
{कूटमुद्गरहस्तस्तु मृत्युस्तं वै समन्वगात्}
{हीनप्रतिज्ञमत्रैवं वधिष्यामीति चिन्तयन्}


\twolineshloka
{सुदर्शनस्तु मनसा कर्मणा चक्षुषा गिरा}
{त्यक्तेर्ष्यस्त्यक्तमन्युश्च स्मयमानोऽब्रवीदिदम्}


\twolineshloka
{सुरतं तेऽस्तु विप्राग्र्य प्रीतिर्हि परमा मम}
{गृहस्थस्य हि धर्मोऽग्र्यः सम्प्राप्तातिथिपूजनम्}


\twolineshloka
{अतिथिः पूजितो यस्य गृहस्थस्य तु गच्छति}
{नान्यस्तस्मात्परो धर्म इति प्राहुर्मनीषिणः}


\twolineshloka
{प्राणाश्च मम दाराश्च यच्चान्यद्विद्यते वसु}
{अतिथिभ्यो मया देयमिति मे व्रतमाहितम्}


\twolineshloka
{निःसन्दिग्धं यथा वाक्यमेतन्मे समुदाहृतम्}
{तेनाहं विप्र सत्येन स्वयमात्मानमालभे}


\twolineshloka
{पृथिवी वायुराकाशमापो ज्योतिश्च पञ्चमम्}
{बुद्धिरात्मा मनः कालो दिशश्चैव गुणा दश}


\twolineshloka
{नित्यमेव हि पश्यन्ति देहिनां देहसंश्रिताः}
{सुकृतं दुष्कृतं चापि कर्म कर्मवतांवर}


\twolineshloka
{यथैषा नानृता वाणी मयाऽद्य समुदीरिता}
{तेन सत्येन मां देवाः पालयन्तु दहन्तु वा}


\twolineshloka
{ततो नादः समभवद्दिक्षु सर्वासु भारत}
{असकृत्सत्यमित्येवं नैतन्मिथ्येति सर्वतः}


\twolineshloka
{उटजात्तु ततस्तस्मान्निश्चक्राम स वै द्विजः}
{वपुषा द्यां च भूमिं च व्याप्य वायुरिवोद्यतः}


\twolineshloka
{स्वरेण विप्रः शैक्षेण त्रींल्लोकाननुनादयन्}
{उवाच चैनं धर्मज्ञं पूर्वमामन्त्र्य नामतः}


\twolineshloka
{धर्मोऽहमस्मि भद्रं ते जिज्ञासार्थं तवानघ}
{प्राप्तः सत्यं च ते ज्ञात्वा प्रीतिर्मे परमा त्वयि}


\twolineshloka
{विजितश्च त्वया मृत्युर्योऽयं त्वामनुगच्छति}
{रन्ध्रान्वेषी तव सदा त्वया धृत्या वशीकृतः}


\twolineshloka
{न चास्ति शक्तिस्त्रैलोक्ये कस्य चित्पुरुषोत्तम}
{पतिव्रतामिमां साध्वीं तवोद्वीक्षितुमप्युत}


\twolineshloka
{रक्षिता त्वद्गुणैरेषा पातिव्रत्यगुणैस्तथा}
{अधृष्या यदियं ब्रूयात्तथा तन्नान्यथा भवेत्}


\twolineshloka
{एषा हि तपसा स्वेन संयुक्ता ब्रह्मवादिनी}
{पावनार्थं च लोकस्य सरिच्छ्रेष्ठा भविष्यति}


\threelineshloka
{अर्धेनौघवती नाम त्वामर्धेनानुयास्यति}
{शरीरेण महाभागा योगो ह्यस्या वशे स्थितः}
{अनया सह लोकांश्च गन्ताऽसि तपसार्जितान्}


\twolineshloka
{यत्र नावृत्तिमभ्येति शाश्वतांस्ताननुत्तमान्}
{अनेन चैव देहेन लोकांस्त्वमभिपत्स्यसे}


\threelineshloka
{निर्जितश्च त्वया मृत्युरैश्वर्यं च तवोत्तमम्}
{पञ्चभूतान्यतिक्रान्तः स्ववीर्याच्च मनोजवः}
{}


\threelineshloka
{गृहस्थधर्मेणानेन कामक्रोधौ च ते जितौ ॥स्नेहो रागश्च तन्द्री च मोहो द्रोहश्च केवलः}
{तव शुश्रूषया राजन्राजपुत्र्या च निर्जिताः ॥भीष्म उवाच}
{}


\twolineshloka
{शुक्लानां तु सहस्रेण वाजिनां रथमुत्तमम्}
{युक्तं प्रगृह्य भगवान्वासवोप्याजगाम तम्}


\twolineshloka
{मृत्युरात्मा च लोकाश्च जिता भूतानि पञ्च च}
{बुद्धिः कालो मदो मोहः कामक्रोधौ तथैव च}


\twolineshloka
{तस्माद्गृहाश्रमस्थस्य नान्यद्दैवतमस्ति वै}
{ऋतेऽतिथिं नरव्याघ्र मनसैतद्विचारय}


\twolineshloka
{अतिथिः पूजितो यद्धि ध्यायते मनसा शुभम्}
{न तत्क्रतुशतेनापि तुल्यमाहुर्मनीपिण}


% Check verse!
सुदत्तं सुकृतं वाऽपि कम्पयेदप्यनर्चितः
\twolineshloka
{पात्रं त्वतिथिमासाद्य शीलाढ्यं यो न पूजयेत्}
{स दत्त्वा तुष्कृतं तस्मै पुण्यमादाय गच्छति}


\twolineshloka
{एतत्ते कथितं पुत्र मयाऽऽख्यानमनुत्तमम्}
{यथा हि विजितो मृत्युर्गृहस्थेन पुराऽभवत्}


\twolineshloka
{धन्यं यशस्यमायुष्यमिदमाख्यानमुत्तमम्}
{बुभूषताऽभिमन्तव्यं सर्वदुश्चरितापहम्}


\twolineshloka
{इदं यः कथयेद्विद्वानहन्यहनि भारत}
{सुदर्शनस्य चरितं पुण्याँल्लोकानवाप्नुयात्}


\chapter{अध्यायः ३}
\twolineshloka
{प्रज्ञाश्रुताभ्यां वृत्तेन शीलेन च यथा भवान्}
{गुणैश्च विविधैः सर्वैर्वयसा च समन्वितः}


\twolineshloka
{भवान्विशिष्टो बुद्ध्या च प्रज्ञया तपसा तथा}
{`सर्वेषामेव जातानां सतामेतन्न संशयः ॥'}


\twolineshloka
{तस्माद्भवन्तं पृच्छामि धर्मं धर्मभृतांवर}
{नान्यस्त्वदन्यो लोकेषु प्रष्टव्योस्ति नराधिप}


\twolineshloka
{क्षत्रियो यदि वा वैश्यः शूद्रो वा राजसत्तम}
{ब्राह्मण्यं प्राप्नुयाद्येन तन्मे व्याख्यातुमर्हसि}


\fourlineindentedshloka
{`ब्राह्मण्यं यदि दुष्प्रापं त्रिभिर्वर्णैर्नराधिप}
{'तपसा वा सुमहता कर्मणा वा श्रुतेन वा}
{ब्राह्मण्यमथ चेदिच्छेत्कथं शक्यं पितामह ॥भीष्म उवाच}
{}


\twolineshloka
{ब्राह्मण्यमतिदुष्प्राप्यं वर्णैः क्षत्रादिभिस्त्रिभिः}
{परं हि सर्वभूतानां स्थानमेतद्युधिष्ठिर}


\twolineshloka
{बह्विस्तु संसरन्योनीर्जायमानः पुनःपनः}
{पर्याये तात कस्मिंश्चिद्ब्राह्मणो नाम जायते}


\twolineshloka
{अत्राप्युदाहरन्तीममितिहासं पुरातनम्}
{मतङ्गस्य च संवादं गर्दभ्याश्च युधिष्ठिर}


\twolineshloka
{द्विजातेः कस्यचित्तात तुल्यवर्णः सुतस्त्वभूत्}
{मतङ्गो नाम नाम्नाऽऽसीत्सर्वैः समुदितो गुणैः}


\twolineshloka
{स यज्ञकारः कौन्तेय पित्रोत्सृष्टः परंतप}
{प्रायाद्गर्दभयुक्तेन रथेनाप्याशुगामिना}


\twolineshloka
{स बालं गर्दभं राजन्वहन्तं मातुरन्तिके}
{निरविध्यत्प्रतोदेन नासिकायां पुनःपुनः}


\twolineshloka
{तं दृश्य नसि निर्भिन्नं गर्दभी पुत्रगृद्धिनी}
{उवाच मा शुचः पुत्र चण्डालस्त्वाभिविध्यति}


\twolineshloka
{ब्राह्मणो दारुणो नास्ति मैत्रो ब्राह्मण उच्यते}
{आचार्यः सर्वभूतानां शास्ता किं प्रहरिष्यति}


\twolineshloka
{अयं तु पापप्रकृतिर्बाले न कुरुते दयाम्}
{स्वयोनिं मानयत्येष भावो भावं नियच्छति}


\twolineshloka
{एतच्छ्रुत्वा मतङ्गस्तु दारुणं रासभीवचः}
{अवतीर्य रथात्तूर्णं रासभीं प्रत्यभाषत}


\fourlineindentedshloka
{ब्रूहि रासभि कल्याणि माता मे येन दूषिता}
{केन मां वेत्सि चण्डालं ब्राह्मण्यं केन मेऽनशत्}
{तत्त्वेनैतन्महाप्राज्ञे ब्रूहि सर्वमशेषतः ॥गर्दभ्युवाच}
{}


\twolineshloka
{ब्राह्मण्यां वृषलेन त्वं मत्तायां नापितेन ह}
{जातस्त्वमसि चण्डालो ब्राह्मण्यं तेन तेऽनशत्}


\twolineshloka
{एवमुक्तो मतङ्गस्तु प्रतिप्रायाद्गृहं पुनः}
{तमागतमभिप्रेक्ष्य पिता वाक्यमथाब्रवीत्}


\threelineshloka
{यस्त्वं यज्ञार्थसंसिद्धौ नियुक्तो गुरुकर्मणि}
{कस्मात्प्रतिनिवृत्तोसि कच्चिन्न कुशलं तव ॥मतङ्ग उवाच}
{}


\twolineshloka
{अन्त्ययोनिरयोनिर्वा कथं स कुशली भवेत्}
{कुशलं तु कुतस्तस्य यस्येयं जननी पितः}


\twolineshloka
{ब्राह्मण्यां वृषलाञ्जातं पितर्वेदयते हि माम्}
{अमानुषी गर्दभीयं तस्मात्तप्स्ये तपो महत्}


\twolineshloka
{एवमुक्त्वा स पितरं प्रतस्थे कृतनिश्चयः}
{ततो गत्वा महारण्यमतपत्सुमहत्तपः}


\twolineshloka
{ततः स तापयामास विबुधांस्तपसाऽन्वितः}
{मतङ्गः सुसुखं प्रेप्सुः स्थानं सुचरितादपि}


\twolineshloka
{तं तथा तपसा युक्तमुवाच हरिवाहनः}
{मतङ्ग तप्स्यसे किं त्वं भोगानुत्सृज्य मानुषान्}


\threelineshloka
{वरं ददामि ते हन्त वृणीष्व त्वं यदिच्छसि}
{यच्चाप्यवाप्यं हृदि ते सर्वं तद्ब्रूहि मा चिरम् ॥मतङ्ग उवाच}
{}


\threelineshloka
{ब्राह्मण्यं कामयानोऽहमिदमारब्धवांस्तपः}
{गच्छेयं तदवाप्येह वर एष वृतो मया ॥भीष्म उवाच}
{}


\twolineshloka
{एतच्छ्रुत्वा तु वचनं तमुवाच पुरदरः}
{मतङ्ग दुर्लभमिदं विप्रत्वं प्रार्थ्यते त्वया}


\twolineshloka
{ब्राह्मण्यं प्रार्थयानस्त्वमप्राप्यमकृतात्मभिः}
{विनशिष्यसि दुर्बुद्धे तदुपारम मा चिरम्}


\twolineshloka
{श्रेष्ठं यत्सर्वभूतेषु तपो यदतिवर्तते}
{तदग्र्यं प्रार्थयानस्त्वमचिराद्विनशिष्यसि}


\twolineshloka
{देवतासुरमर्त्येषु यत्पवित्रं परं स्मृतम्}
{चण्डालयोनौ जातेन न तत्प्राप्यं कथञ्चन}


\chapter{अध्यायः ४}
\twolineshloka
{एवमुक्तो मतङ्गस्तु संशितात्मा यतव्रतः}
{अतिष्ठदेकपादेन वर्षाणां शतमच्युतः}


\twolineshloka
{तमुवाच ततः शक्रः पुनरेव महायशाः}
{ब्राह्मण्यं दुर्लभं तात प्रार्थयानो न लप्स्यसे}


\twolineshloka
{मतङ्ग परमं स्थानं प्रार्थयन्विनशिष्यसि}
{मा कृथाः साहसं पुत्र नैष धर्मपथस्तव}


\threelineshloka
{`विमार्गतो मार्गमाणः सर्वथा नभविष्यसि'}
{न हि शक्यं त्वया प्राप्तुं ब्राह्मण्यमिह दुर्मते}
{अप्राप्यं प्रार्थयानो हि नचिराद्विनशिष्यसि}


\twolineshloka
{मतङ्ग परमं स्थानं वार्यमाणोऽसकृन्मया}
{चिकीर्षस्येव तपसा सर्वथा नभविष्यसि}


\twolineshloka
{तिर्यग्योनिगतः सर्वो मानुष्यं यदि गच्छति}
{स जायते पुल्कसो वा चण्डालो वाऽप्यसंशयः}


\twolineshloka
{पुल्कसः पापयोनिर्वा यः कश्चिदिह लक्ष्यते}
{स तस्यामेव सुचिरं मतङ्ग परिवर्तते}


\twolineshloka
{ततो दशशते काले लभते शूद्रतामपि}
{शूद्रयोनावपि ततो बहुशः परिवर्तते}


\threelineshloka
{ततस्त्रिंशद्गणे काले लभते वैश्यतामपि}
{वैश्यतायां चिरं कालं तत्रैव परिवर्तते}
{ततः षष्टिगुणे काले राजन्यो नाम जायते}


\twolineshloka
{ततः षष्टिगुणे काले लभते ब्रह्मबन्धुताम्}
{ब्रह्मबन्धुश्चिरं कालं ततस्तु परिवर्तते}


\twolineshloka
{ततस्तु द्विशते काले लभते काण्डपृष्ठताम्}
{काण्डपृष्ठश्चिरं कालं तत्रैव परिवर्तते}


\twolineshloka
{ततस्तु त्रिशते काले लभते जपतामपि}
{तं च प्राप्य चिरं कालं तत्रैव परिवर्तते}


\twolineshloka
{ततश्चतुःशते काले श्रोत्रियो नाम जायते}
{श्रोत्रियत्वे चिरं कालं तत्रैव परिवर्तते}


\twolineshloka
{तदेवं शोकहर्षौ तु कामद्वेषौ च पुत्रक}
{अतिमानातिवादौ च प्रविशेते द्विजाधमम्}


\twolineshloka
{तांश्चेज्जयति शत्रून्स तदा प्राप्नोति सद्गतिम्}
{अथ ते वै जयन्त्येनं तालं पक्वमिवापतेत्}


\twolineshloka
{मतङ्ग सम्प्रधार्यैवं यदहं त्वामचूचुदम्}
{वृणीष्व काममन्यं त्वं ब्राह्मण्यं हि सुदुर्लभम्}


\chapter{अध्यायः ५}
\twolineshloka
{एवमुक्तो मतङ्गस्तु संशितात्मा यतव्रतः}
{सहस्रमेकपादेन ततो ध्याने व्यतिष्ठत}


\threelineshloka
{तं सहस्रावरे काले शक्रो द्रष्टुमुपागमत्}
{तदेव च पुनर्वाक्यमुवाच बलवृत्रहा ॥मतङ्ग उवाच}
{}


\threelineshloka
{इदं वर्षसहस्रं वै ब्रह्मचारी समाहितः}
{अतिष्ठमेकपादेन ब्राह्मण्यं नाप्नुयां कथम् ॥शक्र उवाच}
{}


\twolineshloka
{चण्डालयोनो जातेन नावाप्यं वै कथञ्चन}
{अन्यं कामं वृणीष्व त्वं मा वृथा तेऽस्त्वयं श्रमः}


\twolineshloka
{एवमुक्तो मतङ्गस्तु भृशं शोकपरायणः}
{अध्यतिष्ठद्गयां गत्वा सोऽङ्गुष्टेन शतं समाः}


\twolineshloka
{सुदुर्वहं वहन्योगं कृशो धमनिसन्ततः}
{त्वगस्थिभूतो धर्मात्मा स ततापेति नः श्रुतम्}


\threelineshloka
{तं तपन्तमभिद्रुत्य पाणिं जग्राह वासवः}
{वराणामीश्वरो दाता सर्वभूतहिते रतः ॥शक्र उवाच}
{}


\twolineshloka
{मतङ्ग ब्राह्मणत्वं ते विरुद्धमिह दृश्यते}
{ब्राह्मण्यं दुर्लभतरं संवृतं परिपन्थिभिः}


\twolineshloka
{पूजयन्सुखमाप्नोति दुःखमाप्नोत्यपूजयन्}
{ब्राह्मणे सर्वभूतानां योगक्षेमः समाहितः}


\twolineshloka
{ब्राह्मणेभ्योऽनुतृप्यन्ते पितरो देवतास्तथा}
{ब्राह्मणः सर्वभूतानां मतङ्ग पर उच्यते}


% Check verse!
ब्राह्मणः कुरुते तद्धि यथा यद्यच्च वाञ्छति
\twolineshloka
{बह्वीस्तु संसरन्योनीर्जायमानः पनुःपुनः}
{पर्याये तात कस्मिंश्चिद्ब्राह्मण्यमिह विन्दति}


\twolineshloka
{तदुत्सृज्येह दुष्प्रापं ब्राह्मण्यमकृतात्मभिः}
{अन्यं वरं वृणीष्व त्वं दुर्लभोऽयं हि ते वरः}


\twolineshloka
{किं मां तुदसि दुःखार्तं मृतं मारयसे च माम्}
{त्वां तु शोचामि यो लब्ध्वा ब्राह्मण्यं न बुभूषसे}


\twolineshloka
{ब्राह्मण्यं यदि दुष्प्रापं त्रिभिर्वर्णैर्दुरासदम्}
{सुदुर्लभं सदावाप्य नानुतिष्ठन्ति मानवाः}


\twolineshloka
{ये पापेभ्यः पापतमास्तेषामधम एव सः}
{ब्राह्मण्यं योऽवजानीते धनं लब्ध्वेव दुर्लभम्}


\twolineshloka
{दुष्प्रापं खलु विप्रत्वं प्राप्तं दुरनुपालनम्}
{दुरवापमवाप्यैतन्नानुतिष्ठन्ति मानवाः}


\twolineshloka
{एकारामो ह्यहं शक्र निर्न्द्वद्वो निष्परिग्रहः}
{अहिंसादममास्थाय कथं नार्हामि विप्रताम्}


\twolineshloka
{दैवं तु कथमेतद्वै यदहं मातृदोषतः}
{एतामवस्थां सम्प्राप्तो धर्मज्ञः सन्पुरन्दरं}


\twolineshloka
{नूनं दैवं न शक्यं हि पौरुषेणातिवर्तितुम्}
{यदर्थं यत्नवानेव न लभे विप्रतां विभो}


\threelineshloka
{एवंगते तु धर्मज्ञ दातुमर्हसि मे वरम्}
{यदि तेऽहमनुग्राह्यः किचिद्वा सुकृतं मम ॥वैशम्पायन उवाच}
{}


\twolineshloka
{वृणीष्वेति तदा प्राह ततस्तं बलवृत्रहा}
{चोदितस्तु महेन्द्रेण मतङ्गः प्राब्रवीदिदम्}


\twolineshloka
{यथा कामविहारी स्यां कामरूपी विहङ्गमः}
{ब्रह्मक्षत्राविरोधेन पूजां च प्राप्नुयामहम्}


\threelineshloka
{यथा ममाक्षया कीर्तिर्भवेच्चापि पुरन्दर}
{कर्तुमर्हसि तद्देव शिरसा त्वां प्रसादये ॥शक्र उवाच}
{}


\twolineshloka
{मतङ्ग गम्यतां शीघ्रमेवमेतद्भविष्यति}
{स्त्रियः सर्वास्त्वया लोके यक्ष्यन्ते भूतिकर्मणि}


\twolineshloka
{छन्दोदेव इति ख्यातः स्त्रीणां पूज्यो भविष्यसि}
{कीर्तिश्च तेऽतुला वत्स त्रिषु लोकेषु यास्यति}


\twolineshloka
{एवं तस्मै वरं दत्त्वा वासवोऽन्तरधीयत}
{प्राणांस्त्यक्त्वा मतङ्गोपि प्राप तत्स्थानमुत्तमम्}


\twolineshloka
{एवमेव परं स्थानं मर्त्यानां भरतर्षभ}
{ब्राह्मण्यं नाम दुष्प्रापमिन्द्रेणोक्तं महात्मना}


\chapter{अध्यायः ६}
\twolineshloka
{ब्राह्मण्यं यदि दुष्प्राप्यं त्रिभिर्वर्णैर्नराधिप}
{कथं प्राप्तं महाराज क्षत्रियेण महात्मना}


\twolineshloka
{विश्वामित्रेण धर्मात्मन्ब्राह्मणत्वं नरर्षभ}
{श्रोतुमिच्छामि तत्त्वेन तन्मे ब्रूहि पितामह}


\twolineshloka
{तेन ह्यमितवीर्येण वसिष्ठस्य महात्मनः}
{हतं पुत्रशतं सद्यस्तपसाऽपि पितामह}


\twolineshloka
{यातुधानाश्च बहवो राक्षसास्तिग्मतेजसः}
{मन्युनाऽऽविष्टदेहेन सृष्टाः कालान्तकोपमाः}


\twolineshloka
{महान्कुशिकवंशश्च ब्रह्मर्षिशतसंकुलः}
{स्थापितो नरलोकेऽस्मिन्विद्वद्ब्राह्मणसंकुलः}


\twolineshloka
{ऋतीकस्यात्मजश्चैव शुनःशेपो महातपाः}
{विमोक्षितो महासत्रात्पशुतामप्युपागतः}


\twolineshloka
{हरिश्चन्द्रक्रतौ देवांस्तोषयित्वाऽऽत्मतेजसा}
{पुत्रतामनुसंप्राप्तो विश्वामित्रस्य धीमतः}


\twolineshloka
{नाभिवादयते ज्येष्ठं देवरातं नराधिप}
{पुत्राः पञ्चशतं चापि शप्ताः श्वपचतां गताः}


\twolineshloka
{त्रिशङ्कुर्बन्धुभिर्मुक्त ऐक्ष्वाकः प्रीतिपूर्वकम्}
{अवाक्शिरादिवं नीतो दक्षिणामाश्रितोदिशम्}


\twolineshloka
{विश्वामित्रस्य भगिनी नदी देवर्षिसेविता}
{कौशिकीति कृता पुण्या ब्रह्मर्षिसुरसेविता}


\twolineshloka
{तपोविघ्नकरी चैव पञ्चचूडा सुसंमता}
{रम्भा नामाप्सराः शापाद्यस्य शैलत्वमागता}


\twolineshloka
{तथैवास्य भयाद्बद्ध्वा वसिष्ठः सलिले पुरा}
{आत्मानं मञ्जयञ्श्रीमान्विपाशः पुनरुत्थितः}


\twolineshloka
{तदाप्रभृति पुण्या हि विपाशाऽभून्महानदी}
{विख्याता कर्मणा तेन वसिष्ठस्य महात्मनः}


\twolineshloka
{वश्यश्च भगवान्येन देवसेनाग्रगः प्रभुः}
{स्तुतः प्रीतमनाश्चासीच्छापाच्चैनममुञ्चत}


\twolineshloka
{ध्रुवस्यौत्तानपादस्य ब्रह्मर्षीणां तथैव च}
{मध्ये ज्वलति यो नित्यमुदीचीमाश्रितो दिशम्}


\twolineshloka
{तस्यैतानि च कर्माणि तथाऽन्यानि च कौरव}
{क्षत्रियस्येत्यतो जातमिदं कौहतूलं मम}


\twolineshloka
{किमेतदिति तत्त्वेन प्रब्रूहि भरतर्षभ}
{देहान्तरमनासाद्य कथं स ब्राह्मणोऽभवत्}


\twolineshloka
{एतत्तत्वेन मे तात सर्वमाख्यातुमर्हसि}
{मतङ्गस्य यथातत्त्वं तथैवैतद्वदस्व मे}


\twolineshloka
{स्थाने मतङ्गो ब्राह्मण्यं नालभद्भरतर्षभ}
{चण्डालयोनौजातो हि कथं ब्राह्मण्यमाप्तवान्}


\chapter{अध्यायः ७}
\twolineshloka
{श्रूयतां पार्थ तत्त्वेन विश्वामित्रो यथा पुरा}
{ब्राह्मणत्वं गतस्तात् ब्रह्मर्षित्वं तथैव च}


\twolineshloka
{भरतस्यान्ववाये वै मिथिलो नाम पार्थिवः}
{बभूव भरतश्रेष्ठ यज्वा धर्मभृतांवरः}


\twolineshloka
{तस्य पुत्रो महानासीञ्जह्नुर्नाम नरेश्वरः}
{दुहितृत्वमनुप्राप्ता गङ्गा यस्य महात्मनः}


\twolineshloka
{तस्यात्मजस्तुल्यगुणः सिन्धुद्वीपो महायशाः}
{सिन्धुद्वीपाच्च राजर्षिर्बलाकाश्वो महाबलः}


\twolineshloka
{वल्लभस्तस्य तनयः साक्षाद्धर्म इवापरः}
{कुशिकस्तस्य तनयः सहस्राक्षसमद्युतिः}


\twolineshloka
{कुशिकस्यात्मजः श्रीमान्गाधिर्नाम जनेश्वरः}
{अपुत्रः प्रसवेनार्थी वनवासमुपावसत्}


\twolineshloka
{कन्या जज्ञे सुतात्तस्य वने निवसतः सतः}
{नाम्ना सत्यवती नाम रूपेणाप्रतिमा भुवि}


\twolineshloka
{तां वव्रे भार्गवः श्रीमांश्च्यवनस्यात्मसम्भवः}
{ऋचीक इति विख्यातो विपुले तपसि स्थितः}


\twolineshloka
{स तां न प्रददौ तस्मै ऋचीकाय महात्मने}
{दरिद्र इति मत्वा वै गाधिः शत्रुनिबर्हणः}


\threelineshloka
{प्रत्याख्याय पुनर्यान्तमब्रवीद्राजसत्तप्रः}
{शुल्कं प्रदीयतां मह्यं ततो वत्स्यसि मे सुताम् ॥ऋचीक उवाच}
{}


\threelineshloka
{किं प्रयच्छामि राजेन्द्र तुभ्यं शुल्कमहं नृप}
{दुहितुर्ब्रूह्यसंसक्तो माऽभूत्तत्र विचारणा ॥गाधिरुवाच}
{}


\threelineshloka
{चन्द्ररश्मिप्रकाशानां हयानां वातरहसाम्}
{एकतः श्यामकर्णानां सहस्रं दाह भार्गव ॥भीष्म उवाच}
{}


\twolineshloka
{ततः स भृगुशार्दूलश्च्यवनस्यात्मजः प्रभुः}
{अब्रवीद्वरुणं देवमादित्यं पतिमम्भसाम्}


\twolineshloka
{एकतः श्यामकर्णानां हयानां चन्द्रवर्चसाम्}
{सहस्रं वातवेगानां भिक्षे त्वां देवसत्तम}


\twolineshloka
{तथेति वरुणो देव आदित्यो भृगुसत्तमम्}
{उवाच यत्र ते च्छन्दस्तत्रोत्थास्यन्ति वाजिनः}


\twolineshloka
{ध्यातमात्रे ऋचीकेन हयानां चन्द्रवर्चसाम्}
{गङ्गाजलात्समुत्तस्थौ सहस्रं विपुलौजसाम्}


\twolineshloka
{अदूरे कान्यकुब्जस्य गङ्गायास्तीरमुत्तमम्}
{अश्वतीर्थं तदद्यापि मानवाः परिचक्षते}


\twolineshloka
{ततो वै गाधये तात सहस्रं वाजिनां शुभम्}
{ऋचीकः प्रददौ प्रीतः शुल्कार्थं तपतां वरः}


\twolineshloka
{ततः स विस्मितो राजा गाधिः शापभयेन च}
{ददौ तां समलङ्कृत्य कन्यां भृगुसुताय वै}


\twolineshloka
{जग्राह विधिवत्पाणिं तस्या ब्रह्मर्षिसत्तमः}
{सा च तं पतिमासाद्य परं हर्षमवाप ह}


\twolineshloka
{स तुतोष च ब्रह्मर्षिस्तस्या वृत्तेन भारत}
{छन्दयामास चैवैनां वरेण वरवर्णिनीम्}


\twolineshloka
{मात्रे तत्सर्वमाचख्यौ सा कन्या राजसत्तम}
{अथतामब्रवीन्माता सुतां किञ्चिदवाङ्मुखीम्}


\twolineshloka
{ममापि पुत्रि भर्ता ते प्रसादं कर्तुमर्हति}
{अपत्यस्य प्रदानेन समर्थश्च महातपाः}


\twolineshloka
{ततः सा त्वरितं गत्वा तत्सर्वं प्रत्यवेदयत्}
{मातुश्चिकीर्षितं राजन्ऋचीकस्तामथाब्रवीत्}


\twolineshloka
{गुणवन्तं च पुत्रं वै त्वं च साऽथ जनिष्यथ}
{जनन्यास्तव कल्याणि मा भूद्वै प्रणयोऽन्यथा}


\twolineshloka
{तव चैव गुणश्लाघी पुत्र उत्पत्स्यते महान्}
{अस्मद्वंशकरः श्रीमांस्तव भ्राता च वंशकृत्}


\twolineshloka
{ऋतुस्नाता च साऽश्वत्थं त्वं च वृक्षमुदुम्बरम्}
{परिष्वजेतं कल्याणि तत इष्टमवाप्स्यथः}


\twolineshloka
{चरुद्वयमिदं चैव मन्त्रपूतं शुचिस्मिते}
{त्वं च सा चोपभुञ्जीतं ततः पुत्राववाप्स्यथः}


\twolineshloka
{ततः सत्यवती हृष्टा मातरं प्रत्यभाषत}
{यदृचीकेन कथितं तच्चाचख्यौ चरुद्वयम्}


\twolineshloka
{तामुवाच ततो माता सुतां सत्यवतीं तदा}
{पुत्रि पूर्वोपपन्नायाः कुरुष्व वचनं मम}


\twolineshloka
{भर्त्रा य एष दत्तस्ते चरुर्मन्त्रपुरस्कृतः}
{एनं प्रयच्छ मह्यं त्वं मदीयं त्वं गृहाण च}


\twolineshloka
{व्यत्यासं वृक्षयोश्चापि करवाव शुचिस्मिते}
{यदि प्रमाणं वचनं मम मातुरनिन्दिते}


\twolineshloka
{स्वमपत्यं विशिष्टं हि सर्व इच्छत्यनाविलम्}
{व्यक्तं भगवता चात्र कृतमेवं भविष्यति}


\twolineshloka
{ततो मे त्वच्चरौ भावः पादपे च समुध्यमे}
{कथं विशिष्टो भ्राता मे भवेदित्येव चिन्तय}


\twolineshloka
{तथाच कृतवत्यौ ते माता सत्यवती च सा}
{अथ गर्भावनुप्राप्ते उभे ते वै युधिष्ठिर}


\twolineshloka
{दृष्ट्वा गर्भमनुप्राप्तां भार्यां स च महानृषिः}
{उवाच तां सत्यवतीं दुर्मना भृगुसत्तमः}


\twolineshloka
{व्यत्यासेनोपयुक्तस्ते चरुर्व्यक्तं भविष्यति}
{व्यत्यासः पादपे चापि सुव्यक्तं ते कृतः शुभे}


\twolineshloka
{मया हि विश्वं यद्ब्राह्म त्वच्चरौ सन्निवेशितम्}
{क्षत्रवीर्यं च सकलं चरौ तस्या निवेशितम्}


\twolineshloka
{त्रैलोक्यविख्यातगुणं त्वं विप्रं जनयिष्यसि}
{सा च क्षत्रं विशिष्टं वै तत एतत्कृतं मया}


\twolineshloka
{व्यत्यासस्तु कृतो यस्मात्त्वया मात्रा च ते शुभे}
{तस्मात्सा ब्राह्मणं श्रेष्ठं माता ते जनयिष्यति}


\twolineshloka
{क्षत्रियं तूग्रकर्माणं त्वं भद्रे जनयिष्यसि}
{न हि ते तत्कृतं साधु मातृस्नेहेन भामिनि}


\twolineshloka
{सा श्रुत्वा शोकसंतप्ता पपात वरवर्णिनी}
{भूमौ सत्यवती राजंश्छिन्नेव रुचिरा लता}


\twolineshloka
{प्रतिलभ्य च सा सञ्ज्ञां शिरसा प्रणिपत्य च}
{उवाच भार्या भर्तारं गाधेयी भार्गवर्षभम्}


\twolineshloka
{प्रसादयन्त्यां भार्यायां मयि ब्रह्मविदांवर}
{प्रसादं कुरु विप्रर्षे न मे स्यात्त्रत्रियः सुतः}


\twolineshloka
{कामं ममोग्रकर्मा वै पौत्रो भवितुमर्हति}
{न तु मे स्यात्सुतो ब्रह्मन्नेष मे दीयतां वरः}


\twolineshloka
{एवमस्त्विति होवाच स्वां भार्यां सुमहातपाः}
{ततः सा जनयामास जमदग्निं सुतं शुभम्}


\twolineshloka
{विश्वामित्रं चाजनयद्गाधिभार्या यशस्विनी}
{ऋषेः प्रसादाद्राजेन्द्र ब्रह्मर्षि ब्रह्मवादिनम्}


\twolineshloka
{ततो ब्राह्मणतां यातो विश्वामित्रो महातपाः}
{क्षत्रियः सोऽप्यथ तथा ब्रह्मवंशस्य कारकः}


\twolineshloka
{तस्य पुत्रा महात्मानो ब्रह्मवंशविवर्धनाः}
{तपस्विनो ब्रह्मविदो गोत्रकर्तार एव च}


\twolineshloka
{मधुच्छन्दश्च भगवान्देवरातश्च वीर्यवान्}
{अक्षीणश्च शकुन्तश्च बभ्रुः कालपथस्तथा}


\twolineshloka
{याज्ञवल्क्यश्च विख्यातस्तथा स्थूणो महाव्रतः}
{उलूको यमदूतश्च तथर्षिः सैन्धवायनः}


\twolineshloka
{पर्णजङ्घश्च भगवान्गावलश्च महानृषिः}
{ऋषिर्वज्रस्तथा ख्यातः सालङ्कायन एव च}


\threelineshloka
{लीलाढ्यो नारदश्चैव तथा कूर्चामुखः स्मृतः}
{वादुलिर्मुसलश्चैव वक्षोग्रीवस्तथैव च}
{}


\twolineshloka
{आङ्घ्रिको नैकदृक्चैव शिलायूपः सितः शुचिः}
{चक्रको मारुतन्तव्यो वातघ्नोऽथाश्वलायनः}


\twolineshloka
{श्यामायनोऽथ गार्ग्यश्च जाबालिः सुश्रुतस्तथा}
{कारीषिरथ संश्रुत्यः परपौरवतन्तवः}


\twolineshloka
{महानृषिश्च कपिलस्तथर्षिस्ताडकायनः}
{तथैव चोपगहनस्तथर्षिश्चासुरायणः}


\twolineshloka
{मार्दमर्षिर्हिरण्याक्षो जङ्गारिर्बाभ्रवायणिः}
{भूतिर्विभूतिः सूतश्च सुरकृत्तु तथैव च}


\twolineshloka
{अरालिर्नाचिकश्चैव चाम्पेयोज्जयनौ तथा}
{नवतन्तुर्बकनखः सेयनो यतिरेव च}


\threelineshloka
{अम्भोरुदश्चारुमत्स्यः शिरीषी चाथ गार्दभिः}
{ऊर्जयोनिरुदापेक्षी नारदी च महानृषिः}
{विश्वामित्रात्मजाः सर्वे मुनयो ब्रह्मवादिनः}


\twolineshloka
{तथैव क्षत्रियो राजन्विश्वामित्रो महातपाः}
{ऋचीकेनाहितं ब्रह्म परमेतद्युधिष्ठिर}


\twolineshloka
{एतत्ते सर्वमाख्यातं तत्वेन भरतर्षभ}
{विश्वामित्रस्य वै जन्म सोमसूर्याग्नितेजसः}


\twolineshloka
{यत्रयत्र च सन्देहो भूयस्ते राजसत्तम}
{तत्रतत्र च मां ब्रूहि च्छेत्ताऽस्मि तव संशयान्}


\chapter{अध्यायः ८}
\twolineshloka
{श्रुतं मे महदाख्यानमेतत्कुरुकुलोद्वह}
{सुदुष्प्रापं यद्ब्रवीषि ब्राह्मण्यं वदतांवर}


\twolineshloka
{विश्वामिइत्रो महाराज राजा ब्राह्मणतां गतः}
{कथितं भवता सर्वं विस्तरेण पितामह}


\twolineshloka
{तच्च राजन्मया सर्वं श्रुतं बुद्धिमतांवर}
{आगमो हि परोऽस्माकं त्वत्तः कौरवनन्दन}


\twolineshloka
{वीतहव्यश्च नृपतिः श्रुतो मे विप्रतां गतः}
{तदेव तावद्गाङ्गेय श्रोतुमिच्छाम्यहं विभो}


\threelineshloka
{स केन कर्मणा प्राप्तो ब्राह्मण्यं राजसत्तमः}
{वरेण तपसा वाऽपि तन्मे व्याख्यातुमर्हसि ॥भीष्म उवाच}
{}


\twolineshloka
{शृणु राजन्यथा राजा वीतहव्यो महायशाः}
{राजर्षिर्दुर्लभं प्राप्तो ब्राह्मण्यं लोकसत्कृतम्}


\twolineshloka
{मनोर्महात्मनस्तात प्रजा धर्मेण शासतः}
{बभूव पुत्रो धर्मात्मा शर्यातिरिति विश्रुतः}


\twolineshloka
{तस्यान्ववाये द्वौ राजन्राजानौ सम्बभूवतुः}
{हैहयस्तालजङ्घश्च वत्सेषु जयतांवर}


\threelineshloka
{हैहयस्य तु राजेन्द्र दशसु स्त्रीषु भारत}
{शतं बभूव पुत्राणां शूराणामनिवर्तिनाम्}
{तुल्यरूपप्रभावानां बलिनां युद्धशालिनाम्}


% Check verse!
धनुर्वेदे च वेदे च सर्वत्रैव कृतश्रमाः
\twolineshloka
{काशिष्वपि नृपो राजन्दिवोदासपितामहः}
{हर्यश्व इति विख्यातो बभूव जयतांवरः}


\twolineshloka
{स वीतहव्यदायादैरागत्य पुरुषर्षभ}
{गङ्गायमुनयोर्मध्ये सङ्ग्रामे विनिपातितः}


\twolineshloka
{तं तु हत्वा नरपतिं हैहयास्ते महारथाः}
{प्रतिजग्मुः पुरीं रम्यां वत्सानामकुतोभयाः}


\twolineshloka
{हर्यश्वस्य च दायादः काशिराजोऽभ्यषिच्यत}
{सुदेवो देवसंकाशः साक्षाद्धर्म इवापरः}


\twolineshloka
{स पालयामास महीं धर्मात्मा काशिनन्दनः}
{तैर्वीतहव्यैरागत्य युधि सर्वैर्विनिर्जितः}


\twolineshloka
{तमथाजौ विनिर्जित्य प्रतिजग्मुर्यथागतम्}
{सौदेविस्त्वथ काशीशोदिवोदासोऽभ्यषिच्यत}


\twolineshloka
{दिवोदासस्तु विज्ञाय वीर्य तेषां यतात्मनाम्}
{वाराणसीं महातेजा निर्ममे शक्रशासनात्}


\twolineshloka
{विप्रक्षत्रियसम्बाधां वैश्यशूद्रसमाकुलाम्}
{नैकद्रव्योच्चयवतीं समृद्धविपणापणाम्}


\twolineshloka
{गङ्गाया उत्तरे कूले वप्रान्ते राजसत्तम् ॥गोमत्या दक्षिणे कूले शक्रस्येवामरावतीम्}
{}


\twolineshloka
{तत्र तं राजशार्दूलं निवसन्तं महीपतिम्}
{आगत्य हैहया भूयः पर्यधावन्त भारत}


\twolineshloka
{स निष्क्रम्य ददौ युद्धं तेभ्यो राजा महाबलः}
{देवासुरसमं घोरं दिवोदासो महाद्युतिः}


\twolineshloka
{स तु युद्धे महाराज दिनानां दशतीर्दश}
{हतवाहनभूयिष्ठस्ततो दैन्यमुपागमत्}


\twolineshloka
{हतयोधस्ततो राजन्क्षीणकोशश्चक भूमिपः}
{दिवोदासः पुरीं त्यक्त्वा पलायनपरोऽभवत्}


\twolineshloka
{गत्वाऽऽश्रमपदं रम्यं भरद्वाजस्य धीमतः}
{जगाम शरणं राजा कृताञ्जलिररिंदम्}


\twolineshloka
{तमुवाच भरद्वाजो ज्येष्ठः पुत्रो बृहस्पतेः}
{पुरोधाः शीलसम्पन्नो दिवोदासं महीपतिम्}


\threelineshloka
{किमागमनकृत्यं ते सर्वं प्रबूहि मे नृप}
{यत्ते प्रियं तत्करिष्ये न मेऽत्रास्ति विचारणा ॥राजोवाच}
{}


\twolineshloka
{भगवन्वैतहव्यैर्मे युद्धे वंशः प्रणाशितः}
{अहमेकः परिद्यूनो भवन्तं शरणं गतः}


\twolineshloka
{शिष्यस्नेहेन भगवंस्त्वं मां रक्षितुमर्हसि}
{एकशेषः कृतो वंशो मम तैः पापकर्मभिः}


\twolineshloka
{तमुवाच महाभागो भरद्वाजः प्रतापवान्}
{न भेतव्यं न भेतव्यं सौदेव व्येतु ते भयम्}


\twolineshloka
{अहमिष्टिं करिष्यामि पुत्रार्थं ते विशाम्पते}
{वीतहव्यसहस्राणि येन त्वं प्रहरिष्यसि}


\twolineshloka
{तत इष्टिं चकारर्षिस्तस्य वै पुत्रकामिकीम्}
{अथास्य तनयो जज्ञे दैवोदासः प्रतर्दनः}


\twolineshloka
{स जातमात्रो ववृधे समाः सद्यस्त्रयोदश}
{वेदं चापि जगौ कृत्स्नं धनुर्वेदं च भारत}


\twolineshloka
{योगेन च *****विष्टो भरद्वाजेन धीमता}
{कृत्स्नं हि तेजो यल्लोके तदेतद्देहमाविशत्}


\twolineshloka
{ततः स कवची धन्वी स्तूयमानः सुरर्षिभिः}
{बन्दिभिर्वन्द्यमानश्च बभौ सूर्य इवोदितः}


\twolineshloka
{स रथी बद्धनिस्त्रिंशो बभौ दीप्त इवानलः}
{प्रययौ स धनुर्धुन्वन्विवर्षिषुरिवाम्बुदः}


\twolineshloka
{तं दृष्ट्वा परमं हर्षं सुदेवतनयो ययौ}
{मेने च मनसा दग्धान्वैतहव्यान्स पार्थिवः}


\twolineshloka
{ततोसौ यौवराज्ये च स्थापयित्वा प्रतर्दनम्}
{कृतकृत्यं तदाऽऽत्मानं स राजा प्रत्यपद्यत}


\twolineshloka
{ततस्तु वैतहव्यानां वधाय स महीपतिः}
{पुत्रं प्रस्थापयासास प्रतर्दनमरिंदमम्}


\twolineshloka
{सरथः स तु संतीर्य गङ्गामाशु पराक्रमी}
{प्रययौ वीतहव्यानां पुरीं परपुरंजयः}


\twolineshloka
{वैतहव्यास्तु संश्रुत्य रथघोषं समुद्धतम्}
{निर्ययुर्नगराकारै रथैः पररथारुजैः}


\twolineshloka
{निष्क्रम्य ते नरव्याघ्रा दंशिताश्चित्रयोधिनः}
{प्रतर्दनं समाजग्मुः शरवर्षैरुदायुधाः}


\twolineshloka
{शस्त्रैश्च विविधाकारै रथौघैश्च युधिष्ठिर}
{अभ्यवर्षन्त राजानं हिमवन्तमिवाम्बुदाः}


\twolineshloka
{अस्त्रैरस्त्राणि संवार्य तेषां राजा प्रतर्दनः}
{जघान तान्महातेजा वज्रानलसमैः शरैः}


\twolineshloka
{कृत्तोत्तमाङ्गास्ते राजन्भल्लैः शतसहस्रशः}
{अपतन्रुधिरार्द्राङ्गा निकृत्ता इव किंशुकाः}


\twolineshloka
{हतेषु तेषु सर्वेषु वीतहव्यः सुतेष्वथ}
{प्राद्रवन्नगरं हित्वा भृगोराश्रममप्युत}


\threelineshloka
{ययौ भृगुं च शरणं वीतहव्यो नराधिपः}
{अभयं च ददौ तस्मै वीतहव्याय भार्गवः}
{आसनं शिष्यमध्ये च भृगुरन्यत्समादिशत्}


\twolineshloka
{अथानुपदमेवाशु तत्रागच्छत्प्रतर्दनः}
{स प्राप्य चाश्रमपदं दिवोदासात्मजोऽब्रवीत्}


\twolineshloka
{भोभो केऽत्राश्रमे सन्ति भृगोः शिष्या महात्मनः}
{द्रष्टुमिच्छे मुनिमहं तस्याचक्षत मामिति}


\twolineshloka
{स तं विदित्वा तु भृगुर्निश्चक्रामाश्रमात्तदा}
{पूजयामास च ततो विधिना नृपसत्तमम्}


\threelineshloka
{उवाच चैनं राजेन्द्र किं कार्यं ब्रूहि पार्थिव}
{स चोवाच नृपस्तस्मै यदागमनकारणम् ॥राजोवाच}
{}


\twolineshloka
{अयं ब्रह्मन्नितो राजा वीतहव्यो विसर्ज्यताम्}
{अस्य पुत्रैर्हि मे कृत्स्नो ब्रह्मन्वंशः प्रणाशितः}


\threelineshloka
{उत्सादितश्च विषयः काशीनां रत्नसञ्चयः}
{एतस्य वीर्यदृप्तस्य हतं पुत्रशतं मया}
{अस्येदानीं वधादद्य भविष्याम्यनृणः पितुः}


\twolineshloka
{तमुवाच कृपाविष्टो भृगुर्धर्मभृतांवरः}
{नेहास्ति क्षत्रियः कश्चित्सर्वे हीमे द्विजातयः}


\twolineshloka
{एतत्तु वचनं श्रुत्वा भृगोस्तथ्यं प्रतर्दनः}
{पादावुपस्पृश्य शनैः प्रहृष्टो वाक्यमब्रवीत्}


\twolineshloka
{एवमप्यस्मि भगवन्कृतकृत्यो न संशयः}
{य एष राजा वीर्येण स्वजातिं त्याजितो मया}


\twolineshloka
{अनुजानीहि मां ब्रह्मन्ध्यायस्व च शिवेन माम्}
{त्याजितो हि मया जातिमेव राजा भृगूद्वह}


\twolineshloka
{ततस्तेनाभ्यनुज्ञातो ययौ राजा प्रतर्दनः}
{यथागतं महाराज मुक्त्वा विषमिवोरगः}


\twolineshloka
{भृगोर्वचनमात्रेण स च ब्रह्मर्षितां गतः}
{वीतहव्यो महाराज ब्रह्मवादित्वमेव च}


\twolineshloka
{तस्य गृत्समदः पुत्रो रूपेणेन्द्र इवापरः}
{शक्रस्त्वमिति यो दैत्यैर्निगृहीतः किलाभवत्}


\twolineshloka
{ऋग्वेदे वर्तते चाग्र्या श्रुतिर्यस्य महात्मनः}
{यत्र गृत्समदो राजन्ब्राह्मणैः स महीयते}


\twolineshloka
{स ब्रह्मचारी विप्रर्षिः श्रीमान्गृत्समदोऽभवत्}
{पुत्रो गृत्समदस्यापि विप्रः सावैनसोऽभवत्}


\twolineshloka
{सावैनसस्य पुत्रो वै वितस्त्यस्तस्य चात्मजः}
{वितस्त्यस्य सुतस्तस्य शिवस्तश्चात्मजोऽभवत्}


\twolineshloka
{श्रवास्तस्य सुतश्चर्षिः श्रवसश्चाभवत्तमः}
{तमसश्च प्रकाशोऽभूत्तनयो द्विजसत्तमः}


\twolineshloka
{प्रकाशस्य च वागिन्द्रो बभूव जयतांवरः}
{तस्यात्मजश्च प्रमितिर्वेदवेदाङ्गपारगः}


\threelineshloka
{घृताच्यां तस्य पुत्रस्तु रुरुर्नामोदपद्यत}
{प्रमद्वरायां तु रुरोः पुत्रः समुदपद्यत}
{शुनको नाम विप्रर्षिर्यस्य पुत्रोऽथ शौनकः}


\twolineshloka
{एवं विप्रत्वमगमद्वीतहव्यो नराधिपः}
{भृगोः प्रसादाद्राजेन्द्र क्षत्रियः क्षत्रियर्षभ}


\twolineshloka
{एष ते कथितो वंशो राजन्गार्त्समदो मया}
{विस्तरेण महाराज किमन्यदनुपृच्छसि}


\chapter{अध्यायः ९}
\threelineshloka
{पितामह महाप्राज्ञ सर्वसास्त्रविशारद}
{दैवे पुरुषकारे च किंस्विच्छ्रेष्ठतरं भवेत् ॥भीष्म उवाच}
{}


\twolineshloka
{अत्राप्युदाहरन्तीममितिहासं पुरातनम्}
{वसिष्ठस्य च संवादं ब्रह्मणश्च युधिष्ठिर}


\twolineshloka
{दैवमानुषयोः किंस्वित्कर्मणोः श्रेष्ठमित्युत}
{पुरा वसिष्ठो भगवान्पितामहमपृच्छत}


\twolineshloka
{ततः पद्मोद्भवो राजन्देवदेवः पितामहः}
{उवाच मधुरं वाक्यमर्थवद्धेतुभूषितम्}


\twolineshloka
{`बीजतो ह्यङ्कुरोत्पत्तिरङ्कुरात्पर्णसम्भवः}
{पर्णान्नालाः प्रसूयन्ते नालात्स्कन्धः प्रवर्तते}


\twolineshloka
{स्कन्धात्प्रवर्तते पुष्पं पुष्पात्संवर्धते फलम्}
{फलान्निर्वर्तते बीजं बीजात्स्यात्सम्भवः पुनः'}


\twolineshloka
{नाबीजं जायते किञ्चिन्न बीजेने विना फलम्}
{बीजाद्बीजं प्रभवति नाबीजं विद्यते फलम्}


\twolineshloka
{यादृशं वपते बीजं क्षेत्रमासाद्य वापकः}
{सुकृते दुष्कृते वाऽपि तादृशं लभते फलम्}


\twolineshloka
{यथा बीजं विना क्षेत्रमुप्तं भवति निष्फलम्}
{तथा पुरुषकारेण विना दैवं न सिध्यति}


\twolineshloka
{क्षेत्रं पुरुषकारस्तु दैवं बीजमुदाहृतम्}
{क्षेत्रबीजसमायोगात्ततः सस्यं समृद्ध्यते}


\twolineshloka
{कर्मणः फलनिर्वृत्तिं स्वयमश्नाति कारकः}
{प्रत्यक्षं दृश्यते लोके कृतस्याप्यकृतस्य च}


\twolineshloka
{शुभेन कर्मणा सौख्यं दुःखं पापेन कर्मणा}
{कृतं सर्वत्र लभते नाकृतं भुज्यते क्वचित्}


\twolineshloka
{कृती सर्वत्र लभते प्रतिष्ठां भाग्यवीक्षितः}
{अकृती लभते भ्रष्टः क्षते क्षारावसेचनम्}


\twolineshloka
{तपसा रूपसौभाग्यं रत्नानि विविधानि च}
{प्राप्यते कर्मणा सर्वं न दैवादकृतात्मना}


\twolineshloka
{तथा स्वर्गश्च भोगश्च निष्ठा या च मनीषिता}
{सर्वं पुरुषकारेण कृतेनेहोपलभ्यते}


\twolineshloka
{ज्योतींषि त्रिदशा नागा यक्षाश्चन्द्रार्कमारुताः}
{सर्वे पुरुषकारेण मानुष्याद्देवतां गताः}


\twolineshloka
{अर्थो वा मित्रवर्गो वा ऐश्वर्यं वा कुलान्वितम्}
{श्रीश्चापि दुर्लभा भोक्तुं तथैवाकृतकर्मभिः}


\twolineshloka
{शौचेन लभते विप्रः क्षत्रियो विक्रमेण तु}
{वैश्यः पुरुषकारेण शूद्रः शुश्रूषया श्रियम्}


\twolineshloka
{नादातारं भजन्त्यर्था न क्लीबं नापि निष्क्रियम्}
{नाकर्मशीलं नाशूरं तथा नैवातपस्विनम्}


\twolineshloka
{येन लोकास्त्रयः सृष्टा दैत्याः सर्वाश्च देवताः}
{स एष भगवान्विष्णुः समुद्रे तप्यते तपः}


\twolineshloka
{स्वं चेत्कर्मफलं न स्यात्सर्वमेवाफलं भवेत्}
{लोको दैवं समालक्ष्य उदासीनो भवेद्यदि}


\twolineshloka
{अकृत्वा मानुषं कर्म यो दैवमनुवर्तते}
{वृथा श्राम्यति सम्प्राप्य पतिं क्लीबमिवाङ्गना}


\twolineshloka
{न तथा मानुषे लोके फलमस्ति शुभाशुभे}
{यथा त्रिदशलोके हि फलमल्पेन जायते}


\twolineshloka
{कृतः पुरुषकारस्तु दैवमेवानुवर्तते}
{न दैवमकृते किञ्चित्कस्यचिद्दातुमर्हति}


\twolineshloka
{यथा स्थानान्यनित्यानि दृश्यन्ते दैवतेष्वपि}
{कथं कर्म विना दैवं स्थास्यति स्थापयिष्यतः}


\twolineshloka
{न दैवतानि लोकेऽस्मिन्व्यापारं यान्ति कस्यचित्}
{व्यासङ्गं जनयन्त्युग्रमात्माभिभवशङ्कया}


\twolineshloka
{ऋषीणां देवतानां च सदा भवति विग्रहः}
{कस्य वाचा ह्यदैवं स्याद्यतो दैवं प्रवर्तते}


\twolineshloka
{कथं तस्य समुत्पत्तिर्यतो दैवं प्रवर्तते}
{एवं त्रिदशलोकेऽपि प्राप्यते परमं सुखम्}


\twolineshloka
{आत्मैव ह्यात्मनो बन्धुरात्मैव रिपुरात्मनः}
{आत्मैव ह्यात्मनः साक्षी कृतस्याप्यकृतस्य च}


\twolineshloka
{कृतं च विकृतं किञ्चित्सिद्ध्यते गुरुकर्मणा}
{सुकृतं दुष्कृतं कर्म अकृतार्थं प्रपद्यते}


\twolineshloka
{देवानां शरणं पुण्यं सर्वं पुण्यैरवाप्यते}
{पुण्यहीनं नरं प्राप्य किं दैवं प्रकरिष्यति}


\twolineshloka
{पुरा ययातिर्विभ्रष्टश्च्यावितः पतितः क्षितौ}
{पुनरारोपितः स्वर्गं दौहित्रैः पुण्यकर्मभिः}


\twolineshloka
{पुरूरवाश्च राजर्षिर्द्विजैरभिहितः पुरा}
{ऐल इत्यभिविख्यातः स्वर्गं प्राप्तो महीपतिः}


\twolineshloka
{अश्वमेधादिभिर्यज्ञैः सत्कृतः कोसलाधिपः}
{महर्षिशापात्सौदासः पुरुषादत्वमागतः}


\twolineshloka
{अश्वत्थामा च रामश्च मुनिपुत्रौ धनुर्धरौ}
{न गच्छतः स्वर्गलोकं वेददृष्टेन कर्मणा}


\twolineshloka
{वसुर्यज्ञशतैरिष्ट्वा द्वितीय इव वासवः}
{मिथ्याभिधानेनैकेन रसातलतलं गतः}


\twolineshloka
{बलिर्वैरोचनिर्बद्धो धर्मपाशेन दैवतैः}
{विष्णोः पुरुषकारेण पातालसदनः कृपः}


\twolineshloka
{शक्रस्याथ रथोपस्थे विष्ठितो जनमेजयः}
{द्विजस्त्रीणां वधं कृत्वा किं दैवेन न वारितः}


\twolineshloka
{अज्ञानाद्ब्राह्मणं हत्वा स्पृष्टो बालवधेन च}
{वैशंपायनविप्रर्षिः किं दैवेन न वारितः}


\twolineshloka
{गोप्रदानेन मिथ्या च ब्राह्मणेभ्यो महामखे}
{पुरा नृगश्च राजर्षिः कृकलासत्वमागतः}


\twolineshloka
{धुन्धुमारश्च राजर्षिः सत्रेष्वेव जरां गतः}
{प्रीतिदायं परित्यज्य सुष्वाप स गिरिव्रजे}


\twolineshloka
{पाण्डवानां हृतं राज्यं धार्तराष्ट्रैर्महाबलैः}
{पुनः प्रत्याहृतं चैव न दैवाद्भुजसंश्रयात्}


\twolineshloka
{तपोनियमसंयुक्ता मुनयः संशितव्रताः}
{किं ते दैवबलाच्छापमुत्सृजन्ते न कर्मणा}


\twolineshloka
{पापमुत्सृजते लोके सर्वं प्राप्य सुदुर्लभम्}
{लोभमोहसमापन्नं न दैवं त्रायते नरम्}


\twolineshloka
{यथाऽऽग्निः पवनोद्भूतः सुसूक्ष्मोऽपि महान्भवेत्}
{तथा कर्मसमायुक्तं दैवं साधु विवर्धते}


\twolineshloka
{यथा तैलक्षयाद्दीपः प्रम्लानिमुपगच्छति}
{तथा कर्मक्षयाद्दैवं प्रम्लानिमुपगच्छति}


\twolineshloka
{विपुलमपि धनौघं प्राप्य भोगान्त्रियो वापुरुष इह न शक्तः कर्महीनो हि भोक्तुम्}
{सुविहितमपि चार्थं दैवते रक्ष्यमाणंपुरुष इह महात्मा प्राप्नुते नित्ययुक्तः}


\twolineshloka
{व्ययगुमपि साधुं कर्मणा संश्रयन्तेभवती मनुजलोकाद्दैवलोको विशिष्टः}
{बहुतरसुसमृद्ध्या मानुषाणां गृहाणिपितृवनभवनाभं दृश्यते चामराणाम्}


\twolineshloka
{न च फलति विकर्मा जीवलोके न दैवंव्यपनयति विमार्गं नास्ति दैवे प्रभुत्वम्}
{गुरुमिव कृतमग्र्यं कर्म संयाति दैवंनयति पुरुषकारः सञ्चितस्तत्रतत्र}


\twolineshloka
{एतत्ते सर्वमाख्यातं मया वै मुनिसत्तम}
{फलं पुरुषकारस्य सदा संदृश्य तत्त्वतः}


\twolineshloka
{अभ्युत्थानेन दैवस्य समारब्धेन केनचित्}
{विधिना कर्मणा चैव स्वर्गमार्गमवाप्नुयात्}


\chapter{अध्यायः १०}
\threelineshloka
{कर्मणां च समस्तानां फलिनां भरतर्षभ}
{फलानि महतां श्रेष्ट प्रब्रूहि परिपृच्छतः ॥भीष्म उवाच}
{}


\threelineshloka
{हन्त ते कथयिष्यामि यन्मां पृच्छसि भारत}
{रहस्यं यदृषीणां तु तच्छृणुष्व युधिष्ठिर}
{या गतिः प्राप्यते येन प्रेत्यभावे चिरेप्सिता}


\twolineshloka
{येनयेन शरीरेण यद्यत्कर्म करोति यः}
{तेनतेन शरीरेण तत्तत्फलमुपाश्नुते}


\twolineshloka
{यस्यांयस्यामवस्थायां यत्करोति शुभाशुभम्}
{तस्यांतस्यामवस्थायां भुङ्क्ते जन्मनिजन्मनि}


\twolineshloka
{न नश्यति कृतं कर्म चित्तपञ्चेन्द्रियैरिह}
{ते ह्यस्य साक्षिणो नित्यं षष्ठ आत्मा शुभाशुभे}


\twolineshloka
{चक्षुर्दद्यान्मनो दद्याद्वाचं दद्याच्च सूनृताम्}
{अनुव्रजेदुपासीत स यज्ञः पञ्चदक्षिणः}


\twolineshloka
{यो दद्यादपरिक्लिष्टमन्नमध्वनि वर्तते}
{श्रान्तायादृष्टपूर्वाय तस्य पुण्यफलं महत्}


\twolineshloka
{स्थण्डिलेषु शयानानां गृहाणि शयनानि च}
{चीरवल्कलसंवीते वासांस्याभरणानि च}


\twolineshloka
{वाहनानि च यानानि योगात्मनि तपोधने}
{अग्नीनुपशयानस्य राज्ञः पौरुषमेव च}


\twolineshloka
{रसानां प्रतिसंहारे सौभाग्यमनुगच्छति}
{आमिषप्रतिसंहारे पशून्पुत्रांश्च विन्दति}


\twolineshloka
{अवाक्शिरास्तु यो लम्बेदुदवासं च यो वसेत्}
{मण्डूकशायी च नरो लभते चेप्सितां गतिम्}


\twolineshloka
{पाद्यमासनमेवाथ दीपमन्नं प्रतिश्रयम्}
{दद्यादतिथिपूजार्थं स यज्ञः पञ्चदक्षिणः}


\twolineshloka
{वीरासनं वीरशय्यां वीरस्थानमुपासतः}
{अक्षयास्तस्य वै लोकाः सर्वकामगमास्तथा}


\twolineshloka
{धनं लभेत दानेन मौनेनाज्ञां विशांपते}
{उपभोगांश्च तपसा ब्रह्मचर्येण जीवितम्}


\twolineshloka
{रूपमैश्वर्यमारोग्यमहिंसाफलमश्नुते}
{फलमूलाशिनो राज्यं स्वर्गः पर्णाशिनां भवेत्}


\twolineshloka
{प्रायोपवेशिनो राजन्सर्वत्र सुखमुच्यते}
{गवाढ्यः शाकदीक्षायां स्वर्गगामी तृणाशनः}


\twolineshloka
{स्त्रियस्त्रिषवणं स्नात्वा वायुं पीत्वा क्रतुं लभेत्}
{स्वर्गं सत्येन लभते दीक्षया कुलमुत्तमम्}


\twolineshloka
{सलिलाशी भवेद्यस्तु सदाग्निः संस्कृतो द्विजः}
{मरुत्साधयतो राज्यं नाकपृष्ठमनाशिने}


\twolineshloka
{उपवासं च दीक्षायामभिषेकं च पार्थिव}
{कृत्वा द्वादशवर्षाणि वीरस्थानाद्विशिष्यते}


\twolineshloka
{अधीत्य सर्ववेदान्वै सद्यो दुःखाद्विमुच्यते}
{`तत्पाठधारणात्स्वर्गमर्थज्ञानात्परां गतिम्}


\twolineshloka
{वितृष्णानां वेदजपात्स्वर्गमोक्षफलं स्मृतम्}
{'मानसं हि चरन्धर्म स्वर्गलोकमुपाश्नुते}


\twolineshloka
{या दुस्त्यजा दुर्मतिभिर्या न जीर्यति जीर्यतः}
{योसौप्राणान्तिकोरोगस्तांतृष्णां त्यजतः सुखं}


\twolineshloka
{यथा धेनुसहस्रेषु वत्सो विन्दति मातरम्}
{एवं पूर्वकृतं कर्म कर्तारमनुगच्छति}


\twolineshloka
{अचोद्यमानानि यथा पुष्पाणि च फलानि च}
{स्वकालं नातिवर्तन्ते तथा कर्म पुराकृतम्}


\twolineshloka
{जीर्यन्ति जीर्यतः केशा दन्ता जीर्यन्ति जीर्यतः}
{चक्षुः श्रोत्रे च जीर्येते तृष्णैका न तु जीर्यते}


\threelineshloka
{येन प्रीणन्ति पितरस्तेन प्रीतः प्रजापतिः}
{माता च येन प्रीणाति पृथिवी तेन पूजिता}
{येन प्रीणात्युपाध्यायस्तेन स्याद्ब्रह्म पूजितम्}


\threelineshloka
{सर्वे तस्यादृता धर्मा यस्यैते त्रय आदृताः}
{अनादृतास्तु यस्यैते सर्वास्तस्याफलाः क्रियाः ॥वैशम्पायन उवाच}
{}


\twolineshloka
{भीष्मस्यैतद्वचः श्रुत्वा विस्मिताः कुरुपुङ्गवाः}
{आसन्प्रहृष्टमनसः प्रीतिमन्तोऽभवंस्तदा}


\twolineshloka
{यन्मन्त्रे भवति वृथोपयुज्यमानेयत्सोमे भवति वृथाऽभिषूयमाणे}
{यच्चाग्नौ भवति वृथाऽभिहूयमानेतत्सर्वं भवति वृथाऽभिधीयमाने}


\twolineshloka
{इत्येतदृषिणा प्रोक्तमुक्तवानस्मि भारत}
{शुभाशुभफलप्राप्तौ किमतः श्रोतुमिच्छसि}


\chapter{अध्यायः ११}
\threelineshloka
{आनृशंस्यस्य धर्मज्ञ गुणान्भक्तजनस्य च}
{श्रोतुमिच्छामि धर्मज्ञ तन्मे ब्रूहि पितामह ॥भीष्म उवाच}
{}


\twolineshloka
{अत्राप्युदाहरन्तीममितिहासं पुरातनम्}
{वासवस्य च संवादं शुकस्य च महात्मनः}


\twolineshloka
{विषये काशिराजस्य ग्रामान्निष्क्रम्य लुब्धकः}
{सविषं काण्डमादाय मृगयामास वै मृगम्}


\twolineshloka
{तत्र चामिषलुब्धेन लुब्धकेन महावने}
{अविदूरे मृगान्दृष्ट्वा बाणः प्रतिसमाहितः}


\twolineshloka
{तेन दुर्वारितास्त्रेण निमित्तचपलेषुणा}
{महान्वनतरुस्तत्र विद्धो मृगजिघांसया}


\twolineshloka
{स तीक्ष्णविषदिग्धेन शरेणातिबलात्क्षतः}
{उत्सृज्य फलपत्राणि पादपः शोषमागतः}


\twolineshloka
{तस्मिन्वृक्षे तथाभूते कोटरेषु चिरोषितः}
{न जहाति शुको वासं तस्य भक्त्या वनस्पतेः}


\twolineshloka
{निष्प्रचारो निराहारो ग्लानः शिथिलवागपि}
{कृतज्ञः सह वृक्षेण धर्मात्मा सोप्यशुष्यत}


\twolineshloka
{तमुदारं महासत्वमतिमानुषचेष्टितम्}
{समदुःखसुखं दृष्ट्वा विस्मितः पाकशासनः}


\twolineshloka
{ततश्चिन्तामुपगतः शक्रः कथमयं द्विजः}
{तिर्यग्योनावसम्भाव्यमानृशंस्यमवस्थितः}


\twolineshloka
{अथवा नात्र चित्रं हीत्यभवद्वासवस्य तु}
{प्राणिनामपि सर्वेषां सर्वं सर्वत्र दृश्यते}


\twolineshloka
{ततो ब्राह्मणवेषेणि मानुषं रूपमास्थितः}
{अवतीर्य महीं शक्रस्तं पक्षिणमुवाच ह}


\twolineshloka
{शुक भो पक्षिणांश्रेष्ठ दाक्षेयी सुप्रजास्त्वया}
{पृच्छे त्वां शुकमेनं त्वं कस्मान्न त्यजसि द्रुमम्}


\twolineshloka
{अथ पृष्टः शुकः प्राह मूर्ध्ना समभिवाद्य तम्}
{स्वागतं देवराज त्वं विज्ञातस्तपसा मया}


\twolineshloka
{ततो दशशताक्षेण साधुसाध्विति भाषितम्}
{अहो विज्ञानमित्येवं मनसा पूजितस्ततः}


\twolineshloka
{तमेवं शुभकर्माणं शुकं परमधार्मिकम्}
{जानन्नपि च तत्पापं पप्रच्छ बलसूदनः}


\twolineshloka
{निष्पत्रमफलं शुष्कमशरण्यं पतत्त्रिणाम्}
{किमर्थं सेवसे वृक्षं यदा महदिदं वनम्}


\twolineshloka
{अन्येऽपि बहवो वृक्षाः पत्रसंछन्नकोटराः}
{शुभाः पर्याप्तसञ्चारा विद्यन्तेऽस्मिन्महावने}


\threelineshloka
{गतायुषमसामर्थ्यं क्षीणसारं हतश्रियम्}
{विमृश्य प्रज्ञया धीर जहीमं ह्यस्थिरं द्रुमम् ॥भीष्म उवाच}
{}


\twolineshloka
{तदुपश्रुत्य धर्मात्मा शुकः शक्रेण भाषितम्}
{सुदीर्घमतिनिःश्वस्य दीनो वाक्यमुवाच ह}


\twolineshloka
{अनतिक्रमणीयानि दैवतानि शचीपते}
{यत्राभवंस्तत्र भवांस्तन्निबोध सुराधिप}


\twolineshloka
{अस्मिन्नहं द्रुमे जातः साधुभिश्च गुणैर्युते}
{चालभावेन सङ्गुप्तः शत्रुभिश्च न धर्षितः}


\threelineshloka
{किमनुक्रोश्यं वैफल्यमुत्पादयसि मेऽनघ}
{`अनुरक्तस्य भक्तस्य संस्पृशे न च पावकम्}
{'आनृशंस्याभियुक्तस्य भक्तस्यानन्यगस्य च}


\twolineshloka
{अनुक्रोशो हि साधूनां महद्धर्मस्य लक्षणम्}
{अनुक्रोशश्च साधूनां सदा प्रीतिं प्रयच्छति}


\twolineshloka
{त्वमेव दैवतैः सर्वैः पृच्छ्यसे धर्मसंशयात्}
{अतस्त्वं देवदेवानामाधिपत्ये प्रतिष्ठितः}


\twolineshloka
{नार्हसे मां सहस्राक्ष द्रुमं त्याजयितुं चिरात्}
{समस्थमुपजीवन्वै विषमस्थं कथं त्यजेत्}


\twolineshloka
{तस्य वाक्येन सौम्येन हर्षितः पाकशासनः}
{शुकं प्रोवाच धर्मज्ञमानृशंस्येन तोषितः}


\twolineshloka
{वरं वृणीष्वेति तदा स च वव्रे वरं शुकः}
{आनृशंस्यपरो नित्यं तस्य वृक्षस्य सम्भवम्}


\twolineshloka
{विदित्वा च दृढां भक्तिं तां शुके शीलसम्पदम्}
{प्रीतः क्षिप्रमथो वृक्षममृतेनावसिक्तवान्}


\twolineshloka
{ततः फलानि पत्राणि शाखाश्चापि मनोहराः}
{शुकस्य दृढभक्तित्वाच्छ्रीमत्तां प्राप स द्रुमः}


\twolineshloka
{शुकश्च कर्मणा तेन आनृशंस्यकृतेन वै}
{आयुषोन्ते महाराज प्राप शक्रसलोकताम्}


\twolineshloka
{एवमेव मनुष्येन्द्र भक्तिमन्तं समाश्रितः}
{सर्वार्थसिद्धिं लभते शुकं प्राप्य यथा द्रुमः}


\chapter{अध्यायः १२}
\twolineshloka
{यथैव ते नमस्कार्याः प्रोक्ताः शक्रेण मानद}
{एतन्मे सर्वमाचक्ष्व येभ्यः स्पृहयसे नृपः}


\threelineshloka
{उत्तमापद्गतस्यापि यत्र ते वर्तते मनः}
{मनुष्यलोके सर्वस्मिन्यदमुत्रेह चाप्युत ॥भीष्म उवाच}
{}


\twolineshloka
{स्पृहयामि द्विजातिभ्यो येषां ब्रह्म परं धनम्}
{येषां संप्रत्ययः स्वर्गस्तपःस्वाध्यायसाधनम्}


\twolineshloka
{येषां बालाश्च वृद्धाश्च पितृपैतामहीं धुरम्}
{उद्वहन्ति न सीदन्ति तेभ्यो वै स्पृहयाम्यहम्}


\twolineshloka
{विद्यास्वभिविनीतानां दान्तानां मृदुभाषिणाम्}
{श्रुतवृत्तोपपन्नानां सदाऽक्षरविदां सताम्}


\twolineshloka
{संसत्सु वदतां तात हंसानामिव सङ्घशः}
{मङ्गल्यरूपा रुचिरा दिव्यजीमूतनिःस्वनाः}


\twolineshloka
{सम्यगुच्चरिता वाचः श्रूयन्ते हि युधिष्ठिर}
{शुश्रूषमाणे नृपतौ प्रेत्य चेह सुखावहाः}


\twolineshloka
{ये चापि तेषां श्रोतारः सदा सदसि सम्मताः}
{विज्ञानगुणसम्पन्नास्तेभ्यश्च स्पृहयाम्यहम्}


\threelineshloka
{सुसंस्कृतानि प्रयताः शुचीनि गुणवन्ति च}
{ददत्यन्नानि तृप्त्यर्थं ब्राह्मणेभ्यो युधिष्ठर}
{ये चापि सततं राजंस्तेभ्यश्च स्पृहयाम्यहम्}


% Check verse!
शक्यं ह्येवाहवे योद्धुं न दातुमनसूयितुम्
\twolineshloka
{शूरा वीराश्च शतशः सन्ति लोके युधिष्ठिर}
{तेषां संख्यायमानानां दानशूरो विशिष्यते}


\twolineshloka
{`भद्रं तु जन्म सम्प्राप्य भूयो ब्राह्मणको भवेत्}
{बन्धुमध्ये कुले जातः सुदुरापमवाप्नुयात् ॥'}


\twolineshloka
{धन्यः स्यां यद्यहं भूयः सौम्यब्राह्मणकोपि वा}
{कुले जातो धर्मगतिस्तपोविद्यापरायणः}


\twolineshloka
{न मे त्वत्तः प्रियतरो लोकेऽस्मिन्पाण्डुनन्दन}
{त्वत्तश्चापि प्रियतरा ब्राह्मणा एव भारत}


\twolineshloka
{यथा मम प्रियतमास्त्वत्तो विप्राः कुरूत्तम}
{तेन सत्येन गच्छेयं लोकान्यत्र स शन्तनुः}


\twolineshloka
{न मे पिता प्रियतरो ब्रह्मणेभ्यस्तथाऽभवत्}
{न मे पितुः पिता वाऽपि ये चान्येऽपि सुहृञ्जनाः}


\twolineshloka
{न हि मे वृजिनं किञ्चिद्विद्यते ब्राह्मणेष्विह}
{अणु वा यदि वा स्थूलं विद्यते साधुकर्मसु}


\twolineshloka
{कर्मणा मनसा वाऽपि वाचा वाऽपि परंतप}
{यन्मे कृतं ब्राह्मणेभ्यस्तेनाद्य न तपाम्यहम्}


\twolineshloka
{ब्रह्मण्य इति मामाहुस्तया वाचाऽस्मि तोषितः}
{एतदेव पवित्रेभ्यः सर्वेभ्यः परमं स्मृतम्}


\twolineshloka
{पश्यामि लोकानमलाञ्शुचीन्ब्राह्मणतोषणात्}
{तेषु मे तात गन्तव्यमह्नाय च चिराय च}


\twolineshloka
{यथा भर्त्राश्रयो धर्मः स्त्रीणां लोके युदिष्ठिर}
{स देवः सा गतिर्नान्या क्षत्रियस्य तथा द्विजाः}


\twolineshloka
{क्षत्रियः शतवर्षी च दशवर्षी द्विजोत्तमः}
{पितापुत्रौ च विज्ञेयौ तयोर्हि ब्राह्मणो गुरुः}


\twolineshloka
{नारी तु पत्यभावे वै देवरं कुरुते पतिम्}
{पृथिवी ब्राह्मणालाभे क्षत्रियं कुरुते पतिम्}


\twolineshloka
{`ब्राह्मणानुज्ञया ग्राह्यं राज्यं च सपुरोहितैः}
{तद्रक्षणेन स्वर्गोऽस्य तत्कोपान्नरकोऽक्षयः ॥'}


\twolineshloka
{पुत्रवच्चैव ते रक्ष्या उपास्या गुरुवच्च ते}
{अग्निवच्चोपचार्या वै ब्राह्मणाः कुरुसत्तम्}


\twolineshloka
{ऋजुन्सतः सत्यशीलान्सर्वभूतहिते रतान्}
{आशीविषानिव क्रुद्धान्द्विजान्परिचरेत्सदा}


\twolineshloka
{तेजसस्तपसश्चैव नित्यं बिभ्येद्युधिष्ठिर}
{उभे चैते परित्याज्ये तेजश्चैव तपस्तथा}


\twolineshloka
{व्यवसायस्तयोः शीघ्रमुभयोरेव विद्यते}
{हन्युः क्रुद्धा महाराज ब्राह्मणा ये तपस्विनः}


\twolineshloka
{`दूरतो मातृवत्पूज्या विप्रदाराः सुरक्षया}
{अकोपनापराधेन भूयो नरकमश्नुते}


\twolineshloka
{भूयः स्यादुभयं दत्तं ब्राह्मणाद्यदकोपनात्}
{कुर्यादुभयतः शेषं दत्तशेषं न शेषयेत्}


\twolineshloka
{दण्डपाणिर्यथा गोष्ठं पालो नित्यं हि रक्षयेत्}
{ब्राह्मणेषु स्थितं ब्रह्म क्षत्रियः परिपालयेत्}


\twolineshloka
{पितेव पुत्रान्रक्षेथा ब्राह्मणान्ब्रह्मतेजसः}
{गृहे चैषामवेक्षेथाः किंस्विदस्तीति जीवनम्}


\chapter{अध्यायः १३}
\threelineshloka
{अत्राप्युदाहरन्तीममितिहासं पुरातनम्}
{चतुर्णामपि वेदानां संवादं शृणु पुत्रक ॥ऋग्वेद उवाच}
{}


\twolineshloka
{गृहानाश्रयमाणस्य अग्निहोत्रं च जुह्वतः}
{सर्वं सुकृतमादत्ते यः साये नुद्यतेऽतिथिः}


\threelineshloka
{न स्कन्दते न व्यथते नास्योर्ध्वं सर्पते रजः}
{वरिष्ठमग्निहोत्राच्च ब्राह्मणस्य मुखे हुतम् ॥सामवेद उवाच}
{}


\fourlineindentedshloka
{न चेद्धन्ति पितरं मातरं वान ब्राह्मणं नापवादं करोति}
{यत्किंचिदन्यद्वृजिनं करोति}
{प्रीतोऽतिथिस्तदुपहन्ति पापम् ॥अथर्ववेद उवाच}
{}


\twolineshloka
{यत्क्रोधनो यजते यद्ददातियद्वा तपस्तप्यति यज्जुहोति}
{वैवस्वतो हरते सर्वमस्यमोघं चेष्टं भवति क्रोधनस्य}


\chapter{अध्यायः १४}
% Check verse!
भूयस्तु शृणु राजेन्द्र धर्मान्धर्मभृतांवर
\threelineshloka
{अत्राप्युदाहरन्तीममितिहासं पुरातनम्}
{इन्द्राग्न्योः सूर्यशच्योश्च तन्मे निगदतः शृणु ॥इन्द्र उवाच}
{}


\threelineshloka
{राज्ये विप्रान्प्रपश्यामि कामक्रोधविवर्जितान्}
{एतेन सत्यवाक्येन पादः कुम्भस्य पूर्यताम् ॥अग्निरुवाच}
{}


\twolineshloka
{यथाऽहं तत्र नाश्नामि यत्र नाश्नन्ति वै द्विजाःएतेन सत्यवाक्येन पादः कुम्भस्य पूर्यताम् ॥सूर्य उवाच}
{}


\threelineshloka
{यथा गोब्राह्मणस्यार्थे न तपामि यथाबलम्}
{एतेन सत्यवाक्येन पादः कुम्भस्य पूर्यताम् ॥शच्युवाच}
{}


\twolineshloka
{कर्मणा मनसा वाचा नावमन्ये पुरन्दरम्}
{एतेन सत्यवाक्येन पादः कुम्भस्य पूर्यताम्}


\chapter{अध्यायः १५}
\twolineshloka
{अत्राप्युदाहरन्तीममितिहासं पुरातनम्}
{मद्रराजस्य संवादं व्यासस्य च महात्मनः}


\threelineshloka
{वैताने कर्मणि तते कुन्तीपुत्र यथा पुरा}
{उक्तः स भगवान्यज्ञे तथा तत्राशृणोद्भवान् ॥मद्रराज उवाच}
{}


\twolineshloka
{कानि तीर्थानि भगवन्फलार्थाश्चेह केऽऽश्रमाः}
{क इज्यते कश्च यज्ञः को यूपः क्रमते च कः}


\twolineshloka
{कश्चाध्वरे शस्यते गीतिशब्दैःकश्चाध्वरे गीयते वल्गुभाषैः}
{को ब्रह्मशब्दैः स्तुतिभिः स्तूयते चकस्येह वै हविरध्वर्यवः कल्पयन्ति}


\threelineshloka
{वर्णाश्रमे गोफले कश्च सोमेकश्चोंकारः कश्च वेदार्थमार्गः}
{पृष्टस्तन्मे ब्रूहि सर्वं महर्षेलोकज्येष्ठं कस्य विज्ञानमाहुः ॥द्वैपायन उवाच}
{}


\threelineshloka
{लोकज्येष्ठं यस्य विज्ञानमाहुर्योनिज्येष्ठं यस्य वदन्ति जन्म}
{पूतात्मानो ब्राह्मणा वेदमुख्याअस्मिन्प्रश्नो दीयतां केशवाय ॥ब्राह्मण उवाच}
{}


\threelineshloka
{बालो जात्या क्षत्रधर्मार्थशीलोजातो देवक्यां शूरपुत्रेण वीर}
{वेत्तुं वेदानर्हते क्षत्रियो वैदाशार्हाणामुत्तमः पुष्कराक्षः ॥वासुदेव उवाच}
{}


\threelineshloka
{पाराशर्य ब्रूहि यद्ब्राह्मणेभ्यःप्रीतात्मा वै ब्रह्मकल्पः सुमेधाः}
{पृष्टो यज्ञार्थं पाण्डवस्यातितेजाएतच्छ्रेयस्तस्य लोकस्य चैव ॥व्यास उवाच}
{}


\threelineshloka
{उक्तं वाक्यं यद्भवान्मामवोच-त्प्रश्नं चित्रं नाहमत्रोत्सहेऽद्य}
{छेत्तुं विस्पष्टं तिष्ठति त्वद्विधे वैलोकज्येष्ठे विश्वरूपे सुनाभे ॥वासुदेव उवाच}
{}


\threelineshloka
{तत्त्वं वाक्यं ब्रूहि यत्त्वं महर्षेयस्मिन्कृष्णः प्रोच्यते वै यथावत्}
{प्रीतस्तेऽहं ज्ञानशक्त्या यथाव-त्तस्मान्निर्देशे कर्मणां ब्रूहि सिद्धिम् ॥वैशम्पायन उवाच}
{}


\threelineshloka
{उक्तवाक्ये सत्तमे यादवानांकृष्णो व्यासः प्राञ्जलिर्वासुदेवम्}
{विप्रैः सार्धं पूजयन्देवदेवंकृष्णं विष्णुं वासुदेवं बभाषे ॥व्यास उवाच}
{}


\twolineshloka
{आनन्त्यं ते विश्वकर्मंस्तवैवंरूपं पौराणं शाश्वतं च ध्रुवं च}
{कस्ते बुद्ध्येद्वेदवादेषु चैत-ल्लोके ह्यस्मिञ्शासकस्त्वं पितेव}


\chapter{अध्यायः १६}
\twolineshloka
{द्वारकायां यथा प्राह पुराऽयं मुनिसत्तमः}
{वेदविप्रमयत्वं तु वासुदेवस्य तच्छृणु}


\twolineshloka
{यूपं विष्णुं वासुदेवं विजान-न्सर्वान्विप्रान्बोधते तत्वदर्शि}
{विष्णुं क्रान्तं वासुदेवं विजान-न्विप्रो विप्रत्वं गच्छते तत्वदर्शी}


\twolineshloka
{विष्णुर्यज्ञस्त्विज्यते चापि विष्णुःकृष्णो विष्णुर्यश्च कृत्स्नः प्रभुश्च}
{कृष्णो वेदाङ्गं वेदवादाश्च कृष्णएवं जानन्ब्राह्मणो ब्रह्म एति}


\twolineshloka
{स्थानं सर्वं वैष्णवं यज्ञमार्गेचातुर्होत्रं वैष्णवं तत्र कृष्णः}
{सर्वैर्भावैरिज्यते सर्वकामैःपुण्याँल्लोकान्ब्राह्मणाः प्राप्नुवन्ति}


\twolineshloka
{सोमं सद्भावाद्ये च जातं पिबन्तिदीप्तिं कर्म ये विदानाश्चरन्ति}
{एकान्तमिष्टं चिन्तयन्तो दिविस्था-स्ते वै स्थानं प्राप्नुवन्ति व्रतज्ञाः}


\twolineshloka
{ओमित्येतद्ध्यायमानो न गच्छे-द्दुर्गं पन्थानं पापकर्मापि विप्रः}
{सर्वं कृष्णं वासुदेवं हि विग्राःकृत्वा ध्यानं दुर्गतिं न प्रयान्ति}


\twolineshloka
{आज्यं यज्ञः स्रुक्स्रुवौ यज्ञदाताइच्छा पत्नी पत्निशाला हवींषि}
{इध्याः पुरोडाशं सर्वदा होतृकर्ताकृत्स्नं विष्णुं संविजानंस्तमेति}


\twolineshloka
{योगेयोगे कर्मणां चाभिहारेयुक्ते वैताने कर्मणि ब्राह्मणस्य}
{पुष्ट्यर्थेषु प्राप्नुयात्कर्मसिद्धिंशान्त्यर्थेषु प्राप्नुयात्सर्वशान्तिम्ट'}


\chapter{अध्यायः १७}
\twolineshloka
{श्रद्दात्यागं निर्वृतिं चापि पूजांसत्यं धर्मं यः कृतं चाभ्युपैति}
{कामद्वेषौ त्यज्य सर्वेषु तुल्यःश्रद्धापूतः सर्वयज्ञेषु योग्यः}


\twolineshloka
{यस्मिन्यज्ञे सर्वभूताः प्रहृष्टाःसर्वे चारम्भाः शास्त्रदृष्टाः प्रवृत्ताः}
{धर्म्यैरर्थ्यैर्ये यजन्ते ध्रुवं तेपूतात्मानो धर्ममेकं भजन्ते}


\twolineshloka
{एकाक्षरं द्व्यक्षरमेकमेवसदा यजन्ते नियताः प्रतीताः}
{दृष्ट्वा मनागर्चयित्वा स्म विप्राःसतां मार्गं तं ध्रुवं सम्भजन्ते}


\twolineshloka
{पापात्मानः क्रोधरगाभिभूताःकृष्णे भक्ता नाम सङ्कीर्तयन्तः}
{पूतात्मानो यज्ञशीलाः सुमेधायज्ञस्यान्ते कीर्तिलोकान्भजन्ते}


\twolineshloka
{एको वेदो ब्राह्मणानां बभूवचतुष्पादस्त्रिगुणो ब्रह्मशीर्षः}
{पादंपादं ब्राह्मणा वेदमाहु-स्त्रेताकाले तं च तं विद्धि शीर्षम् ॥'}


\chapter{अध्यायः १८}
\twolineshloka
{सर्वे वेदाः सर्ववेद्याः सशास्त्राःसर्वे यज्ञाः सर्व इत्यश्च कृष्णः}
{विदुः कृष्णं ब्राह्मणास्तत्वतो येतेषां राजन्सर्वयज्ञः समाप्तः}


\twolineshloka
{ज्ञेयो योगी ब्राह्मणैर्वेदतत्वै-रारण्यकैः सैष कृष्णः प्रभुत्वात्}
{सर्वान्यज्ञान्ब्राह्मणान्ब्रह्म चैवव्याप्यातिष्ठद्देवदेवस्त्रिलोके}


% Check verse!
स एष देवः शक्रमीशं यजानंप्रीत्या प्राह क्रतुयष्टारमग्र्यम्न मा शक्रो वेदवेदार्थतत्वा-द्भक्तो भक्त्या शुद्धभावप्रधानः
\twolineshloka
{मा जानन्ते ब्रह्मशीर्षं वरिष्ठंविश्वे विश्वं ब्रह्मयोनिं ह्ययोनिम्}
{सर्वत्राहं शाश्वतः शाश्वतेशःकृत्स्नो वेदोऽग्निर्निर्गुणोऽनन्ततेजाः}


\twolineshloka
{सर्वे देवा वासुदेवं यजन्तेततो बुद्ध्या मार्गमाणास्तनूनाम्}
{सर्वान्कामान्प्राप्नुवन्ते विशालां-स्त्रैलोक्येऽस्मिन्कृष्णनामाभिधानात्}


\twolineshloka
{कृष्णो यज्ञैरिज्यते यायजूकैःकृष्णो वीर्यैरिज्यते विक्रमद्भिः}
{कृष्णो वाक्यैरिज्यते सम्मृशानैःकृष्णो मुक्तैरिज्यते वीतमोहैः}


\threelineshloka
{विद्यावन्तः सोमपा ये विपापाइष्ट्वा यज्ञेर्गोचरं प्रार्थयन्ते}
{भवगानुवाच}
{सर्वं क्रान्तं देवलोकं विशाल-मन्ते गत्वा मुक्तिलोकं भजन्ति}


\twolineshloka
{एवं सर्वे त्वाश्रमाः सुव्रता येमां जानन्तो यान्ति लोकानदीनान्}
{ये ध्यानदीक्षामुद्वहन्तो विपापा ज्योतिर्भूत्वा देवलोकं भजन्ति}


\threelineshloka
{पूज्यन्ते मां पूजयन्तः प्रहृष्टामां जानन्तः श्रद्धया वासुदेवम्}
{भक्त्या तुष्टोऽहं तस्य सत्त्वं प्रयच्छेसत्वस्पृष्टो वीतमोहोऽयमेति ॥द्वैपायन उवाच}
{}


\threelineshloka
{ज्योतींषि शुक्तानि च यानि लोकेत्रयो लोका लोकपालास्त्रयश्च}
{त्रयोऽग्नयश्चाहुतयश्च पञ्चसर्वे देवा देवकीपुत्र एव ॥भीष्म उवाच}
{}


\threelineshloka
{व्यासस्यैतद्वचः श्रुत्वा मद्रराजः सहर्षिभिः}
{व्यासं कृष्णं च विधिवत्प्रीतात्मा प्रत्यपूजयत् ॥वैशम्पायन उवाच}
{}


\twolineshloka
{कविः प्रयातस्तु महर्षिपुत्रोद्वैपायनस्तद्वचनं निशम्य}
{जगाम पृथ्वीं शिरसा महात्मानमश्च कृष्णाय चकार भीष्मः ॥'}


\chapter{अध्यायः १९}
\twolineshloka
{गरुडः पक्षिणां श्रेष्ठ इति पूर्वं पितामह}
{उक्तस्त्वया महाबाहो श्वेतवाहं प्रशंसता}


\twolineshloka
{अत्र कौतूहल इति श्रवणे जायते मतिः}
{कथं गरुत्मान्पक्षीणां श्रैष्ठ्यं प्राप परन्तप}


\threelineshloka
{सुपर्णो वैनतेयश्च केन शत्रुश्च भोगिनाम्}
{किंवीर्यः किंबलश्चासौ वक्तुमर्हसि भारत ॥भीष्म उवाच}
{}


\twolineshloka
{वासुदेव महाबाहो देवकी सुप्रजास्त्वया}
{श्रुतं ते धर्मराजस्य मम हर्षविवर्धन}


\twolineshloka
{सुपर्णं शंस इत्येव मामाह कुरुनन्दनः}
{अस्य प्रवक्तुमिच्छामि त्वयाऽज्ञप्तो महाद्युते}


\twolineshloka
{त्वं हि शौरे महाबाहो सुपर्णः प्रोच्यसे पुरा}
{अनादिनिधने काले गरुडश्चासि केशव}


\threelineshloka
{तस्मात्पूर्वं प्रसाद्य त्वां धर्मपुत्राय धीमते}
{गरुडं पततांश्रेष्ठं वक्तुमिच्छामि माधव ॥वासुदेव उवाच}
{}


\twolineshloka
{यथैव मां भवान्वेद तथा वेद युधिष्ठिरः}
{यथा च गरुडो जातस्तथाऽस्मै ब्रूहि तत्वतः'}


\chapter{अध्यायः २०}
\twolineshloka
{युधिष्ठिर महाबाहो शृणु राजन्यथातथम्}
{गरुडं पक्षिणां श्रेष्ठं वैनतेयं महाबलम्}


\twolineshloka
{तथा च गरुडो राजन्सुपर्णश्च यथाऽभवत्}
{यथा च भुजगान्हन्ति तथा मे ब्रुवतः शृणु}


\twolineshloka
{पुराऽहं तात रामेणि जामदग्न्येन धीमता}
{कैलासशिखरे रम्ये मृगान्निघ्नन्सहस्रशः}


\twolineshloka
{तमहं तात दृष्टैव शस्त्रण्युत्सृज्य सर्वशः}
{अभिवाद्य पूर्वं रामाय विनयेनोपतस्थिवान्}


\twolineshloka
{तमहं कथान्ते वरदं सुपर्णस्य बलौजसी}
{अपृच्छं स च मां प्रीतः प्रत्युवाच युधिष्ठिर}


\twolineshloka
{कद्रूश्च विनता चास्तां प्रजापतिसुते उभे}
{ते तु धर्मेणोपयेमे मारीचः कश्यपः प्रभुः}


% Check verse!
प्रादात्ताभ्यां वरं प्रीतो भार्याभ्यां सुमहातपाः
\twolineshloka
{तत्र कद्रूर्वरं वव्रे पुत्राणां दशतः शतम्}
{तुल्यतेजःप्रभावानां सर्वेषां तुल्यजन्मनाम्}


\twolineshloka
{विनता तु वव्रे द्वौ पुत्रौ वीरौ भरतसत्तम}
{कद्रूपुत्रसहस्रेण तुल्यवेगपराक्रमौ}


\twolineshloka
{स तु ताभ्यां वरं प्रादात्तथेत्युक्त्वा महातपाः}
{जनयामास तान्पुत्रां स्ताभ्यामासीद्यथा पुरा}


\twolineshloka
{कद्रूः प्रजज्ञे ह्यण्डानां तथैव दशतःशतम्}
{अण्डे द्वे विनता चैव दर्शनीयतरे शुभे}


\twolineshloka
{तानि त्वण्डानि तु तयोः कद्रूविनतयोर्द्वयोः}
{सोपस्वेदेषु पात्रेषु निदधुः परिचारिणः}


\twolineshloka
{निस्सरन्ति तदाऽण्डेभ्यः कद्रूपुत्रा भुजङ्गमाः}
{पञ्चवर्षशते काले दृष्ट्वाऽमोघबलौजसः}


\twolineshloka
{विनता तेषु जातेषु पन्नगेषु महात्मसु}
{विपुत्रा पुत्रसंतापाद्दण्डमेकं बिभेद ह}


\twolineshloka
{किमनेन करिष्येऽहमिति वाक्यमभाषत}
{नहि पञ्चशते काले पुरा पुत्रौ ददर्श सा}


\twolineshloka
{सापश्यदण्डान्निष्क्रान्तं विनापत्रं मनस्विनम्}
{पूर्वकायोपसम्पन्नं वियुक्तमितरेण ह}


\twolineshloka
{दृष्ट्वा तु तं तथारूपमसमग्रशरीरिणम्}
{पुत्रदुःखान्विताऽशोचत्स च पक्षी तथा गतः}


\twolineshloka
{अब्रवीच्च मुदा युक्तः पर्यश्रुनयनस्तदा}
{मातरं च पलाशी ह हतोऽहमिति चासकृत्}


\twolineshloka
{न त्वया काङ्क्षितः कालो यावानेवात्यगात्पुरा}
{आवां भवाय पुत्रौ ते श्वसनाद्बलवत्तरौ}


\twolineshloka
{ईर्ष्याक्रोधाभिभूतत्वाद्योहमेवं कृतस्त्वया}
{तस्मात्त्वमपि मे मातर्दासीभावं गमिष्यसि}


\twolineshloka
{पञ्चवर्षशतानि त्वं स्पर्धसे वै यया सह}
{दासी तस्या भवित्रीति साश्रुपातमुवाच ह}


\twolineshloka
{एष चैव महाभागे बली बलवतांवरः}
{मोक्षयिष्यति ते मातर्दासीभावान्ममानुजः ॥'}


\chapter{अध्यायः २१}
\twolineshloka
{विनता पुत्रशोकार्ता शापाद्भीता च भारत}
{प्रतीक्षते स्म तं कालं यः पुत्रोक्तस्तदाऽभवत्}


\twolineshloka
{ततोऽप्यतीते पञ्चशते वर्षाणां कालसंयुगे}
{गरुडोऽथ महावीर्यो जज्ञे भुजगभुग्बली}


\twolineshloka
{बन्धुरास्यः शिखी पत्रकोशः कूर्मनखो महान्}
{रक्ताक्षः संहतग्रीवो ह्रस्वपादो महाशिराः}


\twolineshloka
{यस्त्वण्डात्स विनिर्भिन्नो निष्क्रान्तो भरतर्षभ}
{विनतापूर्वजः पुत्रः सोऽरुणो दृश्यते दिवि}


\twolineshloka
{पूर्वां दिशामभिप्रेत्य सूर्यस्योदयनं प्रति}
{अरुणोऽरुणसंकाशो नाम्ना चैवारुणः स्मृतः}


\twolineshloka
{जातमात्रस्तु विहगो गरुडः पन्नगाशनः}
{विहाय मातरं क्षिप्रमगमत्सर्वतो दिशः}


\twolineshloka
{स तदा ववृधेऽतीव सर्वकामैः कदाऽर्चितः}
{पितामहविसृष्टेन भोजनेन विशांपते}


\twolineshloka
{तस्मिंश्च विहगे तत्र यथाकामं विवर्धति}
{कद्रूश्च विनता चैवागच्छतां सागरं प्रति}


\twolineshloka
{ददृशाते तु ते यान्तमुच्चैश्श्रवसमन्तिकात्}
{स्नात्वोपवृत्तं त्वरितं पीतवन्तं च वाजिनम्}


\threelineshloka
{ततः कद्रूर्हसन्त्येव विनतामिदमब्रवीत्}
{हयस्य वर्णः को न्वत्र ब्रूहि यस्ते मतः शुभे ॥विनतोवाच}
{}


\threelineshloka
{एकवर्णो हयो राज्ञि सर्वश्वेतो मतो मम}
{वर्णं वा कीदृशं तस्य मन्यते त्वं मनस्विनि ॥कद्रूरुवाच}
{}


\threelineshloka
{सर्वश्वेतो मतस्तुभ्यं य एष हयसत्तमः}
{ब्रूहि कल्याणि दीव्यावो वर्णान्यत्वेन भामिनि ॥विनतोवाच}
{}


\threelineshloka
{यद्यार्ये दीव्यसि त्वं मे कः पणो नो भविष्यति}
{सा तज्ज्ञात्वा पणेयं वै ज्ञात्वा तु विपणे त्वया ॥कद्रूरुवाच}
{}


\twolineshloka
{जिता दासी भवेर्मे त्वमहं चाप्यसितेक्षणे}
{नैकवर्णैकवर्णत्वे विनते रोचते च ते}


\twolineshloka
{रोचते मे पणे राज्ञि दासीत्वेन न संशयः}
{सत्यमातिष्ठ भद्रं ते सत्ये स्थास्यामि चाप्यहम्'}


\chapter{अध्यायः २२}
\twolineshloka
{विनता तु तथेत्युक्त्वा कृतसंशयना पणे}
{कद्रूरपि तथेत्युक्त्वा पुत्रानिदमुवाच ह}


\twolineshloka
{मया कृतः पणः पुत्रा मिथो विनतया सह}
{उच्चैरश्रवसि गान्धर्वे तच्छृणुध्वं भुजङ्गमाः}


\twolineshloka
{अब्रवं नैकवर्णं तं सैकवर्णमथाब्रवीत्}
{जिता दासी भवेत्पुत्राः सा वाऽहं वा न संशयः}


\twolineshloka
{एकवर्णश्च वाजी स चन्द्रकोकनदप्रभः}
{साऽहं दासी भविष्यामि जिता पुत्रा न संशयः}


\threelineshloka
{ते यूयमश्वप्रवरमाविशध्वमतन्द्रिताः}
{सर्वश्वेतं वालधिषु वाला भूत्वाञ्जनप्रभाः ॥सर्पा ऊचुः}
{}


\twolineshloka
{निकृत्या न जयः श्रेयान्मातः सत्या गिरः शृणु}
{आयत्यां च तदात्वे च न च धर्मोऽत्र विद्यते}


\twolineshloka
{सा त्वं धर्मादपेतं वै कुलस्यैवाहितं तव}
{निकृत्या विजयं मातर्मा स्म कार्षीः कथञ्चना}


\twolineshloka
{यद्यधर्मेण विजयं वयं काङ्क्षामहे क्वचित्}
{त्वया नाम निवार्याः स्म मा कुरुध्वमिति ध्रुवम्}


\twolineshloka
{सा त्वमस्मानपि सतो विपापानृजुबुद्धिनः}
{कल्मषेणाभिसंयोक्तुं काङ्क्षसे लोभमोहिता}


\twolineshloka
{ते वयं त्वां परित्यज्य द्रविष्याम दिशो दश}
{यत्र वाक्यं न ते मातः पुनः श्रोष्याम ईदृशम्}


\threelineshloka
{गुरोरप्यवलिप्तस्य कार्याकार्यमजानतः}
{उत्पथं प्रतिपन्नस्य परित्यागो विधीयते ॥कद्रूरुवाच}
{}


\twolineshloka
{शृणोमि विविधा वाचो हेतुमत्यः समीरिताः}
{वक्रमार्गनिवृत्त्यर्थं तदहं वो न रोचये}


\twolineshloka
{न च तत्पणितं मन्दाः शक्यं जेतुमतोऽन्वथा}
{जिते निकृत्या श्रुत्वैतत्क्षेमं कुरुत पुत्रकाः}


\twolineshloka
{श्वोऽहं प्रभातसमये जिता धर्मेण पुत्रकाः}
{शैलूषिणी भविष्यामि विनताया न संशयः}


\twolineshloka
{इह चामुत्र चार्थाय पुत्रानिच्छन्ति मातरः}
{सेयमीहा विपन्ना मे युष्मानासाद्य सङ्गताम्}


\twolineshloka
{इह वा तारयेत्पुत्रः प्रेत्य वा तारयेत्पितॄन्}
{मात्र चित्रं भवेकिञ्चित्पुनातीति च पुत्रकः}


\twolineshloka
{ते यूयं तारणार्थाय मम पुत्रा मनोजवाः}
{आविशध्वं हयश्रेष्ठं वाला भूत्वाऽञ्जनप्रभाः}


\twolineshloka
{जानाम्यधर्मं सकलं विजिता विनता भवेत्}
{निकृत्या दासभावस्तु युष्मानप्यवपीडयेत्}


\twolineshloka
{निकृत्या विजयो वेति दासत्वं वा पराभवे}
{उभयं निश्चयं कृत्वा जयो वै धार्मिको वरः}


\twolineshloka
{यद्यप्यधर्मो विजयो युष्मानेव स्पृशेत्पुनः}
{गुरोर्वचनमास्थाय धर्मो वा सम्भविष्यति'}


\chapter{अध्यायः २३}
\twolineshloka
{श्रुत्वा तु वचनं मातुः क्रुद्धायास्ते भुजङ्गमाः}
{कृच्छ्रेणैवान्वमोदन्त केचिद्धित्वा दिशो गताः}


\twolineshloka
{ये प्रतस्थुर्दिशस्तत्र क्रुद्धा तानशपद्भृशम्}
{भुजङ्गमानां माताऽसौ कद्रूर्वैरकरी तदा}


\twolineshloka
{उत्पत्स्यति हि राजन्यः पाण्डवो जनमेजयः}
{चतुर्थो धन्विनां श्रेष्ठात्कुन्तीपुत्राद्धनञ्जयात्}


\twolineshloka
{स सर्पसत्रमाहर्ता क्रुद्धः कुरुकुलोद्वहः}
{तस्मिन्सत्रेऽग्निना युष्मान्पञ्चत्वमुपनेष्यति}


\twolineshloka
{एवं क्रुद्धाऽशपन्माता पन्नगान्धर्मचारिणः}
{गुरोः परित्यागकृतं नैतदन्यद्भविष्यति}


\twolineshloka
{एवं शप्ता दिशः प्राप्ताः पन्नगा धर्मचारिणः}
{विहाय मातरं क्रुद्धा गता वैरकरीं तदा}


\twolineshloka
{तत्र ये वृजिनं तस्या अनापन्ना भुजङ्गमाः}
{ते तस्य वाजिनो वाला बभूवुरसितप्रभाः}


\twolineshloka
{तान्दृष्ट्वा वालधिस्थांश्च पुत्रान्कद्रूरथाब्रवीत्}
{विनतामथ संहृष्टा हयोसौ दृश्यतामिति}


\twolineshloka
{एकवर्णो न वा भद्रे पणो नौ सुव्यवस्थितः}
{उतकादुत्तरन्तं तं हयं चैव च भामिनि}


\twolineshloka
{सा त्ववक्रमतिर्देवी विनता जिह्मगामिनीम्}
{अब्रवीद्भगिनीं किञ्चिद्विहसन्तीव भारत}


\twolineshloka
{हन्त पश्याव गच्छावः सुकृतो नौ पणः शुभे}
{दासी वा ते भविष्यामि त्वं वा दासी भविष्यसि}


\twolineshloka
{एवं स्थिरं पणं कृत्वा हयं ते तं ददर्शतुः}
{कृत्वा साक्षिणमात्मानं भगिन्यौ कुरुसत्तम}


\twolineshloka
{सा दृष्ट्वैव हयं मन्दं विनता शोककर्शिता}
{श्वेतं चन्द्रांशुवालं तं कालवालं मनोजवम्}


\twolineshloka
{तत्र सा व्रीलिता वाक्यं विनता साश्रुबिन्दुका}
{उवाच कालवालोऽयं तुरगो विजितं त्वया}


\twolineshloka
{दासी मां प्रेषय स्वार्थे यथा कामवशां शुभे}
{दास्यश्च कामकारा हि भर्तृणां नात्र संशयः'}


\chapter{अध्यायः २४}
\twolineshloka
{ततः कद्रूर्हसन्ती च विनतां धर्मचारिणीम्}
{दासीवत्प्रेषयामास सा च सर्वं चकार तत्}


\twolineshloka
{न विवर्णा न संक्षुब्धा न च क्रुद्धा न दुःखिता}
{प्रेष्यकर्म चकारास्या विनता कमलेक्षणा}


\twolineshloka
{इमा दिशश्चतस्रोऽस्याः प्रेष्यभावेन वर्तिताः}
{अथ स्म वैनतेयं वै बलदर्पौ समीयतुः}


\twolineshloka
{तं दर्पवशमापन्नं परिधावन्तमन्तिकात्}
{ददर्श नारदो राजन्देवर्षिर्दर्पसंयुतम्}


\twolineshloka
{तमब्रवीच्च देवर्षिर्नारदः प्रहसन्निव}
{किं दर्पवशमापन्नो न वै पश्यसि मातरम्}


\twolineshloka
{बलेन दृप्तः सततमहंमानकृतः सदा}
{दासीं पन्नगराजस्य मातुरन्तर्गृहे सतीम्}


\twolineshloka
{तमब्रवीद्वैनतेयः कर्म किं तन्महामते}
{जनयित्री मयि सुते जाता दासी तपस्विनी}


\threelineshloka
{अथाब्रवीदृषिर्वाक्यं दीव्यती विजिता खग}
{निकृत्या पन्नगेन्द्रस्य मात्रा पुत्रैः पुरा सह ॥गरुड उवाच}
{}


\twolineshloka
{कथं जिता निकृत्या सा भगवञ्जननी मम}
{ब्रूहि तन्मे यथावृत्तं श्रुत्वा वेत्स्ये ततः परम्}


\twolineshloka
{ततस्तस्य यथावृत्तं सर्वं तन्नारदस्तदा}
{आचख्यौ भरतश्रेष्ठ यथावृत्तं पतत्रिणः}


\twolineshloka
{तच्छ्रुत्वा वैनतेयस्य कोपो हृदि समाविशत्}
{जगर्हे पन्नगान्सर्वान्मात्रा सह परंतपः}


\twolineshloka
{ततस्तु रोषाद्दुःखाच्च तूर्णमुत्पत्यक पक्षिराट्}
{जगाम यत्र माताऽस्य कृच्छ्रे महति वर्तते}


\twolineshloka
{तत्रापश्यत्ततो दीनां जटिलां मलिनां कृशाम्}
{तोयदेन प्रतिच्छन्नां सूर्याभामिव मातरम्}


\twolineshloka
{तस्य दुःखाच्च रोषाच्च नेत्राभ्यामश्रु चास्रवत्}
{प्रवृत्तिं च निवृत्तिं च पौरुषे प्रतितस्थुषः}


\twolineshloka
{अनुक्त्वा मातरं किञ्चित्पतत्रिवरपुङ्गवः}
{कद्रूमेव स धर्मात्मा वचनं प्रत्यभाषत}


\twolineshloka
{यदि धर्मेण मे माता जिता यद्यप्यधर्मतः}
{ज्येष्ठा त्वमसि मे माता धर्मः सर्वः स मे मतः}


\twolineshloka
{इयं तु मे स्यात्कृपणा मयि पुत्रेऽम्ब दुःखिता}
{अनुजानीहि तां साधु मत्कृते धर्मदर्शिनि}


\twolineshloka
{कद्रूः श्रुत्वास्य तद्वाक्यं वैनतेयस्य धीमतः}
{उवाच वाक्यं दुष्प्रज्ञा परीता दुःखमूर्च्छिता}


\twolineshloka
{नाहं तव न ते मातुर्वैनतेय कथञ्चन}
{कुर्यां प्रियमनिष्टात्मा मां ब्रवीषि खग द्विजा}


\threelineshloka
{तां तदा ब्रुवतीं वाक्यमनिष्टं क्रूरभाषिणीम्}
{दारुणां सूनृताभिस्तामनुनेतुं प्रचक्रमे ॥गरुड रुवाच}
{}


\threelineshloka
{ज्येष्ठा त्वमसि कल्याणि मातुर्मे भामिनि प्रिया}
{सोदर्यी मम चासि त्वं ज्येष्ठा माता न संशयः ॥ 7--24-22xकद्रूरुवाच}
{}


\twolineshloka
{विहङ्गम यथाकामं गच्छ कामगम द्विज}
{सूनृताभिस्त्वया माता नादासी शक्यमण्डज}


\twolineshloka
{अमृतं यद्याहरेस्त्वं विहङ्गं जननीं तव}
{अदासीं मम पश्येमां वैनतेय न संशयः}


\chapter{अध्यायः २५}
\twolineshloka
{तथेत्युक्त्वा तु विहगः प्रतिज्ञाय महाद्युतिः}
{अमृताहरणे वाचं ततः पितरमब्रवीत्}


\twolineshloka
{कामं वै सूनृता वाचो विसृज्य च मुहुर्मुहुः}
{यच्चाप्यनुज्ञां मातुर्वै न च सा ह्यनुमन्यते}


\twolineshloka
{सा मा बहुविधा वाचो वज्रकल्पा विसृज्य वै}
{भगवन्विनता दासी मम माता महाद्युते}


\twolineshloka
{कद्रूः प्रेषयते चैव दासीयमिति चाब्रवीत्}
{आहरामृतमित्येव मोक्ष्यते विनता ततः}


\threelineshloka
{सोहं मातृविमोक्षार्थमाहरिष्य इति ब्रुवन्}
{अमृतं प्रार्थितस्तूर्णमाहर्तुं प्रतिनन्द्य वै ॥पितोवाच}
{}


\twolineshloka
{अमृतं तात दुष्प्रापं देवैरपि कुतस्त्वया}
{रक्ष्यते हि भृशं पुत्र रक्षिभिस्तन्निबोध मे}


\twolineshloka
{गुप्तमद्भिर्भृशं साधु सर्वतः परिवारितम्}
{अनन्तरमथो गुप्तं ज्वलता जातवेदसा}


\twolineshloka
{ततः शतसहस्राणि अयुतान्यर्बुदानि च}
{रक्षन्त्यमृतमत्यन्तं किङ्करा नाम राक्षसाः}


\twolineshloka
{तेषां शक्त्यृष्टिशूलांश्च शतघ्न्यः पट्टसास्तथा}
{आयुधा रक्षिणां तात वज्रकल्पाः शिलास्तथा}


\twolineshloka
{ततो जालेन महता अवनद्धं समन्ततः}
{अयस्मयेन वै तात वृत्रहन्तुः स्म शासनात्}


\fourlineindentedshloka
{तत्त्वमेवंविधं तात कथं प्रार्थयसेऽमृतम्}
{सुरक्षितं वज्रभृतां वैनतेय विहङ्गम}
{इन्द्रेण देवैर्नागैश्च खड्गैर्गिरिजलादिभिः ॥गरुड उवाच}
{}


\twolineshloka
{पुत्रगृद्ध्या ब्रवीष्येतच्छृणु तात विनिश्चयम्}
{बलवानुपायवानस्मि भूयः किं करवाणि ते}


\twolineshloka
{तमब्रवीत्पिता हृष्टः प्रहसन्वै पुनः पुनः}
{यदि तौ भक्षयेस्तात क्रूरौ कच्छपवारणौ}


\threelineshloka
{तथा बलममेयं ते भविता तन्न संशयः}
{अमृतस्यैव चाहर्ता भविष्यसि न संशयः ॥गरुड उवाच}
{}


\threelineshloka
{क्व तौ क्रूरौ महाभाग वर्तेते हस्तिकच्छपौ}
{भक्षयिष्याम्यहं तात बलस्याप्यायनं प्रति ॥कश्यप उवाच}
{}


\twolineshloka
{पर्वतो वै समुद्रान्ते नभः स्तब्ध्वेव तिष्ठति}
{उरगो नाम दुष्प्रापः पुरा देवगणैरपि}


\twolineshloka
{गोरुतानि स विस्तीर्णः पुष्पितद्रुमसानुमान्}
{तत्र पन्थाः कृतस्तात कुञ्जरेण बलीयसा}


\twolineshloka
{गोरुतान्युच्छ्रयस्तस्य नव सप्त च पुत्रक}
{गच्छताऽगच्छता चैव क्षपितः स महागिरिः}


\twolineshloka
{तावान्भूमिसमस्तात कृतः पन्थाः समुत्थितः}
{तेन गत्वा स मातङ्गः पिपासुर्युद्धमिच्छति}


\twolineshloka
{तमतीत्य तु शैलेन्द्रं ह्रदः कोकनदायुतः}
{कनकेति च विख्यातस्तत्र कूर्मो महाबलः}


\twolineshloka
{गोरुतानि स विस्तीर्णः कच्छपः कुञ्जरश्च सः}
{आयामतश्चापि समौ तेजोबलसमन्वितौ}


\twolineshloka
{पुनरावृत्तिमापन्नौ तावेतौ मधुकैटभौ}
{जन्मान्तरे विप्रमूढौ परस्परवधैषिणौ}


\twolineshloka
{यदा स नागो व्रजति पिपासुस्तं जलाशयम्}
{तदैनं कच्छपो रोषात्प्रतियाति महाबलः}


\twolineshloka
{नखैश्च दशनैश्चापि निमज्योन्मज्य वाऽसकृत्}
{विददाराग्रहस्तेन कुञ्जरं तं जलेचरः}


\twolineshloka
{नागराडपि तोयार्थी पिपासुश्चरणैरपि}
{अग्रहस्तेन दन्ताभ्यां निवारयति वारिजम्}


\twolineshloka
{स तु तोयादनुत्तिष्ठन्वारिजो गजयूथपम्}
{नखैश्च दशनैश्चैव द्विरदं प्रतिषेधति}


\twolineshloka
{निवारितो गजश्रेष्ठः पुनर्गच्छति स्वं वनम्}
{पिपासुः क्लिन्नहस्ताग्रो रुधिरेण समुक्षितः}


\twolineshloka
{तौ गच्छ सहितौ पुत्र यदि शक्नोषि भक्षय}
{न तौ पृथक्तया शक्यावप्रमत्तौ बले स्थितौ'}


\chapter{अध्यायः २६}
\twolineshloka
{कथं तौ भगवञ्शक्यौ मया वारणकच्छपौ ॥युगपद्गृहीतं तं मे त्वमुपायं वक्तुमर्हसि ॥कश्यय उवाच}
{}


\twolineshloka
{योद्धुकामे गजे तस्मिन्मुहूर्तं स जलेचरः}
{उत्तिष्ठति जलात्तूर्णं योद्धुकामः पुनःपुनः}


\threelineshloka
{जलजं निर्गतं तात प्रमत्तं चैव वारणम्}
{ग्रहीष्यसि पतङ्गेश नान्यो योगोऽत्र विद्यते ॥भीष्म उवाच}
{}


\twolineshloka
{इत्येमुक्तो विहगस्तद्गत्वा वनमुत्तमम्}
{ददर्श वारणेन्द्रं तं मेघाचलसमप्रभम्}


\twolineshloka
{तां स नागो गिरेर्वीथिं सम्प्राप्त इव भारतः}
{स तं दृष्ट्वा महाभागः सम्प्रहृष्टतनूरुहः}


% Check verse!
बिभक्षयिषतो राजन्दारुणस्य महात्मनः ॥मातङ्गं कच्छपं चैव प्रहर्षः सुमहानभूत्
\twolineshloka
{अथ वेगेन महता खेचरः स महाबलः}
{सङ्कुच्य सर्वगात्राणि कृच्छ्रेणैवान्वपद्यत}


\twolineshloka
{तथा गत्वा तमध्वानं वारणप्रवरो बली}
{निशश्वास महाश्वासः श्रमाद्विश्रमणाय च}


\twolineshloka
{तस्य निश्वासवातेन मदगन्धेन चैव ह}
{उदतिष्ठन्महाकूर्मो वारणप्रतिषेधकः}


\twolineshloka
{तयोः सुतुमुलं युद्धं ददर्श पतगेश्वरः}
{कच्छपेन्द्रद्विरदयोरिन्द्रप्रह्वादयोरिव}


\twolineshloka
{स्पृशन्तमग्रहस्तेन तोयं वारणयूथपम्}
{दन्तैर्नस्तैश्च जलजो वारयामास भारत}


\twolineshloka
{जलजं वारणोऽप्येवं चरणैः पुष्करेण च}
{प्रत्यषेधन्निमज्जन्तमुन्मज्जन्तं तथैव च}


\twolineshloka
{मुहूर्तमभवद्युद्धं तयोर्भीमप्रदर्शनम्}
{अथ तस्माज्जलाद्राजन्कच्छपः स्थलमास्थितः}


\twolineshloka
{स तु नागः प्रभग्रोऽपि पिपासुर्न न्यवर्तत}
{तोयगृध्नुः शनैस्तर्षादपासर्पत पृष्ठतः}


\twolineshloka
{तं दृष्ट्वा जलजस्तूर्णमपसर्पन्तमाहवात्}
{अभिदुद्राव वेगेन वज्रयाणिरिवासुरम्}


\twolineshloka
{तं रोषात्स्थलमुत्तीर्णमसम्प्राप्तं गजोत्तमम्}
{उभावेव समस्तौ तु जग्राह विनतासुतः}


\twolineshloka
{चरणेन तु सव्येन जग्राह स गजोत्तमम्}
{प्रस्पन्दमानं बलवान्दक्षिणेन तु कच्छपम्}


\twolineshloka
{उत्पपात ततस्तूर्णं पन्नगेन्द्रनिषूदतः}
{दिवं खं च समुत्पत्य पक्षाभ्यामपराजितः}


\twolineshloka
{तेन चोत्पतता तूर्णं सङ्गृहीतौ नखैर्भृशम्}
{वज्रगर्भैः सुनिशितैः प्राणांस्तूर्णं मुमोचतुः}


\twolineshloka
{तौ गृह्य बलवांस्तूर्णं स्रस्तपादशिरोधरौ}
{विवल्गन्निव खे क्रीडन्खेचरोऽभिजगाम ह}


\twolineshloka
{अत्तुकामस्ततो वीरः पृथिव्यां पृथिवीपते}
{निरैक्षत न चापश्यद्द्रुमं पर्याप्तमासितुम्}


\twolineshloka
{नैमिषं त्वथ सम्प्राप्य देवारण्यं महाद्युतिः}
{अपश्यत द्रुमं कञ्चिच्छाखास्कन्धसमावृतम्}


\twolineshloka
{हिमवच्छिखरप्रख्यं योजनद्वयमुच्छ्रितम्}
{परिणाहेन राजेन्द्र नल्वमात्रं समन्ततः}


\twolineshloka
{तस्य शाखाऽभवत्काचिदायता पञ्चयोजनम्}
{दृढमूला दृढस्कन्धा वज्रपत्रसमाचिता}


\twolineshloka
{तत्रोपविष्टः सहसा वैनतेयो निगृह्य तौ}
{अत्तुकामस्ततः शाखा तस्य वेगादवापतत्}


\twolineshloka
{तां पतन्तीमभिप्रेक्ष्य प्रेक्ष्य चर्षिगणानधः}
{आसीनान्वसुभिः सार्धं सत्रेण जगतीपते}


\twolineshloka
{वैखानसान्नाम यतीन्वालखिल्यगणानपि}
{तत्र भीराविशत्तस्य पतगेन्द्रस्य भारत}


\threelineshloka
{तान्दृष्ट्वा स यतींस्तत्र समासीनान्सुरैः सह}
{तुण्डेन गृह्य तां शाखामुत्पपात खगेश्वरः}
{तौच पक्षी भुजङ्गाशो व्योम्नि क्रीडन्निवाव्रजत्}


\threelineshloka
{तं दृष्ट्वा गुरुसम्भारं प्रगृह्योत्पतितं खगम्}
{ऋषयस्तेऽब्रुवन्सर्वे गरुडोऽयमिति स्म ह}
{न त्वन्यः क्षमते कश्चिद्यथाऽयं वीर्यवान्खगः}


\twolineshloka
{असौ यच्छति धर्मात्मा गुरुभारसमन्वितः}
{अयं क्रीडन्निवाकाशे तस्माद्गरुड एव सः}


\twolineshloka
{एवं ते समयं सर्वे वसवश्च दिवौकसः}
{अकार्षुः पक्षिराजस्य गरुडेत्येव नाम ह}


\twolineshloka
{स पक्षी पृथिवीं सर्वां परिधावंस्ततस्ततः}
{मुमुक्षः शाखिनः शाखां न स्म देशमपश्यत}


\twolineshloka
{स वाचमशृणोद्दिव्यामुपर्युपरि जल्पतः}
{देवदूतस्य विस्पष्टमाभाष्य गरुडेति च}


\twolineshloka
{वैनतेय कुविन्देषु समुद्रान्ते महाबल}
{पात्यतां शाखिनः शाखा न हि ते धर्मनिश्चयाः}


\twolineshloka
{तच्छ्रुत्वा गरुडस्तूर्णं जगाम लवणाम्भसः}
{उद्देशं यत्र ते मन्दाः कुविन्दाः पापकर्मिणः}


\twolineshloka
{ततो गत्वा ततः शाखां मुमोच पततांवरः}
{तया हता ********* दास्तदा षड्विंशतो नृप}


\twolineshloka
{स देशो राजशार्दूल ख्यातः परमदारुणः}
{शाखापतग इत्येव कुविन्दानां महात्मनाम् ॥'}


\chapter{अध्यायः २७}
\threelineshloka
{हत्वा तं पक्षिशार्दूलः कुविन्दानां जनं व्रती}
{उपोपविश्य शैलाग्रे भक्षयामास तावुभौ}
{वारणं कच्छर्प चैव संहृष्टः स पतत्रिराट्}


\threelineshloka
{तयोः स रुधिरं पीत्वा मेदसी च परन्तप}
{संहृष्टः यततांश्रेष्ठो लब्ध्वा बलमनुत्तमम्}
{जंगाम देवराजस्य भवनं पन्नगाशनः}


\twolineshloka
{तं प्रणम्य महात्मानं पावकं विस्फुलिङ्गिनम्}
{रात्रिदिवं प्रज्वलितं रक्षार्थममृतस्य ह}


\twolineshloka
{तं दृष्ट्वा विहगेन्द्रस्य भयं तीव्रमुपाविशत्}
{नतु तोयान्न रक्षिभ्यो भयमस्योपपद्यते}


\twolineshloka
{पक्षित्वमात्मनो दृष्ट्वा ज्वलन्तं च हुताशनम्}
{पितामहमथो गत्वा ददर्श भुजगाशनः}


\twolineshloka
{तं प्रणम्य महात्मानं गरुडः प्रयताञ्जलिः}
{प्रोवाच तदसन्दिग्धं वचनं पन्नगेश्वरः}


\twolineshloka
{उद्यतं गुरुकृत्ये मां भगवन्धर्मनिश्चितम्}
{विमोक्षणार्थं मातुर्हि दासभावादनिन्दितम्}


\twolineshloka
{कद्रूसकाशममृतं मयाहर्तव्यमीश्वर}
{तदा मे जननी देव दासभावात्प्रमोक्ष्यते}


\twolineshloka
{तत्रामृतं प्रज्वलितो नित्यमीश्वर रक्षति}
{हिरण्यरेता भगवान्पाकशासनशासनात्}


\twolineshloka
{तत्र मे देवदेवेश भयं तीव्रमथाविशत्}
{ज्वलन्तं पावकं दृष्ट्वा पक्षित्वं चात्मनः प्रभो}


\twolineshloka
{समतिक्रमितुं शक्यः कथं स्यात्पावको मया}
{तस्याभ्युपायं वरद वक्तुमीशोऽसि मे प्रभो}


\twolineshloka
{तमब्रवीन्महाभाग तप्यमानं विहङ्गमम्}
{अग्नेः संशमनोपायमुत्सयन्त पुनःपुनः}


\twolineshloka
{पयसा शाम्यते वत्स सर्पिषा च हुताशनः}
{शरीरस्थोपि भूतानां किं पुनः प्रज्वलन्भुवि}


\twolineshloka
{नवनीतं पयो वाऽपि पावके त्वं समादधेः}
{ततो गच्छ यथाकामं न त्वा धक्ष्यति पावकः ॥'}


\chapter{अध्यायः २८}
\twolineshloka
{पितामहवचः श्रुत्वा गरुडः पततांवरः}
{जगाम गोकुलं किञ्चिन्नवनीतजिहीर्षया}


\twolineshloka
{नवनीतं तथाऽपश्यन्मथितं कलशे स्थितम्}
{तदादाय ततोऽगच्छद्यतस्तद्रक्ष्यतेऽमृतम्}


\twolineshloka
{स तत्र गत्वा पतगस्तिर्यक्तोयं महाबलः}
{हुताशनमपक्रम्य नवनीतमपातयत्}


\twolineshloka
{सो र्चिष्मान्मन्दवेगोऽभूत्सर्पिषा तेन तर्पितः}
{धूमकेतुर्न जज्वाल धूममेव ससर्ज ह}


\twolineshloka
{तमतीत्याशु गरुडो हृष्टात्मा जातवेदसम्}
{रक्षांसि समतिक्रामत्पक्षवातेन पातयन्}


\twolineshloka
{ते पतन्ति शिरोभिस्च जानुभिश्चरणैस्तथा}
{उत्सृज्य शस्त्रावरणं पक्षिपक्षसमाहताः}


\twolineshloka
{उत्प्लुत्य चावृतान्नागान्हत्वा चक्रं व्यतीत्य च}
{अरान्तरेण शिरसा भित्त्वा जालं समाद्रवत्}


\twolineshloka
{स भित्त्वा शिरसा जालं वज्रवेगसमो बली}
{उज्जहार ततः शीघ्रममृतं भुजगाशनः}


\twolineshloka
{तदादायाद्रवच्छीघ्रं गरुडः श्वसनो यथा}
{अथ सन्नाहमकरोद्वृत्रहा विबुधैः सह}


\twolineshloka
{ततो मातलिसंयुक्तं हरिभिः स्वर्णमालिभिः}
{आरुरोह रथं शीघ्रं सूर्याग्निसमतेजसम्}


\twolineshloka
{सोऽभ्यद्रवत्पक्षिराजं वृत्रहा पाकशासनः}
{उद्यम्य निशितं वज्रं वज्रहस्तो महाबलः}


\twolineshloka
{तथैव गरुडो राजन्वज्रहस्तं समाद्रवत्}
{ततो वै मातलिं प्राह शीघ्रं वाहय वाजिनः}


\twolineshloka
{अथ दिव्यं महाघोरं गरुडाय ससर्ज ह}
{वज्रं सहस्रनयनस्तिग्मवेगपराक्रमः}


\twolineshloka
{उत्सिसृक्षन्तमाज्ञाय वज्रं वै त्रिदशेश्वरम्}
{तूर्णं वेगपरो भूत्वा जगाम पततांवरः}


\twolineshloka
{पितामहनिसर्गेण ज्ञात्वा लब्धवरः खगः}
{आयुधानां वरं वज्रमथ शक्रमुवाच ह}


\twolineshloka
{वृत्रहन्नेष वज्रस्ते वरो लब्धः पितामहात्}
{अतः सम्मानमाकाङ्क्षन्मुञ्चाम्येकं तनूरुहम्}


\twolineshloka
{एतेनायुधराजेन यदि शक्तोसि वृत्रहन्}
{हन्यास्त्वं परया शक्तया गच्छाम्यहमनामयः}


\twolineshloka
{तत्तु तूर्णं तदा वज्रं स्वेन वेगेन भारत}
{जघान परया शक्त्या न चैनमदहद्भृशम्}


\twolineshloka
{ततो देवर्षयो राजन्गच्छन्तो वै विहायसा}
{दृष्ट्वा वज्रं विवक्तं तं पक्षइपर्णेऽब्रुवन्वचः}


\twolineshloka
{सुपर्णः पक्षिगरुडो यस्य पर्णे वरायुधम्}
{विषक्तं देवराजस्य वृत्रहन्तुः सनातनम्}


\twolineshloka
{एवं सुपर्णो विहगो वैनतेयः प्रतापवान्}
{ऋषयस्तं विजानन्ति चाग्नेयं वैष्णवं पुनः}


\twolineshloka
{वेदाभिष्टुतमत्यर्थं स्वर्गमार्गफलप्रदम्}
{तनुपर्णं सुपर्णस्य जगृहुर्बर्हिणस्तथा}


% Check verse!
मयूराविस्मिताः सर्वे आद्रवन्ति स्म वज्रिणम् ॥'
\chapter{अध्यायः २९}
\twolineshloka
{ततो व्रज्रं सहस्राक्षो दृष्ट्वा सक्तं वरायुधम्}
{ऋषींश्च दृष्ट्वा सहसा सुपर्णमिदमब्रवीत्}


\threelineshloka
{न ते सुपर्ण पश्यामि प्रभावं तेन योधये}
{इत्युक्ते न मया रक्षा शक्या कर्तुमतोऽन्यथा}
{इदं वज्रं मया सार्धं निवृत्तं हि यथागतम्}


\threelineshloka
{ततः सहस्रनयने निवृत्ते गरुडस्तथा}
{कद्रूमभ्यगमत्तूर्णममृतं गृह्य भारतः ॥गरुड उवाच}
{}


\threelineshloka
{तदाहृतं मया शीघ्रममृतं जननीकृते}
{अदासी सा भवत्वद्य विनता धर्मिचारिणी ॥कद्रूरुवाच}
{}


\threelineshloka
{स्वागतं स्वाहृतं चेदममृतं विनतात्मज}
{अदासी जननी तेऽद्य पुत्र कामवशा शुभा}
{}


\twolineshloka
{एवमुक्ते तदा सा च सम्प्राप्ता विनता गृहम् ॥उपनीय यथान्यायं विहगो बलिनांवरः}
{}


\twolineshloka
{स्मृत्वा निकृत्या विजयं मातुः सम्प्रतिपद्य च ॥वधं च भुजगेन्द्राणां ये वालास्तस्य वाजिनः}
{}


\twolineshloka
{बभूवुरसितप्रख्या निकृत्या वै जितं त्वया ॥तामुवाच ततो न्याय्यं विहगो बलिनांवरः}
{}


\twolineshloka
{उञ्जहारामृतं तूर्णमुत्पपात च रंहसा ॥तद्गृहीत्वाऽमृतं तूर्णं प्रयान्तमपराजितम्}
{}


\threelineshloka
{किमर्थं वैनतेय त्वमाहृत्यामृतमुत्तमम्}
{पुनर्हरति दुर्बुद्धे मा जातु वृजितं कृथाः ॥सुपर्णि उवाच}
{}


\twolineshloka
{अमृताहरणं मेऽद्य यत्कृतं जननीकृते}
{भवत्या वचनादेतदाहरामृतमित्युत}


\twolineshloka
{आहृतं तदिदं शीघ्रं मुक्ता च जननी मम}
{हराम्येष पुनस्तत्र यत एतन्मयाऽऽहृतम्}


\fourlineindentedshloka
{यदि मां भवती ब्रूयादमृतं मे च दीयताम्}
{तथा कुर्यां न वा कुर्यां न हि त्वममृतक्षमा}
{मया धर्मेण सत्येन विनता च समुद्धृता ॥भीष्म उवाच}
{}


\threelineshloka
{ततो गत्वाऽथ गरुडः पुरन्दरमुवाच ह}
{इदं मया वृत्रहन्तर्हृतं तेऽमृतमुत्तमम्}
{मात्रर्थे हि तथैवेदं गृहाणामृतमाहृतम्}


\twolineshloka
{माता च मम देवेश दासीत्वमुपजग्मुषी}
{भुजङ्गमानां मातुर्वै सा मुक्ताऽमृतदर्शनात्}


\twolineshloka
{एतच्छ्रुत्वा सहस्राक्षः सुपर्णमनुमन्यते}
{उवाच च मुदा युक्तो दिष्ट्यादिष्ट्येति वासवः}


\twolineshloka
{क्रषयो ये सहस्राक्षमुपासन्ति सुरैः सह}
{ते सर्वे च मुदा युक्ता विश्वेदेवास्तथैव च}


\twolineshloka
{ततस्तमृषयः सर्वे देवाश्च भरतर्षभ}
{ऊचुः पुरन्दरं दृष्टा गरुडो लभतां वरम्}


\twolineshloka
{ततः शचीपतिर्वाक्यमुवाच प्रहसन्निव}
{जनिष्यति हृषीकेशः स्वयमेवैष पक्षिराट्}


\twolineshloka
{केशवः पुण्डरीकाक्षः शूरपुत्रस्य वेश्मनि}
{स्वयं धर्मस्य रक्षार्थं विभज्य भुजगाशनः}


\twolineshloka
{एष ते पाण्डवश्रेष्ठ गरुडोऽथ पतत्रिराट्}
{सुपर्णो वैनतेयश्च कीर्तितो भद्रमस्तु ते}


\twolineshloka
{तदेतद्भरतश्रेष्ठ मयाऽऽख्यानं प्रकीर्तितम्}
{सुपर्णस्य महाबाहो किं भूयः कथयामि ते ॥'}


\chapter{अध्यायः ३०}
\twolineshloka
{ब्राह्मणानां तु ये लोके प्रतिश्रुत्य पितामह}
{न प्रयच्छन्ति लोभात्ते के भवन्ति महामते}


\threelineshloka
{एतन्मे तत्वतो ब्रूहि धर्मे धर्मभृतांवर}
{प्रतिश्रुत्य दुरात्मानो न प्रयच्छन्ति ये नराः ॥भीष्म उवाच}
{}


\twolineshloka
{यो न दद्यात्प्रतिश्रुत्य स्वल्पं वा यदि वा बहु}
{आशास्तस्य हताः सर्वाः क्लीबस्येव प्रजाफलम्}


\twolineshloka
{यां रात्रिं जायते पापो यां च रात्रिं विनश्यति}
{एतस्मिन्नन्तरे यद्यत्सुकृतं तस्य भारत}


\twolineshloka
{यच्च तस्य हुतं किञ्चिद्दत्तं वा भरतर्षभ}
{तपस्तप्तमथो वाऽपि सर्वं तस्योपहन्यते}


\twolineshloka
{अथैतद्वचनं प्राहुर्धर्मशास्त्रविदो जनाः}
{निशास्य भरतश्रेष्ठ बुद्ध्या परमयुक्तया}


\twolineshloka
{अपि चोदाहरन्तीमं धर्मशास्त्रविदो जनाः}
{अश्वानां श्यामकर्णानां सहस्रेण स मुच्यते}


\twolineshloka
{अत्रैवोदाहरन्तीममितिहासं पुरातनम्}
{सृगालस्य च संवादं वानरस्य च भारत}


\threelineshloka
{तौ सखायौ पुरा ह्यास्तां मानुषत्वे परंतप}
{अन्यां योनिं समापन्नौ सृगालीं वानरीं तथा}
{सम्भाषणात्ततः सख्यं तत्रतत्र परस्परम्}


\twolineshloka
{ततः परासून्खादन्तं सृगालं वानरोऽब्रवीत्}
{श्मशानमध्ये सम्प्रेक्ष्य पूर्वजातिमनुस्मरन्}


\twolineshloka
{किं त्वया पापकं पूर्वं कृतं कर्म सुदारुणम्}
{यस्त्वं श्मशाने मृतकान्पूतिकानत्सि कुत्सितान्}


\twolineshloka
{एवमुक्तः प्रत्युवाच सृगालो वानरं तदा}
{ब्राह्मणस्य प्रतिश्रुत्य न मया तदुपाहृतम्}


\threelineshloka
{तत्कृते पापिकां योनिमापन्नोस्मि प्लवङ्गम}
{तस्मादेवंविधं भक्ष्यं भक्षयामि बुभुक्षितः ॥भीष्म उवाच}
{}


\threelineshloka
{सृगालो वानरं प्राह पुनरेव नरोत्तम}
{किं त्वया पातकं कर्म कृतं येनासि वानरः ॥वानर उवाच}
{}


\fourlineindentedshloka
{स चाप्याह फलाहारो ब्राह्मणानां प्लवङ्गमः}
{तस्मान्न ब्राह्मणस्वं तु हर्त्तव्यं विदुषा सदा}
{सीमाविवादे मोक्तव्यं दातव्यं च प्रतिश्रुतम् ॥भीष्म उवाच}
{}


\twolineshloka
{इत्येतद्ब्रुवतो राजन्ब्राह्मणस्य मया श्रुतम्}
{कथां कथयतः पुण्यां धर्मज्ञस्य पुरातनीम्}


\twolineshloka
{श्रुतं चापि मया भूयः कृष्णस्यापि विशांपते}
{कथां कथयतः पूर्वं ब्राह्मणं प्रति पाण्डव}


\twolineshloka
{न हर्तव्यं विप्रधनं क्षन्तव्यं तेषु नित्यशः}
{बालाश्च नावमन्तव्या दरिद्राः कृपणा अपि}


\twolineshloka
{एवमेव च मां नित्यं ब्राह्मणाः संदिशन्ति वै}
{प्रतिश्रुतं भवेद्देयं नाशा कार्या द्विजोत्तमे}


\twolineshloka
{ब्राह्मणो ह्याशया पूर्वं कृतया पृथिवीपते}
{सुसमिद्धो यता दीप्तः पावकस्तद्विधः स्मृतः}


\twolineshloka
{यं निरीक्षेत सङ्क्रुद्ध आशया पूर्वजातया}
{प्रदहेच्च हितं राजन्कक्षमक्षय्यभुग्यथा}


\twolineshloka
{स एव हि यदा तुष्टो वचसा प्रतिनन्दति}
{भवत्यगदसंकाशो विषये तस्य भारत}


\twolineshloka
{पुत्रान्पौत्रान्पशूंश्चैव बान्धवान्सचिवांस्तथा}
{पुरं जनपदं चैव शान्तिरिष्टेन पोषयेत्}


\twolineshloka
{एतद्धि परमं तेजो ब्राह्मणस्येह दृश्यते}
{सहस्रकिरणस्येव सवितुर्धरणीतले}


\twolineshloka
{तस्माद्दातव्यमेवेह प्रतिश्रुत्य युधिष्ठिर}
{यदीच्छेच्छोभनां जातिं प्राप्तुं भरतसत्तम}


\twolineshloka
{ब्राह्मणस्य हि दत्तेन ध्रुवं स्वर्गो ह्यनुत्तमः}
{शक्यः प्राप्तुं विशेषेण दानं हि महती क्रिया}


\twolineshloka
{इतो दत्तेन जीवन्ति देवताः पितरस्था}
{तस्माद्दानानि देयानि ब्राह्मणेभ्यो विजानता}


\twolineshloka
{महद्धि भरतश्रेष्ठ ब्राह्मणस्तीर्थमुच्यते}
{वेलायां न तु कस्यां चिद्गच्छेद्विप्रो ह्यपूजितः}


\chapter{अध्यायः ३१}
\twolineshloka
{मित्रसौहार्दयोगेन उपदेशं करोति यः}
{जात्याऽधरस्य राजर्षे दोषस्तस्य भवेन्न वा}


\threelineshloka
{एतदिच्छामि तत्त्वेन व्याख्यातुं वै पितामह}
{सूक्ष्मा गतिर्हि धर्मस्यि यत्र मुह्यन्ति मानवाः ॥भीष्म उवाच}
{}


\threelineshloka
{अत्र ते वर्तयिष्यामि शृणु राजन्यथाक्रमम्}
{`मदुक्तं वचनं राजन्यथान्यायं यथागमम्}
{'ऋषीणां वदतां पूर्वं श्रुतमासीद्यथा पुरा}


\twolineshloka
{उपदेशो न कर्तव्यो जातिहीनस्य कस्य चित्}
{उपदेशे महान्दोष उपाध्यायस्य भाष्यते}


\twolineshloka
{`नाध्यापयेच्छूद्रमिह तथा नैव च याजयेत्}
{'निदर्शनमिदं राजञ्शृणु मे भरतर्षभ}


\twolineshloka
{दुरुक्तवचने राजन्यथापूर्वं युधिष्ठिर}
{ब्रह्माश्रमपदे वृत्तं पार्श्वे हिमवतः शुभे}


\threelineshloka
{तत्राश्रमपदं पुण्यं नानावृक्षगणायुतम्}
{नानागुल्मलताकीर्णं मृगद्विजनिषेवितम्}
{सिद्धचारणसंघुष्टं रम्यं पुष्पितकाननम्}


\twolineshloka
{व्रतिभिर्बहुभिः कीर्णं तापसैरुपशोभितम्}
{ब्राह्मणैश्च महाभागैः सूर्यज्वलनसन्निभैः}


\twolineshloka
{नियमव्रतसम्पन्नैः समाकीर्णं तपस्विभिः}
{दीक्षितैर्भरतश्रेष्ठ यताहारैः कृतात्मभिः}


\twolineshloka
{वेदाध्ययनघोषैश्च नादितं भरतर्षभ}
{वालखिल्यैश्च बहुभिर्यतिभिश्च निषेवितम्}


\twolineshloka
{तत्र कश्चित्समुत्साहं कृत्वा शूद्रो दयान्वितः}
{आगतो ह्याश्रमपदं पूजितश्च तपस्विभिः}


\twolineshloka
{तांस्तु दृष्ट्वा मुनिगणान्देवकल्पान्महौजसः}
{विविधां वहतो दीक्षां सम्प्राहृष्यत भारत}


\twolineshloka
{अथास्य बुद्धिरभावत्तापस्ये भरतर्षभ}
{ततोऽब्रवीत्कुलपतिं पादौ सङ्गृह्य भारत}


\twolineshloka
{भवत्प्रसादादिच्छामि धर्मं चर्तुं द्विजर्षभ}
{तस्मादभिगतं त्वं मां प्रव्राजयितुमर्हसि}


\threelineshloka
{वर्णावरोऽहं भगवञ्शूद्रो जात्याऽस्मि सत्तम}
{शुश्रूषां कर्तुमिच्छामि प्रपन्नाय प्रसीद मे ॥कुलपतिरुवाच}
{}


\twolineshloka
{न शक्यमिह शूद्रेण लिङ्गमाश्रित्य वर्तितुम्}
{आस्यतां यदि ते बुद्धिः शुश्रूषानिरतो भव}


\twolineshloka
{शुश्रूषया पराँल्लोकानवाप्स्यसि न संशयः ॥भीष्म उवाच}
{}


\twolineshloka
{एवमुक्तस्तु मुनिना स शूद्रोऽचिन्तयन्नृप}
{कथमत्र मया कार्यं श्रुद्धा धर्मपरा च मे}


\twolineshloka
{विज्ञातमेवं भवतु करिष्ये प्रियमात्मनः}
{गत्वाऽऽश्रमपदाद्दूरमुटजं कृतवांस्तु सः}


\twolineshloka
{तत्र वेदीं च भूमिं च देवतायतनानि च}
{निवेश्य भरतश्रेष्ठ नियमस्थोऽभवन्मुनिः}


\twolineshloka
{अभिषेकांश्च नियमान्देवतायतनेषु च}
{बलिं च कृत्वा हुत्वा च देवतां चाप्यपूजयत्}


\twolineshloka
{सङ्कल्पनियमोपेतः फलाहारो जितेन्द्रियः}
{नित्यं सन्निहिताभिस्तु ओषधीभिः फलैस्तथा}


\twolineshloka
{अतिथीन्पूजयामास यथावत्समुपागतान्}
{एवं हि सुमहान्कालो व्यत्यक्रामत तस्य वै}


\twolineshloka
{अथास्य मुनिरागच्चत्सङ्गत्या वै तमाश्रमम्}
{सम्पूज्य स्वागतेनर्षिं विधिवत्समतोषयत्}


\twolineshloka
{अनूकूलाः कथाः कृत्वा यथागतमपृच्छत}
{ऋषिः परमतेजस्वी धर्मात्मा संशितव्रतः}


\twolineshloka
{एवं सुबहुशस्तस्य शूद्रस्य भरतर्षभ}
{सोऽगच्छदाश्रममृषिः शूद्रं द्रष्टुं नरर्षभः}


\twolineshloka
{अथ तं तापसं शूद्रः सोऽब्रवीद्भरतर्षभ}
{पितृकार्यं करिष्यामि तत्र मेऽनुग्रहं कुरु}


\twolineshloka
{बाढमित्येव तं विप्र उवाच भरतर्षभ}
{शुचिर्भूत्वा स शूद्रस्तु तस्यर्षेः पाद्यमानयत्}


\twolineshloka
{अथ दर्भाश्च वन्यांश्च ओषधीर्भरतर्षभ}
{पवित्रमासनं चैव बृसीं चि समुपानयत्}


\twolineshloka
{अथ दक्षिणमावृत्य बृसीं चरमशैर्षिकीम्}
{कृतामन्यायतो दृष्ट्वा तं शूद्रमृषिरब्रवीत्}


\twolineshloka
{कुरुष्वैतां पूर्वशीर्षां भवांश्चोदङ्मुखः शुचिः}
{स च तत्कृतवाञ्शूद्रः सर्वं यदृषिरब्रवीत्}


\twolineshloka
{यथोपदिष्टं मेधावी दर्भार्घादि यथातथम्}
{हव्यकव्यविधिं कृत्स्नमुक्तं तेन तपस्विना}


\twolineshloka
{ऋषिणा पितृकार्येषु सदा धर्मपथे स्थितः}
{पितृकार्ये कृते चापि विसृष्टः स जगाम ह}


\threelineshloka
{अथ दीर्घस्य कालस्य स तप्यञ्शूद्रतापसः}
{वने पञ्चत्वमगमत्सुकृतेन च तेन वै}
{अजायत महाराज वंशे स च महाद्युतिः}


\twolineshloka
{तथैव स ऋषिस्तात कालधर्ममवाप ह}
{पुरोहितकुले विप्रः स जातोऽस्य वशानुगः}


\twolineshloka
{एवं तौ तत्र सम्भूतावुभौ शूद्रमुनी तदा}
{क्रमेण वर्धितौ चापि विद्यासु कुशलावुभौ}


\threelineshloka
{अथर्ववेदे वेदे च बभूवर्षिः सुनिष्ठितः}
{कल्पप्रयोगे चोत्पन्ने ज्योतिषे च परं गतः}
{साङ्ख्ये चैव परा प्रीतिस्तस्य चैवं व्यवर्धत}


\twolineshloka
{पितर्युपरते चापि कृतशौचस्तु पार्थिवः}
{अभिषिक्तः प्रकृतिभी राजपुत्रः स पार्थिवः}


% Check verse!
अभिषिक्तेन स ऋषिरभिषिक्तः पुरोहितः
\twolineshloka
{स तं पुरोधाय सुखमवसद्भरतर्षभः}
{राज्यं शशास धर्मेण प्रजाश्च परिपालयन्}


\threelineshloka
{पुण्याहवाचने नित्यं धर्मकार्येषु चासकृत्}
{उत्स्मयन्प्राहसच्चापि दृष्ट्वा राजा पुरोहितम्}
{एवं स बहुशो राजन्पुरोधसमुपाहसत्}


\twolineshloka
{लक्षयित्वा पुरोधास्तु बहुशस्तं नराधिपम्}
{उत्स्मयन्तं च सततं दृष्ट्वा तं मन्युमानभूत्}


\twolineshloka
{अथ शून्ये पुरोधास्तु सह राज्ञा समागतः}
{कथाभिरनुकूलाभी राजानं चाभ्योरचयत्}


\threelineshloka
{ततोऽब्रवीन्नरेन्द्रं स पुरोधा भरतर्षभ}
{वरमिच्छाम्यहं त्वेकं त्वया दत्तं महाद्युते ॥राजोवाच}
{}


\threelineshloka
{वराणां ते शतं दद्यां किं बतैकं द्विजोत्तम}
{स्नेहाच्च बहुमानाच्च नास्त्यदेयं हि मे तव ॥पुरोहित उवाच}
{}


\threelineshloka
{एकं वै वरमिच्छामि यदि तुष्टिसि पार्थिव}
{प्रतिजानीहि तावत्त्वं सत्यं यद्वद नानृतम् ॥भीष्म उवाचि}
{}


\threelineshloka
{बाढमित्येव तं राजा प्रत्युवाच युधिष्ठिर}
{यदि ज्ञास्यामि वक्ष्यामि अजानन्न तु संवदे ॥पुरोहित उवाच}
{}


\twolineshloka
{पुण्याहवाचने नित्यं धर्मकृत्येषु चासकृत्}
{शान्तिहोमेषु च सदा किं त्वं हससि वीक्ष्य मां}


\twolineshloka
{सव्रीडं वै भवति हि मनो मे हसता त्वया}
{कामया शापितो राजन्नान्यथावक्तुमर्हसि}


\threelineshloka
{भाव्यं हि कारणेनात्र न ते हास्यमकारणम्}
{कौतूहलं मे सुभृशं तत्त्वेन कथमस्व मे ॥राजोवाच}
{}


\twolineshloka
{एवमुक्ते त्वया विप्र यदवाच्यं भवेदपि}
{अवश्यमेव वक्तव्यं शृणुष्वैकमना द्विज}


\twolineshloka
{पूर्वदेहे यथा वृत्तं तन्निबोध द्विजोत्तम}
{जातिं स्मराम्यहं ब्रह्मन्नवधानेन मे शृणु}


\twolineshloka
{शुद्रोऽहमभवं पूर्वं तपसे कृतनिश्चयः}
{ऋषिरुग्रतपास्त्वं च तदाऽभूर्द्विजसत्तम}


\twolineshloka
{प्रीयता हि तदा ब्रह्मन्ममानुग्रहबुद्धिना}
{पितृकार्ये त्वया पूर्वमुपदेशः कृतोऽनघ}


\twolineshloka
{वृस्यां दर्भेषु हव्ये च कव्ये च मुनिसत्तम}
{एतेन कर्मदोषेण पुरोधास्त्वमजायथाः}


\twolineshloka
{अहं राजा च विप्रेन्द्र पश्य कालस्य पर्ययम्}
{मत्कृतस्योपदेशस्य त्वयाऽवाप्तमिदं फलम्}


\twolineshloka
{एतस्मात्कारणाद्ब्रह्मन्प्रहसे त्वां द्विजोत्तम}
{न त्वां परिभवन्ब्रह्मनप्रहसामि गुरुर्भवान्}


\twolineshloka
{विपर्ययेण मे मन्युस्तेन सन्तप्यते मनः}
{जातिं स्मराम्यहं तुभ्यमतस्त्वां प्रहसामि वै}


\twolineshloka
{एवं तवोग्रं हि तप उपदेशेन नाशितम्}
{पुरोहितत्वमुत्सृज्य यतस्व त्वं पुनर्भवे}


\threelineshloka
{इतस्त्वमधमामन्यां मा योनिं प्राप्स्यसे द्विज}
{गृह्यतां द्रविणं विप्र पूतात्मा भव सत्तम ॥भीष्म उवाच}
{}


\twolineshloka
{ततो विसृष्टो राज्ञा तु विप्रो दानान्यनेकशः}
{ब्राह्मणेभ्यो ददौ वित्तं भूमिं ग्रामांश्च सर्वशः}


\twolineshloka
{कृच्छ्राणि चीर्त्वा च ततो यथोक्तानि द्विजोत्तमैः}
{तीर्थानि चापि गत्वा वै दानानि विविधानि च}


\twolineshloka
{दत्त्वा गाश्चैव विप्रेभ्यः पूतात्माऽभवदात्मवान्}
{तमेव चाश्रमं गत्वा चचार विपुलं तपः}


\twolineshloka
{ततः सिद्धिं परां प्राप्तो ब्राह्मणो राजसत्तम}
{सम्मतस्चाभवत्तेषामाश्रमे तन्निवासिनाम्}


\twolineshloka
{एवं प्राप्तो महत्कृच्छ्रमृषिः सन्नृपसत्तम}
{ब्राह्मणेन न वक्तव्यं तस्माद्वर्णावरे जने}


\twolineshloka
{`वर्जयेदुपदेशं च सदैव ब्राह्मणो नृप}
{उपदेशं हि कुर्वाणो द्विजः कृच्छ्रमवाप्नुयात्}


\twolineshloka
{नेषितव्यं सदा वाचा द्विजेन नृपसत्तम}
{न च प्रवक्तव्यमिह किञ्चिद्वर्णावरे नरे ॥'}


\twolineshloka
{ब्राह्मणाः क्षत्रिया वैश्यास्त्रयो वर्णा द्विजातयः}
{एतेषु कथयन्राजन्ब्राह्मणो न प्रदुष्यति}


\twolineshloka
{तस्मात्सद्भिर्न वक्तव्यं कस्यचित्किंचिदग्रतः}
{सूक्ष्मा गतिर्हि धर्मस्य दुर्ज्ञेया ह्यकृतात्मभिः}


\twolineshloka
{तस्मान्मौनेन मुनयो दीक्षां कुर्वन्ति चादृताः}
{दुरुक्तस्य भयाद्राजन्नाभाषन्ते च किञ्चन}


\twolineshloka
{धार्मिका गुणसम्पन्नाः सत्यार्जवसमन्विताः}
{दुरुक्तवाचाभिहितैः प्राप्नुवन्तीह दुष्कृतम्}


\twolineshloka
{उपदेशो न कर्तव्यो ह्यज्ञात्वा यस्यकस्य चित्}
{उपदेशाद्धि तत्पापं ब्राह्मणः समवाप्नुयात्}


\twolineshloka
{विमृश्य तस्मात्प्राज्ञेन वक्तव्यं धर्ममिच्छता}
{सत्यानृतेन हि कृत उपदेशो हिनस्ति हि}


\twolineshloka
{वक्तव्यमिह पृष्टेन विनिश्चयविपर्ययम्}
{स चोपदेशः कर्तव्यो येन धर्ममवाप्नुयात्}


\twolineshloka
{एतत्ते सर्वमाख्यातमुपदेशकृते मया}
{महान्क्लेशो हि भवति तस्मान्नोपदिशेदिह}


\chapter{अध्यायः ३२}
\threelineshloka
{कीदृशे पुरुषे तात स्त्रीषु भरतर्षभ}
{श्रीः पद्मा वसते नित्यं तन्मे ब्रूहि पितामह ॥भीष्म उवाच}
{}


\twolineshloka
{अत्र ते वर्णयिष्यामि यथावृत्तं यथाश्रुतम्}
{रुक्मिणी देवकीपुत्रसन्निधौ पर्यपृच्छत}


\twolineshloka
{नारायणस्याङ्कगतां ज्वलन्तींदृष्ट्वा श्रियं पद्मसमानवक्त्राम्}
{कौतूहलाद्विस्मितचारुनेत्रापप्रच्छ माता मकरध्वजस्यष}


\twolineshloka
{कानीह भूतान्युपसेवसे त्वंसन्तिष्ठसे कानि च सेवसे त्वम्}
{तानि त्रिलोकेश्वरभूतकान्तेतत्त्वेन मे ब्रूहि महर्षिकन्ये}


\threelineshloka
{एवं तदा श्रीराभिभाष्यमाणादेव्या समक्षं गरुडध्वजस्य}
{उवाच वाक्यं मधुराभिधानंमनोहरं चन्द्रमुखी प्रसन्ना ॥श्रीरुवाच}
{}


\twolineshloka
{वसामि नित्यं सुभगे प्रगल्भेदक्षे नरे कर्मणि वर्तमाने}
{अक्रोधने देवपरे कृतज्ञेजितेन्द्रिये नित्यमुदीर्णसत्वे}


\twolineshloka
{नाकर्मशीले पुरुषे वसामिन नास्तिके साङ्करिके कृतघ्ने}
{न भिन्नवृत्ते न नृशंसवृत्तेन चाविनीते न गुरुष्वसूयके}


\twolineshloka
{ये चाल्पतेजोबलसत्त्वमानाःक्लिश्यन्ति कुप्यन्ति च यत्रतत्र}
{न चैव तिष्ठामि तथाविधेषुनरेषु सङ्गुप्तमनोरथेषु}


\threelineshloka
{यश्चात्मनि प्रार्थयते न किञ्चि-द्यश्च स्वभावोपहतान्तरात्मा}
{द्यश्च स्वभावोपहतान्तरात्मा}
{नरेषु नाहं निवसामि स्म्यक्}


\twolineshloka
{वसामि धर्मशीलेषु धर्मज्ञेषु महात्मसु}
{वृद्धसेविषु दान्तेषु सत्वज्ञेषु महात्मसु}


\twolineshloka
{अवन्ध्यकालेषु सदा दानशौचरतषु च}
{ब्रह्मचर्यतपोज्ञानगोद्विजातिप्रियेषु च}


\twolineshloka
{स्त्रीषु कान्तासु शान्तासु देवद्विजपरासु च}
{विशुद्धगृहभाण्डासु गोधान्याभिरतासु च}


\threelineshloka
{स्वधर्मशीलेषु च धर्मवित्सुवृद्धोपसेवानिरते च दान्ते}
{कृतात्मनि क्षान्तिपरे समर्थेक्षान्तासु दान्तासु तथाऽबलासु}
{सत्यस्वभावार्जवसंयुतासुवसागि देवद्विजपूजिकासु}


\twolineshloka
{प्रकीर्णभाण्डामनवेक्ष्यकारिणींसदा च भर्तुः प्रतिकूलवादिनीम्}
{}


\twolineshloka
{लोलामदक्षामवलेपिनीं चव्यपेतशौचां कलहप्रियां च}
{निद्राभिभूतां सततं शयाना-मेवंविधां स्त्रीं परिवर्जयामि}


\twolineshloka
{सत्यासु नित्यं प्रियदर्शनासुसौभाग्ययुक्तासु गुणान्वितासु}
{वसामि नारीषु पतिव्रतासुकल्याणशीलासु विभूषितासु}


\twolineshloka
{यानेषु कन्यासु विभूषणेषुयज्ञेषु मेघेषु च वृष्टिमत्सु}
{वसामि फुल्लासु च पद्मिनीषुनक्षत्रवीथीषु च शारदीषु}


\twolineshloka
{गजेषु गोष्ठेषु तथाऽऽसनेषुसरःसु फुल्लोत्पलपङ्कजेषु}
{नदीषु हंसस्वननादितासुक्रौञ्चावघुष्टस्वरशोभितासु}


\twolineshloka
{विकीर्णकूलद्रुमराजितासुतपस्विसिद्धद्विजसेवितासु}
{वसामि नित्यं सुबहूदकासुसिम्हैर्गजैश्चाकुलितोदकासु}


% Check verse!
मत्ते गजे गोवृषभे नरेन्द्रेसिम्हासने सत्पुरुषेषु नित्यम्
% Check verse!
यस्मिञ्चनो हव्यभुजं जुहोतिगोब्राह्मणं चार्चति देवताश्चकाले च पुष्पैर्बलयः क्रियन्तेतस्मिन्गृहे नित्यमुपैमि वासम्
\twolineshloka
{स्वाध्यायनित्येषु सदा द्विजेषुक्षत्रे च धर्माभिरते सदैव}
{वैश्ये च कृष्याभिरते वसामिशूद्रे च शुश्रूषणनित्ययुक्ते}


\twolineshloka
{नारायणे त्वेकमना वसामिसर्वेण भावेन शरीरभूता}
{तस्मिन्हि धर्मः सुमहान्निविष्टोब्रह्मिण्यता चात्र तथा प्रियत्वम्}


\twolineshloka
{नाहं शरीरेण वसामि देविनैवं मया शक्यमिहाभिधातुम्}
{भावेन यस्मिन्निवसामि पुंसिस वर्धते धर्मयशोर्थकामैः}


\chapter{अध्यायः ३३}
\twolineshloka
{प्रायश्चित्तं कृतघ्नानां प्रतिब्रूहि पितामह}
{मातॄः पितॄन्गुरुंश्चैव येऽवमन्यन्ति मोहिताः}


\threelineshloka
{ये चाप्यन्ते परे तात कृतघ्ना निरपत्रपाः}
{तेषां गतिं महाबाहो श्रोतुमिच्छामि तत्वतः ॥भीष्म उवाच}
{}


\threelineshloka
{कृतघ्नानां गतिस्तात नरके शाश्वतीः समाः}
{मातापितृगुरूणां च ये न तिष्ठन्ति शासने}
{क्रिमिकीटपिपीलेषु जायन्ते स्थावरेषु च}


\twolineshloka
{दुर्लभो हि पुनस्तेषां मानुष्ये पुनरुद्भवः}
{अत्राप्युदाहरन्तीममितिहासं पुरातनम्}


\twolineshloka
{वत्सनाभो महाप्राज्ञो महर्षिः संशितव्रतः}
{वल्मीकभूतो ब्रह्मर्षिस्तप्यते सुमहत्तपः}


\twolineshloka
{तस्मिंश्च तप्यति तपो वासवो भरतर्षभ}
{ववर्ष समुहद्वर्षं सविद्युत्स्तनयित्नुमान्}


\twolineshloka
{तत्र सप्ताहवर्षं तु मुमुचे पाकशासनः}
{निमीलिताक्षस्तद्वर्षं प्रत्यगृह्यत वै द्विजः}


\twolineshloka
{तस्मिन्पतति वर्षे तु शीतवातसमन्विते}
{विशीर्णध्वस्तशिखरो वल्मीकोऽशनिताडितः}


\twolineshloka
{ताड्यमाने ततस्तस्मिन्वत्सनाभे महात्मनि}
{कारुण्यात्तस्य धर्मः स्वमानृशंस्यमथाकरोत्}


\twolineshloka
{चिन्तयानस्य ब्रह्मर्षि तपन्तमतिधार्मिकम्}
{अनुरूपा मतिः क्षिप्रमुपजाता स्वभावजा}


\twolineshloka
{स्वं रूपं माहिषं कृत्वा ********* मनोहरम्}
{रक्षार्थं वत्सनामस्य चतुष्पादुपरिस्थितः}


\threelineshloka
{यदा त्वपगतं वर्षं वृष्टिवातसमन्वितम्}
{ततो महिषरुपेण धर्मो धर्मभृतांवर}
{शनैर्वल्मीकमुत्सृज्य प्राद्रवद्भरतर्षभ}


\twolineshloka
{स्थितेऽस्मिन्वृष्टिसम्पाते वीक्षते स्म महातपाः}
{दिशश्च विपुलास्तत्र गिरीणां शिखराणि च}


\twolineshloka
{दृष्ट्वा च पृथिवीं सर्वां सलिलेन परिप्लुताम्}
{जलाशयान्स तान्दृष्ट्वा विप्रः प्रमुदितोऽभवत्}


\twolineshloka
{अचिन्तयद्विस्मितश्चि विप्रः प्रमुदितोऽभवत् ॥अचिन्तयद्विस्तितश्च वर्षात्केनाभिरक्षितः}
{ततोऽपश्यत्स महिषमवस्थितमदूरतः}


% Check verse!
तिर्यग्योनावपि कथं दृश्यते धर्मवत्सलः
\twolineshloka
{अतो नु भद्रमहिषः शिलापट्टमवस्थितः}
{पीवरश्चैव बल्यश्च बहुमांसो भवेदयम्}


\twolineshloka
{तस्य बुद्धिरियं जाता धर्मसंसक्तजा मुनेः}
{कृतघ्ना नरकं यान्ति ये च विश्वस्तघातिनः}


\twolineshloka
{निष्कृतिं नैव पश्यामि कृतघ्नानां कथञ्चन}
{क्रते प्राणिपरित्यागाद्धर्मज्ञानां वचो यथा}


\twolineshloka
{अकृत्वा भरणं पित्रोरदत्त्वा गुरुदक्षिणाम्}
{कृतघ्नतां च सम्प्राप्य मरणान्ता च निष्कृतिः}


\twolineshloka
{आकाङ्क्षायामुपेक्षायां चोपपातकमुत्तमम्}
{तस्मात्प्राणान्परित्यक्ष्ये प्रायश्चित्तार्थमित्युत}


\fourlineindentedshloka
{स मेरुशिखरं गत्वा निस्यङ्केनान्तरात्मना}
{प्रायश्चित्तं कर्तुकामः शरीरं त्यक्तुमुद्यतः}
{निगृहीतश्च धर्मात्मा हस्ते धर्मेण धर्मवित् ॥धर्मि उवाच}
{}


\twolineshloka
{वत्सनाभ महाप्राज्ञ बहुवर्षशतायुष}
{परितुष्टोस्मि त्यागेन निस्सङ्गेन तथाऽऽत्मनः}


\twolineshloka
{एवं धर्मभृतः सर्वे विमृशन्ति कृताकृतम्}
{न कश्चिद्वत्सनाभस्य यस्य नोपहतं मनः}


\twolineshloka
{यश्चानवद्यश्चरति शक्तो धर्मं च सर्वशः}
{निवर्तस्व महाप्राज्ञ् पूतात्मा ह्यसि शाश्वतः'}


\chapter{अध्यायः ३४}
\threelineshloka
{स्त्रीपुंसयोः सम्प्रयोगे स्पर्शः कस्याधिको भवेत्}
{एतस्मिन्संशये राजन्यथावद्वक्तुमर्हसि ॥भीष्म उवाच}
{}


\twolineshloka
{अत्राप्युदाहरन्तीममितिहासं पुरातनम्}
{भङ्गास्वनेन शक्रस्य यथा वैरमभूत्पुरा}


\twolineshloka
{पुरा भङ्गास्वनो नाम राजर्षिरतिधार्मिकः}
{अपुत्रः पुरुषव्याघ्र पुत्रार्थं यज्ञमाहरत्}


\twolineshloka
{अग्निष्टुतं स राजर्षिरिन्द्रद्विष्टं महाबलः}
{प्रायश्चित्तेषु मर्त्यानां पुत्रकामेषु चेष्यते}


\threelineshloka
{इन्द्रो ज्ञात्वा तु तं यज्ञं महाभागः सुरेश्वरः}
{अन्तरं तस्य राजर्षेरन्विच्छन्नियतात्मनः}
{न चैवास्यान्तरं राजन्स ददर्श महात्मनः}


\twolineshloka
{कस्य चित्त्वथ कालस्य मृगयां गतवान्नृपः}
{इदमन्तरमित्येव शक्रो नृपममोहयत्}


\twolineshloka
{एकाश्वेन च राजर्षिर्भ्रान्त इन्द्रेण मोहितः}
{न दिशोऽविन्दत नृपः क्षुत्पिपासार्दितस्तदा}


\twolineshloka
{इतश्चेतश्च धावन्वै श्रमतृष्णार्दितो नृप}
{सरोऽपश्यत्सुरुचिरं पूर्णं परमवारिणा}


% Check verse!
सोऽवगाह्य सरस्तात पाययामास वाजिनम्
\twolineshloka
{अथ पीतोदकं सोऽश्वं वृक्षे बद्ध्वा नृपोत्तमः}
{अवगाह्य ततः स्नातस्तत्र स्त्रीत्वमवाप्तवान्}


\twolineshloka
{आत्मानं स्त्रीकृतं दृष्ट्वा व्रीडितो नृपसत्तमः}
{चिन्तानुगतसर्वात्मा व्याकुलेन्द्रियचेतनः}


\twolineshloka
{आरोहिष्ये कथं त्वश्वं कथं यास्यामि वै पुरम्}
{इष्टेनाग्निष्टुता चापि पुत्राणां शतमौरसम्}


\twolineshloka
{जातं महाबलानां मे तान्प्रवक्ष्यामि किन्नवहम्}
{दारेषु चात्मकीयेषु पौरजानपदेषु च}


\threelineshloka
{मृदुत्वं च तनुत्वं च पराधीनत्वमेव च}
{`हासभावादि लावण्यं स्त्रीगुणाद्वा कुतूहलम्}
{'स्त्रीगुणा ऋषिभिः प्रोक्ता धर्मतत्त्वार्थदर्शिभिः}


% Check verse!
व्यायामः कर्कशत्वं च वीर्यं च पुरुषे गुणाः
\twolineshloka
{पौरुषं विप्रनष्टं स्त्रीत्वं केनापि मेऽभवत्}
{स्त्रीभावात्पुनरश्वं तं कथमारोढुमुत्सहे}


\threelineshloka
{महता त्वथ खेदेन आरुह्याश्वं नराधिपः}
{पुनरायात्पुरं तात स्त्रीभूतो नृपसत्तमः}
{}


\twolineshloka
{पुत्रा दाराश्च भृत्याश्च पौरजानपदाश्च ते}
{किन्न्विदं त्विति विज्ञाय विस्मयं परमं गताः}


\threelineshloka
{अथोवाच स राजर्षिः स्त्रीभूतो वदतांवरः}
{मृगयामस्मि निर्यातो बलैः परिवृतो दृढम्}
{उद्भान्तः प्राविशं घोरामटवीं दैवमोहितः}


\twolineshloka
{अटव्यां च सुघोंरायां तृष्णार्तो नष्टचेतनः}
{सरः सुरुचिरप्रख्यमपश्यं पक्षिभिर्वृतम्}


\twolineshloka
{तत्रावगाढः स्त्रीभूतो व्यक्तं दैवान्न संशयः}
{`अतृप्त एव पुत्राणां दाराणां च धनस्य च ॥'}


\threelineshloka
{नामगोत्राणि चाभाव्य दाराणां मन्त्रिणां तथा}
{आह पुत्रांस्ततः सोऽथ स्त्रीभूतः पार्तिवोत्तमः}
{सम्प्रीत्या भुज्यतांराज्यं वनं यास्यामि पुत्रकाः}


\twolineshloka
{स पुत्राणां शतं राजा अभिषिच्य वनं गतः}
{गत्वा चैवाश्रमं सा तु तापसं प्रत्यपद्यत}


\twolineshloka
{तापसेनास्य पुत्राणामाश्रमेष्वभवच्छतम्}
{अथ साऽऽदाय तान्सर्वान्पूर्वपुत्रानभाषत}


\twolineshloka
{पुरुषत्वे सुता यूयं स्त्रीत्वे चेमे शतं सुताः}
{एकत्र भुज्यतां राज्यं भ्रातृभावेन पुत्रकाः}


% Check verse!
सहिता भ्रातरस्तेऽथ राज्यं बुभुजिरे तदा
\threelineshloka
{तान्दृष्ट्वा भातृभावेन भुञ्जानान्राज्यमुत्तमम्}
{चिन्तयामास देवेन्द्रो मन्युनाऽथ परिप्लुतः}
{उपकारोऽस्य राजर्षेः कृतो नापकृतं मया}


\twolineshloka
{ततो ब्राह्मणरुपेण देवराजः शतक्रतुः}
{भेदयामास तान्गत्वा नगरं वै नृपात्मजान्}


\threelineshloka
{भ्रातृणां नास्ति सौभ्रात्रं येऽप्येकस्य पितः सुताः}
{कश्यपस्य सुराश्चैव असुराश्व सुतास्तथा}
{}


\twolineshloka
{राज्यहेतोर्विवदिताः कश्यपस्य सुरासुराः ॥यूयं भङ्गास्वनापत्यास्तापसस्येतरे सुताः}
{}


\twolineshloka
{युष्माकं भेदितास्ते तु युद्धेऽन्योन्यमपातयन्}
{तच्छ्रुत्वा तापसी चापि संतप्ता प्ररुरोद ह}


\twolineshloka
{ब्राह्मणच्छद्मनाऽभ्येत्य तामिन्द्रोऽथान्वपृच्छत}
{केन दुःखेन संतप्ता रोदिषि त्वं वरानने}


\twolineshloka
{ब्राह्म्णं तं ततो दृष्ट्वा सा स्त्री करणमब्रवीत्}
{पुत्राणां द्वे शते ब्रह्मन्कालेन विनिपातिते}


\twolineshloka
{अहं राजाऽभव विप्र तत्र पूर्वं शतं मम}
{समुत्पन्नं सुरूपाणां पुत्राणां ब्राह्मणोत्तम}


\twolineshloka
{कदाचिन्मृगयां यात उद्धान्तो गहने वने}
{अवगाढश्च सरसि स्त्रीभूतो ब्राह्मणोत्तम}


% Check verse!
पुत्रान्राज्ये प्रतिष्ठाप्य वनमस्मि ततो गतः
\twolineshloka
{स्त्रियाश्च मे पुत्रशतं तापसेन महात्मना}
{आश्रमे जनितं ब्रह्मन्नीतं तन्नगरं मया}


\twolineshloka
{तेषां च वैरमुत्पन्नं कालयोगेन वै द्विज}
{एतच्छोचाम्यहं ब्रह्मन्दैवेन समभिप्लुता}


% Check verse!
इन्द्रस्तां दुःशितां दृष्ट्वा अब्रवीत्परुषं वचः
\threelineshloka
{पुरा सुदुःसहं भद्रे मम दुःखं त्वया कृतम्}
{इन्द्रद्विष्टेन यजता मामनाहूय धिष्ठितम्}
{इन्द्रोऽहमस्मि दुर्बुद्धे वैरं ते पातितं मया}


\threelineshloka
{इन्द्रं दृष्ट्वा तु राजर्षिः पादयोः शिरसा गतः}
{प्रसीद त्रिदशश्रेष्ठ पुत्रकामेन स क्रतुः}
{इष्टस्त्रिदशशार्दूल तत्र मे क्षन्तुमर्हसि}


% Check verse!
प्रणिपातेन तस्येन्द्रः परितुष्टो वरं ददौ
\twolineshloka
{पुत्रास्ते कतमे राजञ्जीवन्त्वेतत्प्रचक्ष्व मे}
{स्त्रीभूतस्य हि ये जाताः पुरुषस्याथ येऽभवन्}


\twolineshloka
{तापसी तु ततः शक्रमुवाच प्रयताञ्जलिः}
{स्त्रीभूतस्य हि ये पुत्रास्ते मे जीवन्तु वासव}


\twolineshloka
{इन्द्रस्तु विस्मितो दृष्ट्वा स्त्रियं पप्रच्छ तां पुनः}
{पुरुषोत्पादिता ये ते कथं द्वेष्याः सुतास्तव}


\threelineshloka
{स्त्रीभूतस्य हि ये जाताः स्नेहस्तेभ्योऽधिकः कथम्}
{कारणं श्रोतुमिच्छामि तन्मे वक्तुमिहार्हसि ॥स्त्र्युवाच}
{}


\threelineshloka
{स्त्रियास्त्वभ्यधिकः स्नेहो न तता पुरुषस्य वै}
{तस्मात्ते शक्र जीवन्तु ये जाताः स्त्रीकृतस्य वै ॥भीष्म उवाच}
{}


\twolineshloka
{एवमुक्तस्ततस्त्विन्द्रः प्रीतो वाक्यमुवाच ह}
{सर्व एवेह जीवन्तु पुत्रास्ते सत्यवादिनि}


\threelineshloka
{वरं च वृणु राजेन्द्र यं त्वमिच्छसि सुव्रत}
{पुरुषत्वमथ स्त्रीत्वं मत्तो यदभिकाङ्क्षसे ॥स्त्र्युवाच}
{}


\twolineshloka
{स्त्रीत्वमेव वृणे शक्र पुंस्त्वं नेच्छामि वासव}
{एवमुक्तस्तु देवेन्द्रस्तां स्त्रियं प्रत्युवाच ह}


\twolineshloka
{पुरुषत्वं कथं त्यक्त्वा स्त्रीत्वं रोचयसे विभो}
{एवमुक्तः प्रत्युवाच स्त्रीभूतो राजसत्तमः}


\twolineshloka
{स्त्रियाः पुरुषसंयोगे प्रीतिरभ्यधिका सदा}
{एतस्मात्कारणाच्छक्र स्त्रीत्वमेव वृणोम्यहम्}


\twolineshloka
{रमाभि चाधिकं स्त्रीत्वे सत्यं वै देवसत्तम}
{स्त्रीभावेन हि तुष्यामि गम्यतां त्रिदशाधिप}


\twolineshloka
{एवमस्त्विति चोक्त्वा तामापृच्छ्य त्रिदिवं गतः}
{एवं स्त्रिया महाराज अधिका प्रीतिरुच्यते}


\chapter{अध्यायः ३५}
\threelineshloka
{किं कर्तव्यं मनुष्येण लोकयात्राहितार्थिना}
{कथं वै लोकयात्रां तु किंशीलश्च समाचरेत् ॥भीष्म उवाच}
{}


\twolineshloka
{`देवे नारायणे भक्तिः शंकरे साधुपूजया}
{ध्यानेनाथ जपः कार्यः स्वधर्मैः शुचिचेतसा ॥'}


\twolineshloka
{कायेन त्रिविधं कर्म वाचा चापि चतुर्विधम्}
{मनसा त्रिविधं चैव दश कर्मपथांस्त्यजेत्}


\twolineshloka
{प्राणातिपातः स्तैन्यं च परदाराभिमर्शनम्}
{त्रीणि पापानि कायेन सर्वतः परिवर्जयेत्}


\twolineshloka
{असत्प्रलापं पारुष्यं पैशुन्यमनृतं तथा}
{चत्वारि वाचा राजेन्द्र न जल्पेन्नानुचिन्तयेत्}


\twolineshloka
{अनभिध्या परखेषु सर्वसत्वेषु सौहृदम्}
{कर्मणां फलमस्तीति त्रिविधं मनसा चरेत्}


\threelineshloka
{तस्माद्वाक्वायमनसा नाचरेदशुभं नरः}
{शुभान्येवाचरँल्लोके भक्तो नारायणस्य हि}
{तस्यैव तु पदं सूक्ष्मं प्रसादादश्नुयात्परम् ॥'}


\chapter{अध्यायः ३६}
% Check verse!
चित्तं मे दूयते तात लोके परमविन्दतः
\twolineshloka
{अशाश्वतमिदं सर्वं जगत्स्थावरजङ्गमम्}
{क्रते नारायणं पुण्यं प्रतिभाति पितामह}


\threelineshloka
{नारायणो हि विश्वात्मा पुरुषः पुष्करेक्षणः}
{तस्यास्य देवकीसूनोः श्रुतं कृत्स्नं त्वयाऽनघ ॥भीष्म उवाच}
{}


\twolineshloka
{युधिष्ठिर महाप्राज्ञ मया दृष्टं च सङ्गरे ॥युधिष्ठिर उवाच}
{}


\threelineshloka
{त्वत्त एव तु राजेन्द्र राजधर्माश्च पुष्कलाः}
{श्रुतं पुराणमखिलं नारदेन निवेदितम्}
{गुह्यं नारायणाख्यानं त्रिविधक्लेशनाशनम्}


\twolineshloka
{एकान्तिधर्मनियमाः समासव्यासकल्पिताः}
{कथिता वै महाभाग त्वया वै मदनुग्रहात्}


\twolineshloka
{लोकरक्षणकर्तृत्वं तस्यैव हरिमेधसः}
{आतिथेयविधिश्चैव तपांसि नियमाश्च ये}


\twolineshloka
{वेदवादप्रसिद्धाश्च वाजपेयादयो मखाः}
{यज्ञा द्रविणनिष्पाद्या अग्निहोत्रानुपालिताः}


\twolineshloka
{जपयज्ञाश्च विविधा ब्राह्मणानां तपस्विनाम्}
{एकादशविधाः प्रोक्ता हविर्यज्ञा द्विजातिनाम्}


\twolineshloka
{तेषां फलविशेषाश्च उञ्छधर्मास्तथैव च}
{अहन्यहनि ये प्रोक्ता महायज्ञा द्विजातिनाम्}


\twolineshloka
{वेदश्रवणधर्माश्च ब्रह्मयज्ञफलं तथा}
{वेदव्रतविधानं च नियमाश्चैव वैदिकाः}


\twolineshloka
{स्वाहा स्वधा प्रणीते च इष्टापूर्तफलं तथा}
{उत्तरोत्तरसेवायामाश्रमाणां च यत्फलम्}


\twolineshloka
{प्रत्येकशश्च निष्ठायामाश्रमाणां महामते}
{मासपक्षोपवासानां सम्यगुक्तफलं च यत्}


\twolineshloka
{अनाशितानां ये लोका ये च पञ्चतपा नराः}
{वीराध्वानं प्रपन्नानां या गतिश्चाग्निहोत्रिणाम्}


\twolineshloka
{ग्रीष्मे पञ्चतपानां च शिशिरे जलचारिणाम्}
{वर्षे स्थण्डिलशायीनां फलं यत्परिकीर्तितम्}


\twolineshloka
{लोके चक्रचराणां च द्विजानां यत्फलं स्मृतम्}
{अन्नादीनां च दानानां यत्फलं परिकीर्तितम्}


\twolineshloka
{सर्वतीर्थाभिषिक्तानां नराणां च फलोदयः}
{राज्ञां धर्माश्च ये लोके सम्यक्पालयतां प्रजाः}


\twolineshloka
{ये च सत्यव्रता लोके ये तीर्थे कृतश्रमाः}
{मातापितृपरा ये च गुरुवृत्तीश्च संश्रिताः}


\twolineshloka
{गोब्राह्मणपरित्राणे राष्ट्रातिक्रमणे तथा}
{त्यजन्त्यभिमुखाः प्राणान्निर्भयाः सत्वमाश्रिताः}


\twolineshloka
{सहस्रदक्षिणानां च या गतिर्वदनां वर}
{ये च संध्यामुपासन्ते सम्यगुक्ता महाव्रताः}


\twolineshloka
{तथा योगविधानं च यद्ग्रन्थेष्वभिशब्दितम्}
{वेदाद्याः श्रुतयश्चापि श्रुता मे गुरुसत्तम}


\twolineshloka
{सिद्धान्तनिर्णयाश्चापि द्वैपायनमुखोद्गताः}
{श्रुताः पञ्च महायज्ञा येषु सर्वं प्रतिष्ठितम्}


\twolineshloka
{तत्प्रभेदेषु ये धर्मास्तेऽपि वै कृत्स्नशः श्रुताः}
{न च दूषयितुं शक्याः सद्भिरुक्ता हि ते तथा}


\twolineshloka
{एतेषां किल धर्माणामुत्तमो वैष्णवो विधिः}
{रक्षते भगवान्विष्णुर्भक्तमात्मशरीरवत्}


\twolineshloka
{कर्मणो हि कृतेस्येह कामितस्य द्विजोत्तम}
{फलं ह्यवश्यं भोक्तव्यमृषिर्द्वैपायनोऽब्रवीत्}


\twolineshloka
{भोगान्ते चापि पतनं गतिः पूर्वं प्रभाषिता}
{न मे प्रीतिकरास्त्वेते विषोदर्का हि मे मताः}


\threelineshloka
{वधात्कृष्टतरं मन्ये गर्भवासं महाद्युते}
{दिष्टान्ते यानि दुःखानि पुरुषो विन्दते विभो}
{ततः कष्टतराणीह गर्भवासे हि विन्दति}


\twolineshloka
{ततश्चाभ्याधिकां तीव्रां वेदनां लभते नरः}
{गर्भापक्रमणे तात कर्मणासुपसर्पणे}


\threelineshloka
{तस्मान्मे निश्चयो जातो धर्मेष्वेतेषु भारत}
{तदिच्छामि कुरुश्रेष्ठ त्वत्प्रसादान्महामते}
{तं धर्मं चेह वेत्तुं वै यो जराजन्ममृत्युहा}


\threelineshloka
{येनोष्णदा वैतरणी असिपत्रवनं च तत्}
{कुण्डानि चाग्नितप्तानि क्षुरधारापथस्तथा}
{शाल्मलीं च महाघोरामायसीं घोरकण्टकाम्}


\threelineshloka
{मातापितृकते चापि सुहृन्मित्रार्थकारणात्}
{आत्महेतोश्च पापानि कृतानीह नरैश्च यैः}
{तेषां फलोदयं कष्टमृषिर्द्वैपायनोऽब्रवीत्}


\twolineshloka
{कुम्भीपाकप्रदीप्तानां शूलार्तानां च क्रन्दताम्}
{रौरवे क्षिप्यमाणानां प्रहारैर्मथितात्मनाम्}


\twolineshloka
{स्तनतामपकृत्तानां पिबतामात्मशोणितम्}
{तेषामेव प्रवदतां कारुण्यं नास्ति यन्त्रतः}


\twolineshloka
{तृष्णाशुष्कोष्ठकण्ठानां विह्वलानामचेतसाम्}
{सर्वदुःखाभिभूतानां रुजार्तानां च क्रोशताम्}


% Check verse!
वेदनार्ता हि क्रन्दन्ति पूरयन्तो दिशो दश
\twolineshloka
{एकः करोति पापानि सहभोज्यानि बान्धवैः}
{तेषामेकः फलं भुङ्क्ते कष्टं वैवस्वते गृहे}


\twolineshloka
{येन नैतां गतिं गच्छेन्न विण्मूत्रास्थिपिच्छिले}
{विष्ठामूत्रकृमीमध्ये बहुजन्तुनिषेविते}


\twolineshloka
{को गर्भवासात्परतो नरकोऽन्यो विधीयते}
{यत्र वासकृतो योगः कुक्षौ वासो विधीयते}


\threelineshloka
{जातो विस्तीर्णशोकः स्याद्भवेत विगतज्वरः}
{न चैष लभ्यते कामो जातमात्रं हि मानवम्}
{आविशन्तीह दुःखानि मनोवाक्कायिकानि तु}


\twolineshloka
{तैरस्वतन्त्रो भवति पीड्यमानो भयानकैः}
{तैर्गर्भवासं गच्छति अवशो जायते तथा}


\twolineshloka
{अवशश्चेहते जन्तुर्व्रजत्यवश एव हि}
{जरसा रूपविध्वंसं प्राप्नोत्यवश एव तु}


\twolineshloka
{शरीरभेदमाप्नोति जीर्यतेऽवश एव तु}
{एवं ह्यनियतो मृत्युर्भवत्येव सदा नृषु}


\threelineshloka
{गर्भेषु म्रियते कश्चिज्जायमानस्तथाऽपरः}
{जाता म्रियन्ते बहवो यौवनस्थास्तथाऽपरे}
{मध्यभावे तु नश्यन्ति स्थविरो मृत एव तु}


\twolineshloka
{को जन्मनो नोद्विजते स्वयम्भूरपि यो भवेत्}
{कुतस्त्वस्मद्विधस्तात मरणस्य वशानुगः}


\twolineshloka
{नित्याविष्टो भयेनाहं मनसा कुरुसत्तम}
{मुहूर्तमप्यहं शर्म न विन्दामि महामते}


\twolineshloka
{कालात्मनि तिरोभूतो नित्यं तद्गुणवर्जितः}
{अन्नैर्बहुविधैः पुष्टं वस्त्रैर्नानाविधैर्वृतम्}


\twolineshloka
{चन्दनागरुदिग्धाङ्गं मणिमुक्ताविभूषितम्}
{यानैर्बहुविधैर्यातमेकान्तेनैव लालितम्}


\twolineshloka
{यौवनोद्धतरूपाभिर्मन्दविह्वलगामिभिः}
{इष्टिभिरभिरामाभिर्वरस्त्रीभिरयन्त्रितम्}


\twolineshloka
{रमितं सुचिरं कालं शरीरममितप्रभम्}
{अवितृप्ता गमिष्यन्ति हित्वा प्राणांस्तथाऽपरे}


\twolineshloka
{स्वर्गेऽप्यनियता भूतिस्तथैवाकाशसंश्रये}
{देवाऽप्यधिष्ठानवशास्तस्माद्देवं न कामये}


\twolineshloka
{कामानां नास्त्यधिष्ठानमकामस्तु निवर्तते}
{लोकसङ्ग्रहधर्मास्तु सर्व एव न संशयः}


\twolineshloka
{डोलासधर्मा धर्मज्ञ ऋषिर्द्वैपायनोऽब्रवीत्}
{अस्मात्को विषमं दुःखमारोहेत विचक्षणः}


\twolineshloka
{विद्यमाने समे मार्गे डोलाधर्मविवर्जिते}
{को ह्यात्मानं प्रियं लोके डोलासाधर्म्यतांनयेत्}


\threelineshloka
{चराचरैः सर्वभूतैर्गन्तव्यमविशङ्कया}
{अस्माल्लोकात्परं लोकमपाथेयमदैशिकम्}
{घोरं तमः प्रवेष्टव्यमत्रातारमबान्धवम्}


\twolineshloka
{ये तु तं किल धर्मज्ञा धर्मं नारायणेरितम्}
{अनन्यमनसो दान्ताः स्मरन्ति नियतव्रताः}


\twolineshloka
{ततस्तेनैव पश्यन्ति प्राप्नुव्ति परं पदम्}
{रक्षते भगवान्विष्णुर्भक्तानात्मशरीरवत्}


\twolineshloka
{कुलालचक्रप्रतिमे भ्राम्यमाणेषु जन्तुषु}
{मातापितृसहस्राणि सम्प्राप्तानि मया गुरो}


% Check verse!
स्नेहापन्नेन पीतास्तु मातॄणां विविधाः स्तलाः
\threelineshloka
{पुत्रदारसहस्राणि इष्टानिष्टशतानि च}
{प्राप्तान्यधिष्ठानवशादतीतानि तथैव च}
{न क्वचिन्न सुखं प्राप्तं न क्वचिच्छाश्वती स्थितिः}


\threelineshloka
{स्थानैर्महद्भिर्विभ्रंशो दुःखलब्धैः पुनः पुनः}
{धननाशश्च सम्प्राप्तो लब्ध्वा दुःखेन तद्धनम्}
{अध्वगानामिव पथि च्छायामाश्रित्य सङ्गमः}


\twolineshloka
{एवं कर्मवशो लोको ज्ञातीनां हितसङ्गमः}
{विश्रम्य च पुनर्याति कर्मभिर्दर्शितां गतिम्}


\twolineshloka
{एतदीदृशकं दृष्ट्वा ज्ञात्वा चैव समागमम्}
{को न बिभ्येत्कुरुश्रेष्ठ विष्ठान्नस्येव भोजनात्}


\threelineshloka
{बुद्धिश्च मे समुत्पन्ना वैष्णवे धर्मविस्तरे}
{तदेष शिरसा पादौ गतोऽस्मि भगवंस्तव}
{शरणं च प्रपन्नोऽस्मि गन्तव्ये शरणे ध्रुवे}


\threelineshloka
{जन्ममृत्युजराखिन्नस्त्रिभिर्दुःखैर्निपीडितः}
{इच्छामि भवता त्रातुमेभ्यस्त्वत्तो महामते}
{तस्याद्य युगधर्मस्य श्रवणात्कुरुपुङ्गव}


\twolineshloka
{एतदाद्ययुगोद्भूतं त्रेतायां तत्तिरोहितम्}
{स एव धर्ममखिलमृषिर्द्वैपायनोऽब्रवीत् ॥'}


\chapter{अध्यायः ३७}
% Check verse!
सदृशं राजशार्दूल वृत्तस्य च कुलस्य च
\twolineshloka
{को राज्यं विपुलं गृह्य स्फीताकारं पुनर्महत्}
{निर्जितारातिसामन्तं देवराज्योपमं सुखम्}


\twolineshloka
{राज्ये राज्यगुणा ये च तान्व्युदस्य नराधिप}
{दोषं पश्यति राजेन्द्र देहेऽस्मिन्पाञ्चभौतिके}


\twolineshloka
{अतिक्रान्तास्त्वया राजन्वृत्तेन प्रपितामहाः}
{धर्मो विग्रहवान्धीरो विदुरश्च महायशाः}


\twolineshloka
{सञ्जयश्च महातेजा ये चान्ये दिव्यदर्शनाः}
{प्रवृत्तज्ञानसम्पन्नास्तत्वज्ञानविदो नृप}


\twolineshloka
{तेऽतिक्रान्ता महाराज ब्रह्माद्याः ससुरासुराः}
{अनित्यं दुःखसंतप्तं जगदेतन्न संशयः}


\twolineshloka
{एवमेतान्महाबाहो ब्रह्माद्यान्ससुरासुरान्}
{अनित्यान्सततं पश्य मनुष्यादिषु का कथा}


\twolineshloka
{नित्यां तु प्रकृतीमाह याऽसौ प्रसवधर्मिणी}
{अरूपिणीमनिर्देश्यामकृतां पुरुषातिगाम्}


\twolineshloka
{तामत्यन्तसुखां सौम्यां निर्वाणमिति संज्ञिताम्}
{आहुर्ब्रह्मर्षयो ह्याद्यां भुवि चैव महर्षयः}


\twolineshloka
{तया पुरुषरूपिण्या धर्मप्रकृतिकोऽनघ}
{स यात्येव हि निर्वाणं यत्तत्प्रकृतिसंज्ञितम्}


\threelineshloka
{स एष प्राकृतो धर्मो भ्राजत्यादियुगे नृप}
{विकारधर्माः शेषेषु युगेषु भरतर्षभ}
{भ्राजन्तेऽभ्यधिकं वीर संसारपथगोचराः}


\twolineshloka
{प्रकृतीनां च सर्वासामकृता प्रकृतिः स्मृता}
{एवं प्रकृतिधर्मा हि वरां प्रकृतिमाश्रिता}


\twolineshloka
{पश्यन्ति परमां लोके दृष्टादृष्टानुदर्शिनीम्}
{सत्वादियुगपर्यन्ते त्रेतायुगसमुद्भवे}


\threelineshloka
{कामं कामयमानेषु ब्राह्मणेषु तिरोहितः}
{कुपथेषु तु धर्मेषु प्रादुर्भूतेषु कौरव}
{जातो मन्दप्रचारो हि धर्मः कलियुगे नृप}


\twolineshloka
{नित्यस्तु पुरुषो ज्ञेयो विश्वरूपो निरञ्जनः}
{ब्रह्माद्या अपि देवाश्च यं सदा पर्युपासते}


\twolineshloka
{तं च नारायणं विद्धि परं ब्रह्मेति शाश्वतम्}
{तत्कर्म कुरु कायेन ध्यायस्व मनसा च तम्}


\twolineshloka
{कीर्तयस्व च तन्नाम वाचा सर्वत्र भूपते}
{तत्पदं प्राप्नुहि प्राप्यं शाश्वतं चापुनर्भवम्}


\twolineshloka
{इत्येतद्विष्णुमाश्रित्य संसारग्रहमोक्षणम्}
{कथितं ते महाबाहो किं भूयः श्रोतुमिच्छसि ॥'}


\chapter{अध्यायः ३८}
\threelineshloka
{क्लिश्यमानेषु भूतेषु जातीमरणसागरे}
{यत्प्राप्य क्लेशं नाप्नोति तन्मे ब्रूहि पितामह ॥भीष्म उवाच}
{}


\twolineshloka
{अत्राप्युदाहरन्तीममितिहासं पुरातनम्}
{सनत्कुमारस्य सतः संवादं नारदस्य च}


\twolineshloka
{सनत्कुमारो भगवान्ब्रह्मपुत्रो महायशाः}
{पूर्वजातास्त्रयस्तस्य कथ्यन्ते ब्रह्मवादिनः}


\twolineshloka
{सनकः सनन्दनश्चैव तृतीयश्च सनातनः}
{जातमात्राश्च ते सर्वे प्रतिबुद्धा इति श्रुतिः}


\twolineshloka
{चतुर्थश्चैव तेषां स भगवान्योगवित्तमः}
{सनत्कुमार इति वै कथयन्ति महर्षयः}


\twolineshloka
{हैरण्यगर्भः स मुनिर्वसिष्ठः पञ्चमः स्मृतः}
{षष्ठः स्थाणुः स भगवानमेयात्मा त्रिशूलधृत्}


\twolineshloka
{ततोऽपरे समुत्पन्नाः पावकादारुणे क्रतौ}
{मनसा स्वयंभुवो हीमे मरीचिप्रमुखास्तथा}


\twolineshloka
{भुगुर्मरीचेरनुजो भृगोरप्यङ्गिरास्तथा}
{अनुजोङ्गिरसोऽथात्रिः पुलस्त्योत्रेस्तथाऽनुजः}


\twolineshloka
{पुलस्त्यस्यानुजो विद्वान्पुलहो न महाद्युतिः}
{पठ्यन्ते ब्रह्मजा ह्येते विद्वद्बिरमितौजसः}


\twolineshloka
{सर्वमेतन्महाराज कुर्वन्नादिगुरुर्महान्}
{प्रभुर्विभुरनन्तश्रीर्ब्रह्मा लोकपितामहः}


\threelineshloka
{मूर्तिमन्तोऽमृतीभूतास्तेजसाऽतितपोन्विताः}
{सनकप्रभृतयस्तत्र ये च प्राप्ताः परं पदम्}
{कृत्स्नं क्षयमनुप्राप्य विमुक्ता मूर्तिबन्धनात्}


\twolineshloka
{सनत्कुमारस्तु विभुर्योगमास्थाय योगवित्}
{त्रीँल्लोकानचरच्छश्वदैर्येण परेण हि}


\twolineshloka
{रुद्रश्चाप्यष्टगुणितं योगं प्राप्तो महायशाः}
{सूक्ष्ममष्टगुणं राजन्नितरे नृपसत्तम}


\twolineshloka
{मरीचिप्रमुखास्तात सर्वे सृष्ट्यर्थमेव ते}
{नियुक्ता राजशार्दूल तेषां सृष्टिं शृणुष्व मे}


\twolineshloka
{सप्त ब्रह्मण इत्येते पुराणे निश्चयं गताः}
{सर्वे वेदेषु चैवोक्ताः खिलेषु च न संशयः}


\twolineshloka
{इतिहासपुराणे च श्रुतिरेषा सनातनी}
{ब्राह्मणा वरदानेतान्प्राहुर्वेदान्तपारगाः}


\twolineshloka
{एतेषां पितरस्तात पुत्रा इत्यनुचक्षते}
{गणाः सप्त महाराज मूर्तयोऽमूर्तयस्तथा}


\twolineshloka
{पितृणां चैव राजेन्द्र पुत्रा देवा इति श्रुतिः}
{देवैर्व्याप्ता इमे लोका इत्येवमनुशुश्रुम}


\twolineshloka
{कृष्णद्वैपायनाच्चैव देवस्थानात्तथैव त}
{देवलाच्च नरश्रेष्ठ काश्यपाच्च मया श्रुतम्}


\twolineshloka
{गौतमादपि कौण्डिन्याद्बारद्वाजात्तथैव च}
{मार्कण्डेयात्तथैवैतदृषेर्देवमतादपि}


\twolineshloka
{पित्रा च मम राजेन्द्र श्राद्धकाले प्रभाषितम्}
{परं रहस्यं वेदान्तं प्रियं हि परमात्मनः}


\threelineshloka
{अतः परं प्रवक्ष्यामि यन्मां पृच्छसि भारतः}
{तदिहैकमनोबुद्धिः शृणुष्वावहितो नृप}
{स्वायंभुवस्य संवादं नारदस्य च धीमतः}


\twolineshloka
{सनत्कुमारो भगवान्दिव्यं जज्वाल तेजसा}
{अङ्गुष्ठमात्रो भूत्वा वै विचचार महाद्युतिः}


\twolineshloka
{स कदाचिन्महाभागो मेरुपृष्ठं समागमत्}
{नारदेन नरश्रेष्ठ मुनिना ब्रह्मवादिना}


\twolineshloka
{जिज्ञासमानावन्योन्यं सकाशे ब्रह्मणस्ततः}
{ब्रह्म भागवतौ तात परमार्थार्थचिन्तकौ}


\twolineshloka
{मतिमान्मतिमच्छ्रेष्ठं बुद्धिमान्बुद्धिमत्तरम्}
{क्षेत्रवित्क्षेत्रविच्छ्रेष्ठं ज्ञानविज्ज्ञानमत्तमम्}


\threelineshloka
{सनत्कुमारं तत्वज्ञं भगवन्तमरिंदम}
{लोकविल्लोकविच्छ्रेष्ठमात्मविच्चात्मवित्तमम्}
{सर्ववेदार्थकुशलं सर्वशास्त्रविशारदम्}


\threelineshloka
{साङ्ख्ययोगं च यो वेद पाणावामलकं यथा}
{नारदोऽथ नरश्रेष्ठ तं पप्रच्छ महाद्युतिः ॥नारद उवाच}
{}


\twolineshloka
{त्रयोविंशतितत्वस्य अव्यक्तस्य महामुने}
{प्रभवं चाप्ययं चैव श्रोतुमिच्छामि तत्वतः}


\threelineshloka
{अध्यात्ममधिभूतं च अधिदैवं तथैव च}
{कालसङ्ख्याश्च सर्गं च स्रष्टारं पुरुषं प्रभुम्}
{}


\threelineshloka
{यं विस्वमुपजीवन्ति येन सर्वमिदं ततम्}
{यं प्राप्य न निवर्तन्ते तद्भवान्वक्तुमर्हति ॥सनत्कुमार उवाच}
{}


\twolineshloka
{यं विश्वमुपजीवन्ति यमाहुः पुरुषं परम्}
{तं वै शृणु महाबुद्धे नारायणमनामयम्}


\twolineshloka
{एष नारायणो नाम यं विश्वमुपजीवति}
{एष स्रष्टा विधाता च भर्ता पालयिता प्रभुः}


\twolineshloka
{प्राप्यैनं न निवर्तन्ते यतयोऽध्यात्मचिन्तकाः}
{एतावदेव वक्तव्यं मया नारद पृच्छते}


\twolineshloka
{परं न वेद्मि तत्सर्गं यावांश्चायं यथाप्यहम्}
{श्रूयतामानुपूर्व्येण न च सर्गः प्रयत्नतः}


\threelineshloka
{यथा कालपरीमाणं तत्वानामृषिसत्तम}
{अध्यात्ममधिभूतं च अधिदैवं तथैव च}
{कालसंख्यां च सर्गं च सर्वमेव महामुने}


\twolineshloka
{तमसः कुर्वतः सर्गं तामसो ह्यभिधीयते}
{ब्रह्मविद्भिर्द्विजैर्नित्यं नित्यमध्यात्मचिन्तकैः}


\twolineshloka
{पर्यायनामान्येतस्य कथयन्ति मनीषिणः}
{तानि ते सम्प्रवक्ष्यामि तदिहैकमनाः शृणु}


\threelineshloka
{महार्णवोऽर्णवश्चैव सलिलं च गुणास्तथा}
{वेदास्तपांसि यज्ञाश्च धर्माश्च भगवान्विभुः}
{प्राणः सांवर्तकोग्निश्च व्योम कालस्तथैव च}


\twolineshloka
{नामान्येतानि ब्रह्मर्षे शरीरस्येश्वरस्य वै}
{कीर्तितानि द्विजश्रेष्ठ मया शास्त्रानुसारतः}


\twolineshloka
{चतुर्युगसहस्राणि चतुर्युगमरिंदम}
{प्राहुः कल्पसहस्रं वै ब्राह्मणास्तत्वदर्शिनः}


\twolineshloka
{दशकल्पसहस्राणि अव्ययस्य महानिशा}
{तथैव दिवसं प्राहुर्योगाः सांख्याश्च तत्वतः}


\twolineshloka
{निशासुप्तोथ भगवान्क्षपान्ते प्रत्यबुध्यत}
{पश्चाद्बुद्ध्वा ससर्जापस्तासु वीर्यमवासृजत्}


\twolineshloka
{तदण्डमभवद्धैमं सहस्रांशुसमप्रभम्}
{अहंकृत्वा ततस्तस्मिन्ससर्ज प्रभुरीश्वरः}


\threelineshloka
{हिरण्यगर्भं विस्वात्मा ब्रह्माणां जलवन्मुनिम्}
{भूतभव्यभविष्यस्य कर्तारमनघं विभुम्}
{मूर्तिमन्तं महात्मानं विश्वशभुं स्वयंभुवम्}


\twolineshloka
{अणिमा लघिमा प्राप्तिरीशानो ज्योतिषां नरम्}
{चक्रे तिरोधां भगवानेत्कृत्वा महायशाः}


\twolineshloka
{एतस्यापि निशामाहुर्वेदवेदाङ्गपारगाः}
{पञ्चकल्पसहस्राणि अहरेतावदेव च}


\twolineshloka
{स सर्गं कुरुत ब्रह्मा तामसश्चानुपूर्व्यशः}
{सृजतेऽहं त्वहंकारं परमेष्ठिनमव्ययम्}


\twolineshloka
{अहङ्कारेणैव लोका व्याप्ताः साहंकृतेन वै}
{येनाविष्टानि भूतानि मज्जन्त्यव्यक्तसागरे}


\twolineshloka
{देवर्षिदानवनरा यक्षगन्धर्वकिन्नराः}
{उन्मज्जन्ति निमज्जन्ति ऊर्ध्वाधस्तिर्यगेव च}


\twolineshloka
{एतस्यापि निशामाहुस्तृतीयामथ कुर्वतः}
{त्रीणि कल्पसहस्राणि अहरेतावदेव तु}


\twolineshloka
{अहङ्कारस्तु सृजति महाभूतानि पञ्च वै}
{पृथिवी वायुराकाशमापो ज्योतिश्च पञ्चमम्}


\twolineshloka
{एतेषां गुणतत्वानि पञ्च प्राहुर्द्विजातयः}
{शब्दे स्पर्शे च रूपे च रसे गन्धे तथैव च}


\twolineshloka
{गुणेष्वेतेष्वभिरताः पङ्कलग्ना इव द्विपाः}
{नोत्तिष्ठन्त्यवशीभूताः सक्ता अव्यक्तसागरे}


\twolineshloka
{एतेषामिह वै सर्वं चतुर्थमिह कुर्वतः}
{चतुर्युगसहस्रे वै अहोरात्रास्तथैव च}


\twolineshloka
{अनन्त इति विख्यातः पञ्चमः सर्ग उच्यते}
{इन्द्रियाणि दशैकं च यथाश्रुतिनिदर्शनात्}


\twolineshloka
{मनः सर्वमिदं तात विश्वं सर्वमिदं जगत्}
{न तथान्यानि भूतानि बलवन्ति यथा मनः}


\twolineshloka
{एतस्यापि ह वै सर्गं षष्ठमाहुर्द्विजातयः}
{अहः कल्पसहस्रं वै रात्रिरेतावती तथा}


\twolineshloka
{ऊर्ध्वस्रोतस्तु वै सर्गं सप्तमं ब्रह्मणो विदुः}
{अष्टमं चाप्यधःस्रोतस्तिर्यक्तु नवमः स्मृतः}


\twolineshloka
{एतानि नव सर्गाणि तत्वानि च महामुने}
{चतुर्विंशतितत्वानि तत्वसंख्यानि तेऽनघ}


\threelineshloka
{सर्वस्य प्रभवः पूर्वमुक्तो नारायणः प्रभुः}
{अव्ययः प्रभवश्चैव अव्यक्तस्य महामुने}
{प्रवक्ष्याम्यपरं तत्वं यस्य यस्येश्वरश्च यः}


\twolineshloka
{अध्यात्ममधिभूतं च अधिदैवं तथैव च}
{यथाश्रुतं यथादृष्टं तत्वतो वै निबोध मे ॥'}


\chapter{अध्यायः ३९}
\twolineshloka
{अधःस्रोतसि सर्गे च तिर्यक्स्रोतसि चैव हि}
{एताभ्यामीश्वरं विन्द्यादूर्ध्वस्रोतस्तथैव च}


\twolineshloka
{कर्मेन्द्रियाणां पञ्चानामीश्वरो बुद्धिगोचरः}
{बुद्धीन्द्रियाणामपि तु ईश्वरो मन उच्यते}


\twolineshloka
{मनसः पञ्चभूतानि सगुणान्याहुरीश्वरम्}
{भूतानामीश्वरं विद्याद्ब्रह्माणं परमेष्ठिनम्}


\twolineshloka
{भवान्हि कुशलश्चैव धर्मेष्वेषु परेषु वै}
{कालाग्निरह्नः कल्पान्ते जगद्दहति चांशुभिः}


\twolineshloka
{ततः सर्वाणि भूतानि स्थावराणि चराणि च}
{महाभूतानि दग्धानि स्वां योनिं गमितानि वै}


\twolineshloka
{कूर्मपृष्ठनिभा भूमिर्निर्दग्धकुशकण्टका}
{निर्वृक्षा निस्तृणा चैव दग्धा कालाग्निना तदा}


\twolineshloka
{जगत्प्रलीनं जगति जगच्चापि प्रलीयते}
{नष्टिगन्धा तदा सूक्ष्मा जलमेवाभवत्तदा}


\twolineshloka
{ततो मयूखजालेन सूर्यस्त्वापीयते जलम्}
{रसात्मा लीयते चार्के तथा ब्राह्मणसत्तम}


\twolineshloka
{अन्तरिक्षगतान्भूतान्प्रदहत्यनलस्तदा}
{अग्निभूतं तदा व्योम भवतीत्यभिचक्षते}


\twolineshloka
{तं तथा विस्फुरद्वह्निं वायुर्जरयते महान्}
{महता बलवेगेन आदत्ते तं हि भानुमान्}


\twolineshloka
{वायोरपि गुणं स्पर्शमाकाशं ग्रसते यदा}
{प्रशाम्यति तदा वायुः खं तु तिष्ठति नानदत्}


\twolineshloka
{तस्य तं निनदं शब्दमादत्ते वै मनस्तदा}
{स शब्दगुणहीनात्मा तिष्ठते मूर्तिमांस्तु वै}


\twolineshloka
{भुङ्क्ते च स तदा व्योम मनस्तात दिगात्मकम्}
{व्योमात्मनि विनष्टे तु सङ्कल्पात्मा विवर्धते}


\twolineshloka
{सङ्कल्पात्मानमादत्ते चित्तं वै स्वेन तेजसा}
{चित्तं ग्रसत्यहङ्कारस्तदा वै मुनिसत्तम}


\twolineshloka
{विनष्टे च तदा चित्ते अहङ्कारोऽभवन्महान्}
{अहङ्कारं तदादत्ते महान्ब्रह्मा प्रजापतिः}


\twolineshloka
{अभिमाने विनष्टे तु महान्ब्रह्मा विराजते}
{तं तदा त्रिषु लोकेषु मूर्तिष्वेवाग्रमूर्तिजम्}


\threelineshloka
{येन विश्वमिदं कृत्स्नं निर्मितं वै गुणार्थिना}
{मूर्तं जलेचरमपि व्यवसायगुणात्मकम्}
{ग्रसिष्णुर्भगवान्ब्रह्मा व्यक्ताव्यक्तमसंशयम्}


\twolineshloka
{एषोऽव्ययस्य प्रलयो मया ते परिकीर्तितः}
{अध्यात्ममधिभूतं च अधिदैवं च श्रूयताम्}


% Check verse!
आकाशं प्रथमं भूतं श्रोत्रमध्यात्मं शब्दोधिभूतंदिशोधिदैवतं
% Check verse!
वायुर्द्वितीयं भूतं त्वगध्यात्मं स्पर्शोधिभूतं विद्युदधिदैवतंस्यात्
% Check verse!
ज्योतिस्तृतीयं भूतं चक्षुरध्यात्मं रूपमधिभूतं सूर्योधिदैवतंस्यात्
% Check verse!
आपश्चतुर्थं भूतं जिह्वाध्यात्मं रसोधिभूतं वरुणोधिदैवतंस्यात्
% Check verse!
पृथिवी पञ्चमं भूतं ध्राणमध्यात्मं गन्धोधिभूतं वायुरधिदैवतंस्यात्
\twolineshloka
{पाञ्चभौतिकमेतच्चतुष्टयं वर्णितम्}
{अतऊर्ध्वमिन्द्रियमनुवर्णयिष्यामः}


% Check verse!
पादावध्यात्मं गन्तव्यमधिभूतं विष्णुरधिदैवतं स्यात्
% Check verse!
हस्तावध्यात्मं कर्तव्यमधिभूतमिन्द्रोऽधिदैवतं स्यात्
% Check verse!
पायुरध्यात्मं विसर्गोऽधिभूतं मित्रोऽधिदैवतं स्यात्
\twolineshloka
{उपस्थोऽध्यात्ममानन्दोऽधिभूतं प्रजापतिरवधिदैवतं स्यात्}
{}


% Check verse!
वागध्यात्मं वक्तव्यमधिभूतमग्निरधिदैवतं स्यात्
% Check verse!
मनोऽध्यात्मं मन्तव्यमधिभूतं चन्द्रमा अधिदैवतं स्यात्
\twolineshloka
{अहङ्कारोऽध्यात्ममभिमानोऽधिभूतं विरिञ्चोऽधिदैवतं स्यात्}
{}


% Check verse!
बुद्धिरध्यात्मं व्यवसायोऽधिभूतं ब्रह्माधिदैवतं स्यात्
\twolineshloka
{एवमव्यक्तो भगवान्सकृत्कृत्स्नान्कुरुते सम्हरते च}
{कस्मात्क्रीडार्थम्}


% Check verse!
यथाऽऽदित्योंऽशुजालं क्षिपति सम्हरते च एवमव्यक्तो गुणान्सृजतिसम्हरते च
\threelineshloka
{यथाऽर्णवादूर्मिमालानिच यश्चोर्ध्वमुत्तिष्ठते सम्हरते च}
{यथा चान्तरिक्षादभ्रमाकाशमुत्तिष्ठति स्तनितगर्जितोन्मिश्रंतद्वत्तत्रैव प्राणशत्}
{एवमव्यक्तो गुणान्सृजति सम्हरति च}


% Check verse!
यथा कूर्मोऽङ्गानि कामात्प्रसारयते पुनश्च प्रवेशयति एवमव्यक्तोभगवान्लोकान्प्रकाशयति प्रवेशयते च
\threelineshloka
{एवं चेतनश्च भगवान्पञ्चविंशः शुचिस्तेनाधिष्ठिता प्रकृतिश्चेतयतिनित्यं सहधर्मा च}
{भगवतोऽव्यक्तस्य क्रियावतोक्रियावतश्च प्रकृतिःक्रियावानजरामरः क्षेत्रज्ञो नारायणाख्यः पुरुषः ॥भीष्म उवाच}
{}


\twolineshloka
{इत्येतन्नारदायोक्तं कुमारेण च धीमता}
{एतच्छ्रुत्वा द्विजो राजन्सर्वयज्ञफलं लभेत् ॥'}


\chapter{अध्यायः ४०}
\twolineshloka
{आत्मन्यग्नौ समाध्नाय य एते कुरुनन्दन}
{द्विजातयो व्रतोपेता जपयज्ञपरायणाः}


\twolineshloka
{यजन्त्यारम्भयज्ञैश्च मानसं यज्ञमास्थिताः}
{अग्निभ्यश्च परं नास्ति येषामेषोऽव्यवस्थितः}


\threelineshloka
{तेषां गतिर्महाप्राज्ञ कीदृशी किम्पराश्च ते}
{एतदिच्छामि तत्वेन त्वत्तः श्रोतुं पितामह ॥भीष्म उवाच}
{}


\twolineshloka
{अत्र ते वर्तयिष्यामि इतिहासं पुरातनम्}
{वैकुण्ठस्य च संवादं सुपर्णस्य च भारत}


\threelineshloka
{अमृतस्य समुत्पत्तौ देवानामसुरैः सह}
{षष्टिवर्षसहस्राणि दैवासुरमवर्तत}
{}


\twolineshloka
{तत्र देवास्तु दैतेयैर्वध्यन्ते भृशदारुणैः}
{त्रातारं नाधिगच्छन्ति वध्यमाना महासुरैः}


\twolineshloka
{आर्तास्ते देवदेवेशं प्रपन्नाः शरणैषिणः}
{पितामहं महाप्राज्ञं वध्यमानाः सुरेतरैः}


\twolineshloka
{ता दृष्ट्वा देवता ब्रह्मा सम्भ्रान्तेन्द्रियमानसः}
{वैकुण्ठं शरणं देवं प्रतिपेदे च तैः सह}


\twolineshloka
{ततः स देवैः सहितः पद्मयोनिर्नरेश्वर}
{तुष्टाव प्राञ्जलिर्भूत्वा नारायणमनामयम्}


\twolineshloka
{त्वद्रूपचिन्तनान्नाम्नां स्मरणादर्चनादपि}
{तपोयोगादिभिश्चैव श्रेयो यान्ति मनीषिणः}


\twolineshloka
{भक्तवत्सल पद्माक्ष परमेश्वर पापहन्}
{परमात्माऽविकाराद्य नारायण नमोंस्तु ते}


\twolineshloka
{नमस्ते सर्वलोकादे सर्वात्मामितविक्रम}
{सर्वभूतभविष्येश सर्वभूतमहेश्वर}


\twolineshloka
{देवानामपि देवस्त्वं सर्वविद्यापरायणः}
{जगद्वीजसमाहार जगतः परमो ह्यसि}


\twolineshloka
{त्रायस्व देवता वीर दानवाद्यैः सुपीडिताः}
{लोकांश्च लोकपालांश्च ऋषींश्च जयतांवर}


\twolineshloka
{वेदाः साङ्गोपनिषदः सरहस्याः ससङ्ग्रहाः}
{सोङ्काराः सवषट्काराः प्राहुस्त्वां यज्ञमुत्तमम्}


\twolineshloka
{पवित्राणां पवित्रं च मङ्गलानां च मङ्गलम्}
{तपस्विनां तपश्चैव दैवतं देवतास्वपि}


\twolineshloka
{एवमादिपुरस्कारैर्ऋक्सामयजुषां गणैः}
{वैकुण्ठं तुष्टुवुर्देवाः सर्वे ब्रह्मर्षिभिः सह}


\twolineshloka
{ततोऽन्तरिक्षे वागासीन्मेघगम्भीरनिस्वना}
{जेष्यध्वं दानवान्यूयं मयैव सह सङ्गरे}


\twolineshloka
{ततो देवगणानां च दानवानां च युध्यताम्}
{प्रादुरासीन्महातेजाः शार्ङ्गचक्रगदाधरः}


\twolineshloka
{सुपर्णपृष्ठमास्थाय तेजसा प्रदहन्निव}
{व्यधमद्दानवान्सर्वान्बाहुद्रविणतेजसा}


\twolineshloka
{तं समासाद्य समरे दैत्यदानवपुङ्गवाः}
{व्यनश्यन्त महाराज पतङ्गा इव पावकम्}


\twolineshloka
{स विजित्यासुरान्सर्वान्दानवांश्च महामतिः}
{पश्यतामेव देवानां तत्रैवान्तरधीयत}


\twolineshloka
{तं दृष्ट्वान्तर्हितं देवा विष्णुं देवामितद्युतिम्}
{विस्मयोत्फुल्लनयना ब्रह्माणमिदमब्रुवन्}


\twolineshloka
{भगवन्सर्वलोकेश सर्वलोकपितामह}
{इदमत्यद्भुतं वृत्तं तन्नः शंसितुमर्हसि}


\twolineshloka
{दैवासुरेऽस्मिन्सङ्ग्रामे त्राता येन वयं विभो}
{एतद्विज्ञातुमिच्छामः कुतोसौ कश्च तत्वतः}


\twolineshloka
{कोऽयमस्मान्परित्राय तूष्णीमेव यथागतम्}
{प्रतिप्रयातो दिव्यात्मा तं नः शंसितुमर्हसि}


\threelineshloka
{एवमुक्तः सुरैः सर्वैर्वचनं वचनार्थवित्}
{उवाच पद्मनाभस्य पूर्वरूपं प्रति प्रभो ॥ब्रह्मोवाच}
{}


\twolineshloka
{न ह्येनं वेद तत्वेन भुवनं भुवनेश्वरम्}
{सङ्ख्यातुं नैव चात्मानं निर्गुणं गुणिनां वरम्}


\twolineshloka
{अत्र ते वर्तयिष्यामि इतिहासं पुरातनम्}
{सुपर्णस्य च संवादमृषीणां चापि देवताः}


\twolineshloka
{पुरा ब्रह्मर्षयश्चैव सिद्दाश्च भुवनेश्वरम्}
{आश्रित्य हिमवत्पृष्ठे चक्रिरे विविधाः कथाः}


\twolineshloka
{तेषां कथयतां तत्र कथान्ते पततां वरः}
{प्रादुरासीन्महातेजा वाहश्चक्रगदाभृतः}


\twolineshloka
{स तानृषीन्समासाद्य विनयावनताननः}
{अवतीर्य महावीर्यस्तानृषीनभिजग्मिवान्}


\twolineshloka
{अभ्यर्चितः स ऋषिभिः स्वागतेन महाबलः}
{उपाविशत तेजस्वी भूमौ वेगवतां वरः}


\twolineshloka
{तमासीनं महात्मानं वैनतेयं महाद्युतिम्}
{ऋषयः परिपप्रच्छुर्महात्मानस्तपस्विनः}


\twolineshloka
{कौतूहलं वैनतेय परं नो हृदि वर्तते}
{तस्य नान्योस्ति वक्तेह त्वामृते पन्नगाशन}


\twolineshloka
{तदाख्यातमिहेच्छामो भवता प्रश्नमुत्तमम्}
{एवमुक्तः प्रत्युवाच प्राञ्जलिर्विनतासुतः}


\twolineshloka
{धन्योस्म्यनुगृहीतोस्मि यन्मां ब्रह्मर्षिसत्तमाः}
{प्रष्टव्यं प्रष्टुमिच्छन्ति प्रीतिमन्तोऽनसूयकाः}


\twolineshloka
{किं मया ब्रूत वक्तव्यं कार्यं च वदतां वराः}
{यूयं हि मां यथायुक्तं सर्वं वै प्रष्टुमर्हथ}


\twolineshloka
{नमस्कृत्वा ह्यनन्ताय ततस्त ऋषिसत्तमाः}
{प्रष्टुं प्रचक्रमुस्तत्र वैनतेयं महाबलम्}


\twolineshloka
{देवदेवं महात्मानं नारायणमनामयम्}
{भवानुपास्ते वरदं कुतोऽसौ कश्च तत्वतः}


\twolineshloka
{प्रकृतिर्विकृतिर्वाऽस्य कीदृशी क्वनु संस्थितिः}
{एतद्भवन्तं पृच्छामो देवोऽयं क्व कृतालयः}


\twolineshloka
{एथ भक्तप्रियो देवः प्रियभक्तस्तथैव च}
{त्वं प्रियश्चास्य भक्तश्च नान्यः काश्यप विद्यते}


\twolineshloka
{मुष्णन्निव मनश्चक्षूंष्यविभाव्यतनुर्विभुः}
{अनादिमध्यनिधनो न विद्मैनं कुतो ह्यसौ}


\twolineshloka
{वेदेष्वपि च विश्वात्मा गीयते न च विद्महे}
{तत्वतस्तत्वभूतात्मा विभुर्नित्यः सनातनः}


\twolineshloka
{पृथिवी वायुराकाशमापो ज्योतिश्च पञ्चमम्}
{गुणाश्चैषां यथासङ्ख्यं भावाभावौ तथैव च}


\threelineshloka
{तमः सत्वं रजश्चैव भावाश्चैव तदात्मकाः}
{मनो बुद्धिश्च तेजश्च बुद्धिगम्यानि तत्वतः}
{जायन्ते तात तस्माद्धि तिष्ठते तेष्वसौ विभुः}


\twolineshloka
{सञ्चिन्त्य बहुधा बुद्ध्या नाध्यवस्यामहे परम्}
{तस्य देवस्य तत्वेन तन्नः शंस यथातथम्}


\threelineshloka
{एतमेव परं प्रश्नं कौतूहलसमन्विताः}
{एवं भवन्तं पृच्छामस्तन्नः शंसितुमर्हसि ॥सुपर्ण उवाच}
{}


\twolineshloka
{स्थूलतो यस्तु भगवांस्तेनैव स्वेन हेतुना}
{त्रैलोक्यस्य तु रक्षार्थं दृश्यते रूपमास्थितः}


\twolineshloka
{मया तु महदाश्चर्यं पुरा दृष्टं सनातने}
{देवे श्रीवत्सनिलये तच्छृणुध्वमशेषतः}


\twolineshloka
{न स्म शक्यो मया वेत्तुं न भवद्भिः कथञ्चन}
{यथा मां प्राह भगवांस्तथा तच्छ्रुयतां मम}


\twolineshloka
{मयाऽमृतं देवतानां मिषतामृषिसत्तमाः}
{हृतं विपाट्य तं यन्त्रं विद्राव्यामृतरक्षिणः}


\twolineshloka
{देवता विमुखीकृत्य सेन्द्राः समरुतो मृधे}
{उन्मथ्याशु गिरींश्चैव विक्षोभ्य च महोदधिं}


\threelineshloka
{तं दृष्ट्वा मम विक्रान्तं वागुवाचाशरीरिणी}
{प्रीतोस्मि ते वैनतेय कर्मणाऽनेन सुव्रत}
{अवृथा तेऽस्तु मद्वाक्यं ब्रूहि किं करवाणि ते}


\threelineshloka
{तामेवंवादिनीं वाचमहं प्रत्युक्तवांस्तदा}
{ज्ञातुमिच्छामि कस्त्वं हि ततो मे दास्यसे वरम्}
{प्रकृतिर्विकृतिर्वा त्वं देवो वा दानवोपि वा}


\twolineshloka
{ततो जलदगम्भीरं प्रहस्य वदतांवरः}
{उंवाच वरदः प्रीतः काले त्वं माऽभिवेत्स्यसि}


\twolineshloka
{वाहनं भव मे साधो वरं दद्मि तवोत्तमम्}
{न ते वीर्येण सदृशः कश्चिल्लोके भविष्यति}


\twolineshloka
{पतङ्ग पततांश्रेष्ठ न देवो नापि दानवः}
{मत्सखित्वमनुप्राप्तो दुर्धर्षश्च भविष्यसि}


\twolineshloka
{तमब्रवं देवदेवं मामेवंवादिनं परम्}
{प्रयतः प्राञ्जलिर्भूत्वा प्रणम्य शिरसा विभुम्}


\twolineshloka
{एवमेतन्महाबाहो सर्वमेतद्भविष्यति}
{वाहनं ते भविष्यामि यथा वदति मां भवान्}


\twolineshloka
{मम चापि महाबुद्धे निश्चयं श्रूयतामिति}
{ध्वजस्तेऽहं भविष्यामि रथस्थस्य न संशयः}


\twolineshloka
{तथास्त्विति स मामुक्त्वा भूयः प्राह महामनाः}
{न ते गतिविघातोऽद्य भविष्यत्यमृतं विना}


\twolineshloka
{एवं कृत्वा तु समयं देवदेवः सनातनः}
{मामुक्त्वा साधयस्वेति यथाऽभिप्रायतो गतः}


\twolineshloka
{ततोऽहं कृतसंवादो येन केनापि सत्तमाः}
{कौतूहलसमाविष्टः पितरं कश्यपं गतः}


\twolineshloka
{सोहं पितरमासाद्य प्रणिपत्याभिवाद्य च}
{सर्वमेतद्यथातथ्यमुक्तवान्पितुरन्तिके}


\twolineshloka
{श्रुत्वा तु भगवान्मह्यं ध्यानमेवान्वपद्यत}
{स मुहूर्तमिव ध्यात्वा मामाह वदतां वरः}


\twolineshloka
{धन्योस्यनुगृहीतश्च यत्त्वं तेन महात्मना}
{संवादं कृतवांस्तात गुह्येन परमात्मना}


\twolineshloka
{स्थूलदृश्यः स भगवांस्तेन तेनैव हेतुना}
{दृश्यतेऽव्यक्तरूपस्थः प्रधानप्रभवाप्ययः}


\twolineshloka
{मया हि स महातेजा नान्ययोगसमाधिना}
{तपसोग्रेण तेजस्वी तोषितस्तपसांनिधिः}


\twolineshloka
{ततो मे दर्शयामास तोषयन्निव पुत्रक}
{श्वेतपीतारुणनिभः कद्रूकपिलपिङ्गलः}


\twolineshloka
{रक्तनीलासितनिभः सहस्रोदरपाणिमान्}
{द्विसाहस्रमहावक्त्र एकाक्षः शतलोचनः}


\twolineshloka
{अनिष्पन्दा निराहाराः समानाः सूर्यतेजसा}
{तमुपासन्ति परमं गुह्यमक्षरमव्ययम्}


\twolineshloka
{समासाद्य तु तं विश्वमहं मूर्ध्ना प्रणम्य च}
{ऋग्यजुःसामभिः स्तुत्वा शरण्यं शरणं गतः}


\twolineshloka
{महामेघौघधीरेण स्वरेण जयतांवरः}
{आभाष्य पुत्रपुत्रेति इदमाह धृतं वचः}


\twolineshloka
{त्वयाऽभ्युदयकामेन तपश्चीर्णं महामुने}
{अमुक्तस्त्वं समासङ्गैरविमुक्तोऽद्य पश्यसि}


\twolineshloka
{यदा सङ्गैर्विमुक्तश्च गतमोहो गतस्पृहः}
{भविष्यसि सदा ब्रह्म मामनुध्यास्यसे द्विज}


\twolineshloka
{ऐकान्तिकीं मतिं कृत्वा मद्भक्तो मत्परायणः}
{ज्ञास्यसे मां ततो ब्रह्मन्वीतमोहश्च तत्वतः}


\threelineshloka
{तेन त्वं कृतसंवादः स्वतः सर्वहितैषिणा}
{विश्वरूपेण देवेन पुरुषेण महात्मना}
{तमेवाराधय क्षिप्रं तमाराध्य न सीदसि}


\twolineshloka
{सोहमेवं भगवता पित्रा ब्रह्मर्षिसत्तमाः}
{अनुनीतो यथान्यायं स्वमेव भवनं गतः ॥'}


\chapter{अध्यायः ४१}
\twolineshloka
{सोऽहमामन्त्र्य पितरं तद्भावगतमानसः}
{स्वमेवालयमासाद्य तमेवार्थमचिन्तयम्}


\twolineshloka
{तद्भावगतभावात्मा तद्भूतगतमानसः}
{गोविन्दं चिन्तयन्नासे शाश्वतं परमव्ययम्}


\threelineshloka
{धृतं बभूव हृदयं नारायणदिदृक्षया}
{सोहं वेगं समास्थाय मनोमारुतवेगवान्}
{रम्यां विशालां बदरीं गतो नारायणाश्रमम्}


\threelineshloka
{ततस्तत्र हरिं जगतः प्रभवं विभुम्}
{गोविन्दं पुण्डरीकाक्षं प्रणतः शिरसा हरिम्}
{ऋग्यजुस्सामभिश्चैनं तुष्टाव परया मुदा}


% Check verse!
अथापश्यं सुविपुलमश्वत्थं देवसंश्रयम्
\threelineshloka
{चतुर्द्विगुणपीनांसः शङ्कचक्रगदाधरः}
{प्रादुर्बभूव पुरुषः पीतवासाः सनातनः}
{मध्याह्नार्कप्रतीकाशस्तेजसा भासयन्दिशः}


\threelineshloka
{संस्तुतः संविदं कृत्वा व्रजेति श्रेयसे रतः}
{प्रागुदीचीं दिशं देवः प्रतस्थे पुरुषोत्तमः}
{दिशश्च विदिशश्चैव भासयन्स्वेन तेजसा}


\twolineshloka
{तमहं पुरुषं दिव्यं व्रजन्तममितौजसम्}
{अनुवव्राज वेगेन शनैर्गच्छन्तमव्ययम्}


\twolineshloka
{योजनानां सहस्राणि षष्टिमष्टौ तथा शतम्}
{तथा शतसहस्रं च शतं द्विगुणमेव च}


\twolineshloka
{स गत्वा दीर्गमध्वानमपश्यमहमद्भुतम्}
{महान्तं पावकं दीप्तमर्चिष्मन्तमनिन्धनम्}


\twolineshloka
{शतयोजनविस्तीर्णं तस्माद्द्विगुणमायतम्}
{विवेश स महायोगी पावकं पावकद्युतिः}


\twolineshloka
{तत्र शंभुस्तपस्तेपे महादेवः सहोमया}
{स तेन संविदं कृत्वा पावकं समतिक्रमत्}


% Check verse!
श्रमाभिभूतेन मया कथञ्चिदनुगम्यते
\twolineshloka
{गत्वा स दीर्घमध्वानं भास्करेणावभासितम्}
{अभास्करममर्यादं विवेश सुमहत्तमः}


\twolineshloka
{अथ दृष्टिः प्रतिहता मम तत्र बभूव ह}
{यथास्वभावं भूतात्मा विवेश स महाद्युतिः}


\twolineshloka
{ततोऽहमभवं मूढो जडान्धबधिरोपमः}
{दिशश्च विदिशश्चैव न विजज्ञे तमोवृतः}


\twolineshloka
{अविजानन्नहं किञ्चित्तस्मिंस्तमसि संवृते}
{ससंभ्रान्तेन मनसा व्यथां परमिकां गतः}


\twolineshloka
{सोऽहं प्रपन्नः शरणं देवदेवं सनातनम्}
{प्राञ्चलिर्मनसा भूत्वा वाक्यमेतत्तदोक्तवान्}


\twolineshloka
{भगवन्भूतभव्येश भवद्भूतकृदव्यय}
{शरणं सम्प्रपन्नं मां त्रातुमर्हस्यरिंदम}


\twolineshloka
{अहं तु तत्त्वजिज्ञासुः कोसि कस्यासि कुत्र वा}
{सम्प्राप्तः पदवीं देव स मां संत्रातुमर्हसि}


\threelineshloka
{आविर्भूतः पुराणात्मा मामेहीति सनातनः}
{ततोपरान्ततो देवो विश्वस्य गतिरात्मवान्}
{मोहयामास मां तत्र दुर्विभाव्यवपुर्विभुः}


\twolineshloka
{स्वभावमात्मनस्तत्र दर्शयन्स्वयमात्मना}
{श्रमं मे जनयामास भयं चाभयदः प्रभुः}


\twolineshloka
{खिन्न इत्येव मां मत्वा भगवानव्ययोऽच्युतः}
{शब्देनाश्वासयामास जगाहे च तमो महत्}


\twolineshloka
{अहं तमेवानुगतः श्रमालसपदश्चरन्}
{मनसा देवदेवेशं ध्यातुं समुपचक्रमे}


\twolineshloka
{तथागतं तु मां ज्ञात्वा भगवानमितद्युतिः}
{तमः प्रणाशयामास ममानुग्रहकाङ्क्षया}


\twolineshloka
{ततः प्रनष्टे तमसि तमहं दीप्ततेजसम्}
{अपश्यं तेजसा व्याप्तं मध्याह्न इव भास्करम्}


\twolineshloka
{स्वयंप्रभांश्च पुरुषान्स्त्रियश्च परमाद्भुताः}
{अपश्यमहमव्यग्रस्तस्मिन्देशे सहस्रशः}


\twolineshloka
{न तत्र द्योतते सूर्यो नक्षत्राणि तथैव च}
{न तत्र चन्द्रमा भाति न वायुर्वाति पांसुलः}


\twolineshloka
{तत्र तूर्याण्यनेकानि गीतानि मधुराणि च}
{अदृश्यानि मनोज्ञानि श्रूयन्ते सर्वतोदिशम्}


\twolineshloka
{स्रवन्ति वैडूर्यलताः पद्मोत्पलझषाकुलाः}
{मुक्तासिकतवप्राश्च सरितो निर्मलोदकाः}


\twolineshloka
{अगतिस्तत्र देवानामसुराणां तथैव च}
{गन्धर्वनागयक्षाणां राक्षसानां तथैव च}


\twolineshloka
{स्वयंप्रभास्तत्र नरा दृश्यन्तेऽद्भुतदर्शनाः}
{येषां न देवतास्तुल्याः प्रभाभिर्भावितात्मनाम्}


\twolineshloka
{स च तानप्यतिक्रम्य दैवतैरपि पूजितः}
{विवेश ज्वलनं दीप्तमनिन्धनमनौपमम्}


\twolineshloka
{ज्वालाभिर्मां प्रविष्टं च ज्वलन्तं सर्वतोदिशम्}
{दैत्यदानवरक्षोभिर्दैवतैश्चापि दुस्सहम्}


\twolineshloka
{ज्वालामालिनमासाद्य तमग्निमहमव्ययम्}
{अविषह्यतमं मत्वा मनसेदमचिन्तयम्}


\twolineshloka
{मया हि समरेष्वग्निरनेकेषु महाद्युतिः}
{प्रविष्टश्चापविद्धश्च न च मां दग्धवान्क्वचित्}


\twolineshloka
{अयं च दुस्सहः शश्वत्तेजसाऽतिहुताशनः}
{अत्यादित्यप्रकाशार्चिरनलो दीप्यते महान्}


\twolineshloka
{स तथा दह्यमानोपि तेजसा दीप्तवर्चसा}
{प्रपन्नः शरणं देवं शङ्कचक्रगदाधरम्}


\twolineshloka
{भक्तश्चानुगतश्चेति त्रातुमर्हसि मां विभो}
{यथा मां न दहेदग्निः सद्यो देव तथा कुरु}


\twolineshloka
{एवं विलपमानस्य ज्ञात्वा मे वचनं प्रभुः}
{मा भैरिति वचः प्राह मेघगम्बीरनिस्वनः}


\twolineshloka
{स मामाश्वास्य वचनं प्राहेदं भगवान्विभुः}
{मम त्वं विदितः सौम्य यथावत्तत्वदर्शने}


\twolineshloka
{ज्ञापितश्चापि यत्पित्रा तच्चापि विदितं महत्}
{वैनतेय ममाप्येवमहं वेद्यः कथञ्चन}


\twolineshloka
{महदेतत्स्वरूपं मे न ते वेद्यं कथञ्चन}
{मां हि विन्दन्ति विद्वांसो ये ज्ञाने परिनिष्ठिताः ॥निर्ममा निरहङ्कारा निराशीर्बन्धनायुताः}


\twolineshloka
{भवांस्तु सततं भक्तो मन्मनाः पक्षिसत्तम}
{स्थूलंमां वेत्स्यसे तस्माज्जगतः कारणेस्थितम्'}


\chapter{अध्यायः ४२}
\twolineshloka
{एवं दत्ताभयस्तेन ततोऽहमृषिसत्तमाः}
{नष्टखेदश्रमभयः क्षणेन ह्यभवं तदा}


\twolineshloka
{स शनैर्याति भगवान्गत्या लघुपराक्रमः}
{अहं तु सुमहावेगमास्तायानुव्रजामि तम्}


\twolineshloka
{स गत्वा दीर्घमध्वानमाकाशममितह्युतिः}
{मनसाऽप्यगमं देवमाससादात्मतत्ववित्}


\twolineshloka
{अथ देवः समासाद्य मनसः सदृशं जवम्}
{मोहयित्वा च मां तत्र क्षणेनान्तरधीयत}


\twolineshloka
{तत्राम्बुधरधीरेण भोशब्देनानुनादिना}
{अयं भोऽहमिति प्राह वाक्यं वाक्यविशारदः}


\twolineshloka
{शब्दानुसारी तु ततस्तं देशमहमाव्रजम्}
{तत्रापश्यं ततश्चाहं श्रीमद्धंसयुतं सरः}


\twolineshloka
{स तत्सरः समासाद्य भगवानात्मवित्तमः}
{भोशब्दप्रतिसृष्टेन स्वरेण प्रतिवादिना}


\twolineshloka
{विवेश देवः स्वां योनिं मामिदं चाभ्यभाषत}
{विशस्व सलिलं सौम्य सुखमत्र वसामहे}


\twolineshloka
{ततश्च प्राविशं तत्र सह तेन महात्मना}
{दृष्टवानद्भुततरं तस्मिन्सरसि भास्वताम्}


\twolineshloka
{अग्नीनामप्रणीतानामिद्धानामिन्धनैर्विना}
{दीप्तानामाज्यसिक्तानां स्यानेष्वर्चिष्मतां सदा}


\twolineshloka
{दीप्तिस्तेषामनाज्यानां प्राप्ताज्यानामिवाभवत्}
{अनिद्धानामिव सतामिद्धानामिव भास्वताम्}


\twolineshloka
{अथाहं वरदं देवं नापश्यं तत्र सङ्गतम्}
{ततः सम्मोहमापन्नो विषादभगमं परम्}


\twolineshloka
{अपश्यं चाग्निहोत्राणि शतशोऽथ सहस्रशः}
{विधिना सम्प्रणीतानि धिष्ण्येष्वाज्यवतां तदा}


\twolineshloka
{असंभृष्टतलाश्चैव वेदीः कुसुमसंस्तृताः}
{कुशपद्मोत्पलासङ्गाः कलशांश्च हिरण्मयान्}


\threelineshloka
{अग्निहोत्राणि चित्राणि शतशोऽथ सहस्रशः}
{अग्निहोत्राय योग्यानि यानि द्रव्याणि कानिचित्}
{तानि चात्र समृद्धानि दृष्टवानस्म्यनेकशः}


\twolineshloka
{मनोहृद्यतमश्चाग्निः सुरभिः पुण्यलक्षणः}
{आज्यगन्धो मनोग्राही घ्राणचक्षुस्सुखावहः}


\twolineshloka
{तेषां तत्राग्निहोत्राणामीडितानां सहस्रशः}
{समीपे त्वद्भुततममपश्यमहमव्ययम्}


\threelineshloka
{चन्द्रांशुकाशशुभ्राणां तुषारोद्भेदवर्चसाम्}
{विमलादित्यभासानां स्थण्डिलानि सहस्रशः}
{दृष्टान्यग्निसमीपे तु ध्युतिमन्ति महान्ति च}


\threelineshloka
{एषु चाग्निसमीपेषु शुश्राव सुपदाक्षराः}
{प्रभावान्तरितानां तु प्रस्पष्टाक्षरभाषिणाम्}
{ऋग्यजुःसामगानां च मधुराः सुस्वरा गिरः}


\twolineshloka
{सुसंमृष्टतलैस्तैस्तु बृहद्बिर्दीप्ततेजसैः}
{पावकैः पावितात्माहमभवं लघुविक्रमः}


\twolineshloka
{ततोऽहं तेषु धिष्ण्येषु ज्वलमानेषु यज्वनाम्}
{तं देशं प्रणमित्वाऽथ अन्वेष्टुमुपचक्रमे}


\twolineshloka
{तान्यनेकसहस्राणि पर्यटंस्तु महाजवात्}
{अपश्यमानस्तं देवं ततोऽहं व्यथितोऽभवम्}


\twolineshloka
{ततस्तेष्वग्निहोत्रेषु ज्वलत्सु विमलार्चिषु}
{भानुमत्सु न पश्यामि देवदेवं सनातनम्}


\twolineshloka
{ततोऽहं तानि दीप्तानि परीय व्यस्थितेन्द्रियः}
{नान्तं तेषां प्रपश्यामि खेदश्च सहसाभवत्}


\twolineshloka
{विसृत्य सर्वतो दृष्टिं भयमोहसमन्वितः}
{श्रमं परंममापन्नश्चिन्तयानस्त्वचेतनः}


\twolineshloka
{तस्मिन्न खलु वर्तेऽहं लोके यत्रैतदीदृशम्}
{ऋग्यजुस्सामनिर्घोषः श्रूयते न च दृश्यते}


\twolineshloka
{न च पश्यामि तं देवं येनाहमिह चोदितः}
{एवं चिन्तासभापन्नः प्रध्यातुमुपचक्रमे}


\twolineshloka
{ततश्चिन्तयतो मह्यं मोहेनाविष्टचेतसः}
{महाशब्दः प्रादुरासीत्सुभृशं मे व्यथाकरः}


\twolineshloka
{अथाहं सहसा तत्र शृणोमि विपुलध्वनिम्}
{अपश्यं च सुपर्णानां सहस्राण्ययुतानि च}


\twolineshloka
{अभ्यद्रवन्त मामेव विपुलद्युतिरंहसः}
{तेषामहं प्रभावेण सर्वथैवावरोऽभवम्}


\twolineshloka
{सोऽहं समन्ततः सर्वैः सुपर्णैरतितेजसैः}
{दृष्ट्वाऽऽत्मानं परिगतं सम्भ्रमं परमं गतः}


\twolineshloka
{विनयावनतो भूत्वा नमश्चक्रे महात्मने}
{अनादिनिधनायैभिर्नामभिः परमात्मने}


\twolineshloka
{नारायणाय शुद्धाय शाश्वताय ध्रुवाय च}
{भूतभव्यभवेशाय शिवाय शिवमूर्तये}


\twolineshloka
{शिवयोनेः शिवाद्यायि शिवपूज्यतमाय च}
{घोररूपाय महते युगान्तकरणाय च}


\twolineshloka
{विश्वाय विश्वदेवाय विश्वेशाय महात्मने}
{सहस्रोदरपादाय सहस्रनयनाय च}


\twolineshloka
{सहस्रबाहवे चैव सहस्रवदनाय च}
{शुचिश्रवाय महते ऋतुसंवत्सराय च}


\twolineshloka
{ऋग्यजुःसामवक्त्राय अथर्वशिरसे नमः}
{हृषीकेशाय कृष्णाय द्रुहिणोरुक्रमाय च}


\twolineshloka
{बृहद्वेगाय तार्क्ष्याय वराहायैकशृङ्गिणे}
{शिपिविष्टाय सत्याय हरयेऽथ शिखण्डिने}


\twolineshloka
{हुताशायोर्ध्ववक्त्राय रौद्रानीकाय साधवे}
{सिन्धवे सिन्धुवर्षघ्ने देवानां सिन्धवे नमः}


\twolineshloka
{गरुत्मते त्रिनेत्राय सुधर्माय वृषाकृते}
{सम्म्राडुग्रे संकृतये विरजे सम्भवे भवे}


\twolineshloka
{वृषाय वृषरूपाय विभवे भूर्भुवाय व}
{दीप्तसृष्टाय यज्ञाय स्थिराय स्थविराय च}


\twolineshloka
{अच्युताय तुषाराय वीराय च समाय च}
{जिष्णवे पुरुहूताय वसिष्ठाय वराय च}


\twolineshloka
{सत्येशाय सुरेशाय हरयेऽथ शिखण्डिने}
{बर्हिषाय वरेण्याय वसवे विश्ववेधसे}


\twolineshloka
{किरीटिने सुकेशाय वासुदेवाय शुष्मिणे}
{बृहदुक्थ्यसुषेणाय युग्ये दुन्दुभये तथा}


\twolineshloka
{भयेसखाय विभवे भरद्वाजेऽभयाय च}
{भास्कराय च चन्द्राय पद्मनाभाय भूरिणे}


\twolineshloka
{पुनर्वसुभृतत्वाय जीवप्रभविषाय च}
{वषट्काराय स्वाहाय स्वधाय निधनाय च}


\twolineshloka
{ऋचे च यजुषे साम्ने त्रैलोक्यपतये नमः}
{श्रीपद्मायात्मसदृशे धरणीधारिणे परे}


\twolineshloka
{सौम्यासौम्यस्वरूपाय सौम्ये सुमनसे नमः}
{विश्वाय च सुविश्वाय विश्वरूपधराय च}


\twolineshloka
{केशवाय सुकेशाय रश्मिकेशाय भूरिणे}
{हिरण्यगर्भाय नमः सौम्याय वृषरूपिणे}


\threelineshloka
{नारायणाग्र्यवपुषे पुरुहूताय वज्रिणे}
{वर्मिणे वृषसेनाय धर्मसेनाय रोधसे}
{मुनये ज्वरमुक्तायि ज्वराधिपतये नमः}


\twolineshloka
{अनेत्राय त्रिनेत्राय पिङ्गलाय विडूर्मिणे}
{तपोब्रह्मनिधानाय युगपर्यायिणे नमः}


\twolineshloka
{शरणाय शरण्याय भक्तेष्टशरणाय च}
{नमः सर्वभवेशाय भूतभव्यभवाय च}


\twolineshloka
{पाहि मां देवदेवेश कोप्यजोसि सनातनः}
{एवं गतोस्मि शरणं शरण्यं ब्रह्मयोनिनम्}


\twolineshloka
{स्तव्यं स्तवं स्तुतवतस्तत्तमो मे प्रणश्यत}
{भयं च मे व्यपगतं पक्षिणोऽन्तर्हिताऽभवन्}


\threelineshloka
{शृणोमि च गिरं दिव्यामन्तर्धानगतां शिवाम्}
{मा भैर्गरुत्मन्दान्तोसि पुनः सेन्द्रान्दिवौकसः}
{स्वं चैव भवनं गत्वा द्रक्ष्यसे पुत्रबान्धवान्}


\twolineshloka
{ततस्तस्मिन्क्षणेनैव सहसैव महाद्युतिः}
{प्रत्यदृश्यत तेजस्वी पुरस्तात्स ममान्तिके}


\threelineshloka
{समागम्य ततस्तेन शिवेन परमात्मना}
{अपश्यं चाहमायान्तं नरनारायणाश्रमे}
{चतुर्द्विगुणविन्यासं तं च देवं सनातनम्}


\twolineshloka
{यजतस्तानृषीन्देवान्वदतो ध्यायतो मुनीन्}
{युक्तान्सिद्धान्नैष्ठिकांश्च जपतो यजतो गृहे}


\twolineshloka
{पुष्पपूरपरिक्षिप्तं धूपितं दीपितं हुतम्}
{वन्दितं सिक्तसम्मृष्टं नरनारायणाश्रमम्}


\twolineshloka
{तदद्भुतमहं दृष्ट्वा विस्मितोस्मि तदाऽनघाः}
{जगाम शिरसा देवं प्रयतेनान्तरात्मना}


\twolineshloka
{तदत्यद्भुतसङ्कासं किमेतदिति चिन्तयन्}
{नाध्यगच्छं परं दिव्यं तस्य सर्वभवात्मनः}


\threelineshloka
{प्रणिपत्य सुदुर्धर्षं पुनः पुनरुदीक्ष्य च}
{शिरस्यञ्जलिमाधाय विस्मयोत्फुल्ललोचनः}
{अवोचं तमदीनार्थं श्रेष्ठानां श्रेष्ठमुत्तमम्}


\twolineshloka
{नमस्ते भगवन्देव भूतभव्यभवत्प्रभो}
{यदेतदद्भुतं देव मया दृष्टं त्वदाश्रयम्}


\threelineshloka
{अनादिमद्यपर्यन्तं किं तच्छंसितुमर्हसि}
{यदि जानासि सां भक्तं यदि वाऽनुग्रहो मयि}
{शंस सर्वमशेषेण श्रोतव्यं यदि चेन्मया}


\twolineshloka
{स्वभावस्तव दुर्ज्ञेयः प्रादुर्भावो भवस्य च}
{भवद्भूतभविष्येश सर्वथा गहनं भवान्}


\twolineshloka
{ब्रूहि सर्वमशेषेण तदाश्चर्यं महामुने}
{किं तदत्यद्भुतं वृत्तं तेष्वग्निषु समन्ततः}


\twolineshloka
{कानि तान्यग्निहोत्राणि केषां शब्दः श्रुतो मया}
{शृण्वतां ब्रह्म सततमदृश्यानां महात्मनाम्}


\twolineshloka
{एतन्मे भगवन्कृष्ण ब्रूहि सर्वमशेषतः}
{गृणन्त्यग्निसमीपेषु के च ते ब्रह्मराशयः ॥'}


\chapter{अध्यायः ४३}
\twolineshloka
{मां न देवा न गन्धर्वा नासुरा न व राक्षसाः}
{विदुस्तत्वेन सत्वस्थं सूक्ष्मात्मानमवस्थितम्}


\twolineshloka
{चतुर्धाऽहं विभक्तात्मा लोकानां हितकाम्यया}
{भूतभव्यभविष्यादिरनादिर्विश्वकृत्तमः}


\twolineshloka
{पृथिवी वायुराकाशमापो ज्योतिश्च पञ्चमम्}
{मनो बुद्धिश्च चेतश्च तमः सत्वं रजस्तथा}


\twolineshloka
{प्रकृतिर्विकृतिश्चैव विद्याविद्ये शुभाशुभे}
{मत्त एतानि जायन्ते नाहमेभ्यः कथञ्चन}


\twolineshloka
{यत्किंचिच्छ्रेयसा युक्तं श्रेयस्करमनुत्तमम्}
{धर्मयुक्तं च पुण्यं च सोऽहमस्मि निरामयः}


\twolineshloka
{यत्स्वभावात्मतत्वज्ञैः कारणैरुपलक्ष्यते}
{अनादिमध्यनिधनः सोन्तरात्माऽस्मि शाश्वतः}


\twolineshloka
{यत्तु मे परमं गुह्यं रूपं सूक्ष्मार्थदर्शिभिः}
{गृह्यते सूक्ष्मभावज्ञैः सोऽविभाव्योस्मि शाश्वतः}


\twolineshloka
{तत्तु मे परमं गुह्यं येन व्याप्तमिदं जगत्}
{सोहङ्गतः सर्वसत्वः सर्वस्य प्रभवोऽव्ययः}


\twolineshloka
{मत्तो जायन्ति भूतानि मया धार्यन्त्यहर्निशम्}
{मय्येव विलयं यान्ति प्रलये पन्नगाशन}


\twolineshloka
{यो मां यथा वेदयति तथा तस्यास्मि काश्यप}
{मनोबुद्धिगतः श्रेयो विदधामि विहङ्गम}


\twolineshloka
{मां तु ज्ञातुं कृता बुद्धिर्भवता पक्षिसत्तम}
{शृणु योऽहं यतश्चाहं यदर्थश्चाहमुद्यतः}


\twolineshloka
{ये केचिन्नियतात्मानस्त्रेताग्निपरमार्चिताः}
{अग्निकार्यपरा नित्यं जपहोमपरायणाः}


\twolineshloka
{आत्मन्यग्नीन्समाधाय नियता नियतेन्द्रियाः}
{अनन्यमनसस्ते मां सर्वे वै समुपासते}


\twolineshloka
{यजन्तो जपयज्ञैर्मां मानसैश्च सुसंयताः}
{अग्नीनभ्युद्ययुः शश्वदग्निष्वेवाभिसंश्रिताः}


\twolineshloka
{अनन्यकार्याः शुचयो नित्यमग्निपरायणाः}
{य एवंबुद्ध्यो धीरास्ते मां गच्छन्ति तादृशाः}


\twolineshloka
{अकामहतसङ्कल्पा ज्ञाने नित्यं समाहिताः}
{आत्मन्यग्निं समाधाय निराहारा निराशिषः}


\twolineshloka
{विषयेषु निरारम्भा विमुक्ता ज्ञानचक्षुषः}
{अनन्यमनसो धीराः स्वभावनियमान्विताः}


\twolineshloka
{यत्तद्वियति दृष्टं तत्सरः पद्मोत्पलायुतम्}
{तत्राग्नयः सन्निहिता दीप्यन्ते स्म निरिन्धनाः}


\threelineshloka
{ज्ञानामलाशयास्तस्मिन्ये च चन्द्रांशुनिर्मलाः}
{उपासीना गृणन्तोऽग्निमस्पष्टाक्षरभाषिणः}
{आकाङ्क्षमाणाः शुचयस्तेष्वग्रिषु विहङ्गम}


\twolineshloka
{ये मया भावितात्मानो मय्येवाभिरताः सदा}
{उपासते च मामेव ज्योतिर्भूता निरामयाः}


\twolineshloka
{तैर्हि तत्रैव वस्तव्यं नीरागादिभिरच्युतैः}
{निराहारा ह्यनिष्पन्दाश्चन्द्रांशुसदृशप्रभाः}


\twolineshloka
{निर्मला निरहङ्कारा निरालम्बा निराशिषः}
{मद्भक्ताः सततं तेवै भक्तांस्तानपि चाप्यहम्}


\twolineshloka
{चतुर्धाऽहं विभक्तात्मा चरामि जगतो हितः}
{लोकानां धारणार्थाय विधानं विदधामि च}


% Check verse!
यथावत्तदशेषेण श्रोतुमर्हति मे भवान्
\twolineshloka
{एका मूर्तिर्निर्गुणाख्या योगं परममास्थिता}
{द्वितीया सृजते तात भूतग्रामं चराचरम्}


\threelineshloka
{सृष्टं संहरते चैका जगत्स्थावरजङ्गमम्}
{ज्ञातात्मनिष्ठा क्षपयन्मोहयन्तीव मायया}
{क्षपयन्ती मोहयति आत्मनिष्ठा स्वमायया}


\twolineshloka
{चतुर्थी मे महामूर्तिर्जगद्वृद्धिं ददाति सा}
{रक्षते चापि नियता सोहमस्मि नभश्वरः}


\twolineshloka
{मया सर्वमिदं व्याप्तं मयि सर्वं प्रतिष्ठितम्}
{अहं सर्वजगद्बीजं सर्वत्रगतिरव्ययः}


\twolineshloka
{यानि तान्यग्निहोत्राणि ये च चन्द्रांशुराशयः}
{गृणन्ति वेदं सततं तेष्वग्निषु विहङ्गम}


\threelineshloka
{क्रमेण मां समायान्ति सुखिनो ज्ञानसंयुताः}
{तेषामहं तपो दीप्तं तेजः सम्यक्समाहितम्}
{नित्यं ते मयि वर्तन्ते तेषु चाहमतन्द्रितः}


\twolineshloka
{सर्वतो मुक्तसङ्गेन मय्यनन्यसमाधिना}
{शक्यः समासादयितुमहं वै ज्ञानचक्षुषा}


\twolineshloka
{मां स्थूलदर्शनं विद्धि जगतः कार्यकारणम्}
{मत्तश्च सम्प्रसूतान्वै विद्धि लोकान्सदैवतान्}


\twolineshloka
{मया चापि चतुर्धात्मा विभक्तः प्राणिषु स्यिथः}
{आत्मभूतो वासुदेवो ह्यनिरुद्धो मतौ स्यितः}


\twolineshloka
{सङ्कर्षणोऽहङ्कारे च प्रद्युम्नो मनसि स्यितः}
{अन्यथा च चतुर्दा यत्सम्यक्त्वं श्रोतुमर्हसि}


\twolineshloka
{यत्तत्पद्ममभूत्पूर्वं तत्र ब्रह्मा व्यजायत}
{ब्राह्मणश्चापि सम्भूतः शिव इत्यवधार्यताम्}


\twolineshloka
{शिवात्स्कन्दः संवभूव एतत्सृष्टिचतुष्टयम्}
{दैत्यदानवदर्पघ्नमेवं मां विद्धि नित्यशः}


\twolineshloka
{दैत्यदानवरक्षोभिर्यदा धर्मः प्रपीड्यते}
{तदाऽहं धर्मवृद्ध्यर्थं मूर्तिमान्भविताऽऽशुग}


\twolineshloka
{वेदव्रतपरा ये तु धीरा निश्चितबुद्ध्यः}
{योगिनो योगयुक्ताश्च ते मां पश्यन्ति नान्यथा}


\twolineshloka
{पञ्चभिः सम्प्रयुक्तोऽहं विप्रयुक्तश्च पञ्चभिः}
{वर्तमानश्च तेष्वेवं निवृत्तश्चैव तेष्वहम्}


% Check verse!
ये विदुर्जातसङ्कल्पास्ते मां पश्यन्ति तादृशाः
\threelineshloka
{स्वं वायुरापो ज्योतिश्च पृथिवी चेति पञ्चमम्}
{तदात्मकोऽस्मि विज्ञेयो न चान्योस्मीति निश्चितम्}
{}


\twolineshloka
{वर्तमानमतीतं च पञ्चवर्गेषु निश्चलम्}
{शब्दस्पर्शेषु रूपेषु रसगन्धेषु चाप्यहम्}


\twolineshloka
{रजस्तमोभ्यामाविष्टा येषां बुद्धिरनिश्चिता}
{ते न पश्यन्ति मे तत्वं तपसा महता ह्यपि}


\twolineshloka
{नोपवासैर्न नियमैर्न व्रतैर्विविधैरपि}
{द्रष्टुं वा वेदितुं वाऽपि न शक्या परमा गतिः}


\twolineshloka
{महामोहार्थपङ्के तु निमग्रानां गतिर्हरिः}
{एकान्तिनो ध्यानपरा यतिभावाद्ब्रजन्ति माम्}


\twolineshloka
{सत्वयुक्ता मतिर्येषां केवलाऽऽत्मविनिश्चिता}
{ते पश्यन्ति स्वमात्मानं परमात्मानमव्ययम्}


\twolineshloka
{अहिंसा सर्वभूतेषु तेष्ववस्तितमार्जवम्}
{तेष्वेव च समाधाय सम्यगेति च मामजम्}


\twolineshloka
{यदेतत्परमं गुह्यमाख्यानं परमाद्भुतम्}
{यत्तेन तदशेषेण यथावच्छ्रोतुमर्हसि}


\twolineshloka
{ये त्वग्निहोत्रनियता जपयज्ञपरायणाः}
{ते मामुपासते शश्वद्यांस्तांस्त्वं दृष्टवानसि}


\twolineshloka
{शास्त्रदृष्टविधानज्ञा असक्ताः क्वचिदन्यथा}
{शक्योऽहं वेदितुं तैस्तु यन्मे परममव्ययम्}


\twolineshloka
{ये तु सांख्यं च योगं च ज्ञात्वाऽप्यधृतनिश्चयाः}
{न ते गच्छन्ति कुशलाः परां गतिमनुत्तमाम्}


\twolineshloka
{तस्माज्ज्ञानेन शुद्धेन प्रसन्नात्माऽऽन्मविच्छुचिः}
{आसादयति तद्ब्रह्म यत्र गत्वा न शोचति}


\twolineshloka
{शुद्धाभिजनसम्पन्नाः श्रद्धायुक्तेन चेतसा}
{मद्भक्त्या च द्विजश्रेष्ठा गच्छन्ति परमां गतिं}


\twolineshloka
{यद्गह्यं परमं बुद्धेरलिङ्गग्रहणं च यत्}
{तत्सूक्ष्मं गृह्यते विप्रैर्यतिभिस्तत्त्वदर्शिभिः}


\twolineshloka
{न वायुः पवते तत्र न तस्मिञ्ज्योतिषां गतिः}
{न चापः पृथिवी चैव नाकाशं न मनोगतिः}


\twolineshloka
{तस्माच्चैतानि सर्वाणि प्रजायन्ते विहङ्गम}
{सर्वेभ्यश्च स तेभ्यश्च प्रभवत्यमलो विभुः}


\twolineshloka
{स्थूलदर्शनमेतन्मे यद्दृष्टं भवताऽनघ}
{एतत्सूक्ष्मस्य तद्द्वारं कार्याणां कारणं त्वहम्}


\twolineshloka
{दृष्टो वै भवता तस्मात्सरस्यमितविक्रम}
{ब्रह्मणो यदहोरात्रसङ्ख्याभिज्ञैर्विभाव्यते}


\threelineshloka
{एष कालस्त्वया तत्र सरस्यहमुपागतः}
{मां यज्ञमाहुर्यज्ञज्ञा वेदं वेदविदो जनाः}
{मुनयश्चापि मामेव जपयज्ञं प्रचक्षते}


\twolineshloka
{वक्ता मन्ता रसयिता घ्राता द्रष्टा प्रदर्शकः}
{बोद्धा बोधयिता चाहं गन्ता श्रोता चिदात्मकः}


\twolineshloka
{मामिष्ट्वा स्वर्गमायान्ति तथा चाप्नुवते महत्}
{ज्ञात्वा मामेव चैवान्ते निःसङ्गेनान्तरात्मना}


\twolineshloka
{अहं तेजो द्विजातीनां मम तेजो द्विजातयः}
{मम यस्तेजसो देहः सोग्निरित्यवगम्यताम्}


\twolineshloka
{प्राणपालः शरीरेऽहं योगिनामहमीश्वरः}
{सांख्यानामिदमेवाग्रे मयि सर्वमिदं जगत्}


\twolineshloka
{धर्ममर्तं च कामं च मोक्षं चैवार्जवं जपम्}
{तमः सत्वं रजश्चैव कर्मजं च भवाप्ययम्}


\twolineshloka
{स तदाऽहं तथारूपस्त्वया दृष्टः सनातनः}
{ततस्त्वहं परतरः शक्यः कालेन वेदितुम्}


\threelineshloka
{मम यत्परमं गुह्यं शाश्वतं ध्रुवमव्ययम्}
{तदेवं परमो गुह्यो देवो नारायणो हरिः}
{न तच्छक्यं भुजङ्गारे वेत्तुमभ्युदयान्वितैः}


\twolineshloka
{निरारम्भनमस्कारा निराशीर्बन्धनास्तथा}
{गच्छन्ति तं महात्मानः परं ब्रह्म सनातनम्}


\twolineshloka
{स्थूलोऽहमेवं विहग त्वया दृष्टस्तथाऽनघ}
{एतच्चापि न वेत्त्यन्यस्त्वामृते पन्नगाशन}


\twolineshloka
{मा मतिस्तव गान्नाशमेषा गतिरनुत्तमा}
{मद्भक्तो भव नित्यं त्वं ततो वेत्स्यसि मे पदम्}


\twolineshloka
{एतत्ते सर्वमाख्यातं रहस्यं दिव्यमानुषम्}
{एतच्छ्रेयः परं चैतत्पन्थानं विद्धि मोक्षिणाम्}


\twolineshloka
{एवमुक्त्वा स भगवांस्तत्रैवान्तरधीयत}
{पश्यतो मे महायोगी जगामात्मगतिर्गतिम्}


\twolineshloka
{एतदेवंविधं तस्य महिमानं महात्मनः}
{अच्युतस्याप्रमेयस्य दृष्टवानस्मि यत्पुरा}


\twolineshloka
{एतद्वः सर्वमाख्यातं चेष्टितं तस्य धीमतः}
{मयाऽनुभूतं प्रत्यक्षं दृष्ट्वा चाद्भुतकर्मणः ॥'}


\chapter{अध्यायः ४४}
\twolineshloka
{अहो श्रावितमाख्यानं भवताऽत्यद्भुतं महत्}
{पुण्यं यशस्यमायुष्यं स्वर्ग्यं स्वस्त्ययनं महत्}


\twolineshloka
{एतत्पवित्रं देवानामेतद्गुह्यं परंतप}
{एतज्ज्ञानवता ज्ञेयमेषा गतिरनुत्तमा}


% Check verse!
य इमां श्रावयेद्विद्वान्कथां पर्वसुपर्वसुस लोकान्प्राप्नुयात्पुण्यान्देवर्षिभिरभिष्टुतान्
\twolineshloka
{श्राद्धकाले च विप्राणां य इमां श्रावयेच्छुचिः}
{न तत्र रक्षसां भागो नासुराणां च विद्यते}


\twolineshloka
{अनसूयुर्जितक्रोधः सर्वसत्वहिते रतः}
{यः पठेत्सततं युक्तः स व्रजेत्तत्सलोकताम्}


\threelineshloka
{वेदान्पारयते विप्रो राजा विजयवान्भवेत्}
{वैश्यस्तु धनधान्याढ्यः शूद्रः सुखमवाप्नुयात् ॥भीष्म उवाच}
{}


\twolineshloka
{ततस्ते मुनयः सर्वे सम्पूज्य विनतासुतम्}
{स्वानेव चाश्रमाञ्जग्मुर्बभूवुः शान्तितत्पराः}


\twolineshloka
{स्थूलदर्शिभिराकृष्टो दुर्ज्ञेयो ह्यकृतात्मभिः}
{एषा धुतिर्महाराज धर्म्या धर्मभृतांवर}


\threelineshloka
{सुराणां ब्रह्मणा प्रोक्ता विस्मितानां परंतप}
{मयाप्येषा कथा तात कथिता मातुरन्तिके}
{वसुभिः सत्त्वसम्पन्नैस्तवाप्येषा मयोच्यते}


\twolineshloka
{तदग्निहोत्रपरमा जपयज्ञपरायणाः}
{निराशीर्बन्धनाः सन्तः प्रयान्त्यक्षरसात्मतां}


\twolineshloka
{आरम्भयज्ञानुत्सृज्य जपहोमपरायणाः}
{ध्यायन्तो मनसा विष्णुं गच्छन्ति परमां गतिम्}


\twolineshloka
{तदेष परमो मोक्षो मोक्षद्वारं च भारत}
{यथा विनिश्चितात्मानो गच्छन्ति परमां गतिम्}


\chapter{अध्यायः ४५}
% Check verse!
युधिष्ठिर उवाच

युधिष्ठिर उवाच

ययाऽऽपगेय नामानि श्रुतानीह जगत्पतेः

पितामहेशाय विभो नामान्याचक्ष्य शम्भवे ॥बभ्रवे विश्वरूपाय महाभाग्यं च तत्त्वतः

सुरासुरगुरौ देवे शंकरेऽव्यक्तयोनये ॥भीष्म उवाच

अशक्तोऽहं गुणान्यक्तुं महादेवस्य धीमतः

यो हि सर्वगतो देवो न च सर्वत्र दृश्यते ॥ब्रह्मविष्णुसुरेशानां स्रष्टा च प्रभुरेव च

ब्रह्मादयः पिशाचान्ता यं हि देवा उपासते ॥प्रकृतीनां परत्वेन पुरुषस्य च यः परः

चिन्त्यते यो योगविद्भिर्ऋषिभिस्तत्त्वदर्शिभिः

अक्षरं परमं ब्रह्म असच्च सदसच्च यः ॥प्रकृतिं पुरुषं चैव क्षोभयित्वा स्वतेजसा

ब्रह्माणमसृजत्तस्माद्देवदेवः प्रजापतिः ॥को हि शक्तो गुणान्वक्तं देवदेवस्य धीमतः

गर्भजन्मजरायुक्तो मर्त्यो मृत्युसमन्वितः ॥को हि शक्तो भवं ज्ञातुं मद्विधः परमेश्वरम्

क्रते नारायणात्पुत्र शङ्कचक्रगदाधरात् ॥एष विद्वान्गुणश्रेष्ठो विष्णुः परमदुर्जयः ॥दिव्यचक्षुर्महातेजा वीक्ष्यते योगचक्षुषा ॥रुद्रभक्त्या तु कृष्णेन जगद्व्याप्तं महात्मना

तं प्रसाद्य तदा देवं बदर्यां किल भारत ॥अर्थात्प्रियतरत्वं च सर्वलोकेषु वै तदा

प्राप्तवानेव राजेन्द्र सुवर्णाक्षान्महेश्वरात् ॥पूर्णं वर्षसहस्रं तु तप्तवानेष माधवः

प्रसाद्य वरदं देवं चराचरगुरुं शिवम् ॥युगेयुगे तु कृष्णेन तोषितो वै महेश्वरः

भक्त्या परमया चैव प्रीतश्चैव महात्मनः ॥ऐश्वर्यं यादृसं तस्य जगद्योनेर्महात्मनः ॥तदयं दृष्टवान्साक्षात्पुत्रार्थे हरिरच्युतः ॥यस्मात्परतरं चैव नान्यं पश्यामि भारत

व्याख्यातुं देवदेवस्य शक्तो नामान्यशषतः ॥एष शक्तो महाबाहुर्वक्तुं भगवतो गुणान्

विभूतिं चैव कार्त्स्न्येन सत्यां माहेश्वरीं नृप ॥वैशम्पायन उवाच

एवमुक्त्वा तदा भीष्मो वासुदेवं महायशाः

भवमाहात्म्यसंयुक्तमिदमाह पितामहः ॥ पितामह महेशाय नामान्याचक्ष्व शम्भवे

विदुषे विश्वमायाय महाभाग्यं च तत्वतः ॥भीष्म उवाच


\threelineshloka
{युधिष्ठिर उवाच}
{ययाऽऽपगेय नामानि श्रुतानीह जगत्पतेः}
{पितामहेशाय विभो नामान्याचक्ष्य शम्भवे}


\threelineshloka
{बभ्रवे विश्वरूपाय महाभाग्यं च तत्त्वतः}
{सुरासुरगुरौ देवे शंकरेऽव्यक्तयोनये ॥भीष्म उवाच}
{}


\twolineshloka
{अशक्तोऽहं गुणान्यक्तुं महादेवस्य धीमतः}
{यो हि सर्वगतो देवो न च सर्वत्र दृश्यते}


\twolineshloka
{ब्रह्मविष्णुसुरेशानां स्रष्टा च प्रभुरेव च}
{ब्रह्मादयः पिशाचान्ता यं हि देवा उपासते}


\threelineshloka
{प्रकृतीनां परत्वेन पुरुषस्य च यः परः}
{चिन्त्यते यो योगविद्भिर्ऋषिभिस्तत्त्वदर्शिभिः}
{अक्षरं परमं ब्रह्म असच्च सदसच्च यः}


\twolineshloka
{प्रकृतिं पुरुषं चैव क्षोभयित्वा स्वतेजसा}
{ब्रह्माणमसृजत्तस्माद्देवदेवः प्रजापतिः}


\twolineshloka
{को हि शक्तो गुणान्वक्तं देवदेवस्य धीमतः}
{गर्भजन्मजरायुक्तो मर्त्यो मृत्युसमन्वितः}


\twolineshloka
{को हि शक्तो भवं ज्ञातुं मद्विधः परमेश्वरम्}
{क्रते नारायणात्पुत्र शङ्कचक्रगदाधरात्}


% Check verse!
एष विद्वान्गुणश्रेष्ठो विष्णुः परमदुर्जयः ॥दिव्यचक्षुर्महातेजा वीक्ष्यते योगचक्षुषा
\twolineshloka
{रुद्रभक्त्या तु कृष्णेन जगद्व्याप्तं महात्मना}
{तं प्रसाद्य तदा देवं बदर्यां किल भारत}


\twolineshloka
{अर्थात्प्रियतरत्वं च सर्वलोकेषु वै तदा}
{प्राप्तवानेव राजेन्द्र सुवर्णाक्षान्महेश्वरात्}


\twolineshloka
{पूर्णं वर्षसहस्रं तु तप्तवानेष माधवः}
{प्रसाद्य वरदं देवं चराचरगुरुं शिवम्}


\twolineshloka
{युगेयुगे तु कृष्णेन तोषितो वै महेश्वरः}
{भक्त्या परमया चैव प्रीतश्चैव महात्मनः}


% Check verse!
ऐश्वर्यं यादृसं तस्य जगद्योनेर्महात्मनः ॥तदयं दृष्टवान्साक्षात्पुत्रार्थे हरिरच्युतः
\twolineshloka
{यस्मात्परतरं चैव नान्यं पश्यामि भारत}
{व्याख्यातुं देवदेवस्य शक्तो नामान्यशषतः}


\threelineshloka
{एष शक्तो महाबाहुर्वक्तुं भगवतो गुणान्}
{विभूतिं चैव कार्त्स्न्येन सत्यां माहेश्वरीं नृप ॥वैशम्पायन उवाच}
{}


\twolineshloka
{एवमुक्त्वा तदा भीष्मो वासुदेवं महायशाः}
{भवमाहात्म्यसंयुक्तमिदमाह पितामहः}


\twolineshloka
{सुरासुरगुरो देव विष्णो त्वं वक्तुमर्हसि}
{शिवाय शिवरूपाय यन्माऽपृच्छद्युधिष्ठिरः}


\twolineshloka
{नाम्नां सहस्रं देवस्य तण्डिना ब्रह्मवादिना}
{निवेदितं ब्रह्मलोके ब्रह्मणो यत्पुराऽभवत्}


\twolineshloka
{द्वैपायनप्रभृतयस्तथा चेमे तपोधनाः}
{ऋषयः सुव्रता दान्ताः शृण्वन्तु गदतस्तव}


\threelineshloka
{ध्रुवाय नन्दिने होत्रे गोप्त्रे विश्वसृजेऽग्नये}
{महाभाग्यं विभोर्ब्रूहि मुण्डिनेऽथ कपर्दिने ॥वासुदेव उवाच}
{}


\twolineshloka
{न गतिः कर्मणां शक्या वेत्तुमीशस्य तत्त्वतः}
{हिरण्यगर्भप्रमुखा देवाः सेन्द्रा महर्षयः}


\twolineshloka
{न विदुर्यस्य निधनमादिं वा सूक्ष्मदर्शिनः}
{स कथं नाममात्रेण शक्यो ज्ञातुं सतां गतिः}


\threelineshloka
{तस्याहमसुरघ्नस्य कांश्चिद्भगवतो गुणान्}
{भवतां कीर्तयिष्यामि व्रतेशाय यथातथम् ॥वैशम्पायन उवाच}
{}


\threelineshloka
{एवमुक्त्वा तु भगवान्गुणांस्तस्य महात्मनः}
{उपस्पृश्य शुचिर्भूत्वा कथयामास धीमतः ॥वासुदेव उवाच}
{}


\twolineshloka
{शुश्रूषध्वं ब्राह्मणेन्द्रास्त्वं च तात युधिष्ठिर}
{त्वं चापगेय नामानि निशामय जगत्पतेः}


\twolineshloka
{यदवाप्तं च मे पूर्वं साम्बहेतोः सुदुष्करम्}
{यथावद्भगवान्दृष्टो मया पूर्वं समाधिना}


\twolineshloka
{शम्बरे निहते पूर्वं रौक्मिणेयेन धीमता}
{अतीते द्वादशे वर्षे जाम्बवत्यब्रवीद्धि माम्}


\twolineshloka
{प्रद्युम्नचारुदेष्णादीन्रुक्मिण्या वीक्ष्य पुत्रकान्}
{पुत्रार्थिनी मामुपेत्य वाक्यमाह युधिष्ठिर}


\twolineshloka
{शूरं बलवतां श्रेष्ठं कान्तरूपमकल्मषम्}
{आत्मतुल्यं मम सुतं प्रयच्छाच्युत माचिरम्}


\twolineshloka
{न हि तेऽप्राप्यमस्तीह त्रिषु लोकेषु किञ्चन}
{लोकान्सृजेस्त्वमपरानिच्छन्यदुकुलोद्वह}


\twolineshloka
{त्वया द्वादशवर्षाणि व्रतीभूतेन शुष्यता}
{आराध्य पशुभर्तारं रुक्मिण्यां जनिताः सुताः}


\twolineshloka
{चारुदेष्णः सुचारुश्च चारुवेशो यशोधरः}
{चारुश्रवाश्चारुयशाः प्रद्युम्नः सम्भुरेव च}


\twolineshloka
{यथा ते जनिताः पुत्रा रुक्मिण्यां चारुविक्रमाः}
{तथा ममापि तनयं प्रयच्छ मधुसूदन}


\twolineshloka
{इत्येवं चोदितो देव्या तामवोचं सुमध्यमाम्}
{अनुजानीहि मां राज्ञि करिष्ये वचनं तव}


\threelineshloka
{सा च मामब्रवीद्गच्छ शिवाय विजयाय च}
{ब्रह्मा शिवः काश्यपश्च नद्यो देवा मनोऽनुगाः}
{}


\twolineshloka
{क्षेत्रौषध्यो यज्ञवाहाश्छन्दास्यृषिगणाध्वराः}
{समुद्रा दक्षिणा स्तोभा ऋक्षाणि पितरो ग्रहाः}


\twolineshloka
{देवपत्न्यो देवकन्या देवमातर एव च}
{मन्वन्तराणि गावश्च चन्द्रमाः सविता हरिः}


\twolineshloka
{सावित्री ब्रह्मविद्या च ऋतवो वत्सरास्तथा}
{क्षणा लवा मुहूर्ताश्च निमेषा युगपर्ययाः}


\twolineshloka
{रक्षन्तु सर्वत्र गतं त्वां यादव सुखाय च}
{अरिष्टं गच्छ पन्थानमप्रमत्तो भवानघ}


\twolineshloka
{एवं कृतस्वस्त्ययनस्तयाऽहंततोऽभ्यनुज्ञाय नरेन्द्रपुत्रीम्}
{पितुः समीपं नरसत्तमस्यमातुश्च राज्ञश्च तथाऽऽहुकस्य}


\threelineshloka
{गत्वा समावेद्य यदब्रवीन्मांविद्याधरेन्द्रस्य सुता भृशर्ता}
{तानभ्यनुज्ञाय तदाऽतिदुःखा-द्गदं तथैवातिबलं च रामम्}
{अथोचतुः प्रीतियुतौ तदानींततःसमृद्धिर्भवतोऽस्त्वविघ्नम्}


\twolineshloka
{प्राप्यानुज्ञां गुरुजनादहं तार्क्ष्यमचिन्तयम्}
{सोवहद्धिमवन्तं मां प्राप्य चैनं व्यसर्जयम्}


\twolineshloka
{तत्राहमद्बुतान्भावानपश्यं गिरिसत्तमे}
{क्षेत्रं च तपसां श्रेष्ठं पश्याम्यद्भुतमुत्तमम्}


\twolineshloka
{दिव्यं वैयाघ्रपद्यस्य उपमन्योर्महात्मनः}
{पूजितं देवगन्धर्वैर्ब्राह्मया लक्ष्म्या समावृतम्}


\twolineshloka
{धवककुभकदम्बनारिकेलैःकुरवककेतकजम्बुपाटलाभिः}
{वटवरुणकवत्सनाभविल्वैःसरलकपित्थप्रियालसालतालैः}


\twolineshloka
{बदरीकुन्दपुन्नागरैशोकाम्रातिमुक्तकैः}
{मधूकैः कोविदारैश्च चम्पकैः पनसैस्तथा}


\twolineshloka
{वन्यैर्बहुविधैर्वृक्षैः फलपुष्पप्रदैर्युतम्}
{पुष्पगुल्मलताकीर्णं कदलीषण्डशोभितम्}


\twolineshloka
{नानाशकुनिसम्भोज्यैः फलैर्वृक्षैरलङ्कृतम्}
{यथास्थानविनिक्षिप्तैर्भूषितं भस्मराशिभिः}


\threelineshloka
{रुरुवानरशार्दूलसिंहद्वीपिसमाकुलम्}
{कुरङ्गबर्हिणाकीर्णं मार्जारभुजगावृतम्}
{पूगैश्च मृगजातीनां महिषर्क्षनिषेवितम्}


\twolineshloka
{सकृत्प्रभिन्नैश्च गजैर्विभूषितंप्रहृष्टनानाविधपक्षिसेवितम्}
{सुपुष्पितैरम्बुधरप्रकाशै-र्महीरुहाणां च वनैर्विचित्रैः}


\twolineshloka
{नानापुष्परजोमिश्रो गजदानाधइवासितः}
{दिव्यस्त्रीगीतबहुलो मारुतोऽभिमुखो ववौ}


\twolineshloka
{धारानिनादैर्विहगप्राणादैःशुभैस्तथा बृंहितैः कुञ्जराणाम्}
{गीतैस्तथा किन्नराणामुदारैःशुभैः स्वनैः सामगानां च वीर}


\twolineshloka
{अचिन्त्यं मनसाऽप्यन्यैः सरोभिः समलङ्कृतम्}
{विशालैश्चाग्निशरणैर्भूषितं कुसुमावृतैः}


\twolineshloka
{विभूषितं पुण्ययवित्रतोययासदा च जुष्टं नृप जह्नुकन्यया}
{विभूषितं धर्मभृतां वरिष्ठै-र्महात्मभिर्वह्निसमानकल्पैः}


\twolineshloka
{वाय्वाहारैरम्बुपैर्जप्यनित्यैःसम्प्रक्षालैर्योगिभिर्ध्याननित्यैः}
{धूमप्राशैरूष्मपैः क्षीरपैश्चसंजुष्टं च ब्राह्मणेन्द्रैः समन्तात्}


\twolineshloka
{गोचारिणोऽर्थाश्मकुट्टा दन्तोलूखलिकास्तथा}
{मरीचिपाः फेनपाश्च तथैव मृगचारिणः}


\twolineshloka
{अश्वत्थफलभक्षाश्च तथा ह्युदकशायिनः}
{चीचचर्माम्बरधरास्तथा वल्कलधारिणः}


\twolineshloka
{सुदुःखान्नियमांस्तांस्तान्वहतः सुतपोधनान्}
{पश्यन्मुनीन्बहुविधानप्रवेष्टुमुपचक्रमे}


\twolineshloka
{सूपूजितं देवगणैर्महात्मभिःशिवादिभिर्भारतपुण्यकर्मभिः}
{रराज तच्चाश्रममण्डलं सदादिवीव राजञ्शशिमण्डलं यथा}


\twolineshloka
{क्रीडन्ति सर्पैर्नकुला मृगैर्व्याघ्राश्च मित्रवत्}
{प्रभावाद्दीप्ततपसां सन्निकर्षान्महात्मनाम्}


\twolineshloka
{तत्राश्रमपदे श्रेष्ठे सर्वभूतमनोरम}
{सेविते द्विजशार्दूलैर्वेदवेदाङ्गपारगैः}


\twolineshloka
{नानानियमविख्यातैर्ऋषिभि सुमहान्मभिः}
{प्रविशन्नेव चापश्यं जटाचीरधरं प्रभुम्}


\twolineshloka
{तेजसा तपसा चैव दीप्यमानं यथाऽनलम्}
{शिष्यैरनुगतं शान्तं युवानं ब्राह्मणर्वभम्}


% Check verse!
शिरसा वन्दमानं मामुपमन्युरभाषत
\twolineshloka
{स्वागतं पुण्डरीकाक्ष सफलानि तपांसि नः}
{यः पूज्यः पूजयसि मां द्रष्टव्यो द्रष्टुमिच्छसि}


\threelineshloka
{`मनुष्यतानुवृत्त्या त्वा ज्ञात्वा तिष्ठाम सर्वगम्}
{'तमहं प्राञ्जलिर्भूत्वा मृगपक्षिष्वथाग्निषु}
{धर्मे च शिष्यवर्गे च समपृच्छमनामयम्}


\twolineshloka
{ततो मां भगवानाह साम्ना परमवल्गुना}
{लप्स्यसे तनयं कृष्णि आत्मतुल्यमसंशयम्}


\twolineshloka
{तपः सुमहदास्थाय तोषयेशानमीश्वरम्}
{इह देवः सपत्नीकः समाक्रीडत्यधोक्षज}


\twolineshloka
{इहैनं दैवतश्रेष्ठं देवाः सर्षिगणाः पुरा}
{तपसा ब्रह्मचर्येण सत्येन च दमेन च}


\twolineshloka
{तोषयित्वा शुभान्कामान्प्राप्तवन्तो जनार्दन}
{तेजसां सपसां चैव निधिः स भगवानिह}


\twolineshloka
{शुभाशुभान्वितान्भावान्विसृजन्स क्षिपन्नपि}
{आस्ते देव्या सहाचिन्त्यो यं प्रार्थयसि शत्रुहन्}


\twolineshloka
{हिरण्यकशिपुर्योऽभूद्दानवो मेरुकम्पनः}
{तेन सर्वामरैश्वर्यं शर्वात्प्राप्तं समार्बुदम्}


\twolineshloka
{तस्यैव पुत्रप्रवरो दमनो नाम विश्रुतः}
{महादेववराच्छक्रं वर्षार्बुदमयोधयम्}


\twolineshloka
{विष्णोश्चक्रं च तद्धोरं वज्रमाखण्डलस्य च}
{शीर्णं पुराऽभवत्तात ग्रहस्याङ्गेषु केशव}


\twolineshloka
{[यत्तद्भगवता पूर्वं दत्तं चक्रं तवानघ}
{जलान्तरचरं हत्वा दैत्यं च बलगर्वितम्}


\twolineshloka
{उत्पादितं वृषाङ्केन दीप्तज्वलनसन्निभम्}
{दत्तं भगवता तुभ्यं दुर्धषं तेजसाऽद्भुतम्}


\twolineshloka
{न शक्यं द्रष्टुमन्येन वर्जयित्वा पिनाकिनम्}
{सुदर्शनं भवत्येवं भवेनोक्तं तदा तु तत्}


\threelineshloka
{सुदर्शनं तदा तस्य लोके नाम प्रतिष्ठितम्}
{तज्जीर्णमभावत्तात ग्रहस्याङ्गेषु केशव}
{}


\twolineshloka
{ग्रहस्यातिवलस्याङ्गे वरदत्तस्य धीमतः}
{न शस्त्राणि वहन्त्यङ्गे चक्रवज्रशतान्यपि ॥]}


\twolineshloka
{अर्द्यमानाश्च विबुधा ग्रहेणि सुबलीयसा}
{शिवदत्तवराञ्जघ्नुरसुरेन्द्रान्सुरा भृशम्}


\twolineshloka
{तृष्टो विद्युत्प्रभस्यापि त्रिलोकेश्वरतां ददौ}
{शतं वर्षसहस्राणां सर्वलोकेश्वरोऽभवत्}


\twolineshloka
{ममैवानुचरो नित्यं भवितासीति चाब्रवीत्}
{तथा पुत्रसहस्राणामयुतं च ददौ प्रभुः}


\threelineshloka
{कुशद्वीपं च स ददौ राज्येन भगवानजः}
{[तथा शतमुखो नाम धात्रा सृष्टो महासुरः}
{}


\twolineshloka
{येन वर्षशतं साग्रमात्ममांसैर्हुतोऽनलः}
{तं प्राह भववांस्तुष्टः किंकरोमीति शंकरः}


\twolineshloka
{तं वै शतमुखः प्राह योगो भवतु मेऽद्भुतः}
{बलं च दैवतश्रेष्ठ शाश्वतं सम्प्रयच्छ मे}


\twolineshloka
{तथेति भगवानाह तस्य तद्वचनं प्रभुः}
{स्वायंभुवः क्रतुश्चापि पुत्रार्थमभवत्पुरा}


\threelineshloka
{आविश्य योगेनात्मानं त्रीणि वर्षशतान्यपि}
{तस्य चोपददौ पुत्रान्सहस्रं क्रतुसम्मितान्}
{]योगेश्वरं देवगीतं वेत्थ कृष्ण न संशयः}


\twolineshloka
{याज्ञवल्क्य इति ख्यात ऋषिः परमधार्मिकः}
{आराध्य स महादेवं प्राप्तवानतुलं यशः}


\twolineshloka
{वेदव्यासश्च योगात्मा पराशरसुतो मुनिः}
{सोऽपि शंकरमाराध्य प्राप्तवानतुलं यशः}


\twolineshloka
{वालखिल्या मघवता ह्यवज्ञाताः पुरा किल}
{तैः क्रुद्धैर्भगवान्रुद्रस्तपसा तोषितो ह्यभूत्}


\twolineshloka
{तांश्चापि दैवतश्रेष्ठः प्राह प्रीतो जगत्पतिः}
{सुपर्णं सोमहर्तारं तपसोत्पादयिष्यथ}


\twolineshloka
{महादेवस्य रोषाच्च आपो नष्टाः पुराऽभवन्}
{ताश्च सप्तकपालेन देवैरन्याः प्रवर्तिताः}


\twolineshloka
{ततः पानीयमभवत्प्रसन्ने त्र्यम्बके भुवि}
{अत्रिभार्या सुतं दत्तं सोमं दुर्वाससं प्रभो}


\twolineshloka
{अत्रेर्भार्याऽपि भर्तारं संत्यज्य ब्रह्मवादिनी}
{नाहं तव मुने भूयो वशगा स्यां कथञ्चन}


\twolineshloka
{इत्युक्त्वा सा महादेवमगमच्छरणं किल}
{निराहास भयादत्रेस्त्रीणि वर्षशतान्यपि}


\twolineshloka
{अशेत मुसलेष्वेव प्रसादार्थं भवस्य सा}
{तामब्रवीद्धसन्देवो भविता वै सुतस्तव}


\twolineshloka
{विना भर्त्रा चरुद्रेण भविष्यति न संशयः}
{वंशे तवैव नाम्ना तु ख्यातिं यास्यति चेप्सिताम्}


\twolineshloka
{विकर्णश्च महादेवं तथा भक्तसुखावहम्}
{प्रसाद्य भगवान्सिद्धिं प्राप्तवान्मधुसूदन}


\twolineshloka
{शाकल्यः संशितात्मा वै नववर्षशतान्यपि}
{आराधयामास भवं मनोयज्ञेन केशव}


\twolineshloka
{तं चाह भगवांस्तुष्टो ग्रन्थकारो भविष्यसि}
{वत्साक्षया च ते कीतिस्त्रेलोक्ये वै भविष्यति}


\twolineshloka
{अक्षयं च कुलं तेऽस्तु महर्षिभिरलंकृतम्}
{भविष्यति द्विजश्रेष्ठः सूत्रकर्ता सुतस्तव}


\twolineshloka
{सावर्णिश्चापि विख्यात ऋषिरासीत्कृते युगे}
{इह तेन तपस्तप्तं षष्टिवर्षशतान्यथ}


\twolineshloka
{तमाह भगवान्रुद्रः साक्षात्तुष्टोस्मि तेऽनघ}
{ग्रन्थकृल्लोकविख्यातो भवितास्यजरामरः}


\twolineshloka
{शक्रेणि तु पुरा देवो वाराणस्यां जनार्दन}
{आराधितोऽभूद्भक्तेन दिग्वासा भस्मगुष्ठितः ॥आराध्य स महादेवं देवराज्यमवाप्तवान्}


\twolineshloka
{नारदेन तु भक्त्याऽसौ भव आराधितः पुरा}
{तस्य तुष्टो महादेवो जगौ देवगुरुर्गुरुः}


\twolineshloka
{तेजसा तपसा कीर्त्या त्वत्समो न भविष्यति}
{गीतेन वादितव्येन नित्यं मामनुयास्यसि}


\threelineshloka
{`बाणः स्कन्दसमत्वं च कामो दर्पविमोक्षणम्}
{लवणोऽवध्यतामन्यैर्दशास्यश्च पुनर्बलम्}
{अन्तकोऽन्तमनुप्राप्तस्तस्मात्कोऽन्यः परः प्रभुः}


\twolineshloka
{मयाऽपि च यथा दृष्टो देवदेवः पुरा विभो}
{साक्षात्पशुपतिस्तात तच्चापि शृणु माधव}


\twolineshloka
{यदर्थं च मया देवः प्रयतेन तथा विभो}
{आराधितो महातेजास्तच्चापि शृणु विस्तरात्}


\twolineshloka
{यदवाप्तं च मे पूर्वं देवदेवान्महेश्वरात्}
{तत्सर्वं निखिलेनाद्य कथयिष्यामि तेऽनघ}


\threelineshloka
{पुरा कृतयुगे तात ऋषिरासीन्महायशाः}
{व्याघ्रपाद इति ख्यातो वेदवेदाङ्गपारगः}
{तस्याहमभवं पुत्रो धौम्यश्चापि ममानुजः}


\twolineshloka
{कस्यचित्त्थ कालस्य धौम्येन सह माधव}
{आगच्छमाश्रमं क्रीडन्मुनीनां भावितात्मनाम्}


\twolineshloka
{तत्रापि च मया दृष्टा दुह्यमाना पयस्विनी}
{लक्षितं च मया क्षीरं स्वादुतो ह्यमृतोपमम्}


\twolineshloka
{तदाप्रभृति चैवाहमरुदं मधुसूदन}
{दीयतां दीयतां क्षीरं मम मातरितीरिता}


% Check verse!
अभावाच्चैव दुग्धस्य दुःखिता जननी तदा
\twolineshloka
{ततः पिष्टं समालोड्य तोयेन सह माधव}
{आवयोः क्षीरमित्येव पानार्थं समुपानयत्}


\twolineshloka
{अथ गव्यं पयस्तात कदाचित्प्राशितं मया}
{}


\twolineshloka
{पित्राऽहं यज्ञकाले हि नीतो ज्ञातिकुलं महत्}
{तत्र सा क्षरते देवी दिव्या गौः सुरनन्दिनी}


\threelineshloka
{यस्ताहं तत्पयः पीत्वा रसेन ह्यमृतोपमम्}
{ज्ञात्वा क्षीरगुणांश्चैव उपलभ्य हि सम्भवम्}
{स च पिष्टरसस्तात न मे प्रीतिमुपावहत्}


\twolineshloka
{ततोऽहमब्रुवं बाल्याज्जननीमात्मनस्तदा}
{नेदं क्षीरोदनं मातर्यत्त्वं मे दत्तवत्यसि}


\twolineshloka
{ततो मामब्रवीन्माता दुःखशोकसमन्विता}
{पुत्रस्नेहात्परिष्वज्य मूर्ध्नि चाघ्राय माधव}


\twolineshloka
{कुतः क्षीरोदनं वत्स मुनीनां भावितात्मनम्}
{वने निवसतां नित्यं कन्दमूलफलाशिनाम्}


% Check verse!
आस्थितानां नदीं दिव्यां वालखिल्यैर्निषेविताम्कुत क्षीरं वनस्थानां मुनीनां गिरिवासिनाम्
\threelineshloka
{पावनानां वनाशानां वनाश्रमनिवासिनाम्}
{ग्राम्याहारनिवृत्तानामारण्यफलभोजिनाम्}
{नास्ति पुत्र पयोऽरण्ये सुरभीगोत्रवर्जिते}


\twolineshloka
{नदीगह्वरशैलेषु तीर्थेषु विविधेषु च}
{तपसा जप्यनित्यानां शिवो नः परमा गतिः}


\twolineshloka
{अप्रसाद्य विरूपाक्षं वरदं स्थाणुमव्ययम्}
{कुतः क्षीरोदनं वत्स सुखानि वसनानि च}


\twolineshloka
{तं प्रपद्य सदा वत्स सर्वभावेन शङ्करम्}
{तत्प्रसादाच्च कामेभ्यः फलं प्राप्स्यसि पुत्रक}


\twolineshloka
{जनन्यास्तद्वचः श्रुत्वा तदाप्रभृति शत्रुहन्}
{[प्राञ्जलिः प्रणतो भूत्वा इदमम्बामवोचयं}


\twolineshloka
{कोऽयमम्ब महादेवः स कथं च प्रसीदति}
{कुत्र वा वसते देवो द्रष्टव्यो वा कथञ्चन}


\twolineshloka
{तुष्यते वा कथं शर्वो रूपं तस्य च कीदृशम्}
{कथं ज्ञेयः प्रसन्नो वा दर्शयेज्जननी मम}


\twolineshloka
{एवमुक्ता तदा कृष्ण माता मे सुतवत्सला}
{मूर्घन्याध्राय गोविन्द सबाष्पाकुललोचना}


% Check verse!
प्रमार्जन्ती च गात्राणि मम वै मधुसूदन ॥दैन्यमालम्ब्य जननी इदमाह सुरोत्तम
\twolineshloka
{दुर्विज्ञेयो महादेवो दुराधारो दुरन्तकः}
{दुराबाधश्च दुर्ग्राह्यो दुर्द्दश्यो ह्यकृतात्मभिः}


\twolineshloka
{यस्य रूपाण्यनेकानि प्रवदन्ति मनीषिणः}
{स्थानानि च विचित्राणि प्रासादाश्चाप्यनेकशः}


\threelineshloka
{को हि तत्त्वेन तद्वेद ईशस्य चरितं शुभम्}
{कृतवान्यानि रूपाणि देवदेवः पुरा किल}
{क्रीडते च तथा शर्वः प्रसीदति यथाच वै}


\threelineshloka
{हृदिस्थः सर्वभूतानां विश्वरूपो महेश्वरः}
{भक्तानामनुकम्पार्थं दर्शनं च यथाश्रुतम्}
{मुनीनां ब्रुवतां दिव्यमीशानचरितं शुभम्}


\twolineshloka
{कृतवान्यानि रूपाणि कथितानि दिवौकसैः}
{अनुग्रहार्थं विप्राणां शृणु वत्स समासतः}


% Check verse!
तानि ते कीर्तयिष्यामि यन्मां त्वं परिपृच्छसि
\twolineshloka
{ब्रह्मविष्णुसुरेन्द्राणां रुद्रादित्याश्विनामपि}
{विश्वेषामपि देवानां वपुर्धारयते भवः}


\twolineshloka
{नराणां देवनारीणां तथा प्रेतपिशाचयोः}
{किरातशबराणां च जलजानामनेकशः}


\twolineshloka
{करोति भगवान्रूपमाटव्यशबराण्यपि}
{कूर्मो मत्स्यस्तथा शङ्खः प्रवालाङ्कुरभूषणः}


\twolineshloka
{यक्षराक्षससर्पाणां दैत्यदानवयोरपि}
{वपुर्धारयते देवो भूयश्च बिलवासिनाम्}


\twolineshloka
{व्याघ्रसिंहमृगाणां च तरक्ष्वृक्षपतत्त्रिणाम्}
{उलूकश्वशृगालानां रूपाणि कुरुतेऽपि च}


\twolineshloka
{हंसकाकमयूराणां कृकलासकसारसाम्}
{रूपाणि च बलाकानां गृध्रचक्राङ्गयोरपि}


\twolineshloka
{करोति वा स रूपाणि धारयत्यपि पर्वतम्}
{गोरूपं च महादेवो हस्त्यश्वोष्ट्रखराकृतिः}


\twolineshloka
{छागशार्दूलरूपश्च अनेकमृगरूपधृक्}
{अण्डजानां च दिव्यानां वपुर्धारयते भवः}


\twolineshloka
{दण्डी छत्री च कुण्डी च द्विजानां वारणस्तथा}
{षण्मुखो वै बहुमुखस्त्रिनेत्रो बहुशीर्षकः}


\twolineshloka
{अनेककटिपादश्च अनेकोदरवक्त्रधृत्}
{अनेकपाणिपार्श्वश्च अनेकगपसंवृतः}


\twolineshloka
{ऋषिगन्धर्वरूपश्च सिद्धचारणरूपधृत्}
{भस्पपाण्डुरगात्रश्च चन्द्रार्धकृतभूषणः}


\twolineshloka
{अनेकरावसंघुष्टश्चानेकस्तुतिसंस्कृतः}
{सर्वभूतान्तकः सर्वः सर्वलोकप्रतिष्ठितः}


\twolineshloka
{सर्वलोकान्तरात्मा च सर्वगः सर्ववाद्यपि}
{सर्वत्र भगवान्ज्ञेयो हृदिस्थः सर्वदेहिनाम्}


\twolineshloka
{यो हि यं कामयेत्कामं यस्मिन्नर्थऽर्च्यते पुनः}
{तत्सर्वं वेत्ति देवेशस्तं प्रपद्य यदीच्छसि}


\twolineshloka
{नन्दते कुप्यते चापि तथा हुंकारयत्यपि}
{चक्री शूली गदापाणिर्मुसली खड्गपट्टसी}


\twolineshloka
{भूधरो नागमौञ्जी च नागकुण्डलकुण्डली}
{नागयज्ञोपवीती य नागचर्मोत्तरच्छदः}


\twolineshloka
{हसते गायते चैव नृत्यते च मनोहरम्}
{वादयत्यपि वाद्यानि विचित्राणि गणैर्युतः}


\twolineshloka
{वल्गते जृम्बते चैव रुदते रोदयत्यपि}
{उन्मत्तमत्तरूपं च भाषते चापि सुस्वरः}


\twolineshloka
{अतीव हसते रौद्रस्त्रासयन्नयनैर्जनम्}
{जागर्ति चैव स्वपिति जृम्भते च यथासुवम्}


\twolineshloka
{जपते जप्यते चैव तपते तप्यते पुनः}
{ददाति प्रतिगृह्णाति युञ्जते ध्यायतेऽपि च}


\twolineshloka
{वेदीमध्ये तथा यूपे गोष्ठमध्ये हुताशने}
{दृश्यते दृश्यते चापि बालो वृद्धो युवा तथा}


\twolineshloka
{क्रीडते ऋषिकन्याभिर्ऋषिपत्नीभिरेव च}
{ऊर्ध्वकेशो महाशेफो नग्नो विकृतलोचनः}


\twolineshloka
{गौरः श्यामस्तथा कृष्णः पाण्डुरो धूमलोहितः}
{विकृताक्षो विशालाक्षो दिग्वासाः सर्ववासकः}


\twolineshloka
{अरूपस्याद्यरूपस्य अतिरूपाद्यरूपिणः}
{अनाद्यन्तमजस्यान्तं वेत्स्यते कोस्य तत्त्वतः}


\twolineshloka
{हृदि प्राणो मनो जीवो योगात्मा योगसंज्ञितः}
{ध्यानं तत्परमात्मा च भावग्राह्यो महेश्वरः}


\twolineshloka
{वादको गायनश्चैव सहस्रशतलोचनः}
{एकवक्त्रो द्विवक्त्रश्च त्रिवक्त्रोऽनेकवक्त्रकः}


\twolineshloka
{तद्भक्तस्तद्गतो नित्यं तन्निष्ठस्तत्परायणः}
{भज पुत्र महादेवं ततः प्राप्स्यसि चेप्सितं}


\twolineshloka
{जनन्यास्तद्वचः श्रुत्वा तदाप्रभृति शत्रुहन्}
{]मम भक्तिर्महादेवे नैष्ठिकी समपद्यत}


\twolineshloka
{ततोऽहं तप आस्थाय तोपयामास शंकरम्}
{दिव्यं वर्षसहस्रं तु वामाङ्गुष्ठाग्रविष्ठितः}


\twolineshloka
{एकं वर्षशतं चैव फलाहारस्ततोऽभवम्}
{द्वितीयं शीर्णपर्णाशी तृतीयं चाम्बुभोजनः}


\twolineshloka
{शतानि सप्त चैवाहं वायुभक्षस्तदाऽभवम्}
{एकं वर्षसहस्रं तु दिव्यमाराधितो मया}


\twolineshloka
{ततस्तुष्टो महादेवः सर्वलोकेश्वरः प्रभुः}
{एकभक्त इति ज्ञात्वा जिज्ञासां कुरुते तदा}


\twolineshloka
{शक्ररूपं स कृत्वा तु सर्वैर्देवगणैर्वृतः}
{सहस्राक्षस्तदा भूत्वा वज्रपाणिर्महायशाः}


\twolineshloka
{सुधावदातं रक्ताक्षं स्तब्धकर्णं मदोत्कटम्}
{आवेष्टतकरं घोरं चतुर्दष्ट्रं महागजम्}


\twolineshloka
{समास्थितः स भगवान्दीप्यमानः स्वतेजसा}
{आजगाम किरीटी तु हारकेयूरभूषितः}


\twolineshloka
{पाण्डुरेणातपत्रेण ध्रियमाणेन मूर्धनि}
{सेव्यभानोप्सरोभिश्च दिव्यगन्धर्वनादितैः}


\twolineshloka
{ततो मामाह देवेन्द्रस्तुष्टस्तेऽहं द्विजोत्तम}
{वरं वृणीष्व मत्तस्त्वं यत्ते मनसि वर्तते}


\twolineshloka
{शक्रस्य तु वचः श्रुत्वा नाहं प्रीतमनाऽभवम्}
{अब्रवं च तदा कृष्ण देवराजमिदं वचः}


\twolineshloka
{नाहं त्वत्तो वरं काङ्क्षे नान्यस्मादपि दैवतात्}
{महादेवादृते सौम्य सत्यमेतद्ब्रवीमि ते}


\twolineshloka
{सत्यंसत्यं हि नः शक्र वाक्यमेतत्सुनिश्चितम्}
{न यन्मिहेश्वरं मुक्त्वा कथाऽन्या मम रोचते}


\twolineshloka
{पशुपतिवचनाद्भवामि सद्यःकृमिरथवा तरुरप्यनेकशाखः}
{अपशुपतिवरप्रसादजा मेत्रिभुवनराज्यविभूतिरप्यनिष्टा}


\twolineshloka
{[जन्मश्वपाकमध्येऽ-पि मेऽस्तु हरचरणवन्दनरतस्य}
{मा वाऽनीश्वरभक्तोभवानि भवनेऽपि शक्रस्य}


\threelineshloka
{वाय्वम्बुभुजोऽपि सतो}
{नरस्य दुःखक्षयः कुतस्तस्य}
{भवति हि सुरासुरगुरौयस्य न विश्वेश्वरे भक्तिः}


\twolineshloka
{अलमन्याभिस्तेषांकथाभिरप्यन्यधर्मयुक्ताभिः}
{येषां न क्षणमपि रुचितोहरचरणस्मरणविच्छेदः}


\twolineshloka
{हरचरणनिरतमतिनाभवितव्यमनार्जवं युगं प्राप्य}
{संसारभयं न भवतिहरभक्तिरसायनं पीत्वा}


\twolineshloka
{दिवसं दिवसार्धं वा मुहूर्तं वा क्षणं लवम्}
{न ह्यलब्धप्रसादस्य भक्तिर्भवति शङ्करे ॥]}


\twolineshloka
{अपि कीटः पतङ्गो वा भवेयं शङ्कराज्ञया}
{न तु शक्र त्वया दत्तं त्रैलोक्यमपि कामये}


\twolineshloka
{[श्वाऽपि महेश्वरवचना-द्भवामि स हि नः परः कामः}
{त्रिदशगणराज्यमपि खलुनेच्छाम्यमहेश्वराज्ञप्तम्}


\twolineshloka
{न नाकपृष्ठं न च देवराज्यंन ब्रह्मलोकं न च निष्कलत्वम्}
{न सर्वकामानखिलान्वृणोमिहरस्य दासत्वमहं वृणोमि ॥]}


\twolineshloka
{यावच्छशाङ्कधवलामलबद्धमौलि-र्न प्रीयते पशुपतिर्भगवान्ममेशः}
{तावज्जरामरणजन्मशताभिघातै-र्दुःखानि देहविहितानि समुद्वहामि}


\twolineshloka
{दिवसकरशशाङ्कवह्निदीप्तंत्रिभुवनसारमसारमाद्यमेकम्}
{अजरममरमप्रसाद्य रुद्रंजगति पुमानिह को लभेत शान्तिं}


\twolineshloka
{`धिक्तेषां धिक्तेषांपुनरपि च धिगस्तु धिक्तेषाम्}
{येषां न वसति हृदयेकुपथगतिविमोक्षको रुद्रः'}


\threelineshloka
{यदि नाम जन्म भूयोभवति मदीयैः पुनर्दोषैः}
{तस्मिंस्तस्मिञ्जन्मनिभवे भवेन्मेऽक्षया भक्तिः ॥शक्र उवाच}
{}


\threelineshloka
{कः पुनर्भवने हेतुरीशे कारणकारणे}
{येन शर्वादृतेऽन्यस्मात्प्रसादं नाभिकाङ्क्षसि ॥[उपमन्युरुवाच}
{}


\twolineshloka
{सदसद्व्यक्तमव्यक्तं यमाहुर्ब्रह्मवादिनः}
{नित्यमेकमनेकं च वरं तस्माद्वृणीमहे}


\twolineshloka
{अनादिमध्यपर्यन्तं ज्ञानैश्वर्यमचिन्तितम्}
{आत्मानं परमं यस्माद्वरं तस्माद्वृणीमहे}


\twolineshloka
{ऐश्वर्यं सकलं यस्मादनुत्पादितमव्ययम्}
{अबीजाद्बीजसम्भूतं वरं तस्माद्वृणीमहे}


\twolineshloka
{तमसः परमं ज्योतिस्तपस्तद्वृत्तिनां परम्}
{यं ज्ञात्वा नानुशोचन्ति वरं तस्माद्वृणीमहे}


\twolineshloka
{भूतभावनभावज्ञं सर्वभूताभिभावनम्}
{सर्वगं सर्वदं देवं पूजयामि पुरंदर}


\twolineshloka
{हेतुवादैर्विनिर्मुक्तं साङ्ख्ययोगार्थदं परम्}
{यमुपासन्ति तत्त्वज्ञा वरं तस्माद्वृणीमहे}


\twolineshloka
{मघवन्मघवात्मानं यं वदन्ति सुरेश्वरम्}
{सर्वभूतगुरुं देवं वरं तस्माद्वृणीमहे}


\twolineshloka
{यत्पूर्वमसृजद्देवं ब्रह्माणं लोकभावनम्}
{अण्डमाकाशमापूर्य वरं तस्माद्वृणीमहे}


\twolineshloka
{अग्निरापोऽनिलः पृथ्वी खं बुद्धिश्च मनो महान्}
{स्रष्टा चैषां भवेद्योऽन्यो ब्रूहि कः परमेश्वरात्}


\twolineshloka
{मनो मतिरहङ्कारस्तन्मात्राणीन्द्रियाणि च}
{ब्रूहि चैषां भवेच्छक्र कोऽन्योस्ति परमं शिवात्}


\twolineshloka
{स्रष्टारं भुवनस्येह वदन्तीह पितामहम्}
{आराध्य स तु देवेशमश्नुते महतीं क्षियम्}


\twolineshloka
{भगवत्युत्तमैश्वर्यं ब्रह्मविष्णुपुरोगमम्}
{विद्यते वै महादेवाद्ब्रूहि कः परमेश्वरात्}


\twolineshloka
{दैत्यदानवमुख्यानामाधिपत्यारिमर्दनात्}
{कोऽन्यः शक्रोति देवेशाद्दितेः सम्पादितुं सुतान्}


\twolineshloka
{दिक्कालसूर्यतेजांसि ग्रहवाय्विन्दुतारकाः}
{विद्धि त्वेते महादेवाद्ब्रूहि कः परमेश्वरात्}


\twolineshloka
{अथोत्पत्तिविनाशे वा यज्ञस्य त्रिपुरस्य वा}
{दैत्यदानवमुख्यानामाधिपत्यारिमर्दनः}


\twolineshloka
{किं चात्र बहुभिः सूक्तैर्हेतुवादैः पुरंदर}
{सहस्रनयनं दृष्ट्वा त्वामेव सुरसत्तम}


\twolineshloka
{पूजितं सिद्धगन्धर्वैर्देवैश्च ऋषिभिस्तथा}
{देवदेवप्रसादेन तत्सर्वं कुशिकोत्तम}


\twolineshloka
{अव्यक्तमुक्तकेशाय सर्वगस्येदमात्मकम्}
{चेतनाचेतनाद्येषु शक्र विद्धि महेश्वरात्}


\threelineshloka
{भुवाद्येषु महान्तेषु लोकालोकान्तरेषु च}
{द्वीपस्थानेषु मेरोश्च विभवेष्वन्तरेषु च}
{}


\twolineshloka
{भगवन्मघवन्देवं वदन्ते तत्त्वदर्शिनः ॥यदि देवाः सुराः शक्र पश्यन्त्यन्यां भवाकृतिम्}
{}


\twolineshloka
{किं न गच्छन्ति शरणं मर्दिताश्चासुरैः सुराः ॥अभिघातेषु देवानां सयक्षोरगरक्षसाम्}
{}


\threelineshloka
{परस्परविनाशीषु स्वस्थानैश्वर्यदो भवः ॥अन्धकस्याथ शुक्रस्य दुन्दुभेर्महिषस्य च}
{यक्षेन्द्रबलरक्षःसु निवातकवचेषु च}
{वरदानावघाताय ब्रूहि कोऽन्यो महेश्वरात्}


\twolineshloka
{सुरासुरगुरोर्वक्त्रे कस्य रेतः पुरा हुतम्}
{कस्य वाऽन्यस्य रेतस्तद्येन हैमो गिरिः कृतः}


\twolineshloka
{दिग्वासाः कीर्त्यते कोऽन्यो लोके कश्चोर्ध्वरेतसः}
{कस्य चार्धे स्थिता कान्ता अनङ्गः केन निर्जितः}


\twolineshloka
{ब्रूहीन्द्र परमं स्थानं कस्य देवैः प्रशस्यते}
{श्मशाने कस्य क्रीडार्थं नृत्ते वा कोऽभिभाष्यते}


\twolineshloka
{कस्यैश्वर्यं समानं च भूतैः को वाऽपि क्रीडते}
{कस्य तुल्यबला देवगणाश्चैश्वर्यदर्पिताः}


\twolineshloka
{घुष्यते ह्यचलं स्थानं कस्य त्रैलोक्यपूजितम्}
{वर्षते तपते कोऽन्यो ज्वलते तेजसा च कः}


\twolineshloka
{कस्मादोषधिसम्पत्तिः को वा धारयते वसु}
{प्रकामं क्रीडते को वा त्रैलोक्ये सचराचरे}


\twolineshloka
{ज्ञानसिद्धिक्रियायोगैः सेव्यमानश्च योगिभिः}
{ऋषिगन्धर्वसिद्धैश्च विहितं कारणं परम्}


\twolineshloka
{कर्मयज्ञक्रियायोगैः सेव्यमानः सुरासुरैः}
{नित्यं कर्मफलैर्हीनं तमहं कारणं वदे}


\twolineshloka
{स्थूलं सूक्ष्ममनौपम्यमग्राह्यं गुणगोचरम्}
{गुणहीनं गुणाध्यक्षं परं माहेश्वरं पदम्}


\threelineshloka
{विश्वेशं कारणगुरुं लोककालोकान्तकारणम्}
{भूताभूतभविष्यच्च जनकं सर्वकारणम्}
{}


\twolineshloka
{अक्षराक्षरमव्यक्तं विद्याविद्ये कृताकृते}
{धर्माधर्मौ यतः शक्र तमहं कारणं ब्रुवे}


\twolineshloka
{प्रत्यक्षमिह देवेन्द्र पश्य लिङ्गं भगाङ्कितम्}
{देवदेवेन रुद्रेण सृष्टिसंहारहेतुना}


\twolineshloka
{मात्रा पूर्वं ममाख्यातं कारणं लोकलक्षणम्}
{नास्ति चेशात्परं शक्र तं प्रपद्य यदीच्छसि}


\twolineshloka
{प्रत्यक्षं ननु ते सुरेश विदितं संयोगलिङ्गोद्भवंत्रैलोक्यं सविकारनिर्गुणगणं ब्रह्मादिरेतोद्भवम्}
{यद्ब्रह्मेन्द्रहुताशविष्णुसहिता देवाश्च दैत्येश्वरानान्यत्कामसहस्रकल्पितधियः शंसन्ति ईशात्परं}


\twolineshloka
{तं देवं सचराचरस्य जगतो व्याख्यातवेद्योत्तमंकामार्थी वरयामि संयतमना मोक्षाय सद्यः शिवम्हेतुभिर्वा किमन्यैस्तैरीशः कारणकारणम्}
{न शुश्रुम यदन्यस्य लिङ्गमभ्यर्चितं सुरैः}


\twolineshloka
{कस्यान्यस्य सुरैः सर्वैर्लिङ्गं मुक्त्वा महेश्वरम्}
{अर्च्यतेऽर्चितपूर्वं वा ब्रूहि यद्यस्ति ते श्रुतिः}


\twolineshloka
{यस्य ब्रह्म च विष्णुश्च त्वं चापि सहदैवतैः}
{अर्चयध्वं सदा लिङ्गं तस्माच्छ्रेष्ठतमो हि सः}


\twolineshloka
{न पद्माङ्गा न चक्राङ्का न वज्राङ्का यतः प्रजाः}
{लिङ्गाङ्का च भगाङ्का च तस्मान्माहेश्वरी प्रजा}


\twolineshloka
{देव्याःकरणरूपभावजनिताःसर्वाभगाङ्काः स्त्रियोलिङ्गेनापि हरस्य सर्वपुरुषाः प्रत्यक्षचिह्नीकृताः}
{योऽन्यत्कारणमीश्वरात्प्रवदते देव्या च यन्नाङ्कितंत्रैलोक्ये सचराचरे स तु पुमान्बाह्यो भवेद्दुर्मतिः}


\twolineshloka
{पुल्लिङ्गं सर्वमीशानं स्त्रीलिङ्गं विद्धि चाप्यमाम्}
{द्वाभ्यां तनुभ्यां व्याप्तं हि चराचरमिदं जगत् ॥]}


\twolineshloka
{तस्माद्वरमहं काङ्क्षे निधनं वाऽपि कौशिक}
{गच्छ वा तिष्ठ वा शक्र यथेष्टं बलसूदन}


\twolineshloka
{काममेष वरो मेस्तु शापो वाऽथ महेश्वरात्}
{न चान्यां देवतां काङ्क्षे सर्वकामफलामपि}


\twolineshloka
{एवमुक्त्वा तु देवेन्द्रं दुःखादाकुलितेन्द्रियः}
{न प्रसीदति मे देवः किमेतदिति चिन्तयन्}


\twolineshloka
{अथापश्यं क्षणेनैव तमेवैरावतं पुनः}
{हंसकुन्देन्दुसदृशं मृणालरजतप्रभम्}


\twolineshloka
{वृषरूपधरं साक्षात्क्षीरोदमिव सागरम्}
{कृष्णपुच्छं महाकायं मधुपिङ्गललोचनम्}


\twolineshloka
{वज्रसारमयैः शृङ्गैर्निष्टप्तकनकप्रभैः}
{सुतीक्ष्णैर्मृदुरक्ताग्रैरुत्किरन्तमिवावनिम्}


\threelineshloka
{जाम्बूनदेन दाम्ना च सर्वतः समलङ्कृतम्}
{सुवक्त्रखुरनासं च सुकर्णं सुकटीतटम्}
{सुपार्श्वं विपुलस्कन्धं सुरूपं चारुदर्शनम्}


\twolineshloka
{ककुदं तस्य चाभाति स्कन्धमापूर्य धिष्ठितम्}
{तुषारगिरिकूटाभं सिताभ्रशिखरोपमम्}


\twolineshloka
{तभास्थितश्च भगवान्देवदेवः सहोमया}
{अशोभत महादेवः पौर्णमास्यामिवोडुराट्}


\twolineshloka
{`किरीटं च जटाभारः सर्पाद्याभरणानि च}
{वज्रादिशूलमातङ्गगम्भीरस्मितमागतम् ॥'}


\twolineshloka
{तस्य तेजोभवो वह्निः समेघः स्तनयित्नुमान्}
{सहस्रमिव सूर्याणां सर्वमापूर्य धिष्ठितः}


\twolineshloka
{ईश्वरः सुमहातेजाः संवर्तक इवानलः}
{युगान्ते सर्वभूतानां दिधक्षुरिव चोद्यतः}


\twolineshloka
{तेजसा तु तदा व्याप्तं दुर्निरीक्ष्यं समन्ततः}
{पुनरुद्विग्नहृदय किमेतदिति चिन्तयम्}


\twolineshloka
{मुहूर्तमिव तत्तेजो व्याप्य सर्वा दिशो दश}
{प्रशान्तं च क्षणेनैव देवदेवस्य मायया}


\twolineshloka
{अथापश्यं स्थितं स्थाणुं भगवन्तं महेश्वरम्}
{`सौरभेयगतं सौम्यं विधूममिव पावकम्}


\twolineshloka
{प्रशान्तमनसं देवं त्रिनेत्रमपराजितम्}
{सहितं चारुसर्वाङ्ग्या पार्वत्या परमेश्वरम् ॥'}


\twolineshloka
{नीलकण्ठं महात्मनमसक्तं तेजसां निधिम्}
{अष्टादशभुजं स्थाणुं सर्वाभरणभूषितम्}


\twolineshloka
{शुक्लाम्बरधरं देवं शुक्लमाल्यानुलेपनम्}
{शुक्लध्वजमनाधृष्यं शुक्लयज्ञोपवीतिनम्}


\twolineshloka
{गायद्भिर्नृत्यमानैश्च वादयद्भिश्च सर्वशः}
{वृतं पार्श्वचरैर्दिव्यैरात्मतुल्यपराक्रमैः}


\twolineshloka
{बालेन्दुमुकुटं पाण्डुं शरच्चन्द्रमिवोदितम्}
{त्रिभिर्नेत्रैः कृतोद्योतं त्रिभिः सूर्यैरिवोदितैः}


\twolineshloka
{`सर्वविद्याधिपं देवं शरच्चन्द्रसमप्रभम्}
{नयनाह्लादसौम्योऽहमपश्यं परमेश्वरम् ॥'}


\twolineshloka
{अशोभतास्य देवस्य माला गात्रे सितप्रभे}
{जातरूपमयैः पद्मैर्ग्रथिता रत्नभूषिता}


\twolineshloka
{मूर्तिमन्ति तथाऽस्त्राणि सर्वतेजोमयानि च}
{मया दृष्टानि गोविन्द भवस्यामिततेजसः}


\twolineshloka
{इन्द्रायुधसहस्राभं धनुस्तस्य महात्मनः}
{पिनाकमिति विख्यातं स च वै पन्नगो महान्}


\twolineshloka
{सप्तशीर्षो महाकायस्तीक्ष्णदंष्ट्रो विषोल्वणः}
{ज्यावेष्टितमहाग्रीवः स्थितः पुरुषविग्रहः}


\twolineshloka
{शरश्च सूर्यसङ्काशो दृष्टः पाशुपताह्वयः}
{`सहस्रभुजजिह्वास्यो भीषणो नागविंग्रह}


\twolineshloka
{शङ्खशूलासिभिश्चैव पट्टसै रूपवान्स्थितः}
{येन च त्रिपुरं दग्धं सर्वदेवमयः ******}


\twolineshloka
{अद्वितीयमनिर्देश्यं सर्वभूतभयावहम्}
{सस्फुलिङ्गं महाकायं विसृजन्तमिवानलम्}


\twolineshloka
{एकपादं महादंष्ट्रं सहस्रशिरसोदरम्}
{सहस्रभुजजिह्वाक्षमुद्गिरन्तमिवानलम्}


\twolineshloka
{ब्राह्मान्नारायणाच्चैन्द्रादाग्नेयादपि वारुणात्}
{यद्विशिष्टं महाबाहो सर्वशस्त्रविघातनम्}


\twolineshloka
{येन तत्त्रिपुरं दग्ध्वा क्षणाद्भस्मीकृतं पुरा}
{शरेणैकेन गोविन्द महादेवेन लीलया}


\twolineshloka
{निर्दहेत च यत्कृत्स्नं त्रैलोक्यं सचराचरम्}
{महेश्वरभुजोत्सृष्टं निमेषार्धान्न संशयः}


\twolineshloka
{नावध्यो यस्य लोकेऽस्मिन्ब्रह्मविष्णुसुरेष्वपि}
{तदहं दृष्टवांस्तत्र आश्चर्यमिदमुत्तमम्}


\twolineshloka
{गुह्यमस्त्रवरं नान्यत्तत्तुल्यमधिकं हि वा}
{यत्तच्छूलमिति ख्यातं सर्वलोकेषु शूलिनः}


\twolineshloka
{दारयेद्द्यां मही कृत्स्नां शोषयेद्वा महोदधिम्}
{संहरेद्वा जगत्कृत्स्नं विसृष्टं शूलपाणिना}


\twolineshloka
{यौवनाश्वो हतो येन मान्धाता सबलः पुरा}
{चक्रवर्ती महातेजास्त्रिलोकविजयी नृपः}


\twolineshloka
{महाबलो महावीर्यः शक्रतुल्यपराक्रमः}
{करस्थेनैव गोविन्द लवणस्येह रक्षसः}


\twolineshloka
{तच्छूलमतितीक्ष्णाग्रं सुभीमं रोमहर्षणम्}
{त्रिशिखां भ्रुकुटिं कृत्वा तर्जमानमिव स्थितम्}


\twolineshloka
{विधूमं सार्चिषं कृष्णं कालसूर्यमिवोदितम्}
{सर्पहस्तमनिर्देश्यं पाशहस्तामिवान्तकम्}


\twolineshloka
{दृष्टवानस्मि गोविन्द तदस्त्रं रुद्रसन्निधौ}
{परशुस्तीक्ष्णधारश्च दत्तो रामस्य यः पुराः}


\twolineshloka
{महादेवेन तुष्टेन दत्तं भृगुसुताय च}
{कार्तवीर्यो हतो येन चक्रवर्ती महामृधे}


\twolineshloka
{त्रिःसप्तकृत्वः पृथिवी येन निःक्षत्रिया कृता}
{जामदग्न्येन गोविन्द रामेणाक्लिष्टकर्मणा}


\twolineshloka
{दीप्तधारः सुरौद्रास्यः सर्पकण्ठाग्रधिष्ठितः ॥अभवच्छूलिनोऽभ्याशे दीप्तवह्निशतोपमः}
{}


\twolineshloka
{असङ्ख्येयानि चास्त्राणि तस्य दिव्यानि धीमतः}
{प्राधान्यतो मयैतानि कीर्तितानि तवानघ}


\twolineshloka
{सव्यदेशे तु देवस्य ब्रह्मा लोकपितामहः}
{दिव्यं विमानमास्थाय हंसयुक्तं मनोजवम्}


\twolineshloka
{वामपार्श्वगतश्चापि तथा नारायणः स्थितः}
{वैनतेयं समारुह्य शङ्खचक्रगदाधरः}


\twolineshloka
{स्कन्दो मयूरमास्थाय स्थितो देव्याः समीपतः}
{शक्तिघण्टे समादाय द्वितीय इव पावकः}


\twolineshloka
{पुरस्ताच्चैव देवस्य नन्दिं पश्याम्यवस्थितम्}
{शूलं विष्टभ्य तिष्ठन्तं द्वितीयमिव शङ्करम्}


\twolineshloka
{स्वायंभुवाद्या मनवो भृग्वाद्या ऋषयस्तथा}
{शक्राद्या देवताश्चैव सर्व एव समभ्ययुः}


\twolineshloka
{सर्वभूतगणाश्चैव मातरो विविधाः स्थिताः}
{तेऽभिवाद्य महात्मानं परिवार्य समन्ततः}


\twolineshloka
{अस्तुवन्विविधैः सतोत्रैर्महादेवं सुरास्तदा}
{`जगन्मूर्ति महालिङ्गं तन्मध्ये स्फूतरूपिणम्}


\twolineshloka
{ब्रह्मा भवं तदाऽस्तौवीद्रथन्तरमुदीरयन्}
{ज्येष्ठसाम्ना च देवेशं जगौ नारायणस्तदा}


\threelineshloka
{गृणन्ब्रह्म परं शक्रः शतरुद्रियमुत्तमम्}
{ब्रह्मा नारायणश्चैव देवराजश्च कौशिकः}
{अशोभन्त महात्मानस्त्रयस्त्रय इवाग्नयः}


\twolineshloka
{तेषां मध्यगतो देवो रराज भगवाञ्छिवः}
{शरदभ्रविनिर्मुक्तः परिधिस्थ इवांशुमान्}


\threelineshloka
{अयुतानि च चन्द्रार्कानपश्यं दिवि केशव}
{ततोऽहमस्तुवं देवं स्तवेनानेन सुव्रत ॥उपमन्युरुवाच}
{}


\twolineshloka
{नमो देवाधिदेवाय महादेवाय ते नमः}
{शक्ररूपाय शक्राय शक्रवेषधराय च}


\threelineshloka
{नमस्ते वज्रहस्ताय पिङ्गलायारुणाय च}
{पिनाकपाणये नित्यं शङ्खशूलधराय च}
{}


\twolineshloka
{नमस्ते कृष्णवासाय कृष्णकुञ्चितमूर्धज}
{कृष्णाजिनोत्तरीयाय कृष्णाष्टमिरताय च}


\twolineshloka
{शुक्लवर्णाय शुक्लाय शुक्लाम्बरधराय च}
{शुक्लभस्मावलिप्ताय शुक्लकर्मरताय च}


\twolineshloka
{नमोस्तु रक्तवर्णाय रक्ताम्बरधराय च}
{रक्तध्वजपताकाय रक्तस्रगनुलेपिने}


% Check verse!
नमोस्तु पीतवर्णाय पीताम्बरधराय च
\twolineshloka
{नमोस्तूच्छ्रितच्छत्राय किरीटवरधारिणे}
{अर्धहारार्दकेयूर अर्धकुण्डलकर्णिने}


\twolineshloka
{नमः पवनवेगाय नमो देवाय वै नमः}
{सुरेन्द्राय मुनीन्द्राय महेन्द्राय नमोस्तु ते}


\twolineshloka
{नमः पद्मार्धमालाय उत्पलैर्मिश्रिताय च}
{अर्धचन्दनलिप्ताय अर्धस्रगनुलेपिने}


\twolineshloka
{नम आदित्यवक्त्राय आदित्यनयनाय च}
{नम आदित्यवर्णाय आदित्यप्रतिमाय च}


\twolineshloka
{नमः सोमाय सौम्याय सौम्यवक्त्रधराय च}
{सौम्यरूपाय मुख्याय सौम्यदंष्ट्राविभूषिणे}


\twolineshloka
{नमः श्यामाय गौराय अर्धपीतार्धपाण्डवे}
{नारीनरशरीराय स्त्रीपुंसाय नमोस्तु ते}


\twolineshloka
{नमो वृषभवाहाय गजेन्द्रगमनाय च}
{दुर्गमाय नमस्तुभ्यमगम्यागमनाय च}


\twolineshloka
{नमोस्तु गणनीताय गणवृन्दरताय च}
{गुणानुयातमार्गाय गणनित्यव्रताय च}


\twolineshloka
{नमः श्वेताभ्रवर्णाय संध्यारागप्रभाच य}
{अनुद्दिष्टामभिधानाय स्वरूपाय नमोस्तु ते}


\twolineshloka
{नमो रक्ताग्रवासाय रक्तसूत्रधराय च}
{रक्तमालाविचित्राय रक्ताम्बरधराय च}


\twolineshloka
{मणिभूषितमूर्धाय नमश्चन्द्रार्धभूषिणे}
{विचित्रमणिमूर्धाय कुसुमाष्टधराय च}


\twolineshloka
{नमोऽग्निमुखनेत्राय सहस्रशशिलोचने}
{अग्निरूपाय कान्ताय नमोस्तु गहनाय च}


\twolineshloka
{खचराय नमस्तुभ्यं गोचराभिरताय च}
{भूचराय भुवनाय अनन्ताय शिवाय च}


\twolineshloka
{नमो दिग्वाससे नित्यमधिवाससुवाससे}
{नमो जगन्निवासाय प्रतिपत्तिसुखाय च}


\twolineshloka
{नित्यमुद्बद्धमुकुटे महाकेयूरधारिणे}
{सर्पकण्ठोपहाराय विचित्राभरणाय च}


\twolineshloka
{नमस्त्रिनेत्रनेत्राय सहस्रशतलोचने}
{स्त्रीपुंसाय नपुंसाय नमः साङ्ख्याय योगिने}


\twolineshloka
{शंयोरभिस्रवन्ताय अथर्वाय नमोनमः}
{नमः सर्वार्तिनाशाय नमः शोकहराय च}


\twolineshloka
{नमो मेघनिनादाय बहुमायाधराय च}
{बीजक्षेत्राभिपालाय स्रष्टाराय नमोनमः}


\twolineshloka
{नमः सुरासुरेशाय विश्वेशाय नमोनमः}
{मनः पवनवेगाय नमः पवनरूपिणे}


\twolineshloka
{नमः काञ्चनमालाय गिरिमालाय वै नमः}
{नमः सुरारिमालाय चण्डवेगाय वै नमः}


\twolineshloka
{ब्रह्मशिरोपहर्ताय महिषघ्नाय वै नमः}
{नमः स्त्रीरूपधाराय यज्ञविध्वंसनाय च}


\twolineshloka
{नमस्त्रिपुरहर्ताय यज्ञविध्वंसनाय च}
{नमः कामाङ्गनाशाय कालदण्डधराय च}


\twolineshloka
{नमः स्कन्दविशाखाय ब्रह्मदण्डाय वै नमः}
{नमो भवाय शर्वाय विश्वरूपाय वै नमः}


\twolineshloka
{ईशानाय भवघ्नाय नमोस्त्वन्धकघातिने}
{नमो विश्वाय मायायचिन्त्याचिन्त्याय वै नमः}


\twolineshloka
{त्वं नो गतिश्च श्रेष्ठश्च त्वमेव हृदयं तथा}
{]त्वं ब्रह्मा सर्वदेवानां रुद्राणां नीललोहितः}


\twolineshloka
{आत्मा च सर्वभूतानां साङ्ख्ये पुरुष उच्यते}
{ऋषभस्त्वं पवित्राणां योगिनां निष्कलः शिवः}


\twolineshloka
{गृहस्थस्त्वमाश्रगिणामीश्वराणां महेश्वरः}
{कुबेरः सर्वयक्षाणां क्रतूनां विष्णुरुच्यते}


\twolineshloka
{पर्वतानां भवान्मेरुर्नक्षत्राणां च चन्द्रमाः}
{वसिष्ठस्त्वमृषीणां च ग्रहाणां सूर्य उच्यते}


\twolineshloka
{आरण्यानां पशूनां च सिंहस्त्वं परमेश्वरः}
{ग्राम्याणां गोवृषश्चासि भवाँल्लोक्प्रपूजितः}


\twolineshloka
{आदित्यानां भवान्विष्णुर्वसूनां चैव पावकः}
{पक्षिणां वैनतेयस्त्वमनन्तो भ्रुजगेषु च}


\twolineshloka
{सामवेदश्च वेदानां यजुषां शतरुद्रियम्}
{सनत्कुमारो योगानां साङ्ख्यानां कपिलो ह्यसि}


% Check verse!
शक्रोसि मरुतां देव पितॄणां हव्यवाडसि ॥ब्रह्मलोकश्च लोकानां गतीनां मोक्ष उच्यसे
\twolineshloka
{क्षीरोदः सागराणां च शैलानां हिमवान्गिरिः}
{वर्णानां ब्राह्मणश्चासि विप्राणां दीक्षितो द्विजः}


\twolineshloka
{आदिस्त्वमसि लोकानां संहर्ता काल एव चयच्चान्यदपि लोके वै सर्वतेजोधिकं स्मृतम्}
{तत्सर्वं भगवानेव इति मे निश्चिता मतिः}


\twolineshloka
{नमस्ते भगवन्देव नमस्ते भक्तवत्सलः}
{योगेश्वर नमस्तेऽस्तु नमस्ते विस्वसम्भव}


\twolineshloka
{प्रसीद मम भक्तस्य दीनस्य कृपणस्य च}
{अनैश्वर्येणि युक्तस्य गतिर्भव सनातन}


\twolineshloka
{यच्चापराधं कृतवानज्ञात्वा परमेश्वर}
{मद्भक्त इति देवेश तत्सर्वं क्षन्तुमर्हसि}


\twolineshloka
{मोहितश्चास्मि देवेश त्वया रूपविपर्ययात्}
{नार्घ्यं तेन मया दत्तं पाद्यं चापि महेश्वर}


\twolineshloka
{एवं स्तुत्वाऽहमीशानं पाद्यमर्घ्यं च भक्तितः}
{कृताञ्जलिपुटो भूत्वा सर्वं तस्मै न्यवेदयम्}


\twolineshloka
{ततः शीताम्बुसंयुक्ता दिव्यगन्धसमन्विता}
{पुष्पवृष्टिः शुभा तात पपात मम मूर्धनि}


\twolineshloka
{दुन्दुभिश्च तदा दिव्यस्ताडितो देवकिंकरैः}
{ववौ च मारुतः पुण्यः शुचिगन्धः सुखावहः}


\twolineshloka
{ततः प्रीतो महादेवः सपत्नीको वृषध्वजः}
{अब्रवीत्त्रिदशांस्तत्र हर्षयन्निव मां तदा}


\twolineshloka
{पश्यध्वं त्रिदशाः सर्वे उपमन्योर्महात्मनः}
{मयि भक्तिं परां नित्यमेकभावादवस्थिताम्}


\twolineshloka
{एवमुक्तास्तदा कृष्ण सुरास्ते शूलपाणिना}
{ऊचुः प्राञ्जलयः सर्वे नमस्कृत्वा वृषध्वजम्}


\twolineshloka
{भगवन्देवदेवेश लोकनाथ जगत्पते}
{लभतां सर्वकामेभ्यः फलं त्वत्तो द्विजोत्तमः}


\threelineshloka
{एवमुक्तस्ततः शर्वः सुरैर्ब्रह्मादिभिस्तथा}
{आह मां भगवानीशः प्रहसन्निव शङ्करः ॥भगवानुवाच}
{}


\twolineshloka
{वत्सोपमन्यो तृष्टोस्मि पश्य मां मुनिपुङ्गव}
{दृढभक्तोसि विप्रर्षे मया जिज्ञासितो ह्यसि}


\twolineshloka
{अनया चैव भक्त्या ते अत्यर्थं प्रीतिमानहम्}
{तस्मात्सर्वान्ददाम्यद्य कामांस्तव यथोप्सितान्}


\twolineshloka
{एवमुक्तस्य चैवाथ महादेवेन धीमता}
{हर्षादश्रूण्यवर्तन्त रोमहर्षस्त्वजायत}


\twolineshloka
{अब्रवं च तदा देव हर्षगद्गदया गिरा}
{जानुभ्यामवनीं गत्वा प्रणम्य च पुनःपुनः}


\twolineshloka
{अद्य जातो ह्यहं देव सफलं जन्म चाद्य मे}
{यन्मे साक्षान्महादेवः प्रसन्नस्तिष्ठतेऽग्रतः}


\twolineshloka
{यं न पश्यन्ति चैवाद्धा देवा ह्यमितविक्रमम्}
{तमहं दृष्टवान्देवं कोऽन्यो धन्यतरो मया}


\twolineshloka
{एवं ध्यायन्ति विद्वांसः परं तत्त्वं सनातनम्}
{तद्विशेषमतिख्यातं यदजं ज्ञानमक्षरम्}


\twolineshloka
{स एष भगवान्देवः सर्वसत्त्वादिरव्ययः}
{सर्वतत्त्वविधानज्ञः प्रधानपुरुषः परः}


\twolineshloka
{योऽसृजद्दक्षिणादङ्गाद्ब्रह्माणां लोकसम्भवम्}
{वामपार्श्वात्तथा विष्णुं लोकरक्षार्थमीश्वरः}


\twolineshloka
{युगान्ते चैव सम्प्राप्ते रुद्रमीशोऽसृजत्प्रभुः}
{स रुद्रः संहरन्कृत्स्नं जगत्स्थावरजङ्गमम्}


\twolineshloka
{कालो भूत्वा परं ब्रह्म याति संवर्तकानलः}
{युगान्ते सर्वभूतानि ग्रसन्निव व्यवस्थितः}


\twolineshloka
{एष देवो महादेवो जगत्सृष्ट्वा चराचरम्}
{कल्पान्ते चैव सर्वेषां स्मृतिमाक्षिप्य तिष्ठति}


\twolineshloka
{सर्वगः सर्वभूतात्मा सर्वभूतभवोद्भवः}
{आस्ते सर्वगतो नित्यमदृश्यः सर्वदैवतैः}


\twolineshloka
{यदि देयो वरो मह्यं यदि तुष्टोऽसि मे प्रभो}
{भक्तिर्भक्तु मे नित्यं त्वयि देव सुरेश्वर}


\twolineshloka
{अतीतानागतं चैव वर्तमानं च यद्विभो}
{जानीयामिति मे बुद्धिः प्रसादात्सुरसत्तम}


\twolineshloka
{क्षीरोदनं च भुञ्जीयामक्षयं सह बान्धवैः}
{आश्रमे च सदाऽस्माकं सान्निध्यं परमस्तु ते}


\threelineshloka
{एवमुक्तः स मां प्राह भगवाँल्लोकपूजितः}
{महेश्वरो महातेजाश्वराचरगुरुः शिवः ॥श्रीभगवानुवाच}
{}


\twolineshloka
{अजरश्चामरश्चैव भव त्वं दुःखवर्जितः}
{यशस्वी तेजसा युक्तो दिव्यज्ञानसमन्वितः}


\twolineshloka
{ऋषीणामभिगम्यश्च मत्प्रसादाध्भविष्यसि}
{शीलवान्गुणसम्पन्नः सर्वज्ञः प्रियदर्शनः}


\twolineshloka
{अक्षयं यौवनं तेऽस्तु तेजश्चैवानलोपमम्}
{क्षीरोदः सागरश्चैव यत्रयत्रेच्छसि प्रियम्}


\twolineshloka
{तत्र ते भविता कामं सान्निध्यं पयसोनिधेः}
{क्षीरोदनं च भुङ्ख त्वममृतेन समन्वितम्}


\twolineshloka
{बन्धुभिः सहितः कल्पं ततो मामुपयास्यसि}
{अक्षया बान्धवाश्चैव कुलं गोत्रं च ते सदा}


\twolineshloka
{भविष्यति द्विजश्रेष्ठ मयि भक्तिश्च शाश्वती}
{सान्निध्यं चाश्रमे नित्यं करिष्यामि द्विजोत्तम}


\twolineshloka
{तिष्ठ वत्स यथाकामं नोत्कण्ठां च करिष्यति}
{स्मृतस्त्वया पुनर्विप्र करिष्यामि च दर्शनम्}


\twolineshloka
{एवमुक्त्वा स भगवान्सूर्यकोटिसमप्रभः}
{ईशानः स वरान्दत्त्वा तत्रैवान्तरधीयत}


\twolineshloka
{एवं दृष्टो मया कृष्ण देवदेवः समाधिना}
{तदवाप्तं च मे सर्वं यदुक्तं तेन धीमता}


\twolineshloka
{प्रत्यक्षं चैव ते कृष्ण पश्य सिद्धान्व्यवस्थितान्}
{ऋषीन्विद्याधरान्यक्षान्गन्धर्वाप्सरसस्तथा}


\twolineshloka
{पश्य वृक्षलतागुल्मान्सर्वपुष्पफलप्रदान्}
{सर्वर्तुकुसुमैर्युक्तान्सुखपत्रान्सुगन्धिनः}


\threelineshloka
{सर्वमेतन्महाबाहो दिव्यभावसमन्वितम्}
{प्रसादाद्देवदेवस्य ईशअवरस्य महात्मनः ॥वासुदेव उवाच}
{}


\twolineshloka
{एतच्छ्रुत्वा वचस्तस्य प्रत्यक्षमिव दर्शनम्}
{विस्मयं परमं गत्वा अब्रवं तं महामुनिम्}


\twolineshloka
{धन्यस्त्वमसि विप्रेन्द्रि कस्त्वदन्योस्ति पुण्यकृत्}
{यस्य देवाधिदेवस्ते सान्निध्यं कुरुतेऽऽश्रमे}


\threelineshloka
{अपि तावन्ममाप्येवं दद्यात्स भगवाञ्शिवः}
{दर्शं मुनिशार्दूल प्रसादं चापि शङ्करः ॥उपमन्युरुवाच}
{}


\twolineshloka
{[द्रक्ष्यसे पुण्डरीकाक्ष महादेवं न संशयः}
{अचिरेणैव कालेन यथा दृष्टो मयाऽनघ}


\twolineshloka
{चक्षुषा चैव दिव्येन पश्याम्यमितविक्रमम्}
{षष्ठे मासि महादेवं द्रक्ष्यसे पुरुषोत्तम}


\twolineshloka
{षोडशाष्टौ वरांश्चापि प्राप्स्यसि त्वं महेश्वरात्}
{सपत्नीकाद्यदुश्रेष्ठ सत्यमेतद्ब्रवीमि ते}


\twolineshloka
{अतीतानागतं चैव वर्तमानं च नित्यशः}
{विदितं मे महाबाहो प्रसादात्तस्य धीमतः ॥]}


\twolineshloka
{एतान्सहस्रशश्चान्यान्समनुध्यातवान्हरः}
{कस्मात्प्रसादं भगवान्न कुर्यात्तव माधव}


\fourlineindentedshloka
{त्वादृशेन हि देवानां श्लाघनीय समागमः}
{ब्रह्मणअयेनानृशंसेन श्रद्दधानेन चाप्युत}
{जप्यं तु ते प्रदास्यामि येन द्रक्ष्यसि शंकरम् ॥श्रीकृष्ण उवाच}
{}


\twolineshloka
{अब्रवं तमहं ब्रह्मन्त्वत्प्रसादान्महामुने}
{द्रक्ष्ये दितिजसङ्घानां मर्दनं त्रिदशेश्वरम्}


\twolineshloka
{एवं कथयतस्तस्य महादेवाश्रितां कथाम्}
{दिनान्यष्टौ ततो जग्मुर्मुहूर्तमिव भारत}


\twolineshloka
{दिनेऽष्टमे तु विप्रेणि दीक्षितोऽहं यथाविधि}
{दण्डी मुण्डी कुशी चीरि घृताक्तो मेखलीकृतः}


\twolineshloka
{मासमेकं फलाहारो द्वितीयं सलिलाशनः}
{तृतीयं च चतुर्थं च पञ्चमं चानिलाशः}


\twolineshloka
{एकपादेन तिष्ठंश्च ऊर्ध्वबाहुरतन्द्रितः}
{तेजः सूर्यसहस्रस्य अपश्यं दिवि भारत}


\threelineshloka
{तस्य मध्यगतं चापि तेजसः पाण्डुनन्दन}
{इन्द्रायुधपिनद्धाङ्गं विद्युन्मालागवाक्षकम्}
{नीलसैलचयप्रख्यं बलाकाभूषिताम्बरम्}


\twolineshloka
{तत्र स्थितश्च भगवान्देव्या सह महाद्युतिः}
{तपसा तेजसा कान्त्या दीप्तया सह भार्यया}


\twolineshloka
{रराज भगवांस्तत्र देव्या सह महेश्वरः}
{सोमेन सहितः सूर्यो यथा मेघस्थितस्तथा}


\twolineshloka
{संहृष्टरोमा कौन्तेय विस्मयोत्फुल्ललोचनः}
{अपश्यं देवसङ्घानां गतिमार्तिहरं हरम्}


\twolineshloka
{किरीटिनं गदिनं शूलपाणिंव्याघ्राजिनं जटिलं दण्डपाणिम्}
{पिनाकिनं वज्रिणं तीक्ष्णदंष्ट्रंशुभाङ्गदं व्यालयज्ञोपवीतम्}


\twolineshloka
{दिव्यां मालामुरसाऽनेकवर्णांसमुद्वहन्तं गुल्फदेशावलम्बाम्}
{चन्द्रं यथा परिविष्टं ससन्ध्यंवर्षात्यये तद्वदपश्यमेनम्}


\twolineshloka
{प्रमथानां गणैश्चैव समन्तात्परिवारितम्}
{शरदीव सुदुष्प्रेक्ष्यं परिविष्टं दिवाकरम्}


% Check verse!
एकादशशतान्येवं रुद्राणां वृषवाहनम्अस्तुवं नियतात्मानं कर्मभिः शुभकर्मिणम्
\twolineshloka
{आदित्या वसवः साध्या विश्वेदेवास्तथाऽश्विनौ}
{विश्वाभिः स्तुतिभिर्देवं विश्वदेवं समस्तुवन्}


\twolineshloka
{शतक्रतुश्च भगवान्विष्णुश्चादितिनन्दनौ}
{ब्रह्मा रथन्तरं साम ईरयन्ति भवान्तिके}


\twolineshloka
{योगीश्वराः सुबहवो योगदं पितरं गुरुम्}
{ब्रह्मर्षयश्च ससुतास्तथा देवर्षयश्च वै}


\twolineshloka
{पृथिवीं चान्तरिक्षं च नक्षत्राणि ग्रहास्तथा}
{मासार्धमासा क्रतवो रात्रिः संवत्सराः क्षणाः}


\twolineshloka
{मुहूर्ताश्च निमेपाश्च तथैव युगपर्ययाः}
{दिव्या राजन्नमस्यन्ति विद्याः सत्वविदस्तथा}


\twolineshloka
{सनत्कुमारो देवाश्च इतिहासास्तथैव च}
{मरीचिरङ्गिरा अत्रिः पुलस्त्यः पुलहः क्रतुः}


\twolineshloka
{मनवः सप्त सोमश्च अर्थवा सबृहस्पतिः}
{भृगुर्दक्षः कश्यपश्च वसिष्ठः काश्य एव च}


\twolineshloka
{छन्दांसि दीक्षा यज्ञाश्च दक्षिणाः पावको हविः}
{यज्ञोपगानि द्रव्याणि मूर्तिमन्ति युधिष्ठिर}


\twolineshloka
{प्रजानां पालकाः सर्वे सरितः पन्नगा नगाः}
{देवानां मातरः सर्वादेवपत्न्य सकन्यकाः}


\twolineshloka
{सहस्राणि मुनीनां च अयुतान्यर्बुदानि च}
{नमस्यन्ति प्रभुं शान्तं पर्वताःसागरा दिशः}


\twolineshloka
{गन्धर्वाप्सरसश्चैव गीतवादित्रकोविदाः}
{दिव्यतालेषु गायन्तः स्तुवन्ति भवमद्भुतम्}


\threelineshloka
{विद्याधरा दानवाश्च गुह्यका राक्षसास्तथा}
{सर्वाणि चैव भूतानि स्तावराणि चराणि च}
{नमस्यन्ति महाराज वाङ्मनः कर्मभिर्विभुम्}


% Check verse!
पुरस्ताद्धिष्ठितः शर्वो ममासीस्त्रिदशेश्वरः
\twolineshloka
{पुरस्तद्धिष्ठितं दृष्ट्वा ममेशानं च भारत}
{सप्रजापतिशक्रान्तं जगन्मामभ्युदैक्षत}


\twolineshloka
{ईक्षितुं च महादेवं न मे शक्तिरभूत्तदा}
{ततो मामब्रवीद्देवः पश्य कृष्ण वदस्व च}


\twolineshloka
{त्वया ह्याराधितश्चाहं शतशोऽथ सहस्रशः}
{त्वत्समो नास्ति मे कश्चित्त्रिषु लोकषु वै प्रियः}


\threelineshloka
{शिरसा वन्दिते देवे देवी प्रीता ह्युमा तदा}
{ततोऽहमब्रुवं स्थाणुं स्तुतं ब्रह्मादिभिः सुरैः ॥कृष्ण उवाच}
{}


\twolineshloka
{नमोस्तु ते शाश्वत सर्वयोनेब्राह्माधिपं त्वामृषयो वदन्ति}
{तपश्च सत्वं च रजस्तमश्चत्वामेव सत्यं च वदन्ति सन्तः}


\twolineshloka
{त्वं वै ब्रह्मा च रुद्रश्च वरुणोऽग्निर्मनुर्भवः}
{धाता त्वष्टा विधाता च त्वं प्रभुः सर्वतोमुखः}


\twolineshloka
{त्वत्तो जातानि भूतानि स्थावराणि चराणि च}
{त्वया सृष्टमिदं कृत्स्नं त्रैलोक्यं सचराचरम्}


\twolineshloka
{यानीन्द्रियाणीह मनश्च कृत्स्नंये वायवः सप्ति तथैव चाग्नयः}
{ये देवसंस्थास्तव देवताश्चतस्मात्परं त्वामृषयो वदन्ति}


\twolineshloka
{वेदाश्च यज्ञाः सोमश्च दक्षिणा पावको हविः}
{यज्ञोपगं च यत्किंचिद्भगवांस्तदसंशयम्}


\twolineshloka
{इष्टं दत्तमधीतं व्रतानि नियमाश्च ये}
{ह्रीः कीर्तिः श्रीर्द्युतिस्तुष्टिः सिद्धिश्चैव तदर्पणी}


\twolineshloka
{कामः क्रोधो भयं लोभो मदः स्तम्भोथ मत्सरः}
{आधयो व्याधयश्चैव भगवंस्तनवस्तव}


\twolineshloka
{कृतिर्विकारः प्रणयः प्रधानं बीजमव्ययम्}
{मनसः परमा योनिः प्रभावश्चापि शाश्वतः}


\twolineshloka
{अव्यक्तः पावनोऽचिन्त्यः सहस्रांशुर्हिरण्मयः}
{आदिर्गणानां सर्वेषां भवान्वैजीविताश्रयः}


\twolineshloka
{महानात्मा मतिर्ब्रह्मा विश्वः शंभुः स्वयंभुवः}
{बुद्धिः प्रज्ञोपलब्धिश्चसंवित्ख्यातिर्धृतिः स्मृतिः}


\twolineshloka
{पर्यायवाचकैः शब्दैर्महानात्मा विभाव्यते}
{त्वां बुद्ध्वा ब्राह्मणो वेदात्प्रमोहं विनियच्छति}


% Check verse!
हृदयं सर्वभूतानां क्षेत्रज्ञस्त्वमृषिस्तुतः
\twolineshloka
{सर्वतः पाणिपादस्त्वं सर्वतोक्षिशिरोमुखः}
{सर्वतः श्रुतिमाँल्लोके सर्वमावृत्य तिष्ठसि}


\threelineshloka
{फलं त्वमसि दिग्मांशोर्निमेषादिषु कर्मसु}
{त्वं वै प्रबार्चिः पुरुषः सर्वस्य हृदि संश्रितः}
{अणिमा महिमा प्राप्तिरीशानो ज्योतिरव्ययः}


\twolineshloka
{त्वयि बुद्धिर्मतिर्लोकाः प्रपन्नाः संश्रिताश्च ये}
{ध्यानिनो नित्ययोगाश्च सत्यसत्वा जितेन्द्रियाः}


\twolineshloka
{यस्त्वां ध्रुवं वेदयते गुहाशयंप्रभुं पुराणं पुरुषं विश्वरूपम्}
{हिरण्मयं बुद्धिमतां परां गतिंस बुद्धिमान्बुद्धिमतीत्य तिष्ठति}


\twolineshloka
{विदित्वा सप्त सूक्ष्माणि षडङ्गं त्वां च मूर्तितः}
{प्रधानविधियोगस्थस्त्वामेव विशते बुधः}


\twolineshloka
{एवमुक्ते मया पार्थ भवे चार्तिविनाशने}
{चराचरं जगत्सर्गं सिंहनादं तदाऽकरोत्}


\twolineshloka
{तं विप्रसङ्घाश्च सुरासुराश्चनागाः पिशाचाः पितरो वयांसि}
{रक्षोगणा भूतगणाश्च सर्वेमहर्षयश्चैव तदा प्रणेमुः}


\twolineshloka
{मम मूर्ध्नि च दिव्यानां कुसुमानां सुगन्धिनाम्}
{राशयो निपतन्ति स्म वायुश्च सुसुखो ववौ}


\twolineshloka
{निरीक्ष्य भगवान्देवीं ह्युमां मां च जगद्धितः}
{शतक्रतुं चाभिवीक्ष्य स्वयं मामाह शङ्करः}


\twolineshloka
{विद्मः कृष्ण परां भक्तिमस्मासु तव शत्रुहन्}
{क्रियतामात्मनः श्रेयः प्रीतिर्हित्वयि मे परा}


\twolineshloka
{वृणीष्वाष्टौ वरान्कृष्ण दाताऽस्मि तव सत्तम}
{ब्रूहि यादवशार्दूल यानिच्छसि सुदुर्लभान्}


\chapter{अध्यायः ४६}
\twolineshloka
{मूर्ध्ना निपत्य नियतस्तेजःसन्निचये ततः}
{परमं हर्षमागत्य भगवन्तमथाब्रवम्}


% Check verse!
धर्मे दृढत्वं युधि शत्रुघातंयशस्तथाऽग्र्यं परमं बलं चयोगप्रियत्वं तव सन्निकर्षंवृणे सुतानां च शतं शतानि
% Check verse!
एवमस्त्विति तद्वाक्यं मयोक्तः प्राह शङ्करः
\threelineshloka
{ततो मां जगतो माता धारिणी सर्वपावनी}
{उवाचोमा प्रणिहिता शर्वाणी तपसां निधिः}
{दत्तो भगवता पुत्रः साम्बो नाम तवानघ}


\twolineshloka
{मत्तोप्यष्टौ वरानिष्टान्गृहण त्वं ददामि ते}
{प्रणम्य शिरसा सा च मयोक्ता पाण्डुनन्दन}


\threelineshloka
{द्विजेष्वकोपं पितृतः प्रसादंशतं सुतानां परमं च भोगम्}
{कुले प्रीतिं मातृतश्च प्रसादं-शमप्राप्तिं प्रवृणे चापि दाक्ष्यम् ॥उमोवाच}
{}


\twolineshloka
{एवं भविष्यत्यमरप्रभावनाहं मृषा जातु वदे कदाचित्}
{भार्यासहस्राणि च षोडशैवतासु प्रियत्वं च तथाऽक्षयं}


\threelineshloka
{प्रीतिं चाग्र्यां बान्धवानां सकाशा-द्ददामि तेऽहं वपुषः काम्यतां च}
{भोक्ष्यन्ते वै सप्ततिं वै शतानिगृहे तुभ्यमतिथीनां च नित्यम् ॥वासुदेव उवाच}
{}


\twolineshloka
{एवं दत्त्वा वरान्देवो मम देवी च भारत}
{अन्तर्हितः क्षणे तस्मिन्सगणो भीमपूर्वज}


\threelineshloka
{एतदत्यद्भुतं पूर्वं ब्राह्मणायातितेजसे}
{उपमन्यवे मया कृत्स्नं व्याख्यातं पार्थिवोत्तम}
{नमस्कृत्वा तु स प्राह देवदेवाय सुव्रत}


\twolineshloka
{नास्ति शर्वसमो देवो नास्ति शर्वसमा गतिः}
{नास्ति शर्वसमो दाने नास्ति शर्वसमो रणे}


\chapter{अध्यायः ४७}
\twolineshloka
{ऋषिरासीत्कृते तात तण्डिरित्येव विश्रुतः}
{दशवर्षसहस्राणि तेन देवः समाधिना}


\twolineshloka
{आराधितोऽभूद्भक्तेन तस्योदर्कं निशामय}
{स दृष्टवान्महादेवमस्तौषीच्च स्तवैर्विभुम्}


\twolineshloka
{[इति तण्डिस्तपोयोगात्परमात्मानमव्ययम्}
{चिन्तयित्वा महात्मानमिदमाह सुविस्मितः}


\threelineshloka
{यं पठन्ति सदा साङ्ख्याश्चिन्तयन्ति च योगिनः}
{परं प्रधानं पुरुषमधिष्ठातारमीश्वरम्}
{}


\twolineshloka
{उत्पत्तौ च विनाशे च कारणं यं विदुर्बुधाः}
{देवासुरमुनीनां च परं यस्मान्न विद्यते}


\twolineshloka
{अजं तमहमीशानमनादिनिधनं प्रभुम्}
{अत्यन्तसुखिनं देवमनघं शरणं व्रजे}


\twolineshloka
{एवं ब्रुवन्नेव तदा ददर्श तपसान्निधिम्}
{तमव्ययमनौपम्यमचिन्त्यं शाश्वतं ध्रुवम्}


\twolineshloka
{निष्कलं सकलं ब्रह्म निर्गुणं गुणगोचरम्}
{योगिनां परमानन्दमक्षरं मोक्षसंज्ञितम्}


\twolineshloka
{मनोरिन्द्राग्निमरुतां विश्वस्य ब्रह्मणो गतिम््}
{अग्राह्यमचलं शुद्धं बुद्धिग्राह्यं मनोमयम्}


\twolineshloka
{दुर्विज्ञेयमसङ्ख्ये दुष्प्रापमकृतात्मभिः}
{योनि विश्वस्य जगतस्तमसः परतः परम्}


\fourlineindentedshloka
{यः प्राणवन्तमात्मानं ज्योतिर्जीवस्थितं मनः}
{तं देवं दर्शनाकाङ्क्षी बहून्वर्षगणानृषिः}
{तपस्युग्रे स्थितो भूत्वा दृष्ट्वा तुष्टाव चेश्वरम् ॥]तण्डिरुवाच}
{}


% Check verse!
पवित्राणां पवित्रस्त्वं गतिर्गतिमतांवर
\twolineshloka
{अत्युग्रं तेजसां तेजस्तपसां परमं तपः}
{विश्वावसुहिरण्याक्षपुरुहूतनमस्तृत}


\twolineshloka
{भूरिकल्याणद विभो परं सत्यं नमोस्तु ते}
{जातीमरणभीरूणां यतीनां यततां विभो}


\twolineshloka
{निर्वाणद सहस्रांशो नमस्तेऽस्तु सुखाश्रय}
{ब्रह्मा शतक्रतुर्विष्णुर्विश्वेदेवा महर्षयः}


\twolineshloka
{न विदुस्त्वां तु तत्त्वेन कुतो वेत्स्यामहे वयम्}
{त्वत्तः प्रवर्तते सर्वं त्वयि सर्वं प्रतिष्ठितम्}


\twolineshloka
{कालाख्यः पुरुषाख्यश्च ब्रह्माख्यश्च त्वमेव हि}
{तनवस्ते स्मृतास्तिस्रः पुराणज्ञैः सुरर्षिभिः}


\twolineshloka
{अधिपौरुषमध्यात्ममधिभूताधिदैवतम्}
{अधिलोकाधिविज्ञानमधियज्ञस्त्वमेव हि}


\twolineshloka
{त्वां विदित्वाऽऽत्मदेवस्थं दुर्विदं दैवतैरपि}
{विद्वांसो यान्ति निर्मुक्ताः परं भावमनामयम्}


\twolineshloka
{अनिच्छतस्तव विभो जन्ममृत्युरनेकतः}
{द्वारं तु स्वर्गमोक्षाणामाक्षेप्ता त्वं ददासि च}


\twolineshloka
{त्वं वै स्वर्गश्च मोक्षश्च कामः क्रोधस्त्वमेव च}
{सत्वं रजस्ममश्चैव अधश्चोर्ध्वं त्वमेव हि}


\twolineshloka
{ब्रह्मा भवश्च विष्णुश्च स्कन्देन्द्रौ सविता यमः}
{वरुणेन्दू मनुर्धाता विधाता त्वं धनेश्वरः}


\twolineshloka
{भूर्वायुः सलिलाग्निश्च खं वाग्बुद्धिः स्थितिर्मतिः}
{कर्म सत्यानृते चोभे त्वमेवास्ति च नास्ति च}


\twolineshloka
{इन्द्रियाणीन्द्रियार्थाश्च प्रकृतिभ्यः परं ध्रुवम्}
{विश्वाविश्वपरो भावश्चिन्त्याचिन्त्यस्त्वमेव हि}


\twolineshloka
{यच्चैतत्परमं ब्रह्म यच्च तत्परमं पदम्}
{या गतिः साङ्ख्ययोगानां स भवान्नात्र संशयः}


\twolineshloka
{नूनमद्य कृतार्थाः स्म नूनं प्राप्ताः सतां गतिम्}
{यां गतिं प्रार्तयन्तीह ज्ञाननिर्मलबुद्धयः}


\twolineshloka
{अयो मूढाः स्म सुचिरमिमं कालमचेतसा}
{यन्न विद्मः परं देवं शाश्वतं यं विदुर्बुधाः}


\twolineshloka
{सेयमासादिता साक्षात्त्वद्भक्तिर्जन्मभिर्मया}
{भक्तानुग्रहकृद्देवो यं ज्ञात्वाऽमृतमश्नुते}


\twolineshloka
{देवासुरमुनीनां तु यच्च गुह्यं सनातनम्}
{गुहायां निहितं ब्रह्मि दुर्विज्ञेयं मुनेरपि}


\twolineshloka
{स एष भगवान्देवः सर्वकृत्सर्वतोमुखः}
{सर्वात्मा सर्वदर्शी च सर्वगः सर्ववेदिता}


\twolineshloka
{देहकृद्देहभृद्देही देहभुग्देहिनां गतिः}
{प्राणकृत्प्राणभृत्प्राणी प्राणदः प्राणिनां गतिः}


\twolineshloka
{अध्यात्मगतिरिष्टानां ध्यायिनामात्मवेदिनाम्}
{अपुनर्भवकामानां या गतिः सोऽयमीश्वरः}


\fourlineindentedshloka
{अयं च सर्वभूतानां शुभाशुभगतिप्रदः}
{अयं च जन्ममरणे विदध्यात्सर्वजन्तुषु}
{अयं संसिद्धिकामानां या गतिः सोयमीस्वरः}
{}


\twolineshloka
{भूराद्यान्सर्वभुवनानुत्पाद्य सदिवौकसः}
{दधाति देवस्तनुभिरष्टाभिर्यो बिभर्ति च}


\twolineshloka
{अतः प्रवर्तते सर्वमस्मिन्सर्वं प्रतिष्ठितम्}
{अस्मिंश्च प्रलयं याति अयमेकः सनातनः}


\twolineshloka
{अयं स सत्यकामानां सत्यलोकः परं सताम्}
{अपवर्गश्च मुक्तानां कैवल्यं चात्मवेदिनाम्}


\twolineshloka
{अयं ब्रह्मादिभिः सिद्धैर्गुहायां गोपितः प्रुभुः}
{देवासुरमनुष्याणामप्रकाशो भवेदिति}


\twolineshloka
{तं त्वां देवासुरनरास्तत्त्वेन न विदुर्भवम्}
{मोहिताः खल्वनेनैव हृदिस्थेनाप्रकाशिना}


\twolineshloka
{ये चैनं प्रतिपद्यन्ते भक्तियोगेन भाविताः}
{तेषामेवात्मनात्मानं दर्शयत्येष हृच्छयः}


\twolineshloka
{यं ज्ञात्वा न पुनर्जन्म मरणं चापि विद्यते}
{यं विदित्वा परं वेद्यं वेदितव्यं न विद्यते}


\twolineshloka
{यं लब्ध्वा परमं लाभं नाधिकं मन्यते बुधः}
{यां सूक्ष्मां परमां प्राप्तिं गच्छन्नव्ययमक्षयम्}


\twolineshloka
{यं साङ्ख्या गुणतत्त्वज्ञाः साङ्ख्यशास्त्रविशारदाः}
{सूक्ष्मज्ञानतराः सूक्ष्मं ज्ञात्वा मुच्यन्ति बन्धनैः}


\twolineshloka
{यं च वेदविदो वेद्यं वेदान्ते च प्रतिष्ठितम्}
{प्राणायामपरा नित्यं यं विशन्ति जपन्ति च}


\twolineshloka
{ओंकाररथमारुह्य ते विशन्ति महेश्वरम्}
{अयं स देवयानानामादित्यो द्वारमुच्यते}


\twolineshloka
{अयं च पितृयानानां चन्द्रमा द्वारमुच्यते}
{एष काष्ठा दिशश्चैव संवत्सरयुगादि च}


\twolineshloka
{दिव्यादिव्यः परो लाभ अयने दक्षिणोत्तरे}
{एनं प्रजापतिः पूर्वमाराध्य बहुभिः स्तवैः}


\twolineshloka
{प्रजार्थं वरयामास नीललोहितसंज्ञितम्}
{क्रग्भिर्यमनुशासन्ति तत्त्वे कर्मणि बह्वृचाः}


\twolineshloka
{यजुर्भिर्यत्त्रिधा वेद्यं जुह्वत्यध्वर्यवोऽध्वरे}
{सामभिर्यं च गायन्ति सामगाः शुद्धबुद्धयः}


\twolineshloka
{ऋतं सत्यं परं ब्रह्म स्तुवन्त्याथर्वणा द्विजाः}
{यज्ञस्य परमा योनिः परिश्चायं परः स्मृतः}


\twolineshloka
{रात्र्यहः श्रोत्रनयनः पक्षमासशिरोभुजः}
{ऋतुवीर्यस्तपोधैर्यो ह्यब्दगुह्योरुपादवान्}


\twolineshloka
{मृत्युर्यमो हुताशश्च कालः संहारवेगवान्}
{कालस्य परमा योनिः कालश्चायं सनातनः}


\twolineshloka
{चन्द्रादित्यौ सनक्षत्रौ ग्रहाश्च सह वायुना}
{ध्रुवः सप्तर्षयश्चैव भुवनाः सप्त एव च}


\twolineshloka
{प्रधानं महदव्यक्तं विशेषान्तं सवैकृतम्}
{ब्रह्मादिस्तम्बपर्यन्तं भूतादि सदसच्च यत्}


\twolineshloka
{अष्टौ प्रकृतयश्चैव प्रकृतिभ्यश्च यः परः}
{अस्य देवस्य यद्भागं कृत्स्नं सम्परिवर्तते}


\twolineshloka
{एतत्परममानन्दं यत्तच्छाश्वतमेव च}
{एषा गतिर्विरक्तानामेष भावः परः सताम्}


\twolineshloka
{एतत्पदमनुद्विग्नमेतद्ब्रह्म सनातनम्}
{शास्त्रवेदाङ्गविदुषामेतद्ध्यानं परं पदम्}


\twolineshloka
{इयं सा परमा काष्ठा इयं सा परमा कला}
{इयं सा परमा सिद्धिरियं सा परमा गतिः}


\twolineshloka
{इयं सा परमा शान्तिरियं सा निर्वृतिः परा}
{यं प्राप्य कृतकृत्याः स्म इत्यमन्यन्त योगिनः}


\twolineshloka
{इयं तुष्टिरियं सिद्धिरियं श्रुतिरियं स्मृतिः}
{अध्यात्मगतिरिष्टानां विदुषां प्राप्तिरव्यया}


\twolineshloka
{यजतां कामयानानां मखैर्विपुलदक्षिणैः}
{या गतिर्यज्ञशीलानां सा गतिस्त्वं न संशयः}


\twolineshloka
{सम्यग्योगजपैः शान्तिर्नियमैर्देहतापनैः}
{तप्यतां या गतिर्देव परमा सा गतिर्भवान्}


\twolineshloka
{कर्मन्यासकृतानां च विरक्तानां ततस्ततः}
{या गतिर्ब्रह्मिसदने सा गतिस्त्वं सनातन}


\twolineshloka
{अपुनर्भवकामानां वैराग्ये वर्ततां च या}
{प्रकृतीनां लयानां च सा गतिस्त्वं सनातन}


\twolineshloka
{ज्ञानविज्ञानयुक्तानां निरुपाख्या निरञ्जना}
{कैवल्या या गतिर्देव परमा सा गतिर्भवान्}


\twolineshloka
{वेदशास्त्रपुराणोक्ताः पञ्च ता गतयः स्मृताः}
{त्वत्प्रसादाद्धि लभ्यन्ते न लभ्यन्तेऽन्यथा विभो}


\threelineshloka
{इति तण्डिस्तपोराशिस्तुष्टावेसानमात्मना}
{जगौ च परमं ब्रह्म यत्पुरा लोककृज्जगौ ॥उपमन्युरुवाच}
{}


\twolineshloka
{एवं स्तुतो महादेवस्तण्डिना ब्रह्मवादिना}
{उवाच भगवान्देव उमया सहितः प्रभुः}


\twolineshloka
{ब्रह्मा शतक्रतुर्विष्णुर्विश्वेदेवा महर्षयः}
{न विदुस्त्वामिति ततस्तुष्टः प्रोवाच तं शिवः}


\twolineshloka
{अक्षयश्चाव्ययश्चैव भविता दुःखवर्जितः}
{यशस्वी तेजसा युक्तो दिव्यज्ञानसमन्वितः}


\twolineshloka
{ऋषीणामभिगम्यश्च सूत्रकर्ता सुतस्तव}
{मत्प्रसादाद्द्विजश्रेष्ठ भविष्यति न संशयः}


\threelineshloka
{कं वा कामं ददाम्यद्य ब्रूहि यद्वत्स काङ्क्षसे}
{प्राञ्जलिः स उवाचेदं त्वयि भक्तिर्दृढाऽस्तु मे ॥उपमन्युरुवाच}
{}


\twolineshloka
{एतान्दत्त्वा वरान्देवो वन्द्यमानः सुरर्षिभिः}
{स्तूयमानश्च विबुधैस्तत्रैवान्तरधीयत}


\twolineshloka
{अन्तर्हिते भगवति सानुगे यादवेश्वर}
{ऋषिराश्रममागम्य ममैतत्प्रोक्तवानिह}


% Check verse!
यानि च प्रथितान्यादौ तण्डिराख्यातवान्मममानानि मानवश्रेष्ठ तानि त्वं शृणु सिद्धये
\twolineshloka
{दश नामसहस्राणि देवेष्वाह पितामहः}
{सर्वस्य शास्त्रेषु तथा दश नामशतानि च}


\twolineshloka
{गुह्यानीमानि नामानि तण्डिर्भगवतोऽच्युत}
{देवप्रसादाद्देवेशः पुरा प्राह महात्मने}


\chapter{अध्यायः ४८}
\threelineshloka
{ततः स प्रयतो भूत्वा मम तात युधिष्ठिर}
{प्राञ्जलिः प्राह विप्रर्षिर्नामसङ्ग्रहमादितः ॥उपमन्युरुवाच}
{}


\twolineshloka
{ब्रह्मप्रोक्तैर्ऋषिप्रोक्तैर्वेदवेदाङ्गसम्भवै}
{सर्वलोकेषु विख्यातं स्तुत्यं स्तोष्यामि नामभिः}


\twolineshloka
{महद्भिर्विहितैः सत्यैः सिद्धैः सर्वार्थसाधकैः}
{ऋषिणा तण्डिना भक्त्या कृतैर्वेदकृतात्मना}


\twolineshloka
{यथोक्तैः साधुभिः ख्यातैर्मुनिभिस्तत्त्वदर्शिभिः}
{प्रवरं प्रथमं स्वर्ग्यं सर्वभूतहितं शुभम्}


\threelineshloka
{श्रुतेः सर्वत्र जगति ब्रह्मलोकावतारितैः}
{सत्यैस्तत्परमं ब्रह्मि ब्रह्मप्रोक्तं सनातनम्}
{वक्ष्ये यदुकुलश्रेष्ठ शृणुष्वावहितो मम}


\twolineshloka
{वरयैनं भवं देवं भक्तस्त्वं परमेश्वरम्}
{तेन ते श्रावयिष्यामि यत्तद्ब्रह्म सनातनम्}


\twolineshloka
{न शक्यं विस्तरात्कृत्स्नं वक्तुं सर्वस्य केनचित्}
{युक्तेनापि विभूतीनामपि वर्षशतैरपि}


\twolineshloka
{यस्यादिर्मध्यमन्तं च सुरैरपि न गम्यते}
{कस्तस्य शक्नुयाद्वक्तं गुणान्कार्त्स्न्येन माधव}


\twolineshloka
{किन्तु देवस्य महतः संक्षिप्तार्थपदाक्षरम्}
{शक्तितश्चरितं वक्ष्ये प्रसादात्तस्य धीमतः}


\twolineshloka
{अप्राप्य तु ततोऽनुज्ञां न शक्यः स्तोतुमीश्वरः}
{यदा तेनाभ्यनुज्ञातः स्तुतो वै स तदा मया}


\twolineshloka
{अनादिनिधनस्याहं जगद्योनेर्महात्मनः}
{नाम्नां कञ्चित्समुद्देशं वक्ष्याम्यव्यक्तयोनिनः}


\twolineshloka
{वरदस्य वरेण्यस्य विश्वरूपस्य धीमतः}
{शृणु नाम्नां चयं कृष्ण यदुक्तं पद्मयोनिना}


\twolineshloka
{दश नामसहस्राणि यान्याह प्रपितामहः}
{तानि निर्मथ्य मनसा दध्नो घृतमिवोद्धृतम्}


\threelineshloka
{गिरेः सारं यथा हेम पुष्पसारं यथा मधु}
{घृतात्सारं यथा मण्डस्तथैतत्सारमुद्धृतम्}
{}


\twolineshloka
{सर्वपापापहमिदं चतुर्वेदसमन्वितम्}
{प्रयत्नेनाधिगन्तव्यं धार्यं च प्रयतात्मना}


% Check verse!
माङ्गल्यं पौष्टिकं चैव रक्षोघ्नं पावनं महत्
\twolineshloka
{इदं भक्ताय दातव्यं श्रद्दघानास्तिकाय च}
{नाश्रद्दधानरूपाय नास्तिकायाजितात्मने}


\twolineshloka
{यश्चाभ्यसूयते देवं कारणात्मानमीश्वरम्}
{स कृष्ण नरकं याति सहपूर्वैः सहात्मजैः}


\twolineshloka
{इदं ध्यानमिदं योगमिदं ध्येयमनुत्तममम्}
{इदं जप्यमिदं ज्ञानं रहस्यमिदमुत्तम्}


\twolineshloka
{यं ज्ञात्वा अन्तकालेऽपि गच्छेत परमां गतिम्}
{पवित्रं मङ्गलं मेध्यं कल्याणमिदमुत्तमम्}


\twolineshloka
{इदं ब्रह्मा पुरा कृत्वा सर्वलोकपितामहः}
{सर्वस्तवानां राजत्वे दिव्यानां समकल्पयत्}


\twolineshloka
{तदाप्रभृति चैवायमीश्वरस्य महात्मनः}
{स्तवराज इति ख्यातो जगत्यमरपूजितः}


\twolineshloka
{ब्रह्मलोकादयं स्वर्गे स्तवराजोऽवतारितः}
{यतस्तण्डिः पुरा प्राप तेन तण्डिकृतोऽभवत्}


\twolineshloka
{स्वर्गाच्चैवात्र भूर्लोकं तण्डिना ह्यवतारितः}
{सर्वमङ्गलमाङ्गल्यं सर्वपापप्रणाशनम्}


\twolineshloka
{निगदिष्ये महाबाहो स्तवानामुत्तमं स्तवम्}
{ब्रह्मणामपि यद्ब्रह्म पराणामपि यत्परम्}


\twolineshloka
{तेजसामपि यत्तेजस्तपसामपि यत्तपः}
{शान्तीनामपि या शान्तिर्द्युतीनामपि या द्युतिः}


\twolineshloka
{दान्तानामपि यो दान्तो धीमतामपि या च धीः}
{देवानामपि यो देव ऋषीणामपि यस्त्वृषिः}


\twolineshloka
{यज्ञानामपि यो यज्ञः शिवानामपि यः शिवः}
{रुद्राणामपि यो रुद्रः प्रभा प्रभवतामपि}


\twolineshloka
{योगिनामपि यो योगी कारणानां च कारणम्}
{यतो लोकाः सम्भवन्ति नभवन्ति यतः पुनः}


\threelineshloka
{सर्वभूतात्मभूतस्य हरस्यामिततेजसः}
{अष्टोत्तरसहस्रं तु नाम्नां शर्वस्य मे शृणु}
{यच्छ्रुत्वा मनुजव्याघ्र सर्वान्कामानवाप्स्यसि}


\twolineshloka
{स्थिरः स्थाणुः प्रभुर्भानुः, प्रवरो वरदो वरः}
{सर्वात्मा सर्वविख्यातः सर्वः सर्वकरो भवः}


\twolineshloka
{जटी चर्मीं शिखी खङ्गी सर्वाङ्गः सर्वभावनः}
{हरश्च हरिणाक्षश्च सर्वभूतहरः प्रभुः}


\twolineshloka
{प्रवृत्तिश्च निवृत्तिश्च नियतः शाश्वतो ध्रुवः}
{श्मशानवासी भगवान्खचरो गोचरोऽर्दनः}


\twolineshloka
{अभिवाद्यो महाकर्मा तपस्वी भूतभावनः}
{उन्मत्तवेषप्रच्छन्नः सर्वलोकप्रजापतिः}


\twolineshloka
{महारूपो महाकायो वृषरूपो महायशाः}
{महात्मा सर्वभूतात्मा विश्वरूपो महाहनुः}


\twolineshloka
{लोकपालोऽन्तर्हितात्मा प्रसादो हयगर्दभिः}
{पवित्रं च महांश्चैव नियमो नियमाश्रितः}


\twolineshloka
{सर्वकर्मा स्वयंभूत आदिरादिकरो निधिः}
{सहस्राक्षो विशालाक्षः सोमो नक्षत्रसाधकः}


\twolineshloka
{चन्द्रः सूर्यः शनिः केतुर्ग्रहो ग्रहपतिर्वरः}
{अत्रिरत्र्या नमस्कर्ता मृगबाणार्पणोऽनघः}


\twolineshloka
{महातपा घोरतपा अदीनो दीनसाधकः}
{संवत्सरकरो मन्त्रः प्रमाणं परमं तपः}


\twolineshloka
{योगी योज्यो महाबीजो महारेता महाबलः}
{सुवर्णरेताः सर्वज्ञः सुबीजो बीजवाहनः}


\twolineshloka
{दशबाहुस्त्वनिमिषो नीलकण्ठ उमापतिः}
{विश्वरूपः स्वयं श्रेष्ठो बलवीरो बलो गणः}


\twolineshloka
{गणकर्ता गणपतिर्दिग्वासाः काम एव च}
{मन्त्रवित्परमो मन्त्रः सर्वभावकरो हरः}


\twolineshloka
{कमण्डलुधरो धन्वी बाणहस्तः कपालवान्}
{अशनी शतघ्नी खङ्गी पट्टिशी चायुधी महान्}


\twolineshloka
{स्रुवहस्तः सुरूपश्च तेजस्तेजस्करो निधिः}
{उष्णीषी च सुवक्त्रश्च उदग्रो विनतस्तथा}


\twolineshloka
{दीर्घश्च हरिकेशश्च सुतीर्थः कुष्ण एव च}
{शृगालरूपः सिद्धार्थो मुण्डः सर्वशुभंकरः}


\twolineshloka
{अजश्च बहुरूपश्च गन्धधारी कपर्द्यपि}
{ऊर्ध्वरेता ऊर्ध्वलिङ्ग ऊर्ध्वशायी नभःस्थलः}


\twolineshloka
{त्रिजटी चीरवासाश्च रुद्रः सेनापतिर्विभुः}
{अहश्चरो नक्तंचरस्तिग्ममन्युः सुवर्चसः}


\twolineshloka
{गजहा दैत्यहा कालो लोकधाता गुणाकरः}
{सिंहशार्दूलरूपश्चि आर्द्रचर्माम्बरावृतः}


\twolineshloka
{कालयोगी महानादः सर्वकामश्चतुष्पथः}
{निशाचरः प्रेतचारी भूतचारी महेश्वरः}


\twolineshloka
{बहुभूतो बहुधरः स्वर्भिनुरमितो गतिः}
{नृत्यप्रियो नित्यनर्तो नर्तकः सर्वलालसः}


\twolineshloka
{घोरो महातपाः पाशो नित्यो गिरिरुहो नभः}
{सहस्रहस्तो विजयो व्यवसायो ह्यतन्द्रितः}


\twolineshloka
{अधर्षणो धर्षणात्मा यज्ञहा कामनाशकः}
{दक्षयागापहारी च सुसहो मध्यमस्तथा}


\twolineshloka
{तेजोपहारी बलहा मुदितोऽर्थोऽजितो वरः}
{गम्भीरघोषो गम्भीरो गम्भीरबलवाहनः}


\twolineshloka
{न्यग्रोधरूपो न्यग्रोधो वृक्षकर्णस्थितिर्विभुः}
{सुतीक्ष्णदशनश्चैव महाकायो महाननः}


\twolineshloka
{विष्वक्सेनो हरिर्यज्ञः संयुगापीडवाहनः}
{तीक्ष्णतापश्च हर्यश्वः सहायः कर्मकालवित्}


\twolineshloka
{विष्णुप्रसादितो यज्ञः समुद्रो वडवामुखः}
{हुताशनसहायश्च प्रशान्तात्मा हुताशनः}


\twolineshloka
{उग्रतेजा महातेजा जन्यो विजयकालवित्}
{ज्योतिषामयनं सिद्धिः सर्वविग्रह एव च}


\twolineshloka
{शिंखी मुण्डी जटी ज्वाली मूर्तिजो मूर्धगो बली}
{वेणवी पणवी ताली खली कालकटंकटः}


\twolineshloka
{नक्षत्रविग्रहमतिर्गुणबुद्धिर्लयो गमः}
{प्रजापतिर्विश्वबाहुर्विभागः सर्वगोमुखः}


% Check verse!
विमोचनः सुसरणो हिरण्यकवचोद्भवः ॥मेढ्रजो बलचारी च महीचारी स्रुतस्तथा
\twolineshloka
{सर्वतूर्यनिनादी च सर्वातोद्यपरिग्रहः}
{व्यालरूपो गुहावासी गुहो माली तरङ्गवित्}


\twolineshloka
{त्रिदशस्त्रिकालधृक्कर्मसर्वबन्धविमोचनः}
{बन्धनस्त्वसुरेन्द्राणां युधि शत्रुविनाशनः}


\twolineshloka
{साङ्ख्यप्रसादो दुर्वासाः सर्वसाधुनिषेवितः}
{प्रस्कन्दनो विभागज्ञो अतुल्यो यज्ञभागवित्}


\twolineshloka
{सर्ववासः सर्वचारी दुर्वासा वासवोऽमरः}
{हैमो हेमकरो यज्ञः सर्वंधारी धरोत्तमः}


\twolineshloka
{लोहिताक्षो महाक्षश्च विजयाक्षो विशारदः}
{सङ्ग्रहो निग्रहः कर्ता सर्पचीरनिवासनः}


\twolineshloka
{मुख्योऽमुख्यश्च देहश्च काहलिः सर्वकामदः}
{सर्वकासप्रसादश्च सुबलो बलरूपधृत्}


\twolineshloka
{सर्वकामवरश्चैव सर्वदः सर्वतोमुखः}
{आकाशनिर्विरूपश्च निपाती ह्यवशः खगः}


\twolineshloka
{रौद्ररूपोंऽशुरादित्यो बहुरश्मिः सुवर्चसी}
{वसुवेगो महावेगो मनोवेगो निशाचरः}


\twolineshloka
{सर्ववासी श्रियावासी उपदेशकरोऽकरः}
{मुनिरात्मनिरालोकः सम्भग्नश्च सहस्रदः}


\twolineshloka
{पक्षी च पक्षरूपश्च अतिदीप्तो विशाम्पतिः}
{उन्मादो मदनः कामो ह्यश्वत्थोऽर्थकरो यशः}


\twolineshloka
{वामदेवश्च वामश्च प्राग्दक्षिणश्च वामनः}
{सिद्धयोगी महर्षिश्च सिद्धार्थः सिद्धसाधकः}


\twolineshloka
{भिक्षुश्च भिक्षुरूपश्च विपणो मृदुरव्ययः}
{महासेनो विशाखश्च षष्टिभागो गवांपतिः}


\twolineshloka
{वज्रहस्तश्च विष्कम्भी चमूस्तम्भन एव च}
{वृत्तावृत्तकरस्तालो मधुर्मधुकलोचनः}


\twolineshloka
{वाचस्पत्यो वाजसनो नित्यमाश्रमपूजितः}
{ब्रह्मचारी लोकचारी सर्वचारी विचारवित्}


\twolineshloka
{ईशान ईश्वरः कालो निशाचारी पिनाकवान्}
{निमित्तस्थो निमित्तं च नन्दिर्नन्दिकरो हरिः}


\twolineshloka
{नन्दीश्वरश्च नन्दी च नन्दनो नन्दिवर्धनः}
{भगहारी निहन्ता च कालो ब्रह्मा पितामहः}


\twolineshloka
{चतुर्मुखो महालिङ्गश्चारुलिङ्गस्तथैव च}
{लिङ्गाध्यक्षः सुराध्यक्षो योगाध्यक्षो युगावहः}


\twolineshloka
{बीजाध्यक्षो बीजकर्ता अव्यात्माऽनुगतो बलः}
{इतिहासः सकल्पश्च गौतमोऽथ निशाकरः}


\twolineshloka
{दम्भो ह्यदम्भो वैदम्भो वश्यो वशकरः कलिः}
{लोककर्ता पशुपतिर्महाकर्ता ह्यनौषधः}


\twolineshloka
{अक्षरं परमं ब्रह्मि बलवच्छक्र एव च}
{नीतिर्ह्यनीतिः शुद्धात्मा शुद्धो मान्यो गतागतः}


\twolineshloka
{बहुप्रसादः सुस्वप्नो दर्पणोऽथ त्वभित्रजित्}
{वेदकारो मन्त्रकारो विद्वान्समरमर्दनः}


\twolineshloka
{महामेघनिवासी च महाघोरो वशीकरः}
{अग्निज्वालो महाज्वालो अतिधूम्रो हुतो हिवः}


\twolineshloka
{वृषणः शंकरो नित्यं वर्चस्वी धूमकेतनः}
{नीलस्तथाङ्गलुब्धश्च शोभनो निरवग्रहः}


\twolineshloka
{स्वस्तिदः स्वस्तिभावश्च भागी भागकरो लघुः}
{उत्सङ्गश्च महाङ्गश्चि महागर्भपरायणः}


\twolineshloka
{कृष्णवर्णः सुवर्णश्च इन्द्रियं सर्वदेहिनाम्}
{महापादो महाहस्तो महाकायो महायशाः}


\twolineshloka
{महामूर्धा महाभात्रो महानेत्रो निशालयः}
{महान्तको महाकर्णो महोष्ठश्चि महाहनुः}


\twolineshloka
{महानासो महाकम्बुर्महाग्रीवः श्मशानभाक्}
{महावक्षा महोरस्को ह्यन्तरात्मा मृगालयः}


\twolineshloka
{लम्बनो लम्बितोष्ठश्च महामायः पयोनिधिः}
{महादन्तो महादंष्ट्रो महाजिह्वो महामुखः}


\twolineshloka
{महानखो महारोमा महाकेशो महाजटः}
{प्रसन्नश्च प्रसादश्च प्रत्ययो गिरिसाधनः}


\twolineshloka
{स्नेहनोऽस्नेहनश्चैव अजितश्च महामुनिः}
{वृक्षाकारो वृक्षकेतुरनलो वायुवाहनः}


\twolineshloka
{गण्डली मेरुधामा च देवाधिपतिरेव च}
{अथर्वशीर्षः सामास्य ऋक्सहस्रामितेक्षणः}


\twolineshloka
{यजुःपादभुजो गुह्यः प्रकाशो जङ्गमस्तथा}
{अमोघार्थः प्रसादश्च अभिगम्यः सुदर्शनः}


\twolineshloka
{उपकारः प्रियः सर्वः कनकः काञ्चनच्छविः}
{नाभिर्नन्दिकरो भावः पुष्करस्थपतिः स्थिरः}


\twolineshloka
{द्वादशस्त्रासनश्चाद्यो यज्ञो यज्ञसमाहितः}
{नक्तं कलिश्च कालश्च मकरः कालपूजितः}


\twolineshloka
{सगणो गणकारश्च भूतवाहनसारथिः}
{भस्मशयो भस्मगोप्ता भस्मभूतस्तरुर्गणः}


\twolineshloka
{लोकपालस्तथा लोको महात्मा सर्वपूजितः}
{शुक्लस्त्रिशुक्लः सम्पन्नः शुचिर्भूतनिषेवितः}


\twolineshloka
{आश्रमस्थः क्रियादस्थो विश्वकर्ममतिर्वरः}
{विशालशाखस्ताम्रोष्ठो ह्यम्बुजालः सुनिश्चलः}


\twolineshloka
{कपिलः कपिशः शुक्ल आयुश्चैवि परोऽपरः}
{गन्धर्वो ह्यदितिस्तार्क्ष्यः सुविज्ञेयः सुशारदः}


\twolineshloka
{परश्वधायुधो देव अनुकारी सुबान्धवः}
{तुम्बवीणो महाक्रोध ऊर्ध्वरेता जलेशयः}


\twolineshloka
{उग्रो वंशकरो वंशो वंशनादो ह्यनिन्दितः}
{सर्वाङ्गरूपो मायावी सुहृदो ह्यनिलोऽनलः}


\twolineshloka
{बन्धनो बन्धकर्ता च सुबन्धनविमोचनः}
{स यज्ञारिः स कामारिर्महादंष्ट्रो महायुधः}


\twolineshloka
{बहुधा निन्दितः शर्वः शङ्करः शङ्करोऽधनः}
{अमरेशो महादेवी विश्वदेवः सुरारिहा}


\twolineshloka
{अहिर्बुध्न्योऽनिलाभश्च चेकितानो हविस्तथा}
{अजैकपाच्च कापाली त्रिशङ्कुरजितः शिवः}


\twolineshloka
{धन्वन्तरिर्धूमकेतुः स्कन्दो वैश्रवणस्तथा}
{धाता शक्रश्च विष्णुश्च मित्रस्त्वष्टा ध्रुवो धरः}


\twolineshloka
{प्रभावः सर्वगो वायुरर्यमा सविता रविः}
{उषङ्गुश्च विधाता च मांधाता भूतभावनः}


\twolineshloka
{विभुर्वर्णविभावी च सर्वकामगुणावहः}
{पद्मनाभो महागर्भश्चन्द्रवक्त्रोऽनिलोऽनलः}


\twolineshloka
{बलवांश्चोपशान्तश्च पुराणः पुण्यचञ्चुरी}
{कुरुकर्ता कुरुवासी कुरुभूतो गुणौषधः}


\twolineshloka
{सर्वाशयो दर्भचारी सर्वेषां प्राणिनां पतिः}
{देवदेवः सुखासक्तः सदसत्सर्वरत्नवित्}


\twolineshloka
{कैलासगिरिवासी च हिमवद्भिरिसंश्रयः}
{कूलहारी कूलकर्ता बहुविद्यो बहुप्रदः}


\twolineshloka
{वणिजो वर्धकी वृक्षो बकुलश्चन्दनश्छदः}
{सारग्रीवो महाजत्रुरलोलश्च महौषधः}


\twolineshloka
{सिद्धार्थकारी सिद्धार्थश्छदो व्याकरणोत्तरः}
{सिंहनादः सिंहदंष्ट्रः सिंहगः सिंहवाहनः}


\twolineshloka
{प्रभावात्मा जगत्कालस्थालो लोकहितस्तरुः}
{सारङ्गो नवचक्राङ्गः केतुमाली सभावनः}


% Check verse!
भूतालयो भूतपतिरहोरात्रमनिन्दितः
\twolineshloka
{वाहिता सर्वभूतानां निलयश्च विभुर्भवः}
{अमोघः संयतो ह्यश्वो भोजनः प्राणधारणः}


\threelineshloka
{धृतिमान्मतिमान्दक्षः सत्कृतश्च युगाधिपः}
{गोपालिर्गोपतिर्ग्रामो गोचर्मवसनो हरिः}
{}


\twolineshloka
{हिरण्यबाहुश्च तता गुहापालः प्रवेशिनाम्}
{प्रकृष्टारिर्महाहर्षो जितकामो जितेन्द्रियः}


\twolineshloka
{गान्धारश्च सुवासस्च तपःसक्तो रतिर्नरः}
{महागीतो महानृत्यो ह्यप्सरोगणसेवितः}


\twolineshloka
{महाकेतुर्महाधातुर्नैकसानुचरश्चलः}
{आवेदनीय आदेशः सर्वगन्धसुखावहः}


\twolineshloka
{तोरणस्तारणो वातः वरिधी पतिखेचरः}
{संयोगो वर्धनो वृद्धो अतिवृद्धो गुणाधिकः}


\twolineshloka
{नित्य आत्मसहायश्च देवासुरपतिः पतिः}
{युक्तश्च युक्तबाहुश्च देवो दिवि सुपर्वणः}


\twolineshloka
{आषाढश्च सुषांढश्च ध्रुवोऽथ हरिणो हरः}
{वपुरावर्तमानेभ्यो वसुश्रेष्ठो महापथः}


\twolineshloka
{शिरोहारी विमर्शश्च सर्वलक्षणलक्षितः}
{अक्षश्च रथयोगी च सर्वयोगी महाबलः}


\twolineshloka
{समाम्नायोऽसमाम्नायस्तीर्थदेवो महारथः}
{निर्जीवो जीवनो मन्त्रः शुभाक्षो बहुकर्कशः}


\twolineshloka
{रत्नप्रभूतो रत्नाङ्गो महार्णवनिपानवित्}
{मूलं विशालो ह्यमृतो व्यक्ताव्यक्तस्तपोनिधिः}


\twolineshloka
{आरोहणोऽधिरोहश्च शीलधारी महायशाः}
{सेनाकल्पो महाकल्पो योगो युगकरो हरिः}


\twolineshloka
{युगरूपो महारूपो महानागहनो वधः}
{न्यायनिर्वपणः पादः पण्डितो ह्यचलोपमः}


\twolineshloka
{बहुमालो महामालः शशी हरसुलोचनः}
{विस्तारो लवणः कूपस्त्रियुगः सफलोदयः}


\twolineshloka
{त्रिलोचनो विषष्णाङ्गो मणिविद्धो जटाधरः}
{बिन्दुर्विसर्गः सुमुखः शरः सर्वायुधः सहः}


\twolineshloka
{निवेदनः सुखाजातः सुगन्धारो महाधनुः}
{गन्धपाली च भगवानुत्थानः सर्वकर्मणाम्}


\twolineshloka
{मन्थानो बहुलो वायुः सकलः सर्वलोचनः}
{तलस्तालः करस्थाली ऊर्ध्वसंहननो महान्}


\threelineshloka
{धत्रं सुच्छत्रो विख्यातो लोकः सर्वाश्रयः क्रमः}
{मुण्डो विरूपो विकृतो दण्डी कुण्डी विकुर्वणः}
{}


\twolineshloka
{हर्यक्षः ककुभो वज्रो शतजिह्वः सहस्रपात्}
{सहस्रमूर्धा देवेन्द्रः सर्वदेवमयो गुरुः}


\threelineshloka
{सहस्रबाहुः सर्वाङ्गः शरण्यः सर्वलोककृत्}
{पवित्रं त्रिककुन्मन्त्रः कनिष्ठः कृष्णपिङ्गलः}
{}


\twolineshloka
{ब्रह्मदण्डविनिर्माता शतघ्नीपाशशक्तिमान्}
{पद्मगर्भो महागर्भो ब्रह्मगर्भो जलोद्भवः}


\twolineshloka
{गभस्तिर्ब्रह्मकृद्ब्रह्मी ब्रह्मविद्ब्राह्मणो गतिः}
{अनन्तरुपो नैकात्मा तिग्मतेजाः स्वयंभुवः}


\twolineshloka
{ऊर्ध्वगात्मा पशुपतिर्वातरंहा मनोजवः}
{चन्दनी पद्मनालाग्रः सुरभ्युत्तरणो नरः}


\twolineshloka
{कर्णिकारमहास्रग्वी नीलमौलिः पिनाकधृत्}
{उमापतिरुमाकान्तो जाह्नवीधृगुमाधवः}


\twolineshloka
{वरो वराहो वरदो वरेण्यः सुमाहास्वनः}
{महाप्रसादो दमनः शत्रुहा श्वेतपिङ्गलः}


\twolineshloka
{पीतात्मा परमात्मा च प्रयतात्मा प्रधानधृत्}
{सर्वपार्श्वमुखस्त्र्यक्षो धर्मसाधारणो वरः}


\twolineshloka
{चराचरात्मा सूक्ष्मात्मा अमृतो गोवृषेश्वरः}
{साध्यर्षिर्वसुरादित्यो विवस्वान्सविताऽमृतः}


\twolineshloka
{व्यासः सर्गः सुसंक्षेपो विस्तरः पर्ययो नरः}
{ऋतु संवत्सरो मासः पक्षः सङ्ख्यासमापनः}


\twolineshloka
{कला काष्ठा लवा मात्रा मुहूर्ताहःक्षपाः क्षणाः}
{विश्वक्षेत्रं प्रजाबीजं लिङ्गमाद्यस्तु निर्गमः}


\twolineshloka
{सदसद्व्यक्तमव्यक्तं पिता माता पितामहः}
{स्वर्गद्वारं प्रजाद्वारं मोक्षद्वारं त्रिविष्टपम्}


\twolineshloka
{निर्वाणं ह्लादनश्चैव ब्रह्मलोक परा गतिः}
{देवासुरविनिर्माता देवासुरपरायणः}


\twolineshloka
{देवासुरगुरुर्देवो देवासुरनमस्कृतः}
{देवासुरमहामात्रो देवासुगणाश्रयः}


\twolineshloka
{देवासुरगणाध्यक्षो देवासुरगणाग्रणीः}
{देवातिदेवो देवर्षिर्देवासुरवरप्रदः}


\twolineshloka
{देवासुरेश्वरो विश्वो देवासुरमहेश्वरः}
{सर्वदेवमयोचिन्त्यो देवतात्माऽत्मसम्भवः}


% Check verse!
उद्भित्त्रिविक्रमो वैद्यो विरजो नीरजोऽमरः ॥ईड्यो हस्तीश्वरो व्याघ्रो देवसिंहो नरर्षभः
\twolineshloka
{विबुधोऽग्रवरः सूक्ष्मः सर्वदेवस्तपोमयः}
{सुयुक्तः शोभनो वज्री प्रासानां प्रभवोऽव्ययः}


\twolineshloka
{गुहः कान्तो निजः सर्गः पवित्रं सर्वपावनः}
{शृङ्गी शृङ्गप्रियो बभ्रू राजराजो निरामयः}


\twolineshloka
{अभिरामः सुरगणो विरामः सर्वसाधः}
{ललाटाक्षो विश्वदेवो हरिणो ब्रह्मवर्चसः}


\twolineshloka
{स्थावराणां पतिश्चैव नियमन्द्रियवर्धनः}
{सिद्धार्थः सिद्धभूतार्थोऽचिन्त्यः सत्यव्रतः शुचिः}


\twolineshloka
{व्रताधिपः परं ब्रह्म भक्तानां परमा गतिः}
{विमुक्तो मुक्ततेजाश्च श्रीमाञ्श्रीवर्धनो जगत्}


\threelineshloka
{यथा प्रधानं भगवानिति भक्त्या स्तुतो मया}
{यन्न ब्रह्मादयो देवा विदुस्तत्त्वेन नर्षयः}
{स्तोतव्यमर्च्यं वन्द्यं च कः स्तोष्यति जगत्पतिम्}


\twolineshloka
{भक्तिं त्वेवं पुरस्कृत्य मया यज्ञपतिर्विभुः}
{ततोऽभ्यनुज्ञां सम्प्राप्य स्तुतो मतिमतां वरः}


\twolineshloka
{शिवमेभिः स्तुवन्देवं नामभिः पुष्टिवर्धनैः}
{नित्ययुक्तः शुचिर्भक्तः प्राप्नोत्यात्मानमात्मना}


% Check verse!
एतद्धि परमं ब्रह्म परं ब्रह्माधिगच्छति
% Check verse!
ऋषयश्चैव देवाश्च स्तुवन्त्येतेन तत्परम्
\twolineshloka
{स्तूयमानो महादेवस्तष्यते नियतात्मभिः}
{भक्तानुकम्पी भगवानात्मसंस्थाकरो विभुः}


\twolineshloka
{तथैव च मनुष्येषु ये मनुष्याः प्रधानतः}
{आस्तिकाः श्रद्धधानाश्च बहुभिर्जन्मभिः स्तवैः}


\twolineshloka
{भक्त्या ह्यनन्यमीशानं परं देवं सनातनम्}
{कर्मणा मनसा वाचा भावेनामिततेजसः}


\twolineshloka
{शयाना जाग्रमाणाश्च व्रजन्नुपविशंस्तथा}
{उन्मिषन्निमिषंश्चैव चिन्तयन्तः पुनःपनः}


\twolineshloka
{शृण्वन्तः श्रावयन्तश्च कथयन्तश्च ते भवम्}
{स्तुवन्तः स्तूयामानाश्च तुष्यन्ति च रमन्ति च}


\twolineshloka
{जन्मकोटिसहस्रेषु नानासंसारयोनिषु}
{जन्तोर्विगतपापस्य भवे भक्तिः प्रजायते}


\twolineshloka
{उत्पन्ना च भवे भक्तिरनत्या सर्वभावतः}
{भाविनः कारणे चास्य सर्वयुक्तस्य सर्वथा}


\twolineshloka
{एतद्देवेषु दुष्प्रापं मनुष्येषु न लभ्तते}
{निर्विघ्रा निश्चला रुद्रे भक्तिरव्यभिचारिणी}


\twolineshloka
{तस्यैव च प्रसादेन भक्तिरुत्पद्यते नृणाम्}
{येन यान्ति परां सिद्धिं तद्भागवतचेतसः}


\twolineshloka
{ये सर्वभावानुगताः प्रपद्यन्ते महेश्वरम्}
{प्रपन्नवत्सलो देवः संसारात्तान्समुद्धरेत्}


\twolineshloka
{एवमन्ये विकुर्वन्ति देवाः संसारमोचनम्}
{मनुष्याणामृते देवं नान्या शक्तिस्तपोबलम्}


\twolineshloka
{इति तेनेन्द्रकल्पेन भगवान्सदसत्पतिः}
{कृत्तिवासाः स्तुतः कृष्ण तण्डिना शुभबुद्धिना}


\twolineshloka
{स्तवमेतं भगवतो ब्रह्मा स्वयमधारयत्}
{गीयते च स बुद्ध्येत ब्रह्म शंकरसन्निधौ}


\twolineshloka
{इदं पुण्यं पवित्रं च सर्वदा पापनाशनम्}
{योगदं मोक्षदं चैव स्वर्गदं तोषदं तथा}


\twolineshloka
{एवमेतत्पठन्ते य एकभक्त्या तु शङ्करम्}
{या गतिः साङ्ख्ययोगानां व्रजन्त्येतां गतिं तदा}


\twolineshloka
{स्तवमेतं प्रयत्नेन सदा रुद्रस्य सन्निधौ}
{अब्धमेकं चरेद्भक्तः प्राप्नुयादीप्सितं फलम्}


\twolineshloka
{एतद्रहस्यं परमं ब्रह्मणो हृदि संस्थितम्}
{ब्रह्मा प्रोवाच शक्राय शक्रः प्रोवाच मृत्यवे}


\twolineshloka
{मृत्युः प्रोवाच रुद्रेभ्यो रुद्रेभ्यस्तण्डिमागमत्}
{महता तपसा प्राप्तस्तण्डिना ब्रह्मसद्मानि}


\twolineshloka
{तण्डिः प्रोवाच शुक्राय गौतमाय च भार्गवः}
{वैवस्वताय मनवे गौतमः प्राह माधव}


\twolineshloka
{नारायणाय साध्याय समाधिष्ठाय धीमते}
{यमाय प्राह भगवान्साध्यो नारायणोच्युतः}


\twolineshloka
{नाचिकेताय भगवानाह वैवस्वतो यमः}
{मार्कण्डेयाय वार्ष्णेय नाचिकेतोऽभ्यभाषत}


\threelineshloka
{मार्कण्डेयान्मय प्राप्तो नियमेन जनार्दन}
{तवाप्यहममित्रघ्न स्तवं दद्यां ह्यविश्रुतम्}
{}


\twolineshloka
{स्वर्ग्यमारोग्यमायुष्यं धन्यं वेदेनि संमितम्}
{नास्य विघ्रं विकुर्वन्ति दानवा यक्षराक्षसाः}


\threelineshloka
{पिशाचा यातुधाना वा गुह्यका भुजगा अपि}
{यः पठेत शुचिः पार्थ ब्रह्मचारी जितेन्द्रियः}
{अभग्रयोगो वर्षं तु सोऽश्वमेधफलं लभेत्}


\chapter{अध्यायः ४९}
\twolineshloka
{महायोगी तु तं प्राह कृष्णद्वैपायनो मुनिः}
{पठस्व पुत्र भद्रं ते प्रीयतां ते महेश्वरः}


\twolineshloka
{पुरा पुत्र मया मेरौ तप्यता परमं तपः}
{पुत्रहेतोर्महाराज स्तव एषोऽनुकीर्तितः}


\twolineshloka
{लब्धवानीप्सितं काममहं वै पाण्डुनन्दन}
{तथा त्वमपि शर्वाद्धि सर्वान्कामानवाप्स्यसि}


\threelineshloka
{कपिलश्च ततः प्राह साङ्ख्यर्षिर्देवसम्मतः}
{मया जन्मान्यनेकानि भक्त्या चाराधितो भवः}
{प्रीतश्च भगवाञ्ज्ञानं ददौ मम भवान्तकम्}


\twolineshloka
{चारुशीर्षस्ततः प्राह शक्रस्य दयितः सखा}
{आलम्बायन इत्येवं विश्रुतः करुणात्मकः}


\twolineshloka
{मया गोकर्णमासाद्य तपस्तप्त्वा शतं समाः}
{अयोनिजानां दान्तानां धर्मज्ञानां सुवर्चसाम्}


\twolineshloka
{अजराणामदुःखानां शतवर्षसहस्रिणाम्}
{लब्धं पुत्र शतं शर्वात्पुरा पाण्डुनृपात्मज}


\twolineshloka
{वाल्मीकिश्चाह भगवान्युधिष्ठिरमिदं वचः}
{विवादे साग्निमुनिभिर्ब्रह्मघ्नो वै भवानिति}


\twolineshloka
{उक्तः क्षणेन चाविष्टस्तेनाधर्मेण भारत}
{सोऽहमीशानमनघममोघं शरणं गतः}


\twolineshloka
{मुक्तश्चास्मि ततः पापैस्ततो दुःखविनाशनः}
{आह मां त्रिपुरघ्नो वै यशस्तेऽग्र्यं भविष्यति}


\twolineshloka
{जामदग्न्यश्च कौन्तेयमिदं धर्मभृतांवरः}
{ऋषिमध्ये स्थितः प्राह ज्वलन्निव दिवाकरः}


\twolineshloka
{पितृविप्रवधेनाहमार्तो वै पाण्डवाग्रज}
{शुचिर्भूत्वा महादेवं गतोस्मि शरणं नृप}


\twolineshloka
{नामभिश्चास्तुवं देवं ततस्तुष्टोऽभवद्भवः}
{परशुं च ततो देवो दिव्यान्यस्त्राणि चैव मे}


\twolineshloka
{पापं च ते न भविता अजेयश्च भविष्यसि}
{न ते प्रभविता मृत्युरजरश्च भविष्यसि}


\twolineshloka
{आह मां भगवानेवं शिखण्डी शिवविग्रहः}
{तदवाप्तं च मे सर्वं प्रसादात्तस्य धीमतः}


\threelineshloka
{विश्वामित्रस्तदोवाच क्षत्रियोऽहं तदाऽभवम्}
{ब्राह्मणोऽहं भवानीति मया चाराधितो भवः}
{तत्प्रसादान्मया प्राप्तं ब्राह्मण्यं दुर्लभं महत्}


% Check verse!
असितो देवलश्चैव प्राह पाण्डुसुतं नृपम्
\twolineshloka
{शापाच्छक्रस्य कौन्तेय विभो धर्मोऽनशत्तदा}
{तन्मे धर्मं यशश्चाग्र्यमायुश्चैवाददत्प्रभुः}


\twolineshloka
{ऋषिर्गृत्समदो नाम शक्रस्य दयितः सखा}
{प्राहाजमीढं भगवान्बृहस्पतिसमद्युतिः}


\twolineshloka
{वरिष्ठो नाम भगवांश्चाक्षुषस्य मनोः सुतः}
{शतक्रतोरचिन्त्यस्य सत्रे वर्षसहस्रिके}


\twolineshloka
{वर्तमानेऽब्रवीद्वाक्यं साम्नि ह्युच्चारिते मया}
{रथन्तरे द्विजश्रेष्ठ न सम्यगिति वर्तते}


\twolineshloka
{समीक्षस्व पुनर्बुद्ध्या पापं त्यक्त्वा द्विजोत्तम}
{अयज्ञवाहिनं पापमकार्षीस्त्वं सुदुर्मते}


\twolineshloka
{एवमुक्त्वा महाक्रोधः प्राह शम्भुं पुनर्वचः}
{प्रज्ञया रहितो दुःखी नित्यभीतो वनेचरः}


\twolineshloka
{दशवर्षसहस्राणि दशाष्टौ च शतानि च}
{नष्टपानीयपवने मृगैरन्यैश्च वर्जिते}


\twolineshloka
{अयज्ञीयद्रुमे देशे रुरुसिंहनिषेविते}
{भविता त्वं मृगः क्रूरो महादुःखसमन्वितः}


\twolineshloka
{तस्य वाक्यस्य निधने पार्थ जातो ह्यहं मृगः}
{ततो मां शरणं प्राप्तं प्राह योगी महेश्वरः}


\twolineshloka
{अजरश्चामरश्चैव भविता दुखवर्जितः}
{साम्यं ममास्तु ते सौख्यं युवयोर्वर्धतां क्रतुः}


\twolineshloka
{अनुग्रहानेवमेष करोति भगवान्विभुः}
{अयं धाता विधाता च सुखदुःखे च सर्वदा}


\twolineshloka
{अचिन्त्य एष भगवान्कर्मणा मनसा गिरा}
{न मे तात युधिश्रेष्ठ विद्यया पण्डितः समः}


\twolineshloka
{वासुदेवस्तदोवाच पुनर्मतिमतांवरः}
{सुवर्णाक्षो महादेवस्तपसा तोषितो मया}


\twolineshloka
{ततोऽथ भगवानाह प्रीतो मा वै युधिष्ठिर}
{अर्थात्प्रियतरः कृष्ण मत्प्रसादाद्भविष्यसि}


\twolineshloka
{अपराजितश्च युद्धेषु तेजश्चैवानलोपमम्}
{एवं सहस्रशश्चान्यान्महादेवो वरं ददौ}


\twolineshloka
{मणिमन्थेऽथ शैले वै पुरा सम्पूजितो मया}
{वर्षायुतासहस्राणां सहस्रं शतमेव च}


\twolineshloka
{ततो मां भगवान्प्रीत इदं वचनमब्रवीत्}
{वरं वृणीष्व भद्रं ते यस्ते मनसि वर्तते}


\twolineshloka
{ततः प्रणम्य शिरसा इदं वचनमब्रवम्}
{यदि प्रीतो महादेवो भक्त्या परमया प्रभुः}


\threelineshloka
{नित्यकालं तवेशान भक्तिर्भवतु मे स्थिरा}
{एवमस्त्विति भगवांस्तत्रोक्त्वान्तरधीयत ॥जैगीषव्य उवाच}
{}


\threelineshloka
{ममाष्टगुणमैश्वर्यं दत्तं भगवता पुरा}
{यत्नेनान्येन बलिना वाराणस्यां युधिष्ठिर ॥गर्ग उवाच}
{}


\twolineshloka
{चतुःषष्ट्यङ्गमददत्कलाज्ञानं ममाद्भुतम्}
{सरस्वत्यास्तटे तुष्टो मनोयज्ञेन पाण्डव}


\threelineshloka
{तुल्यं मम सहस्रं तु सुतानां ब्रह्मवादिनाम्}
{आयुश्चैव सपुत्रस्य संवत्सरशतायुतम् ॥पराशर उवाच}
{}


\twolineshloka
{प्रसाद्येह पुरा शर्वं मनसाऽचिन्तयं नृप}
{महातपा महातेजा महायोगी महायशाः}


\twolineshloka
{वेदव्यासः श्रियावासो ब्राह्मणः करुणान्वितः}
{अप्यसावीप्सितः पुत्रो मम स्याद्वै महेश्वरात्}


\twolineshloka
{इति मत्वा हृदि मतं प्राह मां सुरसत्तमः}
{मयि सम्भावना यास्याः फलात्कृष्णो भविष्यति}


\twolineshloka
{सावर्णस्य मनोः सर्गे सप्तर्षिश्च भविष्यति}
{वेदानां च स वै वक्ता कुरुवंशकरस्तथा}


\twolineshloka
{इतिहासस्य कर्ता च पुत्रस्ते जगतो हितः}
{भविष्यति महेन्द्रस्य दयितः स महामुनिः}


\fourlineindentedshloka
{अजरश्चामरश्चैव पराशर सुतस्तव}
{एवमुक्त्वा स भगवांस्तत्रैवान्तरधीयत}
{युधिष्ठिर महायोगी वीर्यवानक्षयोऽव्ययः ॥माण्डव्य उवाच}
{}


% Check verse!
अचोरश्चोरशङ्कायां शूले भिन्नो ह्यहं तदा
\twolineshloka
{तत्रस्थेन स्तुतो देवः प्राह मां वै नरेश्वर}
{मोक्षं प्राप्स्यसि शूलाच्च जीविष्यसि समार्बुदम्}


\twolineshloka
{रुजा शूलकृता चैव न ते विप्र भविष्यति}
{आधिभिर्व्याधिभिश्चैव वर्जितस्त्वं भविष्यति}


\twolineshloka
{पादाच्चतुर्थात्सम्भूत आत्मा यस्मान्मुने तव}
{त्वं भविष्यस्यनुपमो जन्म वै सफलं कुरु}


\twolineshloka
{तीर्थाभिषेकं सकलं त्वमविघ्नेन चाप्स्यसि}
{स्वर्गं चैवाक्षयं विप्र विदधामि तवोर्जितम्}


\fourlineindentedshloka
{एवमुक्त्वा तु भगवान्वरेण्यो वृषवाहनः}
{महेश्वरो महाराजः कृत्तिवासा महाद्युतिः}
{सगणो दैवतश्रेष्ठस्तत्रैवान्तरधीयत ॥गालव उवाच}
{}


\twolineshloka
{विश्वामित्राभ्यनुज्ञातो ह्यहं पितरमागतः}
{अब्रवीन्मां ततो माता दुःखिता रुदती भृशम्}


\twolineshloka
{कौशिकेनाभ्यनुज्ञातं पुत्रं देवविभूषितम्}
{न तात तरुणं दान्तं पिता त्वां पश्यतेऽनघ}


\twolineshloka
{श्रुत्वा जनन्या वचनं निराशो गुरुदर्शने}
{नियतात्मा महादेवमपश्यं सोऽब्रवीच्च माम्}


\twolineshloka
{पिता माता च ते त्वं च पुत्र मृत्युविवर्जिताः}
{भविष्यथ विश क्षिप्रं द्रष्टासि पितरं क्षये}


\threelineshloka
{अनुज्ञातो भगवता गृहं गत्वा युधिष्ठिर}
{अपश्यं पितरं तात इष्टिं कृत्वा विनिःसृतम्}
{उपस्पृश्य गृहीत्वेध्मं कुशांश्च शरणाकुरून्}


\twolineshloka
{तान्विसृज्य च मां प्राह पिता सास्राविलेक्षणः}
{प्रणमन्तं परिष्वज्य मूर्ध्न्युपाघ्राय पाण्डव}


\twolineshloka
{दिष्ट्या दृष्टोसि मे पुत्र कृतविद्य इहागत ॥वैशम्पायन उवाच}
{}


\twolineshloka
{एतान्यत्यद्भुतान्येव कर्माण्यथ महात्मनः}
{प्रोक्तानि मुनिभिः श्रुत्वा विस्मयामास पाण्डवः}


\twolineshloka
{ततः कृष्णोऽब्रवीद्वाक्यं पुनर्मतिमतांवरः}
{युधिष्ठिरं धर्मनिधिं पुरुहूतमिवेश्वरः}


\twolineshloka
{उपमन्युर्मयि प्राह तपन्निव दिवाकरः}
{अशुभैः पापकर्माणो ये नराः कलुषीकृताः}


\twolineshloka
{ईशानं न प्रपद्यन्ते तमोराजसवृत्तयः}
{}


% Check verse!
ईश्वरं सम्प्रपद्यन्ते द्विजा भावितभावनाः
\twolineshloka
{सर्वथा वर्तमानोपि यो भक्तः परमेश्वरे}
{सदृशोऽरण्यवासीनां मुनीनां भावितात्मनाम्}


\twolineshloka
{ब्रह्मत्वं केशवत्वं वा शक्रत्वं वा सुरैः सह}
{त्रैलोक्यस्याधिपत्यं वा तुष्टो रुद्रः प्रयच्छति}


\twolineshloka
{मनसाऽपि शिवं तात ये प्रपद्यन्ति मानवाः}
{विधूय सर्वपापानि देवैः सह वसन्ति ते}


\twolineshloka
{भित्त्वाभित्त्वा च कूलानि हुत्वा सर्वमिदं जगत्}
{जयेद्देवं विरूपाक्षं न स पापेन लिप्यते}


\twolineshloka
{सर्वलक्षणहीनोपि युक्तो वा सर्वपातकैः}
{सर्वं तुदति तत्पापं भावयञ्शिवमात्मना}


\twolineshloka
{कीटपक्षिपतङ्गानां तिरश्चामपि केशव}
{महादेवप्रपन्नानां न भयं विद्यते क्वचित्}


\fourlineindentedshloka
{एवमेव महादेवं भक्ता ये मानवा भुवि}
{न ते संसारवशगा इति मे निश्चिता मतिः}
{ततः कृष्णोऽब्रवीद्वाक्यं धर्मपुत्रं युधिष्ठिरम् ॥विष्णुरुवाच}
{}


\threelineshloka
{आदित्यचन्द्रावनिलानलौ चद्यौर्भूमिरापो वसवोऽथ विश्वे}
{धाताऽर्यमा शुक्रबृहस्पती चवेदा यज्ञा दक्षिणा वेदवाहाः}
{}


% Check verse!
सोमो यष्टा यच्च हव्यं हविश्चरक्षा दीक्षा संयमा ये च केचित्
\twolineshloka
{स्वाहा वौषट् ब्राह्मणाः सौरभेयीधर्मं चाग्र्यं कालचक्रं बलं च}
{यशो दमो बुद्धिमतां स्थितिश्चशुभाशुभं ये मुनयश्च सप्त}


\twolineshloka
{अग्र्या बुद्धिर्मनसा दर्शने चस्पर्शश्चाग्र्यः कर्मणां या च सिद्धिः}
{गणा देवानामूष्मपाः सोमपाश्चलेखाः सुयामास्तुषिता ब्रह्मकायाः}


\twolineshloka
{आभासुरा गन्धपा धूमपाश्चवाचा विरुद्धाश्च मनोविरुद्धाः}
{शुद्धाश्च निर्माणरताश्च देवाःस्पर्शाशना दर्शपा आज्यपाश्च}


\twolineshloka
{चिन्त्यद्योता ये च देवेषु मुख्याये चाप्यन्ते देवताश्चाजमीढ}
{सुपर्णगन्धर्वपिशाचदानवायक्षास्तथा चारणपन्नगाश्च}


\twolineshloka
{स्थूलं सूक्ष्मं मृदु चाप्यसूक्ष्मंदुःखं सुखं दुःखमनन्तरं च}
{साङ्ख्यं योगं तत्पराणां परं चशर्वाञ्जातं विद्धि य***********}


\twolineshloka
{तत्सम्भूता भूतकृतो वरेण्यासर्वे देवा भुवनस्यास्य गोपाः}
{आविश्येमां धरणीं येऽभ्यरक्षन्पुरातनीं तस्य देवस्य सृष्टिम्}


\twolineshloka
{विचिन्वन्तस्तपसा तत्स्थवीयःकिञ्चित्तत्त्वं प्राणहेतोर्नतोस्मि}
{ददातु वेदः स वरानिहेष्टा-नभिष्टुतोः नः प्रभुरव्ययः सदा}


\twolineshloka
{इमं स्तवं सन्नियतेन्द्रियश्चभूत्वा शुचिर्यः पुरुषः पठेत}
{अभग्नयोगो नियतो मासमेकंसम्प्राप्नुयादश्वमेधे फलं यत्}


\twolineshloka
{वेदान्कृत्स्नान्ब्राह्मणः प्राप्नुयात्तुजयेन्नृपः पार्थ महीं च कृत्स्नाम्}
{वैश्यो लाभं प्राप्नुयान्नैपुणं चशूद्रो गतिं प्रेत्य तथा सुखं च}


\twolineshloka
{स्तवराजमिमं कृत्वा रुद्राय दधिरे मनः}
{सर्वदोषापहं पुण्यं पवित्रं च यशस्विनः}


\twolineshloka
{यावन्त्यस्य शरीरेषु रोमकूपाणि भारतः}
{तावन्त्यब्दसहस्राणि स्वर्गे वसति मानवः}


\chapter{अध्यायः ५०}
\twolineshloka
{यदिदं सहधर्मेति प्रोच्यते भरतर्षभ}
{पाणिग्रहणकाले तु स्त्रीणामेतत्कथं स्मृतम्}


\twolineshloka
{आर्ष एष भवेद्धर्मः प्राजापत्योऽथवाऽसुरः}
{यदेतत्सहधर्मेति पूर्वमुक्तं महर्षिभिः}


\twolineshloka
{सन्देहः सुमहानेष विरुद्ध इति मे मतिः}
{इह यः सहधर्मो वै प्रेत्यायं विहितः क्वनु}


\twolineshloka
{स्वर्गो मृतानां भवति सहधर्मः पितामह}
{पूर्वमेकस्तु म्रिय********कस्तिष्ठते वद}


\twolineshloka
{नानाधर्मफलोपेता नानाकर्मनिवासिताः}
{नानानिरयनिष्ठान्ता मानुपा बहवो यदा}


\twolineshloka
{अनृताः स्त्रिय इत्येवं सूत्रकारो व्यवस्यति}
{यदाऽनृताः स्त्रियस्तात सहधर्मः कुतः स्मृतः}


\twolineshloka
{अनृताः स्त्रिय इत्येवं वेदेष्वपि हि पठ्यते}
{धर्मो यः पूर्विको दृष्ट उपचारः क्रियाविधिः}


\twolineshloka
{गहरं प्रतिभात्येतन्मम चिन्तयतोऽनिशम्}
{निःसन्देहमिदं सर्वं पितामह यथाश्रुतिः}


\threelineshloka
{यदैतद्यादृशं चैतद्यथा चैतत्प्रवर्तितम्}
{निखिलेन महाप्राज्ञ भवानेतद्ब्रवीतु मे ॥भीष्म उवाच}
{}


\twolineshloka
{अत्राप्युदाहरन्तीममितिहासं पुरातनम्}
{अष्टावक्रस्य संवादं दिशया सह भारत ॥ i}


\twolineshloka
{निर्विष्टुकामस्तु पुरा अष्टावक्रो महातपाः}
{ऋषेरथ वदान्यस्य वव्रे कन्यां महात्मनः}


\twolineshloka
{सुप्रभां नाम वै नाम्ना रूपेणाप्रतिमां भुवि}
{गुणप्रभावशीलेन चारित्रेण च शोभनाम्}


\twolineshloka
{सा तस्यर्षेर्मनो दृष्टा जहार शुभलोचना}
{वनराजी यथा चित्रा वसन्ते कुसुमाञचिता}


% Check verse!
ऋषिस्तमाह देया मे सुता तुभ्यं हि तच्छृणु
\twolineshloka
{`अनन्यस्त्रीजनः प्राज्ञो ह्यप्रवासी प्रियंवदः}
{सुरूपः सम्मतो वीरः शीलवान्भोगभुक्छुचिः}


\twolineshloka
{दारानुमतयज्ञश्च सुनक्षत्रामथोद्वेहेत्}
{सभृत्यः स्वजनोपेत इह प्रेत्य च मोदते}


\twolineshloka
{गच्छ तावद्दिशं पुण्यामुत्तरां द्रक्ष्यसे ततः ॥अष्टावक्र उवाच}
{}


\threelineshloka
{किं द्रष्टव्यं मया तत्र वक्तुमर्हति मे भवान्}
{तथेदानीं मयो कार्यं यथा वक्ष्यति मां भवान् ॥वदान्य उवाच}
{}


\twolineshloka
{धनदं समतिक्रम्य हिमवन्तं च पर्वतम्}
{रुद्रस्यायतनं दृष्ट्वा सिद्धचारणसेवितम्}


\twolineshloka
{संहृष्टैः पार्षदैर्जुष्टं नृत्यद्भिर्विविधाननैः}
{दिव्याङ्गरागैः पैशाचैरन्यैर्नानाविधैः प्रभोः}


\twolineshloka
{पाणितालसुतालैश्च शम्पातालैः समैस्तथा}
{सम्प्रहृष्टैः प्रनृत्यद्भिः शर्वस्तत्र निषेव्यते}


\twolineshloka
{इष्टं किल गिरौ स्थानं तद्दिव्यमिति शुश्रुम}
{नित्यं सन्निहितो देवस्तथा ते पार्षदाः स्मृताः}


\twolineshloka
{तत्र देव्या तपस्तप्तं सङ्करार्थं सुदुश्चरम्}
{अतस्तदिष्टं देवस्य तथोमाया इति श्रुतिः}


% Check verse!
पूर्वे तत्र महापार्श्वे देवस्योत्तरतस्तथा ॥ऋतवः कालरात्रिश्च ये दिव्या ये च मानुषाः
\twolineshloka
{देवं चोपासते सर्वे रूपिणः किल तत्र ह}
{तदतिक्रम्य भवनं त्वया यातव्यमेव हि}


\twolineshloka
{ततो नीलं वनोद्देशं द्रक्ष्यसे मेघसन्निभम्}
{रमणीयं मनोग्राहि तत्र वै द्रक्ष्यसे स्त्रियम्}


\twolineshloka
{तपस्विनीं महाभागां वृद्धां दीक्षामनुष्ठिताम्}
{द्रष्टव्या सा त्वया तत्र सम्पूज्या चैव यत्नतः}


\threelineshloka
{तां दृष्ट्वा विनिवृत्तस्त्वं ततः पाणिं ग्रहीष्यसि}
{यद्येष समयः सर्वः साध्यतां तत्र गम्यताम् ॥अष्टावक्र उवाच}
{}


\threelineshloka
{तथाऽस्तु साधयिष्यामि तत्र यास्याम्यसंशयम्}
{यत्र त्वं वदसे साधो भवान्भवतु सत्यवाक् ॥भीष्म उवाच}
{}


\twolineshloka
{ततोऽगच्छत्स भगवानुत्तरामुत्तरां दिशम्}
{हिमवन्तं गिरिश्रेष्ठं सिद्धचारणसेवितम्}


\twolineshloka
{स गत्वा द्विजशार्दूलो हिमवन्तं महागिरिम्}
{अभ्यगच्छन्नदीं पुण्यां बाहुदां पुण्यदायिनीम्}


\twolineshloka
{अशोके विमले तीर्थे स्नात्वा वै तर्प्य देवताः}
{तत्र वासाय शयने कौशे सुखमुवास ह}


\twolineshloka
{ततो रात्र्यां व्यतीतायां प्रातरुत्थाय स द्विजः}
{स्नात्वा प्रादुश्चकाराग्निं हुत्वा चैवं विधानतः}


\twolineshloka
{रुद्राणीकूपमासाद्य ह्रदे तत्र समाश्वसत्}
{विश्रान्तश्च समुत्थाय कैलासमभितो ययौ}


\twolineshloka
{सोऽपश्यत्काञ्चनद्वारं दीप्यमानमिव श्रिया}
{मन्दाकिनीं च नलिनीं धनदस्य महात्मनः}


\twolineshloka
{अथ ते राक्षसाः सर्वे येऽभिरक्षन्ति पद्मिनीम्}
{प्रत्युत्थिता भगवन्तं माणिभद्रपुरोगमाः}


\twolineshloka
{स तान्प्रत्यर्चयामास राक्षसान्भीमविक्रमान्}
{निवेदयत मां क्षिप्रं धनदायेति चाब्रवीत्}


\twolineshloka
{ते राक्षसास्तथा राजन्भगवन्तमथाब्रुवन्}
{असौ वैश्रवणो राजा स्वयमायाति तेऽन्तिकम्}


\twolineshloka
{विदितो भगवानस्य कार्यमागमनस्य यत्}
{पश्यैनं त्वं महाभागं ज्वलन्तमिव तेजसा}


\twolineshloka
{ततो वैश्रवणोऽभ्येत्य अष्टावक्रमनिन्दितम्}
{विधिवत्कुशलं पृष्ट्वा ततो ब्रह्मर्षिमब्रवीत्}


\twolineshloka
{सुखं प्राप्तो भवान्कच्चित्किंवा मत्तश्चिकीर्षति}
{ब्रूहि सर्वं करिष्यामि यन्मां वक्ष्यसि वै द्विज}


\twolineshloka
{भवनं प्रविश त्वं मे यथाकामं द्विजोत्तम}
{सत्कृतः कृतकार्यश्च भवान्यास्यत्यविघ्नतः}


\twolineshloka
{प्राविशद्भवनं स्वं वै गृहीत्वा तं द्विजोत्तमम्}
{आसनं स्वं ददौ चैव पाद्यमर्घ्यं तथैव च}


\twolineshloka
{अथोपविष्टयोस्तत्र माणिभद्रपुरोगमाः}
{निषेदुस्तत्र कौबेरा यक्षगन्धर्वकिन्नराः}


\twolineshloka
{ततस्तेषां निषण्णानां धनदो वाक्यमब्रवीत्}
{भवच्छन्दं समाज्ञाय नृत्येरन्नप्सरोगणाः}


\twolineshloka
{आतिथ्यं परमं कार्यं शुश्रूषा भवतस्तथा}
{संवर्ततामित्युवाच मुनिर्मधुरया गिरा}


\twolineshloka
{यथोर्वरा मिश्रकेशी रम्भा चैवोर्वशी तथा}
{अलम्बुसा घृताची च चित्रा चित्राङ्गदारुचिः}


\twolineshloka
{मनोहरा सुकेशी च सुमुखी हासिनी प्रभा}
{विद्युता प्रशमी दान्ता विद्योता रतिरेव च}


\twolineshloka
{एताश्चान्याश्च वै बह्व्यः प्रनृत्ताप्सरसः शुभाः}
{अवादयंश्च गन्धर्वा वाद्यानि विविधानि च}


\twolineshloka
{अथ प्रवृत्ते गान्धर्वे दिव्ये ऋषिरुपाविशत्}
{दिव्यं संवत्सरं तत्रारमतैष महातपाः}


\twolineshloka
{ततो वैश्रवणो राजा भगवन्तमुवाच ह}
{साग्रः संवत्सरो यातो विप्रेह तव पश्यतः}


\twolineshloka
{हार्योऽयं विषयो ब्रह्मन्गान्धर्वो नाम नामतः}
{छन्दतो वर्ततां विप्र यथा वदति वा भवान्}


\twolineshloka
{अतिथिः पूजनीयस्त्वमिदं च भवतो गृहम्}
{सर्वमाज्ञाप्यतामाशु परवन्तो वयं त्वयि}


\twolineshloka
{अथ वैश्रवणं प्रीतो भगवान्प्रत्यभाषत}
{अर्चितोस्मि यथान्यायं गमिष्यामि धनेश्वर}


\threelineshloka
{प्रीतोस्मि सदृशं चैव तव सर्वं धनाधिप}
{तव प्रसादाद्भगवन्महर्षेश्च महात्मनः}
{नियोगादद्य यास्यामि वृद्दिमानृद्धिमान्भव}


\twolineshloka
{अथ निष्क्रम्य भगवान्प्रययावुत्तरामुखः}
{`कैलासे सङ्करावासमभिवीक्ष्य प्रणम्य च}


\twolineshloka
{गौरीशं शङ्करं दान्तं शरणागतवत्सलम्}
{गङ्गाधरं गोपतिनं गणावृतमकल्पषम् ॥'}


\twolineshloka
{कैलासं मन्दरं हैमं सर्वाननुचचार ह}
{तानतीत्य महाशैलान्कैरातं स्थानमुत्तमम्}


\twolineshloka
{प्रदक्षिणं तथा चक्रे प्रयतः शिरसा नतः}
{धरणीमवतीर्याथ पूतात्माऽसौ तदाऽभकवत्}


\twolineshloka
{स तं प्रदक्षिणं कृत्वा निर्यातश्चोत्तरामुखः}
{समेन भूमिभागेन ययौ प्रीतिपुरस्कृतः}


\threelineshloka
{ततोऽपरं वनोद्देशं रमणीयमपश्यत}
{सर्वर्तुभिर्मूलफलैः पक्षिभिश्च समन्वितैः}
{रमणीयैर्वनोद्देशैस्तत्रतत्र विभूषितम्}


% Check verse!
तत्राश्रमपदं दिव्यं ददर्श भगवानथ
\twolineshloka
{शैलांश्च विविधाकारान्काञ्चनान्रत्नभूषितान्}
{मणिभूमौ निविष्टाश्च पुष्करिण्यस्तथैव च}


\twolineshloka
{अन्यान्यपि सुरम्याणि ददर्श सुबहून्यथ}
{भृशं तस्य मनो रमे महर्षेर्भावितात्मनः}


\twolineshloka
{स तत्र काञ्चनं दिव्यं सर्वरत्नमयं गृहम्}
{ददर्शाद्भुतसङ्काशं धनदस्य गृहाद्वरम्}


\twolineshloka
{महान्तो यत्र विविधा मणिकाञ्चनपर्वताः}
{विमानानि च रम्याणि रत्नानि विविधानि च}


\twolineshloka
{मन्दारपुष्पैः सङ्कीर्णां तथा मन्दाकिनीं नदीम्}
{स्वयम्प्रभाश्च मणयो वज्रैर्भूमिश्च भूषिता}


\twolineshloka
{नानाविधैश्च भवनैर्विचित्रमणितोरणैः}
{मुक्ताजालविनिक्षिप्तैर्मणिरत्नविभूषितैः}


\twolineshloka
{मनोद्दष्टिहरै रम्यैः सर्वतः संवृतं शुभैः}
{ऋषिभिश्चावृतं तत्र आश्रमं तं मनोहरम्}


\twolineshloka
{ततस्तस्याभवच्चिन्ता कुत्र वासो भवेदिति}
{अथ द्वारं समभितो गत्वा स्थित्वा ततोऽब्रवीत्}


% Check verse!
अतिथिं समनुप्राप्तमभिजानन्तु येऽत्र वै
\twolineshloka
{अथ कन्याः परिवृता गृहात्तस्माद्विनिर्गताः}
{नानारूपाः सप्त विभो कन्याः सर्वा मनोहराः}


\threelineshloka
{यांयामपश्यत्कन्यां वै सासा तस्य मनोऽहरत्}
{न च शक्तो वारयितुं मनोऽस्याथावसीदति}
{ततो धृतिः समुत्पन्ना तस्य विप्रस्य धीमतः}


\threelineshloka
{अथ तं प्रमदाः प्राहुर्भगवान्प्रविशत्विति}
{स च तासां सुरुपेण तस्यैव भवनस्य च}
{कौतूहलं समाविष्टः प्रविवेश गृहं द्विजः}


\twolineshloka
{तत्रापश्यज्जरायुक्तामरजोम्बरधारिणीम्}
{वृद्धां पर्यङ्कमासीनां सर्वाभरणभूषिताम्}


\threelineshloka
{स्वस्तीति तेन चैवोक्ता सा स्त्री प्रत्यवदत्तदा}
{प्रत्युत्थाय च तं विप्रमास्यतामित्युवाच ह ॥अष्टावक्र उवाच}
{}


\twolineshloka
{सर्वाः स्वानालयान्यान्तु एका मामुपतिष्ठतु}
{प्रज्ञाता या प्रशान्ता या शेषा गच्छन्तु च्छन्दतः}


\threelineshloka
{ततः प्रदक्षिणीकृत्य कन्यास्तास्तमृषिं तदा}
{निश्चक्रमुर्गृहात्तस्मात्सा वृद्धाऽथ व्यतिष्ठतः}
{तया सम्पूजितस्तत्र शयने चापि निर्मले}


\twolineshloka
{अथ तां संविशन्प्राह शयने भास्वरे तदा}
{त्वयाऽपि सुप्यतां भद्रे रजनी ह्यतिवर्तते}


\twolineshloka
{संलापात्तेन विप्रेण तथा सा तत्र भाषिता}
{द्वितीये शयने दिव्ये संविवेश महाप्रभे}


\twolineshloka
{अथ सा वेपमानाङ्गी निमित्तं शीतजं तदा}
{व्यपदिश्य महर्षेर्वै शयनं व्यवरोहत}


\twolineshloka
{स्वागतेनागतां तां तु भगवानभ्यभाषत}
{सा जुगूह भुजाभ्यां तु ऋषिं प्रीत्या नरर्षभ}


\twolineshloka
{निर्विकारमृषिं चापि काष्ठकुड्योपमं तदा}
{दुखिता प्रेक्ष्य सञ्जल्पमकार्षीदृषिणा सह}


\twolineshloka
{ब्रह्मन्नकामकरोस्ति स्त्रीणां पुरुषतो धृतिः}
{कामेन मोहिता चाहं त्वां भजन्तीं भजस्व माम्}


\twolineshloka
{प्रहृष्टो भव विप्रर्षे समागच्छ मया सह}
{उपगूह च भां विप्र कामार्ताऽहं भृशं त्वयि}


\twolineshloka
{एतद्वि तव धर्मात्मंस्तपसः पूज्यते फलम्}
{प्रार्थितं दर्शनादेव भजमानां भजस्व माम्}


\twolineshloka
{सद्म चेदं धनं सर्वं यच्चान्यदपि पश्यसि}
{प्रभुस्त्वं भव सर्वत्र मयि चैव न संशयः}


\twolineshloka
{सर्वान्कामान्विधास्यामि रमस्व सहितो मया}
{रमणीये वने विप्र सर्वकामफलप्रदे}


\twolineshloka
{त्वद्वशाऽहं भविष्यामि रंस्यसे च मया सह}
{सर्वान्कामानुपाश्नीमो ये दिव्या ये च मानुषाः}


\twolineshloka
{नातः परं हि नारीणां विद्यते च कदाचन}
{यथा पुरुषसंसर्गः परमेतद्धि नः फलम्}


\threelineshloka
{आत्मच्छन्देन वर्तन्ते नार्यो मन्मथचोदिताः}
{न च दह्यन्ति गच्छन्त्यः सुतप्तैरपि पांसुभिः ॥अष्टावक्र उवाच}
{}


\twolineshloka
{परदारानहं भद्रे न गच्छेयं कथञ्चन}
{दूषितं धर्मशास्त्रज्ञैः परदाराभिमर्शनम्}


\twolineshloka
{`शुद्धक्षेत्रे ब्रह्महत्याप्रायश्चित्तमथोच्यते}
{पुनश्च पातकं दृष्टं विप्रक्षेत्रे विशेषतः'}


\twolineshloka
{भद्रे निर्वेष्टुकामोऽहं तत्रावकिरणं मम}
{`प्रायश्चित्तं महदतो दारग्रहणपूर्वकम्}


\twolineshloka
{बीजं न शुद्ध्यते वोढुरन्यथा कृतनिष्कृतेः}
{मातृतः पितृतः शुद्धो ज्ञेयः पुत्रो यथार्थतः ॥'}


\twolineshloka
{विषयेष्वनभिज्ञोऽहं धर्मार्थं किल सन्ततिः}
{एवं लोकान्गमिष्यामि पुत्रैरिति न संशयः}


\twolineshloka
{भद्रे धर्मं विजानीहि ज्ञात्वा चोपरमस्व ह ॥स्त्र्युवाच}
{}


\twolineshloka
{नानिलोऽग्निर्न वरुणो न चान्ये त्रिदशा द्विज}
{प्रियाः स्त्रीणां यथा कामो रतिशीला हि योषितः}


\twolineshloka
{सहस्रे किल नारीणां प्राप्येतैका कदाचन}
{तथा शतसहस्रेषु यदि काचित्पतिव्रता}


\twolineshloka
{नैता जानन्ति पितरं न कुलं न च मातरम्}
{न भ्रातॄन्न च भर्तारं न च पुत्रान्न देवरान्}


\threelineshloka
{लीलायन्त्यः कुलं घ्नन्ति कूलानीव सरिद्वराः}
{दोषान्सर्वाश्च मत्वाऽऽशु प्रजापतिरभाषत ॥भीष्म उवाच}
{}


\twolineshloka
{ततः स ऋषिरेकाग्रस्तां स्त्रियं प्रत्यभाषत}
{आस्यतांरुचितश्छन्दः किञ्च कार्यं ब्रवीहि मे}


\twolineshloka
{सा स्त्री प्रोवाच भगवन्द्रक्ष्यसे देशकालतः}
{वस तावन्महाभाग कृतकृत्यो भविष्यसि}


\twolineshloka
{ब्रह्मर्षिस्तामथोवाच स तथेति युधिष्ठिर}
{वत्स्येऽहं यावदुत्साहो भवत्या नात्र संशयः}


\twolineshloka
{अथर्षिरभिसम्प्रेक्ष्य स्त्रियं तां जरयाऽर्दिताम्}
{चिन्तां परमिकां भेजे सन्तप्त इव चाभवत्}


\twolineshloka
{यद्यदङ्गं हि सोऽपश्यत्तस्या विप्रर्षभस्तदा}
{नारमत्तत्रतत्रास्य दृष्टी रूपविरागिता}


\twolineshloka
{देवतेयं गृहस्यास्य शापात्किंनु विरूपिता}
{अस्याश्च कारणं वेत्तुं न युक्तं सहसा भया}


\twolineshloka
{इति चिन्ताविषक्तस्य तमर्थं ज्ञातुमिच्छतः}
{व्यगमद्रात्रिशेषः स मनसा व्याकुलेन तु}


\twolineshloka
{अथ सा स्त्री तथोवाच भगवन्पश्य वै रवेः}
{रूपं सन्ध्याभ्रसंरक्तं किमुपस्थाप्यतां तव}


\twolineshloka
{स उवाच ततस्तां स्त्रीं स्नानोदकमिहानय}
{उपासिष्ये ततः सन्ध्यां वाग्यतो नियतेन्द्रियः}


\chapter{अध्यायः ५१}
\twolineshloka
{अथ सा स्त्री तमुवाच विप्रमेवं भवत्विति}
{तैलं दिव्यमुपादाय स्नानशाटीमुपानयत्}


\twolineshloka
{अनुज्ञाता च मुनिना सा स्त्री तेन महात्मना}
{अथास्य तैलेनाङ्गानि सर्वाण्येवाभ्यमृक्षत}


\twolineshloka
{शनैश्चाच्छादितस्तत्र स्नानशालामुपागमत्}
{भद्रासनं ततश्चित्रमृषिरन्वगमन्नवम्}


\threelineshloka
{अथोपविष्टश्च यदा तस्मिन्भद्रासने तदा}
{स्नापयामास शनकैस्तमृषिं सुखहस्तवत्}
{दिव्यं च विधिवच्चक्रे सोपचारं मुनेस्तदा}


\twolineshloka
{जलेन सुसुखोष्णेन तस्या हस्तसुखेन च}
{व्यतीतां रजनीं कृत्स्नां नाजानात्स महाव्रतः}


\twolineshloka
{तत उत्थाय स मुनिस्तदा परमविस्मितः}
{पूर्वस्यां दिशि सूर्यं च सोऽपश्यदुदितं दिवि}


\twolineshloka
{`सन्ध्योपासनमित्येव सर्वपापहरं न मे}
{'तस्य बुद्धिरियं किन्तु मोहस्तत्त्वमिदं भवेत्}


\twolineshloka
{अथोपास्य सहस्रांशुं किं कोरमीत्युवाच ताम्}
{सा चामृतरसप्रख्यं क्रषेरन्नमुपाहरत्}


\twolineshloka
{तस्य स्वादुतयाऽन्नस्य न प्रभूतं चकार सः}
{व्यगमच्चाप्यहःशेष ततः सन्ध्याऽऽगमत्पुनः}


\twolineshloka
{अथ सा स्त्री भगवन्तं सुप्यतामित्यचोदयत्}
{तत्र वै शयने दिव्ये तस्य तस्याश्च कल्पिते}


\threelineshloka
{[पृथक्र्वैव तथा सुप्तौ सा स्त्री स च मुनिस्तदा}
{तथाऽर्थरात्रे सा स्त्री तु शयनं तदुपागमत् ॥] अष्टावक्र उवाच}
{}


\threelineshloka
{न भद्रे परदारेषु मनो मे सम्प्रसज्जति}
{उत्तिष्ठ भद्रे भद्रं ते स्वापं वै विरमस्व च ॥भीष्म उवाच}
{}


\threelineshloka
{सा तदा तेन विप्रेण तथा धृत्या निवर्तिता}
{स्वतन्त्राऽस्मीत्युवाचर्षिं न धर्मच्छलमस्ति ते ॥अष्टावक्र उवाच}
{}


\threelineshloka
{नास्ति स्वतन्त्रता स्त्रीणामस्वतन्त्रा हि योषितः}
{प्रजापतिमतं ह्येतन्न स्त्री स्वातन्त्र्यमर्हति ॥स्त्र्युवाच}
{}


\threelineshloka
{बाधने मैथुनं विप्र मम भक्तिं च पश्य वै}
{अधर्मं प्राप्स्यसे विप्र यन्मां त्वं नाभिनन्दसि ॥अष्टावक्र उवाच}
{}


\threelineshloka
{हरन्ति दोषजातानि नरमिन्द्रियकिङ्करम्}
{प्रभवामि सदा धृत्या भद्रे स्वशयनं व्रज ॥स्त्र्युवाच}
{}


\twolineshloka
{शिरसा प्रणमे विप्र प्रसादं कर्तुमर्हसि}
{भूमौ निपतमानायाः शरणं भव मेऽनघ}


\twolineshloka
{यदि वा दोषजातं त्वं परदारेषु पश्यसि}
{आत्मानं स्पर्शयाम्यद्य पाणिं गृह्णीष्व मे द्विज}


\fourlineindentedshloka
{न दोषो भविता चैव सत्येनैतद्ब्रवीम्यहम्}
{स्वतन्त्रां मां विजानीहि यो धर्मः सोस्तु वै मयि}
{त्वय्यावेशितचित्ता च स्वतन्त्राऽस्मि भजस्व माम् ॥अष्टावक्र उवाच}
{}


\twolineshloka
{स्वतन्त्रा त्वं कथं भद्रे ब्रूहि कारणमत्र वै}
{नास्ति त्रिलोके स्त्री काचिद्या वै स्वातन्त्र्यमर्हति}


\fourlineindentedshloka
{पिता रक्षति कौमारे भर्ता रक्षति यौवने}
{पुत्रस्तु स्थाविरे भावे न स्त्री स्वातन्त्र्यमर्हति}
{`न वृद्धामक्षतां मन्ये च चेच्छा त्वयि मेऽनघे' स्त्र्युवाच}
{}


\threelineshloka
{कौमारं ब्रह्मचर्यं मे कन्यैवास्मि न संशयः}
{पत्नीं कुरुष्व मां विप्र श्रद्धां विजहि मा मम ॥अष्टावक्र उवाच}
{}


\twolineshloka
{यथा मम तथा तुभ्यं यथा तुभ्यं तथा मम}
{जिज्ञासेयमृषेस्तस्य विघ्नः सत्यं नु किं भवेत्}


\twolineshloka
{आश्चर्यं परमं हीदं किन्नु श्रेयो हि मे भवेत्}
{दिव्याभरणवस्त्रा हि कन्येयं मामुपस्थिता}


\twolineshloka
{किं त्वस्याः परमं रूपं जीर्णमासीत्कथं पुनः}
{कन्यारूपमिहाद्यैवं किमिवात्रोत्तरं भवेत्}


\twolineshloka
{यथा मे परमा शक्तिर्न व्युत्थाने कथंचन}
{न रोचते हि व्युत्थानं सत्येनासादयाम्यहम्}


\chapter{अध्यायः ५२}
\threelineshloka
{न बिभेति कथं सा स्त्री शापाच्च परमद्युते}
{कथं निवृत्तो भगवांस्तद्भवान्प्रब्रवीतु मे ॥भीष्म उवाच}
{}


\threelineshloka
{अष्टावक्रोऽन्वपृच्छत्तां रूपं विकुरुषे कथम्}
{न चानृतं ते वक्तव्यं ब्रूहि ब्राह्मणकाम्यया ॥स्त्र्युवाच}
{}


\twolineshloka
{द्यावापृथिव्योर्यत्रैषा काम्या ब्राह्मणसत्तम}
{शृणुष्वावहितः सर्वं यदिदं सत्यविक्रम}


\twolineshloka
{जिज्ञासेयं प्रयुक्ता मे स्थैर्यं कर्तुं तवानघ}
{अव्युत्थानेन ते लोका जिताः सत्यपराक्रम}


\twolineshloka
{उत्तरां मां दिशं विद्धि दृष्टं स्त्रीचापलं च ते}
{स्थविराणामपि स्त्रीणां बाधते मैथुनज्वरः}


\twolineshloka
{`अविश्वासो न व्यसनी नातिसक्तोऽप्रवासकः}
{विद्वान्सुशीलः पुरुषः सदारः सुखमश्नुते ॥'}


% Check verse!
तुष्टः पितामहस्तेऽद्य तथा देवाः सवासवाः
\twolineshloka
{सत्वं येन च कार्येण सम्प्राप्तो भगवानिह}
{प्रेषितस्तेन विप्रेण कन्यापित्रा द्विजर्षभः}


% Check verse!
प्रेषितश्चोपदेशाय तच्च सर्वं श्रुतं त्वया
\twolineshloka
{`नितान्तं स्त्री भोगपरा प्रियवादाप्रवासनात्}
{रक्ष्यते चाकुचेलाद्यैरप्रसङ्गानुवर्तनैः}


\twolineshloka
{अपर्वस्वनिषिद्धासु रात्रिष्वप्यनृतौ व्रजेत्}
{रात्रौ च नातिनियमो न वै ह्यनियमो भवेत् ॥'}


\twolineshloka
{क्षेमैर्गमिष्यसि गृहं श्रमश्च न भविष्यति}
{कन्यां प्राप्स्यसि तां विप्र पुत्रिणी च भविष्यति}


\twolineshloka
{काम्यया पृष्टवांस्त्वं मां ततो व्याहृतमुत्तरम्}
{अनतिक्रमणीया साकृत्स्नैर्लोकैस्त्रिभिः सदा}


\twolineshloka
{गच्छस्व कृतकृत्यस्त्वं किं वाऽन्यच्छ्रोतुमिच्छसि}
{यावद्ब्रवीमि विप्रर्षे अष्टावक्र यथातथम्}


\threelineshloka
{ऋषिणा प्रसादिता चास्मि तव हेतोर्द्विजर्षभ}
{तस्य सम्माननार्थं मे त्वयि वाक्यं प्रभाषितम् ॥भीष्म उवाच}
{}


\twolineshloka
{श्रुत्वा तु वचनं तस्याः स विप्रः प्राञ्जलिः स्थितः}
{अनुज्ञातस्तया चापि स्वगृहं पुनराव्रजत्}


\twolineshloka
{गृहमागत्य विश्रान्तः स्वजनं परिपृच्छ्य च}
{अभ्यागच्छच्च तं विप्रं न्यायतः कुरुनन्दन}


\twolineshloka
{पृष्टश्च तेन विप्रेण दृष्टं त्वेतन्निदर्शनम्}
{प्राह विप्रं तदा विप्रः सुप्रीतेनान्तरात्मना}


\twolineshloka
{भवताऽहमनुज्ञातः प्रास्थितो गन्धमादनम्}
{तस्य चोत्तरतो देशे दृष्टं मे दैवतं महत्}


\twolineshloka
{तया चाहमनुज्ञातो भवांश्चापि प्रकीर्तितः}
{श्रावितश्चापि तद्वाक्यं गृहं चाभ्यागतः प्रभो}


\threelineshloka
{तमुवाच तदा विप्रः सुतां प्रतिगृहाण मे}
{नक्षत्रतिथिसंयोगे पात्रं हि परमं भवान् ॥भीष्म उवाच}
{}


\twolineshloka
{अष्टावक्रस्तथेत्युक्त्वा प्रतिगृह्य च तां प्रभो}
{कन्यां परमधर्मात्मा प्रीतिमांश्चाभवत्तदा}


\twolineshloka
{कन्यां तां प्रतिगृह्यैव भार्यां परमशोभनाम्}
{उवास मुदितस्तत्र श्वाश्रमे विगतज्वरः}


\chapter{अध्यायः ५३}
\twolineshloka
{वैश्ययोन्यां समुत्पन्नाः शूद्रयोन्यां तथैव च}
{ब्रह्मर्षय इति प्रोक्ताः पुराणि द्विजसत्तमाः}


\threelineshloka
{कथमेतन्महाराज तत्त्वं शंसितुमर्हसि}
{विरुद्धमिह पश्यामि वियोनौ ब्राह्मणो भवेत् ॥भीष्म उवाच}
{}


\twolineshloka
{अलं कौतूहलेनात्र निबोध त्वं युधिष्ठिर}
{शुभेतरं शुभं वाऽपि न चिन्तयितुमर्हसि}


\twolineshloka
{ईशन्त्यात्मन इत्येते गतिश्चैषां न सञ्जते}
{ब्रह्मभूयांस इत्येव ऋषयः श्रुतिचोदिताः}


\twolineshloka
{निन्द्या न चैते राजेन्द्र प्रमाणं हि प्रमाणिनाम्}
{लोकोऽनुमन्यते चैतान्प्रमाणं ह्यत्र वै तपः}


\twolineshloka
{एवं महात्मभिस्तात तपोज्ञानसमन्वितैः}
{प्रवर्तितानि कार्याणि प्रमाणान्येव सत्तम}


\twolineshloka
{भार्याश्चतस्रो राजेन्द्र ब्राह्मणस्य स्वधर्मतः}
{ब्राह्मणी क्षत्रिया वैश्या शूद्रा च भरतर्षभ}


\twolineshloka
{राज्ञां तु क्षत्रिया वैश्या शूद्रा च भरतर्षभ}
{वैश्यस्य वैश्या विहिता शूद्रा च भरतर्षभ}


\twolineshloka
{शूद्रस्यैका स्मृता भार्या प्रतिलोमे तु सङ्करः}
{शूद्रायास्तु नरश्रेष्ठ चत्वारः पतयः स्मृताः}


\twolineshloka
{वर्णोत्तमायास्तु पतिः सवर्णस्त्वेक एव सः}
{द्वौ क्षत्रियाया विहितौ ब्राह्मणः क्षत्रियस्तथा}


\twolineshloka
{वैश्यायास्तु नरश्रेष्ठ विहिताः पतयस्त्रयः}
{सवर्णः क्षत्रियश्चैव ब्राह्मणश्च विशाम्पते}


\twolineshloka
{एवंविधिमनुस्मृत्य ततस्ते ऋषिभिः पुरा}
{उत्पादिता महात्मानः पुत्रा ब्रह्मर्षयः पुरा}


\twolineshloka
{पुराणाभ्यामृषिभ्यां तु मित्रेण वरुणेन च}
{वसिष्ठोऽथ तथाऽगस्त्यो बर्हिषव्यस्तथैव च}


\twolineshloka
{कक्षीवानार्ष्टिषेणश्च पुरुषः कष एव च}
{मामतेयस्य वै पुत्रा गौतमश्चात्मजाः स्मृताः}


\twolineshloka
{वत्सप्रियश्च भगवान्स्थूलरश्मिस्तथाक्षणिः}
{गौतमस्यैव तनया ये दास्यां जनिता ह्युत}


\twolineshloka
{कपिञ्जलादो ब्रह्मर्षिश्चाण्डाल्यामुदपद्यत}
{वैनतेयस्तथा पक्षी तुर्यायां च वसिष्ठतः}


\threelineshloka
{प्रसादाच्च वसिष्ठस्य शुक्लाभ्युपगमेन च}
{अदृश्यन्त्याः पिता वैश्यो नाम्ना चित्रमुखः पुरा}
{ब्राह्मणत्वमनुप्राप्तो ब्रह्मर्षित्वं च कौरव}


\twolineshloka
{वैश्यश्चित्रमुखः कन्यां वसिष्ठतनयस्य वै}
{शुभां प्रादाद्यतो जातो ब्रह्मर्षिस्तु पराशरः}


\twolineshloka
{तथैव दाशकन्यायां सत्यवत्यां महानृषिः}
{पराशरात्प्रसूतश्च व्यासो योगमयो मुनिः}


\twolineshloka
{विभण्डकस्य मृग्यां च तपोयोगात्मको मुनिः}
{ऋस्यशृङ्गः समुत्पन्नो ब्रह्मचारी महायशाः}


\twolineshloka
{शार्ङ्ग्यां च मन्दपालस्य चत्वारो ब्रह्मवादिनः}
{जाता ब्रह्मर्षयः पुण्या यैः स्तुतो हव्यवाहनः}


\twolineshloka
{द्रोणश्च स्तम्बमित्रश्च सारिसृक्वश्च बुद्धिमान्}
{जरितारिश्च विख्यातश्चत्वारः सूर्यसन्निभाः}


\twolineshloka
{महर्षेः कालवृक्षस्य शकुन्त्यामेव जज्ञिवान्}
{हिरण्यहस्तो भगवान्महर्षिः काञ्चनप्रभः}


\twolineshloka
{पावकात्तात सम्भूता मनसा च महर्षयः}
{पितामहस्य राजेन्द्र पुरस्त्यपुलहादयः}


\twolineshloka
{सावर्ण्यश्चापि राजर्षिः सवर्णायामजायत}
{मृण्मय्यां भरतश्रेष्ठ आदित्येन विवस्वता}


\twolineshloka
{शाण्डिल्यश्चाग्नितो जातः कश्यपस्याग्रजः प्रभुः}
{शरद्वतः शरस्तम्बात्कृपश्व कृपया सह}


\twolineshloka
{पद्माश्च जज्ञे राजेन्द्र सोस्यपस्य महात्मनः}
{रेणुश्च रेणुका चैव राममाता यशस्विनी}


\twolineshloka
{समुनायाः समुद्भूतः सोमकेन महात्मना}
{अर्कदन्तो महानृषिः प्रथितः पृथिवीतले}


\twolineshloka
{अग्नेराहवनीयाच्च द्रुपदस्येन्द्रवर्चसः}
{धृष्टद्युम्नश्च सम्भूतो वेद्यां कृष्णा च भारत}


\twolineshloka
{व्याघ्रयोन्यां ततो जाता वसिष्ठस्य महात्मनः}
{एकोनविंशतिः पुत्राः ख्याता व्याघ्रपदादयः}


\threelineshloka
{मन्धश्च बादलोमस्च जावालिश्च महानृषिः}
{मन्युश्चैवोपमन्युश्च सेतुकर्णस्तथैव च}
{एते चान्ये च विख्याताः पृथिव्यां गोत्रतां गताः}


\twolineshloka
{विश्वकाशस्य राजर्षेरैक्ष्वाकोस्तु महात्मनः}
{बालाश्वो नाम पुत्रोऽभूच्छिखां भित्त्वा विनिस्सृतः}


\twolineshloka
{मान्धाता चैव राजर्षिर्युवनाश्वेन धीमता}
{स्वयं धृतोऽथ गर्भेण दिव्यास्त्रबलसंयुतः}


\twolineshloka
{गौरिकश्चापि राजर्षिश्चक्रवर्ती महायशाः}
{बाहुदायां समुत्पन्नो नद्यां राज्ञा सुबाहुना}


\twolineshloka
{भूमेश्च पुत्रो नरकः संवर्तश्चैव पुष्कलः}
{अद्भिश्चैव महातेजा ऋषिर्गार्ग्योऽभ्यजायत}


\threelineshloka
{एते चान्ये च बहवो राजन्या ब्राह्मणास्तथा}
{प्रभावेनाभिसम्भूता महर्षीणां महात्मनाम्}
{नासाध्य तपसा तेषां विद्ययाऽऽत्मगुणैः परैः}


\twolineshloka
{अस्मिन्नर्थे च मनुना नीतः श्लोको नराधिप}
{धर्मं प्रणयता राजंस्तं निबोध युधिष्ठिर}


\twolineshloka
{ऋषिणां च नदीनां च साधूनां च महात्मनाम्}
{प्रभवो नाधिगन्तव्यः स्त्रीणां दुश्चरितस्य च}


\twolineshloka
{तन्नात्र चिन्ता कर्तव्या महर्षीणां समुद्भवे}
{यथा सर्वगतो ह्यग्निस्तथा तेजो महात्मसु ॥'}


\chapter{अध्यायः ५४}
\threelineshloka
{पुत्रैः कथं महाराज पुरुषस्तारितो भवेत्}
{यावन्न लब्धवान्पुत्रमफलः पुरुषो नृप ॥भीष्म उवाच}
{}


\twolineshloka
{अत्राप्युदाहरन्तीममितिहासं पुरातनम्}
{नारदेन पुरा गीतं मार्कण्डेयाय पृच्छते}


\twolineshloka
{पर्वतं नारदं चैव असितं देवलं च तम्}
{आरुणेयं च रैभ्यं च एतानत्रागतान्पुरा}


\twolineshloka
{गङ्गायमुनयोर्मध्ये भोगवत्याः समागमे}
{दृष्ट्वा पूर्वं समासीनान्मार्कण्डेयोऽभ्यगच्छत}


\threelineshloka
{ऋषयस्तु मुनिं दृष्ट्वा समुत्थायोन्मुखाः स्थिताः}
{अर्चयित्वाऽर्हतो विप्रं किं कुर्म इति चाब्रुवन् ॥मार्कण्डेय उवाच}
{}


\twolineshloka
{अयं समागमः सद्भिर्यत्नेनासादितो मया}
{अत्र प्राप्स्यामि धर्माणामाचारस्य च निश्चयम्}


\twolineshloka
{ऋजुः कृतयुगे धर्मस्तस्मिन्क्षीणे विमुह्यति}
{युगेयुगे महर्षिभ्यो धर्ममिच्छामि वेदितुम्}


\twolineshloka
{ऋषिभिर्नारदः प्रोक्तो ब्रूहि यत्रास्य संशयः}
{धर्माधर्मेषु तत्वज्ञ त्वं हि च्छेत्ता हि संशयान्}


\twolineshloka
{ऋषिभ्योऽनुमतं वाक्यं नियोगान्नारदस्तदा}
{सर्वधर्मार्थतत्वज्ञं मार्कण़्डेयं ततोऽब्रवीत्}


\twolineshloka
{दीर्घायो तपसा दीप्त वदेवदाङ्गतत्ववित्}
{यत्र ते संशयो ब्रह्मन्समुत्पन्नः स उच्यताम्}


\threelineshloka
{धर्मं लोकोपकारं वा यच्चान्यच्छ्रोतुमिच्छसि}
{तदहं कथयिष्यामि ब्रूहि त्वं सुमहातपाः ॥मार्कण्डेय उवाच}
{}


\threelineshloka
{युगेयुगे व्यतीतेऽस्मिन्धर्मसेतुः प्रणश्यति}
{कथं धर्मच्छलेनाहं प्राप्नुयामिति मे मतिः ॥नारद उवाच}
{}


\twolineshloka
{आसीद्धर्मः पुरा विप्र चतुष्पादः कृते युगे}
{ततो ह्यधर्मः कालेन प्रसूतः किञ्चिदूनतः}


% Check verse!
ततस्त्रेतायुगं नाम प्रवृत्तं धर्मदूषणम्
\twolineshloka
{तस्मिन्नतीते सम्प्राप्तं तृतीयं द्वापरं युगम्}
{तदा धर्मस्य द्वौ पादावधर्मो नाशयिष्यति}


\twolineshloka
{द्वापरे तु परिक्षीणे नन्दिके समुपस्थिते}
{लोकवृत्तं च धर्मं च उच्यमानं निबोध मे}


\fourlineindentedshloka
{चतुर्थं नन्दिकं नाम धर्मः पादावशेषितः}
{ततः प्रभृति जायन्ते क्षीणप्रज्ञायुपो नराः}
{क्षीणप्राणधना लोके धर्माचारबहिष्कृताः ॥मार्कण्डेय उवाच}
{}


\threelineshloka
{एवं विलुलिते धर्मे लोके चाधर्मसंयुते}
{चातुर्वर्ण्यस्य नियतं हव्यं कव्यं नियच्छति ॥नारद उवाच}
{}


\twolineshloka
{मन्त्रपूतं सदा हव्यं कव्यं चैव न नश्यति}
{प्रतिगृह्णन्ति तद्देवा दातुर्न्यायात्प्रयच्छतः}


\threelineshloka
{सत्वयुक्तं च दाता च सर्वान्कामानवाप्नुयात्}
{अवाप्तकामः स्वर्गे च महीयेत यथेप्सितम् ॥मार्कण्डेय उवाच}
{}


\threelineshloka
{चत्वारो ह्यथ ये वर्णा हव्यं कव्यं प्रदास्यति}
{मन्त्रहीनमपन्यायं तेषां दत्तं क्व गच्छति ॥नारद उवाच}
{}


\threelineshloka
{असुरान्गच्छते दत्तं विप्रै रक्षांसि क्षत्रियैः}
{वैश्यैः प्रेतानि वै दत्तं शूद्रैर्भूतानि गच्छति ॥मार्कण्डेय उवाच}
{}


\threelineshloka
{अथ वर्णावरे जाताश्चातुर्वर्ण्योपदेशिनः}
{दास्यन्ति हव्यकव्यानि तेषां दत्तं क्व गच्छति नारद उवाच}
{}


\twolineshloka
{वर्णावराणां भूतानां हव्यकव्यप्रदातृणाम्}
{नैव देवा न पितरः प्रतिगृह्णन्ति तत्स्वयम्}


\twolineshloka
{यातुधानाः पिशाचाश्च भूता ये चापि नैर्ऋताः}
{तेषां सा विहिता वृत्तिः पितृदैवतनिर्गताः}


\twolineshloka
{तेषां सर्वप्रदातॄणां हव्यकव्यं समाहिताः}
{यत्प्रयच्छन्ति विधइवत्तद्वै भुञ्जन्ति देवताः ॥'}


\chapter{अध्यायः ५५}
\threelineshloka
{श्रुतं वर्णावरैर्दत्तं हव्यं कव्यं च नारद}
{सम्प्रयोगे च पुत्राणां कन्यानां च ब्रवीहि मे ॥नारद उवाच}
{}


\twolineshloka
{कन्याप्रदानं प्रत्राणां स्त्रीणां संयोगमेव च}
{आनुपूर्व्यान्मया सम्यगुच्यमानं निबोध मे}


\twolineshloka
{जातमात्रा तु दातव्या कन्यका सदृशे वरे}
{काले दत्तासु कन्यासु पिता धर्मेण युज्यते}


\twolineshloka
{यस्तु पुष्पवतीं कन्यां बान्धवो न प्रयच्छति}
{मासिमासि गते बन्धुस्तस्या भ्रौणघ्न्यमाप्नुते}


\twolineshloka
{यस्तु कन्यां गृहे रुन्ध्याद्ग्राम्यैर्भोगैर्विवर्जिताम्}
{अवध्यातः स कन्याया बन्धुः प्राप्नोति भ्रूणहाम्}


\twolineshloka
{दूषिता पाणिमात्रेण मृते भर्तरि दारिका}
{संस्कारं लभते नारी द्वितीये सा पुनः पतौ}


\threelineshloka
{पुनर्भूर्नाम सा कन्या सपुत्रा हव्यकव्यदा}
{अदूष्या सा प्रसूतीषु प्रजानां दारकर्मणि ॥मार्कण्डेय उवाच}
{}


\threelineshloka
{या तु कन्या प्रसूयेत गर्भिणी या तु वा भवेत्}
{कथं दारक्रियां भूयः सा भवेदृषिसत्तम ॥नारद उवाच}
{}


\twolineshloka
{तत्वार्थनिश्चितं शब्दं कन्यका नयतेऽग्नये}
{तस्मात्कुर्वन्ति वै भावं कुमार्यस्ता न कन्यकाः}


\threelineshloka
{ब्रह्महत्यात्रिभागेन गर्भाधानविशोधितः}
{गृह्णीयात्तां चतुर्भागविशुद्धां सर्जनात्पुनः ॥मार्कण्डेय उवाच}
{}


\threelineshloka
{कथं कन्यासु ये जाता बन्धूनां दूषिताः सदा}
{कस्य ते हव्यकव्यानि प्रदास्यन्ति महामुने ॥नारद उवाच}
{}


\threelineshloka
{कन्यायास्तु पितुः पुत्राः कानीना हव्यकव्यदाः}
{अन्तर्वत्नयास्तु यः पाणिं गृह्णीयात्स सहोढजः ॥मार्कण्डेय उवाच}
{}


\threelineshloka
{अथ येनाहितो गर्भः कन्यायां तत्र नारद}
{कथं पुत्रफलं तस्य भवेदेतत्प्रचक्ष्व मे ॥नारद उवाच}
{}


\threelineshloka
{धर्माचारेषु ते नित्यं दूषकाः कृतशोधनाः}
{बीजं च नश्यते तेषां मोघचेष्टा भवन्ति ते ॥मार्कण्डेय उवाच}
{}


\threelineshloka
{अथ काचिद्भवेत्कन्या क्रीता दत्ता हृताऽपि वा}
{कथं पुत्रकृतं तस्यास्तद्भवेद्दषिसत्तम ॥नारद उवाच}
{}


\fourlineindentedshloka
{क्रीता दत्ता हृता चैव या कन्या पाणिवर्जिता}
{कौमारी नाम सा भार्या प्रसवेदौरसान्सुतान्}
{न पत्न्यर्थे शुभा प्रोक्ता तत्कर्मण्यपराजिते ॥मार्कण्डेय उवाच}
{}


\threelineshloka
{केन मङ्गलकृत्येषु विनियुज्यन्ति कन्यकाः}
{एतदिच्छामि विज्ञातुं तत्वेनेह महामुने ॥नारद उवाच}
{}


\twolineshloka
{नित्यं निवसते लक्ष्मीः कन्यकासु प्रतिष्ठिता}
{शोभना शुभयोग्या च पूज्या मङ्गलकर्मसु}


\twolineshloka
{आकरस्थं यथा रत्नं सर्वकामफलोपगम्}
{तथा कन्या महालक्ष्मीः सर्वलोकस्य मङ्गलम्}


\twolineshloka
{एवं कन्या परा लक्ष्मी रतिस्तोषश्च देहिनाम्}
{महाकुलानां चारित्रवृत्तेन निकषोपलम्}


\twolineshloka
{आनयित्वा स्वकाद्वर्णात्कन्यकां यो भजेन्नरः}
{दातारं हव्यकव्यानां पुत्रकं या प्रसूयति}


\twolineshloka
{साध्वी कुलं वर्धयति साध्वी पुष्टिग्रहे परा}
{साध्वी लक्ष्मी रतिः साक्षात्प्रतिष्ठा सन्ततिस्तथा}


\chapter{अध्यायः ५६}
\threelineshloka
{श्रुतं बहुविधं वृत्तं कन्यकानां महामते}
{इच्छामि योषितां श्रोतुं धर्माधर्मौ परिग्रहे ॥नारद उवाच}
{}


\twolineshloka
{अष्टौ भार्यागमा धर्म्या नराणां दारकर्मणि}
{प्रेत्येह च हिता यास्तु सपुत्रा हव्यकव्यदाः}


\threelineshloka
{साध्वी पाणिगृहीता या कौमारी पाणिवर्जिता}
{भ्रातृभार्या स्वभार्येति प्रसूयेत्पुत्रमौरसम् ॥मार्कण्डेय उवाच}
{}


\threelineshloka
{त्रयो भार्यागमा ज्ञेया यत्र धर्मो न नश्यति}
{पञ्चान्याः पश्चिमा ब्रूहि भार्यास्तासां च ये सुताः ॥नारद उवाच}
{}


\twolineshloka
{सगोत्रभार्या क्रीता च परभार्या च कारिता}
{गतागता च या भार्या आश्रमादाहृता च या}


\twolineshloka
{एता भार्यागमाः पञ्च पुनर्भार्या भवन्ति याः}
{एता भार्या नृणां गम्यास्तत्पुत्रा हव्यकव्यदाः ॥'}


\chapter{अध्यायः ५७}
\threelineshloka
{श्रुता भार्याश्च पुत्राश्च विस्तरेण महामुने}
{आश्रमस्थाः कथं नार्यो न दुष्यन्तीति ब्रूहि भो ॥नारद उवाच}
{}


\twolineshloka
{आश्रमस्थासु नारीषु बान्धवत्वं प्रणश्यति}
{नष्टवंश्या भवन्त्येता बन्धूनामथ भर्तृणाम्}


\twolineshloka
{परदारा मुक्तदोषास्ता नार्योऽऽश्रमसंस्थिताः}
{स्वयमीशाः स्वदेहानां काम्यास्तद्गतमानसाः}


\threelineshloka
{एवं नार्यो न दुष्यन्ति नराणां तत्प्रसूतिषु}
{धर्मपत्न्यो भवन्त्येताः सपुत्रा हव्यकव्यदाः ॥मार्कण्डेय उवाच}
{}


\threelineshloka
{परस्य भार्या या पूर्वं मृते भर्तरि या पुनः}
{अन्यं भजति भर्तारं ससुता असुता कथम् ॥नारद उवाच}
{}


\twolineshloka
{असुता वा प्रसूता वा गृहस्थानां परस्त्रियः}
{परामृष्टेति ता वर्ज्या धर्माचारेषु दूषिताः}


\threelineshloka
{न चासां हव्यकव्यानि प्रतिगृह्णन्ति देवताः}
{यस्तासु जनयेत्पुत्रान्न तैः पुत्रमवाप्नुयात् ॥मार्कण्डेय उवाच}
{}


\threelineshloka
{परक्षेत्रेषु यो बीजं चापलाद्विसृजेन्नरः}
{कथं पुत्रफलं तस्य भवेत्तदृषिसत्तम ॥नारद उवाच}
{}


\twolineshloka
{अस्वामिके परक्षेत्रे यो नरो बीजमुत्सृजेत्}
{स्वयंवृतोऽऽश्रमस्थायां तद्बीजं न विनश्यति}


\twolineshloka
{परक्षेत्रेषु यो बीजं नरो दर्पात्समुत्सृजेत्}
{क्षेत्रिकस्यैव तद्बीजं न बीजी लभते फलम्}


\threelineshloka
{नातः परमधर्म्यं चाप्ययशस्यं तथोत्तरम्}
{गर्भादीनां च बहुभिस्ताश्च त्याज्याः समेष्वपि ॥मार्कण्डेय उवाच}
{}


\threelineshloka
{अथ ये परदारेषु पुत्रा जायन्ति नारद}
{कस्य ते बन्धुदायादा भवन्ति परमद्युते ॥नारद उवाच}
{}


\twolineshloka
{परदारेषु जायेते द्वौ पुत्रौ कुण्डगोलकौ}
{जीवत्यथ पतौ कुण्डो मृते भर्तरि गोलकः}


\twolineshloka
{ते च जाताः परक्षेत्रे देहिनां प्रेत्य चेह च}
{दत्तानि हव्यकव्यानि नाशयन्त्यथ दातृणाम्}


\threelineshloka
{पितुहि नरकायैते गोलकस्तु विशेषतः}
{चण्डालतुल्यौ तज्जौ हि परत्रेह च नश्यतः ॥मार्कण्डेय उवाच}
{}


\threelineshloka
{कस्य ते गर्हिताः पुत्राः पितॄणां हव्यकव्यदाः}
{यस्य क्षेत्रे प्रसूयन्ते यो वा ताञ्जनयेत्सुतान् ॥नारद उवाच}
{}


\twolineshloka
{क्षेत्रिकश्चैव बीजी च द्वावेतौ निरयं गतौ}
{न रक्षति च यो दारान्परदाराश्च गच्छति}


\threelineshloka
{गर्हितास्ते नरा नित्यं धर्माचारबहिष्कृताः}
{कुण्डो भोक्ता च भोगी च कुत्सिताः पितृदैवतैः ॥मार्कण्डेय उवाच}
{}


\threelineshloka
{तथैते गर्हिताः पुत्रा हव्यकव्यानि नारद}
{कस्य नित्यं प्रयच्छन्ति धर्मो वा तेषु किं फलं ॥नारद उवाच}
{}


\threelineshloka
{यातुधानाः पिशाचाश्च प्रतिगृह्णन्ति तैर्हुतम्}
{हव्यं कव्यं च तैर्दत्तं ये च भूता निशाचराः ॥मार्कण्डेय उवाच}
{}


\threelineshloka
{अथ ते राक्षसाः प्रीताः किं प्रयच्छन्ति दातृणाम्}
{किं वा धर्मफलं तेषां भवेत्तदृषिसत्तम ॥नारद उवाच}
{}


\threelineshloka
{न दत्तं नश्यते किञ्चित्सर्वभूतेषु दातृणाम्}
{प्रेत्य चेह च तां पुष्टिमुपाश्नन्ति प्रदायिनः ॥मार्कण्डेय उवाच}
{}


\threelineshloka
{अथ गोलककुण्डाभ्यां सन्ततिर्या भविष्यति}
{तयोर्ये बान्धवाः केचित्प्रदास्यन्ति कथं नु तं ॥नारद उवाच}
{}


\twolineshloka
{साध्वीजाताः सुतास्तेषां तां वृत्तिमनुतिष्ठताम्}
{प्रीणन्ति पितृदैवत्यं हव्यकव्यसमाहिताः}


\twolineshloka
{एवं गोलककुण्डाभ्यां ये च वर्णापदेशिनः}
{हव्यं कव्यं च शुद्धानां प्रतिगृह्णन्ति देवताः ॥'}


\chapter{अध्यायः ५८}
\threelineshloka
{श्रुतं नराणां चापल्यं परस्त्रीषु प्रजायताम्}
{प्रमदानां तु चापल्ये दोषमिच्छामि वेदितुम् ॥नारद उवाच}
{}


\twolineshloka
{एकवर्णे विदोषं तु गमनं पूर्वकालिकम्}
{धाता च समनुज्ञातो विष्णुना तत्तथाऽकरोत्}


\twolineshloka
{भगलिङ्गे महाप्राज्ञ पूर्वमेव प्रजापतिः}
{ससर्ज ताभ्यां संयोगमनुज्ञातश्चकार सः}


\twolineshloka
{अथ विष्णुप्रसादेन भगो दत्तवरः किल}
{तेन चैव प्रसादेन सर्वांल्लोकानुपाश्नुते}


\twolineshloka
{तस्मात्तु पुरुषे दोषो ह्यधिको नात्र संशयः}
{विना गर्भं सवर्णेषु न त्याज्या गमनात्स्त्रियः}


\twolineshloka
{प्रायश्चित्तं यथान्यायं दण्डं कुर्यात्स पण्डितः}
{श्वभिर्वा दंशनं स्नानं सवनत्रितयं निशि}


\threelineshloka
{भूमौ च भस्मशयनं दानं भोगविवर्जितम्}
{दोषगौरवतः कालो द्रव्यगौरवमेव च}
{मर्यादा स्थापिता पूर्वमिति तीर्थान्तरं गते}


\twolineshloka
{तद्योषितीं तु दीर्घायो नास्ति दोषो व्यतिक्रमे}
{भगतीर्थान्तरे शुद्धो विष्णोस्तु वचनादिह}


\twolineshloka
{रक्ष्याश्चैवान्यसंवादैरन्यगेहाद्विचक्षणैः}
{आसां शुद्धौ विशेषेण कर्मणां फलमश्नुते}


\threelineshloka
{नैता वाच्या न वै वध्या न क्लेश्याः शुभमिच्छता}
{विष्णुप्रसादादित्येव भगस्तीर्थान्तरं गतः}
{मासिमासि ऋतुस्तासां दुष्कृतान्यपकर्षति}


\twolineshloka
{स्त्रियस्तोषकरा नॄणां स्त्रियः पुष्टिप्रदाः सदा}
{पुत्रसेतुप्रतिष्ठाश्च स्त्रियो लोके महाद्युते}


\chapter{अध्यायः ५९}
\threelineshloka
{श्रुतं बलं प्रभावश्च योषितां मुनिसत्तम}
{एकस्य बहुभार्यस्य धर्ममिच्छामि वेदितुम् ॥नारद उवाच}
{}


\twolineshloka
{बहुभार्यासु सक्तस्य नारीभोगेषु गेहिनः}
{ऋतौ विमुञ्चमानस्य सांनिध्ये भ्रूणहा स्मृतः}


\twolineshloka
{वृद्धां वन्ध्यां सुव्रता च मृतापत्यामपुष्पिणीम्}
{कन्यां च बहुपुत्रां च वर्जयन्मुच्यते भयात्}


\threelineshloka
{व्याधितो बन्धनस्थो वा प्रवासेष्वथ पर्वसु}
{ऋतुकाले तु नारीणां भ्रूणहत्यां प्रमुञ्चति ॥मार्कण्डेय उवाच}
{}


\threelineshloka
{वैश्यनारीषु वै जाताः परप्रेष्यासु वा सुताः}
{कस्य ते बन्धुदायादा भवन्ति हि महामुने ॥नारद उवाच}
{}


\threelineshloka
{पण्यस्त्रीषु प्रसूता ये यस्य स्त्री तस्य ते सुताः}
{क्रयाच्च कृत्रिमाः पुत्रा प्रदानाच्चैव दत्रिमाः ॥मार्कण्डेय उवाच}
{}


\threelineshloka
{पण्यनारीष्वनियतः पुंसोऽर्थो वर्तते ध्रुवम्}
{अत्र चाहितगर्भायाः कस्य पुत्रं वदन्ति तम् ॥नारद उवाच}
{}


\threelineshloka
{तीर्थभूतासु नारीषु ज्ञायते योऽभिगच्छति}
{ऋतौ तस्य भवेद्गर्भो यं वा नारी न शङ्कते ॥मार्कण्डेय उवाच}
{}


\threelineshloka
{नराणां त्यजतां भार्यां कामक्रोधाद्गुणान्विताम्}
{अप्रसूतां प्रसूतां वा तेषां पृच्छामि निष्कृतिम् ॥नारद उवाच}
{}


\twolineshloka
{अपापां त्यजमानस्य साध्वीं मत्वा यमादितः}
{आत्मवंशस्वधर्मो वा त्यजतो निष्कृतिर्न तु}


\twolineshloka
{यो नरस्त्यजते भार्या पुष्पिणीमप्रसूतिकाम्}
{स नष्टवंशः पितृभिर्युक्तस्त्यज्येत दैवतैः}


\twolineshloka
{भार्यामपत्यसञ्जातां प्रसूतां पुत्रपौत्रिणीम्}
{पुत्रदारपरित्यागी न स प्राप्नोति निष्कृतिम्}


\threelineshloka
{एवं हि भार्यां त्यजतां नराणां नास्ति निष्कृतिः}
{नार्हन्ति प्रमदास्त्यक्तुं पुत्रपौत्रप्रतिष्ठिताः ॥मार्कण्डेय उवाच}
{}


\threelineshloka
{कीदृशीं संत्यजन्भार्यां नरो दोषैर्न लिप्यते}
{एतदिच्छामि तत्वेन विज्ञातुमृषिसत्तम ॥नारद उवाच}
{}


\twolineshloka
{मोक्षधर्मस्थितानां तु अन्योन्यमनुजानताम्}
{भार्यापतीनां मुक्तानामधर्मो न विधीयते}


\twolineshloka
{अन्यसङ्गां गतापत्यां शूद्रगां परगामिनीम्}
{परीक्ष्य त्यजमानानां नराणां नास्ति पातकम्}


\twolineshloka
{पातकेऽपि तु भर्तव्यौ द्वौ तु माता पिता तथा ॥मार्कण्डेय उवाच}
{}


\threelineshloka
{भार्यायां व्यभिचारिण्यां नरस्य त्यजतो रुषा}
{कथं धर्मोऽप्यधर्मो वा भवतीह महामते ॥नारद उवाच}
{}


\twolineshloka
{अनृतेऽपि हि सत्ये वा यो नारीं दूषितां त्यजेत्}
{अरक्षमाणः स्वां भार्यां नरो भवति भ्रूणहा}


\twolineshloka
{अपत्यहेतोर्या नारी भर्तारमतिलङ्घयेत्}
{लोलेन्द्रियेति सा रक्ष्या न सन्त्याज्या कथञ्चन}


\twolineshloka
{नद्यश्च नार्यश्च समस्वभावानैताः प्रमुञ्चन्ति नरावगाढाः}
{स्रोतांसि नद्यो वहते निपातंनारी रजोभिः पुनरेति शौचम्}


\threelineshloka
{एवं नार्यो न दुष्यन्ति व्यभिचारेऽपि भर्तृणाम्}
{मासिमासि भवेद्रागस्ततः शुद्धा भवन्त्युत ॥मार्कण्डेय उवाच}
{}


\twolineshloka
{कानि तीर्थानि भगवन्नृणां देहाश्रितानि वै}
{तानि वै शंस भगवन्याथातथ्येन पृच्छतः}


\threelineshloka
{सर्वतीर्थेषु सर्वज्ञ किं तीर्थं परमं नृणाम्}
{यत्रोपस्पृश्य पूतो यो नरो भवति नित्यशः ॥नारद उवाच}
{}


\twolineshloka
{देवर्षिपितृतीर्थानि ब्राह्मं मध्येऽथं वैष्णवम्}
{नृणां तीर्थानि पञ्चाहुः पाणौ सन्निहितानि वै}


\fourlineindentedshloka
{आद्यतीर्थं तु तीर्थानां वैष्णवो भाग उच्यते}
{यत्रोपस्पृश्य वर्णानां चतुर्णां वर्धते कुलम्}
{पितृदैवतकार्याणि वर्धन्ते प्रेत्य चेह च ॥मार्कण्डेय उवाच}
{}


\threelineshloka
{नराणां कामवृत्तानां या नार्यो निरवग्रहाः}
{यासामभिग्रहो नास्ति ता मे कथय नारद ॥नारद उवाच}
{}


\threelineshloka
{पाशुर्वैश्या नटी गोपी तान्तुकी तुन्नवायिकी}
{नारी किराती शबरी नर्तकी चानवग्रहा ॥मार्कण्डेय उवाच}
{}


\threelineshloka
{एतासु जाता नारीषु सर्ववर्णेषु ये सुताः}
{केषु के बन्धुदायादा भवन्ति ऋषिसत्तम ॥नारद उवाच}
{}


\threelineshloka
{य एताः परिगृह्णन्ति तेषामेव हि ते सुताः}
{सर्वत्र तु प्रवृत्तासु बीजं नश्यति देहिनाम् ॥मार्कण्डेय उवाच}
{}


\fourlineindentedshloka
{सर्वस्त्रीषु प्रवृत्ताश्च साधुवेदविवर्जिताः}
{मानवाः काण्डपृष्ठाश्च वेदमन्त्रबहिष्कृताः}
{नियुक्ता हव्यकव्येषु तेषां दत्तं कथं भवेत् ॥नारद उवाच}
{}


\threelineshloka
{नार्हन्ति हव्यकव्यानि सावित्रीवर्जिता द्विजाः}
{व्रात्येष्वन्नप्रदानं तद्यथा शूद्रेषु वै तथा ॥मार्कण्डेय उवाच}
{}


\threelineshloka
{धर्मेष्वधिकृतानां तु नराणां मुह्यते मनः}
{कथं न विघ्नो भवति एतदिच्छामि वेदितुम् ॥नारद उवाच}
{}


\twolineshloka
{अर्थाश्च नार्यश्च समानमेन-च्छ्रेयांसि पुंसामिह मोहयन्ति}
{रतिप्रमोदात्प्रमदा हरन्तिभोगैर्धनं चाप्युपहन्ति धर्मान्}


\threelineshloka
{हव्यं कव्यं च धर्मात्मा सर्वं तच्छ्रोत्रियोऽर्हति}
{दत्तं हि श्रोत्रिये साधौ ज्वलिताग्नाविवाहुतिः ॥मार्कण्डेय उवाच}
{}


\threelineshloka
{श्रोत्रियाणां कुले जाता वेदार्थविदितात्मनाम्}
{हित्वा कस्मात्त्रयीं विद्यां वार्तां वृत्तिमुपाश्रिताः ॥नारद उवाच}
{}


\twolineshloka
{चातुर्वर्ण्यं पुरा न्यस्तं सुविद्वत्सु द्विजातिषु}
{तस्माद्वर्णौः संविभज्या वृत्तिः सङ्करवर्जिता}


\threelineshloka
{ये चान्ये श्रोत्रिया जाताः संस्कृताः पुत्रगृध्नुभिः}
{पूर्वनिर्वाणनिर्वृत्तां जातां वृत्तिमुपाश्रिताः ॥मार्कण्डेय उवाच}
{}


\threelineshloka
{असंस्कृताः श्रोत्रियजाः संस्कृता ज्ञानिजाः कथम्}
{नारद उवाच}
{}


\twolineshloka
{असंस्कारो वैदिकश्च स मान्यः श्रोत्रियात्मजः}
{शुद्धान्वयः श्रोत्रियस्तु सुविद्वद्भिः समोऽन्यथा}


\twolineshloka
{अनधीयानपुत्राश्च वेदसंस्कारवर्जिताः}
{तस्मात्ते वेदविज्ञाऽपि विप्राः श्रुतिनिकारिणः}


\threelineshloka
{ब्रह्मराशौ पुरा सृष्टा वेदसंस्कारसंस्कृताः}
{तस्मात्तेष्वेव ते जाताः साधवः कुलधारिणः ॥मार्कण्डेय उवाच}
{}


\threelineshloka
{स्वयं क्रीतासु प्रेष्यासु प्रसूयन्ते तु ये नराः}
{कस्य नार्यः सुताश्चैव भवन्ति ऋषिसत्तम ॥नारद उवाच}
{}


\twolineshloka
{स्वदास्यां यो नरो मोहात्प्रसूयेत स पापकृत्}
{इहाभिनिन्दितः प्रेत्य अपत्यं प्रेष्यतां नयेत्}


\threelineshloka
{सा तस्य भार्या पुत्रा ये हव्यकव्यप्रदास्तु ते}
{तस्या ये बान्धवाः केचिद्विषक्ताः प्रेष्यतां गताः}
{सर्वे तस्यास्तु सम्बन्धा मुच्यन्ते प्रेष्यकर्मसु}


\threelineshloka
{एतत्ते कथितं सर्वं यदभिव्याहृतं त्वया}
{अथवा संशयः कश्चिद्भूयः सम्प्रष्टुमर्हसि ॥मार्कण्डेय उवाच}
{}


\threelineshloka
{अमिथ्यादर्शनालोके नारदः सर्वकोविदः}
{प्रत्यक्षदर्शी लोकानां स्वयंभुरिव सत्तमः ॥भीष्म उवाच}
{}


\threelineshloka
{इति सम्भाष्य ऋषिभिर्मार्कण्डेयो महातपाः}
{नारदं चापि सत्कृत्य तेन चैवाभिसत्कृतः}
{आमन्त्रयित्वा ऋषिभिः प्रययावाश्रमं मुनिः}


% Check verse!
ऋषयश्चापि तीर्थानां परिचर्यां प्रचक्रमुः
\twolineshloka
{सुक्षेत्रबीजसंस्कारविशुद्धो ब्रह्मिचर्यया}
{नित्यनैमित्तिकात्स्नातो मनश्शुद्ध्या च शुद्ध्यति ॥'}


\chapter{अध्यायः ६०}
\threelineshloka
{किमाहुर्भरतश्रेष्ठ पात्रं विप्राः सनातनम्}
{ब्राह्मणं लिङ्गिनं चैव ब्राह्मणं वाऽप्यलिङ्गिनम् ॥भीष्म उवाच}
{}


\threelineshloka
{स्ववृत्तिमभिपन्नाय लिङ्गिने चेतराय च}
{देयमाहुर्महाराज उभावेतौ तपस्विनौ ॥युधिष्ठिर उवाच}
{}


\threelineshloka
{श्रद्धया परयाऽपूतो यः प्रयच्छेद्द्विजातये}
{हव्यं कव्यं तथा दानं को दोषः स्यात्पितामह ॥भीष्म उवाच}
{}


\threelineshloka
{श्रद्धापूतो नरस्तात दुर्दान्तोऽपि न संशयः}
{पूतो भवति सर्वत्र किमुत त्वं महाद्युते ॥युधिष्ठिर उवाच}
{}


\threelineshloka
{न ब्राह्मणं परिक्षेत दैवेषु सततं नरः}
{कव्यप्रदाने तु बुधाः परीक्ष्यं ब्राह्मणं विदुः ॥भीष्म उवाच}
{}


\twolineshloka
{न ब्राह्मणः साधयते हव्यं दैवात्प्रसिद्ध्यति}
{देवप्रसादादिज्यन्ते यजमानैर्न संशयः}


\fourlineindentedshloka
{ब्राह्मणान्भरतश्रेष्ठ सततं ब्रह्मवादिनः}
{मार्कण्डेयः पुरा प्राह इति लोकेषु बुद्धिमान्}
{`ब्राह्मणाः पात्रभूताश्च शुद्धा नैवं पितृष्विह ॥युधिष्ठिर उवाच}
{}


\threelineshloka
{अपर्वोऽप्यथवा विद्वान्सम्बन्धी वा यथा भवेत्}
{तपस्वी यज्ञशीलो वा कथं पात्रं भवेत्तु सः ॥भीष्म उवाच}
{}


\twolineshloka
{कुलीनः कर्मकृद्वैद्यस्तथैवाप्यनृशंस्यवान्}
{ह्रीमानृजुः सत्यवादी पात्रं पूर्वे च ये त्रयः}


\twolineshloka
{तत्रेमं शृणु मे पार्थ चतुर्णां तेजसां मतम्}
{पृथिव्याः काश्यपस्याग्नेर्मार्कण्डेयस्य चैव हि}


\fourlineindentedshloka
{पृथिव्युवाच}
{यथा महार्णवे क्षिप्तः क्षिप्रं नेष्टुर्विनश्यति}
{तथा दुश्चरितं सर्वं त्रयीनित्ये निमज्जति ॥काश्यप उवाच}
{}


\threelineshloka
{सर्वे च वेदाः सह षङ्भिरङ्गैःसाङ्ख्यं पुराणं च कुले च जन्म}
{नैतानि सर्वाणि गतिर्भवन्तिशीलव्यपेतस्य नृप द्विजस्य ॥अग्निरुवाच}
{}


\threelineshloka
{अधीयानः पण्डितम्मन्यमानोयो विद्यया हन्ति यशः परेषाम्}
{ब्रह्मन्स तेन लभते ब्रह्मवध्यांलोकास्तस्य ह्यन्तवन्तो भवन्ति ॥मार्कण्डेय उवाच}
{}


\threelineshloka
{अश्वमेधसहस्रं च सत्यं च तुलया धृतम्}
{नाभिजानामि यज्ञं तु सत्यस्यार्धमवाप्नुयात् ॥भीष्म उवाच}
{}


\threelineshloka
{इत्युक्त्वा ते जग्मुराशु चत्वारोऽमिततेजसः}
{पृथिवी काश्यपोऽग्निश्च प्रकृष्टायुश्च भार्गवः ॥युधिष्ठिर उवाच}
{}


\threelineshloka
{यदि ते ब्राह्मणा लोके व्रतिनो भुञ्जते हविः}
{दत्तं ब्राह्मणकामाय कथं तत्सुकृतं भवेत् ॥भीष्म उवाच}
{}


\threelineshloka
{आदिष्टिनो ये राजेन्द्र ब्राह्मणा वेदपारगाः}
{भुञ्जते ब्रह्मकामाय व्रतलुप्ता भवन्ति ते ॥युधिष्ठिर उवाच}
{}


\threelineshloka
{अनेकान्तं बहुद्वारं धर्ममाहुर्मनीषिणः}
{किं निमित्तं भवेदत्र तन्मे ब्रूहि पितामह ॥भीष्म उवाच}
{}


\twolineshloka
{अहिंसा सत्यमकोध आनृशंस्यं दमस्तथा}
{आर्जवं चैव राजेन्द्र निश्चितं धर्मलक्षणम्}


\twolineshloka
{ये तु धर्मं प्रशंसन्तश्चरन्ति पृथिवीमिमाम्}
{अनाचरन्तस्तद्धर्म सङ्करेऽभिरता प्रभो}


\twolineshloka
{तेभ्यो हिरण्यं रत्नं वा गामश्वं वा ददाति यः}
{दशवर्षाणि विष्ठां स भुङ्क्ते निरयमास्थितः}


\twolineshloka
{धनेन पुल्कसानां च तथैवान्तेवसायिनाम्}
{कृतं कर्माकृतं वाऽपि रागमोहेन जल्पताम्}


\threelineshloka
{वैश्वदेवं च ये मूढा विप्राय ब्रह्मचारिणे}
{न ददन्तीह राजेन्द्र ते लोकान्भुञ्जतेऽशुभान् ॥युधिष्ठिर उवाच}
{}


\threelineshloka
{किं परं ब्रह्मचर्यं च किं परं धर्मलक्षणम्}
{किञ्च श्रेष्ठतमं शौचं तन्मे ब्रूहि पितामह ॥भीष्म उवाच}
{}


\threelineshloka
{ब्रह्मचर्यं परं तात मधुमांसस्य वर्जनम्}
{मर्यादायां स्थितो धर्मः शमः शौचस्य लक्षणम् ॥युधिष्ठिर उवाच}
{}


\threelineshloka
{कस्मिन्काले चरेद्धर्म कस्मिन्कालेऽर्थमाचरेत्}
{कस्मिन्काले सुखी च स्यात्तन्मे ब्रूहि पितामह ॥भीष्म उवाच}
{}


\twolineshloka
{काल्यमर्थं निषेवेत ततो धर्ममनन्तरम्}
{पश्चात्कामं निषेवेत न च गच्छेत्प्रसङ्गिताम्}


\twolineshloka
{ब्राह्मणांश्चैव मन्येत गुरूंश्चाप्यभिपूजयेत्}
{सर्वभूतानुलोमश्च मृदुशीलः प्रियंवदः}


\twolineshloka
{अधिकारे यदनृतं यच्च राजसु पैशुनम्}
{गुरोश्चालीकनिर्बन्धः समानि ब्रह्महत्यया}


\twolineshloka
{प्रहरेन्न नरेन्द्रेषु न हन्याद्गां तथैव च}
{भ्रूणहत्यासमं चैतदुभयं ये निषेधते}


\threelineshloka
{नाग्निं परित्यजेज्जातु न च वेदान्परित्यजेत्}
{न च ब्राह्मणमाक्रोशेत्समं तद्ब्रह्महत्यया ॥युधिष्ठिर उवाच}
{}


\threelineshloka
{कीदृशाः साधवो विप्राः केभ्यो दत्तं महाफलम्}
{कीदृशानां च भोक्तव्यं तन्मे ब्रूहि पितामह ॥भीष्म उवाच}
{}


\twolineshloka
{अक्रोधना धर्मपराः सत्यनित्या दमे रताः}
{तादृशाः साधवो विप्रास्तेभ्यो दत्तं महाफलम्}


\twolineshloka
{अमानिनः सर्वसहा दृढार्था विजितेन्द्रियाः}
{सर्वभूतहिता मैत्रास्तेभ्यो दत्तं महाफलम्}


\twolineshloka
{अलुब्धाः शुचयो वैद्या ह्रीमन्तः सत्यवादिनः}
{स्वकर्मनिरता ये च तेभ्यो दत्तं महाफलम्}


\twolineshloka
{साङ्गांश्च चतुरो वेदानधीते यो द्विजर्षभः}
{षड्भ्यः प्रवृत्तः कर्मभ्यस्तं पात्रमृषयो विदुः}


\twolineshloka
{ये त्वेवंगुणजातीयास्तेभ्यो दत्तं महाफलम्}
{सहस्रगुणमाप्नोति गुणार्हाय प्रदायकः}


\threelineshloka
{प्रज्ञाश्रुताभ्यां वृत्तेन शीलेन च समन्वितः}
{तारयेत कुलं सर्वमेकोऽपीह द्विजर्षभः}
{`तृप्ते तृप्ताः सर्वदेवाः पितरो मुनयोपि च ॥'}


\twolineshloka
{गामश्वं वित्तमन्नं वा तद्विधे प्रतिपादयेत्}
{द्रव्याणि चान्यानि तथा प्रेत्यभावे न शोचति}


\twolineshloka
{तारयेत कुलं सर्वमेकोपि ह द्विजोत्तमः}
{किमङ्ग पुनरेवैते तस्मात्पात्रं समाचरेत्}


\twolineshloka
{निशाम्य च गुणोपेतं ब्राह्मणं साधुसम्मतम्}
{दूरादानाय्य सत्कृत्य सर्वतश्चापि पूजयेत्}


\chapter{अध्यायः ६१}
\threelineshloka
{श्राद्धकाले च दैवे च पित्र्येऽपि च पितामह}
{इच्छामीह त्वयाऽऽख्यातं विहितं यत्सुरर्षिभिः ॥भीष्म उवाच}
{}


\twolineshloka
{दैवं पौर्वाह्णिके कुर्यादपराह्णे तु पैतृकम्}
{मङ्गलाचारसम्पन्नः कृतशौचः प्रयत्नवान्}


\twolineshloka
{मनुष्याणां तु मध्याह्ने प्रदद्यादुपपत्तिभिः}
{कालहीनं तु यद्दानं तं भागं रक्षसां विदुः}


\twolineshloka
{लङ्घितं चावलीढं च लाकपूर्वं च यत्कृतम्}
{रजस्वलाभिदृष्टं च तं भागं रक्षसां विदुः}


\twolineshloka
{अवघुष्टं च यद्भुक्तमव्रतेन च भारत}
{परामृष्टं शुना चैव तं भागं रक्षसां विदुः}


\twolineshloka
{केशकीटावपतितं क्षुतं श्वभिरवेक्षितम्}
{रुदितं चावधूतं च तं भागं रक्षसां विदुः}


\twolineshloka
{निरोङ्कारेण यद्भुक्तं सशस्त्रेण च भारत}
{दुरात्मना च यद्भुक्तं तं भागं रक्षसां विदुः}


\twolineshloka
{परोच्छिष्टं च यद्भुक्तं परिभुक्तं च यद्भवेत्}
{दैवे पित्र्ये च सततं तं भागं रक्षसां विदुः}


\twolineshloka
{मन्त्रहीनं क्रियाहीनं यच्छ्राद्धं परिविष्यते}
{त्रिभिर्वर्णैर्नरश्रेष्ठ तं भागं रक्षसां विदुः}


\twolineshloka
{आज्याहुतिं विना चैव यत्किञ्चित्परिविष्यते}
{दुराचारैश्च यद्भुक्तं तं भागं रक्षसां विदुः}


\twolineshloka
{ये भागा रक्षसां प्राप्तास्त उक्ता भरतर्षभ}
{अत ऊर्ध्वं विसर्गस्य परीक्षां ब्राह्मणे शृणु}


\twolineshloka
{यावन्तः पतिता विप्रा जडोन्मत्तास्तथैव च}
{दैवे वाऽप्यथ पित्र्ये वा राजन्नार्हन्ति केतनम्}


\twolineshloka
{श्वित्री क्लीबश्च कुष्ठी च तता यक्ष्महतश्च यः}
{अपस्मारी च यश्चान्धो राजन्नार्हन्ति केतनम्}


\twolineshloka
{चिकित्सका देवलका वृथा नियमधारिणः}
{सोमविक्रयिणश्चैव श्राद्धे नार्हन्ति केतनम्}


\twolineshloka
{गायना नर्तकाश्चैव प्लवका वादकास्तथा}
{कथका योधकाश्चैव राजन्नार्हन्ति केतनम्}


\twolineshloka
{होतारो वृषलानां च वृषलाध्यापकास्तथा}
{तथा वृषलशिष्याश्च राजन्नार्हन्ति केतनम्}


\twolineshloka
{अनुयोक्ता च यो विप्रो अनुयुक्तश्च भारत}
{नार्हतस्तावपि श्राद्धं ब्रह्मविक्रयिणौ हि तौ}


\twolineshloka
{अग्रणीर्यः कृतः पूर्वं वर्णावरपरिग्रहः}
{ब्राह्मणः सर्वविद्योऽपि राजन्नार्हति केतनम्}


\twolineshloka
{अनग्नयश्च ये विप्रा मृतनिर्यातकाश्च ये}
{स्तेनाश्च पतिताश्चैव राजन्नार्हन्ति केतनम्}


\twolineshloka
{अपरिज्ञातपूर्वाश्च गणपूर्वाश्च भारत}
{पुत्रिकापूर्वपुत्राश्च श्राद्धे नार्हन्ति केतनम्}


\twolineshloka
{ऋणकर्ता च यो राजन्यश्च वार्धुषिको नरः}
{प्राणिविक्रयवृत्तिश्च राजन्नार्हन्ति केतनम्}


\twolineshloka
{स्त्रीपूर्वाः काण्डपृष्ठाश्च यावन्तो भरतर्षभ}
{आजपा ब्राह्मणाश्चैव श्राद्धे नार्हन्ति केतनम्}


\twolineshloka
{श्राद्धे दैवे च निर्दिष्टो ब्राह्मणो भरतर्षभ}
{दातुः प्रतिग्रहीतुश्च शृणुष्वानुग्रहं पुनः}


\twolineshloka
{चीर्णव्रता गुणैर्युक्ता भवेयुर्येऽपि कर्षकाः}
{सावित्रीज्ञाः क्रियावन्तस्ते राजन्केतनक्षमाः}


\twolineshloka
{क्षात्रधर्मिणमप्याजौ केतयेत्कुलजं द्विजम्}
{न त्वेव वणिजं तात श्राद्धे च परिकल्पयेत्}


\twolineshloka
{अग्निहोत्री च यो विप्रो ग्रामवासी च यो भवेत्}
{अस्तेनश्चातिथिज्ञश्च स राजन्केतनक्षमः}


\twolineshloka
{सावित्रीं जपते यस्तु त्रिकालं भरतर्षभ}
{भिक्षावृत्तिः क्रियावांश्च स राजन्केतनक्षमः}


\twolineshloka
{उदितास्तमितो यश्च तथैवास्तमितोदितः}
{अहिंस्रश्चाल्पदोषश्च स राजन्केतनक्षमः}


\twolineshloka
{अकल्कको ह्यतर्कश्च ब्राह्मणो भरतर्षभ}
{संसर्ग भैक्ष्यवृत्तिश्च स राजन्केतनक्षमः}


\twolineshloka
{अव्रती कितवः स्तेनः प्राणिविक्रयिको वणिक्}
{सनिष्कृतिः पुनः सोमं पीतवान्केतनक्षमः}


\twolineshloka
{अर्जयित्वा धनं पूर्वं दारुणैरपि कर्मभिः}
{भवेत्सर्वातिथिः पश्चात्स राजन्केतनक्षमः}


\twolineshloka
{ब्रह्मविक्रयनिर्दिष्टं स्त्रिया यच्चार्जितं धनम्}
{अदेयं पितृविप्रेभ्यो यच्च क्लैब्यादुपार्जितम्}


\twolineshloka
{क्रियमाणेऽपवर्गे च यो द्विजो भरतर्षभ}
{न व्याहरति यद्युक्तं तस्याधर्मो गवानृतम्}


\twolineshloka
{श्राद्धस्य ब्राह्मणः कालः प्राप्तं दधि घृतं तथा}
{सोमक्षयश्च मांसं च यदारण्यं युधिष्ठिर}


\twolineshloka
{`मुहूर्तानां त्रयं पूर्वमह्नः प्रातरिति स्मृतम्}
{जपध्यानादिभिस्तस्मिन्विप्रैः कार्यं शुभव्रतम्}


\twolineshloka
{सङ्गवाख्यं त्रिभागं तु मध्याह्नस्त्रिमुहूर्तकः}
{लौकिकं सङ्गवेऽर्धं च स्नानादि ह्यथ मध्यमे}


\twolineshloka
{चतुर्थमपराह्णं तु त्रिमुहूर्तं तु पित्र्यकम्}
{सायाह्नस्त्रिमुहूर्तं च मध्यमं कविभिः स्मृतम्}


% Check verse!
चतुर्थ त्वपराह्णाख्ये श्राद्धं कुर्यात्सदा नृप
\twolineshloka
{प्रागुदीचीमुखा विप्राः विश्वेदेवे च दक्षिणाः}
{श्रावितेषु सुतृप्तेषु पिण्डं दद्यात्सदक्षिणम् ॥'}


\twolineshloka
{श्राद्धापवर्गे विप्रस्य दातारो वोस्त्वितीरयेत्}
{क्षत्रियस्यापि यो ब्रूयात्प्रीयन्तां पितरस्त्विति}


\twolineshloka
{अपवर्गे तु वैश्यस्य श्राद्धकर्मणि भारत}
{अक्षय्यमभिधातव्यं स्वस्ति शूद्रस्य भारत}


\twolineshloka
{पुण्याहवाचनं दैवं ब्राह्मणस्य विधीयते}
{एतदेव निरोङ्कारं क्षत्रियस्य विधीयते}


\twolineshloka
{वैश्यस्य दैवे वक्तव्यं प्रीयन्तां देवता इति}
{`गोर्हिसायां चतुर्भागं पूर्वं विप्रातिकेतिनः}


\twolineshloka
{वर्णावरेषु भुञ्जानं क्रमाच्छूद्रे चतुर्गुणम्}
{नान्यत्र ब्राह्मणो ब्रूयात्पूर्वं विप्रेण केतितः}


\twolineshloka
{अभोजने च दोषः स्याद्वर्जयेच्छूद्रकेतनम्}
{शूद्रान्नरसपुष्टाङ्गो द्विजो नोर्ध्वां गतिं लभेत्}


\twolineshloka
{अशुचिर्नैव चाश्नीयान्नास्तिको मानवर्जितः}
{न पूर्वं लङ्घयेल्लोभादेकवर्णोऽपि पार्थिव}


\twolineshloka
{विप्राः स्मृता भूमिदेवा उपकुर्वाणवर्जिताः}
{'कर्मणामानुपूर्व्येण विदिपूर्वं कृतं शृणु}


\twolineshloka
{जातकर्मादिकाः सर्वास्त्रिषु वर्णेषु भारत}
{ब्रह्मक्षत्रे हि मन्त्रोक्ता वैश्यस्य च युधिष्ठिर}


\twolineshloka
{विप्रस्य रशना मौञ्जी मौर्वी राजन्यगामिनी}
{बाल्वजी ह्येव वैश्यस्य धर्म एष युधिष्ठिर}


\twolineshloka
{`पालाशो द्विजदण्डः स्यादश्वत्थः क्षत्रियस्य तु}
{औदुम्बरश्च वैश्यस्य धर्म एष युधिष्ठिर ॥'}


\threelineshloka
{दातुः प्रतिग्रहीतुश्च धर्माधर्माविमौ शृणु}
{ब्राह्मणस्यानृतेऽधर्मः प्रोक्तः पातकसंज्ञितः}
{चतुर्गुणः क्षत्रियस्य वैश्यस्याष्टगुणः स्मृतः}


\twolineshloka
{नान्यत्र ब्राह्मणोऽश्नीयात्पूर्वं विप्रेण केतितः}
{[यवीयान्पशुहिंसायां तुल्यधर्मा भवेत्स हि}


\twolineshloka
{तथा राजन्यवैश्याभ्यां यद्यश्नीयात्तु केतितः}
{यवीयान्पशुहिंसायां भागार्धं समवाप्नुयात्}


\twolineshloka
{दैवं वाऽ************** पित्र्यंयोऽश्नीयाद्ब्राह्मणादिषु}
{]अस्नातो ब्राह्मणो राजंस्तस्याधर्मोऽनृतं स्मृतम्}


\twolineshloka
{आशौचो ब्राह्मणो राजन्योऽश्नीयाद्ब्राह्मणादिषु}
{ज्ञानपूर्वमथो लोभात्तस्याधर्मो गवानृतम्}


\twolineshloka
{अर्थेनान्येन यो लिप्सेत्कर्मार्थं चैव भारत}
{आमन्त्रयति राजेन्द्र तस्याधर्मोऽनृतं स्मृतम्}


\twolineshloka
{अवेदव्रतचारित्रास्त्रिभिर्वर्णैर्युधिष्ठिर}
{मन्त्रवत्परिविष्यन्ते तस्याधर्मो गवानृतम्}


\chapter{अध्यायः ६२}
\threelineshloka
{पित्र्यं वाऽप्यथवा दैवं दीयते यत्पितामह}
{एतदिच्छाम्यहं ज्ञातुं दत्तं केषु महाफलम् ॥भीष्म उवाच}
{}


\twolineshloka
{येषां दाराः प्रतीक्षन्ते सुवृष्टिमिव कर्षकाः}
{उच्छेषपरिशेषं हि तान्भोजय युधिष्ठिर}


% Check verse!
चारित्रनिरता राजन्ये कृशाः कृशवृत्तयः ॥अर्थिनश्चोपगच्छन्ति तेषु दत्तं महाफलम्
\twolineshloka
{तद्भक्तास्तद्गृहा राजंस्तद्बलास्तदपाश्रयाः}
{अर्थिनश्च भवन्त्यर्थे तेषु दत्तं महाफलम्}


\twolineshloka
{तस्करेभ्यः परेभ्यो वा हृतस्वा भयदुःखिताः}
{अर्थिनो भोक्तुमिच्छन्ति तेषु दत्तं महाफलम्}


\twolineshloka
{अकल्ककस्य विप्रस्य रौक्ष्यात्करकृतात्मनः}
{वटवो यस्य भिक्षन्ति तेभ्यो दत्तं महाफलम्}


\twolineshloka
{हृतस्वा हृतदाराश्च ये विप्रा देशसम्प्लवे}
{अर्थार्थमभिगच्छन्ति तेभ्यो दत्तं महाफलम्}


\twolineshloka
{व्रतोनो नियमस्थाश्च ये विप्राः श्रुतसम्मताः}
{तस्मपाप्त्यर्थमिच्छन्ति तेभ्यो दत्तं महाफलम्}


\twolineshloka
{अत्युत्क्रान्ताश्च धर्मेषु पाषण्डस्रमयेषु च}
{कृशप्राणाः कृशघनास्तेभ्यो दत्तं महाफलम्}


\twolineshloka
{कृतसर्वस्वहरणा निर्दोषाः प्रभविष्णुभिः}
{स्पृहयन्ति च भुक्त्वाऽन्नं तेषुदत्तं महाफलम्}


\twolineshloka
{तपस्विनस्तपोनिष्ठास्तेषां भैक्षचराश्च ये}
{अर्थिनः किञ्चिदिच्छन्ति तेषु दत्तं महाफलम्}


\twolineshloka
{महाफलविधिर्दाने श्रुतस्ते भरतर्षभ}
{निरयं येन गच्छन्ति स्वर्गं चैव हि तत्छृणु}


\twolineshloka
{`व्रतानां पारणार्थाय गुर्वर्थे यज्ञदक्षिणाम्}
{निर्वेशार्थं च विद्वांसस्तेषां दत्तं महाफलम्}


\twolineshloka
{पित्रोश्च रक्षणार्थाय पुत्रदारार्थमेव वा}
{महाव्याधिविमोक्षार्थं तेषु दत्तं महाफलम्}


\twolineshloka
{बालाः स्त्रियश्च वाञ्छन्ति सुभक्तं चाप्यसाधनाः}
{स्वर्गमायान्ति दत्त्वैषां निरयान्नोपयान्ति ते ॥'}


\twolineshloka
{गुर्वर्थमभयार्थं वा वर्जयित्वा युधिष्ठिर}
{येऽनृतं कथयन्ति स्म ते वा निरयगामिनः}


\twolineshloka
{परदाराभिहर्तारः परदाराभिमर्शिनः}
{परदारप्रयोक्तास्ते वै निरयगामिनः}


\twolineshloka
{ये परस्वापहर्तारः परस्वानां च नाशकाः}
{सूचकाश्च परेषां ये ते वा निरयगामिनः}


\twolineshloka
{प्रपाणां च सभानां च संक्रमाणां च भारत}
{अगाराणां च भेत्तारो नरा निरयगामिनः}


\twolineshloka
{अनाथां प्रमदां बालां वृद्धां भीतां तपस्विनीम्}
{वञ्चयन्ति नरा ये च ते वै निरयगामिनः}


\twolineshloka
{वृत्तिच्छेदं गृहच्छेदं दारच्छेदं च भारत}
{मित्रच्छेदं तथाऽऽशायास्ते वै निरयगामिनः}


\twolineshloka
{सूचकाः सेतुभेत्तारः परवृत्त्युपजीवकाः}
{अकृतज्ञाश्च मित्राणां ते वै निरयगामिनः}


\twolineshloka
{पाषण्डा दूषकाश्चैव समयानां च दूषकाः}
{ये प्रत्यवसिताश्चैव ते वै निरयगामिनः}


\twolineshloka
{विषमव्यवहाराश्च विषमाश्चैव वृद्धिषु}
{लाभेषु विषमाश्चैव ते वै निरयगामिनः}


\twolineshloka
{दूतसंव्यवहाराश्च निष्परीक्षाश्च मानवाः}
{प्राणिहिंसाप्रवृत्ताश्चि ते वै निरयगामिनः}


\twolineshloka
{कृताशं कृतनिर्देशं कृतभक्तं कृतश्रमम्}
{भेदैर्ये व्यपकर्षन्ति ते वै निरयगामिनः}


\twolineshloka
{पर्यश्नन्ति च ये दारानग्निभृत्यातिथींस्तथा}
{उत्सन्नपितृदेवेज्यास्ते वै निरयगामिनः}


\twolineshloka
{वेदविक्रयिणश्चैव वेदानां चैव दूषकाः}
{वेदानां लेखकाश्चैव ते वै निरयगामिनः}


\twolineshloka
{चातुराश्रम्यबाह्याश्च श्रुतिबाह्याश्च ये नराः}
{विकर्मभिश्च जीवन्ति ते वै निरयगामिनः}


\twolineshloka
{केशविक्रयिका राजन्विषविक्रयिकाश्च ये}
{क्षीरविक्रयिकाश्चैव ते वै निरयगामिनः}


\twolineshloka
{ब्राह्मणानां गवां चैव कन्यानां च युधिष्ठिर}
{येऽन्तरायान्ति कार्येषु ते वै निरयगामिनः}


\twolineshloka
{शस्त्रविक्रयिकाश्चैव कर्तारश्च युधिष्ठिर}
{शल्यानां धनुषां चैव ते वै निरयगामिनः}


\twolineshloka
{शिलाभिः शङ्कुभिर्वाऽपि श्वर्भ्रैर्वा भरतर्षभ}
{ये मार्गमनुरुन्धन्ति ते वै निरयगामिनः}


\twolineshloka
{उपाध्यायांश्च भृत्यांश्च भक्तांश्च भरतर्षभ}
{ये त्यजन्त्यविकारांस्त्रींस्ते वै निरयगामिनः}


\twolineshloka
{अप्राप्तदमकाश्चैव नासानां वेधकाश्च ये}
{बन्धकाश्च पशूनां ये ते वै निरयगामिनः}


\twolineshloka
{अगोप्तारश्च राजानो बलिषड्भागतस्कराः}
{समर्थाश्चाप्यदातारस्ते वै निरयगामिनः}


\twolineshloka
{क्षान्तान्दान्तांस्तथा प्राज्ञान्दीर्घकालं सहोषितान्}
{त्यजन्ति कृतकृत्या ये ते वै निरयगामिनः}


\twolineshloka
{बालानामथ वृद्धानां दासानां चैव ये नराः}
{अदत्त्वा भक्षयन्त्यग्रे ते वै निरयगामिनः}


\twolineshloka
{एते पूर्वं विनिर्दिष्टाः प्रोक्ता निरयगामिनः}
{भागिनः स्वर्गलोकस्य वक्ष्यामि भरतर्षभ}


\twolineshloka
{सर्वेष्वेव तु कार्येषु दैवपूर्वेषु भारत}
{हन्ति पुत्रान्पशून्कृत्स्नान्ब्राह्मणातिक्रमः कृतः}


\twolineshloka
{दानेन तपसा चैव सत्येन च युधिष्ठिर}
{ये धर्ममनुवर्तन्ते ते नराः स्वर्गगामिनः}


\twolineshloka
{शुश्रूषाभिस्तपोभिश्च विद्यामादाय भारत}
{ये प्रतिग्रहनिःस्नेहास्ते नराः स्वर्गगामिनः}


\twolineshloka
{भयात्पाषात्तथा बाधाद्दारिद्र्याद्व्याधिधर्षणात्}
{तत्कृते धनमीप्सन्ते ते नराः स्वर्गगामिनः}


\twolineshloka
{क्षमावन्तश्च धीराश्च धर्मकार्येषु चोत्थिताः}
{मङ्गलाचारसम्पन्नाः पुरुषाः स्वर्गगामिनः}


\twolineshloka
{निवृत्ता मधुमांसेभ्यः परदारेभ्य एव च}
{निवृत्ताश्चैव मद्येभ्यस्ते नराः स्वर्गगामिनः}


\twolineshloka
{आश्रमाणां च कर्तारः कुलानां चैव भारत}
{देशानां नगराणां च ते नराः स्वर्गगामिनः}


\twolineshloka
{वस्त्राभरणदातारो भक्ष्यपानान्नदास्तथा}
{कुटुम्बानां च भर्तारः पुरुषाः स्वर्गगामिनः}


\twolineshloka
{सर्वहिंसानिवृत्ताश्च नराः सर्वसहाश्च ये}
{सर्वस्याश्रयभूताश्च ते नराः स्वर्गगामिनः}


\twolineshloka
{मातरं पितरं चैव शुश्रुषन्ति जितेन्द्रियाः}
{भ्रातॄणां चैव सस्नेहास्ते नराः स्वर्गगामिनः}


\twolineshloka
{आढ्याश्च बलवन्तश्च यौवनस्थाश्च भारत}
{ये वै जितेन्द्रिया धीरास्ते नराः स्वर्गगामिनः}


\twolineshloka
{अपराधिषु सस्नेहा मृदवो मृदुवत्सलाः}
{आराधनसुखाश्चापि पुरुषाः स्वर्गगामिनः}


\twolineshloka
{सहस्रपरिवेष्टारस्तथैव च सहस्रदाः}
{त्रातारश्च सहस्राणां ते नराः स्वर्गगामिनः}


\twolineshloka
{सुवर्णस्य च दातारो गवां च भरतर्षभ}
{यानानां वाहनानां च ते नराः स्वर्गगामिनः}


\twolineshloka
{वैवाहिकानां द्रव्याणां प्रेष्याणां च युधिष्ठिर}
{दातारो वाससां चैव ते नराः स्वर्गगामिनः}


\twolineshloka
{विहारावसथोद्यानकूपारामसभाप्रपा}
{वप्राणां चैव कर्तारस्ते नराः स्वर्गगामिनः}


\twolineshloka
{निवेशनानां क्षेत्राणां वसतीनां च भारत}
{दातारः प्रार्थितानां च ते नराः स्वर्गगामिनः}


\twolineshloka
{रसानां चाथ बीजानां धान्यानां च युधिष्ठिर}
{स्वयमुत्पाद्य दातारः पुरुषाः स्वर्गगामिनः}


\twolineshloka
{यस्मिंस्तस्मिन्कुले जाता बहुपुत्राः शतायुषः}
{सानुक्रोशा जितक्रोधाः पुरुषाः स्वर्गगामिनः}


\twolineshloka
{एतदुक्तममुत्रार्थं दैवं पित्र्यं च भारत}
{दानधर्मं च दानस्य यत्पूर्वमृषितिः कृतम्}


\chapter{अध्यायः ६३}
\threelineshloka
{इदं मे तत्त्वतो राजन्वक्तुमर्हसि भारत}
{अहिंसयित्वाऽपि कथं ब्रह्महत्या विधीयते ॥भीष्म उवाच}
{}


\twolineshloka
{व्यासमामन्त्र्य राजेन्द्र पुरा यत्पृष्टवानहम्}
{तत्तेऽहं सम्प्रवक्ष्यामि तदिहैकमनाः शृणु}


\twolineshloka
{चतुर्थस्त्वं वसिष्ठस्य तत्त्वमाख्याहि मे मुने}
{अहिंसयित्वा केनेह ब्रह्महत्या विधीयते}


\twolineshloka
{इति पृष्टो मया राजन्पराशरशरीरजः}
{अब्रवीन्निपुणो धर्मे निःसंशयमनुत्तमम्}


\twolineshloka
{ब्राह्मणं स्वयमाहूय भिक्षार्थे कृशवृत्तिनम्}
{ब्रूयान्नास्तीति यः पश्चात्तं विद्याद्ब्रह्मघातिनम्}


\twolineshloka
{मध्यस्थस्येह विप्रस्य योऽनूचानस्य भारत}
{वृत्तिं हरति दुर्बुद्धिस्तं विद्याद्ब्रह्मघातिनम्}


\twolineshloka
{गोकुलस्य तृषार्तस्य जलार्थमभिधावतः}
{उत्पादयति यो विघ्नं तं विद्या** ह्मघातिनम्}


\twolineshloka
{यः प्रवृत्तां श्रुतिं सम्यकू शास्त्र वा मुनिभिः कृतम्}
{दूषयत्यनभिज्ञाय तं विद्याद्ब्रह्मघातिनम्}


\twolineshloka
{आत्मजां रूपसम्पन्नां महतीं सदृशे वरे}
{न प्रयच्छति यः कन्यां तं विद्याद्ब्रह्मधातिनम्}


\twolineshloka
{अधर्मनिरतो मूढो मिथ्या यो वै द्विजातिषु}
{दद्यान्मर्मातिगं शोकं तं विद्याद्ब्रह्मघातिनम्}


\twolineshloka
{चक्षुषा विप्रहीणस्य पङ्गुलस्य जडस्य वा}
{हरेत यो वै सर्वस्वं तं विद्या *** घातिनम्}


\twolineshloka
{आश्रमे वा वने वाऽपि ग्रामे वा यदि वा पुरे}
{अग्निं समुत्सृजेन्मोहात्तं विद्याद्ब्रह्मघातिनम्}


\chapter{अध्यायः ६४}
\threelineshloka
{तीर्थानां दर्शनं श्रेयः स्नानं च भरतर्षभ}
{श्रवणं च महाप्राज्ञ श्रोतुमिच्छामि तत्त्वतः}
{}


\threelineshloka
{पृथिव्यां यानि तीर्थानि पुण्यानि भरतर्षभ}
{वक्तुमर्हसि मे तानि श्रोताऽस्मि नियतं प्रभो ॥भीष्म उवाच}
{}


\twolineshloka
{इममङ्गिरसा प्रोक्तं तीर्थवंशं महाद्युते}
{श्रोतुमर्हसि भद्रं ते प्राप्स्यसे धर्ममुत्तमम्}


\threelineshloka
{तपोवनगतं विप्रमभिगम्य महामुनिम्}
{पप्रच्छाङ्गिरसं धीरं गौतमः संशितव्रतः}
{}


\twolineshloka
{अस्ति मे भगवन्कश्चित्तीर्थेभ्यो धर्मसंशयः}
{तत्सर्वं श्रोतुमिच्छामि तन्मे शंस महामुने}


\threelineshloka
{उपस्पृश्य फलं किं स्यात्तेषु तीर्थेषु वै मुने}
{प्रेत्यभावे महाप्राज्ञ तद्यथाऽस्ति तथा वद ॥अङ्गिरा उवाच}
{}


\twolineshloka
{सप्ताहं चन्द्रभागां वै वितस्तामूर्मिमालिनीम्}
{विगाह्य वै निराहारो निर्मलो मुनिवद्भवेत्}


\twolineshloka
{काश्मीरमण्डले नद्यो याः पतन्ति महास्वनम्}
{ता नदीः सिन्धुमासाद्य शीलवान्स्वर्गमाप्नुयात्}


\twolineshloka
{पुष्करं च प्रभासं च नैमिषं सागरोदकम्}
{देविकामिन्द्रमार्गं च स्वर्णबिन्दुं विगाह्य च}


\twolineshloka
{विबोध्यते विमानस्थः सोऽप्सरोभिरभिष्टुतः}
{हिरण्यबिन्दुमालक्ष्य प्रयतश्चाभिवाद्य च}


\twolineshloka
{कुशेशये च देवत्वं धूयते तस्य किल्बिषम्}
{इन्द्रतोयां समासाद्य गन्धमादनसन्निधौ}


\twolineshloka
{करतोयां कुरुङ्गे च त्रिरात्रोपोषितो नरः}
{अश्वमेधमवाप्नोति विगाह्य प्रयतः शुचिः}


\twolineshloka
{गङ्गाद्वारे कुशावर्ते बिल्वके नीलपर्वते}
{तथा कनखले स्नात्वा धूतपाप्मा दिवं व्रजेत्}


\twolineshloka
{अपां ह्रद उपस्पृश्य वाजिमेधफलं लभेत्}
{ब्रह्मचारी जितक्रोधः सत्यसन्धस्त्वहिंसकः}


\twolineshloka
{यत्र भागीरथी गङ्गा वहते दिशमुत्तरम्}
{महेश्वरस्य त्रिस्थाने यो नरस्त्वभिषिच्यते}


\twolineshloka
{एकमासं निराहारः स पश्यति हि देवताः}
{सप्तगङ्गे त्रिगङ्गे च इन्द्रमार्गे च तर्पयन्}


\twolineshloka
{अर्थान्वै लभते भोक्तुं यो नरो जायते पुनः}
{महाश्रम उपस्पृश्य योऽग्निहोत्रपरः शुचिः}


\twolineshloka
{एकमासं निराहारः सिद्धिं मासेन स व्रजेत्}
{महाह्रद उपस्पृश्य भृगुतुङ्गे त्वलोलुपः}


\twolineshloka
{त्रिरात्रोपोषितो भूत्वा मुच्यते ब्रह्महत्यया}
{कन्याकूप उपस्पृश्य बलाकायां कृतोदकः}


% Check verse!
देवेषु लभते कीर्तिं यशसा च विराजते
\twolineshloka
{देविकायामुपस्पृश्य तथा सुन्दरिकाह्रदे}
{अश्विन्यां रूपवर्चस्कं प्रेत्य वै लभते नरः}


\twolineshloka
{महागङ्गामुपस्पृश्य कृत्तिकाङ्गारके तथा}
{पक्षमेकं निराहारः स्वर्गमाप्नोति निर्मलः}


\twolineshloka
{वैमानिक उपस्पृश्य किङ्किणीकाश्रमे तथा}
{निवासेऽप्सरसां दिव्ये कामचारी महीयते}


\twolineshloka
{कालिकाश्रममासाद्य विपाशायां कृतोदकः}
{ब्रह्मचारी जितक्रोधस्त्रिरात्रं मुच्यते भवात्}


\twolineshloka
{आश्रमे कृत्तिकानां तु स्नात्वा यस्तर्पयेत्पितॄन्}
{तोषयित्वा महादेवं निर्मलः स्वर्गमाप्नुयात्}


\twolineshloka
{महाकूपमुपस्पृश्य त्रिरात्रोपोषितः शुचिः}
{त्रसानां स्थावराणां च द्विपदानां भयं त्यजेत्}


\twolineshloka
{देवदारुवने स्नात्वा धूतपाप्मा कृतोदकः}
{देवशब्दमवाप्नोति सप्तरात्रोषितः शुचिः}


\twolineshloka
{शरस्तम्बे कुशस्तम्बे द्रोणशर्मपदे तथा}
{अपांप्रपतनासेवी सेव्यते सोप्सरोगणैः}


\twolineshloka
{चित्रकूटे जनस्थाने तथा मन्दाकिनीजले}
{विगाह्य वै निराहारो राजलक्ष्म्या निषेव्यते}


\twolineshloka
{श्यामायास्त्वाश्रमं गत्वा उषित्वा चाभिषिच्य च}
{एकपक्षं निराहारस्त्वन्तर्धानफलं लभेत्}


\twolineshloka
{कौशिकीं तु समासाद्य वायुभक्षस्त्वलोलुपः}
{एकविंशतिरात्रेण स्वर्गमारोहते नरः}


\twolineshloka
{मतङ्गवाप्यां यः स्नायादेकरात्रेण सिध्यति}
{विगाहति ह्यनालम्बमन्धकं वै सनातनम्}


\twolineshloka
{नैमिषे स्वर्गतीर्थे च उपस्पृश्य जितेन्द्रियः}
{फलं पुरुषमेधस्य लभेन्मासं कृतोदकः}


\twolineshloka
{गङ्गाह्रद उपस्पृस्य तथा चैवोत्पलावने}
{अश्वमेघमवाप्नोति तत्र मासं कृतोदकः}


\twolineshloka
{गङ्गायमुनयोस्तीर्थे तथा कालञ्जरे गिरौ}
{दशाश्वमेधानाप्नोति तत्र मासं कृतोदकः}


% Check verse!
यष्टिह्रद उपस्पृश्य चान्नदानाद्विशिष्यते
\twolineshloka
{दशतीर्थसहस्राणि तिस्रः कोट्यस्तथाऽपराः}
{समागच्छन्ति माघ्यां तु प्रयागे भरतर्षभ}


\twolineshloka
{माघमासं प्रयागे तु नियतः संशितव्रतः}
{स्नात्वा तु भरतश्रेष्ठ निर्मलः स्वर्गमाप्नुयात्}


\twolineshloka
{मरुद्गण उपस्पृश्य पितॄणामाश्रमे शुचिः}
{वैवस्वतस्य तीर्थे च तीर्थभूतो भवेन्नरः}


\twolineshloka
{तथा ब्रह्मसरो गत्वा भागीरथ्यां कृतोदकः}
{एकमासं निराहारः सोमलोकमवाप्नुयात्}


\twolineshloka
{उत्पातके नरः स्नात्वा अष्टावक्रे कृतोदकः}
{द्वादशाहं निराहारो नरमेधफलं लभेत्}


\twolineshloka
{अश्मपृष्ठे गयायां च निरविन्दे च पर्वते}
{तृतीयां क्रौञ्चपद्यां च ब्रह्महत्यां विशुध्यते}


\twolineshloka
{कलविङ्क उपस्पृश्य विद्याच्च बहुशो जलम्}
{अग्नेः पुरे नरः स्नात्वा अग्निकन्यापुरे वसेत्}


\twolineshloka
{करवीरपुरे स्नात्वा विशालायां कृतोदकः}
{देवह्रद उपस्पृश्य ब्रह्मभूतो विराजते}


\twolineshloka
{पुनरावर्तनन्दां च महानन्दां च सेव्य वै}
{नन्दने सेव्यते दान्तस्त्वप्सरोभिरहिंसकः}


\twolineshloka
{उर्वशीं कृत्तिकायोगे गत्वा चैव समाहितः}
{लौहित्ये विधिवत्स्नात्वा पुण्डरीकफलं लभेत्}


\twolineshloka
{रामह्रद उपस्पृश्य विपाशायां कृतोदकः}
{द्वादशाहं निराहारः कल्मषाद्विप्रमुच्यते}


\twolineshloka
{महाह्रद उपस्पृश्य शुद्धेन मनसा नरः}
{एकमासं निराहारो जमदग्निगतिं लभेत्}


\twolineshloka
{विन्ध्ये संताप्य चात्मानं सत्यसन्धस्त्वहिंसकः}
{विनयात्तप आस्थाय मासेनैकेन सिध्यति}


\threelineshloka
{`मुण्डे ब्रह्मगवा चैव निरिचिं देवपर्वतम्}
{देवह्रदमुपस्पृश्य ब्रह्मभूतो विराजते}
{कुमारपदमास्थाय मासेनैकेन शुध्यति ॥'}


\twolineshloka
{नर्मदायामुपस्पृश्य तता शूर्पारकोदके}
{एकपक्षं निराहारो राजपुत्रो विधीयते}


\twolineshloka
{जम्बूमार्गे त्रिभिर्मासैः संयतः सुसमाहितः}
{अहोरात्रेण चैकेन सिद्धिं समधिगच्छति}


\twolineshloka
{कोकामुखे विगाह्याथ गत्वा चाञ्जलिकाश्रमम्}
{शाकभक्षश्चीरवासाः कुमारीर्विन्दते दश}


\twolineshloka
{वैवस्वतस्य सदनं न स गच्छेत्कदाचन}
{यस्य कन्याह्रद वासो देवलोकं स गच्छति}


\twolineshloka
{प्रभासे त्वेकरात्रेण अमावास्यां ममाहितः}
{सिद्ध्यते तु महाबाहो यो नरो जायतेऽमरः}


\twolineshloka
{उज्जानक उपस्पृस्य आर्ष्टिषेणस्य चाश्रमे}
{पिङ्गायाश्चाश्रमे स्नात्वा सर्वपापैः प्रमुच्यते}


\twolineshloka
{कुल्यायां समुपस्पृश्य जप्त्वा चैवाघमर्षणम्}
{अश्वमेधमवाप्नोति त्रिरात्रोपोषितो नरः}


\twolineshloka
{पिण्डारक उपस्पृश्य एकरात्रोषितो नरः}
{अग्निष्टोममवाप्नोति प्रभातां शर्वरीं शुचिः}


\twolineshloka
{तथा ब्रह्मसरो गत्वा धर्मारण्योपशोभितम्}
{पुण्डरीकमवाप्नोति उपस्पृश्य नरः शुचिः}


\twolineshloka
{मैनाके पर्वते स्नात्वा तथा सन्ध्यामुपास्य च}
{कामं जित्वा च वै मासं सर्वयज्ञफलं लभेत्}


\twolineshloka
{कालोदकं नन्दिकुण्डं तथा चोत्तरमानसम्}
{अभ्येत्य योजनशताद्भूणहा विप्रमुच्यते}


\twolineshloka
{नन्दीश्वरस्य मूर्ति तु दृष्ट्वा मुच्येत किल्बिषैः}
{स्वर्गमार्गे नरः स्नात्वा ब्रह्मलोकं स गच्छति}


\twolineshloka
{विख्यातो हिमवान्पुण्यः शंकरश्वशुरो गिरिः}
{आकरः सर्वरत्नानां सिद्धचारणसेवितः}


\twolineshloka
{`दर्शनाद्गमनात्पूतो भवेदनशनादपि}
{'शरीरमुत्सृजेत्तत्र विधिपूर्वमनाशके}


\threelineshloka
{अध्रुवं जीवितं ज्ञात्वा यो वै वेदान्तगो द्विजः}
{अभ्यर्च्य देवतास्तत्र नमस्कृत्य मुनींस्तथा}
{}


\twolineshloka
{ततः क्रोधं च लोभं च यो जित्वा तीर्थमावसेत्}
{न तेन किञ्चिन्न प्राप्तं तीर्थाभिगमनाद्भवेत्}


\twolineshloka
{यान्यगम्यानि तीर्थानि दुर्गाणि विषमाणि च}
{मनसा तानि गम्यानि सर्वतीर्थसमीक्षया}


\twolineshloka
{इदं मेध्यमिदं पुण्यमिदं स्वर्ग्यमनुत्तमम्}
{इदं रहस्यं वेदानामाप्लाव्यं पावनं तथा}


\threelineshloka
{इदं दद्याद््द्विजातीनां साधोरात्महितस्य च}
{सुहृदां च जपेत्कर्णे शिष्यस्यानुगतस्य च}
{}


\twolineshloka
{दत्तवान्गौतमस्यैतदङ्गिरा वै महातपाः}
{अङ्गिराः समनुज्ञातः काश्यपेन च धीमता}


\twolineshloka
{महर्षीणामिदं जप्यं पावनानां तथोत्तमम्}
{जपंश्चाभ्युत्थितः शश्वन्निर्मलः स्वर्गमाप्नुयात्}


\twolineshloka
{इदं यश्चापि शृणुयाद्रहस्यं त्वङ्गिरोमतम्}
{उत्तमे च कुले जन्म लभेञ्जातीश्च संस्मरेत्}


\chapter{अध्यायः ६५}
\twolineshloka
{बृहस्पतिसमं बुद्ध्या क्षमया ब्रह्मणः समम्}
{पराक्रमे शक्रसममादित्यसमतेजसम्}


\twolineshloka
{गाङ्गेयमर्जुनेनाजौ निहतं भूरितेजसम्}
{भ्रातृभिः सहितोऽन्यैश्च पर्यपृच्छद्युधिष्ठिरः}


\twolineshloka
{शयानं वीरशयने कालाकाङ्क्षिणमच्युतम्}
{आजग्मुर्भरतश्रेष्ठं द्रष्टुकामा महर्षयः}


\twolineshloka
{अत्रिर्वसिष्ठोऽथ भृगुः पुलस्त्यः पुलहः क्रतुः}
{अङ्गिरा गौतमोऽगस्त्यः सुमतिः सुयतात्मवान्}


\twolineshloka
{विश्वामित्रः स्थूलगिराः संवर्तः प्रमतिर्दमः}
{बृहस्पत्युशनोव्यासश्च्यवनः फाश्यपो ध्रुवः}


\twolineshloka
{दुर्वासा जमदग्निश्च मार्कण्डेयोऽथ गालवः}
{भरद्वाजोऽथ रैभ्यश्चक यवक्रीतस्त्रितस्तथा}


\twolineshloka
{स्थूलाक्षः शबलाक्षश्च कण्वो मेधातिथिः कृशः}
{नारदः पर्वतश्चैव सुधन्वाऽथैकतो द्विजः}


\threelineshloka
{नितंभूर्भुवनो धौम्यः शतानन्दोऽकृतव्रणः}
{जामदग्र्यस्तथा रामः कचश्चेत्येवमादयः}
{समागता महात्मानो भीष्मं द्रष्टुं महर्षयः}


\twolineshloka
{तेषां महात्मनां पूजामागतानां युधिष्ठिरः}
{भ्रातृभिः सहितश्चक्रे यतावदनुपूर्वशः}


\twolineshloka
{ते पूजिताः सुखासीनाः कथाश्रक्रुर्महर्षयः}
{भीष्माश्रिताः सुमधुराः सर्वेन्द्रियमनोहराः}


\twolineshloka
{भीष्मस्तेषां कथाः श्रुत्वा ऋषीणां भावितात्मनाम्}
{मेने दिविष्ठमात्मानं तुष्ट्या परमया युतः}


\twolineshloka
{ततस्ते भीष्ममामन्त्र्य पाण्डवांश्च महर्षयः}
{अन्तर्धानं गताः सर्वे सर्वेषामेव पश्यताम्}


\twolineshloka
{तानृषीन्सुमहाभागानन्तर्धानगतानपि}
{पाण्डवास्तुष्टुवुः सर्वे प्रणेमुश्च मुहुर्मुहुः}


\twolineshloka
{प्रसन्नमनसः सर्वे गाङ्गेयं कुरुसत्तमम्}
{उपतस्थुर्यथोद्यन्तमादित्यं मन्त्रकोविदाः}


\twolineshloka
{प्रभावं तपसस्तेषामृषीणां वीक्ष्य पाण्डवाः}
{प्रकाशन्तो दिशः सर्वा विस्मयं परमं ययुः}


\twolineshloka
{महाभाग्यं परं तेषामृषीणामनुचिन्त्य ते}
{पाण्डवाः सह भीष्मेण कथाश्चक्रुस्तदाश्रयाः}


\twolineshloka
{कथान्ते शिरसा पादौ स्पृष्ट्वा भीष्मस्य पाण्डवः}
{धर्म्यं धर्मसुतः प्रश्नं पर्यपृच्छद्युधिष्ठिरः}


\threelineshloka
{के देशाः के जनपदा आश्रमाः के च पर्वताः}
{प्रकृष्टाः पुण्यतः काश्च ज्ञेया नद्यः पितामह ॥भीष्म उवाच}
{}


\twolineshloka
{अत्राप्युदाहरन्तीममितिहासं पुरातनम्}
{शिलोच्छवृत्तेः संवादं सिद्धस्य च युधिष्ठिर}


\twolineshloka
{इमां कश्चित्परिक्रम्य पृथिवीं शैलभूषणाम्}
{असकृद्द्विपदां श्रेष्ठः श्रेष्ठस्य गृहमेधिनः}


\twolineshloka
{शिलवृत्तेर्गृहं प्राप्तः स तेन विधिनाऽर्चितः}
{उवास रजनीं तत्र सुमुखः सुखभागृषिः}


\twolineshloka
{शिलवृत्तिस्तु यत्कृत्यं प्रातस्तत्कृतवाञ्शुचिः}
{कृतकृत्यमुपातिष्ठत्सिद्धं तमतिथिं तदा}


\twolineshloka
{तौ समेत्य महात्मानौ सुखासीनौ कथाः शुभाः}
{चक्रतुर्वेदसम्बद्धास्तच्छेषकृतलक्षणाः}


\threelineshloka
{शिलवृत्तिः कथान्ते तु सिद्धमामन्त्र्य यत्नतः}
{प्रश्नं पप्रच्छ मेधावी यन्मां त्वं परिपृच्छसि ॥शिलवृत्तिरुवाच}
{}


\threelineshloka
{केदेशाः के जनपदाः केऽऽश्रमाः के च पर्वताः}
{प्रकृष्टाः पुण्यतः काश्च ज्ञेया नद्यस्तदुच्यताम् ॥सिद्ध उवाच}
{}


\twolineshloka
{ते देशास्ते जनपदास्तेऽऽश्रमास्ते च पर्वताः}
{येषां भागीरथी गङ्गा मध्येनैति रारिद्वरा}


\twolineshloka
{तपसा ब्रह्मचर्येण यज्ञैस्त्यागेन वा पुनः}
{गतिं तां न लभेज्जन्तुर्गङ्गां संसेव्य यां लभेत्}


\twolineshloka
{स्पष्टानि येषां गाङ्गेयैस्तोयैर्गात्राणि देहिनाम्}
{न्यस्तानि न पुनस्तेषां त्यागः स्वर्गाद्विधीयते}


\twolineshloka
{सर्वाणि येषां गाङ्गेयैस्तोयैः कार्याणि देहिनाम्}
{गां त्यक्त्वा मानवा विप्र दिवि तिष्ठन्ति ते जनाः}


\twolineshloka
{पूर्वे वयसि कर्माणि कृत्वा पापानि ये नराः}
{पश्चाद्गङ्गां निषेवन्ते तेऽपि यान्त्युत्तमां गतिम्}


\twolineshloka
{`युक्ताश्च पातकैस्त्यक्त्वा देहं शुद्धा भवन्ति ते}
{मुच्यन्ते देहसंत्यागाद्गङ्गायमुनसङ्गमे ॥'}


\twolineshloka
{स्नातानां शुचिभिस्तोयैर्गाङ्गेयैः प्रयतात्मनाम्}
{व्युष्टिर्भवति या पुंसां न सा क्रतुशतैरपि}


\twolineshloka
{यावदस्थि मनुष्यस्य गङ्गातोयेषु तिष्ठति}
{तावद्वर्षसहस्राणि स्वर्गलोके महीयते}


\twolineshloka
{अपहत्य तमस्तीव्रं यथा भात्युदये रविः}
{तथाऽपहत्य पाप्मानं भाति गङ्गाजलोक्षितः}


\twolineshloka
{विसोमा इव शर्वर्यो विपुष्पास्तरवो यथा}
{तद्वद्देशा दिशश्चैव हीना गङ्गाजलैः शिवैः}


\twolineshloka
{वर्णाश्रमा यथा सर्वे धर्मज्ञानविवर्जिताः}
{क्रतवश्च यथाऽसोमास्तथा गङ्गां विना जगत्}


\twolineshloka
{यथा हीनं नभोऽर्केण भूः शैलैः खं च वायुना}
{तथा देशा दिशश्चैव गङ्गाहीना न संशयः}


\twolineshloka
{त्रिषु लोकेषु ये केचित्प्राणिनः सर्व एव ते}
{तर्प्यमाणाः परां तृप्तिं यान्ति गङ्गाजलैः शुभैः}


\twolineshloka
{`अन्ये च देवा मुनयः प्रेतानि पितृभिः सह}
{तर्पितास्तृप्तिमायान्ति त्रिषु लोकेषु सर्वशः ॥'}


\twolineshloka
{यस्तु सूर्येण निष्टप्तं गाङ्गेयं पिबते जलम्}
{गवां निर्हरनिर्मुक्ताद्यावकात्तद्विशिष्यते}


\twolineshloka
{इन्द्रव्रतसहस्रं तु यश्चोरत्कायशोधनम्}
{पिवेद्यश्चापि गङ्गाम्यः समौ स्यातां न वा समौ}


\twolineshloka
{तिष्ठेद्युगसहस्रं तु पदेनैकेन यः पुमान्}
{मासमेकं तु गङ्गायां समौ स्यातां न वा सभौ}


\twolineshloka
{लम्बतेऽवाक्शिरा यस्तु युगानामयुतं पुमान्}
{तिष्ठेद्यथेष्टं यश्चापि गङ्गायां स विशिष्यते}


\twolineshloka
{अग्नौ प्रास्तं प्रधूयेत यथा तूलं द्विजोत्तम}
{तथा गङ्गावगाढस्य सर्वपापं प्रधूयते}


\twolineshloka
{भूतानामिह सर्वेषां दुःखोपहतचेतसाम्}
{गतिमन्वेषमाणानां न गङ्गासदृशी गतिः}


\twolineshloka
{भवन्ति निर्विषाः सर्पा यथा तार्क्ष्यस्य दर्शनात्}
{गङ्गाया दर्शनातद्वत्सर्वपापैः प्रमुच्यते}


\threelineshloka
{अप्रतिष्ठाश्च ये केचिदधर्मशरणाश्च ये}
{येषां प्रतिष्ठा गङ्गेह शरणं शर्म वर्म च}
{}


\twolineshloka
{प्रकृष्टैरशुभैर्ग्रस्ताननेकैः पुरुषाधमान्}
{पततो नरके गङ्गा संश्रितान्प्रेत्य तारयेत्}


\twolineshloka
{ते संविभक्ता मुनिभिर्नूनं देवैः सवासवैः}
{येऽभिगच्छन्ति सततं गङ्गां मतिमतांवर}


\twolineshloka
{विनयाचारहीनाश्च अशिवाश्च नराधमाः}
{ते भवन्ति शिवा विप्र ये वै गङ्गामुपाश्रिताः}


\twolineshloka
{यथा सुराणामभृतं पितॄणां च यथा स्वधा}
{सुधा यथा च नागानां तथा गङ्गाजलं नृणाम्}


\twolineshloka
{उपासते यथा बाला मातरं क्षुधयाऽर्दिताः}
{श्रेयस्कामास्तथा गङ्गामुपासन्तीह देहिनः}


\twolineshloka
{स्वायंभुवं यथा स्थानं सर्वेषां श्रेष्ठमुच्यते}
{स्थानानां सरितां श्रेष्ठा गङ्गा तद्वदिहोच्यते}


\twolineshloka
{`उपजीव्या यथा धेनुर्लोकानां ब्राह्ममेव वा}
{हविषां तु यथा सोमस्तरणेषु तथा त्वियम्}


\twolineshloka
{यथोपजीविनां धेनुर्देवादीनां परा स्मृता}
{तथोपजीविनां गङ्गा सर्वप्राणभृतामिह}


\twolineshloka
{देवाः सोमार्कसंस्थानि यता सत्रादिभिर्मखैः}
{अमृतान्युपजीवन्ति तथा गङ्गाजलं नराः}


\twolineshloka
{जाह्नवीपुलिनोत्थाभिः सिकताभिः समुक्षितम्}
{आत्मानं मन्यते लोको दिविष्ठमिव शोभितम्}


\twolineshloka
{जाह्नवीतीरसम्भूतां मृदं मूर्ध्ना बिभर्ति यः}
{बिभर्ति रूपं सोऽर्कस्य तमोनाशाय निर्मलम्}


\twolineshloka
{गङ्गोर्मिभिरथो दिग्धः पुरुषं पवनो यदा}
{स्पृश्यते सोऽस्य पाप्मानं सद्य एवापकर्षति}


\twolineshloka
{व्यसनैरभितप्तस्य नरस्य विनशिष्यतः}
{गङ्गादर्शनजा प्रीतिर्व्यसनान्यपकर्षति}


\twolineshloka
{हंसारावैः कोकरवै रवैरन्यैश्च पक्षिणाम्}
{पस्पर्ध गङ्गा गन्धर्वान्पुलिनैश्च शिलोच्चयान्}


\twolineshloka
{हंसादिभिः सुबहुभिर्विविधैः पक्षिभिर्वृताम्}
{गङ्गां गोकुलसम्बादां दृष्ट्वा स्वर्गोऽपि विस्मृतः}


\twolineshloka
{न सा प्रीतिर्दिविष्ठस्य सर्वकामानुपाश्नतः}
{सम्भवेद्या परा प्रीतिर्गङ्गायाः पुलिने नृणाम्}


\twolineshloka
{वाङ्मन कर्मजैर्ग्रस्तः पापैरपि पुमानिह}
{वीक्ष्य गङ्गां भवेन्मूतो अत्र मे नास्ति संशयः}


\twolineshloka
{सप्तावरान्सप्त परान्पितॄंस्तेभ्यश्च ये परे}
{पुमांस्तारयते गङ्गां वीक्ष्य स्पृष्ट्वाऽवगाह्य च}


\twolineshloka
{श्रुताभिलषिता पीता स्पृष्टा दृष्टाऽवगाहिता}
{गङ्गा तारयते नॄणामुभौ वंशौ विशेषतः}


\twolineshloka
{`तत्तीरगानां तपसा श्राद्धपारायणादिभिः}
{गङ्गाद्वारप्रभृतिभिस्तत्तीर्थैर्न परं नृणाम्}


\twolineshloka
{सायं प्रातः स्मरेद्गङ्गां नित्यं स्नाने तु कीर्तयेत्}
{तर्पणे पितृपूजासु मरणे चापि संस्मरेत् ॥'}


\twolineshloka
{दर्शनात्स्पर्शनात्पानात्तथा गङ्गेति कीर्तनात्}
{पुनात्युपुण्यान्पुरुषाञ्शतशोऽथ सहस्रशः}


\twolineshloka
{य इच्छेत्सफलं जन्म जीवितं श्रुतमेव च}
{स पितॄंस्तर्पयेद्गङ्गामभिगम्य सुरांस्तथा}


\twolineshloka
{न श्रुतैर्न च वित्तेन कर्मणा न च तत्फलम्}
{प्राप्नुयात्पुरुषोऽत्यन्तं गङ्गां प्राप्य यदाप्नुयात्}


\twolineshloka
{जात्यन्धैरिह तुल्यास्ते मृतैः पङ्गुभिरेव च}
{समर्था ये न पश्यन्ति गङ्गां पुण्यजलां शिवाम्}


\twolineshloka
{भूतभव्यभविष्यज्ञैर्महर्षिभिरुपस्थिताम्}
{देवैः सेन्द्रैश्च को गङ्गां नोपसेवेत मानवः}


\twolineshloka
{वानप्रस्थैर्गृहस्थैश्च यतिभिर्ब्रह्मचारिभिः}
{विद्यावद्भिः श्रितां गङ्गां पुमान्को नाम नाश्रयेत्}


\twolineshloka
{उत्कामद्बिश्च यः प्राणैः प्रयतः शिष्टसम्मतः}
{चिन्तयेन्मनसा गङ्गां स गतिं परमांलभेत्}


\twolineshloka
{न भयेभ्यो भयं तस्य न पापेभ्यो न राजतः}
{आदेहपतनाद्गङ्गामुपास्ते यः पुमानिह}


\twolineshloka
{गगनाद्गां पतन्तीं वै महापुण्यां महेश्वरः}
{दधार शिरसा गङ्गां तामेव दिवि सेवते}


\threelineshloka
{अलङ्कृतास्त्रयो लोकाः पथिभिर्विमलैस्त्रिभिः}
{यस्तु तस्या जलं सेवेत्क्रतकृत्यः पुमान्भवेत्}
{}


\twolineshloka
{दिवि ज्योतिर्यथाऽऽदित्यः पितॄणां चैव चन्द्रमाः}
{देवेश यथा नृणां गङ्गा च सरितां तथा}


\twolineshloka
{मात्रा पित्रा सुतैर्दारैर्विमुक्तस्य धनेन वा}
{न भवेद्धि तथा दुःखं यथा गङ्गा वियोगजम्}


\twolineshloka
{नारण्यैर्नेष्टविषयैर्न सुतैर्न धनागमैः}
{तथा प्रसादो भवति गङ्गां वीक्ष्य यथा भवेत्}


\twolineshloka
{पूर्णमिन्दुं यथा दृष्ट्वा नृणां दृष्टिः प्रसीदति}
{तथा त्रिपथगां दृष्ट्वा नृणां दृष्टिः प्रसीदति}


\twolineshloka
{तद्भावस्तद्गतमनास्तन्निष्ठस्तत्परायणः}
{गङ्गांयोऽनुगतो भक्त्यास तस्याः प्रियतां व्रजेत्}


\twolineshloka
{भूस्थैः खस्थैर्दिविष्ठैश्च भूतैरुच्चावचैरपि}
{गङ्गा विगाह्या सततमेतत्कार्यतमं सताम्}


\threelineshloka
{विश्वलोकेषु पुण्यत्वाद्गङ्गायाः प्रथितं यशः}
{`दुर्मृताननपत्यांश्च सा मृताननयद्दिवम्}
{'यत्पुत्रान्सगरस्येतो भस्माख्याननयद्दिवम्}


\twolineshloka
{वाय्वीरिताभिः सुमनोहराभि-र्द्रुताभिरत्यर्थसमुत्थिताभिः}
{गङ्गोर्मिभिर्भानुमतीभिरिद्धाःसहस्ररश्मिप्रतिमा भवन्ति}


\twolineshloka
{पयस्विनीं घृतिनीमत्युदारांसमृद्धिनीं वेगिनीं दुर्विगाह्याम्}
{गङ्गां गत्वा यैः शरीरं विसृष्टंगता धीरस्ते विबुधैः समत्वम्}


\twolineshloka
{अन्धाञ्चडान्द्रव्यहीनांश्च गङ्गायशस्विनी बृहती विश्वरूपा}
{देवैः सेन्द्रैर्मुनिभिर्मानवैश्चनिषेविता सर्वकामैर्युनक्ति}


\twolineshloka
{ऊर्जस्वतीं मधुमतीं महापुण्यां त्रिवर्त्मगाम्}
{त्रिलोकगोप्त्रीं ये गङ्गां संश्रितास्ते दिवं गताः}


\twolineshloka
{यो वत्स्यति द्रक्ष्यति वाऽपि मर्त्य-स्मस्मै प्रयच्छन्ति सुखानि देवाः}
{तद्भाविताः स्पर्शनदर्शनेनइष्टां गतिं तस्य सुरा दिशन्ति}


\twolineshloka
{दक्षां पृश्निं बृहतीं विप्रकृष्टांशिवामृद्धां भागिनीं सुप्रसन्नाम्}
{विभावरीं सर्वभूतप्रतिष्ठांगङ्गां गता ये त्रिदिवं गतास्ते}


\twolineshloka
{ख्यातिर्यस्याः खं दिवं गां च नित्या-मूर्ध्वं दिशो विदिशश्चावतस्थे}
{तस्या जलं सेव्य सरिद्वरायामर्त्याः सर्वे कुतकृत्या भवन्ति}


\twolineshloka
{इयं गङ्गेति नियतं प्रतिष्ठागुहस्य स्क्मस्य च गर्भयोषा}
{प्रातस्त्रिवर्गा घृतवहा विपाप्मागङ्गाऽवतीर्णा वियतो विश्वतोया}


\threelineshloka
{`नारायणादक्षयात्पूर्वजाताविष्णोः पादाच्छिंशुमाराद्ध्रुवाच्च}
{सोमात्सूर्यान्मेरुरूपाच्च विष्णोःसमागता शिवमूर्ध्नो हिमाद्रिम्}
{सत्यावती द्रव्यपरस्य वर्यादिवो भुवश्चापि वीक्ष्यानुरूपा ॥'}


\twolineshloka
{सुताऽवनीध्रस्य हरस्य भार्यादिवो भुवश्चापि कृतानुरूपा}
{भव्या पृथिव्यां भागिनी चापि राज-न्गङ्गा लोकानां पृण्यदा वै त्रयाणाम्}


\twolineshloka
{मधुस्रवा घृतधारा घृतार्चि-र्महोर्मिभिः शोभिता ब्राह्मणैश्च}
{दिवश्च्युता शिरसाऽऽप्ता शिवेनगङ्गाऽवनीध्रात्त्रिदिवस्य माता}


\twolineshloka
{योनिर्वरिष्ठा विरजा वितन्वीशय्याऽचिरा वारिवहा यशोदा}
{विश्वावती चाकृतिरिष्टसिद्धागङ्गोक्षितानां भुवनस्य पन्थाः}


\twolineshloka
{क्षान्त्या मह्या गोपने धारणे चदीप्त्य, कृशानोस्तपनस्य चैव}
{तुल्या गङ्गा सम्मता ब्राह्मणानांगृहस्य ब्रह्मण्यतया च नित्यम्}


\twolineshloka
{ऋषिष्टुतां विष्णुपदीं पुराणांसुपुण्यतोयां मनसाऽपि लोके}
{सर्वात्मना जाह्नवीं ये प्रवन्नास्ते ब्रह्मणः सदनं सम्प्रयाताः}


\twolineshloka
{लोकानिमान्नयति या जननीव पुत्रा-न्सर्वात्मना सर्वगुणोपपन्ना}
{तत्स्थानकं ब्राह्ममभीप्समानै-र्गङ्गा सदैवात्मवशैरुपास्या}


\threelineshloka
{`न तैर्जुष्टां स्पृशतीं विश्वतोया-मिरावज्ञां रवेतीं भूधराणाम्}
{'उसां पृष्टां मिषतीं विश्वभोज्या-मिरावतीं धारिणीं भूधराणाम्}
{शिष्टाश्रयाममृतां ब्रह्मकान्तांगंङ्गां श्रयेदात्मवान्सिद्धिकामः}


\twolineshloka
{प्रसाद्य देवान्सविभून्समस्ता-न्भगीरथस्तपसोग्रेण गङ्गाम्}
{गामानयत्तामभिगम्य शश्व-त्पुंसां भयं नेह चामुत्र विद्यात्}


\twolineshloka
{उदाहृतः सर्वथा ते गुणानांमयैकदेशः प्रसमीक्ष्य बृद्ध्या}
{शक्तिर्न से काचिदिहास्ति वक्तुंगुणान्सर्वान्परिमातुं तथैव}


\twolineshloka
{मेरोः समुद्रस्य च सर्वयत्नैःसङ्ख्योपलानामुदकस्य वाऽपि}
{शक्यं वक्तुं नेह गङ्गाजलानांगुणाख्यानं परिमातुं तथैव}


\twolineshloka
{तस्मादेतान्परया श्रद्धयोक्ता-न्गुणान्सर्वाञ्जाह्नवीयान्सदैव}
{भवेद्वाचा मनसा कर्मणा चभक्त्या युक्तः श्रद्धया श्रद्दधानः}


\twolineshloka
{लोकानिमांस्त्रीन्यशसा वितत्यसिद्धिं प्राप्य महतीं तां दुरापाम्}
{गङ्गाकृतानचिरेणैव लोका-न्यथेष्टमिष्टान्विहरिष्यसि त्वम्}


\threelineshloka
{तव मम च गुणैर्महानुभावाजुषतु भतिं सततं स्वधर्मयुक्तैः}
{अभिमतजनवत्सला हि गङ्गाजगति युनक्ति सुखैश्च भक्तिमन्तम् ॥भीष्म उवाच}
{}


\twolineshloka
{इति परममतिर्गुणानशेषा-ञ्शिलरतये त्रिपथानुयोगरूपान्}
{बहुविधमनुशास्य तथ्यरूपा-न्गगनतलं द्युतिमान्विवेश सिद्धः}


\twolineshloka
{शिलवृत्तिस्तु सिद्धस्य वाक्यैः सम्बोधितस्तदा}
{गङ्गामुपास्य विधिवत्सिद्धिं प्राप सुदुर्लभाम्}


\threelineshloka
{तथा त्वमपि कौन्तेय भक्त्या परमया युतः}
{गङ्गामभ्येहि सततं प्राप्स्यसे सिद्धिमुत्तमाम् ॥वैशम्पायन उवाच}
{}


\twolineshloka
{श्रुत्वेतिहासं भीष्मोक्तं गङ्गायाः स्तवसंयुतम्}
{युधिष्ठिरः परां प्रीतिमगच्छद्धातृभिः सह}


\chapter{अध्यायः ६६}
\threelineshloka
{के पूज्या वै त्रिलोकेऽस्मिन्मानवा भरतर्पभ}
{विस्तरेण तदाचक्ष्व न हि तृप्यामि कथ्यतः ॥भीष्म उवाच}
{}


\twolineshloka
{अत्राप्युदाहरन्तीममितिहासं पुरातनम्}
{नारदस्य च संवादं वासुदेवस्य चोभयोः}


\twolineshloka
{नारदं प्राञ्जलिं दृष्ट्वा पूजयानं द्विजर्षभान्}
{केशवः परिपप्रच्छ भगवन्क्रान्नमस्यमि}


\threelineshloka
{बहुमानपरस्तेषु भगवन्यान्नमस्यसि}
{शक्यं चेच्छ्रोतुमस्माभिर्ब्रूह्येतद्धर्मवित्तम ॥नारद उवाच}
{}


\twolineshloka
{शृणु गोविन्द यानेतान्पूजयाम्यरिमर्दन}
{त्वत्तोऽन्यः कः पुमाँल्लोके श्रोतुमेतदिहार्हति}


\twolineshloka
{वरुणं वायुमादित्यं पर्जन्यं जातवेदसम्}
{स्थाणु स्कन्दं महालक्ष्मीं विष्णुं ब्रह्माणमेव च}


\twolineshloka
{वाचस्पतिं चन्द्रमसमपः पृथ्वीं सरस्वतीम्}
{सततं ये नमस्यन्ति तान्नमस्याम्यहं विभो}


\twolineshloka
{तपोधनान्वेदविदो नित्यं वेदपरायणान्}
{महार्हान्वृष्णिशार्दूल सदा सम्पूजयाम्यहम्}


\twolineshloka
{अभुक्त्वा देवकार्याणि कुर्वते येऽविकत्थनाः}
{सन्तुष्टाश्च क्षमायुक्तास्तान्नमस्याम्यहं विभो}


\twolineshloka
{सम्यग्यजन्ति ये चेष्टीः क्षान्ता दान्ता जितेन्द्रियाः}
{सत्यं धर्मं क्षितिं गाश्च तान्नमस्यामि यादव}


\twolineshloka
{ये वै तपसि वर्तन्ते वने मूलफलाशनाः}
{असञ्चयाः क्रियावन्तस्तान्नमस्यामि यादव}


\twolineshloka
{ये भृत्यभरणे शक्ताः सततं चातिथिव्रताः}
{भुञ्जते देवशेषाणि तान्नमस्यामि यादव}


\twolineshloka
{ये वेद प्राप्य दुर्धर्षा वाग्मिनो ब्रह्मचारिणः}
{याजनाध्यापने युक्ता नित्यं तान्पूजयाम्यहम्}


\twolineshloka
{प्रसन्नहृदयाश्चैव सर्वसत्वेषु नित्यशः}
{अपृष्ठतापान्स्वाध्याये युक्तास्तान्पूजयाम्यहम्}


\twolineshloka
{गुरुप्रसादे स्वाध्याये यतन्तो ये स्थिरव्रताः}
{शुश्रूषवोऽनसूयन्तस्तान्नमस्यामि यादव}


\twolineshloka
{सुव्रता मुनयो ये च ब्राह्मणाः सत्यसङ्गराः}
{वोढारो हव्यकव्यानां तान्नमस्यामि यादव}


\twolineshloka
{भैक्षचर्यासु निरताः कृशा गुरुकुलाश्रयाः}
{निःसुखा निर्धना ये तु तान्नमस्यामि यादव}


\twolineshloka
{निर्ममा निष्प्रतिद्वन्द्वा निष्ठिता निष्प्रयोजनाः}
{ये वेदं प्राप्य दुर्धर्षा वाग्मिनो ब्रह्मवादिनः}


\twolineshloka
{अहिंसानिरता ये च ये च सत्यव्रता नराः}
{दान्ताः शमपराश्चैव तान्नमस्यामि केशव}


\twolineshloka
{देवतातिथिपूजायां युक्ता ये गृहमेधिनः}
{कपोतवृत्तयो नित्यं तान्नमस्यामि यादव}


\twolineshloka
{येषां त्रिवर्गः कृत्येषु वर्तते नोपहीयते}
{शिष्टाचारप्रवृत्ताश्च तान्नमस्याम्यहं सदा}


\twolineshloka
{ब्राह्मणाः श्रुतसम्पन्ना ये त्रिवर्गमनुष्ठिताः}
{अलोलुपाः पुण्यशीलास्तान्नमस्यामि केशव}


\twolineshloka
{`अवन्ध्यकाला येऽलुब्धास्त्रिवर्गे साधनेषु च}
{विशिष्टाचारयुक्ताश्च नारायण नमामि तान् ॥'}


\twolineshloka
{अब्भक्षा वायुभक्षाश्चक सुधाभक्षाश्च ये सदा}
{व्रतैश्च विविधैर्युक्तास्तान्नमस्यामि माधव}


\twolineshloka
{अयोनीनग्नियोनींश्च ब्रह्मयोनींस्तथैव च}
{सर्वभूतात्मयोनींश्च तान्नमस्याम्यहं सदा}


\twolineshloka
{नित्यमेतान्नमस्यामि कृष्ण लोककरानृषीन्}
{लोकज्येष्ठान्कुलज्येष्ठांस्तमोघ्नँल्लोकभास्करान्}


\twolineshloka
{तस्मात्त्वमपि वार्ष्णेय द्विजान्पूजय नित्यदा}
{पूजिताः पूजनार्हा हि सुखं दास्यन्ति तेऽनघ}


\twolineshloka
{अस्मिँल्लोके सदा ह्येते परत्र च सुखप्रदाः}
{चरन्ते मान्यमाना वै प्रदास्यन्ति सुखं तव}


\twolineshloka
{ये सर्वातिथयो नित्यं गोषु च ब्राह्मणेषु च}
{नित्यं सत्ये चाभिरता दुर्गाण्यतितरन्ति ते}


\twolineshloka
{नित्यं शमपरा ये च तथा ये चानसूयकाः}
{नित्यस्वाध्यायिनो ये च दुर्गाण्यतितरन्ति ते}


\twolineshloka
{सर्वान्देवान्नमस्यन्ति य चैकं वेदमाश्रिताः}
{श्रद्दधानाश्च दान्ताश्च दुर्गाण्यतितरन्ति ते}


\twolineshloka
{तथैव विप्रप्रवरान्नमस्कृत्य यतव्रताः}
{भवन्ति ये दानरता दुर्गाण्यतितरन्ति ते}


\twolineshloka
{तपस्विनश्च ये नित्यं कौमारब्रह्मचारिणः}
{तपसा भावितात्मानो दुर्गाण्यतितरन्ति ते}


\twolineshloka
{देवतातिथिभृत्यानां पितॄणां चार्चने रताः}
{शिष्टान्नभोजिनो ये च दुर्गाण्यतितरन्ति ते}


\twolineshloka
{अग्निमाधाय विधिवत्प्रणता धारयन्ति ये}
{प्राप्तः सोमाहुतिं चैव दुर्गाण्यतितरन्ति ते}


\twolineshloka
{मातापित्रोर्गुरुषु च सम्यग्वर्तन्ति ये सदा}
{यथा त्वं वृष्णिशार्दूलेत्युक्त्वैवं विरराम सः}


\twolineshloka
{तस्मात्त्वमपि कौन्तेय पितृदेवद्विजातिथीन्}
{सम्यक्पूजयसे नित्यं गतिमिष्टामवाप्स्यसि}


\chapter{अध्यायः ६७}
\twolineshloka
{पितामह महाप्राज्ञ सर्वशास्त्रविशारद}
{त्वत्तोऽहं श्रोतुमिच्छामि धर्मं भरतसत्तम}


\threelineshloka
{शरणागतं ये रक्षन्ति भूतग्रामं चतुर्विधम्}
{किं तस्य भरतश्रेष्ठ फलं भवति तत्त्वतः ॥भीष्म उवाच}
{}


\twolineshloka
{इदं शृणु महाप्राज्ञ धर्मपुत्र महायशः}
{इतिहासं पुरावृत्तं शरणार्थं महाफलम्}


\twolineshloka
{प्रपात्यमानः श्येनेन कपोतः प्रियदर्शनः}
{वृषदर्भं महाभागं नरेन्द्रं शरणं गतः}


\twolineshloka
{स तं दृष्ट्वा विशुद्धात्मा त्रासादङ्कमुपागतम्}
{आश्वास्याश्वसिहीत्याह न तेऽस्ति भयमण्डज}


\twolineshloka
{भयं ते सुमहत्कस्मात्कुत्र किं वा कृतं त्वया}
{येन त्वमिह सम्प्राप्तो विसंज्ञो भ्रान्तचेतनः}


\twolineshloka
{नवनीलोत्पलापीड चारुवर्ण सुदर्शन}
{दाडिमाशोकपुष्पाक्ष मा त्रसस्वाभयं तव}


\twolineshloka
{मत्सकाशमनुप्राप्तं न त्वां कश्चित्समुत्सहेत्}
{मनसा ग्रहणं कर्तुं रक्षाध्यक्षपुरस्कृतम्}


\threelineshloka
{काशिराज्यं तदद्यैव त्वदर्तं जीवितं तथा}
{त्यजेयं भव विस्रब्धः कपोत भयं तव ॥श्येन उवाच}
{}


\twolineshloka
{ममैतद्विहितं भक्ष्यं न राजंस्त्रातुमर्हसि}
{अतिक्रान्तं च प्राप्तं च प्रयत्नाच्चोपपादितम्}


\twolineshloka
{मांसं च रुधइरं चास्य मज्जा मेदश्च मे हितम्}
{परितोषकरो ह्येष मम माऽस्याग्रतो भव}


\twolineshloka
{तृष्णा मे बाधतेऽत्युग्रा क्षुधा निर्दहतीव माम्}
{मुञ्चैनं नहि शक्ष्यामि राजन्मन्दयितुं क्षुधाम्}


\twolineshloka
{मया ह्यनुसृतो ह्येष मत्पक्षनखविक्षतः}
{किञ्चिदुच्छ्वासनिःश्वासं न राजन्गोप्तुमर्हसि}


\twolineshloka
{यदि स्वविषये राजन्प्रभुस्त्वं रक्षणे नृणाम्}
{खेचरस्य तृषार्तस्य न त्वं प्रभुरथोत्तम}


\twolineshloka
{यदि वैरिषु भृत्येषु स्वजनव्यवहारयोः}
{विषयेष्विन्द्रियाणां च आकाशे मा पराक्रम}


\threelineshloka
{प्रभुत्वं हि पराक्रम्य सम्यक् पक्षहरेषु ते}
{यदि त्वमिह धर्मार्थी मामपि द्रष्टुमर्हसि ॥भीष्म उवाच}
{}


\threelineshloka
{श्रुत्वा श्येनस्य तद्वाक्यं राजर्षिर्विस्मयं गतः}
{सम्भाव्य चैनं तद्वाक्यं तदर्थी प्रत्यभाषत ॥राजोवाच}
{}


\twolineshloka
{गोवृषो वा वराहो वा मृगो वा महिषोपि वा}
{त्वदथर्मद्य क्रियतां क्षुधाप्रशमनाय ते}


\threelineshloka
{शरणागतं न त्यजेयमिति मे व्रतमाहितम्}
{न मुञ्चति ममाङ्गानि द्विजोऽयं पश्य वै द्विज ॥श्येन उवाच}
{}


% Check verse!
न वराहं न चोक्षाणं न चान्यान्विविधान्द्विजान्भक्षयामि महाराज किमन्नाद्येन तेन मे
\twolineshloka
{यस्तु मे विहितो भक्ष्यः स्वयं देवैः सनातनः}
{श्येनाः कपोतान्खादन्ति स्तितिरेषा सनातनी}


\threelineshloka
{उशीनर कपोते तु यदि स्नेहस्तवानघ}
{ततस्त्वं मे प्रयच्छाद्य स्वमांसं तुलया धृतम् ॥राजोवाच}
{}


\twolineshloka
{महाननुग्रहो मेऽद्य यस्त्वमेवमिहात्थ माम्}
{बाढमेव करिष्यामीत्युक्त्वाऽसौ राजसत्तमः}


\twolineshloka
{उत्कृत्योत्कृत्य मांसानि तुलया समतोलयत्}
{अन्तःपुरे ततस्तस्य स्त्रियो रत्नविभूषिताः}


\twolineshloka
{हाहाभूता विनिष्क्रान्ताः श्रुत्वा परमदुःखिताः}
{तासां रुदितशब्देन मन्त्रिभृत्यजनस्य च}


\twolineshloka
{बभूव सुमहान्नादो मेघगम्भीरनिःस्वनः}
{निरुद्धं गगनं सर्वं व्यभ्रं मेघैः समन्ततः}


% Check verse!
मही प्रचलिता चासीत्तस्य सत्येन कर्मणा
\threelineshloka
{स राजा पार्श्वतश्चैव बाहुभ्यामूरुतश्च यत्}
{तानि मांसानि सञ्छिद्य तुलां पूरयतेऽशनैः}
{तथापि न समस्तेन कपोतेन बभूव ह}


\twolineshloka
{अस्थिभूतो यदा राजा निर्मांसो रुधिरस्रवः}
{तुलां ततः समारूढः स्वं मांसक्षयमुत्सृजन्}


\twolineshloka
{ततः सेन्द्रास्त्रयो लोकास्तं नरेन्द्रमुपस्थिताः}
{र्भर्यश्चाकाशगैस्तत्र वादिता देवदुन्दुभिः}


\twolineshloka
{अमृतेनावसिक्तश्च वृषदर्भो नरेश्वरः}
{दिव्यैश्च सुसुखैर्माल्यैरभिवृष्टः पुनःपुनः}


\twolineshloka
{देवगन्धर्वसन्घातैरप्सरोभिश्च सर्वतः}
{नृत्तश्चैवोपगीतश्च पितामह इव प्रभुः}


\twolineshloka
{हेमप्रासादसम्बाधं मणिकाञ्चनतोरणम्}
{सवैडूर्यमणिस्तम्भं विमानं समधिष्ठितः}


\twolineshloka
{स राजर्षिर्गतः स्वर्गं कर्मणा तेन शाश्वतम्}
{शरणागतेषु चैवं त्वं कुरु सर्वं युधिष्ठिर}


\twolineshloka
{भक्तानामनुरक्तानामाश्रितानां च रक्षिता}
{दयावान्सर्वभूतेषु परत्र सुखमेधते}


\twolineshloka
{साधुवृत्तो हि यो राजा सद्वृत्तमनुतिष्ठति}
{किं न प्राप्तं भवेत्तेनि स्वव्याजेनेह कर्मणा}


\twolineshloka
{स राजर्षिर्विशुद्धात्मा धीरः सत्यपराक्रमः}
{काशीनामीश्वरः ख्यातस्त्रिषु लोकेषु कर्मणा}


\twolineshloka
{योऽप्यन्यः कारेयदेवं शरणागतरक्षणम्}
{सोपि गच्छेत तामेव गतिं भरतसत्तम}


\twolineshloka
{इदं वृत्तं हि राजर्षे वृषदर्भस्य कीर्तयन्}
{पूतात्मा वै भवेल्लोके शृणुयाद्यश्च नित्यशः}


\chapter{अध्यायः ६८}
\threelineshloka
{किं राज्ञः सर्वकृत्यानां गरीयः स्यात्पितामह}
{कुर्वन्किं कर्म नृपतिरुभौ लोकौ समश्रुते भीष्म उवाच}
{}


\twolineshloka
{एतद्राज्ञः कृत्यतममभिषिक्तस्य भारत}
{ब्राह्मणानां रक्षणं च पूजा च सुखमिच्छतः}


\twolineshloka
{कर्तव्यं पार्थिवेन्द्रेण तथैव भरतर्षभ}
{श्रोत्रियान्ब्राह्मणान्वृद्धान्नित्यमेवाभिपूजयेत्}


\twolineshloka
{पौरजानपदांश्चापि ब्राह्मणांश्च बहुश्रुतान्}
{सांत्वेन भोगदानेन नमस्कारैः सदाऽर्चयेत्}


\twolineshloka
{एतत्कृत्यतमं राज्ञो नित्यमेवोपलक्षयेत्}
{यथाऽऽत्मानं यथा पुत्रांस्तथैतान्प्रतिपालयेत्}


\twolineshloka
{ये चाप्येषां पूज्यतमास्तान्दृढं प्रतिपूजयेत्}
{तेषु शान्तेषु तद्राष्ट्रं सर्वमेव विराजते}


\twolineshloka
{ते पूज्यास्ते नमस्कार्या मान्यास्ते पितरो यथा}
{तेष्वेव यात्रा लोकानां भूतानामिव वासवे}


\threelineshloka
{अभिचारैरुपायैश्च दहेयुरपि चेतसा}
{निःशेपं कुपिताः कुर्युरुग्राः सत्यपराक्रमाः}
{}


\twolineshloka
{नान्तमेषां प्रपश्यामि न दिशश्चाप्यपावृताः}
{कुपिताः समुदीक्षन्ते दावेषअवग्निशिखा इव}


\threelineshloka
{`मान्यास्तेषां साधवो ये न निन्द्याश्चाप्यसाधवः'}
{बिभ्यत्येषां साहसिका गुणास्तेषामतीव हि}
{कूपा इव तृणच्छन्ना विशुद्धा द्यौरिवापरे}


\twolineshloka
{प्रसह्यकारिणः केचित्कार्पासमृदवोऽपरे}
{सन्ति चैषामतिशठास्तथैवान्ये तपस्विनः}


\twolineshloka
{कृषिगोरक्ष्यमप्येके भैक्ष्यमन्येऽप्यनुष्ठिताः}
{चोराश्चान्येऽनृताश्चान्ये तथाऽन्ये नटनर्तकाः}


\twolineshloka
{सर्वकर्मसहाश्चान्ये पार्थिवेष्वितरेषु च}
{विविधाचारयुक्ताश्च ब्राह्मणा भरतर्षभ}


\twolineshloka
{नानाकर्मसु रक्तानां बहुकर्मोपजीविनाम्}
{धर्मज्ञानां सतां तेषां नित्यमवोनुकीर्तयेत्}


\twolineshloka
{पितॄणां देवतानं च मनुष्योरगरक्षसाम्}
{पुराऽप्येते महाभागा ब्राह्मणा वै जनाधिप}


\twolineshloka
{नैते देवैर्न पितृभिर्न गन्धर्वैर्न राक्षसैः}
{नासुरैर्न पिशाचैश्च शक्या जेतुं द्विजातयः}


\twolineshloka
{अदैवं दैवतं कुर्युर्दैवतं चाप्यदैवतम्}
{यमिच्छेयुः स राजा स्याद्यं द्विष्युः स पराभवेत्}


\twolineshloka
{परिवादं च ये कुर्युर्ब्राह्मणानामचेतसः}
{सत्यं ब्रवीमि ते राजन्विनश्येयुर्न संशयः}


\twolineshloka
{निन्दाप्रशंसाकुशलाः कीर्त्यकीर्तिपरायणाः}
{परिकुप्यन्ति ते राजन्सततं द्विषतां द्विजाः}


\twolineshloka
{ब्राह्मणा यं प्रशंसन्ति पुरुषः स प्रवर्धते}
{ब्राह्मणैर्यः पराकृष्टः पराभूयात्क्षणाद्धि सः}


\twolineshloka
{शका यवनकाम्भोजास्तास्ताः क्षत्रियजातयः}
{वृषलत्वं परिगता ब्राह्मणानामदर्शनात्}


\twolineshloka
{द्राविडाश्च कलिङ्गाश्च पुलिन्दाश्चाप्युशीनराः}
{कोलिसर्पा महिपकास्तास्ताः क्षत्रियजातयः}


\twolineshloka
{वृपलत्वं परिगता ब्राह्मणानामदर्शनात्}
{श्रेयान्पराजयस्तेभ्यो न जयो जयतांवर}


\twolineshloka
{यस्तु सर्वमिदं हन्याद्ब्राह्मणं च न तत्समम्}
{ब्रह्मवध्या महान्दोष इत्याहुः परमर्षयः}


\twolineshloka
{परिवादो द्विजातीनां न श्रोतव्यः कथञ्चन}
{आसीताधोमुखस्तूष्णीं समुत्थाय व्रजेत वा}


\twolineshloka
{न स जातो जनिष्यो वा पृथिव्यामिह कश्चन}
{यो ब्राह्मणविरोधेन सुखं जीवितुमुत्सहेत्}


\twolineshloka
{दुर्ग्राह्यो मुष्टिना वायुर्दुःस्पर्शः पाणिना शशी}
{दुर्धरा पृथिवी मूर्ध्ना दुर्जया ब्राह्मणा भुवि}


\chapter{अध्यायः ६९}
\twolineshloka
{ब्राह्मणानेव सततं भृशं सम्परिपूजयेत्}
{एते हि सोमराजान ईश्वरः सुखदुःखयोः}


\threelineshloka
{एते भोगैरलङ्कारैरन्यैश्चैव किमिच्छकैः}
{सदा पूज्या नमस्कारै रक्ष्याश्च पितृवन्नृपैः}
{ततो राष्ट्राय शान्तिर्हि भूतानामिव वासवात्}


\twolineshloka
{ज्ञानवान्ब्रह्मवर्चस्वी राष्ट्रे वै ब्राह्मणः शुचिः}
{महारथश्च राजन्य एष्टव्यः शत्रुतापनः}


\twolineshloka
{ब्राह्मणं जातिसम्पन्नं धर्मज्ञं संशितव्रतम्}
{बोजयीत गृहे राजन्न तस्मात्परमस्ति वै}


\twolineshloka
{ब्राह्मणेभ्यो हविर्दत्तं प्रतिगृह्णन्ति देवताः}
{पितरः सर्वभूतानां नैतेभ्यो विद्यते परम्}


\twolineshloka
{आदित्यश्चन्द्रमा विष्णुः सङ्करोऽग्निः प्रजापतिः}
{सर्वे ब्राह्मणमाविश्य सदाऽन्नमुपभुञ्जते}


\twolineshloka
{न तस्याश्नन्ति पितरो यस्य विप्रा न भुञ्जते}
{देवाश्चाप्यस्य नाश्नन्ति पापस्य ब्राह्मणद्विषः}


\twolineshloka
{ब्राह्मणेषु तु तुष्टेषु प्रीयन्ते पितरः सदा}
{तथैव देवता राजन्नात्र कार्या विचारणा}


\twolineshloka
{तथैव तेऽपि प्रीयन्ते येषां भवति तद्धविः}
{न च प्रेत्य विनश्यन्ति गच्छन्ति च परां गतिम्}


% Check verse!
येनयेनैव प्रीयन्ते पितरो देवतास्तथा ॥तेनतेनैव प्रीयन्ते पितरो देवतास्तथा
\twolineshloka
{ब्राह्मणादेव तद्भूतं प्रभवन्ति यतः प्रजाः}
{यतश्चायं प्रभवति प्रेत्य यत्र च गच्छति}


\twolineshloka
{वेदैष मार्गं स्वर्गस्य तथैव नरकस्य च}
{आगतानागते चोमे ब्राह्मणो द्विपदांवरः}


\threelineshloka
{ब्राह्मणो द्विपदां श्रेष्ठः स्वधर्मं चैव वेद यः}
{ये चैनमनुवर्तन्ते ते न यान्ति पराभवम्}
{न ते प्रेत्य विनश्यन्ति गच्छन्ति न पराभवम्}


\twolineshloka
{यद्ब्राह्मणमुखात्प्राप्तं प्रतिगृह्णन्ति वै वचः}
{कृतात्मानो महात्मानस्ते न यान्ति पराभवम्}


\twolineshloka
{क्षत्रियाणां प्रतपतां तेजसा च बलेन च}
{ब्राह्मणेष्वेव शाम्यन्ति तेजांसि च बलानि च}


\twolineshloka
{भृगवस्तालजङ्घांश्च नीपानाङ्गिरसोऽजयन्}
{भरद्वाजो वैतहव्यानैलांश्च भरतर्षभ}


\twolineshloka
{चित्रायुधांश्चाप्यजयन्नेते कृष्णाजिनध्वजाः}
{प्रक्षिप्याथ च कुम्भान्वै पारगामिनमारभेत्}


\twolineshloka
{यत्किंचित्कथ्यते लोके श्रूयते पठ्यतेऽपि वा}
{सर्वं तद्ब्राह्मणेष्वेव गूढोऽग्निरिव दारुषु}


\threelineshloka
{अत्राप्युदाहरन्तीममितिहासं पुरातनम्}
{संवादं वासुदेवस्य पृथ्व्याश्च भरतर्षभ ॥वासुदेव उवाच}
{}


\threelineshloka
{मातरं सर्वभूतानां पृच्छे त्वां संशयं शुभे}
{केनस्वित्कर्मणा पापं व्यपोहति नरो गृही ॥पृथिव्युवाच}
{}


\threelineshloka
{ब्राह्मणानेव सेवेत पवित्रं ह्येतदुत्तमम्}
{ब्राह्मणान्सेवमानस्य रजः सर्वं प्रणश्यति}
{भूतो भूतिरतः कीर्तिरतो बुद्धिः प्रजायते}


\twolineshloka
{महारथश्च राजन्य एष्टव्यः शत्रुतापनः}
{इति मां नारदः प्राह सततं सर्वभूतये}


\twolineshloka
{ब्राह्मणं जातिसम्पन्नं धर्मज्ञं संशितं शुचिम्}
{अपरेषां परेषां च परेभ्यश्चैव ये परे}


\twolineshloka
{ब्राह्मणा यं प्रशंसन्ति स मनुष्यः प्रवर्धते}
{अथ यो ब्राह्मणान्क्रुष्टः पराभवति सोचिरात्}


\twolineshloka
{यथा महार्णवे क्षिप्त आमलोष्टो विनश्यति}
{तथा दुश्चरितं विप्रे पराभावाय कल्पते}


\twolineshloka
{पश्य चन्द्रे कृतं लक्ष्म समुद्रे लवणोदकम्}
{तथा भगसहस्रेण महेन्द्रः परिचिह्नितः}


\twolineshloka
{तेषामेव प्रभावेन सहस्रनयनो ह्यसौ}
{शतक्रतुः समभवत्पश्य माधव यादृशम्}


\threelineshloka
{इच्छन्कीर्तिं च भूतिं च लोकांश्च मधुसूदन}
{ब्राह्मणानुमते तिष्ठेत्पुरुषः शुचिरात्मवान् ॥भीष्म उवाच}
{}


\twolineshloka
{इत्येतद्वचनं श्रुत्वा मेदिन्या मधूसूदनः}
{साधुसाध्विति कौरव्य मेदिनीं प्रत्यपूजयत्}


\twolineshloka
{एतां श्रुत्वोपमां पार्थ प्रयतो ब्राह्मणर्षभान्}
{सततं पूजयेथास्त्वं ततः श्रेयोऽभिपत्स्यसे}


\chapter{अध्यायः ७०}
\twolineshloka
{जन्मनैव महाभागो ब्राह्मणो नाम जायते}
{नमस्यः सर्वभूतानामतिथिः प्रश्रिताग्रभुक्}


\threelineshloka
{सर्वेषां सुहृदस्तात ब्राह्मणाः सुमनोमुकाः}
{`सर्वानेते हनिष्यन्ति ब्राह्मणा जातमन्यवः}
{'गीर्भिर्मङ्गलयुक्ताभिरनुध्यायन्ति पूजिताः}


\twolineshloka
{सर्वान्नो द्विषतस्तात ब्राह्मणा जातमन्यवः}
{गीर्भिर्दारुणयुक्ताभिर्हन्युश्चैते ह्यपूजिताः}


\twolineshloka
{अत्र गाथाः पुरा गीताः कीर्तयन्ति पुराविदः}
{सृष्ट्वा द्विजातीन्धाता हि यथापूर्वं समादधत्}


\twolineshloka
{न वोऽन्यदिह कर्तव्यं किञ्चिदूर्ध्वायनं विधि}
{गुप्तो गोपायते ब्रह्मा श्रेयो वस्तेन शोभनम्}


\twolineshloka
{स्वमेव कुर्वतां कर्म श्रीर्वो ब्राह्मी भविष्यति}
{प्रमाणं सर्वभूतानां प्रग्रहाश्च भविष्यथ}


\twolineshloka
{न शौद्रं कर्म कर्तव्यं ब्राह्मणेन विपश्चिता}
{शौद्रं हि कुर्वतः कर्म ब्राह्मी श्रीरुपरुध्यते}


\twolineshloka
{श्रीश्च बुद्धिश्च तेजश्च विभूतिश्च प्रतापिनी}
{स्वाध्यायेनैव महात्म्यं विपुलं प्रतिपत्स्यथ}


\twolineshloka
{हुत्वा चाहवनीयस्थं महाभाग्ये प्रतिष्ठिताः}
{अग्रभोज्याः प्रसूतीनां श्रिया ब्राह्मयाऽनुकल्पिताः}


\twolineshloka
{श्रद्धया परया युक्ता ह्यनभिद्रोहलब्धया}
{दमस्वाध्यायनिरताः सर्वान्कामानवाप्स्यथ}


\twolineshloka
{यच्चैव मानुषे लोके यच्च देवेषु किञ्चन}
{सर्वं वस्तपसा साध्यं ज्ञानेन नियमेन च}


\twolineshloka
{`युष्मत्संमाननां प्रीतिं पावनैः क्षत्रिया भृशम्}
{'अमुत्रेह समायान्ति वैश्यशूद्राधिकास्तथा}


\twolineshloka
{अरक्षिताश्च युष्माभिर्विरुद्धा यान्ति विप्रवम्}
{युष्मत्तेजोधृता लोकास्तद्रक्ष्यथ जगत्त्रयम् ॥'}


\twolineshloka
{इत्येवं ब्रह्मगीतास्ते समाख्याता मयाऽनघ}
{विप्रानुकम्पार्थमिदं तेन प्रोक्तं हि धीमता}


\twolineshloka
{भूयस्तेषां बलं मन्ये यथा राज्ञस्तपस्विनः}
{दुरासदाश्च चण्डाश्च तपसा क्षिप्रकारिणः}


\twolineshloka
{सन्त्येषां सिम्हसत्वाश्च व्याघ्रसत्वास्तथाऽपरे}
{वराहमृगसत्वाश्च गजसत्वास्तथाऽपरे}


\twolineshloka
{सर्पस्पर्शसमाः केचित्तथाऽन्ये मकरस्पृशः}
{विभाष्य घातिनः केचित्तथा चक्षुर्हणोऽपरे}


\twolineshloka
{सन्ति चाशीविषसमाः सन्ति मन्दास्तथाऽपरे}
{विविधानीह वृत्तानि ब्राह्मणानां युधिष्ठिर}


\twolineshloka
{मेकला द्राविडा लाटाः पौण्ड्राः कान्वशिरास्तथा}
{शौण्डिका दरदा दार्वाश्चोराः शबरबर्बराः}


\twolineshloka
{किराता यवनाश्चैव तास्ताः क्षत्रियजातयः}
{वृषत्वमनुप्राप्ता ब्राह्मणानामदर्शनात्}


\twolineshloka
{ब्राह्मणानां परिभवादसुराः सलिलेशयाः}
{ब्राह्मणानां प्रसादाच्च देवाः स्वर्गनिवासिनः}


\twolineshloka
{अशक्यं स्प्रष्टुमाकाशमचाल्यो हिमवान्गिरिः}
{अवार्या सेतुना गङ्गा दुर्जया ब्राह्ममा भुवि}


\twolineshloka
{न ब्राह्मणविरोधेन शक्या शास्तुं वसुन्धरा}
{ब्राह्मणा हि महात्मानो देवानामपि देवताः}


\twolineshloka
{तान्पूजयस्व सततं दानेन परिचर्यया}
{यदीच्छसि महीं भोक्तमिमां सागरमेखलाम्}


\twolineshloka
{प्रतिग्रहेण तेजो हि विप्राणा शाम्यतेऽनघ}
{प्रतिग्रहं ये नेच्छेयुस्तेऽपि रक्ष्यास्त्वया नृप}


\chapter{अध्यायः ७१}
\twolineshloka
{अत्राप्युदाहरन्तीममितिहासं पुरातनम्}
{शक्रशम्बरसंवादं तन्निवोध युधिष्ठिर}


\twolineshloka
{शक्रो ह्यज्ञातरूपेण जटी भूत्वा सुवारुणः}
{विप्ररूपं समास्थाय प्रश्नं पप्रच्छ शम्बरम्}


\threelineshloka
{केन शम्बर वृत्तेन स्वजात्यानधितिष्ठसि}
{श्रेष्ठं त्वां केन मन्यन्ते तद्वै प्रब्रूहि तत्त्वतः ॥शम्बर उवाच}
{}


\twolineshloka
{नासूयामि सदा विप्रान्ब्राह्ममेव च मे मतम्}
{शास्त्राणि वदतो विप्रान्संमन्यामि यथासुखम्}


\twolineshloka
{श्रुत्वा च नावजानामि नापराध्यामि कर्हिचित्}
{अभ्यर्च्याभ्यनुपृच्छामि पादौ गृह्णामि धीमताम्}


\twolineshloka
{ते विस्रब्धाः प्रभाषन्ते संयच्छन्ति च मां सदा}
{प्रमत्तेष्वप्रमत्तोऽस्मि सदा सुप्तेषु जागृमि}


\twolineshloka
{ते मां शास्त्रपथे युक्तं ब्रह्मण्यमनसूयकम्}
{समासिञ्चति शास्तारः क्षौद्रं मध्विव मक्षिकाः}


\twolineshloka
{यच्च भाषन्ति संतुष्टास्तच्च गृह्णाम्यमायया}
{समाधिमात्मनो नित्यमनुलोममचिन्तयम्}


\twolineshloka
{सोहं वागग्रमृष्टानां रसानामवलेहकः}
{स्वजात्यानधितिष्ठामि नक्षत्राणीव चन्द्रमाः}


\twolineshloka
{एतत्पृथिव्याममृतमेतच्चक्षुरनुत्तमम्}
{यद्ब्राह्मणमुखाच्छास्त्रमिह श्रुत्वा प्रवर्तते}


\twolineshloka
{एतत्कारणमाज्ञाय दृष्ट्वा देवासुरं पुरा}
{युद्धं पिता मे हृष्टात्मा विस्मितः समपद्यत}


\threelineshloka
{दृष्ट्वा च ब्राह्मणानां तु महिमानं महात्मनाम्}
{पर्यपृच्छत्कथममी सिद्धा इति निशाकरम् ॥सोम उवाच}
{}


\twolineshloka
{ब्राह्मणास्तपसा सर्वे सिध्यन्ते वाग्बलाः सदा}
{भुजवीर्याश्च राजानो वागस्त्राश्च द्विजातयः}


\twolineshloka
{प्रवसन्वाप्यधीयीत ब्राह्मीर्दुर्वसतीर्वसन्}
{निर्मन्युरपि निर्वाणो यतिः स्यात्समदर्शनः}


\twolineshloka
{अपि च ज्ञानसम्पन्नः सर्वान्वेदान्पितुर्गृहे}
{श्लाघमान इवाधीयाद्ग्राम्य इत्येव तं विदुः}


\twolineshloka
{भूमिरेतौ निगिरति सर्पो बिलशयानिव}
{राजानं चाप्ययोद्धारं ब्राह्मणं चाप्रवासिनम्}


\twolineshloka
{अभिमानः श्रियं हन्ति पुरुषस्याल्पमेधसः}
{गर्भेण दुष्यते कन्या गृहवासेन च द्विजः}


\twolineshloka
{`विद्याविदो लोकविदस्तपोदमसमन्विताः}
{नित्यपूज्याश्च वन्द्याश्च द्विजा लोकद्वयेच्छुभिः}


\threelineshloka
{इत्येतन्मे पिता श्रुत्वा सोमादद्भुतदर्शनात्}
{ब्राह्मणान्पूजयामास तथैवाहं महाव्रतान् ॥भीष्म उवाच}
{}


\twolineshloka
{श्रुत्वैतद्वचनं शक्रो दानवेन्द्रमुखाच्च्युतम्}
{द्विजान्संम्पूजयामास महेन्द्रत्वमवाप च}


\chapter{अध्यायः ७२}
\threelineshloka
{अपूर्वं वा भवेत्पात्रमथवाऽपि चिरोषितम्}
{दूरादभ्यागतं वाऽपि किं पात्रं स्यात्पितामह ॥भीष्म उवाच}
{}


\twolineshloka
{क्रिया भवति केषांचिदुपांशुव्रतमुत्तमम्}
{यो नो याचेत यत्किंचित्सर्वं दद्याम इत्यपि}


\twolineshloka
{अपीडयन्भृत्यवर्गमित्येवमनुशुश्रुम}
{पीडयन्भृत्यवर्गं हि आत्मानमपकर्षति}


\threelineshloka
{अपूर्वं चापि यत्पात्रं यच्चापि स्याच्चिरोषितम्}
{दूरादभ्यागतं चापि तत्पात्रं न विदुर्बुधाः ॥युधिष्ठिर उवाच}
{}


\threelineshloka
{अपीडया च भूतानां धर्मस्याहिंसया तथा}
{पात्रं विद्यामतत्त्वेन यस्मै दत्तं न सन्तपेत् ॥भीष्म उवाच}
{}


\twolineshloka
{ऋत्विक्पुरोहिताचार्याः शिष्यसम्बन्धिबान्धवाः}
{सर्वे पूज्याश्च मान्याश्च श्रुतवन्तोऽनसूयकाः}


\twolineshloka
{अतोऽन्यथा वर्तमानाः सर्वे नार्हन्ति सत्क्रियाम्}
{तस्माद्गुणैः परीक्षेत पुरुषान्प्रणिधाय वै}


\twolineshloka
{अक्रोधः सत्यवचनमहिंसा दम आर्जवम्}
{अद्रोहोऽनभिमानश्च ह्रीस्तितिक्षा दमः शमः}


\twolineshloka
{यस्मिन्नोतानि दृश्यन्ते न चाकार्याणि भारत}
{स्वभावतो निविष्टानि तत्पात्रं मानमर्हति}


\twolineshloka
{तथा चिरोषितं चापि सम्प्रत्यागतमेव च}
{अपूर्वं चैव पूर्वं च तत्पात्रं मानमर्हति}


\twolineshloka
{अप्रामाण्यं च वेदानां शास्त्राणां चाभिलङ्घनम्}
{अव्यवस्था च सर्वत्र एतन्नाशनमात्मनः}


\twolineshloka
{भवेत्पण्डितमानी यो ब्राह्मणो वेदनिन्दकः}
{आन्वीक्षिकीं तर्कविद्यामनुरक्तो निरर्थिकां}


\twolineshloka
{हेतुवादान्बुवन्सत्सु विजेताऽहेतुवादकः}
{आक्रोष्टा चातिवक्ता च ब्राह्मणानां सदैवहि}


\twolineshloka
{सर्वाभिशङ्की मूढश्च बालः कटुकवागपि}
{बोद्धव्यस्तादृशस्तात नरं श्वानं हि तं विदुः}


\threelineshloka
{यथा श्वा भषितुं चैव हन्तु चैवावसज्जते}
{एवं सम्भाषणार्थाय सर्वशास्त्रवधाय च}
{`अल्पश्रुताः कुतर्काश्च दृष्टाः सृष्टाः कुपण्डिताः}


\twolineshloka
{श्रुतिस्मृती चेतिहासपुराणारण्यवेदिनः}
{अनुरुन्ध्याद्बहुज्ञांश्च सारज्ञाश्चैव पण्डिताः ॥'}


\twolineshloka
{लोकयात्रा च द्रष्टव्या धर्मश्चात्महितानि च}
{एवं नरो वर्तमानः शाश्वतीर्वर्धते समाः}


\twolineshloka
{ऋणमुन्मुच्य देवानामृषीणां च तथैव च}
{पितॄणामथ विप्राणामतिथीनां च पञ्चमम्}


\twolineshloka
{पर्यायेण विमुक्तो यः सुनिर्णिक्तेन कर्मणा}
{एवं गृहस्थः कर्माणि कुर्वन्धर्मानन हीयते}


\chapter{अध्यायः ७३}
\threelineshloka
{स्त्रीणां स्वभावमिच्छामि श्रोतुं भरतसत्तम}
{स्त्रियो हि मूलं दोषाणां लघुचित्ता हि ताः स्मृताः? भीष्म उवाच}
{}


\twolineshloka
{अत्राप्युदाहरन्तीममितिहासं पुरातनम्}
{नारदस्य च संवादं पुंश्चल्या पञ्चचूडाया}


\twolineshloka
{लोकाननुचरन्सर्वान्देवर्षिर्नारदः पुरा}
{ददर्शाप्सरसं ब्राह्मीं पञ्चचूडामनिन्दिताम्}


\twolineshloka
{तां दृष्ट्वा चारुसर्वाङ्गीं पप्रच्चाप्सरसं मुनिः}
{संशयो हृदि कश्चिन्मे ब्रूहि तन्मे सुमध्यमे}


\threelineshloka
{एवमुक्ताऽथ सा विप्रं प्रत्युवाचाथ नारदम्}
{विषये सति वक्ष्यामि समर्थां मन्यसे च माम् ॥नारद उवाच}
{}


\twolineshloka
{न त्वामविषये भद्रे नियोक्ष्यामि कथञ्चन}
{स्त्रीणां स्वभावमिच्छामि त्वत्तः श्रोतुं वरानने}


\twolineshloka
{एतच्छ्रुत्वा वचस्तस्य देवर्षेरप्सरोत्तमा}
{प्रत्युवाच न शक्ष्यामि स्त्री सती निन्दितुं स्त्रियः}


\twolineshloka
{विदितास्ते स्त्रियो याश्च यादृशाश्च स्वभावतः}
{न मामर्हसि देवर्षे नियोक्तुं कार्य ईदृशे}


\twolineshloka
{तामुवाच स देवर्षिः सत्यं वद सुमध्यमे}
{मृषावादे भवेद्दोषः सत्ये दोषो न विद्यते}


\twolineshloka
{इत्युक्ता सा कृतमतिरभवच्चारुहासिनी}
{स्त्रीदोषाञ्शाश्वतान्सत्यान्भाषितुं सम्प्रचक्रमे}


\twolineshloka
{कुलीना रूपवत्यश्च नाथवत्यश्च योषितः}
{मर्यादासु न तिष्ठन्ति स दोषः स्त्रीषु नारद}


\twolineshloka
{न स्त्रीभ्यः किञ्चिदन्यद्वै पापीयस्तरमस्ति वै}
{स्त्रियो हि मूलं दोषाणां तथा त्वमपि वेत्थ ह}


\twolineshloka
{समाज्ञातानृद्धिमतः प्रतिरूपान्वशे स्थितान्}
{पतीनन्तरमासाद्य नालं नार्यः परीक्षितुम्}


\twolineshloka
{असद्धर्मस्त्वयं स्त्रीणामस्माकं भवति प्रभो}
{पापीयसो नरान्यद्वै लज्जां त्यक्त्वा भजामहे}


\twolineshloka
{स्त्रियं हि यः प्रार्थयते सन्निकर्षं च गच्छति}
{ईषच्च कुरुते सेवां तमेवेच्छन्ति योषितः}


\twolineshloka
{अनर्थित्वान्मनुष्याणां भयात्परिजनस्य च}
{मर्यादायाममर्यादाः स्त्रियस्तिष्ठन्ति भर्तृषु}


% Check verse!
नासां कश्चिदगम्योस्ति नासां वयसि निश्चयः ॥विरूपं रूपवन्तं वा पुमानित्येव भुञ्जते
\twolineshloka
{न भयान्नाप्यनुक्रोशान्नार्थहेतोः कथञ्चन}
{न ज्ञातिकुलसम्बन्धात्स्त्रियस्तिष्ठन्ति भर्तृषु}


\twolineshloka
{यौवने वर्तमानानां मृष्टाभरणवाससाम्}
{नारीणां खैरवृत्तीनां स्पृहयन्ति कुलस्त्रियः}


\twolineshloka
{याश्च शश्वद्बहुमता रक्ष्यन्ते दयिताः स्त्रियः}
{अपि ताः सम्प्रसज्जन्ते कुब्जान्धजडवामनैः}


\twolineshloka
{पङ्गुष्वथ च देवर्षे ये चान्ये कुत्सिता नराः}
{स्त्रीणामगम्यो लोकेऽस्मिन्नास्ति कश्चिन्महामुने}


\twolineshloka
{यदि पुंसां गतिर्ब्रह्मन्कथंचिन्नोपपद्यते}
{अप्यन्योन्यं प्रवर्तन्ते न हि तिष्ठन्ति भर्तृषु}


\twolineshloka
{`दुष्टाचाराः पापरता असत्या मायया वृताः}
{अदृष्टबुद्धिबहुलाः प्रायेणेत्यवगम्यताम्}


\twolineshloka
{अलाभात्पुरुषाणां हि भयात्परिजनस्य च}
{वधबन्धभयाच्चापि स्वयं गुप्ता भवन्ति ताः}


\twolineshloka
{चलस्वभावा दुःसेव्या दुर्ग्राह्या भावतस्तथा}
{प्राज्ञस्य पुरुषस्येह यथाभावास्तथा स्त्रियः}


\twolineshloka
{नाग्निस्तृप्यति काष्ठानां नापगानां महोदधिः}
{नान्तकः सर्वभूतानां न पुंसां वामलोचनाः}


\twolineshloka
{इदमन्यच्च देवर्षे रहस्यं सर्वयोषिताम्}
{दृष्ट्वैव पुरुषं ह्यन्यं योनिः प्रक्लिद्यते स्त्रियाः}


\twolineshloka
{कामानामपि दातारं कर्तारं मानसान्त्वयोः}
{रक्षितारं न मृष्यन्ति स्वभर्तारमसत्स्त्रियः}


\twolineshloka
{न कामभोगान्विपुलान्नालङ्कारार्थसञ्चयान्}
{तथैव बहुमन्वन्ते यथा रत्यामनुग्रहम्}


\twolineshloka
{अन्तकः शमनो मृत्युः पातालं बडबामुखम्}
{क्षुरधारा विषं सर्पो वह्निरित्येकतः स्त्रियः}


\twolineshloka
{यतश्च भूतानि महान्ति पञ्चयतश्च लोका विहिता विधात्रा}
{यतः पुमांसः प्रमदाश्च निर्मिता-स्ततश्च दोषाः प्रमदासु नारद}


\chapter{अध्यायः ७४}
\twolineshloka
{इमे वै मानवा लोके स्त्रीषु सज्जन्त्यभीक्ष्णशः}
{मोहेन परमाविष्टा देवदृष्टेन कर्मणा}


\twolineshloka
{स्त्रियश्च पुरुषेष्वेव प्रत्यक्षं लोकसाक्षिकम्}
{अत्र मे संशयस्तीव्रो हृदि सम्परिवर्तते}


\twolineshloka
{कथमासां नराः सङ्गं कुर्वते कुरुनन्दनः}
{स्त्रियो वातेषु रज्यन्ते विरज्यन्ते च ताः पुनः}


\threelineshloka
{इति ताः पुरुषव्याघ्र कथं शक्यास्तु रक्षितुम्}
{प्रमदाः पुरुषेणेह तन्मे व्याख्यातुमर्हसि ॥भीष्म उवाच}
{}


\twolineshloka
{एत्वा हि स्वीयमायाभिर्वञ्चयन्तीह मानवान्}
{न चासां मुच्यते कश्चित्पुरुषो हस्तमागतः}


% Check verse!
गावो नवतृणानीव गृह्णन्त्येता नवन्नवम्
\twolineshloka
{शम्बरस्य च या माया माया या नमुचेरपि}
{बलेः कुम्भीनसेश्चैव सर्वास्तां योषितो विदुः}


\twolineshloka
{हसन्तं प्रहसन्त्येता रुद्रन्तं प्ररुदन्ति च}
{अप्रियं प्रियवाक्यैश्च गृह्णते कालयोगतः}


\twolineshloka
{`यदि जिह्वासहस्रं स्याज्जीवेच्च शरदां शतम्}
{अनन्यकर्मा स्त्रीदोषाननुक्त्वा निधनं व्रजेत् ॥'}


\twolineshloka
{उशना वेद यच्छास्त्रं यच्च वेद बृहस्पतिः}
{स्त्री बुद्ध्या न विशिष्येत तास्तु रक्ष्याः कथं नरैः}


\threelineshloka
{अनृतं सत्यमित्याहुः सत्यं चापि तथाऽनृतम्}
{इति यास्ताः कथं वीर संरक्ष्याः पुरुषैरिह}
{`दोषास्पदेऽशुचौ देहे ह्यासां सक्तास्त्वहो नराः'}


\twolineshloka
{स्त्रीणां बुद्ध्यर्थनिष्कर्षादर्थशास्त्राणि शत्रुहन्}
{बृहस्पतिप्रभृतिभिर्मन्ये सद्भिः कृतानि वै}


\twolineshloka
{संपूज्यमानाः पुरुषैर्विकुर्वन्ति मनो नृषु}
{अपास्ताश्च तथा राजन्विकुर्वन्ति मनः स्त्रियः}


% Check verse!
इमाः प्रजा महापाहो धार्मिक्य इति नः श्रुतम्
\twolineshloka
{सत्कृतासत्कृताश्चापि विकुर्वन्ति मनः सदा}
{कस्ताः शक्तो रक्षितुं स्यादिति मे संशयो महान्}


\threelineshloka
{तथा ब्रूहि महाभाग कुरूणां वंशवर्धन}
{यदि शक्या कुरुश्रेष्ठ रक्षा तासां कदाचन}
{}


% Check verse!
कर्तुं वा कृतपूर्वं वा तन्मे व्याख्यातुमर्हसि
\chapter{अध्यायः ७५}
\twolineshloka
{एवमेतन्महाबाहो नात्र मिथ्याऽस्ति किञ्चन}
{यथा ब्रवीषि कौरव्य नारीं प्रति जनाधिप}


\twolineshloka
{अत्र ते वर्तयिष्यामि इतिहासं पुरातनम्}
{यथा रक्षा कृता पूर्वं विपुलेन महात्मना}


% Check verse!
प्रमदाश्च यथा सृष्टा ब्रह्मणा भरतर्षभ ॥यदर्थं तच्च ते तात प्रवक्ष्यामि नराधिप
\threelineshloka
{न हि स्त्रीभ्यः परं पुत्र पापीयः किञ्चिदस्ति वै}
{अग्निर्हिः प्रमदा दीप्तो मायाश्च मयजा विभो}
{क्षुरधारा विषं सर्पो मृत्युरित्येकतः स्त्रियः}


\twolineshloka
{प्रजा इमा महाबाहो धार्मिक्य इति नः श्रुतम्}
{स्वयं गच्छन्ति देवत्वं ततो देवानियाद्भयम्}


\twolineshloka
{अथाभ्यागच्छन्देवास्ते पितामहमरिंदम}
{निवेद्य मानसं चापि तूष्णीमासन्नधोमुखाः}


\twolineshloka
{तेषामन्तर्गतं ज्ञात्वा देवानां स पितामहः}
{मानवानां प्रमोहार्थं कृत्या नार्योऽसृजत्प्रभुः}


\twolineshloka
{पूर्वसर्गे तु कौन्तेय साध्व्यो नार्य इहाभवन्}
{असाध्व्यस्तु समुत्पन्नाः कृत्याः सर्गात्प्रजापतेः}


\twolineshloka
{ताभ्यः कामान्यथाकामं प्रादाद्धि स पितामहः}
{ताः कामलुब्धाः प्रमदाः प्रामथ्नन्त नरान्सदा}


\twolineshloka
{क्रोधं कामस्य देवेशः सहायं चासृजत्प्रभुः}
{असज्जन्त प्रजाः सर्वाः कामक्रोधवशङ्गताः}


\twolineshloka
{`द्विजानां च गुरूणां च महागुरुनृपादिनाम्}
{क्षणस्त्रीसङ्गकामोत्था यातनाहो निरन्तरा}


\twolineshloka
{अरक्तमनसां नित्यं ब्रह्मचर्यामलात्मनाम्}
{तपोदमार्चनाध्यानयुक्तानां शुद्धिरुत्तमा ॥'}


\twolineshloka
{न च स्त्रीणां क्रियाः काश्चिदिति धर्मो व्यवस्थितः}
{निरिन्द्रिया ह्यशास्त्राश्च स्त्रियोऽनृतमिति श्रुतिः}


\twolineshloka
{शय्यासनमलङ्कारमन्नपानमनार्यताम्}
{दुर्वाग्भावं रतिं चैव ददौ स्त्रीभ्यः प्रजापतिः}


\twolineshloka
{न तासां रक्षणं शक्यं कर्तुं पुंसां कथञ्चन}
{अपि विस्वकृता तात कुतस्तु पुरुषैरिह}


\twolineshloka
{वाचा च वधबन्धैर्वा क्लेशैर्वा विविधैस्तथा}
{न शक्या रक्षितुं नार्यस्ता हि नित्यमसंयताः}


\twolineshloka
{इदं तु पुरुषव्याघ्र पुरस्ताच्छ्रुतवानहम्}
{यथा रक्षा कृता पूर्वं विपुलेन गुरुस्त्रियाः}


\twolineshloka
{ऋषिरासीन्महाभागो देवशर्मेति विश्रुतः}
{तस्य भार्या रुचिर्नाम रूपेणासदृशी भुवि}


\twolineshloka
{तस्या रूपेण सम्मत्ता देवगन्धर्वदानवाः}
{विशेषेण तु राजेन्द्र वृत्रहा पाकशासनः}


\twolineshloka
{नारीणां चरितज्ञश्च देवशर्मा महामतिः}
{यथाशक्ति यथोस्साहं भार्यां तामभ्यरक्षत}


\twolineshloka
{पुरंदरं च जानंश्च परस्त्रीकामचारिणम्}
{तस्माद्यत्नेन भार्याया रक्षणं स चकार ह}


\twolineshloka
{स कदाचिदृषिस्तात यज्ञं कर्तुमनास्तदा}
{भार्यासंरक्षणं कार्यं कथं स्यादित्यचिन्तयत्}


\twolineshloka
{रक्षाविधानं मनसा स सञ्चिन्त्य महातपाः}
{आहूय दयितं शिष्यं विपुलं प्राह भार्गवम्}


\twolineshloka
{यज्ञकारो गमिष्यामि रुचिं चेमां सुरेश्वरः}
{यतः प्रार्थयते नित्यं तां रक्षस्व यथाबलम्}


\threelineshloka
{अप्रमत्तेन ते भाव्यं सदा प्रति पुरंदरम्}
{स हि रूपाणि कुरुते विविधानि भृगूत्तम ॥भीष्म उवाच}
{}


\twolineshloka
{इत्युक्तो विपुलस्तेन तपस्वी नियतेन्द्रियः}
{सदैवोग्रतपा राजन्नग्र्यर्कसदृशद्युतिः}


\twolineshloka
{धर्मज्ञः सत्यवादी च तथेति प्रत्यभाषत}
{पुनश्चेदं महाराज पप्रच्छ प्रस्थितं गुरुम्}


\threelineshloka
{कानि रूपाणि शक्रस्य भवन्त्यागच्छतो मुने}
{वपुस्तेजश्च कीदृग्वै तन्मे व्याख्यातुमर्हसि ॥भीष्म उवाच}
{}


\twolineshloka
{ततः स भगवांस्तस्मै विपुलाय महात्मने}
{आचचक्षे यथातत्त्वं मायां शक्रस्य भारत}


\twolineshloka
{बहुमायः स विप्रर्षे बलहा पाकशासनः}
{तांस्तान्विकुरुते भावान्बहूनथ मुहुर्मुहुः}


\twolineshloka
{किरीटी वज्रधृग्धन्वी मुकुटी बद्धकुण्डलः}
{भवत्यथ मुहूर्तेनि चण्डालसमदर्शनः}


\twolineshloka
{शिखी जटी चीरवासाः पुनर्भवति पुत्रक}
{बृहच्छरीरश्च पनश्चीरवासाः पुनः कृशः}


\twolineshloka
{गौरं श्यामं च कृष्णं च वर्णं विकुरुते पुनः}
{विरूपो रूपवांश्चैव युवा वृद्धस्तथैव च}


\twolineshloka
{`प्राज्ञो जडश्च मूकश्च ह्रस्वो दीर्घस्तथैव च}
{'ब्राह्मणः क्षत्रियश्चैव वैश्यः शूद्रस्तथैव च}


\twolineshloka
{प्रतिलोमोऽनुलोमश्च भवत्यथ शतक्रतुः}
{शुकवायसरूपी च हंसकोकिलरूपवान्}


\twolineshloka
{सिंहव्याघ्रगजानां च रूपं धारयते पुनः}
{दैवं दैत्यमथो राज्ञां वपुर्धारयतेऽपि च}


\twolineshloka
{अकृशो मायुभग्राङ्गः शकुनिर्विकृतस्तथा}
{चतुष्पाद्बहुरूपश्च पुनर्भवति बालिशः}


\twolineshloka
{मक्षिकामशकादीनां वपुर्धारयतेऽपि च}
{न शक्यमस्य ग्रहणं कर्तुं विपुल केनचित्}


\twolineshloka
{अपि विश्वकृता तात येन सृष्टमिदं जगत्}
{पुनरन्तर्हितः शक्रो दृश्यते ज्ञानचक्षुषा}


\threelineshloka
{वायुभूतश्च स पुनर्देवराजो भवत्युतः}
{एवंरूपाणि सततं कुरुते पाकशासनः}
{तस्माद्विपुल यत्नेन रक्षेमां तनुमध्यमाम्}


\twolineshloka
{यथा रुचिं नवलिहेद्देवेन्द्रो भृगुसत्तम}
{क्रतावुपहिते न्यस्तं हविः श्वेव दुरात्मवान्}


\twolineshloka
{एवमाख्याय स मुनिर्यज्ञकारोऽगमत्तदा}
{देवशर्मा महाभागस्ततो भरतसत्तम}


\twolineshloka
{विपुलस्तु वचः श्रुत्वा गुरोश्चिन्तामुपेयिवान्}
{रक्षां च परमां चक्रे देवराजान्महाबलात्}


\twolineshloka
{किन्नु शक्यं मया कर्तुं गुरुदाराभिरक्षणे}
{मायावी हि सुरेन्द्रोसौ दुर्धर्षश्चापि वीर्यवान्}


\twolineshloka
{नीपिधायाश्रमं शक्यो रक्षितुं पाकशासनः}
{उटजं वा तथा ह्यस्य नानाविधसरूपता}


\twolineshloka
{वायुरूपेण वा शक्रो गुरुपत्नीं प्रधर्षयेत्}
{तस्मादिमां सम्प्रविश्य रुचिं स्थास्येहमद्य वै}


\twolineshloka
{अथवा पौरुषेणेयं न शक्या रक्षितुं मया}
{बहुरूपो हि भगवाञ्छ्रूयते पाकशासनः}


\twolineshloka
{सोहं योगबलादेनां रक्षिष्ये पाकशासनात्}
{गात्राणि गात्रैरस्याहं सम्प्रवेक्ष्ये हि रक्षितुम्}


\twolineshloka
{यद्युच्छिष्टामिमां पत्नीमद्य पश्यति मे गुरुः}
{शप्स्यत्यसंशयं कोपाद्दिव्यज्ञानो महातपाः}


\twolineshloka
{न चेयं रक्षितुं शक्या यथाऽन्या प्रमदा नृभिः}
{मायावी हि सुरेन्द्रोसावहो प्राप्तोस्मि संशयम्}


\twolineshloka
{अवश्यं करणीयं हि गुरोरिह हि शासनम्}
{यदि त्वेतदहं कुर्यामाश्चर्यं स्यात्कृतं मया}


\twolineshloka
{योगेनाथ प्रविश्येदं गुरुपत्न्याः कलेवरम्}
{असक्तः पद्मपत्रस्थो जलबिन्दुर्यथा चलः}


\twolineshloka
{एवमेव शरीरे ऽस्या निवत्स्यामि समाहितः}
{निर्मुक्तस्य रजोरूपान्नापराधो भवेन्मम}


\threelineshloka
{यथाहि शून्यां पथिकः सभामध्यावसेत्यथि}
{तथाऽद्यावासयिष्यामि गुरुपत्न्याः कलेवरम्}
{एवमेव शरीरे ऽस्य निवत्स्यामि समाहितः}


\twolineshloka
{इत्येवं धर्ममालोक्य वेदवेदांश्च सर्वशः}
{तपश्च विपुलं दृष्ट्वा गुरोरात्मन एव च}


\twolineshloka
{इति निश्चित्य मनसा रक्षां प्रति स भार्गवः}
{अन्वतिष्ठत्परं यत्नं यथा तच्छृणु पार्थिव}


\twolineshloka
{गुरुपत्नीं समासीनो विपुलः स महातपाः}
{उपासीनामनिन्द्याङ्गी कथार्थैः समलोभयत्}


\twolineshloka
{नेत्राभ्यां नेत्रयोरस्या रश्मिं संयोज्य रश्मिभिः}
{विवेश विपुलः कायमाकाशं पवनो यथा}


\twolineshloka
{लक्षणं लक्षणेनैव वदनं वदनेन च}
{अविचेष्टन्नतिष्ठद्वै छायेवान्तर्गतो मुनिः}


\twolineshloka
{ततो विष्टभ्य विपुलो गुरुपत्न्याः कलेवरम्}
{उवास रक्षणे युक्तो न च सा तमबुध्यत}


\twolineshloka
{यं कालं नागतो राजन्गुरुस्तस्य महात्मनः}
{क्रतुं समाप्य स्वगृहं तं कालं सोऽभ्यरक्षत}


\chapter{अध्यायः ७६}
\twolineshloka
{ततः कदाचिद्देवेन्द्रो दिव्यरूपवपुर्धरः}
{इदमन्तरमित्येवमभ्यगात्तमथाश्रमम्}


\twolineshloka
{रूपमप्रतिमं कृत्वा लोभनीयं जनाधिपः}
{दर्शनीयतमो भूत्वा प्रविवेश तमाश्रमम्}


\twolineshloka
{स ददर्श तमासीनं विपुलस्य कलेबरम्}
{निश्चेष्टं स्तब्धनयनं यताऽऽलेख्यगतं तथा}


\twolineshloka
{रुचिं च रुचिरापाङ्गीं पीनश्रोणिपयोधराम्}
{पद्मपत्रविशालाक्षीं सम्पूर्णेन्दुनिभाननाम्}


\twolineshloka
{सा तमालोक्य सहसा प्रत्युत्तातुमियेष ह}
{रूपेण विस्मिता कोऽसीत्यथ वक्तुमिवेच्छती}


\twolineshloka
{उत्थातुकामा तु सती विष्टव्धा विपुलेन सा}
{निगृहीता मनुष्येन्द्र न शशाक विचेष्टितुम्}


\twolineshloka
{तामाबभाषे देवेन्द्रः साम्ना परमवल्*******}
{त्वदर्थमागतं विद्धि देवेन्द्र मां शुचिस्मिते}


\twolineshloka
{क्लिश्यमानमनङ्गेन त्वत्सङ्कल्पभवेन ह}
{तत्पर्याप्नुहि मां सुभ्रु पुरा कालोऽतिवर्तते}


\twolineshloka
{तमेवंवादिनं शक्रं शुश्राव विपुलो मुनिः}
{गुरुपत्नयाः शरीरस्थो ददर्श त्रिदसाधिपम्}


\twolineshloka
{न शशाक च सा राजन्प्रत्युत्थातुमनिन्दिता}
{वक्तुं च नाशकद्राजन्विष्टब्दा विपुलेन सा}


\threelineshloka
{आकारं गुरुपत्न्यास्तु स विज्ञाय भृगूद्वहः}
{निजय्नाह महातेजा योगेन बलवत्प्रभो}
{बबन्ध योगबन्धैश्च तस्याः सर्वेन्द्रियाणि सः}


\twolineshloka
{तां निर्विकारां दृष्ट्वा तु पुनरेव शचीपतिः}
{उवाच व्रीहितो राजंस्तां योगबलमोहिताम्}


\twolineshloka
{एह्येहीति ततः सा तु प्रतिवक्तुमियेष तम्}
{स तां वाच्यं गुरोः पत्न्या विपुलः पर्यवर्तयत्}


\twolineshloka
{भोः किमागमने कृत्यमिति तस्यास्तु निःसृता}
{वक्त्राच्छशाङ्कसदृशाद्वाणी संस्कारभूषणा}


\twolineshloka
{वीडिता सा तु तद्वाक्यमुक्त्वा परवशा तदा}
{पुरन्दरश्च संत्रस्तो बभूव विमना भृशम्}


\twolineshloka
{स तद्वैकृतमालक्ष्य देवराजो विशांपते}
{अवैक्षत सहस्राक्षस्तदा दिव्येन चक्षुषा}


\twolineshloka
{स ददर्श मुनिं तस्याः शरीरान्तरगोचरम्}
{प्रतिबिम्बमिवादर्शे गुरुपत्न्याः शरीरगम्}


\twolineshloka
{स तं घोरेण तपसा युक्तं दृष्ट्वा पुरंदरः}
{प्रावेपत सुसंत्रस्तो व्रीडितश्च तदा विभो}


\twolineshloka
{विमुच्य गुरुपत्नीं तु विपुलः सुमहातपाः}
{स्वकलेबरमाविश्य शक्रं भीतमथाब्रवीत्}


\twolineshloka
{अजितेन्द्रिय दुर्बुद्धे पापात्मक पुरंदर}
{न चिरं पूजयिष्यन्ति देवास्त्वां मानुषास्तथा}


\twolineshloka
{किन्नु तद्विस्मृतं शक्र न तन्मनसि ते स्थितम्}
{गौतमेनासि यन्मुक्तो भगाङ्कपरिचिह्नितः}


\twolineshloka
{जाने त्वां बालिशमतिमकृतात्मानमस्थिरम्}
{मयेयं रक्ष्यते मूढ गच्छ पाप यथागतम्}


\twolineshloka
{नाहं त्वामद्य मूढात्मन्दहेयं हि स्वतेजसा}
{कृपायमानस्तु न ते दग्धुमिच्छामि वासव}


\twolineshloka
{स च घोरतमो धीमान्गुरुर्मे पापचेतसम्}
{दृष्ट्वा त्वां निर्दहेदद्य क्रोधदीप्तेन चक्षुषा}


\twolineshloka
{नैवं तु शक्र कर्तव्यं पुनर्मान्याश्च ते द्विजाः}
{मा गमः ससुतामात्यः क्षयं ब्रह्मबलार्दितः}


\threelineshloka
{अमरोस्मीति यद्बुद्धिं समास्थाय प्रवर्तते}
{मावमंस्था न तपसा न साध्यं नाम किञ्चन ॥भीष्म उवाच}
{}


\twolineshloka
{तच्छ्रुत्वा वचनं शक्रो विपुलस्य महात्मनः}
{न किञ्चिदुक्त्वा व्रीडार्तस्तत्रैवान्तरधीयत}


\twolineshloka
{मुहूर्तयाते तस्मिंस्तु देवशर्मा महातपाः}
{कृत्वा यज्ञं यथाकाममाजगाम स्वमाश्रमम्}


\twolineshloka
{आगतेऽथ गुरौ राजन्विपुलः प्रियकर्मकृत्}
{रक्षितां गुरेव भार्यां न्यवेदयदनिन्दिताम्}


\twolineshloka
{अभिवाद्य च शान्तात्मा स गुरुं गुरुवत्सलः}
{विपुलः पर्युपातिष्ठद्यथापूर्वमशङ्कितः}


\twolineshloka
{विश्रान्ताय ततस्तस्मै सहासीनाय भार्यया}
{निवेदयामास तदा विपुलः शक्रकर्म तत्}


\twolineshloka
{तच्छुत्वा स मुनिस्तुष्टों विपुलस्य प्रतापवान्}
{बभूव शीलवृत्ताभ्यां तपसा नियमेन च}


\twolineshloka
{विपुलस्य गुरौ वृत्तिं भक्तिमात्मनि तत्प्रभुः}
{धर्मे च स्थिरतां दृष्ट्वा साधुसाध्वित्यभाषत}


\twolineshloka
{प्रतिनन्द्य च धर्मात्मा शिष्यं धर्मपरायणम्}
{वरेण च्छन्दयामास देवशर्मा महामतिः}


\twolineshloka
{स्थितिं च धर्मे जग्राह स तस्माद्गुरुवत्सलः}
{अनुज्ञातश्च गुरुणा चचारानुत्तमं तपः}


\twolineshloka
{तथैव देवशर्मापि सभार्यः स महातपाः}
{निर्भयो बलवृत्रघ्नाच्चचार विजने वने}


\chapter{अध्यायः ७७}
\twolineshloka
{विपुलस्त्वकरोत्तीव्रं तपः कृत्वा गुरोर्वचः}
{तपोयुक्तमथात्मानममन्यत स वीर्यवान्}


\twolineshloka
{स तेन कर्मणा स्वर्गं पृथिवीं पृथिवीपते}
{चचार गतभीः प्रीतो लब्धकीर्तिवरो नृप}


\twolineshloka
{उभौ लोकौ जितौ चापि तथैवामन्यत प्रभुः}
{कर्मणा तेन कौरव्य तपसा विपुलेन च}


\twolineshloka
{अथ काले व्यतिक्रान्ते कस्मिंश्चित्कुरुनन्दन}
{रुच्या भगिन्या आदानं प्रभूतदनधान्यवत्}


\twolineshloka
{एतस्मिन्नेव काले तु दिव्या काचिद्वराङ्गना}
{बिभ्रती परमं रूपं जगामाथ विहायसा}


\twolineshloka
{तस्याः शरीरात्पुष्पाणि पतितानि महीतले}
{तस्याश्रमस्याविदूरे दिव्यगन्धानि भारत}


\twolineshloka
{तान्यगृह्णात्ततो राजन्रुचिर्ललितलोचना}
{तदा निमन्त्रकस्तस्या अङ्गेभ्यः क्षिप्रमागमत्}


\twolineshloka
{तस्या हि भगिनी तात ज्येष्ठा नाम्ना प्रभावती}
{भार्या चित्ररथस्याथ बभूवाङ्गेश्वरस्य वै}


\twolineshloka
{पिनह्य तानि पुष्पाणि केशेषु वरवर्णिनी}
{आमन्त्रिता ततोऽगच्छद्रुचिरङ्गपतेर्गृहम्}


\twolineshloka
{पुष्पाणि तानि दृष्ट्वा तु तदाङ्गेन्द्रवराङ्गना}
{भगिनीं चोदयामास पुष्पार्थे चारुलोचना}


\twolineshloka
{सा भर्त्रे सर्वमाचष्ट रुचिः सुरुचिरानना}
{भगिन्या भाषितं सर्वमृषिस्तच्चाभ्यनन्दत}


\twolineshloka
{ततो विपुलमानाय्य देवशर्मा महातपाः}
{पुष्पार्थे चोदयामास गच्छगच्छेति भारत}


\twolineshloka
{विपुलस्तु गुरोर्वाक्यमविचार्य महातपाः}
{स तथेत्यब्रवीद्राजंस्तं च देशं जगाम ह}


\twolineshloka
{यस्मिन्देशे तु तान्यासन्पतितानि नभस्तलात्}
{अम्लानान्यपि तत्रासन्कुसुमान्यपराण्यपि}


\twolineshloka
{स ततस्तानि जग्राह दिव्यानि रुचिराणि च}
{प्राप्तानि स्वेन तपसा दिव्यगन्धानि भारत}


\twolineshloka
{सम्प्राप्य तानि प्रीतात्मा गुरोर्वचनकारकः}
{तदा जगाम तूर्णं च चम्पां चम्पकमालिनीम्}


\twolineshloka
{स वने निर्जने तात ददर्श मिथुनं नृणाम्}
{चक्रवत्परिवर्तन्तं गृहीत्वा पाणिना करम्}


\twolineshloka
{तत्रैकस्तूर्णमगमत्तत्पदे च विवर्तयन्}
{एकस्तु न तदा राजंश्चक्रतुः कलहं ततः}


\twolineshloka
{त्वं शीघ्रं गच्छसीत्येकोऽब्रवीन्नेति तथाऽपरः}
{पतितेति च तौ राजन्परस्परमथोचतुः}


\twolineshloka
{तयोर्विस्पर्धतोरेवं शपथोऽयमभूत्तदा}
{सहसोद्दिश्य विपुलं ततो वाक्यमथोचतुः}


\twolineshloka
{आवयोरनृतं प्राह यस्तस्याभूद्द्विजस्य वै}
{विपुलस्य परे लोके या गतिः सा भवेदिति}


\twolineshloka
{एतच्छ्रुत्वा तु विपुलो विषण्णवदनोऽभवत्}
{एवं तीव्रतपाश्चाहं कष्टश्चायं परिश्रमः}


\twolineshloka
{मिथुनस्यास्य किं मे स्यात्कृतं पापं यथा गतिः}
{अनिष्टा सर्वभूतानां कीर्तिताऽनेन मेऽद्य वै}


\twolineshloka
{एवं सञ्चिन्तयन्नेव विपुलो राजसत्तम}
{अवाङ्मुखो दीनमना दध्यौ दुष्कृतमात्मनः}


\twolineshloka
{ततः षडन्यान्पुरुषानक्षैः काञ्चनराजतैः}
{अपश्यद्दीव्यमानान्वै लोभामर्षान्वितांस्तथा}


\twolineshloka
{कुर्वतः शपथं तेन यः कृतो मिथुनेन तु}
{विपुलं वै समुद्दिश्य तेपि वाक्यमथाब्रुवन्}


\threelineshloka
{लोभमास्थाय योऽस्माकं विषमं कर्तुमुत्सहेत्}
{विपुलस्य परे लोके या गतिस्तामवाप्नुयात्}
{}


% Check verse!
एतच्छ्रुत्वा तु विपुलो नापश्यद्धर्मसङ्करम् ॥जन्मप्रभृति कौरव्य कृतपूर्वमथात्मनः
\twolineshloka
{स प्रदध्यौ तथा राजन्नग्नावग्निरिवाहितः}
{दह्यमानेन मनसा शापं श्रुत्वा तथाविधम्}


\twolineshloka
{तस्य चिन्तयतस्तात बह्वीर्वाचो निशम्य तु}
{इदमासीन्मनसि च रुच्या रक्षणकारितम्}


\twolineshloka
{लक्षणं लक्षणेनैव वदनं वदनेन च}
{विधाय न मया चोक्तं सत्यमेतद्गुरोस्तथा}


\twolineshloka
{एतदात्मनि कौरव्य दुष्कृतं विपुलस्तदा}
{अमन्यत महाभाग तथा तच्च न संशयः}


\twolineshloka
{स चम्पां नगरीमेत्य पुष्पाणि गुरवे ददौ}
{पूजयामास च गुरुं विधिवत्स गुरुप्रियः}


\chapter{अध्यायः ७८}
\threelineshloka
{तमागतमभिप्रेक्ष्य शिष्यं वाक्यमथाब्रवीत्}
{देवशर्मा महातेजा यत्तच्छृणु जनाधिप ॥देवशर्मोवाच}
{}


\threelineshloka
{किं त्वया मिथुनं दृष्टं तस्मिञ्शिष्य महावने}
{ते त्वां जानन्ति निपुणा आत्मा च रुचिरेव च ॥विपुल उवाच}
{}


\threelineshloka
{ब्रह्मर्षे मिथुनं किन्तत्के च षट्पुरुषा विभो}
{ये मां जानन्ति तत्त्वेन यन्मां त्वं परिपृच्छसि ॥देवशर्मोवाच}
{}


\twolineshloka
{यद्वै तन्मिथुनं ब्रह्मिन्नहोरात्रं हि विद्धि तत्}
{चक्रवत्परिवर्तेत तत्ते जानाति दुष्कृतम्}


\twolineshloka
{ये च ते पुरुषा विप्र अक्षैर्दीव्यन्ति हृष्टवत्}
{ऋतूंस्तानभिजानीहि ते ते जानन्ति दुष्कृतम्}


\twolineshloka
{न मां कश्चिद्विजानीत इति कृत्वा न विश्वसेत्}
{नरो रहसि पापात्मा पापकं कर्म वै द्विज}


\twolineshloka
{कुर्वाणं हि नरं कर्म पापं रहसि सर्वदा}
{पश्यन्ति ऋतवश्चापि तथा दिननिशेऽप्युत}


\twolineshloka
{तथैव हि भवेयुस्ते लोकाः पापकृतो यथा}
{कृत्वाऽनाचक्षतः कर्म मम तच्च यथा कृतम्}


\twolineshloka
{ते त्वां हर्षस्मितं दृष्ट्वा गुरोः कर्मानिवेदकम्}
{स्मारयन्तस्तथा प्राहुस्ते यथा श्रउतवान्भवान्}


\twolineshloka
{अहोरात्रं विजानाति ऋतवश्चापि नित्यशः}
{पुरुषे पापकं कर्म शुभं वाऽशुभकर्मिणः}


\twolineshloka
{तत्त्वया मम यत्कर्म व्यभिचाराद्भयात्मकम्}
{नाख्यातमिति जानन्तस्ते त्वामाहुस्तथा द्विज}


\twolineshloka
{तेनैव हि भवेयुस्ते लोकाः पापकृतो यथा}
{कृत्वा नाचक्षतः कर्म मम यच्च त्वया कृतम्}


\twolineshloka
{तथाऽशक्याश्च दुर्वृत्ता रक्षितुं प्रमदा द्विज}
{न च त्वं कृतवान्किंचिदागः प्रीतोस्मि तेन ते}


\twolineshloka
{`मनोदोषविहीनानां न दोषः स्यात्तथा तव}
{अन्यथाऽऽलिङ्ग्यते कान्ता स्नेहेन दुहिताऽन्यथा}


\twolineshloka
{यतेश्च कामुकानां च योषिद्रूपेऽन्यथा मतिः}
{अशिक्षयैव मनसः प्रायो लोकस्तु वञ्च्यते}


\twolineshloka
{लालेत्युद्विजते लोको वक्त्रासव इति स्पहा}
{अबन्धायोग्यमनसामिति मन्त्रात्मदैवकम्}


\twolineshloka
{न रागस्नेहलोभान्धं कर्मिणां तन्महाफलम्}
{निष्कषायो विशुद्धस्त्वं रुच्यावेशान्न दूषितः}


\twolineshloka
{यदि त्वहं त्वां दुर्वृत्तमद्राक्षं द्विजसत्तम}
{शपेयं त्वामहं क्रोधान्न मेऽत्रास्ति विचारणा}


\twolineshloka
{सञ्जन्ति पुरुषे नार्यः पुंसां सोऽर्थश्च पुष्कलः}
{अन्यथा रक्षतः शापोऽभविष्यत्ते मतिश्च मे}


\twolineshloka
{रक्षिता च त्वया पुत्र मम चापि निवेदिता}
{अहं ते प्रीतिमांस्तात स्वस्थः स्वर्गं गमिष्यसि}


\twolineshloka
{इत्युक्त्वा विपुलं प्रीतो देवशर्मा महानृषिः}
{मुमोद स्वर्गमास्थाय सहभार्यः सशिष्यकः}


\twolineshloka
{इदमाख्यातवांश्चापि ममाख्यानं महामुनिः}
{मार्कण्डेयः पुरा राजन्गङ्गाकूले कथान्तरे}


\twolineshloka
{तस्माद्ब्रवीमि पार्थ त्वां स्त्रियो रक्ष्याः सदैव च}
{उभयं दृश्यते तासु सततं साध्वसाधु च}


\twolineshloka
{स्त्रियः साध्व्यो महाभागाः सम्मता लोकमातरः}
{धारयन्ति महीं राजन्निमां सवनकाननाम्}


\twolineshloka
{असाध्व्यश्चापि दुर्वृत्ताः कुलघ्नाः पापनिश्चयाः}
{विज्ञेया लक्षणैर्दुष्टैः स्वगात्रसहजैर्नृप}


\twolineshloka
{एवमेतासु रक्षा वै शक्या कर्तुं महात्मभिः}
{अन्यथा राजशार्दूल न शक्या रक्षितुं स्त्रियः}


\twolineshloka
{एता हि मनुजव्याघ्र तीक्ष्णास्तीक्ष्णपराक्रमाः}
{नासामस्ति प्रियो नाम मैथुने सङ्गमेति यः}


\twolineshloka
{एताः कृत्याश्च कष्टाश्च कृतघ्ना भरतर्षभ}
{न चैकस्मिन्नमन्त्येताः पुरुषे पाण्डुनन्दन}


\threelineshloka
{नासु स्नेहो नरैः कार्यस्तथैवेर्ष्या जनेश्वर}
{खेदमास्थाय भुञ्जीत धर्ममास्थाय चैव ह}
{`अनृताविह पर्वादिदोषवर्जं नराधिप ॥'}


\twolineshloka
{विहन्येतान्यथा कुर्वन्नरः कौरवनन्दन}
{सर्वथा राजशार्दूल युक्तः सर्वत्र युज्यते}


\twolineshloka
{तेनैकेन तु रक्षा वै विपुलेन कृता स्त्रियाः}
{नान्यः शक्तस्त्रिलोकेऽस्मिन्रक्षितुं नृप योषितः}


\chapter{अध्यायः ७९}
\twolineshloka
{यन्मूलं सर्वधर्माणां प्रजनस्य गृहस्य च}
{पितृदेवातिथीनां च तन्मे ब्रूहि पितामह}


\threelineshloka
{अयं हि सर्वधर्माणां धर्मश्चिन्त्यतमो मतः}
{कीदृशाय प्रदेया स्यात्कन्येति वसुधाधिप ॥भीष्म उवाच}
{}


\threelineshloka
{शीलवृत्ते समाज्ञाय विद्यां योनिं च कर्म च}
{अद्भिरेव प्रदातव्या कन्या गुणवते भवेत्}
{ब्राह्मणानां सतामेष नित्यं धर्मो युधिष्ठिर}


\twolineshloka
{आवाह्यमावहेदेवं यो दद्यादनुकूलतः}
{शिष्टानां क्षत्रियाणां च धर्म एष सनातनः}


\threelineshloka
{आत्माभिप्रेतमुत्सृज्य कन्याभिप्रेत एष यः}
{अभिप्रेता च या यस्य तस्मै देया युधिष्ठिर}
{गान्धर्वमिति तं धर्मं प्राहुर्वेदविदो जनाः}


\twolineshloka
{धनेन बहुधा क्रीत्वा सम्प्रलोभ्य च बान्धवान्}
{असुराणां नृपैतं वै धर्ममाहुर्मनीषिणः}


\twolineshloka
{हत्वा छित्त्वा च शीर्षाणि रुदतां रुदतीं गृहात्}
{प्रसह्य हरणं तात राक्षसो विधिरुच्यते}


\twolineshloka
{`सुसां मत्तां प्रमत्तां वा रहो रात्रौ च गच्छति}
{स पापिष्ठो विवाहानां पैशाचः कथितोऽधमः}


\twolineshloka
{पञ्चानां तु त्रयो धर्म्या द्वावधर्म्यौ युधिष्ठिर}
{पैशाचश्चासुरश्चैव न कर्तव्यौ कथञ्चन}


\twolineshloka
{ब्राह्मः क्षात्रोऽथ गान्धर्व एते धर्म्या नरर्षभ}
{पृथग्वा यदि वा मिश्राः कर्तव्या नात्र संशयः}


\twolineshloka
{तिस्त्रो भार्या ब्राह्मणस्य द्वे भार्ये क्षत्रियस्य तु}
{वैश्यः स्वजात्यां विन्देत तास्वपत्यं हिताय हि}


\twolineshloka
{द्विजस्य ब्राह्म्णी श्रेष्ठा क्षत्रिया क्षत्रियस्य तु}
{रत्यर्थमपि शूद्रा स्यान्नेत्याहुरपरे जनाः}


\twolineshloka
{अपत्यजन्म शूद्रायां न प्रशंसन्ति साधवः}
{शूद्रायां जनयविन्प्रः प्रायश्चित्ती विधीयते}


\twolineshloka
{`नातिबालां वहन्त्यन्ते अनित्यत्वात्प्रजार्थिनः}
{वहन्ति कर्मिणस्तस्यामन्तः शुद्धिव्यपेक्षया}


\twolineshloka
{अपरान्वयसम्भूतां संस्वप्नादिविवर्जिताम्}
{कामो यस्यां निषिद्धश्च केचिदिच्छन्ति चापदि}


\twolineshloka
{त्रिंशद्वर्षो दशवर्षां भार्यां विन्देत नग्निकाम्}
{एकविंशतिवर्षो वा सप्तवर्षामवाप्नुयात्}


\twolineshloka
{यस्यास्तु न भवेद्भाता पिता वा भरतर्षभ}
{नोपयच्छेत तां जातु पुत्रिकाधर्मिणी हि सा}


\twolineshloka
{त्रीणि वर्षाण्युदीक्षेत कन्या ऋतुमती सती}
{चतुर्थेऽत्वथ सम्प्राप्ते स्वयं भर्तारमर्जयेत्}


\twolineshloka
{प्रजा न हीयते तस्या रतिश्च भरतर्षभ}
{अतोऽन्यथा वर्तमाना भवेद्वाच्या प्रजापतेः}


\threelineshloka
{असपिण्डा च या मातुरसगोत्रा च या पितुः}
{इत्येतामुपयच्छेत तं धर्मं मनुरब्रवीत् ॥युधिष्ठिर उवाच}
{}


\twolineshloka
{शुल्कमन्येन दत्तं स्याद्ददानीत्याह चापरः}
{बलादन्यः प्रभाषेत धनमन्यः प्रदर्शयेत्}


\threelineshloka
{पाणिग्रहीत्ता चान्यः स्यात्कस्य भार्या पितामह}
{तत्त्वं जिज्ञासमानानां चक्षुर्भवतु नो भवान् ॥भीष्म उवाच}
{}


\twolineshloka
{यत्किञ्चित्कर्म मानुष्यं संस्थानाय प्रदृश्यते}
{मन्त्रवन्मन्त्रितं तस्य मृषावादस्तु पातकः}


\twolineshloka
{भार्यापत्यृत्विगाचार्याः शिष्योपाध्याय एव च}
{मृषोक्ते दण्डमर्हन्ति नेत्याहुरपरे जनाः}


\twolineshloka
{नह्यकामेन संवादं मनुरेवं प्रशंसति}
{अयशस्यमधर्म्यं च यन्मृषा धर्मगोपनम्}


\twolineshloka
{नैकान्तो दोष एकस्मिंस्तदा केनोपपद्यते}
{धर्मतो यां प्रयच्छन्ति यां च क्रीणन्ति भारत}


\twolineshloka
{बन्धुभिः समनुज्ञाते मन्त्रहोमौ प्रयोजयेत्}
{तथा सिध्यन्ति ते मन्त्रा नादत्तायाः कथञ्चन}


\twolineshloka
{यस्त्वत्र मन्इत्रसमयो भार्यापत्योर्मिथः कृतः}
{तमेवाहुर्गरीयांसं यश्चासौ ज्ञातिभिः कृतः}


\threelineshloka
{देवदत्तां पतिर्भार्यां वेत्ति धर्मस्य शासनात्}
{स दैवीं मानुषीं वाचमनृतां पर्युदस्यति ॥युधिष्ठिर उवाच}
{}


\twolineshloka
{कन्यायां प्राप्तसुल्कायां ज्यायांश्चेदाव्रजेद्वरः}
{धर्मकामार्थसम्पन्नो वाच्यमत्रानृतं न वा}


\twolineshloka
{तस्मिन्नुभयतो दोषे कुर्वञ्श्रेयः समाचरेत्}
{अयं नः सर्वधर्माणां धर्मश्चिन्त्यतमो मतः}


\threelineshloka
{तत्त्वं जिज्ञासमानानां चक्षुर्भवतु नो भवान्}
{तदेतत्सर्वमाचक्ष्व न हि तृप्यामि कथ्यताम् ॥भीष्म उवाच}
{}


\threelineshloka
{नैव निष्ठाकरं शुल्कं ज्ञात्वाऽऽसीत्तेन नानृतम्}
{न हि शुल्कपराः सन्तः कन्यां ददति कर्हिचित्}
{अन्यैर्गुणैरुपेतं तु शुल्कं याचन्ति बान्धवाः}


\threelineshloka
{अलङ्कृत्वा वहस्वेति यो दद्यादनुकूलतः}
{यच्च तां च ददत्येवं न शुल्कं विक्रयो न सः}
{प्रतिगृह्य भवेद्देमेव धर्मः सनातनः}


\twolineshloka
{दास्यामि भवते कन्यामिति पूर्वं नभाषितम्}
{ये चाहुर्ये च नाहुर्ये ये चावश्यं वदन्त्युत}


\twolineshloka
{तस्मादाग्रहणात्पाणेर्याचयन्ति परस्परम्}
{कन्यावरः पुरा दत्तो मरुद्भिरिति नः श्रुतम्}


\twolineshloka
{नानिष्टाय प्रदातव्या कन्या इत्यृषिचोदितम्}
{तन्मूलं काममूलस्य प्रजनस्येति मे मतिः}


\twolineshloka
{समीक्ष्य च बहून्दोषान्संवासाद्विद्धि पाणयोः}
{यथा निष्ठाकरं शुल्कं न जात्वासीत्तथा शृणु}


\twolineshloka
{अहं विचित्रवीर्यस्य द्वे कन्ये समुदावहम्}
{जित्वाऽङ्गमागधान्सर्वान्काशीनथ च कोसलान्}


\threelineshloka
{गृहीतपाणिरेकाऽऽसीत्प्राप्तशुल्काऽपराऽभवत्}
{कन्याऽगृहीता तत्रैव विसर्ज्या इति मे पिता}
{अब्रवीदितरां कन्यामावहेति स कौरवः}


\twolineshloka
{अप्यन्याननुपप्रच्छ शङ्कमानः पितुर्वचः}
{अतीव ह्यस्य धर्मेच्छा पितुर्मेऽभ्यधिकाऽभवत्}


\twolineshloka
{ततोहमब्रवं राजन्नाचारेप्सुरिदं वचः}
{आचारं तत्त्वतो वेत्तुमिच्छामि च पुनःपुनः}


% Check verse!
ततो मयैवमुक्ते तु वाक्ये धर्मभृतावरः ॥पिता मम महाराज बीह्लीको वाक्यमब्रवीत्
\twolineshloka
{यदि व शुल्कतो निष्ठा न पाणिग्रहणात्तथा}
{लाजान्तरमुपासीत प्राप्तशुल्क इति स्मृतिः}


\twolineshloka
{न हि धर्मविदः प्राहुः प्रमाणं वाक्यतः स्मृतम्}
{येषां वै शुल्कतो निष्ठा न पाणिग्रहणात्तथा}


\twolineshloka
{प्रसिद्धं भाषितं दाने नैषां प्रत्यायकं पुनः}
{ये मन्यन्ते क्रयं शुल्कं न ते धर्मविदो नराः}


\twolineshloka
{न चैतेभ्यः प्रदातव्या न वोढव्या तथाविधा}
{न ह्येव भार्या क्रेतव्या न विक्रय्यां कथञ्चन}


\twolineshloka
{ये च क्रीणन्ति दासीवद्विक्रीणन्ति तथैव च}
{भवेत्तेषां तथा निष्ठा लुब्धानां पापचेतसाम्}


\twolineshloka
{अस्मिन्नर्थे सत्यवन्तं पर्यपृच्छन्त वै जनाः}
{कन्यायाः प्राप्तशुल्कायाः शुल्कदः प्रशमं गतः}


\twolineshloka
{पाणिग्रहीता वाऽन्यः स्यादत्र नो धर्मसंशयः}
{तन्नश्छिन्धि महाप्राज्ञ त्वं हि वै प्राज्ञसम्मतः}


\twolineshloka
{तत्त्वं जिज्ञासमानानां चक्षुर्भवतु नो भवान्}
{तानेवं ब्रुवतः सर्वान्सत्यवान्वाक्यमब्रवीत्}


\twolineshloka
{यत्रेष्टं तत्र देया स्यान्नात्र कार्या विचारणा}
{कुर्वते जीवतोप्येवं मृते नैवास्ति संशयः}


\twolineshloka
{देवरं प्रविशेत्कन्या तप्येद्वाऽपि तपः पुनः}
{तमेवानुगता भूत्वा पाणिग्राहस्य नाम सा}


\twolineshloka
{लिखन्त्येव तु केषांचिदपरेषां शनैरपि}
{इति ये संवदन्त्यत्र त एतं निश्चयं विदुः}


\twolineshloka
{तत्पाणिग्रहणात्पूर्वमन्तरं यत्र वर्तते}
{सर्वमङ्गलमन्त्रं वै मृषावादस्तु पातकः}


\twolineshloka
{पाणिग्रहणमन्त्राणां निष्ठा स्यात्सप्तमे पदे}
{पाणिग्रहस्य भार्या स्याद्यस्य चाद्भिः प्रदीयते}


\threelineshloka
{इति देयं वदन्त्यत्र त एतं निश्चयं विदुः}
{अनुकूलामनुवशां भ्रात्रा दत्तामुपाग्निकाम्}
{परिक्रम्य यथान्यायं भार्यां विन्देद्द्विजोत्तमः}


\chapter{अध्यायः ८०}
\threelineshloka
{कन्यायां प्राप्तशुल्कायां पतिश्चेन्नास्ति कश्चन}
{तत्र का प्रतिपत्तिः स्यात्तन्मे ब्रूहि पितामह ॥भीष्म उवाच}
{}


\twolineshloka
{या पुत्रकस्य ऋद्धस्य प्रतिपाल्या तदा भवेत्}
{अथवा सा हरेच्छुल्कं क्रीता शुल्कप्रदस्य सा}


\twolineshloka
{तस्यार्थेऽपत्यमीहेव येन न्यायेन शक्नुयात्}
{न तस्मान्मन्त्रवत्कार्यं कश्चित्कुर्वीत किञ्चन}


\twolineshloka
{स्वयंवृतेन साज्ञप्ता पित्रा वै प्रत्यपद्यत}
{तत्तस्यान्ये प्रशंसन्ति धर्मज्ञा नेतरे जनाः}


\twolineshloka
{एतत्तु नापरे चक्रुरपरे जातु साधवः}
{साधूनां पुनराचारो गरीयान्धर्मलक्षणः}


\twolineshloka
{अस्मिन्नेव प्रकारे तुसुक्रतुर्वाक्यमब्रवीत्}
{नप्ता विदेहराजस्य जनकस्य महात्मनः}


\twolineshloka
{असदाचरिते मार्गे कथं स्यादनुकीर्तनम्}
{अनुप्रश्नः संशयो वा सतामेवमुपालभेत्}


\twolineshloka
{असदेव हि धर्मस्य प्रदानं धर्म आसुरः}
{नानुशुश्रुम जात्वेनामिमां पूर्वेषु कर्मसु}


\threelineshloka
{भार्यापत्योर्हि सम्बन्धः स्त्रीपुंसोस्तुल्य एव तु}
{रतिः साधारणो धर्म इति चाह स पार्थिवः ॥युधिष्ठिर उवाच}
{}


\threelineshloka
{अथ केन प्रमाणेन पुंसामादीयते धनम्}
{पुत्रवद्धि पितुस्तस्य कन्या भवितुमर्हति ॥भीष्म उवाच}
{}


\twolineshloka
{यथैवात्मा तथा पुत्रः पुत्रेण दुहिता समा}
{तस्यामात्मनि तिष्ठन्त्यां कथमन्यो धनं हरेत्}


\twolineshloka
{मातुश्च यौतकं यत्स्यात्कुमारीभाग एव सः}
{दौहित्र एव तद्रिक्थमपुत्रस्य पितुर्हरेत्}


\twolineshloka
{ददाति हि स पिण्डान्वै पितुर्मातामहस्य च}
{पुत्रदौहित्रयोरेव विशेषो नास्ति धर्मतः}


\twolineshloka
{अन्यत्र जामया सार्धं प्रजानां पुत्र ईहते}
{दुहिताऽन्यत्र जातेन पुत्रेणापि विशिष्यते}


\twolineshloka
{दौहित्रकेण धर्मेण तत्र पश्यामि कारणम्}
{विक्रीतासु हि ये पुत्रा भवन्ति पितुरेव ते}


\twolineshloka
{असूयवस्त्वधर्मिष्ठाः परस्वादायिनः शठाः}
{आसुरादधिसम्भूता धर्माद्विषमवृत्तयः}


\twolineshloka
{अत्र गाथा यमोद्गीताः कीर्तयन्ति पुराविदः}
{धर्मज्ञा धर्मशास्त्रेषु निबद्धा धर्मसेतुषु}


\twolineshloka
{यो मनुष्यः स्वकं पुत्रं विक्रीय धनमिच्छति}
{कन्यां वा जीवितार्थाय यः शुल्केन प्रयच्छति}


\twolineshloka
{सप्तावरे महाघोरे निरये कालसाह्वये}
{स्वेदं मूत्रं पुरीषं च तस्मिन्मूढः समश्नुते}


\twolineshloka
{आर्षे गोमिथुनं शुल्कं केचिदाहुर्मृषैव तत्}
{अल्पो वा बहु वा राजन्विक्रयस्तावदेव सः}


\twolineshloka
{यद्यप्याचरितः कैश्चिन्नैष धर्मः सनातनः}
{अन्येषामपि दृश्यन्ते लोभतः सम्प्रवृत्तयः}


\twolineshloka
{वश्यां कुमारीं बलतो ये तां समुपभुञ्जते}
{एते पापस्य कर्तारस्तमस्यन्धे च शेरते}


\twolineshloka
{अन्योप्यथ न विक्रेयो मनुष्यः किं पुनः प्रजाः}
{अधर्ममूलैर्हि धनैस्तैर्न धर्मोऽथ कश्चन}


\chapter{अध्यायः ८१}
\twolineshloka
{प्राचेतसस्य वचनं कीर्तयन्ति पुराविदः}
{यस्याः किञ्चिन्नाददते ज्ञातयो न स विक्रयः}


\twolineshloka
{अर्हणं तत्कुमारीणामानृशंस्यं व्रतं च तत्}
{सर्वं च प्रतिदेयं स्यात्कन्यायै तदशेषतः}


\twolineshloka
{पितृभिर्भ्रातृभिश्चापि श्वशुरैरथ देवरैः}
{पूज्या लालयितव्याश्च बहुकल्याणमीप्सुभिः}


\twolineshloka
{यदि वै स्त्री न रोचेत पुमांसं न प्रमोदयेत्}
{अप्रमोदात्पुनः पुंसः प्रजनो न प्रवर्धते}


\twolineshloka
{पूज्या लालयितव्याश्च स्त्रियो नित्यं जनाधिप}
{स्त्रियो यत्र च पूज्यन्ते रमन्ते तत्र देवताः}


\twolineshloka
{अपूजिताश्च यत्रैताः सर्वास्तत्राफलाः क्रियाः}
{तदा चैतत्कुलं नास्ति यदा शोचन्ति जामयः}


\twolineshloka
{जामीशप्तानि गेहानि निकृत्तानीव कृत्यया}
{नैव भान्ति न वर्धन्ते श्रिया हीनानि पार्थिव}


\twolineshloka
{स्त्रियः पुंसां परिददौ मनुर्जिगमिषुर्दिवम्}
{अबलाः स्वल्पकौपीनाः सुहृद सत्यजिष्णवः}


\twolineshloka
{ईर्षवो मानकामाश्च चण्डाश्च सुहृदोऽबुधाः}
{स्त्रियस्तु मानमर्हन्ति ता मानयत मानवाः}


\twolineshloka
{स्त्रीप्रत्ययो हि वै धर्मो रतिभोगाश्च केवलाः}
{परिचर्या नमस्कारास्तदायत्ता भवन्तु वः}


\twolineshloka
{उत्पादनमपत्यस्य जातस्य परिपालनम्}
{प्रीत्यर्थं लोकयात्रायाः पश्यत स्त्रीनिबन्धनम्}


\twolineshloka
{सम्मान्यमानाश्चैता हि सर्वकार्याष्यवाप्स्यथ}
{विदेहराजदुहिता चात्र श्लोकमगायत}


\twolineshloka
{नास्ति यज्ञः स्त्रियाः कश्चिन्न श्राद्धं नोप्रवासकम्}
{धर्मः स्वभर्तृशुश्रूषा तया स्वर्गं जयन्त्युत}


\twolineshloka
{पिता रक्षति कौमारे भर्ता रक्षति यौवने}
{पुत्राश्च स्थाविरे भावे न स्त्री स्वातन्त्र्यमर्हति}


\twolineshloka
{श्रिय एताः स्त्रियो नाम सत्कार्या भूतिमिच्छता}
{लालिताऽनुगृहीता च श्रीः स्त्री भवति भारतः}


\chapter{अध्यायः ८२}
\twolineshloka
{सर्वशास्त्रविधानज्ञ राजधर्मविदुत्तम}
{अतीव संशयच्छेत्ता भवान्वै प्रथितः क्षितौ}


\twolineshloka
{कश्चित्तु संशयो मेऽस्ति तन्मे ब्रूहि पितामह}
{`अस्यामापदि कष्टायामन्यं पृच्छाम कं वयम्' ॥जातेऽस्मिन्संशये राजन्नान्यं पृच्छेम कञ्चन}


\twolineshloka
{यथा नरेण कर्तव्यं धर्ममार्गानुवर्तिना}
{एतत्सर्वं महाबाहो भवान्व्याख्यातुमर्हति}


\twolineshloka
{चतस्रो विहिता भार्या ब्राह्मणस्य पितामह}
{ब्राह्मणी क्षत्रिया वैश्या शूद्रा च रतिमिच्छतः}


\twolineshloka
{तत्र जातेषु पुत्रेषु सर्वासां कुरुसत्तम}
{आनुपूर्व्येण कस्तेषां पित्र्यं दायाद्यमर्हति}


\threelineshloka
{केन वा किं ततो हार्यं पितृवित्तात्पितामह}
{एतदिच्छामि कथितं विभागस्तेषु कः स्मृतः ॥भीष्म उवाच}
{}


\twolineshloka
{ब्राह्मणः क्षत्रियो वैश्यस्त्रयो वर्णा द्विजातयः}
{एतेषु विहितो धर्मो ब्राह्मणस्य युधिष्ठिर}


\twolineshloka
{वैषम्यादथवा लोभात्कामाद्वाऽपि परन्तप}
{ब्राह्मणस्य भवेच्छूद्रा न तु दृष्टा न तु स्मृता}


\twolineshloka
{शूद्रां शयनमारोप्य ब्राह्मणि यात्यधोगतिम्}
{प्रायश्चित्तीयते चापि विधिदृष्टेन कर्मणा}


\twolineshloka
{तत्र जातेष्वपत्येषु द्विगुणं स्वाद्युधिष्ठिर}
{अतस्ते नियमं वित्ते सम्प्रवक्ष्यामि भारत}


\twolineshloka
{लक्षण्यं गोवृषो यानं यत्प्रधानतमं भवेत्}
{ब्राह्मण्यास्तद्धरेत्पुत्र एकांशं वै पितुर्धनात्}


\twolineshloka
{एषं तु दशधा कार्यं ब्राह्मणस्वं युधिष्ठिर}
{तत्र तेनैव हर्तव्याश्चत्वारोंशाः पितुर्धनात्}


\twolineshloka
{त्रियायास्तु यः पुत्रो ब्राह्मणः सोप्यसंशयः}
{स तु मातुर्विशेषेण त्रीनंशान्हर्तुमर्हति}


\twolineshloka
{वर्णे तृतीये जातस्तु वैश्यायां ब्राह्मणादपि}
{द्विरंशस्तेन हर्तव्यो ब्राह्मणस्वाद्युधिष्ठिर}


\twolineshloka
{शूद्रायां ब्राह्मणाज्जातो नित्यादेयधनः स्मृतः}
{अल्पं चापि प्रदातव्यं शूद्रापुत्राय भारत}


\twolineshloka
{दशधा प्रविभक्तस्य धनस्यैव भवेत्क्रमः}
{सवर्णासु तु जातानां समान्भागान्प्रकल्पयेत्}


\twolineshloka
{अब्राह्मणं तु मन्यन्ते शूद्रापुत्रमनैपुणात्}
{त्रिषु वर्णेषु जातो हि ब्राह्मणाद्ब्राह्मणो भवेत्}


\twolineshloka
{स्मृताश्च वर्णाश्चत्वारः पञ्चमो नाधिगम्यते}
{हरेच्च दशमं भागं शूद्रापुत्रः पितुर्धनात्}


\twolineshloka
{तत्तु दत्तं हरेत्पित्रा नादत्तं हर्तुमर्हति}
{अवश्यं हि धनं देयं शूद्रापुत्राय भारत}


\twolineshloka
{आनृशंस्यं परो धर्म इति तस्मै प्रदीयते}
{यत्रतत्र समुत्पन्नं गुणायैवोपपद्यते}


\twolineshloka
{यद्यप्येष सुपुत्रः स्यादपुत्रो यदि वा भवेत्}
{नाधिकं दशमाद्दद्याच्छूद्रापुत्राय भारत}


\twolineshloka
{`स्मृत एकश्चतुर्भागः कन्याभागस्तु धर्मतः}
{अभ्रातृका समग्राहा चार्धास्येत्यपरे विदुः ॥'}


\twolineshloka
{त्रैवार्विकाद्यदा भक्तादधिकं स्याद्द्विजस्य तु}
{यजेत तेन द्रव्येण न वृथा साधयेद्धनम्}


\twolineshloka
{त्रिसहस्रपरो दायः स्त्रियै देयो धनस्य वै}
{भर्त्रा तच्च धनं दत्तं यथार्हं भोक्तुमर्हति}


\twolineshloka
{स्त्रीणां तु पतिदायाद्यमुपभोगफलं स्मृतम्}
{नापहारं स्त्रियः कुर्युः पितृवित्तात्कथञ्चन}


\twolineshloka
{स्त्रियास्तु यद्भवेद्वित्तं पित्रा दत्तं युधिष्ठिर}
{ब्राह्मण्यास्तद्धरेत्कन्या यथा पुत्रस्तथाऽस्य सा}


\fourlineindentedshloka
{सा हि पुत्रसमा राजन्विहिता कुरुनन्दन}
{एवमेव समुद्दिष्टो धर्मो वै भरतर्षभ}
{एवं धर्ममनुस्मृत्य न वृथा साधयेद्धनम् ॥युधिष्ठिर उवाच}
{}


\twolineshloka
{शूद्रायां ब्राह्मणाज्जातो यद्यदेयधनः स्मृतः}
{केन प्रतिविशेषेण दशमोऽप्यस्य दीयते}


\twolineshloka
{ब्राह्मण्यां ब्राह्मणाज्जातो ब्राह्मणः स्यान्न संशयः}
{क्षत्रियायां तथैव स्याद्वैश्यायामपि चैव हि}


\threelineshloka
{कस्मात्तु विषमं भागं भजेरन्नृपसत्तम}
{यदा सर्वे त्रयो वर्णास्त्वयोक्ता ब्राह्मणा इति ॥भीष्म उवाच}
{}


\twolineshloka
{दारा इत्युच्यते लोके नाम्नैकेन परन्तप}
{प्रोक्तेनि चैव नाम्नाऽयं विशेषः सुमहान्भवेत्}


\twolineshloka
{तिस्रः कृत्वा पुरो भार्याः पश्चाद्विन्देत ब्राह्मणीम्}
{सा ज्येष्ठा सा च पूज्या स्यात्सा च ताभ्यो गरीयसी}


\twolineshloka
{स्नानं प्रसाधनं भर्तुर्दन्तधावनमुञ्जनम्}
{हव्यं कव्यं च यच्चान्यद्धर्मयुक्तं गृहे भवेत्}


\twolineshloka
{न तस्यां जातु तिष्ठन्त्यामन्या तत्कर्तुमर्हति}
{ब्राह्मणीत्वेव कुर्याद्वा ब्राह्मणस्य युधिष्ठिर}


\twolineshloka
{अन्नं पानं च माल्यं च वासांस्याभरणानि च}
{ब्राह्मण्यैतानि देयानि भर्तुः सा हि गरीयसी}


\twolineshloka
{मनुनाऽभिहितं शास्त्रं यच्चापि कुरुनन्दन}
{तत्राप्येष महाराज दृष्टो धर्मः सनातनः}


\twolineshloka
{अथ चेदन्यथा कुर्याद्यदि कामाद्युधिष्ठिर}
{यथा ब्राह्मणचाण्डालः पूर्वदृष्टस्तथैव सः}


\twolineshloka
{ब्राह्मण्याः सदृशः पुत्रः क्षत्रियायाश्च यो भवेत्}
{राजन्विशेषो यस्त्वत्र वर्णयोरुभयोरपि}


\twolineshloka
{न तु जात्या समा लोके ब्राह्मण्याः क्षत्रिया भवेत्}
{ब्राह्मण्याः प्रथमः पुत्रो भूयान्स्याद्राजसत्तम}


\twolineshloka
{भूयोभूयोपि संहार्यः पितृवित्ताद्युधिष्ठिर}
{यथा न सदृशी जातु ब्राह्मण्याः क्षत्रिया भवेत्}


\twolineshloka
{क्षत्रियायास्तथा वैश्या न जातु सदृशी भवेत्}
{श्रीश्च राज्यं च कोशश्च क्षत्रियाणां युधिष्ठिर}


\threelineshloka
{विहितं दृश्यते राजन्सागरान्तां च मेदिनीम्}
{क्षत्रियो हि स्वधर्मेण श्रियं प्राप्नोति भूयसीम्}
{राजा दण्डधरो राजन्रक्षा नान्यत्र क्षत्रियात्}


\twolineshloka
{ब्राह्मणा हि महाभाग देवानामपि देवताः}
{तेषु राजा प्रवर्तेत पूजया विधिपूर्वकम्}


\twolineshloka
{प्रणीतमृषिभिर्ज्ञात्वा धर्मं शाश्वतमव्ययम्}
{लुप्यमानं स्वधर्मेण क्षत्रियो रक्षति प्रजाः}


\twolineshloka
{दस्युभिर्ह्रियमाणं च धनं दारांश्च सर्वशः}
{सर्वेषामेव वर्णानां त्राता भवति पार्थिवः}


\threelineshloka
{भूयान्स्यात्क्षत्रियापुत्रो वैश्यापुत्रान्न संशयः}
{भूयस्तेनापि हर्तव्यं पितृवित्ताद्युधिष्ठिर ॥युधिष्ठिर उवाच}
{}


\threelineshloka
{उक्तं ते विधिवद्राजन्ब्राह्मणस्य पितामह}
{इतरेषां तु वर्णानां कथं वै नियमो भवेत् ॥भीष्म उवाच}
{}


\twolineshloka
{क्षत्रियस्यापि भार्ये द्वे विहिते कुरुनन्दन}
{तृतीया च भवेच्छूद्रा न तु दृष्टा न तु स्मूता}


\twolineshloka
{एष एव क्रमो हि स्यात्क्षत्रियाणां युधिष्ठिर}
{अष्टधा तु भवेत्कार्यं क्षत्रियस्वं जनाधिप}


\twolineshloka
{क्षत्रियाया हरेत्पुत्रश्चतुरोंशान्पितुर्धनात्}
{युद्धावहारिकं यच्च पितुः स्वात्स हरेत्तु तत्}


\threelineshloka
{वैश्यापुत्रस्तु भागांस्त्रीञ्शूद्रापुत्रस्तथाष्टमम्}
{एकैव हि भवेद्भार्या वैश्यस्य कुरुनन्दन}
{}


\twolineshloka
{द्वितीया तु भवेच्छूद्रा न तु दृष्टा न तु स्मृता ॥वैश्यस्य वर्तमानस्य वैश्यायां भरतर्षभ}
{}


\twolineshloka
{शूद्रायां चापि कौन्तेय तयोर्विनियमः स्मृतः ॥पञ्चधा तु भवेत्कार्यं वैश्यस्वं भरतर्षभ}
{}


\twolineshloka
{तयोरपत्ये वक्ष्यामि विभागं च जनाधिप ॥वैश्यापुत्रेण हर्तव्याश्चत्वारोंशाः पितुर्धनात्}
{}


\twolineshloka
{पञ्चमस्तु स्मृतोः भागः शूद्रापुत्राय भारत ॥सोपि दत्तं हरेत्पित्रा नादत्तं हर्तुमर्हति}
{}


% Check verse!
त्रिमिर्वर्णैः सदा जातः शूद्रो देयधनो भवेत्
\twolineshloka
{शूद्रस्य स्यात्सवर्णैव भार्या नान्या कथञ्चन}
{समभागाश्च पुत्राः स्युर्यदि पुत्रशतं भवेत्}


\twolineshloka
{जातानां समवर्णायाः पुत्राणामविशेषतः}
{सर्वेषामेव वर्णानां समभागो धनात्स्मृतः}


\twolineshloka
{ज्येष्ठस्य भागो ज्येष्ठः स्यादेकांशो यः प्रधानतः}
{एष दायविधिः पार्थ पूर्वमुक्तः स्वयंभुवा}


\twolineshloka
{समवर्णासु जातानां विशेषोऽस्त्यपरो नृप}
{विवाहवैशिष्ट्यकृतः पूर्वपूर्वो विशिष्यते}


\twolineshloka
{हरेज्ज्येष्ठः प्रधानांशमेकभार्यासुतेष्वपि}
{मध्यमो मध्यमं चैव कनीयांस्तु कनीयसम्}


\twolineshloka
{एवं जातिषु सर्वासु सवर्णः श्रेष्ठतां गतः}
{महर्षिरपि चैतद्वै मारीचः काश्यपोऽब्रवीत्}


\chapter{अध्यायः ८३}
\twolineshloka
{अर्थाश्रयाद्वा कामाद्वा वर्णानां चाप्यनिश्चयात्}
{अज्ञानाद्वापि वर्णानां जायते वर्णसङ्करः}


\threelineshloka
{तेषामेतेन विधिना जातानां वर्णसङ्करे}
{को धर्मः कानि कर्माणि तन्मे ब्रूहि पितामह ॥भीष्म उवाच}
{}


\twolineshloka
{चातुर्वर्ण्यस्य कर्माणि चातुर्वर्ण्यं च केवलम्}
{असृजत्स हि यज्ञार्थे पूर्वमेव प्रजापतिः}


\twolineshloka
{भार्याश्चतस्नो विप्रस्य द्वयोरात्मा प्रजायते}
{आनुपूर्व्याद्द्वयोर्हीनौ मातृजात्यौ प्रसूयतः}


\twolineshloka
{परं शवाद्ब्राह्मणस्यैव पुत्रःशूद्रापुत्रं पारशवं तमाहुः}
{शुश्रूषकः स्वस्य कुलस्य स स्या-त्स्वचारित्रं नित्यमथो न जह्यात्}


\twolineshloka
{सर्वानुपायानथ सम्प्रधार्यसमुद्धरेत्स्वस्य कुलस्य तन्त्रम्}
{ज्येष्ठो यवीयानपि यो द्विजस्यशुश्रूषया दानपरायणः स्यात्}


\twolineshloka
{तिस्रः क्षत्रियसम्बन्धाद्द्वयोरात्माऽस्य जायते}
{हीनवर्णास्तृतीयायां शूद्रा उग्रा इति स्मृतिः}


\twolineshloka
{द्वे चापि भार्ये वैश्यस्य द्वयोरात्माऽस्व जायते}
{शूद्रा शूद्रस्य चाप्येका शूद्रमेव प्रजायते}


\twolineshloka
{अतोऽविशिष्टस्त्वधमो गुरुदारप्रधर्षकः}
{ब्राह्यं वर्णं जनयति चातुर्वर्ण्यविगर्हितम्}


\twolineshloka
{अयाज्यं क्षत्रियो व्रात्यंक सूतं स्तोत्रक्रियापरम्}
{वैश्यो वैदेहकं चापि मौद्गल्यमपवर्जितम्}


\threelineshloka
{शूद्रश्चण्डालमत्युग्रं वध्यघ्नं बाह्यवासिनम्}
{ब्राह्मण्यां सम्प्रजायन्त इत्येते कुलपांसनाः}
{एते मतिमतांश्रेष्ठ वर्णसङ्करजाः प्रभो}


\twolineshloka
{बन्दी तु जायते वैश्यान्मागधो वाक्यजीवनः}
{शूद्रान्निषादो मत्स्यघ्नः क्षत्रियायां व्यतिक्रमात्}


\twolineshloka
{शूद्रादायोगवश्चापि वैश्यायां ग्राम्यधर्मिणः}
{ब्राह्मणैरप्रतिग्राह्यस्तक्षा स्वधनजीवनः}


\twolineshloka
{एतेऽपि सदृशान्वर्णाञ्जनयन्ति स्वयोनिषु}
{मातृजात्या प्रसूयन्ते ह्यवरा हीनयोनिषु}


\twolineshloka
{यथा चतुर्षु वर्णेषु द्वयोरात्माऽस्य जायते}
{आनन्तर्यात्प्रजायन्ते यथा बाह्या प्रधानतः}


\twolineshloka
{ते चापि सदृशं वर्णं जनयन्ति स्वयोनिषु}
{परस्परस्य दारेषु जनयन्ति विगर्हितान्}


\twolineshloka
{यथा शूद्रोऽपि ब्राह्मण्यां जन्तुं बाह्यं प्रसूयते}
{एवं बाह्यतराद्बाह्यश्चातुर्वर्ण्यात्प्रजायते}


\twolineshloka
{प्रतिलोमं तु वर्धन्ते बाह्योद्बाह्यतरात्पुनः}
{हीनाद्धीनाः प्रसूयन्ते वर्णाः पञ्चदशैव तु}


\threelineshloka
{अगम्यागमनाच्चैव जायते वर्णसङ्करः}
{बाह्यानामनुजायन्ते सैरन्ध्र्यां मागधेषु च}
{प्रसाधनोपचारज्ञमदासं दासजीवनम्}


\twolineshloka
{क्षत्रा ह्यायोगवं सूते वागुराबन्धजीवनम्}
{मैरेयकं च वैदेहः सम्प्रसूतेऽथ माधुकम्}


\twolineshloka
{निषादो मद्गुरं सूते दासं नावोपजीविनम्}
{मृतपं चापि चाण्डालः श्वपाकमिति विश्रुतम्}


\twolineshloka
{चतुरो मागधी सूते क्रूरान्मायोपजीविनः}
{मांसं स्वादुकरं क्षौद्रं सौगन्धमिति विश्रुतम्}


\twolineshloka
{वैदेहकाच्च पापिष्ठा क्रूरं मायोपजीविनम्}
{निषादान्मद्रनाभं च खरयानप्रयायिनम्}


\twolineshloka
{चण्डालात्पुल्कसं चापि खराश्वगजभोजिनम्}
{भृतचैलप्रतिच्छन्नं भिन्नभाजनभोजिनम्}


\twolineshloka
{आयोगवीषु जायन्ते हीनवर्णास्तु ते त्रयः}
{क्षुद्रो वैदेहकादन्धो बहिर्ग्रामप्रतिश्रयः}


\twolineshloka
{कारावरो निषाद्यां तु चर्मकारः प्रसूयते}
{चाण्डालात्पाण्डुसौपाकस्त्वक्सारव्यवहारवान्}


\twolineshloka
{आहिण्कों निषादेन वैदेह्यां सम्प्रसूयते}
{चण्डालेन तु सौपाकश्चण्डालसमवृत्तिमान्}


\twolineshloka
{निषादी चापि चण्डालात्पुत्रमन्तेवसायिनम्}
{श्मशानगोचरं सूते बाह्यैरपि बहिष्कृतम्}


\twolineshloka
{इत्येते सङ्करे जाताः पितृमातृव्यतिक्रमात्}
{प्रच्छन्नान्वा प्रकाशा वा वेदितव्याः स्वकर्मभिः}


\twolineshloka
{चतुर्णामेव वर्णानां धर्मो नान्यस्य विद्यते}
{वर्णानां धर्महीनेषु सङ्ख्या नास्तीह कस्यचित्}


\twolineshloka
{यदृच्छयोपसम्पन्नैर्यज्ञसाधुबहिष्कृतैः}
{बाह्या बाह्यैश्च जायन्ते यथावृत्ति यथाश्रयम्}


\threelineshloka
{चतुष्पथश्मशानानि शैलांश्चान्यान्वनस्पतीन्}
{कार्ष्णायसमलङ्कारं परिगृह्य च नित्यशः}
{वसेयुरेते विज्ञाता वर्तयन्तः स्वकर्मभिः}


\twolineshloka
{युञ्जन्तो वाऽप्यलङ्कारांस्तथोपकरणानि च}
{गोब्राह्मणाय साहय्यं कुर्वाणा वै न संशयः}


\threelineshloka
{आनृशंस्यमनुक्रोशः सत्यवाक्यं तथा क्षमा}
{स्वशरीरैरपि त्राणं बाह्यानां सिद्धिकारणम्}
{भवन्ति मनुजव्याघ्र तत्र मे नास्ति संशयः}


\twolineshloka
{यथोपदेशं परिकीर्तितासुनरः प्रजायेत विचार्य बुद्धिमान्}
{निहीनयोनिर्हि सुतोऽवसादये-त्तितीर्षमाणं हि यथोपलो जले}


\twolineshloka
{अविद्वांसमलं लोके विद्वांसमपि वा पुनः}
{नयन्ति ह्यपथं नार्यः कामक्रोधवशानुगम्}


\threelineshloka
{स्वभावश्चैव नारीणां नराणामिह दूषणम्}
{अत्यर्थं न प्रसज्जन्ते प्रमदासु विपश्चितः ॥युधिष्ठिर उवाच}
{}


\threelineshloka
{वर्णापेतमविज्ञाय नरं कलुषयोनिजम्}
{आर्यरूपमिवानार्यं कथं विद्यामहे वयम् ॥भीष्म उवाच}
{}


\twolineshloka
{योनिसङ्कलुषे जातं नानाभावसमन्वितम्}
{कर्मभिः सज्जनाचीर्णैर्विज्ञेयाः शुद्धयोनिकाः}


\twolineshloka
{अनार्यत्वमनाचारः क्रूरत्वं निष्क्रियात्मता}
{पुरुषं व्यञ्जयन्तीह लोके कलुषयोनिजम्}


\twolineshloka
{पित्र्यं वा भजते शीलं मातृजं वा तथोभयम्}
{न कथञ्च सङ्कीर्णः प्रवृतिं स्वां नियच्छति}


\twolineshloka
{यथैव सदृसो रूपे मातावित्रोर्हि जायते}
{व्याघ्रबिन्दोस्तथा योनिं पुरुषः स्वां नियच्छति}


\twolineshloka
{कुले स्रोतसि संच्छन्ने यस्य स्याद्योनिसङ्करः}
{संश्रयत्येव तच्छीलं नरोऽल्पमथवा बहु}


\twolineshloka
{आर्यरूपसमाचारं चरन्तं कृतके पथि}
{सवर्णमन्यवर्णं वा स्वशीलं सास्ति निश्चये}


\twolineshloka
{नानावृत्तेषु भूतेषु नानाक्रमरतेषु च}
{जन्मवृत्तसमं लोके सुश्लिष्टं न विरज्यते}


\twolineshloka
{शरीरमिह सत्वेन न तस्य परिकृष्यते}
{ज्येष्ठमध्यावरं सत्वं तुल्यसत्वं प्रमोदते}


\twolineshloka
{ज्यायांसमपि शीलेन विहीनं नैव पूजयेत्}
{अपि शूद्रं च धर्मज्ञं सद्वृत्तमभिपूजयेत्}


\twolineshloka
{आत्मानमाख्याति हि कर्मभिर्नरःसुशीलचारित्रकुलैः शुभाशुभैः}
{प्रनष्टमप्यात्मकुलं तथा नरःपुनः प्रकाशं कुरुते स्वकर्मतः}


\twolineshloka
{योनिष्वेतासु सर्वासु सङ्कीर्णास्वितरासु च}
{यत्रात्मानं न जनयेद्बुधस्तां परिवर्जयेत्}


\chapter{अध्यायः ८४}
\twolineshloka
{ब्रूहि तात कुरुश्रेष्ठ वर्णानां त्वं पृथक् पृथक्}
{कीदृश्यां कीदृशाश्चापि पुत्राः कस्य च के च ते}


\threelineshloka
{विप्रवादाः सुबहवः श्रूयन्ते पुत्रकारिणाम्}
{अत्र नो मुह्यतां राजन्संशयं छेत्तुमर्हसि ॥भीष्म उवाच}
{}


% Check verse!
`आत्मा पुत्रस्तु विज्ञेयः प्रथमो बहुधा परे
\twolineshloka
{स्वे क्षेत्रे संस्कृते यस्तु पुत्रमुत्पादयेत्स्वयम्}
{तमौरसं विजानीयात्पुत्रं प्रथमकल्पितम्}


\twolineshloka
{अग्निं प्रजापतिं चेष्ट्वा वराय प्रतिपादिता}
{पुत्रिका स्याद्दुहितरि सङ्कल्पे वाऽपि वा सुतः}


\twolineshloka
{तल्पे जातः प्रमीतस्य क्लीबस्य पतितस्य वा}
{स्वधर्मेण नियुक्तो यः स पुत्रः क्षेत्रजः स्मृतः}


\twolineshloka
{माता पिता च दद्यातां यमद्भिः पुत्रमापदि}
{सदृशप्रीतिसंयुक्तो विज्ञेयो दत्रिमः सुतः}


\twolineshloka
{सदृशं तु प्रकुर्याद्यं गुणदोषविचक्षणम्}
{पुत्रं पुत्रगुणैर्युक्तं विज्ञेयः स तु कृत्रिमः}


\twolineshloka
{उत्पद्यते यस्य गूढं न च ज्ञायेत कस्यचित्}
{स भवेद्गूढजो नाम तस्य स्याद्यस्य तल्पतः}


\twolineshloka
{मातापितृभ्यामुत्सृष्टस्तयोरन्यतरेण वा}
{यं पुत्रं प्रतिगृह्णीयादपविद्धः स उच्यते}


\twolineshloka
{पितृवेश्मनि कन्या तु यं पुत्रं जनयेद्रहः}
{तं कानीनं वदन्नाम्ना वोढुः कन्यासमुद्भवे}


\twolineshloka
{या गर्भिणी संस्क्रियते ज्ञाताऽज्ञातापि वा सती}
{वोढुः स गर्भो भवति सहोढ इति उच्यते}


\twolineshloka
{क्रीणीयाद्यस्त्वपर्यार्थं मातापित्रोर्यमन्तिकात्}
{स क्रीतकः सुतस्तस्य सदृशोऽसदृशोपि वा}


\twolineshloka
{या पत्या वा परित्यक्ता विधवा वा स्वकेच्छया}
{उत्पादयति पुनर्भूत्वा स पौनर्भव उच्यते}


\twolineshloka
{सा चेदक्षतयोनिः स्याद्गतप्रत्याङ्गताऽपि वा}
{पौनर्भवेन भर्त्रा सा पुनसंस्कारमर्हति}


\twolineshloka
{मातापितृहनो यः स्यात्त्यक्तो वा स्यादकारणम्}
{आत्मानं स्पर्शयेद्यस्तु स्वयंदत्तस्तु स स्मृतः}


\twolineshloka
{यं ब्राह्मणस्तु शूद्रायां कामादुत्पादयेत्सुतम्}
{स पावयन्नेव शवस्तस्मात्पारशवः स्मृतः}


\twolineshloka
{दास्यां वा दासदास्यां वा यः शूद्रस्य सुतो भवेत्}
{सोऽनुज्ञातो हरेदंशमिति धर्मो व्यवस्थितः}


\twolineshloka
{क्षेत्रजादीन्सुतानेतानेकादश यथोदितान्}
{पुत्रप्रतिनिधीनाहुः क्रियालोपान्मनीषिणः}


\twolineshloka
{प्रातॄणामेकजातानामेकश्चेत्पुत्रवान्भवेत्}
{सर्वांस्तांस्तेन पुत्रेण पुत्रिणो मनुरब्रवीत्}


\twolineshloka
{सर्वासामेकपत्नीनामेका चेत्पुत्रिणी भवेत्}
{सर्वास्तास्तेन पुत्रेण प्राह पुत्रवतीर्मनुः}


\twolineshloka
{आत्मा पुत्रश्च विज्ञेयस्तस्यानन्तरजश्च यः}
{निरुक्तजश्च विज्ञेयः सुतः प्रसृतजस्तथा}


\twolineshloka
{पतितस्य तु भार्याया भर्त्रा सुसमवेतया}
{तथा दत्तकृतौ पुत्रावध्यूढश्च तथाऽपरः}


\threelineshloka
{षडपध्वंसजाश्चापि कानीनापसदास्तथा}
{इत्येते वै समाख्यातास्तान्विजानीहि भारत ॥युधिष्ठिर उवाच}
{}


\threelineshloka
{षडपध्वंसजाः के स्युः के वाऽप्यपसदास्तथा}
{एतत्सर्वं यथातत्त्वं व्याख्यातुं मे त्वमर्हसि ॥भीष्म उवाच}
{}


\twolineshloka
{त्रिषु वर्णेषु ये पुत्रा ब्राह्मणस्य युधिष्ठिर}
{वर्णयोश्च द्वयोः स्यातां यौ राजन्यस्य भारत}


\twolineshloka
{एको द्विवर्ण एवाथ तथाऽत्रैवोपलक्षितः}
{षडपध्वंसजास्ते हि तथैवापसदाञ्शृणु}


\twolineshloka
{चाण्डालो व्रात्यवर्णौ तु ब्राह्मण्यां क्षत्रियासु च}
{वैश्यायां चैव शूद्रस्य लक्ष्यास्तेऽपसदास्त्रयः}


\twolineshloka
{मागधो वामकश्चैव द्वौ वैश्यस्योपलक्षितौ}
{ब्राह्मण्यां क्षत्रियायां च क्षत्रियस्यैक एव तु}


\threelineshloka
{ब्राह्मण्यां लक्ष्यते सूत इत्येतेऽपसदाः स्मृताः}
{पुत्रा ह्येते न शक्यन्ते मिथ्या कर्तुं नराधिप ॥युधिष्ठिर उवाच}
{}


\threelineshloka
{क्षेत्रजं केचिदेवाहुः सुतं केचित्तु शुक्रजम्}
{तुल्यावेतौ सुतौ कस्य तन्मे ब्रूहि पितामह ॥भीष्म उवाच}
{}


\threelineshloka
{रेतजो वा भवेत्पुत्रः पुत्रो वा क्षत्रेजो भवेत्}
{अध्यूढः समयं भित्त्वेत्येतदेव निबोध मे ॥युधिष्ठिर उवाच}
{}


\threelineshloka
{रेतजं विद्म वै पुत्रं क्षत्रेजस्यागमः कथम्}
{अध्यूढं विद्म वै पुत्रं भित्त्वा तु समयं कथम् ॥भीष्म उवाच}
{}


\twolineshloka
{आत्मजं पुत्रमुत्पाद्य यस्त्यजेत्कारणान्तरे}
{न तत्र कारणं रेतः स क्षेत्रस्वामिनो भवेत्}


\twolineshloka
{पुत्रकामो हि पुत्रार्थे यां वृणीते विशाम्पते}
{तत्र क्षेत्रं प्रमाणं स्यान्न वै तत्रात्मजः सुतः}


\twolineshloka
{अन्यत्र क्षेत्रजः पुत्रो लक्ष्यते भरतर्षभ}
{न ह्यात्मा शक्यते हन्तुं दृष्टान्तोपगतो ह्यसौ}


\threelineshloka
{क्वचिच्च कुतकः पुत्रः सङ्ग्रहादेव लक्ष्यते}
{न तत्र रेतः क्षेत्रं वा प्रमाणं स्याद्युधिष्ठिर ॥युधिष्ठिर उवाच}
{}


\threelineshloka
{कीदृशः कृतकः पुत्रः सङ्ग्रहादेव लक्ष्यते}
{शुक्रं क्षेत्रं प्रमाणं वा यत्र लक्ष्यं न भारत ॥भीष्म उवाच}
{}


\twolineshloka
{मातापितृभ्यां यस्त्यक्तः पथि यस्तं प्रकल्पयेत्}
{न चास्य मातापितरौ ज्ञायेतां स हि कृत्रिमः}


\threelineshloka
{अस्वामिकस्य स्वामित्वं यस्मिन्सम्प्रतिलक्ष्यते}
{यो वर्णः पोषयेत्तं च तद्वर्णस्तस्य जायते ॥युधिष्ठिर उवाच}
{}


\threelineshloka
{कथमस्य प्रयोक्तव्यः संस्कारः कस्य वा कथम्}
{देया कन्या कथं चेति तन्मे ब्रूहि पितामह ॥भीष्म उवाच}
{}


\twolineshloka
{आत्मवत्तस्य कुर्वीत संस्कारं स्वामिवत्तथा}
{त्यक्तो मातापितृभ्यां यः सवर्णं प्रतिपद्यते}


\twolineshloka
{तद्गोत्रबन्धुजं तस्य कुर्यात्संस्कारमच्युत}
{अथ देया तु कन्या स्यात्तद्वर्णस्य युधिष्ठिर}


\twolineshloka
{संस्कर्तुं वर्णगोत्रं च मातृवर्णविनिश्चये}
{कानीनाध्यूढजौ वाऽपि विज्ञेयौ पुत्रकिल्बिषौ}


\twolineshloka
{कानीनाध्यूढजौ वाऽपि विज्ञेयौ पुत्रकिल्बिषौ ॥तावपि स्वाविव सुतौ संस्कार्याविति निश्चयः}
{}


\twolineshloka
{क्षेत्रजो वाऽप्यपसदो येऽध्यूढास्तेषु चाप्युत ॥आत्मवद्वै प्रयुञ्जीरन्संस्कारान्ब्राह्मणादयः}
{}


\twolineshloka
{`स्वं जन्मे मातृगोत्रेण संस्कारं ब्राह्मणादयः ॥'धर्मशास्त्रेषु वर्णानां निश्चयोऽयं पदृश्यते}
{}


% Check verse!
एतत्ते सर्वमाख्यातं किं भूयः श्रोतुमिच्छसि
\chapter{अध्यायः ८५}
\threelineshloka
{दर्शने कीदृशः स्नेहः संवासे च पितामह}
{महाभाग्यं गवां चैव तन्मे व्याख्यातुमर्हसि ॥भीष्म उवाच}
{}


\twolineshloka
{हन्त ते कथयिष्यामि पुरावृत्तं महाद्युते}
{नहुषस्य च संवादं महर्षेश्च्यवनस्य च}


\twolineshloka
{पुरा महर्षिश्च्यवनो भार्गवो भरतर्षभ}
{उदवासकृतारम्भो बभूव स महाव्रतः}


\twolineshloka
{निहत्य मानं क्रोधं च प्रहर्षं शोकमेव च}
{वर्षाणि द्वादश मुनिर्जलवासे धृतव्रतः}


\twolineshloka
{आदधत्सर्वभूतेषु विस्रम्यं परमं शुभम्}
{जलेचरेषु सर्वेषु शीतरश्मिरिव प्रभुः}


\twolineshloka
{स्थाणुभूतः शुचिर्भूत्वा दैवतेभ्यः प्रणम्य च}
{गङ्गायमुनयोर्मध्ये जलं सम्प्रविवेश ह}


\twolineshloka
{गङ्गायमुनयोर्वेगं सुभीमं भीमनिःस्वनम्}
{प्रतिजग्राह शिरसा वातवेगसमं जवे}


\twolineshloka
{गङ्गा च यमुना चैव सरितश्च सरांसि च}
{प्रदक्षिणमृषिं चक्रुर्न चैनं पर्यपीडयन्}


\twolineshloka
{अन्तर्जलेषु सुष्वाप काष्ठभूतो महामुनिः}
{ततश्चोर्ध्वस्थितो धीमानभवद्भरतर्षभ}


\threelineshloka
{जलौकसां ससत्वानां बभूव प्रियदर्शनः}
{उपाजिघ्नन्त च तदा मत्स्यास्तं हृष्टमानसाः}
{तत्र तस्यासतः कालः समतीतोऽभवन्महान्}


\twolineshloka
{ततः कदाचित्समये कस्मिंश्चिन्मत्स्यजीविनः}
{तं देशं समुपाजग्मुर्जालहस्ता महाद्युते}


\threelineshloka
{निषादा बहवस्तत्र मत्स्योद्धरणनिश्चयाः}
{व्यायता बलिनः शूराः सलिलेष्वनुवर्तिनः}
{अभ्याययुश्च तं देशं निश्चिता जालकर्म्णि}


\twolineshloka
{जालं ते योजयामासुर्नवसूत्रकृतं दृढम्}
{मत्स्योद्धरणमाकर्षस्तदा भरतसत्तम}


\twolineshloka
{ततस्ते बहुभिर्योगैः कैवर्ता मत्स्यकाङ्क्षिणः}
{गङ्गायमुनयोर्वारि जालेनावकिरन्ति ते}


\twolineshloka
{जालं सुविततं तेषां नवसूत्रकृतं तथा}
{विस्तारायामसम्पन्नं यत्तत्र सलिले क्षमम्}


\twolineshloka
{ततस्ते सुमहच्चैव बलवच्च सुवर्तितम्}
{अवतीर्य ततः सर्वे जालं चकृषिरे तदा}


\twolineshloka
{अभीतरूपाः संहृष्टा अन्योन्यवशवर्तिनः}
{बबन्धुस्तत्र मत्स्यांश्च तथाऽन्याञ्जलचारिणः}


\twolineshloka
{तथा मत्स्यैः परिवृतं च्यवनं भृगुनन्दनम्}
{आकर्षयन्महाराज जालेनाथ यदृच्छया}


\threelineshloka
{नदीशैवलदिग्धाङ्गं हरिश्मश्रुजटाधरम्}
{लग्नैः शङ्खनखैर्गात्रे क्रोडैश्चित्रैरिवार्पितम्}
{}


\twolineshloka
{तं जालेनोद्धृतं दृष्ट्वा ते तदा वेदपारगम्}
{सर्वे प्राञ्जलयो दाशाः शिरोभिः प्रापतन्भुवि}


\twolineshloka
{परिखेदपरित्रासाज्जालस्याकर्षणेन च}
{मत्स्या बभूवुर्व्यापन्नाः स्थलसंस्पर्शनेन च}


\threelineshloka
{स मुनिस्तत्तदा दृष्ट्वा मत्स्यानां कदनं कृतम्}
{बभूव कृपयाविष्टो निःश्वसंश्च पुनःपनः ॥निषादा ऊचुः}
{}


\twolineshloka
{अज्ञानाद्यत्कृतं पापं प्रसादं तत्र नः कुरु}
{करवाम प्रिय किं ते तन्नो ब्रूहि महामुने}


\twolineshloka
{इत्युक्तो मत्स्यमध्यस्थश्च्यवनो वाक्यमब्रवीत्}
{यो मेऽद्य परमः कामस्तं शृणुध्वं मसाहिताः}


\twolineshloka
{प्राणोत्सर्गं विसर्गं वा मत्स्यैर्यास्याम्यहं सह}
{संवासान्नोत्सहे त्यक्तुं सलिलेऽध्युपितानहम्}


\twolineshloka
{इत्युक्तास्ते निषादास्तु सुभृशं भयकम्पिताः}
{सर्वे विवर्णवदना नहुषाय न्यवेदयन्}


\chapter{अध्यायः ८६}
\twolineshloka
{नहुषस्तु ततः श्रुत्वा च्यवनं तं तथाऽऽगतम्}
{त्वरितः प्रययौ तत्रि सहामात्यपुरोहितः}


\twolineshloka
{शौचं कृत्वा यथान्यायं प्राञ्जलिः प्रयतो नृपः}
{आत्मानमाचचक्षे च च्यवनाय महात्मने}


\threelineshloka
{अर्चयामास तं चापि तस्य राज्ञः पुरोहितः}
{सत्यव्रतं महात्मानं देवकल्पं विशाम्पते ॥नहुष उवाच}
{}


\threelineshloka
{करवाणि प्रियं किं ते तन्मे ब्रूहि द्विजोत्तम}
{सर्वं कर्ताऽस्मि भगवन्यद्यपि स्यात्सुदुष्करम् ॥च्यवन उवाच}
{}


\threelineshloka
{श्रमेणि महता युक्ताः कैवर्ता मत्स्यजीविनः}
{मम मूल्यं प्रयच्छैभ्यो मत्स्यानां विक्रयैः सह ॥नहुष उवाच}
{}


\threelineshloka
{सहस्रं दीयतां मूल्यं निषादेभ्यः पुरोहित}
{निष्क्रयार्थे भगवतो यथाऽऽह भृगुनन्दनः ॥च्यवन उवाच}
{}


\twolineshloka
{आत्ममूल्यं च वक्तव्यं न तल्लोकः प्रशंसति}
{तस्मादहं प्रवक्ष्यामि न चात्मस्तुतिम्मवृतः}


\threelineshloka
{सहस्रं नाहमर्हामि किं वा त्वं मन्यसे नृप}
{सदृशं दीयतां मूल्यं स्वबुद्ध्या निश्चयं कुरु ॥नहुष उवाच}
{}


\threelineshloka
{सहस्राणां शतं विप्र निषोदेभ्यः प्रदीयताम्}
{स्यादिदं भगवन्मूल्यं किं वाऽन्यन्मन्यते भवान् ॥च्यवन उवाच}
{}


\threelineshloka
{नाहं शतसहस्रेण निमेयः पार्थिवर्षभ}
{दीयतां सदृशं मूल्यममात्यैः सह चिन्तय ॥नहुष उवाच}
{}


\threelineshloka
{कोटिः प्रदीयतां मूल्यं निषादेभ्यः पुरोहित}
{यदेतदपि नो मूल्यमतो भूयः प्रदीयताम् ॥च्यवन उवाच}
{}


\threelineshloka
{राजन्नार्हाम्यहं कोटिं भूयो वाऽपि महाद्युते}
{सदृशं दीयतां मूल्यं ब्राह्मणैः सह चिन्तय ॥नहुष उवाच}
{}


\threelineshloka
{अर्दं राज्यं समग्रं वा निषादेभ्यः प्रदीयताम्}
{एतत्तुल्यमहं मन्ये किं वाऽन्यन्मन्यसे द्विज ॥च्यवन उवाच}
{}


\threelineshloka
{अर्धं राज्यं समग्रं च मूल्यं नार्हामि पार्थिव}
{सदृशं दीयतां मूल्यमृषिभिः सह चिन्त्यताम् ॥भीष्म उवाच}
{}


\twolineshloka
{महर्षेर्वचनं श्रुत्वा नहुषो दुःखकर्शितः}
{स चिन्तयामास तदा सहामात्यपुरोहितः}


\twolineshloka
{तत्र त्वन्यो वनचरः कश्चिन्मूलफलाशनः}
{नहुषस्य समीपस्थो गविजातोऽभवन्मुनिः}


\twolineshloka
{स तमाभाष्य राजानमब्रवीद्द्विजसत्तमः}
{तोषयिष्याम्यहं क्षिप्तं यथा तुष्टो भविष्यति}


\threelineshloka
{नाहं मिथ्यावचो ब्रूयां खैरेष्वपि कुतोऽन्यथा}
{भवतो यदहं ब्रयां तत्कार्यमविशङ्कया ॥नहुष उवाच}
{}


\twolineshloka
{व्रवीतु भगवान्मूल्यं महर्षेः सदृशं भृगोः}
{परित्रायस्व मामस्मद्विषयं च कुलं च मे}


\threelineshloka
{हन्याद्धि भगवान्क्रुद्धस्त्रैलोक्यमपि केवलम्}
{किं पुनर्मां तपोहीनं बाहुवीर्यपरायणम्}
{}


\threelineshloka
{अगाधाम्भसि मग्नस्य सामात्यस्य सऋत्विजः}
{प्लवो भव महर्षे त्वं कुरु मूल्यविनिश्चयम् ॥भीष्म उवाच}
{}


\twolineshloka
{नहुषस्य वचः श्रुत्वा गविजातः प्रतापवान्}
{उवाच हर्षयन्सर्वानमात्यान्पार्थिवं च तम्}


\twolineshloka
{`ब्राह्मणानां गवां चैव कुलमेकं द्विधा कृतम्}
{एकत्र मन्त्रास्तिष्ठन्ति हविरन्यत्र तिष्ठति ॥'}


\twolineshloka
{अनर्घेया महाराज द्विजा वर्णेषु चोत्तमाः}
{गावश्च पुरुषव्याग्र गौर्मूल्यं परिकल्प्यताम्}


\twolineshloka
{नहुषस्तु ततः श्रुत्वा महर्षेर्वचनं नृप}
{हर्षेण महता युक्तः सहामात्यपुरोहितः}


\twolineshloka
{अभिगम्य भृगोः पुत्रं च्यवनं संशितव्रतम्}
{इदं प्रोवाच नृपते वाचा सन्तर्पयन्निव}


\threelineshloka
{उत्तिष्ठोत्तिष्ठ विप्रर्षे गवा क्रीतोसि भार्गव}
{एतन्मूल्यमहं मन्ये तव धर्मभृतांवर ॥च्यवन उवाच}
{}


\twolineshloka
{उत्तिष्ठाम्येष राजेन्द्र सम्यक् क्रीतोस्मि तेऽनघ}
{गोभिस्तुल्यं न पश्यामि धनं किञ्चिदिहाच्युत}


\twolineshloka
{कीर्तनं श्रवणं दानं दर्शनं चापि पार्थिव}
{गवां प्रशस्यते वीर सर्वपापहरं शिवम्}


\twolineshloka
{गावो लक्ष्म्याः सदा मूलं गोषु पाप्मा न विद्यते}
{अन्नमेव सदा गावो देवानां परमं हविः}


\twolineshloka
{स्वाहाकारवषट्कारौ गोषु नित्यं प्रतिष्ठितौ}
{गावो यज्ञस्य नेत्र्यो वै तथा यज्ञस्य ता मुखम्}


\twolineshloka
{अमृतं ह्यव्ययं दिव्यं क्षरन्ति च वहन्ति च}
{अमृतायतनं चैताः सर्वलोकनमस्कृताः}


\twolineshloka
{तेजसा वपुषा चैव गावो वह्निसमा भुवि}
{गावो हि सुमहत्तेजः प्राणिनां च सुखप्रदाः}


\twolineshloka
{निविष्टं गोकुलं यत्र श्वासं मुञ्चति निर्भयम्}
{विराजयति तं देशं पापं चास्यापकर्षति}


\twolineshloka
{गावः स्वर्गस्य सोपानं गावः स्वर्गेऽपि पूजिताः}
{गावः कामदुहोदेव्यो नान्यत्किञ्चित्परं स्मृतम्}


\threelineshloka
{इत्येतद्गोषु मे प्रोक्तं महात्म्यं भरतर्षभ}
{गुणैकदेशवचनं शक्यं पारायणं न तु ॥निषादा ऊचुः}
{}


\twolineshloka
{दर्शनं कथनं चैव सहास्माभिः कृतं मुने}
{सतां साप्तपदं मैत्रं प्रसादं न कुरु प्रभो}


\twolineshloka
{हवींषि सर्वाणि यथा ह्युपभुङ्क्ते हुताशनः}
{एवं त्वमपि धर्मात्मन्पुरुषाग्निः प्रतापवान्}


\twolineshloka
{प्रसादयामहे विद्वन्भवन्तं प्रणता वयम्}
{अनुग्रहार्थमस्माकमियं गौः प्रतिगृह्यताम्}


\threelineshloka
{`अत्यन्तापदि शक्तानां परित्राणं हि कुर्वताम्}
{या गतिर्विदिता त्वद्य नरके शरणं भवान् ॥च्यवन उवाच}
{}


\twolineshloka
{कृपणस्य च यच्चक्षुर्मुनेराशीविषस्य च}
{नरं समूलं दहति कक्षमग्निरिव ज्वलन्}


\threelineshloka
{प्रतिगृह्णामि वो धेनुं कैवर्ता मुक्तकिल्बिषाः}
{दिवं गच्छत वै क्षिप्रं मत्स्यैर्जालोद्धृतैः सह ॥भीष्म उवाच}
{}


\twolineshloka
{ततस्तस्य प्रभावात्ते महर्षेर्भावितात्मनः}
{निषादास्तेन वाक्येन सह मत्स्यैर्दिवं ययुः}


\twolineshloka
{ततः स राजा नहुषो विस्मितः प्रेक्ष्य धीवरान्}
{आरोहमाणांस्त्रिदिवं मत्स्यांश्च भरतर्षभ}


\twolineshloka
{ततस्तौ गविजश्चैव च्यवनश्च भृगूद्वहः}
{वराभ्यामनुरूपाभ्यां छन्दयामासतुर्नृपम्}


\twolineshloka
{ततो राजा महावीर्यो नहुषः पृथिवीपतिः}
{परमित्यब्रवीत्प्रीतस्तदा भरतसत्तम}


\twolineshloka
{ततो जग्राह धर्मे स स्थितिमिन्द्रनिभो नृपः}
{तथेति चोदितः प्रीतस्तावृषी प्रत्यपूजयत्}


\twolineshloka
{समाप्तदीक्षश्च्यवनस्ततोऽगच्छत्स्वमाश्रमम्}
{गविजश्च महातेजाः स्वमाश्रमपदं ययौ}


\twolineshloka
{निषादाश्च दिवं जग्मुस्ते च मत्स्या जनाधिप}
{नहुषोपि वरं लब्ध्वा प्रविवेश स्वकं पुरम्}


\twolineshloka
{एतत्ते कथितं तात यन्मां त्वं परिपृच्छसि}
{दर्शने यादृशः स्नेहः संवासे वा युधिष्ठिर}


\twolineshloka
{महाभाग्यं गवां चैव तता धर्मविनिश्चयम्}
{किं भूयः कथ्यतां वीर किं ते हृदि विवक्षितम्}


\chapter{अध्यायः ८७}
\twolineshloka
{संशयो मे महाप्राज्ञ सुमहान्सागरोपम}
{तं मे शृणु महाबाहो श्रुत्वा व्याख्यातुमर्हसि}


\twolineshloka
{कौतूहलं मे सुमहज्जामदग्न्यं प्रति प्रभो}
{रामं धर्मभृतां श्रेष्ठं तन्मे व्याख्यातुमर्हसि}


\threelineshloka
{`ब्राह्मे बले सुपूर्णानामेतेषां च्यवनादिनाम्}
{'कथमेष समुत्पन्नो रामः सत्यपराक्रमः}
{कथं ब्रह्मर्षिवंशोऽयं क्षत्रधर्मा व्यजायत}


\twolineshloka
{तदस्य सम्भवं राजन्निखिलेनानुकीर्तय}
{कौशिकश्च कथं वंशात्क्षात्राद्वै ब्राह्मणोऽभवत्}


\twolineshloka
{अहो प्रभावः सुमहानासीद्वै सुमहात्मनोः}
{रामस्य च नरव्याघ्र विश्वामित्रस्य चैव हि}


\threelineshloka
{कथं पुत्रानतिक्रम्य तेषां नप्तृष्वथाभवत्}
{एष दोषो महाप्राज्ञ तत्त्वं व्याख्यातुमर्हसि ॥भीष्म उवाच}
{}


\twolineshloka
{अत्राप्युदाहरन्तीममितिहासं पुरातनम्}
{च्यवनस्य च संवादं कुशिकस्य च भारत}


\twolineshloka
{एतं दोषं पुरा दृष्ट्वा भार्गवश्च्यवनस्तदा}
{आगामिनं महाबुद्धिः स्ववंशे मुनिसत्तमः}


\twolineshloka
{निश्चित्य मनसा सर्वं गुणदोषबलाबलम्}
{दग्धुकामः कुलं सर्वं कुशिकानां तपोधनः}


\threelineshloka
{च्यवनस्तमनुप्राप्य कुशिकं वाक्यमब्रवीत्}
{वस्तुमिच्छा समुत्पन्ना त्वया सह ममानघ ॥कुशिक उवाच}
{}


\twolineshloka
{भगवन्सहधर्मोऽयं पण्डितैरिह चर्यते}
{प्रदानकाले कन्यानामुच्यते च सदा बुधैः}


\threelineshloka
{यत्तु तावदतिक्रान्तं धर्मद्वारं तपोधन}
{तत्कार्यं प्रकरिष्यामि तदनुज्ञातुमर्हसि ॥भीष्म उवाच}
{}


\twolineshloka
{अथासनमुपादाय च्यवनस्य महामुनेः}
{कुशिको भार्यया सार्धमाजगाम यतो मुनिः}


\twolineshloka
{प्रगृह्य राजा भृङ्गारं पाद्यमस्मै न्यवेदयत्}
{कारयामास सर्वाश्च क्रियास्तस्य महात्मनः}


\twolineshloka
{ततः स राजा च्यवनं मधुपर्कं यथाविधि}
{ग्राहयामास चाव्यग्रो महात्मा नियतव्रतः}


\twolineshloka
{सत्कृत्य तं तथा विप्रमिदं पुनरथाब्रवीत्}
{भगवन्परवन्तौ स्वो ब्रूहि किं करवावहे}


\twolineshloka
{यदि राज्यं यदि धनं यदि गाः संशितव्रत}
{यज्ञदानानि च तथा ब्रूहि सर्वं ददामि ते}


\twolineshloka
{इदं गृहमिदं राज्यमिदं धर्मासनं च ते}
{राजा त्वमसि शाध्युर्वीं भृत्योऽहं परवान्स्त्रिया}


\twolineshloka
{एवमुक्ते ततो वाक्ये च्यवनो भार्गवस्तदा}
{कुशिकं प्रत्युवाचेदं मुदा परमया युतः}


\twolineshloka
{न राज्यं कामये राजन्न धनं न च योषितः}
{न च गा न च वै देशान्न यज्ञं श्रूयतामिदम्}


\twolineshloka
{नियमं किञ्चिदारप्स्ये युवयोर्यदि रोचते}
{परिचर्योस्मि यत्ताभ्यां युवाभ्यामविशङ्कया}


\twolineshloka
{एवमुक्ते तदा तेन दम्पती तौ जहर्षतुः}
{प्रत्यब्रूतां च तमृषिमेवमस्त्विति भारत}


\twolineshloka
{अथ तं कुशिको हृष्टः प्रावेशयदनुत्तमम्}
{गृहोद्देशं ततस्तस्य दर्शनीयमदर्शयत्}


\twolineshloka
{इयं शय्या भगवतो यथाकाममिहोष्यताम्}
{प्रयतिष्यावहे प्रीतिमाहर्तुं ते तपोधन}


\twolineshloka
{अथ सूर्योतिचक्राम तेषां संवदतां तथा}
{अथर्षिश्चोदयामास पानमन्नं तथैव च}


\twolineshloka
{तमपृच्छत्ततो राजा कुशिकः प्रणतस्तदा}
{किमन्नजातमिष्टं ते किमुपस्थापयाम्यहम्}


\twolineshloka
{ततः स परया प्रीत्या प्रत्युवाच नराधिपम्}
{औपपत्तिकमाहारं प्रयच्छस्वेति भारत}


\twolineshloka
{तद्वचः पूजयित्वा तु तथेत्याह स पार्थिवः}
{यथोपपन्नमाहारं तस्मै प्रादाज्जनाधिप}


\twolineshloka
{ततः स भुक्त्वा भगवन्दम्पती प्राह धर्मवित्}
{स्वप्तुमिच्छाम्यहं निद्रा बाधते मामिति प्रभो}


\twolineshloka
{ततः शय्यागृहं प्राप्य भगवानृषिसत्तमः}
{संविवेश नरेशस्तु सपत्नीकः स्थितोऽभवत्}


\twolineshloka
{न प्रबोध्योस्मि संसुप्त इत्युवाचाथ भार्गवः}
{संवाहितव्यौ मे पादौ जागर्तव्यं च वां निशि}


\twolineshloka
{अविशङ्कस्तु कुशिकस्तथेत्येवाह धर्मवित्}
{न प्राबोधयतां तौ च दंपती रजनीक्षये}


\twolineshloka
{यथादेशं महर्षेस्तु शुश्रूषापरमौ तदा}
{बभूवतुर्महाराज प्रयतावथ दम्पती}


\twolineshloka
{ततः स भगवान्विप्रः समादिश्य नराधिपम्}
{सुष्वापैकेन पार्श्वेन दिवसानेकविंशतिम्}


\twolineshloka
{स तु राजा निराहारः सभार्यः कुरुनन्दन}
{पर्युपासत तं हृष्टश्च्यवनाराधने रतः}


\twolineshloka
{भार्गवस्तु समुत्तस्थौ स्वयमेव तपोधनः}
{अकिञ्चिदुक्त्वा तु गृहान्निश्चक्राम महातपाः}


\twolineshloka
{तमन्वगच्छतां तौ च क्षुधितौ श्रमकर्शितौ}
{भार्यापती मुनिश्रेष्ठस्तावेतौ नावलोकयत्}


\twolineshloka
{तयोस्तु प्रेक्षतोऽरेव भार्गवाणां कुलोद्वहः}
{अन्तर्हितोभूद्राजेन्द्र ततो राजाऽपतत्क्षितौ}


\twolineshloka
{ततो मुहूर्तादाश्वस्य सह देव्या महामुनेः}
{पुनरन्वेषणे यत्नमकरोत्स महीपतिः}


\chapter{अध्यायः ८८}
\threelineshloka
{तस्मिन्नन्तर्हिते विप्रे राजा किमकरोत्तदा}
{भार्या चास्य महाभागा तन्मे ब्रूहि पितामह ॥भीष्म उवाच}
{}


\twolineshloka
{अदृष्ट्वा स महीपालस्तमृषिं सह भार्यया}
{परिश्रान्तो निववृते व्रीडितो नष्टचेतनः}


\twolineshloka
{स प्रविश्य पुरीं दीनो नाभ्यभाषत किञ्चन}
{तदेव चिन्तयामास च्यवनस्य विचेष्टितम्}


\twolineshloka
{अथ शून्येन मनसा प्रविवेश गृहं नृपः}
{ददर्श शयने तस्मिञ्शयानं भृगुनन्दनम्}


\twolineshloka
{विस्मितौ तमृषिं दृष्ट्वा तदाश्चर्यं विचिन्त्य च}
{दर्शनात्तस्य तु तदा विश्रान्तौ सम्बभूवतुः}


\twolineshloka
{यथास्थानं ततो गत्वा तत्पादौ संववाहतुः}
{अथापरेण पार्श्वेन सुष्वाप स महामुनिः}


\twolineshloka
{तेनैव च स कालेन प्रत्यबुद्ध्यत वीर्यवान्}
{न च तौ चक्रतुः किञ्चिद्विकारं भयशङ्कितौ}


\twolineshloka
{प्रतिबुद्धस्तु स मुनिस्तौ प्रोवाच विशाम्पते}
{तैलाभ्यङ्गो दीयतां मे स्नास्येऽहमिति भारत}


\twolineshloka
{तौ तथेति प्रतिश्रुत्य क्षुधितौ श्रमकर्शितौ}
{शतपाकेन तैलेन महार्हेणोपतस्थतुः}


\twolineshloka
{ततः सुखासीनमृषिं वाग्यतौ संववाहतुः}
{न च पर्याप्तमित्याह भार्गवः सुमहातपाः}


\twolineshloka
{यदा तौ निर्विकारौ तु लक्षमायास भार्गवः}
{तत उत्थाय सहसा स्नानशालां विवेश ह}


\twolineshloka
{क्लृप्तमेव तु तत्रासीत्स्नानीयं पार्थिवोचितम्}
{असत्कृत्य च तत्सर्वं तत्रैवान्तरधीयत}


\twolineshloka
{स मुनिः पुनरेवाथ नृपतेः पश्यतस्तदा}
{नासूयां चक्रतुस्तौ च दम्पती भरतर्षभ}


\twolineshloka
{अथ स्नातः स भगवान्सिंहासनगतः प्रभुः}
{दर्शयामास कुशिकं सभार्यं कुरुनन्दन}


\twolineshloka
{संहृष्टवदनो राजा सभार्यः कुशिको मुनिम्}
{सिद्धमन्नमिति प्रह्वो निर्विकारो न्यवेदयत्}


\twolineshloka
{आनीयतामिति मुनिस्तं चोवाच नराधिपम्}
{स राजा समुपाजह्रे तदन्नं सह भार्यया}


\twolineshloka
{मांसप्रकारान्विविधाञ्शाकानि विविधानि च}
{लेह्यपिष्टविकारांश्च पानकानि लघूनि च}


\twolineshloka
{रसालापूपकांश्चित्रान्मोदकानथ षड्रसान्}
{रसान्नानाप्रकारांश्च वन्यं च मुनिभोजनम्}


\twolineshloka
{फलानि च विचित्राणि राजभोज्यानि भूरिशः}
{बदरेङ्गुदकाश्मर्यभल्लातकफलानि च}


\twolineshloka
{गृहस्थानां च यद्भोज्यं यच्चापि वनवासिनाम्}
{सर्वमाहारयामास राजा शापभयान्मुनेः}


\twolineshloka
{अथ सर्वमुपन्यस्तमग्रतश्च्यवनस्य तत्}
{ततः सर्वं समानीय तच्च शय्यासनं मुनिः}


\twolineshloka
{वस्त्रैः शुभैरवच्छाद्य भोजनोपस्करैः सह}
{सर्वमादीपयामास च्यवनो भृगुनन्दनः}


\twolineshloka
{न च तौ चक्रतुः क्रोधं दम्पती सुमहाव्रतौ}
{तयोः सम्प्रेक्षतोरेव पुनरन्तर्हितोऽभवत्}


\twolineshloka
{तथैव च स राजर्षिस्तस्थौ तां रजनीं तदा}
{सभार्यो वाग्यतः श्रीमान्न चकोपं समाविशत्}


% Check verse!
नित्यसंस्कृतमन्नं तु विविधं राजवेश्मनि ॥शयनानि च मुख्यानि परिषेकाश्च पुष्कलाः
\twolineshloka
{वस्त्रं च विविधाकारमभवत्समुपार्जितम्}
{न शशाक ततो द्रष्टमन्तरं च्यवनस्तदा}


\twolineshloka
{पुनरेव च विप्रर्षिः प्रोवाच कुशिकं नृपम्}
{सभार्यो मां रथेनाशु वह यत्र ब्रवीम्यहम्}


\twolineshloka
{तथेति च प्राह नृपो निर्विशङ्कस्तपोधनम्}
{क्रीडारथोस्तु भगवन्नुत साङ्ग्रामिको रथः}


\twolineshloka
{इत्युक्तः स मुनी राज्ञा तेन हृष्टेन तद्वचः}
{च्यवनः प्रत्युवाचेदं हृष्टः परपुरंजयम्}


\twolineshloka
{सज्जीकुरु रथं क्षिप्रं यस्ते साङ्ग्रामिको मतः}
{सायुधः सपताकश्च शक्तीकनकयष्टिमान्}


\twolineshloka
{किंकिणीस्वननिर्घोषो युक्तस्तोरणकल्पनैः}
{गदास्वङ्गनिबद्धश्च परमेषुशतान्वितः}


\twolineshloka
{ततः स तं तथेत्युक्त्वा कल्पयित्वा महारथम्}
{भार्यां वामे धुरि तदा चात्मानं दक्षिणे तथा}


\twolineshloka
{त्रिदण्डं वज्रसूच्यग्रं प्रतोदं तत्र चादधत्}
{सर्वमेतत्तथा दत्त्वा नृपो वाक्यमऽथाब्रवीत्}


\twolineshloka
{भगवन्क्व रथो यातु ब्रवीतु भृगुनन्दन}
{यत्र वक्ष्यसि विप्रर्षे तत्र यास्यति ते रथः}


\twolineshloka
{एवमुक्तस्तु भगवान्प्रत्युवाचाथ तं नृपम्}
{इतः प्रभृति यातव्यं पदकम्पदकं शनैः}


\twolineshloka
{श्रमो मम यथा न स्यात्तथा मच्छन्दचारिणौ}
{सुसुखं चैव वोढव्यो जनः सर्वश्च पश्यतु}


\twolineshloka
{नोत्सार्याः पथिकाः केचित्तेभ्योदास्ये वसु ह्यहम्}
{ब्राह्मणेभ्यश्च ये कामानर्थयिष्यन्तिमां पथि}


\twolineshloka
{सर्वान्दास्याम्यशेषेण धनं रत्नानि चैव हि}
{क्रियतां निखिलेनैतन्मा विचारय पार्थिव}


\twolineshloka
{तस्य तद्वचनं श्रुत्वा राजा भृत्यांस्तथाऽब्रवीत्}
{यद्यद्ब्रूयान्मुनिस्तत्तत्सर्वं देयमशङ्कितैः}


\twolineshloka
{ततो रत्नान्यनेकानि स्त्रियो युग्यमजाविकम्}
{कृताकृतं च कनकं गजेन्द्राश्चाचलोपमाः}


\twolineshloka
{अन्वगच्छन्त तमृषिं राजामात्याश्च सर्वशः}
{हाहाभूतं च तत्सर्वमासीन्नगरमार्तवत्}


\twolineshloka
{तौ तीक्ष्णाग्रेण सहसा प्रतोदेन प्रतोदितौ}
{पृष्ठे विद्धौ कटे चैव निर्विकारौ तमूहतुः}


\twolineshloka
{वेपमानौ निराहारौ पञ्चाशद्रात्रकर्शितौ}
{कथंचिदूहतुर्धैर्याद्दम्पती तं रथोत्तमम्}


\twolineshloka
{बहुशो भृशविद्धौ तौ क्षरमाणौ क्षतोद्भवम्}
{ददृशाते महाराज पुष्पिताविव किंशुकौ}


\twolineshloka
{तौ दृष्ट्वा पौरवर्गस्तु भृशं शोकसमाकुलः}
{अभिशापभयत्रस्तो न तं किञ्चिदुवाच ह}


\twolineshloka
{द्वन्द्वशश्चाब्रुवन्सर्वे पश्यध्वं तपसो बलम्}
{क्रुद्धा अपि मुनिश्रेष्ठं वीक्षितुं नेह शुक्नुमः}


\twolineshloka
{अहो भगवतो वीर्यं महर्षेर्भावितात्मनः}
{राज्ञश्चापि सभार्यस्य धैर्यं पश्यत यादृशम्}


\threelineshloka
{श्रान्तावपि हि कच्छ्रेण रथमेनं समूहतुः}
{न चैतयोर्विकारं वै ददर्शं भृगुनन्दनः ॥भीष्म उवाच}
{}


\twolineshloka
{ततः स निर्विकारौ तु दृष्ट्वा भृगुकुलोद्वहः}
{वसु विश्राणयामास यथा वैश्रवणस्तथा}


\twolineshloka
{तत्रापि राजा प्रीतात्मा यथादिष्टमथाकरोत्}
{ततोऽस्य भगवान्प्रीतो बभूव मुनिसत्तमः}


\twolineshloka
{अवतीर्य रथश्रेष्ठाद्दम्पती तौ मुमोच ह}
{विमोच्य चैतौ विधिवत्ततो वाक्यमुवाच ह}


\twolineshloka
{स्निग्धगम्भीरया वाचा भार्गवः सुप्रसन्नया}
{ददामि वां वरं श्रेष्ठं तं ब्रूतामिति भारत}


\twolineshloka
{सुकुमारौ च तौ विद्धौ कराभ्यां मुनिसत्तमः}
{पस्पर्शामृतकल्पाभ्यां स्नेहाद्भरतसतम}


\twolineshloka
{अथाब्रवीन्नृपो वाक्यं श्रमो नास्त्यावयोरिह}
{विश्रान्तौ स्वः प्रभावात्ते ध्यानेनैवेह भार्गव}


\twolineshloka
{अथ तौ भगवान्प्राह प्रहृष्टश्च्यनस्तदा}
{न वृथा व्याहृतं पूर्वं यन्मया तद्भविष्यति}


\twolineshloka
{रमणीयः समुद्देशो गङ्गातीरमिदं शुभम्}
{किञ्चित्कालं व्रतपरो निवत्स्यामीह पार्थिव}


\twolineshloka
{गम्यतां स्वपुरं पुत्र विश्रान्तः पुनरेष्यसि}
{इहस्थं मां सभार्यस्त्वं द्रष्टासि श्वो नराधिप}


\twolineshloka
{न च मन्युस्त्वया कार्यः श्रेयस्त्वां समुपस्थितम्}
{यत्काङ्क्षितं हृदिस्थं ते तत्सर्वं हि भविष्यति}


\twolineshloka
{इत्येवमुक्तः कुशिकः प्रहृष्टेनान्तरात्मना}
{प्रोवाच मुनिशार्दूलमिदं वचनमर्थवत्}


\twolineshloka
{न मे मन्युर्महाभाग पूतौ स्वो भगवंस्त्वया}
{संवृतौ यौवनस्थौ स्वो वपुष्मन्तौ बलान्वितौ}


\twolineshloka
{प्रतोदेन व्रणा ये मे सभार्यस्य त्वया कृताः}
{तान्न पश्यामि गात्रेषु स्वस्थोस्मि सह भार्यया}


\twolineshloka
{इमां च देवीं पश्यामि वपुषाऽप्सरसोपमाम्}
{श्रिया परमया युक्ता तथा दृष्टा पुरा मया}


\twolineshloka
{तव प्रसादसंवृत्तमिदं सर्वं महामुने}
{नैतच्चित्रं तु भगवंस्त्वयि सत्यपराक्रम}


\twolineshloka
{इत्युक्तः प्रत्युवाचैनं कुशिकं च्यवनस्तदा}
{आगच्छेथाः सभार्यश्च त्वमिहेति नराधिप}


\twolineshloka
{इत्युक्तः समनुज्ञातो राजर्षिरभिवाद्य तम्}
{प्रययौ वपुषा युक्तो नगरं देवराजवत्}


\twolineshloka
{तत एनमुपाजग्मुरमात्याः सपुरोहिताः}
{बलस्था गणिकायुक्ताः सर्वाः प्रकृतयस्तथा}


\twolineshloka
{तैर्वृतः कुशिको राजा श्रिया परमया ज्वलन्}
{प्रविवेश पुरं हृष्टः पूज्यमानोऽथ बन्दिभिः}


\twolineshloka
{ततः प्रविश्य नगरं कृत्वा पौर्वाह्णिकीः क्रियाः}
{भुक्त्वा सभार्यो रजनीमुवास स महाद्युतिः}


\twolineshloka
{ततस्तु तौ नवमभिवीक्ष्य यौवनंपरस्परं विगतजराविवामरौ}
{ननन्दतुः शयनगतौ वपुर्धरौश्रिया युतौ द्विजवरदत्तया तदा}


\twolineshloka
{अथाप्युषिर्भृगुकुलकीर्तिवर्धन-स्तपोधनो वनमभिराममृद्धिमत्}
{मनीषया बहुविधरत्नभूषितंससर्ज यन्न पुरि शतक्रतोरपि}


\chapter{अध्यायः ८९}
\twolineshloka
{ततः स राजा रात्र्यन्ते प्रतिबुद्धो महामनाः}
{कृतपूर्वाह्णिकः प्रायात्सभार्यस्तद्वनं प्रति}


\threelineshloka
{ततो ददर्श नृपतिः प्रासादं सर्वकाञ्चनम्}
{मणिस्तम्भसहस्राढ्यं गन्धर्वनगरोपमम्}
{तत्र दिव्यानभिप्रायान्ददर्श कुशिकस्तदा}


\threelineshloka
{पर्वतान्रूप्यसानूंश्च नलिनीश्च सपङ्कजाः}
{चित्रशालाश्च विविधास्तोरणानि च भारत}
{शाद्वलोपचितां भूमिं तथा काञ्चनकुट्टिमाम्}


\twolineshloka
{सहकारान्प्रफुल्लांश्च केतकोद्दालकान्वरान्}
{अशोकान्सहकुन्दांश्च फुल्लांश्चैवातिमुक्तकान्}


\twolineshloka
{चम्पकांस्तिलकान्भव्यान्पनसान्वञ्जुलानपि}
{पुष्पितान्कर्णिकारांश्च तत्रतत्र ददर्श ह}


\twolineshloka
{श्यामान्वारणपुष्पांश्च तथाऽष्टपदिका लताः}
{तत्रतत्र परिक्लृप्ता ददर्श स महीपतिः}


\twolineshloka
{रम्यान्पद्मोत्पलधरान्सर्वर्तुकुसुमांस्तथा}
{विमानप्रतिमांश्चापि प्रासादाञ्शैलसन्निभान्}


\twolineshloka
{शीतलानि च तोयानि क्वचिदुष्णानि भारत}
{आसनानि विचित्राणि शयनप्रवराणि च}


\twolineshloka
{पर्यङ्कान्रत्नसौवर्णान्परार्ध्यास्तरणास्तृतान्}
{भक्ष्यं भोज्यमनन्तं च तत्रतत्रोपकल्पितम्}


\twolineshloka
{वाणीवादाञ्छुकांश्चैव शारिका भृङ्गराजकान्}
{कोकिलाञ्छतपत्रांश्च सकोयष्टिककुक्कुभान्}


\twolineshloka
{मयूरान्कुक्कुटांश्चापि दात्यूहाञ्जीवजीवकान्}
{चकोरान्वानरान्हंसान्सारसांश्चक्रसाह्वयान्}


\twolineshloka
{समन्ततः प्रणदतो ददर्श सुमनोहरान्}
{क्वचिदप्सरसां सङ्घान्गन्धर्वाणां च पार्थिव}


\twolineshloka
{कान्ताभिरपरांस्तत्र परिष्वक्रान्ददर्श ह}
{न ददर्श च तान्भूयो ददर्श च पुनर्नृपः}


\twolineshloka
{गीतध्वनिं सुमधुरं तथैवाध्ययनध्वनिम्}
{हंसानसुमधुरांश्चापि तत्र शुश्राव पार्थिवः}


\twolineshloka
{तं दृष्ट्वाऽत्यद्भुतं राजा मनसाऽचिन्तयत्तदा}
{स्वप्नोऽयं चित्तविभ्रंश उताहो सत्यमेव तु}


\twolineshloka
{अहो सह शरीरेण प्राप्तोस्मि परमां गतिम्}
{उत्तरान्वा कुरून्पुण्यानथवाऽप्यमरावतीम्}


\twolineshloka
{किञ्चेदं महदाश्चर्यं सम्पश्यामीत्यचिन्तयत्}
{एवं सञ्चिन्तयन्नेव ददर्श मुनिपुङ्गवम्}


\twolineshloka
{तस्मिन्विमाने सौवर्णे मणिस्तम्भसमाकुले}
{महार्हे शयने दिव्ये शयानं भृगुनन्दनम्}


\twolineshloka
{तमभ्ययात्प्रहर्षेण नरेन्द्रः सह भार्यया}
{अन्तर्हितस्ततो भूयश्च्यवनः शयनं व तत्}


\twolineshloka
{ततोऽन्यस्मिन्वनोद्देशे पुनरेव ददर्श तम्}
{कौश्यां बृस्यां समासीनं जपमानं महाव्रतम्}


% Check verse!
एवं योगबलाद्विप्रो मोहयामास पार्थिवम्
\twolineshloka
{क्षणेन तद्वनं चैव ते चैवाप्सरसां गणाः}
{गन्धर्वाः पादपाश्चैव सर्वमन्तरधीयत}


\twolineshloka
{निःशब्दमभवच्चापि गङ्गाकूलं पुनर्नृप}
{कुशवल्मीकभूयिष्ठं बभूव च यथा पुरा}


\twolineshloka
{ततः स राजा कुशिकः सभार्यस्तेन कर्मणा}
{विस्मयं परमं प्राप्तस्तद्दृष्ट्वा महदद्भुतम्}


\threelineshloka
{ततः प्रोवाच कुशिको भार्या हर्षसमन्वितः}
{पश्य भद्रे यथाभावाश्चित्रा दृष्टाः सुदुर्लभाः}
{प्रसादाद्भृगुमुख्यस्य किमन्यत्र तपोबलात्}


\twolineshloka
{तपसा तदवाप्यं हि यन्न शक्यं मनोरथैः}
{त्रैलोक्यराज्यादपि हि तप एव विशिष्यते}


\twolineshloka
{तपसा हि सुतप्तेनि क्रीडत्येष तपोधनः}
{अहो प्रभावो ब्रह्मर्षेश्च्यवनस्य महात्मनः}


\twolineshloka
{इच्छयैष तपोवीर्यादन्याँल्लोकान्सृजेदपि}
{ब्राह्मणा एव जायेरन्पुण्यवाग्बुद्धिकर्मणा}


\twolineshloka
{उत्सहेदिह कृत्वैव कोऽन्यो वै च्यवनादृते}
{ब्राह्मण्यं दुर्लभं लोके राज्यं हि सुलभं नरैः}


\twolineshloka
{ब्राह्मण्यस्य प्रभावाद्धि रथे युक्तौ स्वधुर्यवत्}
{इत्येवं चिन्तयानः स विदितश्च्यवनस्य वै}


\twolineshloka
{सम्प्रेक्ष्योवाच नृपतिं क्षिप्रमागम्यतामिति}
{इत्युक्तः सहभार्यस्तु सोभ्यगच्छन्महामुनिम्}


\threelineshloka
{शिरसा वन्दनीयं तमवन्दत च पार्थिवः}
{तस्याशिषः प्रयुज्याथ स मुनिस्तं नराधिपम्}
{निषीदेत्यब्रवीद्धीमान्सान्त्वयन्पुरुषर्षभः}


\twolineshloka
{ततः प्रकृतिमापन्नो भार्गवो नृपते नृपम्}
{उवाच श्लक्ष्णया वाचा तर्पयन्निव भारत}


\twolineshloka
{राजन्सम्यग्जितानीह पञ्चपञ्च स्वयं त्वया}
{मनःषष्ठानीन्द्रियाणि कृच्छ्रान्मुक्तोसि तेन वै}


\twolineshloka
{सम्यगाराधितः पुत्र त्वयाऽहं वदतांवर}
{न हि ते वृजितं किञ्चित्सुसूक्ष्ममपि दृश्यते}


\threelineshloka
{अनुजानीहि मां राजन्गमिष्यामि यथागतम्}
{प्रीतोस्मि तव राजेन्द्र वरश्च प्रतिगृह्यताम् ॥कुशिक उवाच}
{}


\twolineshloka
{अग्निमध्ये गतेनेव भगवन्सन्निधौ मया}
{वर्तितं भृगुशार्दूल यन्न दग्धोस्मि तद्बहु}


\twolineshloka
{एष एव वरो मुख्यः प्राप्तो मे भृगुनन्दन}
{यत्प्रीतोसि ममाचारैः कुलं त्रातं च मेऽनघ}


\twolineshloka
{एथ मेऽन्द्रग्रहो विप्र जीविते च प्रयोजनम्}
{एतद्राज्यफलं चैव तपसश्च फलं मम}


\twolineshloka
{यदि त्वं प्रीतिमान्विप्र मयि वै भृगुनन्दन}
{अस्ति मे संशयः कश्चित्तन्मे व्याख्यातुमर्हसि}


\chapter{अध्यायः ९०}
\threelineshloka
{वरश्च गृह्यतां मत्तो यश्च ते संशयो हृदि}
{तं प्रब्रूदि नरश्रेष्ठ सर्वं सम्पादयामि ते ॥कुशिक उवाच}
{}


\twolineshloka
{यदि प्रीतोसि भगवंस्ततो मे वद भार्गव}
{कारणं श्रोतुसिच्छमि मद्गृहे वासकारितम्}


\twolineshloka
{शयनं चैकपार्श्वेन दिवसानेकविंशतिम्}
{न किञ्चिदुक्त्वा गमनं बहिश्च मुनिपुङ्गवः}


\twolineshloka
{अन्तर्धानमकस्माच्च पुनरेव च दर्शनम्}
{पुनश्च शयनं विप्र दिवसानेकविंशतिम्}


\twolineshloka
{तैलाभ्यक्तस्य गमनं भोजनं च गृहे मम}
{समुपानीय विविधं यद्दग्धं जातवेदसा}


\twolineshloka
{निर्याणां च रथेनाशु सहसा यत्कृतं त्वया}
{धनानां च विसर्गश्च वनस्यापि च दर्शनम्}


\twolineshloka
{प्रासादानां बहूनां च काञ्चनानां महामुने}
{मणिविद्रुमपादानां पर्यङ्काणां च दर्शनम्}


\twolineshloka
{पुनश्चादर्शनं तस्य श्रोतुमिच्छामि कारणम्}
{अतीव ह्यत्र मुह्यामि चिन्तयानो भृगूद्वह}


\threelineshloka
{न चैवात्राधिगच्छामि सर्वस्यास्य विनिश्चयम्}
{एतदिच्छामि कार्त्स्न्येन सत्यं श्रोतुं तपोधन ॥च्यवन उवाच}
{}


\twolineshloka
{शृणु सर्वमशेषेणि यदिदं येन हेतुना}
{न हि शक्यमनाख्यातुमेवं पृष्टेन पार्थिव}


\threelineshloka
{पितामहस्य वदतः पुरा देवसमागमे}
{श्रुतवानस्मि यद्राजंस्तन्मे निगदतः शृणु}
{}


\twolineshloka
{ब्रह्मक्षत्रविरोधेन भविता कुलसङ्करः}
{पौत्रस्ते भविता राजंस्तेजोवीर्यसमन्वितः}


\twolineshloka
{ततः स्वकुलरक्षार्थमहं त्वां समुपागतः}
{चिकीर्षन्कुशिकोच्छेदं संदिधक्षुः कुलं तव}


\twolineshloka
{ततोऽहमागम्य पुरे त्वामवोचं महीपते}
{नियमं कञ्चिदारप्सये शुश्रूषा क्रियतामिति}


\twolineshloka
{न च ते दुष्कृतं किञ्चिदहमासादयं गृहे}
{तेन जीवसि राजर्षे न भवेथास्त्वमन्यथा}


\twolineshloka
{एवं बुद्धिं समास्थाय दिवसानेकविंशतिम्}
{सुप्तोस्मि यदि मां कश्चिद्बोधयेदिति पार्थिव}


\twolineshloka
{यदा त्वया सभार्येण संसुप्तो न प्रबोधितः}
{अहं तदैव ते प्रीतो मनसा राजसत्तम}


\twolineshloka
{उत्थाय चास्मि निष्क्रान्तो यदि मां त्वं महीपते}
{पृच्छेः क्व यास्यसीत्येवं शपेयं त्वामिति प्रभो}


\twolineshloka
{अन्तर्हितः पुनश्चास्मि पुनरेव च ते गृहे}
{योगमास्थाय संसुप्तो दिवसानेकविंशतिम्}


\twolineshloka
{क्षुधितौ मामसूयेथां श्रमाद्वेति नराधिप}
{एतां बुद्धिं समास्थाय कर्शितौ वां क्षुधा मया}


\twolineshloka
{न च तेऽभूत्सुसूक्ष्मोपि मन्युर्मनसि पार्थिव}
{सभार्यस्य नरश्रेष्ठ तेन ते प्रीतिमानहम्}


\twolineshloka
{भोजनं च समानाय्य यत्तदादीपितं मया}
{क्रुध्येथा यदि मात्सर्यादिति तन्मर्षितं च मे}


\twolineshloka
{ततोऽहं रथमारुह्य त्वामवोचं नराधिप}
{सभार्यो मां वहस्वेति तच्च त्वं कृतवांस्तथा}


\twolineshloka
{अविशङ्को नरपते प्रीतोऽहं चापि तेन ह}
{धनोत्सर्गेऽपि च कृते न त्वां क्रोधोप्यधर्षयत्}


\twolineshloka
{ततः प्रीतेन ते राजन्पुनरेतत्कृतं तव}
{सभार्यस्य वनं भूयस्तद्विद्धि मनुजाधिप}


\twolineshloka
{प्रीत्यर्थं तव चैतन्मे स्वर्गसंदर्शनं कृतम्}
{यत्ते वनेऽस्मिन्नृपते दृष्टं दिव्यं सुदर्शनम्}


\twolineshloka
{स्वर्गोद्देशस्त्वया राजन्सशरीरेण पार्थिव}
{मुहूर्तमनुभूतोऽसौ सभार्येण नृपोत्तम}


\twolineshloka
{निदर्शनार्थं तपसो धर्मस्य च नराधिप}
{तत्र नासीत्स्पृहा राजंस्तच्चापि विदितं मया}


\twolineshloka
{ब्राह्मण्यं काङ्क्षसे हि त्वं तपश्च पृथिवीपते}
{अवमत्य नरेन्द्रत्वं देवेन्द्रत्वं च पार्थिव}


\twolineshloka
{एवमेतद्यथातत्वं ब्राह्मण्यं तात दुर्लभम्}
{ब्राह्मण्ये सति चर्षित्वमृषित्वे च तपस्विता}


% Check verse!
भविष्यत्येष ते कामः कुशिकात्कौशिको द्विजः
\twolineshloka
{तृतीयं पुरुषं प्राप्य ब्राह्मणत्वं गमिष्यति}
{वंशस्ते पार्थिवश्रेष्ठ भृगूणामेव तेजसा}


\twolineshloka
{पौत्रस्ते भविता विप्रस्तपस्वी पावकद्युतिः}
{`जमदग्नौ महाभाग तपसा भावितात्मनि ॥'}


\twolineshloka
{यः स देवमनुष्याणआं भयमुत्पादयिष्यति}
{त्रयाणामेव लोकानां सत्यमेतद्ब्रवीमि ते}


\threelineshloka
{वरं गृहाण राजर्षे यस्ते मनसि वर्तते}
{तीर्थयात्रां गमिष्यामि पुरा कालोऽतिवर्तते ॥कुशिक उवाच}
{}


\twolineshloka
{एष एव वरो मेऽद्य यत्त्वं प्रीतो महामुने}
{भवत्वेतद्यथार्थत्वं भवेत्पौत्रो ममानघ}


\twolineshloka
{ब्राह्मण्यं मे कुलस्यास्तु भगवन्नेष मे वरः}
{पुनश्चाख्यातुमिच्छामि भगवन्विस्तरेण वै}


\twolineshloka
{कथमेष्यति विप्रत्वं कुलं मे भृगुनन्दन}
{कश्चासौ भविता बन्धुर्मम कश्चापि सम्मतः}


\chapter{अध्यायः ९१}
\twolineshloka
{अवश्यं कथनीयं मे तवैतन्नरपुङ्गव}
{यदर्थं त्वाहमुच्छेत्तुं सम्प्राप्तो मनुजाधिप}


\twolineshloka
{भूगूणां क्षत्रिया याज्या नित्यमेतज्जनाधिप}
{ते च भेदं गमिष्यन्ति दैवयुक्तेन हेतुना}


\twolineshloka
{क्षत्रियाश्च भृगून्सर्वान्वधिष्यन्ति नराधिप}
{आगर्भादनुकृन्तन्तो दैतदण्डनिपीडिताः}


\twolineshloka
{तत उत्पत्स्यतेऽस्माकं कुले गोत्रविवर्धनः}
{और्वो नाम महातेजा ज्वलितार्कसमद्युतिः}


\threelineshloka
{स त्रैलोक्यविनाशाय कोपाग्निं जनयिष्यति}
{महीं सपर्वतवनां यः करिष्यति भस्मसात्}
{}


\threelineshloka
{कञ्चित्कालं तु वह्निं च स एव शमयिष्यति}
{समुद्रे वडवावक्त्रे प्रक्षिप्य मुनिसत्तमः}
{}


\twolineshloka
{पुत्रं तस्य महाराज ऋचीकं भृगुनन्दनम्}
{साक्षात्कृत्स्नो धनुर्वेदः समुपस्थास्यतेऽनघ}


\twolineshloka
{क्षत्रियाणामभावाय दैवयुक्तेनु हेतुना}
{स तु तं प्रतिगृह्यैव पुत्रे संक्रामयिष्यति}


\twolineshloka
{जमदग्नौ महाभागे तपसा भावितात्मनि}
{स चापि भृगुशार्दूलस्तं वेदं धारयिष्यति}


\twolineshloka
{कुलात्तु तव धर्मात्मन्कन्यां सोऽधिगमिष्यति}
{उद्भावनार्थं भवतो वंशस्य भरतर्षभ}


\twolineshloka
{`क्षत्रहन्ता भवेद्धिंस्र इति दैवं सनातनम्}
{नारायणमुपास्यास्य वरात्तं पुत्रमृच्छति ॥'}


\twolineshloka
{गाधेर्दुहितरं प्राप्यि पौत्रीं तव महातपाः}
{ब्राह्मणं क्षत्रधर्माणं पुत्रमुत्पादयिष्यति}


\threelineshloka
{क्षत्रियं विप्रधर्माणं बृहस्पतिमिवौजसा}
{विश्वामित्रं तव कुले गाधेः पुत्रं सुधार्मिकम्}
{तपसा महता युक्तं प्रदास्यति महाद्युते}


\twolineshloka
{स्त्रियौ तु कारणं तत्र परिवर्ते भविष्यतः}
{पितामहनियोगाद्वै नान्यथैतद्भविष्यति}


\threelineshloka
{तृतीये पुरुषे तुभ्यं ब्राह्मणत्वमुपैष्यति}
{भविता त्वं च सम्बन्धी भृगूणां भावितात्मनाम् ॥भीष्म उवाच}
{}


\threelineshloka
{कुशिकस्तु मुनेर्वाक्यं च्यवनस्य महात्मनः}
{श्रुत्वा हृष्टोऽभवद्राजा वाक्यं चेदमुवाच ह}
{एवमस्त्विति धर्मात्मा तदा भरतसत्तम}


\twolineshloka
{च्यवनस्तु महातेजाः पुनरेव नराधिपम्}
{वरार्थं चोदयामास तमुवाच स पार्थिवः}


\twolineshloka
{बाढमेवं ग्रहीष्यामि कामांस्त्वत्तो महामुने}
{ब्रह्मभूतं कुलं मेऽस्तु धर्मे चास्य मनो भवेत्}


\twolineshloka
{एवमुक्तस्तथेत्येवं प्रत्युक्त्वा च्यवनो मुनिः}
{अभ्यनुज्ञाय नृपतिं तीर्थयात्रां ययौ तदा}


\twolineshloka
{एतत्ते कथितं सर्वमशेषेण मया नृप}
{भृगूणां कुशिकानां च अभिसम्बन्धकारणम्}


\twolineshloka
{यथोक्तमृषिणा चापि तदा तदभवन्नृप}
{जन्म रामस्य च मुनेर्विश्वामित्रस्य चैव हि}


\chapter{अध्यायः ९२}
\twolineshloka
{मुह्यामीति निशम्याद्य चिन्तयानः पुनःपुनः}
{हीनां पार्थिवसिंहैस्तैः श्रीमद्भिः पृथिवीमिमाम्}


\twolineshloka
{प्राप्य राज्यानि शतशो महीं जित्वाऽपि भारत}
{कोटिशः पुरुषान्हत्वा परितप्ये पितामह}


\twolineshloka
{का नु तासां वरस्त्रीणामवस्थाऽद्य भविष्यति}
{या हीनाः पतिभिः पुत्रैर्मातुलैर्भ्रातृभिस्तथा}


\twolineshloka
{वयं हि तान्कुरुन्हत्वा ज्ञातींश्च सुहृदोऽपि च}
{अवाक्शीर्षाः पतिष्यामो नरके नात्रं संशयः}


\threelineshloka
{शरीरं योक्तुमिच्छामि तपसोग्रेण भारत}
{उपदिष्टमिहेच्छामि तत्त्वतोऽहं विशाम्पते ॥वैशम्पायन उवाच}
{}


\twolineshloka
{युधिष्ठिरस्य तद्वाक्यं श्रुत्वा भीष्मो महामनाः}
{परीक्ष्य निपुणं बुद्ध्या युधिष्ठिरमभाषत}


\twolineshloka
{रहस्यमद्भुतं चैव शृणु वक्ष्यामि भारत}
{या गतिः प्राप्यते वेन प्रेत्यभावे विशाम्पते}


\twolineshloka
{तपसा प्राप्यते स्वर्गस्तपसा प्राप्यते यशः}
{आयुःप्रकर्षो भोगाश्च लभ्यन्ते तपसा विभो}


\twolineshloka
{ज्ञानं विज्ञानमारोग्यं रूपं सम्पत्तथैव च}
{सौभाग्यं चैव तपसा प्राप्यते भरतर्षभ}


\twolineshloka
{धं प्राप्नोति तपसा मौनं ज्ञानं प्रयच्छति}
{उपभोगांस्तु दानेन ब्रह्मचर्येण जीवितम्}


\twolineshloka
{अहिंसायाः फलं रूपं दीक्षाया जन्म वै कुले}
{फलमूलाशनाद्राज्यं स्वर्गः पर्णशनाद्भवेत्}


\twolineshloka
{पयोभक्षो दिवं याति दानेन द्रविणाधिकः}
{गुरुशुश्रूषया विद्या नित्यश्राद्धेन सन्ततिः}


\twolineshloka
{गवाढ्यः शाकदीक्षाभिः स्वर्गमाहुस्तृणाशनात्}
{स्त्रियस्त्रिषवण्सनानाद्वायुं पीत्वा क्रतुं लभेत्}


\twolineshloka
{नित्यस्नायी दीर्घजीवी सन्ध्ये तु द्वे जपन्द्विजः}
{मन्त्रं साधयतो राजन्नाकपृष्ठमनाशने}


\twolineshloka
{स्थण्डिलेषु शयानानां गृहाणि शयनानि च}
{चीरवल्कलवासोभिर्वासांस्याभाणानि च}


\twolineshloka
{शय्यासनानि यानानि योगयुक्ते तपोधने}
{अग्निप्रवेशे नियतं ब्रह्मलोके महीयते}


\twolineshloka
{रसानां प्रतिसंहारात्सौभाग्यमिह विन्दति}
{आमिषप्रतिसंहारात्प्रजा ह्यायुष्मती भवेत्}


\twolineshloka
{उदवासं वसेद्यस्तु स नराधिपतिर्भवेत्}
{सत्यवादी नरश्रेष्ठ दैवतैः सह मोदते}


\twolineshloka
{कीर्तिर्भवति दानेन तथाऽऽरोग्यमहिंसया}
{द्विजशुश्रूषया राज्यं द्विजत्वं चापि पुष्कलम्}


\twolineshloka
{पानीयस्य प्रदानेन कीर्तिर्भवति शाश्वती}
{अन्नस्य तु प्रदानेन तृप्यन्ते कामभोगतः}


\twolineshloka
{सान्त्वदः सर्वभूतानां सर्वशोकैर्विमुच्यते}
{देवशुश्रूषया राज्यं दिव्यं रूपं निगच्छति}


\twolineshloka
{दीपालोकप्रदानेन चक्षुष्मान्बुद्धिमान्भवेत्}
{प्रेक्षणीयप्रदानेन स्मृतिं मेधां च विन्दति}


\twolineshloka
{गन्धमाल्यप्रदानेन कीर्तिर्भवति पुष्कला}
{केशश्मश्रू धारंयतामग्र्या भवति सन्ततिः}


\twolineshloka
{उपवासं च दीक्षां च अभिषेकं च पार्थिव}
{कृत्वा द्वादशवर्षाणि वीरस्थानाद्विशिष्यते}


\twolineshloka
{दासीदासमलङ्कारान्क्षेत्राणि च गृहाणि च}
{ब्रह्मदेयां सुतां दत्त्वा प्राप्नोति मनुजर्षभ}


\twolineshloka
{क्रतुभिश्चोपवासैश्च त्रिदिवं याति भारत}
{लभते च शिवं ज्ञानं फलपुष्पप्रदो नरः}


\twolineshloka
{सुवर्णशृङ्गैस्तु विराजितानांगवां सहस्रस्य नरः प्रदानात्}
{प्राप्नोति पुण्यं दिवि देवलोक-मित्येवमाहुर्दिवि वेदसङ्घाः}


\twolineshloka
{प्रयच्छते यः कपिलां सवत्सांकांस्योपदोहां कनकाग्रशृङ्गीम्}
{तैस्तैर्गुणैः कामदुघाऽस्य भूत्वानरं प्रदातारमुपैति सा गौः}


\twolineshloka
{यावन्ति रोमाणि भवन्ति धेन्वा-स्तावत्फलं प्राप्य स गोप्रदानात्}
{पुत्रांश्च पौत्रांश्च कुलं च सर्व-मासप्तमं तारयते परत्र}


\twolineshloka
{सदक्षिणां काञ्चनचारुशृङ्गींकांस्योपदोहां द्रविणोत्तरीयाम्}
{धेनुं तिलानां ददतो द्विजायलोका वसूनां सुलभा भवन्ति}


\twolineshloka
{स्वकर्मभिर्मानवं संनिरुद्धंतीव्रान्धकारे नरके पतन्तम्}
{मंहार्णवे नौरिव वायुयुक्तादानं गवां तारयते परत्र}


\twolineshloka
{यो ब्रह्मदेयां तु ददाति कन्यांभूमिप्रदानं च करोति विप्रे}
{ददाति चान्नं विधिवच्च यश्चस लोकमाप्नोति पुरंदरस्य}


\twolineshloka
{नैवेशिकं सर्वगुणोपपन्नंददाति वै यस्तु नरो द्विजाय}
{स्वाध्यायचारित्र्यगुणान्वितायतस्यापि लोकाः कुरुषूत्तरेषु}


\twolineshloka
{धुर्यप्रदानेन गवां तथा वैलोकानवाप्नोति नरो वसूनाम्}
{स्वर्गाय चाहुस्तु हिरण्यदानंततो विशिष्टं कनकप्रदानम्}


\twolineshloka
{छत्रप्रदानेन गृहं वरिष्ठंयानं तथोपानहसम्प्रदाने}
{वस्रप्रदानेन फलं सुरूपंगन्धप्रदानात्सुरभिर्नरः स्यात्}


\twolineshloka
{पुष्पोपगं वाऽथ फलोपगं वायः पादपं स्पर्शयते द्विजाय}
{सश्रीकमृद्धं बहुरत्नपूर्णंलभत्ययत्नोपगतं गृहं वै}


\twolineshloka
{भक्ष्यान्नपानीयरसप्रदातासर्वान्समाप्नोति रसान्प्रकामम्}
{प्रतिश्रयाच्छानसम्प्रदाताप्राप्नोति तान्येव न संशयोऽत्र}


\twolineshloka
{स्रग्धूपगन्धाननुलेपनानिस्नानानि माल्यानि च मानवो यः}
{दद्याद्द्विजेभ्यः स भवेदरोग-स्तथाऽभिरूपश्च नरेन्द्रलोके}


\twolineshloka
{बीजैरशून्यं शयनैरुपेतंदद्याद्गृहं यः पुरुषो द्विजाय}
{पुण्याभिरामं बहुरत्नपूर्णंलभत्यधिष्ठानवरं स राजन्}


\twolineshloka
{सुगन्धचित्रास्तरणोपधानंदद्यान्नरो यः शयनं द्विजाय}
{रूपान्वितां पक्षवतीं मनोज्ञांभार्यामयत्नोपगतां लभेत्सः}


\threelineshloka
{पितामहस्यानवरो वीरशायी भवेन्नरः}
{नाधिकं विद्यते यस्मादित्याहुः परमर्षयः ॥वैशम्पायन उवाच}
{}


\twolineshloka
{तस्य तद्वचनं श्रुत्वा प्रीतात्मा कुरुनन्दनः}
{नाश्रमेऽरोचयद्वासं वीरमार्गाभिकाङ्क्षया}


\twolineshloka
{ततो युधिष्ठिरः प्राह पाण्डवान्पुरुषर्षभ}
{पितामहस्य यद्वाक्यं तद्वो रोचत्विति प्रभुः}


\twolineshloka
{ततस्तु पाण्डवाः सर्वे द्रौपदी च यशस्विनी}
{युधिष्ठिरस्य तद्वाक्यं बाढमित्यभ्यपूजयन्}


\chapter{अध्यायः ९३}
\threelineshloka
{आरामाणां तटाकानां यत्फलं कुरुपुङ्गव}
{तदहं श्रोतुमिच्छामि त्वत्तोऽद्य भरतर्षभ ॥भीष्म उवाच}
{}


\twolineshloka
{सुप्रदर्शा बलवती चित्रा धातुविभूषिता}
{उपेता सर्वभूतैश्च श्रेष्ठा भूमिरिहोच्यते}


\twolineshloka
{तस्याः क्षेत्रविशेषाश्च तटाकानां च बन्धनम्}
{औदकानि च सर्वाणि प्रवक्ष्याम्यनुपूर्वशः}


\twolineshloka
{तटाकानां च वक्ष्यामि कृतानां चापि ये गुणाः}
{त्रिषु लोकेषु सर्वत्र पूजनीयस्तटाकवान्}


\twolineshloka
{अथवा मित्रसदनं मैत्रं मित्रविवर्धनम्}
{कीर्तिसंजननं श्रेष्ठं तटाकानां निवेशनम्}


\twolineshloka
{धर्मस्यार्थस्य कामस्य फलमाहुर्मनीषिणः}
{तटाकसुकृतं देशे क्षेत्रमेकं महाश्रयम्}


\twolineshloka
{चतुर्विधानां भूतानां तटाकमुपलक्षयेत्}
{तटाकानि च सर्वाणि दिशन्ति श्रियमुत्तमाम्}


\twolineshloka
{देवा मनुष्यगन्धर्वाः पितरोरगराक्षसाः}
{स्थावराणि च भूतानि संश्रयन्ति जलाशयम्}


\twolineshloka
{तस्मात्तांस्ते प्रवक्ष्यामि तटाके ये गुणाः स्मृताः}
{या च तत्र फलावाप्तिर्ऋषिभिः समुदाहृताः}


\twolineshloka
{वर्षाकाले तटाके तु सलिलं यस्य तिष्ठति}
{अग्निहोत्रफलं तस्य फलमाहुर्मनीषिणः}


\twolineshloka
{शरत्काले तु सलिलं तटाके यस्य तिष्ठति}
{गोसहस्रस्य स प्रेत्य लभते फलमुत्तमम्}


\twolineshloka
{हेमन्तकाले सलिलं तटाके यस्य तिष्ठति}
{स वै बहुसुवर्णस्य यज्ञस्य लभते फलम्}


\twolineshloka
{यस्य वै शैशिरे काले तटाके सलिलं भवेत्}
{तस्याग्निष्टोमयज्ञस्य फलमाहुर्मनीषिणः}


\twolineshloka
{तटाकं सुकृतं यस्य वसन्ते तु महाश्रयम्}
{अतिरात्रस्य यज्ञस्य फलं स समुपाश्नुते}


\twolineshloka
{निदाघकाले पानीयं तटाके यस्य तिष्ठति}
{वाजिमेधफलं तस्य फलं वै मुनयो विदुः}


\twolineshloka
{स कुलं तारयेत्सर्वं यस्य खाते जलाशये}
{गावः पिबन्ति सलिलं साधवश्च नराः सदा}


\twolineshloka
{तटाके यस्य गावस्तु पिबन्ति तृषिता जलम्}
{मृगपक्षिमनुष्याश्च सोऽश्वमेधफलं लभेत}


\twolineshloka
{यत्पिबन्ति जलं तत्र स्नायन्ते विश्रमन्ति च}
{तटाके यस्य तत्सर्वं प्रेत्यानन्त्याय कल्पते}


\twolineshloka
{दुर्लभं सलिलं तात विशेषेण परत्र वै}
{पानीयस्य प्रदानेन प्रीतिर्भवति शाश्वती}


\twolineshloka
{तिलान्ददत पानीयं दीपान्ददत जागृत}
{ज्ञातिभिः सह मोदध्वमेतत्प्रेत्य सुदुर्लभम्}


\twolineshloka
{सर्वदानैर्गुरुतरं सर्वदानैर्विशिष्यते}
{पानीयं नरशार्दूल तस्माद्दातव्यमेव हि}


\twolineshloka
{एवमेतत्तटाकस्य कीर्तितं फलमुत्तमम्}
{अत ऊर्ध्वं प्रवक्ष्यामि वृक्षाणामवरोपणम्}


\twolineshloka
{स्थावराणां च भूतानां जातयः षट् प्रकीर्तिताः}
{वृक्षगुल्मलतावल्ल्यस्त्वक्सारास्तृणजातयः}


\twolineshloka
{एता जात्यस्तु वृक्षाणां तेषां रोपे गुणास्त्विमे}
{कीर्तिश्च मानुषे लोके प्रेत्य चैव फलं शुभम्}


\twolineshloka
{लभते नाम लोके च पितृभिश्च महीयते}
{देवलोके गतस्यापि नाम तस्य न नश्यति}


\twolineshloka
{अतीतानागते चोभे पितृवंशं च भारत}
{तारयेद्वृक्षरोपी च तस्माद्वृक्षांश्च रोपयेत्}


\twolineshloka
{तस्य पुत्रा भवन्त्येते पादपा नात्र संशयः}
{परलोकगतः स्वर्गं लोकांश्चाप्नोति सोऽव्ययान्}


\twolineshloka
{पुष्णैः सुरगणान्वृक्षाः फलैश्चापि तथा पितॄन्}
{धायया चातिथिं तात पूजयन्ति महीरुहः}


\twolineshloka
{किन्नरोरगरक्षांसि देवगन्धर्वमानवाः}
{तथा ऋषिगणाश्चैव संश्रयन्ति महीरुहान्}


\twolineshloka
{पुष्पिताः फलवन्तश्च तर्पयन्तीह मानवान्}
{वृक्षदं पुत्रवद्वृक्षास्तारयन्ति परत्र तु}


\twolineshloka
{तस्मात्तटाके सद्वृक्षा रोप्याः श्रेयोर्थिना सदा}
{पुत्रवत्परिपाल्याश्च पुत्रास्ते धर्मतः स्मृताः}


\twolineshloka
{तटाककृद्वृक्षरोपी इष्टयज्ञश्च यो द्विजः}
{एते स्वर्गे महीयन्ते ये चान्ये सत्यवादिनः}


\twolineshloka
{तस्मात्तटाकं कुर्वीत आरामांश्चैव रोपयेत्}
{यजेच्च विविधैर्यज्ञैः सत्यं च सततं वदेत्}


\chapter{अध्यायः ९४}
\twolineshloka
{यानीमानि बहिर्वेद्यां दानानि परिचक्षते}
{तेभ्यो विशिष्टं किं दानं मतं ते कुरुपुङ्गव}


\threelineshloka
{कौतूहलं हि परमं तत्र मे विद्यते प्रभो}
{दातारं दत्तमन्वेति यद्दानं तत्प्रचक्ष्व मे ॥भीष्म उवाच}
{}


\twolineshloka
{अभयं सर्वभूतेभ्यो व्यसने चाप्यनुग्रहः}
{यच्चाभिलषितं दद्यात्तृषितायाभियाचते}


\twolineshloka
{भरणे पुत्रदाराणां तद्दानं श्रेष्ठमुच्यते}
{दत्तं दातारमन्वेति तद्दानं भरतर्षभ}


\twolineshloka
{हिरण्यदानं गोदानं पृथिवीदानमेव च}
{एतानि वै पवित्राणि तारयन्त्यपि दुष्कृतात्}


\twolineshloka
{एतानि पुरुषव्याघ्र साधुभ्यो देहि नित्यदा}
{दानानि हि नरं पापान्मोक्षयन्ति न संशयः}


\twolineshloka
{यद्यदिष्टतमं लोके यच्चास्य दयितं गृहे}
{तत्तद्गुणवते देयं तदेवाक्षयमिच्छता}


\twolineshloka
{प्रियाणि लभते नित्यं प्रियदः प्रियकृत्तथा}
{प्रियो भवति भूतानामिह चैव परत्र च}


\twolineshloka
{याचमानमभीमानादनासक्तमकिञ्चनम्}
{यो नार्चति यथाशक्ति स नृशंसो युधिष्ठिर}


\twolineshloka
{अमित्रमपि चेद्दीनं शरणैषिणमागतम्}
{व्यसने योऽनुगृह्णाति स वै पुरुषसत्तमः}


\twolineshloka
{कृशाय कृतविद्याय वृत्तिक्षीणाय सीदते}
{अपहन्यात्क्षुधां यस्तु न तेन पुरुषः समः}


\twolineshloka
{क्रियानियमितान्साधुन्पुत्रदारैश्च कर्शितान्}
{अयाचमानान्कौन्तेय सर्वोपायैर्निमन्त्रयेत्}


\twolineshloka
{आशिषं ये न देवेषु न च मर्त्येषु कुर्वते}
{अर्हन्तो नित्यसत्वस्था यथालब्धोपजीविनः}


\twolineshloka
{आशीविषसमेभ्यश्च तेभ्यो रक्षस्व भारत}
{तान्युक्तैरुपजिज्ञास्य भोगैर्निर्वप रक्ष च}


\twolineshloka
{कृतैरावसथैर्नित्यं सप्रेष्यैः सपरिच्छदैः}
{निमन्त्रयेथाः कौरव्य सर्वभूतसुखावहैः}


\twolineshloka
{यदि ते प्रतिगृह्णीयुः श्रद्धापूतं युधिष्ठिर}
{कार्यमित्येव मन्वाना धार्मिकाः पुण्यकर्मिणः}


\twolineshloka
{विद्यास्नाता व्रतस्नाता धर्ममाश्रित्य जीविनः}
{गूढस्वाध्यायतपसो ब्राह्म्णाः संशितव्रताः}


\twolineshloka
{तेषु शुद्धेषु दान्तेषु स्वदारनिरतेषु च}
{यत्करिष्यसि कल्याणं तत्ते लोके युधाम्पते}


\twolineshloka
{यथाऽग्निहोत्रं सुहुतं सायम्प्रातर्द्विजातिना}
{तथा भवति दत्तं वै विद्वद्भ्यो यत्कृतात्मना}


\twolineshloka
{एष ते विततो यज्ञः श्रद्धापूतः सदक्षिणः}
{विशिष्टः सर्वयज्ञेभ्यो ददतस्तात वर्तताम्}


\twolineshloka
{निवापो दानसदृशः सदृशेषु युधिष्ठिर}
{निवेदयन्पूजयन्वै तेष्वानृण्यं निगच्छति}


\twolineshloka
{य एवं नैव कुप्यन्ते न लुभ्यन्ति तृणेष्वपि}
{त एव नः पूज्यतमा ये चापि प्रियवादिनः}


\twolineshloka
{ये नो न बहुमन्यन्ते न प्रवर्तन्ति याचने}
{पुत्रवत्परिपाल्यास्ते नमस्तेभ्यस्तथाऽभयम्}


\twolineshloka
{ऋत्विक्पुरोहिताचार्या मृदुधर्मपरा हि ये}
{क्षात्रेणापि हि संसृष्टं तेजः शाम्यति तेष्वपि}


\twolineshloka
{ईश्वरो बलवानस्मि राजाऽस्मीति युधिष्ठिर}
{ब्राह्मणान्मास्म पर्यासीर्वासोभिरशनेन च}


\twolineshloka
{यच्छोभार्थं बलार्तं वा वित्तमस्ति तवानघ}
{तेन ते ब्राह्मणाः पूज्याः स्वधर्ममनुतिष्ठता}


\twolineshloka
{नमस्कार्यास्तथा विप्रा वर्तमाना यथातथम्}
{यथासुखं यथोत्साहं ललन्तु त्वयि पुत्रवत्}


\twolineshloka
{को ह्यक्षयप्रसादानां सुहृदामल्पतोषिणाम्}
{वृत्तिमर्हत्युपक्षेप्तुं त्वदन्यः कुरुसत्तम}


\twolineshloka
{यथाऽपत्याश्रयो धर्मः स्त्रीणां लोके सनातनः}
{सदैव सा गतिर्नान्या तथाऽस्माकं द्विजातयः}


\twolineshloka
{यदि नो ब्राह्मणास्तात संत्यजेयुरपूजिताः}
{पश्यन्तो दारुणं कर्म सततं क्षत्रिये स्थितम्}


\twolineshloka
{अवेदानामकीर्तीनामलोकानामयज्विनाम्}
{कोनु स्याज्जीवितेनार्थस्तद्धिनो ब्राह्मणाश्रयम्}


\threelineshloka
{अत्र ते वर्तयिष्यामि यथा धर्मं सनातनम्}
{राजन्यो ब्राह्मणान्राजन्पुरा परिचचार ह}
{वैश्यो राजन्यमित्येव शूद्रो वैश्यमिति श्रुतिः}


\twolineshloka
{दूराच्छूद्रेणोपचर्यो ब्राह्मणोऽग्निरिव ज्वलन्}
{संस्पर्शपरिचर्यसल्तु वैश्येन क्षत्रियेण च}


\twolineshloka
{मृदुभावान्सत्यशीलान्सत्यधर्मानुपालकान्}
{आशीविषानिव क्रुद्धांस्तानुपाचरत द्विजान्}


\threelineshloka
{अपरेषां परेषां च परेभ्यश्चापि ये परे}
{क्षत्रियाणां प्रतपतां तेजसा च बलेन च}
{ब्राह्मणेष्वेव शाम्यन्ति तेजांसि च तपांसि च}


\twolineshloka
{न मे पिता प्रियतरो न त्वं तात तथा प्रियः}
{न मे पितुः पिता राजन्न चात्मा न च जीवितम्}


\twolineshloka
{त्वत्तश्च मे प्रियतरः पृथिव्यां नास्ति कश्चन}
{त्वत्तोऽपि मे प्रियतरा ब्राह्मणा भरतर्षभ}


\twolineshloka
{ब्रवीमि सत्यमेतच्च यथाऽहं पाण्डुनन्दन}
{तेन सत्येन गच्छेयं लोकान्यत्र स शान्तनुः}


\twolineshloka
{पश्येयं च सतां लोकाञ्छुचीन्ब्रह्मपुरस्कृतान्}
{तत्र मे तात गन्तव्यमह्नाय च चिराय च}


\twolineshloka
{सोहमेतादृशान्लोकान्दृष्ट्वा भरतसत्तम}
{यन्मे कृतं ब्राह्मणेषु न तप्ये तेन पार्थिव}


\chapter{अध्यायः ९५}
\threelineshloka
{यौ च स्यातां चरणेनोपपन्नौयौ विद्यया सदृशौ जन्मना च}
{ताभ्यां दानं कतरस्मै विशिष्ट-मयाचमानाय च याचते च ॥भीष्म उवाच}
{}


\twolineshloka
{श्रेयो वै याचतः पार्थ दानमाहुरयाचते}
{अर्हत्तमो वै धृतिमान्कृपणादकृतात्मनः}


\twolineshloka
{क्षत्रियो रक्षणधृतिर्ब्राह्मणोऽनर्थनाधृतिः}
{ब्राह्मणो धृतिमान्विद्वान्देवान्प्रीणाति तुष्टिमान्}


\twolineshloka
{याच्यमाहुरनीशस्य अतिहारं च भारत}
{उद्वेजयन्ति याचन्ति यदा भूतानि दस्युवत्}


\twolineshloka
{म्रियते याचमानो वै तमनु म्रियतेऽददत्}
{ददत्संजीवयत्येनमात्मानं च युधिष्ठिर}


\twolineshloka
{आनृशंस्यं परो धर्मो याचते यत्प्रदीयते}
{अयाचतः सीदमानान्सर्वोपायैर्निमन्त्रयेत्}


\twolineshloka
{यदि वै तादृशा राष्ट्रे वसेयुस्ते द्विजोत्तमाः}
{भस्मच्छन्नानिवाग्नींस्तान्बुध्येथास्त्वं प्रयत्नतः}


\twolineshloka
{तपसा दीप्यमानास्ते दहेयुः पृथिवीमपि}
{अपूज्यमानाः कौरव्य पूजार्हास्तु तथाविधाः}


\twolineshloka
{पूज्या हि ज्ञानविज्ञानतपोयोगसमन्विताः}
{तेभ्यः पूजां प्रयुञ्जीथा ब्राह्मणेभ्यः परन्तपः}


% Check verse!
ददद्बहुविधान्देयानुपच्छन्दयते च तान्
\twolineshloka
{यदग्निहोत्रे सुहुते सायंप्रातर्भवेत्फलम्}
{विद्यावेदव्रतवति तद्दानफलमुच्यते}


\twolineshloka
{विद्यावेदव्रतस्नाता न व्यापाश्रयजीविनः}
{गूढस्वाध्यायतपसो ब्राह्मणान्संशितव्रतान्}


\twolineshloka
{कृतैरावसथैर्हृद्यैः सप्रेष्यैः सपरिच्छदैः}
{निमन्त्रयेथाः कौरव्य कामैश्चान्यैर्द्विजोत्तमान्}


\twolineshloka
{अपि ते प्रतिगृह्णीयुः श्रद्धोपेतं युधिष्ठिर}
{कार्यमित्येव मन्वाना धर्मज्ञाः सूक्ष्मदर्शिनः}


\twolineshloka
{अपि ते ब्राह्मणा भुक्त्वा गताः सोद्धरणान्गृहान्}
{येषां दाराः प्रतीक्षन्ते पर्जन्यमिव कर्षकाः}


\twolineshloka
{अन्नानि प्रातःसवने नियता ब्रह्मचारिणः}
{ब्राह्मणास्तात भुञ्जानास्त्रेताग्निं प्रीणयन्त्युत}


\twolineshloka
{माध्यंदिनं ते सवनं ददतस्तात वर्तताम्}
{गोहिरण्यानि वासांसि तेनेन्द्रः प्रीयतां तव}


\twolineshloka
{तृतीयं सवनं ते वै वैश्वदेवं युधिष्ठिर}
{यद्देवेभ्यः पितृभ्यश्च विप्रेभ्यश्च प्रयच्छसि}


\twolineshloka
{अहिंसा सर्वभूतेभ्यः संविभागश्च सर्वशः}
{दमस्त्यागो धृतिः सत्यं भवत्यवभृथाय ते}


\twolineshloka
{एथ ते चिततो यज्ञः श्रद्धापूतः सदक्षिणः}
{विशिष्टः सर्वयज्ञानां नित्यं तात प्रवर्तताम्}


\chapter{अध्यायः ९६}
\twolineshloka
{दानं यज्ञः क्रिया चेह किंस्वित्प्रेत्य महाफलम्}
{कस्य ज्यायः फलं प्रोक्तं कीदृशेभ्यः कथं कदा}


\twolineshloka
{एतदिच्छामि विज्ञातुं याथातथ्येनि भारत}
{विद्वञ्जिज्ञासमानाय दानधर्मान्प्रचक्ष्व मे}


\threelineshloka
{अन्तर्वेद्यां च यद्दत्तं श्रद्धया चानृशंस्यतः}
{किंस्विन्नैःश्रेयसं तात तन्मे ब्रूहि पितामह ॥भीष्म उवाच}
{}


\twolineshloka
{रौद्रं कर्म क्षत्रियस्य सततं तात वर्तते}
{नास्य वैतानिकफलं विना दानं सुपावनम्}


% Check verse!
न तु पापकृतां राज्ञां याजका द्विजसत्तमाः ॥धने सत्यप्रदातॄणां प्रतिगृह्णन्ति साधवः
\twolineshloka
{प्रतिगृह्णन्ति न तु चेद्यद्रोषादाप्तदक्षिणैः}
{एतस्मात्कारणाद्यज्ञैर्यजेद्राजाऽऽप्तदक्षिणैः}


\twolineshloka
{अथ चेत्प्रतिगृह्णीयुर्दद्यादहरहर्नृपः}
{श्रद्धामास्थाय परमां पावनं ह्येतुदुत्तमम्}


\twolineshloka
{ब्राह्मणांस्तर्पयन्द्रव्यैः स वै यज्ञोऽनुपद्रवः}
{मैत्रान्साधून्वेदविदः शीलवृत्ततपोर्जितान्}


\twolineshloka
{यत्ते ते न करिष्यन्ति कृतं ते न भविष्यति}
{यज्ञान्साधय साधुभ्यः स्वाद्वन्नान्दक्षिणावतः}


\twolineshloka
{इष्टं दत्तं च मन्येथा आत्मानं दानकर्मणा}
{पूजयेथा यायजूकांस्तवाप्यंशो भवेद्यथा}


\threelineshloka
{`विद्वद्भ्यः सम्प्रदानेन तत्राप्यंशोऽस्य पूजया}
{यज्वभ्यश्चाथ विद्वद्भ्यो दत्त्वा लोकं प्रदापयेत्}
{प्रदद्याज्ज्ञानदातॄणां ज्ञानदानांशभाग्यभवेत् ॥'}


\twolineshloka
{प्रजावतो भरेथाश्च ब्राह्मणान्बहुभारिणः}
{प्रजावांस्तेन भवति यथा जनयिता तथा}


\twolineshloka
{यावतः साधुधर्मान्वै सन्तः संवर्धयन्त्युत}
{सर्वस्वैश्चापि भर्तव्या नरा ये बहुकारिणः}


\twolineshloka
{समृद्धः सम्प्रयच्छ त्वं ब्राह्म्णेभ्यो युधिष्ठिर}
{धेनूरनडुहोऽन्नानि च्छत्रं वासांस्युपानहौ}


\twolineshloka
{आज्यानि यजमानेभ्यस्तथाऽन्नानि च भारत}
{अश्ववन्ति च यानानि वेश्मानि शयनानि च}


\twolineshloka
{एते देयाः पुष्टिमद्भिर्लघूपायाश्च भारत}
{अजुगुप्सांश्च विज्ञाय ब्राह्मणान्वृत्तिकर्शितान्}


\twolineshloka
{उपच्छन्नं प्रकाशं वा वृत्त्या तान्प्रतिपालय}
{राजसूयाश्वमेधाभ्यां श्रेयस्तत्क्षत्रियान्प्रति}


\twolineshloka
{एवं पापैर्विनिर्मुक्तस्त्वं पूतः स्वर्गमाप्स्यसि}
{सञ्चयित्वा पुनः कोशं यद्राष्ट्रं पालयिष्यसि}


\twolineshloka
{तेन त्वं ब्रह्मभूयत्वमवाप्स्यसि धनानि च}
{आत्मनश्च परेषां च वृत्तिं संरक्ष भारत}


\twolineshloka
{पुत्रवच्चापि भृत्यान्स्वान्प्रजाश्च परिपालय}
{[योगः क्षेमश्च ते नित्यं ब्राह्मणेष्वस्तु भारत}


\twolineshloka
{तदर्थं जीवितं तेऽस्तु मा तेभ्योऽप्रतिपालनम्}
{अनर्थो ब्राह्मणस्यैष यद्वित्तनिचयो महान्}


\threelineshloka
{क्षिया ह्यभीक्ष्णं संवासो दर्पयेत्सम्प्रमोहयेत्}
{ब्राह्मणेषु प्रमूढेषु धर्मो विप्रणशेद्धुवम्}
{धर्मप्रणाशे भूतानामभावः स्यान्न संशयः}


\twolineshloka
{यो रक्षिभ्यः सम्प्रदाय राजा राष्ट्रं विलुम्पति}
{यज्ञे राष्ट्राद्धनं तस्मादानयध्वमिति ब्रुवन्}


\twolineshloka
{यच्चादाय तदाज्ञप्तं भीतं दत्तं सुदारुणम्}
{यजेद्राजा न तं यज्ञं प्रशंसन्त्यस्य साधवः}


\twolineshloka
{अपीडिताः सुसंवृद्धा ये ददत्यनुकूलतः}
{तादृशेनाप्युपायेन यष्टव्यं नोद्यमाहृतैः}


\twolineshloka
{यदा परिनिषिच्येत निहितो वै यथाविधि}
{तदा राजा महायज्ञैर्यजेत बहुदक्षिणैः}


\twolineshloka
{वृद्धबालधनं रक्ष्यमन्धस्य कृपणस्य च}
{न खातपूर्वं कुर्वीत न रुदन्तीधनं हरेत्}


\twolineshloka
{हृतं कृपणवित्तं हि राष्ट्रं हन्ति नृप श्रियम्}
{दद्याच्च महतो भोगान्क्षुद्भयं प्रणुदेत्सताम्}


\twolineshloka
{येषां स्वादूनि भोज्यानि समवेक्ष्यन्ति बालकाः}
{नाश्नन्ति विधिवत्तानि किन्नु पापतरं ततः}


\twolineshloka
{यदि ते तादृशो राष्ट्रे विद्वान्त्सीदेत्क्षुधा द्विजः}
{भ्रूणहत्यां च गच्छेथाः कृत्वा पापमिवोत्तमम्}


\twolineshloka
{धिक्तस्य जीवितं राज्ञो राष्ट्रे यस्यावसीदति}
{द्विजोऽन्यो वा मनुष्योपि शिबिराह वचो यथा}


\twolineshloka
{यस्य स्म विषये राज्ञः स्नातकः सीदति क्षुधा}
{अवृद्धिमेति तद्राष्ट्रं विन्दते सह राजकम्}


\twolineshloka
{क्रोशन्त्यो यस्य वै राष्ट्राद्ध्रियन्ते तरसा स्त्रियः}
{क्रोशतां पतिपुत्राणां मृतोऽसौ न च जीवति]}


\twolineshloka
{अरक्षितारं हर्तारं विलोप्तारमनायकम्}
{तं वै राजकलिं हन्युः प्रजाः सन्नह्य निर्घृणं}


\twolineshloka
{अहं वो रक्षितेत्युक्त्वा यो न रक्षति भूमिपः}
{स संहत्य निहन्तव्यः श्वेव सोन्माद आतुरः}


\twolineshloka
{पापं कुर्वन्ति यत्किञ्चित्प्रजा राज्ञा ह्यरक्षिताः}
{चतुर्थं तस्य पापस्य राजा विन्दति भारत}


\twolineshloka
{अथाहुः सर्वमेवैति भूयोऽर्धमिति निश्चयः}
{चतुर्थं मतमस्माकं मनोः श्रुत्वानुशासनम्}


\twolineshloka
{शुभं वा यच्च कुर्वन्ति प्रजा राज्ञा सुरक्षिताः}
{चतुर्थं तस्य पुण्यस्य राजा चाप्नोति भारत}


\twolineshloka
{जीवन्तं त्वानुजीवन्तु प्रजाः सर्वा युधिष्ठिर}
{पर्जन्यमिव भूतानि महाद्रुममिवाण्डजाः}


\twolineshloka
{कुबेरमिव रक्षांसि शतक्रतुमिवामराः}
{ज्ञातयस्त्वाऽनुजीवन्तु सुहृदश्च परन्तप}


\chapter{अध्यायः ९७}
\threelineshloka
{इदं देयमिदं देयमितीयं श्रुतिचोदनात्}
{बहुदेयाश्च राजानः किंस्विद्देयमनुत्तमम् ॥भीष्म उवाच}
{}


\twolineshloka
{अति दानानि सर्वाणि पृथिवीदानमुच्यते}
{अचला ह्यक्षया भूमिर्दोग्ध्री कामानिहोत्तमान्}


\twolineshloka
{दोग्ध्री वासांसि रत्नानि पशून्व्रीहियवांस्तथा}
{भूमिदः सर्वभूतेषु शाश्वतीरेधते समाः}


\twolineshloka
{यावद्भूमेरायुरिह तावद्भूमिद एधते}
{न भूमिदानादस्तीह परं किञ्चिद्युधिष्ठिर}


\twolineshloka
{अप्यल्पं प्रददुः सर्वे पृथिव्या इति नः श्रुतम्}
{भूमिमेव ददुः सर्वे भूमिं ते भुञ्जते जनाः}


\twolineshloka
{स्वकर्मैवोपजीवन्ति नरा इह परत्र च}
{भूमिः पतिं महादेवी दातारं कुरुते प्रियम्}


\twolineshloka
{य एतां दक्षिणां दद्यादक्षयां राजसत्तम}
{पुनर्नरत्वं सम्प्राप्य भवेत्स पृथिवीपतिः}


\twolineshloka
{यथा दानं तथा भोग इति धर्मेषु निश्चयः}
{सङ्ग्रामे वा तनुं जह्याद्दद्याच्च पृथिवीमिमांम्}


\twolineshloka
{इत्येतत्क्षत्रबन्धूनां वदन्ति परमाशिषः}
{पुनाति दत्ता पृथिवी दातारमिति शुश्रुम}


\twolineshloka
{अपि पापसमाचारं ब्रह्मघ्नमपि चानृतम्}
{सैव पापं प्लावयति सैव पापात्प्रमोचयेत्}


\twolineshloka
{अपि पापकृतां राज्ञां प्रतिगृह्णन्ति साधवः}
{पृथिवीं नान्यदिच्छन्ति पावनं जगती यतः}


\twolineshloka
{नामास्याः प्रियदत्तेति गुह्यं देव्याः सनातनम्}
{दानं वाऽप्यथवाऽदानं नामास्याः प्रथमप्रियम्}


\twolineshloka
{य एतां विदुषे दद्यात्पृथिवीं पृथिवीपतिः}
{पृथिव्यामेतदिष्टं स राजा राज्यमितो व्रजेत्}


\twolineshloka
{पुनश्चासौ जनिं प्राप्य राजवत्स्यान्न संशयः}
{तस्मात्प्राप्यैव पृथिवीं दद्याद्विप्राय पार्थिवः}


\twolineshloka
{नाभूमिपतिना भूमिरधिष्ठेया कथञ्चन}
{न च वस्त्रेणि वा गूहेदन्तर्धानेन वा चरेत्}


\twolineshloka
{ये चान्ये भूमिमिच्छेयुः कुर्युरेवं न संशयः}
{यः साधोर्भूमिमादत्ते न भूमिं विन्दते तु सः}


\twolineshloka
{भूमिं दत्त्वा तु साधुभ्यो विन्दते भूमिमुत्तमाम्}
{प्रेत्य चेह च धर्मात्मा सम्प्राप्नोति महद्यशः}


\twolineshloka
{`एकाहारकरीं दत्त्वा षष्ठिसाहस्रमूर्ध्वगः}
{तावत्या हरणे पृथ्व्या नरकं द्विगुणोत्तरम् ॥'}


\twolineshloka
{यस्य विप्रास्तु शंसन्ति साधोर्भूमिं सदैव हि}
{न तस्य शत्रवो राजन्प्रशंसन्ति वसुन्धराम्}


\twolineshloka
{यत्किञ्चित्पुरुषः पापं कुरुते वृत्तिकर्शितः}
{अपि गोचर्ममात्रेण भूमिदानेन पूयते}


\twolineshloka
{येऽपि सङ्गीर्णाकर्माणो राजानो रौद्रकर्मिणः}
{तेभ्यः पवित्रमाख्येयं भूमिदानमनुत्तमम्}


\twolineshloka
{अल्पान्तरमिदं शश्वत्पुराणा मेनिरे जनाः}
{यो यजेताश्वमेधेन दद्याद्वा साधवे महीम्}


\twolineshloka
{अपि चेत्सुकृतं कृत्वा शङ्केरन्नपि पण्डिताः}
{अशंक्यमेकमेवैतद्भूमिदानमनुत्तमम्}


\twolineshloka
{सुवर्णं रजतं वस्त्रं मणिमुक्तावसूनि च}
{सर्वमेतन्महाप्राज्ञो ददाति वसुधां ददत्}


\twolineshloka
{तपो यज्ञः श्रुतं शीलमलोभः सत्यसन्धता}
{गुरुदैवतपूजा च एता वर्तन्ति भूमिदम्}


\twolineshloka
{भर्तुनिःश्रेयसे युक्तास्त्यक्तात्मानो रणे हताः}
{ब्रह्मलोकगताः सिद्धा नातिक्रामन्ति भूमिदम्}


\twolineshloka
{यथा जनित्री स्वं पुत्रं क्षीरेण भरते सदा}
{अनुगृह्णाति दातारं तथा सर्वरसैर्मही}


\twolineshloka
{मृत्युर्वैकिंकरो दण्डस्तापो वह्नेः सुदारुणः}
{घोराश्च वारुणाः पाशा नोपसर्पन्ति भूमिदम्}


\twolineshloka
{पितॄंश्च पितृलोकस्थान्देवलोके च देवताः}
{सन्तर्पयति शान्तात्मा यो ददाति वसुन्धराम्}


\twolineshloka
{कृशाय म्रियमाणाय वृत्तिग्लानाय सीदते}
{भूमिं वृत्तिकरीं दत्त्वा सत्री भवति मानवः}


\twolineshloka
{यथा धावति गौर्वत्सं स्रवन्ती वत्सला पयः}
{एवमेव महाभाग भूमिर्भरति भूमिदम्}


\twolineshloka
{हलकृष्टां महीं दत्त्वा सबीजां सफलामपि}
{सोदकं वाऽपि शरणं तथा भवति कामदः}


\twolineshloka
{ब्राह्मणं वृत्तसम्पन्नमाहिताग्निं शुचिव्रतम्}
{नरः प्रतिग्राह्य महीं न याति यमसादनम्}


\twolineshloka
{यथा चन्द्रमसो वृद्धिरहन्यहनि जायते}
{तथा भूमिकृतं दानं सस्येसस्ये विवर्धते}


\twolineshloka
{अत्र गाथा भूमिगीताः कीर्तयन्ति पुराविदः}
{याः श्रुत्वा जामदग्न्येन दत्ता भूः काश्यपाय वै}


\twolineshloka
{मामेवादत्त मां दत्त मां दत्त्वा मामवाप्स्यथ}
{अस्मिँल्लोके परे चैव तद्दत्तं जायते पुनः}


\twolineshloka
{य इमां व्याहृतिं वेद ब्राह्मणो वेदसम्मिताम्}
{श्राद्धस्य हूयमानस्य ब्रह्मभूयं स गच्छति}


\twolineshloka
{कृत्यानामभिशप्तानामरिष्टशमनं महत्}
{प्रायश्चित्तं महीं दत्त्वा पुनात्युभयतो दश}


\twolineshloka
{पुनाति य इदं वेद वेदवादं तथैव च}
{प्रकृतिः सर्वभूतानां भूमिर्वै शाश्वती मता}


\twolineshloka
{अभिषिच्यैव नृपतिं श्रावयेदिममागमम्}
{यथा श्रुत्वा महीं दद्यान्नादद्यात्साधुतश्च तां}


\twolineshloka
{सोऽयं कृत्स्नो ब्राह्मणार्थो राजार्थश्चाप्यसंशयः}
{राजा हि धर्मकुशलः प्रथमं भूतिलक्षणम्}


\twolineshloka
{अथ येषामधर्मज्ञो राजा भवति नास्तिकः}
{न ते सुखं प्रबुध्यन्ति न सुखं प्रस्वपन्ति च}


\twolineshloka
{सदा भवन्ति चोद्विग्नास्तस्य दुश्चरितैर्नराः}
{योगक्षेमा हि बहवो राष्ट्रं नास्याविशन्ति तत्}


\twolineshloka
{अथ येषां पुनः प्राज्ञो राजा भवति धार्मिकः}
{सुखं ते प्रतिबुध्यन्ते सुसुखं प्रस्वपन्ति च}


\twolineshloka
{तस्य राज्ञः शुभे राज्ये कर्मभिर्निर्वृता नराः}
{योगक्षेमेण वृष्ट्या च विवर्धन्ते स्वकर्मभिः}


\twolineshloka
{स कुलीनः स पुरुषः स बन्धुः स च पुण्यकृत्}
{स दाता स च विक्रान्तो यो ददाति वसुन्धरां}


\twolineshloka
{आदित्या इव दीप्यन्ते तेजसा भुवि मानवाः}
{ददन्ति वसुधां स्फीतां ये वेदविदुषि द्विजे}


\twolineshloka
{यथा सस्यानि रोहन्ति प्रकीर्णानि महीतले}
{तथा कामाः प्ररोहन्ति भूमिदानसमार्जिताः}


\twolineshloka
{आदित्यो वरुणो विष्णुर्ब्रह्मा सोमो हुताशनः}
{शूलिपाणिश्च भगवान्प्रतिनन्दन्ति भूमिदम्}


\twolineshloka
{भूमौ जायन्ति पुरुषा भूमौ निष्ठां व्रजन्ति च}
{चतुर्विधो हि लोकोऽयं योऽयं भूमिगुणात्मकः}


\twolineshloka
{एषा माता पिता चैव जगतः पृथिवीपते}
{नानया सदृशं भूतं किञ्चिदस्ति जनाधिप}


\twolineshloka
{अत्राप्युदाहरन्तीममितिहासं पुरातनम्}
{बृहस्पतेश्च संवादमिन्द्रस्य च युधिष्ठिर}


\twolineshloka
{इष्ट्वा क्रतुशतेनाथ महता दक्षिणावता}
{मघवा वाग्विदांश्रेष्ठं पप्रच्छेदं बृहस्पतिम्}


\threelineshloka
{भगवन्केन दानेन स्वर्गतः सुखमेधते}
{यदक्षयमहार्यं च तद्ब्रूहि वदतांवर ॥भीष्म उवाच}
{}


\twolineshloka
{इत्युक्तृः स सुरेन्द्रेण ततो देवपुलोहितः}
{बृहस्पतिर्बृहत्तेजाः प्रत्युवाच शतक्रतुम्}


\threelineshloka
{सुवर्णदानं गोदानं भूमिदानं च वृत्रहन्}
{`विद्यादानं च कन्यानां दानं पापहरं परम्}
{'दददेतान्महाप्राज्ञः सर्वपापैः प्रमुच्यते}


\twolineshloka
{न भूमिदानाद्देवेन्द्र परं किञ्चिदिति प्रभो}
{विशिष्टमिति मन्येऽहं यता प्राहुर्मनीषिणः}


\twolineshloka
{`ब्राह्मणार्थे गवार्थे वा राष्ट्रघातेऽथ स्वामिनः}
{कुलस्त्रीणां परिभवे मृतास्ते भूमिपैः समाः'}


\twolineshloka
{ये शूरा निहता युद्धे स्वर्याता रणगृद्धिनः}
{सर्वे ते विबुधश्रेष्ठ नातिक्रामन्ति भूमिदम्}


\twolineshloka
{भर्तुर्निःश्रेयसे युक्तास्त्यक्तात्मानो रणे हताः}
{ब्राह्मलोकगता युक्ता नातिक्रामन्ति भूमिदम्}


\twolineshloka
{पञ्च पूर्वा हि पुरुषाः षडन्ये वसुधां गताः}
{एकादश ददद्भूमिं परित्रातीह मानवः}


\twolineshloka
{रत्नोपकीर्णां वसुधां यो ददाति पुरंदर}
{स मुक्तः सर्वकलुषैः स्वर्गलोके महीयते}


\twolineshloka
{महीं स्फीतां ददद्राजन्सर्वकामगुणान्विताम्}
{राजाधिराजो भवति तद्धि दानमनुत्तमम्}


\twolineshloka
{सर्वकामसमायुक्तां काश्यपीं यः प्रयच्छति}
{सर्वभूतानि मन्यन्ते मां ददातीति वासव}


\twolineshloka
{सर्वकामदुघां धेनुं सर्वकामगुणान्विताम्}
{ददाति यः सहस्राक्ष स्वर्गं याति स मानवः}


\twolineshloka
{मधुसर्पिःप्रवाहिण्यः पयोदधिवहास्तथा}
{सरितस्तपर्यन्तीह सुरेन्द्र वसुधाप्रदम्}


\twolineshloka
{भूमिप्रदानान्नृपतिर्मुच्यते सर्वकिल्बिषात्}
{न हि भूमिप्रदानेन दानमन्यद्विशिष्यते}


\twolineshloka
{ददाति यः समुन्द्रान्तां पृथिवीं शस्त्रनिर्जिताम्}
{तं जनाः कथयन्तीह यावद्धरति गौरियम्}


\twolineshloka
{पुण्यामृद्धिरसां भूमिं यो ददाति पुरंदर}
{न तस्य लोकाः क्षीयन्ते भूमिदानगुणान्विताः}


\twolineshloka
{सर्वदा पार्थिवेनेह सततं भूतिमिच्छता}
{भूर्देया विधिवच्छक्र पात्रे सुखमभीप्सुनां}


\twolineshloka
{अपि कृत्वा नरः पापं भूमिं दत्त्वा द्विजातये}
{समुत्सृजति तत्पापं जीर्णां त्वचमिवोरगः}


\twolineshloka
{सागरान्सरितः शैलान्काननानि च सर्वशः}
{सर्वमेतन्नरः शक्र ददाति वसुधां ददत्}


\twolineshloka
{तटाकान्युदपानानि स्रोतांसि च सरांसि च}
{स्नेहान्सर्वरसांश्चैव ददाति वसुधां ददत्}


\twolineshloka
{ओषधीर्वीर्यसम्पन्नानगान्पुष्पफलान्वितान्}
{काननोपलशैलांश्च ददाति वसुधां ददत्}


\twolineshloka
{अग्निष्टोमप्रभृतिभिरिष्ट्वा च स्वाप्तदक्षिणैः}
{न तत्फलमवाप्नोति भूमिदानाद्यदश्नुते}


\twolineshloka
{दाता दशानुगृह्णाति दश हन्ति तथा क्षिपन्}
{पूर्वदत्तां हरन्भूमिं नरकायोपगच्छति}


\twolineshloka
{न ददाति प्रतिश्रुत्य दत्त्वाऽपि च हरेत्तु यः}
{स बद्धो वारुणैः पाशैस्तप्यते मृत्युसासनात्}


\twolineshloka
{आहिताग्निं सदायज्ञं कृशवृत्तिं प्रियातिथिम्}
{ये भरन्ति द्विजश्रेष्ठं नोपसर्पन्ति ते यमम्}


\twolineshloka
{ब्राह्मणेष्वनृणीभूतः पार्थिवः स्यात्पुरंदर}
{इतरेषां तु वर्णानां तारयेत्कृशदुर्बलान्}


\twolineshloka
{नाच्छिन्द्यात्स्पर्शितां भूमिं परेण त्रिदशाधिप}
{ब्राह्मणस्य सुरश्रेष्ठ कृशवृत्तेः कदाचन}


\twolineshloka
{यथाश्रु पतितं तेषां दीनानामथ सीदताम्}
{ब्राह्मणानां हृते क्षेत्रे हन्यात्त्रिपुरुषं कुलम्}


\twolineshloka
{भूमिपालं च्युतं राष्ट्राद्यस्तु संस्थापयेत्पुनः}
{तस्य वासः सहस्राक्ष नाकपृष्ठे महीयते}


\twolineshloka
{`सुनिर्मितां सुविक्रीतां सुभृतां स्थापयेन्नृप}
{'इक्षुभिः सन्ततां भूमिं यवगोधूमसालिनीम्}


\twolineshloka
{गोश्ववाहनपूर्णां वा यो ददाति वसुन्धराम्}
{विमुक्तः सर्वपापेभ्यः स्वर्गलोके महीयते ॥'}


\twolineshloka
{निधिगर्भां ददद्भूमिं सर्वरत्नपरिच्छदाम्}
{अक्षयाँल्लभते लोकान्भूमिसत्रं हि तस्य तत}


\twolineshloka
{विधूय कलुषं सर्वं विरजाः सम्मतः सताम्}
{लोके महीयते सद्भिर्यो ददाति वसुन्धराम्}


\twolineshloka
{यथाऽप्सु पतितः शक्र तैलबिन्दुर्विसर्पति}
{तथा भूमिकृतं दानं सस्येसस्ये विवर्धते}


\twolineshloka
{ये रणाग्रे महीपालाः शूराः समितिशोभनाः}
{वध्यन्तेऽभिमुखाः शक्र ब्रह्मलोकं व्रजन्ति ते}


\twolineshloka
{नृत्तगीतपरा नार्यो दिव्यमाल्यविभूषिताः}
{उपतिष्ठन्ति देवेन्द्र यथा भूमिप्रदं दिवि}


\twolineshloka
{मोदते च सुखं स्वर्गे देवगन्धर्वपूजितः}
{यो ददाति महीं सम्यग्विधिनेह द्विजातये}


\twolineshloka
{शतमप्सरसश्चैव दिव्यमाल्यविभूषिताः}
{उपतिष्ठन्ति देवेन्द्र ब्रह्मलोके धराप्रदम्}


\twolineshloka
{उपतिष्ठन्ति पुण्यानि सदा भूमिप्रदं नरम्}
{शङ्खं भद्रासनं छत्रं वराश्वा वरवाहनम्}


\twolineshloka
{भूमिप्रदानात्पुष्पाणि हिरण्यनिचयास्तथा}
{आज्ञा सदाऽप्रतिहता जयशब्दा वसूनि च}


\twolineshloka
{भूमिदानस्य पुण्यानि फलं स्वर्गः पुरन्दर}
{हिरण्यपुष्पाश्चौषध्यः कुशकाञ्चनशाद्वलाः}


% Check verse!
अमृतप्रसवां भूमिं प्राप्नोति पुरुषो ददत्
\threelineshloka
{नास्ति भूमिसमं दानं नास्ति मातृसमो गुरुः}
{नास्ति सत्यसमो धर्मो नास्ति दानसमो निधिः ॥भीष्म उवाच}
{}


\twolineshloka
{एतदाङ्गिरसाच्छ्रुत्वा वासवो वसुधामिमाम्}
{वसुरत्नसमाकीर्णां ददावाङ्गिरसे तदा}


\twolineshloka
{य इदं श्रावयेच्छ्राद्धे भूमिदानस्य संस्तवम्}
{न तस्य रक्षसां भागो नासुराणां भवत्युत}


\twolineshloka
{अक्षयं च भवेद्दत्तं पितृभ्यस्तन्न संशयः}
{तस्माच्छ्राद्धेष्विदं विद्वान्भुञ्जतः श्रावयेद्द्विजान्}


\twolineshloka
{इत्येतत्सर्वदानानां श्रेष्ठमुक्तं तवानघ}
{मया भरतशार्दूल किं भूयः श्रोतुमिच्छसि}


\chapter{अध्यायः ९८}
\twolineshloka
{कानि दानानि लोकेऽस्मिन्दातुकामो महीपतिः}
{गुणाधिकेभ्यो विप्रेभ्यो दद्याद्भरतसत्तम}


\twolineshloka
{केन तुष्यन्ति ते सद्यः किं तुष्टाः प्रदिशन्ति च}
{शंस मे तन्महाबाहो फलं पुण्यकृतं महत्}


\threelineshloka
{दत्तं किं फलवद्राजन्निह लोकें परत्र च}
{भवतः श्रोतुमिच्छामि तन्मे विस्तरतो वद ॥भीष्म उवाच}
{}


\threelineshloka
{इममर्थं पुरा पृष्टो नारदो देवदर्शनः}
{यदुक्तवानसौ वाक्यं तन्मे निगदतः शृणु ॥नारद उवाच}
{}


\twolineshloka
{अन्नमेव प्रशंसन्ति देवा ऋषिगणास्तथा}
{लोकतन्त्रं हि यज्ञाश्च सर्वमन्ने प्रतिष्ठितम्}


\twolineshloka
{अन्नेन सदृशं दानं न भूतं न भविष्यति}
{तस्मादन्नं विशेषेणि दातुमिच्छन्ति मानवाः}


\twolineshloka
{अन्नमूर्जस्करं लोके प्राणाश्चान्ने प्रतिष्ठिताः}
{अन्नेन धार्यते सर्वं विश्वं जगदिदं प्रभो}


% Check verse!
अन्नाद्गृहस्था लोकेऽस्मिन्भिक्षवस्तापसास्तथा ॥अन्नाद्भवन्ति वै प्राणाः प्रत्यक्षं नात्र संशयः
\twolineshloka
{कटुम्बिने सीदते च ब्राह्मणाय महात्मने}
{दातव्यं भिक्षवे चान्नमात्मनो भूतिमिच्छता}


\twolineshloka
{ब्राह्मणायाभिरूपाय यो दद्यादन्नमर्थिने}
{निदधाति निधिं श्रेष्ठं पारलौकिकमात्मनः}


\twolineshloka
{श्रान्तमध्वनि वर्तन्तं वृद्धमर्हमुपस्थितम्}
{अर्ययेद्भूतिमन्विच्छन्गृहस्थो गृहमागतम्}


\twolineshloka
{क्रोधमुत्पतितं हित्वा सुशीलो वीतमत्सरः}
{अन्नदः प्राप्नुते राजन्दिवि चेह च यत्सुखम्}


\twolineshloka
{नावमन्येदभिगतं न प्रणुद्यात्कदाचन}
{अपि श्वपाके शुनि वा नान्नदानं प्रणश्यति}


\twolineshloka
{यो दद्यादपरिक्लिष्टमन्नमध्वनि वर्तते}
{आर्तायादृष्टपूर्वाय स महद्धर्ममाप्नुयात्}


\twolineshloka
{पितॄन्देवानृपीन्विप्रानतिथींश्च जनाधिप}
{यो नरः प्रीणयत्यन्नैस्तस्य पुण्यफलं महत्}


\twolineshloka
{कृत्वाऽतिपातकं कर्म यो दद्यादन्नमर्थिने}
{ब्राह्मणाय विशेषेण न स पापेन मुह्यते}


\twolineshloka
{ब्राह्मणेष्वक्षयं दानमन्नं शूद्रे महाफलम्}
{अन्नदानं हि शूद्रे च ब्राह्मणे च विशिष्यते}


\twolineshloka
{न पृच्छेद्गोत्रचरणं स्वाध्यायं देशमेव च}
{भिक्षितो ब्राह्मणेनान्नं दद्यादेवाविचारतः}


\twolineshloka
{अन्नदस्यान्नदा वृक्षाः सर्वकामफलप्रदाः}
{भवन्ति चेह चामुत्र नृपते नात्र संशयः}


\twolineshloka
{आशंसन्ते हि पितरः सुवृष्टिमिव कर्षकाः}
{अस्माकमपि पुत्रो वा पौत्रो वाऽन्नं प्रदास्यति}


\twolineshloka
{ब्राह्मणो हि महद्भूतं स्वयं देहीति याचते}
{अकामो वा सकामो वा दत्त्वा पुण्यमवाप्नुयात्}


\twolineshloka
{ब्राह्मणः सर्वभूतानामतिथिः प्रसृताग्रभुक्}
{विप्रा यदधिगच्छन्ति भिक्षमाणा गृहं सदा}


\twolineshloka
{सत्कताश्च निवर्तन्ते तदतीव प्रवर्धते}
{महाभागे कुले प्रेत्य जन्म चाप्नोति भारत}


\twolineshloka
{दत्त्वा त्वन्नं नरो लोके तथा स्थानमनुत्तमम्}
{स्विष्टमृष्टान्नदायी तु स्वर्गे वसति सत्कृतः}


\twolineshloka
{अन्नं प्राणा नराणां हि सर्वमन्ने प्रतिष्ठिम्}
{अन्नदः पशुमान्पुत्री धनवान्भोगवानपि}


\twolineshloka
{प्राणवांश्चापि भवति रूपवांश्च तथा नृप}
{अन्नदः प्राणदो लोके सर्वदः प्रोच्यते तु सः}


\twolineshloka
{अन्नं हि दत्त्वाऽतिथये ब्राह्मणाय यथाविधि}
{प्रदाता सुखमाप्नोति दैवतैश्चापि पूज्यते}


\twolineshloka
{ब्राह्मणो हि महद्भूतं क्षेत्रभूतं युधिष्ठिर}
{उप्यते तत्रि यद्बीजं तद्धि पुण्यफलं महत्}


\twolineshloka
{प्रत्यक्षं प्रीतिजननं भोक्तुर्दातुर्भवत्युत}
{सर्वाण्यन्यानि दानानि परोक्षफलवन्त्युत}


\twolineshloka
{अन्नाद्धि प्रसवं यान्ति रतिरन्नाद्धि भारत}
{धर्मार्थावन्नतो विद्धि रोगनाशं तथाऽन्नतः}


\twolineshloka
{अन्नं ह्यमृतमित्याह पुरा कल्पे प्रजापतिः}
{7-98-3ab अन्नं भुवंदिवं खं च सर्वमन्ने प्रतिष्ठितम्}


\twolineshloka
{अन्नप्रणाशे भिद्यन्ते शरीरे पञ्च धातवः}
{बलं बलवतोपीह प्रणश्यत्यन्नहानितः}


\twolineshloka
{आवाहाश्च विवाहाश्च यज्ञाश्चान्नमृते तथा}
{निवर्तन्ते नरश्रेष्ठ ब्रह्म चात्र प्रलीयते}


\twolineshloka
{अन्नतः सर्वमेतद्धि यत्किञ्चित्स्थाणु जङ्गमम्}
{त्रिषु लोकेषु धर्मार्थमन्नं देयमतो बुधैः}


\twolineshloka
{अन्नदस्य मनुष्यस्य बलमोजो यशांसि च}
{कीर्तिश्च वर्धते शश्वत्त्रिषु लोकेषु पार्थिव}


\twolineshloka
{मेघेषूर्ध्वं सन्निधत्ते प्राणानां पवनः पतिः}
{तच्च मेघगतं वारि शक्रो वर्षति भारत}


\twolineshloka
{आदत्ते च रसान्भौनानादित्यः स्वगभस्तिभिः}
{वायुरादित्यतस्तांश्च रसान्देवः प्रवर्षति}


\twolineshloka
{तद्यदा मेघतो वारि पतितं भवति क्षितौ}
{तदा वसुमती देवी स्निग्धा भवति भारत}


\twolineshloka
{ततः सस्यानि रोहन्ति येन वर्तयते जगत्}
{मांसमेदोस्थिशुक्राणां प्रादुर्भावस्ततः पुनः}


\twolineshloka
{सम्भवन्ति ततः शुक्रात्प्राणिनः पृथिवीपते}
{अग्नीषोमौ हि तच्छुक्रं सृजतः पुष्यतश्च ह}


\twolineshloka
{एवमन्नाद्धि सूर्यश्च पवनः शक्रमेव च}
{एक एव स्मृतो राशिस्ततो भूतानि जज्ञिरे}


\threelineshloka
{प्राणान्ददाति भूतानां तेजश्च भरतर्षभ}
{गृहमभ्यागतायाथ यो दद्यादन्नमर्थिने ॥भीष्म उवाच}
{}


\twolineshloka
{नारदेनैवमुक्तोऽहमन्नदानं सदा नृप}
{अनसूयुस्त्वमप्यन्नं तस्माद्देहि गतज्वरः}


\twolineshloka
{दत्त्वाऽन्नं विधिवद्राजन्विप्रेभ्यस्त्वमपि प्रभो}
{यथावदनुरूपेभ्यस्ततः स्वर्गमवाप्स्यसि}


\twolineshloka
{अन्नदानां हि ये लोकास्तांस्त्वं शृणु जनाधिप}
{भवनानि प्रकाशन्ते दिवि तेषां महात्मनाम्}


\twolineshloka
{नानासंस्थानि रूपाणि नानास्तम्भान्वितानि च}
{चन्द्रमण्डलशुभ्राणि किंकिणीजालवन्ति च}


\twolineshloka
{तरुणादित्यवर्णानि स्थावराणि चराणि च}
{अनेकशतभौमानि सान्तर्जलचराणि च}


\twolineshloka
{वैदूर्यार्कप्रकाशानि रौप्यरुक्ममयानि च}
{सर्वकामफलाश्चापि वृक्षा भवनसंस्थिताः}


\twolineshloka
{वाप्यो वीथ्यः सभाः कूपा दीर्घिकाश्चैव सर्वशः}
{घोषवन्ति च यानानि युक्तान्यथ सहस्रशः}


\twolineshloka
{भक्ष्यभोज्यमयाः शैला वासांस्याभरणानि च}
{क्षीरं स्रवन्ति सरितस्तथा चैवान्नपर्वताः}


\twolineshloka
{प्रासादाः पाण्डुराभ्राभाः शय्याश्च कनकोञ्ज्वलाः}
{तान्यन्नदाः प्रपद्यन्ते तस्मादन्नप्रदो भव}


\twolineshloka
{एते लोकाः पुण्यकृता अन्नदानां महात्मनाम्}
{तस्मादन्नं प्रयत्नेन दातव्यं मानवैर्भुवि}


\chapter{अध्यायः ९९}
\threelineshloka
{श्रुतं मे भवतो वाक्यमन्नदानस्य यो विधिः}
{नक्षत्रयोगस्येदानीं दानकल्पं ब्रवीहि मे ॥भीष्म उवाच}
{}


\twolineshloka
{अत्राप्युदाहरन्तीममितिहासं पुरानम्}
{देवक्याश्चैव संवादं समर्षेर्नारदस्य च}


\twolineshloka
{द्वारकामनुसम्प्राप्तं नारदं देवदर्शनम्}
{पप्रच्छेदं वचः प्रश्नं देवकी धर्मदर्शिनी}


\threelineshloka
{तस्याः सम्पृच्छमानाया देवर्षिर्नारदस्ततः}
{आचष्ट विधइवत्सर्वं तच्छृणुष्व विशाम्पते ॥नारद उवाच}
{}


\twolineshloka
{कृत्तिकासु महाभागे पायसेन ससर्पिषा}
{सन्तर्प्य ब्राह्मणान्साधूँल्लोकानाप्नोत्यनुत्तमान्}


\twolineshloka
{रोहिण्यां प्रसृतैर्मार्गैर्मांसैरन्नेन सर्पिषा}
{पयोऽन्नपानं दातव्यमनृणार्थं द्विजातये}


\twolineshloka
{दोग्ध्रीं दत्त्वा सवत्सां तु नक्षत्रे सोमदैवते}
{गच्छन्ति मानुपाल्लोकात्स्वर्गलोकमनुत्तमम्}


\twolineshloka
{आर्द्रायां कृसरं दत्त्वा तिलभिश्रमुपोषितः}
{नरस्तरति दुर्गाणि क्षुरधारांश्च पर्वतान्}


\twolineshloka
{पूपान्पुनर्वसौ दत्त्वा तथैवान्नानि शोभने}
{यशस्वी रूपसम्पन्नो बह्वन्नो जायते कुले}


\twolineshloka
{पुण्येण कनकं दत्त्वा कृतं वाऽकृतमेव च}
{अनालोकेषु लोकेषु सोमवत्स विराजते}


\twolineshloka
{आश्लेषायां तु यो रूप्यमृषभं वा प्रयच्छति}
{स सर्पभयनिर्मुक्तः सम्भवानधितिष्ठति}


\twolineshloka
{मघासु तिलपूर्णानि वर्धमानानि मानवः}
{प्रदाय पुत्रपशुमानिह प्रेत्य च मोदते}


\twolineshloka
{फल्गुनीपूर्वसमये ब्राह्मणानामुपोषितः}
{भक्ष्यान्फाणितसंयुक्तान्दत्त्वा सौभाग्यमृच्छति}


\twolineshloka
{घृतक्षीरसमायुक्तं विधिवत्षष्टिकौदनम्}
{उत्तराविषये दत्त्वा स्वर्गलोके महीयते}


\twolineshloka
{यद्यत्प्रदीयते दानमुत्तराविषये नरैः}
{महाफलमनन्तं तद्भवतीति विनिश्चयः}


\twolineshloka
{हस्ते हस्तिरथं दत्त्वा चतुर्युक्तमुपोषितः}
{प्राप्नोति परमाँल्लोकान्पुण्यकामसमन्वितान्}


\twolineshloka
{चित्रायां वृषभं दत्त्वा पुण्यगन्धांश्च भारत}
{चरन्त्यप्सरसां लोके रमन्ते नन्दने तथा}


\twolineshloka
{स्वात्यामथ धनं दत्त्वा यदिष्टतममात्मनः}
{iप्राप्नोति लोकान्स शुभानिह चैव महद्यशः}


\threelineshloka
{विशाखायामनड्वाहं धेनुं दत्त्वा च दुग्धदाम्}
{सप्रासङ्गं च शकटं सधान्यं वस्त्रसंयुतम्}
{}


\twolineshloka
{पितॄन्देवांश्च प्रीणाति प्रेत्य चानन्त्यमश्नुते}
{न च दुर्गाण्यवाप्नोति स्वर्गलोकं च गच्छति}


\twolineshloka
{दत्त्वा यथोक्तं विप्रेभ्यो वृत्तिमिष्टां स विन्दति}
{नरकादींश्च संक्लेशान्नाप्नोतीति विनिश्चयः}


\twolineshloka
{अनुराधासु प्रावरं वरान्नं समुपोषितः}
{दत्त्वा युगशतं चापि नरः स्वर्गे महीयते}


\twolineshloka
{कालशाकं तु विप्रेभ्यो दत्त्वा मर्त्यः समूलकम्}
{ज्येष्ठायामृद्धिमिष्टां वै गतिमिष्टां स गच्छति}


\threelineshloka
{मूले मूलफलं दत्त्वा ब्राह्मणेभ्यः समाहितः}
{पितॄन्प्रीणयते चापि गतिमिष्टां च गच्छति}
{}


\twolineshloka
{अथ पूर्वास्वषाढासु दधिपात्राण्युपोषितः}
{कुलवृत्तोपसम्पन्ने ब्राह्मणे वेदपारगे}


\twolineshloka
{प्रदाय जायते प्रेत्य कुले सुबहुगोधने}
{उदमन्थं ससर्पिष्कं प्रभूतमधुफाणितम्}


\threelineshloka
{दत्त्वोत्तरास्वषाढासु सर्वकामानवाप्नुयात्}
{दुग्धं त्वभिजिते योगे दत्त्वा मधुघृतप्लुतम्}
{धर्मनित्यो मनीषिभ्यः स्वर्गलोके महीयते}


\twolineshloka
{श्रवणे कम्बलं दत्त्वा वस्त्रान्तरितमेव वा}
{श्वेतेन याति यानेन स्वर्गलोकानसंवृतान्}


\twolineshloka
{गोप्रयुक्तं धनिष्ठासु यानं दत्त्वा समाहितः}
{वस्त्रराशिधनं सद्यः प्रेत्य राज्यं प्रपद्यते}


\twolineshloka
{गन्धाञ्शतभिषग्योगे दत्त्वा सागरुचन्दनान्}
{प्राप्नोत्यप्सरसां सङ्घान्प्रेत्य गन्धांश्च शाश्वतान्}


\twolineshloka
{पूर्वप्रोष्ठपदायोगे राजमाषान्प्रदाय तु}
{सर्वभक्षफलोपेतः स वै प्रेत्य सुखी भवेत्}


\twolineshloka
{औरभ्रमुत्तरायोगे यस्तु मांसं प्रयच्छति}
{स पितॄन्प्रीणयति वै प्रेत्य चानन्त्यमश्नुते}


\twolineshloka
{कांस्योपदोहनां धेनु रेवत्यां यः प्रयच्छति}
{सा प्रेत्य कामानादाय दातारमुपतिष्ठति}


\twolineshloka
{रथमश्वसमायुक्तं दत्त्वाऽश्विन्यां नरोत्तमः}
{हस्त्यश्वरथसम्पन्ने वर्चस्वी जायते कुले}


\threelineshloka
{भरणीषु द्विजातिभ्यस्तिलधेनुं प्रदाय वै}
{गाः सुप्रभूताः प्राप्नोति नरः प्रेत्य यशस्तथा ॥भीष्म उवाच}
{}


\twolineshloka
{इत्येष लक्षणोद्देशः प्रोक्तो नक्षत्रयोगतः}
{देवक्या नारदेनेह सा स्नुषाभ्योऽब्रवीदिदम्}


\chapter{अध्यायः १००}
\twolineshloka
{सर्वान्कामान्प्रयच्छन्ति ये प्रयच्छन्ति काञ्चनम्}
{इत्येवं भगवानत्रिः पितामहसुतोऽब्रवीत्}


\twolineshloka
{पवित्रं शुच्यथायुष्यं पितृणामक्ष्यं च तत्}
{सुवर्णं मनुजेन्द्रेण हरिश्चन्द्रेण कीर्तितम्}


\twolineshloka
{पानीयपरमं दानं दानानां मनुरब्रवीत्}
{तस्मात्कूपांश्च वापीश्च तटाकानि च स्वानयेत्}


\twolineshloka
{सर्वं विनाशयेत्पापं पुरुषस्येह कर्मणः}
{कूपः प्रवृत्तपानीयः सुप्रवृत्तश्च नित्यशः}


\twolineshloka
{सर्वं तारयते वंशं यस्य खाते जलाशये}
{गावः पिबन्ति विप्राश्च साधवश्च नराः सदा}


\twolineshloka
{निदाघकाले पानीयं यस्य तिष्ठत्यवारितम्}
{स दुर्गं विषमं कृत्स्नं न कदाचिदवाप्नुते}


\twolineshloka
{बृहस्पतेर्भगवतः पूष्णश्चैव भगस्य च}
{अश्विनोश्चैव वह्नेश्च प्रीतिर्भवति सर्पिषा}


\twolineshloka
{परमं भेषजं ह्येतद्यज्ञानामेतदुत्तमम्}
{रसानामुत्तमं चैतत्फलानां चैतदुत्तमम्}


\twolineshloka
{फलकामो यशस्कामः पुष्टिकामश्च नित्यदा}
{घृतं दद्याद्द्विजातिभ्यः पुरुषः शुचिरात्मवान्}


\twolineshloka
{घृतं मासे आश्वयुजि विप्रभ्यो यः प्रयच्छति}
{तस्मै प्रयच्छतो रूपं प्रीतौ देवाविहाश्विनौ}


\twolineshloka
{पायसं सर्पिषा मिश्रं द्विजेभ्यो यः प्रयच्छति}
{गृहं तस्य न रक्षांसि धर्षयन्ति कदाचन}


\twolineshloka
{पिपासया न म्रियते सोपच्छन्दश्च जायते}
{न प्राप्नुयाच्च व्यसनं करकान्यः प्रयच्छति}


\twolineshloka
{प्रयतो ब्राह्मणाग्रे यः श्रद्धया परया युतः}
{उपस्पर्शनषड्भागं लभते पुरुषः सदा}


\twolineshloka
{यः साधनार्थं काष्ठानि ब्राह्मणेभ्यः प्रयच्छति}
{प्रतापनार्थं राजेन्द्र वृत्तवद्भ्यः सदा नरः}


\twolineshloka
{सिद्ध्यन्त्यर्थाः सदा तस्य कार्याणि विविधानि च}
{उपर्युपरि शत्रूणां वपुषा दीप्यते च सः}


\twolineshloka
{भगवांश्चापि सम्प्रतो वह्निर्भवति नित्यशः}
{न तं त्यजन्ति पशवः सङ्ग्रामे च जयत्यपि}


\twolineshloka
{पुत्राञ्श्रियं च लभते यश्छत्रं सम्प्रच्छति}
{न चक्षुर्व्याधिं लभते यज्ञभागमथाश्नुते}


\threelineshloka
{निदाघकाले वर्षे वा यश्छत्रं सम्प्रयच्छति}
{नास्य कश्चिन्मनोदाहः कदाचिदपि जायते}
{कृच्छ्रात्स विषमाच्चैव क्षिप्रं मोक्षमवाप्नुते}


\twolineshloka
{प्रदानं सर्वदानानां शकटस्य विशाम्पते}
{एवमाह महाभागः शाण्डिल्यो भगवानृषिः}


\chapter{अध्यायः १०१}
\threelineshloka
{दह्यमानाय विप्राय यः प्रयच्छत्युपानहौ}
{यत्फलं तस्य भवति तन्मे ब्रूहि पितामह ॥भीष्म उवाच}
{}


\twolineshloka
{उपानहौ प्रयच्छेद्यो ब्राह्मणेभ्यः समाहितः}
{मर्दते कण्टकान्सर्वान्विषमान्निस्तरत्यपि}


\twolineshloka
{स शत्रूणामुपरि च सन्तिष्ठति युधिष्ठिर}
{यानं चाश्वतरीयुक्तं तस्य शुभ्रं विशाम्पते}


\threelineshloka
{उपतिष्ठति कौन्तेय रौप्यकाञ्चनभूषितम्}
{शकटं दम्यसंयुक्तं दत्तं भवति चैव हि ॥युधिष्ठिर उवाच}
{}


\threelineshloka
{यत्फलं तिलदाने च भूमिदाने च कीर्तितम्}
{गोदाने चान्नदाने च भूयस्तद्ब्रूहि कौरव ॥भीष्म उवाच}
{}


\twolineshloka
{शृणुष्व मम कौन्तैय तिलदानस्य यत्फलम्}
{निशम्य च यथान्यायं प्रयच्छ कुरुसत्तम}


\twolineshloka
{पितॄणां प्रथमं भोज्यं तिलाः सृष्टाः स्वयंभुवा}
{तिलदानेन वै तस्मात्पितृपक्षः प्रमोदते}


\twolineshloka
{माघमासे तिलान्यस्तु ब्राह्मणेभ्यः प्रयच्छति}
{सर्वसत्वसमाकीर्णं नरकं स न पश्यति}


\twolineshloka
{सर्वसत्रैश्च यजते यस्तिलैर्यजते पितॄन्}
{न चाकामेन दातव्यं तिलैः श्राद्धं कदाचन}


\twolineshloka
{महर्षेः कश्यपस्यैते गात्रेभ्यः प्रसृतास्तिलाः}
{ततो दिव्यं गता भावं प्रदानेषु तिलाः प्रभो}


\twolineshloka
{पौष्टिका रूपदाश्चैव तथा पापविनाशनाः}
{तस्मात्सर्वप्रदानेभ्यस्तिलदानं विशिष्यते}


\twolineshloka
{आपस्तम्बश्च मेधावी शङ्खश्च लिखितस्तथा}
{महर्षिर्गौतमश्चापि तिलदानैर्दिवं गताः}


\twolineshloka
{तिलहोमरता विप्राः सर्वे संयतमैथुनाः}
{समा गव्येन हविषा प्रवृत्तिषु च संस्थिताः}


\twolineshloka
{सर्वेषामिति दानानां तिलदानं विशिष्यते}
{अक्षयं सर्वदानानां तिलदानमिहोच्यते}


\twolineshloka
{उच्छिन्ने तु पुरा हव्ये कुशिकर्षिः परन्तपः}
{तिलैरग्नित्रयं हुत्वा प्राप्तवान्गतिमुत्तमाम्}


\twolineshloka
{इति प्रोक्तं कुरुश्रेष्ठ तिलदानमनुत्तमम्}
{विधानं येन विधिना तिलानामिह शस्यते}


\twolineshloka
{अत ऊर्ध्वं निबोधेदं देवानां यष्टुमिच्छताम्}
{समागमे महाराज ब्रह्मणा वै स्वयंभुवा}


\threelineshloka
{देवाः समेत्य ब्रह्माणं भूमिभागे यियश्रवः}
{शुभं देशमयाचन्त यजेम इति पार्थिव ॥देवा ऊचुः}
{}


\twolineshloka
{भगवंस्त्वं प्रभुर्भूमेः सर्वस्य त्रिदिवस्य च}
{यजेम हि महाभाग यज्ञं भवदनुज्ञया}


\fourlineindentedshloka
{नाननुज्ञातभूमिर्हि यज्ञस्य फलमश्नुते}
{त्वं हि सर्वस्य जगतः स्थावरस्य चरस्य च}
{प्रभुर्भवसि तस्मात्त्वं समनुज्ञातुमर्हसि ॥ब्रह्मोवाच}
{}


\threelineshloka
{ददानि मेदिनीभागं भवद्भ्योऽहं सुरर्षभाः}
{यस्मिन्देशे करिष्यध्वं यज्ञान्काश्यपनन्दनाः ॥दैवा ऊचुः}
{}


\twolineshloka
{भगवन्कृतकामाः स्म यक्ष्महे स्वाप्तदक्षिणैः}
{इमं तु देशं मुनयः पर्युपासन्ति नित्यदा}


\twolineshloka
{ततोऽगस्त्यश्च कण्वश्च भृगुरत्रिर्वृषाकपिः}
{असितो देवलश्चैव देवयज्ञमुपागमन्}


\twolineshloka
{ततो देवा महात्मान ईजिरे यज्ञमच्युतम्}
{तथा समापयामासुर्यथाकालं सुरर्षभाः}


\threelineshloka
{त इष्टयज्ञास्त्रिदशा हिमवत्यचलोत्तमे}
{षष्ठमंशं क्रतोस्तस्य भूमिदानं प्रचक्रिरे}
{}


\twolineshloka
{प्रादेशमात्रं भूमेस्तु यो दद्यादनुपस्कृतम्}
{न सीदति स कृच्छ्रेषु न च दुर्गाण्यवाप्नुते}


\twolineshloka
{शीतवातातपसहां यागभूमिं सुसंस्कृताम्}
{प्रदाय सुरलोकस्थः पुण्यान्तेऽपि न चाल्यते}


\twolineshloka
{मुदितो वसति प्राज्ञः शक्रेण सह पार्थिव}
{पतिश्रयप्रदानाच्च सोऽपि स्वर्गे महीयते}


\twolineshloka
{अध्यापककुले जातः श्रोत्रियो नियतेन्द्रियः}
{गृहे यस्य वसेत्तुष्टः प्रधानं लोकमश्नुते}


\twolineshloka
{तथा गवार्थे शरणं शीतवर्षसहं दृढम्}
{आसप्तमं तारयति कुलं भरतसत्तम}


\twolineshloka
{क्षेत्रभूमिं ददल्लोके शुभां श्रियमवाप्नुयात्}
{रत्नभूमिं प्रदद्यात्तु कुलवंशं प्रवर्धयेत्}


\twolineshloka
{न चोषरां न निर्दग्धां महीं दद्यात्कथञ्चन}
{न श्मशानपरीतां च न च पापनिषेविताम्}


\twolineshloka
{पारक्ये भूमिदेशे तु पितॄणां निर्वपेत्तु यः}
{तद्भूमिं वाऽपि पितृभिः श्राद्धकर्म विहन्यते}


\twolineshloka
{तस्मात्क्रीत्वा महीं दद्यात्स्वल्पामपि विचक्षणः}
{पिण्डः पितृभ्यो दत्तो वै तस्यां भवति शाश्वतः}


\twolineshloka
{अटवी पर्वताश्चैव नद्यस्तीर्थानि यानि च}
{सर्वाण्यस्वामिकान्याद्दुर्न हि तत्र परिग्रहः}


\twolineshloka
{इत्येतद्भूमिदानस्य फलमुक्तं विशाम्पते}
{अतः परं तु गोदानं कीर्तयिष्यामि तेऽनघ}


\twolineshloka
{गावोऽधिकास्तपस्विभ्यो यस्मात्सर्वेभ्य एव च}
{तस्मान्महेश्वरो देवस्तपस्ताभिः सहास्थितः}


\twolineshloka
{ब्राह्मे लोके वसन्त्येताः सोमेन सह भारत}
{यां तां ब्रह्मर्षयः सिद्धाः प्रार्थयन्ति परां गतिम्}


\twolineshloka
{पयसा हविषा दध्ना शकृता चाथ चर्मणा}
{अस्थिभिश्चोपकुर्वन्ति शृङ्गैर्वालैश्च भारत}


\twolineshloka
{नासां शीतातपौ स्यातां सदैताः कर्म कुर्वते}
{न वर्षविषयं वाऽपि दुःखमासां भवत्युत}


\twolineshloka
{ब्राह्मणैः सहिता यान्ति तस्मात्पारमकं पदम्}
{एकं गोब्राह्मणं तस्मात्प्रवदन्ति मनीषिणः}


\threelineshloka
{रन्तिदेवस्य यज्ञे ताः पशुत्वेनोपकल्पिताः}
{अतश्चर्मण्वती राजन्गोचर्मभ्यः प्रवर्तिता}
{पशुत्वाच्च विनिर्मुक्ताः प्रदानायोपकल्पिताः}


\twolineshloka
{ता इमा विप्रमुख्येभ्यो यो ददाति महीपते}
{निस्तरेदापदं कृच्छ्रां विषमस्थोऽपि पार्थिव}


\twolineshloka
{गवां सहस्रदः प्रेत्य नरकं न प्रपद्यते}
{सर्वत्र विजयं चापि लभते मनुजाधिप}


\twolineshloka
{अमृतं वै गवां क्षीरमित्याह त्रिदशाधिपः}
{तस्माद्ददाति यो धेनुममृतं स प्रयच्छति}


\twolineshloka
{अग्नीनामव्ययं ह्येतद्धौम्यं वेदविदो विदुः}
{तस्माद्ददाति यो धेनुं स हौम्यं सम्प्रयच्छति}


\twolineshloka
{स्वर्गो वै मूर्तिमानेष वृषभं यो गवां पतिम्}
{विप्रे गुणयुते दद्यात्स वै स्वर्गे महीयते}


\twolineshloka
{प्राणा वै प्राणिनामेते प्रोच्यन्ते भरतर्षभ}
{तस्माद्ददाति यो धेनुं प्राणानेष प्रयच्छति}


\twolineshloka
{गावः शरण्या भूतानामिति वेदविदो विदुः}
{तस्माद्ददाति यो धेनुं शरणं सम्प्रयच्छति}


\threelineshloka
{न वधार्थं प्रदातव्या न कीनाशे न नास्तिके}
{गोजीविने न दातव्या तथा गौर्भरतर्षभ}
{`गोरसानां न विक्रेतू रसं च यजनस्य च ॥'}


\twolineshloka
{ददत्स तादृशानां वै नरो गां पापकर्मणाम्}
{अक्षयं नरकं यातीत्येवमाहुर्महर्षयः}


\twolineshloka
{न कृशां नापवत्सां वा वन्ध्यां रोगान्वितां तथा}
{न व्यङ्गां न परिश्रान्तां दद्याद्गां ब्राह्मणाय वै}


\twolineshloka
{दशगोसहस्रदः सम्यक् शक्रेण सह मोदते}
{अक्षयाँल्लभते लोकान्नरः शतसहस्रशः}


\twolineshloka
{इत्येतद्गोप्रदानं च तिलदानं च कीर्तितम्}
{तथा भूमिप्रदानं च शृणुष्वान्ने च भारत}


\twolineshloka
{अन्नदानं प्रधानं हि कौन्तेय परिचक्षते}
{अन्नस्य हि प्रदानेन रन्तिदेवो दिवं गतः}


\twolineshloka
{श्रान्ताय क्षुधितायान्नं यः प्रयच्छति भूमिप}
{स्वायंभुवं महात्स्थानं स गच्छति नराधिप}


\twolineshloka
{न हिरण्यैर्न वासोभिर्नान्यदानेन भारत}
{प्राप्नुवन्ति नराः श्रेयो यथा ह्यन्नप्रदाः प्रभो}


\twolineshloka
{अन्नं वै प्रथमं द्रव्यमन्नं श्रीश्च परा मता}
{अन्नात्प्राणः प्रभवति तेजो वीर्यं बलं तथा}


\twolineshloka
{सद्यो ददाति यश्चान्नं सदैकाग्रमना नरः}
{न स दुर्गाण्यवाप्नोतीत्येवमाह पराशरः}


\twolineshloka
{अर्चयित्वा यथान्यायं देवेभ्योऽन्नं निवेदयेत्}
{यदन्ना हि नरा राजंस्तदन्नास्तस्य देवतः}


\twolineshloka
{कौमुद्यां शुक्लपक्षे तु योऽन्नदानं करोत्युत}
{स सन्तरति दुर्गाणि प्रेत्य चानन्त्यमश्नुते}


\twolineshloka
{अभुक्त्वाऽतिथेये चान्नं प्रयच्छेद्यः समाहितः}
{स वै ब्रह्मविदां लोकान्प्राप्नुयाद्भरतर्षभ}


\twolineshloka
{सुकृच्छ्रामापदं प्राप्तश्चान्नदः पुरुषस्तरेत्}
{पापं तरति चैवेह दुष्कृतं चापकर्षति}


\twolineshloka
{इत्येतदन्नदानस्य तिलदानस्य चैव ह}
{भूमिदानस्य च फलं गोदानस्य च कीर्तितम्}


\chapter{अध्यायः १०२}
\twolineshloka
{श्रुतं दानफलं तात यत्त्वया परिकीर्तितम्}
{अन्नदानं विशेषेण प्रशस्तमिह भारत}


\threelineshloka
{पानीयदानमेवैतत्कथं चेह महाफलम्}
{इत्येतच्छ्रोतुमिच्छामि विस्तरेण पितामह ॥भीष्म उवाच}
{}


\threelineshloka
{हन्त ते वर्तयिष्यामि यथावद्भरतर्षभ}
{गदतस्तन्ममाद्येह शृणु सत्यपराक्रम}
{पानीयदानात्प्रभृति सर्वं वक्ष्यामि तेऽनघ}


\twolineshloka
{यदन्नं यच्च पानीयं सम्प्रदायाश्नुते फलम्}
{न ताभ्यां परमं दानं किञ्चिदस्तीति मे मनः}


\twolineshloka
{अन्नात्प्राणभृतस्तात प्रवर्तन्ते हि सर्वशः}
{तस्मादन्नं परं लोके सर्वदानेषु कथ्यते}


\twolineshloka
{अन्नाद्बलं च तेजश्च प्राणिनां वर्धते सदा}
{अन्नादानमतस्तस्माच्छ्रेष्ठमाह प्रजापतिः}


\twolineshloka
{सावित्र्या ह्यपि कौन्तेय श्रूयते वचनं शुभम्}
{यच्चेदं नान्यथा चैतद्देव सत्रे महामखे}


\twolineshloka
{अन्ने दत्ते नरेणेह प्राणा दत्ता भवन्त्युत}
{प्राणदानाद्धि परमं न दानमिह विद्यते}


\twolineshloka
{श्रुतं हि ते महाबाहो लोमशस्यापि तद्वचः}
{प्राणान्दत्त्वा कपोताय यत्प्राप्तं शिबिना पुरा}


\twolineshloka
{यां गतिं लभते दत्त्वा द्विजस्यान्न विशाम्पते}
{ततो विशिष्टां गच्छन्ति प्राणदा इति नः श्रुतं}


\twolineshloka
{अन्नं चापि प्रभवति पानीयात्कुरुसत्तम}
{नीरजातेन हि विना न किञ्चित्सम्प्रवर्तते}


\twolineshloka
{नीरजातश्च भगवान्सोमो ग्रहगणेश्वरः}
{अमृतं च सुधा चैव सुधा चैवामृतं तथा}


\twolineshloka
{अन्नौषध्यो महाराज वीरुधश्च जलोद्भवाः}
{यतः प्राणभृतां प्राणाः सम्भवन्ति विशाम्पते}


\twolineshloka
{देवानाममृतं ह्यन्नं नागानां च सुधा तथा}
{पितॄणां च स्वधा प्रोक्ता पशूनां चापि वीरुधः}


\twolineshloka
{अन्नमेव मनुष्याणां प्राणानाहुर्मनीषिणः}
{तच्च सर्वं नरव्याघ्र पानीयात्सम्प्रवर्तते}


\twolineshloka
{तस्मात्पानीयदानाद्वै न परं विद्यते क्वचित्}
{तच्च दद्यान्नरो नित्यं यदीच्छेद्भूतिमात्मनः}


\twolineshloka
{धन्यं यशस्यमायुष्यं जलदानमिहोच्यते}
{शत्रूंश्चाप्यधि कौन्तेय सदा तिष्ठति तोयदः}


\twolineshloka
{सर्वकामानवाप्नोति कीर्तिं चैव हि शाश्वतीम्}
{प्रेत्य चानन्त्यमश्नाति पापेभ्यश्च प्रमुच्यते}


\twolineshloka
{तोयदो मनुजव्याघ्र स्वर्गं गत्वा महाद्युते}
{अक्षयान्समवाप्नोति लोकानित्यब्रवीन्मनुः}


\chapter{अध्यायः १०३}
\threelineshloka
{तिलानां कीदृशं दानमथ दीपस्य चैव हि}
{अन्नानां वाससा चैव भूय एव ब्रवीहि मे ॥भीष्म उवाच}
{}


\twolineshloka
{अत्राप्युदाहरन्तीममितिहासं पुरातनम्}
{ब्राह्मणस्य च संवादं यमस्य च युधिष्ठिर}


\twolineshloka
{मध्यदेशे महान्ग्रामो ब्राह्मणानां बभूव ह}
{गङ्गायमुनयोर्मध्ये यामुनस्य गिरेरधः}


\twolineshloka
{पर्णशालेति विख्यातो रमणीयो नराधिप}
{विद्वांशस्तत्र भूयिष्ठा ब्राह्मणाश्चावसंस्तथा}


\twolineshloka
{अथ प्राह यमः कञ्चित्पुरुषं कृष्णवाससम्}
{रक्ताक्षमूर्ध्वरोमाणं काकजङ्घाक्षिनासिकम्}


\twolineshloka
{गच्छ त्वं ब्राह्मणग्रामं ततो गत्वा तमानय}
{अगस्त्यं गोत्रतश्चापि नामतश्चापि शर्मिणम्}


\twolineshloka
{शमे निविष्टं विद्वांसमध्यापकमनावृतम्}
{मा चान्यमानयेथास्त्वं सगोत्रं तस्य पार्श्वतः}


\threelineshloka
{स हि तादृग्गुणस्तेन तुल्योऽध्ययनजन्मना}
{अपत्येषु तथा वृत्ते समस्तेनैव धीमता}
{तमानय यथोद्दिष्टं पूजा कार्या हि तस्य मे}


\twolineshloka
{स गत्वा प्रतिकूलं तच्चकार यमशासनम्}
{तमाक्रम्यानयामास प्रतिषिद्धो यमेन यः}


\twolineshloka
{तस्मै यमः समुत्थाय पूजा कृत्वा च वीर्यवान्}
{प्रोवाच नीयतामेष सोऽन्य आनीयतामिति}


\fourlineindentedshloka
{एवमुक्ते तु वचने धर्मराजेन स द्विजः}
{उवाच धर्मराजानं निर्विण्णोऽध्ययनेन वै}
{यो मे कालो भवेच्छेषस्तं वसेयमिहाच्युत ॥यम उवाच}
{}


\twolineshloka
{नाहं कालस्य विहितं प्राप्नोमीह कथञ्चन}
{यो हि धर्मं चरति वै तं तु जानामि केवलम्}


\threelineshloka
{गच्छ विप्र त्वमद्यैव आलयं स्वं महाद्युते}
{ब्रूहि सर्वं यथा स्वैरं करवाणि किमच्युत ॥ब्राह्मण उवाच}
{}


\threelineshloka
{शुद्धदानं च सुमहत्पुण्यं स्यात्तद्ब्रवीहि मे}
{सर्वस्य हि प्रमाणं त्वं त्रैलोक्यस्यापि सत्तम ॥यम उवाच}
{}


\twolineshloka
{शृणु तत्त्वेन विप्रर्पे प्रदानविधिमुत्तमम्}
{तिलाः परमकं दानं पुण्यं चैवेह शाश्वतम्}


\twolineshloka
{तिलाश्च सम्प्रदातव्या यथाशक्ति द्विजर्षभ}
{नित्यदानात्सर्वकामांस्तिला निर्वर्तयन्त्युत}


\twolineshloka
{तिलाञ्श्राद्धे प्रशंसन्ति दानमेतद्ध्यनुत्तमम्}
{तान्प्रयच्छस्व विप्रेभअयो विधिदृष्टेन कर्मणाः}


\twolineshloka
{वैशाख्यां पौर्णमास्यां तु तिलान्दद्याद्द्विजातिषु}
{तिला भक्षयितव्याश्च सदा त्वालम्भं च तैः}


\twolineshloka
{कार्यं सततमिच्छद्भिः श्रेयः सर्वात्मना गृहे}
{तथाऽऽपः सर्वदा देयाः पेयाश्चैव न संशयः}


\twolineshloka
{पुष्करिण्यस्तटाकानि कूपांश्चैवात्र खानयेत्}
{एतत्सुदुर्लभतरमिह लोके द्विजोत्तम}


\threelineshloka
{आपो नित्यं प्रदेयास्ते पुण्यं ह्येतदनुत्तमम्}
{प्रपाश्च कार्या दानार्थं नित्यं ते द्विजसत्तम}
{भुक्तेऽप्यथ प्रदेयं तु पानीयं वै विशेषतः}


\twolineshloka
{`पानीयाभ्यर्थिनं दृष्ट्वा प्रीत्या दत्त्वा त्वरान्वितः}
{वस्त्रे तन्तुप्रमाणेन दीपे निमिषवत्सरम्}


\threelineshloka
{गवां रोमप्रमाणेन स्वर्गभोगमुपाश्नुते}
{जलबिन्दुप्रमाणेन तदेतान्युपवर्तय ॥' ॥भीष्म उवाच}
{}


\twolineshloka
{इत्युक्ते स तदा तेन यमदूतेन वै गृहान्}
{नीतश्च कारयामास सर्वं तद्यमशासनम्}


\twolineshloka
{नीत्वा तं यमदूतोऽपि गृहीत्वा शर्मिणं तदा}
{ययौ स धर्मराजाय न्यवेदयत चापि तम्}


\twolineshloka
{तं धर्मराजो धर्मज्ञं पूजयित्वा प्रतापवान्}
{कृत्वा च संविदं तेन विससर्ज यथागतम्}


\twolineshloka
{तस्यापि च यमः सर्वमुपदेशं चकार ह}
{प्रेत्यैत्य च ततः सर्वं चकारोक्तं यमेन तत्}


\twolineshloka
{तथा प्रशंसने दीपान्यमः पितृहितेप्सया}
{तस्माद्दीपप्रदो नित्यं सन्तारति वै पितॄन्}


\twolineshloka
{दातव्याः सततं दीपास्तस्माद्भरतसत्तम}
{देवतानां पितॄणां च चक्षुष्यं चात्मनां विभो}


\twolineshloka
{रत्नदानं च सुमहत्पुण्यमुक्तं जनाधिप}
{यस्तान्विक्रीय यजते ब्राह्मणो ह्यभयंकरम्}


\twolineshloka
{यद्वै ददाति विप्रेभ्यो ब्राह्मणः प्रतिगृह्य वै}
{उभयोः स्यात्तदक्षय्यं दातुरादातुरेव च}


\twolineshloka
{यो ददाति स्थितः स्थित्यां तादृशाय प्रतिग्रहम्}
{उभयोरक्षयं धर्मं तं मनुः प्राह धर्मवित्}


\twolineshloka
{वाससां सम्प्रदानेनि स्वदारनिरतो नरः}
{सुवस्त्रश्च सुवेषश्च भवतीत्यनुशुश्रुम}


\twolineshloka
{गावः सुवर्णं च तथा तिलाश्चैवानुवर्णिताः}
{बहुशः पुरुषव्याघ्र वेदप्रामाण्यदर्शनात्}


\twolineshloka
{विवाहांश्चैव कुर्वीत पुत्रानुत्पादयेत च}
{पुत्रलाभो हि कौरव्य सर्वलाभाद्विशिष्यते}


\chapter{अध्यायः १०४}
\twolineshloka
{भूय एव कुरुश्रेष्ठ दानानां विधिमुत्तमम्}
{कथयस्व महाप्राज्ञ भूमिदानं विशेषतः}


\twolineshloka
{पृथिवीं क्षत्रियो दद्याद्ब्राह्मणायेष्टिकर्मिणे}
{विधिवत्प्रतिगृह्णीयान्न त्वन्यो दातुमर्हति}


\threelineshloka
{सर्ववर्णैस्तु यच्छक्यं प्रदातुं फलकाङ्क्षिभिः}
{वेदे वा यत्समाख्यातं तन्मे व्याख्यातुमर्हसि ॥भीष्म उवाच}
{}


\twolineshloka
{तुल्यनामानि देयानि त्रीणि तुल्यफलानि च}
{सर्वकामफलानीह गावः पृथ्वी सरस्वती}


\twolineshloka
{यो ब्रूयाच्चापि शिष्याय धर्म्यां ब्राह्मीं सरस्वतीम्}
{पृथिवीगोप्रदानाभ्यां तुल्यं स फलमश्नुते}


\twolineshloka
{तथैव गाः प्रशंसन्ति न तु देयं ततः परम्}
{सन्निकृष्टफलास्ता हि लघ्वर्थाश्च युधिष्ठिर}


\twolineshloka
{मातरः सर्वभूतानां गावः सर्वसुखप्रदाः}
{वृद्धिमाकाङ्क्षता नित्यं गावः कार्याः प्रदक्षिणाः}


\twolineshloka
{सन्ताड्या न तु पादेन गवां मध्ये न च व्रजेत्}
{मङ्गलायतनं देव्यस्तस्मात्पूज्याः सदैव गाः}


\twolineshloka
{प्रचोदनं देवकृतं गवां कर्मसु वर्तताम्}
{पूर्वमेवाक्षरं चान्यदभिधेयं ततः परम्}


\twolineshloka
{प्रचारे वा निवाते वा बुधो नोद्वेजयेत गाः}
{तृषिता ह्यभिवीक्षन्त्यो नरं हन्युः सबान्धवम्}


\twolineshloka
{पितृसद्मानि सततं देवतायतनानि च}
{पूयन्ते शकृता यासां पूतं किमधिकं ततः}


\twolineshloka
{घासमुष्टिं परगवे दद्यात्संवत्सरं तु यः}
{अकृत्वा स्वयमाहारं व्रतं तत्सार्वकामिकम्}


\threelineshloka
{स हि पुत्रान्यशोऽर्थं च श्रियं चाप्यधिगच्छति}
{नाशयत्यशुभं चैव दुःस्वप्नं चाप्यपोहति ॥युधिष्ठिर उवाच}
{}


\threelineshloka
{देयाः किंलक्षणा गावः काश्चापि परिवर्जयेत्}
{कीदृशाय प्रदातव्या न देयाः कीदृशाय च ॥भीष्म उवाच}
{}


\twolineshloka
{असद्वृत्ताय पापाय लुब्धायानृतवादिने}
{हव्यकव्यव्यपेताय न देया गौः कथञ्चन}


\twolineshloka
{भिक्षवे बहुपुत्राय श्रोत्रियायाहिताग्नये}
{दत्त्वा दशगवां दाता लोकानाप्नोत्यनुत्तमान्}


\twolineshloka
{`जुहोति यद्भोजयति यद्ददाति गवां रसैः}
{सर्वस्यैवांशभाग्दाता तन्निमित्तं प्रवर्तितः ॥'}


\twolineshloka
{यश्चैव धर्मं कुरुते तस्य धर्मफलं च यत्}
{सर्वस्यैवांशभाग्दाता तन्निमित्तं प्रवृत्तयः}


\twolineshloka
{यश्चैनमुत्पादयते यश्चैनं त्रायते भयात्}
{यश्चास्य कुरुते वृत्तिं सर्वे ते पितरस्त्रयः}


\twolineshloka
{कल्मषं गुरुशुश्रूषा हन्ति मानो महद्यशः}
{अपुत्रतां त्रयः पुत्रा अवृत्तिं दश धेनवः}


\twolineshloka
{वेदान्तनिष्ठस्य बहुश्रुतस्यप्रज्ञानतृप्तस्य जितेन्द्रियस्य}
{शिष्टस्य दान्तस्य यतस्य चैवभूतेषु नित्यं प्रियवादिनश्च}


\threelineshloka
{यः क्षुद्भयाद्वै न विकर्म कुर्या-न्मृदुश्च शान्तौ ह्यतिथिप्रियश्च}
{शुभे पात्रे ये गुणा गोप्रदानेतावान्दोषो ब्राह्मणस्वापहारे}
{}


% Check verse!
सर्वावस्थं ब्राह्मणस्वापहारेदाराश्चैषां दूरतो वर्जनीयाः
\twolineshloka
{`विप्रदारे परिहृते तद्धनेऽपहृते च तु}
{परित्रायन्ति शक्तास्तु नमस्तेभ्यो मृताश्च ये}


\twolineshloka
{न पालयन्ति निहतान्ये तान्वैवस्वतो यमः}
{दण्डयन्भर्सयन्नित्यं निरयेभ्यो न मुञ्चति}


\twolineshloka
{तथा गवां परित्राणे पीडने च शुभाशुभम्}
{विप्रगोषु विशेषेण रक्षितेषु गृहेषु वा ॥'}


\chapter{अध्यायः १०५}
\twolineshloka
{अत्रैव कीर्त्यते सद्भिर्ब्राह्मणस्वाभिमर्शने}
{नृगेण सुमहत्कृच्छ्रुं यदवाप्तं कुरूद्वह}


\threelineshloka
{निविशन्त्यां पुरा पार्थ द्वारवत्यामिति श्रुतिः}
{अदृश्यत महाकूपस्तृणवीरुत्समावृतः}
{}


\twolineshloka
{प्रयत्नं तत्र कुर्वाणास्तस्मात्कूपाज्जलार्थिनः}
{श्रमेण महता युक्तास्तस्मिंस्तोये सुसंवृते}


\twolineshloka
{ददृशुस्ते महाकायं कृकलासमवस्थितम्}
{तस्य चोद्धरणे यत्नमकुर्वंस्ते सहस्रशः}


\twolineshloka
{प्रग्रहैश्चर्मपट्टैश्च तं बद्ध्वा पर्वतोपमम्}
{नाशक्नुवन्समुद्धर्तुं ततो जग्मुर्जनार्दनम्}


\twolineshloka
{खमावृत्योदपानस्य कृकलासः स्थितो महान्}
{तस्य नास्ति समुद्धर्तेत्येतत्कृष्णे न्यवेदयन्}


\twolineshloka
{स वासुदेवेन समुद्धृतश्चपृष्टश्च कामान्निजगाद राजा}
{नृगस्तदाऽऽत्मानमथो न्यवेदय-त्पुरातनं यज्ञसहस्रयाजिनम्}


\twolineshloka
{तथा ब्रुवाणं तु तमाह माधवःशुभं त्वया कर्म कृतं न पापकम्}
{कथं भवान्दुर्गतिमीदृशीं गतोनरेन्द्र तद्ब्रूहि किमेतदीदृशम्}


\twolineshloka
{शतं सहस्राणि गवां शतं पुनःपुनः शतान्यष्टशतायुतानि}
{नृप द्विजेभ्यः क्व नु तद्गतं तव}


\twolineshloka
{नृगस्ततोऽब्रवीत्कृष्णं ब्राह्मणस्याग्निहोत्रिणः}
{प्रोषितस्य परिभ्रष्टा गौरेका मम गोधने*****}


\twolineshloka
{गवां सहस्रे संख्याता तदा सा पशुपैर्मम}
{सा ब्राह्मणाय मे दत्ता प्रेत्यार्थमभिकाङ्क्षता}


\twolineshloka
{अपश्यत्परिमार्गंश्च तां गां परगृहे द्विजः}
{ममेयमिति चोवाच ब्राह्मणो यस्य साऽभवत्}


\twolineshloka
{तावुभौ समनुप्राप्तौ विवदन्तौ भृशज्वरौ}
{भवान्दाता भवान्हर्तेत्यथ तौ मामवोचताम्}


\twolineshloka
{शतेन शतसङ्ख्येन गवां विनिमयेन वै}
{याचे प्रतिग्रहीतारं स तु मामब्रवीदिदम्}


\twolineshloka
{देशकालोपसम्पन्ना दोग्ध्री शान्ताऽतिवत्सला}
{स्वादुक्षीरप्रदा धन्या मम नित्यं निवेशने}


\twolineshloka
{कृशं च भरते सा गौर्मम पुत्रमपस्तनम्}
{न सा शक्या मया दातुमित्युक्त्वा स जगाम ह}


\threelineshloka
{ततस्तमपरं विप्रं याचे विनिमयेन वै}
{गवां शतसहस्रं हि तत्कृते गृह्यतामिति ॥ब्राह्मण उवाच}
{}


\twolineshloka
{न राज्ञां प्रतिगृह्णामि शक्तोऽहं स्वस्य मार्गणे}
{सैव गौर्दीयतां शीघ्रं ममेति मधुसूदन}


\twolineshloka
{रुक्ममश्वांश्च ददतो रजतस्यन्दनांस्तथा}
{न जग्राह ययौ चापि तदा स ब्राह्मणर्षभः}


\twolineshloka
{एतस्मिन्नेव काले तु चोदितः कालधर्मणा}
{पितृलोकमहं प्राप्य धर्मराजमुपागमम्}


\twolineshloka
{यमस्तु पूजयित्वा मां ततो वचनमब्रवीत्}
{नान्तः सङ्ख्यायते राजंस्तव पुण्यस्य कर्मणः}


\twolineshloka
{अस्ति चैव कृतं पापमज्ञानात्तदपि त्वया}
{चरस्व पापं पश्चाद्वा पूर्वं वा त्वं यथेच्छसि}


\twolineshloka
{रक्षितास्मीति चोक्तं ते प्रतिज्ञा चानृता तव}
{ब्राह्मणस्वस्य चादानं द्विविधस्ते व्यतिक्रमः}


\twolineshloka
{पूर्वं कृच्छं चरिष्येऽहं पश्चाच्छुभमिति प्रभो}
{धर्मराजं ब्रुवन्नेवं पतितोस्मि महीतले}


\twolineshloka
{अश्रौषं पतितश्चाहं यमस्योच्चैः प्रभाषतः}
{वासुदेवः समुद्धर्ता भविता ते जनार्दनः}


\twolineshloka
{पूर्णे वर्षसहस्रान्ते क्षीणे कर्मणि दुष्कृते}
{प्राप्लस्यसे शाश्वताँल्लोकाञ्जितान्स्वेनैव कर्मणा}


\twolineshloka
{कूपेऽऽत्मानमधःशीर्षमपश्यं पतितं च ह}
{तिर्यग्योनिमनुप्राप्तं न च मामजहात्स्मृतिः}


\twolineshloka
{त्वया तु तारितोऽस्म्यद्य किमन्यत्र तपोबलात्}
{अनुजानीहि मां कृष्ण गच्छेयं दिवमद्य वै}


\twolineshloka
{अनुज्ञातः स कृष्णेन नमस्कृत्य जनार्दनम्}
{विमानं दिव्यमास्थाय ययौ दिवमरिन्दमः}


\twolineshloka
{ततस्तस्मिन्दिवं याते नृगे भरतसत्तम}
{वासुदेव इमाञ्श्लोकाञ्जगाद कुरुनन्दन}


\twolineshloka
{ब्राह्मणस्वं न हर्तव्यं पुरुषेण विजानता}
{ब्राह्मणस्वं हृतं हन्ति नृगं ब्राह्मणगौरिव}


\twolineshloka
{सतां समागमः सद्भिर्नाफलः पार्थ विद्यते}
{विमुक्तं नरकात्पश्य नृगं साधुसमागमात्}


\twolineshloka
{प्रदानं फलवत्तत्र द्रोहस्तत्र तथाऽफलः}
{अपहारं गवां तस्माद्वर्जयेत युधिष्ठिर}


\chapter{अध्यायः १०६}
\threelineshloka
{दत्तानां फलसम्प्राप्तिं गवां प्रब्रूहि मेऽनघ}
{विस्तरेण महाबाहो न हि तृप्यामि कथ्यताम् ॥भीष्म उवाच}
{}


\twolineshloka
{अत्राप्युदाहरन्तीममितिहासं पुरातनम्}
{ऋषेरौद्दालकेर्वाक्यं नाचिकेतस्य चोभयोः}


\twolineshloka
{ऋषिरौद्दालकिर्दीक्षामुपगम्य ततः सुतम्}
{त्वं मामुपचरस्वेति नाचिकेतमभाषत}


\twolineshloka
{समाप्ते नियमे तस्मिन्महर्षिः पुत्रमब्रवीत्}
{उपस्पर्शनसक्तस्य स्वाध्यायाभिरतस्य च}


\twolineshloka
{इध्मा दर्भाः सुमनसः कलशश्चाभितो जलम्}
{विस्मृतं मे तदादाय नदीतीरादिहाव्रज}


\twolineshloka
{गत्वानवाप्य तत्सर्वं नदीवेगसमाप्लुतम्}
{न पश्यामि तदित्येवं पितरं सोऽब्रवीन्मुनिः}


\twolineshloka
{क्षुत्पिपासाश्रमाविष्टो मुनिरौद्दालकिस्तदा}
{यमं पश्येति तं पुत्रमशपत्क्रोधमूर्च्छितः}


\twolineshloka
{तथा स पित्राऽभिहतो वाग्वज्रेण कृताञ्जलिः}
{प्रसीदेति ब्रुवन्नेव गतसत्वोऽपतद्भुवि}


\twolineshloka
{नाचिकेतं प्रिता दृष्ट्वा पतितं दुःखमूर्च्छितः}
{किं मया कृतमित्युक्त्वा निपपात महीतले}


\twolineshloka
{तस्य दुःखपरीतस्य स्वं पुत्रमनुशोचतः}
{व्यतीतं तदहःशेषं सा चोग्रा तत्र शर्वरी}


\twolineshloka
{पित्र्येणाश्रुप्रपातेन नाचिकेतः कुरूद्वह}
{प्रास्यन्दच्छयने कौश्ये वृष्ट्या सस्यमिवाप्लुतम्}


\twolineshloka
{स पर्यपृच्छत्तं पुत्रं श्लाघ्यं पर्यागतं पुनः}
{दिव्यैर्गन्धैः समादिग्धं क्षीणस्वप्नमिवोत्थितम्}


\twolineshloka
{अपि पुत्र जिता लोकाः शुभास्ते स्वेन कर्मणा}
{दिष्ट्या चासि पुनः प्राप्तो न हि ते मानुषं वपुः}


\twolineshloka
{प्रत्यक्षदार्शी सर्वस्य वित्रा पृष्टो महात्मना}
{अभ्युत्थाय पितुर्मध्ये महर्षीणां न्यवेदयत्}


\twolineshloka
{कुर्वन्मवच्छासनमाशु यातोह्यहं विशालां रुचिरप्रभासाम्}
{वैवस्वतीं प्राप्य सभामवश्यंसहस्रशो योजनहैमभौमाम्}


\twolineshloka
{दृष्टैव मामभिमुखमापतन्तंगृहं निवेद्यासनमादिदेश}
{वैवस्वतोऽर्घ्यादिभिरर्हणैश्चभवत्कृते पूजयामास मां साः}


\twolineshloka
{ततस्त्वहं तं शनकैरवोचंवृतः सदस्यैरभिपूज्यमानः}
{प्राप्तोऽस्मि ते विषयं धर्मराजलोकानर्हो यानहं तान्विधत्स्व}


\twolineshloka
{यमोऽब्रवीन्मां न मृतोसि सौम्ययमं पश्येत्याह स त्वां तपस्वी}
{पिता प्रदीप्ताग्निसमानतेजान तत्छक्यमनृतं विप्र कर्तुम्}


\twolineshloka
{दृष्टस्तेऽहं प्रतिगच्छस्व तातशोचत्यसौ तव देहस्य कर्ता}
{ददानि किञ्चापि मनःप्रणीतंप्रियातिथेस्तव कामान्वृणीष्व}


\twolineshloka
{तेनैवमुक्तस्तमहं प्रत्यवोचंप्राप्तोस्मि ते विषयं दुर्निवर्त्यम्}
{इच्छाम्यहं पुण्यकृतां समृद्धाँ-ल्लोकान्द्रष्टुं यदि तेऽहं वरार्हः}


\twolineshloka
{यानं समारोप्य तु मां स देवोवाहैर्युक्तं सुप्रभं भानुमत्तम्}
{संदर्शयामास तदाऽऽत्मलोका-न्सर्वास्तथा पुण्यकृतां द्विजेन्द्र}


\twolineshloka
{अपश्यं तत्र वेश्मानि तैजसानि महात्मनाम्}
{नानासंस्थानरूपाणि सर्वरत्नमयानि च}


\twolineshloka
{चन्द्रमण्डलशुभ्राणि किंकिणीजालवन्ति च}
{अनेकशतभौमानि सान्तर्जलवनानि च}


\twolineshloka
{वैडूर्यार्कप्रकाशानि रूप्यरुक्ममयानि च}
{तरुणादित्यवर्णानि स्थावराणि चराणि च}


\twolineshloka
{भक्ष्यभोज्यमयाञ्शैलान्वासांसि शयनानि च}
{सर्वकामफलांश्चैव वृक्षान्भवनसंस्थितान्}


\twolineshloka
{नद्यो वीथ्यः सभा वाप्यो दीर्घिकाश्चैव सर्वशः}
{घोषवन्ति च यानानि युक्तान्धथ सहस्रशः}


\twolineshloka
{क्षीरस्रवा वै सरितो गिरींश्चसर्पिस्तथा विमलं चापि तोयम्}
{वैवस्वतस्यानुमतांश्च देशा-नदृष्टपूर्वान्सुबहूनपश्यम्}


\twolineshloka
{सर्वान्दृष्ट्वा तदहं धर्मराज-मवोचं वै सर्वदेवं सहिष्णुम्}
{क्षीरस्यैताः सर्पिषश्चैव नद्यःशश्वत्स्रोताः कस्य भोज्याः प्रवृत्ताः}


\twolineshloka
{यमोऽब्रवीद्विद्धि भोज्यांस्त्वमेता-न्ये दातारः साधवो गोरसानाम्}
{अन्ये लोकाः शाश्वता वीतशोकैःसमाकीर्णा गोप्रदाने रतानाम्}


\twolineshloka
{न त्वेतासां दानमात्रं प्रशस्तंपात्रं कालो गोविशेषो विधिश्च}
{ज्ञात्वा देयं विप्र गवान्तरं हिदुःखं ज्ञातुं पावकादित्यभूतम्}


\twolineshloka
{स्वाध्यायवान्योऽतिमात्रं तपस्वीवैतानस्थो ब्राह्मणः पात्रमासाम्}
{गोषु क्षान्तं गोशरण्यं कृतज्ञंवृत्तिग्लानं तादृशं पात्रमाहुः}


\twolineshloka
{कृच्छ्रोत्सृष्टाः पोष्णाभ्यागताश्चद्वारैरेतैर्गोविशेषाः प्रशस्ताः}
{अन्तर्जाताः सुक्रयज्ञानलब्धाःप्राणक्रीताः सोदकाः सोद्वहाश्च}


\twolineshloka
{तिस्रो रात्र्यस्त्वद्भिरुपोष्य भूमौतृप्ता गावस्तर्पितेभ्यः प्रदेयाः}
{वत्सैः प्रीताः सुप्रजाः सोपचारा-स्त्र्यहं दत्त्वा गोरसैर्वर्तितव्यम्}


\twolineshloka
{दत्त्वा धेनुं सुव्रतां साधुदोहांकल्याणवत्सामपलायिनीं च}
{यावन्ति रोमाणि भवन्ति तस्या-स्तावद्वर्षाण्यश्नुते स्वर्गलोकम्}


\twolineshloka
{तथाऽनड्वाहं ब्राह्मणेभ्यः प्रदायदान्तं धुर्यं बलवन्तं युवानम्}
{कुलानुजीव्यं वीर्यवन्तं बृहन्तंभुङ्क्ते लोकान्सम्मितान्धेनुदस्य}


\threelineshloka
{वृद्धे ग्लाने सम्भ्रमे वा महार्थेकृष्यर्थं वा हौम्यहेतोः प्रमूत्याम्}
{गुर्वर्थं वा यज्ञसमाप्तये वागां वै दातुं देशकालोऽविशिष्टः ॥नाचिकेत उवाच}
{}


\twolineshloka
{श्रुत्वा वैवस्वतवचस्तमहं पुनरब्रवम्}
{अगोमी गोप्रदातॄणां कथं लोकान्हि गच्छति}


\twolineshloka
{ततोऽब्रवीद्यमो धीमान्गोप्रदानं ततो गतिम्}
{गोप्रदानानुकल्पात्तु गामृते सन्तु गोप्रदाः}


\twolineshloka
{अलाभे यो गवां दद्याद्धृतधेनुं यतव्रतः}
{तस्यैता घृतवाहिन्यः क्षरन्ते वत्सला इव}


\twolineshloka
{घृतालाभे तु यो दद्यात्तिलधेनुं यतव्रतः}
{स दुर्गात्तारितो धेन्वा क्षीरनद्यां प्रमोदते}


\twolineshloka
{तिलालाभे तु यो दद्याज्जलधेनुं यतव्रतः}
{स कामप्रवहां शीतीं नदीमेतामुपाश्नुते}


\twolineshloka
{एवमेतानि मे तत्र धर्मराजो न्यदर्शयत्}
{दृष्ट्वा च परमं हर्षमवापमहमच्युत}


\twolineshloka
{निवेदये चाहमिमं प्रियं तेक्रतुर्महानल्पधनप्रचारः}
{प्राप्तो मया तात स मत्प्रसूतःप्रपत्स्यते वेदविधिप्रवृत्तः}


\twolineshloka
{शापो ह्ययं भवतोऽनुग्रहायप्राप्तो मया यत्र दृष्टो यमो वै}
{दानव्युष्टिं तत्र दृष्ट्वा महात्म-न्निःसंदिग्धान्दानधर्मांश्चरिष्ये}


\twolineshloka
{इदं च मामब्रवीद्धर्मराजःपुनः पुनः सम्प्रहृष्टो महर्षे}
{दानेन यः प्रयतोऽभूत्सदैवविशेषतो गोप्रदानं च कुर्याम्}


\twolineshloka
{शुद्धो ह्यर्थो नावमन्यस्व धर्मा-न्पात्रे देयं देशकालोपपन्ने}
{तस्माद्गावस्ते नित्यमेव प्रदेयामाभूच्च ते संशयः कश्चिदत्र}


\twolineshloka
{एताः पुरा ह्यददन्नित्यमेवसान्तात्मानो दानपथे निविष्टाः}
{तपांस्युग्राण्यप्रतिशङ्कमाना-स्ते वै दानं प्रददुश्चैव शक्त्या}


\twolineshloka
{काले च शक्त्या मत्सरं वर्जयित्वाशुद्धात्मानः श्रद्धिनः पुण्यशीलाः}
{दत्त्वा गा वै लोकममुं प्रपन्नादेदीप्यन्ते पुण्यशीलास्तु नाके}


\twolineshloka
{एतद्दानं न्यायलब्धं द्विजेभ्यःपात्रे दत्तं प्रापणीयं परीक्ष्य}
{काम्याष्टम्यां वर्तितव्यं दशाहंरसैर्गवां शकृता प्रस्नवैर्वा}


\twolineshloka
{देवव्रती स्याद्वृषभप्रदानै-र्वेदावाप्तिर्गोयुगस्य प्रदाने}
{तीर्थावाप्तिर्गोप्रयुक्तप्रदानेपापोत्सर्गः कपिलायाः प्रदाने}


\twolineshloka
{गामप्येकां कपिलां सम्प्रदायन्यायोपेतां कलुषाद्विप्रमुच्येत्}
{गवां रसात्परमं नास्ति किञ्चि-द्गवां प्रदानं सुमहद्वदन्ति}


\twolineshloka
{गावो लोकांस्तारयन्ति क्षरन्त्योगावश्चान्नं संजनयन्ति लोके}
{यस्तं जानन्न गवां हार्दमेतिस वै गन्ता निरयं पापचेताः}


\twolineshloka
{यैस्तद्दत्तं गोसहस्रं शतं वादशार्धं वा दश वा साधुवत्सम्}
{अप्येका वै साधवे ब्राह्मणायसास्यामुष्मिन्पुण्यतीर्था नदी वै}


\twolineshloka
{प्राप्त्या पुष्ट्या लोकसंरक्षणेनगावस्तुल्याः सूर्यपादैः पृथिव्याम्}
{शब्दश्चैकः संनतिश्चोपभोगा-स्तस्माद्गोदः सूर्य इवावभाति}


\twolineshloka
{गुरुं शिष्यो वरयेद्गोप्रदानेस वै गन्ता नियतं स्वर्गमेव}
{विधिज्ञानां सुमहान्धर्म एषविधिं ह्याद्यं विधयः संविशन्ति}


\twolineshloka
{इदं दानं न्यायलब्धं द्विजेभ्यःपात्रे दत्त्वा प्रापयेथाः परीक्ष्य}
{त्वय्याशंसन्त्यमरा दानवाश्चवयं चापि प्रसृते पुण्यशीले}


\twolineshloka
{इत्युक्तोऽहं धर्मराजं द्विजर्षेधर्मात्मानं शिरसाऽभिप्रणम्य}
{अनुज्ञातस्तेन वैवस्वतेनप्रत्यागमं भगवत्पादमूलम्}


\chapter{अध्यायः १०७}
\twolineshloka
{उक्तं ते गोप्रदानं वै नाचिकेतमृषिं प्रति}
{माहात्म्यमपि चैवोक्तमुद्देशेन गवां प्रभो}


\twolineshloka
{नृगेण च महद्दुःखमनुभूतं महात्मना}
{एकापराधादज्ञानात्पितामह महामते}


\twolineshloka
{द्वारवत्यां यथा चासौ निविशन्त्यां समुद्धृतः}
{मोक्षहेतुरभूत्कृष्णस्तदप्यवधृतं मया}


\threelineshloka
{किं त्वस्ति मम संदेहो गवां लोकं प्रति प्रभो}
{तत्त्वतः श्रोतुमिच्छामि गोदा यत्र वसन्त्युत ॥भीष्म उवाच}
{}


\threelineshloka
{अत्राप्युदाहरन्तीममितिहासं पुरातनम्}
{यथाऽपृच्छत्पद्मयोनिमेतदेव शतक्रतुः ॥शक्र उवाच}
{}


\twolineshloka
{स्वर्लोकवासिनां लक्ष्मीमभिभूय स्वयाऽर्चिषा}
{गोलोकवासिनः पश्ये वदतां संशयोऽत्र मे}


\twolineshloka
{कीदृशा भगवँल्लोका गवां तद्बूहि मेऽनघ}
{यानावसन्ति दातार एतदिच्छामि वेदितुम्}


\twolineshloka
{कीदृशाः किंफलाः किंस्वित्परमस्तत्र को गुणः}
{कथं च पुरुषास्तत्र गच्छन्ति विगतज्वराः}


\twolineshloka
{कियत्कालं प्रदानस्य दाता च फलमश्नुते}
{कथं बहुविधं दानं स्यादल्पमपि वा कथम्}


\twolineshloka
{बह्वीनां कीदृशं दानमल्पानां वाऽपि दीदृशम्}
{अदत्त्वा गोप्रदाः सन्ति केन वा तच्च शंस मे}


\twolineshloka
{कथं वा बहुदाता स्यादल्पदात्रा समः प्रभो}
{अल्पप्रदाता बहुदः कथं स्वित्स्यादिहेश्वर}


\twolineshloka
{कीदृशी दक्षिणा चैव गोप्रदाने विशिष्यते}
{एतत्तथ्येन भगवन्मम शंसितुमर्हसि}


\chapter{अध्यायः १०८}
\twolineshloka
{योऽयं प्रश्नस्त्वया पृष्टो गोप्रदानादिकारितः}
{नान्यः प्रष्टास्ति लोकेस्मिंस्त्वत्तोन्यो हि शतक्रतो?}


\twolineshloka
{सन्ति नानाविधा लोका यांस्त्वं शक्र न पश्यसि}
{पश्यामि यानहं लोकानेकपत्न्यश्च याः स्त्रियः}


\twolineshloka
{कर्मभिश्चापि सुशुभैः सुव्रता ऋषयस्तथा}
{सशरीरा हि तान्यान्ति ब्राह्मणाः शुभबुद्धयः}


\twolineshloka
{शरीरन्यासमोक्षेण मनसा निर्मलेन च}
{स्वप्नभूतांश्च ताँल्लोकान्पश्यन्तीहापि सुव्रताः}


\twolineshloka
{ते तु लोकाः सहस्राक्ष शृणु यादृग्गुणान्विताः}
{न तत्र क्रमते कालो न जरा न च पावकः}


\twolineshloka
{तथा नास्त्यशुभं किञ्चिन्न व्याधिस्तत्र न क्लमः}
{यद्यच्च गावो मनसा तस्मिन्वाञ्छन्ति वासव}


\twolineshloka
{तत्सर्वं प्रापयन्ति स्म मम प्रत्यक्षदर्शनात्}
{कामगाः कामचारिण्यः कामात्कामांश्च भुञ्जते}


\twolineshloka
{वाप्यः सरांसि सरितो विविधानि वनानि च}
{गृहाणि पर्वताश्चैव यावद्द्रव्यं च किञ्चन}


\twolineshloka
{मनोज्ञं सर्वभूतेभ्यस्तद्वनं तत्र दृश्यते}
{ईदृशान्विद्दि ताँल्लोकान्नास्ति लोकस्तथाविधः}


\twolineshloka
{तत्र सर्वसहाः क्षान्ता वत्सला गुरुवर्तिनः}
{अहङ्कारैर्विरहिता यान्ति शक्र नरोत्तमाः}


\twolineshloka
{यः सर्वमांसानि न भक्षयीतपुमान्सदा भावितो धर्मयुक्तः}
{मातापित्रोरर्चिता सत्ययुक्तःशुश्रुषिता ब्राह्मणानामनिन्द्यः}


\twolineshloka
{अक्रोधनो गोषु तथा द्विजेषुधर्मे रतो गुरुशुश्रूषकश्च}
{यावज्जीवं सत्यवृत्ते रतश्चदाने रतो यः क्षमी चापराधे}


\twolineshloka
{मृदुर्दान्तो देवपरायणश्चसर्वातिथिश्चापि यथा दयावान्}
{ईदृग्गुणो मानवस्तं प्रयातिलोकं गवां शाश्वतं चाव्ययं च}


\twolineshloka
{न पारदारी पश्यति लोकमेतंन वै गुरुघ्नो न मृषा सम्प्रलापी}
{सदापवादी ब्राह्मणेष्वात्तवैरोदोषैरन्यैर्यश्च युक्तो दुरात्मा}


\twolineshloka
{न मित्रध्रुङ्नैकृतिकः कृतघ्नःशठोऽनुजुर्धर्मविद्वेषकश्च}
{न ब्रह्महा मनसाऽपि प्रपश्ये-द्गवां लोकं पुण्यकृतां निवासम्}


\twolineshloka
{एतत्ते सर्वमाख्यातं नैपुण्येन सुरेश्वर}
{गोप्रदानरतानां तु फलं शृणु शतक्रतो}


\twolineshloka
{दायाद्यलब्धैरर्थैर्यो गाः क्रीत्वा सम्प्रयच्छति}
{धर्मार्जितान्धैः क्रीतान्स लोकानाप्नुतेऽक्षयान्}


\twolineshloka
{यो वै द्यूते धनं जित्वा गाः क्रीत्वा सम्प्रयच्छति}
{स दिव्यमयुतं शक्र वर्षाणां फलमश्नुते}


\twolineshloka
{दायाद्याद्याः स्म वै गावो न्यायपूर्वैरुपार्जिताः}
{प्रदद्यात्ताः प्रदातॄणां सम्भवन्त्यपि च ध्रुवाः}


\twolineshloka
{प्रतिगृह्य तु यो दद्याद्गाः संशुद्धेनि चेतसा}
{तस्यापीहाक्षयाँल्लोकान्ध्रुवान्विद्धि शचीपते}


\twolineshloka
{जन्मप्रभृति सत्यं च यो ब्रूयान्नियतेन्द्रियः}
{गुरुद्विजसहः क्षान्तस्तस्य गोभिः समा गतिः}


\twolineshloka
{न जातु ब्राह्मणो वाच्यो यदवाच्यं शचीपते}
{मनसा गोषु न द्रुह्येद्गोवृत्तिर्गोनुकम्पकः}


\twolineshloka
{सत्ये धर्मे च निरतस्तस्य शक्र फलं शृणु}
{गोसहस्रेण समिता तस्य धेनुर्भवत्युत}


\twolineshloka
{क्षत्रियस्य गुणैरतैरन्वितस्य फलं शृणु}
{सप्तार्धशततुल्या गौर्भवतीति विनिश्चयः}


\twolineshloka
{वैश्यस्यैते यदि गुणास्तस्य पञ्चशतं भवेत्}
{शूद्रस्यापि विनीतस्य चतुर्भागफलं स्मृतम्}


\twolineshloka
{एतच्चैनं योऽनुतिष्ठेत युक्तःसत्ये रतो गुरुशुश्रूषया च}
{दक्षः क्षान्तो देवतार्थी प्रशान्तःशुचिर्बुद्धो धर्मशीलोऽनहंवाक्}


\twolineshloka
{महत्फलं प्राप्यते सद्द्विजायदत्त्वा दोग्ध्रीं विधिनाऽनेन धेनुम्}
{नित्यं दद्यादेकभक्तः सदा चसत्ये स्थितो गुरुशुश्रुषिता च}


\twolineshloka
{वेदाध्यायी गोषु यो भक्तिमांश्चनित्यं दत्त्वा योऽभिनन्देत गाश्च}
{आजातितो यश्च गवां नमेतइदं फलं शक्र निबोध तस्य}


\twolineshloka
{यत्स्यादिष्ट्वा राजमूये फलं तुयत्स्यादिष्ट्वा बहुना काञ्चनेन}
{एततुल्यं फलमप्याहुरग्र्यंसर्वे संन्तस्त्वषयो ये च सिद्धाः}


\twolineshloka
{योऽग्रं भक्तं किञ्चिदप्राश्य दद्या-द्गोभ्यो नित्यं गोव्रती सत्यवादी}
{शान्तोऽलुब्धो गोसहस्रस्य पुण्यंसंवत्सरेणाप्नुयात्सत्यशीलः}


\twolineshloka
{यदकेभक्तमश्नीयाद्दद्यादेकं गवां च यत्}
{दशवर्षाण्यनन्तानि गोव्रती गोनुकम्पकः}


\twolineshloka
{एकेनैव च भक्तेन यः क्रीत्वा गां प्रयच्छति}
{यावन्ति तस्या रोमाणि सम्भवन्ति शतक्रतो}


\twolineshloka
{तावच्छतानां स गवां फलमाप्नोति शाश्वतम्}
{ब्राह्मणस्य फलं हीदं क्षत्रियस्य तु वै शृणु}


\twolineshloka
{पञ्चवार्षिकमेवं तु क्षत्रियस्य फलं स्मृतम्}
{ततोऽर्धेन तु वैश्यस्य शूद्रो वैश्यार्धतः स्मृतः}


\twolineshloka
{यश्चात्मावेक्रयं कृत्वा गाः क्रीत्वा सम्प्रयच्छति}
{यावत्संदर्शयेद्गां वै स तावत्फलमश्नुते}


\twolineshloka
{रोम्णि रोम्णि महाभाग लोकाश्चास्याक्षयाः स्मृताःसङ्ग्रामेष्वर्जयित्वा तु यो वै गाः सम्प्रयच्छति}
{आत्मविक्रयतुल्यास्ताः शाश्वता विद्धि कौशिक}


\twolineshloka
{अभावे यो गवां दद्यात्तिलधेनुं यतव्रतः}
{दुर्गात्स तारितो धेन्वा क्षीरनद्यां प्रमोदत्ते}


\twolineshloka
{न त्वेवासां दानमात्रं प्रशस्तंपात्रं कालो गोविशेषो विधिश्च}
{कालज्ञानं विप्रगवान्तरं हिदुःखं ज्ञातुं पावकादित्यभूतम्}


\threelineshloka
{स्वाध्यायाढ्यं शुद्धयोनिं प्रशान्तंवैतानस्थं पापभीरुं बहुज्ञम्}
{गोषु क्षान्तं नातितीक्ष्णं शरण्यंवृत्तिग्लानं तादृशं पात्रमाहुः}
{}


\twolineshloka
{वृत्तिग्लाने सीदति चातिमात्रंतुष्ट्यर्थं वा होम्यहेतोः प्रसूतेः}
{गुर्वर्थं वा बालसंवृद्धये वाधेनुं दद्याद्देशकाले विशिष्टे}


\twolineshloka
{अन्तर्ज्ञाताः सक्रयज्ञानलब्धाःप्राणैः क्रीतास्तेजसा यौतकाश्च}
{कृच्छ्रोत्सृष्टाः पोषणाभ्यागताश्चद्वारैरेतैर्गोविशेषाः प्रशस्ताः}


% Check verse!
बलान्विताः शीलवयोपपन्नाःसर्वाः प्रशंसन्ति सुगन्धवत्यःयथा हि गङ्गा सरितां वरिष्ठातथाऽर्जुनीनां कपिला वरिष्ठा
\twolineshloka
{तिस्रो रात्रीस्त्वद्भिरुपोष्य भूमौतृप्ता गावस्तर्पितेभ्यः प्रदेयाः}
{वत्सैः पुष्टैः क्षीरपैः सुप्रचारा-स्त्र्यहं दत्त्वा गोरसैर्वर्तितव्यम्}


\twolineshloka
{दत्त्वा धेनुं सुव्रतां साधुदोहांकल्याणवत्सामपलायिनीं च}
{यावन्ति रोमाणि भवन्ति तस्या-स्तावन्ति वर्षामि भवन्त्यमुत्र}


\twolineshloka
{तथाऽनड्वाहं ब्राह्मणाय प्रदायधुर्यं युवानं बलिनं विनीतम्}
{हलस्य वोढारमनन्तवीर्यंप्राप्नोति लोकान्दशधेनुदस्य}


\twolineshloka
{कान्ताराद्ब्राह्म्णान्गाश्च यः परित्राति कौशिक}
{क्षेमेण स विमुच्येत तस्य पुण्यफलं शृणु}


\twolineshloka
{अश्वमेधक्रतोस्तुल्यं फलं भवति शाश्वतम्}
{मृत्युकाले सहस्राक्ष यां वृत्तिमनुकाङ्क्षते}


\twolineshloka
{लोकान्बहुविधान्दिव्यान्यच्चास्य हृदि वर्तते}
{तत्सर्वं समवाप्नोति कर्मणैतेन मानवः}


\twolineshloka
{गोभिश्च समनुज्ञातः सर्वत्र च महीयते}
{यस्त्वेतेनैव कल्पेन गां वनेष्वनुगच्छति}


\twolineshloka
{तृणगोमयपर्णाशी निस्पृहो नियतः शुचिः}
{अकामं तेन वस्तव्यं मुदितेन शतक्रतो}


% Check verse!
मम लोके वसति स लोके वा यत्र चेच्छति
\chapter{अध्यायः १०९}
\threelineshloka
{जान्यो गामपहरेद्विक्रीयाच्चार्थकारणात्}
{एतद्विज्ञातुमिच्छामि कानु तस्य गतिर्भवेत् ॥पितामह उवाच}
{}


\twolineshloka
{भक्तार्थं विक्रयार्थं वा येऽपहारं हि कुर्वते}
{दानार्थं ब्राह्मणार्थाय तत्रेदं श्रूयतां फलम्}


\twolineshloka
{विक्रयार्थं हि यो हिंस्याद्भक्षयेद्वा निरङ्कुशः}
{घातयानं हि पुरुषं येऽनुमन्येयुरर्थिनः}


\twolineshloka
{घातकः खादको वाऽपि तथा यश्चानुमन्यते}
{यावन्ति तस्या रोमाणि तावद्वर्षाणि मज्जति}


\twolineshloka
{ये दोषा यादृशाश्चैव द्विजयज्ञोपघातके}
{विक्रये चापहारे च ते दोषा वै स्मृता गवाम्}


\threelineshloka
{अपहृत्य तु यो गां वै ब्राह्मणाय प्रयच्छति}
{यावद्दाने तु यो गां वै ब्राह्मणाय प्रयच्छति}
{}


\twolineshloka
{सुवर्णं दक्षिणामाहुर्गोप्रदाने महाद्युते}
{सुवर्णं परमित्युक्तं दक्षिणार्थमसंशयम्}


\twolineshloka
{गोप्रदानात्तारयते सप्त पूर्वांस्तथाऽपरान्}
{सुवर्णं दक्षिणां कृत्वा तावद्द्विगुणमुच्यते}


\twolineshloka
{सुवर्णं परमं दानं सुवर्णं दक्षिणा परा}
{सुवर्णं पावनं शक्र पावनानां परं स्मृतम्}


\threelineshloka
{कुलानां पावनं प्राहुर्जातरूपं शतक्रतो}
{एषा मे दक्षिणा प्रोक्ता समासेन महाद्युते ॥भीष्म उवाच}
{}


\twolineshloka
{एतत्पितामहेनोक्तमिन्द्राय भरतर्षभ}
{इन्द्रो दशरथायाह रामायाह पिता तथा}


\twolineshloka
{राघवोपि प्रियभ्रात्रे लक्ष्मणाय यशस्विने}
{ऋषिभ्यो लक्ष्मणेनोक्तमरण्ये वसता प्रभो}


\twolineshloka
{पारम्पर्यागतं चेदमृषयः संशितव्रताः}
{दुर्धरं दारयामासू राजानश्चैव धार्मिकाः}


% Check verse!
उपाध्यायेन गदितं मम चेदं युधिष्ठिर
\twolineshloka
{य इदं ब्राह्मणो नित्यं वदेद्ब्राह्मणसंसदि}
{यज्ञेषु गोप्रदानेषु द्वयोरपि समागमे}


\twolineshloka
{तस्य लोकाः किलाक्षय्या दैवतैः सह नित्यदा}
{इति ब्रह्मा स भगवानुवाच परमेश्वरः}


\chapter{अध्यायः ११०}
\twolineshloka
{विस्रम्भितोऽहं भवता धर्मान्प्रवदता विभो}
{प्रवक्ष्यामि तु संदेहं तन्मे ब्रूहि पितामह}


\twolineshloka
{व्रतानां किं फलं प्रोक्तं कीदृशं वा महाद्युते}
{नियमानां फलं किं च स्वधीतस्य च किं फलम्}


\twolineshloka
{दमस्येह फलं किं च वेदानां धारणे च किम्}
{अध्यापने फलं किं च सर्वमिच्छामि वेदितुम्}


\twolineshloka
{अप्रतिग्राहके किं च फलं लोके पितामह}
{तस्य किञ्च फलं दृष्टं श्रुतं यस्तु प्रयच्छति}


\threelineshloka
{स्वकर्मनिरतानां च शूराणां चापि किं फलम्}
{सत्ये च किं फलं प्रोक्तं ब्रह्मचर्ये च किं फलम्}
{}


\twolineshloka
{पितृशुश्रूषणे किञ्च मातृशुश्रूषणे तथा}
{आचार्यगुरुशुश्रूषा स्वनुक्रोशानुकम्पने}


\threelineshloka
{एतत्सर्वमशेषेण पितामह यथातथम्}
{वेतुमिच्छामि धर्मज्ञ परं कौतूहलं हि मे ॥भीष्म उवाच}
{}


\twolineshloka
{यो व्रतं वै यथोद्दिष्टं तथा सम्प्रतिपद्यते}
{अखण्डं सम्यगारभ्य तस्य लोकाः सनातनाः}


\twolineshloka
{नियमानां फलं राजन्प्रत्यक्षमिह दृश्यते}
{नियमानां क्रतूनां च त्वयाऽवाप्तमिदं फलम्}


\twolineshloka
{स्वधीतस्यापि च फलं दृश्यतेऽमुत्र चेह च}
{इह लोकेऽर्थवान्नित्यं ब्रह्मलोके च मोदते}


\threelineshloka
{दमस्य तु फलं राजञ्शृणु त्वं विस्तरेण मे}
{दान्ताः सर्वत्र सुखिनो दान्ताः सर्वत्र निर्वृताः}
{यत्रेच्छागामिनो दान्ताः सर्वशत्रुनिषूदनाः}


\twolineshloka
{प्रार्थयन्ति च यद्दान्ता लभन्ते तन्न संशयः}
{युज्यन्ते सर्वकामैर्हि दान्ताः सर्वत्र पाण्डव}


\twolineshloka
{स्वर्गे यथा प्रमोदन्ते तपसा विक्रमेण च}
{दानैर्यज्ञैश्च विविधैस्तथा दान्ताः क्षमान्विताः}


\threelineshloka
{दानाद्दमो विशिष्टो हि ददत्किञ्चिद्द्विजातये}
{दाता कुप्यति नो दान्तस्तस्माद्दानात्परंदमः ॥यस्तु दद्यादकुप्यन्हि तस्य लोकाः सनातनाः}
{क्रोधो हन्ति हि यद्दानं तस्माद्दानाद्वरो दमः}


\twolineshloka
{अदृश्यानि महाराज स्थानान्ययुतशो दिवि}
{ऋषीणां सर्वलोकषु यानि ते यान्ति देवताः}


\twolineshloka
{दमेन यानि नृपते गच्छन्ति परमर्षयः}
{कामयाना महत्स्थानं तस्माद्दानात्परं दमः}


\twolineshloka
{`विद्यादानाद्वरं नास्ति वेदविद्या महाफलाः}
{'अध्यापकः परिक्लेशादक्षयं फलमश्नुते}


\threelineshloka
{विधिवत्पावकं हुत्वा ब्रह्मलोके नराधिप}
{अधीत्यापि हि यो वेदान्न्यायविद्भ्यः प्रयच्छति}
{गुरुकर्मप्रशंसी तु सोपि स्वर्गे महीयते}


\twolineshloka
{क्षत्रियोऽध्ययने युक्तो यजने दानकर्मणि}
{युद्धे यश्च परित्राता सोपि स्वर्गे महीयते}


\twolineshloka
{वैश्यः स्वकर्मनिरतः प्रदानाल्लभते महत्}
{शूद्रः स्वकर्मनिरतः स्वर्गं शुश्रूषयाऽऽर्च्छति}


\twolineshloka
{शूरा बहुविधाः प्रोक्तास्तेषामर्थास्तु मे शृणु}
{शूरान्वयानां निर्दिष्टं फलं शूरस्य चैव हि}


\twolineshloka
{यज्ञशूरा दमे शूराः सत्यशूरास्तथा परे}
{युद्धशूरास्तथैवोक्ता दानशूराश्च मानवाः}


\threelineshloka
{`बुद्धिशूरास्तथैवान्ये क्षमाशूरास्तथा परे}
{'साङ्ख्यशूराश्च बहवो योगशूरास्तथा परे}
{अरण्ये गृहवासे च त्यागे शूरास्तथा परे}


\threelineshloka
{आर्जवे च तथा शूराः शमे वर्तन्ति मानवाः}
{तैस्तैश्च नियमैः शूरा बहवः सन्ति चापरे}
{वेदाध्ययनशूराश्च शूराश्चाध्यापने रताः}


\twolineshloka
{गुरुशुश्रूषया शूराः पितृशुश्रूषया परे}
{मातृशुश्रूषया शूरा भैक्ष्यशूरास्तथा परे}


\twolineshloka
{अरण्ये गृहवासे च शूराश्चातिथिपूजने}
{सर्वे यान्ति पराँल्लोकान्स्वकर्मफलनिर्जितान्}


\twolineshloka
{धारणं सर्ववेदानां सर्वतीर्थावगाहनम्}
{सत्यं च ब्रुवतो नित्यं समं वा स्यान्नवा समम्}


\twolineshloka
{अश्वमेधसहस्रं च सत्यं च तुलया धृतम्}
{अश्वमेधसहस्राद्धि सत्यमेव विशिष्यते}


\twolineshloka
{सत्येन सूर्यस्तपति सत्येनाग्निः प्रदीप्यते}
{सत्येन मरुतो वान्ति सर्वं सत्ये प्रतिष्ठितम्}


\twolineshloka
{सत्येन देवाः प्रीयन्ते पितरो ब्राह्मणास्तथा}
{सत्यमाहुः परो धर्मस्तस्मात्सत्यं न लोपयेत्}


\twolineshloka
{मुनयः सत्यनिरता मुनयः सत्यविक्रमाः}
{मुनयः सत्यशपथास्तस्मात्सत्यं विशिष्यते}


\twolineshloka
{सत्यवन्तः सत्यलोके मोदन्ते भरतर्षभ}
{दमः सत्यफलावाप्तिरुक्ता सर्वात्मना मया}


\twolineshloka
{असंशयं विनीतात्मा स वै स्वर्गे महीयते}
{ब्रह्मचर्यस्य च गुणं शृणु त्वं वसुधाधिप}


\twolineshloka
{आजन्ममरणाद्यस्तु ब्रह्मचारी भवेदिह}
{न तस्य किञ्चिदप्राप्यमिति विद्धि नराधिप}


% Check verse!
बह्व्यः कोट्यस्त्वषीणां तु ब्रह्मलोके वसन्त्युतसत्ये रतानां सततं दान्तानामूर्ध्वरेतसाम्
\twolineshloka
{ब्रह्मचर्यं दहेद्राजन्सर्वपापान्युपासितम्}
{ब्राह्मणेन विशेषेण ब्राह्ममो ह्यग्निरुच्यते}


\twolineshloka
{प्रत्यक्षं हि तथा ह्येतद्ब्राह्मणेषु तपस्विषु}
{बिभेति हि यथा शक्रो ब्रह्मचारिप्रधर्षितः}


\twolineshloka
{तद्ब्रह्मचर्यस्य फलमृषीणामिह दृश्यते}
{मातापित्रोः पूजने यो धर्मस्तमपि मे शृणु}


\twolineshloka
{शुश्रूषते यः पितरं न चासूयेत्कदाचन}
{मातरं भ्रातरं वाऽपि गुरुमाचार्यमेव च}


\twolineshloka
{तस्य राजन्फलं विद्धि स्वर्लोके स्थानमर्चितम्}
{न च पश्येत नरकं गुरुशुश्रूषयाऽऽत्मवान्}


\chapter{अध्यायः १११}
\threelineshloka
{विधिं गवां परं श्रोतुमिच्छामि नृप तत्त्वतः}
{येन ताञ्शाश्वताँल्लोकानर्थिनां प्राप्नुयादिह ॥भीष्म उवाच}
{}


\twolineshloka
{न गोदानात्परं किञ्चिद्विद्यते वसुधाधिप}
{गौर्हि न्यायागता दत्ता सद्यस्तारयते कुलम्}


\twolineshloka
{सतामर्थे सम्यगुत्पादितो यःस वै क्लृप्तः सम्यगाभ्यः प्रजाभ्यः}
{तस्मात्पूर्वं ह्यादिकालप्रवृत्तंगोदानार्थं शृणु राजन्विधिं मे}


\twolineshloka
{पुरा गोष्ठे निलीनासु गोषु सन्दिग्धदर्शिना}
{मांधात्रा प्रकृतं प्रश्नं बृहस्पतिरभाषत}


\twolineshloka
{द्विजानामन्त्र्य सत्कृत्य प्रोक्तं कालमुपोष्य च}
{गोदानार्थे प्रयुञ्जीत रोहिणीं नियतव्रतः}


\twolineshloka
{आह्वानं च प्रयुञ्जीत समङ्गे बहुलेति च}
{प्रविश्य च गवां मध्यमिमां श्रुतिमुदाहरेत्}


\twolineshloka
{गौर्मे माता वृषभः पिता मेदिवं गर्भं जगती मे प्रतिष्ठा}
{प्रपद्यैवं शर्वरीमुष्य गोषुपुनर्वाणीमुत्सृजेद्गोप्रदाने}


\twolineshloka
{स तामेकां निशां गोभिः समसख्यः समव्रतः}
{ऐकात्म्यगमनात्सद्यः कलुषाद्विप्रमुच्यते}


\twolineshloka
{उत्सृष्टवृषवत्सा हि प्रदेया सूर्यदर्शने}
{त्रिदिवं प्रतिपत्तव्यमर्थवादाशिषस्तव}


\threelineshloka
{ऊर्जस्विन्य ऊर्जमेधाश्च यज्ञेगर्भोऽमृतस्य जगतश्च प्रतिष्ठा}
{क्षिते रोहः प्रवहः शश्वदेव}
{प्राजापत्याः सर्वमित्यर्थवादाः}


\twolineshloka
{गावो ममैनः प्रणुदन्तु सौर्या-स्तथा सौम्याः स्वर्गयानाय सन्तु}
{आत्मानं मे मातृवच्चाश्रयन्तुतथाऽनुक्ताः सन्तु सर्वाशिषो मे}


\twolineshloka
{शोषोत्सर्गे कर्मभिर्देहमोक्षेसरस्वत्यः श्रेयसे सम्प्रवृत्ताः}
{यूयं नित्यं सर्वपुण्योपवाह्यांदिशध्वं मे गतिमिष्टां प्रसन्नाः}


\twolineshloka
{या वै यूयं सोऽहमद्यैव भावोयुष्मान्दत्त्वा चाहमात्मप्रदाता}
{मनश्च्युता मन एवोपपन्नाःसन्धुक्षध्वं सौम्यरूपाग्र्यदेहाः}


\twolineshloka
{दानस्याग्रे पूर्वमेतद्वदेतगवां दाता विधिवत्पर्वदृष्टम्}
{प्रतिब्रूयाच्छेषमर्धंद्विजातिःप्रतिगृह्णन्वै गोप्रदाने विधिज्ञः}


\twolineshloka
{गां ददानीति वक्तव्यमर्घ्यमुत्सृज्य सुव्रतः}
{ऊढव्या भरितव्या च वैष्णवीति च चोदयेत्}


\twolineshloka
{नाम सङ्कीर्तयेत्तस्या यथासंख्योत्तरं स वै}
{फलं षट्त्रिंशदष्टौ च सहस्राणि च विंशतिः}


\twolineshloka
{एवमेतान्गुणान्विद्याद्गवादीनां यथाक्रमम्}
{गोप्रदाता समाप्नोति समस्तानष्टमे क्रमे}


\twolineshloka
{गोदः शीली निर्भयश्चार्घदातान स्याद्दुःखी वसुदाता च कामम्}
{उपस्योढा भारते यश्च विद्वा-न्विख्यातास्ते वैष्णवाश्चन्द्रलोकाः}


\twolineshloka
{गा वै दत्त्वा गोव्रती स्यात्त्रिरात्रंनिशां चैकां संवसेतेह ताभिः}
{काम्याष्टभ्यां वर्तितव्यं त्रिरात्रंरसैर्वा गोः शकृता प्रश्नवैर्वा}


\twolineshloka
{देवव्रती स्याद्वृषभप्रदानेवेदावाप्तिर्गोयुगस्य प्रदाने}
{तथा गवां विधिमासाद्य यज्वालोकानग्र्यान्विन्दते गोविधिज्ञः}


\twolineshloka
{कामान्सर्वान्पार्थिवानेकसंस्था-न्यो वै दद्यात्कामदुघां च धेनुम्}
{सम्यक्ताः स्युर्हव्यकव्यौघवत्यस्तासामुक्ष्णां ज्यायसां सम्प्रदानम्}


\twolineshloka
{न चाशिष्यायाव्रतायोपकुर्या-न्नाश्रद्दधानाय न वक्रबुद्धये}
{गुह्यो ह्ययं सर्वलोकस्य धर्मोनेमं धर्मं यत्र तत्र प्रजल्पेत्}


\twolineshloka
{सन्ति लोके श्रद्दधाना मनुष्याःसन्ति क्षुद्रा राक्षसमानुषेषु}
{एषामेतद्दीयमानं ह्यनिष्टंये नास्तिक्यं चाश्रयन्तेऽल्पपुण्याः}


\twolineshloka
{बार्हस्पत्यं वाक्यमेतन्निशम्यये राजानो गोप्रदानानि दत्त्वा}
{लोकान्प्राप्ताः पुण्यशीलाः प्रवृत्ता-स्तान्मे राजन्कीर्त्यमानान्निबोध}


\twolineshloka
{उशीनरो विश्वगश्वो नृगश्चभगीरथो विश्रुतो यौवनाश्वः}
{मान्धाता वै मुचुकुन्दश्च राजाभूरिद्युम्नो नैषधः सोमकश्च}


\twolineshloka
{पुरूरवा भरतश्चक्रवर्तीयस्यान्ववाये भरताः सर्व एव}
{तथा वीरो दाशरथिश्च रामोये चाप्यन्ये विश्रुताः कीर्तिमन्तः}


\twolineshloka
{तथा राजा पृथुकर्मा दिलीपोदिवं प्राप्तो गोप्रदानैर्विधिज्ञः}
{यज्ञैर्दानैस्तपसा राजधर्मै-र्मांधाताऽभूद्गोप्रदानैश्च युक्तः}


\threelineshloka
{तस्मात्पार्थ त्वमपीमां मयोक्तांबार्हस्पतीं भारतीं धारयस्व}
{द्विजाग्र्येभ्यः सम्प्रयच्छस्व प्रीतोगाः पुण्या वै प्राप्य राज्यं कुरूणाम् ॥वैशम्पायन उवाच}
{}


% Check verse!
तथा सर्वं कृतवान्धर्मराजोभीष्मेणोक्तो विधिवद्गोप्रदानेस मान्धातुर्वेद देवोपदिष्टंसम्यग्धर्मं धारयामास राजा
\twolineshloka
{इति नृप सततं गवां प्रदानेयवशकलान्सह गोमयैः पिबानः}
{क्षितितलशयनः शिखी यतात्मावृष इव राजवृषस्तदा बभूव}


\twolineshloka
{नरपतिरभवत्सदैव ताभ्यःप्रयतमनास्त्वभिसंस्तुवंश्च गा वै}
{नृपतिधुरि च गामयुक्त भूप-स्तुरगवरैरगमच्च यत्र तत्र}


\chapter{अध्यायः ११२}
\twolineshloka
{ततो युधिष्ठिरे राजा भूयः शान्तनवं नृपम्}
{गोदानविस्तरं धीमान्पप्रच्छ विनयान्वितः}


\twolineshloka
{गोप्रदानगुणान्सम्यक् पुनर्मे ब्रूहि भारत}
{न हि तृप्याम्यहं वीर शृण्वानोऽमृतमीदृशम्}


\threelineshloka
{इत्युक्तो धर्मराजेन तदा शान्तनवो नृपः}
{सम्यगाह गुणांस्तस्मै गोप्रदानस्य केवलान् ॥भीष्म उवाच}
{}


\twolineshloka
{वत्सलां गुणसम्प्रन्ना तरुणीं वस्त्रसंयुताम्}
{दत्त्वेदृशीं गां विप्राय सर्वपापैः प्रमुच्यते}


\twolineshloka
{असुर्या नाम ते लोका गां दत्त्वा तान्न गच्छति}
{}


\threelineshloka
{पीतोदकां जग्धतृणां नष्टक्षीरां निरिन्द्रियाम्}
{जरारोगोपसम्पन्नां जीर्णां वापीमिवाजलाम्}
{दत्त्वा तमः प्रविशति द्विजं क्लेशेन योजयेत्}


\twolineshloka
{रुष्टा दुष्टा व्याधिता दुर्बला वानो दातव्या याश्च मूल्यैरदत्तैः}
{क्लेशैर्विप्रं योऽफलैः संयुनक्तितस्याऽवीर्याश्चाफलाश्चैव लोकाः}


\threelineshloka
{बलान्विताः शीलवीर्योपपन्नाःसर्वे प्रशंसन्ति सुगन्धवत्यः}
{यथा हि गङ्गा सरितां वरिष्ठातथाऽर्जुनीनां कपिला वरिष्ठा ॥युधिष्ठिर उवाच}
{}


\threelineshloka
{कस्मात्समाने बहुलाप्रदानेसद्भिः प्रशस्तं कपिलाप्रदानम्}
{विशेषमिच्छामि महाप्रभावंश्रोतुं समर्थोस्मि भवान्प्रवक्तुम् ॥भीष्म उवाच}
{}


\twolineshloka
{वृद्धानां ब्रुवतां श्रुत्वा कपिलानामथोद्भवम्}
{वक्ष्यामि तदशेषेण रोहिण्यो निर्मिता यथा}


\twolineshloka
{प्रजाः सृजेति चादिष्टः पूर्वं दक्षः स्वयंभुवा}
{नासृजद्वृत्तिमेवाग्रे प्रजानां हितकाम्यया}


\twolineshloka
{यथा ह्यमृतमाश्रित्य वर्तयन्ति दिवौकसः}
{तथा वृत्तिं समाश्रित्य वर्तयन्ति प्रजा विभो}


\twolineshloka
{अचरेभ्यश्च भूतेभ्यश्चराः श्रेष्ठास्ततो नराः}
{ब्राह्मणाश्च ततः श्रेष्ठास्तषु यज्ञाः प्रतिष्ठिताः}


\twolineshloka
{यज्ञैराप्यायते सोमः स च गोषु प्रतिष्ठितः}
{ताभ्यो देवाः प्रमोदन्ते प्रजानां वृत्तिरासु च}


\twolineshloka
{ततः प्रजासु सृष्टासु दक्षाद्यैः क्षुधिताः प्रजाः}
{प्रजापतिमुपाधावन्विनिश्चित्य चतुर्मखम्}


\twolineshloka
{प्रजातान्येव भूतानि प्राक्रोशन्वृत्तिकाङ्क्षया}
{वृत्तिदं चान्वपद्यन्त तृषिताः पितृमातृवत्}


\threelineshloka
{इतीदं मनसा गत्वा प्रजासर्गार्थमात्मनः}
{प्रजापतिर्बलाधानममृतं प्रापिबत्तदा}
{शंसतस्तस्य तृप्तिं तु गन्धात्सुरभिरुत्थिता}


\twolineshloka
{मुखजा साऽसृजद्धातुः सुरभिर्लोकमातरम्}
{दर्शनीयरसं वृत्तिं सुरभिं मुखजां सुताम्}


\twolineshloka
{साऽसृजत्सौरभेयीस्तु सुरभिर्लोकमातृकाः}
{सुवर्णवर्णाः कपिलाः प्रजानां वृत्तिधेनवः}


\twolineshloka
{तासाममृतवृत्तीनां क्षरन्तीनां समन्ततः}
{बभूवामृतजः फेनः स्रवन्तीनामिवोर्मिजः}


\threelineshloka
{स वत्समुखविभ्रष्टो भवस्य भुवि तिष्ठतः}
{शिरस्यवापतत्क्रुद्धः स तदैक्षत च प्रभुः}
{ललाटप्रभवेणाक्ष्णा रोहिणीं प्रदहन्निव}


\twolineshloka
{तत्तेजस्तु ततो रौद्रं कपिलां गां विशाम्पते}
{नानावर्णत्वमनयन्मेघानिव दिवाकरः}


\twolineshloka
{यास्तु तस्मादपक्रम्य सोममेवाभिसंश्रिताः}
{यथोत्पन्नाः स्ववर्णस्था न नीताश्चान्यवर्णतां}


\twolineshloka
{अथ क्रुद्धं महादेवं प्रजापतिरभाषत}
{अमृतेनावसिक्तस्त्वं नोच्छिष्टं विद्यते गवाम्}


\twolineshloka
{यथा ह्यमृतमादाय सोमो विष्यन्दते पुनः}
{तथा क्षीरं क्षरन्त्येता रोहिण्योऽमृतसम्भवाः}


\twolineshloka
{न दुष्यत्यनिलो नाग्निर्न सुवर्णं न चोदधिः}
{नामृतेनामृतं पीतं न वत्सैर्दुष्यते पयः}


\twolineshloka
{इमाँल्लोकान्भरिष्यन्ति हविषा प्रस्रवेण च}
{आसामैश्वर्यमिच्छन्ति सर्वेऽमृतमयं शुभम्}


\twolineshloka
{वृषभं च ददौ तस्मै भगवाँल्लोकभावनः}
{प्रसादयामास मनस्तेन रुद्रस्य भारत}


\twolineshloka
{प्रीतश्चापि महादेवश्चकार वृषभं तदा}
{ध्वजं च वाहनं चैव तस्मात्स वृषभध्वजः}


\twolineshloka
{ततो देवैर्महादेवस्तदा पशुपतिः कृतः}
{ईश्वरः स गवां मध्ये वृषभाङ्कः प्रकीर्तितः}


\twolineshloka
{एवमव्यग्रवर्णानां कपिलानां महौजसाम्}
{प्रदाने प्रथमः कल्पः सर्वासामेव कीर्तितः}


\twolineshloka
{लोकज्येष्ठा लोकवृत्तिप्रवृत्तारुद्रोद्भूताः सोमविष्यन्दभूताः}
{सौम्याः पुण्याः कामदाः प्राणदाश्चगा वै दत्त्वा सर्वकामप्रदः स्यात्}


\twolineshloka
{इदं गवां प्रभवविधानमुत्तमंपठन्सदा शुचिरपि मङ्गलप्रियः}
{विमुच्यते कलिकलुषेण मानवःप्रियान्सुतान्धनपशुमाप्नुयात्सदा}


\threelineshloka
{हव्यं कव्यं तर्पणं शान्तिकर्मयानं वासो वृद्धबालस्य तुष्टिः}
{एतान्सर्वान्गोप्रदाने गुणान्वैदाता राजन्नाप्नुयाद्वै सदैव ॥वैशम्पायन उवाच}
{}


\twolineshloka
{पितामहस्याथ निशम्य वाक्यंराजा सह भ्रातृभिराजमीढः}
{सुवर्णवर्णानडुहस्तथा गाःपार्थो ददौ ब्राह्मणसत्तमेभ्यः}


\twolineshloka
{तथैव तेभ्योपि ददौ द्विजेभ्योगवां सहस्राणि शतानि चैव}
{यज्ञान्समुद्धिश्य च दक्षिणार्थेलोकान्विजेतुं परमां च कीर्तिम्}


\chapter{अध्यायः ११३}
\twolineshloka
{एवमुक्त्वा ततो भीष्मः पुनर्धर्मसुतं नृपम्}
{जनमेजयभूपाल उवाचेदं सहेतुकम्}


\twolineshloka
{एतस्मिन्नेव काले तु वसिष्ठमृषिसत्तमम्}
{इक्ष्वाकुवंशजो राजा सौदासो वदतां वरः}


\twolineshloka
{सर्वलोकचरं सिद्धं ब्रह्मकोशं सनातनम्}
{पुरोहितमभिप्रष्टुमभिवाद्योपचक्रमे}


\threelineshloka
{त्रैलोक्ये भगवन्किंस्वित्पविइत्रं कथ्यतेऽनघ}
{यत्कीर्तयन्सदा मर्त्यः प्राप्नुयात्पुण्यमुत्तमम् ॥भीष्म उवाच}
{}


\twolineshloka
{तस्मै प्रोवाच वचनं प्रणताय हितं तदा}
{गवामुपनिषद्विद्यां नमस्कृत्य गवां शुचिः}


\twolineshloka
{गावः सुरभिगन्धिन्यस्तथा गुग्गुलुगन्धयः}
{गावः प्रतिष्ठा भूतानां गावः स्वस्त्ययनं महत्}


\twolineshloka
{गावो भूतं च भव्यं च गावः पुष्टिः सनातनी}
{गावो लक्ष्म्यास्तथा मूलं गोषु दत्तं न नश्यति}


\twolineshloka
{अन्नं हि परमं गावो देवानां परमं हविः}
{स्वाहाकारवषट्कारौ गोषु नित्यं प्रतिष्ठितौ}


\twolineshloka
{गावो यज्ञस्य हि फलं गोषु यज्ञाः प्रतिष्ठिताः}
{गावो भविष्यं भूतं च गोषु यज्ञाः प्रतिष्ठिताः}


\twolineshloka
{सायं प्रातश्च सततं होमकाले महाद्युते}
{गावो ददति वै हौम्यमृषिभ्यः पुरुषर्षभ}


\twolineshloka
{यानि कानि च दुर्गाणि दुष्कृतानि कृतानि च}
{तरन्ति चैव पाप्मानं धेनुं ये ददति प्रभो}


\twolineshloka
{एकां च दशगुर्दद्याद्दश दद्याच्च गोशती}
{शतं सहस्रगुर्दद्यात्सर्वे तुल्यफला हि ते}


\twolineshloka
{अनाहिताग्निः शतगुरयज्वा च सहस्रगुः}
{समृद्धो यश्च कीनाशो नार्घमर्हन्ति ते त्रयः}


\twolineshloka
{कपिलां ये प्रयच्छन्ति सवत्सां कांस्यदोहनाम्}
{सुव्रतां वस्त्रसंवीतामुभौ लोकौ जयन्ति ते}


\twolineshloka
{युवानमिन्द्रियोपेतं शतेन सहयूथपम्}
{गवेन्द्रं ब्राह्मणेन्द्राय भूरिशृङ्गमलङ्कृतम्}


\twolineshloka
{वृषबं ये प्रयच्छन्ति श्रोत्रियाय परन्तप}
{ऐश्वर्यं तेऽधिगच्छन्ति जायमानाः पुनःपुनः}


\twolineshloka
{नाकीर्तयित्वा गाःक सुप्यात्तासां संस्मृत्य चोत्पतेत्}
{सायं प्रातर्नमस्येच्च गास्ततः पुष्टिमाप्नुयात्}


\twolineshloka
{गवां मूत्रपुरीषस्य नोद्विजेत कथञ्चन}
{न चासां मांसमश्नीयाद्गवां पुष्टिं तथाऽऽप्नुयात्}


\twolineshloka
{गाश्च संकीर्तयन्नित्यं नावमन्येत तास्तथा}
{अनिष्टं स्वप्नमालक्ष्य गां नरः सम्प्रकीर्तयेत्}


\twolineshloka
{गोमयेन सदा स्नायाद्गोकरीषे च संविशेत्}
{श्लोष्ममूत्रपुरीषाणि प्रतिघातं च वर्जयेत्}


\twolineshloka
{सार्द्रे चर्मणि भुञ्जीत निरीक्षेद्वारुणीं दिशम्}
{वाग्यतः सर्पिषा भूमौ गवां व्युष्टिं सदाऽऽश्नुते}


\twolineshloka
{घृतेन जुहुयादग्निं घृतेन स्वस्ति वाचयेत्}
{घृतं दद्याद्धृतं प्राशेद्गवां व्युष्टिं सदाऽऽश्नुते}


\twolineshloka
{गोमत्या विद्यया धेनुं तिलानामभिमन्त्र्य यः}
{सर्वरत्नमयीं दद्यान्न स शोचेत्कृताकृते}


\twolineshloka
{गावो मामुपतिष्ठन्तु हेमशृङ्ग्यः पयोमुचः}
{सुरभ्यः सौरभेय्यश्च सरितः सागरं यथा}


\twolineshloka
{गा वै पश्याम्यहं नित्यं गावः पश्यन्तु मां सदा}
{गावोस्माकं वयं तासां यतो गावस्ततो वयम्}


\twolineshloka
{एवं रात्रौ दिवा चापि समेषु विषमेषु च}
{महाभयेषु च नरः कीर्तयन्मुच्यते भवात्}


\chapter{अध्यायः ११४}
\twolineshloka
{शतं वर्षसहस्राणां तपस्तप्तं सुदुष्करम्}
{गोभिः पूर्वं विसृष्टाभिर्गच्छेम श्रेष्ठतामिति}


\twolineshloka
{लोकेऽस्मिन्दक्षिणानां च सर्वासां वयमुत्तमाः}
{भवेम न च लोप्येम दोषेणेति परन्तप}


\twolineshloka
{अस्मत्पुरीषस्नानेन जनः पुयेत सर्वदा}
{शकृता च पवित्रार्थं कुर्वीरन्देवमानुषाःठ}


\twolineshloka
{तथा सर्वाणि भूतानि स्थावराणि चराणि च}
{प्रदातारश्च लोकान्नो गच्छेयुरिति मानद}


\twolineshloka
{ताभ्यो वरं ददौ ब्रह्मा तपसोऽन्ते स्वयं प्रभुः}
{एवं भवन्विति विभुर्लोकांस्तारयतेति च}


\twolineshloka
{उत्तस्थुः सिद्धकामास्ता भूतभव्यस्य मातरः}
{तपसोऽन्ते महाराज गावो लोकपरायणाः}


\twolineshloka
{तस्माद्गावो महाभागाः पवित्रं परमुच्यते}
{तथैव सर्वभूतानां समतिष्ठन्त मूर्धनि}


% Check verse!
सम्मनवत्सां कपिलां धेनुं दत्त्वा पयस्विनीम् ॥सुव्रतां वस्त्रसंवीतां ब्रह्मलोके महीयते
\twolineshloka
{लोहितां तुल्यवत्सां तु धेनुं दत्त्वा पयस्विनीम्}
{सुव्रतां वस्त्रसंवीतां सूर्यलोके महीयते}


\twolineshloka
{समानवत्सां शबलां धेनुं दत्त्वा पयस्विनीम्}
{सुव्रतां वस्त्रसंवीतीं सोमलोके महीयते}


\twolineshloka
{समानवत्सां श्वेतां तु धेनुं दत्त्वा पयस्विनीम्}
{सुव्रतां वस्त्रसंवीतामिन्द्रलोके महीयते}


\twolineshloka
{समानवत्सां कृष्णां तु धेनु दत्त्वा पयस्विनीम्}
{सुव्रतां वस्त्रसंवीतामग्निलोके महीयते}


\twolineshloka
{समानवत्सां धूम्रां तु धेनुं दत्त्वा पयस्विनीम्}
{सुव्रतां वस्त्रसंवीतां याम्यलोके महीयते}


\twolineshloka
{अपां फेनसवर्णां तु सवत्सां कांस्यदोहनाम्}
{प्रदाय वस्त्रसंवीतां वारुणं लोकमाप्नुते}


\twolineshloka
{वातरेणुसवर्णां तु सवत्सां कांस्यदोहनाम्}
{प्रदाय वस्त्रसंवीतां वायुलोके महीयते}


\twolineshloka
{हिरण्यवर्णां पिङ्गाक्षीं सवत्सां कांस्यदोहनाम्}
{प्रदाय वस्त्रसंवीतां कौबेरं लोकमश्नुते}


\twolineshloka
{पलालधूम्रवर्णां तु सवत्सां कांस्यदोहनाम्}
{प्रदाय वस्त्रसंवीतां पितृलोके महीयते}


\twolineshloka
{सवत्सां पीवरीं दत्त्वा दृतिकण्ठामलङ्कृताम्}
{वैश्वदेवमसम्बाधं स्थानं श्रेष्ठं प्रपद्यते}


\twolineshloka
{समानवत्सां गौरीं तु धेनुं दत्त्वा पयस्विनीम्}
{सुव्रतां वस्त्रसंवीतां वसूनां लोकमाप्नुयात्}


\twolineshloka
{पाण्डुकम्बलवर्णाभां सवत्सां कांस्यकदोहनाम्}
{प्रदाय वस्त्रसंवीतां साध्यानां लोकमाप्नुते}


\twolineshloka
{वैराटपृष्ठमुक्षाणं सर्वरत्नैरलङ्कृतम्}
{प्रददन्मरुतां लोकान्स राजन्प्रतिपद्यते}


\twolineshloka
{वत्सोपपन्नां नीलां गां सर्वरत्नसमन्विताम्}
{गन्धर्वाप्सरसां लोकान्दत्त्वा प्राप्नोति मानवः}


\twolineshloka
{दृतिकण्ठमनड्वाहं सर्वरत्नैरलङ्कृतम्}
{दत्त्वा प्रजापतेर्लोकान्विशोकः प्रतिपद्यते}


\twolineshloka
{गोप्रदानरतो याति भित्त्वा जलदसञ्चयान्}
{विमानेनार्कवर्णेन दिवि राजन्विराजता}


\twolineshloka
{तं चोरुवेषाः सुश्रोण्यः सहस्रं वारयोपितः}
{रमयन्ति नरश्रेष्ठं गोप्रदानरतं नरम्}


\twolineshloka
{वीणानां वल्लकीनां च नू पुराणां च शिञ्जितैः}
{हासैश्च हरिणाक्षीणां सुप्तः सुप्रतिबोध्यते}


\twolineshloka
{यावन्ति रोमाणि भवन्ति धेन्वा-स्तावन्ति वर्षाणि महीयते सः}
{स्वर्गच्युतश्चापि ततो नृलोकेकुले प्रसूतो विपुले विशोकः}


\chapter{अध्यायः ११५}
\twolineshloka
{घृतक्षीरप्रदा गावो घृतयोन्यो घृतोद्भवाः}
{घृतनद्यो घृतावर्तास्ता मे सन्तु सदा गृहे}


\twolineshloka
{घृतं मे हृदये नित्यं घृतं नाभ्यां प्रतिष्ठितम्}
{घृतं सर्वेषु गात्रेषु घृतं मे मनसि स्थितम्}


\twolineshloka
{गावो ममाग्रतो नित्यं गावः पृष्ठत एव च}
{गावो मे सर्वतश्चैव गवां मध्ये वसाम्यहम्}


\twolineshloka
{इत्याचम्य जपेत्सायं प्रातश्च पुरुषः सदा}
{यदह्ना कुरुते पापं तस्मात्स परिमुच्यते}


\twolineshloka
{प्रासादा यत्र सौवर्णा वसोर्धाराश्च कामदाः}
{गन्धर्वाप्सरसो यत्र तत्र यान्ति सहस्रदाः}


\twolineshloka
{नवनीतपङ्काः क्षीरोदा दधिशैवलसङ्कुलाः}
{वहन्ति यत्र वै नद्यस्तत्र यान्ति सहस्रदाः}


\twolineshloka
{गवां शतसहस्रं तु यः प्रच्छेद्यथाविधि}
{परां वृद्धिमवाप्याथ स्वर्गलोके महीयते}


\twolineshloka
{दश चोभयतः प्रेत्य मातापित्रोः पितामहान्}
{दधाति सुकृताँल्लोकान्पुनाति च कुलं नरः}


\twolineshloka
{धेन्वाः प्रमाणेन समप्रमाणांधेनुं तिलानामपि च प्रदाय}
{पानीयवापीः स यमस्य लोकेन यातनां काञ्चिदुपैति तत्र}


\twolineshloka
{पवित्रमग्र्यं जगतः प्रतिष्ठादिवौकसां मातरोऽथाप्रमेयाः}
{अन्वालभेद्दक्षिणतो व्रजेच्चदद्याच्च पात्रे प्रसमीक्ष्य कालम्}


\twolineshloka
{धेनुं सवत्सां कपिलां भूरिशृङ्गींकांस्योपदोहां वसनोत्तरीयाम्}
{प्रदाय तां गाहति दुर्विगाह्यांयाम्यां सभां वीतभयो मनुष्यः}


\twolineshloka
{सुरूपा बहुरूपाश्चि विश्वरूपाश्च मातरः}
{गावो मामुपतिष्ठन्तामिति नित्यं प्रकीर्तयेत्}


\twolineshloka
{नातः पुण्यतरं दानं नातः पुण्यतरं फलम्}
{नातो विशिष्टं लोकेषु भूतं भवितुमर्हति}


\twolineshloka
{त्वचा लोम्नाऽथ शृङ्गैर्वा वालैः क्षीरेण मेदसा}
{यज्ञं वहति सम्भूय किमस्त्यभ्यधिकं ततः}


\twolineshloka
{यया सर्वमिदं व्याप्तं जगत्स्थावरजङ्गमम्}
{तां धेनुं शिरसा वन्दे भूतभव्यस्य मातरम्}


\threelineshloka
{गुणवचनसमुच्चयैकदेशोनृवर मयैष गवां प्रकीर्तितस्ते}
{न च परमिह दानमस्ति गोभ्योभवति न चापि परायणं ततोऽन्यम् ॥भीष्म उवाच}
{}


\twolineshloka
{वरमिदमिति भूमिपो विचिन्त्यप्रवरमृषेर्वचनं ततो महात्मा}
{व्यसृजत नियतात्मवान्द्विजेभ्यःसुबहु च गोधनमाप्तवांश्च लोकान्}


\chapter{अध्यायः ११६}
\threelineshloka
{पवित्राणां पवित्रं यच्छ्रेष्ठं लोके च यद्भवेत्}
{पावनं परमं चैव तन्मे ब्रूहि पितामह ॥भीष्म उवाच}
{}


\twolineshloka
{गावो महार्थाः पुण्याश्च तारयन्ति च मानवान्}
{धारयन्ति प्रजाश्चेमा हविषा पयसा तथा}


\twolineshloka
{न हि पुण्यतमं किञ्चिद्गोभ्यो भरतसत्तम}
{एताः पुण्याः पवित्राश्च त्रिषु लोकेषु सत्तमाः}


\twolineshloka
{देवानामुपरिष्टाच्च गावः प्रतिवसन्ति वै}
{दत्त्वा चैतास्तारयते यान्ति स्वर्गं मनीषिणः}


\twolineshloka
{मान्धाता यौवनाश्वश्च ययातिर्नहुषस्तथा}
{गा वै ददन्तः सततं सहस्रशतसम्मिताः}


\twolineshloka
{गताः परमकं स्थानं देवैरपि सुदुर्लभम्}
{अपि चात्र पुरावृत्तं कथयिष्यामि तेऽनघ}


\twolineshloka
{ऋषीणामुत्तमं धीमान्कृष्णद्वैपायनं शुकः}
{अभिवाद्याह्निकं कृत्वा शुचिः प्रयतमानसः}


\twolineshloka
{पितरं परिपप्रच्छ दृष्टलोकपरावरम्}
{को यज्ञः सर्वयज्ञानां वरिष्ठोऽभ्युपलक्ष्यते}


\twolineshloka
{किञ्च कृत्वा परं स्थानं प्राप्नुवन्ति मनीषिणः}
{केन देवाः पवित्रेण स्वर्गमश्नन्ति वा विभो}


\threelineshloka
{किञ्च यज्ञस्य यज्ञत्वं क्व च यज्ञः प्रतिष्ठितः}
{दानानामुत्तमं किञ्च किञ्च सत्रमितः परम्}
{पवित्राणां पवित्रं च यत्तद्ब्रूहि महामुने}


\twolineshloka
{एतच्छ्रुत्वा तु वचनं व्यासः परमधर्मवित्}
{पुत्रायाकथयत्सर्वं तत्त्वेन भरतर्षभ}


\twolineshloka
{गावः प्रतिष्ठा भूतानां तथा गावः परायणम्}
{गावः पुण्याः पवित्राश्च गोधनं पावनं तथा}


\twolineshloka
{पूर्वमासन्नंशृङ्गा वै गाव इत्यनुशश्रुम}
{शृङ्गार्थे समुपासन्त ताः किल प्रभुमव्ययम्}


\twolineshloka
{ततो ब्रह्मा तु गाः सत्रमुपविष्टाः समीक्ष्य ह}
{ईप्सितं प्रददौ ताभ्यो गोभ्यः प्रत्येकशः प्रभुः}


\twolineshloka
{तासां शृङ्गाण्यजायन्त तस्या यादृङ्भनोगतम्}
{नानावर्णाः शृङ्गवन्त्यस्ता व्यरोचन्त पुत्रक}


\twolineshloka
{ब्रह्मणा वरदत्तास्ता हव्यकव्यप्रदाः शुभाः}
{पुण्याः पवित्राः सुभगा दिव्यसंस्थानलक्षणाः}


\twolineshloka
{गावस्तेजो महद्दिव्यं गवां दानं प्रशस्यते}
{ये चैताः सम्प्रयच्छन्ति साधवो वीतमत्सराः}


\twolineshloka
{ते वै सुकृतिनः प्रोक्ताः सर्वदानप्रदाश्च ते}
{गवां लोकं तथा पुण्यमाप्नुवन्ति च तेऽनघ}


\twolineshloka
{यत्र वृक्षा मधुफला दिव्यपुष्पफलोपगाः}
{पुष्पाणि च सुगन्धीनि दिव्यानि द्विजसत्तम}


\twolineshloka
{सर्वा मणिभयी भूमिः सर्वकाञ्चनवालुकाः}
{सर्वर्तुसुखसंस्पर्शा निष्पङ्का निरजाः शुभाः}


\twolineshloka
{रक्तोत्पलवनैश्चैव मणिखण्डैर्हिरण्मयैः}
{तरुणादित्यसङ्काशैर्भान्ति तत्र जलाशयाः}


\twolineshloka
{महार्हमणिपत्रैश्च काञ्चनप्रभकेसरैः}
{नीलोत्पलविमिश्रैश्च सरोभिर्बहुपङ्कजैः}


\twolineshloka
{करवीरवनैः फुल्लैः सहस्रावर्तसंवृतैः}
{सन्तानकवनैः फुल्लैर्वृक्षेश्च समलङ्कृताः}


\twolineshloka
{निर्मिलाभिश्च मुक्ताभिर्मणिभिश्च महाप्रभैः}
{उद्भूतपुलिनास्तत्र जातरूपैश्चि निम्नगाः}


\twolineshloka
{सर्वरत्नमयैश्चित्रैरवगाढा द्रुमोत्तमैः}
{जातरूपमयैश्चान्यैर्हुताशनसमप्रभैः}


\twolineshloka
{सौवर्णि गिरयस्तत्र मणिरत्नशिलोच्चयाः}
{सर्वरत्नमयैर्भान्ति शृङ्गैश्चारुभिरुच्छ्रितैः}


\twolineshloka
{नित्यपुष्पफलास्तत्र नगाः पत्ररथाकुलाः}
{दिव्यगन्धरसैः पुष्पैः फलैश्च भरतर्षभ}


\twolineshloka
{रमन्ते पुण्यकर्माणस्तत्र नित्यं युधिष्ठिर}
{सर्वकामसमृद्धार्था निःशोका गतमन्यवः}


\twolineshloka
{विमानेषु विचित्रेषु रमणीयेषु भारत}
{मोदन्ते पुण्यकर्माणो विरहन्तो यशस्विनः}


\twolineshloka
{उपक्रीडन्ति तान्राजञ्शुभाश्चाप्सरसां गणाः}
{एतान्लोकानवाप्नोति गां दत्त्वा वै युधिष्ठिर}


\twolineshloka
{यासामधिपतिः पूषा मारुतो बलवान्बले}
{ऐश्वर्ये वरुणो राजा ता मां पान्तु युगन्धराः}


\twolineshloka
{सुरूपा बहुरूपाश्च विश्वरूपाश्च मातरः}
{प्राजापत्या इति ब्रह्मञ्जपेन्नित्यं यतव्रतः}


\twolineshloka
{गाश्च शुश्रूषते यश्च समन्वेति च सर्वशः}
{तस्मै तुष्टाः प्रयच्छन्ति वरानपि सुदुर्लभान्}


\twolineshloka
{द्रुह्येन मनसा वाऽपि गोषु ता हि सुखप्रदाः}
{अर्चयेत सदा चैव नमस्कारैश्च पूजयेत्}


\twolineshloka
{दान्तः प्रीतमना नित्यं गवां व्युष्टिं तथाऽश्नुते}
{त्र्यहमुष्णं पिबेन्मूत्रं त्र्यहमुष्णं पिबेत्पयः}


\twolineshloka
{गवामुष्णं पयः पीत्वा त्र्यहमुष्णं घृतं विबेत्}
{त्र्यहमुष्णं घृतं पीत्वा वायुभक्षो भवेत्र्यहम्}


\twolineshloka
{योन देवाः पवित्रेणि भुञ्जते लोकमुत्तमम्}
{यत्पवित्रं पवित्राणां तद्धृतं शिरसा वहेत्}


\twolineshloka
{घृतेन जुहुयादग्निं घृतेन स्वस्ति वाचयेत्}
{घृतं प्राशेद्धृतं दद्याद्गवां पुष्टिं तथाऽश्नुते}


\twolineshloka
{निर्हृतैश्च यवैर्गोभिर्मासं प्रश्रितयावकः}
{ब्रह्महत्यासमं पापं सर्वमेतेन शुध्यते}


% Check verse!
पराभवार्थं दैत्यानां देवैः शौचमिदं कृतम् ॥ते देवत्वमपि प्राप्ताः संसिद्धाश्च महाबलाः
\twolineshloka
{गावः पवित्राः पुण्याश्च पावनं परमं महत्}
{ताश्च दत्त्वा द्विजातिभ्यो नरः स्वर्गमुपाश्नुते}


\twolineshloka
{गवां मध्ये शुचिर्भूत्वा गोमतीं मनसा जपेत्}
{पूताभिरद्भिराचम्यि शुचिर्भवति निर्मलः}


\twolineshloka
{अग्निमध्ये गवां मध्ये ब्राह्मणानां च संसदि}
{विद्यावेदप्रतस्नाता ब्राह्मणाः पुण्यकर्मिणः}


\twolineshloka
{अध्यापयेरञ्शिष्यान्वै गोमतीं यज्ञसम्मिताम्}
{त्रिरात्रोपोषितो भूत्वा गोमतीं लभते वरम्}


% Check verse!
पुत्रिकामश्च लभते पुत्रं धनमथापि वा ॥पतिकामा च भर्तारं सर्वकामांश्च मानवः
\threelineshloka
{गावस्तुष्टाः प्रयच्छन्ति सेविता वै न संशयः}
{एवमेतां महाभागा यज्ञियाः सर्वकामदाः}
{रोहिण्य इति जानीहि नैताभ्यो विद्यते परम्}


\twolineshloka
{इत्युक्तः स महातेजाः शुकः पित्रा महात्मना}
{पूजयामास तां नित्यं तस्मात्त्वमपि पूजय}


\chapter{अध्यायः ११७}
\twolineshloka
{सुराणामसुराणां च भूतानां च पितामह}
{प्रभुः स्रष्टा च भगवान्मुनिभिः स्तूयते भुवि}


\threelineshloka
{तस्योपरि कथं ह्येष गोलोकः स्थानतां गतः}
{संशयो मे महानेष तन्मे व्याख्यातुमर्हसि ॥भीष्म उवाच}
{}


\twolineshloka
{मनोवाग्बुद्धयस्तावदेकस्थाः कुरुसत्तम}
{ततो मे शृणु कार्त्स्न्येन गोमहाभाग्यमुत्तमम्}


\twolineshloka
{पुण्यं यशस्यमायुष्यं तथा स्वस्त्ययनं महत्}
{कीर्तिर्विहरता लोके गवां यो गोषु भक्तिमान्}


\twolineshloka
{श्रूयते हि पुराणेषु महर्षीणां महात्मनाम्}
{संस्थाने सर्वलोकानां देवानां चापि सम्भवे}


\twolineshloka
{देवतार्थेऽमृतार्थे च यज्ञार्थे चैव भारत}
{सुरभिर्नाम विख्याता रोहिणी कामरूपिणी}


\twolineshloka
{सङ्कल्प्य मनसा पूर्वं रोहिणी ह्यमृतात्मना}
{घोरं तपः समास्थाय निर्मिती विस्वकर्मणा}


\twolineshloka
{पुरुषं चासृजद्भूस्तेजसा तपसा च ह}
{देदीप्यमानं वपुषा समिद्धमिव पावकम्}


\twolineshloka
{सोऽपश्यदिष्टरूपां तां सुरभिं रोहिणीं तदा}
{दृष्ट्वैव चातिविमनाः सोऽभवत्काममोहितः}


\twolineshloka
{तं कामार्तमथो ज्ञात्वा स्वयंभूर्लोकभावनः}
{माऽऽर्तो भव तथा चैष भगवानभ्यभाषत}


\twolineshloka
{ततः स भगवांस्तत्र मार्ताण्ड इति विश्रुतः}
{चकार नाम तं दृष्ट्वा तस्यार्तीभावमुत्तमम्}


\twolineshloka
{सोऽददाद्भगवांस्तस्मै मार्ताण्डाय महात्मने}
{सुरूपां सुरभिं कन्यां तपस्तेजोमयीं शुभाम्}


\twolineshloka
{यथा मयैष चोद्भूतस्त्वं चैवषा च रोहिणी}
{मैथुनं गतवन्तौ च तथा चोत्पत्स्यति प्रजा}


\twolineshloka
{प्रजा भविष्यते पुण्या पवित्रं परमं च वाम्}
{न चाप्यगम्यागमनाद्दोषं प्राप्स्यसि कर्हिचित्}


\twolineshloka
{त्वत्प्रजासम्भवं क्षीरं भविष्यति परं हविः}
{यज्ञेषु चाज्यभागानां त्वत्प्रजामूलजो विधिः}


\twolineshloka
{प्रजाशुश्रूषवश्चैव ये भविष्यन्ति रोहिणि}
{तव तेनैव पुण्येनि गोलोकं यान्तु मानवाः}


\twolineshloka
{इदं पवित्रं परममृषभं नाम कर्हिचित्}
{यद्वै ज्ञात्वा द्विजा लोके मोक्ष्यन्ते योनिसङ्करात्}


\twolineshloka
{एतत्क्रियाः प्रवर्तन्ते मन्त्रब्राह्मणसंस्कृताः}
{देवतानां वितॄणां च हव्यकव्यपुरोगमाः}


\twolineshloka
{तत एतेन पुण्येन प्रजास्तव तु रोहिणि}
{ऊर्ध्वं ममापि लोकस्य वत्स्यन्ते निरुपद्रवाः}


\twolineshloka
{भद्रं तेभ्यश्च भद्रं ते ये प्रजासु भवन्ति वै}
{युगन्धराश्च ते पुत्राः सन्तु लोकस्य धारणे}


\threelineshloka
{यान्यान्कामयसे लोकांस्ताँल्लोकाननुयास्यसि}
{सर्वदेवगणश्चैव तव यास्यन्ति पुत्रताम्}
{तव स्तनसमुद्भूतं पिबन्तोऽमृतमुत्तमम्}


\twolineshloka
{एवमेतान्वरान्सर्वानगृह्णात्सुरभिस्तदा}
{ब्रुवतः सर्वलोकेशान्निर्वृतिं चागमत्पराम्}


\twolineshloka
{सृष्ट्वा प्रजाश्च विपुला लोकसन्धारणाय वै}
{ब्रह्मणा समनुज्ञाता सुरभिर्लोकमाविशत्}


\twolineshloka
{एवं वरप्रदानेन स्वयंभोरेव भारत}
{उपरिष्टाद्गवां लोकः प्रोक्तस्ते सर्वमादितः ॥'}


\chapter{अध्यायः ११८}
\threelineshloka
{सुरभेः काः प्रजाः पूर्वं मार्ताण्डादभवन्पुरा}
{एतन्मे शंस तत्वेनि गोषु मे प्रीयते मनः ॥भीष्म उवाच}
{}


\twolineshloka
{शृणु नामानि दिव्यानि गोमातॄणां विशेषतः}
{याभिर्व्याप्तास्त्रयो लोकाः कल्याणीभिर्जनाधिप}


\twolineshloka
{सुरभ्यः प्रथमोद्भूता याश्च स्युः प्रथमाः प्रजाः}
{मयोच्यमानाः शृणु ताः प्राप्स्यसे विपुलं यशः}


\twolineshloka
{तप्त्वा तपो घोरतपाः सुरभिर्दिप्ततेजसः}
{सुषावैकादश सुतान्रुद्रा ये च्छन्दसि स्तुताः}


\twolineshloka
{अजैकपादहिर्बुध्न्यस्त्र्यम्बकश्च महायशाः}
{वृषाकपिश्च शम्भुश्च कपाली रैवतस्तथा}


\twolineshloka
{हरश्च बहुरूपश्च उग्र उग्रोऽथ वीर्यवान्}
{तस्य चैवात्मजः श्रीमान्विश्वरूपो महायशाः}


\twolineshloka
{एकादशैते कथिता रुद्रास्ते नाम नामतः}
{महात्मानो महायोगास्तेजोयुक्ता महाबलाः}


\twolineshloka
{एते वरिष्ठजन्मानो देवानां ब्रह्मवादिनाम्}
{विप्राणां प्रकृतिर्लोके एत एव हि विश्रुताः}


\twolineshloka
{एत एकादश प्रोक्ता रुद्रास्त्रिभुवनेश्वराः}
{शतं त्वेतत्समाख्यातं शतरुद्रं महात्मनाम्}


\twolineshloka
{सुषुवे प्रथमां कन्यां सुरभिः पृथिवीं तदा}
{विश्वकामदुघा धेनुर्या धारयति देहिनः}


\twolineshloka
{सुतं गोब्राह्मणं राजन्नेकमित्यभिधीयते}
{गोब्राह्मणस्य जननी सुरभिः परिकीर्त्यते}


\twolineshloka
{सृष्ट्वा तु प्रथमं रुद्रान्वरदान्रुद्रसम्भवान्}
{पश्चात्प्रभुं ग्रहपतिं सुषुवे लोकसम्मतम्}


\twolineshloka
{सोमराजानममृतं यज्ञसर्वस्वमुत्तमम्}
{ओषधीनां रसानां च देवानां जीवितस्य च}


\twolineshloka
{ततः श्रियं च मेधां च कीर्तिं देवीं सरस्वतीम्}
{चतस्रः सुषुवे कन्या योगेषु नियताः स्तिताः}


\twolineshloka
{एताः सृष्ट्वा प्रजा एषा सुरभिः कामरूपिणी}
{सुषुवे परमं भूयो दिव्या गोमातरः शुभाः}


\twolineshloka
{पुण्यां मायां मधुश्च्योतां शिवां शीघ्रां सरिद्वराम्}
{हिरण्यवर्णां सुभगां गव्यां पृश्नीं कुथावतीम्}


\twolineshloka
{अङ्गावतीं घृतवतीं दधिक्षीरपयोवतीम्}
{अमोघां सुरमां सत्यां रेवतीं मारुतीं रसाम्}


\twolineshloka
{अजां च सिकतां चैव शुद्धधूमामधारिणीम्}
{जीवां प्राणवतीं धन्यां शुद्धां धेनुं धनावहाम्}


\twolineshloka
{इन्द्रामृद्धिं च शान्ति च शान्तपापां सरिद्वराम्}
{चत्वारिंशतिमेकां च धन्यास्ता दिवि पूजिताः}


\threelineshloka
{भूयो जज्ञे सुरभ्याश्च श्रीमांश्चन्द्रांशुसप्रभः}
{वृपो दक्ष इति ख्यातः कण्ढे मणितलप्रभः}
{}


\twolineshloka
{स्रग्वी ककुद्मान्द्युतिमान्मृणालसदृशप्रभः}
{सुरभ्यनुमते दत्तो ध्वजो माहेशअवरस्तु सः}


\twolineshloka
{सुरभ्यः कामरूपिण्यो गावः पुण्यार्थमुत्कटाः}
{आदित्येभ्यो वसुभ्यश्च विश्वेभ्यश्च ददौ वरान्}


\twolineshloka
{सुरभिस्तु तपस्तप्त्वा सुषुवे गास्ततः पुनः}
{या दत्ता लोकपालानामिन्द्रादीनां युधिष्ठिर}


\twolineshloka
{सुष्टुनां कपिलां चैव रोहिणीं च यशस्विनीम्}
{सर्वकामदुघां चैव मरुतां कामरूपिणीम्}


\twolineshloka
{गावो मृष्टदुघा ह्येतास्ताश्चतस्रोऽत्र संस्तुताः}
{यासां भूत्वा पुरा वत्साः पिबन्त्यमृतमुत्तमम्}


\twolineshloka
{सुष्टुतां देवराजाय वासवाय महात्मने}
{कपिलां धर्मराजाय वरुणाय च रोहिणीम्}


\twolineshloka
{सर्वकामदुघां धेनुं राज्ञे वैश्रवणाय च}
{इत्येता लोकमहिता विश्रुताः सुरभेः प्रजाः}


\twolineshloka
{एतासां प्रजया पूर्णा पृथिवी मुनिपुङ्गवः}
{गोभ्यः प्रभवते सर्वं यत्किंचिदिह शोभनम्}


\twolineshloka
{सुरभ्यपत्यमित्येतन्नामतस्तेऽनुपूर्वशः}
{कीर्तितं ब्रूहि राजेन्द्र किं भूयः कथयामि ते'}


\chapter{अध्यायः ११९}
\twolineshloka
{सुरभ्यास्तु तदा देव्याः कीर्तिर्लक्ष्मीः सरस्वती}
{मेधा च प्रवरा देवी याश्चतस्रोऽभिविश्रुताः}


\threelineshloka
{पृथग्गोभ्यः किमेताः स्युरुताहो गोषु संश्रिताः}
{देवाः के वाऽऽश्रिता गोषु तन्मे ब्रूहि पितामह ॥भीष्म उवाच}
{}


\twolineshloka
{यं देवं संश्रिता गावस्तं देवं देवसंज्ञितम्}
{यद्यद्देवाश्रितं दैवं तत्तद्दैवं द्विजा विदुः}


\threelineshloka
{सर्वेषामेव देवानां पूर्वं किल समुद्भवे}
{अमृतार्थे सुरपतिः सुरभिं समुपस्थितः ॥इन्द्र उवाच}
{}


\threelineshloka
{इच्छेयममृतं दत्तं त्वया देवि रसाधिकम्}
{त्वत्प्रसादाच्छिवं मह्यममरत्वं भवेदिति ॥सुरभिरुवाच}
{}


\twolineshloka
{वत्सो भूत्वा सुपरते पिबस्व प्रस्रवं मम}
{ततोऽमरत्वमपि तस्थानमैन्द्रमवाप्स्यसि}


\threelineshloka
{न च ते वृत्रहन्युद्धे व्यथाऽरिभ्यो भविष्यति}
{बलार्थमात्मनः शक्र प्रस्रवं पिब मे विभो ॥भीष्म उवाच}
{}


\twolineshloka
{ततोऽपिबत्स्नं तस्याः सुरभ्याः सुरसत्तमः}
{अमरत्वं सुरूपत्वं बलं चापदनुत्तमम्}


\threelineshloka
{पुरंदरोऽमृतं पीत्वा प्रहृष्टः समुपस्थितः}
{पुत्रोऽहं तव भद्रं ते ब्रूहि किं करवाणि ते ॥सुरभिरुवाच}
{}


\twolineshloka
{कृतं पुत्र त्वया सर्वमुपयाहि त्रिविष्टपम्}
{पालयस्व सुरान्सर्वाञ्जहि ये सुरशत्रवः}


\twolineshloka
{न च गोब्राह्मणेऽवज्ञा कार्या ते शान्तिमिच्छता}
{गोब्राह्मणस्य निश्वासः शोषयेदपि देवताः}


\twolineshloka
{गोब्राह्मणप्रियो नित्यं स्वस्तिशब्दमुदाहरन्}
{पृथिव्यामन्तरिक्षे च नाकपृष्ठे च विक्रमेत्}


\threelineshloka
{यच्च तेऽन्यद्भवेत्कृत्यं तन्मे ब्रूयाः समासतः}
{तत्ते सर्वं करिष्यामि सत्येनैतद्ब्रवीमि ते ॥इन्द्र उवाच}
{}


\threelineshloka
{इच्छेयं गोषु नियतं वस्तुं देवि ब्रवीमि ते}
{एभिः सुरगणैः सार्धं ममानुग्रहमाचर ॥सुरभिरुवाच}
{}


\twolineshloka
{गवां शरीरं प्रत्यक्षमेतत्कौशिक लक्ष्ये}
{यो यत्रोत्सहते वस्तुं स तत्र वसतां सुरः}


\threelineshloka
{सर्वं पवित्रं परमं गवां गात्रं सुपूजितम्}
{तथा कुरुष्व भद्रं ते यथा त्वं शक्र मन्यसे ॥भीष्म उवाच}
{}


\twolineshloka
{तस्यास्तद्वचनं श्रुत्वा सुरभ्याः सुरसत्तमः}
{सह सर्वैः सुरगणैरभजत्सौरभीं प्रजाम्}


\twolineshloka
{शृङ्गे वक्त्रे च जिह्वायां देवराजः समाविशत्}
{सर्वच्छिद्रेषु पवनः पादेषु मरुतां गणाः}


\twolineshloka
{ककुदं सर्वगो रुद्रः कुक्षौ चैव हुताशनः}
{सरस्वती स्तनेष्वग्र्या श्रीः पुरीषे जगत्प्रिया}


\twolineshloka
{मूत्रे कीर्तिश्च गङ्गा च मेधा पयसि शाश्वती}
{वक्त्रे सोमश्च वै देवो हृदये भगवान्यमः}


\twolineshloka
{धर्मः पुच्छे क्रिया लोम्नि भास्करश्चक्षुषी श्रितः}
{सिद्धाः सन्धिषु सिद्धिश्च तपस्तेजश्च चेष्टने}


\twolineshloka
{एवं सर्वे सुरगणा नियता गात्रवर्त्मसु}
{महती देवता गावो ब्राह्मणैः परिसंस्कृताः}


\twolineshloka
{गामाश्रयन्ति सहिता देवा हि प्रभविष्णवः}
{किमासां सर्वभावेन विदध्याद्भगवान्प्रियम्}


\twolineshloka
{भवांश्च परया भक्त्या पूजयस्व नरेश्वर}
{गावस्तु परमं लोके पवित्रं पावनं हविः}


\twolineshloka
{निपात्य भक्षितः स्वर्गाद्भार्गवः फेनपः किल}
{स च प्राणान्पुनर्लब्ध्वा ततो गोलोकमाश्रितः ॥'}


\chapter{अध्यायः १२०}
\twolineshloka
{कः फेनपेति नाम्नाऽसौ कथं वा भक्षितः पुरा}
{मृत उज्जीवितः कस्मात्कथं गोलोकमाश्रितः}


\fourlineindentedshloka
{विरुद्वे मानुषे लोके तथा समयवर्त्मसु}
{क्रते दैवं हि दुष्प्रपं मानुषेषु विशेषतः}
{संशयो मे महानत्र तन्मे व्याख्यातुमर्हसि ॥ भीष्म उवाच}
{}


\twolineshloka
{श्रूयते भार्गवे वंशे सुमित्रो नाम भारत}
{वेदाध्ययनसम्पन्नो विपुले तपसि स्थितः}


\twolineshloka
{वानप्रस्थाश्रमे युक्तः स्वकर्मनिरतः सदा}
{विनयाचारतत्वज्ञः सर्वधर्मार्थकोविदः}


\twolineshloka
{यत्नात्त्रिषवणस्नायी संध्योपासनतत्परः}
{अग्निहोत्ररतः क्षान्तो जपञ्जुह्वच्च नित्यदाः}


\twolineshloka
{पितृदेवांश्च नियतमतिर्थींश्च स पूजयन्}
{प्राणसन्धारणार्थं च यत्किंचिदुपहारयन्}


\twolineshloka
{गिरिस्त्रिशिखरो नाम यतः प्रभवते नदी}
{कुलजेति पुराणेषु विश्रुता रुद्रनिर्मिता}


\twolineshloka
{तस्यास्तीरे समे देशे पुष्पमालासमाकुले}
{वन्यौषधिद्रुमोपेते नानापक्षिमृगायुते}


\twolineshloka
{व्यपेतदंशमशके ध्वाङ्क्षगृध्रैरसेविते}
{कृष्णदर्भतृणप्राये सुरम्ये ज्योतिरश्मिनि}


\threelineshloka
{सर्वोन्नतैः समैः श्यामैर्याज्ञीयैस्तरुभिर्वृते}
{तत्राश्रमपदं पुण्यं भृगूणामभवत्पुरा}
{}


\twolineshloka
{उवास तत्र नियतः सुमित्रो नाम भार्गवः}
{यथोद्दिष्टेन पूर्वेषां भृगूणां साधुवर्त्मना}


\twolineshloka
{तस्मा आङ्गिरसः कश्चिद्ददौ गां शर्करीं शुभाम्}
{वर्षासु पश्चिमे मासि पौर्णमास्यां शुचिव्रतः}


\twolineshloka
{स तां लब्ध्वा धर्मशीलश्चिन्तयामास तत्परः}
{सुमित्रः परया भक्त्या जननीमिव मातरम्}


\twolineshloka
{तेन सन्धुक्ष्यमाणा सा रोहिणी कामरूपिणी}
{प्रवृद्धिमगमच्छ्रेष्ठा प्राणतश्च सुदर्शना}


\twolineshloka
{सिराविमुक्तपार्श्वान्ता विपुलां कान्तिमुद्वहत्}
{श्यामपार्श्वान्तपृष्ठा सा सुरभिर्मधुपिङ्गला}


\twolineshloka
{बृहती सूक्ष्मरोमान्ता रूपोदग्रा तनुत्वचा}
{कृष्णपुच्छा श्वेतवक्त्रा समवृत्तपयोधरा}


\twolineshloka
{पृष्ठोन्नता पूर्वनता शङ्कुकर्णी सुलोचना}
{दीर्घजिह्वा ह्रस्वशृङ्गी सम्पूर्णदशनान्तरा}


\twolineshloka
{मांसाधिकगलान्ता सा प्रसन्ना शुभदर्शना}
{नित्यं शमयुता स्निग्धा सम्पूर्णोदात्तनिस्वना}


\twolineshloka
{प्राजापत्यैर्गवां नित्यं प्रशस्तैर्लक्षणैर्युता}
{यौवनस्थेव वनिता शुशुभे रूपशोभया}


\twolineshloka
{वृषेणोपगता सा तु कल्या मधुरदर्शना}
{मिथुनं जनयामास तुल्यरूपमिवात्मनः}


\twolineshloka
{संवर्धयामास स तां सवत्सां भार्गवो मुनिः}
{तयोः प्रजाधिसंसर्गात्सहस्रं च गवामभूत्}


\twolineshloka
{गवां जातिसहस्राणि सम्भूतानि परस्परम्}
{ऋषभाणां च राजेन्द्र नैवान्तः प्रतिदृश्यते}


\twolineshloka
{तैराश्रमपदं रम्यमरण्यं चैव सर्वशः}
{समाकुलं समभवन्मेघैरिव नभस्थलम्}


\twolineshloka
{कानि चित्पद्मवर्णानि किंशुकाभानि कानिचित्}
{रुक्मवर्णानि चान्यानि चन्द्रांशुसदृशानि च}


\threelineshloka
{तथा राजतवर्णानि कानिचिल्लोहितानि वै}
{नीललोहितताम्राणि कृष्णानि कपिलानि च}
{नानारागविचित्राणि यूथानि कुलयूथप}


\twolineshloka
{न च क्षीरं सुतस्नेहाद्वत्सानामुपजीवति}
{भार्गवः केवलं चासीद्गवां प्राणायने रतः}


\twolineshloka
{तथा शुश्रूषतस्तस्य गवां हितमवेक्षतः}
{व्यतीयात्सुमाहान्कालो वत्सोच्छिष्टेन वर्ततः}


\twolineshloka
{क्षुत्पिपासापरिश्रान्तः सततं प्रस्रवं गवाम्}
{वत्सैरुच्छिष्टमुदितं बहुक्षीरतया बहु}


\twolineshloka
{पीतवांस्तेन नामास्य फेनपेत्यभिविश्रुतम्}
{गौतमस्याभिनिष्पन्नमेवं नाम युधिष्ठिर}


\chapter{अध्यायः १२१}
\twolineshloka
{कदाचित्कामरूपिण्यो गावः स्त्रीवेषमाश्रिताः}
{ह्रदे क्रीडन्ति संहृष्टा गायन्त्यः पुण्यलक्षणाः ॥?}


\threelineshloka
{ददृशुस्तस्य गावो वै विस्मयोत्फुल्ललोचनाः}
{ऊचुश्च का यूयमिति स्त्रियो मानुषया गिरा ॥स्त्रिय ऊचुः}
{}


\fourlineindentedshloka
{गाव एव वयं सर्वकर्मभिः शोभनैर्युताः}
{सर्वाः स्त्रीवेषधारिण्यो यथाकामं चरामेहे}
{गाव ऊचुः}
{}


\threelineshloka
{गवां गावः परं दैवं गवां गावः परा गतिः}
{कथयध्वमिहास्माकं केन वः सुकृतां गतिः ॥स्त्रिय ऊचुः}
{}


\twolineshloka
{अस्माकं हविषा देवा ब्राह्मणास्तर्पितास्तथा}
{कव्येन पितरश्चैव हव्येनाग्निश्च तर्पितः}


\twolineshloka
{प्रजया च तथाऽस्माकं कृषिरभ्युद्धृता सदा}
{शकटैश्चापि संयुक्ता दशवाहशतेन वै}


\twolineshloka
{तदेतैः सुकृतैः स्फीतैर्वयं याश्चैव नः प्रजाः}
{गोलोकमनुसम्प्राप्ता यः परं कामगोचरः}


\threelineshloka
{यूयं तु सर्वा रोहिण्यः सप्रजाः सहपुङ्गवाः}
{अधोगामिन्य इत्येव पश्यामो दिव्यचक्षुषा ॥गाव ऊचुः}
{}


\twolineshloka
{एवं गवां परं दैवं गाव एव परायणम्}
{स्वपक्ष्यास्तारणीया वः शरणाय गता वयम्}


\threelineshloka
{किमस्माभिः करणीयं वर्तितव्यं कथञ्चन}
{प्राप्नुयाम च गोलोकं भवाम न च गर्हिताः ॥स्त्रिय ऊचुः}
{}


\twolineshloka
{वर्तते रन्तिदेवस्य सत्रं वर्षसहस्रकम्}
{तत्र तस्य नृपस्याशु पशुत्वमुपगच्छतः}


\threelineshloka
{ततस्तस्योपयोगेन पशुत्वे यज्ञसंस्कृताः}
{गोलोकान्प्राप्स्यथ शुभांस्तेन पुण्येन संयुताः ॥भीष्म उवाच}
{}


\twolineshloka
{एतत्तासां वचः श्रुत्वा गवां संहृष्टमानसाः}
{गमनाय मनश्चक्रुरौत्सुक्यं चागमन्परम्}


\twolineshloka
{न हि नो भार्गवो दाता पशुत्वेनोपयोजनम्}
{यज्ञस्तस्य नरेन्द्रस्य वर्तते धर्मतस्तथा}


\twolineshloka
{वयं न चाननुज्ञाताः शक्ता गन्तुं कथञ्चन}
{अवोचन्नथ तत्रत्या भार्गवो वध्यतामयम्}


\twolineshloka
{एतत्सर्वा रोचयत न हि शक्यमतोऽन्यथा}
{लोकान्प्राप्तुं सहास्माभिर्निश्चयः क्रियतामयम्}


\twolineshloka
{न तु तासां समेतानां काचिद्धोरेण चक्षुषा}
{शक्नोति भार्गवं द्रष्टुं सत्कृतेनोपसंयुता}


\twolineshloka
{अथ पद्मसवर्णाभा भास्करांशुसमप्रभाः}
{जपालोहितताम्राक्ष्यो निर्मांसकठिनाननाः}


\twolineshloka
{रोहिण्यः कपिलाः प्राहुः सर्वासां वै समक्षतः}
{मेघस्तनितनिर्घोषास्तेजोभिरभिरञ्चिताः}


\threelineshloka
{वयं हि तं वधिष्यामः सुमित्रं नात्रं संशयः}
{सुकृतं पृष्ठतः कृत्वा किं नः श्रेयो विधास्यथ ॥गाव ऊचुः}
{}


\threelineshloka
{कपिलाः सर्ववर्णेषु प्रधानत्वमवाप्स्यथ}
{गवां शतफला चैकां दत्त्वा फलमवाप्स्यति ॥भीष्म उवाच}
{}


\twolineshloka
{एतद्गवां वचः श्रुत्वा कपिला हृष्टमानसाः}
{चक्रुः सर्वा भार्गवस्य सुमित्रस्य वधे मतिम् ॥'}


\chapter{अध्यायः १२२}
\twolineshloka
{यास्तु गोमातरस्तस्य कामचारिण्य आगताः}
{समीपं हि सुमित्रस्य कृतज्ञाः समुपस्थिताः}


\twolineshloka
{अभिप्रशस्य चैवाहुस्तमृषिं पुण्यदर्शनाः}
{गोलोकादागता वेद वृषगोमातरो वयम्}


\threelineshloka
{सुप्रीताः स्म वरं गृह्ण यमिच्छसि महामुने}
{यद्भि गोषु परां बुद्धिं कृतवानसि नित्यदा ॥सुमित्र उवाच}
{}


\twolineshloka
{प्रीतोस्म्यनुगृहीतोस्मि यन्मां गोमातरः शुभाः}
{सुप्रीतमनसः सर्वास्तिष्ठन्ते च वरप्रदाः}


\threelineshloka
{भवेद्गोष्वेव मे भक्तिर्यथैवाद्य तथा सदा}
{गोघ्नाश्चैवावसीदन्तु नरा ब्रह्मद्विषश्च ये ॥गोमातर ऊचुः}
{}


\threelineshloka
{एवमेतदृषिश्रेष्ठ हितं वदसि नः प्रियम्}
{एहि गच्छ सहाऽस्माभिर्गोलोकमृषिसत्तम ॥सुमित्र उवाच}
{}


\threelineshloka
{यूयमिष्टां गतिं यान्तु न ह्यहं गन्तुमुत्सहे}
{इमा गावः समुत्सृज्य तपस्विन्यो मम प्रियाः ॥भीष्म उवाच}
{}


\twolineshloka
{तास्तु तस्य वचः श्रुत्वा कपिलानां सुदारुणम्}
{नित्युस्तमृषिमुत्क्षिप्य भार्गवं नभ उद्वहन्}


\twolineshloka
{कलेवरं तु तत्रैव तस्य संन्यस्य मातरः}
{निष्कृष्य करणं योगादानयन्भार्गवस्य वै}


\twolineshloka
{सर्वं चास्य तदाचख्युः कपिलानां विचेष्टितम्}
{यदर्थं हरणं गोभिर्गोलोकं लोकमातरः}


\twolineshloka
{ततस्तु कपिलास्तत्र तस्य दृष्ट्वा कलेवरम्}
{तथाप्रतिज्ञं शृङ्गैश्च खुरैश्चाप्यवचूर्णयन्}


\twolineshloka
{ततः संछिद्य बहुधा भार्गवं नृपसत्तम}
{युयुर्यत्रेतरा गावस्तच्च सर्वं न्यवेदयन्}


\twolineshloka
{अथ गोमातृभिः शप्तास्ता गावः पृथिवीचराः}
{अमेध्यवदनाः क्षिप्रं भवध्वं ब्रह्मघातकाः}


\twolineshloka
{एवं कृतज्ञा गावो हि यता गोमातरो नृप}
{ऋषिश्च प्राप्तवाँल्लोकं गावश्च परिमोक्षिताः ॥'}


\chapter{अध्यायः १२३}
\twolineshloka
{ता गावो रन्तिदेवस्य गत्वा यज्ञं मनीषिणः}
{आत्मानं ज्ञापयामासुर्महर्षीणां समक्षतः}


\threelineshloka
{रन्तिदेवस्ततो राजा प्रयतः प्राञ्जलिः शुचिः}
{उवाच गावः प्रणतः किमागमनमित्यपि ॥गाव ऊचुः}
{}


\threelineshloka
{इच्चामस्तव राजेन्द्र सत्रेऽस्मिन्विनियोजनम्}
{पशुत्वमुपसम्प्राप्तुं प्रसादं कर्तुमर्हसि ॥रन्तिदेव उवाच}
{}


\twolineshloka
{नास्ति शक्तो गवां घातं कर्तुं शतसहस्रशः}
{घातयित्वा त्वहं युष्मान्कथमात्मानमुत्तरे}


\threelineshloka
{यः पशुत्वेन संयोज्य युष्मान्स्वर्गं नयेदिह}
{आत्मानं चैव तपसा गावः समुपगम्यताम् ॥गाव ऊचुः}
{}


\threelineshloka
{अस्माकं तारणे युक्तो धर्मात्मा तपसि स्थितः}
{श्रुतोऽस्माभिर्भवान्राजंस्ततस्तु स्वयमागताः ॥रन्तिदेव उवाच}
{}


\twolineshloka
{मम सत्रे पशुत्वं वो यद्येवं हि मनीषितम्}
{समयेनाहमेतेन जुहुयां वो हुताशने}


\threelineshloka
{कदाचिद्यदि वः काचिदकामा विनियुज्यते}
{तदा समाप्तिः सत्रस्य गवां स्यादिति नैष्ठिकी ॥गाव ऊचुः}
{}


\threelineshloka
{एवमस्तु महाराज यथा त्वं प्रब्रवीषि नः}
{अकामाः स्युर्यदा गावस्तदा सत्रं समाप्यताम् ॥भीष्म उवाच}
{}


\twolineshloka
{ततः प्रवृत्ते गोसत्रे रन्तिदेवस्य धीमतः}
{गोसहस्राण्यहरहर्नियुज्यन्ते शमीतृभिः}


\twolineshloka
{एवं बहनि वर्षाणि व्यतीतानि नराधिप}
{गवां वै वध्यमानानां न चान्तः प्रत्यदृश्यत}


\twolineshloka
{गवां चर्मसहस्रैस्तु राशयः पर्वतोपमाः}
{बभूवुः कुरुशार्दूल बहुधा मेघसंनिभाः}


\twolineshloka
{मेदःक्लेदवहा चैव प्रावर्तत महानदी}
{अद्यापि भुवि विख्याता नदी चर्मण्वती शुभा}


\twolineshloka
{ततः कदाचित्स्वं वत्सं गौरुपामन्त्र्य दुःखिता}
{एहि वत्स स्तनं पाहि मा त्वं पश्चात्क्षुदार्दितः}


\twolineshloka
{तप्स्यसे विमना दुःखं घातितायां मयि ध्रुवम्}
{एते ह्यायान्ति चण्डालाः सशस्त्रामां जिघांसवः}


\twolineshloka
{अथ शुश्राव तां वाणीं मानुषीं समुदाहृताम्}
{रन्तिदेवो महाराज ततस्तां समवारयत्}


\twolineshloka
{स्थापयामास गोसत्रमथ तं पार्तिवर्षभ}
{सत्रोत्सृष्टाः परित्यक्ता गावोऽन्याः समुपाश्रिताः}


\twolineshloka
{यास्तस्य राज्ञो निहता गावो यज्ञे महात्मनः}
{ता गोलोकमुपाजग्मुः प्रेक्षिता ब्रह्मवादिभिः}


\twolineshloka
{रन्तिदेवोपि राजर्षिरिष्ट्वा यज्ञं यथाविधि}
{ततः सख्यं सुरपतेस्त्रिदिवं चाक्षयं ययौ}


\twolineshloka
{फेनपो दिवि गोलोके मुमुदे शाश्वतीः समाः}
{अवशिष्टश्च या गवस्ता बभूवुर्वनेचराः}


\twolineshloka
{फेनपाख्यानमेतत्ते गवां माहात्म्यमेव च}
{कथितं पावनं पुण्यं कृष्णद्वैपायनेरितम्}


\twolineshloka
{नारायणोऽपि भगवान्दृष्ट्वा गोषु परं यशः}
{शुश्रूषां परमां चक्रे भक्तिं च भरतर्षभ}


\twolineshloka
{तस्मात्त्वमपि राजेन्द्र गा वै पूजय भारत}
{द्विजेभ्यश्चैव सततं प्रयच्छ कुरुसत्तम}


\chapter{अध्यायः १२४}
\threelineshloka
{पवित्राणां पवित्रं यच्छ्रेष्ठं लोकेषु पूजितम्}
{महाव्रतं महाभाग तन्मे ब्रूहि पितामह ॥भीष्म उवाच}
{}


\twolineshloka
{अत्राप्युदाहरन्तीममितिहासं पुरातनम्}
{पितुः पुत्रस्य संवादं व्यासस्य च शुकस्य च}


\twolineshloka
{ऋषीणामुत्तमं कृष्णं भावितात्मानमच्युतम्}
{पारम्पर्यविशेषज्ञं सर्वशास्त्रार्थकोविदम्}


\twolineshloka
{कृतशौचः शुकस्तत्र कृतजप्यः कृताह्निकः}
{परं नियममास्थाय परं धर्ममुपाश्रितः}


\twolineshloka
{प्रणम्य शिरसा व्यासं सूक्ष्मतत्वार्थदर्शिनम्}
{शुकः पप्रच्छ वै प्रश्नं दानधर्मकुतूहलः}


\twolineshloka
{बहुचित्राणि दानानि बहुशः शंससे मुने}
{महार्थं पावनं पुण्यं किंस्विद्दानं महाफलम्}


\twolineshloka
{केन दुर्गाणि तरति केन लोकानवाप्नुते}
{केन वा महदाप्नोति इह लोके परत्र च}


\threelineshloka
{के वा यज्ञस्य वोढारः केषु यज्ञः प्रतिष्ठितः}
{किञ्च यज्ञस्य यज्ञत्वं किञ्च यज्ञस्य भेषजम्}
{यज्ञानामुत्तमं किञ्च तद्भवान्प्रब्रवीतु मे}


\twolineshloka
{स तस्मै भजमानाय जातकौतूहलाय च}
{व्यासो व्रतनिधिः प्राह गवामिदमनुत्तमम्}


\twolineshloka
{धन्यं यशस्यमायुष्यं लोके श्रुतिसुखावहम्}
{यत्पवित्रं पवित्राणां मङ्गलानां च मङ्गलम्}


\threelineshloka
{सर्वपापप्रशमनं तत्समासेन मे शृणु}
{यदिदं तिष्ठते लोके जगत्स्थावरजङ्गमम्}
{गावस्तत्प्राप्य तिष्ठन्ति गोलोके पुण्यदर्शनाः}


\twolineshloka
{मातरः सर्वभूतानां विश्वस्य जगतश्च ह}
{रुद्राणामिह साध्यानां गाव एव तु मातरः}


\twolineshloka
{रुद्राणां मातरो ह्येता ह्यादित्यानां स्वसा स्मृताः}
{वसूनां च दुहित्रस्ता ब्रह्मसन्तानमूलजाः}


\twolineshloka
{यासामधिपतिः पूषा मरुतो वालबन्धनाः}
{ऐश्वर्यं वरुणो राजा विश्वेदेवाः समाश्रिताः}


\twolineshloka
{य एवं वेद ता गावो मातरो देवपूजिताः}
{स विप्रो ब्रह्मलोकाय गवां लोकाय वा ध्रुवः}


\twolineshloka
{गावस्तु नावमन्येत कर्मणा मनसा गिरा}
{गवां स्थानं परं लोके प्रार्थयेद्यः परां गतिम्}


\twolineshloka
{न पद्भ्यां ताडयेद्गा वै न दण्डेन न मुष्टिना}
{इमां विद्यामुपाश्रित्य पावनीं ब्रह्मनिर्मिताम्}


\twolineshloka
{मातॄणामन्ववाये च न गोमध्ये न गोव्रजे}
{नरो मूत्रपुरीषस्य दृष्ट्वा कुर्याद्विसर्जनम्}


\twolineshloka
{शुद्धाश्चन्दनशीताङ्ग्यश्चन्द्ररश्मिसमप्रभाः}
{सौम्याः सुरभ्यः सुभगा गावो गुग्गुलुगन्धयः}


\twolineshloka
{सर्वे देवाऽविशन्गा वै समुद्रमिव सिन्धवः}
{दिवं चैवान्तरिक्षं च गवां व्युष्टिं समश्नुते}


\twolineshloka
{दधिना जुहुयादग्निं दधिना स्वस्ति वाचयेत्}
{दधि दद्याच्च प्राशेत गवां व्युष्टिं समश्नुते}


\twolineshloka
{घृतेन जुहुयादग्निं घृतेन स्वस्ति वाचयेत्}
{घृतमालभ्य प्राश्नीयाद्गवां व्युष्टिं समश्नुते}


\twolineshloka
{गावः संजीवना यास्तु गावो दानमनुत्तमम्}
{ताः पुण्यगोपाः सुफला भजमानं भजन्तु माम्}


\twolineshloka
{येन देवाः पवित्रेण स्वर्गलोकमितो गताः}
{तत्पवित्रं पवित्राणां मम मूर्ध्नि प्रतिष्ठितम्}


\twolineshloka
{वीणामृदङ्गपणवा गवां गात्रं प्रतिष्ठिताः}
{क्रीडारतिविहारार्थे त्रिषु लोकेषु वर्तते}


\twolineshloka
{न तत्र देवा वर्तन्ते नाग्निहोत्राणि जुह्वति}
{न यज्ञैरिज्यते चात्र यत्र गौर्वै न दृश्यते}


\twolineshloka
{क्षीरं दधि घृथं यासां रसानामुत्तमो रसः}
{अमृतप्रभवा गावस्त्रैलोक्यं येन जीवति}


\twolineshloka
{इमामाहूय धेनुं च सवत्सां यज्ञमातरम्}
{उपाह्वयन्ति यां विप्रा गावो यज्ञहविष्कृतम्}


\twolineshloka
{या मेध्या प्रथमं कर्म इयं धेनुः सरस्वती}
{पौर्णमासेन वत्सेन कामं कामगुणान्विता}


\twolineshloka
{यत्र सर्वमिदं प्रोतं यत्किंचिज्जङ्गमं जगत्}
{स गौर्वै प्रथमा पुण्या सर्वभूतहिते रता}


\twolineshloka
{धारणाः पावनाः पुण्या भावना भूतभावनाः}
{गावो मामभिरक्षन्तु इह लोके परत्र च}


\twolineshloka
{एष यज्ञः सहोपाङ्ग एष यज्ञः सनातनः}
{वेदाः सहोपनिषदो गवां रूपाः प्रतिष्ठिताः}


\twolineshloka
{एतत्तात मया प्रोक्तं गवामिह परं मतम्}
{सर्वतः श्रावयेन्नित्यं प्रयतो ब्रह्मसंसदि}


\twolineshloka
{श्रुत्वा लभेत ताँल्लोकान्ये मया परिकीर्तिताः}
{श्रावयित्वापि प्रीतात्मा लोकांस्तान्प्रतिपद्यते}


\twolineshloka
{धेनुमेकां समादद्यादहन्यहनि पावनीम्}
{तत्तथा प्राप्नुयाद्विप्रः पठन्वै गोमतीं सदा}


\twolineshloka
{अथ धेनुर्न विद्यते तिलधेनुमनुत्तमाम्}
{दद्याद्गोमतिकल्पेन तां धेनुं सर्वपावनीम्}


\twolineshloka
{आह्निकं गोमतीं नित्यं यः पठेत सदा नरः}
{सर्वपापात्प्रमुच्येत प्रयतात्मा य आचरेत्}


\twolineshloka
{घृतं वा नित्यमालभ्य प्राश्य वा गोमतीं जपेत्}
{स्नात्वा वा गोकरीषेण पठन्पापात्प्रमुच्यते}


\twolineshloka
{मनसा गोमतीं जप्येद्गोमत्या नित्यमाह्निकम्}
{न त्वेन दिवसं कुर्याद्व्यर्थं गोमत्यपाठकः}


\twolineshloka
{गोमतीं जपमाना हि देवा देवत्वमाप्नुवन्}
{ऋषित्वमृषयश्चापि गोमत्या सर्वमाप्नुवन्}


\twolineshloka
{बद्धो बन्धात्प्रमुच्येत कृच्छ्रान्मुच्येत सङ्कटात्}
{गोमतीं सेवते यस्तु लभते प्रियसङ्गमम्}


\twolineshloka
{एतत्पवित्रं कार्त्स्न्येन एतद्व्रतमनुत्तमम्}
{एतत्तु पृथिवीपाल पावनं शृण्वतां सदा}


\twolineshloka
{पुत्रकामाश्च ये केचिद्धनकामाश्च मानवाः}
{अद्धाने चोरवैरिभ्यो मुच्यते गोमतीं पठन्}


\twolineshloka
{पूर्ववैरानुबन्धेषु रणए चाप्याततायिनः}
{लभेत जयमेवाशु सदा गोमतिपाठकः ॥'}


\chapter{अध्यायः १२५}
\threelineshloka
{क्षत्रियाश्चैव शूद्राश्च मन्त्रहीनाश्च ये द्विजाः}
{कपिलामुपजीवन्ति कथमेतत्पितर्भवेत् ॥श्रीव्यास उवाच}
{}


\twolineshloka
{क्षत्रियाश्चैव शूद्राश्च मन्त्रहीनाश्च ये द्विजाः}
{कपिलामुपजीवन्ति तेषां वक्ष्यामि निर्णयम्}


\twolineshloka
{कपिलास्तूत्तमा लोके गोषु चैवोत्तमा मताः}
{तासां दाता लभेत्स्वर्गं विधिना यश्च सेवते}


\twolineshloka
{स्पृशेत कपिलां यस्तु दण्डेन चरणेन वा}
{स तेन स्पर्शमात्रेण नरकायोपपद्यते}


\twolineshloka
{मन्त्रेण युञ्ज्यात्कपिलां मन्त्रेणैव प्रमुञ्चते}
{मन्त्रिहीनं तु यो युञ्जात्कृमियोनौ प्रसूयते}


\twolineshloka
{प्रहाराहतमर्माङ्गा दुःखेन च जडीकृता}
{पदानि यावद्गच्छेत तावल्लोकान्कृमिर्भवेत्}


\twolineshloka
{यावन्तो बिन्दवस्तस्याः शोणितस्य क्षितिं गताः}
{तावद्वर्षसहस्राणि नरकं प्रतिपद्यते}


\twolineshloka
{मन्त्रेण युञ्ज्यात्कपिलां मन्त्रेण विनियोजयेत्}
{मन्त्रहीनैरनुयुतो मञ्जयेत्तमसि प्रभो}


\twolineshloka
{कपिलां येऽपि जीवन्ति बुद्धिमोहान्विता नराः}
{तेऽपि वर्षसहस्राणि पतन्ति नरके नृप}


\twolineshloka
{अथ न्यायेन ये विप्राः कपिलामुपयुञ्जते}
{तस्मिंल्लोके प्रमोदन्ते लोकाश्चैषामनामयाः}


% Check verse!
विधिना ये न कुर्वन्ति शूद्रास्तानुपधारय ॥'
\chapter{अध्यायः १२६}
\threelineshloka
{नानावर्णैरुपेतानां गवां किं मुनिसत्तम}
{कपिलाः सर्ववर्णेषु वरिष्ठत्वमवाप्नुवन् ॥व्यास उवाच}
{}


\twolineshloka
{शृणु पुत्र यथा गोषु वरिष्ठाः कपिलाः स्मृताः}
{कपिलत्वं च सम्प्राप्ताः पूज्याश्चि सततं नृषु}


\twolineshloka
{अग्निः पुरापचक्राम देवेभ्य इति नः श्रुतम्}
{देवेभ्यो मां छादयत शरण्याः शरणं गतम्}


\twolineshloka
{ऊचुस्ताः सहितास्तत्र स्वागतं तव पावकः}
{इह गुप्तस्त्वमस्माभिर्न देवैरुपलप्स्यसे}


\twolineshloka
{अथ देवा विवित्सन्तः पावकं परिचक्रमुः}
{गोषु गुप्तं च विज्ञाय ताः क्षिप्रमुपतस्थिरे}


\twolineshloka
{युष्मासु निवसत्यग्निरिति गाः समचूचुदन्}
{प्रकाश्यतां हुतवहो लोकान्न च्छेत्तुमर्हथ}


% Check verse!
एवमस्त्वित्युनुज्ञाय पावकं समदर्शयन्
\twolineshloka
{अधिगम्य पावकं तुष्टास्ते देवाः सद्य एव तु}
{अग्निं प्रचोदयामासुः क्रियतां गोष्वनुग्रहः}


\twolineshloka
{गवां तु यासां गात्रेषु पावकः समवस्थितः}
{कपिलत्वमनुप्राप्ताः सर्वश्रेष्ठत्वमेव च}


\threelineshloka
{महाफलत्वं लोके च ददौ तासां हुताशनः}
{तस्माद्धि सर्ववर्णानां कपिलां गां प्रदापय}
{श्रोत्रियाय प्रशान्ताय प्रयतायाग्निहोत्रिणे}


\twolineshloka
{यावन्ति रोमाणि भन्ति धेन्वायुगानि तावन्ति पुनाति दातॄन्}
{प्रतिग्रहीतॄंश्च पुनाति दत्ताशिष्टे तु गौर्वै प्रतिपादनेन}


\chapter{अध्यायः १२७}
\threelineshloka
{केन वर्णविभागेन विज्ञेया कपिला भवेत्}
{कति वा लक्षणान्यस्या दृष्टानि मुनिभिः पुरा ॥श्रीव्यास उवाच}
{}


\threelineshloka
{शृणु तात यथा गोषु विज्ञेया कपिला भवेत्}
{नेत्रयोः शृङ्गयोश्चैव खुरेषु वृषणेषु च}
{कर्णतो घ्राणतश्चापि षड्विधाः कपिलाः स्मृताः}


\twolineshloka
{एतेषां लक्षणानां तु यद्येकमपि दृश्यते}
{कपिलां तां विजानीयादेवमाहुर्मनीषिणः}


\twolineshloka
{आग्नेयी नेत्रकपिला खुरैर्माहेश्वरी भवेत्}
{ग्रीवायां वैष्णवी ज्ञेया पूष्णो घ्राणादजायत}


\twolineshloka
{कर्णतस्तु वसन्तेन स्वयोनिमभिजायते}
{गायत्र्याश्च वृषणयोरुत्पत्तिः षड्गुणा स्मृता}


\twolineshloka
{एवं गावश्च विप्राश्च गायत्री सत्यमेव च}
{वसन्तश्च सुवर्णश्च एकतः समजायत}


\twolineshloka
{नेत्रयोः कपिलां यस्तु वाहयेत दुहेत वा}
{स पापकर्मा नरकं प्रतिष्ठां प्रतिपद्यते}


\twolineshloka
{नरकाद्विप्रमुक्तस्तु तिर्यग्योनिं निषेवते}
{यदा लभेत मानुष्यं जात्यन्धो जायते नरः}


\twolineshloka
{शृङ्गयोः कपिलां यस्तु वाहयेत दुहेत वा}
{तिर्यग्योनिं स लभते जायमानः पुनः पुनः}


\twolineshloka
{खुरेषु कपिलां यस्तु वाहयेत दुहेत वा}
{तमस्यपारे मज्जेत धनहीनो नराधमः}


\twolineshloka
{कपिलां वालधानेषु वाहयेत दुहेत वा}
{निराश्रयः सदा चैव जायते यदि चेत्कृमिः}


\threelineshloka
{कर्णेन कपिलां यस्तु जानन्नप्युपजीवति}
{सहस्रशः शुचिर्भुत्वा मानुष्यं प्राप्नुयादथ}
{चण्डालः पापयोनिश्च जायते स नराधमः}


\twolineshloka
{घ्राणेन कपिलां यस्तु प्रमादादुपजीवति}
{सोऽपि वर्षसहस्राणि तिर्यग्योनौ प्रजायते}


% Check verse!
व्याधिग्रस्तो जडो रोगी भवेन्मानुष्यमागतः
\twolineshloka
{मधुसर्पिस्सुगन्धास्तु कपिलाः शास्त्रतः स्मृताः}
{एताः समुपजीवेत सोऽपि तिर्यक्षु जायते}


\twolineshloka
{स्थावरत्वमनुप्राप्तो यदि मानुष्यतां लभेत्}
{अल्पायुः स भवेज्जातो हीनवर्णकुलोद्भवः}


\twolineshloka
{ये तु पापा ह्यसूयन्ते कपिलां वाहयन्ति च}
{निरयेषु प्रतिष्ठन्ते यावदाभूतसम्प्लवम्}


\chapter{अध्यायः १२८}
\threelineshloka
{मया गवां पुरीषं वै श्रिया जुष्टमिति श्रुतम्}
{एतदिच्छाम्यहं श्रोतुं संशयोऽत्र हि मे महान् ॥भीष्म उवाच}
{}


\twolineshloka
{अत्राप्युदाहरन्तीममितिहासं पुरातनम्}
{गोभिर्नृपेहं संवादं श्रिया भारतसत्तम}


\threelineshloka
{श्रीः कृत्वेह वपुः कान्तं गोमध्येषु विवेश ह}
{गावोऽथ विस्मितास्तस्या दृष्ट्वा रूपस्य सम्पदम् ॥गाव ऊचुः}
{}


\twolineshloka
{काऽसि देवि कुतो वा त्वं रूपेणाप्रतिमा भुवि}
{विस्मिताः स्म महाभागे तव रूपस्य सम्पदा}


\threelineshloka
{इच्छामस्त्वां वयं ज्ञातुं का त्वं क्व च गमिष्यसि}
{तत्त्वेन हि सुवर्णाभे सर्वमेतद्ब्रवीहि नः ॥श्रीरुवाच}
{}


\twolineshloka
{लोकस्य कान्तिर्भद्रं वः श्रीर्नामाहं परिश्रुता}
{मया दैत्याः परित्यक्ता विनष्टाः शाश्वतीः समाः}


\twolineshloka
{मयाऽभिपन्ना देवाश्च मोदन्ते शाश्वतीः समाः}
{इन्द्रो विवस्वान्सोमश्च विष्णुरापोऽग्निरेव च}


\twolineshloka
{मयाऽभिपन्ना दीप्यन्ते ऋषयो देवतास्तथा}
{यान्नाविशाम्यहं गावस्ते विनश्यन्ति सर्वशः}


\twolineshloka
{धर्मश्चार्थश्च कामश्च मया जुष्टाः सुखान्विताः}
{एवंप्रभावां मां गावो विजानीत सुखप्रदाम्}


\threelineshloka
{इच्छामि चापि युष्मासु वस्तुं सर्वासु नित्यदा}
{आगत्य प्रार्थये युष्माञ्श्रीजुष्टा भवताऽनघाः ॥गाव ऊचुः}
{}


\twolineshloka
{अध्रुवा चपला च त्वं सामान्या बहुभिः सह}
{न त्वामिच्छाम भद्रं ते गम्यतां यत्र रोचते}


\threelineshloka
{वपुष्मन्त्यो वयं सर्वाः किमस्माकं त्वयाऽद्य वै}
{यथेष्टं गम्यतां तत्र कृतकार्या वयं त्वया ॥श्रीरुवाच}
{}


\twolineshloka
{किमेतद्वः क्षमं गावो यन्मां नेहाभिनन्दथ}
{न मां सम्प्रतिगृह्णीध्वं कस्माद्वै दुर्लभां सतीम्}


\twolineshloka
{सत्यश्च लोकवादोऽयं लोके चरति सुव्रताः}
{स्वयं प्राप्ते परिभवो भवतीति विनिश्चयः}


\twolineshloka
{महदुग्रं तपः कृत्वा मां निषेवन्ति मानवाः}
{देवदानवगन्धर्वाः पिशाचोरगराक्षसाः}


\threelineshloka
{क्षममेतद्धि वो गावः प्रतिगृह्णीत मामिह}
{नावमन्या ह्यहं सौम्यास्त्रैलोक्ये सचराचरे ॥गाव ऊचुः}
{}


\twolineshloka
{नावमान्यामहे देवि न त्वां परिभवामहे}
{अध्रुवा चलचित्तासि ततस्त्वां वर्जयामहे}


\threelineshloka
{बहुना च किमुक्तेन गम्यतां यत्र वाञ्छसि}
{वपुष्मन्त्यो वयं सर्वाः किमस्माकं त्वयाऽनघे ॥श्रीरुवाच}
{}


\twolineshloka
{अवज्ञाता भविष्यामि सर्वलोकेषु मानवैः}
{प्रत्याख्यातेति युष्माभिः प्रसादः क्रियतां मम}


\twolineshloka
{महाभागा भवत्यो वै शरण्याः शरणागताम्}
{परित्रायन्तु मां नित्यं भजमानामनिन्दिताम्}


\twolineshloka
{माननामहमिच्छामि भवत्यः सततं शिवाः}
{अप्येकाङ्गेष्वधो वस्तुमिच्छामि च सुकुत्सिते}


\threelineshloka
{न वोऽस्ति कुत्सितं किञ्चिदङ्गेष्वालक्ष्यतेऽनघाः}
{पुण्याः पवित्राः सुभगा अवाग्देशं प्रयच्छथ}
{वसेयं यत्र वो देहे तन्मे व्याख्यातुमर्हथ}


\twolineshloka
{एवमुक्तास्तु ता गावः शुभाः करुणवत्सलाः}
{सम्मान्य सहिताः सर्वाः श्रियमुचुर्नराधिप}


\threelineshloka
{अवश्यं मानना कार्या तवास्माभिर्यशस्विनि}
{शकृन्मूत्रे निवसतां पुण्यमेतद्धि नः शुभे ॥श्रीरुवाच}
{}


\threelineshloka
{दिष्ट्या प्रसादो युष्माभिः कृतो मेऽनुग्रहात्मकः}
{एवं भवतु भद्रं वः पूजिताऽस्मि सुखप्रदाः ॥भीष्म उवाच}
{}


\twolineshloka
{एवं कृत्वा तु समयं श्रीर्गोभिः सह भारत}
{पश्यन्तीनां ततस्तासां तत्रैवान्तरधीयत}


\twolineshloka
{एवं गोशकृतः पुत्र माहात्म्यं तेऽनुवर्णितम्}
{माहात्म्यं च गवां भूयः श्रूयतां गदतो मम}


\chapter{अध्यायः १२९}
\twolineshloka
{ये च गाः सम्प्रयच्छन्ति हुतशिष्टाशिनश्च ये}
{तेषां सत्राणि यज्ञाश्च नित्यमेव युधिष्ठिर}


\twolineshloka
{ऋते दधिघृतेनेह न यज्ञः सम्प्रवर्तते}
{तेन यज्ञस्य यज्ञत्वमतो मूलं च लक्ष्यते}


\twolineshloka
{दानानामपि सर्वेषां गवां दानं प्रशस्यते}
{गावः श्रेष्ठाः पवित्राश्च पावनं ह्येतदुत्तमम्}


\twolineshloka
{पुष्ट्यर्थमेताः सेवेत शान्त्यर्थमपि चैव ह}
{पयो दधि घृतं चासां सर्वपापप्रमोचनम्}


\twolineshloka
{गावस्तेजः परं प्रोक्तमिह लोके परत्र च}
{न गोभ्यः परमं किञ्चित्पवित्रं भरतर्षभ}


\twolineshloka
{अत्राप्युदाहरन्तीममितिहासं पुरातनम्}
{पितामहस्य संवादमिन्द्रस्य च युधिष्ठिर}


\twolineshloka
{पराभूतेषु दैत्येषु शक्रस्त्रिभुवनेश्वरः}
{प्रजाः समुदिताः सर्वाः सत्यधर्मपरायणाः}


\threelineshloka
{अथर्षयः सगन्धर्वाः किन्नरोरगराक्षसाः}
{देवासुरसुपर्णाश्च प्रजानां पतयस्तथा}
{पर्युपासत कौन्तेय कदाचिद्वै पितामहम्}


\twolineshloka
{नारदः पर्वतश्चैव विश्वावसुहहाहुहूः}
{दिव्यतानेषु गायन्तः पर्युपासत तं प्रभुम्}


\twolineshloka
{तत्र दिव्यानि पुष्पाणि प्रावहत्पवनस्तदा}
{आजह्रुर्ऋतवश्चापि सुगन्धीनि पृथक्पृथक्}


\threelineshloka
{तस्मिन्देवसमावाये सर्वभूतसमागमे}
{दिव्यवादित्रसंघुष्टे दिव्यस्त्रीचारणावृते}
{इन्द्रः पप्रच्छ देवेशमभिवाद्य प्रणम्य च}


\twolineshloka
{देवानां भगवन्कस्माल्लोकेशानां पितामह}
{उपरिष्ठाद्गवां लोक एतदिच्छामि वेदितुम्}


% Check verse!
किं तपो ब्रह्मचर्यं वा गोभिः कृतमिहेश्वर ॥देवानामुपरिष्टाद्यद्वसन्त्यरजसः सुखम्
\twolineshloka
{ततः प्रोवाच ब्रह्मा तं शक्रं बलनिषूदनम्}
{अवज्ञातास्त्वया नित्यं गावो बलनिषूदन}


\twolineshloka
{तेन त्वमासां माहात्म्यं न वेत्सि शृणु यत्प्रभो}
{गवां प्रभावं परमं माहात्म्यं च सुरर्षभ}


\twolineshloka
{यज्ञाङ्गं कथिता गावो यज्ञ एव च वासव}
{एताभिश्च विना विज्ञो न वर्तेत कथञ्चन}


\twolineshloka
{धारयन्ति प्रजाश्चैताः पयसा हविषा तथा}
{एतासां तनयाश्चापि कृषियोगमुपासते}


\twolineshloka
{जनयन्ति च धान्यानि बीजानि विविधानि च}
{ततो यज्ञाः प्रवर्तन्ते हव्यं कव्यं च सर्वशः}


\twolineshloka
{पयो दधि घृतं चैव पुण्याश्चैताः सुराधिप}
{वहन्ति विविधान्भोगान्क्षुत्तृष्णापरिपीडिताः}


\twolineshloka
{मुनींश्च धारयन्तीह प्रजाश्चैवापि कर्मणा}
{वासवाऽकूटवाहिन्यः कर्मणा सुकृतेन च}


\threelineshloka
{उपरिष्टात्ततोऽस्माकं वसन्त्येताः सदैव हि}
{एतत्ते कारणं शक्र निवासकृतमध्य वै}
{गावो देवोपरिष्टाद्धि समाख्याताः शतक्रतो}


\twolineshloka
{एता हि वरदत्ताश्च वरदाश्चापि वासव}
{सुरभ्यः पुण्यकर्मिण्यः पावनाः शुभलक्षणाः}


\twolineshloka
{यदर्थं गां गताश्चैव सुरभ्यः सुरसत्तम}
{तच्च मे शृणु कार्स्न्येन वदतो बलसूदन}


\twolineshloka
{पुरा देवयुगे तात दैत्येन्द्रेषु महात्मसु}
{त्रीँल्लोकाननुशासत्सु विष्णौ गर्भत्वमागते}


\twolineshloka
{अदित्यां तप्यमानायां तपो घोरं सुदुश्चरम्}
{पुत्रार्थममरश्रेष्ठ पादेनैकेन नित्यदा}


\twolineshloka
{तां तु दृष्ट्वा महादेवीं तप्यमानां महत्तपः}
{दक्षस्य दुहिता देवी सुरभिर्नाम नामतः}


\twolineshloka
{अतप्यत तपो घोरं हृष्टा धर्मपरायणा}
{कैलासशिखरे रम्ये देवगन्धर्वसेविते}


\twolineshloka
{व्यतिष्ठदेकपादेन परमं योगमास्थिता}
{दशवर्षसहस्राणि दशवर्षशतानि च}


\twolineshloka
{सन्तप्तास्तपसा तस्या देवाः सर्षिमहोरगाः}
{तत्र गत्वा मया सार्धं पर्युपासत तां शुभां}


\twolineshloka
{अथाहमब्रवं तत्र देवीं तां तपसाऽन्विताम्}
{किमर्थं तप्यसे देवि तपो घोरमनिन्दिते}


\threelineshloka
{प्रीतस्तेऽहं महाभागो तपसाऽनेन शोभने}
{वरयस्व वरं देवि दातास्मीति पुरन्दर ॥सुरभिरुवाच}
{}


\threelineshloka
{वरेण भगवन्मह्यं कृतं लोकपितामह}
{एष एव वरो मेऽद्य यत्प्रीतोसि ममानघ ॥ब्रह्मोवाच}
{}


\twolineshloka
{तामेव ब्रुवतीं देवीं सुरभिं त्रिदशेश्वर}
{प्रत्यब्रवं यद्देवेन्द्र तन्निबोध शचीपते}


\twolineshloka
{अलोभकाम्यया देवि तपसा शुचिना च ते}
{प्रसन्नोऽहं वरं तस्मादमरत्वं दादामि ते}


\twolineshloka
{त्रयाणामपि लोकानामुपरिष्टान्निवत्स्यसि}
{मत्प्रसादाच्च विख्यातो गोलोकः सम्भविष्यति}


\twolineshloka
{मानुषेषु च कुर्वाणाः प्रजाः कर्म शुभास्तव}
{निवत्स्यन्ति महाभागो सर्वा दुहितरश्च ते}


\twolineshloka
{मनसा चिन्तिता भोगास्त्वया वै दिव्यमानुषाः}
{यच्च स्वर्गसुखं देवि तत्ते सम्पत्स्यते शुभे}


\threelineshloka
{तस्या लोकाः सहस्राक्ष सर्वकामसमन्विताः}
{न तत्र क्रमते मृत्युर्न जरा न च पावकः}
{न दैन्यं नाशुभं किञ्चिद्विद्यते तत्र वासव}


\twolineshloka
{तत्र दिव्यान्यरण्यानि दिव्यानि भवनानि च}
{विमानानि सुयुक्तानि कामगानि च वासव}


\twolineshloka
{ब्रह्मचर्येण तपसा सत्येन च दमेन च}
{दानैश्च विविधैः पुण्यैस्तथा तीर्थानुसेवनात्}


\twolineshloka
{तपसा महता चैव सुकृतेन च कर्मणा}
{शक्यः समासादयितुं गोलोकः पुष्करेक्षण}


\threelineshloka
{एतत्ते सर्वमाख्यातं मया शक्रानुपृच्छते}
{न ते परिभवः कार्यो गवामसुरसूदन ॥भीष्म उवाच}
{}


\twolineshloka
{एतच्छ्रुत्वा सहस्राक्षः पूजयामास नित्यदा}
{गाश्चक्रे बहुमानं च तासु नित्यं युधिष्ठिर}


\threelineshloka
{एतत्ते सर्वमाख्यातं पावनं च महाद्युते}
{पवित्रं परमं चापि गवां माहात्म्यमुत्तमम्}
{}


\twolineshloka
{कीर्तितं पुरुषव्याघ्र सर्वपापविमोचनम्}
{य इदं कथयेन्नित्यं ब्राह्मणेभ्यः समाहितः}


\twolineshloka
{हव्यकव्येषु यज्ञेषु पितृकार्येषु चैव ह}
{सार्वकामिकमक्षय्यं पितॄंस्तस्योपतिष्ठते}


\twolineshloka
{गोषु भक्तश्च लभते यद्यदिच्छति मानवः}
{स्त्रियोपि भक्ता या गोषु ताश्च काममवाप्नुयुः}


\twolineshloka
{पुत्रार्थीं लभते पुत्रं कन्यार्थी तामवाप्नुयात्}
{धनार्थी लभते वित्तं धर्मार्थी धर्ममाप्नुयात्}


\twolineshloka
{विद्यार्थी चाप्नुयाद्विद्यां सुखार्थी प्राप्नुयात्सुखम्}
{न किञ्चिद्दुर्लभं चैव गवां भक्तस्य भारत}


\chapter{अध्यायः १३०}
\twolineshloka
{उक्तं पितामहेनेदं गवां दानमनुत्तमम्}
{विशेषेण नरेन्द्राणामिह धर्ममवेक्षताम्}


\threelineshloka
{राज्यं हि सततं दुःखमाश्रमाश्च सुदुर्विदाः}
{परिचारेषु वै दुःखं दुर्धरं चाकृतात्मभिः}
{भूयिष्ठं च नरेन्द्राणां विद्यते न शुभा गतिः}


\twolineshloka
{पूयन्ते तत्र नियतं प्रयच्छन्तो वसुन्धराम्}
{सर्वे च कथिता धर्मास्त्वया मे कुरुनन्दन}


\twolineshloka
{एवमेव गवामुक्तं प्रदानं ते नृगेण ह}
{ऋषिणा नाचिकेतेन पूर्वमेव निदर्शितम्}


\twolineshloka
{वेदोपनिषदे चैव सर्वकर्मसु दक्षिणाः}
{सर्वक्रतुषु चोद्दिष्टा भूमिर्गावोऽथ काञ्चनम्}


\twolineshloka
{तत्र श्रुतिस्तु परमा सुवर्णं दक्षिणेति वै}
{एतदिच्छाम्यहं श्रोतुं पितामह यथातथम्}


\twolineshloka
{किं सुवर्णं कथं जातं कस्मिन्काले किमात्मकम्}
{किंदैव किंफलं चैव कस्माच्च परमुच्यते}


\twolineshloka
{कस्माद्दानं सुवर्णस्य पूजयन्ति मनीषिणः}
{कस्माच्च दक्षिणार्थं तद्यज्ञकर्मसु शस्यते}


\threelineshloka
{कस्माच्च पावनं श्रेष्ठं भूमेर्गोभ्यश्च काञ्चनम्}
{परमं दक्षिणार्थे च तद्ब्रवीहि पितामह ॥भीष्म उवाच}
{}


\twolineshloka
{शृणु राजन्नवहितो बहुकारणविस्तरम्}
{जातरूपसमुत्पत्तिमनुभूतं च यन्मया}


\twolineshloka
{पिता मम महातेजाः श्तनुर्निधनं गतः}
{तस्य दित्सुरहं श्राद्धं गङ्गाद्वारमुपागमम्}


\twolineshloka
{तत्रागम्य पितुः पुत्र श्राद्धकर्म समारभम्}
{मातो मे जाह्नवी चात्र साहाय्यमकरोत्तदा}


\twolineshloka
{तत्रागतांस्तपस्सिद्धानुपवेश्य बहूनृषीन्}
{तोयप्रदानात्प्रभृति कार्याण्यहमथारभम्}


\twolineshloka
{तत्समाप्य यथोद्दिष्टं पूर्वकर्म समाहितः}
{दातुं निर्वपणं सम्यग्यथावदहमारभम्}


\twolineshloka
{ततस्तं दर्भविन्यासं भित्त्वा सुरुचिराङ्गदः}
{प्रलम्बाभरणो बाहुरुदतिष्ठद्विशाम्पते}


\twolineshloka
{मुहूर्तमपि तं दृष्ट्वा परं विस्मयमागमम्}
{प्रतिग्रहीता साक्षान्मे पितेति भरतर्षभ}


\threelineshloka
{ततो मे पुनरेवासीत्संज्ञा सञ्चिन्त्य शास्त्रतः}
{नायं वेदेषु विहितो विधिर्हस्त इति प्रभो}
{पिण्डो देयो नरेणेह ततो मतिरभून्मम}


\twolineshloka
{साक्षान्नेह मनुष्यस्य पिण्डं हि पितरः क्वचित्}
{गृह्णन्ति विहितं चेत्थं पिण्डो देयः कुशेष्विति}


\twolineshloka
{ततोऽहं तदनादृत्य पितुर्हस्तनिदर्शनम्}
{शास्त्रप्रामाण्यसूक्ष्मं तु विधइं पिण्डस्य संस्मरन्}


\twolineshloka
{ततो दर्भेषु तत्सर्वमददं भरतर्षभ}
{शास्त्रमार्गानुसारेण तद्विद्धि मनुजर्षभ}


\twolineshloka
{ततः सोऽन्तर्हितो बाहुः पितुर्मम जनाधिप}
{ततो मां दर्शयामासुः स्वप्नान्ते पितरस्तथा}


\twolineshloka
{प्रीयमाणास्तु मामूचुः प्रीताः स्म भरतर्षभ}
{विज्ञानेन तवानेन यन्न मुह्यसि धर्मतः}


\twolineshloka
{त्वया हि कुर्वता शास्त्रं प्रमाणमिह पार्थिव}
{आत्मा धर्मः श्रुतं वेदाः पितरश्चर्षिभिः सह}


\twolineshloka
{साक्षात्पितामहो ब्रह्मा गुरवोऽथ प्रजापतिः}
{प्रमाणमुपनीता वै स्थिताश्च न विचालिताः}


\twolineshloka
{तदिदं सम्यगारब्धं त्वयाऽद्य भरतर्षभ}
{किन्तु भूमेर्गवां चार्थे सुवर्णं दीयतामिति}


\twolineshloka
{एवं वयं च धर्मश्च सर्वे चास्मत्पितामहाः}
{पाविता वै भविष्यन्ति पावनं हि परं हि तत्}


\twolineshloka
{दश पूर्वान्दशैवान्यांस्तथा सन्तारयन्ति ते}
{सुवर्णं ये प्रयच्छन्ति एवं मत्पितरोऽब्रुवन्}


\twolineshloka
{ततोऽहं विस्मितो राजन्प्रतिबुद्धो विशाम्पते}
{सुवर्णदानेऽकरवं मतिं च भरतर्षभ}


\twolineshloka
{इतिहासमिमं चापि शृणु राजन्पुरातनम्}
{जामदग्न्यं प्रति विभो धन्यमायुष्यमेव च}


\twolineshloka
{जामदग्न्येन रामेण तीव्ररोषान्वितेन वै}
{त्रिःसप्तकृत्वः पृथिवी कृता निःक्षत्रिया पुरा}


\twolineshloka
{ततो जित्वा महीं कृत्स्नां रामो राजीवलोचनः}
{आजहार क्रतुं वीरो ब्रह्मक्षत्रेण पूजितम्}


\twolineshloka
{वाजिमेधं महाराज सर्वकामसमन्वितम्}
{पावनं सर्वभूतानां तेजोद्युतिविवर्धनम्}


\twolineshloka
{विपाप्मा च स तेजस्वी तेन क्रतुफलेन च}
{नैवात्मनोऽथ लघुतां जामदग्न्योऽध्यगच्छत}


\twolineshloka
{स तु क्रतुवरेणेष्ट्वा महात्मा दक्षिणावता}
{पप्रच्छागमसम्पन्नानृषीन्देवांश्च भारत}


\twolineshloka
{पावनं यत्परं नॄणामुग्रे कर्मणि वर्तताम्}
{तदुच्यतां महाभागा इति जागघृणोऽब्रवीत्}


\twolineshloka
{इत्युक्ता वेदशास्त्रज्ञास्तमूचुस्ते महर्षयः}
{राम विप्राः सत्क्रियन्तां वेदप्रामाण्यदर्शनात्}


\twolineshloka
{भूयश्च विप्रर्षिगणाः प्रष्टव्याः पावनं प्रति}
{ते यद्ब्रूर्महाप्राज्ञास्तच्चैव समुदाचर}


\twolineshloka
{ततो वसिष्ठं देवर्षिमगस्त्यमथ काश्यपम्}
{तमेवार्तं महातेजाः पप्रच्छ भृगुनन्दनः}


\twolineshloka
{जाता मतिर्मे विप्रेन्द्राः कथं पूयेयमित्युत}
{केन वा कर्मयोगेन प्रदानेनेह केन वा}


\threelineshloka
{यदि वोऽनुग्रहकृता बुद्धिर्मां प्रति सत्तमाः}
{प्रबूत पावनं किं मे भवेदिति तपोधनाः ॥ऋषय ऊचुः}
{}


\twolineshloka
{गाश्च भूमिं च वित्तं च दत्त्वेह भृगुनन्दन}
{पापकृत्पूयते मर्त्य इति भार्गव शुश्रुम}


\twolineshloka
{अन्यद्दानं तु विप्रर्षे श्रूयतां पावनं महत्}
{दिव्यमत्यद्भुताकारमपत्यं जातवेदसः}


\twolineshloka
{दग्ध्वा लोकान्पुरा वीर्यात्सम्भूतमिह शुश्रुम}
{सुवर्णमिति विख्यातं तद्ददत्सिद्धिमेष्यसि}


\twolineshloka
{ततोऽब्रवीद्वसिष्ठस्तं भगवान्संशितव्रतः}
{शृणु राम यथोत्पन्नं सुवर्णमनलप्रभम्}


\twolineshloka
{फलं दास्यति ते यत्तु दाने परमिहोच्यते}
{सुवर्णं यच्च यस्माच्च यथा च गुणवत्तमम्}


\twolineshloka
{तन्निबोध महाबाहो सर्वं निगदतो मम}
{अग्निषोमात्मकमिदं सुवर्णं विद्दि निश्चये}


\twolineshloka
{अजोऽग्निर्वरुणो मेषः सूर्योऽश्च इति दर्शनम्}
{कुञ्जराश्च मृगा नागा महिषाश्चासुरा इति}


\twolineshloka
{कुक्कुटाश्च वराहाश्च राक्षसा भृगुनन्दन}
{इडा गावः पयः सोमो भूमिरित्येव च स्मृतिः}


\twolineshloka
{जगत्सर्वं च निर्मथ्य तेजोराशिः समुत्थितः}
{सुवर्णमेभ्यो विप्रर्षे रत्नं परममुत्तमम्}


\twolineshloka
{एतस्मात्कारणाद्देवा गन्धर्वोरगराक्षसाः}
{मनुष्याश्च पिशाचाश्च प्रयता धारयन्ति तत्}


\twolineshloka
{मुकुटैरङ्गदयुतैरलङ्कारैः पृथग्विधैः}
{सुवर्णविकृतैस्तत्र विराजन्ते भृगूत्तम}


\twolineshloka
{तस्मात्सर्वपवित्रेभ्यः पवित्रं परमं स्मृतम्}
{भूमेर्गोभ्योऽथ रत्नेभ्यस्तद्विद्धि मनुजर्षभ}


\twolineshloka
{पृथिवीं गाश्च दत्त्वेह यच्चान्यदपि किञ्चन}
{विशिष्यते सुवर्णस्य दानं परमकं विभो}


\twolineshloka
{अक्षयं पावनं चैव सुवर्णममरद्युते}
{प्रयच्छ द्विजमुख्येभ्यः पावनं ह्येतदुत्तमम्}


\twolineshloka
{सुवर्णमेव सर्वासु दक्षिणासु विधीयते}
{सुवर्णं ये प्रयच्छन्ति सर्वदास्ते भवन्त्युत}


\twolineshloka
{देवतास्ते प्रयच्छन्ति ये सुवर्णं ददत्यथ}
{अग्निर्हि देवताः सर्वाः सुवर्णं च तदात्मकम्}


\twolineshloka
{तस्मात्सुवर्णं ददता दत्ताः सर्वाः स्म देवताः}
{भवन्ति पुरुषव्याघ्र न ह्यतः परमं विदुः}


\twolineshloka
{भूय एव च महात्म्यं सुवर्णस्य निबोध मे}
{गदतो मम विप्रर्षे सर्वशस्त्रभृतांवर}


\twolineshloka
{मया श्रुतमिदं पूर्वं पुराणे भृगुनन्दन}
{प्रजापतेः कथयतो मनोः स्वायंभुवस्य वै}


\twolineshloka
{शूलपाणेर्भगवतो रुद्रस्य च महात्मनः}
{गिरौ हिमवति श्रेष्ठे तदा भृगुकुलोद्वह}


\twolineshloka
{देव्या विवाहे निर्वृत्ते रुद्राण्या भृगुनन्दन}
{समागमे भगवतो देव्या सह महात्मनः}


\twolineshloka
{ततः सर्वे समुद्विग्रा देवा रुद्रमुपागमन् ॥ते महादेवमासीनं देवीं च वरदामुमाम्}
{}


\twolineshloka
{प्रसाद्य शिरसा सर्वे रुद्रमूचुर्भृगूद्वह ॥अयं समागमो देव देव्या सह तवानघ}
{}


\twolineshloka
{तपस्विनस्तपस्विन्या तेजस्विन्याऽतितेजसः ॥अमोघतेजास्त्वं देव देवी चेयमुमा तथा}
{}


\twolineshloka
{अपत्यं युवयोर्देव बलवद्भविता विभो}
{तन्नूनं त्रिषु लोकेषु न किञ्चिच्छेषयिष्यति}


\twolineshloka
{तदेभ्यः प्रणतेभ्यस्त्वं देवेभ्यः पृथुलोचन}
{वरं प्रयच्छ लोकेश त्रैलोक्यहितकाम्यया}


\twolineshloka
{अपत्यार्थं निगृह्णीष्व तेजः परमकं विभो}
{[त्रैलोक्यसारौ हि युवां लोकं सन्तापयिष्यथ}


\twolineshloka
{तदपत्यं हि युवयोर्देवानभिभवेद्ध्रुवम्}
{न हि ते पृथिवी देवी न च द्यौर्न दिवं विभो}


\twolineshloka
{नेदं धारयितुं शक्ताः समस्ता इति मे मतिः}
{तेजःप्रभावनिर्दग्धं तस्मात्सर्वमिदं जगत्}


\threelineshloka
{तस्मात्प्रसादं भगवन्कर्तुमर्हसि नः प्रभो}
{न देव्यां सम्भवेत्पुत्रो भवतः सुरसत्तम}
{धैर्यादेव निगृह्णीष्व तेजो ज्वलितमुत्तमम्}


\twolineshloka
{इति तेषां कथयतां भगवान्वृषभध्वजः]}
{एवमस्त्विति देवांस्तान्विप्रर्षे प्रत्यभाषत}


\twolineshloka
{इत्युक्त्वा चोर्ध्वमनयद्रेतो वृषभवाहनः}
{ऊर्ध्वरेषः समभवत्ततः प्रभृति चापि सः}


\twolineshloka
{रुद्राणीति ततः क्रुद्धा प्रजोच्छेद तदा कृते}
{देवानथाब्रवीत्तत्र स्त्रीभावात्परुषं वचः}


\twolineshloka
{यस्मादपत्यकामो वै भर्ता मे विनिवर्तितः}
{तस्मात्सर्वे सुरा भूयमनपत्या भविष्यथ}


\twolineshloka
{प्रजोच्छेदो मम कृतो यस्माद्युष्माभिरद्य वै}
{तस्मात्प्रजा वः खगमाः सर्वेषां न भविष्यति}


\twolineshloka
{पावकस्तु न तत्रासीच्छापकाले भृगूद्वह}
{देवा देव्यास्तथा शापादनपत्यास्ततोऽभवन्}


\twolineshloka
{रुद्रस्तु तेजोऽप्रतिमं धारयामास वै तदा}
{प्रस्कन्नं तु ततस्तस्मात्किंचित्तत्रापतद्भुवि}


\twolineshloka
{उत्पपात तदा वह्नौ ववृधे चाद्भुतोपमम्}
{तेजस्तेजसि संयुक्तमेकयोनित्वमागतम्}


\twolineshloka
{एतस्मिन्नेव काले तु देवाः शक्रपुरोगमाः}
{असुरस्तारको नाम तेन सन्तापिता भृशम्}


\twolineshloka
{आदित्या वसवो रुद्रा मरुतोऽथाश्विनावपि}
{साध्याश्च सर्वे संत्रस्ता दैतेयस्य पराक्रमात्}


\twolineshloka
{स्थानानि देवतानां हि विमानानि पुराणि च}
{ऋषीणां चाश्रमाश्चैव बभूवुरसुरैर्हृताः}


\twolineshloka
{ते दीनमनसः सर्वे देवता ऋषयश्च ये}
{प्रजग्मुः शरणं देवं ब्रह्माणमजरं विभुम्}


\chapter{अध्यायः १३१}
\twolineshloka
{असुरस्तारको नाम त्वया दत्तवरः प्रभो}
{सुरानृषींश्च क्लिश्नाति वधस्तस्य विधीयताम्}


\threelineshloka
{तस्माद्भयं समुत्पन्नमस्माकं वै पितामह}
{परित्रायस्व नो देव न ह्यन्या गतिरस्ति नः ॥ब्रह्मोवाच}
{}


\twolineshloka
{समोहं सर्वभूतानामधर्मं नेह रोचये}
{हन्यतां तारकः क्षिप्र सुरर्षिगणबाधिता}


\threelineshloka
{वेदा धर्माश्च नोच्छेदं गच्छेयुः सुरसत्तमाः}
{विहितं पूर्वमेवात्र मया वै व्येतु वो ज्वरः ॥देवा ऊचुः}
{}


\twolineshloka
{वरदानाद्भगवतो दैतेयो बलगर्वितः}
{देवैर्न शक्यते हन्तु स कथं प्रशमं व्रजेत्}


\twolineshloka
{स हि नैव स्म देवानां नासुराणां न रक्षसाम्}
{वध्यः स्यामिति जग्राह वरं त्वत्तः पितामह}


\threelineshloka
{देवाश्च शप्ता रुद्राण्या प्रजोच्छेदे पुरा कृते}
{न भविष्यति वोऽपत्यमिति सर्वे जगत्पते ॥ब्रह्मोवाच}
{}


\twolineshloka
{हुताशनो न तत्रासीच्छापकाले सुरोत्तमाः}
{स उत्पादयिताऽपत्यं वधाय त्रिदशद्विषाम्}


\twolineshloka
{तद्वै सर्वानतिक्रम्य देवदानवराक्षसान्}
{मानुषानथ गन्धर्वान्नागानथ च पक्षिणः}


\twolineshloka
{अस्त्रेणामोघपातेन शक्त्या तं घातयिष्यति}
{यतो वो भयपुत्पन्नं ये चान्ये सुरशत्रवः}


\twolineshloka
{सनातनो हि सङ्कल्पः काम इत्यभिधीयते}
{रुद्रस्य तेजः प्रस्कन्नमग्नौ निपतितं च यत्}


\twolineshloka
{तत्तजोऽग्निर्महद्भूतं द्वितीयमिव पावकम्}
{वधार्थं देवशत्रूणां गङ्गायां जनयिष्यति}


\twolineshloka
{स तु नावाप तं शापं नष्टः स हुतभुक्तदा}
{तस्माद्वो भयहृद्देवाः समुत्पत्स्यति पावकिः}


\twolineshloka
{अन्विष्यतां वै ज्वलनस्तथा चाद्य नियुज्यताम्}
{तारकस्य वधोपायः कथितो वै मयाऽनघाः}


\twolineshloka
{न हि तेजस्विनां शापास्तेजःसु प्रभवन्ति वै}
{बलान्यतिबलं प्राप्य दुर्बलानि भवन्ति वै}


\twolineshloka
{हन्यादवध्यान्वरदानपि चैव तपस्विनः}
{सङ्कल्पाभिरुचिः कामः सनातनतमोऽभवत्}


\twolineshloka
{जगत्पतिरनिर्देश्य सर्वगः सर्वभावनः}
{हृच्छयः सर्वभूतानां ज्येष्ठो रुद्रादपि प्रभुः}


\twolineshloka
{अन्विष्यतां स तु क्षिप्रं तेजोराशिर्हुताशनः}
{स वो मनोगतं कामं देवः सम्पादयिष्यति}


\twolineshloka
{एतद्वाक्यमुपश्रुत्य ततो देवा महात्मनः}
{जग्मु- संसिद्धसङ्कल्पाः पर्येषन्तो विभावसुम्}


\twolineshloka
{ततस्त्रैलोक्यमृषयो व्यचिन्वन्त सुरैः सह}
{काङ्क्षन्तो दर्शनं वह्नेः सर्वे तद्गतमानसाः}


\threelineshloka
{परेण तपसा युक्ताः श्रीमन्तो लोकविश्रुताः}
{लोकानन्वचरन्सिद्धाः सर्व एव भृगूत्तम}
{नष्टमात्मनि संलीनं नाभिजग्मुर्हुताशनम्}


\fourlineindentedshloka
{ततः संजातसंत्रासानग्निदर्शनलालसान्}
{जलेचरः क्लान्तमनास्तेजसाऽग्नेः प्रदीपितः}
{उवाच देवान्मण्डूको रसातलतलोत्थितः}
{}


\twolineshloka
{रसातलतले देवा वसत्यग्निरिति प्रभो}
{सन्तापादिह सम्प्राप्तः पावकप्रभवादहम्}


\twolineshloka
{स संसुप्तो जले देवा भगवान्हव्यवाहनः}
{अपः संसृज्य तेजोभिस्तेन सन्तापिता वयम्}


\twolineshloka
{तस्य दर्शनमिष्टं वो यदि देवा विभावसोः}
{तत्रैवमधिगच्छध्वं कार्यं वो यदि वह्निना}


\twolineshloka
{गम्यतां साधयिष्यामो वयं ह्यग्निभयात्सुराः}
{एतावदुक्त्वा मण्डूकस्त्वरितो जलमाविशत्}


\twolineshloka
{हुताशनस्तु बुबुधे मण्डूकस्य च पैशुनम्}
{शशाप स तमासाद्य न रसान्वेत्स्यसीति वै}


\twolineshloka
{तं वै संयुज्य शापेन मण्डूकं त्वरितो ययौ}
{अन्यत्र वासाय विभुर्न चात्मानमदर्शयत्}


\threelineshloka
{देवास्त्वनुग्रहं चक्रुर्मण्डूकानां भृगूत्तम}
{यत्तच्छृणु महाबाहो गदतो मम सर्वशः ॥देवा ऊचुः}
{}


\twolineshloka
{अग्निशापादजिह्वाऽपि रसज्ञानबहिष्कृताः}
{सरस्वतीं बहुविधां यूममुच्चारयिष्यथ}


\twolineshloka
{बिलवासं गतांश्चैव निराहारानचेतसः}
{गतासूनपि वः शुष्कान्भूमिः सन्धारयिष्यति}


\twolineshloka
{तमोघनायामपि वै निशायां विचरिष्यथ}
{इत्युक्त्वा तांस्ततो देवाः पुनरेव महीमिमाम्}


\twolineshloka
{परीयुर्ज्वलनस्यार्थे न चाविन्दन्हुताशनम्}
{अथ तान्द्विरदः कश्चित्सुरेन्द्रद्विरदोपमः}


\twolineshloka
{अश्वत्थस्थोऽग्निरित्येवमाह देवान्भृगूद्वह}
{शशाप ज्वलनः सर्वान्द्विरदान्क्रोधमूर्च्छितः}


\threelineshloka
{प्रतीपा भवतां जिह्वा भवित्रीति भृगूद्वह}
{इत्युक्त्वा निःसृतोऽश्वत्थादग्निर्वारणसूचितः}
{प्रविवेश शमीगर्भमथ वह्निः सुषुप्सया}


\threelineshloka
{अनुग्रहं तु नागानां यं चक्रुः शृणु तं प्रभो}
{देवा भृगुकुलश्रेष्ठ प्रीत्या सत्यपराक्रमाः ॥देवा ऊचुः}
{}


\threelineshloka
{प्रतीपया जिह्वयाऽपि सर्वाहारान्हरिष्यथ}
{वाचं चोच्चारयिष्यध्वमुच्चैरव्यञ्जिताक्षराम्}
{इत्युक्त्वा पुनरेवाग्निमनुसस्रुर्दिवौकसः}


\twolineshloka
{अश्वत्थान्निःसृतश्चाग्निः शमीगर्भमुपाविशत्}
{शुकेन ख्यापितो विप्र तं देवाः समुपाद्रवन्}


\twolineshloka
{शशाप सुकमग्निस्तु वाग्विहीनो भविष्यसि}
{जिह्वामावर्तयामास तस्यापि हुतभुक्तदा}


\twolineshloka
{दृष्ट्वा तु ज्वलनं देवाः शुकमूचुर्दयान्विताः}
{भविता न त्वमत्यन्तं शुकत्वे नष्टवागिति}


\twolineshloka
{आवृत्तजिह्वस्य सतो वाक्यं कान्तं भविष्यति}
{बालस्येव प्रवृद्धस्य कलमव्यक्तमद्भुतम्}


\twolineshloka
{इत्युक्त्वा तं शमीगर्भे वह्निमालक्ष्य देवताः}
{तदेवायतनं चक्रुः पुण्यं सर्वक्रियास्वपि}


\twolineshloka
{तदाप्रभृति चाप्यग्निः शमीगर्भेषु दृश्यते}
{उत्पादने तथोपायमभिजग्मुश्च मानवाः}


\threelineshloka
{आपो रसातले यास्तु संस्पृष्टाश्चित्रभानुना}
{ताः पर्वतप्रस्रवणैरूष्मां मुञ्चन्ति भार्गव}
{पावकेनाधिशयता सन्तप्तास्तस्य तेजसा}


\twolineshloka
{अथाग्निर्देवता दृष्ट्वा बभूव व्यथितस्तदा}
{किमागमनमित्येवं तानपृच्छत पावकः}


\fourlineindentedshloka
{तमूचुर्विबुधाः सर्वे ते चैव परमर्षयः}
{त्वां नियोक्ष्यामहे कार्ये तद्भवान्कर्तुमर्हति}
{कृते च तस्मिन्भविता तवापि सुमहान्गुणः ॥अग्निरुवाच}
{}


\threelineshloka
{ब्रूत यद्भवतां कार्यं कर्तास्मि तदहं सुराः}
{भवतां तु नियोज्योस्मि मावोत्रास्तु विचारणा ॥देवा ऊचुः}
{}


\twolineshloka
{असुरस्तारको नाम ब्रह्मणो वरदर्पितः}
{अस्मान्प्रबाधते वीर्याद्वधस्तस्य विधीयताम्}


\twolineshloka
{इमान्देवगणांस्तात प्रजापतिगणांस्तथा}
{ऋषींश्चापि महाभाग परित्रायस्व पावक}


\twolineshloka
{अपत्यं तेजसा युक्तं प्रवीरं जनयक प्रभो}
{यद्भयं नोऽसुरात्तस्मान्नाशयेद्धव्यवाहन}


\twolineshloka
{शप्तानां नो महादेव्या नान्यदस्ति परायणम्}
{अन्यत्र भवतो वीर तस्मात्त्रायस्व नः प्रभो}


\twolineshloka
{इत्युक्तः स तथेत्युक्त्वा भगवान्हव्यवाहनः}
{जगामाथ दुराधर्षो गङ्गां भागीरथीं प्रति}


\twolineshloka
{तया चाप्यभवन्मिश्रो गर्भं चास्यां दधे तदा}
{ववृधे स तदा गर्भः कक्षे कृष्णगतिर्यथा}


\twolineshloka
{तेजसा तस्य देवस्य गङ्गा विह्वलचेतना}
{सन्तापमगमत्तीव्रं वोढुं सा न शशाक ह}


\twolineshloka
{आहिते ज्वलनेनाथ गर्भे तेजःसमन्विते}
{गङ्गायामसुरः कश्चिद्भैरवं नादमानदत्}


% Check verse!
अबुद्धिपतितेनाथ नादेन विपुलेन सा ॥वित्रस्तोद्धान्तनयना गङ्गा विप्लुतलोचना
\twolineshloka
{विसंज्ञा नाशकद्गर्भं वोढुमात्मानमेव च}
{सा तु तेजःपरीताङ्गी कम्पमाना च जाह्नवी}


\twolineshloka
{उवाच ज्वलनं विप्र तदा गर्भबलोद्धुता}
{नते शक्ताऽस्मि भगवंस्तेजसोऽस्य विधारणे}


\twolineshloka
{विमूढाऽस्मि कृताऽनेन न मे स्वास्ध्यं यथा पुरा}
{विह्वला चास्मि भगवंश्चेतो नष्टं च मेऽनघ}


\twolineshloka
{धारणे नास्य शक्ताऽहं गर्भस्य तपतांवर}
{उत्स्रक्ष्येऽहमिमं दुःखान्न तु कामात्कथञ्चन}


\twolineshloka
{न तेजसाऽस्ति संस्पर्शो मम देव विभावसो}
{आपदर्थे हि सम्बन्धः सुसूक्ष्मोऽपि महाद्युते}


\twolineshloka
{यदत्र गुणसम्पन्नमितरद्वा हुताशन}
{त्वय्येव तदहं मन्ये धर्माधर्मौ च केवलौ}


\twolineshloka
{तामुवाच ततो वह्निर्धार्यतां धार्यतामिति}
{गर्भो मत्तेजसा युक्तो महागुणफलोदयः}


\twolineshloka
{शक्ता ह्यसि महीं कृत्स्नां वोढुं धारयितुं तथा}
{न हि ते किञ्चिदप्राप्यमन्यतो धारणादृते}


\threelineshloka
{`एवमुक्ता तु सा देवी तत्रैवान्तरधीयत}
{पावकश्चापि तेजस्वी कृत्वा कार्यं दिवौकसाम्}
{जगामेष्टं तदा देशं ततो भार्गवनन्दन ॥'}


\twolineshloka
{सा वह्निना वार्यमाणा देवैरपि सरिद्वरा}
{समुत्ससर्ज तं गर्भं मेरौ गिरिवेर तदा}


\threelineshloka
{समर्था धारणे चापि रुद्रतेजःप्रधर्षिता}
{नाशकत्सा तदा गर्भं सन्धारयितुमोजसा}
{समुत्ससर्ज तं दुःखाद्दीप्तवैश्वानरप्रभम्}


\twolineshloka
{दर्शयामास चाग्निस्तां तदा गङ्गां भृगूद्वह}
{पप्रच्छ सरितां श्रेष्ठां कच्चिद्गर्भः सुखोदयः}


\threelineshloka
{कीदृग्गुणोपि वा देवि कीदृग्रूपश्च दृश्यते}
{तेजसा केन वा युक्तः सर्वमेतद्ब्रवीहि मे ॥गङ्गोवाच}
{}


\twolineshloka
{जातरूपः स गर्भो वै तेजसा त्वमिवानघ}
{सुवर्णो विमलो दीप्तः पर्वतं चावभासयन्}


\twolineshloka
{पद्मोत्पलविमिश्राणां ह्रदानामिव शीतलः}
{गन्धोस्य स कदम्बानां तुल्यो वै तपतांवर}


\threelineshloka
{तेजसा तस्य गर्भस्य भास्करस्येव रश्मिभिः}
{यद्द्रव्यं परिसंसृष्टं पृथिव्यां पर्वतेषु च}
{तत्सर्वं काञ्चनीभूतं समन्तात्प्रत्यदृश्यत}


\twolineshloka
{पर्यधावत शैलांश्च नदीः प्रस्रवणानि च}
{व्यादीपयत्तेजसा च त्रैलोक्यं सचराचरम्}


\threelineshloka
{एवंरूपः स भगवान्पुत्रस्ते हव्यवाहन}
{सूर्यवैश्वानरसमः कान्त्या सोम इवापरः ॥वसिष्ठ उवाच}
{}


\threelineshloka
{एवमुक्त्वा तु सा देवी तत्रैवान्तरधीयत}
{पावकश्चापि तेजस्वी कृत्वा कार्यं दिवौकसाम्}
{जगामेष्टं ततो देशं तदा भार्गवनन्दन}


\threelineshloka
{एतैः कर्मगुणैर्लोके नामाग्नेः परिगीयते}
{हिरण्यरेता इति वै ऋषिभिर्विबुधैस्तथा}
{पृथिवी च तदा देवी ख्याता वसुमतीति वै}


\twolineshloka
{स तु गर्भो महातेजा गाङ्गेयः पावकोद्भवः}
{दिव्यं शरवणं प्राप्य ववृधेऽद्भुतदर्शनः}


\twolineshloka
{ददृशुः कृत्तिकास्तं तु बालार्कसदृशद्युतिम्}
{जातस्नेहास्तु तं बालं पुपुषुः स्तन्यविस्रवैः}


\twolineshloka
{ततः स कार्तिकेयत्वमवाप परमद्युतिः}
{स्कन्नत्वात्स्कन्दतां चापि गुहावासाद्गुहोऽभवत्}


\twolineshloka
{एवं सुवर्णमुत्पन्नमपत्यं जातवेदसः}
{तत्र जाम्बूनदं श्रेष्ठं देवानामपि भूषणम्}


\twolineshloka
{ततःप्रभृति चाप्येतञ्जातरूपमुदाहृतम्}
{रत्नानामुत्तमं रत्नं भूषणानां तथैव च}


\twolineshloka
{पवित्रं च पवित्राणां मङ्गलानां च मङ्गलम्}
{यत्सुवर्णं स भगवानग्निरीशः प्रजापतिः}


\twolineshloka
{पवित्राणां पवित्रं हि कनकं द्विजसत्तमाः}
{अग्नीषोमात्मकं चैव जातरूपमुदाहृतम्}


\chapter{अध्यायः १३२}
\twolineshloka
{अपि चेदं पुरा राम श्रुतं मे ब्रह्मदर्शनम्}
{पितामहस्य यद्वृत्तं ब्रह्मणः परमात्मनः}


\twolineshloka
{देवस्य महतस्तात वारुणीं बिभ्रतस्तनुम्}
{ऐश्वर्ये वारुणेवाऽथ रुद्रस्येशस्य वै प्रभो}


\twolineshloka
{आजग्मुर्मुनयः सर्वे देवाश्चाग्निपुरोगमाः}
{यज्ञाङ्गानि च सर्वाणि वषट्कारश्च मूर्तिमान्}


\twolineshloka
{मूर्तिमन्ति च सामानि यजूंषि च सहस्रशः}
{ऋग्वेदश्चागमत्तत्र पदक्रमविभूषितः}


\twolineshloka
{लक्षणानि स्वरास्तोभा निरुक्ताः स्वरभक्तयः}
{ओंकारश्छन्दसां नेत्रं निग्रहप्रग्रहौ तथा}


\twolineshloka
{वेदाश्च सोपनिषदो विद्या सावित्र्यथापि च}
{भूतं भव्यं भविष्यच्च दधार भगवाञ्शिवः}


\twolineshloka
{संजुहावात्मनाऽऽत्मानं स्वयमेव तदा प्रभो}
{यज्ञं च शोभयामास बहुरूपं पिनाकधृत्}


\twolineshloka
{द्यौर्नभः पृथिवी खं च तथा चैवैष भूपतिः}
{सर्वविद्येश्वरः श्रीमानेष चापि विभावसुः}


\twolineshloka
{एष ब्रह्मा शिवो रुद्रो वरुणोऽग्निः प्रजापतिः}
{कीर्त्यते भगवान्देवः सर्वभूतपतिः शिवः}


\twolineshloka
{तस्य यज्ञः पशुपतेस्तपः क्रतव एव च}
{दीक्षा दीप्तव्रता देवी दिशश्च सदिगीश्वराः}


\threelineshloka
{देवपत्न्यश्च कन्याश्च देवानां चैव मातरः}
{आजग्मुः सहितास्तत्र तदा भृगुकुलोद्वह}
{यज्ञं पशुपतेः प्रीता वरुणस्य महात्मनः}


% Check verse!
स्वयंभुवस्तु ता दृष्ट्वा रेतः समपतद्भुवि
\twolineshloka
{तस्य शुक्रस्य निष्यन्दान्पांसून्सङ्गृह्य भूमितः}
{प्रास्यत्पूषा कराभ्यां वै तस्मिन्नेव हुताशने}


\twolineshloka
{ततस्तस्मिन्सम्प्रवृत्ते सत्रे ज्वलितपावके}
{ब्रह्मणो जुह्वतस्तत्र प्रादुर्भावो बभूवह}


\twolineshloka
{स्कन्नमात्रं च तच्छुक्रं स्रुवेण परिगृह्य सः}
{आज्यवन्मन्त्रतश्चापि सोऽजुहोद्भृगुनन्दन}


\twolineshloka
{ततः स जनयामास भूतग्रामं च वीर्यवान्}
{तस्य तत्तेजसस्तस्माज्जज्ञे लोकेषु तैजसम्}


\threelineshloka
{तमसस्तामसा भावा व्यापि सत्वं तथोभयम्}
{स गुणस्तेजसो नित्यं तमस्याकाशमेव च}
{सर्वभूतेषु च तथा सत्वं तेजस्तथोत्तमम्}


\twolineshloka
{शुक्रे हुतेऽग्नौ तस्मिंस्तु प्रादुरासंस्त्रयः प्रभो}
{पुरुषा वपुषा युक्ताः स्वैः स्वैः प्रसवजैर्गुणैः}


\threelineshloka
{भर्जनाद्भृगुरित्येवमङ्गारेभ्योऽङ्गिराऽभवत्}
{अङ्गारसंश्रयाच्चैव कविरित्यपरोऽभवत्}
{सह ज्वालाभइरुत्पन्नो भृगुस्तस्माद्भृगुः स्मृतः}


\twolineshloka
{मरीचिभ्यो मरीचिस्तु मारीचः कश्यपो ह्यभूत्}
{अङ्गोरभ्योऽङ्गिरास्तात वालखिल्याः कुशोच्चयात्}


% Check verse!
अत्रैवात्रेति च विभो जातमत्रिं वदन्त्यपि
\twolineshloka
{तथा भस्मव्यपोहेभ्यो ब्रह्मर्षिगणसम्मताः}
{वैखानसाः समुत्पन्नास्तपः श्रुतगुणेप्सवः}


\threelineshloka
{अश्रुतोऽस्य समुत्पन्नावश्विनौ रूपसम्मतौ}
{शेषाः प्रजानां पतयः स्रोतोभ्यस्तस्य जज्ञिरे}
{ऋषयो रोमकूपेभ्यः स्वेदाच्छन्दो बलान्मनः}


\twolineshloka
{एतस्मात्कारणादाहुरग्निः सर्वास्तु देवताः}
{ऋषयः श्रुतसम्पन्ना वेदप्रामाण्यदर्शनात्}


\twolineshloka
{यानि दारूणि निर्यासास्ते मासाः पक्षसंज्ञिताः}
{अहोरात्रा मुहूर्ताश्च वीतज्योतिश्च वारुणम्}


\twolineshloka
{रौद्रं लोहितमित्याहुर्लोहितात्कनकं स्मृतम्}
{तन्मैत्रमिति विज्ञेयं धूमाच्च वसवः स्मृताः}


\twolineshloka
{अर्चिषो याश्च ते रुद्रास्तथाऽऽदित्या महाप्रभाः}
{उद्दीप्तास्ते तथाऽङ्गारा ये धिष्ण्येषु दिवि स्थिताः}


\twolineshloka
{अग्निर्नाथश्च लोकस्य तत्परं ब्रह्म तद्भुवम्}
{सर्वकामदमित्याहुस्तत्र हव्यमुपावहन्}


\twolineshloka
{ततोऽब्रवीन्महादेवो वरुणः पवनात्मकः}
{मम सत्रमिदं दिव्यमहं गृहपतिस्त्विह}


\threelineshloka
{त्रीणि पूर्वाण्यपत्यानि मम तानि न संशयः}
{इति जानीत खगमा मम यज्ञफलं हि तत् ॥अग्निरुवाच}
{}


\twolineshloka
{मदङ्गेभ्यः प्रसूतानि मदाश्रयकृतानि च}
{ममैव तान्मपत्यानि मम शुक्लं हुतं हि तत्}


\twolineshloka
{अथाब्रवील्लोकगुरुर्ब्रह्मा लोकपितामहः}
{ममैव तान्यपत्यानि मम शुक्लं हुतं हि तत्}


\twolineshloka
{अहं वक्ता च मन्त्रस्य होता शुक्रस्य चैव ह}
{यस्य बीजं फलं तस्य शुक्रं चेत्कारणं मतम्}


\twolineshloka
{ततोऽब्रुवन्देवगणाः पितामहमुपेत्य वै}
{कृताञ्जलिपुटाः सर्वे शिरोभिरभिवन्द्य च}


\threelineshloka
{वयं च भगवन्सर्वे जगच्च सचराचरम्}
{तवैव प्रसवाः सर्वे तस्मादग्निर्विभावसुः}
{वरुणश्चेश्वरो देवो लभतां काममीप्सितम्}


\twolineshloka
{निसर्गाद्ब्रह्मणश्चापि वरुणो यादसाम्पतिः}
{जग्राह वै भृगु पूर्वमपत्यं सूर्यवर्चसम्}


\twolineshloka
{ईश्वरोऽङ्गिरसं चाग्नेरपत्यार्थमकल्पयत्}
{पितामहस्त्वपत्यं वैकविं जग्राह तत्त्ववित्}


\threelineshloka
{तदा स वारुणिः ख्यातो भृगुः प्रसवकर्मकृत्}
{आग्नेयस्त्वङ्गिराः श्रीमान्कविर्ब्राह्मो महायशाः}
{भार्गवाङ्गिरसौ लोके लोकसन्तानलक्षणौ}


\twolineshloka
{एते विप्रवराः सर्वे प्रजानां पतयस्त्रयः}
{सर्वं सन्तानमेतेषामिदमित्युपधारय}


\twolineshloka
{भृगोस्तु पुत्राः सप्तासन्सर्वे तुल्या भृगोर्गुणैः}
{च्यवनो वज्रशीर्षश्च शुचिरौर्वस्तथैव च}


\twolineshloka
{शुक्रो वरेण्यश्च विभुः सवनश्चेति सप्त ते}
{भार्गवा वारुणाः सर्वे येषां वंशे भवानपि}


\twolineshloka
{अषअटौ चाङ्गिरसः पुत्रा वारुणास्तेऽप्यवारुणाः}
{बृहस्पतिरुचथ्यश्च वयस्यः शान्तिरेव च}


\twolineshloka
{घोरो विरूपः संवर्तः सुधन्वा चाष्टमः स्मृतः}
{एतेऽष्टौ वह्निजाः सर्वे ज्ञाननिष्ठा निरामयाः}


\twolineshloka
{ब्राह्मणाश्च कवेः पुत्रा वारुणास्तेऽप्युदाहृताः}
{अष्टौ प्रसवजैर्युक्ता गुणैर्ब्रह्मविदः शुभाः}


\twolineshloka
{कविः काव्यश्च विष्णुश्च बुद्धिमानुशना तथा}
{भृगुश्च वरुणश्चैव काश्यपोऽग्निश्च धर्मवित्}


\twolineshloka
{अष्टौ कविसुता ह्येते सर्वमेभिर्जगत्ततम्}
{प्रजापतय एते हि प्रजानां यैरिमाः प्रजाः}


\twolineshloka
{एवमङ्गिरसश्चैव कवेश्च प्रसवान्वयैः}
{भृगोश्च भृगुशार्दूल वंशजैः सततं जगत्}


\twolineshloka
{वरुणश्चादितो विप्र जग्राह प्रभुरीश्वरः}
{कविं तात भृगुं चापि तस्मात्तौ वारुणौ स्मृतौ}


\twolineshloka
{जग्राहाङ्गिरसं देवः शिखी तस्माद्भुताशनः}
{तस्मादाङ्गिरसा ज्ञेयाः सर्व एव तदन्वयाः}


\twolineshloka
{ब्रह्मा पितामहः पूर्वं देवताभिः प्रसादितः}
{इमे नः सन्तरिष्यन्ति प्रजाभिर्जगदीश्वराः}


\twolineshloka
{सर्वे प्रजानां पतयः सर्वे चातितपस्विनः}
{त्वत्प्रसादादिमं लोकं धारयिष्यन्ति शाश्वतं}


\twolineshloka
{तथैव वंशकर्तारस्तव तेजोविवर्धनाः}
{भवेयुर्वेदविदुषः सर्वे च कृतिनस्तथा}


\twolineshloka
{देवपक्षचराः सौम्याः प्राजापत्या महर्षयः}
{अनन्तं ब्रह्म सत्यं च तपश्च परमं भुवि}


\twolineshloka
{सर्वे हि वयमेते च तवैव प्रसवाः प्रभो}
{देवानां ब्राह्मणानां च त्वं हि कर्ता पितामह}


\twolineshloka
{मारीचमादितः कृत्वा सर्वे चैवाथ भार्गवाः}
{अपत्यानीति सम्प्रेक्ष्य क्षमयाम पितामह}


\twolineshloka
{अथ स्वेनैव रूपेणि प्रजनिष्यन्ति वै प्रजाः}
{स्थापयिष्यन्ति चात्मानं युगादिनिधने तथा}


\twolineshloka
{इत्युक्तः स तदा तैस्तु ब्रह्मा लोकपितामहः}
{ततेत्येवाब्रवीत्प्रीतस्तेऽपि जग्मुर्यथागतम्}


\twolineshloka
{एवमेतत्पुरावृत्तं तस्य यज्ञे महात्मनः}
{देवश्रेष्ठस्य लोकादौ वारुणीं बिभ्रतस्तनुम्}


\twolineshloka
{अग्निर्ब्रह्म पशुपतिः शर्वो रुद्रः प्रजापतिः}
{अग्नेरपत्यमेतद्वै सुवर्णमिति धारणाः}


\twolineshloka
{अग्न्यभावे च कुरुते वह्निस्थानेषु काञ्चनम्}
{जामदग्न्यप्रमाणज्ञो वेदश्रुतिनिदर्शनात्}


\twolineshloka
{कुशस्तम्बे जुहोत्यग्निं सुवर्णे तत्र च स्थिते}
{वल्मीकस्य वपायां च कर्णे वाजस्य दक्षिणे}


\twolineshloka
{शकटोर्व्या परस्याप्सु ब्राह्मणस्य करे तथा}
{हुते प्रीतिकरीमृद्धिं भगवांस्तत्र मन्यते}


\twolineshloka
{तस्मादग्निपराः सर्वे देवता इति शुश्रुम}
{ब्रह्मणो हि प्रभूतोऽग्निरग्नेरपि च काञ्चनम्}


\twolineshloka
{तस्माद्ये वै प्रयच्छन्ति सुवर्णं धर्मदर्शिनः}
{देवतास्ते प्रयच्छन्ति समस्ता इति नः श्रुतम्}


\twolineshloka
{तस्य वा तपसो लोकान्गच्छतः परमां गतिम्}
{स्वर्लोके राजराज्येन सोभिषिच्येत भार्गव}


\twolineshloka
{आदित्योदयने प्राप्ते विधिमन्त्रपुरस्कृतम्}
{ददाति काञ्चनं यो वाः दुःस्वप्नं प्रतिहन्ति सः}


\twolineshloka
{ददात्युदितमात्रे यस्यस्य पाप्मा विधूयते}
{मध्याह्ने ददतो रुक्मं हन्ति पापमनागतम्}


\twolineshloka
{ददाति पस्चिमां सन्ध्यां यः सुवर्णं यतव्रतः}
{ब्रह्मवाय्वग्निसोमानां सालोक्यमुपयाति सः}


\twolineshloka
{सेन्द्रेषु चैव लोकेषु प्रतिष्ठा विन्दते शुभाम्}
{इह लोके यशः प्राप्य शान्तपाप्मा च मोदते}


\twolineshloka
{ततः सम्पद्यतेऽन्येषु लोकेष्वप्रतिमः सदा}
{अनावृतगतिश्चैव कामचारो भवत्युत}


\twolineshloka
{न च क्षरति तेभ्यश्च यशश्चैवाप्नुते महत्}
{सुवर्णमक्षयं दत्त्वा लोकांश्चाप्नोति पुष्कलान्}


\twolineshloka
{यस्तु सञ्जनयित्वाऽग्निमादित्योदयनं प्रति}
{दद्याद््वै व्रतमुद्दिश्य सर्वकामान्समश्नुते}


\twolineshloka
{अग्निरित्येव तत्प्राहुः प्रदानं च सुखावहम्}
{यथेष्टगुणसंवृत्तं प्रवर्तकमिति स्मृतम्}


\twolineshloka
{एषा सुवर्णस्योत्पत्तिः कथिता ते मयाऽनघ}
{कार्तिकेयस्य च विभो तद्विद्धि भृगुनन्दन}


\twolineshloka
{कार्तिकेयस्तु संवृद्धः कालेन महता तदा}
{देवैः सेनापतित्वेन वृतः सेन्द्रैर्भृगूद्वह}


\twolineshloka
{जघान तारकं चापि दैत्यमन्यांस्तथाऽसुरान्}
{त्रिदशेन्द्राज्ञया ब्रह्मँल्लोकानां हितकाम्यया}


\threelineshloka
{सुवर्णदाने च मया कथितास्ते गुणा विभो}
{तस्मात्सुवर्णं विप्रेभ्यः प्रयच्छ तदतांवर ॥भीष्म उवाच}
{}


\twolineshloka
{इत्युक्तः स वसिष्ठेन जामदग्न्यः प्रतापवान्}
{ददौ सुवर्णं विप्रभ्यो व्यमुच्यत च किल्बिषात्}


\twolineshloka
{एतत्ते सर्वमाख्यातं सुवर्णस्य महीपते}
{प्रदानस्य फलं चैव जन्म चास्य युधिष्ठिर}


\twolineshloka
{तस्मात्त्वमपि विप्रेभ्यः प्रयच्छ कनकं बहुः}
{ददत्सुवर्णं नृपते किल्बिषाद्विप्रमोक्ष्यसि}


\chapter{अध्यायः १३३}
\twolineshloka
{उक्ताः पितामहेनेह सुवर्णस्य विधानतः}
{विस्तरेण प्रदानस्य ये गुणाः श्रुतिलक्षणाः}


\twolineshloka
{यत्तु कारणमुत्पत्तेः सुवर्णस्य प्रकीर्तितम्}
{स कथं तारकः प्राप्तो निधनं तद्ब्रवीहि मे}


\twolineshloka
{उक्तं स दैवतानां हि अवध्य इति पार्थिव}
{कथं तस्याभवन्मृत्युर्विस्तरेण प्रकीर्तय}


\threelineshloka
{एतदिच्छाम्यहं श्रोतुं त्वत्तः कुरुकुलोद्वह}
{कार्त्स्न्येन तारकवधं परं कौतूहलं हि मे ॥भीष्म उवाच}
{}


\twolineshloka
{विपन्नकृत्या राजेन्द्र देवता ऋषयस्तथा}
{कृत्तिकाश्चोदयामासुरपत्यभरणाय वै}


\twolineshloka
{न देवतानां काचिद्धि समर्था जातवेदसः}
{एता हि शक्तास्तं गर्भं सन्धारयितुमोजसा}


\twolineshloka
{षण्णां तासां ततः प्रीतः पावको गर्भधारणात्}
{स्वेन तेजोविसर्गेण वीर्येण परमेण च}


\twolineshloka
{तास्तु षट् कृत्तिका गर्भं पुपुषुर्जातवेदसः}
{षट्सु वर्त्मसु तेजोऽग्नेः सकलं निहितं प्रभो}


\twolineshloka
{ततस्ता वर्धमानस्य कुमारस्य महात्मनः}
{तेजसाऽभिपरीताङ्ग्यो न क्वचिच्छर्म लेभिरे}


\twolineshloka
{ततस्तेजःपरीताङ्ग्यः सर्वाः काल उपस्थिते}
{समं गर्भं सुषुविरे कृतिकास्ता नरर्षभ}


\twolineshloka
{ततस्तं ष़डधिष्ठानं गर्भमेकत्वमागतम्}
{पृथिवी प्रतिजग्राह कृत्तिकानां समीपतः}


\twolineshloka
{स गर्भो दिव्यसंस्थानो दीप्तिमान्पावकप्रभः}
{दिव्यं शरवणं प्राप्य ववृधे प्रियदर्शनः}


\twolineshloka
{ददृशुः कृत्तिकास्तं तु बालमर्कसमद्युतिम्}
{जातस्नेहाश्च सौहार्दात्पुपुषुः स्तन्तविस्रवैः}


\twolineshloka
{अभवत्कार्तिकेयः स त्रैलोक्ये सचराचरे}
{स्कन्नत्वात्स्कन्दतां प्राप्तो गुहावासाद्गुहोऽभवत्}


\twolineshloka
{ततो देवास्त्रयस्त्रिंशद्दिशश्च सदिगीश्वराः}
{रुद्रो धाता च विष्णुश्च यमः पूषाऽर्यमा भगः}


\twolineshloka
{अंशो मित्रश्च साध्याश्च वासवो वसवोऽश्विनौ}
{आपो वायुर्नभश्चन्द्रो नक्षत्राणि ग्रहा रविः}


\twolineshloka
{पृथग्भूतानि चान्यानि यानि देवगणानि वै}
{आजग्मुस्तेऽद्भुतं द्रष्टुं कुमारं ज्वलनात्मजम्}


\twolineshloka
{ऋषयस्तुष्टुवुश्चैव गन्धर्वाश्च जगुस्तथा}
{षडाननं कुमारं तु द्विषडक्षं द्विजप्रियम्}


\threelineshloka
{पीनांसं द्वादशभुजं पावकादित्यवर्चसम्}
{शयानं शरगुल्मस्थं दृष्ट्वा देवाः सहर्षिभिः}
{लेभिरे परमं हर्षं मेनिरे चासुरं हतम्}


\twolineshloka
{ततो देवाः प्रियाण्यस्य सर्व एव समाहरन्}
{क्रीडतः क्रीडनीयानि ददुः पक्षिगणांश्च ह}


\twolineshloka
{सुपर्णोऽस्य ददौ पुत्रं मयूरं चित्रबर्हिणम्}
{राक्षसाश्च ददुस्तस्मै वराहमहिषावुभौ}


\twolineshloka
{कुक्कुटं चाग्निसङ्काशं प्रददावरुणः स्वयम्}
{चन्द्रमाः प्रददौ मेषमादित्यो रुचिरां प्रभाम्}


\twolineshloka
{गवां माता च गा देवी ददौ शतसहस्रशः}
{छागमग्निर्गुणोपेतमिला पुष्पफलं बहु}


\twolineshloka
{सुधन्वा शकटं चैव रथं चाञ्चितकूबरम्}
{वरुणो वारुणान्दिव्यान्सगजान्प्रददौ शुभान्}


\twolineshloka
{सिंहान्सुरेन्द्रो व्याघ्रांश्च द्विपानन्यांश्च दंष्ट्रिणः}
{श्वापदांश्च बहून्घोराञ्शस्त्राणि विविधानि च}


% Check verse!
राक्षसासुरसङ्घाश्च अनुजग्मुस्तमीश्वरम्
\twolineshloka
{वर्धमानवधोपायं प्रार्थयामास तारकः}
{उपायैर्बहुभिर्हन्तुं नाशकच्चापि तं विभुम्}


\twolineshloka
{सैनापत्येन तं देवाः पूजयित्वा गुहालयम्}
{शशंसुर्विप्रकारं तं तस्मै तारककारितम्}


\twolineshloka
{स विवृद्धो महावीर्यो देवसेनापतिः प्रभुः}
{जघानामोघया शक्त्या दानवं तारकं गुहः}


\twolineshloka
{तेन तस्मिन्कुमारेण क्रीडता निहतेऽसुरे}
{सुरेन्द्रः स्थापितो राज्ये देवानां पुनरीश्वरः}


\twolineshloka
{स सेनापतिरेवाथ बभौ स्कन्दः प्रतापवान्}
{ईशो गोप्ता च देवानां प्रियकृच्छङ्करस्य च}


\twolineshloka
{हिरण्यमूर्तिर्भगवानेव एव च पावकिः}
{सदा कुमारो देवानां सैनापत्यमवाप्तवान्}


\twolineshloka
{तस्मात्सुवर्णं मङ्गल्यं रत्नमक्षय्यमुत्तमम्}
{सहजं कार्तिकेयस्य वह्नेस्तेजः परं मतम्}


\twolineshloka
{एवं रामाय कौरव्य वसिष्ठोऽकथयत्पुरा}
{तस्मात्सुवर्णदानाय प्रयतस्व नराधिप}


\twolineshloka
{रामः सुवर्णं दत्त्वा हि विमुक्तः सर्वकिल्बिषैः}
{त्रिविष्टपे महत्स्थानमवापासुलभं नरैः}


\chapter{अध्यायः १३४}
\threelineshloka
{चातुर्वर्ण्यस्य धर्मात्मन्धर्माः प्रोक्ता यथा त्वया}
{तथैव मे श्राद्धविधिं कृत्स्नं प्रब्रूहि पार्थिव ॥वैशम्पायन उवाच}
{}


\threelineshloka
{युधिष्ठिरेणैवमुक्तो भीष्मः शान्तनवस्तदा}
{इमं श्राद्धविधिं कृत्स्नं वक्तुं समुपचक्रमे ॥भीष्म उवाच}
{}


\twolineshloka
{शृणुष्वावहितो राजञ्श्राद्धकर्मविधिं शुभम्}
{धन्यं यशस्यं पुत्रीयं पितृयज्ञं परन्तप}


\twolineshloka
{देवासुरमनुष्याणां गन्धर्वोरगरक्षसाम्}
{पिशाचकिन्नराणां च पूज्या वै पितरः सदा}


\twolineshloka
{पितॄन्पूज्यादितः पश्चाद्देवतास्तर्पयन्ति वै}
{तस्मात्तान्सर्वयत्नेन पुरुषः पूजयेत्सदा}


\twolineshloka
{अन्वाहार्यं महाराज पितॄणां श्राद्धमुच्यते}
{तस्माद्विशेषविधिना विधिः प्रथमकल्पितः}


\threelineshloka
{सर्वेष्वहःसु प्रीयन्ते कृते श्राद्धे पितामहाः}
{`पिण्डान्वाहार्यकं श्राद्धं कुर्यान्मासानुमासिकम्}
{पितृयज्ञं तु निर्वर्त्य विप्रश्चन्द्रक्षयेऽग्निमान्}


\twolineshloka
{पिण्डानां मासिकश्राद्धमन्वाहार्यं विदुर्बुधाः}
{तदामिषेण कुर्वीत प्रयतः प्राञ्जलिः शुचिः ॥'}


\threelineshloka
{प्रवक्ष्यामि तु ते सर्वांस्तिथ्यांतिथ्यां दिने गुणान्}
{येष्वहःसु कृतैः श्राद्धैर्यत्फलं प्राप्यतेऽनघ}
{तत्सर्वं कीर्तयिष्यामि यथावत्तन्निबोध मे}


\twolineshloka
{पितॄनर्च्य प्रतिपदि प्राप्नुयात्स्वगृहे स्त्रियः}
{अभिरूपप्रजायिन्यो दर्शनीया बहुप्रजाः}


\twolineshloka
{स्त्रियो द्वितीयां जायन्ते तृतीयायां तु वाजिनः}
{चतुर्थ्यां क्षुद्रपशवो भवन्ति बहवो गृहे}


\twolineshloka
{पञ्चम्यां बहवः पुत्रा जायन्ते कुर्वतां नृप}
{कुर्वाणास्तु नराः षष्ठ्यां भवन्ति द्युतिभागिनः}


\twolineshloka
{कृषिभागी भवेच्छ्राद्धं कुर्वाणः सप्तमीं नृप}
{अष्टम्यां तु प्रकुर्वाणो वाणिज्ये लाभमाप्नुयात्}


\twolineshloka
{नवम्यां कुर्वतः श्राद्धं भवत्येकशफं बहु}
{विवर्धन्ते तु दशमीं गावः श्राद्धानि कुर्वतः}


\twolineshloka
{कुप्यभागी भवेन्मर्त्यः कुर्वन्नेकादशीं नृप}
{ब्रह्मवर्चस्विनः पुत्रा जायन्ते तस्य वेश्मनि}


\twolineshloka
{द्वादश्यामीहमानस्य नित्यमेव प्रदृश्यते}
{रजतं बहुवित्तं च सुवर्णं च मनोरमम्}


% Check verse!
ज्ञातीनां तु भवेच्छ्रेष्ठः कुर्वञ्श्राद्धं त्रयोदशीम्
\twolineshloka
{अवश्यं तु युवानोऽस्य प्रमीयन्ते नरा गृहे}
{युद्धभागी भवेन्मर्त्यः कुर्वञ्श्राद्धं चतुर्दशीम्}


% Check verse!
अमावास्यां तु निवपन्सर्वकामानवाप्नुयात्
\twolineshloka
{कृष्णपक्षे दशम्यादौ वर्जयित्वा चतुर्दशीम्}
{श्राद्धकर्मणि तिथ्यस्तु प्रशस्ता न तथेतराः}


\twolineshloka
{यथा चैवापरः पक्षः पूर्वपक्षाद्विशिष्यते}
{तथा श्राद्धस्य पूर्वाह्णादपराह्णो विशिष्यते}


\chapter{अध्यायः १३५}
\threelineshloka
{किंस्विद्दत्तं पितृभ्यो वै भवत्यक्षयमीश्वरः}
{किंस्विद्वहुफलं प्रोक्तं किमानन्त्याय कल्पते ॥भीष्म उवाच}
{}


\twolineshloka
{हवींषि श्राद्धकल्पे तु यानि श्राद्धविदो विदुः}
{तानि मे शृणु काम्यानि फलं चैषां युधिष्ठिर}


\twolineshloka
{तिलैर्व्रीहियवैर्माषैरद्भिर्मूलफलैस्तथा}
{दत्तेन मासं प्रीयन्ते श्राद्धेन पितरो नृप}


\twolineshloka
{वर्धमानतिलं श्राद्धमक्षयं मनुरब्रवीत्}
{सर्वेष्वेव तु भोज्येषु तिलाः प्राधान्यतः स्मृताः}


\twolineshloka
{द्वौ मासौ तु भवेत्तुप्तिर्मत्स्यैः तितृगणस्य ह}
{त्रीन्मासानाविकेनाहुश्चतुर्मासं शशेन ह}


\twolineshloka
{आजेन मासान्प्रीयन्ते पञ्चैव पितरो नृप}
{वाराहेण तु षण्मासान्सप्त वै शाकुलेन तु}


\twolineshloka
{मासानष्टौ पार्षतेन रौरवेण नव प्रभो}
{गवयस्य तु मांसेन तृप्तिः स्याद्दशमासिकी}


\twolineshloka
{मांसेनेकादश प्रीतिः पितॄणां माहिषेण तु}
{गव्येन दत्ते श्राद्धे तु संवत्सरमिहोच्यते}


\twolineshloka
{यथा गव्यं तथा युक्तं पायसं सर्पिषा सह}
{वाध्रीणसस्य मांसेन तृप्तिर्द्वादशवार्षिकी}


\twolineshloka
{आन्त्याय भवेद्दतं खङ्गमांसं पितृक्षते}
{कालशाकं च लौहं चाप्यानन्त्यं छाग उच्यते}


\twolineshloka
{गाथाश्चाप्यत्र गायन्ति पितृगीता युधिष्ठिर}
{सनत्कुमारो भगवान्पुरा मध्यभ्यभाषत}


\twolineshloka
{अपि नः स्वकुले जायाद्यो नो दद्यात्त्रयोदशीम्}
{मघासु सर्पिःसंयुक्तं पायसं दक्षिणायने}


\twolineshloka
{आजेन वाऽपि लौहेन मघास्वेव यतव्रतः}
{हस्तिच्छायासु विधिवत्कर्णव्यजनवीजितम्}


\twolineshloka
{एष्टव्या बहवः पुत्रा यद्येकोपि गयां व्रजेत्}
{यत्रासौ प्रथितो लोकेष्वक्ष्यकरणो वटः}


\twolineshloka
{आपो मूलं फलं मांसमन्नं वाऽपि पितृक्षये}
{यत्किञ्चिन्मधुसंमिश्रं तदानन्त्याय कल्पते}


\chapter{अध्यायः १३६}
\twolineshloka
{यमस्तु यानि श्राद्धानि प्रोवाच शशबिन्दवे}
{तानि मे शृणु काम्यानि नक्षत्रेषु पृथक्पृथक्}


\twolineshloka
{श्राद्धं यः कृत्तिकायोगे कुर्वीत सततं नरः}
{अग्नीनाधाय सापत्यो यजेत विगतज्वरः}


\twolineshloka
{अपत्यकामो रोहिण्यां तेजस्कामो मृगोत्तमे}
{क्रूरकर्मा ददच्छ्राद्धमार्द्रायां मानवो भवेत्}


\twolineshloka
{कृषिभागी भवेन्मर्त्यः कुर्वञ्श्राद्धं पुनर्वसौ}
{पुष्टिकामोऽथ पुष्येण श्राद्धमीहेत मानवः}


\twolineshloka
{आश्लेषायां ददच्छ्राद्धं धीरान्पुत्रान्प्रजायते}
{ज्ञातीनां तु भवेच्छ्रेष्ठो मघासु श्राद्धमावपन्}


\twolineshloka
{फल्गुनीषु ददच्छ्राद्धं सुभगः श्राद्धदो भवेत्}
{अपत्यभागुत्तरासु हस्तेन फलभाग्भवेत्}


\twolineshloka
{चित्रायां तु ददच्छ्राद्धं लभेद्रूपवतः सुतान्}
{स्वातियोगे पितॄनर्च्य वाणिज्यमुपजीवति}


\twolineshloka
{बहुपुत्रो विशाखासु पुत्रमीनहन्भवेन्नरः}
{अनुराधासु कुर्वाणो राज्यचक्रं प्रवर्तयेत्}


\twolineshloka
{आधिपत्यं व्रजेन्मर्त्यो ज्येष्ठायामपवर्जयन्}
{नरः कुरुकुलश्रेष्ठ ऋद्धो दमपुरःसरः}


\twolineshloka
{मूले त्वारोग्यमृच्छेत यशोऽषाढासु चोत्तमम्}
{उत्तरासु त्वषाढासु वीतशोकश्चरेन्महीम्}


\twolineshloka
{श्राद्धं त्वभिजितौ कुर्वन्विद्यां श्रेष्ठामवाप्नुयात्}
{श्रवणेषु ददच्छ्राद्धं प्रेत्य गच्छेत्स तद्गतिम्}


\twolineshloka
{राज्यभागी धनिष्ठायां भवेत नियतं नरः}
{नक्षत्रे वारुणे कुर्वन्भिषक्सिद्धिमवाप्नुयात्}


\twolineshloka
{पूर्वप्रोष्ठपदाः कुर्वन्बहून्विन्दत्यजाविकान्}
{उत्तरासु प्रकुर्वाणो विन्दो गाः सहस्रशः}


\twolineshloka
{बहुकुप्यकृतं वित्तं विन्दते रेवतीं श्रितः}
{अश्विनीष्वश्वान्विन्देत भरणीष्वायुरुत्तमम्}


\twolineshloka
{इमं श्राद्धविधिं श्रुत्वा शशबिन्दुस्तथाऽकरोत्}
{अक्लेसेनाजयच्चापि महीं सोऽनुशशास ह}


\chapter{अध्यायः १३७}
\threelineshloka
{कीदृशेभ्यः प्रदातव्यं भवेच्छ्राद्धं पितामह}
{द्विजेभ्यः कुरुशार्दूल तन्मे व्याख्यातुमर्हसि ॥भीष्म उवाच}
{}


\twolineshloka
{ब्राह्मणान्न परीक्षेत क्षत्रियो दानधर्मवित्}
{दैवे कर्मणि पित्र्ये तु न्यायमाहुः परीक्षणम्}


\twolineshloka
{देवताः पावयन्तीह दैवेनैवेह तेजसा}
{उपेत्य तस्माद्देवेभ्यः सर्वेभ्यो दापयेन्नरः}


\twolineshloka
{श्राद्धे त्वथ महाराज परीक्षेद्ब्राह्मणान्बुधः}
{कुलशीलवयोरूपैर्विद्ययाऽभिजनेन च}


\twolineshloka
{तेषामन्ये पङ्क्तिदूष्यास्तथाऽन्ये पङ्क्तिपावनाः}
{अपाङ्क्तेयास्तु ये राजन्कीर्तयिष्यामि ताञ्शृणु}


\twolineshloka
{कितवो भ्रूणहा यक्ष्मीं पशुपालो निराकृतिः}
{ग्रामप्रेष्यो वार्धुषिको गायनः सर्वविक्रयी}


\twolineshloka
{अगारदाही गरदः कुण्डाशी सोमविक्रयी}
{सामुद्रिको राजभृत्यस्तैलिकः कूटकारकः}


\twolineshloka
{पित्रा विभजमानश्च यस्य चोपपतिर्गृहे}
{अभिशस्तस्तथा स्तेनः शिल्पं यश्चोपजीवती}


\twolineshloka
{पर्वकारश्च सूची च मित्रध्रुक् पारदारिकः}
{अव्रतानामुपाध्यायः काण्डपृष्ठस्तथैव च}


\threelineshloka
{श्वभिश्च यः परिक्रामेद्यः शुना दष्ट एव च}
{परिवित्तिश्च यश्च स्याद्दुश्चर्मा गुरुतल्पगः}
{}


\twolineshloka
{कुशीलवो देवलको नक्षत्रैर्यश्च जीवती ॥ईदृशैर्ब्राह्मणैर्भुक्तमपाङ्क्तेयैर्युधिष्ठिर}
{}


\twolineshloka
{रक्षांसि गच्छते हव्यमित्याहुर्ब्रह्मवादिनः ॥श्राद्धं भुक्त्वा त्वधीयीत वृषलीतल्पगश्च यः}
{}


\twolineshloka
{पुरीषे तस्य ते मासं पितरस्तस्य शेरते}
{सोमविक्रयिणे विष्ठा भिषजे पूयशोणितम्}


\threelineshloka
{नष्टं देवलके दत्तमप्रतिष्ठं च वार्धुषे}
{यत्तु वाणिजके दत्तं नेह नामुत्र तद्भवेत्}
{भस्मनीव हुतं हव्यं तथा पौनर्भवे द्विजे}


\twolineshloka
{ये तु धर्मव्यपेतेषु चारित्रापगतेषु च}
{हव्यं कव्यं प्रयच्छन्ति येषां तत्प्रेत्य नश्यति}


\twolineshloka
{ज्ञानपूर्वं तु ये तेभ्यः प्रयच्छन्त्यल्पबुद्धयः}
{पुरीषं भुञ्जते तस्य पितरः प्रेत्य निश्चयः}


\twolineshloka
{एतानिमान्विजानीयादपाङ्क्तेयान्द्विजाधमान्}
{शूद्राणामुपदेशं च ये कुर्वन्त्यल्पचेतसः}


\twolineshloka
{षष्टिं काणः शतं षण्डः श्वित्री यावत्प्रपश्यति}
{पङ्क्त्यां समुपविष्टायां तावद्दूषयते नृप}


\twolineshloka
{यद्वेष्टितशिरा भुङ्क्ते यद्भुङ्क्ते दक्षिणामुखः}
{सोपानत्कश्च यद्भुङ्क्ते सर्वं विद्यात्तदासुरम्}


\twolineshloka
{असूयता च यदत्तं यच्च श्रद्धाविवर्जितम्}
{सर्वं तदसुरेन्द्राय ब्रह्मा भागमकल्पयत्}


\twolineshloka
{श्वानश्च पङ्क्तिदूषाश्च नावेक्षेरन्कथञ्चन}
{तस्मात्परिसृते दद्यात्तिलांश्चान्ववकीरयेत्}


\twolineshloka
{तिलैर्विरहितं श्राद्धं कृतं क्रोधवशेन च}
{यातुधानाः पिशाचाश्च विप्रलुम्पन्ति तद्धविः}


\twolineshloka
{अपाङ्क्तो यावतः पाङ्क्तान्भुञ्जानाननुपश्यति}
{तावत्फलाद्धंशयति दातारं तस्य बालिशम्}


\twolineshloka
{इमे तु भरतश्रेष्ठ विज्ञेयाः पङ्क्तिपावनाः}
{हेतुतस्तान्प्रवक्ष्यामि परीक्षस्वेह तान्द्विजान्}


\twolineshloka
{विद्यावेदव्रतस्नाता ब्राह्मणाः सर्व एव हि}
{सदाचारपराश्चैव विज्ञेयाः सर्वपावनाः}


\twolineshloka
{पाङ्क्तेयांस्तु प्रवक्ष्यामि ज्ञेयास्ते पङ्क्तिपावनाः}
{त्रिणाचिकेतः पञ्चाग्निस्त्रिसुपर्णः षडङ्गवित्}


\twolineshloka
{ब्रह्मदेयानुसन्तानश्छन्दोगो ज्येष्ठसामगः}
{मातापित्रोर्यश्च वश्यः श्रोत्रियो दशपूरुषः}


\twolineshloka
{ऋतुकालाभिगामी च धर्मपत्नीषु यः सदा}
{वेदविद्याव्रतस्नातो विप्रः पङ्क्तिं पुनात्युत}


\twolineshloka
{अथर्वशिरसोऽध्येता ब्रह्मचारी यतव्रतः}
{सत्यवादी धर्मशीलः स्वकर्मनिरतश्च सः}


\twolineshloka
{ये च पुण्येषु तीर्थेषु अभिषेककृतश्रमाः}
{मखेषु च समन्त्रेषु भन्त्यवभृथप्लुताः}


\twolineshloka
{अक्रोधना ह्यचपलाः क्षान्ताः दान्ता जितेन्द्रियाः}
{सर्वभूतहिता ये च श्राद्धेष्वेतान्निमन्त्रयेत्}


\twolineshloka
{एतेषु दत्तमक्षय्यमेते वै पङ्क्तिपावनाः}
{इमे परे महाभागा विज्ञेयाः पङ्क्तिपावनाः}


\twolineshloka
{यतयो मोक्षधर्मज्ञा योगिनश्चरितव्रताः}
{`पञ्चरात्रविदो मुख्यास्तथा भागवताः परे}


\twolineshloka
{वैखानसाः कुलश्रेष्ठा वैदिकाचारचारिणः}
{'ये चेतिहासं प्रयताः श्रावयन्ति द्विजोत्तमान्}


\twolineshloka
{ये च भाष्यविदः केचिद्ये च व्याकरणे रताः}
{अधीयते पुराणं ये धर्मशास्त्राण्यथापि च}


\twolineshloka
{अधीत्य च यथान्यायं विधिवत्तस्य कारिणः}
{उपपन्नो गुरुकुले सत्यवादी सहस्रशः}


\twolineshloka
{अग्र्याः सर्वेषु वेदेषु सर्वप्रवचनेषु च}
{यावदेते प्रपश्यन्ति पङ्क्त्यास्तावत्पुनन्त्युत}


\threelineshloka
{ततो हि पावनात्पङ्क्त्याः पङ्क्तिपावन उच्यते}
{क्रोशादर्धतृतीयाच्च पावयेदेक एव हि}
{ब्रह्मदेयानुसन्तान इति ब्रह्मविदो विदुः}


\twolineshloka
{अनुत्विगनुपाध्यायः स चेदग्रासनं व्रजेत्}
{ऋत्विग्भिरभ्यनुज्ञातः पङ्क्त्या हरति दुष्कृतम्}


\twolineshloka
{अथ चेद्वेदवित्सर्वैः पङ्क्तिदोषैर्विवर्जितः}
{न च स्यात्पतितो राजन्पङ्क्तिपावन एव सः}


\twolineshloka
{तस्मात्सर्वप्रयत्नेन परीक्ष्यामन्त्रयेद्द्विजान्}
{स्वकर्मनिरतान्दान्तान्कुले जातान्बहुश्रतान्}


\twolineshloka
{यस्य मित्रप्रधानानि श्राद्धानि च हवींषि च}
{न प्रीणन्ति पितॄन्देवान्स्वर्गं च न स गच्छति}


\twolineshloka
{यश्च श्राद्धे कुरुते सङ्गतानिन देवयानेन पथा स याति}
{स वै मुक्तः पिप्पलं बन्धनाद्वास्वर्गाल्लोकच्च्यवते श्राद्धमित्रः}


\twolineshloka
{तस्मान्मित्रं श्राद्धकृन्नाद्रियेतदद्यान्मित्रेभ्यः सङ्ग्रहार्थं धनानि}
{यन्मन्यते नैव शत्रुं न मित्रंतं मध्यस्थं भोजयेद्धव्यकव्ये}


\twolineshloka
{यथोषरे बीजमुप्तं न रोहे-न्नचावप्ता प्राप्नुयाद्बीजभागम्}
{एवं श्राद्धं भुक्तमनर्हमाणै-र्न चेह नामुत्र फलं ददाति}


\twolineshloka
{ब्राह्मणो ह्यनधीयानस्तृणाग्निरिव शाम्यति}
{तस्मै श्राद्धं न दातव्यं न हि भस्मनि हूयते}


\twolineshloka
{सम्भोजनी नाम पिशाचदक्षिणासा नैव देवान्न पितॄनुपैति}
{इहैव सा भ्राम्यति हीनपुण्याशालान्तरे गौरिव नष्टवत्सा}


\twolineshloka
{यथाऽग्नौ शान्ते घृतमाजुहोतितन्नैव देवान्न पितॄनुपैति}
{तथा दत्तं नर्तके गायके चयां चानृते दक्षिणामावृणोति}


\twolineshloka
{उभौ हिनस्ति न भुनक्ति चैषाया चानृते दक्षिणा दीयते वै}
{आघातिनी गर्हितैषा पतन्तीतेषां प्रेतान्पातयेद्देवयानात्}


\twolineshloka
{ऋषीणां समये नित्यं ये चरन्ति युधिष्ठिर}
{निश्चिताः सर्वधर्मज्ञास्तान्देवा ब्राह्मणान्विदुः}


\twolineshloka
{स्वाध्यायनिष्ठा ऋषयो ज्ञाननिष्ठास्तथैव च}
{तपोनिष्ठाश्च बोद्धव्याः कर्मनिष्ठाश्च भारत}


\twolineshloka
{कव्यानि ज्ञाननिष्ठेभ्यः प्रतिष्ठाप्यानि भारत}
{तत्र ये ब्राह्मणान्केचिन्न सीदन्ति हि ते नराः}


\twolineshloka
{ये तु निन्दन्ति जल्पेषु न ताञश्राद्धेषु भोजयत्}
{ब्राह्मणा निन्दिता राजन्हन्युस्त्रैपुरुषं कुलम्}


\twolineshloka
{वैखानसानां वचनमृषीणां श्रूयते नृप}
{दूरादेव परीक्षेत ब्राह्मणान्वेदपारगान्}


% Check verse!
प्रियो वा यदि वा द्वेष्यस्तेषां तु श्राद्धमावपेत्
\twolineshloka
{यः सहस्रं सहस्राणां भोजयदेनृचो नरः}
{एकस्तान्मन्त्रवित्प्रीतः सर्वानर्हति भारत}


\chapter{अध्यायः १३८}
\twolineshloka
{केन सङ्कल्पितं श्राद्धं कस्मिन्काले किमात्मकम्}
{भृग्वङ्गिरसिके काले मुनिना कतरेण वा}


\fourlineindentedshloka
{`येन सङ्कल्पितं चैव तन्मे ब्रूहि पितामह}
{'कानि श्राद्धेषु वर्ज्यानि कानि मूलफलानि च}
{धान्यजात्यश्च का वर्ज्यास्तन्मे ब्रूहि पितामह ॥भीष्म उवाच}
{}


\twolineshloka
{यथा श्राद्धं सम्प्रवृत्तं यस्मिन्काले सदात्मकम्}
{येन सङ्कल्पितं चैव तन्मे शृणु जनाधिप}


\twolineshloka
{स्वायम्भुवोऽत्रिः कौरव्य परमर्षिः प्रतापवान्}
{तस्य वंशे महाराज दत्तात्रेय इति स्मृतः}


\twolineshloka
{दत्तात्रेयस्य पुत्रोऽभून्निमिर्नाम तपोधनः}
{निमेश्चाप्यभवत्पुत्रः श्रीमान्नाम श्रिया वृतः}


\twolineshloka
{पूर्णो वर्षसहस्रान्ते स कृत्वा दुष्करं तपः}
{कालधर्मपरीतात्मा निधनं समुपागतः}


\twolineshloka
{निमिस्तु कृत्वा शौचानि विधिदृष्टेन कर्मणा}
{सन्तापमगमत्तीप्रं पुत्रशोकपरायणः}


\twolineshloka
{अथ कृत्वोपकार्याणि चतुर्दश्यां महामतिः}
{तमेव गणयञ्शोकं विरात्रे प्रत्यबुध्यत}


\twolineshloka
{तस्यासीत्प्रतिबुद्धस्य शोकेन व्यथितात्मनः}
{मनः संहृत्य विषये बुद्धिर्विस्तारगामिनी}


\twolineshloka
{ततः सञ्चिन्तयामास श्राद्धकल्पं समाहितः}
{यानि तस्यैव भोज्यानि मूलानि च फलानि च}


\twolineshloka
{उक्तानि यानि चान्यानि यानि चेष्टानि तस्य ह}
{तानि सर्वाणि मनसा विनिश्चित्य तपोधनः}


\twolineshloka
{अमावास्यां महाप्राज्ञो विप्रानानाय्य पूजितान्}
{दक्षिणावर्तिकाः सर्वा बृसीः स्वयमथाकरोत्}


\twolineshloka
{सप्त विप्रांस्ततो भोज्ये युगपत्समुपानयत्}
{ऋते च लवणं भोज्यं श्यामाकान्नं ददौ प्रभुः}


\twolineshloka
{दक्षिणाग्रास्ततो दर्भा विष्टरेषु निवेशिताः}
{पादयोश्चैव विप्राणां ये त्वन्नमुपभुञ्जते}


\twolineshloka
{कृत्वा च दक्षिणाग्नान्वै दर्भान्स प्रयतः शुचिः}
{प्रददौ श्रीमते पिण्डान्नामगोत्रमुदाहरन्}


\twolineshloka
{तत्कृत्वा स मुनिश्रेष्ठो धर्मसङ्करमात्मनः}
{पश्चात्तापेन महता तप्यमानोऽभ्यचिन्तयत्}


\twolineshloka
{अकृतं मुनिभिः पूर्वं किं मयेदमनुष्ठितम्}
{कथं नु शापेन न मां दहेयुर्ब्राह्मणा इति}


\threelineshloka
{ततः सञ्चिन्तयामास वंशकर्तारमात्मनः}
{`बुद्ध्वाऽत्रिं मनसा दध्यौ भगवन्तं समाहितः}
{'ध्यातमात्रस्तथा चात्रिराजगाम तपोधनः}


\twolineshloka
{अथात्रिस्तं तथा दृष्ट्वा पुत्रशोकेन कर्शितम्}
{भृशमाश्वासयामास वाग्भिरिष्टाभिरव्ययः}


\twolineshloka
{निमे सङ्कल्पितस्तेऽयं पितृयज्ञस्तपोधन}
{मतो मे पूर्वदृष्टोऽत्र धर्मोऽयं ब्रह्मणा स्वयम्}


\twolineshloka
{सोयं स्वयभुविहितो धर्मः सङ्कल्पितस्त्वया}
{ऋते स्वयम्भुवः कोऽन्यः श्राद्धे संविधिमाहरेत्}


\twolineshloka
{अथाख्यास्यामि ते पुत्र श्राद्दे यं विधिमुत्तमम्}
{स्वयम्भुविहितं पुत्र तत्कुरुष्व निबोध मे}


\twolineshloka
{कृत्वाऽग्निशरणं पूर्वं मन्त्रपूर्वं तपोधन}
{ततोऽग्नयेऽथ सोमाय वरुणाय च नित्यशः}


\twolineshloka
{विश्वेदवाश्च यि नित्यं पितृभिः सह गोचराः}
{तेभ्यः सङ्कल्पिता भागाः स्वयमेव स्वयम्भुवा}


\twolineshloka
{स्तोतव्या चेह पृथिवी लोकस्यैव तु धारिणी}
{वैष्णवी काश्यपि चेति तथैवेहाक्षयेति च}


\twolineshloka
{उदकानयने चैव स्तोतव्यो वरुणो विभुः}
{ततोऽग्निश्चैव सोमश्च आराध्याविह तेऽनघ}


\twolineshloka
{देवास्तु पितरो नाम निर्मिता ये स्वयम्भुवा}
{ऊष्मपा ये महाभागास्तेषां भागः प्रकल्पितः}


\twolineshloka
{ते श्राद्धेनार्च्यमाना वै विमुच्यन्ते ह किल्बिषात्}
{सप्तकाः पितृवंशास्तु पूर्वदृष्टाः स्वयंभुवा}


\twolineshloka
{विश्वे चाग्निमुखा देवाः संख्याताः पूर्वमेव ते}
{तेषां नामानि वक्ष्यामि भागार्हाणां महात्मनाम्}


\twolineshloka
{सहः कृतिर्विपाप्मा च पुण्यकृत्पावनस्तथा}
{ग्राम्यः क्षेम्यः समूहश्च दिव्यसानुस्तथैव च}


\twolineshloka
{विवस्वान्वीर्यवाञ्श्रीमान्कीर्तिमान्कृत एव च}
{जितात्मा मुनिवीर्यश्च दीप्तरोमा भयङ्करः}


\twolineshloka
{अनुकर्मा प्रतीतश्च प्रदाताऽप्यंशुमांस्तथा}
{शैलाभः परमक्रोधी धीरोष्णी भूपतिस्तथा}


\twolineshloka
{स्रजो वज्रीवरी चैव विश्वेदेवाः सनातनाः}
{विद्युद्वर्चाः सोमवर्चाः सूर्यश्रीश्चेति नामतः}


\twolineshloka
{सोमपः सूर्यसावित्रो दत्तात्मा पुण्डरीयकः}
{उष्णीनाभो नभोदश्च विश्वायुर्दीप्तिरेव च}


\twolineshloka
{चमूहरः सुरेशश्च व्योमारिः शङ्करो भवः}
{ईशः कर्ता कृतिर्दक्षो भुवनो दिव्यकर्मकृत्}


\twolineshloka
{गणितः पञ्चवीर्यश्च आदित्यो रश्मिवांस्तथा}
{सप्तकृत्सोमवर्चाश्च विश्वकृत्कविरेव च}


\twolineshloka
{अनुगोप्ता सुगोप्ता च नप्ता चेश्वर एव च}
{कीर्तितास्ते महाभागाः कालस्य गतिगोचराः}


\twolineshloka
{अश्राद्धेयानि धान्यानि कोद्रवाः पुलकास्तथा}
{हिङ्गुद्रवेषु शाकेषु मूलानां लशुनं तथा}


\twolineshloka
{पलाण्डुः सौभाञ्जनकस्तथा गृञ्जनकादयः}
{कूश्माण्डजात्यलाबुं च कृष्णं लवणमेव च}


\threelineshloka
{ग्राम्यवाराहमांसं च यच्चैवाप्रोक्षितं भवेत्}
{कृष्णाजं जीरकं चैव शीतपाकी तथैव च}
{अङ्कुराद्यास्तथा वर्ज्या इह शृङ्गाटकानि च}


\threelineshloka
{वर्जयेल्लवणं सर्वं तथा जम्बूफलानि च}
{अवक्षुतावरुदितं तथा श्राद्धे च वर्जयेत्}
{}


\twolineshloka
{निवापे हव्यकव्ये वा गर्हितं च सुदर्शनम्}
{पितरश्च हि देवाश्च नाभिनन्दन्ति तद्धविः}


\twolineshloka
{चण्डालश्वपचौ वर्ज्यौ निवापे समुपस्थिते}
{काषायवासाः कुष्ठी वा पतितो ब्रह्महाऽपि वा}


\twolineshloka
{सङ्कीर्णयोनिर्विप्रश्च सम्बन्धी पतितश्च यः}
{वर्जनीय बुधैरेते निवापे समुपस्थिते}


\twolineshloka
{इत्येवमुक्त्वा भगवान्स्ववंश्यं तमृषिं पुरा}
{पितामहसभां दिव्या जगामात्रिस्तपोधनः}


\chapter{अध्यायः १३९}
\twolineshloka
{तथा विधौ प्रवृत्ते तु सर्व एव महर्षयः}
{पितृयज्ञानकुर्वन्त विधिदृष्टेन कर्मणा}


\twolineshloka
{ऋषयो धर्मनित्यास्तु कृत्वा निवपनान्युत}
{तर्पणं चापि कुर्वन्त तीर्थांभोभिर्यतव्रताः}


\twolineshloka
{निवापैर्दीयमानैश्च चातुर्वर्ण्येन भारत}
{तर्पिताः पितरो देवास्तत्रान्नं जरयन्ति वै}


% Check verse!
अजीर्णैस्त्वभिहन्यन्ते ते देवाः पितृभिः सह ॥सोममेवाभ्यपद्यन्त तदा ह्यन्नाभिपीडिताः
\twolineshloka
{तेऽब्रुवन्सोममासाद्य पितरोऽजीर्णपीडिताः}
{निवापान्नेन पीड्यामः श्रेयो नोत्र विधीयताम्}


\twolineshloka
{तान्सोमः प्रत्युवाचाथ श्रेयश्चेदीप्सितं सुराः}
{स्वयंभूसदनं यात स वः श्रेयोऽभिधास्यति}


\threelineshloka
{ते सोमवचनाद्देवाः पितृभिः सह भारत}
{मेरुशृङ्गे समासीनं पितामहमुपागमन् ॥पितर ऊचुः}
{}


\twolineshloka
{निवापान्नेन भगवन्भृशं पीड्यामहे वयम्}
{प्रसादं कुरु नो देव श्रेयो नः संविधीयताम्}


\fourlineindentedshloka
{इति तेषां वचः श्रुत्वा स्वयम्भूरिदमब्रवीत्}
{एष मे पार्श्वतो वह्निर्युष्मच्छ्रेयो विधास्यति}
{अग्निरुवाच}
{}


\twolineshloka
{सहितास्तस्य भोक्ष्यामो निवापे समुपस्थिते}
{जरयिष्यथ चाप्यन्नं मया सार्धं न संशयः}


\twolineshloka
{एतच्छ्रुत्वा तु पितरस्ततस्ते विज्वराऽभवन्}
{एतस्मात्कारणाच्चाग्नेः प्राग्भागो दीयते नृप}


\twolineshloka
{निवापे चाग्निपूर्वं वै निवृत्ते पुरुषर्षभ}
{न ब्रह्मराक्षसास्तं वै निवापं धर्षयन्त्युत}


\twolineshloka
{रक्षांसि नाभिवर्धन्ते स्थिते देवे हुताशने}
{पूर्वं पिण्डं पितुर्दद्यात्ततो दद्यात्पितामहे}


\threelineshloka
{प्रपितामहाय च तत एष श्राद्धविधिः स्मृतः}
{ब्रूयाच्छ्राद्धे च सावित्रीं पिण्डे पिण्डे समाहितः}
{सोमायेति च वक्तव्यं तथा पितृमतेति च}


\twolineshloka
{रजस्वला च या नारी व्यङ्गिता कर्णयोश्च या}
{निवापे नोपतिष्ठेत सङ्ग्राह्या नान्यवंशजा}


\twolineshloka
{जलं प्रतरमाणस्च कीर्तयेत पितामहान्}
{नदीमासाद्य कुर्वीत पितॄणां पितृतर्पणम्}


\twolineshloka
{पूर्वं स्ववंशजानां तु कृत्वाऽद्भिस्तर्पणं पुनः}
{सुहृत्सम्बन्धिवर्गाणां ततो दद्याज्जलाञ्जलिम्}


\twolineshloka
{कल्माषगोयुगेनाथ युक्तेन तरतो जलम्}
{पितरोऽभिलषन्ते वै नावं चाप्यधिरोहिताः}


\twolineshloka
{सदा नावि जलं तज्ज्ञाः प्रयच्छन्ति समाहिताः}
{मासार्धे कृष्णपक्षस्य कुर्यान्निर्वपणानि वै}


% Check verse!
पुष्टिरायुस्तथा वीर्यं श्रीश्चैव पितृभक्तितः
\threelineshloka
{पितामहः पुलस्त्यश्च वसिष्ठः पुलहस्तथा}
{अङ्गिराश्च क्रतुश्चैव कश्यपश्च महानृषिः}
{}


\twolineshloka
{एते कुरुकुलश्रेष्ठ महायोगेश्वराः स्मृताः ॥एते च पितरो राजन्नेष श्राद्धविधिः परः}
{}


\twolineshloka
{प्रेतास्तु पिण्डसम्बन्धान्मुच्यन्ते तेन कर्मणा ॥इत्येषा पुरुषश्रेष्ठ श्राद्धोत्पत्तिर्यथागमम्}
{}


% Check verse!
व्याख्याता पूर्वनिर्दिष्टा किं ते वक्ष्याम्यतः परम्
\chapter{अध्यायः १४०}
\threelineshloka
{द्विजातयो व्रतोपेता हविस्ते यदि भुञ्जते}
{अन्नं ब्राह्मणकामाय कथमेतत्पितामह ॥भीष्म उवाच}
{}


\threelineshloka
{अवेदोक्तव्रताश्चैव भुञ्जानाः कामकारणे}
{वेदोक्तेषु तु भुञ्जाना व्रतलुप्ता युधिष्ठिर ॥युधिष्ठिर उवाच}
{}


\threelineshloka
{यदिदं तप इत्याहुरुपवासं पृथग्जनाः}
{तपः स्यादेतदेवेह तपोऽन्यद्वाऽपि किं भवेत् ॥भीष्म उवाच}
{}


\twolineshloka
{मासार्धमासोपवासाद्यत्तपो मन्यते जनः}
{आत्मतन्त्रोपघाती यो न तपस्वी न धर्मवित्}


\twolineshloka
{त्यागस्य चापि सम्पत्तिः शिष्यते तप उत्तमम्}
{सदोपवासी च भवेद्ब्रह्मचारी तथैव च}


\twolineshloka
{मुनिश्च स्यात्सदा विप्रो देवांश्चैव सदा यजेत्}
{कुटुम्बिको धर्मकामः सदाऽस्वप्नश्च मानवः}


\twolineshloka
{अमृताशी सदा च स्यात्पवित्रं च सदा पठेत्}
{ऋतवादी सदा च स्यान्नियतश्च सदा भवेत्}


\threelineshloka
{विघसाशी कथं च स्यात्सदा चैवातिथिप्रियः}
{अमृताशी सदा च स्यात्पवित्री च सदा भवेत् ॥युधिष्ठिर उवाच}
{}


\threelineshloka
{कथं सदोपवासी स्याद्ब्रह्मचारी च पार्थिव}
{विघसाशी कथं च स्यात्कथं चैवातिथिप्रियः ॥भीष्म उवाच}
{}


\twolineshloka
{अन्तरा सायमाशं च प्रातराशं च यो नरः}
{सदोपवासी भवति यो न भुङ्क्तेऽन्तरा पुनः}


\twolineshloka
{भार्या गच्छन्ब्रह्मचारी ऋतौ भवति चैव ह}
{ऋतवादी सदा च स्याद्दानशीलस्तु मानवः}


\twolineshloka
{अभक्षयन्वृथा मांसममांसाशी भवत्युत}
{दानं ददत्पवित्री स्यादस्वप्नश्च दिवाऽस्वपन्}


\twolineshloka
{भृत्यातिथिषु यो भुङ्क्ते भुक्तवत्सु नरः सदा}
{अमृतं केवलं भुङ्क्ते इति विद्धि युधिष्ठिर}


\twolineshloka
{अभुक्तवत्सु नाश्नाति ब्राह्मणेषु तु यो नरः}
{अभोजनेन तेनास्य जितः स्वर्गो भवत्युत}


\twolineshloka
{देवेभ्यश्च पितृभ्यश्च संश्रितेभ्यस्तथैव च}
{अवशिष्टानि यो भुङ्क्ते तमाहुर्विघसाशिनम्}


\twolineshloka
{तेषां लोका ह्यपर्यन्ताः सदने ब्रह्मणः स्मृताः}
{उपस्थिता ह्यप्सरसो गन्धर्वैश्च जनाधिप}


\twolineshloka
{देवतातिथिभिः सार्धं पितृभिश्चोपभुञ्जते}
{रमन्ते पुत्रपौत्रैश्च तेषां गतिरनुत्तमा}


\chapter{अध्यायः १४१}
\threelineshloka
{ब्राह्मणेभ्यः प्रयच्छन्ति दानानि विविधानि च}
{दातृप्रतिग्रहीत्रोर्वै को विशेषः पितामह ॥भीष्म उवाच}
{}


\twolineshloka
{साधोर्यः प्रतिगृह्णीयात्तथैवासाधुतो द्विजः}
{गुणवत्यल्पदोषः स्यान्निर्गुणे तु निमज्जति}


\twolineshloka
{अत्राप्युदाहरन्तीममितिहासं पुरातनम्}
{वृषादर्भेश्च संवादं सप्तर्षीणां च भारत}


\twolineshloka
{कश्यपोऽत्रिर्वसिष्ठश्च भरद्वाजोऽथ गौतमः}
{विश्वामित्रो जमदग्निः साध्वी चैवाप्यरुन्धती}


\twolineshloka
{सर्वेषामथ तेषां तु गण्डाभूत्परिचारिका}
{शूद्रः पशुसखश्चैव भर्ता चास्या बभूव ह}


\twolineshloka
{ते च सर्वे तपस्यन्तः पुरा चेरुर्महीमिमाम्}
{समाधिना प्रतीक्षन्तो ब्रह्मलोकं सनातनम्}


\twolineshloka
{अथाभवदनावृष्टिर्महती कुरुनन्दन}
{कृच्छ्रप्राणोऽभवद्यत्र लोकोऽयं वै क्षुधान्वितः}


\twolineshloka
{कस्मिंश्चिच्च पुरा यज्ञे याज्येन शिबिसूनुना}
{दक्षिणार्थेऽथ ऋत्विग्भ्यो दत्तः पुत्रोऽनिलः किल}


\twolineshloka
{अस्मिन्कालेऽथ सोऽल्पायुर्दिष्टान्तमगमत्प्रभुः}
{ते तं क्षुधाभिसन्तप्ताः परिवार्योपतस्थिरे}


\twolineshloka
{याज्यात्मजमथो दृष्ट्वा गतासुमृषिसत्तमाः}
{अपचन्त तदा स्थाल्यां क्षुधार्ताः किल भारत}


\twolineshloka
{नाजीव्ये मर्त्यलोकेऽस्मिन्नात्मानं ते परीप्सवः}
{कृच्छ्रामापेदिरे वृत्तिमन्नहेतोस्तपस्विनः}


\threelineshloka
{अटमानोऽथ तान्मार्गे पचमानान्महीपतिः}
{राजा शैब्यो वृषादर्भिः क्लिश्यमानान्ददर्श ह ॥वृषादर्भिरुवाच}
{}


\twolineshloka
{प्रतिग्रहस्तारयति पुष्टिर्वै प्रतिगृह्णताम्}
{मयि यद्विद्यते वित्तं तद्वृणुध्वं तपोधनाः}


\twolineshloka
{`प्रतिग्रहो ब्राह्मणानां सृष्टा वृत्तिरनिन्दिता}
{तस्माद्ददामि वो वित्तं तद्वृणुध्वं तपोधनाः ॥'}


\twolineshloka
{प्रियो हि मे ब्राह्मणो याचमानोदद्यामहं वोऽश्वतरीसहस्रम्}
{धेनूनां दद्यामयुतं समग्र-मेकैकशः सवृषाः सम्प्रसूताः}


\threelineshloka
{अश्वांस्तथा शीघ्रगाञ्श्वेतरूपां-न्मनोजवान्प्रददाम्यर्बुदानि}
{कुलम्भराननडुहः शतं शता-न्धुर्याञ्श्वेतान्सर्वशोऽहं ददामि}
{प्रष्ठौहीनां पीवराणां च ताव-दग्र्या गृष्टीर्धेनवः सुव्रताश्च}


\threelineshloka
{वरान्ग्रमान्व्रीहिरसं यवांश्चरत्नं चान्यद्दुर्लभं किं ददानि}
{नास्मिन्नभक्ष्ये भावमेवं कुरुध्वंपुष्ट्यर्थं वः किं प्रयच्छाम्यहं वै ॥ऋषय ऊचुः}
{}


\twolineshloka
{राजन्प्रतिग्रहो राज्ञं मध्वास्वादो विषोपमः}
{तज्जानमानः कस्मात्त्वं कुरुषे नः प्रलोभनम्}


\twolineshloka
{`दशसूनासमश्चक्री दशचक्रिसमो ध्वजी}
{दशध्वजिसमा वेश्या दशवेश्यासमो नृपः}


\twolineshloka
{दशसूनासहस्राणि यो वाहयति सौनिकः}
{तेन तुल्यो भवेद्राजा घोरस्तस्य प्रतिग्रहः ॥'}


\twolineshloka
{क्षेत्रं हि दैवतमिव ब्राह्मणान्समुपाश्रितम्}
{अमलो ह्येष तपसा प्रीतः प्रीणाति देवताः}


\twolineshloka
{अह्नायेह तपो जातु ब्राह्मणस्योपजायते}
{तद्दाव इव निर्दह्यात्प्राप्तो राजप्रतिग्रहः}


\twolineshloka
{कुशलं सह दानेन राजन्नस्तु सदा तव}
{अर्थिभ्यो दीयतां सर्वमित्युक्त्वा तेन्यतो ययुः}


\twolineshloka
{अपक्वमेव तन्मांसमभूत्तेषां महात्मनाम्}
{अथ हित्वा ययु सर्वे वनमाहारकाङ्क्षिणः}


\twolineshloka
{ततः प्रचोदिता राज्ञा वनं गत्वाऽस्य मन्त्रिणः}
{प्रचीयोदुम्बराणि स्म दातुं तेषां प्रचक्रिरे}


% Check verse!
`दृष्ट्वा फलानि मुनयस्ते ग्रहीतुमुपाद्रवन् ॥'
\twolineshloka
{उदुम्बराण्यथान्यानि हेमगर्भाण्युपाहरन्}
{भृत्यास्तेषां ततस्तानि प्रग्राहितुमुपाद्रवन्}


\twolineshloka
{गुरूणीति विदित्वाऽथ न ग्राह्याण्यत्रिरब्रवीत्}
{नास्मेह मन्दविज्ञाना नास्म मानुषबुद्धयः}


\fourlineindentedshloka
{हैमानीमानि जानीमः प्रतिबुद्धाः स्म जागृम}
{इह ह्येतदुपावृत्तं प्रेत्य स्यात्कटुकोदयम्}
{अप्रतिग्राह्यमेवैतत्प्रेत्येह च सुखेप्सुना ॥वसिष्ठ उवाच}
{}


\threelineshloka
{शतेन निष्कगुणितं सहस्रेण च सम्मितम्}
{तथा बहु प्रतीच्छन्वै पापिष्ठां लभते गतिम् ॥कश्यप उवाच}
{}


\threelineshloka
{यत्पृथिव्यां व्रीहियवं हिरण्यं पशवः स्त्रियः}
{सर्वं तन्नालमेकस्य तस्माद्विद्वाञ्शमं व्रजेत् ॥भरद्वाज उवाच}
{}


\threelineshloka
{उत्पन्नस्य रुरोः शृङ्गं वर्दमानस्य वर्धते}
{प्रार्थना पुरुषस्येव तस्य मात्रा न विद्यते ॥गौतम उवाच}
{}


\threelineshloka
{न तल्लोके द्रव्यमस्ति यल्लोभं प्रतिपूरयेत्}
{समुद्रकल्पः पुरुषो न कदाचन पूर्यते ॥विश्वामित्र उवाच}
{}


\threelineshloka
{कामं कामयमानस्य यदा कामः समृध्यते}
{अथैनमपरः काम इष्टो विध्यति बाणवत् ॥`अत्रिरुवाच}
{}


\threelineshloka
{न जातु कामः कामानामुपभोगेन शाम्यति}
{हविषा कृष्णवर्त्मेव भूय एवाभिवर्धते ॥' जमदग्निरुवाच}
{}


\threelineshloka
{प्रतिग्रहे संयमो वै तपो धारयते ध्रुवम्}
{तद्धनं ब्राह्मणस्येह लुभ्यमानस्य विस्रवेत् ॥अरुन्धत्युवाच}
{}


\threelineshloka
{धर्मार्थं सञ्चयो यो वा द्रव्याणां पक्षसम्मतः}
{तपःसञ्चय एवेह विशिष्टो द्रव्यसञ्चयात् ॥चण्डोवाच}
{}


\threelineshloka
{उग्रादितो भयाद्यस्माद्बिभ्यतीमे ममेश्वराः}
{बलीयसो दुर्बलवद्बिभेम्यहमतः परम् ॥पशुसस्व उवाच}
{}


\threelineshloka
{यद्वै धर्मात्परं नास्ति तादृशं ब्राह्मणा विदुः}
{विनयात्साधु विद्वांसमुपासेयं यथातथम् ॥ऋषय ऊचुः}
{}


\threelineshloka
{कुशलं सह दानेन तस्मै यस्य प्रजा इमाः}
{फलान्युपधियुक्तानि य एवं नः प्रयच्छति ॥भीष्म उवाच}
{}


\twolineshloka
{इत्युक्त्वा हेमगर्भाणि हित्वा तानि फलानि ते}
{ऋषयो जग्मुरन्यत्र सर्व एव दृढव्रताः}


\threelineshloka
{अथ ते मन्त्रिणः सर्वे राजानमिदमब्रुवन्}
{उपधिं शङ्कमानास्ते हित्वा तानि फलानि वै}
{ततोऽन्यत्रैव गच्छन्ति विदितं तेऽस्तु पार्थिव}


\twolineshloka
{इत्युक्तः स तु भृत्यैस्तैर्वृषादर्भिश्चुकोप ह}
{तेषां वै प्रतिकर्तुं च सर्वेषामगमद्गृहम्}


\twolineshloka
{स गत्वाऽऽहवनीयेऽग्नौ तीव्रं नियममास्थितः}
{जुहाव संस्कृतैर्मन्त्रैरेकैकामाहुतिं नृपः}


\twolineshloka
{तस्मादग्नेः समुत्तस्थौ कृत्या लोकभयंकरी}
{तस्या नाम वृषादर्भिर्यातुधानीत्यथाकरोत्}


\threelineshloka
{सा कृत्या कालरात्रीव कृताञ्जलिरुपस्थिता}
{वृषादर्भि नरपतिं किं करोमीति चाब्रवीत् ॥वृषादर्भिरुवाच}
{}


\twolineshloka
{ऋषीणां गच्छ सप्तानामरुन्धत्यास्तथैव च}
{दासीभर्तुश्च दास्याश्च मनसा नाम धारय}


\twolineshloka
{ज्ञात्वा नामानि चैवैषां सर्वानेतान्विनाशय}
{विनष्टेषु तथा स्वैरं गच्छ यत्रेप्सितं तव}


\twolineshloka
{सा तथेति प्रतिश्रुत्य यातुधानी स्वरूपिणी}
{जगाम तद्वनं यत्र विचेरुस्ते महर्षयः}


\chapter{अध्यायः १४२}
\twolineshloka
{अथात्रिप्रमुखा राजन्वने तस्मिन्महर्षयः}
{व्यचरन्भक्षयन्तो वै मूलानि च फलानि च}


\twolineshloka
{अथापश्यन्सुपीनांसपाणिपादमुखोदरम्}
{परिव्रजन्तं स्थूलाङ्गं परिव्राजं शुनस्सखम्}


\threelineshloka
{अरुन्धती तु तं दृष्ट्वा सर्वाङ्गोपचितं शुभम्}
{भवितारो भवन्तो वै नैवमित्यब्रवीदृषीन् ॥वसिष्ठ उवाच}
{}


\threelineshloka
{नैतस्येह यथाऽस्माकमग्निहोत्रमनिर्हुतम्}
{सायं प्रातश्च होतव्यं तेन पीवाञ्शुनस्सखः ॥अत्रिरुवाच}
{}


\threelineshloka
{नैतस्येह यथाऽस्माकं क्षुधया वीर्यमाहतम्}
{कृच्छ्राधीतं प्रनष्टं च तेन पीवाञ्छुनस्सखः ॥विश्वामित्र उवाच}
{}


\threelineshloka
{नैतस्येह यथाऽस्माकं शश्वच्छास्त्रकृतो ज्वरः}
{अलसः क्षुत्परो मूर्खस्तेन पीवाञ्शुनस्सखः ॥जमदग्निरुवाच}
{}


\threelineshloka
{नैतस्येह यथाऽस्माकं भक्तमिन्धनमेव च}
{सञ्चित्यं वार्षिकं चित्ते तेन पीवाञ्शुनस्सखः ॥कश्यप उवाच}
{}


\threelineshloka
{नैतस्येह यथाऽस्माकं चत्वारश्च सहोदराः}
{देहिदेहीति भिक्षन्ति तेन पीवाञ्शुनस्सखः ॥भरद्वाज उवाच}
{}


\threelineshloka
{नैतस्येह यथाऽस्माकं ब्रह्मबन्धोरचेतसः}
{शोको भार्यापवादेन तेन पीवाञ्शुनस्सखः ॥गौतम उवाच}
{}


\threelineshloka
{नैतस्येह यथाऽस्माकं त्रिकौशेयं च राङ्कवम्}
{एकैकं वै त्रिवर्षीयं तेन पीवाञ्शुनस्सखः ॥भीष्म उवाच}
{}


\twolineshloka
{अथ दृष्ट्वा परिव्राट् स तान्महर्षीञ्शुनस्सखः}
{अभिवाद्य यथान्यायं पाणिस्पर्शमथाचरत्}


\twolineshloka
{परिचर्यां वने तां तु क्षुत्प्रतीकारकाङ्क्षिणः}
{अन्योन्येन निवेद्याथ प्रातिष्ठन्त सहैव ते}


\twolineshloka
{एकनिश्चयकार्याश्च व्यचरन्त वनानि ते}
{आददानाः समुद्धृत्य मूलानि च फलानि च}


\twolineshloka
{कदाचिद्विचरन्तस्ते वृक्षैरविरलैर्वृताम्}
{शुचिपूर्णप्रसन्नोदां ददृशुः पद्मिनीं शुभाम्}


\twolineshloka
{बालादित्यवपुःप्रख्यैः पुष्करैरुपशोभिताम्}
{वैडूर्यवर्णसदृशैः पद्मपत्रैरथावृताम्}


\twolineshloka
{नानाविधैश्च विहगैर्जलप्रवरसेविभिः}
{एकद्वारामनादेयां सूपतीर्थामकर्दमाम्}


\twolineshloka
{वृषादर्भिप्रयुक्ता तु कृत्या विकृतदर्शना}
{यातुधानीति विख्याता पद्मिनीं तामरक्षत}


\twolineshloka
{शुनस्सखसहायास्तु बिसार्थं ते महर्षयः}
{पद्मिनीमभिजग्मुस्ते सर्वे कृत्याभिरक्षिताम्}


\twolineshloka
{ततस्ते यातुधानीं तां दृष्ट्वा विकृतदर्शनाम्}
{स्थितां कमलिनीतीरे कृत्यामूचुर्महर्षयः}


\threelineshloka
{एका तिष्ठसि का च त्वं कस्यार्थे किं प्रयोजनम्}
{पद्मिनीतीरमाश्रित्य ब्रूहि त्वं किं चिकीर्षसि ॥यातुधान्युवाच}
{}


\threelineshloka
{याऽस्मि काऽस्म्यनुयोगो मे न कर्तव्यः कथञ्चन}
{आरक्षिणीं मा पद्मिन्या वित्त सर्वे तपोधनाः ॥ऋषय ऊचुः}
{}


\threelineshloka
{सर्व एव क्षुधार्ताः स्म न चान्यत्किंचिदस्ति नः}
{भवत्याः सम्मते सर्वे गृह्णीयाम बिसान्युत ॥यातुधान्युवाच}
{}


\threelineshloka
{समयेन बिसानीतो गृह्णीध्वं कामकारतः}
{एकैको नाम मे प्रोक्त्वा ततो गृह्णीत माचिरम् ॥भीष्म उवाच}
{}


\threelineshloka
{विज्ञाय यातुधानीं तां कृत्यमृषिवधैषिणीम्}
{अत्रिः क्षुधा परीतात्मा ततो वचनमब्रवीत् ॥अत्त्रिरुवाच}
{}


\threelineshloka
{अरात्त्रिरत्त्रिः सा रात्रिर्यां नाधीते त्रिरद्य वै}
{अरात्रिरत्रिरित्येव नाम मे विद्धि शोभने ॥यातुधान्युवाच}
{}


\threelineshloka
{यथोदाहृतमेतत्ते त्वया नाम महाद्युते}
{दुर्धार्यमेतन्मनसा गच्छाऽवतर पद्मिनीम् ॥वसिष्ठ उवाच}
{}


\threelineshloka
{वसिष्ठोऽस्मि वरिष्ठोऽस्मि वसे वासगृहेष्वपि}
{वरिष्ठत्वाच्च वासाच्च वसिष्ठ इति विद्दि माम् ॥यातुधान्युवाच}
{}


\threelineshloka
{नाम नैरुक्तमेतत्ते दुःखव्याभाषिताक्षरम्}
{नैतद्धारयितुं शक्यं गच्छाऽवतर पद्मिनीम् ॥कश्यप उवाच}
{}


\threelineshloka
{कुलंकुलं च कुवमः कुवमः कश्यपो द्विजः}
{काश्यः काशनिकाशत्वादेतन्मे नाम धारय ॥यातुधान्युवाच}
{}


\threelineshloka
{यथोदाहृतमेतत्ते मयि नाम महाद्युते}
{दुर्धार्यमेतन्मनसा गच्छाऽवतर पद्मिनीम् ॥भरद्वाज उवाच}
{}


\threelineshloka
{भरेऽसुतान्भरे पोष्यान्भरे देवान्भरे द्विजान्}
{भरे भार्यामहं व्याजाद्भरद्वाजोऽस्मि शोभने ॥यातुधान्युवाच}
{}


\threelineshloka
{नाम नैरुक्तमेतत्ते दुःखव्याभाषिताक्षरम्}
{नैतद्धारयितुं शक्यं गच्छाऽवतर पद्मिनीम् ॥गौतम उवाच}
{}


\threelineshloka
{गोदमो दमतोऽधूमोऽदमस्ते समदर्शनात्}
{विद्धि मां गोतमं कृत्ये यातुधानि निबोध मां यातुधान्युवाच}
{}


\threelineshloka
{यथोदाहृतमेतत्ते मयि नाम महामुने}
{नैतद्धारयितुं शक्यं गच्छाऽवतर पद्मिनीम् ॥विश्वामित्र उवाच}
{}


\threelineshloka
{विश्वेदेवाश्च मे मित्रं मित्रमस्मि गवां तथा}
{विश्वामित्र इति ख्यातं यातुधानि निबोध मां यातुधान्युवाच}
{}


\threelineshloka
{नाम नैरुक्तमेतत्ते दुःखव्याभाषिताक्षरम्}
{नैतद्धारयितुं शक्यं गच्छाऽवतर पद्मिनीम् ॥जमदग्निरुवाच}
{}


\threelineshloka
{जाजमद्यजजानेऽहं जिजाहीह जिजायिषि}
{जमदग्निरिति ख्यातं ततो मां विद्धि शोभने ॥यातुधान्युवाच}
{}


\twolineshloka
{यथोदाहृतमेतत्ते मयि नाम महामुने}
{नैतद्धारयितुं शक्यं गच्छाऽवतर पद्मिनीम्}


\fourlineindentedshloka
{अरुन्धत्युवाच}
{धरान्धरित्रीं वसुधां भर्तुस्तिष्ठाम्यनन्तरम्}
{मनोऽनुरुन्धती भर्तुरिति मां विद्ध्यरुन्धतीम् ॥यातुधान्युवाच}
{}


\threelineshloka
{नामनैरुक्तमेतत्ते दुःखव्याभाषिताक्षरम्}
{नैतद्धारयितुं शक्यं गच्छाऽवतर पद्मिनीम् ॥गण्डोवाच}
{}


\threelineshloka
{वक्त्रैकदेशे गण्डेति धातुमेतं प्रचक्षते}
{तेनोन्नतेन गण्डेति विद्धि माऽनलसम्भवे ॥यातुधान्युवाच}
{}


\threelineshloka
{नामनैरुक्तमेतत्ते दुःखव्याभाषिताक्षरम्}
{नैतद्धारयितुं शक्यं गच्छावतर पद्मिनीम् ॥पशुसख उवाच}
{}


\threelineshloka
{पशून्यञ्जामि दृष्ट्वाऽहं पशूनां च सदा सखा}
{गौणं पशुसखेत्येवं विद्धि मामग्निसम्भवे ॥यातुधान्युवाच}
{}


\threelineshloka
{नामनैरुक्तमेतत्ते दुःखव्याभाषिताक्षरम्}
{नैतद्धारयितुं शक्यं गच्छाऽवतर पद्मिनीम् ॥शुनःसख उवाच}
{}


\threelineshloka
{एभिरुक्तं यथा नाम नाहं वक्तुमिहोत्सहे}
{शुनःसखसखायं मां यातुधान्युपधारय ॥यातुधान्युवाच}
{}


\threelineshloka
{नाम न व्यक्तमुक्तं वै वाक्यं संदिग्धया गिरा}
{तस्मात्सकृदिदानीं त्वं ब्रूहि यन्नाम ते द्विज ॥शुनःसख उवाच}
{}


\threelineshloka
{सकृदुक्तं मया नाम न गृहीतं त्वया यदि}
{तस्मात्त्रिदण्डाभिहता गच्छ भस्मेति माचिरम् ॥भीष्म उवाच}
{}


\twolineshloka
{सा ब्रह्मदण्डकल्पेन तेन मूर्ध्नि हता तदा}
{कृत्या पपात मेदिन्यां भस्म साच जगाम ह}


\twolineshloka
{शुनःसखश्च हत्वा तां यातुधानीं महाबलाम्}
{भुवि त्रिदण्डं विष्टभ्य शाद्वले समुपाविशत्}


\twolineshloka
{ततस्ते मुनयः सर्वेः पुष्कराणि बिसानि च}
{यथाकाममुपादाय समुत्तस्थुर्मुदाऽन्विताः}


\twolineshloka
{श्रमेण महता युक्तास्ते बिसानि कलापशः}
{तीरे निक्षिप्य पद्मिन्यास्तर्पणं चक्रुरम्भसा}


\threelineshloka
{अथोत्थाय जलात्तस्मात्सर्वे ते समुपागमन्}
{नापश्यंश्चापि ते तानि बिसानि पुरुषर्षभाः ॥ऋषय ऊचुः}
{}


\threelineshloka
{केन क्षुधाभिभूतानामस्माकं पापकर्मणाम्}
{नृशंसेनापनीतानि बिसान्याहारकाङ्क्षिणाम् ॥भीष्म उवाच}
{}


\twolineshloka
{ते शङ्कमानास्त्वन्योन्यं पप्रच्छुर्द्विजसत्तमाः}
{त ऊचुः शपथं सर्वे कुर्म इत्यरिकर्शन}


\threelineshloka
{त उक्त्वा बाढमित्येव सर्व एव तदा समम्}
{क्षुधार्ताः सुपरिश्रान्ताः शपथायोपचक्रमुः ॥अत्रिरुवाच}
{}


\threelineshloka
{स गां स्पृशतु पादेन सूर्यं च प्रतिमेहतु}
{अध्यायेष्वधीयीत बिसस्तैन्यं करोति यः ॥वसिष्ठ उवाच}
{}


\twolineshloka
{अनध्याये पठेल्लोके शुनः स परिकर्षतु}
{परिव्राट् कामवृत्तिस्तु बिसस्तैन्यं करोति यः}


\threelineshloka
{शरणागतं हन्तु मित्रं स्वसुतां चोपजीवतु}
{अर्थान्काङ्क्षतु कीनाशाद्बिसस्तैन्यं करोति यः ॥कश्यप उवाच}
{}


\threelineshloka
{`विष्णुं ब्रह्मण्यदेवेशं देवदेवं जगद्गुरुम्}
{आधातारं विधातारं सन्धातारं जगद्गुरुम्}
{विहाय स भजत्वन्यं बिसस्तैन्यं करोति यः ॥'}


\twolineshloka
{सर्वत्र सर्वं लपतु न्यासलोपं करोतु च}
{कूटसाक्षित्वमभ्येतु बिसस्तैन्यं करोति यः}


\threelineshloka
{वृथा मांसाशनश्चास्तु वृथा दानं करोतु च}
{यातु स्त्रियं दिवा चैव बिसस्तैन्यं करोति यः ॥भरद्वाज उवाच}
{}


\twolineshloka
{नृशंसस्त्यक्तधर्माऽस्तु स्त्रीषु ज्ञातिषु गोषु च}
{ब्राह्मणं चापि जयतां बिसस्तैन्यं करोति यः}


\threelineshloka
{उपाध्यायमधः कृत्वा ऋचोऽध्येतु यजूंषि च}
{जुहोतु च स कक्षाग्नौ बिसस्तैन्यं करोति यः ॥जमदग्निरुवाच}
{}


\twolineshloka
{पुरीषमुत्सृजत्वप्सु हन्तु गां चैव द्रुह्यत्}
{अनृतौ मैथुनं यातु बिसस्तैन्यं करोति यः}


\threelineshloka
{द्वेष्यो भार्योपजीवी स्याद्दुरबन्धुश्च वैरवान्}
{अन्योन्यस्यातिथिश्चास्तु बिसस्तैन्यं करोति यः ॥गौतम उवाच}
{}


\twolineshloka
{अधीत्य वेदांस्त्यजतु त्रीनग्नीनपविध्यतु}
{विक्रीणातु तथा सोमं बिसस्तैन्यं करोति यः}


\threelineshloka
{उदपानोदके ग्रामे ब्राह्मणो वृषलीपतिः}
{तस्य सालोक्यतां यातु बिसस्तैन्यं करोति यः ॥विश्वामित्र उवाच}
{}


\twolineshloka
{जीवतो वै गुरून्भृत्यान्भरन्त्वस्य परे जनाः}
{दरिद्रो बहुपुत्रः स्याद्बिसस्तैन्यं करोति यः}


\twolineshloka
{अशुचिर्ब्रह्मकूटोऽस्तु मिथ्या चैवाप्यहङ्कृतः}
{कर्षको मत्सरी चास्तु बिसस्तैन्यं करोति यः}


\threelineshloka
{हर्षं करोतु भृतको राज्ञश्वास्तु पुरोहितः}
{अयाज्यस्य भवेदृत्विक् बिसस्तैन्यं करोति यः अरुन्धत्युवाच}
{}


\twolineshloka
{नित्यं परिभवेच्छ्वश्रूं भर्तुर्भवतु दुर्मनाः}
{एका स्वादु समश्नातु बिसस्तैन्यं करोति या}


\threelineshloka
{ज्ञातीनां गृहमध्यस्था सक्तूनत्तु दिनक्षये}
{अभोग्या वीरसूरस्तु बिसस्तैन्यं करोति या ॥गण्डोवाच}
{}


\twolineshloka
{अनृतं भाषतु सदा बन्धुभिश्च विरुध्यतु}
{ददातु कन्यां शुल्केन बिसस्तैन्यं करोति या}


\threelineshloka
{साधयित्वा स्वयं प्राशेद्दास्ये जीर्यतु चैव ह}
{विकर्मणा प्रमीयेत बिसस्तैन्यं करोति या ॥पशुसख उवाच}
{}


\threelineshloka
{दास एव प्रजायेतामप्रसूतिरकिञ्चन}
{दैवतेष्वनमस्कारो बिसस्तैन्यं करोति यः ॥शुनःसख उवाच}
{}


\threelineshloka
{अध्वर्यवे दुहितरं वा ददातुच्छन्दोगे वाऽऽचरितब्रह्मचर्ये}
{आथर्वणं वेदमधीत्य विप्रःस्नायीत वा यो हरते बिसानि ॥ऋषय ऊचुः}
{}


\threelineshloka
{इष्टमेतद्द्विजातीनां योऽयं ते शपथः कृतः}
{त्वया कृतं बिसस्तैन्यं सर्वेषां नः शुनःसख ॥शुनःसख उवाच}
{}


\twolineshloka
{न्यस्तमद्यं न पश्यद्भिर्यदुक्तं कृतकर्मभिः}
{सत्यमेतन्न मिथ्यैतद्बिसस्तैन्यं कृतं मया}


\twolineshloka
{मया ह्यन्तर्हितानीह बिसानीमानि पश्यत}
{परीक्षार्थं भगवतां कृतमेवं मयाऽनघाः}


\twolineshloka
{रक्षणार्थं च सर्वेषां भवतामहमागतः}
{यातुधानी ह्यतिक्रूरा कृत्यैषा वो वधैषिणी}


\twolineshloka
{वृषादर्भिप्रयुक्तैषा निहता मे तपोधनाः}
{दुष्टा हिंस्यादियं पापा युष्मान्प्रत्यग्निसम्भवा}


\fourlineindentedshloka
{तस्मादस्म्यागतो विप्रा वासवं मां निबोधत}
{अलोभादक्षया लोकाः प्राप्ता वः सार्वकामिकाः}
{उत्तिष्ठध्वमितः क्षिप्रं तानवाप्नुत वै द्विजाः ॥भीष्म उवाच}
{}


\twolineshloka
{ततो महर्षयः प्रीतास्तथेत्युक्त्वा पुरन्दरम्}
{सहैव त्रिदशेन्द्रेण सर्वे जग्मुस्त्रिविष्टपम्}


\threelineshloka
{एवमेते महात्मानो योगैर्बहुविधैरपि}
{क्षुधा परमया युक्ताश्छन्द्यमाना महात्मभिः}
{नैव लोभं तदा यक्रुस्ततः स्वर्गमवाप्नुवन्}


\twolineshloka
{तस्मात्सर्वास्ववस्थासु नरो लोभं विवर्जयेत्}
{एष धर्मः परो राजंस्तस्माल्लोभं विवर्जयेत्}


\twolineshloka
{इदं नरः सुचरितं समवायेषु कीर्तयन्}
{अर्थभागी च भवति न च दुर्गाण्यवाप्नुते}


\twolineshloka
{प्रीयन्ते पितरश्चास्य ऋषयो देवतास्तथा}
{यशोधर्मार्थभागी च भवति प्रेत्य मानवः}


\chapter{अध्यायः १४३}
\twolineshloka
{अत्रैवोदाहरन्तीममितिहासं पुरातनम्}
{यद्वृत्तं तीर्थयात्रायां शपथं प्रति तच्छृणु}


\twolineshloka
{पुष्करार्थं कृतं स्तैन्यं पुरा भरतसत्तम}
{राजर्षिभिर्महाराज तथैव च द्विजर्षिभिः}


\twolineshloka
{पुरा प्रभासे ऋषयः समग्राःसमेता वै मन्त्रममन्त्रयन्त}
{चराम सर्वां पृथिवीं पुण्यतीर्थांतन्नः कामं हन्त गच्छाम सर्वे}


\twolineshloka
{शुक्रोऽङ्गिराश्चैव कविश्च विद्वां-स्तथा ह्यगस्त्यो नारदपर्वतौ च}
{भृगुर्वसिष्ठः कश्यपो गौतमश्चविश्वामित्रो जमदग्निश्च राजन्}


\twolineshloka
{ऋषिस्तथा गालवोऽथाष्टकश्चभरद्वाजोऽरुन्धती वालखिल्याः}
{शिबिर्दिलीपो नहुषोऽम्बरीषोराजा ययातिर्धुन्धुमारोऽथ पूरुः}


\twolineshloka
{जग्मुः पुरस्कृत्य महानुभावंशतक्रतुं वृत्रहणं नरेन्द्राः}
{तीर्थानि सर्वाणि परिभ्रमन्तोमाघ्यां ययुः कौशिकीं पुण्यतीर्थाम्}


\twolineshloka
{सर्वेषु तीर्थेष्ववधूतपापाजग्मुस्ततो ब्रह्मसरः सुपुण्यम्}
{देवस्य तीर्थे जलमग्निकल्पाविगाह्य ते भुक्तबिसप्रसूनाः}


\twolineshloka
{केचिद्बिसान्यखनंस्तत्र राजन्न-न्ये मृणालान्यखनंस्तत्र विप्राः}
{अथापश्यन्पुष्करं ते ह्रियन्तंह्रदादगस्त्येन समुद्धृतं तत्}


\twolineshloka
{नानाह सर्वानृषिमुख्यानगस्त्यःकेनाहृतं पुष्करं मे सुजातम्}
{युष्माञ्शङ्के पुष्करं दीयतां मेन वै भवन्तो हर्तुमर्हन्ति पद्मम्}


\twolineshloka
{शृणोमि कालो हिंसते धर्मवीर्यंसेयं प्राप्ता वर्तते धर्मपीडा}
{पुराऽधर्मो वर्तते नेह याव-त्तावद्गच्छामः सुरलोकं चिराय}


\twolineshloka
{पुरा वेदान्ब्राह्मणा ग्राममध्येघुष्टस्वरा वृषलान्श्रावयन्ति}
{पुरा राजा व्यवहारानधर्मा-न्पश्यत्यहं परलोकं व्रजामि}


\twolineshloka
{पुरा राजा प्रत्यवरान्गरीयसोमंस्यत्यथैनमनुयास्यन्ति सर्वे}
{धर्मोत्तरं यावदिदं न वर्ततेतावद्व्रजामि परलोकं चिराय}


\twolineshloka
{पुरा प्रपश्यामि परेण मर्त्या-न्बलीयसा दुर्बलान्भुज्यमानान्}
{तस्माद्यास्यामि परलोकं चिरायन ह्युत्सहे द्रष्टुमीदृङ्नृलोके}


\twolineshloka
{तमाहुरार्ता ऋषयो महर्षिंन ते वयं पुष्करं चोरयामः}
{मिथ्याभिशंसा भवता न कार्याशपाम तीक्ष्णैः शपथैर्महर्षे}


\threelineshloka
{ते निश्चितास्तत्र महर्षयस्तुसम्पश्यन्तो धर्ममेतं नरेन्द्राः}
{ततोऽशपन्त शपथान्पर्ययेणसहैव ते पार्थिव पुत्रपौत्रैः ॥भृगुरुवाच}
{}


\threelineshloka
{प्रत्याक्रोशेदिहाक्रुष्टस्ताडितः प्रतिताडयेत्}
{खादेच्च पृष्ठमांसानि यस्ते हरति पुष्करम् ॥वसिष्ठ उवाच}
{}


\threelineshloka
{अस्वाध्यायपरो लोके श्वानं च परिकर्षतु}
{पुरे च भिक्षुर्भवतु यस्ते हरति पुष्करम् ॥कश्यप उवाच}
{}


\threelineshloka
{सर्वत्र सर्वं पणतु न्यासे लोभं करोतु च}
{कूटसाक्षित्वमभ्येतु यस्ते हरति पुष्करम् ॥गौतम उवाच}
{}


\threelineshloka
{जीवत्वहङ्कृतो बुद्ध्या विषमेणासमेन सः}
{कर्षको मत्सरी चास्तु यस्ते हरति पुष्करम् ॥अङ्गिरा उवाच}
{}


\threelineshloka
{अशुचिर्ब्रह्मकूटोस्तु श्वानं च परिकर्षतु}
{ब्रह्महाऽनिकृतिश्चास्तु यस्ते हरति पुष्करम् ॥धुन्धुमार उवाच}
{}


\threelineshloka
{अकृतज्ञस्तु मित्राणां शूद्रायां च प्रजायतु}
{एकः सम्पन्नमश्नातु यस्ते हरति पुष्करम् ॥पुरूरवा उवाच}
{}


\threelineshloka
{चिकित्सायां प्रचरतु भार्यया चैव पुष्यतु}
{श्वशुरात्तस्य वृत्तिः स्याद्यस्ते हरति पुष्करम् ॥दिलीप उवाच}
{}


\threelineshloka
{उदपानप्लवे ग्रामे ब्राह्मणो वृषलीपतिः}
{तस्य लोकान्स व्रजतु यस्ते हरति पुष्करम् ॥शुक्र उवाच}
{}


\threelineshloka
{वृथा मांसं समश्नातु दिवा गच्छतु मैथुनम्}
{प्रेष्यो भवतु राज्ञश्च यस्ते हरति पुष्करम् ॥जमदग्निरुवाच}
{}


\threelineshloka
{अनध्यायेष्वधीयीत मित्रं श्राद्धे च भोजयेत्}
{श्राद्धे शूद्रस्य चाश्नीयाद्यस्ते हरति पुष्करम् ॥शिबिरुवाच}
{}


\threelineshloka
{अनाहिताग्निर्मियतां यज्ञे विघ्नं करोतु च}
{तपस्विभिर्विरुध्येच्च यस्ते हरति पुष्करम् ॥ययातिरुवाच}
{}


\threelineshloka
{अनृतौ व्रतनियतायां भार्यायां स प्रजायतु}
{निराकरोतु वेदांश्च यस्ते हरति पुष्करम् ॥नहुष उवाच}
{}


\threelineshloka
{अतिथिर्गृहसंस्थोऽस्तु कामवृत्तस्तु दीक्षितः}
{विद्यां प्रयच्छतु भृतो यस्ते हरति पुष्करम् ॥अम्बरीष उवाच}
{}


\threelineshloka
{नृशंसस्त्यक्तधर्मोऽस्तु स्त्रीषु ज्ञातिषु गोषु च}
{निहन्तु ब्राह्मणं चापि यस्ते हरति पुष्करम् ॥नारद उवाच}
{}


\threelineshloka
{गृहज्ञानी बहिःशास्त्रं पठतां विस्वरं पदम्}
{गरीयसोऽवजानातु यस्ते हरति पुष्करम् ॥नाभाग उवाच}
{}


\threelineshloka
{अनृतं भाषतु सदा सद्भिश्चैव विरुध्यतु}
{शुल्केन ददतु कन्यां यस्ते हरपि पुष्करम् ॥कविरुवाच}
{}


\threelineshloka
{पदा च गां संस्पृशतु सूर्यं च प्रति मेहतु}
{शरणागतं संत्यजतु यस्ते हरति पुष्करम् ॥विश्वामिइत्र उवाच}
{}


\threelineshloka
{करोतु भृतकोऽवर्षां राज्ञश्चास्तु पुरोहितः}
{ऋत्विगस्तु ह्ययाज्यस्य यस्ते हरति पुष्करम् ॥पर्वत उवाच}
{}


\threelineshloka
{ग्रामे चाधिकृतः सोऽस्तु खरयानेन गच्छतु}
{शुनः कर्षतु वृत्त्यर्थे यस्ते हरति पुष्करम् ॥भरद्वाज उवाच}
{}


\threelineshloka
{सर्वपापसमादानं नृशंसे चानृते च यत्}
{तत्तस्यास्तु सदा पापं यस्ते हरति पुष्करम् ॥अष्टक उवाच}
{}


\threelineshloka
{स राजास्त्वकृतप्रज्ञः कामवृत्तश्च पापकृत्}
{अधर्मेणाभिशास्तूर्वीं यस्ते हरति पुष्करम् ॥गालव उवाच}
{}


\threelineshloka
{पापिष्ठेभ्यो ह्यनर्घार्हः स नरोऽस्तु स्वपापकृत्}
{दत्त्वा दानं कीर्तयतु यस्ते हरति पुष्करम् ॥अरुन्धत्युवाच}
{}


\threelineshloka
{श्वश्र्वाऽपवादं वदतु भर्तुर्भवतु दुर्मनाः}
{एका स्वादु समश्नातु या ते हरति पुष्करम् ॥वालखिल्या ऊचुः}
{}


\threelineshloka
{एकपादेन वृत्त्यर्थं ग्रामद्वारे स तिष्ठतु}
{धर्मज्ञस्त्यक्तधर्मास्तु यस्ते हरति पुष्करम् ॥पशुसख उवाच}
{}


\threelineshloka
{अग्निहोत्रमनादृत्य स सुखं स्वपतु द्विजः}
{परिव्राट् कामवृत्तोस्तु यस्ते हरति पुष्करम् ॥सुरभ्युवाच}
{}


\threelineshloka
{वालजेन निदानेन कांस्यं भवतु दोहनम्}
{दुह्येत परवत्सेन या ते हरति पुष्करम् ॥भीष्म उवाच}
{}


\twolineshloka
{ततस्तु तैः शपथैः शप्यमानै-र्नानाविधैर्बहुभिः कौरवेन्द्र}
{सहस्राक्षो देवराट् सम्प्रहृष्टःसमीक्ष्य तं कोपनं विप्रमुख्यम्}


\threelineshloka
{यथाब्रवीन्मघवा प्रत्ययं स्वंस्वयं समागत्य तमृषिं जातरोषम्}
{ब्रह्मर्षिदेवर्षिनृपर्षिमध्येयं तं निबोधेह ममाद्य राजन् ॥शक्र उवाच}
{}


\twolineshloka
{अध्वर्यवे दुहितरं ददातुछन्दोगे वाऽऽचरितब्रह्मचर्ये}
{अथर्वणं वेदमधीत्य विप्रःस्नायीत यः पुष्करमाददाति}


\threelineshloka
{सर्वान्वेदानधीयीत पुण्यशीलोऽस्तु धार्मिकः}
{ब्रह्मणः सदनं यातु यस्ते हरति पुष्करम् ॥अगस्त्य उवाच}
{}


\threelineshloka
{आशीर्वादस्त्वया प्रोक्तः शपथो बलसूदन}
{दीयतां पुष्करं मह्यमेष धर्मः सनातनः ॥इन्द्र उवाच}
{}


\twolineshloka
{न मया भगवँल्लोभाद्धृतं पुष्करमद्य वै}
{धर्मांस्तु श्रोतुकामेन हृतं न क्रोद्धुमर्हसि}


\twolineshloka
{धर्मश्रुतिसमुत्कर्षो धर्मसेतुरनामयः}
{आर्षो वै शाश्वतो नित्यमव्ययोऽयं मया श्रुतः}


\twolineshloka
{तदिदं गृह्यतां विद्वन्पुष्करं द्विजसत्तम}
{अतिक्रमं मे भगवन्क्षन्तुमर्हस्यनिन्दित}


\twolineshloka
{इत्युक्तः स महेन्द्रेण तपस्वी कोपनो भृशम्}
{जग्राह पुष्करं धीमान्प्रसन्नश्चाभवन्मुनिः}


\twolineshloka
{प्रययुस्ते ततो भूयस्तीर्थानि वनगोचराः}
{पुण्येषु तीर्थेषु तथा गात्राण्याप्लावयन्त ते}


\twolineshloka
{आख्यानं य इदं युक्तः पठेत्वर्वणिपर्वणि}
{न मूर्खं जनयेत्पुत्रं न भवेच्च निराकृतिः}


\twolineshloka
{न तमापत्स्पृशेत्काचिद्विज्वरो न जरावहः}
{विरजाः श्रेयसा युक्तः प्रेत्य स्वर्गमवाप्नुयात्}


\twolineshloka
{यश्च शास्त्रमधीयीत ऋषिभिः परिपालितम्}
{स गच्छेद्ब्रह्मणो लोकमव्ययं च नरोत्तम}


\chapter{अध्यायः १४४}
\twolineshloka
{यदिदं श्राद्धकृत्येषु दीयते भरतर्षभ}
{छत्रं चोपानहौ चैव केनैतत्सम्प्रवर्तितम्}


\twolineshloka
{कथं चैतत्समुत्पन्नं किमर्थं चैव दीयते}
{न केवलं श्राद्धकृत्ये पुण्यकेष्वपि दीयते}


\threelineshloka
{बहुष्वपि निमित्तेषु पुण्यमाश्रित्य दीयते}
{एतद्विस्तरतो ब्रह्मञ्श्रोतुमिच्छामि तत्त्वतः ॥भीष्म उवाच}
{}


\twolineshloka
{शृणु राजन्नवहितश्छत्रोपानहविस्तरम्}
{यथैतत्प्रथितं लोके यथा चैतत्प्रवर्तितम्}


\twolineshloka
{यथा चाक्षय्यतां प्राप्तं पुण्यतां च यथागतम्}
{सर्वमेतदशेषेण प्रवक्ष्यामि नराधिप}


\twolineshloka
{`इतिहासं पुरावृत्तमिदं शृणु नराधिप}
{'जमदग्नेश्च संवादं सूर्यस्य च महात्मनः}


\twolineshloka
{पुरा स भगवान्साक्षाद्धनुषा क्रीडति प्रभो}
{सन्धायसन्धाय शरांश्चिक्षेप किल भार्गवः}


\twolineshloka
{तान्क्षिप्तान्रेणुका सर्वांस्तस्येषून्दीप्ततेजसः}
{आनीय सा तदा तस्मै प्रादादसकृदच्युत}


\twolineshloka
{अथ तेन स शब्देनि ज्यायाश्चैव शरस्य च}
{प्रहृष्टः सम्प्रचिक्षेप सा च प्रत्याजहार तान्}


\twolineshloka
{ततो मध्याह्नमारूढे ज्येष्ठामूले दिवाकरे}
{स सायकान्द्विजो मुक्त्वा रेणुकामिदमब्रवीत्}


\twolineshloka
{गच्छानय विशालाक्षि शरानेतान्धनुश्च्युतान्}
{यावदेतान्पुनः सुभ्रु क्षिपामीति जनाधिप}


\twolineshloka
{सा गच्छन्त्यन्तरा छायां वृक्षमाश्रित्य भामिनी}
{तस्थौ तस्या हि सन्तप्तं शिरः पादौ तथैव च}


\threelineshloka
{स्थिता सा तु मूहूर्तं वै भर्तुः शापभयाच्छुभा}
{ययावानयितुं भूयः सायकानसितेक्षणा}
{प्रत्याजगाम च शरांस्तानादाय यशस्विनी}


\twolineshloka
{सा वै प्रस्विन्नसर्वाङ्गी पद्भ्यां दुःखं नियच्छती}
{उपाजगाम भर्तारं भयाद्भर्तुः प्रवेपती}


\threelineshloka
{स तामृषिस्तदा क्रुद्धो वाक्यमाह शुभाननाम्}
{रेणुके किं चिरेण त्वमागतेति पुनःपुनः ॥रेणुकोवाच}
{}


\twolineshloka
{शिरस्तप्तं प्रदीप्तौ मे पादौ चैव तपोधन}
{सूर्यतेजोनिरुद्धाऽहं वृक्षच्छायां समाश्रिता}


\threelineshloka
{एतस्मात्कारणाद्ब्रह्मंश्चिरायैतत्कृतं मया}
{एतच्छ्रुत्वा मम विभो मा क्रुधस्त्वं पतोधन ॥जमदग्निरुवाच}
{}


\threelineshloka
{अद्यैनं दीप्तकिरणं रेणुके तव दुःखदम्}
{शरैर्निपातयिष्यामि सूर्यमस्त्राग्नितेजसा ॥भीष्म उवाच}
{}


\twolineshloka
{स विष्फार्य धनुर्दिव्यं गृहीत्वा च शरान्बहूंन्}
{अतिष्ठत्सूर्यमभितो यतो याति ततोमुखः}


\twolineshloka
{अथ तं प्रेक्ष्य सन्नद्धं सूर्योऽभ्येत्य वचोऽब्रवीत्}
{द्विजरूपेण कौन्तेय किं ते सूर्योऽपराध्यति}


\twolineshloka
{आदत्ते रश्मिभिः सूर्यो दिवि तिष्ठंस्ततस्ततः}
{रसं हृतं वै वर्षासु प्रवर्षति दिवाकरः}


\twolineshloka
{ततोऽन्नं जायते विप्र मनुष्याणां सुखावहम्}
{अन्नं प्राणा इति यथा वेदेषु परिपठ्यते}


\twolineshloka
{अथाऽभ्रेषु निगूढश्च रश्मिभिः परिवारितः}
{सप्तद्वीपानिमान्ब्रह्मन्वर्षेणाभिप्रवर्षति}


\twolineshloka
{ततस्तदौषधीनां च वीरुधां पुष्पपत्रजम्}
{सर्वं वर्षाभिनिर्वृत्तमन्नं सम्भवति प्रभो}


\twolineshloka
{जातकर्माणि सर्वाणि व्रतोपनयनानि च}
{गोदानानि विवाहाश्च तथा यज्ञसमृद्धयः}


\twolineshloka
{सत्राणि दानानि तथा संयोगा वित्तसञ्चयाः}
{अन्नतः सम्प्रवर्तन्ते यथा त्वं वेत्थ भार्गव}


\twolineshloka
{रमणीयानि यावन्ति यावदारम्भिकाणि च}
{सर्वमन्नात्प्रभवति विदितं कीर्तयामि ते}


\twolineshloka
{सर्वं हि वेत्थ विप्र त्वं यदेतत्कीर्तितं मया}
{प्रसादये त्वां विप्रर्षे किं ते सूर्यो निपात्यते}


\chapter{अध्यायः १४५}
\threelineshloka
{एवं प्रयाचति तदा भास्करे मुनिसत्तमः}
{जमदग्निर्महातेजाः किं कार्यं प्रत्यपद्यत ॥भीष्म उवाच}
{}


\twolineshloka
{स तथा याचमानस्य मुनिरग्निसमप्रभः}
{जमदग्निः शमं नैव जगाम कुरुनन्दन}


\twolineshloka
{ततः सूर्यो मधुरया वाचा तमिदमब्रवीत्}
{कृताञ्जलिर्विप्ररूपी प्रणम्यैनं विशाम्पते}


\threelineshloka
{चलं निमित्तं विप्रर्षे सदा सूर्यस्य गच्छतः}
{कतं चलं भेत्स्यसि त्वं सदा यान्तं दिवाकरम् ॥जमदग्निरुवाच}
{}


\twolineshloka
{स्थिरं चापि चलं चापि जाने त्वां ज्ञानचक्षुषा}
{अवस्यं विनयाधानं कार्यमद्य मया तव}


\threelineshloka
{मध्याह्ने वै निमेषार्धं तिष्टसि त्वं दिवाकर}
{तत्र भेत्स्यामि सूर्य त्वां न मेऽत्रास्ति विचारणा ॥सूर्य उवाच}
{}


\threelineshloka
{असंशयं मां विप्रर्षे भेत्स्यसे धन्विनांवर}
{अपकारिणं मां विद्दि भगवञ्शरणागतम् ॥भीष्म उवाच}
{}


\twolineshloka
{ततः प्रहस्य भगवाञ्जमदग्निरुवाच तम्}
{न भीः सूर्य त्वया कार्या प्रणिपातगतो ह्यसि}


\twolineshloka
{ब्राह्मणेष्वार्जवं यच्च स्थैर्यं च धरणीतले}
{सौम्यतां चैव सोमस्य गाम्भीर्यं वरुणस्य च}


\twolineshloka
{दीप्तिमग्नेः प्रभां मेरोः प्रतापं धनदस्य च}
{एतान्यतिक्रमेद्यो वै स हन्याच्छरणागतम्}


\twolineshloka
{भवेत्स गुरुतल्पी च ब्रह्मिहा च स वै भवेत्}
{सुरापानं स कुर्याच्च यो हन्याच्छरणागतम्}


\threelineshloka
{एतस्य त्वपनीतस्य समाधिं तात चिन्तय}
{यथा सुखगमः पन्था भवेत्त्वद्रश्मितापितः ॥भीष्म उवाच}
{}


\threelineshloka
{एतावदुक्त्वा स तदा तूष्णीमासीद्भृगूत्तमः}
{अथ सूर्योऽददत्तस्मै छत्रोपानहमाशु वै ॥सूर्य उवाच}
{}


\threelineshloka
{महर्षे शिरसस्त्राणां छत्रं मद्रश्मिवारणम्}
{प्रतिगृह्णीष्व पद्भ्यां च त्राणार्थं चर्मपादुके}
{}


\threelineshloka
{अद्यप्रभृति चैवेह लोके सम्प्रचरिष्यति}
{पुण्यकेषु च सर्वेषु परमक्षय्यमेव च ॥भीष्म उवाच}
{}


\twolineshloka
{उपानहौ च च्छत्रं च सूर्येणैतत्प्रवर्तितम्}
{पुण्यमेतदभिख्यातं त्रिषु लोकेषु भारत}


\twolineshloka
{तस्मात्प्रयच्छ विप्रेषु छत्रोपानहमुत्तमम्}
{धर्मस्ते सुमहान्भावी न मेऽत्रास्ति विचारणा}


\twolineshloka
{छत्रं हि भरतश्रेष्ठः यः प्रदद्याद्द्विजातये}
{शुभ्रं शतशलाकं वै स प्रेत्य सुखमेधते}


\twolineshloka
{स शक्रलोके वसति पूज्यमानो द्विजातिभिः}
{अप्सरोभिश्च सततं देवैश्च भरतर्षभ}


\twolineshloka
{उपानहौ च यो दद्याच्छ्लक्ष्णौ स्नेहसमन्वितौ}
{स्नातकाय महाबाहो संशिताय द्विजातये}


\twolineshloka
{सोपि लोकानवाप्नोति देवतैरभिपूजितान्}
{गोलोके स मुदा युक्तो वसति प्रेत्य भारत}


\twolineshloka
{एतत्ते भरतश्रेष्ठ मया कार्त्स्न्येन कीर्तितम्}
{छत्रोपानहदानस्य फलं भरतसत्तम}


\chapter{अध्यायः १४६}
\twolineshloka
{शूद्राणामिह शुश्रूषा नित्यमेवानुवर्णिता}
{कैः कारणैः कतिविधा शुश्रूषा समुदाहृता}


\threelineshloka
{के च शुश्रूषया लोका विहिता भरतर्षभ}
{शुद्राणां भरतश्रेष्ठ ब्रूहि मे धर्मलक्षणम् ॥भीष्म उवाच}
{}


\twolineshloka
{अत्राप्युदाहरन्तीममितिहासं पुरातनम्}
{शूद्राणामनुकम्पार्थं यदुक्तं ब्रह्मवादिना}


\twolineshloka
{वृद्धः पराशरः प्राह धर्मं शुभ्रमनामयम्}
{अनुग्रहार्थं वर्णानां शौचाचारसमन्वितम्}


\twolineshloka
{धर्मोपदेशमकिलं यथावदनुपूर्वशः}
{शिष्यानध्यापयामास शास्त्रमर्थवदर्थवित्}


\twolineshloka
{क्षान्तेन्द्रियेण मानेन शुचिनाऽचापलेन वै}
{अदुर्बलेन धीरेण शान्तेनोत्तरवादिना}


\twolineshloka
{अलुब्धेनानृशंसेन ऋजुना ब्रह्मवादिना}
{चारित्रतत्परेणैव सर्वभूतहितात्मना}


\twolineshloka
{अरयः षड्विजेतव्या नित्यं स्वं देहमाश्रिताः}
{कामक्रोधौ च लोभश्च मानमोहौ मदस्तथा}


\twolineshloka
{विधिना धृतिमास्थाय शुश्रूषुरनहंकृतः}
{वर्णत्रयस्यानुमतो यथाशक्ति यथाबलम्}


\twolineshloka
{कर्मणा मनसा वाचा चक्षुषा च चतुर्विधम्}
{आस्थाय नियमं धीमाञ्शान्तो दान्तो जितेन्द्रियः}


\twolineshloka
{रक्षोयक्षजनद्वेषी शेषान्नकृतभोजनः}
{वर्णत्रयान्मधु यथा भ्रमरो धर्ममाचरेत्}


% Check verse!
यदि शूद्रस्तपः कुर्याद्वेददृष्टेन कर्मणा ॥इह चास्य परिक्लेशः प्रेत्य चास्यासुभागतिः
\twolineshloka
{अधर्म्यमयशस्यं च तपः शूद्रे प्रतिष्ठितम्}
{अमार्गेण तपस्तप्त्वा म्लेच्छेषु फलमश्नुते}


\threelineshloka
{अन्यथा वर्तमानो हि न शूद्रो धर्ममर्हति}
{अमार्गेणि प्रयातानां प्रत्यक्षादुपलभ्यते}
{चातुर्वर्ण्यव्यपेतानां जातिमूर्तिपरिग्रहः}


\twolineshloka
{तथा ते हि शकाश्चीनाः काम्भोजाः पारदास्तथा}
{शबराः पप्लवाश्चैव तुषारयवनास्तथा}


\twolineshloka
{दार्वाश्च दरदाश्चैव उज्जिहानास्तथेतराः}
{वेणाश्च कङ्कणाश्चैव सिम्हला मद्रकास्तथा}


\twolineshloka
{किष्किन्धकाः पुलिन्दाश्च कह्वाश्चान्ध्राः सनीरगाः}
{गन्धिका द्रमिडाश्चैव बर्बराश्चूचुकास्तथा}


\twolineshloka
{किराताः पार्वतेयाश्च कोलाश्चोलाः सकाषकाः}
{आरूकाश्चैव दोहाश्च याश्चान्याम्लेच्छजातयः}


\twolineshloka
{विकृता विकृताचारा दृश्यन्ते क्रूरबुद्धयः}
{अमार्गेणाश्रिता धर्मं ततो जात्यन्तरं गताः}


\twolineshloka
{अमार्गोपार्जितस्यैतत्तपसो विदितं फलम्}
{न नश्यति कृतं कर्म शुभं वा यदि वाऽशुभम्}


\twolineshloka
{अत्राप्येते वसु प्राप्य विकर्म तपसार्जितम्}
{पाषण्डानर्चयिष्यन्ति धर्मकामा वृथा श्रमाः}


\twolineshloka
{एवं चतुर्णां वर्णानामाश्रमाणां च पार्थिव}
{विपरीतं वर्तमाना म्लेच्छा जायन्त्यबुद्धयः}


\twolineshloka
{अध्यायधनिनो विप्राः क्षत्रियाणां बलं धनम्}
{वणिक्कृषिश्च वैश्यानां शूद्राणां परिचारिका}


\threelineshloka
{व्युच्छेदात्तस्य धर्मस्य निरयायोपपद्यते}
{ततो म्लेच्छा भवन्त्येते निर्घृणा धर्मवर्जिताः}
{पुनश्च निरयं तेषां तिर्यग्योनिश्च शाश्वती}


\twolineshloka
{ये तु सत्यथमास्थाय वर्णाश्रमकृतं पुरा}
{सर्वान्विमार्गानुत्सृज्य स्वधर्मविधिमाश्रिताः}


\twolineshloka
{सर्वभूतदयावन्तो दैवतद्विजपूजकाः}
{शास्त्रदृष्टेन विधिना श्रद्धया जितमन्यवः}


\twolineshloka
{तेषां विधिं प्रवक्ष्यामि यथावदनुपूर्वशः}
{उपादानविधइं कृत्स्नं शुश्रूषाधिगमं तथा}


\twolineshloka
{शिष्टोपनयनं चैव मन्त्राणि विविधानि च}
{तथा शिष्यपरीक्षां च शास्त्रप्रामाण्यदर्शनात्}


\twolineshloka
{प्रवक्ष्यामि यथातत्वं यथावदनुपूर्वशः}
{शौचकृत्यस्य शौचार्थान्सर्वानेव विशेषतः}


\twolineshloka
{महाशौचप्रभृतयो दृष्टास्तत्वार्थदर्शिभिः}
{तत्रापि शूद्रो भिक्षूणामिदं शेष च कल्पयेत्}


\threelineshloka
{भिक्षुभिः सुकृतप्रज्ञैः केवलं दर्ममाश्रितैः}
{सम्यद्गर्शनसम्पन्नैर्गताध्वनि हितार्थिभिः}
{अवकाशमिमं मेध्यं निर्मितं तामवीरुधम्}


\twolineshloka
{निर्जनं संवृतं बुद्ध्वा नियतात्मा जितेन्द्रियः}
{सजलं भाजनं स्थाप्य मृत्तिकां च परीक्षिताम्}


\twolineshloka
{परीक्ष्य भूमिं मूत्रार्थी तत आसीत वाग्यतः}
{उदङ्मुखो दिवा कुर्याद्रात्रौ चेद्दक्षिणामुखः}


\twolineshloka
{अन्तर्हितायां भूमौ तु अन्तर्हितशिरास्तथा}
{असमाप्ते तथा शौचे न वाचं किञ्चिदीरयेत्}


\twolineshloka
{कृतकृत्यस्तथाऽऽचम्य गच्छन्नोदीरयेद्वचः}
{शौचार्थमुपविष्टस्तु मृद्गाजनपुरस्कृतः}


\twolineshloka
{स्थाप्यं कमण्डलुं गृह्यि पार्श्वोरुभ्यामथान्तरे}
{शौचं कुर्याच्छनैर्वीरो बुद्धिपूर्वमसङ्करम्}


\twolineshloka
{पाणिना शुद्धमुदकं सङ्गृह्य विधिपूर्वकम्}
{विप्रुषश्च यता न स्युर्यथा चोरू न संस्पृशेत्}


\twolineshloka
{अपाने मृत्तिकास्तिस्रः प्रदेयास्त्वनुपूर्वशः}
{हस्ताभ्यां च तथा विप्रो हस्तं हस्तेन संस्पृशेत्}


\twolineshloka
{अपाने नव देयाः स्युरिति वृद्धानुशासनम्}
{मृत्तिका दीयमाना हि शोधयेद्देशमञ्जसा}


\twolineshloka
{तस्मात्पाणितले देया मृत्तिकास्तु पुनः पुनः}
{बृद्धिपूर्वं प्रयत्नेन यथा नैव स्पृशेत्स्फिजौ}


\twolineshloka
{यथा घातो हि न भवेत्क्लेदजः परिधानके}
{तथा गुदं प्रमार्जेत शौचार्थं तु पुनःपुनः}


\fourlineindentedshloka
{प्रतिपादं ततस्त्यक्त्वा शौचमुत्थाय कारयेत्}
{सव्ये द्वादश देयाः स्युस्तिस्रस्तिस्रः पुनः पुनः}
{देया कूर्परके हस्ते पृष्ठे बन्धे पुनः पनः}
{}


\twolineshloka
{तथैवादर्शके दद्याच्चतस्रस्तूभयोरपि}
{उभयोर्हस्तयोरेवं सप्तसप्त प्रदापयेत्}


\threelineshloka
{ततोऽन्यां मृत्तिकां गृह्य कार्यं शौचं पुनस्तयोः}
{हस्तयोरेवमेतद्धि महाशौच विधीयते}
{ततोऽन्यथा न कुर्वीत विधिरेष सनातनः}


\twolineshloka
{उपस्थे मूत्रशौचं स्यादत ऊर्ध्वं विधीयते}
{अतोऽन्यथा तु यः सुर्यात्प्रायश्चित्तीयते तु सः}


\twolineshloka
{मलोपहतचेलस्य द्विगुणं तु विधीयते}
{सहपादमथोरुभ्यां हस्तशौचमसंशयम्}


\twolineshloka
{अवधीरयमाणस्य सन्देह उपजायते}
{यथायथा विशुद्ध्येत तत्तथा तदुपक्रमे}


\twolineshloka
{सकर्दमं तु वर्षासु गृहमाविश्य सङ्कटम्}
{हस्तयोर्मृत्तिकास्तिस्रः पादयोः षट् प्रदापयेत्}


\twolineshloka
{कामं दत्त्वा गुदे दद्यात्तिस्रः पद्भ्यां तथैव च}
{हस्तशौचं प्रकर्तव्यं मूत्रशौचविधेस्तथा}


\twolineshloka
{मूत्रशौचे तथा हस्तौ पादाभ्यां चानुपूर्वशः}
{नैष्ठिके स्थानशौचे तु महाशौचं विधीयते}


\twolineshloka
{क्षारौषराभ्यां वस्त्रस्य कुर्याच्छौचं मृदा सह}
{लेपगन्धापनयनममेध्यस्य विधीयते}


\twolineshloka
{स्नानशाट्यां मृदस्तिस्रो हस्ताभ्यां चानुपूर्वशः}
{शौचं प्रयत्नतः कृत्वा कम्पमानः समुद्धरेत्}


\twolineshloka
{देयाश्चतस्रस्तिस्रो वा द्वे वाऽप्येकां तथाऽऽपदि}
{कालमासाद्य देशं च शौचस्य गुरुलाघवम्}


\twolineshloka
{विधिनाऽनेन शौचं तु नित्यं कुर्यादतन्द्रितः}
{अविप्रेक्षन्नसम्भ्रान्तः पादौ प्रक्षाल्य तत्परः}


\twolineshloka
{अप्रक्षालितपादस्तु पाणिमामणिबन्धनात्}
{अधस्तादुपरिष्टाच्च ततः पाणिमुपस्पृशेत्}


\twolineshloka
{मनोगतास्तु निश्शब्दा निश्शब्दं त्रिरपः पिबेत्}
{द्विर्मुखं परिमृज्याच्च खानि चोपस्पृशेद्बुधः}


\twolineshloka
{ऋग्वेदं तेन प्रीणाति प्रथमं यः पिबेदपः}
{द्वितीयं तु यजुर्वेदं तृतीयं साम एव च}


\twolineshloka
{मृज्यते प्रथमं तेन अथर्वा प्रीतिमाप्नुयात्}
{द्वितीयेनेतिहासं च पुराणस्मृतिदेवताः}


\twolineshloka
{यच्चक्षुषि समाधत्ते तेनादित्यं तु प्रीणयेत्}
{प्रीणाति वायुं घ्राणं च दिशश्चाप्यथ श्रोत्रयोः}


\twolineshloka
{ब्रह्माणं तेन प्रीणाति यन्मूर्धनि समापयेत्}
{समुत्क्षिपति चापोर्ध्वमाकाशं तेन प्रीणयेत्}


\twolineshloka
{प्रीणाति विष्णुः पद्भ्यां तु सलिलं वै समादधत् ॥प्राङ्मुखोदङ्मुखो वाऽपि अन्तर्जानुरुपस्पृशेत्}
{}


\twolineshloka
{सर्वत्र विधिरित्येष भोजनादिषु नित्यशः ॥अन्नेषु दन्तलग्नेषु उच्छिष्टः पुनराचमेत्}
{}


\twolineshloka
{विधिरेष समुद्दिष्टः शौचे चाभ्युक्षणं स्मृतम् ॥शूद्रस्यैव विधिर्दृष्टो गृहान्निष्क्रमतस्ततः}
{}


\twolineshloka
{नित्यं त्वलुप्तशौचेन वर्तितव्यं कृतात्मना}
{यशस्कामेन भिक्षुभ्यः शुद्रेणात्महितार्थिना'}


\chapter{अध्यायः १४७}
\twolineshloka
{क्षत्रा आरम्भयज्ञास्तु वीर्ययज्ञा विशः स्मृताः}
{शूद्रा परिचरायज्ञा जपयज्ञास्तु ब्राह्मणाः}


\twolineshloka
{शुश्रूषाजीविनः शूद्रा वैश्या विपणिजीविनः}
{अनिष्टनिग्रहः क्षत्रा विप्राः स्वाध्यायजीविनः}


\twolineshloka
{तपसा शोभते विप्रो राजन्यः पालनादिभिः}
{आतिथ्येन तथा वैश्यः शूद्रो दास्येन शोभते}


\twolineshloka
{यतात्मना तु शूद्रेण शुश्रूषा नित्यमेव च}
{कर्तव्या त्रिषु वर्णेषु प्रायेणाश्रमवासिषु}


\twolineshloka
{अशक्तेन त्रिवर्गस्य सेव्या ह्याश्रमवासिनः}
{यथाशक्यं यथाप्रज्ञं यथाधर्मं यथाश्रुतम्}


% Check verse!
विशेषेणैव कर्तव्या शुश्रूषा भिक्षुकाश्रमे
\twolineshloka
{आश्रमाणां तु सर्वेषां चतुर्णां भिक्षुकाश्रमम्}
{प्रधानमिति वर्ण्यन्ते शिष्टाः शास्त्रविनिश्चये}


\twolineshloka
{यच्चोपदिश्यते शिष्टैः श्रुतिस्मृतिविधानतः}
{तथाऽऽस्थेयमशक्तेन स धर्म इति निश्चितः}


\twolineshloka
{अतोऽन्यथा तु कुर्वाणः श्रेयो नाप्नोति मानवः}
{तस्माद्भिक्षुषु शूद्रेण कार्यमात्महितं सदा}


\twolineshloka
{इह यत्कुरुते श्रेयस्तत्प्रेत्य समुपाश्नुते}
{तच्चानसूयता कार्यं कर्तव्यं यद्धि मन्यते}


\twolineshloka
{असूयता तु तस्येह फलं दुःखादवाप्यते}
{प्रियवादी जितक्रोधो वीततन्द्रीरमत्सरः}


\twolineshloka
{क्षमावाञ्शीलसम्पन्नः सत्यधर्मपरायणः}
{आपद्भावेन कुर्याद्धि शुश्रूषां भिक्षुकाश्रमे}


\twolineshloka
{अयं मे परमो धर्मस्त्वनेनेदं सुदुष्करम्}
{संसारसागरं घोरं तरिष्यामि न संशयः}


\twolineshloka
{विभयो देहमुत्सृज्य यास्यामि परमां गतिम्}
{नातः परं ममाप्यन्य एष धर्मः सनातनः}


\twolineshloka
{एवं सञ्चिन्त्य मनसा शूद्रो बुद्धिसमाधिना}
{कुर्यादविमना नित्यं शुश्रूषाधर्ममुत्तमम्}


\twolineshloka
{शुश्रूषानियमेनेह भाव्यं शिष्टाशिना सदा}
{शमान्वितेन दान्तेन कार्याकार्यविदा सदा}


\twolineshloka
{सर्वकार्येषु कृत्यानि कृतान्येव तु दुर्शयेत्}
{यथा प्रियो भवेद्भिक्षुस्तथा कार्यं प्रसाधयेत्}


\twolineshloka
{यदकल्प्यं भवेद्भिक्षोर्न तत्कार्यं समाचरेत्}
{यथाऽऽश्रमस्याविरुद्धं धर्ममात्राभिसंहितम्}


\twolineshloka
{तत्कार्यमविचारेण नित्यमेव शुभार्थिना}
{मनसा कर्ममा वाचा नित्यमेव प्रसादयेत्}


\twolineshloka
{स्थातव्यं तिष्ठमानेषु गच्छमानाननुव्रजेत्}
{आसीनेष्वासितव्यं च नित्यमेवानुवर्तता}


\twolineshloka
{धर्मलब्धेन स्नेहेन पादौ सम्पीडयेत्सदा}
{उद्वर्तनादींश्च तथा कुर्यादप्रतिचोदितः}


\twolineshloka
{नैजकार्याणि कृत्वा तु नित्यं चैवानुचोदितः}
{यथाविधिरुपस्पृश्य संन्यस्य जलभाजनम्}


\twolineshloka
{भिक्षूणां निलयं गत्वा प्रणम्य विधिपूर्वकम्}
{ब्रह्मपूर्वान्गुरूंस्तत्र प्रणम्यि नियतेन्द्रियः}


\twolineshloka
{तथाऽऽचार्यपुरोगाणामनुकुर्यान्नमस्क्रियाम्}
{स्वधर्मचारिणां चापि सुखं पृष्ट्वाऽभिवाद्य च}


\twolineshloka
{यो भवेत्पूर्वसंसिद्धस्तुल्यकर्मा भवेत्सदा}
{तस्मै प्रणामः कर्तव्यो नेतरेषु कदाचन}


\twolineshloka
{अनुक्त्वा तेषु चोत्थाय नित्यमेव यतव्रतः}
{सम्मार्जनमथो गत्वा कृत्वा चाप्युपलेपनम्}


\twolineshloka
{ततः पुष्पबलिं दद्यापुष्पाण्यादाय धर्मतः}
{निष्क्रम्यावसथात्तूर्णमन्यत्कर्म समाचरेत्}


\threelineshloka
{यथोपघातो न भवेत्स्वाध्यायेऽऽश्रमिणां तथा}
{उपघातं तु कुर्वाण एनसा सम्प्रयुज्यते}
{तथाऽऽत्मा प्रणिधातव्यो यथा ते प्रीतिमाप्नुयुः}


\twolineshloka
{परिचारकोऽहं वर्णानां त्रयाणां धर्मतः स्मृतः}
{किमुताश्रमवृद्धानां यथालब्धोपजीविनाम्}


\twolineshloka
{भिक्षूणां गतरागाणां केवलं ज्ञानदर्शिनम्}
{विशेषेण मया कार्या शुश्रूषा नियतात्मना}


\threelineshloka
{तेषां प्रसादात्तपसा प्राप्स्यामीष्टां शुभां गतिम्}
{एवमेतद्विनिश्चित्य यदि सेवेत भिक्षुकान्}
{विधिना स्वोपदिष्टेन प्राप्नोति परमां गतिम् ॥'}


\chapter{अध्यायः १४८}
\twolineshloka
{न तथा सम्प्रदानेन नोपवासादिभिस्तथा}
{इष्टां गतिमवाप्नोति यथा शुश्रूषकर्मणा}


\twolineshloka
{यादृशेन तु तोयेन शुद्धिं प्रकुरुते नरः}
{तादृग्भवति तद्धौतमुदकस्य प्रभावतः}


\twolineshloka
{शूद्रोप्येतेन मार्गेण यादृशं सेवते जनम्}
{तादृग्भवति संसर्गादचिरेण न संशयः}


\twolineshloka
{तस्मात्प्रयत्नतः सेव्या भिक्षवो नियतात्मना}
{उदकग्राहणाद्येन स्नपनोद्वर्तनैस्तथा}


\twolineshloka
{अध्वना कर्शितानां च व्याधितानां तथैव च}
{शुश्रूषां नियतं कुर्यात्तेषामापदि यत्नतः}


\threelineshloka
{दर्भाजिनान्यवेक्षेत भैक्षभाजनमेव च}
{यथाकामं च कार्याणि सर्वाण्येवोपसाधयेत्}
{प्रायश्चित्तं यता न स्यात्तथा सर्वं समाचरेत्}


\twolineshloka
{व्याधितानां तु भिक्षूणां चेलप्रक्षालनादिभिः}
{प्रतिकर्मक्रिया कार्या भेषजानयनैस्तथा}


\twolineshloka
{पिंषणालेपनं चूर्णं कषायमथ साधनम्}
{नान्यस्य प्रतिचारेषु सुखार्थमुपपादयेत्}


\twolineshloka
{भिक्षाटनोऽभिगच्छेत भिषजश्च विपश्चितः}
{ततो विनिष्क्रियार्थानि द्रव्याणि समुपार्जयेत्}


\twolineshloka
{यश्च प्रीतमना दद्यादादद्याद्भेषजं नरः}
{अश्रद्धया हि दत्तानि तान्यभोक्ष्याणि भिक्षुभिः}


\twolineshloka
{श्रद्धया यदुपादत्तं श्रद्धया चोपपादितम्}
{तस्योपभोगाद्धर्मः स्याद्व्याधिभिश्च निवर्त्यते}


\twolineshloka
{आदेहपतनादेवं शुश्रूषेद्विधिपूर्वकम्}
{न त्वेवं धर्ममुत्सृज्य कुर्यात्तेषां प्रतिक्रियाम्}


\threelineshloka
{स्वभावतो हि द्वन्द्वानि विप्रयान्त्युपयान्ति च}
{स्वभावतः सर्वभावा भवन्ति नभवन्ति च}
{सागरस्योर्मिसदृशा विज्ञातव्या गुणात्मकाः}


\twolineshloka
{विद्यादेवं हि यो धीमांस्तत्ववित्तत्वदर्शनः}
{न स लिप्येत पापेन पद्मपत्रमिवाम्भसा ॥'}


\chapter{अध्यायः १४९}
\twolineshloka
{एवं प्रयतितव्यं हि शुश्रूषार्थमतन्द्रितैः}
{सर्वाभिरूपसेवाभिस्तुष्यन्ति यतयो यथा}


\twolineshloka
{नापराध्येत भिक्षोस्तु न चैनमवधीरयेत्}
{उत्तरं च न सन्दद्यात्क्रुद्धं चैव प्रसादयेत्}


\twolineshloka
{श्रेय एवाभिधातव्यं कर्तव्यं च प्रहृष्टवत्}
{तूष्णींभावेन वै तत्र न क्रुद्धमभिसंवदेत्}


\twolineshloka
{नाददीत परस्वानि न गृह्णीयादयाचितम्}
{लब्धालब्धेन जीवेत तथैव परितोषयेत्}


\twolineshloka
{कोपिनं तु न याचेत ज्ञानविद्वेषकारितः}
{स्थावरेषु दयां कुर्याज्जङ्गमेषु चु प्राणिषु}


\threelineshloka
{यथाऽऽत्मनि तथाऽन्येषु समां दृष्टिं निपातयेत्}
{सर्वभूतेषु चात्मानं सर्वभूतानि चात्मनि}
{सम्पश्यमानो विचरन्ब्रह्मभूयाय कल्पते}


\twolineshloka
{हिंसा वा यदि वाऽहिंसां न कुर्यादात्मकारणात्}
{यत्रेतरो भवेन्नित्यं दोषं तत्र न कारयेत्}


\twolineshloka
{एवं स मुच्यते दोषात्परानाश्रित्य वर्तयन्}
{आत्माश्रयेण दोषेण लिप्यते ह्यल्पबुद्धिमान्}


\twolineshloka
{जरायुजाण्डजाश्चैव उद्भिज्जाः स्वेदजाश्व ये}
{अवध्याः सर्व एतैते बुधैः समनुवर्णिताः}


\threelineshloka
{निश्चयार्थं विबुद्धानां प्रायश्चित्तं विधीयते}
{हिंसा यथाऽन्या विहिता तथा दोष प्रयोजयेत्}
{तथोपदिष्टं गुरुणा शिष्यस्य चरतो विधिम्}


\twolineshloka
{न हि लोभः प्रभवति हिंसा वाऽपि तदात्मिका}
{शास्त्रदर्शनमेतद्धि विहितं विश्वयोनिना}


\twolineshloka
{यद्येतदेवं मन्येत शूद्रो ह्यपि च बुद्धिमान्}
{कृतं कृतवतां गच्छेत्किं पुनर्यो निषेवते}


\twolineshloka
{न शूद्रः पतते कश्चिन्न च संस्कारमर्हति}
{नास्याधिकारो धर्मेऽस्ति न धर्मात्प्रतिषेधनम्}


\twolineshloka
{अनुग्रहार्तं मनुना सर्ववर्णेषु वर्णितम्}
{}


\threelineshloka
{यदापवादस्तु भवेत्स्त्रीकृतः परिचारके}
{अभ्रावकाशशयनं तस्य संवत्सरं स्मृतम्}
{तेन तस्य भवेच्छान्तिस्ततो भूयोप्युपाव्रजेत्}


\fourlineindentedshloka
{सवर्णाया भवेदेतद्धीनायास्त्वर्धमर्हति}
{वर्षत्रयं तु वैश्यायाः क्षत्रियायास्तु षट् समाः}
{ब्राह्मण्या तु समेतस्य समा द्वादश कीर्तिताः}
{}


\twolineshloka
{कटाग्निना वा दग्धव्यस्तस्मिन्नेव क्षणे भवेत्}
{शिश्नावपातनाद्वाऽपि विशुद्धिं समवाप्नुयात्}


\twolineshloka
{अनस्थिबन्धमेकं तु यदि प्राणैर्वियोजयेत्}
{उपोष्यैकाहमादद्यात्प्राणायामांस्तु द्वादश}


\twolineshloka
{त्रिः स्नानमुदके कृत्वा तस्मात्पापात्प्रमुच्यते}
{अस्थिबन्धेषु द्विगुणं प्रायश्चित्तं विधीयते}


\twolineshloka
{अनेन विधिना वाऽपि स्थावरेषु न संशयः}
{कायेन पद्भ्यां हस्ताभ्यामपराधात्तु मुच्यते}


\twolineshloka
{अदुष्टं क्षपयेद्यस्तु सर्ववर्णेषु यश्चरेत्}
{तस्याप्यष्टगुणं विद्यात्प्रायश्चित्तं तदेव तु}


\twolineshloka
{चतुर्गुणं कर्मकृते द्विगुणं वाक्प्रदूषिते}
{कृत्वा तु मानसं पापं तथैवैकगुणं स्मृतम्}


\twolineshloka
{तस्मादेतानि सर्वाणि विदित्वा न समाचरेत्}
{सर्वभूतहितार्थं हि कुशलानि समाचरेत्}


\twolineshloka
{एवं समाहितमनाः सेवते यदि यत्तमान्}
{तद्गतिस्तत्समाचारस्तन्मनास्तत्परायणः}


\twolineshloka
{नाभिनन्देत मरणं नाभिनन्देत जीवितम्}
{कालमेव प्रतीक्षेत निर्वेशं भृतको यथा}


\twolineshloka
{एवं प्रवर्तमानस्तु विनीतः प्रयतात्मवान्}
{निर्णयं पुण्यपापाभ्यामचिरेणोपगच्छति'}


\chapter{अध्यायः १५०}
\twolineshloka
{शुश्रूषानिरतो नित्यमरिष्टान्युपलक्षयेत्}
{त्रैवार्षिकं द्विवार्षिकं वा वार्षिकं वा समुत्थितम्}


\twolineshloka
{षाण्मासिकं मासिकं वा साप्तरात्रिकमेव वा}
{सर्वांस्तदर्थान्वा विद्यात्तेषां चिह्नानि लभयेत्}


\twolineshloka
{पुरुषं हिरण्मयं यस्तु तिष्ठन्तं दक्षिणामुखम्}
{लक्षयेदुत्तरेणैव मृत्युस्त्रैवार्षिको भवेत्}


\twolineshloka
{शुद्धमण्डलमादित्यमरश्मिं सम्प्रपश्यतः}
{संवत्सरद्वयेनैव तस्य मत्युं समादिशेत्}


\twolineshloka
{ज्योत्स्नायामात्मनश्छायां सच्छिद्रां यः प्रपश्यति}
{मृत्युं संवत्सरेणैव जानीयात्सुविचक्षणः}


\twolineshloka
{विशिरस्कां यदा छायां पश्येत्पुरुष आत्मनः}
{जानीयादात्मनो मृत्युं षाण्मासेनेह बुद्धिमान्}


\twolineshloka
{कर्णौ पिधाय हस्ताभ्यां शब्दं न शृणुते यदि}
{जानीयादात्मनो मृत्युं मासेनैव विचक्षणः}


\twolineshloka
{शवगन्धमुपाघ्राति अन्यद्वा सुरभिं नरः}
{देवतायतनस्थो वै सप्तरात्रेण मृत्युभाक्}


\twolineshloka
{कर्णनासापनयनं दन्यदृष्टिविरागता}
{लुप्तसंज्ञं हि करणं सद्यो मृत्युं समादिशेत्}


\twolineshloka
{एवमेषामरिष्टानां पश्येदन्यतमं यदि}
{न तं कालं परीक्षेत यथाऽरिष्टं प्रकल्पितम्}


\twolineshloka
{अभ्यासेन तु कालस्य गच्छेत पुलिनं शुचि}
{तत्र प्राणान्प्रमुञ्चेत तमीशानमनुस्मरन्}


\twolineshloka
{ततोऽन्यदेहमासाद्य गान्धर्वं स्थानमाप्नुयात्}
{तत्रस्थो वसते विंशत्पद्मानि सुहहाद्युतिः}


\twolineshloka
{गन्धर्वैश्चित्रसेनाद्यैः सहितः सत्कृतस्तथा}
{नीलवैडूर्यवर्णेन विमानेनावभासयन्}


\twolineshloka
{नभस्थलमदीनात्मा सार्धमप्सरसां गणैः}
{छन्दकामानुसारी च तत्रतत्र महीयते}


\twolineshloka
{मोदतेऽमरतुल्यात्मा सदाऽमरगणैः सह}
{पतितश्च क्षये काले क्षणेन विमलद्युतिः}


\twolineshloka
{वैश्यस्य बहुवित्तस्य कुलेऽग्र्ये बहुगोधने}
{अवाप्य तत्र वै जन्म स पूतो देवकर्मणा}


\twolineshloka
{छन्दसा जागतेनैव प्राप्तोपनयनं ततः}
{क्षौमवस्त्रोपकरणं द्विजत्वं समवाप्य तु}


\twolineshloka
{अधीयमानो वेदार्थान्गुरुशुश्रूषणे रतः}
{ब्रह्मचारी जितक्रोधस्तपस्वी जायते ततः}


\twolineshloka
{अधीत्य दक्षिणां दत्त्वा गुरवे विधिपूर्वकम्}
{कृतदारः समुपैति गृहस्थव्रतमुत्तमम्}


\twolineshloka
{ददाति यजते चैव चज्ञैर्विपुलदक्षिणैः}
{अग्निहोत्रमुपासन्वै जुह्वच्चैव यथाविधिः}


\twolineshloka
{धर्मं सञ्चिनुते नित्यं मृदुगामी जितेन्द्रियः}
{स कालपरिणामात्तु मृत्युना सम्प्रयुज्यते}


\twolineshloka
{संस्कृतश्चाग्निहोत्रेण कृतपात्रोपधानवान्}
{संस्कृतो देहमुत्सृज्य मरुद्भिरुपपद्यते}


\twolineshloka
{मरुद्भिः सहितश्चापि तुल्यतेजा महाद्युतिः}
{बालार्कसमवर्णेन विमानेन विराजता}


\twolineshloka
{सुखं चरति तत्रस्तो गन्थर्वाप्सरसां गणैः}
{विरजोम्बरसंवीतस्तप्तकाञ्चनभूषणः}


\twolineshloka
{छन्दकामानुसारी च द्विगुणं कालमावसेत्}
{सन्निवर्तेत कालेन स्थानादस्मात्परिच्युतः}


\twolineshloka
{अवितृप्तविहारार्थो दिव्यभोगान्विहाय तु}
{सञ्जायते नृपकुले गजाश्वरथसंकुले}


\twolineshloka
{पार्थिवीं श्रियमापन्नः श्रीमान्धर्मपतिर्यथा}
{जन्मप्रभृति संस्कारं चौलोपनयनानि च}


\twolineshloka
{प्राप्य राजकुले तत्र यथावद्विधिपूर्वकम्}
{छ्दसा त्रैष्टुभेनेह द्विजत्वमुपनीयते}


\twolineshloka
{अधीत्य वेदमखिलं धनुर्वेदं च मुख्यशः}
{समावृत्तस्ततः पित्रा यौवराज्येऽभिषिच्यते}


\twolineshloka
{कृतदारक्रियः श्रीमान्राज्यं सम्प्राप्य धर्मतः}
{प्रजाः पालयते सम्यक् षड्भागकृतसंविधिः}


\twolineshloka
{यज्ञैर्बहुभिरीजानः सम्यगाप्तार्थदक्षिणैः}
{प्रशासति महीं श्रीमान्राज्यमिन्द्रसमुद्युतिः}


\twolineshloka
{स्वधर्मनिरतो नित्यं पुत्रपौत्रसहायवान्}
{कालस्य वशमापन्नः प्राणांस्त्यजति संयुगे}


\twolineshloka
{देवराजस्य भवनमिन्द्रलोकमवाप्नुते}
{सम्पूज्यमानस्त्रिदिवैर्विचचार यथासुखम्}


\twolineshloka
{राजर्षिभिः पुण्यकृद्भिर्यथा देवपतिस्तथा}
{तैः स्तूयते बन्दिभिस्तु नानावाद्यैः प्रबोध्यते}


\twolineshloka
{दिव्यजाम्बूनदमयं भ्राजमानं समन्ततः}
{वराप्सरोभिः सम्पूर्णं देवगन्धर्वसेवितम्}


\twolineshloka
{यानमारुह्य विचरेद्यथा शक्रः शचीपतिः}
{स तत्र वसते षष्टिं पद्मानीह मुदान्वितः}


\threelineshloka
{सर्वाँल्लोकाननुचरन्महर्द्धिरवभासयन्}
{अथ पुण्यक्षयात्तस्मात्स्थाप्यते भुवि भारत}
{जायते च द्विजकुले वेदवेदाङ्गपारगे ॥'}


\chapter{अध्यायः १५१}
\twolineshloka
{ततः श्रुतिसमापन्नः संस्कृश्च यथाविधि}
{चौलोपनयनं तस्य यथावत्क्रियते द्विजैः}


\twolineshloka
{ततोऽष्टमे स वर्षे तु व्रतोपनयनादिभिः}
{क्रियाभिर्विधिदृष्टाभिर्ब्रह्मत्वमुपनीयते}


\twolineshloka
{गायत्रेण छन्दसा तु संस्कृतश्चरितव्रतः}
{अधीयमानो मेधावी शुद्धात्मा नियतव्रतः}


\twolineshloka
{अचिरेणैव कालेन साङ्गान्वेदानवाप्नुते}
{समावृत्तः स धर्मात्मा समावृत्तिक्रियस्तथा}


\twolineshloka
{याजनाध्यापनरतः कुशले कर्मणि स्थितः}
{अग्निहोइत्रपरो नित्यं देवतातिथिपूजकः}


\twolineshloka
{यजते विविधैर्यज्ञैर्जपयज्ञैस्तथैव च}
{न्यायागतधनान्वेषी न्यायवृत्तस्तपोधनः}


\twolineshloka
{सर्वभूतहितश्चैव सर्वशास्त्रविशारदः}
{स्वदारपरितुष्टात्मा क्रतुगामी जितेन्द्रियः}


\twolineshloka
{परापवादविरतः सत्यव्रतपरः सदा}
{स कालपरिणामात्तु संयुतः कालधर्मणा}


\twolineshloka
{संस्कृतश्चाग्निहोत्रेण यथावद्विधिपूर्वकम्}
{सोमलोकमवाप्नोति देहन्यासान्न संशयः}


\twolineshloka
{तत्र सोमप्रभैर्देवैरग्निष्वात्तैश्च भास्वरैः}
{तथा बर्हिषदैश्चैव देवैराङ्गिरसैरपि}


\threelineshloka
{विश्वेभिश्चैव देवैश्च तथा ब्रह्मर्षिभिः पुनः}
{देवर्षिभिश्चाप्रतिमैस्तथैवाप्सरसां गणैः}
{साध्यैः सिद्धैश्च सततं सत्कृतस्तत्र मोदते}


\twolineshloka
{जातरूपमयं दिव्यमर्कतुल्यं मनोजवम्}
{देवगन्धर्वसंकीर्णं विमानमधिरोहति}


\twolineshloka
{सौम्यरूपा मनःकान्तास्तप्तकाञ्चनभूषणाः}
{सोमकन्या विमानस्थं रमयन्ति मुदान्विताः}


\twolineshloka
{स तत्र रमते प्रीतः सह देवैः सहर्षिभिः}
{लोकान्सर्वाननुचरन्दीप्ततेजा मनोजवः}


\twolineshloka
{सभां कामजवीं चापि नित्यमेवाभिगच्छति}
{सर्वलोकेश्वरमृषिं नमस्कृत्य पितामहम्}


\threelineshloka
{परमेष्ठिरनन्तश्रीर्लोकानां प्रभवाप्ययः}
{यतः सर्वाः प्रवर्तन्ते सर्गप्रलयविक्रियाः}
{स तत्र वर्तते श्रीमान्द्विशतं द्विजसत्तम}


\twolineshloka
{अथ कालक्षयात्तस्मात्स्थानादावर्तते पुनः}
{जातिधर्मांस्तथा सर्वान्सर्गादावर्तनानि च}


\twolineshloka
{अशाश्वतमिदं सवेमिति चिन्त्योपलभ्य च}
{शाश्वतं दिव्यमचलमदीनमपनर्भवम्}


\twolineshloka
{आस्थास्यत्यभयं नित्यं यत्रावृत्तिर्न विद्यते}
{यत्रि गत्वा न म्रियते जन्म चापि न विद्यते}


\twolineshloka
{गर्भक्लेशामयाः प्राप्ता जायता च पुनः पुनः}
{कायक्लेशाश्च विविधा द्वन्द्वानि विविधानि च}


\twolineshloka
{शीतोष्णसुखदुःखानि ईर्ष्याद्वेषकृतानि च}
{तत्रतत्रोपभुक्तानि न क्वचिच्छाश्वती स्थितिः}


\twolineshloka
{एवं स निश्चयं कृत्वा निर्मुच्यि ग्रहबन्धनात्}
{छित्त्वा भार्यामयं पाशं तथैवापत्यसम्भवम्}


\twolineshloka
{यतिधर्ममुपाश्रित्यि गुरुशुश्रूषणे रतः}
{अचिरेणैव कालेन श्रेयः समभिगच्छति}


\twolineshloka
{योगशास्त्रं च साङ्ख्यं च विदित्वा सोऽर्थतत्वतः}
{अनुज्ञातश्च गुरुणा यथाशास्त्रमवस्थितः}


\twolineshloka
{पुण्यतीर्थानुसेवी च नदीनां पुलिनाश्रयः}
{शून्यागारनिकेतश्च वनवृक्षगुहाशयः}


\twolineshloka
{अरण्यानुचरो नित्यं देवारण्यनिकेतनः}
{एकरात्रं द्विरात्रं वा न क्वचित्सज्जते द्विजः}


\twolineshloka
{शीर्णपर्णभुगेवापि वने चरति भिक्षुकः}
{न भोगार्थमनुप्रेत्य यात्रामात्रं समश्नुते}


\twolineshloka
{धर्मलब्धं समश्नाति न कामात्किञ्चिदश्नुते}
{युगमात्रदृगध्वानं क्रोशादूर्ध्वं न गच्छति}


\twolineshloka
{समो मानावमानाभ्यां समलोष्टाश्मकाञ्चनः}
{सर्वभूताभयकरस्तथैवाभयदक्षिणः}


\twolineshloka
{निर्द्वन्द्वो निर्नमस्कारो निरानन्दपरिग्रहः}
{निर्ममो निरहङ्कारः सर्वभूतनिराश्रयः}


\twolineshloka
{परिसङ्ख्यानतत्वज्ञस्तदा सत्यरतः सदा}
{ऊर्ध्वं नाधो न तिर्यक्च न किञ्चिदभिकामयेत्}


\twolineshloka
{एवं हि रममाणस्तु यतिधर्मं यथाविधि}
{कालस्य परिणामात्तु यथा पक्वफलं तथा}


\twolineshloka
{स विसृज्य स्वकं देहं प्रविशेद्ब्रह्म शाश्वतम्}
{निरामयमनाद्यन्तं गुणसौम्यमचेतनम्}


\twolineshloka
{निरक्षरमबीजं च निरिन्द्रियमजं तथा}
{अजय्यमक्षयं यत्तदभेद्यं सूक्ष्ममेव च}


\threelineshloka
{निर्गुणं च प्रकृतिमन्निर्विकारं च सर्वशः}
{भूतभव्यभविष्यस्य कालस्य परमेश्वरम्}
{}


% Check verse!
अव्यक्तं पुरुषं क्षेत्रमानन्त्याय प्रपद्यते ॥'
\chapter{अध्यायः १५२}
\twolineshloka
{एवं स भिक्षुर्निर्वाणं प्राप्नुयाद्दग्धकिल्बिषः}
{इहस्थो देहमुत्सृज्य नीडं शकुनिवद्यथा}


\twolineshloka
{सत्पथालम्बनादेव शूद्रः प्राप्नोति सद्गतिम्}
{ब्रह्मणः स्थानमचलं स्थानात्स्थानमवाप्नुयात्}


\twolineshloka
{यथा खनन्खनित्रेण जाङ्गले वारि विन्दति}
{अनिर्वेदात्ततः स्थानमीप्सितं प्रतिपद्यते}


\fourlineindentedshloka
{सैषा गतिरनाद्यन्ता सर्वैरप्युपधारिता}
{तस्माच्छूद्रैरनिर्वेदाच्छ्रद्दधानैस्तु नित्यदा}
{वर्तितव्यं यथाशक्त्या यथा प्रोक्तं मनीषिभिः}
{}


\twolineshloka
{यत्करोति तदश्नाति शुभं वा यदि वाऽशुभम्}
{नाकृतं भुज्यते कर्म न कृतं नश्यते फलम्}


\threelineshloka
{तथा शुभसमाचारः शुभमेवाप्नुते फलम्}
{तथाऽशुभसमाचारो ह्यशुभं समवाप्नुते}
{शुभान्येव समादद्याद्य इच्छेद्भूतिमात्मनः}


\twolineshloka
{भूतिश्च नान्यतः शक्त्या शूद्राणामिति निश्चयः}
{क्रते यतीनां शुश्रूषामिति सन्तो व्यवस्थिताः}


\twolineshloka
{तस्मादागमसम्पन्नो भवेत्सुनियतेन्द्रियः}
{शक्यते ह्यागमादेव गतिं प्राप्तुमनामयाम्}


\twolineshloka
{वरा चैषा गतिर्दृष्टा यामन्वेपन्ति साधवः}
{यत्रामृतत्वं लभते त्यक्त्वा दुःखमनन्तरम्}


\threelineshloka
{इमं हि धर्मिमास्थाय येऽपि स्युः पापयोनयः}
{स्त्रियो वैश्याश्च शूद्राश्च प्राप्नुयुः परमां गतिम्}
{किं पुनर्ब्राह्मणो विद्वान्क्षत्रियो वा बहुश्रुतः}


\twolineshloka
{न चाप्यक्षीणपापस्य ज्ञानं भवति देहिनः}
{ज्ञानोपलब्धिर्भवति कृतकृत्यो यदा भवेत्}


\twolineshloka
{उपलभ्य तु विज्ञानं ज्ञानं वाऽप्यनसूयकः}
{तथैव वर्तेद्गुरुषु भूयांसं वा समाहितः}


\twolineshloka
{अथावमन्येत गुरुं तथा तेषु प्रवर्तते}
{व्यर्थमस्य श्रुतं भवति ज्ञानमज्ञानतां व्रजेत्}


\twolineshloka
{गतिं चाप्यशुभां गच्छेन्निरयाय न संशयः}
{प्रक्षीयते तस्य पुण्यं ज्ञानमस्य विरुध्यते}


\twolineshloka
{अदृष्टपूर्वकल्याणो यथा दृष्ट्वा विधिं नरः}
{उत्सेकान्मोहमापद्य तत्वज्ञानमवाप्तवान्}


\twolineshloka
{एवमेवि हि नोत्सेकः कर्तव्यो ज्ञानसम्भवः}
{फलं ज्ञानस्य हि शमः प्रशामाय यतेत्सदा}


\twolineshloka
{उपशान्तेन दान्तेन क्षमायुक्तेन सर्वदा}
{शुश्रूषा प्रतिपत्तव्या नित्यमेवानसूयता}


\twolineshloka
{धृत्या शिश्नोदरं रक्षेत्पाणिपादं च चक्षुषा}
{इन्द्रियार्थांश्च मनसा मनो बुद्धौ समादधेत्}


\twolineshloka
{धृत्याऽऽसीत ततो गत्वा शुद्धदेशं सुसंवृतम्}
{लब्धासनं यथा दृष्टं विधिपूर्वं समाचरेत्}


\twolineshloka
{ज्ञानयुक्तस्तथा देवं हृदिस्थमुपलक्षयेत्}
{आदीप्यमानं वपुषा विधूममनलं यथा}


\twolineshloka
{रश्मिमन्तमिवादित्यं वैद्युताग्निमिवाम्बरे}
{संस्थितं हृदये पश्येदीशं शाश्वतमव्ययम्}


\twolineshloka
{न चायुक्तेन शक्येत द्रष्टुं देहे महेश्वरः}
{युक्तस्तु पश्यते बुद्ध्या सन्निवेश्य मनो हृदि}


\twolineshloka
{अथ त्वेवं न शक्नोति कर्तुं हृदयधारणम्}
{यथासांख्यमुपासीत यथावद्योगमास्थितः}


\twolineshloka
{पञ्च बुद्धीन्द्रियाणीह पञ्च कर्मेन्द्रियाणि वै}
{पञ्च भूतविशेषाश्च मनश्चैव तु षोडश}


\threelineshloka
{तन्मात्राण्यपबि पञ्चैव मनोऽहङ्कार एव च}
{अष्टमं चाप्यथाव्यक्तमेताः प्रकृतिसंज्ञिताः}
{एताः प्रकृतयश्चाष्टौ विकाराश्चापि षोडश}


\twolineshloka
{एवमेतदिहस्थेन विज्ञेयं तत्वबुद्धिना}
{एवं वर्ष्म समुत्तीर्य तीर्णो भवति नान्यथा}


% Check verse!
परिसंख्यानमेवैतन्मन्तव्यं ज्ञानबुद्धिना
\threelineshloka
{अहन्यहनि शान्तात्मा पावनाय हिताय च}
{एवमेव प्रसंख्याय तत्वबुद्धिर्विमुच्यते}
{निष्पलं केवलं भवति शुद्धतत्वार्थतत्ववित्}


\twolineshloka
{भिक्षुकाश्रममास्थाय शुश्रूषानिरतो बुधः}
{शूद्रो निर्मुच्यते सत्वसंसर्गादेव नान्यथा}


\threelineshloka
{सत्संनिक्रषे परिवर्तितव्यंविद्याधिकाश्चापि निषेवितव्याः}
{सवर्णतां गच्छति सन्निकर्षा-न्नीलः खगो मेरुमिवाश्रयन्वै ॥भीष्म उवाच}
{}


\twolineshloka
{इत्येवमाख्याय महामुनिस्तदाचतुर्षु वर्णेषु विधानमर्थवित्}
{शुश्रूषया वृत्तगतिं समाधिनासमाधियुक्तः प्रययौ स्वमाश्रमम् ॥'}


\chapter{अध्यायः १५३}
\threelineshloka
{केषां देवा महाभागाः सन्नमन्ते महात्मनाम्}
{लोकेऽस्मिंस्तनृषीन्सर्वञ्श्रोतुमिच्छामि सत्तम ॥भीष्म उवाच}
{}


\twolineshloka
{इतिहासमिमं विप्राः कीर्तयन्ति पुराविदः}
{अस्मिन्नर्थे महाप्राज्ञास्तं निबोध युधिष्ठिर}


\twolineshloka
{वृत्रं हत्वाऽप्युपावृत्तं त्रिदशानां पुरस्कृतम्}
{महेन्द्रमनुसम्प्राप्तं स्तूयमानं महर्षिभिः}


\twolineshloka
{श्रिया परमया युक्तं रथस्थं हरिवाहनम्}
{मातलिः प्राञ्जलिर्भूत्वा देवमिन्द्रमुघाच ह}


\twolineshloka
{नमस्कृतानां सर्वेषां भगवंस्त्वं पुरस्कृतः}
{येषां लोके नमस्कुर्यात्तान्ब्रवीतु भवान्मम}


\twolineshloka
{तस्य तद्वचनं श्रुत्वा देवराजः शचीपतिः}
{यन्तारं परिपृच्छन्तं तमिन्द्रः प्रत्युवाच सः}


\twolineshloka
{धर्मं चार्थं च कामं च येषां चिन्तयतां मतिः}
{नाधर्मे वर्तते नित्यं तान्नमस्यामि मातले}


\twolineshloka
{ये रूपगुणसम्पन्नाः प्रमदाहृदयंगमाः}
{निवृत्ताः कामभोगेषु तान्नमस्यामि मातले}


\twolineshloka
{स्वेषु भोगेषु सन्तुष्टाः सुवाचो वचनक्षमाः}
{अमानकामाश्चार्घार्हास्तान्नमस्यामि मातले}


\twolineshloka
{धनं विद्यास्तथैश्वर्यं येषां न चलयेन्मतिम्}
{चलितां ये निगृह्णन्ति तान्नित्यं पूजयाम्यहम्}


\twolineshloka
{इष्टैर्दारैरुपेतानां शुचीनामग्निहोत्रिणाम्}
{चतुष्पादकुटुम्बानां मातिले प्रणमाम्यहम्}


\twolineshloka
{महतस्तपसा प्राप्तौ धनस्य विपुलस्य च}
{कत्यागस्तस्य न वै कार्यो योऽऽत्मानं नावबुध्यते}


\twolineshloka
{येषामर्थस्तथा कामो धर्ममूलविवर्धितः}
{धर्मार्थौ तस्य नियतौ तान्नमस्यामि मातले}


% Check verse!
धर्ममूलार्थकामानां ब्राह्मणानां गवामपि ॥पतिव्रतानां नारीणां प्रणामं प्रकरोम्यहम्
\twolineshloka
{ये भुक्त्वा मानुषान्भोगान्पूर्वे वयसि मातले}
{तपसा स्वर्गमायान्ति शश्वत्तान्पूजयाम्यहम्}


\twolineshloka
{असंभोगान्नचासक्तान्धर्मनित्याञ्जितेन्द्रियान्}
{संन्यस्तानचलप्रख्यान्मनसा पूजयामि तान्}


\twolineshloka
{ज्ञानप्रसन्नविद्यानां निरूढं धर्ममीप्सितम्}
{परैः कीर्तितशौचानां मातले तान्नमाम्यहम्'}


\chapter{अध्यायः १५४}
\threelineshloka
{गार्हस्थ्यं धर्ममखिलं प्रब्रूहि भरतर्षभ}
{ऋद्धिमाप्नोति किं कृत्वा मनुष्य इह पार्थिव ॥भीष्म उवाच}
{}


\twolineshloka
{अत्र ते वर्तयिष्यामि पुरावृत्तं जनाधिप}
{वासुदेवस्य संवादं पृथिव्याश्चैव भारत}


\threelineshloka
{संस्तुत्य पृथिवीं देवीं वासुदेवः प्रतापवान्}
{पप्रच्छ भरतश्रेष्ठ मां त्वं यत्पृच्छसेऽद्य वै ॥वासुदेव उवाच}
{}


\threelineshloka
{गार्हस्थ्यं धर्ममाश्रित्य मया वा मद्विधेन वा}
{किमवश्यं धरे कार्यं किं वा कृत्वा सुखं भवेत् ॥पृथिव्युवाच}
{}


\twolineshloka
{ऋषयः पितरो देवा मनुष्याश्चैव माधव}
{पूज्याश्चैवार्चनीयाश्च यथा चैव निबोध मे}


\twolineshloka
{सदा यज्ञेन देवांश्च सदाऽऽतिथ्येन मानुषान्}
{छन्दतस्तर्पणेनापि पितॄन्युञ्जन्ति नित्यशः}


\twolineshloka
{तेन ह्यृषिगणाः प्रीता ब्रह्मचयेणि चानघ}
{नित्यमग्निं परिचरेदभुक्त्वा बलिकर्म च}


\twolineshloka
{कुर्यात्तथैव देवान्वै प्रियं मे मधुसूदन}
{कुर्यादहरहः श्राद्धमन्नाद्येनोदकेन च}


\twolineshloka
{पयोमूलफलैर्वाऽपि पितॄणां प्रीतिमावहेत्}
{सिद्धान्नाद्वैश्वदेवं वै कुर्यादग्नौ यथाविधि}


\twolineshloka
{आग्नीषोमं वैश्वदेवं धान्वन्तर्यमनन्तरम्}
{प्रजानां पतये चैव पृथग्घोमो विधीयते}


\twolineshloka
{तथैव चानुपूर्व्येण बलिकर्म प्रयोजयेत्}
{दक्षिणायां यमायेति प्रतीच्यां वरुणाय च}


\twolineshloka
{सोमाय चाप्युदीच्यां वै वास्तुमध्ये प्रजापतेः}
{धन्वन्तरेः प्रागुदीच्यां प्राच्यां शक्राय माधव}


\twolineshloka
{मनुष्येभ्य इति प्राहुर्बलिं द्वारि गृहस्य वै}
{मरुद्भ्यो दैवतेभ्यश्च बलिमन्तर्गृहे हरेत्}


\twolineshloka
{तथैव विश्वेदेवेभ्यो बलिमाकाशतो हरेत्}
{निशाचरेभ्यो भूतेभ्यो बलिं नक्तं यथा हरेत्}


\twolineshloka
{एवं कृत्वा बलिं सम्यग्दद्याद्भिक्षां द्विजाय वै}
{अलाभे ब्राह्मणस्याग्नावग्रमुद्धृत्य निक्षिपेत्}


\twolineshloka
{यदा श्राद्धं पितृभ्योपि दातुमिच्छेत मानवः}
{तदा पश्चात्प्रकुर्वीत निवृत्ते श्राद्धकर्मणि}


\twolineshloka
{पितॄन्सन्तर्पयित्वा तु बलिं कुर्याद्विधानतः}
{वेश्वदैवं ततः कुर्यात्पश्चाद्ब्राह्म्णभोजनम्}


\threelineshloka
{ततोऽन्नेनावशेषेण भोजयेदतिथीनपि}
{अर्घ्यपूर्वं महाराज ततः प्रीणाति मानवान्}
{अनित्यं हि स्थितो यस्मात्तस्मादतिथिरुच्यते}


\twolineshloka
{आचार्यस्य पितुश्चैव सख्युराप्तस्य चातिथेः}
{इदमस्ति गृहे मह्यमिति नित्यं निवेदयेत्}


\twolineshloka
{ते यद्वदेयुस्तत्कुर्यादिति धर्मो विधीयते}
{गृहस्थः पुरुषः कृष्ण शिष्टाशी च सदा भवेत्}


\twolineshloka
{राजर्त्विजं स्नातकं च गुरुं श्वशुरमेव च}
{अर्चयेन्मधुपर्केण परिसंवत्सरोषितान्}


% Check verse!
श्वभ्यश्च श्वपचेभ्यश्च वयोभ्यश्चावपेद्भुवि
% Check verse!
वैश्वदेवं हि नामैतत्सायं प्रातर्विधीयते
\threelineshloka
{एतांस्तु धर्मान्गार्हस्थ्यान्यः कुर्यादनसूयकः}
{स इहर्द्धिं परां प्राप्य प्रेत्य लोके महीयते ॥भीष्म उवाच}
{}


\twolineshloka
{इति भूमेर्वचः श्रुत्वा वासुदेवः प्रतापवान्}
{तथा चकार सततं त्वमप्येवं सदाऽऽचर}


\twolineshloka
{एतद्गृहस्थधर्मं त्वं चेष्टमानो जनाधिप}
{इह लोके यशः प्राप्य प्रेत्य स्वर्गमवाप्स्यसि}


\chapter{अध्यायः १५५}
\threelineshloka
{आलोकदानं नामैतत्कीदृशं भरतर्षभ}
{कथमेतत्समुत्पन्नं फलं वा तद्ब्रवीहि मे ॥भीष्म उवाच}
{}


\twolineshloka
{अत्राप्युदाहरन्तीममितिहासं पुरातनम्}
{मनोः प्रजापतेर्वादं सुवर्णस्य च भारत}


\twolineshloka
{तपस्वी कश्चिदभवत्सुवर्णो नाम भारत}
{वर्णतो हेमवर्णः स सुवर्ण इति विश्रुतः}


\twolineshloka
{कुलशीलगुणोपेतः स्वाध्यायोपरमं गतः}
{बहून्सुवंशप्रभवान्समतीतः स्वकैर्गुणैः}


\twolineshloka
{स कदाचिन्मनुं विप्रो ददर्शोपससर्प च}
{कुशलप्रश्नमन्योन्यं तौ चोभौ तत्र चक्रतुः}


\twolineshloka
{ततस्तौ सत्यसंकल्पौ मेरौ काञ्चनपर्वते}
{देवर्षिभिः सदा जुष्टे सहितौ संन्यषीदताम्}


\twolineshloka
{तत्र तौ कथयन्तौ स्तां कथा नानाविधाश्रयाः}
{ब्रह्मर्षिदेवदैत्यानां पुराणानां महात्मनाम्}


\twolineshloka
{सुवर्णस्त्वब्रवीद्वाक्यं मनुं स्वायंभुवं प्रति}
{हितार्थं सर्वभूतानां प्रश्नं मे वक्तुमर्हसि}


\threelineshloka
{सुमनोगन्धधूपाद्यैरिज्यन्ते दैवतानि च}
{किमेतत्कथमुत्पन्नं फलं योगं च शंस मे ॥मनुरुवाच}
{}


\twolineshloka
{अत्राप्युदाहरन्तीममितिहासं पुरातनम्}
{शुक्रस्य च बलेश्चैव संवादं वै महात्मनोः}


\twolineshloka
{बलेर्वैरोचनस्येह त्रैलोक्यमनुशासतः}
{समीपमाजगामाशु शुक्रो भृगुकुलोद्वहः}


\twolineshloka
{तमर्घ्यादिभिरभ्यर्च्य भार्गवं सोऽसुराधिपः}
{निषसादासने पश्चाद्विधिवद्भूरिदक्षिणः}


\twolineshloka
{कथेयमभवत्तत्र त्वया या परिकीर्तिता}
{सुमनोधूपदीपानां सम्प्रदाने फलं प्रति}


% Check verse!
ततः पप्रच्छ दैत्येन्द्रः कवीन्द्रं प्रश्नमुत्तमम्
\threelineshloka
{सुमनोधूपदीपानां किं फलं ब्रह्मिवित्तम}
{प्रदानस्य द्विजश्रेष्ठ तद्भवान्वक्तुमर्हति ॥शुक्र उवाच}
{}


\twolineshloka
{अग्नीषोमादिसृष्टौ तु विष्णोः सर्वात्मनः प्रभोः}
{तपः पूर्वं समुत्पन्नं धर्मस्तस्मादनन्तरम्}


\threelineshloka
{एतस्मिन्नन्तरे चैव वीरुदोषध्य एव च}
{सोमस्यात्मा च बहुधा सम्भूतः पृथिवीतले}
{अमृतं च विषं चैव याश्चान्यास्तृणजातयः}


\twolineshloka
{अमृतं मनसः प्रीतिं सद्यस्तृप्तिं ददाति च}
{मनो ग्लपयते तीव्रं विषं गन्धेन सर्वशः}


\twolineshloka
{अमृतं मङ्गलं विद्धि महद्विषममङ्गलम्}
{ओषध्यो ह्यमृतं सर्वा विषं तेजोग्निसम्भवम्}


\twolineshloka
{अमृतं मनो ह्लादयते श्रियं चापि ददाति च}
{तस्मात्सुमनसः प्रोक्ता नरैः सुकृतकर्मभिः}


\twolineshloka
{देवताभ्यः सुमनसो यो ददाति नरः शुचिः}
{तस्मै सुमनसो देवास्तस्मात्सुमनसः स्मृताः}


\twolineshloka
{यंयमुद्दिश्य दीयेरन्देवं सुमनसः प्रभो}
{मङ्गलार्तं स तेनास्य प्रीतो भवति दैत्यप}


\twolineshloka
{ज्ञेयास्तूग्राश्च सौम्याश्च तेजस्विन्यश्च ताः पृथक्}
{ओषध्यो बहुवीर्या हि बहुरूपास्तथैव च}


\twolineshloka
{यज्ञियानां च वृक्षाणामयज्ञीयान्निबोध मे}
{आसुराणि च माल्यानि दैवतेभ्यो हितानि च}


\twolineshloka
{रक्षसामुरगाणां च यक्षाणां च तथा प्रियाः}
{मनुष्याणां पितॄणां च कान्ता यास्त्वनुपूर्वशः}


\twolineshloka
{वन्या ग्राम्याश्चेह तथा कृष्टोप्ताः पर्वताश्रयाः}
{अकण्टकाः कण्टकिनो गन्धरूपरसान्विताः}


\twolineshloka
{द्विविधो हि स्मृतो गन्ध इष्टोऽनिष्टश्च पुष्पजः}
{इष्टगन्धानि देवानां पुष्पाणीति विभावय}


\twolineshloka
{अकण्टकानां वृक्षाणां श्वेतप्रायाश्च वर्णतः}
{तेषां पुष्पाणि देवानामिष्टानि सततं प्रभो}


\threelineshloka
{`पद्मं च तुलसी जातिरापः सर्वेषु पूजिता}
{'जलजानि च माल्यानि पद्मादीनि च यानि वै}
{गन्धर्वनागयक्षेभ्यस्तानि दद्याद्विचक्षणः}


\twolineshloka
{ओपध्यो रक्तपुष्पाश्च कटुकाः कण्टकान्विताः}
{शत्रूणामभिचारार्थमथर्वसु निदर्शिताः}


\twolineshloka
{तीक्ष्णवीर्यास्तु भूतानां दुरालम्भाः सकण्टकाः}
{रक्तभूयिष्ठवर्णाश्च कृष्णाश्चैवोपहारयेत्}


\twolineshloka
{मनोहृदयनन्दिन्यो विमर्दे मधुराश्च याः}
{चारुररूपाः सुमनसो मानुपाणां स्मृता विभो}


\twolineshloka
{न तु श्मशानसम्भूता न देवायतनोद्भवाः}
{सन्नयेत्पुष्टियुक्तेषु विवाहेषु रहःसु च}


\twolineshloka
{गिरिसानुरुहाः सौम्या देवानामुपधारयेत्}
{प्रोक्षिताभ्युक्षिताः सौम्या यथायोगं यथास्मृति}


\twolineshloka
{गन्धेन देवास्तुष्यन्ति दर्शनाद्यक्षराक्षसाः}
{नागाः समुपभोगेन त्रिभिरेतैस्तु मानुषाः}


\twolineshloka
{सद्यः प्रीणाति देवान्वै ते प्रीता भावयन्त्युत}
{सङ्कल्पसिद्धा मर्त्यानामीप्सिताश्च मनोरथाः}


\twolineshloka
{देवाः प्रीणन्ति सततं मानिता मानयन्ति च}
{अवज्ञातावधूताश्च निर्दहन्त्यधमान्नरान्}


\twolineshloka
{अत ऊर्ध्वं प्रवक्ष्यामि धूपदानविधेः फलम्}
{धूपांश्च विविधान्साधूनसाधूंश्च निबोध मे}


\twolineshloka
{निर्यासाः सरलाश्चैव कृत्रिमाश्चैव ते त्रयः}
{इष्टोऽनिष्टो भवेद्गन्धस्तन्मे विस्तरशः शृणु}


\twolineshloka
{निर्यासाः सल्लकीवर्ज्या देवानां दयितास्तु ते}
{गुग्गुलुः प्रवरस्तेषां सर्वेपामिति निश्चयः}


\twolineshloka
{अगुरुः सारिणां श्रेष्ठो यक्षराक्षसभोगिनाम्}
{दैत्यानां सल्लकीजश्च काङ्क्षितो यश्च तद्विधः}


\twolineshloka
{अथ सर्जरसादीनां गन्धैः पार्थिवदारवैः}
{फाणितासवसंयुक्तैर्मनुष्याणां विधीयते}


\twolineshloka
{देवदानवभूतानां सद्यस्तुष्ठिकरः स्मृतः}
{येऽन्ये वैहारिकास्तत्र मानुषाणामिति स्मृताः}


\twolineshloka
{य एवोक्ताः सुमनसां प्रदाने गुणहेतवः}
{धूपेष्वपि परिज्ञेयास्त एव प्रीतिवर्धनाः}


\twolineshloka
{दीपदाने प्रवक्ष्यामि फलयोगमनुत्तमम्}
{यथा येन यदा चैव प्रदेया यादृशाश्च ते}


\twolineshloka
{ज्योतिस्तेजः प्रकाशं वाऽप्यूध्वगं चापि वर्धते}
{प्रदानं तेजसां तस्मात्तेजो वर्धयते नृणाम्}


\twolineshloka
{अन्धतमस्तमिस्रं च दक्षिणायनमेव च}
{उत्तरायणमेतस्माज्ज्योतिर्दानं प्रशस्यते}


\twolineshloka
{यस्मादूर्ध्वगमे तत्तु तमसश्चैव भेषजम्}
{तस्मादूर्ध्वगतेर्दाता भवेदत्रेति निश्चयः}


\twolineshloka
{देवास्तेजस्विनो यस्मात्प्रभावन्तः प्रकाशकाः}
{तामसा राक्षसाश्चैव तस्माद्दीपः प्रदीयते}


\twolineshloka
{आलोकदानाच्चक्षुष्मान्प्रभायुक्तो भवेन्नरः}
{तान्दत्त्वा नोपहिंसेन न हरेन्नोपनाशयेत्}


\twolineshloka
{दीपहार्ता भवेदन्धस्तमोगतिरसुप्रभः}
{दीपप्रदः स्वर्गलोके दीपमाली विराजते}


\twolineshloka
{हविपा प्रथमः कल्पो द्वितीयश्चौषधीरसैः}
{वसामेदोस्थिनिर्यासैर्न कार्यः पुष्टिमिच्छता}


\twolineshloka
{देवालये सभायां च गिरौ चैत्यचतुष्पथे}
{दीपदाता भवेन्नित्यं य इच्छेद्भूतिमात्मनः}


\twolineshloka
{कुलोद्योतो विशुद्धात्मा प्रकाशत्वं च गच्छति}
{ज्योतिषां चैव सालोक्यं दीपदाता भवेन्नरः}


\twolineshloka
{बलिकर्मसु वक्ष्यामि गुणान्कर्मफलोदयान्}
{देवयक्षोरगनृणां भूतानामथ रक्षसाम्}


\twolineshloka
{येषां नाग्रभुजो विप्रा देवतातिथिबालकाः}
{राक्षसानेव तान्विद्धि निर्वषट्कारमङ्गलान्}


\twolineshloka
{तस्मादग्रं प्रयच्छेत देवभ्यः प्रतिपूजितम्}
{शिरसा प्रणतश्चापि हरेद्बलिमतन्द्रितः}


\twolineshloka
{गृह्णन्ति देवता नित्यमाशंसन्ति सदा गृहान्}
{बाह्याश्चागन्तवो येऽन्ये यक्षराक्षसपन्नगाः}


\twolineshloka
{इतो दत्तेन जीवन्ति देवताः पितरस्तथा}
{ते प्रीताः प्रीणयन्त्येनमायुषा यशसा धनैः}


\twolineshloka
{बलयः सह पुष्पैस्तु देवानामुपहारयेत्}
{दधिदुग्धमयाः पुण्याः सुगन्धाः प्रियदर्शनाः}


\twolineshloka
{कार्या रुधिरमांसाढ्या बलयो यक्षरक्षसाम्}
{सुरासवपुरस्कारा लाजोल्लापिकभूषिताः}


\twolineshloka
{नागानां दयिता नित्यं पद्मोत्पलविमिश्रिताः}
{तिलान्गुडसुसम्पन्नान्भूतानामुपहारयेत्}


\twolineshloka
{अग्रदाताऽग्रभोगी स्याद्बलवीर्यसमन्वितः}
{तस्मादग्रं प्रयच्छेत देवभ्यः प्रतिपूजितम्}


\twolineshloka
{ज्वलन्त्यहरहो वेश्म याश्चास्य गृहदेवताः}
{ताः पूज्या भूतिकामेन प्रसृताग्रप्रदायिना}


\twolineshloka
{इत्येतदसुरेन्द्राय काव्यः प्रोवाच भार्गवः}
{सुवर्णाय मनुः प्राह सुवर्णो नारदाय च}


\twolineshloka
{नारदोऽपि मयि प्राह गुणानेतान्महाद्युते}
{त्वमप्येतद्विदित्वेह सर्वमाचर पुत्रक}


\chapter{अध्यायः १५६}
\twolineshloka
{श्रुतं मे भरतश्रेष्ठ पुष्पधूपप्रदायिनाम्}
{फलं बलिविधाने च तद्भूयो वक्तुमर्हसि}


\threelineshloka
{धूपप्रदानस्य फलं प्रदीपस्य तथैव च}
{बलयश्च किमर्थं वै क्षिप्यन्ते गृहमेधिभिः ॥भीष्म उवाच}
{}


\twolineshloka
{अत्राप्युदाहरन्तीममितिहासं पुरातनम्}
{नहुषस्य च संवादमगस्त्यस्य भृगोस्तता}


\twolineshloka
{नहुषो हि महाराज राजर्षिः सुमहातपाः}
{देवराज्यमनुप्राप्तः सुकृतेनेह कर्मणा}


\twolineshloka
{तत्रापि प्रयतो राजन्नहुषस्त्रिदिवे वसन्}
{मानुषीश्चैव दिव्याश्च कुर्वाणो विविधाः क्रियाः}


\twolineshloka
{मानुष्यस्तत्र सर्वाः स्म क्रियास्तस्य महात्मनः}
{प्रवृत्तास्त्रिदिवे राजन्दिव्याश्चैव सनातनाः}


\twolineshloka
{अग्निकार्याणि समिधः कुशाः सुमनसस्तथा}
{बलयश्चान्नलाजाभिर्धूपनं दीपकर्म च}


\twolineshloka
{सर्वं तस्य गृहे राज्ञः प्रावर्तत महात्मनः}
{जपयज्ञान्मनोयज्ञांस्त्रिदिवेऽपि चकार सः}


\twolineshloka
{देवानभ्यर्चयच्चापि विधिवत्स सुरेश्वरः}
{सर्वानेव यथान्यायं यथापूर्वमरिंदम}


\twolineshloka
{अथेनद्रोऽहमिति ज्ञात्वा अहङ्कारं समाविशत्}
{सर्वाश्चैव क्रियास्तस्य पर्यहीयन्त भूपतेः}


\twolineshloka
{सप्तर्षीन्वाहयामास वरदानदमान्वितः}
{परिहीनक्रियश्चैव दुर्बलत्वमुपेयिवान्}


\twolineshloka
{तस्य वाहयतः कालो मुनिमुख्यांस्तपोधनान्}
{अहङ्काराभिभूतस्य सुमहानत्यवर्तत}


\twolineshloka
{अथ पर्यायशः सर्वान्वाहनायोपचक्रमे}
{पर्यायश्चाप्यगस्त्यस्य समपद्यत भारत}


\twolineshloka
{अथागत्य महातेजा भृगुर्ब्रह्मविदांवरः}
{अगस्त्यमाश्रमस्थं वै समुपेत्येदमब्रवीत्}


\threelineshloka
{एवं वयमसत्कारं देवेन्द्रस्यास्य दुर्मतेः}
{नहुषस्य किमर्थं वै मर्षयाम महामुने ॥अगस्त्य उवाच}
{}


\twolineshloka
{कथमेष मया शक्यः शप्तुं यस्य महामुने}
{वरदेन वरो दत्तो भवतो विदितश्च सः}


\twolineshloka
{यो मे दृष्टिपथं गच्छेत्स मे वश्यो भवेदिति}
{इत्यनेन वरं देवो याचितो गच्छता दिवम्}


\twolineshloka
{एवं न दग्धः स मया भवता च न संशयः}
{अन्येनाप्यृषिमुख्येन न दग्धो न च पातितः}


\twolineshloka
{अमृतं चैव पानाय दत्तमस्मै पुरा विभो}
{महात्मना तदर्तं च नास्माभिर्विनिपात्यते}


\twolineshloka
{प्रायच्छत वरं देवः प्रजानां दुःखकारणम्}
{द्विजेष्वधर्मयुक्तानि स करोति नराधमः}


\threelineshloka
{तत्र यत्प्राप्तकालं नस्तद्ब्रू वदतांवर}
{भवांश्चापि यथा ब्रूयात्तत्कर्तास्मि न संशयः ॥भृगुरुवाच}
{}


\twolineshloka
{पितामहनियोगेन भवन्तं सोऽहमागतः}
{प्रतिकर्तुं बलवति नहुषे दर्पमोहिते}


\twolineshloka
{अद्य हि त्वां सुदुर्बुद्धी रथे योक्ष्यति देवराट्}
{अद्यैनमहमुद्वृत्तं करिष्येऽनिन्द्रमोजसा}


\threelineshloka
{अद्येन्द्रं स्थापयिष्यामि पश्यतस्ते शतक्रतुम्}
{सञ्चाल्य पापकर्माणमैन्द्रात्स्थानात्सुदुर्मतिम्}
{}


\twolineshloka
{अद्य चासौ कुदेवेन्द्रस्त्वां पदा धर्षयिष्यति}
{दैवोपहतचित्तत्वादात्मनाशाय मन्दधीः}


\twolineshloka
{व्युत्क्रान्तधर्मं तमहं धर्षणामर्षितो भृशम्}
{अहिर्भवस्वेति रुषा शप्स्ये पापं द्विजद्रुहम्}


\twolineshloka
{तत एनं सुदुर्बुद्धिं धिक्शब्दाभिहतत्विषम्}
{धरण्यां पातयिष्यामि पश्यतस्ते महामुने}


\twolineshloka
{नहुषं पापकर्माणमैश्वर्यबलमोहितम्}
{यथा च रोचते तुभ्यं तथा कर्तास्म्यहं मुने}


\twolineshloka
{एवमुक्तस्तु भृगुणा मैत्रावरुणिरव्ययः}
{अगस्त्यः परमप्रीतो बभूव विगतज्वरः}


\chapter{अध्यायः १५७}
\threelineshloka
{कथं वै स विपन्नश्च कथं वै पातितो भुवि}
{कथं देवेन्द्रतां प्राप्तस्तद्भवान्वक्तुमर्हति ॥भीष्म उवाच}
{}


\twolineshloka
{एवं तयोः संवदतोः क्रियास्तस्य महात्मनः}
{सर्वा एव प्रवर्तन्ते या दिव्या याश्च मानुषाः}


\twolineshloka
{तथैव दीपदानानि सर्वोपकरणानि वै}
{बलिकर्म च यच्चान्यद्वत्सकाश्च पृथग्विधाः}


\twolineshloka
{सर्वास्तस्य समुत्पन्ना देवेन्द्रस्य महात्मनः}
{देवलोके नृलोके च सदाचारपुरस्कृताः}


\twolineshloka
{ताश्चोद्भवन्ति राजेन्द्र समृद्ध्यै गृहमेधिनः}
{धूपप्रदानैर्दीपैश्च नमस्कारैस्तथैव च}


\twolineshloka
{यथा सिद्धस्य चान्नस्य गृह्य चाग्रं प्रदीयते}
{बलयश्च गृहोद्देशे ततः प्रीयन्ति देवताः}


\twolineshloka
{यथा च गृहिणस्तोषो भवेद्वै बलिकर्मणि}
{तथा शतगुणा प्रीतिर्देवतानां प्रजायते}


\twolineshloka
{एवं धूपप्रदानं च दीपदानं च साधवः}
{प्रयच्छति नमस्कारैर्युक्तमात्मगुणावहम्}


\twolineshloka
{स्नानेनाद्भिश्च यत्कर्म क्रियते वै विपश्चिता}
{नमस्कारप्रयुक्तेन तेन प्रीणन्ति देवताः}


\twolineshloka
{पितरश्च महाभागा ऋषयश्च तपोधनाः}
{गृह्याश्च देवताः सर्वाः प्रीयन्ते विधिनाऽर्चिताः}


\twolineshloka
{इत्येतां बुद्धिमास्थाय नहुषः स नरेश्वरः}
{सुरेन्द्रत्वं महत्प्राप्य कृतवानेतदद्भुतम्}


\twolineshloka
{कस्य चित्त्वथ कालस्य भाग्यक्षय उपस्थिते}
{सर्वमेतदवज्ञाय न चकार यथाविधि}


\threelineshloka
{ततः स परिहीणोऽभूत्सुरेन्द्रो बलदर्पतः}
{धूपदीपादिकविधिं न यथावच्चकार ह}
{ततोऽस्य यज्ञविषयो रक्षोभिः परिबाध्यते}


\twolineshloka
{अथागस्त्यमृषिश्रेष्ठं वाहनायाजुहाव ह}
{द्रुतं सरस्वतीकूलात्स्मयन्निव महाबलः}


\threelineshloka
{ततो भृगुर्महातेजा मैत्रावरुणिमब्रवीत्}
{निमीलयस्व नयने जटां यावद्विशामि ते}
{`सुरेन्द्रपतनायेति स च नेत्र न्यमीलयत् ॥'}


\twolineshloka
{ततोऽगस्त्यस्याथ जटां दृष्ट्वा प्राविशदच्युतः}
{भृगुः स सुमहातेजाः पातनाय नृपस्य च}


\threelineshloka
{ततः स देवराट् प्राप्तस्तमृषिं वाहनाय वै}
{ततोऽगस्त्यः सुरपतिं वाक्यमाह विशाम्पते}
{योजयस्वेति मां क्षिप्रं कं च देशं वहामि ते}


\twolineshloka
{यत्र वक्ष्यसि तत्र त्वां नयिष्यामि सुराधिप}
{इत्युक्तो नरहुषस्तेन योजयामास तं मुनिम्}


\threelineshloka
{भृगुस्तस्य जटान्तस्थो बभूव हृषितो भृशम्}
{न चापि दर्शनं तस्य चकार स भृगुस्तदा}
{वरदानप्रभावज्ञो नहुषस्य महात्मनः}


\threelineshloka
{न चुकोप तदाऽगस्त्यो युक्तोऽपि नहुषेण वै}
{तं तु राजा पदैकेन चोदयामास भारत}
{`श्रुतिः स्मृतिः प्रमाणं वा नेतिवादेन देवराट् ॥'}


\twolineshloka
{न चुकोप स धर्मात्मा ततः पादेन देवराट्}
{अगस्त्यस्य तदा क्रुद्धो वामेनाभ्यहनच्छिरः}


\twolineshloka
{तस्मिञ्शिरस्यभिहते स जटान्तर्गतो भृगुः}
{शशाप बलवत्क्रुद्धो नहुषं पापचेतसम्}


\twolineshloka
{यस्मात्पदाऽवधीः क्रोधाच्छिरसीमं महामुनिम्}
{तस्मादाशु महीं गच्छ सर्पो भूत्वा सुदुर्मते}


\twolineshloka
{शप्तोऽथ स तदा तेन सर्पो भूत्वा पपात ह}
{अदृष्टेनाथ भृगुणा भूतले भरतर्षभ}


\twolineshloka
{भृगुं हि यदि सोऽद्राक्षीन्नहुषः पृथिवीपते}
{स शक्तोनाऽभविष्यद्वै पातने तस्य तेजसा}


\twolineshloka
{स तु तैस्तैः प्रदानैश्च तपोभिर्नियमैस्तथा}
{पतितोऽपि महाराज भूतले स्मृतिमानभूत्}


\fourlineindentedshloka
{प्रसादयामास भृगुं शापान्तो मे भवेदिति}
{ततोऽगस्त्यः कृपाविष्टः प्रासादयत तं भृगुम्}
{शापान्तार्थं महाराज स च प्रादात्कृपान्वितः ॥भृगुरुवाच}
{}


\twolineshloka
{राजा युधिष्ठिरो नाम भविष्यति कुरूद्वहः}
{स त्वां मोक्षयिता शापादित्युक्त्वाऽन्तरधीयत}


\twolineshloka
{अगस्त्योऽपि महातेजाः कृत्वा कार्यं शतक्रतोः}
{स्वमाश्रमपदं प्रायात्पूज्यमानो द्विजातिभिः}


\twolineshloka
{नहुषोऽपि त्वया राजंस्तस्माच्छापात्समुद्धृतः}
{जगाम ब्रह्मभवनं पश्यतस्ते जनाधिप}


\twolineshloka
{तदा स पातयित्वा तं नहुषं भूतले भृगुः}
{जगाम ब्रह्मभवनं ब्रह्मणे च न्यवेदयत्}


\threelineshloka
{ततः शक्रं समानाय्य देवानाह पितामहः}
{वरदानान्मम सुरा नहुषो राज्यमाप्तवान्}
{स चागस्त्येन क्रुद्धेन भ्रंशितो भूतलं गतः}


\twolineshloka
{न च शक्यं विना राज्ञा सुखं वर्तयितुं क्वचित्}
{तस्मादयं पुनः शक्रो देवराज्येऽभिषिच्यताम्}


\twolineshloka
{एवं सम्भाषमाणं तु देवाः पार्थ पितामहम्}
{एवमस्त्विति संहृष्टाः प्रत्यूचुस्तं नराधिप}


\twolineshloka
{सोऽभिषिक्तो भगवता देवराज्ये च वासवः}
{ब्रह्मणा राजशार्दूल यथापूर्वं व्यरोचत}


\twolineshloka
{एवमेतत्पुरावृत्तं नहुषस्य व्यतिक्रमात्}
{स च तैरेव संसिद्धो नहुषः कर्मभिः पुनः}


\threelineshloka
{तस्माद्दीपाः प्रदातव्याः सायं वै गृहमेधिभिः}
{दिव्यं चक्षुरवाप्नोति प्रेत्य दीपस्य दायकः}
{पूर्णचन्द्रप्रतीकाशा दीपदाश्च भवन्त्युत}


\twolineshloka
{यावदक्षिनिमेषाणि ज्वलन्ते तावतीः समाः}
{रूपवान्बलवांश्चापि नरो भवति दीपदः}


\twolineshloka
{य इदं शृणुयाद्वापि पठते यो द्विजोत्तमः}
{ब्रह्मलोकमवाप्नोति स च वै नात्र संशयः}


\chapter{अध्यायः १५८}
\threelineshloka
{ब्राह्मणस्वानि ये मन्दा हरन्ति भरतर्षभ}
{नृशंसकारिणो योनिं कां ते गच्छन्ति मानवाः ॥भीष्य उवाच}
{}


\twolineshloka
{पातकानां परं ह्येतद्ब्रह्मस्वहरणं बलात्}
{सान्वयास्ते विनश्यन्ति चण्डालाः प्रेत्य चेह च}


\threelineshloka
{अत्राप्युदाहरन्तीममितिहासं पुरातनमम्}
{चण्डालस्य च संवादं क्षत्रबन्धोश्च भारत ॥राजोवाच}
{}


\twolineshloka
{वृद्धरूपोऽसि चण्डाल बालवच्च विचेष्टसे}
{श्वखराणां रजःसेवी कस्मादुद्विजसे गवाम्}


\threelineshloka
{साधुभिर्गर्हितं कर्म चण्डालस्य विधीयते}
{कस्माद्गोरजसा ध्वस्तमङ्गं तोयेन सिञ्चसि ॥चण्डाल उवाच}
{}


\twolineshloka
{ब्राह्मणस्य गवां राजन्प्रयान्तीनां रजः पुरा}
{सोममुद्ध्वंसयामास तं सोममपिबन्द्विजाः}


\twolineshloka
{भृत्यानामपि राज्ञस्तु रजसा ध्वंसितं मखे}
{तत्पानाच्च द्विजाः सर्वे क्षिप्रं नरकमाविशन्}


\twolineshloka
{दीक्षितश्च स राजाऽपि क्षिप्रं नरकमाविशत्}
{सह तैर्याजकैः सर्वैर्ब्रह्मस्वमुपजीव्य तत्}


\twolineshloka
{येऽपि तत्रापिबन्क्षीरं घृतं दधि च मानवाः}
{ब्राह्मणाः सहराजन्याः सर्वे नरकमाविशन्}


\twolineshloka
{जघ्नुस्ताः पयसा पुत्रांस्तथा पौत्रान्विधूय तान्}
{पशूनवेक्षमाणश्च साधुवृत्तेन दंपती}


\twolineshloka
{अहं तत्रावसं राजन्ब्रह्मचारी जितेन्द्रियः}
{तासां मे रजसा ध्वस्तं भैक्षमासीन्नराधिप}


\twolineshloka
{चण्डालोऽहं ततो राजन्भुक्त्वा तदभवं नृप}
{ब्रह्मस्वहारी न नृपः सोऽप्रतिष्ठां गतिं ययौ}


\twolineshloka
{तस्माद्धरेन्न विप्रस्वं कदाचिदपि किञ्चन}
{न पश्येन्नानुमोदेच्च न हर्तुं किञ्चिदाहरेत्}


\twolineshloka
{ब्रह्मस्वं रजसा ध्वस्तं भुक्त्वा मां पश्य यादृशम्}
{तस्मात्सोमोऽप्यविक्रेयः पुरुषेण विपश्चिता}


\twolineshloka
{एतद्धि धनमुत्कृष्टं द्विजानामविशेषतः}
{विक्रयं त्विह सोमस्य गर्हयन्ति मनीषिणः}


\twolineshloka
{ये चैनं क्रीणते तात ये च विक्रीणते जनाः}
{ते तु वैवस्वतं प्राप्य रौरवं यान्ति सर्वशः}


\twolineshloka
{सोमं तु रजसा ध्वस्तं विक्रीणन्विधिपूर्वकम्}
{श्रोत्रियो वार्धुषी भूत्वा नचिरं स विनश्यति}


\threelineshloka
{नरकं त्रिंशतं प्राप्य स्वविष्ठामुपजीवति}
{ब्रह्मस्वहारी नरकान्यातनाश्चानुभूय तु}
{मलेषु च कृमिर्भूत्वा श्वविष्ठामुपजीवती}


\twolineshloka
{श्वचर्यामतिमानं च सखिदारेषु विप्लवम्}
{तुलयाधारयद्धर्मो ह्यतिमानोऽतिरिच्यते}


\twolineshloka
{म्लानं मां विकलं पश्य विवर्णं हरिणं कृशम्}
{अतिमानेनि मां पश्य पापां गतिमुपागतम्}


\twolineshloka
{अहं वै विपुले तात कुले दनसमन्विते}
{पौराणे जन्मनि विभो ज्ञानविज्ञानपारगः}


\twolineshloka
{अभवं तत्र जानानो ह्येतान्दोषान्मदात्सदा}
{संरब्ध एव भूतानां पृष्ठमांसमभक्षयम्}


\threelineshloka
{सोऽहं वै विदितः सर्वैर्दूरगो वनिनो वने}
{साधूनां परिभावेप्सुर्विप्राणां गर्वितो धनैः}
{इमामवस्थां सम्प्राप्तः पश्य कालस्य पर्ययम्}


\twolineshloka
{आदीप्तमिव चैलान्तं भ्रमरैरिव चार्दितम्}
{धावमानं सुसंरब्दं फश्य मां रजसाऽन्वितम्}


\twolineshloka
{स्वाध्यायैस्तु महात्पापं हरन्ति गृहमेधिनः}
{दानैः पृथग्विधैश्चापि विप्रजात्यां मनीषिणः}


\twolineshloka
{तथा पापकृतं विप्रमाश्रमस्तं महीपते}
{सर्वसङ्गविनिर्मुक्तं छन्दांस्युत्तारयन्त्युत}


\twolineshloka
{अहं हि पापयोन्यां वै प्रसूतः क्षत्रियर्षभ}
{निश्चयं नाधिगच्छामि कथं मुच्येयमित्युत}


\twolineshloka
{जातिस्मरत्वं च मम केनचित्पूर्वकर्मणा}
{शुभने येन मोक्षं वै प्राप्तुमिच्छाम्यहं नृप}


\threelineshloka
{त्वमिमं सम्प्रपन्नाय संशयं ब्रूहि पृच्छते}
{चण्डालत्वात्कथमहं मुच्येयमिति सत्तम ॥राजोवाच}
{}


\twolineshloka
{चण्डाल प्रतिजानीहि येन मोक्षमवाप्स्यसि}
{ब्राह्मणार्थे त्यजन्प्राणान्गतिमिष्टामवाप्स्यसि}


\threelineshloka
{दत्त्वा शरीरं क्रव्याद्भ्यो रणाग्नौ द्विजहेतुकम्}
{हित्वा प्राणान्प्रमोक्षस्ते नान्यथा मोक्षमर्हसि ॥भीष्म उवाच}
{}


\twolineshloka
{इत्युक्तः स तदा तेन ब्रह्मस्वार्थे परन्तप}
{हित्वा रणमुखे प्राणान्गतिमिष्टामवाप ह}


\twolineshloka
{तस्माद्रक्ष्यं त्वया पुत्र ब्रह्मस्वं भरतर्षभ}
{यदीच्छसि महाबाहो शाश्वतीं गतिमात्मनः}


\chapter{अध्यायः १५९}
\threelineshloka
{एको लोकस्तु कृतिनां स्वर्गे लोके पितामह}
{उत तत्रापि नानात्वं सर्वं ब्रूहि पितामह ॥भीष्म उवाच}
{}


\twolineshloka
{कर्मभिः पार्थ नानात्वं लोकानां यान्ति मानवाः}
{पुण्यान्पुण्यकृतो यान्ति पापान्पापकृतो नराः}


\twolineshloka
{अत्राप्युदाहरन्तीममितिहासं पुरातनम्}
{गौतमस्य मुनेस्तात संवादं वासवस्य च}


\twolineshloka
{ब्राह्मणो गौतमः कश्चिन्मृदुर्दान्तो जितेन्द्रियः}
{महावने हस्तिशिशुं परिद्यूनममातृकम्}


\twolineshloka
{तं दृष्ट्वा जीवयामास सानुक्रोशो धृतव्रतः}
{स तु दीर्घेण कालेनि बभूवातिबलो महान्}


\twolineshloka
{तं प्रपन्नं महादीर्घं प्रभूतं सर्वतोमदम्}
{धृतराष्ट्रस्वरूपेण शक्रो जग्राह हस्तिनम्}


\twolineshloka
{ह्रियमाणं तु तं दृष्ट्वा गौतमः संशितव्रतः}
{अभ्यभाषत राजानं धृतराष्ट्रं महातपाः}


\twolineshloka
{मामाहार्षीर्हस्तिनं पुत्रमेनंदुःखात्पुष्टं धृतराष्ट्रात्कृतज्ञ}
{मैत्रं सतां सप्तपदं वदन्तिमित्रद्रोहो नेह राजन्स्पृसेत्त्वाम्}


\twolineshloka
{इध्मोदकप्रदातारं शून्यपालं ममाश्रमे}
{विनीतमाचार्यकुले सुयुक्तं गुरुकर्मणि}


\threelineshloka
{शिष्टं दान्तं कृतज्ञं च प्रियं च सततं मम}
{न मे विक्रोशतो राजन्हर्तुमर्हसि कुञ्जरम् ॥धृतराष्ट्र उवाच}
{}


\threelineshloka
{गवां सहस्रं भवते ददानिशतानि निष्कस्य ददानि पञ्च}
{अन्यच्च वित्तं विविधं महर्षेकिं ब्राह्मणस्येह गजेन कृत्यम् ॥गौतम उवाच}
{}


\threelineshloka
{तवैव गावो हि भवन्तु राज-न्दास्यः सनिष्का विविधं च रत्नम्}
{अन्यच्च वित्तं विविधं नरेन्द्रकिं ब्राह्मणस्येह धनेन कृत्यम् ॥धृतराष्ट्र उवाच}
{}


\threelineshloka
{ब्राह्मणानां हस्तिभिर्नास्ति कृत्यंराजन्यानां नागकुलानि विप्र}
{स्वं वाहनं नयतो नास्त्यधर्मोनागश्रेष्ठं गौतमास्मान्निवर्त ॥गौतम उवाच}
{}


\threelineshloka
{यत्र प्रेतो नन्दति पुण्यकर्मायत्र प्रेतः शोचते पापकर्मा}
{वैवस्वतस्य सदने महात्मन-स्तत्र त्वाऽहं हस्तिनं यातयिष्ये ॥धृतराष्ट्र उवाच}
{}


\threelineshloka
{ये निष्क्रिया नास्तिकाः श्रद्धधानाःपापात्मान इन्द्रियार्थे निसृष्टाः}
{यमस्य ते यातनां प्राप्नुवन्तिपरं गन्ता धृतराष्ट्रो न तत्र ॥गौतम उवाच}
{}


\threelineshloka
{वैवस्वती संयमनी जनानांयत्रानृतं नोच्यते यत्र सत्यम्}
{यत्रांबलान्बलिनो घातयन्तितत्र त्वाऽहं हस्तिनं यातयिष्ये ॥धृतराष्ट्र उवाच}
{}


\threelineshloka
{ज्येष्ठां स्वसारं पितरं मातरं चगुरून्यथाऽमानयन्तश्चरन्ति}
{तथाविधानामेष लोको महर्षेपरं गन्ता धृतराष्ट्रो न तत्र ॥गौतम उवाच}
{}


\threelineshloka
{मन्दाकिनी वैश्रवणस् राज्ञोमहाभागा भोगिजनप्रवेश्या}
{गन्धर्वयक्षैरप्सरोभिश्च जुष्टातत्र त्वाऽहं हस्तिनं यातयिष्ये ॥धृतराष्ट्र उवाच}
{}


\threelineshloka
{अतिथिव्रताः सुव्रता ये जना वैप्रतिश्रयं ददति ब्राह्मणेभ्यः}
{शिष्टाशिनः संविभज्याश्रितेभ्योमन्दाकिनीं तेऽपि हि भूषयन्ति ॥गौतम उवाच}
{}


\threelineshloka
{मेरोरग्रे यद्वनं भाति रम्यंसुपुष्पितं किन्नरीगीतजुष्टम्}
{सुदर्शना यत्र जम्बूर्विशालातत्र त्वाऽहं हस्तिनं यातयिष्ये ॥धृतराष्ट्र उवाच}
{}


\twolineshloka
{ये ब्राह्मणा मृदवः सत्यशीलाबहुश्रुताः सर्वभूताभिरामाः}
{येऽधीयते सेतिहासं पुराणंमध्वाहुत्या जुह्वति वै द्विजेभ्यः}


\threelineshloka
{तथाविधानामेष लोको महर्षेपरं गन्ता धृतराष्ट्रो न तत्र}
{यदन्यत्ते विदितं स्थानमस्तितद्ब्रूहि त्वं त्वरितो ह्येष यामि ॥गौतम उवाच}
{}


\threelineshloka
{सुपुष्पितं किंनरराजजुष्टंप्रियं वनं नन्दनं नारदस्य}
{गन्धर्वाणामप्सरसां च सद्मतत्र त्वाऽहं हस्तिनं यातयिष्ये ॥धृतराष्ट्र उवाच}
{}


\threelineshloka
{ये नृत्यगीते कुशला जनाः सदादेवात्मानः प्रियकामाश्चरन्ति}
{तथाविधानामेष लोको महर्षेपरं गन्ता धृतराष्ट्रो न तत्र ॥गौतम उवाच}
{}


\twolineshloka
{यत्रोत्तराः कुरवो भान्ति रम्यादेवैः सार्धं मोदमाना नरेन्द्रः}
{यत्राग्नियौनाश्च वसन्ति विप्राअब्योनयः पर्वतयोनयश्च}


\threelineshloka
{यत्र शक्रो वर्षति सर्वकामा-न्यत्र स्त्रियः कामचारा भवन्ति}
{यत्र चेर्ष्या नास्ति नारीनराणांतत्र त्वाऽहं हस्तिनं यातयिष्ये ॥धृतराष्ट्र उवाच}
{}


\twolineshloka
{ये सर्वभूतेषु निवृत्तकामाअमांसादा न्यस्तदण्डाश्चरन्ति}
{न हिंसन्ति स्थावरं जङ्गमानिभूतानां ये सर्वभूतात्मभूताः}


\threelineshloka
{निराशिषो निर्ममा वीतरागालाभालाभे तुल्यनिन्दाप्रशंसाः}
{तथाविधानामेष लोको महर्षेपरं गन्ता धृतराष्ट्रो न तत्र ॥गौतम उवाच}
{}


\threelineshloka
{ततोऽपरे भान्ति लोकाः सनातनाःसुपुण्यगन्धा विरजा वीतशोकाः}
{सोमस्य राज्ञः सदने महात्मन-स्तत्र त्वाऽहं हस्तिनं यातयिष्ये ॥धृतराष्ट्र उवाच}
{}


\twolineshloka
{ये दानशीला न प्रतिगृह्णते सदान चाप्यर्थांश्चाददते परेभ्यः}
{येषामदेयमर्हते नास्ति किञ्चि-त्सर्वातिथ्याः सुप्रजना जनाश्च}


\threelineshloka
{ये क्षन्तारो नाभिजल्पन्ति चान्या-ञ्शक्ता भूत्वा सततं पुण्यशीलाः}
{तथाविधानामेष लोको महर्षेपरं गन्ता धृतराष्ट्रो न तत्र ॥गौतम उवाच}
{}


\threelineshloka
{ततोऽपरे भान्ति लोकाः सनातनाविराजसा वितमस्का विशोकाः}
{आदित्यदेवस्य पदं महात्मन-स्तत्र त्वाऽहं हस्तिनं यातयिष्ये ॥धृतराष्ट्र उवाच}
{}


\twolineshloka
{स्वाध्यायशीला गुरुशुश्रूषकाश्चतपस्विनः सुव्रताः सत्यसन्धाः}
{आचार्याणामप्रतिकूलभाषिणोनित्योत्थिता गुरुकर्मस्वचोद्याः}


\threelineshloka
{तथाविधानामेष लोको महर्षेविशुद्धानां भावितो वाग्यतानाम्}
{सत्ये स्थितानां वेदविदां महात्मनांपरं गन्ता धृतराष्ट्रो न तत्र ॥गौतम उवाच}
{}


\threelineshloka
{ततोऽपरे भान्ति लोकाः सनातनाःसुपुण्यगन्धा विरजा विशोकाः}
{वरुणस्य राज्ञः सदने महात्मन-स्तत्र त्वाऽहं हस्तिनं यातयिष्ये ॥धृतराष्ट्र उवाच}
{}


\twolineshloka
{चातुर्मास्यैर्ये यजन्ते जनाः सदातथेष्टीनां दशशतं प्राप्नुवन्ति}
{ये चाग्निहोत्रं जुह्वति श्रद्दधानायथान्यायं त्रीणि वर्षाणि विप्राः}


\threelineshloka
{स्वदारगाणां धर्मकृतां महात्मनांयथोदिते वर्त्मनि सुस्थितानाम्}
{धर्मान्मनामुद्वहतां गतिं तांपरं गन्ता धृतराष्ट्रो न तत्र ॥गौतम उवाच}
{}


\threelineshloka
{इन्द्रस्य लोका विरजा विशोकादुरन्वयाः काङ्क्षिता मानवानाम्}
{तस्याहं ते भवने भूरितेजसोराजन्निमं हस्तिनं यातयिष्ये ॥धृतराष्ट्र उवाच}
{}


\threelineshloka
{शतवर्षजीवी यश्च शूरो मनुष्योवेदाध्यायी यश्च यज्वाऽप्रमत्तः}
{एते सर्वे शक्रलोकं व्रजन्तिपरं गन्ता धृतराष्ट्रो न तत्र ॥गौतम उवाच}
{}


\threelineshloka
{प्राजापत्याः सन्ति लोका महान्तोनाकस्य पृष्ठे पुष्कला वीतशोकाः}
{मनीषिणां सर्वलोकाभयानांतत्र त्वाऽहं हस्तिनं यातयिष्ये ॥धृतराष्ट्र उवाच}
{}


\threelineshloka
{ये राजानो राजसूयाभिषिक्ताधर्मात्मानो रक्षितारः प्रजानाम्}
{ये चाश्वमेधावभृथे प्लुताङ्गा-स्तेषां लोका धृतराष्ट्रो न तत्र ॥गौतम उवाच}
{}


\threelineshloka
{ततःपरं भान्ति लोकाः सनातनाःसुपुण्यगन्धा विरजा वीतशोकाः}
{तस्मिन्नहं दुर्लभे चाप्यधृष्येगवां लोके हस्तिनं यातयिष्ये ॥धृतराष्ट्र उवाच}
{}


\twolineshloka
{यो गोसहस्री शतदो महात्मायो गोशती दश दद्याच्च शक्त्या}
{तथा दशभ्यो यश्च दद्यादिहैकांपञ्चभ्यो वा दानशीलस्तथैकाम्}


\twolineshloka
{ये जीर्यन्ते ब्रह्मचर्येणि विप्राब्राह्मीं वाचं परिरक्षन्ति चैव}
{मनस्विनस्तीर्थयात्रापरा येते तत्र मोदन्ति ततो विमानैः}


\twolineshloka
{प्रभासं मानसं तीर्थ पुष्कराणि महत्सरः}
{पुण्यं च नैमिषं तीर्थं बाहुदां करतोयिनीम्}


\twolineshloka
{गां हयशिरश्चैव विपाशां स्थूलवालुकाम्}
{तूष्णीं गङ्गां शनैर्गङ्गां महाह्रदमथापि च}


\twolineshloka
{गोमतीं कौशिकीं पंपां महात्मानो धृतव्रताः}
{सरस्वतीदृषद्वत्यौ यमुनां ये तु यान्ति च}


\threelineshloka
{तत ते दिव्यसंस्थाना दिव्यमाल्यधराः शिवाः}
{प्रयान्ति पुण्यगन्धाढ्या धृतराष्ट्रो न तत्र वै ॥गौतम उवाच}
{}


\twolineshloka
{यत्र शीतभयं नास्ति न चोष्णभयमण्वपि}
{न क्षुत्पिपासे न ग्लानिर्न दुःखं न सुखं तथा}


\threelineshloka
{न द्वेष्यो न प्रियः कश्चिन्न बन्धुर्न रिपुस्तथा}
{न जरामरणे तत्र न पुण्यं न च पातकम्}
{}


\threelineshloka
{तस्मिन्विरजसि स्फीते प्रज्ञासत्वव्यवस्थिते}
{स्वयम्भुभवने पुण्ये हस्तिनं यातयिष्यति ॥धृतराष्ट्र उवाच}
{}


\twolineshloka
{निर्मुक्ताः सर्वसङ्गैर्ये कृतात्मानो यतव्रताः}
{अध्यात्मयोगसंस्तानैर्युक्ताः स्वर्गगतिं गताः}


\threelineshloka
{ते ब्रह्मभवनं पुण्यं प्राप्नुवन्तीह सात्विकाः}
{न तत्र धृतराष्ट्रस्ते शक्यो द्रष्टुं महामुने ॥गौतम उवाच}
{}


\twolineshloka
{रथन्तरं यत्र बृहच्च गीयतेयत्र वेदिः पुण्यजनैर्वृता च}
{यत्रोपयाति हरिभिः सोमपीथीतत्र त्वाऽहं हस्तिनं यातयिष्ये}


\threelineshloka
{बुध्यामि त्वां वृत्रहणं शतक्रतुंव्यतिक्रमन्तं भुवनानि विश्वा}
{कच्चिन्न वाचा वृजिनं कदाचि-दकार्षं ते मनसोऽभिषङ्कात् ॥शतक्रतुरुवाच}
{}


\threelineshloka
{मघवाऽहं लोकपथं प्रजाना-मन्वागमं परिवादे गजस्य}
{तस्माद्भवान्प्रणतं माऽनुशास्तुब्रवीषि यत्तत्करवाणि सर्वम् ॥गौतम उवाच}
{}


\threelineshloka
{श्वेतं करेणुं मम पुत्रं हि नागंप्रियं तु मे षष्टिवर्षं तु बालम्}
{यो मे वने वसतोऽभूद्द्वितीय-स्तमेव मे देहि सुरेन्द्र नागम् ॥शतक्रतुरुवाच}
{}


\threelineshloka
{अयं सुतस्ते द्विजमुख्यनागआघ्रायते त्वामभिवीक्षमाणः}
{पादौ च ते नासिकयोपजिघ्रतेश्रेयो ममाध्याहि नमश्च तेऽस्तु ॥गौतम उवाच}
{}


\threelineshloka
{शिवं सदैवेह सुरेन्द्र तुभ्यंध्यायामि पूजां च सदा प्रयुञ्जे}
{ममापि त्वं शक्र शिवं ददस्वत्वया दत्तं प्रतिगृह्णामि नागम् ॥शतक्रतुरुवाच}
{}


\twolineshloka
{येषां वेदा निहिता वै गुहायांमनीषिणां सत्यवतां महात्मनाम्}
{तेषां त्वयैकेन महात्मनाऽस्मिबुद्धस्तस्मात्प्रीतिमांस्तेऽहतद्य}


\threelineshloka
{हन्तैहि ब्राह्मणि क्षिप्रं सह पुत्रेण हस्तिना}
{त्वं हि प्राप्तुं शुभाँल्लोकानह्नाय च चिराय च ॥भीष्म उवाच}
{}


\twolineshloka
{स गौतमं पुरुस्कृत्य सह पुत्रेण हस्तिना}
{दिवमाचक्रमे वज्री सद्भिः सह दुरासदम्}


\twolineshloka
{इदं यः शृणुयान्नित्यं यः पठेद्वा जितेन्द्रियः}
{स याति ब्रह्मणो लोकं ब्राह्मणो गौतमो यथा}


\chapter{अध्यायः १६०}
\twolineshloka
{दानं बहुविधाकारं शान्तिः सत्यमहिंसितम्}
{स्वदारतुष्टिश्चोक्ता ते फलं दानस्य वाऽपि यत्}


\threelineshloka
{पितामहस्य विदितं किमन्यत्तपसो बलात्}
{तपसो यत्परं तेऽद्य तन्नो व्याख्यातुमर्हसि ॥भीष्म उवाच}
{}


\twolineshloka
{तपसः प्रक्षयो यावत्ताल्लोको युधिष्ठिर}
{मतं ममात्र कौन्तेय तपो नानशनात्परम्}


\twolineshloka
{अत्राप्युदाहरन्तीममितिहासं पुरातनम्}
{भगीरथस्य संवादं ब्रह्मणश्च महात्मनः}


\twolineshloka
{अतीत्य सुरलोकं च गवां लोकं च भारत}
{ऋषिलोकं च सोऽगच्छद्भगीरथ इति श्रुतम्}


\twolineshloka
{तं तु दृष्ट्वा वचः प्राह ब्रह्मा राजन्भगीरथम्}
{कथं भगीरथाऽऽगास्त्वमिमं लोकं दुरासदम्}


\threelineshloka
{न हि देवा न गन्धर्वा न मनुष्या भगीरथ}
{आयान्त्यतप्ततपसः कथं वै त्वमिहागतः ॥भगीरथ उवाच}
{}


\twolineshloka
{निश्शङ्कमन्नमददां ब्राह्मणोभ्यःशतं सहस्राणि सदैवतानाम्}
{ब्राह्मं व्रतं नित्यमास्थाय विद्व-न्न त्वेवाहं तस्य फलादिहागाम्}


\twolineshloka
{दशैकरात्रान्दशपञ्चरात्रा-नेकादशैकादशकान्क्रतूंश्च}
{ज्योतिष्टोमानां च शतं यदिष्टंफलेन तेनापि च नागतोऽहम्}


\twolineshloka
{यच्चावसं जाह्नवीतीरनित्यःशतं समास्तप्यमानस्तपोऽहम्}
{अदां च तत्राश्वतरीसहस्रंनारीपुरं न च तेनाहमागाम्}


\twolineshloka
{दशायुतानि चाश्वानां गोऽयुतानि च विंशतिम्}
{पुष्करेषु द्विजातिभ्यः प्रादां शतसहस्रशः}


\twolineshloka
{सुवर्णचन्द्रोत्तमधारिणीनांकन्योत्तमानामददं सहस्रम्}
{षष्टिं सहस्राणि विभूषितानांजाम्बूनदैराभरणैर्न तेन}


\twolineshloka
{दशार्बुदान्यददं गोसवेज्या-स्वेकैकशो दश गा ओकनाथ}
{समानवत्साः पयसा समन्विताःसुवर्णकांस्योपदुहा न तेन}


\twolineshloka
{आप्तोर्यामेषु नियतमेकैकस्मिन्दशाददम्}
{गृष्टीनां क्षीरदोग्ध्रीणां रोहिणीनां शतानि च}


\twolineshloka
{दोग्ध्रीणां वै गवां चापि प्रयुतानि दशैव ह}
{प्रादां दशगुणं ब्रह्मन्न तेनाहमिहागतः}


\twolineshloka
{वाजिनां बाह्लिजातानामयुतान्यददं दश}
{कर्काणां हेममालानां न च तेनाहमागतः}


\twolineshloka
{कोटीश्च काञ्चनस्याष्टौ प्रादां ब्रह्मन्दशान्वहम्}
{एकैकस्मिन्क्रतौ तेन फलेनाहं न चागतः}


\twolineshloka
{वाजिनां श्यामकर्णानां हरितानां पितामह}
{प्रादां हेमस्रजां ब्रह्मन्कोटीर्दश च सप्त च}


\twolineshloka
{ईषादन्तान्महाकायान्काञ्चतस्रग्विभूषितान्}
{पद्मिनो वै सहस्राणि प्रादां दश च सप्त च}


\twolineshloka
{अलङ्कृतानां देवेश दिव्यैः कनकभूषणैः}
{रथानां काञ्चनाङ्गानां सहस्राण्यददं दश}


\twolineshloka
{सप्त चान्यानि युक्तानि वाजिभिः समलङ्कृतैः}
{दक्षिणावयवाः केचिद्वेदैर्ये सम्प्रकीर्तिताः}


\twolineshloka
{वाजपेयेषु दशसु प्रादां तेष्वपि चाप्यहम्}
{शक्रतुल्यप्रभावाणामिज्यया विक्रमेण ह}


\threelineshloka
{सहस्रं निष्ककण्ठानामददं दक्षिणामहम्}
{विजित्य भूपतीन्सर्वानर्थैरिष्ट्वा पितामह}
{अष्टभ्यो राजसूयेभ्यो न च तेनाहमागतः}


\twolineshloka
{स्रोतश्च यावद्गङ्गायाश्छन्नमासीज्जगत्पते}
{दक्षिणाभिः प्रवृत्ताभिर्मम नागां च तत्कृते}


\twolineshloka
{वाजिनां च सहस्रे द्वे सुवर्णशतभूषिते}
{वरं ग्रामशतं चाहमेकैकस्यां तिथावदाम्}


\twolineshloka
{तपस्वी नियताहारः शममास्थाय वाग्यतः}
{दीर्घकालं हिमवति गङ्गार्थमचरं तपः}


\twolineshloka
{मूर्ध्ना हरं महादेवं प्रणम्याभ्यर्चयन्नृपः}
{न तेनाप्यहमागच्छं फलेनेह पितामह}


\twolineshloka
{शम्याक्षेपैरयजं यच्च देवा-ञ्शतैः क्रतूनामयुतैश्चापि यच्च}
{त्रयोदशद्वादशाहैश्च देवसपौण्डरीकैर्न च तेषां फलेन}


\twolineshloka
{अष्टौ सहस्राणि ककुद्मिनामहंशुक्लर्षभाणामददं द्विजेभ्यः}
{एकैकं वै काञ्चनं शृङ्गमेभ्यःपत्नीश्चैषामददं निष्ककण्ठीः}


\twolineshloka
{हिरण्यरत्ननिचयानददं रत्नपर्वतान्}
{धनधान्यैः समृद्धांश्च ग्रामाञ्शतसहस्रशः}


\twolineshloka
{शतं शतानां गृष्टीनामददं चाप्यतन्द्रितः}
{इष्ट्वाऽनेकैर्महायज्ञैर्ब्राह्मणेभ्यो न तेन च}


\twolineshloka
{एकादशाहैरयजं सदक्षिणै-र्द्विर्द्वादशाहैरश्वमेधैश्च देव}
{गवां धनैः षोडशभिश्च ब्रह्मं-स्तेषां फलेनेह न चागतोऽस्मि}


\twolineshloka
{निष्कैककण्ठमददं योजनायतंतद्विस्तीर्णं काञ्चनपादपानाम्}
{वनं चूतानां रत्नविभूषितानांनचैव तेषामागतोऽहं फलेन}


\twolineshloka
{तुरायणं हि व्रतमप्यधृष्य-मक्रोधनोऽकरवं त्रिंशदब्दान्}
{शतं गवामष्टशतानि चैवदिनेदिने ह्यददं ब्राह्मणेभ्यः}


\twolineshloka
{पयस्विनीनामथ रोहिणीनांतथैवान्याननडुहो लोकनाथ}
{प्रादां नित्यं ब्राह्मणेभ्यः सुरेशनेहागतस्तेन फलेन चाहम्}


\threelineshloka
{`शम्याक्षेपेण पृथिवीं त्रिधा पर्यचरं यजन्}
{'त्रिंशदग्नीनहं ब्रह्मन्नयजं यच्च नित्यदा}
{अष्टाभिः सर्वमेधैश्च नरमेधैश्च सप्तभिः}


\twolineshloka
{दशभिर्विश्वजिद्भिश्च शतैरष्टादशोत्तरैः}
{न चैव तेषां देवेश फलेनाहमिहागमम्}


\twolineshloka
{सरय्वां बाहुदायां च गङ्गायामथ नैमिषे}
{गवां शतानामयुतमददं न च तेन वै}


\twolineshloka
{इन्द्रेण गुह्यं निहितं वै गुहायांयद्भार्गवस्तपसेहाभ्यविन्दत्}
{जाज्वल्यमानमुशनस्तेजसेहतत्साधयामासमहं वरेण्य}


\threelineshloka
{`ब्राह्मणार्थाय कर्माणि रणं चैव करोमि यत्}
{'ततो मे ब्राह्मणास्तुष्टास्तस्मिन्कर्मणि साधिते}
{`पूजितैर्ब्राह्मणैर्नित्यं न च तेनाहमागतः ॥'}


\twolineshloka
{सहस्रमृषयश्चासन्ये वै तत्र समागताः}
{उक्तस्तैरस्मि गच्छ त्वं ब्रह्मलोकमिति प्रभो}


\twolineshloka
{प्रीतेनोक्तः सहस्रेण ब्राह्मणानामहं प्रभो}
{इमं लोकमनुप्राप्तो मा भूत्तेऽत्र विचारणा}


\threelineshloka
{कामं यथावद्विहितं विधात्रापृष्टेन वाच्यं तु मया यथावत्}
{तपो हि नान्यच्चानशनान्मतं मेनमोस्तु ते देववर प्रसीद ॥भीष्म उवाच}
{}


\twolineshloka
{इत्युक्तवन्तं ब्रह्मि तु राजानं स भगीरथम्}
{पूजयामास पूजार्हं विधिदृष्टेन कर्मणा}


\twolineshloka
{तस्मादनशनैर्युक्तो विप्रान्पूज्य नित्यदा}
{विप्राणां वचनात्सर्वं परत्रेहं च सिध्यति}


\threelineshloka
{वासोभिरन्नैर्गोभिस्च शुभैर्नैवेशिकैरपि}
{शुभैः सुरक्षणैश्चापि स्तोष्या एव द्विजास्तथा}
{एतदेव परं गुह्यमलोभेन समाचर}


\chapter{अध्यायः १६१}
\twolineshloka
{शतायुरुक्तः पुरुषः शतवीर्यश्चि वैदिके}
{कस्मान्म्रियन्ते पुरुषा बाला अपि पितामह}


\twolineshloka
{आयुष्मान्केन भवति अल्पायुर्वाऽपि मानवः}
{केन वा लभते कीर्तिं केन वा लभते श्रियम्}


\threelineshloka
{तपसा ब्रह्मचर्येण जपैर्होमैस्तथा परैः}
{जन्मना यदि वाऽचारात्तन्मे ब्रूहि पितामह ॥भीष्म उवाच}
{}


\twolineshloka
{अत्र तेऽहं प्रवक्ष्यामि यन्मां त्वमनुपृच्छसि}
{अल्पायुर्येन भवति दीर्घायुर्येनि मानवः}


\twolineshloka
{येन वा लभते कीर्तिं येन वा लभते श्रियम्}
{यथा च वर्तयन्पुरुषः श्रेयसा सम्प्रयुज्यते}


\twolineshloka
{आचाराल्लभते ह्यायुराचाराल्लभते श्रियम्}
{आचारात्कीर्तिमाप्नोति पुरुषः प्रेत्य चेह च}


\twolineshloka
{दुराचारो हि पुरुषो नेहायुर्विन्दते महत्}
{त्रसन्ति चास्य भूतानि तथा परिभवन्ति च}


\twolineshloka
{तस्मात्कुर्यादिहाचारं य इच्छेद्भूतिमात्मनः}
{अपि पापशरीरस्य आचारो हन्त्यलक्षणम्}


\twolineshloka
{आचारलक्षणो धर्मः सन्तश्चारित्रलक्षणाः}
{साधूनां च यथावृत्तमेतदाचारलक्षणम्}


\twolineshloka
{अप्यदृष्टं श्रवादेव पुरुषं धर्मचारिणम्}
{स्वानि कर्माणि कुर्वाणं तं जनाः कुर्वते प्रियम्}


\twolineshloka
{ये नास्तिका निष्क्रियाश्च गुरुशास्त्रातिलङ्घिनः}
{अधर्मज्ञा दुराचारास्ते भवन्ति गतायुषः}


\twolineshloka
{विशीला भिन्नमर्यादा नित्यं सङ्कीर्णमैथुनाः}
{अल्पायुषो भवन्तीह नरा निरयगामिनः}


\twolineshloka
{सर्वलक्षणहीनोऽपि समुदाचारवान्नरः}
{श्रद्दधानोऽनसूयुश्च शतं वर्षाणि जीवति}


\twolineshloka
{अक्रोधनः सत्यवादी भूतानामविहिंसकः}
{अनसूयुरजिह्मश्च शतं वर्षाणि जीवति}


\twolineshloka
{लोष्ठमर्दी तृणच्छेदी नखखादी च यो नरः}
{नित्योच्छिष्टः सङ्कुसुको नेहायुर्विन्दते महत्}


\threelineshloka
{ब्राह्मे मुहूर्ते बुध्येत धर्मार्थौ चानुचिन्तयेत्}
{उत्थाय चोपतिष्ठेत पूर्वां सन्ध्यां कृताञ्जलिः}
{एवमेवापरां सन्ध्यां समुपासीत वाग्यतः}


\twolineshloka
{नेक्षेतादित्यमुद्यन्तं नास्तं यन्तं कदाचन}
{नोपसृष्टं न वारिस्थं न मध्यं नभसो गतम्}


% Check verse!
ऋषयो नित्यसन्ध्यत्वाद्दीर्घमायुरवाप्नुवन्
\twolineshloka
{`सदर्भपाणिस्तत्कुर्वन्वाग्यतस्तन्मनाः शुचिः}
{'तस्मात्तिष्ठेत्सदा पूर्वां पश्चिमां चैव वाग्यतः}


\twolineshloka
{ये न पूर्वामुपासन्ते द्विजाः संध्यां न पश्चिमाम्}
{सर्वांस्तान्धार्मिको राजा शूद्रकर्माणि कारयेत्}


\twolineshloka
{परदारा न गन्तव्याः सर्ववर्णेषु कर्हिचित्}
{नहीदृशमनायुष्यं लोके किञ्चन विद्यते}


\twolineshloka
{यादृशं पुरुषस्येह परदारोपसेवनम्}
{तादृशं विद्यते किञ्चिदनायुष्यं नृणामिह}


\twolineshloka
{यावन्तो रोमकूपाः स्युः स्त्रीणां गात्रेषु निर्मिताः}
{तावद्वर्षसहस्राणि नरकं पर्युपासते}


\twolineshloka
{मैत्रं प्रसाधनं स्नानमञ्जनं दन्तधावनम्}
{पूर्वाह्ण एव कुर्वीत देवतानां च पूजनम्}


\threelineshloka
{पुरीषमूत्रे नोदीक्षेन्नाधितिष्ठेत्कदाचन}
{नातिकल्यं नातिसायं न च मध्यंदिने स्थिते}
{नाज्ञातैः सह गच्छेत नैको न वृषलैः सह}


\twolineshloka
{पन्था देयो ब्राह्मणाय गोभ्यो राजभ्य एव च}
{वृद्धाय भारतप्ताय गर्भिण्यै दुर्बलाय च}


\twolineshloka
{प्रदक्षिणं च कुर्वीत परिज्ञातान्वनस्पतीन्}
{चतुष्पदान्मङ्गलांश्च मान्यान्वृद्धान्द्विजानपि}


\twolineshloka
{मध्यंदिने निशाकाले अर्धरात्रे च सर्वदा}
{चतुष्पथं न सेवेत उभे सन्ध्ये तथैव च}


\twolineshloka
{उपानहौ न वस्त्रं च धृतमन्यैर्न धारयेत्}
{ब्रह्मचारी च नित्यं स्यात्पादं पादेन नाक्रमेत्}


\twolineshloka
{अमावास्यां पौर्णमास्यां चतुर्दश्यां च जन्मनि}
{अष्टम्यामथ द्वादश्यां ब्रह्मचारी सदा भवेत्}


\twolineshloka
{वृथा मांसं न खादेनि पृष्ठमांसं तथैव च}
{आक्रोशं परिवादं च पैशुन्यं च विवर्जयेत्}


\twolineshloka
{नारुन्तुदः स्यान्न नृशंसवादीन हीनतो वरमभ्याददीत}
{ययाऽस्य वाचा पर उद्विजेत न तां वदेद्रुशतीं पापलोक्याम्}


\twolineshloka
{अतिवादबाणा मुखतो निःसरन्तियैराहतः शोचति रात्र्यहानि}
{परस्य वा मर्मसु ये पतन्तितान्पण्डितो नावसृजेत्परेषु}


\twolineshloka
{संरोहत्यग्निना दग्धं वनं परशुना हतम्}
{वाचा दुरुक्तं बीभत्सं न संरोहति वाक्क्षतम्}


\twolineshloka
{कर्णिनालीकनाराचान्निर्हरन्ति शरीरतः}
{वाक्शल्यस्तु न निर्हर्तुं शक्यो हृदिशयो हि सः}


\twolineshloka
{हीनाङ्गानतिरिक्ताङ्गान्विद्याहीनान्वयोधिकान्}
{रूपद्रविणहीनांश्च सत्यहीनांश्च नाक्षिपेत्}


\twolineshloka
{नास्तिक्यं वेदनिन्दां च परनिन्दां च कुत्सनम्}
{द्वेषस्तम्भाभिमानं च तैक्ष्ण्यं च परिवर्जयेत्}


\twolineshloka
{परस्य दण्डं नोद्यच्छेत्क्रुद्धो नैनं निपातयेत्}
{अन्यत्रि पुत्राच्छिष्याच्च शिक्षार्थं ताडनं स्मृतं}


\twolineshloka
{न ब्राह्म्णान्परिवदेन्नक्षत्राणि न निर्दिशेत्}
{तिथिं पक्षस्य न ब्रूयात्तथाऽस्यायुर्न रिष्यते}


\threelineshloka
{`अमावास्यामृते नित्यं दन्तधावनमाचरेत्}
{इतिहासपुराणानि दानं वेदं च नित्यशः}
{गायत्रीमननं नित्यं कुर्यात्सन्ध्यां समाहितः ॥'}


\twolineshloka
{कृत्वा मूत्रपुरीषे तु रथ्यामाक्रम्य वा पुनः}
{पादप्रक्षालनं कुर्यात्स्वाध्याये भोजने तथा}


\twolineshloka
{त्रीणि देवाः पवित्राणि ब्राह्मणानामकल्पयन्}
{अदृष्टमद्भिर्निर्णिक्तं यच्च वाचा प्रशस्यते}


\twolineshloka
{यावकं कृसरं मांसं शष्कुलीं पायसं तथा}
{आत्मार्थंतं न प्रकर्तव्यं देवार्थं तु प्रकल्पयेत्}


\twolineshloka
{नित्यमग्निं परिचरेद्भिक्षां दद्याच्च नित्यदा}
{दन्तकाष्ठे च सन्ध्यायां मलोत्सर्गे च मौनगः}


\threelineshloka
{न चाभ्युदितसायी स्यात्प्रायश्चित्ती तथा भवेत्}
{मातापितरमुत्थाय पूर्वमेवाभिवादयेत्}
{आचार्यमथवाऽप्यन्तं तथाऽऽयुर्विदन्ते महत्}


\threelineshloka
{वर्जयेद्दन्तकाष्ठानि वर्जनीयानि नित्यशः}
{भक्षयेच्छास्त्रदृष्टानि पर्वस्वपि विवर्जयेत्}
{उदङ्मुखश्च सततं शौचं कुर्यात्समाहितः}


\threelineshloka
{अकृत्वा देवपूजां च नाचरेद्दन्तधावनम्}
{अकृत्वा देवपूजां च नान्यं गच्छेत्कदाचन}
{अन्यत्र तु गुरुं वृद्धं धार्मिकं वा विचक्षणम्}


\twolineshloka
{अवलोक्यो न चादर्शो मलिनो बुद्धिमत्तरैः}
{न चाज्ञातां स्त्रियं गच्छेद्गर्भिणीं वा कदाचन}


\twolineshloka
{दारसङ्ग्रहणात्पूर्वं नाचरेन्मैथुनं बुधः}
{अन्यथा त्ववकीर्णः स्यात्प्रायश्चित्तं सदाऽऽचरेत्}


\twolineshloka
{नीदीक्षेत्परदारांश्च रहस्येकासनो भवेत्}
{इन्द्रियाणि सदा यच्छेत्स्वग्ने शुद्धमना भवेत्}


\twolineshloka
{उदक्शिरा न स्वपेत तथा प्रत्यक्शिरा न च}
{प्राक्शिरास्तु स्वपेद्विद्वांस्तथा वै दक्षिणाशिराः}


\twolineshloka
{न भग्ने नावशीर्णो च शयने प्रस्वपीत च}
{नान्तर्धानेन संयुक्ते न च तिर्यक्कदाचन}


\twolineshloka
{न चापि गच्छेत्कार्येण समयाद्वाऽपि नास्तिकैः}
{आसनं तु पदाऽऽकृष्य न प्रसज्जेत्तथा नरः}


\threelineshloka
{न नग्नः कर्हिचित्स्नायान्न निशायां कदाचन}
{स्नात्वा च नावमृज्येत गात्राणि सुविचक्षणः}
{न निशायां पुनः स्नायादापद्यग्निद्विजान्तिके}


\twolineshloka
{न चानुलिम्पेदस्नात्वा वासश्चापि न निर्धुनेत्}
{आर्द्र एव तु वासांसि नित्यं सेवेत मानवः}


\twolineshloka
{स्रजश्च नावकृष्येत न बहिर्धारयीत च}
{उदक्यया च सम्भाषां न कुर्वीत कदाचन}


\twolineshloka
{नोत्सृजेत पुरिषं च क्षेत्रे मार्गस्य चान्तिके}
{उभे मूत्रपुरीषे तु नाप्सु कुर्यात्कदाचन}


\threelineshloka
{देवालयेऽथ गोवृन्दे चैत्ये सस्येषु विश्रमे}
{भोक्ष्यं भुक्त्वा क्षुतेऽध्वानं गत्वा मूत्रपुरीषयोः}
{द्विराचामेद्यथान्यायं हृद्गतं तु पिबन्नपः}


\twolineshloka
{अन्नं बुभुक्षमाणस्तु त्रिमुखेन स्पृशेदपः}
{भुक्त्वा चान्नं तथैव त्रिर्द्विः पुनः परिमार्जयेत्}


\twolineshloka
{प्राङ्मुखो नित्यमश्नीयाद्वाग्यतोऽन्नमकुत्सयन्}
{प्रस्कन्दयेच्च मनसा भुक्त्वा चाग्निमुपस्पृशेत्}


\twolineshloka
{आयुष्यं प्राङ्मुखो भुङ्क्ते यशस्यं दक्षिणामुखः}
{धन्यं पश्चान्मुखो भुङ्क्त ऋतं भुङ्क्त उदङ्मुखः}


\twolineshloka
{`अग्रासनो जितक्रोधो बालपूर्वस्त्वलङ्कृतः}
{घृताहुतिविशुद्धान्नं हुताग्निश्च क्षिपन्ग्रसेत् ॥'}


\twolineshloka
{अग्निमालभ्य तोयेन सर्वान्प्राणानुपस्पृशेत्}
{गात्राणि चैव सर्वाणि नाभिं पाणितले तथा}


\twolineshloka
{नाधितिष्ठेत्तुषं जातु केशभस्मकपालिकाः}
{अन्यस्य चाप्यवस्नातं दूरतः परिवर्जयेत्}


\twolineshloka
{शान्तिहोमांश्च कुर्वीत सावित्राणि च धारयेत्}
{निषष्णश्चापि खादेन न तु गच्छन्कदाचन}


\twolineshloka
{मूत्रं नोत्तिष्ठता कार्यं न भस्मनि न गोव्रजे}
{आर्द्रपादस्तु भुञ्जीत नार्द्रपादस्तु संविशेत्}


\threelineshloka
{आर्द्रपादस्तु भुञ्जानो वर्षाणां जीवते शतम्}
{त्रीणि तेजांसि नोच्छिष्ट आलभेत कदाचन}
{अग्निं गां ब्राह्मणं चैव तता ह्यायुर्न रिष्यते}


\twolineshloka
{त्रीणि ज्योतींषि नोच्छिष्ट उदीक्षेत कदाचन}
{सूर्याचन्द्रमसौ चैव नक्षत्राणि च सर्वशः}


\twolineshloka
{ऊर्ध्वं प्राणा ह्युत्क्रमन्ति यूनः स्थविर आगते}
{प्रत्युत्थानाभिवादाभ्यां पुनस्तान्प्रतिपद्यते}


\twolineshloka
{अभिवादयीत वृद्धांश्च दद्याच्चैवासनं स्वयम्}
{कृताञ्जलिरुपासीत गच्छन्तं पृष्ठतोऽन्वियात्}


\twolineshloka
{न चासीतासने भिन्ने भिन्नकांस्यं च वर्जयेत्}
{नैकवस्त्रेण भोक्तव्यं न नग्नः स्नातुमर्हति}


\twolineshloka
{स्पप्तव्यं नैव नग्नेन न चोच्छिष्टोपि संविशेत्}
{उच्छिष्टो न स्पृशेच्छीर्षं सर्वे प्राणास्तदाश्रयाः}


\threelineshloka
{केशग्रहं प्रहारांश्च सिरस्येतान्विवर्जयेत्}
{न संहाताभ्यां पाणिभ्यां कण्डूयेदात्मनः शिरः}
{न चाभीक्ष्णं शिरःस्नायात्तथा स्यायुर्न रिष्यते}


\twolineshloka
{शिरःस्नातस्तु तैलैश्च नाङ्गं किञ्चिदपि स्पृशेत्}
{तिलपिष्टं न चाश्नीयाद्गतसर्वरसं तथा}


\twolineshloka
{नाध्यापयेत्तथोच्छिष्टो नाधीयीत कदाचन}
{वाते च पूतिगन्धे च मनसाऽपि न चिन्तयेत्}


\twolineshloka
{अत्र गाथा यमोद्गीताः कीर्तयन्ति पुराविदः}
{आयुरस्य निकृन्तामि प्रज्ञामस्याददे तथा}


\twolineshloka
{उच्छिष्टो यः प्राद्रवति स्वाध्यायं चाधिगच्छति}
{यश्चानध्यायकालेऽपि मोहादभ्यस्यति द्विजः}


\twolineshloka
{तस्य वेदः प्रणश्येत आयुश्चि परिहीयते}
{तस्माद्युक्तो ह्यनध्याये नाधीयीत कदाचन}


\twolineshloka
{प्रत्यादित्यं प्रत्यनलं प्रतिगां च प्रतिद्विजान्}
{ये मेहन्ति च पन्थानं ते भवन्ति गतायुषः}


\twolineshloka
{उभे मूत्रपुरीषे तु दिवा कुर्यादुदङ्मुखः}
{दक्षिणाभिमुखो रात्रौ तथा हन्युर्न रिष्यते}


\twolineshloka
{त्रीन्कृशान्नावजानीयाद्दीर्घमायुर्जिजीविषु}
{ब्राह्मणं क्षत्रियं सर्पं सर्वे ह्याशीविषास्त्रयः}


\twolineshloka
{दहत्याशीविषः क्रुद्धो यावत्पश्यति चक्षुषा}
{क्षत्रियोपि दहेत्क्रुद्धो यावत्स्पृशति तेजसा}


\twolineshloka
{ब्राह्म्णिस्तु कुलं हन्याद्ध्यानेनावेक्षितेन च}
{तस्मादेतत्त्रयं यत्नादुपसेवेत पण्डितः}


\twolineshloka
{गुरुणा वैरनिर्बन्धो न कर्तव्यः कदाचन}
{अनुमान्यः प्रसाद्यश्च गुरुः क्रुद्धो युदिष्ठिर}


\twolineshloka
{सम्यङ्मिथ्याप्रवृत्तेऽपि वर्तितव्यं गुराविह}
{गुरुनिन्दा दहत्यायुर्मनुष्याणां न संशयः}


\twolineshloka
{दूरादावसथान्मूत्रं दूरात्पादावसेचनम्}
{उच्छिष्टोत्सर्जनं चैव दूरे कार्यं हितैषिणा}


\twolineshloka
{रक्तमाल्यं न धार्यं स्याच्छुक्लं धार्यं तु पण्डितैः}
{वर्जयित्वा तु कमलं तथा कुवलयं प्रभो}


\twolineshloka
{रक्तं शिरसि धार्यं तु तथा वानेयमित्यपि}
{काञ्चनीयाऽपि माला या न सा दुष्यति कर्हिचित्}


% Check verse!
स्नातस्य वर्णकं नित्यमार्द्रं दद्याद्विशांपते
\twolineshloka
{विपर्ययं न कुर्वीत वाससो बुद्धिमान्नरः}
{तथा नान्यधृतं धार्यं न चातिविकृतं तथा}


\twolineshloka
{अन्यदेव भवेद्वासः शयनीये नरोत्तम}
{अन्यद्रथ्यासु देवानामर्चायामन्यदेव हि}


\twolineshloka
{प्रियङ्गुचन्दनाभ्यां च बिल्वेन स्थगरेण च}
{पृथगेवानुलिम्पेत केसरेण च बुद्धिमान्}


\twolineshloka
{उपवासं च कुर्वीत स्नातः शुचिरलङ्कृतः}
{`नोपवासं वृथा कुर्याद्धनं नापहरेदिह ॥'}


\twolineshloka
{पर्वकालेषु सर्वेषु ब्रह्मचारी सदा भवेत्}
{समानमेकपात्रे तु भुञ्जेन्नान्नं जनेश्वर}


\twolineshloka
{नावलीढमवज्ञातमाघ्रातुं भक्षयेदपि}
{तथा नोद्धृतसाराणि प्रेक्षतामप्रदाय च}


\twolineshloka
{नासंनिविष्टो मेधावी नाशुचिर्नि च सत्यु च}
{प्रतिषिद्धान्नं धर्मेषु भक्ष्यान्भुञ्जीत पृष्ठतः}


\twolineshloka
{पिप्पलं च वटं चैव शणशाकं तथैव च}
{उदुम्बरं न खादेश्च भवार्थी पुरुषो नृप}


\twolineshloka
{आजं गव्यं च यन्मांसं मायूरं चैव वर्जयेत्}
{वर्जयेच्छुष्कमांसं च तथा पर्युषितं च यत्}


\twolineshloka
{न पाणौ लवणं विद्वान्प्राश्नीयान्न च रात्रिषु}
{दधिसक्तून्न दोषायां पिबेन्मधु च नित्यशः}


\twolineshloka
{सायं प्रातश्च भुञ्जीत नान्तराले समाहितः}
{वालेन तु न भुञ्जीत परश्राद्धं तथैव च}


\twolineshloka
{वाग्यतो नैकवस्त्रश्च नासंविष्टः कदाचन}
{भूमौ सदैव नाश्नीयान्नाशौचं न च शब्दवत्}


\twolineshloka
{तोयपूर्वं प्रदायान्नमतिथिभ्यो विशाम्पते}
{पश्चाद्भुञ्जीत मेधावी न चाप्यन्यमना नरः}


\twolineshloka
{समानमेकपङ्क्त्यां तु भोज्यमन्नं नरेश्वर}
{विषं हालाहलं भुङ्क्ते योऽप्रदाय सुहृज्जने}


\twolineshloka
{पानीयं पायसं सक्तून्दधि सर्पिर्मधून्यपि}
{निरस्य शेषमन्येषां न प्रदेयं तु कस्यचित्}


\twolineshloka
{भुञ्जानो मनुजव्याघ्र नैव शङ्कां समाचरेत्}
{दधि चाप्यनु पानं वै न कर्तव्यं भवार्थिना}


\twolineshloka
{आचम्य चैव हस्तेनि परिस्राव्य तथोदकम्}
{अङ्गुष्ठं चरणस्याथ दक्षिणस्यावसेचयेत्}


\twolineshloka
{पाणिं मूर्ध्नि समाधाय स्पृष्ट्वा चाग्निं समाहितः}
{ज्ञातिश्रैष्ठ्यमवाप्नोति प्रयोगकुशलो नरः}


\twolineshloka
{अद्भिः प्राणान्समालभ्य नाभिं पाणितले तथा}
{स्पृशंश्चैव प्रतिष्ठेत न चाप्यार्द्रेण पाणिना}


\twolineshloka
{अङ्गुष्ठमूलस्य तले ब्राह्मं तीर्तमुदाहृतम्}
{कनिष्ठिकायाः पश्चात्तु देवतीर्थमिहोच्यते}


\twolineshloka
{अङ्गुष्ठस्य च यन्मध्यं प्रदेशिन्याश्च भारत}
{तेन पित्र्याणि कुर्वीत स्पृष्ट्वाऽऽपो न्यायतः सदा}


\twolineshloka
{परापवादं न ब्रूयान्नाप्रियं च कदाचन}
{न मन्युः कश्चिदुत्पाद्यः पुरुषेणि भवार्थिना}


\twolineshloka
{एतितैस्तु कथां नेच्छेद्दर्शनं च विवर्जयेत्}
{संसर्गं च न गच्छेत तथाऽऽयुर्विन्दते महत्}


\twolineshloka
{न दिवा मैथुनं गच्छेन्न कन्यां न च बन्धकीम्}
{न चास्नातां स्त्रियं गच्छेत्तथाऽऽयुर्विन्दते महत्}


\twolineshloka
{स्वेस्वे तीर्थे समाचम्य कार्ये समुपकल्पिते}
{त्रिः पीत्वाऽऽपो द्विः प्रमृज्य कृतशौचो भवेन्नरः}


\twolineshloka
{इन्द्रियाणि सकृत्स्पृश्य त्रिरभ्युक्ष्य च मानवः}
{कुर्वीत पित्र्यं दैवं च वेददृष्टेन कर्मणा}


\twolineshloka
{ब्राह्मणार्थे च यच्छौचं तच्च मे शृणु कौरव}
{प्रवृत्तं चाहितं चोक्त्वा बहुभोजनवत्तदा}


\twolineshloka
{सर्वशौचेषु ब्राह्मेण तीर्थेन समुपस्पृशेत्}
{निष्ठीव्य तु तथा क्षुत्वा स्पृश्यापो हि शुचिर्भवेत्}


\threelineshloka
{निष्ठिवने मैथुने च क्षुते अक्ष्याविमेचने}
{उदक्या दर्शने तद्वन्नग्नस्याचमनं स्मृतम्}
{स्पृशेत्कर्णं सप्रणवं सूर्यमीक्षेत्सदा तदा}


\threelineshloka
{वृद्धो ज्ञातिस्तथा मित्रमनाथा च स्वसा गुरुः}
{कुलीनः पण्डित इति रक्ष्या निःस्वः स्वशक्तितः}
{गृहे वासयितव्यास्ते धन्यमायुष्यमेव च}


\twolineshloka
{गृहे पारावता धार्याः शुकाश्च सहशारिकाः}
{`देवताप्रतिमाऽऽदर्शश्चन्दनाः पुष्पवल्लिकाः}


\twolineshloka
{शुद्धं जलं सुवर्णं च रजतं गृहमङ्गलम्}
{'गृहेष्वेते न पापाय यथा वै तैलपायिकाः}


\twolineshloka
{उद्दीपकाश्च गृध्रास्च कपोता भ्रमरास्तथा}
{निविशेयुर्यदा तत्र शान्तिमेव तदाऽऽचरेत्}


\threelineshloka
{अमङ्गल्यः सतां शापस्तथाऽऽक्रोशो महात्मनाम्}
{महात्मनोतिगुह्यानि न वक्तव्यानि कर्हिचित्}
{}


\twolineshloka
{अगम्याश्च न गच्छेत राज्ञः पत्नीं सखीस्तथा}
{वैद्यानां बालवृद्धानां भृत्यानां च युधिष्ठिर}


\twolineshloka
{बन्धूनां ब्राह्मणानां च तथा शारणिकस्य च}
{सम्बन्धिनां च राजेन्द्र तथाऽऽयुर्विन्दते महत्}


\twolineshloka
{ब्राह्मणस्तपतिभ्यां च निर्मितं यन्निवेशनम्}
{तदा वसेत्सदा प्राज्ञो भवार्थी मनुजेश्वर}


\twolineshloka
{सन्ध्यायां न स्वपेद्राजन्विद्यां न च समाचरेत्}
{न भुञ्जीत च मेधावी तथाऽऽयुर्विन्दते महत्}


\twolineshloka
{नक्तं न कुर्यात्पित्र्याणि नक्तं चैव प्रसाधनम्}
{पानीयस्य क्रिया नक्तं न कार्या भूतिमिच्छता}


\twolineshloka
{वर्जनीयाश्चैव नित्यं सक्तवो निशि भारत}
{शेषाणि चावदातानि पानीयं चापि भोजने}


\threelineshloka
{सौहित्यं न च कर्तव्यं रात्रौ न च समाचरेत्}
{`न भुक्त्वा मैथुनं गच्छेन्न धावेन्नातिहासकम्}
{'द्विजच्छेदं न कुर्वीत भुक्त्वा न च समाचरेत्}


\twolineshloka
{महाकुले प्रसूतां च प्रशस्तां लक्षणैस्तथा}
{वयसाऽवरां सुनक्षत्रां कन्यामावोढुमर्हति}


\twolineshloka
{अपत्यमुत्पाद्य ततः प्रतिष्ठाप्य कुलं तथा}
{पुत्राः प्रदेया जातेषु कुलधर्मेषु भारत}


\twolineshloka
{कन्या चोत्पाद्य दातव्या कुलपुत्राय धीमते}
{पुत्रा निवेश्याश्च कुलाद्भृत्या लभ्याश्च भारत}


% Check verse!
शिरःस्नातोथ कुर्वीत दैवं पित्र्यमथापि च
\threelineshloka
{`तैलं जन्मदिनेऽष्टम्यां चतुर्दश्यां च पर्वसु}
{'नक्षत्रे न च कुर्वीत यस्मिञ्जातो भवेन्नरः}
{न प्रोष्ठपदयोः कार्यं तथाऽऽग्नेये च भारत}


\twolineshloka
{दारुणेषु च सर्वेषु प्रत्यरिं च विवर्जयेत्}
{ज्योतिषे यानि चोक्तानि तानि सर्वाणि वर्जयेत्}


\twolineshloka
{प्राङ्मुखः श्मश्रुकर्माणि कारयेत्सुसमाहितः}
{उदङ्मुखो वा राजेन्द्र तथाऽऽयुर्विन्दते महत्}


\twolineshloka
{`सामुद्रेणाम्यसा स्नानं क्षौरं श्राद्धेषु भोजनम्}
{अन्तर्वत्नीपतिः कुर्वन्न पुत्रफलमश्नुते}


\threelineshloka
{सतां गुरूणां वृद्धानां कुलस्त्रीणां विशेषतः}
{'परिवादं न च ब्रूयात्परेषामात्मनस्तथा}
{परिवादो ह्यधर्माय प्रोच्यते भरतर्षभ}


\twolineshloka
{वर्जयेद्व्यङ्गिनीं नारीं तथा कन्यां नरोत्तम}
{समार्षां व्यङ्गिकां श्चैव मातुः सकुलजां तथा}


\threelineshloka
{वृद्धां प्रव्रजितां चैव तथैव च पतिव्रताम्}
{तथा निकृष्टवर्णा च वर्णोत्कृष्टां च वर्जयेत्}
{}


\twolineshloka
{अयोनिं च वियोनिं च न गच्छेत विचक्षणः}
{पिङ्गलां कुष्ठिनीं नारीं न त्वमुद्वोढुमर्हसि}


\twolineshloka
{अपस्मारिकुले जातां निहीनां चापि वर्जयेत्}
{श्वित्रिणां च कुले जातां क्षयिणां मनुजेश्वर}


\twolineshloka
{`सुरोमशामतिस्थूलां कन्यां मातृपितृस्थिताम्}
{अलज्जां भ्रातृजां तुष्टां वर्जयेद्रक्तकेशिनीम् ॥'}


\twolineshloka
{लक्षणैरन्विता या च प्रशस्ता या च लक्षणैः}
{मनोज्ञां दर्शनीयां च तां भवान्वोढुमर्हति}


\twolineshloka
{महाकुले निवेष्टव्यं सदृशे वा युधिष्ठिर}
{अवरा पतिता चैव न ग्राह्या भूतिमिच्छता}


\twolineshloka
{अग्नीनुत्पाद्य यत्नेन क्रियाः सुविहिताश्च याः}
{वेदे च ब्राह्मणैः प्रोक्तास्ताश्च सर्वाः समाचरेत्}


\twolineshloka
{न चेर्ष्या स्त्रीषु कर्तव्या रक्ष्या दाराश्च सर्वशः}
{अनायुष्या भवेदीर्ष्या तस्मादीर्ष्यां विवर्जयेत्}


\twolineshloka
{अनायुष्यं दिवा स्वप्नं तथाऽभ्युदितशायिता}
{प्रातर्निशायां च तथा ये चोच्छिष्टा भवन्ति च}


\twolineshloka
{पारदार्यमनायुष्यं नापितोच्छिष्टता तथा}
{यत्नतो वै न कर्तव्यमत्याशश्चैव भारत}


\twolineshloka
{सन्ध्यां न भुञ्ज्यान्न स्नायेन्न पुरीषं समुत्सृजेत्}
{प्रयतश्च भवेत्तस्यां न च किञ्चित्समाचरेत्}


\twolineshloka
{ब्राह्मणान्पूजयेच्चापि तथा स्नात्वा नराधिप}
{देवांश्च प्रणमेत्स्नातो गुरूंश्चाप्यभिवादयेत्}


\twolineshloka
{अनिमन्त्रितो न गच्छेत यज्ञं गच्छेत दर्शकः}
{अनर्चिते ह्यनायुष्यं गमनं तत्र भारत}


\twolineshloka
{न चैकेन परिव्रज्यं न गन्तव्यं तथा निशि}
{`नानापदि परस्यान्नमनिमन्त्रितमाहरेत्}


\twolineshloka
{एकोद्दिष्टं न भुञ्जीत प्रथमं तु विशेषतः}
{सपिण्डीकरणं वर्ज्यं सविधानं च मासिकम्}


% Check verse!
अनागतायां सन्ध्यायां पश्चिमायां गृहे वसेत्
\twolineshloka
{मातुः पितुर्गुरूणां च कार्यमेवानुशासनम्}
{हितं चाप्यहितं चापि न विचार्यं नरर्षभ}


\twolineshloka
{`क्षत्रियस्तु विशेषेण धनुर्वेदं समभ्यसेत्}
{'धनुर्वेदे च वेदे च यत्नः कार्यो नाराधिप}


\twolineshloka
{हस्तिपृष्ठेऽश्वपृष्ठे च रथचर्यासु चैव ह}
{यत्नवान्भव राजेन्द्र यत्नवान्सुवमेधते}


\twolineshloka
{अप्रधृष्यश्च शत्रूमां भृत्यानां स्वजनस्य च}
{प्रजापालनयुक्तश्च नारतिं लभते क्वचित्}


\twolineshloka
{युक्तिशास्त्रं च ते ज्ञेयं शब्दशास्त्रं च भारत}
{गान्धर्वशास्त्रं च कलाः परिज्ञेया नराधिप}


\twolineshloka
{पुराणमितिहासाश्च तथाऽऽख्यानानि यानि च}
{महात्मनां च चरितं श्रोतव्यं नित्यमेव ते}


\twolineshloka
{`मान्यानां माननं कुर्यान्निन्द्यानां निन्दनं तथा}
{गोब्राह्मणार्थे युध्येत प्राणानपि परित्यजेत्}


\threelineshloka
{न स्त्रीषु सज्जेद्द्रष्टव्यं शक्त्या दानरुचिर्भवेत्}
{न ब्राह्मणान्परिभवेत्कार्पण्यं ब्राह्मणैर्वृतम्}
{पतितान्नाभिभाषेत नाह्वयेत रजस्वलाम् ॥'}


\twolineshloka
{पत्नीं रजस्वलां चैव नाभिगच्छेन्न चाह्वयेत्}
{स्नातां चतुर्थे दिवसे रात्रौ गच्छेद्विचक्षणः}


\twolineshloka
{पञ्चमे दिवसे नारी षष्ठेऽहनि पुमान्भवेत्}
{`आषोडशादृतुर्मुख्यः पुत्रजन्मनि शब्दितः'}


\twolineshloka
{एतेन विधिना पत्नीमुपगच्छेत पण्डितः}
{ज्ञातिसम्बन्धिमित्राणि पूजनीयानि सर्वशः}


\twolineshloka
{यष्टव्यं च यथाशक्ति यज्ञैर्विविधदक्षिणैः}
{अत ऊर्ध्वमरण्यं च सेवितव्यं नराधिप}


\twolineshloka
{एष ते लक्षणोद्देश आयुष्याणां प्रकीर्तितः}
{शेषस्त्रैविद्यवृद्धेभ्यः प्रत्याहार्यो युधिष्ठिर}


\twolineshloka
{आचारो भूतिजनन आचारः कीर्तिवर्धनः}
{आचाराद्वर्धते ह्यायुराचारो हन्त्यलक्षणम्}


\twolineshloka
{आगमानां हि सर्वेषामाचारः श्रेष्ठ उच्यते}
{आचारप्रभवो धर्मो धर्मादायुर्विवर्धते}


\twolineshloka
{एतद्यशस्यमायुष्यं स्वर्ग्यं स्वस्त्ययनं महत्}
{अनुकम्प्य सर्ववर्णान्ब्रह्मणा समुदाहृतम्}


\twolineshloka
{य इदं शृणुयान्नित्यं यश्चापि परिकीर्तयेत्}
{स शुभान्प्राप्नुयाल्लोकान्सदाचारपरो नृप}


\chapter{अध्यायः १६२}
\threelineshloka
{कथं ज्येष्ठः कनिष्ठेषु वर्तेत भरतर्षभ}
{कनिष्ठाश्च यथा ज्येष्ठे वर्तेरंस्तद्ब्रवीहि मे ॥भीष्म उवाच}
{}


\twolineshloka
{ज्येष्ठवत्तात वर्तस्व ज्येष्ठोसि हि तथा भवान्}
{गुरोर्गरीयसी वृत्तिर्या च शिष्यस्य भारत}


\twolineshloka
{न गुरावकृतप्रज्ञे शक्यं शिष्येण वर्तितुम्}
{गुरौ हि सदृशी वृत्तिर्यथा शिष्यस्य भारत}


\twolineshloka
{अन्धः स्यादन्धवेलायां जडः स्यादपि वा बुधः}
{परिहारेण तद्ब्रूयाद्यस्तेषां स्याद्व्यतिक्रमः}


\twolineshloka
{प्रत्यक्षं भिन्नहृदया भेदयेयुर्यथाऽहिताः}
{`श्रियाऽभितप्तास्तद्भेदान्नभिन्नाः स्युः समाहिताः'श्रियाऽभितप्ताः कौन्तेय भेदकामास्तथाऽरयः}


\twolineshloka
{ज्येष्ठः कुलं वर्धयति विनाशयति वा पुनः}
{हन्ति सर्वमपि ज्येष्ठः प्रायो दुर्विनयादिह}


\twolineshloka
{अथ यो विनिकुर्वीत ज्येष्ठो भ्राता यवीयसः}
{अज्येष्ठः स्यादभागश्च नियम्यो राजभिश्च सः}


\twolineshloka
{निकृती हि नरो लोकान्पापान्गच्छत्यसंशयम्}
{विफलं तस्य पुत्रत्वं मोघं जनयितुः स्मृतम्}


\twolineshloka
{पित्रोरनर्थायक कुले जायते पापपूरुषः}
{अकीर्तिं जनयत्येव कीर्तिमन्तर्दधाति च}


\threelineshloka
{सर्वे चापि विकर्मस्था भागं नार्हन्ति सोदराः}
{ज्येष्ठोऽपि दुर्विनीतस्तु कनिष्ठस्तु विशेषतः}
{नाप्रदाय कनिष्ठेभ्यो ज्येष्ठः कुर्वीत वेतनम्}


\twolineshloka
{अनुजस्य पितुर्दायो जङ्घाश्रमफलोऽध्वगः}
{स्वयमीहेत लब्धं तु नाकामो दातुमर्हति}


\threelineshloka
{भ्रातॄणामविभक्तानामुत्थानमपि चेत्सह}
{न पुत्रभागं विषमं पिता दद्यात्कदाचन}
{}


\twolineshloka
{न ज्येष्ठो वाऽवमन्येत दुष्कृतः सुकृतोऽपि वा}
{गुरूणामपराधो हि शक्यः क्षन्तव्य एव च}


\twolineshloka
{यदि स्त्री यद्यवरज श्रेयः पश्येत्तदाचरेत्}
{धर्मार्थः श्रेय इत्याहुस्त्रयो ज्ञाता विधायकाः}


\threelineshloka
{दशाचार्यानुपाध्याय उपाध्यायान्पिता दश}
{दश चैव पितॄन्माता सर्वां वा पृथिवीमपि}
{गौरवेणाभिभवति नास्ति मातृसमो गुरुः}


\twolineshloka
{माता गरीयसी यच्च तेनैतां मन्यते गुरुम्}
{ज्येष्ठो भ्राता पितृसमो मृते पितरि भारत}


\twolineshloka
{स ह्येषां वृत्तिदाता स्यात्स चैतान्प्रतिपालयेत्}
{कनिष्ठास्तं नमस्येरन्सर्वे छन्दानुवर्तिनः}


\twolineshloka
{तमेव चोपजीवेरन्यथैव पितरं तथा}
{शरीरमेतौ सृजतः पिता मता च भारत}


\threelineshloka
{आचार्यशिष्टा या जातिः सा सत्या साऽजरामरा}
{ज्येष्ठा मातृसमा चापि भगिनी भरतर्षभ}
{}


% Check verse!
भ्रातुर्भार्या च तद्वत्स्याद्यस्या बाल्ये स्तनं पिबेत्
\chapter{अध्यायः १६३}
\twolineshloka
{सर्वेषामेव वर्णानां म्लेच्छानां च पितामह}
{उपवासे मतिरियं कारणं च न विद्महे}


\threelineshloka
{ब्रह्मक्षत्रेण नियमाः कर्तव्या इति नः श्रुतम्}
{उपवासे कथं तेषां कृतमस्ति पितामह}
{}


\twolineshloka
{नियमांश्चोपवासांश्च सर्वेषां ब्रूहि पार्थिव}
{आप्नोति कां गतिं तात उपवासपरायणः}


% Check verse!
उपवासः परं पुण्यं पवित्रमपि चोत्तमम् ॥उपोष्येह नरश्रेष्ठ किं फलं प्रतिपद्यते
\twolineshloka
{अधर्मान्मुच्यते केन धर्ममाप्नोति वा कथम्}
{स्वर्गं पुण्यं च लभते कथं भरतसत्तम}


\threelineshloka
{उपोष्य चापि किं तेन प्रयोज्यं स्यान्नराधिप}
{धर्मेण च सुखानर्थाँल्लभेद्येन ब्रवीहि मे ॥वैशम्पायन उवाच}
{}


\twolineshloka
{एवं ब्रुवाणं कौन्तेयं धर्मज्ञं धर्मतत्त्ववित्}
{धर्मपुत्रमिदं वाक्यं भीष्मः शान्तनवोऽब्रवीत्}


\twolineshloka
{इदं खलु महाराज श्रुतमासीत्पुरातनम्}
{उपावासविधौ श्रेष्ठा गुणा ये भरतर्षभ}


\twolineshloka
{प्राजापत्यमाङ्गिरसं पृष्टवानस्मि भारत}
{यथा मां त्वं तथैवाहं पृष्टवांस्तं तपोधनम्}


\threelineshloka
{प्रश्नमेतं मया पृष्टो भगवानग्निसम्भवः}
{उपवासविधिं पुण्यमाचष्ट भरतर्षभ ॥अङ्गिरा उवाच}
{}


\twolineshloka
{ब्रह्मक्षत्रे त्रिरात्रं तु विहितं कुरुनन्दन}
{द्विस्त्रिरात्रमथैकाहं निर्दिष्टं पुरुषर्षभ}


\twolineshloka
{वैश्याः शूद्राश्च यन्मोहादुपवासं प्रकुर्वते}
{त्रिरात्रं वा द्विराइत्रं वा तयोर्व्युष्टिर्न विद्यते}


\twolineshloka
{चतुर्थभक्तक्षपणं वैश्ये शूद्रे विधीयते}
{त्रिरात्रं न तु धर्मज्ञैर्विहितं ब्रह्मवादिभिः}


\twolineshloka
{पञ्चम्यां वाऽपि षष्ठ्यां च पौर्णमास्यां च भारत}
{उपोष्य एकभक्तेन नियतात्मा जितेन्द्रियः}


\twolineshloka
{क्षमावान्रूपसम्पन्नः सुरभिश्चैव जायते}
{नानपत्यो भवेत्प्राज्ञो दरिद्रो वा कदाचन}


\threelineshloka
{यजिष्णुः पञ्चमीं षष्ठीं कुले भोजयते द्विजान्}
{अष्टमीमथ कौरव्य कृष्णपक्षे चतुर्दशीम्}
{उपोष्य व्याधिरहितो वीर्यवानभिजायते}


\twolineshloka
{मार्गशीर्षं तु वै मासमेकभक्तेन यः क्षिपेत्}
{भोजयेच्च द्विजाञ्शक्त्या स मुच्येद्व्याधिकिल्बिषैः}


\twolineshloka
{सर्वकल्याणसम्पूर्णः सर्वौषधिसमन्वितः}
{कृषिभागी बहुधनो बहुधान्यश्च जायते}


\twolineshloka
{पौषमासं तु कौन्तेय भक्तेनैकेन यः क्षिपेत्}
{सुभगो दर्शनीयश्च यशोभागी च जायते}


\twolineshloka
{पितृभक्तो माघमासं यः क्षिपेदेकभोजनः}
{श्रीमत्कुले ज्ञातिमध्ये सुभगत्वं प्रपद्यते}


\threelineshloka
{भगदैवतमासं तु एकभक्तेन यः क्षिपेत्}
{`सुभगो दर्शनीयश्च यशोभागी च जायते}
{'स्त्रीषु वल्लभतां याति वश्याश्चास्य भवन्ति ताः}


\twolineshloka
{चैत्रं तु नियतो मासमेकभक्तेन यः क्षिपेत्}
{सुवर्णमणिमुक्ताढ्ये कुले महति जायते}


\twolineshloka
{निस्तरेदेकभक्तेन वैशाखं यो जितेन्द्रियः}
{नरो वा यदि वा नरी ज्ञातीनां श्रेष्ठतां व्रजेत्}


\twolineshloka
{ज्येष्ठामूलं तु यो मासमेकभक्तेन संक्षिपेत्}
{ऐश्वर्यमतुलं श्रेष्ठं पुमान्स्त्री वा प्रपद्यते}


\twolineshloka
{आषाढमेकभक्तेन स्थित्वा मासमतन्द्रितः}
{बहुधान्यो बहुधनो बहुपुत्रश्च जायते}


\threelineshloka
{श्रावणं नियतो मासमेकभक्तेन यः क्षिपेत्}
{रूपद्रविणसम्पन्नः सुखी भवति नित्यशः}
{बहुभार्यो बहुधनो बहुपुत्रश्चि जायते}


\twolineshloka
{प्रौष्ठपादं तु यो मासमेकाहारो भवेन्नरः}
{गवाढ्यं स्फीतमचलमैश्वर्यं प्रतिपद्यते}


\twolineshloka
{तथैवाश्वयुजं मासमेकभक्तेन यः क्षिपेत्}
{प्रज्ञावान्वाहनाढ्यश्च बहुपुत्रश्च जायते}


\twolineshloka
{कार्तिकं तु नरो मासं यः कुर्यादेकभोजनम्}
{शूरश्च बहुभार्यश्च कीर्तिमांश्चैव जायते}


\twolineshloka
{इति मासा नरव्याघ्र क्षिपतां परिकीर्तिताः}
{तिथीनां नियमा ये तु शृणु तानपि पार्थिव}


\twolineshloka
{पक्षेपक्षे गते यस्तु भक्तमश्नाति भारत}
{गवाढ्यो बहुपुत्रश्च दीर्घायुश्च स जायते}


\twolineshloka
{मासिमासि त्रिरात्राणि कृत्वा वर्षाणि द्वादश}
{गणाधिपत्यं प्राप्नोति निःसपत्नमनाविलम्}


\twolineshloka
{एते तु नियमाः सर्वे कर्तव्याः शरदो दश}
{द्वे चान्ये भरतश्रेष्ठ प्रवृत्तिरनुकीर्तिता}


\twolineshloka
{यस्तु प्रातस्तथा सायं भुञ्जानो नान्तरा पिबेत्}
{अहिंसानिरतो नित्यं जुह्वानो जातवेदसम्}


\twolineshloka
{षड्भिः स वर्षैर्नृपते सिध्यते नाइत्र संशयः}
{अग्निष्टोमस्य यज्ञस्य फलं प्राप्नोति मानवः}


\twolineshloka
{अधिवासे सोप्सरसां नृत्यगीतविनादिते}
{रमते स्त्रीसहस्राढ्ये सुकुती विरजा नरः}


\threelineshloka
{तप्तकाञ्चनवर्णाभं विमानमधिरोहति}
{पूर्णं वर्षसहस्रं च ब्रह्मिलोके महीयते}
{तत्क्षयादिह चागम्य माहात्म्यं प्रतिपद्यते}


\twolineshloka
{यस्तु संवत्सरं पूर्णमेकाहारो भवेन्नरः}
{अतिरात्रस्य यज्ञस्य स फलं समुपाश्नुते}


\twolineshloka
{त्रिंशद्वर्षसहस्राणि स्वर्गे च स महीयते}
{तत्क्षयादिह चागम्य माहात्म्यं प्रतिपद्यते}


\twolineshloka
{यस्तु संवत्सरं पूर्णं चतुर्थं भक्तमश्नुते}
{अहिंसानिरतो नित्यं सत्यवाग्विजितेन्द्रियः}


\twolineshloka
{वाजपेयस्य यज्ञस्य स फलं समुपाश्नुते}
{त्रिंशद्वर्षसहस्राणि स्वर्गलोके महीयते}


\twolineshloka
{षष्ठे काले तु कौन्तेय नरः संवत्सरं क्षिपन्}
{अश्वमेधस्य यज्ञस्य फलं प्राप्नोति मानवः}


\twolineshloka
{चक्रवाकप्रयुक्तेन विमानेन स गच्छति}
{चत्वारिंशत्सहस्राणि वर्षाणि दिवि मोदते}


\twolineshloka
{अष्टमेन तु भक्तेन जीवन्संवत्सरं नृप}
{गवामयनयज्ञस्य फलं प्राप्नोति मानवः}


\twolineshloka
{हंससारसयुक्तेन विमानेन स गच्छति}
{पञ्चाशतं सहस्राणि वर्षाणां दिवि मोदते}


\threelineshloka
{पक्षेपक्षे गते राजन्योऽश्नीयाद्वर्षमेव तु}
{षण्मासानशनं तस्य भगवानङ्गिराऽब्रवीत्}
{षष्टिं वर्षसहस्राणि दिवमावसते च सः}


\twolineshloka
{विणानां वल्लकीनां च वेणूनां च विशाम्पते}
{सुघोषैर्मधुरैः शब्दैः सुप्तः स प्रतिबोध्यते}


\twolineshloka
{संवत्सरमिहैकं तु मासिमासि पिबेदपः}
{फलं विश्वजितस्तात प्राप्नोति स नरो नृप}


\twolineshloka
{सिंहव्याघ्रप्रयुक्तेन विमानेन स गच्छति}
{सप्ततिं च सहस्राणि वर्षाणां दिवि मोदते}


\twolineshloka
{मासादूर्ध्वं नरव्याघ्र नोपवासो विधीयते}
{विधिं त्वनशनस्याहुः पार्थ धर्मविदो जनाः}


\twolineshloka
{अनार्तो व्याधिरहितो गच्छेदनशनं तु यः}
{पदेपदे यज्ञफलं स प्राप्नोति न संशयः}


\twolineshloka
{दिवं हंसप्रयुक्तेन विमानेन स गच्छति}
{शतं वर्षसहस्राणां मोदते स दिवि प्रभो}


\twolineshloka
{शतं चाप्सरसः कन्या रमयन्त्यपि तं नरम्}
{आर्तो वा व्याधितो वाऽपि गच्छेदनशनं तु यः}


\twolineshloka
{शतं वर्षसहस्राणां मोदते स दिवि प्रभो}
{काञ्चीनूपुरशब्देन सुप्तश्चैव प्रबोध्यते}


\twolineshloka
{सहस्रहंसयुक्तेन विमानेन तु गच्छति}
{स गत्वा स्त्रीशताकीर्णे रमते भरतर्षभ}


\twolineshloka
{क्षीणस्याप्यायनं दृष्टं क्षतस्य क्षतरोहणम्}
{व्याधितस्यौषधग्रामः क्रुद्धस्य च प्रसादनम्}


\twolineshloka
{दुःखितस्यार्तपूर्वस्य द्रव्याणां प्रतिपादनम्}
{न चैतद्रोचते तेषां ये धनैः सुखमेधिताः}


\twolineshloka
{अतः स कामसंयुक्ते विमाने हेमसन्निभे}
{रमते स्त्रीशताकीर्णे पुरुषोऽलङ्कृतः शुचिः}


\twolineshloka
{स्वस्थः सफलसङ्कल्पः सुखी विगतकल्मषः}
{अनश्नन्देहमुत्सृज्य फलं प्राप्नोति मानवः}


\twolineshloka
{बालसूर्यप्रतीकाशे विमाने सोमवर्चसि}
{वैदूर्यमुक्ताखचिते वीणामुरजनादिते}


\twolineshloka
{पताकादीपिकाकीर्णे दिव्यघण्टानिनादिते}
{स्त्रीसहस्रानुचरिते स नरः सुखमेधते}


\twolineshloka
{यावन्ति रोमकूपाणि तस्य गात्रेषु पाण्डव}
{तावन्त्येव सहस्राणि वर्षाणां दिवि मोदते}


\twolineshloka
{नास्ति वेदात्परं शास्त्रं नास्ति मातृसमो गुरुः}
{न धर्मात्परमो लाभस्तपो नानशनात्परम्}


\twolineshloka
{ब्राह्मणेभ्यः परं नास्ति पावनं दिवि चेह च}
{उपवासैस्तथा तुल्यं तपःकर्म न विद्यते}


\twolineshloka
{उपोष्य विधिवद्देवास्त्रिदिवं प्रतिपेदिरे}
{ऋषयश्च परां सिद्धिमुपवासैरवाप्नुवन्}


\twolineshloka
{दिव्यवर्षसहस्राणि विश्वामित्रेण धीमता}
{क्षान्तमेकेन भक्तेन तेन विप्रत्वमागतम्}


\twolineshloka
{च्यवनो जमदग्निश्च वसिष्ठो गौतमो भृगुः}
{सर्व एव दिवं प्राप्ताः क्षमावन्तो महर्षयः}


\twolineshloka
{इदमङ्गीरसा पूर्वं महर्षिभ्यः प्रदर्शितम्}
{यः प्रदर्शयते नित्यं न स दुःखमवाप्नुते}


\twolineshloka
{इमं तु कौन्तेय यथाक्रमं विधिंप्रवर्तितं ह्यङ्गिरसा महर्षिणा}
{पठेच्च यो वै शृणुयाच्च नित्यदान विद्यते तस्य नरस्य किल्बिषम्}


\twolineshloka
{विमुच्यते चापि स सर्वसङ्करै-र्न चास्य दोषैरभिभूयते मनः}
{वियोनिजानां च विजानते रुतंध्रुवां च कीर्तिं लभते नरोत्तमः}


\chapter{अध्यायः १६४}
\twolineshloka
{पितामहेन विधिवद्यज्ञाः प्रोक्ता महात्मना}
{गुणाश्चैषां यथातथ्यं प्रेत्य चेह च सर्वशः}


\twolineshloka
{न तै शक्या दरिद्रेण यज्ञाः प्राप्तुं पितामह}
{बहूपकरणा यज्ञा नानासम्भारविस्तराः}


\twolineshloka
{पार्थिवै राजपुत्रैर्वा शक्याः प्राप्तुं पितामह}
{नार्थन्यूनैरवगुणैरेकात्मभिरसंहतैः}


\fourlineindentedshloka
{यो दरिद्रैरपि विधिः शक्यः प्राप्तुं सदा भवेत्}
{अर्थन्यूनैरवगुणैरेकात्मभिरसंहतैः}
{तुल्यो यज्ञफलैरेतैस्तन्मे ब्रूहि पितामह ॥भीष्म उवाच}
{}


\twolineshloka
{इदमङ्गिरसा प्रोक्तमुपवासफलात्मकम्}
{विधिं यज्ञफलैस्तुल्यं तन्निबोध युधिष्ठिर}


\threelineshloka
{यस्तु कल्यं तथा सायं भुञ्जानो नान्तरा पिबेत्}
{अहिंसानिरतो नित्यं जुह्वानो जातवेदसम्}
{षड्भिरेव स वर्षैस्तु सिध्यते नात्र संशयः}


\threelineshloka
{तप्तकाञ्चनवर्णं च विमानं लभते नरः}
{देवस्त्रीणामधीवासे नृत्तगीतनिनादिते}
{प्राजापत्ये वसेत्पद्मं वर्षाणामग्निसन्निभे}


\threelineshloka
{त्रीणि वर्षाणि यः प्राशेत्सततं त्वेकभोजनम्}
{धर्मत्नीरतो नित्यमग्निष्टोमफलं लभेत्}
{यज्ञं बहुसुवर्णं वा वासवप्रियमाचरेत्}


\twolineshloka
{सत्यवान्दानशीलश्च ब्रह्मण्यश्चानसूयकः}
{क्षान्तो दान्तो जितक्रोधः स गच्छति परां गतिम्}


\twolineshloka
{पाण्डुराभ्रपतीकाशे विमाने हंसलक्षणे}
{द्वे समाप्ते ततः पद्मे सोप्सरोभिर्वसेत्सह}


\twolineshloka
{द्वितीये दिवसे यस्तु प्राश्नीयादेकभोजनम्}
{सदा द्वादशमासांस्तु जुह्वानो जातवेदसम्}


\twolineshloka
{अग्निकार्यपरो नित्यं नित्यं कल्यप्रबोधनः}
{अग्निष्टोमस्य यज्ञस्य फलं प्राप्नोति मानवः}


\twolineshloka
{हंससारसयुक्तं च विमानं लभते नरः}
{इन्द्रलोके च वसते वरस्त्रीभिः समावृतः}


\twolineshloka
{तृतीये दिवसे यस्तु प्राश्नीयादेकभोजनम्}
{सदा द्वादशमासांस्तु जुह्वानो जातवेदसम्}


\twolineshloka
{अग्निकार्यपरो नित्यं नित्यं कल्यप्रबोधनः}
{अतिरात्रस्य यज्ञस्य फलं प्राप्नोत्यनुत्तमम्}


\threelineshloka
{मयूरहंसयुक्तं च विमानं लभते नरः}
{सप्तर्षीणां सदा लोके सोप्सरोभिर्वसेत्सह}
{निवर्तनं च तत्रास्य त्रीणि पद्मानि वै विदुः}


\fourlineindentedshloka
{दिवसे यश्चतुर्थे तु प्राश्नीयादेकभोजनम्}
{सदा द्वादशमासान्वै जुह्वानो जातवेदसम्}
{वाजपेयस्य यज्ञस्य फलं प्राप्नीत्यनुत्तमम्}
{}


\threelineshloka
{इन्द्रिकन्याभिरूढं च विमान लभते नरः}
{सागरस्य च पर्यन्ते वासवं लोकमावसेत्}
{देवराजस्य च क्रीडां नित्यकालमवेक्षते}


\twolineshloka
{दिवसे पञ्चमे यस्तु प्राश्नीयादैकभोजनम्}
{सदा द्वादशमासांस्तु जुह्वानो जातवेदसम्}


\threelineshloka
{अलुब्धः}
{सत्यवादी च ब्रह्मण्यश्चाविहिंसकः}
{अनसूयुरपापस्थो द्वादशाहफलं लभेत्}


\twolineshloka
{जाम्बूनदमयं दिव्यं विमानं हंसलक्षणम्}
{सूर्यमालासमाभासमारोहेत्पाण्डुरं गुहम्}


\twolineshloka
{आवर्तनानि चत्वारि तथा पद्मानि द्वादश}
{शराग्निपरिमाणं च तत्रासौ वसते सुखम्}


\twolineshloka
{दिवसे यस्तु षष्ठे वै मुनिः प्राशेत भोजनम्}
{सदा द्वादशमासान्वै जुह्वानो जातवेदसम्}


\twolineshloka
{सदा त्रिषवणस्नायी ब्रह्मचार्यनसूयकः}
{गवामयनयज्ञस्य फलं प्राप्नोत्यनुत्तमम्}


\twolineshloka
{अग्निज्वालासमाभासं हंसबर्हिणसेवितम्}
{शातकुम्भसमायुक्तं साधयेद्यानमुत्तमम्}


\twolineshloka
{तथैवाप्सरसामङ्के प्रतिसुप्तः प्रबोध्यते}
{नूपुराणां निनादेन मेखलानां च निःस्वनैः}


\twolineshloka
{कोटीसहस्रं वर्षाणां त्रीणि कोटिशतानि च}
{पद्मान्यष्टादश तथा पताके द्वे तथैव च}


\twolineshloka
{अयुतानि च पञ्चशदृक्षचर्मशतस्य च}
{लोम्नां प्रमाणेन समं ब्रह्मलोके महीयते}


\twolineshloka
{दिवसे सप्तमे यस्तु प्राश्नीयादेकभोजनम्}
{सदा द्वादशमासान्वै जुह्वानो जातवेदसम्}


\twolineshloka
{सरस्वतीं गोपयानो ब्रह्मचर्यं समाचरन्}
{सुमनोवर्णकं चैव मधु मांसं च वर्जयन्}


\twolineshloka
{पुरुषो मरुतां लोकमिन्द्रलोकं च गच्छति}
{तत्रतत्र हि सिद्धार्थो देवकन्याभिरुह्यते}


\twolineshloka
{फलं बहुसुवर्णस्य यज्ञस्य लभते नरः}
{सङ्ख्यामतिगुणां चापि तेषु लोकेषु मोदते}


\twolineshloka
{यस्तु संवत्सरं क्षान्तो भुङ्क्तेऽहन्यष्टमे नरः}
{देवकार्यपरो नित्यं जुह्वानो जातवेदसम्}


\twolineshloka
{पौण्डरीकस्य यज्ञस्य फलं प्राप्नोत्यनुत्तमम्}
{पद्मवर्णनिभं चैव विमानमधिरोहति}


\twolineshloka
{मृष्टाः कनकगौर्यश्च नार्यः श्यामास्तथा पराः}
{वयोरूपविलासिन्यो लभते नात्र संशयः}


\twolineshloka
{यस्तु संवत्सरं भुङ्क्ते नवमेनवमेऽहनि}
{सदा द्वादशमासान्वै जुह्वानो जातवेदसम्}


\twolineshloka
{अश्वमेधसहस्रस्य फलं प्राप्नोत्यनुत्तमम्}
{पुण्डरीकप्रकाशं च विमानं लभते नरः}


\twolineshloka
{दीप्तसूर्याग्नितेजोभिर्दिव्यमालाभिरेव च}
{नीयते रुद्रकन्याभिः सोन्तरिक्षं सनातनम्}


\twolineshloka
{अष्टादशसहस्राणि वर्षाणां कल्पमेव च}
{कोटीशतसहस्रं च तेषु लोकेषु मोदते}


\twolineshloka
{यस्तु संवत्सरं भुङ्क्ते दशाहे वै गतेगते}
{सदा द्वादश मासान्वै जुह्वानो जातवेदसम्}


\twolineshloka
{ब्रह्मकन्यानिवासे स सर्वभूतमनोहरे}
{अश्वमेधसहस्रस्य फलं प्राप्नोत्यनुत्तमम्}


\twolineshloka
{रूपवत्यश्च तं कन्या रमयन्ति सदा नरम्}
{नीलोत्पलनिभैर्वर्णै रक्तोत्पलनिभैस्तथा}


\twolineshloka
{विमानं मण्डलावर्तमावर्तगहनाकुलम्}
{सागरोर्मिप्रतीकाशं स लभेद्यानमुत्तमम्}


\threelineshloka
{विचित्रमणिमालाभिर्नादितं शङ्खनिःस्वनैः}
{स्फाटिकैर्वज्रसारैश्च स्तम्भैः सुकृतवेदिकम्}
{आरोहति महद्यानं हंससारसवाहनम्}


\twolineshloka
{एकादशे तु दिवसे यः प्राप्ते प्राशते हविः}
{सदा द्वादशमासांस्तु जुह्वानो जातवेदसम्}


\threelineshloka
{परिस्त्रियं नाभिलषेद्वाचाथ मनसाऽपि वा}
{अनृतं च न भाषेत मातापित्रोःक कृतेऽपि वा}
{अभिगच्छेन्महादेवं विमानस्थं महाबलम्}


\twolineshloka
{अश्वमेधसहस्रस्य फलं प्राप्नोत्यनुत्तमम्}
{स्वायंभुवं च पश्येत विमानं समुपस्थितम्}


\twolineshloka
{कुमार्यः काञ्चनाभासा रूपवत्यो नयन्ति तम्}
{रुद्राणां तमधीवासं दिवि दिव्यं मनोहरम्}


\twolineshloka
{वर्षाण्यपरिमेयानि युगान्तमपि चावसेत्}
{कोटीशतसहस्रं च दशकोटिशतानि च}


\twolineshloka
{रुद्रं नित्यं प्रणमते देवदानवसम्मतम्}
{स तस्मै दर्शनं प्राप्तो दिवसेदिवसे भवेत्}


\twolineshloka
{दिवसे द्वादशे यस्तु प्राप्तो वै प्राशते हविः}
{सदा द्वादश मासान्वै जुह्वानो जातवेदसम्}


\twolineshloka
{नियमेन समायुक्तः सर्वमेधफलं लभेत}
{आदित्यैर्द्वादशैस्तस्य विमानं संविधीयते}


\twolineshloka
{मणिमुक्ताप्रवालैश्च महार्हैरुपशोभितम्}
{हंसभासा परिक्षिप्तं नागवीथीसमाकुलम्}


\twolineshloka
{मयूरैश्चक्रवाकैश्च कूजद्भिरुपशोभितम्}
{अट्टैर्महद्भिः संयुक्तं ब्रह्मलोके प्रतिष्ठितम्}


\twolineshloka
{नित्यमावसथं राजन्नरनारीसमावृतम्}
{ऋषिरेवं महाभागस्त्वङ्गिराः प्राह धर्मिवित्}


\twolineshloka
{त्रयोदशे तु दिवसे प्राप्ते यः प्राशते हविः}
{सदा द्वादश मासान्वै देवसत्रफलं लभेत्}


\twolineshloka
{रक्तपद्मोदयं नाम विमानं साधयेन्नरः}
{जातरूपप्रयुक्तं च रत्नसञ्चयभूषितम्}


\twolineshloka
{देवकन्याभिराकीर्णं दिव्याभरणभूषितम्}
{पुण्यगन्धोदयं दिव्यं वादित्रैरुपशोभितम्}


\twolineshloka
{तत्र शङ्कुपताके द्वे युगान्तं कल्पमेव च}
{अयुतायुतं तथा पद्मं समुद्रं च तथा वसेत्}


\twolineshloka
{गीतगन्धर्वघोषैश्च भेरीपणवनिःस्वनैः}
{सदा प्रह्लादितस्ताभिर्देवकन्याभिरीड्यते}


\twolineshloka
{चतुर्दशे तु दिवसे यः पूर्णे प्राशते हविः}
{सदा द्वादशमासांस्तु महामेधफलं लभेत्}


\twolineshloka
{अनिर्देश्यवयोरूपा देवकन्याः स्वलङ्कृताः}
{मृष्टतप्ताङ्गदधरा विमानैरुपयान्ति तम्}


\twolineshloka
{कलहंसविनिर्घोषैर्नूपुराणां च निःखनैः}
{काञ्चीनां च समुत्कर्षैस्तत्रतत्र निबोध्यते}


\twolineshloka
{देवकन्यानिवासे च तस्मिन्वसति मानवः}
{जाह्नवीवालुकाकीर्णं पूर्णं संवत्सरं नरः}


\twolineshloka
{यस्तु पक्षे गते भुङ्क्ते एकभक्तं जितेन्द्रियः}
{सदा द्वादशमासांस्तु जुह्वानो जातवेदसम्}


\twolineshloka
{राजसूयसहस्रस्य फलं प्राप्नोत्यनुत्तमम्}
{यानमारोहते दिव्यं हंसबर्हिणलक्षणम्}


\twolineshloka
{मणिमण्डलकैश्चित्रं जातरूपसमावृतम्}
{दिव्याभरणशोभाभिर्वरस्त्रीभिरलङ्कृतम्}


\threelineshloka
{एकस्तम्भं चतुर्द्वारं सप्तभौमं सुमङ्गलम्}
{वैजयन्तीसहस्रैश्च शोभितं गीतनिःस्वनै}
{दिव्यं दिव्यगुणोपेतं विमानमधिरोहति}


\twolineshloka
{मणिमुक्ताप्रवालैश्च भूषितं वैद्युतप्रभम्}
{वसेद्युगसहस्रं च खड्गकुञ्जरवाहनः}


\twolineshloka
{षोडशे दिवसे यस्तु सम्प्राप्ते प्राशते हविः}
{सदा द्वादशमासान्वै सोमयज्ञफलं लभेत्}


\twolineshloka
{सोमकन्यानिवासेषु सोऽध्यावसति नित्यशः}
{सौम्यगन्धानुलिप्तश्च कामचारगतिर्भवेत्}


\twolineshloka
{सुदर्शनाभिर्नारीभिर्मधुराभिस्तथैव च}
{अर्च्यते वै विमानस्थः कामभोगैश्च सेव्यते}


\twolineshloka
{फलं पद्मशतप्रख्यं महाकल्पं दशाधिकम्}
{आवर्तनानि चत्वारि साधयेच्चाप्यसौ नरः}


\twolineshloka
{दिवसे सप्तदशमे यः प्राप्ते प्राशते हविः}
{सदा द्वादशमासान्वै जुह्वानो जातवेदसम्}


% Check verse!
स्थानं वारुणमैन्द्रं च रौद्रं वाऽप्यधिगच्छति
\twolineshloka
{मारुतं शयनं चैव ब्रह्मलोकं स गच्छति}
{तत्र दैवतकन्याभिरासनेनोपचर्यते}


\threelineshloka
{भूर्भुवःस्वश्च देवर्षिर्विश्वरूपमवेक्षते}
{तत्र देवाधिदेवस्य कुमार्यो रमयन्ति तम्}
{द्वात्रिंशद्रूपदारिण्यो मधुराः समलङ्कृताः}


\twolineshloka
{चन्द्रादित्यावुभौ यावद्गगते चरतः प्रभो}
{तावच्चरत्यसौ धीरः सुधातुल्यरसाशनः}


\twolineshloka
{अष्टादशे यो दिवसे प्राश्नीयादेकभोजनम्}
{सदा द्वादशमासान्वै सप्तलोकान्स पश्यति}


\twolineshloka
{रथैः स नन्दिघोषैश्च पृष्ठतः सोऽनुगम्यते}
{देवकन्याधिरूढैस्तु भ्राजमानैः स्वलंकृतैः}


\twolineshloka
{व्याघ्रसिंहप्रयुक्तं च मेघस्वननिनादितम्}
{विमानमुत्तमं दिव्यं सुसुखि ह्यधिरोहति}


\twolineshloka
{तत्र कल्पसहस्रं स कन्याभिः सह मोदते}
{सुधारसं च भुञ्जीत अमृतोपममुत्तमम्}


\twolineshloka
{एकोनविंशदिवसे यो भुङ्क्ते एकभोजनम्}
{सदा द्वादशमासान्वै सप्त लोकान्स पश्यति}


\twolineshloka
{उत्तमं लभते स्थानमप्सरोगणसेवितम्}
{गन्धर्वैरुपगीतं च विमानं सूर्यवर्चसम्}


\twolineshloka
{तत्रामरवरस्त्रीभिर्मोदते विगतज्वरः}
{दिव्याम्बरधरः श्रीमानयुतानां शतंशतम्}


\twolineshloka
{पूर्णेऽथ विंशे दिवसे यो भुङ्क्ते ह्येकभोजनम्}
{सदा द्वादशमासांस्तु सत्यवादी धृतव्रतः}


\twolineshloka
{अमांसाशी ब्रह्मचारी सर्वभूतहिते रतः}
{स लोकान्विपुलान्रम्यानादित्यानामुपाश्नुते}


\twolineshloka
{गन्धर्वैरप्सरोभिश्च दिव्यमाल्यानुलेपनैः}
{विमानैः काञ्चनैर्हृद्यैः पृष्ठतश्चानुगम्यते}


\threelineshloka
{एकविंशो तु दिवसे यो भुङ्क्ते ह्येकभोजनम्}
{सदा द्वादशमासान्वै जुह्वानो जातवेदसम्}
{लोकमौशनसं दिव्यं शक्रलोकं च गच्छति}


\threelineshloka
{अश्विनोर्मरुतां चैव सुखेष्वभिरतः सदा}
{अनभिज्ञश्च दुःखानां विमानवरमास्थितः}
{सेव्यमानो वरस्त्रीभिः क्रीडत्यमरवत्प्रभुः}


\twolineshloka
{द्वाविंशे दिवसे प्राप्ते यो भुङ्क्ते ह्येकभोजनम्}
{सदा द्वादश मासान्वै जुह्वानो जातवेदसम्}


\twolineshloka
{अहिंसानिरतो धीमान्सत्यवागनसूयकः}
{लोकान्वसूनामाप्नोति दिवाकरसमप्रभः}


\twolineshloka
{कामचारी सुधाहारो विमानवरमास्थितः}
{रमते देवकन्याभिर्दिव्याभरणभूषितः}


\threelineshloka
{त्रयोविंशे तु दिवसे प्राशेद्यस्त्वेकभोजनम्}
{सदा द्वादशमासांस्तु मिताहारो जितेन्द्रियः}
{वायोरुशनसस्चैव रुद्रलोकं च गच्छति}


\threelineshloka
{कामचारी कामगमः पूज्यमानोऽप्सरोगणैः}
{अनेकयुगपर्यन्तं विमानवरमास्थितः}
{रमते देवकन्याभिर्दिव्याभरणभूषितः}


\threelineshloka
{चतुर्विंशे तु दिवसे यः प्राप्ते प्राशते हविः}
{सदा द्वादशमासांश्च जुह्वानो जातवेदसम्}
{आदित्यानामधीवासे मोदमानो वसेच्चिरम्}


\threelineshloka
{दिव्यमाल्याम्बरधरो दिव्यगन्धानुलेपनः}
{विमाने काञ्चने दिव्ये हंसयुक्ते मनोरमे}
{रमते देवकन्यानां सहस्रैरयुतैस्तथा}


\twolineshloka
{पञ्चविंशे तु दिवसे यः प्राशेदेकभोजनम्}
{सदा द्वादशमासांस्तु पुष्कलं यानमारुहेत्}


\threelineshloka
{सिंहव्याग्रप्रयुक्तैस्तु मेघनिःस्वननादितैः}
{स रथैर्नन्दिघोषैश्च पृष्ठतो ह्यनुगम्यते}
{देवकन्यासमारूढैः काञ्चनैर्विमलैः शुभैः}


\threelineshloka
{विमानमुत्तमं दिव्यमास्थाय सुमनोहरम्}
{तत्र कल्पसहस्रं वै वसते स्त्रीशतावृते}
{सुधारसं चोपजीवन्नमृतोपममुत्तमम्}


\threelineshloka
{षड्विंशे दिवसे यस्तु प्रकुर्यादेकभोजनम्}
{सदा द्वादशमासांस्तु नियतो नियताशनः}
{जितेन्द्रियो वीतरागो जुह्वानो जातवेदसम्}


\twolineshloka
{स प्राप्नोति महाभागः पूज्यमानोऽप्सरोगणैः}
{सप्तानां मरुतां लोकान्वसूनां चापि सोश्नुते}


\threelineshloka
{विमानैः स्फाटिकैर्दिव्यैःइ सर्वरत्नैरलङ्कृतैः}
{गन्धर्वैरप्सरोभिश्चि पूज्यमानः प्रमोदते}
{द्वेऽर्बुदानां सहस्रे तु दिव्ये दिव्येन तेजसा}


\twolineshloka
{सप्तविंशेऽथ दिवसे यः कुर्यादेकभोजनम्}
{सदा द्वादशमासांस्तु जुह्वानो जातवेदसम्}


\twolineshloka
{फलं प्राप्नोति विपुलं देवलोके च पूज्यते}
{अमृताशी वसंस्तत्र स वितृपः प्रमोदते}


\twolineshloka
{देवर्षिचरिताँल्लोकान्राजर्षिभिरनुष्ठितान्}
{अध्यावसति दिव्यात्मा विमानवरमास्थितः}


\twolineshloka
{स्त्रीभिर्मिनोभिरामाभी रममाणो मदोत्कटः}
{युगकल्पसहस्राणि त्रीण्यावसति वै सुखम्}


\twolineshloka
{योऽष्टाविंशे तु दिवसे प्राश्नीयादेकभोजनम्}
{सदा द्वादशमासांस्तु जितात्मा विजितेन्द्रियः}


\twolineshloka
{फलं देवर्षिचरितं विपुलं समुपाश्नुते}
{भोगवांस्तेजसा भाति सहस्रांशुरिवामलः}


\twolineshloka
{सुकुमार्यश्च नार्यस्तं रममाणाः सुवर्चसः}
{पीनस्तनोरुजघना दिव्याभरणभूषिताः}


\twolineshloka
{रमयन्ति मनःकान्ता विमाने सूर्यसन्निबे}
{सर्वकामगमे दिव्ये कल्पायुतशतं समाः}


\twolineshloka
{एकोनत्रिंशदिवसे यः प्राशेदेकभोजनम्}
{तस्य लोकाः शुभा दिव्या देवराजर्षिपूजिताः ॥तस्य लोकाः शुभा दिव्या देवराजर्षिपूजिताः}


\threelineshloka
{विमानं सूर्यचन्द्राभं दिव्यं समधिगच्छति}
{जातरूपमयं युक्तं सर्वरत्नसमन्वितम्}
{अप्सरोगणसङ्कीर्णं गन्धर्वैरभिनादितम्}


\twolineshloka
{तत्र चैनं शुभा नार्यो दिव्याभरणभूषिताः}
{मनोभिरामा मधुरा रमयन्ति मदोत्कटाः}


\twolineshloka
{भोगवांस्तेजसा युक्तो वैश्वानरसमप्रभः}
{दिव्यो दिव्येन वपुषा भ्राजमान इवामरः}


\twolineshloka
{वसूनां मरुतां चैव साध्यानामश्विनोस्तथा}
{रुद्राणां च तथा लोकं ब्रह्मलोकं च गच्छति}


\twolineshloka
{यस्तु मासे गते भुङ्क्ते एकभक्तं समाहितः}
{सदा द्वादश मासान्वै ब्रह्म्लोकमवाप्नुयात्}


\twolineshloka
{सुधारसकृताहार श्रीमान्सर्वमनोहरः}
{तेजसा वपुषा लक्ष्म्या भ्राजते रश्मिवानिव}


\twolineshloka
{दिव्यमाल्याम्बरधरो दिव्यगन्धानुलेपनः}
{सुखेष्वभिरतो भोगी दुःखानामविजानकः}


\twolineshloka
{स्वयंप्रभाभिर्नारीभिर्विमानस्थो महीयते}
{रुद्रदेवर्षिकन्याभिः सततं चाभिपूज्यते}


\twolineshloka
{नानारमणरूपाभिर्नानारागाभिरेव च}
{नानामधुरभाषाभिर्नानारतिभिरेव च}


\twolineshloka
{विमाने गगनाकारे सूर्यवैडूर्यसन्निभे}
{पृष्ठतः सोमसङ्करो उदर्के चाभ्रसन्निभे}


\twolineshloka
{दक्षिणायां तु रक्ताभे अधस्तान्नीलमण्डले}
{ऊर्ध्वं विचित्रसङ्काशे नैको वसति पूजितः}


\twolineshloka
{यावद्वर्षसहस्रं वै जम्बूद्वीपे प्रवर्षति}
{तावत्संवत्सराः प्रोक्ता ब्रह्मलोकेऽस्य धीमतः}


\twolineshloka
{विप्रुषश्चैव यावन्त्यो निपतन्ति नभस्तलात्}
{वर्षासु वर्षतस्तावन्निवसत्यमरप्रभः}


\twolineshloka
{मासोपवासी वर्षैस्तु दशभिः स्वर्गमुत्तमम्}
{महर्षित्वमथासाद्य सशरीरगतिर्भवेत्}


\twolineshloka
{मुनिर्दान्तो जितक्रोधो जितशिश्नोदरः सदा}
{जुह्वन्नग्नींश्च नियतः सन्ध्योपासनसेविता}


\twolineshloka
{बहुभिर्नियमैरेवं शुचिरश्नाति यो नरः}
{अभ्रावकाशशीलश्च तस्य भानोरिव त्विषः}


\twolineshloka
{दिवं गत्वा शरीरेण स्वेन राजन्यथाऽमरः}
{स्वर्गं पुण्यं यथाकाममुपभुङ्क्ते तथाविधः}


\threelineshloka
{एथ ते भरतश्रेष्ठ यज्ञानां विधिरुत्तमः}
{व्याख्यातो ह्यानुपूर्व्येण उपवासफलात्मकः}
{}


% Check verse!
दरिद्रैर्मनुजैः पार्थ प्राप्यं यज्ञफलं यथा ॥देवद्विजातिपूजायां रतो भरतसत्तमम्
\twolineshloka
{उपवासविधिस्त्वेष विस्तरेण प्रकीर्तितः}
{नियतेष्वप्रमत्तेषु शौचवत्सु महात्मसु}


\twolineshloka
{दम्भद्रोहनिवृत्तेषु कृतबुद्धिषु भारत}
{अचलेष्वप्रकम्पेषु मा ते भूदत्र संशयः}


\chapter{अध्यायः १६५}
\twolineshloka
{पितामहेन कथिता दानधर्माश्रिताः कथाः}
{मया श्रुता ऋषीणां तु संनिधौ केशवस्य च}


\twolineshloka
{पुनः कौतूहलमभूत्तामेवाध्यात्मिकीं प्रति}
{कथाः कथय राजेन्द्र त्वदन्यः क उदाहरेत्}


\twolineshloka
{एष यादवदायादस्तथानुज्ञातुमर्हति}
{ज्ञाते तु यस्मिञ्ज्ञातव्यं ज्ञातं भवति भारत}


\threelineshloka
{पश्चाच्छ्रोष्यामहे राज्ञां श्राव्यान्धर्मान्पितामह}
{कौतूहलमृषीणां तु च्छेतुमर्हसि साम्प्रतम् ॥भीष्म उवाच}
{}


% Check verse!
अत ऊर्ध्वं महाराज साङ्ख्ययोगोभयशास्त्राधि-गतयाथात्म्यदर्शनसम्पन्नयोराचार्ययोःसंवादमनुव्याख्यास्यामः
% Check verse!
तद्यथा भगवन्तं सनत्कुमारमासीनमङ्गुष्ठपर्व-मात्रं महति विमानवरे योजनसहस्रमण्डले तरुण-भास्करप्रतीकाशे शयनीये महति बद्धासनमनु-ध्यायन्तममृतमनावर्तकरममूर्तमक्षयजमथोपदि-ष्टमुपससर्प भगवन्तमाचार्यं भगवानाचार्यो रुद्रः
% Check verse!
तं प्रोवाच स्वागतं महेश्वर ब्रह्मसुत एतदास-नमास्तां भगवान्
\twolineshloka
{इत्युक्ते चासीनो भगवाननन्तरूपो रुद्रस्तंप्रोवाच भगवानपि ध्यानमावर्तयति}
{इत्युक्ते चाह भगवान्सनत्कुमारस्तथेति}


\twolineshloka
{तथेत्युक्तश्च प्रोवाच भगवाञ्शङ्करस्तदा}
{परावरज्ञं सर्वस्य त्रैलोक्यस्य महामुनिम्}


\twolineshloka
{किं वा ध्यानेन द्रष्टव्यं यद्भवाननुपश्यति}
{यच्च ध्यात्वा न शोचन्ति यतयस्तत्वदर्शिनः}


\twolineshloka
{कथय त्वमिमं देवं देहिनां यतिसत्तम}
{यच्च तत्पुरुषं शुद्धमित्युक्तं योगसाङ्ख्ययोः}


\twolineshloka
{किमध्यात्माधिभूतं च तथा चाप्यधिदैवतम्}
{कालसङ्ख्या च का देव द्रष्टव्या तस्य ब्रह्मणः}


\twolineshloka
{सङ्ख्या सङ्ख्यादनस्यैव या प्रोक्ता परमर्षिभिः}
{शास्त्रदृष्टेन मार्गेण यथावद्यतिसत्तम}


\twolineshloka
{यच्च तत्पुरुषं शुद्धं प्रबुद्धमजरं ध्रुवम्}
{बुध्यमानाप्रबुद्धाभ्यां विद्यावेद्यं तथैव च}


\twolineshloka
{विमोक्षं त्रिविधं चैव ब्रूहि मोक्षविदांवर}
{परिसाङ्ख्यं च साङ्ख्यानां ध्यानं योगेषु चार्थवत्}


\twolineshloka
{एकत्वदर्शनं चैव तथा नानात्वदर्शनम्}
{अरिष्टानि च तत्वेन तथैवोत्क्रमणानि च}


\threelineshloka
{दैवतानि च सर्वाणि निखिलेनानुपूर्वशः}
{यान्याश्रितानि देहेषु देहिनां यतिसत्तम}
{सर्वमेतद्यथातत्वमाख्याहि मुनिसत्तम}


\threelineshloka
{श्रेष्ठो भवान्हि सर्वेषां ब्रह्मज्ञानामनिन्दितः}
{चतुर्थस्त्वं त्रयाणां तु ये गताःइ परमां गतिम्}
{ज्ञानेन च प्राकृतेन निर्मुक्तो मृत्युबन्धनात्}


\twolineshloka
{वयं तु वैकृतं मार्गमाश्रिता वै क्षरं सदा}
{परमुत्सृज्य पन्थानममृताक्षरमेव तु}


\twolineshloka
{न्यूने पथि निमग्नास्तु ऐश्वर्येऽष्टगुणे तथा}
{महिमानं प्रगृह्येमं विचरामो यथासुखम्}


\twolineshloka
{न चैतत्सुखमत्यन्तं न्यूनमेतदनन्तरम्}
{मूर्तिमत्परमेतत्स्यादिदमेवं सुसत्तम}


\twolineshloka
{पुनः पुनश्च पतनं मूर्तिमत्युपदिश्यते}
{न पुनर्मृत्युमित्यन्यं निर्मुक्तानां तु मूर्तितः}


\twolineshloka
{मृत्युदोषास्त्वनन्ता वै उत्पद्यन्ते कृतात्मनाम्}
{मर्त्येषु नाकपृष्ठेषु निरयेषु महामुने}


% Check verse!
तत्र मञ्जन्ति पुरुषाः सुखदुःखेन वेष्टिताः
\threelineshloka
{सुखदुःखव्यपेतं च यदाहुरमृतं पदम्}
{तदहं श्रोतुमिच्छामि यथावच्छ्रुतिदर्शनात् ॥सनत्कुमार उवाच}
{}


\twolineshloka
{यदुक्तं भवता वाक्यं तत्वसंज्ञेति देहिनाम्}
{चतुर्विंशतिमेवात्र केचिदाहुर्मनीषिणः}


\threelineshloka
{केचिदाहुस्त्रयोविंशं यथाश्रुतिनिदर्शनात्}
{वयं तु पञ्चविंशं वै तदधिष्ठानसंज्ञितम्}
{तत्वं समधिमन्यामः सर्वतन्त्रप्रलापनात्}


\twolineshloka
{अव्ययश्चैव वै व्यक्तावुभावपि पिनाकधृक्}
{सह चैव विना चैव तावन्योन्यं प्रतिष्ठितौ}


\twolineshloka
{हिरण्मयीं प्रविश्यैष मूर्तिं मूर्तिमतांवर}
{चकार पुरुषस्तात विकारपुरुषावुभौ}


\twolineshloka
{अव्यक्तादेक एवैष महानात्मा प्रसूयते}
{अहङ्कारेण लोकांश्च व्याप्य चाहंकृतेन वै}


\twolineshloka
{पिना सर्वं तदव्यक्तादभफिमन्यस्व शूलधृक्}
{भूतसर्गमहङ्कारात्तृतीयं विद्धि वै क्रमात्}


\twolineshloka
{अहङ्काराच्च भूतेषु चतुर्थं विद्दि वैकृतम्}
{अहङ्काराच्च जातानि युगपद्विबुधेश्वर}


\twolineshloka
{सविशेषाणि भूतानि पञ्च प्राहुर्मनीषिणः}
{चतुर्विंशात्तु वै प्रोक्तात्पञ्चविंशोऽधितिष्ठति}


\twolineshloka
{एते सर्गा मया प्रोक्ताश्चत्वारः प्राकृतास्त्विह}
{अहङ्काराच्च जातानि युगपद्विबुधेश्वर}


\twolineshloka
{अङ्काराच्च भूतेषु विविधार्थं व्यजायत}
{इन्द्रियैर्युगपत्सर्वैः सो नित्यश्च समीक्षते}


\twolineshloka
{मरुत्त्वं सत्वसर्गश्च तुष्टिः सिद्धिस्तथैव च}
{वैकृतानि प्रवक्ष्यामि शृणु तानि महामते}


\twolineshloka
{एषा तत्वचतुर्विंशन्मया शास्त्रानुमानतः}
{वर्णिता तव देवेश पञ्चविंशत्समन्विता}


\twolineshloka
{पञ्चमोऽनुग्रहश्चैव नवैते प्राकृतैः सह}
{ऐन्द्रेप्यहमधोप्यन्यन्ममात्मनि च भास्वरः}


\twolineshloka
{यच्च देहमयं किञ्चित्त्रिषु लोकेषु विद्यते}
{सर्वत्रैवाभिमन्तव्यं त्वया त्रिपुरसूदन}


\twolineshloka
{अन्यथा येऽनुपश्यन्ति ते न पश्यन्ति ब्रह्मज}
{एतदव्यक्तविषयं पञ्चविंशसमन्वितम्}


\twolineshloka
{अनेन कारणेनैव तत्वमाहुर्मनीषिणः}
{विकारमात्रमेतं तु तत्वमाचक्षते परम्}


\threelineshloka
{निस्तत्वश्चैष देवेश बोद्धव्यं तु न बुद्ध्यते}
{यदि बुद्ध्येत्परं बुद्धं बुद्ध्यमानः सुरर्षभः}
{प्रबुद्धो ह्यभिमन्येत योयं नाहमिति प्रभो}


\chapter{अध्यायः १६६}
\threelineshloka
{तत्वसङ्ख्या श्रुता चैषा येषां ब्रह्मविदांवर}
{सर्गसङ्ख्या मया प्रोक्ता नवानामानुपूर्व्यशः}
{प्रवक्ष्यामि तु तेऽध्यात्ममधिभूताधिदैवतम्}


\threelineshloka
{नैतद्युक्तैर्वेदविद्भिर्गृहस्थै-र्मान्यैरेभिस्तपसा वाभिपन्नैः}
{यत्नेन दृष्टं परमात्मतत्वंतत्वेन प्राप्यं तु यथोक्तमेतत्}
{परं परेभ्यस्त्वमृतार्थतत्वंस्वभावसत्वस्थमनीशमीशम्}


\twolineshloka
{कैवल्यतां प्राप्य महासुरोत्तमतवैतदाख्यामि मुनीन्द्रवृत्त्या}
{रहस्यमेवान्यदवाप्य दिव्यंपवित्रपूतस्तव मृत्युजालम्}


\twolineshloka
{पृथ्वीमिमां यद्यपि रत्नपूर्णांदद्यान्न देयं त्वपरीक्षिताय}
{नाश्रद्दधानाय न चान्यबुद्धे-र्नाज्ञानयुक्ताय न विस्मिताय}


% Check verse!
स्वाध्याययुक्ताय गुणान्वितायप्रदेयमेतन्नियतेन्द्रियाय
\twolineshloka
{संक्षेपं चाप्यथैतेषां तत्वानां वृषभध्वज}
{अनुलोमानुजातानां प्रतिलोमप्रमीयतम्}


\twolineshloka
{प्रवक्ष्यामि तमध्यात्मं साधिभूताधिदैवतम्}
{यथांशुजालमर्कस्य तथैतत्प्रवदन्ति वै}


\twolineshloka
{संक्षीणे ब्रह्मदिवसे जगज्जलधिमाविशेत्}
{प्रलीयते जले भूमिर्जलमग्नौ प्रलीयते}


\twolineshloka
{लीयतेऽग्निस्तथा वायौ वायुराकाश एव तु}
{मनसि प्रलीयते खं तु मनोऽहंकार एव च}


\twolineshloka
{अहङ्कारस्तथा तस्मिन्महति प्रविलीयते}
{महानव्यक्त इत्याहुस्तदेकत्वं प्रचक्षते}


\twolineshloka
{अव्यक्तस्य महादेव प्रलयं विद्धि ब्रह्म्णि}
{एवमस्यासकृत्क्रीडामाहुस्तत्वविदो जनाः}


\twolineshloka
{अध्यात्ममधिभूतं च तथैवाप्यधिदैवतम्}
{यथावदुदितं शास्त्रं योगे तु सुमहात्मभिः}


\twolineshloka
{तथैव चेह साङ्ख्ये तु परिसंख्यात्मचिन्तकैः}
{प्रपञ्चितार्थमेतावन्महादेव महात्मभिः}


\twolineshloka
{ब्रह्मेति विद्यादध्यात्मं पुरुषं चाधिदैवतम्}
{प्रभवं सर्वभूतानां रक्षणं तत्र कर्म च}


\twolineshloka
{अध्यात्मं प्राणमित्याहुः क्रतुमप्यधिदैवतम्}
{रथं च यज्ञवाहोऽत्र कर्माहङ्कारमेव च}


\twolineshloka
{अध्यात्मं तु मनो विद्याच्चन्द्रमाश्चाधिदैवतम्}
{दैवं च प्रभवश्चैव कर्म व्याहृतयस्तथा}


\twolineshloka
{विद्यात्तु श्रोत्रमध्यात्ममाकाशमधिदैवतम्}
{सर्वभावाभिधानार्थं शब्दः कर्म सदा स्मृतं}


\twolineshloka
{त्वगध्यात्ममथो विद्याद्वायुरत्राधिदैवतम्}
{सन्निपातय विज्ञानं सर्वकर्म च तत्र ह}


\twolineshloka
{अध्यात्मं चक्षुरित्याहुर्भास्करोऽत्राधिदैवतम्}
{ज्ञापतं सर्ववर्णानां रूपं कर्म सदा स्मृतम्}


% Check verse!
जिह्वेति विद्यादध्यात्ममापश्चात्राधिदैवतम्
\twolineshloka
{पायुरध्यात्ममित्याहुर्यथावद्यतिसत्तमाः}
{विसर्गमधिभूतं च मित्रं चाप्यधिदैवतम्}


\twolineshloka
{उपस्थोऽध्यात्ममित्याहुर्देवदेव पिनाकधृक्}
{अनुभावोऽधिभूतं तु दैवतं च प्रजापतिः}


\twolineshloka
{पादावध्यात्ममित्याहुस्त्रिशूलाङ्क मनीषिणः}
{गन्तव्यमधिभूतं तु विष्णुस्तत्राधिदैवतम्}


\twolineshloka
{वागध्यात्मं तथैवाहुः पिनाकिंस्तत्वदर्शिनः}
{वक्तव्यमधिभूतं तु वह्निस्तत्राधिदैवतम्}


\twolineshloka
{एतदध्यात्ममतुलं साधिभूताधिदैवतम्}
{मया तु वर्णितं सम्यग्देहिनाममरर्षभ}


\twolineshloka
{एतत्कीटपतङ्गे च श्वपाके शुनि हस्तिनि}
{पुत्रिकादंशमशके ब्राह्मणे गवि पार्थिवे}


\twolineshloka
{सर्वमेव हि द्रष्टव्यमन्यथा मा विचिन्तय}
{अतोऽन्यथा ये पश्यन्ति न सम्यक्तेषु दर्शनम्}


\twolineshloka
{देवदानवगन्धर्वयक्षराक्षसकिन्नराः}
{यन्न जानन्ति को ह्येष कुतो वा भगवानिति}


\twolineshloka
{ओमित्येकाक्षरं ब्रह्म यत्तत्सदसतः परम्}
{अनादिमध्यपर्यन्तं कूटस्थमचलं ध्रुवम्}


\twolineshloka
{योगेश्वरं पद्मनाभं विष्णुं जिष्णुं जगत्पतिम्}
{अनादिनिधं देवं देवदेवं सनातनम्}


\twolineshloka
{अपमेयमविज्ञेयं हरिं नारायणं प्रभुम्}
{कृताञ्जलिः शुचिर्भूत्वा प्रणम्य प्रयतोऽर्चयेत्}


\twolineshloka
{अनाद्यन्तं परं ब्रह्म न देवा ऋषयो विदुः}
{एकोयं वेद भगवांस्त्राता नारायणो हरिः}


\twolineshloka
{नारायणादृषिगणास्ततः सिद्धा महोरगाः}
{देवा देवर्षयश्चैव यं विदुर्दुःखभेषजम्}


\twolineshloka
{यमाहुर्विजितक्लेशं यस्मिंश्च विहिताः प्रजाः}
{यस्मिँल्लोकाः स्फुरन्तीमे जाले शकुनयो यथा}


\twolineshloka
{सप्तर्षयो मनः सप्त साङ्ख्यास्तु मुनिदर्शनात्}
{सप्तर्षयश्चेन्द्रियाणि पञ्च बुद्धीन्द्रियाणि च}


% Check verse!
श्रोत्रयोश्च दिशः प्राहुर्मनसि त्वथ चन्द्रमाः
% Check verse!
मनः षष्ठं बुद्धिः सप्तमी ह्यात्मनि स्थापितानिशरीरेषु नात्मनि तस्य हि कारणानि भन्ति सर्वा-ण्यपि सर्वकर्मसु वा विषयेषुवा युञ्जन्ति यथात्म-नि स्वानि कर्माणि प्रवृत्तानि सप्तस्वपि
% Check verse!
विषयाणां व्यापकत्वानि तान्येव स्वपतो भूत-ग्रामस्याजमात्मानं देवलोकस्थानसंमितं देहान्तर-गामिनं मुमुक्षुं वानुप्रतिशन्ति सूक्ष्माणि प्रलीयन्ते
\twolineshloka
{मोक्षकाले तमेकं न कश्चिद्वेत्ति स्वपरम्}
{एवं प्रविष्टेषु भूतेषु को जागर्तीत्युच्यते}


% Check verse!
निद्राप्रसुप्तेषु वाऽत्र जाग्रत्स्वप्नशीलोत्रसदनौच देवद्योतनो भगवांश्चात्र क्षत्रेज्ञोबुद्धिर्वाऽभिसुप्त-स्यापि स्वप्नदर्शनानि पश्यन्ति
\threelineshloka
{अप्रतिबुद्धेषु लोकेषु स एव प्रतिबुध्यते}
{सएष प्राज्ञः}
{अथ तैजसः कायाग्निः स हि सुप्तस्यात्मा वा}


% Check verse!
अविदुषो वाऽप्रतिबुध्यमानस्यान्नं पचतीत्यन्तेवै तिष्ठति एतदात्मानमधिकृत्यानुज्ञातमिति
\chapter{अध्यायः १६७}
\twolineshloka
{प्रभवश्चाप्ययस्तात वर्णितस्तेऽनुपूर्वशः}
{तथाऽध्यात्माधेभूतं च तथैवात्राधिदैवतम्}


\twolineshloka
{निखिलेन तु वक्ष्यामि दैवतानि ह देहिनाम्}
{यान्याश्रितेषु देहेषु यानि पृच्छसि शङ्कर}


\twolineshloka
{वाच्यग्निस्त्वथ जिह्वायां सोमः प्राणे तु मारुतः}
{रूपे चाप्यथ नक्षत्रं जिह्वायां चाप एव तु}


\twolineshloka
{नाभ्यां समुद्रश्च विभुर्नखरोम तथैव च}
{वनस्पतिवनौषध्यस्त्वङ्गेषु मरुतस्तथा}


\twolineshloka
{संवत्सराः पर्वसु च आकाशे दैवमानुषे}
{उदाने विद्युदभवद्व्याने पर्जन्य एव च}


\twolineshloka
{स्तनयोरेव चाकाशं बले चेन्द्रस्तथैव च}
{मनसोऽप्यथ चेशानस्त्वपाने रुद्र एव च}


\twolineshloka
{गन्धर्वाप्सरसो व्याने सत्ये मित्रश्च शङ्कर}
{प्रज्ञायां वरुणश्चैव चक्षुष्यादित्य एव च}


\twolineshloka
{शरीरे पृथिवी चैव पादयोर्विष्णुरेव च}
{पायौ मित्रस्तथोपस्थे प्रजापतिररिंदम}


\twolineshloka
{मूर्ध्नि चैव दिशः प्राहुर्बुद्धौ ब्रह्मा प्रतिष्ठितः}
{बुध्यमानोऽऽत्मनिष्ठः स्यादधिष्ठाता तु शङ्करः}


\twolineshloka
{अबुद्धश्चाभवत्तस्माद्बुध्यमानान्न संशयः}
{आभ्यामन्यः परो बुद्धो वेदवादेषु शङ्कर}


\twolineshloka
{यदाश्रितानि देहेषु दैवतानि पृथक्पृथक्}
{योऽग्नये जायते नित्यमात्मयाजी समाहितः}


\threelineshloka
{य एवमनुपश्येत दैवतानि समाहितः}
{सोत्र योगी भवत्येव य एवमनुपश्यति}
{स सर्वज्ञयाजिभ्यो ह्यात्मयाजी विशिष्यते}


\twolineshloka
{मुखे जुहोति यो नित्यं कृत्स्नं विश्वमिदं जगत्}
{सोत्मवित्प्रोच्यते तज्ज्ञैर्महादेव महात्मभिः}


\twolineshloka
{सर्वेभ्यः परमेभ्यो वै दैवतेभ्यो ह्यात्मयाजिना}
{गन्तव्यं परमाकाङ्क्षन्परमेव च चिन्तयन्}


\twolineshloka
{यथा संक्रमते देहाद्देही त्रिपुरसूदन}
{तथास्य स्थानमाख्यास्ये पृथक्त्वेनेह शङ्कर}


\twolineshloka
{पादाभ्यां वैष्णवं स्थानमाप्नोति विनियोजनात्}
{पायुना मित्रमाप्नोति उपस्थेन प्रजापतिम्}


\twolineshloka
{नाभ्या च वारुणं स्थानं स्तनाभ्यां तु भवो लभेत्बाहुभ्यां वासवं स्थानं श्रोत्राभ्यामाप्नुयाद्दिशः}
{आदित्यं चक्षुषा स्थानं मूर्ध्ना ब्रह्मण एव च}


\twolineshloka
{अथ मूर्धसु यः प्राणान्धारयेत समाहितः}
{बुद्ध्या मानमवाप्नोति द्रव्यावस्थं च संशयः}


\twolineshloka
{अव्यक्तात्परमं शुद्धमप्रमेयमनामयम्}
{तमाहुः परमं नित्यं यद्यदाप्नोति बुद्धिमान्}


\twolineshloka
{बुध्यमानाप्रबुद्धाभ्यां स बुद्ध इति पठ्यते}
{बुध्यमानमबुद्धश्च नित्यमेवानुपश्यति}


\twolineshloka
{विकारपुरुषस्त्वेष बुध्यमान इति स्मृतः}
{पञ्चविंशतितत्वं तत्प्रोच्यते तत्र संशयः}


\twolineshloka
{स एष प्रकृतिस्थत्वात्तस्थुरित्युपदिश्यते}
{महानात्मा महादेव महानत्राधितिष्ठति}


\twolineshloka
{अधिष्ठानादधिष्ठाता प्रोच्यते शास्त्रदर्शनात्}
{एष चेतयते देव मोहजालमबुद्धिमान्}


\twolineshloka
{अव्यक्तस्यैव साधर्म्यमेतदाहुर्मनीषिणः}
{सोहं सोहमतो नित्यादज्ञानादिति मन्यते}


\twolineshloka
{यदि बुध्यति चैवायं मन्येयमिति भास्वरः}
{न प्रबुद्धो न वर्तेत पानीयं मत्स्यको यथा}


\twolineshloka
{देवता निखिलेनैताः प्रोक्तास्त्रिभुवनेश्वर}
{योगकृत्यं तु तावन्मे त्वं निबोधानुपूर्वशः}


\twolineshloka
{शून्यागारेष्वरण्येषु सागरे वा गुहासु वा}
{विष्टम्भयित्वा त्रीन्दण्डानवाप्तो ह्यद्वयो भवेत्}


\twolineshloka
{प्राङ्मुखोदङ्मुखो वापि तथा पश्चान्मुखोपि वा}
{दक्षिणावदनो वापि बद्ध्वा विधिवदासनम्}


\twolineshloka
{स्वस्तिकेनोपसंविष्टः कायमुन्नाम्य भास्वरम्}
{यथोपदिष्टं गुरुणा तथा तद्ब्रह्म धारयेत्}


\twolineshloka
{लघ्वाहारो यतो दान्तस्त्रिकालपरिवर्जकः}
{मूत्रोत्सर्गपुरीषाभ्यामाहारे च समाहितः}


\twolineshloka
{शेषकालं तु युञ्जीत मनसा सुसमाहितः}
{इन्द्रियाणीन्द्रियार्थेभ्यो मनसा विनिवर्तयेत्}


\twolineshloka
{मनस्तथैव सङ्गृह्य बुद्ध्या बुद्धिमतां वर}
{विधावमानं धैर्येण विस्फुरन्तमितस्ततः}


\twolineshloka
{निरुध्य सर्वसङ्कल्पांस्ततो वै स्थिरतां व्रजेत्}
{एकाग्रस्तद्विजानीयात्सर्वं गुह्यतमं परम्}


\twolineshloka
{निवातस्थ इवालोलो यथा दीपोऽतिदीप्यते}
{ऊर्ध्वमेव न तिर्यक्च तथैवाभ्रान्ति ते मनः}


\twolineshloka
{हृदिस्थस्तिष्ठते योसौ तस्यैवाभिमुखो यदा}
{मनो भवति देवेश पाषाणमिव निश्चलम्}


\twolineshloka
{स निर्जने विनिर्घोषे सघोषे चाऽऽवसञ्जने}
{युक्तो यो न विकम्पेत योगी योगविधिः श्रुतः}


\twolineshloka
{ततः पश्यति तद्ब्रह्म ज्वलदात्मनि संस्थितम्}
{विद्युदम्बुधरे यद्वत्तद्वदेकमनाश्रयम्}


\twolineshloka
{तमस्यगाधे तिष्ठन्तं निस्तमस्कमचेतनम्}
{चेतयानमचेतं च दीप्यमानं स्वतेजसा}


\twolineshloka
{अङ्गष्ठपर्वमात्रं तन्नैश्रेयसमनिन्दितम्}
{महद्भूतमनन्तं च स्वतन्त्रं विगतज्वरम्}


\twolineshloka
{ज्योतिषां ज्योतिषं देवं विष्णुमत्यन्तनिर्मलम्}
{नारायणमणीयांसमीश्वराणामधीश्वरम्}


\twolineshloka
{विश्वतः परमं नित्यं विश्वं नारायणं प्रभुम्}
{अविज्ञाय निमज्जन्ति लोकाः संसारसागरे}


\twolineshloka
{यं दृष्ट्वा यतयस्तात न शोचन्ति गतज्वराः}
{जन्ममृत्युभयान्मुक्तास्तीर्णाः संसारसागरम्}


\threelineshloka
{अणिमा लघिमा भूमा प्राप्तिः प्राकम्यमेव च}
{ईशित्वं च वशित्वं च यत्र कामावसायिता}
{एतदष्टुगुणं योगं योगानाममितं स्मृतम्}


\threelineshloka
{दृष्ट्वात्मानं निरात्मानमप्रमेयं सनातनम्}
{ते विशन्ति शरीराणि योगेनानेन भास्वरम्}
{दैत्यदेवमनुष्याणां बलेन बलवत्तमाः}


\twolineshloka
{एतत्तत्वमनाद्यन्तं यद्भवाननुपृच्छति}
{नित्यं वयमुपास्यामो योगधर्मं सनातनम्}


\twolineshloka
{योगधर्मान्न धर्मोस्ति गरीयान्सुरसत्तम}
{एतद्धर्म हि धर्माणामपुनर्भवसंस्कृतम्}


\twolineshloka
{तत्वतः परमस्तीस्ति केचिदाहुर्मनीषिणः}
{केचिदाहुः परं नास्ति ये ज्ञानफलमाश्रिताः}


\twolineshloka
{ज्ञानस्थः पुरुषस्त्वेष विकृतः स्वेन वर्णितः}
{ये ध्यानेनानुपश्यन्ति नित्यं योगपरायणाः}


\twolineshloka
{तमेव पुरुषं देवं केचिदेव महेश्वर}
{नित्यमन्यतमाः प्राहुर्ज्ञानं परमकं स्मृतम्}


\twolineshloka
{ज्ञानमेव विनिर्मुक्ताः सांख्या गच्छन्ति केवलम्}
{चिन्ताध्यात्मनि चान्यत्र योगाः परमबुद्धयः}


\twolineshloka
{उक्तमेतावदेतत्ते योगदर्शनमुत्तमम्}
{साङ्ख्यज्ञानं प्रवक्ष्यामि परिसंख्याविदर्शनम् ॥'}


\chapter{अध्यायः १६८}
\twolineshloka
{इन्द्रियेभ्यो मनः पूर्वमहङ्कारस्ततः परम्}
{अहङ्कारात्परा बुद्धिर्बुद्धेः परतरं महत्}


\twolineshloka
{महतः परमव्यक्तमव्यक्तात्पुरुषः परः}
{एतावदेतत्सांख्यानां दर्शनं देवसत्तम}


\threelineshloka
{अव्यक्तं बुद्ध्यहङ्कारौ महाभूतानि पञ्च च}
{मनस्तथा विशेषाश्च दश चैवेन्द्रियाणि च}
{एतास्तत्वचतुर्विंशन्महापुरुषसम्मिताः}


\twolineshloka
{बुद्ध्यामानेन देवेश चेतनेन महात्मना}
{संयोगमेतयोर्नित्यमाहुरव्यक्तपुंसयोः}


\twolineshloka
{एकत्वं च बहुत्वं च सर्गप्रलयकोटिशः}
{तमःसंज्ञितमेतद्धि प्रवदन्ति त्रिशूलधृक्}


\twolineshloka
{समुत्पाट्य यथाव्यक्ताज्जीवा यान्ति पुनःपुनः}
{आदिरेष महानात्मा गुणानामिति नः प्रभो}


\twolineshloka
{गुणस्थत्वाद्गुणं चैनमाहुरव्यक्तलक्षणम्}
{एतेनाध्युषितो व्यक्तस्त्रिगुणं चेतयत्युत}


\twolineshloka
{अचेतनः प्रकृत्येषु न चान्यमनुबुध्यते}
{बुध्यमानो ह्यहंकारो नित्यं मानाप्रबोधनात्}


\twolineshloka
{विमलस्व विशुद्धस्य नीरुजस्य महात्मनः}
{विमलोदरशीलः स्याद्बुध्यमानाप्रबुद्धयोः}


\twolineshloka
{द्रष्टा भवत्यभोक्ता च सत्वमूर्तिश्च निर्गुणः}
{बुध्यमानाप्रबुद्धाभ्यामन्य एव तु निर्गुणः}


\twolineshloka
{उपेक्षकः शुचिस्ताभ्यामुभाभ्यामयुतस्तथा}
{बुध्यमानो न बुध्येत बुद्धमेवं सनातनम्}


\twolineshloka
{स एव बुद्धेरव्यक्तस्वभावत्वादचेतनः}
{सोहमेव न मेऽन्योस्ति य एवमभिमन्यते}


\twolineshloka
{न मन्यते ममान्योस्ति येन चेतोस्म्यचेतनः}
{एवमेवाभिमन्येत बुध्यमानोप्यनात्मवान्}


\twolineshloka
{अहमेव न मेऽन्योस्ति न प्रबुद्धवशानुगः}
{अव्यक्तस्थो गुणानेष नित्यमेवाभिमन्यते}


\twolineshloka
{तेनाधिष्ठिततत्वज्ञैर्महद्भिरभिधीयते}
{अहङ्कारेण संयुक्तस्ततस्ददभिमन्यते}


\twolineshloka
{क्षेत्रं प्रविश्य दुर्बुद्धिर्बुद्ध्यमानो ह्यनात्मवान्}
{अहमेव सृजत्यन्यद्द्वितीयं लोकसारथिः}


\twolineshloka
{सर्वाभावैरहङ्कारैस्तृतीयं सर्गसंज्ञितम्}
{ततो भूतान्यहङ्कारमहङ्कारो मनोऽसृजत्}


\threelineshloka
{सर्वस्नोतस्यभिमुखं सम्प्रावर्तत बुद्धिमान्}
{तथैव यज्ञे भूतेषु विषयार्थी पुनःपुनः}
{इन्द्रियैः सह शूलाङ्क पञ्च पञ्चभिरेव च}


\twolineshloka
{मनो वेद न चात्मानमङ्कारं प्रजापतिः}
{न वेद वाप्यहङ्कारो बुद्धिं बुद्धिमतांवर}


\twolineshloka
{एवमेते महाभाग नेतरे नयवादिनः}
{अहङ्कारेण संयुक्तः स्रोतस्यभिमुखः सदा}


\twolineshloka
{एवमेष विकारात्मा महापुरुषसञ्ज्ञकः}
{प्रतनोति जगत्कृत्स्नं पुनराददते सकृत्}


\twolineshloka
{ससंज्ञत्वाज्जगत्कृत्स्नमव्यक्तस्य हृदि स्थितम्}
{संविशद्रजनीं कृत्स्नां निशान्ते दिवसागमे}


\twolineshloka
{पुनरात्मा विजयते बहवो निर्गुणास्तथा}
{अज्ञानेन समायुक्तः सोव्यक्तेन तमोत्मना}


\twolineshloka
{यदि ह्येषो नु मन्येत ममास्ति परतो वरः}
{स पुनः पुनरात्मानं न कुर्यादाक्षिपेत च}


\twolineshloka
{एतमव्यक्तविषयं सूक्ष्मं मन्येत बुद्धिमान्}
{पञ्चविंशं महादेव महापुरुषवैकृतम्}


\twolineshloka
{प्रबुद्धौ बुद्धवानेतत्सृजमानमबुद्धवान्}
{गुणान्पुनश्च तानेव सोत्मनात्मनि निक्षिपेत्}


\twolineshloka
{अव्यक्तस्य वशीभूतो योऽज्ञानस्य तमोत्मनः}
{बुध्यमानो ह्यबुद्धस्य बुद्धस्तदनुभुज्यते}


\twolineshloka
{उपेक्षकः शुचिर्व्यग्रः सोलिङ्गः सोव्रणोऽमलः}
{षड्विंशो भगवानास्ते बुद्धः शुद्धो निरामयः}


\twolineshloka
{अव्यक्तादिविशेषान्तमेतद्वैद्या वदन्त्युत}
{एतैरेव विहीनं तु केचिदाहुर्मनीषिणः}


\twolineshloka
{निस्तत्वं बुध्यमानास्तु केचिदाहुर्महामते}
{केचिदाहुर्महात्मानस्तत्वसंज्ञितमेव तु}


\twolineshloka
{तत्वस्य श्रवणादेनं तत्वमेवं वदन्ति वै}
{सत्वसंश्रयणाच्चैव सत्ववन्तं महेश्वर}


\threelineshloka
{एवमेष विकारात्मा बुध्यमानो महाभुज}
{अव्यक्तो भवते व्यक्तौ सत्वंसत्वं तथा गुणौ}
{विद्या च भवते विद्या भवेत्तु ग्रहसंज्ञितम्}


\twolineshloka
{य एवमनुबुद्ध्यन्ते योगसाङ्ख्याश्च तत्वतः}
{तेऽव्यक्तं शङ्करागाढं मुञ्चन्ते शास्त्रबुद्धयः}


\twolineshloka
{तेषामेतत्तु वदतां शास्त्रार्थं सूक्ष्मदर्शिनाम्}
{बुद्धिर्विस्तीर्यते सर्वं तैलबिन्दुरिवाम्भसि}


\twolineshloka
{विद्या तु सर्वविद्यानामवबोध इति स्मृतः}
{येन विद्यामविद्यां च विन्दन्ति यतिसत्तमाः}


\twolineshloka
{सैषा त्रयी परा विद्या चतुर्थ्यान्वीक्षिकी स्मृता}
{यां बुद्ध्यमानो बुद्ध्येत बुद्ध्यात्मनि समं गतः}


\threelineshloka
{अप्रबुद्धमथाव्यक्तमविद्यासंज्ञिकं स्मृतम्}
{विमोहितं तु शोकेन केवलेन समन्वितम्}
{एतद्बुद्ध्या भवेद्बुद्धः किमन्यद्बुद्धिलक्षणम्}


\twolineshloka
{ये त्वेतन्नावबुद्ध्यन्ते ते प्रबुद्धवशानुगाः}
{ते पुनःपुनरव्यक्ताज्जनिष्यन्त्यबुधांत्मनः}


\twolineshloka
{तमेव तुलयिष्यन्ति अबुद्धिवशवर्तिनः}
{ये चाप्यन्ये तन्मनसस्तेप्येतत्फलभागिनः}


\twolineshloka
{विदित्वैनं न शोचन्ति योगोपेतार्थदर्शिनः}
{स्वातन्त्र्यं प्रतिलप्स्यन्ते केवलत्वं च भास्वरम्}


\threelineshloka
{अज्ञानबन्धनान्मुक्तास्तीर्णाः संसारबन्धनात्}
{अज्ञानसागरं घोरमगाधं तमसंज्ञकम्}
{यत्र मज्जन्ति भूतानि पुनःपुनररिंदम्}


\twolineshloka
{एषा विद्या तथाऽविद्या कथिता ते मयाऽर्थतः}
{यस्मिन्देयं च नो ग्राह्यं सांख्याः सांख्यं तथैव च}


\twolineshloka
{तथा चैकत्वनानात्वमक्षरं क्षरमेव च}
{निगदिष्यामि देवेश विमोक्षं त्रिविधं च ते}


\twolineshloka
{बुद्ध्यमानाप्रबुद्धाभ्यामबुद्धस्य प्रपञ्चनम्}
{भूय एव निबोध त्वं देवानां देवसत्तम}


\twolineshloka
{यच्च किञ्च श्रुतं न स्याद्दृष्टं चैव न किञ्चन}
{तच्च ते सम्प्रवक्ष्यामि एकाग्रः शृणु तत्परः ॥'}


\chapter{अध्यायः १६९}
\twolineshloka
{अरिष्टानि प्रवक्ष्यामि तत्वेन शृणु तद्भवान्}
{मध्य उत्तरतस्तात दक्षिणामुखनिष्ठितम्}


\fourlineindentedshloka
{विद्युत्संस्थानपुरुषं यदि पश्येत मानवः}
{वर्षत्रयेण जानीयाद्देहन्यासमुपस्थितम्}
{एतत्फलमंरिष्टस्य शङ्कराहुर्मनीषिणः}
{}


\twolineshloka
{शुद्धमण्डमादित्यमरश्मिमथ पश्यतः}
{वर्षार्धकेन जानीयाद्देहन्यासमुपस्थितम्}


\twolineshloka
{छिद्रां चन्द्रमसश्छायां पादावप्यनपश्यतः}
{संवत्सरेण जानीयाद्देहन्यासमुपस्थितम्}


\twolineshloka
{कनीनिकायामशिरःपुरुषं यदि पश्यति}
{जानीयात्षट्सु मासेषु देहन्यासमुपस्थितम्}


\twolineshloka
{कर्णौ पिधाय हस्ताभ्यां शब्दं न शृणुयाद्यदि}
{विजानीयात्तु मासेन देहन्यासमुपस्थितम्}


\twolineshloka
{आमगन्धमुपाघ्राति सुरभिं प्राप्य भास्वरम्}
{देवतायतनस्थोपि सप्तरात्रेण मृत्युभाक्}


\threelineshloka
{सर्वाङ्गधारणावस्थां धारयेत समाहितः}
{यथा स मृत्युं जयति नान्यथेह महेश्वर}
{यदि जीवितुमिच्छेत चिरकालं महामुने}


\twolineshloka
{अथ नेच्छेच्चिरं कालं त्यजेदात्मानमात्मना}
{केवलं चिन्तयानस्तु निष्कलं स निरामयम्}


\twolineshloka
{अथ तं निर्विकारं तु प्रकृते परमं शुचिः}
{पुरुषं देहसाधर्म्यं देहन्यासमुपाश्नुयात्}


\twolineshloka
{जाग्रतो हि मयोक्तानि तवारिष्टानि तत्वतः}
{धारणाच्चैव सर्वाङ्गे मृत्यु जीयात्सुरर्षभ}


\twolineshloka
{एकत्वदर्शनं भूयो नानात्वं च निबोध मे}
{अक्षरं च क्षरं चैव चतुष्टयविधानतः}


\twolineshloka
{अव्यक्तादीनि तत्वानि सर्वाण्येव महाद्युते}
{आहुश्चतुर्विंशतितमं विकारपुरुषान्वितम्}


\twolineshloka
{एकत्वदर्शनं चैव नानात्वेन वरं स्मृतम्}
{पञ्चविंशतिवर्गः स्यादपवर्गोऽजरामरः}


\twolineshloka
{स निर्विकारः पुरुषस्तत्वेनैवोपदिश्यते}
{स एव पञ्चविंशस्तु विकारः पुरुषः स्मृतः}


\twolineshloka
{यद्येष निर्विकारः स्यात्तत्वं न तु भवेद्भव}
{विकारो विद्यमानस्तु तत्वसंज्ञकमुच्यते}


\twolineshloka
{यद्योषोऽव्यक्ततां नैति व्यतिरेकान्न संशयः}
{तथा भवति निस्तत्वस्तथा सत्वस्तथागुणः}


\twolineshloka
{विकारगुणसंत्यागात्प्रकृत्यन्यत्वता शुचिः}
{तदा नानात्वतामेति सर्गहीनोऽपवर्गभाक्}


\twolineshloka
{बोध्यमानः प्रबुध्येत समो भवति बुद्धिमान्}
{अक्षरश्च भवत्येष यथावा च्युतवान्क्षणात्}


\twolineshloka
{अव्यक्ताव्यक्तिरुक्ता स्यान्निर्गुणस्य गुणाकरात्}
{एतदेकत्वनानात्वमक्षरः क्षर एव च}


\threelineshloka
{व्याख्यातं तव शूलाङ्क तथारिष्टानि चैव हि}
{विमोक्षलक्षणं शेषं तदपीह ब्रवीमि ते}
{यं ज्ञात्वा यतयः प्राप्ताः केवलत्वमनामयम्}


\twolineshloka
{साङ्ख्याश्चाप्यथ योगाश्च दग्धपङ्का गतज्वराः}
{अमूर्तित्वमनुप्राप्ता निर्गुणा निर्भया भव}


\threelineshloka
{विपाप्मानो महादेव मुक्ताः संसारसागरात्}
{सरणे प्रजनादाने गुणानां प्रकृतिः सदा}
{परा प्रमत्ता सततमेतावत्कार्यकारणम्}


\twolineshloka
{असच्चैव च सच्चैव कुरुते स पुनः पुनः}
{चैतन्येन पुराणेन चेतनाचेतनात्परः}


\twolineshloka
{यस्तु चेतयते चेतो मनसा चैकबुद्धिकम्}
{स नैव सन्न चैवासन्सदसन्न च संस्मृतः}


\twolineshloka
{व्यतिरिक्तश्च शुद्धश्च सोऽन्यश्चाप्रकृतिस्तथा}
{उपेक्षकश्च प्रकृतेर्विकारपुरुषः स्मृतः}


\twolineshloka
{विकारपुरुषेणैषा संयुक्ता सृजते जगत्}
{पुनराददते चैव गुणानामन्यथात्मनि}


\twolineshloka
{मत्स्योदकात्ससंज्ञातः प्रकृतेरेव कर्मणः}
{तद्वत्क्षेत्रसहस्राणि स एव प्रकरिष्यति}


\twolineshloka
{क्षेत्रप्रलयतज्ज्ञस्तु क्षेत्रज्ञ इति चोच्यते}
{सयोगो नित्य इत्याहुर्ये जनास्तत्वदर्शिनः}


\twolineshloka
{एवमेष ह्यसत्सच्च विकारपुरुषः स्मृतः}
{विकारापद्यमानं तु विकृतिं प्रवदन्ति नः}


\twolineshloka
{यदा त्वेष विकारस्य प्रकृतानिति मन्यते}
{तदा विकारतामेति विकारान्यत्वतां व्रजेत्}


\threelineshloka
{प्रकृत्या च विकारैश्च व्यतिरिक्तो यदा भवेत्}
{शुचि यत्परमं शुद्धं प्रतिबुद्धं सनातनम्}
{अयुक्तं निष्कलं शुद्धमव्ययं चाजरामरम्}


\twolineshloka
{समेत्य तेन शुद्धेन बुध्यमानः स भास्वरः}
{विमोक्षं भजते व्यक्तादप्रबुद्धादचेतनात्}


\twolineshloka
{उदुम्बराद्वा मशकः प्रलयान्निर्गतो यथा}
{तथाऽव्यक्तस्य संत्यागान्निर्ममः पञ्च विंशकः}


\twolineshloka
{यथा पुष्करपर्णस्थो जलबिन्दुर्न संश्लिषेत्}
{तथैवाव्यक्तविषये न लिप्येत्पञ्चविंशकः}


\twolineshloka
{आकाश इव निःसङ्गस्तथा सङ्गस्त्था वरः}
{पञ्चविंशतिमो बुद्धो बुद्धेनि समतां गतः}


\twolineshloka
{एतद्धि प्रकृतं ज्ञानं तत्वतश्च समुत्थितम्}
{पूर्वजेभ्यस्तथोत्पन्नं ब्रह्मजेभ्यस्तथानघ}


\twolineshloka
{आदिसर्गो महाबाहो तामसेनावृतं परम्}
{प्रतिष्टावयवंदेवमभेद्यमजरामरम्}


\twolineshloka
{सनकः सनन्दनश्चैवि तृतीयश्च सनातनः}
{ते विदुः परमं धर्ममव्ययं व्ययधर्षणम्}


\twolineshloka
{अव्यक्तात्परमात्सूक्ष्मादव्रणान्मूर्तिसंज्ञकात्}
{क्षेत्रज्ञो भगवानास्ते नरायणपरायणः}


\twolineshloka
{अस्माकं सहजातानामुत्पन्नं ज्ञानमुत्तमम्}
{एते हि मूर्तिमन्तो वै लोकान्प्रविचरामहे}


\twolineshloka
{पुनःपुनः प्रजाता वै तत्रतत्र पिनाकधृक्}
{द्वन्द्वैर्विरज्यमानस्य ज्ञानमुत्पन्नमुत्तमम्}


\twolineshloka
{कपिलान्मूलआचार्यात्तत्वबुद्धिविनिश्चयम्}
{योगसांख्यमवाप्तं मे कार्त्स्न्येन मुनिसत्तमात्}


\twolineshloka
{तेन संबोधिताः शिष्या बहवस्तत्वदर्शिनः}
{तद्बुद्ध्वा बहवः शिष्या मयाप्येतन्निदर्शिताः}


\twolineshloka
{जन्ममृत्युहरं तथ्यं ज्ञानं ज्ञेयं सनातनम्}
{यज्ज्ञात्वा नानुशोचन्ति तत्वज्ञाना निरिन्द्रियाः}


\twolineshloka
{शुद्धबीजमलाश्चैव विपङ्का वै निरक्षराः}
{स्वतन्त्रास्ते स्वतन्त्रेण सम्मिता निष्कलाः स्मृताः}


\twolineshloka
{शाश्वताश्चाव्ययाश्चैव तमोग्राह्याश्च भास्वर}
{विपाप्मानस्तथा सर्वे सत्वस्थाश्चापि निर्व्रणाः}


\twolineshloka
{विमुक्ताः केवलाश्चैव वीतमोहभयास्तथा}
{अमूर्तास्ते महाभाग सर्वे च विगतज्वराः}


\twolineshloka
{हिरण्यनाभस्त्रिशिरास्तथा प्रह्लादभास्करौ}
{वसुर्विश्वावसुश्चैव सार्धं पञ्चशिखस्तथा}


\twolineshloka
{गार्ग्योऽथासुरिरावन्त्यो गौतमो वृत एव च}
{कात्यायनोऽथ नमुचिर्हरिश्च दमनश्चि ह}


\twolineshloka
{एते चान्ये च बहवस्तत्वमेवोपदर्शिताः}
{केचिन्मुक्ताः स्थिताः केचिच्छन्दतश्चापरे मृताः}


\twolineshloka
{दर्शितास्त्रिविधं बन्धं विमोक्षं त्रिविधं तथा}
{अज्ञानं चैव रागश्च संयोगं प्राकृतं तथा}


\twolineshloka
{एतेभ्यो बन्धनं प्रोक्तं विमोक्षमपि मे शृणु}
{परितस्तावता सम्यक्सम्बन्धो यावता कृतः}


\threelineshloka
{कृत्स्नक्षयपरित्यागाद्विमोक्ष इति नः श्रुतिः}
{निवृत्तः सर्वसङ्गेभ्यः केवलः पुरुषोऽमलः ॥भीष्म उवाच}
{}


\twolineshloka
{इत्येवमुक्त्वा भगवानीश्वराय महात्मने}
{सनत्कुमारः प्रययावाकाशं समुपाश्रितः ॥'}


\chapter{अध्यायः १७०}
\threelineshloka
{यद्वरं सर्वतीर्थानां तद्ब्रवीहि पितामह}
{यत्र चैव परं शौचं तन्मे व्याख्यातुमर्हसि ॥भीष्म उवाच}
{}


\twolineshloka
{सर्वाणि खलु तीर्थानि गुणवन्ति मनीषिणः}
{यत्तु तीर्थं च शौचं च तन्मे शृणु समाहितः}


\twolineshloka
{अगाधे विमले शुद्धे सत्यतीर्थे धृतिह्रदे}
{स्नातव्यं मानसे तीर्थे सत्वमालम्ब्य शाश्वतम्}


\twolineshloka
{तीर्थशौचं तपो ज्ञानं मार्दवं सत्यमार्जवम्}
{अहिंसा सर्वभूतानामानृशंस्यं दमः शमः}


\twolineshloka
{निर्ममा निरहङ्कारा निर्द्वन्द्वा निष्परिग्रहाः}
{योगिनस्तीर्थभूतास्ते तीर्थं परममुच्यते}


\twolineshloka
{नारायणेऽथ रुद्रे वा भक्तिस्तीर्थं परं मतम्}
{शौचलक्षणमेतत्ते सर्वत्रैवान्ववेक्षतः}


\threelineshloka
{रजस्तमःसत्वमथो येषां निर्धूतमात्मना}
{तीर्थमाचारशुद्धिश्च स्वमार्गपरिमार्गणम्}
{}


\twolineshloka
{सर्वत्यागेष्वभिरताः सर्वज्ञाः समदर्शिनः}
{शौचष्वेतेष्वभिरतास्ते तीर्थशुचयोऽपि च}


\twolineshloka
{नोदकक्लिन्नगात्रस्तु स्नात इत्यभिधीयते}
{स स्नातो यो दमस्नातः स बाह्याभ्यन्तरः शुचिः}


\twolineshloka
{अदृष्टेष्वनपेक्षा ये प्राप्तेष्वर्थेषु निर्ममाः}
{शौचमेव परं तेषां येषां नोत्पद्यते स्पृहा}


\twolineshloka
{प्रज्ञानं शौचमेवेह शरीरस्य विशेषतः}
{तथा निष्किंचनत्वं च मनसश्च प्रसन्नता}


\twolineshloka
{वृत्तं शौचं महाशौचं तीर्थशौचमतः परम्}
{ज्ञानोत्पन्नं च यच्छौचं तच्छौचं परमं स्मृतम्}


\twolineshloka
{मनसा च प्रदीप्तेन ब्रह्मज्ञानजलेन च}
{स्नाता ये मानसे तीर्थे तज्ज्ञाः क्षेत्रज्ञदर्शनाः}


\twolineshloka
{समारोपितशौचस्तु नित्यं भवसमन्वितः}
{केवलं गुणसम्पन्नः शुचिरेव नरः सदा}


\twolineshloka
{शरीरस्थानि तीर्थानि प्रोक्तान्येतानि भारत}
{पृथिव्यां यानि तीर्थानि पुण्यानि शृणु तान्यपि}


\twolineshloka
{शरीरस्य यथोद्देशः शरीरोपरि निर्मिताः}
{तथा पृथिव्या भागाश्च पुण्यानि सलिलानि च}


\threelineshloka
{कीर्तनाच्चैव तीर्थस्य स्नानाच्च पितृतर्पणात्}
{शोध्यं हि पातकं तीर्थे पूता यान्ति सुखं दिवम्}
{}


\twolineshloka
{परिग्रहाच्च साधूनां पृथिव्याश्चैव तेजसा}
{अतीव पुण्यभागास्ते सलिलस्य च तेजसा}


\twolineshloka
{मनसश्च पृथिव्याश्च पुण्यास्तीर्थास्तथापरे}
{उभयोरेव यः स्नायात्स सिद्धिं शीघ्रमाप्नुयात्}


\twolineshloka
{यथा फलं क्रियाहीनं क्रिया वा फलवर्जिता}
{नेह साधयते कार्यं समायुक्ता तु सिध्यति}


\twolineshloka
{एवं शरीरशौचेन तीर्थशौचेन चान्वितः}
{शुचिः सिद्धिमवाप्नोति द्विविधं शौचमुत्तमम्}


\chapter{अध्यायः १७१}
\threelineshloka
{सर्वेषामुपवासानां यच्छ्रेयः सुमहत्फलम्}
{यच्चाप्यसंशयं लोके तन्मे त्वं वक्तुमर्हसि ॥भीष्म उवाच}
{}


\twolineshloka
{शृणु राजन्यथा गीतं स्वयमेव स्वयंभुवा}
{यत्कृत्वा निर्वृतो भूयात्पुरुषो नात्र संशयः}


\twolineshloka
{द्वादश्यां मार्गशीर्षे तु अहोरात्रेण केशवम्}
{अर्च्याश्वमेधं प्राप्नोति दुष्कृतं चास्य नश्यति}


\twolineshloka
{तथैव पौषमासे तु पूज्यो नारायणेति च}
{वाजपेयमवाप्नोति सिद्धिं च परमां व्रजेत्}


\twolineshloka
{अहोरात्रेण द्वादश्यां माघमासे तु माघवम्}
{राजसूयमवाप्नोति कुलं चैव समुद्धरेत्}


\twolineshloka
{तथैव फाल्गुने मासि गोविन्देति च पूजयन्}
{अतिरात्रमवाप्नोति सोमलोकं च गच्छति}


\twolineshloka
{अहोरात्रेण द्वादश्यां चैत्रे विष्णुरिति स्मरन्}
{पौण्डरीकमवाप्नोति देवलोकं च गच्छति}


\twolineshloka
{वैशाखमासे द्वादश्यां पूजयन्मधुसूदनम्}
{अग्निष्टोममवाप्नोति सोमलोकं च गच्छति}


\twolineshloka
{अहोरात्रेण द्वादश्यां ज्येष्ठे मासि त्रिविक्रमम्}
{गवां मेधमवाप्नोति अप्सरोभिश्च मोदते}


\twolineshloka
{आषाढे मासि द्वादश्यां वामनेति च पूजयन्}
{नरमेधमवाप्नोति पुण्यं च लभते महत्}


\twolineshloka
{अहोरात्रेण द्वादश्यां श्रावणे मासि श्रीधरम्}
{पञ्चयज्ञानवाप्नोति विमानस्थश्च मोदते}


\twolineshloka
{तथा भाद्रपदे मासि हृषीकेशेति पूजयन्}
{सौत्रामणिमवाप्नोति पूतात्मा भवते च हि}


\twolineshloka
{द्वादश्यामाश्विने मासि पद्मनाभेति चार्चयन्}
{गोसहस्रफलं पुण्यं प्राप्नुयान्नात्र संशयः}


\twolineshloka
{द्वादश्यां कार्तिके मासि पूज्यो दामोदरेति च}
{गवां यज्ञमवाप्नोति पुमान्स्त्री वा न संशयः}


\twolineshloka
{अर्चयेत्पुण्डरीकाक्षमेवं संवत्सरं तु यः}
{जातिस्मरत्वं प्राप्नोति विन्द्याद्बहु सुवर्णकम्}


\twolineshloka
{अहन्यहनि तद्भावमुपेन्द्रं योऽधिगच्छति}
{समाप्ते भोजयेद्विप्रानथवा दापयेद्धृतम्}


% Check verse!
अतः परं नोपवासो भवतीति विनिश्चयः ॥उवाच भगवान्विष्णुः स्वयमेव पुरातनम् ॥]
\chapter{अध्यायः १७२}
\twolineshloka
{शरतल्पगतं भीष्मं वृद्धं कुरुपितामहम्}
{उपगम्य महाप्राज्ञः पर्यपृच्छद्युधिष्ठिरः}


\threelineshloka
{अङ्गानां रूपसौभाग्यं प्रियं चैव कथं भवेत्}
{धर्मार्थकामसंयुक्तः सुखभागी कथं भवेत् ॥भीष्म उवाच}
{}


\twolineshloka
{मार्गशीर्षस्य मासस्य चन्द्रे मूलेन संयुते}
{पादौ मूलेन राजेन्द्रि जङ्घायामथ रोहिणीम्}


\twolineshloka
{अश्विन्यां सक्थिनी चैव ऊरू चाषाढयोस्तथा}
{गुह्यं तु फाल्गुनी विद्यात्कृत्तिका कटिकास्तथा}


\twolineshloka
{नाभिं भाद्रपदे विद्याद्रेवत्यामक्षिमण्डलम्}
{पृष्ठमेव धनिष्ठासु अनुराधोत्तरास्तथा}


\twolineshloka
{बाहुभ्यां तु विशाखासु हस्तौ हस्तेन निर्दिशेत्}
{पुनर्वस्वङ्गुली राजन्नाश्लेषासु नखास्तथा}


\twolineshloka
{ग्रीवां ज्येष्ठा च राजेन्द्र श्रवणेन तु कर्णयोः}
{मुखं पुष्येण दानेन दन्तोष्ठौ स्वातिरुच्यते}


\twolineshloka
{हासं शतभिषां चैव मघां चैवाथ नासिकाम्}
{नेत्रे मृगशिरो विद्याल्ललाटे मित्रमेव तु}


\twolineshloka
{भरण्यां तु शिरो विद्यात्केशानार्द्रां नराधिप}
{समाप्ते तु घृतं दद्याद्ब्राह्मणे वेदपारगे}


\twolineshloka
{सुभगो दर्शनीयश्च ज्ञानभाग्यथ जानते}
{जायते परिपूर्णाङ्गः पौर्णमास्येव चन्द्रमाः ॥]}


\chapter{अध्यायः १७३}
\twolineshloka
{पितामह महाप्राज्ञ सर्वशास्त्रविशारद}
{श्रोतुमिच्छामि मर्त्यानां संसारविधिमुत्तमम्}


\twolineshloka
{केन वृत्तेन राजेन्द्र वर्तमाना नरा भुवि}
{प्राप्नुवन्त्युत्तमं स्वर्गं कथं च नरकं नृप}


\threelineshloka
{मृतं शरीरमुत्सृज्य काष्ठलोष्टसमं जनाः}
{प्रयान्त्यमुं लोकमितः को वै ताननुगच्छति ॥ भीष्म उवाच}
{}


\twolineshloka
{दूरादायाति भगवान्बृहस्पतिरुदारधीः}
{पृच्छैनं सुमहाभागमेतद्गुह्यं सनातनम्}


\threelineshloka
{नैतदन्येन शक्यं हि वक्तुं केनचिदद्य वै}
{वक्ता बृहस्पतिसमो न ह्यन्यो विद्यते क्वचित् ॥ वैशम्पायन उवाच}
{}


\twolineshloka
{तयोः संवदतोरेवं पार्थगाङ्गेययोस्तदा}
{आजगाम विशुद्धात्मा भगवान्स बृहस्पतिः}


\twolineshloka
{ततो राजा समुत्थाय धृतराष्ट्रपुरोगमः}
{पूजामनुषमां चक्रे सर्वे ते च सभासदः}


\twolineshloka
{ततो धर्मसुतो राजा भगवन्तं बृहस्पतिम्}
{उपगम्य यथान्यायं प्रश्नं पप्रच्छ तत्त्वतः}


\threelineshloka
{भगवन्सर्वधर्मज्ञ सर्वशास्त्रविशारद}
{मर्त्यस्य कः सहायो वै पिता माता सुतो गुरुः}
{ज्ञातिसम्बन्धिवर्गश्च मित्रवर्गस्तथैव च}


\threelineshloka
{मृतं शरीरमुत्सृज्य काष्ठलोष्टसमं जनाः}
{गच्छन्त्यमुं च लोकं वै क एताननुगच्छति ॥ बृहस्पतिरुवाच}
{}


\twolineshloka
{एकः प्रसूयते राजन्नेक एव विनश्यति}
{एकस्तरति दुर्गाणि गच्छत्येकस्तु दुर्गतिम्}


\twolineshloka
{न सहायः पिता माता तथा भ्राता सुतो गुरुः}
{ज्ञातिसम्बन्धिवर्गश्च मित्रवर्गस्तथैव च}


\twolineshloka
{मृतं शरीरमुत्सृज्य काष्ठलोष्टसमं जनाः}
{मुहूर्तमुपयुञ्ज्याथ ततो यान्ति पराङ्मुखाः}


\twolineshloka
{तैस्तच्छरीरमुत्सृष्टं धर्मि एकोऽनुगच्छति}
{तस्माद्धर्मः सहायार्थे सेवितव्यः सदा नृपः}


\twolineshloka
{प्राणी धर्मसमायुक्तो गच्छेत्स्वर्गगतिं पराम्}
{तथैवाधर्मसंयुक्तो नरकं चोपपद्यते}


\twolineshloka
{तस्मान्न्यायागतैरर्थैर्धर्मं सेवेत पण्डितः}
{धर्म एको मनुष्याणां सहायः पारलौकिकः}


\twolineshloka
{लोभान्मोहादनुक्रोशाद्भयाद्वाऽप्यबहुश्रुतः}
{नरः करोत्यकार्याणि परार्थे लोभमोहितः}


\threelineshloka
{धर्मश्चार्थश्च कामश्च त्रितयं जीविते फलम्}
{एतत्त्रयमवाप्तव्यमधर्मपरिवर्जितम् ॥ युधिष्ठिर उवाच}
{}


\twolineshloka
{श्रुतं भगवतो वाक्यं धर्मयुक्तं परं हितम्}
{शरीरनिश्चयं ज्ञातुं बुद्धिस्तु मम जायते}


\threelineshloka
{मृतं शरीरं हि नृणां सूक्ष्ममव्यक्ततां गतम्}
{अचक्षुर्विषयं प्राप्तं कथं धर्मोऽनुगच्छति ॥ बृहस्पतिरुवाच}
{}


\twolineshloka
{पृथिवी वायुराकाशमापो ज्योतिरनन्तरम्}
{बुद्धिरात्मा च सहिता धर्मं पश्यन्ति नित्यदा}


\twolineshloka
{प्राणिनामिह सर्वेषां साक्षिभूतं दिवानिशम्}
{एतैश्च सह धर्मोऽपि तं जीवमनुगच्छति}


\twolineshloka
{त्वगस्थि मांसं शुक्रं च शोणितं च महामते}
{शरीरं वर्जयन्त्येते जीवितेन विवर्जितम्}


\threelineshloka
{ततो धर्मसमायुक्तः स जीवः सुखमेधते}
{इह लोके परे चैव किं भूयः कथयामि ते ॥ युधिष्ठिर उवाच}
{}


\threelineshloka
{तद्दर्शितं भगवता यथा धर्मोऽनुगच्छति}
{एतत्तु ज्ञातुमिच्छामि कथं रेतः प्रवर्तते ॥ बृहस्पतिरुवाच}
{}


\twolineshloka
{अन्नमश्नन्ति यद्देवाः शरीरस्था नरेश्वर}
{पृथिवी वायुराकाशमापो ज्योतिर्मनस्तथा}


\twolineshloka
{ततस्तृप्तेषु राजेन्द्र तेषु भूतेषु पञ्चसु}
{मनःषष्ठेषु शुद्धात्मन्रेतः सम्पद्यते महत्}


\threelineshloka
{ततो गर्भः सम्भवति श्लेषात्स्त्रीपुंसयोर्नृप}
{एतत्ते सर्वमाख्यातं भूयः किं श्रोतुमिच्छसि ॥ युधिष्ठिर उवाच}
{}


\threelineshloka
{आख्यातं मे भगवता गर्भः सञ्जायते यथा}
{यथा जातस्तु पुरुषः प्रपद्यति तदुच्यताम् ॥ बृहस्पतिरुवाच}
{}


\threelineshloka
{आसन्नमात्रः पुरुषस्तैर्भूतैरभिभूयते}
{विप्रयुक्तश्च तैर्भूतैः पुनर्यात्यपरां गतिम्}
{स च भूतसमायुक्तः प्राप्नुते जीव एव हि}


\threelineshloka
{ततोऽस्य कर्म पश्यन्ति शुभं वा यदि वाशुभम्}
{देवताः पञ्चभूतस्थाः किं भूयः श्रोतुमिच्छसि ॥ युधिष्ठिर उवाच}
{}


\threelineshloka
{त्वगस्थिमांसमुत्सृज्य तैश्च भूतैर्विवर्जितः}
{जीवः सह वसन्कृत्स्नं सुखदुःखसहः प्रभो ॥ बृहस्पतिरुवाच}
{}


\twolineshloka
{`भोगवश्यं कर्मवश्यं यातनावश्यमित्यपि}
{एतत्त्रयाणामासाद्य कर्मतः सोऽश्नुते फलम् ॥'}


\twolineshloka
{जीवः कर्मसमायुक्तः शीघ्रं रेतस्त्वमागतः}
{स्त्रीणां पुष्पं समासाद्य सूतिकाले लभेत तत्}


\twolineshloka
{यमस्य पुरुषैः क्लेसं यमस्य पुरुषैर्वधम्}
{दुःखं संसारचक्रं च नरः क्लेशं स विन्दति}


\twolineshloka
{इह लोके स च प्राणी जन्मप्रभृति पार्थिव}
{सुकृतं कर्म वै भुङ्क्ते धर्मस्य फलमाश्रितः}


\twolineshloka
{यदि धर्मं यथाशक्ति जन्मप्रभृति सेवते}
{ततः स पुरुषो भूत्वा सेवते नियतं सुखम्}


\twolineshloka
{अथान्तरा तु धर्मस्याप्यधर्ममुपसेवते}
{सुखस्यानन्तरं दुःखं स जीवोऽप्यधिगच्छति}


\twolineshloka
{अधर्मेण समायुक्तो यमस्य विषयं गतः}
{महद्दुःखं समासाद्य तिर्यग्योनौ प्रजायते}


\twolineshloka
{कर्मणा येन येनेह यस्यां योनौ प्रजायते}
{जीवो मोहसमायुक्तस्तन्मे निगदतः शृणु}


\twolineshloka
{यदेतदुच्यते शास्त्रे सेतिहासे च च्छन्दसि}
{यमस्य विषयं घोरं मर्त्यो लोकः प्रपद्यते}


\twolineshloka
{इह स्थानानि पुण्यानि देवतुल्यानि भूपते}
{तिर्यग्योन्यतिरिक्तानि गतिमन्ति च सर्वशः}


\twolineshloka
{यमस्य भवने दिव्ये ब्रह्मलोकसमे गुणैः}
{कर्मभिर्नियतैर्बद्धो जन्तुर्दुःखान्युपाश्नुते}


\twolineshloka
{येनयेन तु भावेन कर्मणा पुरुषो गतिम्}
{प्रयाति परुषां घोरां तत्ते वक्ष्याम्यतः परम्}


\twolineshloka
{अधीत्य चतुरो वेदान्द्विजो मोहसमन्वितः}
{पतितात्प्रतिगृह्याथ खरयोनौ प्रजायते}


\twolineshloka
{खरो जीवति वर्षाणि दश पञ्च च भारत}
{खरो मृतो बलीवर्दः सप्तवर्षाणि जीवति}


\twolineshloka
{बलीवर्दो मृतश्चापि जायते ब्रह्मराक्षसः}
{ब्रह्मरक्षश्च मासांस्त्रींस्ततो जायेत ब्राह्मणः}


\twolineshloka
{पतितं याजयित्वा तु कृमियोनौ प्रजायते}
{तत्र जीवति वर्षाणि दश पञ्च च भारत}


\twolineshloka
{कृमिभावाद्विमुक्तस्तु ततो जायेत गर्दभः}
{गर्दभः पञ्चवर्षाणि पञ्चवर्षाणि सूकरः}


\twolineshloka
{कुक्कुटः पञ्चवर्षाणि पञ्चवर्षाणि जम्बुकः}
{श्वा वर्षमेकं भवति ततो जायेत मानवः}


\twolineshloka
{उपाध्यायस्त्रियः पापं शिष्यः कुर्यादबुद्धिमान्}
{स जीव इह संसारांस्त्रीनाप्नोति न संशयः}


\twolineshloka
{वृको भवति राजेन्द्र ततः क्रव्यात्ततः खरः}
{ततः प्रेतः परिक्लिष्टः पश्चाज्जायेत ब्राह्मणः}


\twolineshloka
{मनसाऽपि गुरोर्भार्यां यः शिष्यो याति पापकृत्}
{स उग्रान्प्रैति संसारानधर्मेणेह चेतसा}


\twolineshloka
{श्वयोनौ तु स सम्भूतस्त्रीणि वर्षाणि जीवति}
{तत्रापि निधनं प्राप्तः कृमियोनौ प्रजायते}


\twolineshloka
{कृमिभावमनुप्राप्तो वर्षमेकं तु जीवति}
{ततस्तु निधनं प्राप्तो ब्रह्मयोनौ प्रजायते}


\twolineshloka
{यदि पुत्रसमं शिष्यं गुरुर्हन्यादकारणे}
{आत्मनः कामकारेण सोपि हिंस्रः प्रजायते}


\twolineshloka
{पितरं मातरं चैव यस्तु पुत्रोऽवमन्यते}
{सोऽपि राजन्मृतो जन्तुः पूर्वं जायेत गर्दभः}


\twolineshloka
{गर्दभत्वं तु सम्प्राप्य दशवर्षाणि जीवति}
{संवत्सरं तु कुम्भीरस्ततो जायेत मानवः}


\twolineshloka
{पुत्रस्य मातापितरौ यस्य रुष्टावुभावपि}
{गुर्वपध्यानतः सोपि मृतो जायति गर्दभः}


\twolineshloka
{खरो जीवति मासांस्तु दश श्वा च चतुर्दशः}
{बिडालः सप्तमासांस्तु ततो जायेत मानवः}


\twolineshloka
{मातापितरावाक्रुश्य शारिका सम्प्रजायते}
{ताडयित्वा तु तावेव जायते कच्छपो नृप}


\twolineshloka
{कच्छपो दशवर्षाणि त्रीणि वर्षाणि शल्यकः}
{व्यालो भूत्वा च षण्मासांस्ततो जायति मानुषः}


\twolineshloka
{भर्तृपिण्डमुपाश्नन्यो राजद्विष्टानि सेवते}
{सोपि मोहसमापन्नो मृतो जायति वानरः}


\twolineshloka
{वानरो दशवर्षाणि पञ्चवर्षाणि सूकरः}
{श्वाऽथ भूत्वा तु षण्मासांस्ततो जायति मानुषः}


\twolineshloka
{न्यासापहर्ता तु नरो यमस्य विषयं गतः}
{यातनानां शतं गत्वा कृमियोनौ प्रजायते}


\twolineshloka
{तत्र जीवति वर्षाणि दशपञ्च च भारत}
{दुष्कृतस्य क्षयं कृत्वा ततो जायति मानुषः}


\twolineshloka
{असूयकः कुत्सितश्च चण्डालो दुःखमश्नुते}
{विश्वासहर्ता तु नरो मीनो जायति दुर्मतिः}


\twolineshloka
{भूत्वा मीनोऽष्टमासांस्तु मृगो जायति भारत}
{मृगस्तु चतुरो मासांस्ततश्छागः प्रजायते}


\twolineshloka
{छागस्तु निधनं प्राप्य पूर्णे संवत्सरे ततः}
{गौः स सञ्जायते जन्तुस्ततो जायति मानुषः}


\twolineshloka
{धान्यान्यवांस्तिलान्माषान्कुलत्थान्सर्षपांश्चणान्}
{कलायानथ मुद्गांश्च गोधूमानतसीस्तथा}


\twolineshloka
{यस्तु धान्यापहर्ता च मोहाज्जन्तुरचेतनः}
{स जायते महाराज मूषिको निरपत्रपः}


\twolineshloka
{ततः प्रेत्य महाराज मृतो जायति सूकरः}
{सूकरो जातमात्रस्तु रोगेणि म्रियते नृप}


\twolineshloka
{श्वा ततो जायते मूढः कर्मणा तेन पार्थिव}
{भूत्वा श्वा पञ्चवर्षाणि ततो जायति मानवः}


\twolineshloka
{परदाराभिमर्शं तु कृत्वा जायति वै वृकः}
{श्वा शृगालस्ततो गृध्रो व्यालः कङ्को बकस्तथा}


\twolineshloka
{भ्रातुर्भार्यां तु पापात्मा यो धर्मयति मोहितः}
{पुंस्कोकिलत्वमाप्नोति सोऽपि संवत्सरं नृप}


\twolineshloka
{सखिभार्यां गुरोर्भार्यां राजभार्यां तथैव च}
{प्रधर्षयित्वा कामाद्यो मृतो जायति सूकरः}


\twolineshloka
{सूकरः पञ्चवर्षाणि दशवर्षाणि श्वाविधः}
{बिडालः पञ्चवर्षाणि दशवर्षाणि कुक्कुटः}


\twolineshloka
{पिपीलिका तु मासांस्त्रीन्वानरो मासमेव तु}
{एतानासाद्य संसारान्कृमियोनौ प्रजायते}


\twolineshloka
{तत्र जीवति मासांस्तु कृमियोनौ चतुर्दश}
{ततोऽधर्मक्षयं कृत्वा पुनर्जायति मानवः}


\twolineshloka
{उपस्थिते विवाहे तु यज्ञे दानेऽपि वा विभो}
{मोहात्करोति यो विघ्नं स मृतो जायते कृमिः}


\twolineshloka
{कृमिर्जीवति वर्षाणि दश पञ्च च भारत}
{अधर्मिस्य क्षयं कृत्वा ततो जायति मानवः}


\twolineshloka
{पूर्वं दत्त्वा तु यः कन्यां द्वितीये दातुमिच्छति}
{सोपि राजन्मृतो जन्तुः कृमियोनौ प्रजायते}


\twolineshloka
{तत्र जीवति वर्षाणि त्रयोदश युधिष्ठिर}
{अधर्मसंक्षये युक्तस्ततो जायति मानवः}


\twolineshloka
{देवकार्यमकृत्वा तु पितृकार्यमथापि वा}
{अनिर्वाप्य समश्नन्वै मृतो जायति वायसः}


\twolineshloka
{वायसः शतवर्षाणि ततो जायति कुक्कुटः}
{जायते व्यालकश्चापि मासं तस्मात्तु मानुषः}


\twolineshloka
{ज्येष्ठं पितृसमं चापि भ्रातरं योऽवमन्यते}
{सोऽपि मृत्युमुपागम्य क्रौञ्चयोनौ प्रजायते}


\twolineshloka
{क्रौञ्चो जीवति वर्षं तु ततो जायति चीरकः}
{ततो निधनमापन्नो मानुषत्वमुपाश्नुते}


\twolineshloka
{वृषलो ब्राह्मणीं गत्वा कृमियोनौ प्रजायते}
{[ततः सम्प्राप्य निधनं जायते सूकरः पुनः}


\twolineshloka
{सूकरो जातमात्रस्तु रोगेण म्रियते नृप}
{श्वा ततो जायते मूढः कर्मणा तेन पार्थिव}


\twolineshloka
{श्वा भूत्वा कृतकर्माऽसौ जायते मानुषस्ततः}
{तत्रापत्यं समुत्पाद्य मृतो जायति मूषिकः}


\twolineshloka
{कृतघ्नस्तु मृतो राजन्यमस्य विषयं गतः}
{यमस्य पुरुषैः क्रुद्धैर्वधं प्राप्नोति दारणम्}


\twolineshloka
{दण़्डं समुद्गरं शूलमग्निकुम्भं च दारुणम्}
{असिपत्रवनं घोरवालुकं कूटशाल्मलीम्}


\twolineshloka
{एताश्चान्याश्च बह्वीश्च यमस्य विषयं गतः}
{यातनाः प्राप्य तत्रोग्रास्ततो वध्यति भारत}


\twolineshloka
{ततो हतः कृतघ्नः स तत्रोग्रैर्भरतर्षभ}
{संसारचक्रमासाद्य कृमियोनौ प्रजायते}


\twolineshloka
{कृमिर्भवति वर्षाणि दशपञ्च च भारत}
{ततो गर्भं समासाद्य तत्रैव म्रियते शिशुः}


\twolineshloka
{ततो गर्भशतैर्जन्तुर्बहुभि सम्प्रपद्यते}
{संसारांश्च बहून्गत्वा ततस्मिर्यक्षु जायते}


\twolineshloka
{ततो दुःखमनुप्राप्य बहुवर्षगणानिह}
{स पुनर्भवसंयुक्तस्ततः कूर्मः प्रजायते}


\twolineshloka
{दधि हृत्वा बकश्चापि प्लवो मत्स्यानसंस्कृतान्}
{चोरयित्वा तु दुर्बुद्धिर्मधुदंशः प्रजायते}


\twolineshloka
{फलं वा मूलकं हृत्वा अपूपं वा पिपीलिकाः}
{चोरयित्वा च निष्पावं जायते हलगोलकः}


\twolineshloka
{पायसं चोरयित्वा तु तित्तिरित्वमवाप्नुते}
{हृत्वा पिष्टमयं पूपं कुम्भोलूकः प्रजायते}


\twolineshloka
{अयो हृत्वा तु दुर्बुद्धिर्वायसो जायते नरः}
{कांस्यं हृत्वा तु दुर्बद्धिर्हारितो जायते नरः}


\twolineshloka
{राजतं भाजनं हृत्वा कपोतः सम्प्रजायते}
{हृत्वा तु काञ्चनं भाण्डं कृमियोनौ प्रजायते}


\twolineshloka
{पत्रोर्णं चोरयित्वा तु कृकलत्वं निगच्छति}
{कौशिकं तु ततो हृत्वा नरो जायति वर्तकः}


\twolineshloka
{अंशुकं चोरयित्वा तु शुक्रो जायति मानवः}
{चोरयित्वा दुकूलं तु मृतो हंसः प्रजायते}


\threelineshloka
{क्रौञ्चः कार्पासिकं हृत्वा मृतो जायति मानवः}
{चोरयित्वा नरः पट्टं त्वाविकं चैव भारत}
{क्षौमं च वस्त्रमादायशशो जन्तुः प्रजायते}


\twolineshloka
{वर्णान्हृत्वा तु पुरुषो मृतो जायति बर्हिणः}
{हृत्वा रक्तानि वस्त्राणि जायते जीवजीवकः}


\twolineshloka
{वर्णकादींस्तथा गन्धांश्चोरयित्वेह मानवः}
{छुन्दुन्दरित्वमाप्नोति राजँल्लोभपरायणः}


% Check verse!
तत्र जीवति वर्षाणि ततो दश च पञ्च च

अधर्मस्य क्षयं गत्वा ततो जायति मानुषः ॥ 109a चोरयित्वापयश्चापि बलाका सम्प्रजायते ॥ 110a यस्तु चोरयते तैलं नरो मोहसमन्वितः

110b सोपि राजन्मृतो जन्तुस्तैलपायी प्रजायते ॥ 111a अशस्त्रं पुरुषंहत्वा सशस्त्रः पुरुषाधमः

111b अर्थार्थी यदि वा वैरी स मृतो जायतेस्वरः ॥ 112a खरो जीवति वर्षे द्वे ततः शस्त्रेण वध्यते

112b स मृतोमृगयोनौ तु नित्योद्विग्नोऽभिजायते ॥ 113a मृगो वध्यति शस्त्रेण गतेसंवत्सरे तु सः

113b हतो मृगस्ततो मीनः सोपि जालेन बध्यते ॥ 114a मासेचतुर्थे सम्प्राप्ते श्वापदः सम्प्रजायते

114b श्वापदो दशवर्षाणिद्वीपी वर्षाणि पञ्च च ॥ 115a ततस्तु निधनं प्राप्तः कालपर्यायचोदितः

115b अधर्मस्य क्षयं कृत्वा ततो जायति मानुषः ॥ 116a स्त्रियं हत्वा तुदुर्बुद्धिर्यमस्य विषयं गतः

116b बहून्क्लेशान्समासाद्यनरकानेकविंशतिम् ॥ 117a ततः पश्चान्महाराज कृमियोनौ प्रजायते

117bकृमिर्विंशतिवर्षाणि भूत्वा जायति मानुषः ॥ 118a भोजनं चोरयित्वा तुमक्षिका जायते नरः

118b मक्षिकासङ्घवशगो बहून्मासान्भवत्युत ॥ 119aततः पापक्षयं कृत्वा मानुषत्वमवाप्नुते

119b `भक्ष्यं हृत्वा तुपुरुषो जालपाशः प्रजायते ॥ 120a स्वाद्यं हृत्वा तु पुरुषश्चीरपाशःप्रजायते

' 120b धान्यं हृत्वा तु पुरुषो लोमशः सम्प्रजायते ॥ 121a तथापिण्याकसम्मिश्रमशनं चोरयेन्नरः

121b स जायते भृतिधनो दारुणो मूषिकोनरः ॥ 122a दशन्वै मानुषान्नित्यं पापात्मा स विशाम्पते

122b घृतंहृत्वा तु दुर्बुद्धिः काकमद्गुः प्रजायते ॥ 123a मत्स्यमांसमथो हृत्वाकाको जायति दुर्मतिः

123b लवणं चोरयित्वा तु चिरिकाकः प्रजायते

124aविश्वासेन तु निक्षिप्तं यो विनिह्नोति मानवः

124b स गतायुर्नरस्तातमत्स्ययोनौ प्रजायते ॥ 125a मत्स्ययोनिमनुप्राप्य मृतो जायति मानुषः

125b मानुषत्वमनुप्राप्य क्षीणायुरुपपद्यते ॥ 126a पापानि तु नराःकृत्वा तिर्यग्जायन्ति भारत

126b न चात्मनः प्रयाणान्ते धर्मं जानन्तिकञ्चन ॥ 127a ये पापानि नराः कृत्वा निरस्यन्ति व्रतैः सदा

127bसुखदुःखसमायुक्ता व्यथितास्ते भवन्त्युत ॥ 128a अपुमांसः प्रजायन्तेम्लेच्छाश्चापि न संशयः

128b नराः पापसमाचारा लोभमोहसमन्विताः ॥ 129aवर्जयन्ति च पापानि जन्मप्रभृति ये नराः

129b अरोगा रूपवन्तस्तेधनिनश्च भवन्त्युत ॥ 130a स्त्रियोऽप्येतेन कल्पेन कृत्वापापमवाप्नुयुः

130b एतेषामेव जन्तूनां भार्यात्वमुपयान्ति ताः ॥ 131aपरस्वहरणे दोषाः सर्व एव प्रकीर्तिताः

131b एतद्धि लेशमात्रेण कथितंते मयाऽनघ ॥ 132a अपरस्मिन्कथायोगे भूयः श्रोष्यसि भारत

132b एतन्मयामहाराज ब्रह्मणो गदतः पुरा ॥ 133a सुरर्षीणां श्रुतं मध्ये पृष्टश्चापियथातथम्

133b मयाऽपि तच्च कार्त्स्न्येन यथावदनुवर्णितम्

133cएतच्छ्रुत्वा महाराज धर्मे कुरु मनः सदा
\chapter{अध्यायः १७४}
\twolineshloka
{अधर्मस्य गतिर्ब्रह्मन्कथिता मे त्वयाऽनघ}
{धर्मस्य तु गतिं श्रोतुमिच्छामि वदतांवर}


\threelineshloka
{कृत्वा कर्माणि पापानि कथं यान्ति शुभां गतिम्}
{कर्मणा च कृतेनेह केन यान्ति शुभां गतिम् ॥बृहस्पतिरुवाच}
{}


\twolineshloka
{कृत्वा पापानि कर्माणि अधर्मवशमागतः}
{मनसा विपरीतेन निरयं प्रतिपद्यते}


\twolineshloka
{मोहादधर्मं यः कृत्वा पुनः समनुतप्यते}
{मनःसमाधिसंयुक्तो न स सेवेत दुष्कृतम्}


\twolineshloka
{[यथायथा मनस्तस्य दुष्कृतं कर्म गर्हते}
{तथातथा शरीरं तु तेनाधर्मेण मुच्यते}


\twolineshloka
{यदि व्याहरते राजन्विप्राणां धर्मवादिनाम्}
{ततोऽधर्मकृतात्क्षिप्रमपवादात्प्रमुच्यते ॥]}


\threelineshloka
{यथायथा नरः सम्यगध्रममनुभाषते}
{समाहितेन मनसा विमुच्येत तथातथा}
{भुजङ्ग इव निर्मोकात्पूर्वभुक्ताज्जरान्वितात्}


\twolineshloka
{दत्त्वा विप्रस्य दानानि विविधानि समाहितः}
{मनःसमाधिसंयुक्तः सुगतिं प्रतिपद्यते}


\twolineshloka
{प्रदानानि तु वक्ष्यामि यानि दत्त्वा युधिष्ठिर}
{नरः कृत्वाऽप्यकार्याणि ततो धर्मेण युज्यते}


\twolineshloka
{सर्वेषामेव दानानामन्नं श्रेष्ठमुदाहृतम्}
{पूर्वमन्नं प्रदातव्यमृजुना धर्ममिच्छता}


\twolineshloka
{प्राणा ह्यन्नं मनुष्याणां तस्माज्जन्तुश्च जायते}
{अन्ने प्रतिष्ठितो लोकस्तस्मादन्नं प्रशस्यते}


\twolineshloka
{अन्नमेव प्रशंसन्ति देवर्षिपितृमानवाः}
{अन्नस्य हि प्रदानेन स्वर्गमाप्नोति मानवः}


\twolineshloka
{न्यायलब्धं प्रदातव्यं द्विजातिभ्योऽन्नमुत्तमम्}
{स्वाध्यायसमुपेतेभ्यः प्रहृष्टेनान्तरात्मना}


\twolineshloka
{यस्य ह्यन्नमुपाश्नन्ति ब्राह्मणानां शतं दश}
{हृष्टेन मनसा दत्तं न स तिर्यग्गतिर्भवेत्}


\twolineshloka
{ब्राह्मणानां सहस्राणि दश भोज्य नरर्षभ}
{नरोऽधर्मात्प्रमुच्येत योगेष्वभिरतः सदा}


\twolineshloka
{भैक्ष्येणान्नं समाहृत्य दद्यादन्नं द्विजेषु वै}
{सुवर्णदानात्पापानि नश्यन्ति सुबहून्यपि}


\twolineshloka
{दत्त्वा वृत्तिकरीं भूमिं पातकेनापि मुच्यते}
{पारायणैश्च वेदानां मुच्यते पातकैर्द्विजः}


\twolineshloka
{गायत्र्याश्चैव लक्षेण गोसहस्रस्य तर्पणात्}
{वेदार्थं ज्ञापयित्वा तु शुद्धान्विप्रान्यथार्थतः}


\twolineshloka
{सर्वत्यागादिभिश्चैव मुच्यते पातकैर्द्विजः}
{सर्वातिथ्यं परं ह्येषां तस्माद्दानं परं स्मृतम्}


\twolineshloka
{अहिंसन्ब्राह्मणस्वानि न्यायेन परिपाल्य च}
{क्षत्रियस्तरसा प्राप्तमन्नं यो वै प्रयच्छति}


\twolineshloka
{द्विजेभ्यो वेदवृद्धेभ्यः प्रयतः सुसमाहितः}
{तेनापोहति धर्मात्मन्दुष्कृतं कर्म पाण्डव}


\twolineshloka
{षड्भागपरिशुद्धं च कृषेर्भागमुपार्जितम्}
{वैश्योऽददद्द्विजातिभ्यः पापेभ्यः परिमुच्यते}


\twolineshloka
{अवाप्य प्राणसंदेहं कार्कश्येन समार्जितम्}
{अन्नं दत्त्वा द्विजातिभ्यः शूद्रः पापात्प्रमुच्यते}


\twolineshloka
{औरसेन बलेनान्नमर्जयित्वाऽविहिंसकः}
{यः प्रयच्छति विप्रेभ्यो न स दुर्गाणि पश्यति}


\twolineshloka
{न्यायेनैवाप्तमन्नं तु नरो हर्षसमन्वितः}
{द्विजेभ्यो वेदवृद्धेभ्यो दत्त्वा पापात्प्रमुच्यते}


\twolineshloka
{अन्नमूर्जस्करं लोके दत्त्वोर्जस्वी भवेन्नरः}
{सतां पन्थानमावृत्य सर्वपापैः प्रमुच्यते}


\twolineshloka
{`शूद्रान्नं नैव भोक्तव्यं विप्रैर्धर्मपरायणैः}
{आपद्येव स्वदासानां भोक्तव्यं स्वयमुद्यतैः}


\twolineshloka
{दानवद्भिः कृतः पन्था येन यान्ति मनीषिणः}
{ते हि प्राणस्य दातारस्तेभ्यो धर्मः सनातनः}


\twolineshloka
{सर्वावस्थं मनुष्येण न्यायेनान्नमुपार्जितम्}
{कार्यं पात्रागतं नित्यमन्नं हि परमा गतिः}


\twolineshloka
{अन्नस्य हि प्रदानेन नरो रौद्रं न सेवते}
{तस्मादन्नं प्रदातव्यमन्यायपरिवर्जितम्}


\twolineshloka
{यतेद्ब्राह्मणपूर्वं हि भोक्तुमन्नं गृही सदा}
{अवन्ध्यं दिवसं कुर्यादन्नपानीयदानतः}


\twolineshloka
{भोजयित्वा दशशतं नरो वेदविदां नृप}
{न्यायविद्धर्मविदुषामितिहासविदां तथा}


\twolineshloka
{न याति नरकं घोरं संसारांश्च न सेवते}
{सर्वकामसमायुक्तः प्रेत्य चाप्यश्नुते सुखम्}


\twolineshloka
{एवं सुखसमायुक्तो रमते विगतज्वरः}
{रूपवान्कीर्तिमांश्चैव धनवांश्चोपपद्यते}


\twolineshloka
{एतत्ते सर्वमाख्यातमन्नदानफलं महूत्}
{मूलमेतत्तु धर्माणां प्रदानानां च भारत}


\chapter{अध्यायः १७५}
\threelineshloka
{अहिंसा वैदिकं कर्म ध्यानमिन्द्रियसंयमः}
{तपोऽथ गुरुशुश्रूषा किं श्रेयः पुरुषं प्रति ॥बृहस्पतिरुवाच}
{}


\twolineshloka
{सर्वाण्येतानि धर्मस्य पृथग्द्वाराणि नित्यशः}
{शृणु संकीर्त्यमानानि षडेव भरतर्षभ}


\twolineshloka
{हन्त निःश्रेयसं जन्तोरहं वक्ष्याम्यनुत्तमम्}
{अहिंसापाश्रयं धर्मं दान्तो विद्वान्समाचरेत्}


\twolineshloka
{त्रिदण्डं सर्वभूतेषु निधाय पुरुषः शुचिः}
{कामक्रोधौ च संयम्य ततः सिद्धिमवाप्नुते}


\twolineshloka
{अहिंसकानि भूतानि दण्डेन विनिहन्ति यः}
{आत्मनः सुखमन्विच्छन्स प्रेत्य न सुखी भवेत्}


\twolineshloka
{आत्मोपमस्तु भूतेषु यो वै भवति पूरुषः}
{त्यक्तदण्डो जितक्रोधः स प्रेत्य सुखमेधते}


\twolineshloka
{सर्वभूतात्मभूतस्य सर्वभूतानि पश्यतः}
{देवाऽपि मार्गे मुह्यन्ति ह्यपदस्य पदेषिणः}


\twolineshloka
{न तत्परस्य संदध्यात्प्रतिकूलं यदात्मनः}
{एष साङ्ग्राहिको धर्मः कामादन्यः प्रवर्तते}


\twolineshloka
{प्रख्यापने च दाने च सुखदुःखे प्रियाप्रिये}
{आत्मौपम्येन पुरुषः प्रमाणमधिगच्छति}


\fourlineindentedshloka
{यथा परः प्रक्रमते परेषुतथा परे प्रक्रमन्ते परस्मिन्}
{निषेवते स्वसमां जीवलोकेयथा धर्मो नैपुणेनोपदिष्टः}
{वैशम्पायन उवाच}
{}


\twolineshloka
{इत्युक्त्वा तं सुरगुरुधर्मराजं युधिष्ठिरम्}
{दिवामचक्रमे धीमान्पश्यतामेव नस्तदा}


\chapter{अध्यायः १७६}
\threelineshloka
{ततो युधिष्ठिरो राजा शरतल्पे पितामहम्}
{पुनरेव महाराज पप्रच्छ वदतांवरः ॥युधिष्ठिर उवाच}
{}


\twolineshloka
{ऋषयो ब्राह्मणा देवाः प्रशंसन्ति महामते}
{अहिंसालक्षणं धर्मं वेदप्रामाण्यदर्शनात्}


\threelineshloka
{कर्मणा न नरः कुर्वन्हिंसां पार्थिवसत्तम}
{वाचा च मनसा चैवं ततो दुःखात्प्रमुच्यते ॥भीष्म उवाच}
{}


\twolineshloka
{चतुर्विधेयं निर्दिष्टा ह्यहिंसा ब्रह्मवादिभिः}
{एकैकतोऽपि विभ्रष्टा न भवत्यरिसूदन}


\twolineshloka
{यथा सर्वश्चतुष्पाद्वै त्रिभिः पादैर्न तिष्ठति}
{तथैवेयं महीपाल कारणैः प्रोच्यते त्रिभिः}


\twolineshloka
{यथा नागपदेऽन्यानि पदानि पदगामिनाम्}
{सर्वाण्येवापिधीयन्ते पदजातानि कौञ्जरे}


\twolineshloka
{एवं लोकेष्वहिंसा तु निर्दिष्टा धर्मतः पुरा}
{कर्म्णा लिप्यते जन्तुर्वाचा च मनसाऽपि च}


\twolineshloka
{पूर्वं तु मनसा त्यक्त्वा त्यजेद्वाचाऽथ कर्मणा}
{`हिंसां तु नोपयुञ्जीत तथा हिंसा चतुर्विधा}


\twolineshloka
{काये मनसि वाक्येऽपि दोषा ह्येते प्रकीर्तिताः'}
{[न भक्षयति यो मांसं त्रिविधं स विमुच्यते}


\twolineshloka
{त्रिकारणं तु निर्दिष्टं श्रूयते ब्रह्मवादिभिः}
{मनो वाचि तथाऽऽस्वादे दोषा ह्येषु प्रतिष्ठिताः ॥]}


\twolineshloka
{न भक्षयन्त्यतो मांसं तपोयुक्ता मनीषिणः}
{दोषांस्तु भक्षणे राजन्मांसस्येह निबोध मे}


\twolineshloka
{पुत्र मांसोपमं जानन्खादते यो विचेतनः}
{मांसं मोहसमायुक्तः पुरुषः सोऽधमः स्मृतः}


\twolineshloka
{पितृमातृसमायोगे पुत्रत्वं जायते यथा}
{हिंसां कृत्वाऽवशः पापो भूयिष्ठं जायते तथा}


\twolineshloka
{रसश्च हृदि जिह्वाया ज्ञानं प्रज्ञायते यथा}
{तथा शास्त्रेषु नियतं रागो ह्यास्वादिताद्भवेत्}


\twolineshloka
{संस्कृतासंस्कृताः पक्वा लवणालवणास्तथा}
{प्रजायन्ते यथा भावास्तथा चित्तं निरुध्यते}


\twolineshloka
{भेरीमृदङ्गशब्दांश्च तन्त्रीशब्दांश्च पुष्कलान्}
{निषेविष्यन्ति वै मन्दा मांसभक्षाः कथं नराः}


\twolineshloka
{`परेषां धनधान्यानां हिसकाः स्तावकास्तथा}
{प्रशंसकाश्च मांसस्य नित्यं स्वर्गे बहिष्कृताः ॥'}


\twolineshloka
{अचिन्तितमनिर्दिष्टमसङ्कल्पितमेव च}
{रसगृद्ध्याऽभिभूता ये प्रशंसन्ति फलार्थिनः}


% Check verse!
प्रशंसा ह्येव मांसस्य दोषकल्पफलान्विता
\twolineshloka
{`भस्म विष्ठा कृमिर्वाऽपि निष्ठा यस्येदृशी ध्रुवा}
{स कायः परपीडाभिः कथं धार्योविपश्चिता ॥'}


\twolineshloka
{जीवितं हि परित्यज्य बहवः साधवो जनाः}
{स्वमांसैः परमांसानि परिपाल्य दिवं गताः}


\twolineshloka
{एवमेषा महाराज चतुर्भिः कारणैः स्मृता}
{अहिंसा तव निर्दिष्टा सर्वधर्मानुसंहिता}


\chapter{अध्यायः १७७}
\twolineshloka
{अहिंसा परमो धर्म इत्युक्तं बहुशस्त्वया}
{श्राद्धेषु च भवानाह पितॄणां मांसमीप्सितम्}


\twolineshloka
{मांसैर्बहुविधैः प्रोक्तस्त्वया श्राद्धविधिः पुरा}
{अहत्वा च कुतो मांसमेवमेतद्विरुध्यते}


\twolineshloka
{जातो नः संशयो धर्मे मांसस्य परिवर्जने}
{दोषो भक्षयत कः स्यात्कश्चाभक्षयतो गुणः}


\twolineshloka
{हत्वा भक्षयतो वाऽपि परेणोपहृतस्य वा}
{हन्याद्वा यः परस्यार्थे क्रीत्वा वा भक्षयेन्नरः}


\twolineshloka
{एतदिच्छामि तत्त्वेन कथ्यमानं त्वयाऽनघ}
{निश्चयेन चिकीर्षामि धर्ममेतं सनातनम्}


\threelineshloka
{कथमायुरवाप्नोति कथं भवति सत्ववान्}
{कथमव्यङ्गतामेति लक्षण्यो जायते कथम् ॥भीष्म उवाच}
{}


\twolineshloka
{मांसस्याभक्षणाद्राजन्यो धर्मः कुरुनन्दन}
{तं मे शृणु यथातत्त्वं यश्चास्य विधिरुत्तमः}


\twolineshloka
{रूपमव्यङ्गतामायुर्बुद्धिं सत्वं बलं स्मृतिम्}
{प्राप्तुकामैर्नरैर्हिंसा वर्जनीया कृतात्मभिः}


\twolineshloka
{ऋषीणामत्र संवादो बहुशः कुरुनन्दन}
{बभूव तेषां तु मतं यत्तच्छृणु युधिष्ठिर}


\twolineshloka
{यो यजेताश्वमेधेन मासिमासि यतव्रतः}
{वर्जयेन्मधु मांसं च सममेतद्युधिष्ठिर}


\twolineshloka
{सप्तर्षयो वालखिल्यास्तथैव च मरीचिपाः}
{मांसस्याभक्षणं राजन्प्रशंसन्ति मनीषिणः}


\twolineshloka
{न भक्षयति यो मांसं न च हन्यान्न घातयेत्}
{तन्मित्रं सर्वभूतानां मनुः स्वायंभुवोऽब्रवीत्}


\twolineshloka
{अधृष्यः सर्वभूतानां विश्वास्यः सर्वजन्तुषु}
{साधूनां सम्मतो नित्यं भवेन्मांसं विवर्जयन्}


\twolineshloka
{स्वमांसं परमांसेन यो वर्धयितुमिच्छति}
{अविश्वास्योऽवसीदेत्स इति होवाच नारदः}


\twolineshloka
{ददाति यजते चापि तपस्वी च भवत्यपि}
{मधुमांसनिवृत्त्येति प्राह चैवं बृहस्पतिः}


\twolineshloka
{मासिमास्यश्वमेधेन यो यजेत शतं समाः}
{न खादति च यो मांसं सममेतन्मतं मम}


\twolineshloka
{सदा यजति सत्रेण सदा दानं प्रयच्छति}
{सदा तपस्वी भवति मद्यमांसविवर्जनात्}


\twolineshloka
{सर्वे वेदा न तत्कुर्युः सर्वे यज्ञाश्च भारत}
{यो भक्षयित्वा मांसानि पश्चादपि निवर्तते}


\threelineshloka
{`भक्षयित्वा निमित्तेऽपि दुष्करं कुरुते तपः}
{'दुष्करं च रसज्ञेन मांसस्य परिवर्जनम्}
{चर्तुं व्रतमिदं श्रेष्ठं सर्वप्राण्यभयप्रदम्}


\threelineshloka
{सर्वभूतेषु यो विद्वान्ददात्यभयदक्षिणाम्}
{दाता भवति लोके स प्राणानां नात्र संशयः}
{एवं वै परमं धर्मं प्रशंसन्ति मनीषिणः}


\twolineshloka
{प्राणा यथाऽऽत्मनोऽभीष्टा भूतानामपि वै तथा}
{आत्मौपम्येन गन्तव्यं बुद्धिमद्भिः कृतात्मभिः}


\threelineshloka
{विकीर्णकर्णकेनापि तृणमस्पन्दने भयम्}
{किं पुनर्हन्यमानानां तरसा जीवितार्थिनाम्}
{अरोगाणामपापानां पापैर्मांसोपजीविभिः}


\twolineshloka
{मृत्युतो भयमस्तीति शङ्कायां दुःखमुत्तरम्}
{धर्मस्यायतनं तस्मान्मांसस्य परिवर्जनम्}


\twolineshloka
{अहिंसा परमो धर्मस्तथाऽहिंसा परं तपः}
{अहिंसा परमं सत्यं ततो धर्मः प्रवर्तते}


\twolineshloka
{न हि मांसं तृणात्काष्ठादुपलाद्वाऽपि जायते}
{हत्वा जन्तुं ततो मांसं तस्माद्दोषस्तु भक्षणे}


\twolineshloka
{स्वाहास्वाधामृतभुजो देवाः सत्यार्जवप्रियाः}
{राक्षसेन्द्रभयान्मुक्ताः सर्वभूतपरायणाः}


\twolineshloka
{कान्तारेष्वथ घोरेषु दुर्गेषु गहनेषु च}
{रात्रावहनि सन्ध्यासु चत्वरेषु सभासु च}


\twolineshloka
{उद्यतेषु च शस्त्रेषु मृगव्यालहतेषु च}
{अमांसभक्षणाद्राजन्न भयं तेषु विद्यते}


\twolineshloka
{शरण्यः सर्वभूतानां विश्वास्यः सर्वजन्तुषु}
{अनुद्वेगकरो लोके न चाप्युद्विजते सदा}


\twolineshloka
{यदि चेत्स्वादको न स्यान्न तदा घातको भवेत्}
{घातकः खादकार्थाय तद्धातयति वै नरः}


\twolineshloka
{अभक्ष्यमेतदिति वै इति हिंसा निवर्तते}
{खादकक्रमतो हिंसा मृगादीनां प्रवर्तते}


\twolineshloka
{यस्माद्ग्रसति चैवायुर्हिंसकानां महाद्युते}
{तस्माद्विवर्जयेन्मांसं य इच्छेद्भूतिमात्मनः}


\twolineshloka
{त्रातारं नाधिगच्छन्ति रौद्राः प्राणिविहिंसकाः}
{उद्वेजकास्तु भूतानां यथा व्यालमृगास्तथा}


\twolineshloka
{लोभाद्वा बुद्धिमोहाद्वा बलवीर्यार्थमेव च}
{संसर्गादथ पापानामधर्मो रुचितो नृणाम्}


\twolineshloka
{स्वमांसं परमांसेन यो वर्धयितुमिच्छति}
{उद्विग्नराष्ट्रे वसति यत्र यत्राभिजायते}


\twolineshloka
{धन्यं यशस्यामायुष्यं स्वर्ग्यं स्वस्त्ययनं महत्}
{मांसस्याभक्षणं प्राहुर्नियताः परमर्षयः}


\twolineshloka
{इदं तु खलु कौन्तेय श्रुतमासीत्पुरा मया}
{मार्कण्डेयस्य वदतो ये दोषा मांसभक्षणे}


\twolineshloka
{यो हि खादति सांसानि प्राणिनां जीवितैषिणाम्}
{सदा भवति वै पापः प्राणिहन्ता तथैव सः}


\twolineshloka
{धनेन क्रयिको हन्ति खादकश्चोपभोगतः}
{घातको वधबन्धाभ्यामित्येष त्रिविधो वधः}


\twolineshloka
{अखादन्ननुमोदंश्च भावदोषेण मानवः}
{योऽनुमोदति हन्यन्तं सोऽपि दोषेण लिप्यते}


\twolineshloka
{अधृष्यः सर्वभूतानामायुष्मान्निरुजः सदा}
{भवत्यभक्षयन्मासं दयावान्प्राणिनामिह}


\twolineshloka
{हिरण्यदानैर्गोदानैर्भूमिदानैश्च सर्वशः}
{मांसस्याभक्षणे धर्मो विशिष्ट इति नः श्रुतिः}


\twolineshloka
{अप्रोक्षितं वृथामासं विधिहीनं न भक्षयेत्}
{भक्षयन्निरयं याति नरो नास्त्यत्र संशयः}


\twolineshloka
{प्रोक्षिताभ्युक्षितं मांसं तथा ब्राह्म्णकाम्यया}
{अल्पदोषमिति ज्ञेयं विपरिते तु लिप्यते}


\twolineshloka
{खादकस्य कृते जन्तून्यो हन्यात्पुरुषाधमः}
{महादोषकरस्तत्र घातको न तु खादकः}


\twolineshloka
{इज्यायज्ञश्रुतिकृतैर्या मार्गैरबुधोऽधमः}
{हन्याज्जन्तून्मांसगृध्नुः स वै नरकभाङ्नरः}


\twolineshloka
{भक्षयित्वाऽपि यो मांसं पश्चादपि निवर्तते}
{तस्यापि सुमहान्धर्मो यः पापाद्विनिवर्तते}


\twolineshloka
{आहर्ता चानुमन्ता च विशस्ता क्रयविक्रयी}
{संस्कर्ता चोपभोक्ता च घातकाः सर्व एव ते}


\twolineshloka
{इदमन्यत्तु वक्ष्यामि प्रमाणं विधिनिर्मितम्}
{पुराणमुषिभिर्जुष्टं वेदेषु परिनिश्चितम्}


\twolineshloka
{प्रवृत्तिलक्षणे धर्मे फलार्थिभिरभिद्रुते}
{यथोक्तं राजशार्दूल न तु तन्मोक्षकारणम्}


\threelineshloka
{हविर्यत्संस्कृतं मन्त्रैः प्रोक्षिताभ्युक्षितं शुचि}
{वेदोक्तेन प्रमाणेन पितॄणां प्रक्रियासु च}
{प्रवृत्तिधर्मिणा भक्ष्यं नान्यथा मनुरब्रवीत्}


\twolineshloka
{अस्वर्ग्यमयशस्यं च रक्षोवद्भरतर्षभ}
{विधिहीनं नरः पूर्वं मांसं राजन्न भक्षयेत्}


\twolineshloka
{य इच्छेत्पुरुषोऽत्यन्तमात्मानं निरुपद्रवम्}
{स वर्जयेत मांसानि प्राणिनामिह सर्वशः}


\twolineshloka
{श्रूयते हि पुराकल्पे नृणां व्रीहिमयः पशुः}
{येनायजन्त विद्वांसः पुण्यलोकपरायणाः}


\twolineshloka
{ऋषिभिः संशयं पृष्टो वसुश्चेदिपतिः पुरा}
{अभक्ष्यमपि मांसं यः प्राह भक्ष्यमिति प्रभो}


\twolineshloka
{आकाशादवनिं प्राप्तस्ततः स पृथिवीपतिः}
{यस्तदेव पुनश्चोक्त्वा विवेश धरणीतलम्}


\twolineshloka
{प्रजानां हितकामेन त्वगस्त्येन महात्मना}
{आरण्याः सर्वदैवत्याः प्रोक्षितास्तापसैर्मृगाः}


\twolineshloka
{क्रिया ह्येवं न हीयन्ते पितृदैवतसंश्रिताः}
{प्रीयन्ते पितरश्चैव न्यायतो मांसतर्पिताः}


\twolineshloka
{इदं तु शृणु राजेन्द्र मांसस्याभक्षणे गुणाः}
{[अभक्षणे सर्वसुखं मांसस्य मनुजाधिप ॥]}


\twolineshloka
{यस्तु वर्षशतं पूर्णं तपस्तप्येत्सुदारुणम्}
{यश्चैव वर्जयेन्मांसं सममेतन्मतं मम}


\twolineshloka
{कौमुद्यास्तु विशेषेण शुक्लपक्षे नराधिप}
{वर्जयेत्सर्वमांसानि धर्मो ह्यत्र विधीयते}


\twolineshloka
{[चतुरो वार्षिकान्मासान्यो मांसं परिवर्जयेत्}
{चत्वारि भद्राण्याप्नोति कीर्तिमायुर्यशो बलम्}


\twolineshloka
{अथवा मासमेकं वै सर्वमांसान्यभक्षयन्}
{अतीत्य सर्वदुःखानि सुखं जीवेन्निरामयः}


\twolineshloka
{वर्जयन्ति हि मांसानि मासशः पक्षशोपि वा}
{तेषां हिंसानिवृत्तानां ब्रह्मलोको विधीयते ॥]}


\twolineshloka
{मांसं तु कौमुदं पक्षं वर्जितं पार्थ राजभिः}
{सर्वभूतात्मभूतस्थैर्विदितार्थपरावरैः}


\twolineshloka
{नाभागेनाम्बरीषेण गयेन च महात्मना}
{आयुषाऽथानरण्येन दिलीपरघुसूनुभिः}


\twolineshloka
{कार्तवीर्यानिरुद्धाभ्यां नहुषेण ययातिना}
{नृगेण विष्वगश्वेन तथैव शशबिन्दुना}


\twolineshloka
{युवनाश्वेन च तथा शिबिनौशीनरेण च}
{मुचुकुन्देन मान्धात्रा हरिश्चन्द्रेण वा विभो}


\twolineshloka
{सत्यं वदत माऽसत्यं सत्यं धर्मः सनातनः}
{हरिश्चन्द्रश्चरति वै दिवि सत्येन चन्द्रवत्}


\twolineshloka
{श्येनचित्रेण राजेन्द्र सोमकेन वृकेण च}
{रैवते रन्तिदेवेन वसुना सृञ्जयेन च}


\threelineshloka
{एतैश्चान्यैश्च राजेन्द्र कृपेण भरतेन च}
{दुष्यन्तेन करूशेन रामालर्कनलैस्तथा}
{विचकाश्वेन निमिना जनकेन च धीमता}


\twolineshloka
{ऐलेन पृथुना चैंव वीरसेनेन चैव ह}
{इक्ष्वाकुणा शम्भुना च श्वेतेन सगरेण च}


\twolineshloka
{अजेन धुन्धुना चैव तथैव च सुबाहुना}
{हर्यश्वेन च राजेन्द्र क्षुपेण भरतेन च}


\twolineshloka
{एतैश्चान्यैश्च राजेन्द्रि पुरा मांसं न भक्षितम्}
{शारदं कौमुदं मासं ततस्ते स्वर्गमाप्नुवन्}


\twolineshloka
{ब्रह्मलोके च तिष्ठन्ति ज्वलमानाः श्रियाऽन्विताः}
{उपास्यमाना गन्धर्वैः स्त्रीसहस्रसमन्विताः}


\twolineshloka
{तदेतदुत्तमं धर्ममहिंसाधर्मलक्षणम्}
{ये चरन्ति महात्मानो नाकपृष्ठे वसन्ति ते}


\twolineshloka
{मधु मांसं च ये नित्यं वर्जयन्तीह धार्मिकाः}
{जन्मप्रभृति मद्यं च सर्व ते मुनयः स्मृताः}


\twolineshloka
{इमं धर्मममांसादं यश्चरेनच्छ्रावयीत वा}
{अपि चेत्सुदुराचारो न जातु निरयं व्रजेत्}


\twolineshloka
{पठेद्वा च इदं राजञ्शृणुयाद्वाऽप्यभीक्ष्णशः}
{अमांसभक्षणविधिं पवित्रमृषिपूजितम्}


\threelineshloka
{विमुक्तः सर्वपापेभ्यः सर्वकामैर्महीयते}
{विशिष्टतां ज्ञातिषु च लभते नात्र संशयः}
{`अहिंस्रो दानशीलश्च मधुमांसविवर्जितः ॥'}


\twolineshloka
{आपन्नश्चापदो मुच्येद्बद्धो मुच्येत बन्धनात्}
{मुच्येत्तथाऽऽतुरो रोगाद्दुःखान्मुच्येत दुःखितः}


\twolineshloka
{तिर्यग्योनिं न गच्छेत रूपवांश्च भवेन्नरः}
{ऋद्धिमान्वै कुरुश्रेष्ठ प्राप्नुयाच्च महद्यशः}


\twolineshloka
{एतत्ते कथितं राजन्मांसस्य परिवर्जने}
{प्रवृत्तौ च निवृत्तौ च विधानमृषिनिर्मितम्}


\chapter{अध्यायः १७८}
\twolineshloka
{इमे वै मानवा लोके भृशं मांसेषु गृद्धिनः}
{विसृज्य विविधान्भक्ष्यान्महारक्षोगणा इव}


\twolineshloka
{अपूपान्विविधाकाराञ्शाकानि विविधानि च}
{पादपान्रससंयुक्तान्न चेच्छन्ति यथाऽऽमिषम्}


\twolineshloka
{तत्र मे बुद्धिरत्रेव विषये परिमुह्यते}
{न मन्ये रसतः किञ्चिन्मांसतोऽस्तीति किञ्चन}


\twolineshloka
{तदिच्छामि गुणाञ्श्रोतुं मांसस्याभक्षणे प्रभो}
{भक्षणे चैव ये दोषास्तांश्चैव पुरुषर्षभ}


\twolineshloka
{सर्वसत्त्वेन धर्मज्ञ यथावदिह धर्मतः}
{किं वा भक्ष्यमभक्ष्यं वा सर्वमेतद्वदस्व मे}


\threelineshloka
{यथैतद्यादृशं चैव गुणा ये चास्य वर्जने}
{दोषा भक्षयतो येऽपि तन्मे ब्रूहि पितामह ॥भीष्म उवाच}
{}


\twolineshloka
{सर्वमेतन्महाबाहो यथा वदसि भारत}
{न मांसात्परमं किञ्चिद्रसतो विद्यते भुवि}


\twolineshloka
{क्षतक्षीणाभितप्तानां ग्राम्यधर्मरतात्मनाम्}
{अध्वना कर्शितानां च न मांसाद्विद्यते परम्}


\twolineshloka
{सद्यो वर्धयति प्राणान्पुष्टिमग्र्यां दधाति च}
{`नाशो भक्षणदोषः स्याद्दानमेव सदा मतम्}


\twolineshloka
{क्षुधितानां द्विजानां च सर्वेषां चापि जीवितम्}
{दत्त्वा भवति पूतात्मा श्रद्धया लोभवर्जितः}


\threelineshloka
{शिक्षयन्ति न याचन्ते दर्शयन्ति स्वमूर्तिभिः}
{अवस्थेयमदानस्य मा भूदेवं भवानिति}
{दानाद्यः सुशुचिर्मांसं पुनर्नैव च भक्षयेत् ॥'}


\threelineshloka
{न भक्ष्योऽभ्याधिकः कश्चिन्मांसादस्ति परंतप}
{विवर्जने तु बहवो गुणाः कौरवनन्दन}
{ये भवन्ति मनुष्याणां तान्मे निगदतः शृणु}


\twolineshloka
{स्वमांसं परमांसेन यो वर्धयितुमिच्छति}
{नास्ति क्षुद्रतरस्तस्मात्स नृशंसतरो नरः}


\twolineshloka
{न हि प्राणात्प्रियतरं लोके किञ्चन विद्यते}
{तस्माद्दयां नरः कुर्याद्यथाऽऽत्मनि तथा परे}


\twolineshloka
{शुक्राच्च तात सम्भूतिर्मांसस्येह न संशयः}
{भक्षणे तु महान्दोषो मलेन स हि कल्प्यते}


\twolineshloka
{`अहिंसालक्षणो धर्म इति वेदविदो विदुः}
{यदहिंस्रं भवेत्कर्म तत्कुर्यादात्मविन्नरः}


\twolineshloka
{पितृदैवतयज्ञेषु प्रोक्षितं हविरुच्यते}
{'विधिना वेददृष्टेन तद्भुक्त्वेह न दुष्यति}


\twolineshloka
{यज्ञार्थे पशवः सृष्टा इत्यपि श्रूयते श्रुतिः}
{अतोऽन्यथा प्रवृत्तानां राक्षसो विधिरुच्यते}


\twolineshloka
{क्षत्रियाणां तु यो दृष्टो विधिस्तमपि मे शृणु}
{वीर्येणोपार्जितं मांसं यथा भुञ्जन्न दुष्यति}


\twolineshloka
{आरण्याः सर्वदैवत्याः सर्वशः प्रोक्षिता मृगाः}
{अगस्त्येन पुरा राजन्मृगया येन पूज्यते}


\twolineshloka
{`रक्षणार्थाय भूतानां हिंस्रान्हन्यान्मृगान्पुनः}
{'नात्मानमपरित्यज्य मृगया नाम विद्यते}


\threelineshloka
{समतामुपसङ्गम्य रूपं हन्यान्न वा नृप}
{अतो राजर्षयः सर्वे मृगयां यान्ति भारत}
{न हि लिप्यन्ति पापेन न चैतत्पातकं भुवि}


\twolineshloka
{न ह्यतः सदृशं किञ्चिदिह लोके परत्र च}
{यत्सर्वेष्विह भूतेषु दया कौरवन्दन}


\twolineshloka
{न भयं विद्यते जातु नरस्येह दयावतः}
{दयावतामिमे लोकाः परे चापि तपस्विनाम्}


\twolineshloka
{अभयं सर्वभूतेभ्यो यो ददाति दयापरः}
{अभयं तस्य भूतानि ददतीत्यनुशुश्रुम}


\threelineshloka
{क्षतं च स्खलितं चैव पतितं क्लिन्नमाहतम्}
{सर्वभूतानि रक्षन्ति समेषु विषमेषु च}
{}


\twolineshloka
{नैनं व्यालमृगा घ्नन्ति न पिशाचा न राक्षसाः}
{मुच्यते भयकालेषु मोक्षयेद्यो भये परान्}


\twolineshloka
{प्राणदानात्परं दानं न भूतं न भविष्यति}
{न ह्यात्मनः प्रियतरं किञ्चिदस्तीह निश्चितम्}


\twolineshloka
{अनिष्टं सर्वभूतानां मरणं नाम भारत}
{मृत्युकाले हि भूतानां सद्यो जायति वेपथुः}


\twolineshloka
{व्याधिजन्मजरादुःखैर्नित्यं संसारसागरे}
{जन्तवः परिवर्तन्ते मरणादुद्विजन्ति च}


\twolineshloka
{गर्भवासेषु पच्यन्ते क्षाराम्लकटुकै रसैः}
{मूत्रस्वेदपुरीषाणां परुषैर्भृशदारुणैः}


\twolineshloka
{जाताश्चाप्यवशास्तत्र च्छिद्यमानाः पुनःपुनः}
{हन्यमानाश्च दृश्यन्ते विवशा मांसगृद्धिनः}


\twolineshloka
{कुम्भीपाके च पच्यन्ते तां तां योनिमुपागताः}
{आक्रम्य मार्यमाणाश्च त्रास्यन्त्यन्ये पुनःपुनः}


\twolineshloka
{नात्मनोऽस्ति प्रियतरः पृथिवीमनुसृत्य ह}
{तस्मात्प्राणिषु सर्वेषु दयावानात्मवान्भवेत्}


\twolineshloka
{सर्वमांसानि यो राजन्यावज्जीवं न भक्षयेत्}
{आश्वासं विपुलं स्थानं प्राप्नुयान्नात्र संशयः}


\twolineshloka
{ये भक्षयन्ति मांसानि भूतानां जीवितैषिणाम्}
{भक्ष्यन्ते तेऽपि भूतैस्तैरिति मे नास्ति संशयः}


\twolineshloka
{मां स भक्षयते यस्माद्भक्षयिष्ये तमप्यहम्}
{एतन्मांसस्य मांसत्वमनुबुद्ध्यस्व भारत}


\twolineshloka
{घातको वध्यते नित्यं तथा बध्येत बन्धकः}
{आक्रोष्टा क्रुध्यते राजन्द्वेष्टा द्वेष्यत्वमाप्नुते}


\twolineshloka
{येनयेन शरीरेण यद्यत्कर्म करोति यः}
{तेनतेन शरीरेण तत्तत्फलमुपाश्नुते}


\twolineshloka
{अहिंसा परमो धर्मस्तथाऽहिंसा परो दमः}
{अहिंसा परमं दानमहिंसा परमं तपः}


\twolineshloka
{अहिंसा परमं मिमहिंसा परमं सुखम् ॥सर्वयज्ञेषु वा दानं सर्वतीर्थेषु वा प्लुतम्}
{}


\twolineshloka
{सर्वदानफलं वाऽपि नैतत्तुल्यमहिंसया ॥अहिंसस्य तपोऽक्षय्यमहिंस्रो जयते सदा}
{}


\twolineshloka
{अहिंस्रः सर्वभूतानां यथा माता यथा पिता ॥एतत्फलमहिंसायां भूयश्च कुरुपुङ्गव}
{}


% Check verse!
न हि शक्या गुणा वक्तुमपि वर्षशतैरपि
\chapter{अध्यायः १७९}
\twolineshloka
{अकामाश्च सकामाश्च ये हताः स्म महामृधे}
{कां योनिं प्रतिपन्नास्ते तन्मे ब्रूहि पितामह}


\twolineshloka
{दुःखं प्राणपरित्यागः पुरुषाणां महामृधे}
{जानामि चाहं धर्मज्ञ प्राणत्यागं सुदुष्करम्}


\threelineshloka
{समृद्दे वाऽसमृद्धे वा शुभे वा यदि वाऽशुभे}
{`संसारेऽस्मिन्सदाजाताः प्राणिनोऽभिरताःकथम् 'कारणं तत्र मे ब्रूहि सर्वज्ञो ह्यसि मे मतः ॥भीष्म उवाच}
{}


\twolineshloka
{समृद्धे वाऽसमृद्धे वा शुभे वा यदि वाऽशुभे}
{संसारेऽस्मिन्समे जाताः प्राणिनः पृथिवीपते}


\twolineshloka
{निरता येन भावेन तत्र मे शृणु कारणम्}
{सम्यक्चायमनुप्रश्नस्त्वयोक्तस्तु युधिष्ठिर}


\twolineshloka
{अत्र ते वर्तयिष्यामि पुरावृत्तमिदं नृप}
{द्वैपायनस्य संवादं कीटस्य च युधिष्ठिर}


\twolineshloka
{ब्रह्मिभूतश्चरन्विप्रः कृष्णद्वैपायनः पुरा}
{ददर्श कीटं धावन्तं शीघ्रं शकटवर्त्मनि}


\twolineshloka
{गतिज्ञः सर्वभूतानां रुतज्ञश्च शरीरिणाम्}
{सर्वज्ञस्त्वरितं दृष्ट्वा कीटं वचनमब्रवीत्}


\threelineshloka
{कीट संत्रस्तरूपोऽसि त्वरितश्चैव लक्ष्यसे}
{क्व च वासस्तदाचक्ष्व कुतस्ते भयमागतम् ॥कीट उवाच}
{}


\twolineshloka
{शकटव्रजस्य महतो घोषं श्रुत्वा भयं मम}
{आगतं वै महाबुद्धे स्वन एष हि दारुणः}


\twolineshloka
{श्रूयते तु स मा हन्यादिति ह्यस्मादपक्रमे}
{श्वसतां च शृणोम्येनं गोवृषाणां प्रतोद्यताम्}


\twolineshloka
{वहतां सुमहाभारं सन्निकर्षे स्वनं प्रभो}
{नृणां च संवाहयतां श्रूयन्ते विविधाः स्वनाः}


\twolineshloka
{वोढुमस्मद्विधेनैव न शक्यः कीटयोनिना}
{तस्मादतिक्रमाम्येष भयादस्मात्सुदारुणात्}


\threelineshloka
{दुःखं हि मृत्युर्भूतानां जीवितं च सुदुर्लभम्}
{अतो भीतः पलायामि गच्छेयं नापदं यथा ॥भीष्म उवाच}
{}


\twolineshloka
{इत्युक्तः स तु सं प्राह कुतः कीट सुखं तव}
{मरणं ते सुखं मन्ये तिर्यग्योनौ हि वर्तसे}


\threelineshloka
{शब्दं स्पर्शं रसं गन्धं भोगांश्चोच्चावचान्बहून्}
{नाभिजानासि कीट त्वं श्रेयो मरणमेव ते ॥कीट उवाच}
{}


\twolineshloka
{सर्वत्र निरतो जीव इहापि च सुखं मम}
{चेतयामि महाप्राज्ञ तस्मादिच्छामि जीवितुम्}


\twolineshloka
{इहापि विषयः सर्वो यथादेहं प्रवर्तितः}
{मनुष्यास्तिर्यगाश्चैव पृथग्भोगा विशेषतः}


\twolineshloka
{अहमासं मनुष्यो वै शूद्रो बहुधनः प्रभो}
{अब्रह्मण्यो नृशंसश्च कदर्यो बुद्धिजीवनः}


\twolineshloka
{वाक््श्लक्ष्णो ह्यकृतप्रज्ञो द्वेष्टा विश्वस्य कर्मणः}
{मिथोगुप्तनिधिर्नित्यं परस्वहरणे रतः}


\twolineshloka
{भृत्यातिथिजनश्चापि गृहेऽपर्यशितो मया}
{मात्सर्यात्स्वादुकामेन नृशंसेन बुभुक्षता}


\twolineshloka
{देवार्थं पितृयज्ञार्थं न च श्राद्धं कृतं मया}
{न दत्तमन्नकामेषु दत्तमन्नं लुनामि च}


\twolineshloka
{गुप्तं शरणमाश्रित्य भयेषु शरणागतान्}
{त्यक्त्वाऽकस्मान्निशायां च न दत्तमभयं मया}


\twolineshloka
{धनं धान्यं प्रियान्दारान्यानं वासस्तथाऽद्भुतम्}
{श्रियं दृष्ट्वा मनुष्याणामसूयामि निरर्थकम्}


\twolineshloka
{ईर्ष्युः परसुखं दृष्ट्वा अन्यस्य न बुभूषकः}
{त्रिवर्गहन्ता चान्येषामात्मकामानुवर्तकः}


\twolineshloka
{नृशंसगुणभूयिष्ठं पुरा कर्म कृतं मया}
{स्मृत्वा तदनुतप्येऽहं हित्वा प्रियमिवात्मजम्}


\twolineshloka
{शुभानां नाभिजानामि कृतानां कर्मणां फलम्}
{माता च पूजिता वृद्धा ब्राह्मणश्चार्चितो मया}


\twolineshloka
{सकृज्जातिगुणोपेतः सङ्गत्या गृहमागतः}
{अतिथिः पूजितो ब्रह्मंस्तेन मां नाजहात्स्मृतिः}


\twolineshloka
{कर्मणामेव चैवाहं सुखाशामिव लक्षये}
{तच्छ्रोतुमहमिच्छामि त्वत्तः श्रेयस्तपोधन}


\chapter{अध्यायः १८०}
\twolineshloka
{शुभेन कर्मणा यद्वै तिर्यग्योनौ न मुह्यसे}
{ममैव कीट तत्कर्म येन त्वं न प्रमुह्यसे}


\twolineshloka
{अहं त्वां दर्शनादेव तारयामि तपोबलात्}
{तपोबलाद्धि बलवद्बलमन्यन्न विद्यते}


\twolineshloka
{जानामि पापैः स्वकृतैर्गतं त्वां कीट कीटताम्}
{अवाप्स्य्सि परं र्धर्मं मानुष्ये यदि मन्यसे}


\twolineshloka
{कर्मभूमिकृतं देवा भुञ्जते तिर्यगाश्च ये}
{धन्या अपि मनुष्येषु कामार्थाश्च यथा गुणाः}


\twolineshloka
{वाग्बुद्धिपाणिपादैश्च समुपेता विपश्चितः}
{किमायाति मनुष्यस्य मन्दस्यार्थस्य जीवतः}


\twolineshloka
{दैवे यः कुरुते पूजां विप्राग्निशशिसूर्ययोः}
{ब्रुवन्नपि कथां पुण्यां तत्र कीट त्वमेष्यसि}


\fourlineindentedshloka
{गुणभूतानि भूतानि तत्र त्वमुपभोक्ष्यसे}
{क्रमात्तेऽहं विनेष्यामि ब्रह्मत्वं यदि चेच्छसि}
{भीष्म उवाच}
{स तथेति प्रतिश्रुत्य कीटः समवतिष्ठत}


% Check verse!
*तमृषिं द्रष्टुमगमत्सर्वास्वन्यासु योनिषु
\twolineshloka
{श्वाविद्गोधावराहाणां तथैव मृगपक्षिणाम्}
{श्वपाकशूद्रवैश्यानां क्षत्रियाणां च योनिषु}


\threelineshloka
{स कीटे एवमाख्यातमृषिणा सत्यवादिना}
{प्रतिस्मृत्याथ जग्राह पादौ मूर्ध्नि कृताञ्जलिः ॥कीट उवाच}
{}


\twolineshloka
{इदं तदतुलं स्थानमीप्सितं दशभिर्गुणैः}
{यदहं प्राप्य कीटत्वमागतो राजपुत्रताम्}


\twolineshloka
{वहन्ति मामतिबलाः कुञ्जरा हेममालिनः}
{स्यन्दनेषु च काम्भोजा युक्ताःसमरवाजिनः}


\twolineshloka
{उष्ट्राश्वतरयुक्तानि यानानि च वहन्ति माम्}
{सबान्धवः सहामात्यश्चाश्नामि पिशिताशनम्}


\twolineshloka
{गृहेषु स्वनिवासेषु सुखेषु शयनेषु च}
{वरार्हेषु महाभाग स्वपामि च सुपूजितः}


\twolineshloka
{सर्वेष्वपररात्रेषु सूतमागधबन्दिनः}
{स्तुवन्ति मां यथा देवा महेन्द्रं प्रियवादिनः}


\threelineshloka
{प्रसादात्सत्यसन्धस्य भवतोऽमिततेजसः}
{यदहं कीटतां प्राप्य स्मृतिजाता जुगुप्सिताम्}
{ननु नाशोस्ति पापस्य यन्मयोपचितं पुरा}


\threelineshloka
{[नमस्तेऽस्तु महाप्राज्ञ किं करोमि प्रशाधि माम्}
{त्वत्तपोबलनिर्दिष्टमिदं ह्यधिगतं मया ॥] व्यास उवाच}
{}


\twolineshloka
{[अर्चितोऽहं त्वया राजन्वाग्भिरद्य यदृच्छया}
{अद्य ते कीटतां प्राप्य स्मृतिर्जाताजुगुप्सिताम्]}


\threelineshloka
{शूद्रेणार्थप्रधानेन नृशंसेनाततायिना}
{ममैतद्दर्शनं प्राप्तं तच्च वै सुकृतं पुरा}
{तिर्यग्योनौ स्म जातेन मम चाभ्यर्चनात्तथा}


\threelineshloka
{इतस्त्वं राजपुत्रत्वाद्ब्राह्मण्यं समवाप्स्यसि}
{गोब्राह्मणकृते प्राणान्हित्वाऽऽत्मीयान्रणाजिरे ॥भीष्म उवाच}
{}


\twolineshloka
{राजपुत्रः सुखं प्राप्य ईजे चैवाप्तदक्षिणैः}
{अथ चोद्दीप्यत स्वर्गे प्रभूतोप्यव्ययः सुखी}


\twolineshloka
{तिर्यग्योन्याः शूद्रतामभ्यपैतिशूद्रो वैश्यं क्षत्रियत्वं च वैश्यः}
{वृत्तश्लाघी क्षत्रियो ब्राह्मणत्वंस्वर्गं पुण्याद्ब्राह्मणः साधुवृत्तः}


\chapter{अध्यायः १८१}
\twolineshloka
{क्षत्रधर्ममनुप्राप्तः स्मरन्नेव च वीर्यवान्}
{त्यक्त्वा हि कीटतां राजंश्चचार विपुलं तपः}


\threelineshloka
{तस्य धर्मार्थविदुषो दृष्ट्वा तद्विपुलं तपः}
{आजगाम द्विजश्रेष्ठः कृष्णद्वैपायनस्तदा ॥व्यास उवाच}
{}


\twolineshloka
{क्षात्रादेव व्रतात्कीट भूतानां परिपालय}
{क्षात्रं देवव्रतं ध्यासंस्ततो विप्रत्वमेष्यसि}


\twolineshloka
{पाहि सर्वाः प्रजाः सम्यक् शुभाशुभविदात्मवान्}
{शुभैः संविभजन्कामैरशुभानां च भावनैः}


\threelineshloka
{आत्मवान्भव सुप्रीतः स्वधर्माचरणे रतः}
{क्षात्रीं तनुं समुत्सृज्य ततो विप्रत्वमेष्यसि ॥भीष्म उवाच}
{}


\twolineshloka
{सोऽप्यरण्यादभिप्रेत्य पुनरेव युधिष्ठिर}
{महर्षेर्वचनं श्रुत्वा प्रजा धर्मेण पालयन्}


\twolineshloka
{अचिरेणैव कालेन कीटः पार्थिवसत्तम}
{प्रजापालनधर्मेणि प्रेत्य विप्रत्वमागतः}


\threelineshloka
{ततस्तं ब्राह्मणं द्रष्टुं पुनरेव महायशाः}
{आजगाम महाप्राज्ञः कृष्णद्वैपायनस्तदा ॥व्यास उवाच}
{}


\twolineshloka
{भोभो ब्रह्मर्षभ श्रीमन्मा व्यथिष्ठाः कथञ्चन}
{शुभकृच्छुभयोनीषु पापकृत्पापयोनिषु}


\fourlineindentedshloka
{उपपद्यति धर्मज्ञ यथाधर्मं यथाव्रतम्}
{तस्मान्मृत्युभयात्कीट मा व्यथिष्ठाः कथञ्चन}
{धर्मलाभात्परं न स्यात्तस्माद्धर्मं चरोत्तमम् ॥कीट उवाच}
{}


\threelineshloka
{सुखात्सुखतरं प्राप्तो भगवंस्त्वत्कृते ह्यहम्}
{धर्ममूलं शुभं प्राप्य पाप्मा नष्ट इहाद्य मे ॥भीष्म उवाच}
{}


\threelineshloka
{भगवद्वचनात्कीटो ब्राह्मण्यं प्राप्य दुर्लभम्}
{अकरोत्पृथिवीं राजन्यज्ञयूपशताङ्किताम्}
{ततः सालोक्यमगमद्ब्रह्मणो ब्रह्मवित्तमः}


\twolineshloka
{अथ पापहरं कीटः पार्त ब्रह्म सनातनम्}
{स्वकर्मफलनिर्वृत्तं व्यासस्य वचनात्तदा}


\twolineshloka
{कुरुक्षेत्रे युद्धहताः पुण्ये क्षत्रियपुङ्गवाः}
{सम्प्राप्तास्ते गतिं पुण्यां तन्मा त्वं शोच पुत्रक}


\chapter{अध्यायः १८२}
\threelineshloka
{विद्या तपश्च दानं च किमेतेषां विशिष्यते}
{पृच्छामि त्वां सतां श्रेष्ठ तन्मे ब्रूहि पितामह ॥भीष्म उवाच}
{}


\twolineshloka
{अत्राप्युदाहरन्तीममितिहासं पुरातनम्}
{मैत्रेयस्य च संवादं कृष्णद्वैपायनस्य च}


\twolineshloka
{कृष्णद्वैपायनो राजन्नज्ञातचरितं चरन्}
{वाराणस्यामुपातिष्ठन्मैत्रेयं स्वैरिणीकुले}


\twolineshloka
{तमुपच्छन्नमासीनं ज्ञात्वा स मुनिसत्तमः}
{अर्चित्वा भोजयामास मैत्रेयोऽशनमुत्तमम्}


\twolineshloka
{तदन्नमुत्तमं भुक्त्वा गुणवत्सार्वकामिकम्}
{उत्तिष्ठमानोऽस्मयत प्रीतः कृष्णो महामनाः}


\threelineshloka
{तमुत्स्मयन्तं सम्प्रेक्ष्य मैत्रेयः कृष्णमब्रवीत्}
{कारणं ब्रूहि धर्मात्मन्व्यस्मयिष्ठाः कुतश्च ते}
{तपस्विनो धृतिमतः प्रमोदः समुपागतः}


\twolineshloka
{एतदिच्छामि ते विद्वन्नभिवाद्य प्रणम्य च}
{आत्मनश्च तपोभाग्यं सुखभाग्यं ममेह च}


\fourlineindentedshloka
{`तपोभाग्यान्महाभाग सुखभाग्यात्तथैव च}
{'पृथगाचरितं तात पृथगाचरितात्मनः}
{अल्पान्तरमहं मन्ये विशिष्टमपि चान्वयात् ॥व्यास उवाच}
{}


\twolineshloka
{अतिच्छेदातिवादाभ्यां स्मयोऽयं समुपागतः}
{असत्यं वेदवचनं कस्माद्वेदोऽनृतं वदेत्}


\twolineshloka
{त्रीण्येव तु पदान्याहुः पुरुषस्योत्तमं प्रति}
{न द्रुह्येच्चैव दद्याच्च सत्यं चैव परान्वदेत्}


\twolineshloka
{इति वेदोक्तमृषिभिः पुरस्तात्परिकल्पितम्}
{इदानीं चैव नः कृत्यं पुरस्ताच्च परिश्रुतम्}


\twolineshloka
{अल्पोऽपि तादृशो न्यासो भवत्युत महाफलः}
{तृषिताय च यद्दत्तं हृदयेनानसूयता}


\twolineshloka
{तोषितास्त्रिदशा यत्ते दत्त्वैतद्दर्शनं मम}
{अजैषीर्महतो लोकान्महायज्ञैरिव प्रभो}


\twolineshloka
{ततो दानपवित्रेण प्रीतोऽस्मि तपसैव च}
{दूरात्पुण्यवतो गन्धः पुण्यस्यैव च दर्शनात्}


\twolineshloka
{पुण्यश्च वातिगन्धस्ते मन्ये कर्म विधानजम्}
{अथ कर्मार्जितस्तात यथाचैवानुलेपनात्}


\twolineshloka
{शुभं सर्वपवित्रेब्यो दानमेव परं द्विज}
{[नोचेत्सर्वपवित्रेभ्यो दानमेव परं भवेत् ॥]}


\twolineshloka
{यानीमान्युत्तमानीह वेदोक्तानि प्रशंसति}
{तेषां श्रेष्ठतरं दानमिति मे नात्र संशयः}


\twolineshloka
{दानवद्भिः कृतः पन्था येन यान्ति मनीषिणः}
{ते हि प्राणस्य दातारस्तेषु धर्मः प्रतिष्ठितः}


\twolineshloka
{यथा वेदाः स्वधीताश्च यथा चेन्द्रियसंयमः}
{सर्वत्यागो यता चेह तता दानमनुत्तमम्}


\twolineshloka
{त्वं हि तात सुखादेव शुभमेष्यसि शोभनम्}
{सुखात्सुखतरप्राप्तिमाप्नुते मतिमान्नरः}


\twolineshloka
{तन्नः प्रत्यक्षमेवेदमुपलक्ष्यमसंशयम्}
{श्रीमन्तः प्राप्नुवन्त्यर्थान्दानं यज्ञं तथा सुखम्}


\twolineshloka
{सुखादेव परं दुःखं दुःखादन्यत्परं सुखम्}
{दृश्यते हि महाप्राज्ञ नियतं वै स्वभावतः}


\twolineshloka
{विविधानीह वृत्तानि नरस्याहुर्मनीषिणः}
{पुण्यमन्यत्पापमन्यन्न पुण्यं न च पापकम्}


\twolineshloka
{न वृत्तं मन्यतेऽन्यस्य मन्यतेऽन्यस्य पातकम्}
{यथा स्वकर्मनिर्वृत्तं न पुण्यं न च पापकम्}


\twolineshloka
{यज्ञदानतपःशीला नरा वै पुण्यकर्मिणः}
{येऽभिद्रुह्यन्ति भूतानि ते वै पापकृतो जनाः}


\twolineshloka
{द्रव्याण्याददते चैव दुःखं यान्ति पतन्ति च}
{ततोऽन्यत्कर्म यत्किंचिन्न पुण्यं न च पातकम्}


\twolineshloka
{`नित्यं चाकृपणो भुङ्क्ते स्वजनैर्देहि याचतः}
{भाग्यक्षयेण क्षीयन्ते नोपभोगेन सञ्चयाः ॥'}


\twolineshloka
{रमस्वैधस्व मोदस्व देहि दाने रमस्व च}
{न त्वामतिभविष्यन्ति वैद्या न च तपस्विनः}


\chapter{अध्यायः १८३}
\twolineshloka
{एवमुक्तः प्रत्युवाच मैत्रेयः कर्मपूर्वकः}
{अत्यन्तं श्रीमति कुले जातः प्राज्ञो बहुश्रुतः}


\threelineshloka
{असंशयं महाप्राज्ञ यथैवात्थ तथैव तत्}
{अनुज्ञातश्च भवता किञ्चिद्ब्रूयामहं विभो ॥व्यास उवाच}
{}


\threelineshloka
{यद्यदिच्छसि मैत्रेय यावद्यावद्यथायथा}
{ब्रूहि तत्वं महाप्राज्ञ शुश्रुषे वचनं तव ॥मैत्रेय उवाच}
{}


\twolineshloka
{निर्दोषं निर्मलं चैव वचनं सत्यसंहितम्}
{विद्यातपोभ्यां हि भवान्भावितात्मा न संशयः}


\twolineshloka
{भवतो भावितात्मत्वाल्लाभोऽयं सुमहान्मम}
{भूयो बुद्ध्याऽनुपश्यामि सुसमृद्धतपा इव}


\twolineshloka
{अपि मे दर्शनादेव भवतोऽभ्युदयो महान्}
{मन्ये भवत्प्रसादोऽयं बुद्धिकर्मस्वभावतः}


\twolineshloka
{तपः श्रुतं च योनिश्चाप्येतद्ब्राह्मण्यकारणम्}
{त्रिभिर्गुणैः समुदितः स्नातो भवति वै द्विजः}


\twolineshloka
{अस्मिंस्तृप्ते च तृप्यन्ति पितरो दैवतानि च}
{न हि श्रुतवतां किञ्चिदधिकं ब्राह्मणादृते}


\twolineshloka
{`असंस्कारात्क्षत्रवैश्यौ नश्येते ब्राह्मणादृते}
{शूद्रो नश्यत्यशुश्रूषुराश्रमाणां यथार्हतः ॥'}


\twolineshloka
{[अन्धं स्यात्तम एवेदं न प्रज्ञायेत किञ्चन}
{चातुर्वर्ण्यं न वर्तेत धर्माधर्मावृतानृते ॥]}


\twolineshloka
{यथाहि सुकृते क्षेत्रे फलं विन्दति मानवः}
{एवं दत्त्वा श्रुतवते फलं दाता समश्नुते}


\twolineshloka
{ब्राह्मणश्चेन्न विन्देत श्रुतवृत्तोपसंहितः}
{प्रतिग्रहीता दानस्य मोघं स्याद्धनिनां धनम्}


\twolineshloka
{अन्नं ह्यविद्वान्हन्त्येवमविद्वांसं च हन्ति तत्}
{तच्चान्यं हन्ति यच्चान्यत्स भुक्त्वा हन्यतेऽबुधः}


\twolineshloka
{प्राहुर्ह्यन्नमदन्विद्वान्पुनर्जनयतीश्वरः}
{स चान्नाज्जायते तस्मात्सूक्ष्म एष व्यतिक्रमः}


\twolineshloka
{`ब्राह्मं ह्यनुपयोगी यो ददंश्चान्नमसंशयम्}
{यस्तारयति वै विद्वान्पितॄन्देवान्सदाऽमृतान्'}


\twolineshloka
{यदेव ददतः पुण्यं तदेव प्रतिगृह्णतः}
{न ह्येकचक्रं वर्तेत इत्येवमृषयो विदुः}


\twolineshloka
{यत्र वै ब्राह्मणाः सन्ति श्रुतवृत्तोपसंहिताः}
{तत्र दानफलं पुण्यमिह चामुत्र चाश्नुते}


\twolineshloka
{ये योनिशुद्धाः सततं तपस्यभिरता भृशम्}
{दानाध्ययनसम्पन्नास्ते वै पूज्यतमाः सदा}


\twolineshloka
{तैर्हि सद्भिः कृतः पन्था देवयानो न मुह्यते}
{ते हि स्वर्गस्य नेतारो यज्ञवाहाः सनातनाः}


\chapter{अध्यायः १८४}
\twolineshloka
{एवमुक्तः स भगवान्मैत्रेयं प्रत्यभाषत}
{दिष्ट्यैतत्त्वं विजानासि दिष्ट्या ते बुद्धिरीदृशी?}


\twolineshloka
{लोको ह्ययं गुणानेव भूयिष्ठं सम्प्रशंसति}
{रूपमानो वयोमानो धनविद्यामदस्तथा}


\twolineshloka
{दिष्ट्या नाभिभवन्ति त्वां दैवस्तेऽयमनुग्रहः}
{तत्ते बहुगुणं दानं वर्तयिष्यामि तच्छृणु}


\threelineshloka
{यानीहागमशास्त्राणि याश्च काश्चित्प्रिवृत्तयः}
{तानि वेदं पुरस्कृत्य प्रवृत्तानि यथाक्रमम्}
{}


\twolineshloka
{अहं दानं प्रशंसामि भवानपि तपःश्रुतेः}
{तपः पवित्रं वेदस्य तपः स्वर्गस्य साधनम्}


\twolineshloka
{तपसा महदाप्नोति विद्यया चेति नः श्रुतम्}
{तपसैव चापनुदेद्यच्चान्यदपि दुष्कृतम्}


\twolineshloka
{यद्यद्धि किञ्चित्सन्धाय पुरुषस्तप्यते तपः}
{सर्वमेतदवाप्नोति ब्राह्मणो वेदपारगः}


\twolineshloka
{दुरन्वयं दुष्प्रधर्षं दुरापं दुरतिक्रमम्}
{सर्वं वै तपसाऽभ्योति तपो हि बलवत्तरम्}


\twolineshloka
{सुरापः स्वर्णहारी च भ्रूणहा गुरुतल्पगः}
{तपसा तरते सर्वमेनसश्च प्रमुच्यते}


\twolineshloka
{सर्वो वैद्यस्तु चक्षुष्मानपि यादृशतादृशः}
{तपस्विनं तथैवाहुस्ताभ्यां कार्यं सतां मतम्}


\twolineshloka
{सर्वे पूज्याः श्रुतधनास्तथैव च तपस्विनः}
{दानप्रदाः सुखं प्रेत्य प्राप्नुवन्तीह च श्रियम्}


\twolineshloka
{इमं च ब्रह्मलोकं च लोकं च बलवत्तरम्}
{अन्नदानैः सुकृतिनः प्रतिपद्यन्ति लौकिकम्}


\twolineshloka
{पूजिताः पूजयन्त्येते मानिता मानयन्ति च}
{स दाता यत्र यत्रैति सर्वतः सम्प्रणूयते}


\twolineshloka
{अकर्ता चैव कर्ता च लभते यस्य यादृशम्}
{यदि चोर्ध्वं यद्यधो वा स्वान्लोकानभियास्यति}


\twolineshloka
{प्राप्स्यसि त्वन्नपानानि यानि दास्यसि कर्हिचित्}
{मेधाव्यसि कुले जातः श्रुतवाननृशंसवान्}


\twolineshloka
{कौमारदारो व्रतवान्मैत्रेय निरतो भव}
{[एतद्गृहाण प्रथमं प्रशस्तं गृहमेधिनाम्}


\twolineshloka
{यो भर्ता वासितातुष्टो भर्तुस्तुष्टा च वासिता}
{यस्मिन्नेवं कुले सर्वं कल्याणं तत्र वर्तते}


\threelineshloka
{अद्भिर्गात्रान्मलमिव तमोऽग्निप्रभया यथा}
{]दानेन तपसा चैव विष्णोरभ्यर्चनेन च}
{`ब्राह्मणः स महाभाग तरेत्संसारसागरात्}


\twolineshloka
{स्वकर्मशुद्धसत्त्वानां तपोभिर्निर्मलात्मनाम्}
{विद्यया गतमोहानां तारणाय हरिः स्मृतः}


\twolineshloka
{तदर्चनपरो नित्यं तद्भक्तस्तं नमस्कुरु}
{तद्भक्ता न विनश्यन्ति ह्यष्टाक्षरपरायणाः}


\twolineshloka
{प्रणवोपासनपराः परमार्थिपरास्त्विह}
{एतैः पावय चात्मानं सर्वपापमपोह्य च ॥'}


\twolineshloka
{स्वस्ति प्राप्नुहि मैत्रेय गृहान्साधु व्रजाम्यहम्}
{एतन्मनसि कर्तव्यं श्रेय एवं भविष्यति}


\twolineshloka
{तं प्रणम्याथ मैत्रेयः कृत्वा चापि प्रदक्षिणम्}
{स्वस्ति प्राप्नोतु भगवानित्युवाच कृताञ्जलिः}


\chapter{अध्यायः १८५}
\threelineshloka
{स्त्रीणां हि समुदाचारं सर्वधर्मविदांवर}
{श्रोतुमिच्छाम्यहं त्वत्तस्तन्मे ब्रूहि पितामह ॥भीष्म उवाच}
{}


\twolineshloka
{सर्वज्ञां सर्वतत्त्वज्ञां देवलोके मनस्विनीम्}
{कैकेयी सुमना नाम शाण्डिलीं पर्यपृच्छत}


\twolineshloka
{केन वृत्तेन कल्याणि समाचारेण केन वा}
{विधूय सर्वपापानि देवलोकं त्वमागता}


\twolineshloka
{हुताशनशिखेव त्वं ज्वलमाना स्वतेजसा}
{सुता ताराधिपस्येव प्रभया दिवमागता}


\twolineshloka
{अरजांसि च वस्त्राणि धारयन्ती गतक्लमा}
{विमानस्था शुभा भासि सहस्रगुणमोजसा}


\twolineshloka
{न त्वमल्पेन तपसा दानेन नियमेन वा}
{इमं लोकमनुप्राप्ता त्वं हि तत्त्वं वदस्व मे}


\twolineshloka
{इति पृष्टा सुमनया मधुरं चारुहासिनी}
{शाण्डिली निभृतं वाक्यं सुमनामिदमब्रवीत्}


\twolineshloka
{नाहं काषायवसना नापि वल्कलधारिणी}
{न च मुण्डा च जटिला भूत्वा देवत्वमागता}


\twolineshloka
{अहितानि च वाक्यानि सर्वाणि परुषाणि च}
{अप्रमत्ता च भर्तारं कदाचिन्नाहमब्रवम्}


\twolineshloka
{देवतानां पितॄणां च ब्राह्मणानां च पूजने}
{अप्रमत्ता सदा युक्ता श्वश्रूश्वशुरवर्तिनी}


\twolineshloka
{पैशुन्ये न प्रवर्तामि न ममैतन्मनो गतम्}
{अद्वारि न च तिष्ठामि चिरं न कथयामि च}


\twolineshloka
{असद्वा हसितं किञ्चिदहितं वाऽपि कर्मणा}
{रहस्यमरहस्यं वा न प्रवर्तामि सर्वथा}


\twolineshloka
{कार्यार्थे निर्गतं चापि भर्तारं गृहमागतम्}
{आसनेनोपसंयोज्य पूजयामि समाहिता}


\twolineshloka
{यदन्नं नाभिजानाति यद्भोज्यं नाभिनन्दति}
{भक्ष्यं वा यदि वा लेह्यं तत्सर्वं वर्जयाम्यहम्}


\twolineshloka
{कुटुंबार्थे समानीतं यत्किञ्चित्कार्यमेव तु}
{पुनरुत्थाय तत्सर्वं कारयामि करोमि च}


\twolineshloka
{अग्निसंरक्षणपरा गृहशुद्धिं च कारये}
{कुमारान्पालये नित्यं कुमारीं परिशिक्षये}


\twolineshloka
{आत्मप्रियाणि हित्वाऽपि गर्भसंरक्षणे रता}
{बालानां वर्जये नित्यं शापं कोपं प्रतापनम्}


\threelineshloka
{अविक्षिप्तानि धान्यानि नान्नविक्षेपणं गृहे}
{रक्तवत्स्पृहये गेहे गावः सयवसोदकाः}
{समुद्गम्य च शुद्धाऽहं भिक्षां दद्यां द्विजातिषु}


\twolineshloka
{प्रवासं यदि मे याति भर्ता कार्येण केनचित्}
{मङ्गलैर्बहुभिर्युक्ता भवामि नियता तदा}


\twolineshloka
{अञ्जनं रोचनां चैव स्नानं माल्यानुलेपनम्}
{प्रसाधनं च निष्क्रान्ते नामिनन्दामि भर्तरि}


\threelineshloka
{नोत्थापयामि भर्तारं सुखं सुप्तमहं सदा}
{आतुरेष्वपि कार्येषु तेन तुष्यति मे मनः}
{नोत्थापये सुखं सुप्तं ह्यातुरं पालये पतिम्}


\twolineshloka
{नायासयामि भर्तारं कुटुम्बार्थेऽपि सर्वदा}
{गुप्तगुह्या सदा चास्मि सुसंमृष्टनिवेशना}


\threelineshloka
{इमं धर्मपथं नारी पालयन्ती समाहिता}
{अरुन्धतीव नारीणीस्वर्गलोके महीयते ॥भीष्म उवाच}
{}


\twolineshloka
{एतदाख्याय सा देवी सुमनायै तपस्विनी}
{पतिधर्मं महाभागा जगामादर्शनं तदा}


\twolineshloka
{यश्चेदं पाण्डवाख्यानं पठेत्पर्वणि पर्वणि}
{स देवलोकं सम्प्राप्य नन्दने स सुखी वसेत्}


\chapter{अध्यायः १८६}
\threelineshloka
{यज्ज्ञेयं परमं कृत्यमनुष्ठेयं महात्मभिः}
{सारं मे सर्वशास्त्राणां वक्तुमर्हस्यनुग्रहात् ॥भीष्म उवाच}
{}


\twolineshloka
{श्रूयतामिदमत्यन्तं गूढं संसारमोचनम्}
{श्रोतव्यं च त्वया सम्यग्ज्ञातव्यं च विशाम्पते}


\twolineshloka
{पुण्डरीकः पुरा विप्रः पुण्यतीर्थे जपान्वितः}
{नारदं परिपप्रच्छ श्रेयो योगपरं मुनिम्}


% Check verse!
नारदश्चाब्रवीदेनं ब्रह्मणोक्तं महात्मना
\threelineshloka
{शृणुष्वावहितस्तात ज्ञानयोगमनुत्तमम्}
{अप्रभूतं प्रभूतार्थं वेदशास्त्रार्थसंयुतम्}
{}


\twolineshloka
{यः परः प्रकृते प्रोक्तः पुरुषः पञ्चविंशकः}
{स एव सर्वभूतात्मा नर इत्यभिधीयते}


\twolineshloka
{नराज्जातानि तत्वानि नाराणीति ततो विदुः}
{तान्येव चायनं तस्य तेन नारायणः स्मृतः}


\twolineshloka
{नारायणाज्जगत्सर्वं सर्गकाले प्रजायते}
{तस्मिन्नेव पुनस्तच्च प्रलये सम्प्रलीयते}


\twolineshloka
{नारायणः परं ब्रह्म तत्वं नारायणः परः}
{परादपि परश्चासौ तस्मान्नास्ति परात्परः}


\twolineshloka
{वासुदेवं तथा विष्णुमात्मानं च तथा विदुः}
{संज्ञाभेदैः स एवैकः सर्वशास्त्राभिसंस्कृतः}


\twolineshloka
{आलोड्य सर्वशास्त्राणि विचार्य च पुनःपुनः}
{इदमेकं सुनिष्पन्नं ध्येयो नारायणः सदा}


\twolineshloka
{तस्मात्त्वं गहनान्सर्वांस्त्यक्त्वा शास्त्रार्थविस्तरान्}
{अनन्यचेता ध्यायस्व नारायणमजं विभुम्}


\twolineshloka
{मुहूर्तिमपि यो ध्यायेन्नारायणमतन्द्रितः}
{सोऽपि तद्गतिमाप्नोति किं पुनस्तत्परायणः}


\twolineshloka
{नमो नारायणायेति यो वेद ब्रह्म शाश्वतम्}
{अन्त्यकाले जपन्नेति तद्विष्णोः परमं पदम्}


\twolineshloka
{श्रवणान्मननाच्चैव गीतिस्तुत्यर्चनादिभिः}
{आराध्यं सर्वदा ब्रह्म पुरुषेण हितैषिणा}


\twolineshloka
{लिप्यते न स पापेन नारायणपरायणः}
{पुनाति सकलं लोकं सहस्रांशुरिवोदितः}


\twolineshloka
{ब्रह्मचारी गृहस्थोऽपि वानप्रस्थोऽथ भिक्षुकः}
{केशवाराधनं हित्वा नैव याति परां गतिम्}


\twolineshloka
{जन्मान्तरसहस्रेषु दुर्लभा तद्गता मतिः}
{तद्भक्तवत्सलं देवं समराधय सुव्रत}


\twolineshloka
{नारदेनैवमुक्तस्तु स विप्रोऽभ्यर्चयद्धरिम्}
{स्वप्नोऽपि पुण्डरीकाक्षं शङ्खचक्रगदाधरम्}


\twolineshloka
{किरीटकुण्डलधरं लसच्छ्रीवत्सकौस्तुभम्}
{तं दृष्ट्वा देवदेवेशं प्राणमत्सम्भ्रमान्वितः}


\twolineshloka
{अथ कालेन महता तथा प्रत्यक्षतां गतः}
{संस्तुतः स्तुतिभिर्वेदैर्देवगन्धर्वकिन्नरैः}


\twolineshloka
{अथ तेनैव भगवानात्मलोकमधोक्षजः}
{गतः सम्प्रजितः सर्वैः स योगिनिलयो हरिः}


\twolineshloka
{तस्मात्त्वमपि राजेन्द्र तद्भक्तस्तत्परायणः}
{अर्चयित्वा यथायोगं भजस्व पुरुषोत्तमम्}


\twolineshloka
{अजरममरमेकं ध्येयमाद्यन्तशून्यंसगुणमगुणमाद्यं स्थूलमत्यन्तसूक्ष्मम्}
{निरुपममुपमेयं योगिविज्ञानगम्यंत्रिभुवनगुरुमीशं सम्प्रपद्यस्व विष्णुम्}


\chapter{अध्यायः १८७}
\threelineshloka
{साम्नि चापि प्रदानेन च ज्यायः किं भवतो मतम्}
{प्रब्रूहि भरतश्रेष्ठ यदत्र व्यतिरिच्यते ॥भीष्म उवाच}
{}


\twolineshloka
{साम्ना प्रसाद्यते कश्चिद्दानेन च तथाऽपरः}
{पौरुषीं प्रकृतिं ज्ञात्वा तयोरेकतरं भजेत्}


\twolineshloka
{गुणांस्तु शृणु वैराजन्त्सान्त्वस्य पुरुषर्षभ}
{दारुणान्यपि भूतानि सान्त्वेनाराधयेद्यथा}


\twolineshloka
{अत्राप्युदाहरन्तीममिदिहासं पुरातनम्}
{गृहीत्वा रक्षसा मुक्तो द्विजातिः कानने यथा}


\twolineshloka
{कश्चित्तु बुद्धिसम्पन्नो ब्राह्मणो विजने वने}
{गृहीतः कृच्छ्रमापन्नो रक्षसा भक्षयिष्यता}


\twolineshloka
{सुबुद्धिः श्रुतिसम्पन्नो दृष्ट्वा तमतिभीषणम्}
{सामैवास्मै प्रयुञ्जानो न मुमोह न विव्यधे}


\threelineshloka
{रक्षस्तु वाचा सम्पूज्य प्रश्नं पप्रच्छ तं द्विजम्}
{मोक्ष्यसे ब्रूहि मे प्रश्नं केनास्मि हरिणः कृशः ॥भीष्म उवाच}
{}


\twolineshloka
{मुहूर्तमथ सञ्चिन्त्य ब्राह्मणिस्तं निरीक्ष्य सः}
{अभीतवदथाव्यग्रः प्रश्नं प्रतिजगाद ह}


\twolineshloka
{विदेशस्थो विलोकस्थो विना नूनं सुहृज्जनैः}
{विषयानतुलान्भुङ्क्षे तेनासि हरिणः कृशः}


\twolineshloka
{नूनं मित्राणि ते रक्षः साधूपचरितान्यपि}
{स्वदोषात्तु परित्यज्य तेनासि हरिणः कृशः}


\twolineshloka
{अवृत्त्या पीड्यमानोऽपि वृत्त्युपायान्विगर्हयन्}
{महार्थान्ध्यायसे नूनं तेनासि हरिणः कृशः}


\twolineshloka
{परकार्याधिकारस्थाः सद्गुणैरधमा नराः}
{अवजानन्ति नूनं त्वां तेनासि हरिणः कृशः}


\twolineshloka
{गुणवान्निर्गुणानन्यान्नूनं पश्यसि तत्कृतान्}
{प्राज्ञैरपि विनीतात्मा तेनासि हरिणः कृशः}


\twolineshloka
{सम्पीड्यात्मानमार्यत्वात्त्वया कश्चिदुपस्कृतः}
{जितं त्वां मन्यते साधो तेनासि हरिमः कृशः}


\twolineshloka
{क्लिश्यमानान्विमार्गेषु कामक्रोधावृतात्मनः}
{मन्येऽनुध्यायसि जनांस्तेनासि हरिमः कृशः}


\twolineshloka
{प्राज्ञैरपुजितो नूनं प्राज्ञैरप्यभिनिन्दितः}
{ह्रीमानमर्षी दुर्वृत्तस्तेनासि हरिणः कृशः}


\twolineshloka
{नूनं मित्रमुखः शत्रुः कश्चिदार्यवदाचरन्}
{वञ्चयित्वा गतस्त्वां वै तेनासि हरिणः कृशः}


\twolineshloka
{नूनमद्य सतां मध्ये तव वाक्यमनुत्तमम्}
{न भाति कालेऽभिहितं तेनासि हरिणः कृशः}


\twolineshloka
{दृष्टपूर्वाञ्श्रुतपूर्वान्कुपितान्हृदयप्रियान्}
{अनुनेतुं न शक्रोषि तेनासि हरिणः कृशः}


\twolineshloka
{नूनमासञ्जयित्वा त्वा कृत्ये कस्मिंश्चिदीप्सिते}
{कच्चिदर्थयते नित्यं तेनासि हरिणः कृशः}


\twolineshloka
{परोक्षवादिभिर्मिथ्यादोषस्ते सम्प्रदर्शितः}
{तज्ज्ञैर्न पूज्यसे व्यक्तं तेनासि हरिणः कृशः}


\twolineshloka
{नूनं त्वां सद्गुणापेक्षं पूजयानं सुहृत्प्रजाः}
{मायावीति च जानन्ति तेनासि हरिणः कृशः}


\twolineshloka
{अन्तर्गतमभिप्रायंन न नूनं लज्जयेच्छसि}
{विवक्तुं प्राप्य शैथिल्यात्तेनासि हरिणः कृशः}


\twolineshloka
{नानाबुद्धिरुचीँल्लोके मानुषान्नूनमिच्छसि}
{ग्रहीतुं स्वैर्गुणैः सर्वांस्तेनासि हरिणः कृशः}


\twolineshloka
{असत्सु विनिविष्टेषु न गुणान्वदतः स्वयम्}
{गुणास्ते न विराजन्ते तेनासि हरिणः कृशः}


\twolineshloka
{धर्मवृत्तः श्रुतैर्हीनः पदं त्वं रजसान्वितः}
{महत्प्रार्थयसे नूनं तेनासि हरिणः कृशः}


\twolineshloka
{तपःप्रणिहितात्मानं मन्ये त्वारण्यकाङ्क्षिणम्}
{बन्धुवर्गो निगृह्णाति तेनासि हरिमः कृशः}


\twolineshloka
{इष्टभार्यस्य ते नूनं प्रातिवेश्यो महाधनः}
{युवा सुललितः कामी तेनासि हरिणः कृशः}


\twolineshloka
{दुर्विनीतहतः पुत्रो जामाता वाऽप्रमार्जकः}
{दारा वा प्रतिकूलास्ते तेनासि हरिमः कृशः}


\twolineshloka
{भ्रातरोऽतीव विषमाः पिता वा क्षुत्क्षतो मृतः}
{माता ज्योष्ठो गुरुर्वाऽपि तेनासि हरिमः कृशः}


\twolineshloka
{ब्राह्मणो वा हतो गौर्वा ब्रह्मस्वं वापहृतं पुरा}
{देवस्यं वा हृतं काले तेनासि हरिणः कृशः}


\twolineshloka
{हृतदारोऽथ वृद्धो वा लोके द्विष्टोऽथवा नरैः}
{अविज्ञानेन वा वृद्धस्तेनासि हरिणः कृशः}


\twolineshloka
{वार्धकार्थं धनं दृष्ट्वा स्वा श्रीर्वाऽपि परैर्हृता}
{वृत्तिर्वा दुर्जनापेक्षा तेनासि हरिणः कृशः}


\twolineshloka
{सम्पत्कालेन ते धर्मः क्षीणस्तात सुहृद्बुवैः}
{असंन्यासमतिस्तत्र तेनासि हरिणः कृशः}


\twolineshloka
{अविद्वान्भीरुरल्पार्थे विद्याविक्रमदानजम्}
{यशः प्रार्थयसे नित्यं तेनासि हरिणः कृशः}


\twolineshloka
{चिरहाभिलषितं किञ्चित्फलमप्राप्तमेव ते}
{कृतमन्यैरपहृतं तेनासि हरिणः कृशः}


\twolineshloka
{नूनमात्मगतं दोषमपश्यन्किंचिदात्मनि}
{अकारणेऽभिशस्तो हि तेनासि हरिणः कृशः}


\twolineshloka
{सुहृदां दुःखमार्तानां न प्रमोक्ष्यसि हानिजम्}
{अलमर्थगुणैर्हीनं तेनासि हरिणः कृशः}


\twolineshloka
{साधून्गृहस्थान्दृष्ट्वा च तथा साधून्वनेचरान्}
{मुक्तांश्चावसथे सक्तांस्तेनासि हरिमः कृशः}


\twolineshloka
{धर्म्यमर्थ्यं च काम्यं च देशे च रहितं वचः}
{न प्रसिद्ध्यति ते नूनं तेनासि हरिणः कृशः}


\twolineshloka
{दत्तानकुशलैरर्थान्मनीषी संजिजीविषुः}
{प्राप्य वर्तयसे नूनं तेनासि हरिणः कृशः}


\twolineshloka
{परस्परविरुद्धानां प्रियं नूनं चिकीर्षसि}
{सुहृदामुपरोधेन तेनासि हरिणः कृशः}


\twolineshloka
{पापान्विवर्धितान्दृष्ट्वा कल्याणांश्चावसीदतः}
{ध्रुवं गर्हयसे नूनं तेनासि हरिणः कृशः}


\twolineshloka
{एवं सम्पूजितं रक्षो विप्रं तं प्रत्यपूजयत्}
{सहायमकरोच्चैनं सम्पूज्यामुं मुमोच ह}


\chapter{अध्यायः १८८}
\twolineshloka
{जन्म मानुष्यकं प्राप्य कर्मक्षेत्रं सुदुर्लभम्}
{श्रेयोर्थिना दरिद्रेण किं कर्तव्यं पितामह}


\threelineshloka
{दानानामुत्तमं यच्च देयं यच्च यथायथा}
{मान्यान्पूज्यांश्च गाङ्गेय रहस्यं वक्तुमर्हसि ॥वैशम्पायन उवाच}
{}


\threelineshloka
{एवं पृष्टो नरेन्द्रेण पाण्डवेन यशस्विना}
{धर्माणां परमं गुह्यं भीष्मः प्रोवाच पार्थिवम् ॥भीष्म उवाच}
{}


\twolineshloka
{शृणुष्वावहितो राजन्धर्मगुह्यानि भारत}
{यथाहि भगवान्व्यासः पुरा कथितवान्मयि}


\twolineshloka
{देवगुह्यमिदं राजन्यमेनाक्लिष्टकर्मणा}
{नियमस्थेन युक्तेन तपसो महतः फलम्}


\twolineshloka
{येन यः प्रीयते देवः प्रीयन्ते पितरस्तथा}
{ऋषयः प्रमथाः श्रीश्च चित्रगुप्तो दिशां गजाः}


\twolineshloka
{ऋषिधर्मः स्मृतो यत्र सरहस्यो महाफलः}
{महादानफलं चैव सर्वयज्ञफलं तथा}


\twolineshloka
{यश्चैतदेवं जानीयाज्ज्ञात्वा वा कुरुतेऽनघ}
{सदोषोऽदोषवांश्चेह तैर्गुणैः सह युज्यते}


\twolineshloka
{दशसूनासमं चक्रं दशचक्रसमो ध्वजः}
{दशध्वजसमा वेश्या दशवेश्यासमो नृपः}


\twolineshloka
{अर्घेनैतानि सर्वाणि नृपतिः कथ्यतेऽधिकः}
{त्रिवर्गसहितं शास्त्रं पवित्रं पुण्यलक्षणम्}


\twolineshloka
{धर्मिव्याकरणं पुण्यं रहस्यश्रवणं महत्}
{श्रोतव्यं धर्मसंयुक्तं विहितं त्रिदशैः स्वयम्}


\twolineshloka
{पितॄणां यत्र गुह्यानि प्रोच्यन्ते श्राद्धकर्मणि}
{देवतानां च सर्वेषां रहस्यं कथ्यतेऽखिलम्}


\twolineshloka
{ऋषिधर्मः स्मृतो यत्र सरहस्यो महाफलः}
{महायज्ञफलं चैव सर्वदानफलं तथा}


\twolineshloka
{ये पठन्ति सदा मर्त्या येषां चैवोपतिष्ठति}
{श्रुत्वा च फलमाचष्टे स्वयं नारायणः प्रभुः}


\twolineshloka
{गवां फलं तीर्थफलं यज्ञानां चैव यत्फलम्}
{एतत्फलमवाप्नोति यो नरोऽतिथिपूजकः}


\twolineshloka
{श्रोतारः श्रद्धधानाश्च येषां शुद्धं च मानसम्}
{तेषां व्यक्तं जिता लोकाः श्रद्दधानेन साधुना}


\twolineshloka
{मुच्यते किल्बिषाच्चैव न स पापेन लिप्यते}
{धर्मं च लभते नित्यं प्रेत्य लोकगतो नरः}


\twolineshloka
{कस्यचित्त्वथ कालस्य देवदूतो यदृच्छया}
{स्थितो ह्यन्तर्हितो भूत्वा पर्यभाषत वासवम्}


\twolineshloka
{यो तौ कामगुणोपेतावश्विनौ भिषजां वरौ}
{आज्ञयाऽहं तयोः प्राप्तः सनरान्पितृदैवतान्}


\twolineshloka
{कस्माद्धि मैथुनं श्राद्धे दातुर्भोक्तुश्च वर्जितम्}
{केमर्थं च त्रयः पिण्डाः प्रविभक्ताः पृथक्पृथक्}


\twolineshloka
{प्रथमः कस्य दातव्यो मध्यमः क्व च गच्छति}
{उत्तरश्च स्मृतः कस्य एतदिच्छामि वेदितुम्}


\threelineshloka
{श्रद्दधानेन दूतेन भाषितं धर्मसंहितम्}
{पूर्वस्थास्त्रिदशाः सर्वे पितरः पूज्य खेचरम् ॥पितर ऊचुः}
{}


\twolineshloka
{स्वागतं तेऽस्तु भद्रं ते श्रूयतां खेचरोत्तम}
{गूढार्थः परमः प्रश्नो भवता समुदीरितः}


\twolineshloka
{श्राद्धं दत्त्वा च भुक्त्वा च पुरुषो यः स्त्रियं व्रजेत्}
{पितरस्तस्य तं मासं तस्मिन्रेतसि शेरते}


\twolineshloka
{प्रविभागं तु पिण्डानां प्रवक्ष्याम्यनुपूर्वशः}
{पिण्डो ह्यधस्ताद्गच्छंस्तु अप आविश्य भावयेत्}


\threelineshloka
{पिण्डं तु मध्यमं तत्र पत्नीत्वेका समश्नुते}
{पिण्डस्तृतीयो यस्तेषां तं दद्याज्जातवेदसि}
{एष श्राद्धविधिः प्रोक्तो यथा धर्मो न लुप्यते}


\threelineshloka
{पितरस्तस्य तुष्यन्ति प्रहृष्टमनसः सदा}
{प्रजा विवर्धते चास्य अक्षयं चोपतिष्ठति ॥देवदूत उवाच}
{}


\twolineshloka
{आनुपूर्व्येण पिण्डानां प्रविभागः पृथक्पृथक्}
{पितॄणां त्रिषु सर्वेषां निरुक्तं कथितं त्वया}


\twolineshloka
{एकः समुद्धृतः पिण्डो ह्यधस्तात्कस्य गच्छति}
{कं वा प्रीणयते देवं कथं तारयते पितॄन्}


\twolineshloka
{मध्यमं तु तदा पत्नी भुङ्क्तेऽनुज्ञातमेव हि}
{किमर्थं पितरस्तस्य कव्यमेव च भुञ्जते}


\twolineshloka
{अत्र यस्त्वन्तिमः पिण्डो गच्छते जातवेदसम्}
{भवते का गतिस्तस्य कं वा समनुगच्छति}


\threelineshloka
{एतदिच्छाम्यहं श्रोतुं पिण्डेषु त्रिषु या गतिः}
{फलं वृत्तिं च मार्गं च यश्चैनं प्रतिपद्यते ॥पितर ऊचुः}
{}


\twolineshloka
{सुमहानेष प्रश्नो वै यस्त्वया समुदीरितः}
{रहस्यमद्भुतं चापि पृष्टाः स्म गगनेचर}


\threelineshloka
{एतदेव प्रशंसन्ति देवाश्च मुनयस्तथा}
{तेऽप्येवं नाभिजानन्ति पितृकार्यविनिश्चयम्}
{वर्जयित्वा महात्मानं चिरजीविनमुत्तमम्}


\twolineshloka
{पितृभक्तस्तु यो विप्रो वलब्धो महायशाः}
{त्रयाणामपि पिण्डानां श्रुत्वा भगवतो गतिम्}


\twolineshloka
{देवदूतेन यः पृष्टः श्राद्धस्य विदिनिश्चयः}
{गतिं त्रयाणां पिण्डानां शृणुष्वावहितो मम}


\twolineshloka
{अपो गच्छति यो ह्यत्र शशिनं ह्येष प्रीणयेत्}
{शशी प्रीणयते देवान्पितॄंश्चैव महामते}


\twolineshloka
{भुङ्क्ते तु पत्नीं यं चैषामनुज्ञाता तु मध्यमम्}
{पुत्रकामाय पुत्रं तु प्रयच्छन्ति पितामहाः}


\threelineshloka
{हव्यवाहे तु यः पिण्डो दीयते तन्निबोध मे}
{पितरस्तेन तृप्यन्ति प्रीताः कामान्दिशन्ति च}
{एतत्ते कथितं सर्वं त्रिषु पिण्डेषु या गतिः}


\twolineshloka
{ऋत्विग्यो यजमानस्य पितृत्वमनुगच्छति}
{तस्मिन्नहनि मन्यन्ते परिहार्यं हि मैथुनम्}


\twolineshloka
{शुचिना तु सदा श्राद्धं भोक्तव्यं खेचरोत्तम}
{ये मया कथिता दोषास्ते तथा स्युर्न चान्यथा}


\twolineshloka
{तस्मात्स्नातः शुचिः क्षान्तः श्राद्धं भुञ्जीत वै द्विजः}
{प्रजा विवर्दते चास्य चश्चैवं सम्प्रयच्छति}


\twolineshloka
{ततो विद्युत्प्रभो नाम ऋषिराह महातपाः}
{आदित्यतेजसा तस्य तुल्यं रूपं प्रकाशते}


% Check verse!
स च धर्मरहस्यानि श्रुत्वा शक्रमथाब्रवीत्
\threelineshloka
{तिर्यग्योनिगतान्सत्वान्मर्त्या हिंसन्ति मोहिताः}
{कीटान्पिपीलिकान्सर्पान्मेषान्समृगपक्षिणः}
{किल्बिषं सुबहु प्राप्ताः किंस्विदेषां प्रतिक्रिया}


\threelineshloka
{ततो देवगणाः सर्वे ऋषयश्च तपोधनाः}
{पितरश्च महाभागाः पूजयन्ति स्म तं मुनिम् ॥शक्र उवाच}
{}


\threelineshloka
{कुरुक्षेत्रं गयां गङ्गां प्रभासं पुष्कराणि च}
{एतानि मनसा ध्यात्वा अवगाहेत्ततो जलम्}
{तथा मुच्यति पापेनि राहुणा चन्द्रमा यथा}


\twolineshloka
{त्र्यहं स्नातः स भवति निराहारश्च वर्तते}
{स्पृशते यो गवां पृष्ठं वालधिं च नमस्यति}


\twolineshloka
{ततो विद्युत्प्रभो वाक्यमभ्यभाषत वासवम्}
{अयं सूक्ष्मतरो धर्मस्तं निबोध शतक्रतो}


\twolineshloka
{घृष्टो वटकषायेणि अनुलिप्तः प्रियङ्गुणा}
{क्षीरेण षष्टिकान्भुक्त्वा सर्वपापैः प्रमुच्यते}


\threelineshloka
{श्रूयतां चापरं गुह्यं रहस्यमृषिचिन्तितम्}
{श्रुतं मे भाषमाणस्य स्थाणोः स्थाने बृहस्पतेः}
{रुद्रेण सह देवेश तन्निबोध शचीपते}


\threelineshloka
{पर्वतारोहणं कृत्वा एकपादो विभावसुम्}
{निरीक्षेत निराहार ऊर्ध्वबाहुः कृताञ्जलिः}
{तपसा महता युक्त उपवासफलं लभेत्}


\twolineshloka
{रश्मिभिस्तापितोऽर्कस्य सर्वपापमपोहति}
{ग्रीष्मकालेऽथवा शीते एवं पापमपोहति}


\twolineshloka
{ततः पापात्प्रमुक्तस्य द्युतिर्भवति शाश्वती}
{तेजसा सूर्यवद्दीप्तो भ्राजते सोमवत्पनः}


\twolineshloka
{मध्ये त्रिदशवर्गस्य देवराजः शतक्रतुः}
{उवाच मधुरं वाक्यं बृहस्पतिमनुत्तमम्}


\threelineshloka
{धर्मगुह्यं तु भगवन्मानुषाणां सुखावहम्}
{सरहस्याश्च ये दोषास्तान्यथावदुदीरथ ॥बृहस्पतिरुवाच}
{}


\twolineshloka
{प्रतिमेहन्ति ये सूर्यमनिलं द्विषते च ये}
{हव्यवाहे प्रदीप्ते च समिधं ये न जुह्वति}


\twolineshloka
{बालवत्सां च ये धेनुं दुहन्ति क्षीरकारणात्}
{तेषां दोषान्प्रवक्ष्यामि तान्निबोध शचीपते}


\twolineshloka
{भानुमाननिलश्चैव हव्यवाहश्च वासव}
{लोकानां मातरश्चैव गावः सृष्टाः स्वयंभुवा}


\twolineshloka
{लोकांस्तारयितुं शक्ता मर्त्येष्वेतेषु देवताः}
{सर्वे भवन्तः शृणिवन्तु एकैकं धर्मनिश्चयम्}


\twolineshloka
{वर्षाणि षडशीतिं तु दुर्वृत्ताः कुलपांसनाः}
{स्त्रियः सर्वाश्च दुर्वृत्ताः प्रतिमेहन्ति या रविम्}


\threelineshloka
{अनिलद्वेषिणः शक्र गर्भस्था च्यवते प्रजा}
{हव्यवाहस्य दीप्तस्य समिधं ये न जुह्वति}
{अग्निकार्येषु वै तेषां हव्यं नाश्नाति पावकः}


\twolineshloka
{क्षीरं तु बालवत्सानां ये पिबन्तीह मानवाः}
{न तेषां क्षीरपाः केचिज्जायन्ते कुलवर्धनाः}


\twolineshloka
{प्रजाक्षयेण युज्यन्ते कुलवंशक्षयेण च}
{एवमेतत्पुरा दृष्टं कुलवृद्धैर्द्विजातिभिः}


\twolineshloka
{तस्माद्वर्ज्यानि वर्ज्यानि कार्यं कार्यं च नित्यशः}
{भूतिकामेनि मर्त्येन सत्यमेतद्ब्रवीमि ते}


\twolineshloka
{ततः सर्वा महाभाग देवताः समरुद्गणाः}
{ऋषयश्च महाभागाः पृच्छन्ति स्म पितॄंस्ततः}


\twolineshloka
{पितरः केन तुष्यन्ति मर्त्यानामल्पचेतसाम्}
{अक्षयं च कथं दानं भवेच्चैवौर्ध्वदेहिकम्}


\threelineshloka
{आनृण्यं वा कथं मर्त्या गच्छेयुः केन कर्मणा}
{एतदिच्छामहे श्रोतुं परं कौतूहलं हि नः ॥पितर ऊचुः}
{}


\twolineshloka
{न्यायतो वै महाभागाः संशयः समुदाहृतः}
{श्रूयतां येन तुष्यामो मर्त्यानां साधुकर्मणाम्}


\twolineshloka
{नीलषण्डप्रमोक्षेण अणावास्यां तिलोदकैः}
{वर्षासु दीपकैश्चैव पितॄणामनृणो भवेत्}


\twolineshloka
{अक्षयं निर्व्यलीकं च दानमेतन्महाफलम्}
{अस्माकं परितोषश्च अक्षयः परिकीर्त्यते}


\twolineshloka
{श्रद्धधानाश्च ये मर्त्या आहरिष्यन्ति सन्ततिम्}
{दुर्गात्ते तारयिष्यन्ति नरकात्प्रपितामहान्}


\twolineshloka
{पितॄणां भाषितं श्रुत्वा हृष्टरोमा तपोधनः}
{वृद्धगार्ग्यो महातेजास्तानेवं वाक्यमब्रवीत्}


\threelineshloka
{के गुणा नीलषण्डस्य प्रमुक्तस्य तपोधनाः}
{वर्षासु दीपदानेन तथैव च तिलोदकैः ॥पितर ऊचुः}
{}


\twolineshloka
{नीलषण्डस्य लाङ्गूलं तोयमभ्युद्दरेद्यदि}
{षष्टिं वर्षसहस्राणि पितरस्तेन तर्पिताः}


\twolineshloka
{यस्तु शृङ्गगतं पङ्क्तं कूलादुद्धृत्य तिष्ठति}
{पितरस्तेन गच्छन्ति सोमलोकमसंशयम्}


\twolineshloka
{वर्षासु दीपदानेन शशीवच्छोभते नरः}
{तमोरूपं न तस्यास्ति दीपकं यः प्रयच्छति}


\threelineshloka
{अमावास्यां तु ये मर्त्याः प्रयच्छन्ति तिलोदकम्}
{पात्रमौदुम्बरं गृह्य मधुमिश्रं तपोधन}
{कृतं भवति तैः श्राद्धं सरहस्यं यथार्थवत्}


\threelineshloka
{हृष्टपुष्टमनास्तेषां प्रजा भवति नित्यदा}
{कुलवंशस्य वृद्धिस्तु पिण्डदस्य फलं भवेत्}
{श्रद्दधानस्तु यः कुर्यात्पितॄणामनृणो भवेत्}


\twolineshloka
{एवमेव समुद्दिष्टः श्राद्धकालक्रमस्तथा}
{विधिः पात्रं फलं चैव यथावदनुकीर्तितम् ॥]}


\chapter{अध्यायः १८९}
\twolineshloka
{केन ते च भवेत्प्रीतिः कथं तुष्टिं तु गच्छसि}
{इति पृष्टः सुरेन्द्रेण प्रोवाच हरिरीश्वरः}


\twolineshloka
{ब्राह्मणानां परीवादो मम विद्वेषणं महत्}
{ब्राह्मणैः पूजितैर्नित्यं पूजितोऽहं न संशयः}


\twolineshloka
{नित्याभिवाद्या विप्रेन्द्रा भुक्त्वा पादौ तथाऽऽत्मनः}
{तेषां तुष्यामि मर्त्यानां यश्चक्रे च बलिं हरेत्}


\threelineshloka
{वामनं ब्राह्मणं दृष्ट्वा वराहं च जलोत्थितम्}
{उद्धृतां धरणीं चैव मूर्ध्ना धारयते तु यः}
{न तेषामशुभं किञ्चित्कल्मषं चोपपद्यते}


\twolineshloka
{अश्वत्थं रोचनां गां च पूजयेद्यो नरः सदा}
{पूजितं च जगत्तेन सदेवासुरमानुषम्}


\twolineshloka
{तेन रूपेण तेषां च पूजां गृह्णामि तत्त्वतः}
{पूजा ममैषा नास्त्यन्या यावल्लोकाः प्रतिष्ठिताः}


\threelineshloka
{अन्यथा हि वृथा मर्त्याः पूजयन्त्यल्पबुद्ध्यः}
{नाहं तत्प्रतिगृह्णामि न सा तुष्टिकरी मम ॥इन्द्र उवाच}
{}


\twolineshloka
{चक्रं पादौ वराहं च ब्राह्मणं चापि वामनम्}
{उद्धृतां धरणीं चैव किमर्थं त्वं प्रशंससि}


\threelineshloka
{भवान्सृजति भूतानि भावन्संहरति प्रजाः}
{प्रकृतिः सर्वभूतानां समर्त्यानां सनातनी ॥भीष्म उवाच}
{}


\twolineshloka
{सम्प्रहस्य ततो विष्णुरिदं वचनमब्रवीत्}
{चक्रेण निहता दैत्याः पद्म्यां क्रान्ता वसुन्धरा}


\threelineshloka
{वाराहं रूपमास्थाय हिरण्याक्षो निपातितः}
{वामनं रूपमास्थाय जितो राजा मया बलिः}
{}


\twolineshloka
{परितुष्टो भवाम्येवं मानुषाणां महात्मनाम्}
{तन्मां ये पूजयिष्यन्ति नास्ति तेषां पराभवः}


\twolineshloka
{अपि वा ब्राह्म्णं दृष्ट्वा ब्रह्म चारिणमागतम्}
{ब्राह्मणाग्र्याहुतिं दत्त्वा अमृतं तस्य भोजनम्}


\twolineshloka
{ऐन्द्रीं संध्यामुपासित्वा आदित्याभिमुखः स्थितः}
{सर्वतीर्थेषु स स्नातो मुच्यते सर्वकिल्बिषैः}


\threelineshloka
{एतद्वः कथितं गुह्यमखिलेनि तपोधनाः}
{संशयं पृच्छमानानां किं भूयः कथयाम्यहम् ॥बलदेव उवाच}
{}


\twolineshloka
{श्रूयतां परमं गुह्यं मानुषाणां सुखावहम्}
{अजानन्तो यदबुधाः क्लिश्यन्ते भूतपीडिताः}


\twolineshloka
{कल्य उत्थाय यो मर्त्यः स्पृशेद्गां वै घृतं दधि}
{सर्षपं च प्रियङ्गं च कल्मषात्प्रतिमुच्यते}


\threelineshloka
{भूतानि चैव सर्वाणि अग्रतः पृष्ठतोपि वा}
{उच्छिष्टं वाऽपि च्छिद्रेषु वर्जयन्ति तपोधनाः ॥देवा ऊचुः}
{}


\twolineshloka
{प्रगृह्यौदुम्बरं पात्रं तोयपूर्णमुदङ्भुखः}
{उपवासं तु गृह्णीयाद्यद्वा सङ्कल्पयेद्व्रतम्}


\twolineshloka
{देवतास्तस्य तुष्यन्ति कामिकं चापि सिध्यति}
{अन्यथा हि वृथा मर्त्याः कुर्वते स्वल्पबुद्धयः}


\twolineshloka
{उपवासे बलौ चापि ताम्रपात्रं विशिष्यते}
{बलिर्भिक्षा तथाऽर्ध्यं च पितॄणां च तिलोदकं}


\threelineshloka
{ताम्रपात्रेण दातव्यमन्यथाऽल्पफलं भवेत्}
{गुह्यमेतत्समुद्दिष्टं यथा तुष्यन्ति देवताः ॥धर्म उवाच}
{}


\twolineshloka
{राजपौरुषिके विप्रे घाण्टिके परिचारिके}
{गोरक्षके वाणिजके तथा कारुकुशीलवे}


\threelineshloka
{मित्रद्रुह्यनधीयाने यश्च स्याद्वृषलीपतिः}
{एतेषु दैवं पित्र्यं वा न देयं स्यात्कथञ्चन}
{पिण्डदास्तस्य हीयन्ते न च प्रीणाति वै पितॄन्}


\threelineshloka
{अतिथिर्यस्य भग्नाशो गृहात्प्रतिनिवर्तते}
{पितरस्तस्य देवाश्च अग्नयश्च तथैव हि}
{निराशाः प्रतिगच्छन्ति अतिथेरप्रतिग्रहात्}


\threelineshloka
{स्त्रीघ्नैर्गोघ्नैः कृतघ्नैश्च ब्रह्मघ्नैर्गुरुतल्पगैः}
{तुल्यदोषो भवत्येभिर्यस्यातिथितरनर्चितः ॥अग्निरुवाच}
{}


\threelineshloka
{पादमुद्यम्य यो मर्त्यः स्पृशेद्गाश्च सुदुर्मतिः}
{ब्राह्मणं वा महाभागं दीप्यमानं तथाऽनलम्}
{तस्य दोषान्प्रवक्ष्यामि तच्छृणुध्वं समाहिताः}


\threelineshloka
{दिवं स्पृशत्यशब्दोऽस्य त्रस्यन्ति पितरश्च वै}
{वैमनस्यं च देवानां कृतं भवति पुष्कलम्}
{पावकश्च महातेजा हव्यं न प्रतिगृह्णति}


\twolineshloka
{आजन्मनां शतं चैव नरके पच्यते तु सः}
{निष्कृतिं च न तस्यापि अनुमन्यन्ति कर्हिचित्}


\threelineshloka
{तस्माद्गावो न पादेन स्प्रष्टव्या वै कदाचन}
{ब्राह्मणश्च महातेजा दीप्यमानस्तथाऽनलः}
{श्रद्दधानेन मर्त्येन आत्मनो हितमिच्छता}


\threelineshloka
{एते दोषा मया प्रोक्तास्त्रिषु यः पादमुत्सृजेत्}
{विश्वामित्र उवाच}
{}


\twolineshloka
{श्रूयतां परमं गुह्यं रहस्यं धर्मसंहितम्}
{परमान्नेन यो दद्यात्पितॄणामौपहारिकम्}


\twolineshloka
{गजच्छायायां पूर्वस्यां कुतपे दक्षिणामुखः}
{यदा भाद्रपदे मासि भवते बहुले मघा}


\twolineshloka
{श्रूयतां तस्य दानस्य यादृशो गुणविस्तरः}
{कृतं तेन महच्छ्राद्धं वर्षाणीह त्रयोदश}


\twolineshloka
{बहपल् समङ्गे ह्यकुतोभये चक्षेमे च संख्येव हि भूयसी च}
{यथा पुरा ब्रह्मपुरे सवत्साशतक्रतोर्वज्रधरस्य यज्ञे}


\twolineshloka
{भूयश्च या विष्णुपदे स्थितायाविभावसोश्चापि पथे स्थिता या}
{देवाश्च सर्वे सह नारदेनप्रकुर्वते सर्वसहेति नाम}


\twolineshloka
{मन्त्रेणैतेनाभिवन्देत यो वैविमुच्यते पापकृतेन कर्मणा}
{लोकानवाप्नोति पुरंदरस्यगवां फलं चन्द्रमसो द्युतिं च}


\threelineshloka
{एवं हि मन्त्रं त्रिदशाभिजुष्टंपठेत यः पर्वसु गोष्ठमध्ये}
{न तस्यि पापं न भयं न शोकःसहस्रनेत्रस्य च याति लोकम् ॥भीष्म उवाच}
{}


\threelineshloka
{अथ सप्त महाभाग ऋषयो लोकविश्रुताः}
{वसिष्ठप्रमुखाः सर्वे ब्रह्मणं पद्मसम्भवम्}
{}


\twolineshloka
{प्रदक्षिणमभिक्रम्य सर्वे प्राञ्जलयः स्थिताः ॥उवाच वचनं तेषां वसिष्ठो ब्रह्मवित्तमः}
{}


\twolineshloka
{सर्वप्राणिहितं प्रश्नं ब्रह्मिक्षत्रे विशेषतः ॥द्रव्यहीनाः कथं मर्त्या दरिद्राः साधुवर्तिनः}
{}


% Check verse!
प्राप्नुवन्तीह यज्ञस्य फलं केन च कर्मणा ॥एतच्छ्रुत्वा वचस्तेषां ब्रह्मा वचनमब्रवीत्
\twolineshloka
{अहो प्रश्नो महाभाग गूढार्थः परमः शुभः}
{सूक्ष्मः श्रेयांश्च मर्त्यानां भवद्भिः समुदाहृतः}


\twolineshloka
{श्रूयतां सर्वमाख्यास्ये निखिलेन तपोधनाः}
{यथा यज्ञफलं मर्त्यो लभते नात्र संशयः}


\twolineshloka
{पौषमासस्य शुक्ले वै यदा युज्येत रोहिणी}
{तेन नक्षत्रयोगेन आकाशशयनो भवेत्}


\twolineshloka
{एकवस्त्रः शुचिः स्नातः श्रद्दधानः समाहितः}
{सोमस्य रश्मयः पीत्वा महायज्ञफलं लभेत्}


\twolineshloka
{एतद्वः परमं गुह्यं कथितं द्विजसत्तमाः}
{यन्मां भवन्तः पृच्छन्ति सूक्ष्मतत्त्वार्थदर्शिनः]}


\chapter{अध्यायः १९०}
\twolineshloka
{सलिलस्याञ्जलिं पूर्णमक्षताश्च धृतोत्तराः}
{सोमस्योत्तिष्ठमानस्य तज्जलं चाक्षतांश्च तान्}


\twolineshloka
{स्थितो ह्यभिमुखो मर्त्यः पौर्णमास्यां बलिं हरेत्}
{अग्निकार्यं कृतं तेन हुताश्चास्याग्नयस्त्रयः}


\twolineshloka
{वनस्पतिं च यो हन्यादमावास्यामबुद्धिमान्}
{अपि ह्येकेन पत्रेण लिप्यते ब्रह्महत्यया}


\twolineshloka
{दन्तकाष्ठं तु यः खादेदमावास्यामबुद्धिमान्}
{हिंसितश्चन्द्रमा**** पितरश्चोद्विजन्ति च}


\threelineshloka
{हव्यं न तस्य देवाश्च प्रतिगृह्णन्ति पर्वसु}
{कुप्यन्ते पितरश्चास्य कुले वंशोऽस्य हीयते ॥श्रीरुवाच}
{}


\twolineshloka
{प्रकीर्णं भाजनं यत्र भिन्नभाण्डमथासनम्}
{योषितश्चैव हन्यन्ते कश्मलोपहते गृहे}


\threelineshloka
{देवताः पितरश्चैव उत्सवे पर्वणीषु वा}
{निराशाः प्रतिगच्छन्ति कश्मलोपहताद्गृहात् ॥अङ्गिरा उवाच}
{}


\threelineshloka
{यस्तु संवत्सरं पूर्णं दद्याद्दीपं करञ्जके}
{सुवर्चलामूलहस्तः प्रजा तस्य विवर्धते ॥गार्ग्य उवाच}
{}


\twolineshloka
{आतीथ्यं सततं कुर्याद्दीपं दद्यात्प्रतिश्रये}
{वर्जयानो दिवास्वापं न च मांसानि भक्षयेत्}


\twolineshloka
{गोब्राह्मणं न हिंस्याच्च पुष्कराणि च कीर्तयेत्}
{एत श्रेष्ठतमो धर्मः सरहस्यो महाफलः}


\twolineshloka
{अपि क्रतुशतैरिष्ट्वा क्षयं गच्छति तद्धविः}
{न तु क्षीयन्ति ते धर्माः श्रद्दधानैः प्रयोजिताः}


\twolineshloka
{इदं च परमं गुह्यं सरहस्यं निबोधत}
{श्राद्धकल्पे च दैवे च तैर्थिके पर्वणीषु च}


\threelineshloka
{रजस्वला च या नारी श्वित्रिकाऽपुत्रिका च या}
{एताभिश्चक्षुषा दृष्टं हविर्नाश्नन्ति देवताः}
{पितरश्च न तुष्यन्ति वर्षाण्यपि त्रयोदश}


\threelineshloka
{शुक्लवासाः शुचिर्भूत्वा ब्राह्मणान्स्वस्ति वाचयेत्}
{कीर्तयेद्भारतं चैव तथा स्यादक्षयं हविः ॥धौम्य उवाच}
{}


\twolineshloka
{भिन्नभाण्डं च खट्वां च कुक्कुटं शुनकं तथा}
{अप्रशस्तानि सर्वाणि यश्च वृक्षो गृहेरुहः}


\fourlineindentedshloka
{भिन्नभाण्डे कलिं प्राहुः खट्वायां तु धनक्षयः}
{कुक्कुटे शुनके चैव हविर्नाश्नन्ति देवताः}
{वृक्षमूले ध्रुवं सत्वं तस्माद्वृक्षं न रोपयेत् ॥जमदग्निरुवाच}
{}


\twolineshloka
{यो यजेदश्वमेधेन वाजपेयशतेन ह}
{अवाक्शिरा वा लम्बेत सत्रं वा स्फीतमाहरेत्}


\twolineshloka
{न यस्य हृदयं शुद्धं नरकं स ध्रुवं व्रजेत्}
{तुल्यं यज्ञश्च सत्यं च हृदयस्य च शुद्धता}


\twolineshloka
{शुद्धेन मनसा दत्त्वा सक्तुप्रस्थं द्विजातये}
{ब्रह्मलोकमनुप्राप्तः पर्याप्तं तन्निदर्शनम् ॥]}


\chapter{अध्यायः १९१}
\twolineshloka
{किञ्चिद्धर्मं प्रवक्ष्यामि मानुषाणां सुखावहम्}
{सरहस्याश्च ये दोषास्ताञ्शृणुध्वं समाहिताः}


\twolineshloka
{अग्निकार्यं च कर्तव्यं परमान्नेन भोजनम्}
{दीपकश्चापि कर्तव्यः पितॄणां सतिलोदकः}


\twolineshloka
{एतेन विधिना मर्त्यः श्रद्दधानः समाहितः}
{चतुरो वार्षिकान्मासान्यो ददाति तिलोदकाम्}


\twolineshloka
{भोजनं च यथाशक्त्या ब्राह्मणे वेदपारगे}
{पशुबन्धशतस्येह फलं प्राप्नोति पुष्कलम्}


\twolineshloka
{इदं चैवापरं गुह्यमप्रशस्तं निबोधत}
{अग्नेस्तु वृषलो नेता हविर्मूढाश्च योषितः}


\twolineshloka
{मन्यते धर्म एवेति च चाधर्मेणि लिप्यते}
{अग्नयस्तस्य कुप्यन्ति शूद्रयोनिं स गच्छति}


\threelineshloka
{पितरश्च न तुष्यन्ति सहदेवैर्विशेषतः}
{प्रायश्चित्तं तु यत्तत्र ब्रुवतस्तन्निबोध मे}
{यत्कृत्वा तु नरः सम्यक्सुखी भवति विज्वरः}


\twolineshloka
{गवां मूत्रपुरीषेणि पयसा च घृतेन च}
{अग्निकार्यं त्र्यहं कुर्यान्निराहारः समाहितः}


\twolineshloka
{ततः संवत्सरे पूर्णे प्रतिगृह्णन्ति देवताः}
{हृष्यन्ति पितरश्चास्य श्राद्धकाल उपस्थिते}


\twolineshloka
{एष ह्यधर्मो धर्मश्च सरहस्यः प्रकीर्तितः}
{मर्त्यानां स्वर्गकामानां प्रेत्य स्वर्गसुखावह ॥]}


\chapter{अध्यायः १९२}
\twolineshloka
{परदारेषु ये सक्ता अकृत्वा दारसङ्ग्रहम्}
{निराशाः पितरस्तेषां श्राद्धकाले भवन्ति वै}


\twolineshloka
{परदाररतिर्यश्च यश्च वन्ध्यामुपासते}
{ब्रह्मस्वं हरते यश्च समदोषा भवन्ति ते}


\twolineshloka
{असम्भाष्या भवन्त्येते पितॄणां नात्र संशयः}
{देवताः पितरश्चैषां नाभिनन्दन्ति तद्धविः}


\twolineshloka
{तस्मात्परस्य वै दारांस्त्यजेद्वन्ध्यां च योषितम्}
{ब्रह्मस्वं हि न हर्तव्यमात्मनो हितमिच्छता}


\twolineshloka
{श्रूयतां चापरं गुह्यं रहस्यं धर्मसंहितम्}
{श्रद्दधानेन कर्तव्यं गुरुणां वचनं सदा}


\twolineshloka
{द्वादश्यां पौर्णमास्यां च मासिमासि घृताक्षतम्}
{ब्राह्मणेभ्यः प्रयच्छेत तस्य पुण्यं निबोधत}


\twolineshloka
{सोमश्च वर्धते तेन समुद्रश्च महोदधिः}
{अश्वमेधचतुर्भागं फलं सृजति वासवः}


\twolineshloka
{दानेनैतेन तेजस्वी वीर्यवांश्च भवेन्नरः}
{प्रीतश्च भगवान्सोम इष्टान्कामान्प्रयच्छति}


\twolineshloka
{श्रूयतां चापरो धर्मः सरहस्यो महाफलः}
{इदं कलियुगं प्राप्य मनुष्याणां सुखावहः}


\twolineshloka
{कल्यमुत्थाय यो मर्त्यः स्नातः शुक्लेन वाससा}
{तिलपात्रं प्रयच्छेत ब्राह्मणेभ्यः समाहितः}


\twolineshloka
{तिलोदकं च यो दद्यात्पितॄणां मधुना सह}
{दीपकं कृसंर चैव श्रूयतां तस्य यत्फलम्}


\twolineshloka
{तिलपात्रे फलं प्राह भगवान्पाकशांसनः}
{गोप्रदानं च यः कुर्याद्भूमिदानं च शाश्वतम्}


\twolineshloka
{अग्निष्टोमं च यो यज्ञं यजेत बहुदक्षिणम्}
{तिलपात्रं सहैतेन समं मन्यन्ति देवताः}


\twolineshloka
{तिलोदकं सदा श्राद्धे मन्यन्ते पितरोऽक्षयम्}
{दीपे च कृसरे चैव तुष्यन्तेऽस्य पितामहाः}


\twolineshloka
{स्वर्गे च पितृलोके च पितृदेवाभिपूजितम्}
{एवमेतन्मयोद्दिष्टपिदृष्टं पुरातनम् ॥]}


\chapter{अध्यायः १९३}
\twolineshloka
{ततस्त्वृषिगणाः सर्वे पितरश्च सदेवताः}
{अरुन्धतीं तपोवृद्धामपृच्छन्त समाहिताः}


\fourlineindentedshloka
{समानशीलां वीर्येण वसिष्ठस्य महात्मनः}
{त्वत्तो धर्मरहस्यानि श्रोतुमिच्छामहे वयम्}
{यत्ते गुह्यतमं भद्रे तत्प्रभाषितुमर्हसि ॥अरुन्धत्युवाच}
{}


\twolineshloka
{तपोवृद्धिर्मया प्राप्ता भवतां स्मरणेन वै}
{भवतां च प्रसादेन धर्मान्वक्ष्यामि शाश्वतान्}


\twolineshloka
{सगुह्यान्सरहस्यांश्च ताञ्शृणुद्वमशेषतः}
{श्रद्दधाने प्रयोक्तव्या यस्य शुद्धं तथा मनः}


\twolineshloka
{अश्रद्दधानो मानी च ब्रह्महा गुरुतल्पग}
{असम्भाष्या हि चत्वारो नैषां धर्मं प्रकाशयेत्}


\twolineshloka
{अहन्यहनि यो दद्यात्कपिलां द्वादशीः समाः}
{मासिमासि च सत्रेण यो यजेत सदा नरः}


\twolineshloka
{गवां शतसहस्रं च यो दद्याज्ज्येष्ठपुष्करे}
{न तद्धर्मफलं तुल्यमतिथिर्यस्य तुष्यति}


\twolineshloka
{श्रूयतां चापरो धर्मो मनुष्याणां सुखावहः}
{श्रद्दधानेन कर्तव्यः सरहस्यो महाफलः}


\threelineshloka
{कल्यमुत्थाय गोमध्ये गृह्य दर्भान्सहोदकान्}
{निषिञ्चेत गवां शृङ्गे मस्तकेन च तज्जलम्}
{प्रतीच्छेत निराहारस्तस्यि धर्मफलं शृणु}


\threelineshloka
{श्रूयन्ते यानि तीर्थानि त्रिषु लोकेषु कानिचित्}
{सिद्धचारणजुष्टानि सेवितानि महर्षिभिः}
{अभिषेकः समस्तेषां गवां शृङ्गोदकस्य च}


\threelineshloka
{साधुसाध्विति चोद्दिष्टं दैवतैः पितृभिस्तथा}
{भूतैश्चैव सुसंहृष्टैः पूजिता साऽप्यरुन्धती ॥पितामह उवाच}
{}


\threelineshloka
{अहो धर्मो महाभागे सरहस्य उदाहृतः}
{वरं ददामि ते धन्ये तपस्ते वर्दतां सदा ॥यम उवाच}
{}


\twolineshloka
{रमणीया कथा दिव्या युष्मत्तो या मया श्रुता}
{श्रूयतां चित्रगुप्तस्य भाषितं मम च प्रियम्}


\twolineshloka
{रहस्यं धर्मसंयुक्तं शक्यं श्रोतुं महर्षिभिः}
{श्रद्दधानेन मर्त्येन आत्मनो हितमिच्छाता}


\twolineshloka
{न हि पुण्यं तथा पापं कृतं किञ्चिद्विनश्यति}
{पर्वकाले च यत्किंचिदादित्यं चाधितिष्ठति}


\twolineshloka
{प्रेतलोकं गते मर्त्ये तत्तत्सर्वं विभावसुः}
{प्रतिजानाति पुण्यात्मा तच्चि तत्रोपयुज्यते}


\twolineshloka
{किञ्चिद्धर्म प्रवक्ष्यामि चित्रगुप्तमतं शुभम्}
{पानीयं चैव दीपं च दातव्यं सततं तथा}


\twolineshloka
{उपानहौ च च्छत्रं च कपिला च यथातथम्}
{पुष्करे कपिला देया ब्राह्मणे वेदपारगे}


\twolineshloka
{अग्निहोत्रं च यत्नेन सर्वशः प्रतिपालयेत्}
{अयं चैवापरो धर्मश्चित्रगुप्तेन भाषितः}


\twolineshloka
{फलमस्य पृथक्त्वेन श्रोतुमर्हन्ति सत्तमाः}
{प्रलयं सर्वभूतैस्तु गन्तव्यं कालपर्ययात्}


\twolineshloka
{तत्र दुर्गमनुप्राप्ताः क्षुत्तृष्णापरिपीडिताः}
{दह्यमाना विपच्यन्ते न तत्रास्ति पलायनम्}


\twolineshloka
{अन्धकारं तमो घोरं प्रविशन्त्यल्पबुद्धयः}
{तत्र धर्मं प्रवक्ष्यामि येन दुर्गाणि सन्तरेत्}


\twolineshloka
{अल्पव्ययं महार्थं च प्रेत्य चैव सुखोदयम्}
{पानीयस्य गुणा दिव्याः प्रेतलोके विशेषतः}


\twolineshloka
{तत्र पुण्योदका नाम नदी तेषां विधीयते}
{अक्षयं सलिलं तत्र शीतलं ह्यमृतोपमम्}


\twolineshloka
{स तत्र तोयं पिबति पीनीयं यः प्रयच्छति}
{प्रदीपस्य प्रदानेन श्रूयतां गुणविस्तरः}


\threelineshloka
{तमोन्धकारं नियतं दीपदो न प्रपश्यति}
{प्रभां चास्य प्रयच्छन्ति सोमभास्करपावकाः}
{देवताश्चानुमन्यन्ते विमलाः सर्वतो दिशः}


\twolineshloka
{द्योततें च यथाऽऽदित्यः प्रेतलोकगतो नरः}
{तस्माद्दीपः प्रदातव्यः पानीयं च विशेषतः}


\twolineshloka
{कपिलां ये प्रयच्छन्ति ब्राह्मणे वेदपारगे}
{पुष्करे च विशेषेण श्रूयतां तस्य यत्फलम्}


\threelineshloka
{गोशतं सवृषं तेन दत्तं भवति शाश्वतम्}
{पापं कर्म च यत्किञ्चिद्ब्रह्महत्यासमं भवेत्}
{शोधयेत्कपिला ह्येका प्रदत्तं गोशतं यथा}


\twolineshloka
{तस्मात्तु कपिला देया कौमुद्यां ज्येष्ठपुष्करे}
{न तेषां विषमं किञ्चिन्न दुःखं न च कण्टकाः}


\twolineshloka
{उपानहौ च यो दद्यात्पात्रभूते द्विजोत्तमे}
{छत्रदाने सुखां छायां लभते परलोकगः}


\twolineshloka
{न हि दत्तस्य दानस्य नाशोऽस्तीह कदाचन}
{चित्रगुप्तमतं श्रुत्वा हृष्टरोमा विभावसुः}


\twolineshloka
{उवाच देवताः सर्वाः पितॄंश्चैव महाद्युतिः}
{श्रुतं हि चित्रगुप्तस्य धर्मगुह्यं महात्मनः}


\twolineshloka
{श्रद्दधानाश्च ये मर्त्या ब्राह्मणेषु महात्मसु}
{दानमेतत्प्रयच्छन्ति न तेषां विद्यते भयम्}


\twolineshloka
{धर्मदोषास्त्विमे पञ्च येषां नास्तीह निष्कृतिः}
{असंभाष्या अनाचारा वर्जनीया नराधमाः}


\twolineshloka
{ब्रह्महा चैव गोघ्नश्च परदाररतश्च यः}
{अश्रद्दधानश्च नरः स्त्रियं यश्चोपजीवति}


\twolineshloka
{प्रेतलोकगता ह्येते नरके पापकर्मिणः}
{पच्यन्ते वै यथा मीनाः पूयशोणितभोजनाः}


\twolineshloka
{असम्भाष्याः पितॄणां च देवानां चैव पञ्च ते}
{स्नातकानां च विप्राणां ये चान्ये च तपोधनाः}


\chapter{अध्यायः १९४}
\twolineshloka
{ततः सर्वे महाभागा देवाश्च पितरश्च ह}
{ऋषयश्च महाभागाः प्रमथान्वाक्यमब्रुवन्}


\twolineshloka
{भवन्तो वै महाभागा अपरोक्षनिशाचराः}
{उच्छिष्टानशुचीन्क्षुद्रान्कथं हिसथ मानवान्}


\fourlineindentedshloka
{के च स्मृताः प्रतीघाता येन मर्त्यान्न हिंसथ}
{रक्षोघ्नानि च कानि स्युर्यैर्गृहेषु प्रणश्यथ}
{श्रोतुमिच्छाम युष्माकं सर्वमेतन्निशाचराः ॥प्रमथा ऊचुः}
{}


\twolineshloka
{मैथुनेन सदोच्छिष्टाः कृते चैवाधरोत्तरे}
{मोहान्मांसानि खादेत वृक्षमूले च यः स्वपेत्}


\twolineshloka
{आमिषं शीर्षतो यस्य पादतो यश्च संविशेत्}
{तत उच्छिष्टकाः सर्वे बहुच्छिद्राश्च मानवाः}


\twolineshloka
{उदके चाप्यमेध्यानि श्लेष्माणं च प्रमुञ्चति}
{एते भक्ष्याश्च वध्याश्च मानुषा नात्र संशयः}


\threelineshloka
{एवं शीलसमाचारान्धर्षयामो हि मानवान्}
{श्रूयतां च प्रतीघातान्यैर्न शक्नुम हिंसितुम्}
{}


\twolineshloka
{गोरोचनासमालम्भो वचाहस्तश्च यो भवेत्}
{घृताक्षतं च यो दद्यान्मस्तके तत्परायणः}


\twolineshloka
{ये च मांसं न खादन्ति तान्न शक्नुम हिंसितुम्}
{यस्य चाग्निर्गृहे नित्यं दिवारात्रौ च दीप्यते}


\twolineshloka
{तरक्षोश्चर्मदंष्ट्राश्च तथैव गिरिकच्छपः}
{आज्यधूमो बिडालश्च च्छागः कृष्णोऽत पिङ्गलः}


\twolineshloka
{येषामेतानि तिष्ठन्ति गृहेषु गृहमेधिनाम्}
{तान्यधृष्याण्यगाराणि पिशिताशैः सुदारुणैः}


\threelineshloka
{लोकानस्मद्विधा ये च विचरन्ति यथासुखम्}
{तस्मादेतानि गेहेषु रक्षोघ्नानि विशाम्पते}
{एतद्वः कथितं सर्वं यत्र वः संशयो महान् ॥]}


\chapter{अध्यायः १९५}
\twolineshloka
{ततः पद्मप्रतीकाशः पद्मोद्भूतः पितामहः}
{उवाच वचनं देवान्वासवं च शचीपतिम्}


\twolineshloka
{अयं महाबलो नागो रसातलचरो बली}
{तेजस्वी रेणुको नाम महासत्वपराक्रमः}


\twolineshloka
{अतितेजस्विनः सर्वे महावीर्या महागजाः}
{धारयन्ति महीं कृत्स्नां सशैलवनकाननाम्}


\twolineshloka
{भवद्भिः समनुज्ञातो रेणुकस्तान्महागजान्}
{धर्मगुह्यानि सर्वाणि गत्वा पृच्छतु तत्र वै}


\threelineshloka
{पितामहवचःक श्रुत्वा ते देवा रेणुकं तदा}
{प्रेषयामासुरव्यग्रा यत्र ते धरणीधराः ॥रेणुक उवाच}
{}


\fourlineindentedshloka
{अनुज्ञातोऽस्मि देवैश्च पितृभिस्चक महाबलाः}
{धर्मगुह्यानि युष्माकं श्रोतुमिच्छामि तत्त्वतः}
{कथयध्वं महाभागा यद्वस्तत्त्वं मनीषितम् ॥दिग्गजा ऊचुः}
{}


\threelineshloka
{कार्तिके मासि चाश्लेषा बहुलस्याष्टमी शिवा}
{तेन नक्षत्रयोगेन यो ददाति गुडौदनम्}
{इमं मन्त्रं जपच्छ्राद्धे यताहारो ह्यकोपनः}


\twolineshloka
{बलदेवप्रभृतयो ये नागा बलवत्तराः}
{अनन्ता ह्यक्षया नित्यं भोगिनः सुमहाबलाः}


\twolineshloka
{तेषां कुलोद्भवा ये च महाभूता भुजङ्गमाः}
{ते मे बलिं प्रयच्छन्तु बलतेजोभिवृद्धये}


\twolineshloka
{यदा नारायणः श्रीमानुज्जहार वसुन्धराम्}
{तद्बलं तस्य देवस्य धरामुद्धरतस्तथा}


% Check verse!
एवमुक्त्वा बलिं तत्र वल्मीके तु निवेदयेत्
\twolineshloka
{गजेन्द्रकुसुमाकीर्णं नीलवस्त्रानुलेपनम्}
{निर्वपेत्तं तु वल्मीके अस्तं याते दिवाकरे}


\twolineshloka
{एवं तुष्टास्ततः सर्वे अधस्ताद्भारपीडिताः}
{श्रमं तं नावबुध्यामो धारयन्तो वसुन्धराम्}


\twolineshloka
{एवं मन्यामहे सर्वे भारार्ता निरपेक्षिणः}
{ब्राह्मणः क्षत्रियो वैश्यः शूद्रो वा यद्युपोषितः}


\twolineshloka
{एवं संवत्सरं कृत्वा दानं बहुफलं लभेत्}
{वल्मीके बलिमादाय तन्नो बहुफलं मतम्}


\twolineshloka
{ये च नागा महावीर्यास्त्रिषु लोकेषु कृत्स्नशः}
{कृतातिथ्या भवेयुस्ते शतं वर्षाणि तत्त्वतः}


\twolineshloka
{दिग्गजानां च तच्छ्रुत्वा देवताः पितरस्तथा}
{ऋषयश्च महाभागाः पूजयन्ति स्म रेणुकम् ॥]}


\chapter{अध्यायः १९६}
\twolineshloka
{सारमुद्धृत्य युष्माभिः साधुधर्म उदाहृतः}
{धर्मगुह्यमिदं मत्तः शृणुध्वं सर्व एव ह}


\twolineshloka
{येषां धर्माश्रिता बुद्धिःक श्रद्दधानाश्च ये नराः}
{तेषां स्यादुपदेष्टव्यः सरहस्यो महाफलः}


\twolineshloka
{निरुद्विग्नस्तु यो दद्यान्मासमेकं गवाह्निकम्}
{एकभक्तं तथाऽश्नीयाच्छ्रूयतां तस्य यत्फलम्}


\twolineshloka
{इमा गावो महाभागाः पवित्रं परमं स्मृताः}
{त्रीन्लोकान्धानयन्ति स्म सदेवासुरमानुपान्}


\twolineshloka
{तासु चैव महापुण्यं शुश्रूषा च महाफलम्}
{अहन्यहनि धर्मेण युज्यते वै गवाह्निकः}


\twolineshloka
{मया ह्येता ह्यनुज्ञाताः पूर्वमासन्कृते युगे}
{ततोऽहमनुनीतो वै ब्रह्मणा पद्मयोनिना}


\twolineshloka
{तस्माद्व्रजस्थानगतस्तिष्ठत्युपरि मे वृषः}
{रमेऽहं सह गोभिश्च तस्मात्पूज्याः सदैव ताः}


\twolineshloka
{महाप्रभावा वरदा वरं दद्युरुपासिताः}
{ता गावोऽस्यानुमन्यन्ते सर्वकर्मसु युत्फलम्}


% Check verse!
तस्य तत्र चतुर्भागो यो ददाति गवाह्निकम् ॥]
\chapter{अध्यायः १९७}
\twolineshloka
{ममाप्यनुमतो धर्मस्तं शृणुध्वं समाहिताः}
{नीलषण्डस्य शृङ्गाभ्यां गृहीत्वा मृत्तिकां तु यः}


\twolineshloka
{अभिषेकं त्र्यहं कुर्यात्तस्य धर्मं निबोधत}
{शोधयेदशुभं सर्वमाधिपत्यं परत्र च}


\twolineshloka
{यावच्च जायते मर्त्यस्तावच्छूरो भविष्यति}
{इदं चाप्यपरं गुह्यं सरहस्यं निबोधत}


\twolineshloka
{प्रगृह्यौदुम्बरं पात्रं पक्वान्नं मधुना सह}
{सोमस्योत्तिष्ठमानस्यि पौर्णमास्यां बलिं हरेत्}


\twolineshloka
{तस्य धर्मफलं नित्यं श्रद्दधाना निबोधत}
{साध्या रुद्रास्तथाऽऽदित्या विश्वेदेवास्तथाऽश्विनौ}


\twolineshloka
{मरुतो वसवश्चैव प्रतिगृह्णन्ति तं बलिम्}
{सोमश्च वर्धते तेन समुद्रश्च महोदधिः}


\twolineshloka
{एष धर्मो मयोद्दिष्टः सरहस्यः सुखावहः ॥विष्णुरुवाच}
{}


\twolineshloka
{धर्मगुह्यानि सर्वाणि देवतानां महात्मनाम्}
{ऋषीणां चैव गुह्यानि यः पठेदाह्निकं सदा}


\twolineshloka
{शृणुयाद्वाऽनसूयुर्यः श्रद्दधानः समाहितः}
{नास्य विघ्नः प्रभवति भयं चास्य न विद्यते}


\twolineshloka
{ये च धर्माः शुभाः पुण्याः सरहस्या उदाहृताः}
{तेषां धर्मफलं तस्य यः पठेत जितेन्द्रियः}


\twolineshloka
{नास्य पापं प्रभवति न च पापेन लिप्यते}
{पठेद्वा श्रावयेद्वाऽपि श्रुत्वा वा लभते फलम्}


\twolineshloka
{भुञ्जते पितरो देवा हव्यं कव्यमथाक्षयम्}
{श्रावयंश्चापि विप्रेन्द्रान्पर्वसु प्रयतो नरः}


\twolineshloka
{ऋषीणां देवतानां च पितॄणां चैव नित्यदा}
{भवत्यभिमतः श्रीमान्धर्मेषु प्रयतः सदा}


\threelineshloka
{कृत्वाऽपि पापकं कर्म महापातकवर्जितम्}
{रहस्यधर्मं श्रुत्वेमं सर्वपापैः प्रमुच्यते ॥भीष्म उवाच}
{}


\twolineshloka
{एतद्धर्मरहस्यं वै देवतानां नराधिप}
{व्यासोद्दिष्टं मया प्रोक्तं सर्वदेवनमस्कृतम्}


\twolineshloka
{पृथिवी रत्नसंपूर्णा ज्ञानं चेदमनुत्तमम्}
{इदमेव ततः श्राव्यमिति मन्येत धर्मवित्}


\twolineshloka
{नाश्रद्दधानाय न नास्तिकायन नष्टधर्मायि न निर्घृणाय}
{न हेतुदुष्टाय गुरुद्विषे वानानात्मभूताय निवेद्यमेतत्}


\chapter{अध्यायः १९८}
\threelineshloka
{के भोज्या ब्राह्मणस्येह के भोज्याः क्षत्रियस्य ह}
{तथा वैश्यस्य के भोज्याः के शूद्रस्य च भारत ॥भीष्म उवाच}
{}


\twolineshloka
{ब्राह्मणा ब्राह्मणस्येह भोज्या ये चैव क्षत्रियाः}
{वैश्याश्चापि तथा भोज्याः शूद्राश्च परिवर्जिताः}


\twolineshloka
{ब्राह्मणाः क्षत्रिया वैश्या भोज्या वै क्षत्रियस्य ह}
{वर्जनीयास्तु वै शूद्राः सर्वभक्षा विकर्मिणः}


\twolineshloka
{वैश्यास्तु भोज्या विप्राणां क्षत्रियाणां तथैव च}
{नित्याग्नयो विविक्ताश्च चातुर्मास्यरताश्च ये}


\twolineshloka
{शूद्राणामथ यो भुङ्क्ते स भुङ्क्ते पृथिवीमलम्}
{मलं नृणां स पिबति मलं भुङ्क्ते जनस्य च}


\twolineshloka
{शूद्राणां यस्तथा भुङ्क्ते स भुङ्क्ते पृथिवीमलम्}
{पृथिवीमलमश्नन्ति ये द्विजाः शूद्रभोजिनः}


\twolineshloka
{शूद्रस्य कर्मनिष्ठायां विकर्मस्थोपि पच्यते}
{ब्राह्मणः क्षत्रियो वैश्यो विकर्मस्थश्च पच्यते}


\twolineshloka
{स्वाध्यायनिरता विप्रास्तथा स्वस्त्ययने नृणाम्}
{रक्षणे क्षत्रियं प्राहुर्वैश्यं पुष्ट्यर्थमेव च}


\twolineshloka
{करोति कर्म यद्वैश्यस्तद्गत्वा ह्युपजीवति}
{कृषिगोरक्ष्यवाणिज्यमकुत्सा वैश्यकर्मणि}


\twolineshloka
{शूद्रकर्म तु यः कुर्यादवहाय स्वकर्म च}
{स विज्ञेयो यथा शूद्रो न च भोज्यः कदाचन}


\twolineshloka
{चिकित्सकः काण्डपृष्ठः पुराऽध्यक्षः पुरोहितः}
{सांवत्सरो वृथाध्यायी सर्वे ते शूद्रसंमिताः}


\twolineshloka
{शूद्रकर्मस्वथैतेषु यो भुङ्क्ते निरपत्रपः}
{अभोज्यभोजनं भुक्त्वा भयं प्राप्नोति दारुणम्}


\twolineshloka
{कुलं वीर्यं च तेजश्च तिर्यग्योनित्वमेव च}
{स प्रयाति यथा श्वा वै निष्क्रियो धर्मवर्जितः}


\twolineshloka
{भुङ्क्ते चिकित्सकस्यान्नं तदन्नं च पुरीषवत्}
{पुंश्चल्यन्नं च मूत्रं स्यात्कारुकान्नं च शोणितम्}


\twolineshloka
{विद्योपजीविनोऽन्नं च यो भुङ्क्ते साधुसम्मतः}
{तदप्यन्नं यथा शौद्रं तत्साधुः परिवर्जयेत्}


\twolineshloka
{वचनीयस्य यो भुङ्क्ते तमाहुः शोणितं ह्रदम्}
{पिशुनं भोजनं भुङ्क्ते ब्रह्महत्यासमं विदुः}


% Check verse!
असत्कृतमवज्ञातं न भोक्तव्यं कदाचन
\twolineshloka
{व्याधिं कुलक्षयं चैव क्षिप्रं प्राप्नोति ब्राह्मणः}
{नगरीरक्षिणो भुङ्क्ते श्वपचप्रवणो भवेत्}


\twolineshloka
{गोघ्ने च ब्राह्मणघ्ने च सुरापे गुरुतल्पगे}
{भुक्त्वाऽन्नं जायते विप्रो रक्षसां कुलवर्धनः}


\twolineshloka
{न्यासापहारिणो भुक्त्वा कृतघ्ने क्लीबवर्तिनि}
{जायते शबरावासे मध्यदेशबहिष्कृते}


\twolineshloka
{अभोज्याश्चैव भोज्याश्च मया प्रोक्ता यथाविधि}
{किमन्यदद्य कौन्तेय मत्तस्त्वं श्रोतुमिच्छसि ॥]}


\chapter{अध्यायः १९९}
\twolineshloka
{उक्तास्तु भवता भोज्यास्तथाऽभोज्याश्च सर्वशः}
{अत्र मे प्रश्नसन्देहस्तन्मे वद पितामह}


\threelineshloka
{ब्राह्मणानां विशेषेण हव्यकव्यप्रतिग्रहे}
{नानाविधेषु भोज्येषु प्रायश्चित्तानि शंस मे ॥भीष्म उवाच}
{}


\twolineshloka
{हन्त वक्ष्यामि ते राजन्ब्राह्मणानां महात्मनाम्}
{प्रतिग्रहेषु भोज्ये च मुच्यते येन पाप्मनः}


\twolineshloka
{घृतप्रतिग्रहे चैव सावित्री समिदाहुतिः}
{तिलप्रतिग्रहे चैव सममेतद्युधिष्ठिर}


\twolineshloka
{मांसप्रतिग्रहे चैव मधुनो लवणस्य च}
{आदित्योदयनं स्थित्वा पूतो भवति ब्राह्मणः}


\twolineshloka
{काञ्चनं प्रतिगृह्याथ जपमानो गुरुश्रुतिम्}
{कृष्णायसं च विवृतं धारयन्मुच्यते द्विजः}


\threelineshloka
{एवं प्रतिगृहीतेऽथ धने वस्त्रे तथा स्त्रियाम्}
{एवमेव नरश्रेष्ठ सुवर्णस्य प्रतिग्रहे}
{}


\twolineshloka
{अन्नप्रतिग्रहे चैव पायसेक्षुरसे तथा}
{इक्षुतैलपवित्राणां त्रिसन्ध्येऽप्सु निमज्जनम्}


\twolineshloka
{व्रीहौ पुष्पे फले चैव जले पिष्टमये तथा}
{यावके दधिदुग्धे च सावित्रीं शतशोऽन्विताम्}


\twolineshloka
{उपानहौ च च्छत्रं च प्रतिगृह्यौर्ध्वदेहिके}
{जपेच्छतं समायुक्तस्तेन मुच्यते पाप्मना}


\twolineshloka
{क्षेत्रप्रतिग्रहे चैव ग्रहसूतकयोस्तथा}
{त्रीणि रात्राण्युपोषित्वा तेन पापाद्विमुच्यते}


\twolineshloka
{कृष्णपक्षे तु यः श्राद्धं पितॄणामश्नुते द्विजः}
{अन्नमेतदहोरात्रात्पूतो भवति ब्राह्मणः}


\twolineshloka
{न च सन्ध्यामुपासीत न च जाप्यं प्रवर्तयेत्}
{न सङ्किरेत्तदन्नं च ततः पूयेत ब्राह्मणः}


\twolineshloka
{इत्यर्थमपराङ्णे तु पितॄणां श्राद्धमुच्यते}
{यथोक्तानां यदश्नीयुर्ब्राह्मणाः पूर्वकेतिताः}


\twolineshloka
{मृतकस्य तृतीयाहे ब्राह्मणो योऽन्नमश्नुते}
{स त्रिवेलं समुन्मज्ज्य द्वादशाहेन शुध्यति}


\twolineshloka
{द्वादशाहे व्यतीते तु कृतशौचो विशेषतः}
{ब्राह्मणेभ्यो हविर्दत्त्वा मुच्यते तेन पाप्मना}


\twolineshloka
{मृतस्य दशरात्रेण प्रायश्चित्तानि दापयेत्}
{सावित्रीं रैवतीमिष्टिं कूश्माण्डमघमर्षणम्}


\twolineshloka
{मृतकस्य त्रिरात्रे यः समुद्दिष्टे समश्नुते}
{सप्तत्रिषवणं स्नात्वा पूतो भवति ब्राह्मणः}


% Check verse!
सिद्धिमाप्नोति विपुलामापदं चैव नाप्नुयात्
\twolineshloka
{यस्तु शूद्रैः समश्नीयाद्ब्राह्मणोऽप्येकभोजने}
{अशौचं विधिवत्तस्य शौचमत्र विधीयते}


\twolineshloka
{यस्तु वैश्यैः सहाश्नीयाद्ब्राह्मणोऽप्येकभोजने}
{स वै त्रिरात्रं दीक्षित्वा मुच्यते तेन कर्मणा}


\twolineshloka
{क्षत्रियैः सह योऽश्नीयाद्ब्राह्मणोऽप्येकभोजने}
{आप्लुतः सह वासोभिस्तेन मुच्येत पाप्मना}


\twolineshloka
{शूद्रास्य तु कुलं हन्ति वैश्यस्य पशुबान्धवान्}
{क्षत्रियस्य श्रियं हन्ति ब्राह्मणस्य सुवर्चसम्}


\threelineshloka
{प्रायश्चित्तं च शान्तिं च जुहुयात्तेन मुच्यते}
{सावित्रीं रैवतीमिष्टिं कूश्माण्डमघमर्षणम्}
{}


\twolineshloka
{तथोच्छिष्टमथान्योन्यं सम्प्राशेन्नात्र संशयः}
{रोचना विरजा रात्रिर्मङ्गलालम्भनानि च}


\chapter{अध्यायः २००}
\fourlineindentedshloka
{दानेन वर्ततेत्याह तपसा चैव भारत}
{तदेतन्मे मनोदुःशं व्यपोह त्वं पितामह}
{किंस्वित्पृथिव्यां ह्येतन्मे भवाञ्शंसितुमर्हति ॥भीष्म उवाच}
{}


\twolineshloka
{शृणु यैर्धर्मनिरतैस्तपसा भावितात्मभिः}
{लोका ह्यसंशयं प्राप्ता दानपुण्यरतैर्नृपैः}


\twolineshloka
{सत्कृतश्च तथाऽऽत्रेयः शिष्येभ्यो ब्रह्म निर्गुणम्}
{उपदिश्य तदा राजन्गतो लोकाननुत्तमान्}


\twolineshloka
{शिबिरोशीनरः प्राणान्प्रियस्य तनयस्य च}
{ब्राह्मणार्थमुपाकृत्य नाकपृष्ठमितो गतः}


\twolineshloka
{प्रतर्दनः काशिपतिः प्रदाय तनयं स्वकम्}
{ब्राह्मणायातुलां कीर्तिमिह चामुत्र चाश्नुते}


\twolineshloka
{रन्तिदेवश्च सांकृत्यो वसिष्ठाय महात्मने}
{अर्घ्यं प्रदाय विधिवल्लेभे लोकाननुत्तमान्}


\twolineshloka
{दिव्यं शतशलाकं च यज्ञार्थं काञ्चनं शुभम्}
{छत्रं देवावृधो दत्त्वा ब्राह्मणायास्थितो दिवम्}


\twolineshloka
{भगवानम्बरीषश्च ब्राह्मणायामितौजसे}
{प्रदाय सकलं राष्ट्रं सुरोलकमवाप्तवान्}


\twolineshloka
{सावित्रः कुण्डलं दिव्यं यानं च जनमेजयः}
{ब्राह्मणाय च गा दत्त्वा गतो लोकाननुत्तमान्}


\twolineshloka
{वृषादर्भिश्च राजर्षी रत्नानि विविधानि च}
{रम्यांश्चावसथान्दत्त्वा द्विजेभ्यो दिवमागतः}


\twolineshloka
{निमी राष्ट्रं च वैदर्भिः कन्यां दत्त्वा महात्मने}
{अगस्त्याय गतः स्वर्गं सपुत्रपशुबान्धवः}


\twolineshloka
{जामदग्न्यश्च विप्राय भूमिं दत्त्वा महायशाः}
{रामोऽक्षयांस्तथा लोकाञ्जगाम मनसोऽधिकान्}


\twolineshloka
{अवर्षति च पर्जन्ये सर्वभूतानि देवराट्}
{वसिष्ठो जीवयामास येन यातोऽक्षयां गतिम्}


\twolineshloka
{रामो दाशरथिश्चैव हुत्वा यज्ञेषु वै वसु}
{सगतो ह्यक्षयाँल्लोकान्यस्य लोके महद्यशः}


\twolineshloka
{कक्षसेनश्च राजर्षिर्वसिष्ठाय महात्मने}
{न्यासं यथावत्संन्यस्य जगामि सुमहायशः}


\twolineshloka
{करंधमस्य पौत्रस्तु मरुत्तोऽविक्षितः सुतः}
{कन्यामाङ्गिरसे दत्त्वा दिवामाशु जगाम सः}


\twolineshloka
{ब्रह्मदत्तश्च पाञ्चाल्यो राजा धर्मभृतांवरः}
{निधिं शङ्खमनुज्ञाप्य जगाम परमां गतिम्}


\twolineshloka
{राजा मित्रसहश्चैव वसिष्ठाय महात्मने}
{मदयन्तीं प्रियां भार्यां दत्त्वा च त्रिदिवं गतः}


\twolineshloka
{मनोः पुत्रश्च सुद्युम्नो लिखिताय महात्मने}
{दण्डमुद्धृत्य धर्मेण गतो लोकाननुत्तमान्}


\twolineshloka
{सहस्रचित्यो राजर्षिः प्राणानिष्टान्महायशाः}
{ब्राह्मणार्थे परित्यज्य गतो लोकाननुत्तमान्}


\twolineshloka
{सर्वकामैश्च सम्पूर्णं दत्त्वा वेश्म हिरण्मयम्}
{मौद्गल्याय गतः स्वर्गं शतद्युम्नो महीपतिः}


\twolineshloka
{भक्ष्यभोज्यस्य च कृतान्राशयः पर्वतोपमान्}
{शाण्डिल्याय पुरा दत्त्वा सुमन्युर्दिवमास्थितः}


\twolineshloka
{नाम्ना च द्युतिमान्नाम साल्वराजो महाद्युतिः}
{दत्त्वा राज्यमृचीकाय गतो लोकाननुत्तमान्}


\twolineshloka
{मदिराश्वश्च राजर्षिर्दत्त्वा कन्यां सुमध्यमाम्}
{हिरण्यहस्ताय गतो लोकान्देवैरधिष्ठितान्}


\twolineshloka
{लोमपादश्च राजर्षिः शान्तां दत्त्वा सुतां प्रभुः}
{ऋश्यशृङ्गाय विपुलैः सर्वैः कामैरयुज्यत}


\twolineshloka
{कौत्साय दत्त्वा कन्यां तु हंसीं नाम यशस्विनीम्}
{गतोऽक्षयानतो लोकान्राजर्षिश्च भगीरथः}


\twolineshloka
{दत्त्वा शतसहस्रं तु गवां राजा भगीरथः}
{स वत्सानां कोहलाय गतो लोकाननुत्तमान्}


\twolineshloka
{एते चान्ये च बहवो दानेन तपसा च ह}
{युधिष्ठिर गताः स्वर्गं निवर्तन्ते पुनःपुनः}


\twolineshloka
{तेषां प्रतिष्ठिता कीर्तिर्यावत्स्थास्यति मेदिनी}
{गृहस्थैर्दानतपसा यैर्लोका वै विनिर्जिताः}


\twolineshloka
{शिष्टानां चरितं ह्येतत्कीर्तितं मे युधिष्ठिर}
{दानयज्ञप्रजासर्गैरेते हि दिवमास्थिताः}


\twolineshloka
{दत्त्वा तु सततं तेऽस्तु कौरवाणां धुरन्धर}
{दानयज्ञक्रियायुक्ता बुद्धिर्धर्मोपचायिनी}


\twolineshloka
{यत्र ते नृपशार्दूल सन्देहो वै भविष्यति}
{श्वः प्रभाते हि वक्ष्यामि सन्ध्या हि समुपस्थिता}


\chapter{अध्यायः २०१}
\twolineshloka
{श्रुतं मे भवतस्तात सत्यव्रतपराक्रम}
{दानधर्मेण महता ये प्राप्तास्त्रिदिवं नृपाः}


\twolineshloka
{इमांस्तु श्रोतुमिच्छामि ध्रमान्धर्मभृतांवर}
{दानं कतिविधं देयं किं तस्य च फलं लभेत्}


\threelineshloka
{कथं केभ्यस्च धर्म्यं च दानं दातव्यमिष्यते}
{कैः कारणैः कतिविधं श्रोतुमिच्छामि तत्त्वतः ॥भीष्म उवाच}
{}


\twolineshloka
{शृणु तत्त्वेन कौन्तेय दानं प्रति ममानघ}
{यथा दानं प्रदातव्यं सर्ववर्णेषु भारत}


\twolineshloka
{धर्मादर्थाद्भयात्कामात्कारुण्यादिति भारत}
{दानं पञ्चविधं ज्ञेयं कारणैर्यैर्निबोध तत्}


\twolineshloka
{इह कीर्तिमवाप्नोति प्रेत्य चानुत्तमं सुखम्}
{इति दानं प्रदातव्यं ब्राह्मणेभ्योऽनसूयता}


\twolineshloka
{ददाति वा दास्यति वा मह्यं दत्तमनेन वा}
{इत्यर्तिभ्यो निशम्यैव सर्वं दातव्यमर्थिने}


\twolineshloka
{नास्याहं न मदीयोऽयं पापं कुर्याद्विमानितः}
{इति दद्याद्भयादेव दृढं मूढाय पण्डितः}


\twolineshloka
{प्रियो मे यं प्रियोऽस्याहमिति सम्प्रेक्ष्य बुद्धिमान्}
{वयस्यायैवमक्लिष्टं दानं दद्यादतन्द्रितः}


\twolineshloka
{दीनश्च याचते चायमल्पेनापि हि तुष्यति}
{इति दद्याद्दरिद्राय कारुण्यादिति सर्वथा}


\twolineshloka
{इति पञ्चविधं दानं पुण्यकीर्तिविवर्धनम्}
{यथाशक्त्या प्रदातव्यमेवमाह प्रजापतिः ॥]}


\chapter{अध्यायः २०२}
\threelineshloka
{अनुशास्य शुभैर्वाक्यैर्भीष्ममाह महामतिः}
{प्रीत्या पुनः स शुश्रूषुर्वचनं यद्युधिष्ठिरः ॥जनमेजय उवाच}
{}


\twolineshloka
{पितामहो मे विप्रर्षे भीष्मं कालवशं गतम्}
{किमपृच्छत्तदा राजा सर्वसामाजिकं हितम्}


\threelineshloka
{उभयोर्लोकयोर्युक्तं पुरुषार्थमनुत्तमम्}
{तन्मे वद महाप्राज्ञ श्रोतुं कौतूहलं हि मे ॥वैशम्पायन उवाच}
{}


\twolineshloka
{भूय एव महाराज शृणु धर्मसमुच्चयम्}
{यदपृच्छत्तदा राजा कुन्तीपुत्रो युधिष्ठिरः}


\twolineshloka
{शरतल्पगतं भीष्मं सर्वपार्थिवसन्निधौ}
{अजातशत्रुः प्रीतात्मा पुनरेवाभ्यभाषत}


\twolineshloka
{पितामह महाप्राज्ञ सर्वशास्त्रविशारदः}
{श्रूयतां मे हि वचनमर्थित्वात्प्रब्रवीम्यहम्}


\twolineshloka
{परावरज्ञो भूतानां दयावान्सर्वजन्तुषु}
{आगमैर्बहुभिः प्रीतो भवान्नः परमं कुले}


\twolineshloka
{त्वादृशो दुर्लभो लोके साम्प्रतं ज्ञानसंयुतः}
{भवता गुरुणा चैव धन्याश्चैव वयं प्रभो}


\twolineshloka
{अयं स कालः सम्प्राप्तो दुर्लभैर्ज्ञातिबान्धवैः}
{शास्ता तु नास्ति नः कश्चित्त्वदृते पुरुषर्षभ}


\twolineshloka
{तस्माद्धर्मार्थसहितमायत्यां च हितोदयम्}
{आश्चर्यं परमं वाक्यं श्रोतुमिच्छामि भारत}


\twolineshloka
{अयं नारायणः श्रीमन्सर्वपार्थिवसन्निधौ}
{भवन्तं बहुमानाच्च प्रणयाच्चैव सेवते}


\twolineshloka
{अस्यैव तु समक्षं नः पार्तिवानां तथैव च}
{इतिवृत्तं पुराणं च श्रोतॄणां परमं हितम्}


\threelineshloka
{यदि तेऽहमनुग्राह्यो भ्रातृभिः सहितोऽनघ}
{मत्प्रियार्थं हि कौरव्य स्नेहाद्भाषितुमर्हसि ॥वैशम्पायन उवाच}
{}


\twolineshloka
{तस्य तद्वचनं श्रुत्वा स्नेहादागतविक्लवः}
{प्रविबन्निव तं दृष्ट्वा भीष्मो वचनमब्रवीत्}


\threelineshloka
{शृणु राजन्पुरावृत्तिमितिहासं पुरातनम्}
{एतावदुक्त्वा गाङ्गेयः प्रणम्य शिरसा हरिम्}
{धर्मराजं समीक्ष्येदं पुनर्वक्तुं समारभत्}


\chapter{अध्यायः २०३}
\twolineshloka
{अयं नारायणः श्रीमान्पुत्रार्थे व्रतकाङ्क्षया}
{दीक्षितोऽभून्महाबाहुः पुरा द्वादशवार्षिकम्}


\twolineshloka
{दीक्षितं केशवं द्रष्टुमभिजग्मुर्महर्षयः}
{सेवित्वा च महात्मानः प्रीयमाणं जनार्दनम्}


\threelineshloka
{नारदः पर्वतश्चैव कृष्णद्वैपायनस्तथा}
{देवलः काश्यपश्चैव हस्तिकाश्यप एव च}
{जमदग्निश्च राजेन्द्र धौम्यो वाल्मीकिरेव च}


\twolineshloka
{अपरेऽपि तपःसिद्धाः सत्यव्रतपरायणाः}
{शिष्यैरनुगताः सर्वे ब्रह्मविद्भिरकल्मषैः}


\threelineshloka
{केशवस्तानभिगतान्प्रीत्या सम्परिगृह्य च}
{तेषामतिथिसत्कारं पूजनार्थं कुलोचितम्}
{देवकीतनयो हृष्टो देवतुल्यमकल्पयत्}


\twolineshloka
{उपविष्टेषु सर्वेषु विष्टरेषु तदाऽनघ}
{विश्वस्तेष्व्नभितुष्टेषु केशवार्चनया पुनः}


\twolineshloka
{परस्परं कथा दिव्याः प्रावर्तन्त मनोरमाः}
{विष्णोर्नारायणस्यैव प्रसादात्कथयामि ताः}


\twolineshloka
{तस्यैव व्रतचर्यायां मुनिभिर्विस्मितं पुरा}
{यत्र गोवृषभाङ्कस्य प्रभावोऽभून्महात्मनः}


\twolineshloka
{यत्र देवी महादेवमपृच्छत्संशयान्पुरा}
{कथयामास सर्वांस्तान्देव्याः प्रियचिकीर्षया}


\twolineshloka
{उमापत्योश्च संवादं शृणु तात मनोरमम्}
{वर्णाश्रमाणां धर्मश्च तत्र तात समाहितः}


\twolineshloka
{ऋषिधर्मश्च निखिलो राजधर्मस्च पुष्कलः}
{गृहस्थधर्मश्च शुभः कर्मपाकफलानि च}


\twolineshloka
{देवगुह्यं च विविधं दानधर्मविधिस्तथा}
{विधानमत्र प्रोक्तं यद्यमस्य नियमस्य च}


\twolineshloka
{यमलोकविधानं च स्वर्गलोकगतिस्तथा}
{प्राणमोक्षविधिश्चैव तीर्थचर्या च पुष्कला}


\twolineshloka
{मोक्षधर्मविधानं च साङ्ख्ययोगसमन्वितम्}
{स्त्रीधर्मश्च स्वयं देव्या देवदेवाय भाषितः}


\twolineshloka
{एवमादि शुभं सर्वं तत्र तात समाहितम्}
{रुद्राण्याः संशयप्रश्नो यत्र तात प्रवर्तते}


\twolineshloka
{धन्यं यशस्यमायुष्यं धर्म्यं च परमं हितम्}
{पुष्टियोगमिमं दिव्यं कथ्यमानं मया शृणु}


\threelineshloka
{इतिहासमिमं दिव्यं पवित्रं परमं शुभम्}
{सायं प्रातः सदा सम्यक् श्रोतव्यं च बुभूषता ॥भीष्म उवाच}
{}


\twolineshloka
{ततो नारायणो देवः संक्लिष्टो व्रतचर्यया}
{वह्निर्विनिःसृतो वक्त्रात्कृष्णस्याद्भुतदर्शनः}


\twolineshloka
{अग्निना तेन महता निःसृतेन मुखाद्विभोः}
{पश्यतामेव सर्वेषां दग्ध एव नगोत्तमः}


\twolineshloka
{मृगपक्षिगणाकीर्णः श्वापदैरपि सङ्कुलः}
{वृक्षगुल्मलताकीर्णो मथितो दीनदर्शनः}


\twolineshloka
{पुनः स दृष्टमात्रेण हरिणा सौम्यचेतसा}
{स बभूव गिरिः क्षिप्रं प्रफुल्लद्रुमकाननः}


\threelineshloka
{सिद्धचारणसङ्घैश्च प्रसन्नैरुपशोभितः}
{मत्तवारणसंयुक्तो नानापक्षिगणैर्युतः}
{तदद्भुतमचिन्त्यं च सर्वेषामभवद्भृशम्}


\threelineshloka
{तं दृष्ट्वा हृष्टरोमाणः सर्वे मुनिगणास्तदा}
{विस्तिताः परमायत्ताः साध्यसाकुललोचनाः}
{न किञ्चिदब्रुवंस्तत्र शुभं वा यदि वेतरत्}


\twolineshloka
{ततो नारायणो देवो मुनिसङ्घे तु विस्मिते}
{तान्समीक्ष्यैव मधुरं बभाषे पुष्करेक्षणः}


\threelineshloka
{किमर्थं मुनिसङ्घेऽस्मिन्विस्मयोऽयमनुत्तमः}
{एतन्मे संशयं सर्वे याथातथ्येन नन्दिताः}
{ऋषयो वक्तुमर्हन्ति निश्चयेनार्थकोविदाः}


\threelineshloka
{केशवस्य वचः श्रुत्वा तुष्टुवुर्मुनिपुङ्गवाः}
{भवान्सृजति वै लोकान्भवान्संहरति प्रजाः}
{भवाञ्शीतं भवानुष्णं भवान्सत्यं भवान्क्रतुः}


\twolineshloka
{भवानादिर्भवानन्तो भवतोऽन्यन्न विद्यते}
{स्थावरं जङ्गमं सर्वं त्वमेव पुरुषोत्तम}


\twolineshloka
{त्वत्तः सर्वमिदं तात लोकचक्रं प्रवर्तते}
{त्वमेवार्हसि तद्वक्तुं मुखादग्निविनिर्गमम्}


\fourlineindentedshloka
{एतन्नो विस्मयकरं बभूव मधुसूदन}
{ततो विगतसंत्रासा भवाम पुरुषोत्तम}
{यदिच्छेत्तत्र वक्तव्यं कुतोऽस्माकं नियोगतः ॥श्रीभगवानुवाच}
{}


\twolineshloka
{नित्यं हितार्थं लोकानां भवद्भिः क्रियते तपः}
{तस्माल्लोकहितं गुह्यं श्रूयतां कथयामि वः}


\twolineshloka
{असुरः साम्प्रतं कश्चिदहितो लोकनाशनः}
{मायास्त्रकुशलश्चैव बलदर्पसमन्वितः}


\twolineshloka
{बभूव स मया बद्धो लोकानां हितकाम्यया}
{पुत्रेण मे वधो दृष्टस्तस्य वै मुनिपुङ्गवाः}


\twolineshloka
{तदर्थं पुत्रमेवाहं सिसृक्षुर्वनमागतः}
{आत्मनः सदृशं पुत्रमहं जनयितुं व्रतैः}


\twolineshloka
{एवं व्रतपरीतस्य तपस्तीव्रतया मम}
{अथात्मा मम देहस्थः सोग्निर्भूत्वा विनिःसृतः}


\threelineshloka
{विनिःसृत्य गतो द्रष्टुं क्षणेन च पितामहम्}
{ब्रह्मणा मन्मथोऽनङ्गः पुत्रत्वेन प्रकल्पितः}
{अनुज्ञातश्च तेनैव पुनरायान्ममान्तिकम्}


\twolineshloka
{एवं मे वैष्णवं तेजो मम वक्त्राद्विनिःसृतम्}
{तत्तेजसा निर्मथितः पुरतोऽयं गिरिः स्थितः}


\twolineshloka
{दृष्ट्वा दाहं गिरेस्तस्य सौम्यभावतया मम}
{पुनः स दृष्टमात्रेण गिरिरासीद्यथा पुरा}


\twolineshloka
{एतद्गुह्यं यथातथ्यं कथितं वः समासतः}
{भवन्तो व्यथिता येन विस्मिताश्च तपोधनाः}


% Check verse!
ऋषीणामेवमुक्त्वा तु तान्पुनः प्रत्यभाषत
\chapter{अध्यायः २०४}
\twolineshloka
{भवतां दर्शनादेव प्रीतिरभ्यधिका मम}
{भवन्तस्तु तपःसिद्धा भवन्तो दिव्यदर्शनाः}


\twolineshloka
{सर्वत्र गतिमन्तश्च ज्ञानविज्ञानभाविताः}
{गत्यागतिज्ञा लोकानां सर्वे निर्दूतकल्मषाः}


\twolineshloka
{तस्माद्भवद्भिर्यत्किंचिदृष्टं वाऽप्यथवा श्रुतम्}
{आश्चर्यभूतं लोकेषु तद्भवन्तो ब्रुवन्तु मे}


% Check verse!
युष्माभिः कथितं यत्स्यात्तपसा भावितात्मभिःतस्यादमृतसंकाशं वाङ्मधुश्रवणे स्पृहा
\twolineshloka
{रागद्वेषवियुक्तानां सततं सत्यवादिनाम्}
{श्रद्धेयं श्रवणीयं च वचनं हि सतां भवेत्}


\twolineshloka
{तत्संयोगं हितं मेऽस्तु न वृथा कर्तुमर्हथ}
{भवतां दर्शनं तस्मात्सफलं तु भवेन्मम}


\twolineshloka
{तदहं सज्जनमुखान्निःसृतं जनसंसदि}
{कथयिष्याम्यहं बुद्ध्या बुद्धिदीपकरं नृणाम्}


\twolineshloka
{तदन्ये वर्धयिष्यन्ति पूजयिष्यन्ति चापरे}
{वात्सल्यविगताश्चान्ये प्रशंसन्ति पुरातनाः}


\twolineshloka
{एवं ब्रुवति गोविन्दे श्रवणार्थं महर्षयः}
{वाग्भिः साञ्जलिमालाभिरिदमूचुर्जनार्दनम्}


\twolineshloka
{अयुक्तमस्मानेवं त्वं वाचा वरद भाषितुम्}
{त्वच्छासनमुखाः सर्वे त्वदधीनपरिश्रमाः}


\threelineshloka
{एवं पूजयितुं चास्मान्न चैवार्हसि केशव}
{त्वत्तस्त्वन्यं न पश्यामो यल्लोके ते न विद्यते}
{दिवि वा भुवि वा किञ्चित्तत्सर्वं हि त्वया ततम्}


\twolineshloka
{न विद्महे वयं देव कथ्यमानं तवान्तिके}
{एवमुक्तो हृषीकेशः सस्मितं चेदमब्रवीत्}


\fourlineindentedshloka
{अहं मानुषयोनिस्थः साम्प्रतं मुनिपुङ्गवाः}
{तस्मान्मानुषवद्वीर्यं मम जानीत सुव्रताः}
{भवद्भिः कथ्यमानं च अपूर्वमिव तद्भवेत् ॥भीष्म उवाच}
{}


\twolineshloka
{एवं संचोदिताः सर्वे केशवेन महात्मना}
{ऋषयश्चानुवर्तन्ते वासुदेवस्य शासनम्}


\twolineshloka
{ततस्त्वृषिगणाः सर्वे नारदं देवदर्शनम्}
{अमन्यन्त बुधा बुद्ध्या समर्थं तन्निबोधने}


\twolineshloka
{ऋषिरुग्रतपाश्चायं केशवस्य प्रियोऽधिकम्}
{पुराणज्ञश्च वाग्मी च कारणैस्तं च मेनिरे}


% Check verse!
सर्वे तदर्हणं कृत्वा नारदं वाक्यमब्रुवन्
\twolineshloka
{भवता तीर्थयात्रार्थं चरता हिमवद्गिरौ}
{दृष्टं वै यत्तदाश्चर्यं श्रोतॄणां परमं प्रियम्}


\twolineshloka
{अतस्त्वमविशेषेण हितार्थं सर्वमादितः}
{प्रियार्थं केशवस्यास्य स भवान्वक्तुमर्हति}


\threelineshloka
{तदा संयोजितः सर्वैर्ऋषिभिर्नारदस्तदा}
{प्रणम्य शिरसा विष्णुं सर्वलोकहिते रतम्}
{समुद्वीक्ष्य हृषीकेशं वक्तुमेवोपचक्रमे}


\twolineshloka
{ततो नारायणसुहृन्नारदो वदतांवरः}
{शङ्करस्योमया सार्धं संवादमनुभाषत}


\chapter{अध्यायः २०५}
\twolineshloka
{भगवंस्तीर्थयात्रार्थं तथैव चरता मया}
{दिव्यमद्भुतसङ्काशं दृष्टं हैमवतं वनम्}


\twolineshloka
{नानावृक्षसमायुक्तं नानापक्षिगणैर्वृतम्}
{नानरत्नगणाकीर्णं नानाभावसमन्वितम्}


\twolineshloka
{दिव्यचन्दनसंयुक्तं दिव्यधूपेन धूपितम्}
{दिव्यपुष्पसमाकीर्णं दिव्यगन्धेन वासितम्}


\twolineshloka
{सिद्धचारणसंघातैर्भूतसङ्घैर्निषेवितम्}
{वरिष्ठाप्सरसाकीर्णं नागगन्धर्वसङ्कुलम्}


\twolineshloka
{मृदङ्गमुरजोद्धुष्टं शङ्खवीणाभिनादितम्}
{नृत्यद्भिर्भूतसङ्घैश्च सर्वतस्त्वभिशोभितम्}


\twolineshloka
{नानारूपैर्विरूपैश्च भीमरूपैर्भयानकैः}
{सिंहव्याघ्रोरगमृगैर्बिडालवदनैस्तथा}


\twolineshloka
{स्वरोष्ट्रद्वीपिवदनैर्गजवक्त्रैस्तथैव च}
{उलूकश्येनवदनैः काकगृध्रमुखैस्तथा}


\twolineshloka
{एवं बहुविधाकारैर्भूतसङ्घैर्भृशाकुलम्}
{नानद्यमानं बहुधा हरपारिषदैर्भृशम्}


\twolineshloka
{घोररूपं सुदुर्दर्शं रक्षोगणशतैर्वृतम्}
{समाजं तद्वने दृष्टं मया भूतपतेः पुरा}


\twolineshloka
{प्रनृत्ताप्सरसं दिव्यं देवगन्धर्वनादितम्}
{प्रथमे वर्षरात्रे तु मेघसङ्घनिनादितम्}


\twolineshloka
{नानाबर्हिणसंघुष्टं गजयूथसमाकुलम्}
{षट्पदैरुपगीतं च प्रथमे मासि माधवे}


\twolineshloka
{उत्क्रोशत्क्रौञ्चकुररैः सारसैर्जीवजीवकैः}
{मत्ताभिः परपुष्टाभिः कूजन्तीभिः समाकुलम्}


\twolineshloka
{उत्तमावाससङ्काशं भीमरूपतरं ततः}
{द्रष्टुं भवति धर्मस्य धर्मभागिजनस्य च}


\twolineshloka
{ये चोध्वरेतसः सिद्धास्तत्रतत्र समागताः}
{मार्ताण्डरश्मिसञ्चारा विश्वेदेवगणास्तथा}


\twolineshloka
{तथा नागास्तथा दिव्या लोकपाला हुताशनाः}
{वाताश्च सर्वे चायान्ति दिव्यपुष्पसमाकुलाः}


\twolineshloka
{किरन्तः सर्वपुष्पाणि किरन्तोऽद्भुतदर्शनाः}
{ओषध्यः प्रज्वलन्त्यश्च द्योतयन्त्यो दिशो दश}


\twolineshloka
{विहगाश्च मुदा युक्ता नृत्यन्ति च नदन्ति च}
{ततः समन्ततस्तत्र दिव्या दिव्यजनप्रियाः}


\twolineshloka
{तत्र देवो गिरितटे हेमधातुविभूषिते}
{पर्यङ्क इव बभ्राज उपविष्टो महाद्युतिः}


\twolineshloka
{व्याघ्रचर्मपरीधानो गजचर्मोत्तरच्छदः}
{व्यालयज्ञोपवीतश्च लोहितान्त्रविभूषितः}


\twolineshloka
{हरिश्मश्रुजटो भीमो भयकर्ताऽमरद्विषाम्}
{भयहेतुरभक्तानां भक्तानामभयङ्करः}


\twolineshloka
{किन्नरैर्देवगन्धर्वैः स्तूयमानस्ततस्ततः}
{ऋषिभिश्चाप्सरोभिश्च सर्वतश्चैव शोभितः}


\twolineshloka
{तत्र भूतपतेः स्थानं देवदानवसङ्कुलम्}
{सर्वतेजोमयं भूम्ना लोकपालनिषेवितम्}


\twolineshloka
{महोरगसमाकीर्णं सर्वेषां रोमहर्षणम्}
{भीमरूपमनिर्देश्यमप्रधृष्यतमं विभोः}


\threelineshloka
{तत्र भूतपतिं देवमासीनं शिखरोत्तमे}
{ऋषयो भूतसङ्घाश्च प्रणम्य शिरसा हरम्}
{गीर्भिः परमशुद्धाभिस्तुष्टुवुस्च मनोहरम्}


\twolineshloka
{विमुक्ताश्चैव पापेभ्यो बभूवुर्विगतज्वराः}
{ऋषयो बालकिल्याश्च तथा विप्रर्षयश्च ये}


\threelineshloka
{अयोनिजा योनिजाश्च तपःसिद्धा महर्षयः}
{ततस्तं देवदेवेशं भगवन्तमुपासते}
{}


\twolineshloka
{ततस्तस्मिन्क्षणे देवी भूतस्त्रीगणसंवृता}
{हरतुल्याम्बरधरा समानव्रतचारिणी}


\twolineshloka
{काञ्चनं कलशं गृह्य सर्वतीर्थाम्बुपूरितम्}
{पुष्पवृष्ट्याऽभिवर्षन्ती दिव्यगन्धसमावृता}


\twolineshloka
{सरिद्वराभिः सर्वाभिः पृष्ठतोऽनुगता वरा}
{सेवितुं भगवत्पार्श्वमाजगाम शुचिस्मिता}


\twolineshloka
{आगम्य तु गिरेः पुत्री देवदेवस्य चान्तिकम्}
{मनःप्रियं चिकीर्षन्ती क्रीडार्थं शङ्करान्तिके}


\twolineshloka
{मनोहराभ्यां पाणिभ्यां हरनेत्रे पिधाय तु}
{अवेक्ष्य हृष्टा स्वगणान्स्मयन्ती पृष्ठतः स्थिता}


\twolineshloka
{देव्या चान्धीकृते देवे कश्मलं समपद्यत}
{निमीलिते भूतपतौ नष्टचन्द्रार्कतारकम्}


\twolineshloka
{निःस्वाध्यायवषट्कारं तमसा चाभिसंवृतम्}
{विषण्णं भयवित्रस्तं जगदासीद्भयाकुलम्}


\twolineshloka
{हाहाकारमृषीणां च लोकानामभवत्तदा}
{तमोभिभूते सम्भ्रान्ते लोके जीवनसंक्षये}


\twolineshloka
{तृतीयं चास्य सम्भूतं ललाटे नेत्रमायतम्}
{द्वादशादित्यसङ्काशं लोकान्भासाऽवभासयत्}


\twolineshloka
{तत्रक तेनाग्निना तेन युगान्ताग्निनिभेन वै}
{अदह्यत गिरिः सर्वो हिमवानग्रतः स्थितः}


\twolineshloka
{दह्यमाने गिरौ तस्मिन्मृगपक्षिसमाकुले}
{सविद्याधरगन्धर्वे दिव्यौषधसमाकुले}


\twolineshloka
{ततो गिरिसुता चापि विस्मयोत्फुल्ललोचना}
{बभूव च जगत्सर्वं तथा विस्मयसंयुतम्}


\twolineshloka
{पश्यतामेव सर्वेषां देवदानवरक्षसाम्}
{नेत्रजेनाग्निना तेन दग्ध एव नगोत्तमः}


\twolineshloka
{तं दृष्ट्वा मथितं शैलं शैलपुत्री सविक्लवा}
{पितुः सम्मानमिच्छन्ती पपात भुवि पादूयोः}


\twolineshloka
{तं दृष्ट्वा देवदेवेशो देव्या दुःखमनुत्तमम्}
{हैमवत्याः प्रियार्थं च गिरिं पुनरवैक्षत}


\twolineshloka
{दृष्टमात्रे भगवता सौम्ययुक्तेन चेतसा}
{क्षणेन हिमवाञ्शैलः प्रकृतिस्थोऽभवत्पुनः}


\twolineshloka
{हृष्टपुष्टविहङ्गैश्च प्रफुल्लद्रुमकाननः}
{सिद्धचारणसङ्घैश्च प्रीतियुक्तैः समाकुलः}


\twolineshloka
{पितरं प्रकृतिस्थं च दृष्ट्वा हैमवती भृशम्}
{अभवत्प्रीतिसंयुक्ता मुदितात्र पिनाकिनम्}


\threelineshloka
{देवी विस्मयसंयुक्ता प्रष्टुकामा महेश्वरम्}
{हितार्तं सर्वलोकानां प्रजानां हितकाम्यया}
{देवदेवं महादेवी बभाषेदं वचोऽर्थवत्}


\twolineshloka
{भगवन्देवदेवेश शूलपाणे महाद्युते}
{विस्मयो मे महाञ्जातस्तस्मिन्नेत्राग्निसम्भवे}


\twolineshloka
{किमर्थं देवदेवेश ललाटेऽस्मिन्प्रकाशते}
{अतिसूर्याग्निसङ्काशं तृतीयं नेत्रमायतम्}


\twolineshloka
{नेत्राग्निना तु महता निर्दग्धो हिमवानसौ}
{पुनः संदृष्टमात्रस्तु प्रकृतिस्थः पिता मम}


\threelineshloka
{एष मे संशयो देव हृदि मे सम्प्रवर्तते}
{देवदेव नमस्तुभ्यं तन्मे संसितुमर्हसि ॥नारद उवाच}
{}


\twolineshloka
{एवमुक्तस्तया देव्या प्रीयमाणोऽब्रवीद्भवः}
{स्थाने संशयितुं देवि धर्मज्ञे प्रियभाषिणि}


\twolineshloka
{त्वदृते मां हि वै प्रष्टुं न शक्यं केन चेत्प्रिये}
{प्रकाशं यदि वा गुह्यं प्रियार्थं प्रब्रवीम्यहम्}


\twolineshloka
{शृणु तत्सर्वमकिलमस्यां संसदि भामिनि}
{सर्वेषामेव लोकानां कूटस्थं विद्धि मां प्रिये}


\twolineshloka
{मदधीनास्त्रयो लोका यथा विष्णौ तथा मयि}
{स्रष्टा विष्णुरहं गोप्ता इत्येतद्विद्धि भामिनि}


\twolineshloka
{तस्माद्यदा मां स्पृशति शुभं वा यदि वेतरत्}
{तथैवेदं जगत्सर्वं तत्तद्भवति शोभने}


\twolineshloka
{एतद्गुह्यमजानन्त्या त्वया बाल्यादनिन्दिते}
{मन्नेत्रे पिहिते देवि क्रीडनार्थं दृढव्रते}


\twolineshloka
{तत्कृते नष्टचन्द्रार्कं जगदासीद्भयाकुलम्}
{नष्टादित्ये तमोभूते लोके गिरिसुते प्रिये}


\twolineshloka
{तृतीयं लोचनं सृष्टं लोकं संरक्षितुं मया}
{कथितं संशयस्थानं निर्विशङ्का भव प्रिये}


\chapter{अध्यायः २०६}
\twolineshloka
{क्षणज्ञा देवदेवस्य श्रोतुकामा प्रियं हितम्}
{उमादेवी महादेवमपृच्छत्पुनरेव तु}


\twolineshloka
{भगवन्देवदेवेश सर्वदेवनमस्कृत}
{चतुर्मुखो वै भगवानभवत्केन हेतुना}


\threelineshloka
{भगवन्केन ते वक्त्रमैन्द्रमद्भुतदर्शनम्}
{उत्तरं चापि भगवन्पश्चिमं शुभदर्शनम्}
{दक्षिणं च मुखं रौद्रं केनोर्ध्वं जटिलावृतम्}


\threelineshloka
{यथादृशं महाभाग श्रोतुमिच्छामि कारणम्}
{एष मे संशयो देव तन्मे शंसितुमर्हसि ॥महादेव उवाच}
{}


\twolineshloka
{तदहं ते प्रवक्ष्यामि यत्त्वमिच्छसि भामिनि}
{पुराऽसुरो महाघोरौ लोकोद्वेगकरौ भृशम्}


\twolineshloka
{सुन्दोपसुन्दनामानावासतुर्बलगर्वितौ}
{अशस्त्रवध्यौ बलिनौ परस्परहितैषिणौ}


\threelineshloka
{तयोरेव विनाशाय निर्मिता विश्वकर्मणा}
{सर्वतः सारमुद्धृत्य तिलशो लोकपूजिता}
{तिलोत्तमेति विख्याता अप्सराः सा बभूव ह}


\twolineshloka
{देवकार्यं करिष्यन्ती हासभावसमन्विता}
{सा तपस्यन्तमागम्य रूपेणाप्रतिमा भुवि}


\threelineshloka
{मया बहुमता चेयं देवकार्यं करिष्यति}
{इति मत्वा तदा चाहं कुर्वन्तीं मां प्रदक्षिणम्}
{तथैव तां दिदृक्षुश्च चतुर्वक्त्रोऽभवं प्रिये}


\twolineshloka
{ऐन्द्रं मुखमिदं पूर्वं तपश्चर्यापरं सदा}
{दक्षिणं मे मुखं दिव्यं रौद्रं संहरति प्रजाः}


\fourlineindentedshloka
{लोककार्यपरं नित्यं पश्चिमं मे मुखं प्रिये}
{वेदानधीते सततमद्भुतं चोत्तरं मुखम्}
{एतत्ते सर्वमाख्यातं किं भूयः श्रोतुमिच्छसि ॥उमोवाच}
{}


\threelineshloka
{भगवञ्श्रोतुमिच्छामि शूलपाणे वरप्रद}
{किमर्थं नीलता कण्ठे भाति बर्हिनिभा तव ॥महेश्वर उवात}
{}


% Check verse!
एतत्ते कथयिष्यामि शृणु देवि समाहिता
\twolineshloka
{पुरा युगान्तरे यत्नादमृतार्थं सुरासुरैः}
{बलवद्भिर्विमथितश्चिरकालं महोदधिः}


\twolineshloka
{रज्जुना नागराजेन मथ्यमाने महोदधौ}
{विषं तत्र समुद्भूतं सर्वलोकविनाशनम्}


\twolineshloka
{तद्दृष्ट्वा विबुधाः सर्वे तदा विमनसोऽभवन्}
{ग्रस्तं हि तन्मया देवि लोकानां हितकारणात्}


\fourlineindentedshloka
{तत्कृता नीलता चासीत्कण्ठे बर्हिनिभा शुभे}
{तदाप्रभृति चैवाहं नीलकण्ठ इति स्मृतः}
{एतत्ते सर्वमाख्यातं किं भूयः श्रोतुमिच्छसि ॥उमोवाच}
{}


\fourlineindentedshloka
{नीलकण्ठ नमस्तेऽस्तु सर्वलोकसुखावह}
{बहूनामायुधानां त्वं पिनाकं धर्मुमिच्छसि}
{किमर्थं देवदेवेश तन्मे शंसितुमर्हसि ॥महेश्वर उवाच}
{}


\twolineshloka
{शस्त्रागमं ते वक्ष्यामि शृणु धर्म्यं शुचिस्मिते}
{युगान्तरे महादेवि कण्वो नाम महामुनिः}


\threelineshloka
{स हि दिव्यां तपश्चर्यां कर्तुमेवोपचक्रमे}
{तथा तस्य तपो घोरं चरतः कालपर्ययात्}
{वल्मीकं पुनरुद्भूतं तस्यैव शिरसि प्रिये}


\twolineshloka
{वेणुर्वल्मीकसंयोगान्मूर्ध्नि तस्य बभूव ह}
{धरमाणश्च तत्सर्वं तपश्चर्यां तथाऽकरोत्}


\twolineshloka
{तस्मै ब्रह्मा वरं दातुं जगाम तपसार्चितः}
{दत्त्वा तस्मै वरं देवो वेणुं दृष्ट्वा त्वचिन्तयत्}


\twolineshloka
{लोककार्यं समुद्दिश्य वेणुनाऽनेन भामिनि}
{चिन्तयित्वा तमादाय कार्मुकार्थे न्ययोजयत्}


\twolineshloka
{विष्णोर्मम च सामर्थ्यं ज्ञात्वा लोकपितामहः}
{धनुषी द्वे तदा प्रादाद्विष्णवे च ममैव च}


\twolineshloka
{पिनाकं नाम मे चापं शार्ङ्गं नाम हरेर्धनुः}
{तृतीयमवशेषेण गाण्डीवमभवद्धनुः}


\threelineshloka
{तच्च सोमाय निर्दिश्य ब्रह्मा लोकं गतः पुनः}
{एतत्ते सर्वमाख्यातं शस्त्रागममनिन्दिते ॥उमोवाच}
{}


\twolineshloka
{भगवन्देवदेवेश पिनाकपरमप्रिय}
{वाहनेषु तथाऽन्येषु सत्सु भूतपते तव}


\threelineshloka
{अयं तु वृषभः कस्माद्वाहनं स यथाऽभवत्}
{एष मे सशयो देव तन्मे शंसितुमर्हसि ॥महादेव उवाच}
{}


% Check verse!
तदहं ते प्रवक्ष्यामि वाहनं स यथाऽभवत्
\twolineshloka
{आदिसर्गे पुरा गावः श्वेतवर्णाः शुचिस्मिते}
{बलसंहनना गावो दर्पयुक्ताश्चरन्ति ताः}


\twolineshloka
{अहं तु तप आतिष्ठं तस्मिन्काले शुभानने}
{एकपादश्चोर्ध्वबाहुर्लोकार्थं हिमवद्गिरौ}


\twolineshloka
{गावो मे पार्श्वमागम्य दर्पोत्सिक्ताः समन्ततः}
{स्थानभ्रंशं तदा देवि चक्रिरे बहुशस्तदा}


\twolineshloka
{अपचारेण वै तासां मनःक्षोभोऽभवन्मम}
{तस्माद्दग्धा यदा गावो रोषाविष्टेन चेतसा}


\twolineshloka
{तस्मिंस्तु व्यसने घोरे वर्तमाने पशून्प्रति}
{अनेन वृषभेणाहं शमितः सम्प्रसादनैः}


\threelineshloka
{तदाप्रभृति शान्ताश्च वर्णभेदत्वमागताः}
{श्वेतोऽयं वृषभो देवि पूर्वसंस्कारसंयुतः}
{वाहनत्वे ध्वजत्वे मे तदाप्रभृति योजितः}


\twolineshloka
{तस्मान्मे गोपतित्वं च देवैर्गोभिश्च कल्पितम्}
{प्रसन्नश्चाभवं देवि तदा गोपतितां गतः}


\chapter{अध्यायः २०७}
\twolineshloka
{भगवन्सर्वभूतेश शूलपाणे वृषध्वज}
{आवासेषु विचित्रेषु रम्येषु च शुभेशु च}


\twolineshloka
{सत्सु चान्येषु भूतेषु श्मशाने रमसे कथम्}
{केशास्थिकलिले भीमे कपालशतसङ्कुले}


\twolineshloka
{सृगालगृध्रम्पूर्णे शवधूमसमाकुले}
{चिताग्निविषमे घोरे गहने च भयानके}


\threelineshloka
{एवं कलेवरक्षेत्रे दुर्दर्शे रमसे कथम्}
{एष मे संशयो देव तन्मे शंसितुमर्हसि ॥महादेव उवाच}
{}


\twolineshloka
{हन्त ते कथयिष्यामि शृणु देवि समाहिता}
{आवासार्थं पुरा देवि शुद्धान्वेषी शुचिस्मिते}


\twolineshloka
{नाध्यगच्छं चिरं कालं देशं शुचितमं शुभे}
{एष मेऽभिनिवेशोऽभूत्तस्मिन्काले प्रजापतिः}


\twolineshloka
{आकुलः सुमहाघोरः प्रादुरासीत्समन्ततः}
{सम्भूता भूतसृष्टिश्च घोरा लोकभयावहा}


\threelineshloka
{नानावर्णा विरूपाश्च तीक्ष्णदंष्ट्राः प्रहारिणः}
{पिशाचरक्षोवदनाः प्राणिनां प्राणहारिणः}
{इतश्चरन्ति निघ्नन्तः प्राणिनो भृशमेव च}


\twolineshloka
{एवं लोके प्राणिहीने क्षयं याते पितामहः}
{चिन्तयंस्तत्प्रतीकारं मां च शक्तं हि निग्रहे}


\twolineshloka
{एवं ज्ञात्वा ततो ब्रह्मा तस्मिन्कर्मण्योजयत्}
{तच्च प्राणिहितार्थं तु मयाऽप्यनुमतं प्रिये}


\threelineshloka
{तस्मात्संरक्षिता देवि भूतेभ्यः प्राणिनो भयात्}
{अस्माच्छ्मशानान्मेध्यं तु नास्ति किञ्चिदनिन्दिते}
{निःसम्पातान्मनुष्याणां तस्माच्छुचितमं स्मृतं}


\twolineshloka
{भूतसृष्टिं च तां चाहं श्मशाने संन्यवेशयम्}
{तत्रस्थः सर्वभूतानां विनिहन्मि प्रिये भयम्}


\twolineshloka
{न च भूतगणेनाहं विना वसितुमुत्सहे}
{तस्मान्मे सन्निवासाय श्मशाने रोचते मनः}


\twolineshloka
{मेध्यकामैर्द्विजैनित्यं मेध्यमित्यभिधीयते}
{आचरद्भिर्व्रतं नित्यं मोक्षकामैश्च सेव्यते}


\twolineshloka
{स्थानं मे तत्र विहितं वीरस्थानमिति प्रिये}
{कपालशतसम्पूर्णमभिरूपं भयानकम्}


\twolineshloka
{मध्याह्ने सन्ध्ययोस्तत्र नक्षत्रे रुद्रदेवते}
{आयुष्कामैरशुद्धैर्वा न गन्तव्यमिति स्थितिः}


\twolineshloka
{मदन्येन न शक्यं हि निहन्तुं भूतजं भयम्}
{तत्रस्थोऽहं प्रजाः सर्वाः पालयामि दिनेदिने}


\fourlineindentedshloka
{मन्नियोगाद्भूतसङ्घा न च घ्नन्तीह कञ्चन}
{तांस्तु लोकहितार्ताय श्मशाने रमयान्महम्}
{एतत्ते सर्वमाख्यातं किं भूयः श्रोतुमिच्छसि ॥उमोवाच}
{}


\twolineshloka
{भगवन्देवदेवेश त्रिनेत्र वृषभध्वज}
{पिङ्गलं विकृतं भाति रूपं ते तु भयानकम्}


\twolineshloka
{भस्मदिग्धं विरूपाक्षं तीक्ष्णदंष्ट्रं जटाकुलम्}
{व्याघ्रोदरत्वक्संवीतं कपिलश्मश्रुसंततम्}


\threelineshloka
{रौद्रं भयानकं घोरं शूलपट्टससंयुतम्}
{किमर्थं त्वीदृशं रूपं तन्मे शंसितुमर्हसि ॥महेश्वर उवाच}
{}


\twolineshloka
{तदहं कथयिष्यामि शृणु तत्त्वं समाहिता}
{द्विविधो लौकिको भावः शीतमुष्णमिति प्रिये}


% Check verse!
तयोर्हि ग्रसितं सर्वं सौम्याग्नेयमिदं जगत्
\twolineshloka
{सौम्यत्वं सततं विष्णौ मय्याग्नेयं प्रतिष्ठितम्}
{अनेन वपुषा नित्यं सर्वलोकान्बिभर्म्यहम्}


\twolineshloka
{रौद्राकृतिं विरूपाक्षं शूलपट्टससंयुतम्}
{आग्नेयमिति मे रूपं देवि लोकहिते रतम्}


\twolineshloka
{यद्यहं विपरीतः स्यामेतत्त्यक्त्वा शुभानने}
{तदैव सर्वलोकानां विपरीतं प्रवर्तते}


\threelineshloka
{तस्मान्मयेदं ध्रियते रूपं लोकहितैषिणा}
{इति ते कथितं देवि किं भूयः श्रोतुमिच्छसि ॥उमोवाच}
{}


\fourlineindentedshloka
{भगवन्देवदेवेश शूलपाणे वृषध्वज}
{किमर्थं चन्द्ररेखा ते शिरोभागे विरोचते}
{श्रोतुमिच्छाम्यहं देव तन्मे शंसितुमर्हसि ॥महेश्वर उवाच}
{}


\threelineshloka
{तदहं ते प्रवक्ष्यामि शृणु कल्याणि कारणम्}
{पुराऽहं कारणाद्देवि कोपयुक्तः शुचिस्मिते}
{दक्षयज्ञवधार्याय भूतसङ्घैः समावृतः}


\twolineshloka
{तस्मिन्क्रतुवरे घोरे यज्ञभागनिमित्ततः}
{देवा विभ्रंशितास्ते वै येषां भागः क्रतौ कृतः}


\twolineshloka
{सोमस्तत्र मया देवि कुपितेन भृशार्दितः}
{पश्यंश्चानपराधी सन्पादङ्गुष्ठेन ताडितः}


\twolineshloka
{तथापि विकृतेनाहं सामपूर्वं प्रसादितः}
{तन्मे चिन्तयतश्चासीत्पश्चात्तापः पुरा प्रिये}


\fourlineindentedshloka
{तदाप्रभृति सोमं वै शिरसा धारयाम्यम्}
{एवं मे पापहानिस्तु भवेदिति मतिर्मम}
{तदाप्रभृति वै सोमो मूर्ध्नि संदृश्यते सदा ॥नारद उवाच}
{}


\threelineshloka
{एवं ब्रुवति देवेशे विस्मिताः परमर्षयः}
{वाग्भिः साञ्जलिमालाभिरभितुष्टुवुरीश्वरम् ॥ऋषय ऊचुः}
{}


\twolineshloka
{नमः शङ्कर सर्वेश नमः सर्वजगद्भुरो}
{नमो देवादिदेवाय नमः शशिकलाधर}


\twolineshloka
{नमो घोरतराद्धोर नमो रुद्राय शङ्कर}
{नमः शान्ततराच्छान्त नमश्चन्द्रस्य पालक}


\twolineshloka
{नमः सोमाय देवाय नमस्तुभ्यं चतुर्मुख}
{नमो भूतपते शंभो जह्नुकन्याम्बुशेखर}


\twolineshloka
{नमस्त्रिशूलहस्ताय पन्नगाभरणाय च}
{नमोस्तु विषमाक्षाय दक्षयज्ञप्रदाहक}


\threelineshloka
{नमोस्तु बहुनेत्राय लोकरक्षणतत्पर}
{अहो देवस्य महात्म्यमहो देवस्य वै कृपा}
{एवं धर्मपरत्वं च देवदेवस्य चार्हति}


\threelineshloka
{एवं ब्रुवत्सु मुनिषु वचो देव्यब्रवीद्धरम्}
{सम्प्रीत्यर्थं मुनीनां सा क्षणज्ञा परमं हितम् ॥उमोवाच}
{}


\twolineshloka
{भगवन्देवदेवेश सर्वलोकनमस्कृत}
{अस्यैव ऋषिसङ्घस्य मम च प्रियकाम्यया}


\twolineshloka
{वर्णाश्रमकृतं धर्मं वक्तुमर्हस्यशेषतः}
{न तृप्तिरस्ति देवेश श्रवणीयं हि ते वचः}


\threelineshloka
{सधर्मचारिणी चेयं भक्ता चेयमिति प्रभो}
{वक्तुमर्हसि देवेश लोकानां हितकाम्यया}
{याथातथ्येन तत्सर्वं वक्तुमर्हसि शंकर}


\chapter{अध्यायः २०८}
\twolineshloka
{हन्त ते कथयिष्यामि यत्ते देवि मनःप्रियम्}
{शृणु तत्सर्वमखिलं धर्मं वर्णाश्रमाश्रितम्}


\threelineshloka
{ब्राह्मणाः क्षत्रिया वैश्याः शुद्राश्चेति चतुर्विधम्}
{ब्राह्मणा विहिताः पूर्वं लोकतन्त्रमभीप्सता}
{कर्माणि च तदर्हाणि शास्त्रेषु विहितानि वै}


\twolineshloka
{यदीदमेकवर्णं स्याज्जगत्सर्वं विनश्यति}
{सहैव देवि वर्णानि चत्वारि विहितान्यतः}


\twolineshloka
{मुखतो ब्राह्मणाः सृष्टस्तस्मात्ते वाग्विशारदाः}
{बाहुभ्यां क्षत्रियाः सृष्टास्तस्मात्ते बाहुगर्विताः}


\twolineshloka
{ऊरुभ्यामुद्गता वैश्यास्तस्माद्वार्तोपजीविनः}
{शूद्राश्च पादतः सृष्टास्तस्मात्ते परिचाकाः}


\threelineshloka
{तेषां धर्मांश्च कर्माणि शृणु देवि समाहिता}
{विप्राः कृता भूमिदेवा लोकानां धारणे कृताः}
{ते कैश्चिन्नावमन्तव्या ब्राह्मणा हितमिच्छुभिः}


\twolineshloka
{यदि ते ब्राह्मणा न स्युर्दानयोगवहाः सदा}
{उभयोर्लोकयोर्देवि स्थितिर्न स्यात्समासतः}


\twolineshloka
{लोकेषु दुर्लभं किंतु ब्राह्मणत्वमिति स्मृतम्}
{अबुधो वा दरिद्रो वा पूजनीयः सदैव सः}


\twolineshloka
{ब्राह्मणान्योऽवमन्येत निन्देच्च क्रोधयेच्च वा}
{प्रहरेत हरेद्वाऽपि धनं तेषां नराधमः}


\twolineshloka
{कारयेद्धीनकर्मणि कामलोभविमोहनात्}
{स च मामवमन्येत मां क्रोधयति निन्दति}


\twolineshloka
{मामेव प्रहरेन्मूढो मद्धनस्यापहारकः}
{मामेव प्रेषणं कृत्वा निदन्ते मूढचेतनः}


\threelineshloka
{स्वाध्यायो यजनं दानं तस्य धर्म इति स्थितिः}
{कर्मण्यध्यापनं चैव याजनं च प्रतिग्रहः}
{सत्यं शान्तिस्तपः शौचं तस्य धर्मः सनातनः}


\twolineshloka
{विक्रयो रसधान्यानां ब्राह्मणस्य विगर्हितः}
{अनापदि च शूद्रान्नं वृषलीसङ्ग्रहस्तथा}


\twolineshloka
{तप एव सदा धर्मो ब्राह्मणस्य न संशयः}
{स तु धर्मार्थमुत्पन्नः पूर्वं धात्रा तपोबलात्}


\twolineshloka
{तस्योपनयनं धर्मो नित्यं चोदकधारणम्}
{शास्त्रस्य श्रवणं धर्मो देवव्रतनिषेवणम्}


\threelineshloka
{अग्निकार्यं परो धर्मो नित्ययज्ञोपवीतिता}
{शूद्रान्नवर्जनं धर्मो धर्मः सत्पथसेवनम्}
{धर्मो नित्योपवासित्वं ब्रह्मचर्यं परं तथा}


\twolineshloka
{गृहस्थस्य परो धर्मो गृहस्थाश्रमिणस्तथा}
{गृहसम्मार्जनं धर्म आलेपनविधिस्तथा}


\twolineshloka
{अतिथिप्रियता धर्मो धर्मस्त्रेताग्निधारणम्}
{इष्टिर्वा पशुबन्धाश्च विधिपूर्वं परंतप}


\twolineshloka
{दम्पत्योः समशीलत्वं धर्मो वै गृहमेधिनाम्}
{एवं द्विजन्मनो धर्मो गार्हस्थ्ये धर्मधारणम्}


\twolineshloka
{यस्तु क्षत्रगतो धर्मस्त्वया देवि प्रयोदितः}
{तमहं ते प्रवक्ष्यामि शृणु देवि समाहिता}


\twolineshloka
{क्षत्रियास्तु ततो देवि द्विजानां पालने स्मृताः}
{यदि निःक्षत्रियो लोको जगत्स्यादधरोत्तरम्}


\twolineshloka
{रक्षणात्क्षत्रियैरेव जगद्भवति शाश्वतम्}
{तस्याप्यध्ययनं दानं यजनं धर्मतः स्मृतम्}


\twolineshloka
{दीनानां रक्षणं चैव पापनामनुशासनम्}
{सतां सम्पोषणं चैव कर्मषण्मार्गजीवनम्}


\twolineshloka
{उत्साहः शस्त्रजीवित्वं तस्य धर्मः सनातनः}
{भृत्यानां भरणं धर्मः कृते कर्मण्यमोघता}


\twolineshloka
{सम्यग्गुणयुतो धर्मो धर्मः पौरहितक्रिया}
{व्यवहारस्थितिर्नित्यं गुणयुक्तो महीपतिः}


\twolineshloka
{आर्तवित्तप्रदो राजा धर्मं प्राप्नोत्यनुत्तमम्}
{एवं तैर्विहितः पूर्वैर्धर्मः कर्मविधानतः}


\twolineshloka
{तथैव देवि वैश्याश्च लोकयात्राहिताः स्मृताः}
{अन्ये तानुपजीवन्ति प्रत्यक्षफलदा हि ते}


\twolineshloka
{यदि न स्युस्तथा वैश्या न भवेयुस्तथा परे}
{तेषामध्ययनं दानं गजनं धर्म उच्यते}


\twolineshloka
{वैश्यस्य सततं धर्मः पाशुपाल्यं कृषिस्तथा}
{अग्निहोत्रपरिस्पन्दास्त्रयो वर्णा द्विजातयः}


\twolineshloka
{वाणिज्यं सत्पथे स्थानमातिथेयत्वमेव च}
{विप्राणां स्वागतन्यायो वैश्यधर्मः सनातनः}


\twolineshloka
{तिलगन्धरसाश्चैव न विक्रेयाः कथञ्चन}
{वणिक्पथमुपासद्भिर्वैश्यैर्वैश्यपथि स्थितैः}


\twolineshloka
{सर्वातिथ्यं त्रिवर्गस्य यथाशक्ति दिवानिशम्}
{एवं ते विहिता देवि लोकयात्रा स्वयंभुवा}


\twolineshloka
{तथैव शूद्रा विहिताः सर्वधर्मप्रसाधकाः}
{शूद्राश्च यदि ते न स्युः कर्मकर्ता न विद्यते}


\twolineshloka
{त्रयः पूर्वे शूद्रमूलाः सर्वे कर्मकराः स्मृताः}
{ब्राह्मणादिषु शुश्रूषा दासधर्म इति स्मृतः}


\twolineshloka
{वार्ता च कारुकर्माणि शिल्यं नाट्यं तथैव च}
{अहिंसकः शुभाचारो देवतद्विजवन्दकः}


\twolineshloka
{शूद्रो धर्मफलैरिष्टैः स्वधर्मेणोपपद्यते}
{एवमादि तथाऽन्यच्च शूद्रधर्म इति स्मृतः}


% Check verse!
तेऽप्येवं विहिता लोके कर्मयोग्याः शुभानने
\twolineshloka
{एवं चतुर्णां वर्णानां वर्णलोकाः परत्र च}
{विहिताश्च तथा दृष्टा यथावद्धर्मचारिणि}


\threelineshloka
{एष वर्णाश्रयो धर्मः कर्म चैव तदर्पणम्}
{कथितं श्रोतुकामायाः किं भूयः श्रोतुमिच्छसि ॥उमोवाच}
{}


\threelineshloka
{भगवन्देवदेवेश नमस्ते वृषभध्वज}
{श्रोतुमिच्छाम्यहं देव धर्ममाश्रमिणां विभो ॥महेश्वर उवाच}
{}


\twolineshloka
{तथाऽऽश्रमगतं धर्मं शृणु देवि समाहिता}
{आश्रमाणां तु यो धर्मः क्रियते ब्रह्मवादिभिः}


\twolineshloka
{गृहस्थः प्रवरस्तेषां गार्हस्थ्यं धर्ममाश्रितः}
{पञ्चयज्ञक्रियाशौचं दारतुष्टिरतन्द्रिता}


\twolineshloka
{ऋतुकालाभिगमनं दानयज्ञतपांसि च}
{अविप्रवासस्तस्येष्टः स्वाध्यायश्वाग्निपूर्वकम्}


\twolineshloka
{अतिथीनामाभिमुख्यं शक्त्या चेष्टनिमन्त्रणम्}
{अनुग्रहश्च सर्वेषां मनोवाक्कायकर्मभिः}


\threelineshloka
{एवमादि शुभं चान्यत्कुर्यात्तद्वृत्तिमान्गृही}
{एवं सञ्चरतस्तस्य पुण्यलोका न संशयः}
{}


% Check verse!
तथैव वानप्रस्थस्य धर्माः प्रोक्ताः सनातनाः
\twolineshloka
{गृहवासं समुत्सृकज्य निश्चित्यैकमनाः शुभैः}
{वन्यैरेव तदाहारैः वर्तयेदिति च स्थितिः}


\twolineshloka
{भूमिशय्या जटाश्मश्रुचर्मवल्कलधारणम्}
{देवतातिथिसत्कारो महाकृच्छ्राभिपूजनम्}


\threelineshloka
{अग्निहोत्रं त्रिषवणं नित्यं तस्य विधीयते}
{ब्रह्मचर्यं क्षमा शौचं तस्य धर्मः सनातनः}
{एवं स विगते प्राणे देवलोके महीयते}


\twolineshloka
{यतिधर्मास्तथा देवि गृहांस्त्यक्त्वा यतस्ततः}
{आकिञ्चन्यमनारम्भः सर्वतः शौचमार्जवम्}


\threelineshloka
{सर्वत्र भैक्षचर्या च सर्वत्रैव विवासनम्}
{सदा ध्यानपरत्वं च देहशुद्धिः क्षमा दया}
{तत्वानुगतबुद्धित्वं तस्य धर्मविधिर्भवेत्}


\threelineshloka
{ब्रह्मचारी च यो देवि जन्मप्रभृति भिक्षितः}
{ब्रह्मचर्यपरो भूत्वा साधयेद्गुरुमात्मनः}
{सर्वकालेषु सर्वत्र गुरुपूजां समाचरेत्}


\twolineshloka
{भैक्षचर्याग्निकार्यं च सदा जलनिषेवणम्}
{स्वाध्यायः सततं तस्य एष धर्मः सनातनः}


\twolineshloka
{तस्य चेष्टा तु गुर्वर्थमाप्राणान्तमिति स्थितिः}
{गुरोरभावे तत्पुत्रे गुरुवद्वृत्तिमाचरेत्}


\twolineshloka
{एवं सोप्यमलाँल्लोकान्ब्राह्मणः प्रतिपद्यते}
{एष ते कथितो देवि धर्मश्चाश्रमवासिनाम्}


\twolineshloka
{चतुर्णामाश्रमो युक्तो लोक इत्येव विद्यते}
{कथितं ते समासेन किं भूयः श्रोतुमिच्छसि}


\chapter{अध्यायः २०९}
\twolineshloka
{भगवन्देवदेवेश त्रिपुरान्तक शङ्कर}
{अयं त्वृषिगणो देव तपस्तप इति प्रभो}


\fourlineindentedshloka
{तपसा कर्शितो नित्यं तपोऽर्जनपरायणः}
{अस्य किंलक्षणो धर्मः कीदृशश्चागमस्तथा}
{एतदिच्छाम्यहं श्रोतुं तन्मे वद वरप्रद ॥नारद उवाच}
{}


\twolineshloka
{एवं ब्रुवन्त्यां रुद्राण्यामृषयः साधुसाध्विति}
{अब्रुवन्हृष्टमनसः सर्वे तद्गतमानसाः}


\twolineshloka
{शृण्वन्तीमृषिधर्मांस्तु ऋषयश्चाभ्यपूजयन् ॥ऋषय ऊचुः}
{}


\fourlineindentedshloka
{त्वत्प्रसादद्वयं देवि श्रोष्यामः परमं हितम्}
{धन्याः खलु वयं सर्वे पादमूलं तवाश्रिताः}
{इति सर्वे तदा देवीं वाचा समभिपूजयन् ॥महेश्वर उवाच}
{}


\twolineshloka
{न्यायतस्त्वं महाभागे श्रोतुकामा मनस्विनि}
{हन्त ते कथयिष्यामि मुनिधर्मं शुचिस्मिते}


\twolineshloka
{वानप्रस्थं समाश्रित्य क्रियते बहुधा नरैः}
{बहुशाखो बहुविधो ऋषिधर्मः सनातनः}


\twolineshloka
{प्रायशः सर्वभोगार्थमृषिभिः क्रियते तपः}
{तथा सञ्चरतां तेषां देवि धर्मविधिं शृणु}


\twolineshloka
{भूत्वा पूर्वं गृहस्थस्तु पुत्रानृण्यमवाप्य च}
{कलत्रकार्यं संस्थाप्य कारणात्संत्यजेद्गृहम्}


\twolineshloka
{अवस्थाप्य मनो धृत्या व्यवसायपुरःसरः}
{निर्दारो वा सदारो वा वनवासाय संव्रजेत्}


\twolineshloka
{देशाः परमपुण्या ये नदीवनसमन्विताः}
{अबोधमुक्ताः प्रायेण तीर्थायतनसंयुताः}


\twolineshloka
{तत्र गत्वा विधिं ज्ञात्वा दीक्षां कुर्याद्यथागमम्}
{दीक्षित्वैकमना भूत्वा परिचर्यां समाचरेत्}


\twolineshloka
{काल्योत्थानं च शौचं च सर्वदेवप्रणामनम्}
{सकृदालेपनं काये त्यक्तदोषोऽप्रमादिता}


\twolineshloka
{सायंप्रातश्चाभिषेकं चाग्निहोत्रं यथाविधि}
{काले शौचं च कार्यं च जटावल्कलधारणम्}


\twolineshloka
{सततं वनचर्या च समित्कुसुमकारणात्}
{नीवाराग्रयणं काले शाकमूलोपचायनम्}


\twolineshloka
{सदायतनशौचं च तस्य धर्माय चेष्यते}
{अतिथीनामाभिमुख्यं तत्परत्वं च सर्वदा}


\twolineshloka
{पाद्यासनाभ्यां सम्पूज्य तथाऽऽहारनिमन्त्रणम्}
{अग्राम्यपचनं काले पितृदेवार्चनं तथा}


\twolineshloka
{पश्चादतिथिसत्कारस्तस्य धर्मः सनातनः}
{शिष्टैर्धर्मासने चैव धर्मार्थसहिताः कथाः}


\twolineshloka
{प्रतिश्रयविभागश्च भूमिशय्या शिलासु वा}
{व्रतोपवासयोगश्च क्षमा चेन्द्रियनिग्रहः}


\twolineshloka
{दिवारात्रं यथायोगं शौचं धर्मस्य चिन्तनम्}
{एवं धर्माः पुरा दृष्टाः सामान्या वनवासिनाम्}


\twolineshloka
{एवं वै यतमानस्य कालधर्मो यथा भवेत्}
{तथैव सोऽभिजयति स्वर्गलोकं शुचिस्मिते}


\twolineshloka
{तत्र संविदिता भोगाः स्वर्गस्त्रीभिरनिन्दिते}
{परिभ्रष्टो यथा स्वर्गाद्विशिष्टस्तु भवेन्नृषु}


\threelineshloka
{एवं धर्मस्तथा देवि सर्वेषां वनवासिनाम्}
{एतत्ते कथितं सर्वं किं भूयः श्रोतुमिच्छसि ॥उमोवाच}
{}


\threelineshloka
{भगवन्देवदेवेश ऋषीणां चरितं शुभम्}
{विशेषधर्मानिच्छामि श्रोतुं कौतूहलं हि मे ॥महेश्वर उवाच}
{}


% Check verse!
तदहं ते प्रवक्ष्यामि शृणु देवि समाहिता
\twolineshloka
{वननित्यैर्वनरतैर्वानप्रस्थैर्महर्षिभिः}
{वनं गुरुमिवालम्ब्य वस्तुव्यमिति निश्चयः}


\twolineshloka
{वीरशय्यामुपासद्भिर्वारस्थानोपसेविभिः}
{व्रतोपवासैर्बहुलैर्ग्रीष्मे पञ्चतपैस्तथा}


\twolineshloka
{पञ्चयज्ञपरैर्नित्यं पौर्णमास्यापरायणैः}
{मण्डूकशायैर्हेमन्ते शैवालाङ्कुरभोजनैः}


\twolineshloka
{चीरवल्कलसंवीतैर्मृगाजिनधरैस्तथा}
{चातुर्मास्यपरैः कैश्चिद्देवधर्मपरायणैः}


\twolineshloka
{एवंविधैर्वनेवासैस्तप्यते सुमहत्तपः}
{एवं कृत्वा शुभं कर्म पश्चाद्याति त्रिविष्टपम्}


\twolineshloka
{तत्रापि सुमहत्कालं संविहृत्यि यथासुखम्}
{जायते मानुषे लोके दानभोगसमन्वितः}


\twolineshloka
{तपोविशेषसंयुक्ताः कथितास्ते शुचिस्मिते ॥उमोवाच}
{}


\threelineshloka
{भगवन्देवदेवेश तेषु ये दारसंयुताः}
{कीदृशं चरितं तेषां तन्मे शंसितुमर्हसि ॥महेश्व उवाच}
{}


\twolineshloka
{य एकपत्नीधर्माणश्चरन्ति विपुलं तपः}
{विंध्यपादेषु ये केचिद्ये च नैमिशवासिनः}


\twolineshloka
{पुष्करेषु च ये चान्ये नदीवनसमाश्रिताः}
{सर्वे ते विधिदृष्टेन चरन्ति विपुलं तपः}


\twolineshloka
{हिंसाद्रोहविमुक्ताश्च सर्वभूतानुकम्पिनः}
{शान्ता दान्ता जितक्रोधाः सर्वातिथ्यपरायणाः}


\twolineshloka
{प्राणिष्वात्मोपमा नित्यमृतुकालाभिगामिनः}
{स्वदारसहिता देवि चरन्ति व्रतमुत्तमम्}


\twolineshloka
{वसन्ति सुखमव्यग्राः पुत्रदारसमन्विताः}
{तेषां परिच्छदारम्भाः कृतोपकरणानि च}


\fourlineindentedshloka
{गृहस्थवद्द्वितीयं ते यथायोगं प्रमाणतः}
{पोषणार्थं स्वदाराणामग्निकार्यार्थमेव च}
{गावश्च कर्षणं चैव सर्वमेतद्विधीयते}
{}


\twolineshloka
{एवं वनगतैर्देवि कर्तव्यं दारसङ्ग्रहैः}
{ते स्वदारैः समायान्ति पुण्याँल्लोकान्द्दढव्रताः}


\threelineshloka
{पतिभिः सह ये दाराश्चरन्ति विपुलं तपः}
{अव्यग्रभावादैकात्म्यात्ताश्च गच्छन्ति वै दिवम्}
{एतत्ते कथितं देवि किं भूयः श्रोतुमिच्छसि}


\chapter{अध्यायः २१०}
\threelineshloka
{भगवन्देवदेवेश तेषां कर्मफलं प्रभो}
{श्रोतुमिच्छाम्यहं देव प्रसादात्ते वरप्रद ॥महेश्वर उवाच}
{}


% Check verse!
वानप्रस्थगतं सर्वं फलपाकं शृणु प्रिये
\twolineshloka
{अग्नियोगं व्रजन्ग्रीष्मे ततो द्वादशवार्षिकम्}
{रुद्रलोकेऽभिजायेत विधिदृष्टेन कर्मणा}


\twolineshloka
{उपवासव्रतं कुर्वन्वर्षाकाले दृढव्रतः}
{सोमलोकेऽभिजायेत नरो द्वादशवार्षिकम्}


\twolineshloka
{काष्ठवन्मौनमास्थाय नरो द्वादशवार्षिकम्}
{मरुतां लोकमास्थाय तत्र भोगैश्च युज्यते}


\twolineshloka
{कुशशर्करसंयुक्ते स्थण्डिले संविशन्मुनिः}
{यक्षलोकेऽभिजायेत सहस्राणि चतुर्दश}


\threelineshloka
{वर्षाणां भोगसंयुक्तो नरो द्वादशवार्षिकम्}
{वीरासनगतो यस्तु कण्टकाफलकाश्रितः}
{गन्धर्वेष्वभिजायेत नरो द्वादशवार्षिकम्}


\twolineshloka
{वीरस्थायी चोर्ध्वबाहुर्नरो द्वादशवार्षिकम्}
{देवलोकेऽभिजायेत दिव्यभोगसमन्वितः}


\threelineshloka
{पादाङ्गुष्ठेन यस्तिष्ठेदूर्ध्वबाहुर्जितेन्द्रियः}
{इन्द्रलोकेऽभिजायेत सहस्राणि चतुर्दश}
{}


\twolineshloka
{आहारनियमं कृत्वा मुनिर्द्वादशवार्षिकम्}
{नागलोकेऽभिजायेत संवत्सरगणान्बहून्}


\twolineshloka
{एवं दृढव्रता देवि वानप्रस्थाश्च कर्मभिः}
{स्थानेषु तत्र तिष्ठन्ति तत्तद्भोगसमन्विताः}


\twolineshloka
{तेभ्यो भ्रष्टाः पुनर्देवि जायन्ते नृषु भोगिनः}
{वर्णोत्तमकुलेष्वेव धनधान्यसमन्विताः}


\twolineshloka
{एतत्ते कथितं देवि किं भूयः श्रोतुमिच्छसि ॥उमोवाच}
{}


\threelineshloka
{एषां यथा वराणां तु धर्ममिच्छामि मानद}
{कृपया परयाऽऽविष्टस्तन्मे ब्रूहि महेश्वर ॥महेश्वर उवाच}
{}


% Check verse!
धर्मं यथा वराणां त्वं शृणु भामिनि तत्परा
\twolineshloka
{व्रतोपवासशुद्धाङ्गास्तीर्थस्नानपरायणाः}
{धृतिमन्तःक क्षमायुक्ताः सत्यव्रतपरायणाः}


\twolineshloka
{पक्षमासोपवासैश्च कर्शिता धर्मदर्शिनः}
{वर्षैः शीतातपैश्चैव कुर्वन्तः परमं तपः}


\twolineshloka
{कालयोगेन गच्छन्ति शक्रलोकं शुचिस्मिते}
{तत्र ते भोगसंयुक्ता दिव्यगन्धसमन्विताः}


\fourlineindentedshloka
{दिव्यभूषणसंयुक्ता विमानवरसंयुताः}
{विचरन्ति यथाकामं दिव्यस्त्रीगणसंयुताः}
{एतत्ते कथितं देवि किं भूयः श्रोतुमिच्छसि ॥उमोवाच}
{}


\twolineshloka
{तेषां चक्रचराणां च धर्ममिच्छामि वै प्रभो ॥महेश्वर उवाच}
{}


% Check verse!
हन्त ते कथयिष्यामि शृणु शाकटिकं शुभे
\twolineshloka
{संवहन्तो धुरं दानैः शकटानां तु सर्वदा}
{प्रार्थयन्ते यथाकालं शकटैर्भैक्षचर्यया}


\twolineshloka
{तपोर्जनपरा धीरास्तपसा क्षीणकल्मषाः}
{पर्यटन्तो दिशः सर्वाः कामक्रोधविवर्जिताः}


\twolineshloka
{तेनैव कालयोगेन त्रिदिवं यान्ति शोभने}
{तत्र प्रमुदिता भोगैर्विचरन्ति यथासुखम्}


\twolineshloka
{एतत्ते कथितं देवि किं भूयः श्रोतुमिच्छसि ॥उमोवाच}
{}


\twolineshloka
{वैखानसानां वै धर्मं श्रोतुमिच्छाम्यहं प्रभो ॥महेश्वर उवाच}
{}


\threelineshloka
{ते वै वैखानसा नाम वानप्रस्थाः शुभेक्षणे}
{तीव्रेण तपसा युक्ता दीप्तिमन्तः स्वतेजसा}
{सत्यव्रतपरा धीरास्तेषां निष्कल्मषं तपः}


\twolineshloka
{अश्मकुट्टास्तथाऽन्ये च दन्तोलूखलिनस्तथा}
{शीर्णपर्णाशिनश्चान्ये उञ्छवृत्तास्तथा परे}


\twolineshloka
{कपोतवृत्तयश्चान्ये कापोतीं वृत्तिमाश्रिताः}
{पशुप्रचारनिरताः फेनपाश्च तथा परे}


\twolineshloka
{मृगवन्मृगचर्यायां सञ्चरन्ति तथा परे}
{अब्भक्षा वायुभक्षाश्च निराहारास्तथैव च}


\threelineshloka
{केचिच्चरन्ति सद्विष्णोः पादपूजनमुत्तमम्}
{सञ्चरन्ति तपो घोरं व्याधिमृत्युविवर्जिताः}
{स्ववशादेव ते मृत्युं भीषयन्ति च नित्यशः}


\twolineshloka
{इन्द्रलोके तथा तेषां निर्मिता भोगसञ्चयाः}
{अमरैः समतां यान्ति देववद्भोगसंयुताः}


\threelineshloka
{वराप्सरोभिः संयुक्ताश्चिरकालमनिन्दिते}
{एतत्ते कथितं देवि भूयः श्रोतुं किमिच्छसि ॥उमोवाच}
{}


\twolineshloka
{भगवञ्श्रोतुमिच्छामि वालखिल्यांस्तपोधनान् ॥महेस्वर उवाच}
{}


% Check verse!
धर्मचर्यां तथा देवि वालखिल्यगतां शृणु
\twolineshloka
{मृगनिर्मोकवसना निर्द्वन्द्वास्ते तपोधनाः}
{अङ्गुष्ठमात्राः सुश्रोणि तेष्वेवाङ्गेषु संश्रिताः}


\threelineshloka
{उद्यन्तं सततं सूर्यं स्तुवन्तो विविधैः स्तवैः}
{भास्करस्येव किरणैः सहसा यान्ति नित्यदा}
{द्योतयन्तो दिशः सर्वा धर्मज्ञाः सत्यवादिनः}


\twolineshloka
{तेष्वेव निर्मलं सत्यं लोकार्थं तु प्रतिष्ठितम्}
{लोकोऽयं धार्यते देवि तेषामेव तपोबलात्}


\twolineshloka
{महात्मनां तु तपसा सत्येन च शुचिस्मिते}
{क्षमया च महाभागे भूतानां संस्थितिं विदुः}


\threelineshloka
{प्रजार्थमपि लोकार्थं महद्भिः क्रियते तपः}
{तपसा प्राप्यते सर्वं तपसा प्राप्यते फलम्}
{दुष्प्रापमपि यल्लोके तपसा प्राप्यते हि तत्}


\twolineshloka
{पञ्चभूतार्थतत्वे च लोकसृष्टिविर्धनम्}
{एतत्सर्वं समासेन तपोयोगाद्विनिर्मितम्}


\twolineshloka
{तस्मादयं त्वृषिगणस्तपस्तप इति प्रिये}
{धर्मान्वेषी तपः कर्तुं यतते सततं प्रिये}


\fourlineindentedshloka
{अमरत्वं शिवत्वं च तपसा प्रापयेत्सदा}
{एतत्ते कथितं सर्वं शृण्वन्त्यास्ते श्रुतं प्रिये}
{प्रियार्थमृषिसङ्घस्य प्रजानां हितकाम्यया ॥नारद उवाच}
{}


\twolineshloka
{एवं ब्रुवन्तं देवेशमृषयश्चापि तुष्टुवुः}
{भूयः परतरं यत्तु तदाप्रभृति चक्रिरे}


\chapter{अध्यायः २११}
\fourlineindentedshloka
{उक्तस्त्वया त्रिवर्गस्य धर्मश्च परमः शुभः}
{सर्वव्यापी तु यो धर्मो भगवंस्तं ब्रवीतु मे}
{महेश्वर उवाच}
{}


\twolineshloka
{ब्रह्मणा लोकसंसारे सृष्टा धात्रा गुणार्थिना}
{लोकांस्तारयितुं युक्ता मर्त्येषु क्षितिदेवताः}


\twolineshloka
{तेषु तावत्प्रवक्ष्यामि धर्मं शुभफलोदयम्}
{ब्राह्मणेष्वभयो धर्मः परमः शुभलक्षणः}


\twolineshloka
{इमे च धर्मा लोकानां पूर्वं सृष्टाः स्वयंभुवा}
{पृथिव्यां सद्द्विजैर्नित्यं कीर्त्यमानं निबोध मे}


\twolineshloka
{स्वदारनिरतिर्धर्मो नित्यं जप्यं तथैव च}
{सर्वातिथ्यं त्रिवर्गस्य यथाशक्ति दिवानिशम्}


% Check verse!
शूद्रो धर्मपरो नित्यं शुश्रूषानिरतो भवेत्
\twolineshloka
{त्रैविद्यो ब्राह्मणो वृद्धो न चाध्ययनजीवकः}
{त्रिवर्गस्य व्यतिक्रान्तं तस्य धर्मः सनातनः}


\twolineshloka
{षट् कर्माणि च प्रोक्तानि सृष्टानि ब्रह्मणा पुरा}
{धर्मिष्ठानि वरिष्ठानि तानितानि शृणूत्तमे}


\twolineshloka
{यजनं याजनं चैव दानं पात्रे प्रतिग्रहः}
{अध्यापनमध्ययनं षट्कर्मा धर्मभागृजुः}


\twolineshloka
{नित्यः स्वाध्यायतो धर्मः नित्ययज्ञः सनातनः}
{दानं प्रशंसते नित्यं ब्राह्मणेषु त्रिकर्मसु}


\twolineshloka
{अयमेव परो धर्मः संवृतः सत्सु विद्यते}
{गर्भस्थाने विशुद्धानां धर्मस्य नियमो महान्}


\twolineshloka
{पञ्चयज्ञविशुद्धात्मा ऋतुनित्योऽनसूयकः}
{दान्तो ब्राह्मणसत्कर्ता सुसंमृष्टनिवेशनः}


\twolineshloka
{चक्षुःश्रोत्रमनोजिह्वास्निग्धवर्णप्रदः सदा}
{अतिथ्यभ्यागतरतः शेषान्नकृतभोजनः}


\twolineshloka
{पाद्यमर्घ्यं यथान्यायमासनं शयनं तथा}
{दीपं प्रतिश्रयं चैव यो ददाति स धार्मिकः}


\twolineshloka
{प्रातरुत्थाय वै पश्चाद्भोजने तु निमन्त्रयेत्}
{सत्कृत्याऽनुव्रजेद्यश्च तस्य धर्मः सनातनः}


\twolineshloka
{प्रवृत्तिलक्षणो धर्मो गृहस्थेषु विधीयते}
{तदहं कीर्तयिष्यामि त्रिवर्गेषु च यद्यथा}


\twolineshloka
{एकेनांशेन धर्मार्थः कर्तव्यो हितमिच्छता}
{एकेनांशेन कामार्थमेकमंशं विवर्धयेत्}


\twolineshloka
{निवृत्तिलक्षणः पुण्यो धर्मो मोक्षो विधीयते}
{तस्य वृत्तिं प्रवक्ष्यामि तां शृणुष्व समाहिता}


\threelineshloka
{सर्वभूतदया धर्मो निवृत्तेः परम सदा}
{बुभुक्षितं पिपासार्तमतिथिं श्रान्तमागतम्}
{अर्चयन्ति वरारोहे तेषामपि फलं महत्}


\twolineshloka
{पात्रमित्येव दातव्यं सर्वस्मै धर्मकाङ्क्षिभिः}
{आगमिष्यति यत्पात्रं तत्पात्रं तारयिष्यति}


\twolineshloka
{काले सम्प्राप्तमतिथिं भोक्तुकाममुपस्थितम्}
{यस्तं सम्भावयेत्तत्र व्यासोऽयं समुपस्थितः}


\twolineshloka
{तस्य पूजां यथाशक्त्या सौम्यचित्तः प्रयोजयेत्}
{चित्तमूलो भवेद्धर्मो धर्ममूलं भवेद्यशः}


\twolineshloka
{तस्मात्सौम्येन चित्तेन दातव्यं देवि सर्वथा}
{सौम्यचित्तस्तु यो दद्यात्तद्धि दानमनुत्तमम्}


\twolineshloka
{यथाम्बुबिन्दुभिः सूक्ष्मैः पतद्भिर्मेदिनीतले}
{केदाराश्च तटाकानि सरांसि सरितस्तथा}


\twolineshloka
{तोयपूर्णानि दृश्यन्ते अप्रतर्क्योऽतिशोभने}
{अल्पमल्पमपि ह्येकं दीयमानं विवर्धते}


\twolineshloka
{पीडयाऽपि च भृत्यानां दानमेव विशिष्यते}
{पुत्रदारधनं धान्यं न मृताननुगच्छति}


\twolineshloka
{श्रेयो दानं च भोगश्च धनं प्राप्य यशस्विनि}
{दानेन हि महाभागा भवन्ति मनुजाधिपाः}


\twolineshloka
{नास्ति भूमौ दानसमं नास्ति दानसमो निधिः}
{नास्ति सत्यात्परो धर्मो नानृतात्पातकं परम्}


\threelineshloka
{आश्रमे यस्तु तप्येत तपो मूलफलाशनः}
{आदित्याभिमुखो भूत्वा जटावल्कलसंवृतः}
{मण्डूकशायी हेमन्ते ग्रीष्मे पञ्चतपा भवेत्}


\threelineshloka
{सम्यक्तपश्चरन्तीह श्रद्दधाना वनाश्रमे}
{गृहाश्रमस्य ते देवि कलां नार्हन्ति षोडशीम् ॥उमोवाच}
{}


\twolineshloka
{गृहाश्रमस्य या चर्या व्रतानि नियमाश्च ये}
{तथा च देवताः पूज्याः सततं गृहमेधिना}


\threelineshloka
{यद्यच्च परिहर्तव्यं गृहीणातिथिपर्वसु}
{तत्सर्वं श्रोतुमिच्छामि कथ्यमानं त्वया विभो ॥महेश्वर उवाच}
{}


\twolineshloka
{गृहाश्रमस्य यन्मूलं फलं धर्मोऽयमुत्तमम्}
{पादैश्चतुर्भिः सततं धर्मो यत्र प्रतिष्ठितः}


\twolineshloka
{सारभूतं वरारोहे दध्नो घृतमिवोद्धृतम्}
{तदहं ते प्रवक्ष्यामि श्रूयतां धर्मचारिणि}


\twolineshloka
{`दंपत्यलङ्करिष्णुश्च गृहदानरतिर्नरः}
{कलत्रसौख्यं विन्देत नास्ति तत्र विचारणा}


\twolineshloka
{स्त्रियो वा पुरुषो वाऽपि दम्पतीन्पूजयन्ति ते}
{मनोभिलषितान्कामान्प्राप्नुवन्ति न संशयः ॥'}


\twolineshloka
{शुश्रूषन्ते ये पितरं मातरं च गृहाश्रमे}
{भर्तारं चैव या नारी अग्निहोत्रं च ये द्विजाः}


\threelineshloka
{तेषुतेषु च प्रीणन्ति देवा इन्द्रपुरोगमाः}
{पितरः पितृलोकस्थाः स्वधर्मेण स रज्यते ॥उमोवाच}
{}


\threelineshloka
{मातापितृवियुक्तानां का चर्या गृहमेधिनाम्}
{विधवानां च नारीणां भवानेव ब्रवीतु मे ॥महेश्वर उवाच}
{}


\twolineshloka
{देवतातिथिशुश्रूषा गुरुवृद्धाभिवादनम्}
{अहिंसा सर्वभूतानामलोभः सत्यसन्धता}


\twolineshloka
{ब्रह्मचर्यं शरण्यत्वाशौचं पूर्वाभिभाषणम्}
{कृतज्ञत्वमपैशुन्यं सततं धर्मशीलता}


\twolineshloka
{दिने द्विरभिषेकं च पितृदैवतपूजनम्}
{गवाह्निकप्रदानं च संविभागोऽतिथिष्वपि}


\threelineshloka
{दीपप्रतिश्रयं चैव दद्यात्पाद्यासनं तथा}
{पञ्चमेऽहनि षष्ठे वा द्वादशेऽप्यष्टमेऽथवा}
{तदुर्दशे पञ्चदशे ब्रह्मचारी सदा भवेत्}


\twolineshloka
{श्मश्रुकर्म शिरोभ्यङ्गमञ्जनं दन्तधावनम्}
{नैतेष्वहस्तु कुर्वीत तेषु लक्ष्मीः प्रतिष्ठिता}


\twolineshloka
{व्रतोपवासनियमस्तपो दानं च शक्तितः}
{भरणं भृत्यवर्गस्य दीनानामनुकम्पनम्}


\threelineshloka
{परदारान्निवृत्तिश्च स्वदारेषु रतिः सदा}
{शरीरमेकं दंपत्योर्विधात्रा पूर्वनिर्मितम्}
{तस्मात्स्वदारनिरतो ब्रह्मचारी विधीयते}


\twolineshloka
{शीलवृत्तविनीतस्य निगृहीतेन्द्रियस्य च}
{आर्जवे वर्तमानस्य सर्वभूतहितैषिणः}


\twolineshloka
{प्रियातिथेश्च क्षान्तस्य धर्मार्जितधनस्य च}
{गृहाश्रमपदस्थस्य किमन्यैः कृत्यमाश्रमैः}


\twolineshloka
{यथा मातरमाश्रित्य सर्वे जीवन्ति जन्तवः}
{तथा गृहाश्रमं प्राप्य सर्वे जीवन्ति चाश्रमाः}


\twolineshloka
{राजानः सर्वपाषण्डाः सर्वे रङ्गोपजीविनः}
{व्यालग्रहाश्च डम्याश्च चोरा राजभटास्तथा}


\threelineshloka
{सविद्याः सर्वजीवज्ञाः सर्वे वै विचिकित्सकाः}
{दूराध्वानं प्रपन्नाश्च क्षीणपथ्योदना नराः}
{एते चान्ये च बहवस्तर्कयन्ति गृहाश्रमम्}


\threelineshloka
{मार्जारा मूषिकाः श्वानः सूकराश्च शुकास्तथा}
{कपोतका कावटकाः सरीसृपनिषेवणाः}
{अरण्यवासिनश्चान्ये सङ्घा ये मृगपक्षिणाम्}


\twolineshloka
{एवं बहुविधा देवि लोकेऽस्मिन्सचराचराः}
{गृहे क्षेत्रे बिले चैव शतशोऽथ सहस्रशः}


\twolineshloka
{गृहस्थेन कृतं कर्म सर्वैस्तैरिह भुज्यते}
{उपयुक्तं च यत्तेषां मतिमान्नानुशोचति}


\threelineshloka
{धर्म इत्येव सह्कल्प्य यस्तु तस्य फलं शृणु}
{सर्वयज्ञप्रणीतस्य हयमेधेन यत्फलम्}
{वर्षे स द्वादशे देवि फलेनैतेन युज्यते}


\twolineshloka
{आशापाशविमोक्षं च विधिधर्ममनुत्तमम्}
{वृक्षमूलचरो नित्यं शून्यागारिनिवेशनमम्}


\threelineshloka
{नदीपुलिनशायी च नदीतीरमनुव्रजन्}
{विमुक्तः सर्वसङ्गेभ्यः स्नेहबन्धेन वै द्विजः}
{}


\twolineshloka
{आत्मन्येवात्मना भावं समायोज्येह तेन वै}
{आत्मभूतो यताहारो मोक्षदृष्टेन कर्मणा}


\threelineshloka
{पवित्रनित्यो युक्तश्च तस्य धर्मःक सनातनः}
{नैकत्र रमते सक्तो न चैकग्रामगोचरः}
{युक्तोऽप्यटति यो युक्तो न चैकपुलिनाश्रयः}


% Check verse!
एष मोक्षविदां धर्मो वेदोक्तः सत्पथे स्थितः
\chapter{अध्यायः २१२}
\twolineshloka
{देवदेव नमस्तुभ्यं त्र्यक्ष भो वृषभध्वज}
{श्रुतं मे भगवन्सर्वं त्वत्प्रसादान्महेश्वर}


\twolineshloka
{सङ्गृहीतं मया तच्च तव वाक्यमनुत्तमम्}
{इदानीमस्ति संदेहो मानुषेष्विह कश्चन}


\twolineshloka
{तुल्यप्राणशिरःकायो राजाऽयमिति मृश्यते}
{केन कर्मविपाकेन सर्वप्राधान्यमर्हति}


\fourlineindentedshloka
{स चापि दण्डयन्मर्त्यान्भर्त्सयन्विधमन्नपि}
{प्रेत्यभावे कथं लोकाँल्लभते पुण्यकर्मणा}
{राजवृत्तमहं तस्माच्छ्रोतुमिच्छामि मानद ॥महेश्वर उवाच}
{}


% Check verse!
तदहं ते प्रवक्ष्यामि राजधर्मं शुभानने
\twolineshloka
{राजायत्तं हि यत्सर्वं लोकवृत्तं शुभाशुभम्}
{महतस्तपसो देवि फलं राज्यमिति स्मृतम्}


\twolineshloka
{तपोदानमयं राज्यं परं स्थानं विधीयते}
{तस्माद्राज्ञः सदा मर्त्याः प्रणमन्ति यतस्ततः}


\twolineshloka
{न्यायतस्त्वं महाभागे श्रोतुकामाऽसि भामिनि}
{तस्मात्तस्यैव चरितं जगत्पथ्यं शृणु प्रिये}


\threelineshloka
{अराजके पुरा त्वासीत्प्रजानां सङ्कुलं महत्}
{तदृष्ट्वा सङ्कुलं ब्रह्मा मनुं राज्ये न्यवेदयत्}
{तदाप्रभृति संदृष्टं राज्ञां वृत्तं शुभाशुभम्}


\twolineshloka
{तन्मे शृणु वरारोहे तस्य पथ्यं जगद्धितम्}
{यथा प्रेत्य लभेत्स्वर्गं यथा वीर्यं यशस्तथा}


\twolineshloka
{पित्र्यं वा भूतपूर्वं वा स्वयमुत्पाद्य वा पुनः}
{राज्यधर्ममनुष्ठाय विधिवद्भोक्तुमर्हति}


\twolineshloka
{आत्मानमेव प्रथमं विनयैरुपपादयेत्}
{अनु भृत्यान्प्रजाः पश्चादित्येष विनयक्रमः}


\twolineshloka
{स्वामिनं चोषमां कृत्वा प्रजास्तद्वृत्तकाङ्क्षया}
{स्वयं विनयसम्पन्ना भवन्तीह शुभेक्षणे}


\twolineshloka
{स्वस्मात्पूर्वतरा राजा विनयत्येव वै प्रजाः}
{अपहास्यो भवेत्तादृक्स्वदोषस्यानवेक्षणात्}


\twolineshloka
{विद्याभ्यासैर्वृद्धयोगैरात्मानं विनयं नयेत्}
{विद्या धर्मार्थफलिनी तद्विदो वृद्धसंज्ञिताः}


\twolineshloka
{इन्द्रियाणां जयो देवि अत ऊर्ध्वमुदाहृतः}
{अजये सुमहान्दोषो राजानं विनिपातयेत्}


\threelineshloka
{पञ्चैव स्ववशे कृत्वा तदर्थान्पञ्च शोषयेत्}
{षडुत्सृज्य यथायोगं ज्ञानेन विनयेन च}
{सास्त्रचक्षुर्नयपरो भूत्वा भृत्यान्समाहरेत्}


\threelineshloka
{वृत्तश्रुतकुलोपेतानुपधाबिः परीक्षितान्}
{अमात्यानुपधातीतान्सोपसर्पाञ्जितेन्द्रियान्}
{योजयेत यथायोगं यथार्हं स्वेषु कर्मसु}


\twolineshloka
{अमात्या बुद्धिसम्पन्ना राष्ट्रं बहुजनप्रियम्}
{दुराधर्षं पुरश्रेष्ठं कोशः कृच्छ्रसहः स्मृतः}


\threelineshloka
{अनुरक्तं बलं साम्नामद्वैधं मन्त्रमेव च}
{एताः प्रकृतयः स्वेषु स्वामी विनयतत्ववित्}
{}


\twolineshloka
{प्रजानां रक्षणार्थाय सर्वमेतद्विनिर्मितम्}
{आभिः करणभूताभिः कुर्याल्लोकहितं नृपः}


\twolineshloka
{आत्मरक्षा नरेन्द्रस्य प्रजारक्षार्थमिष्यते}
{तस्मात्सततमात्मानं संरक्षेदप्रमादवान्}


\twolineshloka
{भोजनाच्छादनस्नानाद्बहिर्निष्क्रमणादपि}
{नित्यं स्त्रीगणसंयोगाद्रक्षेदात्मानमात्मवान्}


\twolineshloka
{स्वेभ्यश्चैव परेभ्यश्च शश्त्रादपि विषादपि}
{सततं पुत्रदारेभ्यो रक्षेदात्मानमात्मवान्}


\twolineshloka
{सर्वेभ्य एव स्थानेभ्यो रक्षेदात्मानमात्मवान्}
{प्रजानां रक्षणार्थाय प्रजाहितकरो भवेत्}


\fourlineindentedshloka
{प्रजाकार्यं तु तत्कार्यं प्रजासौख्यं तु तत्सुखम्}
{प्रजाप्रियं प्रियं तस्य स्वहितं तु प्रजाहितम्}
{प्रजार्तं तस्य सर्वस्वमात्मार्थं न विधीयते}
{}


\threelineshloka
{प्रकृतीनां हि रक्षार्थं रागद्वेषौ व्युदस्य च}
{उभयोः पक्षयोर्वादं श्रुत्वा चैव यथातथम्}
{तमर्थं विमृशेद्बुद्ध्या स्वयमातत्वदर्शनात्}


\twolineshloka
{तत्वविद्भिश्च बहुभिर्वृद्धैः सह नरोत्तमैः}
{कर्तारमपराधं च देशकालौ नयानयौ}


\twolineshloka
{ज्ञात्वा सम्यग्यथाशास्त्रं ततो दण्डं नयेन्नृषु}
{एवं कुर्वंल्लभेद्धर्मं पक्षपातविवर्जनात्}


\twolineshloka
{प्रत्यक्षाप्तोपदेशाभ्यामनुमानेन वा पुनः}
{बोद्धव्यं सततं राज्ञा देशवृत्तं शुभाशुभम्}


\twolineshloka
{चारैः कर्मप्रवृत्त्या च तद्विज्ञाय विचारयेत्}
{अशुभं निर्हरेत्सद्यो जोषयेच्छुभमात्मनः}


\twolineshloka
{गर्ह्यान्विगर्हयेदेव पूज्यान्सम्पूजयेत्तथा}
{दण्ड्यांश्च दण्डयेद्देवि नात्र कार्या विचारणा}


\twolineshloka
{पञ्चावेक्षन्सदा मन्त्रं कुर्याद्बुद्धियुतैर्नरैः}
{कुलवृत्तश्रुतोपेतैर्नित्यं मन्त्रपरो भवेत्}


\twolineshloka
{कामकारेण वै मुख्यैर्न च मन्त्रमना भवेत्}
{राजा राष्ट्रहितापेक्षं सत्यधर्माणि कारयेत्}


\twolineshloka
{सर्वोद्योगं स्वयं कुर्याद्दुर्गादिषु सदा नृषु}
{देशवृद्धिकरान्भृत्यानप्रमादेन कारयेत्}


\twolineshloka
{देशक्षयकरान्सर्वानप्रियांश्च विवर्जयेत्}
{अहन्यहनि सम्पश्येदनुजीविगणं स्वयम्}


\twolineshloka
{सुमुखः सुप्रियो दत्त्वा सम्यग्वृत्तं समाचरेत्}
{अधर्म्यं परुषं तीक्ष्णं वाक्यं वक्तुं न चार्हति}


\twolineshloka
{असंविश्वास्य वचनं वक्तुं सत्सु न चार्हति}
{नरेनरे गुणान्दोषान्सम्यग्वेदितुमर्हति}


\twolineshloka
{स्वेङ्गितं वृणुयाद्धैर्यं न कुर्यात्क्षुद्रसंविदम्}
{परेङ्गितज्ञो लोकेषु भूत्वा संसर्गमाचरेत्}


\twolineshloka
{स्वतश्च परतश्चैव परस्परभयादपि}
{अमानुषभयेभ्यश्च स्वाः प्रजाः पालयेन्नृपः}


\twolineshloka
{लुब्धाः कठोराश्चाप्यस्य मानवा दस्युवृत्तयः}
{निग्राह्या एव ते राज्ञा सङ्गृहीत्वा यतस्ततः}


\twolineshloka
{कुमारान्विनयोद्बोधैर्जन्मप्रभृति योजयेत्}
{तेषामात्मगुणोपेतं यौवराज्येन योजयेत्}


\twolineshloka
{प्रकृतीनां यथा न स्याद्राज्यभ्रंशो भवेद्भयम्}
{एतत्संचिन्तयेन्नित्यं तद्विधानं तथार्हति}


\twolineshloka
{अराजकं क्षणमपि राज्यं न स्याद्धि शोभने}
{आत्मनोऽनुविधानाय यौवराज्यं सदेष्यते}


\twolineshloka
{कुलजानां च वैद्यानां श्रोत्रियाणां तपस्विनाम्}
{अन्येषां वृत्तियुक्तानां विशेषं कर्तुमर्हति}


\twolineshloka
{आत्मार्थं राज्यतन्त्रार्थं कोशार्थं च समाचरेत्}
{दुर्गाद्राष्ट्रात्समुद्राच्च वणिग्भ्यः पुरुषात्ययात्}


\twolineshloka
{परात्मगुणसाराभ्यां भृत्यपोषणमाचरेत्}
{वाहनानां प्रकुर्वीत पोषणं योधकर्मसु}


\twolineshloka
{सादरः सततं भूत्वा अवेक्षाव्रतमाचरेत्}
{चतुर्दा विभजेत्कोशं धर्मभृत्यात्मकारणात्}


\twolineshloka
{आपदर्थं च नीतिज्ञो देशकालवशेन तु}
{अनाथान्व्यथितान्वृद्धान्स्वे देशे पोषयेन्नृपः}


\twolineshloka
{सन्धिं च विग्रहं चैव तद्विशेषांस्तथा पारन्}
{यथावत्संविमृश्यैव बुद्धिपूर्वं समाचरेत्}


\twolineshloka
{सर्वेषां सम्प्रियो भूत्वा मण्डलं सततं चरेत्}
{शुभेष्वपि च कार्येषु च चैकान्तः समाचरेत्}


\twolineshloka
{स्वतश्च परतश्चैव व्यसनानि विमृश्य सः}
{परेणि धार्मिकान्योगान्नातीयाद्द्वेषलोभतः}


\twolineshloka
{रक्ष्यत्वं वै प्रजाधर्मः क्षत्रधर्मस्तु रक्षणम्}
{कुनृपैः पीडितास्तस्मात्प्रजाः सर्वत्र पालयेत्}


\twolineshloka
{यात्राकालेऽनवेक्ष्यैव पश्चात्कोपफलोदयः}
{तद्युक्ताश्चापदश्चैव शासनादिति चिन्तयेत्}


\twolineshloka
{व्यसनेभ्यो बलं रक्षेन्नयतो व्ययतोपि वा}
{प्रायशो वर्जयेद्युद्धं प्राणरक्षणकारणात्}


\twolineshloka
{कारणादेवि योद्धव्यं नात्मनः परदोषतः}
{सुयुद्धे प्राणमोक्षश्च तस्य धर्माय इष्यते}


\twolineshloka
{अभियुक्तो बलवता कुर्यादापद्विधिं नृपः}
{अनुनीय तथा सर्वान्प्रजानां हितकारणात्}


\twolineshloka
{अन्यप्रकृतियुक्तानां राज्ञां वृत्तिविचारिणाम्}
{अन्यांश्चापत्प्रपन्नानां न तान्संयोक्तुमर्हति}


\threelineshloka
{शुभाशुभं यदा देवि व्रतं तूभयसाधनम्}
{आत्मैव तच्छुभं कुर्यादशुभं योजयेत्परान्}
{}


\twolineshloka
{एवमुद्देशतः प्रोक्तमलेपत्वं यथा भवेत्}
{एष देवि समासेन राजधर्मः प्रकीर्तितः}


\twolineshloka
{एवं संवर्तमानस्तु दण्डयन्भर्त्सयन्प्रजाः}
{निष्कल्मषमवाप्नोति पद्मपत्रमिवाम्भसा}


\twolineshloka
{एवं संवर्तमानस्य कालधर्मो यदा भवेत्}
{स्वर्गलोके तदा राजा त्रिदशैः सह तोष्यते}


\twolineshloka
{द्विविधं राज्यवृत्तं च न्यायभाग्यसमन्वितम्}
{एवं न्यायानुगं वृत्तं कथितं ते शुभेक्षणे}


\twolineshloka
{राज्यं न्यायानुगं तात बुद्धिशास्त्रानुगं भवेत्}
{धर्म्यं पथ्यं यशस्यं च स्वर्ग्यं चैव तथा भवेत्}


\threelineshloka
{राज्यं भाग्यानुगं नाम अयथावत्प्रदृश्यते}
{तत्तु शास्त्रविनिर्मुक्तं सतां कोपकरं भवेत्}
{अधर्म्यमयशस्यं च दुरन्तं च भवेद्ध्रुवम्}


\twolineshloka
{यत्र स्वच्छन्दतः सर्वं क्रियते कर्म राजभिः}
{तत्र भाग्यवशाद्भृत्या लभन्ते न विशेषतः}


\twolineshloka
{यत्र दण्ड्या न दण्ड्यन्ते पूज्यन्ते वा नराधमाः}
{यत्र सन्तोपि हन्यन्ते तत्र भाग्यानुगं भवेत्}


\twolineshloka
{शुभाशुभं यथा यत्र विपरीतं प्रदृश्यते}
{राज्ञि चासुरपक्षे तु तत्र भाग्यानुगं भवेत्}


\twolineshloka
{भाग्यानुगे तु राजानो वर्तमाना यथातथा}
{प्राप्याकीर्तिमनर्थं च इह लोके शुभेक्षणे}


\twolineshloka
{परत्र सुमहाघोरं तमः प्राप्य दुरत्ययम्}
{तिष्ठन्ति नरके देवि प्रलयान्तादिति स्थितिः}


\twolineshloka
{मोक्षं दुष्कृतिनां चापि विद्यते कालपर्ययात्}
{नास्त्येव मोक्षणं देवि राज्ञां दुष्कृतिकारिणाम्}


\twolineshloka
{एतत्सर्वं समासेन राजवृत्तं शुभाशुभम्}
{कथितं ते महाभागे भूयः श्रोतुं किमिच्छसि}


\chapter{अध्यायः २१३}
\threelineshloka
{देवदेव महादेव सर्वदेवनमस्कृत}
{यानि धर्मरहस्यानि श्रोतुमिच्छामि तान्यहम् ॥महेश्वर उवाच}
{}


\twolineshloka
{रहस्यं श्रूयतां देवि मानुषाणां सुखावहम्}
{नपुंसकेषु वन्ध्यासु वियोनौ पृथिवीतले}


\twolineshloka
{उत्सर्गो रेतसस्तेषु न कार्यो धर्मकाङ्क्षिभिः}
{एतेषु बीजं प्रक्षिप्तं न च रोहति वै प्रिये}


\twolineshloka
{यत्र वा तत्र वा बीजं धर्मार्थीं नोत्सृजेत्पुनः}
{नरो बीजविनाशेन लिप्यते ब्रह्महत्यया}


\twolineshloka
{अहिंसा परमो धर्म अहिंसा परमं सुखम्}
{अहिंसा धर्मशास्त्रेषु सर्वेषु परमं पदम्}


\twolineshloka
{देवतातिथिशुश्रूषा सततं धर्मशीलता}
{वेदाध्ययनयज्ञाश्च तपो दानं दमस्तथा}


\twolineshloka
{आचार्यगुरुशुश्रूषा तीर्थाभिगमनं तथा}
{अहिंसाया वरारोहे कलां नार्हन्ति षोडशीम्}


\twolineshloka
{एतत्ते परमं गुह्यमाख्यातं परमार्चितम् ॥उमोवाच}
{}


\twolineshloka
{यद्यधर्मस्तु हिंसायां किमर्थममरोत्तम}
{यज्ञेषु पशुबन्धेषु हन्यन्ते पशवो द्विजैः}


\twolineshloka
{कथं च भगवन्भूयो हिंसमाना नराधिपाः}
{स्वर्गं सुदुर्गमं यान्ति तदा स्म रिपुसूदन}


\twolineshloka
{यस्यैव गोसहस्राणि विंशतिः स्वादिकानि तु}
{अहन्यहनि हन्यन्ते द्विजानां मांसकारणात्}


\twolineshloka
{समांसं तु स दत्त्वाऽन्नं रन्तिदेवो नराधिपः}
{कथं स्वर्गमनुप्राप्तः परं कौतूहलं हि मे}


\twolineshloka
{किन्तु धर्मं न शृण्वन्ति न श्रद्दधति वा श्रुतम्}
{मृयां वै विनिर्गत्य मृगान्हन्ति नराधिपाः}


\threelineshloka
{एतत्सर्वं विशेषेण विस्तरेण वृषध्वज}
{श्रोतुमिच्छामि सर्वज्ञ तत्त्वमद्य ममोच्यताम् ॥ईश्वर उवाच}
{}


\twolineshloka
{बहुमान्यमिदं देवि नास्ति कश्चिदहिंसकः}
{श्रूयतां कारणं चात्र यथाऽनेकविधं भवेत्}


\twolineshloka
{दृश्यते चापि लोकेऽस्मिन्न हि कश्चिदहिंसकः}
{धरणीसंश्रिता देवि सुसूक्ष्मांश्चैव मध्यमान्}


\threelineshloka
{सञ्चरंश्चरणाभ्यां च हन्ति जीवाननेकशः}
{अज्ञानाज्ज्ञानतो वाऽपि यो जीवः शयनासनात्}
{उपाविशञ्शयानश्च हन्ति जीवाननेकशः}


\twolineshloka
{शिरोवस्त्रेषु ये जीवा नरणां स्वेदसम्भवाः}
{तांश्च हिंसन्ति सततं दंशांश्च मशकानपि}


\twolineshloka
{जले जीवास्तथाऽऽकाशे पृथिवी जीवमालिनी}
{एवं जीवाकुले लोके कोसौ स्याद्यस्त्वहिंसकः}


\twolineshloka
{स्थूलमध्यमसूक्ष्मैश्च स्वेदवारिमहीरुहैः}
{दृश्यरूपैरदृश्यैश्च नानारूपैश्च भामिनि}


\twolineshloka
{जीवैस्ततमिदं सर्वमाकाशं पृथिवी तथा}
{अन्योन्यं ते च हिंसन्ति दुर्बलान्बलवत्तराः}


\twolineshloka
{मत्स्या मत्स्यान्ग्रसन्तीह खगाश्चैव खगांस्तथा}
{सरीसृपैश्च जीवन्ति कपोताद्या विहङ्गमाः}


\twolineshloka
{भूचराः खेचराश्चान्ये क्रव्यादा मांसगृद्धिनः}
{समृद्धाः परमांसैस्तु भक्षेरंस्तेऽपि चापरैः}


\twolineshloka
{सत्वैः सत्वानि जीवन्ति शतशोथ सहस्रशः}
{अपीडयित्वा नैवान्यं जीवा जीवन्ति सुन्दरि}


\threelineshloka
{स्थूलकायस्य सत्वस्य खरस्य महिषस्य च}
{जीवस्यैकस्य मांसेन पयसा रुधिरेण वा}
{तृप्यन्ते बहवो जीवाः क्रव्यादा मांसजीविनः}


\twolineshloka
{एको जीवसहस्राणि सदा खादति मानवः}
{अन्नाद्यस्य च भोगेन दान्यसंज्ञानि यानि तु}


% Check verse!
मांसधान्यैः सबीजैश्च भोजनं परिवर्जयेत्
\twolineshloka
{त्रिरात्रं पञ्चरात्रं वा सप्तरात्रं तथाऽपि वा}
{धान्यानि यो न हिंसेताहिंसकः परिकीर्तितः}


\twolineshloka
{नाश्नाति यावतो जीवस्तावत्पुण्येन युज्यते}
{आहारस्य वियोगेन शरीरं परितप्यते}


\twolineshloka
{तप्यमाने शरीरे तु शरीरे चेन्द्रियाणि तु}
{वशे तिष्ठन्ति सुश्रोणि नृपाणामिव किंकराः}


\threelineshloka
{निरुणद्धीन्द्रियाण्येव स सुखी स विचक्षणः}
{इन्द्रियाणां निरोधेन दानेन च दमेन च}
{नरः सर्वमवाप्नोति मनसा यद्यधिच्छति}


% Check verse!
एवं मूलमर्हिसाया उपवासः प्रकीर्तितः
\twolineshloka
{आहारं कुरुते यस्तु भूमिमाक्रमते च यः}
{सर्वे ते हिंसका देवि यथा धर्मेषु दृश्यते}


\twolineshloka
{यथैवाहिंसको देवि तत्वतो ज्ञायते नरः}
{तथा ते सम्प्रवक्ष्यामि श्रूयतां धर्मचारिणि}


\twolineshloka
{फलानि मूलपर्णानि भस्म वा योपि भक्षयेत्}
{अलेख्यमिव निश्चेष्टं तं मन्येऽहमहिंसकम्}


\twolineshloka
{आरम्भा हिंसया युक्ता धूमेनाग्निरिवावृताः}
{तस्माद्यस्तु निराहारस्तं मन्येऽहमहिंसकम्}


\twolineshloka
{यस्तु सर्वं समुत्सृज्य दीक्षित्वा नियतः शुचिः}
{कृत्वा मण्डलमर्यादां सङ्कल्पं कुरुते नरः}


\twolineshloka
{यावज्जीवमनाशित्वा कालकाङ्क्षी दृढव्रतः}
{ध्यानेन तपसा युक्तस्तं मन्येऽहमहिंसकम्}


\twolineshloka
{अन्यथा हि न पश्यामि नरो यः स्यादहिंसकः}
{बहु चिन्त्यमिदं देवि नास्ति कश्चिदहिंसकः}


\twolineshloka
{यतो यतो महाभागे हिंसा स्यान्महती ततः}
{निवृत्तो मधुमांसाभ्यां हिंसा त्वल्पतरा भवेत्}


\twolineshloka
{निवृत्तिः परमो धर्मो निवृत्तिः परमं सुखम्}
{मनसा विनिवृत्तानां धर्मस्य निचयो महान्}


\twolineshloka
{मनःपूर्वागमा धर्मा अधर्माश्च न संशयः}
{मनसा बध्यते चापि मुच्यते चापि मानवः}


\twolineshloka
{निगृहीते भवेत्स्वर्गो विसृष्टे नरको ध्रुवः}
{घातकः शस्त्रमुद्यम्य मनसा चिन्तयेद्यदि}


\twolineshloka
{आयुःक्षयं गतेऽन्येषां मृते तु प्रहराम्यहम्}
{इति यो घातको हन्यान्न स पापेन लिप्यते}


\twolineshloka
{विधिना निहताः पूर्वं निमित्तं स तु घातकः}
{विधिर्हि बलवान्देवि दुस्त्यजं वै पुराकृतम्}


\twolineshloka
{जीवाः पुराकृतेनैव तिर्यग्योनिसरीसृपाः}
{नानायोनिषु जायन्ते स्वकर्मपरिवेष्टिताः}


\twolineshloka
{नानाविधविचित्राङ्गा नानाशौर्यपराक्रमाः}
{नानाभूमिप्रदेशेषु नानाहारश्च जन्तवः}


\twolineshloka
{जायमानस्य जीवस्य मृत्युः पूर्वं प्रजायते}
{सुखं वा यदि वा दुःखं यथापूर्वं कृतं तु वा}


\twolineshloka
{प्राप्नुवन्ति नरा मृत्युं यदा यत्र च येन च}
{नातिक्रान्तुं हि शक्यः स्यान्निदेशः पूर्वकर्मणः}


\twolineshloka
{अप्रमत्तः प्रमत्तेषु विधिर्जागर्ति जन्तुषु}
{न हि तस्य प्रियः कश्चिन्न द्वेष्यो न च मध्यमः}


\twolineshloka
{समः सर्वेषु भूतेषु कालः कालं निरीक्षते}
{गतायुषो ह्याक्षिपते जीवः सर्वस्य देहिनः}


\twolineshloka
{यथा येन च मर्तव्यं नान्यथा म्रियते हि सः}
{दृश्यते न च लोकेऽस्मिन्भूतो भव्यो द्विधा पुनः}


\twolineshloka
{विज्ञानैर्विक्रमैर्वाऽपि नानामन्त्रौषधैरपि}
{यो हि वञ्चयितुं शक्तो विधेस्तु नियतां गतिम्}


% Check verse!
एष तेऽभिहितो देवि जीवहिंसाविधिक्रमः
\chapter{अध्यायः २१४}
\twolineshloka
{श्रूयतां कारणं देवि यथा हि दुरतिक्रमः}
{विधिः सर्वेषु भूतेषु मर्तव्ये समुपस्थिते}


\threelineshloka
{आयुःक्षयेणोपहिताः समागम्य वरानने}
{कीटाः पतङ्गा बहवः स्थूलाः सूक्ष्माश्च मध्यमाः}
{7-214-2dcप्रज्वलत्सु प्रदीपेषु स्वयमेव पतन्ति ते}


\twolineshloka
{बहूनां मृगयूथानां नानावननिषेविणाम्}
{यस्तु कालं गतस्तेषां स वै पाशेन बध्यते}


\twolineshloka
{सूनार्थं देवि बद्धानां क्षीणायुर्यो निबध्यते}
{अवशो घातकस्याथ हस्तं तदहरेति सः}


\twolineshloka
{यथा पक्षिगणाः क्षिप्रं विस्तीर्णाकाशगामिनः}
{क्षीणायुषो निबध्यन्ते शक्ता अपि पलायितुम्}


\twolineshloka
{यथा वारिचरा मीना बहवोऽम्बुजजातयः}
{जालं समधिरोहन्ति स्वयमेव विधेर्वशात्}


\twolineshloka
{शल्यकस्य च जिह्वाग्रं स्वयमारुह्य शोभने}
{आयुःक्षयेणोपहता निबध्यन्ते सरीसृपाः}


\twolineshloka
{कृषतां कर्षकाणां च नास्ति बुद्धिर्विहिंसने}
{अथैषां लाङ्गलाग्राद्यैर्हन्यन्ते जन्तवोऽक्षयाः}


\twolineshloka
{पादाग्रेणैव चैकेन यां हिंसां कुरुते नरः}
{मातङ्गोपि न तां कुर्यात्कूरो जन्मशतैरपि}


\twolineshloka
{म्रियन्ते यैर्हि मर्तव्यं न तान्घ्नन्ति कृषीवलाः}
{कृषामीति मनस्तस्य नास्ति चिन्ता विहिंसने}


\threelineshloka
{तस्माज्जीवसहस्राणि हत्वाऽपि न स लिप्यते}
{विधिना स हतः पूर्वं पश्चात्प्राणि विपद्यते}
{एवं सर्वेषु भूतेषु विधिर्हि दुरतिक्रमः}


\twolineshloka
{गतायुषा मुहूर्तं तु न शक्यमुपजीवितुम्}
{जीवितव्ये न मर्तव्यं न भूतं न भविष्यति}


\twolineshloka
{शुभाशुभं कर्मफलं न शक्यमतिवर्तितुम्}
{तथा ताभिश्च मर्तव्यं मोक्तव्याश्चैव तास्तथा}


\twolineshloka
{रन्तिदेवस्य गावो वै विधेर्हि वशमागताः}
{स्वयमायान्ति गावो वै हन्यन्ते यत्र सुन्दरि}


\twolineshloka
{गवां वै हन्यमानानां रुधिरप्रभवा नदी}
{चर्मण्वतीति विख्याता खुरशृङ्गास्थिदुर्गमा}


\twolineshloka
{रुधिरं तां नदीं प्राप्य तोयं भवति शोभने}
{मेध्यं पुण्यं पवित्रं च गन्धवर्णरसैर्युतम्}


\twolineshloka
{तत्राऽभिषेकं कुर्वन्ति कृतजप्याः कृताह्निकाः}
{द्विजा देवगणाश्चापि लोकपाला महेश्वराः}


\threelineshloka
{तस्य राज्ञः सदा सत्रे स्वयमागम्य सुन्दरि}
{विधिना पूर्वदृष्टेन तन्मांसमुपकल्पितम्}
{मन्त्रवत्प्रतिगृह्णन्ति यतान्यायं यताविधि}


\twolineshloka
{समांसं च सदा ह्यन्नं शतशोऽथ सहस्रशः}
{भुञ्जानानां द्विजातीनामस्तमेति दिवाकरः}


\twolineshloka
{गावो यास्तत्र हन्यन्ते राज्ञस्तस्य क्रतूत्तमे}
{पठ्यमानेषु मन्त्रेषु यथान्यायं यथाविधि}


\twolineshloka
{ताश्च स्वर्गं गता गावो रन्तिदेवश्च पार्थिवः}
{सदा सत्रविधानेन सिद्धिं प्राप्तो नरोत्तमः}


\twolineshloka
{अथ यस्तु सहायार्थमुक्तः स्यात्पार्थिवैर्नरैः}
{भोगानां संविभागेन वस्त्राभरणभूषणैः}


\twolineshloka
{सहभोजनसम्बद्धैः सत्कारैर्विविधैरपि}
{सहायकाले सम्प्राप्ते सङ्ग्रामे शस्त्रमुद्धरेत्}


\threelineshloka
{व्यूढानीके यथा सास्त्रं सेनयोरुभयोरपि}
{हस्त्यश्वरथसम्पूर्णे पदातिबलसङ्कुले}
{चामरच्छत्रशबले ध्वजचर्मायुधोज्ज्वले}


\twolineshloka
{शक्तितोमरकुन्तासिशूलमुद्गरपाणिभिः}
{कूटमुद्गरचापेषु मुसुण्ठीजुष्टमुष्टिभिः}


\threelineshloka
{भिण्डिपालगदाचक्रप्रासकर्पटधारिभिः}
{नानाप्रहरणैर्योधैः सेनयोरुभयोरपि}
{युद्धशौण्डैः प्रगर्जद्भिर्वृषेषु वृषभैरिव}


\twolineshloka
{शङ्खदुन्दुभिनादेन नानातूर्यरवेण च}
{हयहेषितशब्देन कुञ्जराणां तु बृंहितैः}


\twolineshloka
{योधानां सिंहनादैश्च घण्टानां शिञ्जितस्वनैः}
{दिशश्च विदिशश्चैव समन्ताद्बधिरीकृताः}


\threelineshloka
{ग्रीष्मान्तेष्विव गर्जद्भिर्नभशीव बलाहकैः}
{रथनेमिखुरोद्धूतैररुणै रणरेणुभिः}
{कपिलाभिरिवाकाशे छाद्यमाने समन्ततः}


\twolineshloka
{प्रवृत्ते शस्त्रसम्पाते योधानां तत्र सेनयोः}
{तेषां प्रहारक्षतजं रक्तचन्दनसन्निभम्}


\twolineshloka
{तेषामस्राणि गात्रेभ्य स्रवन्ते रणमूर्धनि}
{पलाशाशोकपुष्पाणां जङ्गमा इव राशयः}


\twolineshloka
{रणे समभिवर्तन्त उद्यतायुधपाणयः}
{शोभमाना रणे शूरा आह्वयन्तः परस्परम्}


\twolineshloka
{हन्यमानेष्वभिघ्नत्सु शूरेषु रणसङ्कटे}
{पृष्ठं दत्त्वा च ये तत्र नायकस्य नराधमाः}


\threelineshloka
{अनाहता निवर्तन्ते नायके चाप्यनीप्सति}
{ते दुष्कृतं प्रपद्यन्ते नायकस्याखिलं नराः}
{यच्चास्ति सुकृतं तेषां युज्यते तेन नायकः}


\threelineshloka
{अहिंसा परमो धर्म इति येऽपि नरा विदुः}
{सङ्ग्रामेषु न युध्यन्ते भृत्याश्चैवानुरूपतः}
{नरकं यान्ति ते घोरं भर्तृपिण्डापहारिणइः}


\threelineshloka
{यस्तु प्राणान्परित्यज्य प्रविशेदुद्यतायुधः}
{सङ्ग्राममग्निप्रतिमं पतह्ग इव निर्भयः}
{स्वर्गमाविशते प्रेत्य ज्ञात्वा योधस्य निश्चयम्}


\threelineshloka
{आविष्टश्चैव सत्त्वेन सघृणो जायते नरः}
{प्रहारैर्नन्दयेद्देवि सत्वेनाधिष्ठितो हि सः}
{प्रहारव्यथितश्चैव न वैक्लब्यमुपैति सः}


\threelineshloka
{यस्तु स्वं नायकं रक्षेदतिघोरे रणाङ्कणे}
{तापयन्नरिसैन्यानि सिंहो मृगगणानिव}
{आदित्य इव मध्याह्ने दुर्निरीक्ष्यो रणाजिरे}


\twolineshloka
{निर्दयो यस्तु सङ्ग्रामे प्रहरन्नुद्यतायुधः}
{यजते स तु पूतात्मा सङ्ग्रामेण महाक्रतुम्}


\twolineshloka
{चर्म कृष्णाजिनं तस्य दन्तकाष्ठं धनुः स्मृतम्}
{रथो वेदिर्ध्वजो यूपः कुशाश्च रथरश्मयः}


\twolineshloka
{मानो दर्पस्त्वहङ्कारस्त्रयस्त्रेताग्नयः स्मृताः}
{प्रमोदस्च स्रुवस्तस्य उपाध्यायो हि सारथिः}


\twolineshloka
{स्रुग्भाण्डं चापि यत्किञ्चिद्यज्ञोपकरणानि च}
{आयुधान्यस्य तत्सर्वं समिधः सायकाः स्मृताः}


\threelineshloka
{स्वेदस्रवश्च गात्रेभ्यः क्षौद्रं तस्य यशस्विनः}
{पुरोडाशा नृशीर्षाणि रुधिरं चाहुतिः स्मृता}
{तूणाश्चैव चरुर्ज्ञेया वसोर्धारा वसाः स्मृताः}


\threelineshloka
{क्रव्यादा भूतसङ्घाश्च तस्मिन्यज्ञे द्विजतयः}
{तेषां भक्षान्नपानानि हता नृगजवाजिनः}
{भुञ्जते ते यथाकामं यता यज्ञे किमिच्छति}


\twolineshloka
{निहतानां तु योधानां वस्त्राभरणभूषणम्}
{हिरण्यं च सुवर्णं च यद्वै यज्ञस्य दक्षिणा}


\twolineshloka
{यस्तत्र हन्यते देवि गजस्कन्धगतो नरः}
{ब्रह्मलोकमवाप्नोति रणेष्वभिमुखो हतः}


\twolineshloka
{रथमध्यगतो वाऽपि हयपृष्ठगतोपि वा}
{हन्यते यस्तु सङ्ग्रामे शक्रलोके महीयते}


\twolineshloka
{स्वर्गे हताः प्रपूज्यते हन्ता त्वत्रैव पूज्यते}
{द्वावेतौ सुखमेधेते हन्ता यश्चैव हन्यते}


% Check verse!
तस्मात्सङ्ग्राममासाद्य प्रहर्तव्यमभीतवत्
\twolineshloka
{निर्भयो यस्तु सङ्ग्रामे यस्तु सङ्ग्रामे प्रहरेदुद्यतायुधः}
{यथा नदीसहस्राणि प्रविष्टानि महोदधिम्}


\twolineshloka
{तथा सर्वे न सन्देहो धर्मा धर्मभृतांवरम्}
{प्रविष्टा राजधर्मेण आचारविनयस्तथा}


\twolineshloka
{वेदोक्ताश्चैव ये धर्माः पाषण्डेषु च कीर्तिताः}
{तथैव मानुषा धर्मा धर्माश्चान्ये तथेतरे}


\threelineshloka
{देशजातिकुलानां च ग्रामधर्मास्तथैव च}
{ये धर्माः पार्वतीयेषु ये धर्माः पत्तनादिषु}
{तेषां पूर्वप्रवृत्तानां कर्तव्यं परिरक्षणम्}


\twolineshloka
{धर्म एव हतो हन्ति धर्मो रक्षति रक्षितः}
{तस्माद्धर्मो न हन्तव्यः पार्थिवेन विशेषतः}


\twolineshloka
{प्रजाः पालयते यत्र धर्मेण वसुधाधिपः}
{षट्कर्मनिरता विप्राः पूज्यन्ते पितृदेवताः}


\twolineshloka
{नैव तस्मिन्ननावृष्टिर्न रोगा नाप्युपद्रवाः}
{धर्मशीलाः प्रजाः सर्वाः स्वधर्मनिरते नृपे}


\twolineshloka
{एष्टव्यः सततं देवि युक्ताचारो नराधिपः}
{छिद्रज्ञश्चैव शत्रूणामप्रमत्तः प्रतापवान्}


\twolineshloka
{शूद्राः पृथिव्यां बहवो राज्ञां बहुविनाशकाः}
{तस्मात्प्रमादं सुश्रोणि न कुर्यात्पण्डितो नृपः}


\twolineshloka
{तेषु मित्रेषु त्यक्तेषु तथा मर्त्येषु हस्तिषु}
{विस्रम्भो नोपगन्तव्यः स्नानपानेषु नित्यशः}


\threelineshloka
{राज्ञो वल्लभतामेति कुलं भावयते स्वकम्}
{यस्तु राष्ट्रहितार्थाय गोब्राह्मणकृते तथा}
{बन्दीग्रहाय मित्रार्थे प्राणांस्त्यजति दुस्त्यजान्}


\twolineshloka
{सर्वकामदुघां धेनुं धरणीं लोकधारिणीम्}
{समुद्रान्तां वरारोहे सशैलवनकाननाम्}


\twolineshloka
{दद्याद्देवि द्विजातिभ्यो वसुपूर्णां वसुन्धराम्}
{न तत्समं वरारोहे प्राणत्यागी विशिष्यते}


\twolineshloka
{सहस्रमपि यज्ञानां यजते च यतर्द्धिमान्}
{यज्ञैस्तस्य किमाश्चर्यं प्राणत्यागः सुदुष्करः}


\twolineshloka
{तस्मात्सर्वेषु यज्ञेषु प्राणयज्ञो विशिष्यते}
{एवं सङ्ग्रामयज्ञास्ते यथार्थं समुदाहृताः}


\chapter{अध्यायः २१५}
\twolineshloka
{मृगयात्रां तु वक्ष्यामि शृणु तां धर्मिचारिणि}
{मृगान्हत्वा महीपालो यथा पापैर्न लिप्यते}


\twolineshloka
{निर्मानुषामिमां सर्वे मृगा इच्छन्ति मेदिनीम्}
{भक्षयन्ति च सस्यानि शासितव्या नृपेण ते}


\threelineshloka
{दुष्टानां शासनं धर्मः शिष्टानां परिपालनम्}
{कर्तव्यं भूमिपालेन नित्यं कार्येषु चार्जवम्}
{स्वर्गं मृगाश्च गच्छन्ति स्वयं नृपतिना हताः}


\threelineshloka
{यथा गावो ह्यगोपालास्तथा राष्ट्रमनायकम्}
{तस्मादंशास्तु देवानां गन्धर्वोरगरक्षसाम्}
{राज्ये नियुक्ता राष्ट्रेषु प्रजापालनकारणात्}


\twolineshloka
{अशिष्टशासने चैव शिष्टानां परिपालने}
{तेषां चर्यां प्रवक्ष्यामि श्रूयतामनुपूर्वशः}


\twolineshloka
{यथा प्रचरतां तेषां पार्थिवानां यशस्विनाम्}
{राष्ट्रं धर्मो धनं चैव यशः कीर्तिश्च वर्धते}


\threelineshloka
{नृपाणां पूर्वमेवायं धर्मो धर्मभृतांवर}
{सभाप्रपातटाकानि देवतायतनानि च}
{ब्राह्मणावसथाश्चैव कर्तव्या नृपसत्तमैः}


\twolineshloka
{ब्राह्मणा नावमन्तव्या भस्मच्छन्ना इवाग्नयः}
{कुलमुत्सादयेयुस्ते क्रोधाविष्टा द्विजातयः}


\threelineshloka
{ध्मायमानो यता ह्यग्निर्निर्दहेत्सर्वमिन्धनम्}
{तथा क्रोधाग्निना विप्रा दहेयुः पृथिवीमिमाम्}
{न हि विप्रेषु क्रुद्धेषु राज्यं भुञ्जन्ति भूमिपाः}


\threelineshloka
{परिभूय द्विजान्मोहाद्वातापिनहुषादयः}
{सबन्धुमित्रा नष्टास्ते दग्धा ब्राह्मणमन्युभिः}
{शरीरं चापि शक्रस्य कृतं भगनिरन्तरम्}


\twolineshloka
{ततो देवगणाः सर्वे इन्द्रस्यार्थे महामुनिम्}
{प्रसादं कारयामासुः प्रणासस्तुतिवन्दनैः}


\threelineshloka
{तेन प्रीतेन सुश्रोणि गौतमेन महात्मना}
{तच्छरीरं तु शक्रस्य सहस्रभगचिह्नितम्}
{कृतं नेत्रसहस्रेण क्षणेनैव निरन्तरम्}


\twolineshloka
{छित्त्वा मेषस्य वृषणौ गौतमेनाभिमन्त्रितौ}
{इन्द्रस्य वृषणौ भूत्वा क्षिप्रं वै श्लेषमागतौ}


\twolineshloka
{एवं विप्रेषु क्रुद्धेषु देवराजः शतक्रतुः}
{अशक्तः शासितुं राज्यं किंपुनर्मानुषा भुवि}


\twolineshloka
{क्रोधाविष्टो दहेद्विप्रः शुष्केन्धनमिवानलः}
{भस्मीकृत्य जगत्सर्वं सृजेदन्यज्जगत्पुनः}


\twolineshloka
{अदेवानपि देवान्स कुर्याद्देवानदेवताः}
{तस्मान्नोत्पादयेन्मन्युं मन्युप्रहरणा द्विजाः}


\threelineshloka
{महत्स्वप्यपराधेषु शासनं नार्हति द्विजः}
{न च शस्त्रनिपातानि न च प्राणैर्वियोजनम्}
{दृश्यते त्रिषु लोकेषु ब्राह्ममानामनिन्दिते}


\twolineshloka
{क्रोधाश्च विपुला घोराः प्रसादाश्चाप्यनुत्तमाः}
{तस्मान्नोत्पादयेत्क्रोधं नित्यं पूज्या द्विजातयः}


\twolineshloka
{दृश्यते न स लोकेऽस्मिन्भूते वाऽथ भविष्यति}
{क्रुद्धेषु यो वै विप्रेषु राज्यं भुङ्क्ते नराधिपः}


\twolineshloka
{न चैवापहसेद्विप्रान्नि चैवोपालभेच्च तान्}
{कालमासाद्य कुप्येच्च काले कुर्यादनुग्रहम्}


\twolineshloka
{सम्प्रहासश्च भृत्येषु न कर्तव्यो नराधिपैः}
{लघुत्वं चैव प्राप्नोति आज्ञा चास्य निवर्तते}


\twolineshloka
{भृत्यानां सम्प्रहासेन पार्थिवः परिभूयते}
{अयाच्यानि च याचन्ति अवक्तव्यं ब्रुवन्ति च}


\twolineshloka
{पूर्वमप्यर्पितैर्लोभैः परितोषं न यान्ति ते}
{तस्माद्भृत्येषु नृपतिः सम्प्रहासं विवर्जयेत्}


\twolineshloka
{न विश्वसेदविश्वस्ते विश्वस्ते नातिविश्वसेत्}
{सगोत्रेषु विशेषेण सर्वोपायैर्न विश्वसेत्}


\threelineshloka
{विश्वासाद्भयमुत्पन्नं हन्याद्वृक्षमिवाशनिः}
{प्रमादाद्धन्यते राजा लोभेन च वशीकृतः}
{तस्मात्प्रमादं लोभं च न च कुर्यान्न विश्वसेत्}


\twolineshloka
{भयार्तानां भयत्राता दीनानुग्रहकारणात्}
{कार्याकार्यविशेषज्ञो नित्यं राष्ट्रहिते रतः}


\twolineshloka
{सत्यसन्धः स्थितो राज्ये प्रजापालनतत्परः}
{अलुब्धो न्यायवादी च षड्भागमुपजीवति}


\twolineshloka
{कार्याकार्यविशेषज्ञः सर्वं धर्मेण पश्यति}
{स्वराष्ट्रेषु दयां कुर्यादकार्ये न प्रवर्तते}


\threelineshloka
{ये चैवैनं प्रशंसन्ति ये च निन्दन्ति मानवाः}
{शत्रुं च मित्रवत्पश्येदपराधविवर्जितम्}
{}


\threelineshloka
{अपराधानुरूपेण दुष्टं दण्डेन शासयेत्}
{धर्मः प्रवर्तते तत्र यत्र दण्डरुचिर्नृपः}
{न धर्मो विद्यते तत्र यत्र राजा क्षमान्वितः}


\twolineshloka
{अशिष्टशासनं धर्मः शिष्टानां परिपालनम्}
{वध्यांश्च घातयेद्यस्तु अवध्यानपरिरक्षति}


\threelineshloka
{अवध्या ब्राह्मणा गावो दूताश्चैव पिता तथा}
{विद्यां ग्राहयते यश्च ये च पूर्वोपकारिणः}
{स्त्रियश्चैव न हन्तव्या यच्च सर्वातिथिर्नरः}


\twolineshloka
{धरणीं गां हिरण्यं च सिद्धान्नं च तिलान्घृतम्}
{ददन्नित्यं द्विजातिभ्यो मुच्यते राजकिल्बिषात्}


\threelineshloka
{एवं चरति यो नित्यं राजा राष्ट्रहिते रतः}
{तस्य राष्ट्रं धनं धर्मो यशः कीर्तिश्च वर्धते}
{न च पापैर्न चानर्थैर्युज्यते स नराधिपः}


\twolineshloka
{षड्भागमुपभुञ्जानः प्रजा राजा न रक्षति}
{स्वचक्रपरचक्राभ्यां धर्मैर्वा विक्रमेण वा}


\twolineshloka
{निरुद्योगो नृपो यश्च परराष्ट्रनिघातने}
{स्वराष्ट्रं निष्प्रतापस्य परचक्रेण हन्यते}


\twolineshloka
{यत्पापं सकलं राजा हतराष्ट्रः प्रपद्यते ॥मातुलं भागिनेयं वा मातरं श्वशुरं गुरुम्}
{}


% Check verse!
पितरं वर्जयित्वैकं हन्याद्धातकमागतम्
\twolineshloka
{स्वस्य राष्ट्रस्य रक्षार्थं युद्यमानश्च यो हतः}
{सङ्ग्रामे परचक्रेण श्रूयतां तस्य या गतिः}


\twolineshloka
{विमानेन वरारोहे अप्सरोगणसेवितः}
{शक्रलोकमितो याति सङ्ग्रामे निहतो नृपः}


\twolineshloka
{यावन्तो रोमकूपाः स्युस्तस्य गात्रेषु सुन्दरि}
{तावद्वर्षसहस्राणि शक्रलोके महीयते}


\twolineshloka
{यदि वै मानुषे लोके कदाचिदुपपद्यते}
{राजा वा राजमात्रो वा भूयो भवति वीर्यवान्}


\twolineshloka
{तस्माद्यत्नेन कर्तव्यं स्वराष्ट्रपरिपालनम्}
{व्यवहाराश्च चारश्च सततं सत्यसन्धता}


\twolineshloka
{अप्रमादः प्रमोदश्च व्यवसायेऽप्यचण्डता}
{भरणं चैव भृत्यानां वाहनानां च पोषणम्}


\twolineshloka
{योधानां चैव सत्कारः कृते कर्मण्यमोघता}
{श्रेय एव नरेन्द्राणामिह चैव परत्र च}


\chapter{अध्यायः २१६}
\threelineshloka
{पशवः पशुबन्धेषु ये हन्यन्तेऽध्वरेषु च}
{यूपे निबध्य मन्त्रैश्च यथान्यायं यथाविधि}
{मन्त्राहुतिविपूतास्ते स्वर्गं यान्ति यशस्विनि}


\threelineshloka
{तर्पिता यज्ञभागेषु तेषां मांसैर्वरानने}
{अग्नयस्त्रिदशाश्चैव लोकपाला महेश्वराः}
{}


\twolineshloka
{तेषु तुष्टेषु जायेत यस्य यज्ञस्य यत्फलम्}
{तेन संयुज्यते देवि यजमानो न संशयः}


\twolineshloka
{सपत्नीकः सपुत्रश्चि पित्रा च भ्रातृभिः सह}
{ये तत्र दीक्षिता देवि सर्वे स्वर्गं प्रयान्ति ते}


\twolineshloka
{एतत्ते सर्वमाख्यातं किं भूयः श्रोतुमिच्छसि ॥उमोवाच}
{}


\twolineshloka
{भगवन्सर्वभूतेश शूलपाणे महाद्युते}
{श्रोतुमिच्छाम्यहं वृत्तं सर्वेषां गृहमेधिनाम्}


\twolineshloka
{कीदृशं चरितं तेषां त्रिवर्गसहितं प्रभो}
{प्रत्यायतिः कथं तेषां जीवनार्थमुदाहृतम्}


\threelineshloka
{वर्तमानाः कथं सर्वे प्राप्नुवन्त्युत्तमां गतिम्}
{एतत्सर्वं समासेन वक्तुमर्हसि मानदः ॥महेश्वर उवाच}
{}


\twolineshloka
{न्यायतस्त्वं महाभागे श्रोतुकामाऽसि भामिनि}
{प्रायशो लोकसद्वृत्तमिष्यते गृहवासिनाम्}


\twolineshloka
{तेषां संरक्षणार्थाय राजानः संस्कृता भुवि}
{सर्वेषामथ मर्त्यानां वृत्तिं सामान्यतः शृणुः}


\twolineshloka
{विद्या वार्ता च सेवा च कारुत्वं नाट्यता तथा}
{इत्यते जीवनार्थाय मर्त्यानां विहिताः प्रिये}


\twolineshloka
{अपि जन्मफलं तावन्मानुषाणां विशेषतः}
{विहितं तत्स्ववृत्तेन तन्मे शृणु समाहिताः}


\twolineshloka
{कर्मक्षेत्रं हि मानुष्यं सुखदुःखयुताः परे}
{सर्वेषां प्राणिनां तस्मान्मानुष्ये वृत्तिरिष्यते}


\twolineshloka
{विद्यायोगस्तु सर्वेषां पूर्वमेव विधीयते}
{कार्याकार्यं विजानन्ति विद्यया देवि नान्यथा}


\twolineshloka
{विद्यया स्फीयते ज्ञानं ज्ञानात्तत्वनिदर्शनम्}
{दृष्टतत्वो विनीतात्मा सर्वार्थस्य च भाजनम्}


% Check verse!
शक्यं विद्याविनीतेन लोके संजीवनं शुभम्
\threelineshloka
{आत्मानं विद्यया तस्मात्पूर्वं वृत्वा तु भाजनम्}
{वश्येन्द्रियो जितक्रोधो भूतात्मानं तु भावयेत्}
{भावयित्वा तदाऽऽत्मानं पूजनीयः सतामपि}


\twolineshloka
{कुलानुवृत्तं वृत्तं वा पूर्वमात्मा समाश्रयेत्}
{इत्येतत्कुलवासाय दानकर्म यथा पुरा}


\threelineshloka
{यदि चेद्विद्यया चैव वृत्तिं काङ्क्षेदथात्मनः}
{राजविद्यानुवादेऽपि लोकविद्यामथापि वा}
{तीर्थतश्चापि गृह्णीयाच्छुश्रूषादिगुणैर्युतः}


\threelineshloka
{ग्रन्थतश्चार्थतश्चैव दृढं कुर्यात्प्रयत्नतः}
{एवं विद्याफलं देवि प्राप्नुयान्नान्यथा नरः}
{न्यायाद्विद्याफलानीच्छेदधर्मं तत्र वर्जयेत्}


\twolineshloka
{यदिच्छेद्वार्तया वृत्तिं काङ्क्षेत विधिपूर्वकम्}
{क्षेत्रे जलोपपन्ने च तद्योग्यां कृषिमाचरेत्}


\twolineshloka
{वाणिज्यं वा यथाकालं कुर्यात्तद्देशयोगतः}
{मूल्यमर्थं प्रयासं च विचार्यैव व्ययोदयौ}


\twolineshloka
{पशुसंजीवनं चैव दश गाः पोषयेद्ध्रुवम्}
{बहुप्रकारा बहवः पशवस्तस्य साधकाः}


\twolineshloka
{यः कश्चित्सेवया वृत्तिं काङ्क्षेत मतिमान्नरः}
{यतात्मा श्रवणीयानां भवेद्वै सम्प्रयोजकः}


\twolineshloka
{बुद्ध्या वा कर्मयोगाद्वा बोधनाद्वा समाश्रयेत्}
{मार्गतस्तु समाश्रित्य तदा तत्सम्प्रयोजयेत्}


\twolineshloka
{यथायथा सु तुष्येत तथा संतोषयेत्तु तम्}
{अनुजीविगुणोपेतः कुर्यादात्मार्थमाश्रितम्}


\twolineshloka
{विप्रियं नाचरेत्तस्य एषा सेवा समासतः}
{विप्रयोगात्पुरा तेन गतिमन्यां न लक्षयेत्}


\twolineshloka
{कारुकर्म च नाट्यं च प्रायशो नीचयोनिषु}
{तयोरपि यथायोगं न्यायतः कर्मवेतनम्}


\twolineshloka
{आजीवेभ्योऽपि सर्वभ्यः स्वार्जवाद्वेतनं हरेत्}
{अनार्जवादाहरतस्तत्तु पापाय कल्पते}


\threelineshloka
{सर्वेषां पूर्वमारम्भांश्चिन्तयेन्नयपूर्वकम्}
{आत्मशक्तिमुपायांश्च देशकालौ च युक्तितः}
{कारणानि प्रयासं च प्रक्षेपं च फलोदयम्}


\twolineshloka
{एवमादीनि सञ्चिन्त्य दृष्ट्वा दैवानुकूलताम्}
{अतः परं समारम्भेद्यत्रात्महितमाहितम्}


\twolineshloka
{वृत्तिमेव समासाद्य तां सदा परिपालयेत्}
{देवमानुषविघ्नेभ्यो न पुनर्मन्यते यथा}


\twolineshloka
{पालयन्वर्धयन्भुञ्जंस्तां प्राप्य न विनाशयेत्}
{क्षीयते गिरिसङ्काशमश्नतो ह्यनपेक्षया}


\twolineshloka
{आजीवेभ्यो धनं प्राप्य चतुर्धा विभजेद्बुधः}
{धर्मायार्थाय कामाय आपत्प्रशमनाय च}


% Check verse!
चतुर्ष्वपि विभागेषु विधानं शृणु शोभने
\twolineshloka
{यज्ञार्थं चान्नदानार्थं दीनानुग्रहकारणात्}
{देवब्राह्मणपूजार्थं पितृपूजार्थमेव च}


\twolineshloka
{मूलार्थं सन्निवासार्थं क्रियानित्यैश्चि धार्मिकैः}
{एवमादिषु चान्येषु धर्मार्थं संत्यजेद्धनम्}


\twolineshloka
{धर्मकार्ये धनं दद्यादनवेक्ष्य फलोदयम्}
{ऐश्वर्यस्थानलाभार्थं राजवाल्लभ्यकारणात्}


\twolineshloka
{वार्तायां च समारम्भेऽमात्यमित्रपरिग्रहे}
{आवाहे च विवाहे च पुत्राणां वृत्तिकारणात्}


\twolineshloka
{अर्थोदयसमावाप्तावनर्थस्य विघातने}
{एवमादिषु चान्येषु अर्थार्थं विसृजेद्धनम्}


\twolineshloka
{अनुबन्धं हेतुयुक्तं दृष्ट्वा वित्तं परित्यजेत्}
{अनर्थं बाधते ह्यर्थो अर्तं चैव फलान्युत}


\twolineshloka
{नाधनाः प्राप्नुन्त्यर्थं नरा यत्नशतैरपि}
{तस्माद्धनं रक्षितव्यं दातव्यं च विधानतः}


\twolineshloka
{शरीरपोषणार्थाय आहारस्य विशोषणे}
{नट*****धर्वसंयोगे कामयात्राविहारयोः}


\twolineshloka
{मनःप्रियाणां संयोगे प्रीतिदाने तथैव च}
{एवमादिषु चान्येषु कामार्तं विसृजेद्धनम्}


\twolineshloka
{विचार्य गुणदोषांस्तु त्रयाणां तत्र संत्यजेत्}
{चतुर्थं सन्निदध्याच्च आपदर्थं शुचिस्मिते}


\twolineshloka
{राज्यभ्रंशविनाशार्थं दुर्भिक्षार्थं च शोभने}
{महाव्याधिविमोक्षार्थं वार्धकस्यैव कारणात्}


\threelineshloka
{शत्रूणां प्रतिकाराय साहसैश्चाप्यमर्षणात्}
{प्रस्थाने चान्यदेशार्थमापदां विप्रमोक्षणे}
{एवमादि समुद्दिश्य सन्निदध्यात्स्वकं धनम्}


\twolineshloka
{सुखमर्थवतां लोके कृच्छ्राणां विप्रमोक्षणम्}
{यस्य नास्ति धनं किञ्चित्तस्य लोकद्वयं न च}


\threelineshloka
{अशनादिन्द्रियाणीव सर्वमर्थात्प्रवर्तते}
{निधानमात्रं यस्तेषामन्यथा विलयं व्रजेत्}
{एवं देवि मनुष्याणां लोकानां जीवनं प्रति}


\twolineshloka
{एवं लोकस्य वृत्तस्य लोकवृत्तं पुनः शृणु}
{धन्यं यशस्यमायुष्यं स्वर्ग्यं च परमं यशः}


\twolineshloka
{त्रिवर्गो हि वशे युक्तः सर्वेषां संविधीयते}
{तथा संवर्तमानास्तु लोकयोर्हितमाप्नुयुः}


\twolineshloka
{काल्योत्थानं च शौचं च देवब्राह्मणभक्तितः}
{गुरुणामेव शुश्रूषा ब्राह्मणेष्वभिवादनम्}


\twolineshloka
{प्रत्युत्थानं च वृद्धानां देवस्थानप्रणामनम्}
{आभिमुख्यं पुरस्कृत्य अतिथीनां च पूजनम्}


\twolineshloka
{वृद्धोपदेशकरणं श्रवणं हितपथ्ययोः}
{पोषणं भूत्यवर्गस्य सान्त्वदानपरिग्रहे}


\twolineshloka
{न्यायतः कर्मकरणमन्यायाहितवर्जितम्}
{सम्यग्वृत्तं स्वदारेषु दोषाणां प्रतिषेधनम्}


\twolineshloka
{पुत्राणां विनयं कुर्यात्तत्तत्कार्यनियोजनम्}
{वर्जनं चाशुभार्थानां शुभानां जोषणं तथा}


\threelineshloka
{कुलोचितानां धर्माणां यतावत्परिपालनम्}
{कुलसन्धारणं चैव पौरुषेणैव सर्वशः}
{एवमादि शुभं सर्वं तस्य वृत्तमिति स्थितम्}


\threelineshloka
{वृद्धसेवी भवेन्नित्यं हितार्थं ज्ञानकाङ्क्षया}
{परार्थं नाहरेद्द्रव्यमनामन्त्र्य तु सर्वथा}
{न याचेत परान्धीरः स्वबाहुबलमाश्रयेत्}


\twolineshloka
{स्वशरीरं सदा रक्षेदाहाराचारयोरपि}
{हितं पथ्यं सदाहारं जीर्णं भुञ्जीत मात्रया}


\twolineshloka
{देवतातिथिसत्कारं कृत्वा सर्वं यथाविधि}
{शेषं भुञ्जेच्छुचिर्भूत्वा न च भाषेत विप्रियम्}


\twolineshloka
{प्रतिश्रयं च पानीयं बलिं भिक्षां च सर्वतः}
{गृहस्थवासी सततं तद्याद्गाश्चैव पोषयेत्}


\twolineshloka
{बहिर्निष्क्रमणं चैव कुर्यात्कारणतोपि वा}
{मध्याह्ने वाऽर्धरात्रे वा गमनाय न रोचयेत्}


\twolineshloka
{विषयान्नावगाहेत स्वशक्त्या तु समाचरेत्}
{यथाऽऽयव्ययता लोके गृहस्थानां प्रपूजितम्}


\twolineshloka
{अयशस्करमर्थघ्नं कर्म यत्परपीडनम्}
{भयाद्वा यदि लोभाद्वा न कुर्वीत कदाचन}


\twolineshloka
{बुद्धिपूर्वं समालोक्य दूरतो गुणदोषतः}
{आरभेत तदा कर्भ शुभं वा यदि वेतरत्}


\twolineshloka
{आत्मसाक्षी भवेन्नित्यमात्मनस्तु शुभाशुभे}
{मनसा कर्मणा वाचा न च काङ्क्षेत पातकम्}


\chapter{अध्यायः २१७}
\threelineshloka
{भगवन्भगनेत्रघ्न कालसूदन शङ्कर}
{इमे वर्णाश्च चत्वारो विहिताः स्वस्वभावतः}
{उताहो क्रियया वर्णाः सम्भवन्ति महेश्वर}


\twolineshloka
{एवं मे संशयप्रश्नस्तं मे छेत्तुं त्वमर्हसि ॥महेश्वर उवाच}
{}


\twolineshloka
{स्वभावादेव विद्यन्ते चत्वारो ब्राह्मणादयः}
{एकजात्या सुदुष्प्रापमन्यवर्णत्वमागतम्}


\twolineshloka
{तच्च कर्मविशेषेण पुनर्जन्मनि जायते}
{तस्मात्तेषां प्रवक्ष्यामि तत्सर्वं कर्मपाकजम्}


\twolineshloka
{ब्राह्मणस्तु नरो भूत्वा स्वजातिमनुपालयन्}
{दृढं ब्राह्मणकर्माणि वेदोक्तानि समाचरेत्}


\twolineshloka
{सत्यार्जवपरो भूत्वा दानयज्ञपरस्तथा}
{सत्यां जात्यां समुदितो जातिधर्मान्न हापयेत्}


\twolineshloka
{एवं संवर्तमानस्तु कालधर्मं गतः पुनः}
{स्वर्गलोके हि जायेत स्वर्गभोगाय भामिनि}


\twolineshloka
{तत्क्षये ब्राह्मणो भूत्वा तथैव नृषु जायते}
{एवंस्वकर्मणा मर्त्यः स्वजातिं लभते पुनः}


\twolineshloka
{अपरस्तु तथा कश्चिद्ब्रह्मयोनिसमुद्भवः}
{अवमत्यैव तां जातिमज्ञानतमसा वृतः}


\fourlineindentedshloka
{अन्यथा वर्तमानस्तु जातिकर्माणि वर्जयेत्}
{शूद्रवद्विचरेल्लोके शूद्रकर्माभिलाषवान्}
{शूद्रैः सह चरन्नित्यं शौचमङ्गलवर्जितः}
{}


\twolineshloka
{स चापि कालधर्मस्थो यमलोके सुदण्डितः}
{यदि जायेत मर्त्येषु शूद्र एवाभिजायते}


% Check verse!
शूद्र एव भवेद्देवि ब्राह्मणोऽपि स्वकर्मणा
\threelineshloka
{तथैव शूद्रस्त्वपरः शूद्रकर्माणि वर्जयेत्}
{सत्यार्जवपरो भूत्वा दानधर्मपरस्तथा}
{मन्त्रब्राह्मणसत्कर्ता मनसा ब्राह्मणप्रियः}


\threelineshloka
{एवं युक्तसमाचारः शूद्रोपि मरणं गतः}
{स्वर्गलोके हि जायेत तत्क्षये नृषु जायते}
{ब्राह्मणानां कुले मुख्ये वेदस्वाध्यायसंयुते}


% Check verse!
एवमेव सदा लोके शूद्रो ब्राह्मण्यमाप्नुयात्
\twolineshloka
{एवं क्षत्रियवैश्याश्च जातिधर्मेण संयुताः}
{स्वकर्मणैव जायन्ते विशिष्टेष्वधमेषु च}


\twolineshloka
{एवं जातिविपर्यासः प्रेत्यभावे भवेन्नृणाम्}
{अन्यथा तु न शक्यं तल्लोकसंस्थितिकारणात्}


\fourlineindentedshloka
{तस्माज्जातिं विशिष्टां तु कथंचित्प्राप्य पण्डितः}
{सर्वथा तां तथा रक्षेन्न पुनर्भ्रश्यते यथा}
{इति ते कथितं देवि भूयः श्रोतुं किमिच्छसि ॥उमोवाच}
{}


\threelineshloka
{जन्मप्रभृति कः शुद्धो लभेज्जन्मफलं नरः}
{शोभनाशोभनं सर्वमधइकारवशात्स्वकम् ॥महेश्वर उवाच}
{}


\twolineshloka
{कर्म कुर्वन्न लिप्येत आर्जवेन समाचरेत्}
{आत्मैव तच्छुभं कुर्यादशुभे योजयेत्परान्}


\threelineshloka
{शठेषु शठवत्कुर्योदार्यष्वार्यवदाचरेत्}
{आपत्सु नावसीदेच्च घोरान्सङ्ग्रामयेत्परात्}
{साम्नैव सर्वकार्याणि कर्तुं पूर्वं समारभेत्}


\twolineshloka
{अनर्थाधर्मशोकानां यथा न प्राप्नुयात्स्वयम्}
{प्रीयते तत्तथा कर्तुमेतद्वृत्तं समासतः}


\twolineshloka
{एवं वृत्तं समासाद्य गृहमाश्रित्य मानवाः}
{निराहारा निरुद्वेगाः प्राप्नुवन्त्युत्तमां गतिम्}


\fourlineindentedshloka
{एतज्जन्मफलं नित्यं सर्वेषां गृहवासिनाम्}
{एवं गृहस्थितैर्नित्यं वर्तितव्यमिति स्थितिः}
{एतत्सर्वं मया प्रोक्तं किं भूयः श्रोतुमिच्छसि ॥उमोवाच}
{}


\threelineshloka
{सुरासुरपते देव वरद प्रीतिवर्धन}
{मानुषेष्वेव ये के चिदाढ्याः क्लेशविवर्जिताः}
{भुञ्जाना विविधान्भोगान्दृश्यन्ते निरुपद्रवाः}


% Check verse!
अपरे क्लेशसंयुक्ता दरिद्रा भोगवर्जिताः
\threelineshloka
{किमर्थं मानुषे लोके न समत्वेन कल्पिताः}
{एतच्छ्रोतुं महादेव कौतूहलमतीव मे ॥महेश्वर उवाच}
{}


\twolineshloka
{न्यायतस्त्वं महाभागे श्रोतुकामासि भामिनि}
{शृणु तत्सर्वमखिलं मानुषाणां हितं वचः}


\threelineshloka
{आदिसर्गे पुरा ब्रह्मा समत्वेनासृजत्प्रजाः}
{नित्यं न भवतो ह्यस्य रागद्वेषौ प्रजापतेः}
{तदा तस्मात्सुराः सर्वे बभूवुः समतो नराः}


\twolineshloka
{एवं संवर्तमाने तु युगे कालविपर्ययात्}
{केचित्प्रपेदिरे तत्र विषमं बुद्धिमोहिताः}


\twolineshloka
{तेषां हानिं ततो दृष्ट्वा तुल्यानामेव भामिनि}
{ब्राह्मणास्ते समाजग्मुस्तत्तत्कारणवेदकाः}


\twolineshloka
{कर्तुं नार्हसि देवेश पक्षपातं त्वमीदृशम्}
{पुत्रभावे समे देव किमर्थं नो भवेत्कलिः}


\twolineshloka
{एवमेतैरुपालब्धो ब्रह्मा वचनमब्रवीत्}
{यूयं मा ब्रूत मे दोषं स्वकृतं स्मरथ प्रजाः}


\threelineshloka
{युष्माभिरेव युष्माकं ग्रथितं हि शुभाशुभम्}
{यादृशं कुरुते कर्म तादृशं फलमश्नुते}
{स्वकृतस्य फलं भुङ्क्ते नान्यस्तद्बोक्तुमर्हति}


\twolineshloka
{एवं संबोधितास्तेन कालकर्त्रा स्वयंभुवा}
{पुनर्विवृत्य कर्माणि शुभान्येव प्रपेदिरे}


\threelineshloka
{एवं विज्ञाततत्वास्ते दानधर्मपरायणाः}
{शुभानि विधिवत्कृत्वा कालधर्मगताः पुनः}
{तानि दानफलान्येव भुञ्जते सुखभोगिनः}


% Check verse!
स्वकृतं तु नरस्तस्मात्स्वयमेव प्रपद्यते
\twolineshloka
{अपरे धर्मकामेभ्यो निवृत्ताश्च शुभेक्षणे}
{कदर्या निरनुक्रोशाः प्रायेणात्मपरायणाः}


\threelineshloka
{तादृशा मरणं प्राप्ताः पुनर्जन्मनि शोभने}
{दरिद्राः क्लेशभूयिष्ठा भवन्त्येव न संशयः ॥उमोवाच}
{}


\fourlineindentedshloka
{मानुषेष्वथ ये केचिद्धनधान्यसमन्विताः}
{भोगहीनाः प्रदृश्यन्ते सर्वभोगेषु सत्स्वपि}
{न भुञ्जते किमर्थं ते तन्मे शंसितुमर्हसि ॥महेश्वर उवाच}
{}


\twolineshloka
{परैः संचोदिता धर्मं कुर्वते न स्वकामतः}
{स्वयं श्रद्धां बहिष्कृत्य कुर्वन्ति च रुदन्ति च}


\fourlineindentedshloka
{तादृशा मरणं प्राप्ताः पुनर्जन्मनि शोभने}
{फलानि तानि सम्प्राप्य भुञ्जते न कदाचन}
{रक्षन्तो वर्धयन्तश्च आसते निधिपालवत् ॥उमोवाच}
{}


\threelineshloka
{केचिद्धनवियुक्ताश्च भोगयुक्ता महेश्वर}
{मानुषाः सम्प्रदृश्यन्ते तन्मे शंसितुमर्हसि ॥महेश्वर उवाच}
{}


\twolineshloka
{आनृशंस्यपरा ये तु धर्मकामाश्चि दुर्गताः}
{परोपकारं कुर्वन्ति दीनानुग्रहकारणात्}


\twolineshloka
{प्रतिपद्युः परधनं नष्टं वाऽन्यैर्नरैर्हृतम्}
{नित्यं ये दातुमनसो नरा वित्तेष्वसत्स्वपि}


\twolineshloka
{कालधर्मवशं प्राप्ताः पुनर्जन्मनि ते नराः}
{एते धनविहीनाश्च भोगयुक्ता भवन्त्युत}


\twolineshloka
{धर्मदानोपदेशं वा कर्तव्यमिति निश्चयः}
{इति ते कथितं देवि किं भूयः श्रोतुमिच्छसि}


\chapter{अध्यायः २१८}
\twolineshloka
{भगवन्देवदेवेश त्र्यक्ष भो वृषभध्वज}
{मानुषास्त्रिविधा देव दृश्यन्ते सततं विभो}


\twolineshloka
{आसीना एव भुञ्जन्ते स्थानैश्वर्यपरिग्रहैः}
{अपरे यत्नपूर्वं तु लभन्ते भोगसङ्ग्रहम्}


\threelineshloka
{अपरे यतमानाश्च न लभन्ते तु किञ्चन}
{केन कर्मविपाकेन तन्मे शंसितुमर्हसि ॥महेश्वर उवाच}
{}


% Check verse!
न्यायतस्त्वं महाभागे श्रोतुकामाऽसि भामिनि
\twolineshloka
{ये लोके मानुषा देवि दानधर्मपरायणाः}
{पात्राणि विधिवज्ज्ञात्वा दूरतोप्यनुमानतः}


\twolineshloka
{अभिगम्य स्वयं तत्र ग्राहयन्ति प्रसाद्य च}
{दानादि चेङ्गितैरेव तैरविज्ञातमेव वा}


\twolineshloka
{पुनर्जन्मनि ते देवि तादृशाः शोभना नराः}
{अयत्नतस्तु तान्येव फलानि प्राप्नुवन्त्युत}


% Check verse!
आसीना एव भुञ्जन्ते भोगान्सुकृतभोगिनः
\twolineshloka
{अपरे ये च दानानि ददत्येव प्रयाचिताः}
{यदायदाऽर्थिने दत्त्वा पुनर्दानं च याचिताः}


\twolineshloka
{तावत्कालं ततो देवि पुनर्जन्मनि ते नराः}
{यत्नतः श्रमसंयुक्ताः पुनस्तान्प्राप्नुवन्ति च}


\twolineshloka
{याचिता अपि केचित्तु अदत्त्वैव कथञ्चन}
{अभ्यसूयापरा मर्त्या लोभोपहतचेतसः}


\twolineshloka
{ते पुनर्जन्मनि शुभे यतन्तो बहुधा नराः}
{न प्राप्नुवन्ति मनुजा मार्गन्तस्तेऽपि किञ्चन}


\twolineshloka
{नानुप्तं रोहते सस्यं तद्वद्दानफलं विदुः}
{यद्यद्ददाति पुरुषस्तत्तत्प्राप्नोति केवलम्}


\twolineshloka
{इति ते कथितं देवि भूयः श्रोतुं किमिच्छसि ॥उमोवाच}
{}


\twolineshloka
{भगवन्भगनेत्रघ्न केचिद्वार्धकसंयुताः}
{अभोगयोग्यकाले तु भोगांश्चैव धनानि च}


\threelineshloka
{लभन्ते स्थविरा भूता भोगैश्वर्यं यतस्ततः}
{केन कर्मविपाकेन तन्मे शंसितुमर्हसि ॥महेश्वर उवाच}
{}


% Check verse!
हन्त ते कथयिष्यामि शृणु तत्वं समाहिता
\twolineshloka
{धर्मकार्यं चिरं कालं विस्मृत्य धनसंयुताः}
{प्राणान्तकाले सम्प्राप्ते व्याधिभिश्च निपीडिताः}


\twolineshloka
{आरभन्ते पुनर्धर्मं दातुं दानानि वा नराः}
{ते पुनर्जन्मनि शुभे भूत्वा दुःखपरिप्लुताः}


\twolineshloka
{अतीतयौवने काले स्थविरत्वमुपागताः}
{लभन्ते पूर्वदत्तानां फलानि शुभलक्षणे}


\twolineshloka
{एतत्कर्मफलं देवि कालयोगाद्भवत्युत ॥उमोवाच}
{}


\threelineshloka
{भोगयुक्ता महादेव केचिद्व्याधिपरिप्लुताः}
{असमर्थाश्च तान्भोक्तं भवन्ति किमु कारणम् ॥महेश्वर उवाच}
{}


\twolineshloka
{व्याधियोगपरिक्लिष्टा ये निराशाः स्वजविते}
{आरभन्ते तदा कर्तुं दानानि शुभलक्षणम्}


\threelineshloka
{ते पुनर्जन्मनि शुभे प्राप्य तानि फलान्युत}
{असमर्थाश्च तान्भोक्तुं व्याधितास्ते भवन्त्युत ॥उमोवाच}
{}


\fourlineindentedshloka
{भगवन्देवदेवेश मानुषेष्वेव केचन}
{रूपयुक्ताः प्रदृश्यन्ते शुभाङ्गा प्रियदर्शनाः}
{केन कर्मविपाकेन तन्मे शंसितुमर्हसि ॥महेश्वर उवाच}
{}


% Check verse!
हन्त ते कथयिष्यामि शृणु तत्वं समाहिता
\twolineshloka
{ये पुरा मनुजा देवि लज्जायुक्ताः प्रियंवदाः}
{शक्ताः सुमधुरा नित्यं भूत्वा चैव स्वभावतः}


\threelineshloka
{अमांसभोजिनश्चैव सदा प्राणिदयायुताः}
{प्रतिकर्मप्रदा वाऽपि वस्त्रदा धर्मकारणात्}
{भूमिशुद्धिकरा वाऽपि कारणादग्निपूजकाः}


\threelineshloka
{एवं युक्तसमाचाराः पुनर्जन्मनि ते नराः}
{रूपेण स्पृहणीयास्तु भवन्त्येव न संशयः ॥उमोवाच}
{}


\threelineshloka
{विरूपाश्च प्रदृश्यन्ते मानुषेष्वेव केचन}
{केन कर्मविपाकेन तन्मे शंसितुमर्हसि ॥महेश्वर उवाच}
{}


% Check verse!
तदहं ते प्रवक्ष्यामि शृणु कल्याणि कारणम्
\twolineshloka
{रूपयोगात्पुरा मर्त्या दर्पाहंकारसंयुताः}
{विरूपहासकास्चैव स्तुतिनिन्दादिभिर्भृशम्}


\twolineshloka
{परोपतापनाश्चैव मांसादाश्च तथैव च}
{अभ्यसूयापराश्चैव अशुद्धाश्च तथा नराः}


\fourlineindentedshloka
{एवं युक्तसमाचारा यमलोके सुदण्डिताः}
{कथंचित्प्राप्य मानुष्यं तत्र ते रूपवर्जिताः}
{विरूपाः सम्भवन्त्येव नास्ति तत्र विचारणा ॥उमोवाच}
{}


\fourlineindentedshloka
{भगवन्देवदेवेश केचित्सौभाग्यसंयुताः}
{रूपभोग्यविहीनाश्च दृश्यन्ते प्रमदाप्रियाः}
{केन कर्मविपाकेन तन्मे शंसितुमर्हसि ॥महेश्वर उवाच}
{}


\twolineshloka
{ये पुरा मनुजा देवि सौम्यशीलाः प्रियंवदाः}
{स्वदारैरेव संतुष्टा दारेषु समवृत्तयः}


\threelineshloka
{दाक्षिण्येनैव वर्तन्ते प्रमदास्वप्रियास्वपि}
{न तु प्रत्यादिशन्त्येव स्त्रीदोषान्गुणसंश्रितान्}
{}


\twolineshloka
{अन्नपानीयदाः काले नृणां स्वादुप्रदाश्च ये}
{स्वदारवर्तिनश्चैव धृतिमन्तो निरत्ययाः}


\fourlineindentedshloka
{एवं युक्तसमाचाराः पुनर्जन्मनि शोभने}
{मानुषास्ते भवन्त्येव सततं सुभगा भृशम्}
{अर्थादृतेऽपि ते देवि भवन्ति प्रमदाप्रियाः ॥उमोवाच}
{}


\threelineshloka
{दुर्भगाः सम्प्रदृश्यन्ते आढ्या भोगयुता अपि}
{केन कर्मविपाकेन तन्मे शंसितुमर्हसि ॥महेश्वर उवाच}
{}


% Check verse!
तदहं ते प्रवक्ष्यामि शृणु सर्वं समाहिता
\twolineshloka
{ये पुरा मनुजा देवि स्वदारेष्वनपेक्षया}
{यथेष्टवृत्तयश्चैव निर्लज्जा वीतसम्भ्रमाः}


\twolineshloka
{परेषां विप्रियकरा वाङ्मनःकायकर्मभिः}
{निराश्रया निरानन्दाः स्त्रीणां हृदयकोपनाः}


\threelineshloka
{एवं युक्तसमाचाराः पनर्जन्मनि ते नराः}
{दुर्भगास्तु भवन्त्येव स्त्रीणां हृदयविप्रियाः}
{}


% Check verse!
नास्ति तेषां रतिसुखं स्वदारेष्वपि किञ्चन
\chapter{अध्यायः २१९}
\twolineshloka
{भगवन्देवदेवेश मानुषेष्वेव केचन}
{ज्ञानविज्ञानसम्पन्ना बुद्धिमन्तो विचक्षणाः}


\threelineshloka
{दुर्गतास्तु प्रदृश्यन्ते यतमाना यथाविधि}
{केन कर्मविपाकेन तन्मे शंसितुमर्हसि ॥महेश्वर उवाच}
{}


% Check verse!
तदहं ते प्रवक्ष्यामि शृणु कल्याणि कारणम्
\twolineshloka
{ये पुरा मनुजा देवि श्रुतवन्तोपि केवलम्}
{निरा**** निरन्नाद्या भृशमात्मपरायणाः}


\threelineshloka
{ते पुनर्जन्मनि शुभे ज्ञानबुद्धियुता अपि}
{निष्किञ्चना भवन्त्येव अनुप्तं हि न रोहति ॥उमोवाच}
{}


\fourlineindentedshloka
{मूर्खा लोके प्रदृश्यन्ते वृथा मूढा विचेतसः}
{ज्ञानविज्ञानरहिताः समृद्धाश्च समन्ततः}
{केन कर्मविपाकेन तन्मे शंसितुमर्हसि ॥महेश्वर उवाच}
{}


\twolineshloka
{ये पुरा मनुजा देवि बालिशा अपि सर्वतः}
{समाचरन्ति दानानि दीनानुग्रहकारणात्}


\twolineshloka
{अबुद्धिपूर्वं वा दानं ददत्येव यतस्ततः}
{ते पुनर्जन्मनि शुभे प्राप्नुवन्त्येव तत्तथा}


\threelineshloka
{पण्डितोऽपण्डितो वाऽपि भुङ्क्ते दानफलं नरः}
{बुद्ध्याऽनपेक्षितं दानं सर्वथा तत्फलत्युत ॥उमोवाच}
{}


\fourlineindentedshloka
{भगवन्देवदेवेश मानुषेष्वेव केचन}
{मेधाविनः श्रुतधरा भवन्ति विशदाक्षराः}
{केन कर्मविपाकेन तन्मे शंसितुमर्हसि ॥महेश्वर उवाच}
{}


\twolineshloka
{ये पुरा मनुजा देवि गुरुशुश्रूषका भृशम्}
{ज्ञानार्थं ते तु सङ्गृह्य तीर्थतो विधिपूर्वकम्}


\threelineshloka
{विधिनैव परांश्चैव ग्राहयन्ति च नान्यथा}
{अश्लाघमाना ज्ञानेन प्रशान्ता यतवाचकाः}
{विद्यास्थानानि ये लोके स्थापयन्ति च यत्नतः}


\threelineshloka
{तादृश मरणं प्राप्ताः पुनर्जन्मनि शोभने}
{मेधाविनः श्रुतधरा भवन्ति विशदाक्षराः ॥उमोवाच}
{}


\fourlineindentedshloka
{अपरे मानुषा देव यतन्तोपि यतस्ततः}
{बहिष्कृताः प्रदृश्यन्ते श्रुतविज्ञानबुद्धितः}
{केन कर्मविपाकेन तन्मे शंसितुमर्हसि ॥महेश्वर उवाच}
{}


\twolineshloka
{ये पुरा मनुजा देवि ज्ञानदर्पसमन्विताः}
{श्लाघमानाश्च तत्प्राप्य ज्ञानाहङ्कारमोहिताः}


\twolineshloka
{वदन्ति ये परान्नित्यं ज्ञानाधिक्येन दर्पिताः}
{ज्ञानादसूयां कुर्वन्ति न सहन्ते च चापरान्}


\fourlineindentedshloka
{तादृशा मरणं प्राप्ताः पुनर्जन्मनि शोभने}
{मानुष्यं सुचिरात्प्राप्य तत्र बोधविवर्जिताः}
{भवन्ति सततं देवि यतन्तो हीनमेधसः ॥उमोवाच}
{}


\twolineshloka
{भगवन्मानुषाः केचित्सर्वकल्याणसंयुताः}
{पुत्रैर्दारैर्गुणयुतैर्दासीदासपरिच्छदैः}


\twolineshloka
{परमं बुद्धिसंयुक्ताः स्थानैश्वर्यपरिग्रहैः}
{व्याधिहीना नबाधाश्च रूपारोग्यबलैर्युताः}


\twolineshloka
{धनधान्येन सम्पन्नाः प्रासादैर्यानवाहनैः}
{सर्वोपभोगसंयुक्ता नानाचित्रैर्मनोहरैः}


\threelineshloka
{ज्ञातिभिः सह मोदन्ते अविघ्नं तु दिनेदिने}
{केन कर्मविपाकेन तन्मे शंसितुमर्हसि ॥महेश्वर उवाच}
{}


% Check verse!
तदहं ते प्रवक्ष्यामि शृणु सर्वं समाहिता
\twolineshloka
{ये पुरा मनुजा देवि आढ्या वा इतरेऽपि वा}
{श्रुतवृत्तसमायुक्ता दानकामाः श्रुतप्रियाः}


\twolineshloka
{परेङ्गितपरा नित्यं दातव्यमिति निश्चिताः}
{सत्यसन्धाः क्षमाशीला लोभमोहविवर्जिताः}


\threelineshloka
{दातारः पात्रतो दानं व्रतैर्नियमसंयुताः}
{स्वदुःखमिव संस्मृत्य परदुःखविवर्जिताः}
{सौम्यशीलाः शुभाचारा देवब्राह्मणपूजकाः}


\twolineshloka
{एवंशीलसमाचाराः पुनर्जन्मनि शोभने}
{दिवि वा भुवि वा देवि जायन्ते कर्मभोगिनः}


\twolineshloka
{मानुषेष्वपि ये जातास्तादृशाः सम्भवन्ति ते}
{यादृशास्तु तथा प्रोक्ताः सर्वे कल्याणसंयुताः}


\fourlineindentedshloka
{रूपं द्रव्यं बलं चायुर्भोगैश्वर्यं बलं श्रुतम्}
{इत्येतत्सर्वसाद्गुण्यं दानाद्भवति नान्यथा}
{तपोदानमयं सर्वमिति विद्धि शुभानने ॥उमोवाच}
{}


\twolineshloka
{अथ केचित्प्रदृश्यन्ते मानुषेष्वेव मानुषाः}
{दुर्गताः क्लेशभूयिष्ठा दानभोगविवर्जिताः}


\fourlineindentedshloka
{भयैस्त्रिभिः समाजुष्टा व्याधिक्षुद्भयसंयुताः}
{दुष्कलत्राभिभूताश्च सततं विघ्नदर्शकाः}
{केन कर्मविपाकेन तन्मे शंसितुमर्हसि ॥महेश्वर उवाच}
{}


\twolineshloka
{ये पुरा मनुजा देवि आसुरं भावमाश्रिताः}
{क्रोधलोभसमायुक्ता निरन्नाद्याश्च निष्क्रियाः}


\twolineshloka
{नास्तिकाश्चैव धूर्ताश्च मूर्खाश्चात्मपरायणाः}
{परोपतापिनो देवि प्रायशः प्राणिनिर्दयाः}


\twolineshloka
{एवंयुक्तसमाचाराः पुनर्जन्मनि शोभने}
{कथंचित्प्राप्य मानुष्यं तत्र ते दुःखपीडिताः}


\twolineshloka
{सर्वतः सम्भवन्त्येव पूर्वमात्मप्रमादतः}
{यथा ते पूर्वकथितास्तथा ते सम्भवन्त्युत}


\twolineshloka
{शुभाशुभं कृतं कर्म सुखदुःखफलोदयम्}
{इति ते कथितं देवि भूयः श्रोतुं किमिच्छसि}


\chapter{अध्यायः २२०}
\fourlineindentedshloka
{भगवन्देवदेवेश मम प्रीतिविवर्धन}
{`जात्यन्धाश्चैव दृश्यन्ते जाता वा नष्टचक्षुषः}
{केन कर्मविपाकेन तन्मे शंसितुमर्हसि ॥महेश्वर उवाच}
{}


% Check verse!
हन्त ते कथयिष्यामि शृणु कल्याणि कारणम्
\twolineshloka
{ये पुरा कामकारेणि परवेश्मसु लोलुपाः}
{परस्त्रियोऽभिवीक्षन्ते दुष्टेनैव स्वचक्षुषाः}


\twolineshloka
{अन्धीकुर्वन्ति यन्मर्त्यान्क्रोधलोभसमन्विताः}
{लक्षणज्ञाश्च रूपेषु अयथावत्प्रदर्शकाः}


\twolineshloka
{एवं युक्तसमाचाराः कालधर्मवशास्तु ते}
{दण्डिता यमदण्डेन निरयस्थाश्चिरं प्रिये}


\fourlineindentedshloka
{यदि चेन्मानुषं जन्म लभेरंस्ते तथापि वा}
{स्वभावतो वा जाता वा अन्धा एव भवन्ति ते}
{अक्षिरोगयुता वाऽपि नास्ति तत्र विचारणा ॥उमोवाच}
{}


\twolineshloka
{मुखरोगयुताः केचिद्दृश्यन्ते सततं नराः}
{दन्तकण्ठकपोलस्थैर्व्याधिभिर्बहुपीडिताः}


\threelineshloka
{आदिप्रभृति वै मर्त्या जाता वाऽप्यथ कारणात्}
{केन कर्मविपाकेन तन्मे शंसितुमर्हसि ॥महेश्वर उवाच}
{}


% Check verse!
हन्त ते कथयिष्यामि शृणु देवि समाहिता
\twolineshloka
{कुवक्तारस्तु ये देवि जिह्वया कटुकं भृशम्}
{असत्यं परुषं घोरं गुरून्प्रति परान्प्रति}


\threelineshloka
{जिह्वाबाधां तदाऽन्येषां कुर्वते कोपकारणात्}
{प्रायशोऽनृतभूयिष्ठा नराः कार्यवशेन वा}
{तेषां जिह्वाप्रदेशस्था व्याधयः सम्भवन्ति ते}


\twolineshloka
{कुश्रोतारस्तु ये चार्यं परेषां कर्णनाशकाः}
{कर्णरोगान्बहुविधाँल्लभन्ते ते पुनर्भवे}


\threelineshloka
{दन्तरोगशिरोरोगकर्णरोगास्तथैव च}
{अन्ये दुःखाश्रिता दोषाः सर्वे चात्मकृतं फलम् ॥उमोवाच}
{}


\twolineshloka
{पीड्यन्ते सततं देव मानुषेष्वेव केचन}
{कुक्षिपक्षाश्रितैर्दोषैर्व्याधिभिश्चोदराश्रितैः}


\threelineshloka
{तीक्ष्णिशूलैश्च पीड्यन्ते नरा दुःखपरिप्लुताः}
{केन कर्मविपाकेन तन्मे शंसितुमर्हसि ॥महेश्वर उवाच}
{}


\twolineshloka
{ये पुरा मनुजा देवि कामक्रोधवशा भृशम्}
{आत्मार्थमेव चाहारं भुञ्जन्ते निरपेक्षकाः}


\twolineshloka
{अभक्ष्याहारदानैश्च विश्वस्तानां विषप्रदाः}
{अभक्ष्यभक्षदाश्चैव शौचमङ्गलवर्जिताः}


\twolineshloka
{मांसयुक्तसमाचाराः पुनर्जन्मनि शोभने}
{कथञ्चित्प्राप्य मानुष्यं तत्र ते व्याधिपीडिताः}


\threelineshloka
{तैस्तैर्बहुविधाकारैर्व्याधिभिर्दुःखसंश्रिताः}
{भवन्त्येवं तथा देवि यथा चैवं तथा कृतम् ॥उमोवाच}
{}


\fourlineindentedshloka
{दृश्यन्ते सततं देव व्याधिभिर्मेहनाश्रितैः}
{पीड्यमानास्तथा मर्त्या अश्मरीशर्करादिभिः}
{केन कर्मविपाकेन तन्मे शंसितुमर्हसि ॥महेश्वर उवाच}
{}


\twolineshloka
{ये पुरा मनुजा देवि परदारप्रधर्षकाः}
{तिर्यग्योनिषु धूर्ता वै मैथुनार्थं चरन्ति च}


\twolineshloka
{कामदोषेणि ये धूर्ताः कन्यासु विधवासु च}
{बलात्कारेण गच्छन्ति रूपदर्पसमन्विताः}


\fourlineindentedshloka
{तादृशा मरणं पुनर्जन्मनि शोभने}
{यदि चेन्मानुषं जन्म लभेरंस्ते तथाविधाः}
{मेहनस्थैस्तथा घोरैः पीड्यन्ते व्यधिभिः प्रिये ॥उमोवाच}
{}


\threelineshloka
{भगवन्मानुषाः केचिद्दृश्यन्ते शोषिणः कृशाः}
{केन कर्मविपाकेन तन्मे शंसितुमर्हसि ॥महेश्वर उवाच}
{}


\twolineshloka
{ये पुरा मनुजा देवि मांसलुब्धाः सुलोलुपाः}
{आत्मार्थे स्वादुगृद्धाश्च परभोगोपतापिनः}


\twolineshloka
{अभ्यसूयाश्चोपतापाः परभोगेषु ये नराः}
{एवं युक्तसमाचाराः पुनर्जन्मनि शोभने}


\threelineshloka
{शेषव्याधियुतास्तत्र नरा धमनिसंतताः}
{भवन्त्येव नरा देवि पापकर्मोपभोगिनः ॥उमोवाच}
{}


\threelineshloka
{भगवन्मानुषाः केचित्क्लिश्यन्ते कण्ठरोगिणः}
{केन कर्मविपाकेन तन्मे शंसितुमर्हसि ॥महेश्वर उवाच}
{}


\twolineshloka
{ये पुरा मनुजा देवि परेषां रूपनाशनाः}
{आघातवधबन्धैश्च वृथा दण्डेन मोहिताः}


\twolineshloka
{इष्टनाशकरा ये तु अपथ्याहारदा नराः}
{चिकित्सका वा दुष्टास्च द्वेषलोभसमन्विताः}


\twolineshloka
{निर्दयाः प्राणिहिंसायां मलदाश्चित्तनाशनाः}
{एवंयुक्तसमाचाराः पुनर्जन्मनि शोभने}


\twolineshloka
{यदि वै मानुषं जन्म लभेरंस्तेषु दुःखिताः}
{अत्र ते क्लेशसंयुक्ताः कण्ठरोगशतैर्वृताः}


\fourlineindentedshloka
{केचित्त्वग्दोषसंयुक्ता व्रणकुष्ठैश्च संयुताः}
{श्वित्रकुष्ठयुता वाऽपि बहुधा कृच्छ्रसंयुताः}
{भवन्त्येव नरा देवि यथा तेन कृतं फलम् ॥उमोवाच}
{}


\threelineshloka
{भगवन्मानुषाः केचिदङ्गहीनाश्च पङ्गव}
{केन कर्मविपाकेन तन्मे शंसितुमर्हसि ॥महेश्वर उवाच}
{}


\threelineshloka
{ये पुरा मनुजा देवि लोभमोहसमावृताः}
{प्राणिनां प्राणहिंसार्थमङ्गविघ्नं प्रकुर्वते}
{शस्त्रेणोत्कृत्य वा देवि प्राणिनां चेष्टनाशकाः}


\fourlineindentedshloka
{एवं युक्तसमाचाराः पुनर्जन्मनि शोभने}
{तदङ्गहीना वै प्रेत्य भवन्त्येव न संशयः}
{स्वभावतो वा जाता वा पङ्गवश्च भवन्ति ते ॥उमोवाच}
{}


\threelineshloka
{भगवन्मानुषाः केचिद्ग्रन्थिभिः पिलकैस्तथा}
{क्लिश्यमानाः प्रदृश्यन्ते तन्मे शंसितुमर्हसि ॥महेश्वर उवाच}
{}


\threelineshloka
{ये पुरा मनुजा देवि ग्रन्थिभेदकरा नृणाम्}
{मुष्टिप्रहारपरुषा नृशंसाः पापकारिणः}
{पाटकास्तोटकाश्चैव शूलतुन्नास्तथैव च}


\threelineshloka
{एवंयुक्तसमाचाराः पुनर्जन्मनि शोभने}
{ग्रन्थिभिः पिलकैश्चैव क्लिश्यन्ते भृशदुःखिताः ॥उमोवाच}
{}


\threelineshloka
{भगवन्मानुषाः केचित्पादरोगसमन्विताः}
{दृश्यन्ते सततं देव तन्मे शंसितुमर्हसि ॥महेश्वर उवाच}
{}


\threelineshloka
{ये पुरा मनुजा देवि क्रोधलोभसमन्विताः}
{मनुजा देवतास्थानं स्वपादैर्भ्रंशयन्त्युत}
{जानुभिः पार्ष्णिभिश्चैव प्राणिहिंसां प्रकुर्वते}


\threelineshloka
{एवंयुक्तसमाचाराः पुनर्जन्मनि शोभने}
{पादरोगैर्बहुविधैर्बाध्यन्ते विपदादिभिः ॥उमोवाच}
{}


\twolineshloka
{भगवन्मानुषाः केचिद्दृश्यन्ते बहवो भुवि}
{वातजैः पित्तजै रोगैर्युगपत्सान्निपातकैः}


\fourlineindentedshloka
{रोगैर्बहुविधैर्देव क्लिश्यमानाः सुदुःखिताः}
{असमस्तैः समस्तैश्च आढ्या वा दुर्गतास्तथा}
{केन कर्मविपाकेन तन्मे शंसितुमर्हसि ॥महेश्वर उवाच}
{}


% Check verse!
तदहं ते प्रवक्ष्यामि शृणु कल्याणि कारणम्
\twolineshloka
{ये पुरा मनुजा देवि त्वासुरं भावमाश्रिताः}
{स्ववशाः कोपनपरा गुरुविद्वेषिणस्तथा}


\twolineshloka
{परेषां दुःखजनका मनोवाक्कायकर्मभिः}
{छिन्दन्भिन्दन्स्तुदन्नेव नित्यं प्राणिषु निर्दयाः}


\threelineshloka
{एवंयुक्तसमाचाराः पुनर्जन्मनि शोभने}
{यदि वै मानुषं जन्म लभेरंस्ते तथाविधाः}
{तत्र ते बहुभिर्घोरैस्तप्यन्ते व्याधिभिः प्रिये}


\twolineshloka
{केचिद्वातादिसंयुक्ताः केचित्काससमन्विताः}
{ज्वरातिसारतृष्णाभिः पीड्यमानास्तथा परे}


\fourlineindentedshloka
{पादगुल्मैश्च बहुभिः श्लेष्मदोषसमन्विताः}
{पादरोगैश्च विविधैर्व्रणकुष्ठभगंदरैः}
{आढ्या वा दुर्गता वाऽपि दृश्यन्ते व्याधिपीडिताः}
{}


\threelineshloka
{एवमात्मकृतं कर्म भुञ्जन्ते तत्रतत्र ते}
{अभिभूतुं न शक्यं हि केनचित्स्वकृतं फलम्}
{इति ते कथितं देवि भूयः श्रोतुं किमिच्छसि}


\chapter{अध्यायः २२१}
\twolineshloka
{भगवन्देवदेवेश भूतपाल नमोस्तु ते}
{ह्रस्वाङ्गाश्चैव वक्राङ्गाः कुब्जा वामनकास्तथा}


\threelineshloka
{अपरे मानुषा देव दृश्यन्ते कुणिबाहवः}
{केन कर्मविपाकेन तन्मे शंसितुमर्हसि ॥महेश्वर उवाच}
{}


\twolineshloka
{ये पुरा मनुजा देवि लोभमोहसमन्विताः}
{धान्यमानान्विकुर्वन्ति क्रयविक्रयकारणात्}


\twolineshloka
{कुलदोषं तदा देवि धृतमानेषु नित्यशः}
{अर्धापकर्षणं चैव सर्वेषां क्रयविक्रये}


\twolineshloka
{अङ्गदोषकरा ये तु परेषां कोपकारणात्}
{मांसादाश्चैव ये मूर्खा अयथावत्प्रथाः सदा}


\threelineshloka
{एवंयुक्तसमाचाराः पुनर्जन्मनि शोभने}
{ह्रस्वाङ्गा वामनाश्चैव कुब्जाश्चैव भवन्ति ते ॥उमोवाच}
{}


\fourlineindentedshloka
{भगवन्मानुषाः केचिद्दृश्यन्ते मानुषेषु वै}
{उन्मत्ताश्च पिशाचाश्च पर्यटन्तो यतस्ततः}
{केन कर्मविपाकेन तन्मे शंसितुमर्हसि ॥महेश्वर उवाच}
{}


\twolineshloka
{ये पुरा मनुजा देवि दर्पाहंकारसंयुताः}
{बहुधा प्रलपन्त्येव हसन्ति च परान्भृशम्}


\threelineshloka
{मोहयन्ति परान्भोगैर्मदनैर्लोभकारणात्}
{वृद्धान्गुरूंश्च ये मूर्खा वृथैवापहसन्ति च}
{शौण्डा विदग्धाः शास्त्रेषु सदैवानृतवादिनः}


\threelineshloka
{एवंयुक्तसमाचाराः पुनर्जन्मनि शोभने}
{उन्मत्ताश्च पिशाचाश्च भवन्त्येव न संशयः ॥उमोवाच}
{}


\fourlineindentedshloka
{भगवन्मानुषाः केचिन्निरपत्याः सुदुःखिताः}
{यतन्तो न लभन्त्येव अपत्यानि यतस्ततः}
{केन कर्मविपाकेन तन्मे शंसितुमर्हसि ॥महेश्वर उवाच}
{}


\twolineshloka
{ये पुरा मनुजा देवि सर्वप्राणिषु निर्दयाः}
{घ्नन्ति बालांश्च भुञ्जन्ते मृगाणां पक्षिणामपि}


\twolineshloka
{गुरुविद्वेषिणश्चैव परपुत्राभ्यसूयकाः}
{पितृपूजां न कुर्वन्ति यथोक्तां चाष्टकादिभिः}


\fourlineindentedshloka
{एवंयुक्तसमाचाराः पुनर्जन्मनि शोभने}
{मानुष्यं वा चिरात्प्राप्य निरपत्या भवन्ति ते}
{पुत्रशोकयुताश्चापि नास्ति तत्र विचारणा ॥उमोवाच}
{}


\twolineshloka
{भगवन्मानुषाः केचित्प्रदृश्यन्ते सुदुःखिताः}
{उद्वेगवासनिरताः सोद्वेगाश्च यतव्रताः}


\threelineshloka
{नित्यं शोकसमाविष्टा दुर्गताश्च तथैव च}
{केन कर्मविपाकेन तन्मे शंसितुमर्हसि ॥महेश्वर उवाच}
{}


\twolineshloka
{ये पुरा मनुजा नित्यमुत्क्रोशनपरायणाः}
{भीषयन्ति परानित्यं विकुर्वन्ति तथैव च}


\twolineshloka
{ऋणवृद्धिकराश्चैव दरिद्रेभ्यो यथेष्टतः}
{ऋणार्थमभिगच्छन्ति सततं वृद्धिरूपकाः}


\twolineshloka
{उद्विजन्ते हि तान्दृष्ट्वा धारकाः स्वार्थकारणात्}
{अतिवृद्धिर्न कर्तव्या दरिद्रेभ्यो यथेष्टतः}


\twolineshloka
{ये श्वभिः क्रीडमानाश्च त्रासयन्ति वने मृगान्}
{प्राणिहिंसां तथा देवि कुर्वन्ति च यतस्ततः}


\twolineshloka
{येषां गृहेषु वै श्वानस्त्रासयन्ति वृथा नरान्}
{एवंयुक्तसमाचाराः कालधर्मगताः पुनः}


\twolineshloka
{पीडिता यमदण्डेन निरयस्थाश्चिरं प्रिये}
{कथञ्चित्प्राप्य मानुष्यं तत्र ते दुःखसंयुताः}


\threelineshloka
{कुदेशे दुःखभूयिष्ठे व्याघातशतसङ्कुले}
{जायन्ते तत्र शोचन्तः सोद्वेगाश्च यतस्ततः ॥उमोवाच}
{}


\fourlineindentedshloka
{भगवन्मानुषाः केचिदैश्वर्यज्ञानसंयुताः}
{म्लेच्छभूमिषु दृश्यन्ते म्लेच्छैश्वर्यसमन्विताः}
{केन कर्मविपाकेन तन्मे शंसितुमर्हसि ॥महेश्वर उवाच}
{}


\twolineshloka
{ये पुरा मनुजा देवि धनधान्यसमन्विताः}
{अयथावत्प्रयच्छन्ति श्रद्धावर्जितमेव वा}


\twolineshloka
{अपात्रेभ्यश्च ये दानं शौचमङ्गलवर्जिताः}
{ददत्येव च ये मूर्खाः श्लाघयाऽवज्ञयाऽपि वा}


\fourlineindentedshloka
{एवंयुक्तसमाचाराः पुनर्जन्मनि शोभने}
{कुदेशे म्लेच्छभूयिष्ठे दुर्गमे वनसंकटे}
{म्लेच्छाधिपत्यं सम्प्राप्य जायन्ते तत्रतत्र वै ॥उमोवाच}
{}


\twolineshloka
{भगवन्भगनेत्रघ्न मानुषेषु च केचन}
{क्लीबा नपुंसकाश्चैव दृश्यन्ते षण्डकास्तथा}


\threelineshloka
{नीचकर्मरता नीचा नीचसक्यास्तथा भुवि}
{केन कर्मविपाकेन तन्मे शंसितुमर्हसि ॥महेश्वर उवाच}
{}


% Check verse!
तदहं ते प्रवक्ष्यामि शृणु कल्याणि कारणम्
\twolineshloka
{ये पुरा मनुजा भूत्वा घोरकर्मरतास्तथा}
{पशुपंस्त्वोपगातेन जीवन्ति च रमन्ति च}


\twolineshloka
{पुंस्त्वोपघातिनश्चैव नराणां कोपकारणात्}
{ये धूर्ताः स्त्रीषु गच्छन्ति अयथावद्यथेष्टतः}


\threelineshloka
{कामविघ्नकरा ये तु द्वेषपैशुन्यकारणात्}
{एवंयुक्तसमाचाराः कालधर्मं गतास्तु ते}
{दण्डिता यमदण्डेन निरयस्थाश्चिरं प्रिये}


\twolineshloka
{यदि चेन्मानुषं जन्म लभेरंस्ते तथाविधाः}
{क्लीबा वर्षवराश्चैव षण्डकाश्च भवन्ति ते}


\twolineshloka
{नीचकर्मपरा लोके निर्लज्जा वीतसम्भ्रमाः}
{परान्दीनान्बहिष्कृत्य ते भवन्ति स्वकर्मणा}


\twolineshloka
{यदि चेत्सम्प्रपश्येरंस्ते मुच्यन्ते हि कल्मषात्}
{अत्रापि ते प्रमाद्येयुः पतन्ति नरकालये}


\twolineshloka
{स्त्रीणामपि तथा देवि यथा पुंसां तु कर्मजम्}
{इति ते कथितं देवि भूयः श्रोतुं किमिच्छसि}


\chapter{अध्यायः २२२}
\fourlineindentedshloka
{भगवन्देवदेवेश शूलपाणे वृषध्वज}
{पुंश्चल्य इव या स्त्रीषु नीचवृत्तिरताः स्मृताः}
{केन कर्मविपाकेन तन्मे शंसितुमर्हसि ॥महेश्वर उवाच}
{}


\twolineshloka
{याः पुरा मनुजा देवि बुद्धिमोहसमन्विताः}
{कामरागसमायुक्ताः पतीनतिचरन्ति वै}


\twolineshloka
{प्रतिकूलपरा यास्तु पतीन्प्रति यथा तथा}
{शौचं लज्जां तु विस्मृत्य यथेष्टपरिचारणाः}


\threelineshloka
{एवंयुक्तसमाचारा यमलोके सुदण्डिताः}
{यदि वै मानुषं जन्म लभेरंस्तास्तथाविधाः}
{बहुसाधारणा एव पुंश्चल्यश्च भवन्ति ताः}


\twolineshloka
{पौश्चल्यं यत्तु तद्वृत्तं स्त्रीणां कष्टतमं स्मृतम्}
{ततःप्रभृति ता देवि पतन्त्येव न संशयः}


\threelineshloka
{शोचन्ति चेत्तु तद्वृत्तं मनसा हितमाप्नुयुः ॥उमोवाच}
{भगवन्देवदेवेश प्रमदा विधवा भृशम्}
{}


\threelineshloka
{दृश्यन्ते मानुषा लोके सर्वकल्याणवर्जिताः}
{केन कर्मविपाकेन तन्मे शंसितुमर्हसि ॥महेश्वर उवाच}
{}


\twolineshloka
{याः पुरा मनुजा देवि बुद्धिमोहसमन्विताः}
{कुटुम्बं तत्र वे पत्युर्नाशयन्ति वृथा तथा}


\twolineshloka
{विषदाश्चाग्निदाश्चैव पतीन्प्रति सुनिर्दयाः}
{अन्यासां हि पतीन्यान्ति स्वपतिद्वेषकारणात्}


\fourlineindentedshloka
{एवंयुक्तसमाचारा यमलोके सुदण्डिताः}
{निरयस्थाश्चिरं कालं कथंचित्प्राप्य मानुषम्}
{तत्र ता भोगरहिता विधवास्तु भवन्ति वै ॥उमोवाच}
{}


\fourlineindentedshloka
{भगवन्प्रमदा लोके पत्यौ ज्ञातिषु सत्सुच}
{लिङ्गिन्यः सम्प्रदृश्यन्ते पाषण्डं मतमाश्रिताः}
{केन कर्मविपाकेन तन्मे शंसितुमर्हसि ॥महेश्वर उवाच}
{}


\twolineshloka
{याः पुरा भावदोषेणि लोभमोहसमन्विताः}
{परद्रव्यपरा लोभात्परेषां द्रव्यहारकाः}


\twolineshloka
{अभ्यसूयापरा यास्तु सपत्नीनां प्रदूषकाः}
{ईर्ष्यापराः कोपनाश्च बन्धूनां विफलाः सदा}


\fourlineindentedshloka
{एवंयुक्तसमाचाराः पुनर्जन्मनि ताः स्त्रियः}
{अलक्षणसमायुक्ताः पाषण्डं धर्ममाश्रिताः}
{स्त्रियः प्रव्राजशीलाश्च भवन्त्येव न संशयः ॥उमोवाच}
{}


\fourlineindentedshloka
{भगवन्मानुषाः केचित्कारुवृत्तिसमाश्रिताः}
{प्रदृश्यन्ते मनुष्येषु नीचकर्मरतास्तथा}
{केन कर्मविपाकेन तन्मे शंसितुमर्हति ॥महेश्वर उवाच}
{}


\twolineshloka
{ये पुरा मनुजा देवि स्तब्धमानयुता भृशम्}
{दर्पाहङ्कारसंयुक्ताः केवलात्मपरायणाः}


\twolineshloka
{तादृशा मानुषा देवि पुनर्जन्मनि शोभने}
{कात्वो नटगन्धर्वाः सम्भवन्ति यथा तथा}


\twolineshloka
{नापिता बन्दिनश्चैव तथा वैतालिकाः प्रिये}
{एवंभूतास्त्वधोवृत्तिं जीवन्त्याश्रित्य मानवाः}


\threelineshloka
{परप्रसाधनकरास्ते परैः कृतवेतनाः}
{परावमानस्य फलं भुञ्जते पौर्वदैहिकम् ॥उमोवाच}
{}


\twolineshloka
{भगवन्देवदेवेश मानुषेष्वेव केचन}
{दासभूताः प्रदृश्यन्ते सर्वकर्मपरा भृशम्}


\threelineshloka
{आघातभर्त्सनसहाः पीड्यमानाश्च सर्वशः}
{केन कर्मविपाकेन तन्मे शंसितुमर्हसि ॥महेश्वर उवाच}
{}


% Check verse!
तदहं ते प्रवक्ष्यामि शृणु कल्याणि कारणम्
\twolineshloka
{ये पुरा मनुजा देवि परेषां वित्तहारकाः}
{ऋणवृद्धिकरं कृष्या न्यासदत्तं तथैव च}


\twolineshloka
{निक्षेपकारणाद्दत्तपरद्रव्यापहारिणः}
{प्रमादाद्विस्मृतं नष्टं परेषां धनहारकाः}


\twolineshloka
{वधबन्धपरिक्लेशैर्दासत्वं कुर्वते परान्}
{तादृशा मरणं प्राप्ता दण्डिता यमशासनैः}


\twolineshloka
{कथञ्चित्प्राप्य मानुष्यं तत्र ते देवि सर्वथा}
{दासभूता भविष्यन्ति जन्मप्रभृति मानवाः}


\twolineshloka
{तेषां कर्माणि कुर्वन्ति येषां ते धनहारकाः}
{आसमाप्तेः स्वपापस्य कुर्वन्तीति विनिश्चयः}


\threelineshloka
{पशुभूतास्तथा चान्ये भवन्ति धनहारकाः}
{तत्तथा क्षीयते कर्म तेषां पूर्वापराधजम्}
{अतोऽन्यथा न तच्छक्यं कर्म भोक्तुं सुरासुरैः}


\twolineshloka
{किन्तु मोक्षविधिस्तेषां सर्वता तत्प्रसादनम्}
{अयथावन्मोक्षकामः पुनर्जन्मनि चेष्यते}


\twolineshloka
{मोक्षकामी यथान्यायं कुर्वन्कर्माणि सर्वशः}
{भर्तुः प्रसादमाकाङ्क्षेदायासान्सर्वथा सहन्}


\twolineshloka
{प्रीतिपूर्वं तु यो भर्त्रा मुक्तो मुक्तः स्वपापतः}
{तथाभूतान्कर्मकरान्सदा सन्तोषयेत्पतिः}


\twolineshloka
{यथार्हं कारयेत्कर्म दण्डकारणतः क्षिपेत्}
{वृद्धान्बालांस्तथा क्षीणान्पालयन्धर्ममाप्नुयात्}


% Check verse!
इति ते कथितं देवि भूयः श्रोतुं किमिच्छसि
\chapter{अध्यायः २२३}
\twolineshloka
{भगवन्मानुषेष्वेव मानुषाः समदर्शनाः}
{चण्डाला इव दृश्यन्ते स्पर्शमात्रविदूषिताइः}


\fourlineindentedshloka
{नीचकर्मरता देव सर्वेषां मलहारकाः}
{दुर्गताः क्लेशभूयिष्ठा विरूपा दुष्टचेतसः}
{केन कर्मविपाकेन तन्मे शंसितुमर्हसि ॥महेश्वर उवाच}
{}


% Check verse!
तदहं ते प्रवक्ष्यामि तदेकाग्रमनाः शृणु
\twolineshloka
{ये पुरा मनुजा देवि अतिमानयुता भृशम्}
{आत्मसम्भावनायुक्ताः स्तब्धा दर्पसमन्विताः}


\twolineshloka
{प्रणामं तु न कुर्वन्ति गुरूणामपि पामराः}
{ये स्वधर्मार्पणं कार्यमतिमानान्न कुर्वते}


\threelineshloka
{परान्संनामयन्त्येव आज्ञयात्मनि ये बलात्}
{ऋद्धियोगात्परान्नित्यमवमन्यन्ति मानवान्}
{पानपाः सर्वभक्षाश्च परुषाः कटुका नराः}


\twolineshloka
{एवंयुक्तसमाचाराक दण्डिता यमशासनैः}
{कथंचित्प्राप्य मानुष्यं चण्डालाः सम्भवन्ति ते}


\twolineshloka
{नीचकर्मरताश्चैव सर्वेषां मलहारकाः}
{परेषां वन्दनपरास्ते भवन्त्येव मानिनः}


\threelineshloka
{विरूपाः पापयोनिस्थाः स्पर्शनादिविवर्जिताः}
{कुवृत्तिमुपजीवन्ति भुत्वा ते रजकादयः}
{पुराऽतिमानदोषात्तु भुञ्जते स्वकृतं फलम्}


\fourlineindentedshloka
{तानप्यवस्ताकृपणांश्चण्डालानपि बुद्धिमान्}
{न च निन्देन्नापि कुप्येद्भुञ्जते स्वकृतं फलम्}
{चण्डाला अपि तां जातिं शोचन्तः शुद्धिमाप्नुयुः ॥उमोवाच}
{}


\twolineshloka
{भगवन्मानुषाः केचिदाशापाशशतैर्वृताः}
{परेषां द्वारि तिष्ठन्ति प्रतिषिद्धाः प्रवेशने}


\threelineshloka
{द्रष्टुं ज्ञापयितुं चैव न लभन्ते च यत्नतः}
{केन कर्मविपाकेन तन्मे शंसितुमर्हसि ॥महेश्वर उवाच}
{}


\twolineshloka
{ये पुरा मानुषा देवि ऐश्वर्यस्थानसंयुताः}
{संवादं तु न कुर्वन्ति परैरैश्वर्यमोहिताः}


\twolineshloka
{द्वाराणि न ददत्येव लोभमोहादिभिर्वृताः}
{अवस्थामोहसंयुक्ताः स्वार्थमात्रपरायणाः}


\twolineshloka
{सर्वभोगयुता वाऽपि सर्वेषां निष्फला भृशम्}
{अपि शक्ता न कुर्युस्ते परानुग्रहकारणात्}


\twolineshloka
{निर्दयाश्चैव निर्द्वारा भोगैश्वर्यगतिं प्रति}
{एवंयुक्तसमाचाराः पुनर्जन्मनि शोभने}


\twolineshloka
{यदि चेन्मानुषं जन्म लभेरंस्ते तथाविधाः}
{दुर्गता दुरवस्थाश्च कर्मव्याक्षेपसंयुताः}


\twolineshloka
{अभिधावन्ति ते सर्वे तमर्थमभिवेदिनः}
{राज्ञां वा राजमात्राणां द्वारि तिष्ठन्ति वारिताः}


\threelineshloka
{कर्म विज्ञापितुं द्रष्टुं न लभन्ते कथञ्चन}
{प्रवेष्टुमपि ते द्वारं बहिस्तिष्ठन्ति काङ्क्षया ॥उमोवाच}
{}


\twolineshloka
{भगवन्मानुषाः केचिन्मनुष्येषु बहुष्वपि}
{सहसा नष्टसर्वस्वा भ्रष्टकोशपरिग्रहाः}


\threelineshloka
{दृश्यन्ते मानुषाः केचिद्राजचोरोदकादिभिः}
{केन कर्मविपाकेन तन्मे शंसितुमर्हसि ॥महेश्वर उवाच}
{}


\twolineshloka
{ये पुरा मानुषा देवि आसुरं भावमाश्रिताः}
{परेषां वृत्तिनाशं तु कुर्वते द्वेषलोभतः}


\twolineshloka
{उत्कोचनपराश्चैव पिशुनाश्च तथाविधाः}
{परद्रव्यहरा घोराश्चौर्याद्वाऽन्येन कर्मणा}


\twolineshloka
{निर्दया निरनुक्रोशाः परेषां वृत्तिनाशकाः}
{नास्तिकाऽनृतभूयिष्ठाः परद्रव्यापहारिणः}


\twolineshloka
{एवंयुक्तसमाचारा दण्डिता यमशासनैः}
{निरयस्थाश्चिरं कालं तत्र दुःखसमन्विताः}


\twolineshloka
{यदि चेन्मानुषं जन्म लभेरंस्ते तथाविधाः}
{तत्रस्थाः प्राप्नुवन्त्येव सहसा द्रव्यवाशनम्}


\threelineshloka
{कष्टं तत्प्राप्नुवन्त्येव कारणाकारणादपि}
{नाशं विनाशं द्रव्याणामुपघातं च सर्वशः ॥उमोवाच}
{}


\twolineshloka
{भगवन्मानुषाः केचिद्बान्धवैः सहसा पृथक्}
{कारणादेव सहसा सर्वेषां प्राणनाशनम्}


\threelineshloka
{शस्त्रेण वाऽन्यथा वाऽपि प्राप्नुवन्ति वधं नराः}
{केन कर्मविपाकेन तन्मे शंसितुमर्हसि ॥महेश्वर उवाच}
{}


\twolineshloka
{ये पुरा मनुजा देवि घोरकर्मरतानृताः}
{आसुराः प्रायशो मूर्खाः प्राणिहिंसाप्रिया भृशं}


\threelineshloka
{निर्दयाः प्राणिहिंसायां तथा प्राणिविघातकाः}
{विश्वस्तघातकाश्चैव तथा सुप्तविघातकाः}
{प्रायशोऽनृतभूयिष्ठा नास्तिका मांसभोजनाः}


\twolineshloka
{एवंयुक्तसमाचाराः कालधर्मं गताः पुनः}
{दण्डिता यमदण्डेन निरयस्थाश्चिरं प्रिये}


\threelineshloka
{तिर्यग्योनिं पुनः प्राप्य तत्र दुःखपरिक्षयात्}
{यदि चेन्मानुपं जन्म लभेरंस्ते तथाविधाः}
{तत्र ते प्राप्नुवन्त्येव वधबन्धान्यथा तथा}


\twolineshloka
{आढ्या वा दुर्गता वाऽपि भुञ्जते स्वकृतं फलम्}
{सुप्ता मत्ताश्च विश्वस्तास्तथा ते प्राप्नुवन्त्युत}


\threelineshloka
{प्राणवाधकृतं दुःखं बान्धवैः सहसा पृथक्}
{पुत्रदारविनाशं वा शस्त्रेणान्येन वा वधम् ॥उमोवाच}
{}


\fourlineindentedshloka
{भगवन्मानुषाः केचिद्राजनीतिविशार दैः}
{दण्ड्यन्ते मानुषे लोके मानुषाः सर्वतोभयाः}
{केन कर्मविपाकेन तन्मे शंसितुमर्हसि ॥महेश्वर उवाच}
{}


\threelineshloka
{ये पुरा मनुजा देवि मानुषांश्चेतराणि वा}
{क्लिष्टघातेन निघ्नन्ति प्राणान्प्राणिषु निर्दयाः}
{आसुर घोरकर्माणः क्रूरदण्डवधप्रियाः}


\twolineshloka
{ये दण्डयन्त्यदण्ड्यांश्च राजानः कोपमोहिताः}
{हिंसाहङ्कारपरुषा मांसादा नास्तिकाशुभाः}


\twolineshloka
{केचित्स्त्रीपुरुषघ्नाश्च गुरुघ्नाश्च तथा प्रिये}
{एवंयुक्तसमाचारा प्राणिधर्मं गताः पुनः}


\twolineshloka
{दण्डिता यमदण्डेन निरयस्थाश्चिरं प्रिये}
{पूर्वजन्मकृतं कर्म भुञ्जते तदिह प्रजाः}


\twolineshloka
{इहैव यत्कर्म कृतं तत्परत्र फलत्युत}
{एषा व्यवस्थितिर्देवि मानुषेष्वेव दृश्यते}


\twolineshloka
{न चर्षीणां न देवानाममृतत्वात्तपोबलात्}
{तैरेकेन शरीरेण भुज्यते कर्मणः फलम्}


\twolineshloka
{न तथा मानुषाणां स्यादन्तर्धाय भवेद्धि तत् ॥उमोवाच}
{}


\twolineshloka
{किमर्थं मानुषा लोके दण्ड्यन्ते पृथिवीश्वरैः}
{कृतापराधमुद्दिश्य हन्ता हर्ताऽयमित्युत}


\twolineshloka
{पुत्रार्थी पुत्रकामेष्ट्या इहैव लभते सुतान्}
{तैरेव हि शरीरेण भुञ्जन्ते कर्मणां फलम्}


\threelineshloka
{दृश्यन्ते मानुषे लोके तद्भवान्नानुमन्यते}
{एतन्मे संशयस्थानं तन्मे त्वं छेत्तुमर्हसि ॥महेश्वर उवाच}
{}


\twolineshloka
{स्थाने संशयितं देवि तत्त्वं शृणु समाहिता}
{कर्म कर्मफलं चेति युगपद्भुवि नेष्यते}


\twolineshloka
{ये त्वयाऽभिहिता देवि हन्ता हर्ताऽयमित्यपि}
{तेषां तत्पूर्वकं कर्म दण्ड्यते यत्र राजभिः}


\twolineshloka
{देवि कर्म कृतं चैषां हेतुर्भवति शासने}
{अपराधापरेशेन राजा दण्डयति प्रजाः}


\twolineshloka
{इह लोके व्यवस्थार्थं राजभिर्दण्डनं स्मृतम्}
{उद्वेजनार्थं शेषाणामपराधं तमुद्दिशन्}


\threelineshloka
{पुराकृतफलं दण्डो दण्ड्यमानस्य तद्ध्रुवम्}
{प्रागेव च मया प्रोक्तं तत्र निःसंशया भव ॥उमोवाच}
{}


\threelineshloka
{भगवन्भुवि मर्त्यानां दण्डितानां नरेश्वरैः}
{दण्डेनैव तु तेनेह पापनाशो भवेन्न वा}
{}


\twolineshloka
{एतन्मया संशयितं तद्भवांश्छेत्तुमर्हति ॥महेश्वर उवाच}
{स्थाने संशयितं देवि शृणु तत्वं समाहिता}


\twolineshloka
{ये नृपैर्दण्डिता भूमावपराधापदेशतः}
{यमलोके न दण्ड्यन्ते तत्र ते यमदण्डनैः}


\threelineshloka
{अदण्डिता वा ये मिथ्या मिथ्या वा दण्डिता भुवि}
{तान्यमो दण्डयत्येव स हि वेद कृताकृतम्}
{नातिक्रमेद्यमं कश्चित्कर्म कृत्वेह मानुषः}


\threelineshloka
{राजा यमश्च कुर्वाते दण्डमात्रं तु शोभने}
{उभाभ्यां यमराजभ्यां दण्डितोऽदण्डितोपि वा}
{पश्चात्कर्मफलं भुङ्क्ते नरके मानुषेषु वा}


\threelineshloka
{नास्ति कर्मफलच्छेत्ता कश्चिल्लोकत्रयेऽपि च}
{इति ते कथितं सर्वं निर्विशङ्का भव प्रिये ॥उमोवाच}
{}


\twolineshloka
{किमर्थं दुष्कृतं कृत्वा मानुषा भुवि नित्यशः}
{पुनस्तत्कर्मनाशाय प्रायश्चित्तानि कुर्वते}


\fourlineindentedshloka
{सर्वपापहरं चेति हयमेधं वदन्ति च}
{प्रायश्चित्तानि चान्यानि पापनाशाय कुर्वते}
{तस्मान्मया संशयितं त्वं तच्छेत्तुमिहार्हसि ॥महेश्वर उवाच}
{}


\twolineshloka
{स्थाने संशयितं देवि शृणु तत्वं समाहिता}
{संशयो हि महानेव पूर्वेषां च मनीषिणाम्}


\twolineshloka
{द्विधा तु क्रियते पापं सद्भिश्चासद्भिरेव च}
{अभिसन्धाय वा नित्यमन्यथा वा यदृच्छया}


\twolineshloka
{केवलं चाभिसन्धाय संरम्भाच्च करोति यत्}
{कर्मणस्तस्य नाशस्तु न कथंचन विद्यते}


\twolineshloka
{अभिसन्धिकृतस्यैव नैव नाशोस्ति कर्मणः}
{अश्वमेधसहस्रैश्च प्रायश्चित्तशतैरपि}


\twolineshloka
{अन्यथा यत्कृतं पापं प्रमादाद्वा यदृच्छया}
{प्रायश्चित्ताश्वमेधाब्यां श्रेयसा तत्प्रणश्यति}


\twolineshloka
{लोकसंव्यवहारार्थं प्रायश्चित्तादिरिष्यते}
{विद्ध्येवं पापके कार्ये निर्विशङ्का भव प्रिये}


\twolineshloka
{इति ते कथितं देवि भूयः श्रोतुं किमिच्छसि ॥उमोवाच}
{}


\fourlineindentedshloka
{भगवन्देवदेवेश मानुषाश्चेतरा अपि}
{म्रियन्ते मानुषा लोके कारणाकारणादपि}
{केन कर्मविपाकेन तन्मे शंसितुमर्हसि ॥महेश्वर उवाच}
{}


\twolineshloka
{ये पुरा मनुजा देवि कारणाकारणादपि}
{यथाऽसुभिर्वियुज्यन्ते प्राणिनः प्राणिनिर्दर्यः}


\twolineshloka
{तथैव ते प्राप्नुवन्ति यथैवात्मकृतं फलम्}
{विषदास्तु विषेणैव शस्त्रैः शस्त्रेण घातकाः}


\threelineshloka
{एवमेव यथा लोके मानुषान्घ्नन्ति मानुषाः}
{कारणेनैव तेनाथ तता स्वप्राणनाशनम्}
{प्राप्नुवन्ति पुनर्देवि नास्ति तत्र विचारणा}


\twolineshloka
{इति ते कथितं सर्वं कर्मपाकफलं प्रिये}
{भूयस्तव समासेन कथयिष्यामि तच्छृणु}


\twolineshloka
{सत्यप्रमाणकरणान्नित्यमव्यभिचारि च}
{यैः पुरा मनुजैर्देवि यस्मिन्काले यथा कृतम्}


\twolineshloka
{येनैव कारणेनापि कर्म यत्तु शुभाशुभम्}
{तस्मन्काले तथा देवि कारणेनैव तेन तु}


\twolineshloka
{प्राप्नुवन्ति नराः प्रेत्य निःसन्देहं शुभाशुभम्}
{इति सत्यं प्रजानीहि लोके तत्र विधिं प्रति}


\twolineshloka
{कर्मकर्ता नरो भोक्ता स नास्ति दिवि वा भुवि}
{न शक्यं कर्म चाभोक्तुं सदेवासुरमानुषैः}


\twolineshloka
{कर्मणा ग्रथितो लोक आदिप्रभृति वर्तते}
{एतदुद्देशतः प्रोक्तं कर्मपाकफलं प्रति}


\threelineshloka
{यदन्यच्च मया नोक्तं यस्मिंस्ते कर्मसङ्ग्रहे}
{बुद्धितर्केण तत्सर्वं तथा वेदितुमर्हसि}
{कथितं श्रोतुकामाया भूयः श्रोतुं किमिच्छसि}


\chapter{अध्यायः २२४}
\twolineshloka
{भगवन्देवदेवेश लोकपालनमस्कृत}
{प्रसादात्ते महादेव श्रुता मे कर्मणां गतिः}


\twolineshloka
{सङ्गृहीतं च तत्सर्वं तत्वतोऽमृतसंनिभम्}
{कर्मणा ग्रथितं सर्वमिति वेद शुभाशुभम्}


\twolineshloka
{गोवत्सवच्च जननीं निम्नं सलिलवत्तथा}
{कर्तारं स्वकृतं कर्म नित्यं तदनुधावति}


\twolineshloka
{कृतस्य कर्मणश्चेह नाशो नास्तीति निश्चयः}
{अशुभस्य शुभस्यापि तदप्युपगतं मया}


\threelineshloka
{भूय एव महादेव वरद प्रीतिवर्धन}
{कर्मणां गतिमाश्रित्य संशयान्मोक्तुमर्हसि ॥महेश्वर उवाच}
{}


\threelineshloka
{यत्ते विवक्षितं देवि गुह्यमप्यसितेक्षणे}
{तत्सर्वं निर्विशंका त्वं पृच्छ मां शुभलक्षणे ॥उमोवाच}
{}


\twolineshloka
{एवं व्यवस्थिते लोके कर्ममां वृषभध्वज}
{कृत्वा तत्पुरुषः कर्म शुभं वा यदि वेतरत्}


\threelineshloka
{कर्मणः सुकृतस्येह कदा भुङ्क्ते फलं पुनः}
{इह वा प्रेत्य वा देव तन्मे शंसितुमर्हसि ॥महेश्वर उवाच}
{}


\twolineshloka
{स्थाने संशयितं देवि तद्धि गुह्यतमं नृषु}
{त्वत्प्रियार्थं प्रवक्ष्यामि देवि गुह्यं शुभानने}


\threelineshloka
{पूर्वदेहकृतं कर्म भुञ्जते तदिह प्रजाः}
{इहैव यत्कृतं पुंसां तत्परत्र फलिष्यतै}
{एषा व्यवस्थितिर्देवि मानुषेष्वेव दृश्यते}


\threelineshloka
{देवानामसुराणां च अमरत्वात्तपोबलात्}
{एकेनैव शरीरेण भुज्यते कर्मणां फलम्}
{मानुषैर्न तथा देवि अन्तरं त्वेतदिष्यते}


\chapter{अध्यायः २२५}
\twolineshloka
{भगवन्भगनेत्रघ्न मानुषाणां विचेष्टितम्}
{सर्वमात्मकृतं चेति श्रुतं मे भगवन्मतम्}


\fourlineindentedshloka
{लोके ग्रहकृतं सर्वं मत्वा कर्म शुभाशुभम्}
{तदेव ग्रहनक्षत्रं प्रायशः पर्युपासते}
{एष मे संशयो देव तं मे त्वं छेत्तुमर्हसि ॥महेश्वर उवाच}
{}


% Check verse!
स्थाने संशयितं देवि शृणु तत्वविनिश्चयम्
\twolineshloka
{नक्षत्राणि ग्रहाश्चैव शुभाशुभनिवेदकाः}
{मानवानां महाभागे न तु कर्मकराः स्वयम्}


\twolineshloka
{प्रजानां तु हितार्थाय शुभाशुभविधिं प्रति}
{अनागतमतिक्रान्तं ज्योतिश्चक्रेण बोध्यते}


\twolineshloka
{किन्तु तत्र शुभं कर्म सुग्रहैस्तु निवेद्यते}
{दुष्कृतस्याशुभैरेव समावायो भवेदिति}


\twolineshloka
{तस्मात्तु ग्रहवैषम्ये विषमं कुरुते जनः}
{ग्रहसाम्ये शुभं कुर्याज्ज्ञात्वाऽऽत्मानं तथा कृतम्}


\twolineshloka
{केवलं ग्रहनक्षत्रं न करोति शुभाशुभम्}
{सर्वमात्मकृतं कर्म लोकवादो ग्रहा इति}


\threelineshloka
{पृथग्ग्रहाः पृथक्कर्ता कर्ता स्वं भुञ्जते फलम्}
{इति ते कथितं सर्वं विशङ्कां जहि शोभने ॥उमोवाच}
{}


\fourlineindentedshloka
{भगवन्विविधं कर्म कृत्वा जन्तुः शुभाशुभम्}
{किं तयोः पूर्वकतरं भुङ्क्ते जन्मान्तरे पुनः}
{एष मे संशयो देव तं मे त्वं छेत्तुमर्हसि ॥महेश्वर उवाच}
{}


\twolineshloka
{स्थाने संशयितं देवि तत्ते वक्ष्यामि तत्वतः ॥अशुभं पूर्वमित्याहुरपरे शुभमित्यपि}
{}


% Check verse!
मिथ्या तदुभयं प्रोक्तं केवलं तद्ब्रवीमि ते
\twolineshloka
{मानुषे तु पदे कर्म युगपद्भुज्यते सदा}
{यथाकृतं यथायोगमुभयं भुज्यते क्रमात्}


\twolineshloka
{भुञ्जानाश्चापि दृश्यन्ते क्रमशो भुवि मानवाः}
{ऋद्धिं हानिं सुखं दुःखं तत्सर्वमुभयं भयम्}


\twolineshloka
{दुःखान्यनुभवन्त्याढ्या दरिद्राश्च सुखानि च}
{यौगपद्याद्धि भुञ्जाना दृश्यन्ते लोकसाक्षिकम्}


\twolineshloka
{नरके स्वर्गलोके च न तथा संस्थितिः प्रिये}
{नित्यं दुःखं हि नरके स्वर्गे नित्यं सुखं तथा}


\twolineshloka
{शुभाशुभानामाधिक्यं कर्मणां तत्र सेव्यते}
{निरन्तरं सुखं दुःखं स्वर्गे च नरके भवेत्}


\threelineshloka
{तत्रापि सुमहद्भुक्त्वा पूर्वमल्पं पुनः शुभे}
{एतत्ते सर्वमाख्यातं किं भूयः श्रोतुमिच्छसि ॥उमोवाच}
{}


\threelineshloka
{भगवन्प्राणिनो लोके म्रियन्ते केन हेतुना}
{जाताजाता न तिष्ठन्ति तन्मे शंसितुमर्हसि ॥महेश्वर उवाच}
{}


\twolineshloka
{तदहं ते प्रवक्ष्यामि शृणु सत्यं समाहिता}
{आत्मा कर्मक्षयाद्देहं यथा मुञ्चति तच्छृणु}


\twolineshloka
{शरीरात्मसमाहारो जन्तुरित्यभिधीयते}
{तत्रात्मानं नित्यमाहुरनित्यं क्षेत्रिमुच्यते}


\twolineshloka
{एवं कालेन सङ्क्रान्तं शरीरं जर्झरीकृतम्}
{अकर्मयोग्यं संशीर्णं त्यक्त्वा देही ततो व्रजेत्}


\twolineshloka
{नित्यस्यानित्यसंत्यागाल्लोके तन्मरणं विदुः}
{कालं नातिक्रमेरन्हि सदेवासुरमानवाः}


\twolineshloka
{यथाऽऽकाशे न तिष्ठेत द्रव्यं किञ्चिदचेतनम्}
{तथा धावति कालोऽयं क्षणं किञ्चिन्न तिष्ठति}


\threelineshloka
{स पुनर्जायतेऽन्यत्र शरीरं नवमाविशन्}
{एवंलोकगतिर्नित्यमादिप्रभृति वर्तते ॥उमोवाच}
{}


\twolineshloka
{भगवन्प्राणिनो बाला दृश्यन्ते मरणं गताः}
{अतिवृद्धाश्च जीवन्तो दृश्यन्ते चिरजीविनः}


\threelineshloka
{केवलं कालमरणं न प्रमाणं महेश्वर}
{तस्मान्मे संशय ब्रूहि प्राणिनां जीवकारणम् ॥महेश्वर उवाच}
{}


\twolineshloka
{शृणु तत्कारणं देवि निर्णयस्त्वेक एव सः}
{जीर्णत्वमात्रं कुरुते कालो देहं न पातयेत्}


\twolineshloka
{जीर्णे कर्मणि संघातः स्वयमेव विशीर्यते}
{पूर्वकर्मप्रमाणेन जीवितं मृत्युरेव वा}


\twolineshloka
{यावत्पूर्वकृतं कर्म तावज्जीवति मानवः}
{तत्र कर्मवशाद्बाला म्रियन्ते कालसंक्षयात्}


\threelineshloka
{चिरं जीवन्ति वृद्धाश्च तथा कर्मप्रमाणतः}
{इति ते कथितं देवि निर्विशङ्का भव प्रिये ॥उमोवाच}
{}


\threelineshloka
{भगवन्केन वृत्तेन भवन्ति चिरजीविनः}
{अल्पायुषो नराः केन तन्मे शंसितुमर्हसि ॥महेश्वर उवाच}
{}


\twolineshloka
{शृणु तत्सर्वमखिलं गुह्यं पथ्यतरं नृणाम्}
{येन वृत्तेन सम्पन्ना भवन्ति चिरजीविनः}


\twolineshloka
{अहिंसा सत्यवचनमक्रोधः क्षीन्तिरार्जवम्}
{गुरूणां नित्यशुश्रूषा वृद्धानामपि पूजनम्}


\twolineshloka
{शौचादकार्यसंत्यागात्सदा पथ्यस्य भोजनम्}
{एवमादिगुणं वृत्तं नराणां दीर्घजीविनाम्}


\threelineshloka
{तपसा ब्रह्मचर्येण रसायननिषेवणात्}
{उदग्रसत्त्वा बलिनो भवन्ति चिरजीविनः}
{स्वर्गे वा मानुषे वाऽपि चिरं तिष्ठन्ति धार्मिकाः}


\twolineshloka
{अपरे पापकर्माणः प्रायशोऽनृतवादिनः}
{हिंसाप्रिया गुरुद्विष्टा निष्क्रियाः शौचवर्जिताः}


\twolineshloka
{नास्तिका घोरकर्माणः सततं मांसपानपाः}
{पापाचारा गुरुद्विष्टाः कोपनाः कलहप्रियाः}


\threelineshloka
{एवमेवाशुभाचारास्तिष्ठन्ति नरके चिरम्}
{तिर्यग्योनौ तथाऽत्यन्तमल्पास्तिष्ठन्ति मानवाः}
{तस्मादल्पायुषो मर्त्यास्तादृशाः सम्भवन्ति ते}


\twolineshloka
{अगम्यदेशगमनादपथ्यानां च भोजनात्}
{आयुःक्षयो भवेन्नॄणामायुःक्षयकरा हि ते}


\twolineshloka
{भवन्त्यल्पायुषस्तैस्तैरन्यथा चिरजीविनः}
{एतत्ते कथितं सर्वं भूयः श्रोतुं किमिच्छसि}


\chapter{अध्यायः २२६}
\twolineshloka
{देवदेव महादेव श्रुतं मे भगवन्निदम्}
{आत्मनो जातिसम्बन्धं ब्रूहि स्त्रीपुरुषान्तरम्}


\threelineshloka
{स्त्रीप्राणाः पुरुषप्राणा एकतः पृथगेव वा}
{एष मे संशयो देव तं मे छेत्तुं त्वमर्हसि ॥महेश्वर उवाच}
{}


\twolineshloka
{तदहं ते प्रवक्ष्यामि शृणु सर्वं समाहिता}
{स्त्रीत्वं पुंस्त्वमिति प्राणे स्थितिर्नास्ति शुभेक्षणे}


\twolineshloka
{निर्विकारः सदैवात्मा स्त्रीत्वं पुंस्त्वं न चात्मनि}
{कर्मप्रकारेण तथा जात्यां जात्यां प्रजायते}


\threelineshloka
{कृत्वा कर्म पुमान्स्त्री वा स्त्री पुमानपि जायते}
{स्त्रीभावं यत्पुमान्कृत्वा कर्मणा प्रमदा भवेत् ॥उमोवाच}
{}


\threelineshloka
{भगवन्सर्वलोकेश कर्मात्मा न करोति चेत्}
{कोऽन्यः कर्मकरो देहे तन्मे त्वं वक्तुमर्हसि ॥महेश्वर उवाच}
{}


\twolineshloka
{शृणु भामिनि कर्तारमात्मा हि न च कर्मकृत्}
{प्रकृत्या गुणयुक्तेन क्रियते कर्म नित्यशः}


\twolineshloka
{शरीरं प्राणिनां लोके यथा पित्तकफानिलैः}
{व्याप्तमेभिस्त्रिभिर्दोषैस्तथा व्याप्तं त्रिभिर्गुणैः}


\twolineshloka
{सत्वं रजस्तमश्चैव गुणास्त्वेते शरीरिणः}
{प्रकाशात्मकमेतेषां सत्वं सततमिष्यते}


\twolineshloka
{रजो दुःखत्मकं तत्र तमो मोहात्मकं स्मृतम्}
{त्रिभिरेतैर्गुणैर्युक्तं लोके कर्म प्रवर्तते}


\twolineshloka
{सत्यं प्राणिदया शौचं श्रेयः प्रीतिः क्षमा दमः}
{एवमादि तथाऽन्यश्च कर्म सात्विकमुच्यते}


\threelineshloka
{दाक्ष्यं कर्मपरत्वं च लोभो मोहो विधिं प्रति}
{कलत्रसङ्गो माधुर्यं नित्यमैश्वर्यलुब्धता}
{रजसश्चोद्भवं चैतत्कर्म नानाविधं सदा}


\threelineshloka
{अनृतं चैव पारुष्यं धृतिर्विद्वेषिता भृशम्}
{हिंसाऽसत्यं च नास्तिक्यं निद्रालस्यभयानि च}
{तमसश्चोद्भवं चैतत्कर्म पापयुतं तथा}


\twolineshloka
{तस्माद्गुणमयः सर्वः कार्यारम्भः शुभाशुभः}
{तस्मादात्मानमव्यग्रं विद्ध्यकर्तारमव्ययम्}


\threelineshloka
{सात्विकाः पुण्यलोकेषु राजसा मानुषे पदे}
{तिर्यग्योनौ च नरके तिष्ठेयुस्तामसा नराः ॥उमोवाच}
{}


\threelineshloka
{किमर्थमात्मा भिन्नेऽस्मिन्देहे शस्त्रेण वा हते}
{स्वयं प्रयास्यति तदा तन्मे शंसितुमर्हसि ॥महेश्वर उवाच}
{}


\twolineshloka
{तदहं ते प्रक्ष्यामि शृणु कल्याणि कारणम्}
{एतन्निर्णायकैश्चापि मुह्यन्ते सूक्ष्मबुद्धिभिः}


\twolineshloka
{कर्मक्षये तु सम्प्राप्ते प्राणिनां जन्मधारिणाम्}
{उपद्रवो भवेद्देहे येन केनापि हेतुना}


\twolineshloka
{तन्निमित्तं शरीरी तु शरीरं प्राप्य संक्षयम्}
{अपयाति परित्यज्य ततः कर्मवशेन सः}


\twolineshloka
{देहक्षयेपि नैवात्मा वेदनाभिर्न चाल्यते}
{तिष्ठेत्कर्मफलं यावद्व्रजेत्कर्मक्षये पुनः}


\twolineshloka
{आदिप्रभृति लोकेऽस्मिन्नेवमात्मगतिः स्मृता}
{एतत्ते कथितं देवि किं भूयः श्रोतुमिच्छसि}


\chapter{अध्यायः २२७}
\twolineshloka
{भगवन्देवदेवेश कर्मणैव शुभाशुभम्}
{यथायोगं फलं जन्तुः प्राप्नोतीति विनिश्चयः}


\threelineshloka
{परेषां विप्रियं कुर्वन्यथा सम्प्राप्नुयाच्छुभम्}
{यद्येतदस्मिंश्चेद्देहे तन्मे शंसितुमर्हसि ॥महेश्वर उवाच}
{}


\twolineshloka
{तदप्यस्ति महाभागे अभिसन्धिबलान्नृणम्}
{हितार्थं दुःखमन्येषां कृत्वा सुखमवाप्नुयात्}


\threelineshloka
{दण्डयन्भर्त्सयन्राजा जनान्पुण्यमवाप्नुयात्}
{गुरुः सन्तर्जयञ्शिष्यान्भर्ता भृत्यजनान्स्वकान्}
{उन्मार्गप्रतिपन्नांश्च शास्ता धर्मफलं लभेत्}


\threelineshloka
{चिकित्सकश्च दुःखानि जनयन्हितमाप्नुयात्}
{यज्ञार्थं पशुहिंसां च कुर्वन्नपि न लिप्यते}
{एवमन्ये सुमनसो हिंसकाः स्वर्गमाप्नुयुः}


\twolineshloka
{एकस्मिन्निहते भद्रे बहवः सुखमाप्नुयुः}
{तस्मिन्हते भवेद्धर्मः कुत एव तु पातकम्}


\fourlineindentedshloka
{अहिंसतेति हत्वा तु शुद्धे कर्मणि गौरवात्}
{अभिसन्धेरजिह्मत्वाच्छुद्धे धर्मस्य गौरवात्}
{एतत्कृत्वा तु पापेभ्यो न दोषं प्राप्नुयुः क्वचित् ॥उमोवाच}
{}


\threelineshloka
{चतुर्विधानां जन्तूनां कथं ज्ञानमिह स्मृतम्}
{कृत्रिमं तत्स्वभावं वा तन्मे शंसितुमर्हसि ॥महेश्वर उवाच}
{}


\twolineshloka
{स्थावरं जङ्गमं चैव जगद्द्विविधमुच्यते}
{चतस्रो योनयस्तत्र प्रजानां क्रमशो यथा}


\twolineshloka
{तेषामुद्भिदजा वृक्षा लतावल्ल्यश्च वीरुधः}
{दंशयूकादयश्चान्ये स्वेदजाः क्रिमिजातयः}


\twolineshloka
{पक्षिणश्छिद्कर्णाश्च प्राणिनस्त्वण्डजा मताः}
{मृगव्यालमनुष्यांश्च विद्धि तेषां जरायुजान्}


% Check verse!
एवं चतुर्विधां जातिमात्मा संसृत्य तिष्ठति
\twolineshloka
{स्पर्शेनैकेन्द्रियेणात्मा तिष्ठत्युद्भिदजेषु वै}
{शरीरस्पर्शरूपाभ्यां स्वेदजेष्वपि तिष्ठति}


% Check verse!
पञ्चभिश्चेन्द्रियद्वारैर्जीवन्त्यण्डजरायुजाः
\threelineshloka
{तथा भूम्यम्बुसंयोगाद्भवन्त्युद्भिदजाः प्रिये}
{शीकतोष्णयोस्तु संयोगाज्जायन्ते स्वेदजाः प्रिये}
{अण्डजाश्चापि जायन्ते संयोगात्क्लेदबीजयोः}


\twolineshloka
{शुक्लशोणितसंयोगात्सम्भवन्ति जरायुजाः}
{जरायुजानां सर्वेषां मानुषं पदमुत्तमम्}


\twolineshloka
{अतःपरं तमोत्पत्तिं शृणु देवि समाहिता}
{द्विविधं हि तमो लोके शार्वरं देहजं तथा}


\twolineshloka
{जोतिर्भिश्च तमो लोके नाशं गच्छति शार्वरम्}
{देहजं तु तमो लोके तैः समस्तैर्न शाम्यते}


\twolineshloka
{तमसस्तस्य नाशार्थं नोपायमधिजग्मिवान्}
{तपश्चचार वलिपुलं लोककर्ता पितामहः}


\threelineshloka
{चरतस्तु समुद्भूता वेदाः साङ्गाः सहोत्तराः}
{ताँल्लब्ध्वा मुमुदे ब्रह्मा लोकानां हितकाम्यया}
{देहजं तु तमो घोरमभूत्तैरेव नाशितम्}


\threelineshloka
{कार्याकार्यमिदं चेति वाच्यावाच्यमिदं त्विति}
{यदि चेन्न भवेल्लोके श्रुतं चारित्रदैशिकम्}
{पसुभिर्निर्विशेषं तु चेष्टन्ते मानुषा अपि}


\twolineshloka
{यज्ञादीनां समारम्भः श्रुतेनैव विधीयते}
{यज्ञस्य फलयोगेन देवलोकः समृद्ध्यते}


\twolineshloka
{प्रीतियुक्ताः पुनर्देवा मानुषाणां भवन्त्युत}
{एवं नित्यं प्रवर्धेते रोदसी च परस्परम्}


\twolineshloka
{लोकसन्धारणं तस्माच्छ्रुतमित्यवधारय}
{ज्ञानाद्विशिष्टं जन्तूनां नास्ति लोकत्रयेऽपि च}


\twolineshloka
{सहजं तत्प्रधानं स्यादपरं कृत्रिमं स्मृतम्}
{उभयं यत्र सम्पन्नं भवेत्तत्र तु शोभनम्}


\twolineshloka
{सम्प्रगृह्य श्रुतं सर्वं कृतकृत्यो भवत्युत}
{उपर्युपरि मर्त्यानां देववत्सम्प्रकाशते}


\twolineshloka
{कामं क्रोधं भयं दर्पमज्ञानं चैव बुद्धिजम्}
{तच्छ्रुतं नुदति क्षिप्रं यथा वायुर्बलाहकान्}


\twolineshloka
{अल्पमात्रं कृतो धर्मो भवेज्झानवतां महान्}
{महानपि कृतो धर्मो ह्यज्ञानान्निष्फलो भवेत्}


\threelineshloka
{परावरझो भूतानां ज्ञानवांस्तत्वविद्भवेत्}
{एवं श्रुतफलं सर्वं कथितं ते शुभक्षणे ॥उमोवाच}
{}


\fourlineindentedshloka
{भगवन्मानुषाः केचिज्जातिस्मरणसंयुताः}
{किमर्थमभिजायन्ते जानन्तः पौर्वदैहिकम्}
{एतन्मे तत्वतो देव मानुषेषु वदस्व भो ॥महेश्वर उवाच}
{}


% Check verse!
तदहं ते प्रवक्ष्यामि शृणु तत्वं समाहिता
\twolineshloka
{ये मृताः सहसा मर्त्या जायन्ते सहसा पुनः}
{तेषां पौराणिको बोधः कञ्चित्कालं हि तिष्ठति}


\fourlineindentedshloka
{तस्माज्जातिस्मरा लोके जायन्ते बोधसंयुताः}
{तेषां विवर्धतां संज्ञा स्वप्नवत्सा प्रणश्यति}
{परलोकस्य चास्तित्वे मूढानां कारणं च तत् ॥उमोवाच}
{}


\threelineshloka
{भगवन्मानुषाः केचिन्मृता भूत्वाऽपि सम्प्रति}
{निवर्तमाना दृश्यन्ते देहेष्वेव पुनर्नराः ॥महेश्वर उवाच}
{}


% Check verse!
तदहं ते प्रवक्ष्यामि कारणं शृणु शोभने
\twolineshloka
{प्राणैर्वियुज्यमानानां बहुत्वात्प्राणिनां वधे}
{तथैव नामसामान्याद्यमदूता नृणां प्रति}


\twolineshloka
{वहन्ति ते क्वचिन्मोहादन्यं मर्त्यं तु यामिकाः}
{निर्विकारं हि तत्सर्वं यमो वेद कृताकृतम्}


\fourlineindentedshloka
{तस्मात्संयमनीं प्राप्य यमेनैकेन मोक्षिताः}
{पुनरेव निवर्तन्ते शेषं भोक्तुं स्वकर्मणः}
{स्वकर्मण्यसमाप्ते तु निवर्तन्ते हि मानवाः ॥उमोवाच}
{}


\threelineshloka
{भगवन्सुप्तमात्रेण प्राणिनां स्वप्नदर्शनम्}
{किं तत्स्वभावमन्यद्वा तन्मे शंसितुमर्हसि ॥महेश्वर उवाच}
{}


\twolineshloka
{सुप्तानां तु मनश्चेष्टा स्वप्न इत्यभिधीयते}
{अनागतमतिक्रान्तं पश्यते सञ्चरन्मनः}


\twolineshloka
{निमित्तं च भवेत्तस्मात्प्राणिनां स्वप्नदर्शनम्}
{एतत्ते कथितं देवि भूयः श्रोतु किमिच्छसि}


\chapter{अध्यायः २२८}
\twolineshloka
{भगवन्सर्वभूतेश लोके कर्मक्रियापथे}
{दैवात्प्रवर्तते सर्वमिति केचिद्व्यवस्थिताः}


\fourlineindentedshloka
{अपरे चेष्टया चेति दृष्ट्वा प्रत्यक्षतः क्रियाम्}
{पक्षभेदे द्विधा चास्मिन्संशयस्थं मनो मम}
{तत्त्वं वद महादेव श्रोतुं कौतूहलं हि मे ॥महेश्वर उवाच}
{}


\twolineshloka
{तदहं ते प्रवक्ष्यामि शृणु तत्वं समाहिता}
{तदेवं कुरुते कर्म लोके देवि शुभाशुभम्}


\twolineshloka
{लक्ष्यते द्विविधं कर्म मानुषेष्वेव तच्छृणु}
{पुराकृतं तयोरेकमैहिकं त्वितरस्तथा}


\twolineshloka
{अदृष्टपूर्वं यत्कर्म तद्दैवमिति लक्ष्यते}
{विहीनं दृष्टकरणं तन्मानुषमिति स्मृतम्}


\twolineshloka
{मानुषं तु क्रियामात्रं दैवात्सम्भवते फलम्}
{एवं तदुभयं कर्म मानुषं विद्धि तन्नृषु}


\twolineshloka
{लौकिकं तु प्रवक्ष्यामि दैवमानुषनिर्मितम्}
{कृषौ तु दृश्यते कर्म कर्षणं वपनं तथा}


\twolineshloka
{रोपणं चैव लवनं यच्चान्यत्पौरुषं स्मृतम्}
{दैवादसिद्धिश्च भवेद्दुष्कृतं चास्ति पौरुषे}


\threelineshloka
{सुयत्नाल्लभ्यते कीर्तिर्दुर्यत्नादयशस्तथा}
{एवं लोकगतिर्देवि आदिप्रभृति वर्तते ॥उमोवाच}
{}


\fourlineindentedshloka
{भगवन्सर्वलोकेश सुरासुरनमस्कृत}
{कथमात्मा सदा गर्भं संविशेष्कर्मकारणात्}
{तन्मे वद महादेव तद्धि गुह्यं परं मतम् ॥महेश्वर उवाच}
{}


\twolineshloka
{शृणु भामिनि तत्सर्वं गुह्यानां परमं प्रिये}
{देवगुह्यादपि परमात्मगुह्यमिति स्मृतम्}


\twolineshloka
{देवासुरास्तन्न विदुरात्मनो हि गतागतम्}
{अदृश्यो हि सदैवात्मा जनैः सौक्ष्म्यान्निराश्रयात्}


\threelineshloka
{अतिमायेति मायानामात्ममाया सेदष्यते}
{सोयं चतुर्विधां जातिं संविशत्यात्ममायया}
{मैथुनं शोणितं बीजं दैवमेवात्र कारणम्}


\twolineshloka
{बीजशोणितसंयोगे यदा सम्भवते शुभे}
{तदाऽऽत्मा विशते गर्भमेवमण्डजरायुजे}


% Check verse!
एवं संयोगकाले तु आत्मा गर्भत्वमेयिवान्
\twolineshloka
{कलिलाज्जायते पिण्डं पिण्डात्पेश्यर्बुदं भवेत्}
{व्यक्तिभावगतं चैव कर्म त्वाश्रयते क्रमात्}


\twolineshloka
{एवं विवर्धमानेन कर्मात्मा सह वर्धते}
{एवमात्मगतिं विद्धि यन्मां पृच्छसि सुप्रभे}


% Check verse!
रोपणं चैव लवनं यच्चान्यत्पौरुषं स्मृतम्
\twolineshloka
{काले वृष्टिः सुवापं च प्ररोहः पक्तिरेव च}
{एवमादि तु यच्चान्यत्तद्दैवतमिति स्मृतम्}


\twolineshloka
{पञ्चभूतस्थितिश्चैव ज्योतिषामयनं तथा}
{अबुद्धिगम्यं यन्मर्त्यैर्हेतुभिर्वा न विद्यते}


\twolineshloka
{तादृशं कारणं दैवं शुभं वा यदि वेतरत्}
{यादृशं चात्मना शक्यं तत्पौरुषमिति स्मृतम्}


\threelineshloka
{केवलं फलनिष्पत्तिरेकेन तु न शक्यते}
{पौरुषेणैव दैवेनि युगपद्ग्रथितं प्रिये}
{तयोः समाहितं कर्म शीतोष्णं युगपत्तथा}


\twolineshloka
{पौरुषं तु तयोः पूर्वमारब्धव्यं विजानता}
{आत्मना तु न शक्यं हि न तथा कीर्तिमाप्नुयात्}


\twolineshloka
{खननान्मथनाल्लोके जलाग्निप्रापणं यथा}
{तथा पुरुषकारे तु दैवसम्पत्समाहिता}


\twolineshloka
{नरस्याकुर्वतः कर्म दैवसम्पन्न लभ्यते}
{तस्मात्सर्वसमारम्भो दैवमानुषनिर्मितः}


\twolineshloka
{असुरा राक्षसाश्चैव मन्यन्ते लोकनाशनाः}
{पश्यन्ते न च ते पापाः केवलं मांसभक्षणाः}


\twolineshloka
{प्रच्छादितं हि तत्सर्वं गूढमाया हि देवताः}
{तदहं ते प्रवक्ष्यामि देवि गुह्यं पुरस्सरम्}


\twolineshloka
{आदिकाले नराः सर्वे कृत्वा कर्म शुभाशुभम्}
{भुञ्जते पश्यमानास्ते वृत्तान्तं लोकयोर्द्वयोः}


\twolineshloka
{यथैवात्मकृतं विद्युर्देशान्तरगता नराः}
{विद्युस्तथैवान्तकाले सुकृतं पौर्वदैहिकम्}


\twolineshloka
{एवं व्यवस्थिते लोके सर्वे धर्मरताऽभवन्}
{अचिरेणैव कालेन स्वर्गः सम्पूरितस्तदा}


\twolineshloka
{देवानामपि सम्बाधं दृष्ट्वा ब्रह्माऽप्यचिन्तयत्}
{सञ्चरन्ते कथं स्वर्गं मानुषाः प्रविशन्ति हि}


\twolineshloka
{इत्येवमनुचिन्त्यैव मानुषान्सममोहयत्}
{तदाप्रभृति ते मर्त्या न विदुस्ते पुराकृतम्}


\twolineshloka
{कामक्रोधौ तु तत्काले मानुषेष्ववपातयत्}
{ताभ्यामभिहता मर्त्याः स्वर्गलोकं न पेदिरे}


\twolineshloka
{पुराकृतस्याविज्ञानात्कामक्रोधाभिपीडिताः}
{नैतदस्तीति मन्वाना विकारंश्चक्रिरे पुनः}


\twolineshloka
{अकार्यादिमहादोषानाहरन्त्यात्मकारणात्}
{विस्मृत्य धर्मकार्याणि परलोकभयं तदा}


\threelineshloka
{एवं व्यवस्थिते लोके कश्मलं समपद्यत}
{लोकानां चैव देवानां क्षयायैव तथा प्रिये}
{नरकाः पूरिताश्चासन्प्राणिभिः पापकारिभिः}


\threelineshloka
{पुनरेव तु तान्दृष्ट्वा लोककर्ता पितामहः}
{अचिन्तयत्तमेवार्थं लोकानां हितकारणात्}
{समत्वेन कथं लोके वर्तेतेति मुहुर्मुहुः}


\threelineshloka
{चिन्तयित्वा तदा ब्रह्मा ज्ञानेन तपसा प्रिये}
{अकरोज्ज्ञानदृश्यं तत्परलोकं न चक्षुषा ॥उमोवाच}
{}


\twolineshloka
{भगवन्मृतमात्रस्तु योयं जात इति स्मृतः}
{तथैव दृश्यते जातस्तत्रात्मा तु कथं भवेत्}


\threelineshloka
{गर्भादावेव संविष्ट आत्मा तु भगवन्मम}
{एष मे संशयो देव तन्मे छेत्तुं त्वमर्हसि ॥महेश्वर उवाच}
{}


\twolineshloka
{तदहं ते प्रवक्ष्यामि शृणु तत्वं समाहिता}
{}


\threelineshloka
{अन्यो गर्भगतो भूत्वा तत्रैव निधनं गतः}
{पुनरन्यच्छरीरं तत्प्रविश्य भुवि जायते}
{तत्वविन्नैव सर्वस्तु दैवयोगस्तु सम्भवेत्}


\twolineshloka
{सूतिकाया हितार्थं च मोहनार्थं च देहिनाम्}
{समकर्मविधानत्वादित्येवं विद्धि शोभने}


\fourlineindentedshloka
{काङ्क्षमाणास्तु नरकं भुक्त्वा केचित्प्रयान्ति हि}
{मायासंयामिका नाम यज्जन्ममरणान्तरे}
{इति ते कथितं देवि भूयः श्रोतुं किमिच्छसि ॥उमोवाच}
{}


\twolineshloka
{भगवन्सर्वलोकेश लोकनाथ वृषध्वज}
{नास्त्यात्मा कर्मभोक्तेति मृतो जन्तुर्न जायते}


\twolineshloka
{स्वभावाज्जायते सर्वं यथा वृक्षफलं तथा}
{यथोर्मयः सम्भवन्ति तथैव जगदाकृतिः}


\twolineshloka
{तपोदानानि यत्कर्म तत्र तद्दृश्यते वृथा}
{नास्ति पौनर्भवं जन्म इति केचिद्व्यवस्थिताः}


\twolineshloka
{परोक्षवचनं श्रुत्वा न प्रत्यक्षस्य दर्शनात्}
{तत्सर्वं नास्तिनास्तीति संशयस्थास्तथा परे}


\twolineshloka
{पक्षभेदान्तरे चास्मिंस्तत्वं मे वक्तुमर्हसि}
{उक्तं भगवता यत्तु तत्तु लोकस्य संस्थितिः}


\threelineshloka
{प्रश्नमेतत्तु पृच्छत्या रुद्राण्या परिषत्तदा}
{कौतूहलयुता श्रोतुं समाहितमनाऽभवत् ॥महेश्वर उवाच}
{}


\twolineshloka
{नैतदस्ति महाभागे यद्वदन्तीह नास्तिकाः}
{एतदेवाभिशस्तानां श्रुतविद्वेषिणां मतम्}


\twolineshloka
{सर्वमर्थं श्रुतं दृष्टं यत्प्रागुक्तं मया तव}
{तदाप्रभृति मर्त्यानां श्रुतमाश्रित्य पण्डिताः}


\twolineshloka
{कामान्संछिद्य परिगान्धृत्या वै परमासिना}
{अभियान्त्येव ते स्वर्गं पश्यन्तः कर्मणः पलम्}


\twolineshloka
{एवं श्रद्धाफलं लोके परतः सुमहत्फलम्}
{बुद्धिः श्रद्धा च विनयः कारणानि हितैषिणाम्}


\twolineshloka
{तस्मात्स्वर्गाभिगन्तारः कतिचित्त्वभवन्नराः}
{अन्ये करणहीनत्वान्नास्तिक्यं भावमाश्रिताः}


\twolineshloka
{श्रुतविद्वेषिणो मूर्खा नास्तिका दृढनिश्चयाः}
{निष्क्रियास्तु निरन्नादाः पतन्त्येवाधमां गतिम्}


\twolineshloka
{नास्त्यस्तीति पुनर्जन्म कवयोऽप्यत्र मोहिताः}
{नाधिगच्छन्ति तन्नित्यं हेतुवादशतैरपि}


\twolineshloka
{एषा ब्रह्मकृता माया दुर्विज्ञेया सुरासुरैः}
{किंपुनर्मानवैर्लोके ज्ञातुकामैः कुबुद्धिभिः}


\twolineshloka
{केवलं श्रद्धया देवि श्रुतमात्मनिविष्टया}
{ततोस्तीऽत्येव मन्तव्यं तथा हितमवाप्नुयात्}


\threelineshloka
{दैवगुह्येषु चान्येषु हेतुर्देवि निरर्थकः}
{बधिरान्धवदेवात्र वर्तितव्यं हितैषिणा}
{एतत्ते कथितं देवि ऋषिगुह्यं प्रजाहितम्}


\chapter{अध्यायः २२९}
\twolineshloka
{भगवन्सर्वलोकेश त्रिपुरार्दन शङ्कर}
{कीदृशा यमदण्डास्ते कीदृशाः परिचारकाः}


\fourlineindentedshloka
{कथं मृतास्ते गच्छन्ति प्राणिनो यमसादनम्}
{कीदृशं भवनं तस्य कथं दण्डयति प्रजाः}
{एतत्सर्वं महादेव श्रोतुमिच्छाम्यहं प्रभो ॥महेश्वर उवाच}
{}


\twolineshloka
{शृणु कल्याणि तत्सर्वं यत्ते देवि मनःप्रियम्}
{दक्षिणस्यां दिशि शुभे यमस्य सदनं महत्}


\twolineshloka
{विचित्रं रमणीयं च नानाभावसमन्वितम्}
{पितृभिः प्रेतसङ्घैश्च यमदूतैश्च सन्ततम्}


\twolineshloka
{प्राणिसङ्घैश्च बहुभिः कर्मवश्यैश्च पूरितम्}
{तत्रास्ते दण्डयन्नित्यं यमो लोकहिते रतः}


\twolineshloka
{मायया सततं वेत्ति प्राणिनां यच्छुभाशुभम्}
{मायया संहरंस्तत्र प्राणिसङ्घान्यतस्ततः}


\twolineshloka
{तस्य मायामयाः पाशा न वेद्यन्ते सुरासुरैः}
{को हि मानुषमात्रस्तु देवस्य चरितं महत्}


\threelineshloka
{एवं संवसतस्तस्य यमस्य परिचारकाः}
{गृहीत्वा सन्नयन्त्येव प्राणिनः क्षीणकर्मणः}
{यन केनापदेशेन त्वपदेशसमुद्भवाः}


\twolineshloka
{कर्मणा प्राणिनो लोके उत्तमाधममध्यमाः}
{यथार्हं तान्समादाय नयन्ति यमसादनम्}


\threelineshloka
{धार्मिकानुत्तमान्विद्धि स्वर्गिणस्ते यथाऽमराः}
{त्रिषु जन्म लभन्ते ये कर्मणा मध्यमाः स्मृताः}
{तिर्यङनरकगन्तारो ह्यधमास्ते नराधमाः}


\twolineshloka
{पन्थानस्त्रिविधा दृष्टाः सर्वेषां गतजीविनाम्}
{रमणीयं निराबाधं दुर्दर्शमिति नामतः}


\twolineshloka
{रमणीयं तु यन्मार्गं पताकाध्वजसङ्कुलम्}
{धूपितं सिक्तसंमृष्टं पुष्पमालाभिसङ्कुलम्}


\twolineshloka
{मनोहरं सुखस्पर्शं गच्छतामेव तद्भवेत्}
{निराबाधं यथालोकं सुप्रशस्तं कृतं भवेत्}


\threelineshloka
{तृतीयं यत्तु दुर्दर्शं दुर्गन्धि तमसा वृतम्}
{परुषं शर्कराकीर्णं श्वदंष्ट्राबहुलं भृशम्}
{किमिकीटसमाकीर्णं भजतामतिदुर्गमम्}


\twolineshloka
{मार्गैरेवं त्रिभिर्नित्यमुत्तमाधममध्यमान्}
{सन्नयन्ति यथा काले तन्मे शृणु शुचिस्मिते}


\threelineshloka
{उत्तमानन्तकाले तु यमदूताः सुसंवृताः}
{नयन्ति सुखमादाय रमणीयपथेन वै ॥उमोवाच}
{}


\threelineshloka
{भगवंस्तत्र चात्मानं त्यक्तदेहं निराश्रयम्}
{अदृश्यं कथमादाय सन्नयन्ति यमान्तिकम् ॥महेश्वर उवाच}
{}


\twolineshloka
{शृणु भामिनि तत्स्रवं त्रिविधं देहकारणम्}
{कर्मवश्यं भोगवश्यं दुःखवश्यमिति प्रिये}


\threelineshloka
{मानुषं कर्मवश्यं स्याद्द्वितीयं भोगसाधनम्}
{तृतीयं यातनावश्यं शरीरं मायया कृतम्}
{यमलोके न चान्यत्र दृश्यते यातनायुतम्}


\twolineshloka
{शरीरैर्यातनावश्यैर्जीवानामुच्य भामिनि}
{नयन्ति यामिकास्तत्र प्राणिनो मायया मृतान्}


% Check verse!
मध्यमान्योधवेषेण मध्यमेन पथा तथा
\twolineshloka
{चण्डालवेषास्त्वधमान्गृहीत्वा भर्त्सतर्जनैः}
{आकर्षन्तस्तथा पाशैर्दुर्दर्शेन नयन्ति तान्}


\twolineshloka
{त्रिविधानेवमादाय नयन्ति यमसादनम्}
{धर्मासनगतं दक्षं भ्राजमानं स्वतेजसा}


\threelineshloka
{लोकपालं सहाध्यक्षं तथैव परिषद्गतम्}
{दर्शयन्ति महाभागे यामिकास्तं निवेद्य ते ॥पूजयन्दण्डयन्कांश्चित्तेषां शृण्वञ्शुभाशुभम्}
{व्याहृतो बहुसाहस्रैस्तत्रास्ते सततं यमः}


\threelineshloka
{गतानां तु यमस्तेषामुत्तमानभिपूजया}
{अभिसङ्गृह्य विधिवत्पृष्ट्वा स्वागतकौशलम्}
{प्रस्तुत्य तत्कृतं तेषां लोकं संदिशते यमः}


% Check verse!
यमेनैवमनुज्ञाता यान्ति पश्चात्त्रिविष्टपम्
\twolineshloka
{मध्यमानां यमस्तेषां श्रुत्वा कर्म यथातथम्}
{जायन्तां मानुषेष्वेव इति संदिशते च तान्}


\threelineshloka
{अधमान्पाशसंयुक्तान्यमो नावेक्षते गतान्}
{यमस्य पुरुषा घोराश्चण्डालसमदर्शनाः}
{यातनाः प्रापयन्त्येताँल्लोकपालस्य शासनात्}


\twolineshloka
{भिन्दन्तश्च तुदन्तश्च प्रकर्षन्तो यतस्ततः}
{क्रोशन्तः पातयन्त्येतान्मिथो गर्तेष्ववाङ्मुखान्}


\twolineshloka
{संयामिन्यः शिलास्तेषां पतन्ति शिरसि प्रिये}
{अयोमुखाः कङ्कवला भक्षयन्ति सुदारुणाः}


\twolineshloka
{असिपत्रवने घोरे चारयन्ति तथा परान्}
{तीक्ष्णदंष्ट्रास्तथा श्वानः कांश्चित्तत्र ह्यदन्ति वै}


\threelineshloka
{तत्र वैतरणी नाम नदी ग्राहसमाकुला}
{दुष्प्रवेशा च घोरा च मूत्रशोणितवाहिनी}
{तस्यां सम्मज्जयन्त्येते तृषितान्पाययन्ति तान्}


\twolineshloka
{आरोपयन्ति वै कांश्चित्तत्र कण्टकशल्मलीम्}
{यन्त्रचक्रेषु तिलवत्पीड्यन्ते तत्र केचन}


\twolineshloka
{अङ्गरेषु च दह्यन्ते तथा दुष्कृतकारिणः}
{कुम्भीपाकेषु पच्यन्ते पच्यन्ते सिकतासु वै}


\twolineshloka
{पाट्यन्ते तरुवच्छस्त्रैः पापिनः क्रकचादिभिः}
{भिद्यन्ते भागशः शूलैस्तुद्यन्ते सूक्ष्मसूचिभिः}


\twolineshloka
{एवं त्वया कृतं दोषं तदर्थं दण्डनं त्विति}
{वाचैव घोषयन्ति स्म दण्डमानाः समन्ततः}


\twolineshloka
{एवं ते यातनां प्राप्य शरीरैर्यातनाशयैः}
{प्रसहन्तश्च तद्दुःखं स्मरन्तः स्वापराधजम्}


\twolineshloka
{क्रोशन्तश्च रुद्रन्तश्च न मुच्यन्ते कथञ्चन}
{स्मरन्तस्तत्र तप्यन्ते पापमात्मकृतं भृशम्}


\twolineshloka
{एवं बहुविधा दण्डा भुज्यन्ते पापकारिभिः}
{यातनाभिश्च पच्यन्ते नरकेषु पुनः पुनः}


\threelineshloka
{अपरे यातनां भुक्त्वा मुच्यन्ते तत्र किल्बिषात्}
{पापदोषक्षयकरा यातनाः संस्मृता नृणाम्}
{बहुतप्तं यथा लोहममलं तत्तथा भवेत्}


\chapter{अध्यायः २३०}
\threelineshloka
{भगवंस्ते कथं तत्र दण्ड्यन्ते नरकेषु वै}
{कति ते नरका घोराः कीदृशास्ते महेश्वर ॥महेश्वर उवाच}
{}


\twolineshloka
{शृणु भामिनि तत्सर्वं पञ्चैते नरकाः स्मृताः}
{भूमेरधस्ताद्विहिता घोरा दुष्कृतकर्मणाम्}


\twolineshloka
{प्रथमं रौरवं नाम शतयोजनमायतम्}
{तावत्प्रमाणविस्तीर्णं तामसं पापपीडितम्}


\twolineshloka
{भृशं दुर्गन्धि परुषं क्रिमिभिर्दारुणैर्वृतम्}
{अतिघोरमनिर्देश्यं प्रतिकूलं ततस्ततः}


\twolineshloka
{ते चिरं तत्र तिष्ठन्ति न तत्र शयनासने}
{क्रिमिभिर्भक्ष्यमाणाश्च विष्ठागन्धसमायुताः}


\twolineshloka
{एवंप्रमाणमुद्विग्ना यावत्तिष्ठन्ति तत्र ते}
{यातनाभ्यो दशगुणं नरके दुःखमिष्यते}


% Check verse!
तत्र चात्यन्तिकं दुःखमिष्यते च शुभेक्षणेक्रोशन्तश्च रुद्रन्तश्च वेदनास्तत्रि भुञ्जते
\twolineshloka
{भ्रमन्ति दुःखमोक्षार्थं ज्ञाता कश्चिन्न विद्यते}
{दुःखस्यान्तरमात्रं तु ज्ञानं वा न च लभ्यते}


\twolineshloka
{महारौरवसंज्ञं तु द्वितीयं नरकं प्रिये}
{तस्माद्द्विगुणितं विद्धि माने दुःखे च रौरवात्}


\threelineshloka
{तृतीयं नरकं तत्र कण्टिकावनसंज्ञितम्}
{ततो द्विगुणितं तच्च पूर्वाभ्यां दुःखमानयोः}
{महापातकसंयुक्ता घोरास्तस्मिन्विशन्ति हि}


\twolineshloka
{अग्निकुण्डमिति ख्यातं चतुर्थं नरकं प्रिये}
{एतद्द्विगुणितं तस्माद्यथानिष्टसुखं तथा}


\twolineshloka
{ततो दुःखं हि सुमहदमानुषमिति स्मृतम्}
{भुञ्जते तत्रतत्रैव दुःखं दुष्कृतकारिणः}


\threelineshloka
{तत्र दुःखमनिर्देश्यं वहद्धोरं यथा तथा}
{पञ्चेन्द्रियैरसम्बाधात्पञ्चकष्टमिति स्मृतम्}
{भुञ्जते तत्रतत्रैव दुःखं दुष्कृतकारिणः}


\twolineshloka
{अमानुषार्हजं दुःखं महाभूतैश्च भुञ्जते}
{अतिघोरं चिरं कृत्वा महाभूतानि यान्ति तम्}


\twolineshloka
{पञ्च् कष्टेन हि समं नास्ति दुःखं तथा परम्}
{दुःखस्थानमिति प्राहुः पञ्चकष्टमिति प्रिये}


\twolineshloka
{एवं त्वेतेषु तिष्ठन्ति प्राणिनोः दुःखभागिनः}
{अन्ये च नरकाः सन्ति अवीचिप्रमुखाः प्रिये}


\twolineshloka
{क्रोशन्तश्च रुदन्तश्च वेदनर्ता भृशातुराः}
{केचिद्भमन्तश्चेष्टन्ते केचिद्धावन्ति चातुराः}


\twolineshloka
{आधावन्तो निवार्यन्ते शूलहस्तैर्यतस्ततः}
{रुजार्दितास्तृषायुक्ताः प्राणिनः पापकारिणः}


\twolineshloka
{यावत्पूर्वकृतं तावन्न मुच्यन्ते कथञ्चन}
{क्रिमिभिर्भक्ष्यमाणाश्च वेदनार्तास्तृषान्विताः}


\fourlineindentedshloka
{संस्मरन्तः स्वकं पापं कृतमात्मापराधजम्}
{शोचन्तस्तत्र तिष्ठन्ति यावत्पापक्षयं प्रिये}
{एवं भुक्त्वा तु नरकं मुच्यन्ते पापसंक्षयात् ॥उमोवाच}
{}


% Check verse!
भगवन्कतिकालं ते तिष्ठन्ति नरकेषु वै
\threelineshloka
{महेश्वर उवाच}
{शतवर्षसहस्राणामादिं कृत्वा हि जन्तवः}
{तिष्ठन्ति नरकावासाः प्रलयान्तमिति स्थितिः}


\threelineshloka
{उमोवाच}
{भगवंस्तेषु के तत्र तिष्ठन्तीति वद प्रभो ॥महेश्वर उवाच}
{}


\twolineshloka
{रौरवे शतसाहस्रं वर्षाणामिति संस्थितिः}
{मानुषघ्नाः कृतघ्नाश्च तथैवानृतवादिनः}


\twolineshloka
{द्वितीये द्विगुणं कालं पच्यन्ते तादृशा नराः}
{महापातकयुक्तास्तु तृतीये दुःखमाप्नुयुः}


% Check verse!
एतावन्मानुषसहं परमन्येषु लक्ष्यते
\fourlineindentedshloka
{यक्षा विद्याधराश्चैव काद्रवेयाश्च किंनराः}
{गन्धर्वभूतसङ्घाश्च तेषां पापयुता भृशम्}
{चतुर्थे परिपच्यन्ते तादृशा नरकाः स्मृताः ॥चतुर्थे परितप्यन्ते यावद्युगविपर्ययः}
{}


\twolineshloka
{सहन्तस्तादृशं घोरं पञ्चकष्टे तु यादृशम्}
{तत्रास्य चिरदुःखस्य ह्यधोन्यान्विद्धि मानुषान्}


\twolineshloka
{एवं ते नरकान्भुक्त्वा तत्र क्षपितकल्मषाः}
{नरकेभ्यो विमुक्ताश्च जायन्ते कृमिजातिषु}


\fourlineindentedshloka
{उद्भेदजेषु वा केचिदत्रापि क्षीणकल्मषाः}
{पुनरेव प्रजायन्ते मृगपक्षिषु शोभने}
{मृगपक्षिषु तद्भुक्त्वा लभन्ते मानुषं पदम् ॥उमोवाच}
{}


\twolineshloka
{नानाजातिषु केनैव जायन्ते पापकारिणः ॥महेश्वर उवाच}
{}


\twolineshloka
{तदहं ते प्रवक्ष्यामि यत्त्वमिच्छसि शोभने}
{सर्वदाऽऽत्मा कर्मवशो नानाजातिषु जायते}


\twolineshloka
{यश्च मांसप्रियो नित्यं काकगृध्नान्स संस्पृशेत्}
{सुरापः सततं मर्त्यः सूकरत्वं व्रजेद्भुवम्}


\twolineshloka
{अभक्ष्यभक्षणो मर्त्यः काकजातिषु जायते}
{आत्मघ्नो यो नरः कोपात्प्रेतजातिसु तिष्ठति}


\twolineshloka
{पैशुन्यात्परिवादाच्च कुक्कुटत्वमवाप्नुयात्}
{नास्तिकश्चैव यो मूर्खो मृगजातिं स गच्छति}


\twolineshloka
{हिंसाविहारस्तु नरः क्रिमिकीटेषु जायते}
{अतिमानयुतो नित्यं प्रेत्य गर्दभतां व्रजेत्}


\twolineshloka
{अगम्यागमनाच्चैव परदारनिषेवणात्}
{मूषिकत्वं व्रजेन्मर्त्यो नास्ति तत्र विचारणा}


\twolineshloka
{कृतघ्नो मित्रघाती च सृगालवृकजातिषु}
{कृतघ्नः पुत्रघाती च स्थावरेष्वथ तिष्ठति}


\twolineshloka
{एवमाद्यशुभं कृत्वा नरा निरयगामिनः}
{तांस्तान्भावान्प्रपद्यन्ते स्वकृतस्यैव कारणात्}


\twolineshloka
{एवंजातिषु निर्देश्याः प्राणिनः पापकारिणः}
{कथंचित्पुनरुत्पद्यि लभन्ते मानुषं पदम्}


\threelineshloka
{बहुशश्चाग्निसङ्क्रान्तं लोहं शुचिमयं तथा}
{बहुदुःखाभिसन्तप्तस्तथाऽऽत्मा शोध्यते बलात्}
{तस्मात्सुदुर्लभं चेति विद्धि जन्मसु मानुषम्}


\chapter{अध्यायः २३१}
\twolineshloka
{भगवन्देवदेवेश शूलपाणे वृषध्वज}
{श्रुतं मे परमं गुह्यं प्रसादात्ते वरप्रद}


\twolineshloka
{श्रोतुं भूयोऽहमिच्छामि प्रजानां हितकारणात्}
{शुभाशुभमिति प्रोक्तं कर्म स्वस्वं समासतः}


\twolineshloka
{तन्मे विस्तरतो ब्रूहि शुभाशुभविधइं प्रति}
{अशुभं कीदृशं कर्म प्राणिनो यन्निपातयेत्}


\threelineshloka
{शुभं वापि कथं देव प्रजानामूर्ध्वदं भवेत्}
{एतन्मे वद देवेश श्रोतुकामाऽस्मि कीर्तय ॥महेश्वर उवाच}
{}


\twolineshloka
{तदहं ते प्रवक्ष्यामि तत्सर्वं शृणु शोभने}
{सुकृतं दुष्कृतं चेति द्विविधं कर्मविस्तरम्}


\twolineshloka
{तयोर्यद्दुष्कृतं कर्म तच्च सञ्जायते त्रिधा}
{मनसा कर्मणा वाचा बुद्धिमोहसमुद्भवात्}


\twolineshloka
{मनःपूर्वं तु वा कर्म वर्तते वाङ्मयं ततः}
{जायते वै क्रियायोगमनु चेष्टाक्रमः प्रिये}


\twolineshloka
{अभिद्रोहोऽभ्यसूया च परार्थेषु च वै स्पृहा}
{शुभाशुभानां मर्त्यानां वर्तनं परिवारितम्}


\twolineshloka
{धर्मकार्ये यदाऽश्रद्धा पापकर्मणि हर्षणम्}
{एवमाद्यशुभं कर्म मनसा पापमुच्यते}


\twolineshloka
{अनृतं यच्च परुषमबद्धवचनं कटु}
{असत्यं परिवादश्च पापमेतत्तु वाङ्मयम्}


\twolineshloka
{अगम्यागमनं चैव परदारनिषेवणम्}
{वधबन्धपरिक्लेशैः परप्राणोपतापनम्}


\twolineshloka
{चौर्यं परेषां द्रव्याणां हरणं नाशनं तथा}
{अभक्ष्यभक्षणं चैव व्यसनेष्वविषङ्गता}


\twolineshloka
{दर्पात्स्तम्भाभिमानाच्च परेषामुपतापनम्}
{अकार्याणां च करणमशौचं पानसेवनम्}


\threelineshloka
{दौःशील्यं पापसम्पर्के साहाय्यं पापकर्मणि}
{अधर्म्यमयशस्यं च कार्यं तस्य निषेवणम्}
{एवमाद्यशुभं चान्यच्छारीरं पापमुच्यते}


\twolineshloka
{मानसाद्वाङ्मयं पापं विशिष्टमिति लक्ष्यते}
{वाङ्मयादपि वै पापाच्छारीरं गण्यते बहु}


\twolineshloka
{एवं पापयुतं कर्म त्रिविधं पातयेन्नरम्}
{परापकारजननमत्यन्तं पातकं स्मृतम्}


\twolineshloka
{त्रिविधं तत्कृतं पापं कर्तारं पापकं नयेत्}
{पातकं चापि यत्कर्म कर्मणा बुद्धिपूर्वकम्}


\twolineshloka
{सापदेशमवश्यं तत्कर्तव्यमिति तत्कृतम्}
{कथञ्चित्तत्कृतमपि कर्ता तेन स लिप्यते}


\twolineshloka
{अवश्यं पापदेशेन प्रतिहृन्येत कारणम् ॥उमोवाच}
{}


\twolineshloka
{भगवन्पापकं कर्म यथा कृत्वा न लिप्यते ॥महेश्वर उवाच}
{}


\threelineshloka
{यो नरोऽनपराधी च स्वात्मप्राणस्य रक्षणात्}
{शत्रुमुद्यतशस्त्रं वा पूर्वं तेन हतोपि वा}
{प्रतिहन्यान्नरो हिंस्यान्न स पापेन लिप्यते}


\twolineshloka
{चोरादधिकसंत्रस्तस्तत्प्रतीकारचेष्टया}
{यः प्रजघ्नन्नरो हन्यान्न स पापेन लिप्यते}


\twolineshloka
{ग्रामार्थं भर्तृपिण्डार्थं दीनानुग्रहकारणात्}
{वधबन्धपरिक्लेशान्कुर्वन्पापात्प्रमुच्यते}


\twolineshloka
{दुर्भिक्षे चात्मवृत्त्यर्थमेकायतनगस्तथा}
{अकार्यं वाऽप्यभक्ष्यं वा कृत्वा पपान्न लिप्यते}


\twolineshloka
{विधिरेष गृहस्थानां प्रायेणैवोपदिश्यते}
{अवाच्यं वाऽप्यकार्यं वा देशकालवशेन तु}


\twolineshloka
{बुद्धिपूर्व नरः कुर्वस्तत्प्रयोजनमात्रया}
{किञ्चिद्वा लिप्यते पापैरथवा न च लिप्यते}


\twolineshloka
{एवं देवि विजानीहि नास्ति तत्र विचारणा ॥उमोवाच}
{}


\threelineshloka
{भगवन्पानदोषांश्च पेयापेयत्वकारणम्}
{एतदिच्छाम्यहं श्रोतुं तन्मे वद महेश्वर ॥महेश्वर उवाच}
{}


% Check verse!
हन्त ते कथयिष्यामि पानोत्पत्तिं शुचिस्मिते
\twolineshloka
{पुरा सर्वेऽभवन्मर्त्या बुद्धिमन्तो नयानुगाः}
{शुचयश्च शुभाचाराः सर्वे चोन्मनसः प्रिये}


\twolineshloka
{एवंभूते तदा लोके प्रेष्यकृन्न परस्परम्}
{प्रेष्याबावान्मनुष्याणां कर्मारम्भो ननाश ह}


\twolineshloka
{उभयोर्लोकयोर्नाशं दृष्ट्वा कर्मक्षयात्प्रभुः}
{यज्ञकर्म कथं लोके वर्तेतेति पितामहः}


\threelineshloka
{आज्ञापयत्सुरान्देवि मोहयस्वेति मानुषान्}
{तमसः सारमुद्धृत्य पानं बुद्धिप्रणाशनम्इ}
{न्यपातयन्मनुष्येषु पापदोषावहं प्रिये}


\twolineshloka
{तदाप्रभृति तत्पानान्मुमुहुर्मानवा भुवि}
{कार्याकार्यमजानन्तो वाच्यावाच्यं गुणागुणम्}


\threelineshloka
{केचिद्धसन्ति तत्पीत्वा प्रवदन्ति तथा परे}
{नृत्यन्ति मुदिताः केचिद्गायन्ति शुभाशुभान्}
{}


\twolineshloka
{कलिं ते कुर्वतेऽभीष्टं प्रहरन्ति परस्परम्}
{क्वचिद्धावन्ति सहसा प्रस्खलन्ति पतन्ति च}


\twolineshloka
{अयुक्तं बहु भाषन्ते यत्र क्वचन शोभने}
{नग्ना विक्षिप्य गात्राणि नष्टज्ञाना इवासते}


\twolineshloka
{एवं बहुविधान्भावान्कुर्वन्ति भ्रान्तचेतनाः}
{ये पिबन्ति महामोहं पानं पापयुता नराः}


\twolineshloka
{धृतिं लज्जां च बुद्धिं च पानं पीतं प्रणाशयेत्}
{तस्मान्नराः सम्भवन्ति निर्लज्जा निरपत्रपाः}


\twolineshloka
{बुद्धिसत्वैः परिक्षीणास्तेजोहीना मलान्विताः}
{पीत्वापीत्वा तृषायुक्ताः पानपाः सम्भवन्ति च}


\twolineshloka
{पानकामाः पानकथाः पानकालाभिकाङ्क्षिणः}
{पानार्थं कर्मवश्यास्ते सम्भवन्ति नराधमाः}


\twolineshloka
{पानकामास्तृषायोगाद्बुद्धिसत्वपरिक्षयात्}
{पानदानां प्रेष्यकाराः पानपाः सहसाऽभवन्}


\twolineshloka
{तदाप्रभृति वै लोके दीनैः पानवशैर्नरैः}
{कारयन्ति च कर्माणि बुद्धिमन्तस्तु पानपाः}


\twolineshloka
{कारुत्वमथ दासत्वं प्रेष्यतामेत्य पानपाः}
{सर्वकर्मकराश्चासन्पशुवद्रज्जुबन्धिताः}


\threelineshloka
{पानपस्तु सुरां पीत्वा तदा बुद्धिप्रणाशनात्}
{कार्याकार्यस्य चाज्ञानाद्यथेष्टकरणात्स्वयम्}
{विदुषामविधेयत्वात्पापमेवाभिपद्यते}


\twolineshloka
{परिभूतो भवेल्लोके मद्यपो मित्रभेदकः}
{सर्वकालमशुद्धिं च सर्वभक्षस्तथा भवेत्}


\twolineshloka
{विनष्टो ज्ञातिविद्वद्भ्यः सततं कलिभावगः}
{परुषं कटुकं घोरं वाक्यं वदति सर्वशः}


\twolineshloka
{गुरूनतिवदेन्मत्तः परदारान्प्रधर्षयेत्}
{संविदं कुरुते शौण्डेर्न शृणोति हितं क्वचित्}


\threelineshloka
{एवं बहुविधा दोषाः पानपे सन्ति शोभने}
{केवलं नरकं यान्ति नास्ति तत्र विचारणा}
{तस्मात्तद्वर्जितं सद्भिः पानमात्महितैषिभिः}


\twolineshloka
{यदि पानं न वर्जेरन्सन्तश्चारित्रकारणात्}
{भवेदेतज्जगत्सर्वममर्यादं च निष्क्रियम्}


\twolineshloka
{तस्माद्बुद्धेर्हि रक्षार्थं सद्भिः पानं विवर्जितम्}
{इति ते दुष्कृतं सर्वं कथितं त्रिविधं प्रिये}


\chapter{अध्यायः २३२}
\twolineshloka
{विधानं सुकृतस्यापि भूयः शृणु शुचिस्मिते}
{प्रोच्यते तत्त्रिधा देवि सुकृतं च समासतः}


\twolineshloka
{यदौपरमिकं चैव सुकृतं निरुपद्रवम्}
{तथैव सोपकरणं तावता सुकृतं विदुः}


\twolineshloka
{निवृत्तिः पापकर्मभ्यस्तदौपरमिकं प्रिये}
{मनोवाक्कायजा दोषाः शृणु मे वर्जनाच्छुभम्}


\twolineshloka
{त्रैविध्यदोषोपरमे यस्तु दोषव्यपेक्षया}
{स हि प्राप्नोति सकलं सर्वदुष्कृतवर्जनात्}


\threelineshloka
{प्रथमं वर्जयेद्दोषान्युगपत्पृथगेव वा}
{तथा धर्ममवाप्नोति दोषत्यागो हि दुष्करः}
{दोषसाकल्यसंत्यागान्मुनिर्भवति मानवः}


\twolineshloka
{सौकर्यं पस्य धर्मस्य कार्यारम्भादृतेऽपि च}
{आत्मा च लब्धोपरमो लभन्ते सुकृतं परम्}


\twolineshloka
{अहो नृशंसाः पच्यन्ते मानुषाः स्वल्पबुद्धयः}
{एतादृशं न बुध्यन्ते आत्माधीनं न निर्व्यथाः}


% Check verse!
दुष्कृतत्यागमात्रेण पदमूर्ध्वं हि लभ्यते
\threelineshloka
{पापभीरुत्वमात्रेण दोषाणां परिवर्जनात्}
{सुशोभनो भवेद्देवि क्रजुर्धर्मव्यपेक्षया}
{इत्यौपरमिकं देवि कथितं सुकृतं तव}


\twolineshloka
{श्रुत्वा च वृद्धसंयोगादिन्द्रियाणां च निग्रहात्}
{सन्तोषाच्च धृतेश्चैव शक्यते दोषवर्जनम्}


\threelineshloka
{तदेव धर्ममित्याहुर्दोषसंयमनं प्रिये}
{यमधर्मेण धर्मोस्ति नान्यः शुभतरः प्रिये}
{यमधर्मेण यतयः प्राप्नुवन्त्युत्तमां गतिम्}


\twolineshloka
{ईश्वराणां प्रभवतां दरिद्राणां च वै नृणाम्}
{सफलो दोषसंत्यागो दानादपि शुभादपि}


\twolineshloka
{तपो दानं महादेवि दोषमल्पं हि निर्भरेत्}
{सुकृतं यामिकं चोक्तं वक्ष्ये निरुपसाधनम्}


\twolineshloka
{सुखाभिसन्धिर्लोकानां सत्यं शौचमथार्जवम्}
{व्रतोपवासः प्रीतिश्च ब्रह्मचर्यं दमः शमः}


\twolineshloka
{एवमादि शुभं कर्म सुकृतं नियमाश्रितम्}
{शृणु तेषां विशेषांश्च कीर्तयिष्यामि भामिनि}


\twolineshloka
{सत्यं स्वर्गस्य सोपानं पारावारस्य नौरिव}
{नास्ति सत्यात्परं दानं नास्ति सत्यात्परं तपः}


\twolineshloka
{यथा श्रुतं यथा दृष्टमात्मना यद्यथा कृतम्}
{तथा तस्याविकारेण वचनं सत्यलक्षणम्}


\twolineshloka
{यच्छलेनाभिसंयुक्तं सत्यरूपं मृषैव तत्}
{सत्यमेव प्रवक्तव्यं पारावर्यं विजानता}


\threelineshloka
{दीर्घायुश्च भवेत्सत्यात्कुलसन्तानपालकः}
{लोकसंस्थितिपालश्च भवेत्सत्येन मानवः ॥उमोवाच}
{}


\twolineshloka
{कथं सन्धारयन्मर्त्यो व्रतं शुभमवाप्नुयात् ॥महेश्वर उवाच}
{}


\twolineshloka
{पूर्वमुक्तं तु यत्पापं मनोवाक्कायकर्मभिः}
{व्रतवत्तस्य संत्यागस्तपोव्रतमिति स्मृतम्}


\twolineshloka
{त्याज्यं वा यदि वा जोष्यमव्रतेनि वृथा चरन्}
{तथा फलं न लभते तस्माद्धर्मं वृथा चरेत्}


\twolineshloka
{शुद्धकायो नरो भूत्वा स्नात्वा तीर्थ यथाविधि}
{पञ्चभूतानि चन्द्रार्कौ संध्ये धर्मयमौ पितॄन्}


\twolineshloka
{आत्मनैव तथाऽऽत्मानं निवेद्य व्रतवच्चरेत्}
{व्रतमामरणाद्वाऽपि कालच्छेदेन वा हरेत्}


\twolineshloka
{शाकादिषु व्रतं कुर्यात्तथा पुष्पफलादिषु}
{ब्रह्मचर्यव्रतं कुर्यादुपवासव्रतं तथा}


\threelineshloka
{एवमन्येषु बहुषु व्रतं कार्यं हितैषिणा}
{व्रतभङ्गो यथा न स्याद्रक्षितव्यं तथा बुधैः}
{व्रतभङ्गे महत्पापमिति विद्धि शुभेक्षणे}


\twolineshloka
{औषधार्थं यदज्ञानाद्गुरूणां वचनादपि}
{अनुग्रहार्थं बन्धूनां व्रतभङ्गो न दुष्यते}


\threelineshloka
{व्रतापवर्गकाले तु दैवब्राह्मणपूजनम्}
{नरेण तु यथा विद्धि कार्यसिद्धिं यथाऽऽप्तुयात् ॥उमोवाच}
{}


\twolineshloka
{कथं शौचविधिस्तत्र तन्मे शंसितुमर्हसि ॥महेश्वर उवाच}
{}


\twolineshloka
{बाह्ममाभ्यन्तरं चेति द्विविधं शौचमिष्यते}
{मानसं सुकृतं यत्तच्छौचमाभ्यन्तरं स्मृतम्}


\twolineshloka
{सदाऽऽहारविशुद्धिश्च कायप्रक्षालनं च यत्}
{बाह्यशौचं भवेदेतत्तथैवाचमनादिना}


\threelineshloka
{मृच्चैव शुद्धदेशस्था गोशकृन्मूत्रमेव च}
{द्रव्याणि गन्धयुक्तानि यानि पुष्टिकराणि च}
{एतैः सम्मार्जयेत्कायमम्भसा च पुनः पुनः}


\twolineshloka
{अक्षोभ्यं यत्प्रकीर्णं च नित्यस्रोतं च यज्जलम्}
{प्रायशस्तादृशे मज्जेदन्यथा च विवर्जयेत्}


\threelineshloka
{त्रिस्त्रिराचमनं श्रेष्ठं निष्फेनैर्निर्मलैर्जलैः}
{तथा विण्मूत्रयोः शुद्धिरद्भिर्बहुमृदा भवेत्}
{तथैव जलसंशुद्धिर्यत्संशुद्धं तु संस्पृशेत्}


\twolineshloka
{शकृता भूमिशुद्धिः स्याल्लौहानां भस्मना स्मृतम्}
{तक्षणं घर्षणं चैव दारवाणां विशोधनम्}


\threelineshloka
{दहनं मृण्मयानां च मर्त्यानां कृच्छ्रधारणम्}
{शेषाणां देवि सर्वेषामातपेन जलेन च}
{ब्राह्मणानां च वाक्येन सदा संशोधनं भवेत्}


\fourlineindentedshloka
{अदृष्टमद्भिर्निर्णिक्तं यच्च वाचा प्रशस्यते}
{एवमापदि संशुद्धिरेवं शौचं विधीयते}
{उमोवाच}
{}


\twolineshloka
{आहारशुद्धिस्तु कथं देवदेव महेश्वर ॥महेश्वर उवाच}
{}


\twolineshloka
{अमांसमद्यमक्लेद्यमपर्युषितमेव च}
{अतिकट्वम्ललवणहनं च शुभगन्धि च}


\twolineshloka
{क्रिमिकेशमलैर्हीनं संवृतं शुद्धदर्शम्}
{एवंविधं सदाहार्यं देवब्राह्मणसात्कृतम्}


\twolineshloka
{श्रुतमित्येव तज्ज्ञेयमन्यथा मन्यसेऽशुभम्}
{ग्राम्यादारण्यकैः सिद्धं श्रेष्ठमित्यवधारय}


\threelineshloka
{अतिमात्रगृहीतात्तु अल्पदत्तं भवेच्छुचि}
{यज्ञशेषं हविःशेषं पितृशेषं च निर्मलम्}
{इति ते कथितं देवि भूयः किं श्रोतुमिच्छसि}


\chapter{अध्यायः २३३}
\threelineshloka
{भक्षयन्त्यपरे मांसं वर्जयन्त्यपरे विभो}
{तन्मे वद महादेव भक्ष्याभक्ष्यविनिर्णयम् ॥महेश्वर उवाच}
{}


\twolineshloka
{मांसस्य भक्षणे दोषो यश्चास्याभक्षणे गुणः}
{तदहं कीर्तयिष्यामि तन्निबोध यथातथम्}


\twolineshloka
{इष्टं दत्तमधीतं च क्रतवश्च सदक्षिणाः}
{अमांसभक्षणस्यैव कलां नार्हन्ति षोडशीम्}


\twolineshloka
{आत्मार्थं यः परप्राणान्हिस्यात्स्वादुफलेप्सया}
{व्यालगृध्रसृगालैश्च राक्षसैश्च समस्तु सः}


\twolineshloka
{यो वृथा नित्यमांसाशी स पुमानधमो भवेत्}
{ततः कष्टतरं नास्ति स्वयमाहृत्य भक्षणात्}


\twolineshloka
{स्वमांसं परमांसेन यो वर्धयितुमिच्छति}
{उद्विग्नवासं लभते यत्रयत्रत्रोपजायते}


\twolineshloka
{संछेदनं स्वमांसस्य यथा सञ्जनयेद्रुजम्}
{तथैव परमांसेऽपि वेदितव्यं विजानता}


\twolineshloka
{यस्तु सर्वाणि मांसानि यावज्जीवं न भक्षयेत्}
{स स्वर्गे विपुलं स्थानं लभते नात्र संशयः}


\twolineshloka
{यत्तु वर्षशतं पूर्णं तप्यते परमं तपः}
{यच्चापि वर्जयेन्मांसं सममेतन्न वा समम्}


\twolineshloka
{न हि प्राणैः प्रियतमं लोके किञ्चन विद्यते}
{तस्मात्प्राणिदया कार्या यथात्मनि तथा परे}


\twolineshloka
{सर्वे यज्ञा न तत्कुर्युः सर्वे देवाश्च भामिनि}
{यो मांसरसमास्वाद्य पुनर्मासं विवर्जयेत्}


\twolineshloka
{इत्येवं मुनयः प्राहुर्मांसस्याभक्षणे गुणाः}
{एवं बहुगुणं देवि नृणां मांसविवर्जनम्}


\twolineshloka
{न शक्नुयाद्यदाऽतीय त्यक्तं मांसं कथञ्चन}
{पुण्यं तन्मासमात्रं वा वर्जनीयं विशेषतः}


\fourlineindentedshloka
{न शक्नुयादपि तथा कौमुदीमासमेव च}
{जन्मनक्षत्रतिथिषु सदा पर्वसु रात्रिषु}
{वर्जनीयं तथा मांसं परत्रि हितमिच्छता}
{}


\twolineshloka
{अशक्तः कारणान्मर्त्यो भोक्तुमिच्छेद्विधिं शृणु}
{अनेनि खादन्विधिना कलुषेण न लिप्यते}


\twolineshloka
{सूनायां च गतप्राणान्क्रीत्वा न्यायेन भामिनि}
{ब्राह्मणातिथिपूजार्थं भोक्तव्यं हितमिच्छता}


\threelineshloka
{भैषज्यकारणाद्व्याधौ खादन्पापैर्न लिप्यते}
{पितृशेषं तथैवाश्नन्मांसं नाशुभमृच्छति ॥उमोवाच}
{}


\twolineshloka
{गुरुपूजा कथं देव क्रियते धर्मचारिभिः ॥महेश्वर उवाच}
{}


\twolineshloka
{गुरुपूजां प्रवक्ष्यामि यथावत्तव शोभने}
{कृतज्ञानां परो धर्म इति वेदानुशासनम्}


\twolineshloka
{तस्मात्स्वगुरवः पूज्यास्ते हि पूर्वोपकारिणः}
{गुरूणां च गरीयांसस्त्रयो लोकेषु पूजिताः}


\threelineshloka
{उपाध्यायः पिता माता संपूज्यास्ते विशेषतः}
{ये पितुर्भ्रातरो ज्येष्ठा ये च तस्यानुजास्तथा}
{पितुः पिता च सर्वे ते पूजनीयाः पिता तथा}


\twolineshloka
{मातुर्या भगिनी ज्येष्ठा मातुर्या च यवीयसी}
{मातामही च धात्री च सर्वास्ता मातरः स्मृताः}


\twolineshloka
{उपाध्यायस्य यः पुत्रो यश्च तस्य भवेद्गुरुः}
{ऋत्विग्गुरुः पिता चेति गुरवः सम्प्रकीर्तिताः}


\twolineshloka
{ज्येष्ठो भ्राता नरेन्द्रश्च मातुलः श्वळशुरस्तथा}
{भयत्राता च भर्ता च गुरवस्ते प्रकीर्तिताः}


\twolineshloka
{इत्येष कथितः साध्वि गुरूणां सर्वसङ्ग्रहः}
{अनुवृत्तिं च पूजां च तेषामपि निबोध मे}


\twolineshloka
{आराध्या मातापितरावुपाध्यायस्तथैव च}
{कथंचिन्नावमन्तव्या नरेण हितमिच्छता}


\twolineshloka
{येन प्रीणन्ति पितरस्तेन प्रीतः प्रजापतिः}
{येन प्रीणाति चेन्माता प्रीताः स्युर्देवमातरः}


\twolineshloka
{ये प्रीणात्युपाध्यायो ब्रह्मा तेनाभिपूजितः}
{अप्रीतेषु पुनस्तेषु नरो नरकमेति हि}


\twolineshloka
{गुरूणां वैरनिर्बन्धो न कर्तव्यः कथञ्चन}
{नरः स्वगुरुमप्रीत्या मनसाऽपि न गच्छति}


\twolineshloka
{न ब्रूयाद्विप्रियं तेषामनिष्टं न प्रवर्तयेत्}
{विगृह्य न वदेत्तेषां समीपे स्पर्धया क्वचित्}


\twolineshloka
{यद्यदिच्छन्ति ते कर्तुमस्वतन्त्रस्तदाचरेत्}
{वेदानुशासनसमं गुरुशासनमिष्यते}


\twolineshloka
{कलहांश्च विवादांश्च गुरुभिः सह वर्जयेत्}
{कैतवं परिहासांश्च मन्युकामाश्रयाः कथाः}


\twolineshloka
{गुरूणां योऽनहंवादी करोत्याज्ञामतन्द्रितः}
{न तस्मात्सर्वमर्त्येषु विद्यते पुण्यकृत्तमः}


\twolineshloka
{असूयामपवादं च गुरूणां परिवर्जयेत्}
{तेषां प्रियहितान्वेषी भूत्वा परिचरेत्सदा}


\twolineshloka
{न तद्यज्ञफलं कुर्यात्तपो वाऽऽचरितं महत्}
{यत्कुर्यात्पुरुषस्येह गुरुपूजा सदा कृता}


\twolineshloka
{अनुवृत्तेर्विना धर्मो नास्ति सर्वाश्रमेष्वपि}
{तस्मात्क्षमावृतः क्षान्तो गुरुवृत्तिं समाचरेत्}


\twolineshloka
{स्वमर्थं स्वशरीरं च गुर्वर्थे संत्यजेद्बुधः}
{विवादं धनहेतोर्वा मोहाद्वा तैर्न रोचयेत्}


\twolineshloka
{ब्रह्मचर्यमहिंसा च दानानि विविधानि च}
{गुरुभिः प्रतिषिद्धस्य सर्वमेतपार्थकम्}


\fourlineindentedshloka
{उपाध्यायं पितरं मातरं च}
{येऽभिद्रुह्युर्मनसा कर्मणा वा}
{तेषां पापं भ्रूणहत्याविशिष्टंतेभ्यो नान्यः पापकृदस्ति लोके ॥उमोवाच}
{}


\twolineshloka
{उपवासविधिं तत्र तन्मे शंसितुमर्हसि ॥महेश्वर उवाच}
{}


\twolineshloka
{शरीरमलशान्त्यर्थमिन्द्रियोच्छोषणाय च}
{एकभुक्तोपवासैस्तु धारयन्ते व्रतं नराः}


\twolineshloka
{लभन्ते विपुलं धर्मं तथाऽऽहारपरिक्षयात्}
{बहूनामुपरोधं तु न कुर्यादात्मकारणात्}


\threelineshloka
{जीवोपघातं च तथा स जीवन्धन्य इष्यते}
{तस्मात्पुण्यं लभेन्मर्त्यः स्वयमाहारकर्शनात्}
{तद्गृहस्थैर्यथाशक्ति कर्तव्यमिति निश्चयः}


\threelineshloka
{उपवासार्दिते काये आपदर्शं पयो जलम्}
{भुञ्जन्न प्रतिघाती स्याद्ब्राह्म्णाननुमान्य च ॥उमोवाच}
{}


\twolineshloka
{ब्रह्मचर्यं कथं देव रक्षितव्यं विजानता ॥महेश्वर उवाच}
{}


% Check verse!
तदहं ते प्रवक्ष्यामि शृणु देवि समाहिता
\twolineshloka
{ब्रह्मचर्यं परं शौचं ब्रह्मचर्यं परं तपः}
{केवलं ब्रह्मचर्येणि प्राप्यते परमं पदम्}


\twolineshloka
{सङ्कल्पाद्दर्शनाच्चैव तद्युक्तवचनादपि}
{संस्पर्शादथ संयोगात्पञ्चधा रक्षितं व्रतम्}


\threelineshloka
{व्रतवद्धारितं चैव ब्रह्मचर्यमकल्मषम्}
{नित्यं संरक्षितं तस्य नैष्ठिकानां विधियते}
{तदिष्यते गृहस्थानां कालमुद्दिश्य कारणम्}


\twolineshloka
{जन्मनक्षत्रयोगेषु पुण्यवासेषु पर्वसु}
{देवताधर्मकार्येषु ब्रह्मचर्यव्रतं चरेत्}


\threelineshloka
{ब्रह्मचर्यव्रतफलं लभेद्दारव्रती सदा}
{शौचमायुस्तथाऽऽरोग्यं लभ्यते ब्रह्मचारिभिः ॥उमोवाच}
{}


\threelineshloka
{तीर्थचर्याव्रतं देव क्रियते धर्मकाङ्क्षिभि}
{कानि तीर्थानि लोकेषु तन्मे शंसितुमर्हसि ॥महेश्वर उवाच}
{}


\twolineshloka
{हन्त ते कथयिष्यामि तीर्थस्नानविधिं प्रिये}
{पावनार्थं च शौचार्थं ब्रह्मणा निर्मितं पुरा}


\threelineshloka
{यास्तु लोके महानद्यस्ताः सर्वास्तीर्थसंज्ञिताः}
{तासां प्राक्स्रोतसः श्रेष्ठाः सङ्गमश्च परस्परम्}
{तासां सागरसंयोगो वरिष्ठश्चेति विद्यते}


\twolineshloka
{तासामुभयतः कूलं तत्रतत्र मनीषिभिः}
{देवैर्वा सेवितं देवि तत्तीर्थं परमं स्मृतम्}


\twolineshloka
{समुद्रश्च महातीर्थं पावनं परमं शुभम्}
{तस्य कूलगतास्तीर्था महद्भिश्च समाप्लुताः}


\twolineshloka
{स्रोतसकां पर्वताना च जोषितानां महर्षिभिः}
{अपि कूपं नटाकं वा सेवितुं मुनिभिः प्रिये}


\twolineshloka
{तत्तु तीर्थमिति ज्ञेयं प्रभावात्तु तपस्विनाम्}
{तदा प्रभृति तीर्थत्वं लेभे लोकहिताय वै}


% Check verse!
एवं तीर्थं भवेद्देवि तस्य स्नानविधिं शृणु
\twolineshloka
{जन्मना व्रतभूयिष्ठो गत्वा तीर्थानि काङ्क्षया}
{उपवासत्रयं कुर्यादेकं वा नियमान्वितः}


\twolineshloka
{पुण्यमासेवते काले पौर्णमास्यां यथाविधि}
{बहिरेव शुचिर्भूत्वा तत्तीर्थं मन्मना विशेत्}


\twolineshloka
{त्रिराप्लुत्य जलाभ्यासे दत्त्वा ब्राह्मणदक्षिणाम्}
{अभ्यर्च्य देवायतनं ततः प्रायाद्यथागतम्}


\twolineshloka
{एतद्विधानं सर्वेषां तीर्थंतीर्थमिति प्रिये}
{समीपतीर्थस्नानात्तु दूरतीर्थं सुपूजितम्}


\twolineshloka
{आदिप्रभृतिशुद्धस्य तीर्थस्नानं शुभं भवेत्}
{तपोर्थं पापनाशार्थं शौचार्थं तीर्थगाहनम्}


\twolineshloka
{एवं पुण्येषु तीर्थेषु तीर्थस्नानं शुभं भवेत्}
{एतन्नैयमिकं सर्वं सुकृतं कथितं तव}


\chapter{अध्यायः २३४}
\twolineshloka
{एतदर्थमवाप्नोति नरः प्रेत्य शुभेक्षणे ॥ उमोवाच}
{}


\fourlineindentedshloka
{लोकसिद्धं तु यद्द्रव्यं सर्वसाधारणं भवेत्}
{तददत्सर्विसामान्यं कथं धर्मं लभेन्नरः}
{एवं साधारणे द्रव्ये तस्य स्वत्वं कथं भवेत् ॥ महेश्वर उवाच}
{}


\twolineshloka
{लोके भूतमयं द्रव्यं सर्वसाधारणं तथा}
{तथैव तद्ददन्मर्त्यो लभेत्पुण्यं स तच्छृणु}


\threelineshloka
{दाता प्रतिग्रहीता च देयं सोपक्रमं तथा}
{देशकालौ च यत्त्वेतद्दानं षङ्गुणमुच्यते}
{तेषां सम्पद्विशेषांश्च कीर्त्यमानान्निबोध मे}


\threelineshloka
{आदिप्रभृति यः शुद्धो मनोवाक्कायकर्मभिः}
{सत्यवादीजितक्रोधस्त्वलुब्धो नाभ्यसूयकः}
{श्रुद्धावानास्तिकश्चैवक एवं दाताप्रशस्यते}


\twolineshloka
{शुद्रो दान्तो जितक्रोधस्तथा दीनकुलोद्भवः}
{श्रुतचारित्रसम्पन्नस्तथा बहुकलत्रवान्}


\twolineshloka
{पञ्चयज्ञपरो नित्यं निर्विकारशरीरवान्}
{एतान्पात्रगुणान्विद्धि तादृक्पात्रं प्रशस्यते}


\threelineshloka
{पितृदेवाग्निकार्येषु तस्य दत्तं महाफलम्}
{यद्यदर्हति यो लोके पात्रं तस्य भवेच्च सः}
{मुच्येदापद आपन्नोयेन पात्रं तदस्य तु}


\twolineshloka
{अन्नस्य क्षुधितं पात्रं तृषितस्तु जलस्य वै}
{एवं पात्रेषु नानात्वमिष्यते पुरुषं प्रति}


\twolineshloka
{जारश्चोरश्च षण्डश्च हिंस्रः समयभेदकः}
{लोकविघ्नकराश्चान्ये वर्जितव्याः सर्वशः प्रिये}


\twolineshloka
{परोपघाताद्यद्द्रव्यं चौर्याद्वा लभ्यते नृभिः}
{निर्दयाल्लभ्यते यच्च धूर्तभावेन वै तथा}


\twolineshloka
{अधर्मादर्थमोहाद्वा बहूनामुपरोधनात्}
{यल्लभ्यते धनं देवि तदत्यन्तविगर्हितम्}


\twolineshloka
{तादृशेन कृतं धर्मं निष्फलं विद्धि भामिनि}
{तस्मान्न्यायागतेनैव दातव्यं शुभमिच्छता}


\twolineshloka
{यद्यदात्मप्रियं नित्यं तत्तद्देयमिति स्थितिः}
{उपक्रममिमं विद्धि दातॄणां परमं हितम्}


\threelineshloka
{पात्रभूतं तु दूरस्थमभिगम्य प्रसाद्य च}
{दाता दानं तथा दद्याद्यथा तुष्येत तेन सः}
{एष दानविधइः श्रेष्ठःसमाहूय तु मध्यमः}


\twolineshloka
{पूर्वं च पात्रतां ज्ञात्वा समाहूय निवेद्य च}
{शौचाचमनसंयुक्तं दातव्यं श्रद्धया प्रिये}


\twolineshloka
{याचितॄणां तु परममाभिमुख्यं पुरस्कृतम्}
{संमानपूर्वं सङ्ग्राह्यं दातव्यं देशकालयो}


\twolineshloka
{अपात्रेभ्योपि चान्येभ्यो दातव्यं भूतिमिच्छता ॥पात्राणि सम्परीक्ष्यैव दात्रा वै नाममात्रया}
{}


\twolineshloka
{अतिशक्तया परं दानं यथाशक्ति तु मध्यमम्}
{तृतीयं चापरं दानं नानुरूपमिवात्मनः}


\fourlineindentedshloka
{यथा सम्भाषितं पूर्वं दातव्यं तत्तथैव च}
{पुण्यिक्षेत्रेषु यद्दत्तं पुण्यकालेषु वा यथा}
{तच्छोभनतरंविद्धि गौरवाद्देशकालयोः ॥ उमोवाच}
{}


\twolineshloka
{यश्च पुण्यतमो देशस्तथा कालश्च शंस मे ॥ महेश्वर उवाच}
{}


\threelineshloka
{कुरुक्षेत्रं महानन्यो यश्च देवर्षिसेवितः}
{गिरिर्वरश्च तीर्थानि देशभागेषु पूजितः}
{ग्रहीतुमीप्सितो यत्र तत्र दत्तं महाफमल्}


\twolineshloka
{शरद्वसन्तकालश्च पुण्यमासस्तथैव च}
{शुक्लपक्षश्च पक्षाणां पौर्णमासी च पर्वसु}


\twolineshloka
{पितृदैवतनक्षत्रनिर्मलो दिवसस्तथा}
{तच्छोभनतरं विद्धि चन्द्रसूर्यग्रहे तथा}


\twolineshloka
{प्रतिग्रहीतुर्यः कालो मनसा कीर्तितः शुभे}
{एवमादिष्टकालेषु दत्तं दानं महद्भवेत्}


\twolineshloka
{दाता देयं च पात्रं च उपक्रमयुता क्रिया}
{देशकालं तथा तेषां सम्पच्छुद्धिः प्रकीर्तिता}


% Check verse!
यथैव युगपत्सम्पत्तत्र दानं महद्भवेत्
\twolineshloka
{अत्यल्पमपि यद्दानमेभिः षड्भिर्गुणैर्युतम्}
{भूत्वाऽनन्तं नयेत्स्वर्गं दातारं दोषवर्जितम्}


\threelineshloka
{सुमहद्वाऽपि यद्दानं गुणैरेभिर्विनाकृतम्}
{अत्यल्पफलनिर्योगमफलं वा फलोद्धतम् ॥ उमोवाच}
{}


\threelineshloka
{एवंगुणयुतं दानं दत्तं च फलतां व्रजेत्}
{तदस्ति चेन्महद्देयं तन्मे शंसितुमर्हसि ॥ महेश्वर उवाच}
{}


% Check verse!
तदप्यस्ति महाभागे नराणां भावदोषतः
\twolineshloka
{कृत्वा धर्मं तु विधिवत्पश्चात्तापं करोति चेत्}
{श्लाघया वा यदि ब्रूयाद्वृथा संसदिं यत्कृतम्}


\threelineshloka
{प्रकल्पयेच्च मनसा तत्फलं प्रेत्यभावतः}
{कर्म धर्मकृतं यच्च सततं फलकाङ्क्षया}
{एतत्कृतं वा दत्तं वापरत्र विफलं भवेत्}


\twolineshloka
{एते दोषा विवर्ज्याश्च दातृभिः पुण्यकाङ्क्षिभिः}
{सनातनमिदं वृत्तं सद्भिराचरितं तथा}


\twolineshloka
{अनुग्रहात्परेषां तु गृहस्थानामृणं हि तत्}
{इत्येवं मन आविश्य दातव्यं सततं बुधैः}


\twolineshloka
{एवमेव कृतं नित्यं सुकृतं तद्भवेन्महत्}
{सर्वसाधारणं द्रव्यमेवं दत्त्वा महत्फलम्}


\chapter{अध्यायः २३५}
\threelineshloka
{भगवन्कानि देयानि धर्ममुद्दिश्य मानवैः}
{तान्यहं श्रोतुमिच्छामि तन्मे शंसितुमर्हसि ॥महेश्वर उवाच}
{}


\twolineshloka
{अजस्रं धर्मकार्यं च तथा नैमित्तिकं प्रिये}
{अन्नं प्रतिश्रयो दीपः पानीयं तृणमिन्धनम्}


\twolineshloka
{स्नेहो गन्धश्च भैषज्यं तिलाश्च लवणं तथा}
{एवमादि तथाऽन्यश्च दानमाजस्रमुच्यते}


\twolineshloka
{अजस्रदानात्सततमाजस्रमिति निश्चितम्}
{सामान्यं सर्ववर्णानां दानं शृणु समाहिता}


\twolineshloka
{अन्नं प्राणो मनुष्याणामन्नदः प्राणदो भवेत्}
{तस्माकदन्नं विशेषेण दातुमिच्छति मानवः}


\twolineshloka
{ब्राह्मणायाभिरूपाय यो दद्यादन्नमीप्सितम्}
{निदधाति निधिं श्रेष्ठं सोऽनन्तं पारलौकिकम्}


\twolineshloka
{श्रान्तमध्वपरिश्रान्तमतिथिं गृहमागतम्}
{अर्चयीत प्रयत्नेन स हि यज्ञो वरप्रदः}


\twolineshloka
{कृत्वा तु पातकं कर्म यो दद्यादन्नमर्थिनाम्}
{ब्राह्मणानां विशेषेण सोपहन्ति स्वकं तमः}


\twolineshloka
{पितरस्तस्य नन्दन्ति सुवृष्ट्या कर्षका इव}
{पुत्रो यस्य तु पौत्रो वा श्रोत्रियं भोजयिष्यति}


\twolineshloka
{अपि चण्डालशूद्राणामन्नदानं न गर्हते}
{तस्मात्सर्वप्रयत्नेन दद्यादन्नममत्सरः}


\twolineshloka
{कलत्रं पीडयित्वाऽपि पोषयेदतिथीन्सदा}
{जन्मापि मानुषे लोके तदर्थं हि विधीयते}


\twolineshloka
{अन्नदानाच्च लोकांस्तान्सम्प्रवक्ष्याम्यनिन्दिते}
{भवनानि प्रकाशन्ते दिवि तेषां महात्मनाम्}


\twolineshloka
{अनेकशतभौमानि सान्तर्जलवनानि च}
{वैडूर्यार्चिःप्रकाशपनि हेमरूप्यमयानि च}


\twolineshloka
{नानारूपाणि संस्थानां नानारत्नमयानि च}
{चन्द्रमण्डलशुभ्राणि किङ्किणीजालवन्ति च}


\twolineshloka
{तरुणादित्यवर्णानि स्थावराणि चराणि च}
{यथेष्टभक्ष्यभोज्यानि शयनासनवन्ति च}


\twolineshloka
{सर्वकामफलाश्चात्र वृक्षा भवनसंस्थिताः}
{वाप्यो बह्व्यश्च कूपाश्च दीर्घिकाश्च सहस्रशः}


\twolineshloka
{अरुजानि विशोकानि नित्यानि विविधानि च}
{भवनानि विचित्राणि प्राणदानां त्रिविष्टपे}


\twolineshloka
{विवस्वतश्च सोमस्य ब्रह्मणश्च प्रजापतेः}
{विशन्ति लोकांस्ते नित्यं जगत्यन्नोदकप्रदाः}


\twolineshloka
{तत्र ते सुचिरं कालं विहृत्याप्सरसां गणैः}
{जायन्ते मानुषे लोके सर्वकल्याणसंयुताः}


\twolineshloka
{बलसंहननोपेता नीरोगाश्चिरजीविनः}
{कुलीना मतिमन्तश्च भवन्त्यन्नप्रदा नराः}


\twolineshloka
{तस्मादन्नं विशेषेण दातव्यं भूतिमिच्छता}
{सर्वकालं च सर्वस्य सर्वत्र च सदैव च}


\twolineshloka
{सुवर्णदानं परमं स्वर्ग्यं स्वस्त्ययनं महत्}
{तस्मात्ते वर्णयिष्यामि यथावदनुपूर्वशः}


% Check verse!
अपि पापकृतं क्रूरं दत्तं रुक्मं प्रकाशयेत्
\twolineshloka
{सुवर्णं ये प्रयच्छन्ति श्रोत्रिभेभ्यः सुचेतसः}
{देवतास्ते तर्पयन्ति समस्ता इति वैदिकम्}


\twolineshloka
{अग्निर्हि देवताः सर्वाः सुवर्णं चाग्निरुच्यते}
{तस्मात्सुवर्णदानेन तृप्ताः स्युः सर्वदेवताः}


\twolineshloka
{अग्न्यभावे तु कुर्वन्ति वह्निस्थानेषु काञ्चनम्}
{तस्मात्सुवर्णदातारः सर्वान्कामानवाप्नुयुः}


\twolineshloka
{आदित्यस्य हुताशस्य लोकान्नानाविधाञ्शुभान्}
{काञ्चनं सम्प्रदायाशु प्रविशन्ति न संशयः}


\twolineshloka
{अलङ्कारं कृतं चापि केवलान्प्रविशिष्यते}
{सौवर्णैर्ब्राह्मणं काले तैरलङ्कृत्य भोजयेत्}


\twolineshloka
{य एतत्परमं दानं दत्त्वा सौवर्णमद्भुतम्}
{द्युतिं मेधां वपुः कीर्तिं पुनर्जाते लबेद्ध्रुवम्}


\twolineshloka
{तस्मात्स्वशक्त्या दातव्यं काञ्चनं भुवि मानवैः}
{न ह्येतस्मात्परं लोकेष्वन्यत्पापात्प्रमुच्यते}


\twolineshloka
{अत ऊर्ध्वं प्रवक्ष्यामि गवां दानमनिन्दिते}
{नहि गोभ्यः परं दानं विद्यते जगति प्रिये}


\twolineshloka
{लोकान्सिसृक्षुणा पूर्वं गावः सृष्टाः स्वयंभुवा}
{वृत्त्यर्थं सर्वभूतानां तस्मात्ता मातरः स्मृताः}


\twolineshloka
{लोकज्येष्ठा लोकवृत्त्या प्रवृत्तामय्यायत्ताः सोमनिष्यन्दभूताः}
{सौम्याः पुण्याः प्राणदाः कामदाश्चतस्मात्पूज्याः पुण्यकामैर्मनुष्यैः}


\twolineshloka
{धेनुं दत्त्वा निभृतां सुशीलांकल्याणवत्सां च पयस्विनीं च}
{यावन्ति रोमाणि भवन्ति तस्या-स्तावत्समाः स्वर्गफलानि भुङ्क्ते}


\twolineshloka
{प्रयच्छते यः कपिलां सचेलांसकांस्यदोहां कनकाग्र्यशृङ्गीम्}
{पुत्राश्च पौत्रांश्च कुलं च सर्व-मासप्तमं तारयते परत्र च}


\threelineshloka
{अन्तर्जाताः क्रीतका द्यूतलब्धाः}
{प्राणक्रीताः सोदकाश्चौजसा वा}
{कृच्छ्रोत्सृष्टाः पोषणार्थागताश्चद्वारैरेतैस्ताः प्रलब्धाः प्रदद्यात्}


\twolineshloka
{कृसाय बहुपुत्राय श्रोत्रियायाहिताग्नये}
{प्रदाय नीरुजां धेनुं लोकान्प्राप्नोत्यनुत्तमान्}


\twolineshloka
{नृशंसस्य कृतघ्नस्य लुब्धस्यानृतवादिनः}
{हव्यकव्यव्यपेतस्य न दद्याद्गाः कथञ्चन}


\twolineshloka
{समानवत्सां यो दद्याद्धेर्नुं विप्रे पयस्विनीम्}
{सुवृत्तां वस्त्रसंछन्नां सोमलोके महीयते}


\twolineshloka
{समानवत्सां यो दद्यात्कृष्णां धेनुं पयस्विनीम्}
{सुवृत्तां वस्त्रसंछन्नां लोकान्प्राप्नोत्यपांपतेः}


\twolineshloka
{हिरण्यवर्णां पिङ्गाक्षीं सवत्सां कांस्यदोहनाम्}
{प्रदाय वस्त्रसम्पन्नां यान्ति कौकबेरसद्मनः}


\twolineshloka
{वायुरेणुसवर्मां च सवत्सां कांस्यदोहनाम्}
{प्रदाय वस्त्रसम्पन्नां वायुलोके महीयते}


\twolineshloka
{समानवत्सां यो धेनुं दत्त्वा गौरीं पयस्विनीम्}
{सुवृत्तां वस्त्रसंछन्नामग्निलोके महीयते}


\twolineshloka
{युवानं बलिनं श्यामं शतेन सह यूथपम्}
{गवेन्द्रं ब्राह्मणेन्द्राय भूरिशृङ्गमलंकृतम्}


\twolineshloka
{ऋषभं ये प्रयच्छन्ति श्रोत्रियाणां महात्मनाम्}
{ऐश्वर्यमभिजायन्ते जायमानाः पुनःपुनः}


\twolineshloka
{गवां मूत्रपुरीषाणि नोद्विजेन कदाचन}
{न चासां मांसमश्नीयाद्गोषु भक्तः सदा भवेत्}


\twolineshloka
{ग्रासमुष्टिं परगवे दद्यात्संवत्सरं शुचि}
{अकृत्वा स्वयमाहारं व्रतं तत्सार्वकामिकम्}


\twolineshloka
{गवामुभयतः काले नित्यं स्वस्त्ययनं वदेत्}
{न चासां चिन्तयेत्पापमिति धर्मविदो विदुः}


\twolineshloka
{गावः पवित्रं परमं गोषु लोकाः प्रतिष्ठिताः}
{कथञ्चिन्नावमन्तव्या गावो लोकस्य मातरः}


\twolineshloka
{तस्मादेव गवां दानं विशिष्टमिति कथ्यते}
{गोषु पूजा च भक्तिश्च नरस्यायुष्यतां वहेत्}


\twolineshloka
{अतःपरं प्रवक्ष्यामि भूमिदानं महाफलम्}
{भूमिदानसमं दानं लोके नास्तीति निश्चयः}


\twolineshloka
{गृहयुक्क्षेत्रयुग्वाऽपि भूमिभागः प्रदीयते}
{सुखभोगं निराक्रोशं वास्तुपूर्वं प्रकल्प्य च}


\threelineshloka
{ग्रहीतारमलङ्कृत्य वस्त्रपुष्पानुलोपनैः}
{सभृत्यं सपरीवारं भोजयित्वा यथेष्टतः}
{यो दद्याद्दक्षिणां काले त्रिरद्भिर्गृह्यतामिति}


\twolineshloka
{एवं भूम्यां प्रदत्तायां श्रद्धया वीतमत्सरैः}
{यावत्तिष्ठति सा भूमिस्तावद्दत्तफलं विदुः}


\twolineshloka
{भूमिदः स्वर्गमारुह्य रमते शाश्वतीः समाः}
{अचला ह्यक्षया भूमिः सर्वकामान्दुधुक्षति}


\twolineshloka
{यत्किञ्चित्कुरुते पापं पुरुषो वृत्तिकर्शितः}
{अपि गोकर्णमात्रेण भूमिदानन मुच्यते}


\twolineshloka
{सुवर्णं रजतं वस्त्रं मणिमुक्तावसूनि च}
{सर्वमेतन्महाभागे भूमिदाने प्रतिष्ठितम्}


\twolineshloka
{भर्तुर्निः श्रेयसे युक्तास्त्यक्तात्मानो रणे हताः}
{ब्रह्मलोकाय संसिद्धा नातिक्रामन्ति भूमिदम्}


\twolineshloka
{हलकृष्टां महीं दद्यात्सर्वबीजफलान्विताम्}
{सुकूपशरणां वाऽपि सा भवेत्सर्वकामदा}


\twolineshloka
{निष्पन्नसस्यां पृथिवीं यो ददाति द्विजन्मनाम्}
{विमुक्तः कलुषैः सर्वैः शकलोकं स गच्छति}


\twolineshloka
{यथा जनित्री क्षीरेणि स्वपुत्रमभिवर्धयेत्}
{एवं सर्वफलैर्भूमिर्दातारमभिवर्धयेत्}


\twolineshloka
{ब्राह्मणं वृत्तसम्पन्नमाहिताग्निं शुचिव्रतम्}
{ग्राहयित्वा निजां भूमिं न यान्ति यमसादनम्}


\twolineshloka
{यथा चन्द्रमसो वृद्धिरहन्यहनि दृश्यते}
{तथा भूमेः कृतं दानं सस्येसस्ये विवर्धते}


\twolineshloka
{यथा बीजानि रोहन्ति प्रकीर्णानि महीतले}
{तथा कामाः प्ररोहन्ति भूमिदानगुणार्जिताः}


\twolineshloka
{पितरः पितृलोकस्था देवताश्च दिवि स्थिताः}
{सन्तर्पयन्ति भोगैस्तं यो ददाति वसुन्धराम्}


\twolineshloka
{दीर्घायुष्यं वराङ्गत्वं स्फीतां च श्रियमुत्तमाम्}
{परत्र लभते मर्त्यः सम्प्रदाय वसुन्धराम्}


\twolineshloka
{एतत्सर्वं मयोद्दिष्टं भूमिदानस्य यत्फलम्}
{श्रद्दधानैर्नरैर्नित्यं श्राव्यमेतत्सनातनम्}


\twolineshloka
{अतः परं प्रवक्ष्यामि कन्यादानं यथाविधि}
{कन्या देया महादेवि परेषामात्मनोपि वा}


\twolineshloka
{कन्यां शुद्धव्रताचारां कुलरूपसमन्विताम्}
{यस्मै दित्सति पात्राय तेनापि भृशकामिताम्}


\twolineshloka
{प्रथमं तत्समाकल्प्य बन्धुभिः कृतनिश्चयः}
{कारयित्वा गृहं पूर्वं दासीदासपरिच्छदैः}


\twolineshloka
{गृहोपकरणैश्चैव पशुधान्येन संयुताम्}
{तदर्थिने तदर्हाय कन्यां तां समलङ्कृताम्}


\twolineshloka
{सविवाहं यथान्यायं प्रचच्छेदग्निसाक्षिकम्}
{वृत्त्यातीं यथा कृत्वा सद्गृहे तौ निवेशयेत्}


\twolineshloka
{एवं कृत्वा वधूदानं तस्य दानस्य गौरवात्}
{प्रेत्यभावे महीयेत स्वर्गलोके यथासुखम्}


% Check verse!
पुनर्जातस्य सौभाग्यं कुलवृद्धिं तथाऽऽप्नुयात्
\twolineshloka
{विद्यादानं तथा देवि पात्रभूताय वै ददत्}
{प्रेत्यभावे लभेन्मर्त्यो मेधां वृद्धिं धृतिं स्मृतिम्}


\twolineshloka
{अनुरूपाय शिष्याय यश्च विद्यां प्रयच्छति}
{यथोक्तस्य प्रदानस्य फलमानन्त्यमश्नुते}


\twolineshloka
{दापनं त्वथ विद्यानां दरिद्रेभ्योऽर्थवेदनैः}
{स्वयं दत्तेन तुल्यं स्यादिति विद्धि शुभानने}


\twolineshloka
{एवं ते कथितान्येव महादानानि मानिनि}
{त्वत्प्रियार्थं मया देवि भूयः श्रोतुं किमिच्छसि}


\chapter{अध्यायः २३६}
\threelineshloka
{भगवन्देवदेवेश कथं देयं तिलान्वितम्}
{तस्य तस्य फलं ब्रूहि दत्तस्य च कृतस्य च ॥महेश्वर उवाच}
{}


% Check verse!
तिलकल्पविधिं देवि तन्मे शृणु समाहिता
\twolineshloka
{समृद्धैरसमृद्धैर्वा तिला देया विशेषतः}
{तिलाः पवित्राः पापघ्नाः सुपुण्या इति संस्मृताः}


\threelineshloka
{न्यायतस्तु तिलाञ्शुद्धान्संहृत्याथ स्वशक्तिनः}
{तिलराशिं पुनः कुर्यात्पर्वाताभं सुरस्नकम्}
{महान्तं यदि वा स्तोकं नानाद्रव्यसमन्वितम्}


\threelineshloka
{सुवर्णरजताभ्यां च मणिमुक्ताप्रवालकैः}
{अलङ्कृत्य यथायोगं सपताकं सवेदिकम्}
{सभूषणं सवस्त्रं च शयनासनसंमितम्}


\twolineshloka
{प्रायशः कौमुदीमासे पौर्णमास्यां विशेषतः}
{भोजयित्वा च विधिवद्ब्राह्म्णानर्हतो बहून्}


\twolineshloka
{स्वयं कृतोपवासश्च वृत्तशौचसमन्वितः}
{दद्यात्प्रदक्षिणीकृत्य तिलराशिं सदक्षिणम्}


\twolineshloka
{एकस्यापि बहूनां वा दातव्यं भूतिमिच्छता}
{तस्य दानफलं देवि अग्निष्टोमेन संयुतम्}


\threelineshloka
{केवलं वा तिलैरेव भूमौ कृत्वा गवाकृतिम्}
{सवस्त्रकं सरत्नं च पुंसा गोदानकाङ्क्षिणा}
{तदर्हाय प्रदातव्यं तस्य गोदानतः फलम्}


\twolineshloka
{शरावांस्तिलसम्पूर्णान्सहिरण्यान्सचम्पकान्}
{नृपोऽददद्ब्राह्मणाय सु पुण्यफलभाग्भवेत्}


\twolineshloka
{एवं तिलमयं देयं नरेण हितमिच्छता}
{नानादानफलं भूयः शृणु देवि समाहिता}


\twolineshloka
{बलमायुष्यमारोग्यमन्नदानाल्लभेन्नरः}
{पानीयदस्तु सौभाग्यं रसज्ञानं लभेन्नरः}


\twolineshloka
{वस्त्रदानाद्वपुःशोभामलङ्कारं लभेन्नरः}
{दीपदो बुद्धिवैशद्यं द्युतिशोभां लभेन्नरः}


\threelineshloka
{राजपीडाविमोक्षं तु छत्रदो लभते फलम्}
{दासीदासप्रदानात्तु भवेत्कर्मान्तभाङ्नरः}
{दासीदासं च विविधं लभेत्प्रेत्य गुणान्वितम्}


\threelineshloka
{यानानि वाहनं चैव तदर्हाय ददन्नरः}
{पादरोगपरिक्लेशान्मुक्तः श्वसनवाहवान्}
{विचित्र रमणीयं च लभते यानवाहनम्}


\twolineshloka
{प्रतिश्रयप्रदानं च तदर्हाय तदिच्छते}
{वर्षाकाले तु रात्रौ वा लभेत्पक्षबलं शुभम्}


\twolineshloka
{सेतुकूपतटाकानां कर्ता तु लभते नरः}
{दीर्घायुष्यं च सौभाग्यं तता प्रेत्य गतिं शुभां}


\twolineshloka
{वृक्षसंरोपको यस्तु छायापुष्पफलप्रदः}
{प्रेत्यभावे लभेत्पुण्यमभिगम्यो भवेन्नरः}


\twolineshloka
{यस्तु सङ्क्रमकृल्लोके नदीषु जलहारिणाम्}
{लभेत्पुण्यफलं प्रेत्य व्यसनेभ्यो विमोक्षणम्}


\twolineshloka
{मार्गकृत्सततं मर्त्यो भवेत्सन्तानवान्नरः}
{कायदोषविमुक्तस्तु तीर्थकृत्सततं भवेत्}


\twolineshloka
{औषधानां प्रदानात्तु सततं कृपयाऽन्वितः}
{भवेद्व्याधिविहीनश्च दीर्घायुश्च विशेषतः}


\twolineshloka
{अनाथान्पोषयेद्यस्तु कृपणान्धकपङ्गुकान्}
{स च पुण्यफलं प्रेत्य लभते कृच्छ्रमोक्षणम्}


\twolineshloka
{वेदगोष्ठाः शुभाः शाला भिक्षूणां च प्रतिश्रयम्}
{यः कुर्याल्लभते नित्यं नरः प्रेत्य फलं शुभम्}


\twolineshloka
{प्रासादवासं विविधं यक्षशोभां लभेत्पुनः}
{विविधं विविधाकारं भक्ष्यभोज्यगुणान्वितम्}


\twolineshloka
{रम्यं तं दैवगोवाटं यः कुर्याल्लभते नरः}
{प्रेत्य भावे शुभां जातिं व्याधिमोक्षं तथैव च}


\twolineshloka
{एवं नानाविधं द्रव्यं दानकर्ता लभेत्फलम् ॥उमोवाच}
{}


\threelineshloka
{कृतं दत्तं यथा यावत्तस्य तल्लभते फलम्}
{एतन्मे देवदेवेश तत्र कौतूहलं महत् ॥महेश्वर उवाच}
{}


\threelineshloka
{प्रेत्यिभावे शृणु फलं दत्तस्य च कृतस्य च}
{दानं षङ्गुणयुक्तं तु तदर्हाय यथाविधिः}
{यथाविभवतो दानं दातव्यमिति मानवैः}


\twolineshloka
{बुद्धिमायुष्यमारोग्यं बलं भाग्यं तथाऽऽगमम्}
{रूपेण सप्तधा भूत्वा मानुष्यं फलति ध्रुवम्}


\twolineshloka
{इदं दत्तमिदं देयमित्येवं फलकाङ्क्षया}
{यद्दत्तं तत्तदेव स्यान्न तु किञ्चन लभ्यते}


% Check verse!
ध्रुवं देव्यत्तमे दानं मध्यमे त्वधमं फलम्
\chapter{अध्यायः २३७}
\threelineshloka
{भगवन्देवदेवेश विशिष्टं यज्ञमुच्यते}
{लौकिकं वैदिकं चैव तन्मे शंसितुमर्हसि ॥महेश्वर उवाच}
{}


\twolineshloka
{देवतानां तु पूजा या यज्ञेष्वेव समाहिता}
{यज्ञा वेदेष्वधीताश्च वेदा ब्राह्मणसंयुताः}


\twolineshloka
{इदं तु सकलं दिव्यं दिवि वा भुवि वा प्रिये}
{यज्ञार्थं विद्धि तत्सृष्टं लोकानां हितकाम्यया}


\twolineshloka
{एवं विज्ञाय तत्कर्ता सदारः सततं द्विजः}
{प्रेत्यभावे लभेल्लोकान्ब्रह्मकर्मसमाधिना}


\twolineshloka
{ब्राह्मणेष्वेव तद्ब्रह्म नित्यं देवि समाहितम्}
{तस्माद्विप्रैर्यथाशास्त्रं विधिदृष्टेन कर्मणा}


\twolineshloka
{यज्ञकर्मि कृतं सर्वं देवता अभितर्पयेत्}
{ब्राह्मणाः क्षत्रियाश्चैव यज्ञार्थं प्रायशः स्मृताः}


\threelineshloka
{अग्निष्टोमादिभिर्यज्ञैर्वेदेषु परिकल्पितैः}
{सुशुद्धैर्यजमानैस्च ऋत्विग्भिश्च यथाविधिः}
{शुद्धैर्द्रव्योपकरणैर्यष्टव्यमिति निश्चयः}


\twolineshloka
{तथा कृतेषु यज्ञेषु देवानां तोषणं भवेत्}
{तुष्टेषु सर्वदेवेषु यज्वा यज्ञफलं लभेत्}


\twolineshloka
{देवाः सन्तोषिता यज्ञैर्लोकान्संवर्धयन्त्युत}
{उभयोर्लोकयोर्भूतिर्देवि यज्ञे प्रदृश्यते}


\twolineshloka
{तस्माद्यज्वा दिवं गत्वा अमरैः सह मोदते}
{नास्ति यज्ञसमं दानं नास्ति यज्ञसमो निधिः}


\twolineshloka
{सर्वधर्मसमुद्देशो देवि यज्ञे समाहितः}
{एषा यज्ञकृता पूजा लौकिकीमपरां शृणु}


% Check verse!
देवसत्कारमुद्दिश्य क्रियते लौकिकोत्सवः
\threelineshloka
{देवगोष्ठेऽधिसंस्कृत्य चोत्सवं यः करोति वै}
{यागान्देवोपहारांश्च शुचिर्भूत्वा यथाविधि}
{देवान्सन्तोषयित्वा स देवि धर्ममवाप्नुयात्}


\twolineshloka
{गन्धमाल्यैश्च विविधैः परमान्नेन धूपनैः}
{बह्वीभिः स्तुतिभिश्चैव स्तुवद्भिः प्रयतैर्नरैः}


\twolineshloka
{नृत्तैर्वाद्यैश्च गान्धर्वैरन्यैर्दृष्टिविलोभनैः}
{देवसत्कारमुद्दिश्य कुर्वते ये नरा भुवि}


\twolineshloka
{तेषां भक्तिकृतेनैव सत्कारेणैव पूजिताः}
{तेनैव तोषं संयान्ति देवि देवास्त्रिविष्टपे}


\twolineshloka
{मानुषैश्चोपकारैर्वा शुचिभिः सत्परायणैः}
{ब्रह्मचर्यपरैरेतत्कृतं धर्मफलं लभेत्}


\threelineshloka
{केवलैः स्तुतिभिर्देवि गन्धमाल्यसमाहितैः}
{प्रयतैः शुद्धगात्रैस्तु शुद्धदेशे सुपूजिताः}
{सन्तोषं यान्ति ते देवा भक्तैः सम्पूजितास्तथा}


\twolineshloka
{देवान्सन्तोषयित्वैव देवि धर्ममवाप्नुयात् ॥उमोवाच}
{}


\threelineshloka
{त्रिविष्टपस्था वै भूमौ देवा मानुषचेष्टितम्}
{कथं ज्ञास्यन्ति विधिवत्तन्मे शंसितुमर्हसि ॥महेश्वर उवाच}
{}


\twolineshloka
{तदहं तेप्रवक्ष्यामि यथा तैर्विद्यते प्रिये}
{प्राणिनां तु शरीरेषु अन्तरात्मा व्यवस्थितः}


\twolineshloka
{आत्मानं परमं देवमिति विद्धि शुभेक्षणे}
{आत्मा मनोव्यवस्थानात्सर्वं वेत्ति शुभाशुभम्}


\twolineshloka
{आत्मैव देवास्तद्विद्युरव्यग्रमनसा कृतम्}
{सतां मनोव्यवस्थानाच्छुभं भवति वै नृणाम्}


\twolineshloka
{तस्माद्देवाऽभिसम्पूज्या ब्राह्मणानां तथैव च}
{यज्ञाश्च धर्मकार्याणि गुरुपूजा च शोभने}


\twolineshloka
{शुद्धगात्रैर्व्रतयुतैस्तन्मयैस्तत्परायणैः}
{एवं व्यवस्थितैर्नित्यं कर्तव्यमिति निश्चयः}


\threelineshloka
{एवं कृत्वा शुभाकाङ्क्षी परत्रेह च मोदते}
{अन्यथा मन आविश्य कृतं न फलति प्रिये}
{ऋतेऽपि तु मनो देवि अशुभं फलति ध्रुवम्}


\chapter{अध्यायः २३८}
\threelineshloka
{पितृमेधः कथं देव तन्मे शंसितुमर्हसि}
{सर्वेषां पितरः पूज्याः सर्वसम्पत्प्रदायिनः ॥महेश्व उवाच}
{}


\twolineshloka
{पितृमेधं प्रवक्ष्यामि यथावत्तन्मना शृणु}
{देशकालौ विधानं च तत्क्रियायाः शुभाशुभम्}


\twolineshloka
{लोकेषु पितरः पूज्या देवतानां च देवताः}
{शुचयो निर्मलाः पुण्या दक्षिणां दिशमाश्रिताः}


\twolineshloka
{यथा वृष्टिं प्रतीक्षन्ते भूमिष्ठाः सर्वजन्तवः}
{पितरश्च तथा लोके पितृमेधं शुभेक्षणे}


\twolineshloka
{तस्य देशाः कुरुक्षेत्रं गया गङ्गा सरस्वती}
{प्रभासं पुष्करं चेति तेषु दत्तं महाफलम्}


\twolineshloka
{तीर्थानि सरितः पुण्या विविक्तानि वनानि च}
{नदीनां पुलिनानीति देशाः श्राद्धस्य पूजिताः}


\twolineshloka
{माघप्रोष्ठपदौ मासौ श्राद्धकर्मणि पूजितौ}
{पक्षयोः कृष्णपक्षश्च पूर्वपक्षात्प्रशस्यते}


\twolineshloka
{अमावास्यां त्रयोदश्यां नवम्यां प्रतिपत्सु च}
{तिथिष्वेतासु तुष्यन्ति दत्तेनेह पितामहाः}


\twolineshloka
{पूर्वाह्णे शुक्लपक्षे च रात्रौ जन्मदिनेषु वा}
{युग्मेष्वहस्सु च श्राद्धं न च कुर्वीत पण्डितः}


\twolineshloka
{एष कालो मया प्रोक्तः पितृमेधस्य पूजितः}
{यस्मिंश्च ब्राह्म्णं पात्रं पश्येत्कालः स च स्मृतः}


\twolineshloka
{अपाङ्क्तेया द्विजा वर्ज्या ग्राह्यास्ते पङ्क्तिपावनाः}
{भोजयेद्यदि पापिष्ठाञ्श्राद्धेषु नरकं व्रजेत्}


\twolineshloka
{वृत्तश्रुतकुलोपेतान्सकलत्रान्गुणान्वितान्}
{तदर्हाञ्श्रोत्रियान्विद्धि ब्राह्मणानयुजः शुभे}


\twolineshloka
{एतान्निमन्त्रयोद्विद्वान्पूर्वेद्युः प्रातरेव वा}
{तत्र श्राद्धक्रियां पश्चादारभेत यथाविधि}


\twolineshloka
{त्रीणि श्राद्धे पवित्राणि दौहित्रः कुतपस्तिलाः}
{त्रीणि चात्र प्रशंसन्ति शौचमक्रोधमत्वरा}


\twolineshloka
{कुतपः खङ्गपात्रं च कुशा दर्भास्तिला मधु}
{कालशाकं गजच्छाया पवित्रं श्राद्धकर्मसु}


\twolineshloka
{तिलानवकिरेत्तत्र नानावर्णान्समन्ततः}
{अशुद्धं पितृयज्ञश्च तिलैः शुध्यति शोभने}


\twolineshloka
{नीलकाषायवस्त्रं च भिन्नवर्णं नवव्रणम्}
{हीनाङ्गमशुचिं वाऽपि वर्जयेत्तत्र दूरतः}


\twolineshloka
{कुक्कुटांश्च वराहांश्च नग्नं क्लीबं रजस्वलाम्}
{आयसं त्रपुसीसं च श्राद्धकर्मणि वर्जयेत्}


\twolineshloka
{मांसैः प्रीणन्ति पितरो मुद्गमाषयवैरिह}
{शशरौरवमांसेन षण्मासं तृप्तिरिष्यते}


\twolineshloka
{संवत्सरं च गव्येन हविषा पायसेन च}
{वार्ध्रीणसस्य मांसेन तृप्तिर्द्वादशवार्षिकी}


\twolineshloka
{आनन्त्याय भवेद्दत्तं खङ्गमांसं पितृक्षये}
{पायसं सतिलं क्षौद्रं खङ्गमांसेन सम्मितम्}


\twolineshloka
{महाशकलिनो मस्याश्छागो वा सर्वलोहितः}
{कालशाकमितीत्येव तदानन्त्याय कल्पितम्}


\twolineshloka
{सापूपं सामिषं स्निग्धमाहारमुपकल्पयेत्}
{उपकल्प्य तदाहारं ब्राह्मणानर्चयेत्ततः}


\twolineshloka
{श्मश्रुकर्मशिरः स्नातान्समारोप्यासनं क्रमात्}
{सुगन्धमाल्याभरणैः स्नग्भिरेतान्विभूषयेत्}


% Check verse!
अलङ्कृत्योपविष्टांस्तान्पिण्डावापं निवेदयेत्
\threelineshloka
{ततः प्रस्तीर्य दर्भाणां प्रस्तरं दक्षिणामुखम्}
{तत्समीपेऽग्निमिद्ध्वा च स्वधां च जुहुयात्ततः}
{समीपे त्वग्नीषोमाभ्यां पितृभ्यो जुहुयात्तदा}


\twolineshloka
{तथा दर्भेषु पिण्डांस्त्रीन्निर्वपेद्दक्षिणामुखः}
{अपसव्यमपाङ्गुष्ठं नामधेयपुरस्कृतम्}


\threelineshloka
{एतेन विधिना दत्तं पितॄणामक्षयं भवेत्}
{ततो विप्रान्यथाशक्ति पूजयेन्नियतः शुचिः}
{सदक्षिणं ससम्भारं यथा तुष्यन्ति ते द्विजाः}


\twolineshloka
{यत्र तत्क्रियते तत्र न जल्पन्न जपेन्मिथः}
{नियम्य वाच्यं देहं च श्राद्धकर्म समारभेत्}


\twolineshloka
{ततो निर्वपने वृत्ते तान्पिण्डांस्तदनन्तरम्}
{ब्राह्मणोऽग्निरजो गौर्वा भक्षयेदप्सु वा क्षिपेत्}


\twolineshloka
{पत्नीं वा मध्यमं पिण्डं पुत्रकामो हि प्राशयेत्}
{आधत्त पितरो गर्भं कुमारं पुष्करस्रजम्}


\twolineshloka
{तृप्तानुत्थाप्य तान्विप्रानन्नशेषं निवेदयेत्}
{तच्छेषं बहुभिः पश्चात्सभृत्यो भक्षयेन्नरः}


\twolineshloka
{एष प्रोक्तः समासेन पितृयज्ञः सनातनः}
{पितरस्तेन तुष्यन्ति कर्ता च फलमाप्नुयात्}


\twolineshloka
{अहन्यहनि वा कुर्यान्मासेमासेऽथवा पुनः}
{संवत्सरं द्विः कुर्याच्च चतुर्वाऽपि स्वशक्तितः}


\twolineshloka
{दीर्घायुश्च भवेत्स्वस्थः पितृमेधेन वा पुनः}
{सपुत्रो बहुभृत्यश्च प्रभूतधनधान्यवान्}


\twolineshloka
{श्राद्धदः स्वर्गमाप्नोति निर्मलं विविधात्मकम्}
{अप्सरोगणसंघुष्टं विरजस्कमनन्तरम्}


\twolineshloka
{श्राद्धानि पुष्टिकामा वै ये प्रकुर्वन्ति पण्डिताः}
{तेषां पुष्टिं प्रजां चैव दास्यन्ति पितरः सदा}


\threelineshloka
{धन्यं यशस्यमायुष्यं स्वर्ग्यं शत्रुविनाशनम्}
{कुलसन्धारकं चेति श्राद्धमाहुर्मनीषिणः ॥उमोवाच}
{}


\twolineshloka
{भगवन्देवदेवेश मृतास्ते भुवि जन्तवः}
{नानाजातिषु जायन्ते शीघ्रं कर्मवशात्पुनः}


\twolineshloka
{पितरः स्वस्ति ते तत्र कथं तिष्ठन्ति देववत्}
{पितॄणां कतमो देशः पिण्डानश्नन्ति वै कथम्}


\threelineshloka
{अन्ने दत्ते मृतानां तु कथमाप्यायनं भवेत्}
{एवं मया संशयितं भगवन्वक्तुमर्हसि ॥नारद उवाच}
{}


\threelineshloka
{एतद्विरुद्धं पृच्छन्त्यां रुद्राण्यां परिषद्भृशम्}
{बभूव सर्वा मुदिता श्रोतुं हि परमं हितम् ॥महेश्वर उवाच}
{}


\twolineshloka
{स्थाने संशयितं देवि शृणु कल्याणि तत्वतः}
{गुह्यानां परमं गुह्यं हितानां परमं हितम्}


\twolineshloka
{यथा देवगणा देवि तथा पितृगणाः प्रिये}
{दक्षिणस्यां दिशि शुभे सर्वे पितृगणाः स्थिताः}


\twolineshloka
{प्रेतानुद्दिश्य या पूजा क्रियते मानुषैरिह}
{तेन तुष्यन्ति पितरो न प्रेताः पितरः स्मृताः}


\twolineshloka
{उत्तरस्यां यथा देवा रमन्ते यज्ञकर्मभिः}
{दक्षिणस्यां तथा देवि तुष्यन्ति विविधैर्मखैः}


\threelineshloka
{द्विविधं क्रियते कर्म हव्यकव्यसमाश्रितम्}
{तयोर्हव्यक्रिया देवान्कव्यमाप्यायते पितॄन्}
{}


\twolineshloka
{प्रसव्यं मङ्गलैर्द्रव्यैर्हव्यकर्म विधीयते}
{अपसव्यममङ्गल्यैः कव्यं चापि विधीयते}


\twolineshloka
{सदेवासुरगन्धर्वाः पितॄनभ्यर्चयन्ति च}
{आप्यायन्ते च ते श्राद्धैः पुनराप्याययन्ति तान्}


\twolineshloka
{अनिष्टा च पितॄन्पूर्वं यः क्रियां प्रकरोति चेत्}
{रक्षांसि च पिशाचाश्च फलं भोक्ष्यन्ति तस्य तत्}


\twolineshloka
{हव्यकव्यक्रियास्तस्मात्कर्तव्या भुवि मानुषैः}
{कर्मक्षेत्रं हि मानुष्यं तदन्यत्र न विद्यते}


\twolineshloka
{कव्येन सन्ततिर्दृष्टा हव्ये भूतिः पृथग्विधाः}
{इति ते कथितं देवि देवगुह्यं सनातनम्}


\chapter{अध्यायः २३९}
\threelineshloka
{एवं कृतस्य धर्मस्य श्रोतुमिच्छाम्यहं प्रभो}
{प्रमाणं फलमानानां तन्मे शंसितुमर्हसि ॥महेश्वर उवाच}
{}


% Check verse!
प्रमाणकल्पनां देवि दानस्य शृणु भामिनि
\twolineshloka
{यत्सारस्तु नरो लोके तद्दानं चोत्तमं स्मृतम्}
{सर्वदानविधिं प्राहुस्तदेव भुवि शोभने}


\twolineshloka
{प्रस्थं सारं दरिद्रस्य शतं कोटिधनस्य च}
{प्रस्थसारस्तु तत्प्रस्थं ददन्महदषाप्नुयात्}


\twolineshloka
{कोटिसारस्तु तां कोटिं ददान्महदवाप्नुयात्}
{उभयं तन्महत्तच्च फलेनैव समं स्मृतम्}


\twolineshloka
{धर्मार्थकामभोगेषु शक्त्यभावस्तु मध्यमम्}
{स्वद्रव्यादतिहीनं तु तद्दानमधमं स्मृतम्}


\twolineshloka
{शृणु दत्तस्य वै देवि पञ्चधा फलकल्पनाम्}
{आनन्त्यं च महच्चैव समं हीनं हि पातकम्}


\twolineshloka
{तेषां विशेषं वक्ष्यामि शृणु देवि समाहिता}
{दुस्त्यजस्य च वै दानं पात्र आनन्त्यमुच्यते}


\twolineshloka
{दानं षङ्गुणयुक्तं तु महदित्यभिधीयते}
{यथाश्रद्धं तु वै दानं यथार्हं सममुच्यते}


\twolineshloka
{गुणतस्तु तथा हीनं दानं हीनमिति स्मृतम्}
{दानं पातकमित्याहुः षड्गुणानां विपर्यये}


\twolineshloka
{देवलोके महत्कालमान्त्यस्य फलं विदुः}
{महतस्तु तथा कालं स्वर्गलोके तु पूज्यते}


\twolineshloka
{समस्य तु तदा दानं मानुष्यं भोगमावहेत्}
{दानं निष्फलमित्याहुर्विहीनं क्रियया शुभे}


\threelineshloka
{अथवा म्लेच्छदेशेषु तत्र तत्फलतां व्रजेत्}
{नरकं प्रेत्य तिर्यक्षु गच्छेदशुभदानतः ॥उमोवाच}
{}


\twolineshloka
{अशुभस्यापि दानस्य शुभं स्याच्च फलं कथम् ॥महेश्वर उवाच}
{}


\twolineshloka
{मनसा तत्वतः शुद्धमानृशंस्यपुरःसरम्}
{प्रीत्या तु सर्वदानानि दत्त्वा फलमवाप्नुयात्}


\twolineshloka
{रहस्यं सर्वदानानामेतद्विद्धि शुभेक्षणे}
{अन्यानि धर्मकार्याणि शृणु सद्भिः कृतानि च}


\twolineshloka
{आरामदेवगोष्ठानि संक्रमाः कूप एव च}
{गोवाटश्च तटाकश्च सभा शाला च सर्वशः}


\twolineshloka
{पाषण्डावसथश्चैव पानीयं गोतृणानि च}
{व्याधितानां च भैषज्यमनाथानां च पोषणम्}


\twolineshloka
{अनाथशवसंस्कारस्तीर्थमार्गविशोधनम्}
{व्यसनाभ्यवपत्तिश्च सर्वेषां च स्वशक्तितः}


\twolineshloka
{एतत्सर्वं समासेन धर्मकार्यमिति स्मृतम्}
{तत्कर्तव्यं मनुष्येण स्वशक्त्या श्रद्धया शुभे}


\fourlineindentedshloka
{प्रेत्यभावे लभेत्पुण्यं नास्ति तत्र विचारणा}
{रूपं सौभाग्यमारोग्यं बलं सौख्यं लभेन्नरः}
{स्वर्गो वा मानुषे वाऽपि तैस्तैराप्यायते हि सः ॥उमोवाच}
{}


\threelineshloka
{भगवन्लोकपालेश धर्मस्तु कतिभेदकः}
{दृश्यते परितः सद्भिस्तन्मे शंसितुमर्हसि ॥महेश्वर उवाच}
{}


\twolineshloka
{शृणु देवि समुद्देशान्नानात्वं धर्मसङ्कटे}
{धर्मा बहुविधा लोके श्रुतिभेदमुखोद्भवाः}


% Check verse!
स्मृतिधर्मश्च बहुधा सद्भिराचार इष्यते
\twolineshloka
{देशधर्माश्च दृश्यन्ते कुलधर्मास्तथैव च}
{जातिधर्माश्च वै धर्मा गणधर्माश्च शोभने}


\twolineshloka
{शरीरकालवैषम्यादापद्धर्मश्च दृश्यते}
{एतद्धर्मस्य नानात्वं क्रियते लोकवासिभिः}


\twolineshloka
{कारणात्तत्रतत्रैव फलं धर्मस्य चेष्यते}
{तत्कारणसमायोगे लभेत्कुर्वन्फलं नरः}


\twolineshloka
{अन्यथा न लभेत्पुण्यमतदर्हः समाविशेत्}
{एवं धर्मस्य नानात्वं फलं कुर्वल्लँभेन्नरः}


\twolineshloka
{श्रौतस्मार्तस्तु धर्माणां प्राकृतो धर्म उच्यते}
{इति ते कथितं देवि भूयः श्रोतुं किमिच्छसि}


\chapter{अध्यायः २४०}
\twolineshloka
{भगवन्सर्वभूतेश पुरमर्दन शङ्कर}
{श्रुतं पापकृतां दुःखं यमलोके वरप्रद}


\twolineshloka
{श्रोतुमिच्छाम्यहं देव नृणां सुकृतकर्मणाम्}
{कथं ते भुञ्जते भोगान्स्वर्गलोके महेश्वर}


\threelineshloka
{कथिताः कीदृशा लोका नृणां सुकृतकारिणाम्}
{एतन्मे वद देवश श्रोतुं कौतूहलं हि मे ॥महेश्वर उवाच}
{}


\twolineshloka
{शृणु कल्याणि तत्सर्वं यत्त्वमिच्छसि शोभने}
{विविधाः पुण्यलोकास्ते कर्मकर्मण्यतां गताः}


\twolineshloka
{मेरुं हि कनकात्मानं परितः सर्वतोदिशम्}
{भद्राश्चः केतुमालश्च उत्तराः कुरवस्तथा}


\twolineshloka
{जम्बूवनादयः स्वर्गा इत्येते कर्मवर्जिताः}
{तेषु भूत्वा स्वयंभूताः प्रदृश्यते यतस्ततः}


\twolineshloka
{योजनानां सहस्रं च एकैकं मानमात्रया}
{नित्यं पुष्पफलोपेतास्तत्र वृक्षाः समन्ततः}


\threelineshloka
{आसक्तवस्त्राभरणाः सर्वे कनकसन्निभाः}
{द्विरेफाश्चाण्डजास्तत्र प्रवालमणिसन्निभाः}
{विचित्राश्च मनोज्ञाश्च कूजितैः शोभयन्ति तान्}


\twolineshloka
{कुशेशयवनच्छन्ना नलिन्यश्च मनोरमाः}
{तत्र वान्त्यनिला नित्यं दिव्यगन्धसुखावहाः}


\twolineshloka
{सर्वे चाम्लानमाल्याश्च विरजोम्बरसंवृताः}
{एवं बहुविधा देवि दिव्यभोगाः सुखावहाः}


\twolineshloka
{स्त्रियश्च पुरुषाश्चैव सर्वे सुकृतकारिणः}
{रमन्ते तत्र चान्योन्यं कामरागसमन्विताः}


\twolineshloka
{मनोहरा महाभागाः सर्वे ललितकुण्डलाः}
{एवं तत्र स्थिता मर्त्याः प्रमदाः प्रियदर्शनाः}


\twolineshloka
{नानाभावसमायुक्ता यौवनस्थाः सदैव तु}
{युवत्यः कल्पितास्तत्र कामजा ललितास्तथा}


\twolineshloka
{मनोनुकूला मधुरा भोगिनामुपकल्पिताः}
{प्रमदाश्चोद्भवन्त्येव स्वर्गलोके यथा तथा}


\twolineshloka
{एवंविधाः स्त्रियश्चात्र पुरुषाश्च परस्परम्}
{रमन्ते चेन्द्रियैः स्वस्थै शरीरैर्भोगसंस्कृतैः}


\twolineshloka
{कामहर्षगुणाभ्यस्ता नान्ये क्रोधादयः प्रिये}
{क्षुत्पिपासा न चास्त्यत्रि गात्रक्लेशाश्च शोभने}


\threelineshloka
{सर्वतो रमणीयं च सर्वत्र कुसुमान्विम्}
{यावत्पुण्यफलं तावद्दृश्यन्ते बहुसङ्गताः}
{निरन्तरं भोगयुता रमन्ते स्वर्गवासिनः}


\twolineshloka
{तत्र भोगान्यथायोगं भुक्त्वा पुण्यक्षयात्पुनः}
{नश्यन्ति जायमानास्ते शरीरैः सहसा प्रिये}


\twolineshloka
{स्वर्गलोकात्परिभ्रष्टाः जायन्ते मानुषे पुनः}
{पूर्वपुण्यावशेषेण विशिष्टाः सम्भवन्ति ते}


\threelineshloka
{एषा स्वर्गगतिः प्रोक्ता पृच्छन्त्यास्तव भामिनि}
{अत ऊर्ध्वं पदान्यष्टौ सुकर्माणि शृणु प्रिये}
{भोगयुक्तानि पुण्यानि उच्छ्रितानि परस्परम्}


\twolineshloka
{विद्याधराः किम्पुरुषा यक्षगन्धर्वकिन्नराः}
{अप्सरोदानवा देवा यथाक्रममुदाहृताः}


\twolineshloka
{तेषु स्थानेषु जायन्ते प्राणिनाः पुण्यकर्मणः}
{तेषामपि च ये लोकाः स्वर्गलोकोपमाः स्मृताः}


\twolineshloka
{स्वर्गवत्तत्र ते भोगान्भुञ्जते च रमन्ति च}
{रूपसत्वबलोपेताः सर्वे दीर्घायुषस्तथा}


\twolineshloka
{तेषां सर्वक्रियारम्भो मानुषेष्विव दृश्यते}
{अतिमानुषमैश्वर्यमत्र मायाबलात्कृतम्}


\twolineshloka
{जराप्रसूतिमरणं तेषु स्थानेषु दृश्यते}
{गुणा दोषाश्च सन्त्यत्र आकाशगमनं तथा}


\twolineshloka
{अन्तर्धानं बलं सत्त्वमायुश्च चिरजीवितम्}
{तपोविशेषज्जायन्ते यथा कर्म्णि भामिनि}


\twolineshloka
{देवलोके प्रवृत्तिस्तु तेषामेव विधीयते}
{न तथा देवलोको हि तद्विशिष्टाः सुराः स्मृताः ॥i}


\twolineshloka
{तत्र भोगमनिर्देश्यममृतत्वं च विद्यते}
{विमानगमनं नित्यमप्सरोगणसेवितम्}


\twolineshloka
{एवमन्यच्च तत्कर्म देवताभ्यो विशिष्यते}
{प्रत्यक्षं तव तत्सर्वं देवलोके प्रवर्तनम्}


\twolineshloka
{तस्मान्न वर्णये देवि विदितं च त्वया शुभे}
{तत्सर्वं सुकृतैरेव प्राप्यते चोत्तमं पदम्}


\chapter{अध्यायः २४१}
\twolineshloka
{मानुषेष्वेव जीवत्सु गतिर्विज्ञायते न वा}
{यथा शुभगतिर्जीवो नासौ त्वशुभभागिति}


\twolineshloka
{एतदिच्छाम्यहं श्रोतुं तन्मे शंसितुमर्हसि ॥महेश्वर उवाच}
{}


\twolineshloka
{तदहं ते प्रवक्ष्यामि जीवितं विद्यते यथा}
{द्विविधाः प्राणिनो लोके दैवमासुरमाश्रिताः}


\twolineshloka
{मनसा कर्मणा वाचा प्रतिकला भवन्ति ये}
{तादृशानासुरान्विद्धि मर्त्यास्ते नरकालयाः}


\twolineshloka
{हिंस्राश्चोराश्च धूर्ताश्च परदाराभिमर्शकाः}
{नीचकर्मरता ये च शौचमङ्गलवर्जिताः}


\twolineshloka
{शुचिविद्वेषिणः पापा लोकचारित्रदूषकाः}
{एवं युक्तसमाचारा जीवन्तो नरकालयाः}


\twolineshloka
{लोकोद्वेगकराश्चान्ये पशवश्च सरीसृपाः}
{वृक्षाः कण्टकिनो रूक्षास्तादृशान्विद्धि चासुरान्}


% Check verse!
अपरान्देवपक्षांस्तु शृणु देवि समाहिता
\twolineshloka
{मनोवाक्कर्मभिर्नित्यमनुकूला भवन्ति ये}
{तादृशानमरान्विद्धि ते नराः स्वर्गगामिनः}


\twolineshloka
{शौचार्जवपरा धीराः परार्थं नाहरन्ति ये}
{ये समाः सर्वभूतेषु ते नराः स्वर्गगामिनः}


\twolineshloka
{भयाद्वा वृत्तिहेतोर्वा अनृतं न वदन्ति ये}
{सत्यं वदन्ति सततं ते नराः स्वर्गगामिनः}


\twolineshloka
{धार्मिकाः शौचसम्पन्नाः शुक्ला मधुरवादिनः}
{नाकार्यं मनसेच्छन्ति ते नराः स्वर्गगामिनः}


\threelineshloka
{स्वदुःखमिव मन्यन्ते परेषां दुःखवेदनम्}
{दरिद्रा अपि ये केचिद्याचिताः प्रीतिपूर्वकम्}
{ददत्येव च यत्किञ्चित्ते नराः स्वर्गगामिनः}


\twolineshloka
{आस्तिका मङ्गलपराः सततं वृद्धसेविनः}
{पुण्यकर्मपरा नित्यं ते नराः स्वर्गगामिनः}


% Check verse!
व्रतिनो दानशीलाश्च धर्मशीलाश्च मानवाः ॥ऋजवो मृदवो नित्यं ते नराः स्वर्गगामिनः
\twolineshloka
{गुरुशुश्रूषणपरा देवब्राह्मणपूजकाः}
{कृजज्ञाः कृतविद्याश्च ते नराः स्वर्गगामिनः}


\twolineshloka
{जितेन्द्रिया जितक्रोधा जितमानमदाः स्मृताः}
{लोभमात्सर्यहीना ये ते नराः स्वर्गगामिनः}


\twolineshloka
{निर्मा निरहङ्कारः सानुक्रोशाः स्वबन्धुषु}
{दीनानुकम्पिनो नित्यं ते नराः स्वर्गगामिनः}


\twolineshloka
{ऐहिकेन तु वृत्तेन पारत्रमनुमीयते}
{एवंविधा नरा लोके जीवन्तः स्वर्गगामिनः}


\threelineshloka
{यदन्यच्च शुभं लोके प्रजानुग्रहकारि च}
{पशवश्चैव वृक्षाश्च प्रजानां हितकारिणः}
{तादृशान्देवपक्षस्थानिति विद्धि शुभानने}


\twolineshloka
{शुभाशुभप्रयं लोके सर्वं स्थावरजङ्गमम्}
{दैवं शुभमिति प्राहुरासुरं चाशुभं प्रिये}


\chapter{अध्यायः २४२}
\threelineshloka
{भगवन्मानुषाः केचित्कालधर्ममुपस्थिताः}
{प्राणमोक्षं कथं कृत्वा परत्रि हितमाप्नुयुः ॥महेश्वर उवाच}
{}


\twolineshloka
{हन्त ते कथयिष्यामि शृणु देवि समाहिता}
{द्विविधं मरणं लोके स्वभावाद्यत्नतस्तथा}


\twolineshloka
{तयोः स्वभावं नापायं यत्नतः करणोद्भवम्}
{एतयोरुभयोर्देवि विधानं शृणु शोभने}


\twolineshloka
{कल्याकल्यशरीरस्य यत्नजं द्विविधं स्मृतम्}
{यत्नजं नाम मरणमात्मत्यागो मुमूर्षया}


\threelineshloka
{तत्राकल्यशरीरस्य जरा व्याधिश्च कारणम्}
{महाप्रस्थानगमनं तथा प्रायोपवेशनम्}
{जलावगाहनं चैव अग्निचित्यां प्रवेशनम्}


\twolineshloka
{एवं चतुर्विधः प्रोक्त आत्मत्यागो मुमूर्षताम्}
{एतेषां क्रमयोगेन विधानं शृणु शोभने}


\twolineshloka
{स्वधर्मयुक्तं गार्हस्थ्यं चिरमूढ्वा विधानतः}
{तत्रानृण्यं च सम्प्राप्य वृद्धो वा व्याधितोऽपि वा}


\twolineshloka
{दर्शयित्वा स्वदौर्बल्यं सर्वानेवानुमान्य च}
{सर्वं विहाय बन्धूंश्च क्रमाणां भरणं तथा}


\twolineshloka
{दानानि विधिवत्कृत्वा धर्मिकार्यर्थमात्मनः}
{अनुज्ञाप्य जनं सर्वं वाचा मधुरया ब्रुवन्}


\twolineshloka
{अहतं वस्त्रमाच्छाद्य बद्ध्वा तत्कुशरज्जुना}
{उपस्पृश्च प्रतिज्ञाय व्यवसायपुरसरम्}


\twolineshloka
{परित्यज्य ततो ग्राम्यं धर्मं कुर्याद्यथेप्सितम्}
{महाप्रस्तानमिच्छेच्चेत्प्रतिष्ठेतोत्तरां दिशम्}


\threelineshloka
{भूत्वा तावन्निराहारो यावत्प्राणविमोक्षणम्}
{चेष्टाहानौ शयित्वाऽपि तन्मनाः प्राणमुत्सृजेत्}
{एवं पुण्यकृतां लोकानमलान्प्रतिपद्यते}


\twolineshloka
{प्रायोपवेशनं चेच्छेत्तेनैव विधिना नरः}
{देशे पुण्यतमे श्रेष्ठे निराहारस्तु संविशेत्}


\threelineshloka
{अप्राणं तु शुचिर्भूत्वा कुर्वन्दानं स्वशक्तितः}
{पुण्यं परित्यजेत्प्राणानेष धर्मः सनातनः}
{एवं कलेवरं त्यक्त्वा स्वर्गलोके महीयते}


\twolineshloka
{अग्निप्रवेशनं चेच्छेत्तेनैव विधिना शुभे}
{कृत्वा काष्ठमयं चित्यं पुण्यक्षेत्रे नदीषु वा}


\threelineshloka
{दैवतेभ्यो नमस्कृत्वा कृत्वा चापि प्रदक्षिणम्}
{भूत्वा शुचिर्व्यवसितः प्रविशेदग्निसंस्तरम्}
{सोपि लोकान्यथान्यायं प्राप्नुयात्पुण्यकर्मणाम्}


\twolineshloka
{जलावगाहनं चेच्छेत्तेनैव विधिना शुभे}
{ख्याते पुण्यतमे तीर्थे निमज्जेत्सुकृतं स्मरन्}


\twolineshloka
{सोपि पुण्यतमाँल्लोकान्निःसङ्गात्प्रतिपद्यते}
{ततः कल्यशरीरस्य संत्यागं शृणु तत्वतः}


\twolineshloka
{रक्षार्थं क्षत्रियः श्रेष्ठः प्रजापालनकारणात्}
{योधानां भर्तृपिण्डार्थं गुर्वर्थं ब्रह्मचारिणाम्}


\twolineshloka
{गोब्राह्मणार्थं सर्वेषां प्राणत्यागो विधीयते}
{स्वराज्यरक्षणार्तं वा कुजनैः पीडिताः प्रजाः}


\threelineshloka
{मोक्तुकामस्त्यजेत्प्राणान्युद्धमार्गे यथाविधि}
{सुसन्नद्धो व्यवसितः सम्प्रविश्यापराङ्मुखः}
{एवं राजा मृतः सद्यः स्वर्गलोके महीयते}


\twolineshloka
{तादृशी सुगतिर्नास्ति क्षत्रियस्य विशेषतः}
{भृत्यो वा भर्तृपिण्डार्थं भर्तृकर्मण्युपस्थिते}


\twolineshloka
{कुर्वंस्तत्र तु साहाय्यमात्मप्राणानपेक्षया}
{स्वाम्यर्थं संत्यजेत्प्राणान्पुण्याँल्लोकान्स गच्छति}


\twolineshloka
{स्पृहणीयः सुरगणैस्तत्र नास्ति विचारणा}
{एवं गोब्राह्मणार्थं वा दीनार्थं वा त्यजेत्तनुम्}


\twolineshloka
{सोपि पुण्यमवाप्नोति आनृशंस्यव्यपेक्षया}
{इत्येते जीवितत्यागे मार्गास्ते समुदाहृताः}


\twolineshloka
{कामक्रोधाद्भयाद्वाऽपि यदि चेत्संत्यजेत्तनुम्}
{सोऽनन्तं नरकं याति आत्महन्तृत्वकारणात्}


\twolineshloka
{स्वभावं मरणं नाम न तु चात्मेच्छया भवेत्}
{यथा मृतानां यत्कार्यं तन्मे शृणु यथाविधि}


\twolineshloka
{तत्रापि मरणं त्यागो मूढत्यागाद्विशिष्यते}
{भूमौ संवेशयेद्देहं नरस्य विनशिष्यतः}


\twolineshloka
{निर्जीवं वृणुयात्सद्यो वाससा तु कलेवरम्}
{माल्यगन्धैरलङ्कृत्य सुवर्णेन च भामिनि}


\twolineshloka
{श्मशाने दक्षिणे देशे चिताग्नौ प्रदहेन्मृतम्}
{अथवा निक्षिपेद्भूमौ शरीरं जीववर्जितम्}


\twolineshloka
{दिवा च शुक्लपक्षश्च उत्तरायणमेव च}
{मुमूर्षूणां प्रशस्तानि विपरीतं तु गर्हितम्}


\threelineshloka
{औदकं चाष्टकाश्राद्धं बहुभिर्बहुभिः कृतम्}
{आप्यायनं मृतानां तत्परलोके भवेच्छुभम्}
{एतत्सर्वं मया प्रोक्तं मानुषाणां हितं वचः}


\chapter{अध्यायः २४३}
\twolineshloka
{देवदेव नमस्तेऽस्तु कालसूदन शङ्कर}
{लोकेषु विविधा धर्मास्त्वत्प्रसादान्मया श्रुताः}


\twolineshloka
{विशिष्टं सर्वधर्मेभ्यः शाश्वतं ध्रुवमव्ययम्}
{श्रुतुमिच्छाम्यहं सर्वमत्र मुह्यति मे मनः}


\twolineshloka
{केचिन्मोक्षं प्रशंसन्ति केचिद्यज्ञफलं द्विजाः}
{वानप्रस्थं पुनः केचिद्गार्हस्थ्यं केचिदाश्रमम्}


\twolineshloka
{राजधर्माश्रयं केचित्केचित्स्वाध्यायमेव च}
{ब्रह्मचर्याश्रमं केचित्केचिद्वाक्संयमाश्रयम्}


\twolineshloka
{मातरं पितरं केचित्सेवमाना दिवं गताः}
{अहिंसया परः स्वर्गे सत्येन च महीयते}


\twolineshloka
{आहवेऽभिमुखाः केचिन्निहतास्त्रिदिवं गताः}
{केचिदुञ्छवृत्ते सिद्धाः स्वर्गमार्गं समाश्रिताः}


\twolineshloka
{आर्जवेनापरे युक्ता महतां पूजते रताः}
{ऋजवो नाकपृष्ठे तु शुद्धात्मानः प्रतिष्ठिताः}


\twolineshloka
{एवं बहुविधैर्लोके धर्मद्वारैः सुसंवृतैः}
{ममापि मतिराविद्धा मेघलेखेव वायुना}


\threelineshloka
{एतस्मिन्संशयस्थाने संशयच्छेदकारि यत्}
{वचनं ब्रूहि देवेश निश्चयज्ञानसंज्ञितम् ॥नारद उवाच}
{}


\threelineshloka
{एवं पृष्टः स्वया देव्या महादेवः पिनाकधृक्}
{प्रोवाच मधुरं वाक्यं सूक्ष्ममध्यात्मसंश्रितम् ॥महेश्वर उवाच}
{}


\twolineshloka
{न्यायतस्त्वं महाभागे श्रोतुकामाऽसि निश्चयम्}
{एतदेव विशिष्टं ते यत्त्वं पृच्छसि मां प्रिये}


\twolineshloka
{सर्वत्र विहितो धर्मः स्वर्गलोकफलाश्रितः}
{बहुद्वारस्य धर्मस्य नेहास्ति विफलाः क्रियाः}


\twolineshloka
{यस्मिन्यस्मिंश्च विषये योयो याति विनिश्चयम्}
{तं तमेवाभिजानाति नान्यं धर्मं शुचिस्मिते}


\twolineshloka
{शृणु देवि समासेन मोक्षद्वारसमनुत्तमम्}
{एतद्धि सर्वधर्माणां विशिष्टं शुभमव्ययम्}


\twolineshloka
{नास्ति मोक्षात्परं देवि मोक्ष एव परा गतिः}
{सुखमात्यन्तिकं श्रेष्ठमनिवृत्तं च तद्विदुः}


\twolineshloka
{नात्र देवि जरा मृत्युः शोको वा दुःखमेव वा}
{अनुत्तममचिन्त्यं च तद्देवि परमं सुखम्}


\twolineshloka
{ज्ञानानामुत्तमं ज्ञानं मोक्षज्ञानं विदुर्बुधाः}
{ऋषिभिर्देवसङ्घैश्च प्रोच्यते परमं पदम्}


\twolineshloka
{नित्यमक्षरमक्षोभ्यमजेयं शाश्वतं शिवम्}
{विशन्ति तत्पदं प्राज्ञाः स्पृहणीयं सुरोत्तमैः}


\twolineshloka
{दुःखादिश्च दुरन्तश्च संसारोयं प्रकीर्तितः}
{शोकव्याधिजरादोषैर्मरणेन च संयुतः}


\twolineshloka
{यथा ज्योतिर्गणा व्योम्नि विवर्तन्ते पुनःपुनः}
{तस्य मोक्षस्य मार्गोऽयं श्रुयतां शुभलक्षणे}


\twolineshloka
{ब्रह्मादिस्थावरान्तश्च संसारो यः प्रकीर्तितः}
{संसारे प्राणिनः सर्वे निवर्तन्ते यथा पुनः}


\twolineshloka
{तत्र संसारचक्रस्य मोक्षो ज्ञानेन दृश्यते}
{अध्यात्मतत्वविज्ञानं ज्ञानमित्यभिधीयते}


\twolineshloka
{ज्ञानस्य ग्रहणोपायमाचारं ज्ञानिनस्तथा}
{यथावत्सम्प्रवक्ष्यामि तत्त्वमेकमनाः शृणु}


\twolineshloka
{ब्राह्मणः क्षत्रियो वाऽपि भूत्वा पूर्वं गृहे स्थितः}
{आनृण्यं सर्वतः प्राप्य ततस्तान्संत्यजेद्गृहान्}


% Check verse!
ततः संत्यज्य गार्हस्थ्यं निश्चितो वनमाश्रयेत्
\twolineshloka
{वने गुरुं समाज्ञाय दीक्षितो विधिपूर्वकम्}
{दीक्षां प्राप्य यथान्यायं स्ववृत्तं परिपालयेत्}


\twolineshloka
{गृह्णीयादप्युपाध्यायान्मोक्षज्ञानमनिन्दितः}
{द्विविधं च पुनर्मोक्षं साङ्ख्ययोगमिति स्मृतिः}


\threelineshloka
{पञ्चविंशतिविज्ञानं साङ्ख्यमित्यभिधीयते}
{ऐश्वर्यं देवसारूप्यं योगशास्त्रस्य निर्णयः}
{तयोरन्यतरं ज्ञानं शृणुयाच्छिष्यतां गतः}


\threelineshloka
{नाकालो नाप्यकाषायी नाप्यसंवत्सरोषितः}
{नासाङ्ख्ययोगो नाश्राद्धं गुरुणा स्नेहपूर्वकम्}
{समः शीतोष्णहर्षादीन्विषहेत स वै मुनिः}


\twolineshloka
{अमृष्यः क्षुत्पिपासाभ्यामुचितेभ्यो निवर्तयेत्}
{त्यजेत्सङ्कल्पजान्ग्रन्थीन्सदा ध्यानपरो भवेत्}


\twolineshloka
{कुण्डिकाचमसं शिक्यं छत्रं यष्टिमुपानहौ}
{चेलमित्येव नैतेषु स्थापयेत्साम्यमात्मनः}


\twolineshloka
{गुरोः पूर्वं समुत्तिष्ठेज्जघन्यं तस्य संविशेत्}
{नैवाविज्ञाप्य भर्तारमावश्यकमपि व्रजेत्}


\twolineshloka
{द्विरह्नि स्नानशाटेन संध्ययोरभिषेचनम्}
{एककालाशनं चास् विहितं यतिभिः पुरा}


\twolineshloka
{भैक्षं सर्वत्र गृह्णीयाच्चिन्तयेत्सततं निशि}
{कारणे चापि सम्प्राप्ते न ज्ञाप्येत कदाचन}


\twolineshloka
{ब्रह्मिचर्यं वने वासं शौचमिन्द्रियसंयमः}
{दया च सर्वभूतेषु तस्य धर्मः सनातनः}


\twolineshloka
{विमुक्तः सर्वपापेभ्यो लघ्वाहारो जितेन्द्रियः}
{आत्मयुक्तः परां बुद्धिं लभते पापनाशिनीम्}


\twolineshloka
{यदा भावं न कुरुते सर्वभूतेषु पापकम्}
{कर्मणा मनसा वाचा ब्र्हम सम्पद्यते तदा}


\twolineshloka
{अनिष्ठुरोऽनहङ्कारो निर्द्वन्द्वो वीतमत्सरः}
{वीतशोकभयाबाधं पदं प्राप्नोत्यनुत्तमम्}


\twolineshloka
{तुल्यनिन्दास्तुतिर्मौनी समलोष्टाश्मकाञ्चनः}
{समः शत्रौ च मित्रे च निर्वाणमधिगच्छति}


\twolineshloka
{एवं युक्तसमाचारस्तत्परोऽध्यात्मचिन्तकः}
{ज्ञानाभ्यासेन तेनैव प्राप्नोति परमां गतिम्}


\chapter{अध्यायः २४४}
\threelineshloka
{अनुद्विग्नमतेर्जन्तोरस्मिन्संसारमण्डले}
{शोकव्याधिजरादुःखैर्निर्वाणं नोपपद्यते ॥तस्मादुद्वेगजननं मनोऽवस्थानपं तथा}
{ज्ञानं ते सम्प्रवक्ष्यामि तन्मूलममृतं हि वै}


\twolineshloka
{शोकस्थानसहस्राणि भयस्थानशतानि च}
{दिवसेदिवसे मूढमाविशन्ति न पण्डितम्}


\twolineshloka
{नष्टे धने वा दारे वा पुत्रे पितरि वा मृते}
{अहो दुःखमिति ध्यायञ्शोकस्य पदमाव्रजेत्}


\twolineshloka
{द्रव्येषु समतीतेषु ये शुभास्तान्न चिन्तयेत्}
{ताननाद्रियमाणस्य शोकबन्धः प्रणश्यति}


\twolineshloka
{सम्प्रयोगादनिष्टस्य विप्रयोगात्प्रियस्य च}
{मानुषा मानसैर्दुःखैः संयुज्यन्तेऽल्पबुद्धयः}


\twolineshloka
{मृतं वा यदि वा नष्टं योऽतीतमनुशोचति}
{सन्तापेन च युज्येत तच्चास्य न निवर्तते}


\twolineshloka
{उत्पन्नमिह मानुष्ये गर्भप्रभृति मानवम्}
{विविधान्युपवर्तन्ते दुःखानि च सुखानि च}


\twolineshloka
{तयोरेकतरो मार्गो यद्येनमभिसंनमेत्}
{सुखं प्राप्य न संहृष्येन्न दुःखं प्राप्य संज्वरेत्}


\twolineshloka
{दोषदर्शी भवेत्तत्र यत्र स्नेहः प्रवर्तते}
{अनिष्टेनान्वितं पश्येद्यथा क्षिप्रं विरज्यते}


\twolineshloka
{यथा काष्ठं च काष्ठं च समेयातां महोदधौ}
{समेत्य च व्यपेयातां तद्वज्ज्ञातिसमागमः}


\twolineshloka
{अदर्शनादापतिताः पुनश्चादर्शनं गताः}
{स्नेहस्तत्र न कर्तव्यो विप्रयोगो हि तैर्ध्रुवम्}


\twolineshloka
{कुटुम्बपुत्रदारांश्च शरीरं धनसञ्चयम्}
{ऐश्वर्यं स्वस्तिता चेति न मुह्येत्तत्र पण्डितः}


\twolineshloka
{सुखमेकान्ततो नास्ति शक्रस्यापि त्रिविष्टपे}
{तत्रापि सुमहद्दुःखं न नित्यं लभते सुखम्}


\twolineshloka
{सुखस्यान्तरं दुःखं दुःखस्यानन्तरं सुखम्}
{क्षया निचयाः सर्वे पतनान्ताः समुच्छ्रयाः}


\threelineshloka
{संयोगा *********** मरणान्तं च जीवितम्}
{उच्छ्रयांश्च निपाताश्च दृष्ट्या प्रत्यक्षतस्त्रयम्}
{अनित्यमसुखं चेति व्यवस्येत्सर्वमेव च}


\twolineshloka
{अर्थानामार्जने दुःखमार्जितानां तु रक्षणे}
{नाशे दुःखं व्यये दुःखं धिगर्थं दुःखभाजनम्}


\twolineshloka
{अर्थवन्तं नरं नित्यं पञ्चाभिघ्नन्ति शत्रवः}
{राजा चोरश्च दायादा भूतानि क्षय एव च}


\twolineshloka
{अर्थमेव ह्यनर्थस्य मूलमित्यवधारय}
{न ह्यनर्थाः प्रबाधन्ते नरमर्तविवर्जितम्}


\twolineshloka
{अर्थप्राप्तिर्महद्दुःखमाकिञ्चिन्यं परं सुखम्}
{उपद्रवेषु चार्थानां दुःखं हि नियतं भवेत्}


\twolineshloka
{धनलोभेन तृष्णाया न तृप्तिरुपलभ्यते}
{लब्धाश्रयो विवर्धेत समिद्ध इव पावकः}


\twolineshloka
{जित्वाऽपि पृथिवीं कृत्स्नां चतुःसागरमेखलाम्}
{सागराणां पुनः पारं जेतुमिच्छत्यसंशयम्}


\twolineshloka
{अलं परिग्रहेणेह दोषवान्हि परिग्रहः}
{कोशकारः क्रिमिर्देवि बध्यते हि परिग्रहात्}


\twolineshloka
{एकोऽपि पृथिवीं कृत्स्नामेकच्छत्रां प्रशास्ति च}
{एकस्मिन्नेव राष्ट्रे तु स चापि निवसेन्नृपः}


\twolineshloka
{तस्मिन्राष्ट्रेऽपि नगरमेकमेवाधितिष्ठति}
{नगरेऽपि गृहं चैकं भवेत्तस्य निवेशनम्}


\twolineshloka
{एक एव प्रतिष्ठः स्यादावासस्तद्गृहेऽपि च}
{आवासे शयनं चैकं निशि यत्र प्रलीयते}


\twolineshloka
{शयनस्यार्धमेवास्य स्त्रियाश्चार्धं विधीयते}
{तदनेन प्रसङ्गेन स्वल्पेनैव हि युज्यते}


\twolineshloka
{सर्वं ममेति सम्मूढो बलं पश्यति बालिशः}
{एवं सर्वोपयोगेषु स्वल्पमस्य प्रयोजनम्}


\twolineshloka
{तण्डुलप्रस्थमात्रेण यात्रा स्यात्सर्वदेहिनाम्}
{ततो भूयस्तरो योगो दुःखाय तपनाय च}


\twolineshloka
{नास्ति तृष्णासमं दुःखं नास्ति त्यागसमं सुखम्}
{सर्वान्कामान्परित्यज्य ब्रह्मभूयाय कल्पते}


\twolineshloka
{या दुस्त्यजा दुर्मतिभिर्या न जीर्यति जीर्यतः}
{योऽसौ प्राणान्तिको रोगस्तां तृष्णां त्यजतः सुखम्}


\twolineshloka
{न जातु कामः कामानामुपभोगेन शाम्यति}
{हविषा कुष्णवर्त्मेव भूय एवाभिवर्धते}


\threelineshloka
{अलाभेनैव कामानां शोकं त्यजति पण्डितः}
{आयासविटपस्तीव्रः कामाग्निः कर्षणारणिः}
{इन्द्रियार्थैश्च सम्मोह्य दहत्यकुशलं जनम्}


\twolineshloka
{यत्पृथिव्यां व्रीहियवं हिरण्यं पशवः स्त्रियः}
{नालमेकस्य पर्याप्तमिति पश्यन्न मुह्यति}


\twolineshloka
{यच्च कामसुखं लोके यच्च दिव्यं महत्सुखम्}
{तृष्णाक्षयसुखस्यैते नार्हतः षोडशीं कलाम्}


\twolineshloka
{इन्द्रियाणीन्द्रियार्थेषु नैव धीरो नियोजयेत्}
{मनःषष्ठानि संयम्य नित्यमात्मनि योजयेत}


\twolineshloka
{इन्द्रियाणां विसर्गेण दोषमृच्छत्यसंशयम्}
{संनियम्य नु तान्येव ततः सिद्धिमवाप्नुयात्}


\twolineshloka
{षण्णामात्मनि युक्तानामैश्वर्यं योऽधिगच्छति}
{न च पापैर्न चानर्थैः संयुज्येत विचक्षणः}


\twolineshloka
{अप्रमत्तः सदा रक्षेदिन्द्रियाणि विचक्षणः}
{अरक्षितेषु तेष्वाशु नरो नरकमेति हि}


\twolineshloka
{हृदि काममयश्चित्रो मोहसञ्चयसम्भवः}
{अज्ञानरूढमूलस्तु विवित्सापरिषेचनः}


\twolineshloka
{रोषलोभमहास्कन्धः पुरा दुष्कृतसारवान्}
{आयासविटपस्तीव्रशोकपुष्पो भयाङ्कुरः}


\twolineshloka
{नानासङ्कल्पपत्राढ्यः प्रमादात्परिवर्धितः}
{महतीभिः पिपासाभिः समन्तात्परिवेष्टितः}


\twolineshloka
{संरोहत्यकृतप्रज्ञे पादपः कामसम्भवः}
{नैव रोहति तत्वज्ञे रूढो वा छिद्यते पुनः}


\twolineshloka
{कृच्छ्रोपायेष्वनित्येषु निःसारेषु फलेषु च}
{दुःखादिषु दुरन्तेषु कामयोगेषु का रतिः}


\twolineshloka
{इन्द्रियेषु च जीर्यत्सु च्छिद्यमाने तताऽऽयुषि}
{पुरस्ताच्च स्थिते मृत्यौ किं सुखं पश्यतासुखे}


\twolineshloka
{व्याधिभिः पीड्यमानस्य नित्यं शारीरमानसैः}
{नरस्याकृतकृत्यस्य किं सुखं मरणे सति}


\twolineshloka
{सञ्चिन्वानं तमेवार्थं कामानामवितृप्तकम्}
{व्याघ्रः पशुमिवारण्ये मृत्युरादाय गच्छति}


\threelineshloka
{जन्ममृत्युजरादुःखैः सततं समभिद्रुतः}
{संसारे पच्यमानस्तु पापान्नोद्विजते जनः ॥उमोवाच}
{}


\twolineshloka
{केनोपायेन मर्त्यानां निवर्त्येते जरान्तकौ}
{यद्यस्ति भगवन्मह्यमेतदाचक्ष्व माचिरम्}


\threelineshloka
{तपसा वा सुमहता कर्मणा वा श्रुतेन वा}
{रसायनप्रयोगैर्वा केनात्येति जरान्तकौ ॥महेश्वर उवाच}
{}


\twolineshloka
{नैतदस्ति महाभागे जरामृत्युनिवर्तनम्}
{सर्वलोकेषु जानीहि मोक्षादन्यत्र भामिनि}


\twolineshloka
{न धनेन न राज्येन नोग्रेण तपसाऽपि वा}
{मरणं नातितरते विना मुक्त्या शरीरिणः}


\twolineshloka
{अश्वमेधसहस्राणि वाजपेयशतानि च}
{न तरन्ति जरामृत्यू निर्वाणाधिगमाद्विना}


\twolineshloka
{ऐस्वर्यं धनधान्यं च विद्यालाभस्तपस्तथा}
{रसायनप्रयोगाद्वै न तरन्ति जरान्तकौ}


\twolineshloka
{दानयज्ञतपःशीलरसायनविदोऽपि वा}
{स्वाध्यायनिरता वाऽपि न तरन्ति जरान्तकौ}


\twolineshloka
{देवदानवगन्धर्वकिन्नरोरगराक्षसान्}
{स्ववशे कुरुते कालो न कालस्यास्त्यगोचरः}


\twolineshloka
{न ह्यहानि निवर्तन्ते न मासा न पुनः क्षपाः}
{स्रेयं प्रपद्यते ध्यानमजस्रं ध्रुवमव्ययम्}


\twolineshloka
{स्रवन्ति न निवर्तन्ते स्रोतांसि सरितामिव}
{आयुरादाय मर्त्यानामहोरात्रेषु सन्ततम्}


\twolineshloka
{जीवितं सर्वभूतानामक्षयः क्षपयन्नसौ}
{आदित्यो ह्यस्तमभ्येति पुनः पुनरुदेति च}


\twolineshloka
{यस्यां रात्र्यां व्यतीतायामायुरल्पतरं भवेत्}
{गाधोदके मत्स्य इव किन्नु तस्य कुमारता}


\twolineshloka
{मरणं हि शरीरस्य नियतं ध्रुवमेव च}
{तिष्ठन्नपि क्षणं सर्वः कालस्यैति वशं पुनः}


\twolineshloka
{न म्रियेरन्न जीर्येरन्यदि स्युः सर्वदेहिनः}
{न चानिष्टं प्रवर्तेत शोको वा प्राणिनं क्वचित्}


\twolineshloka
{अप्रमत्तः प्रमत्तेषु कालो भूतेषु तिष्ठति}
{अप्रमत्तस्य कालस्य क्षयं प्राप्तो न मुच्यते}


\twolineshloka
{श्वःकार्यमद्य कुर्वीत पूर्वाह्णे चापराह्णिकम्}
{कोपि तद्वेद यत्रासौ मृत्युना नाभिवीक्षितः}


\twolineshloka
{वर्षास्विदं करिष्यामि इदं ग्रीष्मवसन्तयोः}
{इति बालश्चिन्तयति अन्तरायं न बुध्यति}


\twolineshloka
{इदं मे स्यादिदं मे स्यादित्येवं मनसा नराः}
{अनवाप्तेषु कामेषु ह्रियन्ते मरणं प्रति}


\twolineshloka
{कालपाशेन बद्धानामहन्यहनि जीर्यताम्}
{का श्रद्धा प्राणिनां मार्गे विषमे भ्रमतां सदा}


\twolineshloka
{युवैव धर्मशीलः स्यादनिमित्तं हि जीवितम्}
{फलानामिव पक्वानां सदा हि पतनाद्भयम्}


\threelineshloka
{मर्त्यस्य किं धनैर्दारैः पुत्रैर्भोगैः प्रियैरपि}
{एकाह्ना सर्वमुत्सृज्य मृत्योस्तु वशमन्वियात्}
{}


\twolineshloka
{जायामानांश्च सम्प्रेक्ष्य म्रियमाणांस्तथैव च}
{न संवेगोस्ति चेत्पुंसः काष्ठलोसमो हि सः}


\twolineshloka
{विनाशिनो ह्यध्रुवजीवितस्यकिं बन्धुभिर्मित्रपरिग्रहैश्च}
{विहाय यद्गच्छति सर्वमेवंक्षणेन गत्वा न निवर्तते च}


\twolineshloka
{एवं चिन्तयतो नित्यं सर्वार्थानामनित्यताम्}
{उद्वेगो जायते शीघ्रं निर्वाणस्य पुरस्सरः}


\twolineshloka
{तेनोद्वेगेन चाप्यस्य विमर्शो जायते पुनः}
{विमर्शो नाम वैराग्यं सर्वद्रव्येषु जायते}


\twolineshloka
{वैराग्येण परां शान्तिं लभन्ते मानवाः शुभे}
{मोक्षस्योपनिषद्दिव्यं वैराग्यमिति निश्चितम्}


\twolineshloka
{एतत्ते कथितं देवि वैराग्योत्पादनं वचः}
{एवं सञ्चिन्त्य सञ्चिन्त्य मुच्यन्ते हि मुमुक्षवः}


\chapter{अध्यायः २४५}
\twolineshloka
{साङ्ख्यज्ञानं प्रवक्ष्यामि यथावत्ते शुचिस्मिते}
{यज्ज्ञात्वा न पुनर्मर्त्यः संसारेषु प्रवर्तते}


\threelineshloka
{ज्ञानेनैव विमुक्तास्ते साङ्ख्याः संन्यासकोविदाः}
{शरीरं तु तपो घोरं साङ्ख्याः प्राहुर्निरर्थकम्}
{}


\twolineshloka
{पञ्चविंशतिकं ज्ञानं तेषां ज्ञानमिति स्मृतम्}
{मूलप्रकृतिरव्यक्तमव्यक्ताज्जायते महान्}


\twolineshloka
{महतोऽभूदहङ्कारस्तस्मात्तन्मात्रपञ्चकम्}
{इन्द्रियाणि दशैकं च तन्मात्रेभ्यो भवन्त्युत}


\threelineshloka
{तेभ्यो भूतानि पञ्चास्य शरीरं यैः प्रवर्तते}
{इति क्षेत्रस्य संक्षेपं चतुर्विंशतिरिष्यते}
{पञ्चविंशतित्याहुः पुरुषेणेह सङ्ख्यया}


\twolineshloka
{सत्वं रजस्तमश्चेति गुणाः प्रकृतिसम्भवाः}
{तैः सृजत्यखिलं लोकं प्रकृतिः स्वात्मकैर्गुणैः}


\twolineshloka
{इच्छा द्वेषः सुखं दुःखं संघातश्चेतना धृतिः}
{विकाराः प्रकृतेश्चैते वेदितव्या मनीषिभिः}


\twolineshloka
{लक्षणं चापि सर्वेषां विकल्पं चादितः पृथक्}
{विस्तरेणैव वक्ष्यामि तस्य व्याख्यामहं शृणु}


\twolineshloka
{नित्यमेकमणु व्यापि क्रियाहीनमहेतुकम्}
{अग्राह्यमिन्द्रियैः सर्वैरेतदव्यक्तलक्षणम्}


\twolineshloka
{अव्यक्तं प्रकृतिर्मूलं प्रधानं योनिरव्ययम्}
{अव्यक्तस्यैव नामानि शब्दैः पर्यायवाचकैः}


\twolineshloka
{तत्सूक्ष्मत्वादनिर्देश्यं तत्सदित्यभिधीयते}
{तन्मूलं च जगत्सर्वं तन्मूला सृष्टिरिष्यते}


% Check verse!
सत्वादयः प्रकृतिजा गुणास्तान्प्रब्रवीम्यहम्
\threelineshloka
{सुखं तुष्टिः प्रकाशश्च त्रयस्ते सात्विका गुणाः}
{रागद्वेषौ सुखं दुःखं स्तम्भश्च रजसो गुणाः}
{अप्रकाशो भयं मोहस्तन्द्री च तमसो गुणाः}


\twolineshloka
{श्रद्धा प्रहर्षो विज्ञानमसंमोहो दया धृतिः}
{सत्वे प्रवृत्ते वर्धन्ते विपरीते विपर्ययः}


\twolineshloka
{कामक्रोधौ मनस्तापो लोभो मोहस्तथामृषा}
{प्रवृद्धे परिवर्धन्ते रजस्येतानि सर्वशः}


\twolineshloka
{विषादः संशयो मोहस्तन्द्री निद्रा भयं तथा}
{तमस्येतानि वर्धन्ते प्रवृद्धे हेत्वहेतुकम्}


\twolineshloka
{एवमन्योन्यमेतानि वर्धन्ते च पुनःपुनः}
{हीयन्ते च तथा नित्यमभिभूतानि भूरिशः}


\twolineshloka
{तत्र यत्प्रीतिसंयुक्तं कायेन मनसाऽपि वा}
{वर्तते सात्विको भाव इत्युपेक्षेत तत्तथा}


\twolineshloka
{यदा सन्तापसंयुक्तं चित्तक्षोभकरं भवेत्}
{वर्तते रज इत्येव तदा तदभिचिन्तयेत्}


\twolineshloka
{यदा सम्मोहसंयुक्तं यद्विषादकरं भवेत्}
{अप्रतार्क्यमविज्ञेयं तमस्तदुपधारयेत्}


\twolineshloka
{समासात्सात्विको धर्मः समासाद्राजसं धनम्}
{समासात्तामसः कामस्त्रिवर्गे त्रिगुणाः क्रमात्}


\twolineshloka
{ब्रह्मादिदेवसृष्टिर्या सात्विकीति प्रकीर्त्यते}
{राजसी मानवी सृष्टिस्तिर्यग्योनिस्तु तामिसी}


\twolineshloka
{ऊर्ध्वं गच्छन्ति सत्वस्था मध्ये तिष्ठन्ति राजसाः}
{जघन्यगुणवृतच्तिस्था अधो गच्छन्ति तामसाः}


\twolineshloka
{देवमानुषतिर्यक्षु यद्भूतं सचराचरम्}
{आदिप्रभृति संयुक्तं व्याप्तमेभिस्त्रिभिर्गुणैः}


\twolineshloka
{अतः परं प्रवक्ष्यामि महदादीनि लिङ्गतः}
{विज्ञानं च विवेकश्च महतो लक्षणं भवेत्}


\twolineshloka
{महान्बुद्धिर्मतिः प्रज्ञा नामानि महतो विदुः}
{अहङ्कारः स विज्ञेयो लक्षणेन समासतः}


\twolineshloka
{अहङ्कारेण भूतानां सर्गो नानाविधो भवेत्}
{अहङ्कारनिवृत्तिर्हि निर्वाणायोपपद्यते}


\twolineshloka
{खं वायुरग्निः सलिलं पृथिवी चेति पञ्चमी}
{महाभूतानि भूतानां सर्वेषां प्रभवाप्ययौ}


\twolineshloka
{शब्दः श्रोत्रं तथा खानि त्रयमाकाशसम्भवम्}
{स्पर्शवत्प्राणिनां चेष्टा पवनस्य गुणाः स्मृताः}


\twolineshloka
{रूपं पाकोक्षिणी ज्योतिश्चत्वारस्तेजसो गुणाः}
{रसः स्नेहस्तथा जिह्वा शैत्यं च जलजा गुणाः}


\twolineshloka
{गन्धो घ्राणं शरीरं च पृथिव्यास्ते गुणास्त्रयः}
{इति सर्वगुणा देवि विख्याताः पाञ्चभौतिकाः}


\twolineshloka
{गुणान्पूर्वस्यपूर्वस्य प्राप्नुवन्त्युत्तराणि तु}
{तस्मान्नैकगुणाश्चेह दृश्यन्ते बूतसृष्टयः}


\twolineshloka
{उपलभ्याप्सु ये गन्धं केचिद्ब्रूयुरनैपुणाः}
{अपां गन्धगुणं प्राज्ञा नेच्छन्ति कमलेक्षणे}


\threelineshloka
{तद्गन्धत्वमपां नास्ति पृथिव्या एव तद्गुणः}
{भूमिर्गन्धे रसे स्नेहो ज्योतिश्चक्षुषि संस्थितम्}
{प्राणापानाश्रयोः वायुः खेष्वाकाशः शरीरिणां}


\twolineshloka
{केशास्थिनखदन्तत्वक्पाणिपादशिरांसि च}
{पृष्ठोदरकटिग्रीवाः सर्वं भूम्यात्मकं स्मृतम्}


\twolineshloka
{यत्किञ्चिदपि कायेऽस्मिन्धातुदोषमलाश्रितम्}
{तत्सर्वं भौतिकं विद्धि देहैरेवास्य स्वामिकम्}


\twolineshloka
{बुद्धीन्द्रियाणि कर्णत्वक्चक्षुर्जिह्वाऽथ नासिका}
{कर्मेन्द्रियाणि वाक्पाणिपादौ मेढ्रं गुदस्तथा}


\twolineshloka
{शब्दः स्पर्शश्च रूपं च रसो गन्धश्च पञ्चमः}
{बुद्धीन्द्रियार्थाञ्जानीयाद्भूतेभ्यस्त्वभिनिःसृतान्}


\twolineshloka
{वाक्यं क्रिया गतिः प्रीतिरुत्सर्गश्चेति पञ्चधा}
{कर्मेन्द्रियार्थाञ्जानीयात्ते च भूतोद्भवा मताः}


\twolineshloka
{इन्द्रियाणां तु सर्वेषामीश्वरं मन उच्यते}
{प्रार्थनालक्षणं तच्च इन्द्रियं तु मनः स्मृतम्}


\twolineshloka
{नियुङ्क्ते च सदा तानि भूतानि मनसा सह}
{नियमे च विसर्गे च मनसः कारणं प्रभुः}


\twolineshloka
{इन्द्रियाणीन्द्रियार्थाश्च स्वभावश्चेतना धृतिः}
{भूताभूतविकारश्च शरीरमिति संस्मृतम्}


\twolineshloka
{शरीराच्च परो देही शरीरं च व्यपाश्रितः}
{शरीरिणः शरीरस्य सोऽन्तरं वेत्ति वै मुनिः}


\threelineshloka
{रसः स्पर्शस्च गन्धश्च रूपं शब्दविवर्जितम्}
{अशरीरं शरीरेषु दिदृक्षेत निरिन्द्रियम्}
{}


\twolineshloka
{अव्यक्तं सर्वदेहेषु मर्त्येष्वमरमाश्रितम्}
{यः पश्येत्परमात्मानं बन्धनैः स विमुच्यते}


\twolineshloka
{नैवायं चक्षुषां ग्राह्यो नापरैरिन्द्रियैरपि}
{मनसैव प्रदीप्तेन महानात्मा प्रदृश्यते}


\twolineshloka
{स हि सर्वेषु भूतेषु स्थावरेषु चरेषु च}
{वसत्येको महावीर्यो नानाभावसमन्वितः}


\twolineshloka
{नैव चोर्ध्वं न तिर्यक्च नाधस्तान्न कदाचन}
{इन्द्रियैरिव बुद्ध्या वा न दृश्येत कदाचन}


\twolineshloka
{नवद्वारं पुरं गत्वा स्थितोऽसौ नियतो वशी}
{ईश्वरः सर्वलोकेषु स्थावरस्य चरस्य च}


\twolineshloka
{तमेवाहुरणुभ्योऽणुं तु महद्भ्यो महत्तरम्}
{बहुधा सर्वभूतानि व्याप्य तिष्ठति शाश्वतम्}


\twolineshloka
{क्षेत्रज्ञमेकतः कृत्वा सर्वं क्षेत्रमथैकतः}
{एवं स विमृशेज्ज्ञानी संयतः सततं हृदि}


\twolineshloka
{पुरुषः प्रकृतिस्थो हि भुङ्क्ते प्रकृतिजान्गुणान्}
{अकर्ता लेपको नित्यो मध्यस्थः सर्वकर्मणाम्}


\twolineshloka
{कार्यकारणकर्तृत्वे हेतुः प्रकृतिरुच्यते}
{पुरुषः सुखदुःखानां भोक्तृत्वे हेतुरुच्यते}


\twolineshloka
{अजय्योऽयमचिन्त्योऽयमव्यक्तोऽयं सनातनः}
{देही तेजोमयो देहि तिष्ठतीत्यपरे विदुः}


\twolineshloka
{ज्ञानमूष्मा च वायुश्च शरीरे जीवसंज्ञकः}
{इत्येते निश्चिता बुद्ध्या तत्रैते बुद्धिचिन्तकाः}


\twolineshloka
{अपरे सर्वलोकांश्च व्याप्य तिष्ठन्तमीश्वरम्}
{ब्रुवते केचिदत्रैव तिलतैलवदास्थितम्}


\twolineshloka
{अपरे नास्तिका मूढा हीनत्वात्स्थूललक्षणैः}
{नास्त्यात्मेति विनिश्चित्याप्रज्ञास्ते निरयालयाः}


\twolineshloka
{एवं नानाविधा नैव विमृशन्ति महेश्वरम् ॥उमोवाच}
{}


\threelineshloka
{भगवन्ब्राह्मणो लोके नित्यमक्षरमव्ययम्}
{अस्त्यात्मा सर्वभूतेषु हेतुस्तत्र सुदुर्गमः ॥महेश्वर उवाच}
{}


\twolineshloka
{ऋषिभिश्चापि देवैश्च व्यक्तमेष न दृश्यते}
{दृष्ट्वा तु तं महात्मानं पुनस्तु न निवर्तते}


\threelineshloka
{तस्मात्तद्दर्शनादेव विन्दते परमां गतिम्}
{इति ते कथितो देवि साङ्ख्यधर्मः सनातनः}
{कपिलादिभिराचार्यैः सेवितः परमर्षिभिः}


\chapter{अध्यायः २४६}
\twolineshloka
{साङ्ख्यज्ञाने नियुक्तानां यथावत्कीर्तितं मया}
{योगधर्मं पुनः कृत्स्नं कीर्तयिष्यामि ते शृणु}


\twolineshloka
{स च योगो द्विधा भिन्नो ब्रह्मिदेवर्षिसम्मतः}
{समानमुभयत्रापि वृत्तं शास्त्रप्रचोदितम्}


\twolineshloka
{च चाष्टगुणमैश्वर्यमधिकृत्य विधीयते}
{सायुज्यं सर्वदेवानां योगधर्मं परं श्रिताः}


\twolineshloka
{ज्ञानं सर्वस्य योगस्य मूलमित्यवधारय}
{व्रतोपवासनियमैस्तत्सर्वं चापि बृंहयेत्}


\twolineshloka
{ऐकात्म्यं बुद्धिमनसोरिन्द्रियाणां च सर्वशः}
{आत्मनो वेदितं प्राज्ञे ज्ञानमेतत्तु योगिनाम्}


\twolineshloka
{अर्चयेद्ब्राह्मणानग्निं देवतायतनानि च}
{वर्जयेदशिवं भावं सर्वसत्त्वमुपाश्रितः}


\threelineshloka
{दानमध्ययनं श्रुद्धा व्रतानि नियमास्तथा}
{सत्यमाहारशुद्धिश्च शौचमिन्द्रियनिग्रहः}
{एतैश्च वर्धते तेजः पापं चाप्यवधूयते}


\twolineshloka
{निर्धूतपापस्तेजस्वी लघ्वाहारो जितेन्द्रियः}
{अमोधो निर्मलो दान्तः पश्चाद्योगं समाचरेत्}


\threelineshloka
{अवरुध्यात्मनः पूर्वं मत्स्यघात इवामिषम्}
{एकान्ते विजने देशे सर्वतः संवृते शुचौ}
{कल्पयेदासनं तत्र स्वास्तीर्णं मृदुभिः कुशैः}


\threelineshloka
{उपविश्यासने तस्मिन्नृजुकायशिरोधरः}
{अव्यग्रः सुखमासीनः स्वाङ्गानि न विकम्पयेत्}
{सम्प्रेक्ष्य नासिकाग्रं स्वं दिशश्चानवलोकयन्}


\twolineshloka
{मनोऽवस्थापनं देवि योगस्योपनिषद्भवेत्}
{तस्मात्सर्वप्रयत्नेन मनोऽवस्थापयेत्सदा}


\twolineshloka
{त्वक्छ्रोत्रं च ततो जिह्वा घ्राणं चक्षुश्च संहरेत्}
{पञ्चेन्द्रियाणि सन्धाय मनसि स्थापयेद्बुधः}


\twolineshloka
{सर्वं चापोह्य सङ्कल्पमात्मनि स्थापयेन्मनः}
{यदैतान्यवतिष्ठन्ते मनःषष्ठानि चात्मनि}


\twolineshloka
{प्राणापानौ तदा तस्य युगपत्तिष्ठतो वशे}
{प्राणे हि वशमापन्ने योगसिद्धिर्ध्रुवा भवेत्}


\twolineshloka
{शरीरं चिन्तयेत्सर्वं विपाट्य च समीपतः}
{अन्तर्देहगतिं चापि प्राणानां परिचिन्तयेत्}


\twolineshloka
{ततो मूर्धानमग्निं च शरीरं परिपालयेत्}
{प्राणो मूर्धनि च श्वासो वर्तमाने विचेष्टते}


\twolineshloka
{सज्जस्तु सर्वभूतात्मा पुरुषः स सनातनः}
{मनो बुद्धिरहङ्कारो भूतानि विषयांश्च सः}


\twolineshloka
{बस्तिर्मूलं गुदं चैव पावकं च समाश्रितः}
{वहन्मूत्रं पुरीषं च सदाऽपानः प्रवर्तते}


\threelineshloka
{अतः प्रवृत्तिर्देहषु कर्म चापानसंयुतम्}
{उदीरयन्सर्वधातूनन्त ऊर्ध्वं प्रवर्तते}
{उदान इति तं विद्युरध्यात्मकुशला जनाः}


\twolineshloka
{सन्धौसन्धौ स निर्विष्टः सर्वचेष्टाप्रवर्तकः}
{शरीरेषु मनुष्याणां व्यान इत्युपदिश्यते}


\twolineshloka
{धातुष्वग्नौ च विततः समानोऽग्निः समीरणः}
{स एव सर्वचेष्टानामन्तकाले निवर्तकः}


\twolineshloka
{प्राणानां सन्निपातेषु संसर्गाद्यः प्रजायते}
{ऊष्मा सोग्निरिति ज्ञेयः सोन्नं पचति देहिनाम्}


\twolineshloka
{अपानप्राणयोर्मध्ये व्यानोदानावुपाश्रितौ}
{समन्वितः समानेन सम्यक्पचति पावकः}


\twolineshloka
{शरीरमध्ये नाभिः स्यान्नाभ्यामग्निः प्रतिष्ठितः}
{अग्नौ प्राणाश्च संयुक्ताः प्राणेष्वात्मा व्यवस्थितः}


\twolineshloka
{पक्वाशयस्त्वधो नाभेरूर्ध्वमामाशयस्तथा}
{नाभिर्मध्ये शरीरस्य सर्वप्राणाश्च संश्रिताः}


\twolineshloka
{स्थिताः प्राणादयः सर्वे तिर्यगूर्ध्वमधश्वराः}
{वहन्त्यन्नरसान्नाड्यो दशप्राणाग्निचोदिताः}


\twolineshloka
{योगिनामेष मार्गस्तु पञ्चस्वेतेषु तिष्ठति}
{जितश्रमः समासीनो मूर्धन्यात्मानमादधेत्}


\twolineshloka
{मूर्धन्यात्मानमाधाय भ्रुवोर्मध्ये मनस्तथा}
{सन्निरुध्य ततः प्राणानात्मानं चिन्तयेत्परम्}


\twolineshloka
{प्राणे त्वपानं युञ्जीत प्राणांश्चापानकर्मणि}
{प्राणापानगती रुद्ध्वा प्राणायामपरो भवेत्}


\twolineshloka
{एवमन्तः प्रयुञ्जीत पञ्च प्राणान्परस्परम्}
{विजने सम्मिताहारो मुनस्तूष्णीं निरुच्छ्वसन्}


\twolineshloka
{अश्रान्तश्चिन्तयेद्योगी उत्थाय च पुनःपुनः}
{तिष्ठन्गच्छन्स्वपंस्चापि युञ्जीतैवमतन्द्रितः}


\twolineshloka
{एवं नियुञ्जतस्तस्य योगिनो युक्तचेतसः}
{प्रसीदति मनः क्षिप्रं प्रसन्ने दृश्यते परम्}


\twolineshloka
{विधूम इव दीप्तोऽग्निरादित्य इव रश्मिवान्}
{वैद्युतोऽग्निरिवाकाशे पुरुषो दृश्यतेऽव्ययः}


\twolineshloka
{दृष्ट्वा तदात्मनो ज्योतिरैश्वर्याष्टगुणैर्युतः}
{प्राप्नोति परमं स्थानं स्पृहणीयं सुरैरपि}


\twolineshloka
{इमान्योगस्य दोषांश्च दशैव परिचक्षते}
{दोषैर्विघ्ने वरारोहे योगिनां कविभिः स्मृताः}


\twolineshloka
{कामः क्रोधो भयं स्वप्नः स्नेहमत्यशनं तथा}
{वैचित्यं व्याधिरालस्यं लोभं च दशमं स्मृतम्}


\twolineshloka
{एतैस्तेषां भवेद्विघ्नो दशभिर्देवकारितैः}
{तस्मादेतानपास्यादौ युञ्जीत च परं मनः}


\twolineshloka
{इमानपि गुणानष्टौ योगस्य परिचक्षते}
{गुणैस्तैरष्टभिर्द्रव्यमैश्वर्यमधिगम्यते}


\twolineshloka
{अणिमा महिमा चैव प्राप्तिः प्राकाम्यमेव हि}
{ईशित्वं च वशित्वं च यत्र कामावसायिता}


\twolineshloka
{एतानष्टौ गुणान्प्राप्य कथञ्चिद्योगिनां वराः}
{ईशाः सर्वस्य लोकस्य देवानप्यतिशेरते}


\twolineshloka
{योगोस्ति नैवात्यशिनो न चैकान्तमनश्नतः}
{न चातिस्वप्नशीलस्य नातिजागतरस्तथा}


\twolineshloka
{युक्ताहारविहारस्य युक्तचेष्टस्य कर्मसु}
{युक्तस्वप्नावबोधस्य योगो भवति दुःखहा}


\twolineshloka
{अनेनैव विधानेन सायुज्यं तत्प्रकल्प्यते}
{सायुज्यं देवसात्कृत्वा प्रयुञ्जीतात्मभक्तितः}


\twolineshloka
{अनन्यमनसा देवि नित्यं तद्गतचेतसा}
{सायुज्यं प्राप्यते देवैर्यत्नेन महता चिरात्}


\twolineshloka
{हविर्भिरर्चनैर्होमैः प्रणामैर्नित्यचिन्तया}
{अर्चयित्वा यथाशक्ति स्वकं देशं विशन्ति ते}


\twolineshloka
{सायुज्यानां विशिष्टं च मामकं वैष्णवं तथा}
{मां प्राप्य न निवर्तन्ते विष्णु वा शुभलोचने}


\twolineshloka
{इति ते कथितो देवि योगधर्मः सनातनः}
{न शक्यः प्रष्टुमन्येन योगधर्मस्त्वया विना}


\chapter{अध्यायः २४७}
\twolineshloka
{त्रियक्ष त्रिदशश्रेष्ठ त्र्यम्बक त्रिदशाधिप}
{त्रिपुरान्तक कामाङ्गहर त्रिपथगाधर}


\twolineshloka
{दक्षयज्ञप्रशमन सूलपाणेऽरिसूदन}
{नमस्ते लोकपालेश लोकपालवरप्रद}


\twolineshloka
{नैकशाखमपर्यन्तमध्यात्मज्ञानमुत्तमम्}
{अप्रतर्क्यमविज्ञेयं साङ्ख्ययोगसमन्वितम्}


\twolineshloka
{भवता परिपृष्टेन शृण्वन्त्या मम भाषितम्}
{इदानीं श्रोतुमिच्छामि सायुज्यं त्वद्गतं विभो}


\fourlineindentedshloka
{कथं परिचरन्त्येते भक्तास्त्वां परमेष्ठिनम्}
{आचारः कीदृशस्तेषां केन तुष्टो भवेद्भवान्}
{वर्ण्यमानं त्वया साक्षात्प्रीणयत्यधिकं हि मा ॥महेश्वर उवाच}
{}


\twolineshloka
{हन्त ते कथयिष्यामि मम सायुज्यमद्भुतम्}
{येन ते न निवर्तन्ते युक्ताः परमयोगिनः}


\twolineshloka
{अव्यक्तोऽहमचिन्त्योऽहं पूर्वैरपि मुमुक्षुभिः}
{साङ्ख्ययोगौ मया सृष्टौ सर्वं चापि चराचरम्}


\twolineshloka
{अर्चनीयोऽहमीशोऽहमव्ययोऽहं सनातनः}
{अहं प्रसन्नो भक्तानां ददाम्यमरतामपि}


\twolineshloka
{न मां विदुः सुरगणा मुनयश्च तपोधनाः}
{त्वत्प्रियार्थमहं देवि मद्विभूतिं ब्रवीमि ते}


\twolineshloka
{आश्रमेभ्यश्चतुर्भ्योऽहं चतुरो ब्राह्मणाञ्शुभे}
{मद्भक्तान्निर्मलान्पुण्यान्समानीय तपस्विनः}


\twolineshloka
{व्याचख्येऽहं तथा देवि योगं पाशुपतं महत्}
{गृहीतं तच्च तैः सर्वं मुखाच्च मम दक्षिणात्}


\twolineshloka
{श्रुत्वा तत्त्रिषु लोकेषु स्थापितं चापि तैः पुनः}
{इदानीं च त्वया पृष्टो वदाम्येकमनाः शृणु}


\twolineshloka
{अहं पसुपतिर्नाम मद्भक्ता ये च मानवाः}
{सर्वे पाशुपता ज्ञेया भस्मदिग्धतनूरुहाः}


\twolineshloka
{रक्षार्थं मङ्गलार्थं न पवित्रार्थं च भामिनि}
{लिङ्गार्थं चैव भक्तानां भस्म दत्तं मया पुरा}


\twolineshloka
{तेन संदिग्धसर्वाङ्गा भस्मना ब्रह्मचारिणः}
{जटिला मुण्डिता वाऽपि नानाकारशिखण्डिनः}


\twolineshloka
{विकृताः पिङ्गलाभिस्च नग्ना नानाप्रकारिणः}
{भैक्षं चरन्तः सर्वत्र निःस्पृहा निष्परिग्रहाः}


\twolineshloka
{मृत्पात्रहस्ता मद्भक्ता मन्निवेशितबुद्ध्यः}
{चरन्तो निखिलं लोकं मम हर्षविवर्धनाः}


\twolineshloka
{मम पाशुपतं दिव्यं योगशास्त्रमनुत्तमम्}
{सूक्ष्मं सर्वेषु लोकेषु विमृशन्तश्चरन्ति ते}


\twolineshloka
{एवं नित्याभियुक्तानां मद्भक्तानां तपस्विनाम्}
{उपायं चिन्तयाम्याशु येन मामुपयान्ति ते}


\twolineshloka
{स्थापितं त्रिषु लोकेषु शिवलिङ्गं मया मम}
{नमस्कारेण वा तस्य मुच्यन्ते सर्वकिल्बिषैः}


\twolineshloka
{इष्टं दत्तमधीतं च यज्ञाश्च बहुदक्षिणाः}
{शिवलिङ्गप्रणामस्य कलां नार्हन्ति षोडशीम्}


\twolineshloka
{अर्चया शिवलिङ्गस्य परितुष्याम्यहं प्रिये}
{शिवलिङ्गार्चनायां तु विदानमपि मे शृणु}


\twolineshloka
{गोक्षीरनवनीताभ्यामर्चयेद्यः शिवं मम}
{इष्टस्य हयमेधस्य यत्फलं तत्फलं भवेत्}


\twolineshloka
{घृतमण्डेन यो नित्यमर्चयेद्यः शिवं मम}
{स फलं प्राप्नुयान्मर्त्यो ब्राह्मणस्याग्निहोत्रिणः}


\twolineshloka
{केवलेनापि तोयेन स्नापयेद्यः शिवं मम}
{स चापि लभते पुण्यं प्रियं च लभते नरः}


\twolineshloka
{सघृतं गुग्गुलु सम्यग्धूपयेद्यः शिवान्तिके}
{गोसवस्य तु यज्ञस्य यत्फलं तस्य तद्भवेत्}


\twolineshloka
{यस्तु गुग्गुलपिण्डेन केवलेनापि धूपयेत्}
{तस्य रुक्मप्रधानस्य यत्फलं तस्य तद्भवेत्}


\threelineshloka
{यस्तु नानाविधैः पुष्पैर्मम लिङ्गं समर्चयेत्}
{स हि धेनुसहस्रस्य दत्तस्य फलमाप्नुयात्}
{}


\twolineshloka
{यस्तु देशान्तरं गत्वा शिवलिङ्गं समर्चयेत्}
{तस्मात्सर्वमनुष्येषु नास्ति मे प्रियकृत्तमः}


\twolineshloka
{एवं नानाविधैर्द्रव्यैः शिवलिङ्गं समर्चयेत्}
{मत्समानो मनुष्येषु न पुनर्जायते नरः}


\twolineshloka
{अर्चनाभिर्नमस्कारैरुपहारैः स्तवैरपि}
{भक्तो मामर्चयेन्नित्यं शिवलिङ्गेष्वतन्द्रितः}


\twolineshloka
{पलाशबिल्वपत्राणि राजवृक्षस्रजं तथा}
{अर्कपुष्पाणि मेध्यानि मत्प्रियाणि विशेषतः}


\twolineshloka
{फलं वा यदि वा शाकं पुष्पं वा यदि वा जलम्}
{दत्तं सम्प्रीणयेद्देवि भक्तैर्मद्गतमानसैः}


\twolineshloka
{ममाभिपरितुष्टस्य नास्ति लोकेषु दुर्लभम्}
{तस्मात्ते सततं भक्ता मामेवाभ्यर्चयन्त्युत}


\twolineshloka
{मद्भक्ता न विनश्यन्ति मद्भक्ता वीतकल्मषाः}
{मद्भक्ताः सर्वलोकेषु पूजनीया विशेषतः}


\twolineshloka
{मद्द्वेषिणश्चि ये मर्त्या मद्भक्तद्वेषिणश्च वा}
{यान्ति ते नरकं घोरमिष्ट्वा क्रतुशतैरपि}


\twolineshloka
{एतत्ते सर्वमाख्यातं योगं पाशुपतं महत्}
{मद्भक्तैर्मनुजैर्देवि श्राव्यमेतद्दिनेदिने}


\twolineshloka
{शृणुयाद्यः पठेद्वाऽपि ममेदं धर्मनिश्चयम्}
{स्वर्गं कीर्तिं धनं धान्यं स लभेत नरोत्तमः}


\chapter{अध्यायः २४८}
\threelineshloka
{एवमुक्त्वा महादेवः श्रोतुकामः स्वयं प्रभुः}
{अनुकूलां प्रियां भार्यां पार्श्वस्तामभ्यभाषत ॥महेश्वर उवाच}
{}


\threelineshloka
{परावरज्ञे धर्माणां तपोवननिवासिनाम्}
{दीक्षाविधिदमोपेते सततं व्रतचारिणि}
{पृच्छामि त्वां वरारोहे पृष्टा वद ममेप्सितम्}


\twolineshloka
{सावित्री ब्रह्मणः पत्नी कौशिकस्य शची शुभा}
{लक्ष्मीर्विष्णोः प्रिया भार्या धृतिर्भार्या यमस्य तु}


\twolineshloka
{मार्कण्डेयस्य धूमोर्णा ऋद्धिर्वैश्रवणस्य तु}
{वरुणस्य प्रिया गौरी सवितुश्च सुवर्चला}


\twolineshloka
{रोहिणी शशिनो भार्या स्वाहा चाग्नेरनिन्दिता}
{काश्यपस्यादितिश्चैव वसिष्ठस्याप्यरुन्धती}


\twolineshloka
{एताश्चान्याश्च देव्यस्तु सर्वास्ताः पतिदेवताः}
{श्रूयन्ते लोकविख्यातास्त्वया चैव सहोषिताः}


\twolineshloka
{ताभिश्च पूजिताऽपि त्वमनुवृत्त्यनुभाषणैः}
{तस्मात्तु परिपृच्छामि धर्मज्ञे लोकसम्मते}


\twolineshloka
{स्त्रीधर्मं श्रोतुमिच्छामि त्वयैव समुदाहृतम्}
{सध्रमचारिणी मे त्वं लोकसन्धारिणी तथा}


\twolineshloka
{अयं हि स्त्रीगणस्त्वां तु अनुयाति न मुञ्चति}
{त्वत्प्रसादाद्धितं श्रोतुं स्त्रीवृत्तं शुभलक्षणम्}


\twolineshloka
{त्वया चोक्तं विशेषेण गुणभूतं हि तिष्ठति}
{स्त्रिय एव सदा लोके स्त्रीगणस्य गतिः प्रिये}


\twolineshloka
{शश्वद्गौर्गोषु गच्छेत नान्यत्र रमते नरः}
{एवं लोकगतिर्देवि आदिप्रभृति वर्तते}


\twolineshloka
{प्रमदोक्तं तु यत्किञ्चित्तत्स्त्रीषु बहुमन्यते}
{न तथा मन्यते स्त्रीषु पुरुषोक्तमनिन्दिते}


\threelineshloka
{त्वयैष विदितो ह्यर्थः स्त्रीणां धर्मः सनातनः}
{तस्मात्त्वां प्रति पृच्छामि पृष्टा वद ममेप्सितम् ॥नारद उवाच}
{}


\fourlineindentedshloka
{एवमुक्ता तदा देवी महादेवेन शोभना}
{सोद्वेगा च सलज्जा च नावदत्तत्र किञ्चन}
{पुनः पुनस्तदा देवी देवः किमिति चाब्रवीत् ॥उमोवाच}
{}


\threelineshloka
{भगवन्देवदेवेश सुरासुरनमस्कृत}
{त्वदन्तिके मया वक्तुं स्त्रीणां धर्मः कथं भवेत् ॥ महेश्वरउवाच}
{}


\twolineshloka
{मन्नियोगादवश्यं तु वक्तव्यं तु मम प्रिये ॥उमोवाच}
{}


\twolineshloka
{इमा नद्यो महादेव सर्वतीर्थोदकान्विताः}
{उपस्पर्शनहेतोस्त्वां न त्यजन्ति समीपतः}


\twolineshloka
{एताभिः सह सम्मन्त्र्य प्रवक्ष्यामि तवेप्सितम्}
{अयुक्तं सत्सु तन्त्रेषु तानतिक्रम्य भाषितुम्}


\twolineshloka
{मया सम्मानिताश्चैव भविष्यन्ति सरिद्वराः ॥नारद उवाच}
{}


\twolineshloka
{इति मत्वा महादेवी नदीर्देवीः समाह्वयत्}
{विपाशां च वितस्त्यां च चन्द्रभागां सरस्वतीम्}


\twolineshloka
{शतद्रुं देविक्तां सिन्धुं गौतमीं कौशिकीं तथा}
{यमुनां नर्मदां चैव कावेरीमथ निम्नगाम्}


\fourlineindentedshloka
{तथा देवनदीं गङ्गां श्रेष्ठां त्रिपथगां शुभाम्}
{सर्वतीर्थोदकवहां सर्वपापविनाशिनीम्}
{एता नदीः समाहूय समुद्वीक्ष्येदमब्रवीत् ॥उमोवाच}
{}


\twolineshloka
{हे पुण्याः सरितः श्रेष्ठाः सर्वपापविनाशिकाः}
{ज्ञानविज्ञानसम्पन्नाः शृणुध्वं वचनं मम}


\twolineshloka
{अयं भगवता प्रश्न उक्तः स्त्रीधर्ममाश्रितः}
{न चैकया मया साद्यं तस्माद्वस्त्वानयाम्यहम्}


\threelineshloka
{युष्माभिस्तद्विचार्यैवं वक्तुमिच्छामि शोभनाः}
{तत्कथं देवदेवाय वाच्यः स्त्रीधर्म उत्तमः ॥नारद उवाच}
{}


\twolineshloka
{इति पृष्टास्तथा देव्या महानद्यश्चकम्पिरे}
{तासां श्रेष्ठतमा गङ्गा वचनं त्ववेमब्रवीत्}


\twolineshloka
{धन्याश्चानुगृहीताः स्म अनेन वचनेन ते}
{या त्वं सुरासुरैर्मान्या नदीराद्रियसेऽनघे}


\threelineshloka
{तवैवार्हति कल्याणि एवं सान्त्वप्रसादनम्}
{अशक्यमपि ये मूर्खाः स्वात्मसम्भावनायुताः}
{वाक्यं वदन्ति संसत्सु स्वयमेव यथेष्टतः}


% Check verse!
शक्तो यश्चानहंवादी सुदुर्लभतमो मतः
\twolineshloka
{त्वं हि शक्ता सती देवी वक्तुं प्रश्नमशेषतः}
{व्याहर्तुं नेच्छसि स्त्रीत्वात्संपूजयति नस्तथा}


\twolineshloka
{त्वं हि देवि महादेवी ऊहापोहविशारदा}
{दिव्यज्ञानयुता देवि दिव्यज्ञानेन्धनैधिता}


\fourlineindentedshloka
{त्वमेवार्हसि तद्वक्तुं स्त्रीणां वृत्तं शुभाशुभम्}
{याचामहे वयं श्रोतुममृतं त्वन्मुखोद्गतम्}
{कुरु देवप्रियं देवि वद स्त्रीधर्ममुत्तमम् ॥नारद उवाच}
{}


\threelineshloka
{एवं प्रसादिता देवी गङ्गया लोकपूज्यया}
{प्राह धर्ममशेषेण स्त्रीधर्मं सुरसुन्दरी ॥उमोवाच}
{}


\threelineshloka
{भगवन्देवदेवेश सुरेश्वर महेश्वर}
{त्वत्प्रसादात्सुरश्रेष्ठ तवैव प्रियकाम्यया}
{तमहं कीर्तयिष्यामि यथावच्छ्रोतुमिच्छसि}


\chapter{अध्यायः २४९}
\twolineshloka
{एवं ब्रुवन्त्यां स्त्रीधर्मं देव्यां देवस्य शासनात्}
{ऋषिगन्धर्वयक्षाणां योषितश्चाप्सरोगणाः}


\twolineshloka
{नागभूतस्त्रियश्चैव नद्यश्चैव समागताः}
{श्रुतुकामाः परं वाक्यं सर्वाः पर्यवतस्थिरे}


\twolineshloka
{उमादेवी मुदा युक्ता पुज्यमानाऽङ्गनागणैः}
{आनृशंस्यपरा देवी सततं स्त्रीगणं प्रति}


\threelineshloka
{स्त्रीगणस्य हितार्थाय भवप्रियचिकीर्षया}
{वक्तुं वचनमारेभे स्त्रीणां धर्माश्रयान्वितम् ॥उमोवाच}
{}


\twolineshloka
{भगवन्सर्वभूतेश श्रूयतां वचनं मम}
{ऋतुप्राप्ता सुशुद्दा या कन्या सेत्यभिधीयते}


\twolineshloka
{तां तु कन्यां पिता माता भ्राता मातुल एव वा}
{पितृव्यश्चैव पञ्चते दातुं प्रभवतां गताः}


\twolineshloka
{विवाहाश्च तथा पञ्च तासां धर्मार्थकारणात्}
{कामतश्च मिथो दानमितरेतरकाम्यया}


\twolineshloka
{दत्ता यस्य भवेद्भार्या एतेषां येन केन चित्}
{दातारः सुविमृश्यैव दातुमर्हन्ति नान्यथा}


\twolineshloka
{उत्तमानां तु वर्णानां मन्त्रवत्पाणिसङ्ग्रहः}
{विवाहकारणं चाहुः शूद्राणां सम्प्रयोगतः}


\threelineshloka
{यदा दत्ता भवेत्कन्या तस्माद्भार्यार्थिने स्वकैः}
{तदाप्रभृति सा नारी दशरात्रं विलज्जया}
{मनसा कर्मणा वाचा अनुकूला च सा भवेत्}


\twolineshloka
{इति भर्तृव्रतं कुर्यात्पतिमुद्दिश्य शोभना}
{तदाप्रभृति सा नारी न कुर्यात्पत्युरप्रियम्}


\threelineshloka
{यद्यदिच्छति वै भर्ता ध्रमकामार्थकारणात्}
{तथैवानुप्रिया भूत्वा तथैवोपचरेत्पतिम्}
{पतिव्रतात्वं नारीणामेतदेव सनातनम्}


\twolineshloka
{तादृशी सा भवेन्नित्यं यादृशस्तु भवेत्पतिः}
{शुभाशुभसमाचार एतद्वृत्तं समासतः}


\twolineshloka
{दैवतं सततं साध्वी भर्तारं या तु पश्यति}
{दैवमेव भवेत्तस्याः पतिरित्यवगम्यते}


\twolineshloka
{एतस्मिन्कारणं देव पौराणी श्रूयते श्रुतिः}
{कथयामि प्रसादात्ते शृणु देव समासतः}


\twolineshloka
{कस्य चित्त्वथ विप्रस्य भार्ये द्वे हि बभूवतुः}
{तयोरेका धर्मकामा देवानुद्दिश्य भक्तितः}


\twolineshloka
{भर्तारमवमत्यैव देवतासु समाहिता}
{चकार विपुलं धर्मं पूजयानाऽर्चयाऽन्वितम्}


\twolineshloka
{अपरा धर्मकामा च पतिमुद्दिश्य शोभना}
{भर्तारं दैवतं कृत्वा चकार किल तत्प्रियम्}


\threelineshloka
{एवं विवर्तमाने तु युगपन्मरणेऽध्वनि}
{गते किल महादेव तत्रैका या पतिव्रता}
{देवप्रियायां तिष्ठन्त्यां पुण्यलोकं जगाम सा}


\twolineshloka
{देवप्रिया च तिष्ठन्ती विललाप सुदुःखिता}
{तां यमो लोकपालस्तु बभाषे पुष्कलं वचः}


\twolineshloka
{मा शुचस्त्वं निवर्तस्व न लोकाः सन्ति तेऽनघे}
{स्वधर्मविमुखा सा त्वं तस्माल्लोका न सन्ति ते}


\twolineshloka
{देवता हि पतिर्नार्याः स्थापिता सर्वदैवतैः}
{अवमत्य शुभे तत्त्वं कथं लोकान्गमिष्यसि}


\twolineshloka
{मोहेनि त्वं वरारोहे न जानीषे स्वदैवतम्}
{पतिमत्या स्त्रिया कार्यो धर्मः पत्यर्पणस्त्विति}


\threelineshloka
{तस्मात्त्वं हि निवर्तस्व कुरु पत्याश्रितं हितम्}
{तदा गन्तासि लोकांस्तान्यान्गच्छन्ति पतिव्रताः}
{नान्यथा शक्यते प्राप्तुं पतीनां लोकमुत्तमम्}


\twolineshloka
{यमेनैवंविधं चोक्ता निवृत्ता पुनरेव सा}
{बभूव पतिमालम्ब्य पतिप्रियपरायणा}


\twolineshloka
{एवमेतन्महगादेव दैवतं हि स्त्रियाः पतिः}
{तस्मात्पतिपरा भूत्वा पतीनुपचरेदिति}


\chapter{अध्यायः २५०}
\twolineshloka
{पतिमत्या दिवारात्रं वृत्तान्तं श्रूयतां शुभम्}
{पत्युः पूर्वं समुत्थाय प्रातःक्रम समाचरेत्}


\twolineshloka
{पत्युर्भावं विदित्वा तु पश्चात्सम्बोधयेत्तु तम्}
{नित्यं पौर्वाह्णिकं कार्यं स्वयं कुर्याद्यथाविधि}


\twolineshloka
{निवेद्य च तथाऽऽहारं यथा सम्पद्यतामिति}
{तथैव कुर्यात्तत्सर्वं यथा पत्युः प्रियं भवेत्}


% Check verse!
यथा भर्ता तथा नारी गुरूणां प्रतिपद्यते
\twolineshloka
{शुश्रूषापोषणविधौ पतिप्रियचिकीर्षया}
{भर्तुर्निष्क्रमणे कार्यं संस्मरेदप्रमादतः}


\twolineshloka
{आगतं तु पतिं दृष्ट्वा सहसा परिचारणम्}
{स्वयं कुर्वीत सम्प्रीत्या कायश्रमहरं परम्}


\twolineshloka
{पाद्यसनाभ्यां शयनैर्वाक्यैश्च हृदयप्रियैः}
{अतिथीनामागमेन प्रीतियुक्ता सदा भवेत्}


\twolineshloka
{कर्मणा वचनेनापि तोषयेदतिथीन्सदा}
{मङ्गलं गृहशौचं च सर्वोपकरणानि च}


\twolineshloka
{सर्वकालमवेक्षेत कारयन्ती च कुर्वती}
{धर्मकार्ये तु सम्प्राप्ते तद्वद्धर्मपरा भवेत्}


\twolineshloka
{अर्थकार्ये पुनर्भर्तुः प्रमादालस्यवर्जिता}
{सा यत्नं परमं कुर्यात्तस्यि साहाय्यकारणात्}


\twolineshloka
{धुरन्धरा भवेद्भर्तुः साध्वी धर्मार्थयोः सदा}
{विहारकाले वै भर्तुर्ज्ञात्वा भावं हृदि स्थितम्}


\twolineshloka
{अलङ्कृत्य यथायोगं मन्दहाससमन्वितम्}
{वाक्यैर्मधुरसंयुक्तैः स्मयन्ती तोषयेत्पतिम्}


\twolineshloka
{कठोराणि न वाच्यानि अन्यथा प्रमदान्तरे}
{यस्यां कामी भवेद्भर्ता तस्याः प्रीतिकरी भवेत्}


\twolineshloka
{अप्रमादं पुरस्कृत्य मनसा तोषयेत्पतिम्}
{अनन्तरमथान्येषां भोजनावेक्षणं चरेत्}


\threelineshloka
{दासीदासबलीवर्दांश्चण्डालं च शुनस्तथा}
{अनाथान्कृपणांश्चैव भिक्षुकांश्च तथैव च}
{पूजयेद्बलिभैक्षेण पत्युर्धर्मं विवर्धयेत्}


\twolineshloka
{कुपितं वाऽर्थहीनं वा श्रान्तं वोपचरेत्पतिम्}
{यता स तुष्टः स्वस्थश्च तथा सन्तोषयेत्पतिम्}


\twolineshloka
{यथा कुटुम्बचिन्तायां विवादे वाऽर्थसञ्चये}
{आहूता तत्सहायार्थं तथा प्रियहितं वदेत्}


\twolineshloka
{अप्रियं च हितं ब्रूयात्तस्य धर्मार्थकाङ्क्षया}
{एकान्तचर्याकथनं कलहं वर्जयेत्परैः}


\twolineshloka
{बहिरालोकनं चैव मोहं व्रीडां च पैशुनम्}
{बह्वाशित्वं दिवास्वप्नमेवमादि विवर्जयेत्}


\twolineshloka
{रहस्येकासनं साध्वी न कुर्यादात्मजैरपि}
{यद्यद्दद्यान्नियत्स्वेति न्यासवत्परिपालयेत्}


\twolineshloka
{विस्मृतं वाऽपि यद्द्रव्यं प्रतिपद्यात्स्वशौचतः}
{यत्किञ्चित्पतिना दत्तं लब्ध्वा तत्सा सुकी भवेत्}


\twolineshloka
{अतीवाज्ञामतीर्ष्यां च दूरतः परिवर्जयेत्}
{बालवद्वृद्धवद्भार्या सदैवानुचरेत्पतिम्}


\twolineshloka
{भार्याया व्रतमित्येव कर्तव्यं सततं विभो}
{एतत्पतिव्रतावृत्तमुक्तं देव समासतः}


\twolineshloka
{न च भोगे न चैश्वर्ये न सुखे न धने तथा}
{स्पृहा यस्यास्तथा भर्तुः सा नारीणां पतिव्रता}


\twolineshloka
{पतिर्हि दैवतं स्त्रीणां पतिर्बन्धुः पतिर्गतिः}
{नान्यं गतिमहं पश्ये प्रमदाया यथा पतिः}


\threelineshloka
{जातिष्वपि च वै स्त्रीत्वं विशिष्टं मे मतिः प्रभो}
{कायक्लेशेन महता पुरुषः प्राप्नुयात्फलम्}
{तत्सर्वं लभते नारी सुखेन पतिपूजया}


\threelineshloka
{यथासुखं पतिमती सर्वं पत्यनुकूलतः}
{ईदृशं धर्मसाकल्यं पश्य त्वं प्रमदां प्रति}
{एतद्विसृज्य पच्यन्ते कुस्त्रियः पापमोहिताः}


\twolineshloka
{तपश्चर्या च दानं च पतौ तस्याः समर्पितम्}
{रूपं कुलं यशस्तेजः सर्वं तस्मिन्प्रतिष्ठितम्}


\twolineshloka
{एवं व्रतसमाचाराः स्ववृत्तेनैव शोभनाः}
{स्वभर्त्रा च सम गच्छेत्पुण्यलोकान्सुकर्मणा}


\twolineshloka
{वृद्धो विरुपो बीभत्सो धनवान्निर्धनोऽपि वा}
{एवंभूतोपि वै भर्ता स्त्रीणां भूषणमुत्तमम्}


\twolineshloka
{आढ्यं वा रूपयुक्तं वा विरूपं धनवर्जितम्}
{या पतिं तोषयेत्साध्वी सा पत्नीनां विशिष्यते}


\twolineshloka
{दरिद्रांश्च विरूपाश्च प्रमूढान्कुष्टसंयुतान्}
{पतीनुपचरेल्लोकानक्षयान्प्रतिपद्यते}


\twolineshloka
{एवं प्रवर्तमानायाः पतिः पूर्वं म्रियेत चेत्}
{तदाऽनुमरणं गच्छेत्पुनर्धर्मं चरेत वा}


\twolineshloka
{एतदेवं मया प्रोक्तं स्त्रियस्तु बहुधा स्मृताः}
{देवदानवगन्धर्वा मनुष्या इति नैकधा}


\twolineshloka
{सौम्यशीलाः शुभाचाराः सर्वास्ताः सम्भवन्ति च}
{यथा शुभं प्रवक्ष्यामि स्त्रीणां धर्मं महेश्वर}


\twolineshloka
{आसुर्यश्चैव पैशाच्यो राक्षस्यश्च भन्ति हि}
{तासां वृत्तमशेषेण श्रूयतां लोककारणात्}


\twolineshloka
{न्यायतो वाऽन्यथा प्रोक्ता भावदोषसमन्विताः}
{भर्तॄनुपचरन्त्येव रागद्वेषबलात्कृताः}


\twolineshloka
{स्वधर्मविमुखा भूत्वा प्रदूष्यन्ति यतस्ततः}
{प्रवृद्दविषया नित्यं प्रतिकूलं वदन्ति च}


\twolineshloka
{अर्थान्विनाशयन्त्येवं न गृह्णन्ति हितं क्वचित्}
{स्वबुद्धिनिरता भूत्वा जीवन्ति च यथा तथा}


\twolineshloka
{गुणवत्यः क्वचिद्भूत्वा पतिधर्मपरा इव}
{पुनर्भवन्ति पापिष्ठा विषमं वृत्तमास्थिताः}


\twolineshloka
{अनवस्थितमर्यादा बहुवेषा व्यवस्थिताः}
{असन्तुष्टाश्च लुब्धाश्च ईर्ष्याक्रोधयुता भृशम्}


\twolineshloka
{भोगप्रिया हितद्वेष्याः कामभोगपरायणाः}
{प्रायशोऽनृतभूयिष्ठा गुरूणां प्रतिलोमकाः}


\twolineshloka
{एवंवृत्तसमाचारा आसुरं भावमाश्रिताः}
{अपकारपरा नित्यं सततं कलहप्रिया}


\twolineshloka
{परुषा रुक्षवचना निर्घृणा निरपत्रपाः}
{निःस्नेहाः क्रोधनाश्चैव भर्तृपुत्रस्वबन्धुषु}


\threelineshloka
{घोरा मांसप्रिया नित्यं हसन्ति च रुदन्ति च}
{पतीन्व्यभिचरन्त्येव दुर्मार्गेण यथा तथा}
{बन्धुभिर्भर्त्सिता भूत्वा गृहकार्याणि कुर्वते}


\twolineshloka
{अथवा भर्त्सिता देव निवृत्ताः स्वेषु कर्मसु}
{तथैवात्मवधं घोरं व्यवस्येयुर्न संशयः}


\twolineshloka
{निर्दया निरनुक्रोशाः कुटुम्बार्थविलोपकाः}
{धर्मर्थरहिता घोराः सततं कुर्वते क्रियाः}


\twolineshloka
{अनर्थे निपुणाः पापाः परप्राणेषु निर्दयाः}
{एवंयुक्तसमाचाराः स्त्रियः पैशाचमाश्रिताः}


\fourlineindentedshloka
{अपरा मोहसंयुक्ता निर्लज्जा रोदनप्रियाः}
{अशुद्धा मलदिग्धाङ्ग्य पानमांसरता भृशम्}
{वदन्त्यनृतवाक्यानि हसन्ति विलपन्ति च}
{}


\threelineshloka
{दुष्पसादा महाक्रोधाः स्वप्नशीला निरन्तरम्}
{तामस्यो नष्टतत्वार्था मन्दशीला महोदराः}
{भुञ्जते विविधं सिद्धं भोजनं तीव्रसम्भ्रमाः}


\twolineshloka
{गुणरूपवयोयुक्तं पतिं कामिनमुत्तमम्}
{हित्वाऽन्येनैव गच्छन्ति सर्वधा भृशतापिताः}


\twolineshloka
{निर्लज्जा धर्मसन्दिग्धाः प्रतिकूलाः समन्ततः}
{एवंरूपसमाचाराः स्त्रियो राक्षसमाश्रिताः}


\twolineshloka
{एवंविधानां सर्वासां न परत्र महासुखम्}
{नरकाद्विप्रमुक्तानां मानुष्यं दुर्लभं भवेत्}


\threelineshloka
{कष्टं तत्रापि भुञ्जन्ते स्वकृतं दुःखजं बहु}
{दरिद्राः क्लेशभूयिष्ठा विरूपाः कुत्सिताः परैः}
{विधवा दुर्भगा वाऽपि लभन्ते दुःखमीदृशम्}


\twolineshloka
{शतवर्षसहस्राणि निरयं व्यभिचारिणी}
{व्रजेत्पतिं च पापेन संयोज्य स्वकुलं तथा}


\twolineshloka
{एतद्विज्ञाय पतितं पुनश्चेद्धितमात्मनः}
{कुर्याद्भर्तारमाश्रित्य तथा धर्मवमाप्नुयात्}


\twolineshloka
{अतिसंयान्ति ताँल्लोकान्पुण्यान्परमशोभनान्}
{अवमत्य च याः पूर्वं पतिं दुष्टेन चेतसा}


\twolineshloka
{वर्तमानास्च सततं भर्तॄणां प्रतिकूलतः}
{भर्त्रानुमरणं काले याः कुर्वन्ति तथाविधाः}


\twolineshloka
{कामात्क्रोधाद्भयाल्लोभादपहास्या भवन्ति ताः}
{आदिप्रभृति कुस्त्रीणां तथाऽनुमरणं वृथा}


\twolineshloka
{आदिप्रभृति या साध्वी पत्युः प्रियपरायणा}
{ऊर्ध्वं गच्छति सा पूता भर्त्राऽनुमरणं गता}


\twolineshloka
{एवं मृताया वै लोकानहं पश्यामि चक्षुषा}
{स्पृहणीयान्सुरगणैर्यान्गच्छन्ति पतिव्रताः}


\threelineshloka
{अथवा भर्तरि मृते वैधव्यं धर्ममाश्रिताः}
{तूष्णीं भौमं जले नित्यमञ्जलिस्नानमुत्तमम्}
{व्रतं च पतिमुद्दिश्य कुर्यश्चैव विधिं ततः}


\twolineshloka
{एवं गच्छति सा नारी पतिलोकमनुत्तमम्}
{रमणीयमनिर्देश्यं दुष्प्रापं देवमानुषैः}


\twolineshloka
{प्राप्नुयात्तादृशं लोकं केवलं या पतिव्रता}
{इति ते कथितं देव स्त्रीणां धर्मिमनुत्तमम्}


\threelineshloka
{भवतः प्रियकामिन्या यन्मयोक्तं तवाग्रतः}
{चापल्यान्मम देवेश तद्भवान्क्षन्तुमर्हति ॥नारद उवाच}
{}


\twolineshloka
{एवं वदन्तीं रुद्राणीं लज्जाभावसमन्विताम्}
{प्रशशंस च देवेशो वाचा सञ्जनयन्प्रियम्}


\twolineshloka
{ऋषयो देवगन्धर्वाः प्रमदाश्च सहस्रशः}
{प्रणम्य शिरसा देवीं स्तुतिभिश्चाभितुष्टुवुः}


\twolineshloka
{पूजयामासुरपरे देवदेव मुदा युताः}
{संवादं चिन्तयन्त्यन्ते श्रद्दधानाः सुचेतसः}


\threelineshloka
{ततस्तु देवदेवेशो देवीं वचनमब्रवीत्}
{शृणु कल्याणि मद्वाक्यं संवादोऽयं मया तव}
{पुण्यं पवित्रं ख्यातं च भविता नात्र संशयः}


\twolineshloka
{य इमं श्रावयेद्विद्वान्संवादं चावयोः प्रिये}
{शुचिर्भूत्वा नरान्युक्तान्स तैः स्वर्गं व्रजेत्सुखम्}


\twolineshloka
{यस्त्वेनं शृणुयान्नित्यं संवादं चावयोः शुभम्}
{कीर्तिमायुष्यमारोग्यं लभते स गतिं पराम्}


\chapter{अध्यायः २५१}
\threelineshloka
{पिनाकिन्भगनेत्रघ्न सर्वलोकनमस्कृत}
{महात्म्यं वासुदेवस्य श्रोतुमिच्छाम शङ्कर ॥ईश्वर उवाच}
{}


\twolineshloka
{पितामहादपि वरः शाश्वतः पुरुषो हरिः}
{कृष्णो जाम्बूनदाभासो व्यभ्रे सूर्य इवोदितः}


\twolineshloka
{दशबाहुर्महातेजा देवतारिनिषूदनः}
{श्रीवत्साङ्को हृषीकेशः सर्वदैवतपूजितः}


\twolineshloka
{ब्रह्मा तस्योदरभवस्तस्याहं च शिरोभवः}
{शिरोरुहेभ्यो ज्योतींषि रोमभ्यश्च सुराऽसुराः}


\twolineshloka
{ऋषयो देहसम्भूतास्तस्य लोकाश्च शाश्वताः}
{पितामहगृहं साक्षात्सर्वदेवगृहं च सः}


\twolineshloka
{सोस्याः पृथिव्याः कृत्स्नायाः स्रष्टा त्रिभुवनेश्वरः}
{संहर्ता चैव भूतानां स्थावरस्य चरस्य च}


\twolineshloka
{स हि देववरः साक्षाद्देवनाथः परन्तपः}
{सर्वज्ञः सर्वसंश्लिष्टः सर्वगः सर्वतोमुखः}


\twolineshloka
{परमात्मा हृषीकेशः सर्वव्यापी महेश्वरः}
{न तस्मात्परमं भूतं त्रिषु लोकेषु किञ्चन}


\twolineshloka
{सनातनो वै मधुहा गोविन्द इति विश्रुतः}
{स सर्वान्पार्तिवान्सङ्ख्ये घातयिष्यति मानदः}


\twolineshloka
{सुरकार्यार्थमुत्पन्नो मानुषं वपुरास्थितः}
{न हि देवगणाः शक्तास्त्रिविक्रमविनाकृताः}


\twolineshloka
{भुवने देवकार्याणि कर्तुं नायकवर्जिताः}
{नायकः सर्वभूतानां सर्वदेवनमस्कृतः}


\twolineshloka
{एतस्य देवनाथस्य देवकार्यपरस्य च}
{ब्रह्मभूतस्य सततं ब्रह्मर्षिशरणस्य च}


\twolineshloka
{ब्रह्मा वसति गर्भस्थः शरीरे सुखसंस्थितः}
{शर्वः सुखं संश्रितश्च् शरीरे सुखसंस्थितः}


\twolineshloka
{सर्वाः सुखं संश्रिताश्च शरीरे तस्य देवताः}
{स देवः पुण्डरीकाक्षः श्रीगर्भः श्रीसहोषितः}


\twolineshloka
{शार्ङ्गचक्रायुधः खड्गी सर्वनागरिपुध्वज}
{उत्तमेन स शीलेन दमेन च शमेन च}


\twolineshloka
{पराक्रमेण वीर्येण वपुषा दर्शनेन च}
{आरोहेणि प्रमाणेन धैर्येणार्जवसम्पदा}


\twolineshloka
{आनृशंस्येन रूपेण बलेन न समन्वितः}
{अस्त्रैः समुदितः सर्वैर्दिव्यैरद्भुतदर्शनैः}


\twolineshloka
{योगमायः सहस्राक्षो निरपायो महामनाः}
{वीरो मित्रजनश्लाघी ज्ञातिबन्धुजनप्रियः}


\twolineshloka
{क्षमावांश्चानहंवादी ब्रह्मण्यो ब्रह्मनायकः}
{भयहर्ता भयार्तानां मित्राणां नन्दिवर्धनः}


\twolineshloka
{शरण्यः सर्वभूतानां दीनानां पालने रतः}
{श्रुतवानर्थसम्पन्नः सर्वभूतनमस्कृतः}


\twolineshloka
{समाश्रितानां वरदः शत्रूणामपि धर्मवित्}
{नीतिज्ञो नीतिसम्पन्नो ब्रह्मवादी जितेन्द्रियः}


\twolineshloka
{भवार्थमिह देवानां बुद्ध्या परमया युतः}
{प्राजापत्ये शुभे मार्गे मानवे धर्मसंस्कृते}


\twolineshloka
{समुत्पत्स्यति गोविन्दो मनोर्वंशे महात्मनः}
{अङ्गो नाम मनोः पुत्रो अन्तर्धामा ततः परः}


\twolineshloka
{अन्तर्धाम्नो हविर्धामा प्रजापतिरनिन्दितः}
{प्राचीनबर्हिर्भविता हविर्धाम्नः सुतो महान्}


\twolineshloka
{तस्य प्रचेतःप्रमुखा भविष्यन्ति दशात्मजाः}
{प्राचेतसस्तथा दक्षो भवितेह प्रजापतिः}


\twolineshloka
{दाक्षायण्यास्तथाऽऽदित्यो मनुरादित्यतस्तथा}
{मनोश्च वंशज इला सुद्युम्नश्च भविष्यति}


\twolineshloka
{बुधात्पुरूरवाश्चापि तस्मादायुर्भविष्यति}
{नहुषो भविता तस्माद्ययातिस्तस्य चात्मजः}


\twolineshloka
{यदुस्तस्मान्महासत्वः क्रोष्टा तस्माद्भविष्यति}
{क्रोष्टुश्चैव महान्पुत्रो वृजिनीवान्भविष्यति}


\threelineshloka
{वृजिनीवतश्च भविता उषङ्गुरपराजितः}
{उषङ्गोर्भविता पुत्रः शूरश्चित्ररथस्तथा}
{तस्य त्ववरजः पुत्रः शूरो नाम भविष्यति}


\twolineshloka
{तेषां विख्यातवीर्याणां चरित्रगुणशालिनाम्}
{यज्वनां सुविशुद्धानां वंशे ब्राह्मणसम्मते}


\threelineshloka
{स शूरः क्षत्रियश्रेष्ठो महावीर्यो महायशाः}
{स्ववंशविस्तरकरं जनयिष्यति मानदः}
{वसुदेव इति ख्यातं पुत्रमानकदुन्दुभिम्}


\twolineshloka
{तस्य पुत्रश्चतुर्बाहुर्वासुदेवो भविष्यति ॥दाता ब्राह्मणसत्कर्ता ब्रह्मभूतो द्विजप्रियः}
{}


% Check verse!
राज्ञो मागधसंरुद्धान्मोक्षयिष्यति यादवः
\twolineshloka
{जरासन्धं तु राजानं निर्जित्य गिरिगह्वरे}
{सर्वपार्थिवरत्नाढ्यो भविष्यति स वीर्यवान्}


\twolineshloka
{पृथिव्यामप्रतिहतो वीर्येण च भविष्यति}
{विक्रमेण च सम्पन्नः सर्वपार्थिवपार्थिवः}


\twolineshloka
{शूरसेनेषु भूत्वा स द्वारकायां वसन्प्रभुः}
{पालयिष्यति गां देवीं विजित्य नयवित्सदा}


\twolineshloka
{तं भवन्तः समासाद्य वाङ्भाल्यैरर्हणैर्वरैः}
{अर्चयन्तु यथान्यायं ब्रह्माणमिव शाश्वतम्}


\twolineshloka
{यो हि मां द्रष्टुमिच्छेत ब्रह्माणं च पितामहम्}
{द्रष्टव्यस्तेन भगवान्वासुदेवः प्रतापवान्}


\twolineshloka
{दृष्टे तस्मिन्नहं दृष्टो न मेऽत्रास्ति विचारणा}
{पितामहो वा देवेश इति वित्त तपोधनाः}


\twolineshloka
{स यस्य पुण्डरीकाक्षः प्रीतियुक्तो भविष्यति}
{तस्य देवगणः प्रीतो ब्रह्मपूर्वो भविष्यति}


\twolineshloka
{यश्च तं मानवे लोके संश्रयिष्यति केशवम्}
{तस्य कीर्तिर्जयश्चैव स्वर्गश्चैव भविष्यति}


\twolineshloka
{धर्माणां देशिकः साक्षात्स भविष्यति धऱ्मिभाक्}
{धर्मवद्भिः स देवेशो नमस्कार्यः सदोद्यतैः}


\twolineshloka
{धर्म एव परो हि स्यात्तस्मिन्नभ्यर्चिते विभौ}
{सहि देवो महातेजाः प्रजाहितचिकीर्षया}


\twolineshloka
{धर्मार्थं पुरुषव्याघ्र ऋषिकोटीः ससर्ज ह}
{ताः सृष्टास्तेन विभुना पर्वते गन्धमादने}


\twolineshloka
{सनत्कुमारप्रमुखास्तिष्ठन्ति तपसाऽन्विताः}
{तस्मात्स वाग्मी धर्मज्ञो नमस्यो द्विजपुङ्गवाः}


\threelineshloka
{दिवि श्रेष्ठो हि भगवान्हरिर्नारायणः प्रभुः}
{वन्दितो हि स वन्देत मानितो मानयीत च}
{अर्हितश्चार्हयेन्नित्यं पूजितः प्रतिपूजयेत्}


\twolineshloka
{दृष्टः पश्येदहरहः संश्रितः प्रतिसंश्रयेत्}
{अर्चितश्चार्चयेन्नित्यं स देवो द्विजसत्तमाः}


\twolineshloka
{एतत्तस्यानवद्यस्य विष्णोर्वै परमं व्रतम्}
{आदिदेवस्य महतः सज्जनाचरितं सदा}


\twolineshloka
{भुवनेऽभ्यर्चितो नित्यं देवैरपि सनातनः}
{अभयेनानुरूपेणि युज्यन्ते तमनुव्रताः}


\twolineshloka
{कर्मणा मनसा वाचा स नमस्यो द्विजैः सदा}
{यत्नवद्भिरुपस्थाय द्रष्टव्यो देवकीसुतः}


\twolineshloka
{एष वोऽभिहितो मार्गो मया वै मुनिसत्तमाः}
{तं दृष्ट्वा सर्वशो देवं दृष्टाः स्युः सुरसत्तमाः}


\twolineshloka
{महावराहं तं देवं सर्वलोकपितामहम्}
{अहं चैव नमस्यामि नित्यमेव जगत्पतिम्}


\twolineshloka
{तत्र च त्रितयं दृष्टं भविष्यति न संशयः}
{समस्ता हि वयं देवास्तस्य देहे वसामहे}


\twolineshloka
{तस्य चैवाग्रजो भ्राता सिताद्रिनिचयप्रभः}
{हली बल इति ख्यातो भविष्यति धराधरः}


\twolineshloka
{त्रिशिरास्तस्य दिव्यश्च सातकुम्भमयो द्रुमः}
{ध्वजस्तृणेन्द्रो देवस्य भविष्यति रथाश्रितः}


\twolineshloka
{शिरो नागैर्महाभोगैः परिकीर्णं महात्मभिः}
{भविष्यति महाबाहोः सर्वलोकेश्वरस्य च}


\twolineshloka
{चिन्तितानि समेष्यन्ति शस्त्राण्यस्त्राणि चैव ह}
{अनन्तश्च स अवोक्तो भगवान्हरिरव्ययः}


\threelineshloka
{समादिष्टश्च विबुधैर्दर्शय त्वमिति प्रभो}
{सुपर्णो यस्य वीर्येण कश्यपस्यात्मजो बली}
{अन्तं नैवाशकद्द्रष्टुं देवस्य परमात्मनः}


\twolineshloka
{स च शेषो विचरते परया वै मुदा युतः}
{अन्तर्वसति भोगेन परिरभ्य वसुन्धराम्}


\twolineshloka
{य एव विष्णुः सोऽनन्तो भगवान्वसुधाधरः}
{यो रामः स हृषीकेशो योच्युतः स धराधरः}


\twolineshloka
{तावुभौ पुरुषव्याघ्रौ दिव्यौ दिव्यपराक्रमौ}
{द्रष्टव्यौ माननीयौ च चक्रलाङ्गलधारिणौ}


\twolineshloka
{एष वोऽनुग्रहः प्रोक्तो मया पुण्यस्तपोधनाः}
{यद्भवन्तो यदुश्रेष्ठं पूजयेयुः प्रयत्नतः}


\chapter{अध्यायः २५२}
\twolineshloka
{अथ व्योम्नि महाञ्शब्दः सविद्युत्स्तनयित्नुमान्}
{मेघैश्च गगनं नीलं संरुद्धमभवद्धनैः}


\twolineshloka
{प्रावृषीव च पर्जन्यो ववृषे निर्मलं पयः}
{तमश्वैवाभवद्धोरं दिशश्च न चकाशिरे}


\twolineshloka
{ततो देवगिरौ तस्मिन्रम्ये पुण्ये सनातने}
{न शर्वं भूतसङ्घं वा ददृशुर्मनयस्तदा}


\twolineshloka
{व्यभ्रं च गगनं सद्यः क्षणेन समपद्यत}
{तीर्थयात्रां ततो विप्रा जग्मुश्चान्ये यथागतम्}


\twolineshloka
{तदद्भुतमचिन्त्यं च दृष्ट्वा ते विस्मिताऽभवन्}
{शङ्करस्योमया सार्धं संवादं त्वत्कथाश्रयम्}


\twolineshloka
{स भवान्पुरुषव्याघ्र ब्रह्मभूतः सनातनः}
{यदर्थमनुशिष्टा स्मो गिरिपृष्ठे महात्मना}


\twolineshloka
{द्वितीयं त्वद्भुतमिदं त्वत्तेजःकृतमद्य वै}
{दृष्ट्वा च विस्मिताः कृष्ण सा च नः स्मृतिरागता}


\twolineshloka
{एतत्ते देवदेवस्य माहात्म्यं कथितं प्रभो}
{कपर्दिनो गिरीशस्य महाबाहो जनार्दन}


\twolineshloka
{इत्युक्ताः स तदा कृष्णस्तपोवननिवासिभिः}
{मानयामास तान्सर्वानृषीन्देवकिनन्दनः}


\twolineshloka
{अथर्षयः सम्प्रहृष्टाः पुनस्ते कृष्ममब्रुवन्}
{पुनःपुनर्दर्शयास्मान्सदैव मधुसूदन}


\twolineshloka
{न हि नः सा रति स्वर्गे या च त्वद्दर्शने विभो}
{तदृतं च महाबाहो यदाह भगवान्भवः}


\twolineshloka
{एतत्ते सर्वमाख्यातं रहस्यमरिकर्शन}
{त्वमेव ह्यर्ततत्त्वज्ञः पृष्टोऽस्मान्पृच्छसे यदा}


\twolineshloka
{तदस्माभिरिदं गुह्यं त्वत्प्रियार्थमुदाहृतम्}
{न च तेऽविदितं किञ्चित्त्रिषु लोकेषु विद्यते}


\twolineshloka
{जन्म चैव प्रसूतिश्च यच्चान्यत्कारणं विभो}
{वयं तु बहुचापल्यादशक्ता गुह्यधारणे}


\twolineshloka
{ततः स्थिते त्वयि विभो लघुत्वात्प्रलपामहे}
{न हि किञ्चित्तदाश्चर्यं यन्न वेत्ति भवानिह}


\twolineshloka
{दिवि वा भुवि वा देव सर्वं हि विदितं तव}
{साधयाम वयं कृष्ण बुद्धिं पुष्टिमवाप्नुहि}


\threelineshloka
{पुत्रस्ते सदृशस्तात विशिष्टो वा भविष्यति}
{महाप्रभावसंयुक्तो दीप्तिकीर्तिकरः प्रभुः ॥भीष्म उवाच}
{}


\twolineshloka
{ततः प्रणम्य देवेशं यादवं पुरुषोत्तमम्}
{प्रदक्षिणमुपावृत्य प्रजग्मुस्ते महर्षयः}


\twolineshloka
{सोयं नारायणः श्रीमन्दीप्त्या परमया युतः}
{व्रतं यथावत्तच्चीर्त्वा द्वारकां पुनरागमत्}


\twolineshloka
{पूर्णे च दशमे मासि पुत्रोऽस्य परमाद्भुतः}
{रुक्मिण्यां सम्मतो जज्ञे शूरो वंशधरः प्रभो}


\twolineshloka
{स कामः सर्वभूतानां सर्वभावगतो नृप}
{असुराणां सुराणां च चरत्यन्तर्गतः सदा}


\twolineshloka
{सोयं पुरुषशार्दूलो मेघवर्णश्चतुर्भुजः}
{संश्रितः पाण्डवान्प्रेम्णा भवन्तश्चैनमाश्रिताः}


\twolineshloka
{कीर्तिर्लक्ष्मीर्धृतिश्चैवक स्वर्गमार्गस्तथैव च}
{यत्रैष संस्थितस्तत्र देवो विष्णुस्त्रिविक्रमः}


\twolineshloka
{सेन्द्रा देवास्त्रयस्त्रिंशदेष नात्र विचारणा}
{आदिदेवो महादेवः सर्वभूतप्रतिश्रयः}


\twolineshloka
{अनादिनिधनोऽव्यक्तो महात्मा मधुसूदनः}
{अयं जातो महातेजाः सुराणामर्थसिद्धये}


\twolineshloka
{सुदुस्तरार्थतत्त्वस्य वक्ता कर्ता च माधवः}
{तव पार्थ जयः कृत्स्नस्तव कीर्तिस्तथाऽतुला}


\twolineshloka
{तवेयं पृथिवी देवी कृत्स्ना नारायणाश्रयात्}
{अयं नाथस्तवाचिन्त्यो यस्य नारायणो गतिः}


\twolineshloka
{स भवांस्त्वमुपाध्वर्यू रणाग्नौ हुतवान्नृपान्}
{कृष्णस्रुवेण महता युगान्ताग्निसमेन वै}


\twolineshloka
{दुर्योधनश्च शोच्योसौ सपुत्रभ्रातृबान्धवः}
{कृतवान्योऽबुधः क्रोधाद्धरिगाण्डीविविग्रहम्}


\twolineshloka
{दैतेया दानवेन्द्राश्च महाकाया महाबलाः}
{चक्राग्नौ क्षयमापन्ना दावाग्नौ शलभा इव}


\twolineshloka
{प्रतियोद्धुं न शक्यो हि मानुषैरेव संयुगे}
{विहीनैः पुरुषव्याघ्र सत्त्वशक्तिबलादिभिः}


\twolineshloka
{जयो योगी युगान्ताभः सव्यसाची रणाग्रगः}
{तेजसा हतवान्सर्वं सुयोधनवलं नृप}


\twolineshloka
{यत्तु गोवृषभाङ्केन मुनिभ्यः समुदाहृतम्}
{पुराणं हिमवत्पृष्ठे तन्मे निगदतः शृणु}


\twolineshloka
{यावत्तस्य भवेत्पुष्टिस्तेजो जीप्तिः पराक्रमः}
{प्रभावः सन्नतिर्जन्म कृष्णे तत्त्रिगुणं विभो}


\twolineshloka
{कः शक्नोत्यन्यथा कर्तुं तद्यदि स्यात्तथा शृणु}
{यत्रः कृष्णो हि भगवांस्तत्र पुष्टिरनुत्तमा}


\twolineshloka
{वयं त्विहाल्पमतयः परतन्त्राः सुविक्लबाः}
{ज्ञानपूर्वं प्रपन्नाः स्मो मृत्योः पन्थानमव्ययम्}


\twolineshloka
{भवांश्चाप्यार्जवपरः पूर्वं कृत्वा प्रतिश्रयम्}
{राजवृत्तं न लभते प्रतिज्ञापालने रतः}


\twolineshloka
{अत्येवात्मवधं लोके राजंस्त्वं बहु मन्यसे}
{न हि प्रतिज्ञा या दत्ता तां प्रहातुमरिंदम}


\twolineshloka
{कालेनायं जनः सर्वो निहतो रणमूर्धनि}
{वयं च कालेन हताः कालो हि परमेश्वरः}


\twolineshloka
{न हि कालेन कालज्ञः स्पृष्टः शोचितुमर्हसि}
{कालो लोहितरक्ताक्षः कृष्णो दण्डी सनातनः}


\twolineshloka
{तस्मात्कुन्तीसुत ज्ञातीन्नेह शोचितुमर्हसि}
{व्यपेतमन्युर्नित्यं त्वं भव कौरवनन्दन}


\twolineshloka
{माधवस्यास्य महात्म्यं श्रुतं यत्कथितं मया}
{तदेव तावत्पर्याप्तं सज्जनस्य निदर्शनम्}


\twolineshloka
{व्यासस्य वचनं श्रुत्वा नारदस्य च धीमतः}
{स्वयं चैव महाराज कृष्णस्यार्हतमस्य वै}


\twolineshloka
{प्रभावश्चर्षिपूगस्य कथितः सुमहान्मया}
{महेश्वरस्य संवादं शैलपुत्र्याश्च भारत}


\twolineshloka
{धारयिष्यति यश्चैनं महापुरुषसम्भवम्}
{शृणुयात्कथयेद्वा यः स श्रेयो लभते परम्}


\twolineshloka
{भवितारश्च तस्याथ सर्वे कामा यथेप्सिताः}
{प्रेत्य स्वर्गं च लभते नरो नास्त्यत्र संशयः}


\twolineshloka
{न्याय्यं श्रेयोभिकामेन प्रतिपत्तुं जनार्दनः}
{एष एवाक्षयो विप्रैः स्तुतो राजञ्जनार्दनः}


\twolineshloka
{महेश्वरमुखोत्सृष्टा ये च धर्मगुणाः स्मृताः}
{ते त्वया मनसा धार्याः कुरुराज दिवानिशम्}


\twolineshloka
{एवं ते वर्तमानस्य सम्यग्दण्डधरस्य च}
{प्रजापालनदक्षस्य स्वर्गलोको भविष्यति}


\twolineshloka
{धर्मोणापि सदा राजन्प्रजा रक्षितुमर्हसि}
{यस्तस्य विपुलो दण्डः सम्यग्धर्मः स कीर्त्यते}


\twolineshloka
{य एष कथितो राजन्मया सज्जनसन्निधौ}
{शङ्करस्योमया सार्धं संवादो धर्मसंहितः}


\twolineshloka
{श्रुत्वा वा श्रोतुकामो वाऽप्यर्चयेद्वृषभध्वजम्}
{विशुद्धेनेह भावेन य इच्छेद्भूतिमात्मनः}


\twolineshloka
{एष तस्यानवद्यस्य नारदस्य महात्मनः}
{संदेशो देवपूजार्थं तं तथा कुरु पाण्डव}


\twolineshloka
{एतदत्यद्भुतं वृत्तं पुण्ये हि भवति प्रभो}
{वासुदेवस्य कौन्तेय स्थाणोश्चैव स्वभावजम्}


\twolineshloka
{दशवर्शसहस्राणि बदर्यामेष शाश्वतः}
{तपश्चचार विपुलं सह गाण्डीवधन्वना}


\twolineshloka
{त्रियुगौ पुण्डरीकाक्षौ वासुदेवधनंजयौ}
{विदितौ नारदादेतौ मम व्यासाच्च पार्थिव}


\twolineshloka
{बाल एव महाबाहुश्चकार कदनं महत्}
{कंसस्य पुण्डरीकाक्षो ज्ञातित्रामार्थकारणात्}


\twolineshloka
{कर्मणामस्य कौन्तेय नान्तं सङ्ख्यातुमुत्सहे}
{शाश्वतस्य पुराणस्य पुरुषस्य युधिष्ठिर}


\twolineshloka
{ध्रुवं श्रेयः परं तात भविष्यति तवोत्तमम्}
{यस्य ते पुरुषव्याघ्रः सखा चायं जनार्दनः}


\twolineshloka
{दुर्योधनं तु शोचामि प्रेत्य लोकेऽपि दुर्मतिम्}
{यत्कृते पृथिवी सर्वा विनष्टा सहयद्विपा}


\threelineshloka
{दुर्योधनापराधेन कर्णस्य शकुनेस्तथा}
{दुःशासनवतुर्थानां कुरवो निधनं गताः ॥वैशम्पायन उवाच}
{}


\twolineshloka
{एवं सम्भाषमाणे तु गाङ्गेये पुरुषर्षभे}
{तूष्णीं बभूव कौरव्यो मध्ये तेषां महात्मनाम्}


\twolineshloka
{तच्छ्रुत्वा विस्मयं जग्मुर्धृतराष्ट्रादयो नृपाः}
{सम्पूज्य मनसा कृष्णं सर्वे प्राञ्जलयऽभवन्}


\twolineshloka
{ऋषयश्चापि ते सर्वे नारदप्रमुखास्तदा}
{प्रतिगृह्याभ्यनन्दन्त तद्वाक्यं प्रतिपूज्य च}


\threelineshloka
{इत्येतदखिलं सर्वैः पाण्डवो भ्रातृभिः सह}
{श्रुतवान्सुमहाश्चर्यं पुण्यं भीष्मानुशासनम् ॥वैशम्पायन उवाच}
{}


\twolineshloka
{युधिष्ठिरस्तु गाङ्गेयं विश्रान्तं भूरिदक्षिणम्}
{पुनरेव महाबुद्धिः पर्यपृच्छन्महीपतिः}


\chapter{अध्यायः २५३}
\threelineshloka
{कृपया परया प्रोक्तः सर्वेषां पापकर्मणाम्}
{ज्ञानस्य च परस्येह तन्मे ब्रूहि पितामह ॥भीष्म उवाच}
{}


\twolineshloka
{उपायोऽयं परप्राप्तौ परमः परिकीर्तितः}
{नारायणास्यानुध्यानमर्चनं यजनं स्तुतिः}


\threelineshloka
{श्रवणं तत्कथानां च विद्वत्संरक्षणं तथा}
{विद्वच्छुश्रूषणप्रीतिरुपदेशानुपालनम्}
{स ध्यानेन जपेननाशु मुच्यते प्राकृतोपि वा}


\twolineshloka
{जपश्चतुर्विधः प्रोक्तो वैदिकस्तान्त्रिकोपि च}
{पौराणिकोथ विद्वद्भिः कथितः स्मार्त एव च}


\twolineshloka
{विद्वच्छुश्रूषया ज्ञानं विद्वत्संरक्षणेन च}
{नासाध्यं ज्ञानिनां किञ्चित्तस्माद्रक्ष्यास्त्वया द्विजाः}


\twolineshloka
{सुव्रता बन्धुहीनैका वने पूर्वं यमेन तु}
{आसीदाश्वासिता विद्वत्संरक्षणफलात्किल}


\threelineshloka
{विप्रस्य मरणे हेतुस्तत्पत्नी पितृशोकदा}
{वैश्या त्वमतिलाभोऽयं विप्रकन्येति साम्प्रतम्}
{}


\threelineshloka
{इत्युक्ताऽऽश्वासिताऽपृच्छत्केनैवं पापसंयुता}
{जाता विप्रकुले सम्यक् श्रेयश्चापि ब्रवीहि मे ॥यम उवाच}
{}


\twolineshloka
{अन्यजन्मनि विद्वांसं प्रहारैरभिपीडितम्}
{चोरशङ्काविमोक्षेण मोक्षयित्वा सुजन्मिका}


\twolineshloka
{इत्युक्ताऽष्टाक्षरध्यानजपादिश्रेयसंयुता}
{यमेनानुगृहीताऽभूत्पुण्यलोकनिवासिनी}


\twolineshloka
{तन्नित्यं विदुषां रक्षा तत्परोऽभूर्महीपते}
{तेषां संरक्षणात्सद्यः सर्वपापैः प्रमुच्यते}


\chapter{अध्यायः २५४}
\twolineshloka
{श्रुत्वा धर्मानशेषेण पावनानि च सर्वशः}
{युधिष्ठिरः शान्तनवं पुनरेवाभ्यभाषत}


\twolineshloka
{किमेकं दैवतं लोके किं वाऽप्येकं परायणम्}
{स्तुवन्तः कं कर्मर्चन्तः प्राप्नुयुर्मानवाः शुभम्}


\threelineshloka
{को धर्मः सर्वधर्माणां भवतः परमो मतः}
{किं जपन्मुच्यते जन्तुर्जन्मसंसारबन्धनात् ॥भीष्म उवाच}
{}


\twolineshloka
{जगत्प्रभुं देवदेवमनन्तं पुरुषोत्तमम्}
{स्तुवन्नामसहस्रेण पुरुषः सततोत्थितः}


\twolineshloka
{तमेव चार्चयन्नित्यं भक्त्या पुरुषमव्ययम्}
{ध्यायन्स्तुवन्नमस्यंश्च यजमानस्तमेव च}


\twolineshloka
{अनादिनिधनं विष्णुं सर्वलोकमहेश्वरम्}
{लोकाध्यक्षं स्तुवन्नित्यं सर्वदुःकातिगो भवेत्}


\twolineshloka
{ब्रह्मण्यं सर्वधर्मज्ञं लोकानां कीर्तिवर्धनम्}
{लोकनाथं महद्भूतं सर्वबूतभवोद्भवम्}


\twolineshloka
{एष मे सर्वधर्माणां धर्मोऽधिकतमो मतः}
{यद्भक्त्या पुण्डरीकाक्षं स्तवैरर्चेन्नरः सदा}


\twolineshloka
{परमं यो महत्तेजः परमं यो महत्तपः}
{परमं यो महद्ब्रह्म परमं यः परायणम्}


\twolineshloka
{पवित्राणां पवित्रं यो मङ्गलानां च मङ्गलम्}
{दैवतं देवतानां च भूतानां योऽव्ययः पिता}


\twolineshloka
{यतः सर्वाणि भूतानि भवन्त्यादियुगागमे}
{यस्मिंश्च प्रलयं यान्ति पुनरेव युगक्षये}


\twolineshloka
{तस्य लोकप्रधानस्य जगन्नाथस्य भूपते}
{विष्णोर्नामसहस्रं मे शृणु पापभयापहम्}


\twolineshloka
{यानि नामानि गौणानि विख्यातानि महात्मनः}
{ऋषिभिः परिगीतानि तानि वक्ष्यामि भूतये}


\twolineshloka
{`ऋषिर्नाम्नां सहस्रस्य देवव्यासो महामुनिः}
{छन्दोनुष्टुप्तथा देवो भगवान्देवकीसुतः}


\twolineshloka
{अमुतांशूद्भवो बीजं शक्तिर्देवकिनन्दनः}
{त्रिसाम हृदयं तस्य शान्त्यर्थे विनियुज्यते ॥'}


\twolineshloka
{ओं विश्वं विष्णुर्वषट्कारो भूतभव्यभवत्प्रभुः}
{भूतकृद्भूतभृद्भावो भूतात्मा भूतभावनः}


\twolineshloka
{पूतात्मा परमात्मा च मुक्तानां परमा गतिः}
{अव्ययः पुरुषः साक्षी क्षेत्रज्ञोऽक्षर एव च}


\twolineshloka
{योगो योगविदांनेता प्रधानपुरुषेश्वरः}
{नरसिंहवपुः श्रीमान्केशवः पुरुषोत्तमः}


\twolineshloka
{सर्वः शर्वः शिवः स्थाणुर्भूतादिर्निधिरव्ययः}
{सम्भवो भावनो भर्ता प्रभवः प्रभुरीश्वरः}


\twolineshloka
{स्वयंभूः शंभुरादित्यः पुष्कराक्षो महास्वनः}
{अनादिनिधनो धाता विधाता धातुरुत्तमः}


\twolineshloka
{अप्रमेयो हृषीकेशः पद्मनाभोऽमरप्रभुः}
{विश्वकर्मा मनुस्त्वष्टा स्थविष्ठः स्थविरो ध्रुवः}


\twolineshloka
{अग्राह्यः शाश्वतः कृष्णो लोहिताक्षः प्रतर्दनः}
{प्रभूतस्त्रिककुद्धाम पवित्रं मङ्गलं परम्}


\twolineshloka
{ईशानः प्राणदः प्राणो ज्येष्ठः श्रेष्ठः प्रजापतिः}
{हिरण्यगर्भो भूगर्भो माघवो मधुसूदनः}


\twolineshloka
{ईश्वरो विक्रमी धन्वी मेधावी विक्रमः क्रमः}
{अनुत्तमो दुराधर्षः कृतज्ञः कृतिरात्मवान्}


\twolineshloka
{सुरेशः शरणं शर्म विश्वरेताः प्रजाभवः}
{अहः संवत्सरो व्यालः प्रत्ययः सर्वदर्शनः}


\twolineshloka
{अजः सर्वेश्वरः सिद्धः सिद्धिः सर्वादिरच्युतः}
{वृषाकपिरमेयात्मा सर्वयोगविनिःसृतः}


\twolineshloka
{वसुर्वसुमनाः सत्यः समात्मा सम्मितः समः}
{अमोघः पुण्डरीकाक्षो वृषकर्मा वृषाकृतिः}


\twolineshloka
{रुद्रो बहुशिरा बभ्रुर्विश्वयोनिः शुचिश्रवाः}
{अमृतः शाश्वतः स्थापुर्वरारोहो महातपाः}


\twolineshloka
{सर्वगः सर्वविद्भानुर्विष्वक्सेनो जनार्दनः}
{वेदो वेदविदव्यङ्गो वेदाङ्गो वेदवित्कविः}


\twolineshloka
{लोकाध्यक्षः सुराध्यक्षो धर्माध्यक्षः कृताकृतः}
{चतुरात्मा चतुर्व्यूहश्चतुर्दंष्ट्रश्चतुर्भुजः}


\twolineshloka
{भ्राजिष्णुर्भोजनं भोक्ता सहिष्णुर्जगदादिजः}
{अनघो विजयो जेता विश्वयोनिः पुनर्वसु}


\twolineshloka
{उपेन्द्रो वामनः प्रांशुरमोघः शुचिरूर्जितः}
{अतीन्द्रः सङ्ग्रहः सर्गो धृतात्मा नियमोयमः}


\twolineshloka
{वेद्यो वैद्यः सदायोगी वीरहा माधवो मधुः}
{अतीन्द्रियो महामायो महोत्साहो महाबलः}


\twolineshloka
{महाबुद्धिर्महावीर्यो महाशक्तिर्महाद्युतिः}
{अनिर्देश्यवपुः श्रीमानमेयात्मा महाद्रिधृक्}


\twolineshloka
{महेष्वासो महीभर्ता श्रीनिवासः सताङ्गतिः}
{अनिरुद्धः सुरानन्दो गोविन्दो गोविदाम्पतिः}


\twolineshloka
{मरीचिर्दमनो हंसः सुपर्णो भुजगोत्तमः}
{हिरण्यनाभः सुतपाः पद्मनाभः प्रजापतिः}


\twolineshloka
{अमृत्युः सर्वदृक्सिंहः सन्धाता सन्धिमान्स्थिरः}
{अजो दुर्मर्षणः शास्ता विश्रुतात्मा सुरारिहा}


\twolineshloka
{गुरुर्गरुतमो धाम सत्यः सत्यपराक्रमः}
{निमिषोऽनिमिषः स्रग्वी वाचस्पतिरुदारधीः}


\twolineshloka
{अग्रणीर्ग्रामणीः श्रीमान्न्यायो नेता समीरणः}
{सहस्रमूर्धा विश्वात्मा सहस्राक्षः सहस्रपात्}


\twolineshloka
{आवर्तनो निवृत्तात्मा संवृतः सम्प्रमर्दनः}
{अहः संवर्तको वह्निरनिलो धरणीधरः}


\twolineshloka
{सुप्रसादः प्रसन्नात्मा विश्वदृग्विश्वभुग्विभुः}
{सत्कर्ता सत्कृतिः साधुर्जह्नुर्नारायणो नरः}


\twolineshloka
{असंख्येयोप्रमेयात्मा विशिष्टः शिष्टकृच्छुचिः}
{सिद्धार्थः सिद्धसङ्कल्पः सिद्धिदः सिद्धिसाधनः}


\twolineshloka
{वृषाहिर्वृषभो विष्णुर्वृषपर्वा वृषोदरः}
{वर्धनो वर्धमानश्च विविक्तः श्रुतिसागरः}


\twolineshloka
{सुभुजो दुर्धरो वाग्मी महेन्द्रो वसुदो वसुः}
{नैकरूपो बृहद्रूपः शिपिविष्टः प्रकाशनः}


\twolineshloka
{ओजस्तेजोद्युतिधरः प्रकाशात्मा प्रतापनः}
{ऋद्धः स्पृष्टाक्षरो मन्त्रश्चन्द्रांशुर्भास्करद्युतिः}


\twolineshloka
{अमृतांशूद्भवो भानुः शशबिन्दुः सुरेश्वरः}
{औषधं जगतः सेतुः सत्यधर्मपराक्रमः}


\twolineshloka
{भूतभव्यभवन्नाथः पवनः पावनोऽनलः}
{कामहा कामकृत्कान्तः कामः कामप्रदः प्रभुः}


\twolineshloka
{युगादिकृद्युगावर्तो नैकमायो महाशनः}
{अदृश्यो व्यक्तरूपश्च सहस्रजिदनन्तजित्}


\twolineshloka
{इष्टो विशिष्टः शिष्टेष्ट शिखण्डी नहुषो वृषः}
{क्रोधहा क्रोधकृत्कर्ता विश्वबाहुर्महीधरः}


\twolineshloka
{अच्युतः प्रथितः प्राणः प्राणदो वासवानुजः}
{अपांनिधिरधिष्ठानमप्रमत्तः प्रतिष्ठितः}


\twolineshloka
{स्कन्दः स्कन्दधरो धुर्यो वरदो वायुवाहनः}
{वासुदेवो बृहद्भानुरादिदेवः पुरंदरः}


\twolineshloka
{अशोकस्तारणस्तारः शूरः शौरिर्जनेश्वरः}
{अनुकूलः शतावर्तः पद्मी पद्मनिभेक्षणः}


\twolineshloka
{पद्मनाबोऽरविन्दाक्षः पद्मगर्भः शरीरभृत्}
{महर्द्धिर्ऋद्धो वृद्धात्मा महाक्षो गरुडध्वक्षः}


\twolineshloka
{अतुलः शरभो भीमः समयज्ञो हविर्हरिः}
{सर्वलक्षणलक्षण्यो लक्ष्मीवान्समितिंजयः}


\twolineshloka
{विक्षरो रोहितो मार्गो हेतुर्दामोदरः सहः}
{महीधरो महाभागो वेगवानमिताशनः}


\twolineshloka
{उद्भवः क्षोभणो देवः श्रीगर्भः परमेश्वरः}
{करणं कारणं कर्ता विकर्ता गहनो गुहः}


\twolineshloka
{व्यवसायो व्यवस्थानः संस्थानः स्थानदो ध्रुवः}
{परर्द्धिः परमस्पष्टस्तुष्टः पुष्टः शुभेक्षणः}


\twolineshloka
{रामो विरामो विरजो मार्गो नेयो नयोऽनयः}
{वीरः शक्तिमतांश्रेष्ठो धर्मो धर्मविदुत्तमः}


\twolineshloka
{वैकुण्ठः पुरुषः प्राणः प्राणदः प्रणवः पृथुः}
{हिरण्यगर्भः शत्रुघ्नो व्याप्तो वायुरधोक्षजः}


\twolineshloka
{ऋतुः सुदर्शनः कालः परमेष्ठी परिग्रहः}
{उग्रः संवत्सरो दक्षो विश्रामो विश्वदक्षिणः}


\twolineshloka
{विस्तारः स्थावरः स्थाणुः प्रमाणं बीजमव्ययम्}
{अर्थोऽनर्थो महाकोशो महाभोगो महाधनः}


\twolineshloka
{अनिर्विण्णः स्थविष्ठो भूर्धर्मयूपो महामखः}
{नक्षत्रनेमिर्नक्षत्री क्षमः क्षामः समीहनः}


\twolineshloka
{यज्ञ इज्यो महेज्यश्च क्रतुः सत्रं सताङ्गतिः}
{सर्वदर्शी वमुक्तात्मा सर्वज्ञो ज्ञानमुत्तमम्}


\twolineshloka
{सुव्रतः सुमुखः सूक्ष्मः सुघोषः सुखदः सुहृत्}
{मनोहरो जितक्रोधो वीरबाहुर्विदारणः}


\twolineshloka
{स्वापनः स्ववशो व्यापी नैकात्मा नैककर्मकृत्}
{वत्सरो वत्सलो वत्सी रत्नगर्भो धनेश्वरः}


\twolineshloka
{धर्मगुब्धर्मकृद्धर्मी सदसत्क्षरमक्षरम्}
{अविज्ञाता सहस्रांशुर्विधाता कृतलक्षणः}


\twolineshloka
{गभस्तिनेमिः सत्वस्थः सिंहो भूतमहेश्वरः}
{आदिदेवो महादेवो देवेशो देवभृद्गुरुः}


\twolineshloka
{उत्तरो गोपतिर्गोप्ता ज्ञानगम्यः पुरातनः}
{शरीरी भूतभृद्भोक्ता कपीन्द्रो भूरिदक्षिणः}


\twolineshloka
{सोमपोऽमृतपः सोमः पुरुजित्पुरुसत्तमः}
{विनयोज्यः सत्यसन्धो दाशार्हः सात्वतांपतिः}


\twolineshloka
{जीवो विनयिता साक्षी मुकुन्दोऽमितविक्रमः}
{अम्भोनिधिरनन्तात्मा महोदधिशयोऽन्तकः}


\twolineshloka
{अजो महार्हः स्वाभाव्यो जितामित्रः प्रमोदनः}
{आनन्दो नन्दनो नन्दः सत्यधर्मा त्रिविक्रमः}


\twolineshloka
{महर्षिः कपिलाचार्य कृतज्ञो मेदिनीपतिः}
{त्रिपदस्त्रिदशाध्यक्षो महाशृङ्गः कृतान्तकृत्}


\twolineshloka
{महावराहो गोविन्दः सुषेणः कनकाङ्गदी}
{गुह्यो गभीरो गहनो गुप्तश्चक्रगदाधरः}


\twolineshloka
{वेधाः स्वाङ्गोऽजितः कृष्णो दृढः सङ्कर्षणोच्युतः}
{वरुणो वारुणो वृक्षः पुष्कराक्षो महामनाः}


\twolineshloka
{भगवान्भगहा नन्दी वनमाली हलायुधः}
{आदित्यो ज्योतिरादित्यः सहिष्णुर्गतिसत्तमः}


\twolineshloka
{सुधन्वा खण्डपरशुर्दारुणो द्रविणप्रदः}
{दिवस्पृक्सर्वदृग्व्यासो वाचस्पतिरयोनिजः}


\twolineshloka
{त्रिसामा सामगः साम निर्वाणं भेषजं भिषक्}
{संन्यासकृच्छमः शान्तो निष्ठा शान्तिः परायणम्}


\twolineshloka
{शुभाङ्गः शान्तिदः स्रष्टा कुमुदः कुवलेशयः}
{गोहितो गोपतिर्गोप्ता वृषभाक्षो वृषप्रियः}


\twolineshloka
{अनिवर्ती निवृत्तात्मा संक्षेप्ता क्षेमकृच्छिवः}
{श्रीवत्सवक्षाः श्रीवासः श्रीपतिः श्रीमतांवरः}


\twolineshloka
{श्रीदः श्रीशः श्रीनिवासः श्रीनिधिः श्रीविभावनः}
{श्रीधरः श्रीकरः श्रेयः श्रीमाँल्लोकत्रयाश्रयः}


\twolineshloka
{स्वक्षः स्वङ्गः शतानन्दो नन्दिर्ज्योतिर्गणेश्वरः}
{विजितात्मा विधेयात्मा सत्कीर्तिश्छिन्नसंशयः}


\twolineshloka
{उदीर्णः सर्वतश्चक्षुरनीशः शाश्वतः स्थिरः}
{भूशयो भूषणो भूतिर्विशोकः शोकनाशनः}


\twolineshloka
{अर्चिष्मानर्चितः कुम्भो विशुद्धात्मा विशोधनः}
{अनिरुद्धोऽप्रतिरथः प्रद्युम्नोऽमितविक्रमः}


\twolineshloka
{कालनेमिनिहा वीरः शौरिः सूरजनेश्वरः}
{त्रिलोकात्मा त्रिलोकेशः केशवः केशिहा हरिः}


\twolineshloka
{कामदेवः कामपालः कामी कान्तः कृतागमः}
{अनिर्देश्यवपुर्विष्णुर्वीरोऽनन्तो धनञ्जयः}


\twolineshloka
{ब्रह्मण्यो ब्रह्मकृद्ब्रह्मा ब्रह्म ब्रह्मविवर्धनः}
{ब्रह्मविद्ब्राह्मणो ब्रह्मी ब्रह्मज्ञो ब्राह्मणप्रियः}


\twolineshloka
{महाक्रमो महाकर्मा महातेजा महोरगः}
{महाक्रतुर्महायज्वा महायज्ञो महाहविः}


\twolineshloka
{स्तव्यः स्तवप्रियः स्तोत्रं स्तुतिःस्तोता रणप्रियः}
{पूर्णः पूरयिता पुण्यः पुण्यकीर्तिरनामयः}


\twolineshloka
{मनोजवस्तीर्थकरो वसुरेता वसुप्रदः}
{वसुप्रदो वासुदेवो वसुर्वसुमना हविः}


\twolineshloka
{सद्गतिः सत्कृतिः सत्ता सद्भूतिः सत्परायणः}
{शूरसेनो यदुश्रेष्ठः सन्निवासः सुयामुनः}


\twolineshloka
{भूतावासो वासुदेवः सर्वासुनिलयोऽनलः}
{दर्पहा दर्पहो दृप्तो दुर्धरोऽद्धाऽपराजितः}


\twolineshloka
{विश्वमूर्तिर्महमूर्तिर्दीप्तमूर्तिरमूर्तिमान्}
{अनेकमूर्तिरव्यक्तः शतमूर्तिः शताननः}


\twolineshloka
{एको नैकः सवः कः किं यत्तत्पदमनुत्तमम्}
{लोकबन्धुर्लोकनाथो माधवो भक्तवत्सलः}


\twolineshloka
{सुवर्णवर्णो हेमाङ्गो वराङ्गश्चन्दनाङ्गदी}
{वीरहा विषमः शून्यो घृताशीरचलश्चलः}


\twolineshloka
{अमानी मानदो मान्यो लोकस्वामी त्रिलोकधृत्}
{सुमेधा मेघजो धन्यः सत्यमेधा धराधरः}


\twolineshloka
{तेजो वृषो द्युतिधरः सर्वशस्त्रभृतांवरः}
{प्रग्रहो निग्रहो व्यग्रो नैकशृङ्गो गदाग्रजः}


\twolineshloka
{चतुर्मूर्तिश्चतुर्बाहुश्चतुर्व्यूहश्चतुर्गतिः}
{चतुरात्मा चतुर्भावश्चतुर्वदविदेकपात्}


\twolineshloka
{समावर्तो निवृत्तात्मा दुर्जयो दुरतिक्रमः}
{दुर्लभो दुर्गमो दुर्गो दुरावासो दुरारिहा}


\twolineshloka
{शुभाङ्गो लोकसारङ्गः सुतन्तुस्तन्तुवर्धनः}
{इन्द्रकर्मा महाकर्मा कृतकर्मा कृतागमः}


\twolineshloka
{उद्भवः सुन्दरः सुन्दो रत्ननाभः सुलोचनः}
{अर्को वाजसनः शृङ्गी जयन्तः सर्वविज्जयी}


\twolineshloka
{सुवर्णिबिन्दुरक्षोभ्यः सर्ववागीश्वरेश्वरः}
{महाह्रदो महागर्तो महाभूतो महानिधिः}


\twolineshloka
{कुमुदः कुन्दरः कुन्दः पर्जन्यः पवनोऽनिलः}
{अमृतांशोऽमृतवपुः सर्वज्ञः सर्वतोमुखः}


\twolineshloka
{सुलभः सुव्रतः सिद्धः शत्रुजिच्छत्रुतापनः}
{न्यग्रोधोदुम्बरोश्वत्थश्चापूरान्ध्रनिषूदनः}


\twolineshloka
{सहस्रार्चिः सप्तजिह्वः सप्तैधाः सप्तवाहनः}
{अमूर्तिरनघोऽचिन्त्यो भयकृद्भयनाशनः}


\twolineshloka
{अणुर्बृहत्कृशः स्थूलो गुणभृन्नर्गुणो महान्}
{अधृतः स्वधृतः स्वास्थ्यः प्राग्वंशो वंशवर्धनः}


\twolineshloka
{भारभृत्कथितो योगी योगीशः सर्वकामदः}
{आश्रमः श्रमणः क्षामः सुपर्णो वायुवाहनः}


\twolineshloka
{धनुर्धरो धनुर्वेदो दण्डो दमयिता दमः}
{अपराजितः सर्वसहो नियन्ता नियमो यमः}


\twolineshloka
{सत्ववान्सात्विकः सत्यः सत्यधर्मपरायणः}
{अभिप्रायः प्रियार्होऽर्हःऋ प्रियकृत्प्रीतिवर्धनः}


\twolineshloka
{विहायसगतिर्ज्योतिः सुरुचिर्हुतभुग्विभुः}
{रविर्विरोचनः सूर्यः सविता रविलोचनः}


\twolineshloka
{अन्तो हुतभुग्भोक्ता सुखदो नैकदोऽग्रजः}
{अनिर्विण्णः सदामर्षी लोकाधिष्ठानमद्भुतः}


\twolineshloka
{सनात्सनातनतमः कपिलः कपिरप्ययः}
{स्वस्तिदः स्वस्तिकृत्स्वस्ति स्वस्तिभुक् स्वस्तिदक्षिणः}


\twolineshloka
{अरौद्रः कुण्डली चक्री विक्रम्यूर्जितशासनः}
{शब्दातिगः शब्दसहः शिशिरः शर्वरीकरः}


\twolineshloka
{अक्रूरः पेशलो दक्षो दक्षिणः क्षमिणांवरः}
{विद्वत्तमो वीतभयः पुण्यश्रवणकीर्तनः}


\twolineshloka
{उत्तारणो दुष्कृतिहा पुण्यो दुःखप्ननाशनः}
{वीरहा रक्षणः सन्तो जीवनः पर्यवस्थितः}


\twolineshloka
{अनन्तरूपोऽनन्तश्रीर्जितमन्युर्भयापहः}
{चतुरस्रो गभीरात्मा विदिशो व्यादिशो दिशः}


\twolineshloka
{अनादिर्भूर्भुवो लक्ष्मीः सुवीरो रुचिराङ्गदः}
{जननो जनजन्मादिर्भीमो भीमपराक्रमः}


\twolineshloka
{आधारनिलयो धाता पुष्पहासः प्रजागरः}
{ऊर्ध्वगः सत्पथाचारः प्राणदः प्रणवः पणः}


\twolineshloka
{प्रमाणं प्राणनिलयः प्राणभृत्प्राणजीवनः}
{तत्त्वं तत्त्वविदेकात्मा जन्ममृत्युजरातिगः}


\twolineshloka
{भूर्भुवःस्वस्तरुस्तारः सविता प्रपितामहः}
{यज्ञो यज्ञपतिर्यज्वा यज्ञाङ्गो यज्ञवाहनः}


\twolineshloka
{यज्ञभृद्यज्ञकृद्यज्ञी यज्ञभुग्यज्ञसाधनः}
{यज्ञान्तकृद्यज्ञगुह्यमन्नमन्नाद एव च}


\twolineshloka
{आत्मयोनिः स्वयंजातो वैखानः सामगायनः}
{देवकीनन्दनः स्रष्टा क्षितीशः पापनाशनः}


\threelineshloka
{शङ्खभृन्नन्दकी चक्री शार्ङ्गधन्वा गदाधरः}
{रथाङ्गपाणिरक्षोभ्यः सर्वप्रहरणायुधः}
{सर्वप्रहारणायुध ओंनम इति}


\twolineshloka
{इतीदं कीर्तनीयस्यि केशवस्य महात्मनः}
{नाम्नां सहस्रं दिव्यानामशेषेण प्रकीर्तितम्}


\twolineshloka
{य इदं शृणुयान्नित्यं यश्चापि परिकीर्तयेत्}
{नाशुभं प्राप्नुयात्किञ्चित्सोमुत्रेह च मानवः}


\twolineshloka
{वेदान्तो ब्राह्मणः स्यात्क्षत्रियो विजयी भवेत्}
{वैश्यो धनसमृद्धः स्याच्छूद्रः सुखमवाप्नुयात्}


\twolineshloka
{धर्मार्थी प्राप्नुयाद्धर्ममर्थार्थी चार्थमाप्नुयात्}
{कामानवाप्नुयात्कामी प्रजार्थी प्राप्नुयात्प्रजाम्}


\twolineshloka
{भक्तिमान्यः सदोत्थाय शुचिस्तद्गतमानसः}
{सहस्रं वासुदेवस्य नाम्नातेमत्प्रकीर्तयेत्}


\twolineshloka
{यशः प्राप्नोति विपुलं ज्ञातिप्राधान्यमेव च}
{अचलां श्रियमाप्नोति श्रेयः प्राप्नोत्यनुत्तमम्}


\twolineshloka
{न भयं क्वचिदाप्नोति वीर्यं तेजश्च विन्दति}
{भवत्ययोगो द्युतिमान्बलरूपगुणान्वितः}


\twolineshloka
{रोगार्तो मुच्यते रोगाद्बद्धो मुच्येत बन्धनात्}
{भयान्मुच्येत भीतस्तु मुच्येदापन्न आपदः}


\twolineshloka
{दुर्गाण्यतितरत्याशु पुरुषः पुरुषोत्तमम्}
{स्तुवन्नामसहस्रेण नित्यं भक्तिसमन्वितः}


\twolineshloka
{वासुदेवाश्रयो मर्त्यो वासुदेवपरायणः}
{सर्वपापविशुद्धात्मा याति ब्रह्म सनातनम्}


\twolineshloka
{न वासुदेवभक्तानामशुभं विद्यते क्वचित्}
{जन्ममृत्युजराव्याधिभयं नैवोपजायते}


\twolineshloka
{इमं स्तवमधीयानः श्रद्धाभक्तिसमन्वितः}
{युज्येतात्मसुखक्षान्तिश्रीधृतिस्मृतिकीर्तिभिः}


\twolineshloka
{न क्रोधो न च मात्सर्यं न लोभो नाशुभा मतिः}
{भन्ति कृतपुण्यानां भक्तानां पुरुषोत्तमे}


\twolineshloka
{द्यौः सचन्द्रार्कनक्षत्रा खं दिशो भूर्महोदधिः}
{वासुदेवस्य वीर्येण विधृतानि महात्मनः}


\twolineshloka
{ससुरासुरगन्धर्वं सयक्षोरगराक्षसम्}
{जगद्वशे वर्ततेदं कृष्णस्य सचराचरम्}


\twolineshloka
{इन्द्रियाणि मनो बुद्धिः सत्वं तेजो बलं धृतिः}
{वासुदेवात्मकान्याहुः क्षेत्रं क्षेत्रज्ञ एव च}


\twolineshloka
{सर्वागमानामाचारः प्रथमं परिकल्प्यते}
{आचारप्रभवो धर्मो धर्मस्य प्रभुरच्युतः}


\twolineshloka
{ऋषयः पितरो देवा महाभूतानि धातवः}
{जङ्गमाजङ्गमं चेदं जगन्नारयणोद्भवम्}


\twolineshloka
{योगो ज्ञानं तथा साङ्ख्यं विद्याः शिल्पादिकर्म च}
{वेदाः सास्त्राणि विज्ञानमेतत्सर्वं जनार्दनात्}


\twolineshloka
{एको विष्णुर्महद्भूतं पृथग्भूतान्यनेकशः}
{त्रींल्लोकान्व्याप्य भूतात्मा भुङ्क्ते विश्वभुगव्ययः}


\twolineshloka
{इमं स्तवं भगवतो विष्णोर्व्यासेन कीर्तितम्}
{पठेद्य इच्छेत्पुरुषः श्रेयः प्राप्तुं सुखानि च}


\fourlineindentedshloka
{विश्वेश्वरमजं देवं जगतः प्रभवाप्ययम्}
{भजन्ति ये पुष्कराक्षं न ते यान्ति पराभवम्}
{`न ते यान्ति पराभवम् ओं नम इति ॥अर्जुन उवाच}
{}


\threelineshloka
{पद्मपत्रविशालक्ष पद्मनाभ सुरोत्तम}
{भक्तानामनुरक्तानां त्राता भव जनार्दन ॥भगवानुवाच}
{}


\threelineshloka
{यो मां नामसहस्रेण स्तोतुमिच्छति पाण्डव}
{सोहमेकेन श्लोकेन स्तुत एव न संशयः}
{स्तुत एव न संशय ओं नम इति}


\twolineshloka
{वासनाद्वासुदेवः स्या वासितं ते जगत्त्रयम्}
{सर्वभूतनिवासोसि वासुदेव् नमोस्तु ते}


\threelineshloka
{नमोस्त्वनन्ताय सहस्रमूर्तयेसहस्रपादाक्षिशिरोरुबाहवे}
{सहस्रनाम्ने पुरुषाय शाश्वतेसहस्रकोटियुगधारिणे नमः}
{श्रीसहस्रकोटियुगधारिणे नम इति ॥'}


\chapter{अध्यायः २५५}
\twolineshloka
{पितामह महाप्राज्ञ सर्वशास्त्रविशारद}
{किं जप्यं जपतो नित्यं भवेद्धर्मफलं महत्}


\twolineshloka
{प्रस्थाने वा प्रवेशे वा प्रवृत्ते वाऽपि कर्मणि}
{दैवे वा श्राद्धकाले वा किं जप्यं कर्मसाधनम्}


\threelineshloka
{शान्तिकं पौष्टिकं रक्षा शत्रुघ्नं भयनाशनम्}
{जप्यं यद्ब्रह्म समितं तद्भवान्वक्तुमर्हति ॥भीष्म उवाच}
{}


\twolineshloka
{व्यासप्रोक्तमिमं मन्त्रं शृणुष्वैकमना नृप}
{सावित्र्या विहितं दिव्यं सद्यः पावविमोचनम्}


\twolineshloka
{शृणु मन्त्रविधिं कृत्स्नं प्रोच्यमानं मयाऽनघ}
{यं श्रुत्वा पाण्डवश्रेष्ठ सर्वपापैः प्रमुच्यते}


\twolineshloka
{रात्रावहनि धर्मज्ञ जपन्पापैर्न लिप्यते}
{तत्तेऽहं सम्प्रवक्ष्यामि शृणुष्वैकमना नृप}


\twolineshloka
{आयुष्मान्भवते चैव यं श्रुत्वा पार्थिवात्मज}
{पुरुषस्तु सुसिद्धार्थः प्रेत्य चेह च मोदते}


\twolineshloka
{सेवितं सततं राजन्पुरा राजर्षिसत्तमैः}
{क्षत्रधर्मपैरर्नित्यं सत्यव्रतपरायणैः}


\twolineshloka
{इदमाह्निकमव्यग्रं कुर्वद्भिर्नियतैः सदा}
{नृपैर्भरतशार्दूल प्राप्यते श्रीरनुत्तमा}


\twolineshloka
{नमो वसिष्ठाय महाव्रतायपराशरं वेदनिधइं नमस्ते}
{नमोस्त्वनन्ताय महोरगायनमोस्तु सिद्धेभ्य इहाक्षयेभ्यः}


\twolineshloka
{नमोस्त्वृषिभ्यः परमं परेषांदेवेषु देवं वरदं वराणाम्}
{सहस्रशीर्षाय नमः शिवायसहस्रनामाय जनार्दनाय}


\twolineshloka
{अजैकपादहिर्बुध्न्यः पिनाकी चापराजितः}
{ऋतश्च पितृरूपश्च त्र्यम्बकश्च महेश्वरः}


\twolineshloka
{वृषाकपिश्च शंभुश्च हवनोऽथेश्वरस्तथा}
{एकादशैते प्रथिता रुद्रास्त्रिभुवनेश्वराः}


% Check verse!
शतमेतत्समाम्नातं शतरुद्रे महात्मनाम्
\twolineshloka
{अंशो भगश्च मित्रश्च वरुणश्च जलेश्वरः}
{तथा धाताऽर्यमा चैव जयन्तो भास्करस्तथा}


\twolineshloka
{त्वष्टा पूषा तथैवेन्द्रो द्वादशो विष्णुरुच्यते}
{इत्येते द्वादशादित्याः काश्यपेया इति श्रुतिः}


\twolineshloka
{धरो ध्रुवश्च सोमश्च सावित्रोऽथानिलोऽनलः}
{प्रत्यूषश्च प्रभासश्च वसवोष्टौ प्रकीर्तिताः}


\twolineshloka
{नासत्यश्चापि दस्रश्च स्मृतौ द्वावश्विनावपि}
{मार्तण्डस्यात्मजावेतौ संज्ञानासाविनिर्गतौ}


\twolineshloka
{अतः परं प्रवक्ष्यामि लोकानां कर्मसाक्षिणः}
{अपि यज्ञस्य वेत्तारो दत्तस्य सुकृतस्य च}


\twolineshloka
{अदृश्याः सर्वभूतेषु पश्यन्ति त्रिदशेश्वराः}
{शुभाशुभानि कर्माणि मृत्युः कालश्च सर्वशः}


\threelineshloka
{विश्वेदेवाः पितृगणा मूर्तिमन्तस्तपोधनाः}
{मुनयश्चैव सिद्धाश्च तपोमोक्षपरायणाः}
{शुचिस्मिताः कीर्तयतां प्रयच्छन्ति शुभं नृणाम्}


\twolineshloka
{प्रजापतिकृतानेतान्लोकान्दिव्येन तेजसा}
{वसन्ति सर्वलोकेषु प्रयताः सर्वकर्मसु}


\twolineshloka
{प्राणानामीश्वरानेतान्कीर्तयन्प्रयतो नरः}
{धर्मार्थकामैर्विपुलैर्युज्यते सह नित्यशः}


\twolineshloka
{लोकांश्च लभते पुण्यान्विश्वेश्वरकृताञ्शुभान्}
{एते देवास्त्रयस्त्रिंशत्सर्वभूतगणेश्वराः}


\twolineshloka
{नन्दीश्वरो महाकायो ग्रामणीर्वृषभध्वजः}
{ईश्वराः सर्वलोकानां गणेश्वरविनायकाः}


\twolineshloka
{सौम्या रौद्रा गणाश्चैव योगभूतगणास्तथा}
{ज्योतींषि सरितो व्योम सुपर्णः पतगेश्वरः}


\twolineshloka
{पृथिव्यां तपसा सिद्धाः स्थावराश्च चरास्च ह}
{हिमवान्गिरयः सर्वे चत्वारश्च महार्णवाः}


\twolineshloka
{भवस्यानुचराश्चैव हरतुल्यपराक्रमाः}
{विष्णुर्देवोथ जिष्णुश्च स्कन्दश्चाम्बिकया सह}


\twolineshloka
{कीर्तयन्प्रयतः सर्वान्सर्वपापैः प्रमुच्यते}
{अत ऊर्ध्वं प्रवक्ष्यामि मानवानृषिसत्तमान्}


\twolineshloka
{यवक्रीतश्च रैभ्यश्च अर्वावसुपरावसू}
{औशिजश्चैव कक्षीवान्बलश्चाङ्गिरसः सुतः}


\twolineshloka
{ऋषिर्मेधातिथेः पुत्रः कण्वो बर्हिषदस्तथा}
{ब्रह्मतेजोमयाः सर्वे कीर्तिता लोकभावनाः}


\twolineshloka
{लभन्ते हि शुभं सर्वे रुद्रानलवसुप्रभाः}
{भुवि कृत्वा शुभं कर्म मोदन्ते दिवि दैवतैः}


\twolineshloka
{महेन्द्रगुरवः सप्त प्राचीं वै दिशमाश्रिताः}
{प्रयतः कीर्तयेदेताञ्शक्रलोके महीयते}


\twolineshloka
{उन्मुचुः प्रमुचुश्चैव स्वस्त्यात्रेयश्च वीर्यवान्}
{दृढव्यश्चोर्ध्वबाहुश्च तृणसोमाङ्गिरास्तथा}


\twolineshloka
{मित्रावरुणयोः पुत्रस्तथाऽगस्त्यः प्रतापवान्}
{धर्मराजर्त्विजः सप्त दक्षिणां दिशमाश्रिताः}


\twolineshloka
{दृढेयुश्च ऋतेयुश्च परिव्याधश्च कीर्तिमान्}
{एकतश्च द्वितश्चैव त्रितश्चादित्यसन्निभाः}


\twolineshloka
{अत्रेः पुत्रश्च धर्मात्मा ऋषिः सारस्वतस्तथा}
{वरुणस्यर्त्विजः सप्त पश्चिमां दिशमाश्रिताः}


\twolineshloka
{अत्रिर्वसिष्ठो भगवान्कश्यपश्च महानृषिः}
{गौतमश्च भरद्वाजो विश्वामित्रोथ कौशिकः}


\twolineshloka
{ऋचीकतनयश्चोग्रो जमदग्निः प्रतापवान्}
{धनेश्वरस्य गुरवः सप्तैते उत्तराश्रिताः}


\twolineshloka
{अपरे मुनयः सप्त दिक्षु सर्वास्वधिष्ठिताः}
{कीर्तिस्वस्तिकरा नॄणां कीर्तिता लोकभावनाः}


\twolineshloka
{धर्मः कामश्च कालश्च वसुर्वासुकिरेव च}
{अनन्तः कपिलश्चैव सप्तैते धरणीधराः}


\twolineshloka
{रामो व्यासस्तथा द्रौणिरश्वत्थामा च लोमशः}
{इत्येते मुनयो दिव्या एकैकः सप्तसप्तधा}


\twolineshloka
{शान्तिस्वस्तिकरा लोके दिशांपालाः प्रतीर्तिताः}
{यस्यांयस्यां दिशि ह्येते तन्मुखः शरणं व्रजेत्}


\twolineshloka
{स्रष्टारः सर्वभूतानां कीर्तिता लोकपावनाः}
{संवर्तो मेरुसावर्णो मार्कण्डेयश्च धार्मिकः}


\twolineshloka
{साङ्ख्ययोगौ नारदश्च दुर्वासाश्च महानृषिः}
{अत्यन्ततपसो दान्तास्त्रिषु लोकेषु विश्रुताः}


\twolineshloka
{अपरे रुद्रसङ्काशाः कीर्तिता ब्रह्मलौकिकाः}
{अपुत्रो लभते पुत्रं दरिद्रो लभते धनम्}


\twolineshloka
{तथा धर्मार्थकामेषु सिद्धिं च लभते नरः}
{पृथुं वैन्यं नृपवरं पृथ्वी यस्याभवत्सुता}


\twolineshloka
{प्रजापतिं सार्वभौमं कीर्तयेद्वसुधाधिपम्}
{आदित्यवंशप्रभवं महेन्द्रसमविक्रमम्}


\twolineshloka
{पुरूरवसमैलं च त्रिषु लोकेषु विश्रुतम्}
{बुधस्य दयितं पुत्रं कीर्तयेद्वसुधाधिपम्}


\twolineshloka
{त्रिलोकविश्रुतं वीरं भरतं च प्रकीर्तयेत्}
{गवामयेन यज्ञेन येनेष्टं वै कृते युगे}


\twolineshloka
{रन्तिदेवं महादेवं कीर्तयेत्परमद्युतिम्}
{विश्वजित्तपसोपेतं लक्षण्यं लोकपूजितम्}


\twolineshloka
{तथा श्वेतं च राजर्षिं कीर्तयेत्परमुद्यतिम्}
{सगरस्यात्मजा येन प्लावितास्तारितास्तथा}


\twolineshloka
{हुताशनसमानेतान्महारूपान्महौजसः}
{उग्रकायान्महासत्वान्कीर्तयेत्कीर्तिवर्धनान्}


\twolineshloka
{देवानृषिगणांश्चैव नृपांश्च जगतीश्वरान्}
{साङ्ख्यं योगं च परमं हव्यं कव्यं तथैव च}


\twolineshloka
{कीर्तितं परमं ब्रह्म सर्वश्रुतिपरायणम्}
{मङ्गल्यं सर्वभूतानां पवित्रं बहु कीर्तितम्}


\twolineshloka
{व्याधिप्रशमनं श्रेष्ठं पौष्टिकं सर्वकर्मणाम्}
{प्रयतः कीर्तयेच्चैतान्कल्यं सायं च भारत}


\twolineshloka
{एते वै यान्ति वर्षन्ति भान्ति वान्ति सृजन्ति च}
{एते विनायकाः श्रेष्ठा दक्षाः क्षान्ता जितेन्द्रियाः}


\twolineshloka
{नराणामशुभं सर्वे व्यपोहन्ति प्रकीर्तिताः}
{साक्षिभूता महात्मानः पापस्य सुकृतस्य च}


\twolineshloka
{एतान्वै कल्यमुत्थाय कीर्तयञ्शुभमश्नुते}
{नाग्निचोरभयं तस्य न मार्गप्रतिरोधनम्}


\twolineshloka
{एतान्कीर्तयतां नित्यं दुःस्वप्नो नश्यते नृणाम्}
{मुच्यते सर्वपापेभ्यः स्वस्तिमांश्च गृहान्व्रजेत्}


\twolineshloka
{दीक्षाकालेषु सर्वेषु यः पठेन्नियतो द्विजः}
{न्यायवानात्मनिरतः क्षान्तो दान्तोऽनसूयकः}


\twolineshloka
{रोगार्तो व्याधियुक्तो वा पठन्पापात्प्रमुच्यते}
{वास्तुमध्ये तु पठतः कुले स्वस्त्ययनं भवेत्}


\twolineshloka
{क्षेत्रमध्ये तु पठतः सर्वं सस्यं प्ररोहति}
{गच्छतः क्षेममध्वानं ग्रामान्तरगतः पठन्}


\twolineshloka
{आत्मनश्च सुतानां च दाराणां च धनस्य च}
{बीजानामोषधीनां च रक्षामेतां प्रयोजयेत्}


\twolineshloka
{एतान्सङ्ग्रामकाले तु पठतः क्षत्रियस्य तु}
{व्रजन्ति रिपवो नाशं क्षेमं च परिवर्तते}


\twolineshloka
{एतान्दैवे च पित्र्ये च पठतः पुरुषस्य हि}
{भुञ्जते पितरः कव्यं हव्यं च त्रिदिवौकसः}


\twolineshloka
{न व्याधिश्वापदभयं न द्विपान्न हि तस्करात्}
{कश्मलं लघुतां याति पाप्मना च प्रमुच्यते}


\twolineshloka
{यानपात्रे च याने च प्रवासे राजवेश्मनि}
{परां सिद्धिमवाप्नोति सावित्रीं ह्युत्तमां पठन्}


\twolineshloka
{न च राजभयं तेषां न पिशाचान्न राक्षसात्}
{नाग्न्यम्बुपवनव्यालाद्भयं तस्योपजायते}


\twolineshloka
{चतुर्णामपि वर्णानामाश्रमस्य विशेषतः}
{करोति सततं शान्तिं सावित्रीमुत्तमां पठन्}


\twolineshloka
{नाग्निर्दहति काष्ठानि सावित्री यत्र पठ्यते}
{न तत्र बालो म्रियते न च तिष्ठन्ति पन्नगाः}


\twolineshloka
{न तेषां विद्यते दुःखं गच्छन्ति परमां गतिम्}
{ये शृण्वन्ति महद्ब्रह्म सावित्रीगुणकीर्तनम्}


\twolineshloka
{गवां मध्ये तु पठतो गावोऽस्य बहुवत्सलाः}
{प्रस्थाने वा प्रवासे वा सर्वावस्थां गतः पठेत्}


\twolineshloka
{जपतां जुह्वतां चैव नित्यं च प्रयतात्मनाम्}
{ऋषीणां परमं जप्यं गुह्यमेतन्नराधिप}


\twolineshloka
{याथातथ्येन सिद्धस्य इतिहासं पुरातनम्}
{पराशरमतं दिव्यं शक्राय कथितं पुरा}


\twolineshloka
{तदेतत्ते समाख्यातं तथ्यं ब्रह्म सनातनम्}
{हृदयं सर्वभूतानां श्रुतिरेषा सनातनी}


\twolineshloka
{सोमादित्यान्वयाः सर्वे राघवाः कुरवस्तथा}
{पठन्ति शुचयो नित्यं सावित्रीं प्राणिनां गतिं}


\twolineshloka
{अभ्यासे नित्यं देवानां सप्तर्षीणां ध्रुवस्य च}
{मोक्षणं सर्वकृच्छ्राणां मोचयत्यशुभात्सदा}


\twolineshloka
{वृद्धैः काश्यपगौतमप्रभृतिभिर्भृग्वङ्गिरोत्र्यादिभिःशुक्रागस्त्यबृहस्पतिप्रभृतिभिर्ब्रह्मह्मर्षिभिःसेवितम्}
{भारद्वाजमतं ऋचीकतनयैः प्राप्तं वसिष्ठात्पुनःसावित्रीमधिगम्य शक्रवसुभिः कृत्स्ना जिता दानवाः}


\twolineshloka
{यो गोशतं कनकशृङ्गमयं ददातिविप्राय वेदविदुषे च बहुश्रुताय}
{दिव्यां च भारतकथां कथयेच्च नित्यंतुल्यं फलं भवति तस्य च तस्य चैव}


\twolineshloka
{धर्मो विवर्धति भृगोः परिकीर्तनेनवीर्यं विवर्धति पसिष्ठनमोनतेन}
{सङ्ग्रामजिद्भवति चैव रघुं नमस्य-न्स्यादश्विनौ च परिकीर्तयतो न रोगः}


\twolineshloka
{एषा ते कथिता राजन्सावित्री ब्रह्मशाश्वती}
{विवक्षुरसि यच्चान्यत्तत्ते वक्ष्यामि भारत}


\chapter{अध्यायः २५६}
\threelineshloka
{के पूज्याः के नमस्कार्या कथं वर्तेत केषु च}
{किमाचारः कीदृशेषु पितामह न रिष्यते ॥भीष्म उवाच}
{}


\twolineshloka
{ब्राह्मणानां परिभवः सादयेदपि देवताः}
{ब्राह्मणांस्तु नमस्कृत्य युधिष्ठिर न रिष्यते}


\twolineshloka
{ते पूज्यास्ते नमस्कार्या वर्तेथास्तेषु पुत्रवत्}
{ते हि लोकानिमान्सर्वान्धारयन्ति मनीषिणः}


% Check verse!
ब्राह्मणाः सर्वलोकानां महान्तो धर्मसेतवः ॥धनत्यागाभिरामास्चि वाक्सङ्गमधुराश्च ये
\twolineshloka
{रमणीयाश्च भूतानां नियमेन धृतव्रताः}
{प्रणेतारश्च कोशानां शास्त्राणां च यशस्विनः}


\twolineshloka
{तपो येषां धनं नित्यं वाक्चैव विपुलं बलम्}
{प्रसवाश्चैव धर्माणां धर्मज्ञाः सूक्ष्मदर्शिनः}


\twolineshloka
{धर्मकामाः स्थिता धर्मे सुकृतैर्धर्मसेवतः}
{यान्समाश्रित्य तिष्ठन्ति प्रजाः सर्वाश्चतुर्विधाः}


\twolineshloka
{पन्थानः सर्वनेतारो यज्ञवाहाः सनातनाः}
{पितृपैतामहीं गुर्वीमुद्वहन्ति धुरं सदा}


\twolineshloka
{धुरि ये नावसीदन्ति विषमे सद्धया इव}
{पितृदेवातिथिमुखा हव्यकव्याग्रभोजिनः}


\twolineshloka
{भोजनादेव लोकांस्त्रींस्त्रायन्ते महतो भयात्}
{दीपः सर्वस्य लोकस्य चक्षुश्चक्षुष्मतामपि}


\twolineshloka
{सर्वशिल्पादिनिधयो निपुणाः सूक्ष्मदर्शिनः}
{गतिज्ञाः सर्वभूतानामध्यात्मगतिचिन्तकाः}


\twolineshloka
{आदिमध्यावसानानां ज्ञातारश्छिन्नसंशयाः}
{परावरविशेषज्ञा गन्तारः परमां गतिम्}


\twolineshloka
{विमुक्ता धूतपाप्मानो निर्द्वन्द्वा निष्परिग्रहाः}
{मानार्हा मानिता नित्यं ज्ञानविद्भिर्महात्मभिः}


\twolineshloka
{चन्दने मलपङ्के च भोजनेऽभोजने समाः}
{समं येषां दुकूलं च शाणक्षौमाजिनानि च}


\twolineshloka
{तिष्ठेयुरप्यभुञ्जाना बहूनि दिवसान्यपि}
{शोषयेयुश्च गात्राणि स्वाध्यायैः संयतेन्द्रियाः}


\twolineshloka
{अदैवं दैवतं कुर्युर्दैवतं चाप्यदैवतम्}
{लोकानन्यान्सृजेयुस्ते लोकपालांश्च कोपिताः}


\twolineshloka
{अपेयः सागरो येषामपि सापान्महात्मनाम्}
{येषां कोपाग्निरद्यापि दण्डके नोपशाम्यति}


\twolineshloka
{देवानामपि ये देवाः कारणं कारणस्य च}
{प्रमामस्य प्रमाणं च तस्मान्नाभिभवेद्बुधः}


\twolineshloka
{तेषां वृद्धाश्च बालाश्च सर्वे सन्मार्गदर्शिनः}
{तपोविद्याविशेषात्तु मानयन्ति परस्परम्}


\twolineshloka
{अविद्वान्ब्राह्मणो देवः पात्रं वै पावनं महत्}
{विद्वान्भूयस्तरो देवः पूर्णसागरसन्निभः}


\twolineshloka
{अविद्वांश्चैव विद्वांश्च ब्राह्मणो दैवतं महत्}
{प्रणीतश्चाप्रणीतश्च यथाऽग्निर्दैवतं महत्}


\twolineshloka
{श्मशाने ह्यपि तेजस्वी पावको नैव दुष्यति}
{हविर्यज्ञे च विधिवद्भूय एवाभिशोभते}


\twolineshloka
{एवं यद्यप्यनिष्टेषु वर्तते सर्वकर्मसु}
{सर्वथा ब्राह्मणो मान्यो दैवतं विद्धि तत्परम्}


\chapter{अध्यायः २५७}
\threelineshloka
{कां तु ब्राह्मणपूजायां व्युष्टिं दृष्ट्वा जनाधिप}
{कं वा धर्मोदयं मत्वा तानर्चसि महामते ॥भीष्म उवाच}
{}


\twolineshloka
{अत्राप्युदाहरन्तीममितिहासं पुरातनम्}
{पवनस्य च संवादमर्जुनस्य च भारत}


\twolineshloka
{सहस्रभुजभृच्छ्रीमान्कार्तवीर्योऽभवत्प्रभुः}
{अस्य लोकस्य सर्वस्य माहिष्मत्यां महाबलः}


\twolineshloka
{स तु रत्नाकरवतीं सप्तद्वीपां ससागराम्}
{शशास पृथिवीं सर्वां हैहयः सत्यविक्रमः}


\twolineshloka
{स्ववित्तं तेन दत्तं तु दत्तात्रेयाय कर्मणे}
{क्षत्रधर्मं पुरस्कृत्य विनयं श्रुतमेव च}


\twolineshloka
{आराधयामास च तं कृतवीर्यात्मजो मुनिम्}
{न्यमन्त्रयत संतुष्टो द्विजश्चैनं वरैस्त्रिभिः}


\twolineshloka
{स वरैश्छन्दितस्तेन नृपो वचनमब्रवीत्}
{सहस्रबाहुता मेऽस्तु यूपमध्ये ग्रहो यथा}


\twolineshloka
{मम बाहुसहस्रं तु पश्यन्तां सैनिका रणे}
{विक्रमेणि महीं कृत्स्नां जयेयं संशितव्रत}


\twolineshloka
{तां च धर्मेण सम्प्राप्य पालयेयमतन्द्रितः}
{चतुर्थं तु वरं याचे त्वामहं द्विजसत्तम}


\twolineshloka
{तं ममानुग्रहकृते दातुमर्हस्यनिन्दित}
{अनुशाशन्तु मां सन्तो मिथ्यावृत्तं त्वदाश्रयम्}


\twolineshloka
{इत्युक्तः स द्विजः प्राह तथास्त्विति नराधिपम्}
{एवं समभवंस्तस्य वरास्ते दीप्ततेजसः}


\twolineshloka
{गतः स रथमास्थाय ज्वलनार्कसमद्युतिम्}
{अब्रवीद्वीर्यसंमोहात्को वाऽस्ति सदृशो मम}


\twolineshloka
{धैर्यैर्वीर्यैर्यशःशौर्यौर्विक्रमेणौजसाऽपि वा}
{तद्वाक्यान्ते चान्तरिक्षे वागुवाचाशरीरिणी}


\threelineshloka
{न त्वं मूढ विजानीषे ब्राह्मणं क्षत्रियाद्वरम्}
{सहितो ब्राह्मणेनेह क्षत्रियः शास्ति वै प्रजाः ॥अर्जुन उवाच}
{}


\twolineshloka
{कुर्यां भूतानि तुष्टोऽहं क्रुद्धो नाशं तथा नये}
{कर्म्णा मनसा वाचा न मत्तोस्ति वरो द्विजः}


\twolineshloka
{पूर्वो ब्रह्मोत्तरो वादो द्वितीयः क्षत्रियोत्तरः}
{त्वयोक्तौ हेतुयुक्तौ तौ विशेषस्तत्र दृश्यते}


\twolineshloka
{ब्राह्मणाः संश्रिताः क्षत्रं न क्षत्रं ब्राह्मणाश्रितम्}
{श्रिता ब्रह्मोपधा विप्राः खादन्ति क्षत्रियान्भुवि}


\twolineshloka
{क्षत्रियेष्वाश्रितो धर्मः प्रजानां परिपालनम्}
{क्षत्राद्वृत्तिर्ब्राह्मणानां तैः कथं ब्राह्मणो वरः}


\twolineshloka
{सर्वभूतप्रधानांस्तान्भैक्षवृत्तीनहं सदा}
{आत्मसम्भावितान्विप्रान्स्थापयाम्यात्मनो वशे}


\twolineshloka
{कथितं त्वनयाऽसत्यं गायन्त्या कन्यया दिवि}
{विजेष्याम्यवशान्सर्वान्ब्राह्मणांश्चर्मवाससः}


\twolineshloka
{न च मां च्यावयेद्राष्ट्रात्त्रिषु लोकेषु कश्चन}
{देवो वा मानुषो वाऽपि तस्माज्ज्येष्ये द्विजानहम्}


\twolineshloka
{अद्य ब्रह्मोत्तरं लोकं करिष्ये क्षत्रियोत्तरम्}
{नहि मे संयुगे कश्चित्सोढुमुत्सहते बलम्}


\twolineshloka
{अर्जुनस्य वचः श्रुत्वा वित्रस्ताऽभून्निशाचरी}
{अथैनमन्तरिक्षस्थस्ततो वायुरभाषत}


\twolineshloka
{त्यजैनं कलुषं भावं ब्राह्मणेभ्यो नमस्कुरु}
{एतेषां कुर्वतः पापं राष्ट्रक्षोभो भविष्यति}


\twolineshloka
{अथ च त्वां महीपाल शमयिष्यन्ति वै द्विजाः}
{निरसिष्यन्ति ते राष्ट्राद्धतोत्साहा महाबलाः}


\threelineshloka
{तं राजा कस्त्वमित्याह ततस्तं प्राह मारुतः}
{वायुर्वै देवदूतोस्मि हितं त्वां प्रब्रवीम्यहम् ॥अर्जुन उवाच}
{}


\twolineshloka
{अहो त्वयाऽयं विप्रेषु भक्तिरागः प्रदर्शितः}
{यादृशं पृथिवीभूतं तादृशं ब्रूहि मे द्विजम्}


\twolineshloka
{वायोर्वा सदृशं किञ्चिद्ब्रूहि त्वं ब्राह्मणोत्तमम्}
{अपां वै सदृशं वह्नेः सूर्यस्य नभसोऽपि वा}


\chapter{अध्यायः २५८}
\twolineshloka
{शृणु मूढ गुणान्कांश्चिद्ब्राह्मणानां महात्मनाम्}
{ये त्वया कीर्तिता राजंस्तेभ्योऽथ ब्राह्मणो वरः}


\twolineshloka
{त्यक्त्वा महीत्वं भूमिस्तु स्पर्धया काश्यपस्य ह}
{नाशं जगाम तां विप्रो व्यस्तम्ययत कश्यपः}


\twolineshloka
{अक्षया ब्राह्मणा राजन्दिवि चेह च नित्यदा}
{अपिबत्तेजसा ह्यापः स्वयमेवाङ्गिराः पुरा}


\twolineshloka
{स ताः पिबञ्शीरमिव नातृप्यत महातपाः}
{अपूरयन्महौघेन महीं सर्वां च पार्थिव}


\twolineshloka
{तस्मिन्नहं च क्रुद्धे वै जगत्त्यक्त्वा ततो भयात्}
{व्यतिष्ठमग्निहोत्रे च चिरमङ्गिरसो भयात्}


\twolineshloka
{अथ शप्तश्च भगवान्गौतमेन पुरंदरः}
{अहल्यां कामयानो वै धर्मार्थं च न हिंसितः}


\twolineshloka
{तथा समुद्रो नृपते पूर्णो दृष्टश्च वारिणा}
{ब्राह्मणैरभिशप्तश्च बभूव लवणोदकः}


\twolineshloka
{सुवर्णवर्णो निर्धूमः सङ्गतोर्ध्वशिखः कविः}
{क्रुद्धेनाङ्गिरसा शप्तो गुणैरेतैर्विवर्जितः}


\twolineshloka
{महतश्चूर्णितान्पश्य ये हासन्त महोदधिम्}
{सुवर्णिधारिणा नित्यमवशप्ता द्विजातिना}


\twolineshloka
{सम्मतत्वं द्विजातिभ्यः श्रेष्ठं विद्धि नराधिप}
{गर्भस्थान्ब्राह्मणाञ्शश्वन्नमस्यति किल प्रभुः}


\twolineshloka
{दण्डकानां महद्राज्यं ब्राह्मणेन विनाशितम्}
{तालजङ्घं महाक्षत्रमौर्वेणैकेन नाशितम्}


\twolineshloka
{त्वया च विपुलं राज्यं बलं धर्मं श्रुतं तथा}
{दत्तात्रेयप्रसादेन प्रप्तं परमदुर्लभम्}


\twolineshloka
{अग्निं त्वं यजसे नित्यं कस्माद्ब्राह्मणमर्जन}
{स हि सर्वस्य लोकस्य हव्यवाट् किं न वेत्सि तम्}


\twolineshloka
{अथवा ब्राह्मणश्रेष्ठमनुभूतानुपालकम्}
{कर्तारं जवलोकस्य कस्माज्जानन्विमुह्यसे}


\twolineshloka
{तथा प्रजापतिर्ब्राह्मा अव्यक्तप्रभवोऽव्ययः}
{येनेदं विपुलं विश्वं जनितं स्थावरं चरम्}


\twolineshloka
{अण्डजातं तु ब्रह्माणं केचिदिच्छिन्त्यपण्डिताः}
{अण्डाद्भिन्नाद्बभुः शैला दिशोंऽभः पृथिवी दिवम्}


\twolineshloka
{दृष्टवानेतदेवं हि कथं जायेदजो हि सः}
{स्थानमाकाशमण्डं तु यस्माज्जातः पितामहः}


\twolineshloka
{तिष्ठेत्कथमिति ब्रूयान्न किञ्चिद्धि तदा भवेत्}
{अहङ्कार इति प्रोक्तः सर्वतेजोगतः प्रभुः}


\twolineshloka
{नास्त्यन्तमस्ति तु ब्रह्मा स राजा लोकभावनः}
{इत्युक्तः स तदा तूष्णीमभूद्वायुस्तमब्रवीत्}


\chapter{अध्यायः २५९}
\twolineshloka
{इमां भूमिं द्विजातिभ्यो दित्सुर्वै दक्षिणां पुरा}
{अङ्गो नाम नृपो राजंस्ततश्चिन्तां मही ययौ}


\twolineshloka
{धारिणीं सर्वभूतानामयं प्राप्य वरो नृपः}
{कथमिच्छति मां दातुं द्विजेभ्यो ब्रह्मणः सुताम्}


\twolineshloka
{साहं त्यक्त्वा गमिष्यामि भूमित्वं ब्रह्मणः पदम्}
{अयं सराष्ट्रो नृपतिर्मार्भूदिति ततोऽगमत्}


\twolineshloka
{ततस्तां कश्यपो दृष्ट्वा व्रजन्तीं पृथिवीं तदा}
{प्रविवेश महीं सद्यो युक्तात्मा सुसमाहितः}


\twolineshloka
{ऋद्धा सा सर्वतो जज्ञे तृणौषधिसमन्विता}
{धर्मोत्तरा नष्टभया भूमिरासीत्तमतो नृप}


\twolineshloka
{एवं वर्षसहस्राणि दिव्यानि विपुलव्रतः}
{त्रिंशतं कश्यपो राजन्भूमिरासीदतन्द्रितः}


\twolineshloka
{अथागम्य महाराजन्नमस्कृत्यि च कश्यपम्}
{पृथवी काश्यपी जज्ञे सुता तस्य महात्मनः}


\twolineshloka
{एष राजन्नीदृशो वै ब्राह्मणः कश्यपोऽभवत्}
{अन्यं प्रब्रूहि वा त्वं च कश्यपात्क्षत्रियं वरम्}


\twolineshloka
{तूष्णीं बभूव नृपतिः पवनस्त्वब्रवीद्वचः}
{शृणु राजन्नुचक्ष्यस्य जातस्याङ्गिरसे कुले}


\twolineshloka
{भद्रा सोमस्य दुहिता रूपेण परमा मता}
{तस्यास्तुल्यं पतिं सोम उचथ्यं समपश्यत}


\twolineshloka
{सा च तीव्रं तपस्तेपे महाभागा यशस्विनी}
{उचथ्यं तु महाभागं तत्कृते वरयत्तदा}


\twolineshloka
{तत आहूय चोचथ्यं ददामीति यशस्विनीम्}
{भार्यार्थे स च जग्राह विधिवद्भूरिदक्षिणः}


\twolineshloka
{तां त्वकामयत श्रीमान्वरुणः पूर्वमेव ह}
{स चागम्य वनप्रस्थं यमुनायां जहार ताम्}


\twolineshloka
{जलेश्वरस्तु हृत्वा तामनयस्त्वं पुरं प्रति}
{परमाद्भुतसङ्काशं षट्सहस्रशतह्रदम्}


\twolineshloka
{न हि रम्यतरं किञ्चित्तस्मादन्यत्पुरोत्तमम्}
{वासादैरप्सरोभिश्च दिव्यैः कामैश्च शोभितम्}


\twolineshloka
{तत्र देवस्तया सार्धं रेमे राजञ्जलेश्वरः}
{तदाख्यातमुचथ्याय ततः पत्न्यवमर्दनम्}


\twolineshloka
{तच्छ्रुत्वा नारदात्सर्वमुचथ्यो नारदं तदा}
{प्रोवाच गच्छ ब्रूहि त्वं वरुणं परुषं वचः}


\twolineshloka
{मद्वाक्यान्मुञ्च मे भार्यां कस्मात्तां हृतवानसि}
{लोकपालोसि लोकानां न लोकस्य विलोपकः}


\twolineshloka
{सोमेन दत्ता भार्या मे त्वया चापहृताऽद्य वै}
{इत्युक्तो वचनात्तस्य नारदेन जलेश्वरः}


\twolineshloka
{मुञ्च भार्यामुचथ्यस्य कस्मात्त्वं हृतवानसि}
{इति श्रुत्वा वचस्तस्य सोऽथ तं वरुणोऽब्रवीत्}


\threelineshloka
{ममैषा सुप्रिया भार्या नैनामुत्स्रष्टुमुत्सहे}
{इत्युक्तो वरुणेनाथ नारदः प्राप्य तं मुनिम्}
{उचथ्यमब्रवीद्वाक्यं नातिहृष्टमना इव}


\twolineshloka
{गले गृहीत्वा क्षिप्तोस्मि वरुणेन महामुने}
{न प्रयच्छति ते भार्यां यत्ते कार्यं कुरुष्व तत्}


\twolineshloka
{नारदस्य वचः श्रुत्वा क्रुद्धः प्राज्वलदङ्गिराः}
{अपिबत्तेजसा वारि विष्टभ्य सुमहातपाः}


\twolineshloka
{पीयमाने तु सर्वस्मिंस्तोयेऽपि सलिलेश्वरः}
{सुहृद्भिर्भिक्षमाणोऽपि नैवामुञ्चत तां तदा}


\twolineshloka
{ततः क्रुद्धोऽब्रवीद्भूमिमुचथ्यो ब्राह्मणोत्तमः}
{दर्शय स्वस्थलं भद्रे षट्सहस्रशतह्रदम्}


\twolineshloka
{ततस्तदीरणं जातं समुद्रस्यावसर्पतः}
{तस्माद्देशान्नदीं चैव प्रोवाचासौ द्विजोत्तमः}


\twolineshloka
{अदृश्या गच्छ भीरु त्वं सरस्वति मरून्प्रति}
{अणुण्यभूषो भवतु देशस्त्यक्तस्तया शुभे}


\twolineshloka
{ततश्चूर्णीकृते देशेक भद्रामादाय वारिपः}
{अददाच्छरणं गत्वा भार्यामाङ्गिरसाय वै}


\twolineshloka
{प्रतिगृह्य तु तां भार्यामुचथ्यः सुमनाऽभवत्}
{मुमोच च जगद्दुःखान्मरुतश्चैव निर्मलाः}


\twolineshloka
{ततः स लब्ध्वा तां भार्यां वरुणं प्राह धर्मवित्}
{उचथ्यः सुमहातेजा यत्तुच्छृणु नराधिप}


\twolineshloka
{मयैषा तपसा प्राप्ता क्रोशतस्ते जलाधिप}
{इत्युक्त्वा तामुपादाय स्वमेव भवनं ययौ}


\twolineshloka
{एष राजन्नीदृशो वै उचथ्यो ब्राह्मणर्षभः}
{ब्रवीमि हान्यं ब्रूहि त्वमुचथ्यात्क्षत्रियं वरम्}


\chapter{अध्यायः २६०}
\twolineshloka
{इत्युक्तः स नृपस्तूष्णीमभूद्वायुस्ततोऽब्रवीत्}
{शृणु राजन्नगस्त्यस्य महात्म्यं ब्राह्ममस्य ह}


\twolineshloka
{असुरैर्निर्जिता देवा निरुत्साहाश्च ते कृताः}
{यज्ञाश्चैषां हृताः सर्वे पितॄणां च स्वधास्तथा}


\twolineshloka
{कर्मज्या मानवानां च दानवैर्हैहयर्षभ}
{भ्रष्टैश्वर्यास्ततो देवाश्चेरुः पृथ्वीमिति श्रुतिः}


\twolineshloka
{ततः कदाचित्ते राजन्दीप्तमादित्यवर्चसम्}
{ददृशुस्तेजसाः युक्तमगस्त्यं विपुलव्रतम्}


\twolineshloka
{अभिवाद्य तु तं देवाः पृष्ट्वा कुशलमेव च}
{इदमूचुर्महात्मानं वाक्यं काले जनाधिप}


\twolineshloka
{दानवैर्युधि भग्नाः स्म तथैश्वर्याच्चि भ्रंशिताः}
{तदस्मान्नो भयात्तीव्रात्त्राहि त्वं मुनिपुङ्गवः}


\twolineshloka
{इत्युक्तः स तदा देवैरगस्त्यः कुपितोऽभवत्}
{प्रजज्वाल च तेजस्वी कालाग्निरिव संक्षये}


\twolineshloka
{तेन दीप्तांशुजालेन निर्दग्धा दानवास्तदा}
{अन्तरिक्षान्महाराज निपेतुस्ते सहस्रशः}


\twolineshloka
{दह्यमानास्तु ते दैत्यास्तस्यागस्त्यस्य तेजसा}
{उभौ लोकौ परित्यज्य गताः काष्ठां तु दक्षिणाम्}


\twolineshloka
{बलिस्तु यजते यज्ञमश्वमेधं महीं गतः}
{येन्येऽधस्ता महीस्थाश्च तेन दग्धा महासुराः}


\twolineshloka
{त्यक्तलोकाः पुनः प्राप्ताः सुरैः शान्तभयैर्नृप}
{अथैनमब्रुवन्देवा भूमिष्ठानसु राञ्जहि}


\twolineshloka
{इत्युक्तः प्राह देवान्स न शक्तोस्मि महीगतान्}
{दग्धुं तपो हि क्षीयेन्मे न धक्ष्यामीति पार्थिव}


\twolineshloka
{एवं दग्धा भगवता दानवाः स्वेन तेजसा}
{अगस्त्येन तदा राजंस्तपसा भावितात्मना}


\threelineshloka
{ईदृशश्चाप्यगस्त्यो हि कथितस्ते मयाऽनघ}
{ब्रवीम्यन्यं ब्रूहि वा त्वमगस्त्यात्क्षत्रियं वरम् ॥भीष्म उवाच}
{}


\twolineshloka
{इत्युक्तः स तदा तूष्णीमभूद्वायुस्ततोऽब्रवीत्}
{शृणु राजन्वसिष्ठस्य मुख्यं कर्म यशस्विनः}


\threelineshloka
{`वैखानसविधानेन गङ्गातीरं समाश्रिताः}
{'आदित्याः सत्रमासन्त सरो वैखानसं प्रति}
{वसिष्ठं मनसा गत्वा ज्ञात्वा तत्वस्य गोचरम्}


\twolineshloka
{यजमानांस्तु तान्दृष्वा सर्वान्दीक्षानुकर्शितान्}
{हन्तुमैच्छन्त शैलाभा बलिनो नाम दानवाः}


\twolineshloka
{अदूरात्तु ततस्तेषां ब्रह्मदत्तवरं सरः}
{हता हता वै तत्रैते जीवन्त्याप्लुत्य दानवाः}


\twolineshloka
{ते प्रगृह्य महाघोरान्पर्वतान्परिघान्द्रुमान्}
{विक्षोभयन्तः सलिलमुत्थितं शतयोजनम्}


\twolineshloka
{अभ्यद्रवन्त देवांस्ते सहस्राणि दशैव हि}
{ततस्तैरर्दिता देवाः शरणं वासवं ययुः}


\twolineshloka
{स च तैर्व्यथितः शक्रो वसिष्ठं सरणं ययौ}
{ततोऽभयं ददौ तेभ्यो वसिष्ठो भगवानृषिः}


\twolineshloka
{तदा तान्दुःकितान्ज्ञात्वा आनृशंस्यपरो मुनिः}
{अयत्नेनादहत्सर्वाञ्ज्वलता स्वेन तेजसा}


\twolineshloka
{कैलासं प्रस्थितां चैव नदीं गङ्गां महातपाः}
{आनयत्तत्सरो दिव्यं तया भिन्नं च तत्सरः}


\twolineshloka
{सरो भिन्नं तया नद्या सरयूः सा ततोऽभवत्}
{हताश्च बलिनो यत्र स देशे बलिनोऽभवत्}


\twolineshloka
{एवं सेन्द्रा वसिष्ठेन रक्षितास्त्रिदिवौकसः}
{ब्रह्मदत्तवराश्चैव हता दैत्या महात्मना}


\twolineshloka
{एतत्कर्म वसिष्ठस्य कथितं हि मयाऽनघ}
{ब्रवीम्यन्यं ब्रूहि वा त्वं वसिष्ठात्क्षत्रियं वरम् ॥ 26}


\chapter{अध्यायः २६१}
\twolineshloka
{इत्युक्तस्त्वर्जुनस्तूष्णीमभूद्वायुस्तमब्रवीत्}
{शृणु मे हैहयश्रेष्ठ कर्मात्रेः सुमहात्मनः}


\twolineshloka
{घोरे तमस्ययुध्यन्त सहिता देवदानवाः}
{अविद्यत शरैस्तत्र स्वर्भानुः सोमभास्करौ}


\twolineshloka
{अथ ते तमसा ग्रस्ता निहन्यन्ते स्म दानवैः}
{देवा नृपतिशार्दूल सहैव बलिभिस्तदा}


\twolineshloka
{असुरेर्वध्यमानास्ते क्षीणप्राणा दिवौकसः}
{अपश्यन्त तपस्यन्तमत्रिं विप्रं तपोधनम्}


\twolineshloka
{अथैनमब्रुवन्देवाः शान्तक्रोधं जितेन्द्रियम्}
{असुरेरिषुभिर्विद्धौ चन्द्रादित्याविमावुभौ}


\threelineshloka
{वयं वध्यामहे चापि शत्रुभिस्तमसा वृते}
{नाधिगच्छाम शान्तिं च भयात्त्रायस्व नः प्रभो ॥अत्रिरुवाच}
{}


\twolineshloka
{कथं रक्षामि भवतस्तेऽब्रुवंश्चन्द्रमा भव}
{तिमिरघ्नश्च सविता दस्युहन्ता च नो भव}


\twolineshloka
{एवमुक्तस्तदात्रिर्वै तमोनुदभवच्छशी}
{अपश्यत्सौम्यभावाच्च सोमवत्प्रियदर्शनः}


\twolineshloka
{दृष्ट्वा नातिप्रभं सोमं तथा सूर्यं च पार्थिव}
{प्रकाशमकरोदत्रिस्तपसा स्वेन संयुगे}


\twolineshloka
{जगद्वितिमिरं चापि प्रदीप्तमकरोत्तदा}
{व्यजयच्छत्रुसङ्घांश्च देवानां स्वेन तेजसा}


\threelineshloka
{अत्रिणा दह्यमानांस्तान्दृष्ट्वा देवा महासुरान्}
{पराक्रमैस्तेऽपिं तदा व्यघ्नन्नत्रिसुरक्षिताः}
{उद्भासितश्च सविता देवास्त्राता हतासुराः}


\twolineshloka
{अत्रिणा त्वथ सोमत्वं कृतमुत्तमतेजसा}
{द्विजेनाग्निद्वितीयेनि जपता चर्मवाससा}


\threelineshloka
{फलभक्षेण राजर्षे पश्य कर्मात्रिणा कृतम्}
{तस्यापि विस्तरेणोक्तं कर्मात्रेः सुमहात्मनः}
{ब्रवीम्यन्यं ब्रूहि वा त्वमत्रितः क्षत्रियं वरम्}


\twolineshloka
{इत्युक्तस्त्वर्जुनस्तूष्णीमभूद्वायुस्ततोऽब्रवीत्}
{शृणु राजन्महत्कर्म च्यवनस्य महात्मनः}


\threelineshloka
{अश्विनोः प्रतिसंश्रुत्य च्यवनः पाकशासनम्}
{प्रोवाच सहितो देवैः सोमपावश्विनौ कुरु ॥इन्द्र उवाच}
{}


\twolineshloka
{अस्माभिर्निन्दितावेतौ भवेतां सोमपौ कथम्}
{देवैर्न सम्मितावेतौ तस्मान्मैवं वदस्व नः}


\threelineshloka
{अश्विभ्यां सह नेच्छामः सोमं पातुं महाव्रत}
{यदन्यद्वक्ष्यसे विप्र तत्करिष्याम ते वचः ॥च्यवन उवाच}
{}


\twolineshloka
{पिबेतामश्विनौ सोमं भवद्भिः सहिताविमौ}
{उभावेतावपि सुरौ सूर्यपुत्रौ सुरेश्वर}


\threelineshloka
{क्रियतां मद्वचो देवा यथा वै समुदाहृतम्}
{एतद्वः कुर्वतां श्रेयो भवेन्नैतदकुर्वताम् ॥इन्द्र उवाच}
{}


\threelineshloka
{अश्विभ्यां सह सोमं वै न पास्यामि द्विजोत्तम}
{पिबन्त्वन्ये यथाकामं नाहं पातुमिहोत्सहे ॥च्यवन उवाच}
{}


\threelineshloka
{न चेत्करिष्यसि वचो मयोक्तं बलसूदन}
{मया प्रमथितः सद्यः सोमं पास्यसि वै मखे ॥वायुरुवाच}
{}


\twolineshloka
{ततः कर्म समारब्धं हिताय सहसाऽश्विनोः}
{च्यवनेनि ततो मन्त्रैरभिभूताः सुराऽभवन्}


\twolineshloka
{तत्तु कर्म समारब्धं दृष्ट्वेन्द्रः क्रोधमूर्च्छितः}
{उद्यम्य विपुलं शैलं च्यवनं समुपाद्रवत्}


\twolineshloka
{तथा वज्रेम भगवानमर्षाकुललोचनः}
{तमापतन्तं दृष्ट्वैव च्यवनस्तपसाऽन्वितः}


\twolineshloka
{अद्भिः सिक्त्वाऽस्तम्ययतं सवज्रं सहपर्वतम्}
{अथेन्द्रस्य महाघोरं सोऽसृजच्छत्रुमेव हि}


\twolineshloka
{मदं नामाहुतिमयं व्यादितास्यं महामुनिः}
{तस्य दन्तसहस्रं तु बभूव शतयोजनम्}


\twolineshloka
{द्वियोजनशतास्तस्य दंष्ट्राः परमदारुणाः}
{हनुस्तस्याभवद्भूमावास्यं चास्यास्पृशद्दिवम्}


\twolineshloka
{जिह्वामूले स्थितास्तस्य सर्वे देवाः सवासवाः}
{तिमेरास्यमनुप्राप्ता यथा मत्स्या महार्णवे}


\twolineshloka
{ते सम्मन्त्र्य ततो देवा मदस्यास्य समीपगाः}
{अब्रुवन्सहिताः शक्रं प्रणमास्मै द्विजातये}


\twolineshloka
{अश्विभ्यां सह सोमं च पिबाम विगतज्वराः}
{ततः स प्रणतः शक्रश्चकार च्यवनस्य तत्}


\twolineshloka
{च्यवनः कृतवानेतावश्विनौ सोमपायिनौ}
{ततः प्रत्याहरत्कर्म मदं च व्यभजन्मुनिः}


\twolineshloka
{अक्षेषु मृगयायां च पाने स्त्रीषु च वीर्यवान्}
{एतैर्दोषैर्नरा राजन्क्षयं यान्ति न संशयः}


% Check verse!
तस्मादेतान्नरो नित्यं दूरतः परिवर्जयेतद्
\twolineshloka
{एतत्ते च्यवनस्यापि कर्मि राजन्प्रकीर्तितम्}
{ब्रवीम्यन्यं ब्रूहि वा त्वं क्षत्रियं ब्राह्मणाद्वरम्}


\chapter{अध्यायः २६२}
\twolineshloka
{तूष्णीमासीदर्जुनस्तु पवनस्त्वब्रवीत्पुनः}
{शृणु मे ब्राह्मणेष्वेव मुख्यं कर्म जनाधिप}


\twolineshloka
{मदस्यास्यमनुप्राप्ता यदा सेन्द्रा दिवौकसः}
{तदैव च्यवनेन द्यौर्हृता तेषां वसुन्धरा}


\threelineshloka
{उभौ लोकौ हृतौ मत्वा ते देवा दुःखिताऽभवन्}
{शोकार्ताश्त महात्मानं ब्रह्माणं शरणं ययुः ॥देवा ऊचुः}
{}


\threelineshloka
{मदास्यव्यतिषिक्तानामस्माकं लोकपूजित}
{च्यवनेन हृता भूमिः कपैश्चैव दिवं प्रभो ॥ब्रह्मोवाच}
{}


\twolineshloka
{गच्छध्वं शरणं विप्रानाशु सेन्द्रा दिवौकसः}
{प्रसाद्य तानुभौ लोकाववाप्स्यथ यथापुरम्}


\twolineshloka
{ते ययुः शरणं विप्रानूचुस्ते काञ्जयामहे}
{इत्युक्तास्ते द्विजान्प्राहुर्जयतेह कपानिति}


\twolineshloka
{भूगतान्हि विजेतारो वयमित्यब्रुवन्द्विजाः}
{ततः कर्म समारब्धं ब्राह्मणैः कपनाशनम्}


\twolineshloka
{तच्छ्रुत्वा प्रेषितो दूतो ब्राह्मणेभ्यो धनी कपैः}
{स च तान्ब्राह्मणानाह धनी कपवचो यथा}


\twolineshloka
{भवद्भिः सदृशः सर्वे कपाः किमिह वर्तते}
{सर्वे वेदविदः प्राज्ञाः सर्वे च क्रतुयाजिनः}


\twolineshloka
{सर्वे सत्यव्रताश्चैव सर्वे तुल्या महर्षिभिः}
{श्रीश्चैव रमते तेषु धारयन्ति श्रियं च ते}


\twolineshloka
{वृथा दारान्न गच्छन्ति वृथा मांसं न भुञ्जते}
{दीप्तमग्निं जुह्वते च गुरूणां वचने स्थिताः}


\threelineshloka
{सर्वे च नियतात्मानो बालानां संविभागिनः}
{उपेत्य शनकैर्यान्ति न सेवन्ति रजस्वलाम्}
{स्वर्गातिं चैव गच्छन्ति तथैव शुभकर्मिणः}


\twolineshloka
{अभुक्तवत्सु नाश्नन्ति गर्भिणीवृद्धकादिषु}
{पूर्वाह्णेषु न दीव्यन्ति दिवा चैव न शेरते}


\threelineshloka
{एतैश्चान्यैश्च बहुभिर्गुणैर्युक्तान्कथं कपान्}
{विजेष्यथ निवर्तध्वं निवृत्तानां सुखं हि वः ॥ब्राह्मणा ऊचुः}
{}


\twolineshloka
{कपान्वयं विजेष्यामो ये देवास्ते वयं स्मृताः}
{तस्माद्वध्याः कपाऽस्माकं धनिन्याहि यथागतम्}


\twolineshloka
{धनी गत्वा कपानाह न वो विप्राः प्रियङ्कराः}
{गृहीत्वाऽस्त्राण्यतो विप्रान्कपाः सर्वे समाद्रवन्}


\twolineshloka
{समुदग्रध्वजान्दृष्ट्वा कपान्सर्वे द्विजातयः}
{व्यसृजञ्ज्वलितानग्नीन्कपानां प्राणनाशनान्}


\twolineshloka
{ब्रह्मसृष्टा हव्यभुजः कपान्हत्वा सनातनाः}
{नभसीव यथाऽभ्राणि व्यराजन्त नराधिप}


\twolineshloka
{हत्वा वै दानवान्देवाः सर्वे सम्भूय संयुगे}
{ते नाभ्यजानन्हि तदा ब्राह्मणैर्निहतान्कपान्}


\twolineshloka
{अथागम्य महातेजा नारदोऽकथयद्विभो}
{यथा हता महाभागैस्तेजसा ब्राह्मणैः कपाः}


\twolineshloka
{नारदस्य वचः श्रुत्वा प्रीताः सर्वे दिवौकसः}
{प्रशशंसुर्द्विजांश्चापि ब्राह्मणांश्च यशस्विनः}


\twolineshloka
{तेषां तेजस्तथा वीर्यं देवानां ववृधे ततः}
{अवाप्नुवंश्चामरत्वं त्रिषु लोकेषु पूजितम्}


\threelineshloka
{इत्युक्तवचनं वायुमर्जुनः प्रत्युवाच ह}
{प्रतिपूज्य महाबाहो यत्तच्छृणु नराधिप ॥अर्जुन उवाच}
{}


\twolineshloka
{जीवाम्यहं ब्राह्मणार्थं सर्वथा सततं प्रभो}
{ब्रह्मण्यो ब्राह्मणेभ्यश्च प्रणमामि च नित्यशः}


\twolineshloka
{दत्तात्रेयप्रसादाच्च मया प्राप्तमिदं बलम्}
{लोके च परमा कीर्तिर्धर्मश्चाचरितो महान्}


\threelineshloka
{अहो ब्राह्मणकर्माणि मयि मारुत तत्त्वतः}
{त्वया प्रोक्तानि कार्त्स्न्येन श्रुतानि प्रयतेन च ॥वायुरुवाच}
{}


\twolineshloka
{ब्राह्मणान्क्षात्राधर्मेण पालयस्वेन्द्रियाणि च}
{विप्रेभ्यस्ते भयं घोरं तत्तु कालाद्भविष्यति}


\chapter{अध्यायः २६३}
\twolineshloka
{ब्राह्मणानर्चसे राजन्सततं संशितव्रतान्}
{कं तु धर्मोदयं दृष्ट्वा तानर्चसि जनाधिप}


\threelineshloka
{कां वा ब्राह्मणपूजायां व्युष्टिं दृष्ट्वा महाव्रत}
{तानर्चसि महाबाहो सर्वमेतद्वदस्व मे ॥भीष्म उवाच}
{}


\twolineshloka
{एष ते केशवः सर्वमाख्यास्यति महामतिः}
{व्युष्टिं ब्राह्मणपूजायां दृष्ट्वा व्युष्टिं महाव्रत}


\twolineshloka
{बलं श्रोत्रे वाङ्मनश्चक्षुषी चज्ञानं तथा नविशुद्धं ममाद्य}
{देहन्यासो नातिचिरान्मतो मेन चातितूर्णं सविताऽद्य याति}


\twolineshloka
{उक्ता धर्मा ये पुराणे महान्तोराजन्विप्राणां क्षत्रियाणां विशां च}
{येये शूद्राणां धर्ममुपासते तेतानेव कृष्णदुपशिक्षस्व पार्थः}


\twolineshloka
{अहं ह्येनं वेद्मि तत्त्वेन कृष्णंयोऽयं हि यच्चास्य बलं पुराणम्}
{अमेयात्मा केशवः कौरवेन्द्रसोयं धर्मं वक्ष्यति सर्वमेतत्}


\twolineshloka
{कृष्णः पृथ्वीमसृजत्स्वं दिवं चकृष्णस्यि देहान्मेदिनी सम्बभूव}
{वराहोऽयं भीमबलः पुराणःस पर्वतान्व्यसृजद्वै दिशश्च}


\twolineshloka
{अस्माद्वायुश्चान्तरिक्षं दिवं चदिशश्चतस्रो विदिशश्चतस्रः}
{सृष्टिस्तथैवेयमनुप्रसूतास निर्ममे विश्वमिदं पुराणः}


\twolineshloka
{अस्य नाभ्यां पुष्करं सम्प्रसूतंयत्रोत्पन्नः स्वयमेवामितौजाः}
{येनाच्छिन्नं तत्तमः पार्थ घोरंयत्र तिष्ठन्त्यर्णवास्तच्छयानाः}


\twolineshloka
{कृते युगे धर्म आसीत्समग्र-स्त्रेताकाले यज्ञमनुप्रपन्नः}
{बलं त्वासीद्द्वापरे पार्थ कृष्णःकलौ त्वधर्मः क्षितिमेवाजगाम}


\twolineshloka
{स एव पूर्वं निजघान दैत्या-न्स एव देवश्च बभूव सम्राट्}
{स भूतानां भावनो भूतभव्यःस विश्वस्यास्य जगतश्चाभिगोप्ताः}


\twolineshloka
{यदा धर्मो ग्लाति वंशे सुराणांतदा कृष्णो जायते मानुषेषु}
{धर्मे स्थित्वा स तु वै भावितात्मापरांश्च लोकानपराश्च पाति}


\twolineshloka
{त्याज्यांस्त्यक्त्वा चासुराणां वधेनकार्याकार्ये कारणं चैव पाति}
{कृतं करिष्यत्क्रियते च देवोराहुं सोमं विद्धि च शक्रमेनम्}


\twolineshloka
{स विश्वकर्मा स हि विश्वरूपःस विश्वबुग्विश्वसृग्विश्वजिच्च}
{स शूलभृच्छोणितभृत्कराल-स्तं कर्मभिर्विदितं वै स्तुवन्ति}


\twolineshloka
{तं गन्धर्वाणामप्सरसां च नित्य-मुपतिष्ठन्ते विबुधानां शतानि}
{तं राक्षसाश्च परिसंवदन्तिराजन्यानां स विजिगीषुरेकः}


\twolineshloka
{तमध्वरे शंसितारः स्तुवन्तिरथन्तरे सामगाश्च स्तुवन्ति}
{तं ब्राह्मणा ब्रह्ममन्त्रैः स्तुवन्तितस्मै हविरध्वर्यवः कल्पयन्ति}


\twolineshloka
{स पौराणीं ब्रह्मगुहां प्रविष्टोमहीसत्रं भारताग्रे ददर्श}
{स चैव गामुद्दधाराग्र्यकर्माविक्षोभ्य दैत्यानुरगान्दानवांश्च}


\twolineshloka
{तं घोषार्थे गीर्भिरिन्द्राः स्तुवन्तिस चापीशो भारतैकः पशूनाम्}
{तस्य भक्षान्विविधान्वेदयन्तितमेवाजौ वाहनं वेदयन्ति}


\twolineshloka
{तस्यान्तरिक्षं पृथिवी दिवं चसर्वं वशे तिष्ठति शाश्वतस्य}
{स कुम्भे रेतः ससृजे सुराणांयत्रोत्पन्नमृषिमाहुर्वसिष्ठम्}


\twolineshloka
{स मातरिश्वा विभुरश्ववाजीस रश्मिवान्सविता चादिदेवः}
{तेनासुरा विजिताः सर्व एवतद्विक्रान्तैर्विजितानीह त्रीणि}


\twolineshloka
{स देवानां मानुषाणां पितॄणांतमेवाहुर्यज्ञविदां वितानम्}
{स एव कालं विभजन्नुदेतितस्योत्तरं दक्षिणं चायने द्वे}


\twolineshloka
{तस्यैवोर्ध्वं तिर्यगधश्चरन्तिगभस्तयो मेदिनीं भासयन्तः}
{तं ब्राह्मणा वेदविदो जुषन्तितस्यादित्यो गामुपयुज्य भाति}


\twolineshloka
{स मासिमास्यध्वरकृद्विधत्तेतमध्वरे वेदविदः पठन्ति}
{स एवोक्तश्चक्रमिदं त्रिनाभिसप्ताश्वयुक्तं वहते वै त्रिधामा}


\threelineshloka
{`हिरण्मयः सप्तगूढः ससंवि-च्चतुर्बाहुः पन्नगः पद्मनाभः}
{'महातेजाः सर्वगः सर्वसिंहःकृष्णो लोकान्धारयते यथैकः}
{हंसं तमोघ्नं च तमेव वीरकृष्णं सदा पार्थ कर्तारमेहि}


\twolineshloka
{स एकदा कक्षगतो महात्मातुष्टो विभुः खाण्डवे धूमकेतुः}
{स राक्षसानुरगांश्चावजित्यसर्वत्रगः सर्वमग्नौ जुहोति}


\twolineshloka
{स एव पार्थाय श्वेतमश्वं प्रायच्छ-त्स एवाश्वानथ सर्वांश्चकार}
{सबन्धुरस्तस्य रथस्त्रिचक्र-स्त्रिवृच्छिराश्चतुरश्वस्त्रिनाभिः}


\twolineshloka
{स विहायो व्यदधात्पञ्चनाभिःस निर्ममे गां दिवमन्तरिक्षम्}
{सोऽरण्यानि व्यसृजत्पर्वतांश्चहृषीकेशोऽमितदीप्ताग्नितेजाः}


\twolineshloka
{अलङ्घयद्वै सरितो जिघांस-ञ्शक्रं वज्रं प्रहरन्तं निरास}
{स महेन्द्रः स्तूयते वै महाध्वरेविप्रैरेको ऋक्सहस्रैः पुराणैः}


\twolineshloka
{दुर्वासा वै तेन नान्येन शक्योगृहे राजन्वासयितुं महौजाः}
{तमेवाहुर्ऋषिमेकं पुराणंस विश्वकृद्विदधात्यात्मभावान्}


\twolineshloka
{वेदांश्च यो वेदयतेऽधिदेवोविधींश्च यश्चाश्रयते पुराणान्}
{कामे वेदे लौकिके यत्फलं चविष्वक्सेनः सर्वमेतत्प्रतीहि}


\twolineshloka
{ज्योतींषि शुक्लानि हि सर्वलोकेत्रयो लोका लोकपालास्त्रयश्च}
{त्रयोऽग्रयो व्याहृतयश्च तिस्रःसर्वे देवा देवकीपुत्र एव}


\twolineshloka
{स क्त्सरः स ऋतुः सोऽर्धमासःसोऽहोरात्रः स कला वै स काष्ठाः}
{मात्रा मुहूर्ताश्च लवाः क्षणाश्चविष्वक्सेनः सर्वमेतत्प्रतीहि}


\threelineshloka
{चन्द्रादित्यौ ग्रहनक्षत्रताराःसर्वाणि दर्शान्यथ पौर्णमासम्}
{चन्द्रादित्यौ ग्रहनक्षत्रताराःसर्वाणि दर्शान्यथ पौर्णमासम्}
{}


\twolineshloka
{नक्षत्रयोगा ऋतवश्च पार्थविष्वक्सेनात्सर्वमेतत्प्रसूतम् ॥रुद्रादित्या वसवोऽथाश्विनौ चसाध्याश्च विश्वे मरुतां गणाश्च}
{प्रजापतिर्देवमाताऽदितिश्चसर्व कृष्णादृषयश्चैव सप्त}


\twolineshloka
{वायुर्भूत्वा विक्षिपते च विश्व-मग्निर्भूत्वा दहते विश्वरूपः}
{आपो भूत्वा मज्जयते च सर्वंब्रह्म भूत्वा सृजते विश्वसङ्घान्}


\twolineshloka
{वेद्यं च यद्वेदयते च वेद्यंविधिश्च यश्च श्रयते विधेयम्}
{धर्मे च वेदे च बले च सर्वंचराचरं केशवं त्वं प्रतीहि}


\twolineshloka
{ज्योतिर्भूतः परमोसौ पुरस्ता-त्प्रकाशते यत्प्रभया विश्वरूपः}
{अपः सृष्ट्वा सर्वभूतात्मयोनिःपुराऽकरोत्सर्वमेवाथ विश्वम्}


\twolineshloka
{ऋतूनुत्पातान्विविधान्यद्भुतानिमेघान्विद्युत्सर्वमैरावतं च}
{सर्वं कृष्णात्स्थावरं जङ्गं चविश्वात्मानं विष्णुमेनं प्रतीहि}


\twolineshloka
{विश्वावासं निर्गुणं वासुदेवंसङ्कर्षणं जीवभूतं वदन्ति}
{ततः प्रद्युम्नमनिरुद्धं चतुर्थ-माज्ञापयत्यात्मयोनिर्महात्मा}


\twolineshloka
{स पञ्चधा पञ्चगुणोपपन्नंसञ्चोदयन्विश्वमिदं सिसृक्षुः}
{ततश्चकारावनिमारुतौ चखं ज्योतिरम्भश्चि तथैव पार्थ}


\twolineshloka
{स स्थावरं जङ्गमं चैवमेत-च्चतुर्विधं लोकमिमं च कृत्वा}
{ततो भूमिं व्यदधात्पञ्चबीजांद्यावापृथिव्यग्निरथाम्बुवायू}


\threelineshloka
{तेन विश्वं कृतमेतद्धि राज-न्स जीवयत्यात्मनैवात्मयोनिः}
{ततो देवानसुरान्मानवांश्चलोकानृषींश्चापि पितॄन्प्रजाश्च}
{समासेनि विविधान्पाति लोका-न्सर्वान्सदा भूतपतिः सिसृक्षुः}


\twolineshloka
{शुभाशुभं स्थावरं जङ्गमं चविष्वक्सेनात्सर्वमेतत्प्रतीहि}
{यद्वर्तते यच्च भविष्यतीहसर्वं ह्येतत्केशवं त्वं प्रतीहि}


\twolineshloka
{मृत्युश्चैव प्राणिनामन्तकालेसाक्षात्कृष्णः केशवो देहभाजाम्}
{भूतं च यच्चेह न विद्म किञ्चि-द्विष्वक्सेनात्सर्वमेतत्प्रतीहि}


\twolineshloka
{यत्प्रशस्तं च लोकेषु पुण्यं यच्च शुभाशुभम्}
{तत्सर्वं केशवोऽचिन्त्यो विपरीतमतः परम्}


\twolineshloka
{एतादृशः केशवोऽतश्च भूयोनारायणः परमश्चाव्ययश्च}
{मध्याद्यन्तस्य जगतस्तस्तुषश्चबुभूषतां प्रभवश्चाव्ययश्च}


\chapter{अध्यायः २६४}
\threelineshloka
{ब्रूहि ब्राह्मणपूजायां व्युष्टिं त्वं मधुसूदन}
{वेत्ता त्वमस्य चार्थस्य वेद त्वां हि पितामहः ॥वासुदेव उवाच}
{}


\twolineshloka
{शृणुष्वावहितो राजन्द्विजानां भरतर्षभ}
{यथातत्त्वेन वदतो गुणान्वै कुरुसत्तम}


\twolineshloka
{द्वारवत्यां समासीनं पुरा मां कुरुनन्दन}
{प्रद्युम्नः परिपप्रच्छ ब्राह्मणैः परिकोपितः}


\twolineshloka
{किं फलं ब्राह्मणेष्वस्ति पूजायां मधुसूदन}
{ईश्वरत्वं कुतस्तेषामिहैव च परत्र च}


\twolineshloka
{सदा द्विजातीन्सम्पूज्य किं फलं तत्र मानद}
{एतद्ब्रूहि स्फुटं सर्वं सुमहान्संशयोऽत्र मे}


\twolineshloka
{इत्युक्ते वचने तस्मिन्प्रद्युम्नेन तथा त्वहम्}
{प्रत्यब्रवं महाराज यत्तच्छृणु समाहितः}


\twolineshloka
{व्युष्टिं ब्राह्मणपूजायां रौक्मिणेय निबोध मे}
{एते हि सोमराजान ईश्वराः सुखदुःखयोः}


\twolineshloka
{अस्मिँल्लोके रौक्मिणेय तथाऽमुष्मिंश्च पुत्रक}
{ब्राह्मणप्रमुखं सौम्यं न मेऽत्रास्ति विचारणा}


\twolineshloka
{ब्राह्मणप्रभवं सौख्यमायुः कीर्तिर्यशो बलम्}
{लोका लोकेश्वराश्चैव सर्वे ब्राह्मणपूजकाः}


\twolineshloka
{त्रिवर्गे चापवर्गे च यशःश्रीरोगशान्तिषु}
{देवतापितृपूजासु सन्तोष्याश्चैव नो द्विजाः}


\twolineshloka
{तान्कथं वै नाद्रियेयमीश्वरोस्मीति पुत्रक}
{मा ते मन्युर्महाबाहो भवत्वत्र द्विजान्प्रति}


\twolineshloka
{ब्राह्मणा हि महद्भूतमस्मिँल्लोके परत्र च}
{भस्म कुर्यर्जगदिदं क्रुद्धाः प्रत्यक्षदर्शिनः}


\twolineshloka
{हन्युस्तेऽपि सृजेयुस्च लोकान्लोकेश्वरांस्तथा}
{कथं तेषु न वर्तेरन्सम्यग्ज्ञानात्सुतेजसः}


\twolineshloka
{अवसन्मद्गृहे तात ब्राह्मणो हरिपिङ्गलः}
{चीरवासा बिल्वदण्डी दीर्घश्मश्रुः कृशो महान्}


\twolineshloka
{दीर्घभ्यश्च मनुष्येभ्यः प्रमाणादधिको भुवि}
{स स्वैरं चरते लोकान्ये दिव्या ये च मानुषाः}


\twolineshloka
{इमां गाथां गायमानश्चत्वरेषु सभासु च}
{दुर्वाससं वासयेत्को ब्राह्मणं सत्कृतं गृहे}


\twolineshloka
{रोषणः सर्वभूतानां सूक्ष्मेऽप्यपकृते कृते}
{परिभाषां च मे श्रुत्वा को नु दद्यात्प्रतिश्रयम्}


\twolineshloka
{यो मां कश्चिद्वासयीत न स मां कोपयेदिति}
{यस्मान्नाद्रियते कश्चित्ततोऽहं समवासयम्}


\twolineshloka
{स सम्भुङ्क्ते सहस्राणां बहूनामन्नमेकदा}
{एकदा सोल्पकं भुङ्क्ते नचैवैति पुनर्गृहान्}


\twolineshloka
{अकस्माच्च प्रहसति तथाऽकस्मात्प्ररोदिति}
{न चास्य वयसा तुल्यः पृथिव्यामभवत्तदा}


\twolineshloka
{अथ स्वावसथं गत्वा सशय्यास्तरणानि च}
{अदहत्स महातेजास्ततश्चाभ्यपतत्स्वयम्}


\twolineshloka
{अथ मामब्रवीद्भूयः स मुनिः संशितव्रतः}
{कृष्ण पायसमिच्छामि भोक्तुमित्येव सत्वरः}


\twolineshloka
{तदैव तु मया तस्य चित्तज्ञेन गृहे जनः}
{सर्वाण्यन्नानि पानानि भक्ष्याश्चोच्चावचास्तथा}


\twolineshloka
{भवन्तु सत्कृतानीह पूर्वमेव प्रयोचितः}
{ततोऽहं ज्वलमानं वै पायसं प्रत्यवेदयम्}


\twolineshloka
{तं भुक्त्वैव स तु क्षिप्रं ततो वचनमब्रवीत्}
{क्षिप्रमङ्गानि लिम्पस्व पायसेनेति स स्म ह}


\twolineshloka
{अविमृश्यैव च ततः कृतवानस्मि तत्तथा}
{तेनोच्छिष्टेन गात्राणि शरीरं च समालिपम्}


\twolineshloka
{स ददर्श तदाऽभ्याशे मातरं ते शुभाननाम्}
{तामपि स्मयमानां स पायसेनाभ्यलेपयत्}


\twolineshloka
{मुनिः पायसदिग्धाङ्गीं रथे तूर्णमयोजयत्}
{तमारुद्य रथं चैव निर्ययौ स गृहान्मम}


\twolineshloka
{अग्निवर्णो ज्वलन्धीमान्स द्विजो रथधुर्यवत्}
{प्रतोदेनातुदद्बालां रुक्मिणीं मम पश्यतः}


\twolineshloka
{न च मे स्तोकमप्यासीद्दुः खमीर्ष्याकृतं तदा}
{तथा स राजमार्गेणि महता निर्ययौ बहिः}


\twolineshloka
{तद्दृष्ट्वा महदाश्चर्यं दाशार्हा जातमन्यवः}
{तत्राजल्पन्मिथः केचित्समाभाष्य परस्परम्}


\twolineshloka
{ब्राह्मणा एव जायेरन्नान्यो वर्णः कथञ्चन}
{को ह्येनां रथमास्थाय जीवेदन्य पुमानिह}


\twolineshloka
{आशीविषविषं तीक्ष्णं ततस्तीक्ष्णतरो द्विजः}
{ब्रह्माहिविषदिग्धस्य नास्ति कश्चिच्चिकित्सकः}


\twolineshloka
{तस्मिन्व्रजति दुर्धर्षे प्रास्खलद्रुक्मिणी पथि}
{अमर्षयंस्तथा श्रीमान्स्मितपूर्वमचोदयम्}


\twolineshloka
{ततः परमसंक्रुद्धो रथात्प्रस्कन्द्य स द्विजः}
{पदातिरुत्पथेनैव प्राद्रवद्दक्षिणामुखः}


\twolineshloka
{तमुत्पथेन धावन्तमन्वधावं द्विजोत्तमम्}
{तथैव पायसादिग्धः प्रसीद भगवन्निति}


\twolineshloka
{ततो विलोक्य तेजस्वी ब्राह्मणो मामुवाच ह}
{जितः क्रोधस्त्वया कृष्ण प्रकृत्यैव महाभुज}


\twolineshloka
{न तेऽपराधमिह वै दृष्टवानस्मि सुव्रत}
{प्रीतोस्मि तव गोविन्द वृणु कामान्यथेप्सितान्}


% Check verse!
प्रसन्नस्य च मे तात पश्य व्युष्टिं यथाविधाम्
\twolineshloka
{यावदेव मनुष्याणामन्ने भावो भविष्यति}
{यथैवान्ने तथा तेषां त्वयि भावो भविष्यति}


\threelineshloka
{यावच्च पुण्या लोकेषु त्वयि कीर्तिर्भविष्यति}
{त्रिषु लोकेषु तावच्च वैशिष्ट्यं प्रतिपत्स्यसे}
{सुप्रियः सर्वलोकस्य भविष्यसि जनार्दन}


\twolineshloka
{यत्ते भिन्नं च दग्धं च यच्च किञ्चिद्विनाशितम्}
{सर्वं तथैव द्रष्टासि विशिष्टं जनार्दन}


\twolineshloka
{यावदेतत्प्रलिप्तं ते गात्रेषु मधुसूदन}
{अतो मृत्युभयं नास्ति यावदिच्छसि चाच्युत}


\twolineshloka
{न तु पादतले लिप्ते तस्मात्ते मृत्युरत्र वै}
{नैतन्मे प्रियमित्येवं स मां प्रीतोऽब्रवीत्तदा}


\twolineshloka
{इत्युक्तोऽहं शरीरं स्वं ददर्श श्रीसमायुतम्}
{}


\twolineshloka
{रुक्मिणीं चाब्रवीत्प्रीतः सर्वस्त्रीणां वरं यशः}
{कीर्तिं चानुत्तमां लोके समवाप्स्यसि शोभने}


\twolineshloka
{न त्वां जरा वा रोगो वा वैवर्ण्यं चापि भामिनि}
{स्प्रक्ष्यन्ति पुण्यगन्धा च कृष्णमाराधयिष्यसि}


\twolineshloka
{षोडशानां सहस्राणां बधूनां केशवस्य ह}
{वरिष्ठा च सलोक्या च केशवस्य भविष्यसि}


\twolineshloka
{तव मातरमित्युक्त्वा ततो मां पुनरब्रवीत्}
{प्रस्थितः सुमहातेजा दुर्वासाऽग्निरिव ज्वलन्}


\twolineshloka
{एषैव ते बुद्धिरस्तु ब्राह्मणान्प्रति केशव}
{इत्युक्त्वा स तदा पुत्र तत्रैवान्तरधीयत}


\twolineshloka
{तस्मिन्नन्तर्हिते चाहमुपांशु व्रतमाचरम्}
{यत्किञ्चिद्ब्राह्मणो ब्रूयात्सर्वं कुर्यामिति प्रभो}


\twolineshloka
{एतद्व्रतमहं कृत्वा मात्रा ते सह पुत्रक}
{ततः परमहृष्टात्मा प्राविशं गृहमेव च}


\twolineshloka
{प्रविष्टमात्रश्च गृहे सर्वं पश्यामि तन्नवम्}
{यद्भिन्नं यच्च वै दग्धं तेन विप्रेण पुत्रक}


\twolineshloka
{ततोऽहं विस्मयं प्राप्तः सर्वं दृष्ट्वा नवं दृढम्}
{अपूजयं च मनसा रौक्मिणेय सदा द्विजान्}


\twolineshloka
{इत्यहं रौक्मिणेयस्य पृच्छतो भरतर्षभ}
{माहात्म्यं द्विजमुख्यस्य सर्वमाख्यातवांस्तदा}


\twolineshloka
{तथा त्वमपि कौन्तेय ब्राह्मणान्सततं प्रभो}
{पूजयस्व महाभागान्वाग्भिर्दानैश्च नित्यदा}


\twolineshloka
{एवं व्युष्टिमहं प्राप्तो ब्राह्मणस्य प्रसादजाम्}
{यच्च मामाह भीष्मोऽयं तत्सत्यं भरतर्षभ}


\chapter{अध्यायः २६५}
\twolineshloka
{दुर्वाससः प्रसादात्ते शंकरांशस्य माघव}
{अवाप्तमिह विज्ञानं तन्मे व्याख्यातुमर्हसि}


\threelineshloka
{महाभाग्यं च यत्तस्य नामानि च महात्मनः}
{तत्त्वमो ज्ञातुमिच्छामि सर्वं मतिमतांवर ॥वासुदेव उवाच}
{}


\twolineshloka
{हन्ति ते कीर्तयिष्यामि नमस्कृत्य कपर्दिने}
{यदवाप्तं मया राजञ्श्रेयो यच्चार्जितं यशः}


\twolineshloka
{प्रयतः प्रातरुत्थाय यस्त्वधीयेद्विशाम्पते}
{प्राञ्जलिः शतरुद्रीयं नास्य किञ्चनि दुर्लभम्}


\threelineshloka
{`शिवः सर्वकतो रुद्रः स्रष्टा यस्त्वं शृणुष्व मे}
{'प्रजापतिस्तमसृजत्तमसोऽन्ते महातपाः}
{शङ्करस्त्वसृजत्तात प्रजाः स्थावरजङ्गमाः}


\twolineshloka
{नास्ति किञ्चित्परं भूतं महादेवाद्विशाम्पते}
{इह त्रिष्वपि लोकेषु भूतानां प्रभवो हि सः}


\twolineshloka
{न चैवोत्सहते स्थातुं कश्चिदग्रे महात्मनः}
{न हि भूतं समं तेन त्रिषु लोकेषु विद्यते}


\twolineshloka
{गन्धेनापि हि सङ्ग्रामे तस्य क्रुद्धस्य शत्रवः}
{विसंज्ञा हतभूयिष्ठा वेपनेते च पतन्ति च}


\threelineshloka
{घोरं च निनदं तस्य पर्जन्यनिनदोपम्}
{श्रुत्वा विशीर्येद्धृदयं देवानामपि संयुगे}
{यं चाक्ष्णा घोररूपेण पश्येद्दग्धः पतेदधः}


\twolineshloka
{न सुरा नासुरा लोके न गन्धर्वा न पन्नगाः}
{कुपिते सुखमेधन्ते तस्मिन्नपि गुहागताः}


\twolineshloka
{प्रजापतेश्च दक्षस्य यजतो वितते क्रतौ}
{विव्याध कुपितो यज्ञं निर्भयस्तु भवस्तदा}


% Check verse!
धनुषा वाणमुत्सृज्य सुघोरं विननाद च
\twolineshloka
{ते न शर्म कुतः शान्ति विषादं लेभिरे सुराः}
{विद्धे च सहसा यज्ञे कुपिते च महेश्वरे}


\twolineshloka
{तेन ज्यातलघोषेण सर्वे लोकाः समाकुलाः}
{बभूवुरवशाः पार्थ विषेदुश्च सुरासुराः}


\twolineshloka
{आपश्चुक्षुभिरे चैव चकम्पे च वसुन्धरः}
{व्यद्रवग्निरयश्चापि द्यौः पफाल च सर्वशः}


\twolineshloka
{अन्धेन तमसा लोकाः प्रावृता न चकाशिरे}
{प्रनष्टा ज्योतिषां भाश्च सह सूर्येण भारत}


\twolineshloka
{भृशं भीतास्ततः शान्तिं चत्रुः स्वस्त्ययनानि च}
{ऋषयः सर्वभूतानामात्मनश्च हितैषिमः}


\twolineshloka
{ततः सोऽभ्यद्रवद्देवान्रुद्रो रौद्रपराक्रमः}
{भगस्य नयने क्रुद्धः प्रहारेण व्यशातयत्}


\twolineshloka
{पूषणं चाभिदुद्राव घोरेण वपुषाऽन्वितः}
{पुरोडाशं भक्षयतो दशनान्वै व्यशातयत्}


\twolineshloka
{ततः प्रणेमुर्देवास्ते वेपमानाः स्म शङ्करम्}
{पुनश्च सन्दधे रुद्रो दीप्तं सुनिशितं शरम्}


\twolineshloka
{रुद्रस्य विक्रमं दृष्ट्वा भीता देवाः सहर्षिभिः}
{ततः प्रसादयामासुः शर्वं ते विबुधोत्तमाः}


\twolineshloka
{जेषुश्च शतरुद्रीयं देवाः कृत्वाञ्जलिं तदा}
{संस्तूयमानस्त्रिदशैः प्रससाद महेश्वरः}


\twolineshloka
{रुद्रस्य भागं यज्ञे च विशिष्टं ते त्वकल्पयन्}
{भयेन त्रिदशा राजञ्शरणं च प्रपेदिरे}


\twolineshloka
{तेन चैव हि तुष्टेन स यज्ञः सन्धितोऽभवत्}
{यद्यच्चापहृतं तत्र तत्तथैवान्वजीवयत्}


\twolineshloka
{असुराणां पुराण्यासंस्त्रीणि वीर्यवतां दिवि}
{आयसं राजतं चैव सौवर्णमपि चापरम्}


\twolineshloka
{नाशकत्तानि मघवा भेत्तुं सर्वायुधैरपि}
{अथ सर्वेऽमरा रुद्रं जग्मुः शरणमर्दिताः}


\twolineshloka
{तत ऊचुर्महात्मानो देवाः सर्वे समागताः}
{रुद्र रौद्रा भविष्यन्ति पशवः सर्वकर्मसु}


\twolineshloka
{जहि दैत्यान्सह पुरैर्लोकांस्त्रायस्व मानद}
{स तथोक्तस्तथेत्युक्त्वा कृत्वा विष्णुं शरोत्तमम्}


\twolineshloka
{शल्यमग्निं तथा कृत्वा पुङ्खं वैवस्वतं यमम्}
{ओङ्कारं च धनुः सर्वाञ्ज्यां च सावित्रिमुत्तमां}


\twolineshloka
{ब्रह्माणं सारथिं कृत्वा विनियुज्य च सर्वशः}
{त्रिपर्वणा त्रिशल्येन तेन तानि बिभेद सः}


\twolineshloka
{शरेणादित्यवर्णेन कालाग्निसमतेजसा}
{तेऽसुराः सपुरास्तत्र दग्धा रुद्रेण भारत}


\twolineshloka
{तं चैवाङ्कगतं दृष्ट्वा बालं पञ्चशिखं पुनः}
{उमा जिज्ञासमाना वै कोऽयमित्यब्रवीत्तदा}


\twolineshloka
{असूयतश्च शक्रस्य वज्रेणि प्रहरिष्यतः}
{सवज्रं स्तम्भयामास तं बाहुं परिघोपमम्}


\twolineshloka
{न संयुयुधिरे चैव देवास्तं भुवनेश्वरम्}
{सप्रजापतयः सर्वे तस्मिन्मुमुहुरीश्वरे}


\twolineshloka
{ततो ध्यात्वा च भगवान्ब्रह्मा तममितौजसम्}
{अयं श्रेष्ठ इति ज्ञात्वा ववन्दे तमुमापतिम्}


\twolineshloka
{ततः प्रसादयामासुरुमां रुद्रं च ते सुराः}
{बभूव स तदा बालः प्रययौ तु यथापुरम्}


\twolineshloka
{स चापि ब्राह्मणो भूत्वा दुर्वासा नाम वीर्यवान्}
{द्वारवत्यां मम गृहे चिरं कालमुपावसन्}


\twolineshloka
{विप्रकारान्प्रयुङ्क्ते स्म सुबहून्मम वेश्मनि}
{तानुदारतया चाहं चक्षमे चातिदुःसहान्}


\twolineshloka
{स वै रुद्रः स च शिवः सोग्निः सर्वः स सर्वजित्}
{स चैवेन्द्रश्च वायुश्च सोऽस्विनौ स च विद्युतः}


\twolineshloka
{स चन्द्रमाः स चेशानः स सूर्यो वरुणश्च सः}
{स कालः सोन्तको मृत्युः स यमो रात्र्यहानि च}


\twolineshloka
{मासार्धमासा ऋतवः सन्ध्ये संवत्सरश्च सः}
{सधाता स विधाता च विश्वकर्मा स सर्ववित्}


\twolineshloka
{नक्षत्राणि ग्रहाश्चैव दिशोऽथ प्रदिशस्तथा}
{विश्वमूर्तिरमेयात्मा भगवान्परमद्युतिः}


\twolineshloka
{एकधा च द्विधा चैव बहुधा च स एव हि}
{शतधा सहस्रधा चैव तथा शतसहस्रधा}


\twolineshloka
{ईदृशः स महादेवो भूयश्च भगवानतः}
{न हि शक्या गुणा वक्तुमपि वर्षशतैरपि}


\chapter{अध्यायः २६६}
\twolineshloka
{युधिष्ठिर महाबाहो महाभाग्यं महात्मनः}
{रुद्राय बहुरूपाय बहुनाम्ने निबोध मे}


\twolineshloka
{वदन्त्युग्रं महादेवं तथा स्थाणुं महेश्वरम्}
{एकाक्षं त्र्यम्बकं चैव विश्वरूपं शिवं तथा}


\twolineshloka
{द्वे तनू तस्य देवस्य वेदज्ञा ब्राह्मणा विदुः}
{घोरामन्यां शिवामन्यां ते तनू बहुधा पुनः}


\twolineshloka
{उग्रा घोरा तनुर्याऽस्य सोऽग्निर्विद्युत्स भास्करः}
{शिवा सौम्या च या त्वस्य धर्मस्त्वापोथ चन्द्रमाः}


\twolineshloka
{आत्मनोऽर्धं तु तस्याग्निः सोमोऽर्धं पुनरुच्यते}
{ब्रह्मचर्यं चरत्येका शिवा चास्य तनुस्तथा}


\twolineshloka
{याऽस्य घोरतमा मूर्तिर्जगत्संहरते तथा}
{ईश्वरत्वान्महत्त्वाच्च महेश्वर इति स्मृतः}


\twolineshloka
{यन्निर्दहति यत्तीक्ष्णो यदुग्रो यत्प्रतापवान्}
{मांसशोणितमज्जादो यत्ततो रुद्र उच्यते}


\twolineshloka
{देवानां सुमहान्यच्च यच्चास्य विषयो महान्}
{यच्च विश्वं जगत्पाति महादेवस्ततः स्मृतः}


% Check verse!
धूम्ररूपा जटा यस्माद्धूर्जटीत्यत उच्यते
\twolineshloka
{स मेधयति यन्नित्यं सर्वान्वै सर्वकर्मभिः}
{शिवमिच्छन्मनुष्याणां तस्मादेष शिवः स्मृतः}


\twolineshloka
{दहत्यूध्वं स्थितो यच्च प्राणान्नॄणां स्थिरश्च यत्}
{स्थिरलिङ्गश्च यन्नित्यं तस्मात्स्थाणुरिति स्मृतः}


\twolineshloka
{यदस्य बहुधा रूपं भूतं भव्यं भवत्तथा}
{स्थावरं जङ्गमं चैव बहुरूपस्ततः स्मृतः}


\threelineshloka
{विश्वेदेवाश्च यत्तस्मिन्विश्वरूपस्ततः स्मृतः}
{सहस्राक्षोऽयुताक्षो वा सर्वतोक्षिमयोपि वा}
{चक्षुषः प्रभवं तेजः सर्वतश्चक्षुरेव तत्}


\twolineshloka
{सर्वथा यत्पशून्पाति तैश्च यद्रमते सह}
{तेषामधिपतिर्यच्च तस्मात्पशुपतिः स्मृतः}


\threelineshloka
{नित्येन ब्रह्मचर्येण लिङ्गमस्य यदा स्थितम्}
{`भक्तानुग्रहणार्थाय गूढलिङ्गस्ततः स्मृतः}
{'महयत्यस्य लोकश्च प्रियं ह्येतन्महात्मनः}


\twolineshloka
{विग्रहं पूजयेद्यो वै लिङ्गं वाऽपि महात्मनः}
{लिङ्गं पूजयिता नित्यं महतीं श्रियमश्नुते}


\threelineshloka
{ऋषयश्चापि देवाश्च गन्धर्वाप्सरसस्तथा}
{लिङ्गमेवार्चयन्ति स्म यत्तदूर्ध्वं समास्थितम्}
{}


\twolineshloka
{पूज्यमाने ततस्तस्मिन्मोदते स महेश्वरः}
{सुखं ददाति प्रीतात्मा भक्तानां भक्तवत्सलः}


\twolineshloka
{एष एव श्मशानेषु देवो वसति निर्दहन्}
{यजन्ते ये जनास्तत्र वीरस्थाननिषेविणः}


\twolineshloka
{विषमस्थः शरीरेषु स मृत्युः प्राणिनामिह}
{स च वायुः शरीरेषु प्राणपालः शरीरिणाम्}


\twolineshloka
{तस्य घोराणि रूपाणि दीप्तानि च बहुनि च}
{लोके यान्यस्य पूज्यन्ते विप्रास्तानि विदुर्बुधाः}


\twolineshloka
{नामधेयानि देवेषु बहून्यस्य यथार्थवत्}
{निरुच्यन्ते महत्त्वाच्च विभुत्वात्कर्मभिस्तथा}


\twolineshloka
{वेदे चास्य विदुर्विप्राः शतरुद्रीयमुत्तमम्}
{व्यासेनोक्तं च यच्चापि उपस्थानं महात्मनः}


\twolineshloka
{प्रदाता सर्वलोकानां विश्वसाक्षी निरामयः}
{ज्येष्ठभूतं वदन्त्येनं ब्राह्मणा ऋषयोऽपरे}


\twolineshloka
{प्रथमो ह्येष देवानां मुखादग्निमजीजनत्}
{ग्रहैर्बहुविधैः प्राणान्संरुद्धानुत्सृजत्यपि}


\twolineshloka
{विमोक्षयति तुष्टात्मा शरण्यः शरणागतान्}
{आयुरारोग्यमैश्वर्यं हितं कामांश्च पुष्कलान्}


\twolineshloka
{स ददाति मनुष्येभ्यः स एवाक्षिपते पुनः}
{शक्रादिषु च देवेषु तस्य चैश्वर्यमुच्यते}


\twolineshloka
{स एव स्थापको नित्यं त्रैलोक्यस्य शुभाशुभे}
{ऐश्वर्याच्चैव कामानामीश्वरः पुनरुच्यते}


\threelineshloka
{महेश्वरश्च लोकानां महातामीश्वरश्च सः}
{बहुभिर्विविधै रूपैर्विश्वं व्याप्तमिदं जगत्}
{तस्य देवस्य यद्वक्त्रं समुद्रे वडंवामुखण्}


\chapter{अध्यायः २६७}
\twolineshloka
{इत्युक्तवति तद्वाक्यं कृष्णे देवकिनन्दने}
{भीष्मं शान्तनवं भूयः पर्यपृच्छद्युधिष्ठिरः}


\threelineshloka
{निर्णये वा महाबुद्धे सर्वधर्मविदांवर}
{प्रत्यक्षमागमो वेति किं तयोः कारणं भवेत् ॥भीष्म उवाच}
{}


\twolineshloka
{नास्त्यत्रि संशयः कश्चिदिति मे वर्तते मतिः}
{शृणु वक्ष्यामि ते प्राज्ञ सम्यक्त्वं मेऽनुपृच्छसि}


\twolineshloka
{संशयः सुगमस्तत्र दुर्गमस्तस्य निर्णयः}
{दृष्टं श्रुतमनन्तं हि यत्र संशयदर्शन्}


\twolineshloka
{प्रत्यक्षं कारणं दृष्टं हैतुकं प्राज्ञमानिनः}
{नास्तीत्येवं व्यवस्यन्ति सत्यमागममेव वा}


\twolineshloka
{तदयुक्तं व्यवस्यन्ति बालाः पण्डितमानिनः}
{अथ सञ्चिन्त्यमेवैकं कारणं किं भवेदिति}


\twolineshloka
{शक्यं दीर्घेण कालेन युक्तेनामन्त्रितेन च}
{प्राणयात्रामनेकां च कल्पयानेन भारत}


\twolineshloka
{तत्परेणैव नान्येन शक्यते तत्तु कारणम्}
{हेतूनामन्तमासाद्य विपुलं ज्ञानमुत्तमम्}


\fourlineindentedshloka
{ज्योतिः सर्वस्य लोकस्य विपुलं प्रतिपद्यते}
{न त्वेव गमनं राजन्हेतुतो गमनं तथा}
{अग्राह्यमनिबद्धं च वाचा सम्परिवर्जयेत् ॥युधिष्ठिर उवाच}
{}


\threelineshloka
{प्रत्यिक्षं लोकतः सिद्धिर्लोकश्चागमपूर्वकः}
{शिष्टाचारो बहुविधस्तन्मे ब्रूहि पितामहः ॥भीष्म उवाच}
{}


\twolineshloka
{धर्मस्य ह्रियमाणस्य बलवद्भिर्दुरात्मभिः}
{संस्था यत्नैरपि कृता कालेन प्रतिभिद्यते}


\twolineshloka
{अधर्मो धर्मरूपेण तृणैः कूप इवावृतः}
{ततस्तैर्भिद्यते वृत्तं शृणु चैव युधिष्ठिर}


\twolineshloka
{अवृत्त्या ये तु निन्दनि श्रुतत्यागपरायणाः}
{धर्मविद्वेषिणो मन्दा इत्युक्तस्तेषु संशयः}


\twolineshloka
{अतृप्यन्तस्तु साधूनां यावदागमबुद्धयः}
{परमित्येव सन्तुष्टास्तानुपास्स्व च पृच्छ च}


\twolineshloka
{कामार्थौ पृष्ठतः कृत्वा लोभमोहानुसारिणौ}
{धर्म इत्येव सम्बुद्धस्तानुपास्स्व च पृच्छ च}


\threelineshloka
{न तेषां भिद्यते वृत्तं यज्ञाः स्वाध्यायकर्म च}
{आचारः कारणं चैव धर्मश्चैकस्त्रयं पुनः ॥युधिष्ठिर उवाच}
{}


\twolineshloka
{पुनरेव हि मे बुद्धि संशये परिमुह्यति}
{अपरो मज्जमानस्य परं तीरमपश्यतः}


\threelineshloka
{वेदः प्रत्यक्षमाचारः प्रमाणं तत्त्रयं यदि}
{पृथक्त्वं लभ्यते चैषां धर्मश्चैतत्त्रयं कथम् ॥भीष्म उवाच}
{}


\twolineshloka
{धर्मस्य ह्रियमाणस्य बलवद्भिर्दुरात्मभिः}
{यद्येवं मन्यसे राजंस्त्रिधा धर्मविचारणा}


\twolineshloka
{एक एवेति जानीहि त्रिधा धर्मस्य दर्शनम्}
{पृथक्त्वे च न मे बुद्धिस्त्रयाणामपि वै तथा}


\twolineshloka
{[उक्तो मार्गस्त्रयाणां च तत्तथैव समाचर}
{जिज्ञासा न तु कर्तव्या धर्मस्य परितर्कणात्}


\twolineshloka
{सदैव भरतश्रेष्ठ मा ते भूदत्र संशयः}
{अन्धो जड इवाशङ्की यद्ब्रवीमि तदाचर}


\twolineshloka
{अहिंसा सत्यमक्रोधो दानमेतच्चतुष्टयम्}
{अजातशत्रो सेवस्य धर्म एष सनातनः ॥]}


\twolineshloka
{ब्राह्मणेषु च वृत्तिर्या पितृपैतामहोचिता}
{तामन्वेहि महाबाहो धर्मस्यैते हि देशिकाः}


\twolineshloka
{प्रमाणमप्रमाणं वै यः कुर्यादबुधो जनः}
{न स प्रमाणतामर्हो विषादजननो हि सः}


\twolineshloka
{ब्राह्मणानेव सेवस्व सत्कृत्य बहुमान्य च}
{एतेष्वेव त्विमे लोकाः कृत्स्ना इति निबोध तान्}


\chapter{अध्यायः २६८}
\threelineshloka
{ये च धर्ममसूयन्ते ये चैनं पर्युपासते}
{ब्रवीतु मे भवानेतत्क्व ते गच्छन्ति तादृशाः ॥भीष्म उवाच}
{}


\twolineshloka
{रजसा तमसा चैव समवस्तीर्णचेतसः}
{नरकं प्रतिपद्यन्ते धर्मविद्वेषिमो जनाः}


\twolineshloka
{ये तु धर्मं महाराज सततं पर्युपासते}
{सत्यार्जवपराः सन्तस्ते वै स्वर्गभूजो नराः}


\twolineshloka
{धर्म एव रतिस्तेषामाचार्योपासनाद्भवेत्}
{देवलोकं प्रपद्यन्ते ये धर्मं पर्युपासते}


\twolineshloka
{मनुष्या यदि वा देवाः शरीरमुपताप्य वै}
{धर्मिणः सुखमेधन्ते लोभद्वेषविवर्जिताः}


\threelineshloka
{प्रथमं ब्रह्म्णः पुत्रं धर्ममाहुर्मनीषिणः}
{धर्मिणः पर्युपासन्ते फलं पक्वमिवाशिनः ॥युधिष्ठिर उवाच}
{}


\threelineshloka
{असाधोः कीदृशं शीलं साधोश्चैव तु कीदृशम्}
{ब्रवीतु मे भवानेतत्सन्तोऽसन्तश्च कीदृशाः ॥भीष्म उवाच}
{}


\twolineshloka
{दुराधाराश्च दुर्धर्षा दुर्मुखाश्चाप्यसाधवः}
{साधवः शीलसम्पन्नाः शिष्टाचारस्य लक्षणम्}


\twolineshloka
{राजमार्गे गवां मध्ये जनमध्ये च धर्मिणः}
{नोपसेचन्ति राजेन्द्र सर्गं मूत्रपुरीषयोः}


\twolineshloka
{पञ्चानां यजनं कृत्वा शेषमश्नन्ति साधवः}
{न जल्पन्ति च भुञ्जाना न निद्रान्त्यार्द्रपाणयः}


\twolineshloka
{चित्रभानुमपां देवं गोष्ठं चैव चतुष्पथम्}
{ब्राह्मणं धार्मिकं वृद्धं ये कुर्वन्ति प्रदक्षिणम्}


\twolineshloka
{वृद्धानां भारतप्रानां स्त्रीणां बालातुरस्य च}
{ब्राह्मणानां गवां राज्ञां पन्थानं ददते च ये}


\twolineshloka
{अतिथीनां च सर्वेषां प्रेष्याणां स्वजनस्य च}
{`सामान्यं भोजनं कुर्यात्स्वयं नाग्र्यशनं व्रजेत्}


\twolineshloka
{न सत्यार्जवधर्मस्य तुल्यमन्यच्च विद्यते}
{बहुला नाम गौस्तेन गतिमग्र्यां गता किल}


\twolineshloka
{मुनिशापाद्द्विजः कश्चिद्व्याघ्रतां समुपागतः}
{बहुलां भक्षणरुचिरास्वाद्य शपथेन तु}


\twolineshloka
{विमुच्य पीतवत्सां तां दृष्ट्वा स्मृत्वा पुरातनम्}
{जगाम लोकानमलान्सा स्वराष्ट्रं तथा पुनः}


\twolineshloka
{तस्मात्सत्यार्जवरतो राजन्राष्ट्रं समानवम्}
{तारयित्वा सुखं स्वर्गं गन्तासि भरतर्षभ}


\twolineshloka
{तथा शरणकामानां गोप्ता स्यात्स्वागतप्रदः ॥सायं प्रातर्मनुष्याणामशनं देवनिर्मितम्}
{}


\twolineshloka
{नान्तरा भोजनं दृष्टमुपवासविधिर्हि सः ॥होमकाले यथा वह्निः काले होमं प्रतीक्षते}
{}


\twolineshloka
{ऋतुकाले तथाऽऽधानं पितरश्च प्रतीक्षते}
{नान्यदा गच्छते यस्तु ब्रह्मचर्यं च तत्स्मृतम्}


\twolineshloka
{अमृतं ब्राह्मणा गाव इत्येतत्त्रयमेकतः}
{तस्मादोब्राह्मणान्नित्यमर्चयेत यथाविधि}


\twolineshloka
{यजुषां संस्कृतं मांसमुपभुञ्जत दुष्यति}
{पृष्ठमांसं वृथामांसं पुत्रमांसं च तत्समम्}


\twolineshloka
{स्वदेशे परदेशे वाऽप्यतिथिं नोपवासयेत्}
{कर्म वै सफलं कृत्वा गुरुणां प्रतिपादयेत्}


\twolineshloka
{गुरुभ्यस्त्वासनं देयमभिवाद्याभिपूज्य च}
{गुरुमभ्यर्च्य वर्धन्ते आयुषा यशसा श्रिया}


\twolineshloka
{वृद्धान्नाभिभवेज्जातु न चैतान्प्रेषयेदिति}
{नासीनः स्यात्स्थितेष्वेवमायुरस्य न रिष्यते}


\twolineshloka
{न नग्रामीक्षते नारीं न नग्रान्पुरुषानपि}
{मैथुनं सततं गुप्तं तपश्चैव समाचरेत्}


\twolineshloka
{तीर्थानां गुरवस्तीर्थं शुचीनां हृदयं शुचि}
{दर्शनानां परं ज्ञानं सन्तोषः परमं सुखम्}


\twolineshloka
{सायं प्रातश्च वृद्धानां शृणुयात्पुष्कला गिरः}
{श्रुतमाप्नोति हि नरः सततं वृद्धसेवया}


\twolineshloka
{स्वाध्याये भोजने चैव दक्षिणं पाणिमुद्धरेत्}
{यच्छेद्वाङ्मनसी नित्यमिन्द्रियाणि तथैव च}


\twolineshloka
{संस्कृतं पायसं नित्यं यावकं कृसरं हविः}
{अष्टकाः पितृदैवत्या ग्रहाणामभिपूजनम्}


\twolineshloka
{श्मश्रुकर्मणि मङ्गल्यं क्षुतानामभिनन्दनम्}
{व्याधितानां च सर्वेषामायुषामभिनन्दनम्}


\threelineshloka
{न जातु त्वमिति ब्रूयादापन्नोपि महृत्तरम्}
{त्वंकारो वा वधो वेति विद्वत्सु न विशिष्यते}
{अवराणां समानानां शिष्याणां च समाचरेत्}


\twolineshloka
{पापमाचरते नित्यं हृदयं पापकर्मणाम्}
{ज्ञानपूर्वकृतं कर्म च्छादयन्ते ह्यसाधवः}


\twolineshloka
{ज्ञानपूर्वं विनश्यन्ति गूहमाना महाजने}
{न मां मनुष्याः पश्यन्ति न मां पश्यन्ति देवताः}


% Check verse!
पापेनाभिहितः पापः पापमेवाभिजायते
\twolineshloka
{यथा वार्धुषिको वृद्धिं दिनभेदे प्रतीक्षते}
{धर्मेण पिहितं पापं धर्ममेवाभिवर्धयेत्}


\twolineshloka
{यथा लवणमम्भोभिराप्लुतं प्रविलीयते}
{प्रायश्चित्तहतं पापं तथा सद्यः प्रणश्यति}


\twolineshloka
{तस्मात्पापं न गूहेत गूहमानं विवर्धते}
{कृत्वा तु सादुष्वाख्यायाधर्मं प्रशमयन्त्युत}


\twolineshloka
{आशया सञ्चितं द्रव्यं स्वकाले नोपभुज्यते}
{अन्ये चैतत्प्रद्यन्ते वियोगे तस्य देहिनः}


\threelineshloka
{`तद्धर्मसाधनं नित्यं सङ्कल्पाद्धनमार्जयेत्}
{'मननं सर्वभूतानां धर्ममाहुर्मनीषिणः}
{}


\twolineshloka
{तस्मात्सर्वाणि भूतानि धर्ममेव समासते ॥एक एव चरेद्धर्मं न धर्मध्वजिको भवेत्}
{}


\twolineshloka
{धर्मवाणिजका ह्येते ये धर्ममुपभुञ्जते ॥अर्चेद्देवानदम्भेन सेवेताऽमायया गुरून्}
{}


% Check verse!
धनं निदध्यात्पात्रेषु परत्रार्थं समावृतम्
\chapter{अध्यायः २६९}
\twolineshloka
{नाभागधेयः प्राप्नोति धनं सुबलवानपि}
{भागधेयान्वितस्त्वर्थान्कृशो बालश्च विन्दति}


\twolineshloka
{नालाभकाले लभते प्रयत्नेऽपि कृते सति}
{लाभकालेऽप्रयत्नेन लभते विपुलं धनम्}


\twolineshloka
{कृतयत्नाफलाश्चैव दृश्यन्ते शतशो नराः}
{अयत्नेनैधमानाश्च दृश्यन्ते बहवो जनाः}


\twolineshloka
{यदि यत्नो भवेन्मर्त्यः स सर्वं फलमाप्नुयात्}
{नालभ्यं चोपलभ्येत नृणां भरतसत्तम}


\twolineshloka
{प्रयत्नं कृतवन्तोपि दृश्यन्ते ह्यफला नराः}
{मार्गत्यागशतैरर्थानमार्गश्चापरः सुखी}


\twolineshloka
{अकार्यमसकृत्कृत्वा दृश्यन्ते ह्यधना नराः}
{धनयुक्ताः स्वकर्मस्था दृश्यन्ते चापरेऽधनाः}


\twolineshloka
{अधीत्य नीतिशास्त्राणि नीतियुक्तो न दृश्यते}
{अनभिज्ञश्च साचिव्यं गमितः केन हेतुना}


\twolineshloka
{विद्यायुक्तो ह्यविद्यश्च धनवान्दुर्मतिस्तथा}
{यदि विद्यामुपाश्रित्य नरः सुखमवाप्नुयात्}


\twolineshloka
{न विद्वान्विद्यया हीनं वृत्त्यर्थमुपसंश्रयेत्}
{यथा पिपासां जयति पुरुषः प्राप्य वै जलम्}


% Check verse!
इष्टार्थो विद्यया ह्येव न विद्यां प्रजहेन्नरः
\threelineshloka
{नाप्राप्तकालो म्रियते विद्धः शरशतैरपि}
{तृणाग्रेणापि संस्पृष्टः प्राप्तकालो न जीवति ॥भीष्म उवाच}
{}


\twolineshloka
{ईहमानः समारम्भान्यदि नासादयेद्धनम्}
{उग्रं तपः समारोहेन्न ह्यनुप्तं प्ररोहति}


\twolineshloka
{दानेन भोगी भवति मेधावी वृद्धसेवया}
{अहिंसया च दीर्घायुरिति प्राहुर्मनीषिणः}


\twolineshloka
{तस्माद्दद्यान्न याचेत पूजयेद्धार्मिकानपि}
{सुभाषी प्रियकृच्छान्तः सर्वसत्वाविहिंसकः}


\twolineshloka
{यदा प्रमाणं प्रसवः स्वभावश्च सुखासुखे}
{दंशकीटपिपीलानां स्थिरो भव युधिष्ठिर}


\chapter{अध्यायः २७०}
\twolineshloka
{कार्यते यच्च क्रियते सच्चासच्च कृताकृतम्}
{तत्राश्वसीत सत्कृत्वा असत्कृत्वा न विश्वसेत्}


\twolineshloka
{काल एवान्तराशक्तिर्निग्रहानुग्रहौ ददत्}
{बुद्धिमाविश्य भूतानां धर्माधर्मौ प्रवर्तते}


\twolineshloka
{यदा त्वस्य भवेद्बुद्धिर्धर्मार्थस्य प्रदर्शनात्}
{तदाश्वसीत धर्मात्मा दृढबुद्धिर्न विश्वसेत्}


\twolineshloka
{एतावन्मात्रमेतद्धि भूतानां प्राज्ञलक्षणम्}
{कालयुक्तोऽप्युभयविच्छेषं युक्तं समाचरेत्}


\twolineshloka
{यथा ह्युपस्थितैश्वर्याः पूजयन्ति नरा जनान्}
{एवमेवात्मनाऽऽत्मानं पूजयन्तीह धार्मिकाः}


\threelineshloka
{`भावशुद्धिस्तु तपसा देवतानां च मूजया}
{सनातनेन शुद्ध्या च श्रुतदानजपैरपि}
{न ह्यशुद्धस्तु तां दद्याद्धर्मकाले कथञ्चन ॥'}


\twolineshloka
{न ह्यधर्मतया धर्मं दद्यात्कालः कथञ्चन}
{तस्माद्विशुद्धमात्मानं जानीयाद्धर्मचारिणम्}


\twolineshloka
{स्प्रष्टुमप्यसमर्थो हि ज्वलन्तमिव पावकम्}
{अधर्मः सन्ततो धर्मं कालेन परिरक्षितम्}


\twolineshloka
{कार्यावेतौ हि धर्मोणि धर्मो हि विजयावहः}
{त्रयाणामपि लोकानामालोकः कारणं भवेत्}


\threelineshloka
{न तु कश्चिन्नयेत्प्राज्ञो गृहीत्वैव करे नरम्}
{ऊह्यमानस्तु धर्मेण प्राप्नुयात्परमच्युतम्}
{विश्वास एव कर्तव्यो बहुधर्मे शुभच्छले}


\twolineshloka
{शूद्रोऽहं नाधिकारो मे चातुराश्रम्यसेवने}
{इति विज्ञानमपरे नात्मन्युपदधत्युत}


\twolineshloka
{विशेषेण च वक्ष्यामि चातुर्वर्ण्यस्य लिङ्गतः}
{पञ्चभूतशरीराणां सर्वेषां सदृशात्मनाम्}


\twolineshloka
{लोकधर्मे च धर्मे च विशेषकरणं कृतम्}
{यथैकत्वं पुनर्यान्ति प्राणिनस्तत्र विस्तरः}


\twolineshloka
{अध्रुवो हि कथं लोकः स्मृतो धर्मः कथं ध्रुवः}
{यत्र कालो ध्रुवस्तात तत्र धर्मः सनातनः}


\twolineshloka
{सर्वेषां तुल्यदेहानां सर्वेषां सदृशात्मनाम्}
{कालो धर्मेण संयुक्तः शेष एव स्वयं गुरुः}


\twolineshloka
{एवं सति न दोषोऽस्ति भूतानां धर्मसेवने}
{तिर्यग्योनावपि सतां लोक एव मतो गुरुः}


\chapter{अध्यायः २७१}
\threelineshloka
{शरतल्पगतं भीष्मं पाण्डवोऽथ कुरूद्वहः}
{युधिष्ठिरो हितं प्रेप्सुरपृच्छत्कल्मषापहम् ॥युधिष्ठिर उवाच}
{}


\threelineshloka
{किं श्रेयः पुरुषस्येह किं कुर्वन्सुखमेधते}
{विपाप्मा स भवेत्केन किं वा कल्मषनाशनम् ॥वैशम्पायन उवाच}
{}


\threelineshloka
{तस्मै शुश्रूषमाणाय भूयः शान्तनवस्तदा}
{दैवं वंशं यथान्यायमाचष्ट पुरुषर्षभ ॥भीष्म उवाच}
{}


\twolineshloka
{अयं दैवतवंशो वै ऋषिवंशसमन्वितः}
{त्रिसन्ध्यं पठितः पुत्र कल्मषापहरः परः}


\twolineshloka
{यदह्ना कुरुते पापमिन्द्रियैः पुरुषश्चरन्}
{बुद्धिपूर्वमबुद्धिर्वा रात्रौ यच्चापि सन्ध्ययोः}


\twolineshloka
{मुच्यते सर्वपापेभ्यः कीर्तयन्वै शुचिः सदा}
{नान्धो न बधिरः काले कुरुते स्वस्तिमान्सदा}


\twolineshloka
{तिर्यग्योनिं न गच्छेच्च नरकं सङ्कराणि च}
{न च दुःखमयं तस्य मरणे स न मुह्यति}


\twolineshloka
{देवासुरगकुरुर्देवः सर्वभूतनमस्कृतः}
{अचिन्त्योथाप्यनिर्देश्यः सर्वप्राणो ह्ययोनिजः}


\twolineshloka
{पितामहो जगन्नाथः सावित्री ब्रह्मणः सती}
{वेदभूरथ कर्ता च विष्णुर्नारायणः प्रभुः}


\twolineshloka
{उमापतिर्विरूपाक्षः स्कन्दः सेनापतिस्तथा}
{विशाखो हुतभुग्वायुश्चन्द्रसूर्यौ प्रभाकरौ}


\twolineshloka
{शक्रः शचीपतिर्देवो यमो धूमोर्णया सह}
{वरुणः सह गौर्या च सह ऋद्ध्या धनेश्वरः}


\twolineshloka
{सौम्या गौः सुरभिर्देवी विश्रवाश्च महानृषिः}
{सङ्कल्पः सागरो गङ्गा स्रवन्त्योऽथ मरुद्गणः}


\twolineshloka
{वालखिल्यास्तपःसिद्धाः कृष्णद्वैपायनस्तथा}
{नारदः पर्वतश्चैव विश्वावसुर्हहाहुहूः}


\twolineshloka
{तुम्बुरुश्चित्रसेनश्च देवदूतश्च विश्रुतः}
{देवकन्या महाभागा दिव्याश्चाप्सरसां गणाः}


\twolineshloka
{उर्वशी मेनका रम्भा मिश्रकेशी ह्यलम्बुषा}
{विश्वाची च घृताची च पञ्चचूडा तिलोत्तमा}


\twolineshloka
{आदित्या वसवो रुद्राः साश्विनः पितरोपि च}
{धर्मः श्रुतं तपो दीक्षा व्यवसायः पितामहः}


\twolineshloka
{शर्वर्यो दिवसाश्चैव मारीचः कश्यपस्तथा}
{शुक्रो बृहस्पतिर्भौमो बुधो राहुः शनैश्चरः}


\twolineshloka
{नक्षत्राण्यृतवश्चैव मासाः पक्षाः सवत्सराः}
{वैनतेयाः समुद्राश्च कद्रुजाः पन्नगास्तथा}


\twolineshloka
{शतद्रुश्च विपाशा च चन्द्रभागा सरस्वती}
{सिन्धुश्च देविका चैव प्रभासं पुष्कराणि च}


\twolineshloka
{गङ्गा महानदी वेणा कावेरी नर्मदा तथा}
{कुलम्पुना विशल्या च करतोयाम्बुवाहिनी}


\twolineshloka
{सरयूर्गण्डकी चैव लोहितश्च महानदः}
{ताम्रारुणा वेत्रवती पर्णाशा गौतमी तथा}


\twolineshloka
{गोदावरी च वेण्या च कृष्णवेणा तथाऽद्रिजा}
{दृषद्वती च कावेरी चक्षुर्मन्दाकिनी तथा}


\twolineshloka
{प्रयागं च प्रभासं च पुण्यं नैमिषमेव च}
{तच्च विश्वेश्वरस्थानं यत्र तद्विमलं सरः}


\twolineshloka
{पुण्यतीर्थं सुसलिलं कुरुक्षेत्रं प्रकीर्तितम्}
{सिन्धूत्तमं तपो दानं जम्बूमार्गमथापि च}


\twolineshloka
{हिरण्वती वितस्ता च तथा प्लक्षवती नदी}
{वेदस्मृतिर्वेदवती मालवाऽथाश्ववत्यपि}


\twolineshloka
{भूमिभागास्तथा पुण्या गङ्गाद्वारमथापि च}
{ऋषिकुल्यास्तथा मेध्या नद्यः सिन्धुवहास्तथा}


\twolineshloka
{चर्मण्वती नदी पुण्या कौशिकी यमुना तथा}
{नदी भीमरथी चैव बाहुदा च महानदी}


\twolineshloka
{माहेन्द्रवाणी त्रिदिवा नीलिका च सरस्वती}
{नन्दा चापरनन्दा च तथा तीर्थमहाह्रदः}


\twolineshloka
{गयाऽथ फल्गुतीर्थं च धर्मारण्यं सुरैर्वृतम्}
{तथा देवनदी पुण्या सरश्च ब्रह्मनिर्मितम्}


\twolineshloka
{पुण्यं त्रिलोकविख्यातं सर्वपापहरं शिवम्}
{हिमवान्पर्वतश्चैव दिव्यौषधिसमन्वितः}


\twolineshloka
{विन्ध्यो धातुविचित्राङ्गस्तीर्थवानौषधान्वितः}
{मेरुर्महेन्द्रो मलयः श्वेतश्च रजतावृतः}


\twolineshloka
{शृङ्गवान्मन्दरो नीलो निषधो दर्दुरस्तथा}
{चित्रकूटोऽञ्जनाभश्च पर्वतो गन्धमादनः}


\twolineshloka
{पुण्यः सोमगिरिश्चैव तथैवान्ये महीधराः}
{दिशश्च विदिशश्चैव क्षितिः सर्वे महीरुहाः}


\twolineshloka
{विश्वेदेवा नभश्चैव नक्षत्राणि ग्रहास्तथा}
{पान्तु नः सततं देवाः कीर्तिताऽकीर्तिता मया}


\twolineshloka
{कीर्तयानो नरो ह्येतान्मुच्यते सर्वकिल्बिषैः}
{स्तुवंश्च प्रतिनन्दंश्च मुच्यते सर्वतो भयात्}


\twolineshloka
{सर्वसङ्करपापेभ्यो देवतास्तवनन्दकः}
{देवतानन्तरं विप्रांस्तपःसिद्धांस्तपोधिकान्}


\twolineshloka
{कीर्तितान्कीर्तयिष्यामि सर्वपापप्रमोचनान्}
{यवक्रीतोऽथ रैम्यश्च कक्षीवानौशिजस्तथा}


\twolineshloka
{भृग्वङ्गिरास्तथा कण्वो मेधातिथिरथ प्रभुः}
{बर्ही च गुणसम्पन्नः प्राचीं दिशमुपाश्रिताः}


\twolineshloka
{भद्रां दिशं महाभागा उल्मुचुः प्रमुचुस्तथा}
{मुमुचुश्च महाभागः स्वस्त्यात्रेयश्च वीर्यवान्}


\twolineshloka
{मित्रावरुणयोः पुत्रस्तथाऽगस्त्यः प्रतापवान्}
{दृढायुश्चोर्ध्वबाहुश्च विश्रुतावृषिसत्तमौ}


\twolineshloka
{पश्चिमां दिशमाश्रित्य य एधन्ते निबोध तान्}
{उषङ्गुः सह सोदर्यैः परिव्याधश्च वीर्यवान्}


\twolineshloka
{ऋषिर्दीर्घतमाश्चैव गौतमः काश्यपस्तथा}
{एकतश्च द्वितश्चैव त्रितश्चैव महानृषिः}


\threelineshloka
{अत्रेः पुत्रश्च धर्मात्मा तथा सारस्वतः प्रभुः}
{उत्तरां दिशमाश्रित्य य एधन्ते निबोध तान्}
{}


\twolineshloka
{अत्रिर्वसिष्ठः शक्तिश्च पाराशर्यश्च वीर्यवान्}
{विश्वामित्रो भरद्वाजो जमदग्निस्तथैव च}


\twolineshloka
{ऋचीकपुत्रो रामश्च ऋषिरौद्दालकिस्तथा}
{श्वेतकेतुः कोहलश्च विपुलो देवलस्तथा}


\twolineshloka
{देवशर्मा च धौम्यश्च हस्तिकाश्यप एव च}
{लोमशो नाचिकेतश्च लोमहर्षण एव च}


\twolineshloka
{ऋषिरुग्रश्रवाश्चैव भार्गवश्च्यवनस्तथा}
{एष वै समवायश्च ऋषिदेवसमन्वितः}


\twolineshloka
{आद्यः प्रकीर्तितो राजन्सर्वपापप्रमोचनः}
{नृगो ययातिर्नहुषो यदुः पूरुश्च वीर्यवान्}


\twolineshloka
{धुन्धुमारो दिलीपश्च सगरश्च प्रतापवान्}
{कृशाश्वो यौवनाश्वश्च चित्राश्वः सत्यवांस्तथा}


\twolineshloka
{दुष्यन्तो भरतश्चैव चक्रवर्ती महायशाः}
{पवनो जनकश्चैव तथा दृष्टरथो नृपः}


\twolineshloka
{रघुर्नरवरश्चैव तथा दशरथो नृपः}
{रामो राक्षसहा वीरः शशबिन्दुर्भगीरथः}


\twolineshloka
{हरिश्चन्द्रो मरुत्तश्च तथा द़ढरथो नृपः}
{महोदर्यो ह्यलर्कश्च ऐलश्चैव नराधिपः}


\twolineshloka
{करंधमो नरश्रेष्ठः कध्मोरश्च नराधिपः}
{दक्षोऽम्बरीषः कुकुरो रैवतश्च महायशाः}


\twolineshloka
{कुरु संवरणश्चैव मांधाता सत्यविक्रमः}
{मुचुकुन्दश्च राजर्षिर्जह्नुर्जाह्नविसेवितः}


\twolineshloka
{आदिराजः पृथुर्वैन्यो मित्रभानुः प्रियङ्करः}
{त्रसद्दस्युस्तथा राजा श्वेतो राजर्षिसत्तमः}


\twolineshloka
{महाभिषश्च विख्यातो निमी राजा तथाऽष्टकः}
{आयुः क्षुपश्च राजर्षिः कक्षेयुस्च नराधिपः}


\twolineshloka
{प्रतर्दनो दिवोदासः सुदासः कोसलेश्वरः}
{ऐलो नलश्च राजर्षिर्मनुश्चैव प्रजापतिः}


\twolineshloka
{हविध्रश्च पृषध्रश्च प्रतीपः शान्तनुस्तथा}
{अजः प्राचीनबर्हिश्च तथैक्ष्वाकुर्महायशाः}


\twolineshloka
{अनरण्यो नरपतिर्जानुजङ्घस्तथैव च}
{कक्षसेनश्च राजर्षिर्ये चान्ये चानुकीर्तिताः}


\twolineshloka
{कल्यमुत्थाय यो नित्यं सन्ध्ये द्वेऽस्तमयोदये}
{पठेच्छुचिरनावृत्तः स धर्मफलभाग्भवेत्}


\twolineshloka
{देवा देवर्षयश्चैव स्तुता राजर्षयस्तथा}
{पुष्टिमायुर्यशः स्वर्गं विधास्यन्ति ममेश्वराः}


\twolineshloka
{मा विघ्नं मा च मे पापं मा च मे परिपन्थिनः}
{ध्रुवो जयो मे नित्यः स्यात्परत्र च शुभा गतिः}


\threelineshloka
{`पालय त्वं प्रजाः सर्वाः शान्तात्मा त्वनुशासिता}
{द्वैपायनः स्वयंचक्षुः कृष्णस्तेऽस्तु परायणम् ॥वैशम्पायन उवाच}
{}


% Check verse!
इत्युक्त्वोपासनार्थाय विरराम महामतिः ॥'
\chapter{अध्यायः २७२}
\twolineshloka
{शरतल्पगते भीष्मे कौरवाणां धुरन्धरे}
{शयाने वीरशयने पाण्डवैः समुपस्थिते}


\twolineshloka
{युधिष्ठिरो महाप्राज्ञो मम पूर्वपितामहः}
{धर्माणामागमं श्रुत्वा विदित्वा सर्वसंशयान्}


\threelineshloka
{दानानां च विधिं श्रुत्वा च्छिन्नधर्मार्थसंशयः}
{यदन्यदकरोद्विप्र तन्मे शंसितुमर्हसि ॥वैशम्पायन उवाच}
{}


\twolineshloka
{अभून्मुहूर्तं स्तिमितं सर्वं तद्राजमण्डलम्}
{तूष्णींभूते ततस्तस्मिन्पटे चित्रमिवार्पितम्}


\twolineshloka
{मुहूर्तमिव च ध्यात्वा व्यासःक सत्यवतीसुतः}
{नृपं शयानं गाङ्गेयमिदमाह वचस्त्वरन्}


\twolineshloka
{राजन्प्रकृतिमापन्नः कुरुराजो युधिष्ठिरः}
{सहितो भ्रातृभिः सर्वैः पार्तिवैश्चानुयायिभिः}


\twolineshloka
{उपास्ते त्वां नरव्याघ्र सह कृष्णेन धीमता}
{तमिमं पुरयानाय समनुज्ञातुमर्हसि}


\twolineshloka
{एवमुक्तो भगवता व्यासेन पृथिवीपतिः}
{युधिष्ठिरं सहामात्यमनुजज्ञे नदीसुतः}


\twolineshloka
{उवाच चैनं मधुरं नृपं शान्तनवो नृपः}
{प्रविशस्व पुरीं राजन्धर्मे च ध्रियतां मनः}


\twolineshloka
{यजस्व विविधैर्यजैर्बह्वन्नैः स्वाप्तदक्षिणैः}
{ययातिरिव राजेन्द्र श्रकद्धादमपुरःसरः}


\twolineshloka
{क्षत्रधर्मरतः पार्थ पितॄन्देवांश्च तर्पय}
{श्रेयसा योक्ष्यसे चैव व्येतु ते मानसो ज्वरः}


\twolineshloka
{रञ्जयस्व प्रजाः सर्वाः प्रकृतीः परिसान्त्वय}
{सुहृदः फलसत्कारैरर्चयस्व यथार्हतः}


\twolineshloka
{अनुं त्वां तात जीवन्तु मित्राणि सुहृकदस्तथा}
{चैत्यस्थाने स्थितं वृक्षं फलवन्तमिव द्विजाः}


\twolineshloka
{आगन्तव्यं च भवता समये मम पार्थिव}
{विनिवृत्ते दिनकरे प्रवृत्ते चोत्तरायणे}


\twolineshloka
{तथेत्युक्त्वा च कौन्तेयः सोभिवाद्य पितामहम्}
{प्रययौ सपरीवारो नगरं नागसाह्वयम्}


\twolineshloka
{धृतराष्ट्रं पुरस्कृत्य गान्धारीं च पतिव्रताम्}
{सह तैर्ऋषिभिः सर्वैर्भ्रातृभिः केशवेन च}


\twolineshloka
{पौरजानपदैश्चैव मन्त्रिवृद्धैश्च पार्थिव}
{प्रविवेश कुरुश्रेष्ठः पुरं वारणसाह्वयम्}


\chapter{अध्यायः २७३}
\twolineshloka
{ततः कुन्तीसुतो राजा पौरजानपदं जनम्}
{पूजयित्वा यथान्यायमनुजज्ञे गृहान्प्रति}


\twolineshloka
{सान्त्वयामास नारीश्च हतपुत्रा हतेश्वराः}
{विपुलैरर्थदानैः स तदा पाण्डुसुतो नृपः}


\twolineshloka
{सोभिषिक्तो महाप्राज्ञः प्राप्य राज्यं युधिष्ठिरः}
{अवस्थाप्य नरश्रेष्ठः सर्वाः स्वप्रकृतीस्तथा}


\twolineshloka
{द्विजेभ्यो बलमुख्येभ्यो नैगमेभ्यश्च सर्वशः}
{प्रतिगृह्याशिषो मुख्यास्तथा धर्मभृतांवरः}


\twolineshloka
{उषित्वा शर्वरीः श्रीमान्पञ्चाशन्नगरोत्तमे}
{समयं कौरवाग्र्यस्य सस्मार पुरुषर्षभः}


\twolineshloka
{स निर्ययौ जगपुराद्यजकैः परिवारितः}
{दृष्ट्वा निवृत्तमादित्यं प्रवृत्तं चोत्तरायणम्}


\twolineshloka
{घृतं माल्यं च गन्धांश्च क्षौमाणि च युधिष्ठिरः}
{चन्दनागरुमुख्यानि तथा कालगरूणि च}


\twolineshloka
{प्रस्थाप्यच पूर्वं कौन्तेयो भीष्मसंस्करणाय वै}
{माल्यानि च वरार्हाणि रत्नानि विविधानि च}


\twolineshloka
{धृतराष्ट्रं पुरुस्कृत्य गान्धारीं च यशस्विनीम्}
{मातरं च पृथां धामान्भ्रातॄंश्च पुरुषर्षंभान्}


\twolineshloka
{जनार्दनेनानुगतो विदुरेण च धीमता}
{युयुत्सुना च कौरव्यो युयुधानेन चाभिभो}


\twolineshloka
{महता राजभोगेन पारिबर्हेण संवृतः}
{स्तूयमानो महातेजा भीष्मिस्याग्नीननुव्रजन्}


\twolineshloka
{निश्चक्राम पुरात्तस्माद्यथा देवपतिस्तथा}
{आससाद कुरुक्षेत्रे ततः शान्तनवं नृपः}


\twolineshloka
{उपास्यमानं व्यासेन पाराशर्येण धीमता}
{नारदेन च राजर्षे देवलेनासितेन च}


\twolineshloka
{हतशिष्टैर्नृपैस्चान्यैर्नानादेशसमागतैः}
{रक्षिभिश्च महात्मानं रक्ष्यमाणं समन्ततः}


\twolineshloka
{शयानं वीरशयने ददर्श नृपतिस्ततः}
{ततो रथादवारोहद्धातृभिः सह धर्मराट्}


\twolineshloka
{अभिवाद्याथ कौन्तेयः पितामहमरिंदमम्}
{द्वैपायनादीन्विप्रांश्च तैश्च प्रत्यभिनन्दितः}


\twolineshloka
{ऋत्विग्भिर्ब्रह्मकल्पैश्च भ्रातृभिः सह धर्मजः}
{आसाद्य शरतल्पस्थमृषिभिः परिवारितम्}


\twolineshloka
{अब्रवीद्भरतश्रेष्ठं धर्मराजो युधिष्ठिरः}
{भ्रातृभिः सह कौरव्यः शयानं निम्नगासुतम्}


\twolineshloka
{युधिष्ठिरोऽहं नृपते नमस्ते जाह्नवीसुत}
{शृणोषि चेन्महाबाहो ब्रूहि किं करवाणि ते}


\twolineshloka
{प्राप्तोस्मि समये राजन्नग्नीनादाय ते विभो}
{आचार्यान्ब्राह्मणांश्चैव ऋत्विजो भ्रातरश्च मे}


\twolineshloka
{पुत्रश्च ते महातेजा धृतराष्ट्रो जनेश्वरः}
{उपस्थितः सहामात्यो वासुदेवश्च वीर्यवान्}


\twolineshloka
{हतशिष्टाश्च राजानः सर्वे च कुरुजाङ्गलाः}
{तान्पश्य नरशार्दूल समुन्मीलय लोचने}


\threelineshloka
{यच्चेह किञ्चित्कर्तव्यं तत्सर्वं प्रापितं मया}
{यथोक्तं भवता काले सर्वमेव च तत्कृतम् ॥वैशम्पायन उवाच}
{}


\twolineshloka
{एवमुक्तस्तु गाङ्गेयः कुन्तीपुत्रेण धीमता}
{ददर्श भारतान्सर्वान्स्थितान्सम्परिवार्य ह}


\twolineshloka
{ततश्चलवपुर्भीष्मः प्रगृह्य विपुलं भुजम्}
{ओघमेघस्वरो वाग्मी काले वचनमब्रवीत्}


\twolineshloka
{दिष्ट्या प्राप्तोसि कौन्तेय सहामात्यो युधिष्ठिर}
{परिवृत्तो हि भगवान्सहस्रांशुर्दिवाकरः}


\twolineshloka
{अष्टपञ्चाशतं रात्र्यः शयानस्याद्य मे गताः}
{शरेषु निशिताग्नेषु यथा वर्षशतं तथा}


\twolineshloka
{माघोऽयं समनुप्राप्तो मासः पुण्यो युधिष्ठिर}
{त्रिभागशेषः पक्षोऽयं शुक्लो भवितुमर्हति}


\threelineshloka
{एवमुक्त्वा तु गाङ्गेयो धर्मपुत्रं युधिष्ठिरम्}
{धृतराष्ट्रमथामन्त्र्य काले वचनमब्रवीत् ॥भीष्म उवाच}
{}


\twolineshloka
{राजन्विदितधर्मोसि सुनिर्णीतार्थसंशयः}
{बहुश्रुता हि ते विप्रा बहवः पर्युपासिताः}


\twolineshloka
{वेद शास्त्राणि सूक्ष्माणि धर्मांश्च मनुजेश्वर}
{वेदांश्च चतुरः सर्वान्षडङ्गैरुपबृंहितान्}


\twolineshloka
{न शोचितव्यं कौरव्य भवितव्यं हि तत्तथा}
{श्रुतं देवरहस्यं ते कृष्णद्वैपायनादपि}


\twolineshloka
{यथा पाण्डोः सुता राजंस्तथैव तव धर्मतः}
{तान्पालय स्थितो धर्मे गुरुशुश्रूषणे रतान्}


\twolineshloka
{धर्मराजो हि शुद्धात्मा निदेशे स्थास्यते तव}
{आनृशंस्यपरं ह्येन जानामि गुरुवत्सलम्}


\threelineshloka
{तव पुत्रा दुरात्मानः क्रोधलोभपरायणाः}
{ईर्ष्याभिभूता दुर्वृत्तास्तान्न शोचितुमर्हसि ॥वैशम्पायन उवाच}
{}


\threelineshloka
{एतावधुक्त्वा वचनं धृतराष्ट्रं मनीषिणम्}
{वासुदेवं महाबाहुमभ्यभाषय कौरवः ॥भीष्म उवाच}
{}


\twolineshloka
{भगवन्देवदेवेश सुरासुरनमस्कृत}
{त्रिविक्रम नमस्तुभ्यं शङ्खचक्रगदाधर}


\twolineshloka
{वासुदेवो हिरण्यात्मा पुरुषः सविता विराद्}
{जीवभूतोऽनुरूपस्त्वं परमात्मा सनातनः}


\threelineshloka
{त्वद्भक्तं त्वद्गतिं शान्तमुदारमपरिग्रहम्}
{त्रायस्व पुण्डरीकाक्ष पुरुषोत्तम नित्यशः}
{अनुजानीहि मां कृष्ण वैकुण्ठ पुरुषोत्तम}


\twolineshloka
{रक्ष्याश्च ते पाण्डवेया भवान्येषां परायणम्}
{उक्तवानस्मि दुर्बद्धिं मन्दं दुर्योधनं तदा}


\twolineshloka
{यतः कृष्णस्ततो धर्मो यतो धर्मस्ततो जयः}
{वासुदेवेन तीर्थेन पुत्र संशाम्य पाण्डवैः}


\threelineshloka
{सन्धानस्य परः कालस्तवेति च पुनःपुनः}
{न च मे तद्वचो मूढः कृतवान्स सुमन्दधीः}
{घातयित्वेह पृथिवीं ततः स निधनं गतः}


\twolineshloka
{त्वां तु जानाम्यहं देवं पुराणमृषिसत्तमम्}
{नरेण सहितं देव बदर्यां सुचिरोषितम्}


\twolineshloka
{तथा मे नारदः प्राह व्यासश्च सुमहातपाः}
{नरनारायणावेतौ सम्भूतौ मनुजेष्विति}


\threelineshloka
{स मां त्वमनुजानीहि कृष्ण मोक्ष्ये कलेवरम्}
{त्वयाऽहं समनुज्ञातो गच्छेयं परमां गतिम् ॥वासुदेव उवाच}
{}


\twolineshloka
{अनुजानामि भीष्म त्वां वसून्प्राप्नुहि पार्थिव}
{न तेऽस्ति वृजिनं किञ्चिच्छुद्धात्मैश्वर्यसंयुतः}


\threelineshloka
{पितृभक्तोसि राजर्षे मार्कण्डेय इवापरः}
{तेन मृत्युस्तव वशे स्थितो भृत्य इवानतः ॥वैशम्पायन उवाच}
{}


\twolineshloka
{एवमुक्तस्तु गाङ्गेयः पाण्डवानिदमब्रवीत्}
{धृतराष्ट्रमुखांश्चापि सर्वांश्च सुहृदस्तथा}


\twolineshloka
{प्राणानुत्स्रष्टुमिच्छामि तत्रानुज्ञातुमर्हथ}
{सत्येषु यतितव्यं वः सत्यं हि परमं बलम्}


\twolineshloka
{आनृशंस्यपरैर्भाव्यं सदैव नियतात्मभिः}
{ब्रह्मण्यैर्धर्मशीलैश्च तपोनित्यैश्च भारताः}


\threelineshloka
{इत्युक्त्वा सुहृदः सर्वांस्तथा सम्पूज्य चैव ह}
{`धनं बहुविधं राजन्दत्त्वा नित्यं द्विजातिषु}
{'पुनरेवाब्रवीद्धीमान्युधिष्ठिरमिदं वचः}


\twolineshloka
{ब्राह्मणाश्चैव ते नित्यं प्राज्ञाश्चैव विशेषतः}
{आचार्या ऋत्विजश्चैव पूजनीया जनाधिप}


\chapter{अध्यायः २७४}
\twolineshloka
{एवमुक्त्वा कुरून्सर्वान्भीष्मः शान्तनवस्तदा}
{तूष्णीं बभूव कौरव्यः स मुहूर्तमरिंदम}


\twolineshloka
{धारयामास चात्मानं धारणासु यथाक्रमम्}
{तस्योर्ध्वमगमन्प्राणाः सन्निरुद्धा महात्मनः}


\twolineshloka
{इदमाश्चर्यमासीच्च मध्ये तेषां महात्मनाम्}
{सहितैर्ऋषिभिः सर्वैस्तदा व्यासादिभिः प्रभो}


\threelineshloka
{यद्यन्मुञ्चति गात्रं हि स शन्तनुसुतस्तदा}
{तत्तद्विशल्यमभवद्योगयुक्तस्य वै क्रमात्}
{क्षणेन प्रेक्षतां तेषां विशल्यः सोऽभवत्तदा}


\twolineshloka
{तद्दृष्ट्वा विस्मिताः सर्वे वासुदेवपुरोगमाः}
{सह तैर्मुनिभिः सर्वैस्तदा व्यासादिभिर्नृप}


\twolineshloka
{सन्निरुद्धस्तु तेनात्मा सर्वेष्वायतनेषु च}
{जगाम भित्त्वा मूर्धानं दिवमभ्युत्पपात ह}


\twolineshloka
{देवदुन्दुभिनादश्च पुष्पवर्षैः सहाभवत्}
{सिद्धा ब्रह्मर्षयश्चैव सादुसाध्विति हर्षिताः}


\twolineshloka
{महोल्केव च भीष्मस्य मूर्धदेशाज्जनाधिप}
{निःसृकत्याकाशमाविश्य क्षणेनान्तरधीयत}


\twolineshloka
{एवं स राजशार्दूल नृपः शान्तनवस्तदा}
{समयुज्यत लोकैः स्वैर्भरतानां कुलोद्वहः}


\threelineshloka
{ततस्त्वादाय दारूणि गन्धांश्च विविधान्बहून्}
{चितां चक्रुर्महात्मानः पाण्डवा विदुरस्तथा}
{युयुत्सुश्चापि कौरव्यं प्रेक्षकास्त्वितरेऽभवन्}


\twolineshloka
{युधिष्ठिरश्च गाङ्गेयं धृतराष्ट्रश्च दुःखितौ}
{छादयामासतुरुभौ क्षौमेर्माल्यैश्च कौरवम्}


\threelineshloka
{धारयामास तस्याथ युयुत्सुश्छत्रमुत्तमम्}
{चामरे व्यजने शुभ्रे भीमसेनार्जुनावुभौ}
{उष्णीषे परिगृह्णीतां माद्रीपुत्रावुभौ तथा}


\threelineshloka
{युधिष्ठिरेण सहितो धृतराष्ट्रश्च पादतः}
{वृद्धा स्त्रियः कौरवाणां भीष्मं कुरुकुलोद्वहम्}
{तालवृन्तान्युपादाय पर्यवीजन्त सर्वशः}


\twolineshloka
{ततोस्य विदिवच्चक्रुः पितृमेधं महात्मनः}
{याजका जुहुवुश्चाग्नौ जगुः सामानि समागाः}


\twolineshloka
{ततश्चन्दनकाष्ठैश्च तथा कालीयकैरपि}
{कालागुरुप्रभृतिभिर्गन्धैश्चोच्चावच्चैस्तथा}


\twolineshloka
{समवच्छाद्य गाङ्गेयं सम्प्रज्वाल्य हुताशनम्}
{अपसव्यमकुर्वन्त धृतराष्ट्रमुखाश्चिताम्}


\twolineshloka
{संस्कृत्य च कुरुश्रेष्ठं गाङ्गेयं कुरुसत्तमाः}
{जग्मुर्भागीरथीं पुण्यामृषिजुष्टां कुरूद्वहाः}


\twolineshloka
{अनुगम्यमाना व्यासेन नारदेनासितेन च}
{कृष्णेन भरतस्त्रीभिर्ये च पौराः समागताः}


\twolineshloka
{उदकं चक्रिरे सर्वे गाङ्गेयस्य महात्मनः}
{विधिवत्क्षत्रियश्रेष्ठाः स च सर्वो जनस्तदा}


\threelineshloka
{ततो भागीरथी देवी तनयस्योदके कृते}
{उत्थाय सलिलात्तस्माद्रुदती शोकविह्वला}
{परिदेवयती तत्र कौरकवानभ्यभाषत}


\threelineshloka
{निबोधत यथावृत्तमुच्यमानं मया नृपाः}
{राजवृत्तेन सम्पन्नः प्रज्ञयाभिजनेन च}
{सत्कर्ता कुरुवृद्धानां पितृभक्तो दृढव्रतः}


\twolineshloka
{जामदग्न्येन रामेण यः पुरा न पराजितः}
{दिव्यैरस्त्रैर्महावीर्यः स हतोऽद्य शिखण्डिना}


\twolineshloka
{अश्मसारमयं नूनं हृदयं मम पार्थिवाः}
{अपश्यन्त्याः प्रियं पुत्रं यन्न दीर्यति मेऽद्य वै}


\twolineshloka
{समेतं पार्थिवं क्षत्रं काशिपुर्यां स्वयंपरे}
{विजित्यैकरथेनाजौ कन्याश्चायं जहारह}


\twolineshloka
{यस्य नास्ति बले तुल्यः पृथिव्यामपि कश्चन}
{हतं शिखण्डिना श्रुत्वा यन्न दीर्यति मे मनः}


\twolineshloka
{जामदग्न्यः कुरुक्षेत्रे युधि येन महात्मना}
{पीडितो नातियत्नेन स हतोऽद्य शिखण्डिना}


\twolineshloka
{एवंविधं बहु तदा विलन्पतीं नहानदीम्}
{आश्वासयामास तदा साम्ना दामोदरो विभुः}


\twolineshloka
{समाश्वसिहि भद्रे त्वं मा शुचः शुभदर्शने}
{गतः स परमं लोकं तव पुत्रो न संशयः}


\twolineshloka
{वसुरेष महातेजाः शापदोषेण शोभने}
{मानुषत्वमनुप्राप्तो नैनं शोचितुमर्हसि}


\twolineshloka
{स एष क्षत्रधर्मेण युध्यमानो रणाजिरे}
{धनंजयेन निहतो नैष देवि शिखण्डिना}


\twolineshloka
{भीष्मं हि कुरुशार्दूलमुद्यतेषु महारणे}
{न शक्तः संयुगे हन्तुं साक्षादपि शतक्रतुः}


\twolineshloka
{स्वच्छन्दतस्तव सुतो गतः स्वर्गं शुभानने}
{न शक्ता विनिहन्तुं हि रणे तं सर्वदेवताः}


\threelineshloka
{तस्मान्मा त्वं सरिच्छ्रेष्ठे शोचस्य कुरुनन्दनम्}
{वसूनेष गतो देवि पुत्रस्ते विज्वरा भव ॥वैशम्पायन उवाच}
{}


\twolineshloka
{इत्युक्ता सा तु कृष्णेन व्यासेन तु सरिद्वरा}
{त्यक्त्वा शोकं महाराज स्वं वार्यवततार ह}


\twolineshloka
{सत्कृत्य ते तां सरितं ततः कृष्णमुखा नृप}
{अनुज्ञातास्तया सर्वे न्यवर्तन्त जनाधिपाः}


