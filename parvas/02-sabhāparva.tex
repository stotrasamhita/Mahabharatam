\part{सभापर्व}
\chapter{अध्यायः १}
% Check verse!
श्रीवेदव्यासाय नमः
\threelineshloka
{`नारायणं नमस्कृत्य नरं चैव नरोत्तमम्}
{देवीं सरस्वतीं चैव (व्यासं)ततो जयमुदीरयेत् ॥जनमेजय उवाच}
{}


\threelineshloka
{अर्जुनो जयतां श्रेष्ठो मोचयित्वा मयं तदा}
{किं चकार महातेजास्तन्मे ब्रूहि द्विजोत्तम ॥वैशम्पायन उवाच}
{}


\twolineshloka
{शृणु राजन्नवहितश्चरितं पूर्वकस्य ते}
{मोक्षयित्वा मयं तत्र पार्थः शस्त्रभृतां वरः}


\twolineshloka
{गाण्डिवं कार्मुकश्रेष्ठं तूणी चाक्षयसायकौ}
{दिव्यान्यस्त्राणि राजेन्द्र दुर्लभानि नृपैर्भुवि}


\threelineshloka
{रथध्वजपताकाश्च श्वेताश्वांश्च स वीर्यवान्}
{एतानि पावकात्प्राप्य मुदा परमया युतः}
{तस्थौ पार्थो महावीर्यस्तदा सह मयेन सः'}


\twolineshloka
{ततोऽब्रवीन्मयः पार्थं वासुदेवस्य सन्निधौ}
{`पाण्डवेन परित्रातस्तत्कृतं प्रत्यनुस्मरन्'}


\threelineshloka
{प्राञ्जलिः श्लक्ष्णया वाचा पूजयित्वा पुनः पुनः}
{मय उवाच}
{अस्माच्च कृष्णात्सङ्क्रुद्धात्पावकाच्च दिधक्षतः}


\twolineshloka
{त्वया त्रातोऽस्मि कौन्तेय ब्रूहि किं करवाणि ते}
{`अहं हि विश्वकर्मा वै असुराणां परन्तप}


\twolineshloka
{तस्मात्ते विस्मयं किञ्चित्कुर्यामन्यैः सुदुष्करम् ॥वैशम्पायन उवाच}
{}


\twolineshloka
{एवमुक्तो महावीर्यः पार्थो मायाविदं मयम्}
{ध्यात्वा मुहूर्तं कौन्तेयः प्रहसन्वाक्यमब्रवीत्'}


\twolineshloka
{कृतमेव त्वया सर्वं स्वस्ति गच्छ महाऽसुर}
{प्रीतिमान्भव मे नित्यं प्रीतिमन्तो वयं च ते}


\threelineshloka
{प्रोपकारादर्थं हि नादास्यामीति मे व्रतम्'}
{उवाच}
{युक्तमेतत्त्वयि विभो यदात्थ पुरुषर्षभ}


\twolineshloka
{प्रीतिपूर्वमहं किञ्चित्कर्तुमिच्छामि तेऽर्जुन}
{अहं हि विश्वकर्मा वै दानवानां महाकविः}


\twolineshloka
{`सोऽहं वै त्वत्कृते किञ्चित्कर्तुमिच्छामि पाण्डव}
{`दानवानां पुरा पार्थ प्रासादा हि मया कृताः}


\twolineshloka
{रम्याणि सुखगर्भाणि भोगाढ्यानि सहस्रशः}
{उद्यानानि च रम्याणि सरांसि विविधानि च}


\twolineshloka
{विचित्राणि च वस्त्राणि कामगानि रथानि च}
{नगराणि विशालानि साट्टप्राकारवन्ति च}


\fourlineindentedshloka
{वाहनानि च मुख्यानि विचित्राणि सहस्रशः}
{बिलानि रमणीयानि सुखयुक्तानि वै भृशम्}
{एते कृता मया तस्मादिच्छामि फल्गुन' ॥अर्जुन उवाच}
{}


\twolineshloka
{प्राणकृच्छ्राद्विनिर्मुक्तमात्मानं मन्यसे मया}
{एवं गते न शक्ष्यामि किञ्चित्कारयितुं त्वया}


\threelineshloka
{न चापि तव सङ्कल्पं मोघमिच्छामि दानव}
{कृष्णस्य क्रियतां किञ्चित्तथा प्रतिकृतं मयि ॥वैशम्पायन उवाच}
{}


\twolineshloka
{चोदितो वासुदेवस्तु मयं प्रति नरर्षभ}
{मुहूर्तमिव सन्दध्यौ किमयं चोद्यतामिति}


\twolineshloka
{ततो विचिन्त्य मनसा लोकनाथः प्रजापितः}
{चोदयामास तं कृष्णः सभा वै क्रियतामिति}


\twolineshloka
{यदि त्वं कर्तुकामोऽसि प्रियं शिल्पवतां वर}
{धर्मराजस्य दयितां यादृशीमिह मन्यसे}


\twolineshloka
{यां क्रियां नानुकुर्युस्ते मानवाः प्रेक्ष्य विष्ठिताः}
{मनुष्यलोके सकले तादृशीं कुरु वै सभाम्}


\threelineshloka
{यत्र द्विव्यानभिप्रायान्पश्येम विहितांस्त्वया}
{आसुरान्मानुपांश्चैव तादृशीं कुरु वै सभाम् ॥वैशम्पायन उवाच}
{}


\twolineshloka
{प्रतिगृह्य तु तद्वाक्यं सम्प्रहृष्टो मयस्तदा}
{विमानप्रतिमां चक्रे पाण्डवस्य शुभां सभाम्}


\twolineshloka
{ततः कृष्णश्च पार्थश्च धर्मराजे युधिष्ठिरे}
{सर्वमेतत्समावेद्य दर्शयामासतुर्मयम्}


\twolineshloka
{तस्मै युधिष्ठिरः पूजां यथार्हमकरोत्तदा}
{स तु तां प्रतिजग्राह मयः सत्कृत्य भारत}


\twolineshloka
{स पूर्वदेवचरितं तदा तत्र विशाम्पते}
{कथयामास दैतेयः पाण्डुपुत्रेषु भारत}


\twolineshloka
{स कालं कञ्चिदाश्वस्य विश्वकर्मा विचिन्त्य तु}
{सभां प्रचकमे कर्तुं पाण्डवानां महात्मनाम्}


\twolineshloka
{अभिप्रायेण पार्थानां कृष्णस्य च महात्मनः}
{पुण्येऽहनि महातेजाः कृतकौतुकमङ्गलः}


\twolineshloka
{तर्पयित्वा द्विजश्रेष्ठान्पायसेन सहस्रशः}
{धनं बहुविधं दत्त्वा तेभ्य एव च वीर्यवान्}


\twolineshloka
{सर्वर्तुगुणसम्पन्नां दिव्यरूपां मनोरमाम्}
{दशकिष्कुसहस्रां तां मापयामास सर्वतः}


\chapter{अध्यायः २}
\twolineshloka
{उषित्वा खाण्डवप्रस्थे सुखवासं जनार्दनः}
{पार्थैः प्रीतिसमायुक्तैः पूजनार्होऽभिपूजितः}


\twolineshloka
{गमनाय मतिं चक्रे पितुर्दर्शनलालसः}
{धर्मराजमथामन्त्र्य पृथां च पृथुलोचनः}


\twolineshloka
{ववन्दे चरणौ मूर्ध्ना जगद्वन्द्यः पितृष्वसुः}
{स तया मूर्ध्न्युपाघ्रातः परिष्वक्तश्च केशवः}


\twolineshloka
{ददर्शानन्तरं कृष्णो भगिनीं स्वां महायशाः}
{तामुपेत्य हृषीकेशः प्रीत्या बाष्पसमन्वितः}


\twolineshloka
{अर्थ्यं तथ्यं हितं वाक्यं लघु युक्तमनुत्तरम्}
{उवाच भगवान्भद्रां सुभद्रां भद्रभाषिणीम्}


\twolineshloka
{तया स्वजनगामीनि श्रावितो वचनानि सः}
{सम्पूजितश्चाप्यसकृच्छिरसा चाभिवादितः}


\twolineshloka
{तामनुज्ञाय वार्ष्णेयः प्रतिनन्द्य च भामिनीम्}
{ददर्शानन्तरं कृष्णां धौम्यं चापि जनार्दनः}


\twolineshloka
{ववन्दे च यथान्यायं धौम्यं पुरुषसत्तमः}
{द्रौपदीं सान्त्वयित्वा च सुभद्रां परिदाय च}


\twolineshloka
{भ्रातृनभ्यगमद्विद्वान्पार्थेन सहितो बली}
{भ्रातृभिः पञ्चभिः कृष्णो वृतः शक्र इवामरैः}


\twolineshloka
{यात्राकालस्य योग्यानि कर्माणि गरुडध्वजः}
{कर्तुकामः शुचिर्भूत्वा स्नातवान्समलङ्कृतः}


\twolineshloka
{अर्चयामास देवांश्च द्विजांश्च यदुपुङ्गवः}
{माल्यजाप्यनमस्कारैर्गन्धैरुच्चावचैरपि}


\twolineshloka
{स कृत्वा सर्वकार्याणि प्रतस्थे तस्थुपां वरः}
{उपेत्य स यदुश्रेष्ठो बाह्यकक्षाद्विनिर्गतः}


\twolineshloka
{स्वस्ति वाच्यार्हतो विप्रान्दधिपात्रफलाक्षतैः}
{वसु प्रदाय च ततः प्रदक्षिणमथाकरोत्}


\twolineshloka
{काञ्चनं रथमास्थाय तार्क्ष्यकेतनमाशुगम्}
{गदाचक्रासिशार्ङ्गाद्यैरायुधैरावृतं शुभम्}


\twolineshloka
{सुतिथावथ नक्षत्रे मुहूर्ते च गुणान्विते}
{प्रययौ पुण्डरीकाक्षः शैब्यसुग्रीववाहनः}


\twolineshloka
{अन्वारुरोह चाप्येनं प्रेम्णा राजा युधिष्ठिरः}
{अपास्य चास्य यन्तारं दारुकं यन्तृसत्तमम्}


\twolineshloka
{अभीषून्सम्प्रजग्राह स्वयं कुरुपतिस्तदा}
{उपारुह्यार्जुनश्चाऽपि चामरव्यजनं सितम्}


\twolineshloka
{रुक्मदण्डं बृहद्बाहुर्विदुधाव प्रदक्षिणम्}
{तथैव भीमसेनोऽपि रथमारुह्य वीर्यवान्}


\threelineshloka
{`छत्रं शतशलाकं च दिव्यमाल्योपशोभितम्}
{वैडूर्यमणिदण्डं च चामीकरविभूषितम्}
{}


\twolineshloka
{दधार तरसा भीमः मुच्छत्रं शार्ङ्गधन्वने}
{भीमसेनार्जुनौ चापि यमावरिनिषूदनौ'}


\twolineshloka
{पृष्ठतोऽनुययुः कृष्णमृत्विक्पौरजनैर्वृता}
{स तथा भ्रातृभिः सर्वैः केशवः परवीरहा}


\twolineshloka
{अन्वीयमानः शुशुभे शिष्यैरिव गुरुः प्रियैः}
{`अभिमन्युं च सौभद्रं वृद्धैः परिवृतस्तथा}


\threelineshloka
{रथमारोप्य निर्यातो धौम्यो ब्राह्मणपुङ्गवः}
{इन्द्रप्रस्थमतिक्रम्य क्रोशमात्रं महाद्युतिः'}
{पार्थमामन्त्र्य गोविन्दः परिष्वज्य सुपीडितम्}


\twolineshloka
{युधिष्ठरं पूजयित्वा भीमसेनं यमौ तथा}
{परिष्वक्तो भृशं तैस्तु यमाभ्यामभिवादितः}


\twolineshloka
{योजनार्धमथो गत्वा कृष्णः परपुरञ्जयः}
{युधिष्ठिरं समामन्त्र्य निवर्तस्वेति भारत}


\twolineshloka
{ततोऽभिवाद्य गोविन्दः पादौ जग्राह धर्मवित्}
{उत्थाप्य धर्मराजस्तु मूर्ध्न्युपाघ्राय केशवम्}


\twolineshloka
{पाण्डवो यादवश्रेष्ठं कृष्णं कमललोचनम्}
{गम्यतामित्यनुज्ञाप्य धर्मराजो युधिष्ठिरः}


\twolineshloka
{ततस्तैः संविदं कृत्वा यथावन्मधुसूदनः}
{निवर्त्य च तथा कृच्छ्रात्पाण्डवान्सपदानुगान्}


% Check verse!
स्वां पुरीं प्रययौ हृष्टो यथा शक्रोऽमरावतीम् ॥लोचनैरनुजग्मुस्ते तमादृष्टिपथात्तदा
\twolineshloka
{मनोभिरनुजग्मुस्ते कृष्णं प्रीतिसमन्वयात्}
{अतृप्तमनसामेव तेषां केशवदर्शने}


\twolineshloka
{क्षिप्रमन्तर्दधे शौरिश्चक्षुषां प्रियदर्शनः}
{अकामा एव पार्थास्ते गोविन्दगमानसाः}


\twolineshloka
{निवृत्योपययुस्तूर्णं स्वं पुरं पुरुषर्षभाः}
{स्यन्दनेनाथ कृष्णोऽपि त्वरितं द्वारकामगात्}


\fourlineindentedshloka
{सात्वतेन च वीरेण पृष्ठतो यायिना तदा}
{दारुकेण च सूतेन सहितो देवकीसुतः}
{स गतो द्वारकां विष्णुर्गरुत्मानिव वेगवान् ॥वैशम्पायन उवाच}
{}


\twolineshloka
{निवृत्य धर्मराजस्तु सह भ्रातृभिरच्युतः}
{सुहृत्परिवृतो राजा प्रविवेश पुरोत्तमम्}


\twolineshloka
{विसृज्य सुहृदः सर्वान्भ्रातॄन्पुत्रांश्च धर्मराट्}
{मुमोद पुरुषव्याघ्रो द्रौपद्या सहितो नृप}


\twolineshloka
{केशवोपि मुदा युक्तः प्रविवेश पुरोत्तमम्}
{पूज्यमानो यदुश्रेष्ठैरुग्रसेनमुखैस्तथा}


\twolineshloka
{आहुकं पितरं वृद्धं मातरं च यशस्विनीम्}
{अभिवाद्य बलं चैव स्थितः कमललोचनः}


\twolineshloka
{प्रद्युम्नसाम्बनिशठांश्चरुदेष्णं गदं तथा}
{अनिरुद्धं च भानुं च परिष्वज्य जनार्दनः}


\twolineshloka
{स वृद्धैरभ्यनुज्ञातो रुक्मिण्या भवनं ययौ}
{`स तत्र भवने दिव्ये प्रमुमोद सुखी सुखम्}


\twolineshloka
{एतस्मिन्नन्तरे राजन्मयो दैत्याधिपस्तदा}
{विधिवत्कल्पयामास सभां धर्मसुताय वै}


\chapter{अध्यायः ३}
\twolineshloka
{अथाब्रवीन्मयः पार्थमर्युनं जयतां वरम्}
{आपृच्छे त्वां गमिष्यामि पुनरेष्यामि चाप्यहम्}


\threelineshloka
{`विश्रुतां त्रिषु लोकेषु पार्थ दिव्यां सभां तव}
{प्राणिनां विस्मयकरीं तव प्रियविवर्धिनीम्}
{पाण्डवानां च सर्वेषां करिष्यामि धनञ्जय'}


\twolineshloka
{उत्तरेण तु कैलासं मैनाकं पर्वतं प्रति}
{यियक्षमाणेषु पुरा दानवेषु मया कृतम्}


\twolineshloka
{चित्रं मणिमयं भाण्डं रम्यं बिन्दुसारः प्रति}
{सभायां सत्यसन्धस्य यदासीद्वृषपर्वणः}


\twolineshloka
{आगमिष्यामि तद्गृह्य यदि तिष्ठति भारत}
{ततः सभां करिष्यामि पाण्डवस्य यशस्विनीम्}


\twolineshloka
{मनः प्रह्लादिनीं चित्रां सर्वरत्नविभूषिताम्}
{अस्ति बिन्दुसारस्युग्रा गदा च कुरुनन्दन}


\twolineshloka
{निहिता यौवनाश्वेन राज्ञा हत्वा रणे रिपून्}
{सुवर्णबिन्दुभिश्चित्रा गुर्वी भारसहा दृढा}


\twolineshloka
{सा वै शतसहस्रस्य सम्मिता शत्रुघातिनी}
{अनुरूपा च भीमस्य गाण्डीवं भवतो यथा}


\threelineshloka
{वारुणश्च महाशङ्खो देवदत्तः सुघोषवान्}
{सर्वमेतत्प्रदास्यामि भवते नात्र संशयः ॥वैशम्पायन उवाच}
{}


\twolineshloka
{इत्युक्त्वा सोऽसुरः पार्थं प्रागुदीचीं दिशं गतः}
{अथोत्तरेण कैलासान्मैनाकं पर्वतं प्रति}


\twolineshloka
{हिरण्यशृङ्गः सुमहान्महामणिमयो गिरिः}
{रम्यं बिन्दुसरो नाम यत्र राजा भगीरथः}


\twolineshloka
{द्रुष्टुं भागीरथीं गङ्गामुवास बहुलाः समाः}
{यत्रेष्टं सर्वभूतानामीश्वरेण महात्मना}


\twolineshloka
{आहृताः क्रतवो मुख्याः शतं भरतसत्तम}
{यत्र यूपा मणिमयाश्चैत्याश्चापि हिरण्मयाः}


\twolineshloka
{शोभार्थं विहितास्तत्र न तु दृष्टान्ततः कृताः}
{यत्रेष्ट्वा स गतः सिद्धिं सहस्राक्षः शचीपतिः}


\twolineshloka
{यत्र भूतपतिः सृष्ट्वा सर्वाँल्लोकान्सनातनः}
{अपस्यते तिग्मतेजाः स्थितो भूतैः सहस्रशः}


\twolineshloka
{नरनारायणौ ब्रह्मा यमः स्थाणुश्च पञ्चमः}
{उपासते यत्र परं सहस्रयुगपर्यये}


\twolineshloka
{यत्रेष्टं वासुदेवेन सत्त्रैर्वर्षगणान्बहून्}
{श्रद्दधानेन सततं धर्मसम्प्रतिपत्तये}


\twolineshloka
{सुवर्णमालिनो यूपाश्चैत्याश्चाप्यतिभास्वराः}
{ददौ यत्र सहस्राणि प्रयुतानि च केशवः}


\threelineshloka
{तत्र गत्वा स जग्राह गदां शङ्खं च भारत}
{`तस्माद्गिरेरुपादाय शिलाः सुरुचिरावहाः'}
{स्फाटिकं च सभाद्रव्यं यदासीद्वृषपर्वणः}


\twolineshloka
{किङ्करैः सह रक्षोभिर्यदरक्षन्महद्धनम्}
{तदगृह्णान्मयस्तत्र गत्वा सर्वं महाऽसुरः}


\twolineshloka
{तदाहृत्य च तां चक्रे सोऽसुरोऽप्रतिमां सभाम्}
{विश्रुतां त्रिषु लोकेषु दिव्यां मणिमयीं शुभाम्}


\twolineshloka
{गदां च भीमसेनाय प्रवरां प्रददौ तदा}
{देवदत्तं चार्जुनाय शङ्खप्रवरमुत्तमम्}


\twolineshloka
{यस्य शङ्खस्य नादेन भूतानि प्रचकम्पिरे}
{`स कालं कञ्चिदाश्वस्य विश्वकर्मा विचिन्त्य च}


\twolineshloka
{सभां प्रचक्रमे कर्तुं पाण्डवानां महात्मनाम्}
{अभिप्रायेण पार्थानां कृष्णस्य च महात्मनः}


\threelineshloka
{सर्वर्तुगुणसम्पन्नां दिव्यरूपामलङ्कृताम्}
{तर्पयित्वा द्विजश्रेष्ठान्पायसेन सहस्रशः}
{सभा च सा महाराज शातकुम्भमयद्रुमा}


\twolineshloka
{दशकिष्कुसहस्राणि समन्तादायताभवत्}
{यथा वह्नेर्यथार्कस्य सोमस्य च यथा सभा}


\twolineshloka
{भ्राजमाना तथाऽत्यर्थं दधार परमं वपुः}
{अभिघ्नतीव प्रभया प्रभामर्कस्य भास्वराम्}


\threelineshloka
{प्रबभौ ज्वलमानेव दिव्या दिव्येन वर्चसा}
{नवमेघप्रतीकाशा दिवमाकृत्य विष्ठिता}
{आयता विपुला रम्या विपाप्मा विगतक्लमा}


\twolineshloka
{उत्तमद्रव्यसम्पन्ना रत्नप्राकारतोरणा}
{बहुचित्रा बहुधना सुकृता विश्वकर्मणा}


\twolineshloka
{न दाशार्ही सुधर्मा वा ब्रह्मणो वाथ तादृशी}
{सभा रूपेण सम्पन्ना यां चक्रे मतिमान्मयः}


\twolineshloka
{तां स्म तत्र मयेनोक्ता रक्षन्ति च वहन्ति च}
{सभामष्टौ सहस्राणि किङ्करा नाम राक्षसाः}


\twolineshloka
{अन्तरिक्षचरा घोरा महाकाया महाबलाः}
{रक्ताक्षाः पिङ्गलाक्षाश्च शुक्तिकर्णाः प्रहारिणः}


\twolineshloka
{तस्यां सभायां नलिनीं चकाराप्रतिमां मयः}
{वैदूर्यपत्रविततां मणिनालमयाम्बुजाम्}


\threelineshloka
{पद्मसौगन्धिकवतीं नानाद्विजगणायुताम्}
{पुष्पितैः पङ्कजैश्चित्रां कूर्मैर्मत्स्यैश्च काञ्चनैः}
{चित्रस्फटिकसोपानां निष्पङ्कसलिलां शुभाम्}


\twolineshloka
{मन्दानिलसमुद्धूतां मुक्ताबिन्दुभिराचिताम्}
{महामणिशिलापट्टबद्धपर्यन्तवेदिकाम्}


\twolineshloka
{मणिरत्नचितां तां तु केचिदभ्येत्य पार्थिवाः}
{दृष्ट्वापि नाभ्यजानन्त तेऽज्ञानात्प्रपतन्त्युत}


\twolineshloka
{तां सभामभितो नित्यं पुष्पवन्तो महाद्रुमाः}
{आसन्नानाविधा लोलाः शीतच्छाया मनोरमाः}


\twolineshloka
{काननानि सुगन्धीनि पुष्करिण्यश्च सर्वशः}
{हंसकारण्डवोपेताश्चक्रवाकोपशोभिताः}


\twolineshloka
{जलजानां च पद्मानां स्थलजानां च सर्वशः}
{मारुतो गन्धमादाय पाण्डवान्स्म निषेवते}


\twolineshloka
{ईदृशीं तां सभां कृत्वा मासैः परिचतुर्दशैः}
{निष्ठितां धर्मराजाय मयो राजन्न्यवेदयत्}


\chapter{अध्यायः ४}
\twolineshloka
{`तां तु कृत्वा सभां श्रेष्ठां मयश्चार्जुनमब्रवीत्}
{भूतानां च महावीर्यो ध्वजाग्रे किङ्करो गणः}


\twolineshloka
{तव विष्फारघोषेण मेघवन्निनदिष्यति}
{अयं हि सूर्यसङ्काशो ज्वलनस्य रथो महान् ॥ 2}


\twolineshloka
{इमे च दिविजाः श्वेता वीर्यवन्तो हयोत्तमाः}
{मायामयः कृतो ह्येष ध्वजो वानरलक्षणः}


\twolineshloka
{असज्जमानो वृक्षेषु धूमकेतुरिवोच्छ्रितः}
{बहुवर्णं हि लक्ष्येत ध्वजं वानरलक्षणम्}


\threelineshloka
{ध्वजोत्कटं ह्यनवमं युद्धे द्रक्ष्यसि विष्ठितम्}
{एव वीरः सव्यसाचिन्ध्वजस्यान्ते भविष्यति ॥वैशम्पायन उवाच}
{}


\twolineshloka
{इत्युक्त्वाऽऽलिङ्ग्य वीभत्सुं विसृष्टः प्रययौ मयः'}
{}


\twolineshloka
{ततः प्रवेशनं तस्यां चक्रे राजा युधिष्ठिरः}
{अयुतं भोजयित्वा तु ब्राह्मणानां नराधिपः}


\threelineshloka
{साज्येन पायसेनैव मधुना मिश्रितेन च}
{भक्ष्यैर्मूलैः फलैश्चैव मांसैर्वाराहहारिणैः}
{कृसरेणाथ जीवन्त्या हविष्येण च सर्वशः}


\twolineshloka
{मांसप्रकारैर्विविधैः खाद्यैश्चापि तथा नृप}
{चोष्यैश्च विविधै राजन्पेयैश्च बहुविस्तरैः}


\twolineshloka
{अहतैश्चैव वासोभिर्माल्यैरुच्चावचैरपि}
{तर्पयामास विप्रेन्द्रान्नानादिग्भ्यः समागतान्}


\twolineshloka
{ददौ तेभ्यः सहस्राणि गवां प्रत्येकशः पुनः}
{पुण्याहघोषस्तत्रासीद्दिवस्पृगिव भारत}


\twolineshloka
{वादित्रैर्विविधैर्दिव्यैर्गन्धैरुच्चावचैरपि}
{पूजयित्वा कुरुश्रेष्ठो देवतानि निवेश्य च}


\twolineshloka
{तत्र मल्ला नटा झल्लाः सूता वैतालिकास्तथा}
{उपतस्थुर्महात्मानं धर्मपुत्रं युधिष्ठिरम्}


\twolineshloka
{तथा स कृत्वा पूजां तां भ्रातृभिः सह पाण्डवः}
{तस्यां सभायां रम्यायां रेमे शक्रो यथा दिवि}


\twolineshloka
{सभायामृषयस्तस्यां पाण्डवैः सह आसते}
{आसाञ्चक्रुर्नरेन्द्राश्च नानादेशसमागताः}


\twolineshloka
{असितो देवलः सत्यः सर्पिर्माली महाशिराः}
{अर्वा वसुः सुमित्रश्च मैत्रेयः शुनको बलिः}


\twolineshloka
{बको दाल्भ्यः स्थूलशिराः कृष्णद्वैपायनः शुकः}
{सुमन्तुर्जैमिनिः पैलो व्यासशिष्यास्तथा वयम्}


\twolineshloka
{तित्तिरिर्याज्ञवल्क्यश्च ससुतो रोमहर्षणः}
{अप्सुहोम्यश्च धौम्यश्च अणीमाण्डव्यकौशिकौ}


\twolineshloka
{दामोष्णीपस्त्रैबलीश्च पर्णादो घटजानुकः}
{मौञ्जायनो वायुभक्षः पाराशर्यश्च सारिकः}


\twolineshloka
{बलिवाकः सिनीवाकः सप्तपालः कृतश्रमः}
{जातूकर्णः शिखावांश्च आलम्बः पारिजातकः}


\twolineshloka
{पर्वतश्च महाभागो मार्कण्डेयो महामुनिः}
{पवित्रपाणिः सावर्णो भालुकिर्गालवस्तथा}


\twolineshloka
{जङ्घाबन्धुश्च रैभ्यश्च कोपवेगस्तथा भृगुः}
{हरिबभ्रुश्च कौण्डिन्यो बभ्रुमाली सनातनः}


\twolineshloka
{काक्षीवानौशिजश्चैव नाचिकेतोऽथ गौतमः}
{पैङ्ग्यो वराहः शुनकः शाण्डिल्यश्च महातपाः}


\twolineshloka
{कुक्कुरो वेणुजङ्घोऽथ कालापः कठ एव च}
{मुनयो धर्मविद्वांसो धृतात्मानो जितेन्द्रियाः}


\twolineshloka
{एते चान्ये च बहवो वेदवेदाङ्गपारगाः}
{उपासते महात्मानं सभायामृषिसत्तमाः}


\twolineshloka
{कथयन्तः कथाः पुण्या धर्मज्ञाः शुचयोऽमलाः}
{तथैव क्षत्रियश्रेष्ठा धर्मराजमुपासते}


\threelineshloka
{श्रीमान्महात्मा धर्मात्मा मुञ्जकेतुर्विवर्धनः}
{सङ्ग्रामजिद्दुर्मुखश्च उग्रसेनश्च वीर्यवान्}
{}


\twolineshloka
{कक्षसेनः क्षितिपतिः क्षेमकश्चापराजितः}
{कम्बोजराजः कमठः कम्पनश्च महाबलः}


\threelineshloka
{सततं कम्पयामास यवनानेक एव यः}
{बलपौरुषसम्पन्नान्कृतास्त्रानमितौजसः}
{यथाऽसुरान्कालकेयान्देवो वज्रधरस्तथा}


\twolineshloka
{जटासुरो मद्रकानां च राजाकुन्तिः पुलिन्दश्च किरातराजः}
{तथाङ्गवाङ्गौ सहपुण्ड्रकेणपाण्ड्योड्रराजौ च सहान्ध्रकेण}


\twolineshloka
{अङ्गो वङ्गः सुमित्रश्च शैब्यश्चामित्रकर्शनः}
{किरातराजः सुमना यवनाधिपतिस्तथा}


\twolineshloka
{चाणूरो देवरातश्च भोजो भीमरथश्च यः}
{श्रुतायुधश्च कालिङ्गो जयसेनश्च मागधः}


\twolineshloka
{सुकर्मा चेकितानश्च पुरुश्चामित्रकर्शनः}
{केतुमान्वसुदानश्च वैदेहोऽथ कृतक्षणः}


\twolineshloka
{सुधर्मा चानिरुद्धश्च श्रुतायुश्च महाबलः}
{अनूपराजो दुर्धर्पः क्रमजिच्च सुदर्शनः}


\twolineshloka
{शिशुपालः सहसुतः करूपाधिपतिस्तथा}
{वृष्णीनां चैव दुर्धर्पाः कुमारा देवरूपिणः}


\twolineshloka
{आहुको विपृथुश्चैव गदः सारण एव च}
{अक्रूरः कृतवर्मा च सत्यकश्च शिनेः सुतः}


\twolineshloka
{भीष्मकोऽथाकृतिश्चैव द्युमत्सेनश्च वीर्यवान्}
{केकयाश्च महेष्वासा यज्ञसेनश्च सोमकिः}


\threelineshloka
{केतुमान्वसुमांश्चैव कृतास्त्रश्च महाबलः}
{एते चान्ये च बहवः क्षत्रिया मुख्यसंमताः}
{}


\twolineshloka
{उपासते सभायां स्म कुन्तीपुत्रं युधिष्ठिरम्}
{अर्जुनं ये व संश्रित्य राजपुत्रा महाबलाः}


\threelineshloka
{अशिक्षन्त धनुर्वेदं रौरवाजिनवाससः}
{तत्रैव शिक्षिता राजन्कुमारा वृष्णिनन्दनाः}
{}


\twolineshloka
{रौक्मिणेयश्च साम्बश्च युयुधानश्च सात्यकिः}
{सुधर्मा चानिरुद्धश्च शैव्यश्च नरपुङ्गवः}


\twolineshloka
{एते चान्ये च बहवो राजानः पृथिवीपते}
{धनञ्जयसखा चात्र नित्यमास्ते स्म तुम्बुरुः}


\twolineshloka
{उपासते महात्मानमासीनं सप्तविंशतिः}
{चित्रसेनः सहामात्यो गन्धर्वाप्सरसस्तथा}


\twolineshloka
{गीतवादित्रकुशलाः साम्यतालविशारदाः}
{प्रमाणोऽथ लये स्थाने किन्नराः कृतनिश्रमाः}


\twolineshloka
{सञ्चोदितास्तुम्बुरुणा गन्धर्वसहितास्तदा}
{गायन्ति दिव्यतानैस्ते यथान्यायं मनस्विनः}


\twolineshloka
{पाण्डुपुत्रानृषींश्चैव रमयन्त उपासते}
{तस्यां सभायामासीनाः सुव्रताः सत्यसङ्गराः}


% Check verse!
दिवीव देवा ब्रह्माणं युधिष्ठिरमुपासते
\chapter{अध्यायः ५}
\twolineshloka
{अथ तत्रोपविष्टेषु पाण्डवेषु महात्मसु}
{महत्सु चोपविष्टेषु गन्धर्वेषु च भारत}


\twolineshloka
{वेदोपनिषदां वेत्ता ऋषिः सुरगणार्चितः}
{इतिहासपुराणज्ञः क्रियाकल्पविशेषवित्}


\twolineshloka
{`स्तुतस्तोमग्रहस्तोभपदक्रमविभागवित्}
{शिक्षाक्षरविभागज्ञः पुराकल्पविशेषवित्}


\twolineshloka
{आदिकल्पार्थवेत्ता च कल्पसूत्रार्थतत्त्ववित्}
{ब्रह्मचर्यव्रतपर ऊहापोहविशारदः}


\twolineshloka
{नृत्तगान्धर्वसेवी च सर्वत्राप्रतिमस्तथा}
{अष्टादशानां विद्यानां कोशभूतो महाद्युतिः'}


\twolineshloka
{न्यायविद्धर्मतत्त्वज्ञः षडङ्गविदनुत्तमः}
{ऐक्यसंयोगनानात्वसमवायविशारदः}


\twolineshloka
{वक्ता प्रगल्भो मेधावी स्मृतिमान्नयवित्कविः}
{परापरविभागज्ञः प्रमाणकृतनिश्चयः}


\twolineshloka
{पञ्चावयवयुक्तस्य वाक्यस्य गुणदोषवित्}
{उत्तरोत्तरवक्ता च वदतोपि बृहस्पतेः}


\twolineshloka
{धर्मकामार्थमोक्षेषु यथावत्कृतनिश्चयः}
{तथा भुवनकोशस्य सर्वस्यास्य महामतिः}


\twolineshloka
{प्रत्यक्षदर्शी लोकस्य तिर्यगूर्ध्वमधस्तथा}
{साङ्ख्ययोगविभागज्ञो निर्विवित्सुः सुरासुरान्}


\twolineshloka
{सन्धिविग्रहतत्त्वज्ञस्त्वनुमानविभागवित्}
{षाङ्गुण्यविधियुक्तश्च सर्वशास्त्रविशारदः}


\twolineshloka
{युद्धगान्धर्वसेवी च सर्वत्राप्रतिघस्तथा}
{एतैश्चान्यैश्च बहुभिर्युक्तो गुणगणैर्मुनिः}


\twolineshloka
{लोकाननुचरन्सर्वानागमत्तां सभां नृप}
{नारदः सुमहातेजा ऋषिभिः सहितस्तदा}


\twolineshloka
{पारिजातेन राजेन्द्र पर्वतेन च धीमता}
{सुमुखेन च सौम्येन देवर्षिरमितद्युतिः}


\twolineshloka
{सभास्थान्पाण्डवान्द्रष्टुं प्रीयमाणो मनोजवः}
{जयाशीर्भिस्तु तं विप्रो धर्मराजानमार्चयत्}


\twolineshloka
{तमागतमृषिं दृष्ट्वा नारदं सर्वधर्मवित्}
{सहसा पाण्डवश्रेष्ठः प्रत्युत्थायानुजैः सह}


\twolineshloka
{अभ्यवादयत प्रीत्या विनयावनतस्तदा}
{तदर्हमासनं तस्मै सम्प्रदाय यथाविधि}


\twolineshloka
{गां चैव मधुपर्कं च सम्प्रदायार्घ्यमेव च}
{अर्चयामास रत्नैश्च सर्वकामैश्च धर्मवित्}


\fourlineindentedshloka
{तुतोष च यथावच्च पूजां प्राप्य युधिष्ठिरात्}
{सोऽर्चितः पाण्डवैः सर्वैर्महर्षिर्वेदपारगः}
{धर्मकामार्थसंयुक्तं पप्रच्छेदं युधिष्ठिरम् ॥नारद उवाच}
{}


\twolineshloka
{कच्चिदर्थाश्च कल्पन्ते धर्मे च रमते मनः}
{सुखानि चानुभूयन्ते मनश्च न विहन्यते}


\twolineshloka
{कच्चिदाचरितां पूर्वैर्नरदेवपितामहैः}
{वर्तसे वृत्तिमक्षुद्रां धर्मार्थसहितां त्रिषु}


\twolineshloka
{कच्चिदर्थेन वा धर्मं धर्मेणार्थमथापि वा}
{उभौ वा प्रीतिसारेण न कामेन प्रबाधसे}


\twolineshloka
{कच्चिदर्थं च धर्मं च कामं च जयतां वर}
{विभज्य काले कालज्ञः सदा वरद सेवसे}


\twolineshloka
{कच्चिद्राजगुणैः षड्भिः सप्तोपायांस्तथाऽनघ}
{बलाबलं तथा सम्यक्वतुर्दश परीक्षसे}


\twolineshloka
{कच्चिदात्सानमर्न्वाक्ष्य परांश्च जयतां वर}
{तथा सन्धाय कर्माणि अष्टौ भारत सेवसे}


\twolineshloka
{कच्चित्प्रकृतयः सप्त न लुप्ता भरतर्षभ}
{आढ्यास्तथा व्यसनिनः स्वनुरक्ताश्च सर्वशः}


\twolineshloka
{कच्चिन्न कृतकैर्दूतैर्ये चाप्यपरिशङ्किताः}
{त्वत्तो वा तव चामात्यैर्भिद्यते मन्त्रितं तथा}


\twolineshloka
{मित्रोदासीनशत्रूणां कच्चिद्वेत्सि चिकीर्षितम्}
{कच्चित्सन्धिं यथाकालं विग्रहं चोपसेवसे}


\twolineshloka
{कच्चिद्वृत्तिमुदासीने मध्यमे चानुमन्यसे}
{कच्चिदात्मसमा वृद्धाः शुद्धाः सम्बोधनक्षमाः}


\twolineshloka
{कुलीनाश्चानुरक्ताश्च कृतास्ते वीरमन्त्रिणः}
{विजयो मन्त्रमूलो हि राज्ञो भवति भारत}


\twolineshloka
{कच्चित्संवृतमन्त्रैस्ते अमात्यैः शास्त्रकोविदैः}
{राष्ट्रं सुरक्षितं तात शत्रुभिर्न विलुप्यते}


\twolineshloka
{कच्चिन्निद्रावशं नैषि कच्चित्काले विबुद्ध्यसे}
{कच्चिच्चापररात्रेषु चिन्तयस्यर्थमर्थवित्}


\twolineshloka
{कच्चिन्मन्त्रयसे नैकः कच्चिन्न बहुभिः सह}
{कच्चित्ते मन्त्रितो मन्त्रो न राष्ट्रं परिधावति}


\twolineshloka
{कच्चिदर्थान्विनिश्चित्य लघुमूलान्महोदयान्}
{क्षिप्रमारभसे कर्तुं न विघ्नयसि तादृशान्}


\twolineshloka
{कच्चिन्न स्रवे कर्मान्ताः परोक्षास्ते विशङ्किताः}
{सर्वे वा पुनरुत्सृष्टाः संसृष्टं चात्र कारणम्}


\twolineshloka
{आप्तैरलुब्धैः क्रमिकैस्ते च कच्चिदनुष्ठिताः}
{कच्चिद्राजन्कृतान्येव कृतप्रायाणि वा पुनः}


\threelineshloka
{विदुस्ते वीर कर्माणि नानवाप्तानि कानिचित्}
{कच्चित्कारणिका धर्मे सर्वशास्त्रेषु कोविदाः}
{कारयन्ति कुमारांश्च योधमुख्यांश्च सर्वशः}


\twolineshloka
{कच्चित्सहस्रैर्मूर्खाणामेकं क्रीणासि पण्डितम्}
{पण्डितो ह्यर्थकृच्छ्रेषु कुर्यान्नः श्रेयसं परम्}


\twolineshloka
{कच्चिद्दुर्गाणि सर्वाणि धनधान्यायुधोदकैः}
{यन्त्रैश्च परिपूर्णानि तथा शिल्पिधनुर्धरैः}


\twolineshloka
{एकोप्यमात्यो मेधावी शूरो दान्तो विचक्षणः}
{राजानं राजपुत्रं वा प्रापयेन्महतीं श्रियम्}


\twolineshloka
{कच्चिदष्टादशान्येषु स्वपक्षे दश पञ्च च}
{त्रिभिस्त्रिभिरविज्ञातैर्वेत्सि तीर्थानि चारकैः}


\twolineshloka
{कच्चिद्द्विषामविदितः प्रतिपन्नश्च सरवदा}
{नित्ययुक्तो रिपून्सर्वान्वीक्षसे रिपुसूदन}


\twolineshloka
{कच्चिद्विनयसम्पन्नः कुलपुत्रो बहुश्रुतः}
{अनसूयुरनुप्रष्टा सत्कृतस्ते पुरोहितः}


\twolineshloka
{कच्चिदग्निषु ते युक्तो विधिज्ञो मतिमानृजुः}
{हुतं च होष्यमाणं च काले वेदयते सदा}


\twolineshloka
{कच्चिदङ्गेषु निष्णातो ज्योतिपः प्रतिपादकः}
{उत्पातेषु च सर्वेषु दैवज्ञः कुशलस्तव}


\twolineshloka
{कच्चिन्मुख्या महत्स्वेव मध्यमेषु च मध्यमाः}
{जघन्याश्च जघन्येषु भृत्याः कर्मसु योजिताः}


\twolineshloka
{अमात्यानुपधातीतान्पितृपैतामहाञ्शुचीन्}
{श्रेष्ठाञ्श्रेष्ठेषु कच्चित्त्वं नियोजयसि कर्मसु}


\twolineshloka
{कच्चिन्नोग्रेण दण्डेन भृशमुद्विजसे प्रजाः}
{राष्ट्रं तवानुशासन्ति मन्त्रिणो भरतर्षभ}


\twolineshloka
{कच्चित्त्वां नावजानन्ति याजकाः पतितं यथा}
{उग्रप्रतिग्रहीतारं कामयानमिव स्त्रियः}


\threelineshloka
{कच्चिद्धृष्टश्च शूरश्च मतिमान्धृतिमाञ्शुचिः}
{कुलीनश्चानुरक्तश्च दक्षः सेनापतिस्तथा}
{}


\twolineshloka
{कच्चिद्बलस्य ते मुख्याः सर्वयुद्धविशारदाः}
{धृष्टावदाता विक्रान्तास्त्वया सत्कृत्य मानिताः}


\twolineshloka
{कच्चिद्वलस्य भक्तं च वेतनं च यथोचितम्}
{सम्प्राप्तकाले दातव्यं ददासि न विकर्षसि}


\twolineshloka
{कालातिक्रमणादेते भक्तवेतनयोर्भृताः}
{भर्तुः कुप्यन्ति दौर्गत्यात्सोऽनर्थः सुमहान्स्मृतः}


\twolineshloka
{कच्चित्सर्वेऽनुरक्तास्त्वां कुलपुत्राः प्रधानतः}
{कच्चित्प्राणांस्तवार्थेषु सन्त्यजन्ति सदा युधि}


\twolineshloka
{कच्चिन्नैको बहूनर्थान्सर्वशः साम्परायिकान्}
{अनुशास्ति यथाकामं कामात्मा शासनातिगः}


\twolineshloka
{कच्चित्पुरुषकारेण पुरुषः कर्म शोभयन्}
{लभते मानमधिकं भूयो वा भक्तवेतनम्}


\twolineshloka
{कच्चिद्विद्याविनीतांश्च नराञ्ज्ञानविशारदान्}
{यथार्हं गुणतश्चैव दानेनाभ्युपपद्यसे}


\twolineshloka
{कच्चिद्दारान्मनुष्याणां तवार्थे मृत्युमीयुषाम्}
{व्यसनं चाभ्युपेतानां बिभर्षि भरतर्षभ}


\twolineshloka
{कच्चिद्भयादुपगतं क्षीणं वा रिपुमागतम्}
{युद्धे वा विजितं पार्थ पुत्रवत्परिरक्षसि}


\twolineshloka
{कच्चित्त्वमेव सर्वस्याः पृथिव्याः पृथिवीपते}
{समश्चानभिशङ्क्यश्च यथा माता यथा पिता}


\twolineshloka
{कच्चिद्व्यसनिनं शत्रुं निशम्य भरतर्षभ}
{अभियासि जवेनैव समीक्ष्य त्रिविधं बलम्}


\twolineshloka
{यात्रामारभसे दिष्ट्या प्राप्तकालमरिन्दम}
{पार्ष्णिमूलं च विज्ञाय व्यवसायं पराजयम्}


\threelineshloka
{बलस्य च महाराज दत्त्वा वेतनमग्रतः}
{कच्चिच्च बलमुख्येभ्यः परराष्ट्रे परन्तप}
{उपच्छन्नानि रत्नानि प्रयच्छसि यथार्हतः}


\twolineshloka
{कच्चिदात्मानमेवाग्रे विजित्य विजितेन्द्रियः}
{पारञ्जिगीषसे पार्थ प्रमत्तानजितेन्द्रियान्}


\twolineshloka
{कच्चित्ते यास्यतः शत्रून्पर्वं यान्ति स्वनुष्ठिताः}
{साम दानं च भेदश्च दण्डश्च विधिवद्गुणाः}


\twolineshloka
{तांश्च विक्रमसे जेतुं जित्वा च परिरक्षसि ॥कच्चिदष्टाङ्गसंयुक्ता चतुर्विधबला चमूः}
{}


\twolineshloka
{बलमुख्यैः सुनीता ते द्विषतां प्रतिवर्धिनी ॥कच्चिल्लवं च मुष्टिं च परराष्ट्रे परन्तप}
{}


\twolineshloka
{अविहाय महाराज निहंसि समरे रिपून् ॥कच्चित्स्वपरराष्ट्रेषु बहवोऽधिकृतास्तव}
{}


% Check verse!
अर्थान्समधितिष्ठन्ति रक्षन्ति च परस्परम्
\threelineshloka
{कच्चिदभ्यवहार्याणि गात्रसंस्पर्शनानि च}
{घ्रेयाणि च महाराज रक्षन्त्यनुमतास्तव}
{}


% Check verse!
कच्चित्कोशश्च कोष्ठं च वाहनं द्वारमायुधम् ॥आयश्च कृतकल्याणैस्तव भक्तैरनुष्ठितः
\twolineshloka
{कच्चिदाभ्यन्तरेभ्यश्च बाह्येभ्यश्च विशाम्पते ॥रक्षस्यात्मानमेवाग्रे तांश्च स्वेभ्यो मिथश्च तान्}
{}


\twolineshloka
{कच्चिन्न पाने द्यूते वा क्रीडासु प्रमदासु च}
{प्रतिजानन्ति पूर्वाह्णे व्ययं व्यसनजं तव}


\twolineshloka
{कच्चिदायस्य चार्धेन चतुर्भागेन वा पुनः}
{पादभागैस्त्रिभिर्वापि व्ययः संशुद्ध्यते तव}


\twolineshloka
{कच्चिज्ज्ञातीन्गुरून्वृद्धा-न्वणिजः शिल्पिनः श्रितान्}
{अभीक्ष्णमनुगृहाणिसिधनधान्येन दुर्गतान्}


\twolineshloka
{कच्चिच्चायव्यये युक्ताः सर्वे गणकलेखकाः}
{अनुतिष्ठन्ति पूर्वाह्णे नित्यमायं व्ययं तव}


\twolineshloka
{कच्चिदर्थेषु सम्प्रौढान्हितकामाननुप्रियान्}
{नापकर्षसि कर्मभ्यः पूर्वमप्राप्य किल्बिषम्}


\twolineshloka
{कच्चिद्विदित्वा पुरुषानुत्तमाधममध्यमान्}
{त्वं कर्मस्वनुरूपेषु नियोजयसि भारत}


\twolineshloka
{कच्चिन्न लुब्धाश्चौरा वा वैरिणो वा विशाम्पते}
{अप्राप्तव्यवहारा वा तव कर्मस्वनुष्ठिताः}


\twolineshloka
{कच्चिन्न चौरैर्लुब्धैर्वा कुमारैः स्त्रीबलेन वा}
{त्वया वा पीड्यते राष्ट्रं कच्चित्तुष्टाः कृषीवलाः}


\twolineshloka
{कच्चिद्राष्ट्रे तटाकानि पूर्णानि च बृहन्ति च}
{भागशो विनिविष्टानि न कृषिर्देवमातृका}


\twolineshloka
{कच्चिन्नं भक्तं बीजं च कर्षकस्यावसीदति}
{प्रत्येकं च शतं वृद्ध्या ददास्यृणमनुग्रहम्}


\twolineshloka
{कच्चित्स्वनुष्ठिता तात वार्ता ते साधुभिर्जनैः}
{वार्तायां संश्रितस्तात लोकोऽयं सुखमेधते}


\twolineshloka
{कच्चिच्छूराः कृतप्रज्ञाः पञ्चपञ्च स्वनुष्ठिताः}
{क्षेमं कुर्वन्ति संहत्य राजञ्जनपदे तव}


\twolineshloka
{कच्चिन्नगरगुप्त्यर्थं ग्रामा नगरवत्कृताः}
{ग्रामवच्च कृताः प्रान्तास्ते च सर्वे त्वदर्पणाः}


\twolineshloka
{कच्चिद्बलेनानुगताः समानि विषमाणि च}
{पुराणि चौरान्निघ्नन्तश्चरन्ति विषये तव}


\twolineshloka
{कच्चित्स्रियः सान्त्वयसि कच्चित्ताश्च सुरक्षिताः}
{कच्चिन्न श्रद्दधास्यासां कच्चिद्गुह्यं न भाषसे}


\twolineshloka
{कच्चिदात्ययिकं श्रुत्वा तदर्थमनुचिन्त्य च}
{प्रियाण्यनुभवञ्शेषे न त्वमन्तः पुरे नृप}


\twolineshloka
{कच्चिद्द्वौ प्रथमौ यामौ रात्रेः सुप्त्वा विशाम्पते}
{सञ्चिन्तयसि धर्मार्थौ याम उत्थाय पश्चिमे}


\twolineshloka
{कच्चिद्दर्शयसे नित्यं मनुष्यान्समलङ्कृतः}
{उत्थाय काले कालज्ञैः सह पाण्डव मन्त्रिभिः}


\twolineshloka
{कच्चिद्रक्ताम्बरधराः खड्गहस्ताः स्वलङ्कृताः}
{उपासते त्वामभितो रक्षणार्थमरिन्दम}


\twolineshloka
{कच्चिद्दण्ड्येषु यमवत्पूज्येषु च विशाम्पते}
{परीक्ष्य वर्तसे सम्यगप्रियेषु प्रियेषु च}


\twolineshloka
{कच्चिच्छारीरमाबाधमौषधैर्नियमेन वा}
{मानसं वृद्धसेवाभिः सदा पार्थापकर्षसि}


\twolineshloka
{कच्चिद्वैद्याश्चिकित्सायामष्टाङ्गायां विशारदाः}
{सुहृदश्चानुरक्ताश्च शरीरे ते हिताः सदा}


\twolineshloka
{कच्चिन्न लोभान्मोहाद्वा मानाद्वापि विशाम्पते}
{अर्थिप्रत्यर्थिनः प्राप्तान्नापास्यसि कथञ्चन}


\twolineshloka
{कच्चिन्न लोभान्मोहाद्वा विश्रम्भात्प्रणयेन वा}
{आश्रितानां मनुष्याणां वृत्तिं त्वं संरुणात्सि वै}


\twolineshloka
{कच्चित्पौरा न सहिता ये च ते राष्ट्रवासिनः}
{त्वया सह विरुध्यन्ते परैः क्रीताः कथञ्चन}


\twolineshloka
{कच्चिन्न दुर्बलः शत्रुर्बलेन परिपीडितः}
{मन्त्रेण बलवान्कश्चिदुभाभ्यां च कथञ्चन}


\twolineshloka
{कच्चित्सर्वेऽनुरक्तास्त्वां भूमिपालाः प्रधानतः}
{कच्चित्प्राणांस्त्वदर्थेषु सन्त्यजन्ति त्वया हृताः}


\threelineshloka
{कच्चित्ते सर्वविद्यासु गुणतोऽर्चा प्रवर्तते}
{ब्राह्मणानां च साधूनां तव नैः श्रेयसी शुभा}
{दक्षिणास्त्वं ददास्येषां नित्यं स्वर्गापवर्गदाः}


\twolineshloka
{कच्चिद्धर्मे त्रयीमूले पूर्वैराचरिते जनैः}
{यतमानस्तथा कर्तुं तस्मिन्कर्मणि वर्तसे}


\twolineshloka
{कच्चित्तव गृहेऽन्नानि स्वादून्यश्रन्ति वै द्विजाः}
{गुणवन्ति गुणोपेतास्तवाध्यक्षं सदक्षिणम्}


\twolineshloka
{कच्चित्क्रतूनेकचित्तो वाजपेयांश्च सर्वशः}
{पौण्डरीकांश्च कार्त्स्न्येन यतसे कर्तुमात्मवान्}


\twolineshloka
{कच्चिज्ज्ञातीन्गुरून्वृद्धान्दैवतांस्तापसानपि}
{चैत्यांश्च वृक्षान्कल्याणान्ब्राह्मणांश्च नमस्यसि}


\twolineshloka
{कच्चिच्छोको न मन्युर्वा त्वया प्रोत्पाद्यतेऽनघ}
{अपि मङ्गलहस्तश्च जनः पार्श्वे न तिष्ठति}


\twolineshloka
{कच्चिदेषा च ते बुद्धिर्वृत्तिरेषा च तेऽनघ}
{आयुष्या च यशस्या च धर्मकामार्थदर्शिनी}


\twolineshloka
{एतया वर्तमानस्य बुद्ध्या राष्ट्रं न सीदति}
{विजित्य च महीं राजा सोत्यन्तं सुखमेधते}


\twolineshloka
{कच्चिदार्यो विशुद्धात्मा क्षारितश्चौरकर्मणि}
{अदृष्टशास्त्रकुशलैर्न लोभाद्वध्यते शुचि}


\twolineshloka
{दुष्टो गृहीतस्तत्कारितज्ज्ञैर्दृष्टः सकारणः}
{कच्चिन्न मुच्यते स्तेनो द्रव्यलोभान्नरर्षभ}


\twolineshloka
{उत्पन्नानकच्चिदाढ्यस्य दरिद्रस्य च भारत}
{अर्थान्न मिथ्या पश्यन्ति तवामात्या हृता धनैः}


\threelineshloka
{नास्तिक्यमनृतं क्रोधं प्रमादं दीर्घसूत्रताम्}
{अदर्शनं ज्ञानवतामालस्यं पञ्चवृत्तिताम्}
{एकचिन्तनमर्थानामनर्थज्ञैश्च चिन्तनम्}


\twolineshloka
{निश्चितानामनारम्भं मन्त्रस्यापरिरक्षणम्}
{मङ्गलाद्यप्रयोगं च प्रत्युत्थानं च सर्वतः}


\twolineshloka
{कच्चित्त्वं वर्जयस्येतान्राजदोषांश्चतुर्दश}
{प्रायशोयैर्विनश्यन्ति कृतमूलापि पार्थिवः}


\threelineshloka
{कच्चित्ते सफला वेदाः कच्चित्ते सफळं धनम्}
{कच्चित्ते सफला दाराः कच्चित्ते सफलं श्रुतम् ॥युधिष्ठिर उवाच}
{}


\threelineshloka
{कथं वै सफला वेदाः कथं वै सफलं धनम्}
{कथं वै सफला दाराः कथं वै सफलं श्रुतम् ॥नारद उवाच}
{}


\threelineshloka
{अग्निहोत्रफला वेदा दत्तभुक्तफलं धनम्}
{रतिपुत्रफला दाराः शीलवृत्तफलं श्रुतम् ॥वैशम्पायन उवाच}
{}


\threelineshloka
{एतदाख्याय स मुनिर्नारदो वै महातपाः}
{पप्रच्छानन्तरमिदं धर्मात्मानं युधिष्ठिरम् ॥नारद उवाच}
{}


\twolineshloka
{कच्चिदभ्यागता दूराद्वणिजो लाभकारणात्}
{यथोक्तमवहार्यन्ते शुल्कं शुल्कोपजीविभिः}


\twolineshloka
{कच्चित्ते पुरुषा राजन्पुरे राष्ट्रे च मानिताः}
{उपानयन्ति पण्यानि उपाधाभिरवञ्चिताः}


\twolineshloka
{कच्चिच्छृणोषि वृद्धानां धर्मार्थसहिता गिरः}
{नित्यमर्थविदां तात यथाधर्मार्थदर्शिनाम्}


\twolineshloka
{कच्चित्ते कृषितन्त्रेषु गोषु पुष्पफलेषु च}
{धर्मार्थं च द्विजातिभ्यो दीयेते मधुसर्पिषी}


\twolineshloka
{द्रव्योपकरणं किञ्चित्सर्वदा सर्वशिल्पिनाम्}
{चातुर्मास्यावरं सम्यङ्नियतं सम्प्रयच्छसि}


\twolineshloka
{कच्चित्कृतं विजानीषे कर्तारं च प्रशंससि}
{सतां मध्ये महाराज सत्करोषि च पूजयन्}


\twolineshloka
{कच्चित्सूत्राणि सर्वाणि गृह्णासि भरतर्षभ}
{हस्तिसूत्राश्वसूत्राणि रथसूत्राणि वा विभो}


\twolineshloka
{कच्चिदभ्यस्यते सम्यग्गृहे ते भरतर्षभ}
{धनुर्वेदस्य सूत्रं वै यन्त्रसूत्रं च नागरम्}


\twolineshloka
{कच्चिदस्त्राणि सर्वाणि ब्रह्मदण्डश्च तेऽनघ}
{विषयोगास्तथा सर्वे विदिताः शत्रुनाशनाः}


\twolineshloka
{कच्चिदग्निभयाच्चैव सर्वं व्यालभयात्तथा}
{रोगरक्षोभयाच्चैव राष्ट्रं स्वं परिरक्षशि}


\twolineshloka
{कच्चिदन्धांश्च मूकांश्च पङ्गून्व्यङ्गानबान्धवान्}
{पितेव पासि धर्मज्ञ तथा प्रव्रजितानपि}


\threelineshloka
{षडवर्था महाराज कच्चित्ते पृष्ठतः कृताः}
{निद्राऽऽलस्यं भयं क्रोधो मार्दवं दीर्घसूत्रता ॥वैशम्पायन उवाच}
{}


\twolineshloka
{ततः कुरूणामृषभो महात्माश्रुत्वा गिरो ब्राह्मणसत्तमस्य}
{प्रणम्य पादावभिवाद्य तुष्टोराजाऽब्रवीन्नारदं देवरूपम्}


\threelineshloka
{एवं करिष्यामि यथा त्वयोक्तंप्रज्ञा हि मे भूय एवाभिवृद्धा}
{उक्त्वा तथा चैव चकार राजालेभे महीं सागरमेखलां च ॥नारद उवाच}
{}


\twolineshloka
{एवं यो वर्तते राजा चातुर्वर्ण्यस्य रक्षणे}
{स विहृत्येह सुसुखी शुक्रस्यैति सलोकताम्}


\chapter{अध्यायः ६}
\twolineshloka
{सम्पूज्याथाभ्यनुज्ञातो महर्षेर्वचनात्परम्}
{प्रत्युवाचानुपूर्व्येण धर्मराजो युधिष्ठिरः}


\twolineshloka
{भगवत्याय्यमाहैतं यथावद्धर्मनिश्चयम्}
{यथाशक्ति यथान्यायं क्रियतेऽयं विधिर्मया}


\twolineshloka
{राजभिर्यद्यथा कार्यं, पुरा वै तन्न संशयः}
{यथान्यायोपनीतार्थं कृतं हेतुमदर्थवत्}


\threelineshloka
{वयं तु सत्पथं तेषां यातुमिच्छामहे प्रभो}
{न तु शक्यं तथा गन्तुं यथा तैर्नियतात्मभिः ॥वैशम्पायन उवाच}
{}


\twolineshloka
{एकमुक्त्वा स धर्मात्मा वाक्यं तदभिपूज्य च}
{` तं तु विश्रान्तमासीनं देवर्षिममितद्युतिम्' ॥मुहूर्तात्प्राप्तकालं च दृष्ट्वा लोकचरं मुनिम्}


\twolineshloka
{नादरदं सुस्थमासीनमुपासीनो युधिष्ठिरः}
{अपृच्छत्पाण्डवस्तत्र राजमध्ये माहद्युतिः}


\threelineshloka
{भवात्सञ्चरते लोकान्सदा नानाविधान्बहून्}
{ब्रह्मणा निर्मितान्पूर्वं प्रेक्षमाणो मनोजवः}
{}


\threelineshloka
{ईदृशी भविता काचिद्दृष्टपूर्वा सभा क्वचित्}
{इतो वा श्रेयसी ब्रह्मंस्तन्ममाचक्ष्व पृच्छतः ॥वैशम्पायन उवाच}
{}


\threelineshloka
{तच्छ्रुत्वा नारदस्तस्य धर्मराजस्य भाषितम्}
{पाण्डवं प्रत्युवाचेदं स्मयन्मधुरया गिरा ॥नारद उवाच}
{}


\twolineshloka
{मानुषेषु न मे तात दृष्टपूर्वा न च श्रुता}
{सभा मणिमयी राजन्यथेयं तव भारत}


\twolineshloka
{सभां तु पितृराजस्य वरुणस्य च धीमतः}
{कथयिष्ये तथेन्द्रस्य कैलासनिलयस्य च}


\twolineshloka
{ब्रह्मणश्च सभां दिव्यां कथयिष्ये गतक्लमाम्}
{दिव्यादिव्यैरभिप्रायैरुपेतां विश्वरूपिणीम्}


\twolineshloka
{देवैः पितृगणैः साध्यैर्यज्वभिर्नियतात्मभिः}
{जुष्टां मुनिगणैः शान्तैर्वेदयज्ञैः सदक्षिणैः ॥यदि ते श्रवणे बुद्धिर्वर्तते भरतर्षभ}


\twolineshloka
{नारदेनैवमुक्तस्तु धर्मराजो युधिष्ठिरः}
{प्राञ्जलिर्भ्रातृभिः सार्धं तैश्च सर्वैर्द्विजोत्तमैः}


\twolineshloka
{नारदं प्रत्यवाचेदं धर्मराजो महामनाः}
{सभाः कथय ताः सर्वाः श्रोतुमिच्छामहे वयम्}


\twolineshloka
{किन्द्रव्यास्ताः सभा ब्रह्मन्किंविस्ताराः किमायताः}
{पितामहं च के तस्यां सभायां पर्युपासते}


\twolineshloka
{वासवं देवराजं च यमं वैवस्वतं च के}
{वरुणं च कुबेरं च सभायां पर्युपासते}


\twolineshloka
{एतत्सर्वं यथान्यायं ब्रह्मर्षे वदतस्तव}
{श्रोतुमिच्छाम सहिताः परं कौतूहलं हि नः}


\twolineshloka
{एवमुक्तः पाण्डवेन नारदः प्रत्यभाषत}
{क्रमेण राजन्दिव्यास्ताः श्रूयन्तामिह नः सभाः}


\chapter{अध्यायः ७}
\twolineshloka
{शक्रस्य तु सभा दिव्या भास्वरा कर्मनिर्मिता}
{स्वयं शक्रेण कौरव्य निर्जितार्कसमप्रभा}


\threelineshloka
{विस्तीर्णा योजनशतं शतमध्यर्धमायता}
{वैहायसी कामगमा पञ्चयोजनमुच्छ्रिता}
{}


\twolineshloka
{जराशोकक्लमापेता निरातङ्का शिवा शुभा}
{वेश्मासनवती रम्या दिव्यपादपशोभिता}


\threelineshloka
{तस्यां देवेश्वरः पार्थ सभायां परमासने}
{आस्ते शच्या महेन्द्राण्या श्रिया लक्ष्म्या च भारत}
{}


\twolineshloka
{बिभ्रद्वपुरनिर्देश्यं किरीटी लोहिताङ्गदः}
{विरजोम्बरश्चित्रमाल्यो ह्रीकीर्तिद्युतिभिः सह}


\twolineshloka
{तस्यामुपासते नित्यं महात्मानं शतक्रतुम्}
{मरुतः सर्वशो राजन्सर्वे च गृहमेधिनः}


\twolineshloka
{सिद्धा देवर्षयश्चैव साध्या देवगणास्तथा}
{मरुत्त्वन्तश्च सहिता भास्वन्तो हेममालिनः}


\twolineshloka
{एते सानुचराः सर्वे दिव्यरूपाः स्वलङ्कृताः}
{उपासते महात्मान देवराजमरिन्दमम्}


\twolineshloka
{तथा देवर्षयः सर्वे पार्ते शक्रमुपासते}
{अमला धूतपाप्मानो दीप्यमाना इवाग्नयः}


\twolineshloka
{तेजस्विनः सोमसुतो विशोका विगतज्वराः}
{पराशरः पर्वतश्च तथा सावर्णिगालवौ}


\threelineshloka
{`एकतश्च द्वितश्चैव त्रितश्चैव महामतिः'}
{शङ्खश्च लिखितश्चैव तथा गौरशिरा मुनिः}
{दुर्वासाः क्रोधनः श्येनस्तथा दीर्घतमा मुनिः}


\twolineshloka
{पवित्रपाणिः सावर्णिर्याज्ञवल्क्योऽथ भालुकिः}
{उद्दालकः श्वेतकेतुस्ताण्डो भाण्डायनिस्तथा}


% Check verse!
हविष्मांश्च गरिष्ठश्च हरिश्चन्द्रश्च पार्थिवः ॥हृद्यश्चोदरशाण्डिल्यः पाराशर्यः कृषीवलः
\twolineshloka
{वातस्कन्धो विशाखश्च विधाता काल एव च}
{करालदन्तस्त्वष्टा च विश्वकर्मा च तुम्बुरुः}


\twolineshloka
{अयोनिजा योनिजाश्च वायुभक्षा हुताशिनः}
{ईशानं सर्वलोकस्य वज्रिणं समुपासते}


\twolineshloka
{सहदेवः सुनीथश्च वाल्मीकिश्च महातपाः}
{समीकः सत्यवाक्चैव प्रचेताः सत्यसङ्गरः}


\twolineshloka
{मेधातिथिर्वामदेवः पुलस्त्यः पुलहः क्रतुः}
{मरुत्तश्च मरीचिश्च स्थाणुश्चात्र महातपाः}


\twolineshloka
{कक्षीवान्गौतमस्तार्क्ष्यस्तथा वैश्वानरो मुनिः}
{मुनिः कालकवृक्षीय आश्राव्योऽथ हिरण्मयः}


\twolineshloka
{संवर्तो देवहव्यश्च विष्वक्सेनश्च वीर्यवान्}
{दिव्या आपस्तथौषध्यः श्रद्धा मेधा सरस्वती}


\twolineshloka
{अर्थो धर्मश्च कामश्च विद्युतश्चैव पाण्डव}
{जलवाहास्तथा मेघा वायवः स्तनयित्नवः}


\twolineshloka
{प्राचीदिग्यज्ञवाहाश्च पावकाः सप्तविंशतिः}
{अग्नीषोमौ तथेन्द्राग्नी मित्रश्च सविताऽर्यमा}


\twolineshloka
{भगो विश्वे साध्याश्च गुरुः शुक्रस्तथैव च}
{विश्वावसुश्चित्रसेनः सुमनस्तरुणस्तथा}


\twolineshloka
{यज्ञाश्च दक्षिणाश्चैवं ग्रहास्तोभाश्च भारत}
{यज्ञवाहाश्च ये मन्त्राः सर्वे तत्र समासते}


\twolineshloka
{तथैवाप्सरसो राजन् `रम्भोर्वश्यथ मेनका}
{घृताची पञ्चचूडा च विप्रचित्तिपुरोगमाः}


\twolineshloka
{विद्याधराश्च राजेन्द्र' गन्धर्वाश्च मनोरमाः}
{नृत्यवादित्रगीतैश्च हास्यैश्च विविधैरपि}


\twolineshloka
{रमयन्ति स्म नृपते देवराजं शतक्रतुम्}
{स्तुतिभिर्मङ्गलैश्चैव वस्तुवन्तः कर्मभिस्तथा}


\twolineshloka
{विक्रमैश्च महात्मानं बलवृत्रनिषूदनम्}
{ब्रह्मराजर्षयश्चैव सर्वे देवर्षयस्तथा}


\twolineshloka
{विमानैर्विविधैर्दिव्यैर्दीप्यमाना इवाग्नयः}
{स्रग्विणो भूषिताः सर्वे यान्ति चायान्ति चापरे}


% Check verse!
बृहस्पतिश्च शुक्रश्च नित्यमास्तां हि तत्र वै ॥एते चान्ये च बहवो महात्मानो यतव्रताः
\twolineshloka
{विमानैश्चन्द्रसङ्काशैः सोमवत्प्रियदर्शनाः}
{ब्रह्मणो वचनाद्राजन्भृगुः सप्तर्षयस्तथा}


\twolineshloka
{एषा सभा मया राजन्दृष्टा पुष्करमालिनी}
{शतक्रतोर्महाबाहो याम्यामपि सभां शृणु}


\chapter{अध्यायः ८}
\twolineshloka
{कथयिष्ये सभां याम्यां युधिष्ठिर निबोध ताम्}
{वैवस्वतस्य यां पार्थ विश्वकर्मा चकार ह}


\twolineshloka
{तैजसी सा सभा राजन्बभूव शतयोजना}
{विस्तारायामसम्पन्ना भूयसी चापि पाण्डव}


\twolineshloka
{अर्कप्रकाशा भ्राजिष्णुः सर्वतः कामरूपिणी}
{नातिशीता च चात्युष्णा मनसश्च प्रहर्षिणी}


\twolineshloka
{न शोको न जरा तस्यां क्षुत्पिपासे न चाप्रियम्}
{न च दैन्यं क्लमो वाऽपि प्रतिकूलं न चाप्युत}


\twolineshloka
{सर्वे कामाः स्थितास्तस्यां ये दिव्या ये च मानुषाः}
{रसवच्च प्रभूतं च भक्ष्यं भोज्यमरिन्दम}


\twolineshloka
{लेह्यं चोप्यं च पेयं च हृद्यं स्वादु मनोहरम्}
{पुण्यगन्धाः स्रजस्तस्य नित्यं कामफला द्रुमाः}


\twolineshloka
{रसवन्ति च तोयानि शीतान्युष्णानि चैव हि}
{तस्यां राजर्षयः पुण्यास्तथा ब्रह्मर्षयोऽमलाः}


\twolineshloka
{यमं वैवस्वतं तात प्रहृष्टाः पर्युपासते}
{ययातिर्नहुषः पुरुर्मान्धाता सोमको नुगः}


\twolineshloka
{त्रसदस्युश्च राजर्षिः कृतवीर्यः श्रुतश्रवाः}
{अरिष्टनेमिः सिद्धश्च कृतवेगः कृतिर्निमिः}


\twolineshloka
{प्रतर्दनः शिबिर्मत्स्यः पृथुलाक्षो बृहद्रथः}
{वार्तो मरुत्तः कुषिकः साङ्काश्यः साङ्कृतिर्ध्रुवः}


\twolineshloka
{चतुरश्चः सदस्योर्मिः कार्तवीर्यश्च पार्थिवः}
{भरतः सुरथश्चैव सुनीथो निशठोऽनलः}


\twolineshloka
{दिवोदासश्च सुमना अम्बरीषो भगीरथः}
{व्यश्वः सदश्वो वाघ्र्यश्वः पृथुवेगः पृथुश्रवाः}


\twolineshloka
{पृपदश्वो वसुमनाः क्षुपश्च सुमहाबलः}
{रुषद्रुर्वृषसेनश्च पुरुकुत्सो ध्वजी रथी}


\twolineshloka
{आर्ष्टिषेणो दिलीपश्च महात्मा चाप्युशीनरः}
{औशीनरिः पुण्डरीकः शर्यातिः शरभः शुचिः}


\twolineshloka
{अङ्गो रिष्टश्च वेनश्च दुष्यन्तः सृञ्जयो जयः}
{भाङ्गासुरिः सुनीथश्च निषधोऽथ वहीनरः}


\twolineshloka
{करन्धमो बाह्लिकश्च सुद्युम्नो बलवान्मधुः}
{ऐलो मरुत्तश्च तथा बलवान्पृथिवीपतिः}


\twolineshloka
{कपोतरोमा तृणकः सहदेवार्जुनौ तथा}
{व्यश्वः साश्वः कृशाश्वश्च शशबिन्दुश्च पार्थिवः}


\twolineshloka
{रामो दाशरथिश्चैव लक्ष्मणोऽथ प्रतर्दनः}
{अलर्कः कक्षसेनश्च गयो गौराओश्व एव च}


\twolineshloka
{जामदग्न्यश्च रामश्च नाभागसगरौ तथा}
{भूरिद्युम्नो महाश्वश्च पृथाशअवो जनकस्तथा}


\twolineshloka
{राजा वैन्यो वारिसेनः पुरिजिज्जनमेजयः}
{ब्रह्मदत्तस्त्रिगर्तिश्च राजोपरिचरस्तथा}


\twolineshloka
{इन्द्रद्युम्नो भीमजानुर्गौरपृष्ठोऽनघो लयः}
{पद्मोऽथ मुचुकुन्दश्च भूरिद्युम्नः प्रसेनजित्}


\twolineshloka
{अरिष्टनेमिः सुद्युम्नः पृथुलाश्वोऽष्टकस्तथा}
{शतं मत्स्या नृपतयः शतं नीपाः शतं हयाः}


\twolineshloka
{धृतराष्ट्राश्चैकशतमशीतिर्जनमेजयाः}
{शतं च ब्रह्मदत्तानां वीरिणामीरिणां शतम्}


\twolineshloka
{भीष्णाणां द्वे शतेऽप्यत्र भीमानां तु तथा शतम्}
{शतं च प्रतिविन्ध्यानां शतं नागाः शतं हयाः}


\twolineshloka
{पलाशानां शतं ज्ञेयं शतं काशकुशादयः}
{शान्तनुश्चैव राजेन्द्र पाण्डुश्चैव पिता तव}


\twolineshloka
{उशङ्गवः शतरथो देवराजो जयद्रथः}
{वृषदर्भश्च राजर्षिर्बुद्धिमान्सहमन्त्रिभिः}


\twolineshloka
{अथापरे सहस्राणि ये गताः शसबिन्दवः}
{इष्ट्वाऽश्वमेधैर्बहुभिर्महद्भिर्भूरिदक्षिणैः}


\twolineshloka
{एते राजर्षयः पुण्याः कीर्तिमन्तो बहुश्रुताः}
{तस्यां सभायां राजेन्द्र वैवस्वतमुपासते}


\twolineshloka
{अगस्त्योऽथ मतङ्गश्च कालो मृत्युस्तथैव च}
{यज्वानश्चैव सिद्धाश्च ये न योगशरीरिणः}


\twolineshloka
{अग्निष्वात्ताश्च पितरः फेनपाश्वोष्मपाश्च ये}
{सुधावन्तो बर्हिषदो मूर्तिमन्तस्तथाऽपरे}


\twolineshloka
{कालचक्रं च साक्षाच्च भगवान्हव्यवाहनः}
{नरा दुष्कृतकर्माणो दक्षिणायनमृत्यवः}


\twolineshloka
{कालस्य नयने युक्ता यमस्य पुरुपाश्च ये}
{तस्यां शिंशुपपालाशास्तथा काशकुशादयः}


\fourlineindentedshloka
{उपासते धर्मराजं मूर्तिमन्तो जनाधिप}
{एते चान्ये च बहवः पितृराजसभासदः}
{न शक्याः परिसङ्ख्यातुं नामभिः कर्मभिस्तथा}
{}


\threelineshloka
{असम्बाधा हि सा पार्थ रम्या कामगमा सभा}
{दीर्घकालं तपस्तप्त्वा निर्मिता विश्वकर्मणा}
{}


\twolineshloka
{ज्वलन्ती भासमाना च तेजसा स्वेन भारत}
{तामुग्रतपसो यान्ति सुव्रताः सत्यवादिनः}


\twolineshloka
{शान्ताः सन्यासिनः शुद्धाः पूताः पुण्येन कर्मणा}
{सर्वे भास्वरदेहाश्च सर्वे च विरजोम्बराः}


\twolineshloka
{चित्राङ्गदाश्चित्रमाल्याः सर्वे ज्वलितकुण्डलाः}
{सुकृतैः कर्मभिः पुण्यैः पारिबर्हैश्च भूषिताः}


\twolineshloka
{गन्धर्वाश्च महात्मानः सङ्घशश्चाप्सरोगणाः}
{वादित्रं नृत्तगीतं च हास्यं लास्यं च सर्वशः}


\twolineshloka
{पुण्याश्च गन्धाः शब्दाश्च तस्यां पार्थ समन्ततः}
{दिव्यानि चैव माल्यानि उपतिष्ठन्ति नित्यशः}


\twolineshloka
{शतं शतसहस्राणि धर्मिणां तं प्रजेश्वरम्}
{उपासते महात्मानं रूपयुक्ता मनस्विनः}


\twolineshloka
{ईदृशी सा सभा राजन्पितृराज्ञो महात्मनः}
{वरुणस्यापि वक्ष्यामि सभां पुष्करमालिनीम्}


\chapter{अध्यायः ९}
\twolineshloka
{युधिष्ठिर सभा दिव्या वरुणस्यामितप्रभा}
{प्रमाणेन यथा याम्या शुभप्राकारतोरणा}


\twolineshloka
{अन्तः सलिलमास्थाय विहिता विश्वकर्मणा}
{दिव्यै रत्नमयैर्वृक्षैः फलपुष्पप्रदैर्युता}


\twolineshloka
{नीलपीतैः सिताः श्यामैः सितैर्लोहितकैरपि}
{अवतानैस्तथा गुल्मैर्मञ्जरीजालधारिभिः}


\twolineshloka
{तथा शकुनयस्तस्यां विचित्रा मधुरस्वराः}
{अनिर्देश्या वपुष्मन्तः शतशोऽथ सहस्रशः}


\twolineshloka
{सा सभा सुखसंस्पर्शा न शीता न च धर्मदा}
{वेश्मासनवती रम्या सिता वरुणपालिता}


\twolineshloka
{यस्यामास्ते स वरुणो वारुण्या च समन्वितः}
{दिव्यरत्नाम्बरधरो दिव्याभरणभूषितः}


\twolineshloka
{`द्वितीयेन तु नाम्ना वै गौरीति भुवि विश्रुता}
{पत्न्या सवरुणो देवः प्रमोदति सुखी सुखम्'}


\twolineshloka
{स्रग्विणो दिव्यगन्धाश्च दिव्यगन्धानुलेपनाः}
{आदित्यास्तत्र वरुणं जलेश्वरमुपासते}


\twolineshloka
{वासुकिस्तक्षकश्चैव नागश्चैरावतस्तथा}
{कृष्णश्च लोहितश्चैव पद्मश्चित्रश्च वीर्यवान्}


\twolineshloka
{कम्बलाश्वतरौ नागौ धृतराष्ट्रबलाहकौ}
{मणिमान्कुण्डधारश्च कर्कोटकधनञ्जयौ}


\twolineshloka
{पाणिमान्कुण्डधारश्च बलवान्पृथिवीपते}
{प्रह्रादो मुषिकादश्च तथैव जनमेजयः}


\twolineshloka
{पताकिनो मण्डलिनः फणवन्तश्च सर्वशः}
{` अर्थो धर्मश्च कामश्च वसुः कपिल एव च}


\twolineshloka
{अनन्तश्च महानागो यं दृष्ट्वा जलजेश्वरः}
{अभ्यर्चयति सत्कारैरासनेन च तं विभुः}


\twolineshloka
{वासुकिप्रमुखाश्चैव सर्वे प्राञ्जलयः स्थिताः}
{अनुज्ञाताश्च शेषेण यथार्हमुपविश्य च}


\threelineshloka
{एते चान्ये च बहवः सर्पास्तस्यां युधिष्ठिर}
{`वैनतेयश्च गरुडो ये चास्य परिचारिणः'}
{उपासते महात्मानं वरुणं विगतक्लमाः}


\twolineshloka
{बलिर्वैरोचनो राजा नरकः पृथिवीञ्जयः}
{संह्रादो विप्रचित्तिश्च कालखञ्जाश्च दानवाः}


\twolineshloka
{सुहनुर्दुर्मुखः शङ्खः सुमनाः सुमतिस्ततः}
{घटोदरो महापार्श्वः क्रथनः पिठरस्तथा}


\twolineshloka
{विश्वरूपः स्वरूपश्च विरूपोऽथ महाशिराः}
{दशग्रीवश्च वाली च मेघवासा दशावरः}


\twolineshloka
{टिट्टिभो विटभूतश्च संह्रादश्चेन्द्रतापनः}
{दैत्यदानवसङ्घाश्च सर्वे रुचिरकुण्डलाः}


\twolineshloka
{स्रग्विणो मौलिनश्चैव तथा दिव्यपरिच्छदाः}
{सर्वे लब्धवराः शूराः सर्वे विगतमृत्यवः}


\twolineshloka
{ते तस्यां वरुणं देवं धर्मपाशधरं सदा}
{उपासते महात्मानं सर्वे सुचरितव्रताः}


\twolineshloka
{तथा समुद्राश्चत्वारो नदी भागीरथी च सा}
{कालिन्दी विदिशा वेणा नर्दमा वेगवाहिनी}


\twolineshloka
{विपाशा च शतद्रुश्च चन्द्रभागा सरस्वती}
{इरावती वितस्ता च सिन्धुर्देवनदी तथा}


\twolineshloka
{गोदावरी कृष्णवेणी कावेरी च सरिद्वरा}
{किम्पुना च विशल्या च तथा वैतरणी नदी}


\twolineshloka
{तृतीया ज्येष्ठिला चैव शोणश्चापि महानदः}
{चर्मण्वती तथा चैव पर्णाशा च महानदी}


\twolineshloka
{सरयूर्वारवत्याऽथ लाङ्गली च सरिद्वरा}
{करतोया तथाऽऽत्रेयी लौहित्यश्च महानदः}


\twolineshloka
{लङ्घती गोमती चैव सन्ध्या त्रिस्रोतसी तथा}
{एताश्चन्याश्च राजेन्द्र सुतीर्था लोकविश्रुताः}


\threelineshloka
{सरितः सर्वतश्चान्यास्तीर्थानि च सरांसि च}
{कूपाश्च सप्रस्रवणा देहवन्तो युधिष्ठिर}
{}


\twolineshloka
{पल्वलानि तटाकानि देहवन्त्यथ भारत}
{दिशस्तथा मही चैव तथा सर्वे महीधराः}


\twolineshloka
{उपासते महात्मानं सर्वे जलचरास्तथा}
{गीतवादित्रवन्तश्च गन्धर्वाप्सरसां गणाः}


\twolineshloka
{स्तुवन्तो वरुणं तस्यां सर्व एव समासते}
{महीधरा रत्नवन्तो रसा ये च प्रतिष्ठिताः}


\twolineshloka
{कथयन्तः सुमधुराः कथास्तत्र समासते}
{वारुणश्च तथा मन्त्री सुनाभः पर्युपासते}


\twolineshloka
{पुत्रपौत्रैः परिवृतो गोनाम्ना पुष्करेण च}
{सर्वे विग्रहवन्तस्ते तमीश्वरमुपासते}


\twolineshloka
{एषा मया सम्पतता वारुणी भरतर्षभ}
{दृष्टपूर्वा सभा रम्या कुबेरस्य सभां शृणु}


\chapter{अध्यायः १०}
\twolineshloka
{सभा वैश्रवणी राजञ्शतयोजनमायता}
{विस्तीर्णा सप्ततिश्चैव योजनानि सितप्रभा}


\twolineshloka
{तपसा निर्जिता राजन्स्वयं वैश्रवणेन सा}
{शशिप्रभा प्रावरणा कैलासशिखरोपमा}


\twolineshloka
{गुह्यकैरुह्यमाना सा खे विषक्तेव शोभते}
{दिव्या हेममयैरुच्चैः प्रासादैरुपशोभिता}


\threelineshloka
{महारत्नवती चित्रा दिव्यगन्धा मनोरमा}
{सिताभ्रशिखराकारा प्लवमानेव दृश्यते}
{}


\twolineshloka
{दिव्या हेममयै रङ्गैर्विद्युद्भिरिव चित्रिता}
{तस्यां वैश्रवणो राजा विचित्राभरणाम्बरः}


\threelineshloka
{स्त्रीसहस्रैर्वृतः श्रीमानास्ते ज्वलितकुण्डलः}
{`सह पत्न्या महाराज ऋद्ध्या सह विराजते}
{}


\threelineshloka
{सर्वाभरणभूषिण्या वपुष्मत्या धनेश्वरः'}
{दिवाकरनिभे पुण्ये दिव्यास्तरणसंवृते}
{दिव्यपादोपधाने च निषण्णः परमासने}


\twolineshloka
{मन्दाराणामुदाराणां वनानि परिलोडयन्}
{सौगन्धीकवनानां च गन्धं गन्धवहो वहन्}


\twolineshloka
{नलिन्याश्चालकाख्याया नन्दनस्य वनस्य च}
{शीतो हृदयसंह्लादी वायुस्तमुपसेवते}


\twolineshloka
{तत्र देवाः सगन्धर्वा गणैरप्सरसां वृताः}
{दिव्यतानैर्महाराज गायन्तिस्म सभागताः}


\twolineshloka
{मिश्रकेशी च रम्भा च चित्रसेना शुचिस्मिता}
{चारुनेत्रा घृताची च मेनका पुञ्जिकस्थला}


% Check verse!
विश्वाची सहजन्या च प्रम्लोचा उर्वशी
\twolineshloka
{वर्गा च सौरभेयी च समीची बुद्बुदा लता}
{एताः सहस्रशश्चान्या नृत्तगीतविशारदाः}


\twolineshloka
{उपतिष्ठन्ति धनदं गन्धर्वाप्सरसां गणाः}
{अनिशं दिव्यवादित्रैर्नृत्तगीतैश्च स सभा}


\twolineshloka
{अशून्या रुचिरा भाति गन्धर्वाप्सरसां गणैः}
{किन्नरा नाम गन्धर्वा नरा नाम तथाऽपरे}


\twolineshloka
{माणिभद्रोऽथ धनदः श्वेतभद्रश्च गुह्यकः}
{कशेरको गण्डकण्डूः प्रद्योतश्च महाबलः}


\twolineshloka
{कुस्तम्बरुः पिशाचश्च गजकर्णो विशालकः}
{वराहकर्णस्ताम्रौष्ठः फलकक्षः फलोदकः}


\twolineshloka
{हंसचूडः शइखावर्तो हेमनेत्रो विभीषणः}
{पुष्पाननः पिङ्गलकः शोणितोदः प्रवालकः}


\twolineshloka
{वृक्षवास्यनिकेतश्च चीरवासाश्च भारत}
{एते चान्ये च बहवो यक्षाः शतसहस्रशः}


\twolineshloka
{सदा भगवती लक्ष्मीस्तत्रैव नलकूबरः}
{अहं च बहुशस्तस्यां भवन्त्यन्ये च मद्विधाः}


\twolineshloka
{ब्रह्मर्षयो भवन्त्यत्र तथा देवर्षयोऽपरे}
{क्रव्यादाश्च तथैवान्ये गन्धर्वाश्च महाबलाः}


\twolineshloka
{उपासते महात्मानं तस्यां धनदमीश्वरम्}
{भगवान्भूतसङ्घैश्च वृतः शतसहस्रशः}


\twolineshloka
{उमापतिः पशुपतिः शूलभृद्भगनेत्रहा}
{त्र्यम्बको राजशार्दूल देवी च विगतक्लमा}


\twolineshloka
{वामनैर्विकटैः कुब्जैः क्षतजाक्षैर्महारवैः}
{मेदोमांसाशनैरुग्रैरुग्रधन्वा महाबलः}


\twolineshloka
{नानाप्रहरणैरुग्रैर्वातैरिव महाजवैः}
{वृतः सखायमन्वास्ते सदैव धनदं नृप}


\twolineshloka
{प्रहृष्टाः शतशश्चान्ये बहुशः सपरिच्छदाः}
{गन्धर्वाणां च पतयो विश्वावसुर्हहा हुहूः}


\twolineshloka
{तुम्बुरुः प्रवतश्चैव शैलूषश्च तथाऽपरः}
{चित्रसेनश्च गीतज्ञस्तथा चित्ररथोपि च}


\twolineshloka
{एते चान्ये च गन्धर्वा धनेश्वरमुपासते}
{विद्याधराधिपश्चैव चक्रधर्मा सहानुजैः}


\twolineshloka
{उपाचरति तत्र स्म धनानामीश्वरं प्रभुम्}
{किन्नराः शतशस्तत्र धनानामीश्वरं प्रभुम्}


\twolineshloka
{आसते चापि राजानो भगदत्तपुरोगमाः}
{द्रुमः किम्पुरुषेशश्च उपास्ते धनदेश्वरम्}


\twolineshloka
{राक्षसाधिपतिश्चैव महेन्द्रो गन्धमादनः}
{सह यक्षैः सगन्धर्वैः सह सर्वैर्निशाचरैः}


\twolineshloka
{विभीषणश्च धर्मिष्ठ उपास्ते भ्रातरं प्रभुम्}
{हिमवान्पारियात्रश्च विन्ध्यकैलासमन्दराः}


\twolineshloka
{मलयो दर्दुरश्चैव महेन्द्रो गन्धमादनः}
{इन्द्रकीलः सुनाभश्च तथा दिव्यौ च पर्वतौ}


\twolineshloka
{एते चान्ये च बहवः सर्वे मेरुपुरोगमाः}
{उपासते महात्मानं धनानामीश्वरं प्रभुम्}


\twolineshloka
{न्दीश्वरश्च भगवान्महाकालस्तथैव च}
{शङ्कुकर्णमुखाः सर्वे दिव्याः पारिषदास्तथा}


\twolineshloka
{काष्ठः कुटी मुखो दन्ती विजयश्च तपोधिकः}
{श्वेतश्च वृषभस्तत्र नर्दन्नास्ते महाबलः}


\twolineshloka
{धनदं राक्षसाश्चान्ये पिशाचाश्च उपासते}
{पारिषदैः परिवृतमुपायान्तं महेश्वरम्}


\twolineshloka
{सदा हि देवदेवेशं शिवं त्रैलोक्यभावनम्}
{प्रणम्य मूर्ध्ना पौलस्त्यो बहुरूपमुमापतिम्}


\twolineshloka
{ततोऽभ्यनुज्ञां सम्प्राप्य महादेवाद्धनेश्वरः}
{आस्ते कदाचिद्भगवान्भवो धनपतेः सखा}


\twolineshloka
{निधिप्रवरमुख्यौ च शङ्खपद्मौ धनेश्वरौ}
{सर्वान्निधीन्प्रगृह्याथ उपास्तां वै धनेश्वरम्}


\twolineshloka
{सा सभा तादृशी रम्या मया दृष्टान्तरिक्षगा}
{पितामहसभां राजन्कीर्तयिष्ये निबोध ताम्}


\chapter{अध्यायः ११}
\twolineshloka
{पितामहसभां तात कथ्यमानां निबोध मे}
{शक्यते या न निर्देष्टुमेवंरूपेति भारत}


\twolineshloka
{पुरा देवयुगे राजन्नादित्यो भगवान्दिवः}
{आगच्छन्मानुषं लोकं दिदृक्षुर्विगतक्लमः}


\twolineshloka
{चरन्मानुषरूपेण सभां दृष्ट्वा स्वयम्भुवः}
{स तामकथयन्मह्यं दृष्ट्वा तत्त्वेन पाण्डव}


\twolineshloka
{अप्रमेयां सभां दिव्यां मानसीं भरतर्षभ}
{अनिर्देश्यां प्रभावेण सर्वभूतमनोरमाम्}


\threelineshloka
{श्रुत्वा गुणानहं तस्याः सभायाः पाण्डवर्षभ}
{दर्शनेप्सुस्तथा राजन्नादित्यमिदमब्रुवम्}
{}


\twolineshloka
{भगवन्द्रष्टुमिच्छामि पितामहसभां शुभाम्}
{येन वा तपसा शक्या कर्मणा वाऽपि गोपते}


\twolineshloka
{औषधैर्या तथा युक्तैरुत्तमा पापनाशिनी}
{तन्ममाचक्ष्व भगवन्पश्येयं तां सभां यथा}


\twolineshloka
{स तन्मम वचः श्रुत्वा सहस्रांशुर्दिवाकरः}
{प्रोवाच भारतश्रेष्ठ व्रतं वर्षसहस्रकम्}


\twolineshloka
{ब्रह्मव्रतमुपास्स्व त्वं प्रयतेनान्तरात्मना}
{ततोऽहं हिमवत्पृष्ठे समारब्दो महाव्रतम्}


\twolineshloka
{ततः स भगवान्सूर्यो मामुपादाय वीर्यवान्}
{आगच्छत्तां सभां ब्राह्मीं विपाप्मा विगतक्लमः ?}


\twolineshloka
{एवंरूपेति सा शक्या न निर्देष्टुं नराधिप}
{क्षणेन हि बिभर्त्यन्यदनिर्देश्यं वपुस्तथा}


\twolineshloka
{न वेद परिमाणं वा संस्थानं चापि भारत}
{न च रूपं मया तादृक् दृष्टपूर्वं कदाचन}


\twolineshloka
{सुसुखा सा सदा राजन्न शीता न च घर्मदा}
{न क्षुत्पिपासे न ग्लानिं प्राप्यतां प्राप्तुवन्त्युत}


\twolineshloka
{नानारूपैरिव कृता मणिभिः सा सुभास्वरैः}
{स्तम्भैर्न च धृता सा तु शाश्वती न च सा क्षरा}


% Check verse!
दिव्यैर्नानाविधैर्भावैर्भासद्भिरमितप्रभैः
\twolineshloka
{अति चन्द्रं च सूर्यं च शिखिनं च स्वयम्प्रभा}
{दीप्यते नाकपृष्ठस्था भर्त्सयन्वीव भास्करम्}


\twolineshloka
{तस्यां स भगवानास्ते विदधद्देवमायया}
{स्वयमेकोऽनिशं राजन्सर्वलोकपितामहः}


\twolineshloka
{उपतिष्ठन्ति चाप्येनं प्रजानां पतयः प्रभुम्}
{दक्षः प्रचेताः पुलहो मरीचिः कश्यपः प्रभुः}


\threelineshloka
{भृगुरत्रिर्वसिष्ठश्च गौतमोऽथ तथाङ्गिराः}
{पुलस्त्यश्च कतुश्चैव प्रह्लादः कर्दमस्तथा}
{}


\twolineshloka
{अथर्वाङ्गिरसश्चैव वालखिल्या सरीचिपाः}
{मनोऽन्तरिक्षं विद्याश्च वायुस्तेजो जलं मही}


\twolineshloka
{शब्दस्पर्शौ तथा रूपं रसो गन्धश्च भारत}
{प्रकृतिश्च विकारश्च यच्चान्यत्कारणं भुवः}


\twolineshloka
{अगस्त्यश्च महातेजा मार्कण्डेयश्च वीर्यवान्}
{जमदग्निर्भरद्वाजः संवर्तश्च्यवनस्तथा}


\twolineshloka
{दुर्वासाश्च महाभाग ऋष्णशृङ्गश्च धार्मिकः}
{सनत्कुमारो भगवान्योगाचार्यो महातपाः}


\twolineshloka
{असितो देवलश्चैव जैगीषव्यश्च तत्त्ववित्}
{ऋषभो जितशत्रुश्च महावीर्यस्तथा मणिः}


\twolineshloka
{आयुर्वेदस्तथाऽष्टाङ्गो देहवांस्तत्र भारत}
{चन्द्रमाः सह नक्षत्रैरादित्यश्च गभस्तिमान्}


\twolineshloka
{वायवः क्रतवश्चैव सङ्कल्पः प्राण एव च}
{मूर्तिमन्तो महात्मानो महाव्रतपरायणाः}


\twolineshloka
{एते चान्ये च बहवो ब्रह्माणं समुपस्थिताः}
{अर्थो धर्मश्च कामश्च हर्षो द्वेषस्तपो दमः}


\twolineshloka
{आयान्ति तस्यां सहिता गन्धर्वाप्सरसां गणाः}
{विंशतिः सप्त चैवान्ये लोकपालाश्च सर्वशः}


\twolineshloka
{शुक्रो बृहस्पतिश्चैव बुधोऽङ्कारक एव च}
{शनैश्चरश्च राहुश्च ग्रहाः सर्वे तथैव च}


\twolineshloka
{मन्त्रो रथन्तरं चैव हरिमान्वसुमानपि}
{आदित्याः साधिराजानो नामद्वन्द्वैरुदाहृताः}


\twolineshloka
{मरुतो विश्वकर्मा च वसवश्चैव भारत}
{तथा पितृगणाः सर्वे सर्वाणि च हवींष्यथ}


\twolineshloka
{ऋग्वेदः सामवेदश्च यजुर्वेदश्च पाण्डव}
{अथर्ववेदश्च तथा सर्वशास्त्राणि चैव ह}


\twolineshloka
{इतिहासोपवेदाश्च वेदाङ्गानि च सर्वशः}
{ग्रहा यज्ञाश्च सोमश्च देवताश्चापि सर्वशः}


\twolineshloka
{सावित्री दुर्गतरणी वाणी सप्तविधा तथा}
{मेधा धृतिः श्रुतिश्चैव प्रज्ञा बुद्धिर्यशः क्षमा}


\twolineshloka
{सामानि स्तुतिशस्त्राणि गाथाश्च विविधास्तथा}
{भाष्याणि तर्कयुक्तानि देहवन्ति विशाम्पते}


\twolineshloka
{नाटका विविधाः काव्याः कथाख्यायिकारिकाः}
{तत्रतिष्ठन्ति ते पुण्या ये चान्ये गुरुपूजकाः}


\twolineshloka
{क्षणा लवा मुहूर्ताश्च दिवा रात्रिस्तथैव च}
{अर्धमासाश्च मासाश्च ऋतवः षट् च भारत}


\twolineshloka
{संवत्सराः पञ्ययुगमहोरात्रश्चतुर्विधः}
{कालचक्रं च तद्दिव्यं नित्यमक्षयमव्ययम्}


\twolineshloka
{धर्मचक्रं तथा चापि नित्यमास्ते युधिष्ठिर}
{अदितिर्दितिर्दनुश्चैव सुरसा विनता इरा}


% Check verse!
कालिका सुरभी देवी सरमा चाथ गौतमी
\twolineshloka
{प्रभा कद्रूश्च वै देव्यौ देवतानां च मातरः}
{रुद्राणी श्रीश्च लक्ष्मीश्च भद्रा षष्ठी तथाऽपरा}


\twolineshloka
{पृथिवी गां गता देवी ह्रीः स्वाहा कीर्तिरेव च}
{सुरा देवी शची चैव तथा पुष्टिररुन्धती}


\twolineshloka
{संवृत्तिराशा नियतिः सृष्टिर्देवी रतिस्तथा}
{एताश्चान्याश्चवै देव्य उपतस्थुः प्रजापतिम्}


\twolineshloka
{आदित्या वसवो रुद्रा मरुतश्चास्विनावपि}
{विश्वेदेवाश्च साध्याश्च पितरश्च मनोजवाः}


\twolineshloka
{पितृणां च गणान्विद्धि सप्तैव पुरुषर्षभ}
{मूर्तिमन्तो वै चत्वारस्त्रयश्चापि शरीरिणः}


\twolineshloka
{वैराजश्च महाभागा अग्निष्वात्ताश्च भारत}
{गार्हपत्या नाकचराः पितरो लोकविश्रुताः}


\twolineshloka
{सोमपा एकशृङ्गाश्च चतुर्वेदाः कलास्तथा}
{एते चतुर्षु वर्णेषु पूज्यन्ते पितरो नृप}


\twolineshloka
{एतैराप्यायितैः पुर्वं सोमश्चाप्याय्यते पुनः}
{त एते पितरः सर्वे प्रजापतिमुपस्थिताः}


\twolineshloka
{उपासते च संहृष्टा ब्रह्माणममितौजसम्}
{राक्षसाश्च पिशाचाश्च दानवा गुह्यकास्तथा}


\twolineshloka
{नागाः सुपर्णाः पशवः पितामहमुपासते}
{स्थावरा जङ्गमाश्चैव महाभूतास्तथाऽपरे}


\twolineshloka
{पुरन्दरश्च देवेन्द्रो वरुणो धनदो यमः}
{महादेवः सहोमोऽत्र सदा गच्छति सर्वशः}


\twolineshloka
{महासेनश्च राजेन्द्र सदोपास्ते पितामहम्}
{देवो नारायणस्तस्यां तथा देवर्षयश्च ये}


\threelineshloka
{ऋषयो वालखिल्याश्च योनिजायोनिजास्तथा}
{यच्च किञ्चित्रिलोकेऽस्मिन्दृश्यते स्थाणु जङ्गमम्}
{सर्वं तस्यां मया दृष्टमिति विद्धि नराधिप}


\twolineshloka
{अष्टाशीतिसहस्राणि ऋषीणामूर्ध्वरेतसाम्}
{प्रजावतां च पञ्चाशदृषीणामपि पाण्डव}


\twolineshloka
{ते स्म तत्र यथाकामं दृष्ट्वा सर्वे दिवौकसः}
{प्रणम्य शिरसा तस्मै सर्वे यान्ति यथागमम्}


\twolineshloka
{अतिथीनागतान्देवान्दैत्यान्नागांस्तथा द्विजान्}
{यक्षान्मुपर्णान्कालेयान्गन्धर्वाप्सरसस्तथा}


\twolineshloka
{महाभागानमितधीर्ब्रह्मा लोकपितामहः}
{दयावान्सर्वभूतेषु यथार्हं प्रतिपद्यते}


\twolineshloka
{प्रतिगृह्य तु विश्वात्मा स्वयं स्वयम्भूरमितद्युतिः}
{सान्त्वमानार्थसम्भोगैर्युनक्ति मनुजाधिप}


\twolineshloka
{तथा तैरुपयातैश्च प्रतियद्भिश्च भारत}
{आकुला सा सभातात भवति स्म सुखप्रदा}


\twolineshloka
{सर्वतेजोमयी दिव्या ब्रह्मर्षिगणसेविता}
{ब्राहया श्रिया दीप्यमाना शुशुभे विगतक्लमा}


\twolineshloka
{सा सभा तादृशी दृष्टा मया लोकेषु दुर्लभा}
{सभेयं राजशार्दूल मनुष्येषु यथा तव}


\twolineshloka
{एता मया दृष्टपूर्वाः सभा देवेषु भारत}
{सभेयं मानुषे लोके सर्वश्रेष्ठतमा तव}


\chapter{अध्यायः १२}
\twolineshloka
{प्रायशो राजलोकस्ते कथितो वदतां वर}
{वैवस्वतसभायां तु यथा वदसि मे प्रभो}


\twolineshloka
{वरुणस्य सभायां तु नागास्ते कथिता विभो}
{दैत्येन्द्राश्चापि भूयिष्ठाः सरितः सागरास्तथा}


\twolineshloka
{तथा धनपतेर्यक्षा गुह्यका राक्षसास्तथा}
{गन्धर्वाप्सरसश्चैव भगवांश्च वृषध्वजः}


\twolineshloka
{पितामहसभायां तु कथितास्ते महर्षभः}
{सर्वे देवनिकायाश्च सर्वशास्त्राणि वैव ह}


\twolineshloka
{शक्रस्य तु सभायां तु देवाः सङ्कीर्तिता मुने}
{उद्देशतश्च गन्धर्वा विविधाश्च महर्षयः}


\twolineshloka
{एक एव तु राजर्षिर्हरिश्चन्द्रो महामुने}
{कथितस्ते सभायां वै देवेन्द्रस्य महात्मनः}


\twolineshloka
{किं कर्म तेनाचरितं तपो वा नियतव्रत}
{येनासौ सह शक्रेण स्पर्धदे सुमहायशाः}


\twolineshloka
{पितृलोकगतश्चैव त्वया विप्र पिता मम}
{दृष्टः पाण्डुर्महाभागः कथं वाऽपि समागतः}


\threelineshloka
{किमुक्तवांश्च भगवंस्तन्ममाचक्ष्व सुव्रत}
{त्वत्तः श्रोतुं सर्वमिदं परं कौतूहलं हि मे ॥नारद उवाच}
{}


\twolineshloka
{यन्मां पृच्छसि राजेन्द्र हरिश्चन्द्रं प्रति प्रभो}
{तत्तेऽहं सम्प्रवक्ष्यामि माहात्म्यं तस्य धीमतः}


\twolineshloka
{` इक्ष्वाकूणां कुले जातस्त्रिशङ्कुर्नाम पार्थिवः}
{अयोध्याधिपतिर्वीरो विश्वामित्रेण संस्थितः}


\twolineshloka
{तस्य सत्यवती नाम पत्नी केकयवंशजा}
{तस्यां गर्भः समभवद्धर्मेण कुरुनन्दन}


\threelineshloka
{सा च काले महाभागा राजन्मासं प्रविश्य च}
{कुमारं जनयामास हरिश्चन्द्रमकल्मषम्}
{स वै राजा हरिश्चन्द्रस्त्रैशङ्कव इति स्मृतः'}


\twolineshloka
{स राजा बलवानासीत्सम्राट् सर्वमहीक्षिताम्}
{तस्य सर्वे महीपालाः शासनावनताः स्थिताः}


\twolineshloka
{तेनैकं रथमास्थाय जैत्रं हेमविभूषितम्}
{शस्त्रप्रतापेन जिता द्वीपाः सप्त जनेश्वर}


\twolineshloka
{स निर्जित्य महीं कृत्स्नां सशैलवनकाननाम्}
{आजहार महाराज राजसूयं महाक्रतुम्}


\twolineshloka
{तस्य सर्वे महीपाला धनान्याजह्रुराज्ञया}
{द्विजानां परिवेष्टारस्तस्मिन्यज्ञे च तेऽभवन्}


\twolineshloka
{`समाप्तयज्ञो विधिवद्धरिश्चन्द्रः प्रतापवान्}
{अभिषिक्तश्च शुशुभे साम्राज्येन नराधिपः}


% Check verse!
राजसूयेऽभिषिक्तश्च समाप्तवरदक्षिणे'
\twolineshloka
{प्रादाच्च द्रविणं प्रीत्या याचकानां नरेश्वरः}
{यथोक्तवन्तस्ते तस्मिंस्ततः पञ्चगुणाधिकम्}


\twolineshloka
{अतर्पयच्च विविधैर्वसुभिर्ब्राह्मणांस्तदा}
{प्रसर्पकाले सम्प्राप्ते नानादिग्भ्यः समागतान्}


\fourlineindentedshloka
{भक्ष्यभोज्यैश्च विविधैर्यथाकामपुरस्कृतैः}
{रत्नौघतर्पितैस्तुष्टैर्द्विजैश्च समुदाहृतम्}
{तेजस्वी च यशस्वी च नृपेभ्योऽभ्यधिकोऽभवत्}
{}


\twolineshloka
{एतस्मात्कारणाद्राजन्हरिश्चन्द्रो विराजते}
{तेभ्यो राजसहस्रेभ्यस्तद्विद्वि भरतर्षभ}


\twolineshloka
{समाप्य च हरिश्चन्द्रो महायज्ञं प्रतापवान्}
{अभिषिक्तश्च शुशुभे साम्राज्येन नराधिप}


\threelineshloka
{ये चान्ये च महीपाला राजसूयं महाक्रतुम्}
{यजन्ते ते सहेन्द्रेण मोदन्ते भरतर्षभ}
{}


\twolineshloka
{ये चापि नि नं प्राप्ताः सङ्ग्रामेष्वपलायिनः}
{ते तत्सदनम् ताद्य मोन्दते भरतर्षभ}


\twolineshloka
{तपसा ये च तीव्रेण त्यजन्तीह कलेवरम्}
{ते तत्स्थानं समासाद्य श्रीमन्तो भान्ति नित्यशः}


\twolineshloka
{पिता च त्वाऽऽह कौन्तेय पाण्डुः कौरवनन्दन}
{हरिश्चन्द्रे श्रियं दृष्ट्वा नृपतौ जातविस्मयः}


\twolineshloka
{विज्ञाय मानुषं लोकमायान्तं मां नराधिप}
{प्रोवाच प्रणतो भूत्वा वदेथास्त्वं युधिष्ठिरम्}


\twolineshloka
{समर्थोऽसि महीं जेतुं भ्रातरस्ते स्थिता वशे}
{राजसूयं क्रतुश्रेष्ठमहारस्वेति भारत}


\twolineshloka
{त्वयीष्टवति पुत्रेऽहं हरिश्चन्द्रवदाशु वै}
{मोदिष्ये बहुलाः शश्वत्समाः शक्रस्य संसदि}


\twolineshloka
{एवं भवतु वक्ष्येऽहं तव पुत्रं नराधिपम्}
{भूलोकं यदि गच्छेयमिति पाण्डुमथाब्रुवम्}


\twolineshloka
{तस्य त्वं पुरुषव्याघ्र सङ्कल्पं कुरु पाण्डव}
{गन्तासित्वं महेन्द्रस्य पूर्वैः सह सलोकताम्}


\twolineshloka
{बहुविघ्नश्च नृपते क्रतुरेष स्मृतो महान्}
{छिद्राण्यस्य वाञ्छन्ति यज्ञघ्ना ब्रह्मराक्षसाः}


\twolineshloka
{युद्धं च क्षत्रशमनं पृथिवीक्षयकारणम्}
{किञ्चिदेव निमित्तं च भवत्यत्र क्षयावहम्}


\twolineshloka
{एतत्सञ्चिन्त्य राजेन्द्र यत्क्षमं तत्समाचर}
{अप्रमत्तोत्थितो नित्यं चातुर्वर्ण्यस्य रक्षणे}


\fourlineindentedshloka
{भव एधस्व मोदस्व धनैस्तर्पय च द्विजान्}
{एतत्ते विस्तरेणोक्तं यन्मां त्वं परिपृच्छसि}
{आपृच्छे त्वां गमिष्यामि दाशार्हनगरीं प्रति ॥वैशम्पायन उवाच}
{}


\twolineshloka
{एवमाख्याय पार्थेभ्यो नारदो जनमेजय}
{जगाम तैर्वृतो राजन्नृषिभिर्यैः समागतः}


\twolineshloka
{गते तु नारदे पार्थो भ्रातृभिः सह कौरवः}
{राजसूयं क्रतुश्रेष्ठं चिन्तयामास पार्थिवः}


\chapter{अध्यायः १३}
\twolineshloka
{ऋषेस्तद्वचनं श्रुत्वा निशश्वास युधिष्ठिरः}
{चिन्तयन्राजसूयेष्टिं न लेभे शर्म भारत}


\twolineshloka
{राजर्षीणां च तं श्रुत्वा महिमानं महात्मनाम्}
{यज्वनां कर्मभिः पुण्यैर्लोकप्राप्तिं समीक्ष्य च}


\twolineshloka
{हरिश्चन्द्रं च राजर्षि रोजमानं विशेषतः}
{यज्वानं यज्ञमाहर्तुं राजसूयमियेष सः}


\twolineshloka
{युधिष्ठिरस्ततः सर्वानर्चयित्वा सभासदः}
{प्रत्यर्चितश्च तैः सर्वैर्यज्ञायैव मनो दधे}


\twolineshloka
{स राजसूयं राजेन्द्र कुरूणामृषभस्तदा}
{आहर्तुं प्रवणं चक्रे मनः सञ्चिन्त्य चासकृत्}


\twolineshloka
{भूयश्चाद्भुतवीर्यौजा धर्ममेवानुचिन्तयन्}
{किं हितं सर्वलोकानां भवेदिति मनो दधे}


\twolineshloka
{अनुगृह्णन्प्रजाः सर्वाः सर्वधर्मभृतां वरः}
{अविशेषेण सर्वेषां हितं चक्रे युधिष्ठिरः}


\twolineshloka
{सर्वेषां दीयतां देयं मुष्णन्कोपमदावुभौ}
{साधु धर्मेति धर्मेति नान्यच्छ्रूयेत भाषितम्}


\twolineshloka
{एवं गते ततस्तस्मिन्पितरीवाश्वसञ्जनाः}
{न तस्य विद्यते द्रेष्टा ततोऽस्याजातशत्रुता}


\twolineshloka
{परिग्रहान्नरेन्द्रस्य भीमस्य परिपालनात्}
{शत्रूणां क्षपणाच्चैव बीभत्सोः सव्यसाचिनः}


\twolineshloka
{`बलीनां सम्यगुत्थानान्नकुलस्य यशस्विनः'}
{धीमतः सहदेवस्य धर्माणामनुशासनात्}


\twolineshloka
{वैनत्यात्सर्वतश्चैव नकुलस्य स्वभावतः}
{अविग्रहा वीतभयाः स्वकर्मनिरताः सदा}


\twolineshloka
{निकामवर्षाः स्फीताश्च आसञ्जनपदास्तथा}
{वार्धुषी यज्ञसत्वानि गोरक्षं कर्षणं वणिक्}


\twolineshloka
{विशेषात्सर्वमेवैतत्सञ्जज्ञे राजकर्मणा}
{अनुकर्षं च निष्कर्षं व्याधिपावकमूर्छनम्}


\twolineshloka
{सर्वमेव न तत्रासीद्धर्मनित्ये युधिष्ठिरे}
{दस्युभ्यो वञ्चकेभ्यश्च राज्ञः प्रति परस्परम्}


\twolineshloka
{राजवल्लभतश्चैव नाश्रूयत मृषाकृतम्}
{प्रियं कर्तुमुपस्थातुं बलिकर्म स्वकर्मजम्}


\twolineshloka
{अभिहर्तुं नृपाः षट््सु पृथक््जात्यैश्च नैगमैः}
{ववृधे विषयस्तत्र धर्मनित्ये युधिष्ठिरे}


\twolineshloka
{कामतोऽप्युपयुञ्जानै राजसैर्लोभजैर्जनैः}
{सर्वव्यापी सर्वगुणी सर्वसाहः स सर्वराट्}


\threelineshloka
{यस्मिन्नधिकृतः सम्राड् भ्राजमानो महायशाः}
{यत्र राजन्दश दिशः पितृतो मातृतस्तथा}
{अनुरक्ताः प्रजा आसन्नागोपाला द्विजातयः}


\twolineshloka
{स मन्त्रिणः समानाय्य भ्रातृंश्च वदतां वरः}
{राजसूयं प्रति तदा पुनः पुनरपृच्छत}


\twolineshloka
{ते पृच्छ्यमानाः सहिता वचोऽर्थ्यं मन्त्रिणस्तदा}
{युधिष्ठिरं महाप्राज्ञं यियक्षुमिदमब्रुवन्}


\twolineshloka
{येनाभिषिक्तो नृपतिर्वारुणं गुणमृच्छति}
{तेन राजाऽपि तं कुत्स्नं सम्राड्गुणमभीप्सृति}


\twolineshloka
{तस्य सम्राड्गुणार्हस्य भवतः कुरुनन्दन}
{राजसूयस्य समयं मन्यन्ते सुहृदस्तव}


\twolineshloka
{तस्य यज्ञस्य समयः स्वाधीनः क्षत्रसम्पदा}
{साम्ना षडग्नयो यस्मिंश्चीयन्ते शंसितव्रतैः}


\twolineshloka
{दर्वीहोमानुपादाय सर्वान्यः प्राप्नुते क्रतून्}
{अभिषेकं च यज्ञान्ते सर्वजित्तेन चोच्यते}


\twolineshloka
{समर्थोऽसि महाबाहो सर्वे ते वशगा वयम्}
{अचिरात्त्वं महाराज राजसूयमवाप्स्यसि}


\twolineshloka
{अविचार्य महाराज राजसूये मनः कुरु}
{इत्येवं सुहृदः सर्वे पृथक्च सह चाब्रुवन्}


\twolineshloka
{स धर्म्यं पाण्डवस्तेषां वचः श्रुत्वा विशाम्पते}
{धृष्टमिष्टं वरिष्टं च जग्राह मनसाऽरिहा}


\twolineshloka
{श्रुत्वा सुहृद्वचस्तच्च जानंश्चाप्यात्मनः क्षमम्}
{`स प्रशस्तक्रियारम्भः परीक्षामुपचक्रमे ॥वैशम्पायन उवाच}


\twolineshloka
{चतुर्भिर्भीमसेनाद्यैर्भ्रातृभिः सहितो हितम्}
{एवमुक्तस्तथा पार्थो धर्म एव मनो दधे}


\twolineshloka
{स राजसूयं राजेन्द्रः कुरूणामृषभः क्रतुम्}
{जगाम मनसा सद्य आहरिष्यन्युधिष्ठिरः}


\twolineshloka
{भूयस्त्वद्भुतवीर्योपि धर्ममेवानुपालयन् '}
{पुनः पुनर्मनो दध्रे राजसूयाय भारत}


\twolineshloka
{स भ्रातृभिः पुनर्धीमानृत्विग्निश्च महात्मभिः}
{मन्त्रिभिश्चापि सहितो धर्मराजो युधिष्ठिरः}


\twolineshloka
{धौम्यद्वैपायनाद्यैश्च मन्त्रयामास मन्त्रवित्}
{`विराटद्रुपदाभ्यां च सात्यकेन च धीमता}


\twolineshloka
{युधामन्यूत्तमौजोभ्यां सौभद्रेण च धीमता}
{द्रौपदेयैः परं शूरैर्मन्त्रयामास संवृतः ॥युधिष्ठिर उवाच'}


\threelineshloka
{भवन्तो राजसूयस्य सम्राडर्हस्य सुक्रतोः}
{श्रद्दधानस्य वदत ममावाप्तिः कथं भवेत् ॥वैशम्पायन उवाच}
{}


\twolineshloka
{एवमुक्तास्तु ते तेन राज्ञा राजीवलोचन}
{इदमूचुर्वचः काले धर्मराजं युधिष्ठिरम्}


\twolineshloka
{अर्हस्त्वमसि धर्मज्ञ राजसूयं महाक्रतुम्}
{अथैवमुक्ते नृपतावृत्विग्भिर्ऋषिभिस्तथा}


\twolineshloka
{मन्त्रिणो भ्रातरश्चास्य तद्वचः प्रत्यपूजयन्}
{स तु राजा महाप्राज्ञः पुनरेवात्मनाऽऽत्मवान्}


\twolineshloka
{भूयो विममृशे पार्थो लोकानां हितकाम्यया}
{सामर्थ्ययोगं सम्प्रेक्ष्य देशकालौ व्ययागमौ}


\twolineshloka
{विमृश्य सम्यक् च धिया कुर्वन्प्राज्ञो न सीदति}
{नहि यज्ञसमारम्भः केवलात्मविनिश्चयात्}


\twolineshloka
{भवतीति समाज्ञाय यत्नतः कार्यमुद्वहन्}
{स निश्चयार्थं कार्यस्य कृष्णमेव जनार्दनम्}


\twolineshloka
{सर्वलोकात्परं मत्वा जगाम मनसा हरिम्}
{अप्रमेयं महाबाहुं कामाञ्जातमजं नृषु}


\twolineshloka
{पाण्डवस्तर्कयामास कर्मभिर्देवसंमतैः}
{नास्य किञ्चिदविज्ञातं नास्य किञ्चिदकर्मजम्}


\twolineshloka
{न स किञ्चिन्न विषहेदिति कृष्णममन्यत}
{स तु तां नैष्ठिकीं बुद्धिं कृत्वा पार्थो युधिष्ठिरः}


\twolineshloka
{गुरुवद्भूतगुरवे प्राहिणोद्दूतमञ्जसा}
{शीघ्रगेन रथेनाशु स दूतः प्राप्य यादवान्}


\twolineshloka
{द्वारकावासिनं कृष्णं द्वारवत्यां समासदत्}
{` स प्रभुं प्राञ्जलिर्भूत्वा व्यज्ञापयत माधवम् ॥दूत उवाच}


\twolineshloka
{धर्मराजो हृषीकेश धौम्यव्यासादिभिः सह}
{पाञ्चालमात्स्यसहितैर्भ्रातृभिश्चैव सर्वशः}


% Check verse!
त्वद्दर्शनं महाबाहो काङ्क्षते स युधिष्ठिरः ॥वैशम्पायन उवाच
\twolineshloka
{इन्द्रसेनवचः श्रुत्वा यादवप्रवरो बली'}
{दर्शनाकाङ्क्षिणं पार्थं दर्शनाकाङ्क्षयाच्युतः}


\threelineshloka
{`आमन्त्र्य राजन्सुहृदो वसुदेवं च माधवः'}
{इन्द्रसेनेन सहित इन्द्रप्रस्थमगात्तदा}
{व्यतीत्य विविधान्देशांस्त्वरावान्क्षिप्रवाहनवः}


\twolineshloka
{इन्द्रप्रस्थगतं पार्थमभ्यगच्छज्जनार्दनः}
{स गृहे पितृवद्धात्रा धर्मराजेन पूजितः}


\twolineshloka
{भीमेन च ततोऽपश्यत्स्वसारं प्रीतिमान्पितुः}
{प्रीतः प्रीतेन सुहृदा रेमे स सहितस्तदा}


\threelineshloka
{अर्जुनेन यमाभ्यां च गुरुवत्पर्युपासितः}
{तं विश्रान्तं शुभे देशे क्षणिनं कल्पमच्युतम्}
{धर्मराजः समागम्य ज्ञापयत्स्वप्रयोजनम् ॥युधिष्ठिर उवाच}


\twolineshloka
{प्रार्थितो राजसूयो मे न चासौ केवलेप्सया}
{प्राप्यते येन तत्ते हि विदितं कृष्ण सर्वशः}


\twolineshloka
{यस्मिन्सर्वं सम्भवति यश्च सर्वत्र पूज्यते}
{यश्च सर्वेश्वरो राजा राजसूयं स विन्दति}


\twolineshloka
{तं राजसूयं सुहृदः कार्यमाहुः समेत्य मे}
{तत्र मे निश्चिततमं तव कृष्ण गिरा भवेत्}


\twolineshloka
{केचिद्धि सौहृदा देवे न दोषं परिचक्षते}
{स्वार्थहेतोस्तथैवान्ये प्रियमेव वदन्त्युत}


\twolineshloka
{प्रियमेव परीप्सन्ते केचिदात्मनि यद्धितम्}
{एवम्प्रायाश्च दृश्यन्ते जनवादाः प्रयोजने}


\twolineshloka
{त्वं तु हेतूनतीत्यैतान्कामक्रोधौ व्युदस्य च}
{परमं यत्क्षमं लोके यथावद्वक्तुमर्हसि}


\chapter{अध्यायः १४}
\twolineshloka
{सर्वैर्गुणैर्महाराज राजसूयं त्वमर्हसि}
{जानतस्त्वेव ते सर्वं किञ्चिद्वक्ष्यामि भारत}


\twolineshloka
{जामदग्न्येन रामेण क्षत्रं यदवशेषितम्}
{तस्मादवरजं लोके यदिदं क्षत्रसंज्ञितम्}


\twolineshloka
{कृतोऽयं कलुसङ्कल्पः क्षत्रियैर्वसुधाधिप}
{निदेशवाग्भिस्तत्तेह विदितं भरतर्षभ}


\twolineshloka
{ऐलस्येक्ष्वाकुवंशस्य प्रकृतिं परिचक्षते}
{राजानः श्रेणिबद्धाश्च तथाऽन्ये क्षत्रिया भुवि}


\twolineshloka
{ऐलवंश्याश्च ये राजंस्तथैवेक्ष्वाकवो नृपाः}
{तानि चैकशतं विद्धि कुलानि भरतर्षभ}


\twolineshloka
{ययातेस्त्वेव भोजानां विस्तरो गुणतो महान्}
{भजतेऽद्य महाराज विस्तरं सचतुर्दिशम्}


\twolineshloka
{तेषां तथैव तां लक्ष्मीं सर्वक्षत्रमुपासते}
{इदानीमेव वै राजञ्जरासन्धो महीपतिः}


\twolineshloka
{अभिभूय श्रियं तेषां कुलानामभिषेचितः}
{स्थितो मूर्ध्नि नरेन्द्राणामोजसाऽऽक्रम्य सर्वशः}


\twolineshloka
{सोऽवनीं मध्यमां भुक्त्वा मिथो भेदममन्यत}
{प्रभुर्यस्तु परो राजा यस्मिन्नेकवशे जगत्}


\twolineshloka
{स साम्राज्यं महाराज प्राप्तो भवति योगतः}
{तं स राजा जरासन्धं संश्रित्य किल सर्वशः}


\twolineshloka
{राजन्सेनापतिर्जातः शिशुपालः प्रतापवान्}
{तमेव च महाराज शिष्यवत्समुपस्थितः}


% Check verse!
वक्रः करूषाधिपतिर्मायायोधी महाबलः ॥अपरौ च महावीर्यौ महात्मानौ समाश्रितौ
\threelineshloka
{जरासन्धं महावीर्यं तौ हिंसडिम्बिकावुभौ}
{वक्रदन्तः करूषस्य करभो मेघवाहनः}
{मूर्ध्ना दिव्यं मणिं बिभ्रद्यमद्भुतमणिं विदुः}


\twolineshloka
{मरुं च नरकं चैव शास्ति यो यवनाधिपः}
{अपर्यन्तबलो राजा प्रतीच्यां वरुणो यथा}


\twolineshloka
{भगदत्तो महाराज वृद्धस्तव पितुः सखा}
{स वाचा प्रणतस्तस्य कर्मणा च विशेषतः}


\twolineshloka
{स्नेहबद्धश्च मनसा पितृवद्भक्तिमांस्त्वयि}
{प्रतीच्यां दक्षिणं चान्तं पृथिव्याः प्रति यो नृपः}


\twolineshloka
{मातुलो भवतः शूरः पुरुजित्कुन्तिवर्धनः}
{स ते सन्नितिमानेकः स्नेहतः शत्रुसूदनः}


\twolineshloka
{जरासन्धं गतस्त्वेव पुरा यो न मया हतः}
{पुरुषोत्तमविज्ञातो योसौ चेदिषु दुर्मतिः}


\twolineshloka
{आत्मानं प्रतिजानाति लोकेऽस्मिन्पुरुषोत्तमम्}
{आदत्ते सततं मोहाद्यः स चिह्नं च मामकम्}


\twolineshloka
{वङ्गपुण्ड्रकिरातेषु राजा बलसमन्वितः}
{पौण्ड्रको वासुदेवेति योसौ लोकेऽभिविश्रुतः}


\twolineshloka
{चतुर्थभाङ्महाराज भोज इन्द्रसखो बली}
{विद्याबलाद्यो व्यजयत्स पाण्ड्यक्रथकैशिकान्}


\twolineshloka
{भ्राता तस्याकृतिः शूरो जामदग्न्यसमोऽभवत्}
{स भक्तो मागधं राजा भीष्मकः परवीरहा}


\twolineshloka
{प्रियाण्याचरतः प्रह्वान्सदा सम्बन्धिनस्ततः}
{भजतो न भजत्यस्मानप्रियेषु व्यवस्थितः}


\twolineshloka
{न कुलं सबलं राजन्नभ्यजानात्तथाऽऽत्मनः}
{पश्यमानो यशो दीप्तं जरासन्धमुपस्थितः}


\twolineshloka
{उदीच्याश्च तथा भोजाः कुलान्यष्टादश प्रभो}
{जरासन्धभयादेव प्रतीचीं दिशमास्थिताः}


\twolineshloka
{शूरसेना भद्रकारा बोधाः शाल्वाः पटच्चराः}
{सुस्थलाश्च सुकुट्टाश्च कुलिन्दाः कुन्तिभिः सह}


\twolineshloka
{शाल्वायनाश्च राजानः सोदर्यानुचरैः सह}
{दक्षिणा ये च पञ्चालाः पूर्वाः कुन्तिषु कोसलाः}


\twolineshloka
{तथोत्तरां दिशं चापि परित्यज्य भयार्दिताः}
{मत्स्याः सन्न्यस्तपादाश्च दक्षिणां दिशमाश्रिताः}


\twolineshloka
{तथैव सर्वपञ्चाला जरासन्धभायर्दिताः}
{स्वराज्यं सम्परित्यज्य विद्रुताः सर्वतोदिशम्}


\twolineshloka
{` अग्रतो ह्यस्य पाञ्चालास्तत्रानीके महात्मनः}
{अनिर्गते सारबले मागधेभ्यो गिरिव्रजात्}


\twolineshloka
{उग्रसेनसुतः कंसः पुरा निर्जित्य बान्धवान्'}
{बार्हद्रथसुते देव्यावुपागच्छद्वृथामतिः}


\twolineshloka
{अस्तिः प्रास्तिश्च नाम्ना ते सहदेवानुजेऽबले}
{बलेन तेन स्वज्ञातीनभिभूय वृथामतिः}


\twolineshloka
{श्रैष्ठ्यं प्राप्तः स तस्यासीदतीवापनयो महान्}
{भोजराजन्यवृद्धैश्च पीड्यमानैर्दुरात्मना}


\twolineshloka
{ज्ञातित्राणमभीप्सद्भिरस्मत्सम्भावना कृता}
{दत्वाऽऽक्रूराय सुतनुं तामाहुकसुतां तदा}


\twolineshloka
{सङ्कर्षणद्वितीयेन ज्ञातिकार्यं मया कृतम्}
{हतौ कंससुनामानौ मया रामेण चाप्युत}


\twolineshloka
{`हत्वा कंसं तथैवादौ जरासन्धस्य बिभ्यतः}
{मया रामेण चान्यत्र ज्ञातयः परिपालितः'}


\twolineshloka
{भये तु समतिक्रान्ते जरासन्धे समुद्यते}
{मन्त्रोऽयं मन्त्रितो राजन्कुलैरष्टादशावरैः}


\twolineshloka
{अनारमन्तो निघ्नन्तो महास्त्रैः शत्रुघातिभिः}
{न हन्यामो वयं तस्य त्रिभिर्वर्षशतैर्बलम्}


\twolineshloka
{तस्य ह्यमरसङ्काशौ बलेन बलिनां वरौ}
{नामभ्यां हंसडिबिकावशस्त्रनिधनावुभौ}


\twolineshloka
{तावुभौ सहितौ वीरौ जरासन्धश्च वीर्यवान्}
{त्रयस्त्रयाणां लोकानां पर्याप्ता इति मे मतिः}


\twolineshloka
{न हि केवलमस्माकं यावन्तोऽन्ये च पार्थिवाः}
{तथैव तेषामासीच्च बुद्धिर्बुद्धिमतां वर}


\twolineshloka
{`अष्टादश मया तस्य सङ्ग्रामा रोमहर्षणाः}
{दत्ता न च हतो राजञ्जरासन्धो महाबलः'}


\twolineshloka
{अथ हंस इति ख्यातः कश्चिदासीन्महान्नृपः}
{रामेण स हतस्तत्र सङ्ग्रामेऽष्टादशावरे}


\twolineshloka
{हतो हंस इति प्रोक्तस्य केनापि भारत}
{तच्छ्रुत्वा डिबिको राजन्यमुनाम्भस्यमज्जत}


\twolineshloka
{विना हसेन लोकेऽस्मिन्नाहं जीवितुमुत्सहे}
{इत्येतां मतिमास्थाय डिबिको निधनं गतः}


\twolineshloka
{तथा तु डिबिकं श्रुत्वा हंसः परुपुरञ्जयः}
{प्रपेदे यमुनामेव सोपि तस्यां न्यमज्जत}


\twolineshloka
{तौ स राजा जराजन्धः श्रुत्वा च निधनं गतौ}
{पुरं शून्येन मनसा प्रययौ भरतर्षभ}


\twolineshloka
{ततो वयमित्रघ्न तस्मिन्प्रतिगते नृपे}
{पुनरान्दिनः सर्वे मधुरायां वसामहे}


\threelineshloka
{यदा त्वभ्येत्य पितरं सा वै राजीवलोचना}
{कंसभार्या जरासन्धं दुहिता मागधं नृपम्}
{चोदयत्येव राजेन्द्र पतिव्यसनदुःखिता}


\twolineshloka
{पतिघ्नं मे जहीत्येवं पुनः पुनररिन्दम}
{ततो वयं महाराज तं मन्त्रं पूर्वमन्त्रितम्}


\twolineshloka
{संस्मरन्तो विमनसो व्यपयाता नराधिप}
{पृथक्त्वेन महाराज सङ्क्षिप्य महतीं श्रियम्}


\twolineshloka
{पलायामो भयात्तस्य ससुतज्ञातिबान्धवाः}
{इति सञ्चिन्त्य सर्वे स्म प्रतीचीं दिशमाश्रिताः}


\twolineshloka
{कुशस्थलीं पुरीं रम्यां रैवतेनोपशोभिताम्}
{ततो निवेशं तस्यां च कृतवन्तो वयं नृप}


\twolineshloka
{तथैव दुर्गसंस्कारं देवैरपि दुरासदम्}
{स्त्रियोऽपि यस्यां युध्येयुः किमु वृष्णिमहारथाः}


\twolineshloka
{तस्यां वयममित्रघ्न निवसामोऽकुतोभयाः}
{आलोच्य गिरिमुख्यं तं मागधं तीर्णमेव च}


\twolineshloka
{माधवाः कुरुशार्दूल परां मुदमवाप्नुवन्}
{एवं वयं जरासन्धादभितः कृतकिल्बिषाः}


\twolineshloka
{सामर्थ्यवन्तः सम्बन्धाद्गोमन्तं समुपाश्रिताः}
{त्रियोजनायतं सद्म त्रिस्कन्धं योजनावधि}


\twolineshloka
{योजनान्ते शतद्वारं वीरविक्रमतोरणम्}
{अष्टादशावरैर्नद्धं क्षत्रियैर्युद्धदुर्मदैः}


\twolineshloka
{अष्टादश सहस्राणि भ्रातृणां सन्ति नः कले}
{आहुकस्य शतं पुत्रा एकैकस्त्रिदशावरः}


\twolineshloka
{चारुदेष्णः सह भ्रात्रा चक्रदेवोऽथ सात्यकिः}
{अहं च रोहिणेयश्च साम्बः प्रद्युम्न एव च}


\twolineshloka
{एवमेते रथाः सप्त राजन्नन्यान्निबोध मे}
{कृतवर्मा ह्यनाधृष्टिः समीकः समिर्तिजयः}


\twolineshloka
{कङ्कः शङ्कुश्च कुन्तिश्च सप्तैते वै महारथाः}
{`प्रद्युम्नश्चानिरुद्धश्च भानुरक्रूरसारणौ}


\twolineshloka
{निशठश्च गदश्चैव सप्त चैते महारथाः}
{विकमो झिल्लिबभ्रू च उद्धवोऽथ विदूरथः}


\twolineshloka
{वसुदेवोग्रसेनौ च सप्तैते मन्त्रिपुङ्गवाः}
{प्रसेनजिच्च यमलो राजराजगुणान्वितः}


\twolineshloka
{स्यमन्तको मणिर्यस्य रुक्मं निस्रुवते बहु}
{पुत्रौ चान्धकभोजस्य वृद्धो राजा च ते दश}


\twolineshloka
{वज्रसंहनना वीरी वीर्यवन्तो महाबलाः}
{स्मरन्तो मध्यमं देशं वृष्णिवीरा गतज्वराः}


\twolineshloka
{पाण्डवैश्चापि सततं नाथवन्तो वयं नृप}
{सर्वसम्पद्गुणैः सिद्धे तस्मिन्नेवं व्यवस्थिते}


\twolineshloka
{क्षत्रे सम्राजमात्मानं कर्तुमर्हसि भारत}
{दुर्योधनं शान्तनवं द्रोणं द्रौणायनिं कृपम्}


\twolineshloka
{कर्णं च शिशुपालं च रुक्मिणं च धनुर्धरम्}
{एकलव्यं द्रुमं श्वेतं शैब्यं शकुनिमेव च}


\twolineshloka
{एतानजित्वा सङ्ग्रामे कथं शक्नोषि तं क्रतुम्}
{तथैते गौरवेणैव न योत्स्यन्ति नराधिपाः}


\twolineshloka
{एकस्तत्र बलोन्मत्तः कर्णो वैकर्तनो वृषा}
{योत्स्यते स परामर्षी दिव्यास्रबलगर्वितः'}


% Check verse!
न तु शक्यं जरासन्धे जीवमाने महाबलराजसूयस्त्वयाऽवाप्तुमेषा राजन्मतिर्मम
\twolineshloka
{तेन रुद्धा हि राजानः सर्वे जित्वा गिरिव्रजे}
{कन्दरे पर्वतेन्द्रस्य सिंहेनेव महाद्विपाः}


\threelineshloka
{स हि राजा जरासन्धो यियक्षुर्वसुधाधिपैः}
{`अभिषिक्तः स राजन्यैः सहस्रैरुत चाष्टभिः'}
{महादेवं महात्मानमुमापतिमरिन्दम}


\twolineshloka
{आराध्य तपसोग्रेण निर्जितास्तेन पार्थिवाः}
{प्रतिज्ञायाश्च पारं स गतः पार्थिवसत्तम}


\twolineshloka
{स हि निर्जित्य निर्जित्य पार्थिवान्पृतनागतान्}
{पुरमानीय बद्ध्वा च चकार पुरुषव्रजम्}


\twolineshloka
{वयं चैव महाराज जरासन्धभयात्तदा}
{मधुरा सम्परित्यज्य गता द्वारवतीं पुरीम्}


\twolineshloka
{यदि त्वेनं महाराज यज्ञं प्राप्तुमभीप्ससि}
{यतस्व तेषां मोक्षाय जरासन्धवधाय च}


\twolineshloka
{समारम्भो न शक्योऽयमन्यथा कुरुनन्दन}
{राजसूयस्य कार्त्स्न्येन कर्तुं मतिमतां वर}


\twolineshloka
{इत्येषा मे मती राजन्यथा वा मन्यसेऽनघ}
{एवं गते ममाचक्ष्व स्वयं निश्चित्य हेतुभिः}


\chapter{अध्यायः १५}
\twolineshloka
{उक्तं त्वया बुद्धिमता यन्नान्यो वक्तुमर्हति}
{संशयानां हि निर्मोक्ता त्वन्नान्यो विद्यते भुवि}


\threelineshloka
{गृहे गृहे हि राजानः स्वस्य स्वस्य प्रियङ्कराः}
{न च साम्राज्यमाप्तास्ते सम्रादशब्दो हि कृच्छ्रभाक्}
{}


\twolineshloka
{कथं परानुभावज्ञः स्वं प्रशंसितुमर्हति}
{परेण समवेतस्तु यः प्रशस्यः स पूज्यते}


\twolineshloka
{विशाला बहुला भूमिर्बहुरत्नसमाचिता}
{दूरं गत्वा विजानाति श्रेयो वृष्णिकुलोद्वह}


\twolineshloka
{शममेव परं मन्ये शमात्क्षेमं भवेन्मम}
{आरम्भे पारमेष्ठ्यं तु न प्राप्यमिति मे मतिः}


\threelineshloka
{एवमेते हि जानन्ति कुले जाता मनस्विनः}
{कश्चित्कदाचिदेतेषां भवेच्छ्रेष्ठो जनार्दन}
{}


\twolineshloka
{वयं यैव महाभाग जरासन्धभयात्तदा}
{शङ्किताः स्म महाभाग द्वौरात्म्यात्तस्य चानघ}


\twolineshloka
{अहं हि तव दुर्धर्ष भुजवीर्याश्रयः प्रभो}
{नात्मानं बलिनं मन्ये त्वयि तस्माद्विशङ्किते}


\threelineshloka
{त्वत्सकाशाच्च रामाच्च भीमसेनाच्च माधव}
{अर्जुनाद्वा महाबाहो हन्तुं शक्यो नवेति वै}
{एवं जानन्हि वार्ष्णेय विमृशामि पुनः पुनः}


\twolineshloka
{त्वं मे प्रमाणभूतोऽसि सर्वकार्येषु केशव}
{तच्छ्रुत्वा चाब्रवीद्भीमो वाक्यं वाक्यविशारदः ॥भीम उवाच}


\twolineshloka
{अनारम्भपरो राजा वल्मीक इव सीदति}
{दुर्बलश्चानुपायेन बलिनं योऽधितिष्ठति}


\twolineshloka
{अतन्द्रितश्च प्रायेण दुर्बलो बलिनं रिपुम्}
{जयेत्सम्यक्प्रयोगेण नीत्याऽर्थानात्मनो हितान्}


\twolineshloka
{कृष्णे नयो मयि बलं जयः पार्थे धनञ्जये}
{मागधं साधयिष्याम इष्टिं त्रय इवाग्रयः}


\threelineshloka
{`त्वद्बुद्धिबलमाश्रित्य सर्वं प्राप्स्यति धर्मराद}
{जयोऽस्माकं हि गोविन्द येषां नाथो भवान्सदा' ॥कृष्ण उवाच}
{}


\twolineshloka
{अर्थानारभते बालो नानुबन्धमवेक्षते}
{तस्मादरिं न मृष्यन्ति बालमर्थपरायणम्}


\twolineshloka
{जित्वा जय्यान्यौवनाश्विः पालनाच्च भगीरथः}
{कार्तवीर्यस्तपोवीर्याद्बलात्तु भरतो विभुः}


\twolineshloka
{ऋद्ध्या मरुत्तस्तान्पश्च सम्राजस्त्वनुशुश्रुम}
{`सर्वान्वंश्याननुमृशन्नैते सन्ति युगे युगे'}


\twolineshloka
{साम्राज्यमिच्छतस्ते तु सर्वाकारं युधिष्ठिर}
{निग्राह्यलक्षणं प्राप्तिर्धर्मार्थनयलक्षणैः}


\twolineshloka
{बार्हद्रथो नरासन्धस्तद्विद्धि भरतर्षभ}
{न चैनं प्रत्यत्युद्ध्यन्त कुलान्येकशतं नृपाः}


% Check verse!
तस्मादिह बलादेव साम्राज्यं कुरुते हि सः
\twolineshloka
{रत्नभाजो हि राजानो जरासन्धमुपासते}
{न च तुष्यति तेनापि बाल्यादनयमास्थितः}


\twolineshloka
{मूर्धाभिषिक्तं नृपतिं प्रधानपुरुषो बलात्}
{आदत्ते न च नो दृष्टोऽभागः पुरुषतः क्वचित्}


\twolineshloka
{एवं सर्वान्वशे चक्रे जरासन्धः शतावरान्}
{तं दुर्बलतरो राजा कथं पार्थ उपैष्यति}


\fourlineindentedshloka
{`तण्डुलप्रस्थके राजा कपर्दिनमुपासते'}
{प्रोक्षितानां प्रमृष्टानां राज्ञां पशुपतेर्गृहे}
{पशूनामिव प्रमृष्टानां राज्ञां पशुपतेर्गृहे}
{}


\twolineshloka
{क्षत्रियः शस्त्रमरणो यदा भवति सत्कृतः}
{ततः स्म मागधं सङ्ख्ये प्रतिबाधेम यद्वयम्}


\twolineshloka
{षडशीतिः समानीताः शेषा राजंश्चतुर्दश}
{जरासन्धेन राजानस्ततः क्रूरं प्रवर्त्स्यते}


\twolineshloka
{प्राप्नुयात्स यशो दीप्तं तत्र यो विघ्नमाचरेत्}
{जयेद्यश्च जरासन्धं स सम्राण्णियतं भवेत्}


\chapter{अध्यायः १६}
\twolineshloka
{सम्राड्गुणमभीप्सन्वै युष्मान्स्वार्थपरायणः}
{कथं प्रहिणुयां कृष्ण सोऽहं केवलसाहसात्}


\twolineshloka
{भीमार्जुनावुभौ नेत्रे मनो मन्ये जनार्दनम्}
{मनश्चक्षुर्विहीनस्य कीदृशं जीवितं भवेत्}


\twolineshloka
{जरासन्धबलं प्राप्य दुष्पारं भीमविक्रमम्}
{यमोपि न विजेताऽऽजौ तत्र वः किं विचेष्टितम्}


\twolineshloka
{अस्मिंस्त्वर्थान्तरे युक्तमनर्थः प्रतिपद्यते}
{तस्मान्न प्रतिपत्तिस्तु कार्या युक्ता मता मम}


\threelineshloka
{यथाऽहं विमृशाम्येकस्तत्तावच्छ्रूयतां मम}
{संन्यासं रोचये साधु कार्यस्यास्य जनार्दन}
{प्रतिहन्ति मनो मेऽद्य राजसूयो दूराहरः ॥वैशम्पायन उवाच}


\threelineshloka
{पार्थः प्राप्य धनुः श्रेष्ठमक्षय्यौ च महेषुधी}
{रथं ध्वजं हयांश्चैव युधिष्ठिरमभाषत ॥अर्जुन उवाच}
{}


\threelineshloka
{धनुः शस्त्रं शरा वीर्यं पक्षो भूमिर्यशो बलम्}
{प्राप्तमेतन्मय राजन्दुष्प्रापं यदभीप्सितम्}
{}


\threelineshloka
{कुले जन्म प्रशंसन्ति वैद्याः साधु सुनिष्ठिताः}
{बलेन सदृशं नास्ति वीर्यं तु मम रोचते}
{}


\twolineshloka
{कृतवीर्यकुले जातो निर्वीर्यः किं करिष्यति}
{निर्वीर्ये तु कुले जातो वीर्यवांस्तु विशिष्यते}


\twolineshloka
{क्षत्रियः सर्वशो राजन्यस्य वृत्तिर्द्विषज्जये}
{सर्वैगुणैर्विहीनोऽपि वीर्यवान्हि तरेन्द्रिपून्}


\twolineshloka
{सर्वैरपि गुणैर्युक्तो निर्वीर्यः किं करिष्यति}
{जयस्य हेतुः सिद्धिर्हि कर्म दैवं च संश्रितम्}


% Check verse!
संयुक्तो हि बलैः कश्चित्प्रमादान्नोपयुज्यते
% Check verse!
तेन द्वारेण शत्रुभ्यः क्षीयते सबलो रिपुः
\twolineshloka
{दैन्यं यथा बलवति तथा मोहो बलान्विते}
{तावुभौ नाशकौ हेतू राज्ञा त्याज्यौ जयार्थिना}


\twolineshloka
{जरासन्धिविनाशं च राज्ञां च परिरक्षणम्}
{यदि कुर्याम् यज्ञार्थं किं ततः परमं भवेत्}


\twolineshloka
{अनारम्भे हि नियतो भवेदगुणनिश्चयः}
{गुणान्निः संशयाद्राजन्नैर्गुण्यं मन्यसे कथम्}


\twolineshloka
{काषायं सुलभं पश्चान्मुनीनां शममिच्छताम्}
{साम्राज्यं तु भवेच्छक्यं वयं योत्स्यामहे परान्}


\chapter{अध्यायः १७}
\twolineshloka
{जातस्य भारते वंशे तथा कुन्त्याः सुतस्य च}
{या वै युक्ता मतिः सेयमर्जुनेन प्रदर्शिता}


\twolineshloka
{न स्म मृत्युं वयं विद्म रात्रौ वा यदि वा दिवा}
{न चापि कञ्चिदमरमयुद्धेनानुशुश्रुम}


\twolineshloka
{एतावदेव पुरुषैः कार्यं हृदयतोषणम्}
{नयेन विधिदृष्टेन यदुपक्रमते परान्}


\twolineshloka
{सुनयस्यानपायस्य संयोगे परमः क्रमः}
{सङ्गत्या जायतेऽसाम्यं साम्यं च न भवेद्द्वयोः}


\twolineshloka
{अनयस्यानुपायस्य संयुगे परमः क्षयटः}
{संशयो जायते साम्याज्जयश्च न भवेद्द्वयोः}


\twolineshloka
{ते वयं नयमास्थाय शत्रुदेशसमीपगाः}
{कथमन्तं न गच्छेम वृक्षस्येव नदीरयाः ॥पररन्ध्रे पराक्रान्ताः स्वरन्ध्रावरणे स्थिताः}


\twolineshloka
{व्यूढानीकैरतिबलैर्न युद्व्येदरिभिः सह}
{इति बुद्धिमतां नीतिस्तन्ममापीह रोचते}


\twolineshloka
{अनवद्या ह्यसम्बुद्धाः प्रविष्टाः शत्रुसद्म तत्}
{शत्रुदेशमुपाक्रम्य तं कामं प्राप्नुयामहे}


\twolineshloka
{एको ह्येव श्रियं नित्यं बिभर्ति पुरुषर्षभः}
{अन्तरात्मेव भूतानां तत्क्षयं नैव लक्षये}


\twolineshloka
{अथवैनं निहत्याजौ शेषेणापि समाहताः}
{प्राप्नुयाम ततः स्वर्गं ज्ञातित्राणपरायणाः ॥युधिष्ठिर उवाच}


\twolineshloka
{कृष्ण कोऽयं जरासन्धः किंवीर्यः किम्पराक्रमः}
{यस्त्वां स्पृष्ट्वाऽग्निसदृशं न दग्धः शलभो यथा ॥कृष्ण उवाच}


\twolineshloka
{शृणु राजञ्जरासन्धो यद्वीर्यो यत्पराक्रमः}
{यथा चोपेक्षितोऽस्माभिर्बहुशः कृतविप्रियः}


\twolineshloka
{अक्षौहिणीनां तिसृणां पतिः समरदर्पितः}
{राजा बृहद्रथो नाम मगधाधिपतिर्बली}


\twolineshloka
{रूपवान्वीर्यसम्पन्नः श्रीमानतुलविक्रमः}
{नित्यं दीक्षाङ्किततनुः शतक्रतुरिवापरः}


\twolineshloka
{तेजसा सूर्यसङ्काशः क्षमया पृथिवीसमः}
{यश्चान्तकसमः क्रोधे श्रिया वैश्रवणोपमः}


\threelineshloka
{`स्वराज्यं कारयामास मगधेषु गिरिव्रजे'}
{तस्याभिजनसंयुक्तैर्गणैर्भरतसत्तम}
{व्याप्तेयं पृथिवी सर्वा सूर्यस्येव गभस्तिभिः}


\threelineshloka
{स काशिराजस्य सुते यमजे भरतर्षभः}
{उपयेमे महावीर्यो रूपद्रविणसंयुते}
{2-17-17cतयोश्चकार समयं मिथःस पुरुषर्षभः}


\twolineshloka
{नातिवर्तिष्य इत्येवं पत्नीभ्यां सन्निधौ तदा}
{स ताभ्यां शुशुभे राजा पत्नीभ्यां वसुधाधिपः}


\twolineshloka
{प्रियाम्भामनुरूपाभ्यां करेणुभ्यामिव द्विपः}
{तयोर्मध्यगतश्चापि रराज वसुधाधिपः}


\twolineshloka
{गङ्गायमुनयोर्मध्ये मूर्तिमानिव सागरः}
{विषयेषु निमग्रस्य तस्य यौवनमत्यगात्}


\twolineshloka
{न च वंशकरः पुत्रस्तस्याजायत कश्चन}
{मङ्गलैर्बभिर्होमैः पुत्रकामाभिरिष्टिभिः}


\twolineshloka
{नाससाद नृपश्रेष्ठः पुत्रं कुलविवर्धनम्}
{स भार्याभ्यां च सहितो निर्वेदमगमद्धृशम्}


\twolineshloka
{`राज्यं चापि परित्यज्य तपोवनमथाश्रयत्}
{'वार्यमाणः प्रकृतिभिर्नृपभक्त्या विशाम्पते'}


% Check verse!
अथ काक्षीवतः पुत्रं गौतमस्य महात्मनः ॥शुश्राव तपसि श्रेष्ठमुदारं चण्डकौशिकम्
\twolineshloka
{यदृच्छयाऽऽगतं तं तु वृक्षमूलमुपाश्रितम्}
{पत्नीभ्यां सहितो राजा सर्वयत्नैरतोषयत्}


\twolineshloka
{`बृहद्रथं च स ऋषिर्यथावच्चाभ्यनन्दत}
{'उपविष्टः स तेनाथ अनुज्ञातो महात्मना}


\twolineshloka
{तमपृच्छत्तदा विप्रः किमागमनमित्यथ}
{विप्रैरनुगतस्यैव पत्नीभ्यां सहितस्य च}


\twolineshloka
{स उवाच मुनिं राजा भगवन्नास्ति मे सुतः}
{अपुत्रस्य तु राज्येन वृद्धत्वे किं प्रयोजनम्}


\threelineshloka
{सोऽहं तपश्चरिष्यामि पत्नीभ्यां सहितो वने}
{नाप्रजस्य मुने किर्तिः स्वधा चैवाक्षया भवेत्}
{एवमुक्तः स राज्ञा तु मुनिः कारुण्यमागतः}


\twolineshloka
{तमब्रवीत्सत्यधृतिः सत्यवागृषिसत्तमः}
{परितुष्टोऽस्मि राजेन्द्र वरं वरय सुव्रत}


\twolineshloka
{ततः सभार्यः प्रणतस्तमुवाच बृहद्रथः}
{पुत्रदर्शननैराश्याद्बाष्पसन्दिग्धया गिरा ॥राजोवाच}


\threelineshloka
{भगवन् राज्यमुत्सृज्य प्रस्थितस्य तपोवनम्}
{किं वरेणाल्पभाग्यस्य किं राज्येनाप्रजस्य मे ॥कृष्ण उवाच}
{}


\twolineshloka
{एतच्छ्रुत्वा मुनिर्ध्यानमगमन्क्षुभितेन्द्रियः}
{तस्यैव चाम्रवृक्षस्य छायायां समुपाविशत}


\twolineshloka
{तस्योपविष्टस्य मुनेरुत्सङ्गे निपपात ह}
{अवानमशुकादष्टमेकमाम्रफलं किल}


\twolineshloka
{तत्प्रगृह्य मुनिश्रेष्ठो हृदयेनाभिमन्त्र्य च}
{राज्ञे ददावप्रतिमं पुत्रसम्प्राप्तिकारणम्}


\twolineshloka
{उवाच च महाप्राज्ञस्तं राजानं महामुनिः}
{गच्छ राजन्कृतार्थोऽसि निवर्तस्व नराधिप}


\twolineshloka
{`एष ते तनयो राजन्मा तपेह तपोवने}
{प्रजाः पालय धर्मेण एव धर्मो महीक्षिताम्}


\twolineshloka
{यजस्व विविधैर्यज्ञैरिन्द्रं तर्पय चेन्दुना}
{पुत्रं राज्ये प्रतिष्ठाप्य तत आश्रममाव्रज}


\twolineshloka
{अष्टौ वरान्प्रयच्छामि तव पुत्रस्य पार्थिव}
{ब्रह्मण्यत्वमजेयत्वं युद्धेषु च तथा मतिः}


\twolineshloka
{प्रियातिथेयतां चैव दीनानामन्ववेक्षणम्}
{तथा बलं च सुभहल्लोके कीर्ति च शाश्वतीम्}


\twolineshloka
{अनुरागं प्रजानां चेत्येवमष्टौ वरान्नृप}
{गच्छ त्वं कृतकृत्योऽसि निवर्तस्व जनाधिप'}


\twolineshloka
{अनुज्ञातः स ऋषिणा पत्नीभ्यां सहितो नृपः}
{पौरैरनुगतश्चापि विवेश स्वपुरं ततः}


\twolineshloka
{यथासमयमाज्ञाय तदा स नृपसत्तमः}
{द्वाभ्यामेकं फलं प्रादात्पत्नीभ्यां भरतर्षभ}


\threelineshloka
{मुनेश्च बहुमानेन कालस्य च विपर्ययात्}
{ते तदाम्रं द्विधा कृत्वा भक्षयामासतुः शुभे}
{}


\twolineshloka
{तयोः समभवद्गर्भः फलप्राशनसम्भवः}
{ते च दृष्ट्वा स नृपतिः परां मुदमवाप ह}


\twolineshloka
{अथ काले महाप्राज्ञ यथासमयमागते}
{प्रजायेतामुभे राजञ्शरीरशकले तदा}


\twolineshloka
{एकाक्षिबाहुचरणे अर्धोदरमुखस्फिचे}
{दृष्ट्वा शरीरशकले प्रवेपतुरुभे भृशम्}


\twolineshloka
{उद्विग्रे सह संमन्त्र्य ते भगित्यौ तदाऽबले}
{सजीवे प्राणिशकले तत्यजाते सुदुःखिते}


\twolineshloka
{तयोर्धात्र्यौ सुसंवीते कृत्वा ते गर्भसम्प्लवे}
{निर्गम्यान्तः पुरद्वारात्समुत्सृज्याभिजग्मतुः}


\twolineshloka
{`दुकूलाभ्यां सुसञ्छन्ने पाण्डराभ्यामुभे तदा}
{अज्ञाते कस्यचित्ते तु जहतुस्ते चतुष्पथे}


\twolineshloka
{ततो विविशतुर्धात्र्यौ पुनरन्तः पुरं तदा}
{कथयामासतुरुभे देवीभ्यां तु पृथक्पृथक्'}


\twolineshloka
{ते चतुष्पथनिक्षिप्ते जरा नामाथ राक्षसी}
{जग्राह मनुजव्याघ्र मांसशोणितभोजना}


\twolineshloka
{कर्तुकामा सुखवहे शकले सा तु राक्षसी}
{संयोजयामास तदा विधानबलचोदिता}


\twolineshloka
{ते समानीतमात्रे तु शकले पुरुषर्षभ}
{एकमूर्तिधरो वीरः कुमारः समपद्यत}


\twolineshloka
{ततः सा राक्षसी राजन्विस्मयोत्फुल्ललोचना}
{न शशाक समुद्वोदुं वज्रसारमयं शिशुम्}


\twolineshloka
{बालस्ताम्रतलं मुष्टिं कृत्वा चास्ये निधाय सः}
{प्राक्रोशदतिसंरब्धः सतोय इव तोयदः}


\twolineshloka
{तेन शब्देन सम्भ्रान्तः सहसाऽन्तः पुरे जनः}
{निर्जगाम नरव्याघ्र राज्ञा सह परन्तप}


\twolineshloka
{ते चाबले परिम्लाने पयः पूर्णपयोधरे}
{निराशे पुत्रलाभाय सहसैवाब्यगच्छताम्}


\twolineshloka
{ते च दृष्ट्वा तथाभूते राजानं चेष्टसंततिम्}
{तं च बालं सुबलिनं चिन्तयामास राक्षसी}


\twolineshloka
{नार्हामि विषये राज्ञो वसन्ती पुत्रगृद्धिनः}
{बालं पुत्रमिमं हन्तुं धार्मिकस्य महात्मनः}


\twolineshloka
{सा तं बालमुपादाय मेघलेखेन भास्करम्}
{कृत्वा च मानुषं रूपमुवाच वसुधाधिपम्}


\fourlineindentedshloka
{बृहद्रथ सुतस्तेऽयं मया दत्तः प्रगृह्यताम्}
{तव पत्नीद्वये जातो द्विजातिवरशासनात्}
{धात्रीजनपरित्यक्तो मयाऽयं परिरक्षितः ॥कृष्ण उवाच}
{}


\twolineshloka
{ततस्ते भरतश्रेष्ठ काशिराजसुते शुभे}
{तं बालमभिपद्याशु प्रस्रवैरभ्यषिञ्जताम्}


\twolineshloka
{ततः स राजा संहृष्टः सर्वं तदुपलभ्य च}
{अपृच्छद्धेमगर्भाभां राक्षसीं तामराक्षसीम्}


\twolineshloka
{कात्वं कमलगर्भाभे मम पुत्रप्रदायिनी}
{कामं मा ब्रूहि कल्याणि देवता प्रतिभासि मे}


\chapter{अध्यायः १८}
\twolineshloka
{जरा नामास्मि भद्रं ते राक्षसी कामरूपिणी}
{तव वेश्मनि राजेन्द्र पूजिता न्यवसं सुखम्}


\twolineshloka
{गृहे गृहे मनुष्याणां नित्यं तिष्ठामि राक्षसी}
{गृहदेवीति नाम्ना वै पुरा सृष्टा स्वयंम्भुवा}


\twolineshloka
{दानवानां विनाशाय स्थापिता दिव्यरूपिणी}
{यो मां भक्त्या लिखेत्कुड्ये सपुत्रां यौवनान्विताम्}


\twolineshloka
{गृहे तस्य भवेद्वृद्धिरन्यथा क्षयमाप्नुयात्}
{त्वद्गृहे तिष्ठमानाऽहं पूजिताऽहं सदा विभो}


\twolineshloka
{लिखिता चैव कुड्येषु पुत्रैर्बहुभिरावृता}
{गन्धपुष्पैस्तथा धूपैर्भक्ष्यभोज्यैः सुपूजिता}


\threelineshloka
{साऽहं प्रत्युपकारार्थं चिन्तयन्त्यनिशं नृप}
{तवेमे पुत्रशकले दृष्टवत्यस्मि धार्मिका}
{}


\twolineshloka
{संश्लिषिते मया दैवात्कुमारः समपद्यत}
{तव भाग्यान्महाराज हेतुमात्रमहं त्विह}


\twolineshloka
{मेरुं वा खादितुं शक्ता किं पुनस्तव बालकम्}
{गृहसम्पूजनात्तुष्ट्या मया प्रत्यर्पितस्तव}


\fourlineindentedshloka
{मम नाम्ना च लोकेऽस्मिन्ख्यात एव भविष्यति}
{कृष्ण उवाच}
{एवमुक्त्वा तु सा राजंस्तत्रैवान्तरधीयत}
{स सङ्गृह्य कुमारं तं प्रविवेश गृहं नृपः}


\twolineshloka
{तस्य बालस्य यत्कृत्यं तच्चकार नृपस्तदा}
{आज्ञापयच्च राक्षस्या मगधेषु महोत्सवम्}


\twolineshloka
{तस्य नामाकरोच्चैव पितामहसमः पिता}
{जरया सन्धितो यस्माज्जरासन्धो भवत्वयम्}


\twolineshloka
{सोऽवर्धत महातेजा मागधाधिपतेः सुतः}
{प्रमाणबलसम्पन्नो हुताहुतिरिवानलः}


\twolineshloka
{`एवं स ववृधे राजन्कुमारः पुष्करेक्षणः}
{कालेन महता चापि यौवनस्थो बभूव ह' ॥मातापित्रोर्नन्दकरः शुक्लपक्षे यथा शशी}


\chapter{अध्यायः १९}
\twolineshloka
{कस्यचित्त्वथ कालस्य पुनरेव महातपाः}
{मगधेषुपजक्राम भगवांश्चण्डकौशिकः}


\twolineshloka
{तस्याऽऽगमनसंहृष्टः सामात्यः सपुरः सरः}
{सभार्यः सह पुत्रेण निर्जगाम बृहद्रथः}


\twolineshloka
{पाद्यार्घ्याचमनीयैस्तमर्चयामास भारत}
{स नृपो राज्यसहितं पुत्रं तस्मै न्यवेदयत्}


\twolineshloka
{प्रतिगृह्य च तां पूजां पार्थिवाद्भगवानृषिः}
{उवाच मागधं राजन्प्रहृष्टेनान्तरात्मना}


\twolineshloka
{सर्वमेतन्मया ज्ञातं राजन्दिव्येन चक्षुषा}
{पुत्रस्तु शृणु राजेन्द्र यादृशोऽयं भविष्यति}


\twolineshloka
{अस्य रूपं च सत्वं च बलमूर्जितमेव च}
{एष श्रिया समुदितः पुत्रस्तव न संशयः}


\twolineshloka
{प्रापयिष्यति तत्सर्वं विक्रमेण समन्वितः}
{अस्य वीर्यवतो वीर्यं नानुयास्यन्ति पार्थिवाः}


\twolineshloka
{पततो वैनतेयस्य गतिमन्ये यथा खगाः}
{विनाशमुपयास्यन्ति ये चास्य परिपन्थिनः}


\twolineshloka
{देवैरपि विसृष्टानि शस्त्राण्यस्य महीपते}
{न रुजं जनयिष्यन्ति गिरेरिव नदीरयाः}


\twolineshloka
{सर्वमूर्धाभिषिक्तानामेव मूर्ध्नि ज्वलिष्यति}
{प्रभाहरोऽयं सर्वेषां ज्योतिषामिव भास्करः}


\twolineshloka
{एनमासाद्य राजानः समृद्धबलवाहनाः}
{विनाशमुपयास्यन्ति शलभा इव पावकम्}


\twolineshloka
{एष श्रियः समुदिताः सर्वराज्ञां ग्रहीष्यति}
{वर्षास्विवोदीर्णजला नदीर्नदनदीपतिः}


\twolineshloka
{एष धारयित सम्यक्चातुर्वर्ण्यं महाबलः}
{शुभां शुभवतीं स्फीतां सर्वसस्यधरां धराम्}


\twolineshloka
{अस्याज्ञावशगाः सर्वे भविष्यन्ति नराधिपाः}
{सर्वभूतात्मभूतस्य वायोरिव शरीरिणः}


\twolineshloka
{एष रुद्रं महादेवं त्रिपुरान्तकरं हरम्}
{सर्वलोकेष्वतिबलः साक्षाद्द्रक्ष्यति मागधः}


\twolineshloka
{एवं ब्रुवन्नेव मुनिः स्वकार्यमिव चिन्तयन्}
{विसर्जयामास नृपं बृहद्रथमथारिहन्}


\twolineshloka
{प्रविश्य नगरीं चापि ज्ञातिसम्बन्धिभिर्वृतः}
{अभिषिच्य जरासन्धं मगधाधिपतिस्तदा}


\threelineshloka
{बृहद्रथो नरपतिः परां निर्वृतिमाययौ}
{अभिषिक्ते जरासन्धे तदा राजा बृहद्रथः}
{पत्नीद्वयेनानुगतस्तपोवनचरोऽभवत्}


\twolineshloka
{ततो वनस्थे पितरि मातृभ्यां सह भारत}
{जरासन्धः स्ववीर्येण पार्थिवानकरोद्वशे}


\twolineshloka
{अथ दीर्घस्य कालस्य तपोवनचरो नृपः}
{सभार्यः स्वर्गमगमत्तपस्तप्त्वा बृहद्रथः}


\twolineshloka
{जरासन्धोऽपि नृपतिर्यथोक्तं कौशिकेन तत्}
{वरप्रदानमखिलं प्राप्य राज्यमपालयत्}


\threelineshloka
{हते चैव मया कंसे सहंसडिभिके तदा}
{जरासन्धस्य दुहिता रोदते पार्श्वतः पितुः}
{जातो वै वैरनिर्बन्धो मयासीत्तत्र भारत}


\twolineshloka
{भ्रामयित्वा शतगुणमेकोनं येन भारत}
{गदा क्षिप्ता बलवता मागधेन गिरिव्रजात्}


\twolineshloka
{तिष्ठतो मथुरायां वै कुत्स्नस्याद्भुतकर्मणः}
{एकोनयोजनशते सा पपात गदा शुभा}


\twolineshloka
{दृष्ट्वा पौरैस्तदा सम्यग्गदा चैव निवेदिता}
{गदावसानं तत्ख्यातं मथुरायाः समीपतः}


\twolineshloka
{तस्यास्तां हंसडिभिकावशस्त्रनिधनावुभौ}
{मन्त्रे मतिमतां श्रेष्ठौ नीतिशास्त्रे विशारदौ}


\twolineshloka
{यौ तौ मया ते कथितौ पूर्वमेव महाबलौ}
{त्रयस्त्रयाणां लोकानां पर्याप्ता इति मे मतिः}


\twolineshloka
{एवमेष तदा वीर बलिभिः कुकुरान्धकैः}
{वृष्णिभिश्च महाराज नीतिहेतोरुपेक्षितः}


\chapter{अध्यायः २०}
\twolineshloka
{पतितौ हंसडिभिकौ कंसश्च सगणो हतः}
{जरासन्धस्य निधने कालोऽयं समुपागतः}


\twolineshloka
{न शक्योऽसौ रणे जेतुं सर्वैरपि सुरासुरैः}
{प्राणयुद्धेन जेतव्यः स इत्युपलभामहे}


\twolineshloka
{मयि नीतिर्बलं भीमे रक्षिता चावयोर्जयः}
{मागधं साधयिष्याम इष्टिं त्रय इवाग्नयः}


\twolineshloka
{त्रिभिरासादितोऽस्माभिर्विजने स नराधिपः}
{न सन्देहो यथा युद्धमेकेनाप्युपयास्यति}


\twolineshloka
{अवमानाच्च लोभाच्च बाहुवीर्याच्च दर्पितः}
{भीमसेनेन युद्धाय ध्रुवमप्युपयास्यति}


\twolineshloka
{अलं तस्य महाबाहुर्भीमसेनो महाबलः}
{लोकस्य समुदीर्णस्य निधनायान्तको यथा}


\twolineshloka
{यदि भीमबलं वेत्सि यदि ते प्रत्ययो मयि}
{भीमसेनार्जुनौ शीघ्रं न्यासभूतौ प्रयच्छ मे ॥वैशम्पायन उवाच}


\threelineshloka
{एवमुक्तो भगवता प्रत्युवाच युधिष्ठिरः}
{भीमार्जुनौ समालोक्य सम्प्रहृष्टमुखौ स्थितौ ॥युधिष्ठर उवाच}
{}


\twolineshloka
{अच्युताच्युत मामैवं व्याहरामित्रकर्शन}
{पाण्डवानां भवान्नाथो भवन्तं चाश्रिता वयम्}


\twolineshloka
{यथा वदसि गोविन्द सर्वं तदुपपद्यते}
{नहि त्वमग्रतस्तेषां येषां लक्ष्मीः पराङ्मुखी}


\threelineshloka
{`येषामभिमुखी लक्ष्मीस्तेषां कृष्ण त्वमग्रतः'}
{निहतश्च जरासन्धो मोक्षिताश्च महीक्षितः}
{2-2--11c राजसूयश्च मेलब्धो निदेशे त्व तिष्ठतः}


\threelineshloka
{क्षिप्रमेव यथा त्वेतत्कार्यं समुपपद्यते}
{अप्रमत्तो जगन्नाथ तथा कुरु समुपपद्यते}
{}


\twolineshloka
{त्रिभिर्भवद्भिर्हि विना नाहं जीवितुमुत्सहे}
{धर्मकामार्थरहितो रोगार्त इव दुःखितः}


\twolineshloka
{न शौरिणा विना पार्थो न शौरिः पाण्डवं विना}
{नाजेयोस्त्यनयोऽर्लोके कृष्णयोरिति मे मतिः}


\twolineshloka
{अयं च बलिनां श्रेष्ठः श्रीमानपि वृकोदरः}
{युवाभ्यां सहितो वीर किं न कुर्यान्महायशाः}


\twolineshloka
{सुप्रणीतो बलौघो हि कुरुते कार्यमुत्तमम्}
{अन्धं बलं जडं प्राहुः प्रणेतव्यं विचक्षणैः}


\twolineshloka
{यतो हि निम्नं भवति नयन्ति हि ततो जलम्}
{यतश्छिद्रं ततश्चापि नयन्ते धीवरा जलम्}


\twolineshloka
{तस्मान्नयविधानज्ञं पुरुषं लोकविश्रुतम्}
{वयमाश्रित्य गोविन्दं यतामः कार्यसिद्धये}


\twolineshloka
{एवं प्रज्ञानयबलं क्रियोपायसमन्वितम्}
{पुरस्कृर्वीत कार्येषु कृष्णकार्यार्थसिद्धये}


\fourlineindentedshloka
{एवमेव यदुश्रेष्ठ यावत्कार्याथिसिद्धये}
{अर्जुनः कृष्णमन्वेतु भीमोऽन्वेतु धनञ्जयम्}
{नयो जयो बलं चैव विक्रमे सिद्धिमेष्यति ॥वैशम्पायन उवाच}
{}


\twolineshloka
{एवमुक्तास्ततः सर्वे भ्रातरो विपुलौजसः}
{वार्ष्णेयः पाण्डवेयौ च प्रतस्थुर्मागधं प्रति}


\twolineshloka
{वर्चस्विनां ब्राह्मणानां स्नातकानां परिच्छदैः}
{आच्छाद्य सुहृदां वाक्यैर्मनोज्ञैरभिनन्दिताः}


\threelineshloka
{`माधवः पाण्डवेयौ च प्रतस्थुर्व्रतधारिणः'}
{अमर्षादभितप्तानां ज्ञात्यर्थं मुख्यतेजसाम्}
{रविसोमाग्निवपुषां दीप्तमासीत्तदा वपुः}


\twolineshloka
{इतं मेने जरासन्धं दृष्ट्वा भीमपुरोगमौ}
{एककार्यसमुद्यन्तौ कृष्णौ युद्धेऽपराजितौ}


\twolineshloka
{ईशौ हितौ महात्मानौ सर्वकार्यप्रवर्तिनौ}
{धर्मकामार्थलोकानां कार्याणां च प्रवर्तकौ}


\twolineshloka
{कुरुभ्यः प्रस्थितास्ते तु मध्येन कुरुजाङ्गलम्}
{रम्यं पद्मसरो गत्वा कालकूडमतीत्य च}


\twolineshloka
{गण्डकीं च महाशोणं सदानीरां तथैव च}
{एकपर्वतके नद्यः क्रमेणैत्याव्रजन्त ते}


\twolineshloka
{उत्तीर्य सरयूं रम्यां दृष्ट्वा पूर्वांश्च कोसलान्}
{अतीत्य जग्मुर्मिथिलां मालां चर्मण्वतीं नदीम्}


\twolineshloka
{अतीत्य गङ्गां शोणं च त्रयस्ते प्राङ्मुखास्तदा}
{कुशचीरच्छदा जग्मुर्मागधं क्षेत्रमच्युताः}


\twolineshloka
{ते शश्वद्गोधनाकीर्णमम्बुमन्तं शुभद्रुमम्}
{गोरथं गिरिमासाद्य ददृशुर्मागधं पुरम्}


\chapter{अध्यायः २१}
\twolineshloka
{एष पार्थ महान्भाति पशुमान्नित्यमम्बुमान्}
{निरामयः सुवेश्माढ्यो निवेशो मागधः शुभः}


\twolineshloka
{वैहारो विपुलः शैलो वाराहो वृषभस्तथा}
{तथा ऋषिगिरिस्तात शुभाश्चैत्यकपञ्चमाः}


\twolineshloka
{एते पञ्चमहाशृङ्गाः पर्वताः शीतलद्रुमाः}
{रक्षन्तीवाभिसंहत्य संहताङ्गा गिरिव्रजम्}


\twolineshloka
{पुष्पवेष्टितशाखाग्रैर्गन्धवद्भिर्मनोहरैः}
{निगूढा इव लोध्राणां वनैः कामिजनप्रियैः}


\threelineshloka
{`यत्र दीर्घतमा नाम ऋषिः परमन्त्रितः'}
{शूद्रायां गौतमो यत्र महात्मा संशितव्रतः}
{औशीनर्यामजनयत्काक्षीवाद्यान्सुतान्मुनिः}


\threelineshloka
{गौतमः प्रणयात्तस्माद्यथाऽसौ तत्र सद्मनि}
{भजते मागधं वंशं स नृपाणामनुग्रहः}
{}


\twolineshloka
{अङ्गवङ्गादयश्चैव राजानः सुमहाबलाः}
{गौतमक्षयमभ्येत्य रमन्ते स्म पुरार्जुन}


\twolineshloka
{वनराजीस्तु पश्येमाः पिप्लानां मनोरमाः}
{लोघ्राणां च शुभाः पार्थ गौतमौकः समीपजाः}


\twolineshloka
{अर्बुदः शक्रवापी च पन्नगौ शत्रुतापनौ}
{स्वस्तिकस्यालयश्चात्र मणिनागस्य चोत्तमः}


\twolineshloka
{अपारिहार्या मेघानां मागधा मनुना कृताः}
{कौशिको मणिमांश्चैव चक्राते चाप्यनुग्रहम्}


\twolineshloka
{`पाण्डरे विपुले चैव तथा वाराहकेऽपि च}
{चैत्यके चज गिरिश्रेष्ठे मादङ्गे च शिलोच्चये}


\twolineshloka
{एतेषु पर्वतेन्द्रेषु सर्वसिद्धसमालयाः}
{यतीनामाश्रमाश्चैव मुनीनां च महात्मनाम्}


\twolineshloka
{वृषभस्य तमालस्य महावीर्यस्य वै तथा}
{गन्धर्वरक्षसां चैव नागानां च तथाऽऽलयाः}


\threelineshloka
{कक्षीवतस्तपोवीर्यात्तपोदा इति विश्रुताः}
{पुण्यतीर्थाश्च ते सर्वे सिद्धानां चैव कीर्तिताः}
{मणेश्च दर्शनादेव भद्रं शिवमवाप्नुयात्'}


% Check verse!
एवं प्राप्य पुरं रम्यं दुराधर्पं समन्ततः ॥अर्थसिद्धिं त्वनुपमां जरासन्धोऽभिमन्यते
\threelineshloka
{वयमासादने तस्य दर्पमद्य हरेम हि}
{वैशम्पायन उवाच}
{एवमुक्त्वा ततः सर्वे भ्रातरो विपुलौजसः}


\twolineshloka
{वार्ष्णेयः पाण्डवौ चैव प्रतस्थुर्मागधं पुरम्}
{हृष्टपुष्टजनोपेतं चातुर्वर्ण्यसमाकुलम्}


\twolineshloka
{स्फीतोत्सवमनाधृष्यमासेदुश्च गिरिव्रजतम्}
{ततो द्वारमनासाद्य पुरस्य गिरिमुच्छ्रितम्}


\twolineshloka
{बार्हद्रथैः पूज्यमानं तथा नगरवासिभिः}
{मागधानां तु रुचिरं चैत्यकान्तरमाद्रवन्}


\twolineshloka
{यत्र मांसादमृषभमाससाद बृहद्रथः}
{तं हत्वा मासतालाभिस्तिस्रो भेरीकारयत्}


\twolineshloka
{स्वपुरे स्थापयामास तेन चानह्य चर्मणा}
{यत्र ताः प्राणदन्भेर्यो दिव्यपुष्पावचूर्णिताः}


\twolineshloka
{भङ्क्त्वा भेरीत्रयं तेऽपि चैत्यप्राकारमाद्रवन्}
{द्वारतोऽभिमुखाः सर्वे ययुर्नानायुधास्तधा}


\twolineshloka
{मागधानां सुरुचिरं चैत्यकं तं समाद्रवन्}
{शिरसीव समाघ्नन्तो जरासन्धं जिघांस्वः}


\twolineshloka
{स्थिरं सुविपुलं शृङ्गं सुमहत्तत्पुरातनम्}
{अर्चितं गन्धमाल्यैश्च सततं सुप्रतिष्ठितम्}


\twolineshloka
{विपुलैर्बाहुभिर्वीरस्तेऽभिहत्याभ्यपातयन्}
{ततस्ते मागधं हृष्टाः पुरं प्रविविशुस्तदा}


\twolineshloka
{एतस्मिन्नेव काले तु ब्राह्मणा वेदपारगाः}
{दृष्ट्वा तु दुर्निमित्तानि जरासन्धमदर्शयन्}


\threelineshloka
{पर्युग्न्यकुर्वंश्च नृपं द्विरदस्थं पुरोहिताः}
{ततस्तच्छान्तये राजा जरासन्धः प्रतापवान्}
{दीक्षितो नियमस्थोऽसावुपवासपरोऽभवत्}


\twolineshloka
{स्नातकव्रतिनस्ते तु बाहुशस्त्रा निरायुधाः}
{युयुत्सवः प्रविविशुर्जरासन्धेन भारत}


\twolineshloka
{भक्ष्यमाल्यापणानां च ददृशुः श्रियमुत्तमाम्}
{स्फीतां सर्वगुणोपेतां सर्वकामसमृद्धिनाम्}


\threelineshloka
{तां तु दृष्ट्वा समृद्धिं ते वीथ्यां तस्यां नरोत्तमाः}
{राजमार्गेण गच्छन्तः कृष्णभीमधनञ्जयाः}
{बलाद्गृहीत्वा माल्यानि मालाकारान्महाबलाः}


\twolineshloka
{`कर्पूरशृङ्गं कोष्ठं च सफलं चान्तरापणे}
{वैश्याद्बलाद्गृहीत्वा ते विहृत्य च महारथाः'}


\twolineshloka
{विरागवसनाः सर्वे स्रग्विणो मृष्टकुण्डलाः}
{निवेशनमथाजग्मुर्जरासन्धस्य धीमतः}


\twolineshloka
{गोवासमिव वीक्षन्तः सिंहा हैमवता यथा}
{शालस्तम्भनिभास्तेषां चन्दनागुरुरूषिताः}


\twolineshloka
{अशोभन्त महाराज बाहवो युद्धशालिनाम्}
{तान्दृष्ट्वा द्विरदप्रख्याञ्शालस्कन्दानिवोद्गतान्}


\twolineshloka
{व्यूढोरस्कान्मागधानां विस्मयः समपद्यत}
{`अद्वारेणाभ्यवस्कन्द्य विविशुर्मागधालयम्'}


\twolineshloka
{ते त्वतीत्य जनाकीर्णाः कक्षास्तिस्रो नरर्षभाः}
{अहङ्कारेण राजानमुपतस्थुर्गतव्यथाः}


\threelineshloka
{`भो शब्देनैव राजानमुचुस्ते तु महारथाः'}
{तान्पाद्यमधुपर्कार्हान्गवार्हान्सत्कृतिं गतान्}
{प्रत्युत्थाय जरासन्ध उपतस्थे यथाविधि}


\twolineshloka
{उवाच चैतान्राजाऽसौ स्वागतं वोस्त्विति प्रभुः}
{मौनमासीत्तदा पार्थभीमयोर्जनमेजय}


\twolineshloka
{तेषां मध्ये महाबुद्धिः कृष्णो वचनमब्रवीत्}
{वक्तुं नायाति राजेन्द्र एतयोर्नियमस्थयोः}


\threelineshloka
{अर्वाङ्निशीथात्परतस्त्वाया सार्धं वदिष्यतः}
{यज्ञागारे स्थापयित्वा राजा राजगृहं वदिष्यतः}
{}


\twolineshloka
{ततोऽर्धरात्रे सम्प्राप्ते यातो यत्र स्थिता द्विजाः}
{तस्य ह्येतद्व्रतं राजन्बभूव भुवि विश्रुतम्}


\twolineshloka
{स्नातकान्ब्राह्मणान्प्राप्ताञ्श्रुत्वा स समितिञ्जयः}
{अप्यर्धरात्रे नृपतिः प्रत्युद्गच्छति भारत}


\twolineshloka
{तांस्त्वपूर्वेण वेषेण दृष्ट्वा स नृपसत्तमः}
{उपतस्थे जरान्धो विस्मितश्चाभवत्तदा}


\twolineshloka
{ते तु दृष्ट्वैव राजानं जरासन्धं नरर्षभाः}
{इदमूचुरमित्रघ्नाः सर्वे भरतसत्तम}


\twolineshloka
{स्वस्त्यस्तु कुशलं राजन्निति तत्र व्यवस्थिताः}
{तं नृपं नृपशार्दूल प्रेक्षमाणाः परस्परम्}


\twolineshloka
{तानब्रवीञ्जरासन्धस्तथा पाण्डवयादवान्}
{आस्यतामिति राजेन्द्र ब्राह्मणच्छद्मसंवृतान्}


\twolineshloka
{अथोपविविशुः सर्वे त्रयस्ते पुरुषर्षभाः}
{सम्प्रदीप्तास्त्रयो लक्ष्म्या महाध्वर इवाग्नयः}


\twolineshloka
{तानुवाच जरासन्धः सत्यसन्धो नराधिपः}
{विगर्हमाणः कौरव्य वेषग्रहणवैकृतान् ॥न स्नातकव्रता विप्रा वहिर्माल्यानुलेपनाः}


\twolineshloka
{भवन्तीति नृलोकेऽस्मिन्विदितं मम सर्वशः}
{के यूयं पुष्पवन्तश्च भुजैर्ज्याकृतलक्षणैः}


\twolineshloka
{बिभ्रतः क्षात्रमोजश्च ब्राह्मण्यं प्रतिजानथ}
{एवं विरागवसना बहिर्माल्यानुलेपनाः}


% Check verse!
` क्षत्रिया एव लोकेऽस्मिन्विदिता मम सर्वशः' ॥सत्यं वदत के यूयं सत्यं राजसु शोभते
\twolineshloka
{चैत्यकस्य गिरेः शृङ्गं भित्त्वा किमिह सद्मनि}
{अद्वारेण प्रविष्टाः स्थ निर्भया राजकिल्विषात्}


\twolineshloka
{वदध्वं वाचि वीर्यं च ब्राह्मणस्य विशेषतः}
{कर्म चैतद्विलिङ्गस्थं किं वोऽद्य प्रसमीक्षितम्}


\threelineshloka
{एवं च मामुपास्थाय कस्माच्च विधिर्नाहणाम्}
{प्रणीतां नानुगृह्णीत कार्यं किं वाऽस्मदागमे ॥वैशम्पायन उवाच}
{}


\threelineshloka
{एवमुक्ते ततः कृष्णः प्रत्युवाच महामनाः}
{स्निग्धगमभीरया वाचा वाक्यं वाक्यविशारदः ॥कृष्ण उवाच}
{}


\twolineshloka
{स्नातकान्ब्राह्मणान्राजन्विद्ध्यस्मांस्त्वं नराधिप}
{स्नातकव्रतिनो राजन्ब्राह्मणाः क्षत्रिया विशः}


\twolineshloka
{विशेषनियमाश्चैषामविशेषाश्च सन्त्युत}
{विशेषवांश्च सततं क्षत्रियः श्रियमृच्छति}


\twolineshloka
{पुष्पवत्सु ध्रुवा श्रीश्च पुष्पवन्तस्ततो वयम्}
{क्षत्रियो बाहुवीर्यस्तु न तथा वाक्यवीर्यवान् ॥अम्प्रगल्भं वचस्तस्य तस्माद्बार्हद्रथेरितम्}


\twolineshloka
{स्ववीर्यं क्षत्रियाणां तु बाह्वोर्धाता न्यवेशयत्}
{तद्दिदृक्षसि चेद्राजन्द्रष्टास्यद्य न संशयटः}


\twolineshloka
{अद्वारेण रिम्पोर्गेहं द्वारेण सुहृदो गृहान्}
{प्रविशन्ति नरा धीरा द्वाराण्येतानि धर्मतः}


\twolineshloka
{कार्यवन्तो गृहानेत्य शत्रुतो नार्हणां वयम्}
{प्रतिगृह्णीम् तद्विद्वि एतन्नः शाश्वतं व्रतम्}


\chapter{अध्यायः २२}
\twolineshloka
{न स्मरामि कदा वैरं कृतं युष्माभिरित्युत}
{चिन्तयंश्च न पश्यामि भवतां प्रति वैकृतम्}


\twolineshloka
{वैकृते वा सति कथं मन्यध्वं मामनागसम्}
{अरिं वै ब्रूत हे विप्राः सतां समय एष हि}


\twolineshloka
{अर्थधर्मोपघाताद्धि मनः समुपतप्यते}
{योऽनागसि प्रसजति क्षत्रियो हि न संशयटः}


\twolineshloka
{अतोऽन्यथा चरँल्लोके धर्मज्ञः सन्महारथः}
{वृजिनां गतिमाप्नोति श्रेयसोऽप्युपहन्ति च}


\twolineshloka
{त्रैलोक्ये क्षत्रधर्मो हि श्रेयान्वै साधुचारिणाम्}
{नान्यं धर्मं प्रशंसन्ति ये च धर्मविदो जनाः}


\threelineshloka
{तस्य मेऽद्य स्थितस्येह स्वधर्मे नियतात्मनः}
{अनागसं प्रजानां च प्रमादादिव जल्पथ ॥कृष्ण उवाच}
{}


\twolineshloka
{कुलकार्यं महाबाहो कश्चिदेकः कुलोद्वहः}
{वहते यस्तन्नियोगाद्वयमभ्युद्यतास्त्वयि}


\twolineshloka
{त्वया चोपहृता राजन्क्षत्रिया लोकवासिनः}
{तदागः क्रूरमुत्पाद्य मन्यसे किमनागसम्}


\twolineshloka
{राजा राज्ञः कथं साधून्हिंस्यान्नृपतिसत्तम}
{तद्राज्ञः सन्निगृह्य त्वं रुद्रायोपजिहीर्षसि}


\twolineshloka
{अस्मांस्तदेनोपगच्छेत्कृतं बार्हद्रथ त्वया}
{वयं हिं शक्ता धर्मस्य रक्षणे धर्मचारिणः}


\threelineshloka
{`तस्मादद्योपगच्छामस्तव बार्हद्रथान्तिकम्'}
{मनुष्याणां समालम्भो न च दृष्टः कदाचन}
{स कथं मानुषैर्देवं यष्टुमिच्छसि शङ्करम्}


\twolineshloka
{सवर्णो हि सवर्णानां पशुसञ्ज्ञां करिष्यसि}
{कोऽन्यं एवं यथा हि त्वं जरासन्ध वृथामतिः}


\twolineshloka
{यस्यां यस्यामवस्थायां यद्यत्कर्म करोति यः}
{तस्यां तस्यामवस्थायां तत्फलं समवाप्नुयात्}


\twolineshloka
{ते त्वां ज्ञातिक्षयकरं यममार्तानुसारिणः}
{ज्ञातिवृद्धिनिमित्तार्थं विनिहन्तुमिहागताः}


\twolineshloka
{नास्ति लोके पुमानन्यः क्षत्रियोष्विति चैव तत्}
{मन्यसे स च ते राजन्सुमहान्बुद्धिविप्लवः}


\twolineshloka
{को हि जानन्नभिजनमात्मवान्क्षत्रियो नृप}
{नाविशेत्स्वर्गमतुलं रणानन्तरमव्ययम्}


\twolineshloka
{स्वर्गं ह्येव समास्थाय रणयज्ञेषु दीक्षिताः}
{जयन्ति क्षत्रिया लोकांस्तद्विद्धि मनुजर्षभ}


\twolineshloka
{स्वर्गयोनिर्महद्ब्रह्म स्वर्गयोनिर्महद्यशः}
{स्वर्गं योनिस्तपो युद्धे मृत्युः सोऽव्यभिचारवान्}


\twolineshloka
{एष ह्यैन्द्रो वैजयन्तो गुणैर्नित्यं समाहितः}
{येनासुरान्पराजित्य जगत्पाति शतक्रतुः}


\twolineshloka
{स्वर्गमार्गाय कस्य स्याद्विग्रहो वै यथा तव}
{मागधैर्विपुलैः सैन्यैर्बाहुल्यबलदर्पितः}


\twolineshloka
{माऽवमंस्थाः परान्राजन्नास्ति वीर्यं नरे नरे}
{समं चेजस्त्वया चैव विशिष्टं वा नरेश्वर}


\twolineshloka
{यावदेतदसम्बुद्धं तावदेव भवेत्तव}
{विषह्यमेतदस्माकमतो राजन्ब्रवीमि ते}


\twolineshloka
{जहि त्वं सदृशेष्वेव मानं दर्पं च मागध}
{मा गमः ससुतामात्यः सबलश्च यमक्षयम्}


\twolineshloka
{दम्भोद्भवः कार्तवीर्य उत्तरश्च बृहद्रथः}
{श्रेयसो ह्यवमत्येह विनेशुः सबला नृपाः}


\threelineshloka
{युयुक्षमाणास्त्वत्तो हि न वयं ब्राह्मणा ध्रुवम्}
{शौरिरस्मि हृषीकेशो नृवीरौ पाण्डवाविमौ}
{अनयोर्मातुलेयं च कृष्णं मां विद्धि ते रिपुम्}


\twolineshloka
{त्वामाह्वयामहे राजन्स्थिरो युध्वस्व मागध}
{मुच्छ वा नृपतीन्सर्वान् गच्छ वा त्वं यमक्षयम् ॥` वैशम्पायन उवाच}


% Check verse!
एतच्छ्रुत्वा जरासन्धः क्रुद्धो वचनमब्रवीत्
\twolineshloka
{नाहं कंसः प्रलम्बो वा न बाणो न च मुष्टिकः}
{नरको नेन्द्रतपनो न केशी न च पूतना}


\twolineshloka
{न कालयवनो वाऽपि ये त्वया निहता युधि}
{त्वं तु गोपकुलोत्पन्नो जातिं वै पौर्विकीं स्मर}


\twolineshloka
{योऽस्मद्भयादतिक्रम्य सागरानूपमाश्रितः}
{जन्मभूमिं परित्यज्य मधुरां प्राकृतो यथा}


\twolineshloka
{सोऽधुना कत्थसे शौरे शरदीव यथा घनः}
{अद्यानृण्यं करिष्यामि भोजराजस्य धीमतः}


\twolineshloka
{जामातुरौग्रसेनस्य त्वां निहत्याद्य माधव}
{चिरकाङ्क्षितो मे सङ्ग्रामस्त्वां हन्तुं समुहृद्गुणम्}


\twolineshloka
{दिष्ट्या मे सफलो यत्नः कृतो देवैः सवासवैः}
{क्लीबाविमौ च गोविन्द भीमसेनार्जुनावुभौ}


\twolineshloka
{हिंस्यासि युधि विक्रम्य सिंहः क्षुद्रमृगाविव}
{वैशम्पायन उवाच ॥तस्य रोषाभिभूतस्य जरासन्धस्य गर्जतः}


\twolineshloka
{सर्वभूतानि वित्रेमुषे तत्रासन्समागताः}
{श्रीभगवानुवाच ॥किं गर्जसि जरासन्ध कर्मणा तत्समाचर}


\fourlineindentedshloka
{मम निर्देशकर्तृभ्यां पाण्डवाभ्यां नृपाधम}
{समात्यं ससुतं चाद्य घातयिष्याम्यहं रणे}
{न कथञ्चन जीवन्वै प्रवेक्ष्यसि पुरोत्तमम्' ॥जरासन्ध उवाच}
{}


\twolineshloka
{नाजितान्वै नरपतीनहमादद्मि काश्चन}
{अजितः पर्यवस्थाता कोऽत्र यो न मया जितः}


\twolineshloka
{क्षत्रियस्यैतदेवाहुर्धर्म्यं कृष्णोपजीवनम्}
{विक्रम्य वशमानीय कामतो यत्समाचरेत्}


\twolineshloka
{देवातार्थमुपाहृत्य राज्ञः कृष्ण कथं भयात्}
{अहमद्य विमुच्येयं क्षात्रं व्रतमनुस्मरन्}


\threelineshloka
{सैन्यं सैन्येन व्यूढेन एक एकेन वा पुनः}
{द्वाभ्यां त्रिभिर्वा योत्स्येऽहं युगत्पृथगेव वा ॥वैशम्पाय उवाच}
{}


\twolineshloka
{एवमुक्त्वा जरासन्धः सहदेवाभिषेचनम्}
{अज्ञापयत्तदा राजा ययुत्सुर्भीमकर्मभिः}


\twolineshloka
{स तु सेनापतिं राजा सस्मार भरतर्षभ}
{कौशिकं चित्रसेनं च तस्मिन्युद्ध उपस्थिते}


\twolineshloka
{ययोस्ते नामनी राजन्हंसेति डिबिकेति च}
{पूर्वं सङ्कथिते पुम्भिर्नृलोके लोकसत्कृते}


\twolineshloka
{तं तु राजन्विभुः शौरी राजानं बलिनां वरम्}
{स्मृत्वा पुरुषशार्दूलः शार्दूलसमविक्रमम्}


\twolineshloka
{सत्यसन्धो जरासन्धं भुवि भीमपराक्रमम्}
{भागमन्यस्य निर्दिष्टमवध्यं मधुर्भिर्मृधेः}


\twolineshloka
{नात्मनाऽऽत्मवतां मुख्य इयेष मधुसूदनः}
{ब्राह्मीमाज्ञां पुरस्कृत्य हन्तुं हलधरानुजः}


\chapter{अध्यायः २३}
\twolineshloka
{किमर्थं वैरिणावास्तामुभौ तौ कृष्णमागधौ}
{कथं च निर्जितः सङ्ख्ये जरासन्धेन माधवः}


\twolineshloka
{कश्च कंसो मागधस्य यस्य हेतोः स वैरवान्}
{एतदाचक्ष्व मे सर्वं वैशम्पायन तत्वतः ॥वैशम्पायन उवाच}


\twolineshloka
{यादवानामन्ववाये वसुदेवो महामतिः}
{उदपद्यत वार्ष्णेयो ह्युग्रसेनस्य मन्त्रभृत्}


\twolineshloka
{उग्रसेनस्य कंसस्तु बभूव बलवान्सुतः}
{ज्येष्ठो बहूनां कौरव्य सर्वशस्त्रविशारदः}


\twolineshloka
{जरासन्धस्य दुहिता तस्य भार्याऽतिविश्रुता}
{राज्यशुक्लेन दत्ता सा जरासन्धेन धीमता}


\twolineshloka
{तदर्थमुग्रसेनस्य मधुरायां सुतस्तदा}
{अभिषिक्तस्तदाऽमात्यैः स वै तीव्रपराक्रमः}


\twolineshloka
{ऐश्वर्यबलमत्तस्तु स तदा बलमोहितः}
{निगृह्य पितरं भुङ्क्ते तद्राज्यं मन्त्रिभिः सह}


\twolineshloka
{वसुदेवस्य तत्कृत्यं न शृणोति स मन्दधीः}
{त तेन सह तद्राज्यं धर्मतः पर्यपालयत्}


\twolineshloka
{प्रीतिमान्स तु दैत्येन्द्रो वसुदेवस्य देवकीम्}
{उवाह भार्या स तदा दुहिता देवकस्य या}


\twolineshloka
{तस्यामुद्वाह्यमानायां रथेन जनमेजय}
{उपारुरोह वार्ष्णेयं कंसो भूमिपतिस्तदा}


\twolineshloka
{ततोऽन्तरिक्षे वागासीद्देवदूतस्य कस्यचित्}
{वसुदेवश्च शुश्राव तां वाचं पार्थिवश्च सः}


\twolineshloka
{यामेतां वहमानोऽद्य कंसोद्वहसि देवकीम्}
{अस्या यश्चाष्टमो गर्भः स ते मृत्युर्भविष्यति}


\twolineshloka
{सोऽवतीर्य ततो राजा खड्गमुद्धृत्य निर्मलम्}
{इयेष तस्या मूर्धानं छेत्तुं परमदुर्मतिः}


\twolineshloka
{सान्त्वयन्स तदा कंसं हसन्कोधवशानुगम्}
{राजन्ननुनयामास वसुदेवो महामतिः}


\twolineshloka
{अहिंस्यां प्रमदामाहुः सर्वधर्मेषु पार्थिव}
{अकस्मादबलां नारीं हन्तासीमामनागसीम्}


\twolineshloka
{यच्च तेऽत्र भयं राजञ्शक्यते बाधितुं त्वया}
{इयं शक्या पालयितुं समयं चैव रक्षितुम्}


\twolineshloka
{अस्यास्त्वमष्टमं गर्भं जातमात्रं महीपते}
{विध्वंसय तदा प्राप्तमेवं परिहृतं भवेत्}


\twolineshloka
{एवं स राजा कथितो वसुदेवेन भारत}
{तस्य तद्वचनं चके शूरसेनपतिस्तदा}


\twolineshloka
{ततस्तस्यां सम्बभूवुः कुमाराः सूर्यवर्चसः}
{जाताञ्चातांस्तु तान्सर्वाञ्जघान मधुरेश्वरः}


\twolineshloka
{अथ तस्यां समभवद्बलदेवस्तु सत्तमः}
{याम्यता मायया तं तु यमो राजा विशाम्पते}


\twolineshloka
{देवक्या गर्भमतुलं रोहिण्या जठरेऽक्षिपत्}
{आकृष्य कर्षणात्सम्यक्सङ्कर्षणं इति स्मृतः}


\twolineshloka
{बलश्रेष्ठतया तस्य बलदेव इति स्मृतः}
{पुनस्तस्यां समभवदष्टमो मधुमूदनः}


\twolineshloka
{तस्य गर्भस्य रक्षां तु स चक्रेऽभ्यधिकं नृपः}
{ततः काले रक्षणार्थं वसुदेवस्य तत्वतः}


% Check verse!
उग्रः प्रयुक्तः कंसेन सचिवः क्रूरकर्मकृत्
\twolineshloka
{जातमात्रं वासुदेवमथाकृष्य पिता ततः}
{उपजह्रे परिक्रीतां सुतां गोपस्य कस्यचित्}


\twolineshloka
{अमृष्यमाणस्तं शब्दं देवदूतस्य पार्थिवः}
{वासुदेवं महात्मानमर्पयामास गोकुले}


\twolineshloka
{वासुदेवोपि गोपेषु ववृधेऽब्जमिवाम्भसि}
{अज्ञायमानः कंसेन गूढोऽग्निरिव दारुषु}


\twolineshloka
{विप्रचके तदा सर्वान्बल्लवान्मधुरेश्वरः}
{वर्धमानो महाबाहुस्तेजोबलसमन्वितः}


\twolineshloka
{ततस्ते क्लिश्यमानास्तु पुण्डरीकाक्षमच्युतम्}
{भयेन कामादपरे गणशः पर्यवारयन्}


\twolineshloka
{स तु लब्ध्वा बलं राजन्नुग्रसेनस्य संमतः}
{वसुदेवात्मजः सर्वैर्भ्रातृभिः सहितं पुनः}


\twolineshloka
{निर्जित्य युधि भोजेन्द्रं हत्वा कंसं महाबलः}
{अभ्यषिञ्चत्ततो राज्य उग्रसेनं विशाम्पते}


\twolineshloka
{ततः श्रुत्वा जरासन्धो माधवेन हतं युधि}
{शूरसेनाधिपं चक्रे कंसपुत्रं तदा नृप}


\twolineshloka
{ससैन्यं महदुत्थाप्य वासुदेवं तदा नृप}
{अभ्यषिञ्चत्सुतं तत्र सुताया जनमेजय}


\twolineshloka
{उग्रसेनं च वृष्णींश्च महाबलसमन्वितः}
{स तत्र विप्रकुरुते जरासन्धः प्रतापवान्}


\twolineshloka
{एतद्वैरं कौरवेय जरासन्धस्य माधवे}
{आशासितार्थे राजेन्द्र संरुरोध विनिर्जितान्}


\twolineshloka
{पार्थिवैस्तैर्नृपतिभिर्यक्ष्यमाणः समृद्धिमान्}
{देवश्रेष्ठं महादेवं कृत्तिवासं त्रियम्बकम्}


\twolineshloka
{एतत्सर्वं यथावृत्तं कथितं भरतर्षभ}
{यथा तु स हतो राजा भीमसेनेन तच्छृणु}


\chapter{अध्यायः २४}
\twolineshloka
{ततस्तं निश्चितात्मानं युद्धाय यदुनन्दनः}
{उवाच वाग्मी राजानं जरासन्धमधोक्षजः}


\twolineshloka
{त्रयाणां केन ते राजन्योद्धुमुत्सहते मनः}
{अस्मदन्यतमेनेह सज्जीभवतु को युधि}


\twolineshloka
{एवमुक्तः स नृपतिर्युद्धं वव्रे महाद्युतिः}
{जरासन्धस्ततो राजा भीमसेनेन मागधः}


\twolineshloka
{`धारयन्तं गदां दिव्यां बलं श्रुत्वा च निर्वृतः}
{अर्जुन वासुदेवं च वजर्यित्वा स मागधः}


\twolineshloka
{मत्वा देवं गोप इति बालोऽर्जुन इति स्म ह'}
{आदाय रोजनां माल्यं मङ्गल्यान्यपराणि च}


\twolineshloka
{धारयन्नगदान्मुख्यान्निर्वृतीर्वेदनानि च}
{उपतस्थे जरासन्धं युयुत्सुं वै पुरोहितः}


\twolineshloka
{कृतस्वस्त्ययनो राजा ब्राह्मणेन यशस्विना}
{समनह्यज्जरासन्धः क्षात्रं धर्ममनुस्मरन्}


\twolineshloka
{अवमुच्य किरीटं स केशान्समनुमृज्य च}
{उदतिष्ठज्जरासन्धो वेलातिग इवार्णवः}


\twolineshloka
{उवाच मतिमान्राजा भीमं भीमपराक्रमः}
{भीम योत्स्ये त्वया सार्धं श्रेयसा निर्जितं वरम्}


\twolineshloka
{एवमुक्त्वा जरासन्धो भीमसेनमरिन्दमः}
{प्रत्युद्ययौ महातेजाः शक्रं बल इवासुरः}


\twolineshloka
{ततः संमन्त्र्य कृष्णेन कृतस्वस्त्ययनो बली}
{भीमसेनो जरासन्धमाससाद युयुत्सया}


\twolineshloka
{ततस्तौ नरशार्दूलौ बाहुशस्त्रौ समीयतुः}
{वीरौ परमसंहृष्टावन्योन्यजयकाङ्क्षिणौ}


\twolineshloka
{करग्रहणपूर्वं तु कृत्वा पादाभिवन्दनम्}
{कक्षैः कक्षां विधुन्वानावास्फोटं तत्र चक्रतुः}


\threelineshloka
{स्कन्धे दोर्भ्यां समाहत्य निहत्य च मुहुर्मुहुः}
{अङ्गमङ्गैः समाश्लिष्य पुनरास्फालनं च चक्रतुः}
{}


% Check verse!
चित्रहस्तादिकं कृत्वा सस्फुलिङ्गेन चाशनिम् ॥गलगण्डाभिघातेन सस्फुलिङ्गेन चाशनिम्
\twolineshloka
{बाहुपाशादिकं कृत्वा पादाहतशिरावुभौ}
{उरोहस्तं ततश्चक्रे पूर्णकुम्भौ प्रयुज्य तौ}


\twolineshloka
{करसम्पीडनं कृत्वा गर्जन्तौ वारणाविव}
{नर्दन्तौ मेघसङ्काशौ बाहुप्रहरणावुभौ}


\twolineshloka
{तलेनाहन्यमानौ तु अन्योन्यं कृतवीक्षणौ}
{सिंहाविव सुसंङ्क्रुद्धावाकृष्याकृष्य युध्यताम्}


\twolineshloka
{अङ्गेनाङ्गं समापीड्य बाहुभ्यामुभयोरपि}
{आवृत्य बाहुभिश्चापि उदरं च प्रचक्रतुः}


\twolineshloka
{उभौ कट्यां सुपार्श्वे तु तक्षवन्तौ च शिक्षितौ}
{अधो हस्तं स्वकण्ठे तूदरस्योरसि चाक्षिपत्}


\twolineshloka
{सर्वातिक्रान्तमर्यादं पृष्ठभङ्गं च चक्रतुः}
{सम्पूर्णमूर्च्छां बाहुभ्यां पूर्णकुम्भं प्रचक्रतुः}


\twolineshloka
{तृणपीडं यथाकामं पूर्णयोगं समुष्टिकम्}
{एवमादीनि युद्धानि प्रकुर्वन्तौ परस्परम्}


\twolineshloka
{तयोर्युद्धं ततो द्रष्टुं समेताः पुरवासिनाः}
{ब्राह्मणा वणिजश्चैव क्षत्रियाश्च सहस्रशः}


\twolineshloka
{शूद्राश्च नरशार्दूल स्त्रियो वृद्धाश्च सर्वशः}
{निरन्तरमभूत्तत्र जनौघैरभिसंवृतम्}


\twolineshloka
{तयोरथ भुजाघातान्निग्रहप्रग्रहात्तथा}
{आसीत्सुभीमसम्पातो वज्रपर्वतयोरिव}


\twolineshloka
{उभौ परमसंहृष्टौ बलेन बलिनां वरौ}
{अन्योन्यस्यान्तरं प्रेप्सू परस्परजयैषिणौ}


\twolineshloka
{` शिरोभिरिव तौ मेषौ वृक्षैरिव निशाचरौ}
{पदैरिव शुभावश्वौ तुण्डाभ्यां तित्तिरी इव'}


\twolineshloka
{तद्भीममुत्सार्य जनं युद्धमासीदुपप्लवे}
{बलिनोः संयुगे राजन्वृत्रवासवयोरिव}


\twolineshloka
{प्रकर्षणाकर्षणाभ्यामनुकर्षविकर्षणैः}
{आचकर्षतुरन्योन्यं जानुभिश्चावजघ्नतुः}


\twolineshloka
{ततः शब्देन महता भर्त्सयन्तौ परस्परम्}
{पाषाणसङ्घातनिभैः प्रहारैरभिजघ्नतुः}


\twolineshloka
{`ततो भीमं जरासन्धो जघानोरसि मुष्टिना}
{भीमोषि तं जरासन्धं वक्षस्यभिजघान ह'}


\twolineshloka
{व्यूढोरस्कौ दीर्घभुजौ नियुद्धकुशलावुभौ}
{बाहुभिः समसज्जोतामायसैः परिघैरिव}


\threelineshloka
{कार्तिकस्य तु मासस्य प्रवृत्तं प्रथमेऽहनि}
{तदा तद्युद्धमभवद्दिनानि दश पञ्च च}
{अनाहारं दिवारात्रमविश्रान्तमवर्तत}


\twolineshloka
{तद्वृत्तं तु त्रयोदश्यां समवेतं महात्मनोः}
{चतुर्दश्यां निशायां तु निवृत्तो मागधः क्लमात्}


\twolineshloka
{तं राजानं तथा क्लान्तं दृष्ट्वा राजञ्जनार्दनः}
{उवाच भीमकर्माणं भीमं सम्बोधयन्निव}


\twolineshloka
{क्लान्तः शत्रुर्हि कौन्तेय शक्यः पीडयितुं रणे}
{पीड्यमानो हि कार्त्स्न्येन जह्याज्जीवितमात्मनः}


\threelineshloka
{तस्मात्तेऽद्यैव कौन्तेय पीडनीयो जनाधिपः}
{सममेतेन युध्यस्व बाहुभ्यां भरतर्षभ ॥वैशम्पायन उवाच}
{}


\twolineshloka
{एवमुक्तःस कृष्णेन पाण्डवः परवीरहा}
{जरासन्दस्य तद्रन्ध्रं ज्ञात्वा चक्रे मतिं वधे}


\twolineshloka
{ततस्तमजितं जेतुं जरासन्दं वृकोदरः}
{संरम्भाद्बलिनां श्रेष्ठो जग्राह कुरुनन्दनः}


\chapter{अध्यायः २५}
\twolineshloka
{भीमसेनस्ततः कृष्णमुवाच यदुनन्दनम्}
{बुद्धिमास्थाय विपुलां जरासन्धवधोप्सया}


\twolineshloka
{नायं पापो मया कृष्ण युक्तः स्यादनुरोधितुम्}
{प्रायेण यदुशार्दूल बान्धवक्षयकृत्तव}


\twolineshloka
{एवमुक्तस्ततः कृष्णः प्रत्युवाच वृकोदरम्}
{त्वरयन्पुरुषव्याघ्रो जरासन्धवधेप्सया}


\twolineshloka
{यत्ते दैवं परं सत्वं यच्च ते मातरिश्वनः}
{बलं भीमं जरासन्धे दर्शयाशु तदद्य वै}


\twolineshloka
{`तवैष वध्यो दुर्बुद्धिर्जरासन्धो महारथः}
{इत्यन्तरिक्षे त्वश्रौषं यदा वायुरवाप्यते}


\twolineshloka
{गोमन्ते पर्वतश्रेष्ठे येनैष परिमोक्षितः}
{बलदेवबलं प्राप्य कोऽन्यो जीवेत्तु मागधात्}


\threelineshloka
{तदस्य मृत्युर्विहितस्त्वदृते न महाबल}
{वायुं चिन्त्य महाबाहो जहीमं मगधाधिपम् ॥वैशम्पायन उवाच}
{}


\twolineshloka
{एवमुक्तस्ततो भीमो मनसाऽऽचिन्त्य मारुतम्}
{जनार्दनं नमस्कृत्य परिष्वज्य च फल्गुनम्'}


\twolineshloka
{प्रभञ्जनबलाविष्टो जरासन्धमरिन्दमः}
{उत्क्षिप्य भ्रामयामास बलवन्तं महाबलः}


\twolineshloka
{भ्रामयित्वा शतगुणं जानुभ्यां भरतर्षभ}
{बभञ्ज पृष्टं सङ्क्षिप्य निष्पिष्य विननाद च}


\twolineshloka
{करे गृहीत्वा चरणं द्विधा चक्रे महाबलः}
{तस्य निष्पिष्यमाणस्य पाण्डवस्य च गर्जतः}


\twolineshloka
{अभवत्तुमुलो नादः सर्वप्राणिभयङ्करः}
{वित्रेसुर्मागधाः सर्वे स्त्रीणां गर्भाश्च सुस्रुवुः}


\twolineshloka
{भीमसेनस्य नादेन जरासन्धस्य चैव ह}
{किं नु स्याद्धिमवान्भिन्नः किंनुस्विद्दीर्यते मही}


\twolineshloka
{इति वै मागधा जज्ञुर्भीमसेनस्य निखनात्}
{`ततस्तु भगवान्कृष्णो जरासन्धजिघांसया}


\twolineshloka
{भीमसेनं समालोक्य नलं जग्राह पाणिना}
{द्विधा चिच्छेद वै तत्तु जरासन्धवधं प्रति}


\twolineshloka
{ततस्त्वाज्ञाय तस्यैव पादमुत्क्षिप्य मारुतिः}
{द्विधा बभञ्ज तद्गात्रं प्राक्षिपद्विननाद च}


\twolineshloka
{पुनः सन्धाय तु तदा जरान्धः प्रतापवान्}
{भीमेन च समागम्य बाहुयुद्धं चकार ह}


\twolineshloka
{तयोः समभवद्युद्धं तुमुलं रोमहर्षणम्}
{सर्वलोकक्षयकरं सर्वभूतभयावहम्}


\twolineshloka
{पुनः कृष्णस्तमिरिणं द्विधा विच्छिद्य माधवः}
{व्यत्यस्य प्राक्षिपत्तत्तु जरासन्धवधेप्सया}


\twolineshloka
{भीमसेनस्तदा ज्ञात्वा निर्बिभेद च मागधम्}
{द्विधा व्यत्यस्य पादेन प्राक्षिपच्च ननाद ह}


\twolineshloka
{शुष्कमांसास्थिमेदस्त्वग्भिन्नमस्तिष्कपिण्डकटः}
{शवभूतस्तदा राजन्पिण्डीकृत इवाबबौ'}


\twolineshloka
{ततो राज्ञः कुलद्वारि प्रसुप्तमिव तं नृपम्}
{रात्रौ गतासुमुत्सृज्य निश्चक्रमुररिन्दमाः}


\twolineshloka
{जरासन्धरथं कृष्णो योजयित्वा पताकिनम्}
{आरोप्य भ्रातरौ चैव मोक्षयामास बान्धवान्}


\twolineshloka
{ते वै रत्नुभुजं कृष्णं रत्नार्हं पृथिवीश्वराः}
{राजानं चक्ररासाद्य मोक्षिता महतो भयात्}


\twolineshloka
{अक्षतः शस्त्रसम्पन्नो जितारिः सह राजभिः}
{रथमास्थाय तं दिव्यं निर्जगाम गिरिव्रजात्}


\twolineshloka
{यः स सोदर्यवान्नाम द्वियोधी कृष्णसारथिः}
{अभ्यासघाती सन्दृश्यो दुर्जयः सर्वराजभिः}


\twolineshloka
{भीमार्जुनाभ्यां योधाभ्यामास्थितः कृष्णसारथिः}
{शुशुभे रथवर्योऽसौ दुर्जयः सर्वधन्विभिः}


\twolineshloka
{शक्रविष्णू हि सङ्ग्रामे चेरतुस्तारकामये}
{रथेन तेन वै कृष्ण उपारुह्य ययौ तदा}


\twolineshloka
{`एवमेतौ महाबाहू तदा दुष्करकारिणौ}
{कृष्णप्रणीतौ लोकेऽस्मिन्नथे को द्रष्टुमर्हति}


\twolineshloka
{इत्यवोचन्व्रजन्तं तं जरासन्धपुरालयाः}
{वासुदेवं नरश्रेष्ठं युक्तं वातजवैर्हयैः'}


\twolineshloka
{तप्तचामीकराभेण किङ्किणीजालमालिना}
{मेघनिर्घोषनादेन जैत्रेणामित्रघातिना}


\twolineshloka
{येन शक्रो दानवानां जघान नवतीर्नव}
{तं प्राप्य समहृष्यन्त रथं ते पुरुषर्थभाः}


\twolineshloka
{ततः कृष्णं महाबाहुं भ्रातृभ्यां सहितं तदा}
{रथस्थं मागधा दृष्ट्वा समपद्यन्त विस्मिताः}


\twolineshloka
{हयैर्दिव्यैः समायुक्तो रथो वायुसमो जवे}
{अधिष्ठितः स शुशुभे कृष्णेनातीव भारत}


\twolineshloka
{असङ्गो देवविहितस्तस्मिन्रथवरे ध्वजः}
{योजनाद्ददृशे श्रीमानिन्द्रायुधसमप्रभः}


\twolineshloka
{चिन्तयामास कृष्णोऽथ गरुत्मन्तं स चाभ्ययात्}
{क्षणे तस्मिन्स तेनासीच्चैत्यवृक्ष इवोत्थितः}


\twolineshloka
{व्यादितास्यैर्महानादैः सह भूतैर्ध्वजालयैः}
{तस्मिन्रथवरे तस्थौ गुरुत्मान्पन्नगाशनः}


\twolineshloka
{दुर्निरीक्ष्यो हि भूतानां तेजसाऽऽभ्याधिकं बभौ}
{आदित्य इव मध्याह्ने सहस्रकिरणावृतः}


\twolineshloka
{न सज्जति वृक्षेषु शस्त्रैश्चापि न रिष्यते}
{दिव्यो ध्वजवरो राजन्दृश्यते चेह मानुषैः}


\twolineshloka
{तमास्थाय रथं दिव्यं पर्जन्यसमनिः स्वनम्}
{निर्ययौ पुरुषव्याघ्रः पाण्डवाभ्यां सहाच्युतः}


\twolineshloka
{यं लेभे वासवाद्राजा वसुस्तस्माद्बृहद्रथः}
{बृहद्रथात्क्रमेणैव प्राप्तो बार्हद्रथो रथम्}


\twolineshloka
{स निर्याय महाबाहुः पुण्डरीकेक्षणस्ततः}
{गिरिव्रजाद्बहिस्तस्थौ समदेशे महायशाः}


\twolineshloka
{तत्रैनं नागराः सर्वे सत्कारेणाभ्ययुस्तदा}
{ब्राह्मणप्रमुखा राजन्विधिदृष्टेन कर्मणा}


\twolineshloka
{बन्धनाद्विप्रमुक्ताश्च राजानो मधुसूदनम्}
{पूजयामासुरूचुश्च स्तुतिपूर्वमिदं वचः}


\twolineshloka
{नैतच्चित्रं महाबाहो त्वयि देवकिनन्दने}
{भीमार्जुनबलोपेते धर्मस्य प्रतिपालनम्}


\twolineshloka
{जरासन्धह्रदे घोरे दुःखपङ्के निमज्जताम्}
{राज्ञां समभ्युद्धरमं यदिदं कृतमद्य वै}


\twolineshloka
{विष्णो समवसन्नानां गिरिदुर्गे सुदारुणे}
{दिष्ट्या मोक्षाद्यशो दीप्तमाप्तं ते यदुनन्दन}


\twolineshloka
{किं कर्मः पुरुषव्याघ्र शाधि नः प्रणतिस्थितान्}
{कृतमित्येव तद्विद्वि नृपैर्ययद्यपि दुष्करम् ॥वैशम्पायन उवाच}


\twolineshloka
{तानुवाच हृम्पीकेशः समाश्चास्य महामनाः}
{यिधिष्ठिरो राजसूयं क्रतुमार्हतुमिच्छति}


\twolineshloka
{तस्य धर्मप्रवृत्तस्य पार्थिवत्वं चिकीर्षतः}
{सर्वैर्भवद्भिर्विज्ञाय साहाय्यं क्रियतामिति}


\twolineshloka
{ततः सुप्रीतमनसस्ते नृपा नृपसत्तम}
{तथेत्येवाब्रुवन्सर्वे प्रतिगृह्यास्य तां गिरम्}


\twolineshloka
{रत्नभाजं च दाशार्हं चक्रुस्ते पृथिवीश्वराः}
{कृच्छ्राञ्जग्राह गोविन्दस्तेषां तदनुकम्पया}


\twolineshloka
{जरासन्धात्मजश्चैव सहदेवो महामनाः}
{निर्ययौ सजनामात्यः पुरस्कृत्य पुरोहितम्}


\twolineshloka
{स नीचैः प्रणतो भूत्वा बहुरत्नपुरोगमः}
{सहदेवो नृणां देवं वासुदेवमुपस्थितः}


\threelineshloka
{भयार्ताय ततस्तस्मै कृष्णो दत्त्वाऽभयं तदा}
{आददेऽस्य महार्हाणि रत्नानि पुरुषोत्तमः ॥`सददेव उवाच}
{}


\twolineshloka
{यत्कृतं पुरुषव्याघ्र मम पित्रा जनार्दन}
{तत्ते हृदि महाबाहो न कार्यं पुरुषोत्तम}


\twolineshloka
{त्वां प्रपन्नोऽस्मि गोविन्द प्रासदं कुरु मे प्रभो}
{पितुरिच्छामि संस्कारं कर्तुं देवकिनन्दन}


\threelineshloka
{त्वत्तोऽभ्यनुज्ञां सम्प्राप्य भीमसेनात्तथार्जुनात्}
{निर्भयो विचरिष्यामि यथाकामं यथासुखम् ॥वैशम्पायन उवाच}
{}


\twolineshloka
{एवं विज्ञाप्यमानस्य सहदेवस्य मारिष}
{प्रहृष्टो देवकीपुत्रः पाण्डवै च महारथौ}


\twolineshloka
{क्रियतां संस्क्रिया राजन्पितुस्त इति चाब्रुवन्}
{तच्छ्रुत्वा वासुदेवस्य पार्थयोश्च स मागधः}


\twolineshloka
{प्रविश्य नगरं तूर्णं सह मन्त्रिभिरप्युत}
{चितां चन्दनकाष्ठैश्च कालेयसरलैस्तथा}


\twolineshloka
{कालागरुसुनगन्धैश्च तैलैश्च विविधैरपि}
{घृतधाराशतैश्चैव सुमनोभिश्च भारत}


\twolineshloka
{समन्तादेव कीर्यन्तोऽदहन्त मगधाधिपम्}
{उदकं तस्य चक्रेऽथ सहदेवः सहानुजः}


\twolineshloka
{कृत्वा पितुः स्वर्गगतिं निर्ययौ यत्र केशवः}
{पाण्डवौ च महाभागौ भीमसेनार्जुनावुभौ}


\threelineshloka
{स प्रह्वः प्राञ्जलिर्भूत्वा व्यज्ञापयत माधवम्}
{सहदेव उवाच}
{इमे रत्नानि भूरिणी गोजाविमहिषादयः}


\threelineshloka
{हस्तिनोऽश्वाश्च गोविन्द वासांसि विविधानि च}
{दीयतां धर्मराजाय यथा वा मन्यते भवान् ॥वैशम्पायन उवाच}
{}


\twolineshloka
{भयार्ताय ततस्तस्मै कृत्वा कृष्णोऽभयं तदा}
{अभ्यषिञ्चत राजानं सहदेवं जनार्दनः}


\twolineshloka
{मागधानां महीपालं जरासन्धात्मजं तदा}
{आददे च महार्हाणि रत्नानि पुरुषोत्तमः}


\twolineshloka
{गत्वैकत्वं स कृष्णेन पार्थाभ्यां चापि सत्कृतः}
{विवेश मतिमान्त्राजा पुनर्बार्हद्रथं पुरम्}


\twolineshloka
{पार्थाभायं सहितः कृष्णः सर्वैश्च वसुधाधिपैः}
{यथावयः समागम्य विससर्ज नराधिपान्}


\threelineshloka
{विसृज्य सर्वान्नृपतीन्राजसूये महात्मभिः}
{आगन्तव्यं भवद्भिश्च धर्मराजप्रियेप्सुभिः}
{}


\twolineshloka
{एवमुक्ता माधवेन सर्वे ते वसुधाधिपाः}
{एवमस्त्विति चाप्युक्त्वा समेताः परया मुदा}


\twolineshloka
{भीमार्जुनहृषीकेशैः प्रहृष्टाः प्रययुस्तदा}
{रत्नान्यादाय भूरीणी ज्वलन्तो रिपुसूदनाः'}


\twolineshloka
{कृष्णस्तु सह पार्थाभ्यां श्रिया परमया युतः}
{रत्नान्यादाय भूरिणी प्रययौ पुरुषर्षभः}


\twolineshloka
{इन्द्रप्रस्थमुपागम्य पाण्डवाभ्यां सहाच्युतः}
{समेत्य धर्मराजानं प्रीयमाणोऽभ्यभाषत}


\twolineshloka
{दिष्ट्या भीमेन बलवाञ्जरासन्धो निपातितः}
{राजानो मोक्षिताश्चेमे बन्धनान्नृपसत्तम}


\twolineshloka
{दिष्ट्या कुशलिनौ चेमौ भीमसेनधनञ्जयौ}
{पुनः स्वनगरं प्राप्तावक्षताविति भारत}


\twolineshloka
{ततो युधिष्ठिरः कृष्णं पूजयित्वा यथार्हतः}
{भीमसेनार्जुनौ चैव प्रहृष्टः परिषस्वजे}


\twolineshloka
{ततः क्षीणे जरासन्धे भ्रातृभ्यां विहितं जयम्}
{अजातशत्रुरासाद्य मुमुदे भ्रातृभिः सह}


\twolineshloka
{`हृष्टश्च धर्मराड्वाक्यं जनार्दनमभाषत}
{त्वां प्राप्य पुरुषव्याघ्र भीमसेनेन पातितः}


\twolineshloka
{मागधोऽसौ बलोन्मत्तो जरासन्धः प्रतापवान्}
{राजसूयं क्रतुश्रेष्ठं प्राप्स्यामि विगतज्वरः}


\twolineshloka
{त्वद्बुद्धिबलमाश्रित्य यागार्होऽस्मि जनार्दन}
{पीतं पृथिव्याः क्रुद्धेन यशस्ते पुरुषोत्तम}


\twolineshloka
{जरासन्धवधेनैव प्राप्तास्ते विपुलाः श्रियः ॥वैशम्पायन उवाच}
{एवं सम्भाष्य कौन्तेयः प्रादाद्रथवरं प्रभोः}


% Check verse!
प्रतिगृह्य तु गोविन्दो जरासन्धस्य तं रथम्
\twolineshloka
{प्रहृष्टस्तस्य मुमुदे फल्गुनेन जनार्दनः}
{प्रीतिमानभवद्राजन्धर्मराजपुरस्कृतः'}


\twolineshloka
{यथावयः समागम्य भ्रातृभिः सह पाण्डवः}
{सत्कृत्य पूजयित्वा च विससर्ज नराधिपान्}


\twolineshloka
{युधिष्ठिराभ्यनुज्ञातास्ते नृपा हृष्टमानसाः}
{जग्मुः स्वदेशांस्त्वरिता यानैरुच्चावचैस्ततः}


\twolineshloka
{एवं पुरुषशार्दूलो महाबुद्धिर्जनार्दनः}
{पाण्डवैर्घातयामास जरासन्धमरिं तदा}


\twolineshloka
{घातयित्वा जरासन्धं बुद्धिपूर्वमरिन्दमः}
{धर्मराजमनुज्ञाप्य पृथां कृष्णां च भारत}


\twolineshloka
{सुभद्रां भीमसेनं च फाल्गुनं यमजौ तथा}
{धौम्यमामन्त्रयित्वा च प्रययौ स्वां पुरीं प्रति}


\threelineshloka
{`पाण्डवैरनुधावद्भिर्युधिष्ठिरपुरोगमैः}
{हर्षेण महता युक्तः प्राप्य चानुत्तमं यशः}
{जगाम हृष्टः कृष्णस्तु पुनर्द्वारवतीं पुरीम्'}


\twolineshloka
{तेनैव रथमुख्येन मनसस्तुल्यगामिना}
{धर्मराजविसृष्टेन दिव्येनानादयन्दिशः}


\twolineshloka
{ततो युधिष्ठिरमुखाः पाण्डवा भरतर्षभ}
{प्रदक्षिणमकुर्वन्त कृष्णमक्लिष्टकारिणम्}


\twolineshloka
{ततो गते भगवति कृष्णे देवकिनन्दने}
{जयं लब्ध्वा सुविपुलं राज्ञां दत्त्वाऽभयं तदा}


\twolineshloka
{संवर्धितं यशो भूयः कर्मणा तेन भारत}
{द्रौपद्याः पाण्डवा राजन्परां प्रीतिमवर्धयन्}


\twolineshloka
{तस्मिन्काले तु यद्युक्तं धर्मकामार्थसंहितम्}
{तद्राजा धर्मतश्चक्रे प्रजापालनकीर्तनम्}


\chapter{अध्यायः २६}
\threelineshloka
{`ऋषेस्तद्वचनं स्मृत्वा निशश्वास युधिष्ठिरः}
{धर्म धर्मभृतां श्रेष्ठः कर्तुमिच्छन्परन्तपः}
{}


\twolineshloka
{तस्येङ्गितज्ञो बीभत्सुः सर्वशस्त्रभृतां वरः}
{संविवर्तयिषुः कामं पावकात्पाकशासनिः'}


\threelineshloka
{प्राप्तं प्राप्य धनुः श्रेष्ठमक्षय्यौ च महेषुधी}
{रथं ध्वजं सभां चैव युधिष्ठिरमभाषत ॥अर्जुन उवाच}
{}


\twolineshloka
{धनुरस्त्रं शरा वीर्यं पक्षो भूमिर्यशो बलम्}
{प्राप्तमेतन्मया राजन्दुष्प्रापं यदभीप्सितम्}


\twolineshloka
{तस्य कृत्यमहं मन्ये कोशस्य परिवर्धनम्}
{करमाहारयिष्यामि राज्ञः सर्वान्नृपोत्तम}


\threelineshloka
{विजयाय प्रयास्यामि दिशं धनदपालिताम्}
{तिथावथ मुहूर्ते च नक्षत्रे चाभिपूजिते ॥वैशम्पायन उवाच}
{}


% Check verse!
धनञ्जयवचः श्रुत्वा धर्मराजो युधिष्ठिरःस्निग्धगम्भीरनादिन्यां तं गिरा प्रत्यभाषत
\twolineshloka
{स्वस्ति वाच्यार्हतो विप्रान्प्रयाहि भरतर्षभ}
{दुर्हृदामप्रहर्षाय सुहृदां नन्दनाय च}


\twolineshloka
{विजयस्ते ध्रुवं पार्थ प्रियं काममवाप्स्यसि}
{इत्युक्तः प्रययौ पार्थः सैन्येन महता वृतः}


\twolineshloka
{अग्निदत्तेन दिव्येन रथेनाद्भुतकर्मणा}
{तथैव भीमसेनोऽपि यमौ च पुरुषर्षभौ}


\twolineshloka
{ससैन्याः प्रययुः सर्वे धर्मराजेन पूजिताः}
{दिशं धनपतेरिष्टामजयत्पाकशासनिः}


\twolineshloka
{भीमसेनस्तथा प्राचीं सहदेवस्तु दक्षिणाम्}
{प्रतीचीं नकुलो राजन्दिशं व्यजयतास्त्रवित्}


\twolineshloka
{खाण्डवप्रस्थमध्यस्थो धर्मराजो युधिष्ठिरः}
{आसीत्परमया लक्ष्म्या सुहृद्गणवृतः प्रभुः}


\chapter{अध्यायः २७}
\threelineshloka
{दिशामभिजयं ब्रह्मन्विस्तरेणानुकीर्तय}
{न हि तृप्यामि पूर्वेषां शृण्वानश्चरितं महत् ॥वैशम्पायन उवाच}
{}


\twolineshloka
{धनञ्जयस्य वक्ष्यामि विजयं पूर्वमेव ते}
{यौगपद्येन पार्थैर्हि निर्जितेयं वसुन्धरा}


\twolineshloka
{`अवाप्य राजा राज्यार्धं कुन्तीपुत्रो युधिष्ठिरः}
{महत्त्वे राजशब्दस्य मनश्चक्रे महामनाः}


\twolineshloka
{तदा क्षात्रं विदित्वाऽस्य पृथिवीविजयं प्रति}
{अमर्षात्पार्थिवेन्द्राणां तं समेयाय वारयत्}


\twolineshloka
{तत्समेत्य भुवः क्षात्रं रथनागाश्वपत्तिमत्}
{अभ्ययात्पार्थिवं जिष्णुं मोघं कर्तुं जनाधिप}


\twolineshloka
{तत्पार्थः पार्थिवं क्षात्रं युयुत्सुं परमाहवे}
{प्रत्युद्ययौ महाबाहुस्तरसा पाकशासनिः}


\twolineshloka
{तद्भग्रं पार्थिवं क्षात्रं पार्थेनाक्लिष्टकर्मणा}
{वायुनेव घनानीकं तूलीभूतं ययौ दिशः}


\twolineshloka
{तज्जित्वा पार्थिवं क्षात्रं समरे परवीरहा}
{ययौ तदा वशे कर्तुमुदीचीं पाण्डुनन्दनः'}


\twolineshloka
{पूर्वं कुलिङ्गविषये वशे चक्रे महीपतिम्}
{धनञ्जयो महाबाहुर्नातितीव्रेण कर्मणा}


\twolineshloka
{`तेनैव सहितः प्रायाज्जिष्णुः साल्वपुरं प्रति}
{स साल्वपुरमासाद्य साल्वराजं धनञ्जयटः}


\twolineshloka
{विक्रमेणोग्रधन्वानं वशे चक्रे महामनाः}
{तं पार्थः सहसा जित्वा द्युमत्सेनं महीश्वरम्}


\twolineshloka
{कृत्वा स सैनिकं प्रायात्कटदेशमरिन्दमः}
{तत्र पार्थो रणे जिष्णुः सुनाभं वसुधाधिपम्}


\twolineshloka
{विक्रमेण वशे कृत्वा कृतवाननुसैनिकम्}
{एतेन सहितो राजन्सव्यसाची परन्तपः'}


\twolineshloka
{विजिग्ये शाकलद्वीपे प्रतिविन्ध्यं च पार्थिवम्}
{शाकलद्वीपवासाश्च सप्तद्वीपेषु ये नृपाः}


\twolineshloka
{अर्जुनस्य च सैन्यस्थैर्विग्रहस्तुमुलोऽभवत्}
{तान्सर्वानजयत्पार्तो धर्मराजप्रियेप्सया}


% Check verse!
तैरेव सहितः सर्वैः प्रग्ज्योतिषमुपाद्रवत्
\twolineshloka
{तत्र राजा महानासीद्भगदत्तो विशाम्पते}
{तेनासीत्सुमहद्युद्धं पाण्डवस्य महात्मनः}


\twolineshloka
{स किरातैश्च चीनैश्च वृतः प्राग्ज्योतिषोऽभवत्}
{अन्यैश्च बहुभिर्योधैः सागरानुपवासिभिः}


\twolineshloka
{ततः स दिवसानष्टौ योधयित्वा धनञ्जयम्}
{प्रहसन्नब्रवीद्राजा सङ्ग्रामविगतक्रमम्}


\twolineshloka
{उपपन्नं महाबाहो त्वयि कौरवनन्दन}
{पाकशासनदायादे वीर्यमाहवशोभिनि}


\twolineshloka
{अहं सखा महेन्द्रस्य शक्रादनवरो रणे}
{न शक्ष्यामि च ते तात स्थातुं प्रमुखतो युधि}


\twolineshloka
{त्वमीप्सितं पाण्डवेयं ब्रूहि किं करवाणि ते}
{यद्वक्ष्यसि महाबाहो तत्करिष्यामि पुत्रक ॥अर्जुन उवाच}


\twolineshloka
{कुरूणामृषभो राजा धर्मपुत्रो युधिष्ठिरः}
{धर्मज्ञः सत्यसन्धश्च यज्वा विपुलदक्षिणः}


\fourlineindentedshloka
{तस्य पार्थिवतामीप्से करस्तस्मै प्रदीयताम्}
{भवान्पितृसखश्चैव प्रीयमाणो मयापि च}
{ततो नाज्ञापयामि त्वां प्रीतिपूर्वं प्रदीयताम् ॥भगदत्त उवाच}
{}


\twolineshloka
{कुन्तीमातर्यथा मे त्वं तथा राजा युधिष्ठिरः}
{सर्वमेतत्करिष्यामि किं चान्यत्करवाणि ते}


\chapter{अध्यायः २८}
\twolineshloka
{एवमुक्तः प्रत्युवाच भगदत्तं धनञ्जयः}
{अनेनैव कृतं सर्वं भविष्यत्यनुजानता}


\twolineshloka
{तं विजित्य महाबाहुः कुन्तीपुत्रो धनञ्जयः}
{प्रययावुत्तरां तस्माद्दिशं धनदपालिताम्}


\twolineshloka
{अन्तर्गिरिं च कौन्तेयस्तथैव च बहिर्गिरिम्}
{तथैवोपगिरं चैव विजिग्ये पुरुषर्षभः}


\twolineshloka
{विजित्य पर्वतान्सर्वान्ये च तत्र नराधिपाः}
{तान्वशे स्थापयित्वा स धनान्यादाय सर्वशः}


\twolineshloka
{तैरेव सहितः सर्वैरनुरज्य च तान्नुपान्}
{उलूकवासिनं राजन्बृहन्तमुपजग्मिवान्}


\twolineshloka
{मृदङ्गवरनादेन रथनेमिस्वनेन च}
{हस्तिनां च निनादेन कम्पयन्वसुधामिमाम्}


\twolineshloka
{ततो बृहन्तस्त्वरितो बलेन चतुरङ्गिणा}
{निष्क्रम्य नगरात्तस्माद्योधयामास फाल्गुनम्}


\twolineshloka
{सुमहान्सन्निपातोऽभूद्धनञ्जयबृहन्तयोः}
{न शशाक बृहन्तस्तु सोढुं पाण्डवविक्रमम्}


\twolineshloka
{सोऽविषह्यतमं मत्वा कौन्तेयं पर्वतेश्वरः}
{उपावर्तत दुर्धर्षो रत्नान्यादाय सर्वशः}


\twolineshloka
{स तद्राज्यमवस्थाप्य उलूकसहितो ययौ}
{सेनाबिन्दुमथो राजन्राज्यादाशु समाक्षिपत्}


\twolineshloka
{मोदापुरं वामवेदं सुदामानं सुसङ्कुलम्}
{उलूकानुत्तरांश्चैव तांश्च राज्ञः समानयत्}


\twolineshloka
{तत्रस्थः पुरुषैरेव धर्मराजस्य शासनात्}
{किरीटी जितवान्राजन्देशान्पञ्चगणांस्ततः}


\twolineshloka
{स देवप्रस्थमासाद्य सेनाबिन्दोः पुरं प्रति}
{बलेन चतुरङ्गेण निवेशमकरोत्प्रभुः}


% Check verse!
स तैः परिवृतः सर्वैर्विष्वगश्वं नराधिपम् ॥अभ्यगच्छन्महातेजाः पौरवं पुरुषर्षभ
\twolineshloka
{विजित्य चाहवे शूरान्पार्वतीयान्महारथान्}
{जिगाय सेनया राजन्पुरं पौरवरक्षितम्}


\twolineshloka
{पौरवं युधि निर्जित्य दस्यून्पर्वतवासिनः}
{गणानुत्सवसङ्केतानजयत्सप्त पाण्डवः}


\twolineshloka
{ततः काश्मीरकान्वीरान्क्षत्रियान्क्षत्रियर्षभः}
{व्यजयल्लोहितं चैव मण्डलैर्दशभिः सह}


\twolineshloka
{ततस्त्रिगर्ताः कौन्तेयं दार्वाः कोकनदास्तथा}
{क्षत्रिया बहवो राजन्नुपावर्तन्त सर्वशः}


\twolineshloka
{अभिसारीं ततो रम्यां विजिग्ये कुरुनन्दनः}
{उरगावासिनं रम्यं रोचमानं रणेऽजयत्}


\twolineshloka
{ततः सिंहपुरं रम्यं चित्रायुधसुरक्षितम्}
{प्राधमद्बलमास्थाय पाकशासनिराहवे}


\twolineshloka
{ततः सुह्यांश्च चोलांश्च किरीटी पाण्डवर्षभः}
{सहितः सर्वसैन्येन प्रामथत्कुरुनन्दनः}


\twolineshloka
{ततः परमविक्रान्तो बाह्लीकान्पाकशासनिः}
{महता परिमर्देन वशे चक्रे दुरासदान्}


\twolineshloka
{गृहीत्वा तु बलं सारं फल्गुनः पाण्डुनन्दनः}
{दरदान्सहकाम्भोजैरजयत्पाकशासनिः}


\twolineshloka
{प्रागुत्तरं दिशं ये च वसन्त्याश्रित्य दस्यवः}
{निवसन्ति वने ये च तान्सर्वानजयत्प्रभुः}


\twolineshloka
{लोहान्परमकाम्भोजानृषिकानुत्तरानपि}
{सहितांस्तान्महाराज व्यजयत्पाकशासनिः}


\twolineshloka
{ऋषिकेष्वपि सङ्ग्रामे बभूवातिभयङ्करः}
{तारकामयसङ्काशः परस्त्वृषिकपार्थयोः}


\twolineshloka
{स विजित्य ततो राजन्नृषिकान्रणमूर्धनि}
{शुकोदरसमांस्तत्र हयानष्टौ समानयत्}


\twolineshloka
{मयूरसदृशानन्यानुत्तरानपरानपि}
{जवनानाशुगांश्चैव करार्थं समुपानयत्}


\twolineshloka
{स विनिर्जित्य सङ्ग्रामे हिमवन्तं सनिष्कृटम्}
{श्वेतपर्वतमासाद्य न्यविशत्पुरुषर्षभः}


\chapter{अध्यायः २९}
\twolineshloka
{स श्वेतपर्वतं वीरः समतिक्रम्य वीर्यवान्}
{देशं किम्पुरुषावासं द्रुमपुत्रेण रक्षितम्}


\twolineshloka
{महता सन्निपातेन क्षत्रियान्तकरेण ह}
{अजयत्पाण्डवश्रेष्ठः करे चैनं न्यवेशयत्}


\twolineshloka
{तं जित्वा हाटकं नाम देशं गुह्यकरक्षितम्}
{पाकशासनिरव्यग्रः सहसैन्यः समासदत्}


\twolineshloka
{तांस्तु सान्त्वेन निर्जित्य मानसं सर उत्तमम्}
{ऋषिकुल्यास्तथा सर्वा ददर्श कुरुनन्दनः}


\twolineshloka
{सरो मानसमासाद्य हाटकानभितः प्रभुः}
{गन्धर्वरक्षितं देशमजयत्पाण्डवस्ततः}


\twolineshloka
{तत्र तित्तिरिकल्माषान्मण्डूकाख्यान्हयोत्तमान्}
{लेभे स करसमत्यन्तं गन्धर्वनगरात्तदा}


\twolineshloka
{`हेमकूटमथासाद्य न्यवसत्फल्गुनस्तदा}
{तं हेमकूटं राजेन्द्र समतिक्रम्य पाण्डवः}


\twolineshloka
{हरिवर्षं विवेशाथ सैन्येन महता वृतः}
{तत्र पार्थो ददर्शाथ बहूनिह मनोरमान्}


\twolineshloka
{नगरान्सवनांश्चैव नदीश्च विमलोदकाः}
{पुरुषान्देवकल्पांश्च नारीश्च प्रियदर्शनाः}


\twolineshloka
{तान्सर्वास्तत्र दृष्ट्वाऽथ मुदा युक्तो धनञ्जयटः}
{वशे चक्रे स रत्नानि लेभे च सुबहूनि च}


\twolineshloka
{ततो निषधमासाद्य गिरिस्थानजयत्प्रभुः}
{अथ राजन्नतिक्रम्य निषधं शैलमायतम्}


\twolineshloka
{विवेश मध्यमं वर्षं पार्थो दिव्यमिलावृतम्}
{तत्र दिव्योपमान्दिव्यान्पुरुषान्देवदर्शनान्}


\twolineshloka
{अदृष्टपूर्वान्सुभगान्स ददर्श धनञ्जयः}
{सदनानि च शुभ्राणि नारीश्चाप्सरसंनिभाः}


\twolineshloka
{दृष्ट्वा तानजयद्रम्यान्स तैश्च ददृशे तदा}
{जित्वा च तान्महाभागान्करे च विनिवेश्य च}


\twolineshloka
{रत्नान्यादाय दिव्यानि भूषणान्यासनैः सह}
{उदीचीमथ राजेन्द्र ययौ पार्थो मुदाऽन्वितः}


\twolineshloka
{स ददर्श ततो मेरुं शिखरीणां प्रभुं महत्}
{तं काञ्चनमयं दिव्यं चतुर्वणं दुरासदम्}


\twolineshloka
{उन्नतं शतसाहस्रं योजनानां तु सुस्थितम्}
{ज्वलन्तमचलं मेरुं तेजोराशिमनुत्तमम्}


\twolineshloka
{आक्षिपन्तं प्रभां भानोः स्वशृङ्गैः काञ्चनोज्ज्वलैः}
{काञ्चनाभरणं दिव्यदेवगन्धर्वसेवितम्}


\twolineshloka
{अप्रमेयमनाधृष्यमधर्मबहुलैर्जनैः}
{व्यालैराचरितं धोरैर्दिव्यौषधिविदीपितम्}


\twolineshloka
{स्वर्गमावृत्य तिष्ठन्तमुच्छ्रायेण महागिरिम्}
{अगम्यं मनसाप्यन्यैर्नदीवृक्षसमन्वितम्}


\twolineshloka
{नानाविहगसङ्घैश्च नादितं सुमनोहरैः}
{तं दृष्टा फल्गुनो मेरुं प्रीतिमानभवत्तदा}


\twolineshloka
{मेरोरिलावृतं दिव्यं सर्वतः परिमण़्डितम्}
{मेरोस्तु दक्षिणे पार्श्वे जम्बूर्नाम वनस्पतिः}


\twolineshloka
{नित्यपुष्पफलोपेवः सिद्धचारणसेवितः}
{आस्वर्गमुच्छ्रिता राजंस्तस्य शाखा वनस्पतेः}


\twolineshloka
{यस्य नाम्ना त्विदं द्वीपं जन्बूद्वीपमिति स्मृतम्}
{तां च जम्बूं ददर्शाथ सव्यसाची परन्तपः}


\twolineshloka
{तौ दृष्ट्वाऽप्रतिमौ लोके जम्बूं मेरुं च संस्थितौ}
{प्रतीमानभवद्राजन्सर्वतः स विलोकयन्}


\twolineshloka
{तत्र लेभे ततो जिष्णुः सिद्धैर्दिव्यैश्च चारणैः}
{रत्नानि बहुसाहस्रं दत्तान्याभरणानि च}


\twolineshloka
{वासांसि च महार्हाणि तत्र लब्ध्वाऽर्जुनस्तदा}
{आमन्त्रयित्वा तान्सर्वान्यज्ञमुद्दिश्य वै गुरोः}


\twolineshloka
{अथादाय बहून्रत्नान्गमनाययोपचक्रमे}
{मेरुं प्रदक्षिणीकृत्य प्रवतप्रवरं प्रभुः}


\twolineshloka
{ययौ जम्बूनदीतीरे नदीं श्रेष्ठां विलोकयन्}
{स तां मनोरमां दिव्यां जम्बूस्वादुरसावहाम्}


\twolineshloka
{हैमपक्षिगणैर्जुष्टां सौवर्णजलजाकुलाम्}
{हैमपङ्कां हैमजलां सौवर्णोज्ज्वलवालुकाम्}


\twolineshloka
{क्वचित्मुपिष्पितैः पूर्णां सौवर्णकुसुमोत्पलैः}
{क्वचित्तीररुहैः कीर्णां हैमपुष्पैः सुपुष्पितैः}


\twolineshloka
{तीर्थैश्च रुक्मसोपानैः सर्वतः समलङ्कुताम्}
{विमलैर्मणिजालैश्च नृत्तगीतरवैर्युताम्}


\twolineshloka
{दीप्तैर्हेमवितानैश्च समन्ताच्छोभितां शुभाम्}
{तथाविधां नदीं दृष्ट्वा पार्थस्तां प्रशशंस ह}


\twolineshloka
{अदृष्टपूर्वां राजेन्द्र दृष्ट्वा हर्षमवाप च}
{दर्शनीयां नदीतीरे पुरुषान्सुमनोहरान्}


\twolineshloka
{तान्नदीसलिलाहारान्सदारानमरोपमान्}
{नित्यं सुखमुदा युक्तान्सर्वालङ्कारशोभितान्}


\twolineshloka
{तेभ्यो बहूनि रत्नानि तदा लेभे धनञ्जयः}
{दिव्यजम्बूफलं हैमं भूषणानि च पेशलम्}


\twolineshloka
{लब्ध्वा तान्दुर्लभान्पार्थः प्रतीचीं प्रययौ दिशम्}
{नागानां रक्षितं देशमजयश्च पुनस्ततः}


\twolineshloka
{ततो गन्वा महाराज वारुणीं पाकशासनिः}
{गन्धमादनमासाद्य ततस्तानजयत्प्रभुः}


\twolineshloka
{तं गन्धमादनं राजन्नतिक्रम्य ततोऽर्जुनः}
{केतुमालं ददर्शाथ वर्षं रत्नसमन्वितम्}


\twolineshloka
{सेवितं देवकल्पैश्च नारीभिः प्रियदर्शनैः}
{तं जित्वा चार्जुनो राजन्करे च विनिवेश्य च}


\twolineshloka
{आहृत्य तत्र रत्नानि दुर्लभानि तथार्जुनः}
{पुनश्च परिवृत्याथ माध्यं देशमिलावृतम्}


\twolineshloka
{गत्वा प्राचीं दिशं राजन्सव्यसाची धनञ्जयः}
{मेरुमन्दरयोर्मध्ये शैलोदामभितो नदीम्}


\twolineshloka
{ये ते कीचकवेणूनां छायां रम्यामुपासते}
{कषान्झषांश्च नद्यौ तान्प्रघसान्दीप्तवेणिपान्}


\twolineshloka
{पशुपांश्च कुलिन्दांश्च तङ्कणान्परतङ्कणान्}
{एतान्समस्ताञ्जित्वा च करे च विनिवेश्य च}


\twolineshloka
{रत्नान्यादाय सर्वेभ्यो माल्यवन्तं ततो ययौ}
{तं माल्यवन्तं शैलेन्द्रं समतिक्रम्य पाण्डवः}


\twolineshloka
{भद्राश्वं प्रविवेशाथ वर्षं स्वर्गोपमं शुचिम्}
{तत्र देवोपमान्दिव्यान्पुरुषाञ्शुभसंयुतान्}


\twolineshloka
{जित्वा तान्स्ववशे कृत्वा करे च विनिवेश्य च}
{आहृत्य सर्वतो रत्नान्यसङ्ख्यानि ततस्ततः}


\twolineshloka
{नीलं नाम गिरिं गत्वा तत्रस्थानजयत्प्रभुः}
{ततो जिष्णुरतिक्रम्य पर्वतं नीलमायतम्}


\twolineshloka
{विवेश रम्यकं वर्षं सङ्कीर्णं मिथुनैः शुभैः}
{तं देशमथ जित्वा स करे च विनिवेश्य च}


\twolineshloka
{अजयच्चापि बीभत्सुर्देशं गुह्यकरक्षितम्}
{तत्र लेभे च राजेन्द्र सौवर्णान्मृगपक्षिणः}


\twolineshloka
{अगृह्णाद्यज्ञभूत्यर्थं रमणीयान्मनोहरान्}
{अन्यांश्च लब्ध्वा रत्नानि पाण्डवोऽथ महाबलः}


\twolineshloka
{गन्धर्वरक्षितं देशमजयत्सगणं तदा}
{तत्र रत्नानि दिव्यानि लब्ध्वा राजन्नथार्जुनः}


\twolineshloka
{वर्षं हिरण्वतं नाम विवेशाथ महीपते}
{स तु देशेषु रम्येषु गन्तुं तत्रोपचक्रमे}


\twolineshloka
{मध्ये प्रासादवृन्देषु नक्षत्राणां शशी यथा}
{महापथेषु राजैन्द्र सर्वतो यान्तमर्जुनम्}


\twolineshloka
{प्रासादवरशृङ्गस्थाः परया वीर्यशोभया}
{ददृशुस्तं स्रियः सर्वाः पार्थमात्मयशस्करम्}


\twolineshloka
{तं कलापधरं शूरं सरथं सधनुः करम्}
{सवर्मं सकिरीटं वै संनद्धं सपरिच्छदम्}


\twolineshloka
{सुकुमारं महासत्वं तेजोराशिमनुत्तमम्}
{शक्रोपमममित्रघ्नं परवारणवारणम्}


\twolineshloka
{पश्यन्तः स्त्रीगणास्तत्र शक्तिपाणिं स्म मेनिरे}
{अयं स पुरुषव्याघ्रो रणेऽद्भुतपराक्रमः}


\twolineshloka
{अस्य बाहुबलं प्राप्य न भवन्त्यसुहृद्गणाः}
{इति वाचो ब्रुवन्त्यस्ताः स्त्रियः प्रेम्णा धनञ्जयम्}


\twolineshloka
{तुष्टुवुः पुष्पवृष्टिं च ससृजुस्तस्य मूर्धनि}
{दृष्ट्वा ते तु मुदा युक्ताः कौतूहलसमन्वितः}


\twolineshloka
{रत्नैर्विभूषणैश्चैव अभ्यवर्षंश्च पाण्डवम्}
{अथ जित्वा समस्तांस्तान्करे च विनिवेश्य च}


\twolineshloka
{मणिहेमप्रबालानि शस्त्राण्याभरणानि च}
{एतानि लब्ध्वा पार्थोऽथ शृङ्गवन्तं गिरिं ययौ}


\twolineshloka
{शृङ्गवन्तं च कौरव्यः समतिक्रम्य फल्गुनः}
{उत्तरं हरिवर्षं तु स समासाद्य पाण्डवः}


\twolineshloka
{विद्याधरगणांश्चैव यक्षेन्द्रांश्च विनिर्जयन्}
{तत्र लेभे महात्मा वै वासो दिव्यमनुत्तमम्}


\twolineshloka
{किन्नरद्रुमपत्रांश्च तत्र कृष्णाजिनान्बहून्}
{याज्ञीयांस्तांस्तदा दिव्यांस्तत्र लेभे धनञ्जय'}


\twolineshloka
{उत्तरं हरिवर्षं तु स समासाद्य पाण्डवः}
{इयेष जेतुं तं देशं पाकशासनन्दनः}


\twolineshloka
{तत एनं महावीर्यं महाकाया महाबलाः}
{द्वारपालाः समासाद्य हृष्टा वचनमब्रुवन्}


\twolineshloka
{पार्थ नेदं त्वया शक्यं पुरं जेतुं कथञ्जन}
{उपावर्तस्व कल्याण पर्याप्तमिदमच्युत}


\twolineshloka
{इदं पुरं यः प्रविशेद्घ्रुवं न स भवेन्नरः}
{प्रीयामहे त्वया वीर पर्याप्तो विजयस्तवै}


\twolineshloka
{न चात्र किञ्चिज्जेतव्यमर्जुनात्र प्रदृश्यते}
{उत्तराः कुरुवो ह्येते नात्र युद्धं प्रवर्तते}


\twolineshloka
{प्रविष्टोऽपि हि कौन्तेय नेह द्रक्ष्यसि किञ्चन}
{न हि मानुषदेहेन शक्यमत्राभिवीक्षितुम्}


\twolineshloka
{अथेह पुरुषव्याघ्र किञ्चिदन्यच्चिकीर्षसि}
{तत्प्रब्रूहि करिष्यामो वचनात्तव भारत}


\twolineshloka
{ततस्तानब्रवीद्राजन्नर्जुनः प्रहसन्निव}
{पार्थिवत्वं चिकीर्षामि धर्मराजस्य धीमतः}


\twolineshloka
{न प्रवेक्ष्यामि वो देशं विरुद्धं यदि मानुषैः}
{युधिष्ठिराय यत्किञ्चित्करपण्यं प्रदीयताम्}


\threelineshloka
{`नो चेत्कृष्णेन सहितो योधयिष्यामि सायकैः'}
{ततो दिव्यानि वस्त्राणि दिव्यान्याभरणानि च}
{क्षौमाजिनानि दिव्यानि तस्य ते प्रददुः करम्}


\twolineshloka
{एवं स पुरुषव्याघ्रो विजित्य दिशमुत्तराम्}
{सङ्ग्रामान्सुबहून्कृत्वा क्षत्रियैर्दस्युभिस्तथा}


\twolineshloka
{स विनिर्जित्य राज्ञस्तान्करे च विनिवेश्य तु}
{धनान्यादाय सर्वेभ्यो रत्नानि विविधानि च}


\twolineshloka
{हयांस्तित्तिरिकल्माषाञ्शुकपत्रनिभानपि}
{मयूरसदृशान्यान्सर्वाननिलरंहसः}


\twolineshloka
{वृतः सुमहता राजन्बलेन चतुरङ्गिणा}
{आजगाम पुनर्वीरः शक्रप्रस्थं पुरोत्तमम्}


\twolineshloka
{धर्मराजाय तत्पार्थो धनं सर्वं सवाहनम्}
{न्यवेदयदनुज्ञातस्तेन राज्ञा गृहान्ययौ}


\chapter{अध्यायः ३०}
\twolineshloka
{एतस्मिन्नेव काले तु भीमसेनोऽपि वीर्यवान्}
{धर्मराजमनुज्ञाप्य ययौ प्राचीं दिशं प्रति ॥ 5}


\threelineshloka
{महता बलचक्रेण परराष्ट्रावमर्दिना}
{हस्त्यश्वरथपूर्णेन दंशितेन प्रतापवान्}
{}


\twolineshloka
{वृतो भरतशार्दूलो द्विषच्छोकविवर्धनः}
{स गत्वा नरशार्दूलः पञ्चालानां पुरं महत्}


\twolineshloka
{पञ्चालान्विविधोपायैः सान्त्वयामास पाण्डवः}
{`किञ्चित्करं समादाय विदेहानां पुरं ययौ' ॥ततः स गण्डकाञ्शूरो विदेहान्भरतर्षभः}


\fourlineindentedshloka
{विजित्याल्पेन कालेन दशार्णानजयत्प्रभुः}
{तत्र दाशार्णको राजा सुधर्मा रोमहर्षणम्}
{कृतवान्भीमसेनेम महद्युद्धं निरायुधम्}
{}


\twolineshloka
{भीमसेनस्तु तद्दृष्ट्वा तस्य कर्म महात्मनः}
{अधिसेनापतिं चक्रे सुधर्माणं महाबलम्}


\twolineshloka
{ततः प्राचीं दिशं भीमो ययौ भीमपराक्रमः}
{सैन्येन महता राजन्कम्पयन्निव मेदिनीम्}


\twolineshloka
{सोऽश्वमेधेश्वरं राजन्रोचमानं सहानुगम्}
{जिगाय समरे वीरो बलेन बलिनां वरः}


\twolineshloka
{स तं निर्जित्य कौन्तेयो नातितीव्रेण कर्मणा}
{पूर्वदेशं महावीर्यं विजिग्ये कुरनन्दनः}


\twolineshloka
{ततो दक्षिणमागम्य पुलिन्दनगरं महत्}
{सुकुमारं वशे चक्रे सुमित्रं च नराधिपम्}


\twolineshloka
{ततस्तु धर्मराजस्य शासनाद्भरतर्षभः}
{शिशुपालं महावीर्यमभ्यगाज्जनमेजय}


\twolineshloka
{चेदिराजोऽपि तच्छ्रुत्वा पाण्डवस्य चिकीर्षितम्}
{उपनिष्कम्य नगरात्प्रत्यगृह्णात्परन्तप}


\twolineshloka
{तौ समेत्य महाराज कुरुचेदिवृषौ तदा}
{उभयोरात्मकुलयोः कौशलं पर्यपृच्छताम्}


\threelineshloka
{ततो निवेद्य तद्राष्ट्रं चेदिराजो विशाम्पते}
{उवाच भीमं प्रहसन्किमिदं कुरुषेऽनघ}
{}


\twolineshloka
{तस्य भीमस्तदाचख्यौ धर्मराजचिकीर्षितम्}
{स च तं प्रतिगृह्यैव तथा चक्रे नराधिपः}


\twolineshloka
{ततो भीमस्तत्र राजन्निषित्वा त्रिदशाः क्षपाः}
{सत्कृतः शिशुपालेन ययौ सबलवाहनः}


\chapter{अध्यायः ३१}
\twolineshloka
{ततः कुमारविषये श्रेणिमन्तमथाजयत्}
{कोसलाधिपतिं चैव बृहद्बलमरिन्दमः}


\twolineshloka
{अयोध्यायां तु धर्मज्ञं दीर्घयज्ञं महाबलम्}
{अजयत्णण्डवश्रेष्ठो नातितीव्रेण कर्मणा}


\twolineshloka
{ततो गोपालकक्षं च सोत्तरानपि कोसलान्}
{मल्लानामधिपं चैव पार्थिवं चाजयत्प्रभुः}


\twolineshloka
{ततो हिमवतः पार्श्वं समभ्येत्य जलोद्भवम्}
{सर्वमल्पेन कालेन देशं चक्रे वशं बली}


\twolineshloka
{एवं बहुविधान्देशान्विजिग्ये भरतर्षभः}
{भल्लाटमभितो जिग्ये शुक्तिमन्तं च पर्वतम्}


\twolineshloka
{पाण्डवः सुमहावीर्यो बलेन बलिनां वरः}
{स काशिराजं समरे सुबाहुमनिवर्तिनम्}


\twolineshloka
{वशे चक्रे महाबाहुर्भीमो भीमपराक्रमः}
{ततः सुपार्श्वमभितस्तथा राजपतिं क्रथम्}


\twolineshloka
{युध्यमानं बालत्सङ्ख्ये विजिग्ये पाण्डवर्षभः}
{ततो मत्स्यान्महातेजा मलदांश्च महाबलान्}


\twolineshloka
{अनघानभयांश्चैव पशुभूमिं च सर्वशः}
{निवृत्य च महाबाहुर्मदधारं महीधरम्}


\twolineshloka
{सोमधेयांश्च निर्जित्य प्रत्ययावुत्तरामुखः}
{वत्सभूमिं च कौन्तेयो विजिग्ये बलवान्बलात्}


\twolineshloka
{भर्गाणामधिपं चैव निषादाधिपतिं तथा}
{विजिग्ये भूमिपालांश्च ममिमत्प्रमुखान्बहून्}


\twolineshloka
{ततो दक्षिणमल्लांश्च भोगवन्तं च पर्वतम्}
{तरसैवाजयद्भीमो नातितीव्रेण कर्मणा}


\twolineshloka
{शर्मकान्वर्मकांश्चैव व्यजयत्सान्त्वपूर्वकम्}
{वैदेहकं च राजानं जनकं जगतीपतिम्}


\twolineshloka
{विजिग्ये पुरुषव्याघ्रो नातितीव्रेण कर्मणा}
{शकांश्च बर्बराश्चैव अजयच्छद्मपूर्वकम्}


\twolineshloka
{वैदेहस्थस्तु कौन्तेय इन्द्रपर्वतमन्तिकात्}
{किरातानामधिपतीनजयत्सप्त पाण्डवः}


\twolineshloka
{ततः सुह्यान्प्रसुह्यांश्च सपक्षानतिवीर्यवान्}
{विजित्य युधि कौन्तेयो मागधानभ्यधाद्बली}


\twolineshloka
{दण्डं च दण्डधारं च विजित्य पृथिवीपतीन्}
{तैरेव सहितैः सर्वैर्गिरिव्रजमुपाद्रवत्}


\twolineshloka
{जारासन्धिं सान्त्वयित्वा करे च विनिवेश्य ह}
{तैरेव सहितैः सर्वैः कर्णमब्यद्रवद्बली}


\twolineshloka
{स कम्पयन्निव महीं बलेन चतुङ्गिणा}
{युयुधे पाण्डवश्रेष्ठः कर्णेनामित्रघातिना}


\twolineshloka
{स कर्णं युधि निर्जित्य वशे कृत्वा च भारत}
{ततो विजिग्ये बलवान्राज्ञः पर्वतवासिनः}


\twolineshloka
{अथ मोदागिरौ चैव राजानं बलवत्तरम्}
{पाण्डवो बाहुवीर्येण निजघान महामृधे}


\twolineshloka
{ततः पुण्ड्राधिपं वीरं वासुदेवं समाययौ}
{`इदानीं वृष्णिवीरेण न योत्स्यामीति पौण्ड्रकः}


\twolineshloka
{कृष्णस्य भुजसंत्रासात्करमाशु ददौ नृपः'}
{कौशिकीकच्छनिलयं राजानं च महौजसम्}


\twolineshloka
{उभौ बलभृतौ वीरावुमौ तीव्रपराक्रमौ}
{निर्जित्याजौ महाराज वङ्गराजमुपाद्रवत्}


\twolineshloka
{समुद्रसेन निर्जित्य चन्द्रसेनं च पार्थिवम्}
{ताम्रलिप्तं च राजानं कर्वटाधिपतिं तथा}


\twolineshloka
{सुह्यानामधिपं चैव ये च सागरवासिनः}
{सर्वान्म्लेच्छगणांश्चैव विजिग्ये भरतर्षभः}


\twolineshloka
{एवं बहुविधान्देशान्विजित्य पवनात्मजः}
{वसु तेभ्य उपादाय लौहित्यमगद्बली}


\twolineshloka
{स सर्वान्म्लेच्छनृपतीन्सागरानूपवासिनः}
{करमाहारयामास रत्नानि विविधानि च}


\twolineshloka
{चन्दनागुरुवस्त्राणि मणिमौक्तिककम्बलम्}
{काञ्चनं रजतं चैव विद्रुमं च महाधनम्}


\twolineshloka
{ते कोटीशतसङ्ख्येन कौन्तेयं महता तदा}
{अभ्यवर्षन्महात्मानं धनवर्षेण पाण्डवम्}


\twolineshloka
{इन्द्रप्रस्थमुपागम्य भीमो भीमपराक्रमः}
{निवेदयामास तदा धर्मराजाय तद्धनम्}


\chapter{अध्यायः ३२}
\twolineshloka
{तथैव सहदेवोऽपि धर्मराजेन पूजितः}
{महत्या सेनया राजन्प्रययौ दक्षिणां दिशम्}


\twolineshloka
{स शूरसेनान्कार्त्स्न्येन पूर्वमेवाजयत्प्रभुः}
{मत्स्यराजं च कौरव्यो वशे चक्रे बलाद्बली}


\twolineshloka
{अधिराजाधिपं चैव दन्तवक्रं महाबलम्}
{जिगाय करदं चैव कृत्वा राज्ये न्यवेशयत्}


\twolineshloka
{सुकुमारं वशे चक्रे सुमित्रं च नराधिपम्}
{तथैवापरमत्स्यांश्च व्यजयत्स पटच्चरान्}


\twolineshloka
{निषादभूमिं गोशृङ्गं पर्वतप्रवरं तथा}
{तरसैवाजयद्धीमाञ्श्रेणिमन्तं च पार्थिवम्}


\twolineshloka
{नरराष्ट्रं च निर्जित्य कुन्तिभोजमुपाद्रवत्}
{प्रीतिपूर्वं च तस्यासौ प्रतिजग्राह शासनम्}


\twolineshloka
{ततश्चर्मण्वतीकूले जम्भकस्यात्मजं नृपम्}
{ददर्श वासुदेवेन शेषितं पूर्ववैरिणा}


\twolineshloka
{चक्रे तेन स सङ्ग्रामं सहदेवेन भारत}
{स तमाजौ विनिर्जित्य दक्षिणाभिमुखो ययौ}


\threelineshloka
{सेकानपरसेकांश्च रत्नानि विविधानि च ॥ततस्तेनैव सहितो नर्मदामभितो ययौ}
{2-32-10a`भगदत्तं महाबाहुंक्षत्रियं नरकात्मजम्}
{अर्जुनाय करं दत्तं श्रुत्वा तत्र न्यवर्तत'}


\twolineshloka
{विन्दानुविन्दावावन्त्यौ सैन्येन महतावृतौ}
{जिगाय समरे वीरावाश्विनेयः प्रतापवान्}


\twolineshloka
{ततो रत्नान्युपादाय पुरं भोजकटं ययौ}
{तत्र युद्धमभूद्राजन्दिवसद्वयमच्युत}


\twolineshloka
{स विजित्य दुराधर्षं भीष्मकं माद्रिनन्दनः}
{कोसलाधिपतिं चैव तथा वेणातटाधिपम्}


\twolineshloka
{कान्तारकांश्च समर तथा प्राकोटकान्नृपान्}
{नाटकेयांश्च समरे तथा हेरम्बकान्युधि}


\twolineshloka
{मारुधं च विनिर्जित्य रम्यग्राममथो बलात्}
{नाचीनानर्बुकांश्चैव राजानश्च महाबलः}


\twolineshloka
{तांस्तानाटविकान्सर्वानजयत्पाण्डुनन्दनः}
{नाताधिपं च नृपतिं वशे चक्रे महाबलः}


\twolineshloka
{पुलिन्दांश्च रणे जित्वा ययौ दक्षिणतः पुरः}
{युयुधे पाण्ड्यराजेन दिवसं नकुलानुजः}


\twolineshloka
{तं जित्वा स महाबाहुः प्रययौ दक्षिणापथम्}
{गुहामासादयामास किष्किन्धां लोकविश्रुताम्}


\threelineshloka
{`पुरा वानरराजेन वालिना चाभिरक्षिताम्}
{ततः कोसलराजस्य रामस्यैवानुगेन च}
{सुग्रीवेणाभिगुप्तां तां प्रविष्टस्तमथाह्वयत्'}


\twolineshloka
{तत्र वानरराजाभ्यां मैन्देन द्विविदेन च}
{युयुधे दिवसान्सप्त न च तौ विकृतिं गतौ}


\twolineshloka
{ततस्तुष्टौ महात्मानौ सहदेवाय वानरौ}
{ऊचतुश्चैव संहृष्टौ प्रीतिपूर्वमिदं वचः}


\twolineshloka
{गच्छ पाण्डवशार्दूल रत्नान्यादाय सर्वशः}
{अविघ्नमस्त कार्याय धर्मराजाय धीमते}


% Check verse!
ततो रत्नान्युपादाय पुरीं माहिष्मतीं ययौ ॥तत्र नीलेन राज्ञा स चक्रे युद्धं नरर्षभः
% Check verse!
पाण्डवः परवीरघ्नः सहदेवः प्रतापवान् ॥ततोऽस्य सुमहद्युद्धमासीद्भीरुभयङ्करम्
\twolineshloka
{सैन्यक्षयकरं चैव प्राणानां संशयावहम्}
{चक्रे तस्य हि साहाय्यं भगवान्हव्यवाहनः}


\twolineshloka
{ततो रथा हया नागाः पुरुषाः कवचानि च}
{प्रतीप्तानि व्यदृश्यन्त सहदेवबले तदा}


\threelineshloka
{ततः सुसंभ्रान्तमना बभूव कुरुनन्दनः}
{नोत्तरं प्रतिवक्तुं च शक्तोऽभूज्जनमेजय ॥जनमेजय उवाच}
{}


\threelineshloka
{किमर्थं भगवान्वह्निः प्रत्यमित्रोऽभवद्युधि}
{सहदेवस्य यज्ञार्थं घटमानस्य वै द्विज ॥वैशम्पायन उवाच}
{}


\twolineshloka
{तत्र माहिष्मतीवासी भगवान्हव्यवाहनः}
{श्रूयते हि गृहीतो वै पुरस्तात्पारदारिकः}


\twolineshloka
{नीलस्य राज्ञो दुहिता बभूवतातीव शोभना}
{साऽग्निहोत्रमुपातिष्ठद्बोधनाय पितुः सदा}


% Check verse!
व्यजनैर्धूयमानोऽपि तावत्प्रज्वलते न सः ॥यावच्चारुपुटौष्ठेन वायुना न विधूयते
\twolineshloka
{ततः स भगवानग्निश्चकमे तां सुदर्शनाम्}
{नीलस्य राज्ञः सर्वेषामुपनीतश्च सोऽभवत्}


% Check verse!
ततो ब्रह्मणरूपेण रममाणो यदृच्छया ॥चकमे तां वरारोहां कन्यामुत्पललोचनाम् ॥तं तु राजा यथाशास्त्रमशासद्धार्मिकस्तदा
\twolineshloka
{प्रजज्वाल ततः कोपाद्भगवान्हव्यवाहनः}
{तं दृष्ट्वा विस्मितो राजा जगाम शिरसाऽवनिम्}


\twolineshloka
{ततः कालेन तां कन्यां तथैव हि तदा नृपः}
{प्रददौ विप्ररूपाय वह्रये शिरसा नतः}


\twolineshloka
{प्रतिगृह्य च तां सुभ्रुं नीलराज्ञः सुतां तदा}
{चक्रे प्रसादं भगवांस्तस्य राज्ञो विभावसुः}


\twolineshloka
{वरेण च्छन्दयामास तं नृपं स्विष्टकृत्तमः}
{अभयं च स जग्राह स्वसैन्ये वै महीपतिः}


\twolineshloka
{ततः प्रभृति ये केचिदज्ञानात्तां पुरीं नृपाः}
{जिगीषन्ति बलाद्राजंस्ते दह्यन्ते स्म वह्निना}


\twolineshloka
{तस्यां पुर्यां तदा चैव माहिष्मत्यां कुरूद्वह}
{बभूवुरनतिग्राह्य योषितश्छन्दतः किल}


\threelineshloka
{एवमग्निर्वरं प्रादात्स्त्रीणामप्रतिवारणे}
{स्वैरिण्यस्तत्र च राजानस्तत्पुरं भरतर्षभ}
{}


\twolineshloka
{वर्जयन्ति च राजानस्तत्पुरं भरतर्षभ}
{भयादग्नेर्महाराज तदाप्रभृति सर्वदा}


\fourlineindentedshloka
{सहदेवस्तु धर्मात्मा सैन्यं दृष्ट्वा भयार्दितम्}
{परीतमग्निना राजन्नाकम्पत यथाऽचलः}
{उपस्पृश्य शुचिर्भूत्वा सोऽब्रवीत्पावकं ततः ॥सहदेव उवाच}
{}


\twolineshloka
{त्वदर्थोऽयं समारम्भः कृष्णवर्त्मन्नमोस्तु ते}
{मुखं त्वमसि देवानां यज्ञस्त्वमसि पावक}


\twolineshloka
{पावनात्पावकश्चासि वहनाद्धव्यवाहनः}
{वेदास्त्वदर्थं जाता वै जातवेदास्ततो ह्यसि}


\twolineshloka
{चित्रभानुः सुरेशश्च अनलस्त्वं विभावसो}
{स्वर्गद्वारस्पृशश्चासि हुताशो ज्वलनः शिखी}


\twolineshloka
{वैश्वानरस्त्वं पिङ्गेशः प्लवङ्गो भूरितेजसः}
{कुमारसूस्त्वं भगवान्रुद्रगर्भो हिरण्यकृत्}


\twolineshloka
{अग्निर्ददातु मे तेजो वायुः प्राणं ददातु मे}
{पृथिवी बलमादध्याच्छिवं चापो दिशन्तु मे}


\twolineshloka
{अपां गर्भ महासत्व जातवेदः सुरेश्वर}
{देवानां मुखमग्ने त्वं सत्येन विपुनीहि माम्}


\twolineshloka
{ऋषिभिर्ब्राह्मणैश्चैव दैवतैरसुरैरपि}
{नित्यं सुहुत यज्ञेषु सत्येन विपुनीहि माम्}


\twolineshloka
{धूमकेतुः शिखी च त्वं पापहाऽनिसम्भवः}
{सर्वप्राणिषु नित्यस्थः सत्येन विपुनीहि माम्}


\threelineshloka
{एवं स्तुतोऽसि भगवन्प्रीतेन शिचिना मया}
{तुष्टिं पुष्टिं श्रुतं चैव प्रीति चाग्ने प्रयच्छ मे ॥वैशम्पायन उवाच}
{}


\threelineshloka
{इत्येवं मन्त्रमाग्नेयं पठन्यो जुहुयाद्विभुम्}
{ऋद्धिमान्सततं दान्तः सर्वपापैः प्रमुच्यते ॥सहदेव उवाच}
{}


\twolineshloka
{यज्ञविघ्नमिमं कर्तुं नार्हस्त्वं यज्ञवाहन}
{एवमुक्त्वा तु माद्रेयः कुशैरास्तीर्य मेदिनीम्}


\twolineshloka
{विधिवत्पुरुषव्याघ्रः पावकं प्रत्युपाविशत्}
{प्रमुखे तस्य सैन्यस्य भीतोद्विग्रस्य भारत}


\twolineshloka
{न चैनमत्यगाद्वह्निरुवाच महोदधिः}
{तमुपेत्य शनैर्वह्निरुवाच कुरुनन्दनम्}


\twolineshloka
{सहदेवं नृणां देवं सान्त्वपूर्वमिदं वचः}
{उत्तिष्ठोत्तिष्ठ कौरव्य जिज्ञासेयं कृता मया}


\twolineshloka
{वेद्मि सर्वमभिप्रायं तव धर्मसुतस्य च}
{मया तु रक्षितव्या पूरियं भरतसत्तम}


\twolineshloka
{यावद्राज्ञो हि नीलस्य कुले वंशधरा इति}
{ईप्सितं तु करिष्यामि मनसस्तव पाण्डव}


\twolineshloka
{तत उत्थाय हृष्टात्मा प्राञ्जलिः शिरसा नतः}
{पूजयामास माद्रेयटः पावकं भरतर्षभ}


\twolineshloka
{पावके विनिवृत्ते तु नीलो राजाऽभ्यगात्तदा}
{पावकस्याज्ञया चैनमर्चयामास पार्थिवः}


\twolineshloka
{सत्कारेण नरव्याघ्रं सहदेवं युधां पतिम्}
{प्रतिगृह्य च तां पूजां करे च विनिवेश्य च}


\twolineshloka
{माद्रीसुतस्ततः प्रायाद्विजयी दक्षिणां दिशम्}
{त्रैपुरं स वशे कृत्वा राजानममितौजसम्}


\twolineshloka
{निजग्राह महाबाहुस्तरसा पौरवेश्वरम्}
{आकृतिं कौशिकाचार्यं यत्ने महता ततः}


\twolineshloka
{वशे चक्रे महाबाहुः सुराष्ट्राधिपतिं तदा}
{सुराष्ट्रविषयस्थश्च प्रेषयामास रुक्मिणे}


\twolineshloka
{राज्ञे भोजकटस्थाय महामात्राय धीमते}
{भीष्मकायस धर्मात्मा साक्षादिन्द्रसखाय वै}


\twolineshloka
{च चास्य प्रतिजग्राह ससुतः शासनं तदा}
{प्रीतिपूर्वं महाराज वासुदेवमवेक्ष्य च}


\twolineshloka
{ततः स रत्नान्यादाय पुनः प्रायाद्युधां पतिः}
{ततः शूर्पारकं चैव तालाकटमथापि च}


\twolineshloka
{वशे चक्रे महातेजा दण्डकांश्च महाबलः}
{सागरद्वीपवासांश्च नृपतीन्म्लेच्छयोनिजान्}


\twolineshloka
{निषादान्पुरुषादांश्च कर्णप्रावरणानपि}
{ये च कालमुखा नाम नरराक्षसयोनयः}


\twolineshloka
{कृत्स्नं कोलगिरिं चैव सुरभीपट्टनं तथा}
{द्वीपं ताम्राह्वयं चैव पर्वतं रामकं तथा}


\twolineshloka
{तिमिङ्गिलं च स नृपं वशे कृत्वा महामतिः}
{एकपादांश्च पुरुषान्केरलान्वनवासिनः}


\twolineshloka
{नगरीं सञ्जयन्तीं च पाषण्डं करहाटकम्}
{दूतैरेव वशे चक्रे करं चैनानदापयत्}


\twolineshloka
{पाण्ड्यांश्च द्रविडांश्चैव सहितांश्चोड्रकेरलैः}
{अन्ध्रांस्तावनांश्चैव कलिङ्गानुष्ट्रकर्णिकान्}


\twolineshloka
{आटवीं च पुरीं रम्यां यवनानां पुरं तथा}
{दूतैरेव वशे चक्रे करं चैनानदापयत्}


\twolineshloka
{`तात्रपर्णी ततो गत्वा कन्यातीर्थमतीत्य च}
{दक्षिणां च दिशं सर्वा विजित्य कुरुनन्दनः}


\twolineshloka
{उत्तरं तीरमासाद्य सागरस्योर्मिमालिनः}
{चिन्तयामास कौन्तेयो भ्रातुः पुत्रं घटोत्कचम्}


\twolineshloka
{ततश्चिन्तितमात्रस्तु राक्षसः प्रत्यदृश्यत}
{तं मेरुशिखराकारमागतं पाण्डुनन्दनः}


\twolineshloka
{भृगुकच्छात्ततो धीमान्साम्नैवामित्रकर्शनः}
{आगम्यतामिति प्राह धर्मराजस्य शसनाः}


\twolineshloka
{स राक्षसपरीवारस्तं प्रणम्याशु संस्थितः}
{घटोत्कचं महात्मानं राक्षसं घोरदर्शनम्}


\twolineshloka
{तत्रस्थः प्रेषयामास पौलस्त्याय महात्मने'}
{बिभीषणाय धर्मात्मा प्रीतिपूर्वमरिन्दमः}


\twolineshloka
{स चास्य प्रतिजग्राह शासनं प्रीतिपूर्वकम्}
{तच्च कृष्णकृतं धीमानभ्यमन्यत स प्रभुः}


\twolineshloka
{ततः सम्प्रेषयामास रत्नानि विविधानि च}
{चन्दनागुरुकाष्ठानि दिव्यान्याभरणानि च ॥वासांसि च महार्हाणि मणींश्चैव महाधनान्}


\chapter{अध्यायः ३३}
\twolineshloka
{इच्छाम्यागमनं श्रोतुं हैडिम्बस्य द्विजोत्तम}
{लङ्कायां च गतिं ब्रह्मन्पौलस्त्यस्य च दर्शनम्}


\threelineshloka
{कावेरीदर्शनं चैव आनुपूर्व्या वदस्व मे}
{वैशम्पायन उवाच}
{शृणु राजन्यथा वृत्तं सहेदवस्य साहसम्}


\twolineshloka
{कालनद्वीपगांश्चैव तरसाऽजित्य चाहवे}
{दक्षिणां च दिशं जित्वा चोलस्य विषयं ययौ}


\twolineshloka
{ददर्श पुण्यतोयां वै कावेरीं सरितां वराम्}
{नाजापक्षिगणैर्जुष्टां तापसैरुपशोभिताम्}


\twolineshloka
{------------------------}
{कदम्बैः सप्तपर्णैश्च कश्मर्यामलकैर्वृताम्}


\twolineshloka
{-----महाशाखैः प्लक्षैरौदुम्बरैरपि}
{----- अश्वत्थैः खदिरैर्वृताम्}


\twolineshloka
{-------- सञ्छन्नामश्वकर्णैश्च शोभिताम्}
{बूतैः पुण्ड्रकपत्रैश्च कदलीवनसंवृताम्}


\twolineshloka
{चक्रवाकगणैः कीर्णं प्लवैश्च जलवायसैः}
{समुद्रकाकैः क्रौञ्चैश्च नादितां जलकुक्कुटैः}


\twolineshloka
{एवं खगैश्च बहुभिः सङ्घुष्टां जलवासिभिः}
{आश्रमैर्बहुभिः सक्तां चैत्यवृक्षैश्च शोभिताम्}


\twolineshloka
{शोभितां ब्राह्मणैः शुभ्रैर्वेदवेदाङ्गपारगैः}
{क्वचित्तीररुहैर्वृक्षैर्मालाभिरिव शोभिताम्}


\twolineshloka
{क्वचित्सुपुष्पितैर्वृक्षैः क्वचित्सौगन्धिकोत्पलैः}
{कह्लारकुमुदोत्फुल्लैः कमलैरुपशोभिताम्}


\twolineshloka
{कावेरीं तादृशीं दृष्ट्वा प्रीतिमान्पाण्डवस्तदा}
{अस्मद्राष्ट्रे यथा गङ्गा कावेरी च तथा शुभा}


\twolineshloka
{सहदेवस्तु तां तीर्त्वा नदीमनुचरैः सह}
{दक्षिणं तीरभासाद्य गमनायोपचक्रमे}


\twolineshloka
{आगतं पाण्डवं तत्र श्रुत्वा विषयवासिनः}
{दर्शनार्थं ययुस्ते तु कौतूहलसमन्विताः}


\twolineshloka
{द्रमिडाः पुरुषा राजन्स्रियचश्च प्रियदर्शनाः}
{गत्वा पाण्डुसुतं तत्र ददृशुस्ते मुदाऽन्विताः}


\twolineshloka
{सुकुमारं विशालाक्षं व्रजन्तं त्रिदशोपमम्}
{दर्शनीयतमं लोके नेत्रैरनिमिषैरिव}


\twolineshloka
{आश्चर्यभूतं ददृशुर्द्रमिडास्ते समागताः}
{महासेनोपमं दृष्ट्वा पूजां चक्रुश्च तस्य वै}


\twolineshloka
{रत्नैश्च विविधैरिष्टैर्भोगैरन्यैश्च संमतैः}
{गतिमङ्गलयुक्तार्भिः स्तुवन्तो नकुलानुजम्}


\twolineshloka
{सहदेवस्तु तान्दृष्ट्वा द्रमिलानागतान्मुदा}
{विसृज्य तान्महाबाहुः प्रस्थितो दक्षिणां दिशम्}


\twolineshloka
{दूतेन तरसा चोलं विजित्य द्रमिडेश्वरम्}
{ततो रत्नान्युपादाय पाण्डस्य विषयं ययौ}


\twolineshloka
{दर्शने सहदेवस्य न च तृप्ता नराः परे}
{गच्छन्तमनुगच्छन्तः प्राप्ताः कौतूहलान्विताः}


\twolineshloka
{ततो माद्रीसुतों राजन्मृगसङ्घान्विलोकयन्}
{गजान्वनचरानन्यान्व्याघ्रान्कुष्णमृगान्बहून्}


\twolineshloka
{शुकान्मयूरान्दृष्ट्वा तु गृध्रानारण्यकुक्कुटान्}
{ततो देशं समासाद्य श्वशुरस्य महीपतेः}


\twolineshloka
{प्रेषयामास माद्रेयो दूतान्पाण्ड्याय वै तदा}
{प्रतिजग्राह तस्याज्ञां सम्प्रीत्या मलयध्वजः}


\twolineshloka
{भार्या रूपवती जिष्णोः पाण्ड्यस्य तनया शुभा}
{चित्राङ्गदेति विख्याता द्रमिडी योषितां वरा}


\twolineshloka
{आगतं सहदेवं तु सा श्रुत्वाऽन्तः पुरे पितुः}
{प्रेषयामास सम्प्रीत्या पूजारत्नं च वै बहु}


\twolineshloka
{पाण्ड्योऽपि बहुरत्नानि दूतैः सह मुमोच ह}
{मणिमुक्ताप्रवालानि सहदेवाय कीर्तिमान्}


\twolineshloka
{तां दृष्ट्वाऽप्रतिमां पूजां पाण्डवोऽपि मुदा नृप}
{भ्रातुः पुत्रे बहून्रत्नान्दत्वा वै बभ्रूवाहने}


\twolineshloka
{पाण्ड्यं द्रमिडराजानं श्वशुरं मलयध्वजम्}
{स दूतैस्तं वशे कृत्वा मणलूरेश्वरं तदा}


\twolineshloka
{ततो रत्नान्युपादाय द्रमिडैरावृतो ययौ}
{अगस्त्यस्यालयं दिव्यं देवलोकसमं गिरिम्}


\twolineshloka
{स तं प्रदक्षिमं कृत्वा मलयं भरतर्षभ}
{लङ्घयित्वा तु माद्रेयस्ताम्रपणीं नदीं शुभाम्}


\twolineshloka
{प्रसन्नसलिलां दिव्यां सुशीतां च मनोहराम्}
{समुद्रतीरमासाद्य न्यविशत्पाण्डुनन्दनः}


\chapter{अध्यायः ३४}
\twolineshloka
{सहदेवस्ततो राजा मन्त्रिभिः सह भारत}
{सम्प्रधार्य महाबाहुः सचिवैर्बुद्धिमत्तरैः}


\twolineshloka
{स विचार्य तदा राजन्सहदेवस्त्वरान्वितः}
{चिन्तयामास राजेन्द्र भ्रातुः पुत्रं घटोत्कचम्}


\twolineshloka
{ततश्चिन्तितमात्रे तु राक्षसः प्रत्यदृश्यत}
{अतिदीर्घो महाबाहु सर्वाभरणभूषितः}


\twolineshloka
{नीलजीमूतसङ्काशस्तप्तकाञ्चनकुण्डलः}
{विचित्रहारकेयूरः किङ्किणीमणिभूषितः}


\twolineshloka
{हेममाली महादंष्ट्रः किरीटी कुक्षिबन्धनः}
{ताम्रकेशो हरिश्मश्रुर्भीमाङ्गः कटकाङ्गदः}


\twolineshloka
{रक्तचन्दनदिग्धाङ्गः सूक्ष्माम्बरधरो बली}
{बलेन स ययौ तत्र चालयन्निव मेदिनीम्}


\twolineshloka
{ततो दृष्ट्वा जना राजन्नायान्तं पर्वतोपमम्}
{भयाद्धि दुद्रुवुः सर्वे सिंहात्क्षुद्रमृगा यथा}


\twolineshloka
{आससाद च माद्रेयं पुलस्त्यं रावणो यथा}
{अभिवाद्य ततो राजन्सहदेवं घटोत्कचः}


\twolineshloka
{प्रह्वः कृताञ्जलिस्तस्थौ किं कार्यमिति चाब्रवीत्}
{तं परिष्वज्य बाहुभ्यां मूर्ध्न्युपाघ्राय पाण्डवः}


\twolineshloka
{तं मेरुशिखराकारमागतं पाण्डुनन्दनः}
{पूजयित्वा सहामात्यः प्रीतो वाक्यमुवाच ह}


\twolineshloka
{गच्छ लङ्कां पुरीं वत्स करार्थं मम शासनात्}
{तत्र दृष्ट्वा महात्मानं राक्षसेन्द्रं बिभीषणम्}


\threelineshloka
{रत्नानि राजसूयार्थं विविधानि बहूनि च}
{उपादाय च सर्वाणि प्रत्यागच्छ महाबल ॥वैशम्पायन उवाच}
{}


\twolineshloka
{पाण्डवेनैवमुक्तस्तु मुदा युक्तो घटोत्कचः}
{तथेत्युक्त्वा महाराज प्रतस्ये दक्षिणां दिशम्}


\twolineshloka
{प्रययौ दक्षिणं कृत्वा सहदेवं घटोत्कचः}
{लङ्कामभिमुको राजन्समुद्रं स व्यलोकयत्}


\twolineshloka
{कूर्मग्राहझषाकीर्णं मीननक्रैस्तथाऽऽकुलम्}
{शुक्तिव्रातसमाकीर्णं शङ्कानां निचयाकुलम्}


% Check verse!
स दृष्ट्वा रामसेतुं च चिन्तयन्रामविक्रमम् ॥गत्वा पारं समुद्रस्य दक्षिणं स घटोत्कचः
\twolineshloka
{ददर्श लङ्कां राजेन्द्र नाकपृष्ठोपमां शुभाम्}
{प्राकारेणावृतां रम्यां शुभद्वारैश्च शोभिताम्}


\twolineshloka
{प्रासादैर्बहुसाहस्रैः श्वेतरक्तैश्च सङ्कुलाम्}
{दिव्यदुन्दुभिनिर्ह्रादामुद्यानवनशोभिताम्}


\twolineshloka
{सर्वकालफलैर्वृक्षैः पुष्पितैरुपशोभिताम्}
{पुष्पगन्धैश्च सङ्कीर्णां रमणीयमहारथाम्}


\twolineshloka
{नानारत्नैश्च सम्पूर्णामिन्द्रस्येवामरावतीम्}
{विवेश स पुरीं लङ्कां राक्षसैश्च निषेविताम्}


% Check verse!
ददर्श स पुरीं लङ्कां राक्षसैश्च निषेविताम् ॥नानावेषधरान्दक्षान्नारीश्च प्रियदर्शनाः
\twolineshloka
{दिव्यमाल्याम्बरधरा दिव्यभूषणभूषिताः}
{मदरक्तान्तनयनाः पीनश्रोणिपयोधराः}


\twolineshloka
{भैमसेनिं ततो दृष्ट्वा हृष्टास्ते विस्मयं गताः}
{आससाद गृहं राज्ञ इन्द्रस्य सदनोपमम्}


\twolineshloka
{स द्वारपालमासाद्य वाक्यमेतदुवाच ह}
{घटोत्कच उवाच ॥कुरूणामृष्टबो राजा पाण्डुर्नाम महाबलः}


\twolineshloka
{कनीयांस्तस्य दायादः सहदेव इति श्रुतः}
{तेनाहं प्रेषितो दूतः करार्थं कौरवस्य च}


\twolineshloka
{द्रष्टुमिच्छामि राजेनद्रं त्वं क्षिप्रं मां निवेदय}
{वैशम्पायन उवाच ॥तस्य तद्वचनं श्रुत्वा द्वारपालो महीपते}


\twolineshloka
{तथेत्युक्त्वा विवेशाथ भवनं स निवेदकः}
{प्राञ्जलिः स्रवमाचष्ट स्रवां दूतगिरं तदा}


\twolineshloka
{द्वारपालवचः श्रुत्वा राक्षसेन्द्रो विभीषणः}
{उवाच वाक्यं धर्मात्मा समीपं मे प्रवेश्यताम्}


\twolineshloka
{एवमुक्तस्तु राज्ञा स धर्मज्ञेन महात्मना}
{अथनिष्कम्य सम्भ्रान्तो द्वार्स्थोहैडिम्बमब्रवीत्}


\twolineshloka
{एहि दूत नृपं द्रष्टुं क्षिप्रं प्रविश च स्वयम्}
{द्वारपालवचः श्रुत्वा प्रविवेश घटोत्कचः}


\twolineshloka
{स प्रविश्य ददर्शाथ राक्षसेन्द्रस्य मन्दिरम्}
{ततः कैलाससङ्काशं तत्पकाञ्चनतोरणम्}


\twolineshloka
{प्राकारेण परिक्षिप्तं गोपुरैश्चापि शोभितम्}
{हर्म्यप्रासादसम्बाधं नानारत्नोपशोभितम्}


\twolineshloka
{काञ्चनैस्तापनीयैश्च स्फाटिकै राजतैरपि}
{वज्रवैडूर्यजुष्टैश्च स्तम्भैश्च सुमनोहरैः}


\twolineshloka
{नानाध्वजपताकाभिर्युक्तं मणिविचित्रितम्}
{चित्रमाल्यावृतं रम्यं तप्तकाञ्चनवेदिकम्}


\twolineshloka
{स दृष्ट्वा तत्र सर्वं च भैमसेनिर्मनोहरम्}
{प्रविशन्नेव हैडिम्बः शुश्राव मधुरस्वरम्}


\twolineshloka
{तन्त्रीगीतसमाकीर्णं समतालमिताक्षरम्}
{दिव्यदुन्दुभिनिर्ह्रादं वादित्रसततं शुभम्}


\twolineshloka
{स श्रुत्वा मधुरं शब्दं प्रीतिमानभवत्तदा}
{ततो विगाह्य हैडिम्बो बहुकक्ष्यां मनोरमाम्}


\twolineshloka
{स ददर्श महात्मानं द्वार्स्थेन सह भारत}
{तं विभीषणमासीनं काञ्चने परमासने}


\twolineshloka
{दिवि भास्करसङ्काशं मुक्तामणिविभूषितम्}
{दिव्याभरणचित्राङ्गं दिव्यरूपधरं विभुम्}


\twolineshloka
{दिव्यमाल्याम्बरधरं दिव्यगन्धोक्षितं शुभम्}
{विभ्राजमानं वपुषा सूर्यवैश्वानरप्रभम्}


\twolineshloka
{उपोपविष्टं सचिवैर्देवैरिव शतक्रतुम्}
{यक्षैर्महात्मभिर्दिव्यनारीभिर्हृद्यकान्तिभिः}


\twolineshloka
{गीतैर्मङ्गलयुक्तैश्च पूज्यमानं यथा दिवि}
{चामरे व्यजने चाग्र्ये हेमदण्डे महाधने}


\twolineshloka
{गृहीते वरनारीभ्यां धूयमाने च मूर्धनि}
{अर्चिष्मन्तं श्रिया जुष्टं कुबेरवरुणोपमम्}


\twolineshloka
{धर्मे चैव स्थितं नित्यमद्भुतं राक्षसेस्वरम्}
{राममिक्ष्वाकुनाथं वै स्मरन्तं मनसा सदा}


\twolineshloka
{दृष्ट्वा घटोत्कचो राजन्ववन्दे तं कृताञ्जलिः}
{प्रह्वस्तस्थौ महावीर्यः शक्रं चित्ररथो यथा}


\twolineshloka
{तं दूतमागतं दृष्ट्वा राक्षसेन्द्रो विभीषणः}
{पूजयित्वा यथान्यायं सान्त्वपूर्वं वचोऽब्रवीत् ॥विभीषण उवाच}


\twolineshloka
{कस्य वंशे स सञ्जातः करमिच्छन्महीपतिः}
{तस्यानुजान्समस्तांश्च पुरं देशं च तस्य वै}


\threelineshloka
{त्वां च कार्यं च तत्सर्वं श्रोतुमिच्छामि तत्वतः}
{विस्तरेण मम ब्रूहि सर्वानेतान्पृथक्पृथक् ॥वैशम्पायन उवाच}
{}


\threelineshloka
{एवमुक्तस्तु हैडिम्बः पौलस्त्येन महात्मना}
{कृताञ्जलिरुवाचाथ समर्थमिदमुत्तरम् ॥घटोत्कच उवाच}
{}


\twolineshloka
{सोमवंशोद्भवो राजा पाण्डुर्नाम महाबलः}
{पाण्डोश्च पुत्राः पञ्चासञ्छक्रतुल्यपराक्रमाः}


\twolineshloka
{तेषां ज्येष्ठस्तु नाम्ना वै युधिष्ठिर इति श्रुतः}
{अजातशत्रुर्धर्मात्मा धर्मो विग्रहवानिव}


\twolineshloka
{ततो युधिष्ठिरो राजा प्राप्य राज्यमकारयत्}
{गङ्गाया दक्षिणे तीरे नगरे नागसाह्वये}


\twolineshloka
{तद्दत्वा धृतराष्ट्राय शक्रप्रस्थं ययौ ततः}
{भ्रातृभिः सह राजेन्द्र शक्रप्रस्थेऽन्वमोदत}


\twolineshloka
{गङ्गायमुनयोर्मध्ये ते उभे नगरोत्तमे}
{नित्यं धर्मे स्थितो राजा शक्रप्रस्थे प्रशास्ति नः}


\twolineshloka
{तस्यानुजो महाबाहुर्भीमसेन इति श्रुतः}
{महातेजा महाकीर्तिः शक्रतुल्यपराक्रमः}


\twolineshloka
{दशनागसहस्राणां बले तुल्यः स पाण्डवः}
{तस्यानुजोऽर्जुनो नाम महाबलपराक्रमः}


\twolineshloka
{सुकुमारो महासत्वो लोके वीर्येण विश्रुतः}
{कार्तवीर्यसमो वीर्ये सागरप्रतिमो बले}


\twolineshloka
{जामदग्न्यसमश्चास्त्रे सङ्ख्ये रामसमोऽर्जुनः}
{रूपे शक्रसमः पार्थस्तेजसा भास्करोपमः}


\twolineshloka
{देवदानवगन्धर्वैः पिशाचोरगराक्षसैः}
{मानुषैश्च समस्तैस्तु अजेयः फल्गुनो रणे}


\twolineshloka
{तेन तत्खण्डवं दावं तर्पितं जातवेदसे}
{विजित्य तरसा शक्रं युधि देवगणैः सह}


\twolineshloka
{लब्धान्यस्त्राणि दिव्यानि तर्पयित्वा हुताशनम्}
{तेन लब्धा महाराज दुर्लभा दैवतैरपि}


\twolineshloka
{वासुदेवस्य भगिनी सुभद्रा नाम विश्रुता}
{अर्जुनस्यानुजो राजन्नकुलश्चेति विश्रुतः}


\twolineshloka
{दर्शनीयतमो लोके मूर्तिमानिव मन्मथः}
{तस्यानुजो महातेजाः सहदेव इति श्रुतः}


\twolineshloka
{तेनाहं प्रेषितो राजन्कुमारेण समो रणे}
{अहं घटोत्कचो नाम भीमसेनसुतो बली}


\twolineshloka
{मम माता महाभागा हिडिम्बा नाम राक्षसी}
{पार्थानामुपकारार्थं चरामि पृथिवीमिमाम्}


\twolineshloka
{आसीत्पृथिव्याः सर्वस्या महीपालो युधिष्ठिरः}
{राजसूयं क्रतुश्रेष्ठमाहर्तुमुपचक्रमे}


\twolineshloka
{सन्दिदेश च स भ्रातृन्करार्थं सर्वतोदिशम्}
{वृष्णिवीरेण सहितः सन्दिदेशानुजान्नृपः}


\twolineshloka
{उदीचीमर्जुनस्तूर्णं गत्वा मेरोरथोत्तमः}
{गत्वा शतसहस्राणि योजनानि महाबलः}


\twolineshloka
{जित्वा सर्वान्नृपान्युद्धे हत्वा च तरसा वशी}
{स्वर्गद्वारमुपागम्य रत्नान्यादाय वै भृशम्}


\twolineshloka
{अश्वांश्च विविधान्दिव्यान्सर्वानादाय फल्गुनः}
{धनं बहुविधं राजन्धर्मपुत्राय वै ददौ}


\twolineshloka
{भीमसेनो हि राजेन्द्र जित्वा प्राचीं दिशं बलात्}
{वशे कृत्वा महीपालान्पाण्डवाय धनं ददौ}


\twolineshloka
{दिशं प्रतीचीं नकुलः करार्थं प्रययौ तथा}
{सहदेवो दिशं याम्यां जित्वा सर्वान्महीक्षितः}


\twolineshloka
{मां सन्दिदेश राजेन्द्र करार्थमिह सत्कृतः}
{पार्थानां चरितं तुभ्यं सङ्क्षेपात्समुदाहृतम्}


\threelineshloka
{तमवेक्ष्य महाराज धर्मराजं युधिष्ठिरम्}
{पावनं राजसूयं च भगवन्तं हरिं प्रभुम्}
{एतानवेक्ष्य धर्मज्ञ करं दातुमिहार्हसि ॥वैशम्पायन उवाच}


\twolineshloka
{तेन तद्भाषितं श्रुत्वा राक्षसेन्द्रो बिभीषणः}
{शासनं प्रतिजग्राह धर्मात्मा राक्षसैः सह}


\twolineshloka
{तच्च कृष्णकृतं धीमानित्यमन्यत स प्रभुः}
{ततो ददौ विचित्राणि कम्बलानि कुथानि च}


\twolineshloka
{दान्तकाञ्चनपर्यङ्कान्मणिहेमविचित्रितान्}
{भूषणानि महार्हाणि प्रवालानि मणींश्च सः}


\twolineshloka
{काञ्चनानि च भाण्डनि कलशानि घटानि च}
{कटाहानि विचित्रानि द्रोण्यश्चैव सहस्रशः}


\threelineshloka
{राजतानि च भाण्डानि रत्नगर्भांश्च कुण्डलान्}
{हेमपुष्पानि चान्यानि रुक्ममुख्यानि चापरान्}
{}


\twolineshloka
{शङ्खांश्च चन्द्रसङ्काशांश्चित्रावर्तविचित्रितान्}
{यज्ञस्य तोरणे युक्तान्ददौ तालांश्चतुर्दश}


\twolineshloka
{रुक्मपङ्कजपुष्पाणि शिबिका मणिभूषिताः}
{मुकुटानि महार्हाणि रत्नगर्भांश्च कङ्कणान्}


\twolineshloka
{चन्दनानि च मुख्यानि वासांसि विविधानि च}
{स ददौ सहदेवाय तदा राजा विभीषणाः}


\twolineshloka
{तानि सर्वाणि रत्नानि आजह्रुस्ते निशाचराः}
{अष्टाशीतिसहस्राणि समदा रक्तलोचनाः}


\twolineshloka
{रत्नान्यादाय सर्वाणि प्रतस्थे स घटोत्कचः}
{विभीषणं च राजानमभिवाद्य कृताञ्जलिः}


\twolineshloka
{प्रदक्षिणं परीत्यैव निर्जगाम घटोत्कचः}
{ततो रत्नान्युपादाय हैडिम्बो राक्षसैः सह}


\twolineshloka
{जगाम तूर्णं लङ्कायाः सहदेवपदं प्रति}
{आसेदुः पाण्डवं सर्वे लङ्घयित्वा महोदधिम्}


\twolineshloka
{सहदेवो ददर्शाथ रत्नाहारान्निशाचरान्}
{आगतान्भीमसङ्काशान्हैडिम्बं च तथा नृप}


\twolineshloka
{द्रमिला नैऋतान्दृष्ट्वा दुद्रुवुस्ते भयार्दिताः}
{भैमसेनिस्ततो गत्वा मार्देयं प्राञ्जलिः स्थितः}


\twolineshloka
{प्रीतिमानभवद्दृष्ट्वा रत्नौधं तं च पाण्डवः}
{तं परिष्वज्य पाणिभ्यां दृष्ट्वा तान्प्रीतिमानभूत्}


\twolineshloka
{विसृज्य द्रविडान्सर्वान्गमनायोपचक्रमे}
{न्यवर्तम ततो धीमान्सहदेवो नराधिपः}


\twolineshloka
{एवं विजित्य तरसा सान्त्वेन विजयेन च}
{करदान्पार्थिवान्कृत्वा प्रत्यागच्छदरिन्दमः}


\twolineshloka
{रत्नसालमुपादाय ययौ सहनिशाचरः}
{इन्द्रप्रस्थं विवेशाथ कम्पयन्निव मेदिनीम्}


\twolineshloka
{दृष्ट्वा युधिष्ठिरं राजन्सहदेवः कृताञ्जलिः}
{प्रह्वोऽभिवाद्य तस्थौ स पूजितश्चापि तेन वै}


\twolineshloka
{लङ्काप्राप्तान्धनौघांश्च दृष्ट्वा तान्दुर्लभान्बहून्}
{प्रीतिमानभवद्राजा विस्मयं परमं ययौ}


\twolineshloka
{धर्मराजाय तत्सर्वं निवेद्य भरतर्षभ}
{कोटीसहस्रमधिकं हिरण्यस्य महात्मने}


\twolineshloka
{विविधानि च रत्नानि गोजाविमहिषांस्तथा}
{कृतकर्मा सुखं राजन्नुवास जनमेजय}


\chapter{अध्यायः ३५}
\twolineshloka
{नकुलस्य तु वक्ष्यामि कर्माणि विजयं तथा}
{वासुदेवजितामाशां यथाऽसावजयत्प्रभुः}


\twolineshloka
{निर्याय खाण्डवप्रस्थात्प्रतीचीमभितो दिशम्}
{उद्दिश्य मतिमान्प्रायान्महत्या सेनया सह}


\twolineshloka
{सिंहनादेन महता योधानां गर्जितेन च}
{रथनेमिनिनादैश्च कम्पयन्वसुधामिमाम्}


\twolineshloka
{ततो बहु धनं रम्यं गावाढ्यं धनधान्यवत्}
{कार्तिकेयस्य दयितं रोहीतकुपाद्रवत्}


\twolineshloka
{तत्र युद्धं महच्चासीच्छूरैर्मत्तमयूरकैः}
{मरुभूमिं स कार्त्स्न्येन तथैव बहुधान्यकम्}


\twolineshloka
{शैरीषकं महेत्थं च वशे चक्रे महाद्युतिः}
{आक्रोशं चैव राजर्षि तेन युद्धमभून्महत्}


\twolineshloka
{तान्दशार्णान्स जित्वा च प्रतस्थे पाण्डुनन्दनः}
{शिबींस्त्रिगर्तानम्बष्ठान्मालवान्पञ्च कर्पटान्}


\twolineshloka
{तथा मध्यमकेयांश्च वाटधानान्द्विजानथ}
{पुनश्च परिवृत्याथ पुष्करारण्यवासिनः}


\twolineshloka
{गणानुत्सवसङ्केतान्व्यजयत्पुरुषर्षभः}
{सिन्धुकूलाश्रिता ये च ग्रामणीया महाबलाः}


\twolineshloka
{शूद्राभीरगणाश्चैव ये चाश्रित्य सरस्वतीम्}
{वर्तयन्ति च ये मत्स्यैर्ये च पर्वतवासिनः}


\twolineshloka
{कत्स्नं पञ्चनन्दं चैव तथैवामरपर्वतम्}
{उत्तरज्योतिषं चैव तथा दिव्यकटं पुरम्}


\twolineshloka
{द्वारपालं च तरसा वशे चक्रे महाद्युतिः}
{रामठान्हारहूणांश्च प्रतीच्याश्चैव ये नृपाः}


\twolineshloka
{तान्सर्वान्स वशे चक्रे शासनादेव पाण्डवः}
{तत्रस्थः प्रेषयामास वासुदेवाय भारत}


\twolineshloka
{स चास्य गतभी राजन्प्रतिजग्राह शासनम्}
{ततः शाकलमभ्येत्य मद्राणां पुटभेदनम्}


\twolineshloka
{मातुलं प्रीतिपूर्वेण शल्यं चक्रे वशे बली}
{स तेन सत्कृतो राज्ञा सत्कारार्हो विशाम्पते}


\twolineshloka
{रत्नानि भूरीण्यादाय सम्प्रतस्थे युधां पतिः}
{ततः सागरकुक्षिस्थान्म्लेच्छान्परमदारुणान्}


\threelineshloka
{पह्लवान्वर्बरांश्चैव किरातान्यवनाञ्शकान्}
{ततो रत्नान्यपादाय वशे कृत्वा च पार्थिवान्}
{न्यवर्तत कुरुश्रेष्ठो नकुलश्चित्रमार्गवित्}


\twolineshloka
{करमाणां सहस्राणि कोशं तस्य महात्मनः}
{ऊहर्दश महाराज कृच्छ्रादिव महाधनम्}


\twolineshloka
{इन्द्रप्रस्थगतं वीरमभ्येत्य स युधिष्ठिरम्}
{ततो माद्रीसुतः श्रीमान्धनं तस्मै न्यवेदयत्}


\twolineshloka
{एवं विजित्य नकुलो दिशं वरुणपालिताम्}
{प्रतीचीं वासुदेवेन निर्जितां भरतर्षभ}


\chapter{अध्यायः ३६}
\twolineshloka
{रक्षणाद्धर्मराजस्य सत्यस्य परिपालनात्}
{शत्रूणां क्षपणाच्चैव स्वकर्मनिरताः प्रजाः}


\twolineshloka
{बलीनां सम्यगादानाद्धर्मतश्चानुशासनात्}
{निकामवर्षी पर्जन्यः स्फीतो जनपदोऽभवत्}


\twolineshloka
{सर्वारम्भाः सुप्रवृत्ता गोरक्षा कर्षणं वणिक्}
{विशेषात्सर्वमेवैतत्सञ्जज्ञे राजकर्मणः}


\twolineshloka
{दस्युभ्यो वञ्चकेभ्यो वा राजन्प्रति परस्परम्}
{राजवल्लभतश्चैव नाश्रूयन्त मृषागिरः}


\twolineshloka
{अवर्षं चातिवर्षं च व्याधिपावकमूर्छनम्}
{सर्वमेतत्तदा नासीद्धर्मनित्ये युधिष्ठिरे}


\twolineshloka
{प्रियं कर्तुमुपस्थातुं बलिकर्म स्वभावजम्}
{अभिहर्तुं नृपा जग्मुर्नान्यैः कार्यैः कथञ्चनः}


\twolineshloka
{धर्मैर्धनागमैस्तस्य ववधे निचयो महान्}
{कर्तुं यस्य न शक्येन क्षयो वर्षशतैरपि}


\twolineshloka
{स्वकोष्ठस्य परीमाणं कोशस्य च महीपतिः}
{विज्ञाय राजा कौन्तेयो यज्ञायैव मनो दधे}


\twolineshloka
{सुहृदश्चैव ये सर्वे पृथक्च सहचाब्रुवन्}
{यज्ञकालस्तव विभो क्रियतामत्र साम्प्रतम्}


\twolineshloka
{अथैवं ब्रुवतामेव तेषामभ्याययौ हरिः}
{ऋषिः पुराणो वेदात्मा दृश्यश्चैव विजानताम्}


\twolineshloka
{जगतस्तस्थुषां श्रेष्ठः प्रभवश्चाप्ययश्च ह}
{भूतभव्यभवन्नाथः केशवः केशिसूदनः}


\twolineshloka
{प्राकारः सर्ववृष्णीनामापत्स्वभयदोऽरिहा}
{बलाधिकारे निक्षिप्य सम्यगानकदुन्दुभिम्}


\twolineshloka
{उच्चावचमुपादाय धर्मराजाय माधवः}
{धनौघं पुरुषव्याघ्रो बलेन महता वृतः}


\twolineshloka
{तं धनौघमपर्यन्तं रत्नसागरमक्षयम्}
{नादयन्रथघोषेण प्रविशेश पुरोत्तमम्}


\threelineshloka
{पूर्णमापूरयंस्तेषां द्विषच्छोकावहोऽभवत्}
{असूर्यमिव सूर्येण निवातमिव वायुना}
{कृष्णेन समुपेतेन जहृषे भारतं पुरम्}


\twolineshloka
{तं मुदाऽभिसमागम्य सत्कृत्य च यथाविधि}
{स पृष्ट्वा कुशलं चैव सुखासीनं युधिष्ठिरः}


\twolineshloka
{धौम्यद्वैपायनमुखैर्ऋत्विग्भिः पुरुषर्षभः}
{भीमार्जुनयमैश्चैव सहितः कृष्णमब्रवीत् ॥युधिष्ठिर उवाच}


\twolineshloka
{त्वत्कृते पृथिवी सर्वा मद्वशे कृष्ण वर्तते}
{धनं च बहु वार्ष्णेय त्वत्प्रसादादुपार्जितम्}


\twolineshloka
{सोऽहमिच्छामि तत्सर्वं विधिवद्देवकीसुत}
{उपयोक्तुं द्विजाग्र्येभ्यो हव्यवाहे च माधव}


\twolineshloka
{तदहं यष्टुमिच्छामि दाशार्ह सहितस्त्वया}
{अनुजैश्च महाबाहो तन्माऽनुज्ञातुमर्हसि}


\twolineshloka
{तद्दीक्षापय गोविन्द त्वमात्मानं महाभुज}
{त्वयीष्टवति दाशार्ह विपाप्मा भविता ह्यहम्}


\threelineshloka
{मां वाप्यभ्यनुजानीह सहैभिरनुजैर्विभो}
{अनुज्ञातस्त्वया कृष्ण प्राप्नुयां क्रतुमुत्तमम् ॥वैशम्पायन उवाच}
{}


\threelineshloka
{तं कृष्णःस प्रत्युवाचेदं बहूक्त्वा गुणविस्तरम्}
{त्वमेव राजशार्दूल रम्राडर्हो महाक्रतुम्}
{सम्प्राप्नुहि त्वया प्राप्ते कृतकृत्यास्ततो वयम्}


\threelineshloka
{यदस्वाभीप्सितं यज्ञं मयि श्रेयस्यवस्थिते}
{नियुङ्क्ष्व त्वं च मां कृत्ये सर्वं कर्तास्मि ते वचः ॥युधिष्ठिर उवाच}
{}


\threelineshloka
{सफलः कृष्ण सङ्कल्पः सिद्धिश्च नियता मम}
{यस्यमे त्वं हृषीकेश यथेप्सितमुपस्थितः ॥वैशम्पायन उवाच}
{}


\twolineshloka
{अनुज्ञातस्तु कृष्णेन पाण्डवो भ्रातृभिः सह}
{ईजितुं राजसूयेन साधनान्युपचक्रमे}


\twolineshloka
{ततस्त्वाज्ञापयामास पाण्डवोऽरिनिबर्हणः}
{सहदेवं युधां श्रेष्ठं मन्त्रिणश्चैव सर्वशः}


\twolineshloka
{अस्मिन्क्रतौ यथोक्तानि यज्ञाङ्गानि द्व्जातिभिः}
{यथोपकरणं सर्वं मङ्गलानि च सर्वशः}


\twolineshloka
{अधियज्ञांश्च सम्भारान्दौम्योक्तान्क्षिप्रमेव हि}
{समानयन्तु पुरुषा यथायोगं यथाक्रमम्}


\twolineshloka
{इन्द्रसेनो विशोकश्च पूरश्चार्जुनसारथिः}
{अन्नाद्याहरणे युक्ताः सन्तु मत्प्रियकाम्यया}


\twolineshloka
{सर्वकामाश्च कार्यन्तां रसगन्धसमन्विताः}
{मनोरथप्रीतिकरा द्विजानां कुरुसत्तम ॥वैशम्पायन उवाच}


\twolineshloka
{तद्वाक्यसमकालं च कृतं सर्वं न्यवेदयत्}
{सहदेवो युधां श्रेष्ठो धर्मराजो युधिष्ठिरे}


\twolineshloka
{ततो द्वैपायनो राजन्नृत्विजः समुपानयत्}
{वेदानिव महाभागान्साक्षान्मूर्तिमतो द्विजान्}


\twolineshloka
{स्वयं ब्रह्मत्वमकरोत्तस्य सत्यवतीसुतः}
{धनञ्जयानामृषभः सुसामा सामगोऽभवत्}


\twolineshloka
{याज्ञवल्क्यो बभूवाथ ब्रह्मिष्ठोऽध्वर्युसत्तमः}
{पैलो होता वसोः पुत्रो धौम्येन सहितोऽभवत्}


\twolineshloka
{एतेषां पुत्रवर्गाश्च शिष्याश्च भरतर्षभ}
{बभूवुर्होत्रगाः सर्वे वेदवेदाङ्गपारगाः}


\twolineshloka
{ते वाचयित्वा पुण्याहमूहयित्वा च तं विधिम्}
{शास्त्रोक्तं पूजयामासुस्तद्देवयजनं महत्}


\twolineshloka
{तत्र चक्रुरनुज्ञाताः शरणान्युत शिल्पिनः}
{गन्धवन्ति विशालानि वेश्मानीव दिवौकसाम्}


\twolineshloka
{तत आज्ञापयामास स राजा राजसत्तमः}
{सहदेवं तदा सद्यो मन्त्रिणं पुरुषर्षभः}


\twolineshloka
{आमन्त्रणार्थं दूतांस्त्वं प्रेषयस्वाशुगान्द्रुतम्}
{उपश्रुत्य वचो राज्ञः स दूतान्प्राहिणोत्तदा}


\twolineshloka
{आमन्त्रयध्वं राष्ट्रेषु ब्राह्मणान्भूमिपानथ}
{विनाश्च मान्याञ्शूद्रांश्च सर्वानानयतेति च ॥वैशम्पायन उवाच}


\twolineshloka
{ते सर्वान्पृथिवीपालान्पाण्डवेयस्य शासनात्}
{आमन्त्रयाम्बभूवुस्ते प्रेषयामास चापरान्}


\twolineshloka
{दूताश्च वाहनैर्जग्भू राष्ट्राणि सुबहून्यपि}
{ततो युधिष्ठिरो राजा प्रेषयामास पाण्डवम्}


\threelineshloka
{नकुलं हास्तिनपुरं भीष्माय भरतर्षभ}
{द्रोणाय धृतराष्ट्राय विदुराय कृपाय च}
{भ्रातृणां चैव सर्वेणां येऽनुरक्ता युधिष्ठिरे}


\twolineshloka
{ततस्ते तु यथाकालं कुन्तीपुत्रं युधिष्ठिरम्}
{दीक्षयाञ्चक्रिरे विप्रा राजसूयाय भारत}


\twolineshloka
{` ज्येष्ठामूले अमावास्यां मृगाजिनसमावृतः}
{रौरवाजिनसंवीतो नवनीताक्तदेहवान्'}


\twolineshloka
{दीक्षितः स तु धर्मात्मा धर्मराजो युधिष्ठिरः}
{जगाम यज्ञायतनं वृतो विप्रैः सहस्रशः}


\twolineshloka
{भातृभिर्ज्ञातिभिश्चैव सुहृद्भिः सचिवैः सह}
{क्षत्रियैश्च मनुष्येन्द्रैर्नानादेशसमागतैः}


\twolineshloka
{अमात्यैश्च नरश्रेष्ठो धर्मो विग्रहवानिव}
{आजग्मुर्ब्राह्मणास्तत्र विषयेभ्यस्ततस्ततः}


\twolineshloka
{सर्वविद्यासु निष्णाता वेदवेदाङ्गपारगाः}
{तेषामावसथांश्चक्रुर्धर्मराजस्य शासनात्}


\twolineshloka
{बह्वन्नाच्छादनैर्युक्तान्सगणानां पृथक् पृथक्}
{सर्वर्तुगुणसम्पन्नाञ्शिल्पिनोऽथ सहस्रशः}


\twolineshloka
{तेषु ते न्यवसन्राजन्ब्राह्मणा नृपसत्कृताः}
{कथयन्तः कथा बह्वीः पश्यन्तो नटनर्तकान्}


\twolineshloka
{भुञ्जतां चैव विप्राणां वदतां च महास्वनः}
{अनिशं श्रूयते तत्र मुदितानां महात्मनाम्}


\twolineshloka
{दीयतां दीयतामेषां भुज्यतां भुज्यतामिति}
{एवम्प्रकाराः सञ्जल्पाः श्रूयन्तेस्मात्र नित्यशः}


\twolineshloka
{गवां शतसहस्राणि शयनानां च भारत}
{रुक्मस्य योषितां चैव धर्मराजः पृथक् ददौ}


\twolineshloka
{प्रावर्ततैव यज्ञः स पाण्डवस्य महात्मनः}
{पृथिव्यामेकवीरस्य शक्रस्येव त्रिविष्टपे}


\chapter{अध्यायः ३७}
\twolineshloka
{तत आमन्त्रिता राजन्राजानः सत्कृतास्तदा}
{पुरेभ्यः प्रययुस्तेभ्यो विमानेभ्य इवामराः}


\twolineshloka
{ते वै दिग्भ्यः समापेतुः पार्थिवास्तत्र भारत}
{समादाय महार्हाणि रत्नानि विविधानि च}


\twolineshloka
{तच्छ्रुत्वा धर्मराजस्य यज्ञे यज्ञविदस्तदा}
{राजानः शतशस्तुष्टैर्मनोभिर्मनुर्षभ}


\twolineshloka
{बहु वित्तं समादाय विविधं पार्थिवा ययुः}
{द्रष्टुकामाः सभां चैव धर्मराजं च पाण्डवम्}


\twolineshloka
{स गत्वा हास्तिनपुरं नकुलः समितिञ्जयः}
{भीष्ममामन्त्रयाञ्चक्रे धृतराष्ट्रं च पाण्डवः}


\twolineshloka
{प्रयतः प्राञ्जलिर्भूत्वा भारतानानयत्तदा}
{धृतराष्ट्रं च भीष्मं च विदुरं च महामतिम्}


% Check verse!
दुर्योधनमुखांश्चैव भ्रातॄन्सर्वानथानयत्
\twolineshloka
{सत्कृत्यामन्त्रिताः सर्वे ह्याचार्यप्रमुखास्ततः}
{प्रययुः प्रीतमनसो यज्ञं ब्रह्मपुरस्सराः}


\threelineshloka
{धृतराष्ट्रश्च भीष्मश्च विदुरस्च महामतिः}
{दुर्योधनपुरोगाश्च भ्रातरः सर्व एव ते}
{गान्धारराजः सुबलः शकुनिश्च महाबलः}


\twolineshloka
{अचलो वृषकश्चैव कर्णश्च रथिनां वरः}
{तथा शल्यश्च बलवान्बाह्लिकश्च महाबलः}


\twolineshloka
{सोमदत्तोऽथ कौरव्यो भूरिर्भूरिश्रवाः शलः}
{अश्वत्थामा कृपो द्रोणः सैन्धवश्च जयद्रथः}


\twolineshloka
{यज्ञसेनः सपुत्रश्च साल्वश्च वसुधाधिपः}
{प्राग्ज्योतिषश्च नृपतिर्भगदत्तो महारथः}


\twolineshloka
{स तु सर्वैः सह म्लेच्छैः सागरानूपवासिभिः}
{पार्वतीयाश्च राजानो राजा चैव बृहद्बलः}


\twolineshloka
{पौण्ड्रको वासुदेवश्च वङ्गः कालिङ्गकस्तथा}
{आकर्षाः कुन्तलाश्चैव गालवाश्चान्ध्रकास्तथा}


\twolineshloka
{द्राविडाः सिंहलाश्चैव राजा काश्मीरकस्तथा}
{कुन्तिभोजो महातेजाः पार्थिवो गौरवाहनः}


\twolineshloka
{बाह्लिकाश्चापरे शूरा राजानः सर्व एव ते}
{विराटः सह पुत्राभ्यां मावेल्लश्च महाबलः}


\twolineshloka
{राजानो राजपुत्राश्च नानाजनपदेश्वराः}
{शिशुपालो महावीर्यः सह पुत्रेण भारत}


\twolineshloka
{आगच्छत्पाण्डवेयस्य यज्ञं समरदुर्मदः}
{रामश्चैवानिरुद्धश्च कङ्कश्च सहसारणः}


\twolineshloka
{गदप्रद्युम्नसाम्बाश्च चारुदेष्णश्च वीर्यवान्}
{उल्मुको निशठश्चैव वीरश्चाङ्गावहस्तथा}


\twolineshloka
{वृष्णयो निखिलाश्चान्ये समाजग्मुर्महारथाः}
{एते चान्ये च बहवो राजानो मध्यदेशजाः}


\twolineshloka
{आजग्मुः पाण्डुपुत्रस्य राजसूयं महाक्रतुम्}
{ददुस्तेषामावसथान्धर्मराजस्य शासनात्}


\twolineshloka
{बहुभक्ष्यान्वितान्राजन्दीर्घिकावृक्षशोभितान्}
{तथा धर्मात्मजः पूजां चक्रे तेषां महात्मनाम्}


\twolineshloka
{सत्कृताश्च यथोद्दिष्टाञ्जग्मुरावसथान्नृपाः}
{कैलासशिखरप्रख्यान्मनोज्ञान्द्रव्यभूषितान्}


\twolineshloka
{सर्वतः संवृतानुच्चैः प्राकारैः सुकृतैः सितैः}
{सुवर्णजालसंवीतान्मणिकुट्टिमभूषितान्}


\twolineshloka
{सुखारोहणसोपानान्महासनपरिच्छदान्}
{स्नग्दामसमवच्छन्नानुत्तमागुरुगन्धिनः}


\twolineshloka
{हंसेन्दुवर्णसदृशानायोजनसुदर्शनान्}
{असम्बाधान्समद्वारान्युतानुच्चावचैर्गुणैः}


\twolineshloka
{बहुधातुनिबद्धाङ्गान्हिमवच्छिखरानिव}
{विश्रान्तास्ते ततोऽपश्यन्भूमिपा भूरिदक्षिणम्}


\threelineshloka
{वृतं सदस्यैर्बहुभिर्धर्मराजं युधिष्ठिरम्}
{तत्सदः पार्थिवैः कीर्णं ब्राह्मणैश्च महर्षिभिः}
{भ्राजते स्म तदा राजन्नाकपृष्ठं यथाऽमरैः}


\chapter{अध्यायः ३८}
\twolineshloka
{पितामहं गुरुं चैव प्रत्युद्गम्य युधिष्ठिरः}
{अभिवाद्य ततो राजन्निदं वचनमब्रवीत्}


\twolineshloka
{भीष्मं द्रोणं कृपं द्रौणिं दुर्योधनविविंशती}
{अस्मिन्यज्ञे भवन्तो मामनुगृह्णन्तु सर्वशः}


\twolineshloka
{इदं वः सुमहच्चैव यदिहास्ति धनं मम}
{प्रणयन्तु भवन्तो मां यथेष्टमभिमन्त्रितः}


\twolineshloka
{एवमुक्त्वा स तान्सर्वान्दीक्षितः पाण्डवाग्रजः}
{युयोज स यथायोगमधिकारेष्वनन्तरम्}


\twolineshloka
{`पङ्क्त्यारोपणकार्ये च उच्छिष्टापनये पुनः}
{भोजनावेक्षणे चैव युयुत्सुं समयोजयत्'}


\threelineshloka
{भक्ष्यभोज्याधिकारेषु दुःशासनमयोजयत्}
{परिग्रहे ब्राह्मणानामश्वत्थामानमुक्तवान्}
{}


\twolineshloka
{राज्ञां तु प्रतिपूजार्थं सञ्जयं न्ययोजयत्}
{कृताकृतपरिज्ञाने भीष्णद्रोणौ महामती}


\twolineshloka
{हिरण्यस्य सुवर्णस्य रत्नानां चान्ववेक्षणे}
{दक्षिणानां च वै दाने कृपं राजा न्ययोजयत्}


\threelineshloka
{तथाऽऽन्यान्पुरुषव्याघ्रांस्तस्मिंस्तस्मिन्न्ययोजयत्}
{बाह्लिको धृतराष्ट्रश्च सोमदत्तो जयद्रथः}
{नकुलेन समानीताः स्वामिवत्तत्र रेमिरे}


\twolineshloka
{क्षत्ता व्ययकरस्त्वासीद्विदुरः सर्वधर्मवित्}
{दुर्योधनस्त्वर्हणानि प्रतिजग्राह सर्वशः}


\twolineshloka
{`कुन्ती साध्वी च गान्धारी स्त्रीणां कुर्वन्ति चार्चनम् ॥अन्याः सर्वाः स्नुषास्तासां सन्देशं यान्तु माचिरम्}
{तिष्ठेत्कृष्णान्तिके सोयमर्जुनः कार्यसिद्धये'}


\twolineshloka
{चरणक्षालने कृष्णो ब्राह्मणानां स्वयं ह्यभूत्}
{सर्वलोकसमावृत्तः पिप्रीषुः फलमुत्तमम्}


\twolineshloka
{द्रष्टुकामः सभां चैव धर्मराजं यधिष्ठिरम्}
{न कश्चिदाहरत्तत्र सहस्रावरमर्हणम्}


\twolineshloka
{रत्नैश्च बहुभिस्तत्र धर्मराजमवर्धयत्}
{कथं तु मम कौरव्यो रत्नदानैः समाप्नुयात्}


\twolineshloka
{यज्ञमित्येव राजानः स्पर्धमाना ददुर्धनम्}
{भवनैः सविमानाग्रैः सोदर्कैर्बलसंवृतैः}


\twolineshloka
{लोकराजविमानैश्च ब्राह्मणावसथैः सह}
{कृतैरावसथैर्दिव्यैर्विमानप्रतिमैस्तथा}


\threelineshloka
{विचित्रै रत्ववद्भिश्च ऋद्ध्या परमया युतैः}
{राजभिश्च समावृत्तैरतीव श्रीसमृद्धिभिः}
{अशोभत सदो राजन्कौन्तेयस्य महात्मनः}


\twolineshloka
{ऋद्ध्या तु वरुणं देवं स्पर्धमानो युधिष्ठिरः}
{षडग्निनाथ यज्ञेन सोऽयजद्दक्षिणावता}


\threelineshloka
{सर्वाञ्जनान्सर्वकामैः समृद्धैः समतर्पयत्}
{अन्नवान्बहुभक्ष्यश्च भुक्तवज्जनसंवृतः}
{रत्नोपहारसम्पन्नो बभूव स समागमः}


\twolineshloka
{इडाज्यहोमाहुतिभिर्मन्त्रशिक्षाविशारदैः}
{तस्मिन्हि ततृपुर्देवास्तते यज्ञे महर्षिभिः}


\twolineshloka
{यथा देवास्तथा विप्रा दक्षिणान्नमहाधनैः}
{ततृपुः सर्ववर्णाश्च तस्मिन्यज्ञे मुदाऽन्विताः}


\chapter{अध्यायः ३९}
\twolineshloka
{ततोऽभिषेचनीयेऽह्नि ब्राह्मणा राजभिः सह}
{अन्तर्वेदीं प्रविविशुः सत्कारार्हा महर्षयः}


\twolineshloka
{नारदप्रमुखास्तस्यामन्तर्वेद्यां महात्मनः}
{समासीनाः शुशुभिरे सहराजर्षिभिस्तदा}


\twolineshloka
{समेता ब्रह्मभवने देवा देवर्षयस्तथा}
{कर्मान्तरमुपासन्तो जजल्पुरमितौजसः}


\twolineshloka
{एवमेतन्न चाप्येवमेवं चैतन्न चान्यथा}
{इत्यूचुर्बहवस्तत्र वितण्डां वै परस्परम्}


\twolineshloka
{कृशानर्थांस्ततः केचिदकृशांस्तत्र कुर्वते}
{अकृशांश्च कृशांश्चक्रुर्हेतुभिः शास्त्रनिश्चयैः}


\twolineshloka
{तत्र मेधाविनः केचिदर्थमन्यैरुदीरितम्}
{विचिक्षिपुर्यथा श्येना नभोगतमिवामिषम्}


\twolineshloka
{केचिद्धर्मार्थकुशलाः केचित्तत्र महाव्रताः}
{रेमिरे कथयन्तश्च सर्वभाष्यविदां वराः}


\twolineshloka
{सा वेदिर्वेदसम्पन्नैर्देवद्विजमहर्षिभिः}
{आबभासे समाकीर्णा नक्षत्रैर्द्यौरिवायता}


\twolineshloka
{न तस्यां सन्निधौ शूद्रः कश्चिदासीन्न चाव्रती}
{अन्तर्वेद्यां तदा राजन्युधिष्ठिरनिवेशने}


\twolineshloka
{तां तु लक्ष्मीवतो लक्ष्मीं तदा यज्ञविधानजाम्}
{तुतोष नारदः पश्यन्धर्मराजस्य धीमतः}


\twolineshloka
{अथ चिन्तां समापेदे स मुनिर्मनुजाधिप}
{नारदस्तु तदा पश्यन्सर्वक्षत्रसमागमम्}


\twolineshloka
{सस्मार च पुरावृत्तां कथां तां पुरुषर्षभ}
{अंशावतरणे याऽसौ ब्रह्मणो भवनेऽभवत्}


\twolineshloka
{देवानां सङ्गमं तं तु विज्ञाय कुरुनन्दन}
{नारदः पुण्डरीकाक्षं सस्मार मनसा हरिम्}


\twolineshloka
{साक्षात्स विबुधारिघ्नः क्षत्रे नारायणो विभुः}
{प्रतिज्ञां पालयंश्चेमां जातः परपुरञ्जयः}


\twolineshloka
{सन्दिदेश पुरा योऽसौ विबुधान्भूतकृत्स्वयम्}
{अन्योन्यमभिनिघ्नन्तः पुनर्लोकानवाप्स्यथ}


\twolineshloka
{इति नारायणः शम्भुर्भगवान्भूतभावनः}
{आदिश्य विबुधान्सर्वानजायत यदुक्षये}


\twolineshloka
{क्षितावन्धकवृष्णीनां वंशे वंशभृतां वरः}
{परया शुशुभे लक्ष्म्या नक्षत्राणामिवोडुराट्}


\twolineshloka
{यस्य बाहुबलं सेन्द्राः सुराः सर्व उपासते}
{सोयं मानुषवन्नाम हरिरास्तेऽरिमर्दनः}


\twolineshloka
{अहो बत महद्भूतं स्वयम्भूर्यदिदं स्वयम्}
{आदास्यति पुनः क्षत्रमेवं बलसमन्वितम्}


\twolineshloka
{इत्येतां नारदश्चिन्तां चिन्तयामास सर्ववित्}
{हरिं नारायणं ज्ञात्वा यज्ञैरीज्यं तमीश्वरम्}


\twolineshloka
{तस्मिन्धर्मविदां श्रेष्ठो धर्मराजस्य धीमतः}
{महाध्वरे महाबुद्धिस्तस्थौ स बहुमानतः}


\twolineshloka
{`ततः समुदिता मुख्यैर्गुणैर्गुणवतां वराः}
{बहवो भावितात्मानः पृथक्पृथगरिन्दमाः}


\twolineshloka
{आत्मकृत्यमिति ज्ञात्वा पाञ्चालास्तत्र सर्वशः}
{समीयुर्वृष्णयश्चैव तदाऽनीकाग्रहारिणः}


\twolineshloka
{सदाराः सजनामात्या वहन्तो रत्नसञ्चयान्}
{विकृष्टत्वाच्च देशस्य गुरुभारतया च ते}


\twolineshloka
{ययुः प्रमुदिताः पश्चाद्भगवन्तं समन्वयुः}
{बलशेषं समुदितं परिगृह्य समन्ततः}


\twolineshloka
{अजश्चक्रायुधः शौरिरमित्रगणमर्दनः}
{बलाधिकारे निक्षिप्य संमान्यानकदुन्दुभिम्}


\twolineshloka
{सम्प्रायाद्यादवश्रेष्ठो जयमाने युधिष्ठिरे}
{उच्चावचमुपादाय धर्मराजाय माधवः}


\twolineshloka
{धनौघं पुरतः कृत्वा खाण्डवप्रस्थमाययौ}
{तत्र यज्ञगतान्पश्यंश्चैद्यवर्गसमागतान्}


\twolineshloka
{भूमिपालगणान्सर्वान्सप्रभानिव तोयदान्}
{मेघकायान्निवसतो यूथपानिव यूथपः}


\twolineshloka
{बलिनः सिंहसङ्काशान्महीमावृत्य तिष्ठतः}
{ततो जनौघसम्बाधं राजसागरमव्ययम्}


\twolineshloka
{नादयन्रथघोषेण ह्युपायान्मधुसूदनः}
{असूर्यमिव सूर्येण निवातमिव वायुना}


\twolineshloka
{कृष्णेन समुपेतेन जहर्षे भारतं पुरम्}
{ब्राह्मणक्षत्रियाणां तु पूजार्थं ह्यर्थधर्मवित्}


\twolineshloka
{सहदेवो विशेषज्ञो माद्रीपुत्रः कृतोऽभवत्}
{भगवन्तं तु भूतानां भास्वन्तमिव तेजसा}


\twolineshloka
{विशन्तं यज्ञभूमिं तां सितस्यावरजं प्रभुम्}
{तेजोराशिमृषिं विप्रमदृश्यं वै विजानताम्}


\twolineshloka
{वयोधिकानां वृद्धानां मार्गमात्मनि तिष्ठताम्}
{जगतस्तस्थुषश्चैव प्रभवाप्ययमच्युतम्}


\twolineshloka
{अनन्तमन्तं शत्रूणाममित्रगणमर्दनम्}
{प्रभवं सर्वभूतानामापत्स्वभयमच्युतम्}


\twolineshloka
{भविष्यं भावनं भूतं द्वारवत्यामरिन्दमम्}
{स दृष्ट्वा कृष्णमायान्तं प्रतिपूज्यामितौजसम्}


\twolineshloka
{यथार्हं केशवे वृत्तिं प्रत्यपद्यत पाण्डवः}
{ज्यैष्ठ्यकानिष्ठ्यसंयोगं सम्प्रधार्य गुणागुणैः}


\threelineshloka
{आरिराधयिषुर्धर्मः पूजयित्वा द्विजोत्तमान्}
{महदादित्यसङ्काशमासनं च जगत्पतेः}
{ददौ नासादितं कैश्चित्तस्मिन्नुपविवेश सः'}


\twolineshloka
{ततो भीष्मोऽब्रवीद्राजन्धर्मराजं युधिष्ठिरम्}
{क्रियतामर्हणं राज्ञां यथार्हमिति भारत}


\twolineshloka
{आचार्यमृत्विजं चैव संयुजं च युधिष्ठिर}
{स्नातकं च प्रियं प्राहुः षडर्घार्हान्नृपं तथा}


\twolineshloka
{एतानर्घ्यानभिगतानाहुः संवत्सरोषितान्}
{त इमे कालपूगस्य महतोऽस्मानुपागताः}


\twolineshloka
{एषामेकैकशो राजन्नर्घ आनीयतामिति}
{अथ तैषां वरिष्ठाय समर्थायोपनीयताम् ॥युधिष्ठिर उवाच}


\twolineshloka
{कस्मै भवान्मन्यतेऽर्घमेकस्मै कुरुनन्दन}
{उपनीयमानं युक्तं च तन्मे ब्रूहि पितामह ॥वैशम्पायन उवाच}


\twolineshloka
{ततो भीष्मः शान्तनवो बुद्ध्या निश्चित्य वीर्यवान्}
{वार्ष्णेयं मन्यते कृष्णमर्हणीयतमं भुवि ॥भीष्ण उवाच}


\twolineshloka
{एष ह्येषां समस्तानां तेजोबलपराक्रमैः}
{मध्ये तपन्निवाभाति ज्योतिषामिव भास्करः}


\twolineshloka
{असूर्यमिव सूर्येण निर्वातमिव वायुना}
{भासितं ह्लादितं चैव कृष्णेनेदं सदो हि नः}


\twolineshloka
{तस्मै भीष्माभ्यनुज्ञातः सहदेवः प्रतापवान्}
{उपजह्रेऽथ विधिवद्वार्ष्णेयायार्घ्यमुत्तमम्}


\twolineshloka
{`गामर्घ्यं मधुपर्कं च ह्यानीयोपाहरत्तदा}
{एतस्मिन्नन्तरे राजन्निदमासीत्तदाऽद्भुतम्}


\twolineshloka
{तां दृष्ट्वा क्षत्रियाः सर्वे पूजां कृष्णस्य भूयसीम्}
{सम्प्रेक्ष्यान्योन्यमासीना हृदयैस्तामधारयन्'}


\twolineshloka
{प्रतिजग्राह तां कृष्णः शास्त्रदृष्टेन कर्मणा}
{शिशुपालस्तु तां पूजां वासुदेवे न चक्षमे}


\twolineshloka
{उपालभ्य स भीष्मं च धर्मराजं च संसदि}
{अवाक्षिपद्वासुदेवं चेदिराजो महाबलः}


\twolineshloka
{`तेषामाकारभावज्ञः सहदेवो न चक्षमे}
{मानिनां बलिनां राज्ञां पुरुः सन्दर्शिते पदे}


\twolineshloka
{पुष्पवृष्टिर्महत्यासीत्सहदेवस्य मूर्धनि}
{जन्मप्रभृति वृष्णीना सुनीथः शत्रुरब्रवीत्}


\twolineshloka
{प्रष्टा वियोनिजो राजा प्रतिवक्ता नदीसुतः}
{प्रतिग्रहीता गोपालः प्रदाता च वियोनिजः}


\twolineshloka
{सदस्या मूकवत्सर्वे आसतेऽत्र किमुच्यते}
{इत्युक्त्वा स विहस्याशु पाण्डुं पुनरब्रवीत्}


\twolineshloka
{अतिपश्यसि वा सर्वान्न वा पश्यसि पाण्डव}
{तिष्ठत्स्वन्येषु पूज्येषु गोपमर्चितवानसि}


\twolineshloka
{एते चैवोभये तात कार्यस्य तु विनाशके}
{अतिदृष्टिरदृष्टिर्वा तयोः किं त्वं समास्थितः'}


\chapter{अध्यायः ४०}
\twolineshloka
{नायमर्हति वार्ष्णेयस्तिष्ठत्स्विह महात्मसु}
{महीपतिषु कौरव्य राजवत्पार्थिवार्हणम्}


\twolineshloka
{नायं युक्तः समाचारः पाण्डवेषु महात्मसु}
{यत्कामाद्देवकीपुत्रं पाण्डवार्चितवानसि}


\twolineshloka
{बाला यूयं न जानीध्वं धर्मः सूक्ष्मो हि पाण्डवाः}
{अयं तत्राभ्यतिक्रान्तो ह्यापगेयोऽल्पदर्शनः}


\twolineshloka
{त्वादृशो धर्मयुक्तो हि कुर्वाणः प्रियकाम्यया}
{भवत्यभ्यधिकं भीष्मो लोकेष्ववमतः सताम्}


\twolineshloka
{कथं ह्यराजा दाशार्हो मध्ये सर्वमहीक्षिताम्}
{अर्हणामर्हति तथा यथा युष्माभिरर्चितः}


\twolineshloka
{अथवा मन्यसे कृष्णं स्थविरं कुरुपुङ्गवः}
{वसुदेवे स्थिते वृद्धे कथमर्हति तत्सुतः}


\twolineshloka
{अथवा वासुदेवोऽपि प्रियकामोऽनुवृत्तवान्}
{द्रुपदे तिष्ठति कथं माधवोऽर्हति पूजनम्}


\twolineshloka
{आचार्यं मन्यसे कृष्णमथवा कुरुनन्दन}
{द्रोणे तिष्ठति वार्ष्णेयं कस्मादर्चितवानसि}


\twolineshloka
{ऋत्विजं मन्यसे कृष्णमथवा कुरुनन्दन}
{द्वौपायने स्थिते वृद्धे कथं कृष्णोऽर्चितस्त्वया}


\twolineshloka
{भीष्मे शान्तनवे राजन्स्थिते पुरुषसत्तमे}
{स्वच्छन्दमृत्युके राजन्कथं कृष्णोऽर्चितस्त्वया}


\twolineshloka
{अश्वत्थाम्नि स्थिते वीरे सर्वशास्त्रविशारदे}
{कथं कृष्णस्त्वया राजन्नर्चितः कुरुनन्दन}


\twolineshloka
{दुर्योधने च राजेन्द्रे स्थिते पुरुषसत्तमे}
{कृपे च भारताचार्ये कथं कृष्णस्त्वयाऽर्चितः}


\twolineshloka
{द्रुमं कम्पुरुषाचार्यमतिक्रम्य तथाऽर्चितः}
{भीष्मके चैव दुर्धर्षे पाण्डुवत्कृतलक्षणे}


\twolineshloka
{नृपे च रुक्मिणि श्रेष्ठे एकलव्ये तथैव च}
{शल्ये मद्राधिपे चैव कथं कृष्णस्त्वयार्चितः}


\twolineshloka
{अयं च सर्वराज्ञां वै बलश्लाघी महाबलः}
{जामदग्न्यस्य दयितः शिष्यो विप्रस्य भारत}


\twolineshloka
{येनात्मबलमाश्रित्य राजानो युधि निर्जिताः}
{तं च कर्णमतिक्रम्य कथं कृष्णस्त्वयार्चितः}


\twolineshloka
{नैवर्त्विङ्नैव चाचार्यो न राजा मधुसूदनः}
{अर्चितश्च कुरुश्रेष्ठ किमन्यत्प्रियकाम्यया}


\twolineshloka
{अथवाऽभ्यर्चनीयोऽयं युष्माकं मधुसूदनः}
{किं राजभिरिहानीतैरवमानाय भारत}


\twolineshloka
{वयं तु न भयादस्य कौन्तेयस्य महात्मनः}
{प्रयच्छामः करान्सर्वे न लोभान्न च सान्त्वनात्}


\twolineshloka
{अस्य धर्मप्रवृत्तस्य पार्थिवत्वं चिकीर्षतः}
{करानस्मै प्रयच्छामः सोऽयमस्मान्न मन्यते}


\twolineshloka
{किमन्यदवमनानाद्धे यदेनं राजसंसदि}
{अप्राप्तलक्षणं कृष्णमर्घ्येणार्चितवानसि}


\twolineshloka
{अकस्माद्धर्मपुत्रस्य धर्मात्मेति यशो गतम्}
{को हि धर्मच्युते पूजामेवं युक्तां नियोजयेत्}


\twolineshloka
{योयं वृष्णिकुले जातो राजानं हतवान्पुरा}
{जरासन्धं महात्मानमन्यायेन दुरात्मवान्}


\twolineshloka
{अद्य धर्मात्मता चैव व्यपकृष्टा युधिष्ठिरात्}
{दर्शितं कृपणत्वं च कृष्णेऽर्घ्यस्य निवेदनात्}


\twolineshloka
{यदि भीताश्च कौन्तेयाः कृपणाश्च तपस्विनः}
{ननु त्वयाऽपि बोद्धव्यं यां पूजां माधवार्हसि}


\twolineshloka
{अथवा कृपणैरेतामुपनीतां जनार्दन}
{पूजामनर्हः कस्मात्त्वमभ्यनुज्ञातवानसि}


\twolineshloka
{अयुक्तामात्मनः पूजां त्वं पुनर्बहुमन्यसे}
{हविषः प्राप्य निष्यन्दं प्राशिता श्वेव निर्जने}


\twolineshloka
{न त्वं पार्थिवेन्द्राणामपमानः प्रयुज्यते}
{त्वामेव कुरवो व्यक्तं प्रलम्भन्ते जनार्दन}


\twolineshloka
{क्लीबे दारक्रिया यादृगन्धे वा रूपदर्शनम्}
{अराज्ञो राजवत्पूजा तथा ते मधुसूदन}


\twolineshloka
{दृष्टो युधिष्ठिरो राजा दृष्टो भीष्मश्च यादृशः}
{वासुदेवोऽप्ययं दृष्टः सर्वमेतद्यथातथम्}


\twolineshloka
{इत्युक्त्वा शिशुपालस्तानुत्थाय परमासनात्}
{निर्ययौ सदसस्तस्मात्सहितो राजभिस्तदा}


\chapter{अध्यायः ४१}
\twolineshloka
{ततो युधिष्ठिरो राजा शिशुपालमुपाद्रवत्}
{उवाच चैनं मधुरं सान्त्वपूर्वमिदं वचः}


\twolineshloka
{नेदं युक्तं महीपाल यादृशं वै त्वमुक्तवान्}
{अधर्मश्च परो राजन्पारुष्यं च निरर्थकम्}


\twolineshloka
{न हि धर्मं परं जातु नावबुध्येत पार्थिवः}
{भीष्मः शान्तनवस्त्वेनं मावमंस्थास्त्वमन्यथा}


\twolineshloka
{पश्य चैतान्महीपालांस्त्वत्तो वृद्धतरान्बहून्}
{मृष्यन्ते चार्हणां कृष्णे तद्वत्वं क्षन्तुमर्हसि}


\threelineshloka
{वेद तत्त्वेन कृष्णं हि भीष्णश्चेदिपते भृशम्}
{न ह्येनं त्वं तथा यथैनं वेद कौरवः ॥भीष्म उवाच}
{}


\twolineshloka
{नास्मै देयो ह्यनुनयो नायमर्हति सान्त्वनम्}
{लोकवृद्धतमे कृष्णे योऽर्हणां नाभिमन्यते}


\twolineshloka
{क्षत्रियः क्षत्रियं जित्वा रणे रणकृतां वरः}
{यो मुञ्चति वशे कृत्वा गुरुर्भवति तस्य सः}


\twolineshloka
{अस्यां हि समितौ राज्ञामेकमप्यजितं युधि}
{न पश्यामि महीपालं सात्वतीपुत्रतेजसा}


\twolineshloka
{न हि केवलमस्माकमयमर्च्यतमोऽच्युतः}
{त्रयाणामपि लोकानामर्चनीयो महाभुजः}


\twolineshloka
{कृष्णेन हि जिता युद्धे बहवः क्षत्रियर्षभाः}
{जगत्सर्वं च वार्ष्णेये निखिलेन प्रतिष्ठितम्}


\twolineshloka
{तस्मात्सत्स्वपि वृद्धेषु कृष्णमर्चाम नेतरान्}
{एवं वक्तुं न चार्हस्त्वं मा तेऽभूद्बुद्धिरीदृशी}


\threelineshloka
{ज्ञानवृद्धा मया राजन्बहवः पर्युपासिताः}
{`यस्य राजन्प्रभावज्ञाः पुरा सर्वे च रक्षिताः'}
{तेषां कथयतां शौरेरहं गुणवतो गुणान्}


\twolineshloka
{समागतानामश्रौषं बहून्बहुमतान्सताम्}
{कर्माण्यपि च यान्यस्य जन्मप्रभृति धीमतः}


\twolineshloka
{बहृशः कथ्यमानानि नरैर्भूयः श्रुतानि मे}
{न केवलं वयं कामाच्चेदिगज जनार्दनम्}


\twolineshloka
{न सम्बन्धं पुरस्कृत्य कृतार्थं वा कथञ्चन}
{अर्चामहेऽर्चितं सद्भिर्भुवि भूतसुखावहम्}


% Check verse!
यशः शौर्यं जयं चास्य विज्ञायार्चां प्रयुञ्ज्महेन च कश्चिदिहास्माभिः सुवालोप्यपरीक्षितः
\twolineshloka
{गुणैर्वृद्धानतिक्रम्य हरिरर्च्यतमो मतः}
{ज्ञानवृद्धो द्विजातीनां क्षत्रियाणां बलाधिकः}


\twolineshloka
{वैश्यानां धान्यधनवाञ्शूद्राणामेव जन्मतः}
{पूज्यतायां च गोविन्दे हेतू द्वावपि संस्थितौ}


\twolineshloka
{वेदवेदाङ्गविज्ञानं बलं चाभ्यधिकं तथा}
{नृणां लोके हि कोऽन्योस्ति विशिष्टः केशवादृते}


\twolineshloka
{दानं दाक्ष्यं श्रुतं शौर्यं ह्रीः कीर्तिर्बुद्धिरुत्तमा}
{संनतिः श्रीर्धृतिस्तुष्टिः पुष्टिश्च नियताऽच्युते}


\twolineshloka
{तमिमं लोकसम्पन्नमाचार्यं पितरं गुरुम्}
{अर्घ्यमर्चितमर्चामः सर्वे सङ्क्षन्तुमर्हथ}


\twolineshloka
{ऋत्विग्गुरुर्विवाह्यश्च स्नातको नृपतिः प्रियः}
{सर्वमेतद्धृषीकेशस्तस्मादभ्यर्चितोऽच्युतः}


\twolineshloka
{कृष्ण एव हि लोकानामुत्पत्तिरपि चाप्ययः}
{कृष्णस्य हि कृते विश्वमिदं भूतं चराचरम्}


\twolineshloka
{एष प्रकृतिरव्यक्ता कर्ता चैव सनातनः}
{परश्च सर्वभूतेभ्यस्तस्मात्पूज्यतमोऽच्युतः}


\twolineshloka
{बुद्धिर्मनो महद्वायुस्तेजोऽभः खं मही च या}
{चतुर्विधं च यद्भूतं सर्वं कृष्णे प्रतिष्ठितम्}


\twolineshloka
{आदित्यश्चन्द्रमाश्चैव नक्षत्राणि ग्रहाश्च ये}
{दिशश्च विदिशश्चैव सर्वं कृष्णे प्रतिष्ठितम्}


\twolineshloka
{` एष रुद्रश्च सर्वात्मा ब्रह्मा चैव सनातनः}
{अक्षरं क्षररूपेण मानुषत्वमुपागतः'}


\twolineshloka
{अग्निहोत्रमुखा वेदा गायत्री च्छन्दसां मुखम्}
{राजा मुखं मनुष्याणां नदीनां सागरो मुखम्}


\twolineshloka
{नक्षत्राणां मुखं चन्द्र आदित्यस्तेजसां मुखम्}
{पर्वतानां मुखं मेरुर्गरुडः पततां मुखम्}


\twolineshloka
{ऊर्ध्वं तिर्यगधश्चैव यावती जगतो गतिः}
{सदेवकेषु लोकेषु भगवान्केशवो मुखम्}


\twolineshloka
{अयं तु पुरुषो बालः शिशुपालो न बुध्यते}
{सर्वत्र सर्वदा कृष्णं तस्मादेवं प्रभाषते}


\twolineshloka
{यो हि धर्मं विचिनुयादुत्कृष्टं मतिमान्नरः}
{स वै पश्येद्यथा धर्मं न तथा चेदिराडयम्}


\twolineshloka
{सवृद्धबालेष्वथवा पार्थिवेषु महात्मसु}
{को नार्हं मन्यते कृष्णं को वाप्येनं न पूजयेत्}


\twolineshloka
{अथैनां दुष्कृतां पूजां शिशुपालो व्यवस्यति}
{दुष्कृतायां यथान्यायं तथाऽयं कर्तुमर्हति}


\chapter{अध्यायः ४२}
\twolineshloka
{गाङ्गेयेनैवमुक्तस्तु शिशुपालश्चुकोप तम्}
{क्रुद्धं सुनीथं दृष्ट्वाऽथ सहदेवोऽब्रवीत्तदा}


\twolineshloka
{मतिपूर्वमिदं सर्वं चेदिराज मया कृतम्}
{तन्मे निगदतस्तत्त्वं कारणादत्र मे शृणु}


\twolineshloka
{स पार्थिवानां सर्वेषां गुरुः कृष्णोऽपरो न हि}
{तस्मादभ्यर्चितोऽस्माभिः सर्वे संमन्तुमर्हथ}


\twolineshloka
{यो वा सहते कश्चिद्राज्ञां सबलवाहनः}
{क्षिप्रं युद्धाय निर्यातु तस्य मूर्ध्न्याहितं पदम्}


\twolineshloka
{एवमुक्तो मया हेतुरुत्तरं प्रब्रवीतु मे}
{वैशम्पायन उवाच ॥ततो न व्याजहारैषां कश्चिद्बुद्धिमतां सताम्}


\twolineshloka
{मानिनां बलिनां राज्ञां मध्ये सन्दर्शिते पदे}
{एवमुक्ते सुनीथस्य सहदेवेन केशवे}


\twolineshloka
{स्वभावरक्ते नयने कोपाद्रक्ततरे कृते}
{तस्य कोपं समुद्भूतं ज्ञात्वा भीष्मः प्रतापवान्}


\twolineshloka
{आचचक्षे पुनस्तस्मै कृष्णस्यैवोत्तरान्गुणान्}
{स सुनीथं समामन्त्र्य तांश्च सर्वान्महीक्षितः}


\threelineshloka
{उवाच वदतां श्रेष्ठं इदं मतिमतां वरः}
{सहदेवेन राजानो यदुक्तं केशवं प्रति}
{तत्तथेति विजानीध्वं भूयश्चात्र विबोधत}


\chapter{अध्यायः ४३}
\twolineshloka
{ततो भीष्मस्य तच्छ्रुत्वा वचः काले युधिष्ठिरः}
{ज्ञापनार्थाय सर्वेषां भीष्मं पुनरथाब्रवीत्}


\twolineshloka
{विस्तरेणास्य देवस्य कर्माणीच्छामि सर्वशः}
{श्रोतुं भगवतस्तानि प्रब्रवीहि पितामह}


\twolineshloka
{कर्मणामानुपूर्वा च प्रादुर्भावाश्च ये विभोः}
{यथा च प्रकृतिः कृष्णे तन्मे ब्रूहि पितामह}


\twolineshloka
{एवमुक्तस्तदा भीष्मः प्रोवाच भरतर्षभ}
{युधिष्ठिरममित्रघ्नं तस्मिन्त्राजसमागमे}


\twolineshloka
{समक्षं वासुदेवस्य देवस्येव शतक्रतोः}
{कर्माण्यसुकराण्यन्यैराचचक्षे जनाधिप}


\twolineshloka
{शृण्वतां पार्थिवानां च धर्मराजस्य चान्तिके}
{इदं मतिमतां श्रेष्ठः कृष्णं प्रति विशाम्पते}


\twolineshloka
{नाम्नैवामन्त्र्य राजेन्द्र चेदिराजमरिन्दमम्}
{भीमकर्मा ततो भीष्णो भूयः स इतमब्रवीत्}


\threelineshloka
{करूणामपि राजानं युधिष्ठिरमभाषत}
{भीष्ण उवाच}
{वर्तमानामतीतां च शृणु राजन्युधिष्ठिर}


\twolineshloka
{ईश्वरस्योत्तमस्यैनां कर्मणां गहनां गतिम्}
{अव्यक्तो व्यक्तलिङ्गस्थो य एष भगवान्प्रभुः}


\twolineshloka
{पुरा नारायणो देवः स्वयम्भूः प्रपितामहः}
{सहस्रशीर्षः पुरुषो ध्रवोऽनन्तः सनातनः}


\twolineshloka
{सहस्रास्यः सहस्राश्चः सहस्रचरणो विभुः}
{सहस्रवाहुः सर्वज्ञो देवो नामसहस्रवान्}


\twolineshloka
{सहस्रमुकुटो देवो विश्वरूपो महाद्युतिः}
{अनेकवर्णो देवादिरव्यक्ताद्वै परे स्थितः}


\twolineshloka
{असृजत्सलिलं पूर्वं स च नारायणः प्रभुः}
{ततस्तु भगवांस्तोये ब्रह्माणमसृजत्स्वयम्}


\threelineshloka
{ब्रह्मा चतुर्मुखो लोकान्सर्वांस्तानसृजत्स्वयम्}
{आदिकाले पुरा ह्येवं सर्वलोकस्य चोद्भवः}
{पुरा यः प्रलये प्राप्ते नष्टे स्थावरजङ्गमे}


\twolineshloka
{ब्रह्मादिषु प्रलीनेषु नष्टे लोके चराचरे}
{आभूतसम्प्लवे प्राप्ते प्रलीने प्रकृतौ महान्}


\twolineshloka
{एकस्मिष्ठति सर्वात्मा स तु नारायणः प्रभुः}
{नारायणस्य चाङ्गानि सर्वदैवानि भारत}


\twolineshloka
{शिरस्तस्य दिवं राजन्नाभिः खं चरणौ मही}
{अश्विनौ कर्णयोर्देवौ चक्षुषी शशिभास्करौ}


\twolineshloka
{इन्द्रवैश्वानरौ देवौ मुखं तस्य महात्मनः}
{अन्यानि सर्वदैवानि सर्वाङ्गानि महात्मनः}


\twolineshloka
{सर्वं चापि हरौ संस्थं सूत्रे मणिगणा इव}
{आभूतसम्प्लवान्तेऽथ दृष्ट्वा सर्वं तमोन्वितम्}


\twolineshloka
{नारायणो महायोगी सर्वज्ञः परमात्मवान्}
{ब्रह्मभूतस्तदात्मानं ब्रह्मणमसृजत्स्वयम्}


\twolineshloka
{सोऽध्यक्षः सर्वभूतानां प्रभूतप्रभवोऽच्युतः}
{सनत्कुमारं रुद्रं च सप्तर्षीश्च तपोधनात्}


\twolineshloka
{सर्वमेवासृजद्ब्रह्मा तथा लोकांस्तथा प्रजाः}
{ते च तद्व्यसृजंस्तत्र प्राप्तकाले युधिष्ठिर}


\twolineshloka
{तेभ्योऽभवन्महात्मभ्यो बहुधा ब्रह्म शाश्वतम्}
{कल्पानां बहुकोट्यश्चसमतीतास्तु भारत}


\twolineshloka
{आभूतसम्प्लवाश्चैव बहुधाऽद्धाऽपचक्रमुः}
{मन्वन्तरयुगा राजन्सङ्कल्पो भूतसम्प्लवाः}


\twolineshloka
{चक्रवत्परिवर्तन्ते सर्वं विषमुखं जगत्}
{सृष्ट्वा चतुर्मुखं देवं देवो नारायणः प्रभुः}


\twolineshloka
{स लोकानां हितार्थाय क्षीरोदे वसति प्रभुः}
{ब्रह्मा च सर्वलोकानां लोकस्य च पितामहः}


\twolineshloka
{ततो नारायणो देवः सर्वस्य प्रपितामहः}
{}


\chapter{अध्यायः ४४}
\twolineshloka
{अव्यक्तो व्यक्तलिङ्गस्थो य एव भगवान्प्रभुः}
{नरनारायणो भूत्वा हरिरासीद्युधिष्ठिर}


\fourlineindentedshloka
{ब्रह्मा च शक्रः सूर्यश्च धर्मश्चैव सनातनः}
{बहुशः सर्वभूतात्मा प्रादुर्भवति कार्यतः}
{प्रादुर्भावांस्तु वक्ष्यामि दिव्यान्देवगणैर्युतान्}
{}


\twolineshloka
{सुप्त्वा युगसहस्रं स प्रादुर्भवति कार्यवान्}
{अनेकबहुसाहस्रैर्देवदेवो जगत्पतिः}


\twolineshloka
{ब्रह्माणं कपिलं चैव परमेष्ठिं तथैव च}
{देवान्सप्तर्षिभिश्चैव शङ्करं च महायशाः}


\twolineshloka
{सनत्कुमारं भगवान्मनुं चैव प्रजापतिम्}
{पुरा चक्रे च देवादिः प्रदीप्ताग्निसमप्रभः}


\twolineshloka
{येन चार्णवमध्यस्थौ नष्टेस्थावरजङ्गमे}
{नष्टदेवासुरवरे प्रनष्टोरगराक्षसे}


\twolineshloka
{योद्धुकामौ सुदुर्धर्षौ भ्रातरौ मधुकैठभौ}
{हतौ भगवता तेन ततो दत्त्वा वरं परम्}


\twolineshloka
{भूमिं बद्ध्वा कृतौ पूर्वावजेयौ द्वौ महाऽसुरौ}
{तौ कर्णमलसंभूतौ विष्णोस्तस्य महात्मनः}


\twolineshloka
{महार्णवे प्रस्वपतः शैलराजसमौ स्थितौ}
{तौ विवेश स्वयं वायुर्ब्रह्मणा साधु चोदितः}


\twolineshloka
{तौ दिवं छादयित्वा तु ववृधाते महाऽसुरौ}
{वायुप्रमाणौ तौ दृष्ट्वा ब्रह्मा पर्यमृशच्छनैः}


\twolineshloka
{एकं मृदुतरं वेत्ति कठिनं वेत्ति चापरम्}
{नामनी तु तयोश्चके सविता सलिलोद्भवः}


\twolineshloka
{मृदुस्त्वयं मधुर्नाम कठिनः कैठभः स्वयम्}
{तौ दैत्यौ कृतनामानौ चेरतुर्बलगर्वितौ}


\twolineshloka
{तौ पुराऽथ दिवं सर्वां प्राप्तौ राजन्महासुरौ}
{प्रच्छाद्याथ दिवं सर्वां चेरतुर्मधुकैठभौ}


\twolineshloka
{सर्वमेकार्णवं लोकं योद्धुकामौ सुनिर्भयौ}
{तौ गतावसुरौ दृष्ट्वा ब्रह्मा लोकपितामहः}


\twolineshloka
{एकार्णवाम्बुनिचये तत्रैवान्तरधीयत}
{स पद्मात्पद्मनाभस्य नाभिदेशात्समुत्थितात्}


\twolineshloka
{आससाद स्वयं जन्म तत्पङ्कजमपङ्कजम्}
{पूजयामास वसतिं ब्रह्मा लोकपितामहः}


\twolineshloka
{तावुभौ जलगर्भस्थौ नारायणचतुर्मुखौ}
{बहून्वर्षायुतानप्सु शयानौ न च कम्पितौ}


\twolineshloka
{अथ दीर्घस्य कालस्य तावुभौ मधुकैठभौ}
{आजग्मतुस्तौ तं देशं यत्र ब्रह्मा व्यवस्थितः}


\twolineshloka
{तौ दृष्ट्वा लोकनाथस्तु रोषात्संरक्तलोचनः}
{उत्पपाताथ शयनात्पद्मनाभो महाद्युतिः}


\twolineshloka
{तद्युद्धमभवद्घोरं तयोस्तस्य च भारत}
{एकार्णवे तदा घोरे त्रैलोक्ये जलतां गते}


\twolineshloka
{तदभूत्तुमुलं युद्धं वर्षसङ्ख्यासहस्रशः}
{न च तावसुरौ युद्धे तदा श्रममवापतुः}


\twolineshloka
{अथ दीर्घस्य कालस्य तौ दैत्यौ युद्धदुर्मदौ}
{ऊचतुः प्रीतमनसौ देवं नारायणं प्रभुम्}


\twolineshloka
{प्रीतौ स्वस्तव युद्धेन श्लाघ्यस्त्वं मृत्युरावयोः}
{आवां जहि न यत्रोर्वा सलिलेन पिरप्लुता}


\twolineshloka
{हतौ च तव पुत्रत्वं प्राप्नुयाव सुरोत्तम}
{यो ह्यानां युधि निर्जेता तस्यावां विहितौ सुतौ}


\twolineshloka
{तयोस्तु वचनं श्रुत्वा तदा नारायणः प्रभुः}
{तौ प्रहस्य मृधे दैत्यौ दोर्भ्यां च समपीडयम्}


\twolineshloka
{ऊरुभ्यां निधनं चक्रे तावुभौ मधुकैठबौ}
{तौ हतौ चाप्लुतौ तोये वपुर्भ्यामेकतां गतौ}


\twolineshloka
{मेदो मुमुचतुर्दैत्यौ मज्जमानौ जलोर्मिभिः}
{मेदसा तज्जलं व्याप्तं ताभ्यामन्तर्दधे तदा}


\twolineshloka
{नारायणश्च भगवानसृजद्विविधाः प्रजाः}
{दैत्ययोर्मेदसा छन्ना सर्वा राजन्वसुन्धरा}


\twolineshloka
{तदाप्रभृति कौन्तेय मेदिनीति स्मृता मही}
{प्रभावात्पद्मनाभस्य शाश्वती च कृता नृणाम्}


\chapter{अध्यायः ४५}
\twolineshloka
{प्रादुर्भावसहस्राणि समतीतान्यनेकशः}
{यथाशक्ति तु वक्ष्यामि शृणु तान्कुरुनन्दन}


\twolineshloka
{पुरा कमलनाभस्य स्वपतः सागराम्भसि}
{पुष्करे यत्र सम्भूता देवा ऋषिगणैः सह}


\twolineshloka
{एष पौष्करिको नाम प्रादुर्भावः प्रकीर्तितः}
{पुराणैः कथ्यते यत्र वेदश्रुतिसमाहितः}


\twolineshloka
{वाराहस्तु श्रुतिसुखः प्रादुर्भावो महात्मनः}
{यत्र विष्णुः सुरश्रेष्ठो वाराहं रूपमास्थितः}


\twolineshloka
{उज्जहार महीं तोयात्सशैलवनकाननाम्}
{वेदपादो यूपदंष्ट्रः क्रतुर्दन्तश्चितीमुखः}


\twolineshloka
{अग्निजिह्वो दर्भरोमा ब्रह्मशीर्षो महातपाः}
{अहोरात्रेक्षणो दिव्यो वेदाङ्गः श्रुतिभूषणः}


\twolineshloka
{आज्यनासः स्रुवं तुण्डं सामघोषस्वनो महान्}
{धर्मसत्यमयः श्रीमान्कर्मविक्रमसत्कृतः}


\twolineshloka
{प्रायश्चित्तमुखो धीरः पशुजानुर्महावृषः}
{औद्गात्रहोमलिङ्गोऽसौ पशुबीजमहौषधिः}


\twolineshloka
{बाह्यन्तरात्मा मन्त्रास्थिविकृतः सौम्यदर्शनः}
{वेदिस्कन्धो हविर्गन्धो हव्यकव्याभिवेगवान्}


\twolineshloka
{प्राग्वंशकायो द्युतिमान्नानादीक्षाभिरूर्जितः}
{दक्षिणाहृदयो योगी महाशास्त्रमयो महान्}


\twolineshloka
{उपाकर्मोष्ठरुचकः प्रावर्ग्यावर्तभूषणः}
{शालापत्नीसहायो वै मणिशृङ्गसमुच्छ्रितः}


\twolineshloka
{एवं यज्ञवराहो वै भूत्वा विष्णुः सनातनः}
{महीं सागरपर्यन्तां सशैलवनकाननाम्}


\twolineshloka
{एकार्णवजले भ्रष्टामेकार्णवगतः प्रभुः}
{मज्जन्तीं सलिले तस्मिन्स्वदेवीं पृथिवीं तदा}


\twolineshloka
{उज्जहार विषाणेन मार्गण्डेयस्य पश्यतः}
{शृङ्गेण यः समुद्धृत्य लोकानां हितकाम्यया}


\twolineshloka
{सहस्रशीर्षो देवेशो निर्ममे जगतीं प्रभुः}
{एवं यज्ञवराहेण भूतभव्यभात्मना}


\twolineshloka
{उद्धृता पृथिवी देवी पूज्या वै सागराम्बरा}
{निहता दानवाः सर्वे देवदेवेन विष्णुना}


\twolineshloka
{वाराहः कथितो ह्येष नारसिंहमतो शृणु}
{यत्र भूत्वा मृगेन्द्रेण हिरण्यकशिपुर्हतः}


\chapter{अध्यायः ४६}
\twolineshloka
{दैत्येन्द्रो बलवान्त्राजन्सुरारिर्बलगर्वितः}
{हिरण्यकशिपुर्नाम आसीत्रैलोक्यकण्टकः}


\twolineshloka
{दैत्यानामादिपुरुषो वीर्येणाप्रतिमो बली}
{प्रविश्य जलधं राजंश्चकार तप उत्तमम्}


\twolineshloka
{दशवर्षसहस्राणि शतानि दश पञ्च च}
{व्रतोपवासतस्तस्थौ स्याणुमौनव्रतो दृढः}


\twolineshloka
{ततः शमदमाभ्यां च ब्रह्मचर्येण चानघ}
{ब्रह्मा प्रीतमनास्तस्य तपसा नियमेन च}


\twolineshloka
{ततः स्वयम्भूर्भगवान्स्वयमागम्य भूपते}
{विमानेनार्कवर्णेन हंसयुक्तेन भास्वता}


\twolineshloka
{आदित्यैर्वसुभिः साध्यैर्मरुद्भिर्दैवतैस्तथा}
{रुद्रैर्विश्वसहायैश्च यक्षराक्षसकिन्नरैः}


\twolineshloka
{दिशाभिर्विदिशाभिश्च नदीभिः सागरैः सह}
{नक्षत्रैश्च मुहूर्तैश्च खेचरैश्चापरैर्ग्रहैः}


\twolineshloka
{देवर्षिभिस्तपोयुक्तैः सिद्धैः सप्तर्षिभिस्तदा}
{राजर्षिभिः पुण्यतमैर्गन्धर्वैरप्सरोगणैः}


\twolineshloka
{चराचरगुरुः श्रीमान्वृतः सर्वसुरैस्तथा}
{ब्रह्मा ब्रह्मविदां श्रेष्ठो दैत्यमागम्य चाब्रवीत्}


\threelineshloka
{प्रीतोऽस्मि तव भक्तस्य तपसाऽनेन सुव्रत}
{वरं वरय भद्रं ते यथेष्टं काममाप्नुहि ॥हिरण्यकशिपुरुवाच}
{}


\twolineshloka
{न देवा न च गन्धर्वा न यक्षोरगराक्षसाः}
{न मानुषाः पिशाचाश्च हन्युर्मां देवसत्तम}


\twolineshloka
{ऋषयो वा न मां शापैः क्रुद्धा लोकपितामह}
{शपेयुस्तपसा युक्ता वर एष वृतो मया}


\twolineshloka
{न शस्त्रेण नचास्त्रेण गिरिणा पादपेन च}
{न शुष्केण न चार्देण स्यान्न वाऽन्येन मे वधः}


\twolineshloka
{नाकाशे नाथ भूमौ वा रात्रौ वा दिवसेपि वा}
{नान्तर्वा न बहिर्वापि स्याद्वधो मे पितामह}


\fourlineindentedshloka
{पशुभिर्वा मृगैर्न स्यात्पक्षिभिर्वा सरीसृपैः}
{ददासि ------------------}
{ब्रह्मोवाच}
{}


\twolineshloka
{एते दिव्या---------------}
{सर्वकामवरां---------------}


\twolineshloka
{एवमुक्त्वा स ------------हि}
{रराज ब्रह्मलोके ------------}


\threelineshloka
{ततो देवाश्च -------------}
{वरप्रदानं श्रुत्वै------------- ॥देवा ऊचुः}
{}


\twolineshloka
{वरेणाने भग--------------}
{तत्प्रसीदस्व भगवन्वधोपायोऽस्य चिन्त्यताम् ॥भीष्म उवाच}


\twolineshloka
{ततो लोकहितं वाक्यं श्रुत्वा देवः प्रजापतिः}
{प्रोवाच भगवान्वाक्यं सर्वदेवगणांस्तदा}


\twolineshloka
{अवश्यं त्रिदशास्तेन प्राप्तव्यं तपसः फलम्}
{तपसोऽन्तेऽस्य भगवान्वधं कृष्णः करिष्यति}


\twolineshloka
{एतच्छ्रुत्वा सुराः सर्वे ब्रह्मणा तस्य वै वधम्}
{स्वानि स्थानानि दिव्यानि जग्मुस्ते वै मुदान्विताः}


\twolineshloka
{लब्धमात्रे वरे चापि सर्वास्ता बाधते प्रजाः}
{हिरण्यकशिपुर्दैत्यो वरदानेन दर्पितः}


\twolineshloka
{राज्यं चकार दैत्येन्द्रो दैत्यसङ्घैः समावृतः}
{सप्तद्वीपान्वशेचके लोकालोकान्तरं बलात्}


\twolineshloka
{दिव्यभोगान्समस्तान्वै लोके सर्वानवाप सः}
{देवांस्त्रिभुवनस्थांस्तु पराजित्य महासुरः}


\twolineshloka
{त्रैलोक्यं वशमानीय स्वर्गे वसति दानवः}
{यदा वरमदोन्मत्तो न्यवसद्दानवो दिवि}


\twolineshloka
{अथ लोकान्सगस्तांश्च विजित्य स महाबलः}
{भवेयमहमेवेन्द्रः सोमोऽग्निर्मारुतो रविः}


\twolineshloka
{सलिलं चान्तरिक्षं च नक्षत्राणि दिशो दश}
{अहं क्रोधश्च कामश्च वरुणो वसवोऽर्यमा}


\twolineshloka
{धनदश्च धनाध्यक्षो यक्षकिम्पुरुषाधिपः}
{एते भवेयमित्युक्त्वा स्वयं भूत्वा बलात्स च}


\twolineshloka
{एषां गृहीत्वा स्थानानि तेषां कार्याण्यवाप सः}
{इज्यश्चासीन्मखवरैर्देवकिन्नरसत्तमैः}


\twolineshloka
{नरकस्थान्समानीय स्वर्गस्थांश्च चकार सः}
{एवमादीनि कर्माणि कृत्वा दैत्यपतिर्बली}


\twolineshloka
{आश्रमेषु महाभागान्मुनीन्वै शंसितव्रतान्}
{सत्यधर्मपरान्दान्तान्पुरा धर्षितवांस्तु सः}


\twolineshloka
{याज्ञीयान्कृतबान्दैत्यन्याजकांश्चैव देवताः}
{यत्रयत्र सुरा जग्मुस्तत्रतत्र व्रजत्युत}


\twolineshloka
{स्थानानि देवतानां तु हृत्वा राज्यमकारयत्}
{पञ्चकोट्यश्च वर्षाणि अयुतान्येकषष्टि च}


\twolineshloka
{षष्टिश्चैव सहस्राणां जग्मुस्तस्य दुरात्मनः}
{एतद्वर्षं स दैत्येन्द्रो भोगैश्चर्यमवाप सः}


\twolineshloka
{तेनातिबाध्यमानास्ते दैत्येन्द्रेण बलीयसा}
{ब्रह्मलोकं सुरा जग्मुः शर्वशक्रपुरोगमाः}


\twolineshloka
{पितामहं समासाद्य खिन्नाः प्राञ्जलयोऽब्रुवन् ॥देवा ऊचुः}
{}


\twolineshloka
{भगवन्भूतभव्येश नस्त्रायस्व इहागतान्}
{भयं दितिसुताद्घोराद्भवत्यद्य दिवानिशम्}


\threelineshloka
{भगवन्सर्वदैत्यानां स्वयम्भूरादिकृत्प्रभुः}
{स्रष्टा त्वं हव्यकव्यानामव्यक्तः प्रकृतिर्ध्रुवः ॥ब्रह्मोवाच}
{}


\twolineshloka
{श्रूयतामापदेवं हि दुर्विज्ञेया मयापि च}
{नारायणस्तु पुरुषो विश्वरूपो महाद्युतिः}


\twolineshloka
{अव्यक्तः सर्वभूतानामचिन्त्यो विभुरव्ययः}
{ममापि स तु युष्माकं व्यसने परमा गतिः}


\twolineshloka
{नारायणः परोऽव्यक्तादहमव्यक्तसम्भवः}
{मत्तो जज्ञुः प्रजा लोकाः सर्वे देवासुराश्च ते}


\twolineshloka
{देवा यथाहं युष्माकं तथा नारायणो मम}
{पितामहोऽहं सर्वस्य स विष्णुः प्रपितामहः}


\twolineshloka
{निश्चितं विषुधा दैत्यं स विष्णुस्तं हनिष्यति}
{तस्य नास्ति न शक्यं च तस्माद्व्रजत माचिरम् ॥भीष्म उवाच}


\twolineshloka
{पितामहवचः श्रुत्वा सर्वे ते भरतर्षभ}
{विबुधा ब्रह्मणा सार्धं जग्मुः क्षीरोदधिं प्रति}


\twolineshloka
{आदित्या वसवः साध्या विश्वे च मरुतस्तथा}
{रुद्रा महर्षयश्चैव अश्विनौ च सुरूपिणौ}


\threelineshloka
{अन्ये च दिव्या ये राजंस्ते सर्वे सगणाः सुराः}
{चतुर्मुखं पुरस्कृत्य श्वेतद्वीपमुपागताः ॥देवा ऊचुः}
{}


\twolineshloka
{त्रायस्व नोऽद्य देवेश हिरण्यकशिपोर्वधात्}
{त्वं हि नः परमो धाता ब्रह्मादीनां सुरोत्तम}


\twolineshloka
{उत्फुल्लाम्बुजपत्राक्ष शत्रुपक्षभयङ्कर}
{क्षयाय दितिवंशस्य शरणं त्वं भविष्यसि ॥भीष्ण उवाच}


\twolineshloka
{तद्देवानां वचः श्रुत्वा तदा विष्णुः शुचिश्रवाः}
{अदृश्यः सर्वभूतात्मा वक्तुमेवोपचक्रमे ॥विष्णुरुवाच}


\twolineshloka
{भयं त्यजध्वममरा अभयं वो ददाम्यहम्}
{तदेव त्रिदिवं देवाः प्रतिपद्यत माचिरम्}


\twolineshloka
{एषोऽहं सगणं दैत्यं वरदानेन दर्पितम्}
{अवध्यममरेन्द्राणां दानवेन्द्रं निहन्म्यहम् ॥ 2-46-53ब्रह्मोवाच}


\threelineshloka
{भहवन्देवदेवेश खिन्ना एते भृशं सुराः}
{तस्मात्त्वं जहि दैत्येन्द्रं क्षिप्रं कालोऽस्य माचिरम्}
{एष त्वं सगणं दैत्यं वरदानेन दर्पितम् ॥विष्णुरुवाच}


\twolineshloka
{क्षिप्रमेव करिष्यामि त्वरया दैत्यनाशनम्}
{तस्मात्त्वं विबुधाश्चैव प्रतिपद्यत वै दिवम् ॥भीष्म उवाच}


\twolineshloka
{एवमुक्त्वा तु भगवान्विसृज्य त्रिदिवेश्वरान्}
{नरस्यार्धतनुर्भूत्वा सिंहस्यार्धतनुः पुनः}


\twolineshloka
{नारसिंहेन वपुषा पाणिं संस्पृश्य पाणिना}
{भीमरूपो महातेजा व्यादितास्य इवान्तकः}


\twolineshloka
{हिरण्यकशिपुं राजञ्जगाम हरिरीश्वरः}
{दैत्यास्तमागतं दृष्ट्वा नारसिंहं महाबलम्}


\twolineshloka
{ववर्षुः शस्त्रवर्षैस्ते सुसङ्क्रुद्धास्तदा हरिम्}
{तैः सृष्टसर्वशस्त्राणि भक्षयामास वै हरिः}


\twolineshloka
{जघान न रणे दैत्यान्सहस्राणि बहूनि च}
{तान्निहत्य च दैतेयान्सर्वान्क्रुद्धान्महाबलान्}


\twolineshloka
{अभ्यधावत्सुसङ्क्रुद्धो दैत्येन्द्रं बलगर्वितम्}
{जीमूतघनसङ्काशो जीमूतघननिस्वनः}


\twolineshloka
{जीमूत इव दीप्तौजा जीमूत इव वेगवान्}
{दैत्यं सोऽतिबलं दृप्तं दृप्तशार्दूलविक्रमम्}


\twolineshloka
{दृप्तैर्दैत्यगणैर्गुप्तं खरैर्नखमुकैरुत}
{ततः कृत्वा तु युद्धं वै तेन दैत्येन वै हरिः}


\twolineshloka
{सन्ध्याकाले महातेजा भवनान्ते त्वरान्वितः}
{ऊरौ निधाय दैत्येन्द्रं निर्बिभेद नखैस्तदा}


\twolineshloka
{महाबलं महावीर्यं वरदानेन गर्वितम्}
{दैत्यश्रेष्ठं सुरश्रेष्ठो जघान तरसा हरिः}


\twolineshloka
{हिरण्यकशिपुं हत्वा सर्वदैत्यांश्च वै तदा}
{विबुधानां प्रजानां च हितं कृत्वा महाद्युतिः}


\threelineshloka
{प्रमुमोद हरिर्देवः प्राप्य धर्मं तदा भुवि}
{एष ते नारसिंहोऽत्र कथितः पाण्डुनन्दन}
{}


% Check verse!
शृणु त्वं वामनं नाम प्रादुर्भावं महात्मनः
\chapter{अध्यायः ४७}
\twolineshloka
{पुरा त्रेतायुगे राजन्बलिर्वैरोजनोऽभवत्}
{दैत्यानां पार्थिवो वीरो बलेनाप्रतिमो बली}


\twolineshloka
{तदा बलिर्महाराज दैत्यसङ्घैः समावृतः}
{विजेतुं तरसा शक्रमिन्द्रस्थानमवाप सः}


\twolineshloka
{तेन वित्रासिता देवा बलिनाऽऽखण्डलादयः}
{ब्रह्माणं तु पुरस्कृत्य गत्वा क्षीरोदधिं तदा}


\twolineshloka
{तुष्टुवुः सहिताः सर्वे देवं नारायणं प्रभुम्}
{स तेषां दर्शनं चक्रे विबुधानां हरिस्तदा}


\twolineshloka
{प्रसादजं तस्य विभोरदित्यां जन्म उच्यते}
{अदितेरपि पुत्रत्वमेत्य यादवनन्दनः}


\twolineshloka
{एष विष्णुरिति ख्यात इन्द्रस्यावरजोऽभवत्}
{तस्मिन्नेव च काले तु दैत्येन्द्रो बलवीर्यवान्}


\twolineshloka
{अश्वमेधं क्रतुश्रेष्ठमाहर्तुमुपचक्रमे}
{वर्तमाने तदा यज्ञे दैत्येन्द्रस्य युधिष्ठिर}


\twolineshloka
{स विष्णुर्मानवो भूत्वा प्रच्छन्नो ब्रह्मसंवृतः}
{मुण्डो यत्रोपवीती च कृष्णाजिनधरः शिखी}


\twolineshloka
{पालाशदण्डं सङ्गृह्य वामनोऽद्भुतदर्शनः}
{प्रविश्य स बलेर्यज्ञे वर्तमानो च दक्षिणाम्}


\twolineshloka
{देहीत्युवाच दैत्येन्द्रं विक्रमांस्त्रीनिहैव ह}
{दीयतां त्रिपदीमात्रमित्ययाचन्महासुरम्}


\twolineshloka
{स तथेति प्रतिश्रुत्य प्रददौ विष्णवे तदा}
{तेन लब्ध्वा हिरर्भूमिं जृम्भयामास वै भृशम्}


\twolineshloka
{स शिशुः सदिवं खं च पृथिवीं वच विशाम्पते}
{त्रिभिर्विक्रमणैश्चैव सर्वमाक्रमताभिभूः}


\twolineshloka
{बलेर्बलवतो यज्ञे बलिना विष्णुना पुरा}
{विक्रमैस्त्रिभिरक्षोभ्याः क्षोभितास्ते महासुराः}


\twolineshloka
{विप्रचित्तिमुखाः क्रुद्धाः सर्वसङ्घा महासुराः}
{नानावक्रा महाकाया नानावेषधरा नृप}


\twolineshloka
{नानाप्रहरणा रौद्रा नानामाल्यानुलेपनाः}
{स्वान्यायुधानि सङ्गृह्य प्रदीप्ता इव तेजसा}


\twolineshloka
{क्रममाणं हरि तत्र उपावर्तन्त भारत}
{प्रमथ्य सर्वान्दैतेयान्पादहस्ततलैस्तु तान्}


\twolineshloka
{रूपं कृत्वा महाभीमं जहाराशु स मेदिनीम्}
{सम्प्राप्य दिवमाकाशमादित्यसदने स्थितः}


\twolineshloka
{अत्यरोचत भूतात्मा आदित्यस्यैव तेजसा}
{प्रकाशयन्दिशः सर्वाः प्रदिशश्च महायशाः}


\twolineshloka
{शुशुभे स महाबाहुः सर्वलोकान्प्रकाशयन्}
{तस्य विक्रमतो भूमिं चन्द्रादित्यौ स्तनान्तरे}


\twolineshloka
{नभस्तु क्रममाणस्य नाभ्यां किल तदा स्थितौ}
{परमाक्रममाणस्य नानुभ्यां तौ व्यवस्थितौ}


\twolineshloka
{विष्णोरमितवीर्यस्य वदन्त्येवं द्विजातयः}
{अथास्य पादाक्रमणात्पफालाण्डो युधिष्ठिरः}


\twolineshloka
{तच्छिद्रात्स्यन्दिनी तस्य पादभ्रष्टा तु निम्नगा}
{ससार सागरं सा तु पावनी सागरंगमा}


\twolineshloka
{जहार मेदिनीं सर्वां हत्वा दानवपुङ्गवान्}
{आसुरी श्रियमाहृत्य त्रील्लोकान्स जनार्दन}


\twolineshloka
{सपुत्रदारानसुरान्पाताले संन्यवेशयत्}
{नमुचिः शम्बरश्चैव प्रह्लादश्च महामनाः}


\twolineshloka
{महाभूतानि भूतात्मा सविशेषानि वै हरिः}
{कालं च सकलं राजन्गात्रभूतान्यदर्शयत्}


\twolineshloka
{तस्य गात्रे जगत्सर्वमानीतमधिपश्यति}
{न किञ्चिदस्ति लोकेषु यदनाप्तं महात्मना}


\twolineshloka
{तद्धि रूपमुपेन्द्रस्य देवदानवमानवाः}
{दृष्ट्वा संमुमुहुः सर्वे विष्णुतेजोऽभिपीडिताः}


\twolineshloka
{बलिर्बद्धोऽभिमानी च यज्ञवाटे महात्मना}
{विरोचनकुलं सर्वं पाताले विनिवेशितम्}


\twolineshloka
{एवंविधानि कर्माणि कृत्वा गरुडावाहनः}
{न विस्मयमुपागच्छत्पारमेष्ठ्येन तेजसा}


\twolineshloka
{स सर्वमसुरैश्वर्यं सम्प्रदाय शचीपतेः}
{त्रैलोक्यं च ददौ शक्रे विष्णुर्दानवसूदनः}


\threelineshloka
{एष ते वामनो नाम प्रादुर्भावो महात्मनः}
{वेदविद्भिर्द्विजैरेतच्छ्रूयते वैष्णवं यशः}
{मानुषेषु ततो विष्णोः प्रादुर्भावांस्तथा शृणु}


\chapter{अध्यायः ४८}
\twolineshloka
{विष्णोः पुनर्महाभागः प्रादुर्भावो महात्मनः}
{दत्तात्रेय इति ख्यात ऋषिरासीन्महायशाः}


\twolineshloka
{तेन नष्टेषु वेदेषु क्रियासु च मखेषु च}
{चातुर्वर्ण्ये च सङ्कीर्णे धर्मे शिथिलतां गते}


\twolineshloka
{अभवर्धति चाधर्मे सत्ये नष्टे स्थितेऽनृते}
{प्रजासु क्षीयमाणासु धर्मे चामूलतां पते}


\twolineshloka
{सयज्ञाः सक्रिया वेदाः प्रत्यानीता हि तेन वै}
{चातुर्वर्ण्यमसङ्कीर्णं कृतं तेन महात्मना}


\twolineshloka
{स एव वै यदा प्रादाद्धैहयाधिपतेर्वरम्}
{तं हैहयानामधिपस्त्वर्जुनोऽभिप्रसादयन्}


\twolineshloka
{वनं पर्यचरन्सम्यक्छुश्रूषुरनुसूयकः}
{निर्ममो निरहङ्कारो दीर्घकालमतोषयत्}


\twolineshloka
{आराध्य दत्तात्रेयं हि अगृङ्णात्स वरानिमान्}
{आप्तादाप्ततरान्विप्राद्विद्वान्विद्वन्निषेवितात्}


\twolineshloka
{ऋतेऽमरत्वं विप्रेण दत्तात्रेयेण धीमता}
{वरैश्चतुर्भिः प्रवृत इमान्वव्रे वरान्नृपः}


\twolineshloka
{श्रीमान्मनस्वी बलवान्सत्यवागनसूयकः}
{सहस्रबाहुर्भूयासमेषु मे प्रथमो वरः}


\twolineshloka
{जरायुजाण्डजं सर्वं सर्वं चैव चराचरम्}
{शास्तुमिच्छामि धर्मेण द्वितीयस्त्वेष मे वरः}


\twolineshloka
{पितृन्देवानृषीन्विप्रान्यजेयं विपुलैर्मखैः}
{अमित्रांश्च शितैर्बाणैस्तृतीयो व्रर एष मे}


\twolineshloka
{यस्य नासीन्न भविता न चास्ति सदृशः पुमान्}
{इह वा दिवि वा लोके स मे हन्ता भवेदिति}


\twolineshloka
{सोऽर्जुनः कृतवीर्यस्य वरः पुत्रोऽभवद्युधि}
{स सहस्रं सहस्राणां माहिष्मत्यामवर्धत}


\twolineshloka
{स भूमिमखिलां जित्वा द्वीपांश्चापि समुद्रिणः}
{नभसीवाज्वलत्सूर्यः पुण्यैः कर्मभिर्जुनः}


\twolineshloka
{इन्द्रद्वीपं कशेरुं च कामद्वीपं गभस्तितम्}
{गन्धर्ववरुणद्वीपं सौहृष्टममितप्रभः}


\twolineshloka
{पूर्वैरजितपूर्वांश्च द्वीपनजयदर्जुनः}
{इदं तु कार्तवीर्यस्य बभूवासदृशं जनैः}


\twolineshloka
{न पूर्वे नापरे तस्य गमिष्यन्ति गतिं नृपाः}
{यदर्णवे प्रयातस्य वस्त्रं न परिषिच्यते}


\twolineshloka
{सौवर्णं सर्वमप्यासीद्विमानवरमुत्तमम्}
{चतुर्धा व्यभजद्राष्ट्रं तद्विभज्यान्वपालयत्}


\twolineshloka
{एकांशेनाहरत्सेनामेकांशेनावसद्गृहान्}
{यस्तु तस्य तृतीयांशो राज्ञोऽभूज्जनसङ्ग्रहे}


\twolineshloka
{आप्तः परमकल्याणस्तेन यज्ञानकल्पयत्}
{ये दस्यवो ग्रामचरा अरम्ये च वसन्ति ये}


\twolineshloka
{चतुर्थेन तु सोंऽशेन तान्सर्वान्प्रत्यषेधयत्}
{द्वाराणि नापिधीयन्ते पुरेषु नगरेषु च}


\twolineshloka
{स एव राष्ट्रपालोऽभूत्स्रीपालोऽभवदर्जुनः}
{स ----- सः गोपालो विशाम्पते}


\twolineshloka
{शतं वर्षसहस्राणामनुशिष्यार्जुनो महीम्}
{दत्तात्रेयप्रसादेन एवं राज्यं चकार सः}


\twolineshloka
{एवं बहूनि कर्माणि चक्रे लोकहिताय सः ॥दत्तात्रेय इति ख्यातः प्रादुर्भावो ह्ययं हरेः}
{कथितो भरतश्रेष्ट शृणु भूयो महात्मनः}


\chapter{अध्यायः ४९}
\twolineshloka
{तथा भृगुकुले जन्म यदर्थं च महात्मनः}
{जामदग्न्य इति ख्यातः प्रादुर्भावश्च वैष्णवः}


\twolineshloka
{जमदग्निसुतो राजन्त्रामो नाम स वीर्यवान्}
{हेहयान्तकरो राजन्स रामो बलिनां वरः}


\twolineshloka
{कार्तावीर्यो महावीर्यो बलेनाप्रतिमस्तदा}
{रामेण जामदग्न्येन हतो विषममाचरन्}


\twolineshloka
{तं कार्तवीर्यं राजानं हेहयानामरिन्दमम्}
{रथस्थं पार्थिवं रामः पातयित्वाऽवधीद्रणे}


\twolineshloka
{जम्भस्य यज्ञं हत्वा स ऋत्विजश्चैव संस्तरे}
{जम्भस्य मूर्ध्नि भेत्ता च हन्ता च शतदुन्दुभेः}


\twolineshloka
{स एष कृष्णो गोविन्दो जातो भृगुषु वीर्यवान्}
{सहस्रबाहुमुद्धर्तुं सहस्रजितमाहवे}


\twolineshloka
{क्षत्रियाणां चतुष्पष्टिमयुतानि महायशाः}
{सरस्वत्यां समेतानि एष वै धनुषाऽजयत्}


\twolineshloka
{ब्रह्मद्विषां धे तस्मिन्महस्राणि चतुर्दश}
{पुनर्जघान शूराणामतिक्रूरो रथर्षभः}


\twolineshloka
{ततो राज्ञां सहस्रं स भङ्क्ता पूर्वमरिन्दमः}
{सहस्रं मुसलेनाहन्सहस्रमुदकृन्तत}


\twolineshloka
{चतुर्दशसहस्राणि कृणदूममपाययत्}
{शिष्टान्ब्रह्मद्विषो जित्वा ततोऽस्नायत भार्गवः}


\twolineshloka
{रामरामेत्यमिक्रुष्टो ब्राह्मणैः क्षत्रियार्दितैः}
{निघ्नञ्शतसहस्राणि रामः परशुनाभिभूः}


\twolineshloka
{न ह्यमृष्यत तां वाचमार्तैर्भृशमुदीरिताम्}
{भृगो रामाभिधावेति यदाऽक्रन्दन्द्विजातयः}


\twolineshloka
{काश्मीरान्दरदान्कुन्तीन्क्षुद्रकान्मालवाञ्छवान्}
{चेदिकाशिकरूशांश्च ऋषिकान्क्रथकैशिकान्}


\twolineshloka
{अङ्गान्वङ्गान्कलिङ्गांश्च मागधान्काशिकोसलान्}
{रात्रायणान्वीतिहोत्रान्किरातान्कार्तिकावतान्}


\twolineshloka
{एतानन्यांश्च राजन्यान्देशेदेशे सहस्रशः}
{निकृत्य निशितैर्बाणैः सम्प्रदाय विवस्वते}


\twolineshloka
{कीर्णा क्षत्रियकोटीभिर्मेरुमन्दरभूषणा}
{त्रिः सप्तकृत्वः पृथिवी तेन निःक्षत्रिया कृता}


\twolineshloka
{कृत्वा निःक्षत्रियां चैव भार्गवः स महायशाः}
{इन्द्रगोपकवर्णस्य जीवञ्जीवनिभस्य च}


\twolineshloka
{पूरयित्वा च सरितः क्षतजस्य सरांसि च}
{चकार तर्पणं वीरः पितॄणां तासु तेषु च}


\twolineshloka
{सर्वानष्टादश द्वीपान्वशमानीय भार्गवः}
{सोऽश्वमेधसहस्राणि नरमेधशतानि च}


\twolineshloka
{इष्ट्वा सागरपर्यन्तां काश्यपाय महीं ददौ}
{तस्याग्रेणानुपर्येति भूमिं कृत्वा विपांसुलाम्}


\twolineshloka
{ततः कालकृतां सत्यां भार्गवाय महात्मने}
{गाधामप्यत्र गायन्ति ये पुराणविदो जनाः}


\twolineshloka
{वेदिमष्टादशोत्सेधां हिरण्यस्यातिपौरुषीम्}
{रामेण जामदग्न्येन प्रतिजग्राह काश्यपः}


\twolineshloka
{एवमिष्ट्वा महाबाहुः क्रतुभिर्भूरिदक्षिणैः}
{अन्यद्वर्षशतं रामः सौभे साल्वमयोधयत्}


\twolineshloka
{ततः स भृगुशार्दूलस्तं सौभं योधयन्प्रभुः}
{सुबन्धुरं रथं राजन्नास्थाय भरतर्षभ}


\twolineshloka
{नग्निकानां कुमारीणां गायन्तीनामुपाशृणोत्}
{रामराम महाबाहो भृगूणां कीर्तिवर्धन}


\twolineshloka
{त्यज शस्त्राणि सर्वाणि न त्वं सौभं वधिष्यसि}
{शङ्खचक्रगदापाणिर्देवानामभयङ्गरः}


\twolineshloka
{युधि प्रद्युम्नसाम्बाभ्यां कृष्णः सौभं वधिष्यति}
{तच्छ्रुत्वा पुरुषव्याघ्रस्तत एव वनं ययौ}


\twolineshloka
{न्यस्य सर्वाणि शस्त्राणि कालकाङ्क्षी महायशाः}
{रथं सर्वायुधं चैव शरान्परशुमेव च}


\twolineshloka
{धनूंष्यप्सु प्रतिष्ठाप्य रामस्तेपे परं तपः}
{ह्रियं प्रज्ञां श्रियं कीर्तिं लक्ष्मीं चामित्रकर्शनः}


\twolineshloka
{पञ्चाधिष्ठाय धर्मात्मा तं रथं विससर्ज ह}
{आदिकाले प्रवृत्तं तु व्यभजत्करमीश्वरः}


\threelineshloka
{नाघ्नतं श्रद्धया सौभं न ह्यशक्तो महायशाः}
{जामदग्न्य इति ख्यातो यस्त्वयं भगवानुपिः}
{}


% Check verse!
सोऽस्य भागस्तपस्तेपे भार्गवो लोकविश्रुतः
\chapter{अध्यायः ५०}
\twolineshloka
{शृणु राजंस्ततो विष्णोः प्रादुर्भावं महात्मनः}
{अष्टाविंशे युगे चापि मार्कण्डेयपुरः सरः}


\twolineshloka
{तिथौ नावमिके जज्ञे तथा दशरथादपि}
{कृत्वाऽऽत्मानं महाबाहुश्चतुर्धा विष्णुरव्ययः}


\twolineshloka
{लोके राम इति ख्यातस्तेजसा भास्करोपमः}
{प्रसादनार्थं लोकस्य विष्णुस्तत्र सनातनः}


\twolineshloka
{धर्मार्थमेव कौन्तेय जज्ञे तत्र महायशाः}
{तमप्याहुर्मनुष्येन्द्रं सर्वभूतपतेस्तनुम्}


\twolineshloka
{यज्ञविघ्नकरस्तत्र विश्वामित्रस्य भारत}
{सुबाहुर्निहतस्तेन मारीचस्ताडितो भृशम्}


\twolineshloka
{तस्मै दत्तानि चास्राणि विश्वमित्रेण धीमता}
{वधार्थं सर्वशत्रूणां दुर्वाराणि सुरैरपि}


\twolineshloka
{वर्तमाने महायज्ञे जनकस्य महात्मनः}
{भग्नं माहेश्वरं चापं क्रीडता लीलया भृशम्}


\twolineshloka
{ततस्तु सीतां जग्राह भार्यार्थे जानकीं विभुः}
{नगरीं पुनरासाद्य मुमुदे तत्र सीतया}


\twolineshloka
{कस्यचित्त्वथ कालस्य पित्रा तत्राभिचोदितः}
{कैकेय्याः प्रियमन्विच्छन्वनमभ्यवपद्यत}


\twolineshloka
{यः समाः सर्वधर्मज्ञश्चतुर्दश वने वसन}
{लक्ष्मणानुचरो रामः सर्वभूतहिते रतः}


\twolineshloka
{चतुर्दश वने तीर्त्वा तदा वर्षाणि भारत}
{रूपिणी यस्य पार्श्वस्था सीतेत्यभिहिता जनैः}


\twolineshloka
{पूर्वोचितत्वात्सा लक्ष्मीर्भर्तारमनुशोचति}
{जनस्थाने वसन्कार्यं त्रिदशानां चकार सः}


\twolineshloka
{मारीचं दूषणं हुत्वा खरं त्रिशिरसं तथा}
{चतुर्दश सहस्राणि रक्षसां घोरकर्मणाम्}


\twolineshloka
{जघान रामो धर्मात्मा प्रजानां हितकाम्यय}
{विराधं च कबन्धं च राक्षसौ घोरकर्मिणौ}


\twolineshloka
{जघान च तदा रामो गन्धर्वौ शाषविक्षतौ}
{स रावणस्य भगिनीनासाच्छेदमकारयत्}


\twolineshloka
{भार्यावियोगं तं प्राप्य मृगयन्व्यचरद्वनम्}
{स तस्मादृश्यमूकं तु गत्वा पम्पामतीत्य च}


\twolineshloka
{सुग्रीवं मारुतिं दृष्ट्वा चक्रे मैत्रीं तयोः स वै}
{अथ गत्वा स किष्किन्धां सुग्रीवेण तदा सह}


\twolineshloka
{निहत्य वालिनं युद्धे वानरेद्रं महाबलम्}
{अभ्यपिञ्चत्तदा रामः सुग्रीवं वानरेश्वरम्}


\twolineshloka
{ततः स वीर्यवान्राजंस्त्वरया वै समुत्सुकः}
{विचित्य वायुपुत्रेण लङ्कादेशं निवेदितः}


\twolineshloka
{मेतुं वद्ध्वा समुद्रस्य वानरैः स समुत्सुकः}
{सीतायाः पदमन्विच्छन्रामो लङ्कां विवेश वै}


\twolineshloka
{देवोरगगणानां हि यक्षराक्षसपक्षिणाम्}
{तत्रावद्यं राक्षसेन्द्रं रावणं युधि दुर्जयम्}


\twolineshloka
{युक्तं राक्षसकोटीभिर्भिन्नाञ्जनचयोपमम्}
{दुर्निरीक्ष्यं सुरगणैर्वरदानेन दर्पितम्}


% Check verse!
जघान सचिवैः सार्धं सान्वयं रावणं रणे
\twolineshloka
{त्रैलोक्यकण्टकं वीरं महाकायं महाबलम्}
{रावमं सगणं हत्वा रामो भूतपतिः पुरा}


\twolineshloka
{लङ्कायां तं महात्मानं राक्षसेन्द्रं विभीषणम्}
{अभिषिच्य ततो राम अमरत्वं ददौ तदा}


\twolineshloka
{आरुह्य पुष्पकं रामः सीतामादाय पाण्डव}
{सबलं स्वपुरं गत्वा धर्मराज्यमपालयत्}


\twolineshloka
{दानवो लवणो नाम मधोः पुत्रो महाबलः}
{शत्रुघ्नेन हतो राजंस्तदा रामस्य शासनात्}


\twolineshloka
{एवं बहूनि कर्माणि कृत्वा लोकहिताय सः}
{राजं चकार विधिवद्रामो धर्मभृतां वरः}


\threelineshloka
{शताश्वमेधानाजह्रे ज्योतिरुक्थ्यान्निरर्गलान्}
{नाश्रूयन्ताशुभा वाचो नात्ययः प्राणिनां तदा}
{}


\twolineshloka
{न दस्युजं भयं चासीद्रामे राज्यं प्रशसति}
{ऋषीणां देवतानां च मनुष्याणां तथैव च}


\twolineshloka
{पृथिव्यां धार्मिकाः सर्वे रामे राज्यं प्रशासति}
{नाधर्मिष्ठो नरः कश्चिद्बभूव प्राणिनां क्वचित्}


\twolineshloka
{प्राणापानौ समौ ह्यास्तां रामे राज्यं प्रशासति}
{गाधामप्यत्र गायन्ति ये पुराणविदो जनाः}


\twolineshloka
{श्यामो युवा लोहिताक्षो मातङ्गानामिवर्षभः}
{आजानुबाहुः सुमुखः सिंहस्कन्धो महाबलः}


\twolineshloka
{दशवर्षसहस्राणि दशवर्षशतानि च}
{राज्यं भोगं च सम्प्राप्य शशास पृथिवीमिमाम्}


\twolineshloka
{रामो रामो राम इति प्राजानामभवन्कथाः}
{रामभूतं जगदिदं रामे राज्यं प्रशासति}


\twolineshloka
{ऋग्यजुः सामहीनाश्च न तदाऽसन्द्विजायः}
{उषित्वा दण्डके कार्यं त्रिदशानां चकार सः}


\twolineshloka
{पूर्वापकारिणं तं तु पौलस्त्यं मनुजर्षभम्}
{देवगन्धर्वनागानामरिं स निजघानह}


\twolineshloka
{सत्ववान्गुणसम्पन्नो दीप्यमानः स्वतेजसा}
{एवमेव महाबाहुरिक्ष्वाकुकुलवर्धनः}


\twolineshloka
{रावणं सगणं हत्वा दिवमाक्रमताभिभूः}
{इति दाशरथेः ख्यातः प्रादुर्भावो महात्मनः}


\twolineshloka
{ततः कृष्णो महाबाहुर्भीतानामभयङ्करः}
{अष्टाविंशे युगे राजञ्जज्ञे श्रीवत्सलक्षणः}


\twolineshloka
{पेशलश्च वदान्यश्चलोके बहुमतो नृषु}
{स्मृतिमान्देशकालज्ञः शङ्खचक्रगदासिभृत्}


\twolineshloka
{वासुदेव इति ख्यातो लोकानां हितकृत्सदा}
{वृष्णीनां च कुले जातो भूमेः प्रियचिकीर्षया}


\twolineshloka
{शत्रूणां भयकृद्दाता मधुहेति स विश्रुतः}
{शकटार्जुनरामाणां कीलस्थानान्यसूदयत्}


\twolineshloka
{कंसादीन्निजघानाजौ दैत्यान्मानुषविग्रहान्}
{अयं लोकहितार्थाय प्रादुर्भावो महात्मनः}


\twolineshloka
{कल्की विष्णुयशा नाम भूयश्चोत्पत्स्यते हरिः}
{लेर्युगान्ते सम्प्राप्ते धर्मे शिथिलतां गते}


\twolineshloka
{पाषण्डिनां गणानां हि वधार्थं भरतर्षभ}
{धर्मस्य च विवृद्ध्यर्थं विप्राणां हितकाम्यया}


\twolineshloka
{एते चान्ये च बहवो विष्णोर्देवगणैर्युताः}
{प्रादुर्भावाः पुराणेषु गीयन्ते ब्रह्मवादिभिः}


\chapter{अध्यायः ५१}
\twolineshloka
{एवमुक्ते तु कौन्तेयस्ततः कौरवनन्दनः}
{आबभाषे पुनर्भीष्णे धर्मराजो युधिष्ठिरः}


\twolineshloka
{भूय एव मनुष्येन्द्र उपेन्द्रस्य यशस्विनः}
{जन्म वृष्णिषु विज्ञातुमिच्छामि वदतां वर}


\threelineshloka
{यथैव भगवाञ्जातः क्षिताविह जनार्दनः}
{माधवेषु महाबुद्धिस्तन्मे ब्रूहि पितामह ॥वैशम्पायन उवाच}
{}


\twolineshloka
{एवमुक्तस्ततो भीष्मः केशवस्य महात्मनः}
{माधवेषु तथा जन्म कथयामास वीर्यवान्}


\twolineshloka
{हन्त ते कथयिष्यामि युधिष्ठिर यथातथम्}
{यतो नारायणस्येह जन्म वृष्णिषु कौरव}


\twolineshloka
{पुरा लोके महाराज वर्तमाने कृते युगे}
{आसीत्रैलोक्यविख्यातः सङ्ग्रामस्तारकामयः}


\twolineshloka
{विरोचनो मयस्तारो वराहः श्वेत एव च}
{विप्रचित्तिः प्रलम्बश्च वृत्रजम्भबलादयः}


\twolineshloka
{नमुचिः कालनेमिश्च प्रह्लाद इति विश्रुतः}
{लम्बः किशोरः स्वर्भानुररिष्टो राक्षसेश्वरः}


\twolineshloka
{एते चान्ये च बहवो दैत्यसङ्घाः सहस्रशः}
{नानाशस्त्रधरा राजन्नानाभूषणवाहनाः}


\twolineshloka
{देवतानामभिमुखास्तस्थुर्दैतेयदानवाः}
{देवास्तु युध्यमानास्ते दानवानभ्ययू रणे}


\twolineshloka
{आदित्या वसवो रुद्राः साध्या विश्वे मरुद्गणाः}
{इन्द्रो यमश्च वरुणश्चन्द्रश्चैव धनेश्वरः}


\twolineshloka
{अश्विनौ च महावीर्यौ ये चान्ये देवतागणाः}
{चक्रुर्युद्धं महाघोरं दानवैश्च यथाक्रमम्}


\twolineshloka
{युध्यमानाः समेयुश्च देवा दैतेयदानवैः}
{तद्युद्धमभवद्घोरं देवदानवसङ्कुलम्}


\twolineshloka
{ताभ्यां बलाभ्यां सञ्जज्ञे तुमुलो विग्रहस्तदा}
{तीक्ष्णशस्त्रैः किरन्तोऽथ अभ्ययुर्देवदानवाः}


\twolineshloka
{घ्रन्ति देवान्सगन्धर्वान्सयक्षोरगचारणान्}
{ते वध्यमाना दैतेयैर्देवसङ्घास्तदा रणे}


\twolineshloka
{त्रातारं मनसा जग्मुर्देवं नारायमं प्रभुम्}
{एतस्मिन्नन्तरे तत्र जगाम हरिरीश्वरः}


\twolineshloka
{दीपयञ्ज्योतिषा भूमिं शङ्खचक्रगदाधरः}
{तमागतं सुपर्णस्थं विष्णुं लोकनमस्कृतम्}


\twolineshloka
{दृष्ट्वा मुदा युताः सर्वे भयं त्यक्त्वा रमे सुराः}
{चक्रुर्युद्धं पुनः सर्वे देवा दैतेयदानवैः}


\twolineshloka
{तद्युद्धमभवद्घोरमचिन्त्यं रोमहर्षणम्}
{जघ्रुर्दैत्यान्त्रणे घोराः सर्वे शक्रपुरोगमाः}


% Check verse!
ते बाध्यमाना बिबुधैर्दुद्रुवुदैत्यदानवाः
\twolineshloka
{विद्रुतान्दानवान्दृष्ट्वा तदा भारत संयुगे}
{कालनेमिरिति ख्यातो दानवः प्रत्यदृश्यता}


\twolineshloka
{शत्रुप्रहरणे घोरः शतबाहुः शताननः}
{शतशीर्षः स्थितः श्रीमाञ्छतशृङ्गं इवाचलः}


\twolineshloka
{भास्कराकारमुकुटः शिञ्जिताभरणाङ्गदः}
{धूम्रकेतुर्हरिश्मश्रुर्निर्दष्टोष्ठपुटाननः}


\twolineshloka
{त्रैलोक्यान्तरविस्तारं धारयन्विपुलं वपुः}
{तर्जयन्वै रणे देवाञ्छादयन्वै दिशो दश}


\twolineshloka
{अभ्यधावत्सुसङ्क्रुद्धो व्यादितास्य इवान्तकः}
{तत्र शस्त्रप्रतानैश्च देवान्धर्षितवान्त्रणे}


\twolineshloka
{अभ्याययुः सुरान्सर्वान्पुनस्ते दैत्यदानवाः}
{आपीडयन्त्रणे क्रुद्धास्ततो देवान्युधिष्ठिर}


\twolineshloka
{ते वध्यमाना विबुधाः समरे कालनेमिना}
{दैत्यैश्चैव महाराज दुद्रुवुस्ते दिशो दश}


\threelineshloka
{विबुधान्विद्रुतान्दृष्ट्वा कालनेमिर्महा}
{ञसुरः}
{इन्द्रं यमं च वरुणं वायुं च धनदं रविम्}


\twolineshloka
{एतांश्चान्यान्बलाञ्जित्वा तेषां कार्याण्यवाप सः}
{तान्सर्वान्सहसा जित्वा कालनेमिर्महासुरः}


\twolineshloka
{ददर्श गगने विष्णुं सुपर्णस्थं महाद्युतिम्}
{तं दृष्ट्वा क्रोधताम्राक्षस्तर्जयन्नभ्ययात्तदा}


\twolineshloka
{स बाहुशतमुद्यम्य सर्वास्त्रग्रहणं रणे}
{रोषाद्भारत दैत्येन्द्रो विष्णोरुरसि पातयत्}


\twolineshloka
{दैत्याश्च दानवाश्चैव सर्वे मयपुरोगमाः}
{स्वान्यायुधानि सङ्गृह्य सर्वे विष्णुमुपाद्रवन्}


\threelineshloka
{स ताड्यमानो}
{ञतिबलैर्दैत्यैः सर्वायुघोद्यतैः}
{न चचाल हरिर्युद्धेऽकम्पमान इवाचलः}


\twolineshloka
{पुनरुद्यम्य सङ्क्रुद्धः कालनेमिर्दृढां गदाम्}
{जघान गदया राजंस्तं विष्णुं गरुडं च वै}


\twolineshloka
{तं दृष्ट्वा गुरडं श्रान्तं चक्रमुद्यस्य वै हरिः}
{शतं शिरांसि बाहूंश्च सोच्छिनत्कालनेमिनः}


\twolineshloka
{जघानान्यांस्च तान्सर्वान्समरे दैत्यदानवान्}
{विबुधानामृषीणां च स्वानि स्थानानि वै ददौ}


\twolineshloka
{दत्त्वा सुराणां सुग्रीतो योग्यकर्माणि भारत}
{जगाम ब्रह्मणा सार्धं ब्रह्मलोकं तदा हरिः}


\twolineshloka
{ब्रह्मलोकं प्रविश्याश्च प्राप्य नारायणः प्रभुः}
{पौराणं ब्रह्मसदनं दिव्यं नारायणाश्रयम्}


\twolineshloka
{स प्रविश्य तदा देवः स्तूयमानो महर्षिभिः}
{सहस्रशीर्षा भूत्वा च शयनायोपचक्रमे}


\twolineshloka
{आदिदेवः पुराणात्मा निद्रावशमुपागतः}
{शेते सुखं सदा विष्णुर्मोहयञ्जगदव्ययः}


\twolineshloka
{जग्मुस्तस्याथ वर्षाणि शयानस्य महात्मनः}
{षट््त्रिंशच्छतसाहस्रं मानुषेणेह सङ्ख्यया}


\twolineshloka
{जग्मुः कृतयुगत्रेताद्वापरान्ते बुबोध ह}
{ब्रह्मादिभिः स्तूयमानः सुरैश्चापि सहर्षिभिः}


\twolineshloka
{उत्पत्य शयनाद्विष्णुर्ब्रह्मणा विबुधैः सह}
{देवानां च हितार्थाय ययौ देवसभां प्रति}


\twolineshloka
{मेरोः शिरसि विन्यस्तां ज्वलन्तीं तां शुभां सभाम्}
{विविशुस्ते सुराः सर्वे ब्रह्मणा सह भारत}


\twolineshloka
{जग्मुस्तत्र निषेदुस्ते सा निःशब्दा ह्यभूत्तदा}
{तत्र भूमिरुवाचाथ खेदात्करुणभाषिणी}


\twolineshloka
{राज्ञां बलैर्बलवतां खिन्नास्मि भृशपीडिता}
{नित्यं भारपरिश्रान्ता दुःखं जीवाम्यहं सुराः}


\twolineshloka
{पुरे पुरे च नृपतिः कोटिसङ्ख्यैर्बलैर्वृतः}
{राष्ट्रे राष्ट्रे च शतशो ग्रामाः कुलसहस्रिणः}


\twolineshloka
{भूमिपानां सहस्रैश्च तेषां च बिलनां बलैः}
{ग्रामायुतैः पुरै राष्ट्रैरहं निर्विवरीकृता}


\threelineshloka
{तस्माद्धारयितुं शक्त्या न क्षमासि जनानहम्}
{दैत्यैश्च बाध्यमानास्ताः प्राज नित्यं दुरात्मभिः ॥भीष्ण उवाच}
{}


\twolineshloka
{भूमेस्तु वचनं श्रुत्वा देवो नारायणस्तदा}
{व्यादिश्य तान्सुरान्सर्वान्क्षितौ वस्तुं मनो दधे}


\chapter{अध्यायः ५२}
\twolineshloka
{यच्चके भगवान्विष्णुर्वसुदेवसुतस्तदा}
{तत्तेऽहं सम्प्रवक्ष्यामि शृणु स्रवमशेषतः}


\twolineshloka
{वासुदेवस्य महात्म्यं चरितं च महात्मनः}
{हितार्थं सुरसर्त्यानां लोकानां च हिताय च}


\twolineshloka
{यदा दिवि विभुस्तात न रेमे भगवानसौ}
{ततो व्यादिशय भूतानि विभुर्भूमिसुखावहः}


\twolineshloka
{निग्रहार्थाय दैत्यानां चोदयामास वै तदा}
{मुरुतश्च वसूंश्चैव सूर्याचन्द्रमसावुभौ}


\twolineshloka
{गन्धर्वाप्सरसश्चैव रुद्रादित्यांस्तथाऽश्विनौ}
{जायध्वं मानुषे लोके सर्वलोकमहेश्वराः}


\twolineshloka
{जङ्गमानि विशालाक्षो ह्यात्मार्थमसृजत्प्रभुः}
{जायन्तामिति गोविन्दस्तिर्यग्योनिगतैः सह}


\fourlineindentedshloka
{तानि सर्वाणि सर्वज्ञो व्यजायत यदोः कुले}
{आत्मानमात्मना तात कृत्वा बहुविधं हरिः}
{रत्यर्थमिह गास्तत्र ररक्ष पुरुषोत्तमः}
{}


\twolineshloka
{अजातशत्रो जातस्तु यथेष्ट भुवि भूमिप}
{कीर्त्यमानं मया तात निबोध भरतर्षभ}


\twolineshloka
{सागराः समकम्पन्त मुदा चेलुश्च पर्वताः}
{जज्वलुश्चाग्नयः शान्ता जायमाने जनार्दने}


\twolineshloka
{शिवाः सम्प्रववुर्वाताः प्रशान्तमभवद्रजः}
{ज्योतींषि सम्प्रकाशन्त जायमाने जनार्दने}


\twolineshloka
{देवदुन्दुभयश्चापि सस्वनुर्भृशमम्बरे}
{अभ्यवर्षंस्तदाऽऽगम्य देवताः पुष्पवृष्टिभिः}


\twolineshloka
{गीर्भिर्मङ्गलयुक्ताभिः स्तुवन्वै मधुसदनम्}
{उपतस्थुस्तदा प्रीताः प्रादुर्भावे महर्षयः}


\twolineshloka
{ततस्तानभिसम्प्रेक्ष्य नारदप्रमुखानृषीन्}
{उपानृत्यन्नुपजगुर्गन्धर्वाप्सरसां गणाः}


\twolineshloka
{उपतस्थे च गोविन्दं सहस्राक्षः शचीपतिः}
{अभ्यभाषत तेजस्वी महर्षीन्पूजयंस्तदा}


\twolineshloka
{कृत्वा च देवकार्याणि कृत्वा देवहितानि च}
{खं लोकं लोककृद्देवः पुनर्गच्छति तेजसा}


\twolineshloka
{इत्युक्त्वा ऋषिभिः सार्घं जगाम त्रिदिवं पुनः}
{अभ्यनुज्ञाय तान्सर्वाञ्छादयन्प्रकृतिं पराम्}


\twolineshloka
{नन्दगोपकुले कृष्ण उवास बहुलाः समाः}
{ततः कदाचित्सुप्तं तं शकटस्य त्वधः शिशुम्}


\twolineshloka
{यशोदा सम्परित्यज्य जगाम यमुनां नदीम्}
{शिशुलीलां ततः कुर्वन्स्वहस्तचरणौ क्षिपन्}


\twolineshloka
{रुरोद मधुरं कृष्णः पादावूर्ध्वं प्रसारयन्}
{पादाङ्गुष्ठेन शकटं दारयन्नथ केशवः}


\twolineshloka
{तत्र एकेन पादेन पातयित्वा तथा शिशुः}
{न्युब्जं पयोधराकाङ्क्षी ससार च रुरोद च}


\twolineshloka
{पाटितं शकटं दृष्ट्वा भिन्नभाण्डपुटीकटम्}
{जनास्ते शिशुना तेन विस्मयं परमं ययुः}


\twolineshloka
{प्रत्यक्षं शूरसेनानां दृश्यते महदद्भुतम्}
{शयानेन हतः कंसपक्षवांस्तिग्मतेजसा}


\twolineshloka
{पूतना चापि निहता महाकाया महास्तनी}
{ततः काले महाराज संसक्तौ रामकेशवौ}


\twolineshloka
{कृष्णः सङ्कर्षणश्चोभौ रिङ्खिणौ च बभूवतुः}
{अन्योन्यकिरणाक्रान्तौ चन्द्रसूर्याविवाम्बरे}


\twolineshloka
{विसर्पन्तौ च सर्वत्र महासर्पभुजौ तदा}
{रेजतुः पांसुदिग्धाङ्गौ रामकृष्णौ तदा नृप}


\twolineshloka
{क्वचिच्च जानुभिः स्पृष्टौ क्रीडमानौ क्वचिद्वने}
{पिबन्तौ दधिकुल्यांश्च मथ्यमाने च भारत}


\twolineshloka
{ततः स बालो गोविन्दो नवनीतं तदा क्षयम्}
{ग्रासमानस्तु तत्रायं गोपीभिर्ददृशे तथा}


\twolineshloka
{दाम्नाऽथोलूखले कृष्णो गोपीभिश्च निबन्धितः}
{तत्तथा शिशुना तेन कर्षता चार्जुनावृभौ}


\twolineshloka
{समूलविटपौ भग्नौ तदद्भुतमिवाभवत्}
{ततस्तौ बाल्यमुत्तीर्णौ कृष्णसङ्कर्षणावुभौ}


\twolineshloka
{तस्मिन्नेव व्रजस्थाने सप्तवर्षै बभूवतुः}
{नीलपीताम्बरधरौ पीतश्वेतानुलेपनौ}


\twolineshloka
{बभुवतुर्वत्सपालौ काकपक्षधरावुभौ}
{पर्णवाद्यं श्रुतिसुखं वादयन्तौ वराननौ}


\twolineshloka
{शुशुभाते वनगतौ त्रिशीर्षाविव पन्नगौ}
{मयूराङ्गजकर्णौ तौ पल्लवापीडधारिणौ}


\twolineshloka
{वनमालापरिक्षिप्तौ सालपोताविवोद्गतौ}
{अरविन्दकृतापीडौ रज्जुयज्ञोपवीतिनौ}


\twolineshloka
{सशिक्यतुम्बुरुकरौ गोपवेणुप्रवादकौ}
{क्वचिद्वसन्तावन्योन्यं क्रडमानौ क्वचिद्वने}


\twolineshloka
{पर्णशय्यासु तौ सुप्तौ क्वचिन्निद्रान्तरैषिणौ}
{तौ वत्सान्पालयन्तौ हि शोभयन्तौ महद्वनम्}


\threelineshloka
{चञ्चूर्यन्तौ रमन्तौ च राजन्नेवं तदा शुभम्}
{ततो बृन्दावनं गत्वा वसुदेवसुतावुभौ}
{गोकुलं तत्र कौन्येय चारयन्तौ विजह्रतुः}


\chapter{अध्यायः ५३}
\twolineshloka
{ततः कदचिद्गोविन्दो ज्येष्ठं सङ्कर्षणं विना}
{चचार तद्वनं रम्यं सुस्वरूपो वराननः}


\twolineshloka
{काकपक्षधरः श्रीमाञ्छ्यामः पद्मनिभेक्षणः}
{श्रीवत्सेनोरसा युक्तः शशाङ्क इव लक्ष्मणा}


\twolineshloka
{रज्जुयज्ञोपवीती स पीताम्बरधरो युवा}
{श्वेतचन्द्रनलिप्राङ्गो नीलकुञ्चितमूर्धजः}


\twolineshloka
{राजता बर्हिपत्रेण मन्दमारुतकम्पिना}
{क्वचिद्गायन्क्वञ्चित्क्रीडन्क्वचिन्नृत्यन्क्वचिद्धसन्}


\twolineshloka
{गोपवेणुं सुमधुरं कामं तदपि वादयन्}
{प्रह्लादनार्थं च गवां क क्वचिद्वनगतो युवा}


\twolineshloka
{गोकुले मेघकाले तु चचार द्युतिमान्प्रभुः}
{बहुरम्येषु देशेषु वनस्य वनराजिपु}


\twolineshloka
{तासु कृष्णो मुदा युक्तः क्रीडयन्भरतर्षभ}
{स कदाचिद्वने तस्मिन्गोभिः सह परिव्रजन्}


\twolineshloka
{भाण्डीरं नाम दृष्ट्वाऽथ न्यग्रोधं केशवो महान्}
{तच्छायायां मतिं चक्रे निवासाय तदा प्रभुः}


\twolineshloka
{स तत्र वयसा तुल्यैर्बत्सपालैस्तदाऽनघ}
{रेमे स दिवसं कृष्णः पुरा स्वर्गगतो यथा}


\twolineshloka
{तं क्रीडमानं गोपालाः कृष्णं भाण्डीरवासिनः}
{रमयन्ति स्म बहवो मान्यैः क्रीडनकैस्तदा}


\twolineshloka
{अन्ये स्म परिगायन्ति गोपा मुदितमानसाः}
{गोपालकृष्णमेवान्ये गायन्ति स्म वनप्रियाः}


\twolineshloka
{तेषां सङ्गायतामेव वादयामास केशवः}
{पर्णवाद्यान्तरे वेणुं तुम्बवीणां च तत्र वै}


\twolineshloka
{एवं क्रीडान्तरैः कृष्णो गोपालैर्विजहार सः}
{तेन बालेन कौन्तेय कृतं लोकहितं तदा}


\twolineshloka
{पश्यतां सर्वभूतानां वासुदेवेन भारत}
{ह्रदे निपातता तत्र क्रीडितं नागमूर्धनि}


\twolineshloka
{शासयित्वा तु कालीयं सर्वलोकस्य पश्यतः}
{विजहार ततः कृष्णो बलेदवसहायवान्}


\twolineshloka
{धेनुको दारुणो राजन्दैत्यो रासभविग्रहः}
{तदा तालवने राजन्बलदेवेन वै हतः}


\twolineshloka
{ततः कदाचित्कौन्तेय रामकृष्णौ वनं गतौ}
{चारयन्तौ प्रवृद्धानि गोधनानि शुभाननौ}


\twolineshloka
{विहरन्तौ मुदा युक्तौ वीक्षमाणौ वनानि वै}
{श्वेलयन्तौ प्रगायन्तौ विचिन्वन्तौ च पादपान्}


\twolineshloka
{नामभिर्व्याहरन्तौ च वत्सान्गाश्च परन्तपौ}
{चेरतुर्लोकसिद्धाभिः क्रीडाभिरपराजितौ}


\twolineshloka
{तौ देवौ मानुषीं दीक्षां वहन्तौ सुरपूजितौ}
{तज्जातिगुणयुक्ताभिः क्रीडाभिश्चेरतुर्वनम्}


% Check verse!
एवं बाल्येऽपि गोपालैः क्रीडाभिश्च विजह्रतुः
\twolineshloka
{ततः कृष्णो महातेजास्तदा गत्वा तु गोव्रजम्}
{गिरियज्ञं तमेवैष प्रवृत्तं गोपदारकैः}


\twolineshloka
{बुभुजे पायसं शौरिरीश्वरः सर्वभूतकृत्}
{तं दृष्ट्वा गोपकाः सर्वे कृष्णमेव समर्चयन्}


\twolineshloka
{पूज्यमानस्तदा देवैर्दिव्यं वपुरधारयत्}
{धृतो गोवर्धनो नाम सप्ताहं पर्वतो धृतः}


\twolineshloka
{शिशुना वासुदेवेन गवार्थमरिमर्दन}
{क्रीडमानस्तदा कृष्णः कृतवान्कर्म दुष्करम्}


\threelineshloka
{तदद्भुतमतीवासीत्सर्वलोकस्य भारत}
{देवदेवः क्षितं गत्वा कृष्णं नत्वा मुदान्वितः}
{}


\twolineshloka
{गोविन्द इति तं ह्युक्त्वा ह्यभ्यषिञ्चत्पुरन्दरः}
{इत्युक्त्वाश्लिष्य गोविन्दं पुरुहूतोभ्ययाद्दिवम्}


\twolineshloka
{अथारिष्ट इति ख्यातं दैत्यं वृषभविग्रहम्}
{जघान तरसा कृष्णः पशूनां हितकाम्यया}


\twolineshloka
{केशिनामा ततो दैत्यो राजंस्तुरगविग्रहः}
{तथा वनगतं पार्थ गजायुतबलं हयम्}


\twolineshloka
{कराम्भोरुहवज्रेण जघान मधुसूदनः}
{अथ मल्लं तु चाणूरं निजघान महाऽसुरम्}


\twolineshloka
{सुदामानममित्रघ्न सर्वसैन्यपुरस्कृतम्}
{बालरूपेण गोविन्दो निजघान च भारत}


\twolineshloka
{बलदेवेन चायत्नात्समाजे मुष्टिको हतः}
{ताडितश्च सहामात्यः कंसः कृष्णेन भारत}


\threelineshloka
{हत्वा कंसममित्रघ्नः सर्वेषां पश्यतां तदा}
{अभिषिच्योग्रसेनं तं पित्रोः पादमवन्दत}
{}


% Check verse!
एवमादीनि कर्माणि कृतवान्वै जनार्दनः
\chapter{अध्यायः ५४}
\twolineshloka
{ततस्तौ जग्मतुस्तत्र गुरुं सान्दीपिनिं पुनः}
{गुरुशुश्रूषणायुक्तौ धर्मज्ञौ धर्मचारिणौ}


\twolineshloka
{व्रतमुग्रं महात्मानौ विचरन्ताववन्तिषु}
{अहोरात्रैश्चतुष्पष्ट्या साङ्गान्वेदानवापतुः}


\twolineshloka
{लेख्यं च गणितं चोभौ प्राप्नुतां यदुनन्दनौ}
{गान्धर्ववेदं वैद्यं च सकलं समावापतुः}


\twolineshloka
{हस्तिशिक्षामश्विशिक्षां द्वादशाहेन चाप्नुताम्}
{तावुभौ जग्मतुर्वीरौ गुरुं सान्दीपिनिं पुनः}


\twolineshloka
{धनुर्वेदचिकीर्षार्थं धर्मज्ञौ धर्मचारिणौ}
{ताविष्वासवराचार्यमभिगम्य प्रणम्य च}


\twolineshloka
{तेन वै सत्कृतौ राजंश्चरन्तौ ताववन्तिषु}
{पञ्चाशद्भिरहोरात्रैर्दशाङ्गं सुप्रतिष्ठितम्}


\twolineshloka
{सरहस्यं धनुर्वेदं सकलं ताववापतुः}
{दृष्ट्वा कृतार्थो विप्रेन्द्रो गुर्वर्थे तावचोदयत्}


\twolineshloka
{अयाचतार्थं गोविन्दं तदा सान्दीपिनिर्विभुम्}
{मम पुत्रः समुद्रेऽस्मिंस्तिमिना चापवाहितः}


\twolineshloka
{पुत्रमानय भद्रं ते भक्षितं तिमिना मम}
{आर्ताय गुरवे तत्र प्रतिशुश्राव दुष्करम्}


\twolineshloka
{अशक्यं सर्वभूतेषु कर्तुमन्येन केनचित्}
{यश्च सान्दीपिनेः पुत्रं जहार भरतर्षभ}


\twolineshloka
{सोऽसुरः समरे ताभ्यां समुद्रे विनिपातितः}
{ततः सान्दीपिनेः पुत्रः प्रसादादमितौजसः}


\twolineshloka
{दीर्घकालं कृतः प्रेतः पुनरासीच्छरीरवान्}
{तदशक्यमचिन्त्यं च दृष्ट्वा सुमहदद्भुतम्}


\twolineshloka
{सर्वेषामेव भूतानां विस्मयः समजायत}
{आसनानि च सर्वाणि गवाश्वं च धनादिकम्}


\twolineshloka
{सर्वं तदुपजहाते गुरवे रामकेशवौ}
{गदापरिघयुद्धे च सर्वास्त्रेषु च केशवः}


\twolineshloka
{परमां मुख्यतां प्राप्तः सर्वलोकेषु विश्रुतः}
{कश्च नारायणादन्यः सर्वरत्नविभूषितम्}


\twolineshloka
{रथमादित्यसङ्काशमातिष्ठेत शचीपतेः}
{कस्य चाप्रतिमो यन्ता वज्रपाणेः प्रियः सखा}


\twolineshloka
{मातलिः सङ्गृहीता स्यादन्यत्र पुरुषोत्तमात्}
{भोजराजात्मजो वापि कंसस्तात युधिष्ठिर}


\twolineshloka
{अस्त्रजाते बले वीर्ये कार्तवीर्यसमोऽभवत्}
{तस्य भोजपतेः पुत्राद्भोजराजन्यवर्धनात्}


\twolineshloka
{उद्विजन्ते स्म राजानः सुपर्णादिव पन्नगाः}
{चित्रकार्मुकनिस्त्रिंशविमलप्रासयोधिनः}


\twolineshloka
{शतं शतसहस्राणि पादातास्तस्य भारत}
{अष्टौ शतसहस्राणि शूराणामनिवर्तिनाम्}


\twolineshloka
{अभवन्भोजराजस्य जाम्बूनदमया ध्वजाः}
{रुक्मकाञ्चनकक्ष्यास्तु रथास्तस्य युधिष्ठि}


\twolineshloka
{अभवन्भोजपुत्रस्य द्विपास्तावद्धि तद्बलम्}
{चित्रकार्मुकनिस्त्रिंशविमलप्रासयोधिनाम्}


\twolineshloka
{षोडशाश्वसहस्राणि किंशुकाभानि तस्य वै}
{अपरस्तु महाव्यूहः किशोरणां युधिष्ठिर}


\twolineshloka
{आरोहवरसम्पन्नो दुर्धर्षः केनचिद्बलान्}
{स च षोडशसाहस्रः कंसभ्रातृपुरः सरः}


\twolineshloka
{सुनामा सर्वतस्त्वेनं स कंसं पर्यपालयत्}
{सगणो मिश्रको नाम षष्टिमसाहस्र उच्यते}


% Check verse!
कंसरोषमहावेगां वैवस्वतवशानुगाम् ॥मत्तद्विपमहाग्राहां वैवस्वतवशानुगाम्
\twolineshloka
{शस्रजालमहाफेनां सादिवेगमहाजलाम्}
{गदापरिघपाठीनां नानाकवचशैवलाम्}


\twolineshloka
{रथनागमहावर्तां नानारुधिरकर्दमाम्}
{चित्रकार्मुककल्लोलां रथाश्वकलिलह्रदाम्}


\twolineshloka
{महामृधनदीं घोरां योधावर्तननिस्वनाम्}
{कोऽन्यो नारायणादेत्य कंसहन्ता युधिष्ठिर}


\twolineshloka
{एष शक्ररथे तिष्ठंस्तान्यनीकानि भारत}
{व्यधमद्भोजपुत्रस्य महाभ्राणीव मारुतः}


\twolineshloka
{तं सभास्थं सहामात्यं हत्वा कंसं सहान्वयम्}
{आनयामास मानार्हां देवकीं समुहृद्गणाम्}


\twolineshloka
{यशोदां रोहिणीं चैव अभिवाद्य पुनः पुनः}
{उग्रसेनं च राजानमभिषिच्य जनार्दनः}


\threelineshloka
{अर्चितो यदुमुख्यैश्च भगवान्वासवानुजः}
{ततः पार्थिवमायान्तं सहितं सर्वराजभिः}
{सरस्वत्यां जरासन्धमजयत्पुरुषोत्तमः}


\chapter{अध्यायः ५५}
\twolineshloka
{शूरसेनपुरं त्यक्त्वा ततो यादवनन्दनः}
{द्वारकां भगवान्कृष्णः प्रत्यपद्यत भारत}


\twolineshloka
{ततो महात्मा यानानि रत्नानि विविधानि च}
{यथार्हं पुण्डरीकाक्षो नैर्ऋतात्प्रत्यपद्यत}


\twolineshloka
{तत्र विघ्नं चरन्ति स्म दैतेयाः सहदानवैः}
{ताञ्जघान महाबाहुर्वरमत्तान्महासुरान्}


\twolineshloka
{स विघ्नमकरोत्तत्र नरको नाम नैर्ऋतः}
{अदितिं धर्षयामास कुण्डलार्थं युधिष्ठिर}


\twolineshloka
{न चासुरगणैः सर्वैः सहितैः कर्म तत्पुरा}
{कृतपूर्वं महाघोरं यदकार्षीन्महासुरः}


\twolineshloka
{यं मही सुषुवे देवी यस्य प्राग्ज्योतिषं पुरम्}
{विषयान्तपालाश्चत्वारो यस्यासन्युद्धदुर्मदाः}


\twolineshloka
{आदेवयानमावृत्य पन्थानं पर्यवस्थिताः}
{त्रासनाः सुरसङ्घानां विरूपै राक्षसैः सह}


\twolineshloka
{हयग्रीवो निकुम्भश्च घोरः पञ्चजनस्तदा}
{मुरः पुत्रसहस्रैश्च वरमत्तो महासुरः}


\twolineshloka
{तद्वधार्थं महाबाहुरेष चक्रगदासिभृत्}
{जातो वृष्णिषु देवक्यां वासुदेवो जनार्दनः}


\twolineshloka
{तस्यास्य पुरुषेन्द्रस्यलोकप्रथिततेजसः}
{निवासो द्वारकायां तु विदितो वः प्रधानतः}


\twolineshloka
{अतीव हि पुरी रम्या द्वारका वासवक्षयात्}
{अतिवैराजमप्यद्धा प्रत्यक्षस्ते युधिष्ठिर}


\twolineshloka
{तस्मिन्देवपुरप्रख्ये सा सभा वृष्ण्युपाश्रया}
{सुधर्मेति च विख्याता योजनायतविस्तृता}


\twolineshloka
{तत्र वृष्ण्यन्दकाः सर्वे रामकृष्णपुरोगमाः}
{लोकयात्रामिमां कृत्स्नां परिरक्षन्त आसते}


\twolineshloka
{तत्रासीनेषु सर्वेषु कदाचिद्भरतर्षभ}
{दिव्यगन्धा ववुर्वाताः कुसुमानां च वृष्टयः}


\threelineshloka
{ततः सूर्यसहस्राभस्तेजोराशिर्महाद्भुतः}
{मध्ये तु तेजसस्तस्य पाण्डरं गजमास्थितः}
{}


\twolineshloka
{वृतो देवगणैः सर्वैर्वासवः प्रत्यदृश्यत ॥रामकृष्णौ च राजा च वृष्ण्यन्धकगणैः सह}
{}


\twolineshloka
{उत्पत्य सहसा देवे नमस्कारमकुर्वत ॥सोऽवतीर्य गजात्तूर्णं परिष्वज्य जनार्दनम्}
{}


\twolineshloka
{सस्वजे बलदेवं च राजानं च तमाहुकम् ॥वासुदेवोद्धवौ चैव विकद्रुं च महामतिम्}
{}


\twolineshloka
{प्रद्युम्नसाम्बनिशठाननिरुद्धं च सात्यकिम् ॥गदं सारणमक्रूरं भानुझल्लिविडूरथान्}
{}


\twolineshloka
{तथैव कृतवर्णाणां चारुदेष्णं महाबलम् ॥देवकल्पान्महाराज तान्दाशार्हपुरोगमान्}
{}


\twolineshloka
{पिरिष्वज्य च दृष्ट्वा च भगवान्भूतभावनः ॥वृष्ण्यन्धकमहामात्रान्परिष्वज्याथ वासवः}
{}


\twolineshloka
{प्रगृह्य पूजां तैर्दत्तां भगवान्पाकशासनः ॥सोऽदितेर्वचनात्तात कुण्डलार्थे जनार्दनम्}
{}


\threelineshloka
{उवाच परमप्रीतो जहि भौमं नरेश्व ॥भीष्ण उवाच}
{निहत्य नरकं भौममाहरिष्यामि कुण्डले}
{}


\twolineshloka
{एवमुक्त्वाऽथ गोविन्दो राममेवाभ्यभाषत ॥प्रद्युम्नमनिरुद्धं च साम्बं चाप्रतिमं बले}
{}


\twolineshloka
{एतांश्चोच्त्का तथा तत्र वासुदेवो महायशाः ॥अथारुह्य सुपर्णं वै शङ्खचक्रगदासिभृत्}
{}


\twolineshloka
{ययौ तदा हृषीकेशो देवानां हितकाम्यया ॥तं प्रयान्तममित्रघ्नं देवाः सहपुरन्दराः}
{}


\threelineshloka
{पृष्ठतोऽनुययुः प्रीत्या स्तुवन्तो विष्णुमच्युतम्}
{उग्रान्त्रक्षोगणान्हत्वा नरकस्य महासुरान्}
{}


\twolineshloka
{क्षुरान्तान्मौरवान्पाशान्षट््सहस्रं ददर्श सः ॥स़ञ्छिद्य पाशाच्छस्त्रेण मुरं हत्वा सहान्वयम्}
{}


% Check verse!
शैलसङ्घानतिक्रम्य निशुम्भं च व्यपोथयत्
\twolineshloka
{यः सहस्रसहस्त्वेकः सर्वान्देवानपोथयत्}
{तं जघान महावीर्यं हयग्रीवं महाबलम्}


% Check verse!
अपारतेजा दुर्धर्षः सर्वयादवनन्दनः ॥मध्ये लोहितगङ्गायां भगवान्देवकीसुतः
\threelineshloka
{औदकायां विरूपाक्षं जघान मधुसूदनः}
{ततः प्राग्ज्योतिषं नाम दीप्यमानमिव श्रिया}
{पुरमासादयामास तत्र युद्धमवर्तत}


\twolineshloka
{तद्युद्धमभवद्घोरं तेन भौमेन भारत}
{कुण्डलार्थे सुरेशस्य नरकेण महात्मना}


\twolineshloka
{मुहूर्तं लालयित्वा तु नरकं मधूसूदनः}
{प्रवृत्तचक्रं चक्रेण प्रममाथ बलाद्बली}


\twolineshloka
{चक्रप्रमथितं तस्य पपात सहसा भुवि}
{उत्तमाङ्गं हताङ्गस्य वृत्रे वज्रहते यथा}


\twolineshloka
{भूमिस्तु पतितं दृष्ट्वा प्रायच्छत्कुण्डले सुतम्}
{प्रदाय च महाबाहुमिदं वचनमब्रवीत्}


\threelineshloka
{सृष्टस्त्वयैव मधुहंस्त्वयैव विनिपातितः}
{यथेच्छसि तथा क्रीडा प्रजास्तस्यानुपालय ॥श्रीवासुदेव उवाच}
{}


\twolineshloka
{देवानां च मुनीनां च पितॄणां च महात्मनाम्}
{उद्वेजनीयो भूतानां ब्रह्मद्विद् पुरुषाधमः}


\twolineshloka
{लोकद्विष्टः सुतस्ते तु देवारिर्लोककण्टकः}
{सर्वलोकनमस्कार्यामदितं बाधयद्वली}


\twolineshloka
{कुण्डले हृतवान्दर्पात्ततस्ते निहतः सुतः}
{नैव मन्युस्त्वया कार्यो यत्कृतं मयि भामिनि}


\twolineshloka
{त्वत्प्रभावाच्च ते पुत्रो लब्धवान्गतिमुत्तमाम्}
{तस्माद्गच्छ महाभागे भारावतरणं कृतम्}


\chapter{अध्यायः ५६}
\twolineshloka
{निहत्य नरकं भौमं सत्यभामासहायवान्}
{सहितो लोकपालैश्च ददर्श नरकालयम्}


\threelineshloka
{अथास्य गृहमासाद्य नारकस्य महात्मनः}
{ददर्श धनमक्षय्यं रत्नानि विविधानि च}
{}


\twolineshloka
{मणिमुक्ताप्रवालानि वैडूर्यविकृतानि च}
{विस्ताराल्पांश्चार्कमणीन्विपुलान्स्फाटिकानपि}


\twolineshloka
{जाम्बूनदमयान्येव शातकुम्भमयानि च}
{प्रदीप्तज्वलनाभानि शीतरश्मिप्रभाणि च}


\twolineshloka
{हिरण्यवर्णं रुचिरं श्वेतमभ्यन्तरं गृहम्}
{तदक्षय्यं गृहे दृष्टं नरकस्य धनं बहु}


\twolineshloka
{न हि राज्ञः कुबेरस्य तावद्धनसमुच्छ्रयः}
{दृष्टपूर्वः पुरा साक्षान्महेन्द्रभवनेष्वपि}


\twolineshloka
{हते भौमे निशुम्भे च वासवः सगणोऽब्रवीत्}
{दाशार्हपतिमासीनमाहृत्य मणिकुण्डले}


\twolineshloka
{हेमसूत्रा महाकक्ष्यास्तोमरैर्वीर्यशालिनः}
{विमलानि पताकानि वासांसि विविधानि च}


\twolineshloka
{भीमरूपाश्च मातङ्गाः प्रवालविकृताः कुथाः}
{ते च विंशतिसाहस्रा द्वास्तावत्यः करेणवः}


\twolineshloka
{अष्टौ शतसहस्राणि देशजाश्चोत्तमा हयाः}
{गोभिश्चाविकृतैर्यावत्कामात्तव जनार्दन}


\threelineshloka
{एतत्ते प्रापयिष्याणि वृष्ण्यावासमरिन्दम}
{वसु यत्रिषु लोकेषु धर्मेणावर्जितं त्वया ॥भीष्म उवाच}
{}


\twolineshloka
{देवगन्धर्वरत्नानि दैतेयासुरजानि च}
{यानि सन्ति हिरण्यानि नरकस्य निवेशने}


\twolineshloka
{एतत्तु गरुडे सर्वं क्षिप्रमारोप्य वासवः}
{दार्शार्हपतिना सार्धमुपायान्मणिपर्वतम्}


\twolineshloka
{चित्रग्रथितमेघाभः प्रबभौ मणिपर्वतः}
{हेमचित्रवितानैश्च प्रासादैरुपशोभितः}


\twolineshloka
{हर्म्याणि च विशालानि मणिसोपानवन्ति च}
{तत्रस्था वरवर्णिन्यो ददृशुर्मधुसूदनम्}


\twolineshloka
{गन्धर्वसुरमुख्यानां प्रिया दुहितरस्तदा}
{त्रिविष्टपसमे देशे तिष्ठन्तमपराजितम्}


\twolineshloka
{परिवव्रुर्महाबाहुमेकवेणीधराः स्त्रियः}
{पर्वाः काषायवासिन्यः सर्वाश्च नियतेन्द्रियाः}


\twolineshloka
{व्रतसन्तापजः शोके नात्र कश्चिदपीडयत्}
{अरजांसि च वासांसि बिभ्रत्यः कौशिकान्यपि}


\twolineshloka
{समेत्य यदुसिंहस्य चक्रुरस्याञ्जलिं स्त्रियः}
{ऊचुश्चैनं हृषीकेशं सर्वास्ताः कमलेक्षणाः}


\twolineshloka
{नारदेन समाख्यातमस्माकं पुरुषोत्तम}
{आगमिष्यति गोविन्दः सुरकार्यार्थसिद्धये}


\twolineshloka
{सोऽसुरं नरकं हत्वा निशुम्भं मुरमेव च}
{भौमं च सपरीवारं हयग्रीवं च दानवम्}


\twolineshloka
{तथा पञ्चजनं चैव प्राप्स्यते धनमक्षयम्}
{सोऽचिरेणैव कालेन युष्मन्मोक्ता भविष्यति}


\twolineshloka
{एवमुक्त्वागमद्धीरो देवर्षिर्नारदस्तथा}
{त्वां चिन्तयानाः सततं तपो घोरमुपास्महे}


\threelineshloka
{कालेऽतीते महाबाहुं कदा द्रक्ष्याम माधवम्}
{इत्येवं हृदि सङ्कल्पं कृत्वा पुरुषसत्तम}
{तपश्चराम सततं रक्ष्यमाणा हि दानवैः}


\twolineshloka
{ततोऽस्मत्प्रियकामार्थं भगवान्मारुतोऽब्रवीत्}
{यथोक्तं नारदेनाथ न चिरात्तद्भविष्यति ॥भीष्ण उवाच}


\twolineshloka
{तासां परमनारीणामृषभाक्षं पुरः स्थितम्}
{ददृशुर्देवगन्धर्वा गृष्टीनामिव गोपतिम्}


\threelineshloka
{तस्य चन्द्रोपमं वक्त्रमुदीक्ष्य मुदितेन्द्रियाः}
{सम्प्रहृष्टा महाबाहुमिदं वचनमब्रुवन्}
{}


\twolineshloka
{सत्यव्रत पुरा वायुरिदमस्मानिहाब्रवीत्}
{सर्वभूतहितज्ञश्च महर्षिरपि नारदः}


\twolineshloka
{विष्णुर्नारायणो देवः शङ्खचक्रगदासिभृत्}
{स भौमं नरकं हत्वा भर्ता वो भविता ध्रुवम्}


\twolineshloka
{दिष्ट्या तस्यर्षिमुख्यस्य नारदस्य महात्मनः}
{वचनादेव सत्यं नो भर्ता भवितुमर्हसि}


\twolineshloka
{यत्प्रियं बत पश्याम श्रुतं प्रियमरिन्दम}
{दर्शनेन कृतार्थाः स्मो वयमस्य महात्मनः}


\twolineshloka
{उवाच हि यदुश्रेष्ठः सर्वास्ता जातमन्मथाः}
{यथा ब्रूत विशालाक्ष्यस्तत्सर्वं वो भविष्यति}


\twolineshloka
{ततस्ता गरुडे सर्वाः सरत्नधनसञ्चयाः}
{क्षिप्रमारोपयाञ्चक्रे भगवान्देवकीसुतः}


\twolineshloka
{सपक्षिगणमातङ्गं सव्यालमृगपन्नगम्}
{शाखामृगगणैर्जुष्टं सप्रस्तरशिलातलम्}


\twolineshloka
{न्यङ्कुभिश्च वराहैश्च रुरुभिश्च निषेवितम्}
{सप्रपातमहासानुं विचित्रशिखिसङ्कुलम्}


\twolineshloka
{स महेन्द्रानुजः शौरिश्चकार गुरुडोपरि}
{पश्यतां सर्वभूतानामुत्पाट्य मणिपर्वतम्}


\twolineshloka
{उपेन्द्रं बलदेवं च वासवं च महाबलम्}
{स्वपक्षबलविक्षेपैर्महाद्रिशिखरोपमः}


\twolineshloka
{दिक्षु सर्वासु संरावं स चक्रे गरुडो वहन्}
{आरुजन्पर्वताग्राणि पादपांश्च समुत्क्षिपन्}


\twolineshloka
{सञ्जहार महाभ्राणि वैश्वानरपथं गतः}
{ग्रहनक्षत्रताराणां सप्तर्षीणां स्वतेजसा}


\twolineshloka
{प्रभाजालमतिक्रम्य चाश्विनोश्च परन्तप}
{प्राप्य पुण्यतमं स्थानं देवलोकमरिन्दमः}


\threelineshloka
{शक्रसद्म समासाद्य चावरुह्य जनार्दनः}
{सोऽभिवाद्यादितेः पादावर्चितः सर्वदैवतैः}
{ब्रह्मदक्षपुरोगैश्च प्रजापतिभिरेव च}


\twolineshloka
{अदितेः कुण्डले दिव्ये ददावथ तदा विभुः}
{रत्नान च परार्घ्याणि रामेण सह केशवः}


\twolineshloka
{प्रतिगृह्य च तत्सर्वमदितिर्वासवानुजम्}
{पूजयामास दाशार्हं रामं च विगतज्वरा}


\twolineshloka
{शची महेन्द्रमहिषी कृष्णस्य महिषीं तदा}
{सत्यभामां तु सङ्गृह्य अदित्यै सा न्यवेदयत्}


\twolineshloka
{सा तस्याः सत्यभामायाः कृष्णाप्रियचिकीर्षया}
{वरं प्रादाद्देवमाता सत्यायै विगतज्वरा}


\twolineshloka
{जरां न यास्यसि शुभे यावत्कृष्णोऽस्ति भूतले}
{सर्वगन्धगुणोपेता भविष्यसि वरानने}


\twolineshloka
{विसृज्य सत्यभामा वै पौलोमीं च सुमध्यमा}
{शच्यापि समनुज्ञाता ययौ कृष्णनिवेशनम्}


\twolineshloka
{सम्पूज्यमानस्त्रिदशैर्महर्षिगणसेवितः}
{द्वारकां प्रययौ कृष्णो देवलोकादरिन्दमः}


\twolineshloka
{शीघ्रादेत्य महाबाहुर्दीर्घमध्वानमच्युतः}
{वर्घमानपुरद्वारमाससाद सुरोत्तमः}


\chapter{अध्यायः ५७}
\twolineshloka
{तां पुरी द्वारकीं दृष्ट्वा विभुर्नारायणो हरिः}
{हृष्टः सर्वार्थसम्पन्नः -------}


\twolineshloka
{सोऽपश्यद्वृक्षषण्डांश्च रम्यान्नानाजनान्वहून्}
{समन्ततो द्वारवत्यां नानापुष्पफलान्वितान्}


\twolineshloka
{अर्कचन्द्रप्रतीकाशैर्मेरुकूटनिभैर्गृहैः}
{द्वारकामावृतां रम्यां सुकृतां विश्वकर्मणा}


\twolineshloka
{पद्मषण्डाकुलाभिश्च हंससेवितवारिभिः}
{गङ्गासिन्धुप्रकाशाभिः परिघाभिरलङ्कृताम्}


\twolineshloka
{प्राकारेणार्कवर्णेन पाण्डरेण विराजिताम्}
{वियन्मूर्ध्नि निविष्टेन द्यामिवाभ्रपरिच्छदाम्}


\twolineshloka
{नन्दनप्रतिमैश्वापि मिश्रकप्रतिमैर्वनैः}
{तत्र सा विहिता साक्षान्नगरी विश्वकर्मणा}


\twolineshloka
{------------ जनहर्षिणी}
{------------ प्रसादप्रवरैः शुभा}


\twolineshloka
{तस्मिन्पुरवारश्रेष्ठे दाशार्हाणां यशस्विनाम्}
{नेश्मानि जहृषे दृष्ट्वा भगवान्पाकशासनः}


\twolineshloka
{समुच्छ्रितपताकानि पारिप्लवनिभानि च}
{------------ मेरुकूटनिभानि च}


% Check verse!
--------
% Check verse!
---- सधातुभिरिवाद्रिभिः ॥ 2-57-12a----- -----------जाम्बूनदमयद्वारैर्वैडूर्यविकृतार्गलैः
\twolineshloka
{सर्वर्तुसुखसंस्यर्शैर्महाधनपरिच्छदैः}
{रम्यसानुगृहैः शृङ्गैर्विचित्रैरिव पर्वतैः}


\twolineshloka
{पञ्चवर्णसवर्णैश्च पुष्पवृष्टिसमप्रभैः}
{तुल्यैः पर्जन्यनिर्घोषैर्ह्रादैर्भोगवती यथा}


\twolineshloka
{कृष्णध्वजोपवाह्यैश्च दाशार्हायुधरोहितैः}
{वृष्णिवीरमयूरैश्च स्त्रीसहस्रप्रजाकुलैः}


\twolineshloka
{वासुदेवैन्द्रपर्जन्यैर्गृहमेघैरलङ्कृता}
{ददृशे द्वारकाऽतीव मेघैर्द्यैरिव संवृता}


\twolineshloka
{साक्षाद्भगवतो वेश्म विहितं विश्वकर्मणा}
{ददृशुर्वासुदेवस्य चतुर्योजनमायतम्}


\twolineshloka
{तावदेव सुविस्तीर्णं सुसम्पूर्णं महाधनैः}
{प्रासादवरसम्पन्नं युक्तं जगति पर्वतैः}


\twolineshloka
{यं चकार महाभागस्त्वष्टा वासवचोदितः}
{प्रासादं हेमनाभस्य सर्वतो योजनायतम्}


\twolineshloka
{मेरोरिव गिरेः शृङ्गमुच्छ्रितं काञ्चनालयम्}
{रुक्मिण्याः प्रवरो वासो निर्मितः सुमहात्मना}


\twolineshloka
{सत्यभामा पुनर्वेश्म सदा वसति पाण्डरम्}
{विचित्रमणिसोपानं यं विदुः शीतवानिति}


\twolineshloka
{विमलादित्यवर्णाभिः पताकाभिरलङ्कृतम्}
{व्यक्तबद्धं यथोद्देशे चतुर्दशमहाध्वजम्}


\twolineshloka
{सर्वप्रासादमुख्योऽत्र जाम्बवत्या विभूषितः}
{प्रभाया जृम्भणैश्चित्रैस्त्रैलोक्यमिव भासयन्}


\twolineshloka
{यस्तु पाण्डरवर्णाभस्तयोरन्तरमाश्रितः}
{विश्वकर्माकरोदेनं कैलासशिखरोपमम्}


\twolineshloka
{जाम्बूनदप्रदीप्ताग्रः प्रदीप्तज्वलनोपमः}
{सागरप्रतिमोऽतिष्ठन्मेरुरित्यभिविश्रुतः}


\twolineshloka
{तस्मिन्गान्धारराजस्य दुहिता कुलशालिनी}
{सुकेशी नाम विख्याता केशवेन निवेशिता}


\twolineshloka
{पद्मकूट इति ख्यातः पद्मवर्णो महाप्रभः}
{सुप्रभाया महाबाहो वासः स परमोच्छ्रितः}


\twolineshloka
{यस्तु सूर्यप्रभो नाम प्रासादवर उच्यते}
{लक्षणायाः कुरुश्रेष्ठ स दत्तः शार्ङ्गधन्वना}


\twolineshloka
{वैडूर्यवरवर्णाभः प्रासादो हरितप्रभः}
{श्वेतजाला हि यत्रैव यत्रैव च निवेशिता}


\twolineshloka
{यं विदुः सर्वभूतानि हरिरित्येव भारत}
{सुमित्रविजयावासो देवर्षिगणपूजितः}


\twolineshloka
{महिष्या वासुदेवस्य भूषणं सर्ववेश्मनाम्}
{यस्तु प्रासादमुख्योऽत्र विहितः सर्वशिल्पिभिः}


\twolineshloka
{महिष्या वासुदेवस्य केतुमानिति विश्रुतः}
{प्रसादो विरजो नाम विरजस्को महात्मनः}


\twolineshloka
{उपस्थानगृहं तात केशवस्य महात्मनः}
{यस्तु प्रासादमुख्योऽत्र यं त्वष्टा व्यदधात्स्वयम्}


\twolineshloka
{योजनायतविष्कम्भं सर्वरत्नमयं विभोः}
{तेषां तु विहिताः सर्वे रुक्मदण्डाः पताकिनः}


\twolineshloka
{सदने वासुदेवस्य मार्गसञ्जनना ध्वजाः}
{घण्टाजालानि तत्रैव सर्वेषां निवेशने}


\twolineshloka
{आहृत्य यदुसिंहेन वैजयन्तच्छलो महात्}
{हंसकूटस्य यच्छ्रङ्गमिन्द्रद्युम्नसरो महत्}


\twolineshloka
{षष्टितालसमुत्सेधमर्धयोजनविस्तृतम्}
{सकिन्नरमहानादं तदप्यमिततेजसः}


\twolineshloka
{पश्यतां सर्वभूतानां त्रिषु लोकेषु विश्रुतम्}
{आदित्यपथगं यत्तन्मेरोः शिखरमुत्तमम्}


\twolineshloka
{जाम्बूनदमयं दिव्यं त्रिषु लोकेषु विश्रुतम्}
{तदप्युत्पाट्य कुच्छ्रेण स्वं निवेशनमाहृतम्}


\twolineshloka
{भ्राजमानं पुरा तत्र सर्वौषधिविदीपितम्}
{यमिन्द्रभवनाच्छौरिराजहार परन्तपः}


\twolineshloka
{पारिजातः स तत्रैव केशवेन निवेशितः}
{लेपहस्तशतैर्जुष्टो विमानैश्च हिरण्मयैः}


\twolineshloka
{विहिता वासुदेवेन तत्रैव च महाद्रुमाः}
{पद्माकुलजलोपेता रक्तसौगन्धिकोत्पलाः}


\twolineshloka
{मणिमौक्तिकवालूकाः पुष्करिण्यः सरांसि च}
{तासां परमकूलि शोभयन्ति महाद्रुमाः}


\twolineshloka
{सालतालाश्वकर्णाश्च शतशाखाश्च रोहिणः}
{भल्लातककपित्थाश्च इन्द्रवृक्षाश्च चम्पकाः}


\twolineshloka
{खादिरा मृतकाश्चैव समन्तात्परिरोपिताः}
{ये च हैमवता वृक्षा ये च नन्दनजास्तथा}


\twolineshloka
{आहृत्य यदुसिंहेन तेऽपि तत्र निवेशिताः}
{रत्नपीतारुणप्रख्याः सितपुष्पाश्च पादपाः}


\twolineshloka
{सर्वर्तुफलपूर्णोस्ते ते च काननसिन्धुषु}
{सहस्रपत्रपद्माश्च मन्दराश्च सहस्रशः}


\twolineshloka
{अशोकाः कर्णिकाराश्च तिलका नाग मल्लिकाः}
{कुरका नागपुष्पाश्च चम्पकास्तृणपुल्लिकाः}


\twolineshloka
{सप्तवर्णाः कबन्धाश्च नीपाः कुरवकास्तथा}
{केतकाः केसराश्चैव हिनतालतलताटकाः}


\twolineshloka
{तालाः प्रलम्बा वकुलाः पिण्डिका बीजपूरकाः}
{द्रुतामलकखर्जूरा महिता जम्बुकास्तथा}


\twolineshloka
{आम्राः पनसवृक्षाश्च चम्पकास्तिलतिन्दुकाः}
{लिकुचामृताश्चैव क्षीरिका कर्णिका तथा}


\twolineshloka
{नालिकेरेङ्गुदाश्चैव उत्क्रोशकवनानि च}
{कदली जातमल्ली च पाटली कुमुदोत्पलाः}


\twolineshloka
{नीलोत्पलकपूर्णाश्च वाप्यः कूपाः सहस्रशः}
{फुल्लाशाककपित्थाश्च तैस्तीर्त्वा बन्धुजीवकाः}


\twolineshloka
{प्रियालाशोकवादिर्याः प्राचीनाश्चापि सर्वशः}
{प्रियङ्गुबदरीभिश्च यवैः स्यन्दनचन्दनैः}


\twolineshloka
{शचीपीलुपलाश्चैश्च पलाशवधपिप्लैः}
{उदुम्बरैश्च बिल्वैश्च पालाशैः पारिभद्रकैः}


\twolineshloka
{इन्द्रवृक्षार्जुनैश्चैव अश्वत्थैश्चिरबिल्वकैः}
{भौमगञ्जनवृक्षैश्च भल्लाभैरश्वसाह्वयैः}


\twolineshloka
{सज्जैस्ताम्बूलवल्लीभिर्लवङ्गैः क्रमुकैस्तथा}
{वंशैश्च विविधैस्तत्र समन्तात्परिरोपितैः}


\twolineshloka
{ये च नन्दनजा वृक्षा ये च चैत्ररथे वने}
{सर्वे ते यदुनाथेन समन्तात्परिरोपिताः}


\twolineshloka
{समाहिता महानद्यः पीतलोहितवालुकाः}
{तस्मिन्गृहवरे रम्ये मणिशक्रसवालुकाः}


\twolineshloka
{मत्तबर्हिणनादाश्च कोकिलाश्च मदावहाः}
{बभूवुः परमोपेताः सर्वे जगति पर्वताः}


\twolineshloka
{तत्रैव गजयूथानि तत्र गोमहिषास्तथा}
{निवासाश्च कृतास्तत्र वराहा मृगपक्षिणाम्}


\twolineshloka
{विश्वकर्मकृतः शैलः प्राकारस्तत्र वेश्मनि}
{व्यक्तकिष्कुशतोद्यामः सुधारससमप्रभः}


\twolineshloka
{तेन ते च महाशैलाः सरितश्च सरांसि च}
{परिक्षिप्तानि वै तस्य वनान्युपवनानि च}


\chapter{अध्यायः ५८}
\twolineshloka
{एवमालोकयाञ्चक्रुर्द्वारकामृषभास्त्रयः}
{उपेन्द्रबलदेवौ च वासवश्च महायशाः}


\twolineshloka
{ततस्तं पाण्डरं शौरिर्मूर्ध्नि तिष्ठन्गरुत्मतः}
{प्रीतः शङ्खमुपाध्मासीद्द्विषतां रोमहर्षणम्}


\twolineshloka
{तस्य शङ्खस्य शब्देन सारश्चुक्षुभे भृशम्}
{ररास च नभः सर्वं तच्चित्रमभवत्तदा}


\twolineshloka
{पाञ्चजन्यस्य निर्घोषं निशम्य कुकुरान्दकाः}
{प्रीयमाणाः समाजग्मुरालोक्य मधुसूदनम्}


\twolineshloka
{वसुदेवं पुरस्कृत्य वेणुशङ्खरवैः सह}
{उग्रसेनो ययौ राजा वासुदेवनिवेशनम्}


\twolineshloka
{आनन्दितुं पर्यचन्स्वेषु वेश्मसु देवकी}
{रोहिणी च ययौ देशमाहुकस्य च याः स्त्रियः}


\twolineshloka
{हता ब्रह्मद्विषः सर्वे जयन्त्यन्धकवृष्णयटः}
{एवमुक्तः सह स्त्रीभिरक्षतैर्मधुसूदनः}


\twolineshloka
{ततः शौरिः सुपर्णेन स्वं निवेशनमभ्ययात्}
{चकाराथ यथोद्देशमीश्वरो मणिपर्वतम्}


\twolineshloka
{ततो धनानि रत्नानि सभायां मधुसूदनः}
{निधाय पुण्डरीकाक्षः पितुर्दर्शनलालसः}


\twolineshloka
{ततः सान्दीपिनिं पूर्वं ब्राह्मणं चापि भारत}
{यथान्यायं वासुदेव उपस्पृष्ट्वा महायशाः}


\twolineshloka
{ववन्दे पृथुताम्राक्षः प्रीयमाणो महायशाः}
{तथाऽश्रुपरिपूर्णाक्षमानन्दभृतचेतसम्}


\twolineshloka
{ववन्दे सह रामेण पितरं वासवानुजः}
{ताभ्यां च मूर्ध्न्युपाघ्रातः केशवः परवीरहा}


\twolineshloka
{यथाश्रेष्ठमुपागम्य सात्वतान्यदुनन्दनः}
{सर्वेषां नाम जग्राह दाशार्हाणामधोक्षजः}


\twolineshloka
{ततः सर्वाणि वित्तानि सर्वरत्नमयानि च}
{व्यभजत्तानि तेभ्योऽथ सर्वेभ्यो यदुनन्दनः}


\twolineshloka
{सा केशवमहामात्रैर्महेन्द्रप्रतिमैः सभा}
{शुशुभे वृष्णिशार्दूलैः सिंहैरिव गिरेर्गुहा}


\fourlineindentedshloka
{अथासनगतान्सर्वानुवाच विबुधाधिपः}
{शुभया हर्षयन्वाचा महेन्द्रस्तान्महायशाः}
{कुकुरान्धकमुख्यांश्च तं च राजानमाहुकम् ॥इन्द्र उवाच}
{}


\twolineshloka
{यदर्थं जन्म कृष्णस्य मानुषेषु महात्मनः}
{यत्कृतं वासुदेवेन तद्वक्ष्यामि समासतः}


\twolineshloka
{अयं शतसहस्राणि दानवानामरिन्दमः}
{निहय् पुण्डरीकाक्षः पातालविवरं ययौ}


\twolineshloka
{यच्च नाधिगतं पूर्वैः प्रह्लादबलिशम्बरैः}
{तदिदं शौरिणा वित्तं प्रापितं भवतामिह}


\twolineshloka
{सपाशं मुरमाक्रमय् पाञ्चजन्यं च धीमता}
{शिलासङ्घानतिक्रम्य निशुम्भः सगणो हतः}


\twolineshloka
{हयग्रीवश्च विक्रान्तो दानवो निहतो बली}
{मथितश्च मृधे भौमः कुण्डले चाहृते पुनः}


\twolineshloka
{पुनर्बाणवधे शौरिमादित्या वसुभिः सह}
{मन्मुखा आगमिष्यन्ति साध्याश्च मधुसूदनम्}


\twolineshloka
{एवमुक्त्वा ततः सर्वानामन्त्र्य कुकुरान्धकान्}
{सस्वजे रामकृष्णौ च वसुदेवं च वासवः}


\twolineshloka
{प्रद्युम्नसाम्बप्रमुखाननिरूद्धं च सारणम्}
{बभ्रुं झल्लिं गदं भानुं चारुदेष्णं च वृत्रहा}


\threelineshloka
{सत्कृत्य सारणाक्रूरौ पुनराभाष्य सात्यकिम्}
{सस्वजे वृष्णिराजानमाहुकं कुकुराधिपम्}
{भोजं च कृतवर्णाणमन्यांश्चान्दकवृष्णिषु}


\twolineshloka
{आमन्त्र्य देवप्रवरैर्वासवो वासवानुजम्}
{ततः श्वेताचलप्रख्यं गजमैरावतं प्रभुः}


\twolineshloka
{पश्यतां सर्वभातानामारुरोह शचीपतिः}
{पृथिवीं चान्तरिक्षं च दिवं च वरवारणम्}


\twolineshloka
{मुखाडम्बरनिर्घोषैः पूरयन्तमिवासकृत्}
{हैमयन्त्रमहाकक्ष्यं हिरण्मयविषाणिनम्}


\twolineshloka
{मनोहरकुथास्तीर्णं सर्वरत्नविभूषितम्}
{नित्यस्रुतमदस्रावं क्षरन्तमिव तोयदम्}


\twolineshloka
{दिशागजं महामात्रं काञ्चनस्रजमास्थितः}
{प्रबभौ मन्दराग्रस्थः प्रतपन्भानुमानिव}


\twolineshloka
{ततो वज्मयं भीमं प्रगृह्य परामाङ्कुशम्}
{ययौ बलवता सार्धं पावकेन शचीपतिः}


\twolineshloka
{तं करेणुगजव्रातैर्विमानैश्च मरुद्गणाः}
{पृष्ठतोऽनुययुः प्रीताः कुबेरवरुणग्रहाः}


\twolineshloka
{स वायुपक्षमास्थाय वैश्वानरपथं गतः}
{प्राप्य सूर्यपथं देवस्तत्रैवान्तरधीयत}


\chapter{अध्यायः ५९}
\twolineshloka
{ततः सर्वदशार्हाणामाहुकस्य च याः स्त्रियः}
{नन्दगोपस्य महिषी यशोदा लोकविश्रुता}


\twolineshloka
{रेवती च महाभागा रुक्मिणी च पत्रिव्रता}
{सत्या जाम्बवती चोभे गान्धारी शिंशुमापि च}


\twolineshloka
{विशोका लक्षणा चापि सुमित्रा केतुमा तथा}
{वासुदेवमहिष्योऽन्याः श्रिया सार्धं ययुस्तदा}


\twolineshloka
{विभूतिं द्रष्टुमनसः केशवस्य महात्मनः}
{प्रीयमाणाः सभां जग्मुरालोकयितुमच्युतम्}


\twolineshloka
{देवकी सर्वदेवीनां रोहिणी च पुरस्कृता}
{ददृशुर्देवमासीनं कृष्णं हलभृता सह}


\twolineshloka
{तौ तु पूर्वमतिक्रम्य रोहिणीमभिवाद्य च}
{अभ्यवादयतां देवौ देवकीं रामकेशवौ}


\twolineshloka
{सा ताभ्यामृषभाक्षाभ्यां पुत्राभ्यां शुशुभेऽधिकम्}
{देवकी देवमातेव मित्रेण वरुणेन च}


\twolineshloka
{ततः प्राप्ता यशोदाया दुहिता वै क्षणेन हि}
{जाज्वल्यमाना वपुषा प्रभयाऽतीव भारत}


\twolineshloka
{एकानङ्गेति यामाहुः कन्यां वै कामरूपिणीम्}
{यत्कृते सगणं कंसं जघान पुरुषोत्तमः}


\twolineshloka
{ततः स भगवान्रामस्तामुपाक्रम्य भामिनाम्}
{मूर्ध्न्युपाघ्राय सव्येन परिजग्राह पाणिना}


\twolineshloka
{तां च तत्रोपसम्प्राप्य प्रियामिव सखीमिमाम्}
{दक्षिणेन कराग्रेण पिरजग्राह माधवः}


\twolineshloka
{ददृशुस्तां सभामध्ये भगिनीं रामकृष्णयोः}
{रुक्मपद्मशां पद्मश्रीमिवोत्तमनाभयोः}


\twolineshloka
{अथाक्षतमहावष्ट्या लाजपुष्पघृतैरपि}
{वृष्णयोऽवाकिरन्प्रीताः सङ्कर्षणजनार्दनौ}


\twolineshloka
{सबालाः सहवृद्धाश्च ये ज्ञातिकुलबान्धवाः}
{उपोपविविशुः प्रीता वृष्णयो मधुसूदनम्}


\twolineshloka
{पूज्यमानो महाबाहुः पौराणां रतिवर्धनः}
{विवेश पुरुषव्याघ्रः स्ववेश्म मधुसूदनः}


\twolineshloka
{रुक्मिण्या सहितो देव्या प्रमुमोद सुखी सुखम्}
{अनन्तरं च सत्याया जाम्बवत्याश्च भारत}


\twolineshloka
{सर्वासां च यदुश्रेष्ठो गेहे गेहे विहारवान्}
{जगाम च हृषीकेशो रुक्मिण्याः सदनं पुनः}


\twolineshloka
{एष तात महाबाहो विजयः शार्ङ्गधन्वनः}
{एतदर्थं च जन्माहुर्मानुषेषु महात्मनः}


\chapter{अध्यायः ६०}
\twolineshloka
{द्वारकायां ततः कृष्णः स्वदारेषु दिवानिशम्}
{सुखं लब्ध्वा महाराज प्रमुमोद महायशाः}


\twolineshloka
{पौत्रस्य कारणाच्चक्रे विबुधानां प्रियं तदा}
{सावसवैः सुरैः सर्वैर्दुष्करं भरतर्षभ}


\twolineshloka
{बाणो नामाऽभवद्राजा बलेर्ज्येष्ठसुतो बली}
{वीर्यवान्भरतश्रेष्ठ स च बाहुसहस्रवान्}


\twolineshloka
{ततस्तेपे तपस्तीव्रं सत्वेन मनसा नृप}
{रुद्रमाराधयामास स च बाणः समा बहु}


\twolineshloka
{तस्मै बहुवरा दत्ताः शङ्करेण महात्मना}
{तांश्च लब्ध्वा वरान्बाणो दुर्लभानसुरैर्भुवि}


\twolineshloka
{स शोणितपुरे राज्यं चकाराप्रतिमो बली}
{त्रासिताश्च सुराः सर्वे तेन बाणेन पाण्डव}


\twolineshloka
{विजित्य विबुधान्सेन्द्रान्बाणः संवत्सरान्बहून्}
{अशासत महद्राज्यं कुबेर इव भारत}


\twolineshloka
{ततो राजन्नुषा नाम बाणस्य दुहिता यथा}
{येनोपायेन कौन्तेय अनिरुद्धो महाद्युतिः}


\twolineshloka
{प्राद्युम्निस्तामुषां प्राप्य प्रच्छन्नः प्रमुमोद ह}
{अथ बाणो महातेजास्तदा तत्र युधिष्ठिर}


\twolineshloka
{तं गृह्यनिलयं ज्ञात्वा प्राद्युम्निं सुतया तदा}
{गृहीत्वा कारयामास वस्तुं कारागृहे बलात्}


\twolineshloka
{स कुमारः सुखार्होऽथ तदा दुःखसमन्वितः}
{बामेन घातितो राजन्ननिरुद्धो मुमोह च}


\twolineshloka
{एतस्मिन्नेव काले तु नारदो मुनिपुङ्गवः}
{द्वारकां प्राप्य कौन्तेय कृष्णं दृष्ट्वा वचोऽब्रवीत्}


\twolineshloka
{कृष्ण कृष्ण महाबाहो यदूनां कीर्तिवर्धन}
{पौत्रस्ते बाध्यमानोऽत्र बाणेनामिततेजसा}


\twolineshloka
{कृच्छ्रं प्राप्तोऽनिरुद्धो वै शेते कारागृहे सदा}
{एतदुक्त्वा सुरर्षिर्वै बाणस्याथ पुरं ययौ}


\twolineshloka
{नारदस्य वचः श्रुत्वा ततो राजञ्जनार्दनः}
{जाहूय बलदेवं हि प्रद्युम्नं च महाद्युतिम्}


\twolineshloka
{आरुरोह गरुत्मन्तं ताभ्यां सह जनार्दनः}
{ततः सुपर्णमारुह्य जयाय भरतर्षभ}


\twolineshloka
{जग्मुः क्रुधा महावीर्या बाणस्य नगरं प्रति}
{अथासाद्य महाराज तत्पुरं ददृशुश्च ते}


\twolineshloka
{ताम्रप्राकारसङ्गुप्तां हेमप्रासादसङ्कुलाम्}
{दृष्ट्वा मुदा युताः सर्वे विस्मयं परमं ययुः}


\twolineshloka
{तथा बाणपुरस्यासन्द्वारस्था देवताः सदा}
{महेश्वरो गुहश्चैव भद्रकाली विनायकः}


\twolineshloka
{अथ कृष्णो बलाज्जित्वा द्वारपालान्युधिष्ठिर}
{सुसङ्क्रुद्धो महातेजाः शङ्खचक्रगदासिभृत्}


\twolineshloka
{आससादोत्तरद्वारं शङ्करेणाभिरक्षितम्}
{तत्र तस्थौ महातेजाः शूलपाणिर्महेश्वरः}


\twolineshloka
{पिनाकं सशरं गृह्य बाणस्य हितकाम्यया}
{ज्ञात्वा तमागतं कृष्णं व्यादितास्यमिवान्तकम्}


\twolineshloka
{ततस्तौ चक्रतुर्युद्धं वासुदेवमहेश्वरौ}
{तद्युद्धमभवद्घोरमचिन्त्यं रोमहर्षणम्}


\twolineshloka
{अन्योन्यं तौ ततक्षाते अन्योन्यजयकाङ्क्षिणौ}
{दिव्यान्यस्त्राणि तौ देवौ क्रुद्धौ मुमुचतुस्तदा}


\threelineshloka
{ततः कृष्णो रणं कृत्वा मुहूर्तं शूलपाणिना}
{विजित्य तं महादेवं ततो युद्धे शूलपाणिना}
{अन्यांश्च जित्वा द्वारस्थान्प्रविवेश पुरोत्तमम्}


\threelineshloka
{प्रविश्य बाणमासाद्य स तत्राथ जनार्दनः}
{चक्रे युद्धं महाक्रुद्धस्तेन बाणेन भरतर्षभ}
{}


\twolineshloka
{बाणोऽपि सर्वशस्त्राणि शितानि भरतर्षभ}
{सुसङ्क्रुद्धस्तदा युद्धे पातयामास केशवे}


\twolineshloka
{पुनरुद्यम्य शस्त्राणि सहस्रं सर्वबाहुभिः}
{मुमोच बाणः सङ्क्रुद्धः कृष्णं प्रति रणाजिरे}


\twolineshloka
{ततः कृष्णस्तदा कृत्त्वा तानि सर्वाणि भारत}
{कृत्त्वा मुहूर्तं बाणेन युद्धं राजन्नगोक्षजः}


\twolineshloka
{चक्रमुद्यम्य रोषाद्वै दिव्यं शस्त्रोत्तमं ततः}
{सहस्रबाहूंश्चिच्छेद बाणस्यामिततेजसः}


\twolineshloka
{ततो बाणो महाराज कृष्णेन भृशपीडितः}
{भिन्नबाहुः पपाताशु विशाख इव पादपः}


\twolineshloka
{स पातयित्वा बाणैस्तं बाणं कृष्णस्त्वरान्वितः}
{प्राद्युम्निं मोचयामास क्षिप्रं राजगृहात्तदा}


\twolineshloka
{मोक्षयित्वाऽथ गोविन्दः प्राद्युम्निं सह भार्यया}
{बाणस्य सर्वरत्नानि असङ्ख्यानि जहार सः}


\twolineshloka
{गोधनानि च सर्वस्वं स बाणस्यालये बलात्}
{जहार च हृषीकेशो यदूनां कुलवर्धनः}


\twolineshloka
{ततः स सर्वरत्नानि चाहृत्य मधुसूदनः}
{क्षिप्रमारोपयाञ्चक्रे सर्वस्वं गरुडोपरि}


\twolineshloka
{त्वरयाऽथ स कौन्तेय बलदेवं महाबलम्}
{प्रादुम्निं च महावीर्यमनिरुद्धं महाद्युतिम्}


\twolineshloka
{उषां च सुन्दरीं राजन्भृत्यदारगणैः सह}
{सर्वानेतान्समारोप्य गरुडोपरि वीर्यवान्}


\threelineshloka
{मुदा युक्तो महातेजाः पीताम्बरधरो बली}
{दिव्याभरमचित्राङ्गः शङ्कचक्रगदासिभृत्}
{2-60-38ca आरुरोहगरुत्मन्तमुदयं भास्करो यथा}


\chapter{अध्यायः ६१}
\twolineshloka
{सूदिता द्वारपालाश्च निशुम्भनरकौ हतौ}
{कृतक्षेमः पुनः पन्थाः पुरं प्राग्ज्योतिषं प्रति}


\twolineshloka
{शौरिणा पृथिवीपालास्त्रासिता भरतर्षभ}
{धनुषश्च प्रणादेन पाञ्चजन्यस्वनेन च}


\twolineshloka
{मेघप्रख्यैरनेकैश्च दाक्षिणात्याभिसंवृतम्}
{रुक्मिणं त्रासयामास केशवो भरतर्षभ}


\twolineshloka
{ततः पर्जन्यघोषेण रथेनादित्यवर्चसा}
{उवाह महिषीं भोज्यामेष चक्रगदाधरः}


\twolineshloka
{जारूथ्य आहृतक्रोधः शिशुपालश्च निर्जितः}
{वक्रश्च स हतः सङ्ख्ये शतधन्वा च क्षत्रियः}


\twolineshloka
{इन्द्रद्युम्नो हतः कोपाद्यवनश्च कशेरुकः}
{हतः सौभपतिश्चैव साल्वश्च कृतधन्वना}


\twolineshloka
{पर्वतानां सहस्रं च चक्रेण पुरुषोत्तमः}
{विभज्य पुण्डरीकाक्षो द्युमत्सेनमपोथयत्}


\twolineshloka
{महेन्द्रशिखरे चैव निमेषान्तरचारिणौ}
{जग्राह भरतश्रेष्ठ वानरावभितश्चरौ}


\twolineshloka
{इरावत्यां महाभोजो वह्निसूर्यसमो बले}
{गोपतिस्तालकेतुश्च निहतौ शार्ङ्गधन्वना}


\twolineshloka
{अक्षप्रपत्तने राजन्नवहेलनतत्परौ}
{उभौ तावपि कृष्णेन स्वराष्ट्रे विनिपातितौ}


\twolineshloka
{दग्धा वाराणसी तात केशवेन महात्मना}
{पाण्ड्यं पौण्ड्रं च मात्स्यं च कलिङ्गं च जनार्दनः}


% Check verse!
जघान सहितान्सर्वानङ्गराजं च माधवः
\twolineshloka
{एष चैव शतं हत्वा रथेन क्षत्रपुङ्गवान्}
{गान्धारीमवहत्कृष्णो महिषीं यादवर्षभः}


\twolineshloka
{अथ गाण्डीवधन्वानं क्रीडार्थं मधुसूदनः}
{जिगाय भरतश्रेष्ठ कुन्त्याश्च प्रमुखे विभुः}


\twolineshloka
{द्रौणिं कृपं च कर्णं च भीमसेनं सुयोधनम्}
{युद्धाय सहितान्त्राजञ्जिगाय भरतर्षभ}


\twolineshloka
{बभ्रोश्च प्रियमन्विच्छन्नेष चक्रगदाधरः}
{वेणुदारिवृतां भार्यां प्रममाथ युधिष्ठिर}


\twolineshloka
{पर्याप्तां पृथिवीं सर्वां साश्वां सरथकुञ्जराम्}
{वेणुदारिवशे युक्तां जिगाय मधुसूदनः}


\twolineshloka
{अवाप्य तपसा वीर्यं बलमोजश्च भारत}
{त्रासिताः सगणाः सर्वे बाणेन विबुधाधिपाः}


\twolineshloka
{वज्राशनिगदाबामैस्ताडयद्भिरनेकशः}
{तस्य नासीद्रणे मृत्युर्देवैरपि सवासवैः}


\twolineshloka
{सोऽभिभूतश्च कृष्णेन न हतश्च मगात्मना}
{छित्वा बाहुसहस्रं तु गोविन्देन महात्मना}


\threelineshloka
{एषोऽपीडन्महाबाहुः कंसं च मधुसूदनः}
{अवाप्तं तपसा वीर्यं बलमोजश्च भारत}
{कैटभं चातिलोमानि निजघान जनार्दनः}


\twolineshloka
{जम्बुमैरावतं चैव विरूपं च महायशाः}
{2-61-22bजघान भरतश्रेष्ठशम्बरं चारिमर्दनम्}


\twolineshloka
{एष भोगवतीं गत्वा वासुकिं भरतर्षभ}
{निर्जित्य पुण्डरीकाक्षो रौक्मिणेयममोचयत्}


\twolineshloka
{एवं बहूनि कर्माणि शिशुरेव जनार्दनः}
{कृतवान्पुण्डरीकाक्षः सङ्कर्षणसहायवान्}


\twolineshloka
{एवमेषोऽसुराणां चसुराणामपि सर्वशः}
{भयामयकरः कृष्णः सर्वलोकेश्वरः प्रभुः}


\twolineshloka
{एवमेव महाबाहुः शास्ता सर्वदुरात्मनाम्}
{कृत्वा देवार्थममितं स्वस्थानं प्राप्स्यते पुनः}


\twolineshloka
{एष भोगवतीं पुण्यां रविकान्तिं महायशाः}
{द्वारकामात्मसात्कृत्वा सागरं प्लावयिष्यति}


\twolineshloka
{सुरासुरमनुष्येषु नाभून्न भविता क्वचित्}
{यस्तामध्यवसद्राजा नान्यत्र मधुसूदनात्}


\twolineshloka
{भ्राजमानास्तु वै सर्वे वृष्ण्यन्धकमहारथाः}
{तेजिष्ठं प्रतिपत्स्यन्ते नाकपृष्टं गतासवः}


\twolineshloka
{एवमेव दशार्हाणां विधाय विधिना विधिम्}
{विष्णुर्नारायणः साक्षात्स्वस्थानं प्राप्स्यते ध्रुवम्}


\twolineshloka
{अप्रमेयोऽनियोज्यश्च यत्र कामगमो वशी}
{2-61-32ab मोदतेभगवान्प्रीतो वालः क्रीडानकैरिव}


\twolineshloka
{नैष गर्भत्वमापेदे न योन्यामावसत्प्रभुः}
{आत्मनस्तेजसा कृष्णः सर्वेषां कुरुते गतिम्}


\twolineshloka
{यथा बुद्बुद उत्थाय तत्रैव प्रविलीयते}
{चराचराणि भूतानि तथा नारायणे सदा}


\twolineshloka
{न प्रमातुं महाबाहुः शक्यो भारत केशवः}
{परं हि परतस्तस्माद्विश्वरूपान्न विद्यते}


\chapter{अध्यायः ६२}
\twolineshloka
{एवमुक्त्वा ततो भीष्णो विरराम महाबलः}
{व्याजहारोत्तरं तत्र सहदेवोऽर्थवद्वचः}


\twolineshloka
{केशवं केशिहन्तारमप्रमेयपराक्रमम्}
{पूज्यमानं मया यो वः कृष्णं न सहते नृपाः}


\twolineshloka
{सर्वेषां बलिनां मूर्ध्नि मयेदं निहितं पदम्}
{एवमुक्ते मया सम्यगुत्तरं प्रब्रवीतु सः}


\twolineshloka
{स एव हि मया वध्यो भविष्यति न संशयः}
{मतिमन्तश्च ये केचिदाचार्यं पितरं गुरुम्}


\twolineshloka
{अर्च्यमर्चितमर्घार्हमनुजानन्तु ते नृपाः}
{ततो न व्याजहारैषां कश्चिद्बुद्धिमतां सताम्}


\twolineshloka
{मानिनां बलिनां राज्ञां मध्ये वै दर्शिते पदे}
{ततोऽपतत्पुष्पवृष्टिः सहदेवस्य मूर्धनि}


\twolineshloka
{अदृश्यरूपा वाचश्चाप्यब्रुवन्साधुसाध्विति}
{अविध्यदजिनं कृष्णं भविष्यद्भूतजल्पनः}


\twolineshloka
{सर्वसंशयनिर्मोक्ता नारदः सर्वलोकवित्}
{उवाचाखिलभूतानां मध्ये स्पष्टतरं वचः}


\threelineshloka
{कृष्णं कमलपत्राक्षं नार्चयिष्यन्ति ये नराः}
{जीवन्मृतास्तु ते ज्ञेया न सभाष्याः कदाचना ॥वैशम्पायन उवाच}
{}


\twolineshloka
{पूजयित्वा च पूजार्हान्ब्रह्मक्षत्रविशेषवित्}
{सहदेवो नृणां देवः समापयत कर्म तत्}


\twolineshloka
{तस्मिन्नभ्यर्चिते कृष्णे सुनीथः शत्रुकर्षणः}
{अतिताम्रेक्षणः कोपादुवाच मनुजाधिपान्}


\twolineshloka
{स्थितः सेनापतिर्योऽहं मन्वध्वं किं तु साम्प्रतम्}
{युधि तिष्ठाम सन्नह्य समेतान्वृष्णिपाण्डवान्}


\twolineshloka
{इति सर्वान्समुत्साद्य राज्ञस्तांशअचेदिपुङ्गवः}
{यज्ञोपघाताय ततः सोऽमन्त्रतय राजभिः}


% Check verse!
तत्राहूतागताः सर्वे सुनीथप्रमुखा गणाः.समदृश्यन्त सङ्क्रुद्धा विवर्णवदनास्तथा
\twolineshloka
{युधिष्ठिराभिषेकं च वासुदेवस्य चार्हणम्}
{न स्याद्यथा तथा कार्यमेवं सर्वे तदाऽब्रुवन्}


\twolineshloka
{निष्कर्षान्निश्चयात्सर्वे राजानः क्रोधमूर्छिताः}
{अब्रुवंस्तत्र राजानो निर्वेदादात्मनिश्चयात्}


\twolineshloka
{सुहृद्भिर्वार्यमाणानां तेषां हि वपुराबभौ}
{आमिषादपकृष्टानां सिहानामिव गर्जताम्}


\twolineshloka
{तं बलौघमपर्यन्तं राजसागारमक्षयम्}
{कुर्वाणं समयं कृष्णो युद्धाय बुबुधे तदा}


\chapter{अध्यायः ६३}
\threelineshloka
{ततः सागरसङ्काशं दृष्ट्वा नृपतिमण्डलम्}
{संवर्तवाताभिहतं भीमं क्षुब्धमिवार्णवम्}
{}


\fourlineindentedshloka
{रोषात्प्रचलितं सर्वमिदमाह युधिष्ठिरः}
{भीष्मं मतिमतां मुख्यं वृद्धं कुरुपितामहम्}
{बृहस्पतिं बृहत्तेजाः पुरुहूत इवारिहा}
{}


\twolineshloka
{असौ रोषात्प्रचलितो महान्नृपतिसागरः}
{अत्र यत्प्रतिपत्तव्यं तन्मे ब्रूहि पितामह}


\twolineshloka
{यज्ञस्य च न विघ्नः स्यात्प्रजानां च हितं भवेत्}
{यथा सर्वत्र तत्सर्वं ब्रूहि मेऽद्य पितामह}


\twolineshloka
{इत्युक्तवति धर्मज्ञे धर्मराजे युधिष्ठिरे}
{उवाचेदं वचो भीष्मस्ततः कुरुपितामहः}


\twolineshloka
{मा भैस्त्वं कुरुशार्दूल श्वा सिंहं हन्तुमर्हति}
{शिवः पन्थाः सुनीतोऽत्र मया पूर्वतरं वृतः}


\twolineshloka
{प्रसुप्ते हि यथा सिंहे श्वानस्तात समागताः}
{भषेयुः सहिताः सर्वे तथेमे वसुधाधिपाः}


\twolineshloka
{वृष्णिसिंहस्य सुप्तस्य तथाऽमी प्रमुके स्थिताः}
{भषन्ते तात सङ्क्रुद्धाः श्वानः सिंहस्य सन्निधौ}


\threelineshloka
{न हि सम्बुध्यते यावत्सुप्तः सिंह इवाच्युतः}
{` तदिदं ज्ञातपूर्वं हि तव संस्तोतुमिच्छसि'}
{तेन सिंहीकरोत्येतानसिंहश्चेदिपुङ्गवः}


\twolineshloka
{पार्थिवान्पार्थिवश्रेष्ठ शिशुपालोऽल्पचेतनः}
{सर्वान्सर्वात्मना तात नेतुकामो यमक्षयम्}


\twolineshloka
{नूनमेतत्समादातुं पुनरिच्छत्यधोक्षजः}
{यदस्य शिशुपालस्य तेजस्तिष्ठति भारत}


\twolineshloka
{विप्लुता चास्य भद्रं ते बुद्धिर्बुद्धिमतां वर}
{चेदिराजस्य कौन्तेय सर्वेषां च महीक्षितम्}


\twolineshloka
{आदातुं च नरव्याघ्रो यं यमिच्छत्ययं तदा}
{तस्य विप्लवते बुद्धिरेवं चेदिपतेर्यथा}


\threelineshloka
{चतुर्विधानां भूतानां त्रिषु लोकेषु माधवः}
{प्रभवश्चैव सर्वेषां निधनं च युधिष्ठिर ॥वैशम्पायन उवाच}
{}


\twolineshloka
{इति तस्य वचः श्रुत्वा ततश्चेदिपतिर्नृपः}
{भीष्मं रूक्षाक्षरा वाचः श्रावयामास भारत}


\chapter{अध्यायः ६४}
\twolineshloka
{बिभीषिकाभिर्बह्वीभिर्भीषयन्भीष्म पार्थिवान्}
{न व्यपत्रपसे कस्माद्वृद्धः सन्कुलपांसनः}


\twolineshloka
{युक्तमेतत्तृतीयायां प्रकृतौ वर्तता त्वया}
{वक्तुं धर्मादपेतार्थं त्वं हि सर्वकुरूत्तमः}


\twolineshloka
{नावि नौरिव सम्बद्धा यथान्धो वान्धमन्वियात्}
{तथाभूता हि कौरव्या येषां भीष्म त्वमग्रणीः}


\twolineshloka
{पूतनाघातपूर्वाणि कर्माण्यस्य विशेषतः}
{त्वया कीर्तयताऽस्माकं भूयः प्रव्यथितं मनः}


\twolineshloka
{अवलिप्तस्य मूर्शस्य केशवं स्तोतुमिच्छतः}
{कथं भीष्म न ते जिह्वा शतधेयं विदीर्यते}


\twolineshloka
{यत्र कुत्सा प्रयोक्तव्या भीष्म बालतरैर्नरैः}
{तमिमं ज्ञानवृद्धः सन्गोपं संस्तोतुमिच्छसि}


\twolineshloka
{यद्यनेन हता बाल्ये शकुनिश्चित्रमत्र किम्}
{तौ वाऽश्ववृषभौ भीष्ण यौ न युद्धविशारदौ}


\twolineshloka
{चेतनारहितं काष्ठं यद्यनेन निपातितम्}
{पादेन शकटं भीष्ण तत्र किं कृतमद्भुतम्}


\twolineshloka
{`अर्कप्रमाणौ तौ वृक्षौ यद्यनेन निपातितौ}
{'नागश्च दमितोऽनेन तत्र को विस्मयः कृतः'}


\twolineshloka
{वल्मीकमात्रः सप्ताहं यद्यनेन धृतोऽचलः}
{तदा गोवर्धनो भीष्म न तच्चित्रं मतं मम}


\twolineshloka
{भुक्तमेतेन बह्वन्नं क्रीडता नगमूर्धनि}
{इति ते भीष्ण शृण्वानाः परे विस्मयमागताः}


\threelineshloka
{यस्य चानेन धर्मज्ञ भुक्तमन्नं बलीयसः}
{स चानेन हतः कंस इत्येतत्तु बलीयसः}
{}


\twolineshloka
{स चानेन हतः कंस इत्येतत्तु महाद्भुतम् ॥न ते श्रुतमिदं भीष्म नूनं कथयतां सताम्}
{}


\twolineshloka
{यद्वक्ष्ये त्वामधर्मज्ञं वाक्यं कुरुकुलाधम ॥स्त्रीषु गोषु न शस्त्राणि पातयेद्ब्राह्मणेषु च}
{}


\twolineshloka
{इति सन्तोऽनुशासन्ति सञ्जना धर्मिणः सदा}
{भीष्म लोके हि तत्सर्वं वितथं त्वयि दृश्यते}


\twolineshloka
{ज्ञानवृद्धं च वृद्धं च भूयांसं केशवं मम}
{अजानत इवाख्यासि संस्तुवन्कौरवाधम}


\twolineshloka
{गोघ्रः स्त्रीघ्नश्च सन्भीष्म त्वद्वाक्याद्यदि पूज्यते}
{एवम्भूतश्च यो भीष्म कथं संस्तवमर्हति}


\threelineshloka
{असौ मतिमतां श्रेष्ठो य एष जगतः प्रभुः}
{सम्भावयति चाप्येवं त्वद्वाक्याच्च जनार्दनः}
{एवमेतत्सर्वमिति तत्सर्वं वितथं ध्रुवम्}


\twolineshloka
{आत्मानमात्मनाऽऽधातुं यदि शक्तो जनार्दनः}
{अकामयन्तं तं भीष्म कथं साध्विव पश्यसि}


\twolineshloka
{न गाथा गाथिनं शास्ति बहुचेदपि गायति}
{प्रकृतिं यान्ति भूतानि कुलिङ्गशकुनिर्यथा}


\twolineshloka
{नूनं प्रकृतिरेषा ते जघन्या नात्र संशयटः}
{`नदीसुतत्वात्ते चित्तं चञ्चलं न स्थिरं स्मृतम्'}


\twolineshloka
{अतः पापीयसी चैषां पाण्डवानामपीष्यते}
{येषामर्च्यतमः कृष्णस्त्वं च येषां प्रदर्शकः}


\twolineshloka
{धर्मवांस्त्वमधर्मज्ञः सतां मार्गादवप्लुतः}
{को हि धर्मिणमात्मानं जानञ्ज्ञानविदां वरः}


\twolineshloka
{कुर्याद्यथा त्वया भीषम कृतं धर्ममवेक्षता}
{चेत्त्वं धर्मं विजानासि यदि प्राज्ञा मतिस्तव}


\twolineshloka
{अन्यकामा हि धर्मज्ञा कन्यका प्राज्ञमानिना}
{अम्बा नामेति भद्रं ते कथं साऽपहृता त्वया}


\twolineshloka
{तां त्वयाऽपहृतां भीष्म कन्यां नैषितवान्नृपः}
{भ्राता विचित्रवीर्यस्ते सतां मार्गमनुस्मरन्}


\twolineshloka
{भार्ययोर्यस्य चान्येन मिषतः प्राज्ञमानिनः}
{तव जातान्यपत्यानि सज्जनाचरिते पथि}


\twolineshloka
{को हि धर्मोऽस्ति ते भीषम ब्रह्मचर्यमिदं वृथा}
{यद्धारयसि मोहाद्वा क्लीबत्वाद्वा न संशयः}


\twolineshloka
{न त्वं तव धर्मज्ञ पश्याम्युपचरं क्वचित्}
{न हि ते सेविता वृद्धा य एवं धर्ममब्रवीः}


\twolineshloka
{इष्टं दत्तमधीतं च यज्ञाश्च बहुदक्षिणाः}
{सर्वमेतदपत्यस्य कलां नार्हन्ति षोडशीम्}


\twolineshloka
{व्रतोपवासैर्बहुभिः कृतं भवति भीष्म यत्}
{सर्वं तदनपत्यस्य मोघं भवति निश्चयात्}


\twolineshloka
{सोऽनपत्यश्च वृद्धश्च मिथ्याधर्मानुशासनात्}
{हंसवत्त्वमपीदानीं ज्ञातिभ्यः प्राप्नुया वधम्}


\twolineshloka
{एवं हि कथयन्त्यन्ये नरा ज्ञानविदः पुरा}
{भीष्म यत्तदहं सम्यग्वक्ष्यामि तव शृण्वतः}


\twolineshloka
{वृद्धः किल समुद्रान्ते कश्चिद्धंसोऽभवत्पुरा}
{धर्मवागन्यथावृत्तः पक्षिणः सोऽनुशास्ति च}


\twolineshloka
{धर्म चरत माऽधर्ममिति तस्य वचः किल}
{पक्षिणः शुश्रुवुर्भीष्म सततं धर्मवादिनः}


\twolineshloka
{हंसस्य तु वचः श्रुत्वा मुदिताः सर्वपक्षिणः}
{ऊचुश्चैव स्वगा हंसं परिवार्य च सर्वशः}


\twolineshloka
{कथयस्व भवान्सर्वं पक्षिणां तु समासतः}
{को हि नाम द्विजश्रेष्ठ ब्रूहि नो धर्म उत्तमः ॥हंस उवाच}


\twolineshloka
{प्रजास्वहिंसा धर्मो वै हिंसाऽधर्मः खगव्रजाः}
{एतदेवानुबोद्धव्यं धर्माधर्मः समासतः ॥शिशुपाल उवाच}


\twolineshloka
{वृद्धहंसवचः श्रुत्वा पक्षिणस्ते सुसंहिताः}
{ऊचुश्च धर्मलुब्धास्ते स्मयमाना इवाण्डजाः}


\twolineshloka
{धर्मं यः कुरुते नित्यं लोके धीरतरोऽण्डजः}
{स यत्र गच्छेद्धर्मात्मा तन्मे ब्रूहीह तत्त्वतः ॥हंस उवाच}


\threelineshloka
{बाला यूयं न जानीध्वं धर्मसूक्ष्मं विहङ्गमाः}
{धर्मं यः कुरुते लोके सततं शुभबुद्धिना}
{न चायुषोऽन्ते स्वं देहं त्यक्त्वा स्वर्गं स गच्छति}


\twolineshloka
{तथाऽहमपि च त्यक्त्वा काले देहमिमं द्विजाः}
{स्वर्गलोकं गमिष्यामि इयं धर्मस्य वै गतिः}


\twolineshloka
{एवं धर्मकथां चक्रे स हंसः पक्षिणां भृशम्}
{पक्षिणः शुश्रुवुर्भीष्म सततं धर्ममेव ते}


\twolineshloka
{अथास्य भक्ष्यमाजह्रुः समुद्रजलचारिणः}
{अण्डजा भीष्म तस्यान्ये धर्मार्थमिति शुश्रुम}


\twolineshloka
{ते च तस्य समभ्याशे निक्षिप्याण्डानि सर्वशः}
{समुद्राम्भस्यमोदन्त चरन्तो भीष्म पक्षिणः}


\twolineshloka
{तेषामण्डानि सर्वेषां भक्षयामास पापकृत्}
{स हंसः सम्प्रमत्तानामप्रमत्तः स्वकर्मणि}


\twolineshloka
{ततः प्रक्षीयमाणेषु तेषु तेष्वण्डजोऽपरः}
{अशङ्कत महाप्राज्ञः स कदाचिद्ददर्श ह}


\twolineshloka
{ततः सङ्कथयामास दृष्ट्वा हंसस्य किल्बिषम्}
{तेषां परमदुःखार्तः स पक्षी सर्वपक्षिणाम्}


\twolineshloka
{ततः प्रत्यक्षतो दृष्ट्वा पक्षिणस्ते समीपगाः}
{निजघ्नस्तं तदा हंसं मिथ्यावृत्तं कुरूद्वह}


\twolineshloka
{एवं त्वां हंसधर्माणमपीमे वसुधाधिपाः}
{निहन्युर्भीष्म सङ्क्रुद्धाः पक्षिणस्तं यथाण्डजम्}


\twolineshloka
{गाथामप्यत्र गायन्ति ये पुराणविदो जनाः}
{भीष्म यां तां च ते सम्यक्वथयिष्यामि भारत}


\twolineshloka
{अन्तरात्मन्यभिहते रौषि पत्ररथाशुचि}
{अण्डभक्षणकर्मैतत्तव वाचमतीयते}


\chapter{अध्यायः ६५}
\twolineshloka
{स मे बहुमतो राजा जरासन्धो महाबलः}
{योऽनेन युद्धं नेयेष दाक्षोऽयमिति संयुगे}


\twolineshloka
{केशवेन कृतं कर्म जरासन्धवधे तदा}
{भीमसेनार्जुनाभ्यां च कस्तत्साध्विति मन्यते}


\twolineshloka
{उद्वारेण प्रविष्टेन छद्मना ब्रह्मवादिना}
{दृष्टः प्रभावः कृष्णेन जरासन्धस्य भूपतेः}


\twolineshloka
{येन धर्मात्मनाऽऽत्मानं ब्रह्मण्यमभिजानता}
{प्रेषितं पाद्यमस्मै तद्दातुमग्रे दूरात्मने}


\twolineshloka
{भुज्यतामिति तेनोक्ताः कृष्णबीमधनञ्जयाटः}
{जरासन्धेन कौरव्य कृष्णेन विकृतं कृतम्}


\twolineshloka
{यद्ययं जगतः कर्ता यथैनं मूर्ख मन्यसे}
{कस्मान्न ब्राह्मणं सम्यगात्मानमवगच्छति}


\twolineshloka
{इदं त्वाश्चर्यभूतं मे यदिभे पाण्डवास्त्वया}
{अपकृष्टाः सतां मार्गान्मन्यन्ते तच्च साध्विति}


\twolineshloka
{अथवा नैतदाश्चर्यं येषां त्वमसि भारत}
{स्त्रीसधर्मा च वृद्धश्च सर्वार्थानां प्रदर्शकः ॥वैशम्पायन उवाच}


\twolineshloka
{तस्य तद्वचनं श्रुत्वा रूक्षं रूक्षाक्षरं बहु}
{चकोप बलिनां श्रेष्ठो भीमसेनः प्रतापवान्}


\twolineshloka
{तथा पद्मप्रतीकाशे स्वभावायतविस्तृते}
{भूयः क्रोधाभिताम्राक्षे रक्ते नेत्रे बभूवतुः}


\twolineshloka
{त्रिशिखां भ्रकुटीं चास्य ददृशुः सर्वपार्थिवाः}
{ललाटस्थां त्रिकूटस्थां गङ्गां त्रिपथगामिव}


\twolineshloka
{दन्तान्सन्दशतस्तस्य कोपाद्ददृशुराननम्}
{युगान्ते सर्वभूतानि कालस्येव जिघत्सतः}


\twolineshloka
{उत्पतन्तं तु वेगेन जग्राहैनं मनस्विन्}
{भीष्म एव महाबाहुर्महासेनमिवेश्वरः}


\twolineshloka
{तस्व भीमस्य भीष्मेण वार्यमाणस्य भारत}
{गुरुणा विविधैर्वाक्यैः क्रोधः प्रशममागतः}


\twolineshloka
{नातिचक्राम भीष्मस्य स हि वाक्यमरिन्दमः}
{समुद्वृत्तो घनापाये वेलामिव महोदधिः}


\twolineshloka
{शिशुपालस्तु सङ्क्रुद्धे भीमसेने जनाधिप}
{नाकम्पत तदा वीरः पौरुषे व्यवस्थितः}


\twolineshloka
{उत्पतन्तं तु वेगेन पुनः पुनररिन्दमः}
{न स तं चिन्तयामास सिंहः क्रुद्धो मृगं यथा}


\twolineshloka
{प्रहसंश्चाब्रवीद्वाक्यं चेदिराजः प्रतापवान्}
{भीमसेनमभिक्रुद्धं दृष्ट्वा भीमपराक्रमम्}


\twolineshloka
{मुञ्चैनं भीष्म पश्यन्तु यावदेनं नराधिपः}
{मत्प्रभावविनिर्दग्धं पतङ्गमिव वह्निना}


\twolineshloka
{ततश्चेदिपतेर्वाक्यं श्रुत्वा तत्कुरुसत्तमः}
{भीमसेनमुवाचेदं भीष्मे मतिमतां वरः}


\twolineshloka
{`नैषा चेदिपतेर्बुद्धिर्यत्त्वामाह्वयतेऽच्युतम्}
{भीमसेन महाबाहो कृष्णस्यैव विनिश्चयः'}


\chapter{अध्यायः ६६}
\twolineshloka
{चेदिराजकुले जातख्यक्ष एष चतुर्भुजः}
{रासभारावसदृशं ररास च ननाद च}


\twolineshloka
{तेनास्य मातापितरौ त्रेसतुस्तौ सबान्धवौ}
{वैकृतं तस्यत तौ दृष्ट्वा त्यागायाकुरुतां मतिम्}


\twolineshloka
{ततः सभार्यं नृपतिं सामात्यं सपुरोहितम्}
{चिन्तासंमूढहृदयं वागुवाचाशरीरेणी}


\twolineshloka
{एष ते नृपते पुत्रः श्रीमाञ्जातो बलाधिकः}
{तस्मादस्मान्न भेतव्यमव्यग्रः पाहि वै शिशुम्}


\twolineshloka
{न च वै तस्य मृत्युस्त्वं न कालः प्रत्युपस्थितः}
{यश्च शस्त्रेण हन्ताऽस्य स चोत्पन्नो नराधिप}


\twolineshloka
{संश्रुत्योदाहृतं वाक्यं भूतमन्तर्हितं ततः}
{पुत्रस्नेहाभिसन्तप्ता जननी वाक्यमब्रवीत्}


\twolineshloka
{येनेदमीरितं वाक्यं ममैतं तनयं प्रति}
{प्राञ्जलिस्तं नमस्यामि ब्रवीतु स पुनर्वचः}


\twolineshloka
{याथातथ्येन भगवान्देवो वा यदि वेतरः}
{श्रोतुमिच्छामि पुत्रस्य कोऽस्य मृत्युर्भविष्यति}


\twolineshloka
{अन्तर्भूतं ततो भूतमुवाचेदं पुनर्वचः}
{यस्योत्सङ्गे गृहीतस्य भुजावभ्यधिकावुभौ}


\twolineshloka
{पतिष्यतः क्षितितले पञ्चशीर्षाविवोरगौ}
{तृतीयमेतद्बालस्य ललाटस्थं तु लोचनम्}


\twolineshloka
{निमज्जिष्यति यं दृष्ट्वा सोऽस्य मृत्युर्भविष्यति}
{त्र्यक्षं चतुर्भुजं श्रुत्वा तथा च समुदाहृतम्}


\twolineshloka
{पृथिव्यां पार्थिवाः सर्वे अभ्यागच्छन्दिदृक्षवः}
{तान्पूजयित्वा सम्प्राप्तान्यथार्हं स महीपतिः}


\twolineshloka
{एकैकस्य नृपस्याङ्के पुत्रमारोपयत्तदा}
{एवं राजसहस्राणा पृथक्त्वेन यथाक्रमम्}


\twolineshloka
{शिशुरङ्के समारूढो न तत्प्राय निदर्शनम्}
{एतदेव तु संश्रुत्य द्वारवत्यां महाबलौ}


\twolineshloka
{ततश्चेदिपुरं प्राप्तौ सङ्कर्षणजनार्दनौ}
{यादवौ यादवीं द्रुष्टुं स्वसारं तौ पितुस्तदा}


\twolineshloka
{अभिवाद्य यथान्यायं यथाश्रेष्ठं नृपं च ताम्}
{कुशलानामयं पृष्ट्वा निषण्णौ रामकेशवौ}


\twolineshloka
{साऽभ्यर्च्य तौ तदा वीरौ प्रीत्या चाभ्यधिकं ततः}
{पुत्रं दामोदरोत्सङ्गे देवी संन्यदधात्स्वयम्}


\twolineshloka
{न्यस्तमात्रस्य तस्याङ्के भुजावभ्यधिकावुभौ}
{पेततुस्तच्च नयनं न्यमज्जत ललाटजम्}


\twolineshloka
{तद्दृष्ट्वा व्यथिता त्रस्ता वरं कृष्णमयाचत}
{ददस्व मे वरं कृष्ण भयार्ताया महाभुज}


\twolineshloka
{त्वं ह्यार्तानां समाश्वासो भीतानामभयप्रदः}
{एवमुक्तस्ततः कृष्णः सोऽब्रवीद्यदुनन्दनः}


\twolineshloka
{मा भैस्त्वं देवि धर्मज्ञे न मत्तोऽस्ति भयं तव}
{ददामि कं वरं किं च करवाणि पितृष्वसः}


\twolineshloka
{शक्यं वा यदि वाऽशक्यं करिष्याणि वचस्तव}
{एवमुक्ता ततः कृष्णमब्रवीद्यदुनन्दनम्}


\threelineshloka
{शिशुपालस्यापराधान्क्षमेथास्त्वं महाबल}
{मत्कृते यदुशार्दूल विद्ध्येनं मे वरं प्रभो ॥कृष्ण उवाच}
{}


\threelineshloka
{अपराधशतं क्षाम्यं मया ह्यस्य पितृष्वसः}
{पुत्रस्य ते वधार्हस्य मा त्वं शोके मनः कृथाः ॥भीष्म उवाच}
{}


\threelineshloka
{`स जानन्नात्मनो मृत्युं कृष्णं यदुसुखावहम्'}
{एवमेष नृपः पापः शिशुपाः सुमन्दधीः}
{त्वां समाह्वयते वीर गोविन्दवरदर्पितः}


\chapter{अध्यायः ६७}
\threelineshloka
{नैषा चेदिपतेर्बुद्धिर्यया त्वाह्वयतेऽच्युतम्}
{नूनमेव जगद्भर्तुः कृष्णस्यैव विनिश्चयः}
{}


\twolineshloka
{को हि मां भीमसेनाद्य क्षितावर्हति पार्थिवः}
{क्षेप्तुं कालपरीतात्मा यथैष कुलपांसनः}


\twolineshloka
{एष ह्यस्य महाबाहुस्तेर्जोशश्च हरेर्ध्रुवम्}
{तमेव पुनरादातुं कुरुतेऽत्र मतिं हरिः}


\twolineshloka
{येनैष कुरुशार्दूल शार्दूल इव चेदिराट्}
{गर्जत्यतीव दुर्बुद्धिः सर्वानस्मानचिन्तयन् ॥वैशम्पायन उवाच}


\twolineshloka
{ततो न ममृषे चैद्यस्तद्भीष्मवचनं तदा}
{उवाच चैन सङ्क्रुद्धः पुनर्भीष्ममथोत्तरम् ॥शिशुपाल उवाच}


\twolineshloka
{द्विषतां नोऽस्तु भीष्मैष प्रभावः केशवस्य यः}
{यस्य संस्तववक्ता त्वं बन्दिवत्सततोत्थितः}


\twolineshloka
{संस्तवे चमनो भीष्म परेषां रमते यदि}
{तदा संस्तुहि राज्ञस्त्वमिमं हित्वा जनार्दनम्}


\twolineshloka
{दरदं स्तुहि बाह्लीकमिमं पार्थिवसत्तमम्}
{जायमानेन येनेयभवद्दारिता मही}


\twolineshloka
{वङ्गाङ्गविषयाध्यक्षं सहस्राक्षसमं बले}
{स्तुहि कर्णमिमं भीष्म महाचापविकर्षणम्}


\twolineshloka
{यस्येमे कुण्डले दिव्ये सहजे देवनिर्मिते}
{कवचं च महाबाहो बालार्कसदृशप्रभम्}


\twolineshloka
{वासवप्रतिमो येन जरासन्धोऽतिदुर्जयः}
{विजितो बाहुयुद्धेन देहभेदं च लम्भितः}


\twolineshloka
{द्रोणं द्रौणिं च साधु त्वं पितापुत्रौ महारथौ}
{स्तुहि स्तुत्यावुभौ भीष्म सततं द्विजसत्तमौ}


\twolineshloka
{ययोरन्यतरो भीष्म सङ्क्रुद्धः सचराचराम्}
{इमां वसुमतीं कुर्यान्निः शेषामिति मे मतिः}


\twolineshloka
{द्रोणस्य हि समं युद्धे न पश्यामि नराधिपम्}
{नाश्वत्थाम्नः समं भीष्म न च तौ स्तोतुमिच्छसि}


\twolineshloka
{पृथिव्यां सागरान्तायां यो वैप्रतिसमो भवेत्}
{दुर्योधनं त्वं राजेन्द्रमतिक्रम्य महाभुजम्}


\threelineshloka
{जयद्रथं च राजानं कृतास्त्रं दृढविक्रमम्}
{द्रुमं किम्पुरुषाचार्यं लोके प्रथितविक्रमम्}
{अतिक्रम्य महावीर्यं किं प्रशंससि केशवम्}


\twolineshloka
{वृद्धं च भरताचार्यं तथा शारद्वतं कृपम्}
{अतिक्रम्य महावीर्यं किं प्रशंससि केशवम्}


\twolineshloka
{धनुर्धराणां प्रवरं रुक्मिणं पुरुषोत्तमम्}
{अतिक्रम्य महावीर्यं किं प्रशंससि केशवम्}


\twolineshloka
{भीष्मकं च महावीर्यं दन्तवक्त्रं च भूमिपम्}
{भगदत्तं यूपकेथु जयत्सेनं च मागधम्}


\twolineshloka
{विराटद्रुपदौ चोभौ शकुनिं च बहद्बलम्}
{विन्दानुविन्दावावन्त्यौ पाण्ड्यं श्वेतमथोत्तमम्}


\twolineshloka
{शङ्खं च सुमहाभागं वृषसेनं च मानिनम्}
{एकलव्यं च विक्रान्तं कालिङ्गं च महारथम्}


\fourlineindentedshloka
{अतिक्रम्य महावीर्यं किं प्रशंसति केशवम्}
{शल्यादीनपि कस्मात्त्वं न स्तौषि वसुधाधिपान्}
{स्तवाय यदि ते बुद्धिर्वर्तते भीष्म वसुधाधिपान्}
{}


\twolineshloka
{किं हि शक्यं मया कर्तुं यद्वृद्धानां त्वया नृप}
{पुरा कथयतां नूनं न श्रुतं धर्मवादिनाम्}


\twolineshloka
{आत्मनिन्दात्मपूजा च परनिन्दा परस्तवः}
{अनाचरितमार्याणामिति ते भीष्म न श्रुतम्}


\twolineshloka
{यदस्तव्यमिमं शश्वन्मोहात्संस्तौषि भक्तितः}
{केशवं तच्च ते भीष्म न कश्चिदनुमन्यते}


\twolineshloka
{कथं भोजस्य पुरुषे वत्सपाले दुरात्मनि}
{समावेशयसे सर्वं जगत्केवलकाम्यया}


\twolineshloka
{अथ चैषा न ते बुद्धिः प्रकृतिं याति भारत}
{मयैव कथितं पूर्वं कुलिङ्गशकुनिर्यथा}


\twolineshloka
{कुलिङ्गशकुनिर्नाम पार्श्वे हिमवतः परे}
{भीष्म तस्याः सदा वाचः श्रूयन्तेऽर्थविगर्हिताः}


\twolineshloka
{मा साहसमितीदं सा सततं वाशते किल}
{साहसं चात्मनातीव चरन्ती नावबुध्यते}


\twolineshloka
{सा हि मांसार्गलं भीष्म मुखात्सिंहस्य खादतः}
{दन्तान्तरविलग्नं यत्तदादत्तेऽल्पचेतना}


\twolineshloka
{इच्छतः सा हि सिंहस्य भीष्म जीवत्यसंशयम्}
{तद्वत्त्वमप्यधर्मिष्ठ सदा वाचः प्रभाषसे}


\twolineshloka
{इच्छतां भूमिपालानां भीष्म जीवस्यसंशयम्}
{लोकविद्विष्टकर्मा हि नान्योऽस्ति भवता समः ॥वैशम्पायन उवाच}


\twolineshloka
{ततश्चेदिपतेः श्रुत्वा भीष्मः स कटुकं वचः}
{उवाचेदं वचो राजंश्चेदिराजस्य शृण्वतः}


\twolineshloka
{इच्छतां किल नामाहं जीवाम्येषां महीक्षिताम्}
{सोऽहं न गणयाम्येतांस्तृणेनापि नराधिपान्}


\twolineshloka
{एवमुक्ते तु भीष्मेण ततः सञ्चुक्रुशुर्नृपाः}
{केचिज्जहृषिरे तत्र केचिद्भीष्मं जगर्हिरे}


\twolineshloka
{केचिदूचुर्महेष्वासाः श्रुत्वा भीष्मस्य यद्वचः}
{पापोऽवलिप्तो वृद्धश्च नायं भीष्मोऽर्हति क्षमाम्}


\twolineshloka
{हन्यतां दुर्मतिर्भीष्मः पशुवत्साध्वयं नृपाः}
{सर्वैः समेत्य संरब्धैर्दह्यतां वा कटाग्निना}


\twolineshloka
{इति तेषां वचः श्रुत्वा ततः कुरुपितामहः}
{उवाच मतिमान्भीष्मस्तानेव वसुधाधिपान्}


\twolineshloka
{उक्तस्योक्तस्य नेहान्तमहं समुपलक्षये}
{यत्तु वक्ष्यामि तत्सर्वं शृणुध्वं वसुधाधिपाः}


\twolineshloka
{पशुवद्घातनं वा मे दहनं वा कटाग्निना}
{क्रियतां मूर्ध्नि वो न्यस्तं मयेदं सकलं पदम्}


\twolineshloka
{एष तिष्ठति गोविन्दः पूजितोऽस्माभिरच्युतः}
{यस्य वस्त्वरते बुद्धिर्मरणाय स माधवम्}


\twolineshloka
{कृष्णमाह्वयतामद्य युद्धे चक्रगदाधरम्}
{यादवस्यैव देवस्य देहं विशतु पातितः}


\chapter{अध्यायः ६८}
\twolineshloka
{वचः श्रुत्वैव भीष्मस्य चेदिराडुरुविक्रमः}
{युयुत्सुर्वासुदेवेन वासुदेवमुवाच ह}


\twolineshloka
{आह्वये त्वां रणं गच्छ मया सार्धं जनार्दन}
{यावदद्य निहन्मि त्वां सहितं सर्वपाण्डवैः}


\twolineshloka
{सह त्वया हि मे वध्याः सर्वथा कृष्ण पाण्डवाः}
{नृपतीन्समतिक्रम्य यैरराजा त्वमर्चितः}


\twolineshloka
{ये त्वां दासमराजानं बाल्यादर्चन्ति दुर्मतिम्}
{अनर्हमर्हवत्कृष्ण वध्यास्त इति मे मतिः}


\twolineshloka
{इत्युक्त्वा राजशार्दूल `शार्दूल इव नादयन्}
{पश्यतां सर्वभूतानां शिशुपालः प्रतापवान्}


\twolineshloka
{स रणायैव सङ्क्रुद्धः सन्नद्धः सर्वराजभिः}
{सुनीथः प्रययौ क्षिप्रं पार्थयज्ञजिघांसया}


\threelineshloka
{ततश्चक्रगदापाणिः केशवः केशिहा हरिः}
{सध्वजं रथमास्थाय दारुकेण सुसत्कृतम्}
{भीष्मेण दत्तहस्तोऽसावारुहोह रथोत्तमम्}


\twolineshloka
{तेन पापस्वभावेन कोपितान्सर्वपार्थिवान्}
{आससाद रणे कृष्णः सज्जितैकरथः स्थितः}


\twolineshloka
{ततः पुष्करपत्राक्षं तार्क्ष्यध्वजरथे स्थितम्}
{दिवाकरमिवोद्यन्तं ददृशुः सर्वपार्थिवाः}


\twolineshloka
{आरोपयन्तं ज्यां कृष्णं प्रतपन्तमिवौजसा}
{स्थितं पुष्परथे दिव्ये पुष्पकेतुमिवापरम्}


\twolineshloka
{दृष्ट्वा कृष्णं तथा यान्तं प्रतपन्तमिवौजसा}
{यथार्हं केशवे वृत्तिमवशाः प्रतिपेदिरे}


\twolineshloka
{तानुवाच महाबाहुर्महाऽसुरनिबर्हणः}
{वृष्णिवीरस्तदा राजन्सान्त्वयन्परवीरहा ॥श्रीभगवानुवाच}


\twolineshloka
{अपेत सबलाः सर्व आस्वस्ता मम शासनात्}
{मा दृष्टो दूषयेत्पाप एष वः सर्वपार्थिपाः}


\twolineshloka
{एष नः शत्रुरत्यन्तमेष वृष्णिविमर्दनः}
{सात्वतां सात्वतीपुत्रो वैरं चरति शाश्वतम्'}


\twolineshloka
{प्राग्ज्योतिषपुरं यातानस्माञ्ज्ञात्वा नृशंसकृत्}
{अदहद्द्वारकामेष स्वस्त्रीयः सन्नराधिपाः}


\twolineshloka
{क्रीडतो भोजराजस्य एव रैवतके गिरौ}
{हत्वा बध्वा च तान्सार्वानुपायात्स्वपुरं पुरा}


\twolineshloka
{अश्वमेधे हयं मेध्यमुत्सृष्टं रक्षिभिर्वृतम्}
{पितुर्मे यज्ञविघ्नार्थमहरत्पापनिश्चयः}


\twolineshloka
{सौवीरान्प्रतियातां च बभ्रोरेष तपस्विनः}
{भार्यामभ्यहरन्मोहादकामां तामितो गताम्}


\twolineshloka
{एष मायाप्रतिच्छन्नः कारूशार्थे तपस्विनीम्}
{जहार भद्रां वैशालीं मातुलस्य नृशंसकृत्}


\twolineshloka
{वृष्णिदारान्विलाप्यैव हत्वा च कुकुरान्धकान्}
{पापाबुद्धिरुपातिष्ठत्स प्रविश्य ससम्भ्रमम्}


\twolineshloka
{विशालराज्ञो दुहितां मम पित्रा वृतां सतीम्}
{अनेन कृत्वा सन्धानं करूशेन जिगीषया}


\twolineshloka
{जरासन्धं समाश्रित्य कृतवान्विप्रियाणि मे}
{तानि सर्वाणि सङ्ख्यातुं न शक्नोमि नराधिपाः}


\twolineshloka
{एवमेतदपर्यन्तमेष वृष्णिषु किल्बिषी}
{अस्माकमयमारम्भांश्चकार परभानृजुः}


\twolineshloka
{शतं क्षन्तव्यमस्माभिर्वधार्हाणां किलागसाम्}
{बद्धोऽस्मि समयैर्घोरैर्मातुरस्यैव सङ्गरे}


\twolineshloka
{तत्तथा शतमस्माकं क्षान्तं क्षयकरं मया}
{द्वौ तु मे वधकालेऽस्मिन्न क्षन्तव्यौ कथञ्चन}


\twolineshloka
{यज्ञविघ्नकरं हन्यां पाण्डवानां च दुर्हृदम्}
{इति मे वर्तते भावस्तमतीयां कथं न्वहम्}


\twolineshloka
{पितृष्वसुः कृते दुःखं सुमहन्मर्षयाम्यहम्}
{दिष्ट्या हीदं सर्वराज्ञां सन्निधावद्य वर्तते}


\twolineshloka
{पश्यन्ति हि भवन्तोऽद्य मय्यतीव व्यतिक्रमम्}
{कृतानि तु परोक्षं मे यानि तानि निबोधत}


\twolineshloka
{इमं त्वस्य न शक्ष्यामि क्षन्तुमद्य व्यतिक्रमम्}
{अवलेपाद्वधार्हस्य समग्रे राजमण्डले}


\twolineshloka
{रुक्मिण्यामस्य मूढस्य प्रार्थनाऽऽसीन्मुमूर्षतः}
{न च तां प्राप्तवान्मूढः शूद्रो वेदश्रुतीमिव ॥वैशम्पायन उवाच}


\twolineshloka
{एवमादि ततः सर्वे सहितास्ते नराधिपाः}
{गर्हणं शिशुपालस्य वासुदेवेन विश्रुतः}


\twolineshloka
{वासुदेववचः श्रुत्वा चेदिराजं व्यगर्हयन्}
{रथोपस्थे धनुष्मन्तं शरान्सन्दधतं रुषा}


\twolineshloka
{श्रुत्वाऽपि च विलोक्याशु दुद्रुवुः सर्वपार्थिवाः}
{विहाय परमोद्विग्नाश्चेदिराजं चमूमुखे}


\twolineshloka
{तस्य तद्वचनं श्रुत्वा शिशुपालः प्रतापवान्}
{जहास स्वनवद्धासं वाक्यं चेदमुवाच ह}


\twolineshloka
{मत्पूर्वां रुक्मिणीं कृष्ण संसत्सु परिकीर्तयन्}
{विशेषतः पार्थिवेषु व्रीडां न कुरुषे कथम्}


\twolineshloka
{मन्यमानो हि कः सत्सु पुरुषः परिकीर्तयेत्}
{अन्यपूर्वा स्त्रियं जातु त्वदन्यो मधूसूदन}


\twolineshloka
{क्षमा वा यदि ते श्रद्धा मा वा कृष्ण मम क्षम}
{क्रुद्धाद्वापि प्रसन्नाद्वा किं मे त्वत्तो भविष्यते ॥`वैशम्पायन उवाच}


\twolineshloka
{स तांस्तु विद्रुतान्सर्वान्साश्वपत्तिरथद्विपान्}
{कृष्णतेजोहतान्सर्वान्समीक्ष्य वसुधाधिपान्}


\twolineshloka
{शिशुपालो रथेनैकः प्रत्युपायात्स केशवम्}
{रुषा ताम्रेक्षणो राजञ्छलभः पावकं यथा}


\chapter{अध्यायः ६९}
\twolineshloka
{ततो युद्धाय संनद्धं चेदिराजं युधिष्ठिरः}
{दृष्ट्वा मतिमतां श्रेष्ठो नारदं समुवाच ह ॥युधिष्ठिर उवाच}


\twolineshloka
{अन्तरिक्षे च भूमौ च तेऽस्त्यविदितं क्वचित्}
{यानि राजविनाशाय भौमानि च खगानि च}


\threelineshloka
{निमित्तानीह जायन्ते उत्पाताश्च पृथग्विधाः}
{एतदित्छामि कार्त्स्न्येन श्रोतुं त्वत्तो महामुने ॥वैशम्पायन उवाच}
{}


\twolineshloka
{इत्येवं मितमान्विप्रः कुरुराजस्य धीमतः}
{पृच्छतः सर्वमव्यग्रमाचचक्षे महायशाः ॥नारद उवाच}


\twolineshloka
{पराक्रमं च मार्गं च संनिपातं समुच्छ्रयम्}
{आरोहणं कुरुश्रेष्ठ अन्योन्यं प्रतिसर्पणम्}


\twolineshloka
{पश्मीनां व्यतिसंसर्गं व्यायामं वृत्तिपीडनम्}
{दर्शनादर्शनं चैव अदृश्यानां च दर्शनम्}


\twolineshloka
{हानिं वृद्धिं च ह्रासं च वर्णस्थानं बलाबलम्}
{सर्वमेतत्परीक्षेत ग्रहाणां ग्रहकोविदः}


\twolineshloka
{भौमाः पूर्वं प्रवर्तन्ते खेचराश्च ततः परम्}
{उत्पद्यन्ते च लोकेऽस्मिन्नुत्पाता देवनिर्मिताः}


\twolineshloka
{यदा तु सर्वभूतानां छाया न परिवर्तते}
{अपरेण गते सूर्ये तत्पराभवलक्षणम्}


\twolineshloka
{अच्छाये विमलच्छाया प्रतिच्छायेव दृश्यते}
{यत्र चैत्यकवृक्षाणां तत्र विद्यान्महद्भयम्}


\twolineshloka
{शीर्णपर्णप्रवालाश्च शुष्कपर्णाश्च चैत्यकाः}
{अपभ्रष्टप्रवालाश्च तत्राभावं विनिर्दिशेत्}


\twolineshloka
{स्निग्धपर्णप्रवालाश्च दृश्यन्ते यत्र चैत्यकाः}
{ईहमानाश्च वृक्षाश्च भावस्तत्र न संशयटः}


\twolineshloka
{पुष्पे पुष्पं प्रजायेत फले वा फलमाश्रितम्}
{राजा वा राजमात्रो वा मरणायोपपद्यते}


\twolineshloka
{प्रावृट्छरदि हेमन्ते वसन्ते वापि सर्वशः}
{आकालिकं पुष्पफलं राष्ट्रक्षोभं विनिर्दिशेत्}


\twolineshloka
{नदीनां स्त्रोतसोऽकाले द्योतयन्ति महाभयम्}
{वनस्पतिः पूज्यमानः पूजितोऽपूजितोऽपि वा}


\twolineshloka
{यदा भज्येत वातेन भिद्यते नमितोऽपि वा}
{अग्निवायुभयं विद्याच्छ्रेष्ठो वापि विनश्यति}


\threelineshloka
{दिशः सर्वाश्च दीप्यन्ते जायन्ते राजविभ्रमाः}
{भिद्यमानो यदा वृक्षो निनदेच्चापि पातितः}
{सह राष्ट्रं च पतितं नतं वृक्षं प्रपातयेत्}


\twolineshloka
{अथैनं छेदयेत्कश्चित्प्रतिक्रुद्धो वनस्पतिः}
{छेत्ता भेत्ता पतिश्चैव क्षिप्रमेव नशिष्यति}


\twolineshloka
{देवतानां च पतनं मष्टपानां च पातनम्}
{अचलानां प्रकम्पश्च तत्पराभवलक्षणम्}


\twolineshloka
{निशि चेन्द्रधनुर्दृष्टं ततोपि च महद्भयम्}
{तद्द्रष्टरेव भीतिः स्यान्नान्येषां भरतर्षभ}


% Check verse!
रात्राविन्द्रधनुर्दृष्ट्वा तद्राष्ट्रं परिवर्जयेत्
\twolineshloka
{अर्चा यत्र प्रनृत्यन्त नदन्ति च हसन्ति च}
{उन्मीलन्ति निमीलन्ति राष्ट्रक्षोभं विनिर्दिशेत्}


\twolineshloka
{शिला यदि प्रसिञ्चन्ति स्नेहांश्चोदकसम्भवान्}
{अन्यद्वा विकृतं किञ्चित्तद्भयस्य निदर्शनम्}


\twolineshloka
{म्रियन्ते वा महामात्रा राजा सपरिवारकः}
{पुरस्य या भवेद्व्याधी राष्ट्रे देशे च विभ्रमाः}


\twolineshloka
{देवतानां यदाऽऽवासे राज्ञां वा यत्र वेश्मनि}
{भाण्डागारायुधागारे निविशेत यदा मधु}


\twolineshloka
{सर्वं तदा भवेत्स्थानं हन्यमानं बलीयसा}
{आगन्तुकं भयं तत्र भवेदित्येव निर्दिशेत्}


\twolineshloka
{पादपश्चैव यो यत्र रक्तं स्रवति शोणितम्}
{दन्ताग्रात्कुञ्जरो वापि शृङ्गाद्वा वृषभस्तथा}


\twolineshloka
{पादपाद्राष्ट्रिविभ्रंशः कुञ्जराद्राजविभ्रमः}
{गोब्राह्मणविनाशः स्याद्वृवभस्येति निर्दिशेत्}


\twolineshloka
{छत्रं नरपतेर्यत्र निपतेत्पृथिवीतले}
{सराष्ट्रो नृपती राजन्क्षिप्रमेव विनश्यति}


\twolineshloka
{देवागारेषु वा यत्र राज्ञो वा यत्र वेश्मनि}
{विकृतं यदि दृश्येत नागावासेषु वा पुनः}


\twolineshloka
{तस्य देशस्य पीडा स्याद्राज्ञो जनपदस्य वा}
{अनावृष्टिभयं घोरमतिदुर्भिक्षमादिशेत्}


\twolineshloka
{अर्चाया बाहुभङ्गेन गृहस्थानां भयं भवेत्}
{भग्ने प्रहरणे विद्यात्सेनापतिविनाशनम्}


\twolineshloka
{आगन्तुका तु प्रतिमा स्थानं यत्र न विन्दति}
{जभ्यन्तरेण षण्मासाद्राजा त्यजति तत्पुरम्}


\twolineshloka
{प्रदीर्यते मही यत्र विनदत्यपि पात्यते}
{म्रियते तत्र राजा च तत्र राष्ट्रं विनश्यति}


\twolineshloka
{एणीपदान्वा सर्पान्वा डुण्डुभानथ दीप्यकान्}
{मण्डूको ग्रसते यत्र तत्र राजा विनश्यति}


\twolineshloka
{अभिन्नं वाप्यपक्वं वा यत्रान्नमुपचीयते}
{जीर्यन्ते वा म्रियन्ते वा तदन्नं नोपभुञ्जते}


\twolineshloka
{उदपाने च यत्रापो विवर्धन्ते युधिष्ठिर}
{स्थावरेषु प्रवर्तन्ते निर्गच्छेन्न पुनस्ततः}


\twolineshloka
{अपादं वा त्रिपादं वा द्विशीर्षं वा चतुर्भुजम्}
{स्त्रियो यत्र प्रसूयन्ते ब्रूयात्तत्र पराभवम्}


\twolineshloka
{अजैडकाः स्त्रियो गावो ये चान्ये च वियोनयः}
{विकृतानि प्रजायन्ते तत्र तत्र पराभवः}


\twolineshloka
{नदी यत्र प्रतिस्रोता आवहेत्कलुषोदकम्}
{दिशश्च न प्रकाशन्ते तत्पराभवलक्षणम्}


\twolineshloka
{एतानि च निमित्तानि यानि चान्यानि भारत}
{केशवादेव जायन्ते भौमानि च खगानि च}


\twolineshloka
{चन्द्रादित्यौ ग्रहाश्चैव नक्षत्राणि च भारत}
{वायुरग्निस्तथा चापः पृथिवी च जनार्दनात्}


\twolineshloka
{यस्य देशस्य हानिं वा वृद्धिं वा कर्तुमिच्छति}
{तस्मिन्देशे निमित्तानि तानि तानि करोत्ययम्}


\twolineshloka
{सोसौ चेदिपतेस्तात विनाशं समुपस्थितम्}
{निवेदयति गोविन्दः स्वैरुपायैर्न संशयः}


\twolineshloka
{इयं प्रचलिता भूमिरशिवा वान्ति मारुताः}
{राहुश्चाप्यपतत्सोममपर्वणि विशाम्पते}


\twolineshloka
{सनिर्घाताः पतन्त्युल्कास्तमः सञ्जायते भृशम्}
{चेदिराजविनाशाय हरिरेष विजृम्भते ॥वैशम्पायन उवाच}


\threelineshloka
{एवमुक्त्वा तु देवर्षिर्नारदो विरराम ह}
{ताभ्यां पुरुषसिंहाभ्यां तस्मिन्युद्ध उपस्थिते}
{}


\twolineshloka
{ददृशुर्भूमिपालास्ते घोरानौत्पातिकान्बहून् ॥तत्र वै दृश्यमानानां दिक्षु सर्वासु भारत}
{}


\twolineshloka
{अश्रूयन्त तदा राजञ्छिवानामशिवा रवाः ॥ररास च मही कृत्स्ना सवृक्षवनपर्वता}
{}


\twolineshloka
{अपर्वणि च मध्याह्ने मूर्यं स्वर्भानुरग्रसत् ॥ध्वजाग्रे चेदिराजस्य सर्वरत्नपरिष्कृते}
{}


\twolineshloka
{अपतत्खाच्च्युतो गृध्रस्तीक्ष्णतुण्डः परन्तप ॥आरण्यैः सहसा हृष्टा ग्राम्याश्च मृगपक्षिणः}
{}


\twolineshloka
{चुक्रुशुर्भैरवं तत्र तस्मिन्युद्ध उपस्थिते ॥एवमादिनि घोराणि भौमानि च स्वगानि च}
{}


% Check verse!
औत्पातिकान्यदृश्यन्त सङ्क्रुद्धे शार्ङ्गधन्वनि
\chapter{अध्यायः ७०}
\twolineshloka
{ततो विष्फारयन्राजा महच्चैदिपतिर्धनुः}
{अभियास्यन्हृषीकेशमुवाच मधुसूदनम्}


\twolineshloka
{एकस्त्वमसि मे शत्रुस्तत्त्वां हत्वाऽद्य माधव}
{ततः सागरपर्यन्तां पालयिष्यामि मेदिनीम्}


\fourlineindentedshloka
{द्वैरथं काङ्क्षितं यद्वै तदिदं पर्युपस्थितम्}
{चिरस्य वत मे ----वासुदेव सह त्वया}
{अद्य त्वां निहनिष्यामि भीष्मं च सह पाण्डवैः ॥वैशम्पायन उवाच}
{}


% Check verse!
-----
% Check verse!
----
% Check verse!
-------
% Check verse!
-------
% Check verse!
------ सुनीयं परवीरहा ॥-----
\threelineshloka
{अयं त्वं भामकस्तीक्ष्णश्चेदिराज महाशरः}
{भेत्तुमर्हति वेगेन महाशनिरिवाचलम् ॥वैशम्पायन उवाच}
{}


\twolineshloka
{एवं ब्रुवति गोविन्दे ततश्चेदिपतिः पुनः}
{मुमोच निशितानन्यान्कृष्णं प्रति शरान्बहून्}


\twolineshloka
{अथ बाणार्दितः कृष्णः शार्ङ्गमायम्य दीप्तिमान्}
{मुमोच निशितान्बाणाञ्छतशोथ सहस्रशः}


\twolineshloka
{ताञ्छरांस्तु स चिच्छेद शरवर्षैस्तु चेदिराट्}
{षड्भिश्चान्यैर्जघानाशु केशवं चेदिपुङ्गवः}


\twolineshloka
{ततोऽस्रं सहसा कृष्णः प्रमुमोच जगद्गुरुः}
{अस्त्रेण तन्महाबाहुर्वारयामास चेदिराट्}


\twolineshloka
{ततः शतसहस्रेण शराणां नतपर्वणाम्}
{सर्वतः समवाकीर्य शौरिं दामोदरं तदा}


\twolineshloka
{ननाद बलवान्क्रुद्धः शिशुपालः प्रतापवान्}
{इदं चोवाच संरब्धः केशवं परवीरहा ॥शिशुपाल उवाच}


\twolineshloka
{अद्याङ्गं मामका बाणा भेत्स्यन्ति तव संयुगे}
{हत्वा त्वां समुतामात्यं पाण्डवांश्च तरस्विनः}


\twolineshloka
{अनृण्यमद्यय यास्यामि जरासन्धस्य धीमतः}
{कंसस्य केशिनश्चैव नरकस्य तथैव ह ॥वैशम्पायन उवाच}


\twolineshloka
{इत्युक्त्वा क्रोधताम्राक्षः शिशुपालो जनार्दनम्}
{अदृश्यं शरवर्षेण सर्वतः स चकार ह}


\twolineshloka
{ततोऽस्त्रेणैव चान्योन्यं निकृत्य च शरान्बहून्}
{शरवर्षैस्तदा चैद्यमन्तर्धातुं प्रचक्रमे}


\twolineshloka
{अन्तर्धानगतौ वीरौ शुशुभाते महारथौ}
{तौ दृष्ट्वा सर्वभूतानि साधुसाध्वित्यपूजयन्}


\twolineshloka
{न दृष्टपूर्वमस्माभिर्युद्धमीदृशकं पुरा}
{ततः कृष्णं जघानाशु शुशुपालस्त्रिभिः शरैः}


\threelineshloka
{कृष्णोऽपि बाणैर्विव्याध सुनीथं पञ्चभिर्युधि}
{ततः सुनीथं सप्तत्या नाराचैर्दयद्बली}
{}


\twolineshloka
{ततोऽतिविद्धः कृष्णेन सुनीथः क्रोधमूर्छितः}
{विव्याध निशितैर्बाणैर्वासुदेवं स्तनान्तरे}


\twolineshloka
{पुनः कृष्णं त्रिभिर्विद्ध्वा ननादावसरे नृपः}
{ततोऽतिदारुणं युद्धं सहसा चक्रतुस्तदा}


\twolineshloka
{नौ नखैरिव शार्दूलौ दन्तैरिव महागजौ}
{दंष्ट्राभिरिव पञ्चास्यौ चरणैरिव कुक्कुटौ}


\twolineshloka
{दारयेतां शरैस्तीक्ष्णैरन्योन्यं युधि तावुभौ}
{ततो मुमुचतुः क्रुद्धौ शरवर्षमनुत्तमम्}


\twolineshloka
{शरैरेव शराञ्छित्वा तावुभौ पुरुषर्षभौ}
{चक्रातेऽस्त्रमयं युद्धं घोरं तदतिमानुषम्}


\twolineshloka
{आग्नेयमस्त्रं मुमुचे शिशुपालः प्रतापवान्}
{वारुणास्त्रेण तच्छ्रीघ्रं नाशयामास केशवः}


\twolineshloka
{कौबेरमस्त्रं सहसा चेदिराट् प्रमुमोच ह}
{2-70-29 कौबेरणैवसहसाऽनाशयत्तं जगत्प्रभुः}


\twolineshloka
{याम्यमस्त्रं ततः क्रुद्धो मुमुचे कालमोहितः}
{याम्येनैवास्त्रयोगेन याम्यमस्त्रं व्यनाशयत्}


\twolineshloka
{गान्धर्वेण च गान्धर्वं मानवं मानवेन च}
{वायव्येन च वायव्यं रौद्रं रौद्रेण चाभिभूः}


\twolineshloka
{ऐन्द्रमैन्द्रेण भगवान्वैष्णवेन च वैष्णवम्}
{एवमस्त्राणि कुर्वाणौ युयुधाते महाबलौ}


\twolineshloka
{ततो मायां विकुर्वाणो दमगोषसुतो बली}
{गदामुसलसंयुक्ताञ्छक्तितोमरसायकान्}


\twolineshloka
{परश्वथमुसण्डीश्च ववर्ष युधि केशवम्}
{अमोघास्त्रेण भगवान्नाशयामास केशिहा}


\twolineshloka
{शिलावर्षं महाघोरं ववर्ष युधि चेदिराट्}
{वज्रास्त्रेणाभिसङ्क्रुद्धश्चूर्णं तदकरोत्प्रभुः}


\twolineshloka
{जलवर्षं ततो घोरं व्यस़जच्चेदिपुङ्गवः}
{वायव्यास्त्रेण भगवान्व्याक्षिपच्छतशो हि तत्}


\twolineshloka
{निहत्य सर्वमायां वै सुनीतस्य जनार्दनः}
{स मुहूर्तं चकाराशु द्वन्द्वयुद्धं महारथः}


\twolineshloka
{स बाणयुद्धं कुर्वाणो भर्त्सयामास चेदिराट्}
{दमघोषसुतो धृष्टमुवाच यदुपुङ्गवम्}


\twolineshloka
{अद्य कृष्णमकृष्णं तु कुर्वन्तु मम सायकाः}
{इत्येवमुक्त्वा दुष्टात्मा शरवर्षं जनार्दने}


\twolineshloka
{मुमोच पुरुषव्याघ्रो घोरं वै चेदिपुङ्गवः}
{शरसंङ्कृत्तगात्रस्तु क्षणेन यदुनन्दनः}


\twolineshloka
{रुधिरं परिसुस्राव मदं मत्त इव द्विपः}
{न यन्ता न रथो वापि न चाश्वाः पर्वतोपमाः}


\twolineshloka
{दृश्यन्ते शरसञ्छन्नाः केशवस्य महात्मनः}
{केशवं तदवस्थं तु दृष्ट्वा भूतानि चक्रुशुः}


\twolineshloka
{दारुकस्तु तदा प्राह कृष्णं यादवनन्दनम्}
{नेदृशो दृष्टपूर्वो हि सङ्ग्रामो वै पुरा मया}


\twolineshloka
{स्थातव्यमिति तिष्ठामि त्वत्प्रभावेण माधव}
{अन्यथा न च मे प्राणा धरायेयुर्जनार्दन}


\twolineshloka
{अतः सञ्चिन्त्य गोविन्द क्षिप्रमस्य वधं कुरु}
{एवमुक्तस्तु सूतेन केशवो वाक्यमब्रवीत्}


\twolineshloka
{एष ह्यतिबलो दैत्यो हिरण्यकशिपुः पुरा}
{रिपुः सुराणामभवद्वरदानेन गर्वितः}


\twolineshloka
{तथाऽऽसीद्रावणो नाम राक्षसो ह्यतिवीर्यवान्}
{तेनैव बलवीर्येण बलं नागणयन्मम}


\twolineshloka
{अहं मृत्युश्च भविता काले काले दुरात्मनः}
{न भेतव्यं तथा सूत नैष कश्चिन्मयि स्थिते}


\twolineshloka
{इत्येवमुक्त्वा भगवान्ननर्द गरुडध्वजः}
{पाञ्चजन्यं महाशङ्खं पूरयामास केशवः}


\twolineshloka
{संमोहयित्वा भगवांश्चक्रं दिव्यं समाददे}
{चिच्छेद च सुनीथस्य शिरश्चक्रेण संयुगे'}


\twolineshloka
{स पपात महाबाहुर्वज्राहत इवाचलः}
{ततश्चेदिपतेर्देहात्तेजोऽग्र्यं ददृशुर्नृपाः}


\threelineshloka
{उत्पतन्तं महाराज गगनादिव भास्करम्}
{ततः कमलपत्राक्षं कृष्णं लोकनमस्कृतम्}
{ववन्दे तत्तदा तेजो विवेश च नराधिप}


\twolineshloka
{तदद्भुतममन्यन्त दृष्ट्वा सर्वे महीक्षितः}
{यद्विवेश महाबाहुं तत्तेजः पुरुषोत्तमम्}


\twolineshloka
{अनभ्रे प्रववर्ष द्यौः पपात ज्वलिताशनिः}
{कृष्णेन निहते चैद्ये चचाल न वसुन्धरा}


\twolineshloka
{ततः केचिन्महीपाला नाब्रुवंस्तत्र किञ्चन}
{अतीतवाक्पथे काले प्रेक्षमाणा जनार्दनम्}


\twolineshloka
{हस्तैर्हस्ताग्रमपरे प्रत्यपिंषन्नमर्षिताः}
{अपरे दशनैरोष्ठानदशन्क्रोधमूर्छिताः}


\twolineshloka
{रहश्च केचिद्वार्ष्णेयं प्रशशंसुर्नराधिपाः}
{केचिदेव सुसंरब्धा मध्यस्थास्त्वपरेऽभवन्}


\threelineshloka
{प्रहृष्टाः केशवं जग्मुः संस्तुवन्तो महर्षयः}
{ब्राह्मणाश्च महात्मानः पार्थिवाश्च महाबलाः}
{शशंसुर्निर्वृताः सर्वे दृष्ट्वा कृष्णस्य विक्रमम्}


\twolineshloka
{`सदेवगन्धर्वगणा राजानो भुवि विश्रुताः}
{प्रणामं हि हृषीकेशे प्राकुर्वत महात्मनि}


\twolineshloka
{ये त्वासुरगणाः पक्षाः सम्भूताः क्षत्रिया इह}
{ते निन्दन्ति हृषीकेशं दुरात्मानो गतायुषः}


\twolineshloka
{प्रजापतिगणा ये तु मध्यस्थाश्च महात्मनि}
{ब्रह्मर्षयश्च सिद्धाश्च गन्धर्वोरगचारणाः}


\threelineshloka
{ते वै स्तुवन्ति गोविन्दं दिव्यैर्मङ्गलसंयुतैः}
{परस्परं च नृत्यन्ति गीतेन विविधेन च}
{उपतिष्ठन्ति गोविन्दं प्रीतियुक्ता महात्मनि}


\twolineshloka
{प्रहृष्टाः केशवं जग्मुः संस्तुवन्तो महर्षयः}
{ब्राह्मणाश्चापि सुप्रीताः पाण्डवाश्च महाबलाः}


\twolineshloka
{पाण्डवस्त्वब्रवीद्भातॄन्सत्कारेण महीपतिम्}
{दमघोषात्मजं शूरं संस्कारयत मा चिरम्}


\twolineshloka
{कुरुराजवचः श्रुत्वा भ्रातरस्ते त्वरान्विताः}
{तथा च कृतवन्तस्ते भ्रातुर्वै शासनं तदा}


\twolineshloka
{चेदीनामाधिपत्ये च पुत्रं तस्याज्ञया हरेः}
{अभ्यषिञ्चत तं पार्थः सहितैर्वसुधाधिपैः}


\chapter{अध्यायः ७१}
\twolineshloka
{ततः प्रववृते यज्ञो धर्मराजस्य धीमतः}
{शान्तविघ्नार्हणक्षोभो महर्षिगणसङ्कुलः}


\twolineshloka
{तं तु यज्ञं महाबाहुरासमाप्तेर्जनार्दनः}
{ररक्ष भगवाञ्छौरिः शार्ङ्गचक्रगदाधरः}


\twolineshloka
{तस्मिन्यज्ञे प्रवृत्ते तु वाग्मिनो हेतुवादिनः}
{हेतुवादान्बहून्प्राहुः परपक्षजिगीषवः}


\twolineshloka
{ददृशुस्ते नृपतयो यज्ञस्य विधिमुत्तमम्}
{उपेन्द्रस्येव विहितं सहदेवेन भारत}


\twolineshloka
{तद्यज्ञे न्यवसन्राजन्ब्राह्मणा भृशमुत्सुकाः}
{कथयन्तः कथाः पुण्याः पश्यन्तो नटनर्तकान्}


\twolineshloka
{ददृशुस्तोरणान्यत्र हेमतालमयानि च}
{दीप्तभास्करतुल्यानि प्रदीप्तानीव तेजसा}


\twolineshloka
{स यज्ञस्तोरणैस्तत्र ग्रहैर्द्योरिव सम्बभौ}
{शय्यासनविहारांश्च सुबहून्रत्नसंवृतान्}


\twolineshloka
{घटान्पात्रीकटाहानि कलशानि समन्ततः}
{न ते किञ्चिदसौवर्णमपश्यंस्तत्र पार्थिवाः}


\twolineshloka
{भुञ्जानानां च विप्राणां स्वादुभोज्यैः पृथग्विधैः}
{अनिशं श्रूयते तत्र मुदितानां महात्मनाम्}


\twolineshloka
{दीयतां दीयतामेषां भुज्यतां भुज्यतामिति}
{एवम्प्रकाराः सञ्जल्पाः श्रूयन्ते तत्र नित्यशः}


\twolineshloka
{ओदनानां विकाराणि स्वादूनि च बहूनि च}
{विविधआनि च भक्ष्याणि पेयानि मधुराणि च}


\twolineshloka
{ददुर्द्विजानां सततं राजप्रेष्या महाध्वरे}
{पूर्णे शतसहस्रे तु विप्राणां भुञ्जतां तदा}


\twolineshloka
{स्थापितस्तत्र सञ्ज्ञार्थं शङ्खोऽध्मायत नित्यशः}
{मुहुर्मुहुः प्रणादस्तु तस्य शङ्खस्य भारत}


\twolineshloka
{उत्तमः श्रूयते शब्दः श्रुत्वा विस्मयमागमन्}
{एवं प्रवृत्ते यज्ञे तु तुष्टपुष्टजनायुते}


\twolineshloka
{अन्नस्य बहुशो राजन्नुत्सेधाः पर्वतोपमाः}
{दधिकुल्याश्च ददृशुः सर्पिषश्च ह्रदाञ्जनाः}


\twolineshloka
{जम्बूद्वीपो हि सकलो नानाजनपदायुतः}
{राजन्नदृश्यतैकस्थो राज्ञस्तस्मिन्महाक्रतौ}


\twolineshloka
{तत्र राजसहस्राणि पुरुषाणां ततस्ततः}
{गृहीत्वा धनमाजग्मुस्तस्य राज्ञो महाक्रतौ}


\twolineshloka
{राजानः स्रग्विणश्चैव संमृष्टमणिकुण्डलाः}
{तान्परीविविषुर्विप्राञ्छतशोऽथ सहस्रशः}


\twolineshloka
{विविधान्यन्नपानानि लेह्यानि विविधानि च}
{तेषां नृपोपभोग्यानि ब्राह्मणेभ्यो ददुः स्म ते}


\twolineshloka
{नानाविधानि भक्ष्याणि स्वादुपुष्पफलानि च}
{गुलानि स्वादुक्षौद्राणि ददुस्ते ब्राह्मणेषु वै}


\twolineshloka
{एतानि सततं भुक्त्वा तस्मिन्यज्ञे द्विजातयः}
{परां प्रीतिं ययुः सर्वे मोदमानास्ततस्ततः}


\twolineshloka
{एवं प्रमुदितं सर्वं बहुशो धनधान्यवत्}
{यज्ञवाटं नृपा दृष्ट्वा विस्मयं परमं ययुः}


\twolineshloka
{यथाबद्धूयमानाग्निं राजसूयं महाक्रतुम्}
{पाण्डवस्य यथाशास्त्रं जुहुवुः सर्वयाजकाः}


\twolineshloka
{व्यासधौम्यादयः सर्वे विधिवत्षोडशर्त्विजः}
{स्वस्वकर्माणि चक्रुस्ते पाण्डवस्य महाक्रतौ}


\twolineshloka
{नाषडङ्गविदत्रासीत्सदस्यो नाबहुश्रुतः}
{नाव्रतो नानुपाध्यायो न पापो नाक्षमो द्विजः}


\twolineshloka
{न तत्र कृपणः कश्चिद्दरिद्रो न बभूव ह}
{क्षुधितो दुःखितो वापि प्राकृतो वापि मानुषः}


\twolineshloka
{भोजनं भोजनार्थिभ्यो दापयामास सर्वदा}
{सहदेवो महातेजाः सततं राजशासनात्}


\twolineshloka
{संस्तरे कुशलाश्चापि सर्वकर्माणि याजकाः}
{दिवस दिवसे चक्रुर्यथाशास्त्रार्थचक्षुषः}


\twolineshloka
{ब्राह्मणा देवशास्त्रज्ञः कथाश्चक्रुश्च सर्वतः}
{रेमिरे च कथान्ते तु सर्वे तस्मिन्महाक्रतौ}


\twolineshloka
{सा वेदिर्वेदसम्पन्नैर्देवद्विजमहर्षिभिः}
{आबभासे तदा कीर्णा नक्षत्रैर्द्यौरिवामला}


\twolineshloka
{पाण्डित्यदर्शनार्थाय केचन द्विजसत्तमाः}
{तर्कार्थमागताः केचित्केचिद्विद्याभिमानिनः}


\twolineshloka
{केचिद्दिदृक्षया केचिद्भीत्या राज्ञः प्रतापिनः}
{सर्वेऽप्यवभृथस्नाता याजकाः केचन द्विजाः}


\twolineshloka
{ततो वै हेमयूपांश्च सर्वरत्नसमाचितान्}
{शोभार्थं कारयामास सहदेवो महाद्युतिः}


\twolineshloka
{ददृशुस्तोरणान्यत्र हेमतालमयानि च}
{स यज्ञस्तोरणैस्तैश्च ग्रहैर्द्योरिव सम्बभौ}


\twolineshloka
{तालानां तोरणैर्हैमैर्दान्तैरिव दिशागजैः}
{दिक्षु सर्वासु विन्यस्तैस्तेजोभिर्भास्करैर्यथा}


\twolineshloka
{सकिरीटैर्नृपैश्चैव शुशुभे तत्सदस्तदा}
{देवैर्दिव्यैश्च यक्षैश्च उरगैर्दिव्यमानुषैः}


\twolineshloka
{विद्याधरगणैः कीर्णः पाण्डवस्य महात्मनः}
{स राजसूयः शुशुभे धर्मराजस्य धीमतः}


\twolineshloka
{गन्धर्वगणसङ्कीर्णः शोभितोऽप्सरसां गणैः}
{देवैर्मुनिगणैर्यक्षैर्देवलोक इवापरः}


\twolineshloka
{स किम्पुरुषगीतैश्च किन्नरैरुपशोभितः}
{नारदश्च जगौ तत्र तुम्बुरुश्च महाद्युतिः}


\twolineshloka
{विश्वासुश्चित्रसेनस्तथाऽन्ये गीतकोविदाः}
{रमयन्ति स्म तान्सर्वान्यज्ञकर्मान्तरेष्वथ}


\twolineshloka
{तत्र चाप्सरसः सर्वाः सुन्दर्यः प्रियदर्शनाः}
{ननृतुश्च जगुश्चात्र नित्यं कर्मान्तरेष्वथ}


\twolineshloka
{इतिहासपुराणानि आख्यानानि च सर्वशः}
{ऊचुर्वै शब्दशास्त्रज्ञा नित्यं कर्मान्तरेष्वथ}


\twolineshloka
{भेर्यश्च मुरजाश्चैव मड्डुका गोमुखाश्च ये}
{शृङ्गवंशाम्बुजा वीणाः श्रूयन्ते स्म सहस्रशः}


\threelineshloka
{लोकेऽस्मिन्सर्वविप्राश्च वैश्याः शूद्रा नृपादयः}
{सर्वे म्लेच्छाः सर्वगणास्त्वादिमध्यान्तजास्तथा}
{}


\twolineshloka
{नानादेशसमुद्भूतैर्नानाजातिभिरागतैः}
{पर्याप्त इव लोकोऽयं युधिष्ठिरनिवेशने}


\twolineshloka
{भीष्मद्रोणादयः सर्वे कुरवः ससुयोधनाः}
{वृष्णयश्च समग्राश्च पाञ्चालाश्चापि सर्वशः}


\twolineshloka
{यज्ञेऽस्मिन्सर्वकर्माणि चक्रुर्दासा इव क्रतौ}
{एवं प्रवृत्तो यज्ञः स धर्मराजस्य धीमतः}


\twolineshloka
{शुशुभे च महाबाहो सोमस्येव क्रतुर्यथा}
{वस्त्राणि कम्बलांश्चैव प्रावारांश्चैव सर्वदा}


\twolineshloka
{निष्कहेमजभाण्डानि भूषणानि च सर्वशः}
{प्रददौ तत्र विप्राणां धर्मराजो युधिष्ठिरः}


\twolineshloka
{यानि तत्र महीपालैर्लब्धवान्भरतर्षभः}
{तानि सर्वाणि रत्नानि ब्राह्मणानां ददौ तदा'}


\chapter{अध्यायः ७२}
\twolineshloka
{ततः स कुरुराजस्य सर्वकर्मसमृद्धिमान्}
{यज्ञः प्रीतिकरो राजन्संबभौ विपुलोत्सवः}


\twolineshloka
{शान्तविघ्नः सुखारम्भः प्रभूतधनधान्यवान्}
{अन्नवान्बहुभक्ष्यश्च केशवेन सुरक्षितः}


\twolineshloka
{समापयामास च तं राजसूयं महाक्रतुम्}
{`कोटिसहस्रं प्रददौ ब्राह्मणानां महात्मनाम्}


\twolineshloka
{न करिष्यति तं लोके कश्चिदन्यो महीपतिः}
{याजकाः सर्वकामैश्च सततं ततृपुर्धनैः}


\twolineshloka
{ततश्चावभृथस्नातः स राजा पाण्डुनन्दनः}
{व्यासं धौम्यं वसिष्ठं च नारदं च महामुनिम्}


\threelineshloka
{सुमन्तु जैमिनिं पैलं वैशम्पायनमेव च}
{याज्ञवल्क्यं च कपिलं कपालं कौशिकं तथा}
{सर्वांश्च ऋत्विक्प्रवरान्पूजयामास सत्कृतान् ॥युधिष्ठिर उवाच}


\twolineshloka
{युष्मत्प्रसादात्प्राप्तोऽयं राजसूयो महाक्रतुः}
{जनार्दनप्रसादाद्धि सम्पूर्णो मे मनोरथः ॥वैशम्पायन उवाच}


\twolineshloka
{अथ यज्ञं समाप्यान्ते पूजयामास माधवम्}
{बलदेव च देवेशं भीष्माद्यांश्च कुरूद्वहान्'}


\twolineshloka
{ततस्त्ववभृथस्नातं धर्मात्मानं युधिष्ठिरम्}
{समस्तं पार्थिवं क्षत्रमुपगम्येदमब्रवीत्}


\twolineshloka
{दिष्ट्या वर्धसि धर्मज्ञ साम्राज्यं प्राप्तवानसि}
{आजमीढाजमीढानां यशः संवर्धितं त्वया}


\twolineshloka
{कर्मणैतेन राजेन्द्र धर्मश्च सुमहान्कृतः}
{आपृच्छामो नरव्याघ्र सर्वकामैः सूपूजिताः}


\twolineshloka
{स्वराष्ट्राणि गमिष्यामस्तदनुज्ञातुमर्हसि}
{श्रुत्वा तु वचनं राज्ञां धर्मराजो युधिष्ठिरः}


\twolineshloka
{यथार्हं पूज्य नृपतीन्भ्रातॄन्सर्वानुवाच ह}
{राजानः सर्व एवैते प्रीत्याऽस्मान्प्तमुपागताः}


\twolineshloka
{प्रस्थिताः स्वानि राष्ट्राणि मामापृच्छय परन्तपाः}
{अनुव्रजत भद्रं वो विषयान्तं नृपोत्तमान्}


\twolineshloka
{भ्रातुर्वचनमाज्ञाय पाण्डवा धर्मचारिणः}
{यथार्हं नृपतीन्सर्वानेकैकं समनुव्रजन्}


\twolineshloka
{विरायटमन्वायात्तूर्णं धृष्टह्युम्नः प्रतापवान्}
{धनञ्जयो यज्ञसेनं महात्मानं महारथम्}


\twolineshloka
{भीष्मं च धृतराष्ट्रं च भीमसेनो महाबलः}
{द्रोणं तु ससुतं वीरं सहदेवो युधां पतिः}


\twolineshloka
{नकुलः सुबलं राजन्सहपुत्रं समन्वयात्}
{द्रौपदेयाः ससौभद्राः पार्वतीयान्महारथान्}


\twolineshloka
{अन्वगच्छंस्तथैवान्यान्क्षत्रियान्क्षत्रियर्षभाः}
{एवं सुपूजिताः सर्वे जग्मुर्विप्राः सहस्रशः}


\twolineshloka
{गतेषु पार्थिवेन्द्रेषु सर्वेषु ब्राह्मणेषु च}
{युधिष्ठिरमुवाचेदं वासुदेवः प्रतापवान्}


\twolineshloka
{आपृच्छे त्वां गमिष्यामि द्वारकां कुरुनन्दन}
{राजसूयं क्रतुश्रेष्ठं दिष्ट्या त्वं प्राप्तवानसि}


\twolineshloka
{तमुवाचैवमुक्तस्तु धर्मराजो जनार्दनम्}
{तव प्रसादाद्गोविन्द प्राप्तः क्रतुवरो मया}


\twolineshloka
{क्षत्रं समग्रमपि च त्वत्प्रसादाद्वशे स्थितम्}
{उपादाय बलिं मुख्यं मामेव समुपस्थितम्}


% Check verse!
कथं त्वद्गमनार्थं मे वाणी वितरतेऽनघ
\twolineshloka
{न ह्यहं त्वामृते वीरं रतिं प्राप्नोमि कर्हचित्}
{अवश्यं चैव गन्तव्या भवता द्वारका पुरी ॥वैशम्पायन उवाच}


\twolineshloka
{एवमुक्तः स धर्मात्मा युधिष्ठिरसहायवान्}
{अभिगम्याब्रवीत्प्रीतः पृथां पृथुयशा हरिः}


\twolineshloka
{साम्राज्यं समनुप्राप्ताः पुत्रास्तेऽद्य पितृष्वसः}
{सिद्धार्था वसुमन्तश्च सा त्वं प्रीतिमवाप्नुहि}


\twolineshloka
{अनुज्ञातस्त्वया चाहं द्वारकां गन्तुमुत्सहे}
{सुभद्रां द्रौपदीं चैव सभाजयत केशवः}


\twolineshloka
{निष्क्रम्यान्तः पुरात्तस्माद्युधिष्ठिरसहायवान्}
{स्नातश्च कृतजप्यश्च ब्राह्मणान्स्वस्ति वाच्य च}


\twolineshloka
{ततो मेघवपुः प्रग्व्यं स्यन्दनं च सुकल्पितम्}
{योजयित्वा महाबाहुर्दारुकः समुपस्थितः}


\twolineshloka
{उपस्थितं रथं दृष्ट्वा तार्क्ष्यप्रवरकेतनम्}
{प्रदक्षिणमुपावृत्य समारुह्य महामनाः}


% Check verse!
प्रययौ पुण्डरीकाक्षस्ततो द्वारवतीं पुरीम्
\twolineshloka
{तं पद्भ्यामनुवव्राज धर्मराजो युधिष्ठिरः}
{भ्रातृभिः सहितः श्रीमान्वासुदेवं महाबलम्}


\twolineshloka
{ततो मुहूर्तं सङ्गृह्य स्यन्दनप्रवरं हरिः}
{अब्रवीत्पुण्डरीकाक्षः कुन्तीपुत्रं युधिष्ठिरम्}


\twolineshloka
{अप्रमत्तः स्थितो नित्यं प्रजाः पाहि विशाम्पते}
{पर्जन्यमिव भूतानि महाद्रुममिव द्विजाः}


\twolineshloka
{बान्धवास्त्वोपजीवन्तु सहस्राक्षमिवामराः}
{कृत्वा परस्परेणैव संवादं कृष्णपाण्डवौ}


\twolineshloka
{अन्योन्यं समनुज्ञाप्य जग्मतुः स्वगृहान्प्रति}
{गते द्वारवतीं कृष्णे सात्वतप्रवरे नृप}


\twolineshloka
{महादुर्योधनो राजा शकुनिश्चापि सौबलः}
{`सूतपुत्रश्च गधेयः सह दुःशासनादिभिः}


\twolineshloka
{सर्वकामगुणोपेतैरर्च्यमानास्तु भारत'}
{तस्यां सभायां दिव्यायामवसंस्तत्र पाण्डवैः}


\chapter{अध्यायः ७३}
\twolineshloka
{`अनुसंसार्य नृपतीन्पाण्डवाः पाण्डवाग्रजम्}
{अभिजग्मुर्महेष्वासा धर्मराजं युधिष्ठिरम्}


\twolineshloka
{सोऽनुमेने महाबाहुर्भातॄंश्च सुहृदस्तथा'}
{शिष्यैः परिवृतो व्यासः पुरस्तात्समपद्यत}


\twolineshloka
{सोऽभ्ययादासनात्तूर्णं भ्रातृभिः परिवारितः}
{पाद्येनासनदानेन पितामहमपूजयत्}


\twolineshloka
{अथोपविश्य भगवान्काञ्चने परमासने}
{आस्यतामिति चोवाच धर्मराजं युधिष्ठिरम्}


\twolineshloka
{अथोपविष्टं राजानं भ्रातृभिः परिवारितम्}
{उवाच भगवान्व्यासस्तत्तद्वाक्यविशारदः}


\twolineshloka
{दिष्ट्या वर्धसि कौन्तेय साम्राज्यं प्राप्य दुर्लभम्}
{वर्धिताः कुरवः सर्वे त्वया कुरुकुलोद्वह}


\twolineshloka
{आपृच्छे त्वां गमिष्यामि पूजितोऽस्मि विशाम्पते}
{एवमुक्तः स कृष्णेन धर्मराजो युधिष्ठिरः}


\twolineshloka
{अभिवाद्योपसङ्गृह्य पितामहमथाब्रवीत् ॥युधिष्ठिर उवाच}
{}


\twolineshloka
{संशयो द्विपदां श्रेष्ठ ममोत्पन्नः सुदुर्लभः}
{तस्य नान्योऽस्ति वक्ता वै त्वामृते द्विजपुङ्गव}


\twolineshloka
{उत्पातांस्त्रिविधान्प्राह नारदो भगवानृषिः}
{दिव्यांश्चैवान्तरिक्षांश्च पार्थिवांश्च पितामह}


\twolineshloka
{`सुमहच्च फलं तेषां भवितेति न संशयः'}
{अपि चैद्यस्य पतनाच्छान्तमौत्पातिकं महत् ॥वैशम्पायन उवाच}


\twolineshloka
{राज्ञस्तु वचनं श्रुत्वा पराशरसुतः प्रभुः}
{कृष्णद्वैपायनो व्यास इदं वचनमब्रवीत्}


\twolineshloka
{त्रयोदश समा राजन्नुत्पातानां फलं महत्}
{सर्वक्षत्रविनाशाय भविष्यति विशाम्पते}


\threelineshloka
{त्वामेकं कारणं कृत्वा कालेन भरतर्षभ}
{समेतं पार्थिवं क्षत्रं क्षयं यास्यति भारत}
{दुर्योधनापराधेन भीमार्जुनबलेन च}


\twolineshloka
{स्वप्नं द्रक्ष्यसि राजेन्द्र तस्मिन्काल उपस्थिते}
{तत्तेऽहं सम्प्रवक्ष्यामि तन्निबोध युधिष्ठिर}


\twolineshloka
{यान्तं द्रक्ष्यसि राजेन्द्र क्षपान्ते त्वं वृषध्वजम्}
{नीलकण्ठं भवं स्थाणुं कपालिं त्रिपुरान्तकम्}


\twolineshloka
{उग्रं रुद्रं पशुपतिं महादेवमुमापतिम्}
{हरं शर्वं वृषं शूलं पिनाकिं कृत्तिवाससम्}


\twolineshloka
{कैलासकूडप्रतिमे वृषभेऽवस्थितं शिवम्}
{निरीक्षमाणं सततं पितृराजाश्रितां दिशम्}


\twolineshloka
{एवमीदृशकं स्वप्नं द्रक्ष्यसि त्वं विशाम्पते}
{मा तत्कृते ह्यनुध्याहि कालो हि दुरतिक्रमः}


\twolineshloka
{स्वस्ति तेऽस्तु गमिष्यामि कैलासं पर्वतं प्रति}
{अप्रमत्तः स्थितो दान्तः पृथिवीं परिपालय ॥वैशम्पायन उवाच}


\twolineshloka
{एवमुक्त्वा स भगवान्कैलासं पर्वतं ययौ}
{कृष्णद्वैपायनो व्यासः सह शिष्यैः सहानुगैः}


\twolineshloka
{गते पितामहे राजा चिन्ताशोकसमन्वितः}
{निः श्वसन्नुष्णमसकृत्तमेवार्थं विचिन्तयन्}


\twolineshloka
{कथं तु दैवं शक्येत पौरुषेण प्रबाधितुम्}
{अवश्यमेव भविता यदुक्तं परमर्षिणा}


\twolineshloka
{ततोऽब्रवीन्महातेजाः सर्वान्भ्रातॄन्युधिष्ठिरः}
{श्रुतं वै पुरुषव्याघ्रा यन्मां द्वैपायनोऽब्रवीत्}


\twolineshloka
{तदा तद्वचनं श्रुत्वा मरणे निश्चिता मतिः}
{सर्वक्षत्रस्य निधने यद्यहं हेतुरीप्सितः}


\twolineshloka
{कालेन निर्मितस्तात को ममार्थोऽस्ति जीवतः}
{एवं ब्रुवन्तं राजानं फाल्गुनः प्रत्यभाषत}


\twolineshloka
{मा राजन्कश्मलं घोरं प्रविशो बुद्धिनाशनम्}
{सम्प्रधार्य महाराज यत्क्षमं तत्समाचर ॥वैशम्पायन उवाच}


\twolineshloka
{ततोऽब्रवीत्सत्यधृतिर्भ्रातॄन्सर्वान्युधिष्ठिरः}
{द्वैपायनस्य वचनं तत्रैव समचिन्तयत्}


\twolineshloka
{अद्यप्रभृति भद्रं वः प्रतिज्ञां मे निबोधत}
{त्रयोदश समास्तात को ममार्थो ऽस्ति जीवतः}


\twolineshloka
{न प्रवक्ष्यामि परुषं भ्रातॄनन्यांश्च पार्थिवान्}
{स्थितो निदेशे ज्ञातीनां योक्ष्ये तत्सुमुदाहरन्}


\twolineshloka
{एवं मे वर्तमानस्य स्वसुतेऽष्वितरेषु च}
{भेदो न भविता लोके भेदमूलो हि विग्रहः}


\twolineshloka
{विग्रहं दूरतो रक्षन्प्रियाण्येव समाचरन्}
{वाच्यतां न गमिष्यामि लोकेषु मनुजर्षभाः}


\twolineshloka
{भ्रातृर्ज्येष्ठस्य वचनं पाण्डवाः संनिशम्य तत}
{तमेव समवर्तन्त धर्मराजहिते रताः}


\twolineshloka
{संसत्सु समयं कृत्वा धर्मराड्भ्रातृभिः सह}
{पितॄंस्तर्प्य यथान्यायं देवताश्च विशाम्पते}


\twolineshloka
{कृतमङ्गलकल्यामो भ्रातृभिः पिरवारितः}
{गतेषु क्षत्रियेन्द्रेषु सर्वेषु भरतर्षभ}


\threelineshloka
{युधिष्ठिरः सहामात्यः प्रविवेश पुरोत्तमम्}
{दुर्योधनो महाराज शकुनिश्चापि सौबलः}
{सभायां समणीयायां तत्रैवास्ते नराधिप}


\chapter{अध्यायः ७४}
\twolineshloka
{वसन्दुर्योधनस्तस्यां सभायां पुरुषर्षभ}
{शनैर्ददर्श तां सर्वां सभां शकुनिना सह}


\twolineshloka
{तस्यां दिव्यानभिप्रायान्ददर्श कुरुनन्दनः}
{न दृष्टपूर्वा ये तेन नगरे नागसाह्वये}


\twolineshloka
{स कदाचित्सभामध्ये धार्तराष्ट्रो महीपतिः}
{स्फाटिकं स्थलमासाद्य जलमित्यभिशङ्कया}


\twolineshloka
{स्ववस्त्रोत्कर्षणं राजा कृतवान्बुद्धिमोहितः}
{दुर्मना विमुखश्चैव परिचक्राम तां सभाम्}


\twolineshloka
{ततः स्थले निपतितो दुर्मना व्रीडितो नृपः}
{निः श्वसन्विमुखश्चापि परिचक्राम तां सभाम्}


\twolineshloka
{ततः स्फाटिकतोयां वै स्फाटिकाम्बुजशोभिताम्}
{वापीं मत्वा स्थलमिव सवासाः प्रापतञ्जले}


\twolineshloka
{जले निपतितं दृष्ट्वा भीमसेनो महाबलः}
{जहास जहसुश्चैव किङ्कराश्च सुयोधनम्}


\twolineshloka
{वासांसि च शुभान्यस्मै प्रददू राजशासनात्}
{तथागतं तु तं दृष्ट्वा भीमसेनो महाबलः}


\twolineshloka
{अर्जुनश्च यमौ चोभौ सर्वे ते प्राहसंस्तदा}
{नामर्षयत्ततस्तेषामवहासममर्षणः}


\twolineshloka
{आकारं रक्षमाणस्तु न स तान्समुदैक्षत}
{पुनर्वसनमुत्क्षिप्य प्रतिरिष्यन्निव स्थलम्}


\threelineshloka
{आरुरोह ततः सर्वे जहसुश्च पुनर्जनाः}
{द्वारं तु पिहिताकारं स्फाटिकं प्रेक्ष्य भूमिपः}
{प्रविशन्नाहतो मूर्ध्नि व्याघूर्णित इव स्थितः}


\twolineshloka
{तादृशं च परं द्वारं स्फाटिकोरुकवाटकम्}
{विघट्टयन्कराभ्यां तु निष्क्रम्याग्रे पपात हा}


\twolineshloka
{द्वारं तु वितताकारं समापेदे पुनश्च सः}
{तद्वृत्तं चेति मन्वानो द्वारस्थानादुपारमत्}


\twolineshloka
{एवं प्रलम्भान्विविधान्प्राप्य तत्र विशाम्पते}
{पाण्डवेयाभ्यनुज्ञातस्ततो दुर्योधनो नृपः}


\twolineshloka
{अपहृष्टेन मनसा राजसूये महाक्रतौ}
{प्रेक्ष्य तामद्भुतामृद्धिं जगाम गजसाह्वयम्}


\twolineshloka
{पाण्डवश्रीप्रतप्तस्य ध्यायमानस्य गच्छतः}
{दुर्योधनस्य नृपतेः पापा मतिरजायत}


\twolineshloka
{पार्थान्सुमनसो दृष्ट्वा पार्थिवांश्च वशानुगान्}
{कृत्स्नं चापि हितं लोकमाकुमारं कुरूद्वह}


\twolineshloka
{महिमानं परं चापि पाण्डवानां महात्मनाम्}
{दूर्योधनो धार्तराष्ट्रो विवर्णः समपद्यत}


\twolineshloka
{स तु गच्छन्ननेकाग्नः सभामेकोऽन्वचिन्तयत्}
{श्रियं च तामनुपमां धर्मराजस्य धीमतः}


\twolineshloka
{प्रमत्तो धृतराष्ट्रस्य पुत्रो दुर्योधनस्तदा}
{नाभ्यभाषत्सुबलजं भाषमाणं पुनः पुनः}


\twolineshloka
{अनेकाग्रं तु तं दृष्ट्वा शकुनिः प्रत्यभाषत}
{दुर्योधन कृतोमूलं निःश्वसन्निव गच्छसि ॥दुर्योधन उवाच}


\twolineshloka
{दृष्ट्वेमां पृथिवीं कृत्स्नां युधिष्ठिरवशानुगाम्}
{जितामस्त्रप्रतापेन श्वेताश्वस्य महात्मनः}


\twolineshloka
{तं च यज्ञं तथाभूतं दृष्ट्वा पार्थस्य मातुल}
{यथा शक्रस्य देवेषु तथाभूतं महाद्युतेः}


\twolineshloka
{अमर्षेण तु सम्पूर्णो दह्यमानो दिवानिशम्}
{शुचिशुक्रागमे काले शुष्ये तोयमिवाल्पकम्}


\twolineshloka
{पश्य सात्वतमुख्येन शिशुपालो निपातितः}
{न च तत्र पुमानासीत्कश्चित्तस्य पदानुगः}


\twolineshloka
{दह्यमाना हि राजानः पाण्डवोत्थेन वह्निना}
{क्षान्तवन्तोऽपराधं ते को हि तत्क्षन्तुमर्हति}


\twolineshloka
{वासुदेवेन तत्कर्म यथाऽयुक्तं महत्कृतम्}
{सिद्धं च पाण्डुपुत्राणां प्रतापेन महात्मनाम्}


\twolineshloka
{तथाहि रत्नन्यादाय विविधानि नृपा नृपम्}
{उपातिष्ठन्त कौन्येयं वैश्या इव करप्रदाः}


\twolineshloka
{श्रियं तथागतं दृष्ट्वा ज्वलन्तीमिव पाण्डवे}
{अमर्षवशमापन्नो दह्यामि न तथोचितः}


\twolineshloka
{वह्निमेव प्रवेक्ष््यामि भक्षयिष्यामि वा विषम्}
{अपो वापि प्रवेक्ष्यामिन हि शक्ष्यामि जीवितुम्}


\twolineshloka
{को हि नाम पुमांल्लोके मर्षयिष्यति सत्ववान्}
{सपत्नानृद्ध्यतो दृष्ट्वा हीनमात्मानमेव च}


\twolineshloka
{सोऽहं न स्त्री न चाप्यस्त्री न पुमान्नापुमानपि}
{योऽहं तां मर्षयाम्यद्य तादृशीं श्रियमागताम्}


\twolineshloka
{ईश्वरत्वं पृथिव्याश्च वसुमत्तां च तादृशीम्}
{यज्ञं च तादृशं दृष्ट्वा मादृशः को न संञ्ज्वरेत्}


\twolineshloka
{अशक्तश्चैक एवाहं तामाहर्तुं नृपश्रियम्}
{सहायांश्च न पश्यामि तेन मृत्युं विचिन्तये}


\twolineshloka
{दैवमेव परं मन्ये पौरुषं च निरर्थकम्}
{दृष्ट्वा कुन्तीसुते शुद्धां श्रियं तामहतां तथा}


\twolineshloka
{कृतो यत्नो मया पूर्वं विनाशे तस्य सौबल}
{तच्च सर्वमतिक्रम्य संवृद्धोऽप्स्विव पङ्गजम्}


\threelineshloka
{तेन दैवं परं मन्ये पौरुषं च निरर्थकम्}
{धार्तराष्ट्राश्च हीयन्ते पार्था वर्धन्ति नित्यशः}
{2-74-38c`कृष्णस्तु सुमनास्तेषां विवर्धयति सम्पदः'}


\twolineshloka
{सोऽहं श्रियं च तां दृष्ट्वा सभां तां च तथाविधाम्}
{रक्षिभिश्चावहासं तं परितप्ये यथाऽग्निना}


% Check verse!
अमर्षं च समाविष्टं धृतराष्ट्रे निवेदय
\chapter{अध्यायः ७५}
\twolineshloka
{दुर्योधन न तेऽमर्षः कार्यः प्रति युधिष्ठिरम्}
{भागधेयानि हि स्वानि पाण्डवा भुञ्जते सदा}


\twolineshloka
{विधानं विविधाकारं परं तेषां विधानतः}
{अनेकैरभ्युपायैश्च त्वया न शकिताः पुरा}


\threelineshloka
{आरब्धा हि महाराज पुनः पुनररिन्दम}
{विमुक्ताश्च नरव्याघ्रा भगधेयपुरस्कृताः}
{`उत्साहवन्तः पुरुषा नावसीदन्ति कर्मसु'}


\twolineshloka
{तैर्लब्धा द्रौपदी भार्या द्रुपदश्च सुतैः सह}
{सहायाः पृथिवीपाला वासुदेवश्च वीर्यवान्}


\twolineshloka
{लब्धश्चानभिभूतार्थैः पित्र्योंशः पृथिवीपते}
{विवृद्धस्तेजसा तेषां तत्र का परिदेवना}


\twolineshloka
{धनञ्जयेन गाण्डीवमक्षय्यौ च महेषुधी}
{लब्धान्यस्त्राणि दिव्यानि तोषयित्वा हुताशनम्}


\twolineshloka
{तेन कार्मुकमुख्येन बाहुवीर्येण चात्मनः}
{कृता वशे महीपालास्तत्र का परिदेवना}


\twolineshloka
{अग्निदाहान्मयं चापि मोक्षयित्वा स दानवम्}
{सभां तां कारयामास सव्यसाची परन्तपः}


\twolineshloka
{तेन चैव मयेनोक्ताः किङ्करा नाम राक्षसाः}
{वहन्ति तां सभां भीमास्तत्र का परिदेवना}


\twolineshloka
{यच्चासहायतां राजन्नुक्तवानसि भारत}
{तन्मिथ्या भ्रातरो हीमे तव सर्वे वशानुगाः}


\twolineshloka
{द्रोणस्तव महेष्वासः सह पुत्रेण वीर्यवान्}
{मूतपुत्रश्च राधेयो दृढधन्वा महारथः}


\twolineshloka
{`स एकः समरे सर्वान्पाण्डवान्सहसोमकान्}
{विजेष्यति महाबाहो किं सहायैः करिष्यसि}


\twolineshloka
{भीष्मश्च पुरुषव्याघ्रो गौतमश्च महारथः}
{जयद्रथश्च बलावान्सोमदत्तस्तथैव च'}


\threelineshloka
{अहं च सह सोदर्यैः सौमदत्तिश्च पार्थिवः}
{एतैस्त्वं सहितः सर्वैर्जय कृत्स्नं वसुन्दराम् ॥दुर्योधन उवाच}
{}


\twolineshloka
{त्वया च सहितो राजन्नेतैश्चान्यैर्महारथैः}
{एतानहं विजेष्यामि यदि त्वमनुमन्यसे}


\twolineshloka
{एतेषु विजितेष्वद्य भविष्यति मही मम}
{सर्वे च पृथिवीपालाः सभा सा च महाधना ॥शकुनिरुवाच}


\twolineshloka
{धनञ्जयो वासुदेवो भीमसेनो युधिष्ठिरः}
{नकुलः सहदेवश्च द्रुपदश्च सहात्मजैः}


\twolineshloka
{नैते युधि पराजेतुं शक्या देवगणैरपि}
{महारथा महेष्वासाः कृतास्त्रा युद्धदुर्मदाः}


\twolineshloka
{अहं तु तद्विजानामि विजेतुं येन शक्यते}
{युधिष्ठिरं स्वयं राजंस्तन्निबोध जुषस्व च ॥दुर्योधन उवाच}


\threelineshloka
{अप्रमादेन सुहृदामन्येषां च महात्मनाम्}
{यदि शक्या विजेतुं ते तन्ममाचक्ष्व मातुल ॥शकुनिरुवाच}
{}


\twolineshloka
{द्यूतप्रियश्च कौन्तेयो न स जानाति देवितुम्}
{समाहूतश्च राजेन्द्रो न शक्ष्यति तिवर्तितुम्}


\twolineshloka
{देवने कुशलश्चाहं न मेऽस्ति सदृशो भुवि}
{त्रिषु लोकेषु कौरव्य तं त्वं द्यूते समाह्वय}


\twolineshloka
{तस्याक्षकुशलो राजन्नादास्येऽहमसंशयम्}
{राज्यं श्रियं च तां दीप्तां त्वदर्थं पुरुषर्षभ}


\threelineshloka
{इदं तु सर्वं त्वं राज्ञे दुर्योधन निवेदय}
{अनुज्ञातस्तु ते पित्रा विजेष्ये तान्न संशयः ॥दुर्योधन उवाच}
{}


\twolineshloka
{त्वमेव करुमुख्याय धृतराष्ट्राय सौबल}
{निवेदय यथान्यायं नाहं शक्ष्ये निवेदितुम्}


\chapter{अध्यायः ७६}
\twolineshloka
{अनुभूय तु राज्ञस्तं राजसूयं सुदुर्मतिः}
{`युधिष्ठिरस्य शकुनिर्दुर्योधसुसंयुतः}


\twolineshloka
{विवेश हास्तिनपुरं दुर्योधनमतेन सः}
{वाढमित्येव शकुनिर्दृढं हृदि चकार ह}


\twolineshloka
{अस्वस्थतां चतां दृष्ट्वा धार्तराष्ट्रस्य पापकृत्}
{भारतानां च दुष्टात्मा क्षयाय हि नृपक्षयः'}


\twolineshloka
{प्रियकृन्मतमाज्ञाय पूर्वं दुर्योधनस्य तत्}
{प्रज्ञाचक्षुषमासीनं शकुनिः सौबलस्तदा}


\threelineshloka
{दुर्योधनवचः श्रुत्वा धृतराष्ट्रं जनाधिपम्}
{उपगम्य महाप्राज्ञं शकुनिर्वाक्यमब्रवीत् ॥शकुनिरुवाच}
{}


\twolineshloka
{दुर्योधनो महाराज विवर्णो हरिणः कृशः}
{दीनश्चिन्तापरश्चैव तं विद्धि मनुजाधिप}


\twolineshloka
{न वै परीक्षसे सम्यगसह्यं शत्रुसंभवम्}
{ज्येष्ठपुत्रस्य हृच्छोकं किमर्थं नावबुध्यसे}


\threelineshloka
{`एवमुक्तः शकुनिना धृतराष्ट्रो जनेश्वरः}
{दुर्योधनं समाहूयं इदं वचनमब्रवीत्' ॥धृतराष्ट्र उवाच}
{}


\twolineshloka
{दुर्योधन कृतोमूलं भृशमार्तोऽसि पुत्रक}
{श्रोतव्यश्चेन्मया सोऽर्थो ब्रूहि मे कुरुनन्दन}


\twolineshloka
{अयं त्वां शकुनिः प्राह विवर्णं हरिमं कृशम्}
{चिन्तयंश्च न पश्यामि शोकस्य तव सम्भवम्}


\twolineshloka
{ऐश्वर्यं हि महत्पुत्र त्वयि सर्वं प्रतिष्ठितम्}
{भ्रातरः सुहृदश्चैव नाचरन्ति तवाप्रियम्}


\twolineshloka
{आच्छादयसि प्रावारानश्नासि पिशितौदनम्}
{आजानेया वहन्त्यश्वाः केनासि हरिणः कृशः}


\twolineshloka
{शयनानि महार्हाणि योषितश्च मनोरमाः}
{गुणवन्ति च वेश्मानि विहाराश्च यथासुखम्}


\twolineshloka
{देवनामिव ते सर्वं वाचि बद्धं न संशयः}
{स दीन इव दुर्धर्ष कस्माच्छोचसि पुत्रक}


\twolineshloka
{`मात्रा पित्रा च पुत्रस्य यद्वै कार्यं परं स्मृतम्}
{प्राप्तस्त्वमसि तत्तात निखिलां नः कुलश्रियम्}


\twolineshloka
{उपस्थितः सर्वकामैस्त्रिदिवे वासवो यथा}
{विविधैरन्नपानैश्च प्रवरैः किं नु शोचसि}


\twolineshloka
{निरुक्तं निगमं छन्दः षडङ्गान्यस्त्रशास्त्रवान्}
{अधीती कृतविद्यस्त्वं दशव्याकरणैः कृपात्}


\twolineshloka
{हलायुधात्कृपाद्द्रोणादस्त्रविद्यामधीतवान्}
{भ्राताज्येष्ठः स्थितो राज्ये किमु शोचसि पुत्रक}


\twolineshloka
{पृथग्जनैरलभ्यं यदशनाच्छादनं बहु}
{प्रभुः सन्भुञ्जसे पुत्र संस्तुतः सूतमागधैः}


\twolineshloka
{तस्य ते विदितप्रज्ञ शोकमूलमिदं कथम्}
{लोकेस्मिञ्ज्येष्ठभागन्यस्तन्ममाचक्ष्व पृच्छतः ॥वैशम्पायन उवाच}


\twolineshloka
{तस्य तद्वचनं श्रुत्वा मन्दः क्रोधवशानुगः}
{पितरं प्रत्युवाचेदं स्वमतिं सम्प्रकाशयन् ' ॥दुर्योधन उवाच}


\twolineshloka
{अश्नाम्याच्छादये चाहं यथा कुपुरुषस्तथा}
{अमर्षं धारये चोग्रं निनीषुः कालपर्ययम्}


\twolineshloka
{अमर्षणः स्वाः प्रकृतीरभिभूय परं स्थितः}
{क्लेशान्मुमुक्षुः परजान्स वै पुरुष उच्यते}


\twolineshloka
{सन्तोषो वै श्रियं हन्ति ह्यभिमानं च भारत}
{अनुक्रोशभये चोभे यैर्वृतो नाश्नुते महत्}


\twolineshloka
{न मां प्रीणाति मद्भुक्तं श्रियं दृष्ट्वा युधिष्ठेरे}
{अतिज्वलन्तीं कौन्तेये विवर्णकरणीं मम}


\twolineshloka
{सपत्नानृद्व्यतोत्मानं हीयमानं निशाम्य च}
{अदृश्यामपि कौन्तेय श्रियं पश्यन्निवोद्यताम्}


\twolineshloka
{तस्मादहं विवर्णश्च दीनश्च हरिमः कृशः}
{अष्टाशीतिसहस्राणि स्नातका गृहमेधिनः}


\threelineshloka
{त्रिंशद्दासीक एकैको यान्बिभर्ति युधिष्ठिरः}
{दशान्यानि सहस्राणि यतीनामूर्ध्वरेतसाम्}
{भुञ्जते रुक्मपात्रीभिर्युधिष्ठिरनिवेशने}


\threelineshloka
{कदलीमृगमोकानि कृष्णश्यामारुणानि च}
{काम्भोजः प्राहिणोत्तस्मै परार्ध्यानपि कम्बलान्}
{गजयोषिद्गवाश्वस्य शतशोऽथ सहस्रशः}


\twolineshloka
{त्रिशतं चोष्ट्रवामीनां शतानि विचरन्त्युत}
{राजन्या बलिमादाय समेता हि नृपक्षये}


\twolineshloka
{पृथग्विधानि रत्नान पार्थिवाः पृथिवीपते}
{आहरन्क्रतुमुख्येऽस्मिन्कुन्तीपुत्राय भूरिशः}


\twolineshloka
{न क्वचिद्धि मया तादृग्दृष्टपूर्वो न च श्रुतः}
{यादृग्धनागमो यज्ञे पाण्डुपुत्रस्य धीमतः}


\threelineshloka
{`असत्यं चेदिदं सर्वं सञ्जयं प्रष्टुमर्हसि'}
{अपर्यन्तं धनौघं तं दृष्ट्वा शत्रोरहं नृप}
{शर्म नैवाभिगच्छामि चिन्तयानो विशाम्पते}


\twolineshloka
{ब्रह्मणा वाटधानाश्च गोमन्तः शतसङ्घशः}
{त्रिखर्वं बलिमादाय द्वारि तिष्ठन्ति वारिताः}


\twolineshloka
{कमण्डलूनुपादाय जातरूपमयाञ्शुभान्}
{त्रैखर्वाः प्रतिवेद्यास्मै लेभिरेऽथ प्रवेशनम्}


\twolineshloka
{यथैव मधु शक्राय धारयन्त्यमरस्त्रियटः}
{तदस्मै कांस्यमाहार्षीद्वारुणं कलशोदधिः}


\twolineshloka
{शङ्खप्रवरमादाय वासुदेवोऽभिषिक्तवान्}
{शक्यं रुक्मसहस्रस्य बहुरत्नविभूषितम्}


\twolineshloka
{दृष्ट्वा च मम तत्सर्वं ज्वररूपमिवाभवत्}
{गृहीत्वा तत्तु गच्छन्ति समुद्रौ पूर्वदक्षिणौ}


\threelineshloka
{तथैव पश्चिमं यान्ति गृहीत्वा भरतर्षभ}
{उत्तरं तु न गच्छन्ति विना तात पतत्रिणः}
{तत्र गत्वाऽर्जुनो दण्डमाजहारामितं धनम्}


\twolineshloka
{`कृतां बैन्दुसरै रत्नैर्मयेन स्फाटिकच्छदाम्}
{अपश्यं नलिनीं पूर्णामुदकस्येव भारत}


\twolineshloka
{उत्कर्षन्तं च वासश्च प्राहसन्मां वृकोदरः}
{किङ्कराश्च सभापाला जहसुर्भरतर्षभ}


\twolineshloka
{पित्रोरर्थे विशेषेण प्रावृण्वं तत्र जीवितम्}
{तत्र त्म यदि शक्तः स्यां घातयेयं वृकोदरम्}


% Check verse!
सपत्नेनापहासो हि स मां दहति भारत
\twolineshloka
{तत्र स्फाटिकतोयां हि स्फाटिकाम्बुजशोभिताम्}
{सभां पुष्करिणीं मत्वा पतितोऽस्मि नराधिप}


\twolineshloka
{तत्र मामहसद्भीमः सह पार्थेन सस्वरम्}
{द्रौपदी चसह स्त्रीभिः पातयन्ती मनो मम}


\twolineshloka
{क्लिन्नवस्त्रस्य च जले किङ्करा राजचोदिताः}
{ददुर्वासांसि मेऽन्यानि तच्च दुःखतरं मम}


\twolineshloka
{अस्तम्भा इव तिष्ठन्ति स्तम्भा इव सहस्रशः}
{सोहं तत्राहतो राजन्स्फटिकाभ्यन्तरे विभो}


\twolineshloka
{अद्वारेण विनिर्गच्छन्द्वारसंस्थानरूपिणा}
{अभिहत्य शिलां भूयो ललाटेनास्मि विक्षतः}


\threelineshloka
{आमृशन्निव तां दृष्ट्वा मार्गान्तरमुपाविशम्}
{इदं द्वारमिदं राजन्नद्वारमिति मां प्रति}
{अद्भुतं प्रहसन्वाक्यं बभाषे स वृकोदरः}


\twolineshloka
{स्त्रियश्च तत्र मां दृष्ट्वा जहसुस्तादृशं नृप}
{सर्वं हासकरं तेषां सदस्यानां नरर्षभ}


\twolineshloka
{न श्रुतानि न दृष्टानि यानि रत्नान मे क्वचित्}
{तानि मे तत्र दृष्टानि तेन तप्तोस्मि दुःखितः}


\twolineshloka
{हुताशनं प्रवेक्ष्यामि प्रवेक्ष्यामि महोदधिम्}
{सम्भावितस्य चाकीर्तिर्मरणादतिरिच्यते'}


% Check verse!
इदं चाद्भुतमत्रासीत्तन्मे निगदतः शृणु
\twolineshloka
{पूर्णे शतसहस्रे तु विप्राणां भुञ्जतां सदा}
{स्थापितस्तत्र सञ्ज्ञार्थं शङ्खो ध्मायति नित्यसः}


\twolineshloka
{मुहुर्मुहुः प्रणदतस्तस्य शङ्खस्य भारत}
{अनिशं शब्दमश्रौषं ततो रोमाणि मेऽहृषन्}


\twolineshloka
{पार्थिवैर्बहुभिः कीर्णमुपस्थानं दिदृक्षुभिः}
{अशोभत महाराज नक्षत्रैर्द्यैरिवामला}


\twolineshloka
{सर्वरत्नान्युपादाय पार्थिवा वै जनेश्वर}
{यज्ञे तस्य महाराज पाण्डुपुत्रस्य धीमतः}


\threelineshloka
{वैश्या इव महीपाला द्विजातिपरिवेषकाः}
{न सा श्रीर्देवराजस्य यमस्य वरुणस्य च}
{गुह्यकाधिपतेर्वापि या श्री राजन्युधिष्ठिरे}


\twolineshloka
{तां दृष्ट्वा पाण्डुपुत्रस्य श्रियं परमिक्रामहम्}
{शान्तिं न परिगच्छामि दह्यमानेन चेतसा}


\twolineshloka
{`अप्राप्य पाण्डवैश्वर्यं शमो मम न विद्यते}
{अरीन्बाणैः शाययिष्ये शयिष्ये वा हतः परैः}


\twolineshloka
{एतादृशस्य मे किं तु जीवितेन परन्तप}
{वर्धन्ते पाण्डवा राजन्वयं हि स्थितवृद्धयः' ॥शकुनिरुवाच}


\twolineshloka
{यामेतामतुलां लक्ष्मीं दृष्टवानसि पाण्डवे}
{तस्याटः प्राप्तावुपायं मे शृणु सत्यपराक्रम}


\twolineshloka
{अहमक्षेष्वभिज्ञोऽस्मि पृथिव्यामपि भारत}
{हृदयज्ञः पणज्ञश्च विशेषज्ञश्च देवने}


\twolineshloka
{द्यूतप्रियश्च कौन्तेयो न च जानाति देवितुम्}
{आहूतश्चैष्यति व्यक्तं नित्यमेवाह्वयत्स्वयम्}


\twolineshloka
{नियतं तं विजेष्यामि कृत्वा तु कपटं विभो}
{आनयामि समृद्धिं तां दिव्यां चोपाह्वयस्व तम् ॥वैशम्पायन उवाच}


\twolineshloka
{एवमुक्तः शकुनिना राजा दुर्योधनस्ततः}
{धृतराष्ट्रमिदं वाक्यमपदान्तरमब्रवीत्}


\twolineshloka
{अयमुत्सहते राजञ्श्रियमाहर्तुमक्षवित्}
{द्यूतेन पाण्डुपुत्रस्य तदनुज्ञातुमर्हसि ॥धृतराष्ट्र उवाच}


\twolineshloka
{क्षत्ता मन्त्री महाप्राज्ञः स्थितो यस्यास्मि शासने}
{तेन सङ्गम्य वेत्स्यामि कार्यस्यास्य विनिश्चयम्}


\threelineshloka
{स हि धर्मं पुरस्कृत्य दीर्घदर्शी परं हितम्}
{उभयोटः पक्षयोर्युक्तं वक्ष्यत्यर्थविनिश्चयम् ॥दुर्योधन उवाच}
{}


\twolineshloka
{निवर्तयिष्यति त्वाऽसौ यदि क्षत्ता समेष्यति}
{निवृत्ते त्वयि राजेन्द्र मरिष्येऽहमसंशयम्}


\twolineshloka
{स त्वं मयि मृते राजन्विदुरेण सुखी भव}
{भोक्ष्यसे पृथिवीं कृत्स्नां किं मया त्वं करिष्यसि ॥वैशम्पायन उवाच}


\twolineshloka
{आर्तवाक्यं तु तत्तस्य प्रणयोक्तं निशम्य सः}
{धृतराष्ट्रोऽब्रवीत्प्रेष्यन्दुर्योधनमते स्थितः}


\twolineshloka
{स्थूणासहस्रैर्बृहतीं शतद्वारां सभां मम}
{मनोरमां दर्शनीयामाशु कुर्वन्तुं शिल्पिनः}


\twolineshloka
{ततः संस्तीर्य रत्नैस्तां तक्ष्ण आनाय्य सर्वशः}
{सुकृतां सुप्रवेशां च निवेदयत मेऽशनैः}


\twolineshloka
{दूर्योधनस्य शान्त्यर्थमिति निश्चित्य भूमिपः}
{धृतराष्ट्रो महाराज प्राहिणोद्विदुराय वै}


\twolineshloka
{अपृष्ट्वा विदुरं स्वस्यन नासीत्कश्चिद्विनिश्चयः}
{द्यूते दोषांश्च जानन्स पुत्रस्नेहादकृष्यत}


\twolineshloka
{तच्छ्रुत्वा विदुरो धीमान्कलिद्वारमुपस्थितम्}
{विनाशमुखमुत्पन्नं धृतराष्ट्रमुपाद्रवत्}


\threelineshloka
{सोऽभिगम्य महात्मानं भ्राता भ्रातरमग्रजम्}
{मूर्ध्ना प्रणम्य चरणाविदं वचनमब्रवीत् ॥विदुर उवाच}
{}


\threelineshloka
{नाभिनन्दामि ते राजन्व्यवसायमिमं प्रभो}
{पुत्रैर्भेदो यथा न स्थाद््द्यूतहेतोस्तथा कुरु ॥धृतराष्ट्र उवाच}
{}


\twolineshloka
{क्षत्तः पुत्रेषु पुत्रैर्मे कलहो न भविष्यति}
{यदि देवाः प्रसादं नः करिष्यन्ति न संशयः}


\twolineshloka
{अशुभं वा शुभं वापि हितं वा यदि वाऽहितम्}
{}


\twolineshloka
{मयि सन्निहिते द्रोणे भीष्मे त्वयि च भारत}
{अनयो दैवविहितो न कथञ्चिद्भविष्यति}


\twolineshloka
{गच्छ त्वं रथमास्थाय हयैर्वातसमैर्जवे}
{खाण्डवप्रस्थमद्यैव समानय युधिष्ठिरम्}


\twolineshloka
{न वाच्यो व्यवसायो मे विदुरैतद्ब्रवीमि ते}
{दैवमेव परं मन्ये येनैतदुपपद्यते}


\twolineshloka
{इत्युक्तो विदुरो धीमान्नेदमस्तीति चिन्तयन्}
{आपगेयं महाप्राज्ञमभ्यगच्छत्सुदुः खितः}


\chapter{अध्यायः ७७}
\twolineshloka
{कथं समभवद्द्यूतं भ्रातॄणां तन्महात्ययम्}
{यत्र तद्व्यसनं प्राप्तं पाण्डवैर्मे पितामहैः}


\twolineshloka
{के च तत्र समास्तारा राजानो ब्रह्मवित्तम}
{के चैनमन्वमोदन्त के चैनं प्रत्यषेधयन्}


\threelineshloka
{विस्तरेणैतदिच्छामि कथ्यमानं त्वया द्विज}
{मूलं ह्येतद्विनाशस्य पृथिव्या द्विजसत्तम ॥सौतिरुवाच}
{}


\twolineshloka
{एवमुक्तस्ततो राज्ञा व्यासशिष्यः प्रतापवान्}
{आचचक्षेऽथ यद्वृत्तं तत्सर्वं वेदतत्त्वविद् ॥वैशम्पायन उवाच}


\twolineshloka
{एवमुक्तस्ततो राज्ञा व्यासशिष्यः प्रतापवान्}
{आचचक्षेऽथ यद्वृत्तं तत्सर्वं वेदतत्त्ववित्}


\twolineshloka
{विदुरस्य मतिं ज्ञात्वा धृतराष्ट्रोऽम्बिकासुतः}
{दुर्योधनमिदं वाक्यमुवाच विजने पुनः}


\twolineshloka
{अलं द्यूतेन गान्धारे विदुरो न प्रशंसति}
{न ह्यसौ सुमहाबुद्धिरहितं नो वदिष्यति}


\twolineshloka
{हितं हि परमं मन्ये विदुरो यत्प्रभाषते}
{क्रियतां पुत्र तत्सर्वमेतन्मन्ये हितं तव}


\threelineshloka
{देवर्षिर्वासवागुरुर्देवराजाय धीमते}
{यत्प्राह शास्त्रं भगवान्बृहस्पतिरुदारधीः}
{तद्दे विदुरः सर्वं सरहस्यं महाकविः}


\twolineshloka
{स्थितस्तु वचने तस्य सदाऽहमपि पुत्रक}
{विदुरो वापि मेधावी कुरूणां प्रवरो मतः}


\twolineshloka
{उद्धवो वा महाबुद्धिर्वृष्णीनामर्चितो नृप}
{तदलं पुत्र द्यूतेन द्यूते भेदो हि दृश्यते}


% Check verse!
भेदे विनाशो राज्यस्य तत्पुत्र परिवर्जय
\twolineshloka
{पित्रा मात्रा च पुत्रस्य यद्वै कार्यं परं स्मृतम्}
{प्राप्तस्त्वमसि तन्नाम पितृपैतामहं पदम्}


\twolineshloka
{अधीतवान्कृती शास्त्रे लालितः सततं गृहे}
{भ्रातृज्येष्ठः स्थितो राज्ये विन्दसरे किं न शोभनम्}


\twolineshloka
{पृथग्जनैरलभ्यं यद्भोजनाच्छादनं परम्}
{तत्प्राप्तोसि महाबाहो कस्माच्छोनसि पुत्रक}


\twolineshloka
{स्फीतं राष्ट्रं महाबाहो पितृपैतामहं महत्}
{नित्यमाज्ञापयन्भासि दिवि देवेश्वरो यथा}


\twolineshloka
{`यादृशं च तवैश्वर्यं तदन्येषां सुदुर्लभम्}
{ये चोपभोगास्ते राजन्मया ते परिकीर्तिताः'}


\twolineshloka
{तस्य ते विदितप्रज्ञ शोकमूलमिदं कथम्}
{समुत्थितं दुःखकरं तन्मे शंसितुमर्हसि ॥दुर्योधन उवाच}


\twolineshloka
{अश्नाम्याच्छादयामीति प्रपश्यन्हीनपौरुषः}
{नामर्षं कुरुते यस्तु पुरुषः सोऽधमः स्मृतः}


\twolineshloka
{न मां प्रीणाति राजेन्द्र लक्ष्मीः साधारणी विभो}
{ज्वलितामेव कौन्येये श्रियं दृष्ट्वा च विव्यथे}


\threelineshloka
{सर्वां च पृथिवीं चैव युधिष्ठिरवशानुगाम्}
{स्थिरोऽस्मि योऽहं जीवामि दुःखादेतद्ब्रवीमि ते}
{}


\twolineshloka
{आवर्जिता इवाभान्ति नीपाश्चित्रककौकुराः}
{कारस्कारा लोहजङ्घा युधिष्ठिरनिवेशने}


\twolineshloka
{हिमवत्सागरानुपाः सर्वे रत्नाकरास्तथा}
{अन्त्याः सर्वे पर्युदस्ता युधिष्ठिरनिवेशने}


\twolineshloka
{ज्येष्ठोऽयमिति मां मत्वा श्रेष्ठश्चेति विशाम्पते}
{युधिष्ठिरेण सत्कृत्य युक्तो रत्नपरिग्रहे}


\twolineshloka
{उपस्थितानां रत्नानां श्रेष्ठानामर्घहारिणाम्}
{नादृश्यत परः पारो नापरस्तत्र भारत}


\twolineshloka
{न मे हस्तः समभवद्वसु तत्प्रतिगृह्णतः}
{अतिष्ठन्त मयि श्रान्ते गृग्य दूराहृतं वसु}


\twolineshloka
{कृतां बिन्दुसरोरत्नैर्मयेन स्फाटिकच्छदाम्}
{अपश्यं नलिनीं पूर्णामुदकस्येव भारत}


\twolineshloka
{वस्त्रमुत्कर्षति मयि प्राहसत्स वृकोदरः}
{शत्रोर्ऋद्धिविशेषेण विमूढ रत्तवर्जितम्}


\twolineshloka
{तत्र स्म यदि शक्तः स्यं पातयेऽहं वृकोदरम्}
{यदि कुर्यां समारम्भं भीमं हन्तुं नराधिप}


\twolineshloka
{शिशुपाल इवास्माकं गतिः स्यान्नात्र संशयः}
{सपत्नेनावहासो मे स मां दहति भारत}


\twolineshloka
{पुनश्च तादृशीमेव वापीं जलजशालिनीम्}
{मत्वा शिलासमां तोये पतितोऽस्मि नराधिप}


\twolineshloka
{तत्र मां प्राहसत्कृष्णः पार्थेन सह सुस्वरम्}
{द्रौपदी च सह स्त्रीभिर्व्यथयन्ती मनो मम}


\twolineshloka
{क्लिन्नवस्त्रस्य तु जले किङ्करा राजनोदिताः}
{ददुर्वासांसि मेऽन्यानि तच्च दुःखं परं मम}


\threelineshloka
{प्रलम्भं च शृणुष्वान्यद्वदतो मे नराधिप}
{अद्वारेण विनिर्गच्छन्द्वारसंस्थानरूपिणा}
{अभिहत्य शिलां भूयो ललाटेनास्मि विक्षतः}


\twolineshloka
{तत्र मां यमजौ दूरादालोक्याभिहतं तदा}
{बाहुभिः परिगृह्णीतां शोचन्तौ सहितावुभौ}


\twolineshloka
{उवाच सहदेवस्तु तत्र मां विस्मयन्निव}
{इद द्वारं धार्तराष्ट्र मा गच्छेति पुनः पुनः}


\twolineshloka
{भीमसेनेन तत्रोक्तो धृतराष्ट्रात्मजेति च}
{सम्बोध्य प्रहसित्वा च इतो द्वारं नराधिप}


\twolineshloka
{नामधेयानि रत्नानां पुरस्तान्न श्रुतानि मे}
{यानि दृष्टानि मे तस्यां मनस्तपति तच्च मे}


\chapter{अध्यायः ७८}
\twolineshloka
{यन्मया पाण्डवोयानां दृष्टं तच्छृणु भारत}
{आहृतं भूमिपालैर्हि वसु मुख्यं ततस्ततः}


\twolineshloka
{नाविदं मूढमात्मानं दृष्ट्वाहं तदरेर्धनम्}
{फलतो भूमितो वाऽपि प्रतिपद्यस्व भारत}


\twolineshloka
{और्णान्बैलान्वार्षदंशान् जातरूपपरिष्कृतान्}
{प्रावाराजिनमुख्यांश्च काम्भोजः प्रददौ बहून्}


\twolineshloka
{अश्वांस्तित्तिरिकल्माषांस्त्रिशतं शुकनासिकान्}
{अष्ट्रवामीस्त्रिगर्ताश्च पुष्टाः पीलुशमीङ्गुदैः}


\twolineshloka
{गोपाः स्वीयेन सहितास्तदादाय चतुष्पदम्}
{वसातयोऽन्यद्द्रव्यं द्वारि तस्यावतस्थिरे}


\threelineshloka
{कमण्डलूनुपादाय जातरूपमयञ्छिवान्}
{रत्नानि च हिरण्यं च सुवर्णं चैव केवलम्}
{}


\twolineshloka
{प्रीयमाणः प्रसन्नात्मा स्वयं स्वजनसंवृतः}
{त्रैखर्वो रथमुख्येशः पाण्डवाय न्यवेदयत्}


\twolineshloka
{यश्च स द्विजमुख्येन राज्ञः शङ्खो निवेदितः}
{प्रीत्या दत्तः कुविन्देन धर्मराजाय धीमते}


\twolineshloka
{तं सर्वे भ्रातरो भ्रात्रे ददुः शङ्खं किरीटिने}
{तं प्रत्यगृह्णाद्बीभत्सुस्तोयजं हेममालिनम्}


\twolineshloka
{चित्रं निष्कसहस्रेण भ्राजमानं स्वतेजसा}
{रुचिरं दर्शनीयं च पूजितं विश्वकर्मणा}


\twolineshloka
{अधारयच्च धर्मश्च तं नमस्य पुनः पुनः}
{योऽनादनेऽपि नदति स ननादाधिकं तदा}


\twolineshloka
{प्रणादाद्भूमिपास्तस्य पेतुर्हीनाः खतेजसा}
{धृष्टद्युम्नः पाण्डवाश्च सात्यकिः केशवोऽष्टमः}


\twolineshloka
{सत्वेन स्वेन सम्पन्ना अन्योन्यप्रियकारिणः}
{विसञ्ज्ञान्भूमिपान्दृष्ट्वा मां च ते प्राहसंस्तदा}


\twolineshloka
{ततः प्रहृष्टो बीभत्सुरददाद्धेमशृङ्गिणः}
{शताननडुहान्पञ्च द्विजमुख्याय भारत}


\twolineshloka
{सुमुखेन बलिर्मुख्यः प्रेषितोऽजातशत्रवे}
{कुविन्देन हिरण्यं च वासांसि विविधानि च}


\twolineshloka
{काश्मीरराजो मार्द्वीकं शुद्धं च सरसं मधु}
{बलिं च कुत्स्नमादाय पाण्डवायाभ्युटपागमत्}


\twolineshloka
{यवना हयानुपादाय पार्वतीयान्मनोजवान्}
{आसनानि महार्हाणि कम्बलांश्च महाधनान्}


\twolineshloka
{नवान्सूक्ष्मांश्च हृद्यांश्च परार्थ्यान्सुप्रदर्शनान्}
{अन्यच्च विविधं रत्नं द्वारि ते न्यवतस्थिरे}


\twolineshloka
{श्रुतायुरपि कालिङ्गो मणिरत्नमनुत्तमम्}
{अङ्गः स्त्रियो दर्शनीया जातरूपविभूषिताः}


\twolineshloka
{वङ्गो जाम्बूनदमयान्पर्यङ्गाञ्छतशो नृप}
{दक्षिणात्सागराभ्याशात्प्रावारांश्च परश्शतम्}


\twolineshloka
{औदकानि सरत्नानि बलिं चादाय भारत}
{अन्येभ्यो भूमिपालेभ्यः पाण्डवाय न्यवेदयत्}


\twolineshloka
{दार्दुरं चन्दनं मुख्यं भारं षण्णवति द्रुतम्}
{पाण्डवाय ददौ पाण्ड्यः शङ्खांस्तावत एव च}


\twolineshloka
{चन्दनागरु चानन्तं मुक्तावैडूर्यचित्रिताः}
{चोलश्च केरलश्चोमौ ददतुः पाण्डवाय वै}


\twolineshloka
{अश्मको हेमशृङ्गीश्च दोग्ध्रीर्हेमविभूषितः}
{सवत्साः कुम्भदोहाश्च सहस्राण्यददाद्दश}


\twolineshloka
{सैन्धवानां सहस्राणि हयानां पञ्चविंशतिम्}
{अददात्सैन्धवो राजा हेममाल्यैरलङ्कृतान्}


\twolineshloka
{सौवीरो हस्तिभिर्युक्तान्रथांश्च त्रिशतं परान्}
{जातरूपपरिष्कारान्मणिरत्नविभूषितान्}


\twolineshloka
{मध्यन्दिनार्कप्रतिमांस्तेजसा ज्वलितानिव}
{बलिं च कृत्स्नमादाय पाण्डवाय न्यवेदयत्}


\twolineshloka
{अवन्तिराजो रत्नानि विविधानि सहस्रशः}
{हाराङ्गदांश्च मुख्यान्वै विविधं च विभूषणम्}


\twolineshloka
{दासीनामयुतं चापि बलिमादाय भारत}
{सभाद्वारि नरश्रेष्ठ दिदृक्षुरवतिष्ठते}


\threelineshloka
{दशार्णश्चेदिराजश्च शूरसेनश्च वीर्यवान्}
{वस्त्राणि मुख्यान्यादाय रत्नानि विविधानि च}
{बलिं च कृत्स्नमादाय पाण्डवाय न्यवेदयत्}


\twolineshloka
{काशिराजेन हृष्टेन बली राज्ञि निवेदितः}
{अशीतिगोसहस्राणि शतान्यष्टौ च दन्तिनाम्}


\twolineshloka
{अयुतं च नदीजानां हयानां हेममालिनाम्}
{विविधानि च रत्नानि काशिराजो बलिं ददौ}


\twolineshloka
{कृतक्षणश्च वैदेहः कौसलश्च बृहद्बलः}
{ददतुर्वाजिमुख्यांश्च सहस्राणि चतुर्दश}


\twolineshloka
{शैब्यो वसादिभिः सार्धं त्रिगर्तो मालवैः सह}
{तेभ्यो रत्नानि ददतुरेकैको भूमिपोऽमितम्}


\twolineshloka
{हारान्मुख्यान्परार्ध्यांश्च विविधं च विभूषणम्}
{शतं दासीसहस्राणि कार्पासिकनिवासिनाम्}


\twolineshloka
{श्यामास्तन्वीर्दीर्घकेशीर्हेमाभरणभूषिताः}
{बलिं च कृत्स्नमादाय भारुकच्छो नरर्षभ}


\twolineshloka
{शुद्धान्विप्रोत्तमार्हांश्च कम्बलप्रवरान्ददौ}
{ते सर्वे पाण्डुपुत्रस्य द्वार्यतिष्ठन्दिदृक्षवः}


\twolineshloka
{उपायनं यदा दद्युस्तदा द्वारमलभ्यत}
{इन्द्रकृष्टैर्वर्धयन्ति धान्यैर्नदमुखैस्तु ये}


\twolineshloka
{समुद्रनिकटे जाताः परिसिन्धुनिवासिनः}
{ते वै द्रुमाः पारदाश्च काश्यकैरातकैः सह}


\twolineshloka
{बलिं विविधमादाय रत्नानि विविधानि च}
{अजाविकं गोहिरण्यं खरोष्ट्रं फलवन्मधु}


\twolineshloka
{कम्बलान्विविधांश्चैव द्वारि तिष्ठन्ति वारिताः}
{प्राग्ज्योतिषपतिः शूरो म्लेच्छानामधिपो बली}


\twolineshloka
{यवनैः सहितो राजा भगदत्तो महाबलः}
{आजानेयान्हयाञ्छीघ्रमादायानिलरंहसः}


\twolineshloka
{बलिं च कृत्स्नमादाय द्वारि तिष्ठति वारितः}
{अश्वसारमयान्भाण्डाञ्छुभान्दन्तत्सरूनसीन्}


\twolineshloka
{प्राग्ज्योतिषाधिपो दत्त्वा भगदत्तोऽव्रजत्तदा}
{व्यक्षाङ्ख्यक्षांल्ललाटाक्षान्नानादिग्भ्यः समागतान्}


\twolineshloka
{औष्णीषआनहयांश्चैव बाहुकान्पुरुषादकान्}
{एकपादांश्च तत्राहमपश्यं द्वारि वारितान्}


\twolineshloka
{बल्यर्थं ददतस्तस्य हिरण्यं रजतं वसु}
{इन्द्रगोपकसङ्काशाञ्छुकवर्णान्मनोजवान्}


\twolineshloka
{तथैवेन्द्रायुधनिभान्सन्ध्याभ्रसदृशानपि}
{अनेकवर्णानारण्यान्गृहीत्वाश्वांस्तथा बहून्}


\twolineshloka
{जातरूपमनर्घ्यं च ददुस्तस्यैकपादकाः}
{सिंहलश्च तदा राजा परिगृह्य धनं बहु}


\twolineshloka
{गोशीर्षं हरितश्यामं चन्दनप्रवरं महत्}
{भाराणां शतमेकं तु द्वारि तिष्ठति वारितः}


\threelineshloka
{ये नग्नविषया राजन्बर्बरेयाश्च विश्रुताः}
{शतं दासीसहस्राणां कम्बलांश्च सहस्रशः}
{परिगृह्य महाराज द्वारि तिष्ठन्ति वारिताः}


\twolineshloka
{पौण्ड्राश्च दामलिप्ताश्च यथाकामकृतो नृपाः}
{कालेयकं च रूप्यं च परिगृह्य परिच्छदान्}


\twolineshloka
{अगरून्स्फाटिकांश्चैव दन्ताञ्जातीफलानि च}
{तक्कोलांश्च लवङ्गाश्च कर्पूरांश्च महाबल}


\twolineshloka
{अन्यांश्च विविधान्द्रव्यान्परिगृह्योपतस्थिरे}
{एते सर्वे महात्मानो द्वारि तिष्ठन्ति वारिताः}


\twolineshloka
{शैलेयश्च ततो राजा पत्रोर्णान्परिगृह्य सः}
{द्वारि तिष्ठन्महाराज द्वारपालैर्निवारितः}


\twolineshloka
{चीना हूणाः कषाः काचाः पर्वतान्तरवासिनः}
{आहार्षुर्दशसाहस्रान्विन्नीतान्दिक्षु विंश्रुतान्}


\twolineshloka
{औष्णीकं कम्बलं चैव कीटजं मणिजं तथा}
{प्रमाणरागस्पर्शाढ्यं बाह्वीचीनसमुद्भवम्}


\twolineshloka
{रसान् गन्धान्प्रशंसन्तस्ततो द्वारमलभ्यत}
{खर्वटास्तोमराश्चैव शूरा वर्धनकास्तथा}


\twolineshloka
{चेलान्बहुविधान्गृह्य द्वारि तिष्ठन्ति वारिताः}
{प्राक्कोटा नाटकेयाश्च नन्दीनगरकास्तथा}


\twolineshloka
{नापितास्त्रैपुराश्चैव पञ्चमेयाः सहोरुजाः}
{तथा चाटविकाः सर्वे नानाद्रव्यपरिच्छदान्}


\twolineshloka
{परिगृह्य महाराज द्वारि तिष्ठन्ति वारिताः}
{शकास्तुषाराः कौरव्य रोमकाः शृङ्गिणोश्मकाः}


\twolineshloka
{बलादूरुगमा राजन्गणितं चार्बुदं मया}
{कूटीकृतं सुवर्णं च पद्मकिञ्जल्कसंनिभम्}


\twolineshloka
{शितान्दीर्घानसीनन्यान्यष्टिशक्तिपरश्वथान्}
{श्लक्ष्णं वस्त्रमकार्पसमाविकं मृदु चाजिनम्}


\twolineshloka
{बलं मत्तं समादाय द्वारि तिष्ठन्ति वारिताः}
{आसनानि महार्हाणि यानानि शयनानि च}


\twolineshloka
{मणिकाञ्चनचित्राणि गजदन्तमयानि च}
{रथांश्च विविधाकाराञ्जाम्बूनदपरिष्कृतान्}


\threelineshloka
{हयैर्विनीतैः सम्पन्नान्वैयाघ्रपरिवारितान्}
{विचित्रान्सपरिस्तोमांश्चापानि विविधानि च}
{}


\twolineshloka
{नाराचानर्घनाराचाञ्छस्त्राणि विविधानि च}
{एतद्द्रव्यं महद्गृह्य पूर्वदेशाधिपो नृपः}


\twolineshloka
{प्रविष्टो यज्ञसदनं पाण्डवस्य महात्मनः}
{जन्तुचेलान्द्विसाहस्रान्दुकूलान्ययुतानि च}


\twolineshloka
{कांस्यानि चैव भाण्डानि महार्हाणि कुथानि च}
{एतान्यन्यानि रत्नानि ददौ पार्थस्य वै मुदा}


\twolineshloka
{अन्यान्बहुविधान्राजन्नरः सागरमाश्रिताः}
{रत्नानि विविधान्गृह्य ददुस्ते पाण्डवाय तु}


\twolineshloka
{मालवाश्च ततो राजन्रत्नानि विविधानि च}
{गोधूमानां च राजेन्द्र द्रोणानां कोटिसंमितम्}


\twolineshloka
{अन्यांश्च विविधान्धान्यान्परिगृह्य महाबलः}
{पाण्डवाय ददौ प्रीत्या प्रविवेश महाध्वरम्}


\twolineshloka
{नानारत्नान्बहून्गृह्य सुराष्ट्राधिपतिर्नृपः}
{तैलकुम्भान्महाराज द्रोणानामयुतानि च}


\twolineshloka
{गुडानपि स तान्स्वादून्सहस्रशकटैर्नृपः}
{एतानि सर्वाण्यादाय ददौ कुन्तीसुताय सः}


\twolineshloka
{अन्ये च पार्थिवा राजन्नानादेशसमागताः}
{रत्नानि विविधान्गृह्य ददुस्ते कौरवाय तु}


\twolineshloka
{जम्बूद्वीपे समस्ते तु सराष्ट्रवनपर्वते}
{करं तु न प्रयच्छेत नास्ति पार्थस्य पार्थिवः}


\twolineshloka
{नरः सप्तसु वर्षेसु तद्यज्ञे नास्ति नागतः}
{क्रतुर्नानागणैः कीर्णो बभौ शक्रसदो यथा}


\twolineshloka
{इमांश्च दायान्विविधान्निबोध मम पार्थिव}
{यज्ञाप्थे राजभिर्दत्तान्महतो धनसञ्चयान्}


\twolineshloka
{मेरुमन्दरयोर्मध्ये शैलोदामभितो नदीम्}
{ये ते कीचकवेणूनां छायां रम्यामुपासते}


\twolineshloka
{खषा एकासनाद्यर्हाः प्रदरा दीर्घवेणवः}
{पारदाश्च कुलिन्दाश्च तङ्कणाः परतङ्कणाः}


\twolineshloka
{तद्वै पिपीलिकं नाम उद्धृतं यत्पिपीलिकैः}
{जातरूपं द्रोणमेयमहार्षुः कुञ्जशो नराः}


\twolineshloka
{कृष्णवालांश्च चमराञ्छुक्लवालांस्तथा परान्}
{हिमवत्पुष्पजं चैव स्वादुक्षौद्ररसं बहु}


\twolineshloka
{उत्तरेभ्यः कुरुभ्यश्च व्यूढमाल्यैर्महात्मभिः}
{उत्तरादपि कैलासादोषधीः सुमहाबलाः}


\twolineshloka
{पार्वतीयाश्चराजान आहृत्य प्रणताः स्थिताः}
{अजातशत्रवे राजन्द्वारि तिष्ठन्ति वारिताः}


\twolineshloka
{ये परार्घ्या हिमवतः सूर्योदयगिरेरनु}
{एवंरूपाः समुद्रान्ते लौहित्यमभितश्च ये}


% Check verse!
-----
% Check verse!
-----
% Check verse!
----
% Check verse!
----
% Check verse!
----
\twolineshloka
{शिबित्रैगर्तयौधेया राजन्या मद्रकैः सह}
{वसुतेयाः समौलेया दाहक्षुद्रकमालवैः}


\twolineshloka
{चौण्डिकाश्चौदकाश्चैव साल्वाश्चैव विशम्पते}
{अङ्कवङ्काश्च यवना अनवद्या गयैः सह}


\twolineshloka
{सुजातयः श्रेणिमन्तः श्रेयांसः शस्त्रधारिणः}
{आहार्षुः क्षत्रिया वित्तं शतशोऽजातशत्रवै}


\twolineshloka
{वङ्काः कलिङ्गा मगधास्ताम्रलिप्ताः सपुण्ड्रकाः}
{दुकूलं कौशिकं चैव पत्रोर्णं चैव भारत}


\twolineshloka
{उपावृत्ता नृपास्तस्य ददुः प्रीतिं न चागमन्}
{उच्यन्ते तत्र हि द्वार्स्थैर्बलिमादाय विष्ठिताः}


\twolineshloka
{ईषादन्तान्हेमकक्ष्यान्पद्मवर्णान्कुथावृतान्}
{शैलाभान्नित्यमत्तांश्चाप्यभितः काम्यकं सरः}


\twolineshloka
{क्षमावतः कुलीनांश्च कुञ्जरान्सपरिच्छदान्}
{दत्त्वैकैको दशशतान्द्वारेण प्रविशन्त्विति}


\twolineshloka
{वैदेहकाश्च पुण्ड्राश्च गौलेयास्ताम्रलिप्तकाः}
{मरुकाः काशिका दर्दा भौमेया नटनाटकाः}


\twolineshloka
{कर्णाटाः कांस्यकुट्टाश्च पद्मजालाः सतीनराः}
{दाक्षिणात्याः पुलिन्दाश्च शवेरास्तङ्कणाः शषाः}


\twolineshloka
{बर्बरा यवनाश्चैव गर्गराभीरकास्तथा}
{पल्लवाः शककारूशास्तुम्बकाः काशिकास्तदा}


\twolineshloka
{एते चान्ये च बहवो नानादिगभ्यः समागताः}
{अन्यैश्चोपहृतान्यत्र रत्नानि हि महात्मभिः}


\twolineshloka
{समुद्रसारवैडूर्यान्मुक्ताः शङ्खास्तथैव च}
{शुभावर्ताञ्छुभाञ्छुक्तीः सिंहलाः समुपाहरन्}


\threelineshloka
{सम्भृतान्मणिचीरैश्च श्यामांस्ताम्रान्तलोचनान्}
{राजा चित्ररथो नाम गन्धर्वो वासवानुगः}
{शतानि चत्वार्यददद्धयानां वातरंहसाम्}


\twolineshloka
{तुम्बुरुस्तु प्रमुदितो गन्धर्वो वाजिनां शतम्}
{आम्रपत्रसवर्णानामददद्धेममालिनाम्}


\twolineshloka
{कृती राजा च कौरव्य शूकराणां विशाम्पते}
{अददद्गजरत्नानां शतानि सुबहून्यथ}


\twolineshloka
{विराटेन तु मत्स्येन बल्यर्थं हेममालिनाम्}
{कुञ्जराणां सहस्रे द्वे मत्तानां समुपाहृते}


\twolineshloka
{पांसुराष्ट्राद्वमुदानो राजा षड्विंशतिं गजान्}
{अश्वानां चसहस्रे द्वे राजन्काञ्चनमालिनाम्}


\twolineshloka
{जवसत्वोपपन्नानां वयस्थानां नराधिप}
{बलिं च कृत्स्नमादाय पाण्डवेभ्यो न्यवेदयत्}


\twolineshloka
{यज्ञसेनेन दासीनां सहस्राणि चतुर्दश}
{दासानामयुतं चैव सदाराणां विशाम्पते}


\twolineshloka
{गजयुक्ता महाराज रथाः षड्विंशतिस्तथा}
{राज्यं च कृत्स्नं पार्थेभ्यो यज्ञार्थं वै निवेदितम्}


\twolineshloka
{वासुदेवोऽपि वार्ष्णेयो मानं कुर्वन्किरीटिनः}
{अददद्गजमुख्यानां सहस्राणि चतुर्दश}


\twolineshloka
{आत्मा हि कृष्णः पार्थस्य कृष्णस्यात्मा धनञ्जयः}
{यद्ब्रूयादर्जुनः कृष्णं सर्वं कुर्यादसंशयम्}


\twolineshloka
{कृष्णो धनञ्जयस्यार्थे स्वर्गलोकमपि त्यजेत्}
{तथैव पार्थः कृष्णार्थे प्राणानपि परित्यजेत्}


\twolineshloka
{सुरभींश्चन्दनरसान्हेमकुम्भसमास्थितान्}
{मलयाद्दर्दुराच्चैव चन्दनागुरुसञ्चयान्}


\twolineshloka
{मणिरत्नानि भास्वन्ति काञ्चनं सूक्ष्मवस्त्रकम्}
{चोलपाण्ड्यावपि द्वारं न लेभाते ह्युपस्थितौ}


\twolineshloka
{समुद्रसारं वैदूर्यं मुक्तासङ्घांस्तथैव च}
{शतशश्च कुथांस्तत्र सिंहलाः समुपाहरन्}


\twolineshloka
{संवृता मणिचीरैस्तु श्यामास्ताम्रान्तलोचनाः}
{ता गृहीत्वा नरास्तत्र द्वारि तिष्ठन्ति वारिताः}


\twolineshloka
{प्रीत्यर्थं ब्राह्मणश्चैव क्षत्रियाश्च विनिर्जिताः}
{उपाजह्रुर्विशश्चैव शूद्राः शुश्रूषवस्तथा}


\twolineshloka
{प्रीत्या च बहुमानाच्चाप्युपागच्छन्युधिष्ठिरम्}
{सर्वे म्लेच्छाः सर्ववर्णा आदिमध्यान्तजास्तथा}


\twolineshloka
{नानादेशसमुत्थैश्चन नानाजितिभिरेव च}
{पर्यस्त इव लोकोऽयं युधिष्ठिरनिवेशने}


\twolineshloka
{उच्चावचानुपग्राहान्राजभिः प्रापितान्बहून्}
{शत्रूणां पश्यतो दुःखान्मुमूर्षा मे व्यजायत}


\twolineshloka
{भृत्यास्तु ये पाण्डवानां तांस्ते वक्ष्यामि पार्थिव}
{येषामामं च पक्वं च संविधत्ते युधिष्ठिरः}


\twolineshloka
{अयुतं त्रीणि पद्मानि गजारोहाः ससादिनः}
{रथानामर्बुदं चापि पादाता बहवस्तथा}


\twolineshloka
{प्रमीयमाणमां च पच्यमानं तथैव च}
{विसृज्यमानं चान्यत्र पुण्याहस्वन एव च}


\twolineshloka
{नाभुक्तवन्तं नापीतं नालङ्कृतमसत्कृतम्}
{अपश्यं सर्ववर्णानां युधिष्ठिरनिवेशने}


\threelineshloka
{अष्टाशीतिसहस्राणि स्नातका गृहमेधिनः}
{त्रिंशद्दासीक एकैको यान्बिभर्ति युधिष्ठिरः}
{सुप्रीताः परितृष्टाश्च ते ह्याशंसत्त्यरिक्षयम्}


\twolineshloka
{दशान्यानि सहस्राणि यतीनामूर्ध्वरेतसाम्}
{भुञ्जते रुक्मपात्रीभिर्युधिष्ठिरनिवेशने}


\twolineshloka
{अभुक्तं भुक्तवद्वापि सर्वमाकुब्जवामनम्}
{अभुञ्जाना याज्ञसेनी प्रत्यवैक्षद्विशाम्पते}


\twolineshloka
{द्वौ करौ न प्रयच्छेतां कुन्तीपुत्राय भारत}
{साम्बन्धिकेनपाञ्चालाः सख्येनान्धकवृष्णयः}


\chapter{अध्यायः ७९}
\twolineshloka
{आर्यास्तु ये वै राजानः सत्यसन्धा महाव्रताः}
{पर्याप्तविद्या वक्तारो वेदान्तावभृथप्लुताः}


\twolineshloka
{धृतिमन्तो ह्रीनिषेवा धर्मात्मानो नाशस्विनः}
{मूर्धाभिषिक्तास्ते चैनं राजानः पर्युपांसते}


\twolineshloka
{दक्षिणार्थं समानीत राजभिः कांस्यदोहनाः}
{आरण्या बहुसाहस्रा अपश्यंस्तत्रतत्र गाः}


\twolineshloka
{आजह्रस्तत्र सत्कृत्य स्वयमुद्यम्य भारत}
{अभिषेकार्थमव्यग्रा भाण्डमुच्चावचं नृपाः}


\twolineshloka
{बाह्लीको रथमाहार्षीज्जाम्बूनदविभूषितम्}
{सुदक्षिणस्तु युयुजे श्वेतैः काम्भोजजैर्हयैः}


\twolineshloka
{सुनीथः प्रीतिमांश्चैव ह्यनुकर्षं महाबलः}
{ध्वजं चेदिपतिश्चैवमहार्षीत्स्वयमुद्यतम्}


\twolineshloka
{दाक्षिणात्यः सन्नहनं स्रगुष्णीषे च मागधः}
{वसुदानो महेष्वासो गजेन्द्रं षष्टिहायनम्}


\twolineshloka
{मत्स्यस्त्वक्षान्हेमनद्धानेकलव्य उपानहौ}
{आवन्त्यस्त्वभिषेकार्थमपो बहुविधास्तथा}


\twolineshloka
{चेकितान उपासङ्गं धनुः काश्य उपाहरत्}
{असिं च सुत्सरुं शल्यः शैक्यं काञ्चनभूषणम्}


\twolineshloka
{अभ्यपिञ्चत्ततो धौम्यो व्यासश्च समुहातपाः}
{नारदं च पुरस्कृत्य देवलं चासितं मुनिम्}


\twolineshloka
{प्रीतिमन्त उपातिष्ठन्नभिषेकं महर्षयः}
{जामदग्न्येन सहितास्तथान्ये वेदपारगाः}


\twolineshloka
{अभिजग्मर्महात्मानो मन्त्रवद्भूरिदक्षिणम्}
{महेन्द्रमिव देवेन्द्रं दिवि सप्तर्षयो यथा}


\twolineshloka
{अधारयच्छत्रमस्य सात्यकिः सत्यविक्रमः}
{धनञ्जयश्च व्यजने भीमसेनश्च पाण्डवः}


\twolineshloka
{चामरे चापि शुद्धे द्वे यमौ जगृहतुस्तथा}
{उपागृह्णाद्यमिन्द्राय पुराकल्पे प्रजापतिः}


\twolineshloka
{तमस्मै शङ्खमाहार्षीद्वारुणं कलशोदधिः}
{शैक्यं निष्कसहस्रेण सुकृतं विश्वकर्मणा}


\twolineshloka
{तेनाभिषिक्तः कृष्णेन तत्र मे कश्मलोऽभवत्}
{गच्छन्ति पूर्वादपरं समुद्रं चापि दक्षिणम्}


\twolineshloka
{उत्तरं तु गच्छन्ति विना तात पतत्र्रिभिः}
{तत्र स्म दध्मुः शतशः शङ्खान्मङ्गलकारकान्}


\twolineshloka
{प्राणदन्त समाध्मातास्ततो रोमाणि मेऽहृषन्}
{प्रापतन्भूमिपालाश्च ये तु हीनाः स्वतेजसा}


\twolineshloka
{धृष्टद्युम्नः पाण्डवाश्च सात्यकिः केशवोऽष्टमः}
{सत्वस्था वीर्यसम्पन्ना ह्यन्योन्यप्रियदर्शनाः}


\twolineshloka
{विसञ्ज्ञान्भूमिपान्दृष्ट्वा मां च ते प्राहसंस्तदा}
{ततः प्रहृष्टो बीभत्सुः प्रादाद्धेमविषाणिनाम्}


\twolineshloka
{शतान्यनडुहां पञ्च द्विजमुख्येषु भारत}
{न रन्तिदेवो नाभागो यौवनाश्वो मनुर्न च}


\twolineshloka
{न च राजा पृथुर्वैन्यो न चाप्यासीद्भगीरथः}
{ययातिर्नहुषो वापि यथा राजा युधिष्ठिरः}


\twolineshloka
{यथाऽतिमात्रं कौन्तेयः श्रिया परमया युतः}
{राजसूयमवाप्यैवं हरिश्चन्द्र इव प्रभुः}


\twolineshloka
{एतां दृष्टा श्रियं पार्थे हरिश्चन्द्रे यथा विभो}
{कथं तु जीवितं श्रेयो मम पश्यसि भारत}


\twolineshloka
{अन्धेनेव युगं नद्धं विपर्यस्तं नराधिप}
{कनीयांसो विवर्धन्ते ज्येष्ठा हीयन्त एव च}


\twolineshloka
{एवं दृष्ट्वा नाभिविन्दामि शर्मसमीक्षमाणोऽपि कुरुप्रवीर}
{तेनाहमेवं कृशतां गतश्चविवर्णतां चैव सशोकतां च}


\chapter{अध्यायः ८०}
\twolineshloka
{त्वं वै ज्येष्ठो ज्यैष्ठिनेयः पुत्र मा पाण्डवान्द्विषः}
{द्वेषा ह्यसुखमादत्ते यथैव निघनं तथा}


\twolineshloka
{अव्युत्पन्नं समानार्थं तुल्यमित्रं युधिष्ठिरम्}
{अद्विषन्तं कथं द्विष्यात्त्वादृशो भरतर्षभ}


\twolineshloka
{तुल्याभिजनवीर्यश्च कथं भ्रातुः श्रियं नृप}
{पुत्र कामयसे मोहान्मैवं भूः शाम्य मा शुचः}


\twolineshloka
{अथ यज्ञविभूतिं तां काङ्क्षसे भरतर्षभ}
{ऋत्विजस्तव तन्वन्तु सप्ततन्तुं महाध्वरम्}


\twolineshloka
{आहरिष्यन्ति राजानस्तवापि विपुलं धनम्}
{प्रीत्या च बहुमानाच्च रत्नान्याभरणानि च}


\twolineshloka
{अनार्याचरितं तात परस्वस्पृहणं भृशम्}
{स्वसन्तुष्टः स्वधर्मस्थो यः स वै सुखमेधते}


\twolineshloka
{`मही कामदुघा सा हि वीरपत्नीति चोच्यते}
{तथा वीरस्य भार्या श्रीस्ते इमे हि कलत्रवत्}


% Check verse!
तवाप्यस्ति हे चेद्वीर्यं भोक्ष्यसे हि महीमिमाम्
\twolineshloka
{अयुक्तमिदमेतत्तु परस्वहरणं भृशम्}
{उभयोर्लोकयोर्दुःखं सुहृदां काङ्क्षतोऽनयम्}


\twolineshloka
{अव्यापारः परार्तेषु नित्योद्योगः स्वकर्मसु}
{रक्षणं समुपात्तानामेतद्वैभवलक्षणम्}


\twolineshloka
{विपत्तिष्वव्यथो दक्षो नित्यमुत्थानवान्नरः}
{अप्रमत्तो विनीतात्मा नित्यं भद्राणि पश्यति}


\twolineshloka
{बाहूनिवैतान्मा च्छेत्सीः पाण्डुपुत्रास्तथैव ते}
{भ्रातृणां तद्धनार्थं वै मित्रद्रोहं च मा कुरु}


\twolineshloka
{पाण्डोः पुत्रान्मा द्विषस्वेह राजं-स्तथैव ते भ्रातृधनं समग्रम्}
{मित्रद्रोहे तात महानघर्मःपितामहा ये तव तेऽपि तेषाम्}


\twolineshloka
{अन्तर्वेद्यां ददद्वित्तं कामाननुभवन्प्रियान्}
{क्रीडन्स्त्रीभिर्निरातङ्कः प्रशाम्य भरतर्षभ}


\chapter{अध्यायः ८१}
\twolineshloka
{यस्य नास्ति निजा प्रज्ञा केवलं तु बहुश्रुतः}
{न स जानाति शास्त्रार्थं दर्वी सूपरसानिव}


\twolineshloka
{जानन्वै मोहयति मां नावि नौरिव संयता}
{स्वार्थे किं नावधानं ते उताहो द्वेष्टि मां भवान्}


\twolineshloka
{न सन्तीमे धार्तराष्ट्रा येषां त्वमनुशासिता}
{भविष्यमर्थमाख्यासि सर्वदा कृत्यमात्मनः}


\twolineshloka
{परनेयोऽग्रणीर्यस्य स मार्गान्प्रतिमुह्यति}
{पन्थानमनुगच्छेयुः कथं तस्य पदानुगाः}


\twolineshloka
{राजन्परिणतप्रज्ञो वृद्धसेवी जितेन्द्रियः}
{प्रतिपन्नान्स्वकार्येषु संमोहयसि नो भृशम्}


\twolineshloka
{लोकवृत्ताद्राजवृत्तमन्यदाह बृहस्पतिः}
{तस्माद्राज्ञाऽप्रमत्तेन स्वार्थश्चिन्त्यः सदैव हि}


\twolineshloka
{क्षत्रियस्य महाराज जये वृत्तिः समाहिता}
{स वै धर्मस्त्वधर्मो वा स्ववृत्तौ का परीक्षणा}


\twolineshloka
{प्रकालयेद्दिशः सर्वाः प्रतोदेनेव सारथिः}
{प्रत्यमित्रश्रियं दीप्तां जिघृक्षुर्भरतर्षभ}


\twolineshloka
{प्रच्छन्नो वा प्रकाशो वा योगो योऽरिं प्रबाधते}
{तद्वै शस्त्रं शस्त्रविदां न शस्त्रं छेदनं स्मृतम्}


\twolineshloka
{शत्रुश्चैव हि मित्रं च न लेख्यं न च मातृका}
{यो वै सन्तापयति यं स शत्रुः प्रोच्यते नृप}


\twolineshloka
{असन्तोषः श्रियो मूलं तस्मात्तं कामयाम्यहम्}
{समुच्छ्रये यो यतते स राजन्परमो नयः}


\twolineshloka
{ममत्वं हि न कर्तव्यमैश्वर्ये वा धनेऽपि वा}
{पूर्वावाप्तं हरन्त्यन्ये राजध्रमं हि तं विदुः}


\twolineshloka
{अद्रोहसमंय कृत्वा चिच्छेद नमुचेः शिरः}
{शक्रः साऽभिमता तस्य रिपौ वृत्तिः सनातनी}


\twolineshloka
{द्वावेतौ ग्रसते भूमिः सर्पो बिलशयानिव}
{राजानं चाविरोद्धारं ब्राह्मणं चाप्रवासिनम्}


\twolineshloka
{नास्ति वै जातितः शत्रुः पुरुषस्य विशाम्पते}
{येन साधारणी वृत्तिः स शत्रुर्नेतरो जनः}


\twolineshloka
{शत्रुपक्षं समृध्यन्तं यो मोहात्समुपेक्षते}
{व्याधिराप्यायित इव तस्य मूलं छिनत्ति सः}


\twolineshloka
{अल्पोऽपि ह्यरिरत्यर्थं वर्धमानः पराक्रमैः}
{वल्मीको मूलज इव ग्रसते वृक्षमन्तिकात्}


\twolineshloka
{आजमीढ रिपोर्लक्ष्मीर्मा ते रोचिष्ट भारत}
{एष भारः सत्ववतां न यः शिरसि धिष्ठितः}


\twolineshloka
{जन्मवृद्धिमिवार्थानां यो वृद्धिमाभिकाङ्क्षते}
{एधते ज्ञातिषु स वै सद्यो वृद्धिर्हि विक्रमः}


\twolineshloka
{नाप्राप्य पाण्डवैश्वर्यं संशयो मे भविष्यति}
{अवाप्स्ये वा श्रियं तां हि शयिष्ये वा हतो युधि}


\twolineshloka
{एतादृशस्य किं मेऽद्य जीवितेन विशाम्पते}
{वर्धन्ते पाण्डवा नित्यं वयं स्वस्थिरवृद्धयः}


\chapter{अध्यायः ८२}
\twolineshloka
{यां त्वमेतां श्रियं पाण्डुपुत्रे युधिष्ठिरे}
{तप्यसे तां हरिष्यामि द्यूतेन जयतां वर}


\twolineshloka
{आहूयतां परं राजन्कुन्तीपुत्रो युधिष्ठिरः}
{अगत्वा संशयमहमयुद्ध्वा च चमूमुखे}


\twolineshloka
{अक्षान्क्षिपन्नक्षतः सन्विद्वानविदुषो जये}
{ग्लहान्धनूंषि मे विद्धि शरानक्षांश्च भारत}


\twolineshloka
{अक्षाणां हृदयं मे ज्यां रथं विद्धि ममास्फुरम् ॥दुर्योधन उवाच}
{}


\threelineshloka
{अयमुत्सहते राजच्छ्रियमाहर्तुमक्षवित्}
{द्यूतेन पाण्डुपुत्रेभ्यस्तदनुज्ञातुमर्हसि ॥धृतराष्ट्र उवाच}
{}


\twolineshloka
{स्थितोऽस्मि शासने भ्रातुर्विदुरस्य महात्मनः}
{तेन सङ्गम्य वेत्स्यामि कार्यस्यास्य विनिश्चयम् ॥दुर्योधन उवाच}


\twolineshloka
{`कृष्णादभ्याधिकः सोऽपि क्षत्ता बोद्धा विशाम्पते}
{केवलं धर्ममेवाह न तद्विजयसाधकम्}


\twolineshloka
{जयश्च धर्मतोपेतस्तथैव भरतर्षभ}
{तस्माद्विनयतो जेता तावुभौ च विरोधिनौ'}


\twolineshloka
{व्यपनेष्यति ते बुद्धिं विदुरो मुक्तसंशयः}
{पाण्डवानां हिते युक्तो न तथा मम कौरव}


\twolineshloka
{नारभेतान्यसामर्थ्यात्पुरुषः कार्यमात्मनः}
{मतिसाम्यं द्वयोर्नास्ति कार्येषु कुरुनन्दन}


\twolineshloka
{भयं परिहरन्मत्त आत्मानं परिपालयन्}
{वर्षासु क्लिन्नवटवत्तिष्ठन्नैवावसीदति}


\twolineshloka
{न व्याधयो नापि यमः प्राप्तुं श्रेयः प्रतीक्षते}
{यावदेव भवेत्कल्पस्तावच्छ्रेयः समाचरेत् ॥धृतराष्ट्र उवाच}


\twolineshloka
{सर्वथा पुत्र बलिभिर्विग्रहो मे रोचते}
{वैरं विकारं सृजति तद्वै शस्त्रमनायसम्}


\twolineshloka
{अनर्थमर्थं मन्यसे राजपुत्रसङ्ग्रन्थनं कलहस्यातियाति}
{तद्वै प्रवृत्तं तु यथाकथञ्चित्सृजेदसीन्निशितान्सायकांश्च ॥दुर्योधन उवाच}


\twolineshloka
{द्यूते पुराणैर्व्यवहारः प्रणीत-स्तत्रात्ययो नास्ति न सम्प्रहारः}
{तद्रोचतां शकुनेर्वाक्यमद्यसभां क्षिप्रं त्वमिहाज्ञापयस्व}


\threelineshloka
{स्वर्गद्वारं दीव्यतां नो विशिष्टंतद्वर्तिनां चापि तथैव युक्तम्}
{भवेदेवं ह्यात्मना तुल्यमेवदुरोदरं पाण्डवैस्त्वं कुरुष्व ॥धृतराष्ट्र उवाच}
{}


\twolineshloka
{वाक्यं न मे रोचते यत्त्वयोक्तंयत्ते प्रियं तत्क्रियतां नरेन्द्र}
{पश्चात्तप्स्यसे तदुपाक्रम्य वाक्यंन हीदृशं भावि वचो हि धर्म्यम्}


\twolineshloka
{दृष्टं ह्येतद्विदुरेणै सर्वंविपश्चिता बुद्धिविद्यानुगेन}
{तदेवैतदवशस्याभ्युपैतिमहद्भयं क्षत्रियजीवघाति ॥वैशम्पायन उवाच}


\twolineshloka
{एवमुक्त्वा धृतराष्ट्रो मनीषीदैवं मत्वा परमं दुस्तरं च}
{शशासोच्चैः पुरुषान्पुत्रवाक्येस्थितो राजा दैवसंमूढचेताः}


\twolineshloka
{सहस्रस्तम्भां हेमवैदूर्यचित्रांशतद्वारां तोरणस्फाटिकाढ्याम्}
{सभामग्र्यां क्रोशमात्रायतां मेतद्विस्तारामाशु कुर्वन्तु युक्ताः}


\twolineshloka
{श्रुत्वा तस्य त्वरिता निर्विशङ्काःप्राज्ञा दक्षास्तां तदा चक्रुराशु}
{सर्वद्रव्याण्युपजह्रुः सभायांसहस्रशः शिल्पिनश्चैव युक्ताः}


\twolineshloka
{कालेनाल्पेनाथ निष्ठां गतां तांसभां रम्यां बहुरत्नां विचित्राम्}
{चित्रैर्हैमैरासनैरभ्युपेता-माचख्युस्ते तस्य राज्ञः प्रतीताः}


\twolineshloka
{ततो विद्वान्विदुरं मन्त्रिमुख्य-मुवाचेदं धृतराष्ट्रो नरेन्द्रः}
{युधिष्ठिरं राजपुत्रं च गत्वामद्वाक्येन क्षिप्रमिहानयस्व}


\twolineshloka
{सभेयं मे बहुरत्ना विचित्राशय्यासनैरुपपन्ना महार्हैः}
{सा दृश्यतां भ्रातृभिः सार्धमेत्यमुहृद्द्यूतं वर्ततामत्र चेति ॥वैशम्पायन उवाच}


\twolineshloka
{मतमाज्ञाय पुत्रस्य धृतराष्ट्रो नराधिपः}
{मत्वा च दुस्तरं दैवमेतद्राजंश्चकार ह}


\twolineshloka
{अन्यायेन तथोक्तस्तु विदुरो विदुषां वरः}
{नाभ्यनन्दद्वचो भ्रातुर्वचनं चेदमब्रवीत् ॥विदुर उवाच}


\threelineshloka
{नाभिनन्दे नृपते प्रैषमेतंमैवं कृथाः कुलनाशाद्बिभेमि}
{पुत्रैर्भिन्नः कलहस्ते ध्रुवं स्या-देतच्छङ्के द्यूतकृते नरेन्द्र ॥धृतराष्ट्र उवाच}
{}


\twolineshloka
{नेह क्षत्तः कलहस्तप्स्यते मांन चेद्दैवं प्रतिलोमं भविष्यत्}
{छात्रा तु दिष्टस्य वशे किलेदंसर्वं जगच्चेष्टति न स्वतन्त्रम्}


\twolineshloka
{तदद्य विदुर प्राप्य राजानं मम शासनात्}
{क्षिप्रमानय दुर्धर्षं कुन्तीपुत्रं युधिष्ठिरम्}


\chapter{अध्यायः ८३}
\twolineshloka
{ततः प्रायाद्विदुरोऽश्वैरुदारै-र्महाजवैर्बलिभिः साधु दान्तैः}
{बलान्नियुक्तो धृतराष्ट्रेण राज्ञामनीषिणां पाण्डवानां सकाशे}


\twolineshloka
{सोऽभिपत्य तदध्वानमासाद्य नृपतेः पुरम्}
{प्रविवेश महाबुद्धिः पूज्यमानो द्विजातिभिः}


\twolineshloka
{स राजगृहमासाद्य कुबेरभवनोपमम्}
{अभ्यागच्छत धर्मात्मा धर्मपुत्रं युधिष्ठिरम्}


\twolineshloka
{तं वै राजा सत्यधृतिर्महात्माअजातशत्रुर्विदुरं यथावत्}
{पूजापूर्वं प्रतिगृह्याजमीढ-स्ततोऽपृच्छद्धृतराष्ट्रं सपुत्रम् ॥युधिष्ठिर उवाच}


\twolineshloka
{विज्ञायते ते मनसोऽप्रहर्षःकच्चित्क्षत्तः कुशलेनागतोऽसि}
{कच्चित्पुत्राः स्थविरस्यानुलोमावशानुगाश्चापि विशोऽथ कच्चित् ॥विदुर उवाच}


\twolineshloka
{राजा महात्मा कुशली सपुत्रआस्ते वृतो ज्ञातिभिरिन्द्रकल्पः}
{प्रीतो राजन्पुत्रगुणैर्विनीतोविशोक एवात्मरतिर्महात्मा}


\twolineshloka
{इदं तु त्वां कुरुराजोऽभ्युवाचपूर्वं पृष्ट्वा कुशलं चाव्ययं च}
{इयं सभा त्वत्सभातुल्यरूपाभ्रातॄणां ते दृस्यतामेत्य पुत्र}


\twolineshloka
{समागम्य भ्रातृभिः पार्थ तस्यांसुहृद्द्यूतं क्रियतां रम्यतां च}
{प्रीयामहे भवतां सङ्गमेनसमागताः कुरवश्चापि सर्वे}


\twolineshloka
{दुरोदरा विहिता ये तु तत्रमहात्मना धृतराष्ट्रेण राज्ञा}
{तान्द्रक्ष्यसे कितवान्सन्निविष्टा-नित्यागतोऽहं नृपते तज्जुषस्व ॥युधिष्ठर उवाच}


\twolineshloka
{द्यूते क्षत्तः कलहो विद्यते नःको वै रोचतने बुध्यमानः}
{किं वा भवान्मन्यते युक्तरूपंभवद्वाक्ये सर्व एव स्थिताः स्मः ॥विदुर उवाच}


\twolineshloka
{जानाम्यहं द्यूतमनर्थमूलंकृतश्च यत्नोऽस्य मया निवारणे}
{राजा च मां प्राहिमोत्त्वत्सकाशंश्रुत्वा विद्वञ्श्रेय इहाचरस्व ॥युधिष्ठिर उवाच}


\twolineshloka
{के तत्रान्ये कितवा दीव्यमानाविना राज्ञो धृतराष्ट्रस्य पुत्रैः}
{पृच्छामि त्वां विदुर ब्रूहि नस्तान्यैर्दीव्यामः शतशः सन्निपत्य ॥विदुर उवाच}


\twolineshloka
{गन्धारराजः शकुनिर्विशाम्पतेराजाऽतिदेवी कृतहस्तो मताक्षः}
{विनिंशतिश्चित्रसेनश्च राजासत्यव्रतः पुरुमित्रो जयश्च ॥युधिष्ठिर उवाच}


\twolineshloka
{महाभयाः कितवाः सन्निविष्टामायोपधा देवितारोऽत्र सन्ति}
{धात्रा तु दिष्टस्य वशे किलेदंसर्वं जगत्तिष्ठति न स्वतन्त्रम्}


\twolineshloka
{नाहं राज्ञो धृतराष्ट्रस्य शासना-न्न गन्तुमिच्छामि कवे दुरोदरम्}
{इष्टो हि पुत्रस्य पिता सदैवतदस्मि कर्ता विदुरात्थ मां यथा}


\twolineshloka
{न चाकामः शकुनिना देविताहंन चेन्मां जिष्णुराह्वयिता सभायाम्}
{आहूतोऽहं न निवर्ते कदाचित्तदाहितं शाश्वतं वै व्रतं मे ॥वैशम्पायन उवाच}


\twolineshloka
{एवमुक्त्वा विदुरं धर्मराजःप्रायात्रिकं सर्वमाज्ञाप्य तूर्णम्}
{प्रायाच्छ्वोभूते सगणः सानुयात्रःसह स्त्रीभिर्दौपदामादि कृत्वा}


\twolineshloka
{दैवं हि प्रज्ञां मुष्णाति चक्षुस्तेज इवापतत्}
{धातुश्च वशमन्वेति पाशैरिव नरः सितः}


\twolineshloka
{इत्युक्त्वा प्रययौ राजा सह क्षत्र्रा युधिष्ठिरः}
{अमृष्यमाणस्तस्याथ समाह्वानमरिन्दमः}


\threelineshloka
{बाह्लिकेन रथं यत्तमास्थाय परवीरहा}
{परिच्छन्नो ययौ पार्थो भ्रातृभिः सह पाण्डवः}
{राजश्रिया दीप्यमानो ययौ ब्रह्मपुरः सरः}


\twolineshloka
{`सन्दिदेश ततः प्रेष्यानागतान्नगरं प्रति}
{ततस्ते नृपशार्दूल चक्रुर्वै नृपशासनम्}


\twolineshloka
{ततो राजा महातेजाः संयम्य सपरिच्छदम्}
{ब्राह्मणैः स्वस्ति वाच्याथ प्रययौ मन्दिराद्बहिः}


\twolineshloka
{ब्राह्मणेभ्यो धनं दत्त्वा गत्यर्थं स यथाविधि}
{अन्येभ्यः स तु दत्त्वा च गन्तुमेवोपचक्रमे}


\twolineshloka
{सर्वलक्षणसम्पन्नं राजहंसपरिच्छदम्}
{तमारुह्य महाराजो गजेन्द्रं षष्टिहायनम्}


\twolineshloka
{हारी किरीटी हेमाभः सर्वाभरणभूषितः}
{रराज राजन्पार्थो वै परया नृपशोभया}


\twolineshloka
{रुक्मवेदिगतः प्राज्यो ज्वलन्निव हुताशनः}
{ततो जगाम राजा स प्रहृष्टनरवाहनः}


\twolineshloka
{रथघोषेण महता पूरयन्वै नभः स्थलम्}
{संस्तूयमानः स्तुतिभिः सूतमागधबन्दिभिः}


\twolineshloka
{महासैन्येन सहितो यथादित्यः स्वरश्मिभिः}
{पाण्डुरेणातपत्रेण ध्रियमाणेन मूर्धनि}


\twolineshloka
{बभौ युधिष्ठिरो राजा पौर्णमास्यामिवोडुराट्}
{चामरैर्हेमदण्डैश्च धूयमानः समन्ततः}


\twolineshloka
{जयाशिषः प्रहृष्टानां नराणां पथि पाण्डवः}
{प्रत्यगृह्णाद्यथान्यायं यथआवद्भरतर्षभः}


\twolineshloka
{तथैव सैनिका राजन्राजानमनुयान्ति ये}
{तेषां हलहलाशब्दो दिवं स्तब्धः प्रतिष्ठितः}


\twolineshloka
{नृपस्याग्ने ययौ राजन्भीमसेनो रथी बली}
{उभौ पार्श्वगतौ राज्ञः सतल्पौ वै सुकल्पितौ}


\twolineshloka
{अधिरूढौ यमौ चापि जग्मतुर्भरतर्षभ}
{शोभयन्तौ महासैन्यं तावुभौ रूपशालिनौ}


\twolineshloka
{पृष्ठतोऽनुययौ जिष्णुर्वीरः शस्त्रभृतां वरः}
{श्वेताश्वो गाण्डिवं गृह्य अग्निदत्तं रथं गतः}


\twolineshloka
{सैन्यमध्ये ययौ राजन्कुरुराजो युधिष्ठिरः}
{द्रौपदीप्रमुखा नार्यः सानुगाः सपरिच्छदाः}


\threelineshloka
{आरुह्य ता विचित्राङ्ग्यो यानानि विविधानि च}
{महत्या सेनया राजन्नग्रे यानानि विविधानि च}
{}


\twolineshloka
{समृद्धनरनागाश्वं सपताकरथध्वजम्}
{संनद्धवरनिस्त्रिंशं पथि निर्घोषनिः स्वनम्}


\twolineshloka
{शङ्खदुन्दुभितालानां वेणुवीणानुवनादितम्}
{शुशुभे पाण्डवं सैन्यं प्रयास्यत्तत्तदा नृप}


\twolineshloka
{यथा कुबेरो लङ्कायां पुरा चात्यन्तशोभया}
{महत्या सेनया सार्धं गुरुमिन्द्रं स गच्छति}


\twolineshloka
{तथा ययौ स पार्थोऽपि असङ्ख्येयविभूतिना}
{सुसमृद्धेन सैन्येन यथा वैश्रवणस्तथा}


\twolineshloka
{स सरांसि नदीश्चैव वनान्युपवनानि च}
{अत्यक्रामन्महाराज पुरीं चाभ्यवपद्यत}


\twolineshloka
{स हास्तिनसमीपे तु कुरुराजो युधिष्ठिरः}
{चक्रे निवेशनं तत्र ततः स सहसैनिकाः}


\twolineshloka
{शिवे देशे समे चैव न्यवसत्पाण्डवस्तदा}
{ततोराजन्समाहूय शोकविह्वलया गिरा}


\twolineshloka
{एतद्वाक्यं च सर्वस्वं धृतराष्ट्रचिकीर्षितम्}
{आचचक्षे यथावृत्तं विदुरोऽथ नृपस्य ह}


\threelineshloka
{तच्छ्रुत्वा भाषितं तेन धर्मराजोऽब्रवीदिदम्}
{न मर्षयाम्यहं क्षत्तः समाह्वानं व्रतं हि मे}
{स्वस्त्यस्तु लोके विप्राणां प्रजानां चैव सर्वदा ॥वैशम्पायन उवाच}


\twolineshloka
{प्रविवेश ततो राजा नगरं नागसाह्वयम्}
{धृतराष्ट्रेण चाहूतः कालस्य समयेन च'}


\twolineshloka
{स हास्तिनपुरं गत्वा धृतराष्ट्रगृहं ययौ}
{समियाय च धर्मात्मा धृतराष्ट्रेण पाण्डवः}


\twolineshloka
{तथा भीष्मेण द्रोणेन कर्णेन च कृपेण च}
{समियाय यथान्यायं द्रौणिना च विभुः सह}


\twolineshloka
{समेत्य च महाबाहुः सोमदत्तेन चैव ह}
{दूर्योधनेन सभ्रात्रा सौबलेन च वीर्यवान्}


\twolineshloka
{ये चान्ये तत्र राजानः पूर्वमेव समागताः}
{दुःशासनेन वीरेण सर्वैर्भ्रातृभिरेव च}


\twolineshloka
{जयद्रथेन च तथा कुरुभिश्चापि सर्वशः}
{ततः सर्वैर्महाबाहुर्भ्रातृभिः पिरवारितः}


\twolineshloka
{प्रविवेश गृहं राज्ञो धृतराष्ट्रस्य धीमतः}
{ददर्श तत्र गान्धारीं देवीं पतिमनुव्रताम्}


\twolineshloka
{स्नुषाभिः संवृतां शश्वत्ताराभिरिव रोहिणीम्}
{अभिवाद्य स गान्धारीं तया च प्रतिनन्दितः}


% Check verse!
ददर्श पितरं वृद्धं प्रज्ञाचक्षुषमीश्वरम्
\twolineshloka
{राज्ञा मूर्धन्युपाघ्रातास्ते च कौरवनन्दनाः}
{चत्वारः पाण्डवा राजन्भीमसेनपुरोगमाः}


\twolineshloka
{ततो हर्षः समभवत्कौरवाणां विशाम्पते}
{तान्दृष्ट्वा पुरुषव्याघ्रान्पाण्डवान्प्रियदर्शनान्}


\twolineshloka
{विविशुस्तेऽभ्यनुज्ञाता रत्नवन्ति गृहाणि च}
{ददृशुश्चोपयातारो द्रोपदीप्रमुखाः स्त्रियः}


\twolineshloka
{याज्ञसेन्याः परामृद्धिं दृष्ट्वा प्रज्वलितामिव}
{स्नुषास्ता धृतराष्ट्रस्य नातिप्रमनसोऽभवन्}


\twolineshloka
{ततस्ते पुरुषव्याघ्रा ग्तवा स्त्रीभिस्तु संविदम्}
{कृत्वा व्यायामपूर्वाणि कृत्यानि प्रतिकर्म च}


\twolineshloka
{ततः कृताह्निकाः सर्वे दिव्यचन्दनभूषिताः}
{कल्याणमनसश्चैव ब्राह्मणान्स्वस्ति वाच्य च}


\twolineshloka
{मनोज्ञमशनं भुक्त्वा विविशुः शरणान्यथ}
{उपगीयमाना नारीभिरस्वपन्कुरुपुङ्गवाः}


\twolineshloka
{जगाम तेषां सा रात्रिः पुण्या रतिविहारिणाम्}
{स्तूयमानाश्च विश्रान्ताः काले निद्रामथात्यजन्}


\twolineshloka
{मुखोषितास्ते रजनीं प्रातः सर्वे कृताह्निकाः}
{सभां रम्यां प्रविविशुः कितवैरभिनन्दिताः}


\chapter{अध्यायः ८४}
\twolineshloka
{प्रविश्य तां सभां पार्था युधिष्ठिरपुरोगमाः}
{समेत्य पार्थिवान्सर्वान्पूजार्हानभिपूज्य च}


\twolineshloka
{यथावयः समेयाना उपविष्टा यथार्हतः}
{आसनेषु विचित्रेषु स्पर्द्ध्यास्तरणवत्सु च}


\twolineshloka
{तेषु तत्रोपविष्टेषु सर्वेष्वथ नृपेषु च}
{शकुनिः सौबलस्तत्र युधिष्ठिरमभाषत ॥शकुनिरुवाच}


\threelineshloka
{उपस्तीर्णा सभा राजन्सर्वे त्वयि कृतक्षणाः}
{अक्षानुप्त्वा देवनस्य समयोऽस्तु युधिष्ठिर ॥युधिष्ठिर उवाच}
{}


\twolineshloka
{नितिर्देवनं पापं न क्षात्रोऽत्र पराक्रमः}
{न च नीतिर्ध्रुवा राजन्किं त्वं द्यूतं प्रशंससि}


\threelineshloka
{न हि मानं प्रशंसन्ति निकृतौ कितवस्य हि}
{शकुने मैवं नोऽजैषीरमार्गेण नृशंसवत् ॥शकुनिरुवाच}
{}


\twolineshloka
{यो वेत्ति सङ्ख्या निकृतौ विधिज्ञ-श्चेष्टास्वखिन्नः कितवोऽक्षजासु}
{महामतिर्यश्च जानाति द्यूतंस वै सर्वं सहते प्रक्रियासु}


\threelineshloka
{अक्षग्लहः सोऽभिभवेत्परं न-स्तेनैव दोषो भवतीह पार्थ}
{दीव्यामहे पार्थिव मा विशङ्कांकुरुष्व पाणं च चिरं च मा कृथाः ॥युधिष्ठिर उवाच}
{}


\twolineshloka
{एवमाहायमसितो देवलो मुनिसत्तमः}
{इमानि लोकद्वाराणि यो वै भ्राम्यति सर्वदा}


\twolineshloka
{इदं वै देवनं पापं निकृत्या कितवैः सह}
{धर्मेण तु जयो युद्धे तत्परं न तु देवनम्}


\twolineshloka
{नार्या म्लेच्छन्ति भाषाभिर्मायया न चरन्त्युत}
{अजिह्यमशठं युद्धमेतत्सत्पुरुषव्रतम्}


\twolineshloka
{शक्तितो ब्राह्मणार्थाय शिक्षितुं प्रयतामहे}
{तद्वै वित्तं मातिदेवीर्माजैषीः शकुने परान्}


\threelineshloka
{निकृत्या कामये नाहं सुखान्युत धनानि वा}
{कितवस्येह कृतिनो वृत्तमेतन्न पूज्यते ॥शकुनिरुवाच}
{}


\twolineshloka
{श्रोत्रियः श्रोत्रियानेति निकृत्यैव युधिष्ठिर}
{विद्वानविदुषोऽभ्येति नाहुस्तां निकृतिं जनाः}


\twolineshloka
{अक्षैर्हि शिक्षितोऽभ्येति निकृत्यैव युधिष्ठिर}
{विद्वानविदुषोऽभ्येति नाहुस्तां निकृतिं जनाः}


\threelineshloka
{अकृतास्त्रं कृतास्रश्च दुर्बलं बलवत्तरः}
{एवं कर्मसु सर्वेषु निकृत्यैव युधिष्ठिरः}
{विद्वानविदुषोभ्येति नाहुस्तां निकृतिं जनाः}


\twolineshloka
{एवं त्वं मामिहाभ्येत्य निकृतिं यदि मन्यसे}
{देवनाद्विनिवर्तस्व यदि ते विद्यते भयम् ॥युधिष्ठिर उवाच}


\twolineshloka
{आहूतो न निवर्तेयमिति मे व्रतमाहितम्}
{विधिश्च बलवान्राजन्दिष्टस्यास्मि वशे स्थितः}


\threelineshloka
{अस्मिन्समागमे केन देवनं मे भविष्यति}
{प्रतिपाणश्च कोऽन्योस्ति ततो द्यूतं प्रवर्तताम् ॥दुर्योधन उवाच}
{}


\twolineshloka
{अहं दातास्मि रत्नानां धनानां च विशाम्पते}
{मदर्थे देविता चायं शकुनिर्मातुलो मम ॥युधिष्ठिर उवाच}


\twolineshloka
{अन्येनान्यस्य वै द्यूतं विषमं प्रतिभाति मे}
{एतद्विद्विन्नुपादत्स्व काममेवं प्रवर्ततम्}


\chapter{अध्यायः ८५}
\twolineshloka
{उपोह्यमाने द्यूते तु राजानः सर्व एव ते}
{धृतराष्ट्रं पुरस्कृत्य विविशुस्तां सभां ततः}


\twolineshloka
{भीष्मो द्रोणः कृपश्चैव विदुरश्च महामतिः}
{नातिप्रीतेन मनसा तेऽन्ववर्तन्त भारत}


\twolineshloka
{ते द्वन्द्वशः पृथच्कैव सिंहग्रीवा महौजसः}
{सिंहासनानि भूरिणी विचित्राणि च भेजिरे}


\twolineshloka
{शुशुभे सा सभा राजन्राजभिस्तैः समागतैः}
{देवैरिव महाभागैः समवेतैस्त्रिविष्टपम्}


\twolineshloka
{सर्वे वेदविदः शूराः सर्वे भास्वरमूर्तयः}
{प्रवर्तत महाराज सुहृद्द्यूतमनन्तरम् ॥युधिष्ठिर उवाच}


\twolineshloka
{अयं बुहधनो राजन्सागरावर्तसम्भवः}
{मणिर्हारोत्तरः श्रीमान्कनकोत्तमभूषणः}


\twolineshloka
{एतद्राजन्मम धनं प्रतिपाणोऽस्ति कस्तव}
{येन मां त्वं महाराज धनेन प्रतिदीव्यसे ॥दुर्योधन उवाच}


\twolineshloka
{सन्ति मे मणयश्चैव धनानि सुबहूनि च}
{मत्सरश्च न मेऽर्थेषु जयस्वैनं दुरोदरम् ॥वैशम्पायन उवाच}


\twolineshloka
{ततो जग्राह शकुनिस्तानक्षानक्षतत्त्ववित्}
{जितमित्येव शकुनिर्युधिष्ठिरमभाषत ॥युधिष्ठिर उवाच}


\twolineshloka
{मत्त कैतकेनैव यज्जितोऽस्मि दुरोदरे}
{शकुने हन्त दीव्यामो ग्लहमानाः परस्परम्}


\threelineshloka
{सन्ति निष्कसहस्रस्य भाण्डिन्यो भरिताः शुभाः}
{कोशो हिरण्यमक्षय्यं जातरूपमनेकशः}
{एतद्राजन्मम धनं तेन दीव्याम्यहं त्वया ॥वैशम्पायन उवाच}


\twolineshloka
{कौरवाणां कुलकरं ज्येष्ठं पाण्डवमच्युतम्}
{इत्युक्तः शकुनिः प्राह जितमित्येव तं नृपम् ॥युधिष्ठिर उवाच}


\twolineshloka
{अयं सहस्रसमितो वैयाघ्रः सुप्रतिष्ठितः}
{सुचक्रोपस्करः श्रीमान्किङ्किणीजालमण्डितः}


\twolineshloka
{संह्रादनो राजरथो य इहास्मानुपावहत्}
{जौत्रो रथवरः पुण्यो मेघसागरनिः स्वनः}


\threelineshloka
{अष्टौ यं कुररच्छायाः सदश्वा राष्ट्रसंमताः}
{वहन्ति नैषां मुच्येत पदाद्भूमिमुपस्पृशन्}
{एतद्राजन्धनं मह्यं तेन दीव्याम्यहं त्वया ॥वैशम्पायन उवाच}


\twolineshloka
{एवं श्रुत्वा व्यवसितो निकृतिं समुपाश्रितः}
{जितमित्येव शकुनिर्युधिष्ठिरमभाषत ॥युधिष्ठिर उवाच}


\twolineshloka
{शतं दासीसहस्राणि तरुण्यो हेमभद्रिकाः}
{कम्बुकेयूरधारिण्यो निष्ककण्ठ्यः स्वलङ्कृताः}


\twolineshloka
{महार्हमाल्याभरणाः सुवस्त्राश्चन्दनोक्षिताः}
{मणीन्हेम च बिभ्रत्यश्चतुःषष्टिविशारदाः}


\threelineshloka
{अनुसेवां चरन्तीमाः कुशला नृत्तसामसु}
{स्नातकानाममात्यानां राज्ञां च मम शासनात्}
{एतद्राजन्मम धनं तेन दीव्याम्यहं त्वया ॥वैशम्पायन उवाच}


\threelineshloka
{एतच्छुत्वा व्यवसितो निकृतिं समुपाश्रितः}
{जितमित्येव शकुनिर्युधिष्ठिरमभाषत ॥युधिष्ठिर उवाच}
{}


\twolineshloka
{एतावन्ति च दासानां सहस्राण्युत सन्ति मे}
{प्रदक्षिणानुलोमाश्च प्रावारवसनाः सदा}


\threelineshloka
{प्राज्ञा मेधाविनो दान्ता युवानो मृष्टकुण्डलाः}
{पात्रीहस्ता दिवारात्रमतिथीन्भोजयन्त्युत}
{एतद्राजन्मम धनं तेन दीव्याम्यहं त्वया ॥वैशम्पायन उवाच}


\twolineshloka
{एतच्छ्रुत्वा व्यवसितो निकृतिं समुपाश्रितः}
{जितमित्येव शकुनिर्युधिष्ठिरभाषत ॥युधिष्ठिर उवाच}


\twolineshloka
{सहस्रसङ्ख्या नगा मे मत्तास्तिष्ठन्ति सौबल}
{हेमकक्षाः कृतापीडाः पद्मिनो हेममालिनः}


\twolineshloka
{सुदान्ता राजवहनाः सर्वशब्दक्षमा युधि}
{ईषादन्ता महाकायाः सर्वे चाष्टकरेणवः}


\twolineshloka
{सर्वे च पुरभेत्तारो नवमेघनिभा गजाः}
{एतद्राजन्मम धनं तेन दीव्याम्यहं त्वया ॥वैशम्पायन उवाच}


\twolineshloka
{इत्येवंवादिनं पार्थं प्रहसन्निव सौबलः}
{जितमित्येव शकुनिर्युधिष्ठिरमभाषत ॥युधिष्ठिर उवाच}


\twolineshloka
{रथास्तावन्त एवेमे हेमदण्डाः पताकिनः}
{हयैर्विनीतैः सम्पन्ना रथिभिश्चित्रयोधिभिः}


\threelineshloka
{एकैको ह्यत्र लभते सहस्रपरमां भृतिम्}
{युध्यतोऽयुध्यतो वापि वेतनं मासकालिकम्}
{एतद्राजन्म धनं तेन दीव्याम्यहं त्वया ॥वैशम्पायन उवाच}


\twolineshloka
{इत्येवमुक्ते वचने कृतवैरो दुरात्मवान्}
{जितमित्येव शकुनिर्युधिष्ठिरमभाषत ॥युधिष्ठिर उवाच}


\twolineshloka
{अश्वांस्तित्तिरिकल्माषान्गान्धर्वान्हेममालिनः}
{ददौ चित्ररथस्तुष्टो यांस्तान्गाण्डीवधन्वने}


\twolineshloka
{युद्धे जितः पराभूतः प्रीतिपूर्वमरिन्दमः}
{एतद्राजन्मम धनं तेन दीव्याम्यहं त्वया ॥वैशम्पायन उवाच}


\twolineshloka
{एतच्छ्रुत्वा व्यवसितो निकृतिं समुपाश्रितः}
{जितमित्येव शकुनिर्युधिष्ठिरमभाषत ॥युधिष्ठिर उवाच}


\twolineshloka
{रथानां शकटानां च श्रेष्ठानां चायुतानि मे}
{युक्तान्येव हि तिष्ठन्ति वाहैरुच्चावचैस्तथा}


\twolineshloka
{एवं वर्णस्य वर्णस्य समुच्चीय सहस्रशः}
{यथा समुदिता वीराः सर्वे वीरपराक्रमाः}


\threelineshloka
{क्षीरं पिबन्तस्तिष्ठन्ति भुञ्जानाः शालितण्डुलान्}
{षष्टिस्तानि सहस्राणि सर्वे विपुलवक्षसः}
{एतद्राजन्मम धनं तेन दीव्याम्यहं त्वया ॥वैशम्पायन उवाच}


\threelineshloka
{एतच्छ्रत्वा व्यवसितो निकृति समुपाश्रितः}
{जितमित्येव शकुनिर्युधिष्ठिरमभाषत ॥युधिष्ठिर उवाच}
{}


\twolineshloka
{ताम्रलोहैः परिवृता निधयो ये चतुः शताः}
{पञ्चद्रौणिक एकैकः सुवर्णस्याहतस्य वै}


\twolineshloka
{जातरूपस्य मुख्यस्य नार्घो यस्य हि भारत}
{एतद्राजन्मम धनं तेन दीव्याम्यहं त्वया ॥वैशम्पायन उवाच}


\twolineshloka
{एतच्छ्रुत्वा व्यवसितो निकृतिं समुपाश्रितः}
{जितमित्येव शकुनिर्युधिष्ठिरमभापतः}


\chapter{अध्यायः ८६}
\twolineshloka
{एवं प्रवर्तिते द्यूते घोरे सर्वापहारिणि}
{सर्वसंशयनिर्मोक्ता विदुरो वाक्यमब्रवीत्}


\twolineshloka
{महाराज विजानीहि यत्त्वां वक्ष्यामि भारत}
{मुमूर्षोरौषधमिव न रोचेतापि ते श्रुतम्}


\twolineshloka
{यद्वै पुरा जातमात्रो रुरावगोमायुवद्विस्वरं पापचेताः}
{दुर्योधनो भारतानां कुलघ्नःसोऽयं युक्तो भवतां कालहेतुः}


\twolineshloka
{गृहे वसन्तं गोमायुं त्वं वै मोहान्न बुध्यसे}
{दुर्योधनस्य रूपेण शृणु काव्यां गिरं मम}


\twolineshloka
{मधु वै माध्विको लब्ध्वा प्रपातं नैव बुध्यते}
{आरुह्य तं मज्जति वा पतनं चाधिगच्छति}


\twolineshloka
{सोऽयं मत्तोऽक्षद्यूतेन मधुवन्न पीरक्षते}
{प्रपातं बुध्यते नैव वैरं कृत्वा महारथैः}


\twolineshloka
{विदितं मे महाप्राज्ञ भोजेष्वेवासमञ्जसम्}
{पुत्रं संत्यक्तवान्पूर्वं पौराणां हितकाम्यया}


\twolineshloka
{अन्धका यादवा भोजाः समेताः कंसमत्यजन्}
{नियोगात्तु हते तस्मिन्कृष्णेनामित्रघातिना}


\twolineshloka
{एवं ते ज्ञातयः सर्वे मोदमानाः शतं समाः}
{त्वन्नियुक्तः सव्यसाची निगृह्णातु सुयोधनम्}


\threelineshloka
{निग्रहादस्य पादस्य मोदन्तां कुरवः सुखम्}
{काकेनेमांश्चित्रवर्हाञ्शार्दूलान्क्रोष्टुकेन च}
{क्रीणीष्व पाण्डवान्राजन्मा मज्जीः शोकसागरे}


\twolineshloka
{त्यजेत्कुलार्थे पुरुषं ग्रामस्यार्थे कुलं त्यजेत्}
{ग्रामं जनपदस्यार्थे आत्मार्थे पृथिवीं त्यजेत्}


\twolineshloka
{सर्वज्ञः सर्वभावज्ञः सर्वशत्रुभयङ्करः}
{इति स्म भाषते काव्यो जम्भत्यागे महाऽसुरान्}


\threelineshloka
{हिरण्यष्ठीविनः कांश्चित्पक्षिणो वनगोचरान्}
{गृहे किल कृतावासाँल्लोभाद्राजा न्यपीडयत्}
{स चोपभोगलोभान्धो हिरण्यार्थी परन्तप}


\twolineshloka
{आयतिं च तदात्वं च उभे सद्यो व्यनाशयत्}
{तदर्थकामस्तद्वत्त्वं माद्रुहः पाण्डवान्नृप}


\twolineshloka
{मोहात्मा तप्स्यसे पश्चात्पत्रिहा पुरुषो यथा}
{जातञ्जातं पाण्डवेभ्यः पुष्पमादत्स्व भारत}


\threelineshloka
{मालाकार इवारामे स्नेहं कुर्वन्पुनः पुनः}
{वृक्षानङ्गारकारीव मैनाद्याक्षीः समूलकान्}
{मा गमः समुतामात्यः सबलश्च यमक्षयम्}


\twolineshloka
{समवेतान्हि कः पार्थान्प्रतियुध्येत भारत}
{मरुद्भिः सहितो राजन्नपि साक्षान्मरुत्पतिः}


\twolineshloka
{द्यूतं मूलं कलहस्याभ्युपैतिमिथो भेदं महते दारुणाय}
{तदा स्थितोऽयं धृतराष्ट्रस्य पुत्रोदुर्योधनः सृजते वैरमुग्रम्}


\twolineshloka
{प्रातिपेयाः शान्तनवा भीमसेनाः सबाह्निकाः}
{दुर्योधनापराधेन कृच्छ्रं प्राप्स्यन्ति सर्वशः}


\twolineshloka
{दुर्योधनो मदेनैव क्षेमं व्यपोहति}
{विषाणं गौरिव मदात्स्वयमारुजतेत्मनः}


\twolineshloka
{यश्चित्तमन्वेति परस्य राजन्वीरः कविः स्वामवमत्य दृष्टिम्}
{नावं समुद्र इव बालनेत्रा-मारुह्य घोरे व्यसने निमज्जेत्}


\twolineshloka
{दुर्योधनो ग्लहते पाण्डवेनप्रीयायसे त्वं जयतीति तच्च}
{अतिनर्मा जायते संप्रहारोयतो विनाशः समुपैति पुंसाम्}


\twolineshloka
{आकर्षस्तेऽवाक्फलः सुप्रणीतोहृदि प्रौढो मन्त्रपदः समाधिः}
{युधिष्ठिरेण कलहस्तवाय-मचिन्तितोऽभिमतः स्वबन्धुना}


\twolineshloka
{प्रातिपेयाः शान्तनवाः शृणुध्वंकाव्यां वाचं संसदि कौरवाणाम्}
{वैश्वानरं प्रज्वलितं सुघोरंमा यास्यध्वं मन्दमनुप्रपन्नः}


\twolineshloka
{यदा मन्युं पाण्डवोऽजातशत्रु-र्न संयच्छेदक्षमदाभिभूतः}
{वृकोदरः सव्यसाची यमौ चकोऽत्र द्वीपः स्यात्तुमुले वस्तदानीम्}


\twolineshloka
{महाराज प्रभवस्त्वं धनानांपुरा द्यूतान्मनसा यावदिच्छेः}
{बहुवित्तान्पाण्डवांश्चेज्जयस्त्वंकिं ते तत्स्याद्वसु विन्देह पार्थान्}


\twolineshloka
{जानीमहे देवितं सौबलस्यवेद द्यूते निकृतिं पार्वतीयः}
{यतः प्राप्तः शकुनिस्तत्र यातुमा यूयुधो भारत पाण्डवेयान्}


\chapter{अध्यायः ८७}
\twolineshloka
{परेषामेव यशसा श्लोघसे त्वंसदा क्षत्तः कुत्सयन्धार्तराष्ट्रान्}
{जानीमहे विदुर यत्प्रियस्त्वंबालानिवास्मानवमन्यसे नित्यमेव}


\twolineshloka
{स विज्ञेयः पुरुषोऽन्यत्रकामोनिन्दाप्रशंसे हि तथा युनक्ति}
{जिह्वाऽऽत्मनो हृदयस्थं व्यनक्तिजानीमहे त्वन्मनसः प्रातिकूल्यम्}


\twolineshloka
{उत्सङ्गे च व्याल इवाहितोऽसिमार्जरवत्पोषकं चोपहंसि}
{भर्तृघ्नं त्वां न हि पापीय आहु-स्तस्मात्क्षत्तः किं न बिभेषि पापात्}


\twolineshloka
{जित्वा शत्रून्फलमाप्तं महद्वैमाऽस्मान्क्षत्तः परुषामीह वोचः}
{द्विषद्भिस्त्वं सम्प्रयोगाभिनन्दीमुहुर्देषं यासि नः सम्प्रयोगात्}


\twolineshloka
{अमित्रतां याति नरोऽक्षमं ब्रुव-न्निगूहते गुह्यममित्रसंस्तवे}
{तदाश्रितोऽपत्रप किं नु बाधसेयदिच्छसि त्वं तदिहाभिभाषतसे}


\twolineshloka
{मा नोऽवमंस्था विद्य मानस्तवेदंशिक्षस्व बुद्धिं स्थविराणां सकाशात्}
{यशो रक्षस्व विदुर सम्प्रणीतंमा व्यापृतः परकार्येशु भूस्त्वम्}


\twolineshloka
{अहं कर्तेति विदूर मा च मंस्थामा नो नित्यं परुषाणीह वोचः}
{न त्वां पृच्छामि विदुर यद्धितं मेस्वस्ति क्षत्तर्मा तितिक्षून् क्षिण् त्वम्}


\twolineshloka
{एकः शास्ता न द्वितीयोऽस्ति शास्तागर्भे शयानं पुरुषं शास्ति शास्ता}
{तेनानुशिष्टः प्रवणादिवाम्भोयथा नियुक्तोऽस्ति तथा भवामि}


\threelineshloka
{भिनत्ति शिरसा शैलमहिं भोजयते च यः}
{धीरेव कुरुते तस्य कार्याणामनुशासनम्}
{यो बलादनुशास्तीह सोऽमित्रं तेन विन्दति}


\threelineshloka
{मित्रतामनुवृत्तं तु समुपेक्षत्यपण्डितः}
{दीप्य यः प्रदीप्ताग्निं प्राच्किरं नाभिधावति}
{भस्मापि न स विन्देत शिष्टं क्वचन भारत}


\twolineshloka
{न वासयेत्पारवर्ग्यं द्विषन्तंविशेषतः क्षत्तरहितं मनुप्यम्}
{स यत्रेच्छसि विदुर तत्र गच्छसुसान्त्विता ह्यसती स्त्री जहाति ॥विदुर उवाच}


\twolineshloka
{एतावता पुरुषं ये त्यजन्तितेषां सख्यमन्तवद्ब्रूहि राजन्}
{राज्ञां हि चित्तानि परिप्लुतानिसान्त्वं दत्वा मुसलैर्घातयन्ति}


\twolineshloka
{अबालत्वं मन्यसे राजपुत्रबालोऽहमित्येव सुमन्दबुद्धे}
{यः सौहृदे पुरुषं स्थापयित्वापश्चादेनं दूषयते स बालः}


\twolineshloka
{न श्रेयसे नीयते मन्दबुद्धिःस्त्री श्रोत्रियस्येव गृहे प्रदुष्टा}
{ध्रुवं न रोचेद्भरतर्षभस्यपतिः कुमार्या इव षष्टिवर्षः}


% Check verse!
अतः प्रियं चेदनुकाङ्क्षसे त्वंसर्वेषु कार्येषु हिताहितेषुस्त्रियश्च राजञ्जडपङ्गुकांश्चपृच्छ त्वं वै तादृशांश्चैव सर्वान्
\twolineshloka
{लभ्यते खलु पापीयान्नरोऽनु प्रियवागिह}
{अप्रियस्य हि पथ्यस्य वक्ता श्रोता च दुर्लभः}


\twolineshloka
{यस्तु धर्मपरश्च स्याद्धित्वा भर्तुः प्रियाप्रिये}
{अप्रियाण्याह पथ्यानि तेन राजा सहायवान्}


\twolineshloka
{अव्याधइजं कटुजं तीक्ष्णमुष्णंयशोमुषं परुषं पूतिगन्धिम्}
{सतां पेयं यन्न पिबन्त्यसन्तोमन्युं महाराज पिब प्रशाम्य}


\twolineshloka
{वैचित्रवीर्यस्य यशो धनं चवाञ्छाम्यहं सहपुत्रस्य शश्वत्}
{यथा तथा तेऽस्तु नमश्चतेऽस्तुममापि च स्वस्ति दिशन्तु विप्राः}


\twolineshloka
{आशीविषान्नेत्रविषान्कोपयेन्न च पण्डितः}
{एवं तेऽहं वदामीदं प्रयतः कुरुनन्दन}


\chapter{अध्यायः ८८}
\twolineshloka
{बहुवित्तं पराजैषीः पाण्डवानां युधिष्ठिर}
{आचक्ष्व वित्तं कौन्तेय यदि तेऽस्त्यपराजितम् ॥युधिष्ठिर उवाच}


\twolineshloka
{मम वित्तमसङ्ख्येयं यदहं वेद सौबल}
{अथ त्वं शकुने कस्माद्वित्तं समनुपृच्छसि}


\twolineshloka
{अयुतं प्रयुतं चैव शङ्कुं पद्मं तथार्बुदम्}
{खर्वं शङ्खं निखर्वं च महापद्मं च कोटयः}


\twolineshloka
{मध्यं चैव परार्धं च सपरं चात्र पण्यताम्}
{एतन्मम धनं राजंस्तेन दीव्याम्यहं त्वया ॥वैशम्पायन उवाच}


\twolineshloka
{एतच्छुत्वा व्यवसितो निकृतिं समुपाश्रितः}
{जितमित्येव शकुनिर्युधिष्ठिरमभाषत ॥युधिष्ठिर उवाच}


\threelineshloka
{गवाश्वं बहुधेनूकमसङ्ख्येयमजाविकम्}
{यत्किञ्चिदनु पर्णाशां प्राक् सिन्धोरपि सौबल}
{एतन्मम धनं सर्वं तेन दीव्याम्यहं त्वया ॥वैशम्पायन उवाच}


\twolineshloka
{एतच्छुत्वा व्यवसितो निकृतिं समुपाश्रितः}
{जितमित्येव शकुनिर्युधिष्ठिरमभाषत ॥युधिष्ठिर उवाच}


\threelineshloka
{पुरं जनपदो भूमिरब्राह्मणधनैः सह}
{अब्राह्मणाश्च पुरुषा राजञ्शिष्टं धनं मम}
{एतद्राजन्मम धनं तेन दीव्याम्यहं त्वया ॥वैशम्पायन उवाच}


\twolineshloka
{एतच्छ्रुत्वा व्यवसितो निकृतिं समुपाश्रितः}
{जितमित्येव शकुनिर्युधिष्ठिरमभाषत ॥युधिष्ठिर उवाच}


\threelineshloka
{राजपुत्रा इमे राजञ्छोमन्ते यैर्विभूषिताः}
{कुम्डलानि च निष्काश्च सर्वं राजविभूषणम्}
{एतन्मम धनं राजंस्तेन दीव्याम्यहं त्वया ॥वैशम्पायन उवाच}


\twolineshloka
{एतच्छ्रुत्वा व्यवसितो निकृतिं समुपाश्रितः}
{जितमित्येव शकुनिर्युधिष्ठिरमभाषत ॥युधिष्ठिर उवाच}


\threelineshloka
{श्यामो युवा लोहिताक्षः सिंहस्कन्धो महाभुजः}
{नकुलो ग्लह एवैकी विद्ध्येतन्मम तद्धनम् ॥शकुनिरुवाच}
{}


\twolineshloka
{प्रियस्ते नकुलो राजन्राजपुत्रो युधिष्ठिर}
{अस्माकं वशतां प्राप्तो भूयः केनेह दीव्यसे ॥वैशम्पायन उवाच}


\twolineshloka
{एवमुक्त्वा तु तानक्षाञ्शकुनिः प्रत्यदीव्यत}
{जितमित्येव शकुनिर्युधिष्ठिरमभाषत ॥युधिष्ठिर उवाच}


\twolineshloka
{अयं धर्मान्सहदेवोऽनुशास्तिलोके ह्यस्मिन्पण्डिताख्यां गतश्च}
{अनर्हता राजपुत्रेण तेनदीव्याम्यहं चाप्रियवत्प्रियेण ॥वैशम्पायन उवाच}


\twolineshloka
{एतच्छ्रुत्वा व्यवसितो निकृतिं समुपाश्रितः}
{जितमित्येव शकुनिर्युधिष्ठिरमभाषत ॥शकुनिरुवाच}


\threelineshloka
{माद्रीपुत्रौ प्रियौ राजंस्तवेमौ विजितौ मया}
{गरीयांसौ तु मन्ये भीमसेनधनञ्जयौ ॥युधिष्ठिर उवाच}
{}


\twolineshloka
{अधर्मं चरसे नूनं यो नावेक्षसि वै नयम्}
{यो नः सुमनसां मूढ विभेदं कर्तुमिच्छसि ॥शकुनिरुवाच}


\twolineshloka
{गर्ते मत्तः प्रपतते प्रमत्तः स्थाणुमृच्छति}
{ज्येष्ठो राजन्व्ररिष्ठोऽसि नमस्ते भरतर्षभ}


\twolineshloka
{स्वप्ने तानि न दृश्यन्ते जाग्रतो वा युधिष्ठिर}
{कितवा यानि दीव्यन्तः प्रलपन्त्युत्कटा इव ॥युधिष्ठिर उवाच}


\twolineshloka
{यो नः सङ्ख्ये नौरिव पारनेताजेता रिपूणां राजपुत्रस्तरस्वी}
{अनर्हता लोकवीरेण तेनदीव्याम्यहं शकुने फाल्गुनेन ॥वैशम्पायन उवाच}


\threelineshloka
{एतच्छ्रुत्वा व्यवसितो निकृतिं समुपाश्रितः}
{जितमित्येव शकुनिर्युधिष्ठिरमभाषत ॥शकुनिरुवाच}
{}


\twolineshloka
{अयं मया पाण्डवानां धनिर्धरःपराजितः पाण्डवः सव्यसाची}
{भीमेन राजन्दयितेन दीव्ययत्कैतवं पाण्डव तेऽवशिष्टम् ॥युधिष्ठिर उवाच}


\twolineshloka
{यो नो नेता यो युधि नः प्रणेतायथा वज्री दानवशत्रुरेकः}
{तिर्यक्प्रेक्षी सन्नतभ्रूर्महात्मासिंहस्कन्धो यश्च सदाऽत्यमर्षी}


\twolineshloka
{बलेन तुल्यो यस्यप पुमान्न विद्यतेगदाभृतामग्र्य इहारिमर्दनः}
{अनर्हता राजपुत्रेण तेन दीव्याम्यहं भीमसेनेन राजन् ॥वैशम्पायन उवाच}


\threelineshloka
{एतच्छ्रुत्वा व्यवसितो निकृतिं समुपाश्रितः}
{जितमित्येव शकुनिर्युधिष्ठिरमभाषत ॥शकुनिरुवाच}
{}


\twolineshloka
{बहुवित्तं पराजैषीर्भ्रातॄंश्च सहयद्विपान्}
{आचक्ष्व वित्तं कौन्तेय यदि तेऽस्त्यपराजितम् ॥युधिष्ठिर उवाच}


\twolineshloka
{अहं विशिष्टः सर्वेषां भ्रातॄणां दयितस्तथा}
{कुर्यामहं जितः कर्म स्वयमात्मन्युपल्पुते ॥वैशम्पायन उवाच}


\threelineshloka
{एतच्छ्रुत्वा व्यवसितो निकृतिं समुपाश्रितः}
{जितमित्येव शकुनिर्युधिष्ठिरमभाषत ॥शकुनिरुवाच}
{}


\twolineshloka
{एतत्पापिष्ठमकरोर्यदात्मानं पराजयेः}
{शिष्टे सति धने राजन्पाप आत्मपराजयः ॥वैशम्पायन उवाच}


\twolineshloka
{एवमुक्त्वा मताक्षस्तान् ग्लहे सर्वानवस्थितान्}
{पराजयल्लोकवीरानुक्त्वा राज्ञां पृथक् पृथक् ॥शकुनिरुवाच}


\twolineshloka
{अस्ति ते वै प्रिया राजन् ग्लह एकोऽपराजितः}
{पणस्व कृष्णां पाञ्चालीं तयात्मानं पुनर्जय ॥युधिष्ठिर उवाच}


\twolineshloka
{नैव ह्रस्वा न महती न कृशा नातिरोहिणी}
{नीलकुञ्चितकशी च तया दीव्याम्यहं त्वया}


\twolineshloka
{शारदोत्पलपत्राक्ष्या शारदोत्पलगन्धया}
{शारदोत्पलसेविन्या रूपेण श्रीसमानया}


\twolineshloka
{तथैव स्यादानुशंस्यात्तथ स्याद्रूपसम्पदा}
{तथा स्याच्छीलसम्पत्त्या यामिच्छेत्पुरुषः स्त्रियम्}


\twolineshloka
{सर्वैर्गुणैर्हि सम्पन्नामनुकूलां प्रियंवदाम्}
{यादृशीं धर्मकामार्थसिद्धिमिच्छेन्नरः स्त्रियम्}


\twolineshloka
{चरमं संविशति या प्रथमं प्रतिबुध्यते}
{आगोपालाविपालेभ्यः सर्वं वेद कृताकृतम्}


\twolineshloka
{आभाति पद्मवद्वक्त्रं सस्वेदं मल्लिकेव च}
{वेदीमध्या दीर्घकेशी ताम्रास्या नातिलोमशा}


\twolineshloka
{तयैवंविधया राजन्पाञ्चाल्याहं सुमध्यमा}
{ग्लहं दीव्यामि चार्वङ्ग्या द्रौपद्या हन्त सौबल ॥वैशम्पायन उवाच}


\twolineshloka
{एवमुक्ते तु वचने धर्मराजेन धीमता}
{धिग्धिगित्येव वृद्धानां सभ्यानां निः सृता गिरः}


\twolineshloka
{चुक्षुभे सा सभा राजन्राज्ञां सञ्जत्रिरे शुचः}
{भीष्मद्रोणकृपादीनां स्वेदश्च समजायत}


\twolineshloka
{शिरो गृहीत्वा विदुरो गतसत्व इवाभवत्}
{आस्ते ध्यायन्नधोवक्त्रो निः श्वसन्निव पन्नगः}


\threelineshloka
{`बाह्लीकः सोमदत्तश्च प्रातिपेयश्च सञ्जयः}
{द्रौणिर्भूरिश्रवाश्चैव युयुत्सुर्धृतराष्ट्रजः}
{आसुर्वीक्ष्य त्वधोवक्त्रा निश्वसन्त इवोरगाः'}


\twolineshloka
{धृतराष्ट्रस्तु संहृष्टः पर्यपृच्छत्पुनः पुनः}
{किं जितं किं जितमिति ह्याकारं नाभ्यरक्षत}


\twolineshloka
{जहर्ष कर्णोऽतिभृशं सह दुःशासनादिभिः}
{इतरेषां तु सभ्यानां नेत्रेभ्यः प्रापतञ्जलम्}


\twolineshloka
{सौबलस्त्वभिघायैव जितकाशी मदोत्कटः}
{जितमित्येव तानक्षान्पुररेवान्वपद्यत}


\chapter{अध्यायः ८९}
\twolineshloka
{एहि क्षत्तर्द्रौपदीमानस्वप्रियां भार्यां संमतां पाण्डवानाम्}
{संमार्जतां वेश्म परैतु शीघ्रंतत्रास्तु दासीभिरपुण्यशीला ॥विदुर उवाच}


\twolineshloka
{दर्विभाषं भाषितं त्वादृशेनन मन्द सम्बुद्ध्यसि ----}
{प्रपाते त्वं लम्बमानो न वेत्सिव्याघ्रान्मृगः कोपयसेऽतिवेलम्}


\twolineshloka
{आशीविषास्ते शिरसि पूर्णकोपा महाविषाः}
{मा कोपिष्ठाः सुमन्दात्मन्मा गमस्त्वं यमक्षयम्}


\twolineshloka
{न हि दासीत्वमापन्ना कृष्णा भवितुमर्हति}
{जनीशेन हि राज्ञैषा पणे न्यस्तेति मे मतिः}


\twolineshloka
{अयं दत्ते वेणुरिवात्मघातीफलं राजा धृतराष्ट्रस्य पुत्रः}
{द्यूतं हि वैराय महाभयायमत्तो न ----मन्तकालम्}


\twolineshloka
{नारुन्तुदः स्यान्न -----न हीनताः परमभ्याददीत}
{ययास्य वाचा पर उद्विजेतन तां वदेदुशती पापलोक्याम्}


\twolineshloka
{समुच्चरन्त्यतिवादाश्च वक्त्रा---- शौचति रात्र्यहानि}
{परस्य नामर्मसु ते पतन्तितान्पण्डितो नावसृजेत्परेषु}


\twolineshloka
{अजो हि शस्त्रमगिलत्किलैकःशस्त्रे विपन्ने शिरसास्य भूमौ}
{निकृन्तनं स्वस्य कण्ठस्य घोरंतद्वद्वेरं मा कृथाः पाण्डुपुत्रैः}


\twolineshloka
{न किञ्चिदित्थं प्रवदन्ति पार्थावनेचरं वा गृहमेधिनं वा}
{तपस्विनं वा परिपूर्णविद्यंभषन्ति हैवं श्वनराः सदैव}


\twolineshloka
{द्वारं सुघोरं नरकस्य जिह्यंन बुध्यते धृतराष्ट्रस्य पुत्रः}
{तमन्वेतारो बहवः कुरूणांद्यूतोदये सह दुःशासनेन}


\twolineshloka
{मज्जन्त्यलाबूनि शिलाः प्लवन्तेमुह्यन्ति नावोम्भसि शश्वदेव}
{मूढो राजा धृतराष्ट्रस्य पुत्रोन मे वाचः पथ्यरूपाः शृणोति}


\twolineshloka
{अन्तो नूं भवितायं करूणांसुदारुणः सर्वहरो विनाशः}
{वाचः काव्याः सुहृदां पथ्यरूपान श्रूयन्ते वर्धते लोभ एव ॥वैशम्पायन उवाच}


\twolineshloka
{धिगस्तु क्षत्तारमिति ब्रुवाणोदर्पेण मत्तो धृतराष्ट्रस्य पुत्रः}
{अवैक्षत प्रातिकामीं सभाया-मुवाच चैनं परमार्यमध्ये ॥दुर्योधन उवाच}


\twolineshloka
{त्वं प्रातिकामिन्द्रौपदीमानयस्वन ते भयं विद्यते पाण्डवेभ्यः}
{क्षत्ता ह्ययं विवदत्येव भीतोन चास्माकं वृद्धिकामः सदैव ॥वैशम्पायन उवाच}


\threelineshloka
{एवमुक्तः प्रातिकामी स सूतःप्रायाच्छीघ्रं राजवचो निशम्य}
{प्रविश्य च श्वेव हि सिंहगेष्ठंसमासदन्महिषीं पाण्डवानाम् ॥प्रातिकाम्युवाच}
{}


\twolineshloka
{युधिष्ठिरो द्यूतमदेन मत्तोदुर्योधनो द्रौपदि त्वामजैषीत्}
{सा त्वं प्रपद्यस्व धृतराष्ट्रस्य वेश्मनयामि त्वां कर्मणि याज्ञसेनि ॥द्रौपद्युवाच}


\twolineshloka
{कथं त्वेवं वदसि प्रातिकामि-को हि दीव्येद्भार्यया राजपुत्रः}
{मूडो राजा द्यूतमदेन मत्तोह्यभून्नान्यत्कैतवमस्य किञ्चित् ॥प्रातिकाम्युवाच}


\twolineshloka
{यदा नाभूत्कैतवमन्यदस्यतदाऽदेवीत्पाण्डवोऽजातशत्रुः}
{न्यस्ताः पूर्वं भ्रातरस्तेन राज्ञास्वयं चात्मा त्वमथो राजपुत्रि ॥द्रौपद्युवाच}


\twolineshloka
{गच्छ त्वं कितवं गत्वा सभायां पृच्छ सूतज}
{किं तु पूर्वं पराजैषीरात्मानमथवा नु माम्}


\twolineshloka
{एतज्ज्ञात्वा समागच्छ ततो मां नयं सूतज}
{ज्ञात्वा चिकीर्षितमहं राज्ञो यास्यामि दुःखिता ॥वैशम्पायन उवाच}


\twolineshloka
{सभां गत्वा स चोवाच द्रौपद्यस्तद्वचस्तदा}
{युधिष्ठिरं नरेनद्राणां मध्ये स्थितमिदं वचः}


\twolineshloka
{कस्येशो नः पराजैषीरिति त्वामाह द्रौपदी}
{किं नु पूर्वं पराजैषीरात्मानमथवापि माम् ॥वैशम्पायन उवाच}


\twolineshloka
{युधिष्ठिरस्तु निश्चेता गतसत्व इवाभवत्}
{न तं सूतं प्रत्युवाच वचनं साध्वसाधु वा ॥दुर्योधन उवाच}


\twolineshloka
{इहैवागत्य पाञ्चाली प्रश्नमेनं प्रभाषताम्}
{इहैव सर्वे शृण्वन्तु तस्याश्चैतस्य यद्वचः ॥वैशम्पायन उवाच}


\twolineshloka
{स गत्वा राजभवनं दुर्योधनवशानुगः}
{उवाच द्रौपदीं सूतः प्रातिकामी व्यथन्निव}


\twolineshloka
{सभ्यास्त्वमी राजपुत्र्याह्वयन्तिमन्ये प्राप्तः संशयः कौरवाणाम्}
{न वै समृद्दिं पालयते लघीयान्यस्त्वां सभां नेष्यति राजपुत्रि ॥द्रौपद्युवाच}


\twolineshloka
{एवं नूनं व्यदधात्संविधातास्पर्शावुभौ स्पृशतो वृद्धबालौ}
{धर्मं त्वेकं परमं प्राह लोकेस नः शमं धास्यति गोप्यमानः}


\twolineshloka
{सोऽयं धर्मो मा त्यगात्कौरवान्वैसभ्यान्गत्वा पृच्छ धर्म्यं वचो मे}
{ते मां ब्रूयुर्निश्चितं तत्करिष्येधर्मात्मानो नीतिमन्तो वरिष्ठाः ॥वैशम्पायन उवाच}


\twolineshloka
{श्रुत्वा सूतस्तद्वचो याज्ञसेन्याःसभां गत्वा प्राह वाक्यं तदानीम्}
{अधोमुखास्ते न च किञ्चिदूचु-र्निर्बन्धं तं धार्तराष्ट्रस्य बुद्ध्वा}


\twolineshloka
{युधिष्ठिरस्तु तच्छ्रुत्वा दुर्योधनचिकीर्षितम्}
{द्रौपद्याः संमतं दूतं प्राहिणोद्भरतर्षभ}


\twolineshloka
{एकवस्त्र त्वधोनीवो रोदमाना रजस्वला}
{सभामागम्य पाञ्चालि श्वशुरस्याग्रतो भव}


\twolineshloka
{अथ त्वामागतां दृष्ट्वा राजपुत्रीं सभां तदा}
{सभ्याः सर्वे विनिन्देरन्मनोर्भिर्धृतराष्ट्रजम् ॥वैशम्पायन उवाच}


\twolineshloka
{स गत्वा त्वरितं दूतः कृष्णाया भवनं नृप}
{न्यवेदयन्मतं धीमान्धर्मराजस्य निश्चितम्}


\twolineshloka
{पाण्डवाश्च महात्मानो दीना दुःखसमन्विताः}
{सत्येनातिपरीताङ्गा नोदीक्षन्ते स्म किञ्चन}


\twolineshloka
{ततस्त्वेषां मुखमालोक्य राजादुर्योधनः सूतमुवाच हृष्टः}
{इहैवैतामानय प्रातिकामिन्प्रत्यक्षमस्याः कुरवो ब्रुवन्तः}


\threelineshloka
{ततः सूतस्तस्य वशानुगामीभीतश्च कोपाद्द्रुपदात्मजायाः}
{विहाय मानं पुनरेव सभ्या-नुवाच कृष्णां किमहं ब्रवीमि ॥दूर्योधन उवाच}
{}


\twolineshloka
{दुःशासनैष मम सूतपुत्रोवृकोदरादुद्विजतेऽल्पचेताः}
{स्वयं प्रगृह्यानय याज्ञसेनींकिं ते करिष्यन्त्यवशाः सपत्नाः ॥वैशम्पायन उवाच}


\twolineshloka
{ततः समुत्थाय स राजपुत्रःश्रुत्वा भ्रातुः शासनं रक्तदृष्टिः}
{प्रविश्य तद्वेश्म महारथाना-मित्यब्रवीद्द्रौपदीं राजपुत्रीम्}


\threelineshloka
{एह्येहि पाञ्चालि राजपुत्रीम्}
{दुर्योधनं पश्य विमुक्तलज्जा}
{कुरून्भजस्वायतपत्रनेत्रेधर्मेण लब्धाऽसि सभां परैहि}


\twolineshloka
{ततः समुत्थाय सुदूर्मनाः साविवर्णमामृज्य मुखं करेण}
{आर्ता प्रदुद्राव यतः स्त्रियस्तावृद्धस्य राज्ञः कुरुपुङ्गवस्य}


\twolineshloka
{ततो जवेनाभिससार रोषा-द्दुःशासनस्तामभिगर्जमानः}
{दीर्घेषु नीलेष्वथ चोर्मिमत्सुजग्राह केशेषु नरेन्द्रपत्नीम्}


\twolineshloka
{ये राजसूयावभृथे जलेनमहाक्रतौ मन्त्रपूतेन सिक्ताः}
{ते पाण्डवानां परिभूय वीर्यंबलात्प्रमृष्टा धृतराष्ट्रजेन}


\twolineshloka
{स तां पराकृष्य सभासमीप-मानीय कृष्णामतिदीर्घकेशीम्}
{दुःशासनो नाथवतीमनाथव-च्चकर्ष वायुः कदलीमिवार्ताम्}


\twolineshloka
{सा कृष्णमाणा नमिताङ्गयष्टिःशनैरुवाचाथ रजस्वलाऽस्मि}
{एकं च वासो मम मन्दबुद्धेसभां नेतुं नार्हसि मामनार्य}


\twolineshloka
{ततोऽब्रवीत्तां प्रसभं निगृह्यकेशेशु कृष्णेषु तदा स कृष्णाम्}
{कृष्णं च जिष्णुं च हरिं नरं चत्रायाय विक्रोशति याज्ञसेनि}


\threelineshloka
{रजस्वला वा भव याज्ञसेनिएकाम्बरा वाप्यथवा विवस्त्रा}
{द्यूते जिता चासि कृताऽसि दासीदासीषु वासश्च यथोपजोषम् ॥वैशम्पायन उवाच}
{}


\threelineshloka
{प्रकीर्णकेशी पतितार्धवस्त्रादुःशासनेन व्यवधूयमाना}
{हीमत्यमर्षेण च दह्यमानाशनैरिदं वाक्यमुवाच कृष्णा ॥द्रौपद्युवाच}
{}


\twolineshloka
{इमे समायामुपनीतशास्त्राःक्रियावन्तः सर्व एवेन्द्रकल्पाः}
{गुरुस्थाना गुरवश्चैव सर्वेतेषामग्रे नोत्सहे स्थातुमेवम्}


\twolineshloka
{नशंसकर्मंस्त्वमनार्यवृतमा मा विवस्त्रां कुरु मा विकार्षीः}
{न मर्षयेयुस्तव राजपुत्राःसेन्द्रापि देवा यदि ते सहायाः}


\twolineshloka
{धर्मे स्थितो धर्मसुतो महात्माधर्मश्च सूक्ष्मो निपुणोपलक्ष्यः}
{वाचापि भर्तुः परमाणुमात्र-मिच्छामि दोषं न गुणान्विसृज्य}


\twolineshloka
{इदं त्वकार्यं कुरुवीरमध्येरजस्वलां यत्परिकर्षसे माम्}
{न चापि कश्चित्कुरुतेऽत्र कुत्सांध्रुवं तवेदं मतमभ्युपेतः}


\twolineshloka
{धिगस्तु नष्टः खलु भारतानांधर्मस्तथा क्षत्रविदां च वृत्तम्}
{यत्र ह्यतीतां कुरुधर्मवेलांप्रेक्षन्ति सर्वे कुरवः सभायाम्}


\twolineshloka
{द्रोणस्य भीष्मस्य च नास्ति सत्त्वंक्षत्तुस्तथैवास्य चनास्ति सत्त्वंक्षत्तुस्तथैवास्य महात्मनोपि}
{न लक्षयन्ति कुरुवृद्धमुख्याः ॥वैशम्पायन उवाच}


\twolineshloka
{तथा ब्रुवन्ती करुणं सुमध्यमाभर्तॄन्कटाक्षैः कुपितानपश्यत्}
{सा पाण्डवान्कोपपरीतदेहा-न्सन्दीपयामास कटाक्षपातैः}


\twolineshloka
{हृतेन राज्येन तथा धनेनरत्नैश्च मुख्यैर्न तथा बभूव}
{यथा त्रपाकोपसमीरितेनकृष्णाकटाक्षेण बभूव दुःखम्}


\twolineshloka
{दुःशासनश्चापि समीक्ष्य कृष्णा-मवेक्षमाणां कृपणान्पतींस्तान्}
{आधूय वेगेन विसञ्ज्ञकल्पा-मुवाच दासीति हसन्सशब्दम्}


\twolineshloka
{कर्णस्तु तद्वाक्यमतीव हृष्टःसम्पूजयामास हसन्सशब्दम्}
{गान्धारराजः सुबलस्य पुत्र-स्तथैव दुःशासनमभ्यनन्दत्}


\threelineshloka
{सभ्यास्तु ये तत्र बभूवुरन्येताभ्यामृते धार्तराष्ट्रेण चैव}
{तेषामभूद्दुः खमतीव कृष्णांदृष्ट्वा सभायां परिकृष्यमाणाम् ॥भीष्म उवाच}
{}


\twolineshloka
{न धर्मसौक्ष्म्यात्सुभगे विवेक्तुंशक्रोमि ते प्रश्नमिमं यथावत्}
{अस्वाम्यशक्तः पणितुं परस्वंस्त्रियाश्च भर्तुर्वशतां समीक्ष्य}


\twolineshloka
{त्यजेत सर्वां पृथिवीं समृद्धांयुधिष्ठिरो धर्ममथो न जह्यात्}
{उक्तं जितोऽस्मीति च पाण्डवेनतस्मान्न शक्नोमि विवेक्तुमेतत्}


\threelineshloka
{द्व्यूतेऽद्वितीयः शकुनिर्नरेषुकुन्तीसुतस्तेन निसृष्टकामः}
{न मन्यते तां निकृतिं युधिष्ठिर-स्तस्मान् ते प्रश्नमिमं ब्रवीमि ॥द्रौपद्युवाच}
{}


\twolineshloka
{आहूय राजा कुशलैरनार्यै-र्दुष्टात्मभिर्नैकृतिकैः सभायाम्}
{द्यूतप्रियैर्नातिकृतप्रयत्नःकस्मादयं नाम निसृष्टकामः}


\twolineshloka
{अशुद्धभावैर्निकृतिप्रवृत्तै-रबुध्यमानः कुरुपाण्डवाग्र्यः}
{सम्भूय सर्वैश्च जितोऽपि यस्मा-त्पश्चादयं कैतवमभ्युपेतः}


\twolineshloka
{तिष्ठन्ति चेमे कुरवः सभाया-मीशाः सुतानां च तथा स्नुपाणाम्}
{समीक्ष्य सर्वे मम चापि वाक्यंविब्रूत मे प्रश्नमिमं यथावत्}


\twolineshloka
{न सा सभा यत्र न सन्ति वृद्धान ते वृद्धा ये न वदन्ति धर्मम्}
{नासौ धर्मो यत्र न सत्यमस्तिन तत्सत्यं यच्छलेनानुविद्धम् ॥वैशम्पायन उवाच}


\twolineshloka
{तथा ब्रुवन्तीं करुणं रुदन्ती-मवेक्षमाणां कृपणान्पतींस्तान्}
{दुःशासनः परुषाण्यप्रियाणिवाक्यान्युवाचामधुराणि चैव}


\twolineshloka
{तां कृष्यमाणां च रजस्वलां चस्रस्तोत्तरीयामतदर्हमाणाम्}
{वृकोदरः प्रेक्ष्य युधिष्ठिरं चचकार कोपं परमार्तरूपः}


\chapter{अध्यायः ९०}
\twolineshloka
{भवन्ति गेहे बन्धक्यः कितवानां युधिष्ठिर}
{भवन्ति दीव्यन्ति दया चैवास्ति तावस्वपि}


\twolineshloka
{काश्यो यद्धनमाहार्षीद्द्रव्यं यच्चान्यदुत्तमम्}
{तथाऽन्ये पृथिवीपाला यानि रत्नान्युपाहरन्}


\twolineshloka
{वाहनानि धनं चैव कवचान्यायुधानि च}
{राज्यमात्मा वयं चैव कैतवेन हृतं परैः}


\twolineshloka
{न च मे तत्र कोपोऽभूत्सर्वस्येशो हि नो भवान्}
{इमं त्वतिक्रमं मन्यो द्रौपदी यत्र पण्यते}


\twolineshloka
{एषा ह्यनर्हती बाला पाण्डवान्प्राप्य कौरवैः}
{त्वत्कृते क्लिश्यते क्षुद्रैर्नृशंसैरकृतात्मभिः}


\threelineshloka
{अस्याः कृते मन्युरयं त्वयि राजन्निपात्यते}
{बाहू ते सम्प्रधक्ष्यामि सहदेवाग्निमानयः ॥अर्जुन उवाच}
{}


\twolineshloka
{न पुरा भीमसेन त्वमीदृशीर्वदिता गिरः}
{परैस्ते नाशितं नूनं नृशंसैर्धर्मगौरवम्}


\twolineshloka
{न सकामाः परो कार्या धर्ममेवाचरोत्तमम्}
{भ्रातरं धार्मिकं ज्येष्ठं कोऽतिवर्तितुमर्हति}


\twolineshloka
{आहूतो हि परै राजा क्षात्रं व्रतमनुस्मरन्}
{दीव्यते परकामेन तन्नः कीर्तिकरं महत् ॥भीमसेन उवाच}


\twolineshloka
{एवमस्मिन्कृतं विद्यां यदि नाहं धनञ्जय}
{दीप्तेऽग्नौ सहितौ बाहू निर्दहेयं बलादिव ॥वैशम्पायन उवाच}


\twolineshloka
{तथा तान्दुः खितान्दृष्ट्वा पाण्डवान्धृतराष्ट्रजः}
{कृष्यमाणां च पाञ्चालीं विकर्ण इदमब्रवीत्}


\twolineshloka
{याज्ञसेन्या यदुक्तं तद्वाक्यं विब्रूत पार्थिवाः}
{अविवेकेन वाक्यस्य नरकः सद्य एव नः}


\twolineshloka
{भीष्मश्च धृतराष्ट्रश्च कुरुवृद्धतमावुभौ}
{समेत्य नाहतुः किञ्चिद्विदुरश्च महामतिः}


\twolineshloka
{भारद्वाजश्च सर्वेषामाचार्यः कृप एव च}
{कुत एतावपि प्रश्नं नाहतुर्द्विजसत्तमौ}


\twolineshloka
{ये त्वन्ये पृथिवीपालाः समेताः सर्वतोदिशम्}
{कामक्रोधौ समुत्सृज्य ते ब्रुवन्तु यथामति}


\twolineshloka
{यदितं द्रौपदी वाक्यमुक्तवत्यसकृच्छुभा}
{विमृश्य कस्य कः पक्षः पार्थिवा वदतोत्तरम् ॥वाशम्पायन उवाच}


\twolineshloka
{एवं स बहुशः सर्वानुक्तवांस्तान्सभासदः}
{न च ते पृथिवीपालास्तमूचुः साध्वसाधु वा}


\twolineshloka
{उक्त्वाऽसकृत्तथा सर्वान्विकर्णः पृथिवीपतीन्}
{पाणौ पाणिं विनिष्पिष्य निःश्वसन्निदमब्रवीत्}


\twolineshloka
{विब्रूत पृथिवीपाला वाक्यं मा वा कथञ्चनि}
{मन्ये न्याय्यं यदत्राहं तद्वि वक्ष्यामि कौरवाः}


\twolineshloka
{चत्वार्याहुर्नश्रेष्ठा व्यसनानि महीक्षिताम्}
{मृगयां पानमक्षांश्च ग्राम्ये चैवातिरक्तताम्}


\twolineshloka
{एतेषु हि नरः सक्तो धर्ममुत्सृज्य वर्तते}
{यथाऽयुक्तेन च कृतां क्रियां लोको न मन्यते}


\twolineshloka
{तथेयं पाण्डुपुत्रेण व्यसने वर्तता भृशम्}
{समाहूतेन कितवैरास्थितो द्रौपदीपणः}


\twolineshloka
{साधारणी च सर्वेषां पाण्डवानामनिन्दिता}
{जितेन पूर्वं चानेन पाण्डवेन कृतः पणः}


\twolineshloka
{इयं च कीर्तिता कृष्णा सौबलेन पणार्थिना}
{एतत्सर्वं विचार्याहं मन्ये न विजितामिमाम् ॥वैशम्पायन उवाच}


\twolineshloka
{एतच्छ्रुत्वा महान्नादः सभ्यानामुदतिष्ठत}
{विकर्णं शंसमानानां सौबलं चापि निन्दताम्}


\twolineshloka
{तस्मिन्नुपरते शब्दे राधेयः क्रोधमूर्छितः}
{प्रगृह्य रुचिरं बाहुमिदं वचनमब्रवीत् ॥कर्ण उवाच}


\twolineshloka
{दृश्यन्ते वै विकर्णेह वैकृतानि बहून्यपि}
{तज्जातस्तद्विनाशाय यथाऽग्निररणिप्रजः}


\twolineshloka
{एते न किञ्चिदप्याहुश्चोदिता ह्यपि कृष्णया}
{धर्मेण विजितामेतां मन्यन्ते द्रपदात्मजाम्}


\twolineshloka
{त्वं तु केवलबाल्येन धार्तराष्ट्र विदीर्यसे}
{यद्ब्रवीषि सभाम्ध्ये बालः स्थविरभाषितम्}


\twolineshloka
{न च धर्म यथावत्त्वं कृष्णां च जितेति सुमन्दधीः}
{यद्ब्रवीषि जितां कृष्णां न जितेति सुमन्दधीः}


\twolineshloka
{कथं ह्यविजितां कृष्णां मन्यसे धृतराष्ट्रज}
{यदा सभायां सर्वस्वं न्यस्तवान्पाण्डवाग्रजः}


\twolineshloka
{अभ्यन्तर च सर्वस्वे द्रौपदी भरतर्षभ}
{एवं धर्मजितां कृष्णां मन्यसे न जितां कथम्}


\twolineshloka
{कीर्तिता द्रौपदी वाचा अनुज्ञाता च पाण्डवैः}
{भवत्यविजिता केन हेतुनैषा मता तव}


\twolineshloka
{मन्यसे वा सभामेतामानीतामेकवाससम्}
{अधर्मेणेति तत्रापि शृणु मे वाक्यमुत्तमम्}


\twolineshloka
{एको भर्ता स्त्रिया देवैर्विहितः कुरुनन्दन}
{इयं त्वनेकवशगा बन्धकीति विनिश्चिता}


\twolineshloka
{अस्याः सभामानयनं न चित्रमिति मे मतिः}
{एकाम्बरधरत्वं वाऽप्यथवाऽपि विवस्त्रता}


\twolineshloka
{यच्चैषां द्रविणं किञ्चिद्य चैषा ये च पाण्डवाः}
{सौबलेनेह तत्सर्वं धर्मेण विजितं वसु}


\twolineshloka
{दुःशासन सुबालोऽयं विकर्णः प्राज्ञवादिकः}
{पाण्डवानां च वासांसि द्रौपद्याश्चाप्युपाहर ॥वैशम्पायन उवाच}


\twolineshloka
{तच्छ्रुत्वा पाण्डवाः सर्वे स्वानि वासांसि भारत}
{अवकीर्योत्तरीयाणि सभायां समुपाविशन्}


\twolineshloka
{ततो दुःशासनो राजन्द्रौपद्या वसनं बलात्}
{सभामध्ये सभाक्षिप्य व्यपाक्रष्टुं प्रचक्रमे}


\twolineshloka
{`आकृष्यमाणे वसने विललाप सुदुःखिता}
{ज्ञातं मया विसिष्ठेन पुरा गीतं महात्मना}


\threelineshloka
{महत्यापदि सम्प्राप्ते स्मर्तव्यो भगवान्हरिः}
{इति निश्चित्य मनसा शरणागतवत्सलम्}
{आकृष्यमाणे वसने द्रौपदी कृष्णमस्तरत्}


\twolineshloka
{शङ्खचक्रगदापाणे द्वारकानिलयाच्युत}
{गोविन्द पुण्डरीकाक्ष रक्ष मां शरणागताम्}


\twolineshloka
{हा कृष्ण द्वारकावासिन्क्वासि यादवन्दन}
{इमामवस्थां सम्प्राप्तामनाथां किमुपेक्षसे}


\twolineshloka
{गोविन्द द्वारकावासिन्कृष्ण गोपीजनप्रिय}
{कौरवैः परिभूतां मां किं न जानासि केशव'}


\twolineshloka
{हे नाथ हे रमानाथ व्रजनाथार्तिनाशन}
{कौरवार्णवग्नां मामुद्धरस्व जनार्दन}


\twolineshloka
{कृष्णकृष्ण महायोगिन्विश्वात्मन्विश्व्भावन}
{प्रपन्नां पाहि गोविन्द कुरमध्येऽवसीदतीम्}


\twolineshloka
{इत्यनुस्मृत्य कृष्णं सा हरिं त्रिभुवनेश्वरम्}
{प्रारुदद्दुः खिता राजन्मुखमाच्छाद्य भामिनी}


\twolineshloka
{तस्य प्रसाद्द्रौपद्याः कृष्णमाणेऽम्बरे तदा}
{तद्रूपमपरे वस्त्रं प्रादुरासीदनेकशः}


\twolineshloka
{नानारागविरागाणि वसनान्यथ वै प्रभो}
{प्रादुर्भवन्ति शतशो धर्मस्य परिपालनात्}


\twolineshloka
{ततो हलहलाशब्दस्तत्रासीद्घोरदर्शनः}
{तदद्भुततमं लोके वीक्ष्य सर्वे महीभृतः}


\twolineshloka
{शशंसुर्द्रौपदीं तत्र कुत्सन्तो धृतराष्ट्रजम्}
{`धिग्धिगित्यशिवां वाचमुत्सृजन्कौरवान्प्रति'}


\twolineshloka
{यदा तु वाससां राशिः सभामध्ये समाचितः'}
{तदा दुःशासनः श्रान्तो व्रीडितः समुपाविशत्}


\twolineshloka
{शशाप तत्र भीमस्तु राजमध्ये बृहत्स्वनः}
{क्रोधाद्विस्फुरमाणौष्ठो विनिष्पिष्य करे करम् ॥भीम उवाच}


\twolineshloka
{इदं मे वाक्यमादध्वं क्षत्रिया लोकवासिनः}
{नोक्तपूर्वं नरैरन्यैर्न चान्यो यद्वदिष्यति}


\twolineshloka
{यद्येतदेवमुक्त्वाऽहं न कुर्यां पृथिवीश्वराः}
{पितामहानां पूर्वेषां नाहं गतिमवाप्नुयाम्}


\twolineshloka
{अस्य पापस्य दुर्बुद्धेर्भारतापसदस्य च}
{न पिबेयं बलाद्वक्षो भित्त्वा चेद्रुधिरं युधि ॥वैशम्पायन उवाच}


% Check verse!
तस्य ते तद्वचः श्रुत्वा रौद्रं लोमप्रहर्षणम् ॥प्रचक्रुर्बहुलां पूजां कुसन्तो धृतराष्ट्रजम्
\twolineshloka
{न विब्रुवन्ति कौरव्याः प्रश्नमेतमिति स्म ह}
{सुजनः क्रोशति स्मात्र धृतराष्ट्रं विगर्हयन्}


\threelineshloka
{विदुर उवाच}
{द्रौपदी प्रश्नमुक्त्वैवं रोरवीति त्वनाथवत्}
{वैशम्पायन उवाच}


\twolineshloka
{तस्य ते तद्वचः श्रुत्वा रौद्रं लोमप्रहर्षणम्}
{न च विब्रूत तं प्रश्नं सभ्या धर्मोऽत्र पीड्यते}


\twolineshloka
{सभां प्रपद्यते प्रश्नः प्रज्वलन्निव हव्यवाद्}
{तं वै सत्येन धर्मेण सभ्याः प्रशमयन्त्युत}


\twolineshloka
{धर्म्यं प्रश्नमतो ब्रूयादार्यः सत्येन मानवः}
{विब्रूयुस्तत्र तं प्रश्नं कामक्रोधबलातिगाः}


\twolineshloka
{विकर्णेन यथाप्रज्ञमुक्तः प्रश्नो नराधिपाः}
{भवन्तोऽपि हि तं प्रश्नं विब्रुवन्तु यथामति}


\twolineshloka
{यो हि प्रश्नं न विब्रूयाद्धर्मदर्शी सभां गतः}
{अनृते या फलावाप्तिस्तस्याः सोऽर्धं समश्नुते}


\twolineshloka
{यः पुनर्वितथं ब्रूयाद्धर्मदशीं सभां गतः}
{अनृतस्य फलं कृत्स्नं स प्राप्नोतीति निश्चयः}


\twolineshloka
{अत्राप्युदाहरन्तीममितिहासं पुरातनम्}
{प्रह्लादस्य च संवादं मुनेराङ्गिरसस्य च}


\twolineshloka
{प्रह्लादो नाम दैत्येन्द्रस्तस्य पुत्रो विरोचनः}
{कन्याहेतोराङ्गिरसं सुधन्वानमुपाद्रवत्}


\twolineshloka
{अहं ज्यायानहं ज्यायानिति कन्येप्सया तदा}
{तयोर्देवनमत्रासीत्प्राणयोरिति नः श्रुतम्}


\twolineshloka
{तयोः प्रश्नविवादोऽभूत्प्रह्लादं तावपृच्छताम्}
{ज्यायान्क आवयोरेकः प्रश्नं प्रब्रूहि मा मृषा}


\twolineshloka
{स वै विवदनाद्भीतः सुधन्वानं विलोकयन्}
{तं सुधन्वाब्रवीत्क्रुद्धो ब्रह्मदम्ड इव ज्वलन्}


% Check verse!
यदि वै वक्ष्यसि मृषा प्रह्लादाथ न वक्ष्यसि ॥शतघा ते शिरो वज्री वज्रेण प्रहरिष्यति
\twolineshloka
{सुधन्वना तथोक्तः सन्व्यथितोऽश्वत्थपर्णवत्}
{जगाम कश्यपं दैत्यः परिप्रष्टुं महौजसम् ॥प्रह्लाद उवाच}


\twolineshloka
{त्वं वै धर्मस्य विज्ञाता दैवस्येहासुरस्य च}
{ब्राह्मणस्य महाभाग धर्मकृच्छ्रमिदं शृणु}


\twolineshloka
{यो वै प्रश्नं न विब्रूयाद्वितथं चैव निर्दिशेत्}
{के वै तस्य परे लोकास्तन्ममाचक्ष्व पृच्छतः ॥कश्यप उवाच}


\twolineshloka
{जानन्नविब्रुवन्प्रश्नान्कामात्क्रोधाद्भयात्तथा}
{सहस्रं वारुणान्पाशानात्मनि प्रतिमुञ्चति}


\twolineshloka
{साक्षी वा विब्रुवन्साक्ष्यं गोकर्णशिथिलश्वरन्}
{सहस्रं वारुणान्पाशानात्मनि प्रतिमुञ्जति}


\twolineshloka
{तस्य संवत्सरे पूर्णे पाश एकः प्रमुच्यते}
{तस्मात्सत्यं तु वक्तव्यं जानता सत्यमुञ्जसा}


\twolineshloka
{विद्धो धर्मो ह्यधर्मेण सभां यत्रोपपद्यते}
{न चास्य शल्यं कृन्तन्ति विद्धास्तत्र सभासदः}


\twolineshloka
{अर्धं हरति वै श्रेष्ठः पादो भवति कर्तृषु}
{पादश्चैव सभासत्सु ये न निन्दन्ति निन्दितम्}


\twolineshloka
{अनेना भवति श्रेष्ठो मुच्यन्ते च सभासदः}
{एनो गच्छति कर्तारं निन्दार्हो यत्र निन्द्यते}


\twolineshloka
{वितथं तु वदेयुर्ये धर्मं प्रह्लाद पृच्छते}
{इष्टापूर्तं च ते घ्न्ति सप्तसप्त परावरान्}


\twolineshloka
{हृतस्वस्य हि यद्दुःखं हतपुत्रस्य चैव यत्}
{ऋणिनः प्रति यच्चैव स्वार्थाद्धष्टस्य चैव यत्}


\twolineshloka
{स्त्रियाः पत्या विहीनाया राज्ञा ग्रस्तस्य चैव यत्}
{अपुत्रायाश्च यद्दुःखं व्याघ्राघ्रातस्य चैव यत्}


% Check verse!
अध्यूढायाश्च यद्दुःखं व्याघ्राघ्रातस्य चैव यत् ॥एतानि वै समान्याहुर्दुःखानि त्रिदिवेश्वराः
\twolineshloka
{तानि सर्वाणि दुःखानि प्राप्नोति वितथं ब्रुवन्}
{समक्षदर्शनात्साक्षी श्रवणाच्चेति धारणात्}


\twolineshloka
{तस्मात्सत्यं ब्रुवत्साक्षी धर्मार्थाभ्यां न हीयते}
{कश्यपस्य वचः श्रुत्वा प्रह्लादः पुत्रमब्रवीत्}


\threelineshloka
{श्रेयान्सुधन्वा त्वत्तो वै मत्तः श्रेयांस्तथाङ्गिराः}
{माता सुधन्वाऽयं प्राणानामीश्वरस्तव ॥विरोचन सुधन्वाऽयं प्राणानामीश्वरस्तव ॥सुधन्वोवाच}
{}


\threelineshloka
{पुत्रस्नेहं पिरत्यज्य यस्त्वं धर्मे व्यवस्थितः}
{अनुजानामि ते पुत्रं जीवत्वेव शतं समाः ॥विदुर उवाच}
{}


\twolineshloka
{एवं वै परमं धर्मं श्रुत्वा सर्वे सभासदः}
{यथाप्रश्नं तु कृष्णाया मन्यध्वं तत्र किं परम् ॥वैशम्पायन उवाच}


\twolineshloka
{विदुरस्य वचः श्रुत्वा नोचुः किञ्चन पार्थिवाः}
{कर्णो दुःशासनं त्वाह कृष्णं दासीं गृहान्नय}


\twolineshloka
{तां वेपमानां सव्रीडां प्रलपन्तीं स्म पाण्डवान्}
{दुःशासनः सभामध्ये विचकर्ष तपस्विनीम्}


\chapter{अध्यायः ९१}
\twolineshloka
{पुरस्तात्करणीयं मे न कृतं कार्यमुत्तरम्}
{विह्वलाऽस्मि कृताऽनेन कर्षता बलिना बलात्}


\twolineshloka
{अभिवादं करोम्येषां कुरूणां कुरुसंसदि}
{न मे स्यादपराधोऽयं तदिदं न कृतं मया ॥वैशम्पायन उवाच}


\threelineshloka
{सा तेन च समाधूता दुःखेन च तपस्विनी}
{पतिता विललापेदं सभायामतथोचिता ॥द्रौपद्युवाच}
{}


\twolineshloka
{स्वयंवरे यास्मि नृपैदृष्टा रङ्गे समागतैः}
{न दृष्टपूर्वा चान्यत्र साऽहमद्य सभां गता}


\twolineshloka
{यां न वायुर्न चादित्यो दृष्टवन्तौ पुरा गृहे}
{साऽहमद्य सभामध्ये दृष्टास्मि जनसंसदि}


\twolineshloka
{यां न मृष्यन्ति वातेन स्पृश्यमानां गृहे पुरा}
{स्पृश्यमानां सहन्तेऽद्य पाण्डवास्तां दूरात्मना}


\twolineshloka
{मृष्यन्ति कुरवश्चेमे मन्ये कालस्य पर्ययम्}
{स्नुषां दुहितरं चैव क्लिश्यमानामनर्हतीम्}


\twolineshloka
{किंन्वतः कृपणं भूयो यदहं स्त्री सती शुभा}
{सभामध्यं विगाहेऽद्य क्व नु धर्मो महीक्षिताम्}


\twolineshloka
{धर्म्यं स्त्रियं सभां पूर्वे न नयन्तीति नः श्रुतम्}
{स नष्टः कौरवेयेषु पूर्वो धर्मः सनातनः}


\twolineshloka
{कथं हि भार्या पाण्डुनां पार्षतस्य स्वसा सती}
{वासुदेवस्य च सखी पार्थिवानां सभामियाम्}


\threelineshloka
{तामिमां धर्मराजस्य भार्यां सदृशवर्णजाम्}
{ब्रूत दासीमदासीं वा तत्करिष्यामि कौरवाः}
{}


\twolineshloka
{अयं मां सुदृढं क्षुद्रः कौरवाणां यशोहरः.क्लिश्नाति नाहं तत्सोढुं क्षुद्रः कौरवाणां यशोहरः}
{}


\twolineshloka
{जितां वाऽप्यजितां वापि मन्यध्वं मां यथा नृपाः}
{तथा प्रत्युक्तमिच्छामि तत्करिष्यामि कौरवाः ॥भीष्म उवाच}


\twolineshloka
{उक्तवानस्मि कल्याणि धर्मस्य परमा गतिः}
{लोके न शक्यते ज्ञातुमपि विज्ञैर्महात्मभिः}


% Check verse!
बलवांश्च यथा धर्मं लोके पश्यति पुरुषः ॥स धर्मो धर्मवेलायां भवत्यभिहतः परः
\twolineshloka
{न विवेक्तुं च ते प्रश्नमिमं शक्नोमि निश्चयात्}
{सूक्ष्मत्वाद्गहनत्वाच्च कार्यस्यास्य च गौरवात्}


\twolineshloka
{नूनमन्तः कुलस्यास्य भविता न चिरादिव}
{तथा हि कुरवः सर्वे लोभमोहपरायणाः}


\twolineshloka
{कुलेषु जाताः कल्याणि व्यसनैराहता भृशम्}
{धर्म्यान्मार्गान्न च्यवन्ते येषां नस्त्वं बधूः स्थिता}


\twolineshloka
{उपपन्नं च पाञ्चालि तवेदं वृत्तमीदृशम्}
{यत्कृच्छ्रमपि सम्प्राप्ता धर्ममेवान्ववेक्षसे}


\twolineshloka
{एते द्रोणादयश्चैव वृद्धा धर्मविदो जनाः}
{शून्यैः शरीरैस्तिष्ठन्ति गतासव इवानताः}


\twolineshloka
{युधिष्ठिरस्तु प्रश्नोऽस्मिन्प्रमाणमिति मे मतिः}
{अजितां वा जितां वेति स्वयं व्याख्यातुमर्हति}


\chapter{अध्यायः ९२}
\twolineshloka
{तथा तु दृष्ट्वा बहु तत्र देवींरोरूयमाणां कुररीमिवार्ताम्}
{नोचुर्वचः साध्वथवाऽप्यसाधुमहीक्षितो धार्तराष्ट्रस्य भीताः}


\twolineshloka
{दृष्ट्वा तथा पार्थिवपुत्रपौत्रां-स्तूष्णींभूतान्धृतराष्ट्रस्य पुत्रः}
{स्मयन्निवेदं वचनं बभाषेपाञ्चालराजस्य सुतां तदानीम् ॥दुर्योधन उवाच}


\twolineshloka
{तिष्ठत्वयं प्रश्न उदारसत्वेभीमेऽर्जुने सहदेवे तथैव}
{पत्यौ च ते नकुले याज्ञसेनिवदन्त्वेते वचनं त्वत्प्रसूतम्}


\twolineshloka
{अनीश्वरं विब्रुवन्त्वार्यमध्येयुधिष्ठिरं तव पाञ्चालि हेतोः}
{कुर्वन्तु सर्वे चानृतं धर्मराजंपाञ्चालि त्वं मोक्ष्यसे दासभावात्}


\twolineshloka
{धर्मे स्थितो धर्मसुतो महात्मास्वयं चेदं कथयत्विन्द्रकल्पः}
{ईशो वा ते ह्यनीशोऽथवैषवाक्यादस्य क्षिप्रमेकं भजस्व}


\twolineshloka
{सर्वे हीमे कौरवेयाः सभायांदुःखान्तरे वर्तमानास्तवैव}
{न विब्रुवन्त्यार्यसत्वा यथाव-त्पतींश्च ते समवेक्ष्याल्पभाग्यान् ॥वैशम्पायन उवाच}


\twolineshloka
{ततः सभ्याः कुरुराजस्य तस्यवाक्यं सर्वे प्रशशंसुस्तथोच्चैः}
{चेलावेधांश्चापि चक्रुर्नदन्तोहाहेत्यासीदपि चैवार्तनादः}


\twolineshloka
{श्रुत्वा तुं तद्वाक्यमनोहरं त-द्धर्षश्चासीत्कौरवाणां सभायाम्}
{सर्वे चासन्पार्थिवाः प्रीतिमन्तःकुरुश्रेष्ठं धार्मिकं पूजयन्तः}


\twolineshloka
{युधिष्ठिरं च ते सर्वे समुदैक्षन्त पार्थिवाः}
{किं नु वक्ष्यति धर्मज्ञ इति साचीकृताननाः}


\twolineshloka
{किं नु वक्ष्यति बीभत्सुरजितो युधि पाण्डवः}
{भीमसेनो यमौ चोभौ भृशं कौतूहलान्विताः}


\twolineshloka
{तस्मिन्नुपरते शब्दे भीमसेनोऽब्रवीदिदम्}
{प्रगृह्य रुचिरं दिव्यं भुजं चन्दनचर्चितम्}


\twolineshloka
{यद्येष गुरुरस्माकं धर्मराजो महामनाः}
{न प्रभुः स्यात्कुलस्यास्य न वयं मर्षयेमहि}


\twolineshloka
{ईशो नः पुण्यतपसां प्राणानामपि चेश्वरः}
{मन्यते जितमात्मानं यद्येष विजिता वयम्}


\twolineshloka
{न हि मुच्येत मे जीवन्पदा भूमिमुपस्पृशन्}
{मर्त्यधर्मा परामृश्य पाञ्चाल्या मूर्धजानिमान्}


\twolineshloka
{पश्यध्वं ह्यायतौ वृत्तौ भुजौ मे परिघाविव}
{नैतयोरन्तरं प्राप्य मुच्येतापि शतक्रतुः}


\twolineshloka
{धर्मपाशसितस्त्वेवमधिगच्छामि सङ्कटम्}
{गौरवेण निरुद्धश्च निग्रहादर्जुनस्य च}


\twolineshloka
{धर्मराजनिसृष्टस्तु सिंहः क्षुद्रमृगानिव}
{धार्तराष्टारानिमान्पापान्निष्पषेयं तलासिभिः ॥वैशम्पायन उवाच}


\twolineshloka
{तमुवाच तदा भीष्मो द्रोणो विदुर एव च}
{क्षम्यतामिदमित्येवं सर्वं सम्भाव्यते त्वयि ॥कर्ण उवाच}


\threelineshloka
{त्रयः किलेमे ह्यधना भवन्ति}
{दासः पुत्रश्चास्वतन्त्रा च नारी}
{दासस्य पत्नी त्वधनस्य भद्रेहीनश्वरा दासधनं च सर्वम्}


\twolineshloka
{प्रविश्य राज्ञः परिवारं भजस्वतत्ते कार्यं शिष्टमादिश्यतेऽत्र}
{ईशास्तु सर्वे तव राजपुत्रिभवन्ति वै धार्तराष्ट्रा न पार्थाः}


\twolineshloka
{अन्यं वृणीष्व पतिमाशु भामिनियस्माद्दास्यं न लभसि देवनेन}
{अवाच्या वै पतिषु कामवृत्ति-र्नित्यं दास्ये विदितं तत्तवास्तु}


\twolineshloka
{पराजितो नकुलो भीमसेनोयुधिष्ठरः सहदेवार्जुनौ च}
{दासीभूता त्वं हि वै याज्ञसेनिपराजितास्ते पतयो नैव सन्ति}


\twolineshloka
{प्रयोजनं जन्मनि किं न मन्यतेपराक्रमं पौरुषं चैव पार्थटः}
{पाञ्चाल्यस्य द्रुपदस्यात्मजामिमांसभामध्ये यो व्यदेवीद््ग्लहेषु ॥वैशम्पायन उवाच}


\twolineshloka
{तद्वै श्रुत्वा भीमसेनोऽत्यमर्षीभृशं निशश्वास तदाऽर्तरूपः}
{राजानुगो धर्मपाशानुबद्धोदहन्निवैनं क्रोधसंरक्तदृष्टिः ॥भीम उवाच}


\twolineshloka
{नाहं कुप्ये सूतपुत्रस्य राज-न्नेष सत्यं दासधर्मः प्रदिष्टः}
{किं विद्विषो वै मामेवं व्याहरेयु-र्नादेवीस्त्वं यद्यनया नरेन्द्र ॥वैशम्पायन उवाच}


\twolineshloka
{भीमसेनवचः श्रुत्वा राजा दुर्योधनस्तदा}
{युधिष्ठिरमुवाचेदं तूष्णीम्भूतमचेतनम्}


\twolineshloka
{भीमार्जुनौ यमौ चैव स्थितौ ते नृप शासने}
{प्रश्नं ब्रूहि च कृष्णां त्वमजितां यदि मन्यसे}


\twolineshloka
{एवमुक्त्वा तु कौन्येयमपोह्य वसनं स्वकम्}
{स्मयन्निवेक्ष्य पाञ्चालीमैश्वर्यमदमोहितः}


\twolineshloka
{कदलीदण्डसदृशं सर्वलक्षणसंयुतम्}
{गजहस्तप्रतीकाशं वज्रप्रतिमगौरवम्}


\twolineshloka
{अभ्युत्स्मयित्वा राधेयं भीममाधर्षयन्निव}
{द्रौपद्याः प्रेक्षमाणायाः सव्यमूरुमदर्शयत्}


\twolineshloka
{भीमसेनस्तमालोक्य नेत्रे उत्फाल्य लोहिते}
{प्रोवाच राजमध्ये तं सभां विश्रावयन्निव}


\twolineshloka
{पितृभिः सह सालोक्यं मा स्म गच्छेद्वृकोदरः}
{यद्येतमूरुं गदया न भिन्द्यां ते महाहवे ॥वैशम्पायन उवाच}


\twolineshloka
{क्रुद्धस्य तस्य सर्वेभ्यः स्रोतोभ्यः पावकार्चिषः}
{वृक्षस्येव विनिश्वेरुः कोटरेभ्यः प्रदह्यतः ॥विदुर उवाच}


\twolineshloka
{परं भयं पश्यत भीमसेना-त्तद्बुध्यध्वं पार्थिवाः प्रातिपेयाः}
{दैवेरितो नूनमयं पुरस्ता-त्परोऽनयो भरतेषूदपादि}


\twolineshloka
{अतिद्यूतं कृतमिदं धार्तराष्ट्रयस्मात्स्त्रियं विवदध्वं सभायाम्}
{योगक्षेमौ नश्यतो वः समग्रौपापान्मन्त्रान्कुरवो मन्त्रयन्ति}


\twolineshloka
{इमं धर्मं कुरवो जानताशुध्वस्ते धर्मे परिषत्सम्प्रदुष्येत्}
{इमां चेत्पूर्वं कितवोऽग्लहिष्य-दीशोऽभविष्यदपराजितात्मा}


\twolineshloka
{स्वप्ने यथैतद्विजितं धनं स्या-देवं मन्ये यस्य दीव्यत्यनीशः}
{गान्धारराजस्य वचो निशम्यधर्मादस्मात्कुरवो माऽपयात ॥दुर्योधन उवाच}


\twolineshloka
{भीमस्य वाक्ये तद्वदेवार्जुनस्यस्थितोऽहं वै यमयोश्चैवमेव}
{युधिष्ठिरं ते प्रवदन्त्वनीश-मथो दास्यान्मोक्षसे याज्ञसेनि ॥अर्जुन उवाच}


\twolineshloka
{ईशो राजा पूर्वमासीद््ग्लहे नःकुन्तीसुतो धर्मराजो महात्मा}
{ईशस्त्वयं कस्य पराजितात्मातज्जानीध्वं कुरवः सर्व एव ॥`कर्ण उवाच}


\threelineshloka
{दुश्शासन निबोधेदं वचनं वै प्रभाषितम्}
{किमनेन चिरं वीर नयस्व द्रपदात्मजाम्}
{दासीभावेन भुङ्क्ष्व त्वं यथेष्टं कुरुनन्दन ॥वैशम्पायन उवाच}


\twolineshloka
{ततो गान्धारराजस्य पुत्रः शकुनिरब्रवीत्}
{साधु कर्ण महाबाहो यथेष्टं क्रियतामिति}


\threelineshloka
{ततो दुश्शासनस्तूर्णं द्रुपदस्य सुतां बलात्}
{प्रवेशयितुमारब्धः स चकर्ष दुरात्मवान् ॥ततो विक्रोशति तदा पाञ्चाली वरवर्णिनी ॥द्रौपद्युवाच}
{}


\twolineshloka
{परित्रायस्व मां भीष्ण द्रोण द्रौणे तथा कृप}
{परित्रायस्व विदुर धर्मिष्ठो धर्मवत्सल}


\threelineshloka
{धृतराष्ट्र महाराज परित्रायस्व वै स्नुषाम्}
{गान्धारि त्वं महाभागे सर्वज्ञे सर्वदर्शिनि}
{पिरत्रायस्व मां भीरुं सुयोधनभयार्दिताम्}


\twolineshloka
{त्वमार्ये वीरजननि किं मां पश्यसि यादवीम्}
{क्लिश्यमानामनार्येण न त्रायसिव वधूं स्वकाम्}


\twolineshloka
{यदि लालप्यमानां मां न कश्चित्किञ्चिदब्रवीत्}
{हा हताऽस्मि सुमन्दात्मा सुयोधनवशं गता}


\twolineshloka
{न वै पाण्डुर्नरपतिर्न धर्मो न च देवराट्}
{न चायुर्नाश्विनौ वाऽपि परित्रायन्ति वै स्नुषाम् ॥धिक्कष्टं यदि जीवेयं मन्दभाग्या पतिव्रता ॥विदुर उवाच}


\twolineshloka
{शृणोमि वाक्यं तव राजपुत्रिनेमे पार्थाः किञ्चिदपि ब्रुवन्ति}
{सा त्वं प्रियार्थं शृणु वाक्यमेत-द्यदुच्यते पापमतिः कृतघ्नः}


% Check verse!
सुयोधनः सानुचरः सुदुष्टःसहैव राजा निकृतः सूनुना चयद्येष वाचं महदुच्यमानांन श्रोष्यते पापमतिः सुदुष्टः ॥वैशम्पायन उवाच
\twolineshloka
{इत्येवमुक्त्वा द्रुपदस्य पुत्रींक्षत्ताऽब्रवीद्धृतराष्ट्रस्य पुत्रम्}
{}


% Check verse!
मा क्लिश्यतां वै द्रुपदस्य पुत्रींमा त्वं चरीं द्रक्ष्यसि राजपुत्र ॥विदुर उवाच
\threelineshloka
{यद्येवं त्वं महाराज सङ्क्लेशयसि द्रौपदीम्}
{अचिरेणैव कालेन पुत्रस्ते सह मन्त्रिभिः}
{गमिष्यति क्षयं पापः पाण्डवक्षयकारणात्}


\twolineshloka
{भीमार्जुनाभ्यां क्रुद्धाभ्यां माद्रीपुत्रद्वयेन च}
{तस्मान्निवारय सुतं मा विनाशं विचिन्तय ॥वैशम्पायन उवाच}


\twolineshloka
{एतच्छुत्वा मन्दबुद्धिर्नोत्तरं किञ्चिदब्रवीत्}
{ततो दुर्योधनस्तत्र दैवमोहबलात्कृतः}


\twolineshloka
{अचिन्त्य क्षत्तुर्वचनं हर्षेणायतलोचनः}
{ऊरू दर्शयते पापो द्रौपद्या वै मुहुर्मुहुः}


\twolineshloka
{ऊरौ सन्दर्श्यमाने तु निरीक्ष्य तु सुयोधनम्}
{वृकोदरस्तदालोक्य नेत्रे चोल्फाल्य लोहिते}


\twolineshloka
{एतत्समीक्ष्यात्मनि चावमानंनियम्य मन्युं बलनान्स मानी}
{राजानुजः संसदि कौरवाणांविनिष्क्रमन्वाक्यमुवाच भीमः}


\twolineshloka
{अहं दुर्योधनं हन्ता कर्णं हन्ता धनञ्जयः}
{शकुनिं चाक्षकितवं सहदेवो हनिष्यति}


\twolineshloka
{इदं च भूयो वक्ष्यामि सभामध्ये बृहद्वचः}
{सत्यं देवाः करिष्यन्ति यन्नो युद्धं भविष्यति}


\twolineshloka
{सुयोधनमिमं पापं हन्ताऽस्मि गदया युधि}
{शिरः पादेन चास्याहमधिष्ठास्यामि भूतले}


\twolineshloka
{वक्षः शूरस्य निर्वास्य परुषस्य दुरात्मनः}
{दुश्शासनस्य रुधिरं पास्यामि मृगराडिव ॥अर्जुन उवाच}


\twolineshloka
{भीमसेन न ते सन्ति येषां वैरं त्वया सह}
{नन्दा गृहेषु न बुद्ध्यन्ते महद्भयम्}


\twolineshloka
{नैव वाचा व्यवसितं भीम विज्ञायते सताम्}
{यदि स्थास्यन्ति सङ्ग्रामे क्षत्रधर्मेण वै सह}


\twolineshloka
{दुर्योधनस्य कर्णस्य शकुनेश्च दुरात्मनः}
{दुश्शासनचतुर्थानां भूमिः पास्यति शोणितम्}


\twolineshloka
{असूयितारं वक्तारं प्रहृष्टानां दुरात्मनाम्}
{भीमसेन नियोगात्ते हन्ताऽहं कर्णमाहवे}


\threelineshloka
{कर्णं कर्णानुगांश्चैव रणे हन्ताऽस्मि पत्रिभिः}
{ये चान्ये विप्रयोत्स्यन्ति बुद्धिमोहेन मां नृपाः}
{तान्स्म सर्वञ्छितैर्बाणैर्नेताऽस्मि यमसादनम्}


\twolineshloka
{चलेद्वि हिमवान्स्थानान्निष्प्रभः स्याद्दिवाकरः}
{शैत्यं सोमात्प्रणश्येत मत्सत्यं विचलेद्यदि ॥वैशम्पायन उवाच}


\twolineshloka
{उत्युक्तवति पार्थे तु श्रीमान्माद्रवतीसुतः}
{प्रगृह्य विपुलं बाहुं सहदेवः प्रतापवान्}


\twolineshloka
{सौबलस्य वधप्रेप्सुरिदं वचनमब्रवीत्}
{क्रोधसंरक्तनयनो निश्वस्य च मुहुर्मुहुः ॥सहदेव उवाच}


\twolineshloka
{यानक्षान्मन्यसे मूढ गान्धाराणां यशोहर}
{नैते ह्यक्षाः शिता वाणास्त्वयैते समरे धृताः}


\threelineshloka
{यथा चैवोक्तवान्भीमस्त्वामुद्दिश्य सबान्धवम्}
{कर्ताऽहं कर्मणा चास्य कुरुकार्याणि सर्वशः}
{यदि स्थस्यासि सङ्ग्रामे क्षत्रधर्मेण सौबल ॥वैशम्पायन उवाच}


\twolineshloka
{सहदेववचः श्रुत्वा नकुलोऽपि विशाम्पते}
{दर्शनीयतमो नॄणामिदं वचनमब्रवीत् ॥नकुल उवाच}


\twolineshloka
{सुतेयं यज्ञसेनस्य द्यूतेऽस्मिन्धृतराष्ट्रजैः}
{यैर्वाचः श्राविता रूक्षा धूर्तैर्दुर्योधनप्रियैः}


\twolineshloka
{धार्तराष्ट्रान्सुदुर्वृत्तान्मुमूर्षून्कालचोदितान्}
{दर्शयिष्यामि भूयिष्ठमहं वैवस्वतक्षयम्}


\twolineshloka
{उलूकं च दुरात्मानं सौबलस्य प्रियं सुतम्}
{हन्ताऽहमस्मि समरे मम शत्रुं नराधमम्}


\twolineshloka
{निदेशाद्धर्मराजस्य द्रौपद्याः पदवीं चरन्}
{नर्धार्तराष्टां पृथिवीं कर्तास्मि नचिरादिवा ॥द्रौपद्युवाच}


\twolineshloka
{यस्माच्चोरुं दर्शयसे यस्माच्चोरुं निरीक्षसे}
{तस्मात्तव ह्यधर्मिष्ठ ऊरौ मृत्युर्भविष्यति}


\twolineshloka
{यस्माच्चैवं क्लेशयति भ्राता ते मां दुरात्मवान्}
{तस्मादुधिरमेवास्य पास्यते वै वृकोदरः}


\twolineshloka
{इमं च पापिष्ठमतिं कर्णं समुतबान्धवम्}
{सामात्यं सपरीवारं हनिष्यति धनञ्जयः}


\twolineshloka
{क्षुद्रधर्मं नैकृतिकं शकुनिं पापचेतसम्}
{सहदेवो रणे क्रुद्धो हनिष्यति सबान्धवम् ॥वैशम्पायन उवाच}


\twolineshloka
{एवमुक्ते तु वचने द्रौपद्या धर्मशीलया}
{ततोऽन्तरिक्षात्सुमहत्पुष्पवर्षमवापततम्}


\twolineshloka
{तेषां तु वचनं श्रुत्वा नोचुस्तत्र सभासदः}
{अर्जुनस्य भयाद्राजन्नभून्निश्शब्दमत्र वै}


\chapter{अध्यायः ९३}
\twolineshloka
{द्रौपद्या वचनं श्रुत्वा चुकोपाथ धनञ्जयः}
{स तदा क्रोधताम्राक्ष इदं वचनमब्रवीत्}


\twolineshloka
{अयं तु मां वारयते धर्मराजो युधिष्ठिरः}
{इत्युक्त्वा क्रोधताम्राक्षो धनुरादाय वीर्यवान्}


\twolineshloka
{सव्यसाची समुत्पत्य ताञ्छत्रून्समुदैक्षत}
{उद्यतं फल्गुनं तत्र ददृशुः सर्वपार्थिवाः}


\twolineshloka
{युगान्ते सर्वलोकांस्तु दहन्तमिव पावकम्}
{वीक्षमाणं धनुष्पामिं हन्तुकामं मुहुर्मुहुः}


\twolineshloka
{हन्तुकामं पशून्क्रुद्धं रुद्रं दक्षक्रतौ यथा}
{तथाभूतं नरं दृष्ट्वा विषेदुस्तत्र मानवाः}


\twolineshloka
{धनञ्जयस्यव वीर्यज्ञा निराशा जीविते तदा}
{मृतभूता भवन्सर्वे नेत्रैरनिमिषैरिव}


\twolineshloka
{अर्जुनं धर्मपुत्रं च समुदैक्षन्त पार्थिवाः}
{क्रुद्धं तदाऽर्जुनं दृष्ट्वा पृथिवी च चचाल ह}


\threelineshloka
{खेचराणि च भूतानि वित्रेसुर्वै भयार्दिताः}
{नादित्यो विरराजाथ नापि वान्ति च मारुताः}
{}


\twolineshloka
{न चन्द्रो न च नक्षत्रं द्यौर्दिशोन न विभान्ति ह}
{सर्वमाविद्धमभवज्जगत्स्थावरजङ्गमम्}


\twolineshloka
{उत्पतन्स वभौ पार्तो दिवाकर इवाम्बरे}
{}


\twolineshloka
{पार्थं दृष्ट्वा क्रुद्धं कालान्तकयमोपमम्}
{भीमसेनो मुदा युक्तो युद्धायैव मनो दधे}


\twolineshloka
{पाञ्चाली च ददर्शाथ सुसङ्क्रुद्धं धनञ्जयम्}
{हन्तुकामं रिपून्मर्वान्सुपर्णमिव पन्नगान्}


\twolineshloka
{दुष्प्रेक्षः सोऽभवत्क्रुद्धो युगान्ताग्निरिव ज्वलन्}
{तं दृष्ट्वा तेजसा युक्तं विव्यधुः पुरवासिनः}


\twolineshloka
{उत्पतन्तं तु वेगेन ततो दृष्ट्वा धनञ्जयम्}
{जग्राह स तदा राजा पुरुहूतो यथा हरिम्}


\twolineshloka
{उवाच स घृणी ज्येष्ठो धर्मराजो युधिष्ठिरः}
{मा पार्थ साहसं कार्षीर्मा विनाशं गमेद्यशः}


\twolineshloka
{अहमेतान्पापकृतो द्यूतज्ञान्दग्धुमुत्सहे}
{कन्त्त्वसत्यगतिं दृष्ट्वा क्रोधो नाशमुपैति मे}


% Check verse!
त्वमिमं जगतोऽर्थे वै कोपं संयच्छ पाण्डव ॥वैशम्पायन उवाच
\twolineshloka
{एवमुक्तस्तदा राज्ञा पाण्डवोऽथ धनञ्जयः}
{क्रोधं संशमयन्पार्थो धार्तराष्ट्रं प्रति स्थितः}


\twolineshloka
{तस्मिन्वीरे प्रशान्ते तु पाण्डवे फल्गुने पुनः}
{सुसम्प्रहृष्टमभवज्जगत्स्थावरजङ्गमम्}


\twolineshloka
{वारितं च तथा दृष्ट्वा भ्रात्रा पार्थं वृकोदरः}
{बभूव विमना राजन्नभून्निश्शब्दमत्र वै}


\twolineshloka
{ततो राज्ञो धृतराष्ट्रस्य गेहेगोमायुरुच्चैर्व्याहरदग्निहोत्रे}
{तं रासभाः प्रत्यभाषन्त राज-न्समन्ततः पक्षिणश्चैव रौद्राः}


\twolineshloka
{तं वै शब्दं विदुरस्तत्त्ववेदीशुश्राव घोरं सुबलात्मजा च}
{भीष्मो द्रोणो गौतमश्चापि विद्वान्स्वस्तिस्वस्तीत्यपि चैवाहुरुच्चैः}


\twolineshloka
{ततो गान्धारी विदुरश्चापि विद्वां-स्तमुत्पातं घोरमालक्ष्य राज्ञे}
{निवेदयामासतुरार्तवत्तदाततो राजा वाक्यमिदं बभाषे ॥धृतराष्ट्र उवाच}


\twolineshloka
{हतोऽसि दुर्योधन मन्दबुद्धेयस्त्वं सभायां कुरुपुङ्गवानाम्}
{स्त्रियं समाभाषसि दुर्विनीतविशेषतो द्रौपदीं धर्मपत्नीम्}


\twolineshloka
{वैशम्पायन उवाच ॥ एवमुक्त्वा धृतराष्ट्रो मनीषीहितान्वेषी बान्धवानामपायात्}
{कृष्णां पाञ्चालीमब्रवीत्सान्त्वपूर्वंविमृश्यैतत्प्रज्ञया तत्त्वबुद्धिः}


\twolineshloka
{धृतराष्ट उवाच ॥वरं वृणीष्व पाञ्चालि मत्तो यदभिवाञ्छसि}
{वधूनां हि विशिष्टा मे त्वं धर्मपरमा सती}


\twolineshloka
{द्रौपद्युवाच ॥ ददासि चेद्वरं मह्यंवृणोमि भरतर्षभ}
{सर्वधर्मानुगः श्रीमानदासोऽस्तु युधिष्ठिरः}


\twolineshloka
{मनस्विनमजानन्तो मैवं ब्रूयुः कुमारकाः}
{एतं वै दासपुत्रेति प्रतिविन्ध्यं ममात्मजम्}


\twolineshloka
{राजपुत्रः पुरा भूत्वा यथा नान्यः पुमान्क्वचित्}
{लालितो दासपुत्रत्वं पश्यन्नश्येद्धि भारत ॥धृतराष्ट्र उवाच}


\threelineshloka
{एवं भवतु कल्याणि यथा त्वमभिभाषसे}
{द्वितीयं ते वरं भद्रे ददानि वरयस्व ह}
{मनो हि मे वितरति नैकं त्वं वरमर्हसि ॥द्रौपद्युवाच}


\twolineshloka
{सरथौ सघनुष्कौ न भीमसेनधनञ्जयौ}
{यमौ च वरये राजन्नदासान्स्ववशानहम्}


\threelineshloka
{धृतराष्ट्र उवाच ॥ तथाऽस्तु ते महाभागे यथा त्वं नन्दिनीच्छसि}
{तृतीयं वरयास्मत्तो नासि द्वाभ्यां सुसंस्कृता}
{त्वं हि सर्वस्नुषाणां मे श्रेयसी धर्मचारिणी}


\threelineshloka
{द्रौपद्युवाच}
{लोभो धर्मस्य नाशाय भगवन्नाहमुत्सहे}
{अनर्हा वरमादातुं तृतीयं राजसत्तम}


\twolineshloka
{एकामाहुर्वैश्यवरं द्वौ तु क्षत्रस्त्रियो वरौ}
{त्रयस्तु राज्ञो राजेन्द्र ब्राह्मणस्य शतं वराः}


\twolineshloka
{पापीयांस इमे भूत्वा सन्तीर्णाः पतयो मम}
{वेत्स्यन्ति चैव भद्राणि राजन्पुण्येन कर्मणा}


\chapter{अध्यायः ९४}
\twolineshloka
{या नः श्रुता मनुष्येषु स्त्रियो रूपेण संमताः}
{तासामेतादृशं कर्म न कस्याश्चन शुश्रुम}


\twolineshloka
{क्रोधाविष्टेषु पार्थेषु धार्तराष्ट्रेषु चाप्यति}
{द्रौपदी पाण्डुपुत्राणां कृष्णा शान्तिरिहाभवत्}


\twolineshloka
{अप्लवेऽम्भसि मग्नानामप्रतिष्ठे निमज्जताम्}
{पाञ्चाली पाण्डुपुत्राणां नौरेषां पारगाऽभवत् ॥वैशम्पायन उवाच}


\twolineshloka
{तद्वै श्रुत्वा भीमसेनः कुरुमध्येऽत्यमर्षणः}
{स्त्री गतिः पाण्डुपुत्राणामित्युवाच सुदुर्मनाः ॥भीम उवाच}


\twolineshloka
{त्रीणि ज्योतींषि पुरुष इति वै देवलोऽब्रवीत्}
{अपत्यं कर्म विद्या च यतः सृष्टाः प्रजास्ततः}


\twolineshloka
{अमेध्ये वै गतप्रामे शून्ये ज्ञातिभिरुज्झिते}
{देहे त्रितयमेवैतत्पुरुषस्योपयुज्यते}


\twolineshloka
{तन्नो ज्योतिरभिहतं दाराणामभिमर्शनात्}
{धनञ्जय कथं स्वित्स्यादपत्यमभिमृष्टजम् ॥अर्जुन उवाच}


\twolineshloka
{न चैवोक्ता न चानुक्ता हीनतः परुषा गिरः}
{भारत प्रतिजल्पन्ति सदा तूत्तमपूरुषाः}


\twolineshloka
{स्मरन्ति सुकृतान्येव न वैराणि कृतान्यपि}
{सन्तः प्रतिविजानन्तो लब्धसम्भावनाः स्वयम् ॥भीम उवाच}


\twolineshloka
{इहैवैतानहं सर्वान्हन्मि शत्रून्समागतान्}
{अथ निष्क्रम्य राजेन्द्र समूलान्हन्मि भारत}


\twolineshloka
{किं नो विवदितेनेह किमुक्तेन च भारत}
{अद्यैवैतान्निहन्मीह प्रशाधि पृथिवीमिमाम्}


\twolineshloka
{इत्युक्त्वा भीमसेनस्तु कनिष्ठैर्भ्रातृभिः सह}
{मृगमध्ये यथा सिंहो मुहुर्मुहुरुदैक्षत}


\twolineshloka
{सान्त्व्यमानो वीक्षमाणः पार्थेनाक्लिष्टकर्मणा}
{खिद्यत्येव महाबाहुरन्तर्दाहेन वीर्यवान्}


\twolineshloka
{क्रुद्धस्य तस्य स्रोतोभ्यः कर्णादिभ्यो नराधिप}
{सधूमः सस्फुलिङ्गार्चिः पावकः समजायत}


\twolineshloka
{भ्रुकुटीकृतदुष्प्रेक्ष्यमभवत्तस्य तन्मुखम्}
{युगान्तकाले सम्प्राप्ते कृतान्तस्येव रूपिणः}


\twolineshloka
{युधिष्ठिरस्तमावार्य बाहुना बाहुशालिनम्}
{मैवमित्यब्रवीच्चैनं जोषमास्वेति भारत}


\twolineshloka
{निवार्य च महाबाहुं कोपसंरक्तलोचनम्}
{पितरं समुपातिष्ठद्धृतराष्ट्रं कृताञ्जलिः}


\chapter{अध्यायः ९५}
\twolineshloka
{राजन्किं करवामस्ते प्रशाध्यस्मांस्त्वमीश्वरः}
{नित्यं हि स्थातुमिच्छामस्तव भारत शासने ॥धृतराष्ट्र उवाच}


\twolineshloka
{अजातशत्रो भद्रं ते अरिष्टं स्वस्ति गच्छत}
{अनुज्ञाताः सहधनाः स्वराज्यमनुशासत}


\twolineshloka
{इदं चैवावबोद्धव्यं वृद्धस्य मम शासनम्}
{मया निगदितं सर्वं पथ्यं निःश्रेयसं परम्}


\twolineshloka
{वेत्थ त्वं तात धर्माणां गतिं सूक्ष्मां युधिष्ठिर}
{विनीतोऽसि महाप्राज्ञ वृद्धानां पर्युपासिता}


\twolineshloka
{यतो बुद्धिस्ततः शान्तिः प्रशमं गच्छ भारत}
{नादारुणि पतेच्छस्त्रं दारुण्येतन्निपात्यते}


\twolineshloka
{न वैराण्यभिजानन्ति गुणान्पश्यन्ति नागुणान्}
{विरोधं नाधिगच्छन्ति ये त उत्तमपूरुषाः}


\twolineshloka
{स्मरन्ति सुकृतान्येव न वैराणि कृतान्यपि}
{सन्तः परार्थं कुर्वाणा नावेक्षन्ति प्रतिक्रियाम्}


\threelineshloka
{संवादे परुषाण्याहुर्युधिष्ठिर नराधमाः}
{प्रत्याहुर्मध्यमास्त्वेतेऽनुक्ताः नराधमाः}
{}


\twolineshloka
{न चोक्ता नैव चानुक्तास्त्वहिताः परुषा गिरः}
{प्रतिजल्पन्ति वै धीराः सदा तूत्तमपुरुषाः}


\twolineshloka
{स्मरन्ति सुकृतान्येव न वैराणि कृतान्यपि}
{सन्तः प्रतिविजानन्तो लब्ध्वा प्रत्ययमात्मनः}


\twolineshloka
{असम्भिन्नार्यमर्यादाः साधवः प्रियदर्शनाः}
{तथा चरित्तंमार्येण त्वयाऽस्मिन्सत्समागमे}


\twolineshloka
{दुर्योधनस्य पारुष्यं तत्तात हृदि मा कृथाः}
{मातरं चैव गान्धारीं मां च त्वं गुणकाङ्क्षया}


\twolineshloka
{उपस्थितं वृद्धमन्धं पितरं पश्य भारत}
{प्रेक्षापूर्वं मया द्यूतमिदमासीदुपेक्षितम्}


\twolineshloka
{मित्राणि द्रष्टुकामेन पुत्राणां च बलाबलम्}
{अशोच्याः कुरवो राजन्येषां त्वमनुशासिता}


\twolineshloka
{मन्त्री च विदुरो धीमान्सर्वशास्त्रविशारदः}
{त्वयि धर्मोऽर्जुने धैर्यं भीमसेने पराक्रमः}


\threelineshloka
{शुद्धा च गुरुशुश्रूषा यमयोः पुरुषाग्र्ययोः}
{अजातशत्रो भद्रं ते खाण्डवप्रस्थमाविश}
{भ्रातृभिस्तेऽस्तु सौभ्रात्रं धर्मे ते धीयताम मनः ॥वैशम्पायन उवाच}


\twolineshloka
{इत्युक्तो भरतश्रेष्ठ धर्मराजो युधिष्ठिरः}
{कृत्वाऽऽर्यसमयं सर्वं प्रतस्थे भ्रातृभिः सह}


\twolineshloka
{ते रथान्मेघसङ्काशानास्थाय सह कृष्णया}
{प्रययुर्हृष्टमनस इन्द्रप्रस्थं पुरोत्तमम्}


\chapter{अध्यायः ९६}
\twolineshloka
{अनुज्ञातांस्तान्विदित्वा सरत्नधनसञ्जयान्}
{पाण्डवान्धार्तराष्ट्राणां कथमासीन्मनस्तदा ॥वैशम्पायन उवाच}


\twolineshloka
{अनुज्ञातांस्तान्विदित्वा धृतराष्ट्रेण धीमता}
{राजन्दुः शासनः क्षिप्रं जगाम भ्रातरं प्रति}


\twolineshloka
{दुर्योधनं समासाद्य सामात्यं भरतर्षभ}
{दुःखार्तो भरतश्रेष्ठ इदं वचनमब्रवीत् ॥दुःशासन उवाच}


\twolineshloka
{दुःखेनैतत्समानीतं स्थविरो नाशयत्यसौ}
{शत्रुसाद्गमयद्द्रव्यं तद्बुध्यध्वं महारथाः}


\twolineshloka
{अथ दुर्योधनः कर्णः शकुनिश्चापि सौबलः}
{मिथः सङ्गम्य सहिताः पाण्डवान्प्रति मानिनः}


\twolineshloka
{वैचित्रवीर्यं राजानं धृतराष्ट्रं मनीषिणम्}
{अभिगम्य त्वरायुक्ताः श्लक्ष्णं वचनमब्रुवन् ॥दुर्योधन उवाच}


\twolineshloka
{न त्वयेदं श्रुतं राजन्यज्जगाद बृहस्पतिः}
{शक्रस्य नीतिं प्रवदन्विद्वान्देवपुरोहितः}


\twolineshloka
{सर्वोपायैर्निहन्तव्याः शत्रवः शत्रुसूदन}
{पुरा युद्धाद्बलाद्वापि प्रकुर्वन्ति तवाहितम्}


\twolineshloka
{ते वयं पाण्डवधनैः सर्वान्सम्पूज्य पार्थिवान्}
{यदि तान्योधयिष्यामः किं वै निः परिहास्यति}


\twolineshloka
{अहीनाशीविषान्क्रुद्धान्नाशाय समुपस्थितान्}
{कृत्वा कण्ठे च पृष्ठे च कः समुत्स्रष्टुमर्हति}


\twolineshloka
{आत्तशस्त्रा रथगताः कुपितास्तात पाण्डवाः}
{निःशेषान्नः करिष्यन्ति क्रुद्धा ह्याशीविषा इव}


\twolineshloka
{सन्नद्धो ह्यर्जुनो याति विधृत्य परमेषुधी}
{गाण्डीवं मुहुरादत्ते निःश्वसंश्च निरीक्षते}


\twolineshloka
{गदां गुर्वी समुद्यम्य त्वरितश्च वृकोदरः}
{स्वरथं योजयित्वाऽशु निर्यात इति नः श्रुतम्}


\twolineshloka
{नकुलः खह्गमादाय चर्म चाप्यर्धचन्द्रवत्}
{सहदेवश्च राजा च चक्रुराकारमिङ्गितैः}


\twolineshloka
{ते त्वास्थाय रथान्सर्वे बहुशस्त्रपरिच्छदान्}
{अभिघ्नान्तो रथव्रातान्सेनायोगाय निर्ययुः}


\twolineshloka
{न क्षंस्यन्ते तथाऽस्माभिर्जातु विप्रकृता हि ते}
{द्रौपद्याश्च परिक्लेशं कस्तेषां क्षन्तुमर्हति}


\twolineshloka
{` न पश्यामि रणे क्रद्धुं बीभत्सुं प्रतिवारणम्}
{भीष्मो द्रोणश्च कर्णश्च द्रौणिश्च रथिनां वरः}


\twolineshloka
{कृपश्च वृषसेनश्च विकर्णश्च जयद्रथः}
{वाह्लीकः सोमदत्तश्च भूरिर्भूरिश्रवाः शलः}


\twolineshloka
{शकुनिः ससुतश्चैव नृपाश्चान्ये च कौरवाः}
{नैते सर्वे रणोद्युक्ताः पार्थं सोढुमशक्नुवन्}


\twolineshloka
{अर्जुनेन समो लोके नास्ति वीर्ये धनुर्धरः}
{योऽर्जुनेनार्जुनस्तुल्यो द्विबाहुर्बहुबाहुना ॥धृतराष्ट्र उवाच}


\twolineshloka
{कस्त्वयोक्तः पुमान्वीरो बीभत्सुसमविक्रमः}
{तं ये व्रूहि महावीर्यं श्रोतुमिच्छामि पुत्रक ॥दुर्योधन उवाच}


\twolineshloka
{कार्तवीर्यस्य चरितं शृणु राजन्महात्मनः}
{अव्यक्तप्रभवो ब्रह्मा सर्वलोकपितामहः}


\twolineshloka
{ब्रह्मणोऽत्रिः सुतो विद्वानत्रेः पुत्रो निशाकरः}
{सोमस्य तदु बुधः पुत्रो बुधस्य तु पुरूरवाः}


\twolineshloka
{तस्याप्यध सुतोऽप्यायुरायोस्तु नहुषः सुतः}
{}


% Check verse!

% Check verse!

% Check verse!

% Check verse!

% Check verse!

\twolineshloka
{स चार्जुनोऽथ तेजस्वी तपः परमदुश्चरम्}
{------ सोऽर्जुनोऽत्रिसुतं मुनिम्}


\twolineshloka
{तस्य दत्तो वरान्प्रादाच्चतुरः पार्थिवस्य वै}
{पूर्वं बाहुसहस्रं तु प्रार्थितः परमो वरः}


\twolineshloka
{अधर्मे प्रीयमाणस्य सद्भिस्तत्र निवारणम्}
{धर्मेण पृथिवीं जित्वा धर्मेणैव हि रञ्जनम्}


\twolineshloka
{सङ्ग्रामान्सुबहून्कृत्वा हत्वा चारीन्सहस्रशः}
{सङ्ग्रामे यतमानस्य वधश्चैवाधिकाद्रणै}


\twolineshloka
{तस्य बाहुसहस्रं तु युध्यतः किल भारत}
{रथो ध्वजश्च सञ्जज्ञ इत्येवं मे श्रुतं परा}


\twolineshloka
{तथेयं पृथिवी राजन्त्सप्तद्वीपा सपत्तना}
{ससमुद्राकरा तात विधिनोग्रेण वै जिता}


\threelineshloka
{चार्जुनोऽथ तेजस्वी सप्तद्वीपेश्वरोऽभवत्}
{च राजा महायज्ञानाजहार महाबलः}
{प्रशशास महाबाहुर्महीं स च समा बहूः}


\twolineshloka
{ततोऽर्जुनः कदाचिद्वै राजन्माहिष्मतीपतिः}
{नर्मदां भरतश्रेष्ठं तां तु दारैर्ययौ सह}


\twolineshloka
{ततस्तां स नदीं गत्वा प्रविश्यन्तर्जले तदा}
{कर्तुं राजञ्जलक्रीडां ततो राजोपचक्रमे}


\twolineshloka
{तस्मिन्नेव ततः काले रावमो राक्षसैः सह}
{लङ्काया ईश्वरस्तात तं देशं प्रययौ बली}


\twolineshloka
{ततस्तमर्जुनं दृष्ट्वा नर्मदायां दशाननः}
{नित्यं क्रोधपरो धीरो वरदानेन मोहितः}


\twolineshloka
{अभ्यघावत्सुसङ्क्रुद्धो महेन्द्रं शम्बरो यथा}
{अर्जुनोऽप्यथ तं दृष्ट्वा रावणं प्रत्यवारयत्}


\twolineshloka
{ततस्तौ चक्रतुर्युद्धं रावणश्चार्जुनश्च वै}
{ततस्तु दुर्जयं वीरं वरदानेन दर्पितम्}


\twolineshloka
{राक्षसेन्द्रं मनुष्येन्द्रो जित्वा बध्वा रणे बलात्}
{बध्वा धनुर्ज्यया राजन्विवेशाथ पुरीं स्वकाम्}


\twolineshloka
{स तु तं बन्धितं श्रुत्वा पुलस्त्यो रावणं तदा}
{मोक्षयाणास बन्धाद्वै पुरे दृष्ट्वाऽर्जुनं तदा}


\twolineshloka
{ततः कदाचित्तेजस्वी कार्तवीर्योर्जुनो बली}
{समुद्रतीरं गत्वाथ विरचन्दर्पमोहितः}


\twolineshloka
{अवाकिरच्छितशरैः समुद्रं स तु भारत}
{तं समुद्रो नमस्कृत्य कृताञ्जलिरभाषत}


\threelineshloka
{आशुगान्वीर मा मुञ्च ब्रूहि किं करवाणि ते}
{मदाश्रयाणि सत्वानि त्वद्विसृष्टैर्महेषुभिः}
{बाध्यन्ते राजशार्दूल तेभ्यो देह्यभयं विभो ॥अर्जुन उवाच}


\twolineshloka
{देहि सिन्धुपते युद्धमद्यैव त्वरया मम}
{अथवा पीडयामि त्वां तस्मात्त्वं कुरु माचिरम् ॥समुद्र उवाच}


\twolineshloka
{लोके राजन्महावीर्या बहवो निवसन्ति ये}
{तेषामेकेन राजेन्द्र कुरु युद्धं महाबल ॥अर्जुन उवाच}


\twolineshloka
{मत्समो यदि सङ्ग्रामे वरायुधधरः क्वचित्}
{विद्यते तं ममाचक्ष्व यः समासेत मा मृधे ॥समुद्र उवाच}


\twolineshloka
{महर्षिर्जमदग्निस्तु यदि राजन्परिश्रुतः}
{तस्य पुत्रो रणं दातुं यथावद्वै तवार्हति ॥दुर्योधन उवाच}


\twolineshloka
{समुद्रस्य वचः श्रुत्वा राजा माहिष्मतीपतिः}
{नारदस्य च वै पूर्वं क्रोधेन महता वृतः}


\twolineshloka
{ततः प्रतिययौ शीघ्रं क्रोधेन सह भारत}
{स तमाश्रममागत्य काममेवान्वपद्यत}


\twolineshloka
{स कामं प्रतिकूलानि चकार सह बन्धुभिः}
{आयासं जनयामास रामस्य स महात्मनः}


\twolineshloka
{ततस्तेजः प्रजज्वाल रामस्यामिततेजसः}
{प्रदहन्निव सैन्यानि रश्मिमानिव तेजसा}


% Check verse!
अथ तौ चक्रतुर्युद्धं वृत्रवासवयोरिव
\twolineshloka
{ततः परशुमादाय नृपं बाहुसहस्रिणम्}
{चिच्छेद सहसा रामो बहुशाखमिव द्रुमम्}


\twolineshloka
{तं हतं पतितं दृष्ट्वा समेतास्तस्य बान्धवाः}
{असीनादाय शक्तीश्च रामं ते प्रत्यवारयन्}


\twolineshloka
{रामोऽपि रथमास्थाय धनुरायम्य सत्वरः}
{विसृजन्परमास्त्राणि व्यधमत्पार्थिवान्बली}


\twolineshloka
{ततस्तु क्षत्रिया राजञ्जामदग्न्यभयार्दिताः}
{विविशुर्गिरिदुर्गाणि मृगाः सिंहभयादिव}


\twolineshloka
{तेषां स्वविहितं कर्म तद्भयान्नानुतिष्ठति}
{प्रजा वृषलतां प्राप्ता ब्राह्मणानामदर्शनात्}


\twolineshloka
{तथा च द्रविडाः काचाः पुण्डाश्च शबरैः सह}
{वृषलत्वं परिगता विच्छिन्नाः क्षत्रधर्मिणः}


\twolineshloka
{ततस्तु हतवीरासु क्षत्रियासु पुनः पुनः}
{द्विजैरभ्युदितं क्षत्रं तानि रामो निहत्य च}


\twolineshloka
{ततस्त्रिस्मप्तमे याते रामं वागशरीरिणी}
{दिव्या प्रोवाच मधुरा सर्वलोकपरिश्रुता}


\twolineshloka
{रामराम निवर्तस्व स्वगुणं नात्र पश्यसि}
{क्षत्रबन्धूनिमान्प्रामैर्विप्रयुज्य पुनः पुनः}


\twolineshloka
{तथैव तं महात्मानमृचीकप्रमुखास्तथा}
{रामराम महावीर्य निवर्तस्वेत्यथाब्रुवन्}


\threelineshloka
{पितुर्वधमसमृष्यंस्तु रामः प्रोवाच तानृषीन्}
{नार्हा हन्त भवन्तो मां निवारयितुमित्युत ॥पितर ऊचुः}
{}


\twolineshloka
{नार्हसि क्षत्रबन्धूंस्त्वं निहन्तुं जयतां वर}
{न हि युक्तं त्वया तात ब्राह्मणेनसता नृपान् ॥दुर्योधन उवाच}


\twolineshloka
{पितॄणां वचनं श्रुत्वा क्रोधं त्यक्त्वा स भार्गवः}
{अश्वमेधसहस्राणि नरमेधशतानि च}


\twolineshloka
{इष्ट्वा सागरपर्यन्तां काश्यपाय ददौ महीम्}
{तेन रामेण सङ््ग्रामे तुल्यस्तात दयञ्जयः}


\twolineshloka
{कार्तवीर्येण च रणे तुल्यः पार्थो न संशयः}
{रणे विक्रम्य राजेन्द्र पार्थं जेतुं न शक्यते}


\chapter{अध्यायः ९७}
\twolineshloka
{शृणु राजन्पुराऽचिन्त्यानर्जुनस्य च साहसान्}
{अर्जुनो धन्विनां श्रेष्ठो दुष्करं कृतवान्पुरा}


\twolineshloka
{द्रुपदस्य पुरे राजन्द्रौपद्याश्च स्वयंवर ॥आबालवृद्धसङ्क्षोभे सर्वक्षत्रसमागमे}
{क्षिप्रकारी जले मत्स्यं दुर्निरीक्ष्यं ससर्ज ह}


\twolineshloka
{सर्वैर्नृपैरसाध्यं तत्कार्मुकप्रवरं च वै}
{क्षणेन सज्यमकरोत्सर्वक्षत्रस्य पश्यतः}


\twolineshloka
{ततो यन्त्रमयं विध्वा विसारं फल्गुनो बली}
{कृष्णया हेममाल्येन स्कन्धे स परिवेष्टितः}


\threelineshloka
{ततस्तया वृतं पार्थं दृष्ट्वा सर्वे नृपास्तदा}
{रोषात्सर्वायुधान्गृह्य क्रुद्धा वीरा महौजसः}
{वैकर्तनं पुरस्कृत्य सर्वे पार्थमुपाद्रवन्}


\twolineshloka
{स सर्वान्पार्थिवान्दृष्ट्वा क्रुद्धान्पार्थो महाबलः}
{वारयित्वा शरैस्तीक्ष्णैरजयत्तत्र स स्वयम्}


\twolineshloka
{जित्वा तु तान्महीपालान्सर्वान्कर्णपुरोगमान्}
{लेभे कृष्णां शुभां पार्थो युध्वा वीर्यबलात्तदा}


\twolineshloka
{सर्वक्षत्रसमूहेषु अम्बां भीष्मो यथा पुरा}
{ततः कदाचिद्बीभत्सुस्तीर्ययात्रां ययौ स्वयम्}


\twolineshloka
{अथोलूपीं शुभां तात नागराजसतां तदा}
{नागेष्वाप वराग्र्येषु प्रार्थितोऽथ यथा तथा}


\twolineshloka
{ततो गोदावरीं कृष्णां कावेरीं चावगाहत}
{तत्र पाण्ड्यं समासाद्य तस्य कन्यामवाप सः}


% Check verse!
लब्ध्वा जिष्णुर्मुदं तत्र ततो याम्यां दिशं ययौ
\twolineshloka
{स दक्षिणं समुद्रान्तं गत्वा चाप्सरसां च वै}
{कुमारतीर्थमासाद्य मोक्षयामास चार्जुनः}


\threelineshloka
{ग्राहरूपाश्च ताः पञ्च अतिशौर्येण वै बलात्}
{कन्यातीर्थं समभ्येत्य ततो द्वारवतीं ययौ}
{}


\twolineshloka
{तत्र कृष्णनिदेशात्स सुभद्रां प्राप्य फल्गुनः}
{तामारोप्य रथोपस्थे प्रययौ स्वपुरीं प्रति}


\twolineshloka
{अथादाय गते पार्थे ते श्रुत्वा सर्वयादवाः}
{तमभ्यधावन्त्सङ्क्रुद्धाः सिंहव्याघ्रगणा इव}


\twolineshloka
{प्रद्युम्नः कृतवर्मा च गदः सारणसात्यकी}
{आहुकश्चैव साम्बश्च चारुदेष्णो विदूरथः}


\twolineshloka
{अन्ये च यादवाः सर्वे बलदेवपुरोगमाः}
{एकमेव परे कृष्णं गजवाजिरथैर्युताः}


\twolineshloka
{अथासाद्य वने यान्तं परिवार्य धनञ्जयम्}
{चक्रुर्युद्धं सुसङ्क्रुद्धा बहुकोट्यश्च यादवाः}


\twolineshloka
{एक एव तु पार्थस्तैर्युद्धं चक्रे सुदारुणम्}
{तेन तेषां समं युद्धं मुहूर्तं प्रबभूव ह}


\threelineshloka
{ततः पार्थो रणे सर्वान्वारयित्वा शितैः शरैः}
{बलाद्विजित्य राजेन्द्र वीरस्तान्सर्वयादवान्}
{तां सुभद्रामथादाय शक्रप्रस्थं विवेश ह}


\twolineshloka
{भूयः शृणु महाराज फल्गुनस्य च साहसम्}
{ददौ स वह्नेर्बिभत्सुः प्रार्थितं खाण्डवं वनम्}


\twolineshloka
{लब्धमात्रे तु तेनाथ भगवान्हव्यवाहनः}
{भक्षितुं खाण्डवं राजंस्तत्रस्थानुपचक्रमे}


\threelineshloka
{ततस्तं भक्षयन्तं वै सव्यसाची विभावसुम्}
{रथा धन्वी शरान्गृह्य स कलापयुतः प्रभुः}
{पालयामास राजेन्द्र स्ववीर्येण महाबलः}


\twolineshloka
{ततः श्रुत्वा महेनद्रस्तु मेघांस्तान्सन्दिदेश ह}
{तेनोक्ता मेघसङ्घास्ते ववर्षुरतिवृष्टिभिः}


\twolineshloka
{ततो मेघगाणान्पार्थः शरव्रातैः समान्ततः}
{खगमैर्वारयामास तदाश्चर्यमिवाभवत्}


\threelineshloka
{वारितान्मेघसङ्घांश्च श्रुत्वा क्रुद्धः पुरन्दरः}
{पाण्डरं गजमास्थाय सर्वदेवगणैर्वृतः}
{ययौ पार्थेन संयोद्धुं रक्षार्थं खाण्डवस्य च}


\threelineshloka
{रुद्राश्च मरुतश्चैव वसवश्चाश्विनौ तदा}
{आदित्याश्चैव साध्याश्च निश्वेदेवाश्च भारत}
{गन्धर्वाश्चैव सहिता अन्ये देवगणाश्च ये}


\twolineshloka
{ते सर्वे शस्त्रसम्पन्ना दीप्यमानाः स्वतेजसा}
{धनञ्जयं जिघांसन्तः प्रपेतुर्विबुधाधिपाः}


\twolineshloka
{युगान्ते यानि दृश्यन्ते निमित्तानि महान्त्यपि}
{सर्वाणि तत्र दृश्यन्ते निमित्तानि महीपते}


\twolineshloka
{ततो देवगमाः सर्वे पार्थं समभिदुद्रुवुः}
{असम्भ्रान्तस्तु तान्दृष्ट्वा स तां देवमयीं चमूम्}


\twolineshloka
{त्वरितः फल्गुनो गृह्य तीक्ष्णांस्तानाशुगांस्तदा}
{इन्द्रं देवांश्च सम्प्रेक्ष्य तस्थौ काल इवात्यये}


\twolineshloka
{ततो देवगणाः सर्वे बीभत्सुं सपुरन्दराः}
{अवाकिरञ्छरव्रातैर्मानुषं तं महीपते}


\twolineshloka
{ततः पार्थो महातेजा गाण्डिवं गृह्य सत्वरः}
{वारयामास देवानां शरव्रातैः शरांस्तदा}


\twolineshloka
{पुनः क्रुद्धाः सुराः सर्वे मर्त्यं तं सुभहाबलाः}
{नानाशस्त्रैर्ववर्षुस्तं सव्यसाची महीपते}


\twolineshloka
{तान्पार्थः शस्त्रवर्षान्वै विसृष्टान्विबुधैस्तदा}
{द्विधा त्रिधा स चिच्छेद स एव निशितैः शरैः}


\twolineshloka
{पुनश्च पार्थः सङ्क्रुद्धो मण्डलीकृतकार्मुकः}
{देवसङ्घाञ्छरैस्तीक्ष्णैरर्पयन्वै समन्ततः}


\twolineshloka
{ततो देवगणाः सर्वे युध्वा पार्थेन वै मुहुः}
{रणे जेतुमशक्यं तं ज्ञात्वा ते भरतर्षभ}


\twolineshloka
{शान्तास्ते विबुधाः सर्वे पार्थबाणाभिपीडिताः}
{सद्विपं वासवं त्यक्त्वा दुद्रुवुः सर्वतो दिशम्}


\twolineshloka
{प्राचीं रुद्राः सगन्धर्वा दक्षिणां मरुतो ययुः}
{दिशं प्रतीचीं भीतास्ते वसवश्च तथाऽश्विनौ}


\twolineshloka
{आदित्याश्चैव विश्वे च दुद्रुवुर्वा उदङ्मुखाः}
{साध्याश्चोर्ध्वमुखा भीताश्चिन्तयन्तोऽस्य सायकान्}


\twolineshloka
{एवं सुरगणाः सर्वे प्राद्रवन्त्सर्वतो दिशम्}
{मुहुर्मुहुः प्रेक्षमाणाः पार्थमेव सकार्मुकम्}


\twolineshloka
{विद्रुतान्देवसङ्घांस्तान्रणे दृष्ट्वा पुरन्दरः}
{ततः क्रुद्धो महातेजाः पार्थं बाणैरवाकिरत्}


% Check verse!
पार्थोऽपि शक्रं विव्याथ मानुषो विबुधाधिपम्
\twolineshloka
{ततः सोऽश्ममयं वर्षं व्यसृजद्विबुधाधिपः}
{तच्छरैरर्जुनो वर्षं प्रतिजाघ्नेऽत्यमर्षणः}


\twolineshloka
{अथ संवर्धयामास तद्वर्षं देवराडपि}
{भूय एव महावीर्यं जिज्ञासुः सव्यसाचिनः}


\twolineshloka
{सोऽश्मवर्षं महावेगमिषुभिः पाण्डवोऽपि च}
{विलयं गमयामास हर्षयन्पाकशासनम्}


\twolineshloka
{उपादाय तु पाणिभ्यामङ्गदं नाम पर्वतम्}
{सद्रुमं व्यसृजच्छक्रो जिघांसुः श्वेतवाहनम्}


\twolineshloka
{ततोऽर्जुनो वेगवद्भिर्ज्वलमानैरजिह्यगैः}
{बाणैर्विध्वंसयामास गिरिराजं सहस्रधा}


\twolineshloka
{शक्रं च पातयामास शरैः पार्थो महान्युधि}
{ततः शक्रो महाराज रणे वीरं धनञ्जयम्}


\twolineshloka
{ज्ञात्वा जेतुमशक्यं तं तेजोबलसमन्वितम्}
{परां प्रीति ययौ तत्र पुत्रशौर्येण वासवः}


\twolineshloka
{तदा तत्र न तस्यास्ति दिवि कश्चिन्महायशाः}
{समर्थो निर्जये राजन्नपि साक्षात्प्रजापतिः}


\twolineshloka
{ततः पार्थः शरैर्हत्वा यक्षराक्षसपन्नगान्}
{दीप्ते चाग्नौ महातेजाः पातयामास सन्ततम्}


\twolineshloka
{प्रतिषेधयितुं पार्थं न शेकुस्तत्र केचन}
{दृष्ट्वा निवारितं शक्रं दिवि देवगणैः सह}


\twolineshloka
{यथा सुपर्णः सोमार्थं विबुधानजयत्पुरा}
{तथा जित्वा सुरान्पार्थस्तर्पयामास पावकम्}


\twolineshloka
{ततोऽर्जुनः स्ववीर्येण तर्पयित्वा विभावसुम्}
{रथं ध्वजं च सहयं दिव्यानस्त्रांश्च पाण्डवः}


\twolineshloka
{गाण्डीवं च धनुः श्रेष्ठं तूणी चाक्षयसायकौ}
{एतान्यपि च बीभत्सुर्लेभे कीर्ति च भारत}


\twolineshloka
{भूयोऽपि शृणु राजेन्द्र पार्थो गत्वोत्तरां दिशम्}
{विजित्य नववर्षांश्च सपुरांश्च सपर्वतान्}


\twolineshloka
{जम्बुद्वीपं वशे कृत्वा सर्वं तद्भरतर्षभ}
{बलाज्जित्वा नृपान्सर्वान्करे चविनिवेश्य च}


\threelineshloka
{रत्नान्यादाय सर्वाणि गत्वा चैव पुनः पुरीम्}
{ततो ज्येष्ठं महात्मानं धर्मराजं युधिष्ठिरम्}
{राजसूयं क्रतुश्रेष्ठं कारयामास भारत}


\twolineshloka
{स तान्यन्यानि कर्माणि कृतवानर्जुनः पुरा}
{अर्जुनेन समो वीर्ये त्रिषु लोकेषु न क्वचित्}


\twolineshloka
{देवदानवयक्षाश्च पिशाचोरगराक्षसाः}
{भीष्मद्रोणादयः सर्वे कुरवश्च महारथाः}


\twolineshloka
{लोके सर्वनृपाश्चैव वीराश्चान्ये धनुर्धराः}
{एते पार्थं रणे युक्ताः प्रतियोद्धुं न शक्नुयुः}


\twolineshloka
{अहं हि नित्यं कौरव्य फल्गुनं हृदि संस्थितम्}
{अपश्यं चिन्तयित्वा तं समुद्विग्नोऽस्मि तद्भयात्}


\twolineshloka
{गृहे गृहे च पश्यामि तात पार्थमहं सदा}
{शरगाण्डीवसंयुक्तं पाशहस्तमिवान्तकम्}


\twolineshloka
{अपि पार्थसहस्राणि भीतः पश्यामि भारत}
{पार्थभूतमिदं सर्वं नगरं प्रतिभाति मे}


\twolineshloka
{पार्थमेव हि पश्यामि रहिते तात भारत}
{दृष्ट्वा स्वप्नगतं पार्थमुद्धमामि विचेतनः}


\twolineshloka
{अकारादीनि नामानि अर्जुनग्रस्तचेतसः}
{अश्वाक्षराम्बुजाश्चैव त्रासं सञ्जनयन्ति मे}


\twolineshloka
{नास्ति पार्थादृते तात परवीराद्भयं मम}
{प्रह्लादं वा बलिं वापि हन्याद्धि विजयो रणे}


\twolineshloka
{तस्मात्तेन महाराज युद्धं नस्तात न क्षमम्}
{अहं तस्य प्रभावज्ञो नित्यं दुःखं वहामि च}


\twolineshloka
{पुरा हि दण्डकारण्ये मारीचस्य यथा भयम्}
{भवेद्रामे महावीर्ये तथा पार्थे भयं मम ॥धृतराष्ट्र उवाच}


\twolineshloka
{जानाम्येव महद्वीयं जिष्णोरेतद्दुरासदम्}
{एतद्वीरस्य पार्थस्य कार्षीस्त्वं तु विप्रियम्}


\twolineshloka
{द्यूतं वा शस्त्रयुद्धं वा दुवाक्यं वा कथञ्चन}
{एतेष्वेवं कृते तस्य विग्रहश्चैव वो भवेत्}


\twolineshloka
{तस्मात्त्वं पुत्र पार्थेन नित्यं स्नेहेन वर्तय}
{यश्च पार्थेन सम्बन्धो वर्तते चेन्नरो भुवि}


\twolineshloka
{तस्य नास्ति भयं किञ्चित्रिषु लोकेषु भारत}
{तस्मात्त्वं जिष्णुना वत्स नित्यं स्नेहेन वर्तय ॥दुर्योधन उवाच}


\twolineshloka
{द्यूते पार्थस्य कौरव्य मायया निकृतिः कृता}
{तस्माद्वि नो जयस्तात अन्योपायेन नो भवेत् ॥धृतराष्ट्र उवाच}


\twolineshloka
{उपायश्च न कर्तव्यः पाण्डवान्प्रति भारत}
{पार्थान्प्रति पुरा वत्स बहूपायाः कृतास्त्वया}


\twolineshloka
{तानुपायान्हि कौन्तेया बहुशो व्यतिचक्रमुः}
{तस्माद्वितं जीविताय नः कुलस्य जनस्य च}


\twolineshloka
{त्वं चिकीर्षसि चेद्वत्स समित्रः सहबान्धवः}
{सभ्रातृकस्त्वं पार्थेन नित्यं स्नेहेन वर्तय ॥वैशम्पायन उवाच}


\twolineshloka
{धृतराष्ट्रवचः श्रुत्वा राजा दुर्योधनस्तदा}
{चिन्तयित्वा मुहूर्तं तु विधिना चोदितोऽब्रवीत्'}


\twolineshloka
{पुनर्दीव्याम भद्रं ते वनवासाय पाण्डवैः}
{एवमेतान्वशे कर्तुं शक्ष्यामः पुरुषर्षभ ॥ष}


\twolineshloka
{ते वा द्वादश वर्षाणि वयं वा द्यूतनिर्जिताः}
{प्रविशेम महारण्यमजिनैः प्रतिवासिताः}


\twolineshloka
{त्रयोदशं च स्वजनैरज्ञाताः परिवत्सरम्}
{ज्ञाताश्च पुनरन्यानि वने वर्षाणि द्वादश}


\twolineshloka
{निवसेम वयं ते वा तथा द्यूतं प्रवर्तताम्}
{अक्षानुप्त्वा पुनर्द्यूतमिदं कुर्वन्तु पाण्डवः}


\twolineshloka
{एतत्कृत्यतमं राजन्नस्माकं भरतर्षभ}
{अयं हि शकुनिर्वेद सविद्यामक्षसम्पदम्}


\twolineshloka
{दृढमूलं वयं राज्ये मित्राणि परिगृह्य च}
{सारवद्विपुलं सैन्यं सत्कृत्य च दुरासदम्}


\twolineshloka
{ते च त्रयोदशं वर्षं पारयिष्यन्ति चेद्व्रतम्}
{जेष्यामस्तान्वयं राजत्रोचतां ते परन्तप ॥धृतराष्ट्र उवाच}


\twolineshloka
{तूर्णं प्रत्यानयस्वैतान्कामं व्यध्वगतानपि}
{आगच्छन्तु पुनर्द्यूतमिदं कुर्वन्तु पाण्डवः ॥वैशम्पायन उवाच}


\twolineshloka
{ततो द्रोणः सोमदत्तो बाह्लीकश्चैव गौतमः}
{विदुरो द्रोणपुत्रश्च वैश्यापुत्रश्च वीर्यवान्}


\twolineshloka
{भूरिश्रवाः शान्तनवो विकर्णश्च महारथः}
{मा द्यूतमित्यभाषन्त शमोऽस्त्विति च सर्वशः}


\twolineshloka
{अकामानां च सर्वेषां सुहृदामर्थदर्शिनाम्}
{अकरोत्पाण्डवाह्वानं धृतराष्ट्रः सुतप्रियः}


\twolineshloka
{अथाब्रवीन्महाराज धृतराष्ट्रं जनेश्वरम्}
{पुत्रहार्दाद्धर्मयुक्ता गान्धारी शोककर्शिता}


\twolineshloka
{जाते दुर्योधने क्षत्ता महामतिरभाषत}
{नीयतां परलोकाय साध्वयं कुलपांसनः}


\twolineshloka
{व्यनदज्जातमात्रो हि गोमायुरिव भारत}
{अन्तो नूनं कुलस्यास्य कुरवस्तन्निबोधत}


\twolineshloka
{मा निमज्जीः स्वदोषेण महाप्सु त्वं हि भारत}
{मा बालानामशिष्टानामभिमंस्था मतिं प्रभो}


\twolineshloka
{मा कुलस्य क्षये घोरे कारणं त्वं भविष्यसि}
{बद्धं सेतुं को नु भिन्द्याद्धमेच्छान्तं च पावकम्}


\twolineshloka
{शमे स्थितान्को नु पार्थान्कोपयेद्भरतर्षभ}
{स्मरन्तं त्वामाजमीढं स्मारयिष्याम्यहं पुनः}


\twolineshloka
{शास्त्रं न शास्ति दुर्बुद्धिं श्रेयसे चेतराय च}
{न वै वृद्धो बालमतिर्भवेद्राजन्कथञ्चन}


\twolineshloka
{त्वन्नेत्राः सन्तु ते पुत्रा मा त्वां दीर्णाः प्रहासिषुः}
{तस्मादयं मद्वचनात्त्यज्यतां कुलपांसनः}


\twolineshloka
{तथा ते न कृतं राजन्पुत्रस्नेहान्नराधिप}
{तस्य प्राप्तं फलं विद्धि कुलान्तकरणाय यत्}


\twolineshloka
{शमेन धर्मेण नयेन युक्ताया ते बुद्धिः साऽस्तु ते मा प्रमादीः}
{प्रध्वंसिनी क्रूरसमाहिता श्री-र्मृदुप्रौढा गच्छति पुत्रपौत्रान्}


\twolineshloka
{अथाब्रवीन्महाराजो गान्धारीं धर्मदर्शिनीम्}
{अन्तः कामं कुलस्यास्तु न शक्नोमि निवारितुम्}


\twolineshloka
{यथेच्छन्ति तथैवास्तु प्रत्यागच्छन्तु पाण्डवाः}
{पुनर्द्यूतं च कुर्वन्तु मामकाः पाण्डवैः सह}


\chapter{अध्यायः ९८}
\twolineshloka
{ततो व्यध्वगतं पार्थं प्रातिकामी युधिष्ठिरम्}
{उवाच वचनाद्राज्ञो धृतराष्ट्रस्य धीमतः}


\twolineshloka
{उपास्तीर्णा सभा राजन्नक्षानुप्त्वा युधिष्ठिर}
{एहि पाण्डव दीव्येति पिता त्वाह नराधिपः ॥युधिष्ठिर उवाच}


\twolineshloka
{धातुर्नियोगाद्भूतानि प्राप्नुवन्ति शुभाशुभम्}
{न निवृत्तिस्तयोरस्ति देवतव्यं पुनर्यदि}


\twolineshloka
{अक्षद्यूते समाह्वानं नियोगात्स्थविरस्य च}
{जानन्नपि क्षयकरं नातिक्रमितुमुत्सहे ॥वैशम्पायन उवाच}


\twolineshloka
{असम्भवो हेममयस्य जन्तो-स्तथापि रामो लुलुभे मृगाय}
{प्रायः समासन्नपराभवाणांधियो विपर्यस्ततरा भवन्ति}


\twolineshloka
{इति ब्रुवन्निववृते भ्रातृभिः सह पाण्डवः}
{जानांश्च शकुनेर्मायां पार्थो द्यूतमियात्पुनः}


\twolineshloka
{विविशुस्ते सभां तां तु पुनरेव महारथाः}
{व्यथयन्ति स्म चेतांसि मुहृदां भरतर्षभाः}


\twolineshloka
{यथोपजोषमासीनाः पुनर्द्यूतप्रवृत्तये}
{सर्वलोकविनाशाय दैवेनोपनिपीडिताः ॥शकुनिरुवाच}


\twolineshloka
{अमुञ्चत्स्थविरो यद्वो धनं पूजितमेव तत्}
{महाधनं ग्लहं त्वेकं शृणु भो भरतर्षभ}


\twolineshloka
{वयं वा द्वादशाब्दानि युष्माभिर्द्यूतनिर्जिताः}
{प्रविशेम महारण्यं रौरवाजिनवाससः}


\twolineshloka
{त्रयोदशं च स्वजनैरज्ञाताः परिवत्सरम्}
{ज्ञाताश्च पुनरन्यानि वने वर्षाणि द्वादश}


\twolineshloka
{अस्माभिर्निर्जिता यूयं वने द्वादश वत्सरान्}
{वसध्वं कृष्णया सार्धमजिनैः प्रतिवासिताः}


\twolineshloka
{त्रयोदशं च स्वजनैरज्ञाताः पिरवत्सरम्}
{ज्ञाताश्च पुनरन्यानि वने वर्षाणि द्वादश}


\twolineshloka
{त्रयोदशे च निर्वृत्ते पुनरेव यथोचितम्}
{स्वराज्यं प्रतिपत्तव्यमितरैरथवेतरैः}


\twolineshloka
{अनेन व्यवसायेन सहास्माभिर्युधिष्ठिर}
{अक्षानुप्त्वा पुनर्द्यूतमेहि दीव्यस्व भारत}


\threelineshloka
{अथ सभ्याः सभामध्ये समुच्छ्रितकरास्तदा}
{ऊचुरुद्विग्नमनसः संवेगात्सर्व एव हि ॥सभ्या ऊचुः}
{}


\twolineshloka
{अहो धिग्बान्धवा नैनं बोधयन्ति महद्भयम्}
{बुद्ध्या बुद्ध्येन्न वा बुद्ध्येदयं वै भरतर्षभ ॥वैशम्पायन उवाच}


\twolineshloka
{जनप्रवादान्सुबहूञ्शृण्वन्नपि नराधिपः}
{ह्रिया च धर्मसंयोगात्पार्थो द्यूतमियात्पुनः}


\twolineshloka
{जानन्नापि महाबुद्धिः पुनर्द्यूतमवर्तयत्}
{अप्यासन्नो विनाशः स्यात्कुरूणामितिचिन्तयन् ॥युधिष्ठिर उवाच}


\twolineshloka
{कथं वै मद्विधो राजा स्वधर्ममनुपालयन्}
{आहूतो विनिवर्तेत दीव्यामि शकुने त्वया ॥शकुनिरुवाच}


\twolineshloka
{गवाश्वं बहुधेनुकमपर्यन्तमजाविकम्}
{गजाः कोशो हिरण्यं च दासीदासाश्च सर्वशः}


\twolineshloka
{हित्वा नो ग्लह एवैको वनवासाय पाण्डवाः}
{यूयं वयं वा विजिता वसेम वनमाश्रिताः}


\threelineshloka
{त्रयोदशं च वै वर्णमज्ञाताः स्वजनैस्तथा}
{अनेन व्यवसायेन दीव्याम पुरुषर्षभाः}
{समुत्क्षेपेण चैकेन वनवासाय भारत}


\twolineshloka
{` वैशम्पायन उवाच ॥ एवं दैवबलाविष्टो धर्मराजो युधिष्ठिरः}
{भीष्मद्रोणाऽऽवार्यमाणो विदुरेण च धीमता}


\twolineshloka
{युयुत्सुना कृपेणाथ सञ्जयेन च भारत}
{गान्धार्या पृथया चैव भीमार्जुनयमैस्तथा}


\twolineshloka
{विकर्णेन च वीरेण द्रौपद्या द्रौणिना तथा}
{सोमदत्तेन च तथा बाह्लीकेन च धीमता}


\twolineshloka
{वार्यमाणोपि सततं न च राजन्नियच्छति}
{एवं संवार्यमाणोपि कौन्तेयो हितकाम्यया}


\twolineshloka
{देवकार्यार्थसिद्ध्यर्थं मुहूर्तं कलिमाविशत्}
{अविष्टः कलिना राजञ्छकुनिं प्रत्यभाषत}


\twolineshloka
{एवं भवत्विति तदा वनवासाय दीव्यति'}
{प्रतिजग्राह तं पार्थो ग्लहं जग्राह सौबलः ॥जितमित्येव शकुनिर्युधिष्ठिरमभाषत}


\chapter{अध्यायः ९९}
\twolineshloka
{ततः पराजिताः पार्था वनावासाय दीक्षिताः}
{अजिनान्युत्तरीयाणि जगृहुश्च यथाक्रमम्}


\twolineshloka
{अजिनैः संवृतान्दृष्ट्वा हृतराज्यानरिन्दमान्}
{प्रस्थितान्वनवासाय ततो दुःशासनोऽब्रवीत्}


\twolineshloka
{प्रवृत्तं धार्तराष्ट्रस्य चक्रं राज्ञो महात्मनः}
{पराजिताः पाण्डवेया विपत्तिं परमां गताः}


\twolineshloka
{अद्य देवाः सम्प्रयाताः समैर्वर्त्मभिरस्थलैः}
{गुणज्येष्ठास्तथा श्रेष्ठाः श्रेयांसो यद्वयं परैः}


\twolineshloka
{नरकं पातिताः पार्था दीर्घकालमनन्तकम्}
{मुखाच्च हीना राज्याच्च विनष्टाः शाश्वतीः समाः}


\twolineshloka
{धनेन मत्ता ये ते स्म धार्तराष्ट्रान्प्रहासिषुः}
{ते निर्जिता हृतधना वनमेष्यन्ति पाण्डवः}


% Check verse!
चित्रान्सन्नाहानवमुञ्चन्तु चैषांवासांसि दिव्यानि च भानुमन्ति ॥विवास्यन्तां रुरुचर्माणि सर्वेयथा ग्लहं सौबलस्याभ्युपेताः
\twolineshloka
{न सन्ति लोकेषु पुमांस ईदृशाइत्येव ये भावितबुद्धयः सदा}
{ज्ञास्यन्ति तेत्मानमिमेऽद्य पाण्डवाविपर्यये पाण्ढतिला इवाफलाः}


% Check verse!
इदं हि वासो यदि वेदृशानांमनस्विनां रौरवमाहवेषु ॥आदीक्षितानामजिनानि यद्व-द्वलीयसां पश्यत पाण्डवानाम्
\twolineshloka
{महाप्राज्ञः सौमकिर्यज्ञसेनःकन्यां पाञ्चालीं पाण्डवेभ्यः प्रदाय}
{अकार्षिद्वै सुकृतं नेह किञ्चित्क्लीबाः पार्थाः पतयो याज्ञसेन्याः}


\twolineshloka
{सूक्ष्मप्रावारानजिनोत्तरीयान्दृष्ट्वाऽरण्ये निर्धनानप्रतिष्ठान्}
{कां त्वं प्रीतिं लप्स्यसे याज्ञसेनिपतिं वृणीष्वेह यमन्यमिच्छसि}


\twolineshloka
{एते हि सर्वे कुरवः समेताःक्षान्ता दान्ताः सुद्रविणोपपन्नाः}
{एषां वृणीष्वैकतमं पतित्वेन त्वां नयेत्कालविपर्ययोऽयम्}


\twolineshloka
{यथाऽफलाः षण्ढतिला यथा चर्ममया मृगाः}
{तथैव पाण्डवाः सर्वे यथा काकयवा अपि}


\twolineshloka
{किं पाण्डवांस्ते पतितानुपास्यमोघः श्रमः षण्ढतिलानुपास्य}
{एवं नृशंसः परुषाणि पार्था-नश्रावयद्धृतराष्ट्रस्य पुत्रः}


\twolineshloka
{तद्वै श्रुत्वा भीमसेनोऽत्यमर्षीनिर्भर्त्स्योच्चैः सन्निगृह्यैव रोषात्}
{उवाच चैनं सहसैवोपगम्यसिंहो यथा हैमवतः शृगालम् ॥भीमसेन उवाच}


\twolineshloka
{क्रूर पापजनैर्जुष्टमकृतार्थं प्रभाषसे}
{गान्धारविद्यया हि त्वं राजमध्ये विकत्थसे}


\twolineshloka
{यथा तुदसि मर्माणि वाक्छरैरिह नो भृशम्}
{तथा स्मारयिता तेऽहं कृन्तन्मर्माणि संयुगे}


\twolineshloka
{ये च त्वामनुवर्तन्ते क्रोधलोभवशानुगाः}
{गोप्तारः सानुबन्धांस्तान्नेताऽस्मि यमसादनम् ॥वैशम्पायन उवाच}


\twolineshloka
{एवं ब्रुवाणमजिनैर्विवासितंदुःशासनस्तं परिनृत्यति स्म}
{मध्ये कुरूणां धर्मनिबद्धमार्गंगौर्गौरिति स्माह्वयन्मुक्तलज्जः ॥भीमसेन उवाच}


\twolineshloka
{नृशंस परुषं वक्तुं दुःशासन त्वया}
{निकृत्या हि धनं लब्ध्वा को विकत्थितुमर्हति}


\twolineshloka
{मैव स्म सुकृतां लोकान्गच्छेत्पार्थो वृकोदरः}
{यदि वक्षो हि ते भित्त्वा न पिबेच्छोणितं रणे}


\twolineshloka
{धार्तराष्ट्रान्रणे हत्त्वा मिषतां सर्वधन्विनाम्}
{शमं गन्ताऽस्मि नचिरात्सत्यमेतद्ब्रवीमि ते ॥वैशम्पायन उवाच}


\twolineshloka
{तस्य राजा सिंहगतेः सखेलंदुर्योधनो भीमसेनस्य हर्षात्}
{गतिं स्वगत्याऽनुचकार मन्दोनिर्गच्छतां पाण्डवानां सभायाः}


\twolineshloka
{नैतावता कृतमित्यब्रवीत्तंवृकोदरः सन्निवृत्तार्धकायः}
{शीघ्रं हि त्वां निहतं सानुबन्धंसंस्मार्याहं प्रतिवक्ष्यामि मूढ}


\twolineshloka
{एवं समीक्ष्यात्मनि चावमानंनियम्य मन्युं बलवान्स मानी}
{राजानुगः संसदि कौरवाणांविनिष्कामन्वाक्यमुवाच भीमः}


\twolineshloka
{अहं दुर्योधनं हन्ता कर्णं हन्ता धनञ्जयः}
{शकुनिं चाक्षकितवं सहदेवो हनिष्यति}


\twolineshloka
{इदं च भूयो वक्ष्यामि सभामध्ये बृहद्वचः}
{सत्यं देवाः करिष्यन्ति यन्नो युद्धं भविष्यति}


\twolineshloka
{सुयोधनमिमं पापं हन्ताऽस्मि गदया युधि}
{शिरः पादेन चास्याहमधिष्ठास्यामि भूतले}


\twolineshloka
{वाक्यशूरस्य चैवास्य परुषस्य दुरात्मनः}
{दुःशासनस्य रुधिरं पाताऽस्मि मृगराडिव ॥अर्जुन उवाच}


\twolineshloka
{`भीमसेन न ते सन्ति येषां वैरं त्वया त्विह}
{मत्ता मृगेषु सुखिनो न बुद्ध्यन्ते महद्भयम्'}


\twolineshloka
{नैवं वाचा व्यवसितं भीम विज्ञायते सताम्}
{इतश्चतुर्दशे वर्षे द्रष्टारो यद्भविष्यति ॥भीमसेन उवाच}


\twolineshloka
{दुर्योधनस्य कर्णस्य शकुनेश्च दुरात्मनः}
{दुःशासनचतुर्थानां भूमिः पास्यति शोणितम् ॥अर्जुन उवाच}


\twolineshloka
{असूयितारं द्रष्टारं प्रवक्तारं विकत्थनम्}
{भीमसेन नियोगात्ते हन्ताहं कर्णमाहवे}


\twolineshloka
{अर्जुनः प्रतिजानीते भीमस्य प्रियकाम्यया}
{कर्णं कर्णानुगांश्चैव रणे हन्ताऽस्मि पत्रिभिः}


\twolineshloka
{ये चान्ये प्रतियोत्स्यन्ति बुद्धिमोहेन मां नृपाः}
{तांश्च सर्वानहं बाणैर्नेताऽस्मि यमसादनम्}


\twolineshloka
{चलेद्वि हिमवान्स्थानान्निष्प्रभः स्याद्दिवाकरः}
{शैत्यं सोमात्प्रणश्येत मत्सत्यं विचलेद्यदि}


\twolineshloka
{न प्रदास्यति चेद्राज्यमितो वर्षे चतुर्दशे}
{दुर्योधनोऽभिसत्कृत्य सत्यमेतद्भविष्यति ॥वैशम्पायन उवाच}


\twolineshloka
{इत्युक्तवति पार्थे तु श्रीमान्माद्रवतीसुतः}
{प्रगृह्य विपुलं बाहुं सहदेवः प्रतापवान्}


\twolineshloka
{सौबलस्य वधं प्रेप्सुरिदं वचनमब्रवीत्}
{क्रोधसंरक्तनयनो निः श्वसन्निव पन्नगः ॥सहदेव उवाच}


\twolineshloka
{अक्षान्यान्मन्यसे मूढ गान्धाराणां यशोहर}
{नैतेऽक्षा निशिता बाणास्त्वयैते समरे वृताः}


\twolineshloka
{यथा चैवोक्तवान्भीमस्त्वामुद्दिश्य सबान्धवम्}
{कर्ताहं कर्मणस्तस्य कुरु कार्याणि सर्वशः}


\twolineshloka
{हन्ताऽस्मि तरसा युद्धे त्वामेवहे सबान्धवम्}
{यदि स्थास्यसि सङ्ग्रामे क्षत्रधर्मेण सौबल}


\twolineshloka
{सहदेववचः श्रुत्वा नकुलोऽपि विशाम्पते}
{दर्शनीयतमो नॄणामिदं वचनमब्रवीत्}


\twolineshloka
{सुतेयं यज्ञसेनस्य द्यूतेऽस्मिन्धृतराष्ट्रजैः}
{यैर्वाचः श्राविता रूक्षाः स्थितैर्दुर्योधनप्रिये}


\twolineshloka
{तान्धार्तराष्ट्रान्दुर्वृत्तान्मुमूर्षून्कालनोदितान्}
{गमयिष्यामि भूयिष्ठानहं वैवस्वतक्षयम्}


\twolineshloka
{`उलूकं च दुरात्मानं सौबलस्य सुतं प्रियम्}
{क्रूरं हन्ताऽस्मि समरे तं वै क्रूरं नराधमम्'}


\twolineshloka
{निदेशाद्धर्मराजस्य द्रौपद्याः पदवीं चरन्}
{निर्धार्तराष्ट्रां पृथिवीं कर्ताऽस्मि नचिरादिव ॥वैशम्पायन उवाच}


\twolineshloka
{एवं ते पुरुषव्याघ्राः सर्वे व्यायतबाहवः}
{प्रतिज्ञा बहुलाः कृत्वा धृतराष्ट्रमुपागमन्}


\chapter{अध्यायः १००}
% Check verse!
आमन्त्रयामि भरतांस्तथा वृद्धं पितामहम् ॥राजानं सोमदत्तं च महाराजं च बाह्लिकम्
\fourlineindentedshloka
{द्रोणं कृपं नृपांश्चान्यानश्वत्थामानमेव च}
{विदुरं धृतराष्ट्रं च धार्तराष्ट्रांश्च सर्वशः ॥ 2-100-3a`सौमदत्तिं महावीर्यं विकर्णं च महामतिम्'}
{युयुत्सुं सञ्जयं चैव तथैवान्यान्सभासदः ॥ 2-100-4a`गान्धारीं चमहाभागां मातरं च तथा पृथाम्'}
{सर्वानामन्त्र्य गच्छामि द्रष्टाऽस्मि पुनरेत्य वः ॥वैशम्पायन उवाच}


\twolineshloka
{न च किञ्चिदथोचुस्तं ह्रियाऽऽसन्ना युधिष्ठिरम्}
{मनोभिरेव कल्याणं दध्युस्ते तस्य धीमतः ॥विदुर उवाच}


\twolineshloka
{आर्या पृथा राजपुत्री नारण्यं गन्तुमर्हति}
{सुकुमारी च वृद्धा च नित्यं चैव सुखोचिता}


\threelineshloka
{इह वत्स्यति कल्याणी सत्कृता मम वेश्मनि}
{इति पार्था विजानीध्वमगदं वोऽस्तु सर्वशः ॥पाण्डवा ऊचुः}
{}


\twolineshloka
{तथेत्युक्त्वाऽब्रुवन्सर्वे यथा नो वदसेऽनघ}
{त्वं पितृव्यः पितृसमो वयं च त्वत्परायणाः}


\twolineshloka
{यथाऽऽज्ञापयसे विद्वंस्त्वं हि नः परमो गुरुः}
{यच्चान्यदपि कर्तव्यं तद्विधत्स्व महामते ॥विदुर उवाच}


\twolineshloka
{युधिष्ठिर विजानीहि ममेदं भरतर्षभ}
{नाधर्मेण जितः कश्चिद्व्यथते वै पराजये}


\twolineshloka
{त्वं वै धर्मं विजानीषे युद्धे जेता धनञ्जयः}
{हन्ताऽरीणां भीमसेनो नकुलस्त्वर्थसङ्ग्रही}


\twolineshloka
{संयन्ता सहदेवस्तु धौम्यो ब्रह्मविदुत्तमः}
{धर्मार्थकुशला चैव द्रौपदी धर्मचारिणी}


\twolineshloka
{अन्योन्यस्य प्रियाः सर्वे तथैव प्रियदर्शनाः}
{परैरभेद्याः सन्तुष्टाः को वोन न स्पृहयेदिह}


\twolineshloka
{एष वै सर्वकल्याणः समाधिस्तव भारत}
{नैनं शत्रुर्विषहते शक्रेणापि समोऽप्युत}


\twolineshloka
{हिमवत्यनुशिष्टोऽसि मेरुसावर्णिना पुरा}
{द्वैपायनेन कृष्णेन नगरे वारणावते}


\twolineshloka
{भृगुतुङ्गे च रामेण दृष्टद्वत्यां च शम्भुना}
{अश्रौषीरसि तस्यापि महर्षेरञ्जनं प्रति}


\twolineshloka
{कल्माषीतीरसंस्थस्य गतस्त्वं शिष्यतां भृगोः}
{द्रष्टा सदा नारदस्ते धौम्यस्तेऽयं पुरोहितः}


\twolineshloka
{माहासीः साम्पराये त्वं बुद्धिं तामृषिपूजिताम्}
{पुरूरवसमैलं त्वं बुद्ध्या जयसि पाण्डव}


\twolineshloka
{शक्त्या जयसि राज्ञोऽन्यानृषीन्धर्गोपसेवया}
{ऐन्द्रे जये धृतमना याम्ये कोपविधारणे}


% Check verse!
तथा विसर्गे कौबेरे वारुणे कोपविधारणे ॥आत्मप्रदानं सौम्यत्वमद्भ्यश्चैवोपजीवनम्
\twolineshloka
{भूमेः क्षमा च तेजश्च समग्रं सूर्यमण्डलात्}
{वायोर्बलं प्राप्नुहि त्वं भूतेभ्यश्चात्मसम्पदम्}


\twolineshloka
{अगदं वोऽस्तु भद्रं वो द्रष्टाऽस्मि पुनरागतान्}
{आपद्धर्मार्थकृच्छ्रेषु सर्वकार्येषु वा पुनः}


\twolineshloka
{यथावत्प्रतिपद्येथाः काले काले युधिष्ठिर}
{आपृष्टोऽसीह कौन्तेय स्वस्ति प्राप्नुहि भारत}


\twolineshloka
{कृतार्थं स्वस्मिमन्तं त्वां द्रक्ष्यामः पुनरागतम्}
{न हि वो वृजिनं किञ्चिद्वेद कश्चित्पुराकृतम् ॥वैशम्पायन उवाच}


\twolineshloka
{एवमुक्तस्तथेत्युक्त्वा पाण्डवः सत्यविक्रमः}
{भीष्मद्रोणौ नमस्कृत्य प्रातिष्ठत युधिष्ठिरः}


\chapter{अध्यायः १०१}
\twolineshloka
{ततः सम्प्रस्थिते तत्र धर्मराजे तदा नृप}
{जनाः समन्ताद्द्रष्टुं तं समारुरुहुरातुराः}


\twolineshloka
{ततः प्रासादवर्याणि विमानशिखराणि च}
{गोपुराणि च सर्वाणि वृक्षानन्यांश्च सर्वशः}


\twolineshloka
{अथाधिरुह्य सस्त्रीका उदासीना व्यलोकयन्}
{न हि रथ्यास्तदा शक्या गन्तुं ताश्च जनाकुलाः}


\twolineshloka
{आरुह्य स्मानतास्तत्र दीनाः पश्यन्ति पाण्डवान्}
{पदातिं वर्जितच्छत्रं चेलभूषणवर्जितम्}


\threelineshloka
{वल्कलाजिनसंवीतं पार्थं दृष्ट्वा जनास्तदा}
{ऊचुर्बहुविधा वाचो भृशोपहतचेतसः ॥जना ऊचुः}
{}


\twolineshloka
{यं यान्तमनुयाति स्म चतुरङ्गबलं महत्}
{तमेकं कृष्णया सार्धमनुगच्छन्ति पाण्डवाः}


\twolineshloka
{चत्वारो भ्रातरश्चैव धौम्यश्चैव पुरोहितः}
{भीमार्जुनौ वारयित्वा निकृत्या बद्धकार्मुकौ}


\twolineshloka
{धर्म एवास्थितो येन त्यक्त्वा राज्यं महात्मना}
{या न शक्या पुरा द्रष्टुं भूतैराकाशगैरपि}


\twolineshloka
{तामद्य कृष्णां पश्यन्ति राजमार्गगता जनाः}
{अङ्गरागोचितां कृष्णां रक्तचन्दनसेविनीम्}


\twolineshloka
{वर्षमुष्णं च शीतं च नेष्यत्याशु विवर्णताम्}
{अद्य नूनं पृथा देवी सत्यमाविश्य भाषते}


\twolineshloka
{पुत्रान्स्नुषां च देवी तु द्रष्टुमद्याथ नार्हति}
{निर्गुणस्यापि पुत्रस्य कथं स्याहुः स्वदर्शनम्}


\twolineshloka
{किम्पुनर्यस्य लोकोऽयं जितो वृत्तेन केवलम्}
{आनृशंस्यमनुक्रोशो धृतिः शीलं दमः शमः}


\twolineshloka
{पाण्डवं शोभयन्त्येते षड्गुणाः पुरुषोत्तमम्}
{तस्मादस्योपघातेन प्रजाः परमपीडिताः}


\twolineshloka
{औदकानीव सत्वानि ग्रीष्मे सलिलसङ्क्षयात्}
{पीडया पीडितं सर्वं जगदस्य जगत्पतेः}


\twolineshloka
{मूलस्यैवोपघातेन वृक्षः पुष्पफलोपगः}
{मूलं ह्येष मनुष्याणां धर्मराजो महाद्युतिः}


\twolineshloka
{पुष्पं फलं च पत्रं च शाखास्तस्येतरे जनाः}
{ते भ्रातर इव क्षिप्रं सपुत्राः सहबान्धवाः}


\twolineshloka
{गच्छन्तमनुगच्छामो येन गच्छति पाण्डवः}
{उद्यानानि परित्यज्य क्षेत्राणि च गृहाणि च}


\twolineshloka
{एकदुःखसुखाः पार्थमनुयाम सुधार्मिकम्}
{समुच्छ्रितपताकानि परिध्वस्ताजिराणि च}


\twolineshloka
{उपात्तधनधान्यानि हृतसाराणि सर्वशः}
{रजसाऽप्यवकीर्णानि परित्यक्तानि दैवतैः}


\twolineshloka
{मूषकैः परिधावद्भिरुद्बलैरावृतानि च}
{अपेतोदकधूमानि हीनसंमार्जनानि च}


\twolineshloka
{प्रनष्टबलिकर्मेज्यामन्त्रहोमजपानि च}
{दुष्कालेनेव भग्नानि भिन्नभाजनवन्ति च}


\twolineshloka
{अस्मत्त्यक्तानि वेश्मानि सौबलः प्रतिपद्यताम्}
{वनं नगरमेवास्तु येन गच्छन्ति पाण्डवाः}


\twolineshloka
{अस्माभिश्च परित्यक्तं पुरं सम्पद्यतां वनम्}
{बिलानि दंष्ट्रिणः सर्वे वानि मृगपक्षिणः}


\twolineshloka
{त्यजन्त्वस्मद्भयाद्भीता गजाः सिंहा वनान्यपि}
{अनाक्रान्तं प्रपद्यन्तः सेवमानं त्यजन्तु च}


\twolineshloka
{तृणमांसफलादानां देशांस्त्यक्त्वा मृगद्विजाः}
{वयं पार्थैर्वने सम्यक्सह वत्स्याम निर्वृताः}


\twolineshloka
{इत्येवं विविधा वाचो नानाजनसमीरिताः}
{शुश्राव पार्थः श्रुत्वा च न विचक्रेऽस्य मानसम्}


\twolineshloka
{ततः प्रासादसंस्थास्तु समन्ताद्वै गृहे गृहे}
{ब्राह्मणक्षत्रियविशां शूद्राणां चैव योषितः}


\twolineshloka
{गत्वा स्वगृहजालानि उत्पाट्यावरणानि च}
{ददृशुः पाण्डवान्दीनान्वल्कलाजिनवाससः}


\twolineshloka
{कृष्णां त्वदृष्टपूर्वां तां व्रजन्तीं पद्भिरेव च}
{एकवस्त्रां रुदन्तीं तां मुक्तकेशीं रजस्वलाम्}


\threelineshloka
{दृष्ट्वा तदा स्त्रियः सर्वा विवर्णवदना भृशम्}
{विलप्य बहुधा मोहाद्दुःखशोकेन पीडिताः}
{हाहा धिग्धिग्धिगित्युक्त्वा नेत्रैरश्रूण्यवर्तयन्}


\twolineshloka
{जनस्याथ वजः श्रुत्वा स राजा भ्रातृभिः सह}
{उद्दिश्य वनावासाय प्रतस्थे कृतनिश्चयः ॥वैशम्पायन उवाच}


\twolineshloka
{तस्मिन्सम्प्रस्थिते कृष्णा पृथां प्राप्य यशस्विनीम्}
{अपृच्छद्भृशदुःखार्ता याश्चान्यास्तत्र योषितः}


% Check verse!
ततो निनादः सुमहान्पाण्डवान्तः पुरेऽभवत्
\twolineshloka
{कुन्ती च भृशशन्तप्ता द्रौपदीं प्रेक्ष्य गच्छतीम्}
{शोकविह्वलया वाचा कृच्छ्राद्वचनमब्रवीत्}


\twolineshloka
{वत्से शोको न ते कार्यः प्राप्येदं व्यसनं महत्}
{स्त्रीधर्माणामभिज्ञाऽसि शीलाचारवती तथा}


\twolineshloka
{न त्वां शन्देष्टुमर्हामि भर्तॄन्प्रति शुचिस्मिते}
{साध्वी गुणसमापन्ना भूषितं ते कुलद्वयम्}


\twolineshloka
{सभाग्याः कुरवश्चेमे ये न दग्धास्त्वयाऽनघे}
{अरिष्टं व्रज पन्थानं मदनुध्यानबृंहिता}


\twolineshloka
{भाविन्यर्थे हि सत्स्त्रीणां वैकृतं नोपजायते}
{गुरुधर्माभिगुप्ता च श्रेयः क्षिप्रमवप्स्यसि}


\twolineshloka
{सहदेवश्च मे पुत्रः सदाऽवेक्ष्यो वने वसन्}
{यथेदं व्यसनं प्राप्य नायं सीदेन्महामतिः ॥वैशम्पायन उवाच}


\twolineshloka
{तथेन्युक्त्वा तु सा देवी स्रवन्नेत्रजलाविला}
{शोणिताक्तैकवसना मुक्तकेशी विनिर्ययौ}


\twolineshloka
{तां क्रोशन्तीं पृथा दुःखादनुवव्राज गच्छतीम्}
{अथापश्यन्सुतान्सर्वान्हृताभरणवाससः}


\twolineshloka
{रुरुचर्मावृततनून्ह्रिया किञ्चिदवाह्मुखान्}
{परैः परीतान्संहृष्टैः सुहृद्भिश्चानुशोचितान्}


\threelineshloka
{तदवस्थान्सुतान्सर्वानुपसृत्यातिवत्सला}
{स्वजमानाऽवदच्छेकात्तत्तद्विलपती बहु ॥कुन्त्युवाच}
{}


\twolineshloka
{कथं सद्धर्मचारित्रान्वृत्तस्थितिविभूषितान्}
{अक्षुद्रान्दृढभक्तांश्च दैवतेज्यापरान्सदा}


\twolineshloka
{व्यसनं वः समभ्यागात्कोऽयं विधिविपर्ययः}
{कस्यापध्यानजं चेदमागः पश्यामि वो धिया}


\twolineshloka
{स्यात्तु मद्भाग्यदोषोऽयं याऽहं युष्मानजीजनम्}
{दुःखायासभुजोऽत्यर्थं युक्तानप्युत्तमैर्गुणैः}


\twolineshloka
{कथं वत्स्यथ दुर्गेषु वने ऋद्धिविनाकृताः}
{वीर्यसत्वबलोत्साहतेजोभिरकृशाः कृशाः}


\twolineshloka
{यद्येवमहमज्ञास्यं वने वासं हि वो ध्रुवम्}
{शतशृङ्गान्मृते पाण्डौ नागमिष्यं गजाह्वयम्}


\twolineshloka
{धन्यं वः पितरं मन्ये तपोमेधान्वितं तथा}
{यः पुत्राधिमसम्प्राप्य स्वर्गेच्छामकरोत्प्रियाम्}


\twolineshloka
{धन्यां चातीन्द्रियज्ञानामिमां प्राप्तां परां गतिम्}
{मन्ये तु माद्रीं धर्मज्ञां कल्याणीं सर्वथैव तु}


\twolineshloka
{रत्या मत्या च गत्या च ययाऽहमभिसन्धिता}
{जीवितप्रियतां मह्यं धिङ्मां सङ्क्लेशभागिनीम्}


\threelineshloka
{पुत्रका न विहास्ये वः कृच्छ्रलब्धान्प्रियान्सतः}
{साऽहं यास्यामि हि वनं हा कृष्णे किं जहासि माम्}
{}


\twolineshloka
{अन्तवत्यसुधर्मेऽस्मिन्धात्रा किं नु प्रमादतः}
{ममान्तो नैव विहितस्तेनायुर्न जहाति माम्}


\twolineshloka
{हा कृष्ण द्वारकावासिन्क्वासि सङ्कर्षणानुज}
{कस्मान्न त्रायसे दुःखान्मां चेमांश्च नरोत्तमान्}


\twolineshloka
{अनादिनिधनं ये त्वामनुध्यायन्ति वै नराः}
{तांस्त्वं पासीत्ययं वादः स गतो व्यर्थतां कथम्}


\twolineshloka
{इमे सद्धर्ममाहात्म्ययशोवीर्यानुवर्तिनः}
{नार्हन्ति व्यसनं भोक्तुं नन्वेषां क्रियतां दया}


% Check verse!
सेयं नीत्यर्थविज्ञेषु कथमापदुपागता ॥स्थितेषु कुलनाथेषु कथमापदुपागता
\twolineshloka
{हा पाण्डो हा महाराज क्वासि किं समुपेक्षसे}
{पुत्रान्विवास्यतः साधूनरिभिर्द्यूतनिर्जितान्}


\twolineshloka
{सहदेव निवर्तस्व ननु त्वमसि मे प्रियः}
{शरीरादपि माद्रेय मां मा त्याक्षीः कुपुत्रवत्}


\twolineshloka
{व्रजन्तु ब्रातरस्तेऽमी यदि सत्याभिसन्धिनः}
{मत्परित्राणजं धर्ममिहैव त्वमवाप्नुहि ॥वैशम्पायन उवाच}


\twolineshloka
{एवं विपलतीं कुन्तीमभिवाद्य प्रणम्य च}
{पाण्डवा विगतानन्दा वनायैव प्रवव्रजुः}


\twolineshloka
{विदुरश्चापि तामार्तां कुन्तीमाश्वास्य हेतुभिः}
{प्रावेशयद्गृहं क्षत्ता स्वयमार्ततरः शनैः}


\twolineshloka
{धार्तराष्ट्रस्त्रिस्ताश्च निखिलेनोपलभ्य तत्}
{गमनं परिकर्षं च कृष्णाया द्यूतमण्डले}


\twolineshloka
{रुरुदुः सुस्वनं सर्वा विनिन्दन्त्यः कुरून्भृशम्}
{दध्युश्च सुचिरं कालं करासक्तमुखाम्बुजाः}


\twolineshloka
{राजा च धृतराष्ट्रस्तु पुत्राणामनयं तदा}
{ध्यायन्नुद्विग्नहृदयो न शान्तिमधिजग्मिवान्}


\twolineshloka
{स चिन्तयन्ननेकाग्रः शोकव्याकुलचेतनः}
{क्षत्तुः सम्प्रेषयामास शीघ्रमागम्यतामिति}


\twolineshloka
{ततो जगाम विदुरो धृतराष्ट्रनिवेशनम्}
{तं पर्यपृच्छत्संविग्नो धृतराष्ट्रो जनाधिपः}


\chapter{अध्यायः १०२}
\twolineshloka
{तमागतमथो राजा विदुरं दीर्घदर्शिनम्}
{साशङ्क इव प्रपच्छ धृतराष्टोऽम्बिकासुतः}


\twolineshloka
{कथं गच्छति कौन्तेयो धर्मपुत्रो युधिष्ठिरः}
{भीमसेनः सव्यसाची माद्रीपुत्री च पाण्डवौ}


\twolineshloka
{धौम्यश्चैव कथं क्षत्तर्द्रौपदी च यशस्विनी}
{श्रोतुमिच्छाम्यहं सर्वं तेषां शंस विचेष्टितम् ॥विदुर उवाच}


\twolineshloka
{वस्त्रेण संवृत्य मुखं कुन्तीपुत्रो युधिष्ठिरः}
{बाहू विशालौ सम्पश्यन्भीमो गच्छति पाण्डवः}


\twolineshloka
{सिकता वपन्सव्यसाची राजानमनुगच्छति}
{माद्रीपुत्रः सहदेवो मुखमालिप्य गच्छति}


\twolineshloka
{पांसूपलिप्तसर्वाङ्गो नकुलश्चित्तविह्वलः}
{दर्शनीयतमो लोके राजानमनुगच्छति}


\twolineshloka
{कृष्णा तु केशैः प्रच्छाद्य मुखमायतलोचना}
{दर्शनीया प्ररुदती राजानमनुगच्छति}


\twolineshloka
{धौम्यो रौद्राणि सामानि याम्यानि च विशाम्पते}
{गायन्गच्छति मार्गेषु कुशानादाय पाणिना ॥धृतराष्ट्र उवाच}


\twolineshloka
{विविधानीह रूपाणि कृत्वा गच्छन्ति पाण्डवाः}
{तन्ममाचक्ष्व विदुर कस्मादेवं व्रजन्ति ते ॥विदुर उवाच}


\twolineshloka
{निकृतस्यापि ते पुत्रैर्हृते राज्ये धनेषु च}
{न धर्माच्चलते बुद्धिर्धर्मराजस्य धीमतः}


\twolineshloka
{योऽसौ राजा घृणी नित्यं धार्तराष्ट्रेषु भारत}
{निकृत्या भ्रंशितः क्रोधान्नोन्मीलयति लोचने}


\twolineshloka
{नाहं जनं निर्दहेयं दृष्ट्वा घोरेण चक्षुषा}
{स पिधाय मुखं राजा तस्माद्गच्छति पाण्डवः}


\twolineshloka
{यथा च भीमो व्रजति निगदतः शृणु}
{बाह्वोर्बले नास्ति समो ममेति भरतर्षभ}


\twolineshloka
{बाहू विशालौ कृत्वाऽसौ तेन भीमोपि गच्छति}
{बाहू विदर्शयन्राजन्बाहुद्रविणदर्पितः}


\twolineshloka
{चिकीर्षन्कर्म शत्रुभ्यो बाहुद्रव्यानुरूपतः}
{प्रदिशञ्शरसम्पातान्कुन्तीपुत्रोऽर्जुनस्तदा}


\threelineshloka
{सिकता वपन्सव्यसाची राजानमनुगच्छति}
{असक्ताः सिकतास्तस्य यथा सम्प्रति भारत}
{असक्तं शरवर्षाणि तथा मोक्ष्यति शत्रुषु}


\twolineshloka
{न मे कश्चिद्विजानीयान्मुखमद्येति भारत}
{मुखमालिप्य तेनासौ सहदेवोऽपि गच्छति}


\twolineshloka
{नाहं मनांस्याददेयं मार्गे स्त्रीणामिति प्रभो}
{पांसूपलिप्तसर्वाङ्गो नकुलस्तेन गच्छति}


\twolineshloka
{एकवस्त्रा प्ररुदती मुक्तकेशी रजस्वला}
{शोणितेनाक्तवसना द्रौपदी वाक्यमब्रवीत्}


\twolineshloka
{यत्कृतेऽहमिदं प्राप्ता तेषां वर्षे चतुर्दशे}
{हतपत्यो हतसुता हतबन्धुजनप्रियाः}


\twolineshloka
{बन्धुशोणितदिग्धाङ्ग्यो मुक्तकेश्यो रजस्वलाः}
{एवं कृतोदका भार्याः प्रवेक्ष्यन्ति गजाह्वयम्}


\twolineshloka
{कृत्वा तु नैर्ऋतान्दर्भान्धीरो धौम्यः पुरोहितः}
{सामानि गायन्याम्यानि पुरतो याति भारत}


\threelineshloka
{हतेषु भरतेष्वाजौ कुरूणां गुरवस्तदा}
{एवं सामानि -- न्तीत्युक्त्वा धौम्योपि गच्छति}
{}


\twolineshloka
{`प्रस्थाप्य पाण्डवाञ्शेषान्निः शेषस्ते भविष्यति}
{इति धौम्यो व्यवसितो रौद्रसामानि गायति ॥धृतराष्ट्र उवाच}


\twolineshloka
{किमब्रुवन्नैकृतिकः किं वा नागरिका जनाः}
{तथ्येन मे समाचक्ष्व क्षत्तः सर्वमशेषतः ॥विदुर उवाच}


\threelineshloka
{ब्राह्मणाः क्षत्रिया वैश्याः शूद्रा येऽन्ये वदन्त्यथ}
{तच्छृणुष्व महाराज वक्ष्यते च मया तव ॥प्रकृतय ऊचुः}
{}


\twolineshloka
{हाहा गच्छन्ति नो नाथाः समवेक्षध्वमीदृशम्}
{अहो धिक्कुरुवृद्धानां बालानामिव चेष्टितम्}


\twolineshloka
{राष्ट्रेभ्यः पाण्डुदायादाँल्लोभान्निर्वासयन्ति ये}
{अनाथाः स्म वयं सर्वे वियुक्ताः पाण्डुनन्दनैः}


\twolineshloka
{दुर्विनीतेषु लुब्धेषु का प्रीतिः कौरवेषु नः}
{इति पौराः सुदुः खार्ताः क्रोशन्ति स्म पुनः पुनः}


\twolineshloka
{एवमाकारलिङ्गैस्ते व्यवसायं मनोगतम्}
{कथयन्तश्च कौन्तेया वनं जग्मुर्मनस्विनः}


\twolineshloka
{एवं तेषु नराग्र्येषु निर्यत्सु गजसाह्वयात्}
{अनभ्रे विद्युतश्चासन्भूमिश्च समकम्पत}


\twolineshloka
{राहुरग्रसदादित्यमपर्वणि विशाम्पते}
{उल्का चाप्यपसव्येन पुरं कृत्वा व्यशीर्यत}


\twolineshloka
{प्रत्याहरन्ति क्रव्यादा गृध्रगोमायुवायसाः}
{देवायतनचैत्येषु प्राकाराट्टालकेषु च}


\twolineshloka
{एवमेते महोत्पाताः प्रादुरासन्दुरासदाः}
{भरतानामभावाय राजन्दुर्मन्त्रिते तव ॥वैशम्पायन उवाच}


\twolineshloka
{एवं प्रवदतोरेव तयोस्तत्र विशाम्पते}
{धृतराष्ट्रस्य राज्ञश्च विदुरस्य च धीमतः}


\twolineshloka
{नारदश्च सभामध्ये कुरूणामग्रतः स्थितः}
{महर्षिभिः परिवृतो रौद्रं वाक्यमुवाच ह}


\twolineshloka
{इतश्चतुर्दशे वर्षे विङ्क्ष्यन्तीह कौरवाः}
{दुर्योधनापराधेन भीमार्जुनबलेन च}


\twolineshloka
{इत्युक्त्वा दिवमाक्रम्य क्षिप्रमन्तरधीयत}
{ब्राह्मीं श्रियं सुविपुलां बिभ्रद्देवर्षिसत्तमः}


\twolineshloka
{ततो दुर्योधनः कर्णः शकुनिश्चापि सौबलः}
{द्रोणं द्वीपममन्यन्त राज्यं चास्मै न्यवेदयन्}


\twolineshloka
{अथाब्रवीत्ततो द्रोणो दुर्योधनममर्षणम्}
{दुःशासनं च कर्णं च सर्वानेव च भारतान्}


\twolineshloka
{अवध्यान्पाण्डवान्प्राहुर्देवपुत्रान्द्विजातयः}
{अहं चै शरणं प्राप्तान्वर्तमानो यथाबलम्}


\twolineshloka
{गन्ता सर्वात्मना भक्त्या धार्तराष्ट्रान्सराजकान्}
{नोत्सहेयं परित्यक्तुं दैवं हि बलवत्तरम्}


\twolineshloka
{धर्मतः पाण्डुपुत्रा वै वनं गच्छन्ति निर्जिताः}
{ते च द्वादश वर्षाणि वने वत्स्यन्ति पाण्डवः}


\twolineshloka
{चरितब्रह्मचर्याश्च क्रोधामर्षवशानुगाः}
{वैरं निर्यातयिष्यन्ति महद्दुःखाय पाण्डवाः}


\twolineshloka
{मया च भ्रंशितो राजन्द्रुपदः सखिविग्रहे}
{पुत्रार्थमयजद्राजा वधाय मम भारत}


\twolineshloka
{याजोपयाजतपसा पुत्रं लेभे स पावकात्}
{धृष्टद्युम्नं द्रौपदीं च वेदीमध्यात्सुमध्यमाम्}


\twolineshloka
{धृष्टद्युम्नस्तु पार्थानां स्यालः सम्बन्धतो मतः}
{पाण्डवानां प्रियरतस्तस्मान्मां भयमाविशत्}


\twolineshloka
{ज्वालावर्णो देवदत्तो धनुष्मान्कवची शरी}
{मर्त्यधर्मतया तस्मादद्य मे साध्वसो महान्}


\twolineshloka
{गतो हि पक्षतां तेषां पार्षतः परवीरहा}
{रथातिरथसङ्ख्यायां योऽग्रणीरर्जुनो युवा}


\twolineshloka
{सृष्टप्राणो भृशतरं तेन चेत्सङ्गमो मम}
{किमन्यद्दुःखमधिकं परमं भुवि कौरवाः}


\twolineshloka
{धृष्टद्युम्नो द्रोणमृत्युरिति विप्रथितं वचः}
{मद्वधायाश्रितोऽप्येष लोके चाप्यतिविश्रुतः}


\twolineshloka
{सोऽयं नूनमनुप्राप्तस्त्वत्कृते कालपर्ययः}
{त्वरितं कुरुत श्रेयो नैतदेतावता कृतम्}


\twolineshloka
{मुहूर्तं सुखमेवैतत्तालच्छायेव हैमनी}
{जयध्वं च महायज्ञैर्भोगानश्नीत दत्त च}


\twolineshloka
{इतश्चतुर्दशे वर्षे महत्प्राप्स्यथ वैशसम्}
{दुर्योधन निशम्यैतत्प्रतिपद्य यथेच्छसि}


\twolineshloka
{शमं वा पाण्डुपुत्रेण प्रयुङ्क्ष्व यदि मन्यसे}
{वैशम्पायन उवाच ॥द्रोणस्य --चनं श्रुत्वा धृतराष्ट्रोऽब्रवीदिदम्}


\threelineshloka
{सम्यगाह गुरुः क्षत्तरुपावर्तय पाण्डवान्}
{यदि ते न निवर्तन्ते सत्कृता यान्तु पाण्डवाः}
{}


% Check verse!
सशस्त्ररथपादाता भोगवन्तश्च पुत्रकाः
\chapter{अध्यायः १०३}
\twolineshloka
{वनं गतेषु पार्थेषु निर्जितेषु दुरोदरे}
{धृतराष्ट्रं महाराज तदा चिन्ता समाविशत्}


\twolineshloka
{तं चिन्तयानमासीनं धृतराष्ट्रं जनेश्वरम्}
{निः श्वसन्तमनेकाग्रमिति होवाच सञ्जयः}


\twolineshloka
{अवाप्य वसुसम्पूर्णां वसुधां वसुधाधिप}
{प्रव्राज्य पाण्डवान्राज्याद्राजन्किमनुशोचसि ॥धृतराष्ट्र उवाच}


\twolineshloka
{अशोच्यत्वं कुतस्तेषां येषां वैरं भविष्यति}
{पाण्डवैर्युद्धशौण्डैर्हि बलवद्भिर्महारथैः ॥सञ्जय उवाच}


\twolineshloka
{तवेदं सुकृतं राजन्महद्वैरमुपस्थितम्}
{विनाशो येन लोकस्य सानुबन्धो भविष्यति}


\threelineshloka
{वार्यमाणो हि भीष्मेण द्रोणेन विदुरेण च}
{पाण्डवानां प्रियां भार्यां द्रौपदीं धर्मचारिणीम्}
{}


\twolineshloka
{प्राहिणोदानयेहेति पुत्रो दुर्योधनस्तव}
{सूतपुत्रं सुमन्दात्मा निर्लज्जः प्रातिकामिनम्}


\twolineshloka
{यस्मै देवाः प्रयच्छन्ति पुरुषाय पराभवम्}
{बुद्धिं तस्यापकर्षन्ति सोऽवाचीनानि पश्यति}


\twolineshloka
{बुद्धौ कलुषभूतायां विनाशे समुपस्थिते}
{अनयो नयसङ्काशो हृदयान्नापसर्पति}


\twolineshloka
{अनर्थाश्चार्थरूपेण अर्थाश्चानर्थरूपिणः}
{उत्तिष्ठन्ति विनाशाय नूनं तच्चास्य रोचते}


\twolineshloka
{न कालो दण्डमुद्यम्य शिरः कृन्तति कस्यचित्}
{कालस्य बलमेतार्वाद्वपरीतार्थदर्शनम्}


\twolineshloka
{आसादितमिदं घोरं तुमुलं रोमहर्षणम्}
{पाञ्चालीमपकर्षद्भिः सभामध्ये तपस्विनीम्}


\twolineshloka
{अयोनिजां रूपवतीं कुले जातां विभावसोः}
{को नु तां सर्वधर्मज्ञां परिभूय यशस्विनीम्}


\twolineshloka
{पर्यानयेत्सभामध्ये विना दुर्द्यूतदेविनम्}
{स्त्रीधर्मिणी वरारोहा शोणितेन परिप्लुता}


\twolineshloka
{एकवस्त्राथ पाञ्चाली पाण्डवानभ्यवैक्षत}
{हृतस्वान्हृतराज्यांश्च हृतवस्त्रान्हृतश्रियः}


\twolineshloka
{विहीनान्सर्वकामेभ्यो दासभावमुपागतान्}
{धर्मपाशपरिक्षिप्तानशक्तानिव विक्रमे}


\twolineshloka
{क्रुद्धां चानर्हतीं कृष्णां दुःखितां कुरुसंसदि}
{दुर्योधनश्च कर्णश्च कटुकान्यभ्यभाषताम्}


\twolineshloka
{इति सर्वमिदं राजन्नाकुलं प्रतिभाति मे}
{धृतराष्ट्र उवाच ॥तस्याः कृपणचक्षुर्भ्यां प्रदह्येतापि मेदिनी}


\twolineshloka
{अपि शेषं भवेदद्य पुत्राणां मम सञ्जय}
{भरतानां स्त्रियः सर्वा गान्धार्या सह सङ्गताः}


\twolineshloka
{प्राकोशन्भैरवं तत्र दृष्ट्वा कृष्णां सभागताम्}
{धर्मिष्टां धर्मपत्नीं च रूपयौवनशालिनीम्}


\twolineshloka
{प्रजाभिःसह सङ्गम्य ह्यनुशोचन्ति नित्यशः}
{अग्निहोत्राणि सायाह्ने च चाहूयन्त सर्वशः}


\twolineshloka
{ब्राह्मणाः कुपिताश्चासन्द्रौपद्याः परिकर्षणे}
{आसीन्निष्ठानको घोरो निर्घातश्च महानभूत्}


\twolineshloka
{दिव उल्काश्चापतन्त राहुश्चार्कमुपाग्रसत्}
{अपर्वणि महाघोरं प्रजानां जयन्भयम्}


\twolineshloka
{तथैव रथशालासु प्रादुरासीद्धुताशनः}
{ध्वजाश्चापि व्यशीर्यन्त भरतानामभूतये}


\twolineshloka
{दुर्योधनस्याग्निहोत्रे प्राक्रोशन्भैरवं शिवाः}
{तास्तदा प्रत्यभाषन्त रासभाः सर्वतो दिशः}


\twolineshloka
{प्रातिष्ठत ततो भीष्मो द्रोणेन सह सञ्जय}
{कृपश्च सोमदत्तश्च बाह्लीकश्च महामनाः}


\twolineshloka
{ततोऽहमब्रुवं तत्र विदुरेण प्रचोदितः}
{वरं ददानि कृष्णायै काङ्क्षितं यद्यदिच्छति}


\twolineshloka
{अवृणोत्तत्र पाञ्चाली पाण्डवाना --सताम्}
{सरथान्सधनुष्कांश्चाप्यनुज्ञासिषमप्यहम्}


\twolineshloka
{अथाब्रवीन्महाप्राज्ञो विदुरः सर्वधर्मवित्}
{एतदन्तास्तु भरता यद्व-कृष्णा सभां गता}


\twolineshloka
{यैषा पाञ्चालराजस्य सुता सा श्रीरनुत्तमा}
{पाञ्चाली पाण्डवानेतान्दैवसृष्टोपसर्पति}


\twolineshloka
{तस्याः पार्थाः परिक्लेशं न क्षंस्यन्ते ह्यमर्षणाः}
{वृष्णयो वा महेष्वासाः पाञ्चाला वा महारथाः}


\twolineshloka
{तेन सत्याभिसन्धेन वासुदेवेन रक्षिताः}
{आगमिष्यति बीभत्सुः पञ्चालैः परिवारितः}


\twolineshloka
{तेषां मध्ये महेष्वासो भीमसेनो महाबलः}
{आगमिष्यति धुन्वानो गदां दण्डमिवान्तकः}


\twolineshloka
{ततो गाण्डीवनिर्घोषं श्रुत्वा पार्थस्य धीमतः}
{गदावेगं च भीमस्य नालं सोढुं नराधिपाः}


\twolineshloka
{तत्र मे रोचते नित्यं पार्थैः साम न विग्रहः}
{कुरुभ्यो हि सदा मन्ये पाण्डवान्बलवत्तरान्}


\threelineshloka
{तथा हि बलवान्राजा जरासन्धो महाद्युतिः}
{बाहुप्रहरणेनैव भीमेन निहतो युधि}
{}


\twolineshloka
{तस्य ते शम एवास्तु पाण्डवैर्भरतर्षभ}
{उभयोः पक्षयोर्युक्तं क्रियतामविशङ्कया}


\twolineshloka
{एवं कृते महाराज परं श्रेयस्त्वमाप्स्यसि}
{एवं गावल्गणे क्षत्ता धर्मार्थसहितं वचः}


% Check verse!
उक्तवान्न गृहीतं वै मया पुत्रहितैषिणा
