\part{स्त्रीपर्व}
\chapter{अध्यायः १}
\threelineshloka
{श्रीवेदव्यासाय नमः}
{नारायणं नमस्कृत्य नरं चैव नरोत्तमम्}
{देवीं सरस्वतीं व्यासं ततो जयमुदीरयेत्}


\threelineshloka
{जनमेजय उवाच}
{हते दुर्योधने चैव हते सैन्ये च सर्वशः}
{धृतराष्ट्रो महाराज श्रुत्वा किमकरोन्मुने}


\twolineshloka
{तथैव कौरवो राजा धर्मपुत्रो महामनाः}
{कृपप्रभृतयश्चैव किमकुर्वत ते त्रयः}


\threelineshloka
{अश्वत्थाम्नः कृतं कर्म शापश्चान्योन्यकारितः}
{वृत्तान्मुत्रं ब्रूहि यदभाषत सञ्जयः ॥वैशम्पायन उवाच}
{}


\twolineshloka
{हते पुत्रशते दीनं छिन्नशाखमिव द्रुमम्}
{पुत्रशोकाभिसन्प्तं धृतराष्ट्रं महीपतिम्}


\twolineshloka
{ध्यानमूकत्वमापन्नं चिन्तया समभिप्लुतम्}
{सञ्जयो जयतां श्रेष्ठ राजानं वाक्यमब्रवीत्}


\threelineshloka
{किं शोचसि महाराज नास्ति शोके सहायता}
{अक्षौहिण्यो हताश्चाष्टौ दश चैव विशाम्पते}
{निर्जितेयं वसुमती शून्या स्थास्यति केवलम्}


\twolineshloka
{नानादिग्भ्यः समागम्य नानाजात्या नराधिपाः}
{सहितास्व पुत्रेण सर्वे वै निधनं गताः}


\fourlineindentedshloka
{पितॄणां पुत्रपौत्राणां ज्ञातीनां सुहृदां तथा}
{गुरूणां चानुपूर्व्येण ये चान्येऽनुचरा हताः}
{प्रेतकार्याणि सर्वाणि कारयस्व नराधिप ॥वैशम्पायन उवाच}
{}


\threelineshloka
{तच्छ्रुत्वा करुणं पुत्रपौत्रवधार्दितः}
{पपात भुवि दुर्धर्षो वाताहत इव द्रुमः ॥धृतराष्ट्र उवाच}
{}


\twolineshloka
{हतपुत्रो हतामात्यो हतसर्वसुहृज्जनः}
{दुःखं नूनं गमिष्यामि विचरन्पृथिवीमिमाम्}


\twolineshloka
{किन्नु बन्धुविहीनस्य जीवितेन ममाद्य वै}
{लूनपक्षस्य इव मे वैनतेयस्य सञ्जय}


\twolineshloka
{हृतराज्यो हतसुहृद्वतपुत्रश्च वै तथा}
{न भ्राजिष्ये महाप्राज्ञ क्षीणरश्मिरिवांशुमान्}


\twolineshloka
{न कृतं सुहृदां वाक्यं जामद्ग्न्यस्य च ॥सभामध्ये तु कृष्णेन यच्छ्रेयोऽभिहितं मम}
{}


\twolineshloka
{अलं वैरेण ते राजन्पुत्रः संगृद्यतामिति}
{तच्च वाक्यमकृत्वाऽहं भृशं तप्यामि दुर्मतिः}


% Check verse!
न हि श्रोताऽस्मि भीष्मस्य शर्मयुक्तं प्रभाषितम्
\threelineshloka
{दुर्योधनस्य च तथा वृषभस्येव नर्दतः}
{दुःशासनवधं श्रुत्वा कर्णस्य च विपर्ययम्}
{द्रोणसूर्योपरागं च हृदयं मे विदीर्यते}


\twolineshloka
{न स्मराम्यात्मनः किञ्चित्पुरा सञ्जय दुष्कृतम्}
{यस्येदं फलमद्येह मया मूढेन भुज्यते}


\twolineshloka
{नूनं व्यपकृतं किञ्चिन्मया पूर्वेषु जन्मसु}
{येन मां दुःखभागेषु धाता कर्मसु युक्तवान्}


\twolineshloka
{परिणामश्च वयसः सर्वबन्धुक्षयश्च मे}
{सुहृन्मित्रविनाशश्च दैवयोगादुपागतः}


% Check verse!
कोन्वस्ति दुःखिततरो मत्तोऽन्यो हि पुमान्भुवि
\threelineshloka
{तन्मामद्यैव पश्यन्तु पाण्डवाः संशितव्रताः}
{विवृतं ब्रह्मलोकस्य दीर्घमध्वानमास्थितम् ॥वैशम्पायन उवाच}
{}


\twolineshloka
{तस्य लालप्यमानस्य बहु शोकं विचिन्वतः}
{शोकापहं नरेन्द्रस्य सञ्जयो वाक्यमब्रवीत्}


\twolineshloka
{शोकं राजन्व्यपनुद श्रुतास्ते वेदनिश्चयाः}
{शास्त्रागमाश्च विविधा वृद्धेभ्यो नृपसत्तम}


\twolineshloka
{सृञ्जये पुत्रशोकार्ते यदूर्चुर्मुनयः पुरा}
{यथा यौवनजं दर्पमास्थिते ते सुते नृप}


\twolineshloka
{न त्वया सुहृदां वाक्यं ब्रुवतामवधारितम्}
{स्वार्थश्च न कृतः कश्चिल्लुब्धेन फलगृद्धिना}


\twolineshloka
{असिनैवैकधारेण स्वबुद्ध्या तु विचेष्टितम्}
{प्रायशोऽवृत्तसम्पन्नाः सततं पर्युपासिताः}


\twolineshloka
{यस्य दुःशासनो मन्त्री राधेयश्च दुरात्मवान्}
{शकुनिश्चैव दुष्टात्मा चित्रसेनश्च दुर्मतिः}


\twolineshloka
{अनल्पं येन वै सर्वं शल्यभूतं कृतं जगत्}
{कुरुवृद्धस्य भीष्मस्य गान्धार्या विदुरस्य च}


\twolineshloka
{द्रोणस्य च महाराज कृपस्य च शरद्वतः}
{कृष्णस्य च महाबाहो नारदस्य च धीमतः}


\threelineshloka
{ऋषीणां च तथाऽन्येषां व्यासस्यामिततेजसः}
{न कृतं तेन वचनं तव पुत्रेण भारत}
{क्षपिताः क्षत्रियाः सर्वे शत्रूणां वर्धितं यशः}


\twolineshloka
{अधर्मसंयुतं किञ्चिन्नित्यं युद्धमिति ब्रुवन्}
{मध्यस्थो हि त्वमप्यासीर्न क्षमं किञ्चिदुक्तवान्}


% Check verse!
धूर्धरेण त्वया भारस्तुलया न समं धृतः
\twolineshloka
{आदावेव मनुष्येण वर्तितव्यं यथाक्रमम्}
{यथा नातीतमर्थं वै पश्चात्तापेन युज्यते}


\twolineshloka
{पुत्रगृद्ध्या त्वया राजन्प्रियं तस्य चिकीर्षितम्}
{पश्चात्तापमिमं प्राप्तो न त्वं शोचितुमर्हसि}


\twolineshloka
{मधु यः केवलं दृष्ट्वा प्रपातं नानुपश्यति}
{स भ्रष्टोमधुलोभेन शोचत्येव यथा भवान्}


\twolineshloka
{अर्थान्न शोचन्प्राप्नोति न शोचन्विन्दते सुखम्}
{न शोचञ्श्रियमाप्नोति न शोचन्विन्दते जयम्}


\twolineshloka
{स्वयमुत्पादयित्वाऽग्निं परीतस्तेन योऽग्निना}
{दह्यमानः पुनस्तापं भजते न स पण्डितः}


\twolineshloka
{त्वयैव ससुतेनायं वाक्यवायुसमीरितः}
{लोभाज्येन च संसिक्तो ज्वलितः पार्थपावकः}


\twolineshloka
{तस्मिन्समिद्धे पतिताः शलभा इव ते सुताः}
{तान्केशवाग्निनिर्दग्धान्न त्वं शोचितुमर्हसि}


\twolineshloka
{यच्चाश्रुपातकलिलं वदनं वहसे नृप}
{अशास्त्रदृष्टमेतद्धि न प्रशंसन्ति पण्डिताः}


\threelineshloka
{विस्फुलिङ्गा इव ह्येतान्दहन्ति किल मानवान्}
{जहीहि मन्युं बुद्ध्या वै धास्यात्मानमात्मना ॥ 41 ॥वैशम्पायन उवाच}
{}


\twolineshloka
{एवमाश्वासितं तेन सञ्जयेन महात्मना}
{विदुरो भूय एवाह बुद्धिपूर्वं परन्तप}


\chapter{अध्यायः २}
\twolineshloka
{वैशम्पायन उवाच}
{}


\threelineshloka
{ततोऽमृतरसैर्वाक्यैर्ह्लादयन्पुरुषर्षभम्}
{वैचित्रवीर्यं विदुरो यदुवाच निबोध तत् ॥विदुर उवाच}
{}


\twolineshloka
{उत्तिष्ठ राजन्किं शेषे धारयात्मानमात्मना}
{एषा वै सर्वसत्वानां लोकेश्वर परा गतिः}


\twolineshloka
{सर्वे क्षयान्ता निचयाः पतनान्ताः समुच्छ्रयाः}
{संयोगा विप्रयोगान्ता मरणान्तं च जीवितम्}


\twolineshloka
{यदा शूरं च भीरुं च यमः कर्षति भारत}
{तत्किं न योत्स्यन्ति हि ते क्षत्रियाः क्षत्रियर्षभ}


\twolineshloka
{अयुध्यमानो म्रियते युध्यमानश्च जीवति}
{कालं प्राप्य महाराज न कश्चिदतिवर्तते}


\twolineshloka
{अभावादीनि भूतानि भावमध्यानि भारत}
{अभावनिधनान्येव तत्र का परिदेवना}


\twolineshloka
{न शोचन्मृतमन्वेति न शोचन्म्रियते नरः}
{एवं सांसिद्धिके लोके किमर्थमनुशोचसि}


\twolineshloka
{कालः कर्षति भूतानि सर्वाणि विविधान्युत}
{न कालस्य प्रियः कश्चिन्न द्वेष्यः कुरुसत्तम}


\twolineshloka
{यथा वायुस्तृणाग्राणि संवर्तयति सर्वशः}
{तथा कालवशं यान्ति भूतानि भरतर्षभ}


\twolineshloka
{एकसार्थप्रयातानां सर्वेषां तत्र गामिनाम्}
{यस्य कालः स यात्यग्रे तत्र का परिदेवना}


\twolineshloka
{न चाप्येतान्हतान्युद्धे राजञ्शोचितुमर्हसि}
{प्रमाणं यदि शास्त्राणि गतास्ते परमां गतिम्}


\twolineshloka
{सर्वे स्वाध्यायवन्तो हि सर्वे च चरितव्रताः}
{सर्वे चाभिमुखाः क्षीणास्तत्र का परिदेवना}


\twolineshloka
{अदर्शनादापतिताः पुनश्चादर्शनं गताः}
{नैते तव न तेषां त्वं तत्र का परिदेवना}


\twolineshloka
{हतो हि लभते स्वर्गं जित्वा च लभते यशः}
{उभयं नो बहुगुणं नास्ति निष्फलता रणे}


\twolineshloka
{तेषां कामदुघाँल्लोकानिन्द्रः सङ्कल्पयिष्यति}
{इन्द्रस्यातिथयो ह्येते भवन्ति भरतर्षभ}


\twolineshloka
{न यज्ञैर्दक्षिणावद्भिर्न तपोभिर्न विद्यया}
{स्वर्गं यान्ति तथा मर्त्या यथा शूरा रणे हताः}


\twolineshloka
{शरीराग्निषु शूराणां जुहुवुस्ते शराहुतीः}
{हूयमानाञ्शरांश्चैव सेहुस्तेजस्विनो मिथः}


\twolineshloka
{एवं राजंस्तवाचक्षे स्वर्ग्यं पन्थानमुत्तमम्}
{न युद्धादधिकं किञ्चित्क्षत्रियस्येह विद्यते}


\twolineshloka
{क्षत्रियास्ते महात्मानः शूराः समितिशोभनाः}
{आशिषः परमाः प्राप्ता न शोच्याः सर्व एव हि}


\twolineshloka
{आत्मानमात्मनाऽऽश्वास्य मा शुचः पुरुषर्षभ}
{नाद्य शोकाभिभूतस्त्वं कायमुत्स्रष्टुमर्हसि}


\twolineshloka
{मातापितृसहस्राणि पुत्रदारशतानि च}
{संसारेष्वनुभूतानि कस्य ते कस्य वा वयम्}


\twolineshloka
{शोकस्थानसहसाणि भयस्थानशतानि च}
{दिवसेदिवसे मूढमाविशन्ति न पण़्डितम्}


\twolineshloka
{न कालस्य प्रियः कश्चिन्न द्वेष्यः कुरुसत्तम}
{न मध्यस्थः क्वचित्कालः सर्वं कालः प्रकर्षति}


\twolineshloka
{कालः पचति भूतानि कालः संहरते प्रजाः}
{कालः सुप्तेषु जागर्ति कालो हि दुरतिक्रमः}


\twolineshloka
{अनित्यं यौवनं रूपं जीवितं द्रव्यसञ्चयः}
{आरोग्यं प्रियसंवासो गृद्ध्येदेषु न पण्डितः}


\twolineshloka
{न जानपदिकं दुःखमेकः शोचितुमर्हसि}
{अप्यभावेन युज्येत तच्चास्य न निवर्तते}


\twolineshloka
{अशोचन्प्रतिकुर्वीत यदि पश्येत्पराक्रमम्}
{भैषज्यमेतद्दुःखस्य यदेतन्नानुचिन्तयेत्}


\twolineshloka
{चिन्त्यमानं हि न व्येति भूयश्चापि प्रवर्धते ॥अनिष्टसम्प्रयोगाच्च विप्रयोगात्प्रियस्य च}
{}


\twolineshloka
{मानुषा मानसैर्दुःखैर्युज्यन्ते येऽल्पबुद्धयः ॥नार्थो न धर्मो न सुखं यदेतदनुशोचति}
{}


\twolineshloka
{तच्च नाप्नोति कार्यार्थं त्रिवर्गाच्चैव भ्रश्यते ॥अन्योन्यबाधनावस्थां प्राप्य वैषयिकीं नराः}
{}


\twolineshloka
{असन्तुष्टाः प्रमुह्यन्ति सन्तोषं यान्ति पण्डिताः ॥प्रज्ञया मानसं दुःखं हन्याच्छारीरमौषधैः}
{}


\twolineshloka
{एतद्विज्ञानसामर्थ्यं न बालैः समतामियात् ॥शयानं चानुशेते हि तिष्ठन्तं चानुतिष्ठति}
{}


\twolineshloka
{अनुधावति धावन्तं कर्म पूर्वकृतं नरम् ॥यस्यांयस्यामवस्थायां यत्करोति शुभाशुभम्}
{}


\twolineshloka
{तस्यांतस्यामवस्थायां तत्तत्फलमवाप्नुते ॥[येनयेन शरीरेण यद्यत्कर्म करोति यः}
{}


% Check verse!
तेनतेन शरीरेण तत्फलं समुपाश्नुते ॥आत्मैव ह्यात्मनः साक्षी कृतस्यापकृतस्य च
\twolineshloka
{शुभेन कर्मणा सौख्यं दुःखं पापेन कर्मणा}
{कृतं भवति सर्वत्र नाकृतं विद्यते क्वचित्}


\twolineshloka
{न हि ज्ञानविरुद्धेषु बह्वपायेषु कर्मसु}
{मूलघातिषु सज्जन्ते बुद्धिमन्तो भवद्विघाः ॥]}


\chapter{अध्यायः ३}
\twolineshloka
{वृतराष्ट्र उवाच}
{}


\twolineshloka
{सुमाषितं महाप्राज्ञ शोकोऽयं विगतो मम}
{भूय एव तु वाक्यानि श्रोतुमिच्छामि तत्त्वतः}


\threelineshloka
{अनिष्टानां च संसर्गादिष्टानां च निवर्तनात्}
{कथं हि मानसं दुःखं विप्रयुज्यन्ति पण्डिताः ॥विदुर उवाच}
{}


\twolineshloka
{वतोयतो मनो दुःखात्सुखाद्र विप्रमुच्यते}
{तवत्तसः शगं लब्ध्वा सुगतिं विन्दते बुधः}


\twolineshloka
{अशाधतमिदं सर्वं चिन्त्यमानं नरर्षभ}
{कदलीसन्निभो लोकः सारो ह्यस्य न विद्यते}


\twolineshloka
{[यदा प्राज्ञाश्च मूढाश्च धनवन्तोऽथ निर्धनाः}
{सर्वे पितृवनं प्राप्य स्वपन्ति विगतज्वराः}


\twolineshloka
{निर्मांसैरस्थिभूयिष्ठैर्गात्रैः स्नायुनिबन्धिभिः}
{किं विशेषं प्रपश्यन्ति तत्र तेषां परे जनाः}


\twolineshloka
{येन प्रत्यवगच्छेयुः कुलरूपविशेषणम्}
{कस्मादन्योन्यमिच्छन्ति विप्रलब्धधियो नराः ॥]}


\twolineshloka
{गृहाण्येव हि मर्त्यानामाहुर्देहानि पण्डिताः}
{कालेन विनियुज्यन्ते सत्वमेकं तु शोभनम्}


\twolineshloka
{यथा जीर्णमजीर्णं वा वस्त्रं त्यक्त्वा तु पूरुषः}
{अन्यद्रोचयते वस्त्रमेवं देहाः शरीरिणाम्}


\twolineshloka
{वैचित्रवीर्य सत्यं हि दुःखं वा यदि वा सुखम्}
{प्राप्नुवन्तीह भूतानि स्वकृतेनैव कर्मणा}


\twolineshloka
{कर्मणा प्राप्यते स्वर्गः सुखं दुःखं च भारत}
{ततो वहति तं भारमवशः स्ववशोऽपि वा}


\twolineshloka
{यथा च मृन्मयं भाण्डं चक्रारूढं विपद्यते}
{किञ्चित्प्रक्रियमाणं वा कृतमात्रमथापि वा}


\twolineshloka
{हीनं वाऽप्यवरोप्यं वा अवतीर्णमथापि वा}
{आर्द्रं वाऽप्यथवा शुष्कं पच्यमानमथापि वा}


\twolineshloka
{अवतार्यन्तमापाङ्कादुद्धृतं चापि भारत}
{अथवा परिभुज्जन्तमेवं देहाः शरीरिणाम्}


\twolineshloka
{गर्भस्थो वा प्रसूतो वाऽप्यथवा दशरात्रिकः}
{अर्धमासगतो वाऽपि मासमात्रगतोऽपि वा}


\twolineshloka
{संवत्सरगतो वापि द्विसंवत्सर एव वा}
{यौवनस्थोऽथ मध्यस्थो वृद्धो वापि विपद्यते}


\twolineshloka
{प्राक्कर्मभिस्तु भूतानि भवन्ति नभवन्ति च}
{एवं संबर्धिते लोके किमर्थमनुतप्यसे}


\twolineshloka
{यथा तु सलिलं राजन्क्रीडार्थमनुसञ्चरन्}
{उन्मज्जेच्च निमज्जेच्च चेष्टते च नराधिप}


\twolineshloka
{एवं संसारगहने उन्मज्जननिमज्जने}
{कर्मयोगेन बध्यन्ते क्लिश्यन्ते चाल्पबुद्धयः}


\twolineshloka
{ये तु प्राज्ञाः स्थिता मध्ये संसारान्तगतैषिणः}
{समागमज्ञा भूतानां ते यान्ति परमां गतिम्}


\chapter{अध्यायः ४}
\twolineshloka
{धृतराष्ट्र उवाच}
{}


\threelineshloka
{कथं संसारगहनं विज्ञेयं वदतां वर}
{एतदिच्छाम्यहं श्रोतुं तत्त्वमाख्याहि पृच्छतः ॥विदुर उवाच}
{}


\twolineshloka
{जन्मप्रभृति भूतानां क्रियाः सर्वास्तु लक्षयन्}
{पूर्वमेवेह कलिले वसते किञ्चिदन्तरम्}


\twolineshloka
{ततः स पञ्चमेऽतीते मासे मांसमकल्पयत्}
{ततः सर्वाङ्गसम्पूर्णो गर्भो मासे तु जायते}


\twolineshloka
{अमेध्यमध्ये वसति मांसशोणितलेपने}
{ततस्तु वायुवेगेन ऊर्ध्वपादो ह्यधोमुखः}


\twolineshloka
{योनिद्वारमुपागम्य बहून्क्लेशान्समृच्छति}
{योनिसंपीडनाच्चैव पूर्वकर्मभिरन्वितः}


\twolineshloka
{तस्मान्मुक्तः स संसारादन्यान्पश्यत्युपद्रवान्}
{ग्रहास्तमनुच्छन्ति सारमेया इवामिषम्}


\twolineshloka
{ततः कालान्तरे प्राप्ते व्याधयश्चापि तं तथा}
{उपसर्पन्ति जीवन्तं बध्यमानं स्वकर्मभिः}


\twolineshloka
{बद्धमिन्द्रियपाशैस्तं सङ्गकामुकमातुरम्}
{व्यसनान्यनुवर्तन्ते विविधानि नराधिप}


\twolineshloka
{बाध्यमानश्च तैर्भूयो नैव तृप्तिमुपैति सः}
{[तदा नावैति चैवायं प्रकुर्वन्साध्वसाधु वा ॥]}


\twolineshloka
{तत्रैनं परिपश्यन्ति ये ध्यानपरिनिष्ठिताः}
{अयं न बुध्यते तावद्यमलोकादिहागमे}


\threelineshloka
{यमदूतैर्विकृष्यंश्च मृत्युं कालेन गच्छति}
{वाग्घीनस्य च या मात्रा इष्टानिष्टकृताऽस्य वै}
{भूय एवात्मनाऽऽत्मानं बध्यमानमुपैति सः}


\twolineshloka
{अहो विनिकृतो लोको लोभेन च वशीकृतः}
{लोभक्रोधमदोन्मत्तो नात्मानमवबुध्यते}


\twolineshloka
{कुलीनत्वे च रमते दुष्कुलीनान्विकुत्सयन्}
{धनदर्पेण दृप्तश्च दरिद्रान्परिकुत्सयन्}


\twolineshloka
{मूर्खानिति परानाह नात्मानं समवेक्षते}
{दोषान्क्षिपति चान्येषां नात्मानं शास्तुमिच्छति}


\twolineshloka
{यदा प्राज्ञाश्च मूर्खाश्च धनवन्तश्च निर्धनाः}
{कुलीनाश्चाकुलीनाश्च मानिनोऽथाप्यमानिनः}


\twolineshloka
{सर्वे पितृवनं प्राप्ताः स्वपन्ति विगतत्वचः}
{निर्मांसैरस्थिभूयिष्ठैर्गात्रैः स्नायुनिबन्धनैः}


\twolineshloka
{किं विशेषं प्रपश्यन्ति तत्र तेषां परे जनाः}
{येन प्रत्यवगच्छेयुः कुलरूपविशेषणम्}


\twolineshloka
{यदा सर्वे समं न्यस्ताः स्वपन्ति धरणीतले}
{कस्मादन्योन्यमिच्छन्ति विप्रलब्धुमिहाबुधाः}


\threelineshloka
{प्रत्यक्षं च परोक्षं च यो निशम्य श्रुतिं त्विमाम्}
{[अध्रुवे जीवलोकेऽस्मिन्यो धर्ममनुपालयन्}
{]जन्मप्रभृति वर्तेत प्राप्नुयात्परमां गतिम्}


\twolineshloka
{एवं सर्वं विदित्वा वै यस्तत्त्वमनुवर्तते}
{स प्रमोक्षयते चैव पन्थानं मनुजेश्वर}


\chapter{अध्यायः ५}
\twolineshloka
{धृतराष्ट्र उवाच}
{}


\threelineshloka
{यदिदं धर्मगहनं बुद्ध्या समनुबुध्यते}
{एतद्विस्तरतः सर्वं बुद्धिमार्गं प्रशंस मे ॥विदुर उवाच}
{}


\twolineshloka
{अत्र ते सर्वयिष्यामि नमस्कृत्वा स्वयम्भुवे}
{यथा संसारगहनं वदन्ति परमर्षयः}


% Check verse!
कश्चिन्महति कान्तारे वर्तमानो द्विजः किल ॥महद्दुर्गमनुप्राप्तो वनं क्रव्यादसङ्कुलम्
\twolineshloka
{सिंहव्याघ्रगजर्क्षौघैरतिघोरमहास्वनैः}
{पिशितादैरतिभयैर्महोग्राकृतिभिस्तथा}


\twolineshloka
{समन्तात्सम्परिक्षिप्तं यस्माद्द्रष्टुर्महद्भयम्}
{तदस्य दृष्ट्वा हृदयमुद्वेगमगमत्परम्}


\twolineshloka
{अभ्युच्छ्रयन्ति रमाणि विक्रियाश्च परन्तप}
{स तद्वनं व्यनुसरन्सम्प्रधावन्नितस्ततः}


\threelineshloka
{वीक्षमाणो दिशः सर्वाः शरणं क्व भवेदिति}
{स तेषां नाशमन्विच्छन्प्रद्रुतो भयपीडितः}
{न च निर्याति वै दूरं न वा तैर्विप्रयुज्यते}


\twolineshloka
{अथापश्यद्वनं गूढं समन्ताद्वागुरावृतम्}
{बाहुभ्यां सम्परिष्वक्तः स्त्रिया परमघोरया}


\twolineshloka
{पञ्चशीर्षधरैर्नागैः शैलैरिव समुन्नतैः}
{नभःस्पृशैर्महावृक्षैः परिक्षिप्तं महावनम्}


\twolineshloka
{वनमध्ये च तत्राभूदुदपानः समावृतः}
{वल्लीभिस्तृणनद्धाभिर्गूढाभिरभिसंवृतः}


\twolineshloka
{पपात स द्विजस्तत्र निगूढे सलिलाशये}
{विलग्नश्चाभवत्तस्मिँल्लतासन्तानसङ्कुले}


\twolineshloka
{पनसस्य यथा जातं वृन्तबद्धं महाफलम्}
{स तथा लम्बते तत्र ह्यूर्ध्वपादो ह्यधः शिराः}


\twolineshloka
{अथ तत्रापि चान्योऽस्य भूयो जात उपद्रवः}
{कूपमध्ये महानागमपश्यत महाबलम्}


% Check verse!
कूपपीनाहवेलायामपश्यत् महागजम्
\twolineshloka
{षड्वक्त्रं कृष्णशबलं द्विषट्कपदचारिणम्}
{क्रमेण परिसर्पन्तं वल्लीवृक्षसमावृतम्}


\twolineshloka
{तस्य शाखाप्रशाखासु वृक्षशाखावलम्बिनः}
{नानारूपा मधुकरा घोररूपा भयावहाः}


\twolineshloka
{आसते मधु संवृत्य पूर्वमेव निकेतजाः}
{भूयोभूयः समीहन्ते मधूनि भरतर्षभ}


\twolineshloka
{स्वादनीयानि भूतानां यैर्बालो न वितृप्यते}
{तेषां मधूनां बहुधा धारा प्रस्रवते तदा}


\twolineshloka
{आलम्बमानः स पुमान्धारां पिबति सर्वदा}
{न चास्य तृष्णा विरता पिबमानस्य सङ्कटे}


\twolineshloka
{अभीप्सति च तां नित्यमतृप्तः स पुनः पुनः}
{न चास्य जीविते राजन्निर्वेदः समजायत}


\twolineshloka
{तत्रैव च मनुष्यस्य जीविताशा प्रतिष्ठिता}
{कृष्णाः श्वेताश्च तं वृक्षं निकृन्तन्ति स्म मूषिकाः}


\twolineshloka
{व्यालैश्च तद्वनं दुर्गं स्त्रिया च परमोग्रया}
{कूपाधस्ताच्च नागेन पीनाहे कुञ्जरेण च}


\twolineshloka
{वृक्षप्रपाताच्च भयं मूषिकेभ्यश्च पञ्चमम्}
{मधुलोभान्मधुकरैः षष्ठमाहुर्महद्भयम्}


\twolineshloka
{एवं स वसते तत्र क्षिप्तः संसारसागरे}
{न चैव जीविताशायां निर्वेदमुपगच्छति}


\chapter{अध्यायः ६}
\twolineshloka
{धृतराष्ट्र उवाच}
{}


\twolineshloka
{अहो खलु महद्दुःखं कृच्छ्रवासं वसत्यसौ}
{कथं तस्य रतिस्तत्र तुष्टिर्वा वदतां वर}


\twolineshloka
{स देशः क्वनु यत्रासौ वसते धर्मसङ्कटे}
{कथं वा स विमुच्येत नरस्तस्मान्महाभयात्}


\threelineshloka
{एतन्मे सर्वमाचक्ष्व साधु चेष्टामहे तदा}
{कृपा मे महती जाता तस्याभ्युद्धरणेन हि ॥विदुर उवाच}
{}


\twolineshloka
{उपाख्यानमिदं राजन्मोक्षविद्भिरुदाहृतम्}
{सुगतिं विन्दते येन परलोकेषु मानवः}


\twolineshloka
{उच्यते यत्तु कान्तारं महासंसार एव सः}
{वनं दुर्गं हि यच्चैतत्संसारगहनं हि तत्}


% Check verse!
ये च ते कथिता व्याला व्याधयस्ते प्रकीर्तिताः
\twolineshloka
{या सा नारी बृहत्काया अध्यतिष्ठत तत्र वै}
{तामाहुस्तु जरां प्राज्ञा रूपवर्णविनाशिनीम्}


\threelineshloka
{स यस्तु कूपो नृपते स तु देहः शरीरिणाम्}
{यस्तत्र वसतेऽधस्तान्महाहिः काल एव सः}
{अन्तकः सर्वभूतानां देहिनां प्राणाहार्यसौ}


\twolineshloka
{कूपमध्ये च या जाता वल्ली यत्र स मानवः}
{प्रोतो ययाऽभवल्लग्नो जीविताशा शरीरिणाम्}


\twolineshloka
{स यस्तु कूपपीनाहे तं वृक्षं परिसर्पति}
{षड्वक्त्रः कुञ्जरो राजन्स तु संवत्सरः स्मृतः}


\twolineshloka
{मुखानि ऋतवो मासाः पादा द्वादशा कीर्तिताः}
{ये तु वृंक्षं निकृन्तन्ति मूषिकाः पन्नगास्तथा}


\twolineshloka
{रात्र्यहानि तु तान्याहुर्भूतानां परिचिन्तकाः}
{ये ते मधुकरास्तत्र कामास्ते परिकीर्तिताः}


\twolineshloka
{यास्तु ता बहुशो धाराः स्रवन्ति मधुनिस्रवम्}
{तांस्तु कामरसान्विन्द्याद्यत्र सज्जन्ति मानवाः}


\twolineshloka
{एवं संसारचक्रस्य परिवृत्तिं विदुर्बुधाः}
{येन संसारचक्रस्य पाशांश्छिन्दन्ति सर्वथा}


\chapter{अध्यायः ७}
\twolineshloka
{धृतराष्ट्र उवाच}
{}


\threelineshloka
{अहोऽभिहितमाख्यान भवता तत्त्वदर्शिना}
{भूय एव तु मे हर्षः श्रोतुं वागमृतं तव ॥विदुर उवाच}
{}


\twolineshloka
{शृणु भूयः प्रवक्ष्यामि मार्गस्यैतस्य विस्तरम्}
{यच्छ्रुत्वा विप्रमुच्यन्ते संसाराद्वि विचक्षणाः}


\twolineshloka
{यथा तु पुरुषो राजन्दीर्घमध्वानमास्थितः}
{क्वचित्क्वचिच्छ्रमस्थानं कुरुते वासमेव वा}


\twolineshloka
{एवं संसारपर्याये गर्भवासेषु भारत}
{कुर्वन्ति दुर्बुधा वासं मुच्यन्ते तत्र पण्डिताः}


\twolineshloka
{तस्मादध्वानमेवैतमाहुः शास्त्रविदो जनाः}
{यत्तत्संसारगहनं वनमाहुर्मनीषिणः}


\twolineshloka
{सोयं लोकसमावर्तो मर्त्यानां भरतर्षभ}
{चराणां स्थावराणां च न गृध्येत्तत्र पाण्डितः}


\twolineshloka
{शारीरा मानसाश्चैव मर्त्यानां व्याधयश्च ये}
{प्रत्यक्षाश्च परोक्षाश्च ते व्यालाः कथिता बुधैः}


\twolineshloka
{क्लिश्यमानाश्च तैर्नित्यं मार्यमाणाश्च भारत}
{स्वकर्मभिर्महाव्यालैर्नोद्विजन्त्यल्पबुद्धयः}


\twolineshloka
{अथापि तैर्विमुच्येत व्याधिभिः पुरुषो नृप}
{आवृणोत्येव तं पश्चाज्जरा रूपविनाशिनी}


\twolineshloka
{शब्दरूपरसस्पर्शगन्धैश्च विविधैरपि}
{मज्जमानं महापङ्के निरालम्बे समन्ततः}


\twolineshloka
{संवत्सरर्तवो मासाः पक्षाहोरात्रसन्धयः}
{क्रमेणास्य प्रवृज्जन्ति रूपमायुस्तथैव च}


\twolineshloka
{एते कालस्य विधयो नैताञ्जानान्ति दुर्बुधाः}
{धात्राऽभिलिखितान्याहुः सर्वभूतानि कर्मणा}


\twolineshloka
{रथः शरीरं भूतानां सत्वमाहुस्तु सारथिम्}
{इन्द्रियाणि हयानाहुः कर्मबुद्विस्तु रश्मयः}


\twolineshloka
{तेषां हयानां यो वेगं धावतामनु धावति}
{सतु संसारचक्रेऽस्मिंश्चक्रवत्परिवर्तते}


\twolineshloka
{यस्तान्संयमते बुद्ध्या संयतो न निवर्तते}
{यस्तु संसारचक्रेऽस्मिंश्चक्रवत्परिवर्तते}


\twolineshloka
{[भ्रममाणा न मुह्यन्ति संसारे न भ्रमन्ति ते}
{संसारे भ्रमतां राजन्दुःखमेतद्धि जायते}


\twolineshloka
{तस्पादस्य निवृत्त्यर्थं यत्नमेवाचरेद्बुधः}
{उपेक्षा नात्र कर्तव्या शतशाखः प्रवर्धते}


\twolineshloka
{यतेन्द्रियो नरो राजन्क्रोधलोभनिराकृतः}
{सन्तुष्टः सत्यवादी यः स शान्तिमधिगच्छति ॥]}


\twolineshloka
{याम्यमाहू रथं ह्येनं मुह्यन्ते येन दुर्बुधाः}
{स चैतत्प्राप्नुयाद्राजन्यत्त्वं प्राप्तो नराधिप}


\threelineshloka
{अनुतर्षुलमेवैतद्दुःखं भवति मारिष}
{राज्यनाशः सुहृन्नाशः सुतनाशश्च भारत}
{साधुः परमदुःखानां दुःखभैषज्यमारभेत्}


\twolineshloka
{ज्ञानौषधमवाप्येह दूरपारं महौषधम्}
{छिन्द्याद्दुःखमहाव्याधिं नरः संयतमानसः}


\twolineshloka
{न विक्रमो न चाप्यर्थो न मित्रं न सुहृज्जनः}
{तस्मान्मोचयते दुःखाद्यथात्मा स्थिरनिश्चयः}


\twolineshloka
{तस्मान्मैत्रं समास्थाय शीलमापदि भारत}
{दमस्त्यागोऽप्रमादश्च ते त्रयो ब्रह्मणो हयाः}


\twolineshloka
{शीलरश्मिसमायुक्ते स्थितो यो मानसे रथे}
{त्यक्त्वा मृत्युभयं राजन्ब्रह्मलोकं स गच्छति}


\twolineshloka
{अभयं सर्वभूतेभ्यो यो ददाति महीपते}
{स गच्छति परं स्थानं विष्णोः पदमनामयम्}


\twolineshloka
{न तत्क्रतुसहस्रेण नोपवासैश्च नित्यशः}
{अभयस्य च दानेन यत्फलं प्राप्नुयान्नरः}


\twolineshloka
{न ह्यात्मनः प्रियतरं किञ्चिद्भूतेषु निश्चितम्}
{अनिष्टं सर्वभूतानां मरणं नाम भारत}


% Check verse!
तस्मात्सर्वेषु भूतेषु दया कार्या विपश्चिता
\threelineshloka
{नानायोगसमायुक्ता बुद्धिजालेन संवृताः}
{असूक्ष्मदृष्टयो मन्दा भ्राम्यन्ते तत्रतत्र ह}
{सुसूक्ष्मदृष्टयो राजन्व्रजन्ति ब्रह्मसाम्यताम्}


\twolineshloka
{एवं ज्ञात्वा महाप्राज्ञ स तेषामौर्ध्वदैहिकम्}
{कर्तुमर्हति तेनैव फल प्राप्स्यति वै भवान्}


\chapter{अध्यायः ८}
\twolineshloka
{वैशम्पायन उवाच}
{}


\twolineshloka
{विदुरस्य तु तद्वाक्यं निशम्य कुरुसत्तमः}
{पुत्रशोकाभिसन्तप्तः पपात भुवि मूर्छितः}


\twolineshloka
{तं तथा पतितं भूमौ निःसंज्ञं प्रेक्ष्य बान्धवाः}
{कृष्णद्वैपायनश्चैव क्षत्ता च विदुरस्तथा}


\twolineshloka
{सञ्जयः सुहृदश्चान्ये द्वास्था ये चास्य सम्मताः}
{जलेन सुखशीतेन तालवृन्तैश्च भारत}


\twolineshloka
{पस्पर्शुश्च करैर्गात्रं वीजमानाश्च यत्नतः}
{अन्वासत चिरं कालं धृतराष्ट्रं तथाविधम्}


\twolineshloka
{अथ दीर्घस्य कालस्य लब्धस़ञ्ज्ञो महीपतिः}
{विललाप चिरं कालं पुत्राधिभिरभिप्लुतः}


\twolineshloka
{धिगस्तु खलु मानुष्यं मानुष्ये च परिग्रहम्}
{यतोमूलानि दुःखानि सम्भवन्ति पुनःपुनः}


\twolineshloka
{मित्रनाशेऽर्थनाशे च ज्ञातिसम्बन्धिनामपि}
{प्राप्यते सुमहद्दुःखं विषाग्निप्रतिमं विभो}


\twolineshloka
{येन दह्यन्ति गात्राणि येन प्रज्ञा विनश्यति}
{येनाभिभूतः पुरुषो मरणं प्रतिपद्यते}


\twolineshloka
{तदिदं मरणं प्राप्तं मया भाग्यविपर्ययात्}
{[तस्यान्तं नाधिगच्छामि ऋते प्राणविमोक्षणात्]}


\twolineshloka
{तच्चैवाहं करिष्यामि अद्यैव द्विजसत्तम}
{इत्युक्त्वा तु महात्मानं पितरं ब्रह्मवित्तमम्}


\twolineshloka
{धृतराष्ट्रोऽभवन्मूढः स शोकं परमं गतः}
{अभूच्च तूष्णीं राजाऽसौ ध्यायमानो महीपतिः}


\threelineshloka
{तस्य तद्वचनं श्रुत्वा कृष्णद्वैपायनः प्रभुः}
{पुत्रशोकाभिसन्तप्तं पुत्रं वचनमब्रवीत् ॥व्यास उवाच}
{}


\twolineshloka
{धृतराष्ट्र महाबाहो शृणु वक्ष्यामि पुत्रक}
{श्रुतवानसि मेधावी धर्मार्थकुशलः प्रभो}


\twolineshloka
{न तेऽस्त्यविदितं किञ्चिद्वेदितव्यं परन्तप}
{अनित्यतां हि भूतानां विजानासि न संशयः}


\twolineshloka
{अध्रुवे जीवलोके च स्थाने वा शाश्वते सति}
{जीविते मरणान्ते च कस्माच्छोचसि पुत्रक}


\twolineshloka
{प्रत्यक्षं तव राजेन्द्र वैरस्यास्य समुद्भवः}
{पुत्रं ते कारणं कृत्वा कालयोगेन निर्मितः}


\twolineshloka
{अवश्यं भवितव्ये च कुरूणां सङ्क्षये नृप}
{कस्माच्छोचसि ताञ्शूरान्गतान्परमिकां गतिम्}


\twolineshloka
{जानता च महाबाहो विदुरेण महात्मना}
{यतितं सर्वयत्नेन शमं प्रति जनेश्वर}


\twolineshloka
{न च दैवकृतो मार्गः शक्यो भूतेन केनचित्}
{घटताऽपि चिरं कालं नियन्तुमिति मे मतिः}


\twolineshloka
{देवतानां हि यत्कार्यं मया प्रत्यक्षतः श्रुतम्}
{तत्तेऽहं सम्प्रवक्ष्यामि यथा स्थैर्यं भवेत्तव}


\twolineshloka
{पुराऽहं परितो यातः सभामैन्द्रीं जितक्लमः}
{अपश्यं तत्र च सदा समवेतान्दिवौकसः}


\twolineshloka
{नारदप्रमुखांश्चापि सर्वान्देवर्षिसत्तमान्}
{तत्र चापि मया दृष्टा पृथिवी पृथिवीपते}


\twolineshloka
{कार्यार्थमुपसम्पन्ना देवतानां समीपतः}
{उपगम्य तदा धात्री देवानाह समागतान्}


\twolineshloka
{यत्कार्यं मम युष्माभिर्ब्रह्मणः सदने तदा}
{प्रतिज्ञातं महाभागास्तच्छीघ्रं संविधीयताम्}


\twolineshloka
{तस्यास्तद्वचनं श्रुत्वा विष्णुर्लोकनमस्कृतः}
{उवाच वाक्यं प्रहसन्प्रभुस्तां देवसंसदि}


\twolineshloka
{धृतराष्ट्रस्य पुत्राणां यस्तु ज्येष्ठः शतस्य वै}
{दुर्योधन इति ख्यातः स ते कार्यं करिष्यति}


\twolineshloka
{तं च प्राप्य महीपालं कृतकृत्या भविष्यसि}
{तस्यार्थे पृथिवीपालाः कुरुक्षेत्रं समागताः}


\twolineshloka
{अन्योन्यं घातयिष्यन्ति दृढैः शस्त्रैः प्रहारिणः}
{ततस्ते भविता देवि भारस्य युधि नाशनम्}


\twolineshloka
{गच्छ शीघ्रं स्वकं स्थानं लोकान्धारय शोभने}
{स एष ते सुतो राजँल्लोकसंहारकारणात्}


\twolineshloka
{कलेरंशः समुत्पन्नो गान्धार्या जठरे नृप}
{अमर्षी बलवाञ्शूरः क्रोधनो दुष्प्रसाधनः}


\twolineshloka
{दैवयोगात्समुत्पन्ना भ्रातरस्तस्य तादृशाः}
{शकुनिर्मातुलश्चैव कर्णश्च परमः सखा}


\twolineshloka
{समुत्पन्ना विनाशार्थं पृथिव्यां सहिता नृपाः}
{[यादृशो जायते राजा तादृशोऽस्य जनो भवेत्}


\twolineshloka
{अधर्मो धर्मतां याति स्वामी चेद्धार्मिको भवेत्}
{स्वामिनो गुणदोषाभ्यां भृत्याः स्युर्नात्र संशयः}


\twolineshloka
{दुष्टं राजानमासाद्य गतास्ते तनया नृप}
{]एतमर्थं महाबाहो नारदो वेद तत्त्ववित्}


\twolineshloka
{आत्मापराधात्पुत्रास्ते विनष्टाः पृथिवीपते}
{मा ताञ्शोचस्व राजेन्द्र न हि शोकेऽस्ति कारणम्}


\twolineshloka
{न हि ते पाण्डवाः स्वल्पमपराध्यन्ति भारत}
{पुत्रास्तव दुरात्मानो यैरियं घातिता मही}


\twolineshloka
{नारदेन च भद्रं ते पूर्वमेव न संशयः}
{युधिष्ठिरस्य समितौ राजसूये निवेदितम्}


\twolineshloka
{पाण्डवाः कौरवाः सर्वे समासाद्य परस्परम्}
{नभविष्यन्ति कौन्तेय यत्ते कृत्यं तदाचर}


\twolineshloka
{नारदस्य वचः श्रुत्वा तथाऽकुर्वत पाण्डवाः}
{एतत्ते सर्वमाख्यातं देवगुह्यं सनातनम्}


\twolineshloka
{कथं ते शोकनाशः स्यात्प्राणेषु च दया प्रभो}
{स्नेहश्च पाण्डुपुत्रेषु ज्ञात्वा दैवकृतं विधिम्}


\twolineshloka
{एष चार्थो महाबाहो पूर्वमेव मया श्रुतः}
{कथितो धर्मराजस्य राजसूये कुरूत्तम}


\twolineshloka
{यतितं धर्मपुत्रेण मया गुह्ये निवेदिते}
{अविग्रहे कौरवाणां दैवं तु बलवत्तरम्}


\twolineshloka
{अनतिक्रमणीयो हि विधी राजन्कथञ्चन}
{कृतान्तस्य तु भूतेन स्थावरेण चरेण च}


\twolineshloka
{भवान्धर्मपरो यत्र बुद्धिश्रेष्ठश्च भारत}
{मुह्यते प्राणिनां ज्ञात्वा गतिं चागतिमेव च}


\twolineshloka
{त्वां तु शोकेन सन्तप्तं मुह्यमानं मुहुर्मुहुः}
{ज्ञात्वा युधिष्ठिरो राजा प्राणानपि परित्यजेत्}


\twolineshloka
{कृपालुर्नित्यशो वीरस्तिर्यग्योनिगतेष्वपि}
{स कथं त्वयि राजेन्द्र कृपां नैव करिष्यति}


\twolineshloka
{मम चैव नियोगेन विधेश्चाप्यनिवर्तनाम्}
{पाण्डवानां च कारुण्यात्प्राणान्धारय भारत}


\twolineshloka
{एवं ते वर्तमानस्य लोके कीर्तिर्भविष्यति}
{धर्मार्थः सुमहांस्तात तप्तं स्याच्च तपश्चिरात्}


\threelineshloka
{पुत्रशोकं समुत्पन्नं हुताशं ज्वलितं यथा}
{प्रज्ञाम्भसा महाराज निर्वापय सदा सता ॥वैशम्पायन उवाच}
{}


\twolineshloka
{तच्छ्रुत्वा तस्य वचनं व्यासस्यामिततेजसः}
{मुहूर्तं समनुध्यायन्धृतराष्ट्रोऽभ्यभाषत}


\twolineshloka
{महता शोकजालेन प्रणुन्नोऽस्मि द्विजोत्तम}
{नात्मानमवबुध्यामि मुह्यमानो मुहुर्मुहुः}


\twolineshloka
{इदं तु वचनं श्रुत्वा तव देवनियोगजम्}
{धारयिष्याम्यहं प्राणान्यतिष्ये च न शोचितुं}


\twolineshloka
{एतच्छ्रुत्वा तु वचनं व्यासः सत्यवतीसुतः}
{धृतराष्ट्रस्य राजेन्द्र तत्रैवान्तरधीयत}


\twolineshloka
{[जनमेजय उवाच}
{}


\twolineshloka
{गते भगवति व्यासे धृतराष्ट्रो महीपतिः}
{किमचेष्टत विप्रर्षे तन्मे व्याख्यातुमर्हसि}


\twolineshloka
{तथैव कौरवो राजा धर्मपुत्रो महामनाः}
{कृपप्रभृतयश्चैव किमकुर्वत ते त्रयः}


\threelineshloka
{अश्वत्थाम्नः श्रुतं कर्म शापश्चान्योन्यकारितः}
{वृत्तान्तमुत्तरं ब्रूहि यदभाषत सञ्जयः ॥वैशम्पायन उवाच}
{}


\threelineshloka
{हते दुर्योधने चैव हते सैन्ये च सर्वशः}
{सञ्जयो विगतप्रज्ञो धृतराष्ट्रमुपस्थितः ॥11-9-4 विगता प्रज्ञा व्यासदत्तं दिव्यज्ञानं यस्य स विगतप्रज्ञः ॥सञ्जय उवाच}
{}


\twolineshloka
{आगम्य नानादेशेभ्यो नानाजनपदेश्वराः}
{पितृलोकं गता राजन्सर्वे तव सुतैः सह}


\twolineshloka
{याच्यमानेन सततं तव पुत्रेण भारत}
{घातिता पृथिवी सर्वा वैरस्यान्तं विधित्सता}


\threelineshloka
{पुत्राणामथ पौत्राणां पितॄणां च महीपते}
{आनुपूर्व्येण सर्वेषां प्रेतकार्याणि कारय ॥वैशम्पायन उवाच}
{}


\twolineshloka
{तच्छ्रुत्वा वचनं घोरं सञ्जयस्य महीपतिः}
{गतासुरिव निश्चेष्टो न्यपतत्पृथिवीतले}


\twolineshloka
{तं शयानमुपागम्य पृथिव्यां पृथिवीपतिम्}
{विदुरः सर्वधर्मज्ञ इदं वचनमब्रवीत्}


\twolineshloka
{उत्तिष्ठ राजन्किं शेषे मा शुचो भरतर्षभ}
{एषा वै सर्वसत्वानां लोकेश्वर परा गतिः}


\twolineshloka
{अभावादीनि भूतानि भावमध्यानि भारत}
{अभावनिधनान्येव तत्र का परिदेवना}


\twolineshloka
{न शोचन्मृतमन्वेति न शोचन्म्रियते नरः}
{एवं सांसिद्धिके लोके किमर्थमनुशोचसि}


\twolineshloka
{अयुध्यमानो म्रियते युध्यमानस्तु जीवति}
{कालं प्राप्य महाराज न कश्चिदतिवर्तते}


\twolineshloka
{कालः कर्षति भूतानि सर्वाणि विविधानि च}
{न कालस्य प्रियः कश्चिन्न द्वेष्यः कुरुसत्तम}


\twolineshloka
{यथा वायुस्तृणाग्राणि संवर्तपति सर्वतः}
{तथा कालवशं यान्ति भूतानि भरतर्षभ}


\twolineshloka
{एकसार्थप्रयातानां सर्वेषां तत्र गामिनाम्}
{यस्य कालः प्रयात्यग्रे तत्र का परिदेवना}


\twolineshloka
{यांश्चापि निहतान्युद्धे राजंस्त्वमनुशोचसि}
{न शोच्या हि माहत्मानः सर्वे ते त्रिदिवं गताः}


\twolineshloka
{न यज्ञैर्दक्षिणावद्भिर्न तपोभिर्न विद्यया}
{तथा स्वर्गमुपायान्ति यथा शूरास्तनुत्यजः}


\twolineshloka
{सर्वे वेदविदः शूराः सर्वे सुचरितव्रताः}
{सर्वे चाभिमुखाः क्षीणास्तत्र का परिदेवना}


\twolineshloka
{शरीराग्निषु शूराणां जुहुवुस्ते शराहुतीः}
{हूयमानाञ्शरांश्चैव सेहुरुत्तमपूरषाः}


\twolineshloka
{एवं राजंस्तवाचक्षे स्वर्ग्यं पन्थानमुत्तमम्}
{न युद्धादधिकं किञ्चित्क्षत्रियस्येह विद्यते}


\twolineshloka
{क्षत्रियास्ते महात्मानः शूराः समितिशोभनाः}
{आशिषं परमां प्राप्ता न शोच्याः सर्व एव हि}


\twolineshloka
{आत्मनाऽऽत्मानमाश्वास्य मा शुचः पुरुषर्षभ}
{नाद्य शोकाभिभूतस्त्वं कार्यमुत्स्नष्टुमर्हसि}


\twolineshloka
{वैशम्पायन उवाच}
{}


\twolineshloka
{विदुरस्य तु तद्वाक्यं श्रुत्वा तु पुरुषर्षभः}
{युज्यतां यानमित्युक्त्वा पुनर्वचनमव्रवीत्}


\threelineshloka
{धृतराष्ट्र उवाच}
{शीघ्रमानय गान्धारीं सर्वाश्च भरतस्त्रियः}
{वधूं कुन्तीमुपादाय याश्चान्यास्तत्र योषितः}


\chapter{अध्यायः ९}
\twolineshloka
{`जनमेजय उवाच}
{}


\threelineshloka
{गते व्यासे तु धर्मात्मा धृतराष्ट्रो महीपतिः}
{किमचेष्टत विप्रर्षे तन्मे व्याख्यातुमर्हसि ॥वैशम्पायन उवाच}
{}


\twolineshloka
{एतच्छ्रुत्वा नरश्रेष्ठश्चिरं ध्यात्वा त्वचेतनः}
{सञ्जयं योजयेत्युक्त्वा विदुरं प्रत्यभाषत}


\twolineshloka
{क्षिप्रमानय गान्धारीं सर्वाश्च भरतस्त्रियः}
{कुन्तीं चैव तथा क्षत्तः समानय ममाऽन्तिकम्'}


\twolineshloka
{एवमुक्त्वा स धर्मात्मा विदुरं धर्मवित्तमम्}
{शोकविप्रहतज्ञानो यानमेवान्वरोहत}


\twolineshloka
{गान्धारी पुत्रशोकार्ता भर्तुर्वचननोदिता}
{सह कुन्त्या यतो राजा सह स्त्रीभिरुपाद्रवत्}


\twolineshloka
{ताः समासाद्य राजानं भृशं शोकसमन्विताः}
{आमन्त्र्यान्योन्यमायस्ता भृशमुच्चुक्रुशुस्ततः}


\twolineshloka
{ताः समाश्वासयत्क्षत्ता ताभ्यश्चार्ततरः स्वयम्}
{अश्रुकण्ठीः समारोप्य ततोऽसौ निर्ययौ पुरात्}


\twolineshloka
{ततः प्रणादः सञ्जज्ञे सर्वेषु कुरुवेश्मसु}
{आकुमारं पुरं सर्वमभवच्छोककर्शितम्}


\twolineshloka
{अदृष्टपूर्वा या नार्यः पुरा देवगणैरपि}
{पृथग्जनेन दृश्यन्ते तास्तदा निहतेश्वराः}


\twolineshloka
{प्रकीर्य केशान्सुशुभान्भूषणान्यवमुच्य च}
{एकवस्त्रधरा नार्यः परिपेतुरनाथवत्}


\twolineshloka
{श्वतेपर्वतरूपेभ्यो गृहेभ्यस्ता निराक्रमन्}
{गुहाभ्य इव शैलानां पृषत्यो हतयूथपाः}


\twolineshloka
{तान्युदीर्णानि नारीणां तदा वृन्दान्यनेकशः}
{शोकार्तान्यद्रवन्राजन्किशोरीणामिवाङ्गणे}


\twolineshloka
{प्रगृह्य बाहून्क्रोशन्त्यः पुत्रान्भ्रातॄन्पितॄनपि}
{दर्शयन्ति हता हि स्म युगान्ते लोकसङ्क्षयम्}


\twolineshloka
{विलपन्त्यो रुदन्त्यश्च धावमानास्ततस्ततः}
{शोकेनोपहतज्ञाताः कर्तव्यं न प्रजज्ञिरे}


\twolineshloka
{व्रीडां जग्मुः पुरा याः स्म सखीनामपि योषितः}
{ता एकवस्त्रा निर्लज्जाः श्वश्रूणां पुरतोऽभवन्}


\twolineshloka
{परस्परं सुसूक्ष्मेषु शोकेष्वाश्वासयन्ति याः}
{ताः शोकविह्वला राजन्नवैक्षन्त परस्परम्}


\twolineshloka
{ताभिः परिवृतो राजा रुदतीभिः सहस्रशः}
{निर्ययौ नगराद्दीनस्तूर्णमायोधनं प्रति}


\twolineshloka
{शिल्पिनो वणिजो वैश्याः सर्वे कर्मोपजीविनः}
{ते पार्थिवं पुरस्कृत्य निर्ययुर्नगराद्बहिः}


\twolineshloka
{तेषां विक्रोशमानानामार्तानां कुरुसङ्क्षये}
{प्रादुरासीन्महाञ्शब्दो व्यथयन्भुवनान्युत}


\twolineshloka
{युगान्तकाले सम्प्राप्ते भूतानां दह्यतामिव}
{अभावस्योदयः प्राप्त इति भूतानि मेनिरे}


\twolineshloka
{भृशमुद्विग्नमनसस्ते पौराः कुरुसङ्क्षये}
{प्राक्रोशन्त महाराज स्वनुरक्तास्तदा भृशम्}


\chapter{अध्यायः १०}
\twolineshloka
{वैशम्पायन उवाच}
{}


\twolineshloka
{क्रोशमात्रं ततो गत्वा ददृशुस्तान्महारथान्}
{शारद्वतं कृपं द्रौणिं कृतवर्माणमेव च}


\twolineshloka
{ते तु दृष्ट्वैव राजानं प्रज्ञाचक्षुषमीश्वरम्}
{अश्रुकण्ठा विनिःश्वस्य रुदन्तमिदमब्रुवन्}


\twolineshloka
{पुत्रस्तव महाराज कृत्वा कर्म सुदुष्करम्}
{गतः सानुचरो राजञ्शक्रलोकं महीपते}


\twolineshloka
{दुर्योधनबलान्मुक्ता वयमेव त्रयो रथाः}
{सर्वमन्यत्परिक्षीणं सैन्यं ते भरतर्षभ}


\twolineshloka
{इत्येवमुक्त्वा राजानं कृपः शारद्वतस्ततः}
{गान्धारीं पुत्रशोकार्तामिदं वचनमब्रवीत्}


\twolineshloka
{अभीता युध्यमानास्ते घ्नन्तः शत्रुगणान्बहून्}
{वीरकर्माणि कुर्वाणाः पुत्रास्ते निधनं गताः}


\twolineshloka
{ध्रवं सम्प्राप्य लोकांस्ते निर्मलाञ्शस्त्रनिर्जितान्}
{भास्वरं देहमास्थाय विहरन्त्यमरा इव}


\threelineshloka
{सर्वे ह्यभिमुखा राज्ञि युध्यमाना हता युधि}
{न हि कश्चिद्धि शूराणां युध्यमानः पराङ्मुखः}
{शस्त्रेण निधनं प्राप्तो न च कश्चित्कृताञ्जलिः}


\twolineshloka
{एतां तां क्षत्रियस्याहुः पुराणाः परमां गतिम्}
{शस्त्रेण निधनं सङ्ख्ये तान्न शोचितुमर्हसि}


\twolineshloka
{न चापि शत्रवस्तेषां मुच्यन्ते राज्ञि पाण्डवाः}
{शृणु यत्कृतमस्माभिरश्वत्थामपुरोगमैः}


\twolineshloka
{अधर्मेण हतं श्रुत्वा भीमसेनेन ते सुतम्}
{सुप्तं शिबिरमासाद्य पाण्डूनां कदनं कृतम्}


\twolineshloka
{पाञ्चाला निहताः सर्वे धृष्टद्युम्नपुरोगमाः}
{द्रुपदस्यात्मजाश्चैव द्रौपदेयाश्च पातिताः}


\twolineshloka
{तथा विशसनं कृत्वा पुत्रशत्रुगणस्य ते}
{प्राद्रवाम रणे स्थातुं न हि शक्ष्यामहे त्रयः}


\twolineshloka
{ते हि शूरा महेष्वासाः क्षिप्रमेष्यन्ति पाण्डवाः}
{अमर्षवशमापन्ना वैरं प्रतिजिहीर्षवः}


\twolineshloka
{ते हतानात्मजाञ्श्रुत्वा प्रमत्ताः पुरुषर्षभाः}
{अन्विष्यन्तः पदं शूराः क्षिप्रमेष्यन्ति पाण्डवाः}


\twolineshloka
{तेषां तु किल्बिषं कृत्वा संस्थातुं नोत्सहामहे}
{अनुजानीहि नो राज्ञि मा च शोके मनः कृथाः}


\twolineshloka
{राजंस्त्वमनुजानीहि धैर्यमातिष्ठ चोत्तमम्}
{दिष्टान्तं पश्य चापि त्वं क्षात्रं धर्मं च केवलम्}


\twolineshloka
{इत्येवमुक्त्वा राजानं कृत्वा चाभिप्रदक्षिणम्}
{कृपश्च कृतवर्मा च द्रोमपुत्रश्च भारत}


\twolineshloka
{अवेक्षमाणा राजानं धृतराष्ट्रं मनीषिणम्}
{गङ्गामनु महाराज तूर्णमश्वानचोदयन्}


\twolineshloka
{अपक्रम्य तु ते राजन्सर्व एव महारथाः}
{आमन्त्र्यान्योन्यमुद्विग्नास्त्रिधा ते प्रययुस्तदा}


\twolineshloka
{जगाम हास्तिनपुरं कृपः शारद्वतस्तदा}
{स्वमेव राष्ट्रं हार्दिक्यो द्रौणिर्व्यासाश्रमं ययौ}


\twolineshloka
{एवं ते प्रययुर्वीरा वीक्षमाणाः परस्परम्}
{भयार्ताः पाण्डुपुत्राणामागस्कृत्वा महात्मनाम्}


\twolineshloka
{समेत्य वीरा राजानं तदा त्वनुदिते रषौ}
{विप्रजग्मुर्महात्मानो यथेष्टकमरिन्दमाः}


\twolineshloka
{[समासाद्याथ वै द्रौणिं पाण्डुपुत्रा महारथाः}
{व्यजयंस्ते रणे राजन्विक्रम्य तदनन्तरम् ॥]}


\chapter{अध्यायः ११}
\twolineshloka
{वैशम्पायन उवाच}
{}


\twolineshloka
{हतेषु सर्वेसैन्येषु धर्मराजो युधिष्ठिरः}
{शुश्रुवे पितरं वृद्धं निर्यान्तं गजसाह्वयात्}


\twolineshloka
{सोऽभ्ययात्पुत्रशोकार्तं पुत्रशोकपरिप्लुतः}
{शोचमानो महाराज भ्रातृभिः सहितस्तदा}


\twolineshloka
{अन्वीयमानो वीरेण दाशार्हेण महात्मना}
{युयुधानेन च तथा तथैव च युयुत्सुना}


\twolineshloka
{तमन्वगात्सुदुःखार्ता द्रौपदीशोककर्शिता}
{सह पाञ्चालयोषिद्भिर्यास्तत्रासन्समागताः}


\twolineshloka
{सङ्ग्राममनु बृन्दानि स्त्रीणां भरतसत्तम्}
{कुररीणामिवार्तानां क्रोशन्तीनां ददर्श ह}


\twolineshloka
{ताभिः परिवृतो राजा क्रोशन्तीभिः सहस्रशः}
{ऊर्ध्वबाहुभिरार्ताभी रुदतीभिः प्रियाप्रियैः}


\twolineshloka
{क नु धर्मज्ञता राज्ञः क्व नु साद्यानृशंसता}
{यच्चावधीत्पितॄन्भ्रातॄन्गुरुपुत्रान्सखीनपि}


\twolineshloka
{घातयित्वा कथं द्रोणं भीष्मं चापि पितामहम्}
{मनस्तेऽभून्महाबाहो हत्वा चापि जयद्रथम्}


\threelineshloka
{किं नु राज्येन ते कार्यं पितॄन्भ्रातॄनपश्यतः}
{अभिमन्युं च दुर्धर्षं द्रौपदेयांश्च भारत ॥वैशम्पायन उवाच}
{}


\twolineshloka
{अतीत्य ता महाबाहुः क्रोशन्तीः कुररीरिव}
{ववन्दे पितरं ज्येष्ठं धर्मराजो युधिष्ठिरः}


\twolineshloka
{ततोऽभिवाद्य पितरं क्रमेणामित्रकर्शनाः}
{न्यवेदयन्त नामानि पाण्डवास्तेऽपि सर्वशः}


\twolineshloka
{तमात्मजान्तकरणं पिता पुत्रवधार्दितः}
{अप्रीयमाणः शोकार्तः पाण्डवं परिषस्वजे}


\twolineshloka
{धर्मराजं परिष्वज्य सान्त्वयित्वा च भारत}
{दुष्टात्मा भीममन्वैच्छद्दिधक्षुरिव पावकः}


\twolineshloka
{स कोपपावकस्तस्य शोकवायुसमीरितः}
{भीमसेनमयं दावं दिधक्षुरिव दृश्यते}


\twolineshloka
{तस्य संकल्पमाज्ञाय भीमं प्रत्यशुभं हरिः}
{भीममाक्षिप्य पाणिभ्यां प्रददौ भीममायसम्}


\twolineshloka
{प्रागेव तु महाबुद्धिर्बुद्ध्वा तस्येङ्गितं हरिः}
{संविधानं महाप्राज्ञस्तत्र चक्रे जनार्दनः}


\twolineshloka
{उपगुह्यैव पाणिभ्यां भीमसेनमयस्मयम्}
{बभञ्ज बलवान्राजा मन्यमानो वृकोदरम्}


\twolineshloka
{नागायुतसमप्राणः स राजा भीममायसम्}
{भङ्क्तत्वा विमथितोरस्कः सुस्राव रुधिरं मुखात्}


\twolineshloka
{ततः पपात मेदिन्यां तथैव रुधिरोक्षितः}
{प्रपुष्पिताग्रशिखरः पारिजात इव द्रुमः}


\twolineshloka
{प्रत्यगृह्णाच्च तं विद्वान्सूतो गावल्गणिस्तदा}
{मैवमित्यब्रवीच्चैनं शमयन्सान्त्वयन्निव}


\twolineshloka
{हतो भीम इति ज्ञात्वा गतमन्युर्महामनाः}
{हाहाभीमेति चुक्रोश नृपः शोकसमन्वितः}


\twolineshloka
{तं विदित्वा गतक्रोधं भीमसेनवधार्दितम्}
{वासुदेवो वरः पुंसामिदं वचनमब्रवीत्}


\twolineshloka
{मा शुचो धृतराष्ट्र त्वं नैव भीमस्त्वया हतः}
{आयसी प्रतिमा ह्येषा त्वया निष्पातिता विभो}


\twolineshloka
{त्वां क्रोधवशमापन्नं विदित्वा भरतर्षभ}
{मयाऽपकृष्टः कौन्तेयो मृत्योर्दंष्ट्रान्तरं गतः}


\twolineshloka
{न हि ते राजशार्दूल बले तुल्योऽस्ति कश्चन}
{कः सहेत महाबाहो बाह्वोर्विग्रहणं नरः}


\twolineshloka
{यथान्तकमनुप्राप्य जीवन्कश्चिन्न मुच्यते}
{एवं बाह्वन्तरं प्राप्य तव जीवेन्न कश्चन}


\twolineshloka
{तस्मात्पुत्रेण या तेऽसौ प्रतिमा कारिताऽऽयसी}
{भीमस्य सेयं कौरव्य तवैवोपहृता मया}


\twolineshloka
{पुत्रशोकाभिसन्तप्तं धर्मादपकृतं मनः}
{तव राजेन्द्र तेन त्वं भीमसेनं जिघांससि}


\twolineshloka
{न त्वेतत्ते क्षमं राजन्हन्यास्त्वं यद्वृकोदरम्}
{न हि पुत्रा महाराज जीवेयुस्ते कथञ्चन}


\twolineshloka
{तस्माद्यत्कृतमस्माभिर्मन्यमानैः शमं प्रति}
{अनुमन्यस्व तत्सर्वं मा च शोके मनः कृथाः}


\chapter{अध्यायः १२}
\twolineshloka
{वैशम्पायन उवाच}
{}


\twolineshloka
{तत एनमुपातिष्ठञ्शौचार्थं परिचारकाः}
{कृतशौचं पुनश्चैनं ओवाच मधुसूदनः}


\twolineshloka
{राजन्नधीता वेदास्ते शास्त्राणि विविधानि च}
{श्रुतानि च पुराणानि राजधर्माश्च केवलाः}


\twolineshloka
{एवं विद्वान्महाप्राज्ञः समर्थः सन्बलाबले}
{आत्मापराधात्कस्मात्त्वं कुरुषे कोपमीदृशम्}


\twolineshloka
{उक्तवांस्त्वां तदैवाहं भीष्मद्रोणौ च भारत}
{विदुरः सञ्जयश्चैव वाक्यं राजन्न तत्कृथाः}


\twolineshloka
{स वार्यमाणो नास्माकमकार्षीर्वचनं तदा}
{पाण्डवानधिकाञ्चानन्बले शौर्ये च कौरव}


\twolineshloka
{राजा हि यः स्थिरप्रज्ञः स्वयं दोषानवेक्षते}
{देशकालविभागं च परं श्रयेः स विन्दति}


\twolineshloka
{उच्यमानस्तु यः श्रेयो गृह्णीते नो हिताहिते}
{आपदः समनुप्राप्य सोऽभ्येति विलयं किल}


\twolineshloka
{ततोऽन्यवृत्तमात्मानं समवेक्षस्व भारत}
{राजंस्त्वं ह्यविधेयात्मा दुर्योधनवशे स्थितः}


\twolineshloka
{आत्मापराधादापन्नस्तत्किं भीमं जिघांससि}
{तस्मात्संयच्छ कोपं त्वं स्वमनुस्मृत्य दुष्कृतम्}


\twolineshloka
{यस्तु तां स्पर्धया क्षुद्रः पाञ्चालीमानयत्समाम्}
{स हतो भीमसेनेन वैरं प्रतिजिहीर्षता}


\threelineshloka
{आत्मनोऽतिक्रमं पश्य पुंत्रस्य च दुरात्मनः}
{यदनागसि पाण्डूनां परित्यागस्त्वया कृतः ॥वैशम्पायन उवाच}
{}


\twolineshloka
{एवमुक्तः स कृष्णेन सर्वं सत्यं जनाधिप}
{उवाच देवकीपुत्रं धृतराष्ट्रो महीपतिः}


\twolineshloka
{एवमेतन्महाबाहो यथा वदसि माधव}
{पुत्रस्नेहस्तु बलवान्धर्मान्मां समचालयत्}


\twolineshloka
{दिष्ट्या तु पुरुषव्याघ्रो बलवान्सत्यक्त्रिमः}
{त्वद्गुप्तो नागमत्कृष्ण भीमो बाह्वन्तरं मम}


\twolineshloka
{इदानीं त्वहमव्यग्रो गतमन्युर्गतज्वरः}
{मध्यमं पाण्डवं वीरं स्पष्टुमिच्छामि माधव}


\twolineshloka
{हतेषु पारथिवेन्द्रेषु पुत्रेषु निहतेषु च}
{पाण्डुपुत्रेषु मे धर्मः प्रीतिश्चाप्यवतिष्ठते}


\twolineshloka
{ततः स भीमं च धनञ्जयं चमाद्याश्च पुत्रौ पुरुषप्रवीरौ}
{पस्पर्श गात्रैः प्ररुदन्सुगात्रा--नाश्वास्य कल्याणमुवाच चैनान्}


\chapter{अध्यायः १३}
\twolineshloka
{वैशम्पायन उवाच}
{}


\twolineshloka
{धृतराष्ट्राभ्यनुज्ञातास्ततस्ते कुरुपुङ्गवाः}
{अभ्ययुर्भ्रातरः सर्वे गान्धारीं सहकेशवाः}


\twolineshloka
{ततो ज्ञात्वा हतामित्रं धर्मराजं युधिष्ठिरम्}
{गान्धारी पुत्रशोकार्ता शप्तुमैच्छदनिन्दिता}


\twolineshloka
{तस्याः पापमभिप्रायं विदित्वा पाण्डवान्प्रति}
{ऋषिर्गन्धवतीपुत्रः प्रागेव समबुध्यत}


\twolineshloka
{स गङ्गायामुपस्पृश्य ब्रह्मर्षिः प्रयतः शुचिः}
{तं देशमुपसंपेदे पाराशर्यो मनोजवः}


\twolineshloka
{दिव्येन चक्षुषा ज्ञात्वा मनसाऽनुद्धतेन च}
{सर्वप्राणभृतां भावं सततं स तु बृध्यति}


\twolineshloka
{स स्नुषामब्रवीत्काले कल्याणानि महातपाः}
{शापकालमवाक्षिप्य शमकालमुदीरयन्}


\twolineshloka
{न कोपः पाण्डवे कार्यो गान्धारि शममाप्नुहि}
{रजो निगृह्यतां चैव शृणु चेदं वचो मम}


\twolineshloka
{पुरोक्ता युद्धकाले त्वं पुत्रेण जयमिच्छता}
{जयमाशास्व मे मातर्युध्यमानस्य शत्रुभिः}


\twolineshloka
{सा तथा याच्यमाना त्वं युद्धकाले जयैषिणा}
{उक्तवत्यसि कल्याणि यतो धर्मस्ततो जयः}


\twolineshloka
{वाचाऽप्यतीते मा क्रोधे मनः कुरु यशस्विनि}
{स्मरामि भाषमाणां हि त्वामहं सत्यवादिनीम्}


\twolineshloka
{विग्रहे तुमुले राज्ञां गत्वा परमसंशयम्}
{जितं पाण्डुसुतैर्युद्धे नूनं धर्मस्ततोऽधिकः}


\twolineshloka
{क्षमाशीला पुरा भूत्वा साऽद्य न क्षमसे कथम्}
{अधर्मे जहि धर्मज्ञे यतो धर्मस्ततो जयः}


\threelineshloka
{सा त्वं धर्मं परिस्मृत्य वाचं चोक्तां मनस्विनी}
{कोपं संयच्छ गान्धारि पाण्डवेषु सुतेषु ते ॥गान्धार्युवाच}
{}


\twolineshloka
{भगवन्नाभ्यसूयामि नैतानिच्छामि नश्यतः}
{पुत्रशोकेन तु बलान्मनो विह्वलतीव मे}


\twolineshloka
{यथैव कुन्त्या कौन्तेया रक्षितव्यास्तथा मया}
{तथैव धृतराष्ट्रेण रक्षितव्या यथा त्वया}


\twolineshloka
{दुःशासनापराधेन शकुनेः सौबलस्य च}
{कर्णदुर्योधनाभ्यां च कृतोऽयं कृरुसंक्षयः}


\twolineshloka
{नापराध्यति बीभत्सुर्न च पार्थो वृकोदरः}
{नकुलः सदेवश्च नैव राजा युधिष्ठिरः}


\twolineshloka
{युध्यमाना हि कौरव्याः कृतमामाः परस्परम्}
{निहताः सहिताश्चान्यैस्तत्र नास्त्यप्रियं मम}


\twolineshloka
{किं तु कर्माकरोद्भीमो वासुदेवस्य पश्यतः}
{दुर्योधनं समाहूय गदायुर्द्ध महामनाः}


\twolineshloka
{शिक्षयाऽभ्यधिकं ज्ञात्वा चरन्तं बहुधा रणे}
{अधो नाभ्याः प्रहृतवांस्तन्मे कोपमवर्धयत्}


\twolineshloka
{कथं नु धर्मं धर्मज्ञाः समुद्दिष्टं महात्मभिः}
{त्यजेयुराहवे शूराः प्राणहेतोः कथञ्चनः}


\chapter{अध्यायः १४}
\twolineshloka
{वैशम्पायन उवाच}
{}


\twolineshloka
{तच्छ्रुत्वा वचनं तस्या भीमसेनोऽथ भीतवत्}
{गान्धारीं प्रत्युवाचेदं वचः सानुनयं तदा}


\twolineshloka
{अधर्मो यदि वा धर्मस्त्रासात्तत्र मया कृतः}
{आत्मानं त्रातुकामेन तन्मे त्वं क्षन्तुमर्हसि}


\twolineshloka
{न हि युद्धेन पुत्रस्ते धर्मेण स महाबलः}
{दुःशक्यः केनचिद्धर्तुमतो विषममाचरम्}


\twolineshloka
{अधर्मेण जितः पूर्वं तेन चापि युधिष्ठिरः}
{निकृताश्च सदैव स्म तो विषममाचरम्}


\twolineshloka
{सैन्यस्यैकोऽवशिष्टोऽयं गदायुद्धे च वीर्यवान्}
{न त्यक्ष्यति हृतं राज्यमिति वै तत्कृतं मया}


\twolineshloka
{एकपत्नीं च पाश्चालीमेकवस्त्रां रजस्वलाम्}
{भवत्या विदितं सर्वमुक्तवान्यत्सुतस्तव}


\twolineshloka
{सुयोधनं त्वसंहृत्य न शक्या भूः ससागरा}
{केवला भोक्तुमस्माभिरतश्चैतत्कृतं मया}


\twolineshloka
{तच्चाप्यप्रियमस्माकं पुत्रस्ते समुपाचरत्}
{द्रौपद्या यत्सभामध्ये सव्यमूरुमदर्शयत्}


\twolineshloka
{तत्रैव वध्यः सोऽस्माकं दुराचारोऽप्ब ते सुतः}
{धर्मराजाज्ञया चैव स्थिताः स्म समये पुरा}


\twolineshloka
{वैरमुद्दीपितं राज्ञि पुत्रेण तव यन्महत्}
{क्लेशिताश्च वने नित्यं तत एतत्कृतं मया}


\threelineshloka
{वैरस्यास्य गताः पारं हत्वा दुर्योधनं रणे}
{राज्यं युधिष्ठिरे प्राप्ते वयं च गतमन्यवः ॥गान्धार्युवाच}
{}


\twolineshloka
{न तस्यैवं वधस्तात यं प्रशंससि मे सुतम्}
{कृतवांश्चापि तत्सर्वं यदिदं भाषसे मयि}


\twolineshloka
{हताश्वे नकुले यत्तु वृषसेनेन भारत}
{अपिबः शोणितं सङ्ख्ये दुःशासनशरीरजम्}


\threelineshloka
{सद्भिर्विगर्हितं घोरमनार्यजनसेवितम्}
{क्रूरं कर्माकृथास्तस्मात्तदयुक्तं वृकोदर ॥भीम उवाच}
{}


\twolineshloka
{हताश्वं नकुलं दृष्ट्वा वृषसेनेन संयुगे}
{शत्रूणां तु प्रहृष्टानां त्रासः सञ्जनितो मया}


\twolineshloka
{`स प्रतिज्ञामकरवं पिबाम्यसृगरेरिति'}
{रुधिरं नातिचक्राम दन्तोष्ठादम्ब मा शुचः}


\twolineshloka
{[अन्यस्यापि न पातव्यं रुधिरं किंपुनः स्वकम्}
{यथैवात्मा तथा भ्राता विशेषो नास्ति कश्चन}


\twolineshloka
{वैवस्तितो हि तद्वेद यथावत्कुलनन्दिनि}
{मा कृथा हृदि तन्मातर्न तत्पीतं मयाऽनधे}


\twolineshloka
{केशपक्षपरामर्शे द्रौपद्या द्यूतकारिते}
{क्रोधाद्यदब्रवं चाहं तच्च मे हृदि वर्तते}


\twolineshloka
{क्षत्रधर्माच्च्युतो राज्ञि भवेयं शाश्वतीः समाः}
{प्रतिज्ञां तामनिस्तीर्य ततस्तत्कृतवानहम्}


\threelineshloka
{अनिगृह्य पुरा पुत्रानस्मस्वनपकारिषु}
{न ममार्हसि कल्याणि दोषेण परिशङ्कितुम् ॥गान्धार्युवाच}
{}


\twolineshloka
{वृद्धस्यास्य शतं पुत्रान्निघ्नंस्त्वमपराजितः}
{कस्मानाशेषयः कञ्चिद्येनाल्पमपराधितम्}


\twolineshloka
{सन्तानमावयोस्तात वृद्धयोर्हृतराज्ययोः}
{नाशेषयः कथं यष्टिमेकां वृद्धयुगस्य वै}


\twolineshloka
{शेषे ह्यवस्थिते तात पुत्राणामल्पकेऽपि च}
{मन्ददुःखं भवेदद्य यदि त्वं धर्ममाचरेः}


\chapter{अध्यायः १५}
\twolineshloka
{वैशम्पायन उवाच}
{}


\twolineshloka
{तमेवमुक्त्वा गान्धारी युधिष्ठिरमपृच्छत}
{क्व स राजेति सक्रोधा पुत्रपौत्रवधार्दिता}


\twolineshloka
{तामभ्यगच्छत्कौन्तेयो वेपमानः कृताञ्जलिः}
{युधिष्ठिरस्तु गान्धारीं मधुरं वाक्यमब्रवीत्}


\twolineshloka
{पुत्रहन्ता नृशंसोऽहं तव देवि युधिष्ठिरः}
{शापार्हः पृथिवीनाशे हेतुभूतः शपस्व माम्}


\twolineshloka
{नहि मे जीवितेनार्थो न राज्येन सुखेन वा}
{तादृशान्सुहृदो हत्वा प्रमूढस्य सुहृद्द्रुहः}


\twolineshloka
{तमेवंवादिनं भीतं सन्निकर्षागतं तदा}
{नोवाच किञ्चिद्गान्धारी निःश्वासपरमा नृप}


\twolineshloka
{तस्मावनतदेहस्य पादयोर्निपतिष्यतः}
{युधिष्ठिरस्य नृपतेर्धर्मज्ञा दीर्घदर्शिनी}


\twolineshloka
{अङ्गुल्यग्राणि ददृशे देवी पट्टान्तरेण सा}
{ततः स कुनखीभूतो दर्शनीयनखो नृपः}


% Check verse!
तं दृष्ट्वा चार्जुनोऽगच्छद्वासुदेवस्य पृष्ठतः
\twolineshloka
{एवं सञ्चेष्टमानांस्तानितश्चेतश्च भारत}
{गान्धारी विगतक्रोधा सांत्वयामास मातृवत्}


\twolineshloka
{ते पाण्डवा अनुज्ञाता मातरं वीतमत्सराः}
{अभ्यगच्छन्त सहिताः पृथां पृथुवलक्षसः}


\twolineshloka
{चिरस्य दृष्ट्वा पुत्रान्सा पुत्राधिभिरभिप्लुता}
{बाष्पमाहारयद्देवी वस्त्रेणावृत्य वै मुखम्}


\twolineshloka
{ततो बाष्पं समुत्सृज्य सह पुत्रैस्तदा पृथा}
{अपश्यदेनाञ्शस्त्रौघैर्बहुधा परिविक्षतान्}


\fourlineindentedshloka
{सा तानेकैकशः पुत्रान्संस्पृशन्ती पुनःपुनः}
{अन्वशोचत दुःखार्ता द्रौपदीं निहतात्मजाम्}
{रुदन्तीमथ पाञ्चालीं ददर्श पतितां भुवि ॥द्रौपद्युवाच}
{}


\threelineshloka
{आर्ये पौत्रास्तु ते सर्वे सौभद्रसहिता गताः}
{न त्वां तेऽद्याभिगच्छन्ति चिरं दृष्ट्वा अनिन्दिते}
{किन्नु राज्येन मे कार्यं विहीनायाः सुतैर्वरैः}


\twolineshloka
{तां समाश्वासयामास पृथा पृथललोचना}
{उत्थाप्याङ्केन सुदतीं रुदतीं शोककर्शिताम्}


\threelineshloka
{तयैव सहिता चापि पुत्रैरनुगता पृथा}
{अभ्यगच्छत गान्धारीमार्तामार्ततरा स्वयम् ॥वैशम्पायन उवाच}
{}


\twolineshloka
{तामुवाचाथ गान्धारी सह कुन्त्या यशस्विनीम्}
{मैवं पुत्रीति शोकार्ता पश्य मामपि दुःखितां}


\twolineshloka
{मन्ये लोकविनाशोऽयं कालपर्यायनोदितः}
{अवश्यभावी सम्प्राप्तः स्वभावाद्रोमहर्षणः}


\twolineshloka
{इदं तत्समनुप्राप्तं विदुरस्य वचो महत्}
{असिद्धानुनये कृष्णे यदुवाच महामतिः}


\twolineshloka
{तस्मिन्नपरिहार्येऽर्थे व्यतीते च विशेषतः}
{मा शुचो न हि शोच्यास्ते सङ्ग्रामे निधनं गताः}


\twolineshloka
{यथैवाहं तथैव त्वं को वामाश्वासयिष्यति}
{ममैव ह्यपराधेन कुलमग्र्यं विनाशितम्}


\chapter{अध्यायः १६}
\twolineshloka
{वैशम्पायन उवाच}
{}


\twolineshloka
{एवमुक्त्वा तु गान्धारी कुरूणामवकर्तनम्}
{अपश्यत्तत्र तिष्ठन्ती सर्वं तत्र यथास्थितम्}


\twolineshloka
{पतिव्रता महाभागा समानव्रतचारिणी}
{उग्रेण तपसा युक्ता सततं सत्यवादिनी}


\twolineshloka
{वरदानेन कृष्णस्य महर्षेः पुण्यकर्मणः}
{दिव्यज्ञानबलोपेता विविधं पर्यदेवयत्}


\twolineshloka
{ददर्श सा बुद्धिमती दूरादपि यथाऽन्तिके}
{रणाजिरं तद्वीराणामद्भुतं रोमहर्षणम्}


\twolineshloka
{अस्थिकेशवसाकीर्णं शोणितौघपरिप्लुतम्}
{शरीरैर्बहुसाहस्रैर्विनिकीर्णं समन्ततः}


\twolineshloka
{गजाश्वरथयोधानामावृतं रुधिराविलैः}
{शरीरैरशिरस्कैश्च विदेहैश्च शिरोगणैः}


\twolineshloka
{गजाश्वनरवीराणां निःसत्वैरभिसंवृतम्}
{सृगालवलगोमायुकङ्ककाकनिषेवितम्}


\twolineshloka
{रक्षसां पुरुषादाना मोदनं कुरराकुलम्}
{अशिवाभिः शिवाभिश्च नादितं गृध्रसेवितम्}


\twolineshloka
{ततो व्यासाभ्यनुज्ञातो धृतराष्ट्रो महीपतिः}
{पाण्डुपुत्राश्च ते सर्वे युधिष्ठिरपुरोगमाः}


\twolineshloka
{वासुदेवं पुरस्कृत्य हतबन्धुं च पार्थिवम्}
{कुरुस्त्रियः समादाय जग्मुरायोधनं प्रति}


\twolineshloka
{समासाद्य कुरुक्षेत्रं ताः स्त्रियो निहतेश्वराः}
{अपश्यन्त हतांस्तत्र पुत्रान्भ्रातॄन्पितॄन्पतीन्}


\twolineshloka
{क्रव्यादैर्भक्ष्यमाणांश्च गोमायुवलवायसैः}
{भूतैः पिशाचै रक्षोभिर्विविधैश्च निशाचरैः}


\twolineshloka
{रुद्राक्रीडनिभं दृष्ट्वा तदा विशसनं स्त्रियः}
{कुरर्य इव शोकार्ता विक्रोशन्त्यो निपेतिरे}


\twolineshloka
{अदृष्टपूर्वं पशन्त्यो दुःखार्ता भरतस्त्रियः}
{शरीरेषु स्खलन्त्यश्च न्यपतंश्च परासुवत्}


\twolineshloka
{श्रान्तानां चाप्यनाथानां क्रन्दन्तीनां भृशं तदा}
{पाञ्चालकुरुयोषाणां कृपणं तदभून्महत्}


\twolineshloka
{दुःखोपहतचित्ताभिः समन्तादनुनादितम्}
{दृष्ट्वाऽऽयोधनमत्युग्रं धर्मज्ञा सुबलात्मजा}


\twolineshloka
{ततः सा पुण्डरीकाक्षमामन्त्र्य पुरुषोत्तमम्}
{कुरूणां वैशमं दृष्ट्वा इदं वचनमब्रवीत्}


\twolineshloka
{पश्यैताः पुण्डरीकाक्ष स्नुषा मे निहतेश्वराः}
{प्रकीर्णकेशाः क्रोशन्तीः कुररीरिव माधव}


\twolineshloka
{अमूस्त्वभिसमागम्य स्मरन्त्यो भरतर्षभान्}
{पृथगेवानुपद्यन्ते पुत्रान्भ्रातॄन्पितॄन्पतीन्}


\twolineshloka
{वीरसूभिर्महाबाहो हतपुत्राभिरावृतम्}
{क्वचिच्च वीरपत्नीभिर्हतवीराभिरावृतम्}


\twolineshloka
{शोभितं पुरुषव्याघ्रैः कर्णभीष्माभिमन्युभिः}
{द्रोणद्रुपदशल्यैश्च ज्वलद्भिरिव पावकैः}


\twolineshloka
{काञ्चनैः कवचैर्निष्कैर्मणिभिश्च महाधनैः}
{अद्भुतैर्हस्तकेयूरैः स्रग्भिश्च समलंकृतम्}


\twolineshloka
{वीरबाहुविसृष्टाभिः शक्तिभिः परिघैरपि}
{खङ्गैश्च विविधैस्तीक्ष्णैः सशरैश्च शरासनैः}


\twolineshloka
{क्रव्यादसङ्घैर्मुदितैस्तिष्ठद्भिः सहितैः क्वचित्}
{क्वचिदाक्रीडमानैश्च शयानैश्चापरैः क्वचित्}


\threelineshloka
{एतदेवंविधं वीर सम्पश्यायोधनं विभो}
{`त्वया तु श्रावितं कर्म पुष्कराक्ष महाद्युते'}
{पश्यमाना हि दह्यामि शोकेनाहं जनार्दन}


\twolineshloka
{पाञ्चालानां कुरूणां च विनाशं मधुसूदन}
{पञ्चानामपि भूतानामहं वधमचिन्तयम्}


\twolineshloka
{तान्सुपर्णाश्च गृध्राश्च कर्षयन्त्यसृगुक्षितान्}
{विगृह्य चरणैर्गृध्रा भक्षयन्ति सहस्रशः}


\twolineshloka
{जयद्रथस्य कर्णस्य तथैव द्रोमभीष्मयोः}
{अभिमन्योर्विनाशं च कश्चिन्तयितुमर्हति}


\twolineshloka
{अवध्यकल्पान्निहतान्गतसत्वानचेतसः}
{गृध्रकङ्कवलश्येनश्वसृगालसमावृतान्}


\twolineshloka
{अमर्षवशमापन्नान्दुर्योधनवशे स्थितान्}
{पश्येमान्पुरुषव्याघ्रान्संशान्तान्पावकानिव}


\twolineshloka
{शयनान्युचिताः सर्वे मृदूनि विमलानि च}
{विपन्नास्तेऽद्य वसुधां विवृतामधिशेरते}


\twolineshloka
{बन्दिभिः सततं काले स्तुवद्भिरभिनन्दिताः}
{शिवानामशिवा घोराः शृण्वन्ति विविधा गिरः}


\twolineshloka
{ये पुरा शेरते वीराः शयनेषु यशस्विनः}
{चन्दनागुरुदिग्धाङ्गास्तेऽद्य पांसुषु शेरते}


\twolineshloka
{तेषामाभरणान्येते गृघ्रगोमायुवायसाः}
{आक्षिपन्ति शिवा घोरा विनदन्त्यः पुनः पुनः}


\twolineshloka
{बाणान्वनिनिशितान्पीतान्निस्त्रिंशान्विमला गदाः}
{युद्धाभिमानिनः सर्वे जीवन्त इव बिभ्रति}


\twolineshloka
{सुरूपवर्णा बहवः क्रव्यादैरवघट्टिताः}
{ऋषभप्रतिरूपाक्षाः शेरते च सहस्रशः}


\twolineshloka
{अपरे पुनरालिङ्ग्य गदाः परिघबाहवः}
{शेरतेऽभिमुखाः शूरा दयिता इव योषितः}


\twolineshloka
{बिभ्रतः कवचान्यन्ये विमलान्यायुधानि च}
{न धर्षयन्ति क्रव्यादा जीवन्तीति जनार्दन}


\twolineshloka
{क्रव्यादैः कृष्यमाणानामपरेषां महात्मनाम्}
{शातकौम्भ्यः स्रजश्चित्रा विप्रकीर्णाः समन्ततः}


\twolineshloka
{एते गोमायवो भीमा निहतानां यशस्विनाम्}
{कण्ठान्तरगतान्हारानाक्षिपन्ति सहस्रशः}


\twolineshloka
{सर्वेष्वपररात्रेषु यानवन्दन्त बन्दिनः}
{स्तुतिभिश्च परार्ध्याभिरुपचारैश्च शिक्षिताः}


\twolineshloka
{तानद्य परिदेवन्ति दुःखार्ता परमाङ्गनाः}
{कृपणा वृष्णिशार्दूल दुःखशोकार्दिता भृशम्}


\twolineshloka
{रक्तोत्पलवनानीव विभान्ति रुचिराणि च}
{मुखानि परमस्त्रीणां परिशुष्काणि केशव}


\twolineshloka
{रुदित्वा विरता ह्येता ध्यायन्त्यः सपरिक्लमाः}
{कुरुस्त्रियोऽभिगच्छन्ति तेन तेनैव दुःखिताः}


\twolineshloka
{एतान्यादित्यवर्णानि तपमीयनिभानि च}
{पश्य रोदनताम्राणि वक्त्राणि कुरुयोषिताम्}


\twolineshloka
{श्यामानां वरवर्णानां गौरीणामेकवाससाम्}
{दुर्योधनवरस्त्रीणां पश्य वृन्दानि केशव}


\twolineshloka
{आसामपरिपूर्णार्थं निशम्य परिदेवितम्}
{इतरेरसङ्क्रन्दान्न विजानन्ति योषितः}


\twolineshloka
{एता दीर्घमिवोच्छ्वस्य विक्रुश्य च विलप्य च}
{विस्पन्दमाना दुःखेन वीरा जहति जीवितम्}


\twolineshloka
{बह्व्यो दृष्ट्व शरीराणि क्रोशन्ति विलपन्ति च}
{पाणिभिश्चापरा घ्नन्ति शिरांसि मृदुपाणयः}


\twolineshloka
{शिरोभिः पतितैर्हस्तैः सर्वाङ्गैः खण्डशः कृतैः}
{इतरेतसम्पृक्तैराकीर्णा भाति मेदिनी}


\twolineshloka
{विशिरस्कानथो कायान्दृष्ट्वा ह्येताननिन्दितान्}
{मुह्यन्त्यनुचिता नार्यो विदेहानि शिरांसि च}


\twolineshloka
{शिरः कायेन सन्धाय प्रेक्षमाणा विचेतसः}
{अपश्यन्त्योऽपरं तत्र नेदमस्येऽति दुःखिताः}


\twolineshloka
{बाहूरुचरणानन्यान्विशिखोन्मथितान्पृथक्}
{सन्दधत्योऽसुखाविष्टा मूर्छन्त्येताः पुनःपुनः}


\twolineshloka
{उत्कृत्तशिरसश्चान्यान्विजग्धान्मृगपक्षिभिः}
{दृष्ट्वा काश्चिन्न जानन्ति भर्तॄन्भरतयोषितः}


\twolineshloka
{पाणिभिश्चापरा घ्नन्ति शिरांसि मधुसूदन}
{प्रेक्ष्य भ्रातॄन्पितॄन्पुत्रान्पतींश्च निहतान्परैः}


\threelineshloka
{बाहुभिश्च सखङ्गैश्च शिरोभिश्च सकुण्डलैः}
{अगम्यकल्पा पृथिवी मांसशोणितकर्दमा}
{बभूव भरतश्रेष्ठ प्राणिभिर्गतजीवितैः}


\twolineshloka
{न दुःखेषूचिताः पूर्वं दुःखं गाहन्त्यनिन्दिताः}
{भ्रातृभिः पतिभिः पुत्रैरुपाकीर्णां वसुन्धराम्}


\twolineshloka
{यूथानीव किशोरीणां सुकेशीनां जनार्दन}
{स्नुषाणां धृतराष्ट्रस्य पश्य वृन्दान्यनेकशः}


\twolineshloka
{इतो दुःखतरं किन्नु केशव प्रतिभाति मे}
{यदिमाः कुर्वते सर्वा रवमुच्चावचं स्त्रियः}


\twolineshloka
{नूनमाचरितं पापं मया पूर्वेषु जन्मसु}
{या पश्यामि हतान्पुत्रान्पौत्रान्भ्रातृंश्च माधव}


\twolineshloka
{एवमार्ता विलपती समाभाष्य जनार्दनम्}
{गान्धारी पुत्रशोकार्ता द्वदर्श निहतं सुतम्}


\chapter{अध्यायः १७}
\twolineshloka
{वैशम्पायन उवाच}
{}


\twolineshloka
{दुर्योधनं हतं दृष्ट्वा गान्धारी शोककर्शिता}
{सहसा न्यपतद्भूमौ छिन्नेव कदली वने}


\twolineshloka
{सा तु लब्ध्वा पुनः संज्ञां विक्रुश्य च विलप्य च}
{दुर्योधनमभिप्रेक्ष्य शयानं रुधिरोक्षितम्}


\twolineshloka
{परिष्वज्याथ गान्धारी कृपणं पर्यदेवयत्}
{हाहापुत्रेति शोकार्ता विललापाकुलेन्द्रिया}


\twolineshloka
{सुगूढजत्रु विपुलं हारनिष्कविभूषितम्}
{वारिणा नेत्रजेनोरः सिञ्चन्ती शोकतापिता}


\twolineshloka
{समीपस्थं हृषीकेशमिद वचनमब्रवीत्}
{उपस्थितेऽस्मिन्सङ्ग्रामे ज्ञातीनां संक्षये विभो}


\twolineshloka
{मामयं प्राह वार्ष्णेय प्राञ्चलिर्नृपसत्तमः}
{अस्मिञ्ज्ञातिसमुद्धर्षे जयमम्बा ब्रवीतु मे}


\twolineshloka
{इत्युक्ते जानती सर्वमहं स्वव्यसनागमम्}
{अब्रुवं पुरुषव्याघ्र यतो धर्मस्ततो जयः}


\twolineshloka
{तथा तु युव्यमानस्त्वं सम्प्रगृह्य सुपुत्रक}
{ध्रुवं शखजिताँल्लोकान्प्राप्स्यस्यमरवत्प्रभो}


\twolineshloka
{इत्येवxxx पूर्वं नैवं शोचामि वै प्रभो}
{धृतराष्ट्रं तु शोचामि कृपणं हतबान्धवम्}


\twolineshloka
{अमर्वणं युधां श्रेष्ठं कृतास्त्रं युद्धदुर्मदम्}
{शयानं वीरशयने पश्य माधव मे सुतम्}


\twolineshloka
{बोऽयं मूर्धावसिक्तानाये याति परन्तपः}
{सोऽवं पांसुषु शेतेऽद्य पञ्च कालस्य पर्ययम्}


\twolineshloka
{दुवं दुर्योधनो वीरो गतिं न सुलभां गतः}
{तथा ह्यभिxxxx शेते शयने वीरसेविते}


\twolineshloka
{यं पुरा पर्युपासीना समयन्ति वरस्त्रियः}
{सं वीरशयने सुप्तं रमयन्त्यशिवाः शिवाः}


\twolineshloka
{यं पुरा पर्युपासीना रमयन्ति महीक्षितः}
{महीतलस्यं निहतं गृध्रास्तं पर्युपासते}


\twolineshloka
{यं पुरा व्यजनैरम्यैरुपवीजन्ति योषितः}
{तमद्य पक्षव्यजनै रुपवीजन्ति पक्षिणः}


\twolineshloka
{एष शेते महाबाहुर्बलवान्सत्यविक्रमः}
{सिंहेनेव द्विपः सङ्ख्ये भीमसेनेन पातितः}


\twolineshloka
{पश्य दुर्योधनं कृष्ण शयानं रुधिरोक्षितम्}
{निहतं भीमसेनेन गदां सम्मृज्य भारत}


\twolineshloka
{अक्षौहिणीर्महाबाहुर्दश चैकां च केशव}
{आनयद्यः पुरा सङ्ख्ये सोऽनयान्निधनं गतः}


\twolineshloka
{एष दुर्योधनः शेते महेष्वासो महाबलः}
{शार्दूल इव सिंहेन भीमसेनेन पातितः}


\twolineshloka
{विदुरं ह्यवमत्यैष पितरं चैव मन्दभाक्}
{बालो वृद्धावमानेन मन्दो मृत्युवशं गतः}


\twolineshloka
{निःसपत्ना मही यस्य त्रयोदशसमाः स्थिता}
{स शेते निहतो भूमौ पुत्रो मे पृथिवीपतिः}


\twolineshloka
{अपश्यं कृष्ण पृथिवीं धार्तराष्ट्रानुशासिताम्}
{पूर्णां हस्तिगवाश्वैश्च वार्ष्णेय न तु तच्चिरम्}


\twolineshloka
{तामेवाद्य महाबाहो पश्याम्यन्यानुशासिताम्}
{हीनां हस्तिगवाश्वेन किन्नु जीवामि माधव}


\twolineshloka
{इदं कष्टतरं पश्य पुत्रस्यापि वधान्मम}
{या इमाः पर्युपासन्ते हताञ्शूरान्रणे स्त्रियः}


\twolineshloka
{प्रकीर्णकेशां सुश्रोणीं दुर्योधनभूजाङ्कगाम्}
{रुक्मवेदिनिभां पश्य कृष्ण लक्ष्मणमातरम्}


\twolineshloka
{नूनमेषा पुरा बाला जीवमाने महाभुजे}
{भुजावाश्रित्य रमते सुभुजस्य मनस्विनी}


\twolineshloka
{कथं तु शतधा नेदं हृदयं मम दीर्यते}
{पश्यन्त्या निहतं पुत्रं पौत्रेण सहितं रणे}


\twolineshloka
{पुत्रं रुधिरसंसिक्तमुपजिघ्रत्यनिन्दिता}
{दुर्योधनं तु वामोरूः पाणिना परिमार्जती}


\twolineshloka
{किन्नु शोचति भर्तारं हतपुत्रं मनस्विनी}
{तथा ह्यवस्थिता भाति पुत्रं चाप्यभिवीक्ष्य सा}


\twolineshloka
{स्वशिरः पञ्चशाखाभ्यामभिहत्यायतेक्षणा}
{पतत्युरसि वीरस्य कुरुराजस्य माधव}


\twolineshloka
{पुण्डरीकनिभा भाति पुण्डरीकान्तरप्रभा}
{मुखं प्रमृज्य पुत्रस्य भर्तुश्चैव तपस्विनी}


\twolineshloka
{यदि सत्यागमाः सन्ति यदि वै श्रुतयस्तथा}
{ध्रुवं लोकानवाप्तोऽयं नृपो बाहुबलार्जितान्}


\chapter{अध्यायः १८}
\twolineshloka
{गान्धार्युवाच}
{}


\twolineshloka
{पश्य माधव पुत्रान्मे शतसङ्ख्याञ्जितक्लमान्}
{गदया भीमसेनेन भूयिष्ठं निहतान्रणे}


\twolineshloka
{इदं दुःखतरं मेऽद्य यदिमा मुक्तमूर्धजाः}
{हतपुत्रा रणे बालाः परिधावन्ति मे स्नुषाः}


\twolineshloka
{प्रासादतलचारिण्यश्चरमैर्भूषणान्वितैः}
{कायेनाद्य स्पृशन्तीमां रुधिरार्द्रां वसुन्धराम्}


\twolineshloka
{कृच्छ्रादुत्सारयन्ति स्म गृध्रगोमायुवायसान्}
{दुःखेनार्ता विघूर्णन्त्यो मत्ता इव चरन्त्युत}


\twolineshloka
{एषाऽन्या त्वनवद्याङ्गी करसम्मितमध्यमा}
{घोरमायोधनं दृष्ट्वा निपतत्यतिदुःखिता}


\twolineshloka
{दृष्ट्वा मे पार्थिवसुतामेतां लक्ष्मणमातरम्}
{राजपुत्रीं महाबाहो मनो न ह्युपशाम्यति}


\twolineshloka
{भ्रातॄंश्चान्याः पितॄंश्चान्याः पुत्रांश्च निहतान्भुवि}
{दृष्ट्वा परिपतन्त्येषाः प्रगृह्य सुमहाभुजान्}


\twolineshloka
{मध्यमानां तु नारीणां वृद्धानां चापराजित}
{आक्रन्दं हतबन्धूना दारुणे वैशसे शृणु}


\twolineshloka
{रथनीडानि देहांश्च हतानां गजवाजिनाम्}
{आश्रित्य श्रममोहार्ताः स्थिताः पश्य महाभुज}


\twolineshloka
{अन्या चापहृतं कायाच्चारुकुण्डलमुन्नसम्}
{स्वस्य बन्धोः शिरः कृष्ण गृहीत्वा पश्य तिष्ठति}


\threelineshloka
{पूर्वजातिकृतं पापं मन्ये नाल्पमिवानघ}
{एताभिर्निरवद्याभिर्मया चैवाल्पपुण्यया}
{यदिदं धर्मराजेन घातितं नो जनार्दन}


% Check verse!
न हि नाशोऽस्ति वार्ष्णेय कर्मणोः शुभपापयोः
\twolineshloka
{प्रत्यग्रवयसः पश्य दर्शनीयकुचाननाः}
{कुलेषु जाता हीमत्यः कृष्णपक्ष्माक्षिमूर्धजाः}


\twolineshloka
{हंसगद्गदभाषिण्यो दुःखशोकप्रमोहिताः}
{सारस्य इव वाशन्त्यः पतिताः पश्य माधव}


\twolineshloka
{फुल्लुपद्यप्रकाशानि पुण्डरीकाक्ष योषिताम्}
{अनवद्यानि वक्त्राणि तापयत्येष रश्मिवान्}


\twolineshloka
{सेर्ष्याणां मम पुत्राणां वासुदेवावरोधनम्}
{मत्तमातङ्गदर्पाणां पश्यन्त्यद्य पृथग्जनाः}


\twolineshloka
{शतचन्द्राणि चर्माणि ध्वजांश्चादित्यवर्चसः}
{रौक्माणि चैव वर्माणि निष्कानपि च काञ्चनान्}


\twolineshloka
{शिरस्त्राणानि चैतानि पुत्राणां मे महीतले}
{पश्य दीप्तानि गोविन्द पावकान्सुहुतानिव}


\twolineshloka
{एष दुःशासनः शेते शूरेणामित्रघातिना}
{पीतशोणिxx सर्वाङ्गो युधि भीमेन पातितः}


\twolineshloka
{गदया भीमसेनेन पश्य माधव मे सुतम्}
{द्यूतक्लेशाननुस्मृत्य द्रौपदीचोदितेन च}


\twolineshloka
{उक्ता ह्यनेन पाञ्चाली सभायां द्यूतनिर्जिता}
{प्रियं चिकीर्षता भ्रातुः कर्णस्य च जनार्दन}


\twolineshloka
{सहैव सहदेवेन नकुलेनार्जुनेन च}
{दासीभूताऽसि पाञ्चालि क्षिप्रं प्रविश नो गृहान्}


\twolineshloka
{ततोऽहमब्रुवं कृष्ण तदा दुर्योधनं नृपम्}
{मृत्युपाशपरिक्षिप्तां द्रौपदीं पुत्र वर्जय}


\twolineshloka
{निबोधैनं सुदुर्बुद्धिं मातुलं कलहप्रियम्}
{क्षिप्रमेनं परित्यज्य पुत्र संशाम्य पाण्डवैः}


\twolineshloka
{न बुद्ध्यसे त्वं दुर्बुद्धे भीमसेनममर्षणम्}
{वाङ्गराचैस्तुदंस्तीक्ष्णैरुल्काभिरिव कुञ्जरम्}


\twolineshloka
{तानेवं रहसि क्रुद्धो वाक्शल्यानवधारयन्}
{उत्ससर्ज विषं तेषु सर्पो गोवृषभेष्विव}


\twolineshloka
{एष दुःशासनः शेते विक्षिप्य विपुलौ भुजौ}
{निहतो भीमसेनेन सिहेनेव महागजः}


\twolineshloka
{अत्यर्थमकरोद्रौद्रं भीमसेनोऽत्यमर्षणः}
{दुःशासनस्य यत्क्रुद्धोऽपिबच्छोणितमाहवे}


\chapter{अध्यायः १९}
\twolineshloka
{गान्धार्युवाच}
{}


\twolineshloka
{एष माधव पुत्रो मे विकर्णः पाज्ञसम्मतः}
{भूमौ विनिहतः शेते भीमेन शतधा कृतः}


\twolineshloka
{गजमध्ये हतः शेते विकर्णो मधुसूदन}
{नीलमेघपरिक्षिप्तः शरदीव निशाकरः}


\twolineshloka
{अस्य चापग्रहेणैव पाणिः कृतकिणो महान्}
{कथञ्चिच्छिद्यते गृध्रैरत्तुकामैस्तलत्रवान्}


\twolineshloka
{अस्य भार्याऽऽमिषप्रेप्सून्गृध्रकाकांस्तपस्विनी}
{वारयत्यनिशं बाला न च शक्नोति माधव}


\twolineshloka
{युवा वृन्दारकसमो विकर्णः पुरुषर्षभ}
{सुखोषितः सुखार्हश्च शेते पांसुषु माधव}


\twolineshloka
{कर्णिनालीकनाराचैर्भिन्नमर्माणमाहवे}
{अद्यापि न जहात्येनं लक्ष्मीर्भरतसत्तमम्}


\twolineshloka
{एष सङ्ग्रामशूरणे प्रतिज्ञां पालयिष्यता}
{दुर्मुखोऽभिमुखः शेते हतोऽरिगणहा रणे}


\twolineshloka
{तस्यैतद्वदनं कृष्ण श्वापदैरर्धभक्षितम्}
{विभात्यभ्यधिकं तात सप्तम्यामिव चन्द्रमाः}


\twolineshloka
{शूरस्य हि रणे कृष्ण पश्याननमथेदृशम्}
{स कथं निहतोऽमित्रैः पांसून्ग्रसति मे सुतः}


\twolineshloka
{यस्याहवमुखे सौम्य स्थाता नैवोपपद्यते}
{स कथं दुर्मुखोऽमित्रैर्हतो विबुधलोकजित्}


\twolineshloka
{चित्रसेनं हतं भूमौ शयानं मधुसूदन}
{धार्तराष्ट्रमिमं पश्य प्रतिमानं धनुष्मताम्}


\twolineshloka
{तं चित्रमाल्याभरणं युवत्यः शोककर्शिताः}
{क्रव्यादसङ्घैः सहिता रुदन्त्यः पर्युपासते}


\twolineshloka
{स्त्रीणां रुदितनिर्घोषः श्वापदानां च गर्जितम्}
{चित्ररूपमिदं कृष्ण विमृश्य प्रतिभाति मे}


\twolineshloka
{युवा वृन्दारको नित्यं प्रवरस्त्रीनिषेवितः}
{विविंशतिरसौ शेते ध्वस्तः पांसुषु माधव}


\twolineshloka
{शरसङ्कृत्तवर्माणं वीरं विशसने हतम्}
{परिवार्यासते गृध्राः परिविंशं विविंशतिम्}


\twolineshloka
{प्रविश्य समरे शूरः पाण्डवानामनीकिनीम्}
{स वीरशयने शेते परः सत्पुरुषोचिते}


\twolineshloka
{स्मितोपपन्नं सुनसं सुभ्रु ताराधिपोपमम्}
{अतीव सौम्यवदनं कृष्ण पश्य विविंशतेः}


\twolineshloka
{यं स्म तं पर्युपासन्ते बहुधा वरयोषितः}
{क्रीडन्तमिव गन्धर्वं देवकन्याः सहस्रशः}


\twolineshloka
{हन्तारं परसैन्यानां शूरं समितिशोभनम्}
{निबर्हणममित्राणां दुःसहं विषहेत कः}


\twolineshloka
{दुःसहस्यैतदाभाति शरीरं संवृतं शरैः}
{गिरिरात्मरुहैः फुल्लैः कर्णिकारैरिवाचितः}


\twolineshloka
{शातकौम्भ्या स्रजा भाति कवचेन च भास्वता}
{अग्निनेव गिरिः श्वेतो गतासुरपि दुःसहः}


\chapter{अध्यायः २०}
\twolineshloka
{गान्धार्युवाच}
{}


\twolineshloka
{अध्यर्धगुणमाहुर्यं बले शौर्ये च केशव}
{पित्रा त्वया च दाशार्ह दृप्तं सिंहमिवोत्कटम्}


\twolineshloka
{यो बिभेद चमूमेको मम पुत्रस्य दुर्भिदाम्}
{स भूत्वा मृत्युरन्येषां स्वयं मृत्युवशं गतः}


\twolineshloka
{तस्योपलक्षये कृष्ण कार्ष्णेरमिततेजसः}
{अभिमन्योर्हतस्यापि प्रभा नैवोपशाम्यति}


\twolineshloka
{एषा विराटदुहिता स्नुषा गाण्डीवधन्वनः}
{हतं बाला पतिं वीरं दृष्ट्वा शोचत्यनिन्दिता}


\twolineshloka
{तमेषा हि समागम्य भार्या भर्तारमन्तिके}
{विराटदुहिता कृष्ण पाणिना परिमार्जति}


\twolineshloka
{तस्य वक्त्रमुपाघ्राय सौभद्रस्य मनस्विनी}
{विबुद्धकमलाकारं कम्बुवृत्तशिरोधरम्}


\twolineshloka
{कामरूपवती चैषा परिष्वजति भामिनी}
{लज्जमाना पुरा चैनं माध्वीकमदमूर्च्छिता}


\twolineshloka
{तस्य क्षतजसन्दिग्धं जातरूपपरिष्कृतम्}
{विमुच्य कवचं कृष्ण शीरमभिवीक्षते}


\twolineshloka
{अवेक्षमाणा तं बाला कुष्ण त्वामभिभाषते}
{अयं ते पुण्डरीकाक्ष कसदृशाक्षो निपातितः}


\twolineshloka
{बले वीर्ये च सदृशस्तेजसा चैव तेऽनघ}
{रूपेण च तथाऽत्यर्थं शेते भुवि निपातितः}


\twolineshloka
{अत्यन्तं सुकुमारस्य राङ्कवाजिनशायिनः}
{कच्चिदद्य शरीरं ते भूमौ न परितप्यते}


\threelineshloka
{मातङ्गभुजवर्ष्णाणौ ज्याक्षेपकठिनत्वचौ}
{काञ्चनाङ्गदिनौ शेते निक्षिप्य विपुलौ भुजौ}
{व्यायम्य बहुधा नूनं सुखसुप्तः श्रमादिव}


\twolineshloka
{एवं विलपतीमार्तां किं मां न प्रतिभाषसे}
{न स्मराम्यपराधं ते किं मां न प्रतिभाषसे}


\twolineshloka
{ननु मां त्वं पुरा दूरादभिवीक्ष्याभिभाषसे}
{न स्मराम्यपराधं मे किं मां न प्रतिभाषसे}


\twolineshloka
{आर्यामार्य सुभद्रां तवमिमां श्च त्रिदशोपमान्}
{पितॄन्मां चैव दुःखार्तां विहाय क्व गमिष्यसि}


\twolineshloka
{तस्य शोणितदिग्धान्वै केशानुन्नम्य पाणिना}
{उत्सङ्गे वक्तमाधाय जीवन्तमिव पृच्छति}


\twolineshloka
{स्वस्रीयं वासुदेवस्य पुत्रं गाण्डीवधन्वनः}
{कथं त्वां रणमध्यस्थं जघ्रुरेते महारथाः}


\twolineshloka
{धिगस्तु क्रूरकर्तॄंस्तान्कृपकर्णजयद्रथान्}
{द्रोणद्रौणायतनी चोभौ यैरहं विधवा कृता}


\twolineshloka
{रथर्षभाणां सर्वेषां कथमासीत्तदा मनः}
{बालं त्वां पिरवार्यैकमनेकेषां च जघ्नताम्}


\twolineshloka
{कथं नु पाण्डवानां च पाञ्चालानां तु पश्यताम्}
{त्वं वीर निधनं प्राप्तो नाथवान्सन्ननाथवत्}


\twolineshloka
{दृष्ट्वा बहुभिराक्रन्दे निहतं त्वां पिता तव}
{वीरः पुरुषशार्दूलः कथं जीवति पाण्डवः}


\twolineshloka
{न राज्यलाभो विपुलः शत्रूणां च पराभवः}
{प्रीतिं धास्यति पार्थानां त्वामृते पुष्करेक्षण}


\twolineshloka
{तव शस्त्रजिताँल्लोकान्धर्मेण च दमेन च}
{क्षिप्रमन्वागमिष्यामि तत्र मां प्रतिपालय}


\twolineshloka
{दुर्मरं पुनरप्राप्ते काले भवति केनचित्}
{यदहं त्वां रणे दृष्ट्वा हतं जीवामि दुर्भगा}


\twolineshloka
{कामिदानीं नरव्याघ्र श्लक्ष्णया स्मितया गिरा}
{पितृलोके समेत्यान्यां मामिवामन्त्रयिष्यसि}


\twolineshloka
{नूनमप्सरसां स्वर्गे मनांसि प्रमथिष्यसि}
{परमेण च रूपेण गिरा च स्मितपूर्वया}


\twolineshloka
{प्राप्य पुण्यकृताँल्लोकानप्सरोभिः समेयिवान्}
{सौभद्र विहन्काले स्मरेथाः सुकृतानि मे}


\twolineshloka
{एतावानिह संवासो विहितस्ते मया सह}
{षण्मासान्सप्तमे मासि त्वं वीर निधनं गतः}


\twolineshloka
{इत्युक्तवचनामेनामपकर्षन्ति दुःखिताम्}
{उत्तरां मोघसङ्कल्पां मत्स्यराजकुलस्त्रियः}


\twolineshloka
{उत्तरामपकृष्यैनामार्तामार्ततराः स्वयम्}
{विराटं निहतं दृष्ट्वा क्रोशन्ति विलपन्ति च}


\twolineshloka
{द्रोणास्त्रशरसङ्कृत्तं शयानं रुधिरोक्षितम्}
{विराटं वितुदन्त्येते गृध्रगोमायुवायसाः}


\twolineshloka
{वितुद्यमानं विहगैर्विराटमसितेक्षणाः}
{न शक्नुवन्ति विहगान्निवारयितुमातुराः}


\twolineshloka
{आसामातपतप्तानामायासेन च योषिताम्}
{श्रमेण च विवर्णानां वक्त्राणां विप्लुतं वपुः}


\threelineshloka
{उत्तरं चाभिमन्युं च काम्भोजं च सुदक्षिणम्}
{कार्ष्णिनाऽभिहतं पश्य लक्ष्मणं प्रियदर्शनम्}
{}


% Check verse!
आयोधनशिरोमध्ये शयानं पश्य माधव
\chapter{अध्यायः २१}
\twolineshloka
{गान्धार्युवाच}
{}


\twolineshloka
{आवन्त्यं भीमसेनेन भक्षयन्ति निपातितम्}
{गृध्रगोमायवः शूरं दूरबन्धुमबन्धुवत्}


\twolineshloka
{तं पश्य कदनं कृत्वा शूरणां मधुसूदन}
{शयानं वीरशयने रुधिरेण समुक्षितम्}


\twolineshloka
{तं सृगालाश्च कङ्काश्च क्रव्यादाश्च पृथग्विधाः}
{तेनतेन विकर्षन्ति पश्य कालस्य पर्ययम्}


\twolineshloka
{शयानं वीरशयने शूरमाक्रन्दकारिणम्}
{आवन्त्यमभितो नार्यो रुदत्यः पर्युपासते}


\twolineshloka
{प्रातिपेयं महेष्वासं हत भीमेन बाह्लिकम्}
{प्रसुप्तमिव शार्दूलं पश्य कृष्णमनस्विनम्}


\twolineshloka
{अतीव मुखवर्णोऽस्य निहतस्यापि शोभते}
{सोमस्यवाभिपूर्णस्य पौर्णमास्यां समुद्यतः}


\twolineshloka
{पुत्रशोकाभितप्तेन प्रतिज्ञां चाभिरक्षता}
{पाकशासनिना सङ्ख्ये वार्धक्षत्रिर्निपातितः}


\twolineshloka
{एकादशचमूर्भित्त्वा रक्ष्यमाणं महात्मभिः}
{सत्यं चिकीर्षता पश्य हतमेनं जयद्रथम्}


\twolineshloka
{सिन्धुसौवीरभर्तारं दर्पपूर्णं मनस्विनम्}
{भक्षयन्ति शिवा गृध्रा जनार्दन जयद्रथम्}


\twolineshloka
{संरक्ष्यमाणं भार्याभिरनुरक्ताभिरच्युत}
{भीषयन्त्यो विकर्षन्ति गहनं निम्नमन्तिकात्}


\twolineshloka
{तमेताः पर्युपासन्ते वीक्षमाणा महाभुजम्}
{सिन्धुसौवीरकाम्भोजगान्धारयवनस्त्रियः}


\twolineshloka
{यदा कृष्णामुपादाय प्राद्रवत्केकयैः सह}
{तदैव वध्यः पाण्डूनां जनार्दन जयद्रथः}


\twolineshloka
{दुःशलां मानयद्भिस्तु तदा मुक्तो जयद्रथः}
{कथमद्य न तां कृष्ण मानयन्ति स्म ते पुनः}


\twolineshloka
{सैन्धवं मे सुता बाला प्रस्खलन्तीव दुःखिता}
{प्रमापयन्ती चात्मानमाक्रोशन्तीव पाण्डवान्}


\twolineshloka
{किन्नु दुःखतरं कृष्ण परं मम भविष्यति}
{यत्सुता विधवा बाला स्नुषाश्च निहतेश्वराः}


\twolineshloka
{हाहा धिग्धुःशलां पश्य वीतशोकभयामिव}
{भर्तुः शिर अपश्यन्तीं धावमानामितस्ततः}


\twolineshloka
{वारयामास यः सर्वान्पाण्डवान्पुत्रगृद्धिनः}
{स हत्वा विपुलाः सेनाः स्वयं मृत्युवशं गतः}


\twolineshloka
{तं मत्तमिव मातङ्गं वीरं परमदुर्जयम्}
{परिवार्य रुदन्त्येताः स्त्रियश्चन्द्रोपमाननाः}


\chapter{अध्यायः २२}
\twolineshloka
{गान्धार्युवाच}
{}


\twolineshloka
{आवनत्यं भीमसेनेन भक्षयन्ति निपातितम्}
{गृध्रगोमायवः शूरं दूरबन्धुमबन्धुवत्}


\twolineshloka
{तं पश्य कदनं कृत्वा शूराणां मधुसूदन}
{शयानं वीरशयने रुधिरेण समुक्षितम्}


\twolineshloka
{तं सृगालाश्च कङ्काश्च क्रव्यादाश्च पृथग्विधाः}
{तेनतेन विकर्षन्ति पश्य कालस्य पर्ययम्}


\twolineshloka
{शयानं वीरशयने शूरमाक्रन्दकारिणम्}
{आवन्त्यमभितो नार्यो रुदत्यः पर्युपासते}


\twolineshloka
{प्रातिपेयं महेष्वासं हतं भीमेन बाह्लिकम्}
{प्रसुप्तमिव शार्दुलं पश्य कृष्णमनस्विनम्}


\twolineshloka
{अतीव मुखवर्णोऽस्य निहतस्यापि शोभते}
{सोमस्यवाभिपूर्णस्य पौर्णमास्यां समुद्यतः}


\twolineshloka
{पुत्रशोकाभितप्तेन प्रतिज्ञां चाभिरक्षता}
{पाकशासनिना सङ्ख्ये वार्धक्षत्रिर्निपातितः}


\twolineshloka
{एकादशचमूर्भित्त्वा रक्ष्यमाणं महात्मभिः}
{सत्यं चिकीर्षता पश्य हतमेनं जयद्रथम्}


\twolineshloka
{सिन्धुसौवीरभर्तारं दर्पपूर्णं मनस्विनम्}
{भक्षयन्ति शिवा गृध्रा जनार्दन जयद्रथम्}


\twolineshloka
{संरक्ष्यमाणं भार्याभिरनुरक्ताभिरच्युत}
{भीषयन्त्यो विकर्षन्ति गहनं निम्नमन्तिकात्}


\twolineshloka
{तमेताः पर्यपासन्ते वीक्षमाणा महाभुजम्}
{सिन्धुसौवीरकाम्भोजगान्धारयवनस्त्रियः}


\twolineshloka
{यदा कृष्णामुपादाय प्राद्रवत्केकयैः सह}
{तदैव वध्यः पाण्डूनां जनार्दन जयद्रथः}


\twolineshloka
{दुःशलां मानयद्भिस्तु तदा मुक्तो जयद्रथः}
{कथमद्य न तां कृष्ण मानयन्ति स्म ते पुनः}


\twolineshloka
{सैन्धवं मे सुता बाला प्रस्स्वलन्तीव दुःखिता}
{प्रमापयन्ती चात्मानमाक्रोशन्तीव पाण्डवान्}


\twolineshloka
{किन्नु दुःखतरं कृष्ण परं मम भविष्यति}
{यत्सुता विधवा बाला स्नुषाश्च निहतेश्वराः}


\twolineshloka
{हाहा धिग्धुःशलां पश्य वीतशोकभयामिव}
{भर्तुः शिर अपश्यन्तीं धावमानामितस्ततः}


\twolineshloka
{वारयामास यः सर्वान्पाण्डवान्पुत्रगृद्विनः}
{स हत्वा विपुलाः सेनाः सर्वयं मृत्युवशं गतः}


\twolineshloka
{तं मत्तमिव मातङ्गं वीरं परमदुर्जयम्}
{परिवार्य रुदन्त्येताः स्त्रियश्चन्द्रोपमाननाः}


\chapter{अध्यायः २३}
\twolineshloka
{गान्धार्युवाच}
{}


\twolineshloka
{एष शल्यो हतः शेते साक्षान्नकुलमातुलः}
{धर्मज्ञेन हतस्तात धर्मराजेन संयुगे}


\twolineshloka
{यस्त्वया स्पर्धते नित्यं सर्वत्र पुरुषर्षभ}
{स एष निहतः शेते मद्रराजो महाबलः}


\twolineshloka
{`जयद्रथे यदि ब्रूयुरुपरोधं कथञ्चन}
{मद्रपुत्रे कथं ब्रूयुरुपरोधं विवक्षवः ॥'}


\twolineshloka
{येन सङ्गृह्णता तात रथमाधिरथेर्युधि}
{जयार्थं पाण्डुपुत्राणां तदा तेजोवधः कृतः}


\twolineshloka
{अहो धिक्फश्य शल्यस्य पूर्णचन्द्रसुदर्शनम्}
{मुखं प्द्मपलाशाक्षं काकैरादष्टमव्रणम्}


\twolineshloka
{एषा चामीकराभस्य तप्तकाञ्चनसप्रभा}
{आस्याद्विनिः सृता जिह्वा भक्ष्यते कृष्ण पक्षिभिः}


\twolineshloka
{युधिष्ठिरेण निहतं शल्यं समितिशोभनम्}
{रुदन्त्यः पर्युपासन्ते मद्रराजं कुलाङ्गनाः}


\twolineshloka
{एताः सुसूक्ष्मवसना मद्रराजं नरर्षभम्}
{क्रोशन्त्योऽथ समासाद्य क्षत्रियाः क्षत्रियर्षभम्}


\twolineshloka
{शल्यं निपतितं नार्यः परिवार्याभितः स्थिताः}
{वासिता गृष्टयः पङ्के परिमग्नमिवर्षभम्}


\twolineshloka
{शल्यं शरणदं शूरं पश्येमं वृष्णिनन्दन}
{शयानं वीरशयने शरैर्विशकलीकृतम्}


\twolineshloka
{एष शैलालयो राजा भगदत्तः प्रतापवान्}
{गजाङ्कुशधरः श्रीमाञ्शेते भुवि निपातितः}


\twolineshloka
{यस्य रुक्ममयी माला शिरस्येषा विराजते}
{श्वापदैर्भक्ष्यमाणस्य शोभयन्तीव मूर्धजान्}


\twolineshloka
{एतेन किल पार्थस्य युद्धमासीत्सुदारुणम्}
{रोमहर्षणमत्युग्रं शक्रस्य त्वहिना यथा}


\twolineshloka
{योधयित्वा माहाबाहुरेष पार्थं धनञ्जयम्}
{संशयं गमयित्वा च कुन्तीपुत्रेण पातितः}


\twolineshloka
{यस्य नास्ति समो लोके शौर्ये वीर्ये च कश्चन्}
{स एष निहतः शेते भीष्मो भीष्मकृदाहवे}


\twolineshloka
{पश्य शान्तनवं कृष्ण शयानं सूर्यवर्चसम्}
{युगान्त इव कालेन पातितं सूर्यमम्बरात्}


\twolineshloka
{एष तप्त्वा रणे शत्रूञ्शस्त्रतापेन वीर्यवान्}
{नरसूर्योऽस्तमभ्येति सूर्योऽस्तमिव केशव}


\twolineshloka
{शरतल्पगतं भीष्ममूर्ध्वरेतसमच्युतम्}
{शयानं वीरशयने पश्य शूरनिषेविते}


\twolineshloka
{कर्णिनालीकनाराचैरास्तीर्य शयनोत्तमम्}
{आविश्य शेते भगवान्स्कन्दः शरवणं यथा}


\twolineshloka
{अतूलपूर्णं गाङ्गेयस्त्रिभिर्बाणैः समन्वितम्}
{उपधायोपधानाग्र्यं दत्तं गाण्डीवधन्वना}


\twolineshloka
{पालयानः पितुः शास्त्रमूर्ध्वरेता महायशाः}
{एष शान्तनवः शेते माधवाप्रतिमो युधि}


\twolineshloka
{धर्मात्मा तात सर्वज्ञः पारावर्येण निर्णये}
{अमर्त्य इव मर्त्यः सन्नेष प्राणानधारयत्}


\twolineshloka
{नास्ति युद्धे कृती कश्चिन्न विद्वान््न पराक्रमी}
{यत्र शान्तनवो भीष्मः शेतेऽद्य निहतः परैः}


\twolineshloka
{स्वयमेतेन शूरेण पृच्छयमानेन पाण्डवैः}
{धर्मज्ञेनाहवे मृत्युरादिष्टः सत्यवादिना}


\twolineshloka
{प्रनष्टः कुरुवंशश्च पुनर्येन समुद्वृतः}
{स गतः कुरुभिः सार्धं महाबुद्धिः पराभवम्}


\twolineshloka
{धर्मेण कुरवः केन परिद्रक्ष्यन्ति माधव}
{हते देवव्रते भीष्मे देवकल्पे नरर्षभे}


\twolineshloka
{अर्जुनस्य विनेतारमाचार्यं सात्यकेस्तथा}
{तं पश्य पतितं द्रोणं कुरूणां गुरुमुत्तमम्}


\twolineshloka
{अस्त्रं चतुर्विधं वेद यथैव त्रिदशेश्वरः}
{भार्गवो वा महावीर्यस्तथा द्रोणोऽपि माधव}


\twolineshloka
{यस्य प्रसादाद्बीभत्सुः पाण्डवः कर्म दुष्करम्}
{चकार स हतः शेते नैनमस्त्राण्यपालयन्}


\twolineshloka
{यं पुरोधाय कुरव आह्वयन्ति स्म पाण्डवान्}
{सोऽयं शस्त्रभृतां श्रेष्ठो द्रोणः शस्त्रपृथक्कृतः}


\twolineshloka
{यस्य निर्दहतः सेनां गतिरग्नरिवाभवत्}
{स भूमौ निहतः शेते शान्तार्चिरिव पावकः}


\twolineshloka
{धनुर्मुष्टिरशीर्णश्च हस्तावापश्च माधव}
{द्रोणस्य निहतस्यापि दृश्यते जीवतो यथा}


\twolineshloka
{वेदा यस्माच्च चत्वारः सर्वाण्यस्त्राणि केशव}
{अनपेतानि वै शूराद्यथैवादौ प्रजापतेः}


\twolineshloka
{चन्दनार्हाविमौ तस्य बन्दिभिर्वन्दितौ शुभौ}
{गोमायवो विकर्षन्ति पादौ शिष्यशतार्चितौ}


\twolineshloka
{द्रोणं द्रुपदपुत्रेण निहतं मधुसूदन}
{कृपी कृपणमन्वास्ते दुःखोपहतचेतना}


\twolineshloka
{तां पश्य रुदतीमार्तां मुक्तकेशीमधोमुखीम्}
{हतं पतिमुपासन्तीं द्रोणं शस्त्रभृतां वरम्}


\twolineshloka
{बाणैर्भिन्नतनुत्राणं धृष्टद्युम्नेन केशव}
{उपास्ते वै मृधे द्रोणं जटिला ब्रह्मचारिणी}


\twolineshloka
{प्रेतकृत्ये च यतते कृपी कृपणमातुरा}
{हतस्य समरे भर्तुः सुकुमारी यशस्विनी}


\twolineshloka
{अग्नीनाधाय विविधवच्चितां प्रज्वाल्य सर्वतः}
{द्रोणमाधाय गायन्ति त्रीणि सामानि सामगाः}


\twolineshloka
{कुर्वन्ति च चितामेते जटिला ब्रह्मचारिणः}
{धनुर्भिः शक्तिभिश्चैव रथनीडैश्च माधव}


\twolineshloka
{शरैश्च विविधैरन्यैर्धक्ष्यन्ते भूरितेजसम्}
{त एते द्रोणमादाय गायन्ति च रुदन्ति च}


\twolineshloka
{सामभिस्त्रिभिरन्तस्थैरनुशंसन्ति चापरे}
{अग्नावग्निं समाधाय द्रोणं हुत्वा हुताशने}


\twolineshloka
{गच्छन्त्यभिमुखा गङ्गां द्रोणशिष्या द्विजातयः}
{अपसव्यां चितिं कृत्वा पुरस्कृत्य कृपीं तथा}


\chapter{अध्यायः २४}
\twolineshloka
{गान्धार्युवाच}
{}


\twolineshloka
{सोमदत्तसुतं पश्य युयुधानेन पातितम्}
{वितुद्यमानं विहगैर्बहुभिर्माधवान्तिके}


\twolineshloka
{पुत्रशोकाभिसन्तप्तः शोमदत्तो जनार्दन}
{युयुधानं महेष्वासं गर्हयन्निव दृश्यते}


\twolineshloka
{असौ हि भूरिश्वसो माता शोकपरिप्लुता}
{आश्वासयति भर्तारं सोमदत्तमनिन्दिता}


\twolineshloka
{दिष्ट्या नैनं महाराज दारुणं भरतक्षयम्}
{कुरुसङ्क्रन्दनं घोरं युगान्तमनुपश्यसि}


\twolineshloka
{दिष्ट्या यूपध्वजं पुत्रं वीरं भूरिसहस्रदम्}
{अनेकक्रतुयज्वानं निहतं नानुपश्यसि}


\twolineshloka
{दिष्ट्या स्नुषाणामाक्रन्दे घोरं विलपितं बहु}
{न शृणोषि महाराज सारसीनामिवार्णवे}


\twolineshloka
{एकवस्त्रास्तु पञ्चैताः प्रकीर्णासितमूर्धजाः}
{स्नुषास्ते परिधावन्ति हतापत्या हतेश्वराः}


\twolineshloka
{श्वापदैर्भक्ष्यमाणं त्वमहो दिष्ट्या न पश्यसि}
{छिन्नबाहुं नरव्याघ्रमर्जुनेन निपातितम्}


\twolineshloka
{शलं विनिहतं सङ्ख्ये भूरिश्रवसमेव च}
{स्नुषाश्च विविधाः सर्वा दिष्ट्या नाद्येह पश्यसि}


\twolineshloka
{दिष्ट्या तत्काञ्चनं छत्रं यूपकेतोर्महात्मनः}
{विनीकीर्णं रथोपस्थे सौमदत्तेर्न पश्यसि}


\twolineshloka
{अमूस्तु भूरिश्रवसो भार्याः सात्यकिना हतम्}
{परिवार्यानुशोचन्ति भर्तारमसितेक्षणाः}


\twolineshloka
{एता विलप्य करुणं भर्तृशोकेन कर्शिताः}
{पतन्त्यभिमुखा भूमौ कृपणं बत केशव}


\twolineshloka
{बीभत्सुरतिबीभत्सं कर्मेदमकरोत्कथम्}
{प्रमत्तस्य यदच्छैत्सीद्बाहुं शूरस्य यज्वनः}


\twolineshloka
{ततः पापतरं कर्म कृतवानपि सात्यकिः}
{यस्मात्प्रायोपविष्टस्य प्राहार्षीत्संशितात्मनः}


\twolineshloka
{एको द्वाभ्यां हतः शेते क्षत्रधर्मेण धार्मिकः}
{किन्नु वक्ष्यति वै सत्सु गोष्ठीषु च सभासु च}


\twolineshloka
{अपुण्यमयशस्यं च कर्मेदं सात्यकिः स्वयम्}
{इति यूपध्वजस्यैताः स्त्रियः क्रोशन्ति माधव}


\twolineshloka
{भार्या यूपध्वजस्यैषा करसम्मितमध्यमा}
{कृत्वोत्सङ्गे भुजं भर्तुः कृपणं परिदेवति}


\twolineshloka
{अयं स हन्ता शूराणां मित्राणामभयप्रदः}
{प्रदाता गोसहस्राणां क्षत्रियान्तकरः करः}


\twolineshloka
{अयं स रशनोत्कर्षी पीनस्तनविमर्दनः}
{नाभ्यूरुजघनस्पर्शी नीवीविस्रंसनः करः}


\twolineshloka
{वासुदेवस्य सान्निध्ये पार्थेनाक्लिष्टकर्मणा}
{युध्यतः समरेऽन्येन प्रमत्तस्य निपातितः}


\twolineshloka
{किन्नु वक्ष्यसि संसत्सु कथासु च जनार्दन}
{अर्जुनस्य महत्कर्म स्वयं वा स किरीटभृत्}


\twolineshloka
{इत्येवं गर्हयित्वैषा तूष्णीमास्ते वराङ्गना}
{तामेतामनुशोचन्ति सपत्न्यः स्वामिव स्नुषाम्}


\twolineshloka
{गान्धारराजः शकुनिर्बलवान्सत्यविक्रमः}
{निहतः सहदेवेन भागिनेमेन मातुलः}


\twolineshloka
{यः पुरा हेमदण्डाभ्यां व्यजनाभ्यां स्म वीज्यते}
{स एष पक्षिभिः पक्षैः शयान उपवीज्यते}


\twolineshloka
{यः स्वरूपाणि कुरुते शतशोऽथ सहस्रशः}
{तस्य मायाविनो माया दग्धाः पाण्डवतेजसा}


\twolineshloka
{मायया निकृतिप्रज्ञो जितवान्यो युधिष्ठिरम्}
{सभायां विपुलं राज्यं स जहौ जीवितं कथम्}


\twolineshloka
{शकुन्ताः शकुनिं कृष्ण समन्तात्पर्युपासते}
{कितवं मम पुत्राणां विनाशायोपशिक्षितम्}


\twolineshloka
{एतेनैतन्महद्वैरं प्रसक्तं पाण्डवैः सह}
{वधाय मम पुत्राणामात्मनः सगणस्य च}


\twolineshloka
{यथैव मम पुत्राणां लोकाः शस्त्रजिताः प्रभो}
{एवमस्यापि दुर्बुद्धेर्लोकाः शस्त्रेण वै जिताः}


\twolineshloka
{कथं च नायं तत्रापि पुत्रान्मे भ्रातृभिः सह}
{विरोधयेदृजुप्रज्ञाननृजुर्मधुसूदन}


\chapter{अध्यायः २५}
\twolineshloka
{गान्धार्युवाच}
{}


\twolineshloka
{काम्भोजं पश्य दुर्धर्षं काम्भोजास्तरणोचितम्}
{शयानमृषभस्कन्धं हतं पांसुषु माधव}


\twolineshloka
{अस्य क्षतजसन्दिग्धौ बाहू चन्दनरूषितौ}
{अवेक्ष्य कृपणं भार्या विलपत्यतिदुःखिता}


\twolineshloka
{इमौ तौ परिघप्रख्यौ बाहू चन्दनरूषितौ}
{ययोर्विवरमापन्नां न रतिर्मां पुराऽजहात्}


\twolineshloka
{कां गतिं तु गमिष्यामि त्वया हीना जनेश्वर}
{दूरबन्धुरनाथा च वेपन्ती मधुरस्वरा}


\twolineshloka
{आतपे क्लाम्यमानानां विविधानामिव स्रुजाम्}
{क्लान्तानामपि नारीणां जहाति श्रीर्न वै तनूः}


\twolineshloka
{शयानमभितः शूरं कालिङ्गं मधुसूदन}
{पश्य दीप्ताङ्गदयुतं प्रतिमानं धनुष्मताम्}


\twolineshloka
{मागधानामधिपतिं जयत्सेनं जनार्दन}
{परिवार्य प्ररुदिता मागध्यः पश्य योषितः}


\twolineshloka
{हरिणायतनेत्राणां सुस्वराणां जनार्दन}
{मनः श्रुतिहरो नादो मनो मोहयतीव मे}


\twolineshloka
{प्रकीर्णवस्त्राभरणा रुदत्यः शोककर्शिताः}
{स्वास्तीर्णशयनोपेता मागध्यः शेरते भुवि}


\twolineshloka
{कोसलानामधिपतिं राजपुत्रं बृहद्बलम्}
{भर्तारं परिवार्यैताः पृथक्प्ररुदिताः स्त्रियः}


\twolineshloka
{अस्य गात्रगतान्बाणान्कार्ष्णिबाहुबलेरितान्}
{उद्धरन्त्यसुखाविष्टा मूर्च्छमानाः पुनः पुनः}


\twolineshloka
{आसां सर्वानवद्यानामातपेन परिश्रमात्}
{प्रम्लाननलिनाभानि भान्ति वक्त्राणि माधव}


\twolineshloka
{द्रोणेन निहताः शूरा शेरते रुचिराङ्गदाः}
{धृष्टद्युम्नसुताः सर्वे शिशवो हेममालिनः}


\twolineshloka
{रथाग्न्यगारं चापार्चिं शरशक्तिगदेन्धनम्}
{द्रोणमासाद्य निर्दग्धाः शलभा इव पावकम्}


\twolineshloka
{तथैव निहताः शूराः शेरते रुचिराङ्गदाः}
{द्रोणेनाभिमुखाः सङ्ख्ये भ्रातरः पञ्च केकयाः}


\twolineshloka
{तप्तकाञ्चनवर्माणस्तालध्वजस्थव्रजाः}
{भासयन्ति महीं भासा ज्वलिता इव पावकाः}


\twolineshloka
{द्रोणेन द्रुपदं सङ्ख्ये पश्य माधव पातितम्}
{महाद्विपमिवारण्ये सिंहेन महता हतम्}


\twolineshloka
{पाञ्चालराज्ञो विमलं पुण्डरीकाक्ष पाण्डुरम्}
{आतपत्रं समाभाति शरदीव निशाकरः}


\twolineshloka
{एतास्तु द्रुपदं वृद्धं स्नुषा भार्याश्च दुःखिताः}
{दग्ध्वा गच्छन्ति पाञ्चालराजानमपसव्यतः}


\twolineshloka
{धृष्टकेतुं महात्मानं चेदिपुङ्गवमङ्गनाः}
{द्रोणेन निहतं शूरं हरन्ति हृतचेतसः}


\twolineshloka
{द्रोणास्त्रमभिहत्यैष विमर्दे मधुसूदन}
{महेष्वासो हतः शेते वज्राहत इव द्रुमः}


\twolineshloka
{एष चेदिपतिः शूरो धृष्टकेतुर्महारथः}
{शेते विनिहतः सङ्ख्ये हत्वा शत्रून्सहस्रशः}


\twolineshloka
{वितुद्यमानं विहगैस्तं भार्याः पर्युपाश्रिताः}
{चेदिराजं हृषीकेश हतं सबलबान्धवम्}


\twolineshloka
{दाशार्हपुत्रजं वीरं शयानं सत्यविक्रमम्}
{आरोप्याङ्के रुदन्त्येताश्चेदिराज वराङ्गनाः}


\twolineshloka
{अस्य पुत्रं हृषीकेश सुवक्त्रं चारुकुण्डलम्}
{द्रोणेन समरे पश्य निकृत्तं बहुधा शरैः}


\twolineshloka
{पितरं नूनमाजिस्थं युध्यमानं परैः सह}
{नाजहात्पितरं वीरमद्यापि मधुसूदन}


\twolineshloka
{एवं ममापि पुत्रस्य पुत्रः पितरमन्वगात्}
{दुर्योधनं महाबाहो लक्ष्मणः परवीरहा}


\twolineshloka
{विन्दानुविन्दावावन्त्यौ पतितौ पश्य माधव}
{हिमान्ते पुष्पितौ शालौ मरुता गलिताविव}


\twolineshloka
{काञ्चनाङ्गदवर्माणौ बाणखङ्गधनुर्धरौ}
{ऋषभप्रतिरूपाक्षौ शयानौ विमलस्रजौ}


\twolineshloka
{अवध्याः पाण्डवाः कृष्ण सर्व एव त्वया सह}
{ये मुक्ता द्रोणभीष्माभ्यां कर्णाद्वैकर्तनात्कृपात्}


\twolineshloka
{दुर्योधनाद्द्रोणसुतात्सैन्धवाच्च जयद्रथात्}
{सोमदत्ताद्विकर्णाच्च शूराच्च कृतवर्मणः}


\twolineshloka
{ये हन्युः शस्त्रवेगेन देवानपि नरर्षभाः}
{त इमे निहताः सङ्ख्ये पश्य कालस्य पर्ययम्}


\threelineshloka
{नातिभारोऽस्ति दैवस्य ध्रुवं माधव कश्चन}
{यदिमे निहताः शूराः क्षत्रियैः क्षत्रियर्षभाः}
{`शूराश्च कृतविद्याश्च मम पुत्रा मनस्विनः'}


\twolineshloka
{तदैव निहताः कृष्ण मम पुत्रास्तरस्विनः}
{यदैवाकृतकामस्त्वमुपप्लाव्यं गतः पुनः}


\twolineshloka
{शन्तनोश्चैव पुत्रेण प्राज्ञेन विदुरेण च}
{तदैवोक्तास्मि मा स्नेहं कुरुष्वात्मसुतेष्विति}


\threelineshloka
{तयोर्हि दर्शनं नैतन्मिथ्या भवितुमर्हति}
{अचिरेणैव मे पुत्रा भस्मीभूता जनार्दन ॥वैशम्पायन उवाच}
{}


\twolineshloka
{इत्युक्त्वा न्यपतद्भूमौ गान्धारी शोकमूर्च्छिता}
{दुःखोपहतविज्ञाना धैर्यमुत्सृज्य भारत}


\threelineshloka
{ततः कोपपरीताङ्गी पुत्रशोकपरिप्लुता}
{जगाद शौरिं दोषेण गान्धारी व्यथितेन्द्रिया ॥गान्धार्युवाच}
{}


\twolineshloka
{पाण्डवाः धार्तराष्ट्राश्च क्रुद्धाः कृष्ण परस्परम्}
{उपेक्षिता विनश्यन्तस्त्वया कस्माज्जनार्दन}


\twolineshloka
{शक्तेन बहुभृत्येन विपुले तिष्ठता बले}
{उभयत्र समर्थेन श्रुतवाक्येन चैव ह}


\twolineshloka
{इच्छतोपेक्षितो नाशः कुरूणां मधुसूदन}
{यस्मात्त्वया महाबाहो फलं तस्मादवाप्नुहि}


\twolineshloka
{पतिशुश्रूषया यन्मे तपः किञ्चिदुपार्जितम्}
{तेन त्वां दुरवापेन शपे चक्रगदाधरम्}


\twolineshloka
{यस्मात्परस्परं घ्नन्तो ज्ञातयः कुरुपाण्डवाः}
{उपेक्षितास्ते गोविन्द तस्माज्ज्ञातीन्वधिष्यसि}


\twolineshloka
{त्वमप्युपस्थिते वर्षे षट््त्रिंशे मधुसूदन}
{हतज्ञातिर्हतामात्यो हतपुत्रो वनेचरः}


\twolineshloka
{अनाथवदविज्ञातो लोकेष्वनभिलक्षितः}
{कुत्सितेनाभ्युपायेन निधनं समवाप्स्यसि}


\threelineshloka
{तवाप्येवं हतसुता निहतज्ञातिबान्धवाः}
{स्त्रियः परिपतिष्यन्ति यथैता भरतस्त्रियः ॥वैशम्पयन उवाच}
{}


\twolineshloka
{तच्छ्रुत्वा वचनं घोरं वासुदेवो महामनाः}
{उवाच देवीं गान्धारीमीषदभ्युत्स्मयन्निव}


\twolineshloka
{जानेऽहमेतदप्येवं चीर्णं चरसि क्षत्रिये}
{दैवादेव विनश्यन्ति वृष्णयो नात्र संशयः}


\twolineshloka
{संहर्ता वृष्णिचक्रस्य नान्यो मद्विद्यते शुभे}
{अवध्यास्ते नरैरन्यैरपि वा देवदानवैः}


\threelineshloka
{परस्परकृतं नाशं यतः प्राप्स्यन्ति यादवाः}
{इत्युक्तवति दाशार्हे पाण्डवास्त्रस्तचेतसः}
{बभूवुर्भृशसंविग्ना निराशाश्चापि जीविते}


\chapter{अध्यायः २६}
\twolineshloka
{श्रीभगवानुवाच}
{}


\twolineshloka
{उत्तिष्ठोत्तिष्ठ गान्धारि मा च शोके मनः कृथाः}
{तवैव ह्यपराधेन कुरवो निधनं गताः}


\twolineshloka
{यत्त्वं पुत्रं दुरात्मानमीर्षुमत्यन्तमानिनम्}
{दुर्योधनं पुरस्कृत्य दुष्कृतं साधु मन्यसे}


\twolineshloka
{निष्ठुरं वैरपुरुषं वृद्धानां शासनातिगम्}
{कथमात्मकृतं दोषं मय्याधातुमिहेच्छसि}


\twolineshloka
{मृतं वा यदि वा नष्टं योऽतीतमनुशोचति}
{दुःखेन लभते दुःखं द्वावनर्थौ प्रपद्यते}


\threelineshloka
{तपोर्थीयं ब्राह्मणी धत्त गर्भंगौर्वोढारं धावितारं तुरङ्गी}
{शूद्रा दासं पशुपालं च वैश्यावधार्थीयं त्वद्विधा राजपुत्री ॥वैशंपायन उवाच}
{}


\twolineshloka
{तच्छ्रुत्वा वासुदेवस्य पुनरुक्तं वचोऽप्रियम्}
{तूष्णीं बभूव गान्धारी शोकव्याकुललोचना}


\twolineshloka
{धृतराष्ट्रस्तु राजर्षिर्निगृह्याबुद्धिजं तमः}
{पर्यपृच्छत धर्मज्ञो धर्मराजं युधिष्ठिरम्}


\threelineshloka
{जीवतां परिमाणज्ञः सैन्यानामसि पाण्डव}
{हतानां यदि जानीषे परिमाणं वदस्वं मे ॥युधिष्ठिर उवाच}
{}


\twolineshloka
{दशायुतसहस्राणि सहस्राणि च विंशतिः}
{कोट्यः षष्टिश्च षट् चैव ह्यस्मिन्राजन्मृधे हताः}


\threelineshloka
{आलक्षाणां च वीराणां सहस्राणि चतुर्दश}
{दश चान्यानि राजेन्द्र शतं षष्टिश्च भारत ॥धृतराष्ट्र उवाच}
{}


\threelineshloka
{युधिष्ठिरगतिं कां ते गताः पुरुषसत्तम}
{आचक्ष्व मे महाबाहो सर्वज्ञो ह्यसि मे मतः ॥युधिष्ठिर उवाच}
{}


\twolineshloka
{यैर्हुतानि शरीराणि हृष्टैः परमसंयुगे}
{देवराजसमाँल्लोकान्गतास्ते सत्यविक्रमाः}


\twolineshloka
{ये त्वहृष्टेन मनसा मर्तव्यमिति भारत}
{युध्यमाना हताः सङ्ख्ये गन्धर्वैः सह सङ्गताः}


\twolineshloka
{ये च सङ्ग्रामभूमिष्ठा याचमानाः पराङ्मुखाः}
{शस्त्रेण निधनं प्राप्ता गतास्ते गुह्यकान्प्रति}


\twolineshloka
{पात्यमानाः परैर्ये तु हीयमाना निरायुधाः}
{हीनिषेवा महात्मानः परानभिमुखा रणे}


\twolineshloka
{छिद्यमानाः शितैः शस्त्रैः क्षत्रधर्मपरायणाः}
{हतास्ते ब्रह्मसदनं गता वीराः सुवर्चसः}


\threelineshloka
{ये त्वत्र निहता राजन्नन्तरायोधनं प्रति}
{यथाकथञ्चित्पुरुषास्ते गता ह्युत्तरां गतिम् ॥धृतराष्ट्र उवाच}
{}


\threelineshloka
{केन ज्ञानबलेनैव पुत्र पश्यसि सिद्धवत्}
{तन्मे वद महाबाहो श्रोतव्यं यदि वै मया ॥युधिष्ठिर उवाच}
{}


\twolineshloka
{निदेशाद्भवतः पूर्वं वने विचरता मया}
{तीर्थयात्राप्रसङ्गेन सम्प्राप्तोऽयमनुग्रहः}


\threelineshloka
{देवर्षिर्लोमशो दृष्टस्ततः प्राप्तोऽस्म्यनुस्मृतिम्}
{दिव्यं चक्षुरपि प्राप्तं ज्ञानयोगेन वै पुरा ॥धृतराष्ट्र उवाच}
{}


\threelineshloka
{ये त्वनाथा जनास्तात नाथवन्तश्च पाण़्डव}
{कच्चित्तेषां शरीराणि धक्ष्यन्ति विधिपूर्वकम् ॥युधिष्ठिर उवाच}
{}


\twolineshloka
{न येषां शान्तिकर्तारो न च येऽत्राहिताग्नयः}
{वयं च तेषां कुर्यामो बहुत्वात्तात कर्मणाम्}


\threelineshloka
{यान्सुपर्णाश्च गुध्राश्च विकर्षन्ति ततस्ततः}
{तेषां सङ्कर्षणा लोका भविष्यन्ति न संशयः ॥वैशम्पायन उवाच}
{}


\twolineshloka
{एवमुक्त्वा महाराज कुन्तीपुत्रो युधिष्ठिरः}
{आदिदेश सुधर्माणं धौम्यं सूतं च सञ्जयम्}


\twolineshloka
{विदुरं च महाबुद्धिं युयुत्सुं चैव कौरवम्}
{इन्द्रसेनमुखान्भृत्यान्सूतांश्चैव सहस्रशः}


\twolineshloka
{भवन्तः कारयन्त्वेषां प्रेतकार्याण्यशेषतः}
{यथा नाथवतां किञ्चिच्छरीरं न विनश्यति}


\twolineshloka
{शासनाद्धर्मराजस्य क्षत्ता सूतश्च सञ्जयः}
{सुधर्मा धौम्यसहित इन्द्रसेनादयस्तथा}


\twolineshloka
{चन्दनागुरुकाष्ठानि तथा कालीयकान्युत}
{घृतं तैलं च गन्धांश्च क्षौमाणि वसनानि च}


\twolineshloka
{समाहृत्य महार्हाणि दारुणां चैव सञ्चयान्}
{रथांश्च मृदितास्तत्र नानाप्रहरणानि च}


\twolineshloka
{चिताः कृत्वा प्रयत्नेन यथामुख्यान्नराधिपान्}
{दाहयामासुरव्याग्राः शास्त्रदृष्टेन कर्मणा}


% Check verse!
दुर्योधनं च राजानं भ्रातॄंश्चास्य कर्मणा ॥शल्यं शलं च राजानं भूरिश्रवसमेव च
\twolineshloka
{जयद्रथं च राजानमभिमन्युं च भारत}
{दौःशासनिं लक्ष्मणं च धृष्टकेतुं च पार्थिवम्}


\twolineshloka
{बृहद्वलं सोमदत्तं सृञ्जयं च महारथम्}
{राजानं क्षेमधन्वानं विराटद्रुपदौ तथा}


\twolineshloka
{शिखण्डिनं च पाञ्चाल्यं धृष्टद्युम्नं च पार्षतम्}
{युधामन्युं च विक्रान्तमुत्तमौजसमेव च}


\twolineshloka
{कौसल्यं द्रौपदेयांश्च शकुनिं चापि सौबलम्}
{अचलं वृषकं चैव भगदत्ते च पार्थिवम्}


\twolineshloka
{कर्णं वैकर्तनं चैव सहपुत्रममर्षणम्}
{केकयांश्च महेष्वासांस्त्रिगर्तांश्च महारथान्}


\twolineshloka
{घटोत्कचं राक्षसेन्द्रं बकभ्रातरमेव च}
{अलम्बुसं राक्षसेन्द्रं जलसन्धं च पार्थिवम्}


\threelineshloka
{एतांश्चान्यांस्च सुबहून्पार्थिवांश्च सहस्रशः}
{धृतधाराहुतैर्दीप्तैः पावकैः समदाहयन्}
{}


\twolineshloka
{पितृमेधाश्च केषाञ्चित्प्रावर्तन्त महात्मनाम्}
{सामभिश्चाप्यगायन्त तेऽन्वशासत चापरैः}


\twolineshloka
{साम्नामृचां च नादेन स्त्रीणां च रुदितस्वनैः}
{कश्मलं सर्वभूतानां निशायां समपद्यत}


\twolineshloka
{हुताश्च तत्प्रदीप्ताश्च दीप्यमानाश्च पावकाः}
{नभसीवान्वदृश्यन्त ग्रहास्तन्वभ्रसंवृताः}


\twolineshloka
{ये चाप्यनाथास्तत्रासन्नानादेशसमागताः}
{तांश्च सर्वान्समानीय राशीकृत्वा सहस्रशः}


\twolineshloka
{चित्वा दारुभिरव्यग्रैः प्रभूतैः स्नेहपाचितैः}
{दाहयामास तान्सर्वान्विदुरो राजशासनात्}


\twolineshloka
{कारयित्वा क्रियास्तेषां कुरुराजो युधिष्ठिरः}
{धृतराष्ट्रं पुरस्कृत्य गङ्गामभिमुखोऽगमत्}


\chapter{अध्यायः २७}
\threelineshloka
{वैशम्पायन उवाच}
{कृत्वोदकं ते सुहृदां सर्वेषां पाण्डुनन्दनाः}
{विदुरे धृतराष्ट्रश्च सर्वाश्च भरतस्त्रियः}


