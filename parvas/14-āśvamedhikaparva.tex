\part{आश्वमॆधिकपर्व}
\chapter{अध्यायः १}
\threelineshloka
{नारायणं नमस्कृत्य नरं चैव नरोत्तमम्}
{देवीं सरस्वतीं व्यासं ततो जयमुदीरयेत् ॥वैशम्पायन उवाच}
{}


\twolineshloka
{कृतोदकस्तु राजानं धृतराष्ट्रं युधिष्ठिरः}
{पुरस्कृत्य महाबाहुरुत्तताराकुलेन्द्रियः}


\twolineshloka
{उत्तीर्य तु महाबाहुर्बाष्पव्याकुललोचनः}
{पपात तीरे गङ्गाया व्याधविद्ध इव द्विपः}


\twolineshloka
{तं सीदमानं जग्राह भीमः कृष्णेन चोदितः}
{मैवमित्यब्रवीच्चैनं कृष्णः परबलार्दनः}


\twolineshloka
{तमार्तं पतितं भूमौ श्वसन्तं च पुनः पुनः}
{तदृशुः पार्थिवा राजन्धर्मपुत्रं युधिष्ठिरम्}


\twolineshloka
{त दृष्ट्वा दीनमनसं गतसत्वं नरेश्वरम्}
{भूयः शोकसमाविष्टाः पाण्डवाःक समुपाविशन्}


\twolineshloka
{राजा तु धृतराष्ट्रस्तं तथा दीनो महाभुजम्}
{वाक्यमाह महाबुद्धिः प्रज्ञाचक्षुर्नरेश्वरम्}


\twolineshloka
{उत्तिष्ठ कुरुशार्दूल कुरु कार्यमनन्तरम्}
{क्षत्रधर्मेण कौन्तेय जितेयमवनी त्वया}


\twolineshloka
{भुङ्क्ष सार्धं भ्रातृभिस्तां सुहृद्भिश्च जनेश्वर}
{शोचितव्यं न पश्यामि त्वया धर्मभृतांवर}


\twolineshloka
{शोचितव्यं मया चैव गान्धार्या च महीपते}
{ययोः पुत्रशतं नष्टं स्वप्नलब्धं यथा धनम्}


\twolineshloka
{अश्रुत्वा हितकामस्य विदुरस्य महात्मनः}
{वाक्यानि सुमहार्थानि परितप्यामि दुर्मतिः}


\twolineshloka
{उक्तवान्विदुरो यन्मां धर्मात्मा दिव्यदर्शनः}
{दुर्योधनापराधेन कुलं ते विनशिष्यति}


\twolineshloka
{स्वस्ति चेदिच्छसे राजन्कुलस्य कुरु मे वचः}
{वध्यतामेष दुष्टात्मा मन्दो राजा सुयोधनः}


\twolineshloka
{कर्णश्च शकुनिश्चैव नैनं पश्यतु कर्हिचित्}
{द्यूतसङ्घातमप्येषामप्रमादेन वारय}


\twolineshloka
{अभिषेचय राजानं धर्मात्मानं युधिष्ठिरम्}
{स पालयिष्यति वशी धर्मेण पृथिवीमिमाम्}


\threelineshloka
{अथ नेच्छसि राजानं कुन्तीपुत्रं युधिष्ठिरम्}
{`विनाशमुपयास्तन्ति तव पुत्रा न संशयः}
{'मेढीभूतः स्वयं राज्यं प्रतिगृह्णीष्व पार्थिव}


\twolineshloka
{समं सर्वेषु भूतेषु वर्तमानं नराधिप}
{अनुजीवन्तु सर्वे त्वां ज्ञातयो ज्ञातिवर्धंन}


\twolineshloka
{एवं ब्रुवति कौन्तेय विदुरे दीर्घदर्शिनि}
{दुर्योधनमहं पापमन्ववर्तं वृथामतिः}


\twolineshloka
{अश्रुत्वा तस्य धीरस्य वाक्यानि मधुराण्यहम्}
{फलं प्राप्य महद्दुःखं निमग्नः शोकसागरे}


\twolineshloka
{वृद्धौ हि तेऽद्य पितरौ पश्य नौ दुःखितौ नृप}
{न शोचितव्यं भवता पश्यामीह जनाधिप}


\chapter{अध्यायः २}
\twolineshloka
{एवमुक्तस्तु राजाऽसौ धृतराष्ट्रेण धीमता}
{तूष्णींबभूव मेधावी तमुवाचाथ केशवः}


\twolineshloka
{अतीव मनसा शोकः क्रियमाणो जनाधिप}
{सन्तपयति चैतस्य पूर्वप्रेतान्पितामहान्}


\twolineshloka
{यजस्व विविधैर्यज्ञैर्बहुभिः स्वाप्तदक्षिणैः}
{देवांस्तर्पय सोमेन स्वधया च पितॄनपि}


\threelineshloka
{अतिथीनन्नपानेन कामैरन्यैरकिञ्चनान्}
{`त्वद्विधस्य महाबुद्धे नैतदद्योपपद्यते}
{'विदितं वेदितव्यं ते कर्तव्यमपि ते कृतम्}


\twolineshloka
{श्रुताश्च राजधर्मास्ते भीष्माद्भागीरथीसुतात्}
{कृष्णद्वैपायनाच्चैव नारदाद्विदुरात्तथा}


\twolineshloka
{नेमामर्हसि मूढानां वृत्तिं त्वमनुवर्तितुम्}
{पितृपैतामहं वृत्तमास्थाय धुरमुद्वह}


\twolineshloka
{युक्तं हि यशसा क्षात्रं स्वर्गं प्राप्तुमसंशयम्}
{नहि कश्चिद्धि शूरणां निहतोऽत्र पराङ्मुखः}


\twolineshloka
{त्यज शोकं महाराज भवितव्यं हि तत्तथा}
{न शक्यास्ते पुनर्द्रष्टुं त्वया येऽस्मिन्रणे हताः}


\twolineshloka
{एतावदुक्त्वा गोविन्दो धर्मराजं युधिष्ठिरम्}
{विरराम महातेजास्तमुवाच युधिष्ठिरः}


\twolineshloka
{गोविन्द मयि या प्रीतिस्तव सा विदिता मम}
{सौहृदेन तथा प्रेम्णा सदा मय्यनुकम्पसे}


\twolineshloka
{प्रियं तु मे स्यात्सुमहत्कृतं चक्रगदाधर}
{श्रीमन्प्रीतेन मनसाक सर्वं यादवनन्दन}


\twolineshloka
{यदि मामनुजानीयाद्भवान्गन्तुं तपोवनम्}
{`कृतकृत्यो भविष्यामि इति मे निश्चिता मतिः'}


\threelineshloka
{न हि शान्तिं प्रपश्यामि पातयित्वा पितामहम्}
{`नृशंसः पुरुषव्याघ्रं गुरुं वीर्यबलान्वितम्}
{'कर्णं च पुरुषव्याघ्नं सङ्ग्रामेष्वपलायिनम्}


\twolineshloka
{कर्मणा येन मुच्येयमस्मात्क्रूरादरिंदम}
{कर्मणा तद्विधत्स्वेह येन शुध्यति मे मनः}


\twolineshloka
{तमेवंवादिनं पार्थं व्यासः प्रोवाच धर्मवित्}
{सान्त्वयन्सुमहातेजाः शुभं वचनमर्थवत्}


\twolineshloka
{सुकृता ते मतिस्तात पुनर्बाल्येन मुह्यसे}
{किमाकाशे वयं सर्वे प्रलपामो मुहुर्मुहुः}


\twolineshloka
{विदिताः क्षत्रधर्मास्ते येषां युद्धेन जीविका}
{तथा प्रवृत्तो नृपतिर्नाधिबन्धेन युज्यसे}


\twolineshloka
{मोक्षधर्माश्च निखिला याथातथ्येन ते श्रुताः}
{`यथा वै कामजां मायां परित्युक्तं त्वमर्हसि ॥'}


\twolineshloka
{असकृच्चापि सन्देहाश्छिन्नास्ते कामजा मया}
{अश्रद्दधानो दुर्मेधा लुप्तस्मृतिरसि ध्रुवम्}


% Check verse!
मैवं भव न ते युक्तमिदमज्ञानमीदृशम्
\twolineshloka
{प्रायश्चित्तानि सर्वाणि विदितानि च तेऽनघ}
{राजधर्माश्च ते सर्वे दानधर्माश्च ते श्रुताः}


\twolineshloka
{स कथं सर्वधर्मज्ञः सर्वागमविशारदः}
{परिमुह्यसि भूयस्त्वमज्ञानादिव भारत}


\chapter{अध्यायः ३}
\twolineshloka
{युधिष्ठिर तव प्रज्ञा न सम्यगिति मे मतिः}
{न हि कश्चित्स्वयं मर्त्यः स्ववशः कुरुते क्रियाम्}


\twolineshloka
{ईश्वरेण च युक्तोऽयं साध्वसाधु च मानवः}
{करोत्यसुकरं कर्म तत्रका परिदेवना}


\twolineshloka
{आत्मानं मन्यसे चाथ पापकर्माणमन्ततः}
{शृणु तत्र यथा पापमपकृत्येत भारत}


\twolineshloka
{तपोभिः क्रतुभिश्चैव दानेन च युधिष्ठिर}
{तरन्ति नित्यं पुरुषा ये स्म पापानि कुर्वते}


\twolineshloka
{यज्ञेन तपसा चैव दानेन च नराधिप}
{पूयन्ते नरशार्दूल नरा दुष्कृतकारिणः}


\twolineshloka
{असुराश्च सुराश्चैव पुण्यहेतोर्मखक्रियाम्}
{प्रवर्तन्ते महात्मानस्तस्माद्यज्ञः परायणम्}


\twolineshloka
{यज्ञैरेव महात्मानो बभूवुरधिकाः सुराः}
{ततो देवाः क्रियावन्तो दानवानभ्यधर्षयन्}


\twolineshloka
{राजसूयाश्वमेधौ च सर्वमेधं च भारत}
{नरमेधं च नृपते त्वमाहर युधिष्ठिर}


\twolineshloka
{यजस्व वाजिमेधेन विधिवद्दक्षिणावता}
{बहुकामान्नवित्तेन रामो दाशरथिर्यथा}


\threelineshloka
{यथा च भरतो राजा दौष्यन्तिः पृथिवीपतिः}
{शाकुन्तलो महावीर्यस्तव पूर्वपितामहः ॥युधिष्ठिर उवाच}
{}


\twolineshloka
{असंशयं वाजिमेधः पारयेत्पृथिवीमपि}
{अभिप्रायस्तु मे कश्चित्तं त्वं श्रोतुमिहार्हसि}


\threelineshloka
{इमं ज्ञातिवधं कृत्वा सुमहान्तं द्विजोत्तम}
{`अहमाराधयिष्यामि कथं शोकपरायणः}
{'दानमल्पं न शक्नोमि दातुं वित्तं च नास्ति मे}


\twolineshloka
{न तु बालानिमान्दीनानुत्सहे वसु याचितुम्}
{तथैवाद्रविणान्कृच्छ्रे वर्तमानान्नृपात्मजान्}


\twolineshloka
{स्वयं विनाश्य पृथिवीं यज्ञार्थं द्विजसत्तम}
{करमाहारयिष्यामि कथं शोकपरायणः}


\twolineshloka
{दुर्योधनापराधेनि वसुधायां नराधिपाः}
{प्रनष्टा योजयित्वाऽस्मानकीर्त्या मुनिसत्तम}


\twolineshloka
{दुर्योधनेन पृथिवी क्षपिता जयकारणात्}
{कोशश्चापि विशीर्णोसौ धार्तराष्ट्रस्य दुर्मतेः}


\twolineshloka
{पृथिवी दक्षिणा चात्र वाजिमेधे महाक्रतौ}
{विद्वद्भिः परिदृष्टोऽयं शिष्टो विधिविपर्ययः}


\twolineshloka
{न च प्रतिनिधिं कर्तुं चिकीर्षामि तपोधन}
{अत्र मे भगवन्सम्यक्साचिव्यं कर्तुमर्हसि}


\twolineshloka
{एवमुक्तस्तु पार्थेन कृष्णद्वैपायनस्तदा}
{मुहूर्तमनुसञ्चिन्त्य धर्मराजानमब्रवीत्}


\twolineshloka
{कोशश्चापि विशीर्णोऽयं परिपूर्णो भविष्यति}
{विद्यते द्रविणं पार्थ गिरौ हिमवति स्थितम्}


\threelineshloka
{उत्सृष्टं ब्राह्मणैर्यज्ञे मरुत्तस्य महीपते}
{तदानयस्व कौन्तेय पर्याप्तं तद्भविष्यति ॥युधिष्ठिर उवाच}
{}


\threelineshloka
{कथं यज्ञे मरुत्तस्य द्रविणं तत्समाचितम्}
{कस्मिंश्च काले स नृपो बभूव ददतांवर ॥व्यास उवाच}
{}


\twolineshloka
{यदि शुश्रूषसे पार्थ शृणु कारंधमं नृपम्}
{यस्मिन्काले महावीर्यः स राजाऽऽसीन्महाधनः}


\chapter{अध्यायः ४}
\threelineshloka
{शुश्रूषे तस्य धर्मज्ञ राजर्षेः परिकीर्तनम्}
{द्वैपायन मरुत्तस्य कथां प्रब्रूहि मेऽनघ ॥व्यास उवाच}
{}


\twolineshloka
{आसीत्कृतयुगे तात मनुर्दण्डिधरः प्रभुः}
{तस्य पुत्रो महोष्वासः प्रजातिरभवन्नृपः}


\twolineshloka
{प्रजातेरभवत्पुत्रः क्षुत इत्यभिविश्रुतः}
{क्षुतस्य पुत्र इक्ष्वाकुर्महीपालोऽभवत्प्रभुः}


\twolineshloka
{तस्य पुत्रशतं राजन्नासीत्परमधार्मिकम्}
{तांस्तु सर्वान्महीपालानिक्ष्वाकुरकरोत्प्रभुः}


\twolineshloka
{तेषां ज्येष्ठस्तु विंशोऽभूत्प्रतिमानं धनुष्मताम्}
{विंशस्य पुत्रः कल्याणो विविंशो नाम भारत}


\twolineshloka
{विविंशस्य सुता राजन्बभूवुर्दश पञ्च च}
{सर्वे धनुषि विक्रान्ता ब्रह्मण्याः सत्यवादिनः}


\twolineshloka
{दानधर्मरताः शान्ताः सततं प्रियवादिनः}
{तेषां ज्येष्ठः खनीनेत्रः स तान्सर्वानपीडयत्}


\twolineshloka
{खनीनेत्रस्तु विक्रान्तो जित्वा राज्यमकण्टकम्}
{नाशकद्रक्षितुं राज्यं नान्वरज्यन्त तं प्रजाः}


\twolineshloka
{तमपास्य च तद्राज्ये तस्य पुत्रं सुवर्चसम्}
{अभ्यषिञ्चन्त राजेन्द्र मुदिता ह्यभवंस्तदा}


\twolineshloka
{स पितुर्विक्रियां दृष्ट्वा राज्यान्निरसनं च तत्}
{नियतो वर्तयामास प्रजाहितचिकीर्षया}


\twolineshloka
{ब्रह्मण्यः सत्यवादी च शुचिः शमदमान्वितः}
{प्रजास्तं चान्वरज्यन्त धर्मनित्यं मनस्विनम्}


\twolineshloka
{तस्य धर्मप्रवृत्तस्य व्यशीर्यत्कोशवाहनम्}
{तं क्षीणकोशं सामन्ताः समन्तात्पर्यपीडयन्}


\twolineshloka
{स पीड्यमानो बहुभिः क्षीणकोशाश्ववाहनः}
{आर्तिमार्च्छत्परां राजा सह भृत्यैः पुरेण च}


\twolineshloka
{न चैनमभिहन्तुं ते शक्नुवन्ति बलक्षये}
{सम्यग्वृत्तो हि राजा स धर्मनित्यो युधिष्ठिर}


\twolineshloka
{यदा तु परमामार्तिं गतोऽसौ सपुरो नृपः}
{ततः प्रदध्मौ स करं प्रादुरासीत्ततो बलम्}


\twolineshloka
{ततस्तानजयत्सर्वान्प्रातिसीमान्नराधिपान्}
{एतस्मात्कारणाद्राजन्विश्रुतः स करंधमः}


\twolineshloka
{आवीक्षित्तस्य पुत्रोऽभूत्त्रेतायुगमुखे पुरा}
{इन्द्रादनवरः श्रीमान्देवैरपि सुदुर्जयः}


\threelineshloka
{`कारंधम इति ख्यातो बभूव जगतीपतिः}
{'तस्य सर्वे महीपाला वर्तन्ते स्म वशे तदा}
{स हि सम्राडभूत्तेषां वृत्तेन च बलेन च}


\twolineshloka
{अविक्षिन्नाम धर्मात्मा शौर्येणेन्द्रसमोऽभवत्}
{यज्ञशीलो धर्मरतिर्धृतिमान्संयतेन्द्रियः}


\twolineshloka
{तेजसाऽऽदित्यसदृशः क्षमया पृथिवीसमः}
{बृहस्पतिसमो बुद्ध्या हिमवानिव सुस्थिरः}


\twolineshloka
{कर्मणा मनसा वाचा दमेन प्रशमेन च}
{मनांस्याराधयामास प्रजानां स महीपतिः}


\twolineshloka
{य ईजे हयमेधानां शतेन विधिवत्प्रभुः}
{याजयामास यं विद्वान्स्वयमेवाङ्गिराः प्रभुः}


\threelineshloka
{तस्य पुत्रोऽतिचक्राम पितरं गुणवत्तया}
{मरुत्तो नाम धर्मज्ञश्चक्रवर्ती महायशाः}
{नागायुतसमप्राणः साक्षाद्विष्णुरिवापरः}


\twolineshloka
{स यक्ष्यमाणो धर्मात्मा शातकुम्भमयान्युत}
{कारयामास शुभ्राणि भाजनानि सहस्रशः}


\twolineshloka
{मेरुं पर्वतमासाद्य हिमवत्पार्श्व उत्तरे}
{काञ्चनः सुमहान्पादस्तत्र कर्म चकार सः}


\twolineshloka
{ततः कुण्डानि पात्रीश्च पिठराण्यासंनानि च}
{चक्रुः सुवर्णकर्तारो येषां सङ्ख्या न विद्यते}


\threelineshloka
{तस्यैव च समीपे तु यज्ञवाटो बभूव ह}
{ईजे तत्र स धर्मात्मा विदिवत्पृथिवीपतिः}
{मरुत्तः सहितैः सर्वैः प्रजापालैर्नराधिपः}


\chapter{अध्यायः ५}
\twolineshloka
{पथंवीर्यः समभवत्स राजा ददतांवरः}
{कथं च जातरूपेण समयुज्यत वै नृपः}


\threelineshloka
{क्व च तत्सांप्रतं द्रव्यं भगवन्नवतिष्ठते}
{कथं च शक्यमस्माभिस्तदवाप्नुं तचपोधन ॥व्यास उवाच}
{}


\twolineshloka
{असुराश्चैव देवाश्च दक्षस्यासन्प्रजापतेः}
{अपत्यं बहुलं तात तेऽस्पर्धन्त परस्परम्}


\twolineshloka
{तथैवाङ्गिरसः पुत्रौ पितृतुल्यौ बभूवतुः}
{बृहस्पतिर्बृहत्तेजाः संवर्तश्च तपोधनः}


\twolineshloka
{तावति स्पर्धिनौ राजन्पृथगास्तां परस्परम्}
{बृहस्पतिः स संवर्तं बाधते स्म पुनःपुनः}


\twolineshloka
{स बाध्यमानः सततं भ्रात्रा ज्येष्ठेन भारत}
{अर्थानुत्सृज्य दिग्वासा वने वासमरोचयत्}


\twolineshloka
{वासवोऽप्यसुरान्सर्वान्विजित्य च निपात्य च}
{इन्द्रत्वं प्राप्य लोकेषु ततो वव्रे पुरोहितम्}


\twolineshloka
{पुत्रमङ्गिरसो ज्येष्ठं विप्रज्येष्ठं बृहस्पतिम्}
{याज्यस्त्वङ्गिरसः पूर्वमासीद्राजा करंधमः}


\threelineshloka
{वीर्येणाप्रतिमो लोके वृत्तेन च बलेन च}
{शतक्रतुरिवौजस्वी धर्मात्मा संशितव्रतः}
{}


\twolineshloka
{वाहनं यस्य योधाश्च मित्राणि विविधानि च}
{शयनानि च मुख्यानि महार्हाणि च सर्वशः}


\twolineshloka
{ध्यानादेवाभवद्राजन्मुखवातेन सर्वशः}
{स गुणैः पार्थिवान्सर्वान्वशे चक्रे नराधिपः}


\twolineshloka
{संजीव्य कालमिष्टं च सशरीरो दिवं गतः}
{बभूव तस्य पुत्रस्तु ययातिरिव धर्मवित्}


\twolineshloka
{अविक्षिन्नाम शत्रुंजित्स वशे कृतवान्महीम्}
{विक्रमेण गुणैश्चैव पितेवासीत्स पार्थिवः}


\twolineshloka
{तस्य वासवतुल्योऽभून्मरुत्तो नाम वीर्यवान्}
{पुत्रस्तमनुरक्ताऽभूत्पृथिवी सागराम्बरा}


\twolineshloka
{स्पर्धते स स्म सततं देवराजेन नित्यदा}
{वासवोऽपि मरुत्तेनि स्पर्धते पाण्डुनन्दन}


\twolineshloka
{शुचिः स गुणवानासीन्मरुत्तः पृथिवीपतिः}
{यतमानोपि यं शक्रो न विशेषयति स्म ह}


\twolineshloka
{सोऽशक्नुवन्विशेषाय समाहूय बृहस्पतिम्}
{उवाचेदं वचो देवैः सहितो हरिवाहनः}


\twolineshloka
{बृहस्पते मरुत्तस्य मा स्म कार्षीः कथञ्चन}
{दैवं कर्माथ पित्र्यं वा कर्तासि मम चेत्प्रियम्}


\twolineshloka
{अहं हि त्रिषु लोकेषु सुराणां च बृहस्पते}
{इन्द्रत्वं प्राप्तवानेको मरुत्तस्तु महीपतिः}


\twolineshloka
{कथं ह्यमर्त्यं ब्रह्मंस्त्वं याजयित्वा सुराधिपम्}
{याजयेर्मृत्युसंयुक्तं मरुत्तमविशङ्कया}


\twolineshloka
{मां वा वृणीष्व भद्रं ते मरुत्तं वा महीपतिम्}
{परित्यज्य मरुत्तं वा यथाजोषं भजस्व माम्}


\twolineshloka
{एवमुक्तः स करकव्य देवराज्ञा बृहस्पतिः}
{मुहूर्तमिव सञ्चिन्त्य देवराजानमब्रवीत्}


\twolineshloka
{त्वं भूतानामधिपतिस्त्वयि लोकाः प्रतिष्ठिताः}
{नमुचेर्विश्वरूपस्य निहन्ता त्वं बलस्य च}


\twolineshloka
{त्वमाजहर्थ् देवानामेको वीरश्रियं पराम्}
{त्वं बिभर्षि भुवं द्यां च सदैव बलसूदन}


\twolineshloka
{परोहित्यं कथं कृत्वा तव देवगणेश्वर}
{याजयेयमहं मर्त्यं मरुत्तं पाकशासन}


\twolineshloka
{समाश्वसिहि देवेन्द्र नाहं मर्त्यस्य कर्हिचित्}
{ग्रहीष्यामि स्रुवं यज्ञे शृणु चेदं वचो मम}


\threelineshloka
{हिरण्यरेता नोष्णः स्यात्परिवर्तेत मेदिनी}
{भासं तु न रविः कुर्यान्न तु सत्यं चलेन्मयि ॥व्यास उवाच}
{}


\twolineshloka
{बृहस्पतिवचःक श्रुत्वा शक्रो विगतमत्सरः}
{प्रशस्यैनं विवेशाथ स्वमेव भवनं तदा}


\chapter{अध्यायः ६}
\twolineshloka
{अत्राप्युदाहरन्तीमममितिहासं पुरातनम्}
{बृहस्पतेश्च संवादं मरुत्तस्य च धीमतः}


\twolineshloka
{देवराजस्य समयं कृतमाङ्गिरसेन ह}
{श्रुत्वा मरुत्तो नृपतिर्मन्युमाहारयत्परम्}


\twolineshloka
{सङ्कल्प्य मनसा यज्ञं करंधमसुतात्मजः}
{बृहस्पतिमुपागम्य वाग्मी वचनमब्रवीत्}


\twolineshloka
{भगवन्यन्मया पूर्वमभिगम्य तपोधन}
{कृतोऽभिसन्धिर्यज्ञस्य भवतो वचनाद्गुरो}


\threelineshloka
{तमहं यष्टुमिच्छामि सम्भाराः सम्भृताश्च मे}
{याज्योस्मि भवतः साधो तत्प्राप्नुहि विधत्स्व च ॥बृहस्पतिरुवाच}
{}


\threelineshloka
{न कामये याजयितुं त्वामहं पृथिवीपते}
{वृतोस्मि देवराजेन प्रतिज्ञातं च तस्य मे ॥मरुत्त उवाच}
{}


\threelineshloka
{पित्र्यमस्मि तव क्षेत्रं बहुमन्ये च ते भृशम्}
{तवास्मि याज्यातां प्राप्तो भजमानं भजस्व माम् ॥बृहस्पतिरुवाच}
{}


\twolineshloka
{अमर्त्यं याजयित्वाऽहं याजयिष्ये कथं नरम्}
{मरुत्त गच्छ वा मा वा निवृत्तोस्म्यद्य याजनात्}


\threelineshloka
{न त्वां याजयितास्म्यद्य वृणु यं त्वमिहेच्छसि}
{उपाध्यायं महाबाहो यस्ते यज्ञं करिष्यति ॥व्यास उवाच}
{}


\twolineshloka
{एवमुक्तस्तु नृपतिर्मरुत्तो व्रीडितोऽभवत्}
{प्रत्यागच्छन्सुसंविग्नो ददर्श पथि नारदम्}


\twolineshloka
{देवर्षिणा समागम्य नारदेनि स पार्थिवः}
{विधिवत्प्राञ्जलिस्तस्थावथैनं नारदोऽब्रवीत्}


\twolineshloka
{राजर्षे नातिहृष्टोसि कच्चित्क्षेमं तवानघ}
{क्व गतोसि कुतश्चेदमप्रीतिस्थानमागतम्}


\twolineshloka
{श्रोतव्यं चेन्मया राजन्ब्रूहि मे पार्थिवर्षभ}
{व्यपनेष्यामि ते मन्युं सर्वयत्नैर्नराधिप}


\threelineshloka
{एवमुक्तो मरुत्तः स नारदेन महर्षिणा}
{विप्रलम्भमुपाध्यायात्सर्वज्ञे तं न्यवेदयत् ॥मरुत्त उवाच}
{}


\twolineshloka
{गतोस्म्यङ्गिरसः पुत्रं देवाचार्यं बृहस्पतिम्}
{यज्ञार्थमृत्विजं प्रष्टुं स च मां नाभ्यनन्दत}


\threelineshloka
{प्रत्याख्यातश्च तेनाहं जीवितुं नाद्य कामये}
{परित्यक्तश्च गुरुणा दूषितश्चास्मि नारद ॥व्यास उवाच}
{}


\twolineshloka
{एवमुक्तस्तु राज्ञा स नारदः प्रत्युवाच ह}
{आविक्षितं महाराज वाचा संजीवयन्निव}


\twolineshloka
{राजन्नङ्गिरसः पुत्रः संवर्तो नाम धार्मिकः}
{चङ्कमीति दिशः सर्वा दिग्वासा मोहयन्प्रजाः}


\threelineshloka
{तं गच्च यदि याज्यं त्वां न वाञ्छति बृहस्पतिः}
{प्रसन्नस्त्वां महातेजाः संवर्तो याजयिष्यति ॥मरुत्त उवाच}
{}


\twolineshloka
{संजीवितोऽहं भवता वाक्येनानेन नारद}
{पश्येयं क्व नु संवर्तं शंस मे वदतांवर}


\threelineshloka
{कथं च तस्मै वर्तेयं कथं मां न परित्यजेत्}
{प्रत्याख्यातश्च तेनापि नाहं जीवितुमुत्सहे ॥नारद उवाच}
{}


\twolineshloka
{उन्मत्तवेषं बिभ्रत्स चङ्क्रमीति यथासुखम्}
{वाराणसीं तु नगरीमभीक्ष्णमुपसेवते}


\twolineshloka
{तस्या द्वारं समासाद्य न्यसेथाः कुणपं क्वचित्}
{तं दृष्ट्वा यो निवर्तेत संवर्तः स महीपते}


\twolineshloka
{तं पृष्ठतोऽनुगच्छेथा यत्र गच्छेत्स वीर्यवान्}
{तमेकान्ते समासाद्य प्राञ्जलिः शरणं व्रजेः}


\twolineshloka
{पृच्छेत्त्वां यदि केनाहं तवाख्यात इति स्म ह}
{ब्रूयास्त्वं नारदेनेति स कुत्र इति शत्रुहन्}


\threelineshloka
{स चेत्त्वामनुयुञ्जीत ममानुगमनेप्सया}
{शंसेथा वह्निमारूढं मामपि त्वमशङ्कया ॥व्यास उवाच}
{}


\twolineshloka
{स तथेति प्रतिश्रुत्य पूजयित्वा च नारदम्}
{अभ्यनुज्ञाय राजर्षिर्ययौ वाराणसीं पुरीम्}


\twolineshloka
{तत्र गत्वा यथोक्तं स पुर्या द्वारे महायशाः}
{कुणपं स्थापयामास नारदस्य वचः स्मरन्}


\twolineshloka
{यौगपद्येनि विप्रश्च पुरीद्वारमथाविशत्}
{ततः स कुणपं दृष्ट्वा सहसा संन्यवर्तत}


\twolineshloka
{स तं निवृत्तमालक्ष्य प्राञ्जलिः पृष्ठतोऽन्वगात्}
{आविक्षितो महीपालः संवर्तमुपशिक्षितुम्}


\twolineshloka
{स च तं विजने दृष्ट्वा पांसुभिः कदेमेन च}
{श्लेष्मणा चैव राजानं ष्ठीवनैश्च समाकिरत्}


\twolineshloka
{स तथा बाध्यमानो वै संवर्तेन महीपतिः}
{अन्वगादेव तमृषिं प्राञ्जलिः सम्प्रसादयन्}


\twolineshloka
{ततो निवर्त्य संवर्तः परिश्रान्त उपाविशत्}
{शीतलच्छायमासाद्य न्यग्रोधं बहुशाखिनम्}


\chapter{अध्यायः ७}
\twolineshloka
{कथमस्मि त्वया ज्ञातः केन वा कथितोस्मि ते}
{एतदाचक्ष्व मे तत्त्वमिच्छसे चेन्मम प्रियम्}


\threelineshloka
{सत्यं ते ब्रुवतः सर्वे सम्पत्स्यन्ते मनोरथाः}
{मिथ्या च ब्रुवतो मूर्धा शतधा ते स्फुटिष्यति ॥मरुत्त उवाच}
{}


\threelineshloka
{नारदेन भवान्मह्यमाख्यातो ह्यटता पथि}
{गुरुपुत्रो ममेति त्वं ततो मे प्रीतिरुत्तमा ॥संवर्त उवाच}
{}


\threelineshloka
{सत्यमेतद्भवानाह स मां जानाति सत्रिणम्}
{कथयस्व तदेतन्मे क्वनु सम्प्रति नारदः ॥मरुत उवाच}
{}


\threelineshloka
{भवन्तं कथयित्वा तु मम देवर्षिसत्तमः}
{ततो मामभ्यनुज्ञाय प्रविष्टो हव्यवाहनम् ॥व्यास उवाच}
{}


\twolineshloka
{श्रुत्वा तु पार्थिवस्यैत्संवर्तः प्रमुदं गतः}
{एतावदहमप्येवं शक्नुयामिति सोऽब्रवीत्}


\twolineshloka
{ततो मरुत्तमुन्मत्तो वाचा निर्भर्त्सयन्निव}
{रूक्षया ब्राह्मणि राजन्पुनः पुनरथाब्रवीत्}


\twolineshloka
{वातप्रधानेन मया स्वचित्तवशवर्तिना}
{एवं विकृतरूपेण कथं याजितुमिच्छसि}


\twolineshloka
{भ्राता मम समर्थश्च वासवेन च सङ्गतः}
{वर्तते याजने चैव तेन कर्माणि कारय}


\twolineshloka
{गार्हस्थ्यं चैव याज्याश्च सर्वा गृह्याश्च देवताः}
{पूर्वजेन ममाक्षिप्तं शरीरं वर्जितं त्विदम्}


\twolineshloka
{नाहं तेनाननुज्ञातस्त्वामाविक्षित कर्हिचित्}
{याजयेयं कथंचिद्वै स हि पूज्यतमो मम}


\threelineshloka
{स त्वं बृहस्पतिं गच्च तमनुज्ञाप्य चाव्रज}
{ततोऽहं याजयिष्ये त्वां यदि यष्टुमिहेच्छसि ॥मरुत्त उवाच}
{}


\twolineshloka
{बृहस्पतिं गतः पूर्वमहं संवर्ते तच्छृणु}
{न मां कामयते याज्यं मुनिर्वासववारितः}


\twolineshloka
{अमरं याज्यमासाद्य याजयिष्ये न मानुषम्}
{शक्रेण प्रतिषिद्धोऽहं मरुत्तं मा स्म याजये}


\twolineshloka
{स्पर्धते हि मया विप्र सदा हि स तु पार्थिवः}
{एवमस्त्विति चाप्युक्तो भ्रात्रा ते बलसूदनः}


\twolineshloka
{स मामधिगतं प्रेम्णा याज्यत्वे न बुभूषति}
{देवराजं समाश्रित्य तद्विद्धि मुनिपुङ्गव}


\twolineshloka
{सोहमिच्छामि भवता सर्वस्वेनापि याजितुम्}
{कामये समतिक्रान्तुं वासवं त्वत्कृतैर्गुणैः}


\threelineshloka
{न हि मे वर्तते बुद्धिर्गन्तुं ब्रह्मन्बृहस्पतिम्}
{प्रत्याख्यातो हि तेनास्मि तथाऽनपकृते सति ॥संवर्त उवाच}
{}


\twolineshloka
{चिकीर्षसि यथाकामं सर्वमेतत्त्वयि ध्रुवम्}
{यदि सर्वानभिप्रायान्कर्तासि मम पार्थिव}


\twolineshloka
{याज्यमानं मया हि त्वां बृहस्पतिपुंरदरौ}
{द्विषेतां समभिक्रुद्धावेतदेकं समर्थये}


\threelineshloka
{स्थैर्यमत्र कथं मे स्यात्स त्वं निःसंशयं कुरु}
{कुपितस्त्वां न हीदानीं भस्म कुर्या सवान्धवम् ॥मरुत्त उवाच}
{}


\twolineshloka
{यावत्तपेत्सहस्रांशुस्तिष्ठेरंश्चापि पर्वताः}
{तावल्लोकान्न लभेयं त्यजेयं सङ्गतं यदि}


\threelineshloka
{मा चापि शुभबुद्धित्वं लभेयमिह कर्हिचित्}
{विषयैः सङ्गतं चास्तु त्यजेयं सङ्गतं यदि ॥संवर्त उवाच}
{}


\twolineshloka
{आविक्षित शुभा बुद्धिर्वर्ततां तव कर्मसु}
{याजनं हि ममाप्येव वर्तते हृदि पार्थिव}


\twolineshloka
{अभिधास्ते च ते राजन्नक्षयं द्रव्यमुत्तमम्}
{येन देवान्सग्धर्वाञ्शक्रं चाभिभविष्यसि}


\twolineshloka
{न तु मे वर्तते बुद्धिर्धने याज्येषु वा पुनः}
{विप्रियं तु करिष्यामि भ्रातुश्चेन्द्रस्य चोभयो}


\twolineshloka
{गमयिष्यामि शक्रेण समतामपि ते ध्रुवम्}
{प्रियं च ते करिष्यामि सत्यमेतद्ब्रवीमि ते}


\chapter{अध्यायः ८}
\twolineshloka
{गिरेर्हिमवतः पृष्ठे मुञ्जवान्नाम पर्वतः}
{तप्यते यत्र भगवांस्तपो नित्यमुपापतिः}


\twolineshloka
{वनस्पतीनां मूलेषु शृङ्गेषु विषमेषु च}
{गुहासु शैलराजस्य यथाकामं यथासुखम्}


\twolineshloka
{उमासहायो भगवान्यत्र नित्यं महेश्वरः}
{आस्ते शूली महातेजा नानाभूतगणावृतः}


\twolineshloka
{तत्र रुद्राश्च साध्याश्च विश्वेऽथ वसवस्तथा}
{यमश्च वरुणश्चैव कुबेरश्च सहानुगः}


\twolineshloka
{भूतानि च पिशाचाश्च नासत्यावपि चाश्विनौ}
{गन्धऱ्वाप्सरसश्चैव यक्षा देवर्षयस्तथा}


\twolineshloka
{आदित्या मरुतश्चैव यातुधानाश्च सर्वशः}
{उपासन्ते महात्मानं बहुरूपमुपापतिम्}


\twolineshloka
{रमते भगवांस्तत्र कुबेरानुचरैः सह}
{विकृतैर्विकृताकारैः क्रीडद्भिः पृथिवीपते}


\threelineshloka
{श्रिया ज्वलन्दृश्यते वै बालादित्यसमद्युतिः}
{न रूपं शक्यते तस्य संस्थानं वा कदाचन}
{निर्देष्टुं प्राणिभिः कैश्चित्प्राकृतैर्मांसलोचनैः}


\twolineshloka
{नोष्णं न शिशिरं तत्र न वायुर्न च भास्करः}
{न जारा क्षुत्पिपासे वा न मृत्युर्न भयं नृप}


\twolineshloka
{तस्य शैलस्य पार्श्वेषु सर्वेषु जयतांवर}
{धातवो जातरूपस्य रश्मयः सवितुर्यथा}


\twolineshloka
{रक्ष्यन्ते ते कुबेरस्य सहायैरुद्यतायुधैः}
{चिकीर्षद्भिः प्रियं राजन्कुबेरस्य महात्मनः}


\twolineshloka
{`तत्र गत्वा समन्वास्य महायोगेश्वरं शिवम्}
{कुरु प्रणामं राजर्षे भक्त्या परमया यतुः ॥'}


\twolineshloka
{तस्मै भगवते कृत्वा नमः शर्वाय वेधसे}
{`एभिस्तं नामभिर्देवं सर्वविद्याधरं स्तुहि ॥'}


\twolineshloka
{रुद्राय शितिकण्ठाय सुरूपाय सुवर्चसे}
{कपर्दिने करालाय हर्यक्ष्णे वरदाय च}


\twolineshloka
{त्र्यक्ष्णे पूष्णो दन्तभिदे वामनाय शिवाय च}
{याम्यायाव्यक्तरूपाय सद्वृत्ते शङ्कराय च}


\twolineshloka
{क्षेम्याय हरिकेशाय स्थाणवे पुरुषाय च}
{हरिनेत्राय मुण्डाय क्रुद्धायोत्तरणाय च}


\twolineshloka
{भास्वराय सुतीर्थाय देवदेवाय रंहसे}
{उष्णीषिणे सुवक्त्राय सहस्राक्षाय मीढुषे}


\twolineshloka
{गिरिशाय प्राशान्ताय यतये चीरवाससे}
{बिल्वदण्डाय सिद्धाय सर्वदण्डधराय च}


\twolineshloka
{मृगव्याधाय महते धन्विनेऽथ भवाय च}
{वराय सोमवक्त्राय सिद्धमन्त्राय चक्षुषे}


\twolineshloka
{हिरण्यबाहवे राजन्नुग्राय पतये दिशाम्}
{लेलिहानाय गोष्ठाय सिद्धमन्त्राय वृष्णये}


\twolineshloka
{पशूनां पतये चैव भूतानां पतये नमः}
{वृषाय मातृभक्ताय सेनान्ये मध्यमाय च}


\threelineshloka
{`अभिवक्त्राय पतये सर्वदेवमयाय च}
{'स्रुवहस्ताय पतये धन्विने भार्गवाय च}
{अजाय कृष्णनेत्राय विरूपाक्षाय चैव ह}


\twolineshloka
{तीक्ष्णदंष्ट्राय तीक्ष्णाय वैश्वानरमुखाय च}
{महात्मने चानङ्गाय सर्वाय पतये विशाम्}


\threelineshloka
{`तथा रुद्राय पतये पृथवे कृत्तिवाससे}
{'विलोहिताय दीप्ताय दीप्ताक्षाय महौजसे}
{वसुरेतःसुवपुषे पृथवे कृत्तिवाससे}


\twolineshloka
{कपालमालिने चैव सुवर्णमुकुटाय च}
{महादेवाय कृष्णाय त्र्यम्बकायानघाय च}


\twolineshloka
{क्रोधनायानृशंसाय मृदवे बाहुशालिने}
{दण्डिने तप्ततपसे तथैवाक्रूरकर्मणे}


\twolineshloka
{सहस्रशिरसे चैव सहस्रचरणाय च}
{नमः स्वधास्वरूपाय बहुरूपाय दंष्ट्रिणे}


\twolineshloka
{पिनाकिनं महादेवं महाभोगिनमव्ययम्}
{त्रिशूलहस्तं वरदं त्र्यम्बकं भुवनेश्वरम्}


\twolineshloka
{त्रिपुरघ्नं त्रिनयनं त्रिलोकेशं महौजसम्}
{प्रभवं सर्वभूतानां दातारं धरणीधरम्}


\twolineshloka
{ईशानं शङ्करं सर्वं शिवं विश्वेश्वरं भवम्}
{उमापतिं पशुपतिं विश्वरूपं महेश्वरम्}


\twolineshloka
{विरूपाक्षं दशभुजं विष्यन्दं गोवृषध्वजम्}
{उग्रं स्थाणुं शिवं रौद्रं शर्वं गौरीशमीश्वरम्}


\twolineshloka
{शितिकण्ठमजं शुक्रं पृथुं पृथुहरं वरम्}
{विश्वरूपं विरूपाक्षं बहुरूपमुपापतिम्}


\twolineshloka
{प्रणम्य शिरसा देवमनङ्गाङ्गहरं हरम्}
{शरण्यं शरणं याहि महादेवं चतुर्मुखम्}


\twolineshloka
{`विरोचमानं वपुषा दिव्याभरणभूषितम्}
{अनाद्यन्तमजं शंभुं सर्वव्यापिनमीश्वरम्}


\twolineshloka
{निस्त्रैगुण्यं निरुद्वेगं निर्मलं निधिमोजसाम्}
{प्रणम्य प्राञ्जलिः शर्वं प्रयामि शरणं हरम्}


\twolineshloka
{सम्मान्यं निश्चलं नित्यमकारुण्यमलेपनम्}
{अध्यात्मवेदमासाद्य प्रयामि शरणं मुहुः}


\twolineshloka
{यस्य नित्यं विदुः स्थानं मोक्षमध्यात्मचिन्तकाः}
{योगीशं तत्वमार्गस्थाः कैवल्यं पदमक्षरम्}


\twolineshloka
{यं विदुः सङ्गिनं मुक्ताः सामान्यं समदर्शिनः}
{तं प्रपद्ये जगद्योनिमयोनिं निर्गुणात्मकम्}


\twolineshloka
{असृजद्यस्तु भूतादीन्सप्त लोकान्सनातनान्}
{स्थितः सत्योपरि स्थाणुस्तं प्रपद्ये सनातनम्}


\twolineshloka
{भक्तानां सुलभं तं हि दुर्लभं दूरपातिनाम्}
{अदूरस्थममुं देवं प्रकृतेः परतः स्थितम्}


\threelineshloka
{नमामि सर्वलोकस्थं व्रजामि शरणं शिवम्}
{'एवं कृत्वा नमस्तस्मै महादेवाय रंहसे}
{महात्मने क्षितिपते तत्सुवर्णमवाप्स्यसि}


\twolineshloka
{`लभन्ते गाणपत्यं च तदेकाग्रा हि मानवाः}
{किं पुनः स्वर्णिभाण्डानि तस्मात्त्वं गच्छ मा चिरं}


\threelineshloka
{महत्तरं हि ते लाभं हस्त्यश्वोष्ट्रादिभिः सह}
{'सुवर्णमाहरिष्यन्तस्तत्र गच्छन्तु ते नराः ॥व्यास उवाच}
{}


\twolineshloka
{इत्युक्तः स वचस्तस्य चक्रे कारन्धमात्मजः}
{`गङ्गाधरं नमस्कृत्य लब्धवान्धनमुत्तमम्}


\twolineshloka
{कुबेर इव तत्प्राप्य महादेवप्रसादतः}
{'ततोऽतिमानुषं सर्वं चक्रे यज्ञस्य संविधिम्}


\twolineshloka
{सौवर्णानि च भाण्डानि संचक्रुस्तत्र शिल्पिनः}
{`शालाश्च सर्वसम्भारांस्तत्र संवर्तशासनात् ॥'}


\twolineshloka
{बृहस्पतिस्तु तां श्रुत्वा मरुत्तस्य महीपतेः}
{समृद्धिमति देवेभ्यः सन्तापमकरोद्भृशम्}


\twolineshloka
{सन्तप्यमानो वैवर्ण्यं कृशत्वं चागमत्परम्}
{भविष्यति हि मे शत्रुः संवर्तो वसुमानिति}


\twolineshloka
{तं श्रुत्वा भृशसंतप्तं देवराजो बृहस्पतिम्}
{अभिगम्यामरवृतः प्रोवाचेदं वचस्तदा}


\chapter{अध्यायः ९}
\threelineshloka
{कच्चित्सुखं स्वपिषि त्वं बृहस्पतेकच्चिन्मनोज्ञाः परिचारकास्ते}
{कच्चिद्देवानां सुखकामोसि विप्रकच्चिद्देवास्त्वां परिपालयन्ति ॥बृहस्पतिरुवाच}
{}


\threelineshloka
{सुखं शये शयने देवराजतथा मनोज्ञाः परिचारका मे}
{तथा देवानां सुखकामोस्मि नित्यंदेवाश्च मां सुभृशं पालयन्ति ॥इन्द्र उवाच}
{}


\threelineshloka
{कुतो दुःखं मानसं देहजं वापाण्डुर्विवर्णश्च कुतस्त्वमद्य}
{आचक्ष्व मे ब्राह्मण यावदेता-न्निहन्मि सर्वांस्तव दुःखकर्तॄन् ॥बृहस्पतिरुवाच}
{}


\threelineshloka
{मरुत्तमाहुर्मघवन्यक्ष्यमाणंमहायज्ञेनोत्तमदक्षिणेन}
{संवर्तो याजयतीति मे श्रुतंतदिच्छामि न स तं याजयेत् ॥इन्द्र उवाच}
{}


\threelineshloka
{सर्वान्कामाननुयातोसि विप्रत्वं देवानां मन्त्रयसे पुरोधाः}
{उभौ च ते जरामृत्यू व्यतीतौकिं संवर्तस्तव कर्ताऽद्य विप्र ॥बृहस्पतिरुवाच}
{}


\twolineshloka
{देवैः सह त्वमसुरान्सम्प्रणुद्यजिघांससे चाप्युत सानुबन्धान्}
{यंयं समृद्धं पश्यसि तत्रतत्रदुःखं सपत्नेषु समृद्धिभावः}


\threelineshloka
{अतोस्मि देवेन्द्र विवर्णरूपःसपत्नो मे वर्धते तन्निशम्य}
{सर्वोपायैर्मघवन्संनियच्छसंवर्तं वा पार्थिवं वा मरुत्तम् ॥इन्द्र उवाच}
{}


\threelineshloka
{एहि गच्छ प्रहितो जातवेदोबृहस्पतिं परिदातुं मरुत्ते}
{अयं वै त्वां याजयिता बृहस्पति-स्तथाऽमरं चैव करिष्यतीति ॥अग्निरुवाच}
{}


\threelineshloka
{अहं गच्छामि तव शक्राद्य दूतोबृहस्पतिं परिदातुं मरुत्ते}
{वाचं सत्यां पुरुहूतस्य कर्तुंबृहस्पतेश्चापचितिं चिकीर्षुः ॥व्यास उवाच}
{}


\threelineshloka
{ततः प्रायाद्धूमकेतुर्महात्मावनस्पतीन्वीरुधश्चावमृद्गन्}
{कामाद्धिमान्ते परिवर्तमानःकाष्ठातिगो मातरिश्वेव नर्दन् ॥मरुत उवाच}
{}


\threelineshloka
{आश्वर्यमद्य पश्यामि रूपिणं वह्निमागतम्}
{आसनं सलिलं पाद्यं गां चोपानय वै मुने ॥अग्निरुवाच}
{}


\threelineshloka
{आसनं सलिलं पाद्यं प्रतिनन्दामि तेऽनघ}
{इन्द्रेण तु समादिष्टं विद्धि मां दूतमागतम् ॥मरुत उवाच}
{}


\threelineshloka
{कच्चिच्छ्रीमान्देवराजः सुखी चकच्चिच्चास्मान्प्रीयते धूमकेतो}
{कच्चिद्देवा अस्य वशे यथाव-त्प्रब्रूहि त्वं मम कार्त्स्न्येन देव ॥अग्निरुवाच}
{}


\twolineshloka
{शक्रो भृशं सुसुखी पार्थिवेन्द्रप्रातिं चेच्छत्यजरां वै त्वया सः}
{देवाश्च सर्वे वशगास्तस्य राज-न्संदेशं त्वं शृणु मे देवराज्ञः}


\threelineshloka
{यदर्थं मां प्राहिणोत्त्वत्सकाशंबृहस्पतिं परिदातुं मरुत्ते}
{अयं गुरुर्याजयतां नृप त्वांमर्त्यं सन्तममरं त्वां करोतु ॥मरुत उवाच}
{}


\threelineshloka
{संवर्तोऽयं याजयिता द्विजो मांबृहस्पतेरञ्जलिरेष तस्य}
{न चैवासौ याजयित्वा महेन्द्रंमर्त्यं सन्तं याजयन्नद्य शोभेत् ॥अग्निरुवाच}
{}


\twolineshloka
{ये वै लोका देवलोके महान्तःसम्प्राप्स्यसे तान्देवराजप्रसादात्}
{त्वां चेदसौ याजयेद्वै बृहस्पति-र्नूनं स्वर्गं त्वं जयेः कीर्तियुक्तः}


\threelineshloka
{तथा लोका मानुषा ये च दिव्याःप्रजापतेश्चापि ये वै महान्तः}
{तेते जिता देवराज्यं च कृत्स्नंबृहस्पतिर्याजयेच्चेन्नरेन्द्र ॥संवर्त उवाच}
{}


\threelineshloka
{मा स्मैव त्वं पुनरागाः कथंचि-द्बृहस्पतिं परिदातुं मरुत्ते}
{मा त्वां धक्ष्ये चक्षुषा दारुणेनसंक्रुद्धोऽहं पावक त्वं निबोधः ॥व्यास उवाच}
{}


\twolineshloka
{ततो देवानगमद्धूमकेतु-दीहाद्भीतो व्यथितोऽश्वत्थपर्णवत्}
{तं वै दृष्ट्वा प्राह शक्रो महात्माबृहस्पतेः सन्निधौ हव्यवाहम्}


\fourlineindentedshloka
{यस्त्वं गतः प्रहितो जातवेदोबृहस्पतिं परिदातुं मरुत्ते}
{तत्किं प्राह स नृपो यक्ष्यमाणःकच्चिद्वचः प्रतिगृह्णाति तच्च}
{अग्निरुवाच}
{}


\twolineshloka
{न ते वाचं रोचयते मरुत्तोबृहस्पतेरञ्जलिं प्राहिणोत्सः}
{संवर्तो मां याजयितेत्युवाचपुनः पुनः स मया याच्यमानः}


\fourlineindentedshloka
{उवाचेदं मानुषा ये च दिव्या}
{प्रजापतेर्ये च लोका महान्तः}
{तांश्चेल्लभेयं संविदं तेन कृत्वातथापि नेच्छेयमिति प्रतीतः ॥इन्द्र उवाच}
{}


\threelineshloka
{पुनर्गत्वा पार्थिवं त्वं समेत्यवाक्यं मदीयं प्रापय स्वार्थयुक्तम्}
{पुनर्यद्युक्तो न करिष्यते वच-स्त्वत्तो वज्रं सम्प्रहर्तास्मि तस्मै ॥अग्निरुवाच}
{}


\twolineshloka
{गन्धर्वराड्यात्वयं तत्र दूतोबिभेम्यहं वासव तत्र गन्तुम्}
{संरब्धो मामब्रवीत्तीक्ष्णरोषःसंवर्तो वाक्यं चरितब्रह्मचर्यः}


\threelineshloka
{यद्यागच्छेः पुनरेवं कथंचि-द्बृहस्पतिं परिदातुं मरुत्ते}
{दहेयं त्वां चक्षुषा दारुणेनसंक्रुद्ध इत्येतदवैहि शक्र ॥शक्र उवाच}
{}


\threelineshloka
{त्वमेवान्यान्दहसे जातवेदोन हि त्वदन्यो विद्यते भस्मकर्ता}
{त्वत्संस्पर्शात्सर्वलोको बिभेतिअश्रद्धेयं वदसे हव्यवाह ॥अग्निरुवाच}
{}


\threelineshloka
{दिवं देवेन्द्र पृथिवीं च सर्वांसंवेष्टयेस्त्वं स्वबलेनैव शक्र}
{एवंविधस्येह सतस्तवासौकथं वृत्रस्त्रिदिवं प्राग्जहार ॥इन्द्र उवाच}
{}


\twolineshloka
{नगण्डिकाकारयोगं करेऽणुंन चारिसोमं प्रपिबामि वह्ने}
{न क्षीणशक्तौ प्रहारामि वज्रंको मे सुखाय प्रहरेत मर्त्यः}


\threelineshloka
{प्रव्रजयेयं कालकेयान्पृथिव्या-मपाकर्षन्दानवानन्तरिक्षात्}
{दिवः प्रह्लादमवसानमानयंको मेऽसुखाय प्रहरेत मानवः ॥अग्निरुवाच}
{}


\twolineshloka
{यत्र शर्यातिं च्यवनो याजयिष्य-न्सहाश्विभ्यां सोममगृह्णदेकः}
{तं त्वं क्रुद्धः प्रत्यषेधीः पुरस्ता-च्छर्यातियज्ञं स्म तं महेन्द्र}


\twolineshloka
{वज्रं गृहीत्वा च पुरन्दर त्वंसम्प्राहार्षीश्च्यवनस्यातिघोरम्}
{स ते विप्रः सह वज्रेणि बाहु-मपागृह्णात्तपसा जातमन्युः}


\twolineshloka
{ततो रोषात्सर्वतो घोररूपंसपत्नं ते जनयामास भूयः}
{मदं नाम्ना चासुरं विश्वरूपंयं त्वं दृष्ट्वा चक्षुषी संन्यमीलः}


\twolineshloka
{हनुरेका जगतीस्था तथैकादिवं गता महतो दानवस्य}
{सहस्रं दन्तानां शतयोजनानांसुतीक्ष्णानां घोररूपं बभूव}


\twolineshloka
{वृत्ताः स्थूला रजतस्तम्भवर्णादंष्ट्राश्चतस्रो द्वे शते योजनानाम्}
{स त्वां दन्तान्विदशन्नभ्यधाव-ज्जिघांसया शूलमुद्यम्य घोरम्}


\twolineshloka
{तं नापश्यस्त्वं तदा घोररूपंसर्वे वै त्वां ददृशुर्दर्शनीयम्}
{यस्माद्भीतः प्राञ्जलिस्त्वं महर्षि-मागच्छेथाः शरणं दानवघ्न}


\twolineshloka
{क्षात्राद्बलाद्ब्रह्मबलं गरीयोन ब्रह्मतः किञ्चिदन्यद्गरीयः}
{सोहं जानन्ब्रह्मतेजो यथाव-न्न संवर्त गन्तुमिच्छामि शक्र}


\chapter{अध्यायः १०}
\twolineshloka
{एवमेतद्ब्रह्मबलं गरीयोन ब्राह्मणात्किञ्चिदन्यद्ररीयः}
{आविक्षितस्य तु बलं न मृष्येवज्रमस्मै प्रहरिष्यामि घोरम्}


\twolineshloka
{धृतराष्ट्र गच्छ प्रहितो मरुत्तंसंवर्तेन सङ्गतं तं वदस्वबृहस्पतिं त्वमुपशिक्षस्व राज-न्वज्रं वा ते प्रहरिष्यामि घोरम् ॥व्यास उवाच}
{}


% Check verse!
ततो गत्वा धृतराष्ट्रो नरेन्द्रंप्रोवाचेदं वचनं वासवस्य
\twolineshloka
{गन्धर्वं मां धृतराष्ट्रं निबोधत्वामागतं वक्तुकामं नरेन्द्र}
{ऐन्द्रं वाक्यं शृणु मे राजसिंहयत्प्राह लोकाधिपतिर्महात्मा}


\threelineshloka
{बृहस्पतिं याजकं त्वं वृणीष्ववज्रं वा ते प्रहरिष्यामि घोरम्}
{वचश्चेदेतन्न करिष्यसे मेप्राहैतदेतावदचिन्त्यकर्मा ॥मरुत्त उवाच}
{}


\twolineshloka
{त्वं चैवैतद्वेत्थ पुरन्दरश्चविश्वेदेवा वसवश्चाश्विनौ च}
{मित्रद्रोहे निष्कृतिर्नास्ति लोकेमहत्पापं ब्रह्महत्यासमं तत्}


\threelineshloka
{बृहस्पतिर्याजयतां महेन्द्रंदेवश्रेष्ठं वज्रभृतां वरिष्ठम्}
{संवर्तो मां याजयिताऽद्य राज-न्न ते वाक्यं तस्य वा रोचयामि ॥गन्धर्व उवाच}
{}


\threelineshloka
{घोरो नादः श्रूयतां वासवस्यनभस्तले गर्जतो राजसिंह}
{व्यक्तं वज्रं मोक्ष्यते ते महेन्द्रःक्षेमं राजंश्चिन्त्यतामेष कालः ॥व्यास उवाच}
{}


\threelineshloka
{इत्येवमुक्तो धृतराष्ट्रेण राज-ञ्श्रुत्वा नादं नदतो वासवस्य}
{तपोनित्यं धर्मविदां वरिष्ठंसंवर्तं तं ज्ञापयामास कार्यम् ॥मरुत्त उवाच}
{}


\twolineshloka
{पश्यात्मानं प्लवमानं त्वमारा-दध्वा दूरं तेन न दृश्यतेऽद्य}
{प्रपद्येऽहं शर्म विप्रेन्द्र त्वत्तःप्रयच्छ तस्मादभयं विप्रमुख्य}


\threelineshloka
{अयमायाति वे वज्री दिशो विद्योतयन्दस}
{अमानुषेण घोरेणि सदस्यास्त्रासिता हि नः ॥संवर्त उवाच}
{}


\twolineshloka
{भयं शक्राद्व्येतु ते राजसिंहप्रणोत्स्येऽहं भयमेतत्सुघोरम्}
{संस्तम्भिन्या विद्यया क्षिप्रमेवमा भैस्त्वमस्याभिभवात्प्रतीतः}


\twolineshloka
{अहं संस्तम्भयिष्यामि मा भैस्त्वं शक्रतो नृप}
{यर्वेषामेव देवानां क्षयितान्यायुधानि मे}


\twolineshloka
{दिशो वज्रं व्रजतां वायुरेतुवर्षं भूत्वा वर्षतां काननेषु}
{आपः प्लवन्त्वन्तरिक्षे वृथा चसौदामनी दृश्यते माऽपि भैस्त्वम्}


\threelineshloka
{वह्निर्देवस्त्रातु वा सर्वतस्तेकामान्सर्वान्वर्षतु वासवो वा}
{वज्रं तथा स्थापयतां वधायमहाघोरं पुवमानं जलौघैः ॥मरुत्त उवाच}
{}


\threelineshloka
{घोरः शब्दः श्रूयते वै महास्वनोवज्रस्यैष सहितो मारुतेन}
{आत्मा हि मे प्रव्यथते मुहुर्मुहु-र्न मे स्वास्थ्यं जायते चाद्य विप्र ॥संवर्त उवाच}
{}


\threelineshloka
{वज्रादुग्राद्व्येतु भयं तवाद्यवातो भूत्वा व्रजतु नरेन्द्र वज्रम्}
{भयं त्वक्त्वा वरमन्यं वृणीष्वकं ते कामं तपसा साधयामि ॥मरुत्त उवाच}
{}


\threelineshloka
{इन्द्रः साक्षात्सहसाऽभ्येतु विप्रहविर्यज्ञे प्रतिगृह्णातु चैव}
{स्वंस्वं हविश्चैव जुषन्तु देवाहुतं सोमं प्रतिगृह्णन्तु चैव ॥संवर्त उवाच}
{}


\threelineshloka
{अयमिन्द्रो हरिभिरायाति राज-न्देवैः सर्वैस्त्वरितैः स्तूयमानः}
{मन्त्राहूतो यज्ञमिमं मयाऽद्यपश्यश्वैनं मन्त्रविस्रस्तकायम् ॥व्यास उवाच}
{}


\twolineshloka
{ततो देवैः सहितो देवराजोरथे युङ्क्त्वा तान्हरीन्वाजिमुख्यान्}
{आयाद्यज्ञमथ राज्ञः पिपासु-राविक्षितस्याप्रमेयस्य सोमम्}


\threelineshloka
{तमायान्तं सहितं देवसङ्घैःप्रत्युद्ययौ सपुरोधा मरुत्तः}
{चक्रे पूजां देवराजाय चाग्र्यांयथाशास्त्रं विधिवत्प्रीयमाणः ॥संवर्त उवाच}
{}


\threelineshloka
{सुस्वागतं ते पुरुहूतेह विद्व-न्यज्ञोऽप्ययं सन्निहिते त्वयीन्द्र}
{शोशुभ्यते बलवृत्रघ्न भूयःपिबस्व सोमं सुतमुद्यतं मया ॥मरुत्त उवाच}
{}


\fourlineindentedshloka
{शिवेन मां पश्य नमश्च तेऽस्तुप्राप्तो यज्ञः सफलं जीवितं मे}
{अयं यज्ञं कुरुते मे सुरेन्द्रबृहस्पतेरवरो जन्मना च}
{इन्द्र उवाच}
{}


\twolineshloka
{जानामि ते गुरुमेनं तपोधनंबृहस्पतेरनुजं तिग्मतेजसम्यस्याह्वानादागतोऽहं नरेन्द्रप्रीतिर्मेऽद्य त्वयि मन्युः प्रनष्टः ॥संवर्त उवाच}
{}


\threelineshloka
{यदि प्रीतस्त्वमसि वै देवराजतस्मात्स्वयं शाधि यज्ञे विधानम्}
{स्वयं सर्वान्कुरु भागान्सुरेन्द्रजानात्वयं सर्वलोकश्च देव ॥व्यास उवाच}
{}


\twolineshloka
{एवमुक्तस्त्वाङ्गिरसेन शक्रःसमादिदेश स्वयमेव देवान्}
{सभाः क्रियन्तामावसथाश्च मुख्याःसहस्रशश्चित्रभूताः समृद्धाः}


\twolineshloka
{क्लृप्ताः स्थूणाः कुरुतारोहणानिगन्धर्वाणामप्सरसां च शीघ्रम्}
{यत्र नृत्येरन्नप्सरसः समस्ताःस्वर्गोपमः क्रियतां यज्ञवाटः}


\twolineshloka
{इत्युक्तास्ते चक्रुराशु प्रतीतादिवौकसः शक्रवाक्यान्नरेन्द्र}
{ततो वाक्यं प्राह राजानमिन्द्रःप्रीतो राजन्पूज्यमानो मरुत्तम्}


\twolineshloka
{एष त्वयाऽहमिह राजन्समेत्यये चाप्यन्ते तव पूर्वे नरेन्द्र}
{सर्वाश्चान्या देवताः प्रीयमाणाहविस्तुभ्यं प्रतिगृह्णन्तु राजन्}


\twolineshloka
{आग्नेयं वै लोहितमालभन्तांवैश्वदेवं बहुरूपं हि राजन्}
{निलं चोक्षाणं मेद्यमप्यालभन्तांचलच्छिश्नं सम्प्रदिष्टं द्विजाग्र्याः}


\twolineshloka
{ततो यज्ञो ववृधे तस्य राज-न्यत्र देवाः स्वयमन्नानि जह्रुः}
{यस्मिञ्शक्रो ब्राह्मणैः पूज्यमानःसदस्योऽभूद्धरिमान्देवराजः}


\twolineshloka
{ततः संवर्तश्चैत्यगतो माहात्मायथा वह्निः प्रज्वलितो द्वितीयः}
{हवींष्युच्चैराह्वयन्देवसङ्घा-ञ्जुहावाग्नौ मन्त्रवत्सुप्रतीतः}


\twolineshloka
{ततः पीत्वा बलभित्सोममग्र्यंये चाप्यन्ते सोमपा देवसङ्घाः}
{सर्वेऽनुज्ञाताः प्रययुः पार्थिवेनयथाजोषं तर्पिताः प्रीतिमन्तः}


\twolineshloka
{ततो राजा जातरूपस्य राशी-न्पदेपदे कारयामास हृष्टः}
{द्विजातिभ्यो विसृजन्भूरि वित्तंरराज वित्तेश इवारिहन्ता}


\twolineshloka
{ततो वित्तं विविधं सन्निधाययथोत्साहं कारयित्वा च कोशम्}
{अनुज्ञातो गुरुणां संनिवृत्त्यशशास गामखिलां सागरान्ताम्}


\threelineshloka
{एवंगुणः सम्बभूवेह राजायस्य क्रतौ तत्सुवर्णं प्रभूतम्}
{तत्त्वं समादाय नरेन्द्र वित्तंयजस्व देवांस्तपनीयैर्विधानैः ॥वैशम्पायन उवाच}
{}


\twolineshloka
{ततो राजा पाण्डवो हृष्टरूपःश्रुत्वा वाक्यं सत्यवत्याः सुतस्य}
{मनश्चक्रे तेन वित्तेन यष्टुंततोऽमात्यैर्मन्त्रयामास भूयः}


\chapter{अध्यायः ११}
\twolineshloka
{इत्युक्ते नृपतौ तस्मिन्व्यासेनाद्भुतकर्मणा}
{वासुदेवो महातेजास्ततो वचनमाददे}


\twolineshloka
{तं नृपं दीनमनसं निहतज्ञातिबान्धवम्}
{उपप्लुतमिवादित्यं सधूममिव पावकम्}


\threelineshloka
{निर्विण्णमनसं पार्थं ज्ञात्वा वृष्णिकुलोद्वहः}
{आश्वासनन्धर्मसुतं प्रवक्तुमुपचक्रमे ॥वासुदेव उवाच}
{}


\twolineshloka
{सर्वं जिह्मं मृत्युपदमजिह्मं ब्रह्मणः पदम्}
{एतावान्ज्ञानविषयः किं प्रलापः करिष्यति}


\twolineshloka
{नैव तेऽनुष्ठितं कर्म नैव ते शत्रवो जिताः}
{कथं शत्रुं शरीरस्तमात्मनो नावबुध्यसे}


\twolineshloka
{अत्र ते वर्तयिष्यामि यथाधर्मं यथाश्रुतम्}
{इन्द्रस्य सह वृत्रेण यथा युद्धमवर्तत}


\threelineshloka
{वृत्रेण पृथिवी व्याप्ता पुरा किल नराधिप}
{दृष्ट्वा स पृथिवीं व्याप्तां गन्धस्य विषये हृते}
{धराहरणदुर्गन्धो विषयः समपद्यत}


\twolineshloka
{शतक्रतुश्चुकोपाथ गन्धस्य विषये हृते}
{वृत्रस्य सततः क्रुद्धो घोरं वज्रमवासृजत्}


\twolineshloka
{स वध्यमानो वज्रेण सुभृशं भूरितेजसा}
{विवेश सहसा तोयं जग्राह विषयं ततः}


\twolineshloka
{अप्सु वृत्रगृहीतासु रसे च विषये हृते}
{शतक्रतुरतिक्रुद्धस्तत्र वज्रमवासृजत्}


\twolineshloka
{स वध्यमानो वज्रेण तस्मिन्नमिततेजसा}
{विवेश सहसा ज्योतिर्जग्राह विषयं ततः}


\twolineshloka
{व्याप्ते ज्योतिषि वृत्रेण रूपेऽथ विषये हृते}
{शतक्रतुरतिक्रुद्धस्तत्र वज्रमवासृजत्}


\twolineshloka
{स वद्यमानो वज्रेण तस्मिन्नमिततेजसा}
{विवेश सहसा वायुं जग्राह विषयं ततः}


\twolineshloka
{व्याप्ते वायौ तु वृत्रेण स्पर्शेऽथ विषये हृते}
{शतक्रतुरतिक्रुद्धस्तत्र वज्रमवासृजत्}


\twolineshloka
{स वध्यगानो वज्रेण तस्मिन्नमिततेजसा}
{आकाशमभिदुद्राव जग्राह विषयं ततः}


\twolineshloka
{आकाशे वृत्रभूतेऽथ शब्दे च विषये हृते}
{शतक्रतुरभिक्रुद्धस्तत्र वज्रमवासृजत्}


\twolineshloka
{स वध्यमानो वज्रेण तस्मिन्नमिततेजसा}
{विवेश सहसा शक्रं जग्राह विषयं ततः}


\twolineshloka
{तस्य वृत्रगृहीतस्य मोहः समभवन्महान्}
{रथन्तरेण तं साम्ना वसिष्ठः प्रत्यबोधयत्}


\twolineshloka
{ततो वृत्रं शरीरस्थं जघान भरतर्षभ}
{शतक्रतुरदृश्येन वज्रेणेतीह नः श्रुतम्}


\twolineshloka
{इदं धर्म्यं रहस्यं वै शक्रेणोक्तं महर्षिषु}
{ऋषिभिश्च मम प्रोक्तं तन्निबोध जनाधिप}


\chapter{अध्यायः १२}
\twolineshloka
{द्विवोधो जायते व्याधिः शारीरो मानसस्तथा}
{परस्परं तयोर्जन्म निर्द्वन्द्वं नोपपद्यते}


\twolineshloka
{शरीरे जायते व्याधिः शरीरः स निगद्यते}
{मानसे जायते व्याधिर्मानसस्तु निगद्यते}


\twolineshloka
{शीतोष्णे चैव वायुश्च गुणा राजञ्शरीरजाः}
{तेषां गुणानां साम्यं चेत्तदाहुः स्वस्थलक्षणम्}


% Check verse!
उष्णेन बाध्यते शीतं शीतेनोष्णं च बाध्यते
\twolineshloka
{सत्वं रजस्तमश्चेति त्रय आत्मगुणाः स्मृताः}
{तेषां गुणानां साम्यं चेत्तदाहुः स्वस्थलक्षणम्}


\twolineshloka
{तेषामन्यतमोत्सेके विधानमुपदिश्यते}
{हर्षेण बाध्यते शोको हर्षः शोकेन बाध्यते}


\twolineshloka
{कश्चिद्दुःखे वर्तमानः सुखस्य स्मर्तुमिच्छति}
{कश्चित्सुखे वर्तमानो दुःखस्य स्मर्तुमिच्छति}


\threelineshloka
{स त्वं न दुःखी दुःखस्य न सुखी सुसुखस्य वा}
{स्मर्तुमिच्छसि कौन्तेय दैवं हि बलवत्तरम्}
{अथवा ते स्वभावोऽयं येन पार्थावकृष्यसे}


\twolineshloka
{दृष्ट्वा सभागतां कृष्णामेकवस्त्रां रजस्वलाम्}
{मिषतां पाण्डवेयानां न तस्य स्मर्तुमिच्छसि}


\twolineshloka
{प्रव्राजनं च नगरादजिनैश्च विवासनम्}
{महारण्यनिवासश्च न तस्य स्मर्तुमिच्छसि}


\twolineshloka
{जटासुरात्परिक्लेशश्चित्रसेनेन चाहवः}
{सैन्धवाच्च परिक्लेशो न तस्य स्मर्तुमिच्छसि}


\twolineshloka
{पुनरज्ञातचर्यायां कीचकेन पदा वधः}
{याज्ञसेन्यास्तथा पार्थ न तस्य स्मर्तुमिच्छसि}


\twolineshloka
{यच्च ते द्रोणभीष्माभ्यां युद्धमासीदरिंदम}
{मनसैकेन योद्धव्यं तत्ते युद्धमुपस्थितम्}


\twolineshloka
{तस्मादभ्युपगन्तव्यं युद्धाय भरतर्षभ}
{परमव्यक्तरूपस्य पारं युक्त्या स्वकर्मभिः}


\twolineshloka
{यत्र नैव शरैः कार्यं न भृत्यैर्न च बन्धुभिः}
{आत्मनैकेन योद्धव्यं तत्ते युद्धमुपस्थितम्}


\twolineshloka
{तस्मिन्ननिर्जिते युद्धे कामवस्थां गमिष्यसि}
{एतज्ज्ञात्वा तु कौन्तेय कृतकृत्यो भविष्यसि}


\twolineshloka
{एथां बुद्धिं विनिश्चित्य भूतानामागतिं गतिम्}
{पितृपैतामहे वृत्ते शाधि राज्यं यथोचितम्}


\chapter{अध्यायः १३}
\twolineshloka
{न बाह्यं द्रव्यमुत्सृज्य सिद्धिर्भवति भारत}
{शारीरं द्रव्यमुत्सृज्य सिद्धिर्बवति वा न वा}


\twolineshloka
{बाह्यद्रव्यविमुक्तस्य शारीरेषु च गृह्यतः}
{यो धर्मो यत्सुखं चैव द्विषतामस्तु तत्तव}


\twolineshloka
{द्व्यक्षरस्तु भवेन्मृत्युस्त्र्यक्षरं ब्रह्म शाश्वतम्}
{ममेति द्व्यक्षरो मृत्युर्नममेति च शाश्वतम्}


\twolineshloka
{ब्रह्ममृत्यू ततो राजन्नात्मन्येव व्यवस्थितौ}
{अदृश्यमानौ भूतानि योधयेतामसंशयम्}


\twolineshloka
{अविनाशोऽस्य तत्त्वस्य नियतो यदि भारत}
{भित्त्वा शरीरं भूतानामहिंसां प्रतिपद्यते}


\twolineshloka
{लब्ध्वा हि पृथिवीं कृत्स्नां सहस्थावरजङ्गमाम्}
{ममत्वं यस्य नैव स्यात्किं तया स करिष्यति}


\twolineshloka
{अथवा वसतः पार्थ वने वन्येन जीवतः}
{ममता यस्य वित्तेषु मृत्योरांस्ये स वर्तते}


\twolineshloka
{ब्राह्यान्तराणां शत्रूणां स्वभावं पश्य भारत}
{यन्न पश्यति तद्भूतं मुच्यते स महाभयात्}


\threelineshloka
{कामात्मानं न प्रशंसन्ति लोकेनेहाकामा काचिदस्ति प्रवृत्तिः}
{सर्वे कामा मनसोऽङ्ग प्रभूतायान्पण्डितः संहरते विचिन्त्य}
{भूयोभूयो जन्मनोऽभ्यासयोगा-द्योगी योगं सारमार्गं विचिन्त्य}


\threelineshloka
{दानं च वेदाध्ययनं तपश्चकाम्यानि कर्माणि च वैदिकानि}
{व्रतं यज्ञान्नियमान्ध्यानयोगा-न्कामेन यो नारभते विदित्वा}
{यद्यच्चायं कामयते स धमोनयो धर्मो नियमस्तस्य मूलम्}


\threelineshloka
{अत्र गाथाः कामगीताः कीर्तयन्ति पुराविदः}
{शृणु सङ्कीर्त्यमानास्ता अखिलेन युधिष्ठिर ॥काम उवाच}
{}


% Check verse!
नाहं शक्योऽनुपायेन हन्तुं भूतेन केनचित्
\twolineshloka
{यो मां प्रयतते हन्तुं ज्ञात्वा प्रहरणे बलम्}
{तस्य तस्मिन्प्रहरणे पुनः प्रादुर्भवाम्यहम्}


\twolineshloka
{यो मां प्रयतते हन्तुं यज्ञैर्विविधदक्षिणैः}
{जङ्गमेष्विव धर्मात्मा पुनः प्रादुर्भवाम्यहम्}


\twolineshloka
{यो मां प्रयतते हन्तुं वेदैर्वेदान्तसाधनैः}
{स्थावरेष्विव भूतात्मा तस्य प्रादुर्भवाम्यहम्}


\twolineshloka
{यो मां प्रयतते हन्तुं धृत्या सत्यपराक्रमः}
{भावो भवामि तस्याहं स च मां नावबुध्यते}


\twolineshloka
{यो मां प्रयतते हन्तुं तपसा संशितव्रतः}
{ततस्पपसि तस्यथ पुनः प्रादुर्भवाम्यहम्}


\threelineshloka
{यो मां प्रयतते हन्तुं मोक्षमास्थाय पण्डितः}
{तस्य मोक्षरतिस्थस्य नृत्यामि च हसामि च}
{अवध्यः सर्वभूतानामहमेकः सनातनः}


\twolineshloka
{तस्मात्त्वमपि तं कामं यज्ञैर्विविधदक्षिणैः}
{धर्मे कुरु महाराज तत्र ते स भविष्यति}


\twolineshloka
{मा ते व्यथाऽस्तु निहतान्बन्धून्वीक्ष्य पुनःपुनः}
{न शक्यास्ते पुनर्द्रष्ट्रं येऽहतास्मिन्रणाजिरे}


\twolineshloka
{स त्वमिष्ट्वा महायज्ञैः समृद्धैराप्तदक्षिणैः}
{कीर्तिं लोके परां प्राप्य गतिमग्र्यां गमिष्यसि}


\chapter{अध्यायः १४}
\twolineshloka
{एवं बहुविधैर्वाक्यैर्मुनिभिस्तैस्तपोधनैः}
{समाश्वस्त राजर्षिर्हितबन्धुर्युधिष्ठिरः}


\twolineshloka
{सोऽनुनीतो भगवता विष्टरश्रवसा स्वयम्}
{द्वैपायनेन कृष्णेनि देवस्थानेन चाभिभूः}


\twolineshloka
{नारदेनाथ भीमेन नकुलेन च पार्थिव}
{कृष्णया सहदेवेन विजयेन च धीमता}


\twolineshloka
{अन्यैश्च पुरुषव्याघ्रैर्ब्राह्मणैः शास्त्रदृष्टिभिः}
{व्यजहाच्छोकजं दुःखं सन्तापं चैव मानसम्}


\threelineshloka
{अर्चयामास देवांश्च ब्राह्मणांश्च युधिष्ठिरः}
{कृत्वाऽथ प्रेतकार्याणि सर्वेषां कुरुनन्दनः}
{अन्वशासच्च धर्मात्मा पृथिवीं सागराम्बराम्}


\twolineshloka
{प्रशान्तचेताः कौरव्यः स्वराज्यं प्राप्य केवलम्}
{व्यासं च नारदं चैव तांश्चान्यानब्रवीन्नृपः}


\twolineshloka
{आश्वासितोऽहं प्राग्वृद्धैर्भवद्भिर्मुनिपुङ्गवैः}
{न सूक्ष्ममपि मे किञ्चिद्व्यलीकमिह विद्यते}


\twolineshloka
{अर्थश्च सुमहान्प्राप्तो येन यक्ष्यामि देवताः}
{पुरस्कृत्याद्य भवतः समानेष्यामहे मखम्}


\twolineshloka
{हिमवन्तं त्वया गुप्ता गमिष्यामः पितामह}
{बह्वाश्चर्यो हि देशः स श्रूयते द्विजसत्तम}


\twolineshloka
{तथा भगवता चित्रं कल्याणं बहु भाषितम्}
{देवर्षिणा नारदेन देवस्थानेन चैव ह}


\threelineshloka
{नाभागधेयः पुरुषः कश्चिदेवंविधान्गुरून्}
{लभते व्यसनं प्राप्य सुहृदः साधुसम्मतान् ॥वैशम्पायन उवाच}
{}


\twolineshloka
{एवमुक्तास्तु ते राज्ञा सर्व एव महर्षयः}
{अभ्यनुज्ञाप्य राजानं तथोभौ कृष्णफल्गुनौ}


\twolineshloka
{पश्यतामेव सर्वेषां तत्रैवादर्शनं ययुः}
{ततो धर्मसुतो राजा तत्रैवोपाविशत्प्रभुः}


\threelineshloka
{एवं नातिमहान्कालः स तेषां संन्यवर्तत}
{कुर्वतां शौचकार्याणि भीष्मस्य निधने तदा}
{महादानानि विप्रेभ्यो ददतामौर्ध्वदेहिकम्}


\twolineshloka
{भीष्मकर्णपुरोगाणां कुरूणां कुरुसत्तम}
{सहितो धृतराष्ट्रेण स ददावौर्ध्वदेहिकम्}


\twolineshloka
{ततो दत्त्वा बहुधनं विप्रेभ्यः पाण्डवर्षभः}
{धृताराष्ट्रं पुरुस्कृत्य विवेश गजसाह्वयम्}


\chapter{अध्यायः १५}
\twolineshloka
{स समाश्वास्य पितरं प्रज्ञाचक्षुषमीश्वरम्}
{अन्वशासत धर्मात्मा पृथिवीं भ्रातृभिः सह}


\twolineshloka
{यथा मनुर्महाराजो रामो दाशरथिर्यथा}
{तथा भरतसिंहोऽपि पालयामास मेदिनीम्}


\twolineshloka
{नाधर्म्यमभवत्तत्र सर्वो धर्मरुचिर्जनः}
{बभूव नरशार्दूल यथा कृतयुगे तथा}


\twolineshloka
{कलिमासन्नमाविष्टं निवार्य नृपनन्दनः}
{भ्रातृभिः सहितो धीमान्बभौ धर्मबलोद्धतः}


\twolineshloka
{ववर्ष भगवान्देवः काले देशे यथेप्सितम्}
{निरामयं जगदभूत्क्षुत्पिपासे न किञ्चन}


\twolineshloka
{आधिर्नास्ति मनुष्याणां व्यसने नाभवन्मतिः}
{ब्राह्म्णप्रमुखा वर्णास्ते स्वधर्मोत्तराः शुभाः}


\threelineshloka
{धर्मसत्यप्रधानाश्च सत्यं सद्विषयान्वितम्}
{धर्मासनस्थः सद्भिः स स्त्रीबालातुरवृद्धकान्}
{वर्णक्रमान्पूर्णभृतान्साकल्याद्रक्षणोद्यतः}


\twolineshloka
{अवृत्तिवृत्तिदानाद्यैर्यज्ञाद्यैर्व्याधितैरपि}
{आमुष्मिकं भयं नास्ति लौकिकं कृतमेव}


\twolineshloka
{स्वर्गलोकोपमो लोकस्तदा तस्मिन्प्रशासति}
{बभूव सुखमेवाग्रं तद्विशिष्टतरं परम्}


\twolineshloka
{नार्यः पतिव्रताः सर्वा रूपवत्यः स्वलंकृताः}
{यथोक्तवृत्ताः स्वगुणैर्बभूवुः प्रीतिहेतवः}


\twolineshloka
{पुमांसः पुण्यशीलाढ्याः स्वंस्वं धर्ममनुव्रताः}
{सुखिनः सूक्ष्ममप्येनो न कुर्वन्ति कदाचन}


\twolineshloka
{सर्वे नरांश्च नार्यश्च सततं प्रियवादिनः}
{अजिह्ममनसः शुक्ला बभूवुः श्रमवर्जिताः}


\twolineshloka
{भूषिताः कुण्डलैर्हारैः कटकैः कटिसूत्रकैः}
{सुवाससः सुगन्धाढ्याः प्रायशः पृथिवीतले}


\twolineshloka
{सर्वे ब्रह्मविदो विप्राः सर्वत्र परिनिष्ठिताः}
{वलीपलितहीनास्तु सुखिनो दीर्घदर्शिनः}


\twolineshloka
{इच्छा न जायतेऽन्यत्र वर्णेषु न च सङ्करः}
{मनुष्याणां महाराज मर्यादा सुव्यवस्थिता}


\twolineshloka
{तस्मिञ्शासति राजेन्द्रे मृगव्यालसरीसृपाः}
{अन्योन्यमपि चान्येषु न बाधन्ते कदाचन}


\twolineshloka
{गावः सुक्षीरभूयिष्ठाः सुस्ववालमुखोदराः}
{अपीडिताः कर्षकाद्यैर्हृतव्याधिकवत्सकाः}


\twolineshloka
{अवन्ध्यकाला मनुजाः पुरुषार्थेषु च क्रमात्}
{विषयेष्वनिषिद्धेषु वेदशास्त्रेषु चोद्यताः}


% Check verse!
सुवृत्ता वृषभाः पुष्टा रसनाभाः सुखोदयाः
\twolineshloka
{अतीव मधुरः शब्दः स्पर्शश्चातिसुखं रसम्}
{करूपं दृष्टिक्षमं रम्यं मनोज्ञं गन्धवद्बभौ}


\twolineshloka
{धर्मार्थकामसंयुक्तं मोक्षाभ्युदयसाधनम्}
{प्रह्लादजननं पुण्यं सम्बभूवाथ मानसम्}


\twolineshloka
{स्थावरा बहुपुष्पाढ्याः फलच्छायावहास्तथा}
{सुस्पर्शा विषहीनाश्च सुपत्रत्वक्प्ररोहिणः}


\twolineshloka
{मनोनुकूलाः सर्वेषां चेष्टाभूतापवर्जिताः}
{तथा बभूव राजर्षिस्तद्वृत्तमभवद्भुवि}


\twolineshloka
{सर्वलक्षणसम्पन्नाः पाण्डवा धर्मचारिणः}
{ज्येष्ठानुवर्तिनः सर्वे बभूवुः प्रियदर्शनाः}


\twolineshloka
{सिंहोरस्का जितक्रोधास्तेजोबलसमन्विताः}
{आजानुबाहवः सर्वे दानशीला जितेन्द्रियाः}


\twolineshloka
{तेषु शासत्सु धरणीमृतवः स्वगुणैर्बभुः}
{सुखोदयाय वर्तन्ते ग्रहास्तारागणैः सह}


\twolineshloka
{मही च सस्यबहुला सर्वरत्नगुणोदया}
{कामधुग्धेनुवद्भोगान्फलन्ति स्म सहस्रधा}


\threelineshloka
{मन्वादिभिः कृताः पूर्वं मर्यादा मानवेषु याः}
{अनतिक्रम्य ताः सर्वाः कुलेषु समयानि च}
{अन्वशासत राजानो धर्मपुत्रप्रियंकराः}


\twolineshloka
{महाकुलानि धर्मिष्ठा वर्धयन्तो विशेषतः}
{मनुप्रणीतया वृत्त्या तेऽन्वशासनसुन्धराम्}


\twolineshloka
{राजवृत्तिर्हि सा शश्वद्धर्मिष्ठाऽभून्महीतले}
{प्रायो लोकमतिस्तान राजवृत्तानुगामिनी}


\twolineshloka
{एवं भारतवर्षं स्वं राजा स्वर्गं सुरेन्द्रवत्}
{शशास जिष्णुना सार्धं गोत्रां गाण्डीवधन्वन'}


\chapter{अध्यायः १६}
\threelineshloka
{विजिते पाण्डवेयैस्तु प्रशान्ते च द्विजोत्तम}
{राष्ट्रे ते चत्रतुर्वीरौ वासुदेवधनंजयौ ॥वैशम्पायन उवाच}
{}


\twolineshloka
{विजिते पाण्डवै राजन्प्रशान्ते च विशाम्पतौ}
{राष्ट्रे बभूवतुर्हृष्टौ वासुदेवधनंजयौ}


\twolineshloka
{विजह्राते मुदा युक्तौ दिवि देवश्वराविव}
{तौ वनेषु विचित्रेषु पर्वतेषु ससानुषुः}


\twolineshloka
{तीर्थेषु चैव पुण्येषु पल्वलेषु नदीषु च}
{चंक्रम्यमाणौ संहृष्टावश्विनाविव नन्दने}


\twolineshloka
{इन्द्रप्रस्थे महात्मानौ रेमाते कृष्णपाण्डवौ}
{प्रविश्य तां सभां रम्यां विजह्राते च भारत}


\twolineshloka
{तत्र युद्धकथाश्चित्राः परिक्लेशांश्च पार्थिव}
{कथायोगे कथायोगे कथयामासतुः सदा}


\twolineshloka
{ऋषीणां देवतानां च वंशांस्तावाहतुः सदा}
{प्रीयमाणौ महात्मानौ पुराणावृषिसत्तमौ}


\twolineshloka
{मधुरास्तु कथाश्चित्राश्चित्रार्थपदनिश्चयाः}
{निश्चयज्ञः स पार्थाय कथयामास केशवः}


\twolineshloka
{पुत्रशोकाभिसंतप्तं ज्ञातीनां च सहस्रशः}
{कथाभिः शमयामास पार्तं शौरिर्जनार्दनः}


\twolineshloka
{स तमाश्वास्य विधिवद्विधानज्ञो महातपाः}
{अपहृत्यात्मनो भारं विशश्रामेव सात्वतः}


\twolineshloka
{ततः कथान्ते गोविन्दो गुडाखेशमुवाच ह}
{सान्त्वयञ्श्लक्ष्णया वाचा हेतुयुक्तमिदं वचः}


\twolineshloka
{विजितेयं धरा कृत्स्ना सव्यसाचिन्परंतप}
{त्वद्बाहुबलामाश्रित्य राज्ञा धर्मसुतेन ह}


\twolineshloka
{असपत्नां महीं भुङ्क्ते धर्मराजो युधिष्ठिरः}
{भीमसेनानुभावेन यमयोश्च नरोत्तम}


\twolineshloka
{धर्मेणि राज्ञा धर्मज्ञ प्राप्तं राज्यमकण्टकम्}
{धर्मेण निहतः सङ्ख्ये स च राजा सुयोधनः}


\twolineshloka
{अधर्मरुचयो लुब्धाः सदा चाप्रियवादिनः}
{धार्तराष्ट्रा दुरात्मानः सानुबन्धा निपातिताः}


\twolineshloka
{प्रशान्तामखिलां पार्थ पृथिवीं पृथिवीपतिः}
{भुङ्क्ते धर्मसुतो राजा त्वया गुप्तः कुरूद्वह}


\twolineshloka
{रमे चाहं त्वया सार्धमरण्येष्वपि पाण्डव}
{किमु यत्र जनोऽयं वै पृथा चामित्रकर्मन}


\twolineshloka
{यत्र धर्मसुतो राजा यत्र यत्र भीमो महाबलः}
{यत्र माद्रवतीपुत्रौ रतिस्तत्र परा मम}


\twolineshloka
{तथैव स्वर्गलोकेषु सभोद्देशेषु कौरव}
{रमणीयेषु पुष्णेषु सहितस्य त्वयाऽनध}


\twolineshloka
{कालो महांस्त्वतीतो मे शूरसूनुमपश्यतः}
{बलदेवं च कौरव्यं तथाऽन्यान्वृष्णिपुङ्गवान्}


\twolineshloka
{सोहं गन्तुमभीप्सामि पुरीं द्वारावतीं प्रति}
{रोचतां गमनं तुभ्यं ममापि पुरुषर्षभ}


\twolineshloka
{उक्तो बहुविधं राजा तत्रतत्र युधिष्ठिरः}
{सह भीष्मेण यद्युक्तमस्माभिः शोकर्शितः}


\twolineshloka
{शिष्टो युधिष्टिरोऽस्माभिः शास्ता सन्नपि पाण्डवः}
{तेन तत्तु वचः सम्यग्गृहीतं सुमहात्मना}


\twolineshloka
{धर्मपुत्रे हि धर्मज्ञे कृतज्ञे सत्यवादिनि}
{सत्यं धर्मो मतिश्चाग्र्या स्थितिश्च सततं स्थिरा}


\twolineshloka
{तत्र गत्वा महात्मानं यदि ते रोचतेऽर्जुन}
{अस्मद्गमनसंयुक्तं वचो ब्रूहि जनाधिपम्}


\twolineshloka
{न हि तस्याप्रियं कुर्यां प्राणत्योगेऽप्युपस्थिते}
{कुतो गन्तुं महाबाहो पुरीं द्वारावतीं प्रति}


\twolineshloka
{सर्वं त्विदमहं पार्त त्वत्प्रीतिहितकाम्यया}
{ब्रवीमि सत्यं कौरव्य न मिथ्यैतत्कथञ्चन}


\twolineshloka
{प्रयोजनं च निर्वृत्तमिह वासेन मेऽर्जुन}
{धार्तराष्ट्रो हतो राजा सबलः सपदानुगः}


\twolineshloka
{पृथिवी च वशे तात धर्मपुत्रस्य धीमतः}
{स्थिता समुद्रवसना सशैलवनकानना}


\twolineshloka
{चिता रत्नैर्बहुविधैः कुरुराजस्य पाण्डव}
{धर्मेण राजा धर्मज्ञः पातु सर्वां वसुन्धराम्}


\twolineshloka
{उपास्यमानो मुनिभिः सिद्धैश्चापि महात्मभिः}
{स्तूयमानश्च सततं बन्दिभिर्भरतर्षभ}


\twolineshloka
{तं मया सह गत्वाऽद्य राजानं कुरुवर्धनम्}
{आपृच्छ कुरुशार्दूल गमनं द्वारकां प्रति}


\twolineshloka
{इदं शरीरं वसु यच्च मे गृहेनिवेदितं पार्थ सदा युधिष्ठिरे}
{प्रियश्च मान्यश्च हि मे युधिष्ठिरःसदा कुरूणामधिपो महामतिः}


\twolineshloka
{प्रयोजनं चापि निवासकारणेन विद्यते मे त्वदृते नृपात्मज}
{स्थिता हि पृथ्वी तव पार्थ शासनेगुरोः सुवृत्तस्य युधिष्ठिरस्य च}


\twolineshloka
{इतीदमुक्तः स तदा महात्मनाजनार्दनेनामितविक्रमोऽर्जुनः}
{तथेति दुःखादिव वाक्यमैरय-ज्जनार्दनं सम्प्रतिपूज्य पार्थिव}


\chapter{अध्यायः १७}
\threelineshloka
{सभायां वसतोस्तत्र निहत्यारीन्महात्मनोः}
{केशवार्जुनयोः का नु कथा समभवद्द्विज ॥वैशम्पायन उवाच}
{}


\twolineshloka
{कृष्णेन सहितः पार्थः स्वं राज्यं प्राप्य केवलम्}
{तस्यां सभायां दिव्यायां विजहार मुदा युतः}


\twolineshloka
{तत्र कञ्चित्सभोद्देशं स्वर्गोद्देशसमं नृप}
{यदृच्छया तौ मुदितौ जग्मतुः स्वजनावृतौ}


\twolineshloka
{ततः प्रतीतः कृष्णेन सहितः पाण्डवोऽर्जुनः}
{निरीक्ष्य तां सभां रम्यामिदं वचनमब्रवीत्}


\twolineshloka
{विदितं मे महाबाहो सङ्ग्रामे समुपस्थिते}
{महात्म्यं देवकीपुत्र तच्च ते रूपमैश्वरम्}


\twolineshloka
{यत्तु तद्भवता प्रोक्तं पुरा केशव सौहृदात्}
{तत्सर्वं पुरुषव्याघ्र नष्टं मे व्यग्रचेतसः}


\threelineshloka
{मम कौतूहलं त्वस्ति तेष्वर्थेषु पुनः पुनः}
{भवांस्तु द्वारकां गन्ता नचिरादिव माधव ॥वैशम्पायन उवाच}
{}


\threelineshloka
{एवमुक्तस्तु तं कृष्णः फाल्गुनं प्रत्यभाषत}
{परिष्वज्य महातेजा वचनं वदतांवरः ॥वासुदेव उवाच}
{}


\twolineshloka
{श्रावितस्त्वं मया गुह्यं ज्ञापितश्च सनातनम्}
{धर्मं स्वरूपिणं पार्त सर्वलोकांश्च शाश्वतान्}


\twolineshloka
{अबुद्ध्या यन्न गृह्णीतास्तन्मे सुमहदप्रियम्}
{न च साऽद्य पुनर्भूयः स्मृतिर्मे सम्भविष्यति}


\twolineshloka
{नूनमश्रद्दधानोऽसि दुर्मेधा ह्यसि पाण्डव}
{न च शक्यं पुनर्वक्तुमशेषेण धनंजय}


\twolineshloka
{स हि धर्मः सुपर्याप्तो ब्रह्मणः पदवेदने}
{न शक्यं तन्मया भूयस्तथा वक्तुमशेषतः}


\twolineshloka
{परं हि ब्रह्म कथितं योगयुक्तेन तन्मया}
{इतिहासं तु वक्ष्यामि तस्मिन्नर्थे पुरातनम्}


\twolineshloka
{यथा तां बुद्धिमास्थाय गतिमग्र्यां गमिष्यसि}
{शृणु धर्मभृतांश्रेष्ठ गदतः सर्वमेव मे}


\twolineshloka
{आगच्छद्ब्राह्मणः कश्चित्स्वर्गलोकादरिंदम}
{ब्रह्मलोकाच्च दुर्धर्षः सोस्माभिः पूजितोऽभवत्}


\threelineshloka
{अस्माभिः परिपृष्ठश्च यदाह भरतर्षभ}
{दिव्येन विधिना पार्थ तच्छृणुष्वाविचारयन् ॥ब्राह्मण उवाच}
{}


\twolineshloka
{मोक्षधर्मं समाश्रित्य कृष्ण यन्माऽनुपृच्छसि}
{भूतानामनुकम्पार्थं मन्मोहच्छेदनं विभो}


\twolineshloka
{तत्तेऽहं सम्प्रवक्ष्यामि यथावन्मधुसूदन}
{शृणुष्वावहितो भूत्वा गदतो मम माधव}


\twolineshloka
{कश्चिद्विप्रस्तपोयुक्तः काश्यपो धर्मवित्तमः}
{आससाद द्विजं कंचिद्धर्माणामागतागमम्}


\twolineshloka
{गतागमं सुबहुशो ज्ञानविज्ञानपारगम्}
{लोकतत्त्वार्थकुशलं ज्ञातरं सुखदुःखयोः}


\twolineshloka
{जातिस्मरणतत्त्वज्ञं कोविदं पापपुण्ययोः}
{द्रष्टारमुच्चनीचानां कर्मभिर्देहिनां गतिम्}


\twolineshloka
{चरन्तं मुक्तवत्सिद्धं प्रशान्तं संयतेन्द्रियम्}
{दीप्यमानं श्रिया ब्राह्मया क्रममाणं च सर्वशः}


\twolineshloka
{अन्तर्धानगतिज्ञं च श्रुत्वा तत्त्वेन काश्यपः}
{तथैवान्तर्हितैः सिद्धैर्यान्तं चक्रधरैः सह}


\twolineshloka
{सम्भाषमाणमेकान्ते समासीनं च तैः सह}
{यदृच्छया च गच्छन्तमसक्तं पवनं यथा}


\threelineshloka
{तं समासाद्य मेधावी स तदा द्विजसत्तमः}
{चरणौ धर्मकामो वै स तस्य सुसमाहितः}
{प्रतिपदे यथान्यायं भक्त्या परमया युतः}


\twolineshloka
{विस्मितश्चाद्भुतं दृष्ट्वा काश्यपस्तं द्विजोत्तमम्}
{परिचारेण महता गुरुं तं पर्यतोषयत्}


\twolineshloka
{उपपन्नं च तत्सर्वं श्रुतचारित्रसंयुतम्}
{भौमेनातोषयच्चैनं गुरुवृत्तिं समास्थितः}


\threelineshloka
{तस्मै तुष्टः स शिष्याय यत्प्रसन्नोऽब्रवीद्गुरुः}
{सिद्धिं परामभिप्रेक्ष्य शृणु मत्तो जनार्दन ॥सिद्ध उवाच}
{}


\twolineshloka
{विविधैः कर्मभिस्तात पुण्ययोगैश्च केवलैः}
{गच्छन्तीह गतिं मर्त्या देवलोके च वा स्थितिम्}


\twolineshloka
{न क्वचित्सुखमत्यन्तं न क्वचिच्छाश्वती स्थितिः}
{स्थानाच्च महतो भ्रंशो दुःखलब्दात्पुनः पुनः}


\twolineshloka
{अशुभा गतयः प्राप्ताः कष्टा मे पापसेवनात्}
{काममन्युपरीतेन तृष्णया मोहितेन च}


\twolineshloka
{पुनः पुनश्च मरणं जन्म चैव पुनः पुनः}
{आहारा विविधा भुक्ताःपीता नानाविधाःस्तनाः}


\twolineshloka
{मातरो विविधा दृष्टाः पितरश्च पृथग्विधाः}
{सुखानि च विचित्राणि दुःखानि च मयाऽनघ}


\twolineshloka
{प्रियैर्विवासो बहुशः संवासश्चाप्रियैः सह}
{धननाशश्च सम्प्राप्तो लब्ध्वा दुःखेन तद्धनम्}


\twolineshloka
{अवमानाः सुकष्टाश्च परतः स्वजनात्तथा}
{शारीरा मानसा वाऽपि वेदना भृशदारुणाः}


\twolineshloka
{प्राप्ता विमाननाश्चोग्रा वधबन्धाश्च दारुणाः}
{पतनं निरये चैव यातनाश्च यमक्षये}


\twolineshloka
{जरारोगाश्च सततं व्यसनानि च भूरिशः}
{लोकेऽस्मिन्ननुभूतानि द्वन्द्वजानि भृशं मया}


\twolineshloka
{ततः कदाचिन्निर्वेदान्निकारान्निकृतेन च}
{लोकतन्त्रं परित्यक्तं दुःखार्तेन भृशं मया}


\threelineshloka
{लोकेऽस्मिन्ननुभूयाहमिमं मार्गमनुष्ठितः}
{ततः सिद्धिरियं प्राप्ता प्रसादादात्मनो मया}
{नाहं पुनरिहागन्ता लोकानालोकयाम्यहम्}


\twolineshloka
{आसिद्धेराप्रजासर्गादात्मनोपि गतीः शुभाः}
{उपलब्धा द्विजश्रेष्ठ तथेयं सिद्धिरुत्तमा}


\twolineshloka
{इतः परं गमिष्यामि ततः परतरं पुनः}
{ब्रह्मणः पदमव्यक्तं मा तेऽभूतत्र संशयः}


\twolineshloka
{नाहं पुनरिहागन्ता मर्त्यलोकं परन्तप}
{प्रीतोस्मि ते महाप्राज्ञ ब्रूहि किं करवाणि ते}


\twolineshloka
{यदीप्सुरुपपन्नस्त्वं तस्य कालोऽयमागतः}
{अभिजाने च तदहं यदर्थं मामुपागतः}


\twolineshloka
{अचिरात्तु गमिष्यामि येनाहं त्वामचूचुदम्}
{भृशं प्रीतोस्मि भवतश्चारित्रेण विचक्षण}


% Check verse!
परिपृच्छ यावद्भवतो भाषे यद्यत्तवेप्सितम्
\twolineshloka
{बहुमन्ये च ते बुद्धिं भृशं सम्पूजयामि च}
{येनाहं भवता बुद्धो मेधावी ह्यसि काश्यप}


\chapter{अध्यायः १८}
\threelineshloka
{ततस्तस्योपसङ्गृह्य पादौ प्रश्नान्सुदुर्वचान्}
{पप्रच्छ तांश्च धर्मान्स प्राह धर्मभृतांवरः ॥काश्यप उवाच}
{}


\twolineshloka
{कथं शरीराच्च्यवते कथं चैवोपपद्यते}
{कथं कष्टाच्च संसारात्संसरन्परिमुच्यते}


\twolineshloka
{आत्मानं वा कथं युक्त्वा तच्छरीरं विमुञ्चति}
{शरीरे च विनिर्मुक्तो कथमन्यत्प्रपद्यते}


\threelineshloka
{कथं शुभाशुभे चायं कर्मणी स्वकृते नरः}
{उपभुङ्क्ते क्व वा कर्म विदेहस्योपतिष्ठते ॥ब्राह्मण उवाच}
{}


\threelineshloka
{एवं सञ्चोदितः सिद्धः प्रश्नांस्तान्प्रत्यभाषत}
{आनुपूर्व्येण वार्ष्णेय तन्मे निगदतः शृणुः ॥सिद्ध उवाच}
{}


\threelineshloka
{`अस्मिन्नेवाशु फलदा आयुष्यास्तु क्रियाःस्मृताः}
{आयुःकीर्तिकराणीह यानि कृत्यानि सेवते}
{शरीरग्रहणेऽन्यस्मिंस्तेषु क्षीणेषु सर्वशः}


\twolineshloka
{आयुःक्षयपरीतात्मा विपरीतानि सेवते}
{बुद्धिर्व्यावर्तते चास्य विनाशे प्रत्युपस्थिते}


\twolineshloka
{सत्त्वं बलं च कालं चाविदित्वा चात्मनस्तथा}
{अतिवेलमुपाश्नाति स्वविरुद्धान्यनात्मवान्}


\twolineshloka
{यदाऽयमतिकष्टानि सर्वाण्युपनिषेवते}
{अत्यर्थमपि वा भुङ्क्ते न वा भुङ्क्ते कदाचन}


\twolineshloka
{दुष्टान्नामिषपानं च यदन्योन्यविरोधि च}
{गुरु चाप्यमितं भुङ्क्ते नातिजीर्णे दिवा पुनः}


\twolineshloka
{व्यायाममतिमात्रं च व्यावाय चोपसेवते}
{सततं कर्मलोभाद्वा प्राप्तं वेगं विधारयेत्}


\twolineshloka
{रसाभियुक्तमन्नं वा दिवास्वप्नं च सेवते}
{अपक्वानागते काले स्वयं दोषान्प्रकोपयेत्}


\twolineshloka
{स्वदोषकोपनाद्रोगं लभते मरणान्तिकम्}
{अपि वोद्बन्धनादीनि परीतानि व्यवस्यति}


\twolineshloka
{तस्य तैः कारणैर्जन्तोः शरीरं च्यवते तदा}
{जीवितं प्रोच्यमानं तद्यथावदुपधारय}


\twolineshloka
{ऊष्मा प्रकुपितः काये तीव्रवायुसमीरितः}
{शरीरमनुपर्येत्य सर्वान्प्राणान्रुणद्धि वै}


\twolineshloka
{अत्यर्थं बलवानूष्मा शरीरे परिकोपितः}
{भिनत्ति जीवस्थानानि तानि कर्मणि विद्धि च}


\threelineshloka
{ततः सवेदनः सद्यो जीवः प्रच्यवते क्षरन्}
{शरीरं त्यजते जन्तुश्छिद्यमानेषु मर्मसु}
{वेदनाभिः परीतात्मा तद्विद्धि द्विजसत्तम}


\threelineshloka
{जनीमरणसंविग्राः सततं सर्वजन्तवः}
{दृश्यन्ते संत्यजन्तश्च शरीराणि द्विजर्षभ ॥गर्भसंक्रमणे चापि गर्भाणापुपसर्पणे}
{तादृशीमेव लभते वेदनां मानवः पुनः}


% Check verse!
भिन्नसंधिरथ क्लेदमद्भिः स लभते नरः
\twolineshloka
{यथा पञ्चसु भूतेषु सम्भूतत्वं नियच्छति}
{शैत्यात्प्रकुपितः काये तीव्रवायुसमीरितः}


\threelineshloka
{यः स पञ्चसु भूतेषु प्राणापाने व्यवस्थितः}
{स गच्छत्यूर्ध्वगो वायुः कृच्छ्रान्मुक्त्वा शरीरिणः}
{शरीरं च जहात्येवं निरुच्छ्वासश्च दृश्यते}


\twolineshloka
{स निरूष्मा निरुच्छ्वासो निःश्रीको गतचेतनः}
{कर्मणा सम्परित्यक्तो मृत इत्युच्यते नरः}


\twolineshloka
{स्रोतोभिर्यैर्विजानाति इन्द्रियार्थाञ्शरीरभृत्}
{तैरेव न विजानाति प्राणानाहारसम्भवान्}


% Check verse!
तत्रैव कुरुते काये यः स जीवः सनातनः
\twolineshloka
{तथा यद्यद्भवेन्मुक्तं सन्निपाते क्वचित्क्वचित्}
{तत्तन्मर्म विजानीहि शास्त्रदृष्टं हि तत्तथा}


\twolineshloka
{तेषु मर्मसु भिन्नेषु ततः स समुदीरयन्}
{आविश्य हृदयं जन्तोः सत्त्वं चाशु रुणद्धि वै}


\threelineshloka
{ततः सचेतनो जन्तुर्नाभिजानाति किञ्चन}
{तमसा संवृतज्ञानः संवृतेष्वेव मर्मसु}
{स जीवो निरदिष्ठानश्चाल्यते मातरिश्वना}


\twolineshloka
{ततः स तं महोच्छ्वासं भृशमुच्छ्वस्य दारुणम्}
{निष्क्रमन्कम्पयत्याशु तच्छरीरमचेतनम्}


\twolineshloka
{स जीवः प्रच्युतः कायात्कर्मभिः स्वैः समावृतः}
{अङ्कितः स्वैः शुभैः पुण्यैः पापैर्वाऽप्युपपद्यते}


\twolineshloka
{ब्राह्मणा ज्ञानसम्पन्ना यथावच्छ्रुतनिश्चयाः}
{इतरं कृतपुण्यं वा तं विजानन्ति लक्षणैः}


\twolineshloka
{यथान्धकारे खद्येतं दीप्यमानं ततस्ततः}
{चक्षुष्मन्तः प्रपश्यन्ति तथा च ज्ञानचक्षुषः}


\twolineshloka
{पश्यन्त्येवंविधं सिद्धा जीवं दिव्येन चक्षुषा}
{च्यवन्तं जायमानं च योनिं चानुप्रवेशितम्}


\twolineshloka
{तस्य स्थानानि दृष्टानि विविधानीह शास्त्रतः}
{कर्मभूमिरियं भूमिर्यत्र तिष्ठन्ति जन्तवः}


\twolineshloka
{ततः शुभाशुभं कृत्वा लभन्ते सर्वदेहिनः}
{इहैवोच्चावचान्भोगान्प्राप्नुवन्ति स्वकर्मभिः}


\threelineshloka
{इहैवाशुभकर्माणः कर्मभिर्निरयं गताः}
{अवाग्गतिरियं कष्टा यत्र पच्यन्ति मानवाः}
{तस्मात्सुदुर्लभो मोक्षो रक्ष्यश्चात्मा ततो भृशम्}


\twolineshloka
{ऊर्ध्वं तु जन्तवो गत्वा येषु स्थानेष्ववस्थिताः}
{कीर्त्यमानानि तानीह तत्त्वतः संनिबोध मे}


\twolineshloka
{तच्छ्रुत्वा नैष्ठिकीं बुद्धिं बुद्ध्येथाः कर्मनिश्चयम्}
{तारारूपाणि सर्वाणि यत्रैतच्चन्द्रमण्डलम्}


\twolineshloka
{यत्र विभ्राजते लोके स्वभासा सूर्यमण्डलम्}
{स्थानान्येतानि जानीहि जनानां पुण्यकर्मणाम्}


\twolineshloka
{कर्मक्षयाच्च ते सर्वे च्यवन्ते वै पुनः पुनः}
{तत्रापि च विशेषोस्ति दिवि नीचोच्चमध्यमः}


\twolineshloka
{न च तत्रापि संतोषो दृष्ट्वा दीप्ततरां श्रियम्}
{इत्येता गतयः सर्वाः पृथक्ते समुदीरिताः}


\twolineshloka
{उपपत्तिं तु वक्ष्यामि गर्भस्याहमतः परम्}
{तथावत्तां निगदतः शृणुष्वावहितो द्विज}


\chapter{अध्यायः १९}
\twolineshloka
{शुभानामशुभानां च नेह नाशोस्ति कर्मणाम्}
{प्राप्यप्राप्यानुपच्यन्ते क्षेत्रेक्षेत्रे तथातथा}


\twolineshloka
{यथा प्रसूयमानस्तु फली दद्यात्फलं बहु}
{तथा स्याद्विपुलं पुण्यं शुद्धेन मनसा कृतम्}


\twolineshloka
{पापं चापि तथैव स्यात्पापेन मनसा कृतम्}
{पुरोधाय मनो हीह कर्मण्यात्मा प्रवर्तते}


\twolineshloka
{यथा कर्मसमाविष्टः काममन्युसमावृतः}
{नरो गर्भं प्रविशति तत्रापि शृणु चोत्तरम्}


\twolineshloka
{शुक्रं शोणितसंसृष्टं स्त्रिया गर्भाशयं गतम्}
{क्षेत्रं कर्मजमाप्नोति शुभं वा यदि वाऽशुभम्}


\twolineshloka
{सौक्ष्म्यादव्यक्तयावाच्च न च क्वचन सज्जति}
{सम्प्राप्य ब्राह्मणः कायं तस्मात्तद्ब्रह्म शाश्वतम्}


% Check verse!
तद्बीजं सर्वभूतानां तेन जीवन्ति जन्तवः
\threelineshloka
{स जीवः सर्वभूतानां गर्भमाविश्य भागशः}
{दधाति चेतना सद्यः प्राणस्थानेष्ववस्थितः}
{ततः स्पन्दयतेऽङ्गानि स गर्भश्चेतनान्वितः}


\twolineshloka
{यथा लोहस्य विष्यन्दो निषिक्तो बिम्बविग्रहम्}
{उपैति तद्वज्जानीहि गर्भे जीवप्रवेशनम्}


\threelineshloka
{लोहपिण्डं यथा वह्निः प्रविश्य ह्यतितापयेत्}
{तथा त्वमपि जानीहि गर्भे जीवोपपादनम्}
{}


\twolineshloka
{यथा च दीपः शरणं दीप्यमानः प्रकाशयेत्}
{एवमेव शरीराणि प्रकाशयति चेतनः}


\twolineshloka
{यद्यच्च कुरुते कर्म शुभं वा यदि वाऽशुभम्}
{पूर्वदेहकृतं सर्वमवश्यमुपभुज्यते}


\twolineshloka
{ततस्तु क्षीयते चैव पुनश्चान्यत्प्रचीयते}
{यावत्तु मोक्षयोगस्थं धर्मं नैवावबुध्यते}


\twolineshloka
{तत्ते धर्मं प्रवक्ष्यामि सुखी भवति येन वै}
{आवर्तमानो जातीषु यथाऽन्योन्यासु सत्तम}


\twolineshloka
{दानं व्रतं ब्रह्मचर्यं यथोक्तव्रतधारणम्}
{दमः प्रशान्तता चैव भूतानां चानुकम्पनम्}


\twolineshloka
{संयमश्चानृशंस्य च परस्वादानवर्जनम्}
{व्यलीकानामकरणं भूताना मनसा भुवि}


\twolineshloka
{मातापित्रोश्च शुश्रूषा देवतातितिपूजनम्}
{गुरुपूजा घृणा शौचं नित्यमिन्द्रयसंयमः}


\twolineshloka
{प्रवर्तनं शुभानां च तत्सतां व्रतमुच्यते}
{ततो धर्मः प्रभवति यः प्रजाः पाति शाश्वतीः}


\twolineshloka
{`सद्भिराचरितो धर्मः सदाचारे प्रतिष्ठितः}
{उभयार्थो भवत्येव स्वर्गार्थो मोक्षतस्तथा ॥'}


\twolineshloka
{एवं सत्सु सदा पश्येत्तत्रछाप्येषा ध्रुवा स्थितिः}
{आचारो धर्ममाचष्टे यस्मिन्सन्तो व्यवस्थिताः}


\twolineshloka
{तेषु तत्कर्म निक्षिप्तं यः स धर्मः सनातनः}
{यस्तं समभिपद्येत न स दुर्गतिमाप्नुयात्}


\twolineshloka
{अतो नियम्यते लोकः प्रच्यवन्धर्मवर्त्मसु}
{यश्च योगी च मुक्तश्च स ऐतेभ्यो विशिष्यते}


\twolineshloka
{वर्तमानस्य धर्मेण पुरुषस्य यथा तथा}
{संसारतारणं ह्यस्य कालेन महता भवेत्}


\twolineshloka
{एवं पूर्वकृतं कर्म सर्वो जन्तुः प्रपद्यते}
{सर्वं तत्कारणं येन निकृतोऽप्यमिहागतः}


\twolineshloka
{शरीरग्रहणं चास्य केन पूर्वं प्रकल्पितम्}
{इत्येवं संशये लोके तच्च वक्ष्याम्यतः परम्}


\twolineshloka
{शरीरमात्मनः कृत्वा सर्वलोकपितामहः}
{त्रैलोक्यमसृजद्ब्रह्मा कृत्स्नं स्थावरजङ्गमम्}


\twolineshloka
{ततः प्रधानमसृजच्चेतनां तु शरीरिणाम्}
{यया सर्वमिदं व्याप्तं यां लोके परमां विदुः}


\twolineshloka
{इदं तत्क्षरमित्युक्तं परं त्वमृतमक्षरम्}
{त्रयाणां मिथुनं सर्वमेकैकस्य पृथक्पृथक्}


\twolineshloka
{असृजत्सर्वभूतानि पूर्वदृष्टः प्रजापतिः}
{स्थावराणि च भूतानि इत्येषा पौर्विकी श्रुतिः}


\twolineshloka
{तस्य कालपरीमाणमकरोत्स पितामहः}
{भूतेषु परिवृत्तिं च पनरावृत्तिमेव च}


\twolineshloka
{यथा तु कश्चिन्मेधावी दृष्टात्मा पूर्वजन्मनि}
{यत्प्रवक्ष्यामि तत्सर्वं यथावदुपपद्यते}


\twolineshloka
{सुखदुःखे यथा सम्यगनित्ये यः प्रपश्यति}
{कायं चामेध्यसंस्थानं विनाशं कर्मसंहितम्}


\twolineshloka
{यच्च किञ्चित्सुखं तच्च दुःखं दृष्टमिति स्मरन्}
{संसारसागरं घोरं तरिष्यति सुदुस्तरम्}


\twolineshloka
{जनीमरणरोगैश्च समाविष्टः प्रधानवित्}
{चेतनाव्तसु चैतन्यं समं भूतेषु पश्यति}


\twolineshloka
{निर्विद्यते ततः कश्चिन्मार्गमाणः परं पदम्}
{तस्योपदेशं वक्ष्यामि याथातथ्येन सत्तम}


\twolineshloka
{शाश्वतस्याव्ययस्याथ् यदस्य ज्ञानमुत्तमम्}
{प्रोच्यमानं मया विप्र निबोधदमशेषतः}


\chapter{अध्यायः २०}
\twolineshloka
{यः स्यादेकान्त आसीनस्तूष्णीं किञ्चिदचिन्तयन्}
{पूर्वंपूर्वं परित्यज्य स निरारम्भको भवेत्}


\twolineshloka
{सर्वमित्रः सर्वसहः शमे रक्तो जितेन्द्रियः}
{व्यपेतभयमन्युश्च कामहा मुच्यते नरः}


\threelineshloka
{आत्मवत्सर्वभूतेषु यश्चरेन्नियतः शुचिः}
{`नित्यमेव यथान्यायं यश्चरेन्नियतेन्द्रियः}
{'अमानी निरभीमानः सर्वतो मुक्त एव सः}


\twolineshloka
{जीवितं मरणं चोभे सुखदुःखे तथैव च}
{लाभालाभे प्रियद्वेष्ये यः समः स च मुच्यते}


\twolineshloka
{न कस्यचित्स्पृहयते नावजानाति किचन}
{निर्द्वन्द्वो वीतरागात्मा सर्वथा मुक्त एव सः}


\twolineshloka
{अनमित्रश्च निर्बन्धुरनपत्यश्च यः क्वचित्}
{त्यक्तधर्मार्थकामश्च निराराङ्क्षी च मुच्यते}


\twolineshloka
{नैव धर्मी न चाधर्मी पूर्वोपचितहा च यः}
{क्षीणधातुः प्रशान्तात्मा निर्द्वंद्वः स विमुच्यते}


\twolineshloka
{अकर्मा चाविकाङ्क्षश्च पश्येज्जगदशाश्वतम्}
{अस्वस्तमवशं नित्यं जन्ममृत्युजरायुतम्}


\twolineshloka
{वैराग्यबुद्धिः सततं तावद्दोषव्यपेक्षकः}
{आत्मबन्धविनिर्मोक्षं स करोत्यचिरादिव}


\twolineshloka
{अगन्धमरसस्पर्शमशब्दमपरिग्रहम्}
{अरूपमनभिज्ञेयं दृष्ट्वाऽऽत्मानं विमुच्यते}


\twolineshloka
{पञ्चभूतगुणैर्हीनममूर्तिमदलेपकम्}
{अगुणं गुणभोक्तारं यः पश्यति स मुच्यते}


\twolineshloka
{विहाय सर्वसङ्कल्पान्बुद्ध्या शारीरमानसान्}
{शनैर्निर्वाणमाप्नोति निरिन्धन इवानलः}


\twolineshloka
{सर्वसंस्कारनिर्मुक्तो निर्द्वन्द्वो निष्परिग्रहः}
{तपसा इन्द्रियग्रामं यश्चरेन्मुक्त एव सः}


\twolineshloka
{विमुक्तः सर्वसंस्कारैस्ततो ब्रह्मि सनातनम्}
{परमाप्नोति संशान्तमचलं नित्यमक्षरम्}


\twolineshloka
{अतः परं प्रवक्ष्यामि योगशास्त्रमनुत्तमम्}
{यज्ज्ञात्वा सिद्धमात्मानं लोके पश्यन्ति योगिनः}


\twolineshloka
{तस्योपदेशं वक्ष्यामि यथावत्तन्निबोध मे}
{यैर्योगैर्भावयन्नित्यं पश्यत्यात्मानमात्मनि}


\twolineshloka
{इन्द्रियाणि तु संहृत्य मन आत्मनि धारयेत्}
{तीव्रं तप्त्वा तपः पूर्वं मोक्षयोगं समाचरेत्}


\twolineshloka
{तपस्वी नित्यसङ्कल्पो दम्भाहङ्कारवर्जितः}
{मनीषी मनसा विप्रः पश्यत्यात्मानमात्मनि}


\twolineshloka
{स चेच्छक्रोत्ययं साधुर्योक्तुमात्मानमात्मनि}
{तत एकान्तशीलः स पश्यत्यात्मानमात्मनि}


\twolineshloka
{संयतः सततं युक्त आत्मवान्विजितेन्द्रियः}
{यथा य आत्मनाऽऽत्मानं सम्प्रयुक्तः प्रपश्यति}


\twolineshloka
{यथाहि पुरुषः स्वप्ने दृष्ट्वा पश्यत्यसाविति}
{तथारूपमिवात्मानं साधुयुक्तः प्रपश्यति}


\twolineshloka
{इषीकां च यथा मुञ्जात्कश्चिन्निष्कृष्य दर्शयेत्}
{योगी निष्कृष्य चात्मानं तथा पश्यति देहतः}


\twolineshloka
{मुञ्जं शरीरमित्याहुरिषीकामात्मनि श्रिताम्}
{एतन्निदर्शनं प्रोक्तं योगविद्भिरनुत्तमम्}


\twolineshloka
{यदा हि युक्तमात्मानं सम्यक् पश्यति देहभूत्}
{न तस्येहेश्वरः कश्चित्त्रैलोक्यस्यापि यः प्रभुः}


\twolineshloka
{अन्यान्याश्चैव तनवो यथेष्टं प्रतिपद्यते}
{विनिर्भिद्य जरां मृत्युं न शोचति न हृष्यति}


\twolineshloka
{देवानामपि देवत्वं युक्तः कारयते वशी}
{ब्रह्म चाव्ययमाप्नोति हित्वा देहमशाश्वतम्}


\twolineshloka
{विनश्यत्सु च लोकेषु न भयं तस्य जायते}
{क्लिश्यमानेषु भूतेषु न स क्लिश्यति केनचित्}


\twolineshloka
{दुःखशोकमलैर्घोरैः सङ्गस्नेहसमुद्भवैः}
{न विचाल्यति युक्तात्मा निस्पृहः शान्तमानसः}


\twolineshloka
{नैनं शस्त्राणि विध्यन्ते न मृत्युश्चास्य विद्यते}
{नातः सुखतरं किञ्चिल्लोके क्वचन दृश्यते}


\twolineshloka
{सम्यग्युक्त्वा स आत्मानमात्मन्येव प्रतिष्ठिते}
{विनिवृत्तजरादुःखः सुखं स्वपिति चापि सः}


\twolineshloka
{देहान्यथेष्टमभ्येति हित्वेमां मानुषीं तनुम्}
{निर्वेदस्तु न कर्तव्यो भुञ्जानेन कथञ्चन}


\twolineshloka
{सम्यग्युक्तो यदात्मानमात्मन्येव प्रपश्यति}
{तदैव न स्पृहयते साक्षादपि शतक्रतोः}


\twolineshloka
{योगमेकान्तशीलस्तु यथा विन्दति तच्छृणु}
{दृष्टपूर्वां दिशं चिन्त्य यस्मिन्संनिवसेत्परे}


\threelineshloka
{पुरस्याभ्यन्तरे तस्य मनः स्थाप्यं न बाह्यतः}
{पुरस्याभ्यन्तरे तिष्ठन्यस्मिन्नावसथे वसेत}
{तस्मिन्नावसथे धार्यं सबाद्याभ्यन्तरं मनः}


\twolineshloka
{प्रचिन्त्यावसथे कृत्स्नं यस्मिन्काये स पश्यति}
{तस्मिन्काये मनश्चास्य न च किञ्चन बाह्यतः}


\twolineshloka
{सन्नियम्येन्द्रियग्रामं निर्घोषं निर्जने वने}
{कायमभ्यन्तरं कृत्स्नमेकाग्रः परिचिन्तयेत्}


\twolineshloka
{इन्तांस्तालु च जिह्वां च गलं ग्रीवां तथैव च}
{हृदयं चिन्तयेच्चापि तथा हृदयबन्धनम्}


\twolineshloka
{इत्युक्तः स मया शिष्यो मेधावी मधुसूदन}
{पप्रच्छ पुनरेवेमं मोक्षधर्मं सुदुर्वचम्}


\twolineshloka
{भुक्तं भुक्तमिदं कोष्ठे कथमन्नं विपच्यते}
{कथं रसत्वं व्रजति शोणितत्वं कथं पुनः}


\twolineshloka
{तथा मांसं च मेदश्च स्नाय्वस्थीनि च पोषयेत्}
{कथमेतानि सर्वाणि शरीराणि शरीरिणाम्}


\twolineshloka
{वर्धन्ते वर्धमानस्य वर्धते च कथं बलम्}
{निरासनं निष्कसनं मलानां च पृथक् पृथक्}


\twolineshloka
{कुतो वाऽयं प्रश्वसिति उच्छ्वसित्यपि वा पुनः}
{कं च देशमधिष्ठाय तिष्ठत्यात्माऽयमात्मनि}


\twolineshloka
{जीवः कथं वहति च चेष्टमानः कलेवरम्}
{किंवर्णं कीदृशं चैव निवेशयति वै मनः}


\twolineshloka
{याथातथ्येन भगवन्वक्तुमर्हसि मेऽनघ}
{इति सम्परिपृष्टोऽहं तेन विप्रेणि माधव}


\twolineshloka
{प्रत्यब्रवं महाबाहो यथाश्रुतमरिंदम}
{यथा स्वकोष्ठे प्रक्षिप्य भाण्डं भाण्डमना भवेत्}


\twolineshloka
{तथा स्वकाये प्रक्षिप्य मनोद्वारैरनिश्चलैः}
{आत्मानं तत्र मार्गेत प्रमादं परिवर्जयेत्}


\twolineshloka
{एवं सततमुद्युक्तः प्रीतात्मा नचिरादिव}
{आसादयति तद्ब्रह्म यद्दृष्ट्वा स्यात्प्रधानवित्}


\twolineshloka
{न त्वसौ चक्षुषा ग्राह्यो न च सर्वैरपीन्द्रियैः}
{मनसैव प्रदीपेन महानात्मा प्रदृश्यते}


\twolineshloka
{सर्वतः पाणिपादान्तः सर्वतोक्षिशिरोमुखः}
{सर्वतः श्रुतिमाँल्लोके सर्वमावृत्य तिष्ठति}


\twolineshloka
{जीवो निष्क्रान्तमात्मानं शरीरात्सम्प्रपश्यति}
{स तमुत्सृज्य देहं स्वं पारयेद्ब्रह्म केवलम्}


\twolineshloka
{आत्मानमालोकयति मनसा प्रहसन्निव}
{तदेवमाश्रयं कृत्वा मोक्षं याति ततो मयि}


\twolineshloka
{इदं सर्वरहस्यं ते मया प्रोक्तं द्विजोत्तम}
{आपृच्छे साधयिष्यामि गच्छ विप्र यथासुखम्}


\threelineshloka
{इत्युक्तः स तदा कृष्ण मया शिष्यो महातपाः}
{अगच्छत यथाकामं ब्राह्मणश्छिन्नसंशयः ॥वासुदेव उवाच}
{}


\twolineshloka
{इत्युक्त्वा स तदा वाक्यं मां पार्थ द्विजसत्तमः}
{मोक्षधर्माश्रितं सम्यक् तत्रैवान्तरधीयत}


\twolineshloka
{कच्चिदेतत्त्वया पार्थ श्रुत्मेकाग्रचेतसा}
{तदापि हि रथस्थस्त्वं श्रुतवानेतदेव हि}


\twolineshloka
{नैतत्पार्थ सुविज्ञेयं व्यामिश्रेणेति मे मतिः}
{नरेणाकृतसङ्गेन विशुद्धेनान्तरात्मना}


\twolineshloka
{सुरहस्यमिदं प्रोक्तं देवानां भरतर्षभ}
{कच्चित्त्विदं श्रुतं पार्थ मनुष्येणेह कर्हिचित्}


\twolineshloka
{न ह्येतच्छ्रोतुमर्होऽन्यो मनुष्यस्त्वामृतेऽनघ}
{नैतदन्येनि विज्ञेयं व्यामिश्रेणान्तरात्मना}


\twolineshloka
{क्रियावद्भिर्हि कौन्तेय देवलोकः समावृतः}
{न चैतदिष्टं देवानां मर्त्यैरुपरि वर्तनम्}


\twolineshloka
{परा हि सा गतिः पार्थ यत्तद्ब्रह्म सनातनम्}
{यत्रामृतत्वं प्राप्नोति त्यक्त्वा दुःखं सदा सुखी}


\twolineshloka
{इमं धर्मं समास्थाय येऽपि स्युः पापयोनयः}
{स्त्रियो वैश्यास्तथा शूद्रास्तेऽपि यान्ति परां गतिम्}


\twolineshloka
{किं पुनर्ब्रह्मणाः पार्थ क्षत्रिया वा बहुश्रुताः}
{स्वधर्मरतयो नित्यं ब्रह्मलोकपरायणः}


\twolineshloka
{हेतुमच्चैतदुद्दिष्टमुपायाश्चास्य साधने}
{सिद्धिं फलं च मोक्षश्च दुःखस्य च विनिर्णियः}


\twolineshloka
{नातः परं सुखं त्वन्यत्किंचित्स्याद्भरतर्षभ}
{श्रुतवाञ्श्रद्दधानश्च पराक्रान्तश्च पाण्डव}


\twolineshloka
{यः परित्यज्यते मर्त्यो लोकसारमसरवत्}
{एतैरुपायैः स क्षिप्रं परां गतिमवाप्नुते}


\twolineshloka
{एतावदेव वक्तव्यं नान्तो भूयोस्ति किञ्चन}
{षण्मासान्नित्ययुक्तस्य योगः पार्थ प्रवर्तते}


\chapter{अध्यायः २१}
\twolineshloka
{अत्राप्युदाहरन्तीममितिहासं पुरातनम्}
{दंपत्योः पार्थ संवादो योऽभकवद्भरतर्षभ}


\twolineshloka
{ब्राह्मणी ब्राह्मणं कंचिज्ज्ञानविज्ञानपारगम्}
{दृष्ट्वा विविक्त आसीनं भार्या भर्तारमब्रवीत्}


\twolineshloka
{कं नु लोकं गमिष्यामि त्वामहं पतिमाश्रिता}
{न्यस्तकर्माणमासीनं कीनाशमविचक्षणम्}


\twolineshloka
{भार्याः पतिकृताँल्लोकानाप्नुवन्तीति नः श्रुतम्}
{त्वामहं पतिमासाद्य कां गमिष्यामि वै गतिम्}


\twolineshloka
{एवमुक्तः स शान्तात्मा तामुवाच हसन्निव}
{सुभगे नाभ्यसूयामि वाक्यस्यास्य तवानघे}


\twolineshloka
{ग्राह्यं दृश्यं तथा श्राव्यं यदिदं कर्म विद्यते}
{एतदेव व्यवस्यन्ति कर्म कर्मेति कर्मिणः}


\twolineshloka
{मोहमेव निगच्छन्ति कर्मिणो ज्ञानवर्जिताः}
{नैष्कर्म्य न च लोकेऽस्मिन्मौर्तमित्युपलभ्यते}


\twolineshloka
{कर्मणा मनसा वाचा शुभं वा यदि वाऽशुभम्}
{जन्मादिमूर्तिभेदानां कर्म भूतेषु वर्तते}


\twolineshloka
{रक्षोभिर्वध्यमानेषु दृश्यश्राव्येषु कर्मसु}
{आत्मस्थमात्मना तेन दृष्टमायतनं मया}


\twolineshloka
{यत्र तद्ब्रह्म निर्द्वन्द्वं यत्र सोमः सहाग्निना}
{व्यवायं कुरुते नित्यं धीरो भूतानि धारयन्}


\twolineshloka
{यत्रि ब्रह्मादयो युक्तास्तदक्षरमुपासते}
{विद्वांसः सुव्रता यत्र शान्तात्मानो जितेन्द्रियाः}


\twolineshloka
{घ्राणेन न तदाघ्रेयं नास्वाद्यं चैव जिह्वया}
{स्पर्शनेन तदस्पृश्यं मनसा त्ववगम्यते}


\threelineshloka
{चक्षुषा न विषह्यं च यत्किञ्चिच्छ्रवणात्परम्}
{अगन्धमरसस्पर्शमरूपं शब्दवर्जितम्}
{यतः प्रवर्तते तन्त्रं यत्र चैतत्प्रतिष्ठितम्}


\twolineshloka
{प्राणोऽपानः समानश्च व्यानश्चोदान एव च}
{तत एव प्रवर्न्तते तदेव प्रविशन्ति च}


\twolineshloka
{समानव्यानयोर्मध्ये प्राणापानौ विचेरतुः}
{तस्मिन्सुप्ते प्रलीयेते समानो व्यान एव च}


\twolineshloka
{अपनाप्राणयोर्मध्ये उदानो व्याप्य तिष्ठति}
{तस्माच्छयानं पुरुषं प्राणापानौ न मुञ्चतः}


\twolineshloka
{प्राणोनोपहते यत्तु तमुदानं प्रचक्षते}
{तस्मात्तपो व्यवस्यन्ति तद्भवं ब्रह्मवादिनः}


\twolineshloka
{तेषामन्योन्यसक्तानां सर्वेषां देहचारिणाम्}
{अग्निर्वैश्वानरो मध्ये सप्तधा विहितोऽन्तरा}


\twolineshloka
{घ्राणं जिह्वा च चक्षुश्च त्वक्च श्रोत्रं च पञ्चमम्}
{मनो बुद्धिश्च सप्तैता जिह्वा वैश्वानरार्चिषः}


\twolineshloka
{घ्रेयं दृश्यं च पेयं च स्पृश्यं श्राव्यं तथैव च}
{मन्तव्यमवबोद्धव्यं ताः सप्त समिधो मताः}


\twolineshloka
{घ्राता भक्षयिता द्रष्टा स्प्रष्टा श्रोता च पञ्चमः}
{मन्ता बोद्धा च स्प्तैते भवन्ति परमर्त्विजः}


\twolineshloka
{घ्रेये पेये च दृश्ये च स्पृश्ये श्राव्ये तथैव च}
{मन्तव्येऽप्यथ बोद्धव्ये सुभगे पश्य सर्वदा}


\twolineshloka
{हवींष्याग्निषु होतारः सप्तधा सप्तसप्तसु}
{सम्यक्प्रक्षिप्य विद्वांसो जनयन्ति स्वयोनिषु}


\twolineshloka
{पृथिवी वायुराकाशमापो ज्योतिश्च पञ्चमम्}
{मनो बुद्धिश्च सप्तैता योनिरित्येव शब्दिताः}


\twolineshloka
{हविर्भूतगुणाः सर्वे प्रविशन्त्यग्नियं मुखम्}
{अन्तर्वासमुषित्वा च जायन्ते स्वासु योनिषु}


\twolineshloka
{तत्रैव च निरुध्यन्ते प्रलये भूतभावने}
{ततः संजायते गन्धस्ततः संजायते रसः}


\threelineshloka
{ततः संजायते रूपं ततः स्पर्शोऽभिजायते}
{ततः संजायते शब्दः संशयस्तत्र जायते}
{ततः संजायते निष्ठा जन्मैतत्सप्तधा विदुः}


\twolineshloka
{अनेनैव प्रकारेण प्रगृहीतं पुरातनैः}
{पूर्णाहुतिभिरापूर्णास्ते पूर्यन्ते हि तेजसा}


\chapter{अध्यायः २२}
\twolineshloka
{अत्राप्युदाहरन्तीममितिहासं पुरातनम्}
{निबोध दशहोतृणां विधानमिह यादृशम्}


\twolineshloka
{श्रोत्रं त्वक्चक्षुषी जिह्वा नासिका चरणौ करौ}
{उपस्थं पायुरिति वाग्घोतृणि दश भामिनि}


\twolineshloka
{शब्दस्पर्शौ रूपरसौ गन्धो वाक्यं क्रिया गतिः}
{रेतोमूत्रपुरीषाणां त्यागो दश हवींषि च}


\twolineshloka
{दिशो वायू रविश्चन्द्रः पृथ्व्यग्नी विष्णुरेव च}
{इन्द्रः प्रजापतिर्मित्रमग्नयो दश भामिनि}


\twolineshloka
{दशेन्द्रियाणि होतृणि हवींषि दश भामिनि}
{विषया नाम समिधो हूयन्ते तु दशाग्निषु}


\twolineshloka
{चित्तं स्रुवश्च वित्तं च पवित्रं ज्ञानमुत्तमम्}
{सुविभक्तमिदं पूर्वं जगदासीदिति श्रुतम्}


\threelineshloka
{`ततो विविक्ता वित्तासीत्सा वित्तं पर्यवेक्षते}
{'सर्वमेवात्र विज्ञेयं चित्ते ज्ञानमवेक्षता}
{रेतः शरीरभृत्काये विज्ञाता तु शरीरभृत्}


\twolineshloka
{शरीरभृद्गार्हपत्यस्तस्मादग्निः प्रणीयते}
{मनश्चाहवनीयस्तु तस्मिन्प्रक्षिप्यते हविः}


\threelineshloka
{ततो वाचस्पतिर्जज्ञे तं मनः पर्यवेक्षते}
{रूपं भवति वै वक्त्रं तदनुद्रवते मनः ॥ब्राह्मण्युवाच}
{}


\twolineshloka
{कस्माद्वागभवत्पूर्वं कस्मात्पश्चान्मनोऽभवत्}
{मनसा चिन्तितं पूर्वं वाक्यं समभिपद्यते}


\threelineshloka
{केन विज्ञानयोगेन मतिश्चित्तं समास्थिता}
{समुन्नीता नाध्यगच्छत्को वै तां प्रतिबाधते ॥ब्राह्मण उवाच}
{}


\twolineshloka
{तन्मनस्थः पतिर्भूत्वा तस्मात्प्रेहन्निवायति}
{तां मतिं मनसः प्राहुर्मनस्तस्मादपेक्षते}


\twolineshloka
{प्रश्नं तु वाङ्मनसयोर्यस्मात्त्वमनुपृच्छसि}
{तस्मात्ते वर्तयिष्यामि तयोरेव समाह्वयम्}


\twolineshloka
{उभे वाङ्मनसी गत्वा भूतात्मानमपृच्छताम्}
{आवयोः श्रेष्ठमाचक्ष्व च्छिन्धि नौ संशयं विभो}


\threelineshloka
{मन इत्येवि भगवांस्तदा प्राह सरस्वतीम्}
{अहं वै कामधुक्तुभ्यमिति तं प्राह वागथ ॥ब्राह्मण उवाच}
{}


\twolineshloka
{स्थावरं जङ्गमं चैव विद्ध्युभे मनसी मम}
{स्थावरं मत्सकाशे वै जङ्गमं विषये तव}


\twolineshloka
{यस्तु ते विषये गच्छन्मन्त्रो वर्णः स्वरोपि वा}
{तन्मनो जङ्गमं नाम तस्मादसि गरीयसी}


\twolineshloka
{तस्माद्भवितुमर्हामि स्वयमभ्येत्य शोभने}
{तस्मादुच्छ्वासमासाद्य प्रवक्ष्यामि सरस्वति}


\threelineshloka
{प्राणापानावन्तरे यद्वाग्वै नित्यं स्म तिष्ठति}
{प्रीयमाणा महाभागे विना प्राणांश्च मामपि}
{प्रजापतिमुपाधावत्प्रसीद भगवन्निति}


\twolineshloka
{ततः प्राणः प्रादुरभूद्वाचमाप्याययन्पुनः}
{तस्मादुच्छ्वासमासाद्य न वाग्वदति कर्हिचित्}


\twolineshloka
{घोषिणी जातनिर्घोषा नित्यमेव प्रवर्तते}
{तयोरपि च घोषिण्या निर्घोर्षैव गरीयसी}


\twolineshloka
{गौरिव प्रस्रवत्यर्थान्रसमुत्तमशालिनी}
{सततं स्यन्दते ह्येषा शाश्वतं ब्रह्मवादिनी}


\threelineshloka
{दिव्यादिव्यप्रभावेन भारती गौः शुचिस्मिते}
{एतयोरन्तरं पश्य सूक्ष्मयोर्यतमानयोः ॥ब्राह्मण्युवाच}
{}


\threelineshloka
{अनुत्पन्नेषु वाक्येषु चोद्यमाना सिसृक्षया}
{किंन्नु पूर्वं तदा देवी व्याजहार सरस्वती ॥ब्राह्मण उवाच}
{}


\twolineshloka
{प्राणेन य सम्भवते शरीरेप्राणादपानं प्रतिपद्यते च}
{उदानभूता च विसृज्य देहंव्यानेन सर्वं दिवमावृणोति}


\twolineshloka
{ततः समाने प्रतितिष्ठतीहइत्येव पूर्वं प्रजजल्प वागपि}
{तस्मान्मनः स्थावरत्वाद्विशिष्टंतथा देवी जङ्गमत्वाद्विशिष्टाः}


\chapter{अध्यायः २३}
\twolineshloka
{अत्राप्युदाहरन्तीममितिहासं पुरातनम्}
{सुभगे सप्तहोतॄणां विधानमिह यादृशम्}


\twolineshloka
{घ्राणश्चक्षुश्च जिह्वा च त्वक् श्रोत्रं चैव पञ्चमम्}
{मनो बुद्धिश्च सप्तैते होतारः पृथगाश्रिताः}


\threelineshloka
{सूक्ष्माकाशे समं प्राप्ते न पश्यन्तीतरेतरम्}
{एतद्वै सप्तहोतृत्वं स्वभावाद्विद्धि शोभने ॥ब्राह्मण्युवाच}
{}


\threelineshloka
{सूक्ष्मे तु काशे सम्प्राप्ते कथं नान्योन्यदर्शिनः}
{कथं स्वभावाद्भगवन्नेतदाचक्ष्व मे प्रभो ॥ब्राह्मण उवाच}
{}


\twolineshloka
{गुणज्ञानेषु विज्ञानं गुणज्ञानामभिज्ञता}
{परस्परं गुणानेते नाभिजानन्ति कर्हिचित्}


\twolineshloka
{जिह्वा चक्षुस्तथा श्रोत्रं त्वङ्मनो बुद्धिरेव च}
{न गन्धानधिगच्छन्ति घ्राणस्तानधिगच्छति}


\twolineshloka
{घ्राणं चक्षुस्तथा क्षोत्रं त्वङ्मनो बुद्धिरेव च}
{न रसानधिगच्छन्ति जिह्वा तानधिगच्छति}


\twolineshloka
{घ्राणं जिह्वा तथा श्रोत्रं त्वङ्मनो बुद्धिरेव च}
{न रूपाण्यधिगच्छन्ति चक्षुस्तान्यधिगच्छति}


\twolineshloka
{घ्राणं जिह्वा ततश्चक्षुः श्रोत्रं बुद्धिर्मनस्तथा}
{न स्पर्शानधिगच्छन्ति त्वक्च तानधिगच्छति}


\twolineshloka
{घ्राणं जिह्वा च चक्षुश्च त्वङ्मनो बुद्धिरेव च}
{न शब्दानधिगच्छन्ति श्रोत्रं तानधिगच्छति}


\twolineshloka
{घ्राणं जिह्वा च चक्षुश्च त्वक् श्रोत्रं बुद्धिरेव च}
{सङ्कल्पान्नाधिगच्छन्ति मनस्तानधिगच्छति}


\twolineshloka
{घ्राणं जिह्वा च चक्षुश्च त्वक् श्रोत्रं मन एव च}
{न निष्ठामधिगच्छन्ति बुद्धिस्तामधिगच्छति}


\threelineshloka
{अत्राप्युदाहरन्तीममितिहासं पुरातनम्}
{इन्द्रियाणां च संवादं मनसश्चैव भामिनि ॥मन उवाच}
{}


\twolineshloka
{नाघ्राति मामृते घ्राणं रसं जिह्वा न वेत्ति च}
{रूपं चक्षुर्न गृह्णाति त्वक् स्पर्सं नावबुध्यते}


\twolineshloka
{न श्रोत्रं बुध्यते शब्दं मया हीनं कथञ्चन}
{प्रवरं सर्वबूतानामहमस्मि सनातनम्}


\twolineshloka
{अगाराणीव शून्यानि शान्तार्चिष इवाग्नयः}
{इन्द्रियाणि न भासन्ते मया हीनानि नित्यशः}


\threelineshloka
{काष्ठानीवार्द्रशुष्काणि यतमानैरपीन्द्रियैः}
{गुणार्थान्नाधिगच्छन्ति मामृते सर्वजन्तवः ॥इन्द्रियाण्यूचुः}
{}


\twolineshloka
{एवमेतद्भवेत्सत्यं यथैतन्मन्यते भवान्}
{ऋतेऽस्मानस्मदर्थांस्त्वं भोगान्भुङ्क्ते भवान्यदि}


\twolineshloka
{यद्यस्मासु प्रलीनेषु तप्रणं प्राणधारणम्}
{भोगान्भुङ्क्ते भवान्सत्यं यथैतन्मन्यते तथा}


\twolineshloka
{अथवाऽस्मासु लीनेषु तिष्ठत्सु विषयेषु च}
{यदि सङ्कल्पमात्रेण भुङ्क्ते भोगान्यथार्थवत्}


\twolineshloka
{अथ चेन्मन्यसे सिद्धिमस्मदर्थेषु नित्यदा}
{घ्राणेन रूपमादत्स्व रसमादत्स्व चक्षुषा}


\twolineshloka
{श्रोत्रेण गन्धानादत्स्व स्पर्शानादत्स्व जिह्वया}
{त्वचा च शब्दमादत्स्व बुद्ध्या स्पर्शमथापि च}


\twolineshloka
{बलवन्तो ह्यनियमा नियमा दुर्बलीयसाम्}
{भोगानपूर्वानादत्स्व नोच्छिष्टं भोक्तुमर्हति}


\twolineshloka
{यथा हि शिष्यः शास्तारं श्रुत्यर्थमभिधावति}
{ततः श्रुतमुपादाय श्रुतार्थमुपतिष्ठति}


\twolineshloka
{विषयानेवमस्माभिर्दर्शितानभिमन्यसे}
{अनुभूतानतीतांश्च स्वप्ने जागरणे तथा}


\twolineshloka
{वैमनस्यं गतानां च जन्तूनामल्पचेतसाम्}
{अस्मदर्थे कृते दृश्यते प्राणधारणम्}


\twolineshloka
{बहूनपि हि सङ्कल्पान्मत्वा स्वप्नानुपास्य च}
{बुभुक्षया पीड्यमानो विषयानेन धावति}


\twolineshloka
{अगारमद्वारमिव प्रविश्यसङ्कल्पभोगान्विषयानविन्दन्}
{प्राणक्षये शान्तिमुपैति नित्यंदारुक्षयेऽग्निर्ज्वलितो यथैव}


\twolineshloka
{कामं तु नष्टेषु गुणेषु सङ्गःकामं च नान्योन्यगुणोपलब्धिः}
{अस्मान्विना नास्ति तपोपलब्धि-स्तामप्यृते त्वां न भजेत्प्रहर्षः}


\chapter{अध्यायः २४}
\twolineshloka
{अत्राप्युदाहरन्तीममितिहासं पुरातनम्}
{सुभगे पञ्चहोतॄणां विधानमिह यादृशम्}


\threelineshloka
{प्राणापानवुदानश्च समानो व्यान एव च}
{पञ्चहोतॄंस्तथैतान्वै परं भावं विदुर्बुधाः ॥ब्राह्मण्युवाच}
{}


\threelineshloka
{स्वबावात्सप्तहोतार इति मे पूर्विका मतिः}
{यथा वै पञ्च होतारः परो भावस्तदुच्यताम् ॥ब्राह्मण उवाच}
{}


\twolineshloka
{प्राणेन सम्भृतो वायुरपानो जायते ततः}
{अपाने सम्भृतो वायुस्ततो व्यानः प्रवर्तते}


\twolineshloka
{व्यानेन सम्भृतो वायुस्ततोदानः प्रवर्तते}
{उदाने सम्भृतो वायुः समानो नाम जायते}


\threelineshloka
{तेऽपृच्छन्त पुरो गत्वा पूर्वजातं पितामहम्}
{यो नः श्रेष्ठस्तमाचक्ष्व स नः श्रेष्ठो भविष्यति ॥ब्रह्मोवाच}
{}


\threelineshloka
{यस्मिन्प्रलीने प्रलयं व्रजन्तिसर्वे प्राणाः प्राणभृतां शरीरे}
{यस्मिन्प्रवृत्ते च पुनश्चरन्तिस वै श्रेष्ठो गच्छत यत्र कामः ॥प्राण उवाच}
{}


\threelineshloka
{मयि प्रलीने प्रलयं व्रजन्तिसर्वे प्राणाः प्राणभृतां शरीरे}
{मयि प्रवृत्ते च पुनश्चरन्तिश्रेष्ठो ह्यहं पश्यत मां प्रलीनम् ॥ब्राह्मण उवाच}
{}


\twolineshloka
{प्रामः प्रालीयत ततः पुनश्च प्रचचार ह}
{समानश्चाप्युदानश्च वचो ब्रूतां पुनः शुभे}


\fourlineindentedshloka
{न त्वं सर्वमिदं व्याप्य तिष्ठसीह यथा वयम्}
{न त्वं श्रेष्ठो हि नः प्राण अपानो हि वशे तव}
{प्रचचार पुनः प्राणस्ततोऽपानोऽभ्यभाषत ॥उपान उवाच}
{}


\twolineshloka
{मयि प्रलीने प्रलयं व्रजन्तिसर्वे प्राणाः प्राणभृतां शरीरेमयि प्रवृत्ते च पुनश्चरन्तिश्रेष्ठो ह्यहं पश्यत मां प्रलीनम् ॥ब्राह्मण उवाच}
{}


\twolineshloka
{व्यानश्च तमुदानश्चि भाषमाणमथोचतुः}
{अपान न त्वं श्रेष्ठोसि प्राणो हि वशगस्तव}


\twolineshloka
{अपानः प्रचचाराथ व्यानस्तं पुनरब्रवीत्}
{श्रेष्ठोऽहमस्मि सर्वेषां श्रूयतां येन हेतुना}


\threelineshloka
{मयि प्रलीने प्रलयं व्रजन्तिसर्वे प्राणाः प्राणभृतां शरीरे}
{मयि प्रवृत्ते च पुनश्चरन्तिश्रेष्ठो ह्यहं पश्यत मां प्रलीनम् ॥ब्राह्मण उवाच}
{}


\twolineshloka
{प्रालीयत ततो व्यानः पुनश्च प्रचचार ह}
{प्राणापानावुदानस्च समानश्च तमब्रुवन्}


\threelineshloka
{न त्वं श्रेष्ठोसि नो व्यान समानस्तु वशे तव}
{प्रचचार पुनर्व्यानः समानः पुनरब्रवीत्}
{श्रेष्ठोऽहमस्मि सर्वेषां श्रूयतां येन हेतुना}


\twolineshloka
{मयि प्रलीने प्रलयं व्रजन्तिसर्वे प्राणाः प्राणभृतां शरीरे}
{मयि प्रवृत्ते च पुनश्चरन्तिश्रेष्ठो ह्यहं पश्यत मां प्रलीनम्}


\threelineshloka
{`ततः समानः प्रालिल्ये पुनश्च प्रचचार ह}
{प्राणापानावुदानस्च व्यानश्चैव तमब्रवीत्}
{न त्वं समान श्रेष्ठोसि व्यान एव वशे तव ॥'}


\twolineshloka
{समानः प्रचचाराथ उदानस्तमुवाच ह}
{श्रेष्ठोऽहमस्मि सर्वेषां श्रूयतां येन हेतुना}


\twolineshloka
{मयि प्रलीने प्रलयं व्रजन्तिसर्वे प्राणाः प्राणभृतां शरीरे}
{मयि प्रवृत्ते च पुनश्चरन्तिश्रेष्ठो ह्यहं पश्यत मां प्रलीनम्}


\fourlineindentedshloka
{ततः प्रालीयतोदानः पुनश्च प्रचचार ह}
{प्राणापानौ समानश्च व्यानश्चैव तमब्रुवन्}
{उदानि न त्वं श्रेष्ठोसि व्यानि एव वशे तव ॥ब्राह्मण उवाच}
{}


\twolineshloka
{ततस्तानब्रवीत्सर्वान्स्मयमानः प्रजापतिः}
{सर्वे श्रेष्ठा न च श्रेष्ठाः सर्वे चान्योन्यकाङ्क्षिणः}


\twolineshloka
{सर्वे स्वविषये श्रेष्ठाः सर्वे चान्योन्यधर्मिणः}
{इति तानब्रवीत्सर्वान्समवेतान्प्रजापतिः}


\twolineshloka
{एकः स्थिरश्चराश्चान्ये विशेषात्पञ्च वायवः}
{एक एव च सर्वात्मा बहुधाऽप्युपचीयते}


\twolineshloka
{परस्परस्य सुहृदो भावयन्तः परस्परम्}
{स्वस्ति व्रजत भद्रं वो धारयध्वं परस्परम्}


\chapter{अध्यायः २५}
\threelineshloka
{अत्राप्युदाहरन्तीममितिहासं पुरातनम्}
{नारदस्य च संवादमृषेर्देवमतस्य च ॥देवमत उवाच}
{}


\threelineshloka
{जन्तोः संजायमानस्य किंनु पूर्वं प्रवर्तते}
{प्राणोऽपानः समानो वा व्यानो वोदान एव च ॥नारद उवाच}
{}


\threelineshloka
{येनायं सृज्यते जन्तुस्ततोऽन्यः पूर्वमेति तम्}
{प्राणद्वन्द्वं हि विज्ञेयं तिर्यगूर्ध्वमधश्च यत् ॥देवमत उवाच}
{}


\threelineshloka
{केनायं सृज्यते जन्तुः कश्चान्यः पूर्वमेति तम्}
{प्राणद्वन्द्वं च मे ब्रूहि तिर्यगूर्ध्वमधश्च यत् ॥नारद उवाच}
{}


\twolineshloka
{सङ्कल्पाज्जायते हर्षः शब्दादपि च जायते}
{रसात्संजायते चापि रूपादपि च जायते}


\threelineshloka
{`स्पर्शात्संजायते चापि गन्धादपि च जायते}
{'शुक्राच्छोणितसंसृष्टात्पूर्वं प्राणः प्रवर्तते}
{प्राणेन विकृते शुक्रे ततोऽपानः प्रवर्तते}


\twolineshloka
{शुक्रात्संजायते चापि रसादपि च जायते}
{एतद्रूपमुदानस्य हर्षो मिथुनमन्तरा}


\twolineshloka
{कामात्संजायते शुक्रं शुक्रात्संजायते रजः}
{समानव्यानजनिते सामान्ये शुक्रशोणिते}


\twolineshloka
{प्राणापानाविदं द्वन्द्वमवाक् चोर्ध्वं च गच्छतः}
{व्यानः समानश्चैवोभौ तिर्यग्द्वन्द्वत्वमुच्यते}


\twolineshloka
{अग्निर्वै देवताः सर्वा इति देवस्य शासनात्}
{संजायते हि प्राणेषु ज्ञानं बुद्धिसमन्वितम्}


\twolineshloka
{तस्य धूमस्तमोरूपं रजो भस्म सुतेजसः}
{सर्वं संजायते तस्य यत्र प्रक्षिप्यते हविः}


\twolineshloka
{हविः समानो व्यानश्च इति यज्ञविदो विदुः}
{प्राणापानावाज्यभागौ तयोर्मध्ये हुताशनः}


\twolineshloka
{एतद्रूपमुदानस्य परमं ब्राह्मणा विदुः}
{निर्द्वन्द्वमिति यत्त्वेतत्तन्मे निगदतः शृणुः}


\twolineshloka
{अहोरात्रमिदं द्वन्द्वं तयोर्मध्ये हुताशनः}
{एतद्रूपमुदानस्य परमं ब्राह्मणा विदुः}


\twolineshloka
{`उभे सत्यानृते द्वन्द्वं तयोर्मध्ये हुताशनः}
{एतद्रूपमुदानस्य परमं ब्राह्मणा विदुः ॥'}


\twolineshloka
{सच्चासच्चैव तद्द्वन्द्वं तयोर्मध्ये हुताशनः}
{एतद्रूपमुदानस्य परमं ब्राह्मणा विदुः}


\twolineshloka
{`उभे शुभाशुभे द्वन्द्वं तयोर्मध्ये हुताशनः}
{एतद्रूपमुदानस्य परमं ब्राह्मणा विदुः ॥'}


\twolineshloka
{ऊर्ध्वं समानो व्यानश्च व्यस्यते कर्म तेन तत्}
{द्वितीयं तु समानेन पुनरेव व्यवस्यते}


\twolineshloka
{शान्त्यर्थं वामदेव्यं च शान्तिर्ब्रह्म सनातनम्}
{एतद्रूपमुदानस्य परमं ब्राह्मणा विदुः}


\chapter{अध्यायः २६}
\twolineshloka
{अत्राप्युदाहरन्तीममितिहासं पुरातनम्}
{चातुर्होत्रविधानस्य विदानमिह यादृशम्}


\twolineshloka
{तस्य सर्वस्य विधिवद्विधानमुपदेक्ष्यते}
{शृणु मे गदतो भद्रे रहस्यमिदमुद्भुतम्}


\twolineshloka
{करणं कर्म कर्ता च मोक्ष इत्येव भामिनि}
{चत्वार एते होतारो यैरिदं जगदावृतम्}


\threelineshloka
{होतॄणां साधनं चैव शृणु सर्वमशेषतः}
{घ्राणं जिह्वा च चक्षुश्च त्वक्च श्रोत्रं च पञ्चमम्}
{मनो बुद्धिश्च सप्तैते ज्ञेयाः कारणहेतवः}


\twolineshloka
{गन्धो रसश्च रूपं च शब्दः स्पर्शश्च पञ्चमः}
{मन्तव्यमवबोद्धव्यं सप्तैते कर्महेतवः}


\twolineshloka
{घ्राता भक्षयिता द्रष्टा स्प्रष्टा श्रोता च पञ्चमः}
{मन्ता बोद्धा च सप्तैते विज्ञेयाः कर्तृहेतवः}


\twolineshloka
{स्वगुणान्भक्षयन्त्येते गुणवन्तः शुभाशुभान्}
{असन्तो निर्गुणाश्चैते सप्तैते मोक्षहेतवः}


\twolineshloka
{विदुषां बुध्यमानानां स्वंस्वं स्थानं यथाविधि}
{गुणास्ते देवता भूत्वा सततं भुञ्जते हविः}


\twolineshloka
{`अदन्हवींषि चान्नानि ममत्वेन विहन्यते}
{'आत्मार्थे पाचयन्नन्नं ममत्वेनोपहन्यते}


\twolineshloka
{अभक्ष्यभक्षणं चैव मद्यपानं च हन्ति तम्}
{स चान्नं हन्ति तं चान्नं स हत्वा हन्यते पुनः}


\twolineshloka
{हन्ता ह्यन्नमिदं विद्वान्पुनर्जनयतीश्वरः}
{न चान्नाज्जायते तस्मिन्सूक्ष्मो नाम व्यतिक्रमः}


\twolineshloka
{मनसा मन्यते यच्च यच्च वाचा निगद्यते}
{श्रोत्रेण श्रूयते यच्च चक्षुषा यच्च दृश्यते}


\twolineshloka
{स्पर्शेन स्पृश्यते यच्च घ्राणेन घ्रायते च यत्}
{मनस्येतानि संयम्य हवींष्येतानि सर्वशः}


\threelineshloka
{गुणवत्पावको मह्यं दीप्यतेऽन्तःशरीरगः}
{योगयज्ञः प्रवृत्तो मे ज्ञानं ब्रह्ममयो हविः}
{प्राणस्तोत्रोऽपानशस्त्रः सर्वत्यागसुदक्षिणः}


% Check verse!
कर्ताऽनुमन्ता ब्रह्मात्मा होताऽध्वर्युः कृतस्तुतिःऋतं प्रशास्ता तच्छस्त्रमपवर्गोऽस्य दक्षिणा
\twolineshloka
{ऋचश्चाप्यत्र शंसन्ति नारायणविदो जनाः}
{नारायणाय देवाययदविन्दन्पशून्पुरा}


\twolineshloka
{तत्र सामानि गायन्ति तत्र चाहुर्निदर्शनम्}
{देवं नारायणं भीरु सर्वात्मानं निबोध तम्}


\chapter{अध्यायः २७}
\twolineshloka
{एकः शास्ता न द्वितीयोस्ति शास्तायो हृच्छयस्तमहमनु ब्रवीमि}
{तेनैव युक्तः प्रवणादिवोदकंयथा नियुक्तोस्मि तथा वहामि}


\twolineshloka
{एको गुरुर्नास्ति ततो द्वितीयोयो हृच्छयस्तमहमनु ब्रवीमि}
{तेनानुशिष्टा गुरुणा सदैवपुरा हता दानवाः सर्व एव}


\twolineshloka
{एको बन्धुर्नास्ति ततो द्वितीयोयो हृच्छयस्तमहमनु ब्रवीमि}
{तेनानुशिष्टा बान्धवा बन्धुमन्तःसप्तर्षयः पार्थ दिवि प्रभान्ति}


\twolineshloka
{एकः श्रोता नास्ति ततो द्वितीयोयो हृच्छयस्तमहमनु ब्रवीमि}
{तस्मिन्गुरौ गुरुवासं निरुष्यटशक्रो गतः सर्वलोकामरत्वम्}


\twolineshloka
{एको द्वेष्टा नास्ति ततो द्वितीयोयो हृच्छयस्तमहमनु ब्रवीमि}
{तेनानुशिष्टा गुरुणा सदैवलोके द्विष्टाः पन्नगाः सर्व एव}


\twolineshloka
{अत्राप्युदाहरन्तीममितिहासं पुरातनम्}
{प्रजापतौ पन्नगानां देवर्षीणां च संविदम्}


\twolineshloka
{देवर्षयश्च नागाश्चाप्यसुराश्च प्रजापतिम्}
{पर्यपृच्छन्नुपासीनं श्रेयोः नः प्रोच्यतामिति}


\twolineshloka
{तेषां प्रोवाच भगवाञ्श्रेयः समनुपृच्छताम्}
{ओमित्येकाक्षरं ब्रह्म ते श्रुत्वा प्राद्रवन्दिशः}


\twolineshloka
{तेषां प्रद्रवमाणानामुपदेशं च शृण्वताम्}
{सर्पाणां दंशने भावः प्रवृत्तः पूर्वमे तु}


\twolineshloka
{असुराणां प्रवृत्तस्तु दंभभावः स्वभावजः}
{दानं देवा व्यवसिता दममेव महर्षयः}


\twolineshloka
{एकं शास्तारमासाद्य शब्देनैकेन संस्कृताः}
{नानाव्यवसिताः सर्वे सर्पदेवर्षिदानवाः}


\twolineshloka
{शृणोत्ययं प्रोच्यमानं गृह्णाति च यथातथम्}
{पृच्छतस्तदतो भूयो गुरुरन्यो न विद्यते}


\twolineshloka
{तस्य चानुमते कर्म ततः पश्चात्प्रवर्तते}
{गुरुर्बन्धुश्च शास्ता च द्वेष्टा च हृदि संश्रिताः}


\twolineshloka
{पापेन विचरँल्लोके पापचारी भवत्ययम्}
{शुभेन विचरँल्लोके शुभचारी भवत्युत}


\twolineshloka
{कामचारी तु कामेन य हन्द्रियसुखे रतः}
{ब्रह्मचारी सदैवैष य इन्द्रियजये रतः}


\twolineshloka
{अपेतव्रतकर्मा तु केवलं ब्रह्मणि स्थितः}
{ब्रह्मभूतश्चरँल्लोके ब्रह्मचारी भवत्ययम्}


\twolineshloka
{ब्रह्मैव समिधस्तस्य ब्रह्माग्निर्ब्रह्मसंस्तरः}
{आपो ब्रह्म गुरुर्ब्रह्म स ब्रह्मणि समाहितः}


\twolineshloka
{एतदेवेदृशं सूक्ष्मं ब्रह्मचर्यं विदुर्बुधाः}
{विदित्वा चान्वपद्यन्त क्षेत्रज्ञेनानुदर्शिताः}


\chapter{अध्यायः २८}
\twolineshloka
{संकल्पदंशमशकं शोकहर्षहिमातपम्}
{मोहान्धकारतिमिरं लोभव्याधिसरीसृपम्}


\threelineshloka
{विषयैकात्ययाध्वानं कामक्रोधकिरातकम्}
{तदतीत्य महादुर्गं प्रविष्टोस्मि महद्वनम् ॥ब्राह्मण्युवाच}
{}


\threelineshloka
{क्व तद्वनं महाप्राज्ञ के वृक्षाः सरितश्च काः}
{कियन्तः पर्वताश्चैव कियत्यध्यनि तद्वनम् ॥ब्राह्मण उवाच}
{}


\twolineshloka
{नैतदस्ति पृथग्भावः किञ्चिनद्यत्ततः सुखम्}
{नैतदस्त्यपृथग्भावः किञ्चिद्दुःखतरं ततः}


\twolineshloka
{तस्माद्ध्रस्वतरं नास्ति न ततोस्ति महत्तरम्}
{नास्ति तस्माद्दुःखतरं नास्त्यन्यत्तत्समं सुखम्}


\twolineshloka
{न तत्राविश्यि शोचन्ति न प्रहृष्यन्ति च द्विजाः}
{न च बिभ्यति केभ्यश्चिन्नैभ्यो बिभ्यति केचन}


\twolineshloka
{तस्मिन्वने सप्त महाद्रुमाश्चफलानि सप्तातिथयश्च सप्त}
{सप्ताश्रमाः सप्त समाधयश्चदीक्षाश्च सप्तैतदरण्यरूपम्}


\twolineshloka
{पञ्चवर्णानि दिव्यानि पुष्पाणि च फलानि च}
{सृजन्तः पादपास्तत्र व्याप्य तिष्ठन्ति तद्वन्म्}


\twolineshloka
{सुवर्णानि द्विवर्णानि पुष्पाणि च फलानि च}
{सृजन्तः पादपास्तत्र व्याप्य तिष्ठन्ति तद्वनम्}


\twolineshloka
{`शङ्कराणि त्रिवर्णानि पुष्पाणि च फलानि च}
{सृजन्तः पादपास्तत्र व्याप्य तिष्ठन्ति तद्वनम्'}


\twolineshloka
{सुरभीणि द्विवर्णानि पुष्पाणि च फलानि च}
{सृजन्तः पादपास्तत्र व्याप्य तिष्ठन्ति तद्वनम्}


\twolineshloka
{सुरभीण्येकवर्णानि पुष्पाणि च फलानि च}
{सृजन्तः पादपास्तत्र व्याप्य तिष्ठन्ति तद्वनम्}


\twolineshloka
{बहून्यव्यक्तवर्णानि पुष्पाणि च फलानि च}
{विसृजन्तौ महावृक्षौ तद्वनं व्याप्य तिष्ठतः}


\twolineshloka
{एको वह्निः सुमना ब्राह्मणोत्रपञ्चेन्द्रियाणि समिधश्चात्र सन्ति}
{तेभ्यो वृक्षाःक सप्त फलन्ति दीक्षागुणाः फलान्यतिथयः फलाशाः}


\twolineshloka
{आतिथ्यं प्रतिगृह्णन्ति तत्र सप्त महर्षयः}
{अर्चितेषु प्रलीनेषु तेष्वन्यद्रोचते वनम्}


\twolineshloka
{प्रज्ञावृक्षं मोक्षफलं शान्तिच्छायासमन्वितम्}
{ज्ञानाश्रयं तृप्तितोयमन्तःक्षेत्रज्ञभास्करम्}


\twolineshloka
{येऽधिगच्छन्ति तत्सन्तस्तेषां नास्ति पुनर्भवः}
{ऊर्ध्वं चाधश्च तिर्यक्च तस्य नान्तोऽधिगम्यते}


\twolineshloka
{सप्त स्त्रियस्तत्र चरन्ति सत्या-स्त्ववाङ्मुखा भानुमत्यो जनित्र्यः}
{ऊर्ध्वं रसानाददते प्रजाभ्यःसर्वान्यथा नित्यमनित्यता च}


\twolineshloka
{तत्रैव प्रतितिष्ठन्ति पुनस्तत्रोदयन्ति च}
{सप्त सप्तर्षयः सिद्धा वसिष्ठप्रमुखैः सह}


\twolineshloka
{यशो वर्चो भगश्चैव विजयः सिद्धतेजसः}
{एतमेवानुवर्तन्ते सप्त ज्योतींषि भास्करम्}


\twolineshloka
{ऋषयः पर्वताश्चैव सन्ति तत्र समासतः}
{नद्यश्च परितो वारि वहन्त्यो ब्रह्मिसम्भवम्}


\twolineshloka
{नदीनां सङ्गमश्चैव वैताने समुपहरे}
{स्वात्मतृप्ता यतो यान्ति साक्षादेव पितामहम्}


\twolineshloka
{कृशाशाः सुव्रताः शान्तास्तपसा दग्धकिल्बिषाः}
{आत्मन्यात्मानमावेश्य ब्राह्मणास्तमुपासते}


\twolineshloka
{शमिमप्यत्र शंसन्ति विद्यारण्यविदो जनाः}
{तदरण्यमभिप्रेत्य यथातत्वमजायत}


\twolineshloka
{एतदेवेदृशं पुण्यमरण्यं ब्राह्मणा विदुः}
{विदित्वा चानुतिष्ठन्ति क्षेत्रज्ञेनानुदर्शिना}


\chapter{अध्यायः २९}
\twolineshloka
{गन्धान्न जिघ्रामि रासान्न वेद्मिरूपं न पश्यामि न च स्पृशामि}
{न चापि शब्दान्विविधाञ्शृणोमिन चापि सङ्कल्पमुपैमि कञ्चित्}


\twolineshloka
{अर्थानिष्टान्कामयते स्वभावःसर्वान्द्वेष्यान्प्रद्विषते स्वभावः}
{कामद्वेषानुद्भवतः स्वभावा-त्प्राणापानौ जन्तुदेहान्निवेश्य}


\twolineshloka
{तेभ्यश्चान्यांस्तेषु नित्यांश्च भावा-न्भूतात्मानं अक्षयेऽहं शरीरे}
{तस्मिंस्तिष्ठन्नास्मि सक्तः कथंचि-त्कामक्रोधाभ्यां जरया मृत्युना च}


\twolineshloka
{अकामयानस्य च सर्वकामा- नविद्विषाणस्य च सर्वदोषान्}
{न मे स्वभावेषु भवन्ति लेपा-स्तोयस्य बिन्दोरिव पुष्करेषु}


\twolineshloka
{नित्यस्य चैतस्य भवन्ति नित्यानिरीक्ष्यमाणस्य बहून्स्वभावान्}
{न सज्जते कर्मसु भोगजालंदिवीव सूर्यस्य मयूखजालम्}


\twolineshloka
{अत्राप्युदाहरन्तीममितिहासं पुरातनम्}
{अध्वर्युयतिसंवादं तं निबोध यशस्विनि}


\twolineshloka
{प्रोक्ष्यमाणं पशुं दृष्ट्वा यज्ञकर्मण्यथाब्रवीत्}
{यतिरध्वर्युमासीनो हिंसोयमिति कुत्सयन्}


\twolineshloka
{तमध्वर्युः प्रत्युवाच नायं छागो विनश्यति}
{श्रेयसा योक्ष्यते जन्तुर्यज्ञाच्छ्रुतिरियं तथा}


\twolineshloka
{यो ह्यस्य पार्थिवो भागः पृथिवीं स गमिष्यति}
{यदस्य वारिजं किञ्चिदपस्तत्सम्प्रवेक्ष्यति}


\threelineshloka
{सूर्यं चक्षुर्दिशः श्रोत्रे प्राणोऽस्य दिवमेव च}
{आगमे वर्तमानस्य नमे दोषोस्ति कश्चन ॥यतिरुवाच}
{}


\twolineshloka
{प्राणैर्वियोगे च्छागस्य यदि श्रेयः प्रपश्यसि}
{छागार्थे वर्तते यज्ञो भवतः किं प्रयोजनम्}


\twolineshloka
{अनु त्वां मन्यते माता अनु त्वां मन्यते पिता}
{मन्त्रविज्ञानमुन्नीय परिवर्ते विशेषतः}


\threelineshloka
{एवमेवानुमन्येरंस्तान्भवान्द्रष्टुमर्हति}
{तेषामनुमतिं श्रुत्वा शक्या कर्तुं विचारणा ॥अध्वर्युरुवाच}
{}


\twolineshloka
{प्राणा अप्यस्य च्छागस्य प्रापितास्ते स्वयोनिषु}
{शरीरं केवलं शिष्टं निश्चेष्टमिति मे मतिः}


\twolineshloka
{इन्धनस्य तु तुल्येन शरीरेणि विचेतसा}
{हिंसा हि यष्टुकामानामिन्धनं पशुसंज्ञितम्}


\twolineshloka
{अहिंसा सर्वधर्माणामिति वृद्धानुशासनम्}
{यदहिंस्रं भवेत्कर्म तत्कार्यमिति विद्महे}


\twolineshloka
{अहिंसेति प्रतिज्ञेयं यदि वक्ष्याम्यतः परम्}
{शक्यं बहुविधं वक्तुं भवता कार्यदूषणम्}


\threelineshloka
{अहिंसा सर्वभूतानां नित्यमस्मासु रोचते}
{प्रत्यक्षतः साधयामो न परोक्षमुपास्महे ॥अध्वर्यरुवाच}
{}


\twolineshloka
{भूमेर्गन्धगुणान्भुङ्क्ष्व पिबस्यापोमयान्रसान्}
{ज्योतिषां पश्यते रूपं स्पृशस्यनिलजान्गुणान्}


\twolineshloka
{शृणोष्याकाशजाञ्शब्दान्मनसा मन्यसे मतिम्}
{सर्वाण्येतानि भूतानि प्राणा इति च मन्यसे}


\threelineshloka
{प्राणादाने निवृत्तोसि हिंसायां वर्तते भवान्}
{नास्ति चेष्टा विना हिंसां किं वा त्वं मन्यसे द्विज ॥यतिरुवाच}
{}


\twolineshloka
{अक्षरं च क्षरं चैव द्वैधीभावोऽयमात्मनः}
{अक्षरं तत्र सद्भावः स्वभावः क्षर उच्यते}


\twolineshloka
{प्राणो जिह्वा मनः सत्त्वं सद्भावो रजसा सह}
{भावैरेतैर्विमुक्तस्य निर्द्वन्द्वस्य निराशिषः}


\threelineshloka
{समस्य सर्वभूतेषु निर्ममस्य जितात्मनः}
{समन्तात्परिमुक्तस्य न भयं विद्यते क्वचित् ॥अध्वर्युरुवाच}
{}


\twolineshloka
{सद्भिरेवेह संवादः कार्यो मतिमतांवर}
{भवतो हि मतं श्रुत्वा प्रतिभाति मतिर्मम}


\threelineshloka
{भगवन्भगवद्बुद्ध्या प्रतिबुद्धो ब्रवीम्यहम्}
{व्रतं मन्त्रकृतं कर्तुर्नापराधोस्ति मे द्विज ॥ब्राह्मण उवाच}
{}


\twolineshloka
{उपपत्त्या यतिस्तूष्णीं वर्तमानस्ततः परम्}
{अध्वर्युरपि निर्मोहः प्रचचार महामखे}


\twolineshloka
{एवमेतादृशं मोक्षं सुसूक्ष्मं ब्राह्मणा विदुः}
{विदित्वा चानुतिष्ठन्ति क्षेत्रज्ञेनार्थदर्शिना}


\chapter{अध्यायः ३०}
\twolineshloka
{अत्राप्युदाहरन्तीममितिहासं पुरातनम्}
{कार्तवीर्यस्य संवादं समुद्रस्य च भामिनि}


\twolineshloka
{कार्तवीर्यार्जुनो नाम राजा बाहुसहस्रवान्}
{येन सागरपर्यन्ता धनुषा निर्जिता मही}


\twolineshloka
{स कदाचित्समुद्रान्ते विचरन्बलदर्पितः}
{अवीकिरच्छरशतैः समुद्रमिति नः श्रुतम्}


\twolineshloka
{तं समुद्रो नमस्कृत्य कृताञ्जलिरुवाचह}
{मा मुञ्च वीर नाराचान्ब्रूहि किं करवाणि ते}


\threelineshloka
{मदाश्रयाणि भूतानि त्वद्विसृष्टैर्महेषुभिः}
{वध्यन्ते राजशार्दूल तेभ्यो देह्यभयं विभो ॥अर्जुन उवाच}
{}


\threelineshloka
{मत्समो यदि संग्रामे शरासनधरः क्वचित्}
{विद्यते तं समाचक्ष्व यः समो मे महामृधे ॥समुद्र उवाच}
{}


\twolineshloka
{महर्षिर्जमदग्निस्ते यदि राजन्पुरा श्रुतः}
{तस्य पुत्रस्तवातिथ्यं यथावत्कर्तुमर्हति}


\twolineshloka
{ततः स राजा प्रययौ क्रोधेन महता वृतः}
{स तमाश्रममागम्य राममेवान्वपद्यत}


\twolineshloka
{स रामप्रतिकूलानि चकार सह बन्धुभिः}
{आयासं जनयामास रामस्य च महात्मनः}


\twolineshloka
{ततस्तेजः प्रजज्वाल रामस्यामिततेजसः}
{प्रदहन्रिपुसैन्यानि तदा कमललोचने}


\twolineshloka
{ततः परशुमादाय स तं बाहुसहस्रिणम्}
{चिन्छेद सहसा रामो बहुशाखमिव द्रुमम्}


\twolineshloka
{तं हतं पतितं दृष्ट्वा समेताः सर्वबान्धवाः}
{असीनादाय शक्तीर्श्च भार्गवं पर्यधावयन्}


\twolineshloka
{रामोऽपि धनुरादाय रथमारुह्य सत्वरः}
{विसृजञ्शरवर्षाणि व्यधमत्पार्थिवं बलम्}


\twolineshloka
{ततस्तु क्षत्रियाः केचिज्जमदग्निं निहत्य च}
{विविशुर्गिरिदुर्गाणि मृगाः सिंहार्दिता इव}


\twolineshloka
{तेषां स्वविहितं कर्म तद्भयान्नानुतिष्ठताम्}
{प्रजा वृषलतां प्राप्ता ब्राह्मणानामदर्शनात्}


\twolineshloka
{एवं ते द्रविडाऽभीराः पुण्ड्राश्च शबरैः सह}
{वृषलत्वं परिगता व्युत्थानात्क्षत्रधर्मतः}


\twolineshloka
{ततश्च हतवीरासु क्षत्रियासु पुनः पुनः}
{द्विजैरुत्पादितं क्षत्रं जामदग्न्यो न्यकृन्तत}


\twolineshloka
{एकविंशतिमे याते रामं वागशरीरिणी}
{दिव्या प्रोवाच मधुरा सर्वलोकपरिश्रुता}


\twolineshloka
{रामराम निवर्तस्व कं गुणं तात पश्यसि}
{क्षत्रबन्धूनिमान्प्राणैर्विप्रयोज्य पुनः पुनः}


\twolineshloka
{तथैव तं महात्मानमृचीकप्रमुखास्तदा}
{पितामहा महाभाग निवर्तस्वेत्यथाब्रुवन्}


\threelineshloka
{पितुर्वधममृष्यंस्तु रामः प्रोवाच तानृषीन्}
{नार्हन्तीह भवन्तो मां निवारयितुमित्युत ॥पितर ऊचुः}
{}


\twolineshloka
{नार्हसे क्षत्रबन्धूंस्त्वं निहन्तुं जयतांवर}
{नेह युक्तं त्वया हन्तुं ब्राह्मणेन सता नृपान्}


\chapter{अध्यायः ३१}
\twolineshloka
{अत्राप्युदाहरन्तीममितिहासं पुरातनम्}
{श्रुत्वा च तत्तथा कार्यं भवता द्विजसत्तम}


\twolineshloka
{अलर्को नाम राजर्षिरभवत्सुमहातपाः}
{धर्मज्ञः सत्यवादी च महात्मा सुदृढव्रतः}


\twolineshloka
{स सागरान्तां धनुषा विनिर्जित्य महीमिमाम्}
{कृत्वा सुदुष्करं कर्म मनः सूक्ष्मे समादधे}


\threelineshloka
{स्थितस्य वृक्षमूलेऽथ तस्य चिन्ता बभूव ह}
{उत्सृय सुमहद्राज्यं सूक्ष्मं प्रति महामते ॥अलर्क उवाच}
{}


\twolineshloka
{मनसो मे बलं जातं मनो जित्वा ध्रुवो जयः}
{अन्यत्र बाणानस्यामि शत्रुभिः परिवारितः}


\threelineshloka
{यदिदं चापलात्कर्म सर्वान्मर्त्यांश्चिकीर्षति}
{मनः प्रति सुतीक्ष्णाग्रानहं मोक्ष्यामि सायकान् ॥मन उवाच}
{}


\twolineshloka
{नेमे बाणास्तरिष्यन्ति मामलर्क कथञ्चन}
{तवैव मर्म भेत्स्यन्ति भिन्नमर्मा मरिष्यसि}


\twolineshloka
{अन्यान्बाणान्समीक्षस्व यैस्त्वं मां सूदयिष्यसि}
{तत्छ्रुत्वा स विचिन्त्याथ ततो वचनमब्रवीत्}


\threelineshloka
{आघ्राय सुबहून्गन्धांस्तानेव प्रतिगृध्यति}
{तस्माद्ध्राणं प्रति शरान्प्रतिमोक्ष्याम्यहं शितान् ॥घ्राण उवाच}
{}


\twolineshloka
{नेमे बाणास्तरिष्यन्ति मामलर्क कथञ्चन}
{तवैव मर्म भेत्स्यन्ति भिन्नमर्मा मरिष्यसि}


\twolineshloka
{अन्यान्याणान्समीक्षस्व यैस्त्वं मां सूदयिष्यसि}
{तच्छ्रुत्वा स विचिन्त्याथ ततो वचनमब्रवीत्}


\threelineshloka
{इयं स्वादून्रसान्भुक्त्वा तानेव प्रति गृध्यति}
{तस्माज्जिह्वां प्रति शरान्प्रतिमोक्ष्याम्यहं शितान् ॥जिह्वोवाच}
{}


\twolineshloka
{नेमे वाणास्तरिष्यन्ति मामलर्क कथञ्चन}
{तवैव मर्म भेत्स्यन्ति ततो हास्यसि जीवितम्}


\twolineshloka
{अन्यान्वाणान्समीक्षस्व यैस्त्वं मां सूदयिष्यसि}
{तच्छ्रुत्वा स विचिन्त्याथ ततो वचनमब्रवीत्}


\threelineshloka
{स्पृष्ट्वा त्वग्विविधान्स्पर्शांस्तानेव प्रतिगृध्यति}
{तस्मात्त्वचं पाटयिष्ये विविधैः कङ्कपत्रिभिः ॥त्वगुवाच}
{}


\twolineshloka
{नेमे बाणास्तरिष्यन्ति मामलर्क कथञ्चन}
{तवैव मर्म भेत्स्यन्ति भिन्नमर्मा मरिष्यसि}


\twolineshloka
{अन्यान्बाणान्समीक्षस्व यैस्त्वं मां सूदयिष्यसि}
{तच्छ्रुत्वा स विचिन्त्याथ ततो वचनमब्रवीत्}


\threelineshloka
{श्रुत्वा तु विविधाञ्शब्दांस्तानेव प्रतिगृध्यति}
{तस्माच्छ्रोत्रं प्रति शरान्प्रति मुञ्चाम्यहं शितान् ॥श्रोत्रमुवाच}
{}


\twolineshloka
{नेमे बाणास्तरिष्यन्ति मामलर्क कथञ्चन}
{तवैव मर्म भेत्स्यन्ति ततो हास्यति जीवितम्}


\twolineshloka
{अन्यान्बाणान्समीक्षस्व यैस्त्वं मां सूदयिष्यसि}
{तच्छ्रुत्वा स विचिन्त्याथ ततो वचनमब्रवीत्}


\threelineshloka
{दृष्ट्वा रूपाणि बहुशस्तान्येव प्रतिगृध्यति}
{तस्माच्चक्षुर्हनिष्यामि निशितैः सायकैरहम् ॥चक्षुरुवाच}
{}


\twolineshloka
{नेमे बाणास्तरिष्यन्ति मामलर्क कथञ्चन}
{तवैव मर्म भेत्स्यन्ति भिन्नमर्मा मरिष्यसि}


% Check verse!
अन्यान्बाणान्समीक्षस्व यैस्त्वं मां सूदयिष्यतितच्छ्रुत्वा स विचिन्त्याथ ततो वचनमब्रवीत्
\threelineshloka
{इयं निष्ठा बहुविधा प्रज्ञया त्वद्यवस्यति}
{तस्माद्बुद्धिं प्रति शरान्प्रतिमोक्ष्याम्यहं शितान् ॥बुद्धिरुवाच}
{}


\fourlineindentedshloka
{नेमे बाणास्तरिष्यन्ति मामलर्क कथञ्चन}
{तवैव मर्म भेत्स्यन्ति भिन्नमर्मा मरिष्यसि}
{अन्यान्बाणान्समीक्षस्व यैस्त्वं मां सूदयिष्यसि ॥पितर ऊचुः}
{}


\twolineshloka
{ततोऽलर्कस्तपो घोरं तत्रैवास्थाय दुष्करम्}
{नाध्यगच्छत्परं शक्त्या बाणमेतेषु सप्तसु}


\threelineshloka
{सुसमाहितचेतास्तु स ततोऽचिन्तयत्प्रभुः}
{स विचिन्त्य चिरं कालमलर्को द्विजसत्तम}
{नाध्यगच्छत्परं श्रेयो योगान्मतिमतांवरः}


\twolineshloka
{स एकाग्रं मनः कृत्वा निश्चलो योगमास्थितः}
{इन्द्रियाणि जघानाशु बाणेनैकेन वीर्यवान्}


\twolineshloka
{योगेनात्मानमाविश्य सिद्धिं परमिकां गतः}
{विस्मितश्चापि राजर्षिरिमां गाथां जगाद ह}


\threelineshloka
{अहो कष्टं यदस्माभिः सर्वं बाह्यमनुष्ठितम्}
{भोगतृष्णासमायुक्तैः पूर्वं राज्यमुपासितम्}
{इति पश्चान्मया ज्ञातं योगान्नास्ति परं सुखम्}


\twolineshloka
{इति त्वमनुजानीहि राम मा क्षत्रियान्वधीः}
{तषो घोरमुपातिष्ठ ततः श्रेयोऽभिपत्स्यसे}


\twolineshloka
{इत्युक्तः पितृभिः सोथ तपो घोरं समास्थितः}
{जामदग्न्यो महाभागे सिद्धिं च परमां गतः}


\chapter{अध्यायः ३२}
\twolineshloka
{त्रयो वै रिपवो लोके नवधा गुणतः स्मृताः}
{हर्षः स्तंभोतिमानश्च त्रयस्ते सात्विका गुणाः}


\twolineshloka
{शोकः क्रोधाभिसंरम्भो राजसास्ते गुणाः स्मृताः}
{स्वप्नस्तन्द्रा च मोहश्च त्रयस्ते तामसा गुणाः}


\twolineshloka
{एतान्निकृत्य धृतिमान्बाणसङ्घैरतन्द्रितः}
{जेतुं परानुत्सहते प्रशान्तात्मा जितेन्द्रियः}


\twolineshloka
{अत्र गाथाः कीर्तयन्ति पुराकल्पविदो जनाः}
{अम्बरीषेण या गीता राज्ञा राज्यं प्रशासता}


\twolineshloka
{समुदीर्णेषु दोषेषु बाध्यमानेषु साधुषु}
{जग्राह तरसा राज्यमम्बरीष इति श्रुतिः}


\twolineshloka
{स निगृह्यात्मनो दोषान्साधून्समभिपूज्य च}
{जगाम महतीं सिद्धिं गाथाश्चेमा जगाद ह}


\twolineshloka
{भूयिष्ठं विजिता दोषा निहताः सर्वशत्रवः}
{एको दोषो वरिष्ठश्च वध्यः स न हतो मया}


\twolineshloka
{यत्प्रयुक्तो जन्तुरयं वैतृष्ण्यं नाधिगच्छति}
{तृष्णार्त इव निम्नानि धावमानो न बुध्यते}


\twolineshloka
{अकार्यमपि येनेह प्रयुक्तः सेवते नरः}
{तं लोभमसिभिस्तीक्ष्णैर्निकृत्य सुखमेधते}


\threelineshloka
{लोभाद्धि जायते तृष्णा ततश्चिन्ता प्रवर्तते}
{स लिप्समानो लभते भूयिष्ठं राजसान्गुणान्}
{तदवाप्तौ तु लभते भूयिष्ठं तामसान्गुणान्}


\threelineshloka
{स तैर्गुणैः संहतदेहबन्धनः}
{पुनःपनर्जायति कर्म चेहते}
{जन्मक्षये भिन्नविकीर्मदेहोमृत्युं पुनर्गच्छति जन्मनैव}


\twolineshloka
{तस्मादेतं सम्यगवेक्ष्य लोभंनिगृह्य धृत्याऽऽत्मनि राज्यमिच्छेत्}
{एतद्राज्यं नान्यदस्तीह राज्य-मात्मैव राजा विदितो यथावत्}


\twolineshloka
{इति राज्ञाऽम्बरीषेण गाथा गीता यशस्विना}
{आधिराज्य पुरस्कृत्य लोभमेकं निकृन्तता}


\chapter{अध्यायः ३३}
\twolineshloka
{अत्राप्युदाहरन्तीममितिहासं पुरातनम्}
{ब्राह्मणस्य च संवादं जनकस्य च भामिनि}


\twolineshloka
{ब्राह्मणं जनको राजा सन्नं कस्मिंश्चिदागसि}
{विषये मे न वस्तव्यमिति शिष्ट्यर्थमब्रवीत्}


\twolineshloka
{इत्युक्तः प्रत्युवाचाथ ब्राह्मणो राजसत्तमम्}
{आचक्ष्व विषयं राजन्यावांस्तव वशे स्थितः}


\twolineshloka
{सोऽन्यस्य विषये राज्ञो वस्तुमिच्छाम्यहं विभो}
{वचस्ते कर्तुमिच्छामि यथाशास्त्रं महीपते}


\twolineshloka
{इत्युक्तस्तु तदा राजा ब्राह्मणेन यशस्विना}
{मुहुरुष्णं विनिःस्वस्य न किञ्चित्प्रत्यभाषत}


\twolineshloka
{तमासीनं ध्यायमानं राजानममितौजसम्}
{कश्मलं सहसाऽगच्छद्भानुमन्तमिव ग्रहः}


\twolineshloka
{समाश्वास्य ततो राजा विगते कश्यमे तदा}
{ततो मूहूर्तादिव तं ब्राह्मणं वाक्यमब्रवीत्}


\twolineshloka
{पितृपैतामहे राज्ये वश्ये जनपदे सति}
{विषयं नाधिगच्छामि विचिन्वन्पृथिवीमहम्}


\twolineshloka
{नाध्यगच्छं यदा पृथ्व्यां मिथिला मार्गिता मया}
{नाध्यगच्छं यदा तस्यां स्वप्रजा मार्गिता मया}


\twolineshloka
{नाध्यगच्छं यदा तस्यां तदा मे कश्मलोऽभवत्}
{ततो मे कश्मलस्यान्ते मतिः पुनरुपस्थिता}


\twolineshloka
{तदा न विषयं मन्ये सर्वो वा विषयो मम}
{आत्माऽपि चायं न मम सर्वा वा पृथिवी मम}


\threelineshloka
{यथा मम तथाऽन्येषामिति मन्ये द्विजोत्तम}
{उष्यतां यावदुत्साहो भुज्यतां यावदिष्यते ॥ब्राह्मण उवाच}
{}


\twolineshloka
{पितृपैतामहे राज्ये वश्ये जनपदे सति}
{ब्रूहि कां मतिमास्थाय ममत्वं वर्जितं त्वया}


\threelineshloka
{कां वै बुद्धिं समाश्रित्य सर्वो वै विषयस्तव}
{नावैषि विषयं येन सर्वो वा विषयस्तव ॥जनक उवाच}
{}


\twolineshloka
{अन्तवन्त इहारम्भा विदिताः सर्वकर्मसु}
{नाध्यगच्छमहं तस्मान्ममेदमिति यद्भवेत्}


\twolineshloka
{कस्येदमिति कस्य स्वमिति वेदवचस्तथा}
{नाध्यगच्छमहं बुद्ध्या ममेदमिति यद्भवेत्}


\twolineshloka
{एतां बुद्धिं समाश्रित्य ममत्वं वर्जितं मया}
{शृणु बुद्धिं च यां ज्ञात्वा सर्वत्र विषयो मम}


\twolineshloka
{नाहमात्मार्थमिच्छामि गन्धान्घ्राणगतानपि}
{तस्मान्मे निर्जिता भूमिर्वशे तिष्ठति नित्यदा}


\twolineshloka
{नाहमात्मार्थमिच्छामि रसानास्येऽपि वर्ततः}
{आपो मे निर्जितास्तस्माद्वशे तिष्ठन्ति नित्यदा}


\twolineshloka
{नाहमात्मार्थमिच्छामि रूपं ज्योतिश्च चक्षुषः}
{तस्मान्मे निर्जितं ज्योतिर्वशे तिष्ठति नित्यदा}


\twolineshloka
{नाहमात्मार्थमिच्छामि स्पर्शांस्त्वचि गताश्च ये}
{तस्मान्मे निर्जितो वायुर्वशे तिष्ठति नित्यदा}


\twolineshloka
{नाहमात्मार्थमिच्छामि शब्दाञ्श्रोत्रगतानपि}
{आकाशं मे जितं तस्माद्वशे तिष्ठति नित्यदा}


\twolineshloka
{नाहमात्मार्थमिच्छामि मनो नित्यं मनोन्तरे}
{मनो मे निर्जितं तस्माद्वशे तिष्ठति नित्यदा}


\twolineshloka
{देवेभ्यश्च पितृभ्यश्च भूतेभ्योऽतिथिभिः सह}
{इत्यर्थं सर्व एवेति समारम्भा भवन्ति वै}


\twolineshloka
{ततः प्रहस्य जनकं ब्राह्मणः पुनरब्रवीत्}
{त्वज्जिज्ञासार्थमद्येह विद्धि मां धर्ममागतम्}


\twolineshloka
{त्वमस्य ब्रह्मिनाभस्य दुर्वारस्यानिवर्तिनः}
{सत्वनेमिनिरुद्धस्य चक्रस्यैकः प्रवर्तकः}


\chapter{अध्यायः ३४}
\twolineshloka
{नाहं तथा भूरु चरामि लोकेयथा त्वं मां तर्जयसे स्वबुद्ध्या}
{विप्रोस्मिं मुक्तोस्मि वनेचरोस्मिगृहस्थधर्मा व्रतवांस्तथाऽस्मि}


\twolineshloka
{नाहमस्मि यता मां त्वं पश्यसे च शुभाशुभे}
{मया व्याप्तमिदं सर्वं यत्किञ्चिज्जगतीगतम्}


\twolineshloka
{ये केचिज्जन्तवो लोके जङ्गमाः स्थावराश्च ह}
{तेषां मामन्तकं विद्धि दारुणामिव पावकम्}


\twolineshloka
{राज्ये पृथिव्यां सर्वस्यामथवापि त्रिविष्टपे}
{तथा बुद्धिरियं वेत्ति बुद्धिरेव धनं मम}


\twolineshloka
{एकः पन्था ब्राह्मणानां येन गच्छन्ति तद्विदः}
{गृहेषु वनवासेषु गुरुवासेषु भिक्षुषु}


\twolineshloka
{लिङ्गैर्बहुभिरव्यग्रैरेका बुद्धिरुपास्यते}
{नानालिङ्गाश्रमस्थानां येषां बुद्धिः शमात्मिका}


\threelineshloka
{ते भावमेकमायान्ति सरितः सागरं यथा}
{बुद्ध्याऽयं गम्यते मार्गः शरीरेणि न गम्यते}
{आद्यन्तवन्ति कर्माणि शरीरं कर्मबन्धनम्}


\twolineshloka
{तस्मान्मे सुभगे नास्ति परलोककृतं भयम्}
{तद्भावभावनिरता ममैवात्मानमेष्यसि}


\chapter{अध्यायः ३५}
\twolineshloka
{नेदमल्पात्मना शक्यं वेदितुं वाऽकृतात्मना}
{बहु चाल्पं च संक्षिप्तं विस्तृतं च मतं मम}


\threelineshloka
{उपायं तं मम ब्रूहि येनैषा लभ्यते मतिः}
{तन्मन्ये कारणं कर्म यत एषा प्रवर्तते ॥ब्राह्मण उवाच}
{}


\threelineshloka
{अरणीं ब्राह्मणीं विद्धि गुरुरस्योत्तरारणिः}
{तपःश्रुतेभिमथिनी ज्ञानाग्निर्जायते ततः ॥ब्राह्मणयुवाच}
{}


\threelineshloka
{यदिदं ब्रह्मणो लिङ्गं क्षेत्रज्ञ इति संज्ञितम्}
{ग्रहीतुं येन यच्छक्यं लक्षणं तस्य तद्वद ॥ब्राह्मण उवाच}
{}


\twolineshloka
{अलिङ्गो निर्गुणश्चैव कारणं नास्य विग्रहे}
{उपायमेव वक्ष्यामि येन गृह्येत भावना}


\threelineshloka
{सम्यगप्युपदिष्टस्य ह्यमृतस्येव तृप्यसे}
{कर्मबुद्धिरबुद्धित्वाज्ज्ञानलिङ्गान्निपातितः}
{}


\twolineshloka
{इदं कार्यमिदं नेति न मोक्षेषूपदिश्यते}
{पश्यतः शृण्वतो बुद्धिरात्मनैवोपजायते}


\twolineshloka
{यावन्त इह शक्येरंस्तावतोंऽसान्प्रकल्पयेत्}
{अव्यक्तान्व्यक्तरूपांश्च शतशोऽथ सहस्रशः}


\threelineshloka
{सर्वानुमानयुक्तांश्च सर्वान्प्रत्यक्षहेतुकान्}
{यतः परं न विद्येत ततोऽभ्यासे भविष्यति ॥श्रीभगवानुवाच}
{}


\threelineshloka
{ततस्तु तस्या ब्राह्मण्या मतिः क्षेत्रज्ञसंशये}
{क्षेत्रज्ञानेन परतः क्षेत्रज्ञोऽन्यः प्रवर्तते ॥अर्जुन उवाच}
{}


\threelineshloka
{क्व नु सा ब्राह्मणि कृष्ण क्व चासौ ब्राह्मणर्षभः}
{याभ्यां सिद्धिरियं प्राप्ता तावुभौ वद मेऽच्युत ॥श्रीभगवानुवाच}
{}


\twolineshloka
{मनो मे ब्राह्मणं विद्धि बुद्धिं मे विद्धि ब्राह्मणीम्}
{क्षेत्रज्ञ इति यश्चोक्तः सोऽहमेव धनंजय}


\chapter{अध्यायः ३६}
\threelineshloka
{ब्रह्म यत्परमं ज्ञेयं तन्मे व्याख्यातुमर्हसि}
{भतो हि प्रसादेन सूक्ष्मे मे रमते मतिः ॥वासुदेव उवाच}
{}


\twolineshloka
{अत्राप्युदाहरन्तीममितिहासं पुरातनम्}
{संवादं मोक्षसंयुक्तं शिष्यस्य गुरुणा सह}


\twolineshloka
{कश्चिद्ब्राह्मणमासीनमाचार्यं संशितव्रतम्}
{शिष्यः पप्रच्छ मेधावी किंस्विच्छ्रेयः परंतप}


\twolineshloka
{भगवन्तं प्रपन्नोऽहं निःश्रेयसपराणः}
{याचे त्वां शिरसा विप्र यद्ब्रूयां ब्रूहि तन्मम}


\twolineshloka
{तमेवंवादिनं पार्थ शिष्यं गुरुरुवाच ह}
{कथयस्व प्रवक्ष्यामि यत्र ते संशयो द्विज}


\threelineshloka
{इत्युक्तः स कुरुश्रेष्ठ गुरुणा गुरुवत्सलः}
{प्राञ्जलिः परिपप्रच्छ यत्तच्छृणु महामते ॥शिष्य उवाच}
{}


\twolineshloka
{कुतश्चाहं कुतश्च त्वं तत्सत्यं ब्रूहि यत्परम्}
{कुतो जातानि भूतानि स्थावराणि चराणि च}


\twolineshloka
{केन जीवन्ति भूतानि तेषामायुश्च किं परम्}
{किं सत्यं किं तपो विप्र के गुणाः सद्भिरीरिताः}


\twolineshloka
{के पन्थानः शिवाश्च स्युः किं सुखं कि च दुष्कृतम्}
{एतान्मे भगवन्प्रश्नान्याथातथ्येन सुव्रत}


\twolineshloka
{वक्तुमर्हसि विप्रर्षे यथावदिह तत्त्वतः}
{त्वदन्यः कश्च न प्रश्नानेतान्वक्तुमिहार्हति}


\twolineshloka
{ब्रूहि धर्मविदां श्रेष्ठ परं कौतूहलं मम}
{मोक्षधर्मार्थकुशलो भवाँल्लोकषु गीयते}


\threelineshloka
{सर्वसंशयसंच्छेत्ता त्वदन्यो न च विद्यते}
{संसारभीरवश्चैव मोक्षकामास्तथा वयम् ॥वासुदेव उवाच}
{}


\twolineshloka
{तस्मै सम्प्रतिपन्नाय यथावत्परिपृच्छते}
{शिष्याय गुणयुक्ताय शान्ताय गुरुवर्तिने}


\fourlineindentedshloka
{छायाभूताय दान्ताय यतते ब्रह्माचारिणे}
{तान्प्रश्नानब्रवीत्पार्थ मेधावी स धृतव्रतः}
{गुरुः कुरुकुलश्रेष्ठ सम्यक्सर्वानरिंदम ॥गुरुरुवाच}
{}


\twolineshloka
{ब्रह्मणोक्तमिदं धर्ममृषिप्रवरसेवितम्}
{वेदविद्यासमावाप्तं तत्त्वभूतार्थभावनम्}


\threelineshloka
{ज्ञानं त्वेव परं विद्यः संन्यासं तप उत्तमम्}
{यस्तु वेद निराबाधं ज्ञानतत्त्वं विनिश्चयात्}
{सर्वबूतस्थमात्मानं स सर्वगतिरिष्यते}


\twolineshloka
{ये विद्वान्सहसंवासं विवासं चैव पश्यति}
{तथैवैकत्वनानात्वे स दुःखात्परिमुच्यते}


\twolineshloka
{यो न कामयते किञ्चिन्न किञ्चिदभिमन्यते}
{इह लोकस्थ एवैष ब्रह्मभूयाय कल्पते}


\twolineshloka
{प्रधानगुणतत्त्वज्ञः सर्वभूतविधानवित्}
{निर्ममो निरहङ्कारो मुच्यते नात्र संशयः}


\twolineshloka
{अव्यक्तबीजप्रभवो बुद्धिस्कन्धमयो महान्}
{महाहङ्कारविटप इन्द्रियाङ्कुरकोटरः}


\twolineshloka
{महाभूतविशेषश्च विशेषप्रतिशाखवान्}
{सदापर्णः सदापुष्पः शुभाशुभफलोदयः}


\threelineshloka
{आजीवः सर्वभूतानां ब्रह्मबीजः सनातनः}
{एतज्ज्ञात्वा च तत्त्वानि ज्ञानेन परमासिना}
{छित्त्वा चामरतां प्राप्य जहाति मृत्युजन्मनी}


\twolineshloka
{भूतभव्यभविष्यादिधर्मकामार्थनिश्चयम्}
{सिद्धसङ्घपरिज्ञातं पुराकल्पं सनातनम्}


\twolineshloka
{प्रवक्ष्येऽहं महाप्राज्ञ पदमुत्तममद्य ते}
{बुद्ध्वा यदिहं संसिद्धा भवन्तीह मनीषिणः}


\twolineshloka
{उपगम्यर्षयः पूर्वं जिज्ञासन्तः परस्परम्}
{प्रजापतिभरद्वाजौ गौतमो भार्गवस्तथा}


\twolineshloka
{वसिष्ठः कश्यपश्चैव विश्वामित्रोऽत्रिरेव च}
{मार्गान्सर्वान्परिक्रम्य परिश्रान्तः स्वकर्मभिः}


\twolineshloka
{ऋषिमाङ्गिरसं वृद्धं पुरस्कृत्य तु ते द्विजाः}
{ददृशुर्ब्रह्मभवने ब्रह्माणं वीतकल्मषम्}


\twolineshloka
{तं प्रणम्य महात्मानं सुखासीनं महर्षयः}
{पप्रच्छुर्विनयोपेता नैःश्रेयसमिदं परम्}


\twolineshloka
{कथं कर्म कृतं साधु कथं मुच्येत किल्बिषात्}
{के नो मार्गाः शिवाश्च स्युः किं सत्यं किं च दुष्कृतं}


\twolineshloka
{कौ चोभौ कर्मणां मार्गौ प्राप्नुयुर्दक्षिणोत्तरौ}
{निरयं चापवर्गं च भूतानां प्रभवाप्ययौ}


\threelineshloka
{इत्युक्तः स मुनिश्रेष्ठैर्यदाह प्रपितामहः}
{तत्तेऽहं सम्प्रवक्ष्यामि शृणु शिष्य यथागमम् ॥ब्रह्मोवाच}
{}


\threelineshloka
{सत्याद्भूतानि जातानि स्थावराणि चराणि च}
{तपसा तानि जीवन्ति जीवितं तद्धि सुव्रतम्}
{स्वां योनिं पुनरागम्य वर्तते स्वेन कर्मणा}


% Check verse!
सत्यं हि गुणसंयुक्तं नियतं पञ्चलक्षणम्
\twolineshloka
{ब्रह्म सत्यं तपः सत्यं सत्यं चैव प्रजापतिः}
{सत्याद्भूतानि जातानि सत्यं भूतमयं जगत्}


\twolineshloka
{तस्मात्सत्याश्रया विप्रा नित्यं योगपरायणाः}
{अतीक्रोधसंतापा नियता धर्मसेतवः}


\twolineshloka
{अन्योन्यनियतान्वैद्यान्धर्मसेतुप्रवर्तकान्}
{तानहं सम्प्रवक्ष्यामि शाश्वताँल्लोकभावनान्}


\twolineshloka
{चातुर्विद्यं तथा वर्णांश्चातुराश्रमिकान्पृथक्}
{धर्ममेकं चतुष्पादं नित्यमाहुर्मनीषिणः}


\twolineshloka
{पन्थानं वः प्रवक्ष्यामि शिवं क्षेमकरं द्विजाः}
{नियतं ब्रह्मभावाय यातं पूर्वं मनीषिभिः}


\twolineshloka
{गदन्तस्तु ममाद्येह पन्थानं दुर्विदं परैः}
{निबोधत महाभागा निखिलेन परं पदम्}


\threelineshloka
{ब्रह्मचर्यमिहैवाहुराश्रमं प्रथमं पदम्}
{गार्हस्थ्यं तु द्वितीयं स्याद्वानप्रस्थमतः परम्}
{ततः परं तु विज्ञेयमध्यात्मं परमं पदम्}


\twolineshloka
{ज्योतिराकाशमादित्यो वायुरिन्द्रः प्रजापतिः}
{नोपैति यावदध्यात्मं तावदेतान्न पश्यति}


\twolineshloka
{तस्योपायं प्रवक्ष्यामि पुरस्तात्तं निबोधत}
{फलमूलानिलभुजां मुनीनां वसतां वने}


\twolineshloka
{वानप्रस्थं द्विजातीनां त्रयाणामुपदिश्यते}
{सर्वेषामेव वर्णानां गृहस्थोऽयं विशिष्यते}


\twolineshloka
{श्रद्धालक्षणमित्येकं धर्मं धीराः प्रचक्षते}
{`नैष्ठिकोऽथ यतिर्वाऽपि विरक्तो ब्रह्मदर्शनः ॥'}


\twolineshloka
{इत्येवं देवयाना वः पन्थानः परिकीर्तिताः}
{सद्भिरध्यासिता धीरैः कर्मभिर्धर्मसेतवः}


\twolineshloka
{एतेषां पृथगध्यास्ते यो धर्मं संशितव्रतः}
{कालात्पश्यति भूतानां सदैव प्रभवाप्ययौ}


\twolineshloka
{अतस्तत्त्वानि वक्ष्यामि याथातथ्येन हेतुना}
{विषयस्थानि सर्वाणि वर्तमानानि भागशः}


\twolineshloka
{महानात्मा तथाऽव्यक्तमहंकारस्तथैव च}
{इन्द्रियाणि दशैकं च महाभूतानि पञ्च च}


\twolineshloka
{विशेषाः पञ्चभूतानामित्येषा वैदिकी श्रुतिः}
{चतुर्विंशतिरेषा वस्तत्वानां परिकीर्तिता}


\twolineshloka
{तत्वानामथ यो वेद सर्वेषां प्रभवाप्ययौ}
{स धीरः सर्वभूतेषु न मोऽहमधिगच्छति}


\twolineshloka
{तत्त्वानि यो वेदयते यथातथंगुणांश्च सर्वानखिलाश्च देवताः}
{विधूतपाप्मा प्रविमुच्य बन्धनंस सर्वलोकानमलान्समश्नुते}


\chapter{अध्यायः ३७}
\twolineshloka
{तदव्यक्तमनुद्रिक्तं सर्वव्यापि ध्रुवं स्थिरम्}
{नवद्वारं पुरं विद्यात्त्रिगुणं पञ्चधातुकम्}


\twolineshloka
{एकादशपरिक्षेपं मनोव्याकरणात्मकम्}
{बुद्धिस्वामिकमित्येतत्परमेकादशं भवेत्}


\twolineshloka
{त्रीणि स्रोतांसि यान्यस्मिन्नाप्यायन्ते पुनःपुनः}
{प्रनाड्यस्तिस्र एवैताः प्रवर्न्तते गुणात्मिकाः}


\twolineshloka
{तमो रजस्तथा सत्वं गुणानेतान्प्रचक्षते}
{अन्योन्यमिथुनाः सर्वे तथाऽन्योन्यानुजीविनः}


\twolineshloka
{अन्योन्यापाश्रयाश्चापि तथाऽन्योन्यानुवर्तिनः}
{अन्योन्यव्यतिषक्ताश्च त्रिगुणाः पञ्चधातवः}


\twolineshloka
{तमसो मिथुनं सत्वं सत्वस्य मिथुनं रजः}
{रजसश्चापि सत्वं स्यात्सत्वस्य मिथुनं तमः}


\twolineshloka
{नियम्यते तमो यत्र रजस्तत्र निवर्तते}
{नियम्यते रजो यत्र सत्वं तत्र प्रवर्तते}


\threelineshloka
{नैशात्मकं तमो विद्यात्त्रिगुणं मोहसंज्ञितम्}
{अधर्मलक्षणं चैव नियतं पापकर्मसु}
{तामसं रूपमेतत्तु दृश्यते चापि सङ्गतम्}


\twolineshloka
{प्रकृत्यात्मकमेवाहू रजः पर्यायकारकम्}
{सत्त्वे तु सर्वभूतेषु दृश्यमुत्पत्तिलक्षणम्}


\twolineshloka
{प्रकाशं सर्वभूतेषु लाघवं श्रद्धधानता}
{सात्विकं रूपमेवं तु लाघवं साधुसंमितम्}


\twolineshloka
{एतेषां गुणतत्त्वानि वक्ष्यन्ते तत्त्वहेतुभिः}
{समासव्यासयुक्तानि तत्त्वतस्तानि बोधत}


\twolineshloka
{सम्मोहो ज्ञानमत्यागः कर्मणामविनिर्णयः}
{स्वप्नः स्तंभो भयं लोभः शोकः सुकृतदूषणम्}


\twolineshloka
{अस्मृतिश्चाविपाकश्च नास्तिक्यं भिन्नवृत्तिता}
{निर्विशेषत्वमन्धत्वं जघन्यगुणवृत्तिता}


\twolineshloka
{अकृते कृतमानित्वमज्ञाने ज्ञानमानिता}
{अमैत्री विकृतो भावो ह्यश्रद्धा मूढभावना}


\twolineshloka
{अनार्जवमसंज्ञत्वं कर्म पापमचेतना}
{गुरुत्वं सन्नभावत्वमवशित्वमवाग्गतिः}


\twolineshloka
{सर्व एते गुणा वृत्तास्तामसाः सम्प्रकीर्तिताः}
{ये चान्ये विहिता भावा लोकेऽस्मिन्भावसंज्ञिताः}


\twolineshloka
{तत्रतत्र नियम्यन्ते सर्वे ते तामसा गुणाः}
{परिवादकथा नित्यं मेवब्राह्मणवैरिता}


\twolineshloka
{अत्यागश्चातिमानश्च मोहो मन्युस्तथाऽक्षमा}
{मत्सरश्चैव भूतेषु तामसं वृत्तमिष्यते}


\twolineshloka
{वृथारम्भा हि ये केचिद्वृथा दानानि यानि च}
{वृथाभक्षणमित्येतत्तामसं वृत्तमिष्यते}


\twolineshloka
{अतिवादोऽतितिक्षा च मात्सर्यमभिमानिता}
{अश्रद्दधानता चैव तामसं वृत्तमिष्यते}


\twolineshloka
{एवंविधाश्च ये केचिल्लोकेऽस्मिन्पापकर्मिणः}
{मनुष्या भिन्नमर्यादास्ते सर्वे तामसाः स्मृताः}


\threelineshloka
{तेषां योनीः प्रवक्ष्यामि नियताः पापकर्मिणाम्}
{अवाङ्निरयभावा ये तिर्यङ्निरयगामिनः}
{}


\twolineshloka
{स्थावराणि च भूतानि पशवो वाहनानि च}
{क्रव्यादा दंदशूकाश्च कृमिकीटविहङ्गमाः}


\twolineshloka
{अब्जाता जन्तवश्चैव सर्वे चापि चतुष्पदाः}
{उन्मत्ता बधिरा मूका ये चान्ये पापरोगिणः}


\twolineshloka
{मग्नास्तमसि दुर्वृत्ताः स्वकर्मकृतलक्षणाः}
{अवाक्स्रोतस इत्येते मग्नास्तमसि तामसाः}


\twolineshloka
{तेषामुत्कर्षमुद्रेकं वक्ष्याम्यहमतः परम्}
{यथा ते सुकृतां लोकाँल्लभन्ते पुण्यकर्मिणः}


\twolineshloka
{अन्यथा प्रतिपन्नास्तु विवृद्धा ये च कर्मसु}
{स्वकर्मनिरतानां च ब्राह्मणानां शुभैषिणाम्}


\twolineshloka
{संस्कारेणोर्ध्वमायान्ति यतमानाः सलोकताम्}
{स्वर्गे गच्छन्ति देवानामित्येषा वैदिकी श्रुतिः}


\twolineshloka
{अन्यथा प्रतिपन्नास्ते विबुद्धाः स्वेषु कर्मसु}
{पुनरावृत्तिधर्माणस्ते भवन्तीह मानुषाः}


% Check verse!
पापयोनिं समापन्नाश्चण्डाला मूकचूचुकाः ॥वर्णान्पर्यायशश्चापि प्राप्नुवन्त्युत्तरोत्तरम्
\twolineshloka
{शूद्रयोनिमतिक्रम्य ये चान्ये तामसा गुणाः}
{स्रोतोमध्ये समागम्य वर्तन्ते तामसे गुणे}


\twolineshloka
{अभिष्वङ्गस्तु कामेषु महामोह इति स्मृतः}
{ऋषयो मुनयो देवा मुह्यन्त्यत्र सुखेप्सवः}


\twolineshloka
{तमोमोहो महामोहस्तामिस्रो ह्यन्धसंज्ञितः}
{मरणं त्वन्धतामिस्रस्तामिस्रः क्रोध उच्यते}


\twolineshloka
{वर्णतो गुणतश्चैव योनितश्चैव तत्त्वतः}
{सर्वमेततमो विप्राः कीर्तितं वो यथाविधि}


\twolineshloka
{को न्वेतद्बुध्यते साधु को न्वेतत्साधु पश्यति}
{अतत्त्वे तत्त्वदर्शी यस्तमसस्तच्च लक्षणम्}


\twolineshloka
{तमोगुणा बहुविधाः प्रकीर्तितायथावदुक्तं च तमः परावरम्}
{नरो हि यो वेद गुणानिमान्सदास तामसैः सर्वगुणैः प्रमुच्यते}


\chapter{अध्यायः ३८}
\twolineshloka
{रजोऽहं वः प्रवक्ष्यामि याथातथ्येन सत्तमाः}
{निबोधत महाभागा गुणवृत्तं च राजसम्}


\twolineshloka
{संतापो रूपमायासः सुखदुःखे हिमातपौ}
{ऐश्वर्यं विग्रहः सिद्धिर्हेतुवादोऽरतिः क्षमा}


\twolineshloka
{बलं शौर्यं मदो रोषो व्यायामकलहावपि}
{ईर्ष्येप्सा पैशुनं युद्धं ममत्वं परिपालनम्}


\twolineshloka
{वधबन्धपरिक्लेशाः क्रयो विक्रय एव च}
{निकृन्त च्छिन्धि भिन्धीति परवर्मावकर्तनम्}


\twolineshloka
{उग्रं दारुणमाक्रोशः परवित्तानुरागिता}
{लोकचिन्ताऽनुचिन्ता च मत्सरः परिभाषणम्}


\twolineshloka
{वृथाशास्त्रं मृषावादो विकल्पपरिभाषणम्}
{निन्दा स्तुतिः प्रशंसा च प्रतापः परिधर्षणम्}


% Check verse!
परिचर्या च शुश्रूषा सेवा तृष्णा व्यपाश्रयः ॥व्यूहो नयः प्रमादश्च परिवादः परिग्रहः
\twolineshloka
{संस्कारा ये च लोकेषु प्रवर्तन्ते पृथक्पृथक्}
{नृषु नारीषु भूतेषु द्रव्येषु शरणेषु च}


\twolineshloka
{संतापोऽप्रत्ययश्चैव व्रतानि नियमाश्च ये}
{प्रधानमाशीर्युक्तं च सततं मे भवत्विति}


\twolineshloka
{स्वाहाकारो नमस्कारः स्वधाकारो वषट्क्रिया}
{याजनाध्यापने चोभे यजनाध्ययने अपि}


\twolineshloka
{दानं प्रतिग्रहश्चैव प्रायश्चित्तानि मङ्गलम्}
{इदं मे स्यादिदं मे स्यात्स्नेहो गुणसमुद्भवः}


\twolineshloka
{अभिद्रोहस्तथा माया निकृतिर्मान एव च}
{स्तैन्यं हिंसा जुगुप्सा च परितापः प्रजागरः}


\twolineshloka
{दम्भो दर्पोऽथ रागश्च भक्तिः प्रीतिः प्रमोदनम्}
{द्यूतं च जनवादश्च सम्बन्धाः स्त्रीकृताश्च ये}


\twolineshloka
{नृत्यवादित्रगीतानां प्रसङ्गा ये च केचन}
{सर्व एते गुणा विप्रा राजसाः सम्प्रकीर्तिताः}


\twolineshloka
{भूतभव्यभविष्याणां भावानां भुवि भावनाः}
{त्रिवर्गनिरता नित्यं धर्मोऽर्थः काम इत्यपि}


\twolineshloka
{कामवृत्ताः प्रमोदन्ते सर्वकामसमृद्धिभिः}
{अर्वाक्स्रोतस इत्येते मनुष्या रजसा वृताः}


\threelineshloka
{अस्मिँलोके प्रमोदन्ते जायमानाः पुनःपनः}
{प्रेत्यभाविकमीहन्ते हलौकिकमेव च}
{ददति प्रतिगृह्णन्ति तर्पयन्त्यथ जुह्वति}


\twolineshloka
{रजोगुणा वो बहुधानुकीर्तितायथावदुक्तं गुणवृत्तमेव च}
{नरोपि यो वेद गुणानिमान्सदास राजसैः सर्वगुणैर्विमुच्यते}


\chapter{अध्यायः ३९}
\twolineshloka
{अतः परं प्रवक्ष्यामि तृतीयं गुणमुत्तमम्}
{सर्वभूतहितं लोके सतां धर्ममनिन्दितम्}


\twolineshloka
{आनन्दः प्रीतिरुद्रेकः प्राकाश्यं सुखमेव च}
{अकार्पण्यमसंरम्भः संतोषः श्रद्दधानता}


\twolineshloka
{क्षमा धृतिरहिंसा च समता सत्यमार्जवम्}
{अक्रोधश्चानसूया च शौचं दाक्ष्यं पराक्रमः}


\twolineshloka
{मुधाज्ञानं मुधावृत्तं मुधासेवा मुदाश्रमः}
{एवं यो युक्तधर्मः स्यात्सोमुत्रात्यन्तमश्नुते}


\twolineshloka
{निर्ममो निरहङ्कारो निराशीः सर्वतः समः}
{अकामहत इत्येव सतां धर्मः सनातनः}


\twolineshloka
{विश्रंभो ह्रीस्तितिक्षा च त्यागः शौचमतन्द्रिता}
{आनृशंस्यमसंमोहो दया भूतेष्वपैशुनम्}


\twolineshloka
{हर्षस्तुष्टिर्विस्मयश्च विनयः साधुवृत्तिता}
{शान्तिकर्मविशुद्धिश्च भावशुद्धिर्विमोचनम्}


\twolineshloka
{उपेक्षा ब्रह्मचर्यं च परित्यागश्च सर्वशः}
{निर्ममत्वमनाशीष्ट्वमपरिक्षतधर्मता}


\twolineshloka
{मुधादानं मुधायज्ञो मुधाधीतं मुधाव्रतम्}
{मुधाप्रतिग्रहश्चैव मधाध्रमो मुधातपः}


\twolineshloka
{एवंवृत्तास्तु ये केचिल्लोकेऽस्मिन्सत्वसंश्रयाः}
{ब्राह्मणा ब्रह्मयोनिस्थास्ते धीराः साधुदर्शिनः}


\twolineshloka
{हित्वा सर्वाणि पापानि निःशोका ह्यजरामराः}
{दिव्यं प्राप्य तु ते धीराः कुर्वते वै ततस्तनूः}


\twolineshloka
{ईशित्वं च वशित्वं च लघुत्वं चाणुता तथा}
{विकुर्वते महात्मानो देवास्त्रिदिवगा इव}


\twolineshloka
{ऊर्ध्वस्रोतस इत्येते देवा वैकारिकाः स्मृताः}
{विकुर्वन्तः प्रकृत्या वै दिवं प्राप्तास्ततस्ततः}


\threelineshloka
{यद्यदिच्छन्ति तत्सर्वं भजन्ते विभजन्ति च}
{इत्येतत्सात्विकं वृत्तं कथितं वो द्विजर्षभाः}
{एतद्विज्ञाय लभते विधिवद्यद्यदिच्छति}


\twolineshloka
{प्रकीर्तिताः सत्त्वगुणा विशेषतोयथावदुक्तं गुणवृत्तमेव च}
{नरस्तु यो वेद गुणानिमान्सदागुणान्स भुङ्क्ते न गुणैः स युज्यते}


\chapter{अध्यायः ४०}
\twolineshloka
{नैव शक्या गुणा वक्तुं पृथक्त्वेनैव सर्वशः}
{अविच्छिन्नानि दृश्यन्ते रजः सत्वं तमस्तथा}


\twolineshloka
{अन्योन्यमनुरज्यन्ते ह्यन्योन्येनानुजीविनः}
{अन्योन्यापाश्रयाः सर्वे तथाऽन्योन्यानुवर्तिनः}


\twolineshloka
{यावत्सत्वं रजस्तावद्वर्तते नात्र संशयः}
{यावत्तमश्च सत्वं च रजस्तावदिहोच्यते}


\twolineshloka
{संहत्य कुर्वते यात्रां सहिताः सङ्घचारिणः}
{सङ्घातवृत्तयो ह्येते वर्तन्ते हेत्वहेतुभिः}


\twolineshloka
{उद्रेकव्यतिरिक्तानां तेषामन्योन्यवर्तिनाम्}
{वक्ष्यते तद्यथाऽन्यूनं व्यतिरिक्तं च सर्वशः}


\twolineshloka
{व्यतिरिक्तं तमो यत्र तिर्यग्भावगतं भवेत्}
{अल्पं तत्रथ रजो ज्ञेयं सत्वमल्पतरं तथा}


\twolineshloka
{उद्रिक्तं च रजो यत्र मध्यस्रोतोगतं भवेत्}
{अल्पं तत्र तमो ज्ञेयं सत्वमल्पतरं तथा}


\twolineshloka
{उद्रिक्तं च यदा सत्वमूर्ध्वस्रोतोगतं भवेत्}
{अल्पं तत्र तमो ज्ञेयं रजश्चाल्पतरं तथा}


\twolineshloka
{सत्वं वैकारिकी योनिरिन्द्रियाणां प्रकाशिका}
{न हि सत्वात्परो भावः कश्चिदन्यो विधीयते}


\twolineshloka
{ऊर्ध्वं गच्छन्ति सत्वस्था मध्ये तिष्ठन्ति राजसाः}
{जघन्यगुणसंयुक्ता यान्त्यधस्तामसा जनाः}


\twolineshloka
{तमः शूद्रे रजः क्षत्रे ब्राह्मणे सत्वमुत्तमम्}
{इत्येवं त्रिषु वर्णेषु विवर्तन्ते गुणास्त्रयः}


\twolineshloka
{दूरादपि हि दृश्यनते सहिताः सङ्घचारिणः}
{तमः सत्वं रजश्चैव पृथक्त्वेनानुशुश्रुम}


\twolineshloka
{दृष्ट्वा त्वादित्यमुद्यन्तं कुचोराणां भयं भवेत्}
{अध्वगाः परितप्येयुरुष्णतो दुःखभागिनः}


\twolineshloka
{आदित्यः सत्वमुद्दिष्टः कुचोरास्तु तथा तमः}
{परितापोऽध्वगानां च रजसो गुण उच्यते}


\twolineshloka
{प्राकाश्यं सत्वमादित्यः संतापो रजसो गुणः}
{उपप्लवस्तु विज्ञेयस्तामसस्तस्य पर्वसु}


% Check verse!
एवं ज्योतिष्षु सर्वेषु प्रवर्तन्ते गुणास्त्रयः ॥पर्यायेण च वर्न्तते तत्रतत्र तथातथा
\twolineshloka
{स्थावरेषु तु भावेषु तिर्यग्भावगतं तमः}
{राजसास्तु विवर्तन्ते स्नेहभावस्तु सात्विकैः}


\twolineshloka
{अहस्त्रिधा तु विज्ञेयं त्रिधा रात्रिर्विधीयते}
{मासार्दमासवर्षाणि ऋतवः सन्धयस्तथा}


\twolineshloka
{त्रिधा दानानि दीयन्ते त्रिधा यज्ञः प्रवर्तते}
{त्रिधा लोकास्त्रिधा देवास्त्रिधा विद्यास्त्रिधा गतिः}


\twolineshloka
{भूतं भव्यं भविष्यं च धर्मोऽर्थः काम एव च}
{प्राणापानावुदानश्चाप्येत एव त्रयो गुणाः}


\twolineshloka
{पर्यायेण प्रवर्तन्ते तत्रतत्र तथातथा}
{यत्किञ्चिदिह लोकेऽस्मिन्सर्वमेते त्रयो गुणाः}


\twolineshloka
{त्रयो गुणाः प्रवर्तन्ते ह्यव्यक्ता नित्यमेव तु}
{सत्वं रजस्तमश्चैव गुणसर्गः सनातनः}


\twolineshloka
{तमोऽव्यक्तं शिवं धाम रजो योनिः सनातनः}
{प्रकृतिर्विकारः प्रलयः प्रधानं प्रभवाप्ययौ}


\threelineshloka
{अनुद्रिक्तमनूनं वाऽप्यकम्पमचलं ध्रुवम्}
{सदसच्चैव तत्सर्वमव्यक्तं त्रिगुणं स्मृतम्}
{ज्ञेयानि नामधेयानि नरैरद्यात्मचिन्तकैः}


\twolineshloka
{अव्यक्तनामानि गुणांश्च तत्त्वतोयो वेद सर्वाणि गतीश्च केवलाः}
{विमुक्तदेहः प्रविभागतत्त्ववि-त्स मुच्यते सर्वगुणैर्निरामयः}


\chapter{अध्यायः ४१}
\twolineshloka
{अव्यक्तात्पूर्वमुत्पन्नो महानात्मा महामतिः}
{आदिर्गुणानां सर्वेषां प्रथमः सर्ग उच्यते}


\threelineshloka
{महानात्मा मतिर्विष्णुर्जिष्णुः शंभुश्च वीर्यवान्}
{बुद्धिः प्रज्ञोपलब्धिश्च तता ख्यातिर्धृतिःस्मृतिः}
{}


\twolineshloka
{पर्यायवाचकैः शब्दैर्महानात्मा विभाव्यते}
{तं जानन्ब्राह्मणो विद्वान्प्रमोहं नाधिगच्छति}


\twolineshloka
{सर्वतःपाणिपादं च सर्वतोक्षिशिरोमुखम्}
{सर्वतःश्रुतिमल्लोके सर्वं व्याप्यवतिष्ठति}


\twolineshloka
{महाप्रभावः पुरुषः सर्वस्य हृदि निष्ठितः}
{अणिमा लघिमा प्राप्तिरीशानो ज्योतिरव्ययः}


\twolineshloka
{तत्र बुद्धिमतां लोके सद्भावनिरताश्च ये}
{ध्यानिनो नित्ययोगाश्च सत्यसन्धा जितेन्द्रियाः}


\twolineshloka
{ज्ञानवन्तश्च ये केचिदलुब्धा जितमन्यवः}
{प्रसन्नमनसो धीरा निर्ममा निरहंक्रियाः}


\twolineshloka
{विमुक्ताः सर्व एवैते महत्त्वमुपयान्त्युत}
{आत्मनो महतो वेद यः पुण्यां गतिमुत्तमाम्}


\twolineshloka
{अहङ्कारात्प्रसूतानि महाभूतानि पञ्च वै}
{पृथिवी वायुराकाशमापो ज्योतिश्च पञ्चमम्}


\twolineshloka
{तेषु भूतानि युज्यन्ते महाभूतेषु पञ्चसु}
{ते शब्दस्पर्शरूपेषु रसगन्धक्रियासु च}


\twolineshloka
{महाभूतविनाशान्ते प्रलये प्रत्युपस्थिते}
{सर्वप्राणभृतां धीरा महदुत्पद्यते भयम्}


\twolineshloka
{स धीरः सर्वलोकेषु न मोहमधिगच्छति}
{विष्णुरेवादिसर्गेषु स्वयंभूर्भवति प्रभुः}


\twolineshloka
{एवं हि यो वेद गुहाशयं प्रभुंपरं पुराणं पुरुषं विश्वरूपम्}
{हिरण्मयं बुद्धिमतां परां गतिंस बुद्धिमान्बुद्धिमतीत्य तिष्ठति}


\twolineshloka
{य उत्पन्नो महान्पूर्वमहङ्कारः स उच्यते}
{अहमित्येव सम्भूतो द्वितीयः सर्ग उच्यते}


\twolineshloka
{अहङ्कारश्च भूतादिर्वैकारिक इति स्मृतः}
{तेजसश्चेतना धातुः प्रजासर्गः प्रजापतिः}


\twolineshloka
{देवानां प्रभवो देवो मनसश्च त्रिलोककृत्}
{अहमित्येव तत्सर्वमभिमानः स उच्यते}


\twolineshloka
{अध्यात्मज्ञानतृप्तानां मुनीनां भावितात्मनाम्}
{स्वाध्यायक्रतुसिद्धानामेष लोकः सनातनः}


\twolineshloka
{अहङ्कारेणाहरतो गुणानिमा-न्भूतादिरेवं सृजते स भूतकृत्}
{वैकारिकः सर्वमिदं विचेष्टतेस्वतेजसा रञ्जयते जगत्तथा}


\chapter{अध्यायः ४२}
\twolineshloka
{अहङ्कारात्प्रसूतानि महाभूतानि पञ्च वै}
{पृथिवी वायुराकाशमापो ज्योतिश्च पञ्चमम्}


\twolineshloka
{तेषु भूतानि मुह्यन्ति महाभूतेषु पञ्चसु}
{शब्दस्पर्शनरूपेषु रसगन्धक्रियासु च}


\twolineshloka
{महाभूतविकारान्ते प्रलये प्रत्युपस्थिते}
{सर्वप्राणभूतां धीरा महदुत्पद्यते भयम्}


\twolineshloka
{यद्यस्माज्जायते भूतं तत्र तत्प्रविलीयते}
{लीयन्ते प्रतिलोमानि जायन्ते चोत्तरोत्तरम्}


\twolineshloka
{ततः प्रलीने सर्वस्मिन्भूते स्थावरजङ्गमे}
{स्मृतिमन्तस्तदा धीरा न लीयन्ते कदाचन}


\twolineshloka
{शब्दः स्पर्शस्तथा रूपं रसो गन्धश्च पञ्चमः}
{क्रियाः करणयुक्ताः स्युरनित्या मोहसंज्ञिताः}


\twolineshloka
{लोभप्रजनसम्भूता निर्विशेषा ह्यकिञ्चनाः}
{मांसशोणितसङ्घाता अन्योन्यस्योपजीविनः}


\twolineshloka
{बहिरात्मान इत्येते दीनाः कृपणजीविनः}
{प्राणापानावुदानश्च समानो व्यान एव च}


\twolineshloka
{अन्तरात्मनि चाप्येते नियताः पञ्च वायवः}
{वाङ्मनोबुद्धिरित्येभिः सार्धमष्टात्मकं जगत्}


\twolineshloka
{त्वग्घ्राणश्रोत्रचक्षूंषि रसना वाक्च संयताः}
{विशुद्धं च मनो यस्य बुद्धिश्चाव्यभिचारिणी}


\twolineshloka
{अष्टौ यस्याग्नयो ह्येते दहन्तेऽहङ्क्रियाः सदा}
{स तद्ब्रह्म शुभं याति तस्माद्भूयो न विद्यते}


\twolineshloka
{एकादश च यान्याहुरिन्द्रियाणि विशेषतः}
{अहङ्कारात्प्रसूतानि तानि वक्ष्यामि नामतः}


\twolineshloka
{श्रोत्रं त्वक् चक्षुषी जिह्वा नासिका चैव पञ्चमी}
{पादौ पायुरुपस्थश्च हस्तौ वाग्दशमी भवेत्}


\twolineshloka
{इन्द्रियग्राम इत्येष मन एकादशं भवेत्}
{एतं ग्रामं जयेत्पूर्वं ततो ब्रह्म प्रकाशते}


\twolineshloka
{बुद्धीन्द्रियाणि पञ्चाहुः पञ्च कर्मेन्द्रियाणि च}
{श्रोत्रादीन्यपि पञ्चाहुर्बुद्धियुक्तानि तत्त्वतः}


\twolineshloka
{अविशेषाणि चान्यानि कर्मयुक्तानि यानि तु}
{उभयत्र मनो ज्ञेयं बुद्धिस्तु द्वादशी भवेत्}


\twolineshloka
{इत्युक्तानीन्द्रियाण्येतान्येकादश यथाक्रमम्}
{मन्यन्ते कृतमित्येवं विदित्वा तानि पण्डिताः}


\twolineshloka
{`त्रीणि स्थानानि भूतानां चतुर्थं नोपपद्यते}
{'स्थलमापस्तथाऽऽकाशं जन्म चापि चतुर्विधम्}


\twolineshloka
{अण्डजोद्भिज्जसंस्वेदजरायुजमथापि च}
{चतुर्धा जन्म इत्येतद्भूतग्रामस्य लक्ष्यते}


\twolineshloka
{अपराण्यथ भूतानि खेचराणि तथैव च}
{अण्डजानि विजानीयात्सर्वांश्चैव सरीसृपान्}


\twolineshloka
{स्वेदजाः कृमयः प्रोक्ता जन्तवश्च यथाक्रमम्}
{जन्मद्वितीयमित्येतज्जघन्यतरमुच्यते}


\twolineshloka
{भित्त्वा तु पृथिवीं यानि जायन्ते कालपर्ययात्}
{उद्भिज्जानि च तान्याहुर्भूतानि द्विजसत्तमाः}


\twolineshloka
{द्विपादबहुपादानि तिर्यग्गतिमतीनि च}
{जरायुजानि भूतानि विकृतान्यपि सत्तमाः}


\twolineshloka
{द्विविधा खलु विज्ञेया ब्रह्मयोनिः सनातना}
{तपः कर्म च यत्पुण्यमित्येष विदुषां नयः}


\twolineshloka
{विविधं कर्म विज्ञेयमिज्या दानं च तन्मखे}
{वेदस्याध्ययनं पुण्यमिति वृद्धानुशासनम्}


\twolineshloka
{एतद्यो वेत्ति विधिवत्स मुक्तः स्याद्द्विजर्षभाः}
{विमुक्तः सर्वपापेभ्य इति चैव निबोधत}


% Check verse!
`अतः परं प्रवक्ष्यामि सर्वं विविधमिन्द्रियम् ॥'
\twolineshloka
{आकाशं प्रथमं भूतं श्रोत्रमध्यात्ममुच्यते}
{अधिभूतं तथा शब्दो दिशश्चात्राधिदैवतम्}


\twolineshloka
{द्वितीयं मारुतं भूतं त्वगध्यात्मं च विश्रुतम्}
{स्प्रष्टव्यमधिभूतं तु विद्युत्तत्राधिदैवतम्}


\twolineshloka
{तृतीयं ज्योतिरित्याहुश्चक्षुरध्यात्ममिष्यते}
{अधिभूतं ततो रूपं सूर्यस्तत्राधिदैवतम्}


\twolineshloka
{चतुर्थमापो विज्ञेयं जिह्वा चाध्यात्ममिष्यते}
{अधिभूतं रसश्चात्र सोमस्तत्राधिदैवतम्}


\twolineshloka
{पृथिवी पञ्चमं भूतं घ्राणश्चाध्यात्ममुच्यते}
{अधिभूतं तथा गन्धो वायुस्तत्रादिदैवतम्}


\twolineshloka
{एषु पञ्चसु भूतेषु चतुष्टयविधिः स्मृतः}
{अतः परं प्रवक्ष्यामि सर्वं त्रिविधमिन्द्रियम्}


\twolineshloka
{पादावध्यात्ममित्याहुर्ब्राह्मणास्तत्वदर्शिनः}
{अधिभूतं तु गन्तव्यं विष्णुस्तत्राधिदैवतम्}


\twolineshloka
{अवाग्गतिरपानश्च पायुरध्यात्ममिष्यते}
{अधिभूतं विसर्गश्च मित्रस्तत्राधिदैवतम्}


\twolineshloka
{प्रजनः सर्वभूतानामुपस्थोऽध्यात्ममुच्यते}
{अधिभूतं तथा शुक्रं दैवतं च प्रजापतिः}


\twolineshloka
{हस्तावध्यात्ममित्याहुरद्यात्मविदुषो जनाः}
{अधिभूतं च कर्माणि शक्रस्तत्राधिदैवतम्}


\twolineshloka
{वैश्वदेवी मनःपूर्वा वागध्यात्ममिहोच्यते}
{वक्तव्यमधिभूतं च वह्निस्तत्राधिदैवतम्}


\twolineshloka
{अध्यात्मं मन इत्याहुः पञ्चभूतात्मचारकम्}
{अधिभूतं च सङ्कल्पश्चन्द्रमाश्चाधिदैवतम्}


\twolineshloka
{अहङ्कारस्तथाऽध्यात्मं सर्वसंसारकारणम्}
{अभिमानोऽधिभूतं च रुद्रस्तत्राधिदैवतम्}


\twolineshloka
{अध्यात्मं बुद्धिरित्याहुः षडिन्द्रियविचारिणी}
{अधिभूतं तु विज्ञेयमहस्तत्राधिदैवतम्}


\twolineshloka
{यथावदध्यात्मविधिरेष वः कीर्तितो मया}
{ज्ञानमस्य हि धर्मज्ञाः प्राप्तं ज्ञानवतामिह}


\twolineshloka
{इन्द्रियाणीन्द्रियार्थाश्च महाभूतानि पञ्च च}
{सर्वाण्येतानि संधाय मनसा सम्प्रधारयेत्}


\twolineshloka
{क्षीणे मनसि सर्वस्मिन्न जन्मसुखमिष्यते}
{ज्ञानसम्पन्नसत्त्वानां तत्सुखं विदुषां मतम्}


\twolineshloka
{अतः परं प्रवक्ष्यामि सूक्ष्मभावकरीं शिवाम्}
{निवृत्तिं सर्वभूतेषु मृदुना दारुणेन वा}


\twolineshloka
{गुणागुणमनासङ्गमेकचर्यमनन्तरम्}
{एतद्ब्राह्मणजं वृत्तमाहुरेकपदं सुखम्}


\twolineshloka
{विद्वान्कूर्म इवाङ्गानि कामान्संहृत्य सर्वशः}
{विरजाः सर्वतो मुक्तो यो नरः स सुखी सदा}


\twolineshloka
{कामानात्मनि संयम्य क्षीणतृष्णः समाहितः}
{सर्वभूतसुहृन्मैत्रो ब्रह्मभूयाय कल्पते}


\twolineshloka
{इन्द्रियाणां निरोधेन सर्वेषां विषयैषिणाम्}
{मुनेर्जनपदत्यागादध्यात्माग्निः समिध्यते}


\twolineshloka
{यथाऽग्निरिन्धनैरिद्धो महाज्योतिः प्रकाशते}
{तथेन्द्रियनिरोधेन महानात्मा प्रकाशतदे}


\twolineshloka
{यदा पश्यति भूतानि प्रसन्नात्माऽत्मनो हृदि}
{स्वयंज्योतिस्तदासूक्ष्मात्सूक्ष्मं प्राप्नोत्यनुत्तमम्}


\threelineshloka
{अग्नी रूपं रसं स्रोतो वायुः स्पर्शनमेव च}
{मही गन्धधरा घ्राणमाकाशः श्रवणं तथा}
{`दृश्यमादित्यमेवाहुरध्यात्मविदुषो जनाः ॥'}


\twolineshloka
{रोगशोकसमाविष्टं पञ्चस्रोतःसमावृतम्}
{पञ्चभूतसमायुक्तं नवद्वारं द्विदैवतम्}


\twolineshloka
{रजस्वलमथादृश्यं त्रिगुणं सप्तधातुकम्}
{संसर्गाभिरतं मूढं शरीरमिति धारणा}


\twolineshloka
{दुश्चरं जीवलोकेऽस्मिन्सत्वं प्रति समाश्रितम्}
{एतदेव हि लोकेऽस्मिन्कालचक्रं प्रवर्तते}


\twolineshloka
{एतन्महार्णवं घोरमगाधं मोहसंझितम्}
{विसृजन्संक्षिपेच्चैव मोहयन्स्वापयञ्जगत्}


\twolineshloka
{कामं क्रोधं भयं लोभमभिद्रोहमथानृतम्}
{इन्द्रियाणां निरोधेन सतस्त्यजति दुस्त्यजान्}


\twolineshloka
{यस्यैते निर्जिता लोके त्रिगुणाः पञ्चधातवः}
{व्योम्नि तस्य परं स्थानमानन्तमथ लक्ष्यते}


\twolineshloka
{पञ्चेन्द्रियमहाकूलां मनःस्रोतोभयावहाम्}
{नदीं मोहह्रदां तीर्त्वा कामक्रोधावुभौ जयेत्}


\twolineshloka
{स सर्वदोषनिर्मुक्तस्ततः पश्यति तत्परम्}
{मनो मनसि सन्धाय पश्यन्नात्मानमात्मनि}


\twolineshloka
{सर्ववित्सर्वभूतेषु द्रक्ष्यत्यात्मानमात्मनि}
{एकधा बहुधा चैव विकुर्वाणस्ततस्ततः}


\twolineshloka
{ध्रुवं पश्यति रूपाणि दीपाद्दीपशतं यथा}
{स वै विष्णुश्च मित्रश्च वरुणोऽग्निः प्रजापतिः}


\twolineshloka
{स हि धाता विधाता च स प्रभुः सर्वतोमुखः}
{हृदयं सर्वभूतानां महानात्मा प्रकाशते}


\twolineshloka
{तं विप्रसङ्घाश्च सुरासुराश्चयक्षाः पिशाचाः पितरो वयांसि}
{रक्षोगणा भूतगणाश्च सर्वेमहर्षयश्चैव सदा स्तुवन्ति}


\chapter{अध्यायः ४३}
\twolineshloka
{मनुष्याणां तु राज्यः क्षत्रियो मध्यमो गुणः}
{कुञ्जरो वाहनानां च सिंहश्चारण्यवासिनाम्}


\twolineshloka
{अविः पशूनां सर्वेषामहिस्तु बिलवासिनाम्}
{गवां गोवृषभश्चैव स्त्रीणां पुरुष एव च}


\twolineshloka
{न्यग्रोधो जम्बुवृक्षश्च पिप्पलः शाल्मलिस्तथा}
{शिंशपा मेषशृङ्गश्च तथा कीचकवेणवः}


\twolineshloka
{एते द्रुमाणां राजानो गणानां मरुतस्तथा}
{हिमवान्पारियात्रश्च सह्यो विन्ध्यस्त्रिकूटवान्}


\twolineshloka
{श्वेतो नीलश्च भासश्च राष्ट्रवांश्चैव पर्वतः}
{भृशस्कन्धो महेन्द्रश्च माल्यवान्पर्वतस्तथा}


\twolineshloka
{एते पर्वतराजानो गणानां मरुतस्तथा}
{सूर्यो ग्रहाणामधिपो नक्षत्राणां च चन्द्रमाः}


\twolineshloka
{यमः पितॄणामधिपः सरितामथ सागरः}
{अंभसां वरुणो राजा मरुतामिन्द्र उच्यते}


\twolineshloka
{अर्कोऽधिपतिरुष्णानां ज्योतिषामिन्दुरुच्यते}
{अग्निर्भूतपतिर्नित्यं ब्राह्मणानां बृहस्पतिः}


\threelineshloka
{ओषधीनां पतिः सोमो विष्णुर्बलवतां वरः}
{त्वष्टाऽधिनां पतिः सोमो विष्णुर्बलवतां वरः}
{}


\twolineshloka
{दक्षिणानां तथा यज्ञो वेदानामृच एव च}
{दिशामुदीची विप्राणां सोमो राजा प्रतापवान्}


\twolineshloka
{कुबेरः सर्वरत्नानां देवतानां पुरंदरः}
{एष भूताधिपः सर्गः प्रजानां च प्रजापतिः}


\twolineshloka
{सर्वेषामेव भूतानामहं ब्रह्ममयो महान्}
{भूतं परतरं मत्तो विष्णोर्वाऽपि न विद्यते}


\twolineshloka
{राजाधिराजः सर्वेषां विष्णुर्ब्रह्ममयो महान्}
{ईश्वरं तं विजानीमः स विभुः स प्रजापतिः}


\twolineshloka
{नरकिन्नरयक्षाणां गन्धर्वोरगरक्षसाम्}
{देवदानवनागानां सर्वेषामीश्वरो हि सः}


\twolineshloka
{भगदेवानुयातानां सर्वासां वामलोचना}
{माहेश्वरी महादेवी प्रोच्यते पार्वती हि सा}


\twolineshloka
{उमां देवीं विजानीध्वं नारीणामुत्तमां शुभाम्}
{रतीनां वसुमत्यस्तु स्त्रीणामप्सरसस्तथा}


\twolineshloka
{धर्मिकामाश्च राजानो ब्राह्मणा धर्मसेतवः}
{तस्माद्राजा द्विजातीनां प्रयतेतेह रक्षणे}


\twolineshloka
{राज्ञां हि विषये येषामवसीदन्ति साधवः}
{हीनास्ते स्वगुणैः सर्वैः प्रेत्यावाङ्मार्गगामिनः}


\twolineshloka
{राज्ञां हि विषये येषां साधवः परिरक्षिताः}
{तेऽस्मिँल्लोके प्रमोदन्ते प्रेत्य चानन्दमेव च}


\twolineshloka
{प्राप्नुवन्ति महात्मान इति वित्त द्विजर्षभाः}
{अत ऊर्ध्वं प्रवक्ष्यामि नियतं धर्मलक्षणम्}


\twolineshloka
{अहिंसालक्षणो धर्मो हिंसा चाधर्मलक्षणा}
{प्रकाशलक्षणा देवा मनुष्याः कर्मलक्षणाः}


\twolineshloka
{शब्दलक्षणमाकाशं वायुस्तु स्पर्शलक्षणः}
{ज्योतिषां लक्षणं रूपमापश्च रसलक्षणाः}


\twolineshloka
{धारिणी सर्वभूतानां पृथिवी गन्धलक्षणा}
{स्वरव्यञ्जनसंस्कारा भारती शब्दलक्षणा}


\twolineshloka
{मनसो लक्षणं चिन्ता तथोक्ता बुद्धिरन्वयात्}
{मनसा चिन्तितानर्थान्बुद्ध्या चेह व्यवस्यति}


\twolineshloka
{बुद्धिर्हि व्यवसायेन लक्ष्यते नात्र संशयः}
{लक्षणं मनसो ध्यानमव्यक्तं साधुलक्षणम्}


\twolineshloka
{प्रवृत्तिलक्षणो योगो ज्ञानं संन्यासलक्षणम्}
{तस्माज्ज्ञानं पुरस्कृत्य संन्यसेदिह बुद्धिमान्}


\twolineshloka
{संन्यासी ज्ञानसंयुक्तः प्राप्नोति परमां गतिम्}
{अतीतो द्वन्द्वमभ्येति तमोमृत्युजरातिगः}


\twolineshloka
{धर्मलक्षणसंयुक्तमुक्तं वो विधिवन्मया}
{गुणानां ग्रहणं सम्यग्वक्ष्याम्यहमतः परम्}


\twolineshloka
{पार्तिवो यस्तु गन्धो वै घ्राणेन हि स गृह्यते}
{प्राणस्यश्च तथा वायुर्गन्धज्ञाने विधीयते}


\twolineshloka
{अपां धातू रसो नित्यं जिह्वया स तु गृह्यते}
{जिह्वास्थश्च तथा सोमो रसज्ञाने विधीयते}


\twolineshloka
{तेजसस्तु गुणो रूपं चक्षुषा तच्च गृह्यते}
{चक्षुःस्थश्च ततादित्यो रूपज्ञाने विधीयते}


\twolineshloka
{वायव्यस्तु सदा स्पर्शस्त्वचा प्रज्ञायते च सः}
{त्वक्स्थश्चैव सदा वायुः स्पर्सने स विधीयते}


\twolineshloka
{आकाशस्य गुणो घोषः श्रोत्रेण च स गृह्यते}
{श्रोत्रस्थाश्च दिशः सर्वाः शब्दज्ञाने प्रकीर्तिताः}


\twolineshloka
{मनसश्च गुणश्चिन्ता प्रज्ञया स तु गृह्यते}
{हृदिस्थश्चेतनो धातुर्मनोज्ञाने विधीयते}


\twolineshloka
{बुद्धिरध्यवसायेन ध्यानेन च महांस्तथा}
{निश्चित्य ग्रहणाद्व्यक्तमव्यक्तं नात्र संशयः}


\twolineshloka
{अलिङ्गग्रहणो नित्यः क्षेत्रज्ञो निर्गुणात्मकः}
{तस्मादलिङ्गः क्षेत्रज्ञः केवलं ज्ञानलक्षणः}


\twolineshloka
{अव्यक्तं क्षेत्रमुद्दिष्टं गुणानां प्रभवाप्ययम्}
{सदा पश्याम्यहं लीनं विजानामि शृणोमि च}


\twolineshloka
{पुरुषस्तद्विजानीते तस्मात्क्षेत्रज्ञ उच्यते}
{गुणवृत्तं तथा कृत्स्नं क्षेत्रज्ञः परिपश्यति}


\twolineshloka
{आदिमद्यावसानं तत्सृज्यमानमचेतनम्}
{न गुणा विदुरात्मानं सृज्यमानाः पुनःपुनः}


\twolineshloka
{न सत्यं वेद वै कश्चित्क्षेत्रज्ञस्त्वेव विन्दति}
{गुणानां गुणभूतानां यत्परं परतो महत्}


\twolineshloka
{तस्माद्गुणांश्च तत्वं च परित्यज्येह तत्ववित्}
{क्षीणदोषो गुणान्हित्वा क्षेत्रज्ञं प्रविशत्यथ}


\twolineshloka
{निर्द्वन्द्वो निर्नमस्कारो निःस्वधाकार एव च}
{अचलश्चानिकेतश्च क्षेत्रज्ञः स परो विधिः}


\chapter{अध्यायः ४४}
\twolineshloka
{यदादिमध्यपर्यन्तं ग्रहणोपायमेव च}
{नामलक्षणसंयुक्तं सर्वं वक्ष्यामि तत्त्वतः}


\twolineshloka
{अहः पूर्वं ततो रात्रिर्मासाः शुक्लादयः स्मृताः}
{श्रवणादीनि ऋक्षाणि ऋतवः शिशिरादयः}


\twolineshloka
{भूमिरादिस्तु गन्धानां रसानामाप एव च}
{रूपाणामादिरग्निस्तु स्पर्शादिर्वायुरुच्यते}


\twolineshloka
{शब्दस्यादिस्तथाऽऽकाशमेष भूतकृतो गुणः}
{अतः परं प्रवक्ष्यामि भूतानामादिमुत्तमम्}


\twolineshloka
{आदित्यो ज्योतिषामादिरग्निर्भूतादिरुच्यते}
{सावित्री सर्वविद्यानां देवतानां प्रजापतिः}


\twolineshloka
{ओङ्कारः सर्ववेदानां वचसां प्राण एव च}
{यदस्मिन्नियतं लोके सर्वं सावित्रमुच्यते}


\twolineshloka
{गायत्री च्छन्दसामादिः प्रजानां सर्ग उच्यते}
{गावश्चतुष्पदामादिर्मनुष्याणां द्विजातयः}


\twolineshloka
{श्येनः पतत्रिणामादिर्यज्ञानां हुतमुत्तमम्}
{प्रसर्पिणां तु सर्वेषां ज्येष्ठः सर्पो द्विजोत्तमाः}


\twolineshloka
{कृतमादिर्युगानां च सर्वेषां नात्र संशयः}
{हिरण्यं सर्वरत्नानामोषधीनां यवास्तथा}


\twolineshloka
{सर्वेषां भक्ष्यभोज्यानामन्नं परममुच्यते}
{द्रवाणां चैव सर्वेषां पेयानामाप उत्तमाः}


\twolineshloka
{स्थावराणां तु भूतानां सर्वेषामविशेषतः}
{ब्रह्मक्षेत्रं सदा पुण्यं प्लक्षः प्रवरजः स्मृतः}


\twolineshloka
{अहं प्रजापतीनां च सर्वेषां नात्र संशयः}
{मम विष्णुरचिन्त्यात्मा स्वयंभूरिति संस्मृतः}


\twolineshloka
{पर्वतानां महामेरुः सर्वेषामग्रजः स्मृतः}
{दिशां च प्रदिशां चोर्ध्वं दिक्पूर्वा प्रथमा तथा}


\twolineshloka
{तथा त्रिपथगा गङ्गा नदीनामग्रजा स्मृता}
{तथा सरोदपाननां सर्वेषां सागरोऽग्रजः}


\twolineshloka
{देवदानवभूतानां पिशाचोरगरक्षसाम्}
{नरकिन्नरयक्षाणां सर्वेषामीश्वरः प्रभुः}


\twolineshloka
{आदिर्विश्वस्य जगतो विष्णुर्ब्रह्ममयो महान्}
{ततः परतरं यस्मात्त्रैलोक्ये नेह विद्यते}


\twolineshloka
{आश्रमाणां च सर्वेषां गार्हस्थ्यं नात्र संशयः}
{लोकानामादिरव्यक्तं सर्वस्यान्तस्तदेव च}


\twolineshloka
{अहान्यस्तमयान्तानि उदयान्ता च शर्वरी}
{सुखस्यान्तं सदा दुःखं दुःखस्यान्तं सदा सुखम}


\twolineshloka
{सर्वे क्षयान्ता निचयाः पतनान्ताः समुच्छ्रयाः}
{संयोगाश्च वियोगान्ता मरणान्तं च जीवितम्}


\twolineshloka
{सर्वं कृतं विनाशान्तं जातस्य मरणं ध्रुवम्}
{अशाश्वतं हि लोकेऽस्मिन्सदा स्थावरजङ्गमम्}


\twolineshloka
{इष्टं दत्तं तपोऽधीतं व्रतानि नियमाश्च ये}
{सर्वमेतद्विनाशान्तं ज्ञानस्यान्तो न विद्यते}


\twolineshloka
{तस्माज्ज्ञानेन शुद्धेन प्रशान्तात्मा जितेन्द्रियः}
{निर्ममो निरहंकारो मुच्यते सर्वपाप्मभिः}


\chapter{अध्यायः ४५}
\twolineshloka
{बुद्धिसारं मनस्तम्भमिन्द्रियग्रामबन्धनम्}
{महाभूतारविष्कम्भं निमेषपरिवेष्टनम्}


\twolineshloka
{जराशोकसमाविष्टं व्याधिव्यसनसञ्चरम्}
{देशकालविचारीदं श्रमव्यायामनिःस्वनम्}


\twolineshloka
{अहोरात्रपरिक्षेपं शीतोष्णपरिमण्डलम्}
{सुखदुःखान्तसंश्लेषं क्षुत्पिपासावकीलकम्}


\twolineshloka
{छायातपविलेखं च निमेषोन्मेषविह्वलम्}
{शोकमोहजराकीर्णं वर्तमानमचेतनम्}


\twolineshloka
{मासार्धमासगुणितं विषमं लोकसञ्चरम्}
{तमोनिचयपङ्कं च रजोवेगप्रवर्तकम्}


\twolineshloka
{सत्त्वालङ्कारदीप्तं च गुणसंघातमण्डलम्}
{विरतिग्रहणाभीकं शोकसंहारवर्तनम्}


\twolineshloka
{क्रियाकारणसंयुक्तं रागविस्तारमायतम्}
{लोभेप्सापरिविक्षोभं विविक्तज्ञानसम्भवम्}


\twolineshloka
{भयमोहपरीवारं भूतसंमोहकारकम्}
{आनन्दप्रीतिचारं च कामक्रोधपरिग्रहम्}


\twolineshloka
{महदादिविशेषान्तमव्यक्तं प्रभवाप्ययम्}
{मनोजवनमश्रान्तं कालचक्रं प्रवर्तते}


\twolineshloka
{एतद्द्वन्द्वसमायुक्तं कालचक्रमचेतनम्}
{विसृजेत्संक्षिपेच्चापि बोधयेत्स्वापयेज्जगम्}


\twolineshloka
{कालचक्रप्रवृत्तिं च निवृत्तिं चैव तत्त्वतः}
{यस्तु वेद नरो नित्यं न स भूतेषु मुह्यति}


\twolineshloka
{विमुक्तः सर्वसङ्क्लेशैः सर्वद्वन्द्वातिगो मुनिः}
{विमुक्तः सर्वपापेभ्यः प्राप्नोति परमां गतिम्}


\twolineshloka
{गृहस्थो ब्रह्मचारी च वानप्रस्थोऽथ भिक्षुकः}
{चत्वार आश्रमाः प्रोक्ताः सर्वे गार्हस्थ्यमूलकाः}


\twolineshloka
{यः कश्चिदिह लोकेऽस्मिन्नाश्रमः परिकीर्तितः}
{तस्यान्तगमनं श्रेयः कीर्तिरेषा सनातनी}


\twolineshloka
{संस्कारैः संस्कृतः पूर्वं यथावच्चरितव्रतः}
{जातौ गुणविशिष्टायां समावर्तेत वेदवित्}


\twolineshloka
{स्वदारनिरतो नित्यं शिष्टाचारो जितेन्द्रियः}
{पञ्चभिश्च महायज्ञैः श्रद्दधानो यजेदिह}


\twolineshloka
{देवतातिथिशिष्टाशी निरतो वेदकर्मसु}
{इज्याप्रदानयुक्तश्च यथाशक्ति यथाविधि}


\twolineshloka
{न पाणिपादचपलो न नेत्रचपलो मुनिः}
{न च वागङ्गचपल इति शिष्टस्य गोचरः}


\twolineshloka
{नित्यं यज्ञोपवीती स्याच्छुक्लवासाः शुचिव्रतः}
{नियतो यमदानाभ्यां सदा शिष्टैश्च संविशेत्}


\twolineshloka
{जितशिश्नोदरो मैत्रः शिष्टाचारसमन्वितः}
{वैणवीं धारयेद्यष्टिं सोदकं च कमण्डलुम्}


\threelineshloka
{`त्रीणि धारयते नित्यं कमण्डलुमतन्द्रितः}
{एकमाचमनार्थाय एकं वै पादधावनम्}
{एकं शौचविधानार्थमित्येतत्त्रितयं तथा ॥'}


\twolineshloka
{अधीत्याध्यापनं कुर्यात्तथा यजनयाजने}
{दानं प्रतिग्रहं वाऽपि षङ्गुणां वृत्तिमाचरेत्}


\twolineshloka
{त्रीणि कर्माणि जानीत ब्राह्मणानां तु जीविकाः}
{याजनाध्यापने चोभे शुद्धाच्चापि प्रतिग्रहः}


\twolineshloka
{अथ शेषाणि चान्यानि त्रीणि कर्माणि यानि तु}
{दानमध्ययनं यज्ञो धर्मयुक्तानि तानि तु}


\twolineshloka
{तेष्वप्रमादं कुर्वीत त्रिषु कर्मसु धर्मवित्}
{दान्तो मैत्रः क्षमायुक्तः सर्वभूतसमो मुनिः}


\twolineshloka
{सर्वमेतद्यथाशक्ति विप्रो निर्वर्तयञ्शुचिः}
{एवं युक्तो जयेत्स्वर्गं गृहस्थः संशितव्रतः}


\chapter{अध्यायः ४६}
\twolineshloka
{एवमेतेन मार्गेण पूर्वोक्तेन यथाविधि}
{अधीतवान्यथाशक्ति तथैव ब्रह्मचर्यवान्}


\twolineshloka
{स्वधर्मनिरतो विद्वान्सर्वेन्द्रिययतो मुनिः}
{गुरोः प्रियहिते युक्तः सत्यधर्मपरः शुचिः}


\twolineshloka
{गुरुणा समनुज्ञातो भुञ्जीतान्नमकुत्सयन्}
{हविष्यभैक्ष्यभुक् चापि स्थानासनविहारवान्}


\twolineshloka
{द्विकालमग्निं जुह्वानः शुचिर्भूत्वा समाहितः}
{धारयीत सदा दण्डं बैल्वं पालाशमेव वा}


\twolineshloka
{क्षौमं कार्पासिकं वाऽपि मृगाजिनमथापि वा}
{सर्वं काषायरक्तं वा वासो वाऽपि द्विजस्य ह}


\twolineshloka
{मेखला च भवेन्मौञ्जी जटो नित्योदकस्तथा}
{यज्ञोपवीती स्वाध्यायी अलुप्तनियतव्रतः}


\twolineshloka
{पूताभिश्च तथैवाद्भिः सदा दैवततर्पणम्}
{भावेन नियतः कुर्वन्ब्रह्मचारी प्रशस्यते}


\twolineshloka
{एवं युक्तो जयेत्स्वर्गमूर्ध्वरेताः समाहितः}
{न संसरति जातीषु परमं स्थानमाश्रितः}


\twolineshloka
{संस्कृतः सर्वसंस्कारैस्तथैव ब्रह्मचर्यवान्}
{ग्रामान्निष्क्रम्य चारण्ये मुनिः प्रव्रजितो वसेत्}


\twolineshloka
{चर्मवल्कलसंवासी सायं प्रातरुपस्पृशेत्}
{अरण्यगोचरो नित्यं न ग्रामं प्रविशेत्पुनः}


\twolineshloka
{अर्चयन्नतिथीन्काले दद्याच्चापि प्रतिश्रयम्}
{फलपत्रावरैर्मूलैः श्यामाकेन च वर्तयन्}


\twolineshloka
{स नित्यमुदकं वायुं सर्वं वानेयमाश्रयेत्}
{प्राश्नीयादानुपूर्व्येण यथादीक्षमतन्द्रितः}


\twolineshloka
{समूलफलशाकाद्यैरर्चेदतिथिमागतम्}
{यद्भक्षः स्यात्ततो दद्याद्भिक्षां नित्यमतन्द्रितः}


\twolineshloka
{देवतातिथिपूर्वं च सदा प्राश्नीत वाग्यतः}
{अस्कन्दितमनाश्चैव लघ्वाशी देवताश्रयः}


\twolineshloka
{दान्तो मैत्रः क्षमायुक्तः कशाञ्शमश्रु च धारयन्}
{जुह्वन्स्वाध्यायशीलश्च सत्यधर्मपरायणः}


\twolineshloka
{न्यस्तदेहः सदा दक्षो वननित्यः समाहितः}
{एवं युक्तो जयेत्स्वर्गं वानप्रस्थो जितेन्द्रियः}


\twolineshloka
{गृहस्थो ब्रह्मचारी च वानप्रस्थोऽथवा पुनः}
{य इच्छेन्मोक्षमास्थातुमुत्तमां वृत्तिमाश्रयेत्}


\twolineshloka
{अभयं सर्वभूतेभ्यो दत्त्वा नैष्कर्म्यमाचरेत्}
{सर्वभूतहितो मैत्रः सर्वेन्द्रिययतो मुनिः}


\twolineshloka
{अयाचितमसंक्लृप्तमुपपन्नं यदृच्छया}
{कृत्वा प्राह्णे चरेद्भैक्ष्यं विधूमे भुक्तवज्जने}


\threelineshloka
{वृत्ते शरावसम्पाते भैक्ष्यं लिप्सेत मोक्षवित्}
{लाभेन च न हृष्येत नालाभे विमना भवेत्}
{न चातिभिक्षां भिक्षेत केवलं प्राणयात्रिकः}


\twolineshloka
{यात्रार्थी कालमाकाङ्क्षंश्चरेद्भैक्ष्यं समाहितः}
{लाभं साधारणं नेच्छेन्न भुञ्जीताभिपूजितः}


\twolineshloka
{अभिपूजितलाभाद्वि विजुगुप्सेत भिक्षुकः}
{भुक्तान्यन्नानि तिक्तानि कषायकटुकानि च}


\twolineshloka
{नास्वादयीत भुञ्जानो रसांश्च मधुरांस्तथा}
{यात्रामात्रं च भुञ्जीत केवलं प्राणधारणम्}


\twolineshloka
{असंरोधेन भूतानां वृत्तिं लिप्सेत मोक्षवित्}
{न चान्यमन्नं लिप्सेत भिक्षमाणः कथञ्चन}


\twolineshloka
{न सन्निकाशयेद्धर्मं विविक्ते चारजाश्चरेत्}
{शून्यागारमण्यं वा वृक्षमूलं नदीं तथा}


\twolineshloka
{प्रतिश्रयार्थं सेवेत पार्वतीं वा पुनर्गुहाम्}
{ग्रामैकरात्रिको ग्रीष्मे वर्षास्वेकत्र वा वसेत्}


\twolineshloka
{अध्वा सूर्येणि निर्दिष्टः कीटवच्च चरेन्महीम्}
{दयार्तं चैव भूतानां समीक्ष्य पृथिवीं चरेत्}


\twolineshloka
{सञ्चयांश्च न कुर्वीत स्नेहवासं च वर्जयेत्}
{पूताभिरद्भिर्नित्यं वै कार्यं कुर्वीत मोक्षवित्}


\twolineshloka
{उपस्पृशेदुद्दृताभिरद्भिश्च पुरुषः सदा}
{अहिंसा ब्रह्मचर्यं च सत्यमार्जवमेव च}


\twolineshloka
{अक्रोधश्चानसूया च दमो नित्यमपैशुनम्}
{अष्टस्वेतेषु युक्तः स्याद्व्रतेषु नियतेन्द्रियः}


\twolineshloka
{अपापमशठं वृत्तमजिह्मं नित्यमाचरेत्}
{जोषयेत सदा भोज्यं ग्रासमागतमस्पृहः}


\twolineshloka
{यात्रामात्रं च भुञ्जीत केवलं प्राणयात्रिकम्}
{धर्मलब्धमथाश्नीयान्न काममनुवर्तयेत्}


\twolineshloka
{ग्रासादाच्छादनादन्यन्न गृह्णीयात्कथञ्चन}
{यावदाहारयेत्तावत्प्रतिगृह्णीत नाधिकम्}


\twolineshloka
{परेभ्यो न प्रतिग्राह्यं न च देयं कदाचन}
{दैन्यभावाच्च भूतानां संविभज्य सदा बुधः}


\twolineshloka
{नाददीत परस्वानि न गृह्णीयान्न याचयेत्}
{न किञ्चिद्विषयं भुक्त्वा स्पृहयेत्तस्य वै पुनः}


\twolineshloka
{मृदमापस्तथाऽन्नानि पत्रपुष्पफलानि च}
{असंवृतानि गृह्णीयात्प्रवृत्तानि च कार्यवान्}


\twolineshloka
{न शिल्पजीविकां जीवेद्द्विरन्नं नोत कामयेत्}
{न द्वेष्टा नोपदेष्टा च भवेच्च निरुपस्कृतः}


\twolineshloka
{श्रद्धापूतानि भुञ्जीत निमित्तानि च वर्जयेत्}
{मुधावृत्तिरसक्तश्च सर्वभूतैरसंधितः}


\twolineshloka
{आशीर्युक्तानि सर्वाणि हिंसायुक्तानि यानि च}
{लोकसङ्ग्रहधर्मं च नैव कुर्यान्न कारयेत्}


\twolineshloka
{सर्वभावानतिक्रम्य लघुमात्रः परिव्रजेत्}
{समः सर्वेषु भूतेषु स्थावरेषु चरेषु च}


\twolineshloka
{परं नोद्वेजयेत्कञ्चिन्न च कस्यचिदुद्विजेत्}
{विश्वास्यः सर्वभूतानामग्र्यो मोक्षविदुच्यते}


\twolineshloka
{अनागतं च न ध्यायेन्नातीतमनुचिन्तयेत्}
{वर्तमानमुपेक्षेत कालाकाङ्क्षी समाहितः}


\twolineshloka
{न चक्षुषा न मनसा न वाचा दूषयेत्क्वचित्}
{न प्रत्यक्षं परोक्षं वा किञ्चिद्दुष्टं समाचरेत्}


\twolineshloka
{इन्द्रियाण्युपसंहृत्य कूर्मोऽङ्गानीव सर्वशः}
{क्षीणेन्द्रियमनोबुद्धिर्निरीहः सर्वतत्त्ववित्}


\twolineshloka
{निर्द्वन्द्वो निर्नमस्कारो निःस्वाहाकार एव च}
{निर्ममो निरहङ्कारो निर्योगक्षेम आत्मवान्}


\twolineshloka
{निराशीर्निर्गुणः शान्तो निरासक्तो निराश्रयः}
{आत्मसङ्गी च तत्त्वज्ञो मुच्यते नात्र संशयः}


\twolineshloka
{अपादपाणिपृष्ठं तदशिरस्कमनूदरम्}
{अभिन्नगुणकर्माणं केवलं विमलं स्थिरम्}


\twolineshloka
{अगन्धमरसस्पर्शमरूपाशब्दमेव च}
{अनुगम्यमनासक्तममांसमपि चैव यत्}


\twolineshloka
{निशअचिन्तमव्ययं दिव्यं गृहस्थमपि सर्वदा}
{सर्वभूतस्थमात्मानं ये पश्यन्ति न ते मृताः}


\twolineshloka
{न तत्र क्रमते बुद्धिर्नेन्द्रियाणि न देवताः}
{वेदा यज्ञाश्च लोकाश्च न तपो न व्रतानि च}


\twolineshloka
{यत्र ज्ञानवतां प्राप्तिलिङ्गग्रहणा स्मृता}
{तस्मादलिङ्गधर्मज्ञो धर्मतत्त्वमुपाचरेत्}


\twolineshloka
{गूढधर्माश्रितो विद्वान्विज्ञानचरितं चरेत्}
{अमूढो मूढरूपेण चरेद्धर्ममदूषयन्}


\twolineshloka
{यथैनमवमन्येरन्परे सततमेव हि}
{तथावृत्तश्चरेच्छान्तः सतां धर्मानकुत्सयन्}


\twolineshloka
{य एवं वृत्तसम्पन्नः स मुनिः श्रेष्ठ उच्यते}
{इन्द्रियाणीन्द्रियार्थांश्च महाभूतानि पञ्च च}


\twolineshloka
{मनो बुद्धिरहङ्कारमव्यक्तं पुरुषं तथा}
{एतत्सर्वं प्रसङ्ख्याय यथावत्तत्त्वनिश्चयात्}


\twolineshloka
{ततः स्वर्गमवाप्नोति विमुक्तः सर्वबन्धनैः}
{एतावदन्तवेलायां परिसङ्ख्याय तत्त्ववित्}


\twolineshloka
{ध्यायेदेकान्तमास्थाय मुच्यतेऽथ निराश्रयः}
{निर्मुक्तः सर्वसङ्गेभ्यो वायुराकाशगो यथा}


% Check verse!
क्षीणकोशो निरातङ्कस्तथेदं प्राप्नुयात्परम्
\chapter{अध्यायः ४७}
\twolineshloka
{संन्यासं तप इत्याहुर्वृद्धा निश्चितवादिनः}
{ब्राह्मणा ब्रह्मयोनिस्था ज्ञानं ब्रह्म परं विदुः}


\twolineshloka
{अतिदूरात्मकं ब्रह्म वेदविद्याव्यपाश्रयम्}
{निर्द्वन्द्वं निर्गुणं नित्यमचिन्त्यगुणमुत्तमम्}


\twolineshloka
{ज्ञानेन तपसा चैव धीराः पश्यन्ति तत्परम्}
{निर्णिक्तमनसः पूता व्युत्क्रान्तरजसोऽमलाः}


\twolineshloka
{तपसा क्षेममध्वानं गच्छन्ति परमेश्वरम्}
{संन्यासनिरता नित्यं ये च ब्रह्मविदो जनाः}


\twolineshloka
{तपः प्रदीप इत्याहुराचारो धर्मसाधकः}
{ज्ञानं वै परमं विद्यात्संन्यासं तप उत्तमम्}


\twolineshloka
{यस्तु वेद निराबाधं ज्ञानं तत्त्वविनिश्चयात्}
{सर्वभूतस्थमात्मानं स सर्वविदिहोच्यते}


\twolineshloka
{यो विद्वान्सहवासं च विवासं चैव पश्यति}
{तथैवेकत्वनानात्वे स दुःखात्प्रतिमुच्यते}


\twolineshloka
{यो न कामयते किञ्चिन्न किञ्चिदवमन्यते}
{इह लोकस्थ एवैष ब्रह्मभूयाय कल्पते}


\twolineshloka
{प्रधानगुणतत्त्वज्ञः सर्व भूतविधानवित्}
{निर्ममो निरहङ्कारो मुच्यते नात्र संशयः}


\twolineshloka
{निर्द्वन्द्वो निर्नमस्कारो निःस्वधाकार एव च}
{निर्गुणं नित्यमद्वन्द्वं प्रशमेनैव गच्छति}


\twolineshloka
{हित्वा गुणमयं सर्वं कर्म जन्तुः शुभाशुभम्}
{उभे सत्यानृते हित्वा मुच्यते नात्र संशयः}


\twolineshloka
{अव्यक्तबीजप्रभवो बुद्धिस्कन्धमयो महान्}
{महाहङ्कारविटप इन्द्रियान्तरकोटरः}


\twolineshloka
{महाभूतविशाखश्च विशेषप्रतिशाखवान्}
{सदापत्रः सदापुष्पः शुभाशुभफलोदयः}


\twolineshloka
{आजीव्यः सर्वभूतानां ब्रह्मवृक्षः सनातनः}
{एनं छित्त्वा च भित्त्वा च तत्त्वज्ञानासिना बुधः}


\twolineshloka
{हित्वा सङ्गमयान्पाशान्मृत्युजन्मजरोदयान्}
{निर्ममो निरहङ्कारो मुच्यते नात्र संशयः}


\twolineshloka
{द्वाविमौ पक्षिणौ नित्यौ संक्षेपौ चाप्यचेतनौ}
{एताभ्यां तु परो योन्यश्चेतनावान्स उच्यते}


\twolineshloka
{अचेतनः सत्वसङ्ख्याविमुक्तःसत्त्वात्परं चेतयतेऽन्तरात्मा}
{स क्षेत्रवित्सर्वसङ्ख्यातबुद्धि-र्गुणातिगो मुच्यते सर्वपापैः}


\chapter{अध्यायः ४८}
\threelineshloka
{केचिद्ब्रह्ममयं वृक्षं केचिद्ब्रह्ममयं वनम्}
{केचित्परममव्यक्तं केचित्परमनामयम्}
{मन्यन्ते सर्वमप्येतदव्यक्तप्रभवाव्ययम्}


\twolineshloka
{उच्छ्वासमात्रमपि चेद्योऽन्तकाले समो भवेत्}
{आत्मानमुपसङ्म्य सोमृतत्वाय कल्पते}


\twolineshloka
{निमेषमात्रमपि चेत्संयम्यात्मानमात्मनि}
{गच्छत्यात्मप्रसादेन विदुषां प्राप्तिमव्ययाम्}


\twolineshloka
{प्राणायामैरथ प्राणान्संयम्य स पुनःपुनः}
{दशद्वादशभिर्वापि चतुर्विंशात्समन्ततः}


\twolineshloka
{एवं पूर्वं प्रसन्नात्मा लभते यद्यदिच्छति}
{अव्यक्तात्सत्वमुद्रिक्तममृतत्वाय कल्पते}


\threelineshloka
{सत्वात्परतरं नान्यत्प्रशंसन्तीह तद्विदः}
{अनुमानाद्विजानीमः पुरुषं सत्वसंश्रयम्}
{न शक्यमन्यथा गन्तुं पुरुषं द्विजसत्तमाः}


\twolineshloka
{क्षमा धृतिरहिंसा च समता सत्यमार्जवम्}
{ज्ञानं त्यागोथ संन्यासः सात्विकं वृत्तमिष्यते}


\twolineshloka
{एतेनैवानुमानेन मन्यन्ते वै मनीषिणः}
{सत्वं च पुरुषश्चैव तत्र नास्ति विचारणा}


\twolineshloka
{आहुरेके च विद्वांसो ये ज्ञाने सुप्रतिष्ठिताः}
{क्षेत्रज्ञसत्वयोरैक्यमित्येतन्नोपपद्यते}


\twolineshloka
{पृथग्भूतस्तथा नित्यमित्येतदविचारितम्}
{पृथग्भावश्च विज्ञेयः सहजश्चापि तत्त्वतः}


\twolineshloka
{तथैवैकत्वनानात्वमिष्यते विदुषां नयः}
{मशकोदुम्बरे चैक्यं पृथक्त्वमपि दृश्यते}


\threelineshloka
{मत्स्यो यथाऽन्यः स्यादप्सु सम्प्रयोगस्तथा तयोः}
{सम्बन्धस्तोयबिन्दूनां पर्णैः कोकनदस्य च ॥गुरुरुवाच}
{}


\twolineshloka
{इत्युक्तवन्तस्ते विप्रास्तदा लोकपितामहम्}
{पुनः संशयमापन्नाः पप्रच्छुर्मुनिसत्तमाः}


\chapter{अध्यायः ४९}
\twolineshloka
{को वा स्विदिह धर्माणामनुष्ठेयतमो मतः}
{व्याहतामिव पश्यामो धर्मस्य विविधां गतिम्}


\twolineshloka
{ऊर्द्ध्वं देहाद्वदन्त्येके नैतदस्तीति चापरे}
{केचित्संशयितं सर्वं निःसंशयमथापरे}


\twolineshloka
{अनित्यं नित्यमित्येके नास्त्यस्तीत्यपि चापरे}
{त्रिधेत्येके द्विधेत्येके व्याकीर्णमिति चापरे}


\twolineshloka
{मन्यन्ते ब्राह्मणा एव ब्रह्मज्ञास्तत्त्वदर्शिनः}
{एकमेके पृथक्चान्ये बहुत्वमिति चापरे}


\twolineshloka
{देशकालावुभौ केचिन्नैतदस्तीति चापरे}
{जटाजिनधराश्चान्ये मुण्डाः केचिदसंवृताः}


\twolineshloka
{अश्नानं केचिदिच्छन्ति स्नानमप्यपरे जनाः}
{मन्यन्ते ब्राह्मणा देवा ब्रह्मज्ञास्तत्त्वदर्शिनः}


\twolineshloka
{आहारं केचिदिच्छन्ति केचिच्चानशने रताः}
{कर्म केचित्प्रशंसन्ति प्रशान्ति चापरे जनाः}


\threelineshloka
{केचिन्मोक्षं प्रशंसन्ति केचिद्भोगान्पृथग्विधान्}
{धनानि केचिदिच्छन्ति निर्धनत्वमथापरे}
{उपास्य साधनं त्वेके नैतदस्तीति चापरे}


\twolineshloka
{अहिंसानिरताश्चान्ये केचिद्धिंसापरायणाः}
{पुण्येन यशसा चान्ये नैतदस्तीति चापरे}


\twolineshloka
{सद्भावनिरताश्चान्ये केचित्संशयिते स्थिताः}
{दुःखादन्ये सुखादन्ये ध्यानमित्यपरे जनाः}


\twolineshloka
{यज्ञ इत्यपरे विप्राः प्रदानमिति चापरे}
{तपस्त्वन्ये प्रशंसन्ति स्वाध्यायमपरे जनाः}


\twolineshloka
{ज्ञानं संन्यासमित्येके स्वभावं भूतचिन्तकाः}
{सर्वमेके प्रशंसन्ति न सर्वमिति चापरे}


\twolineshloka
{एवं व्युत्थापिते धर्मे बहुधा विप्रबोधिते}
{निश्चयं नाधिगच्छामः श्रेयः किमिति सत्तम}


\twolineshloka
{इदं श्रेय इदं श्रेय इत्येवं व्युस्थितो जनः}
{यो हि यस्मिन्रतो धर्म स तं पूजयते सदा}


\twolineshloka
{तेन नोऽविहिता प्रज्ञा मनश्च बहुलीकृतम्}
{एतदाख्यातमिच्छामः श्रेयः किमिति सत्तम}


\twolineshloka
{अतः परं तु यद्गुह्यं तद्भवान्वक्तुमर्हति}
{सत्वक्षेत्रज्ञयोश्चापि सम्बन्धः केन हेतुना}


\twolineshloka
{एवमुक्तः स तैर्विप्रैर्भगवाँल्लोकभावनः}
{तेभ्यः शशंस धर्मात्मा याथातथ्येन बुद्धिमान्}


\chapter{अध्यायः ५०}
\twolineshloka
{हन्त वः सम्प्रक्ष्यामि यन्मां पृच्छथ सत्तमाः}
{गुरुणा शिष्यमासाद्य यदुक्तं तन्निबोधत}


\twolineshloka
{समस्तमिह तच्छ्रुत्वा सम्यगेवावधार्यताम्}
{अहिंसा सर्वभूतानामेतत्कृत्यतमं मतम्}


\threelineshloka
{एतत्पदमनुद्विग्नं वरिष्ठं धर्मलक्षणम्}
{ज्ञानं निःश्रेय इत्याहुर्वृद्धा निश्चितदर्शिनः}
{तस्माज्ज्ञानेन शुद्धेन मुच्यते सर्वकिल्बिषैः}


\twolineshloka
{हिंसापराश्च ये केचिद्ये च नास्तिकवृत्तयः}
{लोभमोहसमायुक्तास्ते वै निरयगामिनः}


\twolineshloka
{आशीर्युक्तानि कर्माणि कुर्वते ये त्वतन्द्रिताः}
{तेऽस्मिन्लोके प्रमोदन्ते जायमानाः पुनः पुनः}


\twolineshloka
{कुर्वते ये तु कर्माणि श्रद्धधाना विपश्चितः}
{अनाशीर्योगसंयुक्तास्ते धीराः साधुदर्शिनः}


\twolineshloka
{अतः परं प्रवक्ष्यामि सत्वक्षेत्रज्ञयोर्यथा}
{संयोगो विप्रयोगश्च तन्निबोधत सत्तमाः}


\twolineshloka
{विषयो विषयित्वं च सम्बन्धोऽयमिहोच्यते}
{विषयी पुरुषो नित्यं सत्वं च विषयः स्मृतः}


\threelineshloka
{व्याख्यातं पूर्वकल्पेनि मशकोदुम्बरं यथा}
{भुज्यमानं न जानीते नित्यं सत्वमचेतनम्}
{यस्त्वेवं तं विजानीते यो भुङ्क्ते यश्च भुज्यते}


\twolineshloka
{अनित्यं द्वन्द्वसंयुक्तं सत्वमाहुर्मनीषिणः}
{निर्द्वन्द्वो निष्कलो नित्यः क्षेत्रज्ञो निर्गुणात्मकः}


\twolineshloka
{समः संज्ञानुगश्चैव स सर्वत्र व्यवस्थितः}
{न सज्जते सदा सत्वमापः पुष्करपर्णवत्}


\twolineshloka
{सर्वैरपि गुणैर्विद्वान्व्यतिषक्तो न लिप्यते}
{जलबिन्दुर्यथा लोलः पद्मिनीपत्रसंस्थितः}


\twolineshloka
{एवमेवाप्यसंयुक्तः पुरुषः स्यान्न संशयः}
{द्रव्यमात्रमभूत्सत्वं पुरुषस्येति निश्चयः}


\threelineshloka
{यथा द्रव्यं च कर्ता च संयोगोऽप्यनयोस्तथा}
{यथा प्रदीपमादाय कश्चित्तमसि गच्छति}
{तथा सत्त्वप्रदीपेन गच्छन्ति परमर्षयः}


\twolineshloka
{यावद्द्रव्यं गुणस्तावत्प्रदीपः सम्प्रकाशते}
{क्षीणे द्रव्ये गुणे ज्योतिरन्तर्धानाय गच्छति}


\twolineshloka
{व्यक्तः सत्वगुणस्त्वेवं पुरुषो द्रव्यमुच्यते}
{एतद्विप्रा विजानीत हन्त भूयो ब्रवीमि वः}


\twolineshloka
{सहस्रेणापि दुर्मेधा न बुद्धिमधिगच्छति}
{चतुर्थेनाप्यथांशेन बुद्धिमान्सुखमेधते}


\twolineshloka
{एवं धर्मस्य विज्ञेयं संसाधनमुपायतः}
{उपायज्ञो हि मेधावी सुखमत्यन्तमश्नुते}


\twolineshloka
{यथाऽध्वानमपाथेयः प्रपन्नो मनुजः क्वचित्}
{क्लेशेन याति महता विनश्यत्यन्तराऽपि च}


\twolineshloka
{तथा कर्मसु विज्ञेयं फलं भवति वा न वा}
{पुरुषस्यात्मनिःश्रेयः शुभाशुभनिदर्शनम्}


\twolineshloka
{यथा च दीर्घमध्वानं पद्म्यामेव प्रपद्यते}
{अदृष्टपूर्वं सहसा तत्त्वदर्शनवर्जितः}


\twolineshloka
{तमेव च यथाऽध्वानं रथेनेहाशुगामिना}
{गच्छत्यश्वप्रयुक्तेन तथा बुद्धिमतां गतिः}


\twolineshloka
{ऊर्ध्वं पर्वतमारुह्य नान्ववेक्षेत भूतलम्}
{रथेन रथिनं पश्येत्क्लिश्यमानमचेतनम्}


\twolineshloka
{यावद्रथपथस्तावद्रथेन स तु गच्छति}
{क्षीणे रथपदे विद्वान्रथमुत्सृज्य गच्छति}


\twolineshloka
{एवं गच्छति मेधावी तत्त्वयोगविधानवित्}
{परिज्ञाय गुणज्ञश्च उत्तरादुत्तरोत्तरम्}


\twolineshloka
{यथाऽर्णवं महाघोरमप्लवः सम्प्रगाहते}
{बाहुभ्यामेव सम्मोहाद्वधं वाञ्छत्यसंशयम्}


\twolineshloka
{नावा चापि यथा प्राज्ञो विभागज्ञः स्वरित्रया}
{अश्रान्तः सलिलं गहाच्छीध्रं संतरते ह्रदम्}


\twolineshloka
{तीर्णो गच्छेत्परं पारं नावमुत्सृज्य निर्ममः}
{व्याख्यातं पूर्वकल्पेन यथा रथपदातिनोः}


\twolineshloka
{स्नेहात्सम्मोहमापन्नो नावि दाशो यथा तथा}
{ममत्वेनाभिभूतः संस्तत्रैव परिवर्तते}


\twolineshloka
{नावं न शक्यमारुह्य स्थले विपरिवर्तितुम्}
{तथैव रथमारुह्य नाप्सु चर्या विधीयते}


\twolineshloka
{एवं कर्म कृतं वित्त विषयस्थं पृथक्पृथक्}
{यथा कर्म कृतं लोके तथा तदुपपद्यते}


\twolineshloka
{यन्नैव गन्धि नो रस्यं न रूपस्पर्सशब्दवत्}
{मन्यते न मनो बुद्ध्या तत्प्रधानं प्रचक्षते}


\twolineshloka
{तत्र प्रधानमव्यक्तमव्यक्तस्य गुणो महान्}
{महप्रधानभूतस्य गुणोऽहङ्कार एव च}


\twolineshloka
{अहङ्कारात्तु सम्भूतो महाभूतकृतो गुणः}
{पृथक्त्वेन हि भूतानां विषया वै गुणाःस्मृताः}


\twolineshloka
{बीजधर्मं यथाऽव्यक्तं तथैव प्रसवात्मकम्}
{बीजधर्मा महानात्मा प्रसवश्चेति नः श्रुतम्}


\twolineshloka
{बीजधर्मात्साहङ्कारात्प्रसवश्च पुनःपुनः}
{बीजप्रसवधर्माणि महाभूतानि पञ्च वै}


\twolineshloka
{बीजधर्मिण इत्याहुः प्रसवं च प्रकुर्वते}
{विशेषाः पञ्चभूतानां तेषां वित्त विशेषणम्}


\twolineshloka
{तत्रैकगुणमाकाशं द्विगुणो वायुरुच्यते}
{त्रिगुणं ज्योतिरित्याहुरापश्चापि चतुर्गुणाः}


\twolineshloka
{पृथ्वी पञ्चगुणा ज्ञेया चरस्थावरसङ्कुला}
{सर्वभूतकरी देवी शुभाशुभनिदर्शिनी}


\twolineshloka
{शब्दः स्पर्शस्तथा रूपं रसो गन्धश्च पञ्चमः}
{एते पञ्चगुणा भूमेर्विज्ञेया द्विजसत्तमाः}


\twolineshloka
{पार्थिवश्च सदा गन्धो गन्धश्च बहुधा स्मृतः}
{तस्य गन्धस्य वक्ष्यामि विस्तरेण बहून्गुणान्}


\twolineshloka
{इष्टश्चानिष्टगन्धश्च मधुरोऽम्लः कटुस्तथा}
{निर्हारी संहतः स्निग्धो रूक्षो विशद एव च}


\twolineshloka
{एवं दशविधो ज्ञेयः पार्थिवो गन्ध इत्युत}
{शब्दः स्पर्शस्तथा रूपं द्रवश्चापां गुणाः स्मृताः}


\twolineshloka
{रसज्ञानं तु वक्ष्यामि रसस्तु बहुधा स्मृतः}
{मधुरोऽम्लः कटुस्तिक्तः कषायो लवणस्तथा}


\twolineshloka
{एवं षड्विधविस्तारो रसो वारिमयः स्मृतः}
{शब्दः स्पर्शस्तथा रूपं त्रिगुणं ज्योतिरुच्यते}


\twolineshloka
{ज्योतिषश्च गुणो रूपं रूपं च बहुधा स्मृतम्}
{शुक्लं कृष्णं तथा रक्तं नीलं पीतारुणं तथा}


\twolineshloka
{ह्रस्वं दीर्घं कृशं स्थूलं चतुरश्राणुवृत्तकम्}
{एवं द्वादशविस्तारं तेजसो रूपमुच्यते}


\twolineshloka
{विज्ञेयं ब्राह्मणैर्वृद्धैर्धर्मज्ञैः सत्यवादिभिः}
{शब्दस्पर्शौ च विज्ञेयौ द्विगुणो वायुरुच्यते}


\twolineshloka
{वायोश्चापि गुणः स्पर्शः स्पर्सश्च बहुधा स्मृतः}
{उष्णः शीतः सुखो दुःखः स्निग्धो विशद एव च}


\twolineshloka
{कठिनश्चिक्वणः श्लक्ष्णः पिच्छिलो दारुणो मृदुः}
{एवं द्वादशविस्तारो वायव्यो गुण उच्यते}


% Check verse!
विधिवद्ब्राह्मणैः सिद्धैर्मन्त्रज्ञैस्तत्त्वदार्शिभिः
\twolineshloka
{तत्रैकगुणमाकाशं शब्द इत्येव च स्मृतः}
{तस्य शब्दस्यि वक्ष्यामि विस्तरेण बहून्गुणान्}


\threelineshloka
{षड्जर्षभः सगान्धारो मध्यमः पञ्चमस्तथा}
{अतः परं तु विज्ञेयो निषादो धैवतस्तथा}
{इष्टश्चानिष्टशब्दश्च संहतः प्रतिभानवान्}


\twolineshloka
{एवं बहुविधो ज्ञेयः शब्द आकाशसम्भवः}
{आकाशमुत्तमं भूतमहङ्कारस्ततः परः}


\twolineshloka
{अहङ्कारात्परा बुद्धिर्बुद्धेरात्मा ततः परः}
{तस्मात्तु परमव्यक्तमव्यक्तात्पुरुषः परः}


\twolineshloka
{परावरज्ञो भूतानां विधिज्ञः सर्वकर्मणाम्}
{सर्वभूतात्मभूतात्मा यं प्राप्यानन्त्यमश्नुते}


\chapter{अध्यायः ५१}
\twolineshloka
{भूतानामथ पञ्चानामथेषामीश्वरं मनः}
{नियमे च विसर्गे च भूतानां मन एव च}


\twolineshloka
{अधिष्ठातृमनो नित्यं भूतानां महतां तथा}
{बुद्धिरैश्वर्यमाचष्टे क्षेत्रज्ञश्च स उच्यते}


\twolineshloka
{इनद्रियाणि मनो युङ्क्ते सदश्वानिव सारथिः}
{इन्द्रियाणि मनो बुद्धिः क्षेत्रज्ञे युज्यते सदा}


\twolineshloka
{महदश्वसमायुक्तं बुद्धिसंयमनं रथम्}
{समारुह्य स भूतात्मा समन्तात्परिधावति}


\twolineshloka
{इन्द्रियग्रामसंयुक्तो मनःसारथिरेव च}
{बुद्धिसंयमनो नित्यं महान्ब्रह्ममयो रथः}


\twolineshloka
{एवं यो वेत्ति विद्वान्वै सदा ब्रह्ममयं रथम्}
{स धीरः सर्वलोकेषु न मोहमधिगच्छति}


% Check verse!
अव्यक्तादिविशेषान्तं सहस्थावरजङ्गमम् ॥सूर्यचन्द्रप्रभालोकं ग्रहनक्षत्रमण्डितम्
\twolineshloka
{नदीपर्वतजालैश्च सर्वतः परिभूषितम्}
{विविधाभिस्तथा चाद्भिः सततं समलंकृतम्}


\twolineshloka
{अजितं सर्वभूतानां सर्वप्राणभृतां गतिः}
{एतद्ब्रह्मवनं नित्यं तस्मिंश्चरति क्षेत्रवित्}


\threelineshloka
{लोकेऽस्मिन्यानि सत्वानि त्रसानि स्थावराणि च}
{तान्येवाग्रे प्रलीयन्ते पश्चाद्भूतकृता गुणाः}
{गुणेभ्यः पञ्च भूतानि एष भूतसमुच्छ्रयः}


\twolineshloka
{देवा मनुष्या गन्धर्वाः पिशाचासुरराक्षसाः}
{सर्वे स्वभावतः सृष्टा न क्रियाभ्यो न कारणात्}


\threelineshloka
{एते विश्वसृजो विप्रा जायन्तीह पुनः पुनः}
{तेभ्यः प्रसूतास्तेष्वेव महाभूतेषु पञ्चसु}
{प्रलीयन्ते यथाकालमूर्मयः सागरे यथा}


\twolineshloka
{विश्वसृग्भ्यस्तु भूतेभ्यो महाभूतास्तु सर्वशः}
{भूतेब्यश्चापि पञ्चभ्यो भुक्तो गच्छेत्परां गतिम्}


\twolineshloka
{प्रजापतिरिदं सर्वं मनसैवासृजत्प्रभुः}
{तथैव देवानृषयस्तपसा प्रतिपेदिरे}


\twolineshloka
{तपसश्चानुपूर्व्येण फलमूलाशिनस्तथा}
{त्रैलोक्यं तपसा सिद्धाः पश्यन्तीह समाहिताः}


\twolineshloka
{औषधान्यगदादीनि नानाविद्याश्च सर्वशः}
{तपसैव प्रसिद्ध्यन्ति तपोमूलं हि साधनम्}


\twolineshloka
{यद्दुरापं दुराम्नायं दुराधर्षं दुरन्वयम्}
{तत्सर्वं तपसा साध्यं तपो हि दुरतिक्रमम्}


\twolineshloka
{सुरापो ब्रह्महा स्तेनो भ्रूणहागुरुतल्पगः}
{तपसैव सुतप्तेन मुच्यते किल्बिषात्ततः}


\twolineshloka
{मनुष्याः पितरो देवाः पशवो मृगपक्षिणः}
{यानि चान्यानि भूतानि चराणि स्थावराणि च}


\twolineshloka
{तपःपरायणा नित्यं सिद्ध्यन्ते तपसा सदा}
{तथैव तपसा देवा महाभागा दिवं गताः}


\twolineshloka
{आशीर्युक्तानि कर्माणि कुर्वते ये त्वतन्द्रिताः}
{अहङ्कारसमायुक्तास्ते सकाशे प्रजापतेः}


\twolineshloka
{ध्यानयोगेन शुद्वेन निर्ममा निरहंकृताः}
{आप्नुवन्ति महात्मानो महान्तं लोकमुत्तमम्}


\twolineshloka
{ध्यानयोगमुपागम्य प्रसन्नमतयः सदा}
{सुखोपचयमव्यक्तं प्रविशन्त्यात्मवित्तमाः}


\twolineshloka
{ध्यानयोगादुपागम्य निर्ममा निरहंकृताः}
{अव्यक्तं प्रविशन्तीह महतां लोकमुत्तमम्}


\twolineshloka
{अव्यक्तादेव सम्भूताः समयज्ञा गताः पुनः}
{तमोरजोभ्यां निर्मुक्ताः सत्वमास्थाय केवलम्}


\twolineshloka
{निर्मुक्तः सर्वपापेभ्यः सर्वं त्यजति निष्कलः}
{क्षेत्रज्ञ इति तं विद्याद्यस्तं वेद स वेदवित्}


\twolineshloka
{चित्तं चित्तादुपागम्य मुनिरासीत संयतः}
{यच्चित्तस्तन्मना भूत्वा ग्राह्यमेतत्सनातनम्}


\twolineshloka
{अव्यक्तादिविशेषान्तमविद्यालक्षणं स्मृतम्}
{निबोधत तथा ज्ञानं गुणैर्लक्षणमित्युत}


\twolineshloka
{द्व्यक्षरस्तु भवेन्मृत्युस्त्र्यक्षरं ब्रह्म शाश्वतम्}
{ममेति च भवेन्मृत्युर्न ममेति च शाश्वतम्}


\twolineshloka
{कर्म केचित्प्रशंसन्ति मन्दबुद्धितया नराः}
{ये तु वृद्धा महात्मानो न प्रशंसन्ति कर्म ते}


\twolineshloka
{कर्मणा जायते जन्तुर्मूर्तिमान्षोडशात्मकः}
{पुरुषं ग्रसते विद्या तद्ग्राह्यममृताशिनम्}


\twolineshloka
{तस्मात्कर्मसु निःस्नेहा ये केचित्पारदर्शिनः}
{विद्यामयोऽयं पुरुषो न तु कर्ममयः स्मृतः}


% Check verse!
य एवममृतं नित्यमग्राह्यं शश्वदक्षरम् ॥वश्यात्मानमसंश्लिष्टं यो वेद न मृतो भवेत्
\threelineshloka
{अपूर्वमकृतं नित्यं य एनमविचारिणम्}
{य एवं विन्देदात्मानमग्राह्यममृताशनम्}
{अग्राह्यो ह्यमृतो भवति स एभिः कारणैर्ध्रुवः}


\twolineshloka
{आयोज्य सर्वसंस्कारान्संयम्यात्मानमात्मनि}
{स तद्ब्रह्म शुभं वेत्ति यस्माद्भूयो न विद्यते}


\twolineshloka
{प्रसादे चैव सत्वस्य प्रसादं समवाप्नुयात्}
{लक्षणं हि प्रसादस्य यथा स्यात्स्वप्नदर्शनम्}


\twolineshloka
{गतिरेषा तु मुक्तानां ये ज्ञानपरिनिष्ठिताः}
{प्रवृत्तयश्च याः सर्वाः पश्यन्ति परिणामजाः}


\twolineshloka
{एषा गतिर्विरक्तानामेष धर्मः सनातनः}
{एषा ज्ञानवतां प्राप्तिरेतद्वृत्तमनिन्दितम्}


\twolineshloka
{समेन सर्वभूतेषु निस्पृहेण निराशिषा}
{शक्या गतिरियं गन्तुं सर्वत्र समदर्शिना}


\threelineshloka
{एतद्वः सर्वमाख्यातं मया विप्रर्षिसत्तमाः}
{एवमाचरत क्षिप्रं ततः सिद्धिमवाप्स्यथ ॥गुरुरुवाच}
{}


\twolineshloka
{इत्युक्तास्ते तु मुनयो गुरुणा ब्रह्मणा तथा}
{कृतवन्तो महात्मानस्ततो लोकमवाप्नुवन्}


\threelineshloka
{त्वमप्येतन्महाभाग मयोक्तं ब्रह्मणो वचः}
{सम्यगाचर शुद्धात्भंस्ततः सिद्धिमवाप्स्यसि ॥वासुदेव उवाच}
{}


\twolineshloka
{इत्युक्तः स तदा शिष्यो गुरुणा धर्ममुत्तमम्}
{चकार सर्वं कौन्तेय ततो मोक्षमवाप्तवान्}


\threelineshloka
{कृतकृत्यश्च स तदा शिष्यः कुरुकुलोद्वह}
{तत्पदं समनुप्राप्तो यत्र गत्वा न शोचति ॥अर्जुन उवाच}
{}


\threelineshloka
{को न्वसौ ब्राह्मणः कृष्ण कश्च शिष्यो जनार्दन}
{श्रोतव्यं चेन्मयैतद्वै तत्त्वमाचक्ष्व मे विभो ॥वासुदेव उवाच}
{}


\twolineshloka
{अहं गुरुर्महाबाहो मनः शिष्य च विद्धि मे}
{त्वत्प्रीत्या गुह्यमेतच्च कथितं ते धनंजय}


\twolineshloka
{मयि चेदस्ति ते प्रीतिर्नित्यं कुरुकुलोद्वह}
{अध्यात्ममेतच्छ्रुत्वा त्वं सम्यगाचर सुव्रत}


\twolineshloka
{ततस्त्वं सम्यगाचीर्णो धर्मेऽस्मिन्नरिकर्शन}
{सर्वपापविनिर्मुक्तो मोक्षं प्राप्स्यसि केवलम्}


\twolineshloka
{पूर्वमप्येतदेवोक्तं युद्धकाल उपस्थिते}
{मया तव महाबाहो तस्मादत्र मनः कुरु}


\threelineshloka
{मया तु भरतश्रेष्ठ चिरदृष्टः पिता प्रभुः}
{तमहं द्रष्टमिच्छामि सम्मते तव फल्गुन ॥वैशम्पायन उवाच}
{}


\twolineshloka
{इत्युक्तवचनं कृष्णं प्रत्युवाच धनंजयः}
{`यदिष्टं कुरु सर्वेषामीश्वरोऽस्मान्प्रपालय}


\threelineshloka
{नमस्ते सर्वलोकात्मन्नारायण परात्पर}
{मनोमलात्तपोशक्यं कर्म चाविद्यया हतम्}
{दानमप्यर्थदोषेणि नाम तस्मात्कलौ स्मरेत्}


\twolineshloka
{यदि गन्तुं कृता बुद्धिर्वासुदेव नमोस्तु ते}
{'गच्छावो नगरं कृष्ण गजसाह्वयमद्य वै}


\twolineshloka
{समेत्य तत्रि राजानं धर्मात्मानं युधिष्ठिरम्}
{समनुज्ञाप्य राजानं स्वां पुरीं यातुमर्हसि}


\chapter{अध्यायः ५२}
\twolineshloka
{ततोऽभ्यनोदयत्कृष्णो युज्यतामिति दारुकम्}
{मुहूर्तादिव चाचष्ट युक्तमित्येव दारुकः}


\twolineshloka
{तथैव चानुयात्राणि चोदयामास पाण्डवः}
{सन्नह्यध्वं प्रयास्यामो नगर गजसाह्वयम्}


\twolineshloka
{इत्युक्ताः सैनिकास्ते तु सज्जीभूता विशाम्पते}
{आचख्युः सज्जमित्येवं पार्थायामिततेजसे}


\twolineshloka
{ततस्तौ रथमास्थाय प्रयातौ कृष्णपाण्डवौ}
{विकुर्वाणौ कताश्चित्राः प्रीयमाणौ विशाम्पते}


\twolineshloka
{रथस्थं तु महातेजा वासुदेवं धनञ्जयः}
{पुनरेवाब्रवीद्वाक्यमिदं भरतसत्तम}


\twolineshloka
{त्वत्प्रसादाज्ज्यः प्राप्तो राज्ञा वृष्णिकुलोद्वह}
{निहताः शत्रवश्चापि प्राप्तं राज्यमकण्टकम्}


\fourlineindentedshloka
{नाथवन्तश्च भवता पाण्डवा मधुसूदन}
{भवन्तं प्लवमासाद्य तीर्णाः स्म कुरुसागरम्}
{`भक्तांस्त्वमाश्रितानस्मान्पालयामुत्र चेह च}
{'}


\twolineshloka
{विश्वकर्मन्नमस्तेऽस्तु विश्वात्मन्विश्वसत्तम}
{तथा त्वामभिजानामि तथा चाहं भवान्मतः}


\twolineshloka
{त्वत्तेजःसम्भवो नित्यं हुताशो मधुसूदन}
{रतिः क्रीडामयी तुभ्यं माया ते रोदसी विभो}


\twolineshloka
{त्वयि सर्वमिदं विश्वं यदिदं स्थाणु जङ्गमम्}
{त्वं हि सर्वं विकुरुषे भूतग्रामं चतुर्विधम्}


\twolineshloka
{पृथिवीं चान्तरिक्षं च तथा स्थावरजङ्गमम्}
{हसिंतं तेऽमला ज्योत्स्ना ऋतवश्चेन्द्रियाणि ते}


\twolineshloka
{प्राणो वायुः सततगः क्रोधो मृत्युः सनातनः}
{प्रसादे चापि पद्मा श्रीर्नित्यं त्वयि महामते}


\twolineshloka
{रतिस्तुष्टिर्धृतिः क्षान्तिर्मतिः कान्तिश्चराचरम्}
{त्वमेवेह युगान्तेषु निधनं प्रोच्यसेऽनध}


\twolineshloka
{सुदीर्घेणापि कालेन न ते शक्या गुणा मया}
{आत्मा च परमो वक्तुं नमस्ते नलिनेक्षण}


\twolineshloka
{विदितो मे सुदुर्धर्ष नारदाद्देवलात्तथा}
{कृष्णद्वैपायनाच्चेव तथा कुरुपितामहात्}


\twolineshloka
{त्वयि सर्वं समासक्तं त्वमेवैको जनेश्वरः}
{यच्चानुग्रहसंयुक्तमेतदुक्तं त्वयाऽनघ}


\twolineshloka
{एतत्सर्वमहं सम्यगाचरिष्ये जनार्दन}
{इदं चाद्भुतमत्यन्तं कृतमस्मत्प्रियेप्सया}


\twolineshloka
{यत्पापो निहतः सङ्ख्ये कौरव्यो धृतराष्ट्रजः}
{त्वया दग्धं हि तत्सैन्यं मया विजितमाहवे}


\twolineshloka
{भवता तत्कृतं कर्म येनावाप्तो जयो मया}
{दुर्योधनस्य सङ्ग्रामे तव बुद्धिपराक्रमैः}


\twolineshloka
{कर्णस्य च वधोपायो यथावत्सम्प्रदर्शितः}
{सैन्धवस्य च पापस्य भूरिश्रवस एव च}


\threelineshloka
{`तस्मात्त्वमेव सञ्चिन्त्य हितं कुरु यथा तथा}
{'अहं च प्रीयमाणेन त्वया देवकिनन्दन}
{यदुक्तस्तत्करिष्यामि न हि मेऽत्र विचारणा}


\twolineshloka
{राजानं च समासाद्य धर्मात्मानं युधिष्ठिरम्}
{चोदयिष्यामि धर्मज्ञ गमनार्थं तवानघ}


\threelineshloka
{आहृतं हि ममैतत्ते द्वारकागमनं प्रभो}
{अचिरादेव द्रष्टा त्वं मातुलं मे जनार्दन}
{बलदेवं च दुर्धर्षं तथाऽन्यान्वृष्णिपुङ्गवान्}


\twolineshloka
{एवं सम्भाषमाणौ तौ प्राप्तौ वारणसाह्वयम्}
{तथा विविशतुश्चोभौ सम्प्रहृष्टनराकुलम्}


\twolineshloka
{तौ गत्वा धृतराष्ट्रस्य गृहं शक्रगृहोपमम्}
{ददृशाते महाराज धृतराष्ट्रं जनेश्वरम्}


\twolineshloka
{विदुरं च महाबुद्धिं राजानं च युधिष्ठिरम्}
{भीमसेनं च दुर्धर्षं माद्रीपुत्रौ च पाण्डवौ}


\twolineshloka
{धृतराष्ट्रमुपासीनं युयुत्सुं चापराजितम्}
{गान्धारीं च महाप्रज्ञां पृथा कृष्णां च भामिनीम्}


\twolineshloka
{सुभद्राद्याश्च ताः सर्वा भरतानां स्त्रियस्तथा}
{ददृशाते स्त्रियः सर्वा गान्धारीपरिचारिकाः}


\twolineshloka
{ततः समेत्य राजानं धृतराष्ट्रमरिंदमौ}
{निवेद्य नामधेये स्वे तस्य पादावगृह्णताम्}


\twolineshloka
{गान्धार्याश्च पृथायाश्च धर्मराजस्य चैव हि}
{भीमस्य च महात्मानौ तथा पादावगृह्णताम्}


\threelineshloka
{क्षत्तारं चापि सङ्गृह्य पृष्ट्वा कुशलमव्ययम्}
{`परिष्वज्य महात्मानं वेश्यापुत्रं महारथम्}
{'तैः सार्धं नृपतिं वृद्धं ततस्तौ पर्युपासताम्}


\twolineshloka
{ततो निशि महाराजो धृतराष्ट्रः कुरूद्वहान्}
{जनार्दनं च मेधावी व्यसर्जयत वै गृहान्}


\twolineshloka
{तेऽनुज्ञाता नृपतिना ययुः स्वं स्वं निवेशनम्}
{धनंजयगृहानेव ययौ कृष्णस्तु वीर्यवान्}


\twolineshloka
{तत्रार्चितो यथान्यायं सर्वकामैरुपस्थितः}
{कृष्णः सुष्वाप मेधावी धनंजयसहायवान्}


\threelineshloka
{प्रभातायां तु शर्वर्यां कृत्वा पौर्वाह्णिकीं क्रियाम्}
{धर्मराजस्य भनं जग्मतुः परमार्चितौ}
{यत्रास्ते स सहामात्यो धर्मराजो महाबलः}


\twolineshloka
{तौ प्रविश्य महात्मानौ तद्गृहं परमार्चितम्}
{धर्मराजं ददृशतुर्देवराजमिवाश्विनौ}


\twolineshloka
{समासाद्य तु राजानं वार्ष्णेयकुरुपुङ्गवौ}
{निषीदतुरनुज्ञातौ प्रीयमाणेन तेन तौ}


\twolineshloka
{ततः स राजा मेधावी विवक्षू प्रेक्ष्य तावुभौ}
{प्रोवाच वदतां श्रेष्ठो वचनं राजसत्तमः}


\twolineshloka
{विवक्षू हि युवां मन्ये वीरौ यदुकुरूद्वहौ}
{ब्रूतं कर्तास्मि सर्वं वां नचिरान्मा विचार्यताम्}


\twolineshloka
{इत्युक्तः फल्गुनस्तत्र धर्मराजानमब्रवीत्}
{विनीतवदुपागम्य वाक्यं वाक्यविशारदः}


\twolineshloka
{अयं चिरोषितो राजन्वासुदेवः प्रतापवान्}
{भवन्तं समनुज्ञाप्य पितरं द्रष्टुमिच्छति}


\threelineshloka
{स गच्छेदभ्यनुज्ञातो भवता यदि मन्यसे}
{आनर्तनगरीं वीरस्तदनुज्ञातुमर्हसि ॥युधिष्ठिर उवाच}
{}


\twolineshloka
{पुण्डरीकाक्ष भद्रं ते गच्छ त्वं मदुसूदन}
{पुरीं द्वारवतीमद्य द्रष्टुं शूरसुतं प्रभो}


\twolineshloka
{रोचते मे महाबाहो गमनं तव केशव}
{मातुलश्चिरदृष्टो मे त्वया देवी च देवकी}


\twolineshloka
{समेत्यि मातुलं गत्वा बलदेवं च मानद}
{पूजयेथा महाप्राज्ञ मद्वाक्येन यथाऽर्हतः}


\twolineshloka
{स्मरेथाश्चापि मां नित्यं भीमं च बलिनां वरम्}
{फाल्गुनं सहदेवं च नकुलं चैव मानद}


\twolineshloka
{आनर्तानवलोक्य त्वं पितरं च महाभुजः}
{वृष्णींश्च पुनरागच्छेर्हयमेधे ममानघ}


\twolineshloka
{स गच्छ रत्नान्यादाय विविधानि वसूनि च}
{यच्चप्यन्यन्मनोज्ञं ते तदप्यादत्स्व सात्वत}


\twolineshloka
{इयं च वसुधा कृत्स्ना प्रसादात्तव केशव}
{अस्मानुपागता वीर निहताश्चापि शत्रवः}


\twolineshloka
{स्वर्गापवर्गविषयं त्वद्भक्तानां न दुर्लभम्}
{संसारगहने चेद्धपापाग्निप्रशमाम्बुद ॥'}


\twolineshloka
{एवं ब्रुवति कौरव्ये धर्मराजे युधिष्ठिरे}
{वासुदेवो वरः पुंसामिदं वचनमब्रवीत्}


\twolineshloka
{तवैव रत्नानि धनं च केवलंधरा तु कृत्स्ना तु महाभुजाद्य वै}
{यदस्ति चान्यद्द्रविणं गृहे ममत्वमेव तस्येश्वर नित्यमीश्वरः}


\twolineshloka
{तथेत्यथोक्तः प्रतिपूजितस्तदागदाग्रजो धर्मसुतेन वीर्यवान्}
{पितृष्वसारं त्ववदद्यथाविधिसम्पूजितश्चाप्यगमत्प्रदक्षिणम्}


\twolineshloka
{तया स सम्यक् प्रतिनन्दितस्तत-स्तथैव सर्वैर्विदुरादिभिस्तथा}
{विनिर्ययौ नागपुराद्गदाग्रजोरथेन दिव्येन चतुर्भुजः स्वयम्}


\twolineshloka
{रथे सुभद्रामधिरोप्य भामिनींयुधिष्ठिरस्यानुमते जनार्दनः}
{पितृष्वसुश्चापि तथा महाभुजोविनिर्ययौ पौरजनाभिसंवृतः}


\twolineshloka
{तमन्वयाद्वानरवर्यकेतनःससात्यकिर्माद्रवतीसुतावपि}
{अगाधबुद्धिर्विदुरश्च माधवंस्वयं च भीमो गजराजविक्रमः}


\twolineshloka
{निवर्तयित्वा कुरुराष्ट्रवर्धनां-स्ततः स सर्वान्विदुरं च वीर्यवान्}
{जनार्दनो दारुकमाह सत्वरःप्रचोदयाश्वानिति सात्यकिं तथा}


\twolineshloka
{ततो ययौ शत्रुगणप्रमर्दनःशिनिप्रवीरानुगतो जनार्दनः}
{यथा निहत्यारिगणं शतकतु-र्दिवं तथाऽऽनर्तपुरीं प्रतापवान्}


\chapter{अध्यायः ५३}
\twolineshloka
{तथा प्रयान्तं वार्ष्णेयं द्वारकां भरतर्षभाः}
{परिष्वज्य न्यवर्तन्त सानुयात्राः परंतपाः}


\twolineshloka
{पुनःपुनश्च वार्ष्णेयं पर्यष्वजत फल्गुनः}
{आचक्षुर्विषयाच्चैनं स ददर्श पुनःपुनः}


\twolineshloka
{कृच्छ्रेणैव तु तां पार्तो गोविन्दे विनिवेशिताम्}
{संजहार ततो दृष्टिं कृष्णश्चाप्यपराजितः}


\twolineshloka
{तस्य प्रयाणे यान्यासन्निमित्तानि महात्मनः}
{बहून्यद्भुतरूपाणि तानि मे गदतः शृणु}


\twolineshloka
{वायुर्वेगेन महता रथस्य पुरतो ववौ}
{कुर्वन्निःशर्करं मार्गं विरजस्कमकण्टकम्}


\twolineshloka
{ववर्ष वासवश्चैव तोयं शुचि सुगन्धि च}
{दिव्यानि चैव पुष्पाणि पुरतः शाङ्गधन्वनः}


\twolineshloka
{स प्रयातो महाबाहुः समेषु मरुधन्वसु}
{ददर्शाथ मुनिश्रेष्ठमुदङ्कममितौजसम्}


\threelineshloka
{`महर्षिं सिद्धतपसं सर्वलोकान्तविश्रुतम्}
{'स तं सम्पूज्य तेजस्वी मुनिं पृथुललोचनः}
{पूजितस्तेन च तदा पर्यपृच्छदनामयम्}


\twolineshloka
{स पृष्टः कुशलं तेन सम्पूज्य मधुसूदनम्}
{उदङ्को ब्राह्मणश्रेष्ठस्ततः पप्रच्छ माधवम्}


\twolineshloka
{कच्चिच्छौरे त्वया गत्वा कुरुपाण्डवसद्म तत्}
{कृतं सौभ्रात्रमचलं तन्मे व्याख्यातुमर्हसि}


\twolineshloka
{अपि सन्धाय तान्वीरानुपावृत्तोसि केशव}
{सम्बन्दिनःक स्वदयितान्सततं वृष्णिपुङ्गव}


\twolineshloka
{कच्चित्पाण्डुसुताः पञ्च धृतराष्ट्रस्य चात्मजाः}
{लोकेषु विहरिष्यन्ति त्वया सह परंतप}


\twolineshloka
{स्वराष्ट्रे ते च राजानः कच्चित्प्राप्स्यन्ति वै सुखम्}
{कौरवेषु प्रशान्तेषु त्वया नाथेन केशव}


\threelineshloka
{या मे सम्भावना तात त्वयि नित्यमवर्तत}
{अपि सा सफला तात कृता ते भरतान्प्रति ॥श्रीभगवानुवाच}
{}


\threelineshloka
{कृतो यत्नो मया पूर्वं सौशाम्ये कौरवान्प्रति}
{नाशक्यन्त यदा साम्ये ते स्थापयितुमञ्जसा}
{}


\twolineshloka
{न दिष्टमप्यतिक्रान्तुं शक्यं बुद्ध्या बलेन वा}
{महर्षे विदितं भूयः सर्वमेतत्तवानघ}


\twolineshloka
{तेऽत्यक्रामन्मतिं मह्यं भीष्मस्य विदुरस्य च}
{ततो यमक्षयं जग्मुः समासाद्येतरेतरम्}


\twolineshloka
{पञ्चैव पाण्डवाः शिष्टा इतमित्रा हतात्मजाः}
{धार्तराष्ट्राश्च निहताः सर्वे ससुतबान्धवाः}


\twolineshloka
{इत्युक्तवचने कृष्णे भृशं क्रोधसमन्वितः}
{उदङ्क इत्युवाचैनं रोषादुत्फुल्ललोचनः}


\twolineshloka
{यस्माच्छक्तेन ते कृष्ण न त्राताः कुरुपुङ्गवाः}
{सम्बन्धिनः प्रियास्तस्माच्छप्स्येऽहं त्वामसंशयम्}


\twolineshloka
{न च ते प्रसभं यस्मात्ते निगृह्य निवारिताः}
{तस्मान्मन्युपरीतस्त्वां शप्स्यामि मधुसूदन}


\threelineshloka
{त्वया शक्तेन हि सता मिथ्याचारेणि माधव}
{ते परीताः कुरुश्रेष्ठा नश्यन्तः स्म ह्युपेक्षिताः ॥वासुदेव उवाच}
{}


\twolineshloka
{शृणु मे विस्तरेणेदं यद्वक्ष्ये भृगुनन्दन}
{गृहाणानुनयं चापि तपस्वी ह्यसि भार्गवम्}


\twolineshloka
{श्रुत्वा च मे तदध्यात्मं मुञ्चेथाः शापमद्य वै}
{न च मां तपसाऽल्पेन शक्तोऽभिभवितुं पुमान्}


\twolineshloka
{न च ते तपसो नाशमिच्छामि तपतां वर}
{तपस्ते सुमहद्दीप्तं गुरवश्चापि तोषिताः}


\twolineshloka
{कौमारं ब्रह्मचर्यं ते जानामि द्विजसत्तम}
{दुःखार्जितस्य तपसस्तस्मान्नेच्चामि तेऽव्ययम्}


\chapter{अध्यायः ५४}
\threelineshloka
{ब्रूहि केशव तत्त्वेन त्वमध्यात्ममनिन्दितम्}
{श्रुत्वा श्रेयोऽभिधास्यामि शापं वा ते जनार्दन ॥वासुदेव उवाच}
{}


\twolineshloka
{तमो रजश्च सत्वं च विद्धि भावान्मदाश्रयान्}
{`स्थितिसृष्टिलयाध्यक्षो विष्णुब्रह्मेशसंज्ञितः}


\twolineshloka
{कदाचित्तमसा रुद्रो विष्णुः सत्त्वगुणे स्थितः}
{रजस्यपि तथा ब्रह्मा स्वगुणान्यगुणानुभौ}


\twolineshloka
{प्रणवात्मा च शब्दादींस्त्रिगुणात्मा चराचरम्}
{'तथा रुद्रान्वसून्वाऽपि विद्धि मत्प्रभवान्द्विज}


\twolineshloka
{मयि सर्वाणि भूतानि सर्वभूतेषु चाप्यहम्}
{स्थित इत्यभिजानीहि मा ते भूदत्र संशयः}


\twolineshloka
{तथा दैत्यगणान्सर्वान्यक्षगन्धर्वराक्षसान्}
{नागानप्सरसश्चैव विद्धि मत्प्रभवान्द्विज}


\twolineshloka
{सदसच्चैव यत्प्राहुरव्यक्तं व्यक्तमेव च}
{अक्षरं च क्षरं चैव सर्वमेतन्मदात्मकम्}


\twolineshloka
{ये चाश्रमेषु वै धर्माश्चतुर्षु विहिता मुने}
{वैदिकानि च कर्माणि विद्धि सर्वं मदात्मकम्}


\twolineshloka
{असच्च सच्चैव च यद्विश्वं सदसतः परम्}
{ततः परतरं नास्ति देवदेवात्सनातनात्}


\twolineshloka
{ओंकारप्रभवान्वेदान्विद्धि मां त्वं भृगूद्वह}
{यूपं सोमं चरुं होमं त्रिदशाप्यायनं मखे}


\twolineshloka
{होतारमपि हव्यं च विद्धिं मां भृगुनन्दन}
{अध्वर्युः कल्पकृच्चापि हविः परमसंस्कृतम्}


\twolineshloka
{उद्गाता चापि मां स्तौति गीतघोषैर्महाध्वरे}
{प्रायश्चित्तेषु मां ब्रह्मञ्शान्तिमङ्गलवाचकाः}


\twolineshloka
{स्तुवन्ति विश्वकर्माणं सततं द्विजसत्तम}
{मम विद्धि सुतं धर्ममग्रजं द्विजसत्तम}


\twolineshloka
{मानसं दयितं विप्र सर्वभूतदयात्मकम्}
{तत्राहं वर्तमानैश्च निवृत्तैश्चैव मानवैः}


\twolineshloka
{बह्वीः संसरमाणो वै योनीर्वर्तामि सत्तम}
{लोकसरंक्षणार्थाय धर्मसंस्थापनाय च}


\twolineshloka
{तैस्तैर्वेषैश्च रूपैश्च त्रिषु लोकेषु भार्गव}
{अहं विष्णुरहं ब्रह्मा शक्रोऽथ प्रभवाप्ययः}


\twolineshloka
{भूतग्रामस्य सर्वस्य स्रष्टा संहार एव च}
{अधर्मे वर्तमानानां सर्वेषामहमच्युतः}


\twolineshloka
{धर्मस्य सेतुं बध्नामि चलिते चलिते युगे}
{तास्ता योनीः प्रविश्याहं प्रजानां हितकाम्यया}


\twolineshloka
{यदा त्वहं देवयोनौ वर्तामि भृगुनन्दन}
{तदाऽहं देववत्सर्वमाचरामि न संशयः}


\twolineshloka
{यदा गन्धर्वयोनौ तु वर्तामि भृगुनन्दन}
{तदा गन्धर्ववच्चेष्टा सर्वाश्चेष्टामि भार्गव}


\twolineshloka
{नागयोनौ यदा चैव तदा वर्तामि नागवत्}
{यक्षराक्षसयोन्योस्तु यथावद्विचराम्यहम्}


\twolineshloka
{मानुष्ये वर्तमाने तु कृपणं याचिता मया}
{न च ते जातसम्मोहा वचोऽगृह्णन्त मोहिताः}


\twolineshloka
{भयं च महदुद्दिश्य त्रासिताः कुरवो मया}
{क्रुद्धेन भूत्वा तु पुर्यथावदनुदर्शिताः}


\twolineshloka
{तेऽधर्मेणेह संयुक्ताः परीताःक कालधर्मणा}
{धर्मेण निहता युद्धे गताः स्वर्गं न संशयः}


\twolineshloka
{लोकेषु पाण्डवाश्चैव गताः ख्यातिं द्विजोत्तम}
{एतत्ते सर्वमाख्यातं यन्मां त्वं परिपृच्छसि}


\chapter{अध्यायः ५५}
\twolineshloka
{अभिजानामि जगतः कर्तारं त्वां जनार्दन}
{नूनं भवत्प्रसादोऽयमिति मे नास्ति संशयः}


\twolineshloka
{चित्तं च सुप्रसन्नं मे त्वद्भावगतमच्युत}
{विनिवृत्तश्च मे कोप इति विद्धि परंतप}


\threelineshloka
{यदि त्वनुग्रहं कञ्चित्त्वत्तोऽर्हामि जनार्दन}
{द्रष्टुमिच्छामि ते रूपं वैष्णवं तन्निदर्शय ॥वैशम्पायन उवाच}
{}


\twolineshloka
{ततः स तस्मै प्रीतात्मा दर्शयामास तद्वपुः}
{शाश्वतं वैष्णवं धीमान्ददृशे यद्धनंजयः}


\threelineshloka
{स ददर्श महात्मानं विश्वरूपं महाभुजम्}
{सहस्रसूर्यप्रतिमं दीप्तिमत्पावकोपमम्}
{सर्वमाकाशमावृत्य तिष्ठन्तं सर्वतोमुखम्}


\threelineshloka
{तद्दृष्ट्वा परमं रूपं विष्णोर्वैष्णवमद्भुतम्}
{विस्मयं च ययौ विप्रस्तं दृष्ट्वा परमेश्वरम् ॥उदङ्क उवाच}
{}


\twolineshloka
{`नमोनमस्ते सर्वात्मन्नारायण परात्मक}
{परमात्मन्पद्मनाभ पुण्डरीकाक्ष माधव}


\twolineshloka
{हिरण्यगर्भरूपाय संसारोत्तारणाय च}
{पुरुषाय पुराणाय शान्तश्यामाय ते नमः}


\twolineshloka
{अविद्यातिमिरादित्यं भवव्याधिमहौषधिम्}
{संसारार्णवसारं त्वां प्रणमामि गतिर्भव}


\twolineshloka
{सर्ववेदैकवेद्याय सर्ववेदमयाय च}
{वासुदेवाय नित्याय नमो भक्तप्रियाय ते}


\twolineshloka
{दयया दुःखमोहान्मां सुमुद्धर्तुमिहार्हसि}
{कर्मभिर्बहुभिः पापैर्बद्धं पाहि जनार्दन ॥'}


\twolineshloka
{विश्वकर्मन्नमस्तेऽस्तु विश्वात्मन्विश्वकसम्भव}
{पद्म्यां ते पृथिवी व्याप्ता शिरसा चावृतं नभः}


\twolineshloka
{द्यावापृथिव्योर्यन्मध्यं जठरेण तवावृतम्}
{भुजाभ्यामावृताश्चाशास्त्वमिदं सर्वमच्युत}


\threelineshloka
{संहरस्व पुनर्देव रूपमक्षय्यमुत्तमम्}
{पुनस्त्वां स्वेन रूपेण द्रष्टुमिच्छामि शाश्वतम् ॥वैशम्पायन उवाच}
{}


\threelineshloka
{तमुवाच प्रसन्नात्मा गोविन्दो जनमेजय}
{वरं वृणीष्वेति तदा तमुदङ्कोऽब्रवीदिदम्}
{}


\twolineshloka
{पर्याप्त एष एवाद्य वरस्त्वत्तो महाद्युते}
{यत्ते रूपमिदं कृष्ण पश्यामि प्रभवाप्ययम्}


\threelineshloka
{तमब्रकवीत्पुनः कृष्णो मा त्वमत्र विचारय}
{अवश्यमेतत्कर्तव्यममोघं दर्शनं मम ॥ उदङ्कक उवाच}
{}


\twolineshloka
{अवश्यं करणीयं च यद्येतन्मन्यसे विभो}
{तोयमिच्छामि यत्रेष्टं मरुष्वेतद्धि दुर्लभम्}


\twolineshloka
{ततः संहृत्य तत्तेजः प्रोवाचोदङ्कमीश्वरः}
{एष्टव्ये सति चिन्त्योऽहमित्युक्त्वा द्वारकां ययौ}


\twolineshloka
{ततः कदाचिद्भगवानुदङ्कस्तोयकाङ्क्षया}
{तृषितः परिचक्राम मरौ सस्मार चाच्युतम्}


\twolineshloka
{ततो दिग्वाससं धीमान्मातङ्गं मलपङ्किनम्}
{अपश्यत मरौ तस्मिञ्श्वयूथपरिवारितम्}


\twolineshloka
{भीषणं बद्धनिस्त्रिंशं बाणकार्मुकधारिणम्}
{तस्याधःस्रोतसोऽपश्यद्वारि भूरि द्विजोत्तमः}


\threelineshloka
{स्मरन्नेव च तं प्राह मातङ्गः प्रहसन्निव}
{एह्युदङ्क प्रतीच्छस्व मत्तो वारि भृगूद्वह}
{कृपा हि मे सुमहती त्वां दृष्ट्वा तृट्समाश्रितम्}


\threelineshloka
{इत्युक्तस्तेन स मुनिस्तत्तोयं नाभ्यनन्दन}
{चिक्षेप च स तं धीमान्वाग्भिरुग्राभिरच्युतम्}
{}


\twolineshloka
{पुनःपुनश्च मातङ्ग पिबस्वेति तमब्रवीत्}
{न चापिबत्स सक्रोधः क्षुभितेनान्तरात्मना}


\twolineshloka
{स तथा निश्चयात्तेन प्रत्याख्यातो महात्मना}
{श्वभिः सह महाराज तत्रैवान्तरधीयत}


\twolineshloka
{उदङ्कस्तं तथा दृष्ट्वा ततो व्रीडितमानसः}
{मेने प्रलब्धमात्मानं कृष्णेनामित्रघातिना}


\twolineshloka
{अथ तेनैव मार्गेण शङ्कचक्रगदाधरः}
{आजगाम महाबाहुरुदङ्कश्चैनमब्रवीत्}


\twolineshloka
{न युक्तं तादृशं दातुं त्वया पुरुषसत्तम}
{सलिलं विप्रमुख्येभ्यो मातङ्गस्रोतसा विभो}


\twolineshloka
{इत्युक्तवचनं तं तु महाबुद्धिर्जनार्दनः}
{उदङ्कं श्लक्ष्णया वाचा सान्त्वयन्निदमब्रवीत्}


\twolineshloka
{यादृशेनेह रूपेण योग्यं दातुं धृतेन वै}
{तादृशं खलु ते दत्तं यच्च त्वं नावबुध्यथाः}


\twolineshloka
{मया त्वदर्थमुक्तो वै वज्रपाणिः पुरंदरः}
{उदङ्कायामृतं देहि तोयरूपमिति प्रभुः}


\twolineshloka
{स मामुवाच देवेन्द्र न मर्त्योऽमर्त्यतां व्रजेत्}
{अन्यमस्मै वरं देहीत्यसकृद्भृगुनन्दन}


\twolineshloka
{अमृतं देयमित्येव मयोक्तः स शचीपतिः}
{स मां प्रसाद्य देवेन्द्रः पुनरेवेदमब्रवीत्}


\twolineshloka
{यदि देयमवश्यं वै मातङ्गोऽहं महामते}
{भूत्वाऽमृतं प्रदास्यामि भार्गवाय महात्मने}


\threelineshloka
{यद्येवं प्रतिगृह्णाति भार्गवोऽमृतमद्य वै}
{प्रदातुमेष गच्छामि भार्गवस्यामृतं विभो}
{प्रत्याख्यातस्त्वहं तेन दास्यामि न कथञ्चन}


\threelineshloka
{स तथा समयं कृत्वा तेन रूपेण वासवः}
{उपस्थितस्त्वया चापि प्रत्याख्यातोऽमृतं ददत्}
{चाण्डालरूपी भगवान्सुमहांस्ते व्यतिक्रमः}


\twolineshloka
{यत्तु शक्यं मया कर्तुं भूय एव तवेप्सितम्}
{तोयेप्सां तव दुर्धर्षां करिष्ये सफलामहम्}


\twolineshloka
{येष्वहःसु च ते ब्रह्मन्सलिलेप्सा भविष्यति}
{तदा मरौ भविष्यन्ति जलपूर्णाः पयोधराः}


\twolineshloka
{रसवच्च प्रदास्यन्ति तोयं ते भृगुनन्दन}
{उदङ्गमेघा इत्युक्ताः ख्यातिं यास्यन्ति चापि ते}


\twolineshloka
{इत्युक्तः प्रीतिमान्विप्रः कृष्णेन स बभूव ह}
{अद्याप्युदङ्कमेघाश्च मरौ वर्षन्ति भारत}


\chapter{अध्यायः ५६}
\threelineshloka
{उदङ्कः केन तपसा संयुक्तो वै महामनाः}
{यः शापं दातुकामोऽभूद्विष्णवे प्रभविष्ववे ॥वैशम्पायन उवाच}
{}


\twolineshloka
{उदङ्को महता युक्तस्तपसा जनमेजय}
{गुरभक्तः स तेजस्वी नान्यत्किञ्चिदपूजयत्}


\twolineshloka
{सर्वेषामृषिपुत्राणामेष आसीन्मनोरथः}
{औदङ्कीं गुरुवृत्तिं वै प्राप्नुयामेति भारत}


\twolineshloka
{गौतमस्य तु शिष्याणां बहूनां जनमेजय}
{उदङ्केऽभ्यधिका प्रीतिः स्नेहश्चैवाभवत्तदा}


\twolineshloka
{स तस्य दमशौचाभ्यां विक्रान्तेन च कर्मणा}
{सम्यक्चैवोपचारेण गौतमः प्रीतिमानभूत्}


\twolineshloka
{अथ शिष्यसहस्राणि समनुज्ञाय गौतमः}
{उदङ्कं परया प्रीत्या नाभ्यनुज्ञातुमैच्छत}


\twolineshloka
{तं क्रमेण जरा तात प्रतिपेदे महामुनिम्}
{न चान्वबुध्यत तदा स मुनिर्गुरुवत्सलः}


\twolineshloka
{ततः कदाचिद्राजेन्द्र काष्ठान्यानयितुं ययौ}
{उदङ्कः काष्ठभारं च महान्तं समुपानयत्}


\twolineshloka
{स तद्भाराभिभूतात्मा काष्ठभारमरिंदम}
{निचिक्षेप क्षितौ राजन्परिश्रान्तो बुभुक्षितः}


\twolineshloka
{तस्य काष्ठे विलग्नाऽभूज्जटा रूप्यसमप्रभा}
{ततः काष्ठैः सह तदा पपात धरणीतलम्}


\twolineshloka
{ततः स भारनिष्पिष्टः क्षुधाविष्टश्च भारत}
{दृष्ट्वा तां वयसोऽवस्थां रुरोदार्तस्वरं तदा}


\threelineshloka
{ततो गुरुसुता तस्य पद्मपत्रनिभानना}
{जग्राहाश्रूणि सुश्रोणी करेण पृथुलोचना}
{पितुर्नियोगाद्भावज्ञा शिरसाऽवनता तदा}


\threelineshloka
{तस्या निपेततुर्दग्धौ करौ तैरश्रुबिन्दुभिः}
{न हि तानश्रुपातांस्तु शक्ता धारयितुं मही}
{गौतमस्त्वब्रवीद्विप्रमुदङ्कं प्रीतमानसः}


\threelineshloka
{कस्मात्तात तवाद्येह शोकोत्तरमिदं मनः}
{स स्वैरं ब्रूहि विप्रर्षे श्रोतुमिच्छामि तत्त्वतः ॥उदङ्क उवाच}
{}


\twolineshloka
{भवद्गतेन मनसा भवत्प्रियचिकीर्षया}
{भवद्भक्तिगतेनेह भवद्भावानुगेन च}


\twolineshloka
{जरेयं नाघबुद्धा मे नाभिज्ञातं सुखं च मे}
{शतवर्षोषितं मां हि न त्वमन्यनुजानिथाः}


\threelineshloka
{भवता त्वभ्यनुज्ञाताः शिष्याः प्रत्यवरा मम}
{उपपन्ना द्विजश्रेष्ठ शतशोऽथ सहस्रशः ॥गौतम उवाच}
{}


\twolineshloka
{त्वत्प्रीतियुक्तेन मया गुरुशुश्रूषया तव}
{व्यतिक्रामन्महाकालो नावबुद्धो द्विजर्षभ}


\threelineshloka
{किं त्वद्य यदि ते श्रद्धा गमनं प्रति भार्गव}
{अनुज्ञां प्रतिगृह्य त्वं स्वगृहान्गच्छ माचिरम् ॥उदङ्क उवाच}
{}


\threelineshloka
{गुर्वर्तं कं प्रयच्छामि ब्रूहि त्वं द्विजसत्तम}
{तमुपाहृत्य गच्छेयमनुज्ञातस्त्वया विभो ॥गौतम उवाच}
{}


\threelineshloka
{दक्षिणापरितोषो वै गुरूणां सद्भिरुच्यते}
{तवि ह्याचरतो धर्मं तुष्टोऽहं वै न संशयः}
{इत्थं च परितुष्टं मां विजानीहि भृगूद्वह}


\threelineshloka
{युवा षोडशवर्षो हि यद्यद्य भविता भवान्}
{ददानि पत्नीं कन्यां च स्वां ते दुहितरं द्विज}
{एतामृतेऽङ्गना नान्या त्वत्तेजोऽर्हति सेवितुम्}


\twolineshloka
{ततस्तां प्रतिजग्राह युवा भूत्वा यशस्विनीम्}
{गुरुणा चाभ्यनुज्ञातो गुरुपत्नीमथाब्रवीत्}


\twolineshloka
{किं भवत्यै प्रयच्छामि गुर्वर्थं विनियुङ्क्ष माम्}
{प्रियं हितं च काङ्क्षामि प्राणैरपि धनैरपि}


\threelineshloka
{यद्दुर्लभं हि लोकेऽस्मिन्रत्नमत्यद्भुतं महत्}
{तदानयेयं तपसा न हि मेऽत्रास्ति संशयः ॥अहल्योवाच}
{}


\threelineshloka
{परितुष्टाऽस्मि ते विप्र नित्यं भक्त्या तवानघ}
{पर्याप्तमेतद्भद्रं ते गच्छ तात यथेप्सितम् ॥वैशम्पायन उवाच}
{}


\threelineshloka
{उदङ्कस्तु महाराज पुनरेवाब्रवीद्वचः}
{आज्ञापयस्व मां मातः कर्तव्यं च तव प्रियम् ॥अहल्योवाच}
{}


\threelineshloka
{सौदासपत्न्या विधृते दिव्ये ये मणिकुण्डले}
{ते समानय भद्रं ते गुर्वर्थः सुकृतो भवेत् ॥वैशम्पायन उवाच}
{}


\twolineshloka
{स तथेति प्रतिश्रुत्य जगाम जनमेजय}
{गुरुपत्नीप्रियार्थं वै ते समानयितुं तदा}


\twolineshloka
{स जगाम ततः शीघ्रमुदङ्को ब्राह्मणर्षभः}
{सौदासं पुरुषादं वै भिक्षितुं मणिकुण्डले}


\twolineshloka
{गौतमस्त्वब्रवीत्पत्नीमुदङ्को नात्र दृश्यते}
{इति पृष्टा तमाचष्ट कुण्डलार्थे गतं च सा}


\threelineshloka
{ततः प्रोवाच पत्नीं स न ते सम्यगिदं कृतम्}
{शप्तः स पार्थिवो नूनं ब्राह्मणं तं वधिष्यति ॥अहल्योवाच}
{}


\twolineshloka
{अजानन्त्या नियुक्तः स भगवन्ब्राह्मणो मया}
{भवत्प्रसादान्न भयं किञ्चित्तस्य भविष्यति}


\twolineshloka
{इत्युक्तः प्राह तां पत्नीमेवमस्त्विति गौतमः}
{उदङ्कोऽपि वने शून्ये राजानं तं ददर्श ह}


\chapter{अध्यायः ५७}
\twolineshloka
{स तं दृष्ट्वा तथाभूतं राजानं घोरदर्शनम्}
{दीर्घश्मश्रुधरं नॄणां शोणितेन समुक्षितम्}


\twolineshloka
{चकार न व्यथां विप्रो राजा त्वेनमथाब्रवीत्}
{प्रत्युत्थाय महातेजा भयकर्ता यमोपमः}


\threelineshloka
{दिष्ट्या त्वमसि कल्याण षष्ठे काले ममान्तिकम्}
{भक्ष्यं मृगयमाणस्य सम्प्राप्तो द्विजसत्तम ॥उदङ्ग उवाच}
{}


\threelineshloka
{राजन्गुर्वर्थिनं विद्धि चरन्तं मामिहागतम्}
{न च गुर्वर्थमुद्युक्तं हिंस्यमाहुर्मनीषिणः ॥राजोवाच}
{}


\threelineshloka
{षष्ठे काले ममाहारो विहितो द्विजसत्तम}
{न शक्यस्त्वं समुत्स्रष्टुं क्षुधितेन मयाऽद्य वै ॥उदङ्क उवाच}
{}


\twolineshloka
{एवमस्तु महाराजि समयः क्रियतां तु मे}
{गुर्वर्थमभिनिर्वर्त्य पुनरेष्यामि ते वशम्}


\twolineshloka
{संश्रुतश्च मया योऽर्थो गुरवे राजसत्तम}
{त्वदधीनः स राजेन्द्र तं त्वां भिक्षे नरेश्वर}


\threelineshloka
{ददासि विप्रमुख्येभ्यस्त्वं हि रत्नानि नित्यदा}
{दाता च त्वं नरव्याघ्र पात्रभूतः क्षिताविह}
{पात्रं प्रतिग्रहे चापि विद्धि मां नृपसत्तम}


\twolineshloka
{उपाहृत्य गुरोरर्थं त्वदायत्तमरिंदम}
{समयेनेह राजेन्द्र पुनरेष्यामि ते वशम्}


\threelineshloka
{सत्यं ते प्रतिजानामि नात्र मिथ्या कथञ्चन}
{अनृतं नोक्तपूर्वं मे स्वैरेष्वपि कुतोऽन्यथा ॥सौदास उवाच}
{}


\threelineshloka
{यदि मत्तस्तवायत्तो गुर्वर्थः कृत एव सः}
{यदि चास्ति प्रतिग्राह्यं साम्प्रतं तद्वदस्व मे ॥उदङ्क उवाच}
{}


\threelineshloka
{प्रतिग्राह्यो मतो मे त्वं सदैव पुरुषर्षभ}
{सोहं त्वामनुसम्प्राप्तो भिक्षितुं मणिकुण्डले ॥सौदास उवाच}
{}


\threelineshloka
{पत्न्यास्ते मम विप्रर्षे उचिते मणिकुण्डले}
{वरयार्थं त्वमन्यं वै तं ते दास्यामि सुव्रत ॥उदङ्ग उवाच}
{}


\threelineshloka
{अलं ते व्यपदेशेन प्रमाणं यदि ते वयम्}
{प्रयच्छ कुण्डले मह्यं सत्यवाग्भव पार्थिव ॥वैशम्पायन उवाच}
{}


\twolineshloka
{इत्युक्तस्त्वब्रवीद्राजा तमुदङ्कं पुनर्वचः}
{गच्छ मद्वचनाद्देवीं ब्रूहि देहीति सत्तम}


\threelineshloka
{सैवमुक्ता त्वया नूनं मद्वाक्येन शुचिव्रता}
{प्रदास्यति द्विजश्रेष्ठ कुण्डले ते न संशयः ॥उदङ्क उवाच}
{}


\threelineshloka
{क्व पत्नी भवतः शक्या मया द्रष्टुं नरेश्वर}
{स्वयं वाऽपि भवान्पत्नीं किमर्थं नोपसर्पति ॥सौदास उवाच}
{}


\threelineshloka
{तां द्रक्ष्यति भवानद्य कास्मिंश्चिद्वननिर्झरे}
{षष्ठे काले न हि मया सा शक्या द्रष्टुमद्य वै ॥वैशम्पायन उवाच}
{}


\twolineshloka
{उदङ्कस्तु तथोक्तः स जगाम भरतर्षभ}
{मदयन्तीं च दृष्ट्वा स ज्ञापयत्स्वप्रयोजनम्}


\twolineshloka
{सौदासवचनं श्रुत्वा ततः सा पृथुलोचना}
{प्रत्युवाच महावुद्धिमुदङ्कं जनमेजय}


\twolineshloka
{एवमेतन्महाब्रह्मन्नानृतं वदसेऽनघ}
{अभिज्ञानं तु किञ्चित्त्वं समानयितुमर्हसि}


\twolineshloka
{इमे हि दिव्ये मणिकुण्डले मेदेवाश्च यक्षाश्च महर्षयश्च}
{तैस्तैरुपायैरपहर्तुकामा-श्छिद्रेषु नित्यं परितर्कयन्ति}


\twolineshloka
{निक्षिप्तमेतद्भुवि पन्नगास्तुरत्नं समासाद्य परामृशेयुः}
{यक्षास्तथोच्छिष्टधृतं सुराश्चनिद्रावशाद्वा परिधर्षयेयुः}


\twolineshloka
{छिद्रेष्वेतेष्विमे नित्यं ह्रियते द्विजसत्तम}
{देवराक्षसनागानामप्रमत्तेनि धार्यते}


\twolineshloka
{एते दिवापि भासेते रात्रौ च द्विजसत्तम}
{नक्तं नक्षत्रताराणां प्रभामाक्षिप्य वर्ततः}


\twolineshloka
{एते ह्यामुच्य भगन्क्षुत्पिपासाभयं कुतः}
{विषाग्निश्वापदेभ्यश्च भयं जातु न विद्यते}


\twolineshloka
{ह्रस्वेन चैते आमुक्ते भवतो ह्रस्वके तदा}
{अनुरूपेण चामुक्ते जायेते तत्प्रमाणके}


\twolineshloka
{एवंविधे ममैते वै कुण्डले परमार्चिते}
{त्रिषु लोकेषु विज्ञाते तदभिज्ञानमानय}


\chapter{अध्यायः ५८}
\threelineshloka
{स मित्रसहमासाद्य अभिज्ञानमयाचत}
{तस्मै ददावभिज्ञानं स चेक्ष्वाकुवरस्तदा ॥सौदास उवाच}
{}


\twolineshloka
{न चैवैषा गतिः क्षेम्या न चान्या विद्यते गतिः}
{एतन्मे तत्वमाज्ञाय प्रयच्छ मणिकुण्डले}


\twolineshloka
{इत्युक्तस्तामुदङ्कस्तु भर्तुर्वाक्यमथाब्रवीत्}
{श्रुत्वा च सा तदा प्रादात्ततस्ते मणिकुण्डले}


\threelineshloka
{अवाप्य कुण्डले ते तु राजानं पुनरब्रवीत्}
{किमेतद्गुह्यवचनं श्रोतुमिच्छामि पार्थिव ॥सौदास उवाच}
{}


\twolineshloka
{प्रजाविसर्गाद्विप्रान्वै क्षत्रियाः पूजयन्ति ह}
{विप्रेभ्यश्चापि बहवो दोषाः प्रादुर्भवन्ति नः}


\twolineshloka
{सोहं द्विजेभ्यः प्रणतो विप्राद्दोषमवाप्तवान्}
{गतिमन्यां न पश्यामि मदयन्तीसहायवान्}


\twolineshloka
{न चान्यामपि पश्यामि गतिं गतिमतांवर}
{स्वर्गद्वारस्य गमने स्थाने चेह द्विजोत्तम}


\twolineshloka
{न हि राज्ञा विशेषेण विरुद्धेन द्विजातिभिः}
{शक्यं हि लोके स्थातुं वै प्रेत्य वा सुखमेधितुम्}


\threelineshloka
{तदिष्टे ते मया दत्ते एते स्वे मणिकुण्डले}
{यः कृतस्तेऽद्य समयः सफलं तं कुरुष्य मे ॥उदङ्क उवाच}
{}


\threelineshloka
{राजंस्तथेह कर्तास्मि पुनरेष्यामि ते वशम्}
{प्रश्नं च कञ्चित्प्रष्टुं त्वां व्यवसिष्ये परंतप ॥सौदास उवाच}
{}


\threelineshloka
{ब्रूहि विप्र यथाकामं प्रतिवक्तास्मि ते वचः}
{छेत्तास्मि संशयं तेऽद्य न मेऽत्राश्ति विचारणा ॥उदङ्क उवाच}
{}


\twolineshloka
{प्राहुर्वाक्संयतं विप्रं धर्मनैपुणदर्शिनः}
{मित्रेषु यश्च विषमः स्तेन इत्येव तं विदुः}


\twolineshloka
{स बवान्मित्रतामद्य सम्प्राप्तो मम पार्थिव}
{स मे बुद्धिं प्रयच्छस्व सम्मतां पुरुषर्षभ}


\threelineshloka
{अवाप्तार्थोऽहमद्येह भवांश्च पुरुषादकः}
{भवत्सकाशमागन्तुं क्षमं मम न वेति वै ॥सौदास उवाच}
{}


\twolineshloka
{क्षमं चेदिह वक्तव्यं मया द्विजवरोत्तम}
{मत्समीपं द्विजश्रेष्ट नागन्तव्यं कथञ्चन}


\threelineshloka
{एवं तव प्रपश्यामि श्रेयो भृगुकुलोद्वह}
{आगच्छतो हि ते विप्रि भवेन्मृर्त्युन संशयः ॥वैशम्पायन उवाच}
{}


\twolineshloka
{इत्युक्तः स तदा राजा क्षमं बुद्धिमता हितम्}
{अनुज्ञाप्य स राजानमहल्यां प्रति जग्मिवान्}


\twolineshloka
{गृहीत्वा कुण्डले दिव्ये गुरुपत्न्याः प्रियंकरः}
{जवेन महता प्रायाद्गौतमस्याश्रमं प्रति}


\twolineshloka
{यथा तयो रक्षणं च मदयन्त्याऽभिभाषितम्}
{तथा ते कुण्डले बध्वा तदा कृष्णाजिनेऽनयत्}


\twolineshloka
{स कस्मिंश्चित्क्षुधाविष्टः फलभारसमन्वितम्}
{बिल्वं ददर्श विप्रर्षिरारुरोह च तं ततः}


\twolineshloka
{शाखास्वासज्य तस्यैव कृष्णाजिनमरिंदम}
{पातयामास बिल्वानि तदा स द्विजपुङ्गवः}


\twolineshloka
{अथ पातयमानस्य बिल्वापहृतचक्षुषः}
{न्यपतंस्तानि बिल्वानि तस्मिन्नेवाजिने विभो}


\twolineshloka
{यस्मिंस्ते कुण्डले बद्धे तदा द्विजवरेण वै}
{बिल्वप्रहारैस्तस्याथ व्यशीर्यद्बन्धनं ततः}


\twolineshloka
{सकुण्डलं तदजिनं पपात सहसा तरोः}
{विशीर्णबन्धने तस्मिन्गते कृष्णाजिने महीम्}


\twolineshloka
{अपश्यद्भुजगः कश्चित्ते तत्र मणिकुण्डले}
{ऐरावतकुलोद्भूतः शीघ्रो भूत्वा तदा हि सः}


\twolineshloka
{विदश्यास्येन वल्मीकं विवेशाथ स कुण्डले}
{ह्रियमाणे तु दृष्ट्वा स कुण्डले भुजगेन ह}


\twolineshloka
{पपात वृक्षात्सोद्वेगो दुःखात्परमकोपनः}
{स दण्डकाष्ठमादाय वल्मीकमखनत्तदा}


\twolineshloka
{[अहानि त्रिंशदव्यग्रः पञ्च चान्यानि भारत}
{]क्रोधामर्षाभिसंतप्तस्तदा ब्राह्मणिसत्तमः}


\threelineshloka
{तस्य वेगमसह्यं तमसहन्ती वसुन्धरा}
{दण्डकाष्ठाभिनुन्नाङ्गी चचाल भृशमाकुला}
{ततः खनत एवाथ विप्रर्षेर्धरणीतलम्नागलोकस्य पन्थानं कर्तुकामस्य निश्चयात्}


\threelineshloka
{रथेन हरियुक्तेन तं देशमुपजग्मिवान्}
{वज्रपाणिर्महातेजास्तं ददर्श द्विजोत्तमम् ॥वैशम्पायन उवाच}
{}


\twolineshloka
{स तु तं ब्राह्मणो भूत्वा तस्य दुःखेन दुःखितः}
{उदङ्कमब्रवीद्वाक्यं नैतच्छक्यं त्वयेति वै}


\threelineshloka
{इतो हि नागलोको वै योजनानि सहस्रशः}
{न दण्डकाष्ठसाध्यं च मन्ये कार्यमिदं तव ॥उदङ्क उवाच}
{}


\threelineshloka
{नागलोके यदि ब्रह्मन्न शक्ये कुण्डले मया}
{प्राप्तुं प्राणान्विमोक्ष्यामि पश्यतस्ते द्विजोत्तम ॥वैशम्पायन उवाच}
{}


\twolineshloka
{यदा स नाशकत्तस्य निश्चयं कर्तुमन्यथा}
{वज्रपाणिस्तदा दण्डं वज्रास्त्रेण युयोज ह}


\twolineshloka
{ततो वज्रप्रहारैस्तैर्दार्यमाणा वसुन्धर}
{नागलोकस्य पन्थानमकरोज्जनमेजय}


\twolineshloka
{स तेन मार्गेण तदा नागलोकं विवेश ह}
{ददर्श नागलोकं च योजनानि सहस्रशः}


\twolineshloka
{प्रकारनिचयैर्दिव्यैर्मणिमुक्तास्वलङ्कृतैः}
{उपपन्नं महाभाग शातकुम्भमयैस्तथा}


\twolineshloka
{वापीः स्फटिकसोपाना नदीस्च विमलोदकाः}
{ददर्श वृक्षांश्च बहून्नानाद्विजगणायुतान्}


\twolineshloka
{तस्य लोकस्य च द्वारं स ददर्श भृगूद्वहः}
{पञ्चयोजनविस्तारमायतं शतयोजनम्}


\twolineshloka
{नागलोकमुदङ्कस्तु प्रेक्ष्य दीनोऽभवत्तदा}
{निराशश्चाभवत्तत्र कुण्डलाहरणे पुनः}


\twolineshloka
{तत्र प्रोवाच तुरगस्तं कृष्णश्वेतवालधिः}
{ताम्रास्यनेत्रः कौरव्यः प्रज्वलन्निव तेजसा}


\twolineshloka
{धमस्वापानमेतन्मे ततस्त्वं विप्र लप्स्यसे}
{ऐरावतसुतेनेहि तव्रानीते हि कुण्डले}


\twolineshloka
{मा जुगुप्सां कृथाः पुत्र त्वमत्रार्थे कथञ्चन ॥त्वयैतद्धि समाचीर्णं गौतमस्याश्रमे तदा ॥उदङ्ग उवाच}
{}


\threelineshloka
{कथं भवन्तं जानीयामुपाध्यायाश्रमं प्रति}
{यन्मया चीर्णपूर्वं हि श्रोतुमिच्छामि तद्ध्यहम् ॥अश्व उवाच}
{}


\twolineshloka
{गुरोर्गुरु मां जानीहि ज्वलन्तं जातवेदसम्}
{त्वया ह्यहं सदा विप्र गुरोरर्थेऽभिपूजितः}


\twolineshloka
{विधिवत्सततं विप्र शुचिना भृगुनन्दन}
{तस्माच्छ्रेयो विधास्यामि तवैवं कुरु माचिरम्}


\twolineshloka
{इत्युक्तस्तु तथाऽकार्षीदुदङ्कश्चित्रभानुना}
{ताम्रार्चिः प्रीतिमांश्चापि प्रजज्वाल दिधक्षया}


\twolineshloka
{ततोऽस्य रोमकूपेभ्यो ध्मायमानस्य भारत}
{घनः प्रादुरभूद्धूमो नागलोकभयावहः}


\twolineshloka
{तेन धूमेन महता वर्धमानेन भारत}
{नागलोके महाराज न प्राज्ञायत किञ्चन}


\twolineshloka
{हाहाकृतमभूत्सर्वमैरावतिनिवेशनम्}
{वासुकिप्रमुखानां च नागानां जनमेजय}


\twolineshloka
{न प्राकाशन्त वेश्मानि धूमरुद्धानि भारत}
{नीहारसंवृतानीव वनानि गिरयस्तथा}


\twolineshloka
{ते धूमरक्तनयना वह्नितेजोभितापिताः}
{आजग्मुर्निश्चयं ज्ञातुं भार्गवस्य महात्मनः}


\twolineshloka
{श्रुत्वा च निश्चयं तस्य महर्षेरतितेजसः}
{सम्भ्रान्तनयनाः सर्वे पूजां चक्रुर्यथाविधि}


\twolineshloka
{सर्वे प्राञ्जलयो नागा वृद्धबालपुरोगमाः}
{शिरोभिः प्रणिपत्योचुः प्रसीद भगवन्निति}


\twolineshloka
{प्रसाद्य ब्राह्मणं ते तु पाद्यमर्घ्यं निवेद्य च}
{प्रायच्छन्कुण्डले दिव्ये पन्नगाः परमार्चिते}


\twolineshloka
{ततः स पूजितो नागैस्तदोदङ्कः प्रतापवान्}
{अग्निं प्रदक्षिणं कृत्वा जगाम गुरुसद्म तत्}


\twolineshloka
{स गत्वा त्वरितो राजन्गौतमस्य निवेशनम्}
{प्रायच्छत्कुण्डले दिव्ये गुरुपत्न्यास्तदाऽनघ}


\twolineshloka
{वासुकिप्रमुकानां च नागानां जनमेजय}
{सर्वं शशंस गुरेव यथावद्द्विजसत्तमः}


\twolineshloka
{एवं महात्मना तेन त्रींलोकाञ्जनमेजय}
{परिक्रम्याहृते दिव्ये ततस्ते मणिकुण्डले}


\threelineshloka
{एवंप्रभावः स मुनिरुदङ्को भरतर्षभ}
{परेण तपसा युक्तो यन्मां त्वं परिपृच्छसि}
{}


\chapter{अध्यायः ५९}
\threelineshloka
{उदङ्कस्य वरं दत्त्वा गोविन्दो द्विजत्तम}
{अत ऊर्ध्वं महाबाहुः किं चकार महायशाः ॥वैशम्पायन उवाच}
{}


\twolineshloka
{उदङ्काय वरं दत्त्वा प्रायात्सात्यकिना सह}
{द्वारकामेव गोविन्दः शीघ्रवेगैर्महाहयैः}


\twolineshloka
{सरांसि सरितश्चैव वनानि च गिरींस्तथा}
{अतिक्रम्याससादाथ रम्यां द्वारवतीं पुरीम्}


\twolineshloka
{वर्तमाने महाराज महे रैवतकस्य च}
{उपायात्पुण्डरीकाक्षो युयुधानानुगस्तदा}


\twolineshloka
{अलङ्कृतस्तु स गिरिर्नानारूपैर्विचित्रितैः}
{बभौ रत्नमयैः कोशैः संवृतः पुरुर्षर्षभ}


\twolineshloka
{काञ्चनस्रग्भिरग्र्याभिः सुमनोभिस्तथैव च}
{वासोभिश्च महाशैलः कल्पवृक्षैस्तथैव च}


\twolineshloka
{दीपवृक्षैश्च सौवर्णैरभीक्ष्णमुपशोभितः}
{गुहानिर्झरदेशेषु दिवाभूतो बभूव ह}


\threelineshloka
{एताकाभिर्विचित्राभिः सघण्टाभिः समन्ततः}
{पुंभिः स्त्रीभिश्च संघुष्टः प्रगीत इव चाभवत्}
{अतीव प्रेक्षणीयोऽभून्मेरुर्मुनिगणैरिव}


\twolineshloka
{मत्तानां हृष्टरूपाणां स्त्रीणां पुंसां च भारत}
{गायतां पर्वतेन्द्रस्य दिविस्पृगिव निःस्वनः}


\twolineshloka
{प्रमत्तमत्तसम्मत्तक्ष्वेडितोद्धुष्टसंकुलः}
{तथा किलकिलाशब्दैर्भूधरोऽभून्मनोहरः}


\twolineshloka
{विपणापणवान्रम्यो भक्ष्यभोज्यविहारवान्}
{वस्त्रमाल्योत्करयुतो वीणावेणुमृदङ्गवान्}


\threelineshloka
{सुरामैरेयमिश्रेण भक्ष्यभोज्येन चैव ह}
{दीनान्धकृपणादिभ्यो दीयमानेन चानिशम्}
{बभौ परमकल्याणो महस्तस्य महागिरेः}


\twolineshloka
{पुण्यावसथवान्वीरैः पुण्यकृद्भिर्निषेवितः}
{विहारो वृष्णिवीराणां महे रैवतस्य ह}


\twolineshloka
{स नानावेश्मसंकीर्णो देवलोक इवाबभौ}
{तदा च कृष्णसान्निध्यान्मुदा देवगणैर्युतः}


\twolineshloka
{`स्तुवन्त्यन्तर्हिता देवा गन्धर्वाश्च सहर्षिभिः}
{साधकः सर्वधर्माणामसुराणां विनाशकः}


\twolineshloka
{त्वं स्रष्टा सृज्यमाधारं कारणं धर्मवेदवित्}
{त्वया सत्क्रियते देव ज जानीमोऽत्र मायया}


\twolineshloka
{केवलं त्वाऽभिजानीमः शरणं परमेश्वरम्}
{ब्रह्मादीनां च गोविन्द सान्निध्वं शरणं नमः}


\twolineshloka
{इति स्तुते मानुषैश्च पूजिते देवकीसुते}
{'शक्रसद्मप्रतीकाशो बभूव स हि शैलराट्}


\twolineshloka
{ततः सम्पूज्यमानः स विवेश भवनं शुभम्}
{गोविन्दः सात्यकिश्चैव जग्मतुर्भवनं स्वकम्}


\twolineshloka
{विवेश च प्रहृष्टात्मा चिरकालप्रवासतः}
{कृत्वा नसुकरं कर्म दानवेष्विव वासवः}


\twolineshloka
{उपायान्तं तु वार्ष्णेयं भोजवृष्ण्यन्धकास्तथा}
{अभ्यगच्चन्महात्मानं देवा इव शतक्रतुम्}


\twolineshloka
{स तानभ्यर्च्य मेधावी पृष्ट्वा च कुशलं तदा}
{अभ्यवादयत प्रीतः पितरं मातरं तदा}


\twolineshloka
{ताभ्यां स सम्परिष्वक्तः सान्त्वितश्च महाभुजः}
{उपोपविष्टैः सर्वैस्तैर्वृष्णिभिः परिवारितः}


\twolineshloka
{स विश्रान्तो महातेजाः कृतपादावनेजनः}
{कथयामास तत्सर्वं पृष्टः पित्रा महाहवम्}


\chapter{अध्यायः ६०}
\twolineshloka
{श्रुतवानस्मि वार्ष्णेय सङ्ग्रामं परमाद्भुतम्}
{नराणां वदतां पुत्र कथोद्धातेषु नित्यशः}


\twolineshloka
{त्वं तु प्रत्यक्षदर्शी च कार्यज्ञश्च महाभुजः}
{तस्मात्प्रब्रूहि सङ्ग्रामं याथातथ्येन मेऽनघ}


\twolineshloka
{यथा तदभवद्युद्धं पाण्डवानां महात्मनाम्}
{भीष्मकर्णकृपद्रोणशल्यादिभिरनुत्तमम्}


\threelineshloka
{अन्येषां क्षत्रियाणां च कृतास्त्राणामनेकशः}
{नानावेषाकृतिमतां नानादेशनिवासिनाम् ॥वैशम्पायन उवाच}
{}


\threelineshloka
{इत्युक्तः पुण्डरीकाक्षः पित्रा मातुस्तदाऽन्तिके}
{शशंस कुरुवीराणां सङ्ग्रामे निधनं यथा ॥वासुदेव उवाच}
{}


\twolineshloka
{अत्यद्भुतानि कर्माणि क्षत्रियाणां महात्मनाम्}
{बहुलत्वान्न सङ्ख्यातुं शक्यान्यब्दशतैरपि}


\twolineshloka
{प्राधान्यतस्तु गदतः समासेनैव मे शृणु}
{कर्माणि पृथिवीशानां यथावदमरद्युते}


\twolineshloka
{भीष्मः सेनापतिरभूदेकादशचमूपतिः}
{कौरव्यः कौरवेन्द्राणां देवानामिव पावकिः}


\twolineshloka
{शिखण्डी पाण्डुपुत्राणां नेता सप्तचमूपतिः}
{बभूव रक्षितो धीमाञ्श्रीमता सव्यसाचिना}


\twolineshloka
{तेषां तदभवद्युद्धं दशाहानि महात्मनाम्}
{कुरूणां पाण्डवानां च सुमहद्रोमहर्षणम्}


\twolineshloka
{अयुध्यमानं गाङ्गेयं शिखण्डी तं महाद्युतिम्}
{जघान बहुभिर्बाणैः सह गाण्डीवधन्वना}


\twolineshloka
{अकरोत्स ततः कालं शरतल्पगतो मुनिः}
{अयनं दक्षिणं हित्वा सम्प्राप्ते चोत्तरायणे}


\twolineshloka
{ततः सेनापतिरभूद्द्रोणोऽस्त्रविदुषांवरः}
{प्रवीरः कौरवेन्द्रस्य काव्यो दैत्यपतेरिव}


\twolineshloka
{अक्षौहिणीभिः शिष्टाभिर्नवभिर्द्विजसत्तमः}
{संवृतः समरश्लाघी गुप्तः कृपसुतादिभिः}


\twolineshloka
{धृष्टद्युम्नस्त्वभून्नेता पाण्डवानां महास्त्रवित्}
{गुप्तो भीमेन मेधावी मित्रेण वरुणो यथा}


\twolineshloka
{स च सेनापरिवृतो द्रोणप्रेप्सुर्महामनाः}
{पितुर्निकारान्संस्मृत्य रणे कर्माकरोन्महत्}


\twolineshloka
{तस्मिंस्ते पृथिवीपाला द्रोणपार्षतसङ्गरे}
{नानादिगागता वीराः प्रायशो निधनं गताः}


\twolineshloka
{दिनानि पञ्च तद्युद्धमभूत्परमदारुणम्}
{ततो द्रोणः परिश्रान्तो धृष्टद्युम्नवशं गतः}


\twolineshloka
{ततः सेनापतिरभूत्कर्णो दौर्योधने बले}
{अक्षौहिणीभिः शिष्टाभिर्वृतः पञ्चभिराहवे}


\twolineshloka
{तिस्रस्तु पाण्डुपुत्राणां चम्वो बीभत्सुपालिताः}
{हतप्रवीरभूयिष्ठा बभूवुः समवस्थिताः}


\twolineshloka
{ततः पार्थं समासाद्य पतङ्ग इव पावकम्}
{पञ्चत्वमगमत्सौतिर्द्वितीयेऽहनि दारुणः}


\twolineshloka
{हते कर्णे तु कौरव्या निरुत्साहा हतौजसः}
{अक्षौहिणीभिस्तिसृभिर्मद्रेशं पर्यवारयन्}


\twolineshloka
{हतिवाहनभूयिष्ठाः पाण्डिवास्तु युधिष्ठिरम्}
{अक्षौहिण्या निरुत्साहाः शिष्टया पर्यवारयन्}


\twolineshloka
{अवधीन्मद्रराजानं कुरुराजो युधिष्ठिरः}
{तस्मिंस्तदाऽर्धदिवसे कृत्वा कर्म सुदुष्करम्}


\twolineshloka
{हते शल्ये तु शकुनिं सहदेवो महामनाः}
{आहर्तारं कलेस्तस्य जगानामितविक्रमः}


\twolineshloka
{निहते शकुनौ राजा धार्तराष्ट्रः सुदुर्मनाः}
{अपाक्रामद्गदापाणिर्हतभूयिष्ठसैनिकः}


\twolineshloka
{तमन्वधावत्संक्रुद्धो भीमसेनः प्रतापवान्}
{ह्रदे द्वैपायने चापि सलिलस्थं ददर्श तम्}


\twolineshloka
{इतशिष्टेन सैन्येन समन्तात्पर्यवार्य तम्}
{अथोपविविशुर्हृष्टा ह्रदस्थं पञ्च पाण्डवाः}


\twolineshloka
{विगाह्य सलिलं त्वाशु वाग्बाणैर्भृशविक्षतः}
{उत्थाय स गदापाणिर्युद्धाय समुपस्थितः}


\twolineshloka
{ततः स निहतो राजा धार्तराष्ट्रो महारणे}
{भीमसेनेन विक्रम्य पश्यतां पृथिवीक्षिताम्}


\twolineshloka
{ततस्तत्पाण्डवं सैन्यं प्रसुप्तं शिबिरे निशि}
{निहतं द्रोणपुत्रेण पितुर्वधममृष्यता}


\twolineshloka
{हतपुत्रा हतबला हतमित्रा मया सह}
{युयुधानसहायेन पञ्च शिष्टास्तु पाण्डवाः}


\twolineshloka
{सहैव कृपभोजाभ्यां द्रौणिर्युद्धादमुच्यत}
{युयुत्सुश्चापि कौरव्यो मुक्तः पाण्डवसंश्रयात्}


\twolineshloka
{निहते कौरवेन्द्रे तु सानुबन्धे सुयोधने}
{विदुरः संजयश्चैव धर्मराजमुपस्थितौ}


\threelineshloka
{एवं तदभवद्युद्दमहान्यष्टादश प्रभो}
{यत्र ते पृथिवीपाला निहताः स्वर्गमावसन् ॥वैशम्पायन उवाच}
{}


\twolineshloka
{शृण्वतां तु महाराज कथां तां रोमहर्षणीम्}
{दुःखशोकपरिक्लेशा वृष्णीनामभवंस्तदा}


\chapter{अध्यायः ६१}
\twolineshloka
{कथयन्नेव तु तदा वासुदेवः प्रतापवान्}
{महाभारतयुद्धं तत्कथान्ते पितुरग्रतः}


\twolineshloka
{अभिमन्योर्वधं वीरः सोत्यक्रामन्महामतिः}
{अप्रियं वसुदेवस्य माभूदिति महामनाः}


\twolineshloka
{मा दौहित्रवधं श्रुत्वा वसुदेवो महात्ययम्}
{दुःखशोकाभिसंतप्तो भवेदिति महामतिः}


\twolineshloka
{सुभद्रा तु तमुत्क्रान्तमात्मजस्य वधं रणे}
{आचक्ष्व कृष्ण सौभद्रवधमित्यपतद्भुवि}


\twolineshloka
{तामपश्यन्निपतितां वसुदेवः क्षितौ तदा}
{दृष्ट्वैव च पपातोर्व्यां सोऽपि दुःखेन मूर्छितः}


\twolineshloka
{तषः स दौहित्रवदाद्दुःखशोकसमाहतः}
{वसुदेवो महाराज कृष्णं वाक्यमथाब्रवीत्}


\twolineshloka
{ननु त्वं पुण्डरीकाक्ष सत्यवाग्भुवि विश्रुतः}
{यद्दौहित्रवधं मेऽद्य न ख्यापयसि शत्रुहन्}


\threelineshloka
{तद्भागिनेयनिधनं तत्त्वेनाचक्ष्व मे प्रभो}
{सदृशाक्षस्तव कथं शत्रुभिर्निहतो रणे}
{}


\twolineshloka
{दुर्भरं बत वार्ष्णेय कालेऽप्राप्ते नृभिः सह}
{यत्र मे हृदयं दुःखाच्छतधा न विदीर्यते}


\twolineshloka
{किमब्रवीत्त्वां सङ्ग्रामे सुभद्रां मातरं प्रति}
{मां चापि पुण्डरीकाक्षि चपलाक्षः प्रियो मम}


\twolineshloka
{आहवं पृष्ठतः कृत्वा कच्चिन्न निहतः परैः}
{कच्चिन्मुखं न गोविन्द तेनाजौ विकृतं कृतम्}


\threelineshloka
{स हि कृष्ण महातेजाः श्लाघन्निव ममाग्रतः}
{बालभावेन विजयमात्मनोऽकथयत्प्रभुः}
{}


\twolineshloka
{कच्चिन्न निकृतो बालो द्रोणकर्णाकृपादिभिः}
{धरण्यां निहतः शेते तन्ममाचक्ष्वि केशव}


\twolineshloka
{स हि द्रोणं च भीष्मं च कर्णं च बलीनां वरम्}
{स्पर्धते स्म रणे नित्यं दुहितुः पुत्रको मम}


\twolineshloka
{एवंविधं बहु तदा विलपन्तं सुदुःखितम्}
{पितरं दुःखिततरं गोविन्दो वाक्यमब्रवीत्}


\twolineshloka
{न तेनि विकृतं वक्त्रं कृतं सङ्ग्राममूर्धनि}
{न पृष्ठतः कृतश्चापि सङ्ग्रामस्तेन दुस्तरः}


\twolineshloka
{निहत्य पृथिवीपालान्सहस्रशतसङ्घशः}
{खेदितो द्रोणकर्णाभ्यां दौःशासनिवशं गतः}


\twolineshloka
{एको ह्येकेन सततं युध्यमानो यदि प्रभो}
{न स शक्येत सङ्ग्रामे निहन्तुमपि वज्रिणा}


\twolineshloka
{समाहूते च सङ्ग्रामे पार्थे संशप्तकैस्तदा}
{पर्यवार्यत संक्रुद्धैः स द्रोणादिभिराहवे}


\twolineshloka
{ततः शत्रुवधं कृत्वा सुमहान्तं रेणे पितः}
{दौहित्रस्तव वार्ष्णेय दौःशासनिवशं गतः}


\twolineshloka
{नूनं च स गतः स्वर्गं जहि शोकं महामते}
{न हि व्यसनमासाद्य सीदन्ति कृतबुद्धयः}


\twolineshloka
{द्रोणकर्णप्रभृतयो येन प्रतिसमासिताः}
{रणे महेन्द्रप्रतिमाः स कथं नाप्नुयाद्दिवम्}


\twolineshloka
{स शोकं जहि जुर्धर्ष मा च मन्युवशं गमः}
{शस्त्रपूतां हि स गतिं गतः परपुरंजयः}


\twolineshloka
{तस्मिंस्तु निहते वीरे सुभद्रेयं स्वसा मम}
{दुःखार्ताऽथो सुतं प्राप्य कुररीव ननाद ह}


\twolineshloka
{द्रौपदीं च समासाद्य पर्यतप्यत दुःखिता}
{आर्ये क्व दारकाः सर्वे द्रष्टुमिच्छामि तानहम्}


\twolineshloka
{अस्यास्तु वचनं श्रुत्वा सर्वास्ताः कुरुयोषितः}
{भुजाभ्यां परिगृह्यैनां चुक्रुशुः परमार्तवत्}


\twolineshloka
{उत्तरां चाब्रवीद्भद्रे भर्ता स क्व नु ते गतः}
{क्षिप्रमागमनं मह्यं तस्य त्वं वेदयस्व ह}


\twolineshloka
{ननु नामाद्य वैराटि श्रुत्वा मम गिरं सदा}
{भवनान्निष्पतत्याशु कस्मान्नाभ्येति ते पतिः}


\twolineshloka
{अभिमन्योऽनुशयिनो मातुलास्ते महारथाः}
{कुशलं चाब्रुवन्सर्वे त्वां युयुत्सुमिहागतम्}


\threelineshloka
{आचक्ष्व मेऽद्य सङ्ग्रामं यथापूर्वमरिन्दम}
{कस्मादेवं विलपतीं नाद्येह प्रतिभाषसे}
{}


\twolineshloka
{एवमादि तु वार्ष्णेय्यास्तस्यास्तत्परिदेवितम्}
{श्रुत्वा पृथा सुदुःखार्ता शनैर्वाक्यमथाब्रवीत्}


\twolineshloka
{सुभद्रे वासुदेवेन तथा सात्यकिना रणे}
{पित्रा च लालितो बालः स हतः कालधर्मणा}


\twolineshloka
{ईदृशो मर्त्यधर्मोऽयं मा शुचो यदुनन्दिनि}
{पुत्रो हि तव दुर्धर्षः सम्प्राप्तः परमां गतिम्}


\twolineshloka
{कुले महति जातासि क्षत्रियाणां महात्मनाम्}
{मा शुचश्चपलाक्षं त्वं पद्मपत्रनिभेक्षणे}


\twolineshloka
{उत्तरां त्वमवेक्षस्व गुर्विणीं मा शुचः शुभे}
{पुत्रमेषा हि तस्याशु जनयिष्यति भामिनी}


\twolineshloka
{एवमाश्वासयित्वैनां कुन्ती यदुकुलोद्वह}
{विहाय शोकं दुर्धर्षं श्राद्धमस्य ह्यकल्पयत्}


\twolineshloka
{समनुज्ञाप्य धर्मज्ञं राजानं भीममेव च}
{यमौ यमोपमौ चैव ददौ दानान्यनेकशः}


\twolineshloka
{ततः प्रदाय बह्वीर्गा ब्राह्मणेभ्यो यदूद्वह}
{समाहृष्य तु वार्ष्णेयी वैराटीमब्रवीदिदम्}


\twolineshloka
{वैराटि नेह संतापस्त्वया कार्यो ह्यनिन्दिते}
{भर्तारं प्रति सुश्रोणि गर्भस्थं रक्ष वै शिशुम्}


\twolineshloka
{एवमुक्त्वा ततः कुन्ती विरराम महाद्युते}
{तामनुज्ञाप्य चैवेमां सुभद्रां समुपानयम्}


\twolineshloka
{एवं स निधनं प्राप्तो दौहित्रस्तव मानद}
{संतापं त्यज दुर्धर्ष मा च शोके मनः कृथाः}


\chapter{अध्यायः ६२}
\twolineshloka
{एतच्छ्रुत्वा तु पुत्रस्य वचः शूरात्मजस्तदा}
{विहाय शोकं धर्मात्मा ददौ श्राद्धमनुत्तमम्}


\twolineshloka
{तथैव वासुदेवश्च स्वस्त्रीयस्य महात्मनः}
{दयितस्यि पितुर्नित्यमकरोदौर्ध्वदेहिकम्}


\twolineshloka
{षष्टिं शतसहस्राणि ब्राह्मणानां महौजसाम्}
{विधिवद्भोजयामास भोज्यं सर्वगुणान्वितम्}


\twolineshloka
{आच्छाद्य च महाबाहुर्धनतृष्णामपानुदत्}
{ब्राह्मणानां तदा कृष्णस्तदभूद्रोमहर्षणम्}


\twolineshloka
{सुवर्णं चैव गाश्चैव शयनाच्छादनानि च}
{दीयमानं तदा विप्रः प्रभूतमिति चाब्रुवन्}


\threelineshloka
{वासुदेवोऽथ दाशार्हो बलेदेवः ससात्यकिः}
{अभिमन्योस्तदा श्राद्धमकुर्वन्सत्यकस्तदा}
{अतीव दुःखसंतप्ता न शमं चोपलेभिरे}


\twolineshloka
{तथैव पाण्डवा वीरा नगरे नागसह्वये}
{नोपागच्छन्त वै शान्तिमभिमन्युविनाकृताः}


\twolineshloka
{सुबहूनि च राजेन्द्र दिवसानि विराटजा}
{नाभुङ्क्त पतिदुःखार्ता तदभूत्करुणं महत्}


\twolineshloka
{धियमाणे तु तस्मिंस्तु गर्भे कुक्षिस्थ एव च}
{आजगाम ततो व्यासो ज्ञात्वा दिव्येन चक्षुषा}


\twolineshloka
{समागम्याब्रवीमान्पृथां पृथुललोचनाम्}
{उत्तरां च महातेजाःइ शोकः संत्यज्यतामयम्}


\threelineshloka
{जनिष्यते महातेजाः पुत्रस्तव यशस्विनि}
{प्रभावाद्वासुदेवस्य मम व्याहरणादपि}
{पाण्डवानामयं चान्ते पालयिष्यति मेदिनीम्}


\twolineshloka
{धनञ्जयं च सम्प्रेक्ष्य धर्मराजस्य शृण्वतः}
{व्यासो वाक्यमुवाचेदं हर्षयन्निव भारत}


\twolineshloka
{पौत्रस्तव महाभागो जनिष्यति महामनाः}
{पृथ्वीं सागरपर्यन्तां पालयिष्यति धर्मतः}


\twolineshloka
{तस्माच्छोकं कुरुश्रेष्ठ जहि त्वमरिकर्शन}
{विचार्यमत्र न हि ते सत्यमेतद्भविष्यति}


\twolineshloka
{यच्चापि वृष्णिवीरेणि कृष्णेन कुरुनन्दन}
{पुरोक्तं तत्तथा भावि मा तेऽत्रास्तु विचारणा}


\twolineshloka
{विबुधानां गतो लोकानक्षयानात्मनिर्जितान्}
{न स शोच्यस्त्वया वीरो न चान्यैः कुरुभिस्तथा}


\twolineshloka
{एवं पितामहेनोक्तो धर्मात्मा स धनञ्जयः}
{त्यक्त्वा शोकं महाराज हृष्टरूपोऽभवत्तदा}


\twolineshloka
{पिताऽपि तव धर्मेज्ञ गर्भे तस्मिन्महामते}
{अवर्धत यथाकामं शुक्लपक्षे यथा शसी}


\twolineshloka
{ततः संचोदयामास व्यासो धर्मात्मजं नृपम्}
{अश्वमेधं प्रति तदा ततः सोऽन्तर्हितोऽभवत्}


\twolineshloka
{धर्मराजोपि मेधावी श्रुत्वा व्यासस्य तद्वचः}
{वित्तोपनयने तात चकार गमने मतिम्}


\chapter{अध्यायः ६३}
\twolineshloka
{श्रुत्वैतद्वचनं ब्रह्मन्व्यासेनोक्तं महात्मना}
{अश्वमेधं प्रति तदा किं भूयः प्रचकार ह}


\threelineshloka
{रत्नं च यन्मरुत्तेनि निहितं वसुधातले}
{तदवाप कथं चेति तन्मे ब्रूहि द्विजोत्तम ॥वैशम्पायन उवाच}
{}


\twolineshloka
{श्रुत्वा द्वैपायनवचो धर्मिराजो युधिष्ठिरः}
{भ्रातॄन्सर्वान्समानाय्य काले वचनमब्रवीत्}


\twolineshloka
{अर्जुनं भीमसेनं च माद्रीपुत्रौ यमावपि}
{श्रुतं वो वचनं वीराः सौहृदाद्यन्महात्मना}


\twolineshloka
{कुरूणां हितकामेन प्रोक्तं कृष्णेन धीमता}
{तपोवृद्धेनि महता सुहृदां भूतिमिच्छता}


\twolineshloka
{गुरुणा धर्मशीलेन व्यासेनाद्भुतकर्मणा}
{भीष्मेण च महाप्राज्ञ गोविन्देनि च धीमता}


\twolineshloka
{संस्मृत्य तदहं सम्यक्कर्तुमिच्छामि पाण्डवाः}
{आयत्यां च तदात्वे च सर्वेषां तद्धि नो हितम्}


\twolineshloka
{अनुबन्धे च कल्याणं यद्वचो ब्रह्मवादिनः}
{इयं हि वसुधा सर्वा क्षीणरत्ना कुरूद्वहाः}


\threelineshloka
{तच्चाचष्ट तदा व्यासो मरुत्तस्य धनं नृपाः}
{यद्येतद्वो बहुमतं मन्यध्वं वा क्षमं यदि}
{तदानयामहे सर्वे कतं वा भीम मन्यसे}


\twolineshloka
{इत्युक्तवाक्ये नृपतौ तदा कुरुकुलोद्वह}
{भीमसेनो नृपश्रेष्ठं प्राञ्जलिर्वाक्यमब्रवीत्}


\twolineshloka
{रोचते मे महाबाहो यदिदं भाषितं त्वया}
{व्यासाख्यातस्य वित्तस्य समुपानयनं प्रति}


\twolineshloka
{तत्प्राप्नुयामहे धर्माद्यद्धनं काङ्क्षितं प्रभो}
{कृतमेव महाराज भवेदिति मतिर्मम}


\twolineshloka
{ते वयं प्रणिपातेन गिरीशस्य महात्मनः}
{तदानयामि भद्रं ते समभ्यर्च्य कपर्दिनम्}


\twolineshloka
{`तं विभुं देवदेवेशं शूलपाणिं त्रिलोचनम्}
{अनादिनिधनं शंभुं नमस्यामि महेश्वरम् ॥'}


\twolineshloka
{लोकनाथं गणाध्यक्षं तस्यैवानुचरांश्चि तान्}
{प्रसाद्यार्थमवाप्स्यामो नूनं वाग्बुद्धिकर्मभिः}


\twolineshloka
{रक्षन्ते ये च तद्द्रव्यं किन्नरा रौद्रदर्शनाः}
{ते च वश्या भविष्यन्ति प्रसन्ने वृषभध्वजे}


\twolineshloka
{`स हि देवः प्रसन्नात्मा भक्तानां परमेश्वरः}
{ददात्यमरतां चापि किं पुनः काञ्चनं प्रभुः}


\twolineshloka
{वनस्थास्य पुरा जिष्णोरस्त्रं पाशुपतं महत्}
{रौद्रं ब्रह्मसिरश्चादात्प्रसन्नः किं पुनर्धनम्}


\twolineshloka
{वयं सर्वे हि तद्भक्ताः स चास्माकं प्रसीदति}
{तत्प्रसादादिदं राज्यं प्राप्तं कौरवनन्दन}


\threelineshloka
{अभिमन्योर्वधे वृत्ते प्रतिज्ञाते धनञ्जये}
{जयद्रथवधार्थाय स्वप्ने लोकगुरुर्निशि}
{प्रसाद्य लब्धवानस्त्रमर्जुनः सहकेशवः}


\twolineshloka
{तत्र प्रभातां रजनीं फल्गुनस्याग्रतः प्रभुः}
{जघान सैन्यं शूलेन प्रत्यक्षं सव्यसाचिनः}


\threelineshloka
{कस्तां सेनां महाराज मनसाऽपि प्रधर्षयेत्}
{द्रोणिकर्णबलैर्युक्तां महेष्वासैः प्रहारिभिः}
{ऋते देवान्महेष्वासाद्बहुरूपान्महेश्वरात्}


\threelineshloka
{तस्यैव च प्रसादेव निहतास्तव शत्रवः}
{अश्वमेधस्य संसिद्धिं तव सम्पादयिष्यति}
{'}


\threelineshloka
{श्रुत्वैवं वदतस्तस्य वाक्यं भीमस्य भारत}
{प्रीतो धर्मात्मजो राजा बभूवातीव भारत}
{अर्जुनप्रमुखाश्चापि तथेत्येवाब्रुवन्वचः}


\twolineshloka
{कृत्वा तु पाण्डवाः सर्वे रत्नाहरणनिश्चयम्}
{सेनामाज्ञापयामासुर्नक्षत्रेऽहनि च ध्रुवे}


\twolineshloka
{ततो ययुः पाण्डुसुता ब्राह्मणान्स्वस्ति वाच्य च}
{अर्चयित्वा सुरश्रेष्ठं पूर्वमेव महेश्वरम्}


\twolineshloka
{मोदकैः पायसेनाथ मांसापूपैस्तथैव च}
{आशास्य च महात्मानं प्रययुर्मुदिता भृशम्}


\twolineshloka
{तेषां प्रयास्यतां तत्र मङ्गलानि शुभान्यथ}
{प्राहुः प्रहृष्टमनसो द्विजाग्र्या नागराश्च ते}


\twolineshloka
{ततः प्रदक्षिणीकृत्य शिरोभिः प्रणिपत्य च}
{ब्राह्मणानग्निसहितान्प्रययुः पाण्डुनन्दनाः}


\twolineshloka
{समनुज्ञाप्य राजानं पुत्रशोकसमाहतम्}
{धृतराष्ट्रं सभार्यं वै पृथां च पृथुलोचनाम्}


\threelineshloka
{मूले निक्षिप्य कौरव्यं युयुत्सुं धृतराष्ट्रजम्}
{सम्पूज्यमानाः पौरैश्च ब्राह्मणैश्च मनीषिभिः}
{`प्रययुः पाण्डवा वीरा नियमस्थाः शुचिव्रताः'}


\chapter{अध्यायः ६४}
\twolineshloka
{ततस्ते प्रययुर्हृष्टाः प्रहृष्टनरवाहनाः}
{रथघोषेण महता पूरयन्तो वसुन्धराम्}


\twolineshloka
{संस्तूयमानाः स्तुतिभिः सूतमागधबन्दिभिः}
{स्वेन सैन्येन संवीता यथाऽऽदित्याः स्वरश्मिभिः}


\twolineshloka
{पाण्डुरेणातपत्रेण ध्रियमाणेन मूर्धनि}
{बभौ युधिष्टिरस्तत्र पौर्णमास्यामिवोडुराट्}


\twolineshloka
{जयाशिषः प्रहृष्टानां नराणां पथि पाण्डवः}
{प्रत्यगृह्णाद्यथान्यायं यथावत्पुरुषर्षभः}


\twolineshloka
{तथैव सैनिका राजन्राजानमनुयान्ति ये}
{तेषां हलहलाशब्दो दिवं स्तब्ध्वा व्यतिष्ठत}


\twolineshloka
{सरांसि सरितश्चैव वनान्युपवनानि च}
{अत्यक्रामन्महाराजो गिरिं चाप्यन्वपद्यत}


\threelineshloka
{स्मिन्देशे च राजेन्द्र यत्र तद्द्रव्यमुत्तमम्}
{चक्रे निवेशनं राजा पाण्डवः सहसैनिकैः}
{शिवे देशे समे चैव तदा भरतसत्तम}


\twolineshloka
{अग्रतो ब्राह्मणान्कृत्वा तपोविद्यादमान्वितान्}
{पुरोहितं च कौरव्य वेदवेदाङ्गपारगम्}


\twolineshloka
{आग्निवेश्यं च राजानो ब्राह्मणाः सपुरोधसः}
{कृत्वा शान्तिं यथान्यायं सर्वशः पर्यवारयन्}


\twolineshloka
{कृत्वा तु मध्ये राजानममात्यांश्च यथाविधि}
{षट्पदं नवसङ्ख्यानं निवेशं चक्रिरे जनाः}


\twolineshloka
{मत्तानां वारणेन्द्राणां निवेशं च यथाविधि}
{कारयित्वा स राजेन्द्रो ब्राह्मणानिदमब्रवीत्}


\twolineshloka
{अस्मिन्कार्ये द्विजश्रेष्ठा नक्षत्रे दिवसे शुभे}
{यथा भवन्तो मन्यन्ते कर्तुमर्हन्ति तत्तथा}


\twolineshloka
{न नः कालात्ययो वै स्यादिहैव परिलम्बताम्}
{इति निश्चित्य विप्रेन्द्राः क्रियतां यदनन्तरम्}


\twolineshloka
{श्रुत्वैतद्व********* ब्राह्मणाः सपुरोधसः}
{इदमूचुर्वचो हृष्टा धर्मराजप्रियेप्सवः}


\twolineshloka
{अद्यैव नक्षत्रिमहश्च पुण्यंयतामहे श्रेष्ठतमक्रियासु}
{तपोभिरद्येह वसाम राज-न्नुपोष्यतां चापि भवद्भिरद्य}


\twolineshloka
{श्रुत्वा तु तेषां द्विजसत्तमानांकृतोपवासा रजनीं नरेन्द्राः}
{}


% Check verse!
ऊषुः प्रतीताः कुशसंस्तरेषुयथाऽध्वरे प्रज्वलिता हुताशाः
\twolineshloka
{ततो निशा सा व्यगमन्महात्मनांसंशृण्वतां विप्रसमीरिता गिरः}
{ततः प्रभाते विमले द्विजर्षभावचोऽब्रुवन्धर्मसुतं नराधिपम्}


\chapter{अध्यायः ६५}
\threelineshloka
{क्रियतामुपहारोऽद्य त्र्यम्बकस्य महात्मनः}
{दत्त्वोपहारं नृपते ततः स्वार्थं यतामहे ॥वैशम्पायन उवाच}
{}


\twolineshloka
{श्रुत्वा तु वचनं तेषां ब्राह्मणानां युधिष्ठिरः}
{गिरीशस्य यथान्यायमुपहारमुपाहरत्}


\twolineshloka
{आज्येन तर्पयित्वाऽग्निं विधिवत्संस्कृतेन च}
{मन्त्रसिद्धं चरुं कृत्वा पुरोधाः स ययै तदा}


\twolineshloka
{स गृहीत्वा सुमनसो मन्त्रपूता जनाधिप}
{मोदकैः पायसेनाथ मांसैश्चोपाहरद्बलिंम्}


\threelineshloka
{सुमनोभिश्च चित्राभिर्लाजैरुच्चावचैरपि}
{सर्वं स्विष्टकृतं हुत्वा विधिवद्वेदपारगः}
{किंकराणां ततः पश्चाच्चकार बलिमुत्तमम्}


\twolineshloka
{यक्षेन्द्राय कुबेराय माणिभद्राय चैव ह}
{तथाऽन्येषां च यक्षाणां भूतानां पतयश्च ये}


\twolineshloka
{कृसरेण च मांसेन निवापैस्तिलसंयुतैः}
{ओदनं कुम्भशः कृत्वा पुरोधाः समुपाहरत्}


\twolineshloka
{ब्राह्मणेभ्यः सहस्राणि गवां दत्वा तु भूमिपः}
{नक्तंचराणां भूतानां व्यादिदेश बलि तदा}


\twolineshloka
{धूपगन्धनिरुद्धं तत्सुमनोभिश्च संवृतम्}
{शुशुभे स्थानमत्यर्थं देवदेवस्य पार्थिवः}


\twolineshloka
{कृत्वा पूजां तु रुद्रस्य गणानां चैव सर्वशः}
{ययौ व्यासं पुरस्कृत्य नृपो रत्ननिधिं प्रति}


\twolineshloka
{पूजयित्वा धनाध्वक्षं प्रणिपत्याभिवाद्य च}
{सुमनोभिर्विचित्राभिरपूपैः कृसरेण च}


\twolineshloka
{शङ्खादींश्च निधीन्सर्वान्निधिपालांश्च सर्वशः}
{अर्चयित्वा द्विजाग्र्या स्वस्ति वाच्य च वीर्यवान्}


\twolineshloka
{तेषां पुण्याहघोषेण तेजसा समवस्थितः}
{प्रीतिमान्स कुरुश्रेष्ठः खानयामासं तं निधिम्}


\twolineshloka
{ततः पात्री सकरका बहुरूपा मनोरमाः}
{भृङ्गाराणि कटाहानि कलशान्वर्धमानकान्}


\twolineshloka
{बहूनि च विचित्राणि भाजनानि सहस्रशः}
{उद्धारयामास तदा धर्मराजो युधिष्ठिरः}


\twolineshloka
{तेषां रक्षणमप्यासीन्महान्करपुटस्तथा}
{नद्धं च भाजनं राजंस्तुलार्धमभवन्नृप}


\twolineshloka
{वाहनं पाण्डुपुत्रस्य तत्रासीत्तु विशांपते}
{षष्टिरुष्ट्रसहस्राणि शतानि द्विगुणा हयाः}


\threelineshloka
{वारणाश्च महाराज सहस्रशतसंमिताः}
{शकटानि रथाश्चैव तावदेव करेणवः}
{स्वराणां पुरुषाणां च परिसङ्ख्या न विद्यते}


\twolineshloka
{एतद्वित्तं तदभवद्यदुद्दध्रे युधिष्ठिरः}
{षोडशाष्टौ चतुर्विंशत्सहस्रं भारलक्षणम्}


\twolineshloka
{एतेष्वादाय तद्द्रव्यं पुनरभ्यर्च्य पाण्डवः}
{महादेवं प्रति ययौ पुरं नागाहयं प्रति}


\twolineshloka
{द्वैपायनाभ्यनुज्ञातः पुरस्कृत्य पुरोहितम्}
{गोरुते गोरुते चैव न्यवसत्पुरुषर्षभः}


\twolineshloka
{सा पुराऽभिमुखा राजन्नुवाह महती चमूः}
{कृच्छ्राद्द्रविणभारार्ता हर्षयन्ती कुरूद्वहान्}


\chapter{अध्यायः ६६}
\twolineshloka
{एतस्मिन्नेव काले तु वासुदेवोऽपि वीर्यवान्}
{उपायाद्वृष्णिभिः सार्दं पुरं वारणसाह्वयम्}


\twolineshloka
{समयं वाजिमेधस्य विदित्वा पुरुषर्षभः}
{यथोक्तो धर्मपुत्रेण प्रव्रजन्स्वपुरीं प्रति}


\twolineshloka
{रौक्मिणेयेन सहितो युयुधानेन चैव ह}
{चारुदेष्णेन सांबेनि गदेन कृतवर्मणा}


\twolineshloka
{सारणेन च वीरेण निशठेनोन्मुखेन च}
{बलदेवं पुरस्कृत्य सुभद्रासहितस्तदा}


\twolineshloka
{द्रौपदीमुत्तरां चैव पृथां चाप्यवलोककः}
{समाश्वासयितुं चापि क्षत्रिया निहतेश्वराः}


\twolineshloka
{तानागतान्समीक्ष्यैव धृतराष्ट्रो महीपतिः}
{प्रत्यगृह्णाद्यथान्यायं विदुरश्च महामनाः}


\twolineshloka
{तत्रैव न्यवसत्कृष्णः स्वर्चितः पुरुषोत्तमः}
{विदुरेणि महातेजास्तथैव च युयुत्सुना}


\twolineshloka
{वसत्सु वृष्णिवीरेषु तत्राथ जनमेजय}
{जज्ञे तव पिता राजन्परिक्षित्परवीरहा}


\twolineshloka
{स तु राजा महाराज ब्रह्मास्त्रेणावपीडितः}
{शवो बभूव निश्चेष्टो हर्षशोकविवर्धनः}


\twolineshloka
{हृष्टानां सिंहनादेन जनानां तत्र निःस्वनः}
{आविवश दिशःसर्वाः पुनरेवाभ्युपागमत्}


\twolineshloka
{ततः सोतित्वरः कृष्णो विवेशान्तःपुरं तदा}
{युयुधानद्वितीयो वै व्यथितेन्द्रियमानसः}


\twolineshloka
{ततस्त्वरितमायान्तीं ददर्शं स्वां पितृष्वसाम्}
{क्रोशन्तीमभिधावेति वासुदेवं पुनःपुनः}


\twolineshloka
{पृष्ठतो द्रौपदीं चैव सुभद्रां च यशस्विनीम्}
{विक्रोशन्त्यश्च करुणं पाण्डवानां स्त्रियो नृप}


\twolineshloka
{ततः कृष्णं समासाद्य कुन्ती भोजसुता तदा}
{प्रोवाच राजशार्दूल बाष्पगद्गदया गिरा}


\twolineshloka
{वासुदेव महाबाहो सुप्रजा देवकी त्वया}
{त्वं नो गतिः प्रतिष्ठा च त्वदायत्तमिदं कुलम्}


\twolineshloka
{यदुप्रवीर योऽयं ते स्वस्त्रीयस्यात्मजः प्रभो}
{अश्वत्थाम्ना हतो जातस्तमुज्जीवय केशव}


\twolineshloka
{त्वया ह्येतत्प्रतिज्ञातमैषीके यदुनन्दन}
{अहं संजीवयिष्यामि मृतं जातमिति प्रभो}


\twolineshloka
{सोयं जातो मृतस्तात पश्यैनं पुरुषर्षभ}
{उत्तरां च सुभद्रां च द्रौपदीं मां च माधव}


\twolineshloka
{धर्मपुत्रं च भीमं च फल्गुनं नकुलं तथा}
{सहदेवं च दुर्धर्षं सर्वान्नस्त्रातुमर्हसि}


\twolineshloka
{अस्मिन्प्राणाः समायत्ताः पाण्डवानां ममैव च}
{पाण्डोश्च पिण्डो दाशार्ह तथैव श्वशुरस्य मे}


\twolineshloka
{अभिमन्योश्च भद्रं ते प्रियस्य सदृशस्य च}
{प्रियमुत्पादयाद्य त्वं प्रेतस्यापि जनार्दन}


\twolineshloka
{उत्तरा हि पुरोक्तं वै कथयत्यरिसूदन}
{अभिमन्योर्वचः कृष्ण प्रियत्वात्तन्न संशयः}


\twolineshloka
{अब्रवीत्किल दाशार्ह वैराटीमार्जुनिस्तदा}
{मातुलस्य कुलं भद्रे तव पुत्रो गमिष्यति}


\twolineshloka
{गत्वा वृष्णयन्धककुलं धनुर्वेदं ग्रहीष्यति}
{अस्त्राणि च विचित्राणि नितीशास्त्रं च केवलं}


\twolineshloka
{इत्येतत्प्रणयात्तात सौभद्रः परवीरहा}
{कथयामास दुर्धर्षस्तथा चैतन्न संशयः}


\twolineshloka
{तास्त्वां वयं प्रणम्येह याचामो मधुसूदन}
{कुलस्यास्य हितार्तं च कुरु कल्याणमुत्तमम्}


\twolineshloka
{एवमुक्त्वा तु वार्ष्णेयं पृथा पृथुललोचना}
{उद्धृत्य बाहू दुःखार्ता ताश्चान्याः प्रापतन्भुवि}


\twolineshloka
{अब्रुवंश्च महाराज सर्वाः सास्राविलेक्षणाः}
{स्वस्त्रीयो वासुदेवस्य मृतो जात इति प्रभो}


\twolineshloka
{एवं गते ततः कुन्तीं पर्यगृह्णाज्जनार्दनः}
{भूमौ निपतितां चैनां सान्त्वयामास भारत}


\chapter{अध्यायः ६७}
\twolineshloka
{उत्थितायां पृथायां तु सुभद्रा भ्रातरं तदा}
{दृष्ट्वा चुक्रोश दुःखार्ता वचनं चेदमब्रवीत्}


\twolineshloka
{पुण्डरीकाक्ष पश्य त्वं पौत्रं पार्थस्य धीमतः}
{परिक्षीणेषु कुरुषु परिक्षीणां गतायुषम्}


\twolineshloka
{इषीका द्रोणपुत्रेणि भीमसेनार्थमुद्यता}
{सोत्तरायां निपतिता विजये मयि चैव ह}


\twolineshloka
{सेयं ज्वलन्ती हृदये मयि तिष्टति केशव}
{यन्न पश्यामि दुर्धर्ष सहपुत्रं तु तं प्रभो}


\twolineshloka
{किंनु वक्ष्यति धर्मात्मा धर्मराजो युधिष्ठिरः}
{भीमसेनार्जुनौ चापि माद्रवत्याः सुतौ च तौ}


\twolineshloka
{श्रुत्वाऽभिमन्योस्तनयं जातं च मृतमेव च}
{मुषिता इव वार्ष्णेय द्रोणपुत्रेण पाण्डवाः}


\twolineshloka
{अभिमन्युः प्रियः कृष्ण पितॄणां नात्र संशयः}
{ते श्रुत्वा किंनु वक्ष्यन्ति द्रोणपुत्रास्त्रनिर्जिताः}


\twolineshloka
{भविता नः परं दुःखं किंनु मन्ये जनार्दन}
{अभिमन्योः सुतं कृष्ण मृतं जातमरिंदम}


\twolineshloka
{साऽहं प्रसादये कृष्ण त्वामद्य शिरसा नता}
{पृथेयं द्रौपदी चैव ताः पश्य पुरुषोत्तम}


\twolineshloka
{यदा द्रोणसुतो गर्भान्पाण्डूनां हन्ति माधव}
{तदा किल त्वया द्रौणिः क्रुद्धेनोक्तोऽरिमर्दन}


\twolineshloka
{अकामं त्वां करिष्यामि ब्रह्मबन्धो नराधम}
{अहं संजीवयिष्यामि किरीटितनयात्मजम्}


\twolineshloka
{इत्येद्वचनं श्रुत्वा जानानाऽहं बलं तव}
{प्रसादये त्वां दुर्धर्ष जीवतामभिमन्युजः}


\twolineshloka
{यद्येतत्त्वं प्रतिश्रुत्य न करोषि वच शुभम्}
{सकलं वृष्णिशार्दूल मृतां मामवधारय}


\twolineshloka
{अभिमन्योः सुतो वीर न संजीवति यद्ययम्}
{जीवति त्वयि दुर्धर्ष किं करिष्याम्यहं त्वया}


\twolineshloka
{संजीवयैनं दुर्धर्ष मृतं त्वमभिमन्युजम्}
{सदृशाक्षसुतं वीर सस्यं वर्षन्निवांम्बुदः}


\twolineshloka
{त्वं हि केशव धर्मात्मा सत्यवान्सत्यविक्रमः}
{स तां वाचमृतां कर्तुमर्हसि त्वमरिंदम}


\twolineshloka
{इच्छन्नपि हि लोकांस्त्रीञ्जीवयेथा मृतानिमान्}
{किं पुनर्दयितं जातं स्वस्रीयस्यात्मजं मृतम्}


\twolineshloka
{प्रभावज्ञाऽस्मि ते कृष्ण तस्मात्त्वां याचयाम्यहम्}
{कुरुष्व पाण्डुपुत्राणामिमं परमनुग्रहम्}


\twolineshloka
{स्वसेति वा महाबाहो हतपुत्रेति वा पुनः}
{प्रपन्ना मामियं चेति दयां कर्तुमिहार्हसि}


\chapter{अध्यायः ६८}
\twolineshloka
{एवमुक्तस्तु राजेन्द्र केशिहा दुःखमुर्च्छितः}
{तथेति व्याजहारोच्चैर्ह्लादयन्निव तं जनम्}


\twolineshloka
{वाक्येनैतेन हि तदा तं जनं पुरुषर्षभः}
{ह्लादयामास स विभुर्घर्मार्तं सलिलैरिव}


\twolineshloka
{ततः स प्राविशत्तूर्णं जन्मवेश्म पितुस्तव}
{अर्चितं पुरुषव्याघ्र सितैर्माल्यैर्यथाविधि}


\twolineshloka
{अपां कुम्भैः सुपूर्णैश्च विन्यस्तैः सर्वतोदिशम्}
{घृतेन तिन्दुकालातैः सर्षपैश्च महाभुज}


\threelineshloka
{अस्त्रैश्च विमलैर्न्यस्तैः पावकैश्च समन्ततः}
{वृद्धाभिश्चापि रामाभिः परिवारार्थमावृतः}
{दक्षैश्च परितो धीर भिषग्भिः कुशलैस्तथा}


\twolineshloka
{ददर्श च स तेजस्वी रक्षोघ्रान्यपि सर्वशः}
{द्रव्याणि स्थापितानि स्म विधिवत्कुशलैर्जनैः}


\twolineshloka
{तथायुक्तं च तद्दृष्ट्वा जन्मवेश्म पितुस्तव}
{हृष्टोऽभवद्धृषीकेशः साधुसाध्विति चाब्रवीत्}


\twolineshloka
{तथा ब्रुवति वार्ष्णेये प्रहृष्टवदने तदा}
{द्रौपदी त्वरिता गत्वा वैराटीं वाक्यमब्रवीत्}


\twolineshloka
{अयमायाति ते भर्तुर्मातुलो मधुसूदनः}
{पुराणर्षिरचिन्त्यात्मा समीपमपराजितः}


\twolineshloka
{साऽपि बाष्पकलां वाचं निगृह्याश्रूपि चैव ह}
{असंवीताऽभवद्देवी देववत्कृष्णमीयुषी}


\twolineshloka
{सा तथा दूयमानेन हृदयेन तपस्विनी}
{दृष्ट्वा गोविन्दमायान्तं कृपणं पर्यदेवयत्}


\twolineshloka
{पुण्डरीकाक्ष पश्यावां बालेन हि विनाकृतौ}
{अभिमन्युं च मां चैव हठात्तुल्यं जनार्दन}


\twolineshloka
{वार्ष्णेयमधुहन्वीर शिरसा त्वां प्रसादये}
{द्रोणपुत्रास्त्रनिर्दग्धं जीवयैनं ममात्मजम्}


\twolineshloka
{यदि स्म धर्मराज्ञा वा भीमसेनेन वा पुनः}
{त्वया वा पुण्डरीकाक्ष वाक्यमुक्तमिदं भवेत्}


\twolineshloka
{अजानतीमिषीकेयं जनित्रीं हन्त्विति प्रभो}
{अहमेव विनष्टा स्यां नायमेवं गतो भवेत्}


\twolineshloka
{गर्भस्थस्यास्य बालस्य ब्रह्मास्त्रेण निपातनम्}
{कृत्वा नृशंसं दुर्बुद्धिर्द्रौणिः किं फलमश्नुते}


\twolineshloka
{सा त्वां प्रसाद्य शिरसा याचे शत्रुनिबर्हण}
{प्राणांस्त्यक्ष्यामि गोविन्द नायं संजीवते यदि}


\twolineshloka
{अस्मिन्हि बहवः साधो ये ममासन्मनोरथाः}
{ते द्रोणपुत्रेण हताः किंनु जीवामि केशव}


\twolineshloka
{आसीन्मम मतिः कृष्ण पूर्णोत्सङ्गा जनार्दन}
{अभिवादयिष्ये हृष्टेति तदिदं वितथीकृतम्}


\twolineshloka
{चपलाक्षस्य दायादे मृतेऽस्मिन्पुरुषर्षभ}
{विफला मे कृताः कृष्ण हृदि सर्वे मनोरथाः}


\twolineshloka
{चपलाक्षः किलातीव प्रियस्ते मधुसूदन}
{सुतं पश्य त्वमस्यैनं ब्रह्मास्त्रेण निपातितम्}


\twolineshloka
{कृतघ्नोऽयं नृशंसोऽयं यथाऽस्य जनकस्तथा}
{यः पाण्डवीं श्रियं त्यक्त्वा गतोऽद्य यमसादनं}


\twolineshloka
{मया चैतत्प्रतिज्ञातं रणमूर्धनि केशव}
{अभिमन्यौ हते वीर त्वामेष्याम्यचिरादिति}


\twolineshloka
{तच्च नाकरवं कृष्ण नृशंसा जीवितप्रिया}
{इदानीं मां गतां तत्र किंनु वक्ष्यति फाल्गुनिः}


\chapter{अध्यायः ६९}
\twolineshloka
{सैवं विलप्य करुणं सोन्मादेव तपस्विनी}
{उत्तरा न्यपतद्भूमौ कृपणा पुत्रगृद्धिनी}


\twolineshloka
{तां तु दृष्ट्वा निपतितां हतपुत्रपरिच्छदाम्}
{चुक्रोश कुन्ती दुःखार्ता सर्वाश्च भरतास्त्रियः}


\twolineshloka
{मुहूर्तमिव राजेन्द्र पाण्डवानां निवेशनम्}
{अप्रेक्षणीयमभवदार्तस्वनविनादितम्}


\twolineshloka
{सा मुहूर्तं च राजेन्द्र पुत्रशोकाभिपीडिता}
{कश्मलाभिहता वीर वैराटी त्वभवत्तदा}


\twolineshloka
{प्रतिलभ्य तु सा संज्ञामुत्तरा भरतर्षभ}
{अङ्कमारोप्य तं पुत्रमिदं वचनमब्रवीत्}


\twolineshloka
{धर्मज्ञस्य सुतः संस्त्वं न धर्ममवबुध्यसे}
{यस्त्वं वृष्णिप्रवीरस्य कुरुषे नाभिवादनम्}


\twolineshloka
{पुत्र गत्वा मम वचो ब्रूयास्त्वं पितरं त्विदम्}
{दुर्मरं प्राणिनां वीर काले प्राप्ते कथञ्चन}


\twolineshloka
{याऽहं त्वया विनाऽद्येह पत्या पुत्रेण चैव ह}
{मरणं नाभिगच्छामि हतस्वस्तिरकिञ्चना}


\twolineshloka
{अथवा धर्मराज्ञाऽहमनुज्ञाता महाभुजः}
{भक्षयिष्ये विषं घोरं प्रवेक्ष्ये वा हुताशनम्}


\twolineshloka
{अथवा दुर्भरं तात यदिदं मे सहस्रधा}
{पतिपुत्रविहीनाया हृदयं न विदीर्यते}


\twolineshloka
{उत्तिष्ठ पुत्र पस्येमां दुःखितां प्रपितामहीम्}
{आर्तामुपप्लुतां दीनां निमग्नां शोकसागरे}


\twolineshloka
{आर्यां च पश्य पाञ्चालीं सात्वतीं च तपस्विनीम्}
{मां च पश्य सुदुःखार्तां व्याधविद्धां मृगीमिव}


\twolineshloka
{उत्तिष्ठ पश्य वदनं लोकनाथस्य धीमतः}
{पुण्डरीकपलाशाक्षं पुरेव चपलेक्षणः}


\twolineshloka
{एवं विप्रलपन्तीं तु दृष्ट्वा निपतितां पुनः}
{उत्तरां तां स्त्रियः सर्वाः पुनरुत्थापयन्त्युत}


\twolineshloka
{उत्थाय च पुनर्धैर्यात्तदा मत्स्यपतेः सुता}
{प्राञ्जलिः पुण्डरीकाक्षं भूमावेवाभ्यवादयत्}


\twolineshloka
{श्रुत्वा स तस्या विपुलं विलापं पुरुषर्षभः}
{उपस्पृश्य ततः कृष्णो ब्रह्मास्त्रं प्रत्यसंहरत्}


\twolineshloka
{प्रतिजज्ञे च दाशार्हस्तस्य जीवितमच्युतः}
{अब्रवीच्च विशुद्धात्मा सर्वं विश्रावयज्जगत्}


\twolineshloka
{न ब्रवीम्युत्तरे मिथ्या सत्यमेतद्भविष्यति}
{एष संजीवयाम्येनं पश्यतां सर्वदेहिनाम्}


\twolineshloka
{नोक्तपूर्वं मया मिथ्या स्वैरेष्वपि कदाचन}
{न च युद्धात्परावृत्तस्तथा संजीवतामयम्}


\twolineshloka
{यथा मे दयितो धर्मो ब्राह्मणश्च विशेषतः}
{अभिमन्योः सुतो जातो मृतो जीवत्वयं तथा}


\twolineshloka
{यथाऽहं नाभिजानामि विजये तु कदाचन}
{विरोधं तेन सत्येन मृतो जीवत्वयं शिशुः}


\twolineshloka
{यथा सत्यं च धर्मश्च मयि नित्यं प्रतिष्ठितौ}
{तथा मृतः शिशुरयं जीवतादभिमन्युजः}


\twolineshloka
{यथा कंसश्च केशी च धर्मेण निहतौ मया}
{तेन सत्येन बालोऽयं पुनः संजीवतामिह}


\threelineshloka
{इत्युक्त्वा वासुदेवोऽथ तं बालं भरतर्षभ}
{`पादेन कमलाभेन ब्रह्मरुद्रार्चितेन च}
{पस्पर्श पुण्डरीकाक्ष आपादतलमस्तकम्}


\twolineshloka
{स्पृष्टमात्रस्तु कृष्णेन स बालो भरतर्षभ}
{शनैःशनैर्महाराज प्रापद्यत स चेतनाम् ॥'}


% Check verse!
शनैःशनैर्महाराज प्रास्पन्दत सचेतनः
\chapter{अध्यायः ७०}
\twolineshloka
{ब्रह्मास्त्रं तु यदा राजन्कृष्णेन प्रतिसंहृतम्}
{तदा तद्वेश्म ते पित्रा तेजसाऽभिविदीपितम्}


\twolineshloka
{ततो रक्षांसि सर्वाणि नेशुस्त्यक्त्वा गृहं तु तत्}
{अन्तरिक्षे च वागासीत्साधु केशव साध्विति}


\twolineshloka
{तदस्त्रं ज्वलितं चापि पितामहमगात्तदा}
{ततः प्राणान्पुनर्लेभे पिता तव नरेश्वर}


\twolineshloka
{व्यचेष्टत च बालोसौ यथोत्साहं यताबलम्}
{बभूवुर्मुदिता राजंस्ततस्ता भरतस्त्रियः}


\twolineshloka
{ब्राह्मणा वाचयामासुर्गोविन्दस्यैव शासनात्}
{ततस्ता मुदिताः सर्वाः प्रशशंसुर्जनार्दनम्}


\threelineshloka
{स्त्रियो भरतसिंहानां नावं लब्ध्वेव पारगाः}
{कुन्ती द्रुपदपुत्री च सुभद्रा चोत्तरा तथा}
{स्त्रियश्चान्या नृसिंहानां बभूवुर्हृष्टमानसाः}


\twolineshloka
{तत्र मल्ला नटाश्चैव ग्रन्थिकाः सौख्यशायिकाः}
{सूतमागधसङ्घाश्चाप्यस्तुवंस्तं जनार्दनम्}


\twolineshloka
{कुरुवंशस्तवाख्याभिराशीर्भिर्भरतर्षभ}
{`सभाजयत संहृष्टो महाराज महाजनः ॥'}


\twolineshloka
{उत्थाय तु यथाकालमुत्तरा यदुनन्दनम्}
{अभ्यवादयत प्रीता सह पुत्रेण भारत}


\threelineshloka
{ततस्तस्यै ददौ प्रीतो बहुरत्नं विशेषतः}
{तथैव वृष्णिशार्दूलो नाम चास्याकरोत्प्रभुः}
{पितुस्तव महाराज सत्यसन्धो जनार्दनः}


\twolineshloka
{परिक्षीणे कुले यस्माज्जातोऽयमभिमन्युजः}
{परिक्षिदिति नामास्य भवत्वित्यब्रवीत्तदा}


\twolineshloka
{सोऽवर्धत यथाकालं पिता तव जनाधिप}
{मनःप्रह्लादनश्चासीत्सर्वलोकस्य भारत}


\twolineshloka
{मासजातस्तु ते वीर पिता भवति भारत}
{अथाजग्मुः सुबहुलं रत्नमादाय पाण्डवाः}


\twolineshloka
{`मेरुकूटनिभान्भाण्डान्कलशान्भाजनानि च}
{कृताकृतं महद्भीममादाय पुरुषोत्तमाः}


\twolineshloka
{भारतैर्वाहनैस्तत्र गोरुते गोरुते पथि}
{निवसन्तो ययुर्देवं स्मरन्तः परमेष्ठिनः}


\twolineshloka
{नासीत्तत्र नृपः कश्चिदभारार्तो नृपं विना}
{भीमादयोऽपि यज्ञार्थं वहन्ते किं पुनर्जनाः ॥'}


\twolineshloka
{तान्समीपगताञ्श्रुत्वा निर्ययुर्वृष्णिपुङ्गवाः}
{अलञ्चक्रुश्च माल्यौघैर्नगरं नागसाह्वयम्}


\twolineshloka
{पताकाभिर्विचित्राभिर्ध्वजैश्च विविधैरपि}
{वेश्मानि समलञ्चक्रुः पौराश्चापि जनेश्वरः}


\twolineshloka
{देवतायतनानां च पूजाः सुविविधास्तथा}
{संदिदेशाथ विदुरः पाण्डुपुत्रप्रियेप्सया}


\twolineshloka
{राजमार्गाश्च तत्रासन्सुमनोभिरलङ्कृताः}
{शुशुभे तत्पुरं चापि समुद्रौघनिभस्वनम्}


\twolineshloka
{नर्तकैश्चापि नृत्यद्भिर्गायकानां च निःस्वनैः}
{आसीद्वैश्रवणस्येव निवासस्तत्पुरं तदा}


\twolineshloka
{बन्दिभिश्च नरै राजन्स्त्रीसहायैश्च सर्वशः}
{तत्रतत्र विविक्तेषु समन्तादुपशोभितम्}


\twolineshloka
{पताका धूयमानाश्च समन्तान्मातरिश्वना}
{अदर्शयन्निव तदा कुरून्वै दक्षिणोत्तरान्}


\twolineshloka
{अघोषयंस्तदा चाप पुरुषा राजमार्गतः}
{सर्वरात्रविहारोऽद्य रत्नाभरणलक्षणः}


\chapter{अध्यायः ७१}
\threelineshloka
{तान्समीपगताञ्श्रुत्वा पाण्डवाञ्शत्रुकर्शनः}
{वासुदेवः सहामात्यः प्रययौ ससुहृद्गणः}
{ते समेत्य यथान्यायं प्रत्युद्याता दिदृक्षया}


\twolineshloka
{ते समेत्य तथाधर्मं पाण्डवा वृष्णिभिः सह}
{विविशुः सहिता राजन्पुरं वारणसाह्वयम्}


\twolineshloka
{विशतस्तस्य सैन्यस्य खुरनेमिस्वनेन ह}
{द्यावापृथिव्यौ खं चैव सर्वमासीत्समावृतम्}


\twolineshloka
{ते कोशानग्रतः कृत्वा विविशुः स्वं पुरं तदा}
{पाण्डवाः प्रीतमनसः सामात्याः ससुहृद्गणाः}


\twolineshloka
{ते समेत्य यथान्यायं धृतराष्ट्रं जनाधिपम्}
{कीर्तयन्तः स्वनामानि तस्य पादौ ववन्दिरे}


\twolineshloka
{धृतराष्ट्राभ्यनुज्ञाता गान्धारीं सुबलात्मजाम्}
{कुन्तीं च राजशार्दूल तदा भरतसत्तम}


\twolineshloka
{विदुरं यूजयित्वा च वैश्यापुत्रं समेत्य च}
{पूज्यमानाः स्म ते वीरा व्यरोचन्त विशाम्पते}


\twolineshloka
{ततस्तत्परमाश्चर्यं विचित्रं महदद्भुतम्}
{शुश्रुवुस्ते तदा वीराः पितुस्ते जन्म भारत}


\twolineshloka
{तदुपश्रुत्य तत्कर्म वासुदेवस्य धीमतः}
{पूजार्हं पूजयामासुः कृष्णं देवकीनन्दनम्}


\twolineshloka
{ततः कतिपयाहस्य व्यासः सत्यवतीसुतः}
{आजगाम महातेजा नगरं नागसाह्वयम्}


\twolineshloka
{तस्य सर्वे यथान्यायं पूजा चक्रुः कुरूद्वहाः}
{सह वृष्ण्यन्धकव्याघ्रैरुपासांचक्रिरे तदा}


\twolineshloka
{तत्र नानाविधाकाराः कथाः समभिकीर्त्य वै}
{युधिष्ठिरो धर्मसुतो व्यासं वचनमब्रवीत्}


\twolineshloka
{भवत्प्रसादाद्भगवन्यदिदं रत्नमाहृतम्}
{उपयोक्तुं तदिच्छामि वाजिमेधे महाक्रतौ}


\threelineshloka
{तमनुज्ञातुमिच्छामि भवता मुनिसत्तम}
{त्वदधीना वयं सर्वे कृष्णस्य च महात्मनः ॥व्यास उवाच}
{}


\twolineshloka
{अनुजानामि राजंस्त्वां क्रियतां यदनन्तरम्}
{यजस्व वाजिमेधेन विधिवद्दक्षिणावता}


\twolineshloka
{अश्वमेधो हि राजेन्द्र पावनः सर्वपाप्मनाम्}
{तेनेष्ट्वा त्वं विपात्मा वै भविता नात्र संशयः}


\twolineshloka
{इत्युक्तः स तु धर्मात्मा कुरुराजो युधिष्ठिरः}
{अश्वमेधस्य कौरव्य चकाराहरणे मतिम्}


\twolineshloka
{समनुज्ञाप्य तत्सर्वं कृष्मद्वैपायनं नृपः}
{वासुदेवमथाभ्येत्य वाग्मी वचनमब्रवीत्}


\twolineshloka
{देवकी सुप्रजा देवी त्वया पुरुषसत्तम}
{यद्ब्रूयां त्वां महाबाहो तत्कृथास्त्वमिहाच्युत}


\twolineshloka
{त्वत्प्रभावार्जितान्भोगानश्नीम यदुनन्दन}
{पराक्रमेण बुद्ध्या च त्वयेयं निर्जिता मही}


\twolineshloka
{दीक्षयस्व त्वमात्मानं त्वं हि नः परमो गुरुः}
{त्वयीष्टवति दाशार्ह विपाप्मा भविता ह्यहम्}


\threelineshloka
{त्वं हि यज्ञो गुरुश्च त्वं धर्मज्ञस्त्वं प्रजापतिः}
{त्वं गतिः सर्वभूतानामिति मे निश्चिता मतिः ॥वासुदेव उवाच}
{}


\twolineshloka
{त्वमेवैतन्महाबाहो कर्तुमर्हस्यरिंदम}
{त्वं गतिः सर्वभूतानामिति मे निश्चिता मतिः}


\threelineshloka
{त्वं चाद्यि कुरुवीराणां धर्मेण हि विराजसे}
{गुणीभूताः स्म ते राजंस्त्वं नो राजन्गुरुर्मतः}
{यजस्व मदनुज्ञातः प्राप्य एष क्रतुस्त्वया}


\twolineshloka
{युनक्तु नो भवान्कार्ये यत्र वाञ्छसि भारत}
{सत्यं ते प्रतिजानामि सर्वं कर्तास्मि तेऽनघ}


\twolineshloka
{भीमसेनार्जुनौ चैव तथा माद्रवतीसुतौ}
{इष्टवन्तो भविष्यन्ति त्वयीष्टवति पार्थिवे}


\chapter{अध्यायः ७२}
\twolineshloka
{एवमुक्तस्तु कृष्णेन धर्मपुत्रो युधिष्ठिरः}
{व्यासमामन्त्र्य् मेधावी ततो वाचनमब्रवीत्}


\threelineshloka
{यदा कालं भवान्वेत्ति हयमेधस्य तत्त्वतः}
{दीक्षयस्व तदा मां त्वं त्वय्यायत्तो हि मे क्रुतुः ॥व्यास उवाच}
{}


\twolineshloka
{अयं पैलोथ कौन्तेय याज्ञवल्क्यस्तथैव च}
{विधानं यद्यथा कालं तत्कर्तारौ न संशयः}


\twolineshloka
{चैत्र्यां हि पौर्णमास्यां तु तव दीक्षा भविष्यति}
{सम्भाराः सम्भ्रियन्तां च यज्ञार्थं पुरुषर्षभ}


\twolineshloka
{अश्वविद्याविदश्चैव सूता विप्राश्च तद्विदः}
{मेध्यमश्वं परीक्षन्तां तव यज्ञार्थसिद्धये}


\twolineshloka
{तमुत्सृज यथाशास्त्रं पृथिवीं सागराम्बराम्}
{सपर्येतु यशो नाम्ना तव पार्थिव वर्धयन्}


\twolineshloka
{इत्युक्तः स तथेत्युक्त्वा पाण्डवः पृथिवीपतिः}
{चकार सर्वं राजेन्द्र यथोक्तं ब्रह्मवादिना}


% Check verse!
सम्भारश्चैव राजेन्द्र सर्वे सङ्कल्पितास्तता
\twolineshloka
{स सम्भारान्समाहृत्य नृपो धर्मसुतस्तदा}
{न्यवेदयदमेयात्मा कृष्णद्वैपायनाय वै}


\twolineshloka
{ततोऽब्रवीन्महातेजा व्यासो धर्मात्मजं नृपम्}
{यथाकालं यथायोगं सज्जाः स्म तव दीक्षणे}


\twolineshloka
{स्फ्यश्च कूर्चश्च सौवर्णो यच्चान्यदपि कौरव}
{यत्तु योग्यं भवेत्किञ्चिद्रौक्मं तत्क्रियतामिति}


\threelineshloka
{अश्वश्चोत्सृज्यतामद्य पृथ्व्यामथ यथाक्रमम्}
{सुगुप्तं चरतां चापि यथाशास्त्रं यथाविधि ॥युधिष्ठिर उवाच}
{}


\twolineshloka
{अयमश्वो यता ब्रह्मन्नुत्सृष्टः पृथिवीमिमाम्}
{चरिष्यति यथाकामं तत्र वै संविदीयताम्}


\twolineshloka
{पृथिवीं पर्यटन्तं हि तुरगं कामचारिणम्}
{कः पालयेदिति मुने तद्भवान्वक्तुमर्हति}


\threelineshloka
{इत्युक्तः स तु राजेन्द्र कृष्णद्वैपायनोऽब्रवीत्}
{भीमसेनादवरजः श्रेष्ठः सर्वधनुष्मताम्}
{जिष्णुः सहिष्णुर्धृष्णुस्च स एनं पालयिष्यति}


\threelineshloka
{शक्तः स हि महीं जेतुं निवातकवचान्तकः}
{तस्मिन्ह्यस्त्राणि दिव्यानि दिव्यं संहननं तथा}
{दिव्यं धनश्चेषुधी च स एनमनुयास्यति}


\twolineshloka
{स हि धर्मार्थकुशलः सर्वविद्याविशारदः}
{यथाशास्त्रं नृपश्रेष्ठ चारयिष्यति ते हयम्}


\twolineshloka
{राजपुत्रो महाबाहुः श्यामो राजीवलोचनः}
{अभिमन्योः पिता वीरः स एनमनुयास्यति}


\twolineshloka
{भीमसेनोपि तेजस्वी कौन्तेयोऽमितविक्रमः}
{समर्थो रक्षितुं राष्ट्रं नकुलश्च विशाम्पते}


\twolineshloka
{सहदेवस्तु कौरव्य समायास्यति बुद्धिमान्}
{कुटुम्बतन्त्रं विधिवत्सर्वमेव महायशाः}


\threelineshloka
{स तु सर्वं यथान्यायमुक्त कुरुकुलोद्वहः}
{चकार फल्गुनं चापि संदिदेश हयं प्रति ॥युधिष्ठिर उवाच}
{}


\twolineshloka
{एह्यर्जुन त्वया वीर हयोऽयं परिपाल्यताम्}
{त्वमर्हो रक्षितुं ह्येनं नान्यः कश्चन मानवः}


\twolineshloka
{ये चापि त्वां महाबाहो प्रत्युद्यान्ति नराधिपाः}
{तैर्विग्रहो यथा न स्यात्तथा कार्यं त्वयाऽनघ}


\threelineshloka
{आख्यातव्यश्च भता यज्ञोऽयं मम सर्वशः}
{पार्थिवेभ्यो महाबाहो समये गम्यतामिति ॥वैशम्पायन उवाच}
{}


\twolineshloka
{एवमुक्त्वा स धर्मात्मा भ्रातरं सव्यसाचिनम्}
{भीमं च नकुलं चैव पुरगुप्तौ समादधत्}


\twolineshloka
{कुटुम्बतन्त्रे च तदा सहदेवं युधांपतिम्}
{अनुमान्य महीपालं धृतराष्ट्रं युधिष्ठिरः}


\chapter{अध्यायः ७३}
\twolineshloka
{दीक्षाकाले तु सम्प्राप्ते ततस्ते सुमहर्त्विजः}
{विधिवद्दीक्षयामासुरश्वमेधाय पार्थिवम्}


\twolineshloka
{कृत्वा स पशुमेधांश्च दीक्षितः पाण्डुनन्दनः}
{धर्मराजो महातेजाः सहर्त्विग्भिर्व्यरोचत}


\twolineshloka
{हयश्च हयमेधार्थं स्वयं स ब्रह्मवादिना}
{उत्सृष्टः शास्त्रविधिना व्यासेनामिततेजसा}


\twolineshloka
{स राजा राजधर्मेण दीक्षितो विबभौ तदा}
{हेममाली रुक्मकण्ठः प्रदीप्त इव पावकः}


\twolineshloka
{कृष्णाजिनी दण्डपाणिः क्षौमवासाः स धर्मजः}
{विबभौ द्युतिमान्भूयः प्रजापतिरिवाध्वरे}


\twolineshloka
{तथैवास्यर्त्विजः सर्वे तुल्यवेषा विशांपते}
{बभूवुरर्जुनश्चापि प्रदीप्त इव पावकः}


\twolineshloka
{श्वेताश्वः कपिकेतुश्च ससाराश्वं धनञ्जयः}
{विधिवत्पृथिवीपाल धर्मराजस्य शासनात्}


\twolineshloka
{**क्षिपन्गाण्डिवं राजन्बद्धगोधाङ्गुलित्रवान्}
{तमश्वं पृथिवीपाल मुदा युक्तः ससार च}


\twolineshloka
{अनुमार्गं तदा राजन्नागमत्तत्पुरं विभो}
{द्रष्टुकामं कुरुश्रेष्ठं प्रयास्यनतं धनञ्जयम्}


\twolineshloka
{तेषामन्योन्यसम्मर्दादूष्मेव समाजायत}
{दिदृक्षूणां हयं तं च तं चैव हयसारिणम्}


\twolineshloka
{ततः शब्दो महाराज दिशः खं प्रतिपूरयन्}
{बभूव प्रेक्षतां नॄमां कुन्तीपुत्रं धनंजयम्}


\twolineshloka
{एष गच्छति कौन्तेयस्तुरगश्चैव दीप्तिमान्}
{समन्वेति महाबाहुः संस्पृशन्धनुरुत्तमम्}


\twolineshloka
{एवं शुश्राव वदतां गिरो जिष्णुरुदारधीः}
{स्वस्ति तेऽस्तु व्रजारिष्टं पुनश्चैहीति भारत}


\twolineshloka
{अथापरे मनुष्येन्द्र पुरुषा वाक्यमब्रुवन्}
{नैनं पश्याम सम्मर्दे धनुरेतत्प्रदृश्यते}


\threelineshloka
{एतद्धि भीमनिर्ह्रादं विश्रुतं गाण्डिवं धनुः}
{स्वस्ति गच्छत्वरिष्टो वै पन्थानमकुतोभयम्}
{निवृत्तमेनं द्रक्ष्यामः पुनरेष्यति च ध्रुवम्}


\twolineshloka
{एवमाद्या मनुष्याणां स्त्रीणां च भरतर्षभ}
{शुश्राव मधुरा वाचः पुनःपुनरुदारधीः}


\twolineshloka
{याज्ञवल्क्यस्य शिष्यश्च कुशलो यज्ञकर्मणि}
{प्रायात्पार्थेन सहितः शान्त्यर्थं वेदपारगः}


\threelineshloka
{ब्राह्मणाश्च महीपाल बहवो वेदपारगाः}
{अनुजग्मुर्महात्मानं क्षत्रियाश्च विशाम्पते}
{विधिवत्पृथिवीपाल धर्मराजस्य शासनात्}


\twolineshloka
{पाण्डवैः पृथिवीमश्वो निर्जितामस्त्रतेजसा}
{चचार स महाराज यथादेशं च सत्तम}


\twolineshloka
{तत्र युद्धानि वृत्तानि यान्यासन्पाण्डवस्य ह}
{तानि वक्ष्यामि ते वीर विचित्राणि महान्ति च}


\twolineshloka
{स हयः पृथिवीं राजन्प्रदक्षिणमवर्तत}
{ससारेत्तरतः पूर्वं तन्निबोध महीपते}


\twolineshloka
{अवमृद्रन्स राष्ट्राणि पार्थिवानां हयोत्तमः}
{शनैस्तदा परिययौ श्वेताश्वश्च महारथः}


\twolineshloka
{तत्र सङ्गणना नास्ति राज्ञामयुतशस्तदा}
{येऽयुध्यन्त महाराज क्षत्रिया हतबान्धवाः}


\twolineshloka
{किराता यवना राजन्बहवोऽसिधनुर्धराः}
{म्लेच्छाश्चान्ये बहुविधाः पूर्वं ये निकृता रणे}


\twolineshloka
{आर्याश्च पृथिवीपालाः प्रहृष्टनरवाहनाः}
{समीयुः पाण्डुपुत्रेण बहवो युद्धदुर्मदाः}


\twolineshloka
{एवं वृत्तानि युद्धानि तत्रतत्र महीपते}
{अर्जुनस्य महीपालैर्नानादेशसमागतैः}


\twolineshloka
{यान्यत्र हयतो राजन्प्रवृत्तानि महान्ति च}
{तानि युद्धानि वक्ष्यामि कौन्तेयस्य तवानघ}


\chapter{अध्यायः ७४}
\twolineshloka
{त्रिगर्तैरभवद्युद्धं कृतवैरैः किरीटिनः}
{महारथसमाज्ञातैर्हतानां पुत्रनप्तृभिः}


% Check verse!
ते समाज्ञाय सम्प्राप्तं यज्ञियं तुरगोत्तम*****विषयान्तं ततो वीरा दंशिताः पर्यवारयन्
\twolineshloka
{रथिनो बद्धतूणीराः सदश्वैः समलङ्कृतैः}
{परिवार्य हयं राजन्ग्रहीतुं सम्प्रचक्रमुः}


\twolineshloka
{ततः किरीटी सञ्चिन्त्य तेषां तत्र चिकीर्षितम्}
{वारयामास तान्वीरान्सान्त्वपूर्वमरिंदमः}


\twolineshloka
{तदनादृत्य ते सर्वे शरैरभ्यहनंस्तदा}
{तमोरजोभ्यां संछन्नांस्तान्किरीटी न्यवारयत्}


\twolineshloka
{तानव्रवीत्ततो जिष्णुः प्रहसन्निव भारत}
{निवर्तध्वमधर्मज्ञाः श्रेयो जीवितमेव च}


\twolineshloka
{स हि वीरः प्रयास्यन्वै धर्मराजेन वारितः}
{हतबान्धवा न ते पार्त हन्तव्याः पार्थिवा इति}


\twolineshloka
{स तदा तद्वचः श्रुत्वा धर्मराजस्य धीमतः}
{तान्निवर्तध्वमित्याह न न्यवर्तन्ति चापि ते}


\twolineshloka
{ततस्त्रिगर्तराजानं सूर्यवर्माणमाहवे}
{विनद्य शरजालेनि प्रजहास धनंजयः}


\twolineshloka
{ततस्ते रथघोपेण रथनेमिस्वनेन च}
{पूरयन्तो दिशः सर्वा धनंजयमुपाद्रवन्}


\twolineshloka
{सूर्यवर्मा ततः पार्ते शराणां नतपर्वणाम्}
{शतान्यमुञ्चद्राजेन्द्र लघ्वस्त्रमभिदर्शयन्}


\twolineshloka
{तथैवान्ये महेष्वासा ये च तस्यानुयायिनः}
{मुमुचुः शरवर्षाणि धनंजयवधैषिणः}


\twolineshloka
{स ताञ्ज्यामुखनिर्मुक्तैर्बहुभिः सुबहूञ्शरान्}
{चिच्छेद पाण्डवो राजंस्ते भूमौ न्यपतंस्तदा}


\twolineshloka
{केतुवर्मा तु तेजस्वी तस्यैवावरजो युवा}
{युयुधे भ्रातुरर्थाय पाण्डवेन यशस्विना}


\twolineshloka
{तमापतन्तं सम्प्रेक्ष्य केतुवर्माणमाहवे}
{अभ्यघ्नन्निशितैर्बाणैर्बीभत्सुः परवीरहा}


\twolineshloka
{केतुवर्मण्यभिहते धृतवर्मा महारथः}
{रथेनाशु समुत्पत्य शरैर्जिष्णुमवाकिरत्}


\twolineshloka
{तस्य तां शीघ्रतामीक्ष्य तुतोषातीव वीर्यवान्}
{गुडाकेशो महादेजा बालस्य धृतवर्मणः}


\twolineshloka
{न संदधानं ददृशे नाददानं च तं तदा}
{किरन्तमेवं स शरान्ददृशे पाकशासनिः}


\twolineshloka
{स तु तं पूजयामास धृतवर्माणमाहवे}
{मनसा तु मुहूर्तं वै रणे समभिहर्षयन्}


\twolineshloka
{`न विव्याध रणे क्रुद्धः कुन्तीपुत्रो हसन्निव}
{सौभद्रस्येव तत्कर्म दृष्ट्वा बालस्य विस्मितः}


\twolineshloka
{तं पन्नगमिव क्रुद्धं कुरुवीरः स्मयन्निव}
{प्रीतिपूर्वं महाबाहुः प्राणैर्न व्यपरोपयत्}


\twolineshloka
{स तथा रक्ष्यमाणो वै पार्थेनामिततेजसा}
{धृतवर्मा शरं दीप्तं मुमोच विजये तदा}


\twolineshloka
{स तेन विजयस्तूर्णमासीद्विद्धः करे भृशम्}
{मुमोच गाण्डिवं मोहात्तत्पपाताथ भूतले}


\twolineshloka
{धनुषः पततस्तस्य सव्यसाचिकराद्विभो}
{बभूव सदृशं रूपं शक्रचापस्य भारत}


\twolineshloka
{तस्मिन्निपतिते दिव्ये महाधनुषि पार्थिवः}
{चकार सस्वनं हासं धृतवर्मा महाहवे}


\twolineshloka
{ततो रोषार्दितो जिष्णुः प्रमृज्य रुधिरं करात्}
{धनुरादत्त तद्दिव्यं शरवर्षैर्ववर्ष च}


\twolineshloka
{ततो हलहलाशब्दो दिवस्पृगभवत्तदा}
{नानाविधानां भूतानां तत्कर्माणि प्रशंसताम्}


\twolineshloka
{ततः सम्प्रेक्ष्य संक्रुद्धं कालान्तकयमोपमम्}
{जिष्णुं त्रैगर्तका योधाः परीताः पर्यवारयन्}


\twolineshloka
{अभिसृत्य परीप्सार्थं ततस्ते धृतवर्मणः}
{परिवव्रुर्गुडाकेशं तत्राक्रुद्ध्यद्धनंजयः}


\twolineshloka
{ततो योधाञ्जघानाशु तेषां स दश चाष्ट च}
{महेन्द्रवज्रप्रतिमैरायसैर्बहुभिः शरैः}


\twolineshloka
{तान्सम्प्रभग्नान्सम्प्रेक्ष्य त्वरमाणो धनंजयः}
{शरैराशीविषाकारैर्जघान स्वनवद्धसन्}


\twolineshloka
{ते भग्नमनसः सर्वे त्रैगर्तकमहारथाः}
{दिशोऽभिदुद्रुवू राजन्धनंजयशरार्दिताः}


\twolineshloka
{`हतावशिष्टा हि पराः पार्थं दृष्टपराक्रमाः}
{'तमूचुः पुरुषव्याघ्रं संशप्तकनिषूदनम्}


\threelineshloka
{तवास्म किंकराः सर्वे सर्वे वै वशगास्तव}
{आज्ञापयस्वः नः पार्थ प्रह्वान्प्रेष्यानवस्थितान्}
{करिष्यामः प्रियं सर्वं तव कौरवनन्दन}


\twolineshloka
{एतदाज्ञाय वचनं सर्वांस्तानब्रवीत्तदा}
{जीवितं रक्षत नृपाः शासनं प्रतिगृह्यताम्}


\chapter{अध्यायः ७५}
\twolineshloka
{प्राग्ज्योतिषमथाभ्येत्य व्यचरत्स हयोत्तमः}
{भगदत्तात्मजस्तत्र निर्ययौ रणकर्कशः}


\twolineshloka
{सहयं पाण्डुपुत्रं तु विषयान्तमुपागतम्}
{युयुधे भरतश्रेष्ठ यज्ञदत्तो महीपतिः}


\twolineshloka
{सोभिनिर्याय नगराद्भगदत्तसुतो नृपः}
{अश्वमायान्तमुन्मथ्य नगराभिमुखो ययौ}


\twolineshloka
{तमालक्ष्य महाबाहुः कुरूणामृषभस्तदा}
{गाण्डीवं विक्षिपंस्तूर्णं सहसा समुपाद्रवत्}


\twolineshloka
{ततो गाण्डीवनिर्मुक्तैरिषुभिर्मोहितो नृपः}
{हयमुत्सृज्य तं वीरस्ततः पार्थमुपाद्रवत्}


\twolineshloka
{पुनः प्रविश्य नगरं दंशितः स नृपोत्तमः}
{आरुह्य नागप्रवरं निर्ययौ युद्धकाङ्क्षया}


\twolineshloka
{पाण्डुरेणातपत्रेण ध्रियमाणेन मूर्धनि}
{दोधूयता चामरेण श्वेतेन च महारथः}


\twolineshloka
{ततः पार्थं समासाद्य पाण्डवानां महारथम्}
{आह्वयामास बीभत्सुं बाल्यान्मोहाच्च संयुगे}


\twolineshloka
{स वारणं नगप्रख्यं प्रभिन्नकरटामुखम्}
{प्रेषयामास संक्रुद्धः श्वेताश्वं प्रति पार्थिवः}


\twolineshloka
{विक्षरन्तं महामेघं परवारणवारणम्}
{शस्त्रवत्कल्पितं सङ्ख्ये विवशं युद्धदुर्मदम्}


\twolineshloka
{प्रचोद्यमानः स गजस्तेन राज्ञा महाबलः}
{तदाऽङ्कशेन विबभावुत्पतिष्यन्निवाम्बरम्}


\twolineshloka
{तमापतन्तं सम्प्रेक्ष्य क्रुद्धो राजन्धनंजयः}
{भूमिष्ठो वारणगतं योधयामास भारत}


\twolineshloka
{यज्ञदत्तस्ततः क्रुद्धो मुमोचाशु धनंजये}
{तोमरानग्निसङ्काशाञ्शलभानिव वेगितान्}


\twolineshloka
{अर्जुनस्तानसम्प्राप्तान्गाण्डीवप्रभवैः शरैः}
{द्विधा त्रिधा च चिच्छेद खगमान्खगमैस्तदा}


\twolineshloka
{स तान्दृष्ट्वा तथा छिन्नांस्तोमरान्भगदत्तजः}
{इषूनसक्तांस्त्वरितः प्राहिणोत्पाण्डवं प्रति}


\twolineshloka
{ततोऽर्जुनस्तूर्णतरं रुक्मपुङ्खानजिह्मगान्}
{प्रेषयामास संक्रुद्धो भगदत्तात्मजं प्रति}


\twolineshloka
{स तैर्विद्धो महातेजा यज्ञदत्तो महामृधे}
{भृशाहतः पपातोर्व्यां न त्वेनमजहात्स्मृतिः}


\twolineshloka
{ततः स पुनरारुह्य वारणप्रवरं रणे}
{अव्यग्रः प्रेषयामास जयार्थी विजयं प्रति}


\twolineshloka
{तस्मै बाणांस्ततो जिष्णुर्निर्मुक्ताशीविषोपमान्}
{प्रेषयामास संक्रुद्धो ज्वलितज्वलनोपमान्}


\twolineshloka
{स तैर्विद्धो महानागो विस्रवन्रुधिरं बभौ}
{हिमवानिव शैलेन्द्रो बहुप्रस्रवणस्तदा}


\chapter{अध्यायः ७६}
\twolineshloka
{एवं त्रिरात्रमभवत्तद्युद्धं भरतर्षभ}
{अर्जुनस्य नरेन्द्रेण वृत्रेणेव शतक्रतोः}


\twolineshloka
{ततश्चतुर्थे दिवसे यज्ञदत्तो महाबलः}
{जहास सस्वनं हासं वाक्यं चेदमताब्रवीत्}


\twolineshloka
{अर्जुनार्जुन तिष्ठस्व न मे जीवन्विमोक्ष्यसे}
{त्वां निहत्य करिष्यामि पितुस्तोयं यथाविधि}


\twolineshloka
{त्वया वृद्धो मम पिता भगदत्तः पितुः सखा}
{हतो वृद्धोऽपि बाधित्वा शिशुं मामद्य योधय}


\twolineshloka
{इत्येवमुक्त्वा संक्रुद्धो यज्ञदत्तो नराधिपः}
{प्रेषयामास कौरव्य वारणं पाण्डवं प्रति}


\twolineshloka
{सम्प्रेष्यमाणो नागेन्द्रो यज्ञदत्तेन धीमता}
{उत्पतिष्यन्निवाकाशमभिदुद्राव पाण्डवम्}


\twolineshloka
{अग्रहस्तसुमुक्तेन शीकरेण स नागराट्}
{समौक्षति गुडाकेशं शैलं नील इवाम्बुदः}


\twolineshloka
{स तेन प्रेषितो राज्ञा मेघवद्विनदन्मुहुः}
{मुखाडम्बरसंह्रादैरभ्यद्रवत फल्गुनम्}


\twolineshloka
{स नृत्यन्निव नागेन्द्रो यज्ञदत्तप्रचोदितः}
{आससाद द्रुतं राजन्कौरवाणां महारथम्}


\twolineshloka
{तमायान्तमथालक्ष्य यज्ञदत्तस्य वारणम्}
{गाण्डीवमाश्रित्य बली न व्यकम्पत शत्रुहा}


\twolineshloka
{चुक्रोध बलवच्चापि पाण्डवस्तस्य भूपतेः}
{कार्यविघ्नमनुस्मृत्यि पूर्ववैरं च भारत}


\twolineshloka
{ततस्तं वारणं क्रुद्धः शरजालेन पाण्डवः}
{निवारयामास तदा वेलेव मकरालयम्}


\twolineshloka
{स नागप्रवरः श्रीमानर्जुनेन निवारितः}
{तस्थौ शरैर्विनुन्नाङ्गः श्वाविच्छललितो यथा}


\twolineshloka
{निवारितं गजं दृष्ट्वा भगदत्तसुतो नृपः}
{उत्ससर्ज शितान्बाणानर्जुने क्रोधमूर्छितः}


\twolineshloka
{अर्जुनस्तु महाबाहुः शरैररिनिघातिभिः}
{वारयामास तान्बाणांस्तदद्भुतमिवाभवत्}


\twolineshloka
{ततः पुनरभिक्रुद्धो राजा प्राग्ज्योतिषाधिपः}
{प्रेषयामास नागेन्द्रं बलवत्पर्वतोपमम्}


\twolineshloka
{तमापतन्तं सम्प्रेक्ष्य बलवान्पाकशासनिः}
{नाराचमग्निसङ्काशं प्राहिणोद्वारणं प्रति}


\twolineshloka
{स तेन वारणो राजन्मर्मस्वभिहतो भृशम्}
{पपात सहसा भूमौ वज्ररुग्ण इवाचलः}


\twolineshloka
{स पतञ्शुशुभे नागो धनंजयशराहतः}
{विशन्निव महाशैलो महीं वज्रप्रपीडितः}


\twolineshloka
{तस्मिन्निपतिते नागे यज्ञदत्तस्य पाण्डवः}
{तं न भेतव्यमित्याह ततो भूमिगतं नृपम्}


\twolineshloka
{अब्रवीद्धि महातेजाः प्रस्थितं मां युधिष्ठिरः}
{राजानस्ते न हन्तव्या धनंजय कथञ्चन}


\twolineshloka
{सर्वमेतन्नरव्याघ्र भवत्येतावता कृतम्}
{योधाश्चापि न हन्तव्या धनंजय रणे त्वया}


\twolineshloka
{वक्तव्याश्चापि राजानः सर्वे सह सुहृज्जनैः}
{युधिष्ठिरस्याश्वमेधो भवद्भिरनुभूयताम्}


\twolineshloka
{इति भ्रातृवचः श्रुत्वा न हन्मि त्वां नराधिप}
{उत्तिष्ठ न भयं तेऽस्ति स्वस्तिमान्गच्छ पार्थिव}


\twolineshloka
{आगच्छेथा महाराज परां चैत्रीमुपस्थिताम्}
{तदाऽश्वमेधो भविता धर्मिराजस्य धीमतः}


\twolineshloka
{एवमुक्तः स राजा तु भगदत्तात्मजस्तदा}
{तथेत्वेवाब्रवीद्वाक्यं पाण्डवेनाभिनिर्जितः}


\chapter{अध्यायः ७७}
\twolineshloka
{`जित्वा प्रसाद्य राजानं भगदत्तसुतं तदा}
{विसृज्य याते तुरगे सैन्धवान्प्रति भारत ॥'}


\twolineshloka
{सैन्धवैरभवगद्युद्धं ततस्तस्य किरीटिनः}
{हतशेषैर्महाराज हतानां च सुतैरपि}


\twolineshloka
{तेऽवतीर्णमुपश्रुत्य विषयं श्वेतवाहनम्}
{प्रत्युद्ययुरमृष्यन्ते राजानः पाण्डवर्षभम्}


\twolineshloka
{अश्वं च तं परामृश्य विषयान्ते विषोपमात्}
{न भयं चक्रिरे पार्थाद्भीमसेनादनन्तरात्}


\twolineshloka
{तेऽविदूराद्धनुष्पाणिं यज्ञियस्य हयस्य च}
{बीभत्सुं प्रत्यपद्यन्त् पदातिनमवस्थितम्}


\twolineshloka
{ततस्ते तं महावीर्या राजानः पर्यवारयन्}
{जिगीषन्तो नरव्याघ्रं पूर्वं विनिकृता युधि}


\twolineshloka
{ते नामान्यपि गोत्राणि कर्मणि विविधानि च}
{कीर्तयन्तस्तदा पार्थं शरवर्षैरवाकिरन्}


\twolineshloka
{ते किरन्तः शरव्रातान्वारणप्रतिवारणान्}
{रणे जयमभीप्सन्तः कौन्तेयं पर्यवारयन्}


\twolineshloka
{ते समीक्ष्य च तं कृष्णमुग्रकर्माणमाहवे}
{सर्वे युयुधिरे वीरा रथस्थास्तं पदातिनम्}


\twolineshloka
{ते तमाजघ्निरे वीरं निवातकवचान्तकम्}
{संशप्तकनिहन्तारं हन्तारं सैन्धवस्य च}


\twolineshloka
{ततो रथसहस्रेण गजानामयुतेन च}
{***ष्ठकीकृत्य बीभत्सुं प्रहृष्टमनसोऽभवन्}


\twolineshloka
{** स्मरन्तो वधं वीराः सिन्धुराजस्य चाहवे}
{जयद्रथस्य कौरव्य समरे सव्यसाचिना}


\twolineshloka
{ततः पर्जन्यवत्सर्वे शरवृष्टीरवासृजन्}
{तैः कीर्णः शुशुभे पार्थो रविर्मेघान्तरे यथा}


\twolineshloka
{स शरैः समवच्छन्नश्चकाशे पाण्डवर्षभः}
{पञ्चरान्तरसञ्चारी शकुन्त इव भारत}


\twolineshloka
{ततो हाहाकृतं सर्वं कौन्तेय शरपीडिते}
{त्रैलोक्यमभवद्राजन्रविरासीद्रजोरुणः}


\twolineshloka
{ततो ववौ महाराज मारुतो रोमहर्षणः}
{राहुरग्रसदादित्यं पर्वणीव विशाम्पते}


\twolineshloka
{उल्काश्च जघ्निरे सूर्यं विकीर्यन्त्यः समन्ततः}
{वेपथुश्चाभवद्राजन्कैलासस्य महागिरेः}


\twolineshloka
{मुमुचुः श्वासमत्युष्णं दुःखशोकसमन्विताः}
{सप्तर्षयो जातभयास्तथा देवर्षयोपि च}


\twolineshloka
{शशं चाशु विनिर्भिद्य मण्डलं शशिनोऽपतन्}
{विपरीता दिशश्चापि सर्वा धूमाकुलास्तथा}


\twolineshloka
{रासभारुणसङ्काशा धनुष्मन्तः सविद्युतः}
{आवृत्य गगनं मेघा मुमुचुर्मांसशोणितम्}


\twolineshloka
{एवमासीत्तदा वीरे शरवर्षेण संवृते}
{फल्गुने भरतश्रेष्ठ तदद्भुतमिवाभवत्}


\twolineshloka
{तस्य तेनावकीर्णस्य शरजालेन सर्वतः}
{मोहात्पपात गाण्डीवमावापश्च करादपि}


\twolineshloka
{तस्मिन्मोहमनुप्राप्ते शरजालं महत्तदा}
{सैन्धवा मुमुचुस्तूर्णं गतसत्वे महारथे}


\twolineshloka
{ततो मोहं समापन्नं ज्ञात्वा पार्थं द्विवौकसः}
{सर्वे वित्रस्तमनसस्तस्य शान्तिकृतोऽभवन्}


\twolineshloka
{ततो देवर्षयः सर्वे तथा सप्तर्षयोपि च}
{ब्रह्मर्षचस्च विजयं जेपुः पार्थस्य धीमतः}


\twolineshloka
{ततः प्रदीपिते देवैः पार्थतेजसि पार्थिव}
{तस्थावचलवद्धीमान्सङ्ग्रामे परमास्त्रवित्}


\twolineshloka
{विचकर्ष धनुर्दिव्यं ततः कौरवनन्दनः}
{यन्त्रस्येवेह शब्दोऽभून्महांस्तस्य पुनः पुनः}


\twolineshloka
{ततः स शरवर्षाणि प्रत्यमित्रान्प्रति प्रभुः}
{ववर्ष धनुषा पार्थो वर्षाणीव पुरंदरः}


\twolineshloka
{ततस्ते सैन्धवा योधाः सर्व एव सराजकाः}
{नादृश्यन्त शरैः कीर्णाः शलभैरिव पादपाः}


\twolineshloka
{तस्य शब्देन वित्रेसुर्भयार्ताश्च विदुद्रुवुः}
{मुमुचुस्चाश्रु शोकार्ताः शुशुचुश्चापि सैन्धवाः}


\twolineshloka
{तांस्तु सर्वान्नरव्याघ्रः सैन्धवान्व्यचरद्बली}
{अलातचक्रवद्राजञ्शरजालैः समार्पयत्}


\twolineshloka
{तदिन्द्रजालप्रतिमं बाणजालममित्रहा}
{विसृज्य दिक्षु सर्वासु महेन्द्रि इव वज्रभृत्}


\twolineshloka
{मेघजालनिभं सैन्यं विदार्य शरवृष्टिभिः}
{विबभौ कौरवश्रेष्ठः शरदीव दिवाकरः}


\chapter{अध्यायः ७८}
\twolineshloka
{ततो गाण्डीवभृच्छूरो युद्धाय समुपस्थितः}
{विबभौ युधि दुर्धर्षो हिमवानचलो यथा}


\twolineshloka
{ततस्ते सैन्धवा योधाः पुनरेव व्यवस्थिताः}
{व्यमुञ्चन्त सुसंरब्धाः शरवर्षाणि भारत}


\twolineshloka
{तान्प्रहस्य महाबाहुः पुनरेव व्यवस्थितान्}
{ततः प्रोवाच कौन्तेयो मुमूर्षञ्श्लक्ष्णया गिरा}


\twolineshloka
{युध्यध्वं परया शक्त्या यतध्वं विजये मम}
{कुरुध्वं सर्वकार्याणि महद्वो भयमागतम्}


\twolineshloka
{एष योत्स्यामि सर्वांस्तु निवार्य शरवागुराम्}
{तिष्ठध्वं युद्धमनसो दर्पं शमयितास्मि वः}


\twolineshloka
{एतावदुक्त्वा कौरव्यो रोषाद्गाण्डीवभृत्तदा}
{ततोऽथ वचनं स्मृत्वा भ्रातुर्ज्येष्ठस्य भारत}


\threelineshloka
{न हन्तव्या रणे तात क्षत्रिया विजिगीषवः}
{जेतव्याश्चेति यत्प्रोक्तं धर्मराज्ञा महात्मना}
{चिन्तयामास स तदा फल्गुनः पुरुषर्षभः}


\twolineshloka
{इत्युक्तोऽहं नरेन्द्रेण न हन्तव्या नृपा इति}
{कथितं न मृषेदं स्याद्धर्मराजवचः शुभम्}


\twolineshloka
{न हन्येरंश्च राजानो राज्ञश्चाज्ञा कृता भवेत्}
{इति सञ्चिन्त्य स तदा फल्गुनः पुरुषर्षभः}


\twolineshloka
{प्रोवाच वाक्यं धर्मज्ञः सैन्धवान्युद्धदुर्मदान्}
{बालांस्त्रियो वा युष्माकं न हनिष्ये व्यवस्थितान्}


\twolineshloka
{यश्च वक्ष्यति सङ्ग्रामे तवास्मीति पराजितः}
{एतच्छ्रुत्वा वचो मह्यं कुरुध्वं हितमात्मनः}


\twolineshloka
{ततोऽन्यथा कृच्छ्रगता भविष्यथ मयाऽर्दिताः}
{एवमुक्त्वा तु तान्वीरान्युयुधे कुरुपुङ्गवः}


\twolineshloka
{अत्वरावानसम्भ्रान्तः संक्रुद्धैर्विजिगीषुभिः}
{शतं शतसहस्राणि शराणां नतपर्वणाम्}


\threelineshloka
{मुमुचुः सैन्धवा राजंस्तदा गाण्डीवधन्वनि}
{शरानापततः क्रूरानाशीविषविषोपमान्}
{चिच्छेद निशितैर्बाणैरन्तरा स धनंजयः}


\twolineshloka
{छित्त्वा तु तानाशु चैव कङ्कपत्राञ्शिलाशितान्}
{एकैकमेषां समरे बिभेद निशितैः शरैः}


\twolineshloka
{ततः प्रास्रांश्च शक्तीस्च पुनरेव धनंजये}
{जयद्रथं हतं स्मृत्वा चिक्षिपुः सैन्धवा नृपाः}


\fourlineindentedshloka
{तेषां किरीटी सङ्कल्पं मोघं चक्रे महाबलः}
{सर्वांस्तानन्तरा च्छित्त्वा तदा चुक्रोश पाण्डवः}
{तथैवापततां तेषां योधानां जयगृद्धिनाम्}
{}


\twolineshloka
{तथैवापततां तेषां योधानां यजगृद्धिनाम्}
{शिरांसि पातयामास भल्लैः सन्नतपर्वभिः}


\twolineshloka
{तेषां प्रद्रवतां चापि पुनरेवाभिधावताम्}
{निवर्ततां च शब्दोऽभूत्पूर्णस्येव महोदधेः}


\twolineshloka
{ते वध्यमानास्तु तदा पार्थेनामिततेजसा}
{यथाप्राणं यथोत्साहं योधयामासुरर्जुनम्}


\twolineshloka
{ततस्ते फल्गुनेनाजौ शरैः सन्नतपर्वभिः}
{कृता विसंज्ञा भूयिष्ठा क्लान्तवाहनसैनिकाः}


\twolineshloka
{तांस्तु सर्वान्परिग्लानान्विदित्वा धृतराष्ट्रजा}
{दुःशला बालमादाय नप्तारं प्रययौ तदा}


\twolineshloka
{सुरथस्य सुतं वीरं रथेनाथागमत्तदा}
{शान्त्यर्थं सर्वयोधानामभ्यगच्छत पाण्डवम्}


\twolineshloka
{सा धनंजयमासाद्य रुरोदार्तस्वरं तदा}
{धनंजयोपि तां दृष्ट्वा धनुर्विससृजे प्रभुः}


\twolineshloka
{समुत्सृज्य धनुः पार्थो विधिवद्भगिनी तदा}
{प्राह किं करवाणीति सा च तं प्रत्युवाच ह}


\twolineshloka
{एष ते भरतश्रेष्ठ स्वस्त्रीयस्यात्मजः शिशुः}
{अभिवादयते पार्थ तं पश्य पुरुषर्षभ}


\twolineshloka
{इत्युक्तस्तस्य पितरं स पप्रच्छार्जुनस्तथा}
{क्वासाविति ततो राजन्दुःशला वाक्यमब्रवीत्}


\twolineshloka
{पितृशोकाभिसंतप्तो विषादार्तोऽस्य वै पिता}
{पञ्चत्वमगमद्वीरो यथा तन्मे निशामय}


\threelineshloka
{स पूर्वं पितरं श्रुत्वा हतं युद्धे त्वयाऽनघ}
{त्वामागतं च संश्रुत्य युद्धाय हयसारिणम्}
{पितुश्च मृत्युदुःखार्तोऽजहात्प्राणान्धनंजय}


\twolineshloka
{प्राप्तो बीभत्सुरित्येव नाम श्रुत्वैव तेऽनघ}
{विषादार्तः पपातोर्व्यां ममार च ममात्मजः}


\twolineshloka
{तं दृष्ट्वा पतितं तत्र ततस्तस्यात्मजं प्रभो}
{गृहीत्वा समनुप्राप्ता त्वामद्य शरणैपिणी}


\twolineshloka
{इत्युक्त्वाऽऽर्तस्वरं सा तु मुमोच धृतराष्ट्रजा}
{दीना दीनं स्थितं पार्थमब्रवीच्चाप्यधोमुखम्}


\threelineshloka
{स्वसारं समवेक्षस्व स्वस्रीयात्मजमेव च}
{कर्तुमर्हसि धर्मज्ञ दयां कुरुकुलोद्वह}
{विस्मृत्य कुरुराजानं तं च मन्दं जयद्रथम्}


\twolineshloka
{अभिमन्योर्यथा जातः परिक्षित्परवीरहा}
{तथाऽयं सुरथाज्जातो मम पौत्रो महाभुज}


\twolineshloka
{तमादाय नरव्याघ्र सम्प्रप्तास्मि तवान्तिकम्}
{शमार्थं सर्वयोधानां शृणु चेदं वचो मम}


\twolineshloka
{आगतोऽयं महाबाहो तस्य मन्दस्य पुत्रकः}
{प्रसादमस्य बालस्य तस्मात्त्वं कर्तुमर्हसि}


\twolineshloka
{एष प्रसाद्य शिरसा प्रशमार्थमरिंदम}
{याचते त्वां महाबाहो शमं गच्छ धनंजय}


\twolineshloka
{बालस्य हतबन्धोश्च पार्थ किञ्चिदजानतः}
{प्रसादं कुरु धर्मज्ञ मा मन्युवशमन्वगाः}


\twolineshloka
{तमनार्थं नृशंसं च विस्मृत्यास्य पितामहम्}
{आगस्कारिणमत्यर्तं प्रसादं कर्तुमर्हसि}


\threelineshloka
{एवं ब्रुवत्यां करुणं दुःशलायां धनंजयः}
{संस्मृत्य देवीं गान्धारीं धृतराष्ट्रं च पार्थिवम्}
{उवाच दुःखशोकार्तः क्षत्रधर्मं व्यगर्हयत्}


\twolineshloka
{`धिक्तं दुर्योधनं क्षुद्रं राज्यलुब्धं च मानिनम्}
{'यत्कृते बान्धवाः सर्वे मया नीता यमक्षयम्}


\twolineshloka
{इत्युक्त्वा बहु सान्त्वादि प्रसादमकरोज्जयः}
{परिष्वज्य च तां प्रीतो विससर्ज गृहान्प्रति}


\twolineshloka
{दुःशला चापि तान्योधान्निवार्य महतो रणात्}
{सम्पूज्य पार्थं प्रययौ गृहानेव शुभानना}


\twolineshloka
{एवं निर्जित्य तान्वीरान्सैन्धवान्स धनंजयः}
{अन्वधावत धावन्तं हयं कामविचारिणम्}


\twolineshloka
{ततो मृगमिवाकाशे यथा देवः पिनाकधृक्}
{ससार तं तथा वीरो विधिवद्यज्ञियं हयम्}


\twolineshloka
{स च वाजी यथेष्टेन तांस्तान्देशान्यथाक्रमम्}
{विचचार यथाकामं कर्म पार्थस्य वर्धयन्}


\twolineshloka
{क्रमेण स हयस्त्वेवं विचरन्पुरुषर्षभ}
{मणलूरपतेर्देशमुपायात्सहपाण्डवः}


\chapter{अध्यायः ७९}
\twolineshloka
{श्रुत्वा तु नृपतिः प्राप्तं पितरं बभ्रुवाहनः}
{निर्ययौ विनयेनाथ ब्राह्मणार्यपुरःसरः}


\twolineshloka
{मणलूरेश्वरं त्वेवमुपयातं धनंजयः}
{नाभ्यनन्दत्स मेधावी क्षत्रधर्ममनुस्मरन्}


\twolineshloka
{उवाच च स धर्मात्मा समन्युः फल्गुनस्तदा}
{प्रक्रियेयं न ते युक्ता बहिस्त्वं क्षत्रधर्मतः}


\twolineshloka
{संरक्ष्यमाणं तुरगं यौधिष्ठिरमुपागतम्}
{यज्ञियं विषयान्ते मां नायोत्सीः किंनु पुत्रक}


\twolineshloka
{धिक्त्वामस्तु सुदुर्बुद्धिं क्षत्रधर्माविशारदम्}
{यो मां युद्धाय सम्प्राप्तं साम्नैव प्रत्यगृह्णथाः}


\twolineshloka
{न त्वया पुरुषार्थो हि कश्चिदस्तीह जीवता}
{यस्त्वं स्त्रीवद्युधा प्राप्तं मां साम्ना प्रत्यगृह्णथाः}


\twolineshloka
{यद्यहं न्यस्तशस्त्रस्त्वामागच्छेयं सुदुर्मते}
{प्रक्रियेयं भवेद्युक्ता तावतव नराधम}


\twolineshloka
{तमेवमुक्तं भर्त्रा तु विदित्वा पन्नगात्मजा}
{अमृष्यमाणा भित्त्वोर्वीमुलूपी समुपागमत्}


\twolineshloka
{सा ददर्श तत पुत्रं विमृशन्तमधोमुखम्}
{संतर्ज्यमानमसकृत्पित्रा युद्धार्थिना विभो}


\twolineshloka
{ततः सा चारुसर्वाङ्गी समुपेत्योरगात्मजा}
{उलूपी प्राह वचनं क्षत्रधर्मविशारद}


\twolineshloka
{उलूपीं मां निबोध त्वं मातरं पन्नगात्मजाम्}
{कुरुष्व वचनं पुत्र धर्मस्ते भविता परः}


% Check verse!
युध्यस्वैनं कुरुश्रेष्ठं धनंजयमरिन्दमम् ॥एवमेष हि ते प्रीतो भविष्यति न संशयः
\twolineshloka
{एवमुद्धार्षितो राजा स मात्रा बभ्रुवाहनः}
{मनश्चक्रे महातेजा युद्धाय भरतर्षभ}


\twolineshloka
{सन्नह्य काञ्चनं वर्म शिरस्त्राणं च भानुमत्}
{तूणीरशतसंबाधमारुरोह रथोत्तमम्}


\twolineshloka
{सर्वोपकरणोपेतं युक्तमश्वैर्मनोजवैः}
{सचक्रोपस्करं श्रीमान्हेमभाण्डपरिष्कृतम्}


\twolineshloka
{परमार्चितमुच्छ्रित्य ध्वजं हंसं हिरण्मयम्}
{प्रययौ पार्थमुद्दिश्य स राजा बभ्रुवाहनः}


\twolineshloka
{ततोऽभ्योत्य हयं वीरो यज्ञियं पार्थरक्षितम्}
{ग्राहयामास पुरुषैर्हयशिक्षाविशारदैः}


\twolineshloka
{गृहीतं वाजिनं दृष्ट्वा प्रीतात्मा स धनंजयः}
{पुत्रं रथस्थं भूमिष्ठः संन्यवारयदाहवे}


\twolineshloka
{स तत्र राजा तं वीरं शरसङ्घैरनेकशः}
{अर्दयामास निशितैराशीविषविषोपमैः}


\twolineshloka
{तयोः समभवद्युद्धं पितुः पुत्रस्यक चातुलम्}
{देवासुररणप्रख्यमुभयोः प्रीयमाणयोः}


\twolineshloka
{किरीटिनं प्रविव्याघ शरेणानतपर्वणा}
{जत्रुदेशे नरव्याघ्रं प्रहसन्बभ्रुवाहनः}


\twolineshloka
{सोभ्यगात्सहपुङ्खेन वल्मीकमिव पन्नगः}
{विनिर्भद्य च कौन्तेयं प्रविवेशि महीतलम्}


\twolineshloka
{स गाढवेदनो धीमानालम्ब्य धनुरुत्तमम्}
{दिव्यं तेजः समाविश्य प्रमीत इव सोभवत्}


\twolineshloka
{स संज्ञामुपलभ्याथ प्रशस्य पुरुषर्षभः}
{पुत्रं शक्रात्मजो वाक्यमिदमाह महाद्युतिः}


\twolineshloka
{साधुसाधु महाबाहो वत्स चित्राङ्गदात्मज}
{सदृशं कर्म ते दृष्ट्वा प्रीतिमानस्मि पुत्रक}


\twolineshloka
{विमुञ्चाम्येष ते बाणान्पुत्र युद्धे स्थिरो भव}
{इत्येवमुक्त्वा नाराचैरभ्यवर्षदमित्रहा}


\twolineshloka
{तान्स गाण्डीवनिर्मुक्तान्वज्राशनिसमप्रभान्}
{नाराचानच्छिनद्राज भल्लैः सर्वांस्त्रिधा द्विधा}


\twolineshloka
{तस्य पार्थः शरैर्दिव्यैर्ध्वजं हेमपरिष्कृतम्}
{सुवर्णतालप्रतिमं क्षुरेणापाहरद्रथात्}


\twolineshloka
{हयांश्चास्य महाकायान्महावेगानरिंदम}
{चकार राजन्निर्जावान्प्रहसन्निव पाण्डवः}


\twolineshloka
{स रथादवतीर्याथ राजा परमकोपनः}
{पदातिः पितरं क्रुद्धो योधयामास पाण्डवम्}


\twolineshloka
{सम्प्रीयमाणः पार्थानामृषभः पुत्रविक्रमात्}
{नात्यर्थं पीडयामास पुत्रं वज्रधरात्मजः}


\twolineshloka
{स हन्यमानोऽभिमुखं पितरं बभ्रुवाहनः}
{शरैराशीविषाकारैः पुनरेवार्दयद्बली}


\twolineshloka
{ततः स बाल्यात्पितरं विव्याध हृदि पत्रिणा}
{निशेतेन सुपुङ्खेन बलवद्बभ्रुवाहनः}


\twolineshloka
{स बाणस्तेजसा दीप्तो ज्वलन्निव हुताशनः}
{विवेश पाण्डवं राजन्मर्म भित्त्वाऽतिदुःखकृत्}


\twolineshloka
{स तेनातिभृशं विद्धः पुत्रेण कुरुनन्दनः}
{महीं जगाम मोहार्तस्ततो राजन्धनंजयः}


\threelineshloka
{तस्मिन्निपतिते वीरे कौरवाणां धुरंधरे}
{सोपि मोहं जगामाथ ततश्चित्राङ्गदासुतः}
{ष}


\threelineshloka
{व्यायम्य संयुगे राजा दृष्ट्वा च पितरं हतम्}
{पूर्वमेव स बाणौर्घर्गाढविद्धोऽर्जुनेन ह}
{पपात सोपि धरणीमालिङ्ग्य रणमूर्धनि}


\twolineshloka
{भर्तारं निहतं दृष्ट्वा पुत्रं च पतितं भुवि}
{चित्राङ्गदा परित्रस्ता प्रविवेश रणाजिरे}


\twolineshloka
{शोकसंतप्तहृदया रुदती वेपती भृशम्}
{}


% Check verse!
मणलूरपतेर्माता ददर्श निहतं पतिम्
\chapter{अध्यायः ८०}
\twolineshloka
{ततो बहुतरं भीरर्विलप्य कमलेक्षणा}
{मुमोह दुःखसंतप्ता पपात च महीतले}


\twolineshloka
{प्रतिलभ्य च सा संज्ञां देवी दिव्यवपुर्धरा}
{उलूपीं पन्नगसुतां दृष्ट्वेदं वाक्यमब्रवीत्}


\twolineshloka
{उलूपि पश्य भर्तारं शयानं नितं रणे}
{त्वत्कृते मम पुत्रेण बाणेन समितिंजयम्}


\twolineshloka
{ननु त्वमार्यधर्मज्ञा ननु चासि पतिव्रता}
{यत्त्वत्कृतेऽयं पतितः पतिस्ते निहतो रणे}


\twolineshloka
{किंनु मन्देऽपकराद्धोऽयं यदि तेऽद्य धनंजयः}
{क्षमस्व याच्यमाना वै जीवयस्व धनंजयम्}


\twolineshloka
{ननु त्वमार्ये धर्मज्ञे त्रैलोक्यविदिता शुभे}
{यद्धातयित्वा पुत्रेण भर्तारं नानुशोचसि}


\twolineshloka
{नाहं शोचामि तनयं हतं पन्नगनन्दिनि}
{पतिमेव तु शोचामि यस्यातिथ्यमिदं कृतम्}


\twolineshloka
{इत्युक्त्वा सा तदा देवीमुलूपीं पन्नगात्मजाम्}
{भर्तारमभिगम्येदमित्युवाच यशस्विनी}


\twolineshloka
{उत्तिष्ठ कुरुमुख्यस्य प्रियमुख्य मम प्रिय}
{अयमश्वो महाबाहो मयो ते परिमोक्षितः}


\twolineshloka
{ननु त्वया नाम विभो धर्मराजस्य यज्ञियः}
{अयमश्वोऽनुसर्तव्यः स शेषे किं महीतले}


\twolineshloka
{त्वयि प्राणा ममायत्ताः कुरूणां कुरुनन्दन}
{स कस्मात्प्राणदोऽन्येषां प्राणान्संत्यक्तवानसि}


\twolineshloka
{उलूपि साधु पश्येमं पतिं निपतितं भुवि}
{पुत्रं चेमं समुत्साद्य घातयित्वा न शोचसि}


\twolineshloka
{कामं स्वपितु बालोऽयं भूमौ मृत्युवशं गतः}
{लोहिताक्षो गुडाकेशो विजयः साधु जीवतु}


\twolineshloka
{नापराधोऽस्ति सुभगे नराणां बहुभार्यता}
{प्रमदानां भवत्येष मा ते भूद्बुद्धिरीदृशी}


\twolineshloka
{सख्यं चैतत्कृतं धात्रा शश्वदव्ययमेव तु}
{सख्यं समभिजानीहि सत्यं सङ्गतमस्तु ते}


\twolineshloka
{पुत्रेम घातयित्वैनं पतिं यदि न मेऽद्य वै}
{जीवन्तं दर्शयस्यद्य परित्यक्ष्यामि जीवितम्}


\twolineshloka
{साऽहं दुःखान्विता देवि पतिपुत्रविनाकृता}
{इहैव प्रायमाशिष्ये प्रेक्षन्त्यास्ते न संशयः}


\twolineshloka
{इत्युक्त्वा पन्नगसुतां सपत्नी चैत्रवाहनी}
{ततः प्रायमुपासीना तूष्णीमासीज्जनाधिप}


\twolineshloka
{ततो विलप्य विरता भर्तुः पादौ प्रगृह्य सा}
{उपविष्टा भवद्दीना सोच्छ्वासं पुत्रमीक्षती}


\twolineshloka
{ततः संज्ञां पुनर्लब्ध्वा स राजा बभ्रुवाहनः}
{मातरं तामथालोक्य रणभूमावथाब्रवीत्}


\twolineshloka
{इतो दुःखतरं किंनु यन्मे माता सुखैधिता}
{भूमौ निपतितं वीरमनुशेते मृतं पतिम्}


\twolineshloka
{निहन्तारं रणेऽरीणां सर्वशस्त्रभृतां वरम्}
{मया विनिहतं सङ्ख्ये प्रेक्षते दुर्मरं बत}


\twolineshloka
{अहोऽस्या हृदयं देव्या दृढं यन्न विदीर्यते}
{व्यूढोरस्कं महाबाहुं प्रेक्षन्त्या निहतं पतिम्}


\twolineshloka
{दुर्मरं पुरुषेणेह मन्ये काले ह्यनागते}
{यत्र नाहं न मे माता न वियुक्तौ स्वजीवितात्}


\twolineshloka
{हाहा धिक्कुरुवीरस्य किरीटं काञ्चनं भुवि}
{अपविद्धं हतस्येह मया पुत्रेम पश्यत}


\twolineshloka
{भोभो पश्यत मे वीरं पितरं ब्राह्मणा भुवि}
{शयानं वीरशयने मया पुत्रेण पातितम्}


\twolineshloka
{ब्राह्मणाः कुरुमुख्यस्य ये मुक्ता हयसारिणः}
{कुर्वन्ति शान्तिं कामस्य रणे योऽयं मया हतः}


\twolineshloka
{व्यादिशन्तु च किं विप्राः प्रायश्चित्तमिहाद्य मे}
{आनृशंसस्य पापस्य पितृहन्तू रणाजिरे}


\twolineshloka
{दुश्चरा द्वादश समा हत्वा पितरमद्य वै}
{ममेह सुनृशंसस्य संवीतस्यास्य चर्मणा}


\twolineshloka
{शिरःकपाले चास्यैव भुञ्जतः पितुरद्य मे}
{प्रायश्चित्तं हि नास्त्यन्यद्धत्वाऽद्य पितरं मम}


\twolineshloka
{पश्य नागोत्तमसुते भर्तारं निहतं मया}
{कृतं प्रियं मया तेऽद्य निहत्य समरेऽर्जुनम्}


\twolineshloka
{सोऽहमद्य गमिष्यामि गतिं पितृनिषेविताम्}
{न शक्नोम्यात्मनाऽऽत्मानमहं दारयितुं शुभे}


\twolineshloka
{सा त्वं मयि मृते मातस्तथा गाण्डीवधन्वनि}
{भव प्रीतिमती देवि सत्येनात्मानमालभे}


\twolineshloka
{इत्युक्त्वा स ततो राजा दुःखशोकसमाहतः}
{उपस्पृश्य महाराज दुःखाद्वचनमब्रवीत्}


\twolineshloka
{शृण्वन्तु सर्वभूतानि स्थावराणि चराणि च}
{त्वं च मातर्यथा सत्यं ब्रवीमि भुजगोत्तमे}


\twolineshloka
{यदि नोत्तिष्ठति जयः पिता मे नरसत्तमः}
{अस्मिन्नेव रणोद्देशे शोषयिष्ये कलेवरम्}


\twolineshloka
{नहि मे पितरं हत्वा निष्कृतिर्विद्यते क्वचित्}
{नरकं प्रतिपत्स्यामि ध्रुवं गुरुवधार्दितः}


\twolineshloka
{वीरं हि क्षत्रियं हत्वा गोशतेन प्रमुच्यते}
{पितरं तु निहत्यैवं दुर्लभा निष्कृतिर्मम}


\twolineshloka
{एत एको महातेजाः पाण्डुपुत्रो धनंजयः}
{पिता च मम धर्मात्मा तस्य मे निष्कृतिः कुतः}


\twolineshloka
{इत्येवमुक्त्वा नृपते धनंजयसुतो नृपः}
{उपस्पृश्याभवत्तूष्णीं प्रायोपेतो महामतिः}


\chapter{अध्यायः ८१}
\twolineshloka
{प्रायोपविष्टे नृपतौ मणलूरेश्वरे तदा}
{पितृशोकसमाविष्टे सह मात्रा परंतप}


\twolineshloka
{उलूपी चिन्तयामास तदा संजीवनं मणिम्}
{स चोपातिष्ठत तदा पन्नगानां परायणम्}


\twolineshloka
{तं गृहीत्वा तु कौरव्य नागराजपतेः सुता}
{मनःप्रह्लादनीं वाचं सैनिकानामथाब्रवीत्}


\twolineshloka
{उत्तिष्ठ मा शुचः पुत्र नैव जिष्णुस्त्वया हतः}
{अजेयः पुरुषैरेष तथा देवैः सवासवैः}


\twolineshloka
{मया तु मोहनी नाम मायैषा सम्प्रदर्शिता}
{प्रियार्थं पुरुषेन्द्रस्य पितुस्तेऽद्य यशस्विनः}


\twolineshloka
{जिज्ञासुर्ह्येष पुत्रस्य बलस्य तव कौरव}
{सङ्ग्रामे युद्ध्यतो राजन्नागतः परवीरहा}


\twolineshloka
{तस्मादसि मया पुत्र युद्धाय परिचोदितः}
{मा पापमात्मनः पुत्र शङ्केथा ह्यण्वपि प्रभो}


\twolineshloka
{ऋषिरेष महानात्मा पुराणः शाश्वतोऽक्षरः}
{नैनं शक्तो हि सङ्ग्रामे जेतुं शक्रोऽपि पुत्रक}


\twolineshloka
{अयं तु मे मणिर्दिव्यः समानीतो विशांपते}
{मृतान्मृतान्पन्नगेन्द्रान्यो जीवयति नित्यदा}


\twolineshloka
{एनमस्योरसि त्वं च स्थापयस्व पितुः प्रभो}
{संजीवितं तदा पार्थं स त्वं द्रष्टासि पाण्डवम्}


\twolineshloka
{इत्युक्तः स्थापयामास तस्योरसि मणिं तदा}
{पार्थस्यामिततेजाः स पितुः स्नेहादपापकृत्}


\twolineshloka
{तस्मिन्न्यस्ते मणौ वीरो जिष्णुरुज्जीवितः प्रभुः}
{चिरसुप्त हवोत्तस्थौ मृष्टलोहितलोचनः}


\twolineshloka
{तमुत्थितं महात्मानं लब्धसंज्ञं मनस्विनम्}
{समीक्ष्य पितरं स्वस्थं ववन्दे बभ्रुवाहनः}


\twolineshloka
{उत्थिते पुरुषव्याघ्रे पुनर्लक्ष्मीवति प्रभो}
{दिव्याः सुमनसः पुण्या ववृषे पाकशासनः}


\twolineshloka
{अनाहता दुन्दुभयो विनेदुर्मघनिःस्वनाः}
{साधुसाध्विति चाकाशे बभूव सुमहान्स्वनः}


% Check verse!
उत्थाय च महाबाहुः पर्याश्वस्तो धनंजयः ॥बभ्रुवाहनमालिङ्ग्य समाजिघ्रत मूर्धनि
\twolineshloka
{ददर्श चापि दूरेऽस्य मातरं शोककर्शिताम्}
{उलूप्या सह तिष्ठन्तीं ततोऽपृच्छद्धनंजयः}


\twolineshloka
{किमिदं लक्ष्यते सर्वं शोकविस्मयहर्षवत्}
{रणाजिरममित्रघ्न यदि जानासि शंस मे}


\twolineshloka
{जननी च किमर्थं ते रणभूमिमुपागता}
{नागेन्द्रदुहिता चेयमुलूपी किमिहागता}


\twolineshloka
{जानाम्यहमिदं युद्धं त्वया मद्वचनात्कृतम्}
{स्त्रीणामागमने हेतुमहमिच्छामि वेदितुम्}


\twolineshloka
{तमुवाच तथा पृष्टो मणलूरपतिस्तदा}
{प्रसाद्य शिरसा विद्वानुलूपी पृच्छ्यतामिति}


\chapter{अध्यायः ८२}
\twolineshloka
{किमागमनकृत्यं ते कौरव्यकुलनन्दिनि}
{मणलूरपतेर्मातुस्तथैव च रणाजिरे}


\twolineshloka
{कच्चित्कुशलकामासि राज्ञोऽस्य भुजगात्मजे}
{मम वा चपलापाङ्गि कच्चित्वं शुभमिच्छसि}


\twolineshloka
{कच्चित्ते पृथुलश्रोणि नाप्रियं प्रियदर्शने}
{अकार्षमहमज्ञानादयं वा बभ्रुवाहनः}


\twolineshloka
{कच्चिन्नु राजपूत्री ते सपत्नी चैत्रवाहनी}
{चित्राङ्गदा वरारोहा नापराध्यति किञ्चन}


\threelineshloka
{तमुवाचोरगपतेर्दुहिता प्रहसन्त्यथ}
{न मे त्वमपराद्धोसि न हि मे बभ्रुवाहनः}
{न जनित्री तथाऽस्येयं मम यो प्रेष्यवत्थिता}


\twolineshloka
{श्रूयतां यद्यथा चेदं मया सर्वं विचेष्टितम्}
{न मे कोपस्त्वया कार्यः शिरसा त्वां प्रसादये}


\twolineshloka
{त्वत्प्रियार्थं हि कौरव्य कृतमेतन्मया विभो}
{यत्तच्छृणु महाबाहो निखिलेन धनंजय}


\twolineshloka
{महाभारतयुद्धे यत्त्वया शान्तनवो नृपः}
{अधर्मेण हतः पार्थ तस्यैषा निष्कृतिः कृता}


\twolineshloka
{न हि भीष्मस्त्वया वीर युद्ध्यमानो हि पातितः}
{शिखण्डिना तु संयुक्तस्तमाश्रित्य हतस्त्वया}


\threelineshloka
{तस्य शान्तिमकृत्वा त्वं त्यजेथा यदि जीवितम्}
{कर्मणा तेन पापेन पतेथा निरये ध्रुवम्}
{एषा तु विहिता शान्तिः पुत्राद्यां प्राप्तवानसि}


\twolineshloka
{वसुभिर्वसुधापाल गङ्गया च महामते}
{पुरा हि श्रुतमेतत्ते वसुभिः कथितं मया}


\threelineshloka
{गङ्गायास्तीरमाश्रित्य हते शान्तनवे नृप}
{आप्लुत्य देवा वसवः समेत्य च महानदीम्}
{इदमूचुर्वचो घोरं भागीरथ्या मते तदा}


\twolineshloka
{एष शान्तनवो भीष्मो निहतः सव्यसाचिना}
{अयुद्ध्यमानः सङ्ग्रामे संसक्तोऽन्येन भामिनि}


\twolineshloka
{तदनेनानुषङ्गेण वयमद्य धनञ्जयम्}
{शापेन योजयामेति तथाऽस्त्विति च साऽब्रवीत्}


\twolineshloka
{तदहं पितुरावेद्य प्रविश्य व्यथितेन्द्रिया}
{अभवं स च तच्छ्रुत्वा विषादमगमत्परम्}


\twolineshloka
{पिता तु मे वसून्गत्वा त्वदर्थे समयाचत}
{पुनः पुनः प्रसाद्यैतांस्त एनमिदमब्रुवन्}


\twolineshloka
{पुत्रस्तस्य महाभाग मणलूरेश्वरो युवा}
{स एनं रणमध्यस्थः शरैः पातयिता भुवि}


\twolineshloka
{एवं कृते स नागेन्द्र मुक्तशापो भविष्यति}
{गच्छेति वसुभिश्चोक्तो मम चेदं शशंस सः}


\twolineshloka
{तच्छ्रुत्वा त्वं मया तस्माच्छापादसि विमोक्षितः}
{न हि त्वां देवराजोऽपि समरेषु पराजयेत्}


\twolineshloka
{आत्मा पुत्रः स्मृतस्तस्मात्तेनेहासि पराजितः}
{न हि दोषो मम मतः कथं वा मन्यसे विभो}


\twolineshloka
{इत्येवमुक्तो विजयः प्रसन्नात्माऽब्रवीदिदम्}
{सर्वं मे सुप्रियं देवि यदेतत्कृतवत्यसि}


\twolineshloka
{इत्युक्त्वा सोऽब्रवीत्पुत्रं मणलूरपतिं जयः}
{चित्राङ्गदायाः शृण्वन्त्याः कौरव्यदुहितुस्तदा}


\twolineshloka
{युधिष्ठिरस्याश्वमेधः परिचैत्रीं भविष्यति}
{तत्रागच्छेः सहामात्यो मातृभ्यां सहितो नृप}


\twolineshloka
{इत्येवमुक्तः पार्थेन स राजा बभ्रुवाहनः}
{उवाच पितरं धीमानिदमस्राविलेक्षणः}


\twolineshloka
{उपयास्यामि धर्मज्ञ भवतः सासनादहम्}
{अश्वमेधे महायज्ञे द्विजातिपरिवेषकः}


\twolineshloka
{मम त्वनुग्रहार्थाय प्रविशस्व पुरं स्वकम्}
{भार्याभ्यां सह धर्मज्ञ माभूत्तेऽत्र विचारणा}


\twolineshloka
{उषित्वेह निशामेकां सुखं स्वभवने प्रभो}
{पुनरश्वानुगमनं कर्तासि जयतांवर}


\twolineshloka
{इत्युक्ताः स तु प्रत्रेण तदा वानरकेतनः}
{स्मयन्प्रोवाच कौन्तेयस्तदा चित्राङ्गदासुतम्}


\twolineshloka
{विदितं ते महाबाहो यथा दीक्षां चराम्यहम्}
{न स तावत्प्रवेक्ष्यामि पुरं ते पृथुलोचन}


\threelineshloka
{यथाकामं व्रजत्येष यज्ञियाश्वो नरर्षभ}
{स्वस्ति तेऽस्तु गमिष्यामि न स्थानं विद्यते मम ॥वैशम्पायन उवाच}
{}


\twolineshloka
{स तत्र विधिवत्तेन पूजितः पाकशासनिः}
{}


% Check verse!
भार्याभ्यामभ्यनुज्ञातः प्रायाद्भरतसत्तमः
\chapter{अध्यायः ८३}
\twolineshloka
{स तु वाजी समुद्रान्तां पर्येत्य वसुधामिमाम्}
{निवृत्तोऽभिमुखो राजन्येन वारणसाह्वयम्}


\twolineshloka
{अनुगच्छंश्च तुरगं निवृत्तोऽथ किरीटभृत्}
{यदृच्छया समापेदे पुरं राजगृहं तदा}


\twolineshloka
{तमभ्याशगतं दृष्ट्वा सहदेवात्मजः प्रभो}
{क्षत्रधर्मे स्थितो वीरः समरायाजुहाव ह}


\twolineshloka
{ततः पुरात्स निष्क्रम्य रथी धन्वी शरी तली}
{मेघसन्धिः पदातिं तं धनंजयमुपाद्रवत्}


\twolineshloka
{आसाद्य च महातेजा मेघसन्धिर्धनंजयम्}
{बालभावान्महाराज प्रोवाचेदं न कौशलात्}


\twolineshloka
{किमयं चार्यते वाजी स्इत्रीमध्य इव भारत}
{हयमेनं हरिष्यामि प्रयतस्व विमोक्षणे}


\twolineshloka
{अदत्तानुनयो युद्धे यदि त्वं पितृभिर्मम}
{करिष्यामि तवातिथ्यं प्रहर प्रहरामि च}


\twolineshloka
{इत्युक्तः प्रत्युवाचैनं प्रहसन्निव पाण्डवः}
{विघ्नकर्ता मया वार्य इति मे व्रतमाहितम्}


\twolineshloka
{भ्रात्रा ज्येष्ठेन नृपते तवापि विदितं ध्रुवम्}
{प्रहरस्व यथाशक्ति न मन्युर्विद्यते मम}


\twolineshloka
{इत्युक्तः प्राहरत्पूर्वं पाण्डवं मगधेस्वरः}
{किरञ्शरसहस्राणि वर्षाणीव सहस्रदृक्}


\twolineshloka
{ततो गाण्डीवभृच्छूरो गाण्डीवप्रहितैः शरैः}
{चकार मोघांस्तान्बाणान्सयत्नान्भरतर्षभ}


\twolineshloka
{स मोघं तस्य बाणौघं कृत्वा वानरकेतनः}
{शरात्मुमोच ज्वलितान्दीप्तास्यानिव पन्नगान्}


\twolineshloka
{ध्वजे पताकादण्डेषु रथे यन्त्रे हयेषु च}
{अन्येषु च रथाङ्गेषु न शरीरे न सारथौ}


\twolineshloka
{संरक्ष्यमाणः पार्थेन शरीरे सव्यसाचिना}
{मन्यमानः स्ववीर्यं तन्मागधः प्राहिणोच्छरान्}


\twolineshloka
{ततो गाण्डीवधन्वा तु मागधेन भृशाहतः}
{बभौ वसन्तसमये पलासः पुष्पितो यथा}


\twolineshloka
{अवध्यमानः सोऽभ्यघ्नन्मागधः पाण्डवर्षभम्}
{तेन तस्थौ स कौरव्य लोकवीरस्य दर्शने}


\twolineshloka
{सव्यसाची तु संक्रुद्धो विकृष्य बलवद्धनुः}
{हयांश्चकार निर्जीवान्सारथेश्च शिरोऽहरत्}


\twolineshloka
{धनुश्चास्य महच्चित्रं क्षुरेण प्रचकर्त ह}
{हस्तावापं पताकां च ध्वजं चास्यन्यपातयत्}


\twolineshloka
{स राजा व्यथितो व्यश्वो विधनुर्हतसारथिः}
{गदामादाय कौन्तेयमभिद्रद्राव वेगवान्}


\twolineshloka
{तस्यापतत एवाशु गदां हेमपरिष्कृताम्}
{शरैश्चकर्त बहुधा बहुभिर्गृध्रवाजितैः}


\twolineshloka
{सा गदा शकलीभूता विशीर्णिमणिबन्धना}
{व्याली विमुच्यमानेन पपात धरणीतले}


\twolineshloka
{विरथं विधनुष्कं च गदया परिवर्जितम्}
{`नैच्छत्ताडयितुं धीमानर्जुनः समराग्रणीः}


% Check verse!
तत एनं विमनसं क्षत्रधर्मे व्यवस्थितम् ॥'सान्त्वपूर्वमिदं वाक्यमब्रवीत्कपिकेतनः
\twolineshloka
{पर्याप्तः क्षत्रधर्मोऽयं दर्शितः पुत्र गम्यताम्}
{बह्वेतत्समरे कर्म तव बालस्य पार्थिव}


\twolineshloka
{युधिष्ठिरस्य संदेशो न हन्तव्या नृपा इति}
{तेन जीवसि राजंस्त्वमपराद्धोऽपि मे रणे}


\twolineshloka
{इति मत्वा तदाऽऽत्मानं प्रत्यादिष्टं स्म मागधः}
{तथ्यमित्यभिगम्यैनं प्राञ्जलिः प्रत्यपूजयत्}


\twolineshloka
{परिजितोस्मि भद्रं ते नाहं योद्धुमिहोत्सहे}
{यद्यत्कृत्यं मया तेऽद्य तद्ब्रूहि कृतमेव तु}


\twolineshloka
{तमर्जुनः समाश्वास्य पुनरेवेदमब्रवीत्}
{आगन्तव्यं परां चैत्रीमश्वमेधे नृपस्य नः}


\twolineshloka
{इत्युक्तः स तथेत्युक्त्वा पूजयमास तं हयम्}
{फल्गुनं च युधिश्रेष्ठं विदिवत्सहदेवजः}


\twolineshloka
{ततो यथेष्टमगमत्पुनरेव स कौरवः}
{ततः समुद्रतीरेण वङ्गान्पुण्ड्रान्सकेरलान्}


\twolineshloka
{तत्रतत्र च भूरीणि म्लेच्छसैन्यान्यनेकशः}
{विजिग्ये धनुषा राजन्गाण्डीवेन धनंजयः}


\chapter{अध्यायः ८४}
\twolineshloka
{मागधेनार्चितो राजन्पाण्डवः श्वेतवाहनः}
{दक्षिणां दिशमास्थाय चारयामास तं हयम्}


\twolineshloka
{ततः स पुनरावर्त्य हयः कामचरो बली}
{आससाद पुरीं रम्यां चेदीनां शुक्तिसाह्वयाम्}


\twolineshloka
{शरभेणार्चितस्तत्र शिशुपालसुतेन सः}
{युद्धपूर्वं तदा तेन पूजया च महाबलः}


\twolineshloka
{ततोऽर्चितो ययौ राजंस्तदा स तुरगोत्तमः}
{काशीनङ्गान्कोसलांश्च किरातानाथ तङ्गणात्}


\twolineshloka
{पूजां तत्र यथान्यायं प्रतिगृह्य धनंजयः}
{पुनरावृत्त्य कौन्तेयो दशार्णानगमत्तदा}


\twolineshloka
{तत्र चित्राङ्गदो नाम बलवानरिमर्दनः}
{तेन युद्धमभूत्तस्य विजयस्यातिभैरवम्}


\twolineshloka
{तं चापि वशमानीय किरीटी पुरुषर्षभः}
{निषादराज्ञो विषयमेकलव्यस्य जग्मिवान्}


\twolineshloka
{एकलव्यसुतश्चैनं युद्धेनि जगृहे तदा}
{तत्र चक्रे निषादैः स सग्रामं रोमहर्षणम्}


% Check verse!
ततस्तमपि कौन्तेयः समरेष्वपराजितः ॥जिगाय युधि दुर्धर्षो यज्ञविघ्नार्थमागतम्
\twolineshloka
{स तं जित्वा महाराज नैषादिं पाकशासनिः}
{अर्चितः प्रययौ भूयो दक्षिणं सलिलार्णवम्}


\twolineshloka
{तत्रापि द्रवीडैरान्ध्रै रौद्रैर्माहिषकैरपि}
{तथा कोल्लगिरेयैश्च युद्धमासीन्किरीटिनः}


\threelineshloka
{तांश्चापि विजयो जित्वा नातितीव्रेण कर्मणा}
{तुरङ्गमवशेनाथ सुराष्ट्रानभितो ययौ}
{गोकर्णमथ चासाद्य प्रभासमपि जग्मिवान्}


\twolineshloka
{ततो द्वारवतीं रम्यां वृष्णिवीराभिपालिताम्}
{आससाद हयः श्रीमान्कुरुराजस्य यज्ञियः}


\twolineshloka
{तमुन्मथ्य हयश्रेष्ठं यादवानां कुमारकाः}
{प्रययुस्तांस्तदा राजन्नुग्रसेनो न्यवारयत्}


\twolineshloka
{ततः पुराद्विनिष्क्रम्य वृष्ण्यन्धकपतिस्तदा}
{सहितो वासुदेवेन मातुलेन किरीटिनः}


\threelineshloka
{तौ समेत्य कुरुश्रेष्ठं विधिवत्प्रीतिपूर्वकम्}
{परया भारतश्रेष्ठं पूजया समवस्थितौ}
{ततस्ताभ्यामनुज्ञातो यतयौ येन हयो गतः}


\twolineshloka
{ततः स पश्चिमं देशं समुद्रस्य तदा हयः}
{क्रमेणि व्यचरत्स्फीतं ततः पञ्चनदं ययौ}


\twolineshloka
{तस्मादपि स कौरव्य गन्धारविषयं हयः}
{विचचार यथाकामं कौन्तेयानुगतस्तदा}


\twolineshloka
{ततो गान्धारराजेन युद्धमासीत्किरीटिनः}
{घोरं शकुनिपुत्रेण पूर्ववैरानुसारिणा}


\chapter{अध्यायः ८५}
\threelineshloka
{शकुनस्तनयो वीरो गान्धाराणां महारथः}
{प्रत्युद्ययौ गुडाकेशं सैन्येन महता वृतः}
{हस्त्यश्वरथयुक्तेन पताकाध्वजमालिना}


\twolineshloka
{अमृष्यमाणास्ते योधा नृपस्य शकुनेर्वधम्}
{अभ्ययुः सहिताः पार्थं प्रगृहीतशरासनाः}


\twolineshloka
{स तानुवाच धर्मात्मा बीभत्सुरपराजितः}
{युधिष्टिरस्य वचनं न च ते जगृहुर्हितम्}


\twolineshloka
{वार्यमाणास्तु पार्थेन सान्त्वपूर्वममर्षिताः}
{परिवार्य हयं जग्मुस्ततश्चुक्रोध पाण्डवः}


\twolineshloka
{ततः शिरांसि दीप्ताग्रैस्तेषां चिच्छेद पाण्डवः}
{क्षुरैर्गाण्डीवनिर्मुक्तैर्नातियत्नादिवार्जुनः}


\twolineshloka
{ते वध्यमानाः पार्थेनि हयमुत्सृज्य संप्रमात्}
{न्यवर्तन्त महाराज शरवर्षार्दिता भृशम्}


\twolineshloka
{वितुद्यमानस्तैश्चापि गान्धारैः पाण्डुनन्दनः}
{आदिस्यादिश्य तेजस्वी परानेतानवारयत्}


\twolineshloka
{वध्यमानेषु तेष्वाजौ गान्धारेषु समन्ततः}
{स राजा शकुनेः पुत्रः पाण्डवं प्रत्यवारयत्}


\twolineshloka
{तं युध्यमानं राजानं क्षत्रधर्मे व्यवस्थितम्}
{पार्थोऽब्रवीन्न मे वध्या राजानो राजसासनात्}


\threelineshloka
{अलं युद्धेन ते वीर न तेऽस्त्वद्य पराजयः}
{इत्युक्तस्तदनादृत्य वाक्यमज्ञानमोहितः}
{स शक्रसमकर्माणं समावाकिरदाशुगैः}


\twolineshloka
{तस्य् पार्थः शिरस्त्राणमर्धचन्द्रेण पत्रिणा}
{अपाहरदमेयात्मा जयद्रथशिरो यथा}


\twolineshloka
{तं दृष्ट्वा विस्मयं जग्मुर्गान्धाराः सर्व एव ते}
{इच्छता तेन न हतो राजेत्यपि च ते विदुः}


\threelineshloka
{गान्धारराजपुत्रस्तु पलायनकृतक्षणः}
{ययौ तैरेव सहितस्त्रस्तैः क्षुद्रमृगैरिव}
{}


\twolineshloka
{तेषां तु तरसा पार्थस्तत्रैव परिदावताम्}
{प्रजहारोत्तमाङ्गानि भल्लैः सन्नतपर्वभिः}


\twolineshloka
{उच्छ्रितांस्तु भुजान्केचिन्नाबुध्यन्त शरैर्हृतान्}
{शरैर्गाण्डीवनिर्मुक्तैः पृथुभिः पार्थचोदितैः}


\twolineshloka
{सम्भ्रान्तनरनागाश्वमपतद्विद्रुतं बलम्}
{हतविद्रुतभूयिष्ठमावर्तत मुहुर्मुहुः}


\twolineshloka
{नाभ्यदृस्यन्त वीरस्य केचिदग्रेऽग्र्यकर्मणः}
{रिपवः पात्यमाना वै ये सहेरन्महाशरान्}


\twolineshloka
{ततो गान्धारराजस्य मन्त्रिवृद्धपुरःसरा}
{जननी निर्ययौ भीता पुरस्कृत्यार्घ्यमुत्तमम्}


\twolineshloka
{सा न्यवारयदव्यग्रा तं पुत्रं युद्धदुर्मदम्}
{प्रसादयामास च तं जिष्णुमक्लिष्टकारिणम्}


\twolineshloka
{तां पूजयित्वा बीभत्सुः प्रसादमकरोत्प्रभुः}
{शकुनेश्चापि तनयं सान्त्वयन्निदमब्रवीत्}


\twolineshloka
{न मे प्रियं महाबाहो यत्ते बुद्धिरियं कृता}
{प्रतियोद्धुममित्रघ्न भ्रातैव त्वं ममानघ}


\twolineshloka
{गान्धारीं मातरं स्मृत्वा धृतराष्ट्रकृतेन च}
{तेन जीवसि राजंस्त्वं निहतास्त्वनुगास्तव}


% Check verse!
मैवं भूः शाम्यतां वैरं मा तेऽभूद्बुद्धिरीदृशीआगन्तव्यं परां चैत्रीमश्वमेधे नृपस्य नः
\twolineshloka
{इत्युक्त्वाऽनुययौ पार्थो हयं तं कामचारिणम्}
{ते न्यवर्तन्त गान्धारा हतशिष्टाः स्वकं पुरम्}


\chapter{अध्यायः ८६}
\threelineshloka
{न्यवर्तत ततो वाजी येन नागाह्वयं पुरम्}
{तं निवृत्तं तु शुश्राव चारेणैव युधिष्ठिरः}
{श्रुत्वाऽर्जुनं कुशलिनं स च हृष्टमनाऽभवत्}


\twolineshloka
{विजयस्य च तत्कर्म गान्धारविषये तदा}
{श्रुत्वा चान्येषु देशेषु स सुप्रीतोऽभवत्तदा}


\twolineshloka
{एतस्मिन्नेव काले तु द्वादशीं माघमासिकीम्}
{इष्टं गृहीत्वा नक्षत्रं धर्मराजो युधिष्ठिरः}


\twolineshloka
{समानीय महातेजाः सर्वान्भ्रातॄन्महीपतिः}
{भीमं च नकुलं चैव सहदेवं च कौरव}


\twolineshloka
{प्रोवाचेदं वचः काले तदा धर्मभृतांवरः}
{आमन्त्र्य वदतां श्रेष्ठो भीमं प्रहरतां वरम्}


\twolineshloka
{आयाति भीमसेनासौ सहाश्वेन तवानुजः}
{यथा मे पुरुषाः प्राहुर्ये धनंजयसारिणः}


\twolineshloka
{उपस्थितश्च कालोऽयमभितो वर्तते हयः}
{माघी च पौर्णमासीयं मासः शेषो वृकोदर}


\twolineshloka
{तत्प्रस्थाप्यन्तु विद्वांसो ब्राह्मणा वेदपारगाः}
{वाजिमेधार्थसिद्ध्यर्थं देशं पश्यन्तु यज्ञियम्}


\twolineshloka
{इत्युक्तः स तु तच्चक्रे भीमो नृपतिशासनम्}
{हृष्टः श्रुत्वा गुडोकेशमायान्तं पुरुषर्षभम्}


\twolineshloka
{ततो ययौ भीमसेनः प्राज्ञैः स्थपतिभिः सह}
{ब्राह्मणानग्रतः कृत्वा कुशलान्यज्ञकर्मणि}


\twolineshloka
{तं ससालचयं श्रीमत्सप्रतोलीसुघट्टितम्}
{मापयामास कौरव्यो यज्ञवाटं यताविधि}


\threelineshloka
{प्रासादशतसम्बन्धं मणिप्रवरकुट्टिमम्}
{`सदः स पत्नीसदनं साग्नीध्रमपि चोत्तरम्}
{'कारयामास विदिवद्धेमरत्नविभूषितम्}


\twolineshloka
{स्तंभान्कनकचित्रांश्च तोरणानि बृहन्ति च}
{यज्ञायतनदेशेशु दत्त्वा शुद्धं च काञ्चनम्}


\twolineshloka
{अन्तःपुराणां राज्ञां च नानादेशसमीयुषाम्}
{कारयामास धर्मात्मा तत्रतत्र यथाविधि}


\twolineshloka
{ब्राह्मणानां न वेश्मानि नानादेशसमीयुषाम्}
{कारयामास कौन्तेयो विधिवत्तान्यनेकशः}


\twolineshloka
{तथा सम्प्रेषयामास दूतान्नृपतिशासनात्}
{भीमसेनो महाबाहो राज्ञामक्लिष्टकर्मणाम्}


\twolineshloka
{ते प्रियार्थं कुरुपतेराययुर्नृपसत्तम}
{रत्नान्यनेकान्यादाय स्त्रियोऽश्वानायुधानि च}


\twolineshloka
{तेषां निर्विशतां तेषु शिबिरेषु महात्मनाम्}
{नर्दतः सागरस्येव दिवस्पृगभवत्स्वनः}


% Check verse!
`प्रत्युद्गम्य नमस्कृत्य ब्राह्मणांश्च न्यवेदयत् ॥'
\twolineshloka
{तेषामभ्यागतानां च स राजा कुरुवर्धनः}
{व्यादिदेशान्नपानानि शय्याश्चाप्यतिमानुषाः}


\twolineshloka
{वाहनानां च विविधाः शालाः शालीक्षुगोरसैः}
{उपेता भरतश्रेष्ठो व्यादिदेश स धर्मराट्}


\threelineshloka
{`वर्णाः पृथक्सन्निविष्टा ह्युत्तरोत्तरपूजिताः'}
{तथा तस्मिन्महायज्ञे धर्मराजस्य धीमतः}
{समाजग्मुर्मुनिगणा बहवो ब्रह्मवादिनः}


\twolineshloka
{ये च द्विजातिप्रवरास्तत्रासन्पृथिवीपते}
{समाजग्मुः सशिष्यांस्तान्प्रतिजग्राह कौरवः}


\twolineshloka
{सर्वांश्च ताननुययौ यावदावसथान्प्रति}
{स्वयमेव महातेजा दंभं त्यक्त्वा युधिष्ठिरः}


\twolineshloka
{ततः कृत्वा स्थपतयः शिल्पिनोऽन्ये च ये तदा}
{कृत्स्नं यज्ञविधिं राजन्धर्मराजे न्यवेदयन}


\twolineshloka
{तच्छ्रुत्वा धर्मराजस्तु कृतं सर्वमतन्द्रितः}
{हृष्टरूपोऽभवद्राजन्सह भ्रातृभिरादृतः}


\twolineshloka
{तस्मिन्यज्ञे प्रवृत्ते तु वाग्मिनो हेतुवादिनः}
{हेतुवादान्बहूनाहुः परस्परजिगीषवः}


\twolineshloka
{ददृशुस्तं नृपतयो यज्ञस्य विधिमुत्तमम्}
{देवेन्द्रस्येव विहितं भीमसेनेन भारत}


\twolineshloka
{ददृशुस्तोरणान्यत्र शातकुंभमयानि ते}
{शय्यासनविहारांश्च सुबहून्रत्नसंचयान्}


\twolineshloka
{घटान्पात्रीः कटाहानि कलशान्वर्धमानकान्}
{न हि किञ्चिदसौवर्णमपश्यन्वसुधाधिपाः}


\twolineshloka
{यूपांश्च शास्त्रपठितान्दारवान्हेमभूषितान्}
{उपक्लृप्तान्यथाकालं विधिवद्भूरिवर्चसः}


\twolineshloka
{स्थलजाक जलजा ये च पशवः केचन प्रभो}
{सर्वानेव समानीतानपश्यंस्तत्र ते नृपाः}


\twolineshloka
{गाश्चैव महिषीश्चैव तथा वृद्धस्त्रियोपि च}
{औदकानि च सत्वानि श्वापदानि वयांसि च}


\twolineshloka
{जरायुजाण्डजातानि स्वेदजान्युद्भिदानि च}
{पर्वतानूपजातानि भूतानि ददृशुश्च ते}


\twolineshloka
{एवं प्रमुदितं सर्वं पशुगोधनधान्यतः}
{यज्ञवाटं नृपा दृष्ट्वा परं विस्मयमागताः}


% Check verse!
`अनिशं दीयते च स्म तत्र भोज्यं पृथग्विधम् ॥'ब्राह्मणानां विशां चैव बहुमृष्टान्नमृद्धिमत्
\threelineshloka
{पूर्णे शतसहस्रे तु विप्राणां तत्रि भुञ्जताम्}
{दुन्दुभिर्मेघनिर्घोषो मुहुर्मुहुरताड्यत}
{विननादासकृच्चापि दिवसेदिवसे गते}


\threelineshloka
{एवं स ववृते यज्ञो धर्मराजस्य धीमतः}
{अन्नस्य सुबहून्राजन्नुत्सर्गान्पर्वतोपमान्}
{दधिकुल्याशअच ददृशुः सर्पिषश्च ह्रदाञ्जनाः}


\twolineshloka
{जंबूद्वीपो हि सकलो नानजनपदायुतः}
{राजन्नदृस्यतैकस्थो राज्ञस्तस्य महामखे}


\twolineshloka
{तत्र जातिसहस्राणि पुरुषाणां ततस्ततः}
{गृहीत्वा धमाजग्मुर्बहूनि भरतर्षभ}


\twolineshloka
{स्रग्विणश्तापि ते सर्वे सुमुष्टमणिकुण्डलाः}
{पर्यवेषन्द्विजातींस्ताञ्शतशोऽथ सहस्रशः}


\twolineshloka
{विविधान्यन्नपानानि पुरुषा येऽनुयायिनः}
{ते वै नृपोपभोज्यानि ब्राह्मणानां ददुश्च ह}


\chapter{अध्यायः ८७}
\twolineshloka
{समागतान्वेदविदो राज्ञश्च पृथिवीश्वरान्}
{दृष्ट्वा युधिष्ठिरो राजा भीमसेनमभाषत}


\twolineshloka
{उपयाता नरव्याघ्रा य एते पृथिवीश्वराः}
{एतेषां क्रियतां पूजा पूजार्हा हि नराधिपाः}


\twolineshloka
{इत्युक्तः स तथा चक्रे नरेन्द्रेण यशस्विना}
{भीमसेनो महातेजा यमाभ्यां सह पाण्डवः}


\twolineshloka
{अथाभ्यगच्छद्गोविन्दो वृष्णिभिः सह धर्मजम्}
{बलदेवं पुरस्कृत्य सर्वप्राणभूतां वरः}


\twolineshloka
{युयुधानेन सहितः प्रद्युम्नेन गदेन च}
{निशठेनाथ सांबेन तथैव कृतवर्मणा}


\twolineshloka
{तेषामपि परां पूजां चक्रे भीमो महारथः}
{विविशुस्ते च वेश्मानि रत्नवन्ति च सर्वशः}


\twolineshloka
{युधिष्ठिरसमीपे तु कथान्ते मधुसूदनः}
{अर्जुनं कथयामास बहुसङ्ग्रामकर्शितम्}


\twolineshloka
{स तं प्रपच्छ कौन्तेयः पुनःपुनररिंदमम्}
{धर्मजः शक्रजं जिष्णुं समाचष्ट जगत्पतिः}


\twolineshloka
{आगमद्द्वारकावासी समाप्तः पुरुषो नृप}
{योऽद्राक्षीत्पाण्डवश्रेष्ठ बहुसङ्ग्रामकर्शितम्}


\twolineshloka
{समीपे च महाबाहुमाचष्ट च मम प्रभो}
{कुरु कार्याणि कौन्तेय हयमेधार्थसिद्धये}


\twolineshloka
{इत्युक्तः प्रत्युवाचैनं धर्मराजो युधिष्ठिरः}
{दिष्ट्या स कुशली जिष्णुरुपायाति च माधव}


\twolineshloka
{यदिदं संदिदेशास्मिन्पाण्डवानां बलाग्रणीः}
{तदाख्यातमिहेच्छामि भवता यदुनन्दन}


\twolineshloka
{इत्युक्तो धर्मराजेन वृष्ण्यन्धकपतिस्तदा}
{प्रोवाचेदं वचो वाग्मी धर्मात्मानं युधिष्ठिरम्}


\twolineshloka
{इदमाह महाराज पार्थवाक्यं नरेश्वरः}
{वाच्यो युधिष्ठिरः कृष्ण काले वाक्यमिदं मम}


\twolineshloka
{आगमिष्यन्ति राजानः सर्वे वै कौरवर्षभ}
{प्राप्तानां महतां पूजा कार्या ह्येतत्क्षमं हि नः}


\twolineshloka
{इत्येतद्वचनाद्राजा विज्ञाप्यो मम मानद}
{तथा चात्ययिकं न स्याद्यदर्घाहरणेऽभवत्}


\twolineshloka
{कर्तुमर्हति तद्राजा भवांश्चाप्यनुमन्यताम्}
{राजद्वेषान्न नश्येयुरिमा राजन्पुनः प्रजाः}


\twolineshloka
{इदमन्यच्च कौन्तेय वचः स पुरुषोऽब्रवीत्}
{धनंजयस्य नृपते तन्मे निगदतः शृणु}


\twolineshloka
{उपयास्यति यज्ञं नो मणलूरपतिर्नृपः}
{पुत्रो मम महातेजा दयितो बभ्रुवाहनः}


\twolineshloka
{तं भवान्मदपेक्षार्थं विधिवत्प्रतिपूजयेत्}
{स तु भक्तोऽनुरक्तश्च मम नित्यमिति प्रभो}


\twolineshloka
{इत्येतद्वचनं श्रुत्वा धर्मराजो युधिष्ठिरः}
{अभिनन्द्यास्य तद्वाक्यमिदं वचनमब्रवीत्}


\chapter{अध्यायः ८८}
\twolineshloka
{श्रुतं प्रियमिदं कृष्ण यत्त्वमर्हसि भाषितुम्}
{तन्मेऽमृतरसं पुण्यं मनो ह्लादयति प्रभो}


\twolineshloka
{बहूनि किल युद्धानि विजयस्य नराधिपैः}
{पुनरासन्हृषीकेश तत्रतत्रेति न श्रुतम्}


\twolineshloka
{किंनिमित्तं स नित्यं हि पार्थः सुखविवर्जितः}
{अतीव विजयो धीमानिति मे दूयते मनः}


\twolineshloka
{संचिन्तयामि कौन्तेयं रहो जिष्णुं जनार्दन}
{अतीव दुःखभागी स सततं पाण्डुनन्दनः}


\twolineshloka
{किंनु तस्य शरीरेऽस्ति सर्वलक्षणपूजिते}
{अनिष्टं लक्षणं कृष्ण येन दुःखान्युपाश्नुते}


\threelineshloka
{अतीवासुखभोगी स सततं कुन्तिनन्दनः}
{न हि पश्यामि बीभत्सोर्निन्द्यं गात्रेषु किंचन}
{श्रोतव्यं चेन्मयैतद्वै तन्मे व्याख्यातुमर्हसि}


\twolineshloka
{इत्युक्तः स हृषीकेशो ध्यात्वा सुमहदन्तरम्}
{राजानं भोजराजन्यवर्धनो विष्णुरब्रवीत्}


\twolineshloka
{न ह्यस्य नृपते किञ्चिदनिष्टमुपलक्षये}
{ऋते पुरुषसिंहस्य पिण्डिकेऽस्याधिके यतः}


\twolineshloka
{स ताभ्यां पुरुषव्याघ्रो नित्यमध्वसु वर्तते}
{न चान्यदनुपश्यामि येनासौ दुःखभाजनम्}


\twolineshloka
{इत्युक्तः पुरुषश्रेष्ठस्तदा कृष्णेन धीमता}
{प्रोवाच वृष्णिशार्दूलमेवमेतदिति प्रभो}


\twolineshloka
{कृष्णा तु द्रौप्दी कृष्णं तिर्यक्सासूयमैक्षतप्रतिजग्राह तस्यास्तं प्रणयं चापि केशिहा}
{प्रख्युः सखा हृषीकेशः साक्षादिव धनंजयः}


\twolineshloka
{तत्र भीमादयस्ते तु करवो याजकाश्च ये}
{रेमुः श्रुत्वा विचित्रां तां धनंजयकथां शुभाम्}


\twolineshloka
{तेषां कथयतामेव पुरुषोऽर्जुनसंकथाः}
{उपायाद्वचनाद्दूतो विजयस्य महात्मनः}


\twolineshloka
{सोभिगम्य कुरुश्रेष्ठं नमस्कृत्य च बुद्धिमान्}
{उपायातं नरव्याघ्रं फल्गुनं प्रत्यवेदयत्}


\twolineshloka
{तच्छ्रुत्वा नृपतिस्तस्य हर्षबाष्पाकुलेक्षणः}
{प्रियाख्याननिमित्तं वै ददौ बहुधनं तदा}


\twolineshloka
{ततो द्वितीये दिवसे महाञ्शब्दो व्यवर्धत}
{आगच्छति नरव्याघ्रे कौरवाणां धुरंधरे}


\twolineshloka
{ततो रेणुः समुद्भुतो विबभौ तस्य वाजिनः}
{अभितो वर्तमानस्य यथोच्चैःश्रवसस्तथा}


\twolineshloka
{तत्र हर्षकरीर्वाचो नराणां शुश्रुवेऽर्जुनः}
{दिष्ट्याऽसि पार्थ कुशली धन्यो राजा युधिष्ठिरः}


% Check verse!
कोन्योहि पृथिवीं कृत्स्नां जित्वाहि युधि पार्थिवान्चारयित्वा हयश्रेष्ठमुपागच्छेदृतेऽर्जुनात्
\twolineshloka
{ये व्यतीता महात्मानो राजानः सगरादयः}
{तेषामपीदृशं कर्म न कदाचन शुश्रुम}


\twolineshloka
{नैतदन्ये करिष्यन्ति भविष्या वसुधाधिपाः}
{यत्त्वं कुरुकुलश्रेष्ठ दुष्करं कृतवानसि}


\twolineshloka
{इत्येवं वदतां तेषां पुंसां कर्णसुखा गिरः}
{शृण्वन्विवेश धर्मात्मा फल्गुनो यज्ञसंस्तरम्}


\twolineshloka
{ततो राजा सहामात्यः कृष्णश्च यदुनन्दनः}
{धृतराष्ट्रं पुरस्कृत्य तं प्रत्युद्ययतुस्तदा}


\twolineshloka
{सोऽभिवाद्य पितुः पादौ धर्मराजस्य धीमतः}
{भीमादींश्चापि संपूज्य पर्यष्वजत केशवम्}


\twolineshloka
{तैः समेत्यार्चितस्तांश्च प्रत्यर्च्याथ यथाविधि}
{विशश्राम महाबाहुस्तीरं लब्ध्वेव पारगः}


\chapter{अध्यायः ८९}
\twolineshloka
{एतस्मिन्नेव काले तु स राजा बभ्रुवाहनः}
{मातृभ्यां सहितो धीमान्कुरूनभ्याजगाम ह}


\threelineshloka
{तत्र वृद्धान्यथावत्स कुरूनन्यांश्च पार्थिवान्}
{अभिवाद्य महाबाहुस्तैश्चापि प्रतिनन्दितः}
{प्रविवेश पितामह्याः कुन्त्या भवनमुत्तमम्}


\twolineshloka
{स प्रविश्य महाबाहुः पाण्डवानां निवेशनम्}
{पितामहीमभ्यवन्दत्साम्ना परमवल्गुना}


\threelineshloka
{तथा चित्राङ्गदा देवी कौरवस्यात्मजाऽपि च}
{पृथां कृष्णां च सहिते विनयेनोपजग्मतुः}
{सुभद्रां च यथान्यायं याश्चान्याः कुरुयोषितः}


\twolineshloka
{ददौ कुन्ती ततस्ताभ्यां रत्नानि विविधानि च}
{द्रौपदी च सुभद्रा च याश्चाप्यन्या यदुस्त्रियः}


\twolineshloka
{ऊषतुस्तत्र ते देव्यौ महार्ङशयनासने}
{सुपूजिते स्वयं कुन्त्या पार्थस्य हितकाम्यया}


\twolineshloka
{स च राजा महातेजाः पूजितो बभ्रुवाहनः}
{धूतराष्ट्रं महीपालमुपतस्थे यताविधि}


\twolineshloka
{युधिष्टिरं च राजानं भीमदींश्चापि पाण्डवान्}
{उपागम्य महातेजा विनयेनाभ्यवादयत्}


\twolineshloka
{स तैः प्रेम्या परिष्वक्तः पूजितश्च यथाविधि}
{धनं चास्मै ददुर्भूरि प्रीयमाणा महारथाः}


\twolineshloka
{तथैव च महीपालः कृष्णं चक्रगदाधरम्}
{प्रद्युम्न इव गोविन्दं विनयेनोपतस्थिवान्}


\twolineshloka
{तस्मै कृष्णो ददौ राज्ञे महार्हमतिपूजितम्}
{रथं हेमपरिष्कारं दिव्याश्वयुजमुत्तमम्}


\twolineshloka
{धर्मराजश्च भीमश्च फल्गुनश्च यमौ तथा}
{पृथक्पृथक् च ते चैनं मानार्थाभ्यामयोजयन्}


\twolineshloka
{ततस्तृतीये दिवसे सत्यवत्यात्मजो मुनिः}
{युधिष्ठिरं समभ्येत्य वाग्मी वचनमब्रवीत्}


\twolineshloka
{अद्यप्रभृति कौन्तेय यज्ञस्य समयो हि ते}
{मुहूर्तो यज्ञियः प्राप्तश्चोदयन्तीह याजकाः}


\twolineshloka
{अहीनो नाम राजेन्द्र क्रतुस्तेऽयं विकल्पवान्}
{बहुत्वात्काञ्चनस्यास्य ख्यातो बहुसुवर्णकः}


\twolineshloka
{एवमत्र महाराज दक्षिणाभिर्गुणीकुर}
{श्रीस्त्वां व्रजतु ते राजन्ब्राह्मणा ह्यत्र कारणम्}


\twolineshloka
{त्रीनश्वमेधानत्र त्वं सम्प्राप्य बहुदक्षिणान्}
{ज्ञातिवध्याकृतं पापं प्रहास्यति नराधिप}


\twolineshloka
{पवित्रं परमं चैतत्पावनानां च पावनम्}
{यदश्वमेधावभृथं प्राप्स्यसे कुरुनन्दन}


\twolineshloka
{इत्युक्तः स तु तेजस्वी व्यासेनामितबुद्धिना}
{दीक्षां विवेश धर्मात्मा वाजिमेधाप्तये ततः}


\twolineshloka
{ततो यज्ञं महाबाहुर्वाजिमेधं महाक्रतुम्}
{बह्वन्नदक्षिणं राजा सर्वकामगुणान्वितम्}


\twolineshloka
{तत्र वेदविदो राजंश्चक्रुः कर्माणि याजकाः}
{परिक्रामन्ति शास्त्रज्ञा यतावद्द्विजसत्तमाः}


\twolineshloka
{न तेषां स्खलितं किञ्चिदासीदपहुतं तथा}
{क्रमयुक्तं च युक्तं च चक्रुस्तत्र द्विजर्षभाः}


\twolineshloka
{कृत्वा प्रवर्ग्यं धर्मज्ञा यथावद्द्विजसत्तमाः}
{चक्रस्ते विधिवद्राजंस्तथैवाभिषवं द्विजाः}


\twolineshloka
{अभिषूय ततो राजन्सोमं सोमपसत्तमाः}
{सवनान्यानुपूर्व्येण चक्रुः सास्त्रानुसारिणः}


\twolineshloka
{न तत्र कृपणः कश्चिन्न दरिद्रो बभूव ह}
{क्षुधितो दुःखितो वाऽपि प्राकृतो वाऽपि मानवः}


\twolineshloka
{भोजनं भोजनार्थिभ्यो दापयामास शत्रुहा}
{भीमसेनो महातेजाः सततं राजशासनात्}


\twolineshloka
{संस्तरे कुशलाश्चापि सर्वकार्याणि याजकाः}
{दिवसेदिवसे चक्रुर्यथाशास्त्रानुदर्शात्}


\twolineshloka
{नाषडङ्गविदत्रासीत्सदस्यस्तस्य धीमतः}
{नाव्रतो नानुपाध्यायो न च वादाविचक्षणः}


\twolineshloka
{ततो यूपोच्छ्रये प्राप्ते षड् बैल्वान्भरतर्षभ}
{खादिरान्बिल्वसमितांस्तावतः सर्ववर्णिनः}


\twolineshloka
{देवदारुमयौ द्वौ तु यूपौ कुरुपतेर्मखे}
{श्लेष्मातकमयं चैकं याजकाः समकल्पयन्}


\threelineshloka
{`सर्वानेतान्यथाशास्त्रं याजकाः समकारयन्}
{'शोभार्थं चापरान्यूपान्काञ्चनान्भरतर्षभ}
{स भीमः कारयामास धर्मराजस्य शासनात्}


\twolineshloka
{ते व्यराजन्त राजर्षे वासोभिरुपशोभिताः}
{महेन्द्रानुगता देवा यथा सप्तर्षिभिर्दिवि}


\twolineshloka
{इष्टकाः काञ्चनीश्चात्र चयनार्तं कृता विभो}
{शुशुभे चयनं तच्च दक्षस्येव प्रजापतेः}


\twolineshloka
{चतुश्चित्यश्च तस्यासीदष्टादशकरात्मकः}
{स रुक्मपक्षो निचितस्त्रिकोणो गरुडाकृतिः}


\twolineshloka
{ततो नियुक्ताः पशवो यथाशास्त्रं मनीषिभिः}
{तं तं देवं समुद्दिश्य पक्षिणः पशवश्च ये}


\twolineshloka
{ऋषभाः शास्त्रपठितास्तथा जलचराश्च ये}
{सर्वांस्तानभ्ययुञ्जंस्ते तत्राग्निचयकर्मणि}


\twolineshloka
{यूपेषु नियता चासीत्पशूनां त्रिशती तथा}
{अश्वरत्नोत्तरा यज्ञे कौन्तेयस्य महात्मनः}


\twolineshloka
{स यज्ञः शुशुभे तस्य साक्षाद्देवर्षिसंकुलः}
{गन्धर्वगणसंकीर्णः शोभितोऽप्सरसां गणैः}


\twolineshloka
{स किंपुरुषसंकीर्णः किंनरैश्चोपशोभितः}
{सिद्धविप्रनिवासैश्च समन्तादभिसंवृतः}


\twolineshloka
{तस्मिन्सदसि नित्यास्तु व्यासशिष्या द्विजर्षभाः}
{सर्वशास्त्रप्रणेतारः कुशला यज्ञकर्मसु}


\twolineshloka
{नारदश्च बभूवात्र तुंबुरुश्च महाद्युतिः}
{विश्वावसुश्चित्रसेनस्तथाऽन्ये गीतकोविदाः}


\twolineshloka
{गन्धर्वा गीतकुशला नृत्येषु च विशारदाः}
{रमयन्ति स्म तान्विप्रान्यज्ञकर्मान्तरेषु वै}


\chapter{अध्यायः ९०}
\twolineshloka
{श्रपयित्वा पशूनन्यान्विधिवद्द्विजसत्तमाः}
{तं तुरङ्गं यथाशास्त्रमालभन्त द्विजातयः}


\threelineshloka
{ततः संज्ञप्य तुरगंक विधिवद्याजकर्षभाः}
{उपसंवेशयांचक्रुस्ततस्तां द्रुपदात्मजाम्}
{कलाभिस्तिसृभी राजन्यथाविधि मनस्विनीम्}


\twolineshloka
{उद्धृत्य तु वपां तस्य यथाशास्त्रं द्विजातयः}
{श्रपयामासुरव्यग्रा विधिवद्भरतर्षभ}


\twolineshloka
{तं वपाधूमगन्धं तु धर्मराजः सहानुजैः}
{उपाजिघ्रद्यथासास्त्रं सर्वपापापहं तदा}


\twolineshloka
{शिष्टान्यङ्गानि यान्यासंस्तस्याश्वस्य नराधिप}
{तान्यग्रौ जुहुवुर्धीराः समस्ताः षोडशर्त्विजः}


\twolineshloka
{संस्थाप्यैवं तस्य राज्ञस्तं यज्ञं शक्रतेजसः}
{व्यासः सशिष्यो भगवान्वर्धयामास तं नृपम्}


\twolineshloka
{ततो युधिष्ठिरः प्रादात्सदस्येभ्यो यथाविधि}
{कोटीः सहस्रं निष्काणां व्यासाय तु वसुंधराम्}


\twolineshloka
{प्रतिगृह्य धरां राजन्व्यासः सत्यवतीसुतः}
{अब्रवीद्भरतश्रेष्ठं धर्मराजं युधिष्ठिरम्}


\twolineshloka
{वसुधा भवतस्त्वेषां संन्यस्ता राजसत्तम}
{निष्क्रयो दीयतां मह्यं ब्राह्मणा हि धनार्थिना}


\twolineshloka
{युधिष्ठिरस्तु तान्विप्रान्प्रत्युवाच महामनाः}
{भ्रातृभिः सहितो धीमान्मध्ये राज्ञां महात्मनाम्}


\twolineshloka
{अश्वमेधे महायज्ञे पृथिवी दक्षिणा स्मृता}
{अर्जुनेन जिता चेयमृत्विग्भ्यः प्रापिता मया}


\twolineshloka
{वनं प्रवेक्ष्ये विप्राग्र्या विभजध्वं महीमिमाम्}
{चतुर्धा पृथिवीं कृत्वा चातुर्होत्रप्रमाणतः}


% Check verse!
नाहमादातुमिच्छामि ब्रह्मस्वं द्विजसत्तमाः ॥इदं नित्यं मनो विप्रा भ्रातॄणां चैव मे सदा
\twolineshloka
{इत्युक्तवति तस्मिंस्तु भ्रातरो द्रौपदी च सा}
{एवमेतदिति प्राहुस्तदभूद्रोमहर्षणम्}


\twolineshloka
{ततोऽन्तरिक्षे वागासीत्साधुसाध्विति भारत}
{तथैव द्विजसङ्घानां शंसतां विबभौ स्वनः}


\twolineshloka
{द्वैपायनस्तथा कृष्णः पुनरेव युधिष्ठिरम्}
{प्रोवाच मध्ये विप्राणामिदं सम्पूजयन्मुनिः}


\twolineshloka
{दत्तैषा भवता मह्यं तां ते प्रतिददाम्यहम्}
{हिरण्यं दीयतामेभ्यो ब्राह्मणेभ्यो धराऽस्तु ते}


\twolineshloka
{ततोऽब्रवीद्वासुदेवो धर्मराजं युधिष्ठिरम्}
{यथाऽऽह भगवान्व्यासस्तथा त्वं कर्तुमर्हसि}


\twolineshloka
{इत्युक्तः स कुरुश्रेष्ठः प्रीतात्मा भ्रातृभिः सह}
{कोटिं कोटिं गवां प्रादाद्दक्षिणां त्रिगुणीकृताम्}


\twolineshloka
{न करिष्यति तल्लोके कश्चिदन्यो नराधिपः}
{यत्कृतं कुरुराजेन मरुत्तस्यानुकुर्वता}


\twolineshloka
{प्रतिगृह्य तु तद्द्रव्यं कृष्णद्वैपायनो मुनिः}
{ऋत्विग्भ्यः प्रददौ विद्वांश्चतुर्धा व्यभजंश्च ते}


\twolineshloka
{धरण्या निष्क्रयं दत्त्वा तद्धिरण्यं युधिष्ठिरः}
{दूतपापो जितस्वर्गो मुमुदे भ्रातृभिः सह}


\twolineshloka
{ऋत्विजस्तमपर्यन्तं सुवर्णनिचयं तथा}
{व्यभजन्त द्विजातिभ्यो यथोत्साहं यथासुखम्}


\threelineshloka
{यज्ञवाटे च यत्किञ्चिद्धिरण्यं सवि भूषणम्}
{तोरणानि च यूपांश्च घटान्पात्रीस्तथेष्टकाः}
{युधिष्ठिराभ्यनुज्ञाताः सर्वं तद्व्यभजन्द्विजाः}


\threelineshloka
{अनन्तरं द्विजातिभ्यः क्षत्रिया जह्रिरे वसु}
{तथा विट्शूद्रसङ्घाश्च तथाऽन्ये म्लेच्छजातयः}
{`कालेन महता जह्रुस्तत्सुवर्णं ततस्ततः ॥'}


\twolineshloka
{ततस्ते ब्राह्मणाः सर्वे मुदिता जग्मुरालयान्}
{तर्पिता वसुना तेन धर्मिराजेन धीमिता}


\twolineshloka
{स्वमंशं भगवान्व्यासः कुन्त्यै पादाभिवादितः}
{प्रददौ तस्य महतो हिरण्यस्य महाद्युतिः}


\twolineshloka
{श्वशुरात्प्रीतिदायं तं प्राप्य सा प्रीतमानसा}
{चकार पुण्यकं तेन सुमहत्सङ्घशः पृथा}


\twolineshloka
{गत्वा त्ववभृथं राजा विपाप्मा भ्रातृभिः सह}
{सभाज्यमानः शुशुभे महेन्द्रस्त्रिदशैरिव}


\twolineshloka
{पाण़्डवाश्च महीपालैः समेतैरभिसंवृताः}
{अशोभन्त महाराज ग्रहस्तारागणैरिव}


\twolineshloka
{राजभ्योपि ततः प्रादाद्रत्नानि विविधानि च}
{गजानश्वानलङ्कारान्त्रियो वासांसि काञ्चनम्}


\twolineshloka
{तद्धनौघमपर्यन्तं पार्थः पार्थिवमण्डले}
{विसृजञ्शुशुभे राजन्यथा वैश्रवणस्तथा}


\twolineshloka
{आनीय च तथा वीरं राजानं बभ्रुवाहनम्}
{प्रदाय विपुलं वित्तं गृहात्प्रास्थापयनत्तदा}


\twolineshloka
{दुःशलायाश्च तं पौत्रं बालकं भरतर्षभ}
{स्वराज्येऽथ पितुर्धामान्स्वसुः प्रीत्या न्वयेशयत्}


\twolineshloka
{नृपतींश्चैव तान्सर्वान्सुविभक्तान्सुपूजितान्}
{प्रस्थापयामास वशी कुरुराजो युधिष्ठिरः}


\twolineshloka
{गोविन्दं च महात्मानं बलदेवं महाबलम्}
{तथाऽन्यान्वृष्णिवीरांश्च प्रद्युम्नाद्यान्सहस्रशः}


\twolineshloka
{पूजयित्वा महाराज यथाविधि महाद्युतिः}
{भ्रातृभिः सहितो राजा प्रास्थापरयदरिंदमः}


\twolineshloka
{एवं बभूव यज्ञः स धर्मिराजस्य धीमतः}
{बह्वन्नधनरत्नौघः सुरामैरेयसागरः}


\twolineshloka
{सर्पिःपङ्का ह्रदा यत्र बभूवुश्चान्नपर्वताः}
{रसालकर्दमा नद्यो बभूवुर्भरतर्षभ}


\twolineshloka
{भक्ष्यखाण्डवरागाणां क्रियतां भुज्यतां तथा}
{पशूनां वध्यतां चैव नान्तं ददृशिरे जनाः}


\threelineshloka
{मत्तप्रमत्तमुदितं सुप्रीतयुवतीजनम्}
{मृदङ्गशङ्खनादैस्च मनोरममभूत्तदा}
{}


\twolineshloka
{दीयतां भुज्यतां चापि तत्र शब्दो महानभूत्}
{दीयतां दीयतां चेति दिवारात्रमवारितम्}


\twolineshloka
{तं महोत्सवसंकाशं हृष्टपुष्टजनाकुलम्}
{कथयन्ति स्म पुरुषा नानादेशनिवासिनः}


\twolineshloka
{वर्षित्वा धनधाराभिः कामै रत्नै रसैस्तथा}
{विपाप्मा भरतश्रेष्ठः कृतार्थः प्राविशत्पुरम्}


\chapter{अध्यायः ९१}
\threelineshloka
{पितामहस्य मे यज्ञे धर्मराजस्य धीमतः}
{यदाश्चर्यमभूत्किञ्चित्तद्भवान्वक्तुमर्हति ॥वैशम्पायन उवाच}
{}


\twolineshloka
{श्रूयतां राजशार्दूल महदाश्चर्यमुत्तमम्}
{अश्वमेधे महायज्ञे निवृत्ते यदभूत्प्रभो}


\twolineshloka
{तर्पितेषु द्विजाग्र्येषु ज्ञातिसम्बन्धिबन्धुषु}
{दीनान्धकृपणे वाऽपि तदा भरतसत्तम}


\twolineshloka
{घुष्यमाणे महादाने दिक्षु सर्वासु भारत}
{पतत्सु पुष्पवर्षेषु धर्मराजस्य मूर्धनि}


\twolineshloka
{बिलान्निस्सृत्य नकुलो रुक्मपार्श्वस्तदाऽनघः}
{वज्राशनिसमं नादममुञ्चद्वसुधाधिप}


\twolineshloka
{सकृदुत्सृज्य तन्नादं त्रासयानो मृगद्विजान्}
{मानुपं वचनं प्राह पुष्पोपरिशयो महान्}


\twolineshloka
{सक्तुप्रस्थेन वो नायं यज्ञस्तुल्यो नराधिपाः}
{उच्छवृत्तेर्वदान्यस्य कुरुक्षेत्रनिवासिनः}


\twolineshloka
{तस्य तद्वचनं श्रुत्वा नकुलस्य विशांपते}
{विस्मयं परमं जग्मुः सर्वे ते ब्राह्मणर्षभाः}


% Check verse!
ततः समेत्य नकुलं पर्यपृच्छन्त ते द्विजाः ॥कुतस्त्वं समनुप्राप्तो यज्ञं साधुसमागमम्
\twolineshloka
{किं बलं परमं तुभ्यं श्रुतं किं परायणम्}
{कथं भवन्तं विद्याम यो नो यज्ञं विगर्हसे}


\twolineshloka
{अविलुप्यागमं कृत्स्नं विविधैर्यज्ञियैः कृतम्}
{यथागमं यथान्यायं कर्तव्यं च तथा कृतम्}


\twolineshloka
{पूजार्हाः पूजिताश्चात्र विधिवच्छास्त्रदर्शात्}
{मन्त्राहुतिहुतश्चाग्निर्दत्तं देयममत्सरम्}


\twolineshloka
{तुष्टा द्विजातयश्चात्र दानैर्बहुविधैरपि}
{क्षत्रियाश्चि सुयुद्धेन श्राद्धैश्चापि पितामहाः}


\twolineshloka
{पालनेन विशस्तुष्टाः कामैस्तुष्टा वरस्त्रियः}
{अनुक्रोशैस्तता शूद्रा दानशेषैः पृथग्जनाः}


\twolineshloka
{ज्ञातिसम्बन्धिनस्तुष्टाः शौचेन च नृपस्य नः}
{देवा हविर्भिः पुण्यैस्च रक्षणैः शरणागताः}


\twolineshloka
{यदत्र तथ्यं तद्ब्रूहि सत्यंसत्यं द्विजातिषु}
{यथाश्रुतं यथादृष्टं पृष्टो ब्राह्मणकाम्यया}


\twolineshloka
{श्रद्धेयवाक्यः प्राज्ञस्त्वं दिव्यं रूपं बिभर्षि च}
{समागतश्च विप्रैस्त्वं तद्भवान्वक्तुमर्हति}


\twolineshloka
{इति पृष्टो द्विजैस्तैः स प्रहसन्नकुलोऽब्रवीत्}
{नैषा मृषामया वाणी प्रोक्ता दर्पेण वा द्विजाः}


\twolineshloka
{यन्मयोक्तमिदं वाक्यं युष्माभिश्चाप्युपश्रुतम्}
{सक्तुप्रस्थेन वो नायं यज्ञस्तुल्यो द्विजर्षभाः}


\twolineshloka
{इत्यवश्यं मयैतद्वो वक्तव्यं द्विजसत्तमाः}
{शृणुताव्यग्रमनसः शंसतो मे यथातथम्}


\twolineshloka
{अनुभूतं च दृष्टं च यन्मयाऽद्भुतमुत्तमम्}
{उञ्छवृत्तेर्वदान्यस्य कुरुक्षेत्रनिवासिनः}


\twolineshloka
{स्वर्गं येन द्विजाः प्राप्तः सभार्यः ससुतस्नुषः}
{यथा चार्धं शरीरस्य ममेदं काञ्चनीकृतम्}


\chapter{अध्यायः ९२}
\twolineshloka
{हन्त वः कथयिष्यामि दानस्य फलमुत्तमम्}
{न्यायलब्धस्य सूक्ष्मस्य विप्रदत्तस्य यद्द्विजाः}


\twolineshloka
{धर्मक्षेत्रे कुरुक्षेत्रे धर्मज्ञैर्बहुभिर्वृते}
{उञ्छवृत्तिर्द्विजः कश्चित्कापोतिरभवत्पुरा}


\twolineshloka
{सभार्यः सहपुत्रेण सस्नुषस्तपसि स्थितः}
{बभूव शुक्लवृत्तः स धर्मात्मा नियतेन्द्रियः}


% Check verse!
षष्ठे काले सदा विप्रो भुङ्क्ते तैः सह संवृतः
\twolineshloka
{षष्ठे काले कदाचित्तु तस्याहारो न विद्यते}
{भुङ्क्तेऽन्यस्मिन्कदाचित्स षष्ठे काले द्विजोत्तमः}


\twolineshloka
{कपोतधर्मिणस्तस्य दुर्भिक्षे सति दारुणे}
{नाविद्यत तदा विप्राः संचयस्तन्निबोधत}


\twolineshloka
{क्षीणाषैदिसमावापो द्रव्यहीनोऽभवत्तदा}
{कालेकालेऽस्य सम्प्राप्ते नैव विद्येत भोजनम्}


\twolineshloka
{क्षुधापरिगताः सर्वे प्रातिष्ठन्त तदा तु ते}
{उञ्छं तदा शुक्लपक्षे मध्यं तपति भास्करे}


\twolineshloka
{उष्णार्तश्च क्षुधार्तश्च विप्रस्तपसि संस्थितः}
{उञ्छमप्राप्तवानेव ब्राह्मणः क्षुच्छ्रमान्वितः}


\twolineshloka
{स तथैव क्षुधाविष्टः सार्धं परिजनेन ह}
{क्षपयामास तं कालं कृच्छ्रप्राणो द्विजोत्तमः}


\twolineshloka
{अथ षष्ठे गते काले यवप्रस्थमुपार्जयन्}
{यवप्रस्थं तु तं सक्तूनकुर्वन्त तपस्विनः}


\twolineshloka
{कृतजप्याह्निकास्ते तु हुत्वा चाग्निं यथाविधि}
{कुडवंकुडवं सर्वे व्यभजन्त तपस्विनः}


\twolineshloka
{अथागच्छद्द्विजः कश्चिदतिथिर्भुञ्जतां तदा}
{ते तं दृष्ट्वाऽतिथिं प्राप्तं प्रहृष्टमनसोऽभवन्}


\twolineshloka
{तेऽभिवाद्य सुखप्रश्नं पृष्ट्वा तमतिथिं तदा}
{विशुद्धमनसो दान्ताः श्रद्धादमसमन्विताः}


\twolineshloka
{अनसूया गतक्रोधाः साधवो वीतमत्सराः}
{त्यक्तमानमदक्रोधा धर्मज्ञा द्विजसत्तमाः}


\twolineshloka
{सब्रह्यचर्यं गोत्रं ते तस्य ख्यात्वा परस्परम्}
{कुटीं प्रवेशयामासुः क्षुधार्तमतिथिं तदा}


\threelineshloka
{इदमर्घ्यं च पाद्यं च बृसी चेयं तवानघ}
{शुचयः सक्तवश्चेमे नियमोपार्जिताः प्रभो}
{प्रतिगृह्णीष्व भद्रं ते मया दत्ता द्विजर्षभ}


\twolineshloka
{इत्युक्तः प्रतिगृह्याथ सक्तूनां कुडवं द्विजः}
{भक्षयामास राजेन्द्र न च तुष्टिं जगाम सः}


\twolineshloka
{स उञ्छवृत्तिस्तं प्रेक्ष्य क्षुधापरिगतं द्विजम्}
{आहारं चिन्तयामास कथं तुष्टो भवेदिति}


\twolineshloka
{तस्य भार्याऽब्रवीद्वाक्यं मद्भागो दीयतामिति}
{गच्छत्वेष यथाकामं परितुष्टो द्विजोत्तमः}


\twolineshloka
{इति ब्रुवन्तीं तां साध्वीं भार्यां स द्विजसत्तमः}
{क्षउधापरिगतां ज्ञात्वा तान्सक्तून्नाभ्यनन्दत}


\threelineshloka
{आत्मानुमानतो विद्वान्स तु विप्रर्षभस्तदा}
{जानन्वृद्धां क्षुधार्तां च श्रान्तां ग्लानांतपस्विनीम्}
{त्वगस्थिभूता वेपन्तीं ततो भार्यामुवाच ह}


\twolineshloka
{अपि कीटपतङ्गानां मृगाणां चैव शोभने}
{स्त्रियो रक्ष्याश्च पोष्पाश्च न त्वेवं वक्तुमर्हसि}


\twolineshloka
{अनुकंप्यो नरः पत्न्या पुष्टो रक्षित एव च}
{प्रपतेद्यशसो दीप्तात्स च लोकान्न चाप्नुयात्}


\twolineshloka
{धर्मकामार्थकार्याणि शुश्रूषाकुलसंततिः}
{दारेष्वदीनो धर्मस्च पितॄणामात्मनस्तथा}


\twolineshloka
{न वेत्ति कर्मतो भार्यारक्षणे योऽक्षमः पुमान्}
{अयशो महदाप्नोति नारकांश्चैव गच्छति}


\twolineshloka
{इत्युक्ता सा ततः प्राह धर्मार्थौ नौ समौ द्विज}
{सक्तुप्रस्थचतुर्भागं गृहाणेमं प्रसीद मे}


\twolineshloka
{सत्यं रतिश्च धर्मश्च स्वर्गश्च गुणनिर्जितः}
{स्त्रीणां पतिसमाधीनं काङ्क्षितं च द्विजर्षभ}


\twolineshloka
{ऋतुर्मातु पितुर्बीजं दैवतं परमं पतिः}
{भर्तुः प्रसादान्नारीणां रतिपुत्रफलं तथा}


\twolineshloka
{पालनाद्धि पतिस्त्वं मे भर्ताऽसि भरणाच्च मे}
{पुत्रप्रदानाद्वरदस्तस्मात्सक्तून्प्रयच्छ मे}


\twolineshloka
{जरापरिगतो वृद्धः क्षुधार्तो दुर्बलो भृशम्}
{उपवासपरिश्रान्तो यदा त्वमपि कर्शितः}


\twolineshloka
{इत्युक्तः स तया सक्तून्प्रगृह्येदं वचोऽब्रवीत्}
{द्विजि सक्तूनिमान्भूयः प्रतिगृह्णीष्व सत्तम}


\threelineshloka
{स तान्प्रगृह्य भुक्त्वा च न तुष्टिमगमद्द्विजः}
{तमुञ्छवृत्तिरालक्ष्य ततश्चिन्तापरोऽभवत् ॥पुत्र उवाच}
{}


\twolineshloka
{सक्तूनिमान्प्रगृह्य त्वं देहि विप्राय सत्तम}
{इत्येव सुकृतं मन्ये तस्मादेतत्करोम्यहम्}


\twolineshloka
{भवान्हि परिपाल्यो मे सर्वदैव प्रयत्नतः}
{साधूनां काङ्क्षितं यस्मात्पितुर्वृद्धस्य पालनम्}


% Check verse!
पुत्रार्थो विहितो ह्येष वार्धके परिपालनम् ॥श्रुतिरेषा हि विप्रर्षे त्रिषु लोकेषु शाश्वती
\threelineshloka
{प्राणाधारणमात्रेण शक्यं कर्तुं तपस्त्वया}
{प्राणो हि परमो धर्मः स्थितो देहेषु देहिनाम् ॥पितोवाच}
{}


\twolineshloka
{अपि वर्षसहस्री त्वं बाल एव मतो मम}
{उत्पाद्य पुत्रं हि पिता कृतकृत्यो भवेत्सुतात्}


\twolineshloka
{बालानां क्षुद्बलवती जानाम्येतदहं प्रभो}
{वृद्धोऽहं धारयिष्यामि त्वं बली भव पुत्रक}


\threelineshloka
{जीर्णेन वयसा पुत्र न मां क्षुद्बाधतेऽपि च}
{दीर्घकालं तपस्तप्तं न मे मरणतो भयम् ॥पुत्र उवाच}
{}


\threelineshloka
{अपत्यमस्मि ते पुंसस्त्राणात्पुत्र इति स्मृतः}
{आत्मा पुत्रः स्मृतस्तस्मात्त्राह्यात्मानमिहात्मना ॥पितोवाच}
{}


\twolineshloka
{रूपेण सदृशस्त्वं मे शीलेन च दमेन च}
{परीक्षितश्च बहुधा सक्तूनादद्मि ते सुत}


\twolineshloka
{इत्युक्त्वाऽऽदाय तान्सक्तून्प्रीतात्मा द्विजसत्तमः}
{प्रहसन्निव विप्राय स तस्मै प्रददौ तदा}


\twolineshloka
{भुक्त्वा तानपि सक्तून्स नैव तुष्टो बभूव ह}
{उच्छवृत्तिस्तु धर्मात्मा व्रीडामनुजगाम ह}


\twolineshloka
{तं वै वधूः स्थिता साध्वी ब्राह्मणिप्रियकाम्यया}
{सक्तूनादाय संहृष्टा श्वशुरं वाक्यमब्रवीत्}


\twolineshloka
{संतानात्तव संतानं मम विप्र भविष्यति}
{सक्तूनिमानतिथये गृहीत्वा सम्प्रयच्छ मे}


\twolineshloka
{तव प्रसादान्निर्वृत्ता मम लोकाः किलाक्षयाः}
{पुत्रेण तानवाप्नोति यत्र गत्वा न शोचति}


\twolineshloka
{धर्माद्या हि यथा त्रेता वह्नित्रेता तथैव च}
{तथैव पुत्रपौत्राणां स्वर्गस्त्रेता किलाक्षयः}


\threelineshloka
{पितॄणात्तारयति पुत्र इत्यनुशुश्रुम}
{पुत्रपौत्रैश्च नियतं सादुलोकानुपाश्नुते ॥श्वशुर उवाच}
{}


\twolineshloka
{वातातपविशीर्णाङ्गीं त्वां विवर्णां निरीक्ष्य वै}
{कर्शितां सुव्रताचारे क्षुधाविह्वलचेतसम्}


\twolineshloka
{कथं सक्तून्ग्रहीष्यामि भूत्वा धर्मोपघातकः}
{कल्याणवृत्ते कल्याणि नैव त्वं वक्तुमर्हसि}


\twolineshloka
{षष्ठे काले व्रतवतीं शौचशीलतपोन्विताम्}
{कृच्छ्रवृत्तिं निराहारां द्रक्ष्यामि त्वां कथं शुभे}


\threelineshloka
{बाला क्षुधार्ता नारी च रक्ष्या त्वं सततं मया}
{उपवासपरिश्रान्ता त्वं हि बान्धवनन्दिनी ॥स्नुषोवाच}
{}


\twolineshloka
{गुरोर्मम गुरुस्त्वं वै यतो दैवतदैवतम्}
{देवातिदेवस्तस्मात्त्वं सक्तूनादत्स्व मे प्रभो}


\twolineshloka
{देहः प्राणश्च धर्मश्च शुश्रूषार्थमिदं गुरोः}
{तव विप्र प्रसादेन लोकान्प्राप्स्यामहे शुभान्}


\threelineshloka
{अवेक्ष्या इति कृत्वाऽहं दृढभक्तेति वा द्विज}
{चिन्त्या ममेयमिति वा सक्तूनादानुमर्हसि ॥श्वशुर उवाच}
{}


\twolineshloka
{अनेन नित्यं साध्वी त्वं शीलवृत्तेन शोभसे}
{या त्वं धर्मव्रतोपेता गुरुवृत्तिमवेक्षसे}


\twolineshloka
{तस्मात्सक्तून्ग्रहीष्यामि वधु नार्हसि वञ्चनाम्}
{गणयित्वा महाभागे त्वां हि धर्मभृतां वरे}


\twolineshloka
{इत्युक्त्वा तानुपादाय सक्तून्प्रादाद्द्विजातये}
{ततस्तुष्टोऽभवद्विप्रस्तस्य साधोर्महात्मनः}


\twolineshloka
{प्रीतात्मा स तु तं वाक्यमिदमाह द्विजर्षभम्}
{वाग्मी तदा द्विजश्रेष्टो धर्मः पुरुषविग्रहः}


\twolineshloka
{शुद्धेनि तव दानेन न्यायोपात्तेन धर्मतः}
{यथाशक्ति विसृष्टेन प्रीतोस्मि द्विजसत्तम}


\twolineshloka
{अहो दानं घुष्यते ते स्वर्गे स्वर्गनिवासिभिः}
{गगनात्पुष्पवर्षं च पश्येदं पतितं भुवि}


\twolineshloka
{सुरर्षिदेवगन्धर्वा ये च देवपुरःसराः}
{स्तुवन्तो देवदूताश्च स्थिता दानेन विस्मिताः}


\twolineshloka
{ब्रह्मर्षयो विमानस्था ब्रह्मलोकचरश्च ये}
{काङ्क्षन्ते दर्शनं तुभ्यं दिवं व्रज द्विजर्षभ}


\twolineshloka
{पितृलोकगताः सर्वे तारिताः पितरस्त्वया}
{अनागताश्च बहवः सुबहूनि युगान्युत}


\twolineshloka
{ब्रह्मचर्येण दानेन यज्ञेनि तपसा तथा}
{अगह्वरेण धर्मेण तस्माद्गच्च दिवं द्विज}


\twolineshloka
{श्रद्धया परया यरत्वं तपश्चरसि सुव्रत}
{तस्माद्देवास्तवानेन प्रीता ब्राह्मणसत्तम}


\twolineshloka
{सर्वमेतद्धि यस्मात्ते दत्तं शुद्धेन चेतसा}
{कृच्छ्रकाले ततः स्वर्गो विजितः कर्मणा त्वया}


\twolineshloka
{क्षुधा निर्णुदति प्रज्ञां धर्मबुद्धिं व्यपोहति}
{क्षुधापरिगतज्ञानो धृतिं त्यजति चैव ह}


\twolineshloka
{बुभुक्षां जयते यस्तु स स्वर्गं जयते ध्रुवम्}
{यदा दानरुचिः स्याद्वै तदा धर्मो न सीदति}


\twolineshloka
{अनवेक्ष्य सुतस्नेहं कलत्रस्नेहमेव च}
{धर्ममेव गुरं ज्ञात्वा तृष्णा न गणिता त्वया}


\twolineshloka
{द्रव्यागमो नृणां सूक्ष्मः पात्रे दानं ततः परम्}
{कालः परतरो दानाच्छ्रद्धा चैव ततः परा}


\twolineshloka
{स्वर्गद्वारं सुसूक्ष्मं हि नरैर्माहान्नि दृश्यते}
{सङ्गर्गलं लोभकीलं रागगुप्तं दुरासदम्}


\twolineshloka
{तं तु पश्यन्ति पुरुषा जितदक्रोधा जितेन्द्रियाः}
{ब्राह्मणास्तपसा युक्ता यथाशक्ति प्रदायिनः}


\twolineshloka
{सहस्रशक्तिश्च शतं शतशक्तिर्दशापि च}
{दद्यादपश्च यः शक्त्या सर्वे तुल्यफलाः स्मृताः}


\twolineshloka
{रन्तिदेवो हि नृपतिरपः प्रादादकिंचनः}
{शुद्धेन मनसा विप्र नाकपृष्ठं ततो गतः}


\twolineshloka
{न धर्मः प्रीयते तात दानैर्दत्तैर्महाफलैः}
{न्यायलब्धैर्यथा सूक्ष्मैः श्रुद्धापूतैः स तुष्यति}


\twolineshloka
{गोप्रदानसहस्राणि द्विजेभ्योऽदान्नृगो नृपः}
{एकां दत्त्वा स पारक्यां नरकं समपद्यत}


\twolineshloka
{आत्ममांसप्रदानेन शिबोरौशीनरो नृपः}
{प्राप्य पुण्यकृताँल्लोकान्मोदत दिवि सुव्रतः}


\twolineshloka
{विभवेन नृणां पुण्यं यच्छत्त्या स्वार्जितं न तत्}
{न यज्ञैर्विविधैर्विप्र यथान्यायेन संचितैः}


\twolineshloka
{क्रोधाद्दानफलं हन्ति लोभात्स्वर्गं न गच्छति}
{न्यायवृत्तिर्हि तपसा दानवित्स्वर्गमश्नुते}


% Check verse!
न राजसूयैर्बहुभिरिष्टा विपुलदक्षिणैः ॥न चाश्वमेधैर्बहुभिः फलं सममिदं तव
\twolineshloka
{सक्तुप्रस्थेन विजितो ब्रह्मलोकस्त्वयाऽक्षयः}
{विरजो ब्रह्मसदनं गच्छ विप्र यतासुखम्}


\twolineshloka
{सर्वेषां वो द्विजश्रेष्ठ दिव्यं यानमुपस्थितम्}
{आरोहत यथाकामं धर्मोस्मि द्विज पश्य माम्}


\twolineshloka
{बाधितो हि त्वया देहो लोके कीर्तिः स्थिरा च ते}
{सभार्यः सहपुत्रश्च सस्नुषश्च दिवं व्रज}


\twolineshloka
{इत्युक्तवाक्ये धर्मे तु यानमारुद्य स द्विजः}
{सदारः ससुतश्चैव सस्नुषश्च दिवं गतः}


\twolineshloka
{तस्मिन्विपरे गते स्वर्गं ससुते सस्नुषे तदा}
{भार्याचतुर्थे धर्मज्ञे ततोऽहं निःसृतो बिलात्}


\twolineshloka
{ततस्तु सक्तुगन्धेन क्लेदेन सलिलस्य च}
{दिव्यपुष्पविमर्दाच्च साधोर्दानलवैश्च तैः}


\twolineshloka
{विप्रस्य तपसा तस्य शिरो मे काञ्चनीकृतम्}
{तस्य सत्याभिसन्धस्य सक्तुदानेन चैव ह}


\twolineshloka
{शरीरार्धं च मे विप्राः शातकुंभमयं कृतम्}
{पश्यतेमं सुविपुलं तपसा तस्य धीमतः}


\threelineshloka
{कथमेवंविधं स्याद्वै पार्श्वमन्यदिति द्विजाः}
{तपोवनानि यज्ञांश्च हृष्टोऽभ्योमि पुनः पुनः}
{}


\twolineshloka
{यज्ञं त्वहमिमं श्रुत्वा रुरुराजस्य धीमतः}
{आशया परया प्राप्तो न चाहं काञ्चनीकृतः}


\twolineshloka
{ततो मयोक्तं तद्वाक्यं प्रहस्य ब्राह्मणर्षभाः}
{सक्तुप्रस्थेन यज्ञोऽयं संमितो नेति सर्वथा}


\twolineshloka
{सक्तुप्रस्थलवैस्तैर्हि तदाऽहं काञ्चनीकृतः}
{न हि यज्ञो महानेष सदृशस्तैर्मतो मम}


\twolineshloka
{इत्युक्त्वा नकुलः सर्वान्यज्ञे द्विजवरांस्तदा}
{जगामादर्सनं तेषां विप्रास्ते च ययुर्गृहान्}


\twolineshloka
{एतत्ते सर्वमाख्यातं मया परपुरंजय}
{यदाश्चर्यमभूत्तत्र वाचिमेधे महाक्रतौ}


\twolineshloka
{न विस्मयस्ते नृपते यज्ञे कार्यः कथञ्चन}
{ऋषिकोटिसहस्राणि तपोभिर्ये दिवं गताः}


\twolineshloka
{अद्रोहः सर्वभूतेषु संतोषः शीलमार्जवम्}
{तपो दमश्च सत्यं च प्रदानं चेति संमितम्}


\chapter{अध्यायः ९३}
\twolineshloka
{यज्ञे सक्ता नृपतयस्तपःसका महर्षयः}
{शान्तिव्यवस्थिता विप्राः शमे दम इति प्रभो}


\twolineshloka
{तस्माद्यज्ञफलैस्तुल्यं न किञ्चिदिह दृश्यते}
{इति मे वर्तते बुद्धिस्तथा चैतदसंशयम्}


\twolineshloka
{यज्ञैरिष्ट्वा तु बहवो राजानो द्विजसत्तमाः}
{इह कीर्ति परां प्राप्य प्रेत्य स्वर्गमितो गताः}


\twolineshloka
{देवराजः सहस्राक्षः क्रतुभिर्भूरिदक्षिणैः}
{देवराज्यं महातेजाःक प्राप्तवानखिलं विभुः}


\twolineshloka
{अतो युधिष्ठिरो राजा भीमार्जुनपुरःसरः}
{सदृशो देवराजेन समृद्ध्या विक्रमेण च}


\threelineshloka
{अथ कस्मात्स नकुलो गर्हयामास तं क्रतुम्}
{अश्वमेधं महायज्ञं राज्ञस्तस्य महात्मनः ॥वैशम्पायन उवाच}
{}


\twolineshloka
{यज्ञस्य विधिमग्र्यं वै फलं चापि नराधिप}
{गदतः शृणु मे राजन्यथावदिह भारत}


\twolineshloka
{पुरा शक्रस्य यजतः सर्व ऊचुर्महर्षयः}
{ऋत्विक्षु कर्मव्यग्रेषु वितते यज्ञकर्मणि}


\twolineshloka
{हूयमाने तथा वह्नौ होत्रा गुणसमन्विते}
{देवष्वाहूयमानेषु स्थितेषु परमर्षिषु}


\twolineshloka
{सुप्रतीतैस्तथा विप्रैः स्वागमैः सुस्वरैर्नृप}
{अश्रान्तैश्चापि लघुभिरध्वर्युवृषभैस्तथा}


\twolineshloka
{आलम्भसमये तस्मिन्गृहीतेषु पशुष्वथ}
{महर्षयो महाराज बभूवुः कृपयाऽन्विताः}


\twolineshloka
{ततो दीनान्पशून्दृष्ट्वा ऋषयस्ते तपोधनाः}
{ऊचुः शक्रं समागम्य नायं यज्ञविधिः शुभः}


\twolineshloka
{अपरिज्ञानमेतत्ते महान्तं धर्ममिच्छतः}
{सन्ति यज्ञे बहुगुणा विधिदृष्टाः पुरंदर}


\twolineshloka
{धर्मोपघातकस्त्वेष समारंभस्तव प्रभो}
{नायं धर्मकृतो यज्ञो न हिंसा धर्म उच्यते}


\twolineshloka
{आगमेनैव ते यज्ञं कुर्वन्तु यदि चेच्छसि}
{विधिदृष्टेन यज्ञेन धर्मस्ते सुमहान्भवेत्}


\twolineshloka
{यज्ञं बीजैः सहस्राक्ष त्रिवर्षपरमोषितेः}
{एष धर्मो महाञ्शक्र चिन्तयानोसि गम्यते}


\twolineshloka
{शतक्रतुस्तु तद्वाक्यमृषिभिस्तत्त्वदर्शिभिः}
{उक्तं न प्रतिजग्राह मानमोहवशं गतः}


\twolineshloka
{तेषां विवादः सुमहाञ्शक्रयज्ञे तपस्विनाम्}
{जङ्गमैः स्थावरैर्वाऽपि यष्टव्यमिति भारत}


\twolineshloka
{ते तु खिन्ना विवादेन ऋषयस्तत्त्वदर्शिनः}
{अभिसंधाय शक्रेण पप्रच्छुर्नृपतिं वसुम्}


\threelineshloka
{धर्मसंशयमापन्नान्सत्यं ब्रूहि महामते}
{महाभाग कथं यज्ञेष्वागमो नृपसत्तम}
{यष्टव्यं पशुभिर्मेध्यैरथो बीजैरजैरिति}


\twolineshloka
{तच्छ्रुत्वा तु वसुस्तेषामविचार्य बलाबलम्}
{यथोपनीतैर्यष्टव्यमिति प्रोवाच पार्थिवः}


\twolineshloka
{एवमुक्त्वा स नृपतिः प्रविवेश रसातलम्}
{उक्त्वेह वितथं राजंश्चेदीनामीश्वरः प्रभुः}


\twolineshloka
{तस्मान्नि वाच्यं ह्येकेन बहुज्ञेनापि संशये}
{प्रजापतिमपाहाय स्वयंभुवमृते प्रभुम्}


\twolineshloka
{तेन दत्तानि दानानि पापेनाशुद्धबुद्धिना}
{तानि सर्वाण्यनादृत्य नश्यन्ति विपुलान्यपि}


\twolineshloka
{तस्याधर्मप्रवृत्तस्य हिंसकस्य दुरात्मनः}
{दानेन कीर्तिर्भवति न प्रेत्येह च दुर्मतेः}


\twolineshloka
{अन्यायोपगतं द्रव्यमभीक्ष्णं यो ह्यपण्डितः}
{धर्माभिकाङ्क्षी त्यजति न स धर्मफलं लभेत्}


\twolineshloka
{धर्मवैतंसिको यस्तु पापात्मा पुरुषाधमः}
{ददाति दानं विप्रेभ्यो लोकविश्वासकारणम्}


\twolineshloka
{पापेन कर्मणा विप्रो धनं प्राप्य निरङ्कुशः}
{रागमोहान्वितः सोन्ते कलुषां गतिमश्नुते}


\twolineshloka
{अपि संचयबुद्धिर्हि लोभमोहवशं गतः}
{यज्ञं करोति भूतानि पापेनाशुद्धबुद्धिना}


\twolineshloka
{एवं लब्ध्वा धनं मोहाद्यो हि दद्याद्यजेत वा}
{न तस्य स फलं प्रेत्य भुंक्ते पापधनागमात्}


\twolineshloka
{उञ्छं मूलं फलं शाकमुदपात्रं तपोधनाः}
{दानं विभवतो दत्त्वा नराः स्वर्यान्ति धार्मिकाः}


\twolineshloka
{एष धर्मो महायोगो दानं भूतदया तथा}
{ब्रह्मचर्यं तथा सत्यमनुक्रोशो धृतिः क्षमा}


\twolineshloka
{सनातनस्य धर्मस्य फलमेतत्सनातनम्}
{श्रूयन्ते हि पुरावृत्ता विश्वामित्रादयो नृपाः}


\twolineshloka
{विश्वामित्रोसितश्चैवं जनसश्च महीपतिः}
{कक्षसेनार्ष्टिसेनौ च सिन्धुद्वीपश्च पार्थिवः}


\twolineshloka
{एते चान्ये च बहवः सिद्धिं परमिकां गताः}
{नृपाः सत्यैश्च दानैश्च न्यायलब्धैस्तपोधनाः}


\twolineshloka
{ब्राह्मणाः क्षत्रिया वैश्याः शूद्रा ये चाश्रितास्तपः}
{दानधर्माग्निना शुद्धास्ते स्वर्गं यान्ति भारत}


\chapter{अध्यायः ९४}
\twolineshloka
{धर्मागतेन त्यागेन भगवन्सर्वमस्ति चेत्}
{एतन्मे सर्वमाचक्ष्व कुशलो ह्यसि भाषितुम्}


\twolineshloka
{तस्योञ्छवृत्तेर्यद्वृत्तं सक्तुदाने फलं महत्}
{कथितं तु मम ब्रह्मंस्तथ्यमेतदसंशयम्}


\threelineshloka
{कथं हि सर्वयज्ञेषु निश्चयः परमो भवेत्}
{एतदर्हसि मे वक्तुं निखिलेन द्विजर्षभ ॥वैशम्पायन उवाच}
{}


\twolineshloka
{अत्राप्युदाहरन्तीममितिहासं पुरातनम्}
{अगस्त्यस्य महायज्ञे पुरावृत्तमरिंदम}


\twolineshloka
{पुराऽगस्त्यो महातेजा दीक्षां द्वादशवार्षिकीम्}
{प्रविवेश महाराज सर्वभूतहिते रतः}


\twolineshloka
{तत्राग्निकल्पा होतार आसन्सत्रे महात्मनः}
{मूलाहाराः फलाहाराश्चाश्मकुट्टा मरीचिपाः}


\twolineshloka
{परिपृष्टिका वैघसिकाः प्रसङ्ख्यानास्तथैव च}
{यतयो भिक्षवश्चात्र बभूवुः पर्यवस्थिताः}


\twolineshloka
{सर्वे प्रत्यक्षधर्माणो जितक्रोधा जितेन्द्रियाः}
{दमे स्थिताश्च सर्वे ते हिंसादंभविवर्जिताः}


\twolineshloka
{वृत्ते शुद्धे स्थिता नित्यमिन्द्रियैश्चाप्यबाधिताः}
{उपातिष्ठन्त तं यज्ञं यजन्तस्ते महर्षयः}


\twolineshloka
{यथाशक्त्या भगवता तदन्नं समुपार्जितम्}
{तस्मिन्सत्रे तु यद्वृत्तं यद्योग्यं च तदाऽभवत्}


\threelineshloka
{तथा ह्यनेकैर्मुनिभिर्महान्तः क्रतवः कृताः}
{एवंविधे त्वगस्त्यस्य वर्तमाने तथाऽध्वरे}
{न ववर्ष सहस्राक्षस्तदा भरतसत्तम}


\twolineshloka
{ततः कर्मान्तरे राजन्नगस्त्यस्य महात्मनः}
{कथेयमभिनिर्वृत्ता मुनीनां भावितात्मनाम्}


\twolineshloka
{अगस्त्यो यजमानोसौ ददात्यन्नं विमत्सरः}
{न च वर्षति पर्जन्यः कथमन्नं भविष्यति}


\twolineshloka
{सत्रं चेदं महद्विप्रा मुनेर्द्वादशवार्षिकम्}
{न वर्षिष्यति देवश्च वर्षाण्येतानि द्वादश}


\twolineshloka
{एतद्भवन्तः संचिन्त्य महर्षेरस्य धीमतः}
{अगस्त्यस्यातितपसः कर्तुमर्हन्त्यनुग्रहम्}


\twolineshloka
{इत्येवमुक्ते वचने ततोऽगस्त्यः प्रतापवान्}
{प्रोवाच वाक्यं स तदा प्रसाद्य शिरसा मुनीन्}


\twolineshloka
{यदि द्वादशवर्षाणि न वर्षिष्यति वासवः}
{चिन्तायज्ञं करिष्यामि विधिरेष सनातनः}


\twolineshloka
{यदि द्वादशवर्षाणि न वर्षिष्यति वासवः}
{स्पर्शयज्ञं करिष्यामि विधिरेष सनातनः}


\twolineshloka
{यदि द्वादशवर्षाणि न वर्षिष्यति वासवः}
{व्यायामेनाहरिष्यामि यज्ञानेतान्यतव्रतः}


\twolineshloka
{बीजयज्ञो मयाऽयं वै बहुवर्षसमाचितः}
{बीजैर्हितं करिष्यामि नात्र विघ्नो भविष्यति}


\twolineshloka
{नेदं शक्यं वृथा कर्तुं मम सत्रं कथञ्चन}
{वर्षिष्यतीह वा देवो नवा वर्षं भविष्यति}


\twolineshloka
{अथवाऽभ्यर्थनामिन्द्रो न करिष्यति कामतः}
{स्वयमिन्द्रो भविष्यामि जीवयिष्यामि च प्रजाः}


\twolineshloka
{यो यदाहारजातश्च स तथैव भविष्यति}
{विशेषं चैव कर्तास्मि पुनः पुनरतीव हि}


\twolineshloka
{अद्येह स्वर्णमभ्येतु यच्चान्यद्वसु दुर्लभम्}
{त्रिषु लोकेषु यच्चास्ति तदिहागम्यतां स्वयम्}


\twolineshloka
{दिव्याश्चाप्सरसां सङ्घा गन्धर्वाश्च सकिन्नराः}
{विश्वावसुश्च ये चान्ये तेऽप्युपासन्तु मे मखम्}


\twolineshloka
{उत्तरेभ्यः कुरुभ्यश्च यत्किंचिद्वसु विद्यते}
{सर्वं तदिह यज्ञेषु स्वयमेवोपतिष्ठतु}


\threelineshloka
{स्वर्गः स्वर्गसदश्चैवक धर्मश्च स्वयमेव तु}
{इत्युक्ते सर्वमेवैतदभवत्तपसा मुनेः}
{तस्य दीप्ताग्निमहसस्त्वगस्त्यस्यातितेजसः}


\twolineshloka
{ततस्ते मुनयो हृष्टा ददृशुस्तपसो बलम्}
{विस्मिता वचनं प्राहुरिदं सर्वे महार्थवत्}


\twolineshloka
{प्रीताः स्म तव वाक्येन न त्विच्छामस्तपोवनम्}
{तैरेव यज्ञैस्तुष्टाः स्य न्यायेनेच्छामहे वयम्}


\twolineshloka
{यज्ञं दीक्षां तथा होमान्यच्चान्यन्मृगयामहे}
{`तयोस्संधर्षितैर्यज्ञैर्नान्यतो मृगयामहे ॥'}


\twolineshloka
{न्यायेनोपार्जिताहाराः स्वकर्माभिरता वयम्}
{वेदांश्च ब्रह्मचर्येण न्यायतः प्रार्थयामहे}


\twolineshloka
{न्यायेनोत्तरकालं च गृहेभ्यो निःसृता वयम्}
{धर्मदृष्टैर्विधिद्वारैस्तपस्तप्स्यामहे वयम्}


\twolineshloka
{भवतः सम्यगिष्टा तु बुद्धिर्हिसाविवर्जिता}
{एतामहिंसा यज्ञेषु ब्रूयास्त्वं सततं प्रभो}


\twolineshloka
{प्रीतास्ततो भविष्यामो वयं तु द्विजसत्तम}
{विसर्जिताः समाप्तौ च सत्रादस्माद्व्रजामहे}


\twolineshloka
{तथा कथयतां तेषां देवराजः पुरंदरः}
{ववर्ष सुमहातेजा दृष्ट्वा तस्य तपोबलम्}


\twolineshloka
{आसमाप्तेश्च यज्ञस्य तस्यामितपराक्रमः}
{निकामवर्षी पर्जन्यो बभूव जनमेजय}


\twolineshloka
{प्रसादयामास च तमगस्त्यं त्रिदशेश्वरः}
{स्वयमभ्येत्य राजर्षे पुरस्कृत्य बृहस्पतिम्}


\twolineshloka
{ततो यज्ञसमाप्तौ तान्विससर्ज महामुनीन्}
{अगस्त्यः परमप्रीतः पूजयित्वा यथाविधि}


\chapter{अध्यायः ९५}
\threelineshloka
{कोसौ नकुलरूपेण शिरसा काञ्चनेन वै}
{प्राह मानुषवद्वाचमेतत्पृष्टो वदस्व मे ॥वैशम्पायन उवाच}
{}


\twolineshloka
{एतत्पूर्वं न पृष्टोऽहं न चास्माभिः प्रभाषितम्}
{श्रूयतां नकुलो योसौ यथा वाक्तस्य मानुषी}


\twolineshloka
{श्राद्धं संकल्पयामास जमदग्निः पुरा किल}
{होमधेनुस्तमागाच्च स्वयमेव दुदोह ताम्}


\twolineshloka
{तत्पयः स्थपयामास नवे भाण्डे दृढे शुचौ}
{क्रोधो नकुलरूपेण पिठरं पर्यकर्षयत्}


\twolineshloka
{जिज्ञासुस्तमृषिश्रेष्टं किं कुर्याद्विप्रिये कृते}
{इति सञ्चिन्त्य दुर्मेधा धर्षयामास तत्पयः}


\twolineshloka
{तमाज्ञाय मुनिः क्रोधं नैवास्य स चुकोप ह}
{स तु क्रोधस्ततो राजन्ब्राह्मणीं मूर्तिमास्थितः}


\twolineshloka
{जितोस्मीति भृगुश्रेष्ठ भृगवो ह्यतिरोषणाः}
{लोके मिथ्याप्रवादोयं यत्त्वयाऽस्मि विनिर्जितः}


\threelineshloka
{वशे स्थितोऽहं त्वय्यह्य क्षमावति महात्मनि}
{बिभेमि तपसः साधो प्रसादं कुरु मे प्रभो ॥जमदग्निरुवाच}
{}


\twolineshloka
{साक्षाद्दृष्टोसि मे क्रध गच्छ त्वं विगतज्वरः}
{न त्वयापकृतं मेऽद्य न च मे मन्युरस्ति वै}


\twolineshloka
{यान्समुद्दिश्य संकल्पः पयसोस्य कृतो मया}
{पितरस्ते महाभागास्तेभ्यो बुद्ध्यस्व गम्यताम्}


\twolineshloka
{इत्युक्तो जातसंत्रासस्तत्रैवान्तरधीयत}
{पितॄणामभिषङ्गाच्च नकुलत्वमुपागतः}


\twolineshloka
{स तान्प्रसादयामास शापस्यान्तो भवेदिति}
{तैश्चाप्युक्तः क्षिपन्धर्मं सापस्यान्तमवाप्स्यति}


\twolineshloka
{तैश्चोक्तो यज्ञियान्देशान्धर्मारण्यं तथैव च}
{जुगुप्समानो धावन्स तं यज्ञं समुपासदत्}


\twolineshloka
{धर्मपुत्रमथाक्षिप्य सक्तुप्रस्थेन तेन सः}
{मुक्तः सापात्ततः क्रोधो धर्मो ह्यासीद्युधिष्ठिरः}


\threelineshloka
{एवमेतत्तदा वृत्तं यज्ञे तस्य महात्मनः}
{पश्यतां चापि नस्तत्र नकुलोऽन्तर्हितस्तदा}
{}


\chapter{अध्यायः ९६}
\threelineshloka
{अश्वमेधे पुरावृत्ते केशवं केशिसूदनम्}
{धर्मसंशयमुद्दिश्य किमपृच्छत्पितामहः ॥वैशम्पायन उवाच}
{}


\twolineshloka
{पश्चिमेनाश्वमेधेन यदा स्नातो युधिष्ठिरः}
{तदा राजा नमस्कृत्य केशवं पुनरब्रवीत्}


\twolineshloka
{भववन्वैष्णवा धर्माः किंफलाः किंपरायणाः}
{किं धर्ममधिकृत्याथ भवतोत्पादिताः पुरा}


\twolineshloka
{यदि तेहमनुग्राह्यः प्रियोस्मि मधुसूदन}
{श्रोतव्या यदि मे कृष्ण तन्मे कथय सुव्रत}


\twolineshloka
{पवित्रा किल ते धर्माः सर्वपापप्रणाशनाः}
{सर्वधर्मोत्तमाः पुण्या भगवंस्त्वन्मुखोद्गताः}


\twolineshloka
{याञ्श्रुत्वा ब्रह्महा गोघ्नो मातृहा गुरुतल्पगः}
{पाकर्भेदी कृतघ्नश्च सुरापो ब्रह्मविक्रयी}


\twolineshloka
{मित्रविश्वासघाती च वीरहा भ्रूणहा तथा}
{तपोविक्रयिणश्चैव दानविक्रयिणस्तथा}


\twolineshloka
{आत्मविक्रमयिणो मूढा जीवेद्यश्च विकर्मभिः}
{पापाः शठा नैकृतिका डांभिका दूषकास्तथा}


\twolineshloka
{रसभेदकरा ये च ये च स्युर्ब्रह्मघातकाः}
{शूद्रप्रेष्यकराश्चोरा विप्रा ये च पुरोहिताः}


\twolineshloka
{निक्षेपहारिणः स्त्रीघ्नास्तथा ये पारदारिकाः}
{एते चान्ये च पापा ये मुच्यन्तेतेऽपि किल्बिषात्}


\twolineshloka
{तानाचक्ष्व सुरश्रेष्ठ त्वद्भक्तस्य ममाच्युत ॥वैशंपायन उवाच}
{}


\twolineshloka
{इत्येवं कथिते देवे धर्मपुत्रेण संसदि}
{वसिष्ठाद्यास्तपोयुक्ता मुनयस्तत्वदर्शिनः}


\threelineshloka
{श्रोतुकामाः परं गुह्यं वैष्णवं धर्ममुत्तमम्}
{तथा भागवताश्चैव ततस्तं पर्यवारयन् ॥युधिष्ठिर उवाच}
{}


\twolineshloka
{तत्वतस्तव भावेन पादमूलमुपागतम्}
{यदि जानासि मां भक्तं स्निग्धं वा भक्तवत्सल}


\twolineshloka
{धर्मगुह्यानि सर्वाणि वेत्तुमिच्छामि तत्वतः}
{धर्मान्कथय मे देव यद्यनुग्रहभागहम्}


\twolineshloka
{श्रुता मे मानवा धर्मा वासिष्ठाः काश्यपास्तथा}
{गार्गीया गौतमीयाश्च तथा गोपालकस्य च}


\twolineshloka
{पराशरकृताः पूर्वा मैत्रेयस्य च धीमतः}
{औमा माहेश्वराश्चैव नन्दिधर्माश्च पावनाः}


\twolineshloka
{ब्रह्मणा कथिता ये च कौमाराश्च श्रुता मया}
{धूमायनकृता धर्माः काण्डवैश्वानरा अपि}


\twolineshloka
{भार्गवा याज्ञवल्क्याश्च मार्कण्डेयकृता अपि}
{भारद्वाजकृता ये च बृहस्पतिकृताश्च ये}


\twolineshloka
{कुणेश्च कुणिबाहोश्च विश्वामित्रकृताश्च ये}
{सुमन्तुजैमिनिकृताः शाकुनेयास्तथैव च}


\twolineshloka
{पुलस्त्यपुलहोद्गीताः पावकीयास्तथैव च}
{अगस्त्यगीता मौद्गल्याः शाण्डिल्याः शलभायनाः}


\twolineshloka
{वालखिल्यकृता ये च ये च सप्तर्षिभिस्तथा}
{आपस्तंबकृता धर्माः शङ्खस्य लिखितस्य च}


\twolineshloka
{प्राजापत्यास्तथा याम्या माहेन्द्राश्च श्रुता मया}
{वैयाघ्रव्यासकीयाश्च विभण्डककृताश्च ये}


\twolineshloka
{नारदीयाः श्रुता धर्माः कापोताश्च श्रुता मया}
{तथा विदुरवाक्यानि भृगोरङ्गिरसस्तथा}


\twolineshloka
{क्रौञ्चा मृदङ्गगीताश्च सौर्या हारीतकाश्च ये}
{ये पिशङ्गकृताश्चापि कातपायाः सुवालकाः}


\twolineshloka
{उद्दालककृता धर्मा औशनस्यास्तथैव च}
{वैशंपायनगीताश्च ये चान्येऽप्येवमादितः}


\twolineshloka
{एतेभ्यः सर्वधर्मेभ्यो देव त्वन्मुखनिस्सृताः}
{पावनात्वात्पवित्रत्वाद्विशिष्टा इति मे मतिः}


\threelineshloka
{तस्माद्धि त्वां प्रपन्नस्य त्वद्भक्तस्य च केशव}
{युष्मदीयान्वरान्धर्मान्पुण्यान्कथय मेच्युत ॥वैशंपायन उवाच}
{}


\twolineshloka
{एवं पृष्टस्तु धर्मज्ञो धर्मपुत्रेण केशवः}
{उवाच धर्मान्सूक्ष्मार्थान्धर्मपुत्रस्य हर्षितः}


\twolineshloka
{एवं ते यस्य कौन्तेय यत्नो धर्मेषु सुव्रत}
{तस्य ते दुर्लभो लोके न कश्चिदपि विद्येत}


\twolineshloka
{धर्मः श्रुतो वा दृष्टो वा कथितो वा कृतोपि वा}
{अनुमोदितो वा राजेन्द्र नयतीन्द्रपदं नरम्}


\twolineshloka
{धर्मः पिता च माता च धर्मो नाथः सुहृत्तथा}
{धर्मो भ्राता सखा चैव धर्मः स्वामी परंतप}


\twolineshloka
{धर्मादर्थश्च कामश्च धर्माद्भोगाः सुखानि च}
{धर्मार्दैश्वर्यमेवाग्र्यं धर्मात्स्वर्गगतिः परा}


\twolineshloka
{धर्मोयं सेवितः शुद्धस्त्रायते महतो भयात्}
{धर्माद्द्विजत्वं देवत्वं धर्मः पावयते नरम्}


\twolineshloka
{यदा च क्षीयते पापं कालेन पुरुषस्य तु}
{तदा संजायते बुद्धिर्धर्मं कर्तुं युधिष्ठिर}


\twolineshloka
{जन्मान्तरसहस्रैस्तु मनुष्यत्वं हि दुर्लभम्}
{तद्गत्वापीह यो धर्मं न करोति स्ववञ्चितः}


\twolineshloka
{कुत्सिता ये दरिद्राश्च विरूपा व्याधितास्तथा}
{परद्वेष्याश्च मूर्खाश्च न तैर्धर्मः कृतः पुरा}


% Check verse!
ये च दीर्घायुषः शूराः पण्डिता भोगिनस्तथा ॥नीरोगा रूपसंपन्नास्तैर्धर्मः सुकृतः पुरा
\twolineshloka
{एवं धर्मः कृतः शुद्धो नयते गतिमुत्तमाम्}
{अधर्मं सेवते यस्तु तिर्यग्योन्यां पतत्यसौ}


\twolineshloka
{इदं रहस्यं कौन्तेय शृणु धर्ममनुत्तमम्}
{कथयिष्ये परं धर्मं तव भक्तस्य पाण्डव}


\twolineshloka
{इष्टस्त्वमसि मेऽत्यर्थं प्रपन्नश्चापि मां सदा}
{परमार्थमपि ब्रूयां किं पुनर्धर्मसंहिताम्}


\twolineshloka
{इदं मे मानुषं जन्म कृतमात्मनि मायया}
{धर्मसंस्थापनार्थाय दुष्टानां नाशनाय च}


\twolineshloka
{मानुष्यं भावमापन्नं ये मां गृह्णन्त्यवज्ञया}
{संसारान्तर्हि ते मूढास्तिर्यग्योनिष्वनेकशः}


\twolineshloka
{ये च मां सर्वभूतस्थं पश्यन्ति ज्ञानचक्षुषा}
{मद्भक्तांस्तान्सदा युक्तान्मत्समीपं नयाम्यहम्}


\twolineshloka
{मद्भक्ता न विनश्यन्ति मद्भक्ता वीतकल्मषाः}
{मद्भाक्तानां तु मानुष्ये सफल जन्म पाण्डवा}


\twolineshloka
{अपि पापेष्वभिरता मद्भक्ताः पाण्डुनन्दन}
{मुच्यन्ते पातकैः सर्वैः पद्मपत्रमिवांभसा}


\twolineshloka
{जन्मान्तरसहस्रेषु तपसा भावितात्मनाम्}
{भक्तिरुत्पद्यते तात मनुष्याणां न संशयः}


\twolineshloka
{यच्च रूपं परं गुह्यं कूटस्तमचलं ध्रुवम्}
{न दृश्यते तता देवैर्मद्भक्तैर्दृश्यते यथा}


\twolineshloka
{अपरं यच्च मे रूपं पर्रादुर्भावेषु दृश्यते}
{तदर्चयन्ति सर्वार्थैः सर्वभूतानि पाण्डव}


\twolineshloka
{कल्पकोटिसहस्रेषु व्यतीतेष्वागते च}
{दर्शयामीह तद्रूपं यच्च पश्यन्ति मे सुराः}


\twolineshloka
{स्तित्युत्पत्त्यव्ययकरं यो मां ज्ञात्वा प्रपद्यते}
{अनुगृह्णाम्यहं तं वै संसारान्मोचयामि च}


\twolineshloka
{अहमादिर्हि देवानां सृष्टा ब्रह्मादयो मया}
{प्रकृतिं स्वामवष्टभ्य जगत्सर्वं सृजाम्यहम्}


\twolineshloka
{तमोमूलोहमव्यक्तो रजोमध्ये प्रतिष्ठितः}
{ऊर्ध्वं सत्त्वं विना लोभं ब्रह्मादिस्तंबपर्यतः}


\twolineshloka
{मूर्धानं मे विद्धि दिवं चन्द्रादित्यौ च लोचने}
{गावोग्निर्बाह्मणो वक्त्रं मारुतः श्वसनं च मे}


\threelineshloka
{दिशो मे बाहवश्चाष्टौ नक्षत्राणि च भूषणम्}
{अन्तरिक्षमुरो विद्धि सर्वभूतावकाशकम्}
{मार्गो मेघानिलाभ्यां तु यन्ममोदरमव्ययम्}


\twolineshloka
{पृथिवीमण्डलं यद्वै द्वीपार्णवनगैर्युतम्}
{सर्वसंधारणोपेतं पादौ मम युधिष्ठिर}


\twolineshloka
{स्थितो ह्येकगुणः खेऽहं द्विगुणश्चास्मि मारुते}
{त्रिगुणोग्नौ स्थितोहं वै सलिले च चतुर्गुणः}


\twolineshloka
{शब्दाद्या ये गुणाः पञ्च महाभूतेषु पञ्चसु}
{तन्मात्रासंस्थितः सोहं पृथिव्यां पञ्चधास्थितः}


\twolineshloka
{अहं सहस्रसीर्षस्तु सहस्रवदनेक्षणः}
{सहस्रबाहूदरधृक्सहस्रोरुः सहस्रपात्}


\twolineshloka
{धृत्वोर्वीं सर्वतः सम्यगत्यतिष्ठं दशांगुलम्}
{सर्वभूतात्मभूतस्थः सर्वव्यापी ततोस्म्यहम्}


\twolineshloka
{अचिन्त्योहमनन्तोहमजरोहमजो ह्यहम्}
{अनाद्योऽहमवध्योहमप्रमेयोहमव्ययः}


\twolineshloka
{निर्गुणोहं निगूढात्मा निर्द्वन्द्वो निर्ममो नृप}
{निष्कलो निर्विकारोहं निदानममृतस्य तु}


\twolineshloka
{सुधा चाहं स्वधा चाहं स्वाहा चाहं नराधिप}
{तेजसा तपसा चाहं भूतग्रामं चतुर्विधम्}


\threelineshloka
{स्नेहपाशैर्गुणबद्ध्वा धारयाम्यात्ममायया}
{चातुराश्रमधर्मेहं चातुर्होत्रफलाशनः}
{चतुर्मूर्तिश्चतुर्यज्ञश्चतुराश्रमभावनः}


\twolineshloka
{संहृत्याहं जगत्सर्वं कृत्वा वै गर्भमात्मनः}
{शयामि दिव्ययोगेन प्रलयेषु युधिष्ठिर}


\twolineshloka
{सहस्रयुगपर्यन्तां ब्राह्मीं रात्रिं महार्णवे}
{स्थित्वा सृजामि भूतानि जङ्गमानि स्थिराणि च}


\twolineshloka
{कल्पे कल्पे च भूतानि संहरामि सृजामि च}
{न च मां तानि जानन्ति मायया मोहितानि मे}


\twolineshloka
{मम चैवान्धकारस्य मार्गितव्यस्य नित्यशः}
{प्रशान्तस्येव दीपस्य गतिर्नैपोपलभ्यते}


\twolineshloka
{न तदस्ति क्वचिद्राजन्यत्राहं न प्रतिष्ठितः}
{न च तद्विद्यते भूतं मयि यन्न प्रतिष्ठितम्}


\twolineshloka
{यावन्मित्रं भवेद्भूतं स्थूलं सूक्ष्ममिदं जगत्}
{दीवभूतो ह्यहं तस्मिंस्तावन्मात्रं प्रतिष्ठितः}


\twolineshloka
{किंचात्र बहुनोक्तेन सत्यमेतद्ब्रवीमि ते}
{यद्भूतं यद्भविष्यच्च तत्सर्वमहमेव तु}


\twolineshloka
{मया सृष्टानि भूतानि मन्मयानि च भारत}
{मामेव न विजानन्ति मायया मोहितानि वै}


\twolineshloka
{एवं सर्वं जगदिदं सदेवासुरमानुषम्}
{मत्तः प्रभवते राजन्मय्येव प्रविलीयते}


\chapter{अध्यायः ९७}
\twolineshloka
{एवमात्मोद्भवं सर्वं यगदुद्दिश्य केशवः}
{धर्मान्धर्मात्मजस्याथ पुण्यानकथयत्प्रभुः}


\twolineshloka
{शृणु पाण्डव तत्वेन पवित्रं पापनाशनम्}
{कथ्यमानं मया पुण्यं धर्मशास्त्रफलं महत्}


\twolineshloka
{यः शृणोति शुचिर्भूत्वा एकचित्तस्तपोयुतः}
{स्वर्ग्यं यशस्यामायुष्यं धर्मं ज्ञेयं युधिष्ठिर}


\threelineshloka
{श्रद्दधानस्य तस्येह यत्पापं पूर्वसंचितम्}
{विनश्यत्याशु तत्सर्वं मद्भक्तस्य विशेषतः ॥वैशंपायन उवाच}
{}


\twolineshloka
{एवं श्रुत्वा वचः पुण्यं सत्यं केशवभाषितम्}
{प्रहृष्टमनसो भूत्वा चिन्तयन्तोऽद्भुतं परम्}


\twolineshloka
{देवब्रह्मर्षयः सर्वे गन्धर्वाप्सरसस्तथा}
{भूता यक्षग्रहाश्चैव गुह्यका भुजगास्तथा}


\twolineshloka
{वालखिल्या महात्मानो योगिनस्तत्वदर्शिनः}
{तथा भागवताश्चापि पञ्चकालमुपासकाः}


\threelineshloka
{कौतूहलसमाविष्टाः प्रहृष्टैन्द्रियमानसाः}
{श्रोतुकामाः परं धर्मं वैष्णवं धर्मशासनम्}
{हृदि कर्तुं च तद्वाक्यं प्रणेमुः शिरसा नताः}


\fourlineindentedshloka
{ततस्तान्वासुदेवेन दृष्टान्दिव्येन चक्षुषा}
{विमुक्तपापानालोक्य प्रणम्य शिरसा हरिम्}
{पप्रच्छ केशवं धर्मं धर्मपुत्रः प्रतापवान् ॥युधिष्ठिर उवाच}
{}


\twolineshloka
{कीदृशी ब्राह्मणस्याथ क्षत्रियस्यापि कीदृशी}
{वैश्यस्य कीदृशी देव गतिः शूद्रस्य कीदृशी}


\fourlineindentedshloka
{कथं बध्येत पाशेन ब्राह्मणस्तु यमालये}
{क्षत्रियो वाथ वैश्यो वा शूद्रो वा बध्यते कथम्}
{एतत्कर्मफलं ब्रूहि लोकनाथ नमोस्तु ते ॥वैशंपायन उवाच}
{}


\threelineshloka
{पृष्टोऽथ केशवो ह्येवं धर्मपुत्रेण धीमता}
{उवाच संसारगतिं चातुर्वर्ण्यस्य कर्मजाम् ॥भगवानुवाच}
{}


\twolineshloka
{शृणु वर्णक्रमेणैव धर्मं धर्मभृतां वर}
{नास्ति किंचिन्नरश्रेष्ठ ब्राह्मणस्य तु दुष्कृतम्}


\twolineshloka
{शिखांयज्ञोपवीता ये सन्ध्यां ये चाप्युपासते}
{यैश्च पूर्णाहुतिः प्राप्ता विधिवज्जुह्वते च ये}


\twolineshloka
{वैश्वदेवं च ये चक्रुः पूजयन्त्यतिथींश्च ये}
{नित्यं स्वाध्यायशीलाश्च जपयज्ञपराश्च ये}


\threelineshloka
{सायंप्रातर्हुताशाश्च शूद्रभोजनवर्जिताः}
{डंभानृतविमुक्ताश्च स्वदारनिरताश्च ये}
{पञ्चयज्ञपरा ये च येऽग्निहोत्रमुपासते}


\twolineshloka
{दहन्ति दुष्कृतं येषां हूयमानास्त्रयोऽग्नयः}
{नष्टदुष्कृतकर्माणो ब्रह्मलोकं व्रजन्ति ते}


\threelineshloka
{ब्रह्मलोके पुनः कामं गन्धर्वैर्ब्रह्मगायकैः}
{उद्गीयमानाः प्रयतैः पूज्यमानाः स्वयंभुवा}
{ब्रह्मिलोके प्रमोदन्ते यावदाभूतसंप्लुवम्}


\twolineshloka
{क्षत्रियोपि स्थितो राज्ये स्वधर्मपरिपालकः}
{सम्यक्प्रजाः पालयिता षड्भागनिरतः सदा}


\twolineshloka
{यज्ञदानरतो धीरः स्वदारनिरतः सदा}
{शास्त्रानुसारी तत्वज्ञः प्रजाकार्यपरायणः}


\threelineshloka
{विप्रेभ्यः कामदो नित्यं भृत्यानां भरणे रतः}
{सत्यसन्धः शुचिर्नित्यं लोभडंभविवर्जितः}
{क्षत्रियोप्युत्तमां याति गतिं देवनिषेविताम्}


\twolineshloka
{तत्र दिव्याप्सरोभिस्तु गन्धर्वैश्च विशेषतः}
{सेव्यमानो महातेजाः क्रीडते शक्रपूजितः}


\twolineshloka
{चतुर्थगानि वै त्रिंशत्क्रीडित्वा तत्र देववत्}
{इह मानुष्यलोके तु चतुर्वेदी द्विजो भवेत्}


\twolineshloka
{कृषिगोपालनिरतो धर्मानवेषणतत्परः}
{दानधर्मेपि निरतो विप्रशुश्रूषकस्तथा}


\twolineshloka
{सत्यसन्धः शुचिर्नित्यं लोभडंभविवर्जितः}
{ऋजुः स्वदारनिरतो हिंसाद्रोहविवर्जितः}


\twolineshloka
{वणिग्धर्मान्नमुञ्चन्वै देवब्राह्मणपूजकः}
{वैश्यः स्वर्गतिमाप्नोति पूज्यमानोप्सरोगणैः}


\twolineshloka
{चतुर्युगानि वै त्रिंशत्क्रीडित्वा दश पञ्च च}
{इह मानुष्यलोके च राजा भवति वीर्यवान्}


\twolineshloka
{सुवर्णकोट्यः पञ्चाशद्रत्नानां च शतं तथा}
{हस्त्यश्वरश्वसंयुक्तो महाभोगांश्च सेवते}


\twolineshloka
{त्रयाणामपि वर्णानां शुश्रूषानिरतः सदा}
{विशेषतस्तु विप्राणां दासवद्यस्तु तिष्टति}


\twolineshloka
{अयाचितप्रदाता च सत्यशौचसमन्वितः}
{गुरुदेवार्चनरतः परदारविवर्जितः}


\twolineshloka
{परपीडामकृत्वैव भत्यवर्गं बिभर्ति यः}
{शूद्रोपि स्वर्गमाप्नोति जीवानामभयप्रदः}


\twolineshloka
{स स्वर्गकलोके क्रीडित्वा वर्षकोटिं महातपाः}
{इह मानुषलोके तु वैश्यो धनपतिर्भवेत्}


\twolineshloka
{एवं धर्मात्परं नास्ति महत्संसारमोक्षणम्}
{न च धर्मात्पर किंचित्पापकर्मव्यपोहनम्}


\twolineshloka
{तस्माद्धर्मः सदा कार्यो मानुष्यं प्राप्य दुर्लभम्}
{न हि धर्मानुरक्तानां लोके किंचन दुर्लभम्}


\twolineshloka
{स्वयंभुविहितो धर्मो यो यस्येह नरेश्वर}
{स तेन क्षपयेत्पापं सम्यगाचरितेन च}


\twolineshloka
{सहजं यद्भदेत्कर्म न तत्त्याज्यं हि केनचित्}
{स एव तस्य धर्मो हि तेन सिद्धिं स गच्छति}


\threelineshloka
{विगुणोपि स्वधर्मस्तु पापकर्म व्यपोहति}
{एवमेव तु धर्मोपि क्षीयते पापवर्धनात् ॥युधिष्ठिर उवाच}
{}


\threelineshloka
{भगवन्देवदेवेश श्रोतुं कौतूहलं हि मे}
{शुभस्याप्यशुभस्यापि क्षयवृद्धी यथाक्रमम् ॥भगवानुवाच}
{}


\twolineshloka
{शृणु पार्थिव तत्सर्वं धर्मसूक्ष्मं सनातनम्}
{दुर्विज्ञेयतमं नित्यं यत्र मग्ना महाजनाः}


\twolineshloka
{यथैव शीतमुदकमुष्णेन बहुना वृतम्}
{भवेत्तु तत्क्षणादुष्णं शीतत्वं च विनश्यति}


\twolineshloka
{यथोष्णं वा भवेदल्पं शीतेन बहुना वृतम्}
{शीतलं च भवेत्सर्वमुष्णत्वं च विनश्यति}


\twolineshloka
{एवं च यद्भवेद्भूरि सुकृतं वाऽपि दुष्कृतम्}
{तदल्पं क्षपयेच्छीघ्रं नात्र कार्या विचारणा}


\twolineshloka
{समत्वे सति राजेन्द्र तयोः सुकृतपापयोः}
{गूहितस्य भवेद्वृद्धिः कीर्तितस्य भवेत्क्षयः}


\twolineshloka
{ख्यापनेनानुतापेन प्रायः पापं विनश्यति}
{तथा कृतस्तु राजेन्द्र धर्मो नश्यति मानद}


\twolineshloka
{तावुभौ गूहितौ सम्यग्वृद्धिं यातो न संशयः}
{तस्मात्सर्वप्रयत्नेन न पापं गूहयेद्बुधः}


\twolineshloka
{तस्मादेतत्प्रयत्नेनि कीर्तयेत्क्षयकारणात्}
{तस्मात्संकीर्तयेत्पापं नित्यं धर्मं च गूहयेत्}


\chapter{अध्यायः ९८}
\twolineshloka
{एवं श्रुत्वा वचस्तस्य धर्मपुत्रोऽच्युतस्य तु}
{पप्रच्छ पुनरप्यन्यं धर्मं धर्मात्मजो हरिम्}


\twolineshloka
{वृथा च कति जन्मानि वृथा दानानि कानि च}
{वृथा च जीवितं केषां नराणां पुरुषोत्तम}


\twolineshloka
{कीदृशासु ह्यवस्थासु दानं दत्तं जनार्दन}
{इह लोकेऽनुभवति पुरुषः पुरुषोत्तम}


\twolineshloka
{गर्भस्थः किं समश्नाति किं बाल्ये वाऽपि केशव}
{यौवनस्थेऽपि किं कृष्ण वार्धके वाऽपि किं भवेत्}


\twolineshloka
{सात्विकं कीदृशं दानं राजसं कीदृशं भवेत्}
{तामसं कीदृशं देव तर्पयिष्यति किं प्रभो}


\threelineshloka
{उत्तमं कीदृशं दानं तेषां वा किं फलं भवेत्}
{किं दानं नयति ह्यूर्ध्वं किं गतिं मध्यमां नयेत्}
{गतिं जघन्यामथवा देवदेव ब्रवीहि मे}


\threelineshloka
{एतदिच्छामि विज्ञातुं परं कौतूहलं हि मे}
{त्वदीयं वचनं सत्यं पुण्यं च मधुसूदन ॥वैशंपायन उवाच}
{}


\threelineshloka
{एवं धर्मं प्रयत्नेन पृष्टः पाण्डुसुतेन वै}
{उवाच वासुदेवोऽथ धर्मान्धर्मात्मजस्य तु ॥भगवानुवाच}
{}


\twolineshloka
{शृणु राजन्यथान्यायं वचनं तथ्यमुत्तमम्}
{कथ्यमानं मया पुण्यं सर्वपापव्यपोहनम्}


\twolineshloka
{वृथा च दश जन्मानि चत्वारि च नराधिप}
{वृथा दानानि पञ्चाशत्पञ्चैव च यथाक्रमम्}


\twolineshloka
{वृथा च जीवितं येषां ते च षट् परिकीर्तिताः}
{अनुक्रमेण वक्ष्यामि तानि सर्वाणि पार्थिव}


\threelineshloka
{धर्मघ्नानां वृथा जन्म लुब्धानां पापिनां तथा}
{वृथा पाकं च येऽश्नन्ति परदाररताश्च ये}
{पाकभेदकरा ये च ये च स्युः सत्यवर्जिताः}


\threelineshloka
{मृष्टमश्नाति यश्चैकः क्लिस्यमानैस्तु बान्धवैः}
{पितरं मातरं चैव उपाध्यायं गुरुं तथा}
{मातुलं मातुलानीं च यो निहन्याच्छपेत वा}


\twolineshloka
{ब्राह्मणश्चैव यो भूत्वा सन्ध्योपासनवर्जितः}
{निस्स्वाहो निस्स्वधश्चैव शूद्राणामन्नभुग्द्विजः}


\threelineshloka
{मम वा शंकरस्याथ ब्रह्मणो वा युधिष्ठिर}
{अथवा ब्राह्मणानां तु ये न भक्ता नराधमाः}
{वृथाजन्मान्यथैतेषां पापिनां विद्धि डव}


\twolineshloka
{अश्रद्धयाऽपि यद्दत्तमवामानेन वाऽपि यत्}
{डंभार्थमपि यद्दत्तं यत्पाषण्डिहृतं नृप}


\twolineshloka
{शूद्राचाराय यद्दत्तं यद्दत्त्वा चानुकीर्तितम्}
{रोषयुक्तं तु यद्दत्तं यद्दत्तमनुशोचितम्}


\twolineshloka
{डंभार्जितं च यद्दत्तं यच्च वाऽप्यनृतार्जितम्}
{ब्राह्मणस्वं च यद्दत्तं चौर्येणाप्यार्जितं च यत्}


\twolineshloka
{अभिशास्ताहृतं यत्तु यद्दत्तं पतिते द्विजे}
{निर्ब्रह्माभिहृतं यत्तु यद्दत्तं सर्वयाचकैः}


\twolineshloka
{व्राप्यैस्तु यद्धृतं दानमारूढपतितैश्च यत्}
{यद्दत्तं स्वैरिणीभर्तुः श्वशुराननुवर्तिने}


\twolineshloka
{यद्ग्रामयाचकहृतं यत्कृतघ्नहृतं तथा}
{उपपातकिने दत्तं देवविक्रयिणे च यत्}


\twolineshloka
{स्त्रीजिताय च यद्दत्तं यद्दत्तं राजसेविने}
{गणकाय च यद्दत्तं यच्च कारणिकाय च}


\twolineshloka
{वृषलीपतये दत्तं यद्दत्तं शस्त्रजीविने}
{भृतकाय च यद्दत्तं व्यालग्राहिहृतं च यत्}


\twolineshloka
{पुरोहिताय यद्दत्तं चिकित्सकहृतं च यत्}
{यद्वणिंक्कर्मिणे दत्तं क्षुद्रमन्त्रोपजीविने}


\twolineshloka
{यच्छूद्रजीविने दत्तं यच्च देवलकाय च}
{देवद्रव्याशिने यच्च यद्दत्तं चित्रकर्मिणे}


\twolineshloka
{रङ्गोपजीविने दत्तं यच्च मांसोपजीविने}
{सेवकाय च यद्दत्तं यद्दत्तं ब्राह्मणब्रुवे}


\twolineshloka
{अदेशिने च यद्दत्तं दत्तं वार्धुषिकाय च}
{यदनाचारिणे दत्तं यत्तु दत्तमनग्रये}


\twolineshloka
{असन्ध्योपासिने दत्तं यच्छूद्रग्रामवासिने}
{यन्मिथ्यालिङ्गिने दत्तं दत्तं सर्वाशिने च यत्}


\twolineshloka
{नास्तिकाया च यद्दत्तं धर्मविक्रयिणे च यत्}
{वराकाय च यद्दत्तं यद्दत्तं कूटसाक्षिणे}


\twolineshloka
{ग्रामकूटाय यद्दत्तं दानं पार्तिवपुङ्गव}
{वृथा भवति तत्सर्वं नात्र कार्या विचारणा}


\twolineshloka
{विप्राणामधरा एते लोलुपा ब्राह्मणाधमाः}
{नात्मानं तारयन्त्येते न दातारं युधिष्ठिर}


\twolineshloka
{एतेभ्यो दत्तमात्राणि दानानि सुबहून्यपि}
{वृथा भवन्ति राजेन्द्र भस्मन्याज्याहुतिर्यथा}


\twolineshloka
{एतेषु यत्फलं किंचिद्भविष्यति कथंचन}
{राक्षसाश्च पिशाचाश्च तद्विलुंपन्ति हर्षिताः}


\twolineshloka
{वृथा ह्येतानि दत्तानि कथितानि समासतः}
{जीवितं तु वृथा येषां चच्छृणुष्व युधिष्ठिर}


\twolineshloka
{ये मां न प्रतिपद्यन्ते शङ्करं वा नराधमाः}
{ब्रैह्मणान्वा महीदेवान्वृथा जीवन्ति ते नराः}


\twolineshloka
{हेतुशास्त्रेषु ये सक्ताः कुदृष्टिपथमाश्रिताः}
{देवान्निन्दन्त्यनाचारा वृथा जीवन्ति ते नराः}


\twolineshloka
{कुशलैः कृतशास्त्राणि पठित्वा ये नराधमाः}
{विप्रान्निन्दन्ति यज्ञांश्च वृथा जीवन्ति ते नराः}


\threelineshloka
{ये दुर्गां वा कुमारं वा वायुमग्निं जलं रविम्}
{पितरं मातरं चैव गुरुमिन्द्रं निशाकरम्}
{मूढा निन्दन्त्यनाचारा वृथा जीवन्ति ते नराः}


\twolineshloka
{विद्यमाने धने यस्तु दानधर्मविवर्जितः}
{मृष्टमश्नाति यश्चैको वृथा जीवति सोपि च}


% Check verse!
वृथा जीवितमाख्यातं दानकालं ब्रवीमि ते
\twolineshloka
{तमोनिविष्टचित्तेन दत्तं दानं तु यद्भवेत्}
{तदस्य फलमश्नाति नरो गर्भगतो नृप}


\twolineshloka
{ईर्ष्यामत्सरसंयुक्तो डंभार्थं चार्थकारणात्}
{ददाति दानं यो मर्त्यो बालभावे तदश्नुते}


\twolineshloka
{भोक्तं भोगमशक्तस्तु व्याधिभिः पीडितो भृशम्}
{ददाति दानं यो मर्त्यो वृद्धभावे तदश्नुते}


\twolineshloka
{श्रद्धायुक्तः शुचिः स्नातः प्रसन्नेन्द्रियमानसः}
{ददाति दानं यो मर्त्यो यौवने स तदश्नुते}


\twolineshloka
{स्वयं नीत्वा तु यद्दानं भक्त्या पात्रे प्रदीयते}
{तत्सार्वकालिकं विद्धि दानमामरणान्तिकम्}


\twolineshloka
{राजसं सात्विकं चापि तामसं च युधिष्ठिर}
{दानं दानफलं चैव गतिं च त्रिविधां शृणु}


\twolineshloka
{दानं दातव्यमित्येव मतिं कृत्वा द्विजाय वै}
{उपकारवियुक्ताय यद्दत्तं तद्धि सात्विकम्}


\twolineshloka
{श्रोत्रियाय दरिद्राय बहुभृत्याय पाण्डव}
{दीयते यत्प्रहृष्टेन तत्सात्विकमुदाहृतम्}


\twolineshloka
{वेदाक्षरविहीनाय यत्तु पूर्वोपकारिणे}
{समृद्धाय च यद्दत्तं तद्दानं राजसं स्मृतम्}


\twolineshloka
{संबन्धिने च यद्दत्तं प्रमत्ताय च पाण्डव}
{फलार्थिभिरपात्राय तद्दानं राजसं स्मृतम्}


\twolineshloka
{वैश्वदेवविहीनाय दानमश्रोत्रियाय च}
{दीयते तस्करायापि तद्दानं तामसं स्मृतम्}


\twolineshloka
{सरोषमवधूतं च क्लेशयुक्तमवज्ञया}
{सेवकाय च यद्दतं तत्तामसमुदाहृतम्}


\twolineshloka
{देवाः पितृगणाश्चैव मुनयश्चाग्नयस्तथा}
{सात्विकं दानमश्नन्ति तुष्यन्ति च नरेश्वर}


\twolineshloka
{दानवा दैत्यसङ्गाश्च ग्रहा यक्षाः सराक्षसाः}
{राजसं दानमश्नन्ति वर्जितं पितृदैवतैः}


\twolineshloka
{पिशाचाः प्रेतसङ्घाश्च कश्मला ये मलीमसाः}
{तामसं दानमश्नन्ति गतिं च त्रिविधां शृणु}


\twolineshloka
{सात्विकानां तु दानानामुत्तमं फलमश्नुते}
{मध्यमं राजसानां तु तामसानां तु पश्चिमम्}


\twolineshloka
{अभिगम्योपनीतानां दानानामुत्तमं फलम्}
{मध्यमं तु समाहूय जघन्यं याचते फलम्}


\threelineshloka
{अयाचितप्रदाता यः स याति गतिमुत्तमाम्}
{समाहूय तु यो दद्यान्मध्यमां स गतिं व्रजेत्}
{याचितो यश्च वै दद्याज्जघन्यां स गतिं व्रजेत्}


\twolineshloka
{उत्तमा दैविकी ज्ञेया मध्यमा मानुषी गतिः}
{गतिर्जघन्या तिर्यक्षु गतिरेषा त्रिधा स्मृता}


\twolineshloka
{पात्रभूतेषु विप्रेषु संस्थितेष्वाहिताग्निषु}
{यत्तु निक्षिप्यते दानमक्षयं संप्रकीर्तितम्}


\twolineshloka
{श्रोत्रियाणां दरिद्राणां भरणं कुरु पार्थिव}
{समृद्धानां द्विजातीनां कुर्यास्तेषां तु रक्षणम्}


\twolineshloka
{दरिद्रान्वित्तहीनांश्च प्रदानैः सुष्ठु पूजय}
{आतुरस्यैषधैः कार्यं नीरुजस्य किमौषधैः}


\threelineshloka
{पापं प्रतिग्रहीतारं प्रदातुरपगच्छति}
{प्रतिग्रहीतुर्यत्पुण्यं प्रदातारमुपैति तत्}
{तस्माद्दानं सदा कार्यं परत्र हितमिच्छता}


\twolineshloka
{वेदविद्यावदातेषु सदा शूद्रान्नवर्जिषु}
{प्रयत्नेन विधातव्यो महादानमयो निधिः}


\twolineshloka
{येषां दाराः प्रतीक्षन्ते सहस्रस्येव लम्भनम्}
{भुक्तशेषस्य भक्तस्य तान्निमन्त्रय पाण्डव}


\twolineshloka
{आमन्त्र्य तु निराशानि न कर्तव्यानि भारत}
{कुलानि सुदरिद्राणि तेषामाशा हता भवेत्}


\twolineshloka
{मद्भक्ता ये नरश्रेष्ठ मद्गता मत्परायणाः}
{मद्याजिनो मन्नियमास्तान्प्रयत्नेन पूजयेत्}


\twolineshloka
{तेषां तु पावनायाहं नित्यमेव युधिष्ठिर}
{उभे सन्ध्येऽधितिष्ठामि ह्यस्कन्नं तद्व्रतं मम}


\twolineshloka
{तस्मादष्टाक्षरं मन्त्रं मद्भक्तैर्वीतकल्मषैः}
{सन्ध्याकाले तु जप्तव्यं सततं चात्मशुद्धये}


\threelineshloka
{अन्येषामपि विप्राणां किल्बिषं हि विनश्यति}
{उभे सन्ध्येप्युपासीत तस्माद्विप्रो विशुद्धये}
{}


\twolineshloka
{दैवे श्राद्धेपि विप्रः स नियोक्तव्योऽजुगुप्सया}
{जुगुप्सितस्तु यः श्राद्धं दहत्यग्निरिवेन्धनम्}


\twolineshloka
{भारतं मानवो धर्मो वेदाः साङ्गाश्चिकित्सितम्}
{आज्ञासिद्धानि चत्वारि न हन्तव्यानि हेतुभिः}


\twolineshloka
{न ब्राह्मणान्परीक्षेत दैवे कर्मणि धर्मवित्}
{महान्भवेत्परीवादो ब्राह्मणानां परीक्षणे}


\twolineshloka
{ब्राह्मणानां परीवादं यः कुर्यात्स नराधमः}
{रासभानां शुनां योनिं गच्छेत्पुरुषदूषकः}


\twolineshloka
{श्वत्वं प्राप्नोति निन्दित्वा परीवादात्खरो भवेत्}
{कृमिर्भवत्यभिभवात्कीटो भवति मत्सरात्}


\twolineshloka
{दुर्वृत्ता वा सुवृत्ता वा प्राकृता वा सुसंस्कृताः}
{ब्राह्मणा नावमन्तव्या भस्मच्छन्ना इवाग्नयः}


\twolineshloka
{क्षत्रियं चैव सर्पं च ब्राह्मणं च बहुश्रुतम्}
{नावमन्येत मेधावी कृशानपि कदाचन}


\twolineshloka
{एतत्त्रयं हि पुरुषं निर्दहेदवमानितम्}
{तस्मादेतत्प्रयत्नेन नावमन्येत बुद्धिमान्}


\twolineshloka
{यथा सर्वास्ववस्थासु पावको दैवतं महत्}
{तथा सर्वास्ववस्थासु ब्राह्मणो दैवतं महत्}


\twolineshloka
{व्यङ्गाः काणाश्च कुब्जाश्च वामनाङ्गास्तथैव च}
{सर्वे दैवे नियोक्तव्या व्यामिश्रा वेदपारगैः}


\twolineshloka
{मन्युं नोत्पादयेत्तेषां न चारिष्टं समाचरेत्}
{मन्युप्रहरणा विप्रा न विप्राः शश्त्रपाणयः}


\twolineshloka
{मन्युना घ्नन्ति ते शत्रून्वज्रेणेन्द्र इवासुरान्}
{ब्राह्मणो हि महद्दैवं जातिमात्रेण जायते}


\twolineshloka
{ब्राह्मणः सर्वभूतानां धर्मकोशस्य गुप्तये}
{किं पुनर्ये च कौन्तेय सन्ध्यां नित्यमुपासते}


\twolineshloka
{यस्यास्येन समश्नन्ति हव्यानि त्रिदिवौकसः}
{कव्यानि चैव पितरः किं भूतमधिकं ततः}


\twolineshloka
{उत्पत्तिरेव विप्रस्य मूर्तिधर्मस्य शाश्वती}
{स हि धर्मार्थमुत्पन्नो ब्रह्मभूयाय कल्पते}


\threelineshloka
{स्वमेव ब्राह्मणो भुङ्क्ते स्वं वस्ते स्वं ददाति च}
{आनृशंस्याद्ब्राह्मणस्य भुञ्जते हीतरे जनाः}
{तस्मात्ते नावमन्तव्या मद्भक्ता हि द्विजाः सदा}


\twolineshloka
{आरण्यकोपनिषदि ये तु पश्यन्ति मां द्विजाः}
{निगूढं निष्कलावस्थं तान्प्रयत्नेन पूजय}


\twolineshloka
{स्वगृहे वा प्रवासे वा दिवारात्रमथापि वा}
{श्रद्धया ब्राह्मणाः पूज्या मद्भक्ता ये च पाण्डव}


\fourlineindentedshloka
{नास्ति विप्रसमं दैवं नास्ति विप्रसमो गुरुः}
{नास्ति विप्रात्परो बन्धुर्नास्ति विप्रात्परो निधिः}
{नास्ति विप्रात्परं तीर्थं न पुण्यं ब्राह्मणात्परम्}
{}


\twolineshloka
{न पवित्रं परं विप्रान्न द्विजात्पावनं परम्}
{नास्ति विप्राप्तरो धर्मो नास्ति विप्रात्परा गतिः}


\twolineshloka
{पापकर्मसमाक्षिप्तं पतन्तं नरके नरम्}
{त्रायते पात्रमप्येकं पात्रभूते तु तद्द्विजे}


\twolineshloka
{बालाहिताग्नयो ये च शान्ताः शूद्रान्नवर्जिताः}
{मामर्चयन्ति तद्भक्तास्तेभ्यो दत्तमिहाक्षयम्}


\twolineshloka
{प्रदानैः पूजितो विप्रो वन्दितो वापि संस्कृतः}
{सम्भाषितो वा दृष्टो वा मद्भक्तो दिवमुन्नयेत्}


\twolineshloka
{ये पठन्ति नमस्यन्ति ध्यायन्ति पुरुषास्तु माम्}
{स तान्स्पृष्ट्वा च दृष्ट्वा च नरः पापैः प्रमुच्यते}


\threelineshloka
{मद्भक्ता मद्गतप्राणा मद्गीता मत्परायणाः}
{बीजयोनिविशुद्धा यै श्रोत्रियाः संयतेन्द्रियाः}
{शूद्रान्नविरता नित्यं ते पुनन्तीह दर्शनात्}


\twolineshloka
{स्वंय नीत्वा विशेषेण दानं तेषां गृहेष्वथ}
{निवापयेत्तु यद्भक्त्या तद्दानं कोटिसंमितम्}


\twolineshloka
{जाग्रतः स्वपतो वापि प्रवासेषु गृहेष्वथ}
{हृदये न प्रणश्यामि यस्य विप्रस्य भावतः}


\twolineshloka
{स पूजितो वा दृष्टो वा स्पृष्टो वापि द्विजोत्तमः}
{सम्भाषितो वा राजेन्द्र पुनात्येव नरं सदा}


\twolineshloka
{एवं सर्वास्ववस्थासु सर्वदानानि पाण्डव}
{मद्भक्तेभ्यः प्रदत्तानि स्वर्गमार्गप्रदानि वै}


\chapter{अध्यायः ९९}
\twolineshloka
{श्रुत्वैवं सात्विकं दानं राजसं तामसं तथा}
{पृथक्पृथक्त्वेन गतिं फलं चापि पृथक्पृथक्}


\twolineshloka
{अवितृप्तः प्रहृष्टात्मा पुण्यं धर्मामृतं पुनः}
{युधिष्ठिरो धर्मरतः केशवं पुनरब्रवीत्}


\twolineshloka
{बीजयोनिविशुद्धानां लक्षणानि वदस्व मे}
{बीजदोषेण लोकेश जायन्ते च कथं नराः}


% Check verse!
आचारदोषं देवेशं वक्तुगर्हस्यशेषतः ॥ब्राह्मणानां विशेषं च गुणदोषौ च केशव
\fourlineindentedshloka
{चातुर्वर्ण्यस्य कुत्स्नस्य वर्तमानाः प्रतिग्रहे}
{केन विप्रा विशेषेण तरन्ते तारयन्ति च}
{एतान्कथय देवेश त्वद्भक्तस्य नमोस्तु ते ॥भगवानुवाच}
{}


\twolineshloka
{शृणु राजन्यथावृत्तं बीजयोनिं शुभाशुभम्}
{येन तिष्ठति लोकोयं विनश्यति च पाण्डव}


\twolineshloka
{अविप्लुतब्रह्मचर्यो यस्तु विप्रो यथाविधि}
{स बीजं नाम विज्ञेयं तस्य बीजं शुभं भवेत्}


\threelineshloka
{कन्या चाक्षतयोनिः स्यात्कुलीना पितृमातृतः}
{ब्राह्मणादिषु विवाहेषु परिणीता यथाविधि}
{सा प्रशस्ता वरारोहा तस्या योनिः प्रशस्यते}


\twolineshloka
{मनसा कर्मणा वाचा या गच्छेत्परपूरुषम्}
{योनिस्तस्या नरश्रेष्ठ गर्भाधानं न चार्हति}


\twolineshloka
{स्वैरिण्या यस्तु पापात्मा संतानार्थमिहेच्छति}
{स कुलान्पातयत्याशु दशपूर्वान्दशापरान्}


\threelineshloka
{दुष्टयोनौ तु यो मोहाद्रेतः सिञ्चति मूढधीः}
{तद्रेतसा समुत्पन्नः षडङ्गविदपि द्विजः}
{साधुभिः स बहिष्कार्यः श्वापाक इव पार्थिवा}


\twolineshloka
{कर्मणा मनसा वाचा या भवेत्स्वैरचारिणी}
{सा कुलघ्रीति विज्ञेया तस्यां जातः श्वपाचकः}


\twolineshloka
{दैवे पित्र्ये तथा दाने भोजने सहभाषणे}
{शयने सहसंबन्धे न योग्या दृष्टयोनिजाः}


\twolineshloka
{न तस्माद्दुष्टयोन्यां तु गर्भमुत्पादयेद्बुधः}
{मोहेन कुरुते यस्तु कुलं हन्ति त्रिपूरुषम्}


\threelineshloka
{कानीकनश्च सहोञश्च तथोभौ कुण्डगोलकै}
{आरूढपतिताज्जातः परितस्यापि यः सुतः}
{षडेते विप्रचण्डाला निकृष्टाः श्वपचादपि}


\twolineshloka
{यो यत्र तत्र वा रेतः सिक्त्वा शूद्रासु वा चरेत्}
{कामचारी स पापात्मा बीजं तस्याशुभं भवेत्}


\twolineshloka
{अशुद्धं तद्भवेद्बीजं शुद्धां योनि न चार्हति}
{दूषयत्यपि तां योनिं शुना लीढं हविर्यथा}


\twolineshloka
{शूद्रयोनौ पतेद्बीजं हाहाशब्दं द्विजन्मनः}
{कुर्यात्पुरीषगर्तेषु पतितोऽस्मीति दुःखितः}


\twolineshloka
{मामधःपातयन्नेष पापात्मा काममोहितः}
{अधोगतिं प्रजोत्क्षिप्रमिति शप्त्वा पतेत्तु तत्}


\twolineshloka
{आत्मा हि शुक्लमुद्दिष्टं दैवतं परमं महत्}
{तस्मात्सर्वप्रयत्नेन निरुन्ध्याच्छुक्लमात्मनः}


\twolineshloka
{आयुस्तेजो बलं वीर्यं प्रज्ञा श्रीश्च महद्यशः}
{पुण्यं च मत्प्रियत्वं च लभते ब्रह्मचर्यया}


\twolineshloka
{अविप्लुतब्रह्मचर्यैर्गृहस्थश्रममाश्रितैः}
{पञ्चयज्ञपरैर्धर्मः स्थाप्यते पृथिवीतले}


\twolineshloka
{सायंप्रातस्तु ये सन्ध्यां सम्यङ्नित्यमुपासते}
{नावं वेदमयीं कृत्वा तरन्ते तारयन्ति च}


\twolineshloka
{यो जपेत्पावनीं देवीं गायत्रीं वेदमातरम्}
{न सीदेत्प्रतिगृह्णानः पृथिवीं च ससागराम्}


\twolineshloka
{ये चास्य दुःस्थिताः केचिद्ग्रहाः सूर्यादयो दिवि}
{ते चास्य सौम्या जायन्ते शिवाः शुभकरास्तथा}


\twolineshloka
{यत्र यत्र स्थिताश्चैव दारुणाः पिशिताशनाः}
{घोररूपा महाकाया धर्षयन्ति न तं द्विजम्}


\twolineshloka
{पुनन्तीह पृथिव्यां च चीर्णवेदव्रता नराः}
{चतुर्णामपि वेदानां सा हि राजन्गरीयसी}


\threelineshloka
{अचीर्णिव्रतवेदा ये विकर्मफलमाश्रिताः}
{ब्राह्ममा नाममात्रेण तेऽपि पूज्या युधिष्ठिर}
{किं पुनर्यस्तु सन्ध्ये द्वे नित्यमेवोपतिष्ठते}


\twolineshloka
{शीलमध्ययनं दानं शौचं मार्दवमार्जवम्}
{तस्माद्वेदाद्विशिष्टानि मनुराह प्रजापतिः}


\twolineshloka
{भूर्भुवस्स्वरिति ब्रह्म यो वेदनिरतो द्विजः}
{स्वदारनिरतो दान्तः स विद्वान्स च भूसुरः}


\twolineshloka
{सन्ध्यामुपासते ये वै नित्यमेव द्विजोत्तमाः}
{ते यान्ति नरशार्दूल ब्रह्मलोकं न संशयः}


\twolineshloka
{सावित्रीमात्रसारोपि वरो विप्रः सुयन्त्रितः}
{नायन्त्रितश्चतुर्वेदी सर्वाशी सर्वविक्रयी}


\threelineshloka
{सावित्रीं चैव वेदांश्च तुलयाऽतोलयन्पुरा}
{सदेवर्षिगणाश्चैव सर्वे ब्रह्मपुरस्सराः}
{चतुर्णामपि वेदानां सा हि राजन्गरीयसी}


\twolineshloka
{यथा विकसिते पुष्पे मधु गृह्णन्ति षट्पदाः}
{एवं गृहीता सावित्री सर्ववेदे च पाण्डवः}


\twolineshloka
{तस्मात्तु सर्ववेदानां सावित्री प्राण उच्यते}
{निर्जीवा हीतरे वेदा विना सावित्रिया नृपः}


\twolineshloka
{नायन्त्रितश्चतुर्वेदी शीलभ्रष्टः स कुत्सितः}
{शीलवृत्तसमायुक्तः सावित्रीपाठको वरः}


\threelineshloka
{सहस्रपरमां देवीं शतमध्यां दशावराम्}
{सावित्रीं जप कौतेय सर्वपापप्रणाशिनीम् ॥युधिष्ठिर उवाच}
{}


\threelineshloka
{त्रैलोक्यनाथ हे कृष्ण सर्वभूतात्मको ह्यसि}
{नानायोगपर श्रेष्ठ तुष्यसे केन कर्मणा ॥भागवानुवाच}
{}


\twolineshloka
{यदि भारसहस्रं तु गुग्गुल्वादि प्रधूपयेत्}
{करोति चेन्नमस्कारमुपहारं च कारयेत्}


\twolineshloka
{स्तौति यः स्तुतिभिर्मां च ऋग्यजुस्सामभिः सदा}
{न तोषयति चेद्धिप्रान्नाहं तुष्यामि भारत}


\twolineshloka
{ब्राह्मणे पूजिते नित्यं पूजितोस्मि न संशयः}
{आक्रुष्टे चाहमाक्रुष्टो भवामि भरतर्षभ}


\twolineshloka
{परा मयि गतिस्तेषां पूजयन्ति द्विजं हि ये}
{यदहं द्विजरूपेण वसामि वसुधातले}


\twolineshloka
{यस्तान्पूजयति प्राज्ञो मद्गतेनान्तरात्मना}
{तमहं स्वेन रूपेण पश्यामि नरपुङ्गव}


\twolineshloka
{कुब्जाः काणा वामनाश्च दरिद्रा व्याधितास्तथा}
{नावमान्या द्विजाः प्राज्ञैर्मम रूपा हि ते द्विजाः}


\twolineshloka
{ये केचित्सागरान्तायां पृथिव्यां द्विजसत्तमाः}
{मम रूपं हि तेष्वेवमर्चितेष्वर्चितोऽस्म्यहम्}


\twolineshloka
{बहवस्तु न जानन्ति नरा ज्ञानबहिष्कृताः}
{यदहं द्विजरूपेण वसामि वसुधातले}


\twolineshloka
{अवमन्यन्ति ये विप्रान्स्वधर्मान्पातयन्ति ते}
{प्रेषणैः प्रेषयन्ते च शुश्रूषां कारयन्ति च}


\twolineshloka
{मृतांश्चात्र परत्रेमान्यमदूता महाबलाः}
{निकृन्तन्ति यथाकामं सूत्रमार्गेण शिल्पिनः}


\twolineshloka
{आक्रोशपरिवादाभ्यां ये रमन्ते द्विजातिषु}
{तान्मृतान्यमलोकस्थान्निपात्य पृथिवीतले}


\twolineshloka
{आक्रम्योरसि पादेन क्रूरः संरक्तलोचनः}
{अग्निवर्णैस्तु संदंशैर्यमो जिह्वां समुद्धरेत्}


\twolineshloka
{ये च विप्रान्निरीक्षन्ते पापाः पापेन चक्षुषा}
{अब्रह्मण्याः श्रुतेर्बाह्या नित्यं ब्रह्मद्विषो नराः}


\twolineshloka
{तेषां घोरा महाकाया वक्रतुण्डा महाबलाः}
{उद्धरन्ति मुहूर्तेन स्वगाश्चक्षुर्यमाज्ञया}


\threelineshloka
{यः प्रहारं द्विजेन्द्राय दद्यात्कुर्याच्च शोणितम्}
{अस्थिभङ्गं च यः कुर्यात्प्राणैर्वा विप्रयोजयेत्}
{सोनुपूर्व्येण यातीमान्नरकानेकविंशतिम्}


\threelineshloka
{शूलमारोप्यते पश्चाज्ज्वलने परिपच्यते}
{बहुवर्षसहस्राणि पच्यमानस्त्ववाक्छिराः}
{नावमुच्येत दुर्मेदा न तस्य क्षीयते गतिः}


\twolineshloka
{ब्राह्मणान्वा विचार्यैव व्रजन्तै वधकाङ्क्षया}
{शतर्वषसहस्राणि तामिस्रे परिपच्यते}


\twolineshloka
{उत्पाद्य शोणितं गात्रात्संरंभमतिपूर्वकम्}
{सपर्ययेण यातीमान्नरकानेकविंशतिम्}


\twolineshloka
{तस्मान्नाकुशलं ब्रूयान्न शुष्कां गतिमीरयेत्}
{न ब्रूयात्परुषां वाणीं न चैवैनानतिक्रमेत्}


\twolineshloka
{ये विप्राञ्श्लक्ष्णया वाचा पूजयन्ति नरोत्तमाः}
{अर्चितश्च स्तुतश्चैव तैर्भवामि न संशयः}


\twolineshloka
{तर्जयन्ति च ये विप्रान्क्रोशयन्ति च भारत}
{आक्रुष्टस्तर्जितश्चाहं तैर्भवामि न संशयः}


\twolineshloka
{यश्चन्दनैश्चागरुधूपदीपै-रभ्यर्चयेत्काष्ठमयीं ममार्चाम्}
{तेनार्चितो नैव भवामि सम्य-ग्विप्रार्चनादस्मि समर्चितोऽहम्}


\twolineshloka
{विप्रप्रसादाद्धरणीधरोऽहंविप्रप्रसादादसुराञ्जयामि}
{विप्रप्रसादाच्च सदक्षिणोऽहंविप्रप्रसादादजितोऽहमस्मि}


\chapter{अध्यायः १००}
\threelineshloka
{देवदेवेश दैत्यघ्नि परं कौतूहलं हि मे}
{एतत्कथय सर्वज्ञ त्वद्भक्तस्य च केशव}
{मानुषस्य च लोकस्य धर्मलोकस्य चान्तरम्}


\twolineshloka
{कीदृशं किंप्रमाणं वा किमधिष्ठानमेव च}
{तरन्ति मानुषा देव केनोपायेन माधव}


\twolineshloka
{त्वगस्थिमांसनिर्मुक्ते पञ्चभूतविवर्जिते}
{कथयस्व महादेव सुखदुःखमशेषतः}


\twolineshloka
{जीवस्य कर्मलोकेषु कर्मभिस्तु शुभाशुभैः}
{अनुबद्धस्य तैः पाशैर्नीयमानस्य दारुणैः}


\twolineshloka
{मृत्युदूतैर्दुराधर्षैर्घोरैर्घोरपरक्रमैः}
{वध्यस्याक्षिप्यमाणस्य विद्रुतस्य यमाज्ञया}


\threelineshloka
{पुण्यपापकृदातिष्ठेत्सुखदुःखमशेषतः}
{यमदूतैर्द्रुराधर्षैर्नीयते वा कथं पुनः}
{किं वा तत्र गता देव कर्म कुर्वन्ति मानवाः}


\twolineshloka
{कथं धर्मपरा यान्ति देवताद्विजपूजकाः}
{कतं वा पापकर्माणो यान्ति प्रेतपुरं नराः}


\threelineshloka
{किं रूपं किं प्रमाणं वा वर्णः को वाऽस्य केशव}
{जीवस्य गच्छतो नित्यं यमलोकं ब्रवीहि मे ॥भगवानुवाच}
{}


\twolineshloka
{शृणु राजन्यथावृतं यन्मां त्वं परिपृच्छसि}
{तत्तेऽहं कथयिष्यामि मद्भक्तस्य नरेश्वर}


\twolineshloka
{षडशीतिसहस्राणि योजनानां युधिष्ठिर}
{मानुष्यस्य च लोकस्य यमलोकस्य चान्तरम्}


\twolineshloka
{न तत्र वृक्षच्छाया वा न तटाकं सरोपि वा}
{न वाप्यो दीर्घिका वाऽपि न कूपो वा युधिष्ठिर}


\twolineshloka
{न मण्टपं सभा वाऽपि न प्रपा न निकेतनम्}
{न पर्वतो नदी वाऽपि न भूमेर्विवरं क्वचित्}


\twolineshloka
{न ग्रामो नाश्रमो वाऽपि नोद्यानं वा वनानि च}
{न किंचिदाश्रयस्थानं पथि तस्मिन्युधिष्ठिर}


\twolineshloka
{जन्तोर्हि प्राप्तकालस्य वेदनार्तस्य वै भृशम्}
{कारणैस्त्यक्तदेहस्य प्राणैः कण्ठगतैः पुनः}


\twolineshloka
{शरीराच्चाल्यते जीवो ह्यवशो मातरिश्वना}
{निर्गतो वायुभूतस्तु षट्कोशात्तु कलेवरात्}


\twolineshloka
{शरीरमन्यत्तद्रूपं तद्वर्णं तत्प्रमाणतः}
{अदृश्यं तत्प्रविष्टस्तु सोप्यदृष्टोऽथ केनचित्}


\threelineshloka
{सोन्तरात्मा देहवतामष्टाङ्गो यस्तु संचरेत्}
{छेदनाद्भेदनाद्दाहात्ताडनाद्वा न नश्यति}
{}


\twolineshloka
{नाना रूपधरैर्घौरैः प्रचण्डेश्चण्डसाधनैः}
{नीयमानो दुराधर्षैर्यमदूतैर्यमाज्ञया}


\twolineshloka
{पुत्रदारमयैः पाशैः संनिरूद्धोऽवशो बलात्}
{स्वकर्मभिश्चानुगतः कृतैः सुकृतदुष्कृतैः}


\twolineshloka
{आक्रन्दमानः करुणं बन्धुभिर्दुःखपीडितैः}
{त्यक्त्वा बन्धुजनं सर्वं निरपेक्षस्तु गच्छति}


\twolineshloka
{मातृभिः पितृभिश्चैव भ्रातृभिर्मातुलैस्तथा}
{दारैः पुत्रैर्वयस्यैश्च रुदद्भिस्त्यज्यते पुनः}


\twolineshloka
{अदृश्यमानस्तैर्दीनैरश्रुपूर्णमुखेक्षणैः}
{स्वशरीरं परित्यज्य वायुभूतस्तु गच्छति}


\twolineshloka
{अन्धकारमपारं तं महाघेरं तमोवृतम्}
{दुःखान्तं दुष्प्रतारं च दुर्गमं पापकर्मणाम्}


\twolineshloka
{दुःसहायं दुरन्तं च दुर्निरीक्षं दुरासदम्}
{दुरापमतिदुःखं च पापिष्ठानां नरोत्तम}


\twolineshloka
{ऋषिभिः कथ्यमानं तत्पारंपर्येण पार्थिव}
{त्रासं जनचति प्रायः श्रूयमाणं कथास्वपि}


\twolineshloka
{अवश्यं चैव गन्तव्यं तदध्वानं युधिष्ठिर}
{प्राप्तकालेन संत्यज्य बन्धून्भोगान्धनानि च}


\twolineshloka
{जरायुजैरण्डजैश्च स्वेदजैरुद्भिदैस्तथा}
{जङ्गमैः स्थिरसंज्ञैश्च गन्तव्यं यमसादनम्}


\twolineshloka
{देवासुरैर्मनुष्याद्यैर्वैवस्वतवशानुगैः}
{स्त्रीपुंनपुंसकैश्चापि पृथिव्यां जीवसंज्ञितैः}


\twolineshloka
{मध्यमैर्युवभिर्वाऽपि बालैर्वृद्धैस्तथैव च}
{जातमात्रैश्च गर्भस्थैर्गन्तव्यः स महापथः}


\twolineshloka
{पूर्वाह्णे वाऽपराह्णे वा सन्ध्याकालेऽथवा पुनः}
{प्रदोषे वाऽर्धरात्रे वा प्रत्युषे वाऽप्युपस्थिते}


\twolineshloka
{प्रवासस्थैर्वनस्थैर्वा पर्वतस्थैर्जले स्थितैः}
{क्षेत्रस्थैर्वा नभःस्थेर्वा गृहमध्यगतैरपि}


\threelineshloka
{भुञ्जद्भिर्वा पिबद्भिर्वा खादद्भिर्वा नरोत्तम}
{आसीनैर्वा स्थितैर्वापि शयनीयगतैरपि}
{जाग्रद्भिर्वा प्रसुप्तैर्वा गन्तव्यः स महापथः}


\twolineshloka
{मृत्युदूतैर्दुराधर्षैः प्रचण्डैश्चण्डसासनैः}
{आक्षिप्यमाणा ह्यवशाः प्रयान्ति यमसादनम्}


\twolineshloka
{क्वचिद्भीतैः क्वचिन्मत्तैः प्रस्खलद्भिः क्वचित्क्वचित्}
{क्रन्दद्भिर्वेदनार्तैस्तु गन्तव्यं यमसादनम्}


\twolineshloka
{निर्भर्त्स्यमानैरुद्विग्रैर्विधूतैर्भयविह्वलैः}
{तुद्यमानशरीरैश्च गन्तव्यं तर्जितैस्तथा}


\twolineshloka
{कण्टकाकीर्णमार्गेण तप्तवालुकपांसुना}
{दह्यमानैस्तु गन्तव्यं नरैर्दानविवर्जितैः}


\twolineshloka
{काष्ठोपलशिलाघातैर्दण्डोल्मुककशाङ्कुशैः}
{हन्यमानैर्यमपुरं गन्तव्यं धर्मवर्जितैः}


\twolineshloka
{मेदःशोणितपूयाद्यैर्वक्त्रैर्गात्रैश्च सव्रणैः}
{दग्धक्षतजकीर्णैश्च गन्तव्यं जीवघातकैः}


\twolineshloka
{वेदनार्तैस्च कूजद्भिर्विक्रोशद्भिश्च विस्वरम्}
{वेदनार्तैः पतद्भिश्च गन्तव्यं जीवघातकैः}


\twolineshloka
{भग्रपादोरुहस्तङ्गैर्भग्रजङ्घाशिरोधरैः}
{छिन्नकर्णोष्ठनासैश्च गन्तव्यं जीवघातकैः}


\twolineshloka
{शक्तिभिर्भिण्डिपालैश्च शङ्कुतोमरसायकैः}
{तुद्यमानैस्तु शूलाग्रैर्गन्तव्यं जीवघातकैः}


\twolineshloka
{श्वभिर्व्याघ्रैर्वृकैः काकैर्भक्ष्यमाणाः समन्ततः}
{तुद्यमानाश्च गच्छन्ति राक्षसैर्मांसघातिभिः}


\twolineshloka
{महिषैश्च मृगैश्चापि सूकरैः पृषतैस्तथा}
{भक्ष्यमाणैस्तदध्वानं गन्तव्यं मांसखादिभिः}


\twolineshloka
{सूचीसुतीक्ष्णतुण्डाभिर्मक्षिकाभिः समन्ततः}
{तुद्यमानैश्च गन्तव्यं पापिष्ठैर्बालघातकैः}


\twolineshloka
{विस्रब्धं स्वामिनं मित्रं स्त्रियं वा घ्रन्ति ये नराः}
{शस्त्रैर्निर्भिद्यमानैश्च गन्तव्यं यमसादनम्}


\twolineshloka
{खादयन्ति च ये जीवान्दुःखमापादयन्ति च}
{राक्षसैश्च श्वभिश्चैव भक्ष्यमाणा व्रजन्ति च}


\twolineshloka
{ये हरन्ति च वस्त्राणि शय्यां प्रावरणानि च}
{ते यान्ति विद्रुता नग्नाः पिशाचा इव तत्पथं}


\twolineshloka
{गाश्च धान्यं हिरण्यं वा बलात्क्षेत्रं गृहं तथा}
{ये हरन्ति दुरात्मानः परस्वं पापकारिणः}


\twolineshloka
{पाषाणैरुल्मुकैर्दण्डैः काष्ठघातैश्च चर्झरैः}
{हन्यमानैः क्षताकीर्णैर्नन्तव्यं तैर्यमालयम्}


\twolineshloka
{ब्रह्मस्वं ये हरन्तीह नरा नरकनिर्भयाः}
{आक्रोशन्तीह ये नित्यं प्रहरन्ति च ये द्विजान्}


\twolineshloka
{शुष्ककण्ठा निबद्धास्ते छिन्नजिह्वाक्षिनासिकाः}
{पूयशोणितदुर्गन्धा भक्ष्यमाणाश्च जंबुकैः}


\twolineshloka
{चण्डालैर्भीषणैश्चण्डैस्तुद्यमानाः समन्ततः}
{क्रोशन्तः करुणं घोरं गच्छन्ति यमासादनम्}


\twolineshloka
{तत्र चापि गताः पापा विष्ठाकूपेष्वनेकशः}
{जीवन्तो वर्षकोटीस्तु क्लिश्यन्ते वेदनात्ततः}


\twolineshloka
{ततश्च मुक्ताः कालेन लोके चास्मिन्नराधमाः}
{विष्ठाक्रिमित्वं गच्छन्ति जन्मकोटिशतं नृप}


\twolineshloka
{विद्यमानधनैर्यैस्तु लोभडंभानृतान्वितैः}
{श्रोत्रियेभ्यो न दत्तानि दानानि कुरुपुङ्गव}


\twolineshloka
{ग्रीवापाशनिबद्धास्ते हन्यमानाश्च राक्षसैः}
{क्षुत्पिपासाश्रमार्तास्तु यान्ति प्रेतपुरं नराः}


\twolineshloka
{अदत्तदाना गच्छन्ति शुष्ककण्ठास्यतालुकाः}
{अन्नं पानीयसहितं प्रार्थयन्तः पुनःपुनः}


\fourlineindentedshloka
{स्वामिन्बुभक्षातृष्णार्ता गन्तुं नैवाद्य शक्नुमः}
{ममान्नं दीयतां स्वामिन्पानीयं दीयतां मम}
{इति ब्रुवन्तस्तैर्दूतैः प्राप्यन्ते वै यमालयम् ॥वैशंपायन उवाच}
{}


\twolineshloka
{तच्छ्रुत्वा वचनं विष्णोः पपात भुवि पाण्डवः}
{निस्संज्ञो भयसंत्रस्तो मूर्छया समभिप्लुतः}


\twolineshloka
{ततो लब्ध्वा शनैः संज्ञां समाश्वस्तोच्युतेन सः}
{नेत्रे प्रक्षाल्य तोयेन भूयः केशवमब्रवीत्}


\threelineshloka
{भीतोस्म्यहं महादेव श्रुत्वा मार्गस्य विस्तरम्}
{केनोपायेन तं मार्गं तरन्ति पुरुषाः सुखम् ॥भगवानुवाच}
{}


\twolineshloka
{इह ये धार्मिका लोके जीवगातविवर्जिताः}
{गुरुशुश्रूषणे युक्ता देवब्राह्मणपूजकाः}


\twolineshloka
{अस्मान्मानुष्यलोकात्ते समार्याः सहबान्धवाः}
{यमध्वानं तु गच्छन्ति यथावत्तं निबोध मे}


\twolineshloka
{ब्राह्मणेभ्यः प्रदानानि नानारूपाणि पाण्डव}
{ये प्रयच्छन्ति विप्रेभ्यस्ते सुखं यान्ति तत्फलम्}


\twolineshloka
{अन्नं ये च प्रयच्छन्ति ब्राह्मणेभ्यः सुसंस्कृतम्}
{क्षोत्रियेभ्यो विशेषेण प्रीत्या परमया युताः}


\twolineshloka
{ते विमानैर्महात्मानो यान्ति चित्रैर्यमालयम्}
{सेव्यमाना वरस्त्रीभिरप्सरोभिर्महापथम्}


\twolineshloka
{ये च नित्यं प्रभाषन्ते सत्यं निष्कल्मषं वचः}
{ते च यान्त्यमलाभ्राभैर्विमानैर्वृषयोजितैः}


\twolineshloka
{कपिलाद्यानि पुण्यानि गोप्रदानानि ये नराः}
{ब्राह्मणेभ्यः प्रयच्छन्ति श्रोत्रियेभ्यो विशेषतः}


\twolineshloka
{ते यान्त्यमलवर्णाभैर्विमानैर्वृषयोजितैः}
{वैवस्वतपुरं प्राप्य ह्यप्सरोभिर्निषेविताः}


\twolineshloka
{उपानहौ च च्छत्रं च शयनान्यासनानि च}
{विप्रेभ्यो ये प्रयच्छन्ति वस्त्राण्याभरणानि च}


\twolineshloka
{ते यान्त्यश्वैर्वृषैर्वाऽपि कुञ्जरैरप्यलङ्कृताः}
{धर्मराजपुरं रम्यं सौवर्णच्छत्रशोभिताः}


\twolineshloka
{ये च भक्ष्याणि दास्यन्ति भोज्यं पेयं तथैव च}
{स्निग्धान्नान्यापि विप्रेभ्यः श्रद्धया परया युताः}


\twolineshloka
{ते यान्ति काञ्चनार्यानैः सुखं वैवस्वतालयम्}
{वरस्त्रीभिर्यथाकामं सेव्यमानाः सहस्रशः}


\twolineshloka
{ये च क्षीरं प्रयच्छन्ति घृतं दधि गुडं मधु}
{ब्राह्मणेभ्यः प्रयत्नेन श्रद्दधानाः सुसंस्कृताः}


\twolineshloka
{चक्रवाकप्रयुक्तैस्तु यानै रुक्ममयैः शुभैः}
{यान्ति गन्धर्ववादित्रैः सेव्यमाना यमालयम्}


\twolineshloka
{ये फलानि प्रयच्छन्ति पुष्पाणि सुरभीणि च}
{हंसयुक्तैर्विमानैस्तु यान्ति धर्मपुरं नराः}


\fourlineindentedshloka
{ये प्रयच्छन्ति विप्रेभ्यो विचित्रान्नं घृताप्लुतम्}
{ते व्रजनत्यमलाभ्राभैर्विमानैर्वायुवेगिभिः}
{पुरं तत्प्रेतनाथस्य नानाजनसमाकुलम्}
{}


\twolineshloka
{पानीयं ये प्रयच्छन्ति सर्वभूतप्रजीवनम्}
{ते सुतृप्ताः सुखं यान्ति भवनैर्हंसचोदितैः}


\twolineshloka
{ये तिलं तिलधेनुं वा धृतधेनुमथापि च}
{श्रोत्रियेभ्यः प्रयच्छन्ति सौम्यभावसमन्विताः}


\twolineshloka
{सूर्यमण्डलसंकाशैर्यानैस्ते यान्ति निर्मलैः}
{गीयमानैस्तु गन्धर्वैर्वैवस्वतपुरं नृप}


\twolineshloka
{येषां वाप्यश्च कूपाश्च तटाकानि सरांसि च}
{दीर्घिकाः पुष्करिण्यश्च सजलाश्च जलाशयाः}


\threelineshloka
{यानैस्ते यान्ति चन्द्राभैर्दिव्यघण्टानिनादितैः}
{चामरैस्तालवृन्तैश्च वीज्यमाना महाप्रभाः}
{नित्यतृप्ता महात्मानो गच्छन्ति यमसादनम्}


\twolineshloka
{येषां देवगृहाणीह चित्राण्यायतनानि च}
{मनोहराणि कान्तानि दर्शनीयानि भान्ति च}


\twolineshloka
{ते व्रजन्त्यमलाभ्राभैर्विमानैर्वायुवेगिभिः}
{पुरं तत्प्रेतनाथस्य नानाजनपदाकुलम्}


\twolineshloka
{वैवस्वतं च पश्यन्ति सुखचित्तं सुखस्थितम्}
{यमेन पूजिता यान्ति देवसालोक्यतां ततः}


\threelineshloka
{देवानुद्दिश्य लोकेषु प्रपासु करकोद्धृतम्}
{शीतलं सलिलं रम्यं तृषितेभ्यो दिशन्ति ये}
{ते तु तृप्ति परां यान्ति प्राप्य सौख्यं महापथम्}


\twolineshloka
{काष्ठपादुकदा यान्ति तदध्वानं सुखं नराः}
{सौवर्णमणिपीठे तु पादं कृत्वा स्थोत्तमे}


\twolineshloka
{आरामान्वृक्षषण्डांश्च रोपयन्ति च ये नराः}
{संवर्धयन्ति चाव्यग्रं फलपुष्पोपशोभितम्}


\threelineshloka
{वृक्षच्छायासु रम्यासु शीतलासु स्वलङ्कृताः}
{यान्ति ते वाहनैर्दिव्यैः पूज्यमाना मुहुर्मुहुः}
{}


\twolineshloka
{सेव्यमानाः सुरूपाभिरुत्तमाभिः प्रयत्नतः}
{स्त्रीभिः कनकवर्णाभिर्यथाकामं यथासुखम्}


\twolineshloka
{अश्वयानं तु गोयानं हस्तियानमथापि च}
{ये प्रयच्छन्ति विप्रेभ्यो विमानैः कनकोपमैः}


\threelineshloka
{सुवर्णं रजतं वाऽपि विद्रुमं मौक्तिकं तथा}
{ये प्रयच्छन्ति ते यान्ति विमानैः कनकोज्ज्वलैः}
{ते व्रजन्ति वरस्त्रीभिः सेव्यमाना यथासुखम्}


\twolineshloka
{भूमिदा यान्ति तं लोकं सर्वकामैः सुतर्पिताः}
{उदितादित्यसंकाशैर्विमानैर्वृषयोजितैः}


\twolineshloka
{कन्यां ये च प्रयच्छन्ति विप्राय श्रोत्रियाय च}
{दिव्यकन्यावृता यान्ति विमानैस्ते यमालयम्}


\twolineshloka
{सुगन्धान्गन्धसंयोगान्पुष्पाणि सुरभीणि च}
{प्रयच्छन्ति द्विजाग्रेभ्यो भक्तया परमया युताः}


% Check verse!
सुगन्धाः धर्मपुरं यानैर्विचित्रैरप्यलङ्कृताः ॥यान्ति धर्मपुरं यानैर्विचित्रैरप्यलङ्कृताः
\twolineshloka
{दीपया यान्ति यानैश्च द्योतयन्तो दिशो दश}
{आदित्यसदृशाकारैर्दीप्यमाना इवाग्नयः}


\twolineshloka
{गृहावसथदातारो गृहैः काञ्चनवेदिकैः}
{व्रजन्ति बालसूर्याभैर्धर्मराजपुरं नराः}


\twolineshloka
{जलभाजनदातारः कुण्डिकाकरकप्रदाः}
{पूज्यमाना वरस्त्रीभिर्यान्ति तृप्ता महागजैः}


\twolineshloka
{पादाभ्यङ्गं शिरोभ्यह्गं पानं पादोदकं तथा}
{ये प्रयच्छन्ति विप्रेभ्यस्ते यान्त्यश्वैर्यमालयम्}


\twolineshloka
{विश्रामायन्ति ये विप्राञ्श्रान्तानध्वनि कर्शितान्}
{चक्रवाकप्रयुक्तेन यान्ति यानेन तेऽपि च}


\twolineshloka
{स्वागतेन च यो विप्रान्पूजयेदासनेन च}
{स गच्छति तदध्वानं सुखं परमनिर्वृतः}


\twolineshloka
{नमो ब्रह्मण्यदेवेति यो मां दृष्ट्वाऽभिवादयेत्}
{व्रतीवं प्रयतो नित्यं स सुखं तत्पदं व्रजेत्}


\twolineshloka
{नमः सर्वसहाभ्यश्चेत्यभिख्याय दिनेदिने}
{नमस्करोति नित्यं गां स सुखं याति तत्पथं}


\twolineshloka
{नमोस्तु प्रियदत्तायेत्येवंवादी दिनेदिने}
{भूमिमाक्रमते प्रातः सयनादुत्थितश्च यः}


\twolineshloka
{सर्वकामैः स तृप्तात्मा सर्वभूषणभूषितः}
{याति यानेन दिव्येन सुकं वैवस्वतालयम्}


\twolineshloka
{अनत्तराशिनो ये तु डंभानृतविवर्जिताः}
{तेऽपि सारसयुक्तेन यान्ति यानेन वै सुखम्}


\twolineshloka
{ये चाप्येकेन भुक्तेन डंभानृतविवर्जिताः}
{हंसयुक्तैर्विमानैस्तु सुखं यान्ति यमालयम्}


\twolineshloka
{चतुर्थेन च भुक्तेन वर्तन्ते ये जितेन्द्रियाः}
{यान्ति ते धर्मनगरं यानैर्बर्हिणयोजितैः}


\twolineshloka
{तृतीयदिवसेनेह भुञ्जते ये जितेन्द्रियाः}
{तेऽपि हस्तिरथं यान्ति तत्पथं कनकोज्ज्वलैः}


\threelineshloka
{षष्ठान्नकालिको यस्तु वर्षमेकं तु वर्तते}
{कामक्रोधवनिर्मुक्तः शुचिर्नित्यं जितेन्द्रियः}
{स याति कुञ्जरस्थैस्तु जयशब्दरवैर्युतः}


\twolineshloka
{पक्षोपवासिनो यान्ति यानैः शार्दूलयोजितैः}
{धर्मराजपूरं रम्यं दिव्यस्त्रीगणसेवितम्}


\twolineshloka
{ये च मासोपवासं वै कुर्वते संयतेन्द्रियाः}
{तेऽपि सूर्यादयप्रख्यैर्यान्ति यानैर्यमालयम्}


\twolineshloka
{अग्निप्रवेशं यश्चापि कुरुते मद्गतात्मना}
{स यात्यग्निप्रकाशेन विमानेन यमालयम्}


\twolineshloka
{गोकृते स्त्रीकृते चैव हत्वा विप्रकृतेऽपि च}
{ते यान्त्यमरकन्याभिः सेव्यमाना रविप्रभाः}


\twolineshloka
{ये च कुर्वन्ति मद्भक्तास्तीर्थयात्रां जितेन्द्रियाः}
{ते पन्थानं महात्मानो यानैर्यान्ति सुनिर्वृताः}


\twolineshloka
{ये यजन्ति द्विजश्रेष्ठाः क्रतुभिर्भूरिदक्षिणैः}
{हंससारससंयुक्तैर्यानैस्ते यान्ति तत्पथम्}


\twolineshloka
{परपीडामकृत्वैव भृत्यान्बिभ्रति ये नराः}
{तत्पथं ससुखं यान्ति विमानैः काञ्चनोज्ज्वलैः}


\threelineshloka
{ये समाः सर्वभूतेषु जीवानामभयप्रदाः}
{क्रोधलोभविनिर्मुक्ता निगृहीतेन्द्रियास्तथा}
{}


\twolineshloka
{पूर्णचन्द्रप्रतीकाशैर्विमानैस्ते महाप्रभाः}
{यान्ति वैवस्वतपुरं देवगन्धर्वसेविताः}


\threelineshloka
{ये मामेकान्तभावेन देवं त्र्यंबकमेव वा}
{पूजयन्ति नमस्यन्ति स्तुवन्ति च दिनेदिने}
{धर्मराजपुरं यान्ति यानैस्तेऽर्कसमप्रभैः}


\twolineshloka
{पूजितास्तत्र धर्मेण स्वयं माल्यादिभिः शुभैः}
{यान्त्येव धर्मलोकं वा रुद्रलोकमथापि वा}


\chapter{अध्यायः १०१}
\twolineshloka
{श्रुत्वा यमपुराध्वानं जीवानां गमनं तथा}
{धर्मपुत्रः प्रहृष्टात्मा केशवं पुनरब्रवीत्}


\twolineshloka
{देवदेवेश दैत्यघ्न ऋषिसङ्घैरभिष्टुत}
{भगवन्भवहञ्श्रीमन्सहस्रादित्यसन्निभ}


\twolineshloka
{सर्वसंभव धर्मज्ञ सर्वधरमप्रवर्तक}
{सर्वदानफलं सौम्य कथयस्व ममाच्युत}


\twolineshloka
{दानं देयं कथं कृष्ण कीदृशाय द्विजाय वै}
{कीदृशं वा तपः कृत्वा तत्फलं कुत्र भुज्यते}


\twolineshloka
{एवमुक्तो हृषीकेशो धर्मपुत्रेण धीमता}
{उवाच धर्मपुत्राय पुण्यान्धर्मान्महोदयान्}


\twolineshloka
{शृणुष्वावहितो राजन्पूतं पापघ्नमुत्तमम्}
{सर्वदानफलं सौम्य न श्राव्यं पापकर्मणम्}


\twolineshloka
{यच्छ्रुत्वा पुरुषः स्त्री वा नष्टपापः समाहितः}
{तत्क्षणात्पूततां याति पापकर्मरतोपि वा}


\twolineshloka
{एकाहमपि कौन्तेय भूमावुत्पादितं जलम्}
{सप्त तारयते पूर्वान्वितृष्णा यत्र गौर्भवेत्}


\threelineshloka
{पानीयं परमं लोके जीवानां जीवनं स्मृतम्}
{पानीयस्य प्रदानेन तृप्तिर्भवति पाण्डव}
{पानीयस्य गुणा दिव्याः परलोके गुणावहाः}


\twolineshloka
{तत्र पुष्पोदकी नाम नदी परमपावनी}
{कामान्ददाति राजेन्द्र तोयदानां यमालये}


\twolineshloka
{शीतलं सलिलं ह्यत्र ह्यक्षय्यममृतोपमम्}
{शीततोयप्रदादॄणां भवेन्नित्यं सुखावहम्}


\twolineshloka
{ये चाप्यतोयदातारः पूयस्तेषां विधीयते}
{}


\fourlineindentedshloka
{प्रणश्यत्यंबुपानेन बुभुक्षा च युधिष्ठिर}
{तृपीषस्य न चान्नेन पिपासाऽपि प्रणश्यति}
{तस्मात्तोयं सदा देयं तृषितेभ्यो विजानता}
{}


\twolineshloka
{अग्नेर्मूर्तिः क्षितेर्योनिरमृतस्य स संभवः}
{अतोंभः सर्वभूतानां मूलमित्युच्यते बुधैः}


\twolineshloka
{अद्भिः सर्वाणि भूतानि जीवन्ति प्रभवन्ति च}
{तस्मात्सर्वेषु दानेषु तोयदानं विशिष्यते}


\twolineshloka
{सर्वदानतपोयज्ञैर्यत्प्राप्यं फलमुत्तमम्}
{तत्सर्वं तोयदानेन प्राप्यते नात्र संशयः}


\twolineshloka
{ये प्रयच्छन्ति विप्रेभ्यस्त्वन्नदातं सुसंस्कृतम्}
{तैस्तु दत्ताः स्वयं प्राणा भवन्ति भरतर्षभ}


\threelineshloka
{अन्नाद्रक्तं च शुक्लं च अन्ने जीवः प्रतिष्ठितः}
{इन्द्रियाणि च बुद्धिश्च पुष्णन्त्यन्नेन नित्यशः}
{अन्नहीनानि सीदन्ति सर्वभूतानि पाण्डव}


\twolineshloka
{तेजो बलं च रूपं च सत्वं वीर्यं धृतिर्द्युतिः}
{ज्ञानं मेधा तथाऽऽयुश्च सर्वमन्ने प्रतिष्ठितम्}


\twolineshloka
{देवमानवर्तिर्यक्षु सर्वलोकेषु सर्वदा}
{सर्वकालं हि सर्वेषां सर्वमन्ने प्रतिष्ठितम्}


\twolineshloka
{अन्नं प्रजापते रूपमन्नं प्रजननं स्मृतम्}
{सर्वभूतमयं चान्नं जीवश्चान्नमयः स्मृतः}


\twolineshloka
{अन्नेनाधिष्ठितः प्राण अपानो व्यान एव च}
{उदानश्च समानश्च धारयन्ति शरीरिणम्}


% Check verse!
शयनोत्थानगमनग्रहणाकर्षणानि चसर्वसत्वकृतं क्रम चान्नादेव प्रवर्तते
\twolineshloka
{चतुर्विधानि भूतानि जङ्गमानि स्थिराणि च}
{अन्नाद्भवन्ति राजेन्द्र सृष्टिरेषा प्रजापतेः}


\twolineshloka
{विद्यास्थानानि सर्वाणि सर्वयज्ञाश्च पावनाः}
{अन्नाद्यस्मात्प्रवर्तन्ते तस्मादन्नं परं स्मृतम्}


\twolineshloka
{देवा रुद्रादयः सर्वे पितरोऽप्यग्नयस्तथा}
{यस्मादन्नेन तुष्यन्ति तस्मादन्नं विशिष्यते}


\twolineshloka
{यस्मादन्नात्प्रजाः सर्वाः कल्पेकल्पेऽसृजत्प्रभुः}
{तस्मादन्नात्परं दानं न भूतं न भविष्यति}


\twolineshloka
{यस्मादन्नात्प्रवर्तन्ते धर्मार्थौ काम एव च}
{तस्मादन्नात्परं दानं नामुत्रेह च पाण्डव}


\twolineshloka
{यक्षरक्षोग्रहा नागा भूतान्यन्ते च दानवाः}
{तुष्यन्त्यन्नेन यस्मात्तु तस्मादन्नं परं भवेत्}


\twolineshloka
{परान्नमुपभुञ्जनो यत्कर्म कुरुते शुभम्}
{तच्छुभस्यैकभागस्तु कर्तुर्भवति भारत}


\twolineshloka
{अन्नदस्य त्रयो भागा भन्ति पुरुषर्षभ}
{तस्यादन्नं प्रदातव्यं ब्राह्मणेभ्यो विशेषतः}


\twolineshloka
{ब्राह्मणाय दरिद्राय योऽन्नं संवत्सरं नृप}
{श्रोत्रियाय प्रयच्छेद्वै पाकभेदविवर्जितः}


\twolineshloka
{डंभानृतविमुक्तस्तु परां भक्तिमुपागतः}
{स्वधर्मेणार्जितफलं तस्य पुण्यफलं शृणु}


\threelineshloka
{शतवर्षसहस्राणि कामगः कामरूपधृत्}
{मोदतेऽमरलोकस्थः पूज्यमानोप्सरोगणैः}
{ततश्चापि च्युतः कालान्नरलोके द्विजो भवेत्}


\twolineshloka
{अग्नभिक्षां च यो दद्याद्दरिद्राय द्विजातये}
{षण्मासान्वार्षिकं श्राद्दं तस्य पुण्यफलं शृणु}


\twolineshloka
{गोसहस्रप्रदानेन यत्पुण्यं समुदाहृतम्}
{तत्प्रण्यफलमाप्नोति नरो वै नात्र संशयः}


\twolineshloka
{अथ संवत्सरं दद्यादग्रभिक्षामयाचते}
{प्रच्छद्यैव स्वयं नीत्वा तस्य पुण्यफलं शृणु}


\twolineshloka
{कपिलानां सहस्रैस्तु यद्देयं पुण्यमुच्यते}
{तत्सर्वमखिलं प्राप्य शक्रलोके महीयते}


\twolineshloka
{स शक्रभवने रम्ये र्षिकोटिशतं नृप}
{यथाकामं महातेजाः क्रीडत्यप्सरसांगणैः}


\twolineshloka
{अन्नं च यस्तु वै दद्याद्द्विजाय नियतव्रतः}
{दशवर्षामि राजेन्द्र तस्य पुण्यफलं शृणु}


\twolineshloka
{कपिला शतसहस्रस्य विधिदत्तस्य यत्फलम्}
{तत्पुण्यफलमासाद्य पुरन्दरपुरं व्रजेत्}


\twolineshloka
{स शक्रभवने रम्ये कामरूपी यथासुखम्}
{शतकोटिसमा राजन्क्रीडतेऽमरपूजितः}


\twolineshloka
{शक्रलोकावतीर्णश्च इह लोके महाद्युतिः}
{चतुर्वेदी द्विजः श्रीमाञ्जायते राजपूजितः}


\twolineshloka
{अध्वश्रान्ताय विप्राय क्षुधितायान्नकाङ्क्षिणे}
{देशकालाभियाताय दीयते पाण्डुनन्दन}


\twolineshloka
{याचतेऽन्नं न दद्याद्यो विद्यामाने धनागमे}
{स लुब्धो नरकं याति कृमीणां कालसूत्रकम्}


\twolineshloka
{तत्र नरके घोरे लोभमोहविचेतनः}
{दशरव्षसहस्राणि क्लिश्यते वेदनार्दितः}


\threelineshloka
{तस्माच्च नरकान्मुक्तः कालेन महता हि सः}
{दरिद्रो मानुषे लोके चण्डालेष्वपि जायते}
{}


\twolineshloka
{यस्तु पांसुलपादश्च दूराध्वश्रमकर्शितः}
{क्षुत्पिपासाश्रमश्रान्त आर्तः खिन्नगतिर्द्विजः}


\threelineshloka
{पृच्छन्वै ह्यन्नदातारं गृहमभ्येत्य याचयेत्}
{तं पूजयेत्तु यत्नेन सोऽतिथिः स्वर्गसंक्रमः}
{तस्मिंस्तुष्टे नरश्रेष्ठ तुष्टाः स्युः सर्वदेवताः}


\twolineshloka
{न तथा हविषा होमैर्न पुष्पैर्नानुलेपनैः}
{अग्नयः पार्थ तुष्यन्ति यथा ह्यतिथिपूजनात्}


\threelineshloka
{कपिलायां तु दत्तायां विधिवज्ज्येष्ठपुष्करे}
{न तत्फलमवाप्नोति यत्फलं विप्रबोजनान्}
{}


\twolineshloka
{द्विजपादोदकक्लिन्ना यावत्तिष्ठति मेदिनी}
{तावत्पुष्करपत्रेण पिबन्ति पितरो जलम्}


\twolineshloka
{देवमाल्यापनयनं द्विजोच्छिष्टापमार्जनम्}
{श्रान्तसंवाहनं चैव तथा पादावसेचनम्}


\twolineshloka
{प्रतिश्रयप्रदानं च तथा शय्यासनस्य च}
{एकैकं पाण्डवश्रेष्ठ गोप्रदानाद्विशिष्यते}


\twolineshloka
{पादोदकं पादघृतं दीपमन्नं प्रतिश्रयम्}
{ये प्रयच्छन्ति विप्रेभ्यो नोपसर्पन्ति ते यमम्}


\twolineshloka
{विप्रातिथ्ये कृते राजन्भक्त्या शुश्रूषितेऽपि च}
{देवाः शुश्रूषिताः सर्वे त्रयस्त्रिंशदरिन्दम}


\twolineshloka
{अभ्यागतो ज्ञातपूर्वो ह्यज्ञातोऽतिथिरुच्यते}
{तयोः पूजां द्विजः कुर्यादिति पौराणिकी श्रुतिः}


\twolineshloka
{पादाभ्यङ्गन्नपानैस्तु योऽतिर्थिं पूजयेन्नरः}
{पूजितस्तेन राजेन्द्र भवामीह न संशयः}


\twolineshloka
{शीघ्रं पापाद्विनिर्मुक्तो मया चानुग्रहीकृतः}
{विमानेनेन्दुकल्पेन मम लोकं स गच्छति}


\twolineshloka
{अभ्यागतं श्रान्तमनुव्रजन्तिदेवाश्च सर्वे पितरोऽग्नयश्च}
{तस्मिन्द्विजे पूजिते पूजिताः स्यु-र्गते निराशाः पितरो व्रजन्ति}


\twolineshloka
{अतिर्थिर्यस्य भग्नाशो गृहात्प्रतिनिवर्तते}
{पितरस्तस्य नाश्नन्ति दशवर्षणि पञ्च च}


\threelineshloka
{वर्जितः पितृभिर्लुब्धः स देवैरग्निभिः सह}
{निरयं रौरवं गत्वा दशवर्षाणि पञ्च च}
{ततश्चापि च्युतः कालादिह चोच्छिष्टभुग्भवेत्}


\twolineshloka
{वैश्वदेवान्तिके प्राप्तमतिथिं यो न पूजयेत्}
{चण्डालत्वमवाप्नोति सद्य एव न संशयः}


\twolineshloka
{निर्वासयति यो विप्रं देशकालगतं गृहात्}
{पतितस्तत्क्षणादेव जायते नात्र संशयः}


\threelineshloka
{नरके रौरवे घोरे वर्षकोटिं स पच्यते}
{ततश्चापि च्युतः कालादिह लोके नराधमः}
{श्वा वै द्वादशजन्मानि जायते क्षुत्पिपासितः}


\twolineshloka
{चण्डालोप्यतिथिः प्राप्तो देशकालेऽन्नकाङ्क्षयाः}
{अभ्युद्गम्यो गृहस्थेन पूजनीयश्च सर्वदा}


\twolineshloka
{अनर्चयित्वा योऽश्नाति लोभमोहविचेतनः}
{स चण्डालत्वमापन्नो दश जन्मानि पाण्डव}


\twolineshloka
{निराशमतिथिं कृत्वा भुञ्जनो यः प्रहृष्टवान्}
{न जानाति किलात्मानं विष्ठकूपे निपातितं}


\twolineshloka
{मोघं ध्रुवं प्रोर्णयति मोघमस्य तु पच्यते}
{मोघमन्नं सदाऽश्नाति योतिथिं न च पूजयेत्}


\twolineshloka
{साङ्गोपाङ्गांस्तु यो वेदान्पठतीह दिनेदिने}
{न चातिथिं पूजयति वृथा भवति स द्विजः}


\twolineshloka
{पाकयज्ञमहायज्ञैः सोमसंस्थाभिरेव च}
{ये यजन्ति न चार्चन्ति गृहेष्वतिथिमागतम्}


\twolineshloka
{तेषां यशोभिकामानां दत्तमिष्टं च यद्भवेत्}
{वृथा भवति तत्सर्वमाशया हि तया हतम्}


\twolineshloka
{देशं कालं च पात्रं च स्वशक्तिं च निरीक्ष्य च}
{अल्पं समं महद्वापि कुर्यादातिथ्यमात्मवान्}


\twolineshloka
{सुमुखः सुप्रसन्नात्मा धीमानतिथिमागतम्}
{स्वागतेनासनेनाद्भिरन्नाद्येन च पूजयेत्}


\twolineshloka
{हितः प्रियो वा द्वेष्यो वा मूर्खः पण्डित एव वा}
{प्राप्तो यो वैश्वदेवान्ते सोतिथिः स्वर्गसंक्रमः}


\twolineshloka
{क्षुत्पिपासाश्रमार्ताय देशकालगताय च}
{सत्कृत्यान्नं प्रदातव्यं यज्ञस्य फलमिच्छता}


\threelineshloka
{भोजयेदात्मनः श्रेष्ठान्विधइवद्धव्यकव्ययोः}
{अन्नं प्राणो मनुष्यणामन्नदः प्राणदो भवेत्}
{तस्मादन्नं विशेषेण दातव्यं भूतिमिच्छता}


\twolineshloka
{अन्नदः सर्वकामैस्तु सुतृप्तः सुष्ट्वलङ्कृतः}
{पूर्णचन्द्रप्रकाशेन विमानेन विराजते}


\twolineshloka
{सेव्यमानो वरस्त्रीभिर्मम लोकं स गच्छति}
{क्रीडित्वा तु ततस्तस्मिन्वर्षकोटिं यथाऽमरः}


\twolineshloka
{ततस्चापि च्युतः कालादिह लोके महायशाः}
{वेदशास्त्रार्थतत्वज्ञो भोगवान्ब्राह्मणो भवेत्}


\threelineshloka
{यथाश्रद्धं तु यः कुर्यान्मनुष्येषु प्रजायते}
{महाधनपतिः श्रीमान्वेदवेदाङ्गपारगः}
{सर्वशास्त्रार्थतत्वज्ञो भोगवान्ब्राह्मणो भवेत्}


\twolineshloka
{सर्वातिथ्यं तु यः कुर्याद्वर्षमेकमकल्मषः}
{धर्मार्जितधनो भूत्वा पाकभेदविवर्जितः}


\twolineshloka
{देवानिव स्वयं विप्रानर्चयित्वा पितॄनपि}
{विप्रानग्राशनाशी यस्तस्य पुण्यफलं शृणु}


\twolineshloka
{वर्षेणैकेन यावन्ति पिण्डान्यश्नन्ति ये द्विजाः}
{तावद्वर्षाणि राजेन्द्र मम लोके महीयते}


\twolineshloka
{ततश्चापि च्युतः कालादिह लोके महायशाः}
{वेदसास्त्रार्थतत्वज्ञो भोगवान्ब्राह्मणो भवेत्}


\twolineshloka
{सर्वातिथ्यं तु यः कुर्याद्यथाश्रद्धं नरेश्वर}
{अकालनियमेनापि सत्यवादी जितेन्द्रियः}


\twolineshloka
{सत्यसन्धो चितक्रोधः शाखाधर्मविवर्जितः}
{अधर्मभीरुर्धर्मिष्ठो मायामात्सर्यवर्जितः}


\twolineshloka
{श्रद्दधानः सुचिर्नित्यं पाकबेदविवर्जितः}
{स विमानेन दिव्येन दिव्यरूपी महायशाः}


\threelineshloka
{पुरन्दरपुरं याति गीयमानोप्सरोगणैः}
{मन्वन्तरं तु तत्रैव क्रीडित्वा देवपूजितः}
{मानुष्यलोकमागम्य भोगवान्ब्राह्मणो भवेत्}


\twolineshloka
{दशजन्मानि विप्रत्वमाप्नुयाद्राजपूजितः}
{जातिस्मरश्च भवति यत्रयत्रोपजायते}


\chapter{अध्यायः १०२}
% Check verse!
अतः परं प्रवक्ष्यामि भूमिदानमनुत्तमम्
\twolineshloka
{यः प्रयच्छति विप्राय भूमिं रम्यां सदक्षिणाम्}
{श्रोत्रियाय दरिद्राय साग्निहोत्राय पाण्डव}


\twolineshloka
{स सर्वकामतृप्तात्मा सर्वरत्नविभूषितः}
{सर्वपापविनिर्मुक्तो दीप्यमानोऽर्कवत्सदा}


\twolineshloka
{बालसूर्यप्रकाशेन विचित्रध्वजशोभिना}
{याति यानेन दिव्येन मम लोकं महायशाः}


\twolineshloka
{तत्र दिव्याङ्गनाभिस्तु सेव्यमानो यथासुखम्}
{कामगः कामरूपी च क्रीडत्यप्सरसांगणैः}


\twolineshloka
{यावद्बिभर्ति लोकान्वै भूमिः कुरुकुलोद्वह}
{तावद्भूमिप्रदः काले मम लोके महीयते}


\twolineshloka
{न हि भूमिप्रदानाद्वै दानमन्यद्विशिष्यते}
{न चापि भूमिहरणात्पापमान्यद्विशिष्यते}


\twolineshloka
{दानान्यन्यानि हीयन्ते कालेन कुरुपुङ्गव}
{भूमिदानस्य पुण्यस्य क्षयो नैवोपपद्यते}


\twolineshloka
{ब्राह्मणाय दरिद्राय भूमिं दत्तां तु यो नरः}
{न हिंसति नरव्याघ्र तस्य पुण्यफलं शृणु}


\twolineshloka
{सप्तद्वीपसमुद्रान्ता रत्नसंचयसंकुला}
{सशैलवनदुर्गाढ्या तेन दत्ता मही भवेत्}


\twolineshloka
{भूमिं दृष्ट्वा दीयमानां श्रोत्रियायाग्निहोत्रिणे}
{सर्वभूतानि मन्यन्ते मां ददातीति हर्षवत्}


\twolineshloka
{सुवर्णमणिरत्नानि धनानि च वसूनि च}
{सर्वदानानि वै राजन्ददाति वसुधां ददत्}


\twolineshloka
{सागरान्सरितः शैलान्समानि विषमाणि च}
{सर्वगन्धरसांश्चैव ददाति वसुधां ददत्}


\twolineshloka
{ओषधीः फलसंपन्ना नानापुष्पसमन्विताः}
{कमलोत्पलषण्डांश्च ददाति वसुधां ददत्}


\twolineshloka
{धर्मं कामं तथा चार्थं वेदान्यज्ञांस्तथैव च}
{स्वर्गमार्गगतिं चैव ददाति वसुधां ददत्}


\twolineshloka
{अग्निष्टोमादिभिर्यज्ञैर्ये यजन्ते सदक्षिणैः}
{न तत्फलं लभन्ते ते भूमिदानस्य यत्फलम्}


\threelineshloka
{श्रोत्रिया महीं दत्त्वा यो न हिंसति पाण्डव}
{तद्दानं कथयिष्यन्ति यावल्लोकाः प्रतिष्ठिताः}
{तावत्स्वर्गोपभोगानां भोक्तारः पाण्डुनन्दन}


\twolineshloka
{सस्यपूर्णां महीं यस्तु श्रोत्रियाय प्रयच्छति}
{पितरस्तस्य तृप्ययन्ति यावदाभूतसंप्लुवम्}


\twolineshloka
{मम रुद्रस्य सवितुस्त्रिदशानां तथैव च}
{प्रीतये विद्धि राजेन्द्र भूमिर्दत्ता द्विजाय वै}


\twolineshloka
{तेन पुण्येन पूतात्मा दाता भूमेर्युधिष्ठिर}
{मम सालोक्यमाप्नोति नात्र कार्या विचाराणा}


\twolineshloka
{यत्किंचित्कुरुते पापं पुरुषे वृत्तिकर्शितः}
{स च गोकर्णमात्रेण भूमिदानेन शुध्यति}


\twolineshloka
{मासोपवासे यत्पुण्यं कृच्छ्रे चान्द्रायणेऽपि च}
{भूमिगोकर्णमात्रेण तत्पुण्यं तु विधीयते}


\threelineshloka
{सर्वतीर्थाभिषेके च यत्पुण्यं समुदाहृतम्}
{भूमिगोकर्णमात्रेण तत्पुण्यं तु विधीयते ॥युधिष्ठिर उवाच}
{}


\threelineshloka
{देवदेव नमस्तेऽस्तदु वासुदेव सुरेश्वर}
{गोकर्णस्य प्रमाणं वै वक्तुमर्हसि तत्वतः ॥भगवानुवाच}
{}


\twolineshloka
{शृणु गोकर्णमात्रस्य प्रमाणं पाण्डुनन्दन}
{त्रिंशद्दण्डप्रमाणेन प्रमितं सर्वतो दिशम्}


\twolineshloka
{प्रत्यक्प्रागपि राजेन्द्र तत्तथा दक्षिणोत्तरम्}
{गोकर्णं तद्विदः प्राहुः प्रमाणं धरणेर्नृप}


\twolineshloka
{सवृषं गोशतं यत्र सुखं तिष्ठत्ययन्त्रितम्}
{सवत्सं कुरुशार्दूल तच्च गोकर्णमुच्यते}


\twolineshloka
{किंकरा मृत्युदण्डाश्च कुंभीपाकाश्च दारुणाः}
{घोराश्च वारुणाः पाशा नोपसर्पन्ति भूमिदम्}


\twolineshloka
{निरया रौरवाद्याश्च तथा वैतरणी नदी}
{तीव्राश्च यातनाः कृष्टा नोपसर्पन्ति भूमिदम्}


\twolineshloka
{चित्रगुप्ताः कलिः कालः कृतान्तो मृत्युरेव च}
{यमश्च भगवान्साक्षात्पूजयन्ति महीप्रदम्}


\twolineshloka
{रुद्रः प्रजापतिः शक्रः सुरा ऋषिगणास्तथा}
{अहं च प्रीतिमान्राजन्पूजयामो महीप्रदम्}


\twolineshloka
{कृशभृत्यस्य कृशगोः कृशाश्वस्य कृतातिथेः}
{भूमिर्देया नरश्रेष्ठ स निधइः पारलौकिकः}


\twolineshloka
{सीदमानकुटुंबाय श्रोत्रियायाग्निहोत्रिणे}
{व्रस्थाय दरिद्राय भूमिर्देया नराधिप}


\twolineshloka
{यथा हि धात्री क्षीरेण पुत्रं वर्धयति स्वयम्}
{दातारमनुगृह्णाति दत्ता ह्येवं वसुंधरा}


\twolineshloka
{यथा बिभर्ति गौर्वत्सं सृजन्ती क्षीरमात्मनः}
{तथा सर्वगुणोपेता भूमिर्वहति भूमिदम्}


\twolineshloka
{यथा बीजनि रोहन्ति जलसिक्तानि भूपते}
{तथा कामाः प्ररोहन्ति भूमिदस्य दिनेदिने}


\twolineshloka
{यथा तेजस्तु सूर्यस्य तमः सर्वं व्यपोहति}
{तथा पापं नरस्येह भूमिदानं व्यपोहति}


\twolineshloka
{दाता दशानुगृह्णाति यो हरेद्दश हन्ति च}
{अतीतान्यागतानीह कुलानि कुरुपुङ्गव}


\twolineshloka
{आश्रुत्य भूमिदानं तु दत्त्वा यो वा हरेन्पुनः}
{स बद्धो वारुणैः पाशैः क्षिप्यते पूयशोणिते}


\twolineshloka
{स्वदत्तां परदत्तां वा यो हरेत् वसुंधराम्}
{न तस्य नरकाद्धोराद्विद्यते निष्कृतिः क्वचित्}


\twolineshloka
{ब्राह्म्णस्य हृते क्षेत्रे हन्याद्द्वादश पूर्वजान्}
{स गच्छेत्कृमियोनिं च न च मुच्येत जातु सः}


\twolineshloka
{दत्त्वा भूमिं द्विजेन्द्राय यस्तामेवोपजीवति}
{गवां शतसहस्रस्य हन्तुः स लभते फलम्}


\threelineshloka
{सोधश्शिरास्तु पापात्मा कुंभीपाकेषु पच्यते}
{दिव्यैर्वर्षसहस्रैस्तु कुंभीपाकाद्विजनिस्सृतः}
{इह लोके भवेत्स श्वा रातजन्मनि पाण्डव}


\threelineshloka
{दत्त्वा भूमि द्विजेन्द्राणां यस्तामेवोपजीवति}
{स मूढो याति दुष्टात्मा नरकानेकविंशतिम्}
{नरकेभ्यो विनिर्मुक्तः शुनां योनिं स गच्छति}


\twolineshloka
{हलकृष्टा मही देया सबीजा सस्यमालिनी}
{अथवा सोदका देया दरिद्राय द्विजातये}


\twolineshloka
{एवं दत्ता मही राजन्प्रहृष्टेनान्तरात्मना}
{सर्वान्कामानवाप्नोति मनसा चिन्तितानि च}


\twolineshloka
{बहुभिर्वसुधा दत्ता दीयते च नराधिपैः}
{यस्य यस्य यदा भूमिस्तस्य तस्य तदा फलम्}


\twolineshloka
{यः प्रयच्छति कन्यां वै सुरूपां श्रोत्रियाय वै}
{स ब्रह्मदेयो राजेन्द्र तस्य पुण्यफलं शृणु}


\twolineshloka
{बलीवर्दसहस्राणां दत्तानां धुर्यवाहिनाम्}
{यत्पुण्यं लभते राजन्कन्यादानेन तत्फलम्}


\twolineshloka
{गवां शतसहस्रस्य सम्यग्धत्तस्थ यत्फलम्}
{तत्फलं समवाप्नोतिः यः प्रयच्छति कन्यकाम्}


\twolineshloka
{यावन्ति चैव रोमाणि कन्यायाः कुरुपुङ्गव}
{तावद्वर्षसहस्राणि मम लोके महीयते}


\twolineshloka
{ततश्चापि च्युतः कालादिह लोके स जायते}
{षडङ्गविच्चतुर्वेदी सर्वलोकार्चितो द्विजः}


\twolineshloka
{यः सुवर्णं दरिद्राय ब्राह्मणाय प्रयच्छति}
{श्रोत्रियाय सुवृत्ताय बहुपुत्राय पाण्डव}


\twolineshloka
{स मुक्तः सर्वपापेभ्यो बालसूर्यसमप्रभः}
{विमानं दिव्यामारूढः कामगः कामभोगवान् ॥वर्षकोटिं महातेजा मम लोके महीयते}


\twolineshloka
{ततः कालावतीर्णश्च सोस्मिँल्लोके हि जायते}
{वेदवेदाङ्गविद्विप्रः कोटीधनपतिर्भवेत्}


\twolineshloka
{यश्च रूप्यं प्रयच्छेद्वै दरिद्राय द्विजातये}
{कृशवृत्तेः कृशगवे स मुक्तः सर्वकिल्बिषैः}


\twolineshloka
{पूर्णचन्द्रप्रकाशेन विमानेन विराजता}
{कामरूपि यथाकामं स्वर्गलोके महीयते}


\twolineshloka
{ततोऽवतीर्णः कालेन लोके चास्मिन्महायशाः}
{सर्वलोकार्चितः श्रीमान्राजा भवति वीर्यवान्}


\twolineshloka
{तिलपर्वतकं यस्तु श्रोत्रियाय प्रयच्छति}
{विशेषेण दरिद्राय तस्यापि शृणु यत्फलम्}


\twolineshloka
{पुण्यं वृषायुतोत्सर्गे यत्प्रोक्तं पाण्डुनन्दन}
{तत्पुण्यं समनुप्राप्य तत्क्षणाद्विरजा भवेत्}


\twolineshloka
{यथा त्वचं भुजङ्गो वै त्यक्त्वा शुद्धतनुर्भवेत्}
{तता तिलप्रदानाद्वै पापं त्यक्त्वा विसुद्ध्यति}


\twolineshloka
{तिलषण्डं प्रयुञ्जानो जांबूनदविभूषितम्}
{विमानं दिव्यमारूढः पितृलोके महीयते}


\twolineshloka
{षष्टिं वर्षसहस्राणि कामरूपी महायशाः}
{तिलप्रदाता रमते पितृलोके यथासुखम्}


\twolineshloka
{यः प्रयच्छति विप्राय तिलधेनुं नराधिप}
{श्रोत्रियाय दरिद्राय शृणु तस्यापि यत्फलम्}


\twolineshloka
{गोसहस्रप्रदानेन यत्पुण्यं समुदाहृतम्}
{तत्पुण्यफलामाप्नोति तिलधेनुप्रदो नरः}


\twolineshloka
{तिलानां कुडवैर्यस्तु तिलधेनुं प्रयच्छति}
{तावत्कोटिसमा राजन्स्वर्गलोके महीयते}


\threelineshloka
{अष्टाढकतिलैः कृत्वा तिलधेनु नराधिप}
{द्वात्रिंशन्निष्कसंयुक्तं विषुवे यः प्रयच्छति}
{मद्भक्त्या मद्गतात्मा वै तस्य पुण्यफलं शृणु}


\twolineshloka
{कन्यादानसहस्रस्य विधिदत्तस्य यत्फलम्}
{तत्पुण्यं समनुप्राप्तो मम लोके महीयते}


\twolineshloka
{मम लोकावतीर्णश्च सोस्मिँल्लोकेऽभिजायते}
{ऋग्यजुस्सामवेदानां पारगो ब्राह्मणर्षभः}


\twolineshloka
{गां तु यस्तु दरिद्राय श्रोत्रियाय प्रयच्छति}
{प्रसन्नां क्षीरिणीं पुण्यां सवत्सां कांस्यदोहिनीं}


\twolineshloka
{यत्किंचिद्दुष्कृतं कर्म तस्य पूर्वकृतं नृपः}
{तत्सर्वं तत्क्षणादेव विनश्यति न संशयः}


\twolineshloka
{यानं च वृषसंयुक्तं दीप्यमानं स्वलङ्कृतम्}
{आरूढः कामगं दिव्यं गोलोकमधिगच्छति}


\twolineshloka
{यावन्ति चैव रोमाणि तस्या गोस्तु नराधिप}
{तावद्वर्षसहस्राणि गवां लोके महीयते}


\twolineshloka
{गोलोकादवतीर्णस्तु लोकेऽस्मिन्ब्राह्मणो भवेत्}
{सत्रयाजी वदन्यश्च सर्वराजभिरर्चितः}


\twolineshloka
{तिलं गावः सुवर्णं चाप्यन्नं कन्या वसुंधरा}
{तारयन्तीह दत्तानि ब्राह्मणेभ्यो महाभुज}


\twolineshloka
{ब्राह्मणं वृत्तसंपन्नमाहिताग्निमलोलुपम्}
{तर्पयेद्विधिवद्राजन्स निधिः पारलौकिकः}


\twolineshloka
{आहिताग्नि दरिद्रं च श्रोत्रियं च जितेन्द्रियम्}
{शूद्रान्नवर्जितं चैव द्विजं यत्नेन पूजयेत्}


\twolineshloka
{आहिताग्निः सदा पात्रमह्निहोत्री च वेदवित्}
{पात्राणामपि तत्पात्रं शूद्रान्नं यस्य नोदरे}


\twolineshloka
{यच्च वेदमयं पात्रं यच्च पात्रं तपोमयम्}
{असंकीर्णं च यत्पात्रं तत्पात्रं तारयिष्यति}


\twolineshloka
{नित्यस्वाध्यायनिरतास्त्वसंकीर्णेन्द्रियाश्च ये}
{पञ्चयज्ञपरा नित्यं पूजितास्तारयन्ति ते}


\twolineshloka
{ये क्षान्तिदान्ताः श्रुतिपूर्णकर्णाजितेन्द्रियाः प्राणिवधै निवृत्ताः}
{प्रतिग्रहे संकुचिता गृहस्था-स्ते ब्राह्मणास्तारयितुं समर्थाः}


\twolineshloka
{नित्योदकी नित्ययज्ञोपवीतीनित्यस्वाध्यायी वृषलान्नवर्जी}
{क्रतौ गच्छन्विधिवच्चापि जुह्व-त्स ब्राह्मणस्तारयितुं समर्थः}


\twolineshloka
{ब्राह्मणो यस्तु मद्भक्तो मद्रागी मत्परायणः}
{मयि संन्यस्तकर्मा च स विप्रस्तारयेद्ध्रुवम्}


\twolineshloka
{द्वादशाक्षरतत्वज्ञश्चतुर्व्यूहविभागवित्}
{अच्छिद्रपञ्चकालज्ञःक स विप्रस्तारयिष्यति}


\chapter{अध्यायः १०३}
\twolineshloka
{वासुदेवेन दानेषु कथितेषु यथाक्रमम्}
{अवितृप्तश्च धर्मेषु केशवं पुनरब्रवीत्}


\twolineshloka
{देव धर्मामृतमिदं शृण्वतोपि परंतप}
{न विद्यते सुरश्रेष्ठ मम तृप्तिर्हि माधव}


\twolineshloka
{अनडुत्संप्रदानस्य यत्फलं तु विधीयते}
{तत्फलं कथयस्वेह तव भक्तस्य मेऽच्युत}


\twolineshloka
{यानिचान्यानिदानानित्वया नोक्तानि कानिचित् ॥तान्याचक्ष्व सुरश्रेष्ठ तेषां चानुक्रमात्फलम् ॥भगवानुवाच}
{}


\twolineshloka
{पवित्रत्वात्सुपुण्यत्वात्पावनत्वात्तथैव च}
{शृणु धर्मामृतं श्रेष्ठं दत्तस्यानडुहः फलम्}


\twolineshloka
{दशधेनुसमोऽनड्वानेकोपि कुरुपुङ्गव}
{मेदोमांसविपुष्टाङ्गो नीरोगः कोपवर्जितः}


\twolineshloka
{युवा भद्रः सुशीलश्च सर्वदोषविवर्जितः}
{धुरं धारयति क्षिप्रं दत्तो विप्राय पाण्डव}


\twolineshloka
{स तेन पुण्यदानेन वर्षकोटिं युधिष्ठिर}
{यथाकामं महादेजा गवां लोके महीयते}


\threelineshloka
{यश्च दद्यादनडुहौ द्वौ युक्तौ च धुरंधरौ}
{सुवृत्ताय दरिद्राय श्रोत्रियाय विशेषतः}
{तस्य यत्पुण्यमाख्यातं तच्छृणुष्व युधिष्ठिर}


% Check verse!
सहस्रगोप्रदानेन यत्प्रोक्तं फलमुत्तमम् ॥तत्पुण्यफलमाप्नोति याति लोकान्स मामकान्
\twolineshloka
{यावन्ति चैव रोमाणि तयोरनुडुहोर्नृप}
{तावद्वर्षसहस्राणि मम लोके महीयते}


\twolineshloka
{दरिद्रायैव दातव्यं न समृद्धाय पाण्डव}
{वर्षाणां हि तटाकेषु फलं नैव पयोधिषु}


\twolineshloka
{यस्तु दद्यादनडुहं दरिद्राय द्विजातये}
{स तेन पुण्यदानेन पुतात्मा कुरुपुङ्गव}


\twolineshloka
{विमानं दिव्यमारूढो दिव्यरूपी यथासुखम्}
{मम लोकेषु रमते यावदाभूतसंप्लुवम्}


\threelineshloka
{गृहं दीपप्रभायुक्तं शय्यासनविभूषितम्}
{भाजनोपस्करैर्युक्तं धनधान्यैरलङ्कृतम्}
{दासीगोभूमिसंयुक्तमन्यूनं सर्वसाधनैः}


\twolineshloka
{ब्राह्मणाय दरिद्राय श्रोत्रियाय युधिष्ठिर}
{दद्यात्सदक्षिणं यस्तु तस्य पुण्यफलं शृणु}


\twolineshloka
{देवाः पितृगणाश्चैव ह्यग्नयो ऋषयस्तथा}
{प्रयच्छन्ति प्रहृष्टा वै यानमादित्यसन्निभम्}


\threelineshloka
{तेन गच्छेच्छ्रिया युक्तो ब्रह्मलोकमनुत्तमम्}
{स्त्रीसहस्रावृते रम्ये भवने तत्र काञ्चने}
{मोदते ब्रह्मलोकस्थो यावदाभूतसप्लवम्}


\threelineshloka
{शय्यं प्रस्तरणोपेतां यः प्रयच्छति पाण्डव}
{अर्चयित्वा द्विजं भक्त्या वस्त्रमाल्यानुलेपनैः}
{भोजयित्वा विचित्रान्नं तस्य पुण्यफलं शृणु}


\twolineshloka
{धेनुदानस्य यत्पुण्यं विधिदत्तस्य पाण्डव}
{तत्पुण्यं तमनुप्राप्य पितृलोके महीयते}


\twolineshloka
{शिल्पमध्ययनं वाऽपि विद्यां मन्त्रौषधानि च}
{यः प्रयच्छति विप्राय तस्य पुण्यफलं शृणु}


\twolineshloka
{आहिताग्निसहस्रस्य पूजितस्यैव यत्फलम्}
{तत्पुण्यफलमाप्नोति यस्तु शय्यां प्रयच्चति}


\twolineshloka
{छन्दोभिः संप्रयुक्तेन विमानेन विराजता}
{सप्तर्षिलोकान्व्रजति पूज्यते ब्रह्मवादिभिः}


\twolineshloka
{चतुर्युगानि वै त्रिंशत्क्रीडित्वा तत्र देववत्}
{इह मानुष्यके लोके विप्रो भवति वेदवित्}


\twolineshloka
{विश्रामयति यो विप्रं श्रान्तमध्वनि कर्शितम्}
{कविनश्यति तदा पापं तस्य वर्षकृतं नृप}


\twolineshloka
{अथ प्रक्षालयेत्पादौ तस्य तोयेन भक्तिमान्}
{दशवर्षकृतं पापं व्यपोहति न संशयः}


\twolineshloka
{घृतेन वाऽथ तैलेन पादौ तस्य तु पूजयेत्}
{तद्द्वादसमारूढं पापमाशु व्यपोहति}


\twolineshloka
{धेनुकाञ्चनदत्तस्य यत्पुण्यं समुदाहृतम्}
{तत्पुण्यफलमाप्नोति यस्त्वेनं विप्रमर्चयेत्}


\twolineshloka
{स्वागतेन तु यो विप्रं पूजयेदासनेन च}
{प्रत्युत्थानेन वा राजन्स देवानां प्रियो भवेत्}


\twolineshloka
{स्वागतेनाग्नयो राजन्नासनेन शतक्रतुः}
{प्रत्युत्थानेन पितरः प्रीति यान्त्यतिथिप्रियाः}


\twolineshloka
{अग्निशक्रपितॄणां च तेषां प्रीत्या नराधिप}
{संवत्सरकृतं पापं तस्य सद्यो विनश्यति}


\twolineshloka
{यः प्रयच्छति विप्राय आसनं माल्यभूषितम्}
{स याति मणिचित्रेण रथेनेन्द्रनिकेतनम्}


\twolineshloka
{पुरन्दरासने तत्र दिव्यनारीविभूषितः}
{षष्टिं वर्षसहस्राणि क्रीडत्यप्सरसां गणैः}


\twolineshloka
{वाहनं यः प्रयच्छेत ब्राह्मणाय युधिष्ठिर}
{स याति रत्नचित्रेण वाहनेन सुरालयम्}


\twolineshloka
{स तत्र कामं क्रीडित्वा सेव्यमानोप्सरोगणैः}
{इह राजा भवेद्राजन्नात्र कार्या विचारणा}


\twolineshloka
{पादपं पल्लवाकीर्णं पुष्पतिं फलितं तथा}
{गन्धमाल्यैरथाभ्यर्च्य वस्त्राभरणभूषितम्}


\twolineshloka
{यः प्रयच्छति विप्राय श्रोत्रियाय सदक्षिणम्}
{भोजयित्वा यथाकामं तस्य पुण्यफलं शृणु}


\twolineshloka
{जांबूनदविचित्रेण विमानेन विराजता}
{पुरन्दरपुरं याति जयशब्दरवैर्युतः}


\twolineshloka
{ततः शक्रपुरे रम्ये तस्य कल्पकपादपः}
{ददाति चेप्सितं सर्वं मनसा यद्यदिच्छति}


\twolineshloka
{यावन्ति तस्य पत्राणि पुष्पाणि च फलानि च}
{तावद्वर्षसहस्राणि शक्रलोके महीयते}


\twolineshloka
{शक्रलोकावतीर्णश्च मानुष्यं लोकमागतः}
{रथाश्वगजसंपूर्णं पुरं राज्यं च वक्ष्यति}


\threelineshloka
{स्थापयित्वा तु मद्भक्त्या यो मत्प्रतिकृति नरः}
{आलयं विधिवत्कृत्वा पूजाकर्म च कारयेत्}
{स्वयं वा पूजयेद्भक्त्या तस्य पुण्यफलं शृणु}


\threelineshloka
{अश्वमेधसहस्रस्य यत्पुण्यं समुदाहृतम्}
{तत्फलं समवाप्नोति मत्सालोक्यं प्रपद्यते}
{न जाने निर्गमं तस्य मम लोकाद्युधिष्ठिर}


\twolineshloka
{देवालये विप्रगृहे गोवाटे चत्वरेऽपि वा}
{प्रज्वालयति यो दीपं तस्य पुण्यफलं शृणु}


\twolineshloka
{आरुद्य काञ्चनं यानं द्योतयन्सर्वतो दिशम्}
{गच्छेदादित्यलोकं स सेव्यमानः सुरोत्तमैः ॥छ}


\twolineshloka
{तत्र प्रकामं क्रीडित्वा वर्षकोटिं महातपाः}
{इह लोके भवेद्विप्रो वेदवेदाङ्गपारगः}


\twolineshloka
{देवालयेषु वा राजन्ब्राह्मणावसथेषु वा}
{चत्वरे वा चतुष्के वा रात्रौ वा यदि वा दिवा}


\twolineshloka
{नानागन्धर्ववाद्यानि धर्मश्रावणिकानि च}
{यस्तु कारयते भक्त्या मद्गतेनान्तरात्मना}


\twolineshloka
{तस्य देवा नरश्रेष्ठ पितरश्चापि हर्षिताः}
{सुप्रीताः संप्रयच्चन्ति विमानं कामगं सुखम्}


\twolineshloka
{स च तेन पिमानेन याति देवपुरं नरः}
{तत्र दिव्याप्सरोभिस्तु सेव्यमानः प्रमोदते}


\twolineshloka
{देवलोकावतीर्णस्तु सोस्मिँल्लोके नराधिप}
{वेदवेदाङ्गतत्वज्ञो भोगवान्ब्राह्मणो भवेत्}


\twolineshloka
{चत्वरे वा सभायां वा विस्तीर्णि वा सभाङ्गणे}
{कृत्वाऽग्निकुण्डं विपुलं स्थण्डिलं वा युधिष्ठिर}


\twolineshloka
{तत्राग्निं चतुरो मासाञ्ज्वालयेद्यस्तु भक्तिमाम्}
{समाप्तेषु च मासेषु पौष्यादिषु ततो द्विजान्}


\twolineshloka
{भोजयेत्पायसं मृष्टं मद्गतेनान्तरात्मना}
{दक्षिणां च यथाशक्ति ब्राह्म्णेभ्यो निवेदयत्}


\twolineshloka
{एवमग्निं तु यः कुर्यान्नित्यमेवार्चयेत्तु माम्}
{तस्य पुण्यफलं यद्वै तन्निबोध युधिष्ठिर}


\twolineshloka
{तेनाहं शङ्करश्चैव पितरो ह्यग्नयस्तथा}
{यास्यामः परमां प्रीतिं नात्र कार्याविचारणा}


\twolineshloka
{षष्टिं वर्षसहस्राणि षष्टिं वर्षशतानि च}
{सोस्मत्प्रीतिकरः श्रीमान्मम लोके महीयते}


\twolineshloka
{मम लोकावतीर्णश्च अस्मिँल्लोके महायशाः}
{वेदवेदाङ्गविद्विप्रो जायते राजपूजितः}


\threelineshloka
{यः करोति नरश्रेष्ठ भरणं ब्राह्मणस्य तु}
{श्रोत्रियस्याभिजातस्य दरिद्रस्य विशेषतः}
{तस्य पुण्यफलं यद्वै तन्निबोध युधिष्ठिर}


\twolineshloka
{गवां कोटिप्रदानेन यत्पुण्यं समुदाहृतम्}
{तत्सर्वफलमाप्नोति वर्षेणैकेन पाण्डव}


\twolineshloka
{काञ्चनेन विचित्रेण विमानेनार्कशोभिना}
{स याति मामकं लोकं दिव्यस्त्रीगणसेवितः}


\threelineshloka
{गीयमानो वरस्त्रीभिर्वर्षाणां कोटिविंशतिम्}
{क्रीडित्वा मामके तत्र सर्वदेवैरभिष्टुतः}
{मानुष्यमवतीर्णस्तु वेदविद्ब्राह्मणो भवेत्}


\twolineshloka
{करकां कर्णिकां वाऽपि महद्वा जलभाजनम्}
{यः प्रयच्छति विप्राय तस्य पुण्यफलं शृणु}


\threelineshloka
{ब्रह्मकूर्चे तु यत्पीते फलं प्रोक्तं नरादिप}
{तत्पुण्यफलमाप्नोति जलभाजनदो नरः}
{सुतृप्तः सर्वसौगन्धः प्रहृष्टेन्द्रियमानसः}


\twolineshloka
{हंससारसयुक्तेन विमानेन विराजता}
{स याति वारुणं लोकं दिव्यगन्धर्वसेवितम्}


\twolineshloka
{पानीयं यः प्रयच्छेद्वै जीवानां जीवनं परम्}
{ग्रीष्मे च त्रिषु मासेषु तस्य पुण्यफलं शृणु}


\twolineshloka
{कपिलाकोटिनानस्य यत्पुण्यं तु विधीयते}
{तत्पुण्यफलमाप्नोति पानीयं यः प्रयच्छति}


\twolineshloka
{पूर्णचन्द्रप्रकासेन विमानेन विराजता}
{स गच्छेच्चन्द्रभवनं सेव्यमानोप्सरोगणैः}


\twolineshloka
{त्रिंशत्कोटियुगं तत्र दिव्यगन्धर्वसेवितः}
{क्रीडित्वा मानुषे लोके चतुर्वेदी द्विजो भवेत्}


\twolineshloka
{शिरोभ्यङ्गप्रदानेन तेजस्वी प्रियदर्शनः}
{सुभगो रूपवाञ्शूरः पण्डितश्च भवेद्द्विजः}


\twolineshloka
{वस्त्रदायी तु तेजस्वी सर्वत्र प्रियदर्शनः}
{सुभगो भवति श्रीमान्स्त्रीणां नित्यं मनोरमः}


\threelineshloka
{उपानहौ च च्छत्रं च यो ददाति नरोत्तमः}
{स याति रथमुख्येन काञ्चनेन विराजता}
{शक्रलोकं महातेजाः सेव्यमानोप्सरोगणैः}


\twolineshloka
{काष्ठपादुकदा यान्ति विमानैर्वृक्षनिर्मितैः}
{धर्मराजपुरं रम्यं सेव्यमानाः सुरोत्तमैः}


\twolineshloka
{दन्तकाष्ठप्रादनेन प्रियवाक्यो भवेन्नरः}
{सुगन्धवदनः श्रीमान्मेदासौभाग्यसंयुतः}


\twolineshloka
{क्षीरं दधि घृतं वाऽपि गुडं मधुरसं तथा}
{ये प्रयच्छन्ति विप्रेभ्यः परां भक्तिमुपागताः}


\twolineshloka
{ते वृषैरश्वयानैश्च श्वेतस्रग्दामभूषिताः}
{उपगीयमाना गन्धर्वैर्यान्तीश्वरपुरं नराः}


\twolineshloka
{तत्र दिव्याप्सरोभिस्तु सेव्यमाना यथासुखम्}
{षष्टिवर्षसहस्राणि मोदन्ते देवसन्निभाः}


% Check verse!
ततः कालावतीर्णाश्च जायन्ते त्विह मानवाः ॥प्रभूतधनधान्याश्च भोगवन्तो नरोत्तमाः
\twolineshloka
{वैशाखे मासि वैशाखे दिवसे पाण्डुनन्दन}
{वैवस्वतं समुद्दिश्य परां भक्तिमुपागताः}


\twolineshloka
{अभ्यर्च्य विधिवद्विप्रांस्तिलान्गुडसमन्वितान्}
{ये प्रयच्छन्ति विप्रेभ्यस्तेषां पुण्यफलं शृणु}


\threelineshloka
{गोप्रदानेन यत्पुण्यं विधिवत्पाण्डुनन्दन}
{तत्पुण्यं समनुप्राप्तो यमलोके महीयते}
{ततश्चापि च्युतः कालादिह राजा भविष्यति}


\twolineshloka
{तस्मिन्नेव दिने विप्रान्भोजयित्वा सुदक्षिणम्}
{तोयपूर्णानि दिव्यानि भाजनानि दिशन्ति ये}


\twolineshloka
{ते यान्त्यादित्यवर्णाभैर्विमानैर्वरुणालयम्}
{तत्र दिव्याङ्गनाभिस्तु रमन्ते कामकामिनः}


\twolineshloka
{ततोऽवतीर्णाः कालेन ते चास्मिन्मानुषे पुनः}
{भोगवन्तो द्विजश्रेष्ठ भविष्यन्ति न संशयः}


\threelineshloka
{अनन्तराशी यश्चापि वर्तते व्रतवत्सदा}
{सत्यवाक्क्रोधरहितः शुचिः स्नानरतः सदा}
{स विमानेन दिव्येन याति शक्रपुरं नरः}


\twolineshloka
{तत्र दिव्याप्सरोभिस्तु वर्षकोटिं महातपाः}
{क्रीडित्वा मानुषे लोके जायते वेदविद्द्विजः}


\threelineshloka
{एकभुक्तेन यश्चापि वर्षमेकं तु वर्तते}
{ब्रह्मचारी जितक्रोधः सत्यशौचसमन्वितः}
{स विमानेन दिव्येन याति शक्रपुरं नरः}


\twolineshloka
{दशकोटिसहस्राणि क्रीडित्वाऽप्सरसां गणैः}
{इह मानुष्यके लोके वेदविद्ब्राह्मणो भवेत्}


\twolineshloka
{चतुर्थकाले यो भुङ्क्ते ब्रह्मचारी जितेन्द्रियः}
{वर्तते चैकवर्षं तु तस्य पुण्यफलं शृणु}


\twolineshloka
{चित्रबर्हिणयुक्तेन विचित्रध्वजशोभिना}
{याति यानेन दिव्येन स महेन्द्रपुरं नरः}


\twolineshloka
{अकृशाभिर्वरस्त्रीभिः सेव्यमानो यथासुखम्}
{ततो द्वादशकोटिं स समाः सम्यक्प्रमोदते}


\twolineshloka
{शक्रलोकावतीर्णस्तु लोके चास्मिन्नराधिप}
{भवेद्वै ब्राह्मणो विद्वान्क्षमावान्वेदपारगः}


\threelineshloka
{षष्ठकाले तु योऽश्नाति वर्षमेकमकल्मषः}
{ब्रह्मचर्यव्रतैर्युक्तः शुचि क्रोधविवर्जितः}
{तपोयुक्तस्य तस्याथ शृणुष्व फलमुत्तमम्}


\twolineshloka
{अत्यादित्यप्रकाशेन विमानेनार्कसंनिभः}
{स याति मम लोकान्वै दिव्यनारीनिषेवितः}


\twolineshloka
{तत्र साध्यैर्मरुद्भिस्तु पूज्यमानो यथासुखम्}
{पश्यन्नेव सदा मां तु क्रीडत्यप्सरसां गणैः}


\twolineshloka
{पक्षोपवासं यश्चापि कुरुते मद्गतात्मना}
{समाप्ते तु व्रते तस्मिंस्तर्पयेच्छ्रोत्रियान्द्विजान्}


\threelineshloka
{सोपि गच्छति दिव्येन विमानेन महातपाः}
{द्योतयनप्रभया व्योम मम लोकं प्रपद्यते}
{स तत्र मोदते कामं कामरूपी यथासुखम्}


\threelineshloka
{त्रिंशत्कोटिसमा राजन्क्रीडित्वा तत्र देववत्}
{इह मानुष्यके लोके पूजनीयो द्विजो भवेत्}
{त्रयाणामपि वेदानां साङ्गानां पारगे भवेत्}


\twolineshloka
{यश्च मासोपवासं वै कुरुते मद्गतात्मना}
{जितेन्द्रियो जितक्रोधोजितधीः स्नानतत्परः}


\twolineshloka
{समाप्ते नियमे तत्र भोजयित्वा द्विजोत्तमान्}
{दक्षिणां च ततो दद्यात्प्रहृष्टेनान्तरात्मना}


\twolineshloka
{स गच्छति महातेजा ब्रह्मलोकमनु****म्}
{सिंहयुक्तेन यानेन दिव्यस्त्रीगणसेवितः}


\twolineshloka
{स तत्र ब्रह्मणो लोके दिव्यर्षिगणसेवितः}
{शतकोटिसमा राजन्यथाकामं प्रमोदते}


\twolineshloka
{ततः कालावतीर्णश्च सोस्मिँल्लोके द्विजो भवेत्}
{षडङ्गविच्चतुर्वेदी त्रिंशज्जन्मान्यरोगवान्}


\twolineshloka
{यस्त्यक्त्वा सर्वकर्माणि शुचिः क्रोधविवर्जितः}
{महाप्रस्तानमेकाग्नो याति मद्गतमानसः}


\twolineshloka
{स गच्छेदिन्द्रसदनं विमानेन महातपाः}
{महामणिविचित्रेण सौवर्णन विराजता}


\twolineshloka
{शतकोटिसमास्तत्र सुराधिपतिपूजितः}
{नाकपृष्ठे निवसति दिव्यस्त्रीगणसेवितः}


\twolineshloka
{शक्रलोकावतीर्णश्च मानुषेषूपजायते}
{राज्ञां राजा महातेजाः सर्वलोकार्चितः प्रभुः}


\threelineshloka
{प्रायोपवेशं यश्चापि कुरुते मद्गतात्मना}
{नमो ब्रह्मण्यदेवायेत्युक्त्वा मन्त्रं समाहितः}
{अन्तःस्वस्थो जितक्रोधस्तस्य पुण्यफलं शृणु}


\twolineshloka
{कामगः कामरूपी च बालसूर्यसमप्रभः}
{स विमानेन दिव्येन याति लोकाननामयान्}


\twolineshloka
{स्वर्गत्स्वर्गं महातेजा गत्वा चैव यथासुखम्}
{मम लोकेषु रमते यावदाभूतसंप्लवम्}


\twolineshloka
{अग्निप्रवेशं यश्चापि कुरुते मद्गतात्मना}
{सोपि यानेन दिव्येन मम लोकं प्रपद्यते}


\threelineshloka
{तत्र सर्वगुणोपेतः पश्यन्नेव स मां सदा}
{त्रिंशत्कोटिसमा राजन्मोदते मम संनिधौ}
{ततोऽवतीर्णः कालेन वेदविद्ब्राह्मणो भवेत्}


\twolineshloka
{कर्षणं साधयन्यस्तु मां प्रपन्नः शुचिव्रतः}
{नमो ब्रह्मण्यदेवायेत्येतन्मन्त्रमुदाहरन्}


\threelineshloka
{बालसूर्यप्रकाशेन विमानेन विराजता}
{मम लोकं समासाद्य वर्षकोटिं प्रमोदते}
{मम लोकावतीर्णश्च सोस्मिँल्लोके नृपो भवेत्}


\twolineshloka
{निवेशयति मन्मूर्त्यामात्मानं मद्गतः शुचिः}
{रुद्रदक्षिणमूर्त्यां वा चतुर्दश्यां विशेषतः}


\twolineshloka
{सिद्धैर्ब्रह्मर्षिभिश्चैव देवलोकैश्च पूजितः}
{गन्धर्वैर्भूतसङ्घैश्च गीयमानो महातपाः}


\twolineshloka
{प्रविशेत्स महातेजा मां वा शङ्करमेव वा}
{न स्यात्पुनर्भवो राजन्नात्र कार्या विचारणा}


\twolineshloka
{गोकृते स्त्रीकृते चैव गुरुविप्रकृतेऽपि वा}
{हन्यन्ते ये तु राजेन्द्र शक्रलोकं व्रजन्ति ते}


\twolineshloka
{तत्र जांबूनदमये विमाने कामगामिनि}
{मन्वन्तरं प्रमोदन्ते दिव्यनारीनिषेविताः}


\twolineshloka
{आश्रुतस्याप्रादनेन दत्तस्य हरणेन च}
{जन्मप्रभृति यद्दत्तं तत्सर्वं तु विनश्यति}


\twolineshloka
{नाऽगोप्रदास्तत्र पयः पिबन्तिनाभूमिदा भूमिमथाश्नुवन्ति}
{यान्यान्कामान्ब्राह्मणेभ्यो ददातितांस्तान्कामान्स्वर्गलोके च भुङ्क्ते}


\twolineshloka
{यद्यदिष्टतमं द्रव्यं न्यायनोपार्जितं च यत्}
{तत्तद्गुणवते देयं तदेवाक्षयमिच्छता}


\twolineshloka
{अनुपोष्य त्रिरात्राणि तीर्थान्यनभिगम्य च}
{अदत्त्वा काञ्चनं गां दरिद्रो नाम जायते}


\twolineshloka
{दानं यत्तत्फलं नैव श्रोत्रियाय न दीयते}
{श्रोत्रिया यत्र नाश्नन्ति न देवास्तत्र भुञ्जते}


\twolineshloka
{श्रोत्रियेभ्यः परं नास्ति परमं दैवतं महत्}
{निधानं चापि राजेन्द्र नास्माच्छ्रोत्रियभोजनम्}


\chapter{अध्यायः १०४}
\threelineshloka
{विप्रयोगे शरीरस्य सेन्द्रियस्य विशेषतः}
{अन्तरा वर्तमानस्य गतिः प्राणस्य कीदृशी ॥भगवानुवाच}
{}


\twolineshloka
{शुभाशुभकृतं सर्वं प्राप्नोतीह फलं नरः}
{न तु सर्वस्य भूतस्य पञ्चत्वं विद्यते नृप}


\twolineshloka
{पञ्चत्वं पाण्डवश्रेष्ठ भूरिभूतिकरं नृणाम्}
{तेषां पञ्च महायज्ञान्ये कुर्वन्ति द्विजोत्तम}


\twolineshloka
{पञ्चत्वं पञ्चभिर्भूतैर्वियोगं संप्रचक्षते}
{न जायते न म्रियते पुरुषः शाश्वतः सदा}


\twolineshloka
{प्रायेण मरणं नाम पापिनामेव पाण्डव}
{येषां तु न गतिः पुण्या तेषां मरणमुच्यते}


\threelineshloka
{प्रायेणाकृतकृत्यस्तु मृत्योरुद्विजते जनः}
{कृतकृत्याः प्रतीक्षन्ते मृत्युं प्रियमिवातिथिम् ॥युधिष्ठिर उवाच}
{}


\threelineshloka
{पञ्च यज्ञाः कथं देव क्रियन्तेऽत्र द्विजातिभिः}
{तेषां नाम च देवेश वक्तुमर्हस्यशेषतः ॥श्रीभगवानुवाच}
{}


\twolineshloka
{शृणु पञ्च महायज्ञान्कीर्त्यमानान्युधिष्ठिर}
{यैरेव ब्रह्मसालोक्यं लभ्यते गृहमेधिना}


\twolineshloka
{ऋभुजज्ञं ब्रह्मयज्ञं भूतयज्ञं च पाण्डव}
{नृयज्ञं पितृयज्ञं च पञ्च यज्ञान्प्रचक्षते}


\threelineshloka
{तर्पणं ऋभुयज्ञः स्यात्स्वाध्यायो ब्रह्मयज्ञकः}
{भूतयज्ञो बलिर्यज्ञो नृयज्ञोऽतिथिपूजनम्}
{पितॄनुद्दिश्य यत्कर्म पितृयज्ञः प्रकीर्तितः}


\twolineshloka
{हुतं चाप्यहुतं चैव तथा प्रहुतमेव च}
{प्राशितं बलिदानं च पाकयज्ञान्प्रचक्षते}


\twolineshloka
{वैश्वदेवादयो होमा हुतमित्युच्यते बुधैः}
{अहुतं च भवेद्दत्तं प्रहुतं ब्राह्मणाशितम्}


\twolineshloka
{प्राणाग्निहोत्रहोमं च प्राशितं विधिवद्विदुः}
{बलिकर्म च राजेन्द्र पाकयज्ञाः प्रकीर्तिताः}


\twolineshloka
{केचित्पञ्च महायज्ञान्पाकयज्ञान्प्रचक्षते}
{अपरे ब्रह्मयज्ञादीन्महायज्ञविदो विदुः}


\twolineshloka
{सर्व एते महायज्ञाः सर्वथा परिकीर्तिताः}
{बुभुक्षितान्ब्राह्मणांस्तु यथाशक्ति न हापयेत्}


\twolineshloka
{अहन्यहनि ये त्वेतानकृत्वा भुञ्जते स्वयम्}
{केवलं मलमश्नन्ति ते नार न च संशयः}


\threelineshloka
{तस्मात्स्नात्वा द्विजो विद्वान्कुर्यादेतान्दिनेदिने}
{अतोऽन्यथा तु भुञ्जन्वै प्रायश्चित्ती भवेद्द्विजः ॥युधिष्टिर उवाच}
{}


\threelineshloka
{देवदेवेशक दैत्यघ्न त्वद्भक्तस्य जनार्दन}
{वक्तुमर्हसि देवेश स्नानस्य च विधिं मम ॥भगवानुवाच}
{}


\twolineshloka
{शृणु पाण्डव तत्सर्व पवित्रं पापनाशनम्}
{स्नात्वा येन विधानेन मुच्यन्ते किल्बिषाद्द्विजाः}


\twolineshloka
{मृदं च गोमयं चैव तिलं दर्भांस्तथैव च}
{पुष्पाण्यपि यथान्यायमादाय तु जलं व्रजेत्}


\twolineshloka
{नद्यां स्नात्वा न च स्नायदन्यत्र द्विजसत्तम}
{सति प्रभूते पयसि नाल्पे स्नायात्कदाचन}


\twolineshloka
{गत्वोदकसमीपं तु शुचौ देशे मनोरमे}
{ततो मृद्गोमयादीनि तत्र विप्रो विनिक्षिपेत्}


\twolineshloka
{बहिः प्रक्षाल्य पादौ च द्विराचम्य प्रयत्नतः}
{प्रदक्षिणं समावृत्य नमस्कुर्यात्तु तज्जलम्}


\twolineshloka
{न च प्रक्षालयेद्विद्वांस्तीर्थमद्भिः कदाचन}
{न च पादेन वा हन्याद्धस्तेनान्येन तज्जलम्}


\twolineshloka
{सर्वदेवमया ह्यापो मन्मयाः पाण्डुनन्दन}
{तस्मात्तास्तु न हन्तव्यास्त्वद्भिः प्रक्षालयेत्स्थलं}


\twolineshloka
{केवलं प्रथमं मज्जेन्नाङ्गानि विमृशेद्बुधः}
{तत्तु तीर्थं समासाद्य कुर्यादाचमनं पुनः}


\threelineshloka
{गोकर्णाकृतिवत्कृत्वा करं त्रिः प्रपिबेज्जलम्}
{द्विस्तत्परिमृजेद्वक्त्रं पादावभ्युक्ष्य चात्मनः}
{शीर्षण्यांस्तु ततः प्राणान्सकृदेवतु संस्पृशेन}


\twolineshloka
{बाहू द्वौ च ततः स्पृष्ट्वा हृदयं नाभिमेव च}
{प्रत्यङ्गमुदकं स्पृष्ट्वा मूर्धानं तु पुनः स्पृशेत्}


\twolineshloka
{आपः पुनन्त्वित्युक्त्वा च पुनराचमनं चरेत्}
{सोङ्कारव्याहृतीर्वाऽपि सदसस्पतिमित्यृचम्}


\threelineshloka
{आचम्य मृत्तिकाः पश्चात्त्रिधा कृत्वा समालभेत्}
{ऋचेदं विष्णुरित्यङ्गमुत्तमाधममध्यमम्}
{आलभ्य वारुणैः सूक्तैर्नमस्कृत्य जलं ततः}


\twolineshloka
{स्रवन्ती चेत्प्रतिस्रोतः प्रत्यर्कं चान्यवारिषु}
{मज्जेदोमित्युदाहृत्य न च विक्षोभयेज्जलम्}


\twolineshloka
{गोमयं च त्रिधा कृत्वा जले पूर्वं समालभेत्}
{सव्याहृतीकां सप्रणवां गायत्रीं च जपेत्पुनः}


\threelineshloka
{पुनराचमनं कृत्वा मद्गतेनान्तरात्मना}
{आपोहिष्ठेति तिसृभिर्ऋग्भिः पूतेन वारिणा}
{तथा तरत्समन्दीभिः सिञ्चेच्चतसृभिः क्रमात्}


\twolineshloka
{गोसूक्तेनाश्वसूक्तेन शुद्धवर्गेणि चात्मनः}
{वैष्णवैर्वारुणैः सूक्तैः सावित्रैरिन्द्रदेवतैः}


\twolineshloka
{वामदैव्येन चात्मानमन्यैर्मन्मयसामभिः}
{स्थित्वाऽन्तःसलिले सूक्तं जपेद्वाचाऽघमर्षणं}


\twolineshloka
{सव्याहृतीकां सप्रणवां गायत्रीं वा ततो जपेत्}
{आश्वासमोक्षात्प्रणवं जपेद्वा मामनुस्मरन्}


% Check verse!
उत्प्लुत्य तीर्थमासाद्य धौते शुक्ले च वाससी ॥शुद्धे चाच्छादयेत्कक्षे न कुर्यात्परिपाशके
\threelineshloka
{पाशेन बद्ध्वा कक्षे यत्कुरुते कर्म वैदिकम्}
{राक्षसा दानवा दैत्यास्तद्विलुंपन्ति हर्षिताः}
{तस्मत्सर्वप्रयत्नेन कक्ष्यापाशं न धारयेत्}


\twolineshloka
{ततः प्रक्षाल्य पादौ च हस्तौ चैव मृदा शनैः}
{आचम्य पुनराचामेत्पुनः सावित्रिया द्विजः}


\threelineshloka
{प्राङ्मुखोदङ्मुखो वाऽपि ध्यायन्वेदान्समाहितः}
{जले जलगतः शुद्धः स्थल एव स्थलस्थितः}
{उभयत्र स्थितस्तस्मादाचामेदात्मशुद्धये}


\twolineshloka
{दर्भेषु दर्भपाणिः सन्प्राङ्मुखः सुसमाहितः}
{प्राणायामांस्ततः कुर्यान्मद्गतेनान्तरात्मना}


% Check verse!
सहस्रकृत्वः सावित्रीं शतकृत्वस्तु वा जपेत्
\twolineshloka
{समाहितो जपेत्तस्मात्सावित्र्या चाभिमन्त्र्य च}
{मन्देहानां विनाशाय रक्षसां विक्षिपेज्जलम्}


% Check verse!
उद्वर्गोसीत्यथा चान्तःप्रायश्चित्तजलं क्षिपेत्
% Check verse!
अथादाय सुपुष्पाणि तोयमञ्जलिना द्विजःप्रक्षिप्य प्रतिसूर्यं च व्योममुद्रां प्रकल्पयेत्
\twolineshloka
{ततो द्वादशकृत्वस्तु सूर्यस्येकाक्षरं जपेत्}
{ततः षडक्षरादीनि षट्कृत्वः परिवर्तयेत्}


\twolineshloka
{प्रदक्षिणं परामृश्य मुद्रया स्वमुखान्तरे}
{ऊर्ध्वबाहुस्ततो भूत्वा सूर्यमीक्षेत्समाहितः}


\twolineshloka
{तन्मण्डलस्थं मां ध्यायंस्तोजोमूर्तिं चतुर्भुजम्}
{उदुत्यं च जपेन्मन्त्रं चित्रं तच्चक्षुरित्यपि}


\twolineshloka
{सावित्रीं च यथाशत्ति जप्त्वा सूक्तं च मामकम्}
{मन्मयानि च सामानि पुरुषव्रतमेव च}


\twolineshloka
{ततश्चालोकयेदर्कं हंसः शुचिषदित्यपि}
{प्रदक्षिणं समावृत्य नमस्कृत्य दिवाकरम्}


\twolineshloka
{ततस्तु तर्पयेदद्भिर्ब्रह्म्णां मां च शङ्करम्}
{प्रजापतिं च देवांश्च तथा देवमुनीनपि}


\twolineshloka
{साङ्गानपि तथा वेदानितिहासान्क्रतूनपि}
{पुराणानि च सर्वाणि कुलान्यप्सरसां तथा}


\threelineshloka
{क्रतून्संवत्सरं चैव कलाकाष्ठात्मकं तथा}
{भूतग्रामांश्च भूतानि सरितः सागरांस्तथा}
{शैलाञ्शैलस्थितान्देवानोषधीः सवनस्पतीः}


\twolineshloka
{तर्पयेदुपवीती च प्रत्येकं तृप्यतामिति}
{अन्वारभ्य च सव्येन पाणिना दक्षिणेन तु}


\twolineshloka
{निवीती तर्पयेद्विद्वानृषीन्मन्त्रकृतस्तथा}
{मरीच्यादीनृषींश्चैव नारदाद्यान्समाहितः}


\twolineshloka
{प्राचीनावीत्यथैतास्तु तर्पयेद्देवताः पितॄत्}
{ततस्तु कव्यवाडग्निं सों वैवस्वतं तथा}


\threelineshloka
{ततश्चार्यमणं चापि ह्यग्निष्वात्तांस्तथैव च}
{सोमपांश्चैव दर्भेषु सतिलैरेव वारिभिः}
{तृप्यतामिति पश्चात्तु स पितॄंस्तर्पयेत्ततः}


\twolineshloka
{पितॄन्पितामहांश्चैव तथैव प्रपितामहान्}
{पितामहीस्तता चापि तथैव प्रपितामहीः}


\twolineshloka
{मातरं चात्मनश्चैव गुरुमाचार्यमेव च}
{पितृमातृष्वसारौ च तथा मातामहीमपि}


\twolineshloka
{उपाध्यायान्सखीन्बन्धूञ्शिष्यर्त्विग्ज्ञातिबांधवान्}
{प्रमीतानानृशंस्यार्थं तर्पयेत्तानमत्सरः}


\twolineshloka
{तर्पयित्वा तथाऽऽचम्य स्नानवस्त्रं प्रपीडयेत्}
{वृत्तिं भृत्यजनस्याहुः स्नानं पानं च तद्विदः}


\threelineshloka
{अतर्पयित्वा तान्पूर्वं स्नानवस्त्रं न पीडयेत्}
{पीडयेच्चेत्पुरा मोहाद्देवाः सर्पिगणास्तथा}
{पितरस्तु निराशास्ते शप्त्वा यान्ति यथागतं}


\twolineshloka
{प्रक्षाल्य तु मृदा पादावाचम्य प्रयतः पुनः}
{दर्भेषु दर्भपाणिः सन्स्वाध्यायं तु समारभेत्}


\twolineshloka
{वेदमादौ समारभ्य ततोपर्युपरि क्रमात्}
{यदधीतेऽन्वहं शक्त्या तत्स्वाध्यायं प्रचक्षते}


\twolineshloka
{ऋचो वाऽपि यजुर्वाऽपि सामगायमथापि च}
{इतिहासपुराणानि यथाशक्ति न हापयेत्}


\twolineshloka
{उत्थाय तु नमस्कृत्य दिशो दिग्देवता अपि}
{ब्रह्मणं च ततश्चाग्निं पृथिवीमोषधीस्तथा}


\twolineshloka
{वाचं वाचस्पतिं चैव मां चैव सरितस्तथा}
{नमस्कृत्य तथाऽद्भिस्तु प्रणवादि च पूर्ववत्}


\twolineshloka
{ततो नमोऽद्भ्य इत्युक्त्वा नमस्कुर्यात्तु तज्जलम्}
{घृणिः सूर्यस्तथाऽऽदित्यस्तं प्रणम्य स्वमूर्धनि}


\threelineshloka
{ततस्त्वालोकयन्नर्कं प्रणवेन समाहितः}
{ततो मामर्चयेत्पुष्पैर्मत्प्रियैरेव नित्यशः ॥युधिष्ठिर उवाच}
{}


\threelineshloka
{त्वत्प्रियाणि प्रसूनानि त्वदधिष्ठानि माधव}
{सर्वाण्याचक्ष्व देवेश त्वद्भक्तस्य ममाच्युत ॥भगवानुवाच}
{}


\twolineshloka
{शृणुष्वावहितो राजन्पुष्पाणि प्रियकृन्ति मे}
{कुमुदं करवीरं च चणकं चंपकं तथा}


\twolineshloka
{मल्लिकाजातिपुष्पं च नन्द्यावर्तं च नन्दिकम्}
{पलाशपुष्पपत्राणि दूर्वा भृङ्गकमेव च}


\twolineshloka
{वनमाला च राजेन्द्र मत्प्रियाणि विशेषतः}
{सर्वेषामपि पुष्पाणां सहस्रगुणमुत्पलम्}


\twolineshloka
{तस्मात्पद्मं तथा राजन्पद्मात्तु सतपत्रकम्}
{तस्मात्सहस्रपत्रं तु पुण्डरीकं ततः परम्}


\threelineshloka
{पुण्डरीकसहस्रात्तु तुलसी गुणतोऽधिका}
{बकपुष्पं ततस्तस्मात्सौवर्णं तु ततोऽधिकम्}
{सौवर्णात्तु प्रसूनाच्च मत्प्रियं नास्ति पाण्डव}


\threelineshloka
{पुष्पाभावे तुलस्यास्तु पत्रैर्मामर्चयेत्पुनः}
{पत्रालाभे तु शाखाभिः शाकालाभे शिफालवैः}
{शिफाभावे मृदा तत्र भक्तिमानर्चयेत माम्}


\twolineshloka
{वर्जनीयानि पुष्पाणि शृणु राजन्समाहितः}
{किङ्किणी मुनिपुष्पं च धुर्धूरं पाटलं तथा}


\twolineshloka
{तथाऽतिमुक्तकं चैव पुन्नागं नक्तमालिकम्}
{यौधिकं क्षीरिकापुष्पं निर्गुण्डी लाङ्गुली जपा}


\twolineshloka
{कर्णिकारं तथाऽशोकं शल्मलीपुष्पमेव च}
{ककुभाः कोविदाराश्च वैभीतकमथापि च}


\threelineshloka
{कुरण्टकप्रसूनं च कल्पकं कालकं तथा}
{अङ्केलं गिरिकर्णी च नीलान्येव च सर्वशः}
{एकपर्णानि चान्यानि सर्वाण्येव विवर्जयेत्}


\twolineshloka
{अर्कपुष्पाणि वर्ज्यानि अर्कपत्रस्तितानि च}
{व्याघृताः पिचुमन्दानि सर्वाण्येव विवर्जयेत्}


\threelineshloka
{अन्यैस्तु शुक्लपत्रैस्तु गन्धवद्भिर्नराधिप}
{अवर्ज्यैस्तैर्यथालाभं मद्भक्तो मां समर्चयेत् ॥युधिष्ठिर उवाच}
{}


\threelineshloka
{कथं त्वमर्चनीयोसि मूर्तयः कीदृशास्तु ते}
{वैखानसाः कथं ब्रुयूः कथं वा पाञ्चरात्रिकाः ॥भगवानुवाच}
{}


\twolineshloka
{शृणु पाण्डव तत्सर्वमर्चनाक्रममात्मनः}
{स्थण्डिले पद्मकं कृत्वा चाष्टपत्रं सकर्णिकम्}


\twolineshloka
{अष्टाक्षरविधानेन ह्यथवा द्वादशाक्षरैः}
{वैदिकैरथ मन्त्रैश्च मम सूक्तेन वा पुनः}


\twolineshloka
{स्थापितं मां ततस्तस्मिन्नर्ययित्वा विचक्षणः}
{पुरुषं च ततः सत्यमत्युतं च युधिष्टिर}


\threelineshloka
{अनिरुद्धं च मां प्राहुर्वैखानसविदो जनाः}
{अन्ये त्वेवं विजानन्ति मां राजन्पाञ्चरात्रिकाः}
{}


\twolineshloka
{वासुदेवं च राजेन्द्र संकर्षणमथापि वा}
{प्रद्युम्नं चानिरुद्धं च चतुर्मूर्तिं प्रचक्षते}


\threelineshloka
{एताश्चान्याश्च राजेन्द्र संज्ञाभेदेन मूर्तयः}
{विद्ध्यार्थान्तरा एवं मामेवं चार्चयेद्भुधः ॥युधिष्ठिर उवाच}
{}


\threelineshloka
{त्वद्भक्ताः कीदृसा देव कानि तेषं व्रतानि च}
{एतत्कथय देवेश त्वद्भक्तस्य ममाच्युत ॥भगवानुवाच}
{}


\twolineshloka
{अनन्यदेवताभक्ता चे मद्भक्तजनप्रियाः}
{मामेव शरणं प्राप्ता मद्भक्तास्ते प्रकीर्तिताः}


\twolineshloka
{स्वर्ग्याण्यपि यशस्यानि मत्प्रियाणि विशेषतः}
{मद्भक्तः पाण्डवश्रेष्ठ व्रतानीमानि धारयेत्}


\twolineshloka
{नान्यदाच्छादयेद्वस्त्रं मद्भक्तो जलतारणे}
{स्वस्थस्तु न दिवा स्वप्येन्मधुमांसानि वर्जयेत्}


\twolineshloka
{प्रदक्षिणं व्रजेद्विप्रान्गामश्वत्थं हुताशनम्}
{न धावेत्पतिते वर्षे नाग्नभिक्षां च लोपयेत्}


\twolineshloka
{प्रत्यक्षलवणं नाद्यत्सौभाञ्जनकरञ्जनौ}
{ग्रासमुष्टिं गवपे दद्याद्धान्याम्लं चैव वर्जयेत्}


\twolineshloka
{तथा पर्युषितं चापि पक्वं परगृहागतम्}
{अनिवेदितं च यद्द्रव्यं तत्प्रयत्नेन वर्जयेत्}


\twolineshloka
{विभितककरञ्जानां छायां दूरे विवर्जयेत्}
{पिप्रदेवपरीवादान्न वदेत्पीडितोपि सन्}


\twolineshloka
{सात्विका राजसाश्चापि तामसाश्चापि पाण्डव}
{मामर्चयन्ति मद्भक्तास्तेषामीदृग्विदा गतिः}


\twolineshloka
{तामसास्तिमिरं यान्ति राजसा रज एव तत्}
{सात्विकाः सत्वसंपन्नाः सत्वमेव प्रयान्ति ते}


\twolineshloka
{ये सिद्धाः सन्ति साङ्ख्येन योगसत्वबलेन च}
{नभस्यादित्यचन्द्राभ्यां पश्यन्ति पदविस्तरम्}


\twolineshloka
{एकस्तंभे नवद्वारे त्रिस्थूणे पञ्चसाक्षिके}
{एतस्मिन्देहनगरे राजसस्तु सदा भवेत्}


\twolineshloka
{उदिते सवितर्याप्य क्रियायुक्तस्य धीमतः}
{चतुर्वेदविदश्चापि देहे षड्वृषलाः स्मृताः}


\twolineshloka
{क्षत्रियाः सप्त विज्ञेया वैश्यास्त्वष्टौ प्रकीर्तिताः}
{नियताः पाण्डवश्रेष्ठ शूद्राणामेकविंशतिः}


\twolineshloka
{कामः क्रोधश्च लोभश्च मोहश्च मद एव च}
{महामोहश्च इत्येते देहे षड्वृषलाः स्मृताः}


\twolineshloka
{गर्वः स्तंभो ह्यहङ्कार ईर्ष्या च द्रोह एव च}
{पारुष्यं क्रूरता चैव सप्तैत क्षत्रियाः स्मृताः}


\twolineshloka
{तीक्ष्णता निकृतिर्माया शाठ्यं डंभो ह्यनार्जवम्}
{पैशुन्यमनृतं चैव वेश्यास्त्वष्टौ प्रकीर्तिताः}


\twolineshloka
{तृष्णा बुभुक्षा निद्रा च ह्यालस्यं चाघृणादयः}
{आधिश्चापि विषादश्च प्रमादो हीनसत्वता}


\twolineshloka
{भयं विक्लबता जाड्यं पापकं मन्युरेव च}
{आशा चाश्रद्दधानत्वमनवस्थाप्यमन्त्रणम्}


\twolineshloka
{आशौचं मलिनत्वं च शूद्रा ह्येते प्रकीर्तिताः}
{यस्मिन्नेते न दृश्यन्ते स वै ब्राह्मण उच्यते}


\twolineshloka
{येषुयेषु हि भावेषु यत्कालं वर्तते द्विजः}
{तत्कालं वै स विज्ञेयो ब्राह्मणो ज्ञानदुर्बलः}


\twolineshloka
{प्राणानायम्य यत्कालं येन मां चापि चिन्तयेत्}
{तत्कालो वै द्विजो ज्ञेयः शेषकालो ह्यथेतरः}


\twolineshloka
{तस्मात्तु सात्विको भूत्वा शुचिः क्रोधविवर्जितः}
{मामर्चयेत्तु सततं मत्प्रियत्वं यदीच्छति}


\twolineshloka
{अलोलजिह्वः समुपस्थितो धृतिंनिधाय चक्षुर्युगमात्रमेव तत}
{मनश्च वाचं च निगृह्य चञ्चलंभयान्निवृत्तो मम भक्त उच्यते}


\twolineshloka
{ईदृशाध्यात्मिनो ये तु ब्राह्मणा नियतेन्द्रियाः}
{तेषां श्राद्धेषु तृप्यन्ति तेन तृप्ताः पितामहाः}


\twolineshloka
{धर्मो जयति नाधर्मः सत्यं जयति नानृतम्}
{क्षमा जयति न क्रोधः क्षमावान्ब्राह्मणो भवेत्}


\chapter{अध्यायः १०५}
\twolineshloka
{दानपुण्यफलं श्रुत्वा तपःपुण्यफलानि च}
{धर्मपुत्रः प्रहृष्टात्मा केशवं पुनरब्रवीत्}


\twolineshloka
{या चैषा कपिला देव पूर्वमुत्पादिता विभो}
{होमधेनुः सदा पुण्या चतुर्वक्त्रेण माधव}


\twolineshloka
{सा कथं ब्राह्मणेभ्यो हि देया कस्मिन्दिनेऽपि वा}
{कीदृशाय च विप्राय दातव्या पुण्यलक्षणा}


\threelineshloka
{कथिता कपिला प्रोक्ता स्वयमेव स्वयंभुवा}
{कैर्वा देयाश्च ता देव श्रोतुमिच्छामि तत्वतः ॥वैशम्पायन उवाच}
{}


\twolineshloka
{एवमुक्तो हृषीकेशो धर्मपुत्रेण संसदि}
{अब्रवीत्कपिलासङ्क्यां तासां माहात्म्यमेव च}


\twolineshloka
{शृणु पाण्डव तत्वेन पवित्रं पावनं परम्}
{यच्छ्रुत्वा पापकर्मा पि नरः पापात्प्रमुच्यते}


\twolineshloka
{अग्निमद्योद्भवां दिव्यामग्निज्वालासमप्रभाम्}
{अग्निज्वालोज्ज्वलच्छृङ्गीं प्रदीप्ताङ्गारलोचनाम्}


\twolineshloka
{अग्निपुच्छामग्निखुरामग्निरोमप्रभान्विताम्}
{तामग्नेयीमग्निजिह्वामग्निग्रीवां ज्वलत्प्रभाम्}


\twolineshloka
{भुञ्जते कपिलां ये तु शूद्रा लोभेन मोहिताः}
{पतितांस्तान्विजानीयाच्चण्डालसदृशा हि ते}


\twolineshloka
{न तेषां ब्राह्ममः कश्चिद्गृहे कुर्यात्प्रतिग्रहम्}
{दूराच्च परिहर्तव्या महापातकिनोपि ते}


\twolineshloka
{सार्वकालं हि ते सर्वैर्वर्जिताः पितृदैवतैः}
{ते सदा ह्यप्रतिग्राह्या ह्यसंभाष्याश्च पापिनः}


\twolineshloka
{पिबन्ति कपिलां यावत्तावत्तेषां पितामहाः}
{अमेध्यमुपभुञ्जन्ति भूम्यां वै श्वसृगालवत्}


\twolineshloka
{कपिलाया दधि क्षीरं घृतं तक्रमथापि वा}
{ये शूद्रा उपभुञ्जन्ति तेषां गतिमिमां शृणु}


\twolineshloka
{कपिलोपजीवी शूद्रस्तु मृतो गच्छति रौरवम्}
{क्लिश्यते रौरवे घोरे वर्षकोटिशतं वसन्}


\twolineshloka
{ततश्चापि च्युतः कालाच्छ्वानयोनिं स गच्छति}
{श्वयोन्याश्च परिभ्रष्टो विष्ठायां जायते क्रिमिः}


\twolineshloka
{विष्ठाकूपेषु पापिष्ठो दुर्गन्धेषु सहस्रशः}
{तत्रतत्रोपजायेत नोत्तारं तत्रथ विन्दति}


\twolineshloka
{ब्राह्मणश्चैव यस्तेषां गृहे कुर्यात्प्रतिग्रहम्}
{ततः प्रभृति तस्यापि पितरः स्युरमेध्यपाः}


\twolineshloka
{न तेन सार्धं सम्भाषेन्न चाप्येकासनं व्रजेत्}
{स नित्यं वर्जनीयो हि दूरात्तु ब्राह्मणाधमः}


\twolineshloka
{यस्तेन सह सम्भाषेदेकशय्यां व्रजेत् वा}
{प्राजापत्यं चरेत्कृच्छ्रं स च तेन विसुद्ध्यति}


\twolineshloka
{कपिलोपजीविनः शूद्राद्यः करोति प्रतिग्रहम्}
{प्रायश्चित्तं भवेत्तस्य विप्रस्यैतन्न संशयः}


\twolineshloka
{ब्रह्मकूर्चं प्रकुर्वीति चान्द्रायणमथापि वा}
{मुच्यते किल्बिषात्तस्मादेतेन ब्राह्मणो हि सः}


\twolineshloka
{कपिला ह्यग्निहोत्रार्थे विप्रार्थे वा स्वयंभुवा}
{सर्वं तेजः समुद्धृत्य निर्मिता ब्रह्मणा पुरा}


\twolineshloka
{पवित्रं च पवित्राणां मङ्गलानां च मङ्गलम्}
{पुण्यानां परमं पुण्यं कपिला पाण्डुनन्दन}


\twolineshloka
{तपसा तप एवाग्र्यं व्रातनामुत्तमं व्रतम्}
{दानानां परमं दानं निदानं ह्येतदक्षयम्}


\twolineshloka
{पृथिव्यां यानि तीर्थानि पुण्यान्यायतनानि च}
{पवित्राणि च रम्याणि सर्वलोकेषु पाण्डव}


% Check verse!
एभ्यस्तेजः समुद्धृत्य ब्रह्मणा लोककर्तृणा ॥लोकनिस्तरणायैव निर्मिताः कपिलाः स्वयम्
\twolineshloka
{सर्वतेजोमयी ह्येषां कपिला पाण्डुनन्दन}
{सदाऽमृतमयी मेध्या शुचिः पावनमुत्तमम्}


\twolineshloka
{क्षीरेण कपिलायास्तु दध्ना वा सघृतेन वा}
{होतव्यान्यग्निहोत्राणि सायं प्रातर्द्विजातिभिः}


\twolineshloka
{कपिलाया घृतेनापि दध्ना क्षीरेण वा पुनः}
{जुह्वते योऽग्निहोत्राणि ब्राह्मणा विधिवत्प्रभो}


\twolineshloka
{पूजयन्त्यतिथींश्चैव परां भक्तिमुपागताः}
{शूद्रान्नाद्विरता नित्यं डंभानृतविवर्जिताः}


\twolineshloka
{ते यान्त्यादित्यसङ्काशैर्विमानैर्द्विजसत्तमाः}
{सूर्यमण्डलमध्येन ब्रह्मलोकमनुत्तमम्}


\twolineshloka
{ब्रह्मणो भवने दिव्ये कामगाः कामरूपिणः}
{ब्रह्मणा पूज्यमानास्तु मोदन्ते कल्पमक्षयम्}


\twolineshloka
{एवं हि कपिला राजन्पुण्या मन्त्रामृतारणिः}
{आदावेवाग्निमध्ये तु मैत्रेयी ब्रह्मनिर्मिता}


\twolineshloka
{शृङ्गाग्रे कपिलायास्तु सर्वतीर्थानि पाण्डव}
{ब्रह्मणो हि नियोगेन निवसन्ति दिनेदिने}


\twolineshloka
{प्रातरुत्थाय यो मर्त्यः कपिलाशृङ्गमस्तकात्}
{यश्च्युतामंबुधारां वै शिरसा प्रयतः शुचिः}


\twolineshloka
{स तेन पुण्यतीर्थेन सहसा हतकिल्विषः}
{जन्मत्रयकृतं पापं प्रदहत्यग्निवतृणम्}


\twolineshloka
{मूत्रेण कपिलायास्तु यश्च प्राणानुपस्पृशेत् ॥स्नानेन तेन पुण्येन नष्टपापः स मानवः}
{त्रिंशद्वर्षकृतात्पापान्मुच्यते नात्र संशयः}


\twolineshloka
{प्रातरुत्थाय यो भक्तया प्रयच्छेत्तृणमुष्टिकम्}
{तस्य नश्यति तत्पापं त्रिंशद्रात्रकृतं नृप}


\twolineshloka
{प्रातरूत्थाय यद्भक्त्या कुर्याद्यस्मात्प्रदक्षिणम्}
{प्रदक्षिणीकृता तेन पृथिवी नात्र संशयः}


\twolineshloka
{प्रदक्षिणेन चैकेन श्रद्धायुक्तेन पाण्डव}
{दशरात्रकृतं पापं तस्य तन्नश्यति ध्रुवम्}


\twolineshloka
{कपिलापञ्चगव्येन यः स्नायात्तु शुचिर्नरः}
{स गङ्गाद्येषु तीर्थेषु स्नातो भवति पाण्डव}


\twolineshloka
{तेन स्नानेन तस्यापि श्रद्दायुक्तस्य पार्थिव}
{दशरात्रकृतं पापं तत्क्षणादेव नश्यति}


\twolineshloka
{दृष्ट्वा तु कपिलां भक्त्या श्रुत्वा हुङ्कारनिस्वनम्}
{व्यपोहति नरः पापमहोरात्रकृतं नृप}


\twolineshloka
{यत्र वा तत्र वा चाङ्गे कपिलां यः स्पृशेच्छुचिः}
{संवत्सरकृतं पापं विनाशयति पाण्डव}


\twolineshloka
{गोसहस्रं तु यो दद्यादेकां च कपिलां नरः}
{समं तस्य फलं प्राह ब्रह्मा लोकपितामहः}


\twolineshloka
{यस्त्वेवं कपिलां हन्यान्नरः कश्चित्प्रमादतः}
{गोसहस्रं हतं तेन भवेन्नात्र विचारणा}


\twolineshloka
{यश्चैकां कपिलां दद्याच्छ्रोत्रियायाहिताग्नये}
{गवां शतसहस्रं तु दत्तं भवति पाण्डव}


\twolineshloka
{दश वै कपिलाः प्रोक्ताः स्वयमेव स्वयंभुवा}
{यो दद्याच्छ्रोत्रियेभ्यो वै स्वर्गं गच्छति तच्छृणु}


\twolineshloka
{प्रथमा स्वर्णकपिला द्वितीया गौरपिङ्गला}
{तृतीया रक्तपिङ्गाक्षी चतुर्थीं गलपिङ्गला}


\twolineshloka
{पञ्चमी बभ्रुवर्णाभा षष्ठी च श्वेतपिङ्गला}
{सप्तमी रङ्गपिङ्गाक्षी त्वष्टमी खुरपिङ्गला}


\twolineshloka
{नवमी पाटला ज्ञेया दशमी पुच्छपिङ्गला}
{दशैताः कपिलाः प्रोक्तास्तारयन्ति नरान्सदा}


\twolineshloka
{मङ्गल्याश्च पवित्राश्च सर्वपापप्रणाशनाः}
{एवमेव ह्यनड्वाहो दश प्रोक्ता नरेश्वर}


\twolineshloka
{ब्राह्मणो वाहयेत्तांस्तु नान्यो वर्णः कथंचन}
{न वाहयेच्च कपिलां क्षेत्रे वाऽध्वनि वा द्विजः}


\twolineshloka
{वाहयेद्धुङ्कृतेनैव शाखया वा सपत्रया}
{न दण्डेन न वा यष्ट्या न पाशेन न वा पुनः}


\twolineshloka
{न क्षुत्तृष्णाश्रमश्रान्तान्वाहयेद्विकलेन्द्रियान्}
{अतृप्तेषु न भुञ्जीयात्पिबेत्पीतेषु चोदकम्}


\twolineshloka
{शुश्रूषोर्मातरश्चैताः पितरस्ते प्रकीर्तिताः}
{अह्नां पूर्वत्र भागे च धुर्याणां वाहनं स्मृतम्}


\threelineshloka
{विश्रामेन्मध्यमे भागे भागे चान्ते यथासुखम्}
{यत्र च त्वरया कृत्यं संशयो यत्र वाऽध्वनि}
{}


\twolineshloka
{वाहयेत्तत्र धुर्यांस्तु न स पापेन लिप्यते ॥भ्रूणहत्यासमं पापं तस्य स्यात्पाण्डुनन्दन}
{}


\twolineshloka
{अन्यथा वाहयन्राजन्निरयं याति रौरवम् ॥रुधिरं पातयेत्तेषां यस्तु मोहान्नराधिप}
{}


\twolineshloka
{तेन पापेन पापात्मा नरकं यात्यसंशयम् ॥नरकेषु च सर्वेषु समाः स्थित्वा शतंशतम्}
{}


% Check verse!
इह मानुष्यके लोके बलीवर्दो भविष्यति ॥तस्मात्तु मुक्तिमन्विच्छन्दद्यात्तु कपिलां नरः
\twolineshloka
{कपिलां वाहयेद्यस्तु वृषलो लोभमोहितः}
{तेन देवास्त्रयस्त्रिंशत्पितरश्चापि वाहिताः}


\twolineshloka
{स देवैः पितृभिर्नित्यं वध्यमानस्तु दुर्मतिः}
{नरकान्नरकं घोरं गच्छेदाप्रलयं नृप}


\threelineshloka
{ब्रह्मा रुद्रस्तथाऽग्निश्च कपिलानां गतिं गताः}
{तस्मात्ते न निहन्तव्याः पूज्याश्चैव न संशयः}
{निःश्वसन्ति यदा श्रान्तास्तदा हन्युस्च तत्कुलं}


\twolineshloka
{यावन्ति तेषां रोमाणि तावद्वर्षशतं नृप}
{नरकेषूपपच्यन्ते तत्र तद्वाहका नराः}


\twolineshloka
{कपिला सर्वयज्ञेषु दक्षिणार्थं विधीयते}
{तस्मात्तद्दक्षिणा देया यज्ञेष्वेव द्विजातिभिः}


\threelineshloka
{होमार्तं चाग्निहोत्रस्य यां प्रयच्छेत्प्रयत्नतः}
{श्रोत्रियाय दरिद्राय श्रान्तायामिततेजसे}
{तेन दानेन पूतात्मा मम लोके महीयते}


\twolineshloka
{यावन्ति चैव रोमाणि कपिलाङ्गे युधिष्ठिर}
{तावद्वर्षसहस्राणि स्वर्गलोके महीयते}


\threelineshloka
{सुवर्णखुरशृङ्गीं च कपिलां यः प्रयच्छति}
{विषुवे चायने चापि सोऽश्वमेधफलं लभेत्}
{तेनाश्वमेधतुल्येन मम लोकं स गच्छति}


\twolineshloka
{स्वर्णशृङ्गीं रूप्यखुरां सवत्सां कांस्यदोहिनीम्}
{वस्त्रैरलङ्कृतां पुष्टां गन्धैर्माल्यैश्च शोभिताम्}


\twolineshloka
{पवित्रं हि पवित्राणां सुरव्णमिति मे मतिः}
{तस्मात्सुवर्णाभरणा दातव्या साऽग्निहोत्रिणे}


\twolineshloka
{एवं दत्त्वा तु राजेन्द्र सप्तपूर्वान्परानपि}
{तारयिष्यति राजेन्द्र नात्र कार्या विचारणा}


\threelineshloka
{अग्निष्टोमसहस्रस्य वाजपेयं च तत्समम्}
{वाजपेयसहस्रस्य अश्वमेधं च तत्समम्}
{अश्वमेधसहस्रस्य राजसूयं च तत्समम्}


\threelineshloka
{कपिलानां सहस्रेण विदिदत्तेन पाण्डव}
{राजसूयफलं प्राप्य मम लोके महीयते}
{न तस्य पुनरावृत्तिर्विद्यते कुरुपुङ्गव}


\twolineshloka
{प्रयच्छते यः कपिलां सवत्सां कांस्यदोहिनीम्}
{सुवर्णकुरशृङ्गाङ्गीं सर्वालङ्कारशोभिताम्}


\threelineshloka
{तैस्तैर्गुणैः कामदुघा च भूत्वानरं प्रदातारमुपैति सा गौः}
{स्वकर्मभिश्चाप्यनुबध्यमानंतीव्रान्धकारे नरके पतन्तम्}
{महार्णवे नौरिव वायुनीतादत्ता हि गौस्तारयते मनुष्यम्}


\twolineshloka
{पुत्रांश्च पौत्रांश्च कुलं च सर्व-मासप्तमं तारयते यथावत्}
{यावन्मनुष्यान्पृथिवी बिभर्तितावत्प्रदातारमृतं परत्र}


\twolineshloka
{यथौषधं मन्त्रकृतं नरस्यप्रयुक्तमात्रं विनिहन्ति रोगान्}
{तथैव दत्ता कपिला सुपात्रेपापं नरस्याशु निहन्ति सर्वम्}


\twolineshloka
{यथैव दृष्ट्वा भुजगाः सुपर्णंनश्यन्ति दूराद्विवशा भयार्ताः}
{तथैव दृष्ट्वा कपिलाप्रदाना-न्नस्यन्ति पापानि नरस्य शीघ्रम्}


\twolineshloka
{यथा त्वचं वै भुजगो विहायपुनर्नवं रूपमुपैति पुण्यम्}
{तथैव मुक्तः पुरुषः स्वपापै-र्विरज्यते वै कपिलाप्रदानात्}


\twolineshloka
{यथाऽन्धकारं भवने विलग्नंदीप्तो हि निर्यातयति प्रदीपः}
{तथा नरः पापमपि प्रलीनंनिष्क्रामयेद्वै कपिलाप्रदानात्}


\twolineshloka
{यावन्ति रोमाणि भवन्ति तस्यावत्सान्वितायाश्च शरीरजानि}
{तावत्प्रदाता युगवर्षकोटिंस ब्रह्मलोके रमते मनुष्यः}


\twolineshloka
{यस्याहिताग्नेरतिथिप्रियस्यशूद्रान्नदूरस्य जितेन्द्रियस्य}
{सत्यव्रतस्याध्ययनान्वितस्यदत्ता हि गौस्तारयते परत्र}


\chapter{अध्यायः १०६}
\twolineshloka
{एवं श्रुत्वा परं पुण्यं कपिलादानमुत्तमम्}
{धर्मपुत्रः प्रहृष्टात्मा केशवं पुनरब्रवीत्}


\twolineshloka
{देवदेवश कपिला यदा विप्राय दीयते}
{कथं सर्वेषु चाङ्गेषु तस्यास्तिष्ठन्ति देवताः}


\twolineshloka
{याश्चैताः कपिलाः प्रोक्ता दश चैव त्वया मम}
{तासां कति सुरश्रेष्ठ् कपिलाः पुण्यलक्षणाः}


\threelineshloka
{कथं वाऽनुगृहीतास्ताः सुरैः पितृगणैरपि}
{केन युक्ताश्च वर्णेन श्रोतुं कौतूहलं मे ॥वैशम्पायन उवाच}
{}


\twolineshloka
{युधिष्ठिरेणैवमुक्तः केशवः सत्यवाक्तदा}
{गुह्यानां परमं गुह्यं वक्तुमेवोपचक्रमे}


% Check verse!
शृणु राजन्पवित्रं वै रहस्यं धर्ममुत्तमम् ॥ग्रहणीयं सत्यमिदं न श्राव्यं हेतुवादिभिः
\twolineshloka
{यदा वत्सस्य पादौ द्वौ प्रसवे शिरसा सह}
{दृश्येते दानकालं तत्तमाहुर्मुनिसत्तमाः}


\twolineshloka
{अन्तरिक्षगतो वत्सो यावद्भूमिं न यास्यति}
{गौस्तावत्पृथिवी ज्ञेया तस्माद्देया तु तादृशी}


\threelineshloka
{यावन्ति धेन्वा रोमाणि सवत्साया युधिष्ठिर}
{यावत्यः सिकताश्चापि गर्भोदकपरिप्लुताः}
{तावद्वर्षसहस्राणि दाता स्वर्गे महीयते}


\twolineshloka
{सुवर्णाभरणां कृत्वा सव्तसां कपिलां तिलैः}
{प्रच्छाद्य तां तु दद्याद्वै सर्वरत्नैरलङ्कृताम्}


\twolineshloka
{ससमुद्रा नदी तेन सशैलवनकानना}
{चतुरन्ता भवेद्दत्ता नात्र कार्या विचारणा}


\twolineshloka
{पृथिवीदानतुल्येन तेन दानेन मानवः}
{संसारसागरात्तीर्णो याति लोकं प्रजापतेः}


\twolineshloka
{ब्रह्महा भ्रूणहा गोघ्नो यदि वा गुरुतल्पगः}
{महापातकयुक्तोपि एतद्दानेन शुद्ध्यति}


\twolineshloka
{इदं पठति यः पुण्यं कपिलादानमुत्तमम्}
{प्रातरुत्थाय मद्भक्त्या तस्य पुण्यफलं शृणु}


\twolineshloka
{मनसा कर्मणा वाचा मतिपूर्वं युधिष्ठिर}
{पापं रात्रिकृतं हन्यादस्याध्यायस्य पाठकः}


\twolineshloka
{इदमावर्तमानस्तु श्राद्धे यस्तर्पयेद्द्विजान्}
{तस्याप्यमृतमश्नन्ति पितरोऽत्यन्तहर्षिताः}


\twolineshloka
{यश्चेदं शृणुयाद्भक्त्या मद्गतेनान्तरात्मना}
{तस्य रात्रिकृतं सर्वुं पापमाशु प्रणश्यति}


\threelineshloka
{अतः परं विशेषं तु कपिलानां ब्रवीमि ते}
{याश्चैताः कपिलाः प्रोक्ता दश राजन्मया तव}
{तासां चतस्रः प्रवराः पुण्याः पापप्रणाशनाः}


\twolineshloka
{सुवर्णकपिला पुण्या तथा रक्ताक्षिपिङ्गला}
{पिङ्गलाक्षी च या गौश्च या स्यात्पिङ्गलपिङ्गला}


\twolineshloka
{एताश्चतस्रः प्रवराः पवित्राः पापनाशनाः}
{नमस्कृता वा दृष्टा वा घ्नन्ति पापं नरस्य तु}


\twolineshloka
{यस्यैताः कपिलाः सन्ति गृहे पापप्रणाशनाः}
{तत्र श्रीर्विजयः कीर्तिः स्फीता नित्यं युधिष्ठिर}


\twolineshloka
{एतासां प्रीतिमायाति क्षीरेण तु वृषध्वजः}
{दध्ना च त्रिदशाः सर्वे घृतेन तु हुताशनः}


\twolineshloka
{पितरः पितामहाश्चैव तथैव प्रपितामहाः}
{सकृद्देत्तेन तुष्यन्ति वर्षकोटिं युधिष्ठिर}


\twolineshloka
{कपिलाया घृतं क्षीरं दधि पायसमेव वा}
{श्रोत्रियेभ्यः सकृद्दत्वा नरः पापैः प्रमुच्यते}


\twolineshloka
{उपवासं तु यः कृत्वाऽप्यहोरात्रं जितेन्द्रियः}
{कपिलापञ्चगव्यं तु पीत्वा चान्द्रायणात्परम्}


\twolineshloka
{सौम्ये मुहूर्ते तत्प्रशय शुद्धात्मा शुद्धमानसः}
{क्रोधानृतविनिर्मुक्तो मद्गतेनान्तरात्मना}


\threelineshloka
{कपिलापञ्चगव्येन सुमन्त्रेण पृथक्पथक्}
{यो मत्प्रतिकृतिं वाऽपि शङ्कराकृतिमेव वा}
{स्नापयेद्विषुवे यस्तु सोऽश्वमेधफलं लभेत्}


\twolineshloka
{स मुक्तपापः शुद्धात्मा यानेनांबरशोभिना}
{मम लोकं व्रजेन्मुक्तो रुद्रलोकमथापि वा}


\twolineshloka
{ब्रह्मणा तु पुरा सृष्टा कपिला काञ्चनप्रभा}
{अग्निकुण्डात्परैर्मन्त्रैर्होमधेनुर्महाप्रभा}


\twolineshloka
{सृष्टमात्रां तु तां दृष्ट्वा देवा रुद्रादयो दिवि}
{सिद्धा ब्रह्मर्षयश्चैव वेदाः साङ्गाः सहाध्वरैः}


\twolineshloka
{सागराः सरितश्चैव पर्वताः सबलाहकाः}
{गन्धर्वाप्सरसो यक्षाः पन्नगाश्चाप्युपस्थिताः}


\twolineshloka
{सर्वे विस्मयमापन्नाः शिखिमध्ये महाप्रभाम्}
{मन्त्रैशअच विविधैस्तां तु तुष्टुवुस्तामनेकशः}


\twolineshloka
{कृताञ्जलिपुटाः सर्वे नातिशृङ्गीं त्रिलोचनाम्}
{मूर्ध्ना प्रणम्य तां भूमौ सवत्साममृतारणिम्}


\twolineshloka
{ऊचुः प्राञ्जलयः सर्वे चतुर्वक्त्रं पितामहम्}
{आज्ञापय महादेव किं ते कुर्मः कथं विभो}


\twolineshloka
{एवमुक्तः सुरैः सर्वैर्ब्रह्मा वचनमब्रवीत्}
{भवन्तोप्यनुगृह्णन्तु दोग्ध्रीमेनां पयस्विनीम्}


\twolineshloka
{होमधेनुरियं ज्ञेया ह्यग्नीन्संतर्पयिष्यति}
{अतोऽग्निस्तर्पितः सर्वान्भवतस्तर्पयिष्यति}


\twolineshloka
{प्रीताः क्षीरामृतेनास्या जातवीर्यपराक्रमाः}
{जयिष्यथ यथाकामं दानवान्सर्व एव तु}


\twolineshloka
{जातवीर्यबलैर्युक्ताः सत्ववन्तो जितेन्द्रियाः}
{असङ्ख्येयबलाः सर्वे पालयिष्यथ वै प्रजाः}


\twolineshloka
{पालिताश्च प्रजाः सर्वा भवद्भिरिह धर्मतः}
{पूजयिष्यन्ति वा नित्यं यज्ञैर्विविधदक्षिणैः}


\threelineshloka
{एवमुक्ताः सुराः सर्वे ब्रह्मणा परमेष्ठिना}
{ततः संहृष्टवदनाः कपिलायै वरं ददुः ॥देवता ऊचुः}
{}


\twolineshloka
{यस्माल्लोकहितायाद्य ब्रह्मणा त्वं विनिर्मिता}
{तस्मात्पूता पवित्रा च भव पापव्यपोहिनी}


\twolineshloka
{ये त्वां दृष्ट्वा नमस्यन्ति स्पृष्ट्वा चापि करैर्नराः}
{तेषां वर्षकृतं पापं त्वद्भक्तानां प्रणश्यति}


\fourlineindentedshloka
{अकामकृतमज्ञानमदृष्टं यच्च पातकम्}
{त्वां दृष्ट्वा ये नमस्यन्ति नराः सर्वसहेति च}
{तेषां तद्विलयं याति तमः सूर्योदये यथा ॥भगवानुवाच}
{}


\twolineshloka
{इत्युक्त्वाऽस्यै वरं दत्त्वा प्रययुस्ते यथागतम्}
{लोकनिस्तरणार्थाय सा च लोकांश्चचार ह}


\threelineshloka
{तस्यामेव समुद्भूता ह्येताश्च कपिला नव}
{विचरन्ति महीमेनां लोकानुग्रहकारणात्}
{तस्मात्तु कपिला देया परत्र हितमिच्छता}


\twolineshloka
{यदा च दीयते राजन्कपिला ह्यग्निहोत्रिणे}
{तदा च शृङ्गयोस्तस्या विष्णुरिन्द्रश्च तिष्ठति}


\twolineshloka
{चन्द्रवज्रधरौ चापि तिष्ठतः शृङ्गमूलयोः}
{शृङ्गमध्ये तथा ब्रह्मा ललाटे गोवृषध्वजः}


\twolineshloka
{कर्णयोरश्विनौ देवौ चक्षुषी शशिभास्करौ}
{दन्तेषु मरुतो देवा जिह्वायां वाक्सरलस्वती}


\twolineshloka
{रोमकूपेषु मुनयश्चर्मण्येव प्रजापतिः}
{निश्श्वासेषु स्थिता वेदाः सषडङ्गपदक्रमाः}


\twolineshloka
{नासापुटे स्थिता गन्धाः पुष्पाणि सुरभीणि च}
{अधरे वसवः सर्वे मुखे चाग्निः प्रतिष्ठितः}


\twolineshloka
{साध्या देवाः स्थिताः कक्षे ग्रीवायां पार्वती स्थिता}
{पृष्ठे च नक्षत्रगणाः ककुद्देशे नभस्स्थलम्}


\twolineshloka
{अपाने सर्वतीर्थानि गोमूत्रे जाह्नवी स्वयम्}
{इष्टतुष्टमया लक्ष्मीर्गोमये वसती तदा}


\twolineshloka
{नासिकायां सदा देवी ज्येष्ठा वसति भामिनी}
{श्रोणीतटस्थाः पितरो रमा लाङ्गूलमाश्रिता}


\twolineshloka
{पार्श्वयोरुभयोः सर्वे विश्वेदेवाः प्रतिष्ठिताः}
{तिष्ठत्युरसि तासां तु प्रीतः शक्तिधरो गुहः}


\twolineshloka
{जानजङ्घोरुदेशेषु पञ्च तिष्ठन्ति वायवः}
{खुरमध्येषु गन्धर्वाः खुराग्रेषु च पन्नगाः}


\threelineshloka
{चत्वारः सागराः पूर्णास्तस्या एव पयोधराः}
{रतिर्मेधा क्षमा स्वाहा श्रद्धा शान्तिर्धृतिः स्मृतिः}
{}


\twolineshloka
{कीर्तिर्दीप्तिः क्रिया कान्तिस्तुष्टिः पुष्टिस्च सन्ततिः}
{दिशश्च प्रदिशश्चैव सेवन्ते कपिलां सदा}


\twolineshloka
{देवा पितृगणाश्चापि गन्धर्वाप्सरसां गणाः}
{लोका द्वीपार्णवाश्चैव गङ्गाद्याः सरितस्तथा}


\twolineshloka
{देवाः पितृगणाश्चापि वेदाः साङ्गाः सहाध्वरैः}
{वेदोक्तैर्विविधैर्मन्त्रैः स्तुवन्ति हृषितास्तथा}


\twolineshloka
{विद्याधराश्च ये सिद्धा भूतास्तारागणास्तथा}
{पुष्पवृष्टिं च वर्षन्ति प्रनृत्यन्ति च हर्षिताः}


\twolineshloka
{ब्रह्मणोत्पादिता देवी वह्निकुण्डान्महाप्रभा}
{नमस्ते कपिले पुण्ये सर्वदेवैर्नमस्कृते}


\twolineshloka
{कपिलेऽथ महासत्वे सर्वतीर्थमये शुभे}
{दातारं स्वजनोपेतं ब्रह्मलोकं नय स्वयम्}


\threelineshloka
{अहो रत्नमिदं पुण्यं सर्वदुःखघ्नमुत्तमम्}
{अहो धर्मार्जितं शुद्धमिदमग्र्यं महाधनम्}
{इत्याकाशस्थितस्ते तु सर्वदेवा जपन्ति च}


\threelineshloka
{तस्याः प्रतिग्रहीता च भुङ्क्ते यावद्द्विजोत्तमः}
{तावद्देवगणाः सर्वे कपिलामर्चयन्ति च}
{स्वर्णशृङ्गीं रूप्यखुरां गन्धैः पुष्पैः सुपूजिताम्}


\threelineshloka
{वस्त्राभ्यामहताभ्यां तु यावत्तिष्ठत्यलङ्कृता}
{तावद्यदिच्छेत्कपिला मन्त्रपूता सुसंस्कृता}
{भूलोकवासिनः सर्वान्ब्रह्मलोकं नयेत्स्वयम्}


\twolineshloka
{भूरश्वः कनकं गावो रूप्यमश्वं तिला यवाः}
{दीयमानानि विप्राय प्रहृष्यन्ति दिनेदिने}


\twolineshloka
{अथ त्वश्रोत्रियेभ्यो वै तानि दत्तानि पाण्डव}
{तथा निन्दन्त्यथात्मानमशुभं किंनु नः कृतं}


\twolineshloka
{अहो रक्षःपिशाचैश्च लुप्यमानाः समन्ततः}
{यास्यामो निरयं शीघ्रमिति शोचन्ति तानि वै}


\threelineshloka
{एतान्यपि द्विजेभ्यो वै श्रोत्रियेभ्यो विशेषतः}
{दीयमानानि वर्धन्ते दातारं तारयन्ति च ॥युधिष्ठिर उवाच}
{}


\threelineshloka
{देवदेवेश दैत्यघ्न कालः को हव्यकव्ययोः}
{के तत्रि पूजामर्हन्ति वर्जनीयाश्च के द्विजाः ॥भगवानुवाच}
{}


\twolineshloka
{दैवं पूर्वाह्णिकं ज्ञेयं पैतृकं चापराह्णिकम्}
{कालहीनं च यद्दानं तद्दानं राजसं विदुः}


\twolineshloka
{अवघुष्टं च यद्भुक्तमनृतेन च भारत}
{परामृष्टं शुना वाऽपि तद्भागं राक्षसं विदुः}


\twolineshloka
{यावन्तः पतिता विप्रा जडोन्मत्तादयोपि च}
{दैवे च पित्र्ये ते विप्रा राजन्नार्हन्ति सत्क्रियां}


\twolineshloka
{क्लीबः प्लङ्गी च कृष्ठी च राजयक्ष्मान्वितश्च यः}
{अपस्मारी च यश्चापि पित्र्ये नार्हति सत्कृतिम्}


\twolineshloka
{चिकित्सका देवलका मिथ्यानियमधारिणः}
{सोमविक्रयिणश्चापि श्राद्धे नार्हन्ति सत्कृतिम्}


\twolineshloka
{एकोद्दिष्टे च ये चान्नं भुञ्जते विधिवद्द्बिजाः}
{चान्द्रायणमकृत्वा ते पुनर्नार्हन्ति सत्कृतिम्}


\twolineshloka
{गायका नर्तकाश्चैव प्लवका वादकास्तथा}
{कथका यौधिकाश्चैव श्राद्धे नार्हन्ति सत्कृतिम्}


\twolineshloka
{अनग्नयश्च ये विप्राः शवनिर्यातकाश्च ये}
{स्तेनाश्चापि विकर्मस्था राजन्नार्हन्ति सत्कृतिम्}


\twolineshloka
{अपरिज्ञातपूर्वाश्च गणपुत्राश्च ये द्विजाः}
{पुत्रिकापुत्रकाश्चापि श्राद्धे नार्हन्ति सत्कृतिम्}


\twolineshloka
{रणकर्ता च यो विप्रो यश्च वाणिज्यको द्विजः}
{प्राणिविक्रयवृत्तिश्च श्राद्धे नार्हन्ति सत्कृतिम्}


\twolineshloka
{चीर्णव्रतगुणैर्युक्ता नित्यं स्वाध्यायतत्पराः}
{सावित्रीज्ञाः क्रियावन्तस्ते श्राद्धे सत्कृतिक्षमाः}


\twolineshloka
{श्राद्धस्य ब्राह्मणः कालः प्राप्तं दघि घृतं तथा}
{दर्भाः सुमनसः क्षेत्रं तत्काले श्राद्धदो भवेत्}


\twolineshloka
{चारित्रनिरता राजन्कृशा ये कृशवृत्तयः}
{अर्थिनश्चोपगच्छन्ति तेभ्यो दत्तं महत्फलम्}


\twolineshloka
{तपस्विनश्च ये विप्रास्तथा भैक्षचराश्च ये}
{अर्थिनः केचिदिच्छन्ति तेषां दत्तं महत्फलम्}


\twolineshloka
{एवं धर्मभूतां श्रेष्ठ ज्ञात्वा सर्वात्मना तदा}
{र्श्रोत्रियाय दरिद्राय प्रयच्छानुपकारिणे}


\twolineshloka
{दानं यत्ते प्रियं किंचिच्छ्रोत्रियाणां च यत्प्रियम्}
{तत्प्रयच्छस्व धर्मज्ञ यदीच्छसि तदक्षयम्}


% Check verse!
निरयं ये च गच्छन्ति तच्छृणुष्व युधिष्ठिर
\twolineshloka
{गुर्वर्थं वा भयार्थं वा नोचेदन्यत्र पाण्डव}
{वदन्ति येऽनृतं विप्रास्ते वै निरयगामिनः}


\twolineshloka
{परदारापहर्तारः परदारामिमर्शकाः}
{परारप्रयोक्तारस्तें वै निरयगामिनः}


\twolineshloka
{सूचकाः संधिभेत्तारः परद्रव्योपजीविनः}
{अकृतज्ञाश्च मित्राणां ते वै निरयगामिनः}


\twolineshloka
{वर्णाश्रमाणां ये बाह्याः पाषण्डाश्चैव पापिनः}
{उपासते च तानेव ते सर्वे नरकालयाः}


\twolineshloka
{वेदविक्रयिणश्चैव वेदानां चैव दूषकाः}
{वेदानां लेखकाश्चैव ते वै निरयगामिनः}


\twolineshloka
{रसविक्रयिणो राजन्विषविक्रयिणश्च ये}
{क्षीरविक्रयिणश्चापि ते वै निरयगामिनः}


\twolineshloka
{चण्डालेभ्यश्च ये क्षीरं प्रयच्छन्ति नराधमाः}
{अर्थार्थमथवा स्नेहात्ते वै निरयगामिनः}


% Check verse!
पशूनां दमक**यैव तथा नासानुवेधकाःपुंस्त्वहिंसाकरस्चैव ते वै निरयगामिनः
\twolineshloka
{अदातारः समर्था ये द्रव्याणां लोभकारणात्}
{दीनानन्धान्न पश्यन्ति ते वै निरयगामिनः}


\twolineshloka
{क्षान्तान्दान्तान्कृशान्प्राज्ञान्दीर्घकालं सहोषितान्}
{त्यजन्ति कृतकृत्या ये ते वा निरयगामिनः}


\twolineshloka
{बालानामपि वृद्धानां श्रान्तानां चापि ये नराः}
{अदत्त्वाऽश्नान्ति मृष्टान्नं ते वै निरयगामिनः}


\twolineshloka
{एते पूर्वर्षिभिः प्रोक्ता नरा निरयगामिनः}
{ये स्वर्गं समनुप्राप्तास्ताञ्शृणुष्व युधिष्ठिर}


\twolineshloka
{दानेन तपसा चैव सत्येन च दमेन च}
{ये धर्ममनुवर्तन्ते ते नराः स्वर्गगामिनः}


\twolineshloka
{शुश्रूषयाऽप्युपाध्यायाच्छ्रुतमादाय पाण्डव}
{ये प्रतिग्रहनिस्स्नेहास्ते नराः स्वर्गगामिनः}


\twolineshloka
{मधुमांसासवेभ्यस्तु निवृत्ता व्रतवत्तु ये}
{परदारनिवृत्ता ये ते नराः स्वर्गगामिनः}


\twolineshloka
{मातरं पितरं चैव शुश्रूषन्ति च ये नराः}
{भ्रहातॄणामपि सस्नेहास्ते नराः स्वर्गगामिनः}


\twolineshloka
{ये तु भोजनकाले तु निर्याताश्चातिथिप्रियाः}
{द्वाररोधं न कुर्वन्ति ते नराः स्वर्गगामिनः}


\twolineshloka
{वैवाहिकं तु कन्यानां दरिद्राणां च ये नराः}
{कारयन्ति च कुर्वन्ति ते नराः स्वर्गगामिनः}


\twolineshloka
{रसानामथ बीजानामोषधीनां तथैव च}
{दातारः श्रद्धयोपेतास्ते नराः स्वर्गगामिनः}


\twolineshloka
{क्षेमाक्षेमं च मार्गेषु समानि विषमाणि च}
{अर्थिनां ये च वक्ष्यन्ति ते नराः स्वर्गगामिनः}


\threelineshloka
{पर्वद्वये चतुर्दश्यामष्टम्यां संध्ययोर्द्वयोः}
{आर्द्रायां जन्मनक्षत्रे विषुवे श्रवणेऽथवा}
{ये ग्राम्यधर्मविरतास्ते नराः स्वर्गगाप्तिनः}


\twolineshloka
{हव्यकव्यविधानं च नरकस्वर्गगामिनौ}
{धर्माधर्मौ च कथितौ किं भूयः श्रोतुमिच्छसि}


\chapter{अध्यायः १०७}
\threelineshloka
{इदं मे तत्वतो देव वक्तुमर्हस्यशेषतः}
{हिंसामकृत्वा यो मर्त्यो ब्रह्महत्यामवाप्नुयात् ॥भगवानुवाच}
{}


\twolineshloka
{ब्राह्मणं स्वयमाहूय भिक्षार्थं वृत्तिकर्शितम्}
{ब्रूयान्नास्तीति यः पश्चात्तमाहुर्ब्रह्मघातकम्}


\twolineshloka
{मध्यस्थस्येह विप्रस्य योऽनूचानस्य भारत}
{वृत्तिं हरति दुर्बुद्धइस्तमाहुर्ब्रह्मघातकम्}


\twolineshloka
{आश्रमे वाऽऽलये वाऽपि ग्रामे वा नगरेऽपि वा}
{अग्निं यः प्रक्षिपेत्क्रुद्धस्तमाहुर्ब्रह्मघातकम्}


\twolineshloka
{गोकुलस्य तृषार्तस्य जलान्ते वसुधाधिप}
{उत्पादयति यो विघ्नं तमाहुर्ब्रह्मघातकम्}


\twolineshloka
{यः प्रवृत्तां श्रुतिं सम्यक्छास्त्रं वा मुनिभिः कृतम्}
{दूषयत्यनभिज्ञाय तमाहुर्ब्रह्मघातकम्}


\twolineshloka
{चक्षुषा वाऽपि हीनस्य पङ्गोर्वाऽपि जडस्य वा}
{हरेद्वै यस्तु सर्वस्वं तमाहुर्ब्रह्मघातकम्}


\twolineshloka
{गुरुं त्वंकृत्यं हुंकृत्य अतिक्रम्य च शासनम्}
{वर्तते यस्तु मूढात्मा तमाहुर्ब्रह्मघातकम्}


\twolineshloka
{क्रोधाद्वा यदि वा द्वेषादाक्रुष्टस्तर्जितोपि वा}
{ऋतौ स्त्रियं वा नोपेयात्तमाहुर्ब्रह्मघातकम्}


\threelineshloka
{यावत्सारो भवेद्दीनस्तन्नाशे यस्य दुःस्थितिः}
{तत्सर्वस्वं हरेद्यो वै तमाहुर्ब्रह्मघातकम् ॥युधिष्ठिर उवाच}
{}


\threelineshloka
{सप्वेषामपि दानानां यत्तु दानं विशिष्यते}
{अभोज्यान्नाश्च ये विप्रास्तान्ब्रवीहि सुरोत्तम ॥भगवानुवाच}
{}


\twolineshloka
{अन्नमेव प्रशंसन्ति देवा ब्रह्मपुरस्सराः}
{अन्नेन सदृशं दानं न भूतं न भविष्यति}


\twolineshloka
{अन्नमूर्जस्करं लोके ह्यन्नात्प्राणाः प्रतिष्ठिताः}
{अबोज्यान्नान्मया राजन्वक्ष्यमाणान्निबोध मे}


\twolineshloka
{दीक्षितस्य कदर्यस्य क्रुद्धस्य निकृतस्य च}
{अभिशप्तस्य षण्डस्य पाकभेदकरस्य च}


\twolineshloka
{चिकित्सकस्य दूतस्य तथा चोच्छिष्टभोजिनः}
{उग्रान्नं सूतकान्नं च शूद्रोच्छेषणमेव च}


\twolineshloka
{द्विषदन्नं न भोक्तव्यं पतितान्नं च यच्छ्रुतम्}
{तथा च पिशुनस्यान्नं यज्ञविक्रयिणस्तथा}


\twolineshloka
{शैलूषं तन्तुवायान्नं कृतघ्नस्यान्नमेव च}
{अंबष्ठकनिषादानां रङ्गावतरकस्य च}


\twolineshloka
{सुवर्णकर्तुर्वैणस्यि शस्त्रविक्रयिणस्तथा}
{सूताना शौण्डिकानां च वैद्यस्य रजकस्य च}


\twolineshloka
{स्त्रीजितस्य नृशंसस्य तथा माहिषिकस्य च}
{अनिर्दशानां प्रेतानां गणिकानां तथैव च}


\twolineshloka
{बन्दिनो द्यूतकर्तुश्च तथा द्यूतविदामपि}
{परिवित्तस्य यच्चान्नं परिवेत्तुस्तथैव च}


\twolineshloka
{यश्चाग्रदिधिषुर्विप्रो दिधिषूपपिस्तथा}
{तयोरप्युभयोरन्नं राज्ञां चापि विवर्जयेत्}


\twolineshloka
{राजान्नं तेज आदत्ते शूद्रान्नं ब्रह्मवर्चसम्}
{आयुः सुवर्णकारान्नं यशश्चार्मावकृन्तिनः}


\twolineshloka
{गणान्नं गणिकान्नं च लोकेभ्यः परिकीर्तितम्}
{पूयं चिकित्सकस्यान्नं शुक्लं तु वृषलीपतेः}


\twolineshloka
{विष्ठा वार्धुषिकस्यान्नं तस्मात्तत्परिवर्जयेत्}
{तेषां त्वगस्थिरोमाणि भुङ्क्ते योऽन्नं तु भक्षयेत्}


\twolineshloka
{अमात्यान्नमथैतेषां भुक्त्वा तु त्रियहं क्षिपेत्}
{मत्या भुक्त्वा सकृद्वाऽपि प्राजापत्यं चरेद्द्विजः}


\twolineshloka
{दानानां च फलं यद्वै शृणु पाण्डव तत्वतः}
{जलदस्तृप्तिमाप्नोति सुखमक्षयमन्नदः}


\twolineshloka
{तिलदश्च प्रजामिष्टां दीपदश्चक्षुरुत्तमम्}
{भूमदो भूमिमाप्नोति दीर्घमायुर्हिरण्यदः}


\twolineshloka
{गृहदोऽग्र्याणि वेश्मानि रूप्यदो रूपमुत्तमम्}
{वासोदश्चन्द्रसालोक्यमश्विसालोक्यमश्वदः}


\twolineshloka
{अनडुद्दः श्रियं जुष्टां गोदो गोलोकमश्नुते}
{यानशय्याप्रदो भार्यामैश्वर्यमभयप्रदः}


\twolineshloka
{धान्यदः शाश्वतं सौख्यं ब्रह्मदो ब्रह्मसाम्यताम्}
{सर्वेषामेव दानानां ब्रह्मदानं विशिष्यते}


\threelineshloka
{हिरण्यभूगवाश्वाजवस्त्रशय्यासनादिषु}
{योर्चितः प्रतिगृह्णाति दद्यादुचितमेव च}
{तावुभौ गच्छतः स्वर्गं नरकं च विपर्यये}


\twolineshloka
{अनृतं न वदेद्विद्वांस्तपस्तप्त्वा न विस्मयेत्}
{नार्तोप्यभिभवेद्विप्रान्न दत्त्वा परिकीर्तयेत्}


\twolineshloka
{यज्ञोऽनृतेन क्षरति तपः क्षरति विस्मयात्}
{आयुर्विप्रावमानेन दानं तु परिकीर्तनात्}


\twolineshloka
{एकः प्रजायते जन्तुरेक एव प्रमीयते}
{अकोऽनुभुङ्क्ते सुकृतमेकश्चाप्नोति दुष्कृतम्}


\twolineshloka
{मृतं शरीरमुत्सृज्य काष्ठलोष्टसमं क्षितौ}
{विमुखा बान्धवा यान्ति धर्मस्तमनुवर्तते}


\twolineshloka
{अनागतानि कार्याणि कर्तुं गणयते मनः}
{शरीरकं समुद्दिश्य स्मयते नूनमन्तकः}


\twolineshloka
{तस्माद्धर्मसहासस्तु धर्मं संचिनुयात्सदा}
{धर्मेण हि सहायेन तमस्तरति दुस्तरम्}


\twolineshloka
{येषां तटाकानि बहूदकानिसभाश्च कूपाश्च शुभाः प्रपाश्च}
{अन्नप्रदानं मधुरा च वाणीयमस्य ते निर्विषया भवन्ति}


\chapter{अध्यायः १०८}
\threelineshloka
{अनेकान्तं बहुद्वारं धर्ममाहुर्मनीपिणः}
{किंलक्षणोसौ भति तन्मे ब्रूहि जनार्दन ॥भगवानुवाच}
{}


\threelineshloka
{शृणु राजन्समासेन धर्मशौचविधिक्रमम्}
{अहिंसा शौचमक्रोधमानृशंस्यं दमः शमः}
{आर्जवं चैव राजेन्द्र निश्चितं धर्मलक्षणम्}


\twolineshloka
{ब्रह्मचर्यं तपः क्षान्तिर्मधुमांसस्य वर्जनम्}
{मर्यादायां स्थितिश्चैव शमः शौचस्य लक्षणम्}


\twolineshloka
{बाल्ये विद्यां निषेवेत यौवने दारसंग्रहम्}
{वार्धके मौनमातिष्ठेत्सर्वदा धर्ममाचरेत्}


\twolineshloka
{ब्राह्मणान्नावमन्येत गुरुन्परिवदेन्न च}
{यतीनामनुकूलः स्यादेष धर्मः सनातनः}


\twolineshloka
{यतिर्गुरुर्द्विजातीनां वर्णानां ब्राह्मणो गुरुः}
{पतिरेव गुरुः स्त्रीणां सर्वेषां पार्थिवो गुरुः}


\twolineshloka
{यद्गृहस्तार्जितं पापं ज्ञानतोऽज्ञानतोपि वा}
{निर्दहिष्यति तत्सर्वमेकरात्रोषितो यतिः}


\twolineshloka
{दुर्वृत्ता वा सुवृत्ता वा ज्ञानिनोऽज्ञानिनोपि वा}
{गृहस्थैर्यतयः पूज्याः परत्र हितकाङ्क्षिभिः}


\twolineshloka
{एकदण्डी त्रिदण्डी वा शिखी वा मुण्डितोपि वा}
{काषायदण्डधारोपि यतिः पूज्यो न संशयः}


\twolineshloka
{अपूजितो गृहस्थैर्वा तथा चाप्यवमानितः}
{यतिर्वाऽप्यतिथिर्वाऽपि नरके पातयिष्यतः}


\twolineshloka
{तस्मात्तु यत्नतः पूज्या मद्भक्ता मत्परायणाः}
{मयि संन्यस्तकर्माणः परत्र हितकाङ्क्षिभिः}


\twolineshloka
{प्रहरेन्न द्विजान्विप्रो गां न हन्यात्कदाचन}
{भ्रूणहत्यासमं चैव उभयं यो निषेवते}


\twolineshloka
{नाग्निं मुखेनोपधमेन्न च पादौ प्रतापयेत्}
{नाधः कुर्यात्कदाचित्तु न पृष्ठं परितापयेत्}


\twolineshloka
{नान्तरा गमनं कुर्यान्न चामेध्यं विनिक्षिपेत्}
{उच्छिष्टो न स्पृशेदग्निमाशौचस्थो न जातुचित्}


\twolineshloka
{श्वचण्डालादिभिः स्पृष्टो नाङ्गमग्नौ प्रतापयेत्}
{सर्वदेवमयो वह्निस्तस्माच्छुद्धः सदा स्पृशेत्}


\twolineshloka
{प्राप्तमूत्रपुरीषस्तु न स्पृशेद्वह्निमात्मवान्}
{यावत्तु धारयेद्वेगं तावदप्रयतो भवेत्}


\twolineshloka
{पचनाग्निं न गृह्णीयात्परवेश्मनि जातुचित्}
{तस्मिन्पक्वेन चान्नेन यत्कर्म कुरुते शुभम्}


\twolineshloka
{तस्यैव तच्छुभस्यार्धमग्निदस्य भवेन्नृप}
{तस्माद्गृहगतं वह्निं प्रकुर्यादविनाशितम्}


\threelineshloka
{प्रमादाद्यदि वाऽज्ञानात्तस्य नाशो भविष्यति}
{गृह्णीयात्तु मथित्वा वा श्रोत्रियागारतोपि वा ॥युधिष्ठिर उवाच}
{}


\threelineshloka
{कीदृशाः साधवो विप्रास्तेभ्यो दत्तं महाफलम्}
{कीदृशेभ्यो हि दातव्यं तन्मे ब्रूहि जनार्दन ॥भगवानुवाच}
{}


\twolineshloka
{अक्रोधनाः सत्यपरा धर्मनित्या जितेन्द्रियाः}
{तादृशाः साधवो विप्रास्तेभ्यो दत्तं महाफलम्}


\twolineshloka
{अमानिनः सर्वसहा दृष्टार्था विजितेन्द्रियाः}
{सर्वभूतहिता मैत्रास्तेभ्यो दत्तं महाफलम्}


\twolineshloka
{अलुब्धाः शुचयो वैद्या हीमन्तः सत्यवादिनः}
{स्वधर्मनिरता ये तु तेभ्यो दत्तं महाफलम्}


\twolineshloka
{साङ्गंश्च चतुरो वेदान्योऽधीयेत दिनेदिने}
{शूद्रान्नं यस्य नो देहे तत्पात्रमृषयो विदुः}


\twolineshloka
{प्रज्ञाश्रुताभ्यां वृत्तेन शीलेन च समन्वितः}
{तारयेत्तत्कुलं सर्वमेकोपीह युधिष्ठिर}


\fourlineindentedshloka
{गामश्वमन्नं वित्तं वा तद्विधे प्रतिपादयेत्}
{निशम्य तु गुणोपेतं ब्राह्मअणं साधुसंमतम्}
{दूरादाहृत्य सत्कृत्य तं प्रयत्नेन पूजयेत् ॥युधिष्टिर उवाच}
{}


\threelineshloka
{धर्माधर्मविधिस्त्वेवं भीमं भीष्मेण भाषितम्}
{भीष्मवाक्यात्सारभूतं वद धर्मं सुरेश्वर ॥भगवानुवाच}
{}


\twolineshloka
{अन्नेन धार्यते सर्वं जगदेतच्चराचरम्}
{अन्नात्प्रभवति प्राणः प्रत्यक्षं नास्ति संशयः}


\twolineshloka
{कलत्रं पीडयित्वा तु देशे काले च शक्तितः}
{दातव्यं भिक्षवे चान्नमात्मनो भूतिमिच्छता}


\twolineshloka
{विप्रमध्वपरिश्रान्तं बालं वृद्धमथापि वा}
{अर्चयेद्गुरुवत्प्रीतो गृहस्थो गृहमागतम्}


\twolineshloka
{क्रोधमुत्पतितं हित्वा सुशीलो वीतमत्सरः}
{अर्चयेदतिथिं प्रीतः परत्र हितभूतये}


\twolineshloka
{अतिथिं नावमन्येत नानृतां गिरमीरयेत्}
{न पृच्छेद्गोत्रचरणं नाधीतं वा कदाचन}


\twolineshloka
{चण्डालो वा श्वपाको वा काले यः कश्चिदागतः}
{अन्नेन पूजनीयः स्यात्परत्र हितमिच्छता}


\twolineshloka
{पिधाय तु गृहद्वारं भुक्ते योऽन्नं प्रहृष्टवान्}
{स्वर्गद्वारपिधानं वै कृतं तेन युधिष्ठिर}


\twolineshloka
{पितॄन्देवानृषीन्विप्रानतिथींश्च निराश्रयान्}
{यो नरः प्रीणयत्यन्नैस्तस्य पुण्यफलं महत्}


\twolineshloka
{कृत्वा तु पापं बहुशो यो दद्यादन्नमर्थिने}
{ब्राह्मणाय विशेषेण सर्वपापैः प्रमुच्यते}


\twolineshloka
{अन्नदः प्राणदो लोके प्राणदः सर्वदो भवेत्}
{तस्मादन्नं विशेषेण दातव्यं भूतिमिच्छता}


\twolineshloka
{अन्नं ह्यमृतमित्याहुरन्नं प्रजननं स्मृतम्}
{अन्नप्रणाशो सीदन्ति शरीरे पञ्च धातवः}


\twolineshloka
{बलं बलवतो नश्येदन्नहीनस्य देहिनः}
{तस्मादन्नं विशेषेण श्रद्धयाऽश्रद्धयापि वा}


\twolineshloka
{आदत्ते हि रसं सर्वमादित्यः स्वगभस्तिभिः}
{वायुस्तस्मात्समादाय रसं मेधेषु धारयेत्}


\twolineshloka
{तत्तु मेघगतं भूमौ शक्रो वर्षति तादृशम्}
{तेन दिग्धा भवेद्देवी मही प्रीता च ****}


\twolineshloka
{तस्यां सस्यानि रोहन्ति यैर्जीवन्त्यखिलाः प्रजाः}
{मांसमेदोस्थिमज्जानां सम्भवस्तेभ्य एव हि}


\twolineshloka
{एवं सूर्यश्च पवनो मेघः शक्रस्तथैव च}
{एक एव स्थितो राशिर्यतो भूतानि जज्ञिरे}


\twolineshloka
{भवनानि च दिव्यानि दिवि तेषां महात्मनाम्}
{नानासंस्थानि भूतानि नानाभूमिगतानि च}


\twolineshloka
{चन्द्रमण्डलशुभ्राणि किङ्किणीजालवन्ति च}
{तरुणादित्यवर्णानि स्थावराणि चराणि च}


\twolineshloka
{अनेकशतसङ्ख्यानि सान्तर्जलवनानि च}
{तत्र पुष्पफलोपेताः कामदाः सुरपादपाः}


\twolineshloka
{वाप्यो बद्धसभाः कूपा दीर्घिकाश्च सहस्रशः}
{भक्ष्यभोज्यमयाः शैला वासांस्याभरणानि च}


\twolineshloka
{क्षीरस्रवन्त्यः सरितस्तथा चैवान्नपर्वताः}
{घोषवन्ति च यानानि युक्तान्यथ सहस्रशः}


\twolineshloka
{प्रासादप्रवराः शुभ्राः शय्याश्च कनकोज्ज्वलाः}
{अन्नदास्तत्र तिष्ठन्ति तस्मादन्नप्रदो भवेत्}


\chapter{अध्यायः १०९}
\threelineshloka
{अन्नदानफलं श्रुत्वा प्रीतोस्मि मधुसूदन}
{भोजनस्य विधिं वक्तुं देवदेव त्वमर्हसि ॥भगवानुवाच}
{}


\twolineshloka
{बोजनस्य द्विजातीनां विधानं शृणु पाण्डव}
{स्नातः शुचिः शुचौ देशे निर्जने हुतपावकः}


\twolineshloka
{मण्डलं कारयित्वा च चतुरश्रं द्विजोत्तमः}
{क्षत्रियश्चेत्ततो वृत्तं वैश्योऽर्धेन्दुसमाकृतिम्}


\twolineshloka
{आर्द्रपादस्तु भुञ्जीयात्प्राङ्मुखश्चासने शुचौ}
{पादाभ्यां धरणीं स्पृष्ट्वा पादेनैकेन वा पुनः}


\twolineshloka
{नैकवासास्तु भुञ्जीयान्न चान्तर्धाय वा द्विजः}
{न भिन्नपात्रे भुञ्जीत पर्णपृष्ठे तथैव च}


\twolineshloka
{अन्नं पूर्वं नमस्कुर्यात्प्रहृष्टेनान्तरात्मना}
{नान्यदालोकयेदन्नान्न जुगुप्सेन तत्परः}


\twolineshloka
{जुगुप्सितं च यच्चान्नं राक्षसा एव भुञ्जते}
{पाणिना जलमुद्धृत्य कुर्यादन्नं प्रदक्षिणम्}


\twolineshloka
{अपेयं तद्विजानीयात्पीत्वा चान्द्रायणं चरेत्}
{परिवेषजलादन्यत्पेयमेव तु मन्त्रवत्}


% Check verse!
पञ्च प्राणाहुतीः कुर्यात्समन्त्रं तु पृथक्पृथक्
\twolineshloka
{यथा रसं न जानाति जिह्वा प्रामाहुतौ नृप}
{तथा समाहितः कुर्यात्प्राणाहुतिमतन्द्रिकतः}


\twolineshloka
{विदित्वाऽन्नमथान्नादं पञ्च प्राणांश्च पाण्डव}
{यः कुर्यादाहुतीः पञ्च तेनेष्टाः पञ्चवायवः}


\twolineshloka
{अतोऽन्यथा तु भुञ्जानो ब्राह्मणो ज्ञानदुर्बलः}
{तेनान्नेनासुरान्प्रेतान्राक्षसांस्तर्पयिष्यति}


\twolineshloka
{वक्त्रप्रमाणान्पिण्डांश्च ग्रसेदेकैकशः पुनः}
{वक्त्राधिकं तु यत्पिण्डमात्मोच्छिष्टं तदुच्यते}


\twolineshloka
{पिण्डावशिष्टमन्यच्च वक्त्रान्निस्सृतमेव च}
{अभोज्यं तद्विजानीयाद्भुक्त्वा चान्द्रायणं चरेत्}


\twolineshloka
{स्वमुच्छिष्टं तु यो भुङ्क्ते यो भुङ्क्ते मुक्तभोजनम्}
{चान्द्रायणं चरेन्कृच्छ्रं प्राजापत्यमथापि वा}


\threelineshloka
{स्त्रीपात्रभुङ्नरः पापः स्त्रीणामुच्छिष्टभुक्तथा}
{तया सह च यो भुङ्क्ते स भुङ्क्ते मद्यमेव हि}
{न तस्य निष्कृतिर्दृष्टा मुनिभिस्तत्वदर्शिभिः}


\twolineshloka
{पिबतः पतिते तोये भोजने मुखनिस्सृते}
{अभोज्यं तद्विजानीयाद्भुक्त्वा चान्द्रायणं चरेत्}


\twolineshloka
{पीतशेषं तु तन्नाम न पेयं पाण्डुनन्दन}
{पिबेद्यदि हि तन्मोहाद्द्विजश्चान्द्रायणं चरेत्}


\twolineshloka
{पानीयानि पिबेद्यते तत्पात्रं द्विजसत्तमः}
{अनुच्छिष्टं भवेत्तावद्यावद्भूमौ न निक्षिपेत्}


\twolineshloka
{मौनी वाऽप्यथवा भूमौ नावलोक्य दिशस्तथा}
{भुञ्जीत विधिवद्विप्रो न चोच्छिष्टं प्रदापयेत्}


\twolineshloka
{सदा चात्यशनं नाद्यान्नातिहीनं च कर्हिचित्}
{यथाऽन्नेन व्यथा न स्यात्तथा भुञ्जीत नित्यशः}


\threelineshloka
{उदक्यामपि चण्डालं श्वानं सूकरमेव वा}
{भुञ्जानो यदि वा पश्येत्तदन्नं च परित्यजेत्}
{भुञ्जानो ह्यत्यजन्मोहाद्द्विजश्चान्द्रायणं चरेत्}


\twolineshloka
{केशकीटोपपन्नं च मुखमारुतवीजितम्}
{अभोज्यं तद्विजानीयाद्भुक्त्वा चान्द्रायणं चरेत्}


\twolineshloka
{उत्थाय च पुनः स्पृष्टं पादस्पृष्टं च लङ्घितम्}
{अन्नं तद्राक्षसं विद्यात्तस्मात्तत्परिवर्जयेत्}


\twolineshloka
{राक्षसोच्छिष्टभुग्विप्रः सप्त पूर्वान्परानपि}
{नरके रौरवे घोर स पितॄन्पातयिष्यति}


\fourlineindentedshloka
{तस्मिन्नाचमनं कुर्याद्यस्मिन्पात्रे स भुक्तवान्}
{यद्युत्तिष्ठत्यनाचान्तो भुक्तवानासनात्ततः}
{स्नानं सद्यः प्रकुर्वीत सोन्यथाऽप्रयतो भवेत् ॥युधिष्ठिर उवाच}
{}


\threelineshloka
{तृणमुष्टिविधानं च तृणमाहात्म्यमेव च}
{इक्षोः सोमसमुद्भूतिं वक्तुमर्हसि मानदः ॥भगवानुवाच}
{}


\twolineshloka
{पितरो वृषभा ज्ञेयो गावो लोकस्य मातरः}
{तासां तु पूजया राजन्पूजिताः पितृदेवताः}


\twolineshloka
{सभा प्रपा गृहाश्चापि देवतायतनानि च}
{शुद्ध्यन्ति शकृता यासांकिं भूतमधिकं ततः}


\twolineshloka
{ग्रासमुष्टिं परगवे दद्यात्संवत्सरं तु यः}
{अकृत्वा स्वयमाहारं प्राप्तस्तत्सार्वकालिकम्}


\twolineshloka
{गावो मे मातरः सर्वाः पितरश्चैव गोवृषाः}
{ग्रासमुष्टिं मया दत्तं प्रतिगृह्णीत मातरः}


\twolineshloka
{इत्युक्त्वाऽनेन मन्त्रेण गायत्र्या वा समाहितः}
{अभिमन्त्र्य ग्रासमुष्टिं तस्य पुण्यफलं शृणु}


\twolineshloka
{यत्कृतं दुष्कृतं तेन ज्ञानतोऽज्ञानतोपि वा}
{तस्य नश्यति तत्सर्वं दुःखप्नं च विनश्यति}


\twolineshloka
{तिलाः पिवित्राः पापघ्ना नारायणसमुद्भवाः}
{तिलाञ्श्राद्धे प्रशंसनति दानमेतदनुत्तमम्}


\twolineshloka
{तिलान्दद्यात्तिलान्भक्ष्यात्तिलानप्रातरुपस्पृशेत्}
{तिलंतिलमिति ब्रूयात्तिलाः पापहरा हि ते}


\threelineshloka
{क्रीत्वा प्रतिगृहीत्वा वा न विक्रेया द्विजातिभिः}
{भोजनाभ्यञ्जनाद्दानाद्योन्यत्तु कुरुते तिलैः}
{कृमिर्भूत्वा श्वविष्ठायां पितृभिः सह मज्जति}


\twolineshloka
{तिलान्नपीडयेद्विप्रो यन्त्रचक्रे स्वयं नृप}
{पीडयन्हि द्विजो मोहान्नरकं याति रौरवम्}


\twolineshloka
{इक्षुवंशोद्भवः सोमः सोमवंशोद्भवा द्विजाः}
{इक्षुं न पीडयेत्तस्मादिक्षुघात्यात्मघातकः}


\threelineshloka
{इक्षुदण्डसहस्राणामेकैकेन युधिष्ठिर}
{ब्रह्महत्यामवाप्नोति द्विजश्चेद्यन्त्रपीडकः}
{तस्मान्न पीडयेदिक्षुं यन्त्रचक्रे द्विजोत्तमः}


\chapter{अध्यायः ११०}
\threelineshloka
{समुच्चयं च धर्माणां भोज्याभोज्यं तथैव च}
{श्रुतं मया त्वत्प्रसादादापद्धर्मं ब्रवीहि मे ॥भगवानुवाच}
{}


\twolineshloka
{दुर्भिक्षे राष्ट्रसंबाधेऽप्याशौचे मृतसूतके}
{धर्मकालेऽध्वनि तथा नियमो येन लुप्यते}


\twolineshloka
{दूराध्वगमनात्खिन्नो द्विजालाभेऽथ शूद्रतः}
{अकृतन्नं तु यत्किंचिद्गृह्णीयादात्मवृत्तये}


\twolineshloka
{आतुरो दुःखितो वाऽपि तथाऽऽर्तो वा बुभुक्षितः}
{भुञ्जन्नविधिना विप्रः प्रायश्चित्तीयते न च}


\twolineshloka
{निमन्त्रितस्तु यो विप्रो विदिवद्धव्यकव्ययोः}
{मांसादीन्यपि भुञ्जानः प्रायश्चित्तीयते न च}


\twolineshloka
{अष्टौ तान्यव्रतघ्नानि आपो मूलं घृतं पयः}
{हविर्ब्राह्मणकाम्या च गुरोर्वचनमौषधम्}


\twolineshloka
{अशक्तो विधिवत्कर्तुं प्रायश्चित्तानि यो नरः}
{विदुषां वचनेनापि दानेनापि विशुद्ध्यति}


\threelineshloka
{अनृतावृतुकाले वा दिवा रात्रौ तथाऽपि वा}
{प्रोषितस्तु स्त्रियं गच्छेत्प्रायश्चित्तीयते न च ॥युधिष्ठिर उवाच}
{}


\threelineshloka
{प्रशस्याः कीदृशा विप्रा निन्द्याश्चापि सुरेश्वर}
{अष्टकायाश्च कः कालस्तन्मे कथय सुव्रत ॥भगवानुवाच}
{}


\twolineshloka
{सत्यसन्धं द्विजं दृष्ट्वा स्थानाद्वेपति भास्करः}
{एष मे मण्डलं भित्त्वा याति ब्रह्म सनातनम्}


\twolineshloka
{कुलीनः कर्मकृद्वैद्यस्तथा चाप्यानृशंस्यवान्}
{श्रीमानृजउः सत्यवादी पात्रं सर्व इमे द्विजाः}


\twolineshloka
{एते चाग्रासनस्थास्ते भुञ्जानाः प्रथमं द्विजाः}
{तस्यां पङ्क्त्यां तु ये वान्ये तान्पुनत्येव दर्शनात्}


\twolineshloka
{मद्भक्ता ये द्विजश्रेष्ठा मद्भक्ता मत्परायणाः}
{तान्पङ्क्तिपावनान्विद्धि पूज्यांश्चैव विशेषतः}


\twolineshloka
{निन्द्याञ्शृणु द्विजान्राजन्नपि वा वेदपारगान्}
{ब्राह्मणच्छद्मना लोके चरतः पापकारिणः}


\twolineshloka
{अनग्निरनधीयानः प्रतिग्रहरुचिस्तु यः}
{यतस्ततस्तु भुञ्जानस्तं विद्याद्ब्रह्मदूषकम्}


\twolineshloka
{मृतसूतकपुष्टाङ्गो यश्च शूद्रान्नभुग्द्विजः}
{अहं चापि न जानामि गतिं तस्य नराधिप}


\twolineshloka
{शूद्रान्नरसपुष्टाङ्गोऽप्यधीयानो हि नित्यशः}
{जपतो जुह्वतो वाऽपि गतिरूर्ध्वं न विद्यते}


\twolineshloka
{आहिताग्निश्च यो विप्रः शूद्रान्नान्न निवर्तते}
{पञ्च तस्य प्रणश्यन्ति आत्मा ब्रह्म त्रयोऽग्नयः}


\twolineshloka
{शूद्रप्रेषणकर्तुश्च ब्राह्मणस्य विशेषतः}
{भूमावन्नं प्रदातव्यं श्वसृगालसमो हि सः}


% Check verse!
प्रेतभूतं तु यः शूद्रं ब्राह्ममो ज्ञानदुर्बलः ॥अनुगच्छेन्नीयमानं त्रिरात्रमशुचिर्भवेत्
\twolineshloka
{त्रिरात्रे तु ततः पूर्णे नदीं गत्वा समुद्रगाम्}
{प्राणायामशतं कृत्वा घृतं प्राश्य विशुद्ध्यति}


\twolineshloka
{अनाथं ब्राह्मणं प्रेतं ये वहन्ति द्विजोत्तमाः}
{पदेपदेऽश्वमेधस्य फलं ते प्राप्नुवन्ति हि}


\twolineshloka
{न तेषामशुभं किंचित्पापं वा शुभकर्मणाम्}
{जलावगाहनादेव सद्यः शौचं विधीयते}


\twolineshloka
{शूद्रवेश्मनि विप्रेणि क्षीरं वा यदि वा दधि}
{निवृत्तेन न भोक्तव्यं विद्धि शूद्रान्नमेव तत}


\twolineshloka
{विप्राणां भोक्तुकामानामत्यन्तं चान्नकाङ्क्षिणाम्}
{यो विघ्नं कुरुते मर्त्यस्ततो नान्योस्ति पापकृत्}


\twolineshloka
{सर्वे च वेदाः सहषङ्भिरङ्गैःसाङ्ख्यं पुराणं च कुले च जन्म}
{नैतानि सर्वाणि गतिर्भवन्तिशीलव्यपेतस्य नृप द्विजस्य}


\twolineshloka
{ग्रहोपरागे विषुवेऽयनान्तेपित्र्ये मघासु स्वसुते च जाते}
{गयेषु पिण्डेषु च पाण्डुपुत्रदत्तं भवेन्निष्कसहस्रतुल्यम्}


\twolineshloka
{वैशाखमासस्य तु या तृतीयाऽ-नवद्याऽसौ कार्तिकशुक्लपक्षे}
{नभस्यमासस्य च कृष्णपक्षेत्रयोदशी पञ्चदशी न माघे}


\threelineshloka
{उपप्लवे चन्द्रमसो रवेश्चश्राद्धस्य कालो ह्ययनद्वये च}
{पानीयमप्यत्रि तिलैर्विमिश्रंदद्यात्पितृभ्यः प्रयतो मनुष्यः}
{श्राद्धं कृतं तेन समासहस्रंरहस्यमेतत्पितरो वदन्ति}


\twolineshloka
{यस्त्वेकपङ्क्त्यां विषमं ददातिस्नेहाद्भयाद्वा यदि वाऽर्थहेतोः}
{क्रूरं दुराचारमनात्मवन्तंब्रह्मघ्नमेनं कवयो वदन्ति}


\twolineshloka
{धनानि येषां विपुलानि सन्तिनित्यं रमन्ते परलोकमूढाः}
{तेषामयं शत्रुवरघ्नलोकोनान्यत्सुखं देहसुखे रतानाम्}


\twolineshloka
{ये चैव मुक्तास्तपसि प्रयुक्ताःस्वाध्यायशीला जरयन्ति देहम्}
{जितेन्द्रिया भूतहिते निविष्ट-स्तेषामसौ चापि परश्च लोकः}


\threelineshloka
{ये चैव विद्यां न तपो न दानंन चापि मूढाः प्रजने यतन्ते}
{न चापि गच्छन्ति सुखानि भोगां-स्तेषामयंक चापि परश्च नास्ति ॥युधिष्ठिर उवाच}
{}


\threelineshloka
{नारायण पुराणेश लोकावास नमोस्तु ते}
{श्रोतुमिच्छामि कार्त्स्न्येन धर्मसारसमुच्चयम् ॥भगवानुवाच}
{}


\twolineshloka
{धर्मसारं महाप्राज्ञि मनुना प्रोक्तमादितः}
{प्रवक्ष्यामि मनुप्रोक्तं पौराणां श्रुतिसंहितम्}


\twolineshloka
{अग्निचित्कपिला सत्री राजा भिक्षुर्महोदधिः}
{दृष्टमात्रात्पुनत्येते तस्मात्पश्येत तान्सदा}


\twolineshloka
{गौरकस्यैव दातव्या न बहूनां युधिष्ठिर}
{सा गौर्विक्रयमापन्ना दहत्यासप्तमं कुलम्}


\twolineshloka
{बहूनां न प्रदातव्या गौर्वस्त्रं शयनं स्त्रियः}
{तादृग्भूतं तु तद्दानं दातारं नोपतिष्ठति}


\twolineshloka
{आक्रम्य ब्राह्मणैर्भुक्तमनार्याणां च वेश्मनि}
{गोभिश्च पुण्यं तत्तेषां राजसूयाद्विशिष्यते}


\twolineshloka
{मा ददात्विति यो ब्रूयाद्ब्राह्मणेषु च गोषु च}
{तिर्यग्योनिशतं गत्वा चण्डालेषूपजायते}


\twolineshloka
{ब्राह्मणस्वं च यद्दैवं दरिद्रस्यैव यद्धनम्}
{गुरोश्चापि हृतं राजन्स्वर्गस्थानपि पातयेत्}


\twolineshloka
{धर्मं जिज्ञासमानानां प्रमाणं परमं श्रुतिः}
{द्वितायं धर्मशास्त्राणि तृतीयं लोकसंग्रहः}


\twolineshloka
{आसमुद्राच्च यत्पूर्वादासमुद्राच्च पश्चिमात्}
{हिमाद्रिविन्ध्ययोर्मध्यमार्यावर्तं प्रचक्षते}


\twolineshloka
{सरावतीदृषद्वत्योर्देवनद्योर्यदन्तरम्}
{तद्देवनिर्मितं देशं ब्रह्मवर्तं प्रचक्षते}


\twolineshloka
{यस्मिन्देशे य आचारः पारंपर्यक्रमागतः}
{वर्णानां सान्तरालानां स सदाचार उच्यते}


\twolineshloka
{कुरुक्षेत्रं च मत्स्याश्च पाञ्चालाः शूरसेनयः}
{एते ब्रह्मर्षिदेशास्तु ब्रह्मावर्तादनन्तराः}


\twolineshloka
{एतद्देशप्रसूतस्य सकाशादग्रजन्मनः}
{स्वं चारित्रं च गृह्णीयुः पृथिव्यां सर्वमानवाः}


\twolineshloka
{हिमवद्विन्ध्ययोर्मध्यं यत्प्राग्विशसनादपि}
{प्रत्यगेव प्रयागात्तु मध्यदेशः प्रकीर्तितः}


\twolineshloka
{कृष्णसारस्तु चरति मृगो यत्र स्वभावतः}
{स ज्ञेयो याज्ञिको देशो म्लेच्छदेशस्ततः परम्}


\twolineshloka
{एतान्विज्ञाय देशांस्तु संश्रयेरन्द्विजातयः}
{शूद्रस्तु यस्मिन्कस्मिन्वा निवसेद्वृत्तिकर्शितः}


\twolineshloka
{आचारः प्रथमो धर्मो ह्यहिंसा सत्यमेव च}
{दानं चैव यथाशक्ति नियमाश्च यमैः सह}


\twolineshloka
{वैदिकैः कर्मभिः पुण्यैर्निषेकादिर्द्विजन्मनाम्}
{कार्यः शरीरसंस्कारः पावनः प्रेत्य चेह च}


\threelineshloka
{गर्भहोमैर्जातकर्मनामचौलोपनायनैः}
{स्वाध्यायैस्तद्व्रतैश्चैव विवाहस्नातकव्रतैः}
{महायज्ञैश्च यज्ञैश्च ब्राह्मीयं क्रियते तनुः}


\twolineshloka
{धर्मोर्थौ यदि न स्यातां शुश्रुषा वाऽपि तद्विधा}
{विद्या तस्मिन्नवप्तव्या शुभं बीजमिवोषरे}


\twolineshloka
{लौकिकं वैदिकं वाऽपि तथाऽऽध्यात्मिकमेव वा}
{यस्माज्ज्ञानमिदं प्राप्तं तं पूर्वमभिवादयेत्}


\twolineshloka
{सव्येन सव्यं संगृह्य दक्षिणेन तु दक्षिणम्}
{न कुर्यादेकहस्तेन गुरोः पादाभिवादनम्}


\twolineshloka
{निषेकादीनि कर्माणि यः करोति यथाविधि}
{अध्यापयति चैवेनं स विप्रो गुरुरुच्यते}


\twolineshloka
{कृत्वोपनयनं वेदान्योध्यापयति नित्यशः}
{सकल्पान्सरहस्यांश्च स चोपाध्याय उच्यते}


\twolineshloka
{साङ्गांश्च वेदानध्याप्य शिक्षयित्वा व्रतानि च}
{विवृणोति च मन्त्रार्थानाचार्यः सोभिधीयते}


\twolineshloka
{उपाध्यायाद्दशाचार्य आचार्याणां शतं पिता}
{पितुः शतगुणं माता गौरवेणातिरिच्यते}


\twolineshloka
{एतेषामपि सर्वेषां गरीयान्ज्ञानदो गुरुः}
{गुरोः परतरं किंचिन्न भूतं न भविष्यति}


\twolineshloka
{तस्मात्तेषां वशे तिष्ठिच्छुश्रूषापरमो भवेत्}
{अवमानाद्धि तेषां तु नरकं स्यान्न संशयः}


\twolineshloka
{हीनाङ्गानतिरिक्ताङ्गान्विद्याहीनान्वयोधिकान्}
{रूपद्रविणहीनांश्च जातिहीनांस्च नाक्षिपेत्}


\twolineshloka
{शपता यत्कृतं पुण्यं शप्यमानं तु गच्छति}
{शप्यमानस्य यत्पापं शपन्तमनुगच्छति}


\twolineshloka
{नास्तिक्यं वेदनिन्दां च देवतानां च कुत्सनम्}
{द्वेषं डंभं च मानं च क्रोधं तैक्ष्ण्यं विवर्जयेत्}


\chapter{अध्यायः १११}
\threelineshloka
{स्वभक्तस्य हृषीकेश धर्माधर्ममजानतः}
{धर्मं पुण्यतमं देव पृच्छतः कथयस्व मे ॥भगवानुवाच}
{}


\fourlineindentedshloka
{यदेतदग्निहोत्रं वै सृष्टं वर्णत्रयस्य तु}
{मन्त्रवद्यद्धुतं सम्यग्विधिना चाप्युपासितम्}
{आहिताग्निं नयत्यूर्ध्वं सपत्नीकं सबान्धवम् ॥युधिष्ठिर उवाच}
{}


\twolineshloka
{कथं तद्ब्राह्मणैर्देव होतव्यं क्षत्रियैः कथम्}
{वैश्यैर्वा देवदेवेश कथं वा सुहुतं भवेत्}


\twolineshloka
{कस्मिन्काले कतं कस्य आधेयोऽग्निः सुरोत्तम}
{आहितस्य कथं वाऽपि सम्यगाचरणं भवेत्}


\twolineshloka
{कत्यग्नयः किमात्मानः स्थानं किं कस्य वा विभो}
{कतरस्मिन्हुते स्थाने कं व्रजेदाग्निहोत्रिकः}


\twolineshloka
{अग्निहोत्रनिमित्तं च किमुत्पन्नं पुराऽनघ}
{कथमेवाथ हूयन्ते प्रीयन्ते च सुराः कथम्}


\twolineshloka
{विधिवन्मन्त्रवत्कृत्वा पूजितास्त्वग्नयः कथम्}
{कां गतिं वदतांश्रेष्ठ नयन्ति ह्यग्निहोत्रिणः}


\twolineshloka
{दुर्हुताश्चापि भगवन्नविज्ञातास्त्रयोऽग्नयः}
{किमाहिताग्नेः कुर्वन्ति दुश्चीर्णा वाऽपि केशव}


\fourlineindentedshloka
{उत्सन्नाग्निस्तु पापात्मा कां योनिं देव गच्छति}
{एतत्सर्वं समासेन भक्त्या ह्युपगतस्य मे}
{वक्तुमर्हसि सर्वज्ञ सर्वाधिकं नमोस्तु ते ॥भगवानुवाच}
{}


\twolineshloka
{शृणु राजन्महापुण्यमिदं धर्मामृतं परम्}
{यत्तु तारयते युक्तान्ब्राह्मणानग्निहोत्रिणः}


\twolineshloka
{ब्र्हमित्वेनासृजं लोकानहमादौ महाद्युते}
{सृष्टोऽग्निर्मुखतः पूर्वं लोकानां हितकाम्यया}


\twolineshloka
{यस्मादग्रे स भूतानां सर्वेषां निर्मितो मया}
{तस्मादग्नीत्यभिहितः पुराणज्ञैर्मनीषिभिः}


\twolineshloka
{यस्मात्तु सर्वकृत्येषु पूर्वमस्मै प्रदीयते}
{आहुतिर्दीप्यमानाय तस्मादग्नीति कथ्यते}


\twolineshloka
{यस्माच्च तु नयत्यग्रां गतिं विप्रान्सुपूजितः}
{यस्माच्च नयनाद्राजन्देवेष्वग्नीति कथ्यते}


\twolineshloka
{तस्माच्च दुर्हुतः सोयमलं भक्षयितुं क्षणात्}
{यजमानं नरश्रेष्ठ क्रव्यादोऽग्निस्ततः स्मृतः}


\twolineshloka
{सर्वभूतात्मको राजन्देवानामेष वै मुखम्}
{प्रथमं मन्मुखात्सृष्टो लोकार्थे पचनः प्रभुः}


\twolineshloka
{सृष्टमात्रो जगत्सर्वमत्तुमैच्छत्पुरा खलु}
{ततः प्रशमितः सोग्निरूपास्यैव मया पुरा}


% Check verse!
सततोपासनात्सोयमौपासन इति स्मृतः
\twolineshloka
{आहुतिः सर्वमाख्याति तस्मिन्वसति सोनलः}
{आवसथ्य इति ख्यातस्तेनासौ ब्रह्मवादिभिः}


\twolineshloka
{तस्मात्पञ्च महायज्ञा वर्तन्ते यस्य धर्मतः}
{सोममण्डलमध्येन गतिस्तस्य द्विजन्मनः}


\twolineshloka
{ते च सप्तर्षयः सिद्धाः संयतेन्द्रियबुद्धयः}
{गता ह्यमरसायुज्यं ते ह्यग्न्यर्चनतत्पराः}


\twolineshloka
{अपरे चावसथ्यं च पचनाग्निं प्रचक्षते}
{तस्मिन्पञ्च महायज्ञा वैश्वदेवश्च वर्तते}


\twolineshloka
{स्थलीपाकं च गार्हं च सर्वे चास्मिन्प्रतिष्ठिताः}
{गृह्यकर्मवहो यस्मात्तस्माद्गृहपतिस्तु सः}


\twolineshloka
{औपासनं चावसथ्यं सभ्यं पचनपावकम्}
{आहुर्ब्रह्मविदः केचिन्मतमेतन्ममापि च}


\twolineshloka
{अग्निहोत्रप्रकारं च शृणु राजन्समाहितः}
{त्रयाणां गुणनामानां वह्नीनामुच्यते मया}


\twolineshloka
{गृहाणां हि पतित्वं हि गृहपत्यमिति स्मृतम्}
{गृहपत्यं तु यस्यासीत्तत्तस्माद्गार्हपत्यता}


\twolineshloka
{यजमानं तु यस्मात्तु दक्षिणां तु गतिं नयेत्}
{दक्षिणाग्निं तमाहुस्ते दक्षिणायतनं द्विजाः}


\twolineshloka
{आहुतिः सर्वमाख्याति हव्यं वै वहनं स्मृतम्}
{सर्वहव्यवहो वह्निर्गतश्चाहवनीयताम्}


\twolineshloka
{यं चाक्सथ्यं जुहुयान्मूलाग्निं विधिवद्द्विजः}
{आवसथ्यं तु तं चाग्निं पचनाग्निं प्रचक्षते}


\twolineshloka
{तेषां स भागतो वह्निः सभ्य इत्यभिधीयते}
{आवतथ्यस्तु यो वह्निः प्रथमः स प्रजापतिः}


\twolineshloka
{ब्रह्मा च गार्हपत्योऽग्निस्तस्मिन्नेव हि सोभवत्}
{दक्षिणाग्निस्त्वयं रुद्रः क्रोधात्मा चण्ड एव सः}


\twolineshloka
{अहमाहवनीयोऽग्निराहोमाद्यस्य वै मुखे}
{सभ्योऽग्निः पञ्चमो यस्तु स्कन्द एव नराधिप}


\twolineshloka
{पृथिवी गार्हपत्योऽग्निरन्तरिक्षं च दक्षिणः}
{स्वर्गमाहवनीयोऽग्निरेवमग्नित्रयं स्मृतम्}


\twolineshloka
{वृत्तश्च गार्हपत्योऽग्निर्यस्माद्वृत्ता च मेदिनी}
{अर्धचन्द्राकृतिस्तं वै दक्षिणाग्निस्तथा भवेत्}


\twolineshloka
{चतुरश्रं ततः स्वर्गं निर्मलं तु निरामयम्}
{तस्मादाहवनीयोऽग्निश्चतुरश्रो भवेन्नृप}


\twolineshloka
{जुहुयाद्गार्हपत्यं यो भुवं जयति स द्विजः}
{जुहोति दक्षिणं यस्तु स जयत्यन्तरिक्षकम्}


\twolineshloka
{पृथिवीमन्तरिक्षं च दिवं ऋषिगणैः सह}
{जयत्याहवनीयं यो जुहुयाद्भक्तिमान्नरः}


\twolineshloka
{आभिमुख्येन होमस्तु यस्य यज्ञेषु वर्तते}
{तेनाप्याहवनीयत्वं गतो वह्निर्महाद्युतिः}


\twolineshloka
{आहोमादग्निहोत्रेषु यज्ञैर्वा यत्र सर्वशः}
{यस्मात्तस्मात्प्रवर्तन्ते ततो ह्याहवनीयता}


\twolineshloka
{यस्त्वावसथ्यं जुहुयान्मूलाग्निं विधिवद्द्विजः}
{स तु सप्तर्षिलोकेषु सपत्नीकः प्रमोदते}


\twolineshloka
{इष्टो भवति सर्वाग्नेरग्निहोत्रं च तद्भवेत्}
{त्राणाद्वै यजमानस्य ह्यग्निहोत्रमिति स्मृतम्}


\twolineshloka
{होइज्येषु विषादो वै विषादो दुःखमुच्यते}
{दुःखं तापत्रयं प्रोक्तं तापं हि नरकं विदुः}


\twolineshloka
{आध्यात्मिकं चाधिदैवमाधिभौतिकमेव च}
{एतत्तापत्रयं प्रोक्तमात्मविद्भिर्नराधिप}


\twolineshloka
{यस्माद्वै त्रायते दुःखाद्यजमानं हुतोऽनलः}
{तस्मात्तु विधिवत्प्रोक्तमग्निहोत्रमिति श्रुतौ}


\twolineshloka
{तदग्निहोत्रं सृष्टं वै ब्रह्मणा लोककर्तृणा}
{वेदाश्चाप्यग्निहोत्रं तु जज्ञिरे स्वयमेव तु}


% Check verse!
अग्निहोत्रफला दारा दत्तभुक्तफलं धनम् ॥रतिपुत्रफला दारा दत्तभुक्तफलं धनम्
\twolineshloka
{त्रिवेदमन्त्रसंयोगादग्निहोत्रं प्रवर्तते}
{ऋग्यजुःसामभिः पुण्यैः स्थाप्यते सूत्रसंयुतैः}


\twolineshloka
{वसन्ते ब्राह्मणस्य स्यादाधेयोऽग्निर्नराधिप}
{वसन्तो ब्राह्मणो ज्ञेयो वेदयोनिः स उच्यते}


\twolineshloka
{अग्न्याधेयं तु येनाथ वसन्ते क्रियतेऽनघ}
{तस्य श्रीर्बह्मवृद्धिश्च ब्राह्मणस्य विवर्धते}


\threelineshloka
{क्षत्रियस्याग्निराधेयो ग्रीष्मे श्रेष्ठः स वै नृप}
{येनाधानं तु वै ग्रीष्मे क्रियते तस्य वर्धते}
{श्रीः प्रजाःक पशवश्चैव वित्तं तेजो बलं यशः}


\twolineshloka
{शरदृतौ तु वैश्यस्य ह्याधानीयो हुताशनः}
{शरद्रात्रं स्वयं वैश्यो वैश्ययोनिः स उच्यते}


\twolineshloka
{शरद्याधानमेवं वै क्रियते येन पाण्डव}
{तस्यापि श्रीः प्रजाऽऽयुश्च पशवोऽर्थश्च वर्धते}


\twolineshloka
{पशवः सर्व एवैते त्रिभिः सर्वैरलङ्कृताः}
{अग्निहोत्रा ह्यवर्तन्ते तैरेव ध्रियते जगत्}


\twolineshloka
{ग्राम्यारण्याश्च पशवो पृक्षाश्चैव तृणानि च}
{फलान्योषधयश्चापि अग्निहोत्रकृते हविः}


\twolineshloka
{रसाः स्नेहास्तथा गन्धा रत्नानि मणयस्तथा}
{काञ्चनानि च लोहानि ह्यग्निहोत्रकृतेऽभवन्}


\twolineshloka
{आयुर्वेदो धनुर्वेदो मीमांसा न्यायविस्तरः}
{धर्मशास्त्रं च तत्सर्वमग्निहोत्रकृते कृतम्}


\twolineshloka
{छन्दः शिक्षा च कल्पश्च तथा व्याकरणानि च}
{शास्त्रं ज्योतिर्निरुक्तं चाप्यग्निहोत्रकृते कृतम्}


\twolineshloka
{इतिहासपुराणं च गाथाश्चोपनिषत्तथा}
{आथर्वणानि कर्माणि चाग्निहोत्रकृते कृतम्}


\twolineshloka
{यच्चैतस्यां पृथिव्यां वै किंचिदस्ति चराचरम्}
{तत्सर्वमग्निहोत्रस्य कृते सृष्टं स्वयंभुवा}


\twolineshloka
{अग्निहोत्रस्य दर्शस्य पूर्णमासस्य चाप्यथ}
{युपेष्टिपशुबन्धानां सोमपानक्रियावताम्}


\twolineshloka
{तिथिनक्षत्रयोगानां मुहूर्तकरणात्मकम्}
{कालस्य वेदनार्थं तु ज्योतिर्ज्ञानं पुराऽनघ}


\twolineshloka
{ऋग्यजुः साममन्त्राणां श्लोकतत्वार्थचिन्तनात्}
{प्रत्यापत्तिविकल्पानां छन्दोज्ञानं प्रल्पितम्}


\twolineshloka
{वर्णाक्षरपदार्थानां सन्धिलिङ्गं प्रकीर्तितम्}
{नामधातुविवेकार्थं पुरा व्याकरणं स्मृतम्}


\twolineshloka
{यूपवेद्यध्वरार्थं तु प्रोक्षणश्रवणाय तु}
{यज्ञदैवतयोगार्थं शिक्षाज्ञानं प्रकल्पितम्}


\twolineshloka
{यज्ञपात्रपवित्रार्थं द्रव्यसंभारणाय च}
{सर्वयज्ञविकल्पाय पुराकल्पं प्रकल्पितम्}


\twolineshloka
{नामधातुविकल्पानां तत्वार्थनियमाय च}
{सर्ववेदनिरुक्तानां निरुक्तमृषिभिः कृतम्}


\twolineshloka
{वेद्यर्थं पृथिवी सृष्टा संभारार्थ तथैव च}
{इध्मार्थमथ यूपार्थं ब्रह्मा चक्रे वनस्पतिम्}


\twolineshloka
{ग्राम्यारण्याश्च पशवो जायन्ते यज्ञकारणात्}
{मन्त्राणां विनियोगं च प्रोक्षितं श्रपणं तथा}


\twolineshloka
{अनुयाजप्रयाजाश्च मरुतां शंसिनस्तथा}
{औद्गात्रं चैव साम्नां वै प्रतिप्रस्थानमेव च}


\twolineshloka
{विष्णुक्रमाणां क्रमणं दक्षिणावभृथं तथा}
{त्रिकासमर्चनं चैव स्यानेषूपहृतं तथा}


\threelineshloka
{देवताग्रहणं मोक्षं हविषां श्रपणं तथा}
{नावबुद्ध्यन्ति ये विप्रा निन्दन्ति च पशोर्वधम्}
{ते यान्ति नरकं घोरं रौरवं तमसाऽऽवृतम्}


\twolineshloka
{शतवर्षसहस्राणि तत्र स्थित्वा नराधमाः}
{कृमिर्भिर्भक्ष्यमाणाश्च तिष्ठेयुः पूयशोणिते}


\threelineshloka
{वृक्षा यूपत्वमिच्छन्ति पशुत्वं पशवस्तथा}
{तृणानीच्छन्ति बर्हिष्ट्वमोषध्यश्च हविष्यताम्}
{सोमत्वं च लताः सर्वा वेदित्वं च वसुंधर}


\twolineshloka
{यस्मात्पशुत्वमिच्छन्ति पशवः स्वर्गलिप्सया}
{तस्मात्पशुवधे हिंसा नास्ति यज्ञेषु पाण्डव}


\twolineshloka
{यूपास्तन्मन्त्रसंस्कारैर्दर्भाश्च पशवस्तथा}
{यजमानेन सहिताः स्वर्गं यान्ति नरेश्वर}


\twolineshloka
{यावत्कालं हि यज्वा वै स्वर्गलोके महीयते}
{तावत्कालं प्रमोदन्ते पशवो ह्यध्वरे हताः}


\threelineshloka
{अहिंसा वैदिकं कर्म ब्रह्मकर्मेति तत्कृतम्}
{वेदोक्तं ये न कुर्वन्ति हिंसाबुद्ध्या क्रतून्द्विजाः}
{सद्यः शूद्रत्वामायान्ति प्रेत्य चण्डालतामपि}


\twolineshloka
{गावो यज्ञार्थमुत्पन्ना दक्षिणार्थं तथैव च}
{सुवर्णं रजतं चैव पात्रक्रुभार्थमेव च}


\twolineshloka
{दर्भाः संस्तरणार्तं तु रक्षसां रक्षणाय च}
{पजनार्थं द्विजाः सृष्टास्तारका दिवि देवताः}


\twolineshloka
{क्षत्रिया रक्षणार्थं तु वैश्या वार्तानिमित्ततः}
{शुश्रूषार्थं त्रयाणां वै शूद्राः सृष्टाः स्वयंभुवा}


\twolineshloka
{एवमेतज्जगत्सर्वमग्निहोत्रकृते कृतम्}
{नावबुध्यन्ति ये चैव नरास्तु तमसा वृताः}


\twolineshloka
{ते यान्ति नरकं घोरं रौरवं नाम विश्रुतम्}
{रौरवाद्विप्रयुक्तास्तु कृमियोनिं व्रजन्ति ते}


\twolineshloka
{यथोक्तमग्निहोत्राणां शुश्रूषन्ति च ये द्विजाः}
{तैर्दत्तं सहुतं चेष्टं दत्तमध्यापितं भवेत्}


\twolineshloka
{एवमिष्टं च पूर्तं च यद्विप्रैः क्रियते नृप}
{तत्सर्वं सम्यगाहृत्य चादित्ये स्थापयाम्यहम्}


\twolineshloka
{मया स्थापितमादित्ये लोकस्य सुकृतं हि तत्}
{धारयेद्यत्सहस्रांशुः सुकृतं ह्यग्निहोत्रिणाम्}


\twolineshloka
{यावत्कालं तु तिष्ठन्ति लोके चाप्यग्निहोत्रिणः}
{तावदेव हि पुण्येन दीप्यते रविणाऽम्बरे}


\threelineshloka
{स्वर्गे स्वर्गं गतानां तु वीर्याद्भवति वीर्यवान्}
{तत्र ते ह्युपभुञ्जन्ति ह्यग्निहोत्रस्य तत्फलम्}
{समानरूपा देवानां तिष्ठन्त्याबूतसंप्लवम्}


\twolineshloka
{वृथाऽग्निना च ये केचिद्दह्यन्ते ह्यग्निहोत्रिणः}
{न तेऽग्निहोत्रिणां लोके मानसाऽपि व्रजन्ति ते}


\twolineshloka
{वीरघ्रास्तु दुराचारा दरिद्रास्तु नराधमाः}
{विकला व्याधिताश्चापि जायन्ते शूद्रयोनिषु}


\twolineshloka
{तस्मादप्रोषितैर्नित्यमग्निहोत्रं द्विजातिभिः}
{होतव्यं विधिवद्राजन्नूर्ध्वामिच्छन्ति ये गतिम्}


\twolineshloka
{आत्मवन्नावमन्तव्यमग्निहोत्रं युधिष्टिर}
{न त्याज्यं क्षणिमप्येतदग्निहोत्रं युधिष्ठिर}


\twolineshloka
{वृद्धत्वेऽप्यग्निहोत्रं ये गृह्णन्ति विधिवद्द्विजाः}
{शूद्रान्नाद्विरता दान्ताः संयतेन्द्रियबुद्धयः}


\twolineshloka
{पञ्चयज्ञपरा नित्यं लोभक्रोधविवर्जिताः}
{द्विकालमतिथींश्चैव पूजयन्ति च भक्तितः}


\twolineshloka
{तेऽपि सूर्योदयप्रख्यैर्विमानैर्वायुवेगिभिः}
{मम लोके प्रमोदन्ते दृष्ट्वा मां च युधिष्ठिर}


\twolineshloka
{मन्वन्तरं च तत्रैकं मोदिता द्विजसत्तमाः}
{इह मानुष्यके लोके जायन्ते द्विजसत्तमाः}


\threelineshloka
{बालाहिताग्रयो ये च शूद्रान्नाद्विरताः सदा}
{क्रोधलोभविनिर्मुक्ताः प्रातस्स्नानपरायणाः}
{यथोक्तमग्निहोत्रं वै जुह्वते विजितेन्द्रियाः}


\twolineshloka
{आतिथेयाः सदा सौम्या द्विकालं मत्परायणाः}
{ते यान्त्यपुनरावृत्तिं भित्त्वा चादित्यमण्डलम्}


\twolineshloka
{मम लोकं सपत्नीका यानैः सूर्योदयप्रभैः}
{तत्र बालार्कसङ्काशाः कामगाः कामरूपिणः}


\twolineshloka
{ऐश्वर्यगुणसंपन्नाः क्रीडन्ति च यथासुखम्}
{इत्येषामाहिताग्नीनां विभूतिः पाण्डुनन्दन}


\threelineshloka
{श्रुतिं केचिन्निन्दमानाः श्रुतिं दूष्यन्त्यबुद्धयः}
{प्रमाणं न च कुर्वन्ति ये यान्तीहापि दुर्गतिम्}
{}


\twolineshloka
{प्रमाणमितिहासं च वेदान्कुर्वति ये द्विजाः}
{ते यान्त्यमरसायुज्यं नित्यमास्तिक्यबुद्धयः}


\chapter{अध्यायः ११२}
\threelineshloka
{चक्रायुध नमस्तेऽस्तु देवेश गरुडध्वज}
{चान्द्रायणविधिं पुण्यमाख्याहि भगवन्मम ॥भगवानुवाच}
{}


\twolineshloka
{शृणु पाण्डव तत्वेन सर्वपापप्रणाशनम्}
{पापिनो येन शुद्ध्यन्ति तत्ते वक्ष्यामि सर्वशः}


\twolineshloka
{ब्राह्मणः क्षत्रियो वाऽपि वैश्यो वा चरितव्रतः}
{यथावत्कर्तुकामो वै तस्यैवं प्रथमा क्रिया}


\twolineshloka
{शोधयेत्तु शरीरं स्वं पञ्चगव्येन यन्त्रितः}
{सशिरः कृष्णपक्षस्य ततः कुर्वीत वापनम्}


\twolineshloka
{शुक्लवासाः शुचिर्भूत्वा मौञ्जीं बध्नीत मेखलाम्}
{पालाशद्ण्डमादाय ब्रह्मचारिव्रते स्थितः}


\threelineshloka
{कृतोपलासः पूर्वं तु शुक्लप्रतिपदि द्विजः}
{नदीसंगमतीर्थेषु शुचौ देशे गृहेऽपि वा}
{गोमयेनोपलिप्रेऽथ स्थण्डिलेऽग्निं निधापयन्}


\twolineshloka
{आघारावाज्यभागौ च प्रणवं व्याहृतीस्तथा}
{वारुणं चैव पञ्चैव हुत्वा सर्वान्यथाक्रमम्}


\twolineshloka
{सत्याय विष्णवे चेति ब्रह्मर्षिभ्योऽथ ब्रह्मणे}
{विश्वेभ्यो हि च देवेभ्यः सप्रजापतये तथा}


\twolineshloka
{षडुक्त्वा जुहुयात्पश्चात्प्रायश्चित्ताहुतिं द्विजः}
{अतः समापयेदग्निं शान्तिं कृत्वाऽथ पौष्टिकिम्}


\threelineshloka
{प्रणम्य् चाग्निं सोमं च भस्म दिग्ध्वा यथाविधि}
{नदीं गत्वा विशुद्धात्मा सोमाय वरुणाय च}
{आदित्याय नमस्कृत्वा ततः स्नायात्समाहितः}


\twolineshloka
{उत्तीर्योदकमाचम्य चासीनः पूर्वतोमुखः}
{प्राणायामं ततः कृत्वा पवित्रैरभिषेचनम्}


\twolineshloka
{आचान्तस्त्वमिवीक्षेत ऊर्ध्वबाहुर्दिवाकरम्}
{कृताञ्जलिपुटः स्थित्वा कुर्याच्चैव प्रदक्षिणम्}


\twolineshloka
{नारायणं वा रुद्रं वा ब्रह्मणमथवाऽपि च}
{वारुणं मन्त्रमूक्तं वा प्राग्भोजनमथापि वा}


\threelineshloka
{वीरघ्नमृषभफं वाऽपि तथा चाप्यघमर्षणम्}
{गायत्रीं मम देवीं वा सावित्रीं वा जपेत्ततः}
{शतं वाऽष्टशतं वाऽपि सहस्रमथवा परम्}


\twolineshloka
{ततो मध्याह्नकाले वै पायसं यावकं हि वा}
{पाचयित्वा प्रयत्नेन प्रयतः सुसमाहितः}


\threelineshloka
{पात्रं तु सुसमादाय सौवर्णं राजतं तु वा}
{ताम्रं वा मृन्मयं वापि औदुंबरमथापि वा}
{}


\twolineshloka
{वृक्षाणां याज्ञियानं तु पर्णैरार्द्रैरकुत्सितैः}
{पुटकेन तु गुप्तेनि चरेद्भैक्षं समाहितः}


\twolineshloka
{ब्राह्मणानां गृहाणां तु सप्तानां नापरं व्रजेत्}
{गोदोहमात्रं तिष्ठित्तु वाग्यतः संयतेन्द्रियः}


% Check verse!
न हसेन्न च वीक्षेत नाभिभाषेत वा स्त्रियम्
\twolineshloka
{दृष्ट्वा मूत्रं पुरीषं वा चण्डालं वा रजस्वलाम्}
{पतितं च तथा श्वानमादित्यमवलोकयेत्}


\twolineshloka
{यो हि पादुकमारुह्य सर्वदा प्रचरेद्द्विजः}
{तं दृष्ट्वा पापकर्माणमादित्यमवलोकयेत्}


\threelineshloka
{ततस्त्वावसथं प्राप्तो भिक्षां निक्षिप्य भूतले}
{प्रक्षाल्य पादावाजान्वोर्हस्तावाकूर्परं पुनः}
{आचम्य वारिणा तेन वह्निं विप्रांश्च पूययेत्}


\twolineshloka
{पञ्च सप्ताथवा कुर्याद्भागान्भैक्षस्य तस्य वै}
{तेषामन्यतमं पिण्डमादित्याय निवेदयेत्}


\twolineshloka
{ब्रह्मणे चाग्नये चैव सोमाय वरुणाय च}
{विश्वेभ्यश्चैव देवेभ्यो दद्यादन्नं यथाक्रमम्}


% Check verse!
अवशिष्टमथैकं तु वक्त्रमात्रं प्रकल्पयेत्
\threelineshloka
{अङ्गुल्यग्रे स्थितं पिण्डं गायत्र्या चाभिमन्त्रयेत्}
{अङ्गुलीभिस्त्रिभिः पिण्डं प्राश्नीयात्प्राङ्मुखःकशुचिः}
{}


\twolineshloka
{यथा च वर्धते सोमो ह्रसते च यथा पुनः}
{तथा पिण्डाश्च वर्धन्ते ह्रसन्ते च दिनेदिने}


\twolineshloka
{त्रिकालं स्नानमस्योक्तं द्विकालमथवा सकृत्}
{ब्रह्मचारी सदा वाऽपि न च वस्त्रं प्रपीडयेत्}


\twolineshloka
{स्थान न दिवसं तिष्ठेद्रात्रौ वीरासनं व्रजेत्}
{भवेत्स्थण्डिलशायी वाऽप्यथा वृक्षमूलिकः}


\twolineshloka
{वल्कलं यदि वा क्षौमं शाणं कार्पासकं तथा}
{आच्छादनं भवेत्तस्य वस्त्रार्थं पाण्डुनन्दन}


\twolineshloka
{एवं चान्द्रायणे पूर्णे मासस्यान्ते प्रयत्नवान्}
{ब्राह्मणान्भोजयेद्भक्त्या दद्याच्चैव च दक्षिणाम्}


\twolineshloka
{चान्द्रायणेन चीर्णेन यत्कृतं तेन दुष्कृतम्}
{तत्सर्वं तत्क्षणादेव भस्मीभवति काष्ठवत्}


\twolineshloka
{ब्रह्महत्या च गोहत्या सुवर्णस्तैन्यमेव च}
{भ्रूणहत्या सुरापानं गुरोर्दारव्यतिक्रमः}


\twolineshloka
{एवमन्यानि पापानि पातकीयानि यानि च}
{चान्द्रायणेन नश्यन्ति वायुना पांसवो यथा}


\twolineshloka
{अनिर्दशाया गोः क्षीरमौष्ठ्रमाविकमेव च}
{मृतसूतकयोश्चान्नं भुक्त्वा चान्द्रायणं चरेत्}


\twolineshloka
{उपपातकिनश्चान्नं पतितान्नं तथैव च}
{शूद्रस्योच्छेषणं चैव भुक्त्वा चान्द्रायणं चरेत्}


\twolineshloka
{आकाशस्थं तु हस्तस्थमधः स्रस्तं तथैव च}
{परहस्तस्थितं चैव भुक्त्वा चान्द्रायणं चरेत्}


\twolineshloka
{अथाग्रेधिषोरन्नं दिधिषूपपतेस्तता}
{परिवेत्तुस्तथा चान्नं परिवित्तान्नमेव च}


\twolineshloka
{कुण्डान्नं गोलकान्नं च देवलान्नं तथैव च}
{तथा पुरोहितस्यान्नं भुक्त्वा चान्द्रायणं चरेत्}


\twolineshloka
{सुराऽऽसवं विषं सर्पिर्लाक्षा लवणमेव च}
{तैलं चापि च विक्रीणन्द्विजश्चान्द्रायणं चरेत्}


\twolineshloka
{एकोद्दिष्टं तु यो भुङ्क्ते जनमध्यगतोऽपि यः}
{भिन्नभाण्डेषु यो भुङ्क्ते द्विजश्चान्द्रायणं चरेत्}


\twolineshloka
{यो भुङ्क्तेऽनुपनीतेन यो भुङ्क्ते च स्त्रिया सह}
{कन्यया सह यो भुङ्क्ते द्विजश्चान्द्रायणं चरेत्}


\twolineshloka
{उच्छिष्टं स्थापयेद्विप्रो यो मोहाद्भोजनान्तरे}
{दद्याद्वा यदि वा मोहाद्विजश्चान्द्रायणं चरेत्}


\twolineshloka
{तुम्बकोशातकं चैव पलाण्डुं गृञ्जनं तथा}
{छत्राकं लशुनं चैव भुक्त्वा चान्द्रायणं चरेत्}


\twolineshloka
{द्विजः पर्युषितं चान्नं पक्वं परगृहागतम्}
{विपक्वं च तथा मांसं भुक्त्वा चान्द्रायणं चरेत्}


\twolineshloka
{उदक्यया शुना वाऽपि चण्डालैर्वा द्विजोत्तमः}
{दृष्टमन्नं तु भुञ्जानो द्विजश्चान्द्रायणं चरेत्}


\twolineshloka
{****तत्पुरा विशुद्ध्यर्थमृषिभिश्चरितं व्रतम्}
{पावनं सर्वभूतानां पुण्यं पाण्डवचोदितम्}


\twolineshloka
{एतेन वसवो रुद्राश्चादित्याश्च दिवं गताः}
{एतदद्य परं गुह्यं पवित्रं पापनाशनम्}


% Check verse!
यथोक्तमेतद्यः कुर्याद्द्विजः पापप्रणाशनम् ॥स दिवं याति पूतात्मा निर्मलादित्यसंनिभः
\chapter{अध्यायः ११३}
\threelineshloka
{केशवेनैवमुक्ते तु चान्द्रायणविधिक्रमे}
{अपृच्छत्पुनरन्यांश्च धर्मान्धर्मात्मजो नृप ॥युधिष्ठिर उवाच}
{}


\threelineshloka
{सर्वभूतपते श्रीमन्सर्वभूतनमस्कृत}
{सर्वभूतहितं धर्मं सर्वज्ञ कथयस्व नः ॥भगवानुवाच}
{}


\twolineshloka
{यद्दरिद्रजनस्यापि स्वर्ग्यं सुखकरं भवेत्}
{सर्वपापप्रशमनं तच्छृणुष्व युधिष्टिर}


\twolineshloka
{कार्तिकाद्यास्तु ये मासा द्वादशैव प्रकीर्तिताः}
{तेष्वेकबुक्तनिमः सर्वेषामुच्यते मया}


\twolineshloka
{कार्तिके यस्तु वै मासे नन्दायां संयतेन्द्रियः}
{एकभुक्तेन मद्भक्तो मासमेकं तु वर्तते}


\twolineshloka
{जलं वा न पिबेन्मासे नान्तरं भोजनात्परम्}
{आदित्यरूपं मां नित्यमर्चयन्सुसमाहितः}


\twolineshloka
{व्रतान्ते भोजयेद्विप्रान्दक्षिणां सम्प्रदाय च}
{क्रोधलोभविनिर्मुक्तस्तस्य पुण्यफलं शृणु}


\threelineshloka
{विधिवत्कपिलादाने यत्पुण्यं समुदाहृतम्}
{तत्पुण्यं समनुप्राप्य सूर्यकलोके महीयते}
{ततश्चापि च्युतः कालात्पुरुषेषूपजायते}


\twolineshloka
{मार्गशीर्षं तु यो मासमेकभुक्तेन वर्तते}
{कामं क्रोधं च लोभं च परित्यज्य यथाविधि}


\twolineshloka
{स्नात्वा चादित्यरूपं मामर्ययेन्नियतेन्द्रियः}
{जपेच्चैव च गायत्रीं मामिकां वाग्यतेन्द्रियः}


\twolineshloka
{मासे परिसमाप्ते तु भोजयित्वा द्विजाञ्शुचिः}
{तानर्चयति मद्भक्त्या तस्य पुण्यफलं शृणु}


\threelineshloka
{अग्निहोत्रे कृते पुण्यमाहिताग्नेस्तु यद्भवेत्}
{तत्पुण्यं फलमासाद्य योनेनांऽबरशोभिना}
{सह सप्तर्षिलोकेषु यथाकामं यथासुखम्}


\twolineshloka
{ततश्चापि च्युतः कालाद्धरिवर्षेषु जायते}
{तत्र प्रकामं क्रीडित्वा राजा पश्चाद्भविष्यति}


\twolineshloka
{******** क्षिपेदेवमेकभुक्तेन यो नरः}
{अर्चयन्नेव मां नित्यं मद्गतेनान्तरात्मना}


\twolineshloka
{अहिंसासत्यसहितः क्रोधहर्षविवर्जितः}
{एवं युक्तस्य राजेन्द्र शृणु यत्फलमुत्तमम्}


\twolineshloka
{विप्रातिथ्यसहस्रेषु यत्पुण्यं समुदाहृतम्}
{तत्पुण्यं समनुप्राप्य शक्रलोके महीयते}


\twolineshloka
{अवतीर्णस्ततः कालादिलावर्षेषु जायते}
{तत्र स्थित्वा चिरं कालमस्मिन्विप्रो भविष्यति}


\threelineshloka
{माघमासं सदा यस्तु वर्तते चैकभुक्ततः}
{मदर्चनपरो भूत्वा डंभक्तोधविवर्जितः}
{मामिकामपि सावित्रीं सन्ध्यायां तु जपेद्बुधः}


\threelineshloka
{दत्त्वा दु दक्षिणामन्ते भोजयित्वा द्विजानपि}
{नमिस्करोति तान्भक्त्या मद्गतेनान्तरात्मना}
{त्रिकालस्नानयुक्तस्य तस्य पुण्यफलं शृणु}


\twolineshloka
{नीलकण्ठप्रयुक्तेन योनेनांऽबरशोभिना}
{पितृलोकं व्रजेच्छ्रीमान्सेव्यमानोप्सरोणैः}


\twolineshloka
{तत्र प्रकामं क्रीडित्वा भद्राश्वेषूपजायते}
{ततश्च्युतश्चतुर्वेदी विप्रो भवति भूतले}


\twolineshloka
{यश्चरेत्फाल्गुनं मासमेकभुक्तो जितेन्द्रियः}
{नमो ब्रह्मण्यदेवायेत्येतन्मन्त्रं जपेत्सदा}


\twolineshloka
{पायसं भोजयेद्विप्रान्व्रतान्ते संयतेन्द्रियः}
{मदर्चनपरोऽक्रोधस्तस्य पुण्यफलं शृणु}


\twolineshloka
{विमानं सारसैर्युक्तमारूढः कामगामि च}
{नक्षत्रलोके रमते नक्षत्रसदृशाकृतिः}


\twolineshloka
{ततश्चापि च्युतः कालात्केतुमालेषु जायते}
{तत्र प्रकामं क्रीडित्वा मानुषेषु मुनिर्भवेत्}


\twolineshloka
{चैत्रमासं तु यो राजन्नेकभुक्तेन वर्तते}
{ब्रह्मचारी च मद्भक्त्या तस्य पुण्यफलं शृणु}


\twolineshloka
{यदग्निहोत्रिणः पुण्यं यथोक्तं व्रतचारिणः}
{तत्पुण्यफकलमासाद्य चन्द्रलोके महीयते}


\twolineshloka
{ततोऽवतीर्णो जायेत वर्षे रमणके पुनः}
{भुक्त्वा कामांस्ततस्तस्मिन्निह राजा भविष्यति}


\twolineshloka
{वैशाखे यस्तु मासे वै ह्येकभुक्तेन वर्तते}
{द्विजमग्रासने कृत्वा भुञ्जन्भूमौ च वाग्यतः}


\twolineshloka
{नमो ब्रह्मण्यदेवायेत्यर्चयित्वा दिवाकरम्}
{व्रतान्ते भोजयेद्विप्रांस्तस्य पुण्यफलं शृणु}


\twolineshloka
{फलं यद्विधिवत्प्रोक्तमग्निष्टोमातिरात्रयोः}
{तत्पुण्यफलमासाद्य देवलोके महीयते}


\twolineshloka
{ततो हैमवते वर्षे जायते कालपर्ययात्}
{तत्र प्रकामं क्रीडित्वा विप्रः पश्चाद्भविष्यति}


\twolineshloka
{ज्येष्ठमासं तु यो विप्रो ह्येकभुक्तेन वर्तते}
{विप्रमग्रासने कृत्वा भूमौ भुञ्जन्यथाविधि}


\twolineshloka
{नमो ब्रह्मण्यदेवायेत्युच्चरन्मां समाहितः}
{डंभानृतविनिर्मुक्तस्तस्य पुण्यफलं शृणु}


\twolineshloka
{चीर्णे चान्द्रायणे सम्यग्यत्पुण्यं समुदाहृतम्}
{तत्पुण्यफलमासाद्य देवलोके महीयते}


\twolineshloka
{अथोत्तरकुरौ वर्षे जायते निर्गतस्ततः}
{ततश्चापि च्युतः कालादिह लोके द्विजो भवेत्}


\twolineshloka
{आषाढमासं यो राजन्नेकभुक्तेन वर्तते}
{ब्रह्मचारी जितक्रोधो मदर्चनपरायणः}


\twolineshloka
{विप्रमग्रासने कृत्वा भूमौ भुञ्जञ्जितेन्द्रियः}
{कृत्वा त्रिषवणं स्नानमष्टाक्षरविधानतः}


\twolineshloka
{व्रतान्ते भोजयेद्विद्वान्पायसेन युधिष्ठिर}
{गुडोदनेन राजेन्द्र तस्य पुण्यफलं शृणु}


\twolineshloka
{कपिलाशतदानस्य यत्पुण्यं पाण्डुनन्दन}
{तत्पुण्यफलमासाद्य देवलोके महीयते}


\twolineshloka
{ततोऽवतीर्णः काले तु शाकद्वीपे तु जायते}
{ततश्चापि च्युतः कालादिह विप्रो भविष्यति}


\threelineshloka
{श्रावणं यः क्षिपेन्मासमेकभुक्तेन वर्तते}
{नमो ब्रह्मण्यदेवायेत्युक्त्वा मामर्चयेत्सदा}
{विप्रमाग्रासने कृत्वा भूमौ भुञ्जन्यथाविधि}


\twolineshloka
{पायसेनार्चयन्विप्राञ्जितक्रोधो जितेन्द्रियः}
{लोभमोहविनिर्मुक्तस्तस्य पुण्यफलं शृणु}


\twolineshloka
{कपिलादानस्य यत्पुण्यं विधिदत्तस्य पाण्डव}
{तत्पुण्यं सम***प्राप्य शक्रलोके महीयते}


\twolineshloka
{ततश्चापि च्युत कालात्कुशद्वीपे प्रजायते}
{तत्रि प्रकामं क्रीडित्वा विप्रो भवति मानुषे}


\twolineshloka
{यस्तु भाद्रपदं मासमेकभुक्तेन वर्तते}
{ब्रह्मचारी जितक्रोधः सत्यसन्धो जितेन्द्रियः}


\twolineshloka
{विप्रमग्रासने कृत्वा पाकभेदविवर्जितः}
{नमो ब्रह्मण्यदेवायेत्युक्त्वाऽस्य चरणौ स्पृशेत्}


\twolineshloka
{तिलानपि घृतं वाऽपि व्रतान्ते दक्षिमां ददत्}
{मद्भक्तस्य नरश्रेष्ठ तस्य पुण्यफलं शृणु}


\twolineshloka
{यत्फलं विधिवत्प्रोक्तं राजसूयाश्वमेधयोः}
{तत्पुण्यफलमासाद्य शक्रलोके महीयते}


\twolineshloka
{ततश्चापि च्युतः कालाज्जायते धनदालये}
{तत्र प्रकामं क्रीकडित्वा राजा भवति मानुषे}


\threelineshloka
{यश्चाप्याश्वयुजं मासमेकभुक्तेन वर्तते}
{मद्गायत्रीं जपेद्भक्त्या मद्गतेनान्तरात्मना}
{द्विसन्ध्यं वा त्रिसन्ध्यं वा शतमष्टोत्तरं तु वा}


\twolineshloka
{विप्रमग्रासने कृत्वा संयतेन्द्रियमानसः}
{व्रतान्ते भोजयेद्विप्रांस्तस्य पुण्यफलं शृणु}


\twolineshloka
{अश्वमेधस्य यत्पुण्यं विधिवत्पाण्डुनन्दन}
{तत्पुण्यफलमासाद्य मम लोके महीयते}


\twolineshloka
{ततश्चापि च्युतः कालाच्छ्वेतद्वीपे प्रजायते}
{तत्र भुक्त्वा च भोगांश्च ततो विप्रवरो भवेत्}


\chapter{अध्यायः ११४}
\threelineshloka
{एवं संवत्सरं पूर्णमेकभुक्तेन यः क्षिपेत्}
{तस्य पुण्यफलं यद्वै तन्ममाचक्ष्व केशव ॥भगवानुवाच}
{}


\twolineshloka
{शृणु पाण्डव तत्त्वं मे वचनं पुण्यमुत्तमम्}
{यदकृत्वाऽथवा कृत्वा नरः पापैः प्रमुच्यते}


\twolineshloka
{एकभुक्तेन वर्तेत नरः संवत्सरं तु यः}
{ब्रह्मचारी जितक्रोधो ह्यधश्शायी जितेन्द्रियः}


\threelineshloka
{शुचिश्चि स्नानतो व्यग्रः सत्यवागनसूयकः}
{अर्चन्नेव तु मां नित्यं मद्गतेनान्तरात्मना}
{सन्ध्ययोस्तु जपेन्नित्यं मद्गायत्रीं समाहितः}


\twolineshloka
{नमो ब्रह्मण्यदेवायेत्यसकृन्मां प्रणम्य च}
{विप्रमग्रासने कृत्वा यावकं भैत्रमेव वा}


\twolineshloka
{भुक्त्वा तु वाग्यतो भूमावाचान्तस्य द्विजन्मनः}
{नमोऽस्तु वासुदेवायेत्युक्त्वा तु चरणौ स्पृशेत्}


\twolineshloka
{मासेमासे समाप्ते तु भोजयित्वा द्विजाञ्शुचीन्}
{संवत्सरे ततः पूर्णे दद्यात्तु व्रतदक्षिणाम्}


\threelineshloka
{नवनीतमयीं गां वा तिलधेनुमथापि वा}
{विप्रहस्तच्युतैस्तोयैः सहिरण्यैः समुक्षितः}
{तस्य पुण्यफलं राजन्कथ्यमानं मया शृणु}


\threelineshloka
{दशजन्मकृतं पापं ज्ञानतोऽज्ञानतोपि वा}
{तद्विनश्यति तस्याशु नात्र कार्या विचारणा ॥युधिष्ठिर उवाच}
{}


\threelineshloka
{सर्वेषामुपवासानां यच्छ्रेयः सुमहत्फलम्}
{यच्च निःश्रेयसं लोके तद्भवान्वक्तुमर्हति ॥भगवानुवाच}
{}


\twolineshloka
{शृणु राजन्यथापूर्वं मयाऽभीष्टं तु मोदते}
{तथा ते कथयिष्यामि मद्भक्ताय युधिषठिर}


\fourlineindentedshloka
{यस्तु भक्त्या शुचिर्भूत्वा पञ्चम्यां मे नराधिप}
{उपवासव्रतं कुर्यात्त्रिकालं चार्चयंस्तु माम्}
{सर्वक्रतुफलं लब्ध्वा मम लोके महीयते ॥युधिष्ठिर उवाच}
{}


\threelineshloka
{भगवन्देवदेवेश पञ्चमी नाम का तव}
{तामहं श्रोतुमिच्छामि कथयस्व ममानघ ॥भगवानुवाच}
{}


\twolineshloka
{पर्वद्वयं च द्वादश्यां श्रवणं च नराधिप}
{मत्पञ्चमीति विख्यातां मत्प्रिया च विशेषतः}


\twolineshloka
{तस्मात्तु ब्राह्मणश्रेष्ठैर्मन्निवेशितबुद्धिभिः}
{उपवासस्तु कर्तव्यो मत्प्रियार्तं विशेषतः}


\twolineshloka
{द्वादश्यामेव वा कुर्यादुपवासमशक्नुवन्}
{तेनाहं परमां प्रीति यास्यामि नरपुङ्गव}


\twolineshloka
{अहोरात्रेण द्वादश्यां मार्गशीर्षेण केशवम्}
{उपोष्य पूजयेद्यो मां सोऽस्वमेधफलं लभेत्}


\twolineshloka
{द्वादश्यां पुष्यमासे तु नाम्ना नारायणं तु माम्}
{उपोष्य पूजयेद्यो मां वाजिमेधफलं लभेत्}


\twolineshloka
{द्वादश्यां माघमासे तु मामुपोष्य तु माधवम्}
{पूजयेद्यः समाप्नोति राजसूयफलं नृप}


% Check verse!
द्वादश्यां फाल्गुने मासि गोविन्दाख्यमुपोष्य माम् ॥पूजयेद्यः समाप्नोति ह्यतिरात्रफलं नृप
\twolineshloka
{द्वादश्यां मासि चैत्रे तु मां विष्णुं समुपोष्य यः}
{पूजयंस्तदवाप्नोति पौण्डरीकस्य यत्फलम्}


\twolineshloka
{द्वादश्यां मासि वैशाखे मधुसूदनसंज्ञितम्}
{उपोष्य पूजयेद्यो मां सोग्निष्टोमस्य पाण्डव}


\twolineshloka
{द्वादश्यां ज्येष्ठमासे तु मामुपोष्य त्रिविक्रमम्}
{अर्ययेद्यः समाप्नोति गवां मेधफलं नृप}


\twolineshloka
{आषाढे वामनाख्यं मां द्वादश्यां समुपोष्य यः}
{नरमेधस्य स फलं प्राप्नोति भरतर्षभ}


\twolineshloka
{द्वादश्यां श्रावणे मासि श्रीधराख्यमुपोष्य माम्}
{पूजयेद्य समाप्नोति पञ्चयज्ञफलं नृप}


\twolineshloka
{मासे भाद्रपदे यो मां हृषीकेशाख्यमर्चयेत्}
{उपोष्य स समाप्नोति सौत्रामणिफलं नृप}


\twolineshloka
{द्वादश्यामाश्वयुङ्मासे पद्मनाभमुपोष्य माम्}
{अर्चयेद्यः समाप्नोति गोसहस्रफलं नृप}


\twolineshloka
{द्वादश्यां कार्तिके मासि मां दामोदरसंज्ञितम्}
{उपोष्य पूजयेद्यस्तु सर्वक्रतुफलं नृप}


\twolineshloka
{केवलेनोपवासेन द्वादश्यां पाण्डुनन्दन}
{यत्फलं पूर्वमुद्दिष्टं तस्यार्धं लभते नृप}


\twolineshloka
{श्रावणेऽप्येवमेवं मामर्चयेद्भक्तिमान्नरः}
{मम सालोक्यमाप्नोति नात्र कार्या विचारणा}


\twolineshloka
{मासेमासे समभ्यर्च्य क्रमशो मामतन्द्रितः}
{पूर्मे संवत्सरे कुर्यात्पुनः संवत्सरं तु माम्}


\twolineshloka
{अविघ्नमर्चयानस्तु यो मद्भक्तो मत्परायणः}
{अविघ्नमर्चयानस्तु मम सायुज्यमाप्नुयात्}


\twolineshloka
{अर्चयेत्प्रीतिमान्यो मां द्वादस्यां वेदसंहिताम्}
{स पूर्वोक्तफलं राजँल्लभते नात्र संशयः}


\twolineshloka
{गन्धं पुष्पं फलं तोयं पत्रं वा मूलमेव वा}
{द्वादश्यां मम यो दद्यात्तत्समो नास्ति मत्प्रियः}


\threelineshloka
{एतेन विधिना सर्वे देवाः शक्रपुरोगमाः}
{मद्भक्ता नरशार्दूल स्वर्गलोकं तु भुञ्जते ॥वैशम्पायन उवाच}
{}


\twolineshloka
{एवं वदति देवेशे केशवे पाडुनन्दनः}
{कृताञ्जलिः स्तोत्रमिदं भक्त्तया धर्मात्मजोऽब्रवीत्}


\twolineshloka
{सर्वलोकेश देवेश हृषीकेशक नमोस्तु ते}
{सहस्रशिरसे नित्यं सहस्राक्ष नमोस्तु ते}


\twolineshloka
{त्रयीमय त्रयीनाथ त्रयीस्तुत नमोनमः}
{यज्ञात्मन्यज्ञसंभूत यज्ञनाथ नमोनमः}


\twolineshloka
{चतुर्मूर्ते चतुर्बाहो चतुर्व्यूह नमोनमः}
{लोकात्मँल्लोककृन्नाथ लोकावास नमोनमः}


\twolineshloka
{सृष्टिसंहारकर्त्रे तु नरसिंह नमोनमः}
{भक्तप्रिय नमस्तेऽस्तु कृष्ण नाथ नमोनमः}


\twolineshloka
{लोकप्रिय नमस्तेऽस्तु भक्तवत्सल ते नमः}
{ब्रह्मवास नमस्तेऽस्तु ब्रह्मनाथ नमोनमः}


\twolineshloka
{रुद्ररूप नमस्तेऽस्तु रुद्रकर्मिरताय ते}
{पञ्चयज्ञ नमस्तेऽस्तु सर्वयज्ञ नमोनमः}


\twolineshloka
{कृष्णप्रिय नमस्तेऽस्तु कृष्णनाथ नमोनमः}
{योगिप्रिय नमस्तेऽस्तु योगिनाथ नमोनमः}


\threelineshloka
{हयवक्त्र नमस्तेऽस्तु चक्रपाणे नमोनमः}
{पञ्चभूत नमस्तेऽस्तु पञ्चायुध नमोनमः ॥वैशम्पायन उवाच}
{}


\twolineshloka
{भक्तिगद्गदया वाचा स्तुवत्येवं युधिष्ठिरे}
{गृहीत्वा केशवो हस्ते प्रीतात्मा तं न्यवारयत्}


\threelineshloka
{निवार्य च पुनर्वाचा भक्तिनम्रं युधिष्ठिरम्}
{वक्तुमेव नरश्रेषठ धर्मपूत्रं प्रचक्रमे ॥भगवानुवाच}
{}


\threelineshloka
{अन्यवत्किमिदं राजन्मां स्तौषि नरपुङ्गव}
{तिष्ठ पृच्छ यथापूर्वं धर्मपूत्र युधिष्ठिर ॥युधिष्ठिर उवाच}
{}


\twolineshloka
{भगवंस्त्वत्प्रसादात्तु धर्मं स्मृत्वा पुनःपुनः}
{न शान्तिरस्ति मे देव नृत्यतीव च मे मनः}


\threelineshloka
{इदं च धर्मसंपन्नं वक्तुमर्हसि माधव}
{कृष्णपक्षेषु द्वादश्यामर्चनीयः कथं भवेत् ॥भगवानुवाच}
{}


\twolineshloka
{शृणु राजन्यथापूर्वं तत्सर्वं कथयामि ते}
{परमं कृष्णद्वादश्यामर्चनायां फलं मम}


\twolineshloka
{एकादश्यामुपोष्याथ द्वादश्यामर्चयेत्तु माम्}
{विप्रानपि यथालाभं पूजयेद्भक्तिमान्नरः}


\twolineshloka
{स गच्छेद्दक्षिणामूर्तिं मां वा नात्र विचारणा}
{चन्द्रसालोक्यमथवा ग्रहनक्षत्रपूजितः}


\chapter{अध्यायः ११५}
\threelineshloka
{केशवेनैवमाख्याते धर्मपुत्रः पुनःपुनः}
{पप्रच्छ दानकालस्य विशेषं च विधिं नृप ॥युधिष्ठिर उवाच}
{}


\threelineshloka
{देव किं फलमाख्यातं विषुवेष्मवरेश्वर}
{सूर्येन्दूपप्लवे चैव वस्तुमर्हति तत्फलम् ॥भगवानुवाच}
{}


\twolineshloka
{शृणुष्व राजन्विषुवे सोमार्कग्रहणेषु च}
{व्यतीपातेऽयने चैव दानं स्यादक्षयं फलम्}


\twolineshloka
{राजन्नयनयोर्मध्ये विषुवं सम्प्रचक्षते}
{समे रात्रिदिने तत्र सन्ध्यायां विषुवे नृप}


\twolineshloka
{ब्रह्माऽहं शङ्करश्चापि तिष्ठामः सहिताः सकृत्}
{क्रियाकरणकार्याणामेकीभावत्वकारणात्}


\twolineshloka
{अस्माकमेकीभूतानां निष्कलं परमं पदम्}
{तन्मुहूर्तं परं पुण्यं राजन्विषुवसंज्ञितम्}


\twolineshloka
{तदेवाद्यक्षरं ब्रह्म परं ब्रह्मेति कीर्तितम्}
{तस्मिन्मुहूर्ते सर्वे तु चिन्तयन्ति परं पदम्}


\twolineshloka
{देवाश्च वसवो रुद्राः पितरश्चाश्विनौ तथा}
{साध्याश्च विश्वे गन्धर्वाः सिद्धा ब्रह्मर्षयस्तथा}


\twolineshloka
{सोमादयो ग्रहाश्चैव सरितः सागरास्तथा}
{मरुतोत्सरसो नागा यक्षराक्षसगुह्यकाः}


\twolineshloka
{एते चान्ये च राजेन्द्र विषुवे संयतेन्द्रियाः}
{सोपवासाः प्रयत्नेन भवन्ति ध्यानतत्पराः}


\threelineshloka
{अन्नं गावस्तिलान्भूमिं कन्यादानं तथैव च}
{गृहमायतनं धान्यं वाहनं शयनं तथा}
{}


\twolineshloka
{यच्चान्यच्च मया प्रोक्तं तत्प्रयच्छ युधिष्ठिर}
{दीयते विषुवेष्वेवं श्रोत्रियेभ्यो विशेषतः}


\twolineshloka
{तस्य दानस्य कौन्तेय क्षयं नैवोपपद्यते}
{वर्धतेऽहरहः पुण्यं तद्दानं कोटिसंमितम्}


\twolineshloka
{विषुवे स्नपनं यस्तु मम कुर्याद्धरस्य वा}
{अर्चनां च यतान्यायं तस्य पुण्यफलं शृणु}


\twolineshloka
{दशजन्मकृतं पापं तस्य सद्यो विनश्यति}
{दशानामश्वमेधानामिष्टानां लभते फलम्}


\twolineshloka
{विमानं दिव्यमारूढः कामरूपी यथासुखम्}
{स याति मामकं लोकं रुद्रलोकमथापि वा}


\twolineshloka
{तत्रस्थैर्देवगन्धर्वेर्गीयमानो यथासुखम्}
{दिव्यवर्षसहस्राणि कोटिमेकं तु मोदते}


\twolineshloka
{ततश्चापि च्युतः कालादिह लोके द्विजोत्तमः}
{चतुर्णामपि वेदानां पारगो ब्रह्मिविद्भवेत्}


\twolineshloka
{चन्द्रसूर्यग्रहे व्योम्नि मम वा शङ्करस्य वा}
{गायत्रीं मामिकां वाऽपि जपेद्यः शङ्करस्य वा}


\twolineshloka
{शङ्खर्तूर्यस्वनैश्चैव कांस्यघण्टास्वनैरपि}
{कारयेत्तु ध्वनिं भक्त्या तस्य पुण्यफलं शृणु}


\twolineshloka
{गान्धर्वैर्होमजप्यैश्च जप्तैरुत्कृष्टनामभिः}
{दुर्बलोपि भवेद्राहुः सोमश्च बलवान्भवेत्}


\twolineshloka
{सूर्येन्दूपप्लवे चैव श्रोत्रियेभ्यः प्रदीयते}
{तत्सहस्रगुणं भूत्वा दांतारमुपतिष्ठति}


\twolineshloka
{महापातकयुक्तोपि यद्यपि स्यान्नरोत्तम}
{निष्पापस्तत्क्षणादेव तेन दानेन जायते}


\twolineshloka
{चन्द्रसूर्यप्रकाशेन विमानेन विराजता}
{याति सोमपुरं रम्यं सेव्यमानोप्सरोगणैः}


\twolineshloka
{यावदृक्षाणि तिष्ठन्ति गगने शशिना सह}
{तावत्कालं स रजेन्द्र सोमलोके महीयते}


\threelineshloka
{ततश्चापि च्युतः कालादिह लोके युधिष्ठिर}
{वेदवेदाङ्गविद्विप्रः कोटीधनपतिर्भवेत् ॥युधिष्ठिर उवाच}
{}


\threelineshloka
{भगवंस्तव गायत्री जप्यते च कथं विभो}
{किं वा तस्य फळं देव ममाचक्ष्व सुरेश्वर ॥भगवानुवाच}
{}


\twolineshloka
{द्वादश्यां विषुवे चैव चन्द्रसूर्यग्रहे तथा}
{अयने श्रवणे चैव व्यतीपाते तथैव च}


\fourlineindentedshloka
{अश्वत्तदर्शने चैव तथा मद्दर्शनेऽपि च}
{जप्या तु मम गायत्री चाथवाऽष्टाक्षरं नृप}
{आर्जितं दुष्कृतं तस्य नाशयेन्नात्र संशयः ॥युधिष्ठिर उवाच}
{}


\threelineshloka
{अश्वत्थदर्शनं चैव किं त्वद्दर्शनसंमितम्}
{एतत्कथय मे देव परं कौतूहलं हि मे ॥भगवानुवाच}
{}


\twolineshloka
{अहमश्वत्थरूपेण पालयामि जगत्त्रयम्}
{अस्वत्थो न स्थितो यत्र नाहं तत्र प्रतिष्ठितः}


\twolineshloka
{यत्राहं संस्थितो राजन्नस्वत्थश्चापि तिष्ठति}
{यस्त्वेनमर्चयेद्भक्त्या स मां साक्षात्समर्चति}


\twolineshloka
{यस्त्वेनं प्रहरेत्कोपान्मामेव प्रहरेत्तु सः}
{तस्मात्प्रदक्षिणं कुर्यान्न चिन्द्यादेनमन्वहम्}


\twolineshloka
{व्रतस्य पारणं तीर्थमार्जवं तीर्थमुच्यते}
{देवशुश्रूषणं तीर्तचं गुरुशुश्रूषणं तथा}


\twolineshloka
{पितृशुश्रूषणं तीर्थं मातृशुश्रूषणं तथा}
{दाराणां तोषणं तीर्तं गार्हस्थ्यं तीर्थमुच्यते}


\twolineshloka
{आतिथेयः परं तीर्थं ब्रह्मतीर्थं सनातनम्}
{ब्रह्मिचर्यं परं तीर्तं त्रेताग्निस्तीर्थमुच्यते}


% Check verse!
मूलं धर्मं तु विज्ञाय मनस्तत्रावधार्यताम् ॥गच्छ तीर्थानि कौन्तेय धर्मो धर्मेण वर्धते
\twolineshloka
{द्विविधं तीर्थमित्याहुः स्थावरं जङ्गमं तथा}
{स्तावराज्जङ्गमं तीर्थं ततो ज्ञानपरिग्रहः}


\twolineshloka
{कर्म्णाऽपि विशुद्धस्य पुरुषस्येह भारत}
{हृदये सर्वतीर्थानि तीर्थभूतः स उच्यते}


\twolineshloka
{गुरुतीर्थं परं ज्ञानमतस्तीर्तं न विद्यते}
{ज्ञानतीर्तं परं तीर्तं ब्रह्मतीर्तं सनातनम्}


\twolineshloka
{क्षमा तु परमं तीर्तं सर्वतीर्थेषु पाण्डव}
{क्षमावतामयं लोकः परश्चैव क्षमावताम्}


\twolineshloka
{मानितोऽमानितो वाऽपि पूजितोऽपूजितोपि वा}
{आक्रुष्टस्तर्जितो वाऽपि क्षमावांस्तीर्थमुच्यते}


\twolineshloka
{क्षमा यशः क्षमा दानं क्षमा यज्ञः क्षमा दमः}
{क्षमाऽहिंसा क्षमा धर्मः क्षमा चेन्द्रियनिग्रहः}


\twolineshloka
{क्षमा दया क्षमा यज्ञः क्षमयैव धृतं जगत्}
{क्षमावान्ब्राह्मणो देवः क्षमावान्ब्राह्मणो वरः}


\twolineshloka
{क्षमावानाप्नुयात्स्वर्गं क्षमावानाप्नुयाद्यशः}
{क्षमावान्प्राप्नुयान्मोक्षं तस्मात्साद्युः स उच्यते}


\twolineshloka
{आत्मा नदी भारतपुण्यतीर्थ-मात्मा तीर्थं सर्वतीर्थप्रधानम्}
{आत्मा यज्ञः सततं मन्यते वैस्वर्गो मोक्षः सर्वमात्मन्यधीनम्}


\threelineshloka
{आचारनैर्मल्यमुपागतेनसत्यक्षमानिस्तुलशीतलेन}
{ज्ञानांबुना स्नाति हि नित्यमेवंकिं तस्य भूयः सलिलेन तीर्थम् ॥युधिष्ठिर उवाच}
{}


\threelineshloka
{भगवन्सर्वपापघ्नं प्रायश्चित्तमदुष्करम्}
{त्वद्भक्तस्य सुरश्रेष्ठ मम त्वं वक्तुमर्हसि ॥भगवानुवाच}
{}


\twolineshloka
{रहलस्यमिदमत्यर्थमश्राव्यं पापकर्मणाम्}
{अधार्मिकाणामश्राव्यं प्रायश्चित्तं ब्रवीमि ते}


\twolineshloka
{पावनं ब्राह्ममं दृष्ट्वा मद्गतेनान्तरात्मना}
{नमो ब्रह्मण्यदेवायेत्यभिवादनमाचरेत्}


\twolineshloka
{प्रदक्षिणं च यः कुर्यात्पुनरष्टाक्षरेण तु}
{तेन तुष्टेन विप्रेणि तत्पापं क्षपयाम्यहम्}


\twolineshloka
{पोत्रकृष्टां वराहस्य मृत्तिकां शिरसा वहन्}
{प्राणायामशतं कृत्वा नरः पापैः प्रमुच्यते}


\twolineshloka
{दक्षिणावर्तशङ्खाद्वा कपिलाशृङ्गतोपि वा}
{प्राक्स्रोतसं नदीं गत्वा ममायतनसंनिधौ}


\twolineshloka
{सलिलेन तु यः स्नायात्सकृदेव रविग्रहे}
{तस्य यत्संचितं पापं तत्क्षणादेव नश्यति}


\threelineshloka
{मस्तकान्निस्सृतैस्तोयैः कपिलाया युधिष्ठिर}
{गोमूत्रेणापि यः स्नायाद्रोहिण्यां मम वा दिने}
{विप्रपादच्युतैर्वाऽपि तोयैः पापं प्रणश्यति}


\twolineshloka
{नमस्येद्यस्तु मद्भक्त्या शिंशुमारं प्रजापतिम्}
{चतुर्दशाङ्गसंयुक्तं तस्य पापं प्रणस्यति}


\twolineshloka
{ततश्चतुर्दशाङ्गानि शृणु तस्य युधिष्ठिर}
{शिरो धर्मो हनुर्ब्रह्मा वृषावुत्तरदक्षिणौ}


\twolineshloka
{हृदयं तु भवेद्विष्णुरंसौ स्यातां तथाऽश्विनौ}
{अत्रिर्मध्यं भवेद्राजँल्लिङ्गं संवत्सरं भवेत्}


\twolineshloka
{मित्रावरुणकौ पादावूरुद्वन्द्वं हुताशनः}
{ततः पश्चाद्भवेदिन्द्रस्ततः पश्चात्प्रजापतिः}


\twolineshloka
{अभयं च ततः पश्चात्स एव ध्रुवसंज्ञिकः}
{एतान्यङ्गानि सर्वाणि शिंशुमारप्रजापतेः}


\threelineshloka
{विबेत्तु पञ्चगव्यं यः पौर्णमास्यामुपोष्य तु}
{तस्य नश्यति यत्पापं तत्पापं पूर्वसंचितम् ॥तथैव ब्रह्मकूर्च तु समन्त्रं तु पृथक्पृथक्}
{}


\twolineshloka
{मासिमासि विबेद्यस्तु तस्य पापं प्रणश्यति ॥पात्रं च ब्रह्मकूर्चं च शृणु तत्र च भारत}
{}


\twolineshloka
{पालाशं पद्मपत्रं च ताम्रं वाऽथ हिरण्ययम्}
{सादयित्वा तु गृह्णीयात्तत्तु पात्रमुदाहृतम्}


\twolineshloka
{गायत्र्या गृह्णते मूत्रं गन्धद्वारेति गोमयम्}
{आप्यायस्वेति च क्षीरं दधिक्राण्वेति वै दधि}


\twolineshloka
{तेजोसिशुक्लमित्याज्यं देवस्यत्वेति कुशोदकम्}
{आपोहिष्ठेत्यृचा गृह्यि यवचूर्णं यथाविधि}


\twolineshloka
{ब्रह्मणे च यथा हुत्वा समिद्धे च हुताशने}
{आलोड्य प्रणवेनैव निर्मथ्य प्रणवेन तु}


\twolineshloka
{उद्धृत्य प्रणवेनैव पिबेतु प्रणवेन तु}
{महताऽपि स पापेन त्वचेवाहिर्विमुच्यते}


\twolineshloka
{भद्रं न हति यः पादं पठन्नृक्सहितां तदा}
{अन्तर्जले वाऽभ्यादित्ये तस्य पापं प्रणश्यति}


\twolineshloka
{मम सूक्तं जपेद्यस्तु नित्यं मद्गतमानसः}
{न पापेन स लिप्येत पद्मपत्रमिवांभसा}


\chapter{अध्यायः ११६}
\threelineshloka
{कीदृशा ब्राह्मणाः पुण्या भावशुद्धाः सुरेश्वर}
{यत्कर्म सफलं नेति कथयस्य ममानघ ॥भगवानुवाच}
{}


\twolineshloka
{शृणु पाण्डव तत्सर्वं ब्राह्मणानां यथाक्रमम्}
{सफलं निष्फलं चैव तेषां कर्म ब्रवीमि ते}


\twolineshloka
{त्रिदण्डधारणं मौनं जटाधारणमुण्डनम्}
{वल्कलाजिनसंवासो व्रतचर्याऽभिषेचनम्}


\twolineshloka
{अग्निहोत्रं गृहे वासः स्वाध्यायं दारसत्क्रिया}
{सर्वाण्येतानि वै मिथ्या यदि भावो न निर्मलः}


\twolineshloka
{अग्निहोत्रं वृथा राजन्वृथा वेदास्तथैव च}
{शीलेन देवास्तुष्यन्ति श्रुतयस्तत्र कारणम्}


\twolineshloka
{क्षान्तः दान्तं जितक्रोधं जितात्मानं जितेन्द्रियम्}
{तमग्र्यं ब्राह्मणं मन्ये शेषाः शूद्रा इति स्मृताः}


\twolineshloka
{अग्निहोत्रव्रतपरान्स्वाध्यायनिरताञ्शुचीन्}
{उपवासरतान्दान्तांस्तादेवा ब्राह्मणान्विदुः}


\twolineshloka
{न जात्या पुजीतो राजन्गुणाः कल्याणकारणाः}
{चण्डालमपि वृत्तस्थं तं देवा ब्राह्मणं विदुः}


\twolineshloka
{मनश्शौचं कर्मशौचं कुलशौचं च भारत}
{शरीरशौचं वाक्छौचं शौचं पञ्चविधं स्मृतम्}


\twolineshloka
{पञ्चस्वेतेषु शौचेषु हृदिं शौचं विशिष्यते}
{हृदयस्य च शौचेन स्वर्गं गच्छन्ति मानवाः}


\twolineshloka
{अग्निहोत्रपरिभ्रष्टः प्रसक्तः क्रयविक्रयैः}
{वर्णसङ्करकर्ता च ब्राह्मणो वृषलैः समः}


\twolineshloka
{यस्य वेदश्रुतिर्नष्टा कर्षकश्चापि यो द्विजः}
{विकर्मसेवी कौन्तेय स वै वृषल उच्यते}


\twolineshloka
{वृषो हि धर्मो विज्ञेयस्तस्य यः कुरुते लयम्}
{वृषलं तं विदुर्देवा निकृष्टं श्वपचादपि}


\twolineshloka
{स्तुतिभिर्ब्रह्मगीताभिर्यः शूद्रं स्तौति मानवः}
{न तु मां स्तौति पापात्मा स तु चण्डालतः समः}


\twolineshloka
{श्वदृतौ तु यथा क्षीरं ब्रह्म वै वृषले तथा}
{दुष्टतामेति तत्सर्वं शुना लीढं हविर्यथा}


\twolineshloka
{अङ्गानि वेदाश्चत्वारो मीमांसान्यायविस्तरः}
{धर्मशास्त्रं पुराणं च विद्या ह्येताश्चतुर्दश}


\twolineshloka
{यान्युक्तानि मया सम्यग्विद्यास्थानानि भारत}
{उत्पन्नानि पवित्राणि भुवनार्थं तथैव च}


\threelineshloka
{तस्मात्तानि न शूद्रस्य स्प्रष्टव्यानि युधिष्ठिर}
{सर्वं त्रीण्यपवित्राणि पञ्चामेध्यानि भारत}
{}


\twolineshloka
{श्वा च शूद्रः श्वपाकश्च अपवित्राणि पाण्डव ॥गायकः कुक्कुटो यूपो ह्युदक्या वृषलीपतिः}
{}


\twolineshloka
{प़ञ्चैते स्युरमेध्याश्च स्प्रष्टव्या न कदातन}
{स्पृष्ट्वैतानष्ट वै विप्रः सचेलो जलमाविशेत्}


\twolineshloka
{मद्भक्ताञ्शूद्रसामान्यादवमन्यन्ति ये नराः}
{नरकेष्वेव तिष्ठन्ति वर्षकोटिं नराधमाः}


\twolineshloka
{चण्डालमपि मद्भक्तं नावमन्येत बुद्धिमान्}
{अवमानात्पतन्त्येव नरके रौरवे नराः}


\twolineshloka
{मम भक्तस्य भक्तेषु प्रीतिरभ्यधिका मम}
{तस्मान्मद्भक्तभक्ताश्च पूजनीया विशेषतः}


\twolineshloka
{कीटपक्षिमृगाणां च मयि संन्यस्तचेतसाम्}
{ऊर्ध्वामेव गतिं विद्धि किं पुनर्ज्ञानिनां नृणाम्}


\twolineshloka
{पत्रं वाऽप्यथवा पुष्पंक फलं वाऽप्यप एव वा}
{ददाति मम शूद्रो यच्छिरसा धारयामि तत्}


\twolineshloka
{विप्रानेवार्चयेद्भक्त्या शूद्रप्रायांश्च मत्प्रियान्}
{तेषां तेनैव रूपेण पूजां गृह्णामि भारत}


\twolineshloka
{वेदोत्तेनैव मार्गेण सर्वभूतहृदि स्थितम्}
{मामर्चयन्ति ये पिप्रा मत्सायुज्यं व्रजन्ति ते}


\twolineshloka
{मद्भक्तानां हितायैव प्रादुर्भावः कृतो मया}
{प्रदुर्भावकृता काचिदर्चनीया युधिष्ठिर}


\twolineshloka
{आसामन्यतमां मूर्तिं यो मद्भक्त्या समर्चति}
{तेनैव परितुष्टोऽहं भविष्यामि न संशयः}


\threelineshloka
{मृदा च मणिरत्नैश्च ताम्रेण रजतेन च}
{कृत्वा प्रतिकृतिं कुर्यादर्चनां काञ्चनेन वा}
{पुण्यं दशगुणं विद्यादेतेषामुत्तरोत्तरम्}


\threelineshloka
{जपकामो भवेद्राजा विद्याकामो द्विजोत्तमः}
{वैश्यो वा धनकामस्तु शूद्रः सुखफलप्रियः}
{सर्वकामाः स्त्रियो वाऽपि सर्वान्कामानवाप्नुयुः}


\chapter{अध्यायः ११७}
\threelineshloka
{कीदृशानां तु शूद्राणां नानुगृह्णासि चार्चनम्}
{उद्वेगस्तव कस्माद्धि तन्मे ब्रूहि सुरेश्वर ॥भगवानुवाच}
{}


\twolineshloka
{अव्रतेनाप्यभक्तेन स्पृष्टां शूद्रेण चार्चनाम्}
{तां वर्जयामि राजेन्द्र श्वपाकविहितामिव}


\twolineshloka
{नन्वहं शङ्करश्चापि गावो विप्रास्तथैव च}
{अश्वत्थोऽमररूपं हि त्रयमेतद्युधिष्ठिर}


\twolineshloka
{एतत्त्रयं हि मद्भक्तो नावमन्येत कर्हिचित्}
{अवमानितमेतत्तु दहत्यासप्तमं कुलम्}


\threelineshloka
{अश्वत्थो ब्राह्मणा गावो मन्मयास्तारयन्ति हि}
{तस्मादेतत्प्रयत्नेन त्रयं पूजय पाण्डव ॥युधिष्ठिर उवाच}
{}


\threelineshloka
{ब्राह्मणेनैव देहेन कथं शूद्रत्वमाप्नुयात्}
{ब्रह्म वा नश्यति कथं वक्तुं देव त्वमर्हसि ॥भगवानुवाच}
{}


\twolineshloka
{कूपस्नानं तु यो विप्रः कुर्याद्द्वादशवार्षिकम्}
{स तेनैव शरीरेण शूद्रत्वं यात्यसंशयम्}


\twolineshloka
{यस्तु राजाश्रयेणैव जीवेद्द्वादशवार्षिकम्}
{स शूद्रत्वं व्रजेद्विप्रो वेदानां पारगोपि सन्}


\twolineshloka
{पत्तने नगरे वाऽपि यो द्वादशसमा वसेत्}
{स शूद्रत्वं व्रजेद्विप्रो नात्र कार्या विचारणा}


\twolineshloka
{उत्पादयति यः पुत्रं शूद्रायां काममोहितः}
{तस्य कायगतं ब्रह्म सद्य एव विनश्यति}


\twolineshloka
{यः सोमलतिकां विप्रः केवलं भक्षयेद्वृथा}
{तस्य कायगतं ब्रह्म सद्य एव विनश्यति}


\twolineshloka
{मैथुनं कुरुते यस्तु जिह्वायां ब्राह्मणो नृप}
{तस्य कायगतं ब्रह्म सद्य एव विनश्यति}


\twolineshloka
{विप्रत्वं दुर्लभं प्राप्य दुर्मर्गैरेवमादिभिः}
{विनाशयन्ति ये तत्तु ताञ्शोचामि युधिष्ठिर}


\twolineshloka
{तस्मात्सर्वप्रत्नेन मत्प्रियो यो युधिष्ठिर}
{जातिभ्रंशकरं कर्म न कुर्यादीदृशं द्विजः}


\chapter{अध्यायः ११८}
\threelineshloka
{देशान्तरगते विप्रे संयुक्ते कालधर्मणा}
{शरीरनाशे संप्राप्ते कथं प्रेतत्वकल्पना ॥भगवानुवाच}
{}


\twolineshloka
{श्रूयतामाहिताग्नेस्तु तथाभूतस्य संस्क्रिया}
{पालाशबृन्दैः प्रतिमा कर्तव्या कल्पचोदिता}


\twolineshloka
{त्रीणि षष्टिशतान्याहुरस्थीन्यस्य युधिष्ठिर}
{तेषां विकल्पना कार्या यथाशास्त्रं विनिश्चितम्}


\twolineshloka
{अशीत्यर्धं शिरसि च ग्रीवायां दश एव च}
{बाह्वोश्चापि शतं दद्यादङ्गुलीषु पुनर्दश}


\twolineshloka
{उरसि त्रिंशतं दद्याज्जठरे वाऽपि विंशतिम्}
{वृषणे द्वादशार्धं तु शिश्ने चाष्टार्धमेव च}


\threelineshloka
{दद्यात्तु शतमूर्वोस्तु षष्ट्यर्थं जानुजङ्घयोः}
{दश दद्याच्चरणयोरेषा प्रेतस्य निष्कृतिः ॥युधिष्ठिर उवाच}
{}


\threelineshloka
{विशेषतीर्थं सर्वेषामशक्तानामनुग्रहात्}
{भक्तानां तारणार्तं तु वक्तुमर्हसि धर्मतः ॥भगवानुवाच}
{}


\twolineshloka
{पावनं सर्वतीर्थानां सत्यं गायन्ति सामगाः}
{सत्यस्य वचनं तीर्थमहिंसा तीर्थमुच्यते}


\twolineshloka
{तपस्तीर्थं दया तीर्थं शीलं तीर्थं युधिष्ठिर}
{अल्पसंतोषकं तीर्थं नारी तीर्थं पतिव्रता}


\twolineshloka
{संतुष्टो ब्राह्मणस्तीर्थं ज्ञानं वा तीर्थमुच्यते}
{मद्भक्तः सततं तीर्थं शङ्करस्य विशेषतः}


\twolineshloka
{यतयस्तीर्थमित्येवं विद्वांसस्तीर्थमुच्यते}
{शरण्यपुरुषस्तीर्थमभयं तीर्थमुच्यते}


\twolineshloka
{त्रैलोक्येऽस्मिन्निरुद्विग्नो न बिभेमि कुतश्चन}
{न दिवा यदि वा रात्रावुद्वेगः शूद्रलङ्घनात्}


\twolineshloka
{न भयं देवदैत्येभ्यो रक्षोभ्यश्चैव मे नृप}
{शूद्रवक्त्राच्च्युतं ब्रह्म भयं तु मम सर्वदा}


\twolineshloka
{तस्मात्सप्रणवं शूद्रो मन्नामापि न कीर्तयेत्}
{प्रणवं हि परं लोके ब्रह्म ब्रह्मविदो विदुः}


\twolineshloka
{द्विजशुश्रूषणं धर्मः शूद्राणां भक्तितो मयि}
{तेन गच्छन्ति ते स्वर्गं चिन्तयन्तो हि मां सदा}


\twolineshloka
{द्विजशुश्रूषया शूद्रः परं श्रेयोऽधिगच्छति}
{द्विजशुश्रूषणादन्यन्नास्ति शूद्रस्य निष्कृतिः}


\twolineshloka
{रागो द्वेषश्च मोहश्च पारुष्यं चानृशंसता}
{शाठ्यं च दीर्घवैरित्वमतिमानमनार्जवम्}


\twolineshloka
{अनृतं चापवादं च पैशुन्यमतिलोभता}
{हिंसा स्तेयो मृषावादो वञ्चना रोषलोभता}


\threelineshloka
{अबुद्धिता च नास्तिक्यं भयमालस्यमेव च}
{अशौचं चाकृतज्ञत्वं डंभता स्तंभ एव च}
{निकृतिश्चाप्यविज्ञानं जातके शूद्रमाविशेत्}


\threelineshloka
{सृष्ट्वा पितामहः शूद्रमभिभूतं तु तामसैः}
{द्विजशुश्रूषणं धर्मं शूद्राणां तु प्रयुक्तवान्}
{नश्यन्ति तामसा भावाः शूद्रस्य द्विजभक्तितः}


\twolineshloka
{पत्रं पुष्पं फलं तोयं यो मे भक्त्या प्रयच्छति}
{तदहं भक्त्युपहृतं मूर्ध्ना गृह्णामि शूद्रतः}


\twolineshloka
{अग्रजो वाऽपि यः कश्चित्सर्वपापसमन्वितः}
{यदि मां सततं ध्यायेत्सर्वपापैः प्रमुच्यते}


\twolineshloka
{विद्याविनयसंपन्ना ब्राह्मणा वेदपारगाः}
{मयि भक्ति न कुर्वन्ति चण्डालसदृशा हि ते}


\twolineshloka
{वृथा दानं वृथा तप्तं वृथा चेष्टं वृथा हुतम्}
{वृथाऽऽतिथ्यं च तत्तस्य यो न भक्तो मम द्विजः}


\twolineshloka
{यत्कृतं च हुतं चापि यदिष्टं दत्तमेव च}
{अभक्तिमत्कृतं सर्वं राक्षसा एव भुञ्जते}


\twolineshloka
{स्थावरे जङ्गमे वाऽपि सर्वभूतेषु पाण्डव}
{समत्वेन यदा कुर्यान्मद्भक्तो मित्रशत्रुषु}


\twolineshloka
{आनृशंस्यमहिंसा च यथा सत्यं तथाऽऽर्जवम्}
{अद्रोहश्चैव भूतानां मद्गतानां व्रतं नृप}


\twolineshloka
{नम इत्येव यो ब्रूयान्मद्भक्तं श्रुद्धयाऽन्वितः}
{तस्याक्षयाऽभवँल्लोकाः श्वपाकस्यापि पार्थिव}


\twolineshloka
{किं पुनर्ये यजन्ते मां सदारं विधिपूर्वकम्}
{मद्भक्ता मद्गतप्राणाः कथयन्तश्च मां सदा}


\twolineshloka
{बहुवर्षसहस्राणि तपस्तप्यति यो नरः}
{नासौ पदमवाप्नोति मद्भक्तैर्यदवाप्यते}


\twolineshloka
{मामेव तस्माद्राजेन्द्र ध्यायन्नित्यमतन्द्रितः}
{अवाप्स्यति ततः सिद्धिं द्रक्ष्यत्येव परं पदम्}


\twolineshloka
{अपार्थकं प्रभाषन्तः शूद्रा भागवता इति}
{न शूद्रा भगवद्भक्ता विप्रा भागवताः स्मृताः}


\twolineshloka
{द्वादशाक्षरतत्वज्ञश्चतुर्व्यूहविभागवित्}
{अच्छिद्रपञ्चकालज्ञः स वै भागवतः स्मृतः}


\twolineshloka
{ऋग्वेदेनैव होता च यजुषाऽध्वर्युरेव च}
{सामवेदेन चोद्गाता पुण्येनाभिष्टुवन्ति माम्}


\twolineshloka
{अथर्वशिरसा चैव नित्यमाथर्वाणा द्विजाः}
{स्तुवन्ति सततं ये मां ते वै भागवताः स्मृताः}


\twolineshloka
{वेदाधीनाः सदा यज्ञा यज्ञाधीनास्तु देवताः}
{देवता ब्राह्मणाधीनास्तस्माद्विप्रास्तु देवताः}


\twolineshloka
{अनाश्रित्योच्छ्रयं नास्ति मुख्यमानश्रयमाश्रयेत्}
{रुद्रं समाश्रिता देवा रुद्रो ब्रह्माणमाश्रितः}


\twolineshloka
{ब्रह्मा मामाश्रितो राजन्नाहं कंचिदुपाश्रितः}
{ममाश्रयो न कश्चित्तु सर्वेषामाश्रयो ह्यहम्}


\twolineshloka
{एवमेतन्मया प्रोक्तं रहस्यमिदमुत्तमम्}
{धर्मप्रियस्य ते नित्यं राजन्नेवं समाचर}


\twolineshloka
{इदं पवित्रमाख्यानं पुण्यं वेदेन संमितम्}
{यः पठेन्मामकं धर्ममहन्यहनि पाण्डव}


\twolineshloka
{धर्मोति वर्धते तस्य बुद्धिश्चापि प्रसीदति}
{पापक्षयमुपेत्यैवं कल्याणं च विवर्धते}


\twolineshloka
{एतत्पुण्यं पवित्रं च पामनाशनमुत्तमम्}
{श्रोतव्यं श्रद्धया युक्तैः श्रोत्रियैश्च विशेषतः}


\twolineshloka
{श्रावयेद्यस्त्विदं भक्त्या प्रयतोथ शृणोति वा}
{स गच्छेन्मम सायुज्यं नात्र कार्या विचारणा}


\threelineshloka
{यश्चेमं श्रावयेच्छ्राद्धे मद्भक्तो मत्परायणः}
{पितरस्तस्य तृप्यन्ति यावदाभूतसंप्लवम् ॥वैशम्पायन उवाच}
{}


\twolineshloka
{श्रुत्वा भागवतान्धर्मान्साक्षाद्विष्णोर्जगद्गुरोः}
{प्रहृष्टमनसो भूत्वा चिन्तयन्तोद्भुताः कथाः}


\twolineshloka
{ऋषय पाण्डवाश्चैव प्रणेमुस्तं जनार्दनम्}
{पूजयामास गोविन्दं धर्मपुत्रः पुनः पुनः}


\twolineshloka
{देवा ब्रह्मर्षयः सिद्धा गन्धर्वाप्सरसस्तथा}
{ऋषयस्च महात्मानो गुह्यका भुजगास्तता}


\twolineshloka
{वालखिल्या महात्मानो योगिनस्तत्वदर्शिनः}
{तथा भागवताश्चापि पञ्चकालमुपासकाः}


\twolineshloka
{कौतूहलसमायुक्ता भगवद्भक्तिमागताः}
{श्रुत्वा तु परमं पुण्यं वैष्णवं धर्मशासनम्}


\twolineshloka
{विमुक्तपापाः पूतास्ते संवृत्तास्तत्क्षणेन् तु}
{प्रणम्य शिरसा विष्णु प्रतिनन्द्य च ताः कथाः}


\threelineshloka
{द्रष्टारो द्वारकायां वै वयं सर्वे जगद्गुरुम्}
{इति प्रहृष्टमनसो ययुर्देवगणैः सह}
{सर्वे ऋषिगणा राजन्ययुः स्वंस्वं निवेशनम्}


\threelineshloka
{गतेषु तेषु सर्वेषु केशवः केशिहा हरिः}
{सस्मार दारुकं राजन्स च सात्यकिना सह}
{समीपस्थोऽभवत्सूतो याहि देवेति चाब्रवीत्}


\threelineshloka
{ततो विषण्णवदनाः पाण्डवाः पुरुषोत्तमम्}
{अञ्जलिं मूर्ध्नि संधाय नेत्रैरश्रुपरिप्लुतैः}
{पिबन्तः सततं कृष्णं नोचुरार्ततरास्तदा}


\twolineshloka
{कृष्णोपि भगवान्देवः पृथामामन्त्र्य चार्तवत्}
{धृतराष्ट्रं च गान्धारीं विगदुरं द्रौपदीं तथा}


\threelineshloka
{कृष्णद्वैपायनं व्यासमृषीनन्यांस्च मन्त्रिणः}
{सुभद्रामात्मजयुतामुत्तरां स्पृश्य पाणिना}
{निर्गत्य वेश्मनस्तस्मादारुरोह तदा रथम्}


\twolineshloka
{वाजिभिः शैव्यसुग्रीवमेघपुष्पबलाहकैः}
{युक्तं तु ध्वजभूतेन पतगेन्द्रेण धीमता}


\threelineshloka
{अन्वारुरोह चाप्येन प्रेम्या राजा युधिष्ठिरः}
{अपास्य चाशु यन्तारं दारुकं सूतसत्तमम्}
{अभीशून्प्रतिजग्राह स्वयं कुरुपतिस्तदा}


\twolineshloka
{उपारुह्यार्जुनश्चापि चामरव्यजनं शुभम्}
{रुक्मदण्डं बृहन्मूर्ध्नि दुधावाभिप्रदक्षिणम्}


\twolineshloka
{तथैव भीमसेनोपि रथमारुह्य वीर्यवान्}
{छत्रं शतशलाकं च दिव्यमाल्योपशोभितम्}


\twolineshloka
{वैडूर्यमणिदण़्डं च चामीकरविभूषितम्}
{दधार तरसा भीमश्छत्रं तच्छार्ङ्गधन्वनः}


\twolineshloka
{उपारुह्य रथं शीघ्रं चामरव्यजने सिते}
{नकुलः सहदेवश्च धूयमानौ जनार्दनम्}


\twolineshloka
{भीमसेनोऽर्जुनश्चैव यमावप्यरिसूदनौ}
{पृष्ठतोऽनुययुः कृष्णं माशब्द इति हर्षिताः}


\twolineshloka
{त्रियोजने व्यतीते तु परिष्वज्य च पाण्डवान्}
{विसृज्य कृष्णस्तान्सर्वान्प्रणतान्द्वारका ययौ}


\twolineshloka
{तथा प्रणम्य गोविन्दं तदाप्रभृति पाण्डवाः}
{कपिलाद्यानि दानानि ददुर्धर्मपरायणाः}


\twolineshloka
{मधुसूदनवाक्यानि स्मृत्वास्मृत्वा पुनःपुनः}
{मनसा पूजयामासुर्हदयस्थानि पाण्डवाः}


\twolineshloka
{युधिष्ठिरस्तु धर्मात्मा हृदि कृत्वा जनार्दनम्}
{तद्भक्तस्तन्मना युक्तस्तद्याजी तत्परोऽभवत्}


\twolineshloka
{एवमुक्तं पुरावृत्तं वैष्णवं धर्मशासनम्}
{मया ते कथितं राजन्पिवित्रं पापनाशनम्}


\twolineshloka
{तच्छृणुष्व महाराज विष्णुप्रोक्तं कुरूद्वह}
{तेन गच्छसि नान्येन तद्विष्णोः परमं पदम्}


