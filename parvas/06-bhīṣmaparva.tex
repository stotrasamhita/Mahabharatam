\part{भीष्मपर्व}
\chapter{अध्यायः १}
\twolineshloka
{श्रीवेदव्यासाय नमःनारायणं नमस्कृत्य नरं नरोत्तमम्}
{देवीं सरस्वतीं चैव ततो जयमुदीरयेत् ॥ १}


% Check verse!
कथं युयुधिरे वीराः कुरुपाण्डवसोमकाःपार्थिवाश्च महात्मानो नानादेशसमागताः ॥ १
\twolineshloka
{यथा युयुधिरे वीराः कुरुपाण्डवसोमकाः}
{कुरुक्षेत्रे तपःक्षेत्रे शृणु त्वं पृथिवीपते ॥ २}


\twolineshloka
{तेऽवतीर्य कुरुक्षेत्रं पाण्डवाः सहसोमकाः}
{कौरवानभ्यवर्तन्त जिगीषन्तो महाबलाः ॥ ३}


\twolineshloka
{वेदाध्ययनसंपन्नाः सर्वे युद्दाभिनन्दिनः}
{आशंसन्तो जयं युद्धे बलेनाभिमुखा रणे ॥ ४}


\twolineshloka
{अभियाय च दुर्धर्षां धार्तराष्ट्रस्य वाहिनीं}
{प्राङ्मुखाः पश्चिमे भागे न्यवशन्त ससैनिकाः ५}


\twolineshloka
{स्यमन्तपञ्चकाद्बाह्यं शिबिराणि सहस्रशः}
{कारयामास विधिवत्कुन्तीपुत्रो युधिष्ठिरः ॥ ६}


\twolineshloka
{शून्येव पृथिवी सर्वा बालवृद्धावशेषिता}
{निरश्वपुरुषेवासीद्रथकुञ्जरवर्जिता ॥ ७}


\twolineshloka
{यावत्तपति सुर्यो हि जम्बुद्वीपस्य मण्डलं}
{तावदेव समावृत्तं बलं पार्थिवसत्तम ॥ ८}


\twolineshloka
{एकस्थाः सर्ववर्णास्ते मण्डलं बहुयोजनम्}
{पर्याक्रामन्त देशांश्च नदीः शैलान्वनानि च}


\twolineshloka
{तेषां युधिष्ठिरो राजा सर्वेषां पुरुषर्षभ}
{व्यादिदेश सवाहानां भक्ष्यभोज्यमनुत्तमम्}


\twolineshloka
{संज्ञाय विविधास्तात तेषां चक्रे युधिष्ठिरः}
{एवंवादी वेदितव्यः पाण्डवेयोऽयमित्युत}


\twolineshloka
{अभिज्ञानानि सर्वेषां संज्ञाश्चाभरणानि च}
{योजयामास कौरव्यो युद्धकाल उपस्थिते}


\twolineshloka
{दृष्ट्वा ध्वजाग्रं पार्थस्य धार्तराष्ट्रो महामनाः}
{सह सर्वैर्महीपालैः प्रत्यव्यूहत पाण्डवम्}


\twolineshloka
{पाण्डुरेणातपत्रेण ध्रियमाणेन मूर्धनि}
{मध्ये नागसहस्रस्य भ्रातृभिः परिवारितः}


\twolineshloka
{दृष्ट्वा दुर्योधनं हृष्टाः पाञ्चाला युद्धनन्दिनः}
{दध्मुः प्रीता महाशङ्खान्भेरीर्जघ्नुः सहस्रशः}


\twolineshloka
{ततः प्रहृष्टां स्वां सेनामभिवीक्ष्याथ पाण्डवाः}
{बभूवुर्हृष्टमनसो वासुदेवश्च वीर्यवान्}


\twolineshloka
{स्वयोधान्हर्षयन्तौ च वासुदेवधनञ्जयौ}
{दध्मतुः पुरुषव्याघ्रौ दिव्यौ शङ्खौ रथे स्थितौ}


\twolineshloka
{पाञ्चजन्यस्य शङ्खस्य देवदत्तस्य चोभयोः}
{श्रुत्वा तु निनदं योधाः शकृन्मूत्रं प्रसुस्रुवुः}


\twolineshloka
{यथा सिंहस्य नदतः स्वनं श्रुत्वेतरे मृगाः}
{त्रयेयुर्निनदं श्रुत्वा तथाऽसीदत तद्बलम्}


\twolineshloka
{उदतिष्ठद्रजो भौमं न प्राज्ञायत किंचन}
{अस्तं गत इवादित्यः सैन्येन रजसा वृतः}


\twolineshloka
{सवर्ष तत्र पर्जन्यो मांसशोणितवृष्टिमान्}
{व्युक्षन्सर्वाणि सैन्यानि तदद्भुतमिवाभवत्}


\twolineshloka
{वायुस्ततः प्रादूरभून्नीचैः शर्करकर्षणः}
{विनिघ्नंस्तान्यनीकानि शतशोऽथ सहस्रशः}


\twolineshloka
{उभे सैन्ये च राजेन्द्र युद्धाय मुदिते भृशम्}
{कुरुक्षेत्रे स्थिते यत्ते सागरक्षुभितोपमे}


\twolineshloka
{तयोस्तु सेनयोरासीदद्भुतः स तु संगमः}
{युगान्ते समनुप्राप्ते द्वयोः सागरयोरिव}


\threelineshloka
{शून्याऽऽसीत्पृथिवी सर्वा वृद्धबालावशेषिता}
{निरश्वपुरुषेवासीद्रथकुञ्जरवर्जिता}
{तेन सेनासमूहेन समानीतेन कौरवैः}


\twolineshloka
{ततस्ते समयं चक्रुः कुरुपाण्डवसोमकाः}
{धर्मान्संस्थापयामासुर्युद्धानां भरतर्षभ}


\twolineshloka
{निवृत्ते विहिते युद्धे स्यात्प्रीतिर्नः परस्परम्}
{यथापरं यथायोगं न च स्यात्कस्यचित्पुनः}


\twolineshloka
{वाचा युद्धे प्रवृत्तानां वाचैव प्रतियोधनम्}
{निष्क्रान्ताः पृतनामध्यान्न हन्तव्याः कदाचन}


\twolineshloka
{रथी च रथिना योध्यो गजेन गजधूर्गतः}
{अश्वेनाश्वी पदातिश्च पादातेनैव भारत}


\twolineshloka
{यथायोगं यथाकामं यथोत्साहं यथाबलम्}
{समाभाष्य प्रहर्तव्यं न विश्वस्ते न विह्वले}


\twolineshloka
{परेण सह संयुक्तः प्रमत्तो विमुखस्तथा}
{क्षीणशस्त्रो विवर्मा च न हन्तव्यः कदाचन}


\twolineshloka
{न सूतेषु न धुर्येषु न च शस्त्रोपनायिषु}
{न भेरीशङ्खवादेषु प्रहर्तव्यं कथंचन}


\twolineshloka
{एवं ते समयं कृत्वा कुरुपाण्डवसोमकाः}
{विस्मयं परमं जग्मुः प्रेक्षमाणाः परस्परम्}


\twolineshloka
{निविश्य च महात्मानस्ततस्ते पुरुषर्षभाः}
{हृष्टरूपाः समुनसो बभूवुः सहसैनिकाः}


\chapter{अध्यायः २}
\twolineshloka
{वैशंपायन उवाच}
{}


\twolineshloka
{ततः पूर्वापरे सैन्ये समीक्ष्य भगवानृषिः}
{सर्ववेदविदां श्रेष्ठो व्यासः सत्यवतीसुतः}


\twolineshloka
{भविष्यति रणे घोरे भरतानां पितामहः}
{प्रत्यक्षदर्शी भगवान्भूतभव्यभविष्यवित्}


\threelineshloka
{वैचित्रवीर्यं राजानं रहस्स्थमिदमब्रवीत्}
{शोचन्तमार्तं ध्यायन्तं पुत्राणामनयं तदा ॥व्यास उवाच}
{}


\twolineshloka
{राजन्परिकालास्ते पुत्राश्चान्ये च पार्थिवाः}
{ते हिंसन्तीव सङ्ग्रामे समासाद्येतरेतरम्}


\twolineshloka
{तेषु कालपरीतेषु विनश्यत्स्वेव भारत}
{कालपर्यायमाहाय मा स्म शोके मनः कृथा}


\twolineshloka
{यदि चेच्छसि संग्रामं द्रुष्टुमेनं विशांपते}
{चक्षुर्ददानि ते पुत्र युद्धमेतन्निशामया}


\threelineshloka
{धृतराष्ट्र उवाच}
{न रोचये ज्ञातिवधं द्रष्टुं ब्रह्मर्षिसत्तम}
{युद्धमेतत्त्वशेषेण श्रृणुयां तव तेजसा}


\fourlineindentedshloka
{वैशंपायन उवाच}
{तस्मिन्ननिच्छति द्रष्टुं संग्रामं श्रोतुमिच्छति}
{वराणामीश्वरो व्यासः संजयाय वरं ददौ ॥व्यास उवाच}
{}


\twolineshloka
{एष ते संजयो राजन्युद्धमेतद्वदिष्यति}
{एतस्य सर्वं संग्रामे न परोक्षं भविष्यति}


\twolineshloka
{चक्षुषा संजयो राजन्दिव्येनैव समन्वितः}
{कथयिष्यति ते युद्धं सर्वज्ञश्च भविष्यति}


\twolineshloka
{प्रकाशं वाऽप्रकाशं वा दिवा वा यदि वा निशि}
{मनसा चिन्तितमपि सर्वं वेत्स्यति संजयः}


\twolineshloka
{नैनं शस्त्राणि भेत्स्यन्ति नैनं बाधिष्यते श्रमः}
{गावल्गणिरयं जीवन्युद्धादस्माद्विमोक्ष्यते}


\twolineshloka
{अहं तु कीर्तिमेतेषां करूणां भरतर्षभ}
{पाण्डवानां च सर्वेषां प्रथयिष्यामि मा शुचः}


\threelineshloka
{दिष्टमेतन्नरव्याघ्र नाभिशोचितुमर्हसि}
{न चैव शक्यं संयन्तुं यतो धर्मस्ततो जयः ॥वैशंपायन उवाच}
{}


\twolineshloka
{एवमुक्त्वा स भगवान्कुरूणां प्रतितामहः}
{पुनरेव महाभागो धृतराष्ट्रमुवाच ह}


\twolineshloka
{इह युद्धे महाराज भविष्यति महान्क्षयः}
{तथेह च निमित्तानि भयदान्युपलक्षये}


\twolineshloka
{श्येना गृध्राश्च काकाश्च काङ्काश्च सहिता बकैः}
{संपतन्ति ध्वजाग्रेषु समवायांश्च कुर्वते}


\twolineshloka
{अभ्यग्रं च प्रपश्यन्ति युद्धमानन्दिनो द्विजाः}
{क्रव्यादा भक्षयिष्यन्ति मांसानि गजवाजिनां}


\twolineshloka
{कटाकटेति वाशन्तो भैरवा भयवेदिनः}
{कङ्काः क्रोशन्ति मध्याह्ने दक्षिणामभितो दिशं}


\twolineshloka
{उभे पूर्वापरे सन्ध्ये नित्यं पश्यामि भारत}
{उदयास्तमने सूर्यं कबन्धैः परिवारितम्}


\twolineshloka
{श्वेतलोहितपर्यन्ताः कृष्णाग्रीवाः सविद्युतः}
{त्रिवर्णाः परिघाः सन्धौ भानुमावारयन्त्युत}


\threelineshloka
{ज्वलितार्के .......त्रं निर्विशेषदिनक्षपम्}
{चन्द्रोऽभूतग्निवर्णश्च पद्मवर्णे नभस्तले}
{}


\twolineshloka
{आलक्षे प्रभया हीनां पौर्णमासीं च कीर्तिकीम्}
{चन्द्रोऽभूदग्निवर्णश्च पद्मवर्णे नभस्तले}


\twolineshloka
{स्वप्स्यन्ति निहता वीरा भूमिमावृत्य पार्थिवाः}
{राजानो राजपुत्राश्च शूराः परिघबाहवः}


\twolineshloka
{अन्तरिक्षे वराहस्य पृषदंशकस्य चोभयोः}
{प्रणादं युध्यतो रात्रौ रौद्रं नित्यं प्रलक्षये}


\twolineshloka
{देवताप्रतिमाश्चैव कम्पन्ति च हसन्ति च}
{वमन्ति रुधिरं चास्यैः स्विद्यन्ति प्रपतन्ति च}


\twolineshloka
{अनाहता दुन्दुभयः प्रणदन्ति विशांपते}
{अयुक्ताश्च प्रवर्तन्ते क्षत्रियाणां महारथाः}


\twolineshloka
{कोकिलाः शतपत्राश्च चाषा भासाः शुकास्तथा}
{सारसाश्च मयूराश्च वाचो मुञ्चन्ति दारुणाः}


\twolineshloka
{गृहीतशस्त्राः क्रोशन्ति चर्मिणो वाजिपृष्ठगाः}
{अरुणोदये प्रदृश्यन्ते शतशः शलभव्रजाः}


\twolineshloka
{उभे सन्ध्ये प्रकाशन्ते दिशो दाहसमन्विते}
{पर्जन्यः पांसुवर्षी च मांसवर्षी च भारत}


\twolineshloka
{या चैषा विश्रुता राजंस्त्रैलोक्ये साधुसंमता}
{अरुन्धती तयाप्येष वसिष्ठः पृष्ठतः कृतः}


\twolineshloka
{रोहिणीं पीडयन्नेष स्थितो राजञ्शनैश्चरः}
{व्यावृत्तं लक्ष्म सोमस्य भविष्यति महद्भयम्}


\twolineshloka
{अनभ्रे च महाघोरस्तनितं श्रूयते भृशम्}
{वाहनानां च रुदतां निपतन्त्युश्रुबिन्दवः}


\chapter{अध्यायः ३}
\twolineshloka
{व्यास उवाच}
{}


\twolineshloka
{खरा गोषु प्रजायन्ते रमन्ते मातृभिः सुताः}
{अनार्तवं पुष्पफलं दर्शयन्ति वनद्रुमाः}


\twolineshloka
{गर्भिण्योऽजातपुत्राश्च जनयन्ति विभीषणान्}
{क्रव्यादाः पक्षिभिश्चापि सहा श्नन्ति परस्परम्}


\twolineshloka
{त्रिविषाणाश्चतुर्नेत्राः पञ्चपादा द्विमेहनाः}
{द्विशीर्षाश्च द्विपुच्छाश्च दंष्ट्रिणः पशवोऽशिवाः}


\twolineshloka
{जायन्ते विवृतास्याश्च व्याहरन्तोऽशिवा गिरः}
{त्रिपदाः शिखिनस्तार्क्ष्याश्चतुर्दंष्ट्रा विषाणिनः}


\twolineshloka
{तथैवान्याश्च दृश्यन्ते स्त्रियो वै ब्रह्मवादिनाम्}
{वैनतेयान्मयूरांश्च जनयन्ति पुरे तव}


\twolineshloka
{गोवत्सं वडवा सूते श्वा सृगालं महीपते}
{कुक्कुरान्करभाश्चैव शुकाश्चाशुवादिनः}


\twolineshloka
{स्त्रियः काश्चित्प्रजायन्ते चतस्रः पञ्च कन्यकाः}
{जातमात्राश्च नृत्यन्ति गायन्त च हसन्ति च}


\twolineshloka
{पृथग्जनस्य सर्वस्य क्षुद्रकाः प्रहसन्ति च}
{नृत्यन्ति परिगायन्ति वेदयन्तो महद्भयम्}


\twolineshloka
{प्रतिमाश्चालिखन्त्येताः सशस्त्राः कालचोदिताः}
{अन्योन्यमभिधावन्ति शिशवो दण्डपाणयः}


\twolineshloka
{अन्योन्यमभिमृद्गन्ति नगराणि युयुत्सवः}
{पद्मोत्पलानि वृक्षेषु जायन्ते कुमुदानि च}


\twolineshloka
{विष्व्वाताश्च वान्त्युग्रा रजो नाप्युपशाम्यति}
{अभीक्ष्णं कम्पते भूमिरर्कं राहुरुपैति च}


\twolineshloka
{श्वेतो ग्रहस्तथा चित्रां समतिक्रम्य तिष्ठति}
{अभावं हि विशेषेण कुरूणां तत्र पश्यति}


\twolineshloka
{धूमकेतुर्महाघोरः पुष्यं चाक्रम्य तिष्ठति}
{सेनयोरशिवं घोरं करिष्यति महाग्रहः}


\twolineshloka
{मघास्वङ्गारको वक्रः श्रवणे च बृहस्पतिः}
{भगं नक्षत्रमाक्रम्य सूर्यपुत्रेण पीड्यते}


\twolineshloka
{शुक्रः प्रोष्ठपदे पूर्वे समारुह्य विरोचते}
{उत्तरे तु परिक्रम्य सहितः समुदीक्षते}


\twolineshloka
{श्यामो ग्रहः प्रज्वलितः सधूम इव पावकः}
{ऐन्द्रं तेजस्वि नक्षत्रं ज्येष्ठामाक्रम्य तिष्ठति}


\threelineshloka
{ध्रुवं प्रज्वलितो घोरमपसव्यं प्रवर्तते}
{रोहिणीं पीडयत्येवमुभौ च शशिभास्करो}
{चित्रास्वात्यन्तरे चैव विष्ठितः परुषग्रहः}


\twolineshloka
{वक्रानुवक्रं कृत्वा च श्रवणं पावकप्रभः}
{ब्रह्मराशिं समावृत्य लोहिताङ्गो व्यवस्थितः}


\twolineshloka
{सर्वसस्यपरिच्छन्ना पृथिवी सस्यमालिनी}
{पञ्चशीर्षा यवाश्चापि शतशीर्षाश्च शालयः}


\twolineshloka
{प्रधानाः सर्वलोकस्य यास्वायत्तमिदं जगत्}
{ता गावः प्रस्नुता वत्सैः शोणितं प्रक्षरन्त्युत}


\twolineshloka
{निश्चेरुरर्चिषश्चापात्खङ्गाश्च ज्वलिता भृशम्}
{व्यक्तं पश्यन्ति शस्त्राणि संग्रामं समुपस्थितम्}


\twolineshloka
{अग्निवर्णा यथा भासः शस्त्राणामुदकस्य च}
{कवचानां ध्वजानां च भविष्यति महाक्षयः}


\twolineshloka
{पृथिवी शोणितावर्ता ध्वजोडुपसमाकुला}
{कुरूणां वैशसे राजन्पाण्डवैः सह भारत}


\twolineshloka
{दिक्षु प्रज्वलितास्याश्च व्याहरन्ति मृगद्विजाः}
{अत्याहितं दर्शयन्तः क्षत्रियाणां महद्भयम्}


\twolineshloka
{एकपक्षाक्षिचरणाः शकुन्ताः खेचरा निशि}
{रौद्रं वदन्ति संरब्धाः शोणितं छर्दयन्ति च}


\twolineshloka
{ग्रहौ ताम्रारुणनिभौ प्रज्वलन्ताविव स्थितौ}
{सप्तर्षीणामुदाराणां समवच्छाद्य वै प्रभाम्}


\twolineshloka
{संवत्सरस्थायिनौ च ग्रहौ प्रज्वलितावुभौ}
{विशाखयोः समीपस्थौ बृहस्पतिशनैश्चरौ}


\twolineshloka
{चन्द्रादित्यावुभौ ग्रस्तावेकाह्ना हि त्रयोदशीम्}
{अपर्वमि ग्रहं यातौ प्रजासंक्षयमिच्छतः}


\twolineshloka
{अशोभिता दिशः सर्वाः पांसुवर्षैः समन्ततः}
{उत्पातमेघा रौद्राश्च रात्रौ वर्षन्ति शोणितम्}


\twolineshloka
{कृत्तिकां पीडयंस्तीक्ष्णैर्नक्षत्रं पृथिवीपते}
{अभीक्ष्णवाता वायन्ते धूमकेतुमवस्थिताः}


\threelineshloka
{विषमं जनयन्त्येत आक्रन्दजननं महत्}
{त्रिषु पूर्वेषु सर्वेषु नक्षत्रेषु विशांपते}
{बुधः संपततेऽभीक्ष्णं जनयन्प्राणिनां भयम्}


\threelineshloka
{चतुर्दशीं पञ्चदशीं भूतपूर्वां च षोडशीम्}
{इमां तु नाभिजानामि अमावास्यां त्रयोदशीम्}
{चन्द्रसूर्यावुभौ ग्रस्तावेकान्हा हि त्रयोदशीम्}


\threelineshloka
{अपर्वणि ग्रहेणैतौ प्रजाः संक्षपयिष्यतः}
{मांसवर्षं पुनस्तीव्रमासीत्कृष्णचतुर्दशीम्}
{शोणितैर्वक्रसंपूर्णा अतृप्तास्तत्र राक्षसाः}


\twolineshloka
{प्रतिस्नोतोवहा नद्यः सरितः शोणितोदकाः}
{फेनायमानाः कूपाश्च कूर्दन्ति वृषभा इव}


\twolineshloka
{पतन्त्युल्काः सनिर्घाताः शक्राशनिसमप्रभाः}
{अद्य चैव निशां व्युष्टामनयं समवाप्स्यथ}


\threelineshloka
{विनिःसृत्य महोल्काभिस्तिमिरं सर्वतो दिशम्}
{अन्योन्यमुपतिष्ठद्भिस्तत्र चोक्तं महर्षिभिः}
{भूमिपालसहस्राणां भूमिः पास्यति शोणितम्}


\twolineshloka
{कैलासमन्दराभ्यां तु तथा हिमवतो विभो}
{सहस्रशो महाशब्दं शिखराणि पतन्ति च}


\twolineshloka
{महाभूता भूमिकम्पे चत्वारः सागराः पृथक्}
{वेलामुद्वर्तयन्तीव क्षोभयन्तो वसुन्धराम्}


\twolineshloka
{वृक्षानुन्मथ्य वान्त्युग्रा वाताः शर्करकर्षिणः}
{आभग्राः सुमहावातैरशनीभिः समाहताः}


\twolineshloka
{वृक्षाः पतन्ति चैत्याश्च ग्रामेषु नगरेषु च}
{नीललोहितपीतश्च भवत्यग्निर्हुतो द्विजैः}


\twolineshloka
{वामार्चिर्दुष्टगन्धश्च मुञ्चन्वै दारुणं स्वनम्}
{स्पर्शा गन्धा रसाश्चैव विपरीता महीपते}


\twolineshloka
{धूमं ध्वजाः प्रमुञ्चन्ति कम्पमाना मुहुर्मुहुः}
{मुञ्चन्त्यङ्गारवर्षं च भेर्यश्च पटहास्तथा}


\twolineshloka
{शिखराणां समृद्धानामुपरिष्टात्समन्ततः}
{वायसाश्च रुवन्त्युग्रं वामं मण्डलमाश्रिताः}


\twolineshloka
{पक्वापक्वेऽतिसुभृशं वावाश्यन्ते वयांसि च}
{निलीयन्ते ध्वजाग्रेषु क्षयाय पृथिवीक्षिताम्}


\twolineshloka
{ध्यायन्तः प्रकिरन्तश्च व्याला वेपथुसंयुताः}
{दीनास्तुरंगमाः सर्वे वारणाः सलिलाश्रयाः}


\twolineshloka
{`एवंविधं दुर्निमित्तं क्षयाय पृथीवीक्षिताम्}
{भौमं दिव्यं चान्तरिक्षं त्रिविधं जायतेऽनिशम्'}


\threelineshloka
{एतच्छ्रुत्वा भवानत्र प्राप्तकालं व्यवस्यताम्}
{यथा लोकः समुच्छेदं नायं गच्छेत भारत ॥वैशंपायन उवाच}
{}


\twolineshloka
{पितुर्वचो निशम्यैतद्धृतराष्ट्रोऽब्रवीदिदम्}
{दिष्टमेतत्पुरा मन्ये भविष्यति नरक्षयः}


\twolineshloka
{राजानः क्षत्रधर्मेण यदि वध्यन्ति संयुगे}
{वीरलोकं समासाद्य सुखं प्राप्स्यन्ति केवलम्}


\threelineshloka
{इह कीर्तिं परे लोके दीर्घकालं महत्सुखम्}
{प्राप्स्यन्ति पुरुषव्याघ्राः प्राणांस्त्यक्त्वा महाहवे ॥वैशंपायन उवाच}
{}


\twolineshloka
{एवं मुनिस्तथेत्सुक्त्वा कवीन्द्रो राजसत्तम}
{धृतराष्ट्रेण पुत्रेण ध्यानमन्वगमत्परम्}


\twolineshloka
{स मुहूर्तं तथा ध्यात्वा पुनरेवाब्रवीद्वचः}
{असंशयं पार्थिवेन्द्र कालः संक्षयते जगत्}


\twolineshloka
{सृजते च पुनर्लोकान्नेह विद्यति शाश्वतम्}
{ज्ञातीनां वै कुरूणां च संबन्धिसुहृदां तथा}


\twolineshloka
{धर्म्यं दर्शय पन्थानं समर्थो ह्यसि वारणे}
{क्षुद्रं जातिवधं प्राहुर्मा कुरुष्व ममाप्रियम्}


\twolineshloka
{कालोऽयं पुत्ररूपेण तव जातो विशांपते}
{न वधः पूज्यते वेदे हितं नैव कथंचन}


\twolineshloka
{हन्यात्स एनं यो हन्यात्कुलधर्मं स्विकां तनुम्}
{कालेनोत्पथगन्तसि शक्ये सति यथाऽऽपदि}


\twolineshloka
{कुलस्यास्य विनाशाय तथैव च महीक्षिताम्}
{अनर्थो राज्यरूपेण तव जातो विशांपते}


\twolineshloka
{लुप्तधर्मा परेणासि धर्मं दर्शय वै सुतान्}
{किं ते राज्येन दुर्धर्षयेन प्राप्तोऽसि किल्विषम्}


\threelineshloka
{यशो धर्मं च कीर्तिं च पालयन्स्वर्गमाप्स्यसि}
{लभन्तां पाण्डवा राज्यं शमं गच्छन्तु कौरवाः ॥वैशंपायन उवाच}
{}


\threelineshloka
{एवं ब्रुवति विप्रेन्द्रे धृतराष्ट्रोऽम्बिकासुतः}
{प्रशस्य वाक्यं वाक्यज्ञो वाक्यं चैवाब्रवीत्पुनः ॥धृतराष्ट्र उवाच}
{}


\twolineshloka
{यथा भवान्वेत्ति तथैव वेत्ताभावाभावौ विदितौ मे यथार्थौ}
{स्वार्थे हि संमुह्यति तात लोकोमां चापि लोकात्मकमेव विद्धि}


\twolineshloka
{प्रसादये त्वामतुलप्रभावंत्वं नो गतिर्दर्शयिता च धीरः}
{न चापि ते वशगा मे सुताश्चन चाधर्मं कर्तुमर्हा हि मे मतिः}


\threelineshloka
{त्वं हि धर्मः पवित्रं च यशः कीर्तिर्धृतिः स्मृतिः}
{कुरूणां पाण्डवानां च मान्यश्चापि पितामहः ॥व्यास उवाच}
{}


\threelineshloka
{वैचित्रवीर्य नृपते यत्ते मनसि वर्तते}
{अभिधत्स्व यथाकामं छेत्ताऽस्मि तव संशयम् ॥धृतराष्ट्र उवाच}
{}


\threelineshloka
{यानि लिङ्गानि संग्रामे भवन्ति विजयिष्यताम्}
{तानि सर्वाणि भगवञ्छ्रोतुमिच्छामि तत्त्वतः ॥व्यास उवाच}
{}


\twolineshloka
{प्रसन्नभाः पावक ऊर्ध्वरश्मिःप्रदक्षिणावर्तशिखो विधूमः}
{पुण्या गन्धाश्चाहुतीनां प्रवान्तिजयस्यैतद्भाविनो रूपमाहुः}


\twolineshloka
{गम्भीरघोषाश्च महास्वनाश्चशङ्खा मृदङ्गाश्च नदन्ति यत्र}
{विशुद्धरश्मिस्तपनः शशी चजयस्येतद्भाविनो रूपमाहुः}


\twolineshloka
{इष्टा वाचः प्रसृता वायसानांसंप्रस्थितानां च गमिष्यतां च}
{ये पृष्ठतस्ते त्वरयन्ति राज-न्ये चाग्रतस्ते प्रतिषेधयन्ति}


\twolineshloka
{कल्याणवाचः शकुना राजहंसाःशुकाः क्रौञ्चाः शतपत्राश्च यत्र}
{प्रदक्षिणाश्चैव भवन्ति सङ्ख्येध्रुवं जयस्तत्र वदन्ति विप्राः}


\twolineshloka
{अलंकारैः कवचैः केतुभिश्चसुखप्रणादैर्हेषितैर्वा हयानाम्}
{भ्राजिष्मती दुष्प्रतिवीक्षणीयायेषां चमूस्ते विजयन्ति शत्रून्}


\twolineshloka
{हृष्टा वाचस्तथा सत्वं योधानां यत्र भारत}
{न म्लायन्ति स्रजश्चैव ते तरन्ति रणोदधिम्}


\twolineshloka
{`प्रयाणे वायसो वामे दक्षिणे प्रविविक्षताम्}
{पश्चात्संसाधयत्यर्थं पुरस्तात्प्रतिषेधति ॥ '}


\twolineshloka
{शब्दरूपरसस्पर्शगन्धाश्चाविकृताः शुभाः}
{सदा हर्षश्च योधानां जयतामिह लक्षणम्}


\twolineshloka
{अनुगा वायवो वान्ति तथाऽभ्राणि वयांसि च}
{अनुप्लवन्ति मेघाश्च तथैवेन्द्रधनूंषि च}


\twolineshloka
{एतानि जयमानानां लक्षणानि विशांपते}
{भवन्ति विपरीतानि मुमूर्षूणां जनाधिप}


\twolineshloka
{अल्पायां वा महत्यां वा सेनायामिति निश्चयः}
{हर्षो योधगणस्यैको जयलक्षणमुच्यते}


\twolineshloka
{एको दीर्णो दारयति सेनां सुमहतीमपि}
{तां दीर्णामनुदीर्यन्ते योधाः शूरतरा अपि}


\twolineshloka
{दुर्निवर्त्या तदा चैव प्रभग्ना महती वमूः}
{अपामिव महावेगा त्रस्ता मगगणा इव}


\twolineshloka
{नैव शक्या समाधातुं सन्निपाते महाचमूः}
{दीर्ण इत्येव दीर्यन्ते सुविद्वांसोऽपि भारत}


\threelineshloka
{भीतान्भग्नांश्च संप्रेक्ष्य भयं भूयोऽभिवर्धते}
{प्रभग्ना सहसा राजन्दिशो विद्रवते चमूः}
{नैव स्थापयितुं शक्या शूरैरपि महाचमूः}


\twolineshloka
{सत्कृत्य महतीं सेनां चतुरङ्गां महीपतिः}
{उपायपूर्वं मेधावी यतेत सततोत्थितः}


\twolineshloka
{उपायविजयं श्रेष्ठमाहुर्भेदेन मध्यमम्}
{जघन्य एष विजयो यो युद्धेन विशांपते}


\twolineshloka
{महान्दोषः सन्निपातस्तस्याद्यः क्षय उच्यते}
{परस्परज्ञाः संहृष्टा व्यवधूताः सुनिश्चिताः}


\twolineshloka
{अपि पञ्चाशतं शूरा मृद्गन्ति महतीं चमूम्}
{अपि वा पञ्च षट् सप्त विजयन्त्यनिवर्तिनः}


\twolineshloka
{न वैनतेयो गरुडः प्रशंसति महाजनम्}
{दृष्ट्वा सुपर्णोऽपचितिं महत्या अपि भारत}


\threelineshloka
{न बाहुल्येन सेनाया जयो भवति नित्यशः}
{अध्रुवो हि जयो नाम दैवं चात्र परायणम्}
{जयवन्तो हि संग्रामे कृतकृत्या भवन्ति हि}


\chapter{अध्यायः ४}
\twolineshloka
{वैशंपायन उवाच}
{}


\twolineshloka
{एवमुक्त्वा ययौ व्यासो धृतराष्ट्रय धीमते}
{धृतराष्ट्रोऽपि तच्छ्रुत्वा ध्यानमेवान्वपद्यत}


\twolineshloka
{स मुहूर्तमिव ध्यात्वा विनिःश्वस्य मुहुर्मुहुः}
{संजयं संशितात्मानमपृच्छद्रतर्षभ}


\twolineshloka
{संजयेमे महीपालाः शूरा युद्धाभिनन्दिनः}
{अन्योन्यमभिनिघ्नन्ति शस्त्रैरुच्चावचैरिह}


\twolineshloka
{पार्थिवाः पृथिवीहेतोः समभित्यज्य जीवितम्}
{न वा शाम्यन्ति निघ्नन्तो वर्धयन्ति यमक्षयम्}


\twolineshloka
{भौममैश्वर्यमिच्छन्तो न मृष्यन्ते परस्परम्}
{मन्ये बहुगुणा भूमिस्तन्ममाचक्ष्व संजय}


\twolineshloka
{बहूनि च सहस्राणि प्रयुतान्यर्बुदानि च}
{कोट्यश्च लोकवीराणां समेताः कुरुजाङ्गले}


\twolineshloka
{देशानां च परीमाणं नगराणां च संजय}
{श्रोतुमिच्छामि तत्त्वेन यत एते समागताः}


\threelineshloka
{दिव्यबुद्धिप्रदीपेन युक्तस्त्वं ज्ञानचक्षुषा}
{प्रभावात्तस्य विप्रर्षेर्व्यासस्यामिततेजसः ॥संजय उवाच}
{}


\twolineshloka
{यथाप्रज्ञं महाप्राज्ञ भौमान्वक्ष्यामि ते गुणान्}
{शास्त्रचक्षुरवेक्षस्व नमस्ते भरतर्षभ}


\twolineshloka
{द्विविधानीह भूतानि चराणि स्थावराणि च}
{त्रसानां त्रिविधा योनिरण्डस्वेदजरायुजाः}


\twolineshloka
{त्रसानां खलु सर्वेषां श्रेष्ठा राजञ्जरायुजाः}
{जरायुजानां प्रवरा मानवाः पशवश्च ये}


\twolineshloka
{नानारूपधरा राजंस्तेषां भेदाश्चतुर्दश}
{वेदोक्ताः पृथिवीपाल येषु यज्ञाः प्रतिष्ठिताः}


\twolineshloka
{ग्राम्याणां पुरुषाः श्रेष्ठाः सिंहाश्चारण्यवासिनाम्}
{सर्वेषामेव भूतानामन्योन्येनोपजीवनम्}


\twolineshloka
{उद्भिञ्जाः स्थावराः प्रोक्तास्तेषां पञ्चैव जातयः}
{वृक्षगुल्मलतावल्ल्यस्त्वक्सारास्तृणजातयः}


\twolineshloka
{तेषां विंशतिरेकोना महाभूतेषु पञ्चसु}
{चतुर्विशतिरुद्दिष्टा गायत्री लोकसंमता}


\twolineshloka
{य एतां वेद गायत्रीं पुण्यां सर्वगुणान्विताम्}
{तत्त्वेन भरतश्रेष्ठ स लोके न प्रणश्यति}


\twolineshloka
{अरण्यवासिनः सप्त सप्तैषां ग्रामवासिनः}
{सिंहा व्याघ्रा वराहाश्च महिषा वारणास्तथा}


\twolineshloka
{ऋक्षाश्च वानराश्चैव सप्तारण्याः स्मृता नृप}
{गौरजाविमनुष्याश्च अश्वाश्वतरगर्दभाः}


\twolineshloka
{एते ग्राम्याः समाख्याताः पशवः सप्त साधुभिः}
{एते वै पशवो राजन्ग्राम्यारण्याश्चतुर्दश}


\twolineshloka
{भूमौ च जायते सर्वं भूमौ सर्वं विनश्यति}
{भूमिः प्रतिष्ठा भूतानां भूमिरेव सनातनम्}


\twolineshloka
{यस्य भूमिस्तस्य सर्वं जगत्स्थावरजङ्गमम्}
{तत्रातिगृद्धा राजानो विनिघ्नन्तीतरेतरम्}


\chapter{अध्यायः ५}
\twolineshloka
{धृतराष्ट्र उवाच}
{}


\twolineshloka
{नदीनां पर्वतानां च नामधेयानि संजय}
{तथा जनपदानां च ये चान्ये भूमिमाश्रिताः}


\threelineshloka
{प्रमाणं च प्रमाणज्ञ पृथिव्या मम सर्वतः}
{निखिलेन समाचक्ष्व काननानि च संजय ॥संजय उवाच}
{}


\twolineshloka
{पञ्चेमानि महाराज महाभूतानि संग्रहात्}
{जगतीस्थानि सर्वाणि समान्याहुर्मनीषिणः}


\twolineshloka
{भूमिरापस्तथा वायुरग्निराकाशमेव च}
{गुणोत्तराणि सर्वाणि तेषां भूमिः प्रधानतः}


\twolineshloka
{शब्दः स्पर्शश्च रूपं च रसो गन्धश्च पञ्चमः}
{भूमेरेते गुणाः प्रोक्ता ऋषिभिस्तत्त्ववेदिभिःक}


\threelineshloka
{चत्वारोऽप्सु गुणा राजन्गन्धस्तत्र न विद्यते}
{शब्दः स्पर्शश्च रूपं च तेजसोऽथ गुणास्त्रयः}
{शब्दः स्पर्शश्च वायौ द्वौ अकाशे शब्द एव तु}


\twolineshloka
{एते पञ्च गुणा राजन्महाभूतेषु पञ्चसु}
{वर्तन्ते सर्वलोकेषु येषु भूताः प्रतिष्ठिताः}


% Check verse!
अन्योन्यं नाभिवर्तन्ते साम्यं भवति वै यदा
\twolineshloka
{यदा तु विषमीभावमाविशन्ति परस्परम्}
{तदा देहैर्देहवन्तो व्यतिरोहन्ति नान्यथा}


\twolineshloka
{आनुपूर्व्या विनश्यन्ति जायन्ते चानुपूर्वशः}
{सर्वाम्यपरिमेयाणि तदेषां रूपमैश्वरम्}


\twolineshloka
{तत्रतत्र हि दृश्यन्ते धातवः पाञ्चभौतिकाः}
{तेषां मनुष्यास्तर्केण प्रमाणानि प्रचक्षते}


\twolineshloka
{अचिन्त्याः खलु ये भावा न तांस्तर्केण साधयेत्}
{प्रकृतिभ्यः परं यत्तु तदचिन्त्यस्य लक्षणम्}


\twolineshloka
{सुदर्शनं प्रवक्ष्यामि द्वीपं तु कुरुनन्दन}
{परिमण्डलो महाराज द्वीपोऽसौ चक्रसंस्थितः}


\twolineshloka
{नदीजालप्रतिच्छन्नः पर्वतैश्चाभ्रसन्निभैः}
{पुरैश्च विविधाकारै रम्यैर्जनपदैस्तथा}


\twolineshloka
{वृक्षैः पुष्पफलोपेतैः संपन्नधनधान्यवान्}
{लवणेन समुद्रेण समन्तात्परिवारितः}


\twolineshloka
{यथा हि पुरुषः पश्येदादर्शे मुखमात्मनः}
{एवं सुदर्शनद्वीपो दृश्यते चन्द्रमण्डले}


\threelineshloka
{द्विरशस्तु ततः प्लक्षो द्विरंशः शाल्मलिर्महान्}
{द्विरंशः पिप्पलस्तस्य द्विरंशश्च कुशो महान्}
{सर्वौषधिसमापन्नः पर्वतैः परिवारितः}


\twolineshloka
{आपस्ततोऽन्या विज्ञेयाः शेषः संक्षेप उच्यते}
{ततोऽन्य उच्यते चायमेनं संक्षेपतः श्रृणु}


\chapter{अध्यायः ६}
\twolineshloka
{धृतराष्ट्र उवाच}
{}


\twolineshloka
{उक्तो द्वीपस्य संक्षेपो विधिवद्बुद्धिमंस्त्वया}
{तत्त्वज्ञश्चासि सर्वस्य विस्तारं ब्रूहि सञ्जय}


\threelineshloka
{यावान्भूम्यवकाशोऽयं दृश्यते शशलक्षणे}
{तस्य प्रमाणां प्रब्रूहि ततो वक्ष्यसि पिप्पलम् ॥वैशंपायन उवाच}
{}


\fourlineindentedshloka
{एवं राज्ञा स पृष्टस्तु संजयो वाक्यमब्रवीत्}
{6-6-3xa संजयउवाच}
{प्रागायता महाराज षडेते वर्षपर्वताः}
{अवगाढा ह्युभयतः समुद्रौ पूर्वपश्चिमौ}


\twolineshloka
{हिमवान्हेमकूटश्च निषधश्च नगोत्तमः}
{नीलश्च वैदूर्यमयः श्वेतश्च शशिसन्निभः}


\twolineshloka
{सर्वधातुविचित्रश्च श्रृङ्गवान्नाम पर्वतः}
{एते वै पर्वता राजन्सिद्धचारणसेविताः}


\twolineshloka
{एषामन्तरविष्कम्भा योजनानि सहस्रशः}
{तत्र पुण्या जनपदास्तानि वर्षाणि भारत}


\threelineshloka
{वसन्ति तेषु सत्वानि नानाजातीनि सर्वशः}
{इदं तु भारतं वर्षं ततो हैमवतं परम् ॥ 6-6-8a` ततःकिंपुरुषावासं वर्षं हिमवतः परम्'}
{हेमकूटात्परं चैव हरिवर्षं प्रचक्षते}


\twolineshloka
{दक्षिणेन तु नीलस्य निषधस्योत्तरेण तु}
{व्रागायतो महाभाग माल्यवान्नाम पर्वतः}


\threelineshloka
{ततः परं माल्यवतः पर्वतो गन्धमादनः}
{परिमण्डलस्तयोर्मध्ये मेरुः कनकपर्वतः}
{तरुणादित्यसंकाशो विधूम इव पावकः}


\threelineshloka
{योजनानां सहस्राणि षोडशाधः किल स्मृतः}
{ऊर्ध्वं च चतुरशीतिर्द्वात्रिंशन्मूर्ध्नि विस्तृतः}
{अधस्ताच्चतुरशीतिर्योजनानां महीपते}


\twolineshloka
{ऊर्ध्वंमधश्च तिर्यक्व मेरुरावृत्य तिष्ठति}
{तस्य पार्श्वेष्वमी द्वीपाश्चत्वारः सस्थिता विभो}


\twolineshloka
{भद्राश्वः केतुमालश्च जम्बूद्वीपश्च भारत}
{उत्तराश्चैव कुरवः कृतपुण्यप्रतिश्रयाः}


\twolineshloka
{विहगः सुमुखो यस्तु सुपर्णस्यात्मजः किल}
{स वै विचिन्तयामास सौवर्णान्वीक्ष्य वायसान्}


\twolineshloka
{मेरुरुत्तममध्यानामधमानां च पक्षिणाम्}
{अविशेषकरो यस्मात्तस्मादेनं त्यजाम्यहम्}


\twolineshloka
{तमादित्योऽनुपर्येति सततं ज्योतिषां वरः}
{चन्द्रमाश्च सनक्षत्रो वायुश्चैव प्रदक्षिणः}


\twolineshloka
{सपर्वतो महाराज दिव्यपुष्पफलान्वितः}
{भवनैरावृतः सर्वैर्जाम्बूनदपरिष्कृतैः}


\twolineshloka
{तत्र देवगणा राजन्गन्धर्वासुरराक्षसाः}
{अप्सरोगणसंयुक्ताः शैले क्रीडन्ति सर्वदा}


\twolineshloka
{तत्र ब्रह्मा च रुद्रश्च शक्रश्चापि सुरेश्वरः}
{समेत्य विविधैर्यज्ञैर्यजन्तेऽनेकदक्षिणैः}


\twolineshloka
{तुम्बुरुर्नारदश्चैव विश्वावसुर्हहाहुहूः}
{अभिगम्यामरश्रेष्ठांस्तुष्टुवुर्विविधैः स्तवैः}


\twolineshloka
{सप्तर्षयो महात्मानः कश्यपश्च प्रजापतिः}
{तत्र गच्छन्ति भद्रं ते सदा पर्वणि पर्वणि}


\twolineshloka
{तस्यैव मूर्धन्युशनाः काव्यो दैत्यैर्महीपते}
{इमानि तस्य रत्नानि तस्येमे रत्नपर्वताः}


\twolineshloka
{तस्मात्कुबेरो भगवांश्चतुर्थं भागमश्रुते}
{ततः कलांशं वित्तस्य मनुष्येभ्यः प्रयच्छति}


\twolineshloka
{पार्श्वे तस्योत्तरे दिव्यं सर्वर्तुकुसुमैश्चितम्}
{कर्णिकारवनं रम्यं शिलाजालसमुद्गतम्}


\twolineshloka
{तत्र साक्षात्पशुपतिर्दिव्यैर्भूतैः समावृतः}
{उमासहायो भगवान्रमते भूतभावनः}


\twolineshloka
{कर्णिकारमयी मालां बिभ्रत्पादावलम्बिनीम्}
{त्रिभिर्नेत्रैः कृतोद्योतस्त्रिभिः सूर्यैरिवोदितैः}


\twolineshloka
{तमुग्रतपसः सिद्धाः सुव्रताः सत्यवादिनः}
{पश्यन्ति नहि दुर्वृत्तैः शक्यो द्रुष्टुं महेश्वरः}


\twolineshloka
{तस्य शैलस्य शिखरात्क्षीरधारा नरेश्वर}
{विश्वरूपाऽपरिमिता भीमनिर्घातनिःस्वना}


\twolineshloka
{पुण्या पुण्यतमैर्जुष्टा गङ्गा भागीरथी शुभा}
{प्लवन्तीव प्रवेगेन ह्रदे चन्द्रमसः शुभे}


\twolineshloka
{तया ह्युत्पादितः पुण्यः स ह्रदः सागरोपमः}
{तां धारयामास तदा दुर्धरां पर्वतैरपि}


\twolineshloka
{शतं वर्षसहस्राणां शिरसैव महेश्वरः}
{मेरोस्तु पश्चिमे पार्श्वे केतुमालो महीपते}


\twolineshloka
{जम्बूखण्डे तु तत्रैव महाजनपदो नृप}
{आयुर्दशसहस्राणि वर्षाणां तत्र भारत}


\twolineshloka
{सुवर्णवर्णाश्च नराः स्त्रियश्चाप्सरसोपमाः}
{अनामया वीतशोका नित्यं मुदितमानसाः}


\twolineshloka
{जायन्ते मानवास्तत्र निष्टप्तकनकप्रभाः}
{गन्धमादनश्रृङ्गेषु कुबेरः सह राक्षसैः}


\twolineshloka
{संवृतोऽप्सरसां सङ्घैर्मोदते गुह्यकाधिपः}
{गन्धमादनपार्श्वे तु पर त्वपरगण्डिकाः}


\threelineshloka
{एकादशसहस्राणि वर्षाणां परमायुषः}
{तत्र कृष्णा नरा राजंस्तेजोयुक्ता महाबलाः}
{स्त्रियश्चोत्पलवर्णाभाः सर्वाः सुप्रियदर्शनाः}


\twolineshloka
{नीलात्परतरं श्वेतं श्वेताद्धैरण्यकं परम्}
{वर्षमैरष्वतं राजन्नानाजनपदावृतम्}


\twolineshloka
{धनुःसंस्थे महाराज द्वे वर्षे दक्षिणोत्तरे}
{इलावृत्तं मध्यमं तु पञ्च दीर्घाणि चैव हि}


\twolineshloka
{उत्तरोत्तरमेतेभ्यो वर्षमुद्रिच्यते गुणैः}
{आयुःप्रमाणमारोग्यं धर्मतः कामतोऽर्थतः}


\twolineshloka
{समन्वितानि भूतानि तेषु वर्षेषु भारत}
{एवमेषा महाराज पर्वतैः पृथिवी चिता}


\twolineshloka
{हेमकूटस्तु सुमहान्कैलासो नाम पर्वतः}
{6-6-41bयत्र वैश्रवणोराजन्गुह्यकैः सह मोदते}


\twolineshloka
{तत्र देवो महादेवो नित्यमास्ते सहोमया}
{शीते शिलातले रम्ये देवर्षिगणपूजितःक}


\twolineshloka
{अस्त्युत्तरेण कैलासं मैनाकं पर्वतं प्रति}
{हिरण्यश्रृङ्गः सुमहान्दिव्यो मणिमयो गिरिः}


\twolineshloka
{तस्य पार्श्वे महद्दिव्यं शुभ्रं काञ्चनवालुकम्}
{रम्यं बिन्दुसरो नाम यत्र राजा भगीरथः}


\twolineshloka
{दृष्ट्वा भागीरथीं गङ्गामुवास बहुलाः समाः}
{यूपा मणिमयास्तत्र चैत्याश्चापि हिरण्मयाः}


\twolineshloka
{तत्रेष्ट्वा तु गतः सिद्धिं सहस्राक्षो महायशाः}
{स्रष्टा भूतपतिर्यत्र सर्वलोकान्सनातनान्}


\twolineshloka
{उपास्यते तिग्मतेजा यत्र भूतैः समन्ततः}
{नरनारायणौ ब्रह्मा मनुः स्थाणुश्च पञ्चमः}


\twolineshloka
{तत्र दिव्या त्रिपथगा प्रथमं तु प्रतिष्ठिता}
{ब्रह्मलोकादपक्रान्ता सप्तधा प्रतिपद्यते}


\twolineshloka
{वस्वौकसारा नलिनी पावनी च सरस्वती}
{जम्बूनदी च सीता च गङ्गा सिंधुश्च सप्तमी}


\twolineshloka
{अचिन्त्या दिव्यसंकाशा प्रभोरेषैव संविधिः}
{उपासते यत्र सत्रं सहस्रयुगपर्यये}


\twolineshloka
{दृश्याऽदृश्या च भवति तत्र तत्र सरस्वती}
{एता दिव्याः सप्त गङ्गास्त्रिषु लोकेषु विश्रुताः}


\twolineshloka
{रक्षांसि वै हिमवति हेमकूटे तु गुह्यकाः}
{सर्पा नागाश्च निषधे गोकर्णं च तपोवनम्}


\threelineshloka
{देवासुराणां सर्वेषां श्वेतपर्वत उच्यते}
{गन्धर्वा निषधे नित्यं नीले ब्रह्मर्षयस्तथा}
{श्रृङ्गवांस्तु महाराज देवानां प्रतिसंचरः}


\twolineshloka
{इत्येतानि महाराज सप्त वर्षाणि भागशः}
{भूतान्युपनिविष्टानि गतिमन्ति ध्रुवाणि च}


\twolineshloka
{तेषामृद्धिर्बहुविधा दृश्यते दैवमानुषी}
{अशक्या परिसंख्यातुं श्रद्धेया तु बुभूषता}


\threelineshloka
{यां तु पृच्छसि मां राजन्दिव्यामेतां शशाकृतिम्}
{पार्श्वे शशस्य द्वे वर्षे उक्ते ये दक्षिणोत्तरे}
{कर्णौ तु शाकद्वीपश्च काश्यपद्वीप एव च}


\twolineshloka
{ताम्रपर्णीशिरो राजञ्छ्रीमान्मलयपर्वतः}
{एतद्द्वितीयं द्वीपस्य दृश्यते शशसंस्थितम्}


\chapter{अध्यायः ७}
\twolineshloka
{धृतराष्ट्र उवाच}
{}


\threelineshloka
{मेरोरथोत्तरं पार्श्वं पूर्वं चाचक्ष्व संजय}
{निखिलेन महाबुद्धे माल्यवन्तं च पर्वतम् ॥संजय उवाच}
{}


\twolineshloka
{दक्षिणेन तु नीलस्य मेरोः पार्श्वे तथोत्तरे}
{उत्तराः कुरवो राजन्पुण्याः सिद्धनिषेविताः}


\twolineshloka
{तत्र वृक्षा मधुफला नित्यपुष्पफलोपगाः}
{पुष्पाणि च सुगन्धीनि रसवन्ति फलानि च}


\twolineshloka
{सर्वकामफलास्तत्र केचिद्वृक्षा जनाधिप}
{अपरे क्षीरिणो नाम वृक्षास्तत्र नराधिप}


\twolineshloka
{ये क्षरन्ति सदा क्षीरं षड्रसं चामृपोतमम्}
{वस्त्राणि च प्रसूयन्ते फलेष्वाभरणानि च}


\threelineshloka
{सर्वा मणिमयी भूमिः सूक्ष्मकाञ्चनवालुका}
{सर्वर्तुसुखसंस्पर्शा निष्पङ्का च जनाधिप}
{पुष्करिष्यः शुभास्तत्र सुखस्पर्शा मनोरमाः}


\twolineshloka
{देवलोकच्युताः सर्वे जायन्ते तत्र मानवाः}
{शुक्लाभिजनसंयन्नाः सर्वे सुप्रियदर्शनाः}


\twolineshloka
{मिथुनानि च जायन्ते स्त्रियश्चाप्सरसोपमाः}
{तेषां ते क्षीरिणां क्षीरं पिबन्त्यमृतसन्निभम्}


\twolineshloka
{मिथुनं जायते काले समं तच्च प्रवर्धते}
{तुल्यरूगुणोपेतं समवेषं तथैव च}


\twolineshloka
{एवमेवानुरूपं च चक्रवाकसमं विभो}
{निरापगाम ते लोका नित्यं मुदितमानसाः}


\twolineshloka
{दशवर्षसहस्राणि शशवर्षशतानि च}
{जीवन्ति ते महाराज न चान्योन्यं जहत्युत}


\twolineshloka
{भारुण्डा नाम शकुनास्तीक्ष्णतुण्डा महाबलाः}
{तान्निहेरन्तीह मृतान्दरीषु प्रक्षिपन्ति च}


\twolineshloka
{उत्तराः कुरवो राजन्व्याख्यातास्ते समासतः}
{मेरोः पार्श्वमहं पूर्वं वक्ष्याम्यथ यथातथम्}


\twolineshloka
{तस्य मूर्धाभिषेकस्तु भद्राश्वस्य विशांपते}
{भद्रसालवनं यत्र कालाम्रश्च महाद्रुमः}


\twolineshloka
{कालाम्रस्तु महाराज नित्यपुष्पफलः शुभः}
{द्रुमश्च योजनोत्सेधः सिद्धचारणसेवितः}


\twolineshloka
{तत्र ते पुरुषाः श्वेतास्तेजोयुक्ता महाबलाः}
{स्त्रियः कुमुदवर्णाश्च सुन्दर्यः प्रियदर्शनाः}


\twolineshloka
{चन्द्रप्रभाश्चन्द्रवर्णाः पूर्वचन्द्रनिभाननाः}
{चन्द्रशीतलगात्र्यश्च नृत्तिगीतविशारदाः}


\twolineshloka
{दशवर्षसहस्राणि तत्रायुर्भरतर्षभ}
{कालाम्ररसपीतास्ते नित्यं संस्थितयौवनाः}


\twolineshloka
{दक्षिणेन तु नीलस्य निषधस्योत्तरेण तु}
{सुदर्शनो नाम महाञ्जम्बूवृक्षः सनातनः}


\twolineshloka
{सर्वकामफलः पुण्यः सिद्धचारणसेवितः}
{तस्य नाम्ना समाख्यातो जम्बूद्वीपः सनातनः}


\twolineshloka
{योजनानां सहस्त्रं च शतं च भरतर्षभ}
{उत्सेधो वृक्षराजस्य दिवस्पृङ्भनुजेश्वर}


\twolineshloka
{अरत्नीनां सहस्रं च शतानि दश पञ्च च}
{परिणाहस्तु वृक्षस्य फलानां रसभेदिनाम्}


\twolineshloka
{पतमानानि तान्युर्वी कुर्वन्ति विपुलं स्वनम्}
{मुञ्चन्ति च रसं राजंस्तस्मिन्रजतसन्निभम्}


\twolineshloka
{तस्या जम्बाः फलरसो नदी भूत्वा जनाधिप}
{मेरुं प्रदक्षिणं कृत्वा संप्रयात्युत्तरान्कुरून्}


\twolineshloka
{तत्र तेषां मनःशान्तिर्न पिपासा जनाधिप}
{तस्मिन्फलरसे पीते न जरा बाधते च तान्}


\twolineshloka
{तत्र जाम्बूनदं नाम कनकं देवभूषणम्}
{इन्द्रगोपकसंकाशं जायते भास्वरं तु तत्}


\twolineshloka
{तरुणादित्यवर्णाश्च जायन्ते तत्र मानवाः}
{तथा माल्यवतः श्रृङ्गे दृश्यते हव्यवाट् सदा}


\twolineshloka
{नाम्ना संवर्तको नाम कालाग्निर्भरतर्षभ}
{तथा माल्यवतः श्रृङ्गे पूर्वपूर्वानुगण्डिका}


\twolineshloka
{योजनानां सहस्राणि पञ्चषण्माल्यवानथ}
{महारजतसंकाशा जायन्ते तत्र मानवाः}


\threelineshloka
{ब्रह्मलोकच्युताः सर्वे सर्वे सर्वेषु साधवः}
{तपस्तप्यन्ति ते तीव्रं भवन्ति ह्यूर्ध्वरेतसः}
{रक्षणार्थं तु भूतानां प्रविश्यन्ते दिवाकरम्}


\twolineshloka
{षष्टिस्तानि सहस्राणि षष्टिरेव शतानि च}
{अरुणस्याग्रतो यान्ति परिवार्य दिवाकरम्}


\twolineshloka
{षष्टिं वर्षसहस्राणि षष्टिमेव शतानि च}
{आदित्यतापतप्तास्ते विशन्ति शशिमण्डलम्}


\chapter{अध्यायः ८}
\twolineshloka
{धृतराष्ट्र उवाच}
{}


\threelineshloka
{वर्षाणां चैव नामानि पर्वतानां च संजय}
{आचक्ष्व मे यथातत्त्वं ये च पर्वतवासिनः ॥संजय उवाच}
{}


\twolineshloka
{दक्षिणेन तु श्वेतस्य नीलस्यैवोत्तरेण तु}
{वर्षं रमणकं नाम जायन्ते तत्र मानवाः}


\twolineshloka
{शुक्लाभिजनसंपन्नाः सर्वे सुप्रियदर्शनाः}
{निःसपत्नाश्च ते सर्वे जायन्ते तत्र मानवाः}


\twolineshloka
{दशवर्षसहस्राणि शतानि दश पञ्च च}
{जीवन्ति ते महाराज नित्यं मुदितमानसाः}


\twolineshloka
{दक्षिणे शृङ्गिणश्चैव श्वेतस्याथोत्तरेण तु}
{वर्षं हिरण्मयं नाम यत्र हैरण्वती नदी}


\twolineshloka
{यत्र चायं महाराज पक्षिराट् पतगोत्तमः}
{यक्षानुगा महाराज धनिनः प्रियदर्शनाः}


\twolineshloka
{महाबलास्तत्र जना राजन्मुदितमानसाः}
{एकादशसहस्राणि वर्षाणां ते जनाधिप}


\twolineshloka
{आयुःप्रमाणं जीवन्ति शतानि दश पञ्च च}
{श्रृङ्गाणि वै श्रृङ्गवतस्त्रीण्येव मनुजाधिप}


\threelineshloka
{एकं मणिमयं तत्र तथैकं रौक्ममद्भुतम्}
{सर्वरत्नमयं चैकं भवनैरुपशोभितम्}
{तत्र स्वयंप्रभा देवी नित्यं वसति शाण्डिली}


\twolineshloka
{उत्तरेण तु श्रृङ्गस्य समुद्रान्ते जनाधिप}
{वर्षमैरावतं नाम तस्माच्छृङ्गवतः परम्}


\twolineshloka
{न तत्र सूर्यस्तपति न जीर्यन्ते च मानवाः}
{चन्द्रमाश्च सनक्षत्रो ज्योतिर्भूत इवावृतः}


\twolineshloka
{पद्मप्रभाः पद्मवर्णाः पद्मपत्रनिभेक्षणाः}
{पद्मपत्रसुगन्धाश्च जायन्ते तत्र मानवाः}


\twolineshloka
{अनिष्पन्दा इष्टगन्धा निराहारा जितेन्द्रियाः}
{देवलोकच्युताः सर्वे तथा विरजसो नृप}


\twolineshloka
{त्रयोदशसहस्राणि वर्षाणां ते जनाधिप}
{आयुःप्रमाणं जीवन्ति नरा भरतसत्तम}


\twolineshloka
{क्षीरोदस्य समुद्रस्य तथैवोत्तरतः प्रभुः}
{हरिर्वसति वैकुण्ठः शकटे कनकोञ्ज्वले}


\twolineshloka
{अष्टचक्रं हि तद्यानं भूतयुक्तं मनोजवम्}
{अग्निवर्णं महातेजो जाम्बूनदविभूषितम्}


\twolineshloka
{स प्रभुः सर्वभूतानां विभुश्च भरतर्षभ}
{संक्षेपो विस्तरश्चैव कर्ता कारयिता तथा}


\threelineshloka
{पृथिव्यापस्तथाऽऽकाशं वायुस्तेजश्च पार्थिव}
{स यज्ञः सर्वभूतनामास्यं तस्य हुताशनः ॥वैशंपायन उवाच}
{}


\twolineshloka
{एवमुक्तः संजयेन धृतराष्ट्रो महामनाः}
{ध्यानमन्वगमद्राजा पुत्रान्प्रति जनाधिप}


\twolineshloka
{स विचिन्त्य महातेजाः पुनरेवाब्रवीद्वचः}
{असंशयं सूतपुत्र कालः संक्षिपते जगत्}


\twolineshloka
{सृजते च पुनः सर्वं नेह विद्यति शाश्वतम्}
{नरो नारायणश्चैव सर्वज्ञः सर्वभूतहृत्}


% Check verse!
देवा वैकुण्ठ इत्याद्दुर्वेदा विष्णुरिति प्रभुम्
\chapter{अध्यायः ९}
\twolineshloka
{धृतराष्ट्र उवाच}
{}


\twolineshloka
{यदिदं भारतं वर्षं यत्रेदं मूर्च्छितं बलम्}
{यत्रातिमात्रलुब्धोऽयं पुत्रो दुर्योधनो मम}


\threelineshloka
{यत्र गृद्धा पाण्डुपुत्रा यत्र मे सञ्जते मनः}
{एतन्मे तत्त्वमाचक्ष्व त्वं हि मे बुद्धिमान्मतः ॥संजय उवाच}
{}


\twolineshloka
{न तत्र पाण्डवा गृद्धाः शृणु राजन्वचो मम}
{गृद्धो दुर्योधनस्तत्र शकुनिश्चापि सौबलः}


\twolineshloka
{अपरे क्षत्रियाश्चैव नानाजनपदेश्वराः}
{ये गृद्धा भारते वर्षे न मृष्यन्ति परस्परम्}


\twolineshloka
{अत्र ते कीर्तयिष्यामि वर्षं भारत भारतम्}
{प्रियमिन्द्रस्य देवस्य मनोर्वैवस्वतस्य च}


\twolineshloka
{पृथोस्तु राजन्वैन्यस्य तथेक्ष्वाकोर्महात्मनः}
{ययातेरम्बरीषस्य मान्धातुर्नहुषस्य च}


\twolineshloka
{तथैव मुचुकुन्दस्य शिबेरौशीनरस्य च}
{ऋषभस्य तथैलस्य नृगस्य नृपतेस्तथा}


\twolineshloka
{कुशिकस्य च दुर्धर्ष गाधेश्चैव महात्मनः}
{सोमकस्य च दुर्धर्ष दिलीपस्य तथैव च}


\twolineshloka
{अन्येषां च महाराज क्षत्रियाणां बलीयसाम्}
{सर्वेषामेव राजेन्द्र प्रियं भारत भारतम्}


\twolineshloka
{तत्ते वर्षं प्रवक्ष्यामि यथायथमरिन्दम}
{श्रृणु मे गदतो राजन्यन्मां त्वं परिपृच्छसि}


\twolineshloka
{महेन्द्रो मलयः सह्यः शुक्तिमानृक्षवानपि}
{विन्ध्यश्च पारियात्रश्च सप्तैते कुलपर्वताः}


\twolineshloka
{तेषां सहस्रशो राजन्पर्वतास्ते समीपतः}
{अविज्ञाताः सारवन्तो विपुलाश्चित्रसानवः}


\twolineshloka
{अन्ये ततोऽपरिज्ञाता ह्रस्वा ह्रस्वोपजीविनः}
{आर्या म्लेच्छाश्च कौरव्य तैर्मिश्राः पुरुषा विभो}


\twolineshloka
{नदीं पिबन्ति विपुलां गङ्गां सिन्धुं सरस्वतीम्}
{गोदावरीं नर्मदां च बाहुदां च महानदीम्}


\twolineshloka
{शतद्रूं चन्द्रभागां च यमुनां च महानदीम्}
{दृषद्वतीं विपाशां च विपापां स्थूलवालुकाम्}


\twolineshloka
{नदीं वेत्रवतीं चैव कृष्णवेणीं च निम्नगाम्}
{इरावतीं वितस्तां च पयोष्णीं देविकामपि}


\twolineshloka
{वेदस्मृतां वेदवतीं त्रिदिवामिक्षुलां कृमिम्}
{करीषिणीं चित्रवाहां चित्रसेनां च निम्नगाम्}


\twolineshloka
{गोमतीं धूतपापां च वन्दनां च महानदीम्}
{कौशिकीं त्रिविदां कृत्यां निचितां लोहितारणीं}


\twolineshloka
{रहस्यां शतकुम्भां च सरयूं च तथैव च}
{चर्मण्वतीं वेत्रवतीं हस्तिसोमां दिशं तथा}


\twolineshloka
{शरावतीं पयोष्णीं च वेणां भीमरथीमपि}
{कावेरीं चुलुकां चापि वाणीं शतबलामपि}


\twolineshloka
{नीवारामहितां चापि सुप्रयोगां जनाधिप}
{पवित्रां कुण्डलीं सिन्धुं राजनीं पुरमालिनीम्}


\twolineshloka
{पूर्वाभिरामां वीरां च भीमामोघवतीं तथा}
{पाशाशिनीं पापहरां महेन्द्रां पाटलावतीम्}


\twolineshloka
{करीषिणीमसिक्नीं च कुशचीरां महानदीम्}
{मकरीं प्रवरां मेनां हेमां घृतवतीं तथा}


\twolineshloka
{पुरावतीमनुष्णां च शैब्यां कापीं च भारत}
{सदानीरामधृष्यां च कुशधारां महानदीम्}


\twolineshloka
{सदाकान्तां शिवां चैव तथा वीरवतीमपि}
{वस्त्रां सुवस्त्रां गौरीं च कम्पनां सहिरण्वतीम्}


\twolineshloka
{वरां वीरकरां चापि पञ्चमीं च महानदीम्}
{रथचित्रां ज्योतिरथां विश्वामित्रां कपिञ्जलाम्}


\twolineshloka
{उपेन्द्रां बहुलां चैव कुवीरामम्बुवाहिनीम्}
{विनदीं पिञ्जलां वेणां तुङ्गवेणां महानदीम्}


\twolineshloka
{विदिशां कृष्णवेणां च ताम्रां च कपिलामपि}
{खलुं सुवामां वेदाश्वां हरिश्रावां महापगाम्}


\twolineshloka
{शीघ्रां च पिच्छिलां चैव भारद्वाजीं च निम्नगाम्}
{कौशिकीं निम्नगां शोणा बाहुदामथ चन्द्रमाम्}


\twolineshloka
{दुर्गां चित्रशिलां चैव ब्रह्मवेध्यां बृहद्वतीम्}
{यवक्षामथ रोहीं च तथा जाम्बूनदीमपि}


\twolineshloka
{सुनसां तमसां दासीं वसामन्यां वराणसीम्}
{नीलां धृतवतीं चैव पर्णाशां च महानदीम्}


\twolineshloka
{मानवीं वृषभां चैव ब्रह्ममेध्यां बृहद्ध्वनीम्}
{एताश्चान्याश्च बहुधा महानद्यो जनाधिप}


\twolineshloka
{सदानिरामयां कृष्णां मन्दगां मन्दवाहिनीम्}
{ब्राह्मणीं च महागौरीं दुर्गामपि च भारत}


\twolineshloka
{चित्रोपलां चित्ररथां मञ्चुलां वाहिनीं तथा}
{मन्दाकिनीं वैतरणीं कोषां चापि महानदीम्}


\twolineshloka
{शुक्तिमतीमनङ्गां च तथैव वृषसाह्वयाम्}
{लोहित्यां करतोयां च तथैव वृषकाह्वयाम्}


\twolineshloka
{कुमारीमृषिकुल्यां च मारिषां च सरस्वतीम्}
{मन्दाकिनीं सुपुण्यां च सर्वां गङ्गां च भारत}


\twolineshloka
{विश्वस्य मातरः सर्वाः सर्वाश्चैव महाफलाः}
{तथा नद्यस्त्वप्रकाशाः शतशोऽथ सहस्रशः}


\twolineshloka
{इत्येताः सरितो राजन्समाख्याता यथास्मृति}
{अत ऊर्ध्वं जनपदान्निबोध गदतो मम}


\twolineshloka
{तत्रेमे कुरुपाञ्चालाः शाल्वा माद्रेयजाङ्गलाः}
{शूरसेनाः पुलिन्दाश्च बोधा मालास्तथैव च}


\twolineshloka
{मत्स्याः कुशल्याः सौशल्याः कुंतयः कांतिकोसलाः}
{चेदिमत्स्यकरूशाश्च भोजाः सिन्धुपुलिन्दकाः}


\twolineshloka
{उत्तमाश्च दशार्णाश्च मेकलाश्चोत्कलैः सह}
{पञ्चालाः कोसलाश्चैव नैकपृष्ठा धुरन्धराः}


\twolineshloka
{गोधा मद्रकलिङ्गाश्छ काशयोऽपरकाशयः}
{जठराः कुकुराश्चैव सदशार्णाश्च भारत}


\twolineshloka
{कुन्तयोऽवन्तयश्चैव तथैवापरकुन्तयः}
{गोमन्ता मन्दकाः सण्डा विदर्भा रूपवाहिकाः}


\twolineshloka
{अश्मकाः पाण्डुराष्ट्राश्च गोपराष्ट्राः करीतयः}
{अधिराज्यकुशाद्यश्च मल्लराष्ट्रं च केवलम्}


\twolineshloka
{वारवास्यायवाहाश्च चक्राश्चक्रातयः शकाः}
{विदेहा मगधाः स्वक्षा मलजा विजयास्तथा}


\twolineshloka
{अङ्गा वङ्गाः कलिङ्गाश्च यकृल्लोमान एव च}
{मल्लाः सुदेष्णाः प्रह्लादा माहिकाः शशिकास्तथा}


\twolineshloka
{बाह्लीका वाटधानाश्च आभीराः कालतोयकाः}
{अपरान्ताः परान्ताश्च पञ्चालाश्चर्ममण्डलाः}


\twolineshloka
{अटवीशिखराश्चैव मेरुभूताश्च मारिष}
{उपावृत्तानुपावृत्ताः स्वराष्ट्राः केकयास्तथा}


\twolineshloka
{कुन्दापरान्ता माहेयाः कक्षाः सामुद्रनिष्कुटाः}
{अन्ध्राश्च बहवो राजन्नन्तर्गिर्यास्तथैव च}


\twolineshloka
{बहिर्गिर्याङ्गमलजा मगधा मानवर्जकाः}
{समन्तराः प्रावृषेया भार्गवाश्च जनाधिप}


\twolineshloka
{पुण्ड्रा भर्गाः किराताश्च सुदृष्टा यामुनास्तथा}
{शका निषादा निषधास्तथैवानर्तनैर्ऋताः}


\twolineshloka
{दुर्गालाः प्रतिमत्स्याश्च कुन्तलाः कोसलास्तथा}
{तीरग्रहाः शूरसेना ईजिकाः कन्यका गुणाः}


\twolineshloka
{तिलभारा मसीराश्च मधुमन्तः सकुन्दकाः}
{काश्मीराः सिन्धुसौवीरा गान्धारा दर्शकास्तथा}


\twolineshloka
{अभीसारा उलूताश्च शैवला बाह्लिकास्तथा}
{दार्वी च वानवा दर्वा वातजामरथोरगाः}


\twolineshloka
{बहुवाद्याश्च कौरव्य सुदामानः सुमल्लिकाः}
{वध्राः करीषकाश्चापि कुलिन्दोपत्यकास्तथा}


\twolineshloka
{वनायवो दशापार्श्वरोमाणः कुशबिन्दवः}
{कच्छा गोपालकक्षाश्च जाङ्गलाः कुरुवर्णकाः}


\twolineshloka
{किराता बर्बराः सिद्धा वैदेहास्ताम्रलिप्तकाः}
{ओण्ड्रा म्लेच्छाः सैसिरिध्राः पार्वतीयाश्च मारिष}


\twolineshloka
{अथापरे जनपदा दक्षिणा भरतर्षभ}
{द्रविडाः केरलाः प्राच्या भूषिका वनवासिकाः}


\twolineshloka
{कर्णाटका महिषका विकल्पा मूषकास्तथा}
{झिल्लिकाः कुन्तलाश्चैव सौहृदानभकाननाः}


\twolineshloka
{कौकुट्टकास्तथा चोलाः कोङ्कणा मालवा नराः}
{समङ्गाः करकाश्चैव कुकुराङ्गारमारिषाः}


\twolineshloka
{ध्वजिन्युत्सवसंकेतास्त्रिगर्ताः साल्वसेनयः}
{व्यूकाः कोकबकाः प्रोष्ठाः सर्मवेगवशास्तथा}


\twolineshloka
{तथैव विन्ध्यचुलिकाः पुलिन्दा वल्कलैः सह}
{मालवा बल्लवाश्चैव तथैवापरबल्लवाः}


\twolineshloka
{कुलिन्दाः कालदाश्चैव कुण्डलाः करटास्तथा}
{मूषकास्तनबालाश्च सनीपा घटसृञ्जयाः}


\twolineshloka
{अठिदाः पाशिवाटाश्च तनयाः सुनयास्तथा}
{ऋषिका विदभाः काकास्तङ्गणाः परतङ्गणाः}


\twolineshloka
{उत्तराश्चापरे म्लेच्छाः क्रूरा भरतसत्तम}
{यवनाश्चीनकाभ्योजा दारुणा म्लेच्छजातयः}


\twolineshloka
{सकृद्ग्रहाः कुलत्थाश्च हूणाः पारसिकैः सह}
{तथैव रमणाश्चीनास्तथैव दशमालिकाः}


\twolineshloka
{क्षत्रिकयोपनिवेशाश्च वैश्यशूद्रकुलानि च}
{शूद्राभीराश्च दरदाः काश्मीराः पशुभिः सह}


\twolineshloka
{खाशीराश्चान्तचाराश्च पह्लवा गिरिगह्वराः}
{आत्रेयाः सभरद्वाजास्तथैव स्तनपोषिकाः}


\twolineshloka
{प्रोषकाश्च कलिङ्गाश्च किरातानां च जातयः}
{तोमरा हन्यमानाश्च तथैव करभञ्जकाः}


\twolineshloka
{एते चान्ये जनपदाः प्राच्योदीच्यास्तथैव च}
{उद्देशमात्रेण मया देशाः संकीर्तिता विभो}


\twolineshloka
{यथागुणबलं चापि त्रिवर्गस्य महाफलम्}
{दुह्याद्धेनुः कामधुक्क भूमिः सम्यगनुष्ठिता}


\twolineshloka
{तस्यां गृद्ध्यन्ति राजानः शूरा धर्मार्थक्रोविदाः}
{ते त्यजन्त्याहवे प्राणान्वसुगृद्धास्तरस्विनः}


\twolineshloka
{देवमानुषकायानां कामं भूमिः परायणम्}
{अन्योन्यस्यावलुम्पन्ति सारमेया यथाऽऽमिषम्}


\twolineshloka
{राजानो भरतश्रेष्ठ भोक्तुकामा वसुन्धराम्}
{न चापि तृप्तिः कामानां विद्यतेऽद्यापि कस्यचित्}


\twolineshloka
{तस्मात्परिग्रहे भूमेर्यतन्ते कुरुपाण्डवाः}
{साम्ना भेदेन दानेन दण्डेनैव च भारत}


\twolineshloka
{पिता भ्राता च पुत्राश्च खं द्यौश्च नरपुङ्गव}
{भूमिर्भवति भूतानां सम्यगच्छिद्रदर्शना}


\chapter{अध्यायः १०}
\twolineshloka
{धृतराष्ट्र उवाच}
{}


\twolineshloka
{भारतस्यास्य वर्षस्य तथा हैमवतस्य च}
{प्रमाणमायुषः सूत बलं चापि शुभाशुभम्}


\threelineshloka
{अनागतमतिक्रान्तं वर्तमानं च संजय}
{आचक्ष्व मे विस्तरेण हरिवर्षं तथैव च ॥संजय उवाच}
{}


\twolineshloka
{चत्वारि भारते वर्षे युगानि भरतर्षभ}
{कृतं त्रेता द्वापरं च तिष्यं च कुरुवर्धन}


\twolineshloka
{पूर्वं कृतयुगं नाम ततस्रेतायुगं प्रभो}
{संक्षेपाद्द्वापरस्याथ ततस्तिष्यं प्रवर्तते}


\twolineshloka
{चत्वारि तु सहस्राणि वर्षाणां कुरुसत्तम}
{आयुःसंख्या कृतयुगे संख्याता राजसत्तम}


\twolineshloka
{तथा त्रीणि सहस्राणि त्रेतायां मनुजाधिप}
{द्वे सहस्रे द्वापरे तु भुवि तिष्ठन्ति सांप्रतम्}


\twolineshloka
{न प्रमाणस्थितिर्ह्यस्ति तिष्येऽस्मिन्भरतर्षभ}
{गर्भस्थाश्च म्रियन्तेऽत्र तथा जाता म्रियन्ति च}


\twolineshloka
{महाबला महासत्वाः प्रज्ञागुणसमन्विताः}
{प्रजायन्ते च जाताश्च शतशोऽथ सहस्रशः}


\twolineshloka
{जाताः कृतयुगे राजन्धनिनः प्रयदर्शनाः}
{प्रजायन्ते च जाताश्च मुनयो वै तपोधनाः}


\twolineshloka
{महोत्साहा महात्मानो धार्मिकाः सत्यवादिनः}
{प्रियदर्शना वपुष्मन्तो महावीर्या धनुर्धराः}


\twolineshloka
{वरार्हा युधि जायन्ते क्षत्रियाः शूरसत्तमाः}
{त्रेतायां क्षत्रिया राजन्सर्वे वै चक्रवर्तिनः}


\twolineshloka
{सर्ववर्णाश्च जायन्ते सदा चैव च द्वापरे}
{महोत्साहा वीर्यवन्तः परस्परजयैषिणः}


\twolineshloka
{तेजसाल्पेन संयुक्ताः क्रोधनाः पुरुषा नृप}
{लुब्धा अनृतकाश्चैव तिष्ये जायन्ति भारत}


\twolineshloka
{ईर्ष्या मानस्तथा क्रोधो मायाऽसूया तथैव च}
{तिष्ये भवति भूतानां रागो लोभश्च भारत}


\twolineshloka
{संक्षेपो वर्तते राजन्द्वापरेऽस्मिन्नराधिप}
{गुणोत्तरं हैमवतं हरिवर्षं ततः परम्}


\chapter{अध्यायः ११}
\twolineshloka
{धृतराष्ट्र उवाच}
{}


\twolineshloka
{जम्बूखण्डस्त्वया प्रोक्तो यथावदिह सञ्जय}
{विष्कम्भमस्य प्रब्रूहि परिमाणं तु तत्त्वतः}


\twolineshloka
{समुद्रस्य प्रमाणं च सम्यगच्छिद्रदर्शनम्}
{शाकद्वीपं च मे ब्रूहि कुशद्वीपं च सञ्जय}


\threelineshloka
{शाल्मलिं चैव तत्त्वेन क्रौञ्चद्वीपं तथैव च}
{ब्रूहि गावल्गणे सर्वं राहोः सोमार्कयोस्तथा ॥सञ्जय उवाच}
{}


\twolineshloka
{राजन्सुबहवो द्वीपा यैरिदं सन्ततं जगत्}
{सप्त द्वीपान्प्रवक्ष्यामि चन्द्रादित्यौ ग्रहं तथा}


\twolineshloka
{अष्टादश सहस्राणि योजनानि विशांपते}
{षट् शतानि च पूर्णानि विष्कम्भो जम्बुपर्वतः}


\twolineshloka
{लावणस्य समुद्रस्य विष्कम्भो द्विगुणः स्मृतः}
{नानाजनपदाकीर्णो मणिविद्रुमचित्रितः}


\twolineshloka
{नैकधातुविचित्रैश्च पर्वतैरुपशोभितः}
{सिद्धचारणसंकीर्णः सागरः परिमण्डलः}


\twolineshloka
{शाकद्वीपं च वक्ष्यामि यथावदिह पार्थिव}
{शृणु मे त्वं यथान्यायं ब्रुवतः कुरुनन्दन}


\twolineshloka
{जम्बूद्वीपपरमाणेन द्विगुणः स नराधिप}
{विष्कम्भेण महाराज सागरोऽपि विभागशः}


\threelineshloka
{क्षीरोदो भरतश्रेष्ठ येन संपरिवारिताः}
{तत्र पुण्या जनपदास्तत्र न म्रियते जनः}
{कुत एव हि दुर्भिक्षं क्षमातेजोयुता हि ते}


\threelineshloka
{शाकद्वीपस्य संक्षेपो यथावद्भरतर्षभ}
{उक्त एष महाराज किमन्यत्कथयामि ते ॥धृतराष्ट्र उवाच}
{}


\threelineshloka
{शाकद्वीपस्य संक्षेपो यथावदिह सञ्जय}
{उक्तस्त्वया महाप्राज्ञ विस्तरं ब्रूहि तत्त्वतः ॥सञ्जय उवाच}
{}


\twolineshloka
{तथैव पर्वता राजन्सप्तात्र मणिभूषिताः}
{रत्नाकरास्तथा नद्यस्तेषां नामानि मे शृणु}


% Check verse!
अतीव गुणवत्सर्वं तत्र पुण्यं जनाधिप
\twolineshloka
{देवर्षिगन्धर्वयुतः प्रथमो मेरुरुच्यते}
{प्रागायतो महाराज मलयो नाम पर्वतः}


\twolineshloka
{ततो मेघाः प्रवर्तन्ते प्रभवन्ति च सर्वशः}
{ततः परेण कौरव्य जलधारो महागिरिः}


\twolineshloka
{ततो नित्युमुपादत्ते वासवः परमं जलम्}
{ततो वर्षं प्रभवति वर्षकाले जनेश्वर}


\twolineshloka
{उच्चैर्गिरी रैवतको यत्र नित्यं प्रतिष्ठिता}
{रेवती दिवि नक्षत्रं पितामहकृतो विधिः}


\fourlineindentedshloka
{उत्तरेण तु राजेन्द्र श्यामो नाम महागिरिः}
{नवमेघप्रभः प्रांशुः श्रीमानुज्ज्वलविग्रहः}
{यतः श्यामत्वमापन्नाः प्रजा जनपदेश्वर ॥धृतराष्ट्र उवाच}
{}


\threelineshloka
{सुमहान्संशयो मेऽद्य प्रोक्तोऽयं सञ्जय त्वया}
{प्रजाः कथं सूतपुत्र संप्राप्ताः श्यामतामिह ॥सञ्जय उवाच}
{}


\twolineshloka
{सर्वेष्वेव महाराज द्वीपेषु कुरुनन्दन}
{गौरः कृष्णश्च पतगस्तयोर्वर्णान्तरे नृप}


% Check verse!
श्यामो यस्मात्प्रवृत्तो वै तस्माच्छ्यामो गिरिःस्मृतः
\twolineshloka
{ततः परं कौरवेन्द्र दुर्गशैलो महोदयः}
{केसरः केसरयुतो यतो वातः प्रवर्तते}


\twolineshloka
{तेषां योजनविष्कम्भो द्विगुणः प्रविभागशः}
{वर्षाणि तेषु कौरव्य सप्तोक्तानि मनीषिभिः}


\twolineshloka
{महामेरुर्महाकाशो जलदः कुमुदोत्तरः}
{जलधारो महाराज सुकुमार इति स्मृतः}


\twolineshloka
{रेवतस्य तु कौमारः श्यामस्य मणिकाञ्चनः}
{केसरस्याथ मौदाकी परेण तु महापुमान्}


\twolineshloka
{परिवार्य तु कौरव्य दैर्घ्यं ह्रस्वत्वमेव च}
{जम्बूद्वीपेन संक्यातस्तस्य मध्ये महाद्रुमः}


\twolineshloka
{शाको नाम महाराज प्रजा तस्य सदानुगा}
{तत्र पुण्या जनपदाः पूज्यते तत्र शंकरः}


\twolineshloka
{तत्र गच्छन्ति सिद्धाश्च चारणा दैवतानि च}
{धार्मिकाश्च प्रजा राजंश्चत्वारोऽतीव भारत}


\twolineshloka
{वर्णाः स्वकर्मनिरता न च स्तेनोऽत्र दृश्यते}
{दीर्घायुषो महाराज जरामृत्युविवर्जिताः}


\twolineshloka
{प्रजास्तत्र विवर्धन्ते वर्षास्विव समुद्रगाः}
{नद्यः पुण्यजलास्तत्र गङ्गा च बहुधा गता}


\twolineshloka
{सुकुमारी कुमारी च शीताशी वेणिका तथा}
{महानदी च कौरव्य तथा मणिजला नदी}


\twolineshloka
{चक्षुर्वर्धनिका चैव नदी भरतसत्तम}
{तत्र प्रवृत्ताः पुण्योदा नद्यः कुरुकुलोद्वह}


\twolineshloka
{सहस्राणां शतान्येव यतो वर्षति वासवः}
{न तासां नामधेयानि परिमाणं तथैव च}


\twolineshloka
{शक्यन्ते परिसंख्यातुं पुण्यास्ता हि सरिद्वराः}
{तत्र पुण्या जनपदाश्चत्वारो लोकसंमताः}


\twolineshloka
{मङ्गाश्च मशकाश्चैव मानसा मन्दगास्तथा}
{मङ्गा ब्राह्मणभूयिष्ठाः स्वकर्मनिरता नृप}


\twolineshloka
{मशकेषु च राजन्या धार्मिकाः सर्वकामदाः}
{मानसाश्च महाराज वैश्यधर्मोपजीविनः}


\twolineshloka
{सर्वकामसमायुक्ताः शूरा धर्मार्थनिश्चिताः}
{शूद्रास्तु मन्दगा नित्यं पुरुषा धर्मशीलिनः}


\twolineshloka
{न तत्र राजा राजेन्द्र न दण्डो न च दण्डिकाः}
{स्वधर्मेणैव धर्मज्ञास्ते रक्षन्ति परस्परम्}


\twolineshloka
{एतावदेव शक्यं तु तत्र द्वीपे प्रभाषितुम्}
{एतदेव च श्रोतव्यं शाकद्वीपे महौजसि}


\chapter{अध्यायः १२}
\twolineshloka
{सञ्जय उवाच}
{}


\twolineshloka
{उत्तरेषु च कौरव्य द्वीपेषु श्रूयते कथा}
{एवं तत्र महाराज ब्रुवतश्च निबोध मे}


\twolineshloka
{घृततोयः समुद्रोऽत्र दधिमण्डोदकोऽपरः}
{सुरोदः सागरश्चैव तथान्यो जलसागरः}


\twolineshloka
{परस्परेण द्विगुणाः सर्वे द्वीपा नराधिप}
{पर्वताश्च महाराज समुद्रैः परिवारिताः}


\twolineshloka
{गौरस्तु मध्यमे द्वीपे गिरिर्मानःशिलो महान्}
{पर्वतः पश्चिमे कृष्णो नारायणसखो नृप}


\twolineshloka
{तत्र रत्नानि दिव्यानि स्वयं रक्षति केशवः}
{प्रसन्नश्चाभवत्तत्र प्रजानां व्यदधत्सुखम्}


\twolineshloka
{कुशस्तम्बः कुशद्वीपे मध्ये जनपदैः सह}
{संपूज्यते शाल्मलिश्च द्वीपे शाल्मलिके नृप}


\twolineshloka
{क्रौञ्चद्वीपे महाक्रौञ्चो गिरी रत्नचयाकरः}
{संपूज्यते महाराज चातुर्वर्ण्येन नित्यदा}


\twolineshloka
{गोमन्तः पर्वतो राजन्सुमहान्सर्वधातुमान्}
{यत्र नित्यं निवसति श्रीमान्कमललोचनः}


\twolineshloka
{मोक्षिभिः संस्तुतो नित्यं प्रभुर्नारायणो हरिः}
{कुशद्वीपे तु राजेन्द्र पर्वतो विद्रुमैश्चितः}


\twolineshloka
{सुनामा नाम दुर्धर्षो द्वितीयो हेमपर्वतः}
{द्युतिमान्नाम कौरव्य तृतीयः कुमुदो गिरिः}


\twolineshloka
{चतुर्थः पुष्पवान्नाम पञ्चमस्तु कुशेशयः}
{षष्ठो हरिगिरिर्नाम षडेते पर्वतोत्तमाः}


\twolineshloka
{तेषामन्तरविष्कम्भो द्विगुणः सर्वभागशः}
{औद्भिदं प्रथमं वर्षं द्वितीयं वेणुमण्डलम्}


\twolineshloka
{तृतीयं सुरथाकारं चतुर्थं कम्बलं स्मृतम्}
{धृतिमत्पञ्चमं वर्षं षष्ठं प्रभाकरम्}


\twolineshloka
{सप्तमं कापिलं वर्षं सप्तैते वर्षलम्भकाः}
{एतेषु देवगन्धर्वाः प्रजाश्च जगतीश्वर}


\twolineshloka
{विहरन्ते रमन्ते च न तेषु म्रियते जनः}
{न तेषु दस्यवः सन्ति म्लेच्छजात्योपि वा नृप}


\twolineshloka
{गौरप्रायो जनः सर्वः सुकुमारश्च पार्थिव}
{अवशिष्टेषु सर्वेषु वक्ष्यामि मनुजेश्वर}


\twolineshloka
{यथाश्रुतं महाराज तदव्यग्रमनाः श्रृणु}
{क्रौञ्चद्वीपे महारा क्रौञ्चो नाम महागिरिः}


\threelineshloka
{क्रौञ्चात्परो वामनको वामनादन्धकारकः}
{अन्धकारात्परो राजन्मैनाकः पर्वतोत्तमः}
{}


\twolineshloka
{मैनाकात्परतो रादन्गोविन्दो गिरिरुत्तमः}
{गोविन्दात्परतो राजन्निबिडो नाम पर्वतः}


\twolineshloka
{परस्तु द्विगुणस्तेषां विष्कम्भो वंशवर्धन}
{देशांस्तत्र प्रवक्ष्यामि तन्मे निगदतः श्रृणु}


\twolineshloka
{क्रोञ्चस्य कुशलो देशो वामनस्य मनोनुगः}
{मनोनुगात्परश्चोष्णो देशः कुरुकुलोद्वह}


\twolineshloka
{उष्णात्परः प्रावरकः प्रावारादन्धकारकः}
{अन्धकारकदेशात्तु मुनिदेशः परः स्मृतः}


\twolineshloka
{मुनिदेशात्परश्चैव प्रोच्यते दुन्दुभिस्वनः}
{सिद्धचारणसंकीर्णो गौरप्रायो जनाधिप}


\twolineshloka
{एते देशा महाराज देवगन्धर्वसेविताः}
{पुष्करे पुष्करो नाम पर्वतो मणिरत्नवान्}


\twolineshloka
{तत्र नित्यं प्रभवति स्वयं देवः प्रजापतिः}
{तं पर्युपासते नित्यं देवाः सर्वे महर्षयः}


\twolineshloka
{वाग्भिर्मनोनुकूलाभिः पूजयन्तो जनाधिप}
{जम्बूद्वीपात्प्रवर्तन्ते रत्नानि विविधान्युत}


\twolineshloka
{द्वीपेषु तेषु सर्वेषु प्रजानां कुरुसत्तम}
{ब्रह्मचर्येण सत्येन प्रजानां हि दमेन च}


\threelineshloka
{आरोग्यायुःप्रमाणाभ्यां द्विगुणं द्विगुणं ततः}
{एको जनपदो राजन्द्वीपेष्वेतेषु भारत}
{उक्ता जनपदा येषु धर्मश्चैकः प्रदृश्यते}


\twolineshloka
{ईश्वरो दण्डमुद्यम्य स्वयमेव प्रजापतिः}
{द्वीपानेतान्महाराज रक्षंस्तिष्ठति नित्यदा}


\twolineshloka
{स राजा स शिवो राजन्स पिता प्रतितामहः}
{गोपायति नरश्रेष्ठ प्रजाः सजडपण्डिताः}


\twolineshloka
{भोजनं चात्र कौरव्य प्रजाः स्वयमुपस्थितम्}
{सिद्धमेव महाबाहो तद्धि भुञ्जन्ति नित्यदा}


\twolineshloka
{ततः परं समा नाम दृश्यते लोकसंस्थितिः}
{चतुरश्र महाराज त्रयस्त्रिंशत्तु मण्डलम्}


\twolineshloka
{तत्र तिष्ठन्ति कौरव्य चत्वारो लोकसंमताः}
{दिग्गजा भरतश्रेष्ठ वामनैरावतादयः}


\twolineshloka
{सुप्रतीकस्तदा राजन्प्रभिन्नकरटामुखः}
{तस्याहं परिमाणं तु न संख्यातुमिहोत्सहे}


\twolineshloka
{असंख्यातः स नित्यं हि तिर्यगूर्ध्वमधस्तथा}
{तत्र वै वायवो वान्ति दिग्भ्यः सर्वाभ्य एव हि}


\twolineshloka
{असंबद्धा महाराज तान्निगृह्णन्ति ते गजाः}
{पुष्करैः पद्मसंकाशैर्विकसद्भिर्महाप्रभैः}


\fourlineindentedshloka
{शतधा पुनरेवाशु ते तान्मुञ्चन्ति नित्यशः}
{श्वसद्भिर्मुच्यमानास्तु दिग्गजैरिह मारुताः}
{आगच्छन्ति महाराज ततस्तिष्ठन्ति वै प्रजाः ॥धृतराष्ट्र उवाच}
{}


\threelineshloka
{परो वै विस्तरोऽत्यर्थं त्वया सञ्जय कीर्तितः}
{दर्शितं द्वीपसंस्थानमुत्तरं ब्रूहि सञ्जय ॥सञ्जय उवाच}
{}


\twolineshloka
{उक्ता द्वीपा महाराज ग्रहं वै शृणु तत्त्वतः}
{स्वर्भानोः कौरवश्रेष्ठ यावदेव प्रमाणतः}


\twolineshloka
{परिमण्डलो महाराज स्वर्भानुः श्रूयते ग्रहः}
{योजनानां सहस्राणि विष्कम्भो द्वादशास्य वै}


\twolineshloka
{परिणाहेन षट्त्रिंशद्विपुलत्वेन चानघ}
{षष्टिमाहुः शतान्यस्य बुधाः पौराणिकास्तथा}


\twolineshloka
{चन्द्रमास्तु सहस्राणि राजन्नेकादश स्मृतः}
{विष्कम्भेण कुरुश्रेष्ठ त्रयस्त्रिंशत्तु मण्डलम्}


% Check verse!
एकोनषष्टिविष्कम्भं शीतरश्मेर्महात्मनः
\twolineshloka
{सूर्यस्त्वष्टौ सहस्राणि द्वे चान्ये कुरुनन्दन}
{विष्कम्भेण ततो राजन्मण्डलं त्रिंशता समम्}


\twolineshloka
{अष्टपञ्चाशतं राजन्विपुलत्वेन चानघ}
{श्रूयते परमोदारः पतगोऽसौ विभावसुः}


\twolineshloka
{एतत्प्रमाणमर्कस्य निर्दिष्टमिह भारत}
{स राहुश्छादयत्येतौ यथाकालं महात्तया}


\twolineshloka
{चन्द्रादित्यौ महाराज संक्षेपोऽयमुदाहृतः}
{इत्येतत्ते महाराज पृच्छतः शास्त्रचक्षुषा}


\twolineshloka
{सर्वमुक्तं यथातत्त्वं तस्माच्छममवाप्नुहि}
{यथोद्दिष्टं मया प्रोक्तं सनिर्माणमिदं जगत्}


\twolineshloka
{तस्मादाश्वस कौरव्य पुत्रं दुर्योधनं प्रति}
{श्रुत्वेदं भरतश्रेष्ठ भूमिपर्व मनोनुगम्}


\twolineshloka
{श्रीमान्भवति राजन्यः सिद्धार्थः साधुसंमतः}
{आयुर्बलं च कीर्तिश्च तस्य तेजश्च वर्धते}


\twolineshloka
{यः शृणोति महीपाल पर्वणीदं यतव्रतः}
{प्रीयन्ते पितरस्तस्य तथैव च पितामहाः}


\twolineshloka
{इदं तु भारतं वर्षं यत्र वर्तामहे वयम्}
{पूर्वैः प्रवर्तितं पुण्यं तत्सर्वं श्रुतवानसि}


\chapter{अध्यायः १३}
\twolineshloka
{वैशंपायन उवाच}
{}


\twolineshloka
{अथ गावल्गणिर्विद्वान्संयुगादेत्य भारत}
{प्रत्यक्षदर्शी सर्वस्य भूतभव्यभविष्यवित्}


\threelineshloka
{ध्यायते धृतराष्ट्राय सहसोत्पत्य दुःखितः}
{आचष्ट निहतं भीष्मं भरतानां पितामहम् ॥सञ्जय उवाच}
{}


\twolineshloka
{सञ्जयोऽहं महाराज नमस्ते भरतर्षभ}
{हतो भीष्मः शान्तनवो भरतानां पितामहः}


\twolineshloka
{ककुदं सर्वयोधानां धाम सर्वधनुष्मताम्}
{शरतल्पगतः सोऽद्य शेते कुरुपितामहः}


\twolineshloka
{यस्य वीर्यं समाश्रित्य द्यूतं पुत्रस्तवाकरोत्}
{स शेते निहतो राजन्संख्ये भीष्मः शिखण्डिना}


\twolineshloka
{यः सर्वान्पृथिवीपालान्समवेतान्महामृधे}
{जिगायैकरथेनैव काशिपुर्यां महारथः}


\twolineshloka
{जामदग्न्यं रणे रामं योऽयुध्यदपसंभ्रमः}
{न हतो जामदग्न्येन स हतोऽद्य शिखण्डिना}


\twolineshloka
{महेन्द्रसदृशः शौर्ये स्थैर्ये च हिमवानिव}
{समुद्र इव गाम्भीर्यो सहिष्णुत्वे धरासमः}


\twolineshloka
{शरदंष्ट्रो धनुर्वक्रः खङ्गिजिह्वो दुरासदः}
{नरसिंहः पिता तेऽद्य पाञ्चाल्येन निपातितः}


\twolineshloka
{पाण्डवानां महासैन्यं यं दृष्ट्वोद्यतमाहवे}
{प्रावेपत भयोद्विग्नं सिंह दृष्ट्वेव गोगणः}


\twolineshloka
{परिरक्ष्य स सेनां ते दशरात्रमनीकहा}
{जगामास्तमिवादित्यः कृत्वा कर्म सुदुष्करम्}


\twolineshloka
{यः स शक्र इवाक्षोभ्यो वर्षन्बाणान्सहस्रशः}
{जघान युधि योधानामर्बुदं दशभिर्दिनैः}


\twolineshloka
{स शेते निहतो भूमौ वातभग्न इव द्रुमः}
{तव दुर्मन्त्रिते राजन्यथा नार्हः स भारत}


\chapter{अध्यायः १४}
\twolineshloka
{धृतराष्ट्र उवाच}
{}


\twolineshloka
{कथं कुरूणामृषभो हतो भीष्मः शिखण्डिना}
{कथं रथार्त्स न्यपतित्पिता मे वासवोपमः}


\twolineshloka
{कथमाचक्ष्व मे योधा हीना भीष्मेण संजय}
{बलिना देवकल्पेन गुर्वर्थे ब्रह्मचारिणा}


\twolineshloka
{तस्मिन्हते महाप्राज्ञे महेष्वासे महाबले}
{महासत्वे नरव्याघ्रे किमु आसीन्मनस्तव}


\twolineshloka
{आर्तिं परामाविशति मनः शंससि मे हतम्}
{कुरूणामृषभं वीरमकम्पं पुरुषर्षभम्}


\twolineshloka
{के तं यान्तमनुप्राप्ताः के वास्यसन्पुरोगमाः}
{केऽतिष्ठन्के न्यवर्तन्त केऽन्ववर्तन्त सञ्जय}


\twolineshloka
{के शूरा रथशार्दूलमद्भुतं क्षत्रियर्षभम्}
{तथाऽनीकं गाहमानं सहसा पृष्ठतोऽन्वयुः}


\twolineshloka
{यस्तमोर्क इवापोहन्परसैन्यममित्रहा}
{सहस्ररश्मिप्रतिमः परेषां भयमादधत्}


\twolineshloka
{अकरोद्दुष्करं कर्म रणे पाण्डुसुतेषु यः}
{ग्रसमानमनीकानि य एनं पर्यवारयन्}


\twolineshloka
{कृतिनं तं दुराधर्षं सञ्जयास्य त्वमन्तिके}
{कथं शान्तनवं युद्धे पाण्डवाः प्रत्यवारयन्}


\twolineshloka
{निकृन्तन्तमनीकानि शरदंष्ट्रं मनस्विनम्}
{चापव्यात्ताननं घोरमसिजिह्वं दुरासदम्}


\twolineshloka
{अनर्हं पुरुषव्याघ्रं ह्रीमन्तमपराजितम्}
{पातयामास कौन्तेयः कथं तमजितं युधि}


\twolineshloka
{उग्रधन्वानमुग्रेषुं वर्तमानं रथोत्तमे}
{परेषामुत्तमाङ्गानि प्रचिन्वन्तमथेषुभिः}


\twolineshloka
{पाण्डवानां महत्सैन्यं यं दृष्ट्वोद्यतमाहवे}
{कालाग्निमिव दुर्धर्षं समचेष्टत नित्यशः}


\twolineshloka
{परिकृष्य स सेनां तु दशरात्रमनीकहा}
{दगामास्तमिवादित्यः कृत्वा कर्म सुदुष्करम्}


\twolineshloka
{यः स शक्र इवाक्षय्यं वर्षं शरमयं क्षिपन्}
{जघान युधि योधानामर्बुदं दशभिर्दिनैः}


\twolineshloka
{स शेते निहतो भूमौ वातभुग्न इव द्रुमः}
{मम दुर्मन्त्रितेनाजौ यथा नार्हति भारतः}


\twolineshloka
{कथं शान्तनवं दृष्ट्वा पाण्डवानामनीकिनी}
{प्रहर्तुमशकत्तत्र भीष्मं भीमपराक्रमम्}


\twolineshloka
{कथं भीष्मेण संग्रामं प्राकुर्वन्पाण्डुनन्दनाः}
{कथं च नाजयद्भीष्मो द्रोणे जीवति सञ्जय}


\twolineshloka
{कृपे सन्निहिते तत्र भरद्वाजात्मजे तथा}
{भीष्मः प्रहरतां श्रेष्ठः कथं स निधनं गतः}


\twolineshloka
{कथं चातिरथस्तेन पाञ्चाल्येन शिखण्डिना}
{भीष्मो विनिहतो युद्धे देवैरपि दुरासदः}


\twolineshloka
{यः स्पर्धते रणे नित्यं जामदग्न्यं महाबलम्}
{अजितं जामदग्न्येन शक्रतुल्यपराक्रमम्}


\twolineshloka
{तं हतं समरे भीष्मं महारथकुलोदितम्}
{सञ्जयाचक्ष्व मे वीरं येन शर्म न विद्महे}


\twolineshloka
{मामकाः के महेष्वासा नाजहुः सञ्जयाच्युतम्}
{दुर्योधनसमादिष्टाः के वीराः पर्यवारयन्}


\twolineshloka
{यच्छिखण्डिमुखाः सर्वे पाण्डवा भीष्ममभ्ययुः}
{कच्चित्ते कुरवः सर्वे नाजहुः सञ्जयाच्युतम्}


\twolineshloka
{अश्मसारमयं नूनं हृदयं मुदृढं मम}
{यच्छ्रुत्वा पुरुषव्याघ्रं हतं भीष्मं न दीर्यते}


\twolineshloka
{यस्मिन्सत्यं च मेधा च नीतिश्च भरतर्षभे}
{अप्रमेयाणि दुर्धर्षे कथं स निहतो युधि}


\twolineshloka
{मौर्वीघोषस्तनयुत्नुः पृषत्कपृषतो महान्}
{धनुर्ह्रादमहाशब्दो महामेघ इवोन्नतः}


\twolineshloka
{योऽभ्यवर्षत कौन्तेयान्सपाञ्चालान्ससृञ्जयान्}
{निघ्नन्परथान्वीरो दानवानिव वज्रभृत्}


\twolineshloka
{इष्वस्त्रसागरं घोरं बाणग्रहं दुरासदम्}
{कार्मुकोर्मिणमक्षय्यमद्वीपं चलमप्लवम्}


\twolineshloka
{गदासिमकरावासं हयावर्त्तं गजाकुलम्}
{पदातिमत्स्यकलिलं शङ्खदुन्दुभिनिःस्वनम्}


\twolineshloka
{हयान्गजपदातींश्च रथांश्च तरसा बहून्}
{निमञ्जयन्तं समरे परवीरापहारिणम्}


\twolineshloka
{विदह्यमानं कोपेन तेजसा च परंतपम्}
{वेलेव मकरावासं के वीराः पर्यवारयन्}


\twolineshloka
{भीष्मो यदकरोत्कर्म समरे सञ्जयारिहा}
{दुर्योधनहितार्थाय के तस्यास्य पुरोऽभवन्}


\twolineshloka
{के रक्षन्दक्षिणं चक्रं भीष्मस्यामिततेजसः}
{पृष्ठतः के परान्वीरानपासेधन्यतव्रताः}


\twolineshloka
{के पुरस्तादवर्तन्त रक्षन्तो भीष्ममन्तिके}
{के रक्षन्नुत्तरं चक्रं वीरा वीरकस्य युध्यतः}


\twolineshloka
{वामे चक्रे वर्तमानाः केऽघ्नन्सञ्जय सृञ्जयान्}
{अग्नतोऽग्न्यमनीकेषु केऽभ्यरक्षन्दुरासदम्}


\twolineshloka
{पार्श्वतः केऽभ्यरक्षन्त गच्छन्तो दुर्गमां गतिम्}
{समूहे के परान्वीरान्प्रत्ययुध्यन्त सञ्जय}


\twolineshloka
{रक्ष्यमाणः कथं वीरैर्गोप्यमानाश्च तेन ते}
{दुर्जयानामनीकानि नाजयंस्तरसा युधि}


\twolineshloka
{सर्वलोकेश्वरस्येव परमस्य प्रजापतेः}
{कथं प्रहर्तुमपि ते शेकुः सञ्जय पाण्डवाः}


\twolineshloka
{यस्मिन्द्वीपे समाश्वस्य युध्यन्ते कुरवः परैः}
{तं निमग्नं नरव्याघ्रं भीष्मं शंससि सञ्जय}


\twolineshloka
{यस्य वीर्यं समाश्रित्य मम पुत्रो बृहद्बलः}
{न पाण्डवानगणयत्कथं स निहतः परैः}


\twolineshloka
{यः पुरा विबुधैः सर्वैः सहाये युद्धदुर्मदः}
{काङ्क्षितो दानवान्ध्नद्भिः पिता मम महाव्रतः}


\twolineshloka
{यस्मिज्जाते महावीर्ये शान्तनुर्लोकविश्रुतः}
{शोकं दैन्यं च दुःस्वं च प्राजहात्स च तत्क्षणे}


\twolineshloka
{प्रोक्तं परायणं प्राज्ञं स्वधर्मनिरतं शुचिम्}
{वेदवेदाङ्गतत्वज्ञं कथं शंससि मे हतम्}


\twolineshloka
{सर्वास्त्रविनयोपेतं शान्तं दान्तं मनस्विनम्}
{हतं शान्तनवं श्रुत्वा मन्ये शेषं हतं बलम्}


\twolineshloka
{धर्मादधर्मो बलवान्संप्राप्त इति मे मतिः}
{यत्र वृद्धं गुरुं हत्वा राज्यमिच्छन्ति पाण्डवाः}


\twolineshloka
{जामदग्न्यः पुरा रामः सर्वास्त्रविदनुत्तमः}
{अम्बार्थमुद्यतः सङ्ख्ये भीष्मेण युधि निर्जितः}


\twolineshloka
{तमिन्द्रसमकर्माणं ककुदं सर्वधन्विनाम्}
{हतं शंससि मे भीष्मं किं नु दुःखमतः परम्}


\threelineshloka
{असकृत्क्षत्रियव्राताः सङ्ख्ये येन विनिर्जिताः}
{जामदग्न्येन वीरेण परवीरनिघातिना}
{न हतो यो महाबुद्धिः स हतोऽद्य शिखण्डिना}


\twolineshloka
{तस्मान्नूनं महावीर्याद्भार्गवाद्युद्धदुर्मदात्}
{तेजोवीर्यबलैर्यूयाञ्शिखण्डी द्रुपदात्मजः}


\twolineshloka
{यः शूरं कृतिनं युद्धे सर्वशस्त्रविशारदम्}
{परमास्त्रविदं वीरं जघान भग्तर्षभम्}


\twolineshloka
{के वीरास्तममित्रघ्नमन्वयुः शस्त्रसंसदि}
{शंस मे तद्यथा चासीद्युद्भं भीष्मस्य पाण्डवैः}


\twolineshloka
{योषेव हतवीरा मे सेन्म पुत्रस्य सञ्जय}
{अगोपमिव चोद्भ्रान्तं गोकुलं तद्बलं मम}


\twolineshloka
{पौरुषं सर्वलोकस्य परं यस्मिन्महाहवे}
{परासक्ते च वस्तस्मिन्कथमासीन्मनस्तदा}


\twolineshloka
{जीवितेऽप्यद्य सामर्थ्यं किमिवास्मासु सञ्जय}
{घातयित्वा महावीर्यं पितरं लोकधार्मिकम्}


\twolineshloka
{अगाधे सलिले मग्नां नावं दृष्ट्वेव पारगाः}
{भीष्मे हते भृशं दुःखान्मन्ये शोचन्ति पुत्रकाः}


\twolineshloka
{अद्रिसारमयं नूनं हृदयं मम सञ्जय}
{यच्छ्रुत्वा पुरुषव्याघ्रं हतं भीष्मं न दीर्यते}


\twolineshloka
{यस्मिन्नस्त्राणि मेधा च नीतिश्च पुरुषर्षभे}
{अप्रमेयाणि दुर्धर्षे कथं स निहतो युधि}


\twolineshloka
{न चास्त्रेण न शौर्येण तपसा मेधया न च}
{न धृत्या न पुनस्त्यागान्मृत्योः कश्चिद्विमुच्यते}


\twolineshloka
{कालो नूनं महावीर्यः सर्वलोकदुरत्ययः}
{यत्र शान्तनवं भीष्मं हतं शंससि सञ्जय}


\twolineshloka
{पुत्रशोकाभिसंतप्तो महद्दुःखमचिन्तयन्}
{आशंसेऽहं परं त्राणं भीष्माच्छान्तनुनन्दनात्}


\twolineshloka
{यदादित्यमिवापश्यत्पतितं भ्रुवि सञ्जय}
{दुर्योधनः शान्तनवं किं तदा प्रत्यपद्मत}


\twolineshloka
{नाहं स्वेषां परेषां वा बुद्ध्या सञ्जय चिन्तयन्}
{शेषं किंचित्प्रपश्यामि प्रत्यनीके महीक्षिताम्}


\twolineshloka
{दारुणः क्षत्रधर्मोऽयमृषिभिः संप्रदर्शितः}
{यत्र शान्तनवं हत्वा राज्यमिच्छन्ति पाण्डवाः}


\twolineshloka
{वयं वा राज्यमिच्छामो घातयित्वा महाव्रतम्}
{क्षत्रधर्मे स्थिताः पार्था नापराध्यन्ति पुत्रकाः}


\twolineshloka
{एतदार्येण कर्तव्यं कृच्छ्रास्वापत्सु सञ्जय}
{पराक्रमः पराशक्तिस्तत्तु तस्मिन्प्रतिष्ठितम्}


\twolineshloka
{अनीकानि विनिघ्नन्तं ह्रीमन्तमपराजितम्}
{कथं शान्तनवं तातं पाण्डुपुत्रा न्यवारयन्}


\twolineshloka
{यथायुक्तान्यनीकानि कथं युद्धं महात्मभिः}
{कथं वा निहतो भीष्मः पिता सञ्जय मे परैः}


\twolineshloka
{दुर्योधनश्च कर्णश्च शकुनिश्चापि सौबलः}
{दुःशासनश्च कितवो हते भीष्मे किमब्रुवन्}


\threelineshloka
{यच्छरीरैरुपास्तीर्णां नरवारणवाजिनाम्}
{शरशक्तिमहाखङ्गतोमराक्षां महाभयाम्}
{प्राविशन्कितवा मन्दाः सभां युद्धदुरासदाम्}


\threelineshloka
{प्राणद्यूते प्रतिभये केऽदीव्यन्त नरर्षभाः}
{के जीयन्ते जितास्तत्र कृतलक्ष्या निपातिताः}
{अन्ये भीष्माच्चान्तनवात्तन्ममाचक्ष्व सञ्जय}


\twolineshloka
{न हि मे शान्तिरस्तीह श्रुत्वा देवव्रतं हतम्}
{पितरं भीमकर्माणं भीष्ममाहवशोभिनम्}


\twolineshloka
{आर्तिं मे हृदये रूढां महतीं पुत्रहानिजाम्}
{त्वं हि मे सर्पिषेवाग्निमुद्दीपयसि सञ्जय}


\twolineshloka
{महान्तं भारमुद्यम्य विश्रुतं सार्वलौकिकम्}
{दृष्ट्वा विनिहतं भीष्मं मन्ये शोचन्ति पुत्रकाः}


\twolineshloka
{श्रोष्यामि तानि दुःखानि दुर्योधनकृतान्यहम्}
{तस्मान्मे सर्वमाचक्ष्व यद्वृत्तं तत्र सञ्जय}


\twolineshloka
{यद्वृत्तं तत्र संग्रामे मन्दस्याबुद्धिसंभवम्}
{अपनीतं सुनीतं यत्तन्ममाचक्ष्व सञ्जय}


\twolineshloka
{यत्कृतं तत्र संग्रामे भीष्मेण जयमिच्छता}
{तेजोयुक्तं कृतास्त्रेण शंस तच्चाप्यशेषतः}


\twolineshloka
{यथा तदभवद्युद्धं कुरुपाण्डवसेनयोः}
{क्रमेण येन यस्मिंश्च काले यच्च यथाऽभवत्}


\chapter{अध्यायः १५}
\twolineshloka
{सञ्जय उवाच}
{}


\twolineshloka
{त्वद्युक्तोऽयमनुप्रश्नो महाराज यथार्हसि}
{न तु दुर्योधने दोषमिममासङ्क्तुमर्हसि}


\twolineshloka
{य आत्मनो दुश्चरितादशुभं प्राप्नुयान्नरः}
{एनसा तेन नान्यं स उपाशङ्कितुमर्हति}


\twolineshloka
{महाराज मनुष्येषु निन्द्यं यः सर्वमाचरेत्}
{स वध्यः सर्वलोकस्य निन्दितानि समाचरन्}


\twolineshloka
{निकारो निकृतिप्रज्ञैः पाण्डवैस्त्वत्प्रतीक्षया}
{अनुभूतः सहामात्यैः क्षान्तश्च सुचिरं वने}


\twolineshloka
{हयानां च गजानां च राज्ञां चामिततेजसाम्}
{प्रत्यक्षं यन्मया दृष्टं दृष्टं योगबलेन च}


\twolineshloka
{शृणु तत्पृथिवीपाल मा च शोके मनः कृथाः}
{दिष्टमेतत्पुरा नूनमिदमेव नराधिप}


\twolineshloka
{नमस्कृत्वा प्रवक्ष्यामि पाराशर्याय धीमते}
{यस्य प्रसादाद्दिव्यं तत्प्राप्तं ज्ञानमनुत्तमम्}


\twolineshloka
{दृष्टिश्चातीन्द्रिया राजन्दूराच्छ्रवणमेव च}
{परचित्तस्य विज्ञानमतीतानागतस्य च}


\twolineshloka
{व्युत्थितोत्पत्तिविज्ञानमाकाशे च गतिः शुभा}
{अस्त्रैरसङ्गो युद्धेषु वरदानान्महात्मनः}


\twolineshloka
{श्रृणु मे विस्तरेणेदं विचित्रं परमाद्भुतम्}
{भरतानामभूद्युद्धं यथा तद्रोमहर्षणम्}


\twolineshloka
{तेष्वनीकेषु यत्तेषु व्यूढेषु च विधानतः}
{दुर्योधनो महाराज दुःशासनमथाब्रवीत्}


\twolineshloka
{दुःशासन रथास्तूर्णं युज्यन्तां भीष्मरक्षिणः}
{अनीकानि च सर्वाणि शीघ्रं त्वमनुचोदय}


\twolineshloka
{अयं स मामभिप्राप्तो वर्षपूगाभिचिन्तितः}
{पाण्डवानां ससैन्यानां कुरूणां च समागमः}


\twolineshloka
{नातः कार्यतमं मन्ये रणे भीष्मस्य रक्षणात्}
{हन्युद्गुप्तो ह्यसौ पार्थान्सोमकांश्च ससृञ्जयान्}


\twolineshloka
{अब्रवीच्च विशुद्धात्मा नाहं हन्यां शिखण्डिनाम्}
{श्रूयते स्त्री ह्यसौ पूर्वं तस्मादूर्त्यो रणे मम}


\twolineshloka
{तस्माद्भीष्मो रक्षितव्यो विशेषेणेति मे मतिः}
{शिखण्डिनो वधे यत्ताः सर्वे तिष्ठन्तु मामकाः}


\twolineshloka
{तथा प्राच्याः प्रतीच्याश्च दाक्षिणात्योत्तरापथाः}
{सर्वथाऽस्त्रेषु कुशलास्ते रक्षन्तु पितामहम्}


\twolineshloka
{अरक्ष्यमाणं हि वृको हन्यात्सिंहं महाबलम्}
{मा सिंहं जम्बुकेनेव घातयामः शिखण्डिना}


\threelineshloka
{वामं चक्रं युधामन्युरुत्तमौजाश्च दक्षिणम्}
{गोप्तारौ फाल्गुनं प्राप्तौ फाल्गुनोपि शिखण्डिनः}
{}


\twolineshloka
{संरक्ष्यमाणः पार्थेन भीष्मेण च विवर्जितः}
{यथा न हन्याद्गाङ्गेयं दुःशासन तथा कुरु}


\chapter{अध्यायः १६}
\twolineshloka
{सञ्जय उवाच}
{}


\twolineshloka
{ततो रजन्यां व्युष्टायां शब्दः समभवन्महान्}
{क्रोशतां भूमिपलानां युज्यतां युज्यतामिति}


\twolineshloka
{शङ्खदुन्दुभिघोषैश्च सिंहनादैश्च भारत}
{हयहेषितनादैश्च रथनेमिस्वनैस्तथा}


\twolineshloka
{गाजानां बृहतां चैव योधानां चापि गर्जताम्}
{क्ष्वेलितास्फोटितोत्क्रुष्टैस्तुमुलं सर्वतोऽभवत्}


\twolineshloka
{उदतिष्ठन्महाराज सर्वं युक्तमशेषतः}
{सूर्योदये महत्सैन्यं कुरुपाण्डवसेनयोः}


\twolineshloka
{राजेन्द्र तव पुत्राणां पाण्डवानां तथैव च}
{दुष्प्रधृष्याणि चास्त्राणि सशस्रकवचानि च}


\twolineshloka
{ततः प्रकाशे सैन्यानि समदृश्यन्त भारत}
{त्वदीयानां परेषां च शस्त्रवन्ति महान्ति च}


\twolineshloka
{तत्र नागा रथाश्चैव जाम्बूनदपरिष्कृताः}
{विभ्राजमाना दृश्यन्ते मेघा इव सविद्युतः}


\twolineshloka
{रथानीकान्यदृश्यन्त नगराणीव भूरिशः}
{अतीव शुशुभे तत्र पिता ते पूर्णचन्द्रवत्}


\twolineshloka
{धनुर्भिर्ऋष्टिभिः खङ्गैर्गदाभिः शक्तितोमरैः}
{योधाः प्रहरणैः शुभ्रैस्तेष्वनीकेष्ववस्थिताः}


\twolineshloka
{गजाः पदाता रथिनस्तुरगाश्च विशांपते}
{व्यतिष्ठन्वागुराकाराः शतशोऽथ सहस्रशः}


\twolineshloka
{ध्वजा बहुविधाकारा व्यदृश्यन्त समुच्छ्रिताः}
{स्वेषां चैव परेषां च द्युतिमन्तः सहस्रशः}


\twolineshloka
{काञ्चना मणिचित्राङ्गा ज्वलन्त इव पावकाः}
{अर्चिष्मन्तो व्यरोचन्त गजारोहाः सहस्रशः}


\twolineshloka
{महेन्द्रकेतवः शुभ्रा महेन्द्रसदनेष्विव}
{सन्नद्धास्ते प्रवीराश्च ददृशुर्युद्धकाङ्क्षिणः}


\twolineshloka
{उद्यतैरायुधैश्चित्रास्तलबद्धाः कलापिनः}
{ऋषभाक्षा मनुष्येन्द्राश्चमूमुखगता बभुः}


\twolineshloka
{शकुनिः सौबलः शल्य आवन्त्योऽथ जयद्रथः}
{विन्दानुविन्दौ कैकेयाः काम्भोजश्च सुदक्षिणः}


\twolineshloka
{श्रुतायुधश्च कालिङ्गो जयत्सेनश्च पार्थिवः}
{बृहद्बलश्च कौशल्यः कृतवर्मा च सात्वतः}


\twolineshloka
{दशैते पुरुषव्याघ्राः शूराः परिघबाहवः}
{अक्षौहिणीनां पतयो यज्वानो भूरिदक्षिणाः}


\twolineshloka
{एते चान्ये च बहवो दुर्योधनवशानुगाः}
{राजानो राजपुत्राश्च नीतिमन्तो महारथाः}


\twolineshloka
{सन्नद्धाः समदृश्यन स्वेष्वनीकेष्ववस्थिताः}
{बद्धकृष्णाजिनाः सर्वे बलिनो युद्धशालिनः}


\twolineshloka
{हृष्टा दुर्योधनस्यार्थे ब्रह्मलोकाय दीक्षिताः}
{समर्था दश वाहिन्यः परिगृह्य व्यवस्थिताः}


\twolineshloka
{एकादशी धार्तराष्ट्री कौरवाणां महाचमूः}
{अग्रतः सर्वसैन्यानां यत्र शान्तनवोऽग्रणीः}


\twolineshloka
{श्वेतोष्णीषं श्वेतहयं श्वेतवर्माणमच्युतम्}
{अपश्याम महाराज भीष्मं चन्द्रमिवोदितम्}


\twolineshloka
{हेमतालध्वजं भीष्मं राजते स्यन्दते स्थितम्}
{श्वेताभ्र इव तीक्ष्णांशुं ददृशुः कुरुपाण्डवाः}


\threelineshloka
{दृष्ट्वा चमूमुखे भीष्मं समकम्पन्त पाण्डवाः}
{सृञ्जयाश्च महेष्वासा धृष्टद्युम्नपुरोगमाः}
{जृम्भमाणं महासिंहं दृष्ट्वा क्षुद्रमृगा यथा}


\twolineshloka
{धृष्टद्युम्नमुखाः सर्वे समुद्विविजिरे मुहुः}
{एकादशैताः श्रीजुष्टा वाहिन्यस्तव पार्थिव}


\twolineshloka
{पाण्डवानां तथा सप्त महापुरुषपालिताः}
{उन्मत्तमकरावर्तौ महाग्राहसमाकुलौ}


\threelineshloka
{युगान्ते समवेतौ द्वौ दृश्येते सागराविव}
{नैव नस्तादृशो राजन्दृष्टपूर्वो न च श्रुतः}
{अनीकानां समेतानां कौरवाणां तथाविधः}


\chapter{अध्यायः १७}
\twolineshloka
{सञ्जय उवाच}
{}


\twolineshloka
{यथा स भगवान्व्यासः कृष्णद्वैपायनोऽब्रवीत्}
{तथैव सहिताः सर्वे समाजग्मुर्महीक्षितः}


\twolineshloka
{मघाविषयगः सोमस्तद्दिनं प्रत्यपद्यत}
{दीप्यमानाश्च संपेतुर्दिवि सप्त महग्रहाः}


\twolineshloka
{द्विधाभूत इवादित्य उदये प्रत्यदृश्यत}
{ज्वलन्त्या शिखया भूयो भानुमानुदितो रविः}


\twolineshloka
{ववाशिरे च दीप्तायां दिशि गोमायुवायसाः}
{लिप्समानाः शरीराणि मांसशोणितभोजनाः}


\twolineshloka
{अहन्यहनि पार्थानां वृद्धः कुरुपितामहः}
{भरद्वाजात्मजश्चैव प्रातरुत्थाय संयतौ}


\twolineshloka
{जयोऽस्तु पाण्डुपुत्राणामित्यूचतुरिन्दमौ}
{युयुधाते तवार्थाय यथा स समयः कृतः}


\twolineshloka
{सर्वधर्मविशेषज्ञः पिता देवव्रतस्तव}
{समानीय महीपालानिदं वचनमब्रवीत्}


\twolineshloka
{इदं वः क्षत्रिया द्वारं स्वर्गायापावृतं महत्}
{गच्छध्वं तेन शक्रस्य ब्रह्मणः सहलोकताम्}


\twolineshloka
{एष वः शाश्वतः पन्थाः पूर्वैः पूर्वतरैः कृतः}
{संभावयध्वमात्मानमव्यग्रमनसो युधि}


\twolineshloka
{नाभागोऽथ ययातिश्च मान्धाता नहुषो नृगः}
{संसिद्धाः परमं स्थानं गताः कर्मभिरीदृशैः}


\twolineshloka
{अधर्मः क्षत्रियस्यैष यद्व्याधिमरणं गृहे}
{यदयोनिधनं याति सोऽस्य धर्मः सनातनः}


\twolineshloka
{एवमुक्ता महीपाला भीष्मेण भरतर्षभ}
{निर्ययुः स्वान्यनीकानि शोभयन्तो रथोत्तमैः}


\twolineshloka
{स तु वैकर्तनः कर्णः सामात्यः सह बन्धुभिः}
{न्यासितः समरे शस्त्रं भीष्मेण भरतर्षभ}


\twolineshloka
{अपेतकर्णाः पुत्रास्ते राजनश्चैव तावकाः}
{निर्ययुः सिंहनादेन नादयन्तो दिशो दश}


\twolineshloka
{श्वेतैश्छत्रैः पताकाभिर्ध्वजवारणवाजिभिः}
{तान्यनीकानि शोभन्ते गजैरथ पदातिभिः}


\twolineshloka
{भेरीपणवशब्दैश्च दुन्दुभीनां च निःस्वनैः}
{रथनेमिनिनादैश्च बभूवाकुलिता मही}


\twolineshloka
{काञ्चनाङ्गदकेयूरैः कार्मुकैश्च महारथाः}
{भ्राजमाना व्यराजन्त साग्नयः पर्वता इव}


\twolineshloka
{तालेन महता भीष्मः पञ्चतारेण केतुना}
{विमलादित्यसंकाशस्तस्थौ कुरुचमूपरि}


\twolineshloka
{ये त्वदीया महेष्वासा राजानो भरतर्षभ}
{अवर्तन्त यथादेशं राजञ्शान्तनवस्य ते}


\threelineshloka
{स तु गोवासनः शैब्यः सहितः सर्वराजभिः}
{ययौ मातङ्गराजेन राजार्हेण पताकिना}
{पद्मवर्णस्त्वनीकानां सर्वेषामग्रतः स्थितः}


\twolineshloka
{अश्वत्थामा ययौ यत्तः सिंहलाङ्गूलकेतुना}
{श्रुतायुधश्चित्रसेनः पुरुमित्रो विविंशतिः}


\twolineshloka
{शल्यो भूरिश्रवाश्चैव विकर्णश्च महारथः}
{एते सप्त महेष्वासा द्रोणपुत्रपुरोगमाः}


\twolineshloka
{स्यन्दनैर्वरवर्माणो भीष्मस्यासन्पुरोगमाः}
{तेषामपि महोत्सेधाः शोभयन्तो रथोत्तमान्}


\twolineshloka
{भ्राजमाना व्यरोचन्त जाम्बूनदमया ध्वजाः}
{जाम्बूनदमयी वेदी कमण्डलुविभूषिता}


\twolineshloka
{केतुराचार्यमुख्यस्य द्रोणस्य धनुषा सह}
{अनेकशतसाहस्रमनीकमनुकर्षतः}


\twolineshloka
{महान्दुर्योधनस्यासीन्नागो मणिमयो ध्वजः}
{तस्य पौरवकालिङ्गकाम्भोजाः समुदक्षिणाः}


\threelineshloka
{क्षेमधन्वा च शल्यश्च तस्थुः प्रमुखतो रथाः}
{स्यन्दनेन महार्हेण केतुना वृषभेण च}
{प्रकर्षन्नेव सेनाग्नं मागधस्य कृपो ययौ}


\twolineshloka
{तदङ्गपतिनां गुप्तं कृपेण च मनस्विना}
{शारदाम्बुधरप्रख्यं प्राच्यानां सुमहद्बलम्}


\twolineshloka
{अनीकप्रमुखे तिष्ठन्वराहेण महायशाः}
{शुशुभे केतुमुख्येन राजतेन जयद्रथः}


\twolineshloka
{शतं रतसहस्राणां तस्यासन्वशवर्तिनः}
{अष्टौ नागसहस्राणि सादिनामयुतानि षट्}


\twolineshloka
{तत्सिन्धुपतिना राज्ञा पालितं ध्वजिनीमुखम्}
{अनन्तरथनागाश्वमशोभत महद्बलम्}


\twolineshloka
{षष्ट्या रथसहस्रैस्तु नागानामयुतेन च}
{पतिः सर्वकलिङ्गानां ययौ केतुमता सह}


\twolineshloka
{तस्य पर्वतसंकाशा व्यरोचन्त महागजाः}
{यन्त्रतोमरतूणीरैः पताकाभिः सुशोभिताः}


\twolineshloka
{शुशुभे केतुमुख्येन पावकेन कलिङ्गकः}
{श्वेतच्छत्रेण निष्केण चामरव्यजनेन च}


\twolineshloka
{केतुमानपि मातङ्गं विचित्रपरमाङ्कुशम्}
{आस्थितः समरे राजन्मेघस्थ इव भानुमान्}


\twolineshloka
{तेजसा दीप्यमानस्तु वारणोत्तममास्थितः}
{भगदत्तो ययौ राजा यथा वज्रधरस्तथा}


\twolineshloka
{गजस्कन्धगतावास्तां भगदत्तेन संमितौ}
{विन्दानुविन्दावावन्त्यौ केतुमन्तमनुव्रतौ}


\twolineshloka
{स रथानीकवान्व्यूहो हस्त्यङ्गो नृपशीर्षवान्}
{वाजिपक्षः पतत्युग्रः प्रहसन्सर्वतोमुखः}


\twolineshloka
{द्रोणेन विहितो राजन्राज्ञा शान्तनवेन च}
{तथैवाचार्यपुत्रेण बाह्लीकेन कृपेम च}


\chapter{अध्यायः १८}
\twolineshloka
{सञ्जय उवाच}
{}


\twolineshloka
{ततो मुहूर्तात्तुमुलः शब्दो हृदयकम्पनः}
{अश्रूयत महाराज योधानां प्रयुयुत्सताम्}


\twolineshloka
{शङ्खदुन्दुभिघोषैश्च वारणानां च बृंहितैः}
{नोमिघोषै रथानां च दीर्यतीव वसुन्धरा}


\twolineshloka
{हयानां हेषमाणानां योधानां चैव गर्जताम्}
{क्षणेनैव नभो भूमिः शब्देनापूरितं तदा}


\twolineshloka
{पुत्राणां तव दुर्धर्ष पाण्डवानां तथैव च}
{समकम्पन्त सैन्यानि परस्परसमागमे}


\twolineshloka
{तत्र नागा रथाश्चैव जाम्बूनदविभूषिताः}
{भ्राजमाना व्यदृश्यन्त मेघा इव सविद्युतः}


\twolineshloka
{ध्वजा बहुविधाकारास्तावकानां नराधिप}
{काञ्चनाङ्गदिनो रेजुर्ज्वलिता इव पावकाः}


\twolineshloka
{स्वेषां चैव परेषां च समदृश्यन्त भारत}
{महेन्द्रकेतवः शुभ्रा महेन्द्रसदनेष्विव}


\twolineshloka
{काञ्चनैः कवचैर्वीरा ज्वलनार्कसमप्रभैः}
{सन्नद्धाः समदृश्यन्त ज्वलनार्कसमप्रभाः}


\twolineshloka
{कुरुयोधवरा राजन्विचित्रायुधकार्मुकाः}
{उद्यतैरायुथैश्चित्रैस्तलबद्धाः पताकिनः}


\threelineshloka
{ऋषभाक्षा महेष्वासाश्चमूमुखगता बभुः}
{पृष्ठगोपास्तु भीष्मस्य पुत्रास्तव नराधिप}
{दुःशासनो दुर्विषहो दुर्मुखो दुःसहस्तथा}


\twolineshloka
{विविंशतिश्चित्रसेनो विकर्णश्च महारथः}
{सत्यव्रतः पुरुमित्रो जयो भूरिश्रवाः शलः}


\twolineshloka
{रथा विंशतिसाहस्रास्तथैषामनुयायिनः}
{अभीषाहाः शूरसेनाः शिबयोऽथ वसातयः}


\twolineshloka
{शाल्वा मत्स्यास्तथाम्बष्ठास्त्रैगर्ताः केकयास्तथा}
{सौवीराः कैतवाः प्राच्याः प्रतीच्येदीच्यवासिनः}


\twolineshloka
{द्वादशैते जनपदाः सर्वे शूरास्तनुत्यजः}
{महता रथवंशेन ते ररक्षुः पितामहम्}


\twolineshloka
{अनीकं दशसाहस्रं कुञ्जराणां तरस्विनाम्}
{मागधो यत्र नृपतिस्तद्रथानीकमन्वयात्}


\twolineshloka
{रथानां चक्ररक्षाश्च पादरक्षाश्च दन्तिनाम्}
{अभवन्वाहिनीमध्ये शतानामयुतानि षट्}


\twolineshloka
{पादाताश्चाग्रतो गच्छन्धनुश्चर्मासिपाणयः}
{अनेकशतसाहस्रा नखरप्रासयोधिनः}


\twolineshloka
{अक्षौहिण्यो दशैका च तव पुत्रस्य भारत}
{अदृश्यत महाराज गङ्गेव यमुनान्तरा}


\chapter{अध्यायः १९}
\twolineshloka
{धृतराष्ट्र उवाच}
{}


\twolineshloka
{अर्क्षांहिणीं दशैकां च व्यूढां दृष्ट्वा युधिष्ठिरः}
{कथमल्पेन सैन्येन प्रत्यव्यूहत पाण्डवः}


\threelineshloka
{यो वेद मानुषं व्यूहं दैवं गान्धर्वमासुरम्}
{कथं भीष्मं स कौन्तेयः प्रत्ययूहत सञ्जय ॥सञ्जय उवाच}
{}


\twolineshloka
{धार्तराष्ट्राण्यनीकानि दृष्ट्वा व्यूढानि पाण्डवः}
{अभ्यभाषत धर्मात्मा धर्मराजो धनञ्जयम्}


\twolineshloka
{महर्षेर्वचनात्तात वेदयन्ति बृहस्पतेः}
{संहतान्योधयेदल्पान्कामं विस्तारयेद्बहून्}


\twolineshloka
{सूचीमुखमनीकं स्यादल्पानां बहुभिः सह}
{अस्माकं च तथा सैन्यमल्पीयः सुतरां परैः}


\twolineshloka
{एतद्वचनमाज्ञाय महर्षेर्व्यूह पाण्डव}
{एतच्छ्रुत्वा धर्मराजं प्रत्यभाषत पाण्डवः}


\twolineshloka
{एष व्यूहामि ते व्यूहं राजसत्तम दुर्जयम्}
{अचलं नाम वज्राख्यं विहितं वज्रपाणिना}


\twolineshloka
{यः स वात इवोद्भूतः समरे दुःसहः परैः}
{स नः पुरो योत्सयते वै भीमः प्रहरतां वरः}


\twolineshloka
{तेजांसि रिपुसैन्यानां मृद्गन्पुरुषसत्तमः}
{अग्नेऽग्रणीर्योत्स्यति नो युद्धोपायविचक्षणः}


\twolineshloka
{यं दृष्ट्वा कुरवः सर्वे दुर्योधनपुरोगमाः}
{निवर्तिष्यन्ति संत्रस्ताः सिंहं क्षुद्रमृगा यथा}


\twolineshloka
{तं सर्वे संश्रयिष्यामः प्राकारमकुतोभयाः}
{भीमं प्रहरतां श्रेष्ठं देवाराजमिवामराः}


\twolineshloka
{न हि सोऽस्ति पुमाँल्लोके यः संक्रुद्धं वृकोदरम्}
{द्रष्टुमत्युग्रकर्माणं विषहेत नरर्षभम्}


\twolineshloka
{भीमसेनो गदां बिभ्रद्वज्रसारमयीं दृढाम्}
{चरन्वेगेन महता समुद्रमपि शोषयेत्}


\twolineshloka
{भीमसेनं तदा राजन्दर्शयस्व महाबलम्}
{केकया धृष्टकेतुश्च चेकितानश्च वीर्यवान्}


\twolineshloka
{एते गच्छन्तु सामात्याः प्रकर्षन्तो जनाधिप}
{धृतराष्ट्रस्य दायादानिति बीभत्सुरब्रवीत्}


\twolineshloka
{ब्रुवाणं तु तथा पार्थं सर्वसैन्यानि भारत}
{अपूजयंस्तदा वाग्भिरनुकूलाभिराहवे}


\twolineshloka
{एवमुक्त्वा महाबाहुस्तथा चक्रे धनञ्जयः}
{व्यूह्य तानि बलान्याशु प्रययौ फल्गुनस्तथा}


\twolineshloka
{संप्रयातान्कुरून्दृष्ट्वा पाण्डंवानां महाचमूः}
{गङ्गेव पूर्णा स्तिमिता स्पन्दमाना व्यदृश्यत}


\twolineshloka
{भीमसेनोऽग्रणीस्तेषां धृष्टद्युम्नश्च वीर्यवान्}
{नकुलः सहदेवश्च धृष्टकेतुश्च पार्थिवः}


\twolineshloka
{विराटश्च ततः पश्चाद्राजाथाक्षौहिणीवृतः}
{भ्रातृभिः सह पुत्रैश्च सोऽभ्यरक्षत पृष्ठतः}


\twolineshloka
{चक्ररक्षौ तु भीमस्य माद्रीपुत्रौ महाद्युती}
{द्रौपदेयाः ससौभद्राः पृष्ठगोपास्तरस्विनः}


\twolineshloka
{धृष्टद्युम्नश्च पाञ्चाल्यस्तेषां गोप्ता महारथः}
{सहितः पृतनाशूरैः रथमुख्यैः प्रभद्रकैः}


\twolineshloka
{शिखण्डी तु ततः पश्चादर्जुनेनाभिरक्षितः}
{यत्तो भीष्मविनाशाय प्रययौ भरतर्षभ}


\twolineshloka
{पृष्ठतोऽप्यर्जुनस्यासीद्युयुधानो महाबलः}
{चक्ररक्षौ तु पाञ्चाल्यौ युधामन्यूत्तमोजसौ}


\twolineshloka
{राजा तु मध्यमानीके कुन्तीपुत्रो युधिष्ठिरः}
{बृहद्भीः कुञ्जरैर्मत्तैश्चलद्भिरचलैरिव}


\threelineshloka
{अक्षौहिण्याथ पाञ्चाल्यो यज्ञसेनो महामनाः}
{विराटमन्वयात्पश्चात्पाण्डवार्थं पराक्रमी ॥ 6-19-27aतेषामादित्यचन्द्राभाः कनकोत्तमभूषणाः}
{नानाचिह्नधरा राजन्रथेष्वासन्महाध्वजाः}


\twolineshloka
{समुत्सार्य ततः पश्चाद्धृष्टद्युम्नो महारथः}
{भ्रातृभिः सहपुत्रैश्च सोऽभ्यरक्षद्युधिष्ठिरम्}


\twolineshloka
{त्वदीयानां परेषां च रथेषु विपुलान्ध्वजान्}
{अभिभूयार्जुनस्यैको रथे तस्थौ महाकपिः}


\twolineshloka
{पदातास्त्वग्रतोऽगच्छन्नसिशक्त्यृष्टिपाणयः}
{अनेकशतसाहस्रा भीमसेनस्य रक्षिणः}


\twolineshloka
{वारणा दशसाहस्राः प्रभिन्नकरटामुखाः}
{शूरा हेममयैर्जालैर्दीप्यमाना इवाचलाः}


\twolineshloka
{क्षरन्त इव जीमूता महार्हाः पद्मगन्धिनः}
{राजानमन्वयुः यश्चाज्जीमूता इव वार्षिकाः}


\twolineshloka
{भीमसेनो गदां भीमां प्रकर्षन्परिघोपमाम्}
{प्रचकर्ष महासैन्यं दुराधर्षो महामनाः}


\twolineshloka
{तमर्कमिव दुष्प्रेक्ष्यं तपन्तमिव वाहिनीम्}
{न शेकुः सर्वयोधास्ते प्रतिवीक्षितुमन्तिके}


\twolineshloka
{वज्रो नामैष स व्यूहो निर्भयः सर्वतोमुखः}
{चापविद्युद्ध्वजो घोरो गुप्तो गाण्डीवधन्वना}


\twolineshloka
{यं प्रतिव्यूहय् तिष्ठन्ति पाण्डवास्तव वाहिनीम्}
{अजेयो मानुषे लोके पाण्डवैरभिरक्षितः}


\twolineshloka
{संध्यां तिष्ठत्सु सैन्येषु सूर्यस्योदयनं प्रति}
{प्रववौ पृष्ठतो वायुर्निरभ्रे स्तनयित्नुमान्}


\twolineshloka
{विष्वग्वाताश्च विववुर्नीचैः शर्करकर्षिणः}
{रजश्चोद्धूयत महत्तम आच्छादयञ्जगत्}


\twolineshloka
{पपात महती चोल्का प्राङ्भुखी भरतर्षभ}
{उद्यन्तं सूर्यमाहत्य व्यशीर्यत महास्वना}


\twolineshloka
{अथ संनह्यमानेषु सैन्येषु भरतर्षभ}
{निष्प्रभोऽभ्युद्ययौ सूर्यः सघोषं भूश्चचाल च}


\threelineshloka
{व्यशीर्यत सनादा च भूस्तदा भरतर्षभ}
{निर्घाता बहवो राजन्दिक्षु सर्वासु चाभवन्}
{प्रादुरासीद्रजस्तीव्रं न प्राज्ञायत किंचन}


\twolineshloka
{ध्वजानां धूयमानानां सहसा मातरिश्वना}
{किङ्किणीजालबद्धानां काञ्चनस्रग्वराम्बरैः}


\twolineshloka
{महतां सपताकानामादित्यसमतेजसाम्}
{सर्वं झणझणीभूतमासीत्तालवनेष्विव}


\twolineshloka
{एवं ते पुरुषव्याघ्राः पाण्डवा युद्धनन्दिनः}
{व्यवस्थिताः प्रतिव्यूह्य तव पुत्रस्य वाहिनीम्}


\twolineshloka
{ग्रसन्त इव मञ्जानो योधानां भरतर्षभ}
{दृष्ट्वाग्रतो भीमसेनं गदापाणिमवस्थितम्}


\chapter{अध्यायः २०}
\twolineshloka
{सूर्योदये सञ्जय के नु पूर्वंयुयुत्सवो हृष्यमाणा इवासन्}
{मामका वा भीष्मनेत्राः समीपेपाण्डवा वा भीमनेत्रास्तदानीम्}


\threelineshloka
{केषां जघन्यौ समसूर्यौ सवायूकेषां सेनां श्वापदाश्चाभषन्त}
{केषां यूनां मुखवर्णाः प्रसन्नाःसर्वं ह्येतद्ब्रूहि तत्त्वं यथावत् ॥सञ्जय उवाच}
{}


\twolineshloka
{उभे सेने तुल्यमिवोपयातेउभे व्यूहे हृष्टरूपे नरेन्द्र}
{उभे चित्रे वनराजिप्रकाशेतथैवोभे नागरथाश्वपूर्णे}


\twolineshloka
{उभे सेने बृहत्यौ भीमरूपेतथैवोभे भारत दुर्विषह्ये}
{तथैवोभे स्वर्गजयाय सृष्टेतथैवोभे सत्पुरुषोपजुष्टे}


\twolineshloka
{पश्चान्मुखाः कुरवो धार्तराष्ट्राःस्थिताः पार्थाः प्राङ्मुखा योत्स्यमानाः}
{दैत्येन्द्रसेनेव च कौरवाणांदेवेन्द्रसेनेव च पाण्डवानाम्}


\twolineshloka
{शीतो वायुः पृष्ठतः पाण्डवानांधार्तराष्ट्राञ्श्वापदा व्याहरन्त}
{गजेन्द्राणां मदगन्धांश्च तीव्रा-न्न सेहिरे तव पुत्रस्य नागाः}


\twolineshloka
{दुर्योधनो हस्तिनं पद्मवर्णंसुवर्णकक्षं जालवन्तं प्रभिन्नम्}
{समास्थितो मध्यगतः कुरूणांसंस्तूयमानो बन्दिभिर्मागधैश्च}


\twolineshloka
{चन्द्रप्रभं श्वेतमथातपत्रंसौवर्णस्रग्भ्राजति चोत्तमाङ्गे}
{तं सर्वतः शकुनिः पार्वतीयैःसार्धं गान्धारैर्याति गान्धारराजः}


\twolineshloka
{भीष्मोऽग्रतः सर्वसैन्यस्य वृद्धःश्वेतच्छत्रः श्वेतधनुः सखङ्गः}
{श्वेतोष्णीषः पाण्डुरेण ध्वजेनश्वेतैरश्वैः श्वेतशैलप्रकाशैः}


\twolineshloka
{तस्य सैन्ये धार्तराष्ट्राश्च सर्वेबाह्लीकानामेकदेशः शलश्च}
{ये चाम्बष्ठाः क्षत्रिया ये च सिन्धो-स्तथा सौवीराः पञ्चनदाश्च शूराः}


\twolineshloka
{शोणैर्हयै रुक्मरथो महात्माद्रोणो धनुष्पाणिरदीनसत्वः}
{आप्तो गुरुः प्रथितः सर्वराज्ञांपश्चाच्चमूमिन्द्र इवाभियाति}


\twolineshloka
{वार्धक्षत्रिः सर्वसैन्यस्य मध्येभूरिश्रवाः पुरुमित्रो जयश्च}
{साल्वा मत्स्याः केकयाश्चेति सर्वेगजानीकैर्भ्रातरो योत्स्यमानाः}


\twolineshloka
{शारद्वतश्चोत्तरधूर्महात्मामहेष्वासो गौतमश्चित्रयोधी}
{शकैः किरातैर्यवनैः पह्लवैश्चसार्धं चमूमुत्तरतोऽभियाति}


\twolineshloka
{महारथैर्वष्णिभोजैः सुगुप्तंसुराष्ट्रकैर्विदितैरात्तशस्त्रैः}
{बृहद्बलं कृतवर्माभिगुप्तंबलं त्वदीयं दक्षिणेनाभियाति}


\twolineshloka
{संशप्तकानामयुतं रथानांमृत्युर्जयो वार्जुनस्येति सृष्टाः}
{येनार्जुनस्तेन राजन्कृतास्त्राःप्रयातारस्ते त्रिगर्ताश्च शूराः}


\twolineshloka
{साग्रं शतसहस्रं तु नागानां तव भारत}
{नागेनागे रथशतं शतमश्वा रथेरथे}


\twolineshloka
{अश्वेऽश्वे दश धानुष्का धानुष्के शत चर्मिणः}
{एवं व्यूढान्यनीकानि भीष्मेण तव भारत}


\twolineshloka
{संव्यूह्य मानुषं व्यूहं दैवं गान्धर्वमासुरम्}
{दिवसेदिवसे प्राप्ते भीष्मः शान्तनवोऽग्रणीः}


\twolineshloka
{महारथौघविपुलः समुद्र इव घोषवान्}
{भीष्मेण धार्तराष्ट्राणां व्यूहः प्रत्यङ्मुखो युधि}


\twolineshloka
{अनन्तरूपा ध्वजिनी नरेन्द्रभीमा त्वदीया न तु पाण्डवानाम्}
{तां चैव मन्ये बृहतीं दुष्प्रघर्षांयस्या नेता केशवश्चार्जुनश्च}


\chapter{अध्यायः २१}
\twolineshloka
{सञ्जय उवाच}
{}


\twolineshloka
{बृहतीं धार्तराष्ट्रस्य सेनां दृष्ट्वा समुद्यताम्}
{विषादमगमद्राजा कुन्तीपुत्रो युधिष्ठिरः}


\twolineshloka
{व्यूहं भीष्मेण चाभेद्यं कल्पितं प्रेक्ष्य पाण्डवः}
{अभेद्यमिव संप्रेक्ष्य विवर्णोऽर्जुनमब्रवीत्}


\twolineshloka
{धनञ्जय कथं शक्यमस्माभिर्योद्धुमाहवे}
{धार्तराष्ट्रैर्महाबाहो येषां योद्धा पितामहः}


\twolineshloka
{अक्षोभ्योऽयमभेद्यश्च भीष्मेणामित्रकर्षिणा}
{कल्पितः शास्त्रदृष्टेन विधिना भूरिवर्चसा}


\twolineshloka
{ते वयं संशय्नं प्राप्ताः ससैन्याः शत्रुकर्षण}
{कथमस्मान्महाव्यूहादुत्थानं नो भविष्यति}


\twolineshloka
{अथार्जुनोऽब्रवीत्पार्थं युधिष्ठिरममित्रहा}
{विषण्णमिव संप्रेक्ष्य तव राजन्ननीकिनीम्}


\twolineshloka
{प्रज्ञयाभ्यधिकाञ्शूरान्गुणयुक्तान्बहूनपि}
{जयन्त्यल्पतरा येन तन्निबोध विशांपते}


\twolineshloka
{तत्र ते कारणं राजन्प्रवक्ष्याम्यनसूयवे}
{नारदस्तमृषिर्वेद भीष्मद्रोणौ च पाण्डव}


\twolineshloka
{एनमेवार्थमाश्रित्य युद्धे देवासुरेऽब्रवीत्}
{पितामहः किल पुरा महेन्द्रादीन्दिवौकसः}


\twolineshloka
{न तथा बलवीर्याभ्यां जयन्ति विजिगीषवः}
{यथा सत्यानृशंस्याभ्यां धर्मेणैवोद्यमेन च}


\twolineshloka
{त्वक्त्वाऽधर्मं तथा सर्वे धर्मं चोत्तममास्थिताः}
{युध्यध्वमतहंकारा यतो धर्मस्ततो जयः}


\twolineshloka
{एवं राजन्विजानीहि ध्रुवोऽस्माकं रणे जयः}
{यथा तु नारदः प्राह यतः कृष्णस्ततो जयः}


\twolineshloka
{गुणभूतो जयः कृष्णे पृष्ठतोऽभ्येति माधवम्}
{तद्यथा विजयश्चास्य सन्नतिश्चापरो गुणः}


\twolineshloka
{अनन्ततेजा गोविन्दः शत्रुपूगेषु निर्व्यथः}
{पुरुषः सनातनमयो यतः कृष्णस्ततो जयः}


\twolineshloka
{पुरा ह्येष हरिर्भूत्वा विकुण्ठोऽकुण्ठसायकः}
{सुरासुरानवस्फूर्जन्नब्रवीत्के जयन्त्विति}


\twolineshloka
{अनु कृष्णं जयेमेति यैरुक्तं तत्र तैर्जितम्}
{तत्प्रसादाद्धि त्रैलोक्यं प्राप्तं शक्रादिभिः सुरैः}


\twolineshloka
{तस्य ते न व्यथां कांचिदिह पश्यामि भारत}
{यस्य ते यजमाशास्ते विश्वभुक् त्रिदिवेश्वरः}


\chapter{अध्यायः २२}
\twolineshloka
{ततो युधिष्ठिरो राजा स्वां सेनां समचोदयत्}
{प्रतिव्यूहन्ननीकानि भीष्मस्य भरतर्षभ}


\twolineshloka
{यथोद्दिष्टान्यनीकानि प्रत्यव्यूहन्त पाण्डवाःक}
{स्वर्गं परममिच्छन्तः सुयुद्धेन कुरूद्वहाः}


\twolineshloka
{मध्ये शिखण्डिनोऽनीकं रक्षितं सव्यसाचिना}
{धृष्टद्युम्नश्चरन्नग्रे भीमसेनेन पालितः}


\twolineshloka
{अनीकं दक्षिणं राजन्युयुधानेन पालितम्}
{श्रीमता सात्वताग्र्येण शक्रेणेव धनुष्मता}


\twolineshloka
{महेन्द्रयानप्रतिमं रथं तुसोपस्करं हाटकरत्नचित्रम्}
{युधिष्ठिरः काञ्चनभाण्डयोक्रंसमास्थितो नागबलस्य मध्ये}


\twolineshloka
{समुच्छ्रितं दन्तशलाकमस्यसुपाण्डुरं छत्रमतीव भाति}
{प्रदक्षिणं चैनमुपाचरन्तमहर्षयः संस्तुतिभिर्महेन्द्रम्}


\twolineshloka
{पुरोहिताः शत्रुवधं वदन्तोब्रह्मर्षिसिद्धाः श्रुतवन्त एनम्}
{जप्यैश्च मन्त्रैश्च महौषधीभिःसमन्ततः स्वस्त्ययनं ब्रुवन्तः}


\twolineshloka
{ततः स वस्त्राणि तथैव गाश्चफलानि पुष्पाणि तथैव निष्कान्}
{कुरूत्तमो ब्राह्मणसान्महात्माकुर्वन्ययौ शक्र इवामरेशः}


\twolineshloka
{सहस्रसूर्यः शतकिङ्किणीकःपरार्ध्यजाम्बूनदहेमचित्रः}
{रथोऽर्जुनस्याग्निरिवार्चिमालीविभ्राजते श्वेतहयः सुचक्रः}


\twolineshloka
{तमास्थितः केशवसंगृहीतंकपिध्वजो गाण्डिवबाणपाणिः}
{धनुर्धरो यस्य समः पृथिव्यांन विद्यते नो भविता कदाचित्}


\twolineshloka
{उद्धर्तयिष्यंस्तव पुत्रसेना-मतीव रौद्रं स बिभर्ति रूपम्}
{अनायुधो यः सुभुजो भुजाभ्यांनराश्वनागान्युधि भस्म कुर्यात्}


\twolineshloka
{स भीमसेनः सहितो यमाभ्यांवृकोदरो वीररथस्य गोप्ता}
{तं तत्र सिंहर्षभमत्तखेलंलोके महेन्द्रप्रतिमानकल्पम्}


\twolineshloka
{समीक्ष्य सेनाग्रगतं दुरासदंसंविव्यथुः पङ्कगता यथा द्विपाः}
{वृकोदरं वारणाजदर्पंयोधास्त्वदीया भयविग्नसत्त्वाः}


\threelineshloka
{अनीकमध्ये तिष्ठन्तं राजपुत्रं दुरासदम्}
{अब्रवीद्भरतश्रेष्ठं गुडाकेशं जनार्दनः ॥वासुदेव उवाच}
{}


\twolineshloka
{य एष रोषात्प्रतपन्बलस्थोयो नः सेनां सिंह इवेक्षते च}
{स एष भीष्मः कुरुवंशकेतु-र्येनाहृतास्त्रिशतं वाजिमेधाः}


\twolineshloka
{एतान्यनीकानि महानुभावंगूहन्ति मेघा इव रश्मिमन्तम्}
{एतानि हत्वा पुरुषप्रवीरकाङ्क्षस्व युद्धं भरतर्षभेण}


\chapter{अध्यायः २३}
\twolineshloka
{सञ्जय उवाच}
{}


\threelineshloka
{धार्तराष्ट्रबलं दृष्ट्वा युद्धाय समुपस्थितम्}
{अर्जुनस्य हितार्थाय कृष्णो वचनमब्रवीत् ॥श्रीभगवानुवाच}
{}


\threelineshloka
{शुचिर्भूत्वा महाबाहो संग्रामाभिमुखे स्थितः}
{पराजयाय श्त्रुणां दुर्गास्तोत्रमुदीरय ॥सञ्जय उवाच}
{}


\threelineshloka
{एवमुक्तोऽर्जुनः सङ्ख्ये वासुदेवेन धीमता}
{अवतीर्य रथात्पार्थः स्तोत्रमाह कृताञ्जलिः ॥अर्जुन उवाच}
{}


\twolineshloka
{नमस्ते सिद्धसेनानि आर्ये मन्दरवासिनि}
{कुमारि कालि कापालि कपिले कृष्णपिङ्गले}


\twolineshloka
{भद्रकालि नमस्तुभ्यं महाकालि नमोऽस्तु ते}
{चण्डि चण्डे नमस्तुभ्यं तारिणि वरवर्णिनि}


\twolineshloka
{कात्यायनि महाभागे करालि विजये जये}
{शिखिपिच्छध्वजधरे नानाभरणभूषिते}


\twolineshloka
{अट्टशूलप्रहरणे स्वङ्गखेटकधारिणि}
{गोपेन्द्रस्यानुजे ज्येष्ठे नन्दगोपकुलोद्भवे}


\twolineshloka
{महिषासृक्प्रिये नित्यं कौशिकि पीतवासिनि}
{अट्टहासे कोकमुखे नमस्तेऽस्तु रणप्रिये}


\twolineshloka
{उभे शाकम्भरि श्वेते कृष्णे कैटभनाशिनि}
{हिरण्याश्चि विरूपाक्षि सुधूम्राक्षि नमोऽस्तु ते}


\twolineshloka
{वेदश्रुति महापुण्ये ब्रह्मण्ये जातवेदसि}
{जम्बूकटकचैत्येषु नित्यं सन्निहितालये}


\twolineshloka
{त्वं ब्रह्मविद्या विद्यानां महानिद्रा च देहिनाम्}
{स्कन्दमातर्भगवति दुर्गे कान्तारवासिनि}


\twolineshloka
{स्वाहाकारः स्वधा चैव कला काष्ठा सरस्वती}
{सावित्री वेदमाता च तथा वेदान्त उच्यते}


\twolineshloka
{स्तुतासि त्वं महादेवि विशुद्धेनान्तरात्मना}
{जयो भवतु मे नित्यं त्वत्प्रसादाद्रणाजिरे}


\twolineshloka
{कान्तारभयदुर्गेषु भक्तानां चालयेषु च}
{नित्यं वससि पाताले युद्धे जयसि दानवान्}


\twolineshloka
{त्वं जम्भनी मोहिनी च माया ह्रीः श्रीस्तथैव च}
{संध्या प्रभावती चैव सावित्री जननी तथा}


\threelineshloka
{तुष्टिः पुष्टिर्धृतिर्दीप्तिश्चन्द्रादित्यविवर्धिनी}
{भूतिर्भूतिमतां सङ्ख्ये वीक्ष्यसे सिद्धचारणैः ॥सञ्जय उवाच}
{}


\threelineshloka
{ततः पार्थस्य विज्ञाय भक्तिं मानववत्सला}
{6-23-17bअन्तरिक्षगतोवाच गोविन्दस्याग्रतः स्थिता ॥ देव्युवाच}
{}


\twolineshloka
{स्वल्पेनैव तु कालेन शत्रूञ्जेष्यसि पाण्डव}
{नरस्त्वमसि दुर्धर्ष नारायणसहायवान्}


\twolineshloka
{अजेयस्त्वं रणेऽरीणामपि वज्रभृतः स्वयम्}
{इत्येवमुक्त्वा वरदा क्षणेनान्तरधीयत}


\twolineshloka
{लब्ध्वा वरं तु कौन्तेयो मेने विजयमात्मनः}
{आरुरोह ततः पार्थो रथं परमसंमतम्}


\twolineshloka
{कृष्णार्जुनावेकरथौ दिव्यौ शङ्खौः प्रदध्मतुः}
{य इदं पठते स्तोत्रं कल्य उत्थाय मानवः}


\twolineshloka
{यक्षरक्षःपिशाचेभ्यो न भयं विद्यते सदा}
{न चापि रिपवस्तेभ्यः सर्पाद्या ये च दंष्ट्रिणः}


\twolineshloka
{न भयं विद्यते तस्य सदा राजकुलादपि}
{विवादेक जयमाप्नोति बद्धो मुच्यति बन्धनात्}


\twolineshloka
{दुर्गं तरति चावश्यं तथा चोरैर्विमुच्यते}
{संग्रामे विजयेन्नित्यं लक्ष्मीं प्राप्नोति केवलाम्}


\twolineshloka
{आरोग्यबलसंपन्नो जीवेद्वर्षशतं तथा}
{एतद्दृष्टं प्रसादात्तु मया व्यासस्य धीमतः}


\twolineshloka
{मोहादेतौ न जानन्ति नरनारायणावृषी}
{तव पुत्रा दुरात्मानः सर्वे मन्युवशानुगाः}


\threelineshloka
{प्राप्तकालमिदं वाक््यं कालपाशेन कुण्ठिताः}
{द्वैपायनो नारदश्च कण्वो रामस्तथानघः}
{अवारयंस्तव सुतं न चासौ तद्गृहीतवान्}


\twolineshloka
{यत्र धर्मो द्युतिः कान्तिर्यत्र ह्रीः श्रीस्तथा मतिः}
{यतो धर्मस्ततः कृष्णो यतः कृष्णस्ततो जयः}


\chapter{अध्यायः २४}
\twolineshloka
{धृतराष्ट्र उवाच}
{}


\twolineshloka
{केषां प्रहृष्टास्तत्राग्रे योधा युध्यन्ति सञ्जय}
{उदग्रमनसः के वा के वा दीना विचेतसः}


\twolineshloka
{के पूर्वं प्राहरंस्तत्र युद्धे हृदयकम्पने}
{मामकाः पाण्डवेया वा तन्ममाचक्ष्व सञ्जय}


\threelineshloka
{कस्य सेनासमुदये गन्धो माल्यसमुद्भवः}
{वायुः प्रदक्षिणश्चैव योधानामभिगर्जताम् ॥सञ्जय उवाच}
{}


\twolineshloka
{उभयोः सेनयोस्तत्र योधा जहृषिरे तदा}
{स्रजः समाः सुगन्धानामुभयत्र समुद्भवः}


\twolineshloka
{संहतानामनीकानां व्यूढानां भरतर्षभ}
{संसर्गात्समुदीर्णानां विमर्दः सुमहानभूत्}


\twolineshloka
{वादित्रशब्दस्तुमुलः शङ्खभेरीविमिश्रितः}
{शूराणां रणशूराणां गर्जतामितरेतरम्}


\threelineshloka
{उभयोः सेनयो राजन्महान्व्यतिकरोऽभवत्}
{अन्योन्यं वीक्षमाणानां योधानां भरतर्षभ}
{कुञ्जराणां च नदतां सैन्यानां च प्रहृष्यताम्}


\chapter{अध्यायः २५}
\twolineshloka
{धृतराष्ट्र उवाच}
{}


\threelineshloka
{धर्मक्षेत्रे कुरुक्षेत्रे समवेता युयुत्सवः}
{मामकाः पाण्डवाश्चैव किमकुर्वत सञ्जय ॥सञ्जय उवाच}
{}


\twolineshloka
{दृष्ट्वा तु पाण्डवानीकं व्यूढं दुर्योधनस्तदा}
{आचार्यमुपसंगम्य राजा वचनमब्रवीत्}


\twolineshloka
{पश्यैतां पाण्डुपुत्राणामाचार्य महतीं चमूम्}
{व्यूढां द्रुपदपुत्रेण तव शिष्येण धीमता}


\twolineshloka
{अत्र शूरा महेष्वासा भीमार्जुनसमा युधि}
{युयुधानो विराटश्च द्रुपदश्च महारथः}


\twolineshloka
{धृष्टकेतुश्चेकितानः काशिराजश्च वीर्यवान्}
{पुरुजित्कृन्तिभोजश्च शैब्यश्च नरपुङ्गवः}


% Check verse!
युधामन्युश्च विक्रान्त उत्तमौजाश्च नरपुङ्गवः ॥सौभद्रो द्रौपदेयाश्च सर्व एव महारथाः
\twolineshloka
{अस्माकं तु विशिष्टा ये तान्निबोध द्विजोत्तम}
{नायका मम सैन्यस्य संज्ञार्थं तान्ब्रवीमि ते}


\twolineshloka
{भवान्भीष्मश्च कर्णश्च कृपश्च समितिंजयः}
{अश्वत्थांमा विकर्णश्च सौमदत्तिर्जयद्रथः}


\twolineshloka
{अन्ये च बहवः शूरा मदर्थे त्यक्तजीविताः}
{नानाशस्त्रप्रहरणाः सर्वे युद्धविशारदाः}


\twolineshloka
{अपर्याप्तं तदस्माकं बलं भीष्माभिरक्षितम्}
{पर्याप्तं त्विदमेतेषां बलं भीमाभिरक्षितम्}


\twolineshloka
{अयनेषु च सर्वेषु यथाभागमवस्थिताः}
{भीष्ममेवाभिरक्षन्तु भवन्तः सर्व एव हि}


\twolineshloka
{तस्य संजनयन्हर्षं कुरुवृद्धः पितामहः}
{सिंहनादं विनद्योच्चैः शङ्खं दध्मौ प्रतापवान्}


\twolineshloka
{ततः शङ्खाश्च भेर्यश्च पणवानकगोमुखाः}
{सहसैवाभ्यहन्यन्त स शब्दस्तुमुलोऽभवत्}


\twolineshloka
{ततः श्वेतैर्हयैर्युक्ते महति स्यन्दते स्थितौ}
{माधवः पाण्डवश्चैव दिव्यौ शङ्खौ प्रदध्मतुः}


\twolineshloka
{पाञ्चजन्यं हृषीकेशो देवदत्तं धनंजयः}
{पौण्ड्रं दध्मौ महाशङ्खं भीमकर्मा वृकोदरः}


\twolineshloka
{अनन्तविजयं राजा कुन्तीपुत्रो युधिष्ठिरः}
{नकुलः सहदेवश्च सुघोषमणिपुष्पकौ}


\twolineshloka
{काश्यश्च परमेष्वासः शिखण्डी च महारथः}
{धृष्टद्युम्नो विराटश्च सात्यकिश्चापराजितः}


\twolineshloka
{द्रुपदो द्रौपदेयाश्च सर्वशः पृथिवीपते}
{सौभद्रश्च महाबाहुः शङ्खान्दध्मुः पृथक्पृथक्}


\twolineshloka
{स घोषो धार्तराष्ट्राणां हृदयानि व्यदारयत्}
{नभश्च पृथिवीं चैव तुमुलो व्यनुनादयन्}


\twolineshloka
{अथ व्यवस्थितान्दृष्ट्वा धार्तराष्ट्रान्कपिध्वजः}
{प्रवृत्ते शस्त्रसंपाते धनुरुद्यम्य पाण्डवः}


\threelineshloka
{हृषीकेशं तदा वाक्यमिदमाह महीपते}
{अर्जुन उवाच}
{सेनयोरुभयोर्मध्ये रथं स्थापय मेऽच्युत}


\twolineshloka
{यावदेतान्निरीक्षेऽहं योद्धुकामानवस्थितान्}
{कैर्मया सह योद्धव्यमस्मिन्रणसमुद्यमे}


\threelineshloka
{योत्स्यमानानवेक्षेऽहं य एतेऽत्र समागताः}
{धार्तराष्ट्रस्य दुर्बुद्धेर्युद्धे प्रियचिकीर्षवः ॥सञ्जय उवाच}
{}


\twolineshloka
{एवमुक्तो हृषीकेशो गुडाकेशेन भारत}
{सेनयोरुभयोर्मध्ये स्थापयित्वा रथोत्तमम्}


\twolineshloka
{भीष्मद्रोणप्रमुखतः सर्वेषां च महीक्षिताम्}
{उवाच पार्थ पश्यैतान्समवेतान्कुरूनिति}


\twolineshloka
{तत्रापश्यत्स्थितान्पार्थः पितॄनथ पितामहान्}
{आचार्यान्मातुलान्भ्रातॄन्पुत्रान्पौत्रान्सखींस्तथा}


\twolineshloka
{श्वशुरान्सुहृदश्चैव सेनयोरुभयोरपि}
{तान्समीक्ष्य स कौन्तेयः सर्वान्बन्धूनवस्थितान्}


\threelineshloka
{कृपया परयाऽविष्टो विषीदन्निदमब्रवीत्}
{अर्जुन उवाच}
{दृष्ट्वेमं स्वजनं कृष्ण युयुत्सुं समुपस्थितम्}


\twolineshloka
{सीदन्ति मम गात्राणि मुखं च परिशुष्यति}
{वेपथुश्च शरीरे मे रोमहर्षश्च जायते}


\twolineshloka
{गाण्डीवं स्रंसते हस्तात्त्वक्वैव परिदह्यते}
{न च शक्नोम्यवस्थातुं भ्रमतीव च मे मनः}


\twolineshloka
{निमित्तानि च पश्यामि विपरीतानि केशव}
{न च श्रेयोऽनुपश्यामि हत्वा स्वजनमाहवे}


\twolineshloka
{न काङ्क्षे विजयं कृष्ण न च राज्यं सुखानि च}
{किं नो राज्येन गोविन्द किं भोगैर्जीवितेन वा}


\twolineshloka
{येषामर्थे काङ्क्षितं नो राज्यं भोगाः सुखानि च}
{त इमेऽवस्थिता युद्धे प्राणांस्त्यक्त्वा धनानि च}


\twolineshloka
{आचार्याः पितरः पुत्रास्तथैव च पितामहाः}
{मातुलाः श्वशुराः पौत्राः स्यालाः संबन्धिनस्तथा}


\twolineshloka
{एतान्न हन्तुमिच्छामि घ्नतोऽपि मधुसूदन}
{अपि त्रैलोक्यराज्यस्य हेतोः किं नु महीकृते}


\twolineshloka
{निहत्य धार्तराष्ट्रान्नः का प्रीतिः स्याज्जनार्दन}
{पापमेवाश्रयेदस्मान्हत्वैतानाततायिनः}


\twolineshloka
{तस्मान्नार्हा यवं हन्तुं धार्तराष्ट्रान्स्वबान्धवान्}
{स्वजनं हि कथं हत्वा सुखिनः स्याम माधव}


\twolineshloka
{यद्यप्येते न पश्यन्ति लोभोपहतचेतसः}
{कुलक्षयकृतं दोषं पित्रद्रोहे च पातकम्}


\twolineshloka
{कथं न ज्ञेयमस्माभिः पापादस्मान्निवर्तितुम्}
{कुलक्षयकृतं दोषं प्रपश्यद्भिर्जनार्दन}


\twolineshloka
{कुलक्षये प्रणश्यन्ति कुलधर्माः सनातनाः}
{धर्मे नष्टे कुलं कृत्स्नमधर्मोऽभिभवत्युत}


\twolineshloka
{अधर्माभिभवात्कृष्ण प्रदुष्यन्ति कुलस्त्रियः}
{स्त्रीषु दुष्टासु वार्ष्णेय जायते वर्णसङ्करः}


\twolineshloka
{सङ्करो नरकायैव कुलघ्नानां कुलस्य च}
{पतन्ति पितरो ह्येषां कुप्तपिण्डोदकक्रियाः}


\twolineshloka
{दोषैरेतैः कुलघ्नानां वर्णसङ्करकारकैः}
{उत्साद्यन्ते जातिधर्माः कुलधर्माश्च शाश्वताः}


\twolineshloka
{उत्सन्नकुलधर्माणां मनुष्याणां जनार्दन}
{नरके नियतं वासो भवतीत्यनुशुश्रुम}


\twolineshloka
{अहो बत महत्पापं कर्तुं व्यवसिता वयम्}
{यद्राज्यसुखलोभेन हन्तुं स्वजनमुद्यताः}


\threelineshloka
{यदि मामप्रतीकारमशस्त्रं शस्त्रपाणयः}
{धार्तराष्ट्रा रणे हन्युस्तन्मे क्षेमतरं भवेत् ॥सञ्जय उवाच}
{}


\twolineshloka
{एवमुक्त्वार्जुनः सङ्ख्ये रथोपस्थ उपाविशत्}
{विसृज्य सशरं चापं शोकसंविग्नमानसः}


\chapter{अध्यायः २६}
\twolineshloka
{सञ्जय उवाच}
{}


\threelineshloka
{तं तथा कृपयाविष्टमश्रुपूर्णाकुलेक्षणम्}
{विषीदन्तमिदं वाक्यमुवाच मधुसूदनः ॥श्रीभगवानुवाच}
{}


\twolineshloka
{कुतस्त्वा कश्मलमिदं विषमे समुपस्थितम्}
{अनार्यजुष्टमस्वर्ग्यमकीर्तिकरमर्जुन}


\threelineshloka
{क्लैब्यं मा स्म गमः पार्थ नैतत्त्वय्युपपद्यते}
{क्षुद्रं हृदयदौर्बल्यं त्यक्त्वोत्तिष्ठ परंतप ॥अर्जुन उवाच}
{}


\twolineshloka
{कथं भीष्ममहं सङ्ख्ये द्रोणं च मधुसूदन}
{इषुभिः प्रतियोत्स्यामि पूजार्हावरिसूदन}


\twolineshloka
{गुरूनहत्वा हि महानुभावान्श्रेयो भोक्तुं भेक्षमपीह लोके}
{हत्वार्थकामांस्तु गुरूनिहैवभुञ्जीय भोगान्रुधिरप्रदिग्धान्}


\twolineshloka
{न चैतद्वीद्मः कतरन्नो गरीयोयद्वा जयम यदि वा नो जयेयुः}
{यानेव हत्वा न जिजीविषाम-स्तेऽवस्थिताः प्रमुखे धार्तराष्ट्राः}


\twolineshloka
{कार्पण्यदोषोपहतस्वभावःपृच्छामि त्वां धर्मसंमूढचेताः}
{यच्छ्रेयः स्यान्निश्चितं ब्रूहि तन्मेशिष्यस्तेऽहं शाधि मां त्वां प्रपन्नम्}


\threelineshloka
{न हि प्रपश्यामि ममापनुद्या-द्यच्छोकमुच्छोषणमिन्द्रियाणाम्}
{अवाप्य भूमावसपत्नमृद्धंराज्यं सुराणामपि चाधिपत्यम् ॥सञ्जय उवाच}
{}


\twolineshloka
{एवमुकत्वा हृषीकेशं गुडाकेशं परंतप}
{न योत्स्य इति गोविन्दमुक्त्वा तूष्णीं बभूव ह}


\threelineshloka
{तमुवाच हृषीकेशः प्रहसन्निव भारत}
{सेनयोरुभयोर्मध्ये विषीदन्तमिदं वचः ॥श्रीभगवानुवाच}
{}


\twolineshloka
{अशोच्यानन्वशोचस्त्वं प्रज्ञावादांश्च भाषसे}
{गतासूनगतासूंश्च नानुशोचन्ति पाण्डिताः}


\twolineshloka
{नत्वेवाहं जातु नासं न त्वं नेमे जनाधिपाः}
{न चैव नभविष्यामः सर्वे वयमतः परम्}


\twolineshloka
{देहिनोऽस्मिन्यथा देहे कौमारं यौवनं जरा}
{तथा देहान्तरप्राप्तिर्धीरस्तत्र न मुह्यति}


\twolineshloka
{मात्रास्पर्शास्तु कौन्तेय शीतोष्णसुखदुःखदाः}
{आगमापायिनोऽनित्स्यास्तांस्तितिक्षस्व भारत}


\twolineshloka
{यं हि व्यथयन्त्येते पुरुषं पुरुषर्षभ}
{समदुःखसुखं धीरं सोऽमृतत्वाय कल्पते}


\twolineshloka
{नासतो विद्यते भावो नाभावो विद्यते सतः}
{उभयोरपि दृष्टोऽन्तस्त्वनयोस्तत्त्वदर्शिभिःक}


\twolineshloka
{अविनाशि तु तद्विद्धि येन सर्वमिदं ततम्}
{विनाशमव्ययस्यास्य न कश्चित्कर्तुमर्हति}


\twolineshloka
{अन्तवन्त इमे देहा नित्यस्योक्ताः शरीरिणः}
{अनाशिनोऽप्रमेयस्य तस्माद्युध्यस्व भारत}


\twolineshloka
{य एनं वेत्ति हन्तारं यश्चैनं मन्यते हतम्}
{उभौ तौ न विजानीतो नायं हन्ति न हन्यते}


\twolineshloka
{न जायते म्रियते वा कदाचि-न्नायं भूत्वा भविता वा न भूयः}
{अजो नित्यः शाश्वतोऽयं पुराणोन हन्तये हन्यमाने शरीरे}


\twolineshloka
{वेदाविनाशिनं नित्यं य एनमजमव्ययम्}
{कथं स पुरुषः पार्थ कं घातयति हन्ति कम्}


\twolineshloka
{वासांसि जीर्णानि यथा विहायनवानि गृह्णानि नरोऽपराणि}
{तथा शरीराणि विहाय जीर्णा-न्यन्यानि संयाति नवानि देही}


\twolineshloka
{नैनं छिन्दन्ति शस्त्राणि नैनं दहति पावकः}
{न चैनं क्लेदयन्त्यापो न शोषयति मारुतः}


\twolineshloka
{अच्छेद्योऽयमदाह्योऽयमक्लेद्योऽशोष्य एव च}
{नित्यः सर्वगतः स्थाणुरचलोऽयं सनातनः}


\twolineshloka
{अव्यक्तोऽयमचिन्त्योऽयमविकार्योऽयमुच्यते}
{तस्मादेवं विदित्वैनं नानुशोचितुमर्हसि}


\twolineshloka
{अथ चैनं नित्यजातं नित्यं वा मन्यसे मृतम्}
{तथापि कत्वं महाबाहो नैनं शोचितुमर्हसि}


\twolineshloka
{जातस्य हि ध्रुवो मृत्युर्ध्रुवं जन्म मृतस्य च}
{तस्मादपरिहार्येऽर्थे न त्वं शोचितुमर्हसि}


\twolineshloka
{अव्यक्तादीनि भूतानि व्यक्तमध्यानि भारत}
{अव्यक्तनिधनान्येन तत्र का परिदेवना}


\twolineshloka
{आश्चर्यवत्पश्यति कश्चिदेन-माश्चर्यवद्वदति तथैव चान्यः}
{आश्चर्यवच्चैनमन्यः श्रृणोतिश्रुत्वाऽप्येनं वेद न चैव कश्चित्}


\twolineshloka
{देही नित्यमवध्योऽयं देहे सर्वस्य भारत}
{तस्मात्सर्वाणि भूतानि न त्वं शोचितुमर्हसि}


\twolineshloka
{स्वधर्ममपि चावेक्ष्य न विकम्पितुमर्हसि}
{धर्म्याद्धि युद्धाच्छ्रेयोऽन्यत्क्षत्रियस्य न विद्यते}


\twolineshloka
{यदृच्छया चोपपन्नं स्वर्गद्वारमपावृतम्}
{सुखिनः क्षत्रियाः पार्थ लन्ते युद्धमीदृशम्}


\twolineshloka
{अथ चेत्त्वमिमं धर्म्यं संग्रामं न करिष्यसि}
{ततः स्वधर्मं कीर्तिं च हित्वा पापमवाप्स्यसि}


\twolineshloka
{अकीर्तिं चापि भूतानि कथयिष्यन्ति तेऽव्ययाम्}
{संभावितस्य चाकीर्तिर्मरणादतिरिच्यते}


\twolineshloka
{भयाद्रणादुपरतं मंस्यन्ते त्वां महारथाः}
{येषां च त्वं बहुमतो भूत्वा यास्यसि लाघवम्}


\twolineshloka
{अवाच्यवादांश्चक बहून्वदिष्यन्ति तवाहिताः}
{निन्दन्तस्तव सामर्थ्यं ततो दुःखतरं नु किम्}


\twolineshloka
{हतो वा प्राप्स्यसि स्वर्गं जित्वा वा भोक्ष्यसे महीम्}
{तस्मादुत्तिष्ठ कौन्तेय युद्धाय कृतनिश्चयः}


\twolineshloka
{सुखदुःखे समे कृत्वा लाभालाभौ जयाजयौ}
{ततो युद्धाय युज्यस्व नैवं पापमवाप्स्यसि}


\twolineshloka
{एषा तेऽभिहिता साङ्ख्ये बुद्धिर्योगे त्विमां श्रृणु}
{बुद्ध्या युक्तो यया पार्थ कर्मबन्धं प्रहास्यसि}


\twolineshloka
{नेहाभिक्रमनाशोऽस्ति प्रत्यवायो न विद्यते}
{स्वल्पमप्यस्य धर्मस्य त्रायते महतो भयात्}


\twolineshloka
{व्यवसायात्मिका बुद्धिरेकेह कुरुनन्दन}
{बहुशाखा ह्यनन्ताश्च बुद्धयोऽव्यवसायिनाम्}


\twolineshloka
{यामिमां पुष्पितां वाचं प्रवदन्त्यविपश्चितः}
{वेदवादरताः पार्थ नान्यदस्तीति वादिनः}


\twolineshloka
{कामात्मानः स्वर्गपरा जन्मकर्मफलप्रदाम्}
{क्रियाविशेषबहुलां भोगैश्वर्यगतिं प्रति}


\twolineshloka
{भोगैश्वर्यप्रसक्तानां तयाऽपहृतचेतसाम्}
{व्यवसायात्मिका बुद्धिः समाधौ न विधीयते}


\twolineshloka
{त्रैगुण्यविषया वेदा निस्त्रैगुण्यो भवार्जुन}
{निर्द्वन्द्वो नित्यसत्वस्थो निर्योगक्षेम आत्मवान्}


\twolineshloka
{यावानर्थ उदपाने सर्वतः संप्लुतोदके}
{तावान्सर्वेषु वेदेषु ब्राह्मणस्य विजानतः}


\twolineshloka
{कर्मण्येवाधिकारस्ते मा फलेषु कदा च न}
{मा कर्मफलहेतुर्भूर्मा ते सङ्गोऽस्त्वकर्मणि}


\twolineshloka
{योगस्थः कुरु कर्माणि सङ्गं त्यक्त्वा धनंजय}
{सिद्ध्यसिद्ध्योः समो भूत्वा समत्वं योग उच्यते}


\twolineshloka
{दूरेण ह्यवरं कर्म बुद्धियोगाद्धनंजय}
{बुद्धौ शरणमन्विच्छ कृपणाः फलहेतवः}


\twolineshloka
{बुद्धियुक्तो जहातीहक उभे सुकृतदुष्कृते}
{तस्माद्योगाय युज्यस्व योगः कर्मसु कौशलम्}


\twolineshloka
{कर्मजं बुद्धियुक्ता हि फलं त्यक्त्वा मनीषिणः}
{जन्मबन्धविनिर्मुक्ताः पदं गच्छन्त्यनामयम्}


\twolineshloka
{यदा ते मोहकलिलं बुद्धिर्व्यतितरिष्यति}
{तदा गन्तासि निर्वेदं श्रोतव्यस्य श्रुतस्य च}


\threelineshloka
{श्रुतिविप्रतिपन्ना ते यदा स्थास्यति निश्चला}
{समाधावचला बुद्धिस्तदा योगमवाप्स्यसि ॥अर्जुन उवाच}
{}


\threelineshloka
{स्थितप्रज्ञस्य का भाषा समाधिस्थस्य केशव}
{स्थितधीः किं प्रभाषेत किमासीत व्रजेत किं ॥श्रीभगवानुवाच}
{}


\twolineshloka
{प्रजहाति यदा कामान्सर्वान्पार्थ मनोगतान्}
{आत्मन्येवात्मना तुष्टः स्थितप्रज्ञस्तदोच्यते}


\twolineshloka
{दुःखेष्वनुद्विग्नमनाः सुखेषु विगतस्पृहः}
{वीतरागभयक्रोधः स्थितधीर्मुनिरुच्यते}


\twolineshloka
{यः सर्वत्रानभिस्नेहस्तत्तत्प्राप्य शुभाशुभम्}
{नाभिनन्दति न द्वेष्टि तस्य प्रज्ञा प्रतिष्ठिता}


\twolineshloka
{यदा संहरते चायं कूर्मोऽङ्गानीव सर्वशः}
{इन्द्रियाणीन्द्रियार्थेभ्यस्तस्य प्रज्ञा प्रतिष्ठिता}


\twolineshloka
{विषया विनिवर्तन्ते निराहारस्य देहिनः}
{रसवर्जं रसोऽप्यस्य परं दृष्ट्वा निवर्तते}


\twolineshloka
{यततो ह्यपि कौन्तेय पुरुषस्य विपश्चितः}
{इन्द्रियाणि प्रमाथीनि हरन्ति प्रसभं मनः}


\twolineshloka
{तानि सर्वाणि संयम्य युक्त आसीत मत्परः}
{वशे हि यस्येन्द्रियामि तस्य प्रज्ञा प्रतिष्ठिता}


\twolineshloka
{ध्यायतो विषयान्पुंसः सङ्गस्तेषूपजायते}
{सङ्गात्संजायते कामः कामात्क्रोधोऽभिजायते}


\twolineshloka
{क्रोधाद्भवति संमोहः संमोहात्स्मृतिविभ्रमः}
{स्मृतिभ्रंशाद्बुदिनाशो बुद्धिनाशात्प्रणश्यति}


\twolineshloka
{रागद्वेषवियुक्तैस्तु विषयानिन्द्रियैश्चरन्}
{आत्मवश्यैर्विधेयात्मा प्रसादमधिगच्छति}


\twolineshloka
{प्रसादे सर्वदुःखानां हानिरस्योपजायते}
{प्रसन्नचेतसो ह्याशु बुद्धिः पर्यवतिष्ठते}


\twolineshloka
{नास्ति बुद्धिरयुक्तस्य न चायुक्तस्य भावना}
{न चाभावयतः शान्तिरशान्तस्य कुतः सुखम्}


\twolineshloka
{इन्द्रियाणां हि चरतां यन्मनोऽनुविधीयते}
{तदस्य हरति प्रज्ञां वायुर्नावमिवाम्भसि}


\twolineshloka
{तस्माद्यस्य महाबाहो निगृहीतानि सर्वशः}
{इन्द्रियाणीन्द्रियार्थेभ्यस्तस्य प्रज्ञा प्रतिष्ठिता}


\twolineshloka
{या निशा सर्वभूतानां तस्यां जागर्तिं संयमी}
{यस्यां जाग्रति भूतानि सा निशा पश्यतो मुनेः}


\twolineshloka
{आपूर्यमाणमचलप्रतिष्ठंसमुद्रमापः प्रविशन्ति यद्वत्}
{तद्वत्कामा यं प्रविशन्ति सर्वेस शान्तिमाप्नोति न कामकामी}


\twolineshloka
{विहाय कामान्यः सर्वान्पुमांश्चरति निःस्पृहः}
{निर्ममो निरहंकारः स शान्तिमधिगच्छति}


\twolineshloka
{एषा ब्राह्मी स्थितिः पार्थ नैनां प्राप्य विमुह्यति}
{स्थित्वास्यामन्तकालेऽपि ब्रह्म निर्वाणमृच्छति}


\chapter{अध्यायः २७}
\twolineshloka
{अर्जुन उवाच}
{}


\twolineshloka
{ज्यायसी चेत्कर्मणस्ते मता बुद्धिर्जनार्दन}
{तत्किं कर्मणि घोरे मां नियोजयसि केशव}


\threelineshloka
{व्यामिश्रेणेव वाक्येन बुद्धिं मोहयसीव मे}
{तदेकं वद निश्चित्य येन श्रेयोऽहमाप्नुयाम् ॥श्रीभगवानुवाच}
{}


\twolineshloka
{लोकेऽस्मिन्द्विविधा निष्ठा पुरा प्रोप्ता मयाऽनघ}
{ज्ञानयोगेन साङ्ख्यानां कर्मयोगेन योगिनाम्}


\twolineshloka
{न कर्मणामनारम्भान्नैष्कर्म्यं पुरुषोऽश्रुते}
{न च संन्यसनादेव सिद्धिं समधिगच्छति}


\twolineshloka
{न हि कश्चित्क्षणमपि जातु तिष्ठत्यकर्मकृत्}
{कार्यते ह्यवशः कर्म सर्वः प्रकृतिजैर्गुणैः}


\twolineshloka
{कर्मेन्द्रियाणि संयम्य य आस्ते मनसा स्मरन्}
{इन्द्रियार्थान्विमूढात्मा मिथ्याचाराः स उच्यते}


\twolineshloka
{यस्त्विन्द्रियाणि पनसा नियम्यारमभतेऽर्जुन}
{कर्मेन्द्रियैः कर्मयोगमसक्तः स विशिष्यते}


\twolineshloka
{नियतं कुरु कर्म त्वं ज्यायो ह्यकर्मणः}
{शरीरयात्रापि च ते न प्रतिद्ध्येदकर्मणः}


\twolineshloka
{यज्ञार्थात्कर्मणोऽन्यत्र लोकोऽयं कर्मबन्धनः}
{तदर्थं कर्म कौन्तेय मुक्तसङ्गः समाचर}


\twolineshloka
{सहयज्ञाः प्रजाः सृष्ट्वा पुरोवाच प्रजापतिः}
{अनेन प्रसविष्यध्वमेष वोऽस्त्विष्टकामधुक्}


\twolineshloka
{देवान्भावयताऽनेन ते देवा भावयन्तु वः}
{परस्परं भावयन्तः श्रेयः परमवाप्स्यथ}


\twolineshloka
{इष्टान्भोगान्हि वो देवा दास्यन्ते यज्ञभाविताः}
{तैर्दत्तानप्रदायैभ्यो यो भुङ्क्ते स्तेन एव सः}


\twolineshloka
{यज्ञशिष्टाशिनः सन्तो मुच्यन्ते सर्वकिल्बिषैः}
{भुञ्जते ते त्वघं पापा ये पचन्त्यात्मकारणात्}


\twolineshloka
{अन्नाद्भवन्ति भूतानि पर्जन्यादन्नसंभवः}
{यज्ञाद्भवति पर्जन्यो यज्ञः कर्मसमुद्भवः}


\twolineshloka
{कर्म ब्रह्मोद्भवं विद्धि ब्रह्माऽक्षरसमुद्भवम्}
{तस्मात्सर्वगतं ब्रह्म नित्यं यज्ञे प्रतिष्ठितम्}


\twolineshloka
{एवं प्रवर्तितं चक्रं नानुवर्तयतीह यः}
{अघायुरिन्द्रियारामो मोघं पार्थ स जीवति}


\twolineshloka
{यस्त्वात्मरतिरेव स्यादात्मतृप्तश्च मानवः}
{आत्मन्येव च संतुष्टस्तस्य कार्यं न विद्यते}


\twolineshloka
{नैव तस्य कृतेनार्थो नाकृतेनेह कश्चन}
{न चास्य सर्वभूतेषु कश्चिदर्थव्यपाश्रयः}


\twolineshloka
{तस्मादसक्तः सततं कार्यं कर्म समाचर}
{असक्तो ह्याचरन्कर्म परमाप्नोति पुरूषः}


\twolineshloka
{कर्मणैव हि संसिद्धिमास्थिता जनकादयः}
{लोकसंग्रहमेवापि संपश्यन्कर्तुमर्हसि}


\twolineshloka
{यद्यदाचरति श्रेष्ठस्तत्तदेवेतरो जनः}
{स यत्प्रमाणं कुरुते लोकस्तदनुवर्तते}


\twolineshloka
{न मे पार्थास्ति कर्तव्यं त्रिषु लोकेषु किंचन}
{नानवाप्तमवाप्तव्यं वर्त एव च कर्मणि}


\twolineshloka
{यदि ह्यहं न वर्तेयं जातु कर्मण्यतन्द्रितः}
{मम वर्न्मानुवर्तन्ते मनुष्याः पार्थ सर्वशः}


\twolineshloka
{उत्सीदेयुरिमे लोका न कुर्यां कर्म चेदहम्}
{संकरस्य च कर्ता स्यामुपहन्यामिमाः प्रजाः}


\twolineshloka
{सक्ताः कर्मण्यविद्वांसो यथा कुर्वन्ति भारत}
{कुर्याद्विद्वांस्तथाऽसक्तश्चिकीर्षुर्लोकसंग्रहम्}


\twolineshloka
{न बुद्धिभेदं जनयेदज्ञानां कर्मसङ्गिनाम्}
{जोषयेत्सर्वकर्माणि विद्वान्युक्तः समाचरन्}


\twolineshloka
{प्रकृतेः क्रियमाणानि गुणैः कर्माणि सर्वशः}
{अहङ्कारविमूढात्मा कर्ताहमिति मन्यते}


\twolineshloka
{तत्त्ववित्तु महाबाहो गुणकर्मविभागयोः}
{गुणाऽनुणेषु वर्तन्त इति मत्वा न सञ्जते}


\twolineshloka
{प्रकृतेर्गुणसंमूढाः सञ्जन्ते गुणकर्मसु}
{तानकृत्स्नविदो मन्दान्कृत्स्नविन्न विचालयेत्}


\twolineshloka
{मयि सर्वाणि कर्माणि संन्यस्याध्यात्मचेतसा}
{निराशीर्निर्ममो भूत्वा युध्यस्व विगतज्वरः}


\twolineshloka
{ये मे मतमिदं नित्यमनुतिष्ठन्ति मानवाः}
{श्रद्धावन्तोऽनसूयन्तो मुच्यन्ते तेऽपि कर्मभिः}


\twolineshloka
{ये त्वेतदभ्यसूयन्तो नानुतिष्ठन्ति मे मतम्}
{सर्वज्ञानविमूढांस्तान्विद्धि नष्टानचेतसः}


\twolineshloka
{सदृशं चेष्टते स्वस्याः प्रकृतेर्ज्ञानवानपि}
{प्रकृतिं यान्ति भूतानि निग्रहः किं करिष्यति}


\twolineshloka
{इन्द्रियस्येन्द्रियस्यार्थे रागद्वेषौ व्यवस्थितौ}
{तयोर्न वशमागच्छेत्तौ ह्यस्य परिपन्थिनौ}


\threelineshloka
{श्रेयान्स्वधर्मो विगुणः परधर्मात्स्वनुष्ठितात्}
{स्वधर्मे निधनं श्रेयः परधर्मो भयावहः ॥अर्जुन उवाच}
{}


\threelineshloka
{अथ केन प्रयुक्तोऽयं पापं चरति पूरुषः}
{अनिच्छन्नपि वार्ष्णेय बलादिव नियोजितः ॥श्रीभगवानुवाच}
{}


\twolineshloka
{काम एष क्रोध एष रजोगुणसमुद्भवः}
{महाशनो महापाप्मा विद्ध्येनमिह वैरिणम्}


\twolineshloka
{धूमेनाव्रियते वह्निर्यथाऽऽदर्शो मलेन च}
{यथोल्बेनावृतो गर्भस्तथा तेनेदमावृतम्}


\twolineshloka
{आवृतं ज्ञानमेतेन ज्ञानिनो नित्यवैरिणा}
{कामरूपेण कौन्तेय दुष्पूरेणानलेन च}


\twolineshloka
{इन्द्रियाणि मनो बुद्धिरस्याधिष्ठानमुच्यते}
{एतैर्विमोहयत्येष ज्ञानमावृत्त्य देहिनम्}


\twolineshloka
{तस्मात्त्वमिन्द्रियाण्यादौ नियम्य भरतर्षभ}
{पाप्मानं प्रजहि ह्येनं ज्ञानविज्ञाननाशनम्}


\twolineshloka
{इन्द्रियामि पराण्याहुरिन्द्रियेभ्यः परं मनः}
{मनसस्तु परा बुद्धिर्यो बुद्धेः परतस्तु सः}


\twolineshloka
{एवं बुद्धेः परं बुद्धाः संस्तभ्यात्मानमात्मना}
{जहि शत्रुं महाबाहो कामरूपं दुरासदम्}


\chapter{अध्यायः २८}
\twolineshloka
{श्रीभगवानुवाच}
{}


\twolineshloka
{इमं विवस्वते योगं प्रोक्तवानहमव्ययम्}
{विवस्वान्मनवे प्राह मनुरिक्ष्वाकवेऽब्रवीत्}


\twolineshloka
{एवं परम्पराप्राप्तमिमं राजर्षयोऽविदुः}
{स कालेनेह महता योगो नष्टः परंतप}


\threelineshloka
{स एवायं मया तेऽद्य योगः प्रोक्तः पुरातनः}
{भक्तोऽसि मे सखा चेति रहस्यं ह्येतदुत्तमम् ॥अर्जुन उवाच}
{}


\threelineshloka
{अपरं भवतो जन्म परं जन्म विवस्वतः}
{कथमेतद्विजानीयां त्वमादौ प्रोक्तवानिति ॥श्रीभगवानुवाच}
{}


\twolineshloka
{बहूनि मे व्यतीतानि जन्मानि तव चार्जुन}
{तान्यहं वेद सर्वाणि न त्वं वेत्थ परंतप}


\twolineshloka
{अजोऽपि सन्नव्ययात्मा भूतानामीश्वरोऽपि सन्}
{प्रकृतिं स्वामधिष्ठाय संभवाम्यात्ममायया}


\twolineshloka
{यदा यदा हि धर्मस्य ग्लानिर्भवति भारत}
{अभ्युत्थानमधर्मस्य तदाऽऽत्मानं सृजाम्यहम्}


\twolineshloka
{परित्राणाय साधूनां विनाशाय च दुष्कृताम्}
{धर्मसंस्थापनार्थाय संभवामि युगे युगे}


\twolineshloka
{जन्म कर्म च मे दिव्यमेवं यो वेत्ति तत्त्वतः}
{त्वक्त्वा देहं पुनर्जन्म नैति मामेति सोऽर्जुन}


\twolineshloka
{वीतरागभयक्रोधा मन्मया मामुपाश्रिताः}
{बहवो ज्ञानतपसा पूता मद्भावमागताः}


\twolineshloka
{ये यथा मां प्रपद्यन्ते तांस्तथैव भजाम्यहम्}
{मम वर्त्मानुवर्तन्ते मनुष्याः पार्थ सर्वशः}


\twolineshloka
{काङ्क्षन्तः कर्मणां सिद्धिं यजन्त इह देवताः}
{क्षिप्रं हि मानुषे लोके सिद्धिर्भवति कर्मजा}


\twolineshloka
{चातुर्वर्ण्यं मया सृष्टं गुणकर्मविभागशः}
{तस्य कर्तारमपि मां विद्ध्यकर्तारमव्ययम्}


\twolineshloka
{न मां कर्माणि लिम्पन्तिक न मे कर्मफले स्पृहा}
{इति मां योऽभिजानाति कर्मभिर्न स बध्यते}


\twolineshloka
{एवं ज्ञात्वा कृतं कर्म पूर्वैरपि मुमुक्षुभिः}
{कुरु कर्मैव तस्मात्त्वं पूर्वैः पूर्वतरं कृतम्}


\twolineshloka
{किं कर्म किमकर्मेति कवयोऽप्यत्र मोहिताः}
{तत्ते कर्म प्रवक्ष्यामि यज्ज्ञात्वा मोक्ष्यसेऽशुभात्}


\twolineshloka
{कर्मणो ह्यपि बोद्धव्यं बोद्धव्यं च विकर्मणः}
{अकर्मणश्च बोद्धव्यं गहना कर्मणो गतिः}


\twolineshloka
{कर्मण्यकर्म यः पश्येदकर्मणि च कर्म यः}
{स बुद्धिमान्मनुष्येषु स युक्तः कृत्स्नकर्मकृत्}


\twolineshloka
{यस्य सर्वे समारम्भाः कामसंकल्पवर्जिताः}
{ज्ञानाग्निदग्धकर्माणं तमाहुः पण्डितं बुधाः}


\twolineshloka
{त्यक्त्वा कर्मफलासङ्गं नित्यतृप्तो निराश्रयः}
{कर्मण्यभिप्रवृत्तोऽपि नैव किंचित्करोति सः}


\twolineshloka
{निराशीर्यतचित्तात्मा त्यक्तसर्वपरिग्रहः}
{शारीरं केवलं कर्म कुर्वन्नाप्नोति किल्बिषम्}


\twolineshloka
{यदृच्छालाभसंतुष्टो द्वन्द्वातीतो विमत्सरः}
{समः सिद्धावसिद्धौ च कृत्वापि न निबध्यते}


\twolineshloka
{गतसङ्गस्य मुक्तस्य ज्ञानावस्थितचेतसः}
{यज्ञायाचरतः कर्म समग्रं प्रविलीयते}


\twolineshloka
{ब्रह्मार्पणं ब्रह्म हविर्ब्रह्माग्नौ ब्रह्मणा हुतम्}
{ब्रह्मैव तेन गन्तव्यं ब्रह्म कर्म समाधिना}


\twolineshloka
{दैवमेवापरे यज्ञं योगिनः पर्युपासते}
{ब्रह्माग्नावपरे यज्ञं यज्ञेनैवोपजुह्वति}


\twolineshloka
{श्रोत्रादीनीन्द्रियाण्यन्ये संयमाग्निषु जुह्वति}
{शब्दादीन्विषयानन्य इन्द्रियाग्निषु जुह्वति}


\twolineshloka
{सर्वाणीन्द्रियकर्माणि प्राणकर्माणि चापरे}
{आत्मसंयमयोगाग्नौ जुह्वति ज्ञानदीपिते}


\twolineshloka
{द्रव्ययज्ञास्तपोयज्ञा योगयज्ञास्तथापरे}
{स्वाध्यायज्ञानयज्ञाश्च यतयः संशितव्रताः}


\twolineshloka
{अपाने जुह्वति प्राणं प्राणेऽपानं तथाऽपरे}
{प्राणापानगती रुद्ध्वा प्राणायामपरायणाः}


\twolineshloka
{अपरे नियताहाराः प्राणान्प्राणेषु जुह्वति}
{सर्वेऽप्येते यज्ञविदो यज्ञक्षपितकल्मषाः}


\twolineshloka
{यज्ञशिष्टामृतभुजो यान्ति ब्रह्म सनातनम्}
{नायं लोकोऽस्त्ययज्ञस्य कुतोऽन्यः कुरुसत्तम}


\twolineshloka
{एवं बहुविधा यज्ञा वितता ब्रह्मणो मुखे}
{कर्मजान्विद्धि तान्सर्वानेवं ज्ञात्वा विमोक्ष्यसे}


\twolineshloka
{श्रेयान्द्रव्यमयाद्यजाज्ज्ञानयज्ञः परंतप}
{सर्वं कर्माखिलं पार्थ ज्ञाने परिसमाप्यते}


\twolineshloka
{तद्विद्धि प्रणिपातेन परिप्रश्नेन सेवया}
{उपदेक्ष्यन्ति ते ज्ञानं ज्ञानिनस्तत्त्वदर्शिनः}


\twolineshloka
{यज्ज्ञात्वा न पुनर्मोहमेवं यास्यसि पाण्डव}
{येन भूतान्यशेषेण द्रक्ष्यस्यात्मन्यथो मयि}


\twolineshloka
{अपि चेदसि पापेभ्यः सर्वेभ्यः पापकृत्तमः}
{सर्वं ज्ञानप्लवेनैव वृजिनं संतरिष्यसि}


\twolineshloka
{यथैधांसि समिद्धोऽग्निर्भस्मासात्कुरुतेऽर्जुन}
{ज्ञानाग्निः सर्वकर्माणि भस्मसात्कुरुते यथा}


\twolineshloka
{न हि ज्ञानेन सदृशं पवित्रमिह विद्यते}
{तत्स्वयं योगसंसिद्धः कालेनात्मनि विन्दति}


\twolineshloka
{श्रद्धावाँल्लभते ज्ञानं तत्परः संयतेन्द्रियः}
{ज्ञानं लब्ध्वा परां शान्तिमचिरेणाधिगच्छति}


\twolineshloka
{अज्ञश्चाश्रद्दधानश्च संशयात्मा विनश्यति}
{नायं लोकोऽस्ति न परो न सुखं संशयात्मनः}


\twolineshloka
{योगसंन्यस्तकर्माणं ज्ञानसंछिन्नसंशयम्}
{आत्मवन्तं न कर्माणि निबध्नन्ति धनंजय}


\twolineshloka
{तस्मादज्ञानसंभूतं हृत्स्थं ज्ञानासिनाऽऽत्मनः}
{छित्त्वैनं संशयं योगमातिष्ठोत्तिष्ठ भारत}


\chapter{अध्यायः २९}
\twolineshloka
{अर्जुन उवाच}
{}


\threelineshloka
{संन्यासं कर्मणां कृष्ण कपुनर्योगं च शंससि}
{यच्छ्रेय एतयोरेकं तन्मे ब्रूहि सुनिश्चितम् ॥श्रीभगवानुवाच}
{}


\twolineshloka
{संन्यासः कर्मयोगश्च निःश्रेयसकरावुभौ}
{तयोस्तु कर्मसंन्यासात्कर्मयोगो विशिष्यते}


\twolineshloka
{ज्ञेयः स नित्यसंन्यासी यो न द्वेष्टि न काङ्क्षति}
{निर्द्वन्द्वो हि महाबाहो सुखं बन्धात्प्रमुच्यते}


\twolineshloka
{साङ्ख्ययोगौ पृथग्बालाः प्रवदन्ति न पण्डिताः}
{एकमप्यास्थितः सम्यगुभयोर्विन्दते फलम्}


\twolineshloka
{यत्साङ्ख्यैः प्राप्यते स्थानं तद्योगैरपि गम्यते}
{एकं साङ्ख्यं च योगं च यः पश्यति स पश्यति}


\twolineshloka
{संन्यासस्तु महाबाहो दुःखमाप्तुमयोगतः}
{योगयुक्तो मुनिर्ब्रह्म नचिरेणाधिगच्छति}


\twolineshloka
{योगयुक्तो विशुद्धात्मा विजितात्मा जितेन्द्रियः}
{सर्वभूतात्मभूतात्मा कुर्वन्नपि न लिप्यते}


\twolineshloka
{नैव किंचित्करोमीति युक्तो मन्येत तत्त्ववित्}
{पश्यञ्शृण्वन्स्पृशञ्जिघ्रन्नश्रन्गच्छन्स्वपन्श्वसन्}


\twolineshloka
{प्रलपन्विसृजन्गृह्णन्नृन्मिषन्निमिषन्नपि}
{इन्द्रियाणीन्द्रियार्थेषु वर्तन्त इति धारयन्}


\twolineshloka
{ब्रह्मण्याधाय कर्माणि सङ्गं त्यक्त्वा करोति यः}
{लिप्यते न स पापेन पद्मपत्रमिवाम्भसा}


\twolineshloka
{कायेन मनसा बुद्ध्या केवलैरिन्द्रियैरपि}
{योगिनः कर्म कुर्वन्ति सङ्गं त्यक्त्वात्मशुद्धये}


\twolineshloka
{युक्तः कर्मफलं त्यक्त्वा शान्तिमाप्नोति नैष्ठिकीम्}
{अयुक्तः कामकारेण फले सक्तो निबध्यते}


\twolineshloka
{सर्वकर्माणि मनसा संन्यस्यास्ते सुखं वशी}
{नवद्वारे पुरे देही नैव कुर्वन्न कारयन्}


\twolineshloka
{न कर्तृत्वं न कर्माणि लोकस्य सृजति प्रभुः}
{न कर्मफलसंयोगं स्वभावस्तु प्रवर्तते}


\twolineshloka
{नादत्ते कस्य चित्पापं न चैव सुकृतं विभुः}
{अज्ञानेनावृतं ज्ञानं तेन मुह्यन्ति जन्तवः}


\twolineshloka
{ज्ञानेन तु तदज्ञानं येषां नाशितमात्मनः}
{तेषामादित्यवज्ज्ञानं प्रकाशयति तत्परम्}


\twolineshloka
{तद्बुद्धयस्तदात्मानस्तन्निष्ठास्तत्परायणाः}
{गच्छन्त्यपुनरावृत्तिं ज्ञाननिर्धूतकल्मषाः}


\twolineshloka
{विद्याविनयसंपन्ने ब्राह्मणे गवि हस्तिनि}
{शुनि चैव श्वपाके च पण्डिताः समदर्शिनः}


\twolineshloka
{इहैव तैर्जितः सर्गो येषां साम्ये स्थितं मनः}
{निर्दोषं हि समं ब्रह्म तस्माद्ब्रह्मणि ते स्थिताः}


\twolineshloka
{न प्रहृष्येत्प्रियं व्याप्य नोद्विजेत्प्राप्य चाप्रियम्}
{स्थिरबुद्धिरसंमूढो ब्रह्मविद्ब्रह्मणि स्थितः}


\twolineshloka
{बाह्यस्पर्शेष्वसक्तात्मा विन्दत्यात्मनि यत्सुखम्}
{स ब्रह्मयोगयुक्तात्मा सुखमक्षय्यमश्रुते}


\twolineshloka
{ये हि संस्पर्शजा भोगा दुःखयोनय एव ते}
{आद्यन्तवन्तः कौन्तेय न तेषु रमते बुधः}


\twolineshloka
{शक्नोतीहैव यः सोढुं प्राक्शरीरविमोक्षणात्}
{कामक्रोधोद्भवं वेगं स युक्तः स सुखी नरः}


\twolineshloka
{योऽन्तःसुखोऽन्तरारामस्तथान्तर्ज्योतिरेव यः}
{स योगी ब्रह्मनिर्वाणं ब्रह्मभूतोऽधिगच्छति}


\twolineshloka
{लभन्ते ब्रह्मनिर्वाणमृषयः क्षीणकल्मषाः}
{छिन्नद्वैधा यतात्मानः सर्वभूतहिते रताः}


\twolineshloka
{कामक्रोधवियुक्तानां यतीनां यतचेतसाम्}
{अभितो ब्रह्मनिर्वाणं वर्तते विदितात्मनाम्}


\twolineshloka
{स्पर्शान्कृत्वा बहिर्बाह्यांश्चक्षुश्चैवान्तरे भ्रुवोः}
{प्राणापानौ समौ कृत्वा नासाभ्यन्तरचारिणौ}


\threelineshloka
{यतेन्द्रियमनोबुद्धिर्मुनिर्मोक्षपरायणः}
{विगतेच्छाभयक्रोधो यः सदा मुक्त एव सः}
{}


\twolineshloka
{भोक्तारं यज्ञतपसां सर्वलोकमहेश्वरम्}
{सुहृदं सर्वभूतानां ज्ञात्वा मां शान्तिमृच्छति}


\chapter{अध्यायः ३०}
\twolineshloka
{श्रीभगवानुवाच}
{}


\twolineshloka
{अनाश्रितः कर्मफलं कार्यं कर्म करोति यः}
{स संन्यासी च योगी च न निरग्निर्न चाक्रियः}


\twolineshloka
{यं संन्यासमिति प्राहुर्योगं तं विद्धि पाण्डव}
{न ह्यसंन्यस्तसिंकल्पो योगी भवति कश्चन}


\twolineshloka
{आरुरुक्षोर्मुनेर्योगं कर्म कारणमुच्यते}
{योगारूढस्य तस्यैव शमः कारणमुच्यते}


\twolineshloka
{यदा हि नेन्द्रियार्थेषु न कर्मस्वनुषञ्जते}
{सर्वसंकल्पसंन्यासी योगारूढस्तदोच्यते}


\twolineshloka
{उद्धरेदात्मनात्मानं नात्मानमवसादयेत्}
{आत्मैव ह्यात्मनो बन्धुरात्मैव रिपुरात्मनः}


\twolineshloka
{बन्धुरात्माऽऽत्मनस्तस्य येनात्मैवात्मना जितः}
{अनात्मनस्तु शत्रुत्वे वर्तेतात्मैव शत्रुवत्}


\twolineshloka
{जितात्मनः प्रशान्तस्य परमात्मा समाहितः}
{शीतोष्णसुखदुःखेषु तथा मानापमानयोः}


\twolineshloka
{ज्ञानविज्ञानतृप्तात्मा कूटस्थो विजितेन्द्रियः}
{युक्त इत्युच्यते योगी समलोष्ठाश्मकाञ्चनः}


\twolineshloka
{सुहृन्मित्रार्युदासीनमध्यस्थद्वेष्यबन्धुषु}
{साधुष्वपि च पापेषु समबुद्धिर्विशिष्यते}


\twolineshloka
{योगी युञ्जीत सततमात्मानं रहसि स्थितः}
{एकाकी यतचित्तात्मा निराशीरपरिग्रहः}


\twolineshloka
{शुचौ देशे प्रतिष्ठाप्य स्थिरमासनमात्मनः}
{नात्युच्छ्रितं नातिनीचं चैलाजिनकुशोत्तरम्}


\twolineshloka
{तत्रैकाग्रं मनः कृत्वा यतचित्तेन्द्रियक्रियः}
{उपविश्यासने युञ्ज्याद्योगमात्मविशुद्धये}


\twolineshloka
{समं कायशिरोग्रीवं धारयन्नचलं स्थिरः}
{संप्रेक्ष्य नासिकाग्रं स्वं दिशश्चानवलोकथन्}


\twolineshloka
{प्रशान्तात्मा विगतभीर्ब्रह्मचारिव्रते स्थितः}
{प्रनः संयम्य मच्चित्तो युक्त आसीत मत्परः}


\twolineshloka
{युञ्जन्नेवं सदात्मानं योगी नियतमानसः}
{शान्तिं निर्वाणपरमां मत्संस्थामधिगच्छति}


\twolineshloka
{नात्यश्रतस्तु योगोऽस्ति न चैकान्तमनश्नतः}
{न चातिस्वप्रशीलस्य जाग्रतो नैव चार्जुन}


\twolineshloka
{युक्ताहारविहारस्य युक्तचेष्टस्य कर्मसु}
{युक्तस्वप्नाववोधस्य योगो भवति दुःखहा}


\twolineshloka
{यदा विनियतं चित्तमात्मन्येवावतिष्ठते}
{निस्पृहः सर्वकामेभ्यो युक्त इत्युच्यते तदा}


\twolineshloka
{यथा दीपो निवातस्थो नेङ्गते सोपमा स्मृता}
{योगिनो यतचित्तस्य युञ्जतो योगमात्मनः}


\twolineshloka
{यत्रोपरमते चित्तं निरुद्धं योगसेवया}
{यत्र चैवात्मनात्मानं पश्यन्नात्मनि तुष्यति}


\twolineshloka
{मुखमात्यन्तिकं यत्तद्वुद्धिग्राह्यमतीन्द्रियम्}
{वेत्ति यत्र न चैवायं स्थितश्चलति तत्त्वतः}


\twolineshloka
{यं लब्ध्वा चापरं लाभं मन्यते नाधिकं ततः}
{यस्मिन्स्थितो न दुःखेन गुरुणापि विचाल्यते}


\twolineshloka
{तं विद्याद्दुःखसंयोगवयोगं योगसंज्ञितम्}
{स निश्चयेन योक्तव्यो योगोऽनिर्विण्णचेतसा}


\twolineshloka
{संकल्पप्रभवान्कामांस्त्यक्त्वा सर्वानशेषतः}
{मनसैवेन्द्रियग्रामं विनिवम्य समन्ततः}


\twolineshloka
{शनैः शनैरुपरमेद्भुद्ध्या धृतिगृहीतया}
{आत्मसंस्थं मनः कृत्वा न किंचिदपि चिन्तयेत्}


\twolineshloka
{यतो यतो निश्चरति मनश्चञ्चलमस्थिरम्}
{ततस्ततो नियम्यैतदात्मन्येव वशं नयेत्}


\twolineshloka
{प्रशान्तमनसं ह्येनं योगिनं मुखमुत्तमम्}
{उपैति श्चान्तरजसं ब्रह्मभूतमकल्मषम्}


\twolineshloka
{युञ्चयेवं सदात्मानं योगी विमतकल्मषःक}
{मुखेन ब्रह्मसंस्कपर्शमत्यन्तं सुखमश्रुते}


\twolineshloka
{सर्वभूतस्थमात्मानं सर्वभूतानि चात्मनि}
{ईक्षते कयोगयुक्तात्मा सर्वत्र समदर्शनः}


\twolineshloka
{यो मां पश्यति सर्वत्र सर्वं च मयि पश्यति}
{तस्याहं न प्रणश्यामि स च मे न प्रणश्यति}


\twolineshloka
{सर्वभूतस्थितं यो मां भजत्येकत्वमास्थितः}
{सर्वथा वर्तमानोऽपि स योगी मयि वर्तते}


\threelineshloka
{आत्मौपम्येन सर्वत्र समं पश्यति योऽर्जुन}
{सुखंक वा यदि वा दुःखं स योगी मरमो मतः ॥अर्जुन उवाच}
{}


\twolineshloka
{योऽयं योगस्त्वया प्रोक्तः साम्येन मधुसूदन}
{एतस्याहं न पश्यामि चञ्चलत्वात्स्थितिं स्थिरां}


\threelineshloka
{चञ्चलं हि मनः कृष्ण प्रमाथि बलवद्दृढम्}
{तस्याहं निग्रहं मन्ये वायोरिव सुदुष्करम् ॥श्रीभगवानुवाच}
{}


\twolineshloka
{असंशयं महाबाहो मनो दुर्निग्रहं चलम्}
{अभ्यासेन तु कौन्तेय वैराग्येण च गृह्यते}


\threelineshloka
{असंयतात्मना योगो दुष्प्राप इति मे मतिः}
{वश्यात्मना तु यतता शक्योऽऽवाप्तुमुपायतः ॥अर्जुन उवाच}
{}


\twolineshloka
{अयतिः श्रद्धयोपेतो योगाच्चलितमानसः}
{अप्राप्य योगसंसिद्धिं कां गतिं कृष्ण गच्छति}


\twolineshloka
{कच्चिन्नोभयविभ्रष्टश्छिन्नाभ्रमिव नश्यति}
{अप्रतिकष्ठो महाबाहो विमूढो ब्रह्मणः पथि}


\threelineshloka
{एतन्मे संशयं कृष्ण छेत्तुमर्हस्यशेषतः}
{त्वदन्यः संशयस्यास्य छेत्ता न ह्युपपद्यते ॥श्रीभगवानुवाच}
{}


\twolineshloka
{पार्थ नैवेह नामुत्र विनाशस्तस्य विद्यते}
{न हि कल्याणकृत्कश्चिद्दुर्गतिं तात गच्छति}


\twolineshloka
{प्राप्य पुण्यकृतां लोकानुषित्वा शाश्वतीः समाः}
{शुचीनां श्रीमतां गेहे योगभ्रष्टोऽभिजायते}


\twolineshloka
{अथवा योगिनामेव कुले भवति धीमताम्}
{एतद्धि दुर्लभतरं लोके जन्म यदीदृश्यम्}


\twolineshloka
{तत्र तं बुद्धिसंयोगं लभते पौर्वदेहिकम्}
{यतते च ततो भूयः संसिद्धौ कुरुनन्दन}


\twolineshloka
{पूर्वाभ्यासेन तेनैव ह्रियते ह्यवशेऽपि सः}
{जिज्ञासुरपि योगस्य शब्दब्रह्मातिवर्तते}


\twolineshloka
{प्रयत्नाद्यतमानस्तु योगी संशुद्धकिल्विषः}
{अनेकन्मसंसिद्धस्ततो याति परां गतिम्}


\twolineshloka
{तपस्विभ्योऽधिको योगी ज्ञानिभ्योपि मतोऽधिकः}
{कर्मिभ्यश्चाधिको योगी तस्माद्योगी भावार्जुन}


\twolineshloka
{योगिनामपि सर्वेषां मद्गतेनान्तरात्मना}
{श्रद्धावान्भजते यो मां स मे युक्ततमो मतः}


\chapter{अध्यायः ३१}
\twolineshloka
{श्रीभगवानुवाच}
{}


\twolineshloka
{मय्यासक्तमनाः पार्थ योगं युञ्जन्मदाश्रयः}
{असंशयं समग्रं मां यथा ज्ञास्यसि तच्छृणु}


\twolineshloka
{ज्ञानं तेऽहं सविज्ञानमिदं वक्ष्याम्यशेषतः}
{यज्ज्ञात्वा नेह भूयोऽन्यज्ज्ञातव्यमवशिष्यते}


\twolineshloka
{मनुष्याणां सहस्रेषु कश्चिद्यतति सिद्धये}
{यततामपि सिद्धानां कश्चिन्मां वेत्ति तत्त्वतः}


\twolineshloka
{भूमिरापोऽनलो वायुः स्वं मनो बुद्धिरेव च}
{अहंकार इती यं मे भिन्ना प्रकृतिरष्टधा}


\twolineshloka
{अपरेयमितस्त्वन्यां प्रकृतिं विद्धि मे पराम्}
{जीवभूतां महाबाहो ययेदं धार्यते जगत्}


\twolineshloka
{एतद्योनीनि भूतानि सर्वाणीत्युपधारय}
{अहं कृत्स्नस्य जगतः प्रभवः प्रलयस्तथा}


\twolineshloka
{मत्तः परतरं नान्यत्किंचिदस्ति धनंजय}
{मयि सर्वमिदं प्रोतं सूत्रे मणिगणा इव}


% Check verse!
रसोऽहमंप्सु कौन्तेय प्रभाऽस्मि शशिसूर्ययोः ॥प्रणवः सर्ववेदेषु शब्दः खे पौरुषं नृषु
\twolineshloka
{पुण्यो गन्धः पृथिव्यां च तेजश्चास्मि विभावसौ}
{जीवनं सर्वभूतेषु तपश्चास्मि तपस्विषु}


\twolineshloka
{बीजं मां सर्वभूतानां विद्धि पार्थ सनातनम्}
{बुद्धिर्बुद्धिमतामस्मि तेजस्तेजस्विनामहम्}


\twolineshloka
{बलं बलवतां चाहं कामरागविवर्जितम्}
{धर्माविरुद्धो भूतेषु कामोऽस्मि भरतर्षभ}


\twolineshloka
{ये चैव सात्विका भावा राजसास्तामसाश्च ये}
{मत्त एवेति तान्विद्धि न त्वहं तेषु ते मयि}


\twolineshloka
{त्रिभिर्गुणमयैर्भावैरेभिः सर्वमिदं जगत्}
{मोहितं नाभिजानाति मामेभ्यः परमव्ययम्}


\twolineshloka
{दैवी ह्येषा गुणमयी मम माया दुरत्यया}
{मामेव ये प्रपद्यन्ते मायामेतां तरन्ति ते}


\twolineshloka
{न मां तुष्कृतिनो मूढाः प्रपद्यन्ते नराधमाः}
{माययाऽपहृतज्ञाना आसुरं भावमाश्रिताः}


\twolineshloka
{चुत्रर्विधा भजन्ते मां जनाः सुकृतिनोऽर्जुन}
{आर्तो जिज्ञासुरर्थार्थी ज्ञानी च भरतर्षभ}


\twolineshloka
{तेषां ज्ञानी नित्ययुक्त एकभक्तिर्विशिष्यते}
{प्रियो हि ज्ञानिनोऽत्यर्थमहं स च मम प्रियः}


\twolineshloka
{उदाराः सर्व एवैते ज्ञानी त्वात्मैव मे मतम्}
{आस्थितःक स हि युक्तात्मा मामेवानुत्तमां गतिं}


\twolineshloka
{बहूनां जन्मनामन्ते ज्ञानवान्मां प्रपद्यते}
{वासुदेवः सर्वमिति स महात्मा सुदुर्लभः}


\twolineshloka
{कामैस्तैस्तैर्हृतज्ञानाः प्रपद्यन्तेऽन्यदेवताः}
{तं तं नियममास्थाय प्रकृत्या नियताः स्वया}


\twolineshloka
{यो यो यां यां तनुं भक्तः श्रद्धयार्चितुमिच्छति}
{तस्य तस्याचलां श्रद्धां तामेव विदधाम्यहम्}


\twolineshloka
{स तया श्रद्धया युक्तस्तस्याराधनमीहते}
{लभते च ततः कामान्मयैव विहितान्हितान्}


\twolineshloka
{अन्तवत्तु फलं तेषां तद्भवत्यल्पमेधसाम्}
{देवान्देवयजो यान्ति मद्भक्ता यान्ति मामपि}


\twolineshloka
{अव्यक्तं व्यक्तिमापन्नं मन्यन्ते मामबुद्धयः}
{परं भावमजानन्तो ममाव्ययमनुत्तमम्}


\twolineshloka
{नाहं प्रकाशः सर्वस्य योगमायासमावृतः}
{मूढोऽयं नाभिजानाति लोको मामजमव्ययम्}


\twolineshloka
{वेदाहं समतीतानि वर्तमानानि चार्जुन}
{भविष्याणि च भूतानि मां तु वेद न कश्चन}


\twolineshloka
{इच्छाद्वेषसमुत्थेन द्वन्द्वमोहेन भारत}
{सर्वभूतानि संमोहं सर्गे यान्ति परंतप}


\twolineshloka
{योषां त्वन्तगतं पापं जनानां पुण्यकर्मणाम्}
{ते द्वन्द्वमोहनिर्मुक्त भजन्ते मां दृढव्रताः}


\twolineshloka
{जरामरणमोक्षाय मामाश्रित्य यतन्ति ये}
{ते ब्रह्म तद्विदुः कृत्स्नमध्यात्मं कर्म चाखिलम्}


\twolineshloka
{साधिभूताधिदैवं मां साधियज्ञं च ये विदुः}
{प्रयाणकालेऽपि च मां ते विदुर्युक्तचेतसः}


\chapter{अध्यायः ३२}
\twolineshloka
{अर्जुन उवाच}
{}


\twolineshloka
{किं तद्ब्रह्म किमध्यात्मं किं कर्म पुरुषोत्तम्}
{अधिभूतं च किं प्रोक्तमधिदैवं किमुच्यते}


\threelineshloka
{अधियज्ञः कथं कोऽत्र देहेऽस्मिन्मधुसीदन}
{प्रयाणकाले च कथं ज्ञेयोऽसि नियतात्मभिः ॥श्रीभगवानुवाच}
{}


\twolineshloka
{अक्षरं ब्रह्म परमं स्वभावोऽध्यात्ममुच्यते}
{भूतभावोद्भवकरो विसर्गः कर्मसंज्ञितः}


\twolineshloka
{अधिभूतं क्षरो भावः पुरुषश्चाधिदैवतम्}
{अधियज्ञोऽहमेवात्र देहे देहभृतां वर}


\twolineshloka
{अन्तकाले च मामेव स्मरन्मुक्त्वा कलेवरम्}
{यः प्रयाति स मद्भावं याति नास्त्यत्र संशयः}


\twolineshloka
{यं यं वापि स्मरन्भावं त्यजत्यन्ते कलेवरम्}
{तं तमेवैति कौन्तेय सदा तद्भावभावितः}


\twolineshloka
{तस्मात्सर्वेषु कालेषु मामनुस्मर युध्य च}
{मय्यर्पितमनोबुद्धिर्मामेवैष्यस्यसंशयः}


\twolineshloka
{अभ्यासयोगयुक्तेन चेतसाऽनान्यगामिना}
{परमं पुरुषं दिव्यं याति पार्थानुचिन्तयन्}


\twolineshloka
{कविं पुराणमनुशासितार-मणोरणीयांसमनुस्मरेद्यः}
{सर्वस्य धातारमचिन्त्यरूप-मादित्यवर्णं तमसः परस्तात्}


\twolineshloka
{प्रयाणकाले मनसाऽचलेनभक्त्या युक्तो योगबलेन चैव}
{भ्रुवोर्मध्ये प्राणमावेश्य सम्य-क्स तं परं पुरुषमुपैति दिव्यम्}


\twolineshloka
{यदक्षरं वेदविदो वदन्तिविशन्ति यद्यतयो वीतरागाः}
{यदिच्छन्तो ब्रह्मचर्यं चरन्तितत्ते पदं संग्रहेण प्रवक्ष्ये}


\twolineshloka
{सर्वद्वाराणि संयम्य मनो हृदि निरुध्य च}
{मूर्ध्न्याधायात्मनः प्राणमास्थितो योगधारणां}


\twolineshloka
{ओमित्येकाक्षरं ब्रह्म व्याहरन्मामनुस्मरन्}
{यः प्रयाति त्यजन्दें स याति परमां गतिम्}


\twolineshloka
{अनन्यचेताः सततं यो मां स्मरति नित्यशः}
{तस्याहं सुलभः पार्थ नित्ययुक्तस्य योगिनः}


\twolineshloka
{मामुपेत्य पुनर्जन्म दुःखालयमशाश्वतम्}
{नाप्नुवन्ति महात्मानः संसिद्धिं परमां गताः}


\twolineshloka
{आब्रह्मभुवनाल्लोकाः पुनरावर्तिनोऽर्जुन}
{मामुपेत्य तु कौन्तेय पुनर्जन्म न विद्यते}


\twolineshloka
{सहस्रयुगपर्यन्तमहर्यद्ब्रह्मणो विदुः}
{रात्रिं युगसहस्रां तां तेऽहोरात्रविदो जनाः}


\twolineshloka
{अव्यक्ताद्व्यक्तयः सर्वाः प्रभवन्त्यहरागमे}
{रात्र्यामे प्रलीयन्ते तत्रैवाव्यक्तसंज्ञके}


\twolineshloka
{भूतग्रामः स एवायं भूत्वा भूत्वा प्रलीयते}
{रात्र्यगमेऽवशः पार्थत प्रभवत्यहरागमे}


\twolineshloka
{परस्तस्मात्तु भावोऽन्यो व्यक्तोऽव्यक्तात्सनातनः}
{यः स सर्वेषु भूतेषुक नश्यत्सु न विनश्यति}


\twolineshloka
{अव्यक्तोऽक्षर इत्युक्तस्तमाहुः परमां गतिम्}
{यं प्राप्य न निवर्तन्ते तद्धाम परमं मम}


\twolineshloka
{पुरुषः स परः पार्थ भक्त्या लभ्यस्त्वनन्यया}
{यस्यान्तस्थानि भूतानि येन सर्वमिदं ततम्}


\twolineshloka
{यत्र काले त्वनावृत्तिमावृत्तिं चैव योगिनः}
{प्रयाता यान्ति तं कालं वक्ष्यामि भरतर्षभ}


\twolineshloka
{अग्निर्ज्योतिरहः शुक्लः षण्मासा उत्तरायणम्}
{तत्र प्रयाता गच्छन्ति ब्रह्म ब्रह्मविदो जनाः}


\twolineshloka
{धूमो रात्रिस्तथा कृष्णः षण्मसा दक्षिणायनम्}
{तत्र चान्द्रमसं ज्योतिर्योगी प्राप्य निवर्तते}


\twolineshloka
{शुक्लकृष्णे गती ह्येते जगतः शाश्वते मते}
{एकया यात्यनावृत्तिमन्ययाऽऽवर्तते पुनः}


\twolineshloka
{नैते सृती पार्थ जानन्योगी मुह्यति कश्चन}
{तस्मात्सर्वेषु कालेषु योगयुक्तो भवार्जुन}


\twolineshloka
{वेदेषु यज्ञेषु तपःसु चैवदानेषु यत्पुण्यफलं प्रदिष्टम्}
{अत्येति तत्सर्वमिदं विदित्वायोगी परं स्थानमुपैति चाद्यम्}


\chapter{अध्यायः ३३}
\twolineshloka
{श्रीभगवानुवाच}
{}


\twolineshloka
{इदं तु ते गुह्यतमं प्रवक्ष्याम्यनसूयवे}
{ज्ञानं विज्ञानसहितं यज्ज्ञात्वा मोक्ष्यसेऽशुभात्}


\twolineshloka
{राजविद्या राजगुह्यं पवित्रमिदमुत्तमम्}
{प्रत्यक्षावगमं धर्म्यं सुसुखं कर्तुमव्ययम्}


\twolineshloka
{अश्रद्दधानां पुरुषा धर्मस्यास्य परंतप}
{अप्राप्य मां निवर्तन्ते मृत्युसंसारवर्त्मनि}


\twolineshloka
{मया ततमिदं सर्वं जगदव्यक्तमूर्तिना}
{मत्स्थानि सर्वभूतानि न चाहं तेष्ववस्थितःक}


\twolineshloka
{न च मत्स्थानि भूतानि पश्य मे योगमैश्वरम्}
{भूतभृन्न च भूतस्थो ममात्मा भूतभावनः}


\twolineshloka
{यथाऽऽकाशस्थितो नित्यं वायुः सर्वत्रगो महान्}
{तथा सर्वाणि भूतानि मत्स्थानीत्युपधारय}


\twolineshloka
{सर्वभूतानि कौन्तेय प्रकृतिं यान्ति मामिकाम्}
{कल्पक्षये पुनस्तानि कल्पादौ विसृजाम्यहम्}


\twolineshloka
{प्रकृतिं स्वामवष्टभ्य विसृजामि पुनः पुनः}
{भूतग्राममिमं कृत्स्नमवशं प्रकृतेर्वशात्}


\twolineshloka
{न च मां तानि कर्माणि निबध्नन्ति धनंजय}
{उदासीनवदासीनमसक्तं तेषु कर्मसु}


\twolineshloka
{मयाऽऽध्यक्षेण प्रकृतिः सूयते सचराचरम्}
{हेतुनाऽनेन कौन्तेय जगद्विपरिवर्तते}


\twolineshloka
{अवजानन्ति मां मूढा मानुषीं तनुमाश्रितम्}
{परं भावमजानन्तो मम भूतमहेश्वरम्}


\twolineshloka
{मोघाशा मोघकर्माणो मोघज्ञाना विचेतसः}
{राक्षसीमासुरीं चैव प्रकृतिं मोहिनीं श्रिताः}


\twolineshloka
{महात्मानस्तु मां पार्थ दैवीं प्रकृतिमाश्रिताः}
{भजन्त्यनन्यमनसो ज्ञात्वा भूतादिमव्ययम्}


\twolineshloka
{सततं कीर्तयन्तो मां यतन्तश्च दृढव्रताः}
{नमस्यन्तश्च मां भक्त्या नित्ययुक्ता उपासते}


\twolineshloka
{ज्ञानयज्ञेन चाप्यन्ते यजन्तो मामुपासते}
{एकत्वेन पृथक्त्कवेन बहुधा विश्वतोमुम्}


\twolineshloka
{अहं क्रतुरहं यज्ञः स्वधाऽहमहमौषधम्}
{मन्त्रोऽहमहमेवाज्यमहमग्निरहं हुतम्}


\twolineshloka
{पिताऽहमस्य जगतो माता धाता पितामहः}
{वेद्यं पवित्रमोंकार ऋक् साम यजुरेव च}


\twolineshloka
{गतिर्भर्ता प्रभुः साक्षी निवासः शरणं सुहृत्}
{प्रभवः प्रलयः स्थानं निधानं बीजमव्ययम्}


\twolineshloka
{तपाम्यहमहं वर्षं निगृह्णाम्युत्सृजामि च}
{अमृतं चैव मृत्युश्च सदसच्चाहमर्जुन}


\twolineshloka
{त्रैविद्या मां सोमपाः पूतपापायज्ञेरिष्ट्वा स्वर्गतिं प्रार्थयन्ते}
{ते पुण्यमासाद्य सुरेन्द्रलोक-मश्रन्ति दिव्यान्दिवि देवभोगान्}


\twolineshloka
{ते तं भुक्त्वा स्वर्गलोकं विशालंक्षीणो पुण्ये पर्त्यलोकं विशन्ति}
{एवं त्रयीधर्ममनुप्रपन्नागतागतं कामकामा लभन्ते}


\twolineshloka
{अनन्याश्चिन्तयन्तो मां ये जनाः पर्युपासते}
{तेषां नित्याभियुक्तानां योगक्षेमं वहाम्यहम्}


\twolineshloka
{येऽप्यन्यदेवताभक्ता यजन्ते श्रद्धयान्विताः}
{तेऽपि मामेव कौन्तेय यजन्त्यविधिपूर्वकम्}


\twolineshloka
{अहं हि सर्वयज्ञानां भोक्ता च प्रभुरेव च}
{न तु मामभिजानन्ति तत्त्वेनातश्च्यवन्ति ते}


\twolineshloka
{यान्ति देवव्रता देवान्पितॄन्यान्ति पितृव्रताः}
{भूतानि यान्ति भूतेज्या यान्ति मद्याजिनोपि माम्}


\twolineshloka
{पत्रं पुष्पं फलं तोयं यो मे भक्त्या प्रयच्छति}
{तदहं भक्त्युपहृतमश्नामि प्रयतात्मनः}


\twolineshloka
{यत्करोषि यदश्रासि यज्जुहोषि ददासि यत्}
{यत्तपस्यसि कौन्तेय तत्कुरुष्व मदर्पणम्}


\twolineshloka
{शुभाशुभफलैरेवं मोक्ष्यसे कर्मबन्धनैः}
{संन्यासयोगयुक्तात्मा विमुक्तो मामुपैष्यसि}


\twolineshloka
{समोऽहं सर्वभूतेषु न मे द्वेष्योऽस्ति न प्रियः}
{ये भजन्ति तु मां भक्त्या मयि ते तेषु चाप्यहम्}


\twolineshloka
{अपि चेत्सुदुराचारो भजते मामनन्यभाक्}
{साधुरेव स मन्तव्यः सम्यग्व्यवसितो हि सः}


\threelineshloka
{क्षिप्रं भवति धर्मात्मा शश्वच्छान्तिं निगच्छति}
{कौन्तेय प्रतिजानीहि न मे भक्तः प्रणश्यति}
{}


\twolineshloka
{मां हि पार्थ व्यपाश्रित्य येऽपि स्युः पापयोनयः}
{स्त्रियो वैश्यास्तथा शूद्रास्तेऽपि यान्ति परां गतिं}


\twolineshloka
{किं पुनर्ब्राह्मणाः पुण्या भक्ता राजर्षयस्तथा}
{अनित्यमसुखं लोकमिमं प्राप्य भजस्व माम्}


\twolineshloka
{मन्मना भव मद्भक्तो मद्याजी मां नमस्कुरु}
{मामेवैष्यसि युक्त्वैवमात्मानं मत्परायणः}


\chapter{अध्यायः ३४}
\twolineshloka
{श्रीभगवानुवाच}
{}


\twolineshloka
{भूय एव महाबाहो श्रृणु मे परमं वचः}
{यत्तेऽहं प्रीयमाणाय वक्ष्यामि हितकाम्यया}


\twolineshloka
{न मे विदुः सुरगणाः प्रभवं न महर्षयः}
{अहमादिर्हि देवानां महर्षीणां च सर्वशः}


\twolineshloka
{यो मामजमनादिं च वेत्ति लोकमहेश्वरम्}
{असंमूढः स मर्त्येषु सर्वपापैः प्रमुच्यते}


\twolineshloka
{बुद्धिर्ज्ञानमसंमोहः क्षमा सत्यं दमः शमः}
{सुखं दुःखं भवो भवो भयं चाभयमेव च}


\twolineshloka
{अहिंसा समता तुष्टिस्तपो दानं यशोऽयशः}
{भवन्ति भावा भूतानां मत्त एव पृथग्विधाः}


\twolineshloka
{महर्षयः सप्त पूर्वे चत्वारो मनवस्तथा}
{मद्भाव मानसा जाता येषां लोक इमाः प्रजाः}


\twolineshloka
{एतां विभूतिं योगं च मम यो वेत्ति तत्त्वतः}
{सोऽविकम्पेन योगेन युज्यते नात्र संशयः}


\twolineshloka
{अहं सर्वस्य प्रभवो मत्तः सर्वं प्रवर्तते}
{इति मत्वा भजन्ते मां बुधा भावसमन्विताः}


\twolineshloka
{मच्चित्ता मद्गतप्राणा बोधयन्तः परस्परम्}
{कथयन्तश्च मां नित्यं तुष्यन्ति च रमन्ति च}


\twolineshloka
{तेषां सततयुक्तानां भजतां प्रीतिपूर्वकम्}
{ददामि बुद्धियोगं तं येन मामुपयान्ति ते}


\threelineshloka
{तेषामेवानुकम्पार्थमहमज्ञानजं तमः}
{नाशयाम्यात्मभावस्थो ज्ञानदीपेन भास्वता ॥अर्जुन उवाच}
{}


\twolineshloka
{परं ब्रह्म परं धाम पवित्रं परमं भवान्}
{पुरुषं शाश्वतं दिव्यमादिदेवमजं विभुम्}


\twolineshloka
{आहुस्त्वामृषयः सर्वे देवर्षिर्नारदस्तथाक}
{असितो देवलो व्यासः स्वयं चैव ब्रवीषि मे}


\twolineshloka
{सर्वतेमदृतं मन्ये यन्मां वदसि केशव}
{न हि ते भगवन्व्यक्तिं विदुर्देवा न दानवाः}


\twolineshloka
{स्वयमेवात्मनात्मानं वेत्थ त्वं पुरुषोत्तम}
{भूतभावन भूतेश देवदेव जगत्पते}


\twolineshloka
{वक्तुमर्हस्यशेषेण दिव्या ह्यात्मविभूतयः}
{याभिर्विभूतिभिर्लोकानिमांस्त्वं व्याप्य तिष्ठसि}


\twolineshloka
{कथं विद्यामहं योगिंस्त्वां सदा परिचिन्तयन्}
{केषु केषु च भावेषु चिन्त्योऽसि भगवन्मया}


\threelineshloka
{विस्तरेणात्मनो योगं विभूतिं च जनार्दन}
{भूयः कथय तृप्तिर्हि शृण्वतो नास्ति मेऽमृतम् ॥श्रीभगवनुवाच}
{}


\twolineshloka
{हन्त ते कथयिष्यामि दिव्या ह्यात्मविभूतयः}
{प्राधान्यतः कुरुश्रेष्ठ नास्त्यन्तो विस्तरस्य मे}


\twolineshloka
{अहमात्मा गुडाकेश सर्वभूताशयस्थितःक}
{अहमादिश्च मध्यं च भूतानामन्त एव च}


\twolineshloka
{आदित्यानामहं विष्णुर्ज्योतिषां रविरंशुमान्}
{मरीचिर्मरुतामस्मि नक्षत्राणामहं शशी}


\twolineshloka
{वेदानां सामवेदोऽस्मि देवानामस्मि वासवः}
{इन्द्रियाणां मनश्चास्मि भूतानामस्मि चेतना}


\twolineshloka
{रुद्राणां शंकरश्चास्मि वित्तेशो यक्षरक्षसाम्}
{वसूनां पावकश्चास्मि मेरुः शिखरिणामहम्}


\twolineshloka
{पुरोधसां च मुख्यं मां विद्धि पार्थ बृहस्पतिम्}
{सेनानीनामहं स्कन्दः सरसामस्मि सागरः}


\twolineshloka
{महर्षीणां भृगुरहं गिरामस्म्येकमक्षरम्}
{यज्ञानां जपयज्ञोऽस्मि स्थावराणां हिमालयः}


\twolineshloka
{अश्वत्थः सर्ववृक्षाणां देवर्षीणां च नारदः}
{गन्धर्वाणां चित्ररथः सिद्धानां कपिलो मुनिः}


\twolineshloka
{उच्चैः श्रवसमश्वानां विद्धि माममृतोद्भवम्}
{ऐरावतं गजेन्द्राणां नराणां च नराधिपम्}


\twolineshloka
{आयुधानामहं वज्रं धेनूनामस्मि कामधुक्}
{प्रजनश्चास्मि कन्दर्पः सर्पाणामस्मि वासुकिः}


\twolineshloka
{अनन्तश्चास्मि नागानां वरुणो यादसामहम्}
{पितॄणामर्यमा चास्मि यमः संयमतामहम्}


\twolineshloka
{प्रह्लादश्चास्मि दैत्यानां कालः कलयतामहम्}
{मृगाणां च मृगेन्द्रोऽहं वैनतेयश्च परिणाम्}


\twolineshloka
{पवनः पवतामस्मि रामः शस्त्रभृतामहम्}
{झषाणां मकरश्चास्मि स्रोतसामस्मि जाह्नवी}


\twolineshloka
{सर्गाणामादिरन्तश्च मध्यं चैवाहमर्जुन}
{अध्यात्मविद्या विद्यानां वादः प्रवदतामहम्}


\twolineshloka
{अक्षराणामकारोऽस्मि द्वन्द्वः सामासिकस्य च}
{अहमेवाक्षयः कालो धाताऽहं विश्वतोमुखः}


\twolineshloka
{मृत्युः सर्वहरश्चाहमुद्भवश्च भविष्यताम्}
{कीर्तिः श्रीर्वाक्क नारीणां स्मृतिर्मेधा धृतिः क्षमा}


\twolineshloka
{बृहत्साम कतथा साम्नां गायत्री छन्दसामहम्}
{मासानां मार्गशीर्षोऽहमृतूनां कुसुमाकरः}


\twolineshloka
{द्यूतं छलयतामस्मि तेजस्तेजस्विनामहम्}
{जयोऽस्मि व्यवसायोऽस्मि सत्त्वं सत्त्ववतामहम्}


\twolineshloka
{वृष्णीनां वासुदेवोऽस्मि पाण्डवानां धनञ्जयः}
{मुनीनामप्यहं व्यासः कवीनामुशना कविः}


\twolineshloka
{दण्डो दमयतामस्मि नीतिरस्मि जिगीषताम्}
{मौनं चैवास्मि गुह्यानां ज्ञानं ज्ञानवतामहम्}


\twolineshloka
{यच्चापि सर्वभूतानां बीजं तदहमर्जुन}
{न तदस्ति विना यत्स्यान्मया भूतं चराचरम्}


\twolineshloka
{नान्तोऽस्ति मम दिव्यानां विभूतीनां परंतप}
{एष तूद्देशतः प्रोक्तो विभूतेर्विस्तरो मया}


\twolineshloka
{यद्यद्विभूतितत्सत्त्वं श्रीमदूर्जितमेव वा}
{तत्तदेवावगच्छ त्वं मम तेजोंशसंभवम्}


\twolineshloka
{अथवा बहुनैतेन किं ज्ञातेन तवार्जुन}
{विष्टभ्याहमिदं कृत्स्नमेकांशेन स्थितो जगत्}


\chapter{अध्यायः ३५}
\twolineshloka
{अर्जुन उवाच}
{}


\twolineshloka
{मदनुग्रहाय परमं गुह्यमध्यात्मसंज्ञितम्}
{यत्त्वयोक्तं वचस्तेन मोहोऽयं विगतो मम}


\twolineshloka
{भवाप्ययौ हि भूतानां श्रुतौ विस्तरशो मया}
{त्वत्तः कमलपत्राक्ष माहात्म्यमपि चाव्ययम्}


\twolineshloka
{एवमेतद्यथात्थ त्वमात्मानं परमेश्वर}
{द्रष्टुमिच्छामि ते रूपमैश्वरं पुरुषोत्तम}


\threelineshloka
{मन्यसे यदि तच्छक्यं मया द्रष्टुमिति प्रभो}
{योगेश्वर ततो मे त्वं दर्शयात्मानमव्ययम् ॥श्रीभगवानुवाच}
{}


\twolineshloka
{पश्य मे पार्थ रूपाणि शतशोऽथ सहस्रशः}
{नानाविधानि दिव्यानि नानावर्णाकृतीनि च}


\twolineshloka
{पश्यादित्यान्वसून्रुद्रानश्विनौ मरुतस्तथा}
{बहून्यदृष्टपूर्वाणि पश्याऽश्चर्याणि भारत}


\twolineshloka
{इहैकस्थं जगत्कृत्स्नं पश्याद्य सचराचरम्}
{मम देहे गुडाकेश यच्चान्यद्द्रष्टुमिच्छसि}


\threelineshloka
{न तु मां शक्ष्यसे द्रष्टुमनेनैव स्वचक्षुषा}
{दिव्यं ददामि ते चक्षुः पश्य मे योगमैश्वरम् ॥सञ्जय उवाच}
{}


\twolineshloka
{एवमुक्त्वा ततो राजन्महायोगेश्वरो हरिः}
{दर्शयामास पार्थाय मरमं रूपमैश्वरम्}


\twolineshloka
{अनेकवक्रनयनमनेकाद्भुतदर्शनम्}
{अनेकदिव्याभारणं दिव्यानेकोद्यतायुधम्}


\twolineshloka
{दिव्यमाल्याम्बरधरं दिव्यगन्धानुलेपनम्}
{सर्वाश्चर्यमयं देवमनन्तं विश्वतोमुखम्}


\twolineshloka
{दिवि सूर्यसहस्रस्य भवेद्युगपदुत्थिता}
{यदि भाः सदृशी सा स्याद्भासस्तस्य महात्मनः}


\twolineshloka
{तत्रैकस्थं जगत्कृत्स्नं प्रविभक्तमनेकधा}
{अपश्यद्देवदेवस्य शरीरे पाण्डवस्तदा}


\threelineshloka
{ततः स विस्मयाविष्टो हृष्टरोमा धऩञ्जयः}
{प्रणम्य शिरसा देवं कृताञ्जलिरभाषत ॥अर्जुन उवाच}
{}


\twolineshloka
{पश्यामि देवांस्तव देव देहेसर्वांस्तथा भूतविशेषसङ्घान्}
{ब्रह्माणमीशं कमलासनस्थ-मृषींश्च सर्वानुरगांश्च दिव्यान्}


\twolineshloka
{अनेकबाहूदरवक्रनेत्रंपश्यामि त्वां सर्वतोऽनन्तरूपम्}
{नान्तं न मध्यं न पुनस्तवादिंपश्यामि विश्वेश्वर विश्वरूप}


\twolineshloka
{किरीटिनं गदिनं चक्रिणं चतेजोराशिं सर्वतो दीप्तिमन्तम्}
{पश्यामि त्वां दुर्निरीक्ष्यं समन्ता-द्दीप्तानलार्कद्युतिमप्रमेययम्}


\twolineshloka
{त्वमक्षरं परमं वेदितव्यंत्वमस्य विश्वस्य परं निधानम्}
{त्वमव्ययः शाश्वतधर्मगोप्तासनातनस्त्वं पुरुषो मतो मे}


\twolineshloka
{अनादिमध्यान्तमनन्तवीर्य-मनन्तबाहुं शशिसूर्यनेत्रम्}
{पश्यामि त्वां दीप्तहुताशवक्रंस्वतेजसा विश्वमिदं तपन्तम्}


\twolineshloka
{द्यावापृथिव्योरिदमन्तरं हिव्याप्तं त्वयैकेन दिशश्च सर्वाः}
{दृष्ट्वाद्भुतं रूपमुग्रं तदेवंलोकत्रयं प्रव्यथितं महात्मन्}


\twolineshloka
{अमी हि त्वा सुरसङ्घा विशन्तिकेचिद्भीताः प्राञ्जलयो गृणान्ति}
{स्वस्तीत्युक्त्वा महर्षिसिद्धसङ्घाःस्तुवन्ति त्वां स्तुतिभिः पुष्कलाभिः}


\twolineshloka
{रुद्रादित्या वसवो ये च साध्याविश्वेऽश्विनौ मरुतश्चोष्मपाश्च}
{गन्धर्वयक्षासुरसिद्धसङ्घावीक्षन्ते त्वां विस्मिताश्चैव सर्वे}


\twolineshloka
{रूपं महत्ते बहुवक्रनेत्रंमहाबाहो बहुबाहूरुपादम्}
{बहूदरं बहुदंष्ट्राकरालंदृष्ट्वा लोकाः प्रव्यथितास्तथाहम्}


\twolineshloka
{नभःस्पृशं दीप्तमनेकवर्णंव्यात्ताननं दीप्तविशालनेत्रम्}
{दृष्ट्वा हि त्वा प्रव्यथितान्तरात्माधृतिं न विन्दामि शमं च विष्णो}


\twolineshloka
{दंष्ट्राकरालानि च ते सुखानिदृष्ट्वैव कालानलसन्निभानि}
{दिशो न जाने न लभे च शर्मप्रसीद देवेश जगन्निवास}


\twolineshloka
{अमी च त्वां धृतराष्ट्रस्य पुत्राःसर्वे सहैवावनिपालसङ्घैः}
{भीष्मो द्रोणः सूतपुत्रस्तथासौसहास्मदीयैरपि योधमुख्यैः}


\twolineshloka
{वक्राणि ते त्वरमाणा विशन्तिदंष्ट्राकरालानि भयानकानि}
{केचिद्विलग्ना दशनान्तरेषुसंदृश्यन्ते चूर्णितैरुत्तमाङ्गैः}


\twolineshloka
{यथा नदीनां बहवोऽम्बुवेगाःसमुद्रमेवाभिमुखा द्रवन्ति}
{तथा तवामी नरलोकवीराविशन्ति वक्राण्यभिविज्वलन्ति}


\twolineshloka
{यथा प्रदीप्तं ज्वलनं पतङ्गाविशन्ति नाशाय समृद्धवेगाः}
{तथैव नाशाय विशन्ति लोका-स्तवापि वक्राणि समृद्धवेगाः}


\twolineshloka
{लेलिह्यसे ग्रसमानः समन्ता-ल्लोकान्समग्रान्वदनैर्ज्वलद्भिःक}
{तेजोभिरापूर्य जगत्समग्रंभासस्तवोग्राः प्रतपन्ति विष्णो}


\threelineshloka
{आख्याहि मे को भवानुग्ररूपोनमोऽस्तु ते देववर प्रसीद}
{विज्ञातुमिच्छामि भवन्तमाद्यंन हि प्रजानामि तव प्रवृत्तिम् ॥श्रीभगवानुवाच}
{}


\twolineshloka
{कालोऽस्मि लोकक्षयकृत्प्रवृद्धोलोकान्समाहर्तुमिह प्रवृत्तः}
{ऋतेऽपि त्वा न भविष्यन्ति सर्वेयेऽवस्थिताः प्रत्यनीकेषु योधाः}


\twolineshloka
{तस्मात्त्वमुत्तिष्ठ यशो लभस्वजित्वा शत्रून्भुङ्क्ष्व राज्यं समृद्धम्}
{मयैवैते निहताः पूर्वमेवनिमित्तमात्रं भव सव्यसाचिन्}


\threelineshloka
{द्रोणं च भीष्मं च जयद्रथं चकर्णं तथान्यानपि योधवीरान्}
{मया हतांस्त्वं जहि माव्यथिष्ठायुद्ध्यस्व जेतानि रणे सपत्नान् ॥सञ्जय उवाच}
{}


\threelineshloka
{एतच्छ्रुत्वा वचनं केशवस्यकृताञ्जलिर्वेपमानः किरीटी}
{नमस्कृत्वा भूय एवाह कृष्णंसगद्गदं भीतभीतः प्रणम्य ॥अर्जुन उवाच}
{}


\twolineshloka
{स्थाने हृषीकेश तव प्रकीर्त्याजगत्प्रहृष्यत्यनुरज्यते च}
{रक्षांसि भीतानि दिशो द्रवन्तिसर्वे नमस्यन्ति च सिद्धसङ्घाः}


\twolineshloka
{कस्माच्च ते न नमेरन्महात्मन्गरीयसे ब्रह्मणोऽप्यादिकर्त्रे}
{अनन्त देवेश जगन्निवासत्वमक्षरं सदसत्तत्परं यत्}


\twolineshloka
{त्वमादिदेवः पुरुषः पुराण-स्त्वमस्य विश्वस्य परं निधानम्}
{वेत्ताऽसि वेद्यं च परं च धामत्वया ततं विश्वमनन्तरूप}


\twolineshloka
{वायुर्यमोऽग्निर्वरुणः शशाङ्कःप्रजापतिस्त्वं प्रपितामहश्च}
{नमो नमस्तेऽतु सहस्रकृत्वःपुनश्च भूयोऽपि नमो नमस्ते}


\twolineshloka
{नमः पुरस्तादथ पृष्ठतस्तेनमोऽस्तु ते सर्वत एव सर्व}
{अनन्तवीर्यामितविक्रमस्त्वंसर्वं समाप्नोषि ततोऽसि सर्वः}


\twolineshloka
{सखेति मत्वा प्रसभं यदुक्तंहे कृष्ण हे यादव हे सखेति}
{अजानता महिमानं तवेदंमया प्रमादात्प्रणयेन वाऽपि}


\twolineshloka
{यच्चापहासार्थमसत्कृतोऽसिविहारशय्यासनभोजनेषु}
{एकोऽथवाप्यच्युत तत्समक्षंतत्क्षामये त्वामहमप्रमेयम्}


\twolineshloka
{पितासि लोकस्य चराचरस्यत्वमस्य पूज्यश् गुरुर्गरीयान्}
{न त्वत्समोऽस्त्यभ्यधिकः कुतोऽन्योलोकत्रयेऽप्यप्रतिमप्रभावः}


\twolineshloka
{तस्मात्प्रणम्य प्रणिधाय कायंप्रसादये त्वामहमीशमीड्यम्}
{पितेव पुत्रस्य सखेव सख्युःप्रियः प्रियायार्हसि देव सोढुम्}


\twolineshloka
{अदृष्टपूर्वं हृषितोऽस्मि दृष्ट्वाभयेन च प्रव्यथितं मनो मे}
{तदेव मे दर्शय देव रूपंप्रसीद देवेश जगन्निवास}


\threelineshloka
{किरीटिनं गदिनं चक्रहस्त-मिच्छामि त्वां द्रष्टुमहं तथैव}
{तेनैव रूपेण चतुर्भुजेनसहस्रबाहो भव विश्वमूर्ते ॥श्रीभगवानुवाच}
{}


\twolineshloka
{मया प्रसन्नेन तवार्जुनेदंरूपं परं दर्शितमात्मयोगात्}
{तेजोमयं विश्वमनन्तमाद्यंयन्मे त्वदन्येन न दृष्टपूर्वम्}


\twolineshloka
{न वेद यज्ञाध्ययनैर्न दानै-र्न च क्रियाभिर्न तपोभिरुग्रैः}
{एवंरूपः शक्य अहं नृलोकेद्रष्टुं त्वदन्येन कुरुप्रवीर}


\threelineshloka
{मा ते व्यथा मा च विमूढभावोदृष्ट्वा रूपं घोरमीदृङ्भमेदम्}
{व्यपेतभीः प्रीतमनाः पुनस्त्वंतदेव मे रूपमिदं प्रपश्य ॥सञ्जय उवाच}
{}


\threelineshloka
{इत्यर्जुनं वासुदेवस्तथोक्त्वास्वकं रूपं दर्शयामास भूयः}
{आश्वासयामास च भीतमेनंभूत्वा पुनः सौम्यवपुर्महात्मा ॥अर्जुन उवाच}
{}


\threelineshloka
{दृष्ट्वेदं मानुषं रूपं तव सौम्यं जनार्दन}
{इदानीमस्मि संवृत्तः सचेताः प्रकृतिं गतः ॥श्रीभगवानुवाच}
{}


\twolineshloka
{सुदुर्दर्शमिदं रूपं दृष्टवानसि यन्मम}
{देवा अप्यस्य रूपस्य नित्यं दर्शनकाङ्क्षिणः}


\twolineshloka
{नाहं वेदैर्न तपसा न दानेन न चेज्यया}
{शक्य एवंविधो द्रष्टुं दृष्टवानसि मां यथा}


\twolineshloka
{भक्त्या त्वनन्यया शक्य अहमेवंविधोऽर्जुन}
{ज्ञातुं द्रष्टुं च तत्त्वेन प्रवेष्टुं च परंतप}


\twolineshloka
{मत्कर्मकृन्मत्परमो मद्भक्तः सङ्गवर्जितः}
{निर्वैरः सर्वभूतेषु यः स मामेति पाण्डव}


\chapter{अध्यायः ३६}
\twolineshloka
{अर्जुन उवाच}
{}


\threelineshloka
{एवं सततयुक्ता ये भक्तास्त्वां पर्युपासते}
{ये चाप्यक्षरमव्यक्तं तेषां के योगवित्तमाः ॥श्रीभगवानुवाच}
{}


\twolineshloka
{मय्यावेश्य मनो ये मां नित्ययुक्ता उपासते}
{श्रद्धया परयोपेतास्ते मे युक्ततमा मताः}


\twolineshloka
{ये त्वक्षरमनिर्देश्यमव्यक्तं पर्युपासते}
{सर्वत्रगमचिन्त्यं च कूटस्थमचलं ध्रुवम्}


\twolineshloka
{संनियम्येन्द्रियग्रामं सर्वत्र समबुद्धयः}
{ते प्राप्नुवन्ति मामेव सर्वभूतहिते रताःक}


\twolineshloka
{क्लेशोऽधिकतरस्तेषामव्यक्तासक्तचेतसाम्}
{अव्यक्ता हि गदिर्दुःखं देहवद्भिरवाप्यते}


\twolineshloka
{ये तु सर्वाणि कर्माणि मयि संन्यस्य मत्पराः}
{अनन्येनैव योगेन मां ध्यायन्त उपासते}


\twolineshloka
{तेषामहं समुद्धर्ता मृत्युसंसारसागरात्}
{भवामि न चिरात्पार्थ मय्यावेशितचेतसाम्}


\twolineshloka
{मय्येव मन आधत्स्व मयि बुद्धिं निवेशय}
{निवसिष्यसि मय्येव अत ऊर्ध्वं न शंशकयः}


\twolineshloka
{अथ चित्तं समाधातुं न शक्नोषि मयि स्थिरम्}
{अभ्यासयोगेन ततो मामिच्छाऽऽप्तुं धनंजय}


\twolineshloka
{अभ्यासेऽप्यसमर्थोऽसि मत्कर्मपरमो भव}
{मदर्थमपि कर्माणि कुर्वन्सिद्धिमवाप्य्यसि}


\twolineshloka
{अथैतदप्यशक्तोऽसि कर्तुं मद्योगमाश्रितः}
{सर्वकर्मफलत्यागं ततः कुरु यतात्मवान्}


\twolineshloka
{श्रेयो हि ज्ञानमभ्यासाज्ज्ञानाद्ध्यानं विशिष्यते}
{ध्यानात्कर्मफलत्यागस्त्यागाच्छान्तिरनन्तरम्}


\twolineshloka
{अद्वेष्टा सर्वभूतानां मैत्रः करुण एव च}
{निर्ममो निरहंकारः समदुःखसुखः क्षमी}


\twolineshloka
{संतुष्टः सततं योगी यतात्मा दृढनिश्चयः}
{मय्यर्पितमनोबुद्धिर्यो मे भक्तः स मे प्रियः}


\twolineshloka
{यस्मान्नोद्विजते लोको लोकान्नोद्विजते च यः}
{हर्षामर्षभयोद्वेगैर्मुक्तो यः स च मे प्रियः}


\twolineshloka
{अनपेक्षः शुचिर्दक्ष उदासीनो गतव्यथः}
{सर्वारम्भपरित्यागी यो मद्भक्तः स मे प्रियः}


\twolineshloka
{यो न हृष्यति न द्वेष्टि न शोचति न काङ्क्षति}
{शुभाशुभपरित्यागी भक्तिमान्यः स मे प्रियः}


\twolineshloka
{समः शत्रौ च मित्रे च तथा मानापमानयोः}
{शीतोष्णसुखदुःखेषु समः सङ्गविवर्जितः}


\twolineshloka
{तुल्यनिन्दास्तुतिर्मौनी संतुष्टो येन केन चित्}
{अनिकेतः स्थिरमतिर्भक्तिमान्मे प्रियो नरः}


\twolineshloka
{ये तु धर्मामृतमिदं यथोक्तं पर्युपासते}
{श्रद्दधाना मत्परमा भक्तास्तेऽतीव मे प्रियाः}


\chapter{अध्यायः ३७}
\twolineshloka
{अर्जुन उवाच}
{}


\threelineshloka
{प्रकृतिं पुरुषं चैव क्षेत्रं क्षेत्रज्ञमेव च}
{एतद्वेदितुमिच्छामि ज्ञानं ज्ञेयं च केशव ॥श्रीभगवानुवाच}
{}


\twolineshloka
{इदं शरीरं कौन्तेय क्षेत्रमित्यभिधीयते}
{एतद्यो वेत्ति तं प्राहुः क्षेत्रज्ञ इति तद्विदः}


\twolineshloka
{क्षेत्रज्ञं चापि मां विद्धि सर्वक्षेत्रेषु भारत}
{क्षेत्रक्षेत्रज्ञयोर्ज्ञानं यत्तज्ज्ञानं मतं मम}


\twolineshloka
{तत्क्षेत्रं यच्च यादृक् यद्विकारि यतश्च यत्}
{स च यो यत्प्रभावश्च तत्समासेन मे श्रृणु}


\twolineshloka
{ऋषिभिर्बहुधा गीतं छन्दोभिर्विविधैः पृथक्}
{ब्रह्मसूत्रपदैश्चैव हेतुमद्भिर्विनिश्चितैः}


\twolineshloka
{महाभूतान्यहङ्कारो बुद्धिरव्यक्तमेव च}
{इन्द्रियाणि दशैकं च पञ्च चेन्द्रियगोचराः}


\twolineshloka
{इच्छा द्वेषः सुखं दुःखं सङ्घातश्चेतना धृतिः}
{एतत्क्षेत्रं समासेन सविकारमुदाहृतम्}


\twolineshloka
{अमानित्वमदम्भित्वमहिंसा क्षान्तिरार्जवम्}
{आचार्योपासनं शौचं स्थैर्यमात्मविनिग्रहः}


\twolineshloka
{इन्द्रियार्थेषु वैराग्यमनहङ्कार एव च}
{जन्ममृत्युजराव्याधिदुःखदोषानुदर्शनम्}


\twolineshloka
{असक्तिरनभिष्वङ्गः पुत्रदारगृहादिषु}
{नित्यं च समचित्तत्वमिष्टानिष्टोपपत्तिषु}


\twolineshloka
{मयि चानन्ययोगेन भक्तिरव्यभिचारिणी}
{विविक्तदेशसेवित्वमरतिर्जनसंसदि}


\twolineshloka
{अध्यात्मज्ञाननित्यत्वं तत्त्वज्ञानार्थदर्शनम्}
{एतज्ज्ञानमिति प्रोक्तमज्ञानं यदतोऽन्यथा}


\twolineshloka
{ज्ञेयं यत्तत्प्रवक्ष्यामि यज्ज्ञात्वाऽमृतमश्रुते}
{अनादिमत्परं ब्रह्म न सत्तन्नासदुच्यते}


\twolineshloka
{सर्वतः पाणिपादं तत्सर्वतोऽक्षिशिरोमुखम्}
{सर्वतः श्रुतिमल्लोके सर्वमावृत्य तिष्ठति}


\twolineshloka
{सर्वेन्द्रियगुणाभासं सर्वेन्द्रियविवर्जितम्}
{असक्तं सर्वभृच्चैव निर्गुणं गुणभोक्तृ च}


\twolineshloka
{बहिरन्तश्च भूतानामचरं चरमेव च}
{सूक्ष्मत्वात्तदविज्ञेयं दूरस्थं चान्तिके च तत्}


\twolineshloka
{अविभक्तं च भूतेषु विभक्तमिव च स्थितम्}
{भूतभर्तृ च तज्ज्ञेयं ग्रसिष्णु प्रभविष्णु च}


\twolineshloka
{ज्योतिषामपि तज्ज्योतिस्तमसः परमुच्यते}
{ज्ञानं ज्ञेयं ज्ञानगम्यं हृदि सर्वस्य विष्ठितम्}


\twolineshloka
{इति क्षेत्रं तथा ज्ञानं ज्ञेयं चोक्तं समासतः}
{मद्भक्त एतद्विज्ञाय मद्भावायोपपद्यते}


\twolineshloka
{प्रकृतिं पुरुषं चैव विद्ध्यनादी उभावपि}
{विकारांश्च गुणांश्चैव विद्धि प्रकृतिसंभवान्}


\twolineshloka
{कार्यकारणकर्तृत्वे हेतुः प्रकृतिरुच्यते}
{पुरुषः सुखदुःखानां भोक्तृत्वे हेतुरुच्यते}


\twolineshloka
{पुरुषः प्रकृतिस्थो हि भुङ्क्ते प्रकृतिजान्गुणान्}
{कारणं गुणसङ्गोऽस्य सदसद्योनिजन्मसु}


\twolineshloka
{उपद्रष्टानुमन्ता च भर्ता भोक्ता महेश्वरः}
{परमात्मेति चाप्युक्तो देहेऽस्मिन्पुरुषः परः}


\twolineshloka
{य एवं वेत्ति पुरुषं प्रकृतिं च गुणैः सह}
{सर्वथा वर्तमानोऽपि न स भूयोऽभिजायते}


\twolineshloka
{ध्यानेनात्मनि पश्यन्ति केचिदात्मानमात्मना}
{अन्ये साङ्ख्येन योगेन कर्मयोगेन चापरे}


\threelineshloka
{अन्ये त्वेवमजानन्तः श्रुत्वाऽन्येभ्य उपासते}
{तेऽपि चातितरन्त्येव मृत्युं श्रुतिपरायणाः ॥ 6-37-27aयावत्संजायते किंचित्सत्त्वं स्थावरजङ्गमम्}
{क्षेत्रक्षेत्रज्ञसंयोगात्तद्विद्धि भरतर्षभ}


\twolineshloka
{समं सर्वेषु भूतेषु तिष्ठन्तं परमेश्वरम्}
{विनश्यत्स्वविनश्यन्तं यः पश्यति स पश्यति}


\twolineshloka
{समं पश्यन्हि सर्वत्र समवस्थितमीश्वरम्}
{न हिनस्त्यात्मनात्मानं ततो याति परां गतिं}


\twolineshloka
{प्रत्यत्यैव च कर्माणि क्रियमाणानि सर्वशः}
{यः पश्यति तथात्मानमकर्तारं स पश्यति}


\twolineshloka
{यदा भूतपृथग्भावमेकस्थमनुपश्यति}
{तत एव च विस्तारं ब्रह्म संपद्यते तदा}


\twolineshloka
{अनादित्वान्निर्गुणत्वात्परमात्मायमव्ययः}
{शरीरस्थोऽपि कौन्तेय न करोति न लिप्यते}


\twolineshloka
{यथा सर्वगतं सौक्ष्म्यादाकाशं नोपलिप्यते}
{सर्वत्रावस्थितो देहे तथात्मा नोपलिप्यते}


\twolineshloka
{यथा प्रकाशयत्येकः कृत्स्नं लोकमिमं रविः}
{क्षेत्रं क्षेत्री तथा कृत्स्नं प्रकाशयति भारत}


\twolineshloka
{क्षेत्रक्षेत्रज्ञयोरेवमन्तरं ज्ञानचक्षुषा}
{भूतप्रकृतिमोक्षं च ये विदुर्यान्ति ते परम्}


\chapter{अध्यायः ३८}
\twolineshloka
{श्रीभगवानुवाच}
{}


\twolineshloka
{परं भूयः प्रवक्ष्यामि ज्ञानानां ज्ञानमुत्तमम्}
{यज्ज्ञात्वा मुनयः सर्वे परां सिद्धिमितो गताः}


\twolineshloka
{इदं ज्ञानमुपाश्रित्य मम साधर्म्यमागताः}
{सर्गेऽपि नोपजायन्ते प्रलये न व्यथन्ति च}


\twolineshloka
{मम योनिर्महद्ब्रह्म तस्मिन्गर्भं दधाम्यहम्}
{संभवः सर्वभूतानां ततो भवति भारत}


\twolineshloka
{सर्वयोनिषु कौन्तेय मूर्तयः संभवन्तिः याः}
{तासां ब्रह्म महद्योनिरहं बीजप्रदः पिता}


\twolineshloka
{सत्वं रजस्तम इति गुणाः प्रकृतिसंभवाः}
{निबध्नन्ति महाबाहो देहे देहिनमव्ययम्}


\twolineshloka
{तत्र सत्त्वं निर्मलत्वात्प्रकाशकमनामयम्}
{सुखसङ्गेन बध्नाति ज्ञानसङ्गेन चानघ}


\twolineshloka
{रजो रागात्मकं विद्धि तृष्णासङ्गसमुद्भवम्}
{तन्निबध्नाति कौन्तेय कर्मसङ्गेन देहिनम्}


\twolineshloka
{तमस्त्वज्ञानजं विद्धि मोहनं सर्वदेहिनाम्}
{प्रमादालस्यनिद्राभिस्तन्निबध्नाति भारत}


\twolineshloka
{सत्वं सुखे संजयति रजः कर्मणि भारत}
{ज्ञानमावृत्य तु तमः प्रमादे संजयत्युत}


\twolineshloka
{रजस्तमश्चाभिभूय सत्वं भवति भारत}
{रजः सत्वं तमश्चैव तमः सत्वं रजस्तथा}


\twolineshloka
{सर्वद्वारेषु देहेऽस्मिन्प्राकाश उपजायते}
{ज्ञानं यदा तदा विद्याद्विवृद्धं सत्वमित्युत}


\twolineshloka
{लोभः प्रवृत्तिरारम्भः कर्मणामशमः स्पृहा}
{रजस्येतानि जायन्ते विवृद्धे भरतर्षभ}


\twolineshloka
{अप्रकाशोऽप्रवृत्तिश्च प्रमादो मोह एव च}
{तमस्येतानि जायन्ते विवृद्धे कुरुनन्दन}


\twolineshloka
{यदा सत्वे प्रवृद्धे तु प्रलयं याति देहभृत्}
{तदोत्तमविदां लोकानमलान्प्रतिपद्यते}


\twolineshloka
{रजसि प्रलयं गत्वा कर्मसङ्गिषु जायते}
{तथा प्रलीनस्तमसि मूढयोनिषु जायते}


\twolineshloka
{कर्मणः सुकृतस्याहुः सात्विकं निर्मलं फलम्}
{रजसस्तु फलं दुःखमज्ञानं तमसः फलम्}


\twolineshloka
{सत्वात्संजायते ज्ञानं रजसो लोभ एव च}
{प्रमादमोहौ तमसो भवतोऽज्ञानमेव च}


\twolineshloka
{ऊर्ध्वं गच्छन्ति सत्वस्था मध्ये तिष्ठन्ति राजसाः}
{जघन्यगुणवृत्तस्था अघो गच्छन्ति तामसाः}


\twolineshloka
{नान्यं गुणेभ्यः कर्तारं यदा द्रष्टाऽनुपश्यति}
{गुणेभ्यश्च परं वेत्ति मद्भावं सोऽधिगच्छति}


\threelineshloka
{गुणानेतानतीत्य त्रीन्देही देहसमुद्भवान्}
{जन्ममृत्युजरादुःखैर्विमुक्तोऽमृतमश्रुते ॥अर्जुन उवाच}
{}


\threelineshloka
{कैर्लिङ्गैस्त्रीन्गुणानेतानीतो भवति प्रभो}
{किमाचारः कथं चैतांस्त्रीन्गुणानतिवर्तते ॥श्रीभगवानुवाच}
{}


\twolineshloka
{प्रकाशं च प्रवृत्तिं च मोहमेव च पाण्डव}
{न द्वेष्टि संप्रवृत्तानि न निवृत्तानि काङ्क्षति}


\twolineshloka
{उदासीनवदासीनो गुणैर्यो न विचाल्यते}
{गुणा वर्तन्त इत्येव योऽवतिष्ठति नेङ्गते}


\twolineshloka
{समदुःखसुखः स्वस्थः समलोष्टाश्मकाञ्चनः}
{तुल्यप्रियाप्रियो धीरस्तुल्यनिन्दात्मसंस्तुतिः}


\twolineshloka
{मानापमानयोस्तुल्यस्तुल्यो मित्रारिपक्षयोः}
{सर्वारम्भपरित्यागी गुणातीतः स उच्यते}


\twolineshloka
{मां च योऽव्यभिचारेण भक्तियोगेन सेवते}
{स गुणान्समतीत्यैतान्ब्रह्म भूयाय कल्पते}


\twolineshloka
{ब्रह्मणो हि प्रतिष्ठाहममृतस्याव्ययस्य च}
{शाश्वतस्य च धर्मस्य सुखस्यैकान्तिकस्य च}


\chapter{अध्यायः ३९}
\twolineshloka
{श्रीभगवानुवाच}
{}


\twolineshloka
{ऊर्ध्वमूलमधःशाखमश्वत्थं प्राहुरव्ययम्}
{छन्दांसि यस्य पर्णानि यस्तं वेद स वेदवित्}


\twolineshloka
{अधश्चोर्ध्वं प्रसृतास्तस्य शाखागुणप्रवृद्धा विषयप्रवालाः}
{अधश्च मूलान्यनुसंततानिकर्मानुबन्धीनि मनुष्यलोके}


\twolineshloka
{न रूपमस्येह तथोपलभ्यतेनान्तो न चादिर्न च संप्रतिष्ठा}
{अश्वत्थमेनं सुविरूढमूल-मसङ्गशस्त्रेण दृढेन छित्वा}


\twolineshloka
{ततः पदं तत्परिमार्गितव्यंयस्मिन्गता न निवर्तन्ति भूयः}
{तमेव चाद्यं पुरुषं प्रपद्येयतः प्रवृत्तिः प्रसृता पुराणी}


\twolineshloka
{निर्मानमोहा जितसङ्गदोषाअध्यात्मनित्या विनिवृत्तकामाः}
{द्वद्वैर्विमुक्ताः सुखदुःखसंज्ञै-र्गच्छन्त्यमूढाः पदमव्ययं तत्}


\twolineshloka
{न तद्भासयते सूर्यो न शशाङ्को न पावकः}
{यद्गत्वा न निवर्तन्ते तद्धाम परमं मम}


\twolineshloka
{ममैवांशो जीवलोके जीवभूतः सनातनः}
{मनःषष्ठानीन्द्रियाणि प्रकृतिस्थानि कर्षति}


\twolineshloka
{शरीरं यदवाप्नोति यच्चाप्युत्क्रामतीश्वरः}
{गृहीत्वैतानि संयाति वायुर्गन्धानिवाशयात्}


\twolineshloka
{श्रोत्रं चक्षुः स्पर्शनं च रसनं घ्राणमेव च}
{अधिष्ठाय मनश्चायं विषयानुपसेवते}


\twolineshloka
{उत्क्रामन्तं स्थितं वाऽपि भुञ्जानं वा गुणान्वितम्}
{विमूढा नानुपश्यन्ति पश्यन्ति ज्ञानचक्षुषः}


\twolineshloka
{यतन्तो योगिनश्चैनं पश्यन्त्यात्मन्यवस्थितम्}
{यतन्तोऽप्यकृतात्मानो नैनं पश्यन्त्यचेतसः}


\twolineshloka
{यदादित्यगतं तेजो जगद्भासयतेऽखिलम्}
{यच्चन्द्रमसि यच्चाग्नौ तत्तेजो विद्धि मार्मकम्}


\twolineshloka
{गामाविश्य च भूतानि धारयम्यहमोजसा}
{पुष्णामि चौषधीः सर्वाः सोमो भूत्वा रसात्मकः}


\twolineshloka
{अहं वैश्वानरो भूत्वा प्राणिनां देहमाश्रितः}
{प्राणापानसमायुक्तः पचाम्यन्नं चतुर्विधम्}


\twolineshloka
{सर्वस्य चाहं हृदि सन्निविष्टोमत्तः स्मृतिर्ज्ञानमपोहनं च}
{वेदैश्च सर्वैरहमेव वेद्योवेदान्तकृद्वेदविदेव चाहम्}


\twolineshloka
{द्वाविमौ पुरुषौ लोके क्षरश्चाक्षर एव च}
{क्षरः सर्वाणि भूतानि कूटस्थोऽक्षर उच्यते}


\twolineshloka
{उत्तमः पुरुषस्त्वन्यः परमात्मेत्युदाहृतः}
{यो लोकत्रयमाविश्य बिभर्त्यव्यय ईश्वरः}


\twolineshloka
{यस्मात्क्षरमतीतोऽहमक्षरादपि चोत्तमः}
{अतोऽस्मि लोके वेदे च प्रस्थितः पुरुषोत्तमः}


\twolineshloka
{यो मामेवमसंमूढो जानाति पुरुषोत्तमम्}
{स सर्वविद्भजति मां सर्वभावेन भारत}


\twolineshloka
{इति गुह्यतमं शास्त्रमिदमुक्तं मयाऽनघ}
{एवद्बुद्ध्वा बुद्धिमान्स्यात्कृतकृत्यश्च भारत}


\chapter{अध्यायः ४०}
\twolineshloka
{श्रीभगवानुवाच}
{}


\twolineshloka
{अभयं सत्वसंशुद्धिर्ज्ञानयोगव्यवस्थितिः}
{दानं दमश्च यज्ञश्च स्वाध्यायस्तप आर्जवम्}


\twolineshloka
{अहिंसा सत्यमक्रोधस्त्यागः शान्तिरपैशुनम्}
{दया भूतेष्वलोलुत्वं मार्दवं ह्रीरचापलम्}


\twolineshloka
{तेजः क्षमा धृतिः शौचमद्रोहो नातिमानिता}
{भवन्ति संपदं दैवीमभिजातस्य भारत}


\twolineshloka
{दम्भो दर्पोऽभिमानश्च क्रोधः पारुष्यमेव च}
{अज्ञानं चाभिजातस्य पार्थ संपदमासुरीम्}


\twolineshloka
{दैवी संपद्विमोक्षाय निबन्धायासुरी मता}
{मा शुचः संपदं दैवीमभिजातोऽसि पाण्डव}


\twolineshloka
{द्वौ भूतसर्गौ लोकेऽस्मिन्दैव आसुर एव च}
{दैवो विस्तरशः प्रोक्त आसुरं पार्थ मे शृणु}


\twolineshloka
{प्रवृत्तिं च निवृत्तिं च जना न विदुरासुराः}
{न शौचं नापि चाचारो न सत्यं तेषु विद्यते}


\twolineshloka
{असत्यमप्रतिष्ठं ते जगदाहुरनीश्वरम्}
{अपरस्परसंभूतं किमन्यत्कामहैतुकम्}


\twolineshloka
{एतां दृष्टिमवष्टभ्य नष्टात्मानोऽल्पबुद्धयः}
{प्रभवन्त्युग्रकर्माणः क्षयाय जगतोऽहिताः}


\twolineshloka
{काममाश्रित्य दुष्पूरं दम्भमानमदान्विताः}
{मोहाद्गृहीत्वाऽसद्ग्राहान्प्रवर्तन्तेऽशुचिव्रताः}


\twolineshloka
{चिन्तामपरिमेयां च प्रलयान्तामुपाश्रिताः}
{कामोपभोगपरमा एतावदिति निश्चिताः}


\twolineshloka
{आशापाशशतैर्बद्धाः कामक्रोधपरायणाः}
{ईहन्तो कामभोगार्थमन्यायेनार्थसंचयान्}


\twolineshloka
{इदमद्य मया लब्धमिमं प्राप्स्ये मनोरथम्}
{इदमस्तीदमपि मे भविष्यति पुनर्धनम्}


\twolineshloka
{असौ मया हतः शत्रुर्हनिष्ये चापरानपि}
{ईश्वरोऽहमहं भोगी सिद्धोऽहं बलवान्सुखी}


\twolineshloka
{आढ्योऽभिजनवानस्मि कोऽन्योस्ति सदृशो मया}
{यक्ष्ये दास्यामि मोदिष्य इत्यज्ञानविमोहिताः}


\twolineshloka
{अनेकचित्तविभ्रान्त मोहजालसमावृताः}
{प्रसक्ताः कामभोगेषु पतन्ति नरकेऽशुचौ}


\twolineshloka
{आत्मसंभाविताः स्तब्धा धनमानमदान्विताः}
{यजन्ते नामयज्ञैस्ते दम्भेनाविधिपूर्वकम्}


\twolineshloka
{अहङ्कारं बलं दर्पं कामं क्रोधं च संश्रिताः}
{मामत्मपरदेहेषु प्रद्विषन्तोऽभ्यसूयकाः}


\twolineshloka
{तानहं द्विषतः क्रूरान्संसारेषु नराधमान्}
{क्षिपाम्यजस्रमशुभानासुरीष्वेव योनिषु}


\twolineshloka
{आसुरीं योनिमापन्ना मूढा जन्मनिजन्मनि}
{मामप्राप्यैव कौन्तेय ततो यान्त्यधमां गतिम्}


\twolineshloka
{त्रिविधं नरकस्येदं द्वारं नाशनमात्मनः}
{कामः क्रोधस्तथा लोभस्तस्मादेतत्रयं त्यजेत्}


\twolineshloka
{एतैर्विमुक्तः कौन्तेय तमोद्वारैस्त्रिभिर्नरः}
{आचरत्यात्मनः श्रेयस्ततो याति परां गतिम्}


\twolineshloka
{यः शास्त्रविधिमुत्सृज्य वर्तते कामकारतः}
{न स सिद्धिमवाप्नोति न सुखं न परां गतिम्}


\twolineshloka
{तस्माच्छास्त्रं प्रमाणं ते कार्याकार्यव्यवस्थितौ}
{ज्ञात्वा शास्त्रविधानोक्तं कर्म कर्तुमिहार्हसि}


\chapter{अध्यायः ४१}
\twolineshloka
{अर्जुन उवाच}
{}


\threelineshloka
{ये शास्त्रविधिमुत्सृज्य यजन्ते श्रद्धयान्विता}
{तेषां निष्ठा तु का कृष्ण सत्वमाहो सजस्तमः ॥श्रीभगवानुवाच}
{}


\twolineshloka
{त्रिविधा भवति श्रद्धा देहिनां सा स्वभावजा}
{सात्विकी राजसी चैव तामसी चेति तां श्रृणु}


\twolineshloka
{सत्वानुरूपा सर्वस्य श्रद्धा भवति भारत}
{श्रद्धामयोऽयं पुरुषो यो यच्छ्रद्धः स एव सः}


\twolineshloka
{यजन्ते सात्विका देवान्यक्षरक्षांसि राजसाः}
{प्रेतान्भूतगणांश्चान्ये यजन्ते तामसा जनाः}


\twolineshloka
{अशास्त्रविहितं घोरं तप्यन्ते ये तपो जनाः}
{दम्भाहङ्कारसंयुक्ताः कामरागबलान्विताः}


\twolineshloka
{कर्शयन्तः शरीरस्थं भूतग्राममचेतसः}
{मां चैवान्तः शरीरस्थं तान्विद्ध्यासुरनिश्चयान्}


\twolineshloka
{आहारस्त्वपि सर्वस्य त्रिविधो भवति प्रियः}
{यज्ञस्तपस्तथा दानं तेषां भेदमिमं श्रृणु}


\twolineshloka
{आयुःसत्वबलारोग्यसुखप्रीतिविवर्धनाः}
{रस्याःस्रिग्धाः स्थिरा हृद्या आहाराःसात्विकप्रियाः}


\twolineshloka
{कट्वम्ललवणात्युष्णतीक्ष्णरूक्षविदाहिनः}
{आहारा राजसस्येष्टा दुःखशोकामयप्रदाः}


\twolineshloka
{यातयामं गतरसं पूति पर्युषितं च यत्}
{उच्छिष्टमपि चामेध्यं भोजनं तामसप्रियम्}


\twolineshloka
{अफलाकाङ्क्षिभिर्यज्ञो विधिदृष्टो य इज्यते}
{यष्टव्यमेवेति मनः समाधाय स सात्विकः}


\twolineshloka
{अभिसंघाय तु फलं दम्भार्थमपि चैव यत्}
{इज्यते भरतश्रेष्ठ तं यज्ञं विद्धि राजसम्}


\twolineshloka
{विधिहीनमसृष्टान्नं मन्त्रहीनमदक्षिणम्}
{श्रद्धाविरहितं यज्ञं तामसं परिचक्षते}


\twolineshloka
{देवद्विजगुरुप्राज्ञपूजनं शौचमार्जवम्}
{ब्रह्मचर्यमहिंसा च शारीरं तप उच्यते}


\twolineshloka
{अनुद्वेगकरं वाक्यं सत्यं प्रियहितं च यत्}
{स्वाध्यायाभ्यसनं चैव वाङ्भयं तप उच्यते}


\twolineshloka
{मनःप्रसादः सम्यत्वं मौनमात्मविनिग्रहः}
{भावसंशुद्धिरित्येतत्तपो मानसमुच्यते}


\twolineshloka
{श्रद्धया परया तप्तं तपस्तत्रिविधं नरैः}
{अफलाकाङ्क्षिभिर्युक्तैः सात्विकं परिचक्षते}


\twolineshloka
{सत्कारमानपूजाराथं तपो दम्भेन चैव यत्}
{क्रियते तदिह प्रोक्तं राजसं चलमध्रुवम्}


\twolineshloka
{मूढग्राहेणात्मनो यत्पीडया क्रियते तपः}
{परस्योत्सादनार्थं वा तत्तामसमुदाहृतम्}


\twolineshloka
{दातव्यमिति यद्दानं दीयतेऽनुपकारिणे}
{देशे काले च पात्रे च तद्दानं सात्विकं स्मृतम्}


\twolineshloka
{यत्तु प्रत्युपकारार्थं फलमुद्दिश्य वा पुनः}
{दीयते च परिक्लिष्टं तद्दानं राजसं स्मृतम्}


\twolineshloka
{अदेशकाले यद्दानमपात्रेभ्यश्च दीयते}
{असत्कृतमवज्ञातं तत्तामसमुदाहृतम्}


\twolineshloka
{ओंतत्सदिति निर्देशो ब्रह्मणस्त्रिविधः स्मृताः}
{ब्राह्मणास्तेन वेदाश्च यज्ञाश्च विहिताः पुरा}


\twolineshloka
{तस्मादोमित्युदाहृत्य यज्ञदानतपःक्रियाः}
{प्रवर्तन्ते विधानोक्ताः सततं ब्रह्मवादिनाम्}


\twolineshloka
{तदित्यनभिसंधाय फलं यज्ञतपःक्रियाः}
{दानक्रियाश्च विविधाः क्रियन्ते मोक्षकाङ्क्षिभिः}


\twolineshloka
{सद्भावे साधुभावे च सदित्येतत्प्रुयुज्यते}
{प्रशस्ते कर्मणि तथा सच्छब्दः पार्थ युज्यते}


\twolineshloka
{यज्ञे तपसि दाने च स्थितिः सदिति चोच्यते}
{कर्म चैव तदर्थीयं सादित्येवाभिधीयते}


\twolineshloka
{अश्रद्धया हुतं दत्तं तपस्तप्तं कृतं च यत्}
{असदित्युच्यते पार्थ न च तत्प्रेत्य नो इह}


\chapter{अध्यायः ४२}
\twolineshloka
{अर्जुन उवाच}
{}


\threelineshloka
{संन्यासत्य महाबाहो तत्त्वमिच्छामि वेदितुम्}
{त्यागस्य च हृषीकेश पृथक्केशिनिपूदन ॥श्रीभगवानुवाच}
{}


\twolineshloka
{काम्यानां कर्मणां न्यासं संन्यासं कवयो विदुः}
{सर्वकर्मफलत्यागं प्राहुस्त्यागं विचक्षणाः}


\twolineshloka
{त्याज्यं दोषवदित्येके कर्म प्राहुर्मनीषिणः}
{यज्ञदानतपःकर्म न त्याज्यमिति चापरे}


\twolineshloka
{निश्चयं श्रृणु मे तत्र त्यागे भरतसत्तम}
{त्यागो हि पुरुषव्याघ्र त्रिविधः संप्रकीर्तितः}


\twolineshloka
{यज्ञदानतपःकर्म न त्याज्यं कार्यमेव तत्}
{यज्ञो दानं तपश्चैव पावनानि मनीषिणाम्}


\twolineshloka
{एतान्यपि तु कर्माणि सङ्गं त्यक्त्वा फलानि च}
{कर्तव्यानीति मे पार्थ निश्चितं मतमुत्तमम्}


\twolineshloka
{नियतस्य तु संन्यासः कर्मणो नोपपद्यते}
{मोहात्तस्य परित्यागस्तामसः परिकीर्तितः}


\twolineshloka
{दुःखमित्येव यत्कर्म कायक्लेशभयात्त्यजेत्}
{स कृत्वा राजसं त्यागं नैव त्यागफलं लभेत्}


\twolineshloka
{कार्यमित्येव यत्कर्म नियतं क्रियतेऽर्जुन}
{सङ्गं त्यक्त्वा फलं चैव स त्यागः सात्विको मतः}


\twolineshloka
{न द्वेष्ट्यकुशलं कर्म कुशले नानुषज्जते}
{त्यागी सत्त्वसमाविष्टो मेधावी छिन्नसंशयः}


\twolineshloka
{न हि देहभृता शक्यं त्यक्तुं कर्माण्यशेषतः}
{यस्तु कर्मफलत्यागी स त्यागीत्यभिधीयते}


\twolineshloka
{अनिष्टमिष्टं मिश्रं च त्रिविधं कर्मणः फलम्}
{भवत्यत्यागिनीं प्रेत्य न तु संन्यासिनां क्वचित्}


\twolineshloka
{पञ्चैतानि महाबाहो कारणानि निबोध मे}
{साङ्ख्ये कृतान्ते प्रोक्तानि सिद्धये सर्वकर्मणां}


\twolineshloka
{अधिष्ठानं तथा कर्ता करणं च पृथग्विधम्}
{विविधाश्च पृथक्केष्टा दैवं चैवात्र पञ्चमम्}


\twolineshloka
{शरीरवाङ्भनोभिर्यत्कर्म प्रारभते नरः}
{न्याय्यं वा विपरीतं वा पञ्चैते तस्य हेतवः}


\twolineshloka
{तत्रैवं सति कर्तारमात्मानं केवलं तु यः}
{पश्यत्यकृतबुद्धित्वान्न स पश्यति दुर्मतिः}


\twolineshloka
{यस्य नाहंकृतो भवो बुद्धिर्यस्य न लिप्यते}
{हत्वाऽपि स इमाँल्लोकान्न हन्ति न निबध्यते}


\twolineshloka
{ज्ञानं ज्ञेयं परिज्ञाता त्रिविधा कर्मचोदना}
{करणं कर्म कर्तेति त्रिविधः कर्मसंग्रहः}


\twolineshloka
{ज्ञानं कर्म च कर्ता च त्रिधैव गुणभेदतः}
{प्रोच्यते गुणसंख्याने यथावच्छृणु तान्यपि}


\twolineshloka
{सर्वभूतेषु येनैकं भावमव्ययमीक्षते}
{अविभक्तं विभक्तेषु तज्ज्ञानं विद्धि सात्विकं}


\twolineshloka
{पृथक्त्वेन तु यज्ज्ञानं नानाभावान्पृथग्विधान्}
{वेत्ति सर्वेषु भूतेषु तज्ज्ञानं विद्धि राजसम्}


\twolineshloka
{यत्तु कृत्स्नवदेकस्मिन्कार्ये सक्तमहैतुकम्}
{अतत्त्वार्थवदल्पं च तत्तासममुदाहृतम्}


\twolineshloka
{नियतं सङ्गरहितमरागद्वेषतः कृतम्}
{अफलप्रेप्सुना कर्म यत्तत्सात्विकमुच्यते}


\twolineshloka
{यत्तु कामेप्सुना कर्म ताहङ्कारेण वा पुनः}
{क्रियते बहुलायासं तद्राजसमुदाहृतम्}


\twolineshloka
{अनुबन्धं क्षयं हिंसामनपेक्ष्य च पौरुषम्}
{मोहादारभ्यते कर्म यत्तत्तामसमुच्यते}


\twolineshloka
{मुक्तसङ्गोनहंवादी धृत्युत्साहसमन्वितः}
{6-42-26bसिद्ध्यसिद्ध्योर्निर्विकारः कर्ता सात्विक उच्यते}


\twolineshloka
{रागी कर्मफलप्रेप्सुर्लुब्धो हिंसात्मकोऽशुचिः}
{हर्षशोकान्वितः कर्ता राजसः परिकीर्तितः}


\twolineshloka
{अयुक्तः प्राकृतः स्तब्धः शठो नैकृतिकोऽलसः}
{विषादी दीर्घसूत्री च कर्ता तामस उच्यते}


\twolineshloka
{बुद्धेर्भेदं धृतेश्चैव गुणतस्त्रिविधं श्रृणु}
{प्रोच्यमानमशेषेण पृथक्त्वेन धनंजय}


\twolineshloka
{प्रवृत्तिं च निवृत्तिं च कार्याकार्ये भयाभये}
{बन्धं मोक्षं च या वेत्ति बुद्धिः सा पार्थ सात्विकी}


\twolineshloka
{यया धर्ममधर्मं च कार्यं चाकार्यमेव च}
{अयथावत्प्रजानाति बुद्धिः सा पार्थ राजसी}


\twolineshloka
{अधर्मं धर्ममिति या मन्यते तमसाऽऽवृता}
{सर्वार्थान्विपरीतांश्च बुद्धिः सा पार्थ तामसी}


\twolineshloka
{धृत्या यया धारयते मनःप्रणेन्द्रियक्रियाः}
{योगेनाव्यभिचारिण्या धृतिः सा पार्थ सात्विकी}


\twolineshloka
{यया तु धर्मकामार्धान्धृत्या धारयतेऽर्जुन}
{प्रसङ्गेन फलाकाङ्क्षी धृतिः सा पार्थ राजसी}


\twolineshloka
{यथा स्वप्नं भयं शोकं विषादं भदेमेव च}
{न विमुञ्चति दुर्मेधा धृतिः सा पार्थ तामसी}


\twolineshloka
{सुखं त्विदानीं त्रिविधं श्रुणु मे भारतर्षभ}
{अभ्यासाद्रमते यत्र दुःखान्तं च निगच्छति}


\twolineshloka
{यत्तदग्रे विषमिव परिणामेऽमृतोपमम्}
{तत्सुखं सात्विकं प्रोक्तमात्मबुद्धिप्रसादजम्}


\twolineshloka
{विषयेन्द्रियसंयोगाद्यत्तदग्रेऽमृतोपभम्}
{परिणामे विषमिव तत्सुखं राजसं स्मृतम्}


\twolineshloka
{यदग्रे चानुबन्धे च सुखं मोहनमात्मनः}
{निद्रालस्यप्रमादोत्थं तत्ताभसमुदाहृतम्}


\twolineshloka
{न तदस्ति पृथिव्यां वा दिवि देवेषु वा पुनः}
{सत्त्वं प्रकृतिजैर्मुक्तं यदेमिः स्यात्रिभिर्गुणैः}


\twolineshloka
{ब्राह्मणक्षत्रियविशां शूद्राणां च परंतप}
{कर्माणि प्रविभक्तानि स्वभावप्रभवैर्गुणैः}


\twolineshloka
{शमो दमस्तपः शौचं क्षान्तिरार्जवमेव च}
{ज्ञानं विज्ञानमास्तिक्यं ब्रह्मकर्म स्वभावजम्}


\twolineshloka
{शौर्यं तेजो धृतिर्दाक्ष्यं युद्धे चाप्यपलायनम्}
{दानमीश्वरभावश्च क्षात्रं कर्म स्वभावजम्}


\twolineshloka
{कृषिगोरक्ष्यरभावश्च क्षात्रं कर्म स्वभावजम्}
{परिचर्यात्मकं कर्म शूद्रस्यापि स्वभावजम्}


\twolineshloka
{स्वे स्वे कर्मण्यभिरतः संसिद्धिं लभते नरः}
{स्वकर्मनिरत्तः सिद्धिं यथा विन्दति तच्छृणु}


\twolineshloka
{यतः प्रवृत्तिर्भूतानां योन सर्वमिदं ततम्}
{स्वकर्मणा तमभ्यर्च्य सिद्धिं विन्दति मानवःक}


\twolineshloka
{श्रेयान्स्वधर्मो विगुणः परधर्मात्स्वनुष्ठितात्}
{स्वभावनियतं कर्म कुर्वन्नाप्नोति किल्विषम्}


\twolineshloka
{सहजं कर्म कौन्तेय सदोषमपि न त्यजेत्}
{सर्वारम्भा हि दोषेण धूमेनाग्निरिवावृताः}


\twolineshloka
{असक्तबुद्धिः सर्वत्र जितात्मा विगतस्पृहः}
{नैष्कर्म्यसिद्धिं परमां संन्यासेनाधिगच्छति}


\twolineshloka
{सिद्धिं प्राप्तो यथा ब्रह्म तथाऽऽप्नेति निबोध मे}
{समासेनैव कौन्तेय निष्ठा ज्ञानस्य या परा}


\twolineshloka
{बुद्ध्या विशुद्धया युक्तो धृत्याऽऽत्मानं नियम्य च}
{शब्दादीन्विषयांस्त्यक्त्वा रागद्वेषौ व्युदस्य च}


\twolineshloka
{विविक्तसेवी लध्वाशी यतकाक्कायमानसः}
{ध्यानयोगपरो नित्यं वैराग्यं समुपाश्रितः}


\twolineshloka
{अहङ्कारं बलं दर्पं कामं क्रोधं परिग्रहम्}
{विमुच्य निर्ममः शान्तो ब्रह्मभूयाय कल्पते}


\twolineshloka
{ब्रह्मभूतः प्रसन्नात्मा न शोचतिक न काङ्क्षति}
{समः सर्वेषु भूतेषु मद्तभक्तिं लभते पराम्}


\twolineshloka
{भक्त्या मामभिजानाति यावान्यश्चास्मि तत्त्वतः}
{ततो मां तत्त्वतो ज्ञात्वा विशते तदनन्तरम्}


\twolineshloka
{सर्वकर्माण्यपि सदा कुर्वाणो मद्व्यपाश्रयः}
{मत्प्रसादादवाप्नोति शाश्वतं पदमव्ययम्}


\twolineshloka
{चेतसा सर्वकर्माणि मयि संन्यस्य मत्परः}
{बुद्धियोगमपाश्रित्य मच्चित्तः सततं भव}


\twolineshloka
{मच्चित्तः सर्वदुर्गाणि मत्प्रसादात्तरिष्यसि}
{अथ चेत्त्वमहङ्कारान्न श्रोष्यसि विनङ्क्ष्यसि}


\twolineshloka
{यदहङ्कारमाश्रित्य न योत्स्य इति मन्यसे}
{मिथ्यैष व्यवसायस्ते प्रकृतिस्त्वां नियोक्ष्यति}


\twolineshloka
{स्वभावजेन कौन्तेय निबद्धः स्वेन कर्मणा}
{कर्तुं नेच्छसि यन्मोहात्करिष्यस्यवशोपि तत्}


\twolineshloka
{ईश्वरः सर्वभूतानां हृद्देशेऽर्जुन तिष्ठति}
{भ्रामयन्सर्वभूतानि यन्त्रारूढानि मायया}


\twolineshloka
{तमेव शरणं गच्छ सर्वभावेन भारत}
{तत्प्रसादात्परां शान्तिं स्थानं प्राप्स्यसि शाश्वतं}


\twolineshloka
{इति ते ज्ञानमाख्यातं गुह्याद्गुह्यतरं मया}
{विमृश्यैतदशेषेण यथेच्छसि तथा कुरु}


\twolineshloka
{सर्वगुह्यतमं भूयः श्रृणु मे परमं वचः}
{इष्टेऽसि मे दृढमिति ततो वक्ष्यामि ते हितम्}


\twolineshloka
{मन्मना भव मद्भक्तो मद्याजी मां नमस्कुरु}
{मामेवैष्यसि सत्यं ते प्रतिजाने प्रियोऽसि मे}


\twolineshloka
{सर्वधर्मान्परित्यज्य मामेकं शरणं व्रज}
{अहं त्वा सर्वपापेभ्यो मोक्षयिष्यामि मा शुचः}


\twolineshloka
{इदं ते नातपस्काय नाभक्ताय कदाचन}
{न चाशुश्रूषवे वाच्यं न च मां योऽभ्यसूयति}


\twolineshloka
{य इदं परमं गुह्यं मद्भक्तेष्वभिधास्यति}
{भक्तिं मयि परां कृत्वा मामेवैष्यत्यसंशयः}


\twolineshloka
{न च तस्मान्मनुष्येषु कश्चिन्मे प्रियकृत्तमः}
{भविता न च मे तस्मादन्यः प्रियतरो भुवि}


\twolineshloka
{अध्येष्यते च य इमं धर्म्यं संवादमावयोः}
{ज्ञानयज्ञेन तेनाहमिष्टः स्यामिति मे मतिः}


\twolineshloka
{श्रद्धावाननसूयश्च श्रृणुयादपि यो नरः}
{सोपि मुक्तः शुभाँल्लोकान्प्राप्नुयात्पुण्यकर्मणां}


\threelineshloka
{कच्चिदेतच्छ्रुतं पार्थ त्वयैकाग्रेण चेतसा}
{कच्चिदज्ञानसंमोहः प्रनष्टस्ते धनंजय ॥अर्जुन उवाच}
{}


\threelineshloka
{नष्टो मोहः स्मृतिर्लब्धा त्वत्प्रसादान्मयाऽच्युत}
{स्थितोऽस्मि गतसंदेहः करिष्ये वचनं तव ॥सञ्जय उवाच}
{}


\twolineshloka
{इत्यहं वासुदेवस्य पार्थस्य च महात्मनः}
{संवादमिममश्रौषमद्भुतं रोमहर्षणम्}


\twolineshloka
{व्यासप्रसादाच्छ्रुतवानेतद्गुह्यमहं परम्}
{योगं योगेश्वरात्कृष्णात्साक्षात्कथयतः स्वयम्}


\twolineshloka
{राजन्संस्मृत्य संस्मृत्य संवादमिममद्भुतम्}
{केशवार्जुवयोः पुण्यं हृष्यामि च मुहुर्मुहुः}


\twolineshloka
{तच्च संस्मृत्य संस्मृत्य रूपमत्यद्भुतं हरेः}
{विस्मयो मे महान्राजन्हृष्यामि च पुनः पुनः}


\twolineshloka
{यत्र योगेश्वरः कृष्णो यत्र पार्थो धनुर्धरः}
{तत्र श्रीर्विजयो भूतिर्ध्रुवा नीतिर्मतिर्मम}


\chapter{अध्यायः ४३}
\twolineshloka
{वैशंपायन उवाच}
{}


\twolineshloka
{गीता सुगीता कर्तव्या किमन्यैः शास्त्रसंग्रहैः}
{या स्वयं पद्मनाभस्य सुखपद्माद्विनिःसृता}


\twolineshloka
{सर्वशास्त्रमयी गीता सर्वदेवमयो हरिः}
{सर्वतीर्थमयी गङ्गा सर्वदेवमयो मनुः}


\twolineshloka
{गीता गङ्गा च गायत्री गोविन्देति हृदि स्थिते}
{चतुर्गकारसंयुक्ते पुनर्जन्म न विद्यते}


\twolineshloka
{षट्शतानि सविंशानि श्लोकानां प्राह केशवः}
{अर्जुनः सप्तपञ्चाशत्सप्तषष्टिं तु सञ्जयः}


\fourlineindentedshloka
{धृतराष्ट्रः श्लोकमेकं गीताया मानमुच्यते}
{भारतामृतसर्वस्वगीताया मथितस्य च}
{सारमुद्धृत्य कृष्णेन अर्जुनस्य मुखे हुतम् ॥सञ्जय उवाच}
{}


\twolineshloka
{ततो धनंजयं दृष्ट्वा बाणगाण्डीवधारिणम्}
{पुनरेव महानादं व्यसृजन्त महारथाःक}


\twolineshloka
{पाण्डवाः सोमकाश्चैव ये चैषामनुयायिनः}
{दध्मुश्च मुदिताः शङ्खान्वीराः सागरसंभवान्}


\twolineshloka
{ततो भेर्यश्च पेश्यश्च क्रकचा गोविषाणिकाः}
{सहसैवाभ्यहन्यन्त ततः शब्दो महानभूत्}


\twolineshloka
{तथा देवाः सगन्धर्वाः पितरश्च जनाधिप}
{सिद्धचारणसङ्घाश्च समीयुस्ते दिदृक्षया}


\twolineshloka
{ऋषयश्च महाभागाः पुरस्कृत्य शतक्रतुम्}
{समीयुस्तत्र सहिता द्रुष्टुं तद्वैशसं महत्}


\twolineshloka
{` ते सेने स्तिमिते सञ्जे वीक्षमाणे परस्परम्}
{गङ्गायमुनयोर्वेगो यथैवैत्य परस्परम्}


\twolineshloka
{एवं प्रवत्ते ते सेने निःशब्दे जनसंसदि}
{चित्रे पट इवालेख्ये दर्शनीयतरे शुभे ॥'}


\twolineshloka
{ततो युधिष्ठिरो दृष्ट्वा युद्धाय समवस्थिते}
{ते सेने सागरप्रख्ये मुहुः प्रज्वलिते नृप}


\twolineshloka
{विमुच्य कवचं वीरो निक्षिप्य च वरायुधम्}
{अवरुह्य रथात्क्षिप्रं पद्म्यामेव कृताञ्जलिः}


\twolineshloka
{पितामहमभिप्रेक्ष्य धर्मराजो युधिष्ठिरः}
{वाग्यतः प्रययौ धीरः प्राङ्भुखो रिपुवाहिनीम्}


\twolineshloka
{तं प्रयान्तमभिप्रेभ्य कुन्तीपुत्रो धनञ्जयः}
{अवतीर्य रथात्तूर्णं भ्रातृभिः सहितोऽन्वयात्}


\threelineshloka
{वासुदेवश्च भगवान्पृष्ठतोऽनुजगाम तम्}
{तथा मुख्याश्च राजानस्तच्चित्ता जग्मुरुत्सुकाः ॥अर्जुन उवाच}
{}


\threelineshloka
{किं ते व्यवसितं राजन्यदस्मानपहाय वै}
{प्रद्म्यामेव प्रयातोऽसि प्राङ्मुखो रिपुवाहिनीम् ॥भीमसेन उवाच}
{}


\threelineshloka
{क्व गमिष्यसि राजेन्द्र निक्षिप्तकवचायुधः}
{दंशितेष्वरिसैन्येषु भ्रातॄनुत्सृज्य पार्थिव ॥नकुल उवाच}
{}


\threelineshloka
{एवं गते त्वयि ज्येष्ठे मम भ्रातरि भारत}
{भीमे दुनोति हृदयं ब्रूहि गन्ता भवान्क्व नु ॥सहदेव उवाच}
{}


\threelineshloka
{अस्मिन्रणसमूहे वै वर्तमाने महाभये}
{उत्सृज्य क्व नु गन्तासि शत्रूनभिमुखो नृप ॥सञ्जय उवाच}
{}


\twolineshloka
{एवमाभाष्यमाणोऽपि भ्रातृभिः कुरुनन्दनः}
{नोवाच वाग्यतः किंचिद्गच्छत्येव युधिष्ठिरः}


\twolineshloka
{तानुवाच महाप्राज्ञो वासुदेवो महामनाः}
{अभिप्रायोऽस्य विज्ञातो मयेति प्रहसन्निव}


\twolineshloka
{एष भीष्मं तथा द्रोणं गौतमं शल्यमेव च}
{अनुमान्य गुरून्सर्वान्योत्स्यते पार्थिवोऽरिभिः}


\twolineshloka
{श्रूयते हि पुराकल्पे गुरूनननुमान्य यः}
{युध्यते स भवेद्व्यक्तमपध्यातो महत्तरैः}


\twolineshloka
{अनुमान्य यथाशास्त्रं यस्तु युध्येन्महत्तरैः}
{ध्रुवस्तस्य जयो युद्धे भवेदिति मतिर्मम}


\twolineshloka
{एवं ब्रुवति कृष्णेऽत्र धार्तराष्ट्रचमूं प्रति}
{हाहाकारो महानासीन्निःशब्दास्त्वपरेऽभवन्}


\threelineshloka
{दृष्ट्वा युधिष्ठिरं दूराद्धार्तराष्ट्रस्य सैनिकाः}
{मिथः संकथयांचक्रुरेषो हि कुलपांसनः ॥ 6-43-29aव्यक्तं भीतइवाभ्येति राजासौ भीष्ममन्तिकम्}
{युधिष्ठिरः ससोदर्यः शरणार्थं प्रयाचकः}


\twolineshloka
{धनञ्जये कथं नाथे पाण्डवे च वृकोदरे}
{नकुले सहदेवे च भीतिरभ्येति पाण्डवम्}


\twolineshloka
{न नूनं क्षत्रियकुले जातः संप्रथिते भुवि}
{यथाऽस्य हृदयं भीतमल्पसत्वस्य संयुगे}


\twolineshloka
{ततस्ते सैनिकाः सर्वे प्रशंसन्ति स्म कौरवान्}
{हृष्टाः सुमनसो भूत्वा चेलानि दुधुवुश्च ह}


\twolineshloka
{व्यनिन्दंश्च तथा सर्वे योधास्तव विशांपते}
{युधिष्ठिरं ससोदर्यं सहितं केशवनेन हि}


\twolineshloka
{ततस्तत्कौरवं सैन्यं धिक्वृत्वा तु युधिष्ठिरम्}
{निःशब्दमभवत्तूर्णं पुनरेव विशांपते}


\twolineshloka
{किं नु वक्ष्याति राजाऽसौ किं भीष्मः प्रतिवक्ष्यति}
{किं भीमः समरश्लाघी किं नु कृष्णार्जुनाविति}


\twolineshloka
{विवक्षितं किमस्येति संशयः सुमहानभूत्}
{उभयो सेनयो राजन्युधिष्ठिरकृते तदा}


\twolineshloka
{सोऽवगाह्य चमूं शत्रोः शरशक्तिसमाकुलाम्}
{भीष्ममेवाभ्यात्तूर्णं भ्रातृभिः परिवारितः}


\threelineshloka
{तमुवाच ततः पादौ कारभ्यां पीड्य पाण्डवः}
{भीष्मं शान्तनवं राजा युद्धाय समुपस्थितम् ॥युधिष्ठिर उवाच}
{}


\threelineshloka
{आमन्त्रये त्वां दुर्धर्ष त्वया योत्स्यामहे सह}
{अनजानीहि मां तात आशिषश्च प्रयोजय ॥भीष्म उवाच}
{}


\twolineshloka
{यद्येवं नाभिगच्छेथा युधि मां पृथिवीपते}
{शपेयं त्वां महाराज परीभावाय भारत}


\twolineshloka
{प्रीतोऽहं पुत्र युध्यस्व जयमाप्नुहि पाण्डव}
{यत्तेऽभिलषितं चान्यत्तदवाप्नुहि संयुगे}


\twolineshloka
{व्रियतां च वरः पार्थ किमस्मत्तोऽभिकाङ्क्षसि}
{एवं गते महाराज न तवास्ति पराजयः}


\twolineshloka
{अर्थस्य पुरुषो दासो दासस्त्वर्थो न कस्यचित्}
{इति सत्यं महाराज बद्धोऽस्म्यर्थेन कौरवैः}


\threelineshloka
{अतस्त्वां क्लीबवद्वाक्यं ब्रवीमि कुरुनन्दन}
{भृतोऽस्त्म्यर्थेन कौरव्य युद्धादन्यत्किमिच्छसि ॥युधिष्ठिर उवाच}
{}


\threelineshloka
{मन्त्रयस्व महाबाहो हितैषी मम नित्यशः}
{युध्यस्व कौरवस्यार्थे ममैष सततं वरः ॥भीष्म उवाच}
{}


\threelineshloka
{राजन्किमत्र साह्यं ते करोमि कुरुनन्दन}
{कामं योत्स्ये परस्यार्थे ब्रूहि यत्ते विवक्षितम् ॥युधिष्ठिर उवाच}
{}


\threelineshloka
{कथं जयेयं संग्रामे भवन्तमपराजितम्}
{एतन्मे मन्त्रय हितं यदि श्रेयः प्रपश्यसि ॥भीष्म उवाच}
{}


\threelineshloka
{नैनं पश्यामि कौन्तेय यो मां युध्यन्तमाहवे}
{विजयेत पुमान्कश्तित्साक्षादपि शतक्रतुः ॥युधिष्ठिर उवाच}
{}


\threelineshloka
{हन्त पृच्छामि तस्मात्त्वां पितामह नमोस्तु ते}
{वधोपायं ब्रवीहि त्वमात्मनः समरे परैः ॥भीष्म उवाच}
{}


\threelineshloka
{न स्म तं तात पश्यामि समरे यो जयेत माम्}
{न तावन्मृत्युकालोऽपि पुनरागमनं कुरु ॥सञ्जय उवाच}
{}


\twolineshloka
{ततो युधिष्ठिरो वाक्यं भीष्मस्य कुरुनन्दन}
{शिरसा प्रतिजग्राह भूयस्तमभिवाद्य च}


\twolineshloka
{प्रायात्पुनर्महाबाहुराचार्यस्य रथं प्रति}
{पश्यतां सर्वसैन्यानां मध्येन भ्रातृभिः सह}


\twolineshloka
{स द्रोणमभिवाद्याथ कृत्वं चाभिप्रदक्षिणम्}
{उवाच राजा दुर्धर्षमात्मनिःश्रेयसं वचः}


\threelineshloka
{आमन्त्रये त्वां भगवन्योत्स्ये विगितकल्मषः}
{कथं जये रिपून्सर्वाननुज्ञातस्त्वया द्विज ॥द्रोण उवाच}
{}


\twolineshloka
{यदि मां नाभिगच्छेथा युद्धाय कृतनिश्चयः}
{शपेयं त्वां महाराज परीभावाय सर्वशः}


\twolineshloka
{तद्युधिष्ठिर तुष्टोऽस्मि पूजितश्च त्वयाऽनघ}
{अनुजानामि युध्यस्व विजयं समवाप्नुहि}


\twolineshloka
{करवाणि च ते कामं ब्रूहि त्वमभिकाङ्क्षितम्}
{एवं गते महाराज युद्धादन्यत्किमिच्छसि}


\twolineshloka
{अर्थस्य पुरुषो दासो दासस्त्वर्थो न कस्यचित्}
{इति सत्यं महाराज बद्धोऽस्म्यर्थेन कौरवैःक}


\threelineshloka
{ब्रवीम्येतत्क्लीबवत्त्वां युद्धादन्यत्किमिच्छसि}
{योत्स्येऽहं कौरवस्यार्थे तवाशास्यो जयो मया ॥युधिष्ठिर उवाच}
{}


\threelineshloka
{जयमाशास्स्व मे ब्रह्मन्मन्त्रयस्व च मद्धितम्}
{युध्यस्व कौरवस्यार्थे वर एष वृतो मया ॥द्रोण उवाच}
{}


\twolineshloka
{ध्रुवस्ते विजयो राजन्यस्य मन्त्री हरिस्तव}
{अहं त्वामभिजानामि रणे शत्रून्विजेष्यसि}


\threelineshloka
{यतो धर्मस्ततः कृष्णो यतः कृष्णस्ततो जयः}
{युध्यस्व गच्छ कौन्तेय पृच्छ मां किं ब्रवीमि ते ॥युधिष्ठिर उवाच}
{}


\threelineshloka
{पृच्छामि त्वां द्विजश्रेष्ठ श्रृणु यन्मेऽभिकाङ्क्षितम्}
{कथं जयेयं संग्रामे भवन्तमपराजितम् ॥द्रोण उवाच}
{}


\threelineshloka
{न तेऽस्ति विजयस्तावद्यावद्युद्ध्याम्यहं रणे}
{ममाशु निधने राजन्यतस्व सह सदरैः ॥युधिष्ठिर उवाच}
{}


\threelineshloka
{हन्त तस्मान्महाबाहो वधोपायं वदात्मनः}
{आचार्य प्रणिपत्यैष पृच्छामि त्वां नमोऽस्तु ते ॥द्रोण उवाच}
{}


\twolineshloka
{न शत्रुं तात पश्यामि यो मां हन्याद्रथे स्थितम्}
{युध्यमानं सुसंरब्धं शरवर्षौघवर्षिणम्}


\threelineshloka
{ऋते प्रायगतं रजन्न्यस्तशस्त्रमचेतनम्}
{हन्यान्मां युधि योधानां सत्यमेतद्ब्रवीमि ते}
{}


\threelineshloka
{शस्त्रं चाहं रणे जह्यां श्रुत्वा तु महदप्रियम्}
{श्रद्धेपवाक्यात्पुरुषादेतत्सत्यं ब्रवीति ते ॥सञ्जय उवाच}
{}


\twolineshloka
{एतच्छ्रुत्वा महाराज भारद्वाजस्य धीमतः}
{अनुमान्य तमाचार्यं प्रायाच्छारद्वतं प्रति}


\twolineshloka
{सोऽभिवाद्य कृपं राजा कृत्वा चापि प्रदक्षिणम्}
{उवाच दुर्धर्षतमं वाक्यं वाक्यविदां वरः}


\threelineshloka
{अनुमानये त्वां योत्स्येऽहं गुरो विगतकल्मषः}
{जयेयं च रिपून्सर्वाननुज्ञातस्त्वयाऽनघ ॥कृप उवाच}
{}


\twolineshloka
{यदि मां नाभिगच्छेथा युद्धाय कृतनिश्चयः}
{शपेयं त्वां महाराज परीभावाय सर्वशः}


\threelineshloka
{अर्थस्य पुरुषो दासो दासस्त्वर्थो न कस्यचित्}
{इति सत्यं महाराज बद्धोऽस्म्यर्थेन कौरवैः}
{}


\threelineshloka
{तेषामर्थे महाराज योद्धव्यमिति मे मतिः}
{अतस्त्वां क्लीबवद््ब्रूयांयुद्धादन्यत्किमिच्छसि ॥युधिष्ठिर उवाच}
{}


\twolineshloka
{हन्त पृच्छामि ते तस्मादाचार्य श्रृणु मे वचः}
{इत्युक्त्वा व्यथितो राजा नोवाच गतचेतनः ॥सञ्जय उवाच}


\twolineshloka
{तं गौतमः प्रत्युवाच विज्ञायास्य विवक्षितम्}
{अवध्योऽहं महीपाल यद्ध्यस्व जयमाप्नुहि}


\twolineshloka
{प्रीतस्तेऽभिगमेनाहं जयं तव नराधिप}
{आशासिष्ये सदोत्थाय सत्यमेतद्ब्रवीमि ते}


\twolineshloka
{एतच्छ्रुत्वा महाराज गौतमस्य विशांपते}
{अनुमान्य कृपं राजा प्रययौ येन मद्रराट्}


\twolineshloka
{स शल्यमभिवाद्याथ कृत्वा चाभिप्रदक्षिणम्}
{उवाच राजा दुर्धर्षमात्मनिःश्रेयसं वचः}


\threelineshloka
{अनुमानये त्वां दुर्धर्ष योत्स्ये विगतकल्मषः}
{जयेयं नु परान्राजन्ननुज्ञातस्त्वया रिपून् ॥शल्य उवाच}
{}


\twolineshloka
{यदि मां नाधिगच्छेथा युद्धाय कृतनिश्चयः}
{शपेयं त्वां महाराज परीभावाय वै रणे}


\twolineshloka
{तुष्टोऽस्मि पूजितश्चास्मि यत्काङ्क्षसि तदस्तु ते}
{अनुजानामि चैव त्वां युध्यस्व जयमाप्नुहि}


\twolineshloka
{ब्रूहि चैष परं वीर केनार्थः किं ददामि ते}
{एवं गते महाराज युद्धादन्यत्किमिच्छसि}


\twolineshloka
{अर्थस्य पुरुषो दासो दासस्त्वर्थो न कस्यचित्}
{इति सत्यं महाराज बद्धोस्म्यर्थेन कौरवैः}


\threelineshloka
{करिष्यामि हि ते कामं भागिनेय यथेप्सितम्}
{ब्रवीम्यतः क्लीबवत्त्वां युद्धादन्यत्किमिच्छसि ॥युधिष्ठिर उवाच}
{}


\threelineshloka
{मन्त्रयस्व महाराज नित्यं मद्धितमुत्तमम्}
{कामं युद्ध्य परस्यार्थे वरमेतं वृणोम्यहम् ॥शल्य उवाच}
{}


\threelineshloka
{किमत्र ब्रूहि साह्यं ते करोमि नृपसत्तम}
{कामं योत्स्ये परस्यार्थे बद्धोऽस्म्यर्थेन कौरवैः ॥युधिष्ठिर उवाच}
{}


\threelineshloka
{स एव मे वरः शल्य उद्योगे यस्त्वया कृतः}
{सूतपुत्रस्य संग्रामे कार्यस्तेजोवधस्त्वया ॥शल्य उवाच}
{}


\threelineshloka
{संपत्स्यत्येष ते कामः कुन्तीपुत्र यथेप्सितम्}
{गच्छ युध्यस्व विस्रब्धः प्रतिजाने वचस्तव ॥सञ्जय उवाच}
{}


\twolineshloka
{अनुमान्याथ कौन्तेयो मातुलं मद्रकेश्वरम्}
{निर्जगाम महासैन्याद्भ्रतृभिः परिवारितः}


\twolineshloka
{वासुदेवस्तु राधेयमाहवेऽभिजगाम वै}
{तत एनमुवाचेदं पाण्डवार्थे गदाग्रजः}


\twolineshloka
{श्रुतं मे कर्ण भीष्मस्य द्वोषात्किल न योत्स्यसे}
{अस्मान्वरय राधेय यावद्भीष्मो न हन्यते}


\threelineshloka
{हते तु भीष्मे राधेय पुनरेष्यसि संयुगम्}
{धार्तराष्ट्रस्य साहाय्यं यदि पश्यसि चेत्समम् ॥कर्ण उवाच}
{}


\threelineshloka
{न विप्रियं करिष्यामि धार्तराष्ट्रस्य केशव}
{त्यक्तप्राणं हि मां विद्धि दुर्योधनहितैषिणम् ॥सञ्जय उवाच}
{}


\twolineshloka
{तच्छ्रुत्वा वचनं कृष्णः संन्यवर्तत भारत}
{युधिष्ठिरपुरोगैश्च पाण्डवैः सह संगतः}


\twolineshloka
{अथ सैन्यस्य मध्ये तु प्राक्रोशत्पाण्डवाग्रजः}
{योऽस्मान्वृणोति तमहं वरये साह्यकारणात्}


\twolineshloka
{अथ तान्समभिप्रेक्ष्य युयुत्सुरिदमब्रवीत्}
{प्रीतात्मा धर्मराजानं कुन्तीपुत्रं युधिष्ठिरम्}


\threelineshloka
{अहं योत्स्यामि भवतः संयुगे धृतराष्ट्रजान्}
{युष्मदर्थं महाराज यदि मां वृणुषेऽनघ ॥युधिष्ठिर उवाच}
{}


\twolineshloka
{एह्येहि सर्वे योत्स्यामस्तव भ्रातॄनपण्डितान्}
{युयुत्सो वासुदेवश्च वयं च ब्रूम सर्वशः}


\twolineshloka
{वृणोमि त्वां महाबाहो युध्यस्व मम कारणात्}
{त्वयि पिण्डश्च तन्तुश्च धृतराष्ट्रस्य दृश्यते}


\threelineshloka
{भजस्वास्मान्राजपुत्र भजमानान्महाद्युते}
{न भविष्यति दुर्बुद्धिर्धार्तराष्ट्रोऽत्यमर्षणः ॥सञ्जय उवाच}
{}


\twolineshloka
{ततो युयुत्सुः कौरव्यान्परित्यज्य सुतांस्तव}
{जगाम पाण्डुपुत्राणां सेनां विश्राव्य दुन्दुभिं}


\twolineshloka
{ततो युधिष्ठिरो राजा संप्रहृष्टःक सहानुजः}
{जग्राह कवचं भूयो दीप्तिमत्कनकोञ्ज्वलम्}


\twolineshloka
{प्रत्यपद्यन्त ते सर्वे स्वरथान्पुरुषर्षभाः}
{ततो व्यूहं यथापूर्वं प्रत्यव्यूहन्त ते पुनः}


\twolineshloka
{अवादयन्दुन्दुभींश्च शतशश्चैव पुष्करान्}
{सिंहनादांश्च विविधान्विनेदुः पुरुषर्षभाः}


\twolineshloka
{रथस्थान्पुरुषव्याघ्रान्पाण्डवान्प्रेक्ष्य पार्थिवाः}
{धृष्टद्युम्नादयः सर्वे पुनर्जहृषिरे तदा}


\twolineshloka
{गौरवं पाण्डुपुत्राणां मान्यान्मानयतां च तान्}
{दृष्ट्वा महीक्षितस्तत्र पूजयांचक्रिरे भृशम्}


\twolineshloka
{सौहृदं च कृपां चैव प्राप्तकालं महात्मनाम्}
{दयां च ज्ञातिषु परां कथयांचक्रिरे नृपाः}


\twolineshloka
{साधुसाध्विति सर्वत्र निश्चेरुः स्तुतिसंहिताः}
{वाचः पुण्याः कीर्तिमतां मनोहृदयहर्षणाः}


\twolineshloka
{म्लेच्छाश्चार्याश्च ये तत्र ददृशुः शुश्रुवुस्तथा}
{वृत्तं तत्पाण्डुपुत्राणां रुरुदुस्ते सगद्गदाः}


\twolineshloka
{ततो जघ्नुर्महाभेरीः शतशश्च सहस्रशः}
{शङ्खांश्च गोक्षीरनिभान्दध्मुर्हृष्टा मनस्विनः}


\chapter{अध्यायः ४४}
\twolineshloka
{धृतराष्ट्र उवाच}
{}


\threelineshloka
{एवं व्यूढेष्वनीकेषु मामकेष्वितरेषु च}
{के पूर्वं प्राहरंस्तत्र कुरवःक पाण्डवा नु किम् ॥सञ्जय उवाच}
{}


\twolineshloka
{भ्रातृभिः सहितो राजन्पुत्रो दुःशासनस्तव}
{भीष्मं प्रमुखतः कृत्वा प्रययौ सह सेनया}


\twolineshloka
{तथैव पाण्डवाः सर्वे भीमसेनपुरोगमाः}
{भीष्मेण युद्धमिच्छन्तः प्रययुर्हृष्टमानसाः}


\twolineshloka
{क्ष्वेडाः किलकिलाशब्दाः क्रकचा गोविषाणिकाः}
{भेरीमृदङ्गमुरजा हयकुञ्जरनिःस्वनाः}


\twolineshloka
{उभयोःक सेनयोर्ह्यासंस्ततस्तेऽस्मान्समाद्रवन्}
{वयं तान्प्रतिर्दन्तस्तदाऽऽसीत्तुमुलं महत्}


\twolineshloka
{महान्त्यनीकानि महासमुच्छ्रयेसमागमे पाण्डवधार्तराष्ट्रयोः}
{चकम्पिरे शङ्खमृदङ्गनिःस्वनैःप्रकम्पितानीव वनानि वायुना}


\twolineshloka
{नरेन्द्र नागाश्वरताकुलाना-मभ्यागतानामशिवे मुहूर्ते}
{बभूव घोषस्तुमुलश्चमूनांवातोद्धुतानामिव सागराणाम्}


\twolineshloka
{तस्मिन्समुत्थिते शब्दे तुमुले रोमहर्षणे}
{भीमसेनो महाबाहुः प्राणदद्गोवृषो यथा}


\twolineshloka
{शङ्खदुन्दुभिनिर्घोषं वारणानां च बृंहितम्}
{सिंहनादं च सैन्यानां भीससेनरवोऽभ्यभूत्}


\twolineshloka
{हयानां हेषमाणानामनीकेषु सहस्रशः}
{सर्वानभ्यभवच्चब्दान्भीमस्य नदतः स्वनः}


\twolineshloka
{तं श्रुत्वा निनदं तस्य सैन्यास्तव वितत्रसुः}
{जीमूतस्येव नदतः शक्राशनिसमस्वनम्}


\twolineshloka
{वाहनानि च सर्वाणि शकृन्मूत्रं प्रमुस्रुवुः}
{शब्देन तस्य वीरस्य सिंहस्येवेतरेक मृगाः}


\twolineshloka
{दर्शयन्घोरमात्मानं महाभ्रमिव नादयन्}
{विभीषयंस्तव सुतान्भीमसेनः समभ्ययात्}


\twolineshloka
{तमायान्तं महेष्वासं सोदर्याः पर्यवारयन्}
{छादयन्तः शरव्रातैर्मेघा इव दिवाकरम्}


\twolineshloka
{दुर्योधनश्च पुत्रस्ते दुर्मुखो दुःसहः शलः}
{दुःशासनश्चातिरथस्तथा दुर्मर्षणो नृपः}


\twolineshloka
{विविंशतिश्चित्रसेनो विकर्णश्च महारथः}
{पुरुमित्रो जयो भोजः सौमदत्तिश्च वीर्यवान्}


\twolineshloka
{महाचापानि धुन्वन्तो मेघा इव सविद्युतः}
{आददानाश्च नाराचान्निर्मुक्ताशीविषोपमान्}


\twolineshloka
{अथ ते द्रौपदीपुत्राः सौभद्रश्च महारथः}
{नकुलः सहदेवश्च धृष्टद्युम्नश्च पार्षतः}


\twolineshloka
{धार्तराष्ट्रान्प्रतिययुरर्दयन्तः शितैः शरैः}
{वज्रैरिव महावेगैः शिखराणि धराभृताम्}


\twolineshloka
{तस्मिन्प्रथमसंग्रामे भीमज्यातलनिःस्वने}
{तावकानां परेषां च नासीत्कश्चित्पराङ्मुखः}


\twolineshloka
{लाघवं द्रोणशिष्याणामपश्यं भरतर्षभ}
{निमित्तवेधिनां चैव शरानुत्सृजतां भृशम्}


\twolineshloka
{नोपशाम्यति निर्घोषो धनुषां कूजतां तथा}
{विनिश्चेरुः शरा दीप्ताज्योतींषीव नभस्तलात्}


\twolineshloka
{सर्वे त्वन्ये महीपालाः प्रेक्षका इव भारत}
{ददृशुर्दर्शनीयं तं भीमं ज्ञातिसमागमम्}


\twolineshloka
{ततस्ते जातसंरम्भाः परस्परकृतागसः}
{अन्योन्यस्पर्धया राजन्व्यायच्छन्त महारथाः}


\twolineshloka
{कुरुपाण्डवसेने ते हस्त्यश्वरथसंकुले}
{शुशुभाते रणेऽतीव पटे चित्रार्पिते इव}


\twolineshloka
{ततस्ते पार्थिवाः सर्वे प्रगृहीतशरासनाः}
{सहसैन्याः समापेतुः पुत्रस्य तव शासनात्}


\twolineshloka
{युधिष्ठिरेण चदिष्टाः पार्थिवास्ते सहस्रशः}
{विनदन्तः समापेतुः पुत्रस्य तव वाहिनीम्}


\twolineshloka
{उभयोः सेनयोस्तीव्रः सैन्यानां स समागमः}
{अन्तर्धीयत चादित्यः सैन्येन रजसा वृतः}


\twolineshloka
{प्रयुद्धानां प्रभग्नानां पुनरावर्तिनामपि}
{नात्र स्वेषां परेषां वा विशेषः समदृश्यत}


\twolineshloka
{तस्मिस्तु तुमुले युद्धे वर्तमाने महाभये}
{अति सर्वाण्यनीकानि पिता तेऽभिव्यरोचत}


\chapter{अध्यायः ४५}
\twolineshloka
{सञ्जय उवाच}
{}


\twolineshloka
{पूर्वाह्णे तस्य रौद्रस्य युद्धमह्नो विशांपते}
{प्रावर्तत महाघोरं राज्ञां देहावकर्तनम्}


\twolineshloka
{कुरूणां सृञ्जयानां च जिगीषूणां परस्परम्}
{सिंहानामिव संह्रादो दिवमुर्वी च नादयन्}


\twolineshloka
{आसीत्किलकिलाशब्दस्तलशङ्खरवैः सह}
{जझिरे सिंहनादाश्च शूराणां प्रतिगर्जताम्}


\twolineshloka
{तलत्राभिहताश्चैव ज्याशब्दा भरतर्षभ}
{पत्तीनां पादशब्दश्च वाजिनां च महास्वनः}


\twolineshloka
{तोत्राङ्कुशनिपातश्च आयुधानां च निःस्वनः}
{घण्टाशब्दश्च नागानामन्योन्यमभिधावताम्}


\twolineshloka
{तस्मिन्समुदिते शब्दे तुगुले रोमहर्षणे}
{बभूव रथनिर्घोषः पर्जन्यनिनदोपमःक}


\twolineshloka
{ते मनः क्रूरमाधाय समभित्यक्तजीविताः}
{पाण्डवानभ्यवर्तन्त सर्व एवोच्छ्रितध्वजाः}


\twolineshloka
{अथ शान्तनवो राजन्नभ्यधावद्धनञ्जयम्}
{प्रगृह्य कार्मुकं घोरं कालदण्डोपमं रणे}


\twolineshloka
{अर्जुनोऽपि धनुर्गृह्य गाण्डीवं लोकविश्रुतम्}
{अभ्यधावत तेजस्वी गाङ्गेयं रणमूर्धनि}


\twolineshloka
{तावुभौ कुरुशार्दूलौ परस्परवधैषिणौ}
{गाङ्गेयस्तु रणे पार्थं विद्ध्वा नाकम्पयद्बली}


\twolineshloka
{तथैव पाण्डवो राजन्भीष्मं नाकम्पयद्युधि}
{सात्यकिस्तु महेष्वासः कृतवर्माणमभ्यात्}


\twolineshloka
{तयोः समभयवद्युद्धं तुमुलं रोमहर्षणम्}
{सात्यकिः कृतवर्माणं कृतवार्मा च सात्यकिम्}


\twolineshloka
{आनर्च्छतुः शरैर्घोरैस्तक्षमाणौ परस्परम्}
{तौ शराचितसर्वाङ्गौ शुशुभाते महाबलौ}


\twolineshloka
{वसन्ते पुष्पशबलौ पुष्पिताविव किंशुकौ}
{अभिमन्युर्महेष्वासं बृहद्बलमयोधयत्}


\twolineshloka
{ततः कोसलराजाऽसावभिमन्योर्विशांपते}
{ध्वजं चिच्छेद समरे सारथिं च व्यपातयत्}


\twolineshloka
{सौभद्रस्तु ततः क्रुद्धः पातिते रथसारथौ}
{बृहद्बलं महाराज विव्याध नवभिः शरैः}


\twolineshloka
{अथापराभ्यां भल्लाभ्यां शिताभ्यामरिमर्दनः}
{ध्वजमेकेन चिच्छेद पार्ष्णिमेकेन सारथिम्}


\twolineshloka
{अन्योन्यं च शरैः क्रुद्धौ ततक्षाते परस्परम्}
{मानिनं समरे दृप्तं कृतवैरं महारथम्}


\twolineshloka
{भीमसेनस्तव सुतं दुर्योधनमयोधयत्}
{तावुभौ नरशार्दूलौ कुरुमुख्यौ महाबलौ}


\twolineshloka
{अन्योन्यं शरवर्षाभ्यां ववृषाते रणाजिरे}
{तौ वीक्ष्य तु महात्मानौ कृतिनौ चित्रयोधिनौ}


\twolineshloka
{विस्मयः सर्वभूतानां समपद्यत भारत}
{दुःशासनस्तु नकुलं प्रत्युद्याय महाबलम्}


\twolineshloka
{अविध्यन्निशितैर्बाणैर्बहुभिर्मर्मभेदिभिः}
{तस्य माद्रीसुतः केतुं सशरं च शरासनम्}


\twolineshloka
{चिच्छेद निशितैर्बाणैः प्रहसन्निव भारत}
{अथैनं पञ्चविंशत्या क्षुद्रकाणां समार्पयत्}


\twolineshloka
{पुत्रस्तु तव दुर्धर्षो नकुलस्य महाहवे}
{तुरङ्गांश्चिच्छिदे बाणैर्ध्वजं चैवाभ्यपातयत्}


\twolineshloka
{दुर्मुखः सहदेवं च प्रत्युद्याय महाबलम्}
{विव्याध शरवर्षेण यतमानं महाहवे}


\threelineshloka
{सहदेवस्ततो वीरो दुर्मुखस्य महारणे}
{शरेण भृतशीक्ष्णेन पातयामास सारथिम् ॥ 6-45-27aतावन्योन्यंसमासाद्य समरे युद्धदुर्मदौ}
{6-45-27bत्रासयेतां शरैर्घोरैःकृतप्रतिकृतैषिणौ}


\twolineshloka
{युधिष्टिरः स्वयं राजा मद्रराजानमभ्ययात्}
{तस्य मद्राधिपश्चापं द्विधा चिच्छेद मारिष}


\twolineshloka
{तदपास्य धनुश्छिन्नं कुन्तीपुत्रो युधिष्ठिरः}
{अन्यत्कार्मुकमादाय वेगवद्बलवत्तरम्}


\twolineshloka
{ततो मद्रेश्वरं राजा शरैः सन्नतपर्वभिः}
{छादयामास संक्रुद्धस्तिष्ठ तिष्ठेति चाब्रवीत्}


\twolineshloka
{धृष्टद्युम्नस्ततो द्रोणमभ्यद्रवत भारत}
{तस्य द्रोणः सुसंक्रुद्धः परासुकरणं दृढम्}


\twolineshloka
{त्रिधा चिच्छेद समरे पाञ्चाल्यस्य तु कार्मुकम्}
{शरं चैव महाघोरं कालदण्डमिवापरम्}


\twolineshloka
{प्रेषयामास समरे सोऽस्य काये न्यमञ्जत}
{अथान्यद्धनुरादाय सायकांश्च चतुर्दश}


\twolineshloka
{द्रोणं द्रुपदपुत्रस्तु प्रतिविव्याध संयुगे}
{तावन्योन्यं सुसंक्रुद्धौ चक्रतुः शुभृशं रणम्}


\twolineshloka
{सौमदत्तिं रणे शङ्खो रभसं रभसो युधि}
{प्रत्युद्ययौ महाराज तिष्ठतिष्ठेति चाब्रवीत्}


\twolineshloka
{तस्य वै दक्षिणं वीरो निर्बिभेद रणे भुजम्}
{समदत्तिस्तथा शङ्खं जत्रुदेशे समाहनत्}


\twolineshloka
{तयोस्तदभवद्युद्धं घोररूपं विशांपते}
{दृप्तयोः समरे पूर्वं वृत्रवासवयोरिव}


\twolineshloka
{बाह्लीकं तु रणे क्रुद्ध क्रुद्धरूपो विशांपते}
{अभ्यद्रवदमेयात्मा धृष्टकेतुर्महारथः}


\twolineshloka
{बाह्लीकस्तु रणे राजन्धृष्टकेतुममर्षणः}
{शरैर्बहुभिरानर्च्छत्सिंहनादमथानदत्}


\twolineshloka
{चेदिराजस्तु संक्रुद्धो बाह्लीकं नवभिः शरैः}
{विव्याध समरे तूर्णं मत्तो मत्तमिव द्विपम्}


\twolineshloka
{तौ तत्र समरे क्रुद्धौ नर्दन्तौ च पुनः पुनः}
{समीयतुः सुसंक्रुद्धावङ्गारवबुधाविव}


\twolineshloka
{राक्षसं रौद्रकर्माणं क्रूरकर्मा घटोत्कचः}
{अलम्बुसं प्रत्युदियाद्बलं शक्र इवाहवे}


\twolineshloka
{घटोत्कचस्ततः क्रुद्धो राक्षसं तं महाबलम्}
{नवत्या सायकैस्तीक्ष्णैर्दारयामास भारत}


\twolineshloka
{अलम्बुसस्तु समरे भैमसेनिं महाबलम्}
{बहुधा दारयामास शरैः सन्नतपर्वभिः}


\twolineshloka
{व्यभ्राजेतां ततस्तौ तु संयुगे शरविक्षतौ}
{यथा देवासुरे युद्धे बलशक्रौ महाबलौ}


\twolineshloka
{शिखण्डी समरे राजन्द्रौणिमभ्युद्ययौ बली}
{अश्वत्थामा ततः क्रुद्दः शिखण्डिनमुपस्थितम्}


\twolineshloka
{नाराचेन सुतीक्ष्णेन भृशं विद्ध्वा ह्यकम्पयत्}
{शिखण्ड्यपि ततो राजन्द्रोणपुत्रमताडयत्}


\twolineshloka
{सायकेन सुपीतेन तीक्ष्णेन निशितेन च}
{तौ जघ्नतुस्तदान्योन्यं शरैर्बहुविधैर्मृधे}


\twolineshloka
{भगदत्तं रणे शूरं विराटो वाहिनीपतिः}
{अभ्ययात्त्वरितो राजंस्ततो युद्धमवर्तत}


\twolineshloka
{विराटो भगदत्तं तु शरवर्षण भारत}
{अभ्यवर्षत्सुसंक्रुद्धो मेघो वृष्ट्या इवाचलम्}


\twolineshloka
{भगदत्तस्ततस्तूर्णं विराटं पृथिवीपतिम्}
{छादयामास समरे मेघः सूर्यमिवोदितम्}


\twolineshloka
{बृहत्क्षत्रं तु कैकेयं कृपः शारद्वतो ययौ}
{तं कृपः शरवर्षेण च्छादयामास भारत}


\twolineshloka
{गौतमं कैकयः क्रुद्धः शरवृष्ट्याऽभ्यपूरयत्}
{तावन्योन्यं हयान्हत्वा धनुश्छित्वा च भारत}


\twolineshloka
{विरथावसियुद्धाय समीयतुरमर्षणौ}
{तयोस्तदभवद्युद्धं घोररूपं सुदारुणम्}


\twolineshloka
{द्रुपदस्तु ततो राजन्सैन्धवं वै जयद्रथम्}
{अभ्युद्ययौ हृष्टरूपो हृष्टरूपं परंतपः}


\twolineshloka
{ततः सैन्धवको राजा द्रुपदं विशिखैस्त्रिभिः}
{ताजयामास समरे स च तं प्रत्यबिध्यत}


\twolineshloka
{तयोस्तदभवद्युद्धं घोररूपं सुदारुणम्}
{ईक्षणप्रीतिजननं शुक्राङ्गरकयोरिव}


\twolineshloka
{विकर्णस्तु सुतस्तुभ्यं सुतसोमं महाबलम्}
{अभ्ययाञ्जवनैरश्वैस्ततो युद्धमवर्तत}


\twolineshloka
{विकर्णः सुतसोमं तु विद्ध्वा नाकम्पयच्छरैः}
{सुतसोमो विकर्णं च तदद्भुतमिवाभवत्}


\twolineshloka
{सुशर्माणं नरव्याघ्रश्चेकितानो महारथः}
{अभ्यद्रवत्सुसंक्रुद्धः पाण्डवार्थे पराक्रमी}


\twolineshloka
{शुशर्मा तु महाराज चेकितानं महारथम्}
{महता शरवर्षेण वारयामास संयुगे}


\twolineshloka
{चेकितानोऽपि संरब्धः सुशर्माणां महाहवे}
{प्राच्छादयत्तमिषुभिर्महामेघ इवाचलम्}


\twolineshloka
{शकुनिः प्रतिविन्ध्यं तु पराक्रान्तं पराक्रमी}
{अभ्यद्रवत राजेन्द्र मत्तः सिंह इव द्विपम्}


\twolineshloka
{यौधिष्ठिरस्तु संक्रुद्धः सौबलं निशितैः शरैः}
{व्यदारयत संग्रामे मघवानिव दानवम्}


\twolineshloka
{शकुनिः प्रतिविन्ध्यं तु प्रतिविध्यन्तमाहवे}
{व्यदारयन्महाप्राज्ञः शरैः सन्नतपर्वभिः}


\twolineshloka
{सुदक्षिणं तु राजेन्द्र काम्भोजानां महारथम्}
{श्रुतकर्मा पराक्रान्तमभ्यद्रवत संयुगे}


\twolineshloka
{सुदक्षिणस्तु समरे साहदेविं महारथम्}
{विद्ध्वा नाकम्पयत वै मैनाकमिव पर्वतम्}


\twolineshloka
{श्रुतकर्मा ततः क्रुद्धः काम्भोजानां महारथम्}
{शरैर्बहुभिरानर्च्छद्दारयन्निव सर्वशः}


\twolineshloka
{इरावानथ संक्रुद्धः श्रुतायुषमरिन्दमम्}
{प्रत्युद्ययौ रणे यत्तो यत्तरूपं परंतपः}


\twolineshloka
{आर्जुनिस्तस्य समरे हयान्हत्वा महारथः}
{ननाद बलवान्नादं तत्सैन्यं प्रत्यपूरयत्}


\twolineshloka
{श्रुतायुस्तु ततः क्रुद्धः फाल्गुनेः समरे हयान्}
{निजघान गदाग्रेण ततो युद्धमवर्तत}


\twolineshloka
{विन्दानिविन्दावावन्त्यौ कुन्तिभोजं महारथम्}
{ससेनं ससुतं वीरं संसस़ञ्जतुराहवे}


\twolineshloka
{तत्राद्भुतमपश्याम तयोर्घोरं पराक्रमम्}
{अयुध्येतां स्थिरौ भूत्वा महत्या सेनया सह}


\twolineshloka
{अनुविन्दस्तु गदया कुन्तिभोजमताडयत्}
{कुन्तिभोजश्च तं तूर्णं शरव्रातैरवाकिरत्}


\twolineshloka
{कुन्तिभोजसुतश्चापि विन्दं विव्याध सायकैः}
{स च तं प्रतिविव्याध तदद्भुतमिवाभवत्}


\twolineshloka
{केकया भ्रातरः पञ्च गान्धारान्पञ्च मारिष}
{ससैन्यास्ते ससैन्यांश्च योधयामासुराहवे}


\twolineshloka
{वीरबाहुश्च ते पुत्रो वैराटिं रथसत्तमम्}
{उत्तरं योधयामास विव्याध निशितैः शरैः}


\twolineshloka
{उत्तरश्चापि तं वीरं विव्याध निशितैः शरैः}
{चेदिराट् समरे राजन्नुलूकं समभिद्रवत्}


\twolineshloka
{तथैव शरवर्षेण उलूकं समविद्ध्यत}
{उलूकश्चापि तं बाणैर्निशितैर्लोमवाहिभिः}


\twolineshloka
{तयोर्युद्धं समभवद्धोररूपं विशांते}
{दारयेतां सुसंक्रुद्धावन्योन्यमपराजितौ}


\twolineshloka
{एवं द्वन्द्वसहस्राणि रथवारणवाजिनाम्}
{पदातीनां च समरे तव तेषां च संकुले}


\twolineshloka
{मुहूर्तमिव तद्युद्धमासीन्मधुरदर्शनम्}
{तत उन्मत्तवद्राजन्न प्राज्ञायत किंचन}


\twolineshloka
{गजो गजेन समरे रथिनं च रथी ययौ}
{अश्वोऽश्वं समभिप्रायात्पदातिश्च पदातिनम्}


\twolineshloka
{ततो युद्धं सुदुर्धर्षं व्याकुलं समपद्यत}
{शूराणां समरे तत्र समासाद्येतरेतरम्}


\twolineshloka
{तत्र देवर्षयः सिद्धाश्चारणाश्च समागताः}
{प्रेक्षन्त तद्रणं घोरं देवासुरसमं भुवि}


\twolineshloka
{ततो दन्तिसहस्राणि र्थानां चापि मारिष}
{अश्वौघाः पुरुषौघाश्च विपरीतं समाययुः}


\twolineshloka
{तत्रतत्र प्रदृश्यन्ते रथवारणपत्तयः}
{सादिनश्च नरव्याघ्र युध्यमाना मुहुर्मुहुः}


\chapter{अध्यायः ४६}
\twolineshloka
{सञ्जय उवाच}
{}


\twolineshloka
{राजञ्शतसहस्राणि तत्रतत्र पदातिनाम्}
{निर्मर्यादं प्रयुद्धानि तत्ते वक्ष्यामि भारत}


\twolineshloka
{न पुत्रः पितरं जज्ञे पिता वा पुत्रमौरसम्}
{न भ्राता भ्रातरं तत्र स्वस्त्रीयं न च मातुलः}


\twolineshloka
{न मातुलं च स्वस्रीयो न सखायं सखा तथा}
{आविष्टा इव युध्यन्ते पाण्डवाः कुरुभिः सह}


\twolineshloka
{रथानीकं नरव्याघ्राः केचिदभ्यपतन्रथैः}
{अभज्यन्त युगैरेव युगानि भरतर्षभ}


\twolineshloka
{रथेषाश्च रतेषाभिः कूरा रथकूबरैः}
{संगतैः सहिताः केचित्परस्परजिघांसवः}


\twolineshloka
{न शेकुश्चलितुं केचित्सन्निपत्य रथा रथैः}
{प्रभिन्नास्तु महाकायाः सन्निपत्य गजा गजैः}


\twolineshloka
{बहुधा दारयन्क्रुद्धा विषाणैरितरेतरम्}
{सतोरणपताकैश्च वारणा वरवारणैः}


\twolineshloka
{अभिसृत्य महाराज वेगवद्भिर्महागजैः}
{दन्तैरभिहतास्तत्र चुक्रुशुः परमातुराः}


\twolineshloka
{अभिनीताश्च शिक्षाभिस्तोत्राङ्कुशसमाहताः}
{अप्रभिन्नाः प्रभिन्नानां संमुखाभिमुखा ययुः}


\twolineshloka
{प्रभिन्नैरपि संसक्ताः केचित्तत्र महागजाः}
{क्रौञ्चवन्निनदं कृत्वा दुद्रुवुः सर्वतो दिशम्}


\twolineshloka
{सम्यक्प्रणीता नागाश्च प्रभिन्नकरटामुखाः}
{ऋषितोमरनाराचैर्निर्विद्धा वरवारणाः}


\twolineshloka
{प्रणेदुर्भिन्नमर्माणो निपेतुश्च गतासवः}
{प्राद्रवन्त दिशः केचिन्नदन्तो भैरवान्रवान्}


\twolineshloka
{गजानां पादरक्षास्तु व्यूढोरस्काः प्रहारिणः}
{ऋष्टिभिश्च धनुर्भिश्च विमलैश्च परश्वथैः}


\threelineshloka
{गदाभिर्मुसलैश्चैव भिन्दिपालैः सतोमरैः}
{आयसैः परिघैश्चैव निस्तिरंशैर्विमलै शितैः}
{}


\twolineshloka
{प्रगृहीतैः सुसंरब्धा द्रवमाणास्ततस्ततः}
{व्यदृश्यन्त महाराज परस्परजिघांसवः}


\twolineshloka
{राजमानाश्च निस्त्रिंशाः संसिक्ता नरशोणितैः}
{प्रत्यदृश्यन्त शूराणामन्योन्यमभिधावताम्}


\twolineshloka
{अवक्षिप्तावधूतानामसीनां वीरबाहुभिः}
{संजज्ञे तुमुलः शब्दः पततां परमर्मसु}


\twolineshloka
{गदामुसलरुग्णानां भिन्नानां च वरासिभिः}
{दन्तिदन्तावभिन्नानां मृदितानां च दन्तिभिः}


\twolineshloka
{तत्र तत्र नरौघाणां क्रोशतामितरेतरम्}
{शुश्रुवुर्दारुणा वाचः प्रेतानामिव भारत}


\twolineshloka
{हयैरपि हयारोहाश्चामरापीडधारिभिः}
{हंसैरिव महावेगैरन्योन्यमभिविद्रुताः}


\twolineshloka
{तैर्विमुक्ता महाप्रासा जाम्बूनदविभूषणाः}
{आशुगा विमलास्तीक्ष्णाः संपेतुर्भुजगोपमाः}


\twolineshloka
{अश्वैरग्न्यजवैः केचिदाप्लुत्य महतो रथात्}
{शिरांस्याददिरे वीरा रथिनामश्वसादिनः}


\twolineshloka
{बहूनपि हयारोहान्भल्लैः सन्नतपर्वभिः}
{रथी जघान संप्राप्य बाणगोचरमागतान्}


\twolineshloka
{नवमेघप्रतीकाशाश्चाक्षिप्य तुरगान्गजाः}
{पादैरेव विमृद्गन्ति मत्ताः कनकभूषणाः}


\twolineshloka
{पाठ्यमानेषु कुम्भेषु पार्श्वेष्वपि च वारणाः}
{प्रासैर्विनिहताः केचिद्विनेदुः परमातुराः}


\twolineshloka
{साश्वारोहान्हयान्कांश्चिदुन्मथ्य वरवारणाः}
{सहसा चिक्षिपुस्तत्र संकुले भैरवे सति}


\twolineshloka
{साश्वारोहान्विषाणाग्रैरुत्क्षिप्य तुरगान्गजाः}
{रथौघानभिमृद्गन्तः सध्वजानभिचक्रमुः}


\twolineshloka
{पुंस्त्वादतिमदत्वाच्च केचित्तत्र महागजाः}
{साश्वारोहान्हयाञ्जघ्नुः करैः सचरणैस्तथा}


\threelineshloka
{अश्वारोहैश्च समरे हस्तिसादिभिरेव च}
{प्रतिमानेषु गात्रेषु पार्श्वष्वभि च वारणान्}
{आशुगा विमलास्तीक्ष्णाः संपेतुर्भुजगोपमाः}


\twolineshloka
{नराश्वकायान्निर्भिद्य लौहानि कवचानि च}
{निपेतुर्विमलाः शक्त्यो वीरबाहुभिरर्पिताः}


\twolineshloka
{महोल्ककाप्रतिमा घोरास्तत्र तत्र विशांपते}
{द्वीपिचर्मावनद्धैश्च व्याघ्रचर्मच्छदैरपि}


\twolineshloka
{विकोशैर्विमलैः खङ्गैरभिजध्नुः परान्रणे}
{अभिप्लुतमभिक्रुद्धमेकपार्श्वावदारितम्}


\twolineshloka
{विदर्शयन्तः संपेतुः खङ्गचर्मपरश्वथैः}
{केचिदाक्षिप्य करिणः साश्वानपि रथान्करैः}


\twolineshloka
{विकर्षन्तो दिशः सर्वाः संपेतुः सर्वशब्दगाः}
{शङ्कुभिर्दारिताः केचित्संभिन्नाश्च परश्वथैः}


\twolineshloka
{हस्तिभिर्मृदिताः केचित्क्षुण्णाश्चान्ये तुरंगमैः}
{रथनेमिनिकृत्ताश्च निकृत्ताश्च परश्वथैः}


\twolineshloka
{व्याक्रोशन्त नरा राजंस्तत्रतत्र स्म बान्धवान्}
{पुत्रानन्ये पितॄनन्ये भ्रातॄंश्च सह बन्धुभिः}


\twolineshloka
{मातुलान्भागिनेयांश्च परानपि च संयुगे}
{विकीर्णान्त्राः सुबहवो भग्नसक्थाश्च भारत}


\twolineshloka
{बाहुभिश्चापरे छिन्नैः पार्श्वेषु च विदारिताः}
{क्रन्दन्तः समदृश्यन्त तृषिता जीवितेप्सवः}


\twolineshloka
{तृषापरिगताः केचिदल्पसत्वा विशांपते}
{भूमौ निपतिताः सङ्ख्ये मृगयांचक्रिरे जलम्}


\twolineshloka
{रुधिरौघपरिक्लिन्नाः क्लिश्यमानाश्च भारत}
{व्यनिन्दन्भृशमात्मानं तव पुत्रांश्च संगतान्}


\threelineshloka
{अपरे क्षत्रिकयाः शूराः कृतवैराः परस्परम्}
{नैव शस्त्रं विमुञ्चन्ति नैव क्रन्दन्ति मारिष}
{तर्जयन्ति च संहृष्टास्तत्रतत्र परस्परम्}


\twolineshloka
{आदश्य दशनैश्चापि क्रोधात्स्वरदनच्छदम्}
{भ्रुकुटीकुटिलैर्वक्रैः प्रेक्षन्ति च परस्परम्}


\twolineshloka
{अपरे क्लिश्यमानास्तु शरार्ता व्रणपीडिताः}
{निष्कूजाः समपद्यन्त दृढसत्वा महाबलाः}


\threelineshloka
{अन्ये च विरथाः शूरा रथमन्यस्य संयुगे}
{प्रार्थयाना निपतिताः संक्षुण्णा वरवारणैः}
{अशोभन्त महाराज सपुष्पा इव किंशुकाः}


\twolineshloka
{संबभूवुरनीकेषु बहवो भैरवस्वनाः}
{वर्तमाने महाभीमे तस्मिन्वीरवरक्षये}


\twolineshloka
{निजघान पिता पुत्रं पुत्रश्च पितरं रणे}
{स्वस्त्रीयो मातुलं चापि स्वस्रीयं चापि मातुलः}


\twolineshloka
{सखा सखायं च तथा संबन्धी बान्धवं तथा}
{एवं युयुधिरे तत्र कुरवः पाण्डवैः सह}


\twolineshloka
{वर्तमाने तथा तस्मिन्निर्मर्यादे भयानके}
{भीष्ममासाद्य पार्थानां वाहिनी समकम्पत}


\threelineshloka
{केतुना पञ्चतारेण तालेन भारतर्षभ}
{राजतेन महाबाहुरुच्छ्रितेन महारथे}
{बभौ भीष्मस्तदा राजंश्चन्द्रमा इव मेरुणा}


\chapter{अध्यायः ४७}
\twolineshloka
{सञ्जय उवाच}
{}


\twolineshloka
{गतपूर्वाह्णभूयिष्ठि तस्मिन्नहनि दारुणे}
{वर्तमाने तथा रौद्रे महावीरवरक्षये}


\twolineshloka
{दुर्मुखः कृतवर्मा च कृपः शल्यो विविंशतिः}
{भीष्मं जुगुपुरासाद्य तव पुत्रेण चोदिताः}


\twolineshloka
{एतैरतिरथैर्गुप्तः पञ्चभिर्भरतर्षभः}
{पाण्डवानामनीकानि विजगाहे महारथः}


\twolineshloka
{चेदिकाशिकरूषेषु पञ्चालेषु च भारत}
{भीष्मस्य बहुधा तालश्चलत्केतुरदृश्यत}


\twolineshloka
{स शिरांसि रणेऽरीणां रथांश्च सयुगध्वजान्}
{निचकर्त महावेगैर्भल्लैः सन्नतपर्वभिः}


\twolineshloka
{नृत्यतो रथमार्गेषु भीष्मस्य भरतर्षभ}
{भृशमार्तस्वरं चक्रुर्नागा मर्मणि ताडिताः}


\twolineshloka
{अभिमन्युः सुसंक्रुद्धः पिशह्गैस्तुरगोत्तमैः}
{संयुक्तं रथमास्थाय प्रायाद्भीष्मरथं प्रति}


\twolineshloka
{जाम्बूनदविचित्रेण कर्णिकारेण केतुना}
{अभ्यवर्तत भीष्मं च तांश्चैव रथसत्तमान्}


\twolineshloka
{स तालकेतोस्तीक्ष्णेन केतुमाहत्य पत्रिणा}
{भीष्मेण युयुधे वीरस्तस्य चानुरथैः सह}


\twolineshloka
{कृतवर्माणमेकेन शल्यं पञ्चभिराशुगैः}
{विद्ध्वा नवभिरानर्च्छच्छिताग्रैः प्रतिपामहम्}


\twolineshloka
{पूर्णायतविसृष्टेन सम्यक्प्रणिहितेन च}
{ध्वजमेकेन विव्याध जाम्बूनदपरिष्कृतम्}


\twolineshloka
{दुर्मुखस्य तु भल्लेन सर्वावरणभेदिना}
{जहार सारथेः कायाच्छिरः सन्नतपर्वणा}


\twolineshloka
{धनुश्चिच्छेद भल्लेन कार्तस्वरविभूषितम्}
{कृपस्य निशिताग्रेण तांश्च तीक्ष्णमुखैः शरैः}


\twolineshloka
{जघान परमक्रुद्धो नृत्यन्निव महारथः}
{तस्य लाघवमुद्वीक्ष्य तुतुषुर्देवता अपि}


\twolineshloka
{लब्धलक्षतया कार्ष्णेः सर्वे भीष्ममुखा रथाः}
{सत्ववन्तममन्यन्त साक्षादिव धनंजयम्}


\twolineshloka
{तस्य लाघवमार्गस्थमलातसदृशप्रभम्}
{दिशः पर्यपतच्चापं गाण्डीवमिव घोषवत्}


\twolineshloka
{तमासाद्य महावेगैर्भीष्मो नव्रभिराशुगैः}
{विव्याध समरे तूर्णमार्जुनिं परवीरहा}


\twolineshloka
{ध्वजं चास्य त्रिभिर्भल्लैश्चिच्छेद परमौजसः}
{सारथिं च त्रिभिर्बाणैराजघान यतव्रतः}


\twolineshloka
{तथैव कृतवर्मा च कृपः शल्यश्च मारिष}
{विद्ध्वा नाकम्पयत्कार्ष्णिं मैनाकमिव पर्वतम्}


\twolineshloka
{स तैः परिवृतः शूरो धार्तराष्ट्रैर्महारथैः}
{ववर्ष शरवर्षाणि कार्ष्णिः पञ्चरथान्प्रति}


\twolineshloka
{ततस्तेषां सहस्राणि संवार्य शरवृष्टिभिः}
{ननाद बलवान्कार्ष्णिर्भीष्माय विसृज्यञ्शरान्}


\twolineshloka
{तत्रास्य सुमहद्राजन्बाह्वोर्बलमदृश्यत}
{यतमानस्य समरे भीष्ममर्दयतः शरैः}


\twolineshloka
{पराक्रान्तस्य तस्यैव भीष्मोऽपि प्राहिणोच्छरान्}
{स तांश्चिच्छेद समरे भीष्मचापच्युताञ्शरान्}


\twolineshloka
{ततो ध्वजमामोघेंषुर्भीष्मस्य नवभिः शरैः}
{चिच्छेद समरे वीरस्तत उच्चुक्रुशुर्जनाः}


\twolineshloka
{स राजतो महास्कन्धस्तालो हेमविभूषितः}
{सौभद्रविशिखैश्छिन्नः पपात भुवि भारत}


\twolineshloka
{तं तु सौभद्रविशिखैः पातितं भरतर्षभ}
{दृष्ट्वा भीमो ननादोच्चैः सौभद्रमभिहर्षयन्}


\twolineshloka
{अथ भीष्मो महास्राणि दिव्यानि सुबहूनि च}
{प्रादुश्चक्रे महारौद्रे रणे तस्मिन्महाबलः}


\twolineshloka
{ततः शरसहस्रेण सौभद्रं प्रतितामहः}
{अवाकिरदमेयात्मा तदद्भुतमिवाभवत्}


\twolineshloka
{ततो दश महेष्वासाः पाण्डवानां महारथाः}
{रक्षार्थमभ्यधावन्त सौभद्रं त्वरिता रथैः}


\twolineshloka
{विराटः सह पुत्रेण धृष्टद्युम्नश्च पार्षतः}
{भीमश्च केकयाश्चैव सात्यकिश्च विशांपते}


\threelineshloka
{तेषां जवेनापततां भीष्मः शान्तनवो रणे}
{पाञ्चाल्यं त्रिभिरानर्च्छत्सात्यकिं नवभिः शरैः}
{}


\twolineshloka
{पूर्णायतविसृष्टेन क्षुरेण निशितेन च}
{ध्वजमेकेन चिच्छेद भीमसेनस्य पत्रिणा}


\twolineshloka
{जाम्बूनदमयः श्रीमान्केसरी स नरोत्तम}
{पपात भीमसेनस्य भीष्मेण मथितो रथात्}


\twolineshloka
{ततो भीमस्त्रिभिर्विद्ध्वा भीष्मं शान्तनवं रणे}
{कृपमेकेन विव्याध कृतवर्माणमष्टभिः}


\twolineshloka
{प्रगृहीताग्रहस्तेकन वैराटिरपि दन्तिना}
{अभ्यद्रवत राजानं मद्राधिपतिमुत्तरः}


\threelineshloka
{तस्य वारणराजस्य जवेनापततो रथे}
{शल्यो निवारयामास वेगमप्रतिमं शरैः}
{}


\twolineshloka
{तस्य क्रुद्धः स नागेन्द्रो बृहतः साधुवाहिनः}
{पदा युगमधिष्ठाय जघान चतुरो हयान्}


\twolineshloka
{स हताश्वे रथे तिष्ठन्मद्राधिपतिरायसीम्}
{उत्तरान्तकरीं शक्तिं चिक्षेप भुजगोपमाम्}


\twolineshloka
{तया भिन्नतनुत्राणः प्रविश्य विपुलं तमः}
{स पपात गजस्कन्धात्परमुक्ताङ्कुशतोमरः}


\twolineshloka
{असिमादाय शल्योऽपि अवप्लुत्य रथोत्तमात्}
{तस्य वारणराजस्य चिच्छेदाथ महाकरम्}


\twolineshloka
{भिन्नमर्मा शरशतैश्छिन्नहस्तः सवारणः}
{भीममार्तस्वरं कृत्वा पपात च ममार च}


\twolineshloka
{एतदीदृशकं कृत्वा मद्रराजो नराधिप}
{आरुरोह रथं तूर्णं भास्वरं कृतवर्मणः}


\twolineshloka
{उत्तरं वै हतं दृष्ट्वा वैराटिर्भ्रातरं तदा}
{कृतवर्मणा च सहितं दृष्ट्वा शल्यमवस्थितम्}


\twolineshloka
{श्वेतः क्रोधात्प्रजज्वाल हविषा हव्यवाडिव}
{स विस्फार्य महच्चापं शक्रचापोपमं बली}


\twolineshloka
{अभ्यधाव़ञ्जिघांसन्वै शल्यं मद्राधिपं बली}
{महता रथवंशेन समन्तात्परिवारितः}


\twolineshloka
{मुञ्चन्बाणमयं वर्षं प्रायाच्छल्यरथं प्रति}
{तमापतन्तं संप्रेक्ष्य मत्तवारणविक्रमम्}


\twolineshloka
{तावकानां रथाः सप्त समन्तात्पर्यवारयन्}
{मद्रराजमभीप्सन्तो मृत्योर्दंष्ट्रान्तरं गतम्}


\twolineshloka
{बृहद्बलश्च कौसल्यो जयत्सेनश्च मागधः}
{तथा रुक्मरथो राजञ्शल्यपुत्रः प्रतापवान्}


\twolineshloka
{विन्दानुविन्दावावान्त्यौ काम्भोजश्च सुदक्षिणः}
{बृहत्क्षत्रस्य दायादः सैन्धवश्च जयद्रथः}


\twolineshloka
{नानावर्णविचित्राणि धनूंषि च महात्मनाम्}
{विस्फारितानि दृश्यन्ते तोयदेष्विव विद्युतः}


\twolineshloka
{ते तु बाणमयं वर्षं श्वेतमूर्धन्यपातयन्}
{निदाघान्तेऽनिलोद्धूता मेघा इव नगे जलम्}


\twolineshloka
{ततः क्रुद्धो महेष्वासः सप्तभल्लैः सुतेजनैः}
{धनूंषि तेषामाच्छिद्य ममर्द पृतनापतिः}


\twolineshloka
{निकृत्तान्येव तानि स्म समदृश्यन्त भारत}
{ततस्ते तु निमेषार्धात्प्रत्यपद्यन्धनूंषि च}


\threelineshloka
{सप्त चैव पृषत्कांश्च श्वेतस्योपर्यपातयन्}
{ततः पुनरमेयात्मा भल्लैः सप्तभिराशुगैः}
{निचकर्त महाबाहुस्तेषां चापानि धन्विनाम्}


\twolineshloka
{ते निकृत्तमहाचापास्त्वरमाणा महारथाः}
{रथशक्तीः परामृश्य विनेदुर्भैरवान्रवान्}


\twolineshloka
{अन्वयुर्भरतश्रेष्ठ सप्त श्वेतरथं प्रति}
{ततस्ता ज्वलिताः सप्त महेन्द्राशनिनिःस्वनाः}


\twolineshloka
{अप्राप्ताः सप्तभिर्भल्लैश्चिच्छेद परमास्त्रवित्}
{6-47-57bततःसमादाय शरं सर्वकायविदारणम्}


\twolineshloka
{प्राहिणोद्भरतश्रेष्ठ श्वेतो रुक्मरथं प्रति}
{तस्य देहे निपतितो बाणो वज्रातिगो महान्}


\twolineshloka
{ततो रुक्मरथो राजन्सायकेन दृढाहतः}
{निषसाद रथोपस्थे कश्मलं चाविशन्महत्}


\twolineshloka
{तं विसंज्ञं विमनसं त्वरमाणस्तु सारथिः}
{अपोवाह नसंभ्रान्तः सर्वलोकस्य पश्यतः}


\twolineshloka
{ततोऽन्यान्षट् समादाय श्वेतो हेमविभूषितान्}
{तेषां षण्मां महाबाहुर्ध्वजशीर्षण्यपातयत्}


\twolineshloka
{हयांश्च तेषां निर्भिद्य सारथींश्च परंतप}
{शरैश्चैतान्समाकीर्य प्रायाच्छल्यरथं प्रति}


\twolineshloka
{ततो हलहलाशब्दस्तव सैन्येषु भारत}
{दृष्ट्वा सेनापतिं तूर्णं यान्तं शल्यरथं प्रति}


\twolineshloka
{ततो भीष्मं पुरस्कृत्य तव पुत्रो महाबलः}
{वृतस्तु सर्वसैन्येन प्रायाच्छ्वेतरथं प्रति}


\twolineshloka
{मृत्योरास्यमनुप्राप्तं मद्रराजममोचयत्}
{ततो युद्धं समभवत्तुमुलं रोमहर्षणम्}


\twolineshloka
{तावकानां परेषां च व्यतिषक्तरथद्विपम्}
{सौभद्रे भीमसेन च सात्यकौ च महारथे}


\threelineshloka
{कैकेये च विराटे च धृष्टद्युम्ने च पार्षते}
{एतेषु नरसिंहेषु चेदिमत्स्येषु चैव ह}
{ववर्ष शरवर्षाणि कुरुवृद्धः पितामहः}


\chapter{अध्यायः ४८}
\twolineshloka
{धृतराष्ट्र उवाच}
{}


\twolineshloka
{एवं श्वेते महेष्वासे प्राप्ते शल्यरथं प्रति}
{कुरवः पाण्डवेयाश्च किमकुर्वत सञ्जय}


\threelineshloka
{भीष्मः शान्तनवः किं वा तन्ममाचक्ष्व पृच्छतः}
{सञ्जय उवाच}
{राजञ्शतसहस्राणि ततः क्षत्रियपुङ्गवाः}


\twolineshloka
{श्वेतं सेनापतिं शूरं पुरस्कृत्य महारथाः}
{राज्ञो बलं दर्शयन्तस्तव पुत्रस्य भारत}


\twolineshloka
{शिखण्डिनं पुरस्कृत्य त्रातुमैच्छन्महारथाः}
{अभ्यवर्तन्त भीष्मस्य रथं हेमपरिष्कृतम्}


\twolineshloka
{जिघांसन्तं युधांश्रेष्ठं तदासीत्तुमुलं महत्}
{तत्तेऽहं संप्रवक्ष्यामि महावैशसमच्युत}


\twolineshloka
{तावकानां परेषां च यथा युद्धमवर्तत}
{तत्राकरोद्रथोपस्थाञ्शून्याञ्शान्तनवो बहून्}


\twolineshloka
{तत्राद्भुतं महच्चक्रे शरैरार्च्छद्रथोत्तमान्}
{समावृणोच्छरैरर्कमर्कतुल्यप्रतापवान्}


\twolineshloka
{नुदन्समन्तात्समरे रविरुद्यन्यथा तमः}
{तेनाजौ प्रेषिता राजञ्शराः शतसहस्रशः}


\twolineshloka
{क्षत्रियान्तकराः सङ्ख्ये महावेगा महाबलाः}
{शिरांसि पातयामासुर्वीराणां शतशो रणे}


\twolineshloka
{गजान्कण्टकसन्नाहान्वज्रेणेव शिलोच्चयान्}
{रथा रथेषु संसक्ता व्यदृश्यन्त विशांपते}


\twolineshloka
{एके रथं पर्यवहंस्तुरगाः सतुरंगमम्}
{युवानं निहतं वीरं लम्बमानं सकार्मुकम्}


\twolineshloka
{उदीर्णाश्च हया राजन्वहनक्तस्तत्रतत्र ह}
{बद्धस्वङ्गनिषङ्गाश्च विध्वस्तशिरसो हताः}


\twolineshloka
{शतशः पतिता भूमौ वीरशय्यासु शेरते}
{परस्पेरण धावन्तः पतिताः पुनरुत्थिताः}


\twolineshloka
{उत्थाय च प्रधावन्तो द्वन्द्वयुद्धमवाप्नुवन्}
{पीडिताः पुनरन्योन्यं लुठन्तो रणमूर्धनि}


\twolineshloka
{सचापाः सनिषङ्गाश्च जातरूपपरिष्कृताः}
{विस्रब्धहतवीराश्च शतशः परिपीडिताः}


\twolineshloka
{तेनतेनाभ्यधावन्त विसृजन्तश्च भारत}
{मत्तो गजः पर्यवर्तद्धयांश्च हतसादिनः}


\twolineshloka
{सरथा रथिनश्चापि विमृद्गन्तः समन्ततः}
{स्यान्दनादपतत्कश्चिन्निहतोऽन्येन सायकैः}


\twolineshloka
{हतसारथिरप्युच्चैः पपात काष्ठवद्रथः}
{युध्यमानस्य संग्रामे व्यूढे रजसि चोत्थिते}


\twolineshloka
{धनुःकूजितविज्ञानं तत्रासीत्प्रतियुद्ध्यतः}
{गात्रस्पर्शेन योधानां व्यज्ञास्त परिपन्थिनम्}


\twolineshloka
{युद्ध्यमानं शरै राजन्सिञ्जिनीध्वजिनीरवात्}
{अन्योन्यं वीरसंशब्दो नाश्रूयत भटैः कृतः}


\twolineshloka
{शब्दायमाने संग्रामे पटहे कर्णदारिणि}
{युध्यमानस्य संग्रामे कुर्वतः पौरुषं स्वकम्}


\twolineshloka
{नाश्रौषं नामगोत्राणि कीर्तनं च परस्परम्}
{भीष्मचापच्युतैर्बाणैरार्तानां युध्यतां मृधे}


\twolineshloka
{परस्परेषां वीराणां मनांसि समकम्पयन्}
{तस्मिन्नत्याकुले युद्धे दारुणे रोमहर्षणे}


\twolineshloka
{पिता पुत्रं च समरे नाभिजानाति कश्चन}
{चक्रे भग्ने युगे च्छिन्ने एकधुर्ये हये हतः}


\twolineshloka
{आक्षिप्तः स्यन्दनाद्वीरः समारथिरजिह्मगैः}
{एवं च समरे सर्वे वीराश्च विरथीकृताः}


\twolineshloka
{तेन तेन स्म दृश्यन्ते धावमानाः समंततः}
{गजो हतः शिरश्छिन्नं मर्म भिन्नं हयो हतः}


\twolineshloka
{अहतः कोपि नैवासीद्भीष्मे नघ्नति शात्रवान्}
{श्वेतः कुरूणामकरोत्क्षयं तस्मिन्महाहवे}


\twolineshloka
{राजपुत्रान्रथोदारानवधीच्छतसङ्घशः}
{चिच्छेद रथिनां बाणैः शिरांसि भरतर्षभ}


\twolineshloka
{साङ्गदा बाहवश्चैव धनूंषि च समंततः}
{रथेषां रथचक्राणि तूणीराणि युगानि च}


\twolineshloka
{छत्राणि च महार्हाणि पताकाश्च विशांपते}
{हयैघाश्च रथौघाश्च नरौघाश्चैव भारत}


\twolineshloka
{वारणाः शतशश्चैव हताः श्वेतेन भारत}
{वयं श्वेतभयाद्भीता विहाय रथसत्तमम्}


\twolineshloka
{अपयातास्तथा पश्चाद्विभुं पश्याम धृष्णवः}
{शरपातमतिक्रम्य कुरवः कुरुनन्दन}


\twolineshloka
{भीष्मं शान्तनवं युद्धे स्थिताः पश्याम सर्वशः}
{अदीनो दीनसमये भीष्मोऽस्माकं महाहवे}


\twolineshloka
{एकस्तस्थौ नरव्याघ्रो गिरिर्मेरुरिवाचलः}
{आददान इव प्राणान्सविता शिशिरात्यये}


\twolineshloka
{गभस्तिभिरिवादित्यस्तस्थौ शरमरीचिमान्}
{स मुमोच महेष्वासः शरसङ्घाननेकशः}


\twolineshloka
{निघ्नन्नमित्रान्समरे वज्रपाणिरिवासुरान्}
{ते वध्यमाना भीष्मेण प्रजहुस्तं महाबलम्}


\twolineshloka
{स्वयूथादिव ते यूथान्मुक्तं भूमिषु दारुणम्}
{तमेवमुपलक्ष्यैको हृष्टः पुष्टः परंतप}


\twolineshloka
{दुर्योधनप्रिये युक्तः पाण्डवान्परिशोचयन्}
{जीवितं दुस्त्यजं त्यक्त्वा भयं च सुमहाहवे}


\twolineshloka
{पातयामास सैन्यानि पाण्डवानां विशांपते}
{प्रहरन्तमनीकानि पिता देवव्रतस्तव}


\twolineshloka
{दृष्ट्वा सेनापतिं भीष्मस्त्वरितः श्वेतमभ्ययात्}
{स भीष्मं शरजालेन महता समवाकिरत्}


\twolineshloka
{श्वेतं चापि तथा भीष्मः शरौघैः समवाकिरत्}
{तौ वृषाविव नर्दन्तौ मत्ताविव महाद्विपौ}


\twolineshloka
{व्याघ्राविव सुसंरब्धावन्योन्यमभिजघ्नतुः}
{अस्त्रैरस्त्राणि संवार्य ततस्तौ पुरुषर्षभौ}


\twolineshloka
{भीष्मः श्वेतश्च युयुधे परस्परवधैषिणौ}
{एकाह्ना निर्दहेद्भीष्मः पाम्डवानामनीकिनीम्}


\twolineshloka
{शरैः परमसंक्रुद्धो यदि श्वेतो न पालयेत्}
{पितामहं ततो दृष्ट्वा श्वेतो विमुखीकृतम्}


\twolineshloka
{प्रहर्षं पाण्डवा जग्मुः पुत्रस्ते विमनाभवत्}
{ततो दुर्योधनः क्रुद्धः पार्थिवैः परिवारितः}


\twolineshloka
{ससैन्यः पाण्डवानीकमभ्यद्रवत संयुगे}
{दुर्मुखः कृतवर्मा च कृपः शल्यो विशांपतिःक}


\twolineshloka
{भीष्मं जुगुपुरासाद्य तव पुत्रेण नोदिताः}
{दृष्ट्वा तु पार्थिवैः सर्वैर्दुर्योधनपुरोगमैः}


\twolineshloka
{पाण्डवानामनीकानि वध्यमानानि संयुगे}
{श्वेतो गाङ्गेयमुत्सृज्य तव पुत्रस्य वाहिनीम्}


\twolineshloka
{नाशयामास वेगेन वायुर्वृक्षानिवौजसा}
{द्रावयित्वा चमूं राजन्वैराटिः क्रोधमूर्च्छितः}


\twolineshloka
{आपतत्सहसा भूयो यत्र भीष्मो व्यवस्थितः}
{तौ तत्रोपगतौ राजञ्शरदीप्तौ महाबलौ}


\twolineshloka
{अयुध्येतां महात्मानौ यथोभौ वृत्रवासवौ}
{अन्योन्यं तु महाराज परस्परवधैषिणौ}


\twolineshloka
{निगृह्य कार्मुकं श्वेतो भीष्मं विव्याध सप्तभिः}
{पराक्रमं ततस्तस्य पराक्रम्य पराक्रमी}


\twolineshloka
{तरसा वारयामास मत्तो मत्तमिव द्विपम्}
{श्वेतः शान्तनवं भूयः शरैः सन्नतपर्वभिः}


\twolineshloka
{विव्याध प़ञ्चविंशत्या तदद्भुतमिवाभवत्}
{तं प्रत्यविध्यद्दशभिर्भीष्मः शान्तनवस्तदा}


\twolineshloka
{स विद्धस्तेन बलवान्नाकम्पत यथाऽचलः}
{वैराटिः समरे क्रुद्धो भृशमायम्य कार्मुकम्}


\twolineshloka
{आजघान ततो भीष्मं श्वेतः क्षत्रियनन्दनः}
{संप्रहस्य तत श्वेतः सृक्किणी परिसंलिहन्}


\twolineshloka
{घनुश्चिच्छेद भीष्मस्य नवभिर्दशधा शरैः}
{संधाय विशिखं चैव शरं लोमप्रवाहिनम्}


\twolineshloka
{उन्ममाथ ततस्तालं ध्वजशीर्षं महात्मनः}
{केतुं निपतितं दृष्ट्वा भीष्मस्य तनयास्तव}


\twolineshloka
{हतं भीष्मममन्यन्त श्वेतस्य वशमागतम्}
{पाण्डवाश्चापि संहृष्टा दध्मुः शङ्खान्मुदा युताः}


\twolineshloka
{भीष्मस्य पतितं केतुं दृष्ट्वा तालं महात्मनः}
{ततो दुर्योधनः क्रोधात्स्वमनीकमनोदयत्}


\twolineshloka
{यत्ता भीष्मं परीप्सध्वं रक्षमाणाः समंततः}
{मा नः प्रपश्यमानानां श्वेतान्मृत्युमवाप्स्यति}


\twolineshloka
{भीष्मः शान्तनवः शूरस्तथा सत्यं ब्रवीमि वः}
{राज्ञस्तु वचनं श्रत्वा त्वरमाणा महारथाः}


\twolineshloka
{बलेन चतुरङ्गेण गाङ्गेयप्रन्वपालयन्}
{बाह्लीकः कृतवर्मा च शलः शल्यश्च भारत}


\twolineshloka
{जलसन्धो विकर्णश्च चित्रसेनो विविंशतिः}
{त्वरमाणास्त्वराकाले परिवार्य समंततः}


\twolineshloka
{शस्त्रवृष्टिं सुतुमुलां श्वेतस्योपर्यपातयन्}
{तान्क्रुद्धो निशितैर्बाणैस्त्वरमाणो महारथः}


\twolineshloka
{अवारयदमेयात्मा दर्शयन्पाणिलाघवम्}
{स निवार्य तु तान्सर्वान्केसरी कुञ्जराविव}


\twolineshloka
{महता शरवर्षेण भीष्मस्य धनुराच्छिनत्}
{ततोऽन्यद्धनुरादाय भीष्मः शान्तनवो युधि}


\twolineshloka
{श्वेतं विव्याध राजेन्द्र कङ्कपत्रैः शितैः शरैः}
{ततः सेनापतिः क्रुद्धो भीष्मं बहुभिरायसैः}


\twolineshloka
{विव्याध समरे राजन्सर्वलोकस्य पश्यतः}
{ततः प्रव्यथितो राजा भीष्मं दृष्ट्वा निवारितम्}


\twolineshloka
{प्रवीरं सर्वलोकस्य श्वेतेन युधि वै तदा}
{निष्ठानकश्च सुमहांस्तव सैन्यस्य चाभवत्}


\twolineshloka
{तं वीरं वारितं दृष्ट्वा श्वेतेन शरविक्षतम्}
{हतं श्वेतेन मन्यन्ते श्वेतस्य वशमागतम्}


\twolineshloka
{ततः क्रोधवशं प्राप्तः पिता देवव्रतस्तव}
{ध्वजमुन्मतितं दृष्ट्वा तां च सेनां निवारिताम्}


\twolineshloka
{श्वेतं प्रति महाराज व्यसृजत्प्तायकान्बहून्}
{तानावार्य रणे श्वेतो भीष्मस्य रथिनां वरः}


\twolineshloka
{धनुश्चिच्छेद भल्लेन पुनरेव पितुस्तव}
{उत्सृज्य कार्मुकं राजन्गाङ्गेयः क्रोधमूर्च्छितः}


\twolineshloka
{अन्यत्कार्मुकमादाय विपुलं बलवत्तरम्}
{तत्र संधाय विपुलान्भल्लान्सप्त शिलाशितान्}


\twolineshloka
{चतुर्भिश्च जघानाश्वाञ्श्वेतस्य पृतनापतेः}
{ध्वजं द्वाभ्यां तु चिच्छेद सप्तमेन च सारथे}


\twolineshloka
{शिरश्चिच्छेद भल्लेन संक्रुद्धोऽलघुविक्रमः}
{हताश्वसूतात्सु रथादवप्लुत्य महाबलः}


\twolineshloka
{अमर्षवशमापन्नो व्याकुलः समपद्यत}
{विरथं रथिनां श्रेष्ठं श्वेतं दृष्ट्वा पितामहः}


\twolineshloka
{ताडयामास निशितैः शरसङ्घैः समंततः}
{स ताड्यमानः समरे भीष्मचापच्युतैः शरैः}


\twolineshloka
{स्वरथे धनुरुत्सृज्य शक्तिं जग्राह काञ्चनीम्}
{ततः शक्तिं रणे श्वेतो जग्राहोग्रं महाभयाम्}


\twolineshloka
{कालदण्डोपमां घोरां मृत्योर्जिह्वामेव श्वसन्}
{अब्रवीच्च तदा श्वेतो भीष्मं शान्तनवं रणे}


\twolineshloka
{तिष्ठेदानीं सुसंरब्धः पश्य मां पुरुषो भव}
{एवमुक्त्वा महेष्वासो भीष्मं युधि पराक्रमी}


\twolineshloka
{ततः शक्तिममेयात्मा चिक्षेप भुजगोपमाम्}
{पाण्डवार्थे पराक्रान्तस्तवानर्थं चिकीर्षुकः}


\twolineshloka
{हाहाकारो महानासीत्पुत्राणां ते विशांपते}
{दृष्ट्वा शक्तिं महाघोरां मृत्योर्दण्डसमप्रभाम्}


\twolineshloka
{श्वेतस्य करनिर्मुक्तां निर्मुक्तोरगसन्निभाम्}
{अपतत्सहसा राजन्महोल्केव नभस्तलात्}


\twolineshloka
{ज्वलन्तीमन्तरिक्षे तां ज्वालाभिरिव संवृताम्}
{असंभ्रान्तस्तदा राजन्पिता देवव्रतस्तव}


\twolineshloka
{अष्टभिर्नवभिर्भीष्मः शक्तिं चिच्छेद पत्रिभिः}
{उत्कृष्टहेमविकृतां निकृतां निशितैः शरैः}


\twolineshloka
{उच्चुक्रुशुस्ततः सर्वे तावका भरतर्षभ}
{शक्तिं विनिहतां दृष्ट्वा वैराटिः क्रोधमूर्च्छितः}


\twolineshloka
{कालोपहतचेतास्तु कर्तवक्यं नाभ्यजानत}
{क्रोधसंमूर्च्छितो राजन्वैराटिः प्रहसन्निव}


\twolineshloka
{गदां जग्राह संहृष्टो भीष्मस्य निधनं प्रति}
{क्रोधेन रक्तनयनो दण्डपाणिरिवान्तकः}


\twolineshloka
{भीष्मं समिदुद्राव जलौघ इव पर्वतम्}
{तस्य वेगमसंवार्यं मत्वा भीष्मः प्रतापवान्}


\twolineshloka
{प्रहारविप्रमोक्षार्थं सहसा धरणींगतः}
{श्वेतः क्रोधसमाविष्टो भ्रामयित्वा तु तां गदाम्}


\twolineshloka
{रथे भीष्मस्य चिक्षेप यथा देवो धनेश्वरः}
{तया भीष्मनिपातिन्या स रथो भस्मसात्कृतः}


\twolineshloka
{सध्वजः सह सूतेन साश्वः सयुगबन्धुरः}
{विरथं रथिनां श्रेष्ठं भीष्मं दृष्ट्वा रथोत्तमकाः}


\twolineshloka
{अभ्यधावन्त सहिताः शल्यप्रभृतयो रथाः}
{ततोऽन्यं रथमास्थाय धनुर्विस्फार्य दुर्मनाः}


\twolineshloka
{शनकैरभ्यायाच्छ्वेतं गाङ्गेयः प्रहसन्निव}
{एतस्मिन्नन्तरे भीष्मः शुश्राव विपुलां गिरम्}


\twolineshloka
{आकाशादीरितां दिव्यामात्मनो हितसंभवाम्}
{भीष्मभीष्म महाबाहो शीघ्रं यत्नं कुरुष्व वै}


\twolineshloka
{एष ह्यस्य जये कालो निर्दिष्टो विश्वयोनिना}
{एतच्छ्रुत्वा तु वचनं देवदूतेन भाषितम्}


\twolineshloka
{संप्रहृष्टमना भूत्वा वधे तस्य मनो दधे}
{विरथं रथिनां श्रेष्ठं श्वेतं दृष्ट्वा पदातिनम्}


\twolineshloka
{सहितास्त्वभ्यवर्तन्त परीप्सन्तो महारथाः}
{सात्यकिर्भीमसेनश्च धृष्टद्युम्नश्च पार्षतः}


\twolineshloka
{कैकेयो धृष्टकेतुश्च अभिमन्युश्च वीर्यवान्}
{एतानापततः सर्वान्द्रोणशल्यकृपैः सह}


\twolineshloka
{अवारयदमेयात्मा वारिवेगानिवाचलः}
{स निरुद्धेषु सर्वेषु पाण्डवेषु महात्मसु}


\twolineshloka
{श्वेतः खङ्गमथाकृष्य भीष्मस्य धनुराच्छिनात्}
{तदपास्य धनुश्छिन्नं त्वरमाणः पितामहः}


\twolineshloka
{देवदूतवचः श्रुत्वा वधे तस्य मनो दधे}
{ततः प्रचरमाणस्तु पिता देवव्रतस्तव}


\twolineshloka
{अन्यत्कार्मुकमादाय त्वरमाणो महारथः}
{क्षणेन सज्यमकरोच्छक्रचापसमप्रभम्}


\twolineshloka
{पिता ते भरतश्रेष्ठ श्वेतं दृष्ट्वा महारथैः}
{वृतं तं मनुजव्याघ्रैर्भीमसेनपुरोगमैःक}


\twolineshloka
{अभ्यवर्तत गाङ्गेयः श्वेतं सेनापतिं द्रुतम्}
{आपतन्तं ततो भीष्मो भीमसेनं प्रतापवान्}


\twolineshloka
{आजघ्ने विशिखैः षष्ट्या सेनान्यं स महारथः}
{अभिमन्युं च समरे पिता देवव्रतस्तव}


\threelineshloka
{आजघ्ने भरतश्रेष्ठस्त्रिभिः सन्नतपर्वभिः}
{सात्यकिं च शतेनाजौ भरतानां पितामहः ॥ 6-48-110aधृष्टद्युम्नंच विंशत्या कैकेयं चापि पञ्चभिः}
{तांश्च सर्वान्महेष्वासान्पिता देवव्रतस्तव}


\twolineshloka
{वारयित्वा शरैर्घोरैः श्वेतमेवाभिदुद्रुवे}
{ततः शरं मृत्युसमं भारसाधनमुत्तमम्}


\twolineshloka
{विकृष्य बलवान्भीष्मः समाधत्त दुरासदम्}
{ब्रह्मास्त्रेण सुसंयुक्तं तं शरं लोमवाहिनम्}


\twolineshloka
{ददृशुर्देवगन्धर्वाः पिशाचोरगराक्षसाः}
{स तस्य कवचं भित्वा हृदयं चामितौजसः}


\twolineshloka
{जगाम धरणीं बाणो महाशनिरिव ज्वलन्}
{अस्तं गच्छन्यथाऽऽदित्यः प्रभाप्रादाय सत्वरः}


\twolineshloka
{एवं जीवितमादाय श्वेतदेहाञ्जगाम ह}
{तं भीष्मेण नरव्याघ्रं तथा विनिहतं युधि}


\twolineshloka
{प्रपतन्तमपश्याम गिरेः शृङ्गमिव च्युतम्}
{अशोचन्पाण्डवास्तत्र क्षत्रियाश्च महारथाः}


\twolineshloka
{प्रहृष्टाश्च सुतास्तुभ्यं कुरवश्चापि सर्वशः}
{6-48-117bततोदुःशासनो राजञ्श्वेतं दृष्ट्वा निपातितम्}


\twolineshloka
{वादित्रनिनदैर्घोरैर्नृत्यति स्म समंततः}
{तस्मिन्हते महेष्वासे भीष्मेणाहवशोभिना}


\twolineshloka
{प्रावेपन्त महेष्वासाः शिखण्डिप्रमुखा रथाः}
{ततो धनंजयो राजन्वार्ष्णेयश्चापि सर्वशः}


\twolineshloka
{अवहारं शनैश्चक्रुर्निहते वाहिनीपतौ}
{ततोऽवहारः सैन्यानां तव तेषां च भारत}


\threelineshloka
{तावकानां परेषां च नर्दतां च मुहुर्मुहुः}
{पार्था विमनसो भूत्वा न्यवर्तन्त महारथाः}
{चिन्तयन्तो वधं घोरं द्वैरथेन परंतपाः}


\chapter{अध्यायः ४९}
\twolineshloka
{धृतराष्ट्र उवाच}
{}


\twolineshloka
{( श्वेते सेनापतौ तात संग्रमे निहते परैः}
{किमकुर्वन्महेष्वासाः पाञ्चालाः पाण्डवैः सह}


\twolineshloka
{सेनापतिं समाकर्ण्य श्वेतं युधि निपातितम्}
{तदर्थं यततां चापि परेषां प्रपलायिनाम्}


\twolineshloka
{मनः प्रीणाति मे वाक्यं जयं संजय श्रृण्वतः}
{प्रत्युपायं चिन्तयन्तः सञ्जनाः प्रस्रवन्ति मे}


\twolineshloka
{स हि वीरोऽनुरक्तश्च वृद्धः कुरुपतिस्तदा}
{कृतं वैरं सदा तेन पितुः पुत्रेण धीमता}


\twolineshloka
{तस्योद्वेगभयाच्चापि संश्रितः पाण्डवान्पुरा}
{सर्वं बलं परित्यज्य दुर्गं संश्रित्य तिष्ठति}


\twolineshloka
{पाण्डवानां प्रतापेन दुर्गं देशं निवेश्य च}
{सपत्नान्सततं बाधन्नार्यवृत्तिमनुष्ठितः}


\twolineshloka
{आश्चर्यं वै सदा तेषां पुरा राज्ञां सुदुर्मतिः}
{ततो युधिष्ठिरे भक्तः कथं सञ्जय सूदितः ॥)}


\twolineshloka
{प्रक्षिप्तः संमतः क्षुद्रः पुत्रो मे पुरुषाधमः}
{न युद्धं रोचयेद्भीष्मो न चाचार्यः कथंचन}


\twolineshloka
{न कृपो न च गान्धारी नाहं सञ्जय रोचये}
{न वासुदेवो वार्ष्णेयो धर्मराजश्च पाण्डवः}


\twolineshloka
{न भीमो नार्जुनश्चैव न यमौ पुरुषर्षभौ}
{वार्यमाणो मया नित्यं गान्धार्या विदुरेण च}


\twolineshloka
{जामदग्न्येन रामेण व्यासेन च महात्मना}
{दुर्योधनो युध्यमानो नित्यमेव हि सञ्जय}


\twolineshloka
{कर्णस्य मतमास्थाय सौबलस्य च पापकृत्}
{दुःशासनस्य च तथा पाण्डवान्नान्वचिन्तयत्}


\twolineshloka
{तस्याहं व्यसनं घोरं मन्ये प्राप्तं तु सञ्जय}
{श्वेतस्य च विनाशेन भीष्मस्य विजयेन च}


\twolineshloka
{संक्रुद्धः कृष्णसहितः पार्थः किमकरोद्युधि}
{अर्जुनाद्धि भयं भूयस्तन्मे तात न शाम्यति}


\twolineshloka
{स हि शूरश्च कौन्तेयः क्षिप्रकारी धनंजयः}
{मन्ये शरैः शरीराणि शत्रूणां प्रमथिष्यति}


\twolineshloka
{ऐन्द्रिमिन्द्रानुजसमं महेन्द्रसदृशं बले}
{अमोघक्रोधसंकल्पं दृष्ट्वा वः किमभून्मनः}


\twolineshloka
{तथैव वेदविच्छूरो ज्वलनार्कसमद्युतिः}
{इन्द्रास्त्रविदमेयात्मा प्रपतन्समितिंजयः}


\twolineshloka
{वज्रसंस्पर्शरूपाणामस्त्राणां च प्रयोजकः}
{सखङ्गाक्षेपहस्तस्तु घोषं चक्रे महारथः}


\twolineshloka
{स सञ्जय महाप्राज्ञो द्रुपदस्यात्मजो बली}
{धृष्टद्युम्नः किमकरोच्छ्वेते युधि निपातिते}


\twolineshloka
{पुरा चैवापराधेन वधेन च चमूपतेः}
{मन्ये मनः प्रजज्वाल पाण्डवानां महात्मनाम्}


\fourlineindentedshloka
{तेषां क्रोधं चिन्तयंस्तु अहःसु च निशासु च}
{न शान्तिमधिगच्छामि दुर्योदनकृतेन हि}
{कथं चाभून्महायुद्धं सर्वमाचक्ष्व सञ्जय ॥सञ्जय उवाच}
{}


\twolineshloka
{श्रृणु राजन्स्थिरो भूत्वा तवापनयनो महान्}
{न च दुर्योधने दोषमिमामाधातुमर्हसि}


\twolineshloka
{गतोदके सेतुबन्धो यादृक्तादृङ्भतिस्तव}
{संदीप्ते भवने यद्वत्कूपस्य खननं तथा}


\twolineshloka
{गतपूर्वाह्णभूयिष्ठे तस्मिन्नहनि दारुणे}
{तावकानां परेषां च पुनर्युद्धमवर्तत}


\twolineshloka
{श्वेतं तु निहतं दृष्ट्वा विराटस्य चमूपतिम्}
{कृतवर्मणा च सहितं दृष्ट्वा शल्यमवस्थितम्}


\twolineshloka
{शङ्खः क्रोधात्प्रजज्वाल हविषा हव्यवाडिव}
{स विस्फार्य महच्चापं शक्रचापोपमं बली}


\twolineshloka
{अभ्यधावज्जिघांसन्वै शल्यं मद्राधिपं युधि}
{महता रथसङ्घेन समन्तात्परिरक्षितः}


\twolineshloka
{सृजन्बाणमयं वर्षं प्रायाच्छल्यरथं प्रति}
{तमापतन्तं संप्रेक्ष्य मत्तवारणविक्रमम्}


\twolineshloka
{तावकानां रथाः सप्त समन्तात्पर्यवारयन्}
{मद्रराजं परीप्सन्तो मृत्योर्दंष्ट्रान्तरं गतम्}


\twolineshloka
{बृहद्बलश्च कौसल्यो जयत्सेनश्च मागधः}
{तथा रुक्मरथो राजन्पुत्रः शल्यस्य मानितः}


\twolineshloka
{विन्दानुविन्दावावन्त्यौ काम्भोजश्च सुदक्षिणःक}
{बृहत्क्षत्रस्य दायादः सैन्धवश्च जयद्रथः}


\twolineshloka
{नानाधातुविचित्राणि कार्मुकाणि महात्मनाम्}
{विस्फारितान्यदृश्यन्त तोयदेष्विव विद्युतः}


\twolineshloka
{ते तु बाणमयं वर्षं शङ्खमूर्ध्निं न्यपातयन्}
{निदाघान्तेऽनिलोद्धृता मेघा इव नगे जलम्}


\twolineshloka
{ततः क्रुद्धो महेष्वासः सप्तभल्लैः सुतेजनैः}
{धनूंषि तेषामाच्छिद्य ननर्द पृतनापतिः}


\twolineshloka
{ततो भीष्मो महाबाहुर्विनद्य जलदो यथा}
{तालमात्रं धनुर्गृह्य शङ्खमभ्यद्रवद्रणे}


\twolineshloka
{तमुद्यन्तमुदीक्ष्याथ महेष्वासं महाबलम्}
{संत्रस्ता पाण्डवी सेना वातवेगहतेव नौः}


\twolineshloka
{ततोऽर्जुनः संत्वरितः शङ्खस्यासीत्पुरःसरः}
{भीष्माद्रक्ष्योऽयमद्येति ततो युद्धमवर्तत}


\twolineshloka
{हाहाकारो महानासीद्योधानां युधि युध्यताम्}
{तेजस्तेजसि संपृक्तमित्येवं विस्मयं ययुः}


\twolineshloka
{अथ शल्यो गदापाणिरवतीर्य महारथात्}
{शङ्खस्य चतुरो वाहानहनद्भरतर्षभ}


\twolineshloka
{स हताश्वाद्रथात्तूर्णं खङ्गमादाय विद्रुतः}
{बीभत्सोश्च रथं प्राप्य पुनः शान्तिमविन्दत}


\twolineshloka
{ततो भीष्मरथात्तूर्णमुत्पतन्ति पतत्रिणः}
{यैरन्तरीक्षं भूमिश्च सर्वतः समवस्तृता}


\twolineshloka
{पञ्चालानथ मत्स्यांश्च केकयांश्च प्रभद्रकान्}
{भीष्मः प्रहरतां श्रेष्ठः पातयामास पत्रिभिः}


\twolineshloka
{उत्सृज्य समरे राजन्पाण्डवं सव्यसाचिनम्}
{अभ्यद्रवत पाञ्चाल्यं द्रुपदं सेनया वृतम्}


\twolineshloka
{प्रियं संबन्धिनं राजञ्शरानवकिरन्बहून्}
{अग्निनेव प्रदग्धानि वनानि शिशिरात्यये}


\twolineshloka
{शरदग्धान्यदृश्यन्त सैन्यानि द्रुपदस्य ह}
{अत्यतिष्ठद्रणे भीष्मो विधूम इव पावकः}


\twolineshloka
{मध्यन्दिने यथाऽऽदित्यं तपन्तमिव तेजसा}
{न शेकुः पाण्डवेयस्य योधा भीष्मं निरीक्षितुम्}


\twolineshloka
{वीक्षांचक्रुः समंतात्ते पाण्डवा भयपीडिताः}
{त्रातारं नाध्यगच्छन्त गावः शीतार्दिता इव}


\twolineshloka
{सा तु यौधिष्ठिरी सेना गाङ्गेयशरपीडिता}
{सिंहेनेव विनिर्भिन्ना शुक्ला गौरिव गोपतेः}


\twolineshloka
{हते विप्रुद्रुते सैन्ये निरुत्साहे विमर्दिते}
{हाहाकारो महानासीत्पाण्डुसैन्येषु भारत}


\twolineshloka
{ततो भीष्मः शान्तनवो नित्यं मण्डलकार्मुकः}
{मुमोच बाणान्दीप्ताग्रानहीनाशीविषानिव}


\twolineshloka
{शरैरेकायनीकुर्वन्दिशः सर्वा यतव्रतः}
{जघान पाण्डवरथानादिश्यादिश्य भारत}


\twolineshloka
{ततः सैन्येषु भग्नेषु मथितेषु च सर्वशः}
{प्राप्ते चास्तं दिनकरे न प्राज्ञायत किंचन}


\twolineshloka
{भीष्मं च समुदीर्यन्तं दृष्ट्वा पार्था महाहवे}
{अवहारमकुर्वन्त सैन्यानां भरतर्षभ}


\chapter{अध्यायः ५०}
\twolineshloka
{सञ्जय उवाच}
{}


\twolineshloka
{कृतेऽवहारे सैन्यानां प्रथमे भरतर्षभ}
{भीष्मे च युद्धसंरब्धे हृष्टे दुर्योधने तथा}


\twolineshloka
{धर्मराजस्ततस्तूर्णमभिगम्य जनार्दनम्}
{भ्रातृभिः सहितः सर्वैः सर्वैश्चैव जनेश्वरैः}


\twolineshloka
{शुचा परमया युक्तश्चिन्तयानः पराजयम्}
{वार्ष्णेयमब्रवीद्रजान्दृष्ट्वा भीष्मस्य विक्रमम्}


\twolineshloka
{कृष्ण पश्य महेष्वासं भीष्मं भीमपराक्रमम्}
{शरैर्दहन्तं सैन्यं मे ग्रीष्मे कक्षमिवानलम्}


\twolineshloka
{कथमेनं महात्मानं शक्ष्यामः प्रतिवीक्षितुम्}
{लेलिह्यमानं सैन्यं मे हविष्मन्तमिवानलम्}


\twolineshloka
{एतं हि पुरुषव्याघ्रं धनुष्मन्तं महाबलम्}
{दृष्ट्वा विप्रद्रुर्त सैन्यं समरे मार्गणाहतम्}


\twolineshloka
{शक्यो जेतुं यमः क्रुद्धो वज्रपाणिश्च संयुगे}
{वरुणः पाशभृद्वापि कुबेरो वा गदाधरः}


\twolineshloka
{न तु भीष्मो महाजेताः शक्यो जेतुं महाबलः}
{सोऽहमेवं गते मग्नो भीष्मागाधजलेऽप्लुवे}


\twolineshloka
{आत्मनो बुद्धिदौर्बल्याद्भीष्ममासाद्य केशव}
{वनं यात्सामि वार्ष्णेय श्रेयो मे तत्र जीवितुम्}


\twolineshloka
{न त्वेतान्पृथिवीपालान्दातुं भीष्माय मृत्यवे}
{क्षपयिष्यति सेनां मे कृष्ण भीष्मो महास्त्रवित्}


\twolineshloka
{यथाऽनलं प्रज्वलितं पतङ्गाः समभिद्रुताः}
{विनाशायोपगच्छन्ति तथा मे सनिको जनः}


\twolineshloka
{क्षयं नीतोऽस्मि वार्ष्णेय राज्यहेतोः पराक्रमी}
{भ्रातश्चैव मे वीराः कर्शिताः शरपीडिताः}


\twolineshloka
{मत्कृते भ्रातृहार्देन राज्याद्भ्रष्टास्तथा सुखात्}
{जीवितं बहुमन्येऽहं जीवितं ह्यद्य दुर्लभम्}


\twolineshloka
{जीवितस्य च शेषेण तपस्तप्स्यामि दुश्चरम्}
{न घातयिष्यामि रणे मित्राणीमानि केशव}


\twolineshloka
{रथान्मे बहुसाहस्रान्दिव्यैरस्त्रैर्महाबलः}
{घातयत्यनिशं भीष्मः प्रवराणां प्रहारिणाम्}


\twolineshloka
{किं नु कृत्वा हितं मे स्याद्ब्रूहि माधव माचिरम्}
{मध्यस्थमिव पश्यामि समरे सव्यसाचिनम्}


\twolineshloka
{एको भीमः परं शक्त्या युध्यत्येव महाभुजः}
{केवलं बाहुवीर्येण क्षत्रधर्ममनुस्मरन्}


\twolineshloka
{गदया वीरघातिन्या यथोत्साहं महामनाः}
{करोत्यसुकरं कर्म रथाश्वनरपङ्क्तिषु}


\twolineshloka
{नालमेष क्षयं कर्तुं परसैन्यस्य मारिष}
{आर्जवेनैव युद्धेन वीर वर्षशतैरपि}


\twolineshloka
{एकोऽस्त्रवित्सखा तेऽयं सोऽप्यस्मान्समुपेक्षते}
{निर्दह्यमानान्भीष्मेण द्रोणेन च महात्मना}


\twolineshloka
{दिव्यान्यस्त्राणि भीष्मस्य द्रोणस्य च महात्मनः}
{धक्ष्यन्ति क्षत्रियान्सर्वान्प्रयुक्तानि पुनः पुनः}


\twolineshloka
{कृष्ण भीष्मः सुसंरब्धः सहितः सर्वपार्थिवैः}
{क्षपयिष्यति नो नूनं यादृशोऽस्य पराक्रमः}


\twolineshloka
{स त्वं पश्य महाभाग योगेश्वर महारथम्}
{भीष्मं यः शमयेत्संख्ये दावाग्निं जलदो यथा}


\twolineshloka
{तव प्रसादाद्गोविन्द पाण्डवा निहतद्विषः}
{स्वराज्यमनुसंप्राप्ता मोदिष्यन्ते सबान्धवाः}


\twolineshloka
{एवमुक्त्वा ततः पार्थो ध्यान्नास्ते महामनाः}
{चिरमन्तर्मना भूत्वा शोकोपहतचेतनः}


\threelineshloka
{शोकार्तं तमथो ज्ञात्वा दुःखोपहतचेतसम्}
{अब्रवीत्तत्र गोविन्दो हर्षयन्सर्वपाण्डवान् ॥ 6-50-27aमाशुचोभरतश्रेष्ट न त्वं शोचितुमर्हसि}
{यस्य ते भ्रातरः शूराः सर्वलोकेषु धन्विनःक}


\twolineshloka
{अहं च प्रियकृद्राजन्सात्यकिश्च महायशाः}
{विराटद्रुपदौ चेमौ धृष्टद्युम्नश्च पार्षतः}


\twolineshloka
{तथैव सबलाश्चेमे राजानो राजसत्तम}
{त्वत्प्रसादं प्रतीक्षन्ते त्वद्भक्ताश्च विशांपते}


\twolineshloka
{एष ते पार्षतो नित्यं हितकामः प्रिये रतः}
{सैनापत्यमनुप्राप्तो धृष्टद्युम्नो महाबलः}


\threelineshloka
{शिखण्डी च महाबाहो भीष्मस्य निधनं किल}
{एतच्छ्रुत्वा ततो धर्मो धृष्टद्युम्नं महारथम्}
{अब्रवीत्समितौ तस्यां वासुदेवस्य श्रृण्वतः}


\twolineshloka
{धृष्टद्युम्न निबोधेदं यत्त्वां वक्ष्यामि मारिष}
{नातिक्रम्यं भवेत्तच्च वचनं मम भाषितम्}


\twolineshloka
{भवान्सेनापतिर्मह्यं वासुदेवेन संमितः}
{कार्तिकेयो यथा नित्यं देवानामभवत्पुरा}


\twolineshloka
{तथा त्वमपि पाण्डूनां सेनानीः पुरुषर्षभ}
{स त्वं पुरुषशार्दूल विक्रम्य जहि कौरवान्}


\twolineshloka
{अहं च तेऽनुयास्यामि भीमः कृष्णश्च मारिष}
{माद्रीपुत्रौ च सहितौ द्रौपदेयाश्च दंशिताः}


\twolineshloka
{ये चान्ये पृथिवीपालाः प्रधानाः पुरुषर्षभ}
{तत उद्धर्षयन्सर्वान्धृष्टद्युम्नोऽभ्यभाषत}


\twolineshloka
{अहं द्रोणान्तकः पार्थ विहितः शम्भुना पुरा}
{रणे भीष्मं कृपं द्रोणं तथा शल्यं जयद्रथम्}


\twolineshloka
{सर्वानद्य रणे दृप्तान्प्रतियोत्स्यामि पार्थिव}
{अथोत्क्रुष्टं महेष्वासैः पाण्डवैर्युद्धदुर्मदैः}


\twolineshloka
{समुद्यते पार्थिवेन्द्रे पार्षते शत्रुसूदने}
{तमब्रवीत्ततः पार्थः पार्षतं पृतनापतिम्}


\twolineshloka
{व्यूहः क्रौञ्चारुणो नाम सर्वशत्रुनिबर्हणः}
{यं बृहस्पतिरिन्द्राय तदा देवासुरेऽब्रवीत्}


\twolineshloka
{तं यथावत्प्रतिव्यूहं परानीकविनाशनम्}
{अदृष्टपूर्वं राजानः पश्यन्तु कुरुभिः सह}


\twolineshloka
{यथोक्तः स नृदेवेन विष्णुर्वज्रभृता यथा}
{प्रभाते सर्वसैन्यानामग्रे चक्रे धनञ्जयम्}


\twolineshloka
{आदित्यपथगः केतुस्तस्याद्भुतमनोरमः}
{शासनात्पुरुहूतस्य निर्मितो विश्वकर्मणा}


\twolineshloka
{इन्द्रायुधसवर्णाभिः पताकाभिरलङ्कृतः}
{6-50-44bआकाशग इवाकाशेगन्धर्वनगरोपमः}


\twolineshloka
{नृत्यमान इवाभाति रथचर्यासु मारिष}
{तेन रत्नवता पार्थः स च गाण्डीवधन्वना}


\twolineshloka
{बभूव परमोपेतः सुमेरुरिव भानुना}
{शिरोभूद्द्रुपदो राजा महत्या सेनया वृतः}


\twolineshloka
{कुन्तिभोजश्च चैद्यश्च चक्षुर्भ्यां तौ जनेश्वरौ}
{दाशार्णकाः प्रभद्राश्च दाशेरकगणैः सह}


\twolineshloka
{अनूपकाः किराताश्च ग्रीवायां भरतर्षभ}
{पटच्चरैश्च पौण्ड्रैश्च राजन्पौरवकैस्तथा}


\twolineshloka
{निषादैः सहितश्चापि पृष्ठमासीद्युधिष्ठिरः}
{पक्षौ तु भीमसेनश्च धृष्टद्युम्नश्च पार्षतः}


\twolineshloka
{द्रौपदेयाभिमन्युश्च सात्यकिश्च महारथः}
{पिशाचा दारदाश्चैव पुण्ड्राः कुण्डीविषैः सह}


\twolineshloka
{मारुता धेनुकाश्चैव तंगमाः परतंगणाः}
{बाह्लिकास्तित्तिराश्चैव चोलाः पाण्ड्याश्च भारत}


\twolineshloka
{एते जनपदा राजन्दक्षिणं पक्षमाश्रिताः}
{अग्निवेशास्तुद्दुण्डाश्च मालवा दानभारयः}


\twolineshloka
{शबरा उद्भसाश्चैव वत्साश्च सह नाकुलैः}
{नकुलः सहदेवश्च वामं पक्षं समाश्रिताः}


\twolineshloka
{रथानामयुतं पक्षौ शिरस्तु नियुतं तथा}
{पृष्ठमर्बुदमेवासीत्सहस्रामि च विंशतिः}


\twolineshloka
{ग्रीवायां नियुतं चापि सहस्रामि च सप्ततिः}
{पक्षकोटिप्रपक्षेषु पक्षान्तेषु च वारणाः}


\twolineshloka
{जग्मुः परिवृता राजंश्चलन्त इव पर्वताः}
{जघनं पालयामास विराटः सह केकयैः}


\twolineshloka
{काशिराजश्च शैब्यश्च रथानामयुतैस्त्रिभिः}
{एवमेनं महाव्यूहं व्यूह्य भारत पाण्डवाः}


\threelineshloka
{सूर्योधयं त इच्छतः स्थिता युद्धाय दंशिताः}
{तेषामादित्यवर्मानि विमलानि महान्ति च}
{श्वेतच्छत्राण्यशोभन्त वारणेषु रथेषु च}


\chapter{अध्यायः ५१}
\twolineshloka
{सञ्जय उवाच}
{}


\twolineshloka
{क्रौञ्चं दृष्ट्वा ततो व्यूहमभेद्यं तनयस्तव}
{रक्षमाणं महाघोरं पार्थेनामिततेजसा}


\twolineshloka
{आचार्यमुपसंगम्य कृपं शल्यं च पार्थिव}
{सौमदत्तिं विकर्णं च सोऽश्वत्थामानमेव च}


\twolineshloka
{दुःशाकसनादीन्भ्रातॄंश्च सर्वनेव च भारत}
{अन्यांश्च सुबहूञ्शूरान्युद्धाय समुपागतान्}


\twolineshloka
{प्राहेदं वचनं काले हर्षयंस्तनयस्तव}
{नानाशस्त्रप्रहरणाः सर्वे युद्धविशारदाः}


\twolineshloka
{एकैकशः समर्था हि यूयं सर्वे महारथाः}
{पाण्डुपुत्रान्रणे हन्तुं ससैन्यान्किमु संहताः}


\twolineshloka
{अपर्याप्तं तदस्माकं बलं भीष्माभिरक्षितम्}
{पर्याप्तमिदमेतेषां बलं भीमाभिरक्षितम्}


\twolineshloka
{संस्थानाः शूरसेनाश्च वेत्रिकाः कुकुरास्तथा}
{आरोचकास्त्रिगर्ताश्च मद्रका यवनास्तथा}


\twolineshloka
{शत्रुंजयेन सहितास्तथा दुःशासनेन च}
{विकर्णेन च वीरेण तथा नन्दोपनन्दकैः}


\twolineshloka
{चित्रसेनेन सहिताः सहिताः पारिभद्रकैः}
{भीष्ममेवाभिरक्षन्तु सह सैन्यपुरस्कृताः}


\twolineshloka
{ततो भीष्मश्च द्रोणश्च तव पुत्राश्च मारिष}
{अव्यूहन्त महाव्यूहं पाण्डूनां प्रतिबाधकम्}


\twolineshloka
{भीष्मः सैन्येन महता समनन्तात्परिवारितः}
{ययौ प्रकर्षन्महतीं वाहिनीं सरराडिव}


\twolineshloka
{तमन्वायान्महेष्वासो भारद्वाजः प्रतापवान्}
{कुन्तलैश्च दशार्णैश्च मागधैश्च विशांपते}


\twolineshloka
{विदमर्भैर्मेकलैश्चैव कर्णप्रावरणैरपि}
{सहिताः सर्वसैन्येन भीष्ममहावशोभिनम्}


\twolineshloka
{गान्धाराः सिन्धुसौवीराः शिबयोऽथ वसातयः}
{शकुनिश्च स्वसैन्येन भीष्ममाहवशोभिनम्}


\twolineshloka
{ततो दुर्योधनो राजा सहितः सर्वसोदरैः}
{अश्वातर्केर्विकर्णैश्च तथा चाम्बष्ठकोसलैः}


\twolineshloka
{दरदैश्च शकैश्चैव तथा क्षुद्रकमालवैः}
{अभ्यरक्षत संहृष्टः सौबलेयस्य वाहिनीम्}


\twolineshloka
{भूरिश्रवाः शलः शल्यो भगदत्तश्च मारिषः}
{विन्दानुविन्दावावन्त्यौ वामं पार्श्वमपालयन्}


\twolineshloka
{सौमदत्तिः सुशर्मा च काम्भोजश्च सुदक्षिणः}
{श्रुतायुश्चाच्युतायुश्च दक्षिणं पक्षमास्थिताः}


\twolineshloka
{अश्वत्थामा कृपश्चैव कृतवर्मा च सात्वतः}
{महत्या सेनया सार्धं सेनापृष्टे व्यवस्थिताः}


\twolineshloka
{पृष्ठगोपास्तु तस्यासन्नानादेश्या जनेश्वराः}
{6-51-20bकेतुमान्वसुदानश्च पुत्रः काश्यस्य चाभिभूः}


\twolineshloka
{ततस्ते तावकाः सर्वे हृष्टा युद्धाय भारत}
{दध्मुः शङ्खान्मुदा युक्ताः सिंहनादांस्तथोन्नदन्}


\twolineshloka
{तेषां श्रुत्वा तु हृष्टानां वृद्धः कुरुपितामहः}
{सिंहनादं विनद्योच्चैः शङ्खं दध्मौ प्रतापवान्}


\twolineshloka
{ततः शङ्खाश्च भेर्यश्च पेश्यश्च विविधाः परैः}
{आनकाश्चाभ्यहन्यन्त स शब्दस्तुमुलोऽभवत्}


\twolineshloka
{ततः श्वेतैर्हयैर्युक्ते महति स्यन्दने स्थितौ}
{प्रदध्मतुः शङ्खवरौ हेमरत्नपरिष्कृतौ}


\twolineshloka
{पाञ्चजन्यं हृषीकेशो देवदत्तं धनंजयः}
{पौण्ड्रं दध्मौ महाशङ्खं भीमकर्मा वृकोदरः}


\twolineshloka
{अन्तविजयं राजा कुन्तीपुत्रो युधिष्ठिरः}
{नकुलः सहदेवश्च सुघोषमणिपुष्पकौ}


\twolineshloka
{काशिराजश्च शैब्यश्च शिखण्डी च महारथः}
{धृष्टद्युम्नो विराटश्च सात्यकिश्च महारथः}


\twolineshloka
{पाञ्चाल्याश्च महेष्वासा द्रौपद्याः पञ्च चात्मजाः}
{सर्वे दध्युर्महाशङ्खान्सिंहनादांश्च नेदिरे}


\twolineshloka
{स घोषः सुमहांस्तत्र वीरैस्तैः समुदीरितः}
{नभश्च पृथिवीं चैव तुमुलो व्यनुनादयन्}


\twolineshloka
{एवमेते महाराज प्रहृष्टाः कुरुपाण्डवाः}
{पुनर्युद्धाय संजग्मुस्तापयानाः परस्परम्}


\chapter{अध्यायः ५२}
\twolineshloka
{धृतराष्ट्र उवाच}
{}


\threelineshloka
{एवं व्यूढेष्वनीकेषु मामकेष्वितरेषु च}
{कथं प्रहरतां श्रेष्ठाः संप्रहारं प्रचक्रिरे ॥सञ्जय उवाच}
{}


\threelineshloka
{तथा व्यूढेष्वनीकेषु सन्नद्धरुचिरध्वजाः}
{`तावकाः पाण्डवैः सार्धं यथाऽयुध्यन्त तच्छृणु}
{'अपारमिव संदृश्य सागरप्रतिमं बलम्}


\twolineshloka
{तेषां मध्ये स्थितो राजन्पुत्रो दुर्योधनस्तव}
{अब्रवीत्तावकान्सर्वान्युद्ध्यध्वमिति दंशिताः}


\twolineshloka
{ते मनः क्रूरमाधाय समभित्यक्तजीविताः}
{पाण्डवानभ्यवर्तन्त सर्व एवोच्छ्रितध्वजाः}


\twolineshloka
{ततो युद्धं समभवत्तुमुलं रोमहर्षणम्}
{तावकानां परेषां च व्यतिषक्तरथद्विपम्}


\twolineshloka
{मुक्तास्तु रथिभिर्बाणा रुक्मपुङ्खाः सुतेजिताःक}
{सन्निपेतुरकुण्ठाग्रा नागेषु च हयेषु च}


\twolineshloka
{तथा प्रवृत्ते संग्रामे धनुरुद्यम्य दंशितः}
{अभिपत्य महाबाहुर्भीष्मो भीमपराक्रमः}


\twolineshloka
{सौभद्रे भीमसेने च सात्यकौ च महारथे}
{कैकेये च विराटे च धृष्टद्युम्ने च पार्षते}


\twolineshloka
{एतेषु नरवीरेषु चेदिमत्स्येषु चाभिभूः}
{ववर्ष शरवर्षाणि वृद्धः कुरुपितामहः}


\twolineshloka
{अभिद्यत ततो व्यूहस्तस्मिन्वीरसमागमे}
{सर्वेषामेव सैन्यानामासीद्व्यतिकरो महान्}


\twolineshloka
{सादिनो ध्वजिनश्चैव हतप्रवरवाजिनः}
{विप्रद्रुतरथानीकाः समपद्यन्त पाण्डवाः}


\twolineshloka
{अर्जुनस्तु नरव्याघ्रो दृष्ट्वा भीष्मं महारथम्}
{वार्ष्णेयमब्रवीत्क्रुद्धो याहि यत्र पितामहः}


\twolineshloka
{एष भीष्मः सुसंक्रुद्धो वार्ष्णेय मम वाहिनीम्}
{नाशयिष्यति सुव्यक्तं दुर्योधनहिते रतः}


\twolineshloka
{एष द्रोणः कृपः शल्यो विकर्णश्च जनार्दन}
{धार्तराष्ट्राश्च सहिता दुर्योधनपुरोगमाः}


\twolineshloka
{पाञ्चालान्निहनिष्यन्ति रक्षिता दृढधन्वना}
{सोऽहं भीष्मं वधिष्यामि सैन्यहेतोर्जनार्दन}


\twolineshloka
{तमब्रवीद्वासुदेवो यत्तो भव धनंजय}
{एष त्वां प्रापयिष्यामि पितामहरथं प्रति}


\twolineshloka
{एवमुक्त्वा ततः शौरी रथं तं लोकविश्रुतम्}
{प्रापयामास भीष्मस्य रथं प्रति जनेश्वर}


\twolineshloka
{चलद्बहुपातकेन बलाकावर्णवाजिना}
{समुच्छ्रितमहाभीमनदद्वानरकेतुना}


\twolineshloka
{महता मेघनादेन रथेनामिततेजसा}
{विनिघ्नन्कौरवानीकं शूरसेनांश्च पाण्डवः}


\twolineshloka
{प्रायाच्छरणदः शीघ्रं सुहृदां हर्षवर्धनः}
{तमापतन्तं वेगेन प्रभिन्नमिव वारणम्}


\threelineshloka
{त्रसयनतं रणे शूरान्मर्दयन्तं च सायकैः}
{सैन्धवप्रमुखैर्गुप्तः प्राच्यसौवीरकेकयैः}
{सहसा प्रत्युदीयाय भीष्मः शान्तनवोऽर्जुनम्}


\twolineshloka
{को हि गाण्डीवधन्वानमन्यः कुरुपितामहात्}
{द्रोणवैकर्तनाभ्यां वा रथी संयातुमर्हति}


\twolineshloka
{ततो भीष्मो महाराज सर्वलोकमहारथः}
{अर्जुनं सप्ततसप्तत्या नाराचानां समाचिनोत्}


\twolineshloka
{द्रोणश्च पञ्चविंशत्या कृपः पञ्चाशता शरैः}
{दुर्योधनश्रतुःषष्ट्या शल्यश्च नवभिः शरैः}


\twolineshloka
{सैन्धवो नवभिश्चैव शकुनिश्चापि पञ्चभिः}
{विकर्णो दशभिर्भल्लै राजन्विव्याध पाण्डवम्}


\twolineshloka
{स तैर्विद्धो महेष्वासः समन्तान्निशितैः शरैः}
{न विव्यथे महाबाहुर्भिद्यमान इवाचलः}


\twolineshloka
{स भीष्मं पञ्चविंशत्या कृपं च नवभिः शरैः}
{द्रोणं षष्ट्या नरव्याघ्रो विकर्णं च त्रिभिः शरैः}


\twolineshloka
{शल्यं चैव त्रिभिर्बाणै राजानं चैव पञ्चभिः}
{प्रत्यविध्यदमेयात्मा किरीटी भरतर्षभ}


\twolineshloka
{तं सात्यकिर्विराटश्च धृष्टद्युम्नश्च पार्षतः}
{सौभद्रो द्रौपदेयाश्च परिवव्रुर्धनंजयम्}


\twolineshloka
{ततो द्रोणं महेष्वासं गाङ्गेयस्य प्रिये रतम्}
{अभ्यवर्तत पाञ्चाल्यः संयुक्तः सह सोमकैः}


\twolineshloka
{भीष्मस्तु रथिनां श्रेष्ठो राजन्विव्याध पाण्डवम्}
{अशीत्या निशितैर्बाणैस्ततोऽक्रोशन्त पाण्डवाः}


\twolineshloka
{तेषां तु निनदं श्रुत्वा सहितानां प्रहृष्टवत्}
{प्रविवेश ततो मध्यं नरसिंहः प्रतापवान्}


\twolineshloka
{तेषां महारथानां स मध्यं प्राप्य धनंजयः}
{चिक्रीड धनुषा राजँल्लक्षं कृत्वा महारथान्}


\threelineshloka
{`क्षत्रियाणां शिरांस्युग्रैः कृन्तञ्शस्त्रैर्महारथः}
{शून्यान्कृत्वा रथोपस्थान्व्यचरत्फल्गुनस्तदा}
{'}


\twolineshloka
{ततो दुर्योधनो राजा भीष्ममाह जनेश्वरः}
{पीड्यमानं स्वकं सैन्यं दृष्ट्वा पार्थेन संयुगे}


\threelineshloka
{एष पाण्डुसुतस्तात कृष्णेन सहितो बली}
{यततां सर्वसैन्यानां मूलं नः परिकृन्तति}
{त्वयि जीवति गाङ्गेय द्रोणे च रथिनां वरे}


\threelineshloka
{त्वत्कृते चैव कर्णोऽपि न्यस्तशस्त्रो विशांपते}
{न युध्यति रणे पार्थान्हितकामः सदा मम}
{स तथा कुरु गाङ्गेय यथा हन्येत फल्गुनः}


\twolineshloka
{एवमुक्तस्ततो राजन्पिता देवव्रतस्तव}
{धिक् क्षात्रं धर्ममित्युक्त्वा प्रायात्पार्थं रथं प्रति}


\twolineshloka
{उभौ श्वेतहयौ राजन्संसक्तौ प्रेक्ष्य पार्थिवाः}
{सिंहनादान्भृशं चक्रुः शङ्खान्दध्मुश्च मारिषत}


\twolineshloka
{द्रौणिर्दुर्योधनश्चैव विकर्णश्च तवात्मजः}
{परिवार्य रणे भीष्मं स्थिता युद्धाय मारिष}


\twolineshloka
{तथैव पाण्डवाः सर्वे परिवार्य धनञ्जयम्}
{स्थिता युद्धाय महते ततो युद्धमवर्तत}


\twolineshloka
{गाङ्गेयस्तु रणे पार्थमानर्च्छन्नवभिः शरैः}
{तमर्जुनः प्रत्यविध्यद्दशभिर्मर्मभेदिभिः}


\twolineshloka
{ततः शरसहस्रेण सुप्रयुक्तेन पाण्डवः}
{अर्जुनः समरश्लाघी भीष्मस्यावारयद्दिशः}


\twolineshloka
{शरजालं ततस्तत्तु शरजालेन मारिष}
{वारयामास पार्थस्य भीष्मः शान्तनवस्तदा}


\twolineshloka
{उभौ परमसंहृष्टावुभौ युद्धाभिनन्दिनौ}
{निर्विशेषमयुध्येतां कृतप्रतिकृतैषिणौ}


\twolineshloka
{भीष्मचापविमुक्तानि शरजालानि शङ्घशः}
{शीर्यमाणान्यदृश्यन्त भिन्नान्यर्जुनसायकैःक}


\twolineshloka
{तथैवार्जुनमुक्तानि शरजालानि सर्वशः}
{गाङ्गेयशरनुन्नानि प्रापतन्त महीतले}


\threelineshloka
{अर्जुनः पञ्चविंशत्या भीष्ममार्च्छच्छितैः शरैः}
{भीष्मोपि समरे पार्थं विव्याध निशितैः शरैः}
{}


\twolineshloka
{अन्योन्यस्य हयान्विद्ध्वा ध्वजौ च सुमहाबलौ}
{रथेषां रथचक्रे च चिक्रीडतुररिन्दमौ}


\twolineshloka
{ततः क्रुद्धो महाराज भीष्मः प्रहरतां वरः}
{वासुदेवं त्रिभिर्बाणैराजघान स्तनान्तरे}


\twolineshloka
{भीष्मचापच्युतैस्तैस्तु निर्विद्धो मधुसूदनः}
{विरराज रणे राजन्सपुष्प इव किंशुकः}


\twolineshloka
{ततोऽर्जुनो भृशं क्रुद्धो निर्विद्धं प्रेक्ष्य माधवम्}
{सारथिं कुरुवृद्धस्य निर्बिभेद शितैः शरैः}


\twolineshloka
{यतमानौ तु तौ वीरावन्योन्यस्य वधं प्रति}
{न शक्नुतां तदाऽन्योन्यमभिसन्धातुमाहवे}


\twolineshloka
{तौ मण्डलानि चित्राणि गतत्प्रत्यागतानि च}
{अदर्शयेतां बहुधा सूतसामर्थ्यलाघवात्}


\twolineshloka
{अन्तरं च प्रहारेषु तर्कयन्तौ परस्परम्}
{राजन्नन्तरमार्गस्थौ स्थितावास्तां मुहुर्मुहुः}


\twolineshloka
{उभौ सिंहरवोन्मिश्रं शङ्खशब्दं च चक्रतुः}
{तथैव चापनिर्घोषं चक्रतुस्तौ महारथौ}


\twolineshloka
{तयोः शङ्खनिनादेन रथनेमिस्वनेन च}
{दारिता सहसा भूमिश्चकम्पे च ननाद च}


\twolineshloka
{नोभयोरन्तरं कश्चिद्ददृशे भरतर्षभ}
{बलिनौ युद्धदुर्धर्षावन्योन्यसदृशावुभौ}


\twolineshloka
{चिह्नमात्रेण भीष्मं तु प्रजज्ञुस्तत्र कौरवाः}
{तथा पाण्डुसुताः पार्थं चिह्नमात्रेण जज्ञिरेक}


\twolineshloka
{तयोर्नृवरयोर्दृष्ट्वा तादृशं तं पराक्रमम्}
{विस्कमयं सर्वभूतानि जग्मुर्भारतसंयुगे}


\twolineshloka
{न तयोर्विवरं कश्चिद्रणे पश्यति भारत}
{धर्मे स्थितस्य हि यथा न कश्चिद्वृजिनं क्वचित्}


\twolineshloka
{उभौ च शरजालेन तावदृश्यो बभूवतुः}
{प्रकाशौ च पुनस्तूर्णं बभूवतुरुभौ रणे}


\twolineshloka
{तत्र देवाः सगन्धर्वाश्चारणाश्चर्षिभिः सह}
{अन्योन्यं प्रत्यभाषन्त तयोर्दृष्ट्वा पराक्रमम्}


\twolineshloka
{न शक्यौ युधि संरब्धौ जेतुमेतौ कथंचन}
{सदेवासुरगन्धर्वैर्लोकैरपि महारथौ}


\twolineshloka
{आश्चर्यभूतं लोकेषु युद्धमेतन्महाद्भुतम्}
{नैतादृशानि युद्धानि भविष्यन्ति कथंचन}


\twolineshloka
{न हि शक्यो रणे जेतुं भीष्मः पार्थेन धीमता}
{सधनुः सरथः साश्वः प्रवपन्सायकान्रणे}


\twolineshloka
{तथैव पाण्डवं युद्धे देवैरपि दुरासदम्}
{न विजेतुं रणे भीष्म उत्सहेत धनुर्धरम्}


\threelineshloka
{आलोकादपि युद्धं हि सममेतद्भविष्यति}
{इति स्म वाचोऽश्रूयन्त प्रोच्चरन्त्यकस्ततस्ततः}
{गाङ्गेयार्जुनयोः सङ्ख्ये स्तवयुक्ता विशांपते}


\twolineshloka
{त्वदीयास्तु तदा योधाः पाण्डवेयाश्च भारत}
{अन्योन्यं समरे जघ्नुस्तयोस्तत्र समागमे}


\twolineshloka
{शितधारैस्तथा स्वङ्गैर्विमलैश्च परश्वधैः}
{शरैरन्यैश्च बहुभिः सस्त्रैर्नानाविधैरपि}


\threelineshloka
{उभयोः सेनयोः शूरा न्यकृन्तन्त परस्परम्}
{वर्तमाने तथा घोरे तस्मिन्युद्धे सुदारुणे}
{द्रोणपाञ्चाल्ययो राजन्महानासीत्समागमः}


\chapter{अध्यायः ५३}
\twolineshloka
{धृतराष्ट्र उवाच}
{}


\twolineshloka
{कथं द्रोणो महेष्वासः पाञ्चाल्यश्चापि पार्षतः}
{उभौ समीयतुर्यत्तौ तन्ममाचक्ष्व सञ्जय}


\twolineshloka
{दिष्टमेव परं मन्ये पौरुषादिति मे मतिः}
{यत्र शान्तनवो भीष्मो नातरद्युधि पाण्डवम्}


\threelineshloka
{भीष्मो हि समरे क्रुद्धो हन्याल्लोकांश्चराचरान्}
{स कथं पाण्डवं युद्धे नातरत्सञ्जयौजसा ॥सञ्जय उवाच}
{}


\twolineshloka
{श्रृणु राजन्स्थिरो भूत्वा युद्धमेतत्सुदारुणम्}
{न शक्याः पाण्डवा जेतुं देवैरपि सवासवैः}


\twolineshloka
{द्रोणस्तु निशितैर्बाणैर्धृष्टद्युम्नमविध्यत}
{सारथिं चास्य भल्लेन रथनीडादपातयत्}


\twolineshloka
{तथाऽस्य चतुरो वाहांश्चतुर्भिः सायकोत्तमैः}
{पीडयामास संक्रुद्धो धृष्टद्युम्नस्य मारिष}


\twolineshloka
{धृष्टद्युम्नस्ततो द्रोणं नवत्या निशितैः शरैः}
{विव्याध प्रहसन्वीरस्तिष्ठतिष्ठेति चाब्रवीत्}


\twolineshloka
{ततः पुनरमेयात्मा भारद्वाजः प्रतापवान्}
{शरैः प्रच्छादयामास धृष्टद्युम्नममर्षणम्}


\twolineshloka
{आददे च शरं घोरं पार्षतान्तचिकीर्षकया}
{शक्राशनिसमस्पर्शं कालदण्डमिवापरम्}


\twolineshloka
{हाहाकारो महानासीत्सर्वसैन्येषु भारत}
{तमिषुं संधितं दृष्ट्वा भारद्वाजेन संयुगे}


\twolineshloka
{तत्राद्भुतमपश्याम धृष्टद्युम्नस्य पौरुषम्}
{यदेकः समरे वीरस्तस्थौ गिरिरिवाचलः}


\twolineshloka
{तं च दीप्तं शरं घोरमायान्तं मृत्युमात्मनः}
{चिच्छेद शरवृष्टिं च भारद्वाजे मुमोच ह}


\twolineshloka
{तत उच्चुक्रुशुः सर्वे पाञ्चालाः पाण्डवैः सह}
{धृष्टद्युम्नेन तत्कर्म कृतं दृष्ट्वा सुदुष्करम्}


\twolineshloka
{ततः शक्तिं महावेगां स्वर्णवैडूर्यभूषिताम्}
{द्रोणस्य निधनाकाङ्क्षी चिक्षेप स पराक्रमी}


\twolineshloka
{तामापतन्तीं सहसा शक्तिं कनकभूषिताम्}
{त्रिधा चिच्छेद समरे भारद्वाजो हसन्निव}


\twolineshloka
{शक्तिं विनिहतां दृष्ट्वा धृष्टद्युम्नः प्रतपवान्}
{ववर्ष शरवर्षाणि द्रोणं प्रति जनेश्वरः}


\twolineshloka
{शरवर्षं ततस्तत्तु सन्निवार्य महायशाः}
{द्रोणो द्रुपदपुत्रस्य मध्ये चिच्छेद कार्मुकम्}


\twolineshloka
{स च्छिन्नधन्वा समरे गदां गुर्वी महायशाः}
{द्रोणाय प्रेषयामास गिरिसारमयीं बली}


\twolineshloka
{सा गदा वेगवन्मुक्ता प्रायाद्द्रोणजिघांसया}
{तत्राद्भुतमपश्याम भारद्वाजस्य पौरुषम्}


\twolineshloka
{लाघवाद्व्यंसयामास गदां हेमविभूषिताम्}
{व्यंसयित्वा गदां तां च प्रेषयामास पार्षतम्}


\twolineshloka
{भल्लान्सुनिशितान्पीतातान्रुक्मपुङ्खान्सुदारुणान्}
{ते तस्य कवचं भित्त्वा पपुः शोणितमाहवे}


\twolineshloka
{अथान्यद्धनुरादाय धृष्टद्युम्नो महारथः}
{द्रोणं युधि पराक्रम्य शरैर्विव्याध पञ्चभिः}


\twolineshloka
{रुधिराक्तौ ततस्तौ तु शुशुभाते नरर्षभौ}
{वसन्तसमये राजन्पुष्पिताविव किंशुकौ}


\twolineshloka
{अमर्षितस्ततो राजन्पराक्रम्य चमूमुखे}
{द्रोणो द्रुपदपुत्रस्य पुनश्चिच्छेद कार्मुकम्}


\twolineshloka
{अथैनं छिन्नधन्वानं शरैः सन्नतपर्वभिः}
{अभ्यवर्षदमेयात्मा वृष्ट्या मेघ इवाचलम्}


\twolineshloka
{सारथिं चास्य भल्लेन रथनीजादपातयत्}
{अथास्य चतुरो वाहांश्चतुर्भिर्निशितैः शरैः}


\twolineshloka
{पातयामास समरे सिंहनादं ननाद च}
{ततोऽपरेण भल्लेन हस्ताच्चापमथाच्छिनत्}


\twolineshloka
{स च्छिन्नधन्वा विरथो हताश्वो हतसारथिः}
{गदापाणिरवारोहत्स्व्यापयन्पौरुषं महत्}


\twolineshloka
{तामस्य विशिस्वैस्तूर्णं पातयामास भारत}
{रथादनवरूढस्य तदद्भुतमिवाभवत्}


\twolineshloka
{ततः स विपुलं चर्म शतचन्द्रं च भानुमत्}
{खङ्गं च विपुलं दिव्यं प्रगृह्य सुभुजो बली}


\twolineshloka
{अभिदुद्राव वेगेन द्रोणस्य वधाकाङ्क्षया}
{आमिषार्थी यथा सिंहो वने मत्तमिव द्विपम्}


\twolineshloka
{तत्राद्भुतमपश्याम भारद्वाजस्य पौरुषम्}
{लाघवं चास्त्रयोगं च बलं बाह्वोश्च भारत}


\twolineshloka
{यदेनं शरवर्षेण वारयामास पार्षतम्}
{न शशाक ततो गन्तुं बलवानपि संयुगे}


\twolineshloka
{निवारितस्तु द्रोणेन धृष्टद्युम्नो महारथः}
{न्यावारयच्छरौघांस्तांश्चर्मणा कृतहस्तवत्}


\twolineshloka
{ततो भीमो महाबाहुः सहसाऽभ्यपतद्बली}
{साहाय्यकारी समरे पार्षतस्य महात्मनः}


\twolineshloka
{स द्रोमं निशितैर्बाणै राजन्विव्याध सप्तभिः}
{पार्षतं च रथं तूर्णं स्वकमारोहयत्तदा}


\twolineshloka
{ततो दुर्योधनो राजन्कलिङ्गं समचोदयत्}
{सैन्येन महता युक्तं भारद्वाजस्य रक्षणे}


\twolineshloka
{ततः सा महती सेन कलिङ्गानां जनेश्वर}
{भीममभ्युद्ययौ तूर्णं तव पुत्रस्य शासनात्}


\twolineshloka
{पाञ्चाल्यमथ संत्यज्य द्रोणोऽपि रथिनां वरः}
{विराटद्रुपदौ वृद्धौ वारयामास संयुगे}


\twolineshloka
{धृष्टद्युम्नोऽपि समरे धर्मराजानमभ्ययात्}
{ततः प्रववृते युद्धं तुमुलं रोमहर्षणम्}


\twolineshloka
{कलिङ्गानां च समरे भीष्मस्य च महात्मनः}
{जगतः प्रक्षयकरं घोररूपं भयावहम्}


\chapter{अध्यायः ५४}
\twolineshloka
{धृतराष्ट्र उवाच}
{}


\twolineshloka
{मम पुत्रसमादिष्टः कलिङ्गोः वाहिनीपतिः}
{कथमद्भुतकर्माणं भीमसेनं महाबलम्}


\threelineshloka
{चरन्तं गदया वीर दण्डहस्तमिवान्तकम्}
{योधयामास समरे कलिङ्गः सह सेनया ॥सञ्जय उवाच}
{}


\twolineshloka
{पुत्रेण तव राजेन्द्र स तथोक्तो महाबलः}
{महत्या सेनया गुप्तः प्रायाद्भीमरथं प्रति}


\twolineshloka
{तामापतन्तीं महतीं कलिङ्गानां महाचमूम्}
{रथाश्वनागकलिलां प्रगृहीतमहायुधाम्}


\twolineshloka
{भीमसेनः कलिङ्गानामार्च्छद्भारत वाहिनीम्}
{केतुमन्तं च नैषादिमायान्तं सह चेदिभिः}


\twolineshloka
{ततः श्रुतायुः संक्रुद्धो राज्ञा केतुमता सह}
{आससाद रणे भीमं व्यूढानीकेषु चेदिषु}


\twolineshloka
{रथैरनेकसाहस्रैः कलिङ्गानां नराधिप}
{अयुतेन गजानां च निषादैःक सह केतुमान्}


\twolineshloka
{भिमसेनं रणे राजन्समन्तात्पर्यवारयत्}
{चेदिमत्स्यकरूषाश्च भीमसेनपदानुगाः}


\twolineshloka
{अभ्यधावन्त समरे निषादान्सह राजभिः}
{ततः प्रववृते युद्धं घोररूपं भयावहम्}


\twolineshloka
{न प्राजानन्त योधाः स्वान्परस्परजिघांसया}
{घोरमासीत्ततो युद्धं भीमस्य सहसा परैः}


\twolineshloka
{यथेन्द्रस्य महाराज महत्या दैत्यसेनया}
{तस्य सैन्यस्य संग्रामे युध्यमानस्य भारत}


\twolineshloka
{बभूव सुमहाञ्शब्दः सागरस्येव गर्जतः}
{अन्योन्यं स्म तदा योधा विकर्षन्तो विशांपते}


\twolineshloka
{महीं चक्रुश्चितां सर्वां शशलोहितसन्निभाम्}
{योधांश्च स्वान्परान्वापि नाभ्यजानञ्जिघांसया}


\twolineshloka
{स्वानप्याददते स्वाश्च शूराः परमदुर्जयाः}
{विमर्दः सुमहानासीदल्पानां बहुभिः सह}


\twolineshloka
{कलिङ्गैः सह चैदीनां निषादैश्च विशांपते}
{कृत्वा पुरुषकारं तु यथाशक्ति महाबलाः}


\twolineshloka
{भीमसेनं परित्यज्य संन्यवर्तनक्त चेदयः}
{सर्वैः कलिङ्गैरासन्नः सन्निवृत्तेषु चेदिषु}


\twolineshloka
{स्वबाहुबलमस्थाय अभ्यवर्षन्त पाण्डवम्}
{न चचाल रथोपस्थाद्भीमसेनो महाबलः}


\twolineshloka
{शितैरवाकिरद्बाणैः कलिङ्गानां वरूथिनीम्}
{कालिङ्गस्तु महेष्वासः पुत्रश्चास्य महारथः}


\twolineshloka
{शक्रदेव इति ख्यातो जघ्नतुः पाण्डवं शरैः}
{ततो भीमो महाबाहुर्विधुन्वन्रुचिरं धनुः}


\twolineshloka
{योधयामास कालिङ्गं स्वबाहुबलमाश्रितः}
{शक्रदेवस्तु समरे विसृजन्सायकान्बहून्}


\twolineshloka
{अश्वाञ्जघान समरे भीमसेनस्य सायकैः}
{तं दृष्ट्वा विरथं तत्र भीमसेनमरिन्दमम्}


\twolineshloka
{शक्रदेवोऽभिदुद्राव शरैरवकिरञ्सितैः}
{भीमस्योपरि राजेन्द्र शक्रदेवो महाबलः}


\twolineshloka
{ववर्ष शरवर्षाणि तपान्ते जलदो यथा}
{हताश्वे तु रथे तिष्ठन्भीमसेनो महाबलः}


\twolineshloka
{शक्रदेवाय चिक्षेप सर्वशैक्यायसीं गदाम्}
{स तया निहतो राजन्कालिङ्गतनयो रथात्}


\twolineshloka
{विरथः सह सूतेन जगाम धरणीतलम्}
{कहतमात्मसुतं दृष्ट्वा कलिङ्गानां जनाधिपः}


\threelineshloka
{रथैरनेकसाहस्रैर्भीमस्यावारयद्दिशः}
{`अयुतेन गजानां च निषादैः परिवारितः}
{'ततो भीमो महावेगां त्यक्त्वा गुर्वी महागदां}


\twolineshloka
{निस्त्रिंशमाददे घोरं चिकीर्षुः कर्म दारुणम्}
{चर्म चाप्रतिमं राजन्नार्षभं पुरुषर्षभ}


\twolineshloka
{नक्षत्रैरर्धचन्द्रैश्च शातकुम्भमयैश्चितम्}
{कालिङ्गस्तु ततः क्रुद्धो धनुर्ज्यामवमृज्य च}


\twolineshloka
{प्रगृह्य च शरं घोरमेकं सर्पविषोपमम्}
{प्राहिणोद्भीमसेनाय वधाकाङ्क्षी जनेश्वरः}


\twolineshloka
{तमापतन्तं वेगेन प्रेरितं निशितं शरम्}
{भीमसेनो द्विधा राजंश्चिच्छेद विपुलासिना}


\twolineshloka
{उदक्रोशच्च संहृष्टस्त्रासयानो वरूथिनीम्}
{कालिङ्गोऽथ ततः क्रुद्धो भीमसेनाय संयुगे}


\twolineshloka
{तोमरान्प्राहिणोच्छीघ्रं चतुर्दश शिलाशितान्}
{तानप्राप्तान्महाबाहुः खगतानेव पाण्डवः}


\twolineshloka
{चिच्छेद सहसा राजन्नसंभ्रान्तो वरासिना}
{नित्यत्य तु रणे भीमस्तोमरान्वै चतुर्दश}


\twolineshloka
{भानुमन्तं ततो भीमः प्राद्रवत्पुरुषर्षभः}
{भानुमांस्तु ततो भीमं शरवर्षेण च्छादयन्}


\twolineshloka
{ननाद बलवन्नादं नादयानो नभस्तलम्}
{न च तं ममृषे भीमः सिंहनादं महाहवे}


\twolineshloka
{ततः शब्देन महता विननाद महास्वनः}
{तेन नादेन वित्रस्ता कलिङ्गानां वरूथिनी}


\twolineshloka
{न भीमं समरे मेने मानुषं भरतर्षभ}
{ततो भीमो महाबाहुर्नर्दित्वा विपुलं स्वनम्}


\twolineshloka
{सासिर्वेगवदाप्लुत्य दन्ताभ्यां वारणोत्तमम्}
{आरुरोह ततो मध्यं नागराजस्य मारिष}


\twolineshloka
{ततो मुमोच कालिङ्गः शक्तिं तामकरोद्द्वीधा}
{खङ्गेन पृथुना मध्ये भानुमन्तमथाच्छिनत्}


\twolineshloka
{सोऽन्तरायुधिनं हत्वा राजपुत्रमरिन्दमः}
{गुरुभारसहे स्कन्धे नागस्यासिमपातयत्}


\twolineshloka
{छिन्नस्कन्धः स विनदन्पपात गजयूथपः}
{आरुगणः सिन्धुवेगन सानुमानिव पर्वतः}


\twolineshloka
{ततस्तस्मादप्लुत्य गजाद्भारत भारतः}
{खङ्गपाणिरदीनात्मा तस्थौ भूमौ सुदंशितः}


\twolineshloka
{स चचार बहून्मार्गानभितः पातयन्गजान्}
{अग्निचक्रमिवाविद्धं सर्वतः प्रत्यदृश्यत}


\threelineshloka
{अश्वबृन्देषु नागेषु रथानीकेषु चाभिभूः}
{पदातीनां च सङ्घेषु विनिघ्नञ्शोणितोक्षितः}
{}


\twolineshloka
{श्येनवद्व्यचरद्भीमो रणेऽरिषु बलोत्कटः}
{छिन्दंस्तेषाकं शरीराणि शिरांसि च महाबलः}


\twolineshloka
{खङघ्गेकन शितधारेण संयुगे गजयोधिनाम्}
{पदातिरेकः संक्रुद्धः शत्रूणां भयवर्धनः}


\twolineshloka
{संमाहयामास स तान्कालान्तकयमोपमः}
{मूढाश्च ते तमेवाजौ विनदन्तः समाद्रवन्}


\twolineshloka
{सासिमुत्तमवेगेन विचरन्तं महारणे}
{निकृत्य रथिनां चाजौ रथेषाश्च युगानि च}


\twolineshloka
{जघान रथिनश्चापि बलवान्रिपुमर्दनः}
{भीमसेनश्चरन्मार्गान्सुबहून्प्रत्यदृश्यत}


\twolineshloka
{भ्रान्तमाविद्धमुद्भ्रान्तमाप्लुतं प्रसृतं प्लुतम्}
{संपातं समुदीर्णं च दर्शयामास पाण्डवः}


\twolineshloka
{केचिदग्रासिना छिन्नाः पाण्डवेन महात्मना}
{विनेदुर्भिन्नमर्माणो निपेतुश्च गतासवः}


\threelineshloka
{छइन्नदन्ताग्रहस्ताश्च भिन्नकुम्भास्तथाऽपरे}
{वियोधाः स्वान्यनीकानि जघ्नुर्भारत वारणाः}
{}


\twolineshloka
{निपेतुरुर्व्यां च तथा विनदन्तो महारवान्}
{छिन्नांश्च तोमरान्राजन्महामात्रशिरांसि च}


\twolineshloka
{परिस्तोमान्विचित्रांश्च कक्ष्याश्च कनकोज्ज्वलाः}
{ग्रैवेयाण्यथ शक्तीश्च पताकाः कणपांस्तथा}


\twolineshloka
{तूणीरानथ यन्त्राणि धनूंषि च}
{भिन्दिपालानि शुभ्राणि तोत्राणि चाङ्कुशैः सह}


\twolineshloka
{घण्टाश्च विविधा राजन्हेमगर्भान्त्सरूनपि}
{पततः पातितांश्चैव पश्यामः सह सादिभिः}


% Check verse!
छिन्नगात्रावरकरैर्निहतैश्चाप वारणैः ॥आसीद्भूमिःक समास्तीर्णा पतितैर्भूधरैरिव
\twolineshloka
{विमृद्यैवं महानागान्ममर्दाश्वान्महाबलः}
{अश्वारोहवरांश्चैव पातयामास संयुगे}


\twolineshloka
{तद्धोरमभवद्युद्धं तस्य तेषां च भारत}
{खलीनान्यथ योक्राणि कक्ष्याश्च कनकोज्ज्वलाः}


\twolineshloka
{परिस्तोमाश्च प्रासाश्च ऋष्टयश्च महाधनाः}
{कवचान्यथ चर्माणि चित्राण्यास्तरणानि च}


\twolineshloka
{तत्रतत्रापविद्धआनि व्यदृश्यन्त महाहवे}
{प्रासैर्यन्त्रैर्विचित्रैश्च शस्त्रैश्च विमलैस्तथा}


\twolineshloka
{स चक्रे वसुधां कीर्णां शबलैः कुसुमैरिव}
{आप्लुत्य रथिनः काश्चित्परामृश्य महाबलः}


\twolineshloka
{पातयामास खङ्गेन सध्वजानपि पाण्डवः}
{मुहुरुत्पततो दिक्षु धावतश्च यशस्विनःक}


\twolineshloka
{मार्गांश्च चरतश्चित्रं व्यस्मयन्त रणे जनाः}
{स जघान पदा कांश्चिद्व्याक्षिप्यान्यानपोथयत्}


\twolineshloka
{खह्गेनान्यांश्च चिच्छेद नादेनान्यांश्च भीषयन्}
{ऊरुवेगेन चाप्यन्यान्पातयामास भूतले}


\twolineshloka
{अपरे चैनमालोक्य भयात्पञ्चत्वमागताः}
{एवं सा बहुला सेना कलिङ्गानां तरस्विनाम्}


\twolineshloka
{परिवार्य रणे भीष्मं भीमसेनमुपाद्रवत्}
{ततः कालिङ्गयैन्यानां प्रमुखे भरतर्षभ}


\threelineshloka
{श्रुतायुषमभिप्रेक्ष्य भीमसेनः समभ्ययात्}
{तमायान्तमभिप्रेक्ष्य कालिङ्गो नवभिः शरैः}
{}


\twolineshloka
{भीमसेनममेयात्मा प्रत्यविध्यत्स्तनान्तरे}
{कलिङ्गबाणाभिहतस्तोत्रार्दित इव द्विपः}


\twolineshloka
{भीमसेनः प्रजज्वाल क्रोदेनाग्निरिवैधितः}
{अथाशोकः समादाय रथं हेमपरिष्कृतम्}


\twolineshloka
{भीमं संपादयामास रथेन रथसारथिः}
{तामरुह्य रथं तूर्णं कौन्तेयः शत्रुसूदनः}


\twolineshloka
{कलिङ्गमभिदुद्राव तिष्ठितिष्ठेति चाब्रवीत्}
{ततः श्रुतायुर्बलावान्भीमाय निशिताञ्शरान्}


\twolineshloka
{प्रेषयामास संक्रुद्धो दर्शयन्पाणिलाघवम्}
{स कार्मुकवरोत्सृष्टैर्नवभिर्निशितैः शरैः}


\twolineshloka
{समाहतो महाराज कलिङ्गेन महात्मना}
{संचुक्रुशे भृशं भीमो दण्डाहत इवोरगः}


\twolineshloka
{क्रुद्धश्च चापमायम्य बलवद्बलिनां वरः}
{कलिङ्गमवधीत्पार्थो भीमः सप्तभिरायसैः}


\twolineshloka
{क्षुराभ्यां चक्ररक्षौ च कलिङ्गस्य महाबलौ}
{सत्यदेवं च सत्यं च प्राहिणोद्यमसादनम्}


\twolineshloka
{ततः पुनमेयात्मा नासचैर्निशितैस्त्रिभिः}
{केतुमन्तं रणे भीमोऽगमयद्यमसादनम्}


\twolineshloka
{ततः कलिङ्गाः सन्नद्धा भीमसेनममर्षणम्}
{अनीकैर्बहुसाहस्रैः क्षत्रियाः समवारयन्}


\twolineshloka
{ततः शक्तिगदाखङ्गतोमरर्ष्टिपरश्वथैः}
{कलिङ्गाश्च ततो राजन्भीमसेनमवाकिरन्}


\twolineshloka
{संनिवार्य स तां घोरां शरवृष्टिं समुत्थिताम्}
{गदामादाय तरसा संनिपत्य महाबलः}


\twolineshloka
{भीमः सप्तशतान्वीराननयद्यमसादनम्}
{पुनश्चैव द्विसाहस्रान्कलिङ्गानरिमर्दनः}


\twolineshloka
{प्राहिणोन्मृत्युलोकाय तदद्भुतमिवाभवत्}
{एवं स तान्यनीकानि कलिङ्गानां पुनःपुनः}


\twolineshloka
{बिभेद समरे तूर्णं प्रेक्ष्य भीष्मं महारथम्}
{हतारोहाश्च मातङ्गाः पाण्डवेन कृता रणे}


\twolineshloka
{विप्रजग्मुरनीकेषु मेघा वातहता इव}
{मृद्गन्तः स्वान्यनीकानि विनदन्तः शरातुराः}


\twolineshloka
{ततो भीमो महाबाहुः खङ्गहस्तो महाभुजःक}
{संप्रहृष्टो महाघोषं शङ्खं प्राध्मापयद्बली}


\twolineshloka
{सर्वकालिङ्गसैन्यानां मनांसि समकम्पयत्}
{मोहश्चापि कलिङ्गानामाविवेश परंतप}


\twolineshloka
{प्राकम्पन्त च सैन्यानि वाहनानि च सर्वशः}
{भीमेन समरे राजन्गजेन्द्रेणेव सर्वशः}


\twolineshloka
{मार्गान्बहून्विचरता धावता च ततस्ततः}
{मुहुरुत्पतता चैव संमोहः समपद्यत}


\twolineshloka
{भीमसेनभयत्रस्तं सैन्यं च समकम्पत}
{क्षोभ्यमाणमसंबाधं ग्राहेणेव महत्सरः}


\twolineshloka
{त्रासितेषु च सर्वेषु भीमेनाद्भुतकर्मणा}
{पुनरावर्तमानेषु विद्रवत्सु च सङ्घशः}


\twolineshloka
{सर्वकालिङ्गयोधेषु पाण्डूनां ध्वजिनीपतिः}
{अब्रवीत्स्वान्यनीकानि युध्यध्वमिति पार्षतः}


\threelineshloka
{सेनापतिवचः श्रुत्वा शिखण्डिप्रमुखा गणाः}
{भीममेवाभ्यवर्तन्त रथानीकैः प्रहारिभिः ॥ 6-54-93aधर्मराजश्चतान्सर्वानुपजग्राह पाण्डवः}
{महता मेघवर्णेन नागानीकेन पृष्ठतः}


\twolineshloka
{एवं संनोद्य सर्वामि स्वान्यनीकानि पार्षतः}
{भीमसेनस्य जग्राह पार्ष्णि सत्पुरुषैर्वृतः}


\twolineshloka
{न हि पञ्चालराजस्य लोके कश्चन विद्यते}
{भीमसात्यकयोरन्यः प्राणेभ्यः प्रियकृत्तमः}


\twolineshloka
{सोऽपश्यच्च कलिङ्गेषु चरन्तमरिसूदनः}
{भीमसेनं महाबाहुं पार्षतः परवीरहा}


\twolineshloka
{ननर्द बहुधा राजन्हृष्टश्चासीत्परंतपः}
{शङ्खं दध्मौ च समरे सिंहनादं ननाद च}


\threelineshloka
{स च पारावताश्वस्य रथे हेमपरिष्कृते}
{कोविदारध्वजं दृष्ट्वा भीमसेनः समाश्वसत्}
{}


% Check verse!
धृष्टद्युम्नस्तु तं दृष्ट्वा कलिङ्गैः समभिद्रुतम् ॥भीमसेनममेयात्मा त्राणायाजौ समभ्ययात्
\threelineshloka
{तौ दूरात्सात्यकिं दृष्ट्वा धृष्टद्युम्नवृकोदरौ}
{कलिङ्गान्समरे वीरौ योधयेतां मनस्विनौ}
{}


\twolineshloka
{स तत्र गत्वा शैनेयो जवेन जयतां वरः}
{पार्थपार्षतयोः पार्षिं जग्राह पुरुषर्षभः}


\twolineshloka
{स कृत्कवा दारुणं कर्म प्रगृहीतशरासनः}
{आस्थितो रौद्रमात्मानं कलिङ्गानन्ववैक्षत}


\twolineshloka
{कलिङ्गप्रभवां चैव मांसशोणितकर्दमाम्}
{रुधिरस्यन्दिनीं तत्र भीमः प्रावर्तयन्नदीम्}


\twolineshloka
{अन्तरेण कलिङ्गानां पाण्डवानां च वाहिनीम्}
{तां संततार दुस्तारां भीमसेनो महाबलः}


\twolineshloka
{भीमसेनं तथा दृष्ट्वा प्राक्रोशंस्तावका नृप}
{कालोऽयं भीमरूपेण कलिङ्गैः कसह युध्यते}


\twolineshloka
{ततः शान्तनवो भीष्मः श्रुत्वा तं निनदं रणे}
{अभ्ययात्त्वरितो भीमं व्यूढानीकः समंततः}


\twolineshloka
{तं सात्यकिर्भीमसेनो धृष्टद्युम्नश्च पार्षतः}
{अभ्यद्रवन्त भीष्मस्य रथं हेमपरिष्कृतम्}


\twolineshloka
{परिवार्य तु ते सर्वे गाङ्गेयं तरसा रणे}
{त्रिभिस्त्रिभिः शरैर्घोरैर्भीष्ममानर्च्छुरोजसा}


\twolineshloka
{प्रत्यविध्यत तान्सर्वान्पिता देवव्रतस्तव}
{यतमानान्महेष्वासांस्त्रिभिस्त्रिभिरजिह्मगैः}


\twolineshloka
{ततः शरसहस्रेण सन्निवार्य महारथान्}
{हयान्काञ्चनसन्नाहान्भीमस्य न्यहनच्छरैः}


\twolineshloka
{हताश्वे स रथे तिष्ठन्भीमसेनः प्रतापवान्}
{शक्तिं चिक्षेप तरसा गाङ्गेयस्य रथं प्रति}


\twolineshloka
{अप्राप्तामथ तां शक्तिं पिता देवव्रतस्तव}
{त्रिधा चिच्छेद समरे सा पृथिव्यामशीर्यत}


\twolineshloka
{ततः शैक्यायसीं गुर्वी प्रगृह्य बलवान्गदाम्}
{भीमसेनस्ततस्तूर्णं पुप्लुवे मनुजर्षभ}


\twolineshloka
{सात्यकोऽपि ततस्तूर्णं भीमस्य प्रियकाम्यया}
{गाङ्गेयसारथिं तूर्णं पातयामास सायकैः}


\twolineshloka
{भीष्मस्तु निहते तस्मिन्सारथौ रथिनां वरः}
{वातायमानैस्तैरश्वैरपनीतो रणाजिरात्}


\twolineshloka
{भीमसेनस्ततो राजन्नपयाते महाव्रते}
{प्रजज्वाल यथा वह्निर्दहन्कक्षमिवैधितः}


\twolineshloka
{स हत्वा सर्वकालिङ्गान्सेनामध्ये व्यतिष्ठत}
{नैनमभ्युत्सहन्केचित्तावका भरतर्षभ}


\twolineshloka
{धृष्टद्युम्नस्तमारोप्य स्वरथे रथिनां वरः}
{पश्यतां सर्वसैन्यानामपोवाहक यशस्विनम्}


\twolineshloka
{संपूज्यमानः पाञ्चाल्यैर्मत्स्यैश्च भरतर्षभ}
{धृष्टद्युम्नं परिष्वज्य समेयादथ सात्यकिम्}


\twolineshloka
{अथाब्रवीद्भीमसेनं सात्यकिः सत्यविक्रमः}
{प्रहर्षयन्यदुव्याघ्रो धृष्टद्युम्नस्य पश्यतः}


\twolineshloka
{दिष्ट्या कालिङ्गराजश्च राजपुत्रश्च केतुमान}
{शक्रदेवश्च कालिङ्गः कलिङ्गाश्च मृधे हताः}


\twolineshloka
{स्वबाहुबलवीर्येण नागाश्वरथसंकुलः}
{महापुरुषभूयिष्ठो धीरयोधनिषेवितः}


\threelineshloka
{महाव्यूहः कलिङ्गानामेकेन मृदितस्त्वया}
{एवमुक्त्वा शिनेर्नप्ता दीर्घबाहुररिन्दम}
{रथाद्रथमभिद्रुत्य पर्यष्वजत पाण्डवम्}


\twolineshloka
{ततः स्वरथमास्थाय पुनरेव महारथः}
{तावकानवधीत्क्रुद्धो भीमस्य बलमादधत्}


\chapter{अध्यायः ५५}
\twolineshloka
{सञ्जय उवाच}
{}


\twolineshloka
{ततोऽपराह्णभूयिष्ठे तस्मिन्नहनि भारत}
{रथनागाश्वपत्तीनां सादिनां च महाक्षये}


\twolineshloka
{द्रोणपुत्रेण शल्येन कृपेण च महात्मना}
{समसज्जत पाञ्चाल्यस्त्रिभिरैतैर्महारथैः}


\twolineshloka
{स लोकविदितानश्वान्निजघान महाबलः}
{द्रौणेः पाञ्चालदायादः शितैर्दशभिराशुगैः}


\twolineshloka
{ततः शल्यरथं तूर्णमास्थाय हतवाहनः}
{द्रौणिः पाञ्चालदायादमभ्यवर्षदथेषुभिः}


\twolineshloka
{धृष्टद्युम्नं तु संयुक्तं द्रौणिना वीक्ष्य भारत}
{सौभद्रोऽभ्यपतत्तूर्णं विकिरन्निशिताञ्शरान्}


\twolineshloka
{स शल्यं प़ञ्चविंशत्या कृपं च नवभिः शरैः}
{अश्वत्थामानमष्टाभिर्विव्याध पुरुषर्षभः}


\twolineshloka
{आर्जुनिं तु ततस्तूर्णं द्रौणिर्विव्याध पत्रिणा}
{शल्योऽथ दशभिश्चैव कृपश्च निशितैस्त्रिभिः}


\twolineshloka
{लक्ष्मणस्तव पौत्रस्तु सौभद्रं समवस्थितम्}
{अभ्यवर्तत संहृष्टस्ततो युद्धमवर्तत}


\twolineshloka
{दौर्योधनिः सुसंक्रुद्धः सौभद्रं परवीरहा}
{विव्याध समरे राजंस्तदद्भुतमिवाभवत्}


\twolineshloka
{अभिमन्युः सुसंक्रुद्धो भ्रातरं भरतर्षभ}
{शरैः पञ्चाशतै राजन्क्षिप्रहस्तोऽभ्यविध्यत}


\twolineshloka
{लक्ष्मणोऽपि पुनस्तस्य धनुश्चिच्छेद पत्रिणा}
{मुष्टिदेशे महाराज ततस्ते चुक्रुशुर्जनाः}


\twolineshloka
{तद्विहाय धनुश्छिन्नं सौभद्रं परवीरहा}
{अन्यदादत्तवांश्चित्रं कार्मुकं वेगवत्तरम्}


\twolineshloka
{तौ तत्र समरे युक्तौ कृतप्रतिकृतैषिणौ}
{अन्योन्यं विशिस्वैस्तीक्ष्णैर्जघ्नतुः पुरुषर्षभौ}


\twolineshloka
{ततो दुर्योधनो राज दृष्ट्वा पुत्रं महारथम्}
{पीडितं तव पौत्रेण प्रायात्तत्र प्रजेश्वरः}


\twolineshloka
{सन्निवृत्ते तव सुते सर्व एव जनाधिपाः}
{आर्जुनिं रथवंशेन समन्तात्पर्यवारयन्}


\twolineshloka
{स तैः परिवृतः शूरैः शूरो युधि सुदुर्जयैः}
{न स्म प्रव्यथते राजन्कृष्णतुल्यपराक्रमः}


\twolineshloka
{सौभद्रमथ संसक्तं दृष्ट्वा तत्र धनंजयः}
{अभिदुद्राव वेगेन त्रातुकामः स्वमात्मजम्}


\twolineshloka
{ततः सरथनागाश्वा भीष्मद्रोणपुरोगमाः}
{अभ्यवर्तन्त राजानः सहिताः सव्यसाचिनम्}


\twolineshloka
{उद्वूतं सहसा भौमं नागाश्वरथपत्तिभिः}
{दिवाकररथं प्राप्य रजस्तीव्रमदृश्यत}


\twolineshloka
{तानि नागसहस्राणि भूमिपालशतानि च}
{तस्य बाणपथं प्राप्य नाभ्यवर्तन्त सर्वशः}


\twolineshloka
{प्रणेदुः सर्वभूतानि बभूवुस्तिमिरा दिशः}
{कुरूणां चानयस्तीव्रः समदृश्यत दारुणः}


\twolineshloka
{नाप्यन्तरिक्षं न दिशो न भूमिर्न च भास्करः}
{प्रजज्ञे भरतश्रेष्ठ सस्त्रसङ्घैः किरीटिनः}


\twolineshloka
{सादिता रथनागाश्च हताश्वा रथिनो रणे}
{विप्रद्रुतरथाः केचिद्दृश्यन्ते रथयूथपाः}


\twolineshloka
{विरथा रथिनश्चान्ये धावमानाः समन्ततः}
{तत्रतत्रैव दृश्यन्ते सायुधाः साङ्गदैर्भुजैः}


\twolineshloka
{हयारोहा हयांस्त्यक्त्वा गजारोहाश्च दन्तिनः}
{अर्जुनस्य भयाद्राजन्समन्ताद्विप्रदुद्रुवुः}


\twolineshloka
{रथेभ्यश्च गजेभ्यश्च हयेभ्यश्च नराधिपाः}
{पतिताः पात्यमानाश्च दृश्यन्तेऽर्जुनसायकैः}


\twolineshloka
{सगदानुद्यतान्बाहून्सखङ्गांश्च विशांपते}
{सप्रासांश्च सतूणीरान्सशरान्सशरासनान्}


\twolineshloka
{साङ्कुशाकन्सपताकांश्च तत्रतत्रार्जुनो नृणाम्}
{निचकर्त शरैरुग्रै रौद्रं वपुरधारयत्}


\twolineshloka
{परिघाणां प्रदीप्तानां मुद्गराणां च मारिष}
{प्रासानां भिन्दिपालानां निस्त्रिशानां च संयुगे}


\twolineshloka
{परश्वथानां पीक्ष्णानां तोमराणां च भारत}
{वर्मणां चापविद्धानां काञ्चनानां च भूमिप}


\twolineshloka
{ध्वजानां चर्मणां चैव व्यजनानां च सर्वशः}
{छत्राणां हेमदण्डानां तोमराणां च भारत}


\twolineshloka
{प्रतोदानां च योक्राणां कशानां चैव मारिष}
{राशयः स्मात्र दृश्यन्ते विनिकीर्णा रणक्षितौ}


\twolineshloka
{नासीत्तत्र पुमान्कश्चित्तव सैन्यस्य भारत}
{योऽर्जुनं समरे शूरं प्रत्युद्यायात्कथंचन}


\twolineshloka
{यो यो हि समरे पार्थं प्रत्युद्याति विशांपते}
{स सङ्ख्ये विशिखैस्तीक्ष्णैः परलोकाय नीयते}


\twolineshloka
{तेषु विद्रवमाणेषु तव योधेषु सर्वशः}
{अर्जुनो वासुदेवश्च दध्मतुर्वारिजोत्तमौ}


\twolineshloka
{तत्पभग्नं बलं दृष्ट्वा पिता देवव्रतस्तव}
{अब्रवीत्समरे शूरं भारद्वाजं स्मयन्निव}


\twolineshloka
{एष पाण्डुसुतो वीर कृष्णेन सहितो बली}
{तथा करोति सैन्यानि यथा कुर्याद्धऩञ्जयः}


\twolineshloka
{न ह्येष समरे शक्यो विजेतुं हि कथंचन}
{यथास्य दृश्यते रूप कालान्तकयमोपमम्}


\twolineshloka
{न निवर्तयितुं चापि शक्येयं महती चमूः}
{अन्योन्यप्रेया पश्य द्रवतीयं वरूथिनी}


\twolineshloka
{एष चास्तं गिरिश्रेष्ठं भानुमान्प्रतिपद्यते}
{चक्षूंषि सर्वलोकस्य संहरन्निव सर्वथा}


\twolineshloka
{तत्रावहारं संप्राप्तं मन्येऽहं पुरुषर्षभ}
{श्रान्ता भीताश्च नो योधा न योत्स्यन्ति कथंचन}


\twolineshloka
{एवमुक्त्वा ततो भीष्मो द्रोणमाचार्यसत्तमम्}
{अवहारमथो चक्रे तावकानां महारथः}


\twolineshloka
{`ततः सरथनागाश्वा जंय प्राप्य समोपकाः}
{पाञ्चालाः पाण्डवाश्चैव प्रणेदुश्च पुनः पुनः}


\twolineshloka
{प्रययुः शिबिरायैव धनञ्जयपुरस्कृताः}
{वादित्रघोषैः संहृष्टा प्रणद्यन्तो महारथाः ॥'}


\twolineshloka
{ततोऽवहारः सैन्यानां तव तेषां च भारत}
{अस्तं गच्छति सूर्येऽभूत्सन्ध्याकाले च वर्तति}


\chapter{अध्यायः ५६}
\twolineshloka
{सञ्जय उवाच}
{}


\twolineshloka
{प्रभातायां च शर्वर्यां भीष्मः शान्तनस्तदा}
{अनीकान्यनुसंयाने व्यादिदेशाय भारत}


\threelineshloka
{गारुडं च महाव्यूहं चक्रे शान्तनवस्तदा}
{पुत्राणां ते जयाकाङ्क्षी भीष्मः कुरुपितामहः}
{}


\twolineshloka
{गरुडस्य स्वयं तुण्डे पिता देवव्रतस्तव}
{चक्षुषी च भरज्वाजः कृतवर्मा च सात्वतः}


\twolineshloka
{अश्वत्थामा कृपश्चैव शिर आस्तां यशस्विनौ}
{त्रैगर्त्तैरथ कैकेयैर्वाटधानैश्च संयुगे}


\twolineshloka
{भूरिश्रवाः शलः शल्यो भगदत्तश्च मारिष}
{मद्रकाः सिन्धुसौवीरास्तथा पाञ्चनदाश्च ये}


\twolineshloka
{जयद्रथेन सहिता ग्रीवायां सन्निवेशिताः}
{पृष्ठे दुर्योधनो राजा सोदर्यैः सानुगैर्वृतः}


\twolineshloka
{विन्दानुविन्दावावन्त्यौ काम्भोजश्च शकैः सह}
{पच्छमासन्महाराज शूरसेनाश्च सर्वशः}


\twolineshloka
{मागधाश्च कलिङ्गाश्च दासेरकगणैः सह}
{दक्षिणं पक्षमासाद्य स्थिता व्यूहस्य दंशिताः}


\twolineshloka
{काकरुशा विकुञ्चाश्च मुण्डाः कुण्डीवृषास्तथा}
{बृहद्वलेन सहिता वामं पार्श्वमवस्थिताः}


\twolineshloka
{व्यूढं दृष्ट्वा तु तत्सैन्यं सव्यसाची परंतपः}
{धृष्टद्युम्नेन सहितः प्रत्यव्यूहत संयुगे}


\twolineshloka
{अर्धचन्द्रेण व्यूहेन व्यूहन्तमतिदारुणम्}
{दक्षिणं शृङ्गमास्वय भिमसेनो व्यरोचत}


\twolineshloka
{नानासंस्त्रौपसंषत्रैर्नानादेश्यैर्नृपैर्वृतः}
{तदन्येव विराटश्च द्रुपदश्च महारथः}


\twolineshloka
{तदनन्तरमेवासीन्नीलो नीलायुधैः सह}
{नीलादनन्तरश्चैव धृष्टकेतुर्महाबलः}


\twolineshloka
{चेदिकाशिकरूपेश्च पौरवैरपि संवृतः}
{धृष्टद्युम्नः शिखण्डी च पाञ्चालाश्च प्रभद्रकाः}


\threelineshloka
{मध्ये सैन्यस्य महतः स्थिता युद्धाय भारत}
{तत्रैव धर्मराजोऽपि गजानीकेन संवृतः}
{}


\twolineshloka
{ततस्तु सात्यकी राजन्द्रौपद्याः पञ्च चात्मजाः}
{अभिमन्युस्ततः शूर इरावांश्च ततः परम्}


\twolineshloka
{भैमसेनिस्ततो राजन्केकयाश्च महारथाः}
{एते सर्वे महाराज वामं पार्श्वमुपाश्रिताः}


\threelineshloka
{सर्वस्य जगतो गोप्ता गोप्ता यस्य जनार्दनः}
{`तत्रानुरथिनां श्रेष्ठो वामश्रृङ्गे व्यवस्थितः}
{'एवमेतं महाव्यूहं प्रत्यव्यूहन्त पाण्डवाः}


\twolineshloka
{वधार्थं तव पुत्रामां तत्पक्षे ये च सङ्गताः}
{ततः प्रववृते युद्धं व्यतिषक्तरथद्विपम्}


\twolineshloka
{तावकानां परेषां च निघ्नतामितरेतरम्}
{हयौघाश्च रथौघाश्व तत्र तत्र विशांपते}


\twolineshloka
{संपतन्तो व्यदृश्यन्त निघ्नन्तस्ते परस्परम्}
{धावतां च रथोघानां निघ्नतां च पृथक् पृथक्}


\threelineshloka
{बभूव तुमुलः शब्दो विमिश्रो दुन्दुभिस्वनैः}
{दिवस्पृङ्वरवीराणां निघ्नतामितरेतरम्}
{संप्रहारे सुतुमुले तव तेषां च भारत}


\chapter{अध्यायः ५७}
\twolineshloka
{सञ्जय उवाच}
{}


\twolineshloka
{ततो व्यूढेष्वनीकेषु तावकेषु परेषु च}
{धनञ्जयो रथानीकमवधीत्तव भारत}


\twolineshloka
{शरैरथिरथो युद्धे दारयन्रथयूथपान्}
{ते वध्यमानाः पार्थेन कालेनेव युगक्षये}


\twolineshloka
{धार्तराष्ट्रा रणे यत्नात्पाण्डवान्प्रत्ययोधयन्}
{प्रार्थयाना यशो दीप्तां मृत्युं कृत्वा निवर्तनम्}


\twolineshloka
{एकाग्रमनसो भूत्वा पाण्डवानां वरूथिनीम्}
{बभञ्जुर्बहुशो राजंस्ते चासञ्जन्त संयुगे}


\twolineshloka
{द्रवद्भिरथभग्नैश्च परिवर्तद्भिरेव च}
{पाण्डवैः कौरवेयैश्च न प्राज्ञायत किंचन}


\twolineshloka
{उदतिष्ठद्रजो भौमं छादयानं दिवाकरम्}
{न दिशः प्रदिशो वापि जज्ञिरेऽत्र समागताः}


\twolineshloka
{अनुमानेन संज्ञाभिर्नामगोत्रैश्च संयुगे}
{अवर्तत तदा युद्धं तत्र तत्र विशांपते}


\twolineshloka
{न व्यूहो भिद्यते तत्र कौरवाणां कथञ्चन}
{रक्षितः सत्यसन्धेन भारद्वाजेन संयुगे}


\twolineshloka
{तथैव पाण्डवानां च रक्षितः सव्यसाचिना}
{नाभिद्यत महाव्यूहो भीमेन च सुरक्षितः}


\twolineshloka
{सेनाग्रादपि निष्पत्य प्रायुध्यंस्तत्र मानवाः}
{उभयोः सेनयो राजन्व्यतिषक्तरथद्विपाः}


\twolineshloka
{हयारोहैर्हयारोहाः पात्यन्ते स्म महाहवे}
{ऋष्टिभिर्विमलाभिश्च प्रासैरपि च संयुगे}


\twolineshloka
{रथी रथिनमासाद्य शरैः कनकभूषणैःक}
{पातयामास समरे तस्मिन्नतिभयंकरे}


\twolineshloka
{गजारोहा गजारोहान्नाराचशरतोमरैः}
{संसक्तान्पातयामासुस्तव तेषां च सर्वशः}


\twolineshloka
{कश्चिदुत्पत्य समरे वस्वारणमास्थितः}
{केशपक्षे परामृश्य जहार समरे शिरः}


\twolineshloka
{अन्ये द्विरददन्ताग्रनिर्भिन्नहृदया रणे}
{वेमुश्च रुधिरं वीरा निःश्वसन्तः समन्ततः}


\twolineshloka
{कश्चित्करिविषाणस्थो वीरो रणविशारदाः}
{प्रावेपच्छक्तिनिर्भिन्नो गजशिक्षास्त्रवेदिना}


\twolineshloka
{पत्तिमङ्घा रणे पक्तीन्भिमन्दिपालपरश्वथैः}
{न्यतापतयन्त संहृष्टाः परस्परकृतागसः}


\twolineshloka
{रथी च समरे राजन्नासाद्य गजयूथपम्}
{स गजं पातयामास गजी च रथिनां वरम्}


\twolineshloka
{रथिनं च हयारोहः प्रासेन भरतर्षभ}
{पातयामास समरे रथी च हयसादिनम्}


\twolineshloka
{पदाती रथिनं सङ्ख्ये रथी चापि पदातिनम्}
{न्यपातयच्छितैः शस्त्रैः सेनयोरुभयोरपि}


\twolineshloka
{गजारोहा हयारोहान्पातयांचक्रिरे तदा}
{हयारोहा गजस्थांश्च तदद्भुतमिवाभवत्}


\twolineshloka
{गजारोहवरैश्चापि तत्रतत्र पदातयः}
{पातिताः समदृश्यन्त तैश्चापि गजयोधिनः}


\twolineshloka
{पत्तिसङ्घा हयारोहैः सादिसङ्घाश्च पत्तिभिः}
{पात्यमाना व्यदृश्यन्त शतशोऽथ सहस्रशः}


\twolineshloka
{ध्वजैस्तत्रापविद्धैश्च कार्मुकैस्तोमरैस्तथा}
{प्रासैस्तथा गदाभिश्च परिघैः कम्पनैस्तथा}


\twolineshloka
{शक्तिभिः कवचैश्चित्रैः कणपैरङ्कुशैरपि}
{निस्त्रिंशैर्विमलैश्चापि स्वर्णपुङ्खैः शरैस्तथा}


\twolineshloka
{परिस्तोमैः कुथाभिश्च कम्बलैश्च महाधनैः}
{भूर्भाति भरतश्रेष्ठ स्रग्दामैरिव चित्रिता}


\twolineshloka
{नराश्वकायैः पतितैर्दन्तिभिश्च महाहवे}
{अगम्यरूपा पृथिवी मांसशोणितकर्दमा}


\twolineshloka
{प्रशशाम रजो भौमं व्युक्षितं रणशोणितैः}
{दिशश्च विमलाः सर्वाः संबभूवुर्जनेश्वर}


\twolineshloka
{उत्थितान्यगणेयानि कबन्धानि समन्ततः}
{चिह्नभूतानि जगतो विनाशार्थाय भारत}


\twolineshloka
{तस्मिन्युद्धे महारौद्रे वर्तमाने सुदारुणे}
{प्रत्यदृश्यन्त रथिनो धावमानाः समन्ततः}


\twolineshloka
{ततो भीष्मश्च द्रोणश्च सैन्धवश्च जयद्रथः}
{पुरुमित्रो जयो भोजः शल्यश्चापि ससौबलः}


\twolineshloka
{एते समरदुर्धर्षाः सिंहतुल्यपराक्रमाः}
{पाण्डवानामनीकानि बभञ्जुःस्म पुनःपुनः}


\twolineshloka
{तथैव भीमसेनोऽपि राक्षसश्च घटोत्कचः}
{सात्यकिश्चेकितानश्च द्रोपदेयाश्च भारत}


\twolineshloka
{तावकांस्तव पुत्रांश्च सहितान्सर्वराजभिः}
{द्रावयामासुराजौ ते त्रिदशा दानवानिव}


\twolineshloka
{तथा ते समरेऽन्योन्यं निघ्नन्तः क्षत्रियर्षभाः}
{रक्तोक्षिता घोररूपा विरेर्जुर्दानवा इव}


\twolineshloka
{विनिर्जित्य रिपून्वीरा सेकनयोरुभयोरपि}
{व्यदृस्यन्त महामात्रा ग्रहा इव नभस्तले}


\threelineshloka
{ततो रथसहस्रेण पुत्रो दुर्योधनस्तव}
{अभ्ययात्पाण्डवं युद्धे राक्षसं च घटोत्कचम् ॥ 6-57-38aतथैवपाण्डवाः सर्वे महत्या सेनया सह}
{द्रोणभीष्मौ रणे यत्तौ प्रत्युद्ययुररिन्दमौ}


\twolineshloka
{किरीटी च ययौ क्रुद्धः समन्तात्पार्थिवोत्तमान्}
{आर्जुनिः सात्यकिश्चैव ययतुः सौबलं बलम्}


\twolineshloka
{ततः प्रववृते भूयः संग्रामो रोमहर्षणः}
{तावकानां परेषां च समरे विजयैषिणाम्}


\chapter{अध्यायः ५८}
\twolineshloka
{सञ्जय उवाच}
{}


\twolineshloka
{ततस्ते पार्थिवाः क्रुद्धाः फल्गुनं वीक्ष्य संयुगे}
{रथैरनेकसाहस्रैः समन्तात्पर्यवारयन्}


\twolineshloka
{अथैनं रथबृन्देन कोष्ठकीकृत्य भारत}
{शरैः सुबहुसाहस्रैः समन्तादभ्यवारयन्}


\twolineshloka
{शक्तीश्च विमलाकस्तीक्ष्णा गदाश्च परिघैः सह}
{प्रासान्परश्वथांश्चैव मुद्गरान्मुसलानपि}


\twolineshloka
{चिक्षिषुः समरे क्रुद्धाः फल्गुनस्य रथं प्रति}
{शस्त्राणामथ तां वृष्टिं शलभानामिवायतिम्}


\twolineshloka
{रुरोध सर्वतः पार्थः शरैः कनकभूषणैः}
{तत्र तल्लाघवं दृष्ट्वा बीभत्सोरतिमानुषम्}


\twolineshloka
{देवदानवगन्धर्वाः पिशाचोरगराक्षसाः}
{साधुसाध्विति राजेन्द्र फल्गुनं प्रत्यपूजयन्}


\twolineshloka
{सात्यकिश्चाभिमन्युश्च महत्या सेकनया वृतौ}
{गान्धारान्समरे शूराञ्जग्मतुः सहसौबलान्}


\twolineshloka
{तत्र सौबलकाः क्रुद्धा वार्ष्णेयस्य रथोत्तमम्}
{तिशलश्चिच्छिदुः क्रोधाच्छस्त्रैर्नानाविधैर्युधि}


\twolineshloka
{सात्यकिस्तु रथं त्यक्त्वा वर्तमाने भयावहे}
{अभिमन्यो रथं तूर्णमारुरोह परंतपः}


\twolineshloka
{तावेकरथसंयुक्तौ सौबलेयस्य वाहिनीम्}
{व्यधमेतां शितैस्तूर्णं शरैःक सन्नतपर्वभिः}


\twolineshloka
{द्रोणभीष्मौ रणे यत्तौ धर्मराजस्य वाहिनीम्}
{नाशयेतां शरैस्तीक्ष्णैः कङ्कपत्रपरिच्छदैः}


\twolineshloka
{ततो धर्मसुतोक राजा माद्रीपुत्रौ च पाण्डवौ}
{मिषतां सर्वसैन्यानां द्रोणानीकमुपाद्रवन्}


\twolineshloka
{तत्रासीत्सुमहद्युद्धं तुमुलं रोमहर्षणम्}
{यथा देवासुरं युद्धं पूर्वमासीत्सुदारुणम्}


\twolineshloka
{कुर्वाणौ सुमहत्कर्म भीमसेनघटोत्कचौ}
{दुर्योधनस्ततोऽभ्येत्य तावुभावप्यवारयत्}


\twolineshloka
{तत्राद्भुतमपश्याम हैडिम्बस्य पराक्रमम्}
{अतीत्य पीतरं युद्धे यदयुध्यत भारत}


\twolineshloka
{भीमसेनस्तु संक्रुद्धो दुर्योधनममर्षणम्}
{हृद्यविध्यत्पृषत्केन प्रहसन्निव पाण्डवः}


\twolineshloka
{ततो दुर्योधननो राजा प्रहारवरपीडितः}
{निषसाद रथोपस्थे कश्मलं च जगाम ह}


\twolineshloka
{तं विसंज्ञं विदित्वा तु त्वरमाणोऽस्य सारथिः}
{अपोवाह रणाद्राजंस्ततः सैन्यमभज्यत}


\twolineshloka
{ततस्तां कौरवीं सेनां द्रवमाणां समन्ततः}
{निघ्नन्भीमः शरैस्तीक्ष्णैरनुवव्राज पृष्ठतः}


\twolineshloka
{पार्षतश्च रथश्रोष्ठो धर्मपुत्रश्च पाण्डवः}
{द्रोणस्य पश्यतः सैन्यं गाङ्गेयस्य च पश्यतः}


\twolineshloka
{जघ्नतुर्विशिकैस्तीक्ष्णैः परानीकविनाशनैः}
{द्रवमाणं तु तत्सैन्यं तव पुत्रस्य संयुगे}


\twolineshloka
{नाशक्नुतां वारयितुं भीष्मद्रोणौ महारथौ}
{वार्यमाणं च भीष्मेण द्रोणेन च महात्मना}


\twolineshloka
{विद्रवत्येव तत्सैन्यं पश्यतोर्द्रोणभीष्मयोः}
{ततो रथसहस्रेषु विद्रवत्सु ततस्ततःक}


\twolineshloka
{तावास्थितावेकरथं सौभद्रशिनिपुङ्गवौ}
{सौबलीं समरे सेनां शातयेतां समन्ततः}


\twolineshloka
{शुशुभाते तदा तौ तु शैनेयकुरुपुङ्गवौ}
{अमावास्यां गतौ यद्वत्सोमसूर्यौ नभस्तले}


\twolineshloka
{अर्जुनस्तु ततः क्रुद्धस्तव सैन्यं विशांपते}
{ववर्ष शरवर्षेण धाराभिरिव तोयदः}


\twolineshloka
{वध्यमानं ततस्तत्र शरैः पार्थस्य संयुगे}
{दुद्राव कौरवं सैन्यं विषादभयकम्पितम्}


\twolineshloka
{द्रवतस्तान्समालक्ष्य भीष्मद्रोणौ महारथौ}
{न्यवारयेतां संरब्धौ दुर्योधनहितैषिणौ}


\twolineshloka
{ततो दुर्योधनो राजा समाश्वस्य विशांपते}
{न्यवर्तयत तत्सैन्यं द्रवमाणां समन्ततः}


\twolineshloka
{यत्र यत्र हि पुत्रं ते ये ये पश्यन्ति भारत}
{तत्र तत्र न्यवर्तन्त क्षत्रियाणां महरथाः}


\twolineshloka
{तान्निवृत्तान्समीक्ष्यैव ततोऽन्येऽपीतरे जनाः}
{अन्योन्यस्पर्धया राजँल्लज्जया चावतस्थिरे}


\twolineshloka
{पुनरावर्ततां तेषां वेग आसीद्विशांपते}
{पूर्यतः सागरस्येव चन्द्रस्योदयनं प्रति}


\twolineshloka
{अन्निवृत्तांस्ततस्तांस्तु दृष्ट्वा राजा सुयोधनः}
{अब्रवीत्त्वरितो गत्वा भीष्मं शान्तनवं वचः}


\twolineshloka
{पितामह निबोधेदं यत्त्वां वक्ष्यामि भारत}
{नानुरूपमहं मन्ये त्वयि जीवति कौरव}


\twolineshloka
{द्रोणो चास्त्रविदां श्रेष्ठे सपुत्रे ससुहृञ्जने}
{कृपे चैव महेष्वासे द्रवते यद्वरूथिनी}


\twolineshloka
{न पाण्डवान्प्रतिलांस्तव मन्ये कथंचन}
{तथा द्रोणस्य सङ्ग्रमे द्रौमेश्चैव कृपस्य च}


\threelineshloka
{अनुग्राह्याः पाण्डुसुतास्तव नूनं पितामह}
{यथेमां क्षमसे वीर वध्यमानां वरूथिनीम् ॥ 6-58-38aसोऽस्मिवाच्यस्त्वया राजन्पूर्वमेव समागमे}
{न योत्स्ये पाण्डवान्सङ्ख्ये नापि पार्षतसात्यकी}


\twolineshloka
{श्रुत्वा तु वचनं तुभ्यमाचार्यस्य कृपस्य च}
{कर्णेन सहितः कृत्यं चिन्तयानस्तदैव हि}


\twolineshloka
{यदि नाहं परित्याज्यो युवाभ्यामिह संयुगे}
{विक्रमेणानुरूपेण युध्येतां पुरुषर्षभौ}


\twolineshloka
{एतच्छ्रुत्वा वचो भीष्मः प्रहसन्वै मुहुर्मुहुः}
{अब्रवीत्तनयं तुभ्यं क्रोधादुद्वृत्य चक्षुषी}


\twolineshloka
{बहुशोऽसि मया राजंस्तथ्यमुक्तो हितं वचः}
{अजेयाः पाण्डवा युद्धे देवैरपि सवासवैः}


\twolineshloka
{यत्तु शक्यं मया कर्तुं वृद्धेनाद्य गतायुषा}
{करिष्यामि यथाशक्ति पश्येदानीं सबान्धवः}


\twolineshloka
{अद्य पाण्डुसुतानेकः ससैन्यान्सह बन्धुभिः}
{सोऽहं निवारयिष्यामि सर्वलोकस्य पश्यतः}


\twolineshloka
{एवमुक्ते तु भीष्मेण पुत्रास्तव जनेश्वर}
{दध्मुः शङ्खान्मुदा युक्ता भेरीः संजघ्निरे भृशम्}


\twolineshloka
{पाण्डवा हि ततो राजञ्श्रुत्वा तं निनदं महत्}
{दध्मुः शङ्खांश्च भेरीश्च मुरजांश्चाप्यनादयन्}


\chapter{अध्यायः ५९}
\twolineshloka
{धृतराष्ट्र उवाच}
{}


\twolineshloka
{प्रतिज्ञाते ततस्तस्मिन्युद्धे भीष्मेण दारुणे}
{क्रोधितो मम पुत्रेण दुःखितेन विशेषतः}


\threelineshloka
{भीष्मः किमकरोत्तत्र पाण्डवेयेषु भारत}
{पितामहे वा पञ्चालास्तन्ममाचक्ष्व सञ्जय ॥स़ञ्जय उवाच}
{}


\twolineshloka
{गतपूर्वाह्णभूयिष्ठे तस्मिन्नहनि भारत}
{पश्चिमां दिशमास्थाय स्थिते चापि दिवाकरे}


\twolineshloka
{जयं प्राप्तेषु हृष्टेषु पाण्डवेषु महात्ससु}
{सर्वधर्मविशेषज्ञः पिता देवव्रतस्तव}


\twolineshloka
{अभ्ययाञ्जवनैरश्वैः पाण्डवानामनीकिनीम्}
{महत्या सेनया गुप्तस्तव पुत्रैश्च सर्वशः}


\twolineshloka
{प्रावर्तत ततो युद्धं तुमुलं रोमहर्षणम्}
{अस्माकं पाण्डवैः सार्धमनायात्तव भारत}


\twolineshloka
{घनुषां कूजतां तत्र तलानां चाभिहन्यताम्}
{महान्समभवच्छब्दो गिरीणामिव दीर्यताम्}


\twolineshloka
{तिष्ठ स्थितोऽस्मि विद्ध्यैनं निवर्तस्व स्थिरो भव}
{स्थिरोऽस्मि प्रहरस्वेति शब्दोऽश्रूयत सर्वशः}


\twolineshloka
{काञ्चनेषु तनुत्रेषु किरीटेषु ध्वजेषु च}
{शिलानामिव शैलेषु पतितानामभूद्ध्वनिः}


\twolineshloka
{पतितान्युत्तमाङ्गानि बाहवश्च विभूषिताः}
{व्यचेष्टन्त महीं प्राप्य शतशोऽथ सहस्रशः}


\twolineshloka
{हृतोत्तमाह्गाः केचित्तु तथैवोद्यतकार्मुकाः}
{प्रगृहीतायुधाश्चापि तस्थुः पुरुषसत्तमाःक}


\twolineshloka
{प्रावर्तत महावेगा नदी रुधिरवाहिनी}
{मातङ्गाङ्गशिला रौद्रा मांसशोणितकर्दमा}


\twolineshloka
{वरश्वनरनागानां शरीरप्रभवा तदा}
{परलोकार्णवमुखी गृध्रगोमायुमोदिनी}


\twolineshloka
{न दृष्टं न श्रुतं वापि युद्धमेतादृशं नृप}
{यथा तव सुतानां च पाण्डवानां च भारत}


\twolineshloka
{नासीद्रथपथस्तत्र यौधैर्युधि निपातितैः}
{गजैश्च पतितैर्नीलैर्गिरिश्रृङ्गैनिरवावृतः}


\twolineshloka
{विकीर्णैः कवचैश्चित्रैः शिरस्त्रणैश्च मारिष}
{शुशुभे तद्राणस्थानं शरदीव नभस्तलम्}


\twolineshloka
{विनिर्भिन्नाः शरैः केचिदन्त्रापीडप्रकर्षिणः}
{अभीताः समरे शत्रूनभ्यधावन्त दर्पिताः}


\twolineshloka
{तात भ्रातः सखे बन्धो वयस्य मम मातुल}
{मा मां परित्यजेत्यन्ये चुक्रुशुः पतिता रणे}


\twolineshloka
{अथाभ्येहि त्वमागच्छ किं भीतोसि क्व यास्यसि}
{स्थितोऽहं समरे मा भैरिति चान्ये विचुक्रुशुः}


\twolineshloka
{तत्र भीष्मः शान्तनवो नित्यं मण्डलकार्मुकः}
{मुमीच बाणान्दीप्ताग्रानहीनाशीविषानिव}


\twolineshloka
{शरैरेकायनीकुर्वन्दिशः सर्वा यतव्रतः}
{जघान पाण्डवरथानादिश्य भारतर्षभ}


\twolineshloka
{स नृत्यन्वै रथोपस्थे दर्शयन्पाणिलाघवम्}
{अलातचक्रवद्राजंस्तत्र तत्र स्म दृश्यते}


\twolineshloka
{तमेकं समरे शूरं पाण्डवाः सृञ्जयैः सह}
{अनेकशतसाहस्रं समपश्यन्त लाघवात्}


\twolineshloka
{मायाकृतात्मानमिव भीष्मं तत्र स्म मेनिरे}
{पूर्वस्यां दिशि तं दृष्ट्वा प्रतीच्यां ददृशुर्जनाः}


\twolineshloka
{उदीच्यां चैवमालोक्य दक्षिणस्यां पुनः प्रभो}
{एवं स समरे शूरो गाङ्गेयः प्रत्यदृश्यत}


\twolineshloka
{न चैवं पाण्डवेयानां कश्चिच्छक्नोति वीक्षितुम्}
{विशिखानेव पश्यन्ति भीष्मचापच्युतान्बहून्}


\twolineshloka
{कुर्वाणं समरे कर्म सूदयानं च वाहिनीम्}
{व्याक्रोशन्त रणे तत्र नरा बहुविधा बहु}


\twolineshloka
{अमानुषेण रूपेण चरन्तं पितरं तव}
{शलभा इव राजानः पतन्ति विधिचोदिताः}


\twolineshloka
{भीष्माग्निमभिसंक्रुद्धं विनाशाय सहस्रशः}
{न हि मोघः शरः कश्चिदासीद्भीष्मस्य संयुगे}


\twolineshloka
{नरनागाश्वकायेषु बहुत्वाल्लघुयोधिनः}
{भिनत्त्येकेन बाणेन सुमुखेन पतत्रिणा}


\twolineshloka
{गजं कंकटसन्नद्धं वज्रेणेव शिलोच्चयम्}
{द्वौ त्रीनपि गजारोहान्पिण्डितान्वर्मितानपि}


\twolineshloka
{नाराचेन सुमुक्तेन निजघान पिता तव}
{यो यो भीष्मं नरव्याघ्रमभ्येति युधि कश्चन}


\twolineshloka
{मुहूर्तदृष्टः स मया पतितो भुवि दृश्यते}
{एवं सा धर्मराजस्य वध्यमाना महाचमूः}


\twolineshloka
{भीष्मेणातुलवीर्येण व्यशीर्यत संहस्रधा}
{प्राकम्पत महासेना शरवर्षेण तापिता}


\twolineshloka
{पश्यतो वासुदेवस्य पार्थस्याथ शिखण्डिनः}
{यतमानाऽपि ते वीरा द्रवमाणान्महारथान्}


\twolineshloka
{नाशक्नुवन्वारयितुं भीष्मबाणप्रपीडितान्}
{महेन्द्रसमवीर्येण वध्यमाना महाचमूः}


\twolineshloka
{अभज्यत महाराज न च द्वौ सह धावतः}
{आविद्धरनागाश्वं पतितध्वजकूबरम्}


\twolineshloka
{अनीकं पाण्डुपुत्राणां हाहाभूतमचेतनम्}
{जघानात्र पिता पुत्रं पुत्रश्च पितरं तथा}


\twolineshloka
{प्रियं सखायं चाक्रन्दे सखा दैवबलात्कृतः}
{6-59-39bविमुच्यकवचान्यन्ये पाण्डुपुत्रस्य सैनिकाः}


\twolineshloka
{विनुक्तकेशा धावन्तः प्रत्यदृश्यन्त भारत}
{तद्गोकुलमिवोद्धान्तमुद्धान्तरथयूथपम्}


\twolineshloka
{ददृशे पाण्डुपुत्रस्य सैन्यमार्तस्वरं तदा}
{प्रभज्यमानं सैन्यं तु दृष्ट्वा यादवनन्दनः}


\twolineshloka
{उवाच पार्थं बीभत्सुं निगृह्य रथमुत्तमम्}
{असं स कालः संप्राप्तः पार्थ पस्तेऽभिकाङ्क्षितः}


\threelineshloka
{प्रहरस्व नरव्याघ्र न चेन्मोहाद्विमुह्यसे}
{यत्त्वया कथितं वीर पुरा राज्ञां समागमे}
{}


\twolineshloka
{भीष्मद्रोणमुखान्सर्वान्धार्तराष्ट्रस्य सैनिकान्}
{सानुबन्धान्हनिष्यामि ये मां योत्स्यन्ति संयुगे}


\twolineshloka
{इति तत्कुरु कौन्तेय सत्यं वाक्यमरिन्दम}
{बीभत्सो पश्य सैन्यं स्वं भज्यमानं ततस्ततः}


\twolineshloka
{द्रवतश्च महीपालान्पश्य यौधिष्ठिरे बले}
{दृष्ट्वा हि भीष्मं समरे व्यात्ताननमिवान्तकम्}


\twolineshloka
{भयार्ताः प्रपलायन्ते सिंहात्क्षुद्रमृगा इव}
{एवमुक्तः प्रत्युवाच वासुदेवं धनंजयः}


\threelineshloka
{नोदयाश्वान्यतो भीष्मो विगाहे तद्वलार्णवम्}
{पातयिष्यामि दुर्धर्षं वृद्धं कुरुपितामहम् ॥सञ्जय उवाच}
{}


\twolineshloka
{ततोऽश्वान्रजतप्रकख्यान्नोदयामास माधवः}
{यतो भीष्मरथो राजन्दुष्प्रेक्ष्यो रश्मिमानिव}


\twolineshloka
{ततस्तत्पुनरावृत्तं युधिष्ठिरबलं महत्}
{दृष्ट्वा पार्थं महाबाहुं भीष्मायोद्यतमाहवे}


\twolineshloka
{ततो भीष्मःक कुरुश्रेष्ठ सिंहवद्विनदन्मुहुः}
{धनञ्जयरथं शीघ्रं शरवर्षैरवाकिरत्}


\twolineshloka
{क्षणेन स रथस्तस्य सहयः कसहसारथिः}
{शरवर्षेण महता संछन्नो न प्रकाशते}


\twolineshloka
{वासुदेवस्त्वसंभ्रान्तो धैर्यमास्थाय सत्त्ववान्}
{चोदयामास तानश्वान्विभिन्नान्भीष्मसायकैः}


\twolineshloka
{ततः पार्थो धनुर्गृह्य दिव्यं जलदनिःस्वनम्}
{पातयामास भीष्मस्य धनुश्छित्वा त्रिभिः शरैः}


\twolineshloka
{स च्छिन्नधन्वा कौरव्यः पुनरन्यन्महद्धनुः}
{निमिषान्तरमात्रेण सज्यं चक्रे पिता तव}


\twolineshloka
{विचकर्ष ततो दोर्भ्यां धनुर्जलदनिःस्वनम्}
{अथास्य तदपि क्रुद्धश्चिच्छेद धनुरर्जुनः}


\twolineshloka
{तस्य तत्पूजयामास लाघवं शन्तनोः सुतः}
{साधु पार्थ महाबाहो साधु भो पाण्डुनन्दन}


\twolineshloka
{त्वय्येवैतद्युक्तरूपं महत्कर्म धनंजय}
{प्रीतोऽस्मि सुभृशं पुत्र कुरु युद्धं मया सह}


\twolineshloka
{इति पार्थं प्रशस्याथ प्रगृह्यान्यन्महद्धनुः}
{मुमोच समरे वीरः शरान्पार्थरथं प्रति}


\twolineshloka
{अदर्शयद्वासुदेवो हययाने परं बलम्}
{मोघान्कुर्वञ्शरांस्तस्य मण्डलान्व्यचरल्लुघु}


\twolineshloka
{तथा भीष्मस्तु सुदृढं वासुदेवधनंजयौ}
{विव्याध निशितैर्बाणैः सर्वगात्रेषु भारत}


\twolineshloka
{शुशुभाते नरव्याघ्रौ तौ भीष्मशरविक्षतौ}
{गोवृषाविव संरब्धौ विषाणोल्लेखनाङ्क्तितौ}


\threelineshloka
{पुनश्चापि सुसंरब्धः शरैः शतसहस्रशः}
{कृष्णयोर्युधि संरब्धो भीष्माः प्राच्छादयद्दिशः}
{`पार्थोऽपि समरे क्रुद्धो भीष्मस्यावारयद्दिशः ॥'}


\twolineshloka
{वार्ष्णेयं च शरैस्तीक्ष्णैः कम्पयामास रोषितः}
{मुहुरभ्यर्दयन्भीष्मः प्रहस्य स्वनवत्तदा}


\twolineshloka
{ततस्तु कृष्णः समरे दृष्ट्वा भीष्मपराक्रमम्}
{संप्रेक्ष्य च महाबाहुः पार्थस्य मृदुयुद्धताम्}


\twolineshloka
{भीष्मं च शरवर्षाणि सृजन्तमनिशं युधि}
{प्रतपन्तमिवादित्यं मध्यमासाद्य सेनयोः}


\twolineshloka
{वारन्वरान्विनिघ्नन्तं पाण्डुपुत्रस्य सैनिकान्}
{युगान्तमिव कुर्वाणं भीष्मं यौधिष्ठिरे बले}


\twolineshloka
{अमृष्यमाणो भगवान्केशवः परवीरहा}
{अचिन्तयदमेयात्मा नास्ति यौधिष्ठिरं बलम्}


\twolineshloka
{एकाह्ना हि रणे भीष्मो नाशयेद्देवदानवान्}
{किं नु पाण्डुसुतान्युद्धे सबलान्सपदानुगान्}


\twolineshloka
{द्रवते च महासैन्यं पाण्डवस्य महात्मनः}
{एते च कौरवास्तूर्णं प्रभग्नान्वीक्ष्य सोमकान्}


\twolineshloka
{प्राद्रवन्ति रणे दृष्ट्वा हर्षयन्तः पितामहम्}
{सोहं भीष्मं निहन्म्यद्य पाण्डवार्थाय दंशितः}


\twolineshloka
{भारमेतं विनेष्यामि पाण्डवानां महात्मनाम्}
{अर्जुनो हि शरैस्तीक्ष्णैर्वध्यमानोऽपि संयुगे}


\threelineshloka
{कर्तव्यं नाभिजानाति रणे भीष्मस्य गौरवात्}
{तथा चिन्तयतस्तस्य भूय एव पितामहः}
{प्रेषयामास संक्रुद्धः शरान्पार्थरथं प्रति}


\twolineshloka
{तेषां बहुत्वासु भृसं शराणांदिशश्च सर्वाः पिहिता बभूवुः}
{न चान्तरिक्षं न दिशो न भूमि-र्न भास्करोऽदृश्यत रश्मिमालीक}


\twolineshloka
{ववुश्च वातास्तुमुलाः सधूमादिशश्च सर्वाः क्षुभिता बभूवुः}
{द्रोणो विकर्णोऽथ जयद्रथश्चभूरिश्रवाः कृतवर्मा कृपश्च}


\twolineshloka
{श्रुतायुरम्बष्ठपतिश्च राजाविन्दानुविन्दौ च सुदक्षिणश्च}
{प्राच्याश्च सौवीरगणाश्च सर्वेवसातयः क्षुद्रकमालवाश्च}


\twolineshloka
{किरीटिनं त्वरमाणाभिस्रु-र्निदेशगाः शान्तनवस्य राज्ञः}
{तं वाजिपादातरथौघजालै-रनेकसाहस्रशतैर्ददर्श}


\twolineshloka
{किरीटिनं संपरिवार्यमाणंशिनेर्नप्ता वारणयूथपैश्च}
{ततस्तु दृष्ट्वार्जुनवासुदेवौपदातिनागाश्वरथैः समन्तात्}


\twolineshloka
{अभिद्रुतौ शस्त्रभृतां वरिष्ठौशिनिप्रवीरोऽभिससार तूर्णम्}
{स तान्यनीकानि महाधनुष्मान्शिनिप्रवीरः सहसाऽभिपत्य}


\twolineshloka
{चकार साहाय्यमथार्जुनस्यविष्णुर्यथा वृत्रनिषूदनस्य}
{विशीर्णनागाश्वरथध्वजौघंभीष्मेण वित्रासितसर्वयोधम्}


\twolineshloka
{युधिष्ठिरानीकमभिद्रवन्तंप्रोवाच संदृश्य शिनिप्रवीरः}
{क्व क्षत्रिया यास्यथ नैष धर्मःसतां पुरस्तात्कथितः पुराणैः}


\twolineshloka
{मा स्वां प्रतिज्ञां त्यजत प्रवीराःस्वं वीरधर्मं परिपालयध्वम्}
{तान्वासवानन्तरजो निशाम्यनरेन्द्रमुख्यान्द्रवतः समन्तात्}


\twolineshloka
{पार्थस्य दृष्टवा मृदुयुद्धातां चभीष्मं च सङ्ख्ये समुदीर्यमाणम्}
{अमृष्यमाणः स ततो महात्मायशस्विनं सर्वदशार्हभर्ता}


\twolineshloka
{उवाच शैनेयमभिप्रशंसन्दृष्ट्वा कुरूनापततः समग्रान्}
{ये यान्ति ते यान्तु शिनिप्रवीरयेऽपि स्थिताः सात्वत तेऽपि यान्तु}


\twolineshloka
{भीष्मं रथात्पश्य निपात्यमानंद्रोणं च सङ्ख्ये सगणं मयाऽद्य}
{न सारथेः सात्वत कौरवाणांक्रुद्धस्य मुच्येत रणेऽद्य कश्चित्}


\twolineshloka
{तस्मादहं गृह्य रथाङ्गमुग्रंप्राणं हरिष्यामि महाव्रतस्य}
{निहत्य भीष्मं सगणं तथाजौद्रोणं च शैनेय रथप्रवीरौ}


\threelineshloka
{प्रीतिं करिष्यामि धनंजयस्यराज्ञश्च भीमस्य तथाऽश्विनोश्च}
{निहत्य सर्वान्धृतराष्ट्रपुत्रां-स्तत्पक्षिणो ये च नरेन्द्रमुख्याः}
{राज्येन राजानमजातशत्रुंसंपादयिष्याम्यहमद्य हृष्टः}


\twolineshloka
{` इतीदमुक्त्वा स महानुभावःसस्मार चक्रं निशितं पुराणम्}
{सुदर्शनं चिन्तितमात्रमेवतस्याग्रहस्तं स्वयमारुरोह}


\twolineshloka
{तच्चक्रपद्मं प्रगृहीतमाजौरराज नारायणबाहुनालम्}
{यथादिपद्मं तरुणार्कवर्णंरराज नारायणनाभिजातमक्}


\twolineshloka
{तत्कृष्णकोपोदयसूर्यबुद्धंक्षुरान्ततीक्ष्णाग्रसुजातपत्रम्}
{तेस्यैव देहोरुसरःप्ररूढंरराज नारायणबाहुनालम् ॥'}


\threelineshloka
{ततः सुनाभं वसुदेवपुत्रःसूर्यप्रभं वज्रसमप्रभावम्}
{क्षुरान्तमुद्यम्य भुजेन चक्रंरथादवप्लुत्य विसृज्य वाहान्}
{संकम्पयन्गां चरणैर्महात्मावेगेन कृष्मः प्रससार भीष्मम्}


\threelineshloka
{मदान्धमाजौ समुदीर्णदर्पंसिंहो जिघांसन्निव वारणेन्द्रम्}
{सोऽभिद्रवन्भीष्ममनीकमध्येक्रुद्धो महेन्द्रावरजः प्रमाथी}
{व्यालम्बिपीताम्बरधृक्ककाशेघनो यथा खे तडितावनद्धः}


\twolineshloka
{तमात्तचक्रं प्रणदन्तमुच्चैःक्रुद्धं महेन्द्रावरजं समीक्ष्य}
{सर्वाणि भूतानि भृशं विनेदुःक्षयं कुरूणामिव चिन्तयित्वा}


\twolineshloka
{स वासुदेवः प्रगृहीतचक्रःसंवर्तयिष्यन्निव सर्वलोकम्}
{अभ्युत्पतँल्लोकगुरुर्बभासेभूतानि धक्ष्यन्निव धूमकेतुः}


\threelineshloka
{तमाद्रवन्तं प्रगृहीतचक्रंदृष्ट्वा देवं शान्तनवस्तदानीम्}
{असंभ्रमं तद्विचकर्ष दोर्भ्यांमहाधनुर्गामडिवतुल्यघोषम्}
{उवाच भीष्मस्तमनन्तपौरुषंगोविन्दमाजावविमूढचेताः}


\twolineshloka
{एह्येहि देवेश जगन्निवासनमोस्तु ते माधव चक्रपाणे}
{प्रसह्य मां पातय लोकनाथरथोत्तमात्सर्वशरण्य सङ्ख्ये}


\twolineshloka
{त्वया हतस्यापि ममाऽद्य कृष्णश्रेयः परिस्मिन्निह चैव लोके}
{संभावितोऽस्म्यन्धकवृष्णिनाथलोकैस्त्रिभिर्वीर तवाभियानात्}


\twolineshloka
{रथादवप्लुत्य ततस्त्वरावान्पार्थोऽप्यनुद्रुत्य यदुप्रवीरम्}
{जग्राह पीनोत्तमलम्बबाहुंबाह्वोर्हरिं व्यायतपीनबाहुः}


\twolineshloka
{निगृह्यमणाश्च तदाऽऽदिदेवोभृश सरोषः किल नाम योगी}
{आदाय वेगेन जगाम विष्णु-र्जिष्णुं महावात इवैकवृक्षम्}


\twolineshloka
{पार्थस्तु विष्टभ्य बलेन पादौभीष्मान्तिकं तूर्णमभिद्रवन्तम्}
{बलान्निजग्राह हरिं किरीटीपदेऽथ राजन्दशमे कथंचित्}


\twolineshloka
{अवस्थितं च प्रणिपत्य कृष्णंप्रीतोऽर्जुनः काञ्चनचित्रमाली}
{उवाच कोपे प्रतिसंहरेतिगतिर्भवान्केशव पाण्डवानाम्}


\twolineshloka
{न हास्यते कर्म यथाप्रतिज्ञंपुत्रैः शपे केशव सोदरैश्च}
{अन्तं करिष्यामि यथा कुरूणांत्वयाऽहमिन्द्रानुजसंप्रयुक्तः}


\twolineshloka
{ततः प्रतिज्ञां समयं च तस्यजनार्दनः प्रीतमना निशम्य}
{स्थितः प्रिये कौरवसत्तमस्यरथं सचक्रः पुनरारुरोह}


\twolineshloka
{`ततः प्रतिज्ञां समवाप्य भीष्मःकृताञ्चलिः स्तुत्यमथाकरोद्वै}
{त्रैविक्रमे यस्य वपुर्बभासेतथैव दृष्ट्वा तु समुञ्ज्वलन्तम्}


\twolineshloka
{प्रगृह्य शङ्खं द्विषताकं निहन्तास तानभूषून्पुनराददानः}
{भीष्मं शरैः संपरिवार्य सङ्ख्येचिच्छेद भूरिश्रवसश्च चापम्}


\twolineshloka
{शल्यं च विद्ध्वा नवभिः पृषत्कै-र्दुर्योधं वक्षसि निर्बिभेद}
{'विनादयामास ततो दिशश्चस पाञ्चजन्यस्य रवेण शौरिः}


\twolineshloka
{व्याविद्धनिष्काङ्गदकुण्डलं तंरजोविकीर्णाञ्चितपद्मनेत्रम्}
{विशुद्धदंष्ट्रं प्रगृहीतशङ्खंविचुक्रुशुः प्रेक्ष्य कुरुप्रवीराः}


\twolineshloka
{मृदङ्गभेरीपणवप्रणादानेमिस्वना दुन्दुभिनिःस्वनाश्च}
{ससिंहनादाश्च बभूवुरुग्राःसर्वेष्वनीकेषु ततः कुरूणाम्}


\twolineshloka
{गाण्डीवघोषः स्तनयित्नुकल्पोजगाम पार्थस्य नभो दिशश्च}
{जग्मुश्च बाणा विमलाः प्रसन्नाःसर्वा दिशः पाण्डवचापमुक्ताः}


\twolineshloka
{तं कौरवाणामधिपो जवेनभीष्मेण भूरिश्रवसा च सार्धम्}
{अभ्युद्ययावुद्यतबाणपाणिःकक्षं दिधक्षन्निव धूमकेतुः}


\twolineshloka
{अथार्जुनाय प्रजिघाय भल्लान्भूरिश्रवाः सप्त सुवर्णपुङ्खान्}
{दुर्योदनस्तोमरमुग्रवेगंशल्यो गदां शान्तनवश्च शक्तिम्}


\twolineshloka
{स सप्तभिः सप्त शरप्रवेकान्संवार्य भीरिश्रवसा विसृष्टान्}
{शितेन दुर्योधनबाहुमुक्तंक्षुरेण तत्तोमरमुन्ममाथ}


\twolineshloka
{ततः शुभामापततीं स शक्तिंविद्युत्प्रभां शान्तनवेन मुक्ताम्}
{गदां च मद्राधिपबाहुमुक्तांद्वाभ्यां शराभ्यां निचर्त वीरः}


\twolineshloka
{ततो भुजाभ्यां बलवद्विकृष्यचित्रं धनुर्गाण्डिवमप्रमेयम्}
{माहेन्द्रमस्रं विधिवत्सुघोरंप्रादुश्चकाराद्भुतमन्तरिक्षे}


\twolineshloka
{तेनोत्तमास्त्रेण ततो महात्मासर्वाण्यनीकानि महाधनुष्मान्}
{शरौघजालैर्विमलाग्निवर्णै-र्निवारयामास किरीटमाली}


\threelineshloka
{शिलीमुखाः पार्थधनुःप्रमुक्तारथान्ध्वजाग्राणि धनूंषि बाहून्}
{निकृत्य देहान्विविशुः परेषांनरेन्द्रनागेन्द्रतुरङ्गमाणाम्}
{}


\twolineshloka
{ततो दिशः सोऽनुदिशश्च पार्थःशरैः सुधारैः समरे वितत्य}
{गाण्डीवशब्देन मनांसि तेषांकिरीटमाली व्यथयांचकार}


\twolineshloka
{तस्मिंस्तथा घोरतमे प्रवृत्तेशङ्खस्वना दुन्दुभिनिःस्वनाश्च}
{अन्तर्हिता गाण्डिवनिःस्वनेनबभूवुरुग्राश्वरथप्रणादाः}


\twolineshloka
{गाण्डीवशब्दं तमथो विदित्वाविराटराजप्रमुखाः प्रवीराः}
{पाञ्चालराजो द्रुपदश्च वीर-स्तं देशमाजग्मुरदीनसत्त्वाः}


\twolineshloka
{सर्वाणि सैन्यानि तु तावकानियतोयतो गाण्डिवजः प्रणादः}
{ततस्ततः सन्नतिमेव जग्मु-र्न तं प्रतीपोऽभिससार कश्चित्}


\twolineshloka
{तस्मिन्सुघोरे कनृप संप्रहारेहताः प्रवीराः सरथाश्वसूताः}
{गजाश्च नाराचनिपाततप्तामहापताकाः शुभरुक्मकक्ष्याः}


\twolineshloka
{परीतसत्वाः सहसा निपेतुःकिरीटिना भिन्नतनुत्रकायाः}
{दृढं हताः पात्रिभिरुग्रवेगैःपार्थेन भल्लैर्विमलैः शिताग्नैः}


\twolineshloka
{निकृत्तयन्त्रा निहतेन्द्रकीलाध्वजा महान्तो ध्वजिनिमुखेषु}
{पदातिसङ्घाश्च रथाश्च सङ्ख्येहयाश्च नागाश्च धनञ्जयेन}


\twolineshloka
{बाणाहतास्तूर्णमपेतसत्वाविष्टभ्य गात्राणि निपेतुरुर्व्याम्}
{ऐन्द्रेण तेनास्त्रवरेण राजन्महाहवे भिन्नतनुत्रदेहाः}


\twolineshloka
{ततः शरौघैर्निशितैः किरीटिनानृदेहशस्त्रक्षतलोहितोदा}
{नदी सुघोरा नरमेदफेनाप्रवर्तिता तत्र रणाजिरे वै}


\twolineshloka
{वेगेन साऽतीव पृथुप्रवाहापरेतनागाश्वशरीररोधाः}
{नरेन्द्रमञ्जोच्छ्रितमांसपङ्काःप्रभूतरक्षोगणभूतसेविता}


\twolineshloka
{शिरःकपालाकुलकेशशाद्वलाशरीरसङ्घातसहस्रवाहिनी}
{विशीर्णनानाकवचोर्मिसंकुलानराश्वनागास्थिनिकृत्तशर्करा}


\twolineshloka
{श्वकङ्कशालावृकगृध्रकाकैःक्रव्यादसङ्घैश्च तरक्षुभिश्च}
{उपेतकूलां ददृशुर्मनुष्याःक्रूरां महावैतरणीप्रकाशाम्}


\twolineshloka
{प्रवर्तितामर्जुनबाणसङ्घै-र्मेदोवसासृकप्रवहां सुभीमाम्}
{हतप्रवीरां च तथैव दृष्ट्वासेनां कुरूणामथ फल्गुनेन}


\twolineshloka
{ते चेदिपाञ्चालकरूषमत्स्याःपार्थाश्च सर्वे सहिताः प्रणेदुः}
{जयप्रगल्भाःक पुरुषप्रवीराःसंत्रासयन्तः कुरुवीरयोधान्}


\twolineshloka
{हतप्रवीराणि बलानि दृष्ट्वाकिरीटिना शत्रुभयावहेन}
{वित्रास्य सेनां ध्वजिनीपतीनांसिंहो मृगाणामिव यूथसङ्घान्}


\twolineshloka
{विनेदतुस्तावतिहर्षयुक्तौगाण्डीवधन्वा च जनार्दनश्च}
{ततो रविं संवृतरश्मिजालंदृष्ट्वा भृशं शस्त्रपरिक्षताङ्गाः}


\twolineshloka
{तदैन्द्रमस्त्रं विततं च घोर-मसह्यमुद्वीक्ष्य युगान्तकल्पम्}
{अथापयानं कुरवः सभीष्माःसद्रोणदुर्योधनबाह्लिकाश्च}


\twolineshloka
{चक्रुर्निशां सन्धिगतां समीक्ष्यविभावसोर्लोहितरागयुक्ताम्}
{अवापय कीर्ति च यशश्च लोकेविजित्य शत्रूंश्च धनंजयोऽपि}


\twolineshloka
{ययौ नरेन्द्रैः सहसोदरैश्चसमाप्तकर्मा शिबिरं निशायाम्}
{ततः प्रजज्ञे तुमुलः कुरूणांनिशामुखे घोरतमः प्रणादः}


\twolineshloka
{रणे रथानामयुतं निहत्यहता गजाः सप्तशतार्जुनेन}
{प्राज्याश्च सौवीरगणाश्च सर्वेनिपातिताः क्षुद्रकमालवाश्च}


\twolineshloka
{महत्कृतं कर्म धनञ्जयेनकर्तुं यथा नार्हति कश्चिदन्यः}
{श्रुतायुरम्बष्ठपतिश्च राजातथैव दुर्मर्षणचित्रसेनौ}


\twolineshloka
{द्रोणः कृपः सैन्धवबाह्लिकौ चभूरिश्रवाः शल्यशलौ च राजन्}
{अन्ये च योधाः शतशः समेताःक्रुद्धेन पार्थेन रणस्य मध्ये}


\twolineshloka
{स्वबाहुवीर्येण जिताः सभीष्माःकिरीटिना लोकमहारथेन}
{इति ब्रुवन्तः शिबिराणि जग्मुःसर्वे गणा भारत ये त्वदीयाः}


\twolineshloka
{उल्कासहस्रैश्च सुसंप्रदीप्तै-र्विभ्राजमानैश्च तथा प्रदीप्रैः}
{किरीटिवित्रासितसर्वयोधाचक्रे निवेशं ध्वजिनी कुरूणाम्}


\chapter{अध्यायः ६०}
\twolineshloka
{सञ्जय उवाच}
{}


\twolineshloka
{व्युष्टां निशां भारत भारताना-मनीकिनीनां प्रमुखे महात्मा}
{ययौ सपत्नान्प्रति जातकोपोवृतः समग्रेण बलेन भीष्मः}


\twolineshloka
{तं द्रोणदुर्योधनबाह्लिकाश्चतथैव दुर्मर्षणचित्रसेनौ}
{जयद्रथश्चातिबलो बलौघै-र्नृपास्तथान्ये प्रययुः समन्तात्}


\twolineshloka
{स तैर्महद्भिश्च महारथैश्चतेजस्विभिर्वीर्यवद्भिश्च राजन्}
{रराज राजा स तु राजमुख्यै-र्वृतः सदेवैरिव वज्रपाणिः}


\twolineshloka
{तस्मिन्ननीकप्रमुखे विषक्तादोधूयमानाश्च महापताकाः}
{सुरक्तपीतासितपाण्डुराभामहागजस्कन्धगता विरेजुः}


\twolineshloka
{सा वाहिनी शान्तनवेन गुप्तामहारथैर्वारणवाजिभिश्च}
{बभौ सविद्युत्स्तनयित्नुकल्पाजलागमे द्यौरिव जातमेघा}


\twolineshloka
{ततो रणायाभिमुखी प्रयाताप्रत्यर्जुनं शान्तनवाभिगुप्ता}
{सेनामहोग्ना सहसा कुरूणांवेगो यथा भीम इवापगायाः}


\twolineshloka
{तं व्यालनानाविधगूढसारंगजाश्वपादातरथौघपक्षम्}
{व्यूहं महामेघसमं महात्माददर्श दूरात्कपिराजकेतुः}


\twolineshloka
{विनिर्ययौ केतुमता रथेननरर्षभः श्वेतहयेन काले}
{` जये धृतः शत्रुवरूथिनीनांयथा सुरेन्द्रोऽसुरवाहिनीनाम्}


\twolineshloka
{नारायणेनेन्द्र इवाभिगुप्तःशशीव सूर्येण समेयिवान्यथा}
{तथा महात्मा सह केशवेनवरीथिनीनां प्रमुखे रराज '}


\twolineshloka
{सोपस्करं सोत्तरबन्धुरेषंयत्तं यदूनामृषभेण सङ्ख्ये}
{कपिध्वजं प्रेक्ष्य रथं विषेदुःसहैव पुत्रैस्तव कौरवेयाः}


\twolineshloka
{प्रकल्पितं गुप्तमुदायुधेनकिरीटिना लोकमहारथेन}
{तं व्यूहराजं ददृशुस्त्वदीया-श्चतुश्चतुर्व्यालसहस्रकीर्णम्}


\twolineshloka
{यथैव पूर्वेऽहनि धर्मराज्ञाव्यूहः कृतः कौरवसत्तमेन}
{यथा न भूतो भुवि मानुषेषुन दृष्टपूर्वो न च संश्रुतश्च}


\twolineshloka
{ततो यथादेशमुपेत्य तस्थुःपाञ्चालमुख्याः सह चेदिमुख्यैः}
{ततः समादेशसमाहतानिभेरीसहस्राणि विनेदुराजौ}


\twolineshloka
{शङ्खस्वनास्तूर्यरवाः प्रणेदुःसर्वेष्वनीकेषु ससिंहनादाः}
{ततः सबाणानि महास्वनानिविस्फार्यमाणानि धनूंषि वीरैः}


\twolineshloka
{क्षणेन भेरीपणवप्रणादा-नन्तर्दधुः शङ्खमहास्वनांश्च}
{तच्छङ्खशब्दावृतमन्तरिक्ष-मुद्धूतभौमद्रुतरेणुजालम्}


\twolineshloka
{महानुभावाश्च ततः प्रकाश-नालोक्य वीराः सहसाऽभिपेतुः}
{रथी रथेनाभिहतः ससूतःपपात साश्वः सरथः सकेतुः}


\twolineshloka
{गजो गजेनाभिहताः पपातपदातिना चाभिहतः पदातिः}
{आवर्तमानान्यभिवर्तमानै-र्घोरीकृतान्यद्भुतदर्शनानि}


\twolineshloka
{प्रासैश्च खङ्गैश्च समाहतानिसदश्वबृन्दानि सदश्वबृन्दैः}
{सुवर्णतारागणभूषितानिसूर्यप्रभाभानि शरावराणि}


\twolineshloka
{विदार्यमाणानि परश्वथैश्चप्रासैश्च स्वङ्गैश्च निपेतुरुर्व्याम्}
{गजैर्विषणापरगात्ररुग्णाःकेचित्ससूता रथिनः प्रपेतुः}


\twolineshloka
{गजर्षभाश्चापि जगर्षभेणनिपातिता बाणहताः पृथिव्याम्}
{गजौघवेगोद्धतसादितानांश्रुत्वा विषेदुः सहसा मनुष्याः}


\twolineshloka
{आर्तस्वनं सादिपदातियूनांविषाणगात्रावरताडितानाम्}
{संभ्रान्तनागाश्वरथे मुहूर्तेमहाक्षये सादिपदातियूनाम्}


\twolineshloka
{महारथैः संपरिवार्यमाणंसंदृश्य दूरात्कपिराजकेतुम्}
{तं पञ्चतालोच्छ्रिततालकेतुःसदश्ववेगाद्भुतवीर्ययानः}


\twolineshloka
{महास्त्रबाणाशनिदीप्तमन्तंकिरीटिनं शान्तनवोऽभ्यधावत्}
{तथैव शक्रप्रतिमप्रभाव-मिन्द्रात्मजं द्रोणमुखा विसस्नुः}


\twolineshloka
{कृपश्च शल्यश्च विविंशतिश्चदुर्योधनः सौमदत्तिश्च राजन्}
{ततो रथानां प्रमुखादुपेत्यसर्वास्त्रिवित्काञ्चनचित्रवर्मा}


\twolineshloka
{जवेन शूरोऽभिससार सर्वां-स्तानर्जुनस्यात्मभवोऽभिमन्युः}
{तेषां महास्त्राणि महारथाना-मसह्यकर्मा विनिहत्य कार्ष्णिः}


\twolineshloka
{बभौ महामन्त्रहुतार्चिमालीसदोगतः सन्भगवानिवाग्निः}
{ततः स तूर्णं रुधिरोदफेनांकृत्वा नदीमाशु रणे रिपूणाम्}


\twolineshloka
{जगाम सौभद्रमतीत्य भीष्मोमहारथं पार्थमदीनसत्वः}
{ततः प्रहस्याद्भुतविक्रमेणगाण्डीवमुक्तेन शिलाशितेन}


\twolineshloka
{विपाठजालेन महास्त्रजालंविनाशयामास किरीटमाली}
{तमुत्तमं सर्वधनुर्धराणा-मसक्तकर्मा कपिराजकेतुः}


\twolineshloka
{भीष्मं महात्माऽभिववर्ष तूर्णंशरौघजालैर्विमलैश्च भल्लैः}
{तथैव भीष्माहतमन्तरिक्षेमहास्त्रजालं कपिराजकेतोः}


\threelineshloka
{विशीर्यमाणं ददृशुस्त्वदीयादिवाकरेणेव तमोभिभूतम्}
{एवंविधं कार्मुकभीमनाद-मदीनवत्सत्पुरुषोत्तमाभ्याम्}
{ददर्श लोकः कुरुसृंजयाश्चतद्द्वैरथं भीष्मधनञ्जयाभ्याम्}


\chapter{अध्यायः ६१}
\twolineshloka
{सञ्जय उवाच}
{}


\twolineshloka
{द्रौणिर्भूरिश्रवाः शत्यश्चित्रसेनश्च मारिषः}
{पुत्रः सांयमनेश्चैव सौभद्रं पर्यवारयन्}


\twolineshloka
{संसक्तमतितेजोभिस्तमेकं ददृशुर्जनाः}
{पञ्चभिर्मनुजव्याघ्रैर्गजैः सिंहशिशुं यथा}


\twolineshloka
{नातिलक्ष्यतया कश्चिन्न शौर्ये न पराक्रमे}
{बभूव सदृशः कार्ष्णेर्नास्त्रे नापि च लाघवे}


\twolineshloka
{तथा तमात्मजं युद्धे विक्रमन्तमरिन्दमम्}
{दृष्ट्वा पार्थः सुसंयत्तं सिंहनादमथानदत्}


\twolineshloka
{पीडयानं तु तत्सैन्यं पौत्रं तव विशांपते}
{दृष्ट्वा त्वदीया राजेन्द्र समन्तात्पर्यवारयन्}


\twolineshloka
{ध्वजिनीं धार्तराष्ट्राणां दीनशत्रुरदीनवत्}
{प्रत्युद्ययौ स सौभद्रस्तेजसा च बलेन च}


\threelineshloka
{तस्य लाघवमार्गस्थमादित्यसदृशप्रभम्}
{व्यदृश्यत महच्चापं समरे युध्यतः परैः}
{}


\twolineshloka
{स द्रौणिमिषुणैकेन विद्ध्वा शल्यं च पञ्चभिः}
{ध्वजं सांयमनेश्वैव सोऽष्टाभिश्चिच्छिदे ततः}


\twolineshloka
{रुक्मदण्डां महाशक्तिं प्रेषितां सौमदत्तिन}
{शितेनोरगसंकाशां पत्रिणाभिजघान ताम्}


\twolineshloka
{शल्यस्य च महावेगानस्यतः समरे शरान्}
{`धनुश्चिच्छेद भल्लेन तीव्रवेगन फाल्गुनिः'जघानार्जुनदायादश्चतुर्भिश्चतुरो हयान्}


\twolineshloka
{भूरिश्रवाश्च शल्यश्च द्रौणिः सांयमनिः शलः}
{नाभ्यवर्तन्तं संरब्धाः कार्ष्णेर्बाहुबलोदयात्}


\twolineshloka
{ततस्त्रिगर्ता राजेन्द्र मद्राश्च सह केकयैः}
{पञ्चविंशतिसाहस्रास्तव पुत्रेण चोदिताः}


\twolineshloka
{धनुर्वेदविदो मुख्या अजेयाः शत्रुभिर्युधि}
{सह पुत्रं जिघांसन्तं परिवव्रुः किरीटिनम्}


\twolineshloka
{तौ तु तत्र पितापुत्रौ परिक्षिप्तौ महारथौ}
{ददर्श राजन्पाञ्चल्यः सेनापतिररिन्दम}


\twolineshloka
{स वारणरथोघानां सहस्रैर्बहुभिर्वृतः}
{वाजिभिः पत्तिभिश्चैव वृतः शतसहस्रशः}


\twolineshloka
{`पारावताश्वः स रथमास्थाय परवीरहा'धनुर्विष्फार्य संक्रुद्धो नोदयित्वा च वाहिनीम्}
{ययौ तं मद्रकानीकं केकयांश्च परंतप}


\twolineshloka
{तेन कीर्तिमता गुप्तमनीकं दृढधन्वना}
{संरब्धरथनागाश्वं योत्स्यमानमशोभत}


\twolineshloka
{सोऽर्जुनप्रमुखे यान्तं पाञ्चालकुलवर्धनः}
{त्रिभिः शारद्वतं बाणैर्जत्रदेशे समार्पयत्}


\twolineshloka
{ततः स मद्रकान्हत्वा दशैव दशभिः शरैः}
{पृष्ठरक्षं जघानाशु भल्लेन कृतवर्मणः}


\twolineshloka
{दमनं चापि दायादं पौरवस्य महात्मनः}
{जघान विमलाग्नेण नाराचेन परंतपः}


\twolineshloka
{ततः सांयमनेः पुत्रः पाञ्चाल्यं युद्धदुर्मदम्}
{अविध्यत्रिंशता बाणैर्दशभिश्चास्य सारथिम्}


\twolineshloka
{सोऽतिविद्धो महेष्वासः सृक्किणी परिसंलिहन्}
{भल्लेन भृशतीक्ष्णेन निचकर्तास्य कार्मुकम्}


\twolineshloka
{अथैनं पञ्चविंशत्या क्षिप्रमेव समार्पयत्}
{अश्वांश्चास्यावधीद्राजन्नुभौ तौ पार्ष्णिसारथी}


\twolineshloka
{स हताश्वे रते तिष्ठन्ददर्श भरतर्षभ}
{पुत्रः सांयमनेः पुत्रं पाञ्चाल्यस्य महात्मनः}


\twolineshloka
{स प्रगृह्य महाघोरं निस्त्रिंशवरमायसम्}
{पदातिस्तूर्णमानर्च्छद्रथस्यं पुरुषर्षभः}


\twolineshloka
{तं महौघमिवायान्तं स्वात्पतन्तमिवोरगम्}
{भ्रान्तावरणनिस्त्रिंशं कालोत्सृष्टमिवान्तकम्}


\twolineshloka
{दीप्यमानमिवादित्यं मत्तवारणविक्रमम्}
{अपश्यन्पाण्डवास्तत्र धृष्टद्युम्नश्च पार्षतः}


\twolineshloka
{तस्य पाञ्चालदायादः प्रतीपमभिधावतः}
{शितनिस्त्रिंशहस्तस्य शरावरणधारिणः}


\twolineshloka
{बाणवेगमतीतस्य तथाभ्याशमुपेयुषः}
{त्वरन्सेनापतिः क्रुद्धो बिभेद गदया शिरः}


\twolineshloka
{तस्य राजन्सनिस्त्रिंशं सुप्रभं च शरावरम्}
{हतस्य पततो हस्ताद्वेगेन न्यपतद्भुवि}


\twolineshloka
{तं निहत्य गदाग्रेण स लेभे परमां मुदम्}
{पुत्रः पाञ्चालराजस्य महात्मा भीमविक्रमः}


\twolineshloka
{तस्मिन्हते महेष्वासे शल्यपुत्रे महारथे}
{हाहाकारो महानासीत्तव सैन्यस्य मारिष}


\twolineshloka
{ततः सांयमनिः क्रुद्धो दृष्ट्वा निहतामात्मजम्}
{अभिदुद्राव वेगेन पाञ्चाल्यं युद्धदुर्मदम्}


\twolineshloka
{तौ तत्र समरे शूरौ समेतौ युद्धदुर्मदौ}
{ददृशुः सर्वराजानः कुरवः पाण्डवास्तथा}


\twolineshloka
{ततः सांयमनिः क्रुद्धः पार्षतं परवीरहा}
{आजघान त्रिभिर्बाणैस्तोत्रैरिव महाद्विपम्}


\twolineshloka
{तथैव पार्षतं शूरं शल्यः समितिशोभनः}
{आजघानोरसि क्रुद्धस्ततो युद्धमवर्तत}


\chapter{अध्यायः ६२}
\twolineshloka
{धृतराष्ट्र उवाच}
{}


\twolineshloka
{दैवमेव परं मन्ये पौरुषादपि सञ्जय}
{यत्सैन्यं मम पुत्रस्य पाण्डुपुत्रेण वध्यते}


\twolineshloka
{नित्यं हि मामकांस्तात हतानेव हि शंससि}
{अव्यग्रांश्च प्रहृष्टांश्च नित्यं शंशसि पाण्डवान्}


\threelineshloka
{`विभग्नांश्च प्रनष्टांश्च नित्यं शंससि मामकान्}
{'हीनान्पुरुषकारेण मामकानद्य सञ्जय}
{पातितान्पात्यमानांश्च हतानेव च शंससि}


\twolineshloka
{युध्यमानान्यथाशक्ति घटमानाञ्जयं प्रति}
{पाण्डवा हि जयन्त्येव जीयन्ते चैव मामकाः}


\twolineshloka
{सोऽहं तीव्राणि दुःस्वानि दुर्योधनकृतानि च}
{श्रोष्यामि सततं तात दुःसहानि बहूनि च}


\threelineshloka
{तमुपायं न पश्यामि जीयेरन्येन पाण्डवान्}
{मामका विजयं युद्धे प्राप्नुयुर्येन सञ्जय ॥सञ्जय उवाच}
{}


\twolineshloka
{क्षयं मनुष्यदेहानां गजवाजिरथक्षयम्}
{श्रृणु राजन्स्थिरो भूत्वा तवैवापनयो महान्}


\twolineshloka
{धृष्टद्युम्नस्तु शल्येन पीडितो नवभिः शरैः}
{पीडयामास संक्रुद्धो मद्राधिपतिमायसैःक}


\twolineshloka
{तत्राद्भुतमपश्याम पार्षस्य पराक्रमम्}
{न्यवारयत यस्तूर्णं शल्यं समितिशोभनम्}


\twolineshloka
{नान्तरं दृश्यते तत्र तयोश्च रथिनोस्तदा}
{मुहूर्तमिव तद्युद्धं तयोः सममिवाभवत्}


\twolineshloka
{ततः शल्यो महाराज धृष्टद्युम्नस्य संयुगे}
{धनुश्चिच्छेद भल्लेन पीतेन निशितेन च}


\twolineshloka
{अथैनं शरवर्षेण च्छादयामास संयुगे}
{गिरिं जलागमे यद्वञ्जलदा जलवृष्टिभिः}


\twolineshloka
{अभिमन्युस्ततः क्रुद्धो धृष्टद्युम्ने च पीडिते}
{अभिदुद्राव वेगेन मद्रराजरथं प्रति}


\twolineshloka
{ततो मद्राधिपस्थं कार्ष्णिः प्राप्यातिकोपनः}
{आर्तायनिममेयात्मा विव्याध निशितैः शरैः}


\twolineshloka
{ततस्तु तावका राजन्परीप्सन्तोऽर्जुनं रणे}
{मद्रराजरथं तूर्णं परिवार्यावतस्थिरे}


\twolineshloka
{दुर्योधनो विकर्णश्च दुःशासनविविंशती}
{दुर्मर्षणो दुःसहश्च चित्रसेनोऽथ दुर्मुखः}


\twolineshloka
{सत्यव्रतश्च भद्रं ते पुरुमित्रश्च भारत}
{एते मद्राधिपरथं पालयन्तः स्थिता रणे}


\twolineshloka
{तान्भीमसेनः संक्रुद्धो धृष्टद्युम्नश्च पार्षतः}
{द्रौपदेयाभिमन्युश्च माद्रीपुत्रौ च पाण्डवौ}


\twolineshloka
{धार्तराष्ट्रान्दश रथान्दशैव प्रत्यवारयन्}
{नानारूपाणि शस्त्राणि विसृजन्तो विशांपते}


\twolineshloka
{अभ्यवर्तन्तं संहृष्टाः परस्परवधैषिणः}
{ते वै समेयुः संग्रामे राजन्दुर्मन्त्रिते तव}


\twolineshloka
{तस्मिन्दशरथे क्रुद्धे वर्तमाने महाभये}
{तावकानां परेषां वा प्रेक्षका रथिनोऽभवन्}


\twolineshloka
{शस्त्राण्यनेकरूपाणि विसृजन्तो महारथाः}
{अन्योन्यमभिनर्दन्तः संप्रहारं प्रचक्रिरे}


\twolineshloka
{ते तदा जातसंरम्भाः सर्वेऽन्योन्यं जिघांसवः}
{अन्योन्यमभिमर्दन्तः स्पर्धमानाः परस्परम्}


\twolineshloka
{अन्योन्यस्पर्धया राजञ्ज्ञातयः संगता मिथः}
{महास्राणि विमुञ्चन्तः समापेतुरमर्षिणः}


\twolineshloka
{दुर्योधनस्तु संक्रुद्धो धृष्टद्युम्नं महारणे}
{विव्याध निशितैर्बाणैश्चतुर्भिः समरे द्रुतम्}


\twolineshloka
{दुर्मर्षणश्च विंशत्या चित्रसेनश्च प़ञ्चभिः}
{दुर्मुखो नवभिर्बाणैर्दुःसहश्चापि सप्तभिः}


\twolineshloka
{विविंशतिः पञ्चभिश्च त्रिभिर्दुःशासनस्तथा}
{तान्प्रत्यविध्यद्राजेन्द्र पार्षतः शत्रुतापनः}


\twolineshloka
{एकैकं पञ्चविंशत्या दर्शयन्पाणिलाघवम्}
{सत्यव्रतं च समरे पुरुमित्रं च भारत}


\twolineshloka
{अभिमन्युरविध्यत्तु दशभिर्दशभिः शरैः}
{माद्रीपुत्रौ तु समरे मातुलं मातृनन्दनौ}


\twolineshloka
{अविध्येतां शरैस्तीक्ष्णैस्तदद्भुतमिवाभवत्}
{ततः शल्यो महाराज स्वस्त्रीयौ रथिनां वरौ}


\twolineshloka
{शरैर्बहुभिरानर्च्छत्कृतप्रतिकृतैषिणौ}
{छाद्यमानौ ततस्तौ कतु माद्रीपुत्रौ न चेलतुः}


\twolineshloka
{अथ दुर्योधनं दृष्ट्वा भीमसेनो महाबलः}
{विधित्सुः कलहस्यान्तं गदां जग्राह पाण्डवः}


\twolineshloka
{तमुद्यतगदं दृष्ट्वा कैलासमिव श्रृङ्गिणम्}
{भीमसेनं महाबाहुं पुत्रास्ते प्राद्रवन्भयात्}


\twolineshloka
{दुर्योधनस्तु संक्रुद्धो मागधं समचोदयत्}
{अनीकं दशसाहस्रं कुञ्जराणां तरस्विनाम्}


\twolineshloka
{गजानीकेन सहितस्तेन राजा सुयोधनाः}
{मागधं पुरतः कृत्वा भीमसेनं समभ्ययात्}


\twolineshloka
{आपतन्तं च तं दृष्ट्वा गजानीकं वृकोदरः}
{गदापाणिरवारोहद्रथांत्सिंह इवोन्नदन्}


\twolineshloka
{अद्रिसारमयीं गुर्वीं प्रगृह्य महतीं गदाम्}
{अभ्यधावद्गजानीकं व्यादितास्य इवान्तकः}


\twolineshloka
{स गजान्गदया निघ्नन्व्यचरत्समरे बली}
{भीमसेनो महाबाहुः सवज्र इव वासवः}


\twolineshloka
{तस्य नादेन महता मनोहृदयकम्पिना}
{व्यत्यचेष्टन्त संहत्य गजा भिमस्य गर्जतः}


\twolineshloka
{ततस्तु द्रौपदीपुत्राः सौभद्रश्च महारथः}
{नकुलः सहदेवश्च धृष्टद्युम्नश्च पार्षतः}


\twolineshloka
{पृष्ठं भीमस्य रक्षन्तः शरवर्षेण वारणान्}
{अभ्यवर्षन्त धावन्तो मेघा इव गिरिव्रजान्}


\threelineshloka
{नाकुलिस्तु शतानीकः समरे शत्रुपूगहा}
{क्षुरैः क्षुरप्रैर्भल्लैश्च पीतैश्चाञ्जलिकैः शितैः}
{न्यहनच्चोत्तमाङ्गानि पाण्डवो गजयोधिनाम्}


\twolineshloka
{शिरोभिः प्रपतद्भिश्च बाहुभिश्च विभूषितैः}
{अश्मवृष्टिरिवाभाति पाणिभिश्च सहाङ्कुशैः}


\twolineshloka
{हृतोत्तमाङ्गाः स्कन्धेषु गजानां गजयोधिनः}
{अदृश्यन्ताचलाग्रेकषु द्रुमा भग्नशिखा इव}


\twolineshloka
{धृष्टद्युम्नहतानन्यानपश्याम महागजान्}
{पततः पात्यमानांश्च पार्षतेन महात्मना}


\twolineshloka
{मागधोऽथ महीपालो गजमैरावणोपमम्}
{प्रेषयामास समरे सौभद्रस्य रथं प्रति}


\twolineshloka
{तमापतन्तं संप्रेक्ष्य मागधस्य महागजम्}
{जघानैकेषुणा वीरःत सौभद्रः परवीरहा}


\twolineshloka
{तस्यावर्जितनागस्य कार्ष्णिः परपुरंजयः}
{राज्ञो रजतपुङ्खेन भल्लेनापाहरच्छिरः}


\twolineshloka
{विगाह्य तद्गजानीकं भीमसेनोऽपि पाण्डवः}
{व्यचरत्समरे मृद्गन्गजानिन्द्रो गिरीनिव}


\twolineshloka
{एकप्रहारनिहतान्भीमसेनेन दन्तिनः}
{अपश्याम रणे तस्मिन्गिरीन्वज्रहतानिव}


\twolineshloka
{भग्नदन्तान्भग्नकरान्भग्नसक्थीश्च वारणान्}
{भग्नपृष्ठत्रिकानन्यान्निहतान्पर्वतोपमान्}


\twolineshloka
{नदतः सीदतश्चान्यान्विमुखान्समरे गतान्}
{विद्रुतान्भयसंविग्नांस्तथा विशकृतोऽपरान्}


\twolineshloka
{भीमसेनस्य मार्गेषु पतितान्पर्वतोपमान्}
{अपश्यं निहतान्नागान्राजन्निष्ठीवतोपरान्}


\twolineshloka
{वमन्तो रुधिरं चान्ये भिन्नकुम्भा महागजाः}
{विह्नलन्तो गता भूमिं शैला इव धरातले}


\twolineshloka
{मेदोरुधिरदिग्धाङ्गो वसामज्जासमुक्षितः}
{व्यचरत्समरे भीमो दण्डपाणिरिवान्तकः}


\twolineshloka
{गजानां रुधिरक्लिन्नां गदां बिभ्रद्वृकोदरः}
{घोरः प्रतिभयश्चासीत्पिनाकीव पिनाकभृत्}


\twolineshloka
{संमथ्यमानाः क्रुद्धेन भीमसेनेन दन्तिनः}
{सहसा प्राद्रवन्क्लिष्ठा मृद्गन्तस्तव वाहिनीम्}


\twolineshloka
{तं हि वीरं महेष्वासं सौभद्रप्रमुखा रथाः}
{पर्यरक्षन्त युध्यन्तं वज्रायुधमिवामराः}


\twolineshloka
{शोणिताक्तां गदां बिभ्रदुक्षितां गजशोणितैः}
{कृतान्त इव रोद्रात्मा भीमसेनो व्यदृश्यत}


\twolineshloka
{व्यायच्छमानं गदया दिक्षु सर्वासु भारत}
{अपश्याम रणे भीमं नृत्यन्तमिव शङ्करम्}


\twolineshloka
{यमदण्डोपमां गुर्वीमिन्द्राशनिसमकस्वनाम्}
{अपश्याम महाराय रौद्रां विशसनीं गदाम्}


\twolineshloka
{विमिश्रां केशमञ्जाभिः प्रदिग्धां रुधिरेण च}
{पिनाकमिव रुद्रस्य क्रुद्धस्याभिघ्नतः पशून्}


\twolineshloka
{यथा पशूनां सङ्घातं यष्ट्या पालः प्रकालयेत्}
{तथा भीमो गजानीकं गदया समकालयत्}


\twolineshloka
{गदया वध्यमानास्ते मार्गणैश्च समन्ततः}
{स्वान्यनीकानि मृद्गन्तः प्राद्रवन्कुञ्जरास्तव}


\twolineshloka
{महावात इवाभ्राणि विधमित्वा स वारणान्}
{अतिष्ठत्तुमुले भीमः श्मशान इव शूलभृत्}


\chapter{अध्यायः ६३}
\twolineshloka
{सञ्जय उवाच}
{}


\twolineshloka
{हते तस्मिन्गजानीके पुत्रो दुर्योधनस्तव}
{भीमसेनं ध्नतेत्येवं सर्वसैन्यान्यचोदयत्}


\twolineshloka
{ततः सर्वाण्यनीकानि तव पुत्रस्य शासनात्}
{अभ्यद्रवन्भीमसेनं नदन्तं भैरवान्रवान्}


\twolineshloka
{तं बलौघमपर्यन्तं देवैरपि सुदुःसहम्}
{आपतन्तं सुदुष्पारं समुद्रिमिव पर्वणि}


\twolineshloka
{रथनागाश्वकलिलं शङ्खदुन्दुभिनादितम्}
{अनन्तरथपादातं रजसा सर्वतो वृतम्}


\twolineshloka
{तं भीमसेनः समरे महोदधिमिवापरम्}
{सेनासागरमक्षोभ्यं वेलेव समवारयत्}


\twolineshloka
{तदाश्चर्यमपश्याम पाण्डवस्य महात्मनः}
{भीमसेनस्य समरे राजन्कर्मातिमानुषम्}


\twolineshloka
{उदीर्णान्पार्थिवान्सर्वान्साश्वान्सरथकुञ्जरान्}
{असंभ्रमं भीमसेनो गदया समवारयत्}


\twolineshloka
{स संवार्य बलौघांस्तन्गदया रथिनां वरः}
{अतिष्ठत्तुमुले भीमो गिरिर्मेरुरिवाचलः}


\twolineshloka
{तस्मिन्सुतुमुले घोरे काले परमदारुणे}
{भ्रातरश्चैव पुत्राश्च धृष्टद्युम्नश्च पार्षतः}


\twolineshloka
{द्रौपदेयाभिमन्युश्च शिखण्डी चापराजितः}
{न प्राजहन्भीमसेनं भये जाते महाबलम्}


\twolineshloka
{ततः शैक्यायसीं गुर्वीं प्रगृह्य महतीं गदाम्}
{अधावत्तावकान्योधान्दण्डपाणिरिवान्तकः}


\twolineshloka
{पोययन्रथबृन्दानि वाजिबृन्दनि चाभिभूः}
{कर्षयन्रथवृन्दानि बाहुवेगेन पाण्डवः}


\twolineshloka
{विनिध्रन्यचतत्सङ्ख्ये युगान्ते कालविद्विभुः}
{ऊरुवेगेन संकर्षन्रथजालानि पाण्डवः}


\twolineshloka
{बलानि संममर्दाशु नड्वलानीव कुञ्जरः}
{मृद्गन्रथेभ्यो रथिनो गजेभ्यो गजयोधिनः}


\twolineshloka
{सादिनश्चाश्वपृष्ठेभ्यो भूमौ वापि पदातिनः}
{गदया व्यवमत्सर्वान्वातो वृक्षानिवौजसा}


\twolineshloka
{भीमसेनो महाबाहुस्तव पुत्रस्य वै बले}
{सापि प्रञ्जानसामासैः प्रदिग्धा रुधिरेण च}


\twolineshloka
{अदृश्यत महारौद्रा गदा नागाश्वपातनी}
{तत्रतत्र हतैश्चापि मनुष्यगजवाजिभिः}


\twolineshloka
{रणाङ्गणं समभवन्मृत्योरावाससन्निभम्}
{पिनाकमिव रुद्रस्य क्रुद्धस्याभिघ्नतः पशून्}


\twolineshloka
{यमदण्डोपमामुग्रामिन्द्राशनिसमस्वनाम्}
{ददृशुर्भीमसेनस्य रौद्रां विशसनीं गदाम्}


\twolineshloka
{आविध्यतो गदां तस्य कौन्तेयस्य महात्मनः}
{बभौ रूपं महाघोरं कालस्येव युगक्षये}


\twolineshloka
{तं तथा महतीं सेनां द्रावयन्तं पुनः पुनः}
{दृष्ट्वा मृत्युमवायान्तं सर्वे विमनसोऽभवन्}


\twolineshloka
{यतोयतः प्रेक्षते स्म गदामुद्यम्य पाण्डवः}
{तेनतेन स्म दीर्यन्ते सर्वसैन्यानि भारत}


\twolineshloka
{प्रदारयन्तं सैन्यानि बलेनामितविक्रमम्}
{ग्रसमानमनीकानि व्यादितास्यमिवान्तकम्}


\twolineshloka
{तं तथा भीमकर्माणं प्रगृहीतमहागदम्}
{दृष्ट्वा वृकोदरं भीष्मः सहसैव समभ्ययात्}


\twolineshloka
{महता रथघोषेण रथेनादित्यवर्चसा}
{छादयञ्शरवर्षेण पर्जन्य इव वृष्टिमान्}


\twolineshloka
{तमायान्तं तथा दृष्ट्वा व्यात्ताननमिवान्तकम्}
{भीष्मं भीमो महाबाहुः प्रत्युदीयादमर्षितः}


\twolineshloka
{तस्मिन्क्षणे सात्यकिः सत्यसन्धःशिनिप्रवीरोऽभ्यपतत्पितामहम्}
{निघ्नन्नमित्रान्धनुषा दृढेनसंकम्पयंस्तव पुत्रस्य सैन्यम्}


\twolineshloka
{तं यान्तमश्वै रजतप्रकाशैःशरान्वपन्तं निशितान्सुपुङ्खान्}
{नाशक्नवन्धारयितुं तदानींसर्वे गणा भारत ये त्वदीयाः}


\twolineshloka
{अविध्यदेनं दशभिः पृषत्कैः-रलम्बुसो राक्षसोऽसौ तदानीम्}
{शरैश्चतुर्भिः प्रतिविद्ध्य तं चनप्ता शिनेरभ्यपतद्रथेन}


\twolineshloka
{अन्वागतं वृष्णिवरं निशम्यतं शत्रुमध्ये परिवर्तमानम्}
{प्रद्रावयन्तं कुरुपुङ्गवांश्चपुनः पुनश्च प्रणदन्तमाजौ}


\twolineshloka
{योधास्त्वदीयाः शरवर्षैरवर्ष-न्मेघा यथा भूधरमम्बुवेगैः}
{तथापि तं धारयितुं न शेकु-र्मध्यंदिने सूर्यमिवातपन्तम्}


\twolineshloka
{न तत्र कश्चिन्नविषण्ण आसी-दृते राजन्सोमदत्तस्य पुत्रात्}
{स वै समादाय धनुर्महात्माभिरिश्रवा भारत सौमदत्तिः}


% Check verse!
दृष्ट्वा रथान्स्वान्व्यपनीयमाना-न्प्रत्युद्ययौ सात्यकिं योद्धुमिच्छन्
\chapter{अध्यायः ६४}
\twolineshloka
{सञ्जय उवाच}
{}


\twolineshloka
{ततो भूरिश्रवा राजन्सात्यकिं नवभिः शरैः}
{प्राविध्यद्भृशसंक्रुद्धस्तोत्रैरिव महाद्विपम्}


\twolineshloka
{कौरवं सात्यकिश्चैव शरैः सन्नतपर्वभिः}
{अवारयदमेयात्मा सर्वलोकस्य पश्यतः}


\twolineshloka
{ततो दुर्योधनो राजा सोदर्यैः परिवारितः}
{सोमदत्तिं रणे यत्तः समन्तात्पर्यवारयत्}


\twolineshloka
{तं चैव पाण्डवाः सर्वे सात्यकिं रभसं रणे}
{परिवार्य स्थिताः सङ्ख्ये समन्तात्सुमहौजसः}


\twolineshloka
{भीमसेनस्तु संक्रुद्धो गजामुद्यम्य भारत}
{दुर्योधनमुखान्सर्वान्पुत्रांस्ते पर्यवारयत्}


\twolineshloka
{रथैरनेकसाहस्रैः क्रोधामर्षसमन्वितः}
{नन्दकस्तव पुत्रस्तु भीमसेनं महाबलम्}


\twolineshloka
{विव्याध विशिखैः षड्भिः कङ्कपत्रैः शिलाशितैः}
{दुर्योधनस्तदा राजन्भीमसेनं महारथम्}


\twolineshloka
{आजघानोरसि क्रुद्धो मार्गणैर्नवभिः शितैः}
{ततो भीमो महाबाहुः स्वरथं सुमहाबलः}


\twolineshloka
{आरुरोह रथश्रेष्ठं विशोकं चेदमब्रवीत्}
{एते महारथाः शूरा धार्तराष्ट्राः समागताः}


\twolineshloka
{मामेव भृशसंक्रुद्धा हन्तुमभ्युद्यता युधि}
{मनोरथद्रुमोऽस्माकं चिन्तितो बहुवार्षिकः}


\twolineshloka
{सफलः सूत चाद्येह योऽहं पश्यामि सोदरान्}
{यत्राशोकसमुत्क्षिप्ता रेणवो रथनेमिभिः}


\twolineshloka
{प्रयास्यन्त्यन्तरिक्षं हि शरबृन्दैर्दिगन्तरे}
{तत्र तिष्ठति सन्नद्धः स्वयं राजा सुयोधनः}


\twolineshloka
{भ्रातरश्चास्य सन्नद्धाः कुलपुत्रा मदोत्कटाः}
{एतानद्य हनिष्यामि पश्यतस्ते न संशयः}


\twolineshloka
{तस्मान्ममाश्वान्संग्रमे यत्तः संयच्छ सारथे}
{एवमुक्त्वा ततः पार्थस्तव पुत्रं विशांपते}


\twolineshloka
{विव्याध निशितैस्तीक्ष्णैः शरैः कनकभूषणैः}
{नन्दकं च त्रिभिर्भाणैरभ्यविध्यत्स्तनान्तरे}


\twolineshloka
{तं तु दुर्योधनः षष्ट्या विद्ध्वा भीमं महाबलम्}
{त्रिभिरन्यैः सुनिशितैर्विशोकं प्रत्यविध्यत}


\twolineshloka
{भिमस्य च रणे राजन्धनुश्चिच्छेद भासुरम्}
{मुष्टिदेशे भृशं तीक्ष्णौस्त्रिभिर्भल्लैर्हसन्निव}


\twolineshloka
{समरे प्रेक्ष्य यन्तारं विशोकं तु वृकोदरः}
{पीडितं विशिखैस्तीक्ष्णैस्तव पुत्रेण धन्विना}


\twolineshloka
{अमृष्यमाणः संरब्धो धनुर्दिव्यं परामृशम्}
{पुत्रस्य ते महाराज वधार्थं भरतर्षभ}


\twolineshloka
{समादधत्सुसंक्रुद्धः क्षुरप्रं रोमवाहिनम्}
{तेन चिच्छेद नृपतेर्भीमः कार्मुकमुत्तमम्}


\twolineshloka
{सोऽपविद्ध्य धनुश्छिन्नं पुत्रस्ते क्रोधमूर्च्छितः}
{अन्यत्कार्मुकमादत्त सत्वरं वेगवत्तरम्}


\twolineshloka
{संदधे विशिखं घोर कालमृत्युसमप्रभम्}
{तेनाजघान संक्रुद्धो भीमसेनं स्तनान्तरे}


\twolineshloka
{स गाढविद्धो व्यथितः स्यान्दनोपस्थ आविशत्}
{स निषण्णो रथोपस्थे मूर्च्छामभिजगाम ह}


\twolineshloka
{तं दृष्ट्वा व्यथितं भीममभिमन्युपुरोगमाः}
{नामृष्यन्त महेष्वासाः पाण्डवानां महारथः}


\twolineshloka
{ततस्तु तुमुलां वृष्टिं शस्त्राणां तिग्मतेजसाम्}
{पातयामासुरव्यग्राः पुत्रस्य तव मूर्धनि}


\twolineshloka
{प्रतिलभ्य ततः संज्ञां भीमसेनो महाबलः}
{दुर्योधनं त्रिभिर्विद्ध्वा पुनर्विव्याध पञ्चभिः}


\twolineshloka
{शल्यं च पञ्चविंशत्या शरैर्विव्याध पाण्डवः}
{रुक्मपुङ्खैर्महेष्वासः स विद्धो व्यपयाद्रणात्}


\twolineshloka
{प्रत्युद्ययुस्ततो भीमं तव पुत्राश्चतुर्दश}
{सेनापतिः सुषेणश्च जलसन्धः सुलोचनः}


\twolineshloka
{उद्रो भीमरथो भीमो वीरबाहुरलोलुपः}
{दुर्मुखो दुष्प्रधर्षश्च विवित्सुर्विकटः समः}


\twolineshloka
{विसृजन्तो बहून्बाणान्क्रोधसंरक्तलोचनाः}
{भीमसेनमभिद्रुत्य विव्यधुः सहिता भृशम्}


\twolineshloka
{पुत्रांस्तु तव संप्रेक्ष्य भीमसेनो महाबलः}
{सृक्किणी विलिहन्वीरः पशुमध्ये यथा वृकः}


\twolineshloka
{अभिपत्य महाबाहुर्गरुत्मानिव वेगितः}
{सेनापतेः क्षुरप्रेण शिरश्चिच्छेद पाण्डवः}


\twolineshloka
{संप्रहस्य च हृष्टात्मा त्रिभिर्बाणैर्महाभुजः}
{जलसन्धं विनिर्भिद्य सोऽनयद्यमसादनम्}


\twolineshloka
{सुषेणं च ततो हत्वा प्रेषयामास मृत्यवे}
{उग्रस्य सशिरस्त्राणं शिरश्चन्द्रोपमं भुवि}


\twolineshloka
{पातयामास भल्लेन कुण्डलाभ्यां विभूषितम्}
{वीरबाहुं च सप्तत्या साश्वकेतुं ससारथिम्}


\twolineshloka
{निनाय समरे वीरः परलोकाय पाण्डवः}
{भीमं भीमरथं चोभौ भीमसेनो हसन्निव}


\twolineshloka
{पुत्रौ ते दुर्मदौ राजन्ननयद्यमसादनम्}
{ततः सुलोचनं भीमः क्षुरप्रेण महामृधे}


\twolineshloka
{मिषतां सर्वसैन्यानामनयद्यमसादनम्}
{पुत्रास्तु तव तं दृष्ट्वा भीमसेनपराक्रमम्}


\twolineshloka
{शेषा येऽन्येऽभवंस्तत्र ते भीमस्य भयार्दिताः}
{विप्रद्रुता दिशो राजन्वध्यमाना महात्मना}


\twolineshloka
{ततोऽब्रवीच्छान्तनवः सर्वानेव महारथान्}
{एष भीमो रणे क्रुद्धो धार्तराष्ट्रान्महारथान्}


\twolineshloka
{यथा प्राग्र्यान्यथा ज्येष्ठान्यथा शूरांश्च संगतान्}
{निपातयत्युग्रधन्वा तं प्रगृह्णीत माचिरम्}


\twolineshloka
{एवमुक्त्वा ततः सर्वे धार्तराष्ट्रस्य सैनिकाः}
{अभ्यद्रवन्त संक्रुद्धा भीमसेनं महाबलम्}


\twolineshloka
{भगदत्तः प्रभिन्नेन कुञ्जरेण विशांपते}
{अभ्ययात्सहसा तत्रयत्र भीमो व्यवस्थितः}


\twolineshloka
{आपतन्नेव च रणे भीमसेनं शिलीमुखैः}
{अदृश्यं समरे चक्रे जीमूत इव भास्करम्}


\twolineshloka
{अभिमन्युमुखास्तत्तु नामृष्यन्त महारथाः}
{भीमस्याच्छादनं शङ्ख्ये स्वबाहुबलमाश्रिताः}


\twolineshloka
{त एनं शरवर्षेण समन्तात्पर्यवारयन्}
{6-64-46bगजं च शरवृष्ट्यातु बिभिदुस्ते समन्ततः}


\twolineshloka
{स शस्त्रवृष्ट्याऽभिहतः समस्तैस्तैर्महारथैः}
{प्राग्ज्योतिषगजो राजन्नानालिङ्गैः सुतेजनैः}


\twolineshloka
{संजातरुधिरोत्पीडः प्रेक्षणीयोऽभवद्रणे}
{गभस्तिभिरिवार्कस्य संस्यूतो जलदो महान्}


\twolineshloka
{संचोदितो मदस्रावी भगदत्तेन वारणः}
{अभ्यधावत तन्सर्वान्कालोत्सृष्ट इवान्तकः}


\twolineshloka
{द्विगुणं जवमास्थाय कम्पयंश्चरणैर्महीम्}
{तस्य तत्सुमहद्रूपं दृष्ट्वा सर्वे महारथाः}


\twolineshloka
{असह्यं मन्यमानाश्च नातिप्रमनसोऽभवन्}
{ततस्तु नृपतिः क्रुद्धो भीमसेनं स्तनान्तरे}


\threelineshloka
{आजघान महाराज शरेणानतपर्वणा}
{सोऽतिविद्धो महेष्वासस्तेन राज्ञा महारथः}
{}


\twolineshloka
{मूर्च्छयाऽभिपरीतात्मा ध्वजयष्टिं समाश्रयत्}
{तांस्तु भीतान्समालक्ष्य भीमसेनं च मूर्च्छितम्}


\twolineshloka
{ननाद बलवन्नादं भगदत्तः प्रतापवान्}
{ततो घटोत्कचो राजन्प्रेक्ष्य भीमं तथाऽऽगतं}


\twolineshloka
{संक्रुद्धो राक्षसो घोरस्तत्रैवान्तरधीयत}
{स कृत्वा दारुणां मायां भूरूणां भयवर्धिनीम्}


\twolineshloka
{अदृश्यत निमेषार्धाद्धोररूपं समास्थितः}
{ऐरावतं समारूढः स वै मायाकृतं स्वयम्}


\twolineshloka
{तस्य चान्येऽपि दिङ्वागा बभूवुरनुयायिनः}
{अञ्जनो वमानश्चैव महापद्मश्च सुप्रभः}


\twolineshloka
{त्रय एते महानागा राक्षसैः समधिष्ठिताः}
{महाकायास्त्रिधा राजन्प्रस्रवन्तो मदं बहु}


\twolineshloka
{तेजोवीर्यबलोपेता महाबलपराक्रमाः}
{घटोत्कचस्तु स्वं नागं चोदयामास तं तदा}


\twolineshloka
{सगजं भगदत्तं तु हन्तुकामः परंतपः}
{ते चान्ये चोदिता नागा राक्षसैस्तैर्महाबलैः}


\twolineshloka
{परिपेतुः सुसंरब्धाश्चतुर्दंष्ट्राश्चतुर्दिशम्}
{भगदत्तस्य तं नागं विषाणैरभ्यपीडयन्}


\twolineshloka
{स पीड्यमानस्तैर्नागैर्वेदनार्तः शराहतः}
{अनदत्सुमहानादमिन्द्राशनिसमस्वनम्}


\twolineshloka
{`व्यवर्तत महाघोषो भैमसेनिशरार्दितः}
{मृदित्वा सर्वसैन्यानि तव पुत्रस्य भारत'}


\twolineshloka
{तस्य तं नदतो नादं सुघोरं भीमनिःश्वनम्}
{श्रुत्वा भीष्मोऽब्रवीद्द्रोणं राजानं च सुयोधनम्}


\twolineshloka
{एष युध्यति संग्रामे हैडिम्बेन दुरात्मना}
{भगदत्तो महेष्वासः कृच्छ्रे च परिवर्तते}


\twolineshloka
{राक्षसश्च महामायः स च राजाऽतिकोपनः}
{एतौ समेतौ समरे कालमृत्युसमावुभौ}


\twolineshloka
{श्रीयते चैव हृष्टानां पाण्डवानां महास्वनः}
{हस्तिनश्चैव सुमहान्भीतस्य रुदितध्वनिः}


\twolineshloka
{तत्र गच्छाम भद्रं वो राजानं परिरक्षितुम्}
{अरक्ष्यमाणः समरे क्षिप्रं प्राणान्विमोक्ष्यति}


\twolineshloka
{ते त्वरध्वं महावीर्याः किं चिरेण प्रयामहे}
{महान्हि वर्तते रौद्रः संग्रामो रोमहर्षणः}


\twolineshloka
{भक्तश्च कुलपुत्रश्च शूरश्च पृतनापतिः}
{युक्तं तस्य परित्राणं कर्तुमस्माभिरच्युत}


\twolineshloka
{भीष्मस्य तद्वचः श्रुत्वा सर्व एव महारथाः}
{द्रोणभीष्मौ पुरस्कृत्य भगदत्तपरीप्सया}


\twolineshloka
{उत्तमं जवमास्थाय प्रययुर्यत्र सोऽभवत्}
{तान्प्रयातान्समालोक्य युधिष्ठिरपुरोगमाः}


\twolineshloka
{पञ्चालाः पाण्डवैः सार्धं पृष्ठतोऽनुययुः परान्}
{तान्यनीकान्यथालोक्य राक्षसेन्द्रः प्रतापवान्}


\twolineshloka
{ननाद सुमहानादं विस्फोटमशनेरिव}
{तस्य तं निनदं श्रुत्वा दृष्ट्वा नागांश्च युध्यतः}


\twolineshloka
{भीष्मः शान्तनवो भूयो भारद्वाजमभाषत}
{न रोचते मे संग्रामो हैडिम्बेन दुरात्मना}


\twolineshloka
{बलवीर्यमसाविष्टः ससहायश्च सांप्रतज्}
{नैष शक्यो युधा जेतुमपि वज्रभृता स्वयम्}


\twolineshloka
{लब्धलक्षः प्रहारी च वयं च श्रान्तवाहनाः}
{पाञ्चालैः पाण्डवेयैश्च दिवसं क्षतविक्षताः}


\threelineshloka
{इदानीं युधि निर्जेतुं न शक्योऽसौ स राक्षसः}
{अस्तमभ्येति सविता रात्रौ योद्धुं कन शक्यते}
{अवहारमतः कुर्मः श्वो योत्स्यामः परै सह}


\twolineshloka
{पितामहवचः श्रुत्वा यथा चक्रुःस्म कौरवाः}
{उपायेनापयानं ते घटोत्कचभयार्दिताः}


\twolineshloka
{कौरवेषु निवृत्तेषु पाण्डवा जितकाशिनः}
{सिंहनादान्भृशं चक्रुः शङ्खान्दध्मुश्च भारत}


\twolineshloka
{एवं तदभवद्युद्धं दिवसं भरतर्षभ}
{पाण्डवानां कुरूणां च पुरस्कृत्य घटोत्कचम्}


\twolineshloka
{कौरवास्तु ततो राजन्प्रययुः शिबिरं स्वकम्}
{व्रीडमाना निशाकाले पाण्डवेयैः पराजिताः}


\twolineshloka
{शरविक्षतगात्रास्तु पाण्डपपुत्रा महारथाः}
{युद्धे सुमनसो भूत्वा जग्मुः स्वशिबिरं प्रति}


\twolineshloka
{पुरस्कृत्य महाराज भीमसेनघटोत्कचौ}
{पूजयन्तस्तदान्योन्यं मुदा परमया युताः}


\twolineshloka
{नदन्तो विविधान्नादांस्तूर्यस्वनविमिश्रितान्}
{सिंहनादांश्च कुर्वन्तो विमिश्राञ्शङ्खनिः स्वनैः}


\threelineshloka
{विनदन्तो महात्मानः कम्पयन्तश्च मेदिनीम्}
{घट्टयन्तश्च मर्माणि तव पुत्रस्य मारिष}
{}


\twolineshloka
{प्रयाताः शिबिरायैव निशाकाले परंतप}
{दुर्योधनस्तु नृपतिर्दीनो भ्रातृवधेन च}


\threelineshloka
{मुहूर्तं चिन्तयामास बाष्पशोकसमाकुलः}
{ततः कृत्वा विधिं सर्वं शिबिरस्य यथाविधि}
{प्रदध्यौ शोकसंतप्तो भ्रातृव्यसनकर्शितः}


\chapter{अध्यायः ६५}
\twolineshloka
{धृतराष्ट्र उवाच}
{}


\twolineshloka
{भयं मे सुमहञ्जातं विस्मयश्चैव सञ्जय}
{श्रुत्वा पाण्डुकुमाराणां कर्म देवैः सुदुष्करम्}


\twolineshloka
{पुत्राणां च पराभावं श्रुत्वा सञ्जय सर्वशः}
{चिन्ता मे महती सूत भविष्यति कथं त्विति}


\twolineshloka
{ध्रुवं विदुरवाक्यानि धक्ष्यन्ति हृदयं मम}
{तथा हि दृश्यते सर्वं दैवयोगेन सञ्जय}


\twolineshloka
{यत्र भीष्ममुखान्सर्वाञ्शस्त्रज्ञान्योधसत्तमान्}
{पाण्डवानामनीकेषु योधयन्ति प्रहारिणः}


\twolineshloka
{केनावध्या महात्मानः पाण्डुपुत्रा महाबलाः}
{केन दत्तवरास्तात किं वा ज्ञानं विदन्ति ते}


\twolineshloka
{येन क्षयं न गच्छन्ति दिवि तारागणा इव}
{पुनःपुनर्न मृष्यामि हतं सैन्यं तु पाण्डवैः}


\twolineshloka
{मय्येव दण्डः पतति दैवात्परमदारुणः}
{यथाऽवध्याः पाण्डुसुता यथा वाध्याश्च मे सुताः}


\twolineshloka
{एतन्मे सर्वमाचक्ष्व याथातथ्येन सञ्जय}
{न हि पारं प्रपश्यामि दुःखस्यास्य कथंचन}


\twolineshloka
{समुद्रस्येव महतो भुजाभ्यां प्रतरन्नरः}
{पुत्राणां व्यसनं मन्ये ध्रुवं प्राप्तं सुदारुणम्}


\twolineshloka
{घातयिष्यति मे स्रवान्पुत्रान्भीमो न शंसयः}
{न हि पश्यामि तं वीरं यो मे रक्षेत्सुतान्रणे}


\twolineshloka
{ध्रुवं विनाशः संप्राप्तः पुत्रामां मम सञ्जय}
{तस्मान्मे कारणं सूत शक्तिं चैव विशेषतः}


\twolineshloka
{पृच्छतो वै यथातत्त्वं सर्वमाख्यातुमर्हसि}
{दुर्योधनश्च यच्चक्रे दृष्ट्वा स्वान्विमुखान्रणे}


\twolineshloka
{भीष्मद्रोणौ कृपश्चैव सौबलश्च जयद्रथः}
{द्रौणिर्वापि महेष्वासो विकर्णो वा महाबलः}


\threelineshloka
{निश्चयो वापि कस्तेषां तदा ह्यासीन्महात्मानाम्}
{विमुखेषु महाप्राज्ञ मम पुत्रेषु सञ्जय ॥सञ्जय उवाच}
{}


\twolineshloka
{श्रृणु राजन्नवहितः श्रुत्वा चैवावधारय}
{नैव मन्त्रकृतं किंचिन्नैव मायां तथाविधाम्}


\twolineshloka
{न वै विभीषिकां कांचिद्राजन्कुर्वन्ति पाण्डवाः}
{युध्यन्ति ते यथान्यायं शक्तिमन्तश्च संयुगे}


\twolineshloka
{धर्मेण सर्वकार्याणि जीवितादीनि भारत}
{आरभन्ते सदा पार्थाः प्रार्थयाना महद्यशः}


\twolineshloka
{न ते युद्धान्निवर्तन्ते धर्मोपेता महाबलाः}
{श्रिया परमया युक्ता यतो धर्मस्ततो जयः}


\twolineshloka
{तेनावध्या रणे पार्था जययुक्ताश्च पार्थिव}
{तव पुत्रा दुरात्मानः पापेष्वभिरताः सदा}


\twolineshloka
{निष्ठुरा हीनकर्माणस्तेन हीयन्ति संयुगे}
{सुबहूनि नृशंसानि पुत्रैस्तव जनेश्वरे}


\twolineshloka
{निकृतानीह पाण्डूनां नीचैरिव यथा नरैः}
{सर्वं च तदनादृत्य पुत्राणां तव किल्बिषम्}


\threelineshloka
{धर्ममूलाः सदैवासन्पाण्डवाः पाण्डुपूर्वज}
{न चैतान्बहुमन्यन्ते पुत्रास्तव विशांपते}
{}


\twolineshloka
{तस्य पापस्य सततं क्रियमाणस्य कर्मणः}
{सांप्रतं सुमहद्धोरं फलं प्राप्तं जनेश्वर}


\twolineshloka
{स त्वं भुङ्क्ष्व महाराज सपुत्रः समुहृञ्जनः}
{नावबुध्यसि यद्राजन्वार्यमाणः सुहृज्जनैः}


\twolineshloka
{विदुरेणाथ भीष्मेण द्रोणेन न महात्मना}
{तथा मया चाप्यसकृद्वार्यमाणो न बुध्यसे}


\twolineshloka
{वाक्यं हितं च पथ्यं च मर्त्यः पथ्यमिवौषधम्}
{पुत्राणां मतमास्थाय जितान्मन्यसि पाण्डवान्}


\twolineshloka
{शृणु भूयो यथातत्त्वं यन्मां त्वं परिपृच्छसि}
{कारणं भरतश्रेष्ठ पाण्डवानां जयं प्रति}


\twolineshloka
{तत्तेऽहं कथयिष्यामि यथाश्रुतमरिन्दम}
{दुर्योधनेन संपृष्ट एतमर्थं पितामहः}


\twolineshloka
{दृष्ट्वा भ्रातॄन्रणे सर्वान्निर्जितांस्तु महारथान्}
{शोकसंमूढहृदयो निशाकाले स्म कौरवः}


\threelineshloka
{पितामहं महाप्राज्ञं विनयोपगम्य ह}
{यदब्रवीत्सुतस्तेऽसौ तन्मे शृणु जनेश्वर ॥दुर्योधन उवाच}
{}


\twolineshloka
{भवान्द्रोणश्च कर्णश्च कृपो द्रौणिस्तथैव च}
{कृतवर्मा च हार्दिक्यः काम्भोजश्च सुदक्षिणः}


\twolineshloka
{भूरिश्रवा विकर्णश्च भगदत्तश्च वीर्यवान्}
{महारथाः समाख्याताः कुलपुत्रास्तनुत्यजः}


\twolineshloka
{त्रयाणामपि लोकानां पर्याप्ता इति मे मतिः}
{पाण्डवानां समस्ताश्च न तिष्ठन्ति पराक्रमे}


\threelineshloka
{तत्र मे संशयो जातस्तत्त्वमाचक्ष्व पृच्छतः}
{यं समाश्रित्य कौन्तेय जयन्त्यस्मान्क्षणेक्षणे ॥भीष्म उवाच}
{}


\twolineshloka
{शृणु राजन्वचो मह्यं यथा वक्ष्यामि कौरव}
{बहुशश्च मयोक्तोऽसि न च मे तत्त्वया कृतम्}


\twolineshloka
{क्रियतां पाण्डवैः सार्ध समो भरतसत्तम}
{एतत्क्षेममहं मन्ये पृथिव्यास्तव वा विभो}


\twolineshloka
{भुङ्क्ष्वेमां पृथिवीं राजन्भ्रातृभिः सहितः सुखी}
{दुर्हृदस्तापयन्सर्वान्नन्दयंश्चापि बान्धवान्}


\twolineshloka
{न च मे क्रोशतस्तात श्रुतवानसि वै पुरा}
{तदिदं समनुप्राप्तं यत्पाण्डूनवमन्यसे}


\twolineshloka
{यश्च हेतुरवध्यत्वे तेकषामक्लिष्टकर्मणाम्}
{तं शृणुष्व महाबाहो मम कीर्तयतः प्रभो}


\twolineshloka
{नास्ति लोकेषु तद्भूतं भविता नो भविष्यति}
{यो जयेत्पाण्डवान्सर्वान्पालिताञ्छार्ङ्गधन्वना}


\twolineshloka
{यत्तु मे कथितं तात मुनिभिर्भावितात्मभिः}
{पुराणगीतं धर्मज्ञ तच्छृणुष्व यथातथम्}


\twolineshloka
{पुरा किल सुराः सर्वे ऋषयश्च समागताः}
{पितामहमुपासेदुः पर्वत गन्धमादने}


\twolineshloka
{तेषां मध्ये समासीनः प्रजापतिरपश्यत}
{विमानं प्रज्वलद्भासा स्थितं प्रवरमम्बरे}


\twolineshloka
{ध्यानेनावेद्य तद्ब्रह्मा कृत्वा च नियतोऽञ्जलिम्}
{नमश्चकार हृष्टात्मा पुरुषं परमेश्वरम्}


\twolineshloka
{ऋषयस्त्वथ देवाश्च दृष्ट्वा ब्रह्माणमुत्थितम्}
{स्थिताः प्राञ्जलयः सर्वे पश्यन्तो महदद्भुतम्}


\twolineshloka
{यथावच्च तमभ्यर्च्य ब्रह्मा ब्रह्मविदां वरः}
{जगाद जगतः स्रष्टा परं परमधर्मवित्}


\twolineshloka
{विश्वावसुर्विश्वमूर्तिर्हि विश्वेविष्वक्सेनो विश्वकर्मा वशी च}
{विश्वेश्वरो वासुदेवोऽसि तस्मा-द्योगात्मानं देवतं त्वामुपैमि}


\twolineshloka
{जय विश्वमहादेव जय लोकहिते रत}
{जय योगीश्वर विभो जय योगपरावर}


\twolineshloka
{पद्मगर्भविशालाक्ष जय लोकेश्वरेश्वर}
{भूतभव्यभवन्नाथ जय सौम्यात्मजात्मज}


\twolineshloka
{असङ्क्येयगुणाधार जय सर्वपरायण}
{नारायण सुदुष्पार जय शार्ङ्गधनुर्धर}


\twolineshloka
{जय सर्वगुणोपेत विश्वमूर्ते निरामय}
{विश्वेश्वर महाबाहो जय लोकार्थतत्पर}


\twolineshloka
{महोरगवराहाद्य हरिकेश विभो जय}
{हरिवासदिशामीश विश्ववासामिताव्यय}


\twolineshloka
{व्यक्ताव्यक्तामितस्थान नियतेन्द्रिय सत्त्क्रिय}
{असङ्ख्येयात्मभावज्ञ जय गम्भीर कामद}


\twolineshloka
{अनन्त विदित ब्रह्मन्नित्यभूतविभावन}
{कृतकार्य कृतप्रज्ञ धर्मज्ञ विजयावह}


\twolineshloka
{गुह्यात्मन्सर्वयोगात्मन्स्फुटसंभूतसंभव}
{भूताद्य लोकत्त्वेश जय भूतविभावन}


\twolineshloka
{आत्मयोगे महाभाग कल्पसंक्षेपतत्पर}
{उद्भावनमनोभाव जय ब्रह्म जयप्रिय}


\twolineshloka
{निसर्गसर्गनिरत कामेश परमेश्वर}
{अमृतोद्भवसद्भाव मुक्तात्मन्विजयप्रद}


\twolineshloka
{प्रजापतिपते देव पद्मनाभ महाबल}
{आत्मभूतमहाभूत सत्वात्मञ्जय सर्वदा}


\twolineshloka
{पादौ तव धा देवी दिशो बाहु दिवं शिरः}
{मूर्तिस्तेऽहं सुराः कायश्चन्द्रादित्यौ च चक्षुषी}


\twolineshloka
{बलं तपश्च सत्यं च कर्म धर्मात्मकं तव}
{तेजोऽग्निः पवनः श्वास आपस्ते स्वेदसंभवाः}


\twolineshloka
{अश्विनौ श्रवणौ नित्यं देवी जिह्वा सरस्वती}
{वेदाः संस्कारनिष्ठा हि त्वयीदं जगदाश्रितम्}


\twolineshloka
{न सङ्ख्यानं परीमाणं न तेजो न पराक्रमम्}
{न बलं योगयोगीश जानीमस्ते परंतप}


\twolineshloka
{त्वद्भक्तिनिरता देव नियमैस्त्वां समाश्रिताः}
{अर्चयामः सदा विष्णो परमेशं महेश्वरम्}


\twolineshloka
{ऋषयो देवगन्धर्वा यक्षराक्षसपन्नगाः}
{पिशाचा मानुषाश्चैव मृगपक्षिसरीसृपाः}


\twolineshloka
{एवमादि मया सृष्टं पृथिव्यां त्वत्प्रसादजम्}
{पद्मनाभ विशालाक्ष कृष्ण दुःश्वप्रणाशन}


\twolineshloka
{त्वं गतिः सर्वभूतानां त्वं नेता त्वं जगद्गुरुः}
{त्वत्प्रसादेन देवेश सुखिनो विबुधाः सदा}


\twolineshloka
{पृथिवी निर्भया देव त्वत्प्रसादात्सदाऽभवत्}
{तस्माद्भव विशालाक्ष यदुवंशविवर्धनः}


\twolineshloka
{धर्मसंस्थापनार्थाय दैत्यानां च बधाय च}
{जगतो धारणार्थाय विज्ञाप्यं कुरु मे विभो}


\twolineshloka
{यत्तत्परमकं गुह्यं त्वत्प्रसादादिदं विभो}
{वासुदेव तदेतत्ते मयोद्गीतं यथातथम्}


\twolineshloka
{सृष्ट्वा संकर्षणं देवं स्वयमात्मानमात्मना}
{कृष्ण त्वमात्मनास्राक्षीः प्रद्युम्नं चात्मसंभवम्}


\twolineshloka
{प्रद्युम्नादनिरुद्धं त्वं यं विदुर्विष्णुमव्ययम्}
{अनिरुद्धोऽसृजन्मां वै ब्रह्माणं लोकधारिणम्}


\threelineshloka
{वासुदेवमयः सोऽहं त्वयैवास्मि विनिर्मितः}
{तस्माद्याचामि लोकेश चतुरात्मानमात्मना}
{विभज्य भागशोत्मानं व्रज मानुषतां विभो}


\twolineshloka
{तत्रासुरवधं कृत्वा सर्वलोकसुखाय वै}
{धर्मं प्राप्य यशः प्राप्य योगं प्राप्स्यति तत्त्वतः}


\twolineshloka
{त्वां हि ब्रह्मर्षयो लोके देवाश्चामितविक्रम}
{तैस्तैर्हि नामभिर्युक्ता गायन्ति परमात्मकम्}


\twolineshloka
{स्थिताश्च सर्वे त्वयि भूतसङ्घाःकृत्वाश्रयं त्वां वरदं सुबाहो}
{अनादिमध्यान्तमपारयोगंलोकस्य सेतुं प्रवदन्ति विप्राः}


\chapter{अध्यायः ६६}
\twolineshloka
{भीष्म उवाच}
{}


\twolineshloka
{ततः स भगवान्देवो लोकानामीश्वरेश्वरः}
{ब्रह्माणं प्रत्युवाचेदं स्रिग्धगम्भीरया गिरा}


\twolineshloka
{विदितं तात योगान्मे सर्वमेतत्तवेप्सितम्}
{तथा तद्भवितेत्युक्त्वा तत्रैवान्तरधीयत}


\twolineshloka
{ततो देवर्षिगन्धर्वा विस्मयं परमं गताः}
{कौतूहलपराः सर्वे पितामहमथाब्रुवन्}


\twolineshloka
{कोन्वयं यो भगवता प्रणम्य विनयाद्विभो}
{वाग्भिः स्तुतो वरिष्ठाभिः श्रोतुमिच्छाम तं वयम्}


\twolineshloka
{एवमुक्तस्तु भगवान्प्रत्युवाच पितामहः}
{देवब्रह्मर्षिगन्धर्वान्सर्वान्मधुरया गिरा}


\twolineshloka
{यत्तत्परं भविष्यं च भवितव्यं च यत्परम्}
{भूतात्मा च प्रभुश्चैव ब्रह्म यच्च परं पदम्}


\threelineshloka
{तेनास्मि कृतसंवादः प्रसन्नेन सुरर्षभाः}
{जगतोऽनुग्रहार्थाय याचितो मे जगत्पतिः}
{}


\twolineshloka
{मानुषं लोकमातिष्ठ वासुदेव इति श्रुतः}
{अस्रुराणां वधार्थाय संभवस्व महीतले}


\twolineshloka
{संग्रामे निहता ये ते दैत्यदानवराक्षसाः}
{त इमे नृषु संभूता घोररूपा महाबलाः}


\twolineshloka
{तेषां वधार्थं भगवान्नरेण सहितो वशी}
{मानुषीं योनिमास्थाय चरिष्यति महीतले}


\twolineshloka
{नरनारायणौ यौ तौ पुराणावृषिसत्तमौ}
{सहितो मानुषे लोके संभूतावमितद्युती}


\twolineshloka
{अजेयौ समरे यत्तौ सहितैरमरैरपि}
{मूढास्त्वेतौ न जानन्ति नरनारायणावृषी}


\twolineshloka
{तस्याहमग्रजः पुत्रः सर्वस्य जगतः प्रभुः}
{वासुदेवोऽर्चनीयो वः सर्वलोकमहेश्वरः}


\twolineshloka
{तथा मनुष्योऽयमिति कदाचित्सुरसत्तमाः}
{नावज्ञेयो महावीर्यः शङ्खचक्रगदाधरः}


\twolineshloka
{एतत्परमकं गुह्यमेतत्परमकं पदम्}
{एतत्परमकं ब्रह्म एतत्परमकं यशः}


\twolineshloka
{एतदक्षरमव्यक्तमेतद्वै शाश्वतं महः}
{यत्तत्पुरुषसंज्ञं वै गीयते ज्ञायते न च}


\twolineshloka
{एतत्परमकं तेज एतत्परमकं सुखम्}
{एतत्परमकं सत्यं कीर्तितं विश्वकर्मणा}


\twolineshloka
{तस्मात्सेन्द्रैः सुरैः सर्वैर्लोकैश्चामितविक्रमःक}
{नावज्ञेयो वासुदेवो मानुषोऽयमिति प्रभुः}


\twolineshloka
{यश्च मानुषमात्रोऽयमिति ब्रूयात्स मन्दधीः}
{हृषीकेशमवज्ञानात्तमाहुः पुरुषाधमम्}


\twolineshloka
{योगिनं तं महात्मानं प्रविष्टं मानुषीं तनुम्}
{अवमन्येद्वासुदेवं तमाहुस्तमसं जनाः}


\twolineshloka
{देवं चराचरात्मानं श्रीवत्साङ्कं सुवर्चसम्}
{पद्मनाभं न जानाति तमाहुस्तामसं बुधाः}


\twolineshloka
{किरीटकौस्तुभधरं मित्राणामभयंकरम्}
{अवजानन्महात्मानं घोरे तमसि मञ्जति}


\fourlineindentedshloka
{एवं विदित्वा तत्त्वार्थं लोकानामीश्वरेश्वरः}
{वासुदेवो नमस्कार्यः सर्वलोकैः सुरोत्तमाः}
{तस्यामहात्माजो ब्रह्मा सर्वस्य जगतः पतिः ॥भीष्म उवाच}
{}


\twolineshloka
{एवमुक्त्वा स भगवान्देवान्सर्षिगकणान्पुरा}
{विसृज्य सर्वभूतानि जगाम भवनं स्वकम्}


\twolineshloka
{ततो देवाः सगन्धर्वा मुनयोऽप्सरसोऽपि च}
{कथां तां ब्रह्मणा गीतां श्रुत्वा प्रीता दिवं ययुः}


\twolineshloka
{एतच्छ्रुतं मया तात ऋषीणां भावितात्मनाम्}
{वासुदेवं कथयतां समवाये पुरातनम्}


\twolineshloka
{रामस्य जामदग्न्यस्य मार्कण्डेयस्य धीमतः}
{व्यासनारदयोश्चापि सकाशाद्भरतर्षभ}


\twolineshloka
{एतमर्थं च विज्ञाय श्रुत्वा च प्रभुमव्ययम्}
{वासुदेवं महात्मानं लोकानामीश्वरेश्वरम्}


\twolineshloka
{यस्य चैवात्मजो ब्रह्मा सर्वस्य जगतः पिता}
{कथं न वासुदेवोऽयमर्च्यश्चेज्यश्च मानवैः}


\twolineshloka
{वारितोऽसि मया तात मुनिर्भिर्वेदपारगैः}
{मा गच्छ संयुगं तेन वासुदेवेन धन्विना}


\twolineshloka
{मा पाण्डवैः सार्धमिति तत्त्वं मोहान्न बुध्यसे}
{मन्ये त्वां राक्षसं क्रूरं तथा चासि तमोवृतः}


\twolineshloka
{यस्माद्द्विषसि गोविन्दं पाण्डवं तं धनञ्जयम्}
{नरनारायणौ देवी कोऽन्यो द्विष्याद्धि मानवः}


\twolineshloka
{तस्माद्ब्रवीमि ते राजन्नेष वै शाश्वतोऽव्ययः}
{सर्कवलोकमयो नित्यः शास्ता धाता धरो ध्रुवः}


\twolineshloka
{यो धारयति लोकांस्त्रींश्चराचरगुरुः प्रभुः}
{योद्धा जयश्च जेता च सर्वप्रकृतिरीश्वरः}


\twolineshloka
{राजन्सर्वमयो ह्येष तमोरागविवर्जितः}
{यतः कृष्णस्ततो धर्मो यतो धर्मस्ततो जयः}


\twolineshloka
{तस्य माहात्म्ययोगेन योगेनात्ममयेन च}
{धृताः पाण्डुसुता राजञ्जयश्चैषां भविष्यति}


\twolineshloka
{श्रेयोयुक्तां सदा बुद्धिं पाण़्डवानां दधाति यः}
{बलं चैव रणे नित्यं भयेभ्यश्चैव रक्षति}


\twolineshloka
{स एष शाश्वतो देवाः सर्वगुह्यमयः शिवः}
{वासुदेव इति ज्ञेयो यन्मां पृच्छसि भारत}


\twolineshloka
{ब्राह्मणैः क्षत्रिकयैर्वैश्यैः शूद्रैश्च कृतलक्षणैः}
{सेव्यतेऽभ्यर्च्यते चैव नित्ययुक्तैः स्वकर्मभिःक}


\twolineshloka
{द्वापरस्य युगस्यान्ते आदौ कलियुगस्य च}
{सात्वतं विधिमास्थाय गीतः संकर्षणेन वै}


\twolineshloka
{स एष सर्वं सुरमर्त्यलोकंसमुद्रकक्ष्यान्तरितां पुरीं च}
{युगेयुगे मानुषं चैव वासंपुनःपुनः सृजते वासुदेवाःक}


\chapter{अध्यायः ६७}
\twolineshloka
{दुर्योधन उवाच}
{}


\threelineshloka
{वासुदेवो महद्भूतं सर्वलोकेषु कथ्यते}
{तस्यागमं प्रतिष्ठां च ज्ञातुमिच्छे पितामह ॥भीष्म उवाच}
{}


\twolineshloka
{वासुदेवो महद्भूतं सर्वदैवतदैवतम्}
{न परं पुण्डरीकाक्षाद्दृश्यते भरतर्षभ}


\threelineshloka
{श्रुतं मे तात रामस्य जामदग्न्यस्य जल्पतः}
{नारदस्य च देवर्षेः कृष्णद्वैपायनस्य च}
{असितो देवलश्चापि वालखिल्यास्तपोधनाः}


\twolineshloka
{मार्कण्डेयश्च गोविन्दे कथयत्यद्भुतं महत्}
{सर्वभूतादिभूतात्मा महात्मा पुरुषोत्तमः}


\twolineshloka
{आपो वायुश्च तेजश्च त्रयमेतदकल्पयत्}
{स सृष्ट्वा पृथिवीं देवीं सर्वलोकेश्वरः प्रभुः}


\twolineshloka
{अप्सु वै शयनं चक्रे महात्मा पुरुषोत्तमः}
{सर्वदेवमयो देवः शयानः शयने सुखम्}


\twolineshloka
{मुखतः सोऽग्निमसृजत्प्राणाद्वायुमथापि च}
{सरस्वतीं च वेदांश्च मनसः ससृजेऽच्युत}


\twolineshloka
{एषक लोकान्ससर्जादौ देवांश्च ऋषिभिः सह}
{अमृतं चैव मृत्युं च प्रजानां प्रभवाप्ययौ}


\twolineshloka
{एष धर्मश्च धर्मज्ञो वरदः सर्वकामदः}
{एष कर्ता च कार्यं च आदेरादिः स्वयं प्रभुः}


\twolineshloka
{भूतं भव्यं भविष्यच्च पूर्वमेतदकल्पयत्}
{उभे सन्ध्ये दिवः खं च नियमांश्च जनार्दनः}


\twolineshloka
{ऋषींश्चैव हि गोविन्दस्तपश्चैवाभ्यकल्पयत्}
{स्रष्टारं जगतश्चापि महात्मा प्रभुरव्ययः}


\twolineshloka
{अग्रजं सर्वभूतानां संकर्षणमकल्पयत्}
{शेषं चाकल्पयद्देवमनन्तमिति तं विदुः}


\threelineshloka
{यो धारयति भूतानि धरां चेमां सपर्वताम्}
{ध्यानयोगेन विप्राश्च तं विदन्ति महौजसम्}
{कर्णस्रोतोभवं चापि मधुं नाम महासुरम्}


\threelineshloka
{तमुग्रमुग्रकर्माणमुग्रां बुद्धइं समास्थितम्}
{हरन्तं ब्रह्मणो वेदाञ्जघान ब्रह्मणः पिता}
{ब्रह्मणोपचितिं कुर्वन्महात्मा पुरुषोत्तमः}


\twolineshloka
{तस्य तात वधादेव देवदानवमानवाः}
{मधुसूदनमित्याहुर्ऋषयश्च जनार्दनम्}


\twolineshloka
{वराहं नारसिंहं च त्रिविक्रममिति प्रभुम्}
{एष धाता विधाता च सर्वेषां प्रामिनां हरिः}


\twolineshloka
{परं हि पुण्डरीकाक्षान्न भूतं न भविष्यति}
{मुखतः सोसृजद्विप्रान्बाहुभ्यां क्षत्रियांस्तथा}


\twolineshloka
{वैश्यांश्चाप्यूरुतो राजञ्शूद्रान्पद्भ्यां तथैवच}
{तपसा नियतं देवं निधानं सर्वदेहिनाम्}


\twolineshloka
{ब्रह्मभूतममावास्यां पौर्णमास्यां तथैव च}
{योगभूतं परिचरन्केशवं महदाप्नुयात्}


\twolineshloka
{केशवः परमं तेजः सर्वलोकपितामहः}
{एवमार्हुर्हृषीकेशं मुनयो वै नराधिप}


\twolineshloka
{एवमेनं विजानीहि आचार्यं पितरं गुरुम्}
{कृष्णो यस्य प्रसीदेत लोकास्तेनाक्षया जिताः}


\twolineshloka
{यश्चैवैनं भयस्थाने केशवं शरणं व्रजेत्}
{सदा नरः पठंश्चेदं स्वस्तिमान्स सुखी भवेत्}


\twolineshloka
{ये च कृष्णं प्रपद्यन्ते ते न मुह्यन्ति मानवाः}
{भये महति मग्नांश्च पाति नित्यं जनार्दनःक}


\threelineshloka
{स तं युधिष्ठिरो ज्ञात्वा याथातथ्येन भारत}
{सर्वात्मना महात्मानं केशवं जगदीश्वरम्}
{प्रपन्नः शरणं राजन्योगानां प्रभुमीश्वरम्}


\chapter{अध्यायः ६८}
\twolineshloka
{भीष्म उवाच}
{}


\twolineshloka
{स्तवं वै ब्रह्मसंयुक्तं शृणु कृष्णस्य भारत}
{ब्रह्मर्षिभिश्च देवैश्च यः पुरा कथितो भुवि}


\twolineshloka
{साध्यानामपि देवानां देवदेवेश्वरः प्रभुः}
{लोकभावनभावज्ञ इति त्वां नारदोऽब्रवीत्}


\twolineshloka
{भूतं भव्यं भविष्यच्च मार्कण्डेयोऽभ्युवाच ह}
{यज्ञं त्वां चैव यज्ञानां तपश्च तपसामपि}


\twolineshloka
{देवानामपि देवं च त्वामाह भगवान्भृगुः}
{पुराणां चैव परमं विष्णो रूपं तवेति च}


\twolineshloka
{वासुदेवो वसूनां त्वं शक्रं स्थापयिता तथा}
{देवदेवोऽसि देवानामिति द्वैपायनोऽब्रवीत्}


\twolineshloka
{पूर्वे प्रजानिसर्गे च दक्षमाहुः प्रजापतिम्}
{स्रष्टारं सर्वभूतानामङ्गिरास्त्वां तथाऽब्रवीत्}


\twolineshloka
{अव्यक्तं ते शरीरोत्थं व्यक्तं ते मनसि स्थितम्}
{देवास्त्वत्संभवाश्चैव देवलस्त्वसितोऽब्रवीत्}


\twolineshloka
{शिरसा ते दिव्यं व्याप्तं बाहुभ्यां पृथिवी तथा}
{जठरं ते त्रयो लोकाः पुरुषोऽसि सनातनः}


\twolineshloka
{एवं त्वामभिजानन्ति तपसा भाविता नराः}
{आत्मदर्शनतृप्तानामृषीणां चासि सत्तमः}


\twolineshloka
{राजर्षीणामुदाराणामाहवेष्वनिवर्तिनाम्}
{सर्वधर्मप्रधानानां त्वं गतिर्मधुसूदन}


\twolineshloka
{इति नित्यं योगविर्द्भिर्भगवान्पुरुषोत्तमः}
{सनत्कुमारप्रमुखैः स्तूयतेऽभ्यर्च्यते हरिः}


\threelineshloka
{एष ते विस्तरस्तात संक्षेपश्च प्रकीर्तितः}
{केशवस्य यथा तत्त्वं सुप्रीतो भज केशवम् ॥सञ्जय उवाच}
{}


\twolineshloka
{पुण्यं श्रुत्वैतदाख्यानं महाराज सुतस्तव}
{केशवं बहुमेने स पाण्डवांश्च महारथान्}


\twolineshloka
{तमब्रवीन्महाराज भीष्मः शान्तनवः पुनः}
{माहात्म्यं ते श्रुतं राजन्केशवस्य महात्कमनः}


\twolineshloka
{नरस्य च यथातत्त्वं यन्मां त्वं पृच्छसे नृप}
{यदर्थं नृषु संभूतौ नरानारायणावृषी}


\twolineshloka
{अवध्यौ च यथा वीरौ संयुगेष्वपराजितौ}
{यथा च पाण्डवा राजन्नवध्या युधि कस्यचित्}


\twolineshloka
{प्रीतिमान्हि दृढं कृष्णः पाण्डवेषु यशस्विषु}
{तस्माद्ब्रवीमि राजेन्द्र शमो भवतु पाण्डवैः}


\twolineshloka
{पृथिवीं भुङ्क्ष्व सहितो भ्रातृभिर्बलिभिर्वशी}
{नरानारायणौ देवाववज्ञाय नशिष्यसि}


\twolineshloka
{एवमुक्त्वा तव पिता तूष्णीमासीद्विशांपते}
{व्यसर्जयच्च राजानं शयनं च विवेश ह}


\twolineshloka
{राजा च शिबिरं प्रायात्प्रणिपत्य महात्मने}
{शिश्ये च शयने शुभ्रे रात्रिं तां भरतर्षभ}


\chapter{अध्यायः ६९}
\twolineshloka
{सञ्जय उवाच}
{}


\twolineshloka
{व्युषितायां तु शर्वर्यामुदिते च दिवाकरे}
{उभे सेने महाराज युद्धायैव समीयतुः}


\twolineshloka
{अभ्यधावन्त संक्रुद्धाः परस्परजिगीषवः}
{ते सर्वे सहिता युद्धे समालोक्य परस्परम्}


\twolineshloka
{पाण्डवा धार्तराष्ट्राश्च राजन्दर्मन्त्रिते तव}
{व्यूहौ च व्यूह्य संरब्धाः संप्रहृष्टाः प्रहारिणः}


\twolineshloka
{अरक्षन्मकरव्यूहं भीष्मो राजन्समन्ततः}
{तथैव पाण्डवा राजन्नरक्षन्व्यूहमात्मनः}


\twolineshloka
{`अजातशत्रुः शत्रूणां मनांसि समकम्पयत्}
{'श्येनवद्व्यूह्य तं व्यूहं धौम्यस्य वचनात्स्वयम्}


\twolineshloka
{स हि तस्य सुविज्ञात अग्निचित्येषु भारत}
{मकरस्तु महाव्यूहस्तव पुत्रस्य धीमतः}


\twolineshloka
{स्वयं सर्वेण सैन्येन द्रोणेनानुमतस्तदा}
{यथाव्यूहं शान्तनवः सोऽन्ववर्तत तत्पुनः}


\twolineshloka
{स निर्ययौ महाराज पिता देवव्रतस्तव}
{महता रथवंशेन संवृतो रथिनां वरः}


\twolineshloka
{इतरेतरमन्वीयुर्यथाभागमवस्थिताः}
{रथिनः पत्तयश्चैव दन्तिनः सादिनस्तथा}


\twolineshloka
{तान्दृष्ट्वाऽभ्युद्यतान्सङ्ख्ये पाण्डवा हि यशस्विनः}
{श्येनव्यूहेन संव्यूह्य समनह्यन्त संयुगे}


\twolineshloka
{अशोभत मुखे तस्य भीमसेनो महाबलः}
{नेत्रे शिखण्डी दुर्धर्षो धृष्टद्युम्नश्च पार्षतः}


\twolineshloka
{शीर्षे तस्याभवद्वीरः सात्यकिः सत्यविक्रमः}
{विधुन्वन्गाण्डिवं पार्थो ग्रीवायामभवत्तदा}


\twolineshloka
{अक्षौहिण्या समं तत्र वामपक्षोऽभवत्तदा}
{महात्मा द्रुपदः श्रीमान्सह पुत्रेण संयुगे}


\twolineshloka
{दक्षिणश्चाभवत्पक्षः कैकेयोऽक्षौहिणीपतिः}
{पृष्ठतो द्रौपदेयाश्च सौभद्रश्चापि वीर्यवान्}


\twolineshloka
{पृष्ठे समभवच्छ्रीमान्स्वयं राजा युधिष्ठिरःक}
{भ्रातृभ्यां सहितो वीरो यमाभ्यां चारुविक्रमः}


\twolineshloka
{प्रविश्य तु रणे भीमो मकरं मुखतस्तदा}
{भीष्ममासाद्य संग्रामे च्छादयामास सायकैः}


\twolineshloka
{ततो भीष्मो महास्राणि पातयामास भारत}
{मोहयन्पाण्डुपुत्राणां व्यूढं सैन्यं महाहवे}


\twolineshloka
{संमुह्यति तदा सैन्ये त्वरमाणो धनंजयः}
{भीष्मं शरसहस्रेण विव्याध रणमूर्धनि}


\twolineshloka
{प्रतिसंवार्य चास्राणि भीष्ममुक्तानि संयुगे}
{स्वेनानीकेन हृष्टेन युद्धाय समुपस्थितः}


\twolineshloka
{ततो दुर्योधनो राजा भारज्वाजमभाषत}
{पूर्वं दृष्ट्वा वधं घोरं बलस्य बलिनां वरः}


\twolineshloka
{भ्रातॄणां च वधं युद्धे स्मरमाणो महारथः}
{आचार्यः सततं हि त्वं हितकामो ममानघ}


\twolineshloka
{वयं हि त्वां समाश्रित्य भीष्मं चैव पितामहम्}
{देवानपि रणे जेतुं प्रार्थयामो न संशयः}


\twolineshloka
{किमु पाण्डुसुताकन्युद्धे हीनवीर्यपराक्रमान्}
{स तथा कुरु भद्रं ते यथा वध्यन्ति पाण्डवाः}


\twolineshloka
{एवमुक्तस्ततो द्रोणस्तव पुत्रेण मारिष}
{तत्र संप्रेक्ष्य राजनं संक्रुद्ध इव निश्चसन्}


\twolineshloka
{बालिशस्त्वं न जानीषे पाण्डवानां पराक्रमम्}
{न शक्या हि यथा जेतुं पाण्डवा हि महाबलाः}


\threelineshloka
{यथाबलं यथावीर्यं कर्म कुर्यामहं हि ते}
{इत्युक्तवा ते सुतं राजन्नभ्यपद्यत वाहिनीम्}
{अभिनत्पाण्डवानीकं प्रेक्षमाणस्य सात्यकेः}


\twolineshloka
{सात्यकिस्तु ततो द्रौणं वारयामात भारत}
{तयोः प्रववृते युद्धं घोररूपं भयावहम्}


\twolineshloka
{शैनेयं तु रणे क्रुद्धो भारद्वाजः प्रतापावान्}
{अविध्यन्निशितैर्बाणैर्जत्रुदेशे हसन्निव}


\twolineshloka
{भीमसेनस्ततः क्रुद्धो भारद्वाजमविध्यत}
{संरक्षन्सात्यकिं राजन्द्रोणाच्छस्त्रभृतां वरात्}


\twolineshloka
{ततो द्रोणश्च भीष्मश्च तथा शल्यश्च मारिष}
{भीमसेनं रणे क्रुद्धाश्छादयांचक्रिरे शरैः}


\twolineshloka
{तत्राभिमन्युः संक्रुद्धो द्रौपदेयाश्च मारिष}
{विव्यधुर्निशितैर्बैणैः सर्वांस्तानुद्यतायुधान्}


\twolineshloka
{द्रोणभीष्मौ तु संक्रुद्धावापतन्तौ महाबलौ}
{प्रत्युद्ययौ शिखण्डी तु महेष्वासो महाहवे}


\twolineshloka
{प्रगृह्य बलवद्वीरो धनुर्जलदनिःश्वनम्}
{अभ्यवर्षच्छरैस्तूर्णं छादयानो दिवाकरम्}


\twolineshloka
{शिखण्डिनं समासाद्य भरतानां पितामहः}
{अवर्जयत संग्रामं स्त्रीत्वं तस्यानुसंस्मरन्}


\twolineshloka
{ततो द्रोणो महाराज अभ्यद्रवत तं रणे}
{रक्षमाणस्तदा भीष्मं तव पुत्रेण चोदितः}


\twolineshloka
{शिखण्डी तु समासाद्य द्रोणं शस्त्रभृतां वरम्}
{अवर्जयत संत्रस्तो युगान्ताग्निमिवोल्बणम्}


\twolineshloka
{ततो बलेन महता पुत्रस्तव विशांपते}
{जुगोप भीष्ममासाद्य प्रार्थयानो महद्यशः}


\twolineshloka
{तथैव पाण्डवा राजन्पुरुस्कृत्य धनञ्जयम्}
{भीष्ममेवाभ्यवर्तन्त जये कृत्वा दृढां मतिम्}


\twolineshloka
{तद्युद्धमभवद्धोरं देवानां दानवैरिव}
{जयमाकाङ्क्षतां सङ्क्ष्ये यशश्च सुमहाद्भुतम्}


\chapter{अध्यायः ७०}
\twolineshloka
{सञ्जय उवाच}
{}


\twolineshloka
{अकरोत्तुमुलं युद्धं भीष्मः शान्तनवस्तदा}
{भीमसेनभयादिच्छन्पुत्रांस्तारयितुं तव}


\twolineshloka
{पूर्वाह्णे तन्महारौद्रं राज्ञां युद्धमवर्तत}
{कुरूणां पाण्डवानां च मुख्यशूरविनाशनम्}


\twolineshloka
{तस्मिन्नाकुलसंग्रामे वर्तमाने महाभये}
{अभवत्तुमुलः शब्दः संस्पृशन्गगनं महत्}


\twolineshloka
{नदद्भिश्च महानागैर्हेषमाणैश्च वाजिभिः}
{भेरीशङ्खनिनादैश्च तुमुलं समपद्यत}


\twolineshloka
{युयुत्सवस्ते विक्रान्ता विजयाय महाबलाःक}
{अन्योन्यमभिगर्जन्तो गोष्ठिष्विव महर्षभाः}


\twolineshloka
{शिरासं पात्यमानानां समरे निशितैः शरैः}
{अश्यवृष्टिरिवाकाशे बभूव भरतर्षभ}


\twolineshloka
{कुण्डलोष्णीषधारीणि जातरूपोज्ज्वलानि च}
{पतितानि स्म दृश्यन्ते शिरांसि भरतर्षभ}


\twolineshloka
{विशिखोन्मथितैर्गात्रैर्बाहुभिश्च सकार्मुकैः}
{सहस्ताभरणैश्चान्यैरभवच्छादिता मही}


\twolineshloka
{कवचोपहितैर्गात्रैर्हस्तैश्च समलङ्कृतैः}
{मुखैश्च चन्द्रसंकाशै रक्तान्तनयनैः शुभैः}


\twolineshloka
{गजवाजिमनुष्याणां सर्वगात्रैश्च भूपते}
{आसीत्सर्वा समास्तीर्णा मुहूर्तेन वसुंधराः}


\twolineshloka
{रजोमेघैश्च तुमुलैः शस्त्रविद्युत्प्रकाशिभिः}
{आयुधानां च निर्घोषः स्तनयित्नुसमोऽभवत्}


\twolineshloka
{स संप्रहारस्तुमुलःक कटुकः शोणितोदकः}
{प्रावर्तत कुरूणां च पाण्डवानां च भारत}


\twolineshloka
{तस्मिन्महाभये घोरे तुमुले रोमहर्षणे}
{ववृषुः शरवर्षाणि क्षत्रिया युद्धदुर्मदाः}


\twolineshloka
{आक्रोशन्कुञ्जरास्तत्र शरवर्षप्रतापिताः}
{तावकानां परेषां च संयुगे भरतर्षभ}


\twolineshloka
{संरब्धानां च वीराणां धीराणाममितौजसाम्}
{धनुर्ज्यातलशब्देन न प्राज्ञायत किंचन}


\twolineshloka
{उत्थितेषु कबन्धेषु सर्वतः शोमितोदके}
{समरे पर्यधावन्त नृपा रिपुवधोद्यताः}


\twolineshloka
{शरशक्तिगदाभिस्ते खङ्गैश्चामिततेजसः}
{निजघ्नुः समरेऽन्योन्यं शूराः परिघबाहवः}


\twolineshloka
{बभ्रमुः कुञ्जराश्चात्र शरैर्विद्धा निरङ्कुशाः}
{अश्वाश्च पर्यधावन्त हतारोहा दिशो दश}


\twolineshloka
{उत्पत्य निपतन्त्यन्ये शरघातप्रपीडिताः}
{तावकानां परेषां च योधा भरतसत्तम}


\twolineshloka
{वाहानामुत्तमाङ्गानां कार्मुकाणां च भारत}
{गदानां परिघाणां च हस्तानां चोरुभिः सह}


\twolineshloka
{पादानां भूषणानां च केयूराणां च सङ्घशः}
{राशयस्तत्र दृश्यन्ते भीष्मभीमसमागमे}


\twolineshloka
{अश्वानां कुञ्जराणां च रथानां चानिवर्तिनाम्}
{सङ्घाताः स्म प्रदृश्यन्ते तत्रतत्र विशांपते}


\twolineshloka
{गदाभिरसिभिः प्रासैर्बाणैश्च नतपर्वभिः}
{जघ्नुः परस्पर तत्र क्षत्रियाः काल आगते}


\twolineshloka
{अपरे बाहुभिर्वीरा नियुद्धकुशला युधि}
{बहुधा समसञ्जन्त आयसैः परिघैरिव}


\twolineshloka
{मुष्टिभिर्जानुभिश्चैव तलैश्चैव विशांपते}
{अन्योन्यं जघ्रिरे वीरास्तावकाः पाण्डवैः सह}


\twolineshloka
{पतितैः पात्यमानैश्च विचेष्टद्भिश्च भूतले}
{घोरमायोधनं जज्ञे तत्रतत्र जनेश्वर}


\twolineshloka
{विरथा रथिनश्चात्र निस्त्रिंशवरधारिणःक}
{अन्योन्यमभिधावन्तः परस्परवधैषिणः}


\twolineshloka
{ततो दुर्योधनो राजा कलिङ्गैर्बहुभिर्वृतः}
{पुरस्कृत्य रणे भीष्मं पाण्डवानभ्यवर्तत}


\twolineshloka
{तथैव पाण्डवाः सर्वे परिवार्य वृकोदरम्}
{भीष्ममभ्यद्रवन्क्रुद्धास्ततो युद्धमवर्तत}


\chapter{अध्यायः ७१}
\twolineshloka
{सञ्जय उवाच}
{}


\twolineshloka
{दृष्ट्वा भीष्मेण संसक्तान्भ्रातॄनन्यांश्च पार्थिवान्}
{समभ्यधावद्गांङ्गेयमुद्यतास्त्रो धनञ्जयः}


\twolineshloka
{पाञ्चजन्यस्य निर्घोषं धनुषो गाण्डिवस्य च}
{ध्वजं च दृष्ट्वा पार्थस्य सर्वान्नो भयमाविशत्}


\twolineshloka
{सिंहलाङ्गूलमाकाशे ज्वलन्तमिव पर्वतम्}
{असञ्जमानं वृक्षेषु धूमकेतुमिवोत्थितम्}


\twolineshloka
{बहुवर्णं विचित्रं च दिव्यं वानरलक्षणम्}
{अपश्याम महाराज ध्वजं गाण्डीवधन्वनः}


\threelineshloka
{सुवर्णपृष्ठं गाण्डीवं रणे द्रक्ष्यामि भारत}
{विद्युतं मेघमध्यस्थां भ्राजमानामिवाम्बरे}
{ददृशुर्गाण्डिवं योधा रुक्मपृष्ठं महामृधे}


\twolineshloka
{आशुश्रुम भृशं चास्य शक्रस्येवाभिगर्जतः}
{सुघोरं तलयोः शब्दं निघ्नतस्तव वाहिनीम्}


\twolineshloka
{चण्डवातो यथा मेघः सविद्युत्स्तनयित्नुमान्}
{दिशः संप्लावयन्सर्वाः शरवर्षैः समन्ततः}


\twolineshloka
{समभ्यधावद्गाङ्गेयं भैरवास्त्रो धनञ्जयः}
{दिशं प्राचीं प्रतीचीं च न जानीमोऽस्त्रमोहिताः}


\twolineshloka
{कान्दिग्भूता श्रान्तपत्रा हताश्वा हतचेतसः}
{अन्योन्यमभिसंश्लिष्य योधास्ते भरतर्षभ}


\twolineshloka
{भीष्ममेवाभ्यलीयन्त सह सर्वैस्तवात्मजैः}
{तेषामार्तायनमभूद्भीष्मः शान्तनवो रणे}


\twolineshloka
{समुत्पतन्ति वित्रस्ता रथेभ्यो रथिनस्तथा}
{सादिनश्चाश्वपृष्ठेभ्यो भूमौ चापि पदातयः}


\twolineshloka
{श्रुत्वा गाण्डीवनिर्घोषं विस्फूर्जितमिवाशनेः}
{सर्वसैन्यानि भीतानि व्यवालीयन्त भारत}


\twolineshloka
{अथ काम्भोजैजरश्वैर्महद्भिः शीघ्रगामिभिः}
{गोपानां बहुसाहस्त्रैर्बालैर्गापायनैर्वृतः}


\twolineshloka
{मद्रसौवीरगान्धारैस्त्रैगर्तैश्च विशांपते}
{सर्वकालिङ्गमुख्यैश्च कलिङ्गाधिपतिर्वृतः}


\twolineshloka
{नानानरगणौघैश्च दुःशासनपुरःसरः}
{जयद्रथश्च नृपतिः सहितः सर्वराजभिः}


\twolineshloka
{हयारोहवराश्चैव तव पुत्रेण चोदिताः}
{चतुर्दशसहस्राणि सौबलं पर्यवारयन्}


\twolineshloka
{ततस्ते सहिताः सर्वे विभक्तरथवाहनाः}
{अर्जुनं समरे जघ्नुस्तावका भरतर्षभ}


\threelineshloka
{चेदिकाशिपदातैश्च रथैः पाञ्चालसृञ्जयैः}
{पाण्डवाः सहिताः सर्वे धृष्टद्युम्नपुरोगमाः}
{तावकान्समरे जघ्नुर्धर्मपुत्रेण चोदिताः}


\twolineshloka
{रथिभिर्वारणैरश्वैः पादातैश्च समीरितम्}
{घोरमायोधनं चक्रे महाभ्रसदृशं रजः}


\twolineshloka
{तोमरप्रासनाराचगजाश्वरथयोधिनाम्}
{बलेन महता भीष्मः समसञ्जत्किरीटिना}


\twolineshloka
{आवन्त्यः काशिराजेन भीमसेनेन सैन्धवः}
{अजातशत्रुर्मद्राणामृषभेण यशश्विना}


\twolineshloka
{सहपुत्रः सहामात्यः शल्येन समसञ्जत}
{विकर्णः सहदेवेन चित्रसेनः शिखण्डिना}


\twolineshloka
{मत्स्या दुर्योधनं जग्मुः शंकुनिं न विशांपते}
{द्रुपदश्चेकितानश्च सात्यकिश्चक महारथः}


\twolineshloka
{द्रोणेन समसञ्जन्त सपुत्रेण महात्मना}
{कृपश्च कृतवर्मा च धृष्टद्युम्नमभिद्रुतौ}


\twolineshloka
{एवं प्रव्रजिताश्वानि भ्रान्तनागरथानि च}
{सैन्यानि समसञ्जन्त प्रयुद्धानि समन्ततः}


\twolineshloka
{निरभ्रे विद्युतस्तीव्रा दिशश्च रजसा वृताः}
{प्रादुरासन्महोत्काश्च सनिर्घाता विशांपते}


\twolineshloka
{प्रादुर्भूतो महावातः पांसुवर्षं पपात च}
{नभस्यन्तर्दधे सूर्यः सैन्येन सजसा वृतः}


\twolineshloka
{प्रमोहः सर्वसत्वानामतीव समपद्यत}
{रजसा चाभिभूतानामस्त्रजालैश्च तुद्यताम्}


\twolineshloka
{वीरबाहुविसृष्टानां सर्वावरणभेदिनाम्}
{संघातः शरजालानां तुमुलः समपद्यत}


\twolineshloka
{प्रकाशं चक्रुराकाशमुद्यतानि भुजोत्तमैः}
{नक्षत्रविमलाभानि शस्त्राणि भरतर्षभ}


\twolineshloka
{आर्षभाणि विचित्राणि रुक्मजालावृतानि च}
{संपेतुर्दिक्षु सर्वासु चर्माणि भरतर्षभ}


\threelineshloka
{सूर्यवर्णैश्च निस्त्रिंशैः पात्यमानानि सर्वशः}
{दिक्षु सर्वास्वदृश्यन्त शरीराणि शिरांसि च}
{}


\twolineshloka
{भग्नचक्राक्षनीडाश्च निपातितमहाध्वजाः}
{हताश्वाः पृथिवीं जग्मुस्तत्रतत्र महारथाः}


\twolineshloka
{परिपेतुर्हयाश्चात्र केचिच्छस्त्रकृतव्रणाः}
{रथान्विपरिकर्षन्तो हतेषु रथयोधिषु}


\threelineshloka
{शराहता भिन्नदेहा बद्धयोक्रा हयोत्तमाः}
{युगानि पर्यकर्षन्त तत्रतत्र स्म भारत}
{}


\twolineshloka
{अदृश्यन्त ससूताश्च साश्वाः सरथयोधिनः}
{एकेन बलिना राजन्वारणेन विमर्दिताः}


\twolineshloka
{गन्धहस्तिमदस्रावमाघ्राय बहवो रणे}
{सन्निपाते बलौघानां गजैर्ममृदिरे गजाः}


\twolineshloka
{सतोमरैर्महामात्रैर्निपतद्भिर्गतासुभिः}
{बभूवायोधनं छन्नं नाराचाभिहतैर्गजैः}


\twolineshloka
{सन्निपाते बलौघानां प्रेषितैर्वरवारणैः}
{निपेतुर्युधि संभग्नाः सयोधाः सध्वजा गजाः}


\twolineshloka
{नागराजोपमैर्हस्तैर्नागैराक्षिप्य संयुगे}
{व्यदृश्यन्त महाराज संभग्ना रथकूबराः}


\twolineshloka
{विशीर्णरथसङ्घाश्च केशेष्वाक्षिप्य दन्तिभिः}
{द्रुमशाखा इवाविध्य निष्पिष्टा रथिनो रणे}


\twolineshloka
{रथेषु च रथान्युद्धे संसक्तान्वरवारणाः}
{विकर्षन्तो दिशः सर्वाः संपेतुः सर्वशब्दगाः}


\twolineshloka
{तेषां तथा कर्षतां तु गजानां रूपमाबभौ}
{सरःसु नलिनीजालं विषक्तमिव कर्षताम्}


\twolineshloka
{एवं संछादितं तत्र बभूवायोधनं महत्}
{सादिभिश्च पदातैश्च सध्वजैश्च महारथैः}


\chapter{अध्यायः ७२}
\twolineshloka
{सञ्जय उवाच}
{}


\twolineshloka
{शिखण्डी सह मत्स्येन विराटेन विशांपते}
{भीष्ममाशु महेष्वासमाससाद सुदुर्जयम्}


\twolineshloka
{द्रोणं कृपं विकर्णं च महेष्वासं महाबलम्}
{राज्ञश्चान्यान्रणे शूरान्बहूनार्च्छद्धनंजयः}


\twolineshloka
{सैन्धवं च महेष्वासं सामात्यं सह बन्धुभिः}
{प्राच्यांश्च दाक्षिणात्यांश्च भूमिपान्भूमिपर्षभ}


\twolineshloka
{पुत्रं च ते महेष्वासं दुर्योधनममर्षणम्}
{दुःसहं चैव समरे भीमसेनोऽभ्यवर्तत}


\twolineshloka
{सहदेवस्तु शकुनिमुलूकं च महारथम्}
{पितापुत्रौ महेष्वासावभ्यवर्तत दुर्जयौ}


\twolineshloka
{युधिष्ठिरो महाराज गजानीकं महारथः}
{समवर्तत संग्रामे पुत्रेण निकृतस्तव}


\twolineshloka
{माद्रीपुत्रस्तु नकुलः शूरसंक्रन्दनो युधि}
{त्रिगर्तानां बलैः सार्धं समसञ्जत पाण्डवः}


\twolineshloka
{अभ्यवर्तन्त संक्रुद्धाः समरे साल्वकेकयान्}
{सात्यकिश्चेकितानश्च सौभद्रश्च महारथः}


\threelineshloka
{धृष्टकेतुश्च समरे राक्षसश्च घटोत्कचः}
{नाकुलिश्च शतानीकः समरे रथपुङ्गवः}
{पुत्राणां ते रथानीकं प्रत्युद्याताः सुदुर्जयाः}


\twolineshloka
{सेनापतिरमेयात्मा धृष्टद्युम्नो महाबलः}
{द्रोणेन समरे राजन्समियायोग्रकर्मणा}


\twolineshloka
{एवमेते महेष्वासास्तावकाः पाण्डवैः सह}
{समेत्य समरे शूराः संप्राहारं प्रचक्रिरे}


\twolineshloka
{मध्यंदिनगते सूर्ये नभस्याकुलतां गते}
{कुरवः पाण्डवेयाश्च निजघ्नुरितरेतरम्}


\twolineshloka
{ध्वजिनो हेमचित्राङ्गा विचरन्तो रणाजिरे}
{सपताका रथा रेजुर्वैयाघ्रपरिवारणाः}


\twolineshloka
{समेतानां च समरे जिगीषूणां परस्परम्}
{बभूव तुमुलः शब्दः सिंहानामिव नर्दताम्}


\twolineshloka
{तत्राद्भुतमपश्यास संप्रहारं सुदारुणम्}
{यदकुर्वन्रणे शूराः सृञ्जयाः कुरुभिः सह}


\twolineshloka
{नैव खं न दिशो राजन्न सूर्यं शत्रुतापन}
{विदिशो वापि पश्यामः शरैर्मुक्तैः समन्ततः}


\twolineshloka
{शक्तीनां विमलाग्राणां तोमराणां तथास्यताम्}
{निस्त्रिंशानां च पीतानां नीलोत्पलनिभाः प्रभाः}


\twolineshloka
{कवचानां विचित्राणां भूषणानां प्रभास्तथा}
{खं दिशः प्रदिशश्चैव भासयामासुरोजसा}


\twolineshloka
{वपुर्भिश्च नरेन्द्राणां चन्द्रसूर्यसमप्रभैः}
{विरराज तदा राजंस्तत्रतत्र रणाङ्गणम्}


\twolineshloka
{रथसङ्घा नरव्याघ्राः समायान्तश्च संयुगे}
{विरेजुः समरे राजन्ग्रहा इव नभस्तले}


\twolineshloka
{भीष्मस्तु रथिनां श्रेष्ठो भीमसेनं महाबलम्}
{अवारयत संक्रुद्धः सर्वसैन्यस्य पश्यतः}


\twolineshloka
{ततो भीष्मविनिर्मुक्ता रुक्मपुङ्खाः शिलाशिताः}
{अभ्यङ्गन्तमरे भीमं तैलघौताः सुतेजनाः}


\twolineshloka
{तस्य शक्तिं महावेगं भीमसेनो महाबलः}
{क्रुद्धाशीविषसंकाशं प्रेषयामास भारत}


\twolineshloka
{तामापतन्तीं सहसा रुक्मदण्डं दुरासदाम्}
{चिच्छेद समरे भीष्मः शरैः सन्नतपर्वभिः}


\twolineshloka
{ततोऽपरेण भल्लेन पीतेन निशितेन च}
{कार्मुकं भीमसेनस्य द्विधा चिच्छेद भारत}


\twolineshloka
{अपास्य तु धनुश्छिन्नं भीमसेनो महाबलः}
{शरैर्बहुभिरानर्च्छद्भीष्मं शान्तनवं युधि}


\twolineshloka
{सात्यकिस्तु ततस्तूर्णं भीष्ममासाद्य संयुगे}
{आकर्णप्रहितैस्त्रीक्ष्णैर्निशितैस्तिग्मतेजनैः}


\twolineshloka
{शरैर्बहुभिरानर्च्छत्पितरं ते जनेश्वर}
{ततः संधाय वै तीक्ष्णं शरं परमदारुणम्}


\twolineshloka
{वार्ष्णेयस्य रथाद्भीष्मः पातयामास सारथिम्}
{तस्याश्वाः प्रद्रुता राजन्निहते रथसारथौ}


\twolineshloka
{तेन तेनैव धावन्ति मनोमारुतरंहसः}
{ततः सर्वस्य सैन्यस्य निःस्वनस्तुमुलोऽभवत्}


\twolineshloka
{हाहाकारश्च संजज्ञे पाण्डवानां महात्मनाम्}
{अभ्यद्रवत गृह्णीत हयान्यच्छत धावत}


\twolineshloka
{इत्यासीत्तुमुलः शब्दो युयुधानरथं प्रति}
{एतस्मिन्नेव काले तु भीष्मः शान्तनवस्तदा}


\threelineshloka
{न्यहनत्पाण्डवीं सेनामासुरीमिव वृत्राहा}
{ते वध्यमाना भीष्मेण पाञ्चालाः सोमकैः सह}
{स्थिरां युद्धे मतिं कृत्वा भीष्ममेवाभिदुद्रुवुः}


\twolineshloka
{धृष्टद्युम्नमुखाश्चापि पार्थाः शान्तनवं रणे}
{अभ्यधावञ्जिगीषन्तस्तव पुत्रस्य वाहिनीम्}


\twolineshloka
{तथैव कौरवा राजन्भीष्मद्रोणपुरोगमाः}
{अभ्यधावन्त वेगेन ततो युद्धमवर्तत}


\chapter{अध्यायः ७३}
\twolineshloka
{सञ्जय उवाच}
{}


\twolineshloka
{विराटोऽय त्रिभिर्बाणैर्भीष्ममार्च्छन्महारथम्}
{विव्याध तुरमांत्रास्य त्रिभिर्बाणैर्महारथः}


\twolineshloka
{तं प्रत्यविध्यद्दशभिर्भीष्मः शान्तनवः शरैः}
{रुक्मपुङ्खैर्महेष्वासः कृतहस्तो महाबलः}


\twolineshloka
{द्रौणिर्गाण्डीवधन्वानं भीमधन्वा महारथः}
{अविध्यदिषुभिः षङ्भिर्दृढहस्तः स्तनान्तरे}


\twolineshloka
{कार्मुकं तस्य चिच्छेद फल्गुनः परवीरहा}
{अविध्यच्च भृशं तीक्ष्णैः पत्रिभिः शत्रुकर्शनः}


\twolineshloka
{सोऽन्यत्कार्मुकमादाय वेगवान्क्रोधमूर्च्छितः}
{अमृष्यमाणः पार्थेन कार्मुकच्छेदमाहवे}


\twolineshloka
{अविध्यत्फल्गुनं राजन्नवत्या निशितैः शरैः}
{वासुदेवं च सप्तत्या विव्याध परमेषुभिः}


\twolineshloka
{ततः क्रोधाभिताम्राक्षः कृष्णेन सह फल्गुनः}
{दीर्घमुष्णं च निःश्वस्य चिन्तयित्वा पुनःपुनः}


\twolineshloka
{धनुः प्रपीड्य वामेन करेणामित्रकर्शनः}
{गाण्डीवधन्वा संक्रुद्धः शितान्सन्नतपर्वणः}


\twolineshloka
{जीवितान्तकरान्घोरान्समादत्त शिलीमुखान्}
{तैस्तूर्णं समरेऽविध्यद्रौणिं बलवतां वरः}


\twolineshloka
{तस्य ते कवचं भित्त्वा पपुः शोणितमाहवे}
{न विव्यथे च निर्भिन्नो द्रौणिर्गाण्डीवधन्वना}


\twolineshloka
{तथैव च शरान्द्रौणिः प्रविमुञ्चन्नविह्वलः}
{तस्थौ च समरे राजंस्त्रातुमिच्छन्महाव्रतम्}


\twolineshloka
{तस्य तत्सुमहत्कर्म शशंसुः कुरुसत्तमाः}
{यत्कृष्णाभ्यां समेताभ्यामभ्यापतत संयुगे}


\twolineshloka
{स हि नित्यमनीकेषु युध्यतेऽभयमास्तितः}
{अस्त्रग्रामं ससंहारं द्रोणात्प्राप्य सुदुर्लभम्}


\twolineshloka
{ममैष आचार्यसुतो द्रोणस्यापि प्रियः सुतः}
{ब्राह्मणश्च विशेषेण माननीयो ममेति च}


\twolineshloka
{समास्थआय मतिं वीरो बीभत्सुः सत्रुतापनः}
{कृपां चक्रे रथश्रेष्ठो भारद्वाजसुतं प्रति}


\twolineshloka
{द्रौणिं त्यक्त्वा ततो युद्धे कौन्तेयः श्वेतवाहनः}
{युयुधे तावकान्निघ्नंस्त्वरमाणः पराक्रमी}


\twolineshloka
{दुर्योधनस्तु दशभिर्गार्ध्रपत्रैः शिलाशितैः}
{भीमसेनं महेष्वासं रुक्मपुङ्खैः समार्पयत्}


\twolineshloka
{भीमसेनः सुसंक्रुद्धः परासुकरणं दृढम्}
{चित्रं कार्मुकमादत्त शरांश्च निशितान्दश}


\twolineshloka
{आकर्णप्रहितैस्तीक्ष्णैर्वेगवद्भिरजिह्मगैः}
{अविध्यत्तूर्णमव्यग्रः कुरुराजं महोरसि}


\twolineshloka
{तस्य काञ्चनसूत्रस्थः शरैः संछादितो मणिः}
{रराजोरसि खे सूर्यो ग्रहैरिव समावृतः}


\twolineshloka
{पुत्रस्तु तव तेजस्वी भीमसेनेन ताडित}
{नामृष्यत यथा नागस्तलशब्दं मदोत्कटः}


\twolineshloka
{ततः शरैर्महाराज रुक्मपुङ्खैः शिलाशितैः}
{भीमं विव्याध संक्रुद्धस्त्रासयानो वरूथिनीम्}


\twolineshloka
{तौ युध्यमानौ समरे भृशमन्योन्यविक्षतौ}
{पुत्रौ ते देवसंकाशौ व्यरोचेतां महाबलौ}


\twolineshloka
{चित्रसेनं नरव्याघ्रं सौभद्रः परवीरहा}
{अविध्यद्दशभिर्बाणैः पुरुमित्रं च सप्तभिः}


\twolineshloka
{सत्यव्रतं च सप्तत्या विद्ध्वा शक्रसमो युधि}
{नृत्यन्निव रणे वीर आर्तिं नः समजीजनत्}


\twolineshloka
{तं प्रत्यविध्यद्दशभिश्चित्रसेनः शिलीमुखैः}
{सत्यव्रतश्च नवभिः पुरुमित्रश्च सप्तभिः}


\twolineshloka
{स विद्धो विक्षरन्रक्तं शत्रुसंवारणं महत्}
{चिच्छेद चित्रसेनस्य चित्रं कार्मुकमार्जुनिः}


\twolineshloka
{भित्त्वा चास्य तनुत्राणं शरेणोरस्यताडयत्}
{ततस्ते तावका वीरा राजपुत्रा महारथाः}


\twolineshloka
{समेत्य युधि संरब्धा विव्यधुर्निशितैः शरैः}
{तांश्च सर्वाञ्शरैस्तीक्ष्णैर्जघान परमास्त्रवित्}


\twolineshloka
{तस्य दृष्ट्वा तु तत्कर्म परिवव्रुः सुतास्तव}
{दहन्तं समरे सैन्यं वने कक्षं यथोल्बणम्}


\twolineshloka
{अपेतशिशिरे काले समिद्धमिव पावकम्}
{अत्यरोचत सौभद्रस्तव सैन्यानि नाशयन्}


\twolineshloka
{तत्तस्य चरितं दृष्ट्वा पौत्रस्तव विशांपते}
{लक्ष्मणोऽभ्यपतत्तूर्णं सात्तीपुत्रमाहवे}


\twolineshloka
{अभिमन्युस्तु संक्रुद्धो लक्ष्मणं शुभलक्षणम्}
{विव्याध निशितैः षङभिः सारथिं च त्रिभिः शरैः}


\twolineshloka
{तथैव लक्ष्मणो राजन्सौभद्रं निशितैः शरैः}
{अविध्यत महाराज तदद्भुतमिवाभवत्}


\twolineshloka
{तस्याश्वांश्चतुरो हत्वा सारथिं च महाबलः}
{अभ्यद्रवत सौभद्रो लक्ष्मणं निशितैः शरैः}


\twolineshloka
{हताश्वे तु रथे तिष्ठँल्लक्ष्मणः परवीरहा}
{शक्तिं चिक्षेप संक्रुद्धः सौभद्रस्य रथं प्रति}


\twolineshloka
{तामापतन्तीं सहसा घोररूपां दुरासदाम्}
{अभिमन्युः शरैस्तीक्ष्णैश्चिच्छेद भुजगोपमाम्}


\twolineshloka
{ततः स्वरथमारोप्य लक्ष्मणं गौतमस्तदा}
{अपोवाह रथेनाजौ सर्वसैन्यस्य पश्यतः}


\twolineshloka
{ततः समाकुले तस्मिन्वर्तमाने महाभये}
{अभ्यद्रवञ्जिघांसन्तः परस्परवधैषिणः}


\twolineshloka
{तावकाश्च महेष्वासाः पाण्डवाश्च महारथाः}
{जुह्वन्तः समरे प्राणान्निजघ्नुरितरेतरम्}


\twolineshloka
{मुक्तकेशा विकवचा विरथाश्छिन्नकार्मुकाः}
{बाहुभइः समयुध्यन्त सृञ्जयाः कुरुभिः सह}


\twolineshloka
{ततो भीष्मो महाबाहुः पाण्डवानां महात्मनाम्}
{सेनां जघान संक्रुद्धो दिव्यैरस्त्रैर्महाबलः}


\twolineshloka
{हतेश्वरैर्गजैस्तत्र नरैरश्वैश्च पातितैः}
{रथिभिः सादिभिश्चैव समास्तीर्यत मेदिनी}


\chapter{अध्यायः ७४}
\twolineshloka
{सञ्जय उवाच}
{}


\twolineshloka
{अथ राजन्महाबाहुः सात्यकिर्युद्धदुर्मदः}
{विकृष्य चापं समरे भारसाहमनुत्तमम्}


\threelineshloka
{यत्तत्सख्युस्तु पूर्वेण अर्जुनादुपशिक्षितम्}
{प्रगाढं लघु चित्रं च दर्शयन्हस्तलाघवम्}
{प्रामुञ्चत्पुङ्खसंयुक्ताञ्शरानाशीविषोपमान्}


\twolineshloka
{तस्य विक्षिपतस्छापं शरानन्यांश्च मुञ्चतः}
{आददानस्य भूयश्च संदधानस्य चापरान्}


\twolineshloka
{क्षिपतश्च परांस्तस्य रणे शत्रून्विनिघ्नतः}
{ददृशे रूपमत्यर्थं मेघस्येव प्रवर्षतः}


\twolineshloka
{तमुदीर्यन्तमालोक्य राजा दुर्योधस्ततः}
{रथानामयुतं तस्य प्रेषयामास भारत}


\twolineshloka
{तांस्तु सर्वान्महेष्वासान्सात्यकिः सत्यविक्रमः}
{जघान परमेष्वासो दिव्येनास्त्रेण वीर्यवान्}


\twolineshloka
{स कृत्कवा दारुणं कर्म प्रगृहीतशरासनः}
{आससाद ततो वीरो भूरिश्रवसमाहवे}


\twolineshloka
{स हि संदृश्य सेनां ते युयुधानेन पातिताम्}
{अभ्यधावत संक्रुद्धः कुरूणां कीर्तिवर्धनः}


\twolineshloka
{इन्द्रायुधसवर्णं तु विस्फार्य सुमहद्धनुः}
{सृष्टवान्वज्रसंकाशाञ्शरानाशीविषोपमान्}


\twolineshloka
{सहस्रशो माहाराज दर्शयन्पाणिलाघवम्}
{शरांस्तान्मृत्युसंस्पर्शान्सात्यकेश्च पदानुगाः}


\twolineshloka
{न विषेहुस्तदा राजन्दुद्रुवुस्ते समन्ततः}
{विहाय सात्यकिं राजन्समरे युद्धदुर्मदम्}


\twolineshloka
{तं दृष्ट्वा युयुधानस्य सुता दश महाबलाः}
{महारथाःक समाख्याताश्चित्रवर्मायुधध्वजाः}


\twolineshloka
{समासाद्य महेष्वासं भूरिश्रवसमाहवे}
{ऊचुः सर्वे सुसंरब्धा यूपकेतुं महारणे}


\twolineshloka
{भोभो कौरवादायाद सहास्माभिर्महाबल}
{एहि युध्यस्व संग्रामे समस्तैः पृथगेव वा}


\twolineshloka
{अस्मान्वा त्वं पराजित्य यशः प्राप्नुहि संयुगे}
{वयं वा त्वां पराजित्य प्रीतिं धास्यामहे पितुः}


\twolineshloka
{एवमुक्तस्तदा शूरैस्तानुवाच महाबलः}
{वीर्यश्लाघी नरश्रेष्ठस्तान्दृष्ट्वा समवस्थितान्}


\twolineshloka
{साध्विदं कथ्यते वीरा यंद्येवं मतिरद्य वः}
{युध्यध्वं सहिता यत्ता निहनिष्यामि वो रणे}


\twolineshloka
{एवमुक्ता महेष्वासास्ते वीराः क्षिप्रकारिणः}
{महता शरवर्षेण अभ्यधावन्नरिन्दमम्}


\twolineshloka
{सोऽपराह्णे महाराज संग्रामस्तुमुलोऽभवत्}
{एकस्य च बहूनां च समेतानां रणाजिरे}


\twolineshloka
{तमेकं रथिनां श्रेष्ठं शरैस्ते समवाकिरन्}
{प्रावृषीव यथा मेरुं सिषिचुर्जलदा नृप}


\twolineshloka
{तैस्तु मुक्ताञ्शरान्घोरान्यमदण्डाशनिप्रभान्}
{असंप्राप्तानसंभ्राकन्तिश्चिच्छेदाशु महारथः}


\twolineshloka
{तत्राद्भुतमपश्याम सौमदत्तेः पराक्रमम्}
{यदेको बहुबिर्युद्धे समसञ्जदभीतवत्}


\twolineshloka
{विसृज्य शरवृष्टिं तां दश राजन्महारथाः}
{परिवार्य महाबाहुं निहन्तुमुपचक्रमुः}


\twolineshloka
{समदत्तिस्ततः क्रुद्धस्तेषां चापानि भारत}
{चिच्छेद समरे राजन्युध्यमानो महारथैः}


\twolineshloka
{अथैषां छिन्नधनुषां शरैः सन्नतपर्वभिः}
{चिच्छेद समरे राजञ्शिरांसि भरतर्षभ}


\twolineshloka
{ते हता न्यपतन्राजन्वज्रभग्ना इव द्रुमाः}
{6-74-26bतान्दृष्ट्वानिहतान्वीरो रणे पुत्रान्महाबलान्}


\twolineshloka
{वार्ष्णेयो विनदराजन्भूरिश्रवसमभ्ययात्}
{रथं रथेन समरे पीडयित्वा महाबलौ}


\twolineshloka
{तावन्योन्यं हि समरे निहत्य रथवाजिनः}
{विरथावभिवल्गन्तौ समेयातां महारथौ}


\twolineshloka
{प्रगृहीतमहाखङ्गौ तौ चर्मवरधारिणौ}
{शुशुभाते नरव्याघ्रौ युद्धाय समवस्थितौ}


\twolineshloka
{असह्यमसियुद्धाय भूरिश्रवसमाहवे}
{मत्वा वृकोदरस्तूर्णमभिप्लुत्य महारथः}


\twolineshloka
{ततः सात्यकिमभ्येत्य निस्त्रिंशवरधारिणम्}
{भीमसेनस्त्वरन्राजन्रथमारोपयत्तदा}


\twolineshloka
{तवापि तनयो राजन्भूरिश्रवसमाहवे}
{आरोपयद्रथं तूर्णं पश्यतां सर्वधन्विनाम्}


\twolineshloka
{तस्मिस्तथा वर्तमाने रणे भीष्मं महारथम्}
{अयोधयन्त संरब्धाः पाण्डवा भरतर्षभ}


\twolineshloka
{लोहितायति चादित्ये त्वरमाणो धनञ्जयः}
{पञ्चविंशतिसाहस्रान्निजघान महारथान्}


\twolineshloka
{ते हि दुर्योधनादिष्टास्तदा पार्थनिबर्हणे}
{संप्राप्यैव गता नाशं शलभा इव पावकम्}


\twolineshloka
{ततो मत्स्याः केकयाश्च धनुर्वेदविशारदाः}
{परिवव्रुस्तदा पार्थं सहपुत्रं महारथम्}


\twolineshloka
{एवस्मिन्नेव काले तु सूर्येऽस्तमुपगच्छति}
{सर्वेषां चैव सैन्यानां प्रमोहः समजायत}


\twolineshloka
{अवहारं ततश्चक्रे पिता देवव्रतस्तव}
{संध्याकाले महाराज सैन्यानां श्रान्तवाहनः}


\twolineshloka
{पाण्डवानां कुरूणां च परस्परसमागमे}
{ते सेने भृशसंविग्ने ययतुः स्वं निवेशनम्}


\twolineshloka
{ततः स्वशिबिरं गत्वा न्यविशंस्तत्र भारत}
{पाण्डवाः सृञ्जयैः सार्धं कुरवश्च यथाविधिः}


\chapter{अध्यायः ७५}
\twolineshloka
{सञ्जय उवाच}
{}


\twolineshloka
{विहृत्य तु ततो राजन्सहिताः कुरुपाण्डवाः}
{व्यतीतायां तु शर्वर्यां पुनर्युद्धाय निर्ययुः}


\twolineshloka
{ततः शब्दो महानासीत्तव तेषां च भारत}
{युज्यतां रथमुख्यानां कल्प्यतां चैव दन्तिनाम्}


\twolineshloka
{संनह्यतां पदातीनां हयानां चैव भारत}
{शङ्खदुन्दुभिनादश्च तुमुलः सर्वतोऽभवत्}


\twolineshloka
{ततो युधिष्ठिरो राजा धृष्टद्युम्नमभाषत}
{व्यूहं व्यूहस्व पाञ्चाल मकरं शत्रुनाशनम्}


\twolineshloka
{एवमुक्तस्तु पार्थेन धृष्टद्युम्नो महारथः}
{व्यादिदेश यथान्यायं रथिनो रथिनां वरः}


\twolineshloka
{शिरोऽभूद्द्रुपदस्तस्य पाण्डवश्च धनञ्जयः}
{चक्षुषी सहदेवश्च नकुलश्च महारथः}


\twolineshloka
{तुण्डमासीन्महाराज भीमसेनो महाबलः}
{सौभद्रो द्रौपदेयाश्च राक्षसश्च घटोत्कचः}


\threelineshloka
{सात्यकिर्धर्मराजश्च व्यूहग्रीवां समास्थिताः}
{पृष्ठमासीन्महाराज विराटो वाहिनीपतिः ॥ 6-75-9aधृष्टद्युम्नोमहाराज महत्या सेनया वृतः}
{केकया भ्रातरः पञ्च वामपार्श्वं समाश्रिताः}


\twolineshloka
{धृष्टकेतुर्नरव्याघ्रश्चेकितानश्च वीर्यवान्}
{दक्षिणं पक्षमाश्रित्य स्थितौ व्यूहस्य रक्षणे}


\twolineshloka
{पादयोस्तु महाराज स्थितः श्रीमान्महारथः}
{कुन्तिभोजः शतानीको महत्या सेनया वृतः}


\twolineshloka
{शिखण्डी तु महेष्वासः सोमकैः संवृतो बली}
{इरावांश्च ततः पुच्छे मकरस्य व्यवस्थितौ}


\twolineshloka
{एवमेतं महाव्यूहं व्यूह्य भारत पाण़्डवाः}
{सूर्योदये महाराज पुनर्युद्धाय दंशिताःक}


\twolineshloka
{कौरवानभ्ययुस्तूर्णं हस्त्यश्वरथपत्तिभिः}
{समुच्छ्रितैर्ध्वजैश्छत्रैः शस्त्रैश्च विमलैः शितैः}


\twolineshloka
{व्यूढं दृष्ट्वा तु तत्सैन्यं पिता देवव्रतस्तव}
{कौञ्चेन महता राजन्प्रत्यव्यूहत वाहिनीम्}


\twolineshloka
{तस्य तुण्डे महेष्वासो भारद्वाजो व्यरोचत}
{अश्वत्थामा कृपश्चैव चक्षुरास्तां जनेश्वर}


\twolineshloka
{कृतवर्मा तु सहितः काम्भोजैरथ बाह्लिकैः}
{शिरस्यासीन्नरश्रेष्ठः श्रेष्ठः सर्वधनुष्मताम्}


\twolineshloka
{ग्रीवायां शूरसेनश्च तव पुत्रश्च मारिष}
{दुर्योधनो महाराज राजभिर्बहुभिर्वृतः}


\twolineshloka
{प्राग्जोतिषस्तु सहितो मद्रसौवीरकेकयैः}
{उरस्यभून्नरश्रेष्ठ महत्या सेनया वृतः}


\threelineshloka
{पृष्ठे चास्तां महेष्वासावावन्त्यौ सपदानुगौः}
{स्वसेनया च सहितः शुशर्मा प्रस्थलाधिपः}
{वामं पक्षं समाश्रित्य दंशितः समवस्थितः}


\twolineshloka
{तुषारा यवनाश्चैव शकाश्च सह चूचुपैः}
{दक्षिणं पक्षमाश्रित्य स्थिता व्यूहस्त भारत}


\twolineshloka
{श्रुतायुश्च शातायुश्च सौमदत्तिश्च मारिष}
{व्यूहस्य जघने तस्थू रक्षमाणाः परस्परम्}


\twolineshloka
{ततो युद्धाय संजग्मुः पाण्डवाः कौरवैः सह}
{सूर्योदये महाराज प्रावर्तत जनक्षयः}


\twolineshloka
{प्रतीयू रथिनो नामान्नागाश्च रथिनो ययुः}
{हयारोहान्रथारोहा रथिनश्चापि सादिनः}


\twolineshloka
{सादिनश्च हयान्राजन्रथिनश्च महारणे}
{हस्त्यारोहान्हयारोहा रथिनः सादिनस्तथा}


\twolineshloka
{रथिनः पत्तिभिः सार्धं सादिनश्चापि पत्तिभिः}
{अन्योन्यं समरे राजन्प्रत्यधावन्नमर्षिताः}


\twolineshloka
{भीमसेनार्जुनयमैर्गुप्ता चान्यैर्महारथैः}
{शुशुभे पाण्डवी सेना नक्षत्रैरिव शर्वरी}


\twolineshloka
{तथा भीष्मकृपद्रोणशल्यदुर्योधनादिभिः}
{तवापि च बभौ सेना ग्रहैर्द्यौरिव संवृताः}


\twolineshloka
{भीमसेनस्तु कौन्तेयो द्रोणं दृष्ट्वा पराक्रमी}
{अभ्ययाञ्जवनैरश्वैर्भारद्वाजस्य वाहिनीम्}


\twolineshloka
{द्रोणास्तु समरे क्रुद्धो भीमं नवभिरायसैः}
{विव्याध समरश्लाघी मर्माण्युद्दिष्य वीर्यवान्}


\twolineshloka
{दृढाहतस्तो भीमो भारद्वाजस्य संयुगे}
{सारथिं प्रेषयामास यमस्य सदनं प्रति}


\twolineshloka
{स संगृह्य स्कवयं वाहान्भारद्वाजः प्रतापवान्}
{व्यधमत्पाण्डवीं सेनां तूलराशिमिवानलः}


\twolineshloka
{ते वद्यमाना द्रोणेन भीष्मेण च नरोत्तमाः}
{सृञ्जयाः तावकं सैन्यं भीमार्जुनपरिक्षतम्}


\twolineshloka
{तथैव तावकं सैन्यं भीमार्जुनपरिक्षतम्}
{मुह्यते तत्रतत्रैव समदेव वराङ्गना}


\twolineshloka
{अभिद्येतां ततो व्यूहौ तस्मिन्वीरवरक्षये}
{आसीद्व्यतिकरो घोरस्तव तेषां च भारत}


\twolineshloka
{तदद्भुतमपश्याम तावकानां परैः सह}
{एकायनगताः सर्वे यदयुध्यन्त भारत}


\twolineshloka
{प्रतिसंवार्य चास्त्राणि तेऽन्योन्यस्य विशांपते}
{युयुधुः पाण्डवाश्चैव कौरवाश्च महाबलाः}


\chapter{अध्यायः ७६}
\twolineshloka
{धृतराष्ट्र उवाच}
{}


\twolineshloka
{एवं बहुगुणं सैन्यमेवं बहुविधं परम्}
{व्यूढमेवं यथाशास्त्रममोघं चैव सञ्जय}


\twolineshloka
{जुष्टमस्माकमत्यन्तमभिकामं च नः सदा}
{प्रहृष्टं व्यसनोपेतं पुरस्ताद्दृष्टविक्रमम्}


\twolineshloka
{नातिवृद्धमबालं च न कृशं न च पीवरम्}
{लघुवृत्तायतप्रायं सागराकारमव्ययम्}


\twolineshloka
{आत्तसन्नाहशस्त्रं च बहुशस्त्रपरिग्रहम्}
{असियुद्धे नियुद्धे च गदायुद्धे च कोविदम्}


\twolineshloka
{प्रासर्ष्टितोमरेष्वाजौ परिघेष्वायसेषु च}
{भिण्डिपालेषु शक्तीषु मुसलेषु च सर्वशः}


\twolineshloka
{कम्पनेषु च चापेषु कणपेषु च सर्वशः}
{क्षेणणीयेषु चित्रेषु मुष्टयुद्धेषु च क्षमम्}


\twolineshloka
{अपरोक्षं च विद्यासु व्यायामे च कृतश्रमम्}
{शस्त्रग्रहणविद्यासु सर्वासु परिनिष्ठितम्}


\twolineshloka
{आरोहे पर्यवस्कन्दे सरणे सान्तरप्लुते}
{सम्यक्प्रहरणे याने व्यपयाने च कोविदम्}


\twolineshloka
{नागाश्वरथयानेषु बहुशः सुपरीक्षितम्}
{परीक्ष्य च यथान्यायं वेतनेनोपपादितम्}


\twolineshloka
{न गोष्ठ्या नोपकारेण न च बन्धुनिमित्ततः}
{न सौहृदबलैर्वापि नाकुलीनपरिग्रहैः}


\twolineshloka
{समृद्धजनमार्यं च तुष्टसंबन्धिबान्धवम्}
{कृतोपकारभूयिष्ठं यशस्वि च मनस्वि च}


\twolineshloka
{स्वजनैस्तु नरैर्मुख्यैर्बहुशो दृष्टकर्मभिः}
{लोकपालोपमैस्तात पालितं लोकविश्रुतम्}


\twolineshloka
{बहुभिः क्षत्रियैर्गुप्तं पृथिव्यां लोकसंमतैः}
{अस्मानभिगतैः कामात्सबलैः सपदानुगैः}


\twolineshloka
{महोदधिमिवापूर्णमापगाभिः समन्ततः}
{अपक्षपक्षिसंकाशै रथैर्नागैश्च संवृतम्}


\twolineshloka
{नानायोधजलं भीमं वाहनोर्मितरङ्गिणम्}
{क्षेपण्यसिगदाशक्तिशरप्राससामाकुलम्}


\twolineshloka
{ध्वजभूषणसंबाधं रत्नपट्टसुसंचितम्}
{परिधावद्भिरश्वैश्च वायुवेगविकम्पितम्}


\twolineshloka
{अपारमिव गर्जन्तं सागरप्रतिमं महत्}
{द्रोणभीष्मासिसंगप्तं गुप्तं च कृतवर्मणा}


\twolineshloka
{कृपदुःशासनाभ्यां च जयद्रथमुखैस्तथा}
{भगदत्तविकर्णाभ्यां द्रौणिसौबलबाह्लिकैः}


\twolineshloka
{गुप्तं प्रवीरैर्लौकैश्च सारवद्भिर्महात्मभिः}
{यदहन्यत सैन्यं मे दिष्टमत्र परायणम्}


\twolineshloka
{नैतादृशं समुद्योगं दृष्टवन्तो हि मानुषाः}
{ऋषयो वा महाभागाः पुराणा भुवि सञ्जय}


\twolineshloka
{ईदृशोऽपि बलौघस्तु संयुक्तः शस्त्रसंपदा}
{वध्यते यत्र संग्रामे किमन्यद्भागधेयतः}


\twolineshloka
{विपरीतमिदं सर्वं प्रतिभाति हि सञ्जय}
{यत्रेदृशं बलं घोरं नावधीद्युधि पाण्डवान्}


\twolineshloka
{पाण्डवार्थाय नियतं देवास्तत्र समागताः}
{युध्यन्ते मामकं सैन्यं यथाऽवध्यत सञ्जय}


\twolineshloka
{उक्तोऽपि विदुरेणाहं हितं पथ्यं च नित्यशः}
{न च जग्राह तन्मन्दः पुत्रो दुर्योधनो मम}


\twolineshloka
{तथ्यां मन्ये मतिं तस्य सर्वज्ञस्य महात्मनः}
{आसीत्तथाऽऽगतं तात येन दृष्टमिदं पुरा}


\twolineshloka
{अथवा भाव्यमेवं हि संजयैतेन सर्वथा}
{पुरा धात्रा यथा दिष्टं तत्तथा न तदन्यथा}


\chapter{अध्यायः ७७}
\twolineshloka
{सञ्जय उवाच}
{}


\twolineshloka
{आत्मदोषात्त्वया राजन्प्राप्तं व्यसनमीदृशम्}
{न हि दुर्योधनस्तानि पश्यते भरतर्षभ}


\twolineshloka
{यानि त्वं पश्यसे राजन्धर्मसंकरकारणात्}
{तव दोषात्पुरा वृत्तं द्यूतमेतद्विशांपते}


\twolineshloka
{तव दोषेण युद्दं च प्रवृत्तं सह पाण्डवैः}
{त्वमेवाद्य फलं भुङ्क्ष्व कृत्वा किल्बिषमात्मना}


\twolineshloka
{आत्मना हि कृतं कर्म आत्मनैवोपभुज्यते}
{इह वा प्रेत्य वा राजंस्त्वया प्राप्तं यथातथम्}


\twolineshloka
{तस्माद्राजन्स्थिरो भूत्वा प्राप्येदं व्यसनं महत्}
{शृणु युद्धं यथा वृत्तं शंसतो मे नराधिप}


\twolineshloka
{भीमसेनः सुनिशितैर्बाणैर्भित्त्वा महाचमूम्}
{आससाद ततो वीरः सर्वान्दुर्योधनानुजान्}


\twolineshloka
{दुःशासनं दुर्विषहं दुःसहं दुर्मदं जयम्}
{जयत्सेनं विकर्णं च चित्रसेनं सुदर्शनम्}


\twolineshloka
{चारुचित्रं सुवर्माणं दुष्कर्णं कर्णमेव च}
{एतांश्चान्यांश्च सुबहून्समीपस्थान्महारथान्}


\twolineshloka
{धार्तराष्ट्रान्सुसंक्रुद्धान्दृष्ट्वा भीमो महारथः}
{भीष्मेण समरे गुप्तां प्रविवेश महाचमूम्}


\twolineshloka
{अथालोक्य प्रविष्टं तमूचुस्ते सर्व एव तु}
{जीवग्राहं निगृह्णीमो वयमेनं नराधिपाः}


\twolineshloka
{स तैः परिवृतः पार्थो भ्रातृभिः कृतनिश्चयैः}
{प्रजासंहरणे सूर्यः क्रूरैरिव महाग्रहैः}


\threelineshloka
{संप्राप्य मध्यं सैन्यस्य न भीः पाण्डवमाविशत्}
{यथा देवासुरे युद्धे महेन्द्रं प्राप्य दानवान्}
{}


\twolineshloka
{ततः शतसहस्राणि रथिनां सर्वशः प्रभो}
{उद्यतानि शरैस्तीव्रैस्तमेकं परिवव्रिरे}


\twolineshloka
{स तेषां प्रवरान्योधान्हस्त्यश्वरथसादिनः}
{जघान समरे शूरो धार्तराष्ट्रानचिन्तयन्}


\twolineshloka
{तेषां व्यवसितं ज्ञात्वा भीमसेनो जिघृक्षताम्}
{समस्तानां वधे राजन्मतिं चक्रे महामनाः}


\twolineshloka
{ततो रथं समुत्सृज्य गदामादाय पाण्डवः}
{उवाच सारथिं भीमं स्थीयतामिति भारत}


\threelineshloka
{यावदेतान्हनिष्यामि धार्तराष्ट्रान्सहानुगान्}
{इत्युक्त्वा भीमसेनस्तु प्रविश्य महतीं चमूम्}
{जघान धार्तराष्ट्राणां तद्बलौघमहार्णवम्}


\twolineshloka
{गदया भिमसेनेन ताडिता वारणोत्तमाः}
{भिन्नकुम्भा महाकाया भिन्नपृष्ठास्तथैवच}


\twolineshloka
{भिन्नगात्राः सहारोहैः शेरते पर्वता इव}
{रथाश्च भग्नास्तिलशः सयोधाः शतशो रणे}


\twolineshloka
{अश्वाश्च सादिनश्चैव पादातैः सह भारत}
{तत्राद्भुतमपश्याम भीमसेनस्य विक्रमम्}


\twolineshloka
{यदेकः समरे राजन्बहुभिः समयोधयत्}
{अन्तकाले प्रजाः सर्वा दण्डपाणिरिवान्तकः}


\twolineshloka
{भीमसेने प्रविष्टे तु धृष्टद्युम्नोऽपि पार्षतः}
{द्रोणमुत्सृज्य तरसा ययौ यत्र वृकोदरः}


\twolineshloka
{विदार्य महतीं सेनां तावकानां नरर्षभः}
{आससाद रथं शून्यं भीमसेनस्य संयुगे}


\twolineshloka
{दृष्ट्वा विशोकं समरे भीमसेनस्य सारथिम्}
{धृष्टद्युम्नो महाराज दुर्मना गतचेतनः}


\twolineshloka
{अपृच्छद्बाष्पसंरुद्धो निःश्वसन्वाचमीरयन्}
{मम प्राणैः प्रियतमः क्व भीम इति दुःखितः}


\twolineshloka
{विशोकस्तमुवाचेदं धृष्टद्युम्नं कृताञ्जलिः}
{संस्थाप्य मामिह बली पाण्डवेयः पराक्रमी}


\twolineshloka
{प्रविष्टो धार्तराष्ट्राणामेतद्बलमहार्णवम्}
{मामुक्त्वा पुरुषव्याघ्रः प्रीतियुक्तमिदं वचः}


\threelineshloka
{प्रतिपालय मां सूत नियम्याश्वान्मुहूर्तकम्}
{यावदेवान्निहन्म्यद्य य इमे मद्वधोद्यताः}
{अभ्यधावद्गदापाणिस्तद्बलं स महाबलः}


\twolineshloka
{ततो दृष्ट्वा प्रधावन्तं गदाहस्तं महाबलम्}
{सर्वेषामेव सैन्यानां संहर्षः समजायत}


\twolineshloka
{तस्मिन्सुतुमुले युद्धे वर्तमाने भयानके}
{भित्त्वा राजन्महाव्यूहं प्रविवेश सखा तव}


\twolineshloka
{विशोकस्य वचः श्रुत्वा धृष्टद्युम्नोऽथ पार्षतः}
{प्रत्युवाच ततः सूतं रणमध्ये महाबलः}


\twolineshloka
{न हि मे जीवितेनापि विद्यतेऽद्य प्रयोजनम्}
{भीमसेनं रणे हित्वा स्नेहमुत्सृज्य पाण्डवैः}


\twolineshloka
{यदि यामि विना भीमं किं मां क्षत्रं वदिष्यति}
{एकायनगते भीमे मयि चावस्थिते युधि}


\twolineshloka
{तस्य न स्वस्ति कुर्वन्ति देवाः शक्रपुरोगमाः}
{यः सहायान्परित्यज्य स्वस्तिमानाव्रजेद्गृहम्}


\threelineshloka
{`रौरवे नरके मञ्जेदप्लवे दुस्तरे नृभिः}
{'मम भीमः सखा चैव संबन्धी च महाबलः}
{भक्तोऽस्मान्भक्तिमांश्चाहं तमप्यरिनिषूदनम्}


\twolineshloka
{सोऽहं तत्र गमिष्यामि यत्र यातो वृकोदरः}
{निघ्नन्तं मां रिपून्पश्य दानवानिव वासवम्}


\twolineshloka
{एवमुक्त्वा ततो वीरो ययौ मध्येन वाहिनीम्}
{भीमसेनस्य मार्गेषु गदाप्रमथितैर्गजैः}


\twolineshloka
{स ददर्श तदा भीमं दहन्तं रिपुवाहिनीम्}
{वातो वृक्षानिव बलात्प्रभञ्जन्तं रणे रिपून्}


\twolineshloka
{ते वध्यमानाः समरे रथिनः सादिनस्तथा}
{पादाता दन्तिनश्चैव चक्रुरात्स्वरं महत्}


\twolineshloka
{हाहाकारश्च संजज्ञे तव सैन्यस्य मारिष}
{वध्यतो भीमसेनेन कृतिना चित्रयोधिना}


\twolineshloka
{ततः कृतास्त्रास्ते सर्वे परिवार्य वृकोदरम्}
{अभीताः समवर्तन्त शस्त्रवृष्ट्या परंतप}


\twolineshloka
{अभिद्रुतं शस्त्रभृतां वरिष्ठंसमन्ततः पाण्डवं लोकवीरः}
{सैन्येन घोरेण सुसंहितेनदृष्ट्वा बली पार्षतो भीमसेनम्}


\twolineshloka
{अथोपगच्छच्छरविक्षताङ्गंपदातिनं क्रोधविषं वमन्तम्}
{आश्वासयन्पार्षतो भीमसेनंगदाहस्तं कालमिवान्तकाले}


\twolineshloka
{विशल्यमेनं च चकार तूर्ण-मारोपयच्चात्मरथे महात्मा}
{भृशं परिष्वज्य च भीमसेन-माश्वासयामास स शत्रुमध्ये}


\twolineshloka
{तथा तस्मिन्वर्तमानेऽतिवेगंभ्रातॄनथोपेत्य तवापि पुत्रः}
{तस्मिन्विमर्दे तव संप्रवृत्तेदृष्ट्वा रणे वाक्यमिदं बभाषे}


% Check verse!
अयं दुरात्मा द्रुपदस्य पुत्रःसमागतो भीमसेनेन सार्धम्
\twolineshloka
{तं याम सर्वे महता बलेनमा वो रिपुः प्रार्थयतामनीकम्}
{श्रुत्वा तु वाक्यं तममृष्यमाणाज्येष्ठज्ञया नोदिता धार्तराष्ट्राः}


\twolineshloka
{वधाय निष्पेतुरुदायुधास्तेयुगक्षये केतवो यद्वदुग्राः}
{प्रगृह्य चास्त्राणि धनूंषि वीराज्यां नेमिघोषैः प्रविकम्पयन्तः}


\twolineshloka
{शरैरवर्षन्द्रुपदस्य पुत्रंयथाम्बुदा भूधरं वारिजालैः}
{निहत्य तांश्चापि शरैः सुतीक्ष्णै-र्न विव्यथे समरे चित्रयोधी}


\fourlineindentedshloka
{समभ्युदीर्णांश्च तवात्मजांस्तथानिशाम्य वीरानभितः स्थितान्रणे}
{जिघांसुरुग्रो द्रुपदात्मजो युवा}
{प्रमोहनास्त्रं युयुजे महारथः}
{क्रुद्धो भृशं तव पुत्रेषु राज-न्दैत्येषु यद्वत्समरे महेन्द्रः}


\twolineshloka
{`स वै ततोऽस्त्रं सुमहाप्रभावं'प्रमोहनं द्रोणदत्तं महात्मा}
{प्रयोजयामास उदारकर्मातस्मिन्रणे तव सैन्यस्य राजन् ॥'}


\twolineshloka
{ततो व्यमुह्यन्त रणे नृवीराःप्रमोहनास्त्राहतबुद्धिसत्वाःप्रदुद्रुवुः कुरवश्चैव सर्वेसवाजिनागाः सरथाः समन्तात्}
{परीतकालानिव नष्टसंज्ञा-न्मोहोपेतांस्तव पुत्रान्निशम्य}


\twolineshloka
{एतस्मिन्नेव काले तु भीमः प्रहरतां वरः}
{विश्रम्य च तदा राजन्पीत्वाऽमृतरसं जलम्}


\twolineshloka
{पुनः सन्नह्य संक्रुद्धो योधयामास संयुगे}
{धृष्टद्यम्नेन सहितः कालयामास भारत}


\twolineshloka
{एतस्मिन्नन्तरे राजन्द्रोणः शस्त्रभृतां वरः}
{द्रुपदं त्रिभिरासाद्य शरैर्विव्याध दारुणैः}


\threelineshloka
{सोऽतिविद्धस्ततो राजन्रणे द्रोणेन पार्थिवः}
{अपायाद्द्रुपदो राजन्पूर्ववैरमनुस्मरन्}
{}


\twolineshloka
{जित्वा तु द्रुपदं द्रोणः शङ्खं दध्मौ प्रतापवान्}
{तस्य शङ्खस्वनं श्रुत्वा वित्रेसुः सर्वसोमकाः}


\twolineshloka
{अथ शुश्राव तेजस्वी द्रोणः शस्त्रभृतां वरः}
{प्रमोहनास्त्रेण रणे मोहितानात्मजांस्तव}


\twolineshloka
{ततो द्रोणो महाराज त्वरितोऽभ्याययौ रणात्}
{तत्रापश्यन्महेष्वासो भारद्वाजः प्रतापवान्}


\twolineshloka
{धृष्टद्युम्नं च भीमं च विचरन्तौ महारणे}
{मोहाविष्टांश्च ते पुत्रानपश्यत्स महारथः}


\twolineshloka
{ततः प्रज्ञांस्त्रमदाय मोहनास्त्रं व्यनाशयत्}
{ततः प्रत्यागतप्राणास्तव पुत्रा महारथाः}


\twolineshloka
{पुनर्युद्धाय समरे प्रत्युद्याता जिगीषवः}
{ततो युधिष्ठिरः प्राह समाहूय स्वसैनिकान्}


\twolineshloka
{गच्छन्तु पदवीं शक्त्या भीमपार्षतयोर्युधि}
{सौभद्रप्रमुखा वीरा रथा द्वादश दंशिताः}


\twolineshloka
{प्रवृत्तिमधिगच्छन्तु न हि शुद्ध्यति मे मनः}
{प्रवृत्तिर्भिमसेनस्य पार्षतस्य च संयुगे}


\twolineshloka
{विज्ञेया समरे शीघ्रं प्रविशध्वं रथार्णवम्}
{गच्छन्तु परया शक्त्या भवन्त इति मे मतिः}


\threelineshloka
{त एवं समनुज्ञाताः शूरा विक्रान्तयोधिनः}
{बाढमित्येवमुक्त्वा तु सर्वे पुरुषमानिनः}
{मध्यंदिनगते सूर्ये प्रययुः सर्व एव हि}


\twolineshloka
{केकया द्रौपदेयाश्च धृष्टकेतुश्च वीर्यवान्}
{अभिमन्युं पुरस्कृत्य महत्या सेनया वृताः}


\twolineshloka
{ते कृत्वा समरे व्यूहं सूचीमुखमरिन्दमम्}
{बिभिदुर्धार्तराष्ट्राणां तद्रथानीकमाहवे}


\twolineshloka
{तान्प्रयातान्महेष्वासानभिमन्युपुरोगमान्}
{भीमसेनभयाविष्टा धृष्टद्युम्नविमोहिता}


\twolineshloka
{न संवारयितुं शक्ता तव सेना जनाधिप}
{मदमूर्च्छन्वितात्मा वै प्रमदेवाध्वनि स्थिता}


\twolineshloka
{तेऽभिजाता महेष्वासाः सुवर्णविकृतध्वजाः}
{परीप्सन्तोऽभ्यधावन्त धृष्टद्युम्रवृकोदरौ}


\twolineshloka
{तौ च दृष्ट्वा महेष्वासावभिमन्युपुरोगमान्}
{बभूवतुर्मुदायुक्तौ निघ्नन्तौ तव वाहिनीम्}


\threelineshloka
{`द्रोणमिष्वस्त्रकुशलं सर्वविद्यासु पारगम्}
{'दृष्ट्वा तु सहसायान्तं पाञ्चाल्यो गुरुमात्मनः}
{नाशंसत वधं वीरः पुत्राणां तव पार्षतः}


\twolineshloka
{ततो रथं समारोप्य कैकेयस्य वृकोदरम्}
{अभ्यधावत्सुसंक्रुद्धो द्रोणमिष्वस्त्रपारगम्}


\twolineshloka
{तस्याभिपततस्तूर्णं भारद्वाजः प्रतापवान्}
{क्रुद्धश्चिच्छेद बाणेन धनुः शत्रुनिबर्हणः}


\twolineshloka
{अन्यांश्च शतशो बाणान्प्रेषयामास पार्षते}
{दुर्योधनहितार्थाय भर्तृपिण्डमनुस्मरन्}


\twolineshloka
{अथान्यद्धनुरादाय पार्षतः परवीरहा}
{द्रोणं विव्याध विंशत्या रुक्मपुङ्खैः शिलाशितैः}


\twolineshloka
{तस्य द्रोणः पुनश्चापं चिच्छेदामित्रकर्शनःक}
{हयांश्च चतुरस्तूर्णं चतुर्भिः सायकोत्तमैः}


\threelineshloka
{वैवस्वतक्षयं घोरं प्रेषयामास भारत}
{सारथिं चास्य भल्लेन प्रेषयामास भारत}
{}


\twolineshloka
{हताश्वात्स रथात्तूर्णमवप्लुत्य महारथः}
{आरुरोह महाबाहुरभिमन्योर्महारथम्}


\twolineshloka
{ततः सरथनागाश्वा समकम्पत वाहिनी}
{पश्यतो भीमसेनस्य पार्षतस्य च पश्यतः}


\twolineshloka
{तत्प्रभग्नं बलं दृष्ट्वा द्रोणेनामिततेजसा}
{नाशक्रुवन्वारयितुं समस्तास्ते महारथाः}


\twolineshloka
{वध्यमानं तु तत्सैन्यं द्रोणेन निशितैः शरैः}
{व्यभ्रमत्तत्रतत्रैव क्षोभ्यमाण इवार्णवः}


\threelineshloka
{तथा दृष्ट्वा च तत्सैन्यं जहृषे तावकं बलम्}
{दृष्ट्वाचार्यं सुसंक्रुद्धं तपन्तं रिपुवाहिनीम्}
{तुष्टुवुः सर्वतो योधाः साधुसाध्विति भारत}


\chapter{अध्यायः ७८}
\twolineshloka
{सञ्जय उवाच}
{}


\twolineshloka
{ततो दुर्योधनो राजा महोत्प्रत्यागतस्मृतिः}
{शरवर्षैः पुनर्भीमं प्रत्यवारयदच्युतम्}


\twolineshloka
{एकीभूतास्ततश्चैव तव पुत्रा महारथाः}
{समेत्य समरे भीमं योधयामासुरुद्यताः}


\twolineshloka
{भीमसेनोऽपि समरे संप्राप्य स्वरथं पुनः}
{समारुह्य महाबाहुर्ययौ येन तवात्मजः}


\twolineshloka
{प्रगृह्य च महावेगं परासुकरणं दृढम्}
{सज्यं शरासनं सङ्ख्ये शरैर्विव्याध ते सुतम्}


\twolineshloka
{ततो दुर्योधनो राजा भीमसेनं महाबलम्}
{नाराचेन सुतीक्ष्णेन भृशं मर्मण्यताडयत्}


\twolineshloka
{सोऽतिविद्धो महेष्वासस्तव पुत्रेण धन्विना}
{क्रोधसंरक्तनयनो वेगेनाक्षिप्य कार्मुकम्}


\twolineshloka
{दुर्योधनं त्रिभिर्बाणैर्बाह्वोरुरसि चार्पयत्}
{स तत्र शुशुभे राजा शिखरैर्गिरिराडिव}


\twolineshloka
{तौ दष्ट्वा समरे क्रुद्धौ विनिघ्नन्तौ परस्परम्}
{दुर्योधनानुजाः सर्वे शूराः संत्यक्तजीविताः}


\twolineshloka
{संस्मृत्य मन्त्रितं पूर्वं निग्रहे भीमकर्मणः}
{निश्चयं परमं कृत्वा निग्रहीतुं प्रचक्रमुः}


\twolineshloka
{तानापतत एवाजौ भीमसेनो महाबलः}
{प्रत्युद्ययौ महाराज गजः प्रतिगजानिव}


\twolineshloka
{भृशं क्रुद्धश्च तेजस्वी नाराचेन समार्पयत्}
{चित्रसेनं महाराज तव पुत्रं महायशाः}


\twolineshloka
{तथेतरांस्तव सुतांस्ताडयामास भारत}
{शरैर्बहुविधैः सङ्ख्ये रुक्मपुङ्खैः सुतेजनैः}


\twolineshloka
{ततः संप्रेक्ष्य पुत्रैस्ते भीमसेनं समावृतम्}
{अभिमन्युप्रभृतयस्ते द्वदश महारथाः}


\twolineshloka
{प्रेषिता धर्मराजेन भीमसेनपदानुगाः}
{प्रतिजग्मुर्महाराज तव पुत्रान्महाबलान्}


\twolineshloka
{दृष्ट्वा रथस्थांस्ताञ्शूरान्सूर्याग्निसमतेजसः}
{सर्वानेव महेष्वासान्भ्राजमानाञ्श्रिया वृतान्}


\twolineshloka
{महाहवे दीष्यमानान्सुवर्णमकुटोञ्ज्वलान्}
{तत्यजुः समरे भीमं तव पुत्रा महाबलाः}


\twolineshloka
{तान्नामृष्यत कौन्तेयो जीवमाना गता इति}
{अन्वीय च पुनः सर्वांस्तव पुत्रानपीडयत्}


\twolineshloka
{अथाभिमन्युः समरे भीमसेनेन संगतः}
{पार्षतेन च ते सर्वे कैकया द्रौपदीसुताः}


\threelineshloka
{तादृष्ट्वा समरे क्रुद्धांस्तव सैन्ये महारथाः}
{दुर्योधनप्रभृतयः प्रगृहीतशरासनाः}
{भृशमर्श्वैः प्रजवितैः प्रययुर्यत्र ते रथाः}


\twolineshloka
{अपराह्णे महाराज प्रावर्तत महारणः}
{तावकानां च बलिनां परेषां चैव भारत}


\twolineshloka
{अभिमन्युर्विकर्णस्य हयान्हत्वा महाहवे}
{अथैनं पञ्चविंशत्या क्षुद्रकाणां समार्पयत्}


\threelineshloka
{हताश्वं रथमुत्सृज्य विकर्णस्तु महारथः}
{आरुरोह रथं राजंश्चित्रसेनस्य भारत}
{योधयामास समरे तदद्भुतमिवाभवत्}


\twolineshloka
{स्थितावेकरथे तौ तु भ्रातरौ कुलवर्धनौ}
{आर्जिनिः शरजालेन च्छादयामास भारत}


\twolineshloka
{चित्रसेनो विकर्णश्च कार्ष्णिं पञ्चभिरायसैः}
{विव्यधाते न चाकम्पत्कार्ष्णिर्मेरुरिव स्थितः}


\twolineshloka
{दुःशासनस्तु समरे केकयान्पञ्च मारिप}
{योधयामास राजेन्द्र तदद्भुतमिवाभवत्}


\twolineshloka
{द्रौपदेया रणे क्रुद्धा दुर्योधनमवारयन्}
{शरैराशीविषाकारैः पुत्रं तव विशांपते}


\twolineshloka
{पुत्रोऽपि तव दुर्धर्षो द्रौपद्यास्तनयान्रणे}
{सायकैर्निशितै राजन्नाजघान पृथक्पृथक्}


\twolineshloka
{तैश्चापि विद्धः शुशुभे रुधिरेम समुक्षितः}
{गिरिः प्रस्रवणैर्यद्वद्गैरिकादिविमिश्रितैः}


\twolineshloka
{भीष्मोऽपि समरे राजन्पाण्डवानामनीकिनीम्}
{कालयामास बलवान्पालः पशुगणानिव}


\twolineshloka
{ततो गाण्डीवनिर्घोषः प्रादुरासीद्विशांपते}
{दक्षिणेन वरूथिन्याः पार्थस्यरीन्विनिघ्नतः}


\twolineshloka
{उत्तस्थुः समरे तत्र कबन्धानि समन्ततः}
{कुरूणां चैव सैन्येषु पाण्डवानां च भारत}


\twolineshloka
{शोणितोदं शरावर्तं गजद्वीपं हयोर्मिणम्}
{रथनौभिर्नरव्याघ्राः प्रतेरुः सैन्यसागरम्}


\twolineshloka
{छिन्नहस्ता विकवचा विदेहाश्च नरोत्तमाः}
{दृश्यन्ते पतितास्तत्र शतशोऽथ सहस्रशः}


\twolineshloka
{निहतैर्मत्तमातङ्गैः शोणितौघपरिप्लुतैः}
{भूर्भाति भरतश्रेष्ठ पर्वतैराचिता यथा}


\twolineshloka
{तत्राद्भुतमपश्याम तव तेषां च भारत}
{न तत्रासीत्पुमान्कश्चिद्यो युद्धं नाभिकाङ्क्षति}


\twolineshloka
{एवं युयुधिरे वीराः प्रार्थयाना महद्यशः}
{तावकाः पाण्डवैः सार्धमाकाङ्क्षन्तो जयं युधि}


\chapter{अध्यायः ७९}
\twolineshloka
{सञ्जय उवाच}
{}


\twolineshloka
{ततो दुर्योधनो राजा लोहितायति भास्करे}
{संग्रामरभसो भीमं हन्तुकामोऽभ्यधावत}


\twolineshloka
{तमायान्तमभिप्रेक्ष्य नृवीरं दृढवैरिणम्}
{भीमसेनः सुसंक्रुद्ध इदं वचनमब्रवीत्}


\twolineshloka
{अयं स कालः संप्राप्तो वर्षपूगाभिवाञ्छितः}
{अद्य त्वां निहनिष्यामि यदि नोत्सृजसे रणम्}


\twolineshloka
{अद्य कुन्त्याः परिक्लेशं वनवासं च कृत्स्नशः}
{द्रौपद्याश्च परिक्लेशं प्रणेष्यामि हते त्वयि}


\twolineshloka
{यत्पुरा मत्सरीभूत्वा पाण्डवानवमन्यसे}
{तस्य पापस्य गान्धारे पश्य व्यसनमागतम्}


\twolineshloka
{कर्णस्य मतमास्थाय सौबलस्य च यत्पुरा}
{अचिन्त्य पाण्डवान्कामाद्यथेष्टं कृतवानसि}


\twolineshloka
{याचमानं च यन्मोहाद्दाशार्हमवमन्यसे}
{उलूकस्य समादेशं यद्ददासि च हृष्टवत्}


\twolineshloka
{तेन त्वां निहनिष्यामि सानुबन्धं सबान्धवम्}
{शमीकरिष्ये तत्पापं यत्पुरा कृतवानसि}


\twolineshloka
{एवमुक्त्वा धनुर्घोरं विकृष्योद्भ्राम्य चासकृत्}
{समाधत्त शरान्घोरान्महाशनिसमप्रभान्}


\twolineshloka
{षड्विंशतिमसं क्रुद्धो मुमोचाशु सुयोधने}
{ज्वलिताग्निशिखाकारान्वज्रकल्पानजिह्मगान्}


\twolineshloka
{ततोऽस्य कार्मुकं द्वाभ्यां सूतं द्वाभ्यां च विव्यधे}
{चतुर्भिरश्वाञ्जवनाननयद्यमसादनम्}


\twolineshloka
{द्वाभ्यां च सुविकृष्टाभ्यां शराभ्यामरिमर्दनः}
{छत्रं चिच्छेद समरे राज्ञस्तस्य नरोत्तम्}


\twolineshloka
{षङ्भिश्च तस्य चिच्छेद ज्वलन्तं ध्वजमुत्तमम्}
{छित्त्वा तं च ननादोच्चैस्तव पुत्रस्य पश्यतः}


\twolineshloka
{रथाच्च स ध्वजः श्रीमान्नानारत्नविभूषितात्}
{पपात सहसा भूमौ विद्युञ्जलधरादिव}


\twolineshloka
{ज्वलन्तं सूर्यसंकाशं नागं मणिमयं शुभम्}
{ध्वजं कुरुपतेश्छिन्नं ददृशुः सर्वपार्थिवाः}


\twolineshloka
{अथैनं दशभिर्बाणैस्तोत्रैरिव महाद्विपम्}
{आजघान रणे वीरं स्मयन्निव महारथः}


\twolineshloka
{स गाढविद्धो व्यथितो भीमसेनेन संयुगे}
{निषसाद रथोपस्थे मूर्च्छाभिहतचेतनः}


\twolineshloka
{ततः स राजा सिन्धूनां रथश्रेष्ठो महाबलः}
{दुर्योधनस्य जग्राह पार्ष्णिं स्वपुरुषैर्वृतः}


\twolineshloka
{कृपश्च रथिनां श्रेष्ठस्तव पुत्रमचेतनम्}
{आरोपयद्रथं राजन्दुर्योधनममर्षणम्}


\twolineshloka
{परिवार्य ततो भीमं जेतुकामो जयद्रथः}
{रथैरनेकसाहस्रैर्भीमस्यावारयद्दिशः}


\twolineshloka
{धृष्टकेतुस्ततो राजन्नभिमन्युश्च वीर्यवान्}
{केकया द्रौपदेयाश्च तव पुत्रानयोधयन्}


\twolineshloka
{चित्रसेनः सुचित्रश्च चित्राङ्गश्चित्रदर्शनः}
{चारुचित्रः सुचारुश्च तथा नन्दोपनन्दकौ}


\twolineshloka
{अष्टावेते महेष्वासाः सुकुमारा यशस्विनः}
{अभिमन्युरथं राजन्समन्तात्पर्यवारयन्}


\twolineshloka
{आजघान ततस्तूर्णमभिमन्युर्महामनाः}
{एकैकं पञ्चभिर्बाणैः शितैः सन्नतपर्वभिः}


\twolineshloka
{वज्रमृत्युप्रतीकाशैर्विचित्रायुधनिःसृतैः}
{अमृष्यमाणास्ते सर्वे सौभद्रं रथसत्तमम्}


\twolineshloka
{ववृषुर्मार्गणैस्तीक्ष्णैर्गिरिं मेरुमिवाम्बुदाः}
{स पीड्यमानः समरे कृतास्त्रो युद्धदुर्मदः}


\twolineshloka
{अभिमन्युर्महाराज तावकान्समकम्पयत्}
{यथा देवासुरे युद्धे वज्रपाणिर्महासुरान्}


\twolineshloka
{विकर्णस्य ततो भल्लान्प्रेषयामास भारत}
{चतुर्दश रथश्रेष्ठो घोरानाशीविषोपमान्}


\twolineshloka
{स तैर्विकर्णस्य रथात्पातयामास वीर्यवान्}
{ध्वजं सूतं हयांश्चैव नृत्यमान इवाहवे}


\twolineshloka
{पुनश्चान्याञ्शरान्पीतानकुण्ठाग्राञ्शिलाशितान्}
{प्रेषयामास संक्रुद्धो विकर्णाय महाबलः}


\twolineshloka
{ते विकर्णं समासाद्य कङ्कबर्हिणवाससः}
{भित्त्वा देहं गता भूमिं ज्वलन्त इव पन्नगाः}


\twolineshloka
{ते शरा हेमपुङ्खाग्रा व्यदृश्यन्त महीतले}
{विकर्णरुधिरक्लिन्ना वमन्त इव शोणितम्}


\twolineshloka
{विकर्णं वीक्ष्य निर्भिन्नं तस्यैवान्ये सहोदराः}
{अभ्यद्रवन्त समरे सौभद्रप्रमुखान्रथान्}


\twolineshloka
{अभियात्वा तथैवान्यान्रथांस्तान्सूर्यवर्चसः}
{अविध्यन्समरेऽन्योन्यं संरम्भाद्युद्धदुर्मदाः}


\twolineshloka
{दुर्मुखः श्रुतकर्माणं विद्ध्वा सप्तभिराशुगैः}
{ध्वजमेकेन चिच्छेद सारथिं चास्य सप्तमिः}


\twolineshloka
{अश्वाञ्जाम्बूनदैर्जालैः प्रच्छन्नान्वातरंहसः}
{जघान षङ्भिरासाद्य सारथिं चाभ्यपातयत्}


\twolineshloka
{स हताश्वे रथे तिष्ठञ्श्रुतकर्मा महारथः}
{शक्तिं चिक्षेप संक्रुद्धो महोल्कां ज्वलितामिव}


\twolineshloka
{सा दुर्मुखस्य विमलं वर्म भित्त्वा यशस्विनः}
{विदार्य प्राविशद्भूभिं दीप्यमाना स्वतेजसा}


\twolineshloka
{दुर्मुखो विह्वलस्तत्र निषसाद रणे विभो}
{विसंज्ञं प्रेक्ष्य ते सर्वे भ्रातरः पर्यवारयन्}


\twolineshloka
{तं दृष्ट्वा विरथं तत्र सुतसोमो महारथः}
{पश्यतां सर्वसैन्यानां रथमारोपयत्स्वकम्}


\twolineshloka
{श्रुतकीर्तिस्तथा वीरो जयत्सेनं सुतं तव}
{अभ्ययात्समरे राजन्हन्तुकामो यशस्विनम्}


\twolineshloka
{तस्य विक्षिपतश्चापं श्रुतकीर्तेर्महास्वनम्}
{चिच्छेद समरे तूर्णं जयत्सेनः सुतस्तव}


\twolineshloka
{क्षुरप्रेण सुतीक्ष्णेन प्रहसन्निव भारत}
{तं दृष्ट्वा च्छिन्नधन्वानं शतानीकः सहोदरम्}


\twolineshloka
{अभ्यपद्यत तेजस्वी सिंहवन्निनदन्मुहुः}
{शतानीकस्तु समरे दृढं विस्पार्य कार्मुकम्}


\twolineshloka
{विव्याध दशभिस्तूर्णं जयत्सेनं शिलीमुखैः}
{ननाद सुमहानादं प्रभिन्न इव वारणः}


\twolineshloka
{अथान्येन सुतीक्ष्णेन सर्वावरणभेदिना}
{शतानीको जयत्सेनं विव्याध हृदये भृशम्}


\twolineshloka
{तथा तस्मिन्वर्तमाने दुष्कर्णो भ्रातुरन्तिके}
{मुमोचास्मैशितान्बाणांस्तीक्ष्णानाशीविषोपमान्चिच्छेद समरे चापं नाकुलेः क्रोधमूर्च्छितः}


\twolineshloka
{अथान्यद्धनुरादाय भारसाहमनुत्तमम्}
{समादत्त शरान्घोराञ्शतानीको महाबलः}


\twolineshloka
{तिष्ठतिष्ठेति चामन्त्र्य दुष्कर्णं भ्रातुरग्रतः}
{मुमोचास्मै शितान्बाणाज्ज्वलितान्पन्नगानिव}


\twolineshloka
{ततोऽस्य धनुरेकेन द्वाभ्यां सूतं च मारिष}
{चिच्छेद समरे तूर्णं तं च विव्याध सप्तभिः}


\twolineshloka
{अश्वान्मनोजवांस्तस्य कर्बुरान्वातरंहसः}
{जघान निशितैस्तूर्णं सर्वान्द्वादशभिः शरैः}


\twolineshloka
{अथापरेण भल्लेन सुयुक्तेनाशुपातिना}
{दुष्कर्णं नाकुलिः क्रुद्धो विव्याध हृदये भृशम्}


\twolineshloka
{स पपात ततो भूमौ वज्राहत इव द्रुमः}
{दुष्कर्णं व्यथितं दृष्ट्वा पञ्च राजन्महारथाः}


\twolineshloka
{दिघांसन्तः शतानीकं सर्वतः पर्यवारयन्}
{छाद्यमानं शरव्रतैः शतानीकं यशस्विनम्}


\twolineshloka
{अभ्यधावन्त संक्रुद्धाः केकयाः पञ्च सोदराः}
{तानभ्यापततः प्रेक्ष्य तव पुत्रा महारथाः}


\twolineshloka
{प्रत्युद्ययुर्महाराज गजानिव महागजाः}
{दुर्मुखो दुर्जयश्चैव तथा दुर्मर्षणो युवा}


\twolineshloka
{शत्रुंजयः सत्रुसहः सर्वे क्रुद्धा यशस्विनः}
{प्रत्युद्याता महाराज केकयान्भ्रातरः समम्}


\twolineshloka
{रथैर्नगरसंकाशैर्हयैर्युक्तैर्मनोजवैः}
{नानावर्णविचित्राभिः पताकाभिरलङ्कृतैः}


\twolineshloka
{शरचापधरा वीरा विचित्रकवचध्वजाः}
{विविशुस्ते परं सैन्यं सिंहा इव वनाद्वनम्}


\twolineshloka
{तेषां सुतुमुलं युद्धं व्यतिषक्तरथद्विपम्}
{अवर्तत महारौद्रं निघ्नतामितरेतरम्}


\twolineshloka
{अन्योन्यागस्कृतां राजन्यमराष्ट्रविवर्धनम्}
{मुहूर्तास्तमिते सूर्ये चक्रुर्युद्धं सुदारुणम्}


\twolineshloka
{रथिनः सादिनश्चाथ व्यकीर्यन्त सहस्रशः}
{ततः शान्तनवः क्रुद्धः शरैः सन्नतपर्वभिः}


\twolineshloka
{नाशयामास सेनां तां भीष्मस्तेषां महात्मनाम्}
{पञ्चालानां च सैन्यानि शरैर्निन्ये यमक्षयम्}


\twolineshloka
{एवं भित्त्वा महेष्वासः पाण्डवानामनीकिनीम्}
{कृत्वाऽवहारं सैन्यानां ययौ स्वशिबिरं नृप}


\twolineshloka
{नाशयामासतुर्वीरौ धृष्टद्युम्नवृकोदरौ}
{करवाणामनीकानि शरैः सन्नतपर्वभिः}


\twolineshloka
{धर्मराजोऽपि संप्रेक्ष्य धृष्टद्युम्नवृकोदरौ}
{मूर्ध्नि चैतावुपाघ्राय प्रहृष्टः शिबिरं ययौ}


\twolineshloka
{` अर्जुनो वासुदेवश्च कौरवाणामनीकिनीम्}
{'हत्वा विद्राव्य च शरैः शिबिरायैव जग्मतुः}


\chapter{अध्यायः ८०}
\twolineshloka
{सञ्जय उवाच}
{}


\twolineshloka
{क्षत्रियास्ते महाराज परस्परकृतागसः}
{जग्मुः स्वशिबिराण्येव रुधिरेण समुक्षिताः}


\twolineshloka
{विश्रम्य च यथान्यायं पूजयित्वा परस्परम्}
{सन्नद्धाः समदृश्यन्त भूयो युद्धचिकीर्षवः}


\twolineshloka
{ततस्तव सुतो राजंश्चिन्तयाभिपरिप्लुतः}
{विस्रवच्छोणिताक्ताङ्गः पप्रच्छेदं पितामहम्}


\twolineshloka
{सैन्यानि रौद्राणि भयानकानिव्यूढानि सम्यग्बहुलध्वजानि}
{विदार्य हत्वा च निपीड्य शूरां-स्ते पाण्डवा लब्धजयाः प्रहृष्टाः}


\twolineshloka
{संमोह्य सर्वान्युधि कीर्तिमन्तोव्यूहं च तं मकरं मृत्युकल्पम्}
{प्रविश्य भीमेन रणे हतोऽस्मिघोरैः शरैर्मृत्युदण्डप्रकाशैः}


\twolineshloka
{क्रुद्धं तमुद्वीक्ष्य भयेन राज-न्संमूर्च्छितो न लभे शान्तिमद्य}
{इच्छे प्रसादात्तव सत्यसन्धप्राप्तुं जयं पाण्डवेयांश्च हन्तुम्}


\twolineshloka
{तेनैवमुक्तः प्रहसन्महात्मादुर्योधनं मन्युगतं विदित्वा}
{तं प्रत्युवाचाविमना मनस्वीगङ्गासुतः सस्त्रभृतां वरिष्ठः}


\twolineshloka
{परेण यत्नेन विगाह्य सेनांसर्वात्मनाऽहं तव राजपुत्र}
{इच्छामि दातुं विजयं सुखं चन चात्मानं छादयेऽहं त्वदर्थे}


\twolineshloka
{एते तु रौद्रा बहवो महारथायशस्विनः शूरतमाः कृतास्त्राः}
{ये पाण्डवानां समरे सहायाजितक्लमा रोषविषं वमन्ति}


\twolineshloka
{केनेह शक्याः सहसा विजेतुंवीर्योद्धताः कृतवैरास्त्वया च}
{अहं ह्येतान्प्रतियोत्स्यामि राज-न्सर्वात्मना जीवितं त्यज्य वीर}


\fourlineindentedshloka
{रणे तवार्थाय महानुभावन जीवितं रक्षितव्यं ममाद्य}
{सर्वांस्तवार्थाय सदेवदैत्याँ-ल्लोकान्दहेयं किमु शत्रुसेनाम् ॥ 6-80-12aतान्पाण्डवान्योधयिष्यामि राजन्प्रियं च ते सर्वमहं करिष्ये}
{सञ्जय उवाच}
{श्रुत्वा पितुस्ते वचनं प्रतीतोदुर्योधनः प्रीतमना बभूव}


\twolineshloka
{सर्वाणि सैन्यानि ततः प्रहृष्टोनिर्गच्छतेत्याह नृपांश्च सर्वान्}
{तदाक्षया तानि विनिर्ययुर्द्रुतंगजाश्वपादातरथायुतानि}


\twolineshloka
{प्रहर्षयुक्तानि तु तानि राज-न्महान्ति नानाविधशस्त्रवन्ति}
{स्थितानि नागाश्वपदातिमन्तिविरेजुराजौ तव राजन्बलानि}


\twolineshloka
{शस्त्रास्त्रविद्भिर्नरवीरयोधै-रधिष्ठिताः सैन्यगणास्त्वदीयाः}
{रथौघपादातगजाश्वसङ्घैःप्रयाद्भिराजौ विधिवत्प्रणुन्नैः}


\twolineshloka
{समुद्धतं वै तरुणार्कवर्णंरजो बभौ च्छादयत्सूर्यश्मीन्}
{रेजुः पताका रथदन्तिसंस्थावातेरिता भ्राम्यमाणाः समन्तात्}


\twolineshloka
{नानालिङ्गैः समरे तत्र राजन्मेघैर्युता विद्युतः खे यथैव}
{वृन्दैः स्थिताश्चापि सुसंप्रयुक्ता-श्चकाशिरे दन्तिगणाः समन्तात्}


\twolineshloka
{धनूंषि विष्फारयतां नृपाणांबभूव शब्दस्तुमुलोऽतिघोरः}
{विमथ्यतो देवमहासुरौघै-र्यथार्णवस्यादियुगे तदानीम्}


\twolineshloka
{तदुग्रनागं बहुरूपवर्णंतवात्मजानां समुदीर्णकोपम्}
{बभूव सैन्यं रिपुसैन्यहन्त-युगान्तमेघौघनिभं तदानीम्}


\chapter{अध्यायः ८१}
\twolineshloka
{सञ्जय उवाच}
{}


\threelineshloka
{अथात्मजं तव पुनर्गाङ्गेयो ध्यानमास्थितम्}
{अब्रवीद्भरतश्रेष्ठः संप्रहर्षकर वचः ॥भीष्म उवाच}
{}


\twolineshloka
{अहं द्रोणश्च शल्यश्च कृतवर्मा च सात्वतः}
{अश्वत्थामा विकर्णश्च भगदत्तोऽथ सौबलः}


\twolineshloka
{विन्दानुविन्दावावन्त्यौ बाह्लीकः सह बाह्लिकैः}
{त्रिगर्तराजो बलवान्मागधश्च सुदुर्जयः}


\threelineshloka
{बृहद्बलश्च कौसल्यश्चित्रसेनो विविंशतिः}
{`कृपश्च सह सोदर्यैस्तव राजन्पदानुगाः}
{'रथाश्च बहुसाहस्राः शोभनाश्च महाध्वजाः}


\twolineshloka
{देशजाश्च हया राजन्स्वारूढा हयसादिभिः}
{गजेन्द्राश्च मदोद्वृत्ताः प्रभिन्नकरटामुखाः}


\twolineshloka
{पादाताश्च तथा शूरा नानाप्रहरणा युधि}
{नानादेशसमुत्पन्नास्त्वदर्थे योद्धुमुद्यताः}


\twolineshloka
{एते चान्ये च बहवस्त्वदर्थे त्यक्तजीविताः}
{देवानपि रणे जेतुं समर्था इति मे मतिः}


\twolineshloka
{अवश्यं हि मया राजंस्तव वाच्यं हितं सदा}
{अशक्याः पाण्डवा जेतुं देवैरपि सवासवैः}


\twolineshloka
{वासुदेवसहायाश्च महेन्द्रसमविक्रमाः}
{सर्वथाऽहं तु राजेन्द्र करिष्ये वचनं तव}


\threelineshloka
{पाण्डवांश्च रणे जेष्ये मां वा जेष्यन्ति पाण्डवाः}
{सञ्जय उवाच}
{एवमुक्त्वा ददावस्मै विशल्यकरणीं शुभाम्}


\twolineshloka
{ओषधीं वीयसंपन्नां विशल्यश्चाभवत्तदा}
{ततः प्रभाते विमले स्वेन सैन्येन वीर्यवान्}


\twolineshloka
{अव्यूहत स्वयं व्यूहं भीष्मो व्यूहविशारदः}
{मण़्डलं मनुजश्रेष्ठो नानाशस्त्रसमाकुलम्}


\twolineshloka
{संपूर्णं योधमुख्यैश्च तथा दन्तिपदातिभिः}
{रथैरनेकसाहस्रैः समन्तत्परिवारितम्}


\twolineshloka
{अश्ववृन्दैर्महद्भिश्च रिष्टितोमरधारिभिः}
{नागेनागे रथाः सप्त सप्त चाश्वा रथेरथे}


\twolineshloka
{अन्वश्वं दश धानुष्का धानुष्के दश चर्मिणः}
{एवं व्यूहं महाराज तव सैन्यस्य दंशितम्}


\twolineshloka
{स्थितं रणाय महते भीष्मेण युधि पालितम्}
{दशाश्वानां सहस्राणि दन्तिनां च तथैव च}


\twolineshloka
{रथानामयुतं चापि पुत्राश्च तव दंशिताः}
{चित्रसेनादयः शूरा अभ्यरक्षन्तितामहम्}


\twolineshloka
{रक्ष्यमाणाः स तैः शूरैर्गोप्यमानाश्च तेन ते}
{सन्नद्धाः समदृश्यन्त राजानश्च महाबलाः}


\twolineshloka
{दुर्योधनस्तु समरे दंशितो रथमास्थितः}
{व्यराजत श्रिया जुष्टो यथा शक्रस्त्रिविष्टपे}


\twolineshloka
{ततः शब्दो महानासीत्पुत्राणां तव भारत}
{रथघोषश्च विपुलो वादित्राणां च निःस्वनः}


\twolineshloka
{भीष्मेण धार्तराष्ट्राणां व्यूढः प्रत्यङ्भुखो युधि}
{मण्डलः स महाव्यूहो दुर्भेद्योऽमित्रघातनः}


\twolineshloka
{सर्वतः शुशुभे राजन्रणेऽरीणां दुरासदः}
{मण्डलं तु समालोक्य व्यूहं परमदुर्ययम्}


\twolineshloka
{स्वयं युधिष्ठिरो राजा वज्रं व्यूहमथाकरोत्}
{तथा व्यूढेष्वनीकेषु यथास्थानमवस्थिताः}


\twolineshloka
{रथिनः सादिनः सर्वे सिंहनादमथानदन्}
{बिभित्सवस्ततो व्यूहं निर्ययुर्युद्धकाङ्क्षिणः}


\twolineshloka
{इतरेतरतः शूराः सहसैन्याः प्रहारिणः}
{भारद्वाजो ययौ मत्स्यं द्रौणिश्चापि सिखण्डिनम्}


\twolineshloka
{स्वयं दुर्योधनो राजा पार्षतं समुपाद्रवत्}
{नकुलः सहदेवश्च मद्रराजानमीयतुः}


\twolineshloka
{विन्दानुविन्दावावन्त्यौ युयुधानमभिद्रुतौ}
{सर्वे नृपास्तु समरे धनंजयमयोधयन्}


\twolineshloka
{भीमसेनो रणे यान्तं हार्दिक्यं समवारयत्}
{चित्रसेनं विकर्णं च तथा दुर्मर्षणं विभुः}


\twolineshloka
{आर्जुनिः समरे राजंस्तव मुत्रानयोधयत्}
{प्राग्ज्योतिषो महेष्वासो हैडिम्बं राक्षसोत्तमम्}


\twolineshloka
{अभिदुद्राव वेगेन मत्तो मत्तमिव द्विपम्}
{अलम्बुसस्तदा राजन्सात्यकिं युद्धदुर्मदम्}


\twolineshloka
{ससैन्यं समरे क्रुद्धो राक्षसः समुपाद्रवत्}
{भूरिश्रवा रणे यत्तो धृष्टकेतुमयोधयत्}


\twolineshloka
{श्रुतायुषं च राजानं धर्मपुत्रो युधिष्ठिरः}
{चेकितानश्च समरे कृपमेवान्वयोधयत्}


\twolineshloka
{शेषाः प्रतिययुर्यत्ता भीष्ममेव महारथम्}
{ततो राजसमूहास्ते परिवव्रुर्धनंजयम्}


\twolineshloka
{शक्तितोमरनाराचगदापरिघपाणयः}
{अर्जुनोऽथ भृशं क्रुद्धो वार्ष्णेयमिदमब्रवीत्}


\twolineshloka
{पश्य माधव सैन्यानि धार्तराष्ट्रस्य संयुगे}
{व्यूढानि व्यूहविदुषा गाङ्गेयेन महात्मना}


\twolineshloka
{युद्धाभिकामाञ्शूरांश्च पश्य माधव दंशितान्}
{त्रिगर्तराजं सहितं भ्रातृभिः पश्य केशव}


\twolineshloka
{अद्यैतान्नाशयिष्यामि पश्यतस्ते जनार्दन}
{य इमे मां यदुश्रेष्ठ योद्धुकामा रणाजिरे}


\twolineshloka
{एतदुक्त्वा तु कौन्तेयो धनुर्ज्यामवमृद्य च}
{ववर्ष शरवर्षाणि नराधिपगणान्प्रति}


\twolineshloka
{तेऽपि तं परमेष्वासाः शरवर्षैरपूरयन्}
{तटाकं वारिधाराभिर्यथा प्रावृषि तोयदाः}


\twolineshloka
{हाहाकारो महानासीत्तव सैन्ये विशांपते}
{छाद्ममानौ रणे कृष्णौ शरैर्दृष्ट्वा महारणे}


\twolineshloka
{देवा देवर्षयश्चैव गन्धर्वाश्च सहोरगैः}
{विस्मयं परमं जग्मुर्दृष्ट्वा कृष्णौ तथाऽऽगतौ}


\twolineshloka
{ततः क्रुद्धोऽर्जुनो राजन्नैन्द्रमस्त्रमुदैरयत्}
{तत्राद्भुतमपश्याम विजयस्य पराक्रमम्}


\twolineshloka
{शस्त्रवृष्टिं परैर्मुक्तां शरौघैर्यदवारयत्}
{न च तत्राप्यनिर्भिन्नः कश्चिदासीद्विशांपते}


\twolineshloka
{तेषां राजसहस्राणां हयानां दन्तिनां तथा}
{द्वाभ्यां त्रिभिः शरैश्चान्यान्पार्थो विव्याध मारिष}


\twolineshloka
{ते हन्यमानाः पार्थेन भीष्मं शान्तनवं ययुः}
{अगाधे मञ्जमानानां भीष्मः पोतोऽभवत्तदा}


\twolineshloka
{आपतद्भिस्तु तैस्तत्र प्रभग्नं तावकं बलम्}
{संचुक्षुभे महाराज वातैरिव महार्णवः}


\chapter{अध्यायः ८२}
\twolineshloka
{सञ्जय उवाच}
{}


\twolineshloka
{प्रवृत्तमात्रे संग्रामे निवृत्ते च सुशर्मणि}
{भग्नेषु चापि वीरेषु पाण्डवेन महात्मना}


\twolineshloka
{क्षुभ्यमाणे बले तूर्णं सागरप्रतिमे तव}
{प्रत्युद्याते च गाङ्गेये त्वरितं विजयं प्रति}


\threelineshloka
{दृष्ट्वा दुर्योधनो राजा रणे पार्थस्य विक्रमम्}
{त्वरमाणः समभ्येत्य सर्वांस्तानब्रवीन्नृपान् ॥ 6-82-4aतेषांतु प्रमुखे शूरं सुशर्माणं महाबलम्}
{मध्ये सर्वस्य सैन्यस्य भृशं संहर्षयन्निव}


\twolineshloka
{एष भीष्मः शान्तनवो योद्धुकामो धनञ्जयम्}
{सर्वात्मना कुरुश्रेष्ठस्त्यक्त्वा जीवितमात्मनः}


\twolineshloka
{तं प्रयान्तं रणे वीरं सर्वसैन्येन भारतम्}
{संयत्ताः समरे सर्वे पालयध्वं पितामहम्}


\twolineshloka
{बाढमित्येवमुक्त्वा तु तान्यनीकानि सर्वशः}
{नरेन्द्राणां महाराज समाजग्मुः पितामहम्}


\twolineshloka
{ततः प्रयातः सहसा भीष्मः शान्तनवोऽर्जुनम्}
{रणे भारतमायान्तमाससाद महाबलः}


\twolineshloka
{महाश्वेताश्वयुक्तेन भीमवानरकेतुना}
{महता मेघनादेन रथेनातिविराजता}


\twolineshloka
{समरे सर्वसैन्यानामुपयानं धनञ्जयम्}
{अभवत्तुमुलो नादो भयाद्दृष्ट्वा किरीटिनम्}


\twolineshloka
{अभीशुहस्तं कृष्णं च दृष्ट्वादित्यमिवापरम्}
{मध्यंदिनगतं सङ्ख्ये न शेकुः प्रतिवीक्षितुम्}


\twolineshloka
{तथा शान्तनवं भीष्मं श्वेताश्वं श्वेतकार्मुकम्}
{न शेकुः पाण्डवा द्रष्टुं श्वेतं ग्रहमिवोदितम्}


\twolineshloka
{स सर्वतः परिवृतस्त्रिगर्तैः सुमहात्मभिः}
{भ्रातृभिः सह पुत्रैश्च तथाऽन्यैश्च महारथैः}


\twolineshloka
{भारद्वाजस्तु समरे मत्स्यं विव्याध पत्रिणा}
{ध्वजं चास्य शरेणाजौ धनुश्चैकेन चिच्छिदे}


\twolineshloka
{तदपास्य धनुश्छिन्नं विराटो वाहिनीपतिः}
{अन्यदादत्त वेगेन धनुर्भारसहं दृढम्}


\twolineshloka
{शरांश्चाशीविषाकाराज्ज्वलितान्पन्नगानिव}
{द्रोणं त्रिभिश्च विव्याध चतुर्भिश्चास्य वाजिनः}


\twolineshloka
{ध्वजमेकेन विव्याधक सारथिं चास्य पञ्चभिः}
{धनुरेकेषुणाऽविध्यत्तत्राक्रुध्यद्द्विजर्षभः}


\twolineshloka
{तस्य द्रोणोऽवधीदश्वाञ्शरैः सन्नतपर्वभिः}
{अष्टाभिर्भरतश्रेष्ठ सूतमेकेन पत्रिणा}


\twolineshloka
{स हताश्वादवप्लुत्य स्यन्दनाद्धतसारथिः}
{आरुरोह रथं तूर्णं पुत्रस्य रथिनां वरः}


\twolineshloka
{ततस्तु तौ पितापुत्रौ भारद्वाजं रथे स्थितौ}
{महता शरवर्षेण वारयामाससतुर्बलात्}


\twolineshloka
{भारद्वाजस्ततः क्रुद्धः शरमाशीविषोपमम्}
{चिक्षेप समरे तूर्णं शङ्खं प्रति जनेश्वरः}


\twolineshloka
{स तस्य हृदयं भित्त्वा पीत्त्वा शोणितमाहवे}
{जगाम धरणीं बाणो लोहितार्द्रवरच्छदः}


\twolineshloka
{स पपात रणे तूर्णं भारद्वाजशराहतः}
{धनुस्त्यक्त्वा शरांश्चैव पितुरेव समीपतः}


\twolineshloka
{हतं तमात्मजं दृष्ट्वा विराटः प्राद्रवद्भयात्}
{उत्सृज्य समरे द्रोणं व्यात्ताननमिवान्तकम्}


\twolineshloka
{भारद्वाजस्ततस्तूर्णं पाण्डवानां महाचमूम्}
{दारयामास समरे शतशोऽथ सहस्रशः}


\twolineshloka
{शिखण्डी तु महाराज द्रौणिमासाद्य संयुगे}
{आजघान भ्रवोर्मध्ये नाराचैस्त्रिभिराशुगैः}


\twolineshloka
{स बभौ रथशार्दूलो ललाटे संस्थितैस्त्रिभिः}
{शिखरैः काञ्चनमयैर्मेरुस्त्रिभिरिवोच्छ्रितैः}


\twolineshloka
{अश्वत्थाम ततः क्रुद्धो निमेषार्धाच्छिखण्डिनः}
{ध्वजं सूतमथो राजंस्तुरगानायुधानि च}


\twolineshloka
{शरैर्बहुभिराच्छिद्य पातयामास संयुगे ष}
{स हताश्वादवप्लुत्य रथाद्वै रथिनां वरः}


\twolineshloka
{खङ्गमादाय सुशितं विमलं च शरावरम्}
{श्येनवद्व्यचरत्क्रुद्धः शिखण्डी शत्रुतापनः}


\twolineshloka
{सखङ्गस्य महाराज चरतस्तस्य संयुगे}
{नान्तरं ददृशे द्रौणिस्तदद्भुतमिवाभवत्}


\twolineshloka
{ततः शरसहस्राणि बहूनि भरतर्षभ}
{प्रेषयामास समरे द्रौणिः परमकोपनः}


\twolineshloka
{तामापतन्तीं समरे शरवृष्टिं सुदारुणाम्}
{असिना तीक्ष्णधारेण चिच्छेद बलिनां वरः}


\twolineshloka
{ततोऽस्य विमलं द्रौणिः शतचन्द्रं मनोरमम्}
{चर्माच्छिनदसिं चास्थ खण्डयामास संयुगे}


\twolineshloka
{शितैस्तु बहुशो राजंस्तं च विव्याध पत्रिभिः}
{शिखण्डी तु ततः खङ्गं खण्डितं तेन सायकैः}


\twolineshloka
{आविध्य व्यसृजत्तूर्णं ज्वलन्तमिव पन्नगम्}
{तमापतन्तं सहसा कालानलसमप्रभम्}


\twolineshloka
{चिच्छेद समरे द्रौणिर्दर्शयन्पाणिलाघवम्}
{शिखण्डिनं च विव्याध शरैर्बहुभिरायसैः}


\twolineshloka
{शिखण्डी तु भृशं राजंस्तड्यमानः शितैः शरैः}
{आरुरोह रथं तूर्णं माधवस्य महात्मनः}


\twolineshloka
{सात्यकिश्चापि संक्रुद्धो राक्षसं क्रूरमाहवे}
{अलम्बुसं शरैस्तीक्ष्णैर्विव्याध बलिनां वरः}


\twolineshloka
{राक्षसेन्द्रस्ततस्तस्य धनुश्चिच्छेद भारत}
{अर्धचन्द्रेण समरे तं च विव्याध सायकैः}


\twolineshloka
{मायां च राक्षसीं कृत्वा शरवर्षैरवाकिरत्}
{तत्राद्भुतमपश्याम शैनेयस्य पराक्रमम्}


\twolineshloka
{असंभ्रमस्तु समरे वध्यमानः शितैः शरैः}
{ऐन्द्रमस्त्रं च वार्ष्णेयो योजयामास भारत}


\twolineshloka
{विजयाद्यदनुप्राप्तं माधवेन यशस्विना}
{तदस्त्रं भस्मासात्कृत्वा मायां तां राक्षसीं तदा}


\twolineshloka
{अलम्बुसं शरैरन्यैरभ्याकिरत सर्वतः}
{पर्वतं वारिधाराभिः प्रावृषीव बलाहकः}


\twolineshloka
{तत्तथा पीडितं तेन माधवेन यशस्विना}
{प्रदुद्राव भयाद्रक्षस्त्यक्त्वा सात्यकिमाहवे}


\twolineshloka
{........ माघवता जित्वा भारत सात्यकिः}
{शैनेयः प्राणद़ज्जित्वा योधानां तव पश्यताम्}


\twolineshloka
{न्यहनत्तावकांश्चापि सात्यकिः सत्यविक्रमः}
{निशितैर्बहुभिर्बाणैस्तेऽद्रवन्त भयार्दिताः}


\twolineshloka
{एतस्मिन्नेव काले तु द्रुपदस्यात्मजो बली}
{धृष्टद्युम्नो महाराज पुत्रं तव जनेश्वरम्}


\twolineshloka
{छादयामास समरे शरैः सन्नतपर्वभिः}
{स च्छाद्यमानो विशिखैर्धृष्टद्युम्नेन भारत}


\twolineshloka
{विव्यथे न च राजेन्द्र तव पुत्रो जनेश्वर}
{धृष्टद्युम्नं च समरे तूर्णं विव्याध पत्रिभिःक}


\twolineshloka
{षष्ट्या च त्रिंशता चैव तदद्भुतमिवाभवत्}
{तस्य सेनापतिः क्रुद्धो धनुश्चिच्छेद मारिष}


\twolineshloka
{हयांश्च चतुरः शीघ्रं निजघान महाबलः}
{शरैश्चैनं सुनिशितैः क्षिप्रं विव्याध सप्तभिः}


\twolineshloka
{स हताश्वान्महाबाहुरवप्लुत्य रथाद्बली}
{पदातिरसिमुद्यम्य प्राद्रवत्पार्षतं प्रति}


\twolineshloka
{शकुनिस्तं समभ्येत्य राजगृद्धी महाबलः}
{राजानं सर्वलोकस्य रथमारोपयत्स्वकम्}


\twolineshloka
{ततो नृपं पराजित्य पार्षतः परवीरहा}
{न्यहनत्तावकं सैन्यं वज्रपाणिरिवासुरान्}


\twolineshloka
{कृतवर्मा रणे भीमं शरैरार्च्छन्महारथः}
{प्रच्छादयामास च तं महामेघो रविं यथा}


\twolineshloka
{ततः प्रहस्य समरे भीमसेनः परंतपः}
{प्रेषयामास संक्रुद्धः सायंकान्कृतवर्मणे}


\twolineshloka
{तैरर्द्यमानोऽतिरथः सात्वतः शस्त्रकोविदः}
{नाकम्पत महाराज भीमं चार्च्छच्छितैः शरैः}


\twolineshloka
{तस्याश्वांश्चतुरो हत्वा भीमसेनो महारथः}
{सारथिं पातयामास सध्वजं सुपरिष्कृतम्}


\twolineshloka
{शरैर्बहुविधैश्चैनमाचिनोत्परवीरहा}
{शकलीकृतसर्वाङ्गः श्वावित्तु शललैर्यथा}


\twolineshloka
{हताश्वश्च ततस्तूर्णं सुबलस्य रथं ययौ}
{स्वालस्य ते महाराज तव पुत्रस्य पश्यतः}


\twolineshloka
{भीमसेनोऽपि संक्रुद्धस्तव सैन्यमुपाद्रवत्}
{निजघान च संक्रुद्धो दण्डपाणिरिवान्तकः}


\chapter{अध्यायः ८३}
\twolineshloka
{धृतराष्ट्र उवाच}
{}


\twolineshloka
{बहूनि हि विचित्राणि द्वैरथानि स्म सञ्जय}
{पाण्डूनां मामकैः सार्धमश्रौषं तव जल्पतः}


\twolineshloka
{न चैव मामकं किंचिद्धृष्टं शंससि सञ्जय}
{नित्यं पाण्डुसुतान्हृष्टानभग्नान्संप्रशंससि}


\threelineshloka
{जीयमानान्विमनसो मामकान्विगतौजसः}
{वदसे संयुगे सूत दिष्टमेतन्न संशयः ॥सञ्जय उवाच}
{}


\twolineshloka
{यथाशक्ति यथोत्साहं युद्धे चेष्टन्ति तावकाः}
{दर्शयानाः परं शक्त्या पौरुषं पुरुषर्षभ}


\twolineshloka
{गङ्गायाः सुरनद्या वै स्वादुभूतं यथोदकम्}
{महोदधिसमभ्याशे लवणत्वं निगच्छति}


\twolineshloka
{तथा तत्पौरुषं राजंस्तावकानां परंतप}
{प्राप्य पाण्डुसुतान्वीरान्व्यर्थं भवति संयुगे}


\twolineshloka
{घटमानान्यथाशक्ति कुर्वाणान्कर्म दुष्करम्}
{न दोषेण कुरुश्रेष्ठ कौरवान्गन्तुमर्हसि}


\twolineshloka
{तवापराधात्सुमहान्सपुत्रस्य विशांपते}
{पृतिव्याः प्रक्षयो घोरो यमराष्ट्रविवर्धनः}


\twolineshloka
{आत्मदोषात्समुत्पन्नं शोचितुं नार्हसे नृप}
{न हि रक्षन्ति राजानः सर्वथाऽत्रापि जीवितम्}


\twolineshloka
{युद्धे सुकृतिनां लोकानिच्छन्तो वसुधाधिपाः}
{चमूं विगाह्य युद्ध्यन्ते नित्यं स्वर्गपरायणाः}


\twolineshloka
{पूर्वाह्णे तु महाराज प्रावर्तत जनक्षयः}
{तं त्वमेकमना भूत्वा शृणु देवासुरोपमम्}


\twolineshloka
{आवन्त्यौ तु महेष्वासौ महासेनौ महाबलौ}
{युधामन्युमभिप्रेक्ष्य समेयता रणोत्कटौ}


\twolineshloka
{तेषां प्रववृते युद्धं समुहद्रोणहर्षणम्}
{युधामन्युः सुसंक्रुद्धो भ्रातरौ देवरूपिणौ}


\twolineshloka
{विव्याध निशितैस्तूर्णं शरैः सन्नतपर्वभिः}
{तावेनं प्रत्यविध्येतां समरे चित्रयोधिनौ}


\twolineshloka
{युध्यतां हि तथा राजन्विशेषो न व्यदृश्यत}
{यततां शत्रुनाशाय कृतप्रतिकृतैषिणाम्}


\twolineshloka
{युधामन्युस्ततो राजन्ननुविन्दस्य सायकैः}
{चतुर्भिश्चतुरो वाहाननयद्यमसादनम्}


\twolineshloka
{भल्लाभ्यां च सुतीक्ष्णाभ्यां धनुः केतुं च मारिष}
{चिच्छेद समरे राजंस्तदद्भुतमिवाभवत्}


\twolineshloka
{त्यक्त्वाऽनुविन्दोऽथ रथं विन्दस्य रथमास्थितः}
{धनुर्गृहीत्वा परमं भारसाधनमुत्तमम्}


\twolineshloka
{तावेकस्थौ रणे वीरावावन्त्यौ रथिनां वरौ}
{शरान्मुमुचतुस्तूर्णं युधामन्यौ महात्मनि}


\twolineshloka
{ताभ्यां मुक्ता महावेगाः शराः काञ्चनभूषणाः}
{दिवाकरपथं प्राप्य च्छादयामासुरम्बरम्}


\twolineshloka
{युधामन्यू रणे क्रुद्धो भ्रातरौ तौ महारथौ}
{ववर्ष शरवर्षेण सारथिं चाप्यपातयत्}


\twolineshloka
{तस्मिंस्तु पतिते भूमौ गतसत्वे तु सारथौ}
{रथः प्रदुद्राव दिशः समुद्धान्तहयस्ततः}


\twolineshloka
{तौ स जित्वा महाराज यज्ञसेनसुतः प्रभुः}
{पौरुषं ख्यापयंस्तूर्णं व्यधमत्तव वाहिनीम्}


\twolineshloka
{सा वध्यमाना समरे धार्तराष्ट्री महाचमूः}
{वेगान्बहुविधांश्चक्रे विषं पीत्वेन मानवः}


\twolineshloka
{हैडिम्बो राक्षसेन्द्रस्तु भगदत्तं समाद्रवत्}
{रथेनादित्यवर्णेन सध्वजेन महाबलः}


\twolineshloka
{ततः प्राग्ज्योतिषो राजा नागराजं समास्थितःक}
{यथा वज्रधरः पूर्वं संग्रामे तारकामये}


\twolineshloka
{तत्र देवाः सगन्धर्वा ऋषयश्च समागताः}
{विशेषं न स्म विविदुर्हैडिम्बभगदत्तयोः}


\twolineshloka
{यथा सुरपतिः शक्रस्त्रासयामास दानवान्}
{तथैव समरे राजा द्रावयामास पाण्डवान्}


\twolineshloka
{तेन विद्राव्यमाणास्ते पाण्डवाः सर्वतो दिशम्}
{त्रातारं नाभ्यगच्छन्तः स्वेष्वनीकेषु भारत}


\twolineshloka
{भैमसेनिं रथस्थं तु तत्रापश्याम भारत}
{शेषा विमनसो भूत्वा प्राद्रवन्त महारथाः}


\twolineshloka
{निवृत्तेषु तु पाण्डूनां पुनः सैन्येषु भारत}
{आसीन्निष्ठानको घोरस्तव सैन्यस्य संयुगे}


\twolineshloka
{घटोत्कचस्ततो राजन्भगदत्तं महारणे}
{शरैः प्रच्छादयामास मेरुं गिरिमिवाम्बुदः}


\twolineshloka
{निहत्य ताञ्शरान्राजा राक्षसस्य धनुश्र्युतान्}
{भैमसेनिं रणे तूर्णं सर्वमर्मस्वताडयत्}


\twolineshloka
{स ताड्यमानो बहुभिः शरैः सन्नतपर्वभिः}
{न विव्यथे राक्षसन्द्रो भिद्यमान इवाचलः}


\twolineshloka
{तस्य प्राग्ज्योतिषः क्रुद्धस्तोमरांश्च चतुर्दश}
{प्रेषयामास समरे तांश्चिच्छेद स राक्षसः}


\twolineshloka
{स तांश्छित्त्वा महाबाहुस्तोमरान्निशितैः शरैः}
{भगदत्तं च विव्याध सप्तत्या कङ्कपत्रिभिः}


\twolineshloka
{ततः प्राग्ज्योतिषो राजा प्रहसन्निव भारत}
{तस्याश्वांश्चतुरः सङ्ख्ये पातयामास सायकैः}


\twolineshloka
{स हताश्वे रथे तिष्ठन्राक्षसेन्द्रः प्रतापवान्}
{शक्तिं चिक्षेप वेगेन प्रग्ज्योतिषगजं प्रति}


\twolineshloka
{तामापतन्तीं सहसा हेमदण्डां सुवेगिनीम्}
{त्रिधा चिच्छेद नृपतिः सा व्यकीर्यत मेदिनीं}


\twolineshloka
{शक्तिं विनिहतां दृष्ट्वा हैडिम्बः प्राद्रवद्भयात्}
{यथेन्द्रस्य रणात्पूर्वं नमुचिर्दैत्यसत्तमः}


\twolineshloka
{तं विजित्य रणे शूरं विक्रान्तं ख्यातपौरुषम्}
{अजय्यं समरे वीरं यमेन वरुणेन च}


\twolineshloka
{पाण्डवीं समरे सेनां संममर्द स कुञ्जरः}
{यथा वनगजो राजन्मृद्गश्चरति पद्मिनीम्}


\twolineshloka
{मद्रेश्वरस्तु समरे यमाभ्यां समसञ्जत}
{स्वस्त्रीयौ छादयांचक्रे शरौघैः पाण्डुनन्दनौ}


\twolineshloka
{सहदेवस्तु समरे मातुलं दृश्य संगतम्}
{अवारयच्छरौघेण मेघो यद्वद्दिवाकरम्}


\twolineshloka
{छाद्यमानः शरौघेण हृष्टरूपतरोऽभवत्}
{तयोश्चाप्यभवत्प्रीतिरतुला मातृकारणात्}


\twolineshloka
{ततः प्रहस्य समरे नकुलस्य महारथः}
{` ध्वजं चिच्छेद बाणेन धनुश्चैकेन मारिष}


\twolineshloka
{अथैनं छिन्नधन्वानं छादयामास भारत}
{निजघान रणे तं तु सूतं चास्य न्यपातयत्}


\twolineshloka
{ततः प्रसह्य समरे नकुलस्य महारथः}
{अश्वांश्च चतुरो राजंश्चतुर्भिः सायकोत्तमैः}


\twolineshloka
{प्रेषयामास समरे यमस्य सदनं प्रति}
{हताश्वात्तु रथात्तूर्णमवप्लुत्य महारथः}


\twolineshloka
{आरुरोह ततो यानं भ्रातुरेव यशस्विनः}
{एकस्थौ तु रणे शूरौ दृढे विक्षिप्य कार्मुके}


\twolineshloka
{मद्रराजरथं तूर्णं च्छादयामासतुः क्षणात्}
{स च्छाद्यमानो बहुभिः शरैः सन्नतपर्वभिः}


\twolineshloka
{स्वस्त्रीयाभ्यां नरव्याघ्रो नाकम्पत यथाऽचलः}
{प्रहसन्निव तां चापि शस्त्रवृष्टिं जघान ह}


\twolineshloka
{सहदेवस्ततः क्रुद्धः शरमुद्गृह्य वीर्यवान्}
{मद्रराजमभिप्रेक्ष्य प्रेषयामास भारत}


\twolineshloka
{स शरः प्रेषितस्तेन गरुडानिलवेगवान्}
{मद्रराजं विनिर्भिद्य निपपात महीतले}


\twolineshloka
{स गाढविद्धो व्यथितो रथोपस्थे महारथः}
{निषसाद महाराज कश्मलं च जगाम ह}


\twolineshloka
{तं विसंज्ञं निपतितं सूतः संप्रेक्ष्य संयुगे}
{अपोवाह रथेनाजौ यमाभ्यामभिपीडितम्}


\twolineshloka
{दृष्ट्वा मद्रेश्वररथं धार्तराष्ट्राः पराङ्भुखम्}
{सर्वे विमनसो भूत्वा नेदमस्तीत्यचिन्तयन्}


\twolineshloka
{निर्जित्य मातुलं सङ्ख्ये माद्रीपुत्रौ महारथौ}
{दध्यतुर्मुदितौ शङ्खौ सिंहनादं च नेदतुः}


\twolineshloka
{अभिदुद्रुवतुर्हृष्टौ तव सैन्यं विशांपते}
{यथा दैत्यचमूं राजन्निन्द्रोपेन्द्राविवामरौ}


\chapter{अध्यायः ८४}
\twolineshloka
{सञ्जय उवाच}
{}


\twolineshloka
{ततो युधिष्ठिरो राजा मध्यं प्राप्ते दिवाकरे}
{श्रुतायुपमभिप्रेक्ष्य प्रेषयामास वाजिनः}


\twolineshloka
{ततस्तु त्वरितो राजञ्श्रुतायुषमरिन्दमन्}
{निजघ्ने सायकैस्तीक्ष्णैर्नवभिर्नतपर्वभिःक}


\twolineshloka
{स संवार्य रमे राजा प्रेषितान्धर्मसूनुना}
{शरान्सप्त महेष्वासः कौन्तेयाया समार्षयत्}


\twolineshloka
{ते तस्व कवचं भित्त्वा पपुः शोणितमाहवे}
{असूनिव विचिन्वन्तो देहे तस्य महात्मनः}


\twolineshloka
{पाण्डवस्तु भृशं क्रुद्धो विद्धस्तेन महात्मना}
{रणे वराहकर्णेन राजानं हृद्यविध्यता}


\twolineshloka
{अथापरेण भल्लेन केतुं तस्य महात्मनः}
{रथश्रेष्ठो रथात्तूर्णं भूमौ पार्थो न्यपातयत्}


\twolineshloka
{केतु निपतितं दृष्ट्वा श्रुतायुः स तु पार्थिवः}
{पाण्डवं विशिखैतीक्ष्णै राजन्विव्याध सप्तभिः}


\twolineshloka
{ततः क्रोघात्प्रजज्वल धर्मपुत्रो युधिष्ठिरः}
{यथा युगान्ते भूतानि दिधक्षुरिव पावकः}


\twolineshloka
{क्रुद्धं तु पाण्डवं दृष्ट्वा देवगन्धर्वराक्षसाः}
{प्रविव्यधुर्महाराज व्याकुलं जाप्यभूज्जगत्}


\twolineshloka
{सर्वेषां चैव भूतानामिदमासीन्मनोगतम्}
{त्रींल्लोकान्म संक्रुद्धो नृपोऽयं धक्ष्यतीति वै}


\twolineshloka
{ऋष..... देवाश्च चक्रुः स्वस्त्ययनं महत्}
{लोकानां नृपाशान्त्यर्थं क्रोधिते पाण्डवे तदा}


\twolineshloka
{स च क्रोधसमाविष्टः सृक्विणी परिसंलिहन}
{इवारात्सवपुर्घोरं युगान्तादित्यसन्निभम्}


\twolineshloka
{ततः सैन्यानि सर्वाणि तावकानि विशांपते}
{निराशान्यभवंस्तत्र जीवितं प्रति भारत}


\twolineshloka
{स तु धैर्येण तं कोपं सन्निवार्य महायशाः}
{श्रुतायुषः प्रचिच्छेद मुष्टिदेशे महाधनुः}


\twolineshloka
{अथैनं छिन्नधन्वानं नाराचेन स्तनान्तरे}
{निर्बिभेद रणे राजा सर्वसैन्यस्य पश्यतः}


\twolineshloka
{सत्वरं च रणे राजंस्तस्य वाहान्महात्मनः}
{निजघान शरैः क्षिप्रं सूतं च सुमहाबलः}


\twolineshloka
{हताश्वं तु रथं त्यक्त्वा दृष्ट्वा राज्ञोऽस्य पौरुषम्}
{विप्रदुद्राव वेगेन श्रुतायुः समरे तदा}


\twolineshloka
{तस्मिञ्जिते महेष्वासे धर्मपुत्रेण संयुगे}
{दुर्योधनबलं राजन्सर्वमासीत्पराङ्भुखम्}


\twolineshloka
{एवं जित्वा महाराज धर्मपुत्रो युधिष्ठिरः}
{व्यात्ताननो यथा कालस्तव सैन्यं जघान ह}


\twolineshloka
{चेकितानस्तु वार्ष्णेयो गौतमं रथिनां वरम्}
{प्रेक्षतां सर्वसैन्यानां छादयामास सायकैः}


\twolineshloka
{संनिवार्य शरांस्तांस्तु कृपः शारद्वतो युधि}
{चेकितानां रणे यत्तं राजन्विव्याध पत्रिभिः}


\twolineshloka
{अथापरेण भल्लेन धनुश्चिच्छेद मारिष}
{सारथिं चास्य समरे क्षिप्रहस्तो न्यपातयत्}


\twolineshloka
{अश्वांस्छास्यावधीद्राजन्नुभौ तौ पार्ष्णिसारथी}
{अवप्लुत्य रथात्तूर्णं गदां जग्राह सात्वतः}


\twolineshloka
{स तया वीरघातिन्या गदया गदिनां वरः}
{गौतमस्य हयान्हत्वा सारथिं च न्यपातयत्}


\twolineshloka
{भूमिष्ठो गौतमस्तस्य शरांश्चिक्षेप षोडशक}
{शरास्ते सात्वतं भित्त्वा प्राविशन्धरणीतलम्}


\twolineshloka
{चेकितानस्ततः क्रुद्धः पुनश्चिक्षेप तां गदाम्}
{गौतमस्य वधाकाङ्क्षी वृत्रस्येव पुरन्दरः}


\twolineshloka
{तामापतन्तीं विमलामश्यमगर्भां महागदाम्}
{शरैरनेकसाहस्त्रैर्वारयामास गौतमः}


\twolineshloka
{चेकितानस्ततः खङ्गं क्रोधादुद्धृत्य भारत}
{लाघवं परमास्थाय गौतमं समुपाद्रवत्}


\twolineshloka
{गौतमोऽपि धनुस्त्यक्त्वा प्रगृह्यासिं सुसंयतः}
{वेगेन महता राजंश्चेकितानमुपाद्रवत्}


\twolineshloka
{तावुभौ बलसंपन्नौ निस्त्रिंशवरधारिणौ}
{निस्त्रिंशाभ्यां सुतीक्ष्णाभ्यामन्योन्यं संततक्षतुः}


\twolineshloka
{निस्त्रिंशवेगाभिहतौ ततस्तौ पुरुषर्षभौ}
{धरणीं समनुप्राप्तौ सर्वभूतनिषेविताम्}


\twolineshloka
{मूर्छयाऽभिपरीताङ्गौ व्यायामेन तु मोहितौ}
{ततोऽभ्यधावद्वेगेन भीमसेनः सुहृत्तया}


\twolineshloka
{चेकितानं तथाभूतं दृष्ट्वा समरदुर्मदः}
{रथमारोपयच्चैनं सर्वसैकन्यस्य पश्यतः}


\twolineshloka
{तथैव शकुनि शूरः स्यालस्तव विशांपतेक}
{आरोपयद्रथं तूर्णं गौतमं रथिनां वरम्}


\twolineshloka
{सौमदत्तिं ततः क्रुद्धो धृष्टकेतुर्महाबलः}
{नवत्या सायकैः क्षिप्रं राजन्विव्याध वक्षसि}


\twolineshloka
{सौमदत्तिरुरस्थैस्तैर्भृशं बाणैरशोभत}
{मध्यंदिने महाराज रश्मिभिस्तपनो यथा}


\twolineshloka
{भूरिश्रवास्तु समरे धृष्टकेतुं महारथम्}
{हतसूतहयं चक्रे विरथं सायकोत्तमैः}


\twolineshloka
{विरथं तं समालोक्य हताश्वं हतसारथिम्}
{महता शरवर्षेण च्छादयामास संयुगे}


\twolineshloka
{स तु तं रथमुत्सृज्य धृष्टकेतुर्महामनाः}
{आरुरोह ततो यानं शतानीकस्य मारिष}


\twolineshloka
{चित्रसेनो विकर्णश्च राजन्दुर्मर्षणस्तथा}
{रथिनो हेमसन्नाहाः सौभद्रमभिदुद्रुवुः}


\twolineshloka
{अभिमन्योस्ततस्तैस्तु घोरं युद्धमवर्तत}
{शरीरस्य यथा राजन्वातपित्तकफैस्त्रिभिः}


\twolineshloka
{विरथांस्तव पुत्रांस्तु कृत्वा राजन्महाहवे}
{न जघान नरव्याघ्रः स्मरन्भीमवचस्तदा}


\twolineshloka
{ततो राज्ञां बहुशतैर्गजाश्वरथयायिभिः}
{संवृतं समरे भीष्मं देवैरपि दुरासदम्}


\twolineshloka
{प्रयान्तं शीघ्रमुद्वीक्ष्य परित्रातुं सुतांस्तव}
{अभिमन्युं समुद्दिश्य बालमेकं महारथम्}


\twolineshloka
{वासुदेवमुवाचेदं कौन्तेयः श्वेतवाहनः}
{चोदयाश्वान्हृषीकेश यत्रैते बहुला रथाः}


\twolineshloka
{एते हि बहवः शूराः कृतास्त्रा युद्धदुर्मदाः}
{यथा हन्युर्न नः सेनां तथा माधव चोदय}


\twolineshloka
{एवमुक्तः स वार्ष्णेयः कौन्तेयेनामितौजसा}
{रथं श्वेतहयैर्युक्तं प्रेषयामास संयुगे}


\twolineshloka
{निष्ठानको महानासीत्तव सैन्यस्य मारिष}
{यदर्जुनो रणे क्रुद्धः संयातस्तावकान्प्रति}


\twolineshloka
{समासाद्य तु कौन्तेयो राज्ञस्तान्भीष्मरक्षिणः}
{सुशर्माणमथो राजन्निदं वचनमब्रवीत्}


\twolineshloka
{जानामि त्वां युधां श्रेष्ठमत्यन्तं पूर्ववैरिणम्}
{अनयस्याद्य संप्राप्तं फलं पश्य सुदारुणम्}


\twolineshloka
{अद्य ते दर्शयिष्यामि पूर्वप्रेतान्पितामहान्}
{एवं संजल्पतस्तस्य बीभत्सोः शत्रुघातिनः}


\twolineshloka
{श्रुत्वापि परुषं वाक्यं सुशर्मा रथयूथपः}
{न चैनमब्रवीत्किंचिच्छुभं वा यदि वाऽशुभम्}


\twolineshloka
{अभिगम्यार्जुनं वीरं राजभिर्बहुभिर्वृतः}
{पुरस्तात्पृष्ठतश्चैव पार्श्वतश्चैव सर्वतः}


\twolineshloka
{परिवार्यार्जुनं सङ्ख्ये तव पुत्रैर्महारथः}
{शरैः संछादयामास मेघैरिव दिवाकरम्}


\twolineshloka
{ततः प्रवृत्तः सुमहान्संग्रामः शोणितोदकः}
{तावकानां च समरे पाण्डवानां च भारत}


\chapter{अध्यायः ८५}
\twolineshloka
{सञ्जय उवाच}
{}


\twolineshloka
{स ताड्यमानस्तु शरैर्धनंजयःपदाहतो नाग इव श्वसन्बली}
{बाणांश्च बाणेन महारथानांचिच्छेद चापानि रणे प्रसह्य}


\twolineshloka
{संछिद्य चापानि च तानि राज्ञांतेषां रणे वीर्यवतां क्षणेन}
{विव्याध बाणैर्युगपन्महात्मानिःशेषतां तेष्वथ मन्यमानः}


\twolineshloka
{निपेतुराजौ रुधिरप्रदिग्धा-स्ते ताडिताः शक्रसुतेन राजन्}
{विभिन्नगात्राः पतितोत्तमाङ्गागतासवश्छिन्नतनुक्रकायाः}


\twolineshloka
{महीं गताः पर्थबलाभिभूताविचित्ररूपा युगपद्विनेशुः}
{दृष्ट्वा हतांस्तान्युधि राजपुत्रां-स्त्रिगर्तराजः प्रययौ रथेन}


\twolineshloka
{तेषां रथानामथ पृष्ठगोपाद्वात्रिंशदन्येऽभ्यपतन्त पार्थम्}
{तथैव ते तं परिवार्य पार्थंविकृष्य चापानि महारवाणि}


\twolineshloka
{अवीवृषन्बाणमहौघवृष्ट्यायथा गिरिं तोयधरा जलौघैः}
{संपीड्यमानस्तु शरौघवृष्ट्याधनंजयस्तान्युधि जातरोषः}


\twolineshloka
{षष्ट्या शरैः संयति तैलधौतै-र्जघान तानप्यथ पृष्ठगोपान्}
{रथांश्च तांस्तनवजित्य सङ्ख्येधनञ्जयः प्रीतमना यशस्वी}


\twolineshloka
{अथात्वरद्भीष्मवधाय जिष्णु-र्बलानि राजन्समरे निहत्य}
{त्रिगर्तराजो निहतान्समीक्ष्यमहात्मना तानथ बन्धुवर्गान्}


% Check verse!
रणे पुरस्कृत्य नराधिपांस्तान्जगाम पार्थं त्वरितो वधायअभिद्रुतं चास्त्रभृतां वरिष्ठंधनंजयं वीक्ष्य शिखण्डिमुख्याः
\twolineshloka
{अभ्यद्ययुस्ते शितशस्त्रहस्तारिरक्षिषन्तो रथमर्जुनस्य}
{पार्थोऽपि तानापततः समीक्ष्यत्रिगर्तराज्ञा सहितान्नृवीरान}


\twolineshloka
{विध्वंसयित्वा समरे धनुष्मान्गाण्डीवमुक्तैर्निशितैः पृषत्कैः}
{भीष्मं यियासुर्युधि संददर्शदुर्योधनं सैन्धवादींश्च राज्ञः}


\threelineshloka
{संवारयिष्णूनभिवारयित्वामुहूर्तमायोध्य बलेन वीरः}
{उत्सृज्य राजानमनन्तवीर्योजयद्रथादींश्च नृपान्महौजाः}
{ययौ ततो भीमबलो मनस्वीगाङ्गेयमाजौ शरचापपाणिः}


\twolineshloka
{भीष्मोऽपि दृष्ट्वा समरे कृतास्त्रान्स पाण्डवानां रथिनोऽभ्युदारान्}
{विहाय संग्राममुखे धनंजयंजवेन पार्थं पुनराजगाम}


\threelineshloka
{युधिष्ठिरश्च प्रबलो महात्मासमाययौ त्वरितो जातकोपः}
{मद्राधिपं समभित्यज्य सङ्ख्येस्वभागमाप्तं तमनन्तकीर्तिः}
{सार्धं स माद्रीसुतभीमसेनै-र्भीष्मं ययौ शान्तनवं रणाय}


\twolineshloka
{तैः संप्रयुक्तैः स महारथाग्र्यै-र्गङ्गासुतः समरे चित्रयोधी}
{न विव्यधे शान्तनवो महात्मासमागतैः पाण्डुसुतैः समस्तैः}


\twolineshloka
{अथैत्य राजा युधि सत्यसन्धोजयद्रथोऽत्युग्रबलो मनस्वी}
{चिच्छेद चापानि महारथानांप्रसह्य तेषां धनुषा वरेण}


\twolineshloka
{युधिष्ठिरं भीमसेनं यमौ चपार्थं कृष्णं युधि संजातकोपः}
{दुर्योदनः क्रोधविषो महात्माजघान बाणैरनलप्रकाशैः}


\twolineshloka
{कृपेण शल्येन शलेन चैवतथा विभो चित्रसेनेन चौजौ}
{विद्धाः शरैस्तेऽतिविवृद्धकोपै-र्देवा यथा दैत्यगणैः समेतैः}


\twolineshloka
{छिन्नायुधं शान्तनवेन राजाशिखण्डिनं प्रेक्ष्य च जातकोपः}
{अजातशत्रुः समरे महात्माशिखण्डिनं क्रुद्ध उवाच वाक्यम्}


\twolineshloka
{उक्त्वा तथा त्वं पितुरग्रतो मा-महं हनिष्यामि महाव्रतं तम्}
{भीष्मं शरौघैर्विमलार्कवर्णैःसत्यं वदामीति कृता प्रतिज्ञा}


\twolineshloka
{त्वया च नैनां सफलां करोषिदेवव्रतं यन्न निहंसि युद्धे}
{मिथ्याप्रतिज्ञो भव मात्र वीररक्षस्व धर्मं स्वकुलं यशश्च}


\twolineshloka
{प्रेक्षस्व भीष्मं युधि भीमवेगंसर्वांस्तपन्तं मम सैन्यसङ्घान्}
{शरौघजालैरतितिग्मवेगैःकालं यथा कालकृतं क्षणेन}


\twolineshloka
{निकृत्तचापः समरेऽनपेक्षःपराजितः शान्तनवेन चाजौ}
{विहाय बन्धूनथ सदरांश्चक्व यास्यसे नानुरूपं तवेदम्}


\twolineshloka
{दृष्ट्वा हि भीष्मं तमनन्तवीर्यंभग्नं च सैन्यं द्रवमाणमेवम्}
{भीतोऽसि नूनं द्रुपदस्य पुत्रतथा हि ते मुखवर्णोऽप्रहृष्टः}


\twolineshloka
{अज्ञायमाने च धनंजये तुमहाहावे संप्रसक्ते नृवीरे}
{कथं हि भीष्मात्प्रथितः पृथिव्यांभयं त्वमद्य प्रकरोषि वीर}


\twolineshloka
{स धर्मराजस्य वचो निशम्य 6-85-26bरूक्षाक्षरं विप्रलापानुबद्धम्}
{प्रत्यादेशं मन्यमानो महात्माप्रतत्वरे भीष्मवधाय राजन्}


\twolineshloka
{तमापतन्तं महता जवेनशिखण्डिनं भीष्ममभिद्रवन्तम्}
{निवारयामास हि शल्य एन-मस्त्रेण घोरेण सुदुर्जयेन}


\threelineshloka
{स चापि दृष्ट्वा समुदीर्यमाण-मस्त्रं युगान्ताग्निसमप्रकाशम्}
{न संमुमोह द्रुपदस्य पुत्रोराजन्महेन्द्रप्रतिमप्रभावः}
{तस्थौ च तत्रैव महाधनुष्मान् शरैस्तदस्त्रं प्रतिबाधमानः}


\twolineshloka
{अथाददे वारुणमन्यदस्त्रंशिखण्ड्यथोऽग्रं प्रतिघातमस्य}
{तदस्त्रमस्त्रेण विदार्यमाणंखस्थाः सुरा ददृशुः पार्थिवाश्च}


\twolineshloka
{भीष्मस्तु राजन्समरे महात्माधनुश्च चित्रं ध्वजमेव चापि}
{छित्त्वाऽनदत्पाण्डुसुतस्य वीरोयुधिष्ठिरस्याजमीढस्य राज्ञः}


\twolineshloka
{ततः समुत्सृज्य धनुः सबाणंयुधिष्ठिरं वीक्ष्य भयाभिभूतम्}
{गदां प्रगृह्याभिपपात सङ्ख्येजयद्रथं भीमसेनः पदातिः}


\twolineshloka
{तमापतन्तं सहसा जवेनजयद्रथः सगदं भीमसेनम्}
{विव्याध घोरैर्यमदण्डकल्पैःशितैः शरैः पञ्चशरैः समन्तात्}


\twolineshloka
{अचिन्तयित्वा स शरांस्तरस्वीवृकोदरः क्रोधपरीतचेताः}
{जघान वाहान्समरे समन्तात्सुसंमतान्सिन्धुराजस्य सङ्ख्ये}


% Check verse!
ततोऽभिवीक्ष्याप्रतिमप्रभाव-स्तवात्मजस्त्वरमाणो रथेनअभ्यायौ भीमसेनं निहन्तुंसमुद्यतास्त्रः सुरराजकल्पः
\twolineshloka
{जयद्रथो भग्रवाहं रथं तत्यक्त्वा ययौ यत्र राजा कुरूणाम्}
{भयेन भीमस्य समूढचेताःससौबलस्तत्र युद्धस्य भीतः}


\twolineshloka
{भीमोऽप्यथैनं सहसा विनद्यप्रत्युद्ययौ गदया तर्जयानः}
{समुद्यतां तां यमदण्डकल्पांदृष्ट्वा गदां ते कुरवः समन्तात्}


\twolineshloka
{विहाय सर्वे तव पुत्रमुग्रंपातं गदायाः परिहर्तुकामाः}
{अपक्रान्तास्तुमुले संप्रमर्देसुदूरुणे भारत मोहनीये}


\threelineshloka
{अमूढचेतास्त्वथ चित्रसेनोमहागदामापतन्तीं निरीक्ष्य}
{रथं समुत्सृज्य पदातिराजौप्रगृह्य खङ्गं विपुलं च चर्म}
{अवप्लुतः सिंह उवाचलाग्रा-ज्जगामान्यं भ्रमप भूमिदेशम्}


\twolineshloka
{गदापि सा प्राप्य रथं सुचित्रंसाश्व ससूतं विनिहत्य सङ्ख्ये}
{जगाम भूमिं ज्वलिता महोल्कभ्रष्टाऽम्बराद्गामिव संपतन्ती}


\twolineshloka
{आश्चर्यभूतं सुमहत्त्वदीयादृष्ट्वैव तद्भारत संप्रहृष्टाः}
{सर्वे विनेदुः सहिताः समन्ता-त्पुपूजिरे तव पुत्रस्य शौर्यम्}


\chapter{अध्यायः ८६}
\twolineshloka
{सञ्जय उवाच}
{}


\twolineshloka
{विरथं तं समासाद्य चित्रसेनं यशस्विनम्}
{रथमारोपयामास विकर्णस्तनयस्तव}


\twolineshloka
{तस्मिंस्तथा वर्तमाने तुमुले संकुले भृशम्}
{भीष्मः शान्तनवस्तूर्णं युधिष्ठिरमुपाद्रवत्}


\twolineshloka
{ततः सरथनागाश्वाः समकम्पन्त सृञ्जयाः}
{मृत्योरास्यमनुप्राप्तं मेनिरे च युधिष्ठिरम्}


\twolineshloka
{युधिष्ठिरोऽपि कौरव्यो यमाभ्यां सहितः प्रभुः}
{महेष्वासं नरव्याघ्रं भीष्मं शान्तनवं ययौ}


\twolineshloka
{ततः शरसहस्राणि प्रमुञ्चन्पाण्डवो युधि}
{भीष्मं संछादयामास यथा मेघो दिवाकरम्}


\twolineshloka
{तेन सम्यक्प्रणीतानि शरजालानि मारिष}
{प्रतिजग्राह गाङ्गयः शतशोऽथ सहस्रशः}


\twolineshloka
{तथैव शरजालानि भीष्मोणास्तानि मारिष}
{आकाशे समदृश्यन्त स्वगमानां व्रजा इव}


\twolineshloka
{निमेषार्धेन कौन्तेयं भीष्मः शान्तनवो युधि}
{अदृश्यं समरे चक्रे शरजालेन भारत}


\twolineshloka
{ततो युधिष्ठिरे राजा कौरव्यस्य महात्मनः}
{नाराचं प्रेषयामास क्रुद्ध आशीविषोपमम्}


\twolineshloka
{असंप्राप्तं ततस्तं तु क्षुरप्रेण महारथः}
{चच्छेद समरे राजन्भीष्मस्तस्य धनुश्च्युतम्}


\twolineshloka
{तं तु च्छित्वा रणे भीष्मो नाराचं कलसंमितम्}
{नजघ्ने कौरवेन्द्रस्य हयान्काञ्चनभूषणान्}


\twolineshloka
{हताश्वं तु रथं त्यक्त्वा धर्मपुत्रो युधिष्ठिरः}
{आरुरोह रथं तूर्णं नकुलस्य महात्मनःक}


\twolineshloka
{यमावपि हि संक्रुद्धः समासाद्य रणे तदा}
{शरैः संछादयामास भीष्मः परपुरंजयः}


\twolineshloka
{तौ तु दृष्ट्वा महाराज भीष्मबाणप्रपीडितौ}
{जगाम परमां चिन्तां भीष्मस्य वधकाङ्क्षया}


\twolineshloka
{ततो युधिष्ठिरो वश्यान्राज्ञस्तान्समचोदयत्}
{भीष्मं शान्तनवं सर्वे निहतेति सुहृद्गणान्}


\twolineshloka
{ततस्ते पार्थिवाः सर्वे श्रुत्वा पार्थस्य भाषितम्}
{महता रथवंशेन परिवव्रुः पितामहम्}


\twolineshloka
{स समन्तात्परिवृतः पिता देवव्रतस्तव}
{चिक्रीड धनुषा राजन्पातयानो महारथान्}


\twolineshloka
{तं चरन्तं रणे पार्था ददृशुः कौरवं युधि}
{मृगमध्यं प्रविश्येव यथा सिंहशिशुं वने}


\twolineshloka
{तर्जयानं रमे वीरांस्त्रासयानं च सायकैः}
{दृष्ट्वा त्रेसुर्महाराज सिंहं मृगगणा इव}


\twolineshloka
{रणे भारत सिंहस्य ददृशुः क्षत्रिया गतिम्}
{अग्नेर्वायुसहायस्य यथा कक्षं दिधक्षतः}


\twolineshloka
{शिरांसि रथिनां भीष्मः पातयामास संयुगे}
{तालेभ्यः परिपक्वानि फलानि कुशलो नरः}


\twolineshloka
{पतद्भिश्च महाराज शिरोभिर्धरणीतले}
{बभूव तुमुलः शब्दः पततामश्मनामिव}


\twolineshloka
{तस्मिन्सुतुमुले युद्धे वर्तमाने भयानके}
{सर्वेषामेव सैन्यानामासीद्व्यतिकरो महान्}


\twolineshloka
{भिन्नेषु तेषु व्यूहेषु क्षत्रिया इतरेतरम्}
{एकमेकं समाहूय युद्धायैवावतस्थिरे}


\twolineshloka
{शिखण्डी तु समासाद्य भरतानां पितामहम्}
{अभिदुद्राव वेगेन तिष्ठतिष्ठेति चाब्रवीत्}


\twolineshloka
{अनादृत्य ततो भीष्मस्तं शिखण्डिनमाहवे}
{प्रययौ सृञ्जयान्क्रुद्धः स्त्रीत्वं तस्य विचिन्तयन्}


\twolineshloka
{सृंजयास्तु ततो दृष्ट्वा हृष्टं भीष्म महारणे}
{सिंहनादांश्च विविधांश्चक्रुः शङ्खविमिश्रितान्}


\twolineshloka
{ततः प्रववृते युद्धं व्यतिषक्तरथद्विपम्}
{पश्चिमां दिशमासाद्य स्थिते सवितरि प्रभो}


\twolineshloka
{धृष्टद्युम्नोऽथ पाञ्चाल्यः सात्यकिश्च महारथः}
{पीडयन्तौ भृशं सैन्यं शक्तितोमरवृष्टिभिः}


\twolineshloka
{शस्त्रैश्च बहुभी राजञ्जघ्नतुस्तावकान्रणे}
{ते हन्यमानाः समरे तावका भरतर्षभ}


\twolineshloka
{आर्यां युद्धे मतिं कृत्वा न त्यजन्ति स्म संयुगम्}
{यथोत्साहं तु समरे निजघ्नुस्तावका रणे}


\twolineshloka
{तत्राक्रन्दो महानासीत्तावकानां महात्मानाम्}
{वध्यतां समरे राजन्पार्षतेन महात्मना}


\twolineshloka
{तं श्रुत्वा निनदं घोरं तावकानां महारथौ}
{विन्दानुविन्दावावन्त्यौ पार्षतं प्रत्युपस्थितौ}


\twolineshloka
{तौ तस्य तुरगान्हत्वा त्वरमाणौ महारथौ}
{छादयामासतुरुभौ शरवर्षेण पार्षतम्}


\twolineshloka
{अवप्लुत्याथ पाञ्चाल्यो रथात्तूर्णं महाबलः}
{आरुरोहक रथं तूर्णं सात्यकेस्तु महात्मनः}


\twolineshloka
{ततो युधिष्ठिरो राजा महत्या सेनया वृतः}
{आवन्त्यौ समरे क्रुद्धावभ्ययात्स परंतपौ}


\twolineshloka
{तथैव तव पुत्रोऽपि सर्वोद्योगेन मारिष}
{विन्दानुविन्दौ समरे परिवार्यावतस्थिवान्}


\twolineshloka
{अर्जुनश्चापि संक्रुद्धः क्षत्रियान्क्षत्रियर्षभः}
{अयोधयत संग्रामे वज्रपाणिरिवासुरान्}


\twolineshloka
{द्रोणास्तु समरे क्रुद्धः पुत्रस्य प्रियकृत्तव}
{व्यधमत्सर्वपाञ्चालांस्तूलराशिमिवानलः}


\twolineshloka
{दुर्योधनपुरोगास्तु पुत्रास्तव विशांपते}
{परिवार्य रणे भीष्मं युयुधुः पाण्डवैः सह}


\twolineshloka
{ततो दुर्योधनो राजा लोहितायति भास्करे}
{अब्रवीत्तावकान्सर्वांस्त्वरध्वमिति भारत}


\twolineshloka
{युध्यतां तु तथा तेषां कुर्वतां कर्म दुष्करम्}
{अस्तं गिरिमथारूढे अप्रकाशति भास्करे}


\twolineshloka
{प्रावर्तत नदी घोरा शोणितौघतरङ्गिणी}
{गोमायुगणसंकीर्णा क्षणेन क्षणदामुखे}


\twolineshloka
{शिवाभिरशिवाभिश्च रुवद्भिर्भैरवं रवम्}
{घोरमायोधनं जज्ञे भूतसङ्घैः समाकुलम्}


\twolineshloka
{राक्षसाश्च पिशाचाश्च तथान्ये पिशिताशिनः}
{समन्ततो व्यदृश्यन्त शतशोऽथ सहस्रशः}


\twolineshloka
{अर्जुनोऽथ सुशर्मादीन्राज्ञस्तान्सपदानुगान्}
{विजित्य पृतनामध्ये ययौ स्वशिबिरं प्रति}


\twolineshloka
{युधिष्टिरोऽपि कौरव्यो भ्रातृभ्यां सहितस्तथा}
{ययौ स्वशिबिरं राजा निशायां सेनया वृतः}


\twolineshloka
{भीमसेनोऽपि राजेन्द्र दुर्योधनमुखान्रथान्}
{अवजित्य ततः सङ्ख्ये ययौ स्वशिबिरं प्रति}


\twolineshloka
{दुर्योधनोऽपि नृपतिः परिवार्य महारणे}
{भीष्मं शान्तनवं तूर्णं प्रयातः शिबिरं प्रति}


\twolineshloka
{द्रोणो द्रौणिः कृपः शल्यः कृतवर्मा च सात्वतः}
{परिवार्य चमूं सर्वाः प्रययुः शिबिरं प्रति}


\twolineshloka
{तथैव सात्यकी राजन्धृष्टद्युम्नश्च पार्षतः}
{परिवार्य रणे योधान्ययतुः शिबिरं प्रति}


\twolineshloka
{एवमेते महाराज तावकाः पाण्डवैः सह}
{पर्यवर्तन्त सहिता निशाकाले परंतप}


\twolineshloka
{ततः स्वशिबिरं गत्वा पाण्डवाः कुरवस्तथा}
{न्यवसन्त महाराज पूजयन्तः परस्परम्}


\twolineshloka
{रक्षां कृत्वा ततः शूरा न्यस्य गुल्मान्यथाविधि}
{अपनीय च शल्यानि स्नात्वा च विविधैर्जलैः}


\twolineshloka
{कृतस्वस्त्ययनाः सर्वे स्तूयमानाश्च बन्दिभिः}
{गीतवादित्रशब्देन व्यक्रीडन्त यशस्विनः}


\twolineshloka
{मुहूर्तादिव तत्सर्वमभवत्स्वर्गसन्निभम्}
{न हि युद्धकथां कांचित्तत्राकुर्वन्महारथाः}


\twolineshloka
{ते प्रसुप्ते बले तत्र परिश्रान्तजने नृप}
{हस्त्यश्वबहुले रात्रौ प्रेक्षणीये बभूवतुः}


\chapter{अध्यायः ८७}
\twolineshloka
{सञ्जय उवाच}
{}


\twolineshloka
{परिणाम्य निशां तां तु सुखसुप्ता जनेश्वराः}
{कुरवः पाण्डवाश्चैव पुनर्युद्धाय निर्ययुः}


\twolineshloka
{ततः शब्दो महानासीत्सेनयोरुभयोर्नृप}
{निर्गच्छमानयोः सङ्ख्ये यथा सागरयोरिव}


\twolineshloka
{ततो दुर्योधनो राजा चित्रसेनो विविंशतिः}
{भीष्मश्च रथिनां श्रेष्ठो भारद्वाजश्च वै द्विजः}


\twolineshloka
{एकीभूताः सुसंयत्ताः कौरवाणां महाचमूम्}
{व्यूहाय विदधू राजन्पाण्डवान्प्रतिदंशितान्}


\twolineshloka
{कूर्मव्यूहं ततः कृत्वा पिता तव विशांपते}
{सागरप्रतिमं घोरं वाहनोर्मितरङ्गिणम्}


\twolineshloka
{अग्रतः सर्वसैन्यानां भीष्मः शान्तनवो ययौ}
{मालवैर्दाक्षिणात्यैश्च आवन्त्यैश्च समन्वितः}


\twolineshloka
{ततोऽनन्तरमेवासीद्भारद्वाजः प्रतापवान}
{पुलिन्दैः पारदैश्चैव तथा क्षुद्रकमालवैः}


\twolineshloka
{द्रोणादनन्तरं यत्तो भगदत्तः प्रतापवान्}
{मगधैश्च कलिङ्गैश्च पिशाचैश्च विशांपते}


\twolineshloka
{प्राग्ज्योतिषादनु नृपः कौसल्योऽथ बृहद्बलः}
{मेकलैः कुरुविन्दैश्च त्रैपुरैश्च समन्विताः}


\twolineshloka
{बृहद्बलादनु नृपस्त्रिगर्तः प्रस्थलाधिपः}
{काम्भोजैर्बहुभिः सार्धं यवनैश्च सहस्रशः}


\twolineshloka
{द्रौणिस्तु रभसः शूरस्त्रैगर्तादनु भारत}
{प्रययौ सिंहनादेन नादयानो धरातलम्}


\twolineshloka
{तथा सर्वेण सैन्येन राजा दुर्योधनस्तदा}
{द्रौणेरनन्तरं प्रायात्सोदर्यैः परिवारितः}


\twolineshloka
{दुर्योधनादनु ततः कृपः शारद्वतो ययौ}
{एवमेष महाव्यूहः प्रययौ सागरोपमः}


\twolineshloka
{रेजुस्तत्र पताकाश्च श्वेतच्छत्राणि भारत}
{अङ्गदान्यत्र चित्राणि महार्हाणि धनूंषि च}


\twolineshloka
{तं तु दृष्ट्वा महाव्यूहं तावकानां महारथः}
{युधिष्ठिरोऽब्रवीत्तूर्णं पार्षतं पृतनापतिम्}


\twolineshloka
{पश्य व्यूहं महेष्वास निर्मितं सागरोपमम्}
{प्रतिव्यूहं रणे शूर कुरु क्षिप्रं महारथ}


\twolineshloka
{ततः स पार्षतः क्रूरो व्यूहं चक्रे सुदारुणम्}
{श्रृङ्गाटकं महाराज परव्यूहविनाशनम्}


\twolineshloka
{श्रृङ्गाभ्यां भीमसेनश्च सात्यकिश्च महारथः}
{रथैरनेकसाहस्रैस्तथा हयपदातिभिः}


\twolineshloka
{नाभावभून्नरश्रेष्ठः श्वेताश्वः कृष्णसारथिः}
{मध्ये युधिष्ठिरो राजा माद्रीपुत्रौ च पाण्डवौ}


\twolineshloka
{अथेतरे महेष्वासाः सहसैन्या नराधिपाः}
{व्यूहं तं पूरयामासुर्व्यूहशास्त्रविशारदाः}


\twolineshloka
{एवमेतं महाव्यूहं व्यूह्य भारत पाण्डवाः}
{अतिष्ठन्समरे शूरा योद्धुकामा जयैषिणः}


\twolineshloka
{भेरीशब्दैश्च विमलैर्विमिश्रैः शङ्खनिःस्वनैः}
{क्ष्वेडितास्फोटितोत्क्रुष्टैर्नादिताः सर्वतो दिशः}


\threelineshloka
{ततः शूराः समासाद्य समरे ते परस्परम}
{नेत्रैरनिमिषै राजन्नवैक्षन्त परस्परम् ॥ 6-87-24a`मनोभिस्तेमनुष्येन्द्र युद्धं योधाः प्रचक्रिरे}
{'पुनराहूय तेऽन्योन्यं शरीरैरपि चक्रिरे}


\twolineshloka
{ततः प्रववृते युद्धं घोररूपं भयावहम्}
{तावकानां परेषां च निघ्नतामितरेतरम्}


\twolineshloka
{नाराचा निशिताः सङ्ख्ये संपतन्ति स्म भारत}
{व्यात्तानना भयकरा उरगा इव सङ्घशः}


\twolineshloka
{निष्पेतुर्विमलाः शक्त्यस्तैलधौताः सुतेजनाः}
{अम्बुदेभ्यो यथा राजन्भ्राजमानाः शतह्रदाः}


\twolineshloka
{गदाश्च विमलैः पट्टैः पिनद्धाः स्वर्णभूषिताः}
{पतन्त्यस्तत्र दृस्यन्ते गिरिशृङ्गोपमाः शुभाः}


\twolineshloka
{निस्त्रिंशाश्च व्यदृश्यन्त विमलाम्बरसन्निभाः}
{आर्षभाणि च चर्माणि शतचन्द्राणि भारतत}


\twolineshloka
{अशोभन्त रणे राजन्पात्यमानानि सर्वशः}
{तेऽन्योन्यं समरे सेने युध्यमाने नराधिप}


\twolineshloka
{अशोभेतां यथा देवदैत्यसेने समुद्यते}
{अभ्यद्रवन्त समरे तेऽन्योन्यं वै समन्ततः}


\twolineshloka
{रथास्तु रथिभिस्तूर्णं प्रेषिताः परमाहवे}
{युगैर्युगानि संश्लिष्य युयुधुः पार्थिवर्षभाः}


\twolineshloka
{दन्तिनां युध्यमानानां संघर्षात्पावकोऽभवत्}
{दन्तेषु भरतश्रेष्ठ सधूमः सर्वतो दिशम्}


\twolineshloka
{प्रासैरभिहताः केचिद्गजयोधाः समन्ततः}
{पतमानाः स्म दृश्यन्ते गिरिशृङ्गान्नगा इव}


\twolineshloka
{पादाताश्चाप्यदृश्यन्त निघ्नन्तोऽथ परस्परम्}
{चित्ररूपधराः शूरा नखरप्रासयोधिनः}


\twolineshloka
{अन्योन्यं ते समासाद्य कुरुपाण्डवसैनिकाः}
{अस्त्रैर्नानाविधैर्घोरै रणे निन्युर्यमक्षयम्}


\twolineshloka
{ततः शान्तनवो भीष्मो रथघोषेण नादयन्}
{अभ्यागमद्रणे पार्थान्धनुःशब्देन मोहयन्}


\twolineshloka
{पाण्डवानां रथाश्चापि नदन्तो भैरवं स्वनम्}
{अभ्यद्रवन्त संयत्ता धृष्टद्युम्नपुरोगमाः}


\twolineshloka
{ततः प्रववृते युद्धं तव तेषां च भारत}
{नराश्वरथनागानां व्यतिषक्तं परस्परम्}


\chapter{अध्यायः ८८}
\twolineshloka
{सञ्जय उवाच}
{}


\twolineshloka
{भीष्मं तु समरे क्रुद्धं प्रतपन्तं समन्ततः}
{न शेकुः पाण्डवा द्रष्टुं तपन्तमिव भास्करम्}


\twolineshloka
{ततः सर्वाणि सैन्यानि धर्मपुत्रस्य शासनात्}
{अभ्यद्रवन्त गाङ्गेयं मर्दयन्तं शितैः शरैः}


\twolineshloka
{स तु भीष्मो रणश्लाघी तोमकान्सहसृज्जयान्}
{पाञ्चालांश्च महेष्वासान्पातयामास सायकैः}


\twolineshloka
{ते वध्यमाना भीष्मेण पाञ्चालाः पाण्डवैः सह}
{भीष्ममेवाभ्ययुस्तूर्णं त्यक्त्वा मृत्युकृतं भयम्}


\twolineshloka
{स तेषां रथिनां वीरो भीष्मः शान्तनवो युधि}
{चिच्छेद सहसा राजन्बाहूनथ शिरांसि च}


\twolineshloka
{विरथान्रथिनश्चक्रे पिता देवव्रतस्तव}
{पतितान्युत्तमाङ्गानि हयेभ्यो हयसादिनाम्}


\twolineshloka
{निर्मनुष्यांश्च मातङ्गाञ्शयानान्पर्वतोपमान्}
{अपश्याम महाराज भीष्मास्त्रेण प्रमाथितान्}


\twolineshloka
{न तत्रासीत्पुमान्कश्चित्पाण्डवानां विशांपते}
{अन्यत्र रथिनां श्रेष्ठद्भीमसेनान्महाबलात्}


\twolineshloka
{स हि भीष्मं समासाद्य छादयामास सायकैः}
{ततो निष्टानको घोरो भीष्मभीमसमागमे}


\twolineshloka
{बभूव सर्वसैन्यानां घोररूपो भयानकः}
{तथैव पाण्डवा हृष्टाः सिंहनादमथानदन्}


\twolineshloka
{ततो दुर्योधनो राजा सोदर्यैः परिवारितः}
{भीष्मं जुगोप समरे वर्तमाने जनक्षये}


\twolineshloka
{भीमस्तु सारथिं हत्वा भीष्मस्य रथिनां वरः}
{प्रद्रुताश्वे रथे तस्मिन्द्रवमाणे समन्ततः}


\twolineshloka
{सुनाभस्य शरेणाशु शिरश्चिच्छेद भारत}
{क्षुरप्रेण सुतीक्ष्णेन स हतो न्यपतद्भुवि}


\twolineshloka
{हते तस्मिन्महाराज तव पुत्रे महारथे}
{नामृष्यन्त रणे शूराः सोदराः सप्त संयुगे}


\twolineshloka
{आदित्यकेतुर्बह्वाशी कुण्डधारो महोदरः}
{अपराजितः पण्डितको विशालाक्षः सुदुर्जयः}


\twolineshloka
{पाण्डवं चित्रसन्नाहा विचित्रकवचध्वजाः}
{अभ्यद्रवन्त संग्रामे योद्धुकामारिमर्दनाः}


\twolineshloka
{महोदरस्तु समरे भीमं विव्याध पत्रिभिः}
{नवभिर्वद्रसंकाशैर्नमुचिं वृत्रहा यथा}


\twolineshloka
{आदित्यकेतुः सप्तत्या बह्वाशी चापि पञ्चभिः}
{नवत्या कुण्डधारश्च विशालाक्षश्च पञ्चभिः}


\twolineshloka
{अपराजितो महाराज पराजिष्णुर्महारथम्}
{शरैर्बहुभिरानर्च्छद्भीमसेनं महाबलम्}


\twolineshloka
{रणे पण्डितकश्चैनं त्रिभिर्बाणैः समार्पयत्}
{स तं न ममृषे भीमः शत्रुभिर्वधमाहवे}


\twolineshloka
{धनुः प्रपीड्य वामेन करेणामित्रकर्सनः}
{शिरश्चिच्छेद समरे शरेणानतपर्वणा}


\twolineshloka
{अपराजितस्य सुनसं तव पुत्रस्य संयुगे}
{पराजितस्य भीमेन कृत्तं गामपतच्छिरः}


\twolineshloka
{अथापरेण भल्लेन कुण्डधारं महारथम्}
{प्राहिणोन्मृत्युलोकाय सर्वलोकस्य पश्यतः}


\twolineshloka
{ततः पुनरमेयात्मा प्रसंधाय शिलीमुखम्}
{प्रेषयामास समरे पण्डितं प्रति भारत}


\twolineshloka
{स शरः पण्डितं हत्वा विवेश धरणीतलम्}
{यथा नरं निहत्याशु भुजगः कालचोदितः}


\twolineshloka
{विशालाक्षशिरश्छित्त्वा पातयामास भूतले}
{त्रिभिः शरैरदीनात्मा स्मरन्क्लेशं पुरातनम्}


\twolineshloka
{महोदरं महेष्वासं नाराचेन स्तनान्तरे}
{विव्याध समरे राजन्स हतो न्यपतद्भुवि}


\twolineshloka
{आदित्यकेतोः केतुं च च्छित्त्वा बाणेन संयुगे}
{भल्लेन भृशतीक्ष्णेन शिरश्चिच्छेद भारत}


\twolineshloka
{बह्वाशिनं तत भीमः शरेणानतर्वणा}
{प्रेषयामास संक्रुद्धो यमस्य सदनं प्रति}


\twolineshloka
{प्रदुद्रुवुस्ततस्तेऽन्ये पुत्रास्तव विशांपते}
{मन्यमाना हि तत्सत्यं सभायां तस्य भाषितम्}


\twolineshloka
{ततो दुर्योधनो राजा भ्रातृव्यसनकर्शितः}
{अब्रवीत्तावकान्योधान्भीमोऽयं वध्यतामिति}


\twolineshloka
{एवमेते महेष्वासाः पुत्रास्त्व विशांपते}
{भ्रातॄन्संदृश्य निहतान्प्रस्मरंस्ते हि तद्वचः}


\twolineshloka
{यदुक्तवान्महाप्राज्ञः क्षत्ता हितमनामयम्}
{तदिदं समनुप्राप्तं वचनं दिव्यदर्शिनः}


\twolineshloka
{लोभमोहसमाविष्टः पुत्रप्रीत्या जनाधिप}
{न बुध्यसे पुरा यत्तत्तथ्यमुक्तं वचो महत्}


\twolineshloka
{तथैव च वधार्थाय पुत्राणां पाण्डवो बली}
{नूनं जातो महाबाहुर्यथा हन्ति स्म कौरवान्}


\twolineshloka
{ततो दुरयोधनो राजा भीष्ममासाद्य संयुगे}
{दुःखेन महताऽऽविष्टो विललाप सुदुःखितः}


\twolineshloka
{निहता भ्रातरः शूरा भीमसेनेन मे युधि}
{यतमानास्तथान्येऽपि हन्यन्ते सर्वसैनिकाः}


\twolineshloka
{भवांश्च मध्यस्थतया नित्यमस्मानुपेक्षते}
{सोऽहं कुपथमारूढः पश्य दैवमिदं मम}


\twolineshloka
{एतच्छ्रुत्वा वचः क्रूरं पिता देवव्रतस्तव}
{दुर्योधनमिदं वाक्यमब्रवीत्समाश्रुलोचनः}


\twolineshloka
{उक्तमेतन्मया पूर्वं द्रोणेन विदुरेण च}
{गान्धार्या च यशस्विन्या तत्त्वं तात न बुद्धवान्}


\twolineshloka
{समयश्च मया पूर्वं कृतो वै शत्रुकर्शन}
{नाहंयुधि नियोक्तव्यो नाप्याचार्यः कथंचना}


\twolineshloka
{यय हि धर्तराष्ट्राणां भीमो द्रक्ष्यति संयुगे}
{हनिष्यति रणे नित्यं सत्यमेतद्ब्रवीमि ते}


\twolineshloka
{स त्वं राजन्स्थिरो भूत्वा रणे कृत्वा दृढां मतिम्}
{योधयस्व रणे पार्थान्स्वर्गं कृत्वा परायणम्}


\twolineshloka
{न शक्यः पाण्डवा जेतुं सेन्द्रैरपि सुरासुरैः}
{तस्माद्युद्धे स्थिरां कृत्वा मतिं युद्ध्यस्व भारत}


\chapter{अध्यायः ८९}
\twolineshloka
{धृतराष्ट्र उवाच}
{}


\twolineshloka
{दृष्ट्वा मे निहतान्पुत्रान्बहूनेकेन सञ्जय}
{भीष्मो द्रोणः कृपश्चैव किमकुर्वत संयुगे}


\twolineshloka
{अहन्यहनि मे पुत्राः क्षयं गच्छन्ति सञ्जय}
{मन्येऽहं सर्वथा पुत्रान्दैवेनोपहतान्भृशम्}


\twolineshloka
{यत्र मे तनयाः सर्वे जीयन्ते न जयन्त्युत}
{यत्र भीष्मस्य द्रोणस्य कृपस्य च महात्मनः}


\twolineshloka
{सौमदत्तेश्च वीरस्य भगदत्तस्य चोभयोः}
{अश्वत्थाम्नस्तथा तात शूराणामनिवर्तिनाम्}


\twolineshloka
{अन्येषां चैव शूराणां मध्यगास्तनया मम}
{यदहन्यन्त संग्रामे किमन्यद्भागधेयतः}


\twolineshloka
{न हि दुर्योधनो मन्दः पुरा प्रोक्तमबुध्यत}
{वार्यमाणो मया तात भीष्मेण विदुरेण च}


\twolineshloka
{गन्धार्या चैव दुर्मेधाः सततं हितकाम्यया}
{नाबुध्यत पुरा मोहात्तस्य प्राप्तमिदं फलम्}


\twolineshloka
{यद्भीमसेनः समरे पुत्रान्मम विचेतसः}
{अहन्यहनि संक्रुद्धो नयते यमसादनम्}


\twolineshloka
{इदं तत्समनुप्राप्तं क्षत्तुर्वचनमुत्तमम्}
{न बुद्धवानस्मि विभो प्रोच्यमानमिदं तदा}


\twolineshloka
{निवारय सुतं द्यूतात्पाण्डवान्मा द्रुहेति च}
{सुहृदां हितकामानां ब्रुवतां तत्तदेव च}


\twolineshloka
{न शृणोमि पुरा वाक्यं मर्त्यः पथ्यमिवौषधम्}
{तदिदं समनुप्राप्तं वचनं साधु भाषितम्}


\twolineshloka
{विदुरद्रोणभीष्माणां तथाऽन्येषां हितैषिणाम्}
{अकृत्वा वचनं पथ्यं क्षयं गच्छन्ति कौरवाः}


\threelineshloka
{तदेतत्समतिक्रान्तं पूर्वमेव हि सञ्जय}
{तस्मान्मे वद तत्त्वेन यथा युद्धमवर्तत ॥सञ्जय उवाच}
{}


\twolineshloka
{मध्याह्ने मुमहारौद्रः संग्रामः समुपद्यत}
{लकक्षयकरो राजंस्तन्मे निगदतः शृणु}


\twolineshloka
{ततः सर्वाणि सैन्यानि धर्मपुत्रस्य शासनात्}
{संरब्धान्यभ्यवर्तन्त भीष्ममेव जिघांसया}


\twolineshloka
{धृष्टद्युम्नः शिखण्डी च सात्यकिश्च महारथः}
{युक्तानीका महाराज भीष्ममेव समभ्ययुः}


\twolineshloka
{विराटो द्रुपदश्चैव सहिताः सर्वसोमकैः}
{अभ्यद्रवन्त संग्रामे भीष्ममेव महारथम्}


\twolineshloka
{केकया धृष्टकेतुश्च कुन्तिभोजश्च दंशितः}
{शुक्तानीका महाराज भीष्ममेव समभ्ययुः}


\twolineshloka
{अर्जुनो द्रौपदश्चैव कुन्तिभोजश्च दंशितः}
{दुर्योधनसमादिष्टान्राज्ञः सर्वान्समभ्ययुः}


\twolineshloka
{अभिमन्युस्तथा शूरो हैडिम्बश्च महारथः}
{भीमसेनश्च संक्रुद्धस्तेऽभ्यधावन्त कौरवान्}


\twolineshloka
{त्रिधाभूतैरवध्यन्त पाण्डवैः कौरवा युधि}
{तथैव कौरवै राजन्नवध्यन्त परे रणे}


\twolineshloka
{द्रोणस्तु रथिनां श्रेष्ठः सोमकान्सृञ्जयैः सह}
{अभ्यधावत संक्रुद्धः प्रेषयिष्यन्यमक्षयम्}


\twolineshloka
{तत्राक्रन्दो महानासीत्सृञ्जयानां महात्मानाम्}
{वध्यतां समरे राजन्भारद्वाजेन धन्विना}


\twolineshloka
{द्रोणेन निहतास्तत्र क्षत्रिया बहवो रणे}
{विचेष्टन्तो हृदृश्यन्त व्याधिक्लिष्टा वरा इव}


\twolineshloka
{कूजतां क्रन्दतां चैव स्तनतां चैव भारत}
{अनिशं शुश्रुवे शब्दः क्षुत्क्लिष्टानां नृणामिव}


\twolineshloka
{तथैव कौरवेयाणां भीमसेनो महाबलः}
{चकार कदनं घोरं क्रुद्धः काल इवापरः}


\twolineshloka
{वध्यतां तत्र सैन्यानामन्योन्येन महारणे}
{प्रावर्तत नदी घोरा रुधिरौघपरवाहिनी}


\twolineshloka
{स संग्रामो महाराज घोररूपोऽभवन्महान्}
{कुरूणां पाणडवानां च यमराष्ट्रविवर्धनः}


\twolineshloka
{ततो भीमो रणे क्रुद्धो रभसश्च विशेषतः}
{गजानीकं समासाद्य प्रेषयामास मृत्यवे}


\twolineshloka
{तत्र भारत भीमेन नाराचाभिहता गजाः}
{पेतुर्नेदुश्च सेदुश्च दिशश्च परिबभ्रमुः}


\twolineshloka
{छिन्नहस्ता महानागाश्छिन्नगात्राश्च मारिष}
{क्रौञ्चवद्व्यनदन्भीताः पृथिवीमधिशेरते}


\twolineshloka
{नकुलः सहदेवश्च हयानीकमभिद्रुतौ}
{ते हयाः काञ्चनापीडा रुक्मभाण्डपरिच्छदाः}


\twolineshloka
{वध्यमाना व्यदृश्यन्त शतशोऽथ सहस्रशः}
{पतिद्भिस्तुरगै राजन्समास्तीर्यत मेदिनी}


\twolineshloka
{निर्जिह्वैश्च श्वसद्भिश्च कूजद्भिश्च गतासुभिः}
{हयैर्बभौ नरश्रेष्ठ नानारूपधरैर्धरा}


\twolineshloka
{अर्जुनेन हतैः सङ्ख्ये तथा भारत राजभिः}
{प्रबभौ वसुधा घोरा तत्रतत्र विशांपते}


\twolineshloka
{रथैर्भग्नैर्ध्वजैश्छिन्नैर्निकृत्तैश्च महायुधैः}
{चामरैर्व्यजनैश्चैव च्छत्रैश्च सुमहाप्रभैः}


\twolineshloka
{हरैर्निष्कैः सकेयूरैः शिरोभिश्च सकुण्डलैः}
{उष्णीषैरपविद्धैश्च पताकाभिश्च सर्वशः}


\twolineshloka
{अनुकर्षैः शुभै राजन्योक्रैश्चैव सरश्मिभिः}
{संकीर्णा वसुधा भाति वसन्ते कुसुमैरिव}


\twolineshloka
{एवमेष क्षयो वृत्तः पाण्डूनामपि भारत}
{क्रुद्धे शान्तनवे भीष्मे द्रोणे च रथसत्तमे}


\twolineshloka
{अश्वत्थाम्नि कृपे चैव तथैव कृतवर्मणि}
{तथेतरेषु क्रुद्धेषु तावकानामपि क्षयः}


\chapter{अध्यायः ९०}
\twolineshloka
{सञ्जय उवाच}
{}


\twolineshloka
{वर्तमाने तथा रौद्रे राजन्वीरवरक्षये}
{शकुनिः सौबलः श्रीमान्पाण्डवान्समुपाद्रवत्}


\twolineshloka
{तथैव सात्वतो राजन्हार्दिक्यः परवीरहा}
{अभ्यद्रवत संग्रामे पाण्डवानां वरूथिनीम्}


\twolineshloka
{ततः काम्भोजमुख्यानां नदीजानां च वाजिनाम्}
{आरट्टानां महीजानां सिन्धुजानां च सर्वशः}


\twolineshloka
{वनायुजानां शुभ्राणां तथा पर्वतवासिनाम्}
{वाजिनां बहुभिः सङ्ख्ये समन्तात्परिवारयन्}


\twolineshloka
{ये चापरे तित्तिरिजा जवना वातरंहसः}
{सुवर्णालङ्कृतैरेतैर्वर्मवद्भिः सुकल्पितैः}


\twolineshloka
{हयैर्वातजवैर्मुख्यैः पाण्डवस्य सुतो बली}
{अभ्यवर्तत तत्सैन्यं हृष्टरूपः परंतपाः}


\twolineshloka
{अर्जुनस्य सुतः श्रीमानिरावान्नाम वीर्यवान्}
{सुतायां नागराजस्य जातः पार्थेन धीमता}


\twolineshloka
{ऐरावतेन सा दत्ता अनपत्या महात्मना}
{पत्यौ हते सुपर्णेन कृपणा दीनचेतना}


\twolineshloka
{कार्यार्थं तां च जग्राह पार्थः कामवशानुगाम्}
{एवमेष समुत्पन्नः परक्षेत्रेऽर्जुनात्मजः}


\twolineshloka
{स नागलोके संवृद्धो मात्रा च परिरक्षितः}
{पितृव्येण परित्यक्तः पार्थद्वेषाद्दुरात्मना}


\twolineshloka
{खमुत्पत्य महाराज नागराट् सत्यविक्रमः}
{इन्द्रलोकं जगामाशु श्रुत्वा तत्रार्जुनं गतम्}


\twolineshloka
{सोऽभिगम्य महाबाहुः पितरं सत्यविक्रमः}
{अभ्यवादयदव्यग्रो विनयेन कृताञ्जलिः}


\twolineshloka
{न्यवेदयत चात्मानमर्जुनस्य महात्मनः}
{इरावानस्मि भद्रं ते पुत्रश्चाहं तव प्रभो}


\twolineshloka
{मातुः समागमो यश्च तत्सर्वं प्रत्यवेदयत्}
{तच्च सर्वं यथावृत्तमनुसस्मार पाण्डवः}


\twolineshloka
{परिष्वज्य सुतं चापि आत्मनः सदृशं गुणैः}
{प्रीतिमाननयत्पार्थो देवराजनिवेशनम्}


\twolineshloka
{सोऽर्जुनेन समाज्ञप्तो देवलोके तदा नृप}
{प्रीतिपूर्वं महाबाहुः स्वकार्यं प्रति भारत}


\twolineshloka
{`स चापि नरशार्दूलः शार्दूलसमविक्रमः}
{'अब्रवीच्च तदा पार्थमयमस्मि तवानघ}


\twolineshloka
{स्थितः प्रेष्यश्च पुत्रश्च सर्वथा त्वं प्रशाधि माम्}
{कं करोमि च ते कामं कं च कामं त्वमिच्छसि}


\twolineshloka
{परिष्वज्य सुतं प्रेम्णा वासविः प्रत्युवाच तम्}
{प्रीतिपूर्वं च कार्यं च कार्या प्रीतिश्च मानदः ॥'}


\twolineshloka
{युद्धकाले तु साहाय्यं दातव्यं नो भवेदिति}
{बाढमित्येवमुक्त्वा तु युद्धकाल इहागतः}


\twolineshloka
{कामवर्णजवैरश्वैः सर्वतः शुशुभे वृतः}
{ते हयाः काञ्चनापीडा नानावर्णा मनोजवाः}


\twolineshloka
{उत्पेतुः सहसा राजन्हंसा इव महोदधौ}
{ते त्वदीयान्समासाद्य हयसंघान्मनोजवान्}


\twolineshloka
{क्रोडैः क्रोडानभिघ्नन्तो घोणाभिश्च परस्परम्}
{निपेतुः सहसा राजन्सुवेगाभिहता भुवि}


\twolineshloka
{निपतद्भिस्तथा तैश्च हयसङ्घैः परस्परम्}
{शुश्रुवे दारुणः शब्दः सुपर्णपतने यथा}


\twolineshloka
{तथैव तावका राजन्समेत्यान्योन्यमाहवे}
{परस्परवधं घोरं चक्रुस्ते हयसादिनः}


\twolineshloka
{तस्मिंस्तथा वर्तमाने संकुले तुमुले भृशम्}
{उभयोरपि संशान्ता हयसङ्घाः समन्ततः}


\twolineshloka
{प्रक्षीणसायकाः शूरा निहताश्वाः श्रमातुराः}
{विलयं समनुप्राप्तास्तक्षमाणाः परस्परम्}


\twolineshloka
{ततः क्षीणे हयानिके किंचिच्छेषे च भारत}
{सौबलस्यानुजाः शूरा निर्गता रणमूर्धनि}


\twolineshloka
{वायुवेगसमस्पर्शाञ्जवे वायुसमांश्च ते}
{आरुह्य बलसंपन्नान्वयःस्थांस्तुरगोत्तमान्}


\twolineshloka
{गजो गवाक्षो वृषकश्चर्मवानार्जयः शुकः}
{षडेते बलसंपन्ना निर्ययुर्महतो बलात्}


\twolineshloka
{वार्यमाणाः शकुनिना स्वैश्च योधैर्महाबलैः}
{सन्नद्धा युद्धकुशला रौद्ररूपा महाबलाः}


\twolineshloka
{तदनीकं महाबहो भित्त्वा परमदुर्जयम्}
{बलेन महता युक्ताः स्वर्गाय विजयैषिणः}


\twolineshloka
{विविशुस्ते तदा हृष्टा गान्धारा युद्धदुर्मदाः}
{तान्प्रविष्टांस्तदा दृष्ट्वा इरावानपि वीर्यवान्}


\twolineshloka
{अब्रवीत्समरे योधान्स्वान्विचित्रहयस्थितान्}
{यथैते धार्तराष्ट्रस्य योधाः सानुगवाहनाः}


\twolineshloka
{हन्यन्ते समरे सर्वे तथा नीतिर्विधीयताम्}
{बाढमित्येवमुक्त्वा ते सर्वे योधा इरावतः}


\twolineshloka
{जघ्नुस्तेषां बलानीकं दुर्जयं समरे परैः}
{तदनीकमनीकेन समरे वीक्ष्य पातितम्}


\twolineshloka
{अमृष्यमाणास्ते सर्वे सुबलस्यात्मजा रणे}
{इरावन्तमभिद्रुत्य सर्वतः पर्यवारयन्}


\twolineshloka
{ताडयन्तः शितैः प्रासैश्चोदयन्तः परस्परम्}
{ते शूराः पर्यधावन्त कुर्वन्तो महदाकुलम्}


\twolineshloka
{इरावानथ निर्भिन्नः प्रासैस्तीक्ष्णैर्महात्मभिः}
{स्रवता रुधिरेणाक्तस्तोत्रैर्विद्ध इव द्विपः}


\twolineshloka
{उरस्यपि च पृष्ठे च पार्श्वयोश्च भृशाहतः}
{एको बहुभिरत्यर्थं धैर्याद्राजन्न विव्यथे}


\twolineshloka
{इरावानपि संक्रुद्धः सर्वास्तान्निशितैः शरैः}
{मोहयामास समरे विद्ध्वा परपुरंजयः}


\twolineshloka
{प्रासानुत्कृष्य तरसा स्वशरीरादरिन्दमः}
{तैरेव ताडयामास सुबलस्यात्मजान्रणे}


\twolineshloka
{विकृष्य च शितं खङ्गं गृहीत्वा च शरावरम्}
{पदातिर्द्रुतमागच्छज्जिघांसुः सौबलान्युधि}


\twolineshloka
{ततः प्रत्यागतप्राणाः सर्वे ते सुबलात्मजाः}
{भूयः क्रोधसमाविष्टा इरावन्तमभिद्रुताः}


\twolineshloka
{इरावानपि खङ्गेन दर्शयन्पाणिलाघवम्}
{अभ्यवर्तत देवांश्च मानुषांश्चैव संयुगे}


\twolineshloka
{लाघवेनाथ चरतः सर्वे ते सुबलात्मजाः}
{अन्तरं नाभ्यगच्छन्त चरन्तः शीघ्रगैर्हयैः}


\twolineshloka
{भूमिष्ठमथ तं सह्ख्ये ह्यन्तरिक्षादवप्लुतम्}
{परिवार्य भृशं सर्वे ग्रहीतुमुपचक्रमुः}


\twolineshloka
{अथाभ्याशगतानां तेषां गात्राण्यकृन्तत}
{असिहस्तोऽस्त्रहस्तानां तेषां गात्राण्यकृन्तत}


\twolineshloka
{आयुधानि च सर्वेषां बाहूनपि विभूषितान्}
{अपतन्त निकृत्ताङ्गा मृता भूमौ गतासवः}


\twolineshloka
{वृषकस्तु महाराज बहुधा विपरिक्षतः}
{अमुच्यत महारौद्रात्तस्माद्वीरावकर्तनात्}


\twolineshloka
{तान्सर्वान्पतितान्दृष्ट्वा भीतो दुर्योधनस्ततः}
{अभ्यधावत संक्रुद्धो राक्षसं घोरदर्शनम्}


\twolineshloka
{आर्श्यशृङ्गिं महेष्वासं मायाविनमरिन्दमम्}
{वैरिणं भीमसेनस्य पूर्वं बकवधेन वै}


\twolineshloka
{पश्य वीर यथा ह्येष फल्गुनस्य सुतो बली}
{मायावी विप्रियं कर्तुमकार्षीन्मे बलक्षयम्}


\twolineshloka
{त्वं च कामगमस्तात मायास्त्रे च विशारदः}
{कृतवैरश्च पार्थेन तस्मादेनं रणे जहि}


\twolineshloka
{बाढमित्येवमुक्त्वा तु राक्षसो घोरदर्शनः}
{प्रययौ सिंहनादेन यत्रार्जुनसुतो युवा}


\twolineshloka
{स योधैर्युद्धकुशलैर्विमलप्रासयोधिभिः}
{वीरैः प्रहारिभिर्युक्तैः स्वैरनीकैः समावृतः}


% Check verse!
[हतशेषैर्महाराज द्विसाहस्त्रैर्हयोत्तमैः]निहन्तुकामः समरे इरावन्तं महाबलम्
% Check verse!
इरावानपि संक्रुद्धस्त्वरमाणः पराक्रमीहन्तुकाममयित्रघ्नो राक्षसं प्रत्यवारयत्
\twolineshloka
{तमापतन्तं संप्रेक्ष्य राक्षसः सुमहाबलः}
{त्वरमाणस्ततो मायां प्रयोक्तुमुपचक्रमे}


\twolineshloka
{तेन मायामयाः सृष्टा हयास्तावन्त एव हि}
{स्वारूढा राक्षसैर्घोरैः शूलपट्टसधारिभिः}


\twolineshloka
{ते संरब्धाः समागम्य द्विसाहस्राः प्रहारिणः}
{अचिराद्गमयामासुः प्रेतलोकं परस्परम्}


\twolineshloka
{तस्मिंस्तु निहते सैन्ये तावुभौ युद्धदुर्मदौ}
{संग्रामे समतिष्ठेतां यथा वै वृत्रवासवौ}


\twolineshloka
{आद्रवन्तमभिप्रेक्ष्य राक्षसं युद्धदुर्मदम्}
{इरावाथ संरब्धः प्रत्यधावन्महाबलः}


\twolineshloka
{समभ्याशगतस्याजौ तस्य खङ्गेन दुर्मदः}
{चिच्छेद कार्मुकं दीप्तं शरवापं च कञ्चुकम्}


\twolineshloka
{स निकृत्तं धनुर्दृष्ट्वा खं जवेन समाविशत्}
{इरावन्तमभिक्रुद्धं मोहयन्निव मायया}


\twolineshloka
{ततोऽन्तरिक्षमुत्पत्य इरावानपि राक्षसम्}
{विमोहयित्वा मायाभिसत्स्य गात्रामि सायकैः}


\twolineshloka
{चिच्छेद सर्वमर्मज्ञः कामरूपो दुरासदः}
{तथा स राक्षसश्रेष्ठः शरैः कृत्तः पुनः पुनः}


\twolineshloka
{संबभूव महाराज समवाप च यौवनम्}
{माया हि सहराज समवाप च यौवनम्}


\twolineshloka
{एवं तद्राक्षसस्याङ्गं छिन्नंछिन्नं व्यरोहत}
{इरावानपि संक्रुद्धो राक्षसं त महाबलम्}


\threelineshloka
{परश्वथेन तीक्ष्णेन चिच्छेद च पुनः पुनः}
{स तेन बलिना वीरश्छिद्यमान इव द्रुमः ॥ 6-90-71aराक्षसोव्यनदद्धोरं स शब्दस्तुमुलोऽभवत्}
{परश्वथक्षतं रक्षः सुस्राव बहुशोणितम्}


\twolineshloka
{ततश्रुक्रोध बलवांश्चक्रे वेगं च संयुगे}
{आर्श्यशृङ्गिस्ततो दृष्ट्वा समरे शत्रुमूर्चितम्}


\twolineshloka
{कृत्वा घोरं महद्रूपं ग्रहीतुमुपचक्रमे}
{अर्जुनस्य सुतं वीरमिरावन्तं यशस्विनम्}


\twolineshloka
{संग्रामशिरसो मध्ये सर्वेषां तत्र पश्यताम्}
{तां दृष्ट्वा तादृशीं मायां राक्षसस्य दुरात्मनः}


\twolineshloka
{इरावानपि संक्रुद्धो मायां स्रष्टुं प्रचक्रमे}
{तस्य क्रोधाभिभूतस्य समरेष्वनिवर्तिनः}


\twolineshloka
{योऽन्वयो मातृकस्तस्य स एनमभिपेदिवान्}
{स नागैर्बहुभी राजन्निरावान्संवृतो रणे}


\twolineshloka
{दधार सुमहद्रूपमनन्त इव भोगवान्}
{ततो बहुविधैर्नागैश्छादयामास राक्षसम्}


\twolineshloka
{छाद्यमानस्तु नागैः स ध्यात्वा राक्षसपुङ्गवः}
{सौपर्णं रूपमास्थाय भक्षयामास पन्नगम्}


\twolineshloka
{मायया भक्षिते तस्मिन्नन्वये तस्य मातृके}
{विमोहितमिरावन्तं न्यहनद्राक्षसोऽसिना}


\twolineshloka
{सकुण्डलं समुकुटं पद्मेन्दुसदृशप्रभम्}
{इरावतः शिरो रक्षः पातयामास भूतले}


\twolineshloka
{तस्मिंस्तु निहते वीरे राक्षसेनार्जुनात्मजे}
{विशोकाः समपद्यन्त धार्तराष्ट्राः सराजकाः}


\twolineshloka
{तस्मिन्महति संग्रामे तादृशे भैरवे पुनः}
{महान्व्यतिकरो घोरः सेनयोः समपद्यत}


\twolineshloka
{गजा हयाः पदाताश्च विमिश्रा दन्तिभिर्हताः}
{रथाश्वा दन्तिनश्चैव पत्तिभिस्तत्र सूदिताः}


\twolineshloka
{तथा पत्तिरथौघाश्च हयाश्च बहवो रणे}
{रथिभिर्निहता राजंस्तव तेषां च संकुले}


\twolineshloka
{अजानन्नर्जुनश्चापि निहतं पुत्रमौरसम्}
{जघान समरे शूरान्राज्ञस्तान्भीष्मरक्षिणः}


\twolineshloka
{तथैव तावका राजन्सृञ्जयाश्च सहस्रशः}
{जुह्वतः समरे प्राणान्निजघ्रुरितरेतरम्}


\twolineshloka
{मुक्तकेशा विकवचा विरथाश्छिन्नकार्मुकाः}
{बाहुभिः समयुध्यन्त समवेताः परस्परम्}


\twolineshloka
{तथा मर्मातिगैर्भीष्मो निजघान महारथान्}
{कम्पयन्समरे सेनां पाण्डवानां परंतपः}


\twolineshloka
{तेन यौधिष्ठिरे सैन्ये बहवो मानवा हताः}
{दन्तिनः सादिनश्चैव रथिनोऽथ हयास्तथा}


\twolineshloka
{तत्र भारत भीष्मस्य रणे दृष्ट्वा पराक्रमम्}
{अत्यद्भुतमपश्याम शक्रस्येव पराक्रमम्}


\twolineshloka
{तथैव भीमसेनस्य पार्षतस्य च भारत}
{रौद्रमासीद्रणे युद्धं सात्यकस्य च धन्विनः}


\twolineshloka
{दृष्ट्वा द्रोणस्य विक्रान्तं पाण्डवान्भयमाविशत्}
{एक एव रणे शक्तो निहन्तुं सर्वसैनिकान्}


\twolineshloka
{किं पुनः पृथिवीशूरैर्योधव्रातैः समावृतः}
{इत्यब्रुवन्महाराज रणे द्रोणेन पीडिताः}


\twolineshloka
{वर्तमाने तथा रौद्रे संग्रामे भरतर्षभ}
{उभयोः सेनयोः शूरा नामृष्यन्त परस्परम्}


\twolineshloka
{आविष्टा इव युध्यन्ते रक्षोभूतैर्महाबलैः}
{तावकाः पाण्डवेयाश्च संरब्धास्तात धन्विनः}


\twolineshloka
{न स्म पश्यामहे कंचित्प्राणान्यः परिरक्षति}
{दैवासुराभे समरे तस्मिन्योधा नराधिप}


\chapter{अध्यायः ९१}
\twolineshloka
{धृतराष्ट्र उवाच}
{}


\threelineshloka
{इरावन्तं तु निहतं दृष्ट्वा पार्था महारथाः}
{संग्रामे किमकुर्वन्त तन्ममाचक्ष्व सञ्जय ॥सञ्जय उवाच}
{}


\twolineshloka
{इरावन्तं तु निहतं संग्रामे वीक्ष्य राक्षसः}
{व्यनदत्सुमहानादं भैमसेनिर्घटोत्कचः}


\twolineshloka
{नदतस्तस्य शब्देन पृथिवी सागराम्बरा}
{सपर्वतवना राजंश्चचाल सुभृशं तदा}


\twolineshloka
{अन्तरिक्षं दिशश्चैव सर्वाश्च प्रदिशस्तथा}
{`चेलुश्च सहसा तत्र तेन नादेन नादिताः'}


\twolineshloka
{तं श्रुत्वा सुमहानादं तव सैन्यस्य भारत ॥ऊरुस्तम्भः समभवद्वेपथुः स्वेद एव च}
{}


\twolineshloka
{सर्व एव महाराज तावका दीनचेतसः ॥सर्वतः समचेष्टन्त सिंहभीता गजा इव}
{}


\twolineshloka
{नर्दित्वा सुमहानादं निर्घातमिव राक्षसः ॥ज्वलितं शूलमुद्यम्य रूपं कृत्वा विभीषणम्}
{}


\twolineshloka
{नानारूपप्रहरणैर्वृतो राक्षसपुङ्गवैः ॥आजघान सुसंक्रुद्धः कालान्तकयमोपभः}
{}


\twolineshloka
{तमापतन्तं संप्रेक्ष्य संक्रुद्धं भीमदर्शनम् ॥स्वबलं च भयात्तस्य प्रायशो विमुखीकृतम्}
{}


\twolineshloka
{ततो दुर्योधनो राजा घटोत्कचमुपाद्रवत् ॥प्रगृह्य विपुलं चापं सिंहवद्विनदन्मुहुः}
{}


% Check verse!
पृष्ठतोऽनुययौ चैनं स्रवद्भिः पर्वतोपमैः ॥कुञ्जरैर्दशसाहस्रैर्वङ्गानामधिपः स्वयम्
\twolineshloka
{पुत्रं तव महाराज चुकोप स निशाचरः}
{ततः प्रववृते युद्धं तुमुलं रोमहर्षणम्}


\twolineshloka
{राक्षसानां च राजेन्द्र दुर्योधनबलस्य च}
{गजानीकं च संप्रेक्ष्य मेघबृन्दमिवोदितम्}


\twolineshloka
{अभ्यधावन्त संक्रुद्धा राक्षसाः शस्त्रपाणयः}
{नदन्तो विविधान्नादान्मेघा इव सविद्युतः}


\twolineshloka
{शरशक्त्यृष्टिनाराचैर्निघ्नन्तो गजयोधिनः}
{भिण्डिपालैस्तथा शूलैर्मुद्गरैः सपरश्वथैः}


\twolineshloka
{पर्वताग्रैश्च वृक्षैश्च निजघ्नुस्ते महागजान्}
{भिन्नकुम्भान्विरुधिरान्भिन्नगात्रांश्च वारणान्}


\twolineshloka
{अपश्याम महाराज वध्यमानान्निशाचरैः}
{तेषु प्रक्षीयमाणेषु भग्नेषु गजयोधिषु}


\twolineshloka
{दुर्योधनो महाराज राक्षसान्समुपाद्रवत्}
{अमर्षवशमापन्नस्त्यक्त्वा जीवितात्मनः}


\twolineshloka
{मुमोच निशितान्बाणान्राक्षसेषु परंतप}
{जघान च महेष्वासः प्रधानांस्तत्र राक्षसान्}


\twolineshloka
{संक्रुद्धो भरतश्रेष्ठ पुत्रो दुर्योधनस्तव}
{वेगवन्तं महारौद्रं विद्युज्जिह्वं प्रमाथिनम्}


\twolineshloka
{शरैश्चतुर्भिश्चतुरो निजघान महाबलः}
{ततः पुनरमेयात्मा शरवर्षं दुरासदम्}


\twolineshloka
{मुमोच भरतश्रेष्ठो निशाचरबलं प्रति}
{तत्तु दृष्ट्वा महत्कर्म पुत्रस्य तव मारिष}


\twolineshloka
{क्रोधेनाभिप्रजज्वाल भैमसेनिर्महाबलः}
{स विष्फार्य महच्चापमिन्द्राशनिसमप्रभम्}


\twolineshloka
{अभिदुद्राव वेगेन दुर्योधनमरिन्दमम्}
{तमापतन्तमुद्वीक्ष्य कालसृष्टमिवान्तकम्}


\threelineshloka
{न विव्यथे महाराज पुत्रो दुर्योधनस्तव}
{अथैनमब्रवीत्क्रुद्धः क्रूरः संरक्तलोचनः ॥घटोत्कच उवाच}
{}


\twolineshloka
{अद्यानृण्यं गमिष्यामि पितॄणां मातुरेव च}
{ये त्वया सुनृशंसेन दीर्घकालं प्रवासिताः}


\twolineshloka
{यच्च ते पाण्डवा राजंश्छलद्यूते पराजिताः}
{यच्चैव द्रौपदी कृष्णा एववस्त्रा रजस्वला}


\twolineshloka
{सभामानीय दुर्बुद्धे बहुधा क्लेशिता त्वया}
{तव च प्रियकामेन आश्रमस्था दुरात्मना}


\twolineshloka
{सैन्धवेन परामृष्टा परिभूय पितॄन्मम}
{एतेषामपमानानामन्येषां च कुलाधम}


\twolineshloka
{अन्तमद्य गमिष्यामि यदि नोत्सृजसे रणम्}
{एवमुक्त्वा तु हैडिम्बो महद्विष्फार्य कार्मुकम्}


\threelineshloka
{संदश्य दशनैरोष्ठं सृकिणी परिसंलिहन्}
{शरवर्षेण महता दुर्योधनमवाकिरत्}
{पर्वतं वारिधाराभिः प्रावृषीव बलाहकः}


\chapter{अध्यायः ९२}
\twolineshloka
{सञ्जय उवाच}
{}


\twolineshloka
{ततस्तद्बाणवर्षं तु दुःसहं दानवैरपि}
{दधार युधि राजेन्द्रो यथा वर्षं महाद्विपः}


\twolineshloka
{ततः क्रोधसमाविष्टो निःश्वसन्निव पन्नगः}
{संशयं परमं प्राप्तः पुत्रस्ते भरतर्षभ}


\twolineshloka
{मुमोच निशितांस्तीक्ष्णान्नाराचान्पञ्चविंशतिम्}
{तेऽपतन्सहसा राजंस्तस्मिन्राक्षसपुङ्गवे}


\twolineshloka
{आशीविषा इव क्रुद्धाः पर्वते गन्धमादने}
{स तैर्विद्धः स्रवन्रक्तं प्रभिन्न इव कुञ्जरः}


\twolineshloka
{दध्रे मतिं विनाशाय राज्ञः स पिशिताशनः}
{जग्राह च महाशक्तिं गिरीणामपि दारिणीम्}


\twolineshloka
{संप्रदीप्तां महोल्काभामशनिं ज्वलितामिव}
{तमागच्छन्महाबाहुर्जिघांसुस्तनयं तव}


\twolineshloka
{तामुद्यतामभिप्रेक्ष्य वङ्गानामधिपस्त्वरन्}
{कुञ्जरं गिरिसंकाशं राक्षसं प्रत्यचोदयत्}


\twolineshloka
{स नागप्रवरेणाजौ बलिना शीघ्रगामिना}
{यतो दुर्योधनरथस्तं मार्गं प्रत्यषेधयत्}


\twolineshloka
{पन्थानं वारयामास कुञ्जरेण सुतस्य ते}
{मार्गमावारितं दृष्ट्वा राज्ञा वङ्गेन धीमता}


\twolineshloka
{घटोत्कचो महाराज क्रोधसंरक्तलोचनः}
{उद्यतां तां महाशक्तिकं तस्मिंश्चिक्षेप वारणे}


\twolineshloka
{स तयाऽभिहतो राजंस्तेन बाहुप्रमुक्तया}
{संजातरुधिरोत्पीडः पपात च ममार च}


\twolineshloka
{पतत्यथ गजे चापि वङ्गानामीश्वरो बली}
{जवेन समभिद्रुत्य जगाम धरणीतलम्}


\twolineshloka
{दुर्योधोऽपि संप्रेक्ष्य पतितं वरवारणम्}
{प्रभग्नं च बलं दृष्ट्वा जगाम परमां व्यथाम्}


\threelineshloka
{`अशक्तः प्रतियोद्धुं वै दृष्ट्वा तस्य पराक्रमम्}
{'क्षत्रधर्मं पुरस्कृत्य आत्मनश्चातिमानिताम्}
{प्राप्तेऽपक्रनये राजा तस्थौ गिरिरिवाचलः}


\twolineshloka
{राधान च शितं बाणं कालाग्निसमतेजसम्}
{सुयोच धरमक्रुद्धस्तस्मिन्घोरे निशाचरे}


\twolineshloka
{तमापतन्तं संप्रेक्ष्य वणिमिन्द्राशनिप्रभम्}
{लाघवान्मोचयामास महात्मा वै घटोत्कचः}


\twolineshloka
{भूयश्च निननादोषं क्रोधसंरक्तलोचनः}
{त्रासयामास रौन्यानि युगन्ते जलदो यथा}


\twolineshloka
{तं श्रुत्वा निनिदं घोरं तस्व भीमस्य रक्षसः}
{आचार्यमुसंगम्य भीष्मः शान्तनवोऽब्रवीत्}


\twolineshloka
{यथैव निनन्दो घोरः श्रूयते राक्षसेरितः}
{हैडिम्बो युध्यते नूनं राज्ञा दुर्योधनेन च}


\twolineshloka
{नैष शक्यो हि संग्रामे जेतुं भूतेन केनचित्}
{तत्र गच्छत भद्रं वो राजानं परिरक्षितुम्}


\twolineshloka
{अभिद्रुत्य महाबाहुं राक्षसेन प्रपीडितम्}
{एतद्धि परमं कृत्यं सर्वेषां नः परंतपाः}


\twolineshloka
{पितामहवचः श्रुत्वा त्वरमाणा महारथाः}
{उत्तमं जवमास्थाय प्रययुर्यत्र कौरवः}


\twolineshloka
{द्रोणश्च सोमदत्तश्च बाह्लीकोऽथ जयद्रथः}
{कृपो भिरिश्रवाः शल्य आवन्त्यः स बृहद्बलः}


\twolineshloka
{अश्वत्थामा विकर्णश्च चित्रसेनो विविंशतिः}
{रथाश्चानेकसाहस्रा ये तेषामनुयायिनः}


\twolineshloka
{अभिद्रुतं परीप्सन्तः पुत्रं दुर्योधनं तव}
{तदनीकमनाधृष्यं पालितं तु महारथैः}


\twolineshloka
{पाततायिनमायान्तं प्रेक्ष्य राक्षससत्तमः}
{...कम्पत महाबाहुर्मैनाक इव पर्वतःक}


\twolineshloka
{प्रगृह्य विपुलं चापं ज्ञातिभिः परिवारितः}
{शूलमुद्गरहस्तैश्च नानाप्रहरणैरपि}


\twolineshloka
{ततः समभवद्युद्धं तुमुलं रोमहर्षणम्}
{राक्षसानां च मुख्यस्य दुर्योधनबलस्य च}


\twolineshloka
{धनुषां कूजतां शब्दः सर्वतस्तुमुलो रणे}
{अश्रूयत महाराज वंशानां दह्यतामिव}


\twolineshloka
{शस्त्राणां पात्यमानानां कवचेषु शरीरिणाम्}
{शब्दः समभवद्राजन्गिरीणामिव भिद्यताम्}


\twolineshloka
{वीरबाहुविसृष्टानां तोमराणां विशांपते}
{रूपमासीद्वियत्स्थानां सर्पाणामिव सर्पताम्}


\twolineshloka
{ततः परमसंक्रुद्धो विष्फार्य सुमहद्धनुः}
{राक्षसन्द्रो महाबाहुर्विनदन्भैरव रवम्}


\twolineshloka
{आचार्यस्यार्धचन्द्रेण क्रुद्धश्चिच्छेद कार्मुकम्}
{सोमदत्तस्य भल्लेन ध्वजं चोन्मथ्य चानदत्}


\twolineshloka
{बाह्लीकं च त्रिभिर्बाणैः प्रत्यविध्यत्स्तनान्तरे}
{कृपमेकेन विव्याध चित्रसेनं त्रिभिः शरैः}


\twolineshloka
{पूर्णायतविसृष्टेन सम्यक्प्रणिहितेन च}
{जत्रुदेशे समासाद्य विकर्णं समताडयत्}


\twolineshloka
{न्यषीदत्स्वरथोपस्थे शोणितेन परिप्लुतः}
{ततः पुनरमेयात्मा नाराचान्दश पञ्च च}


\twolineshloka
{भूरिश्रवसि संक्रुद्धः प्राहिणोद्भरतर्षभ}
{ते वर्म भित्त्वा तस्याशु विविशुर्धरणीतलम्}


\twolineshloka
{विविंशतेश्च द्रौणेश्च यन्तारौ समताडयत्}
{तौ पेततू रथोपस्थे रश्मीनुत्सृज्य वाजिनाम्}


\twolineshloka
{सिन्धुराज्ञोऽर्धचन्द्रेण वाराहं स्वर्णभूषितम्}
{उन्ममाथ महाराज द्वितीयेनाच्छिनद्धनुः}


\twolineshloka
{चतुर्भिरथ नाराचैरावन्त्यस्य महात्मनः}
{जघान चतुरो वाहान्क्रोधसंरक्तलोचनः}


\twolineshloka
{पूर्णायतविसृष्टेन पीतेन निशितेन च}
{निर्बिभेद महाराज राजपुत्रं बृहद्बलम्}


\twolineshloka
{स गाढविद्धो व्यथितो रथोपस्थ उपाविशत्}
{भृशं क्रोधेन चाविष्टो रथस्थो राक्षसाधिपः}


\twolineshloka
{चिक्षेप निशितांस्तीक्ष्णाञ्छरानाशीविषोपमान्}
{बिभिदुस्ते महाराज शल्यं युद्धविशारदम्}


\chapter{अध्यायः ९३}
\twolineshloka
{सञ्जय उवाच}
{}


\twolineshloka
{विमुखीकृत्य वर्सांस्तु तावकान्युधि राक्षसः}
{जिघांसुभरतश्रेष्ठ दुर्योधनमुपाद्रवत्}


\twolineshloka
{तमापतन्तं संप्रेक्ष्य राजानं प्रति वेगितम्}
{अभ्यधावञ्जिघांसन्तस्तावका युद्धदुर्मदाःक}


\twolineshloka
{तालमात्राणि चापानि विकर्षन्तो महारथाः}
{तमेकमभ्यधावन्त नदन्तः सिंहसङ्घवत्}


\twolineshloka
{अथैनं शरवर्षेण समन्तात्पर्यवाकिरन्}
{पर्वतं वारिधाराभिः शरदीव वलाहकाः}


\twolineshloka
{स गाढविद्धो व्यथितस्तोत्रार्दित इव द्विपः}
{उत्पपात तदाऽऽकाशं समन्ताद्वैनतेयवत्}


\twolineshloka
{व्यनदत्सुमहानादं जीमूत इव शारदः}
{दिशः खं विदिशश्चैव नादयन्भैरवस्वनः}


\twolineshloka
{राक्षसस्य तु तं शब्दं श्रुत्वा राजा युधिष्ठिरः}
{उवाच भरतश्रेष्ठ भीमसेनमरिन्दमम्}


\twolineshloka
{युध्यते राक्षसो नूनं धार्तराष्ट्रैर्महारथैः}
{यथाऽस्य श्रूयते शब्दो नदतो भैरवं स्वनम्}


\twolineshloka
{अतिभारं च पश्यामि तस्मिन्राक्षसपुङ्गवे}
{पितामहश्च संक्रुद्धः पाञ्चालान्हन्तुमुद्यतः}


\twolineshloka
{तेषां च रक्षणार्थाय युध्यते फल्गुनः परैः}
{एतज्ज्ञात्वा महाबाहो कार्यद्वयमुपस्थितम्}


\twolineshloka
{गच्छ रक्षस्व हैडिम्बं शंशयं परमं गतम्}
{भ्रातुर्वचनमाज्ञाय त्वरमाणो वृकोदरः}


\twolineshloka
{प्रययौ सिंहनादेन त्रासयन्सर्वपार्थिवान्}
{वेगेन महता राजन्पर्वकाले यथोदधिःक}


\twolineshloka
{तमन्वगात्सत्यधृतिः सौचित्तिर्युद्धदुर्मदः}
{श्रेणिमान्वसुदानश्च पुत्रः काश्यस्य चाभिभूः}


\twolineshloka
{अभिमन्युमुखाश्चैव द्रौपदेया महारथाः}
{क्षत्रदेवश्च विक्रन्तः क्षत्रधर्मा तथैव च}


\twolineshloka
{अनूपाधिपतिश्चैव नीलः स्वबलमास्थितः}
{महता रथवंशेन हैडिम्बं पर्यवारयन्}


\twolineshloka
{कुञ्चरैश्च सदा मत्तैः षट्सहस्रैः प्रहारिभिःक}
{अभ्यरक्षन्त सहिता राक्षसेन्द्रं घटोत्कचम्}


\twolineshloka
{सिंहनादेन महता नेमिघोषेण चैव ह}
{खुरशब्दनिपातैश्च कम्पयन्तो वसुंधराम्}


\twolineshloka
{तेषामापततां श्रुत्वा शब्दं तं तावकं बलम्}
{भीमसेनभयोद्विग्नं विवर्णवदनं तथा}


\twolineshloka
{परिवृत्य महाराज परित्यज्य घटोत्कचम्}
{ततः प्रववृते युद्धंक तत्र तेषां महात्मनाम्}


\twolineshloka
{तावकानां परेषां च संग्रामेष्वनिवर्तिनाम्}
{नानारूपाणि शस्त्राणि विसृजन्तो महारथाः}


\twolineshloka
{अन्योन्यमभिधावन्तः संप्रहारं प्रचक्रिरे}
{व्यतिष्यक्तं महारौद्रं युद्धं भीरुभयावहम्}


\twolineshloka
{हया हयैः समाजग्मुः पादाताश्च पदातिभिः}
{`रथा रथैः समागच्छन्नागा नागैश्च संयुगे'अन्योन्यं समरे राजन्प्रार्थयानाः समभ्ययुः}


\twolineshloka
{सहसा चाभवत्तीव्रं सन्निपातन्महद्रजः}
{गजाश्वरथपत्तीनां पादनेमिसमुद्धतम्}


\twolineshloka
{धूम्रारुणं रजस्तीव्रं रणभूमिं समावृणोत्}
{नैव स्वे न परे राजन्समजानन्परस्परम्}


\twolineshloka
{पिता पुत्रं न जानीते पुत्रो वा पितरं तथा}
{निर्मर्यादे तथाभूते वैशसे रोमहर्षणे}


\twolineshloka
{शस्त्राणां भरतश्रेष्ठ मनुष्याणां च र्जताम्}
{सुमहानभवच्छब्दः प्रेतानामिव भारत}


\twolineshloka
{गजवाजिमनुष्याणां शोणितान्रतरङ्गिणी}
{प्राववर्तत नदी तत्र केशशैवलशाद्वला}


\twolineshloka
{नराणां चैव कायेभ्यः शिरसां पततां रणे}
{शुश्रुवे सुमहाञ्छब्दः पततामश्मनामिव}


\twolineshloka
{विशिरस्कैर्मनुष्यैश्च छिन्नगात्रैश्च वारणैः}
{अश्वैः संभिन्नदेहैश्च संकीर्णाऽभूद्वसुंधरा}


\twolineshloka
{नानाविधानि शस्त्राणि विसृजन्तो महारथाः}
{अन्योन्यमभिधावन्तः संप्रहारार्थमुद्यताः}


\twolineshloka
{हया हयान्समासाद्य प्रेषिता हयसादिभिः}
{समाहत्य रणेऽन्योन्यं निपेतुर्गतजीविताः}


\twolineshloka
{नरा नरान्समासाद्य क्रोधरक्तेक्षणा भृशम्}
{उरांस्युरोभिरन्योन्यं समाश्लिष्य निजघ्निरे}


\twolineshloka
{प्रेषिताश्च महामात्रैर्वारणाः परवारणैः}
{अभ्यघ्नन्त विषणाग्रैर्वारणानेव संयुगे}


\twolineshloka
{ते जातरुधिरोत्पीडाः पताकाभिरलङ्कृताः}
{संसक्ताः प्रत्यदृश्यन्त मेघा इव सविद्युतः}


\twolineshloka
{केचिद्भिन्ना विषाणाग्रैर्भिन्नकुम्भाश्च तोमरैः}
{विनदन्तोऽभ्यधावन्त गर्जमाना धना इव}


\twolineshloka
{केचिद्धस्तैर्द्विधाच्छिन्नैश्छिन्नगात्रास्तथाऽपरे}
{निपेतुस्तुमुले तस्मिंश्छिन्नपक्षा इवाद्रयः}


\twolineshloka
{पार्श्वैस्तु दारितैरन्ये वारणैर्वरवारणाः}
{मुमुचुः शोणितं भूरि धातूनिव महीधराः}


\twolineshloka
{नाराचनिहतास्त्वन्ये तथा विद्धाश्च तोमरैः}
{विनदन्तोऽभ्यधावन्त विशृङ्गा इव पर्वताः}


\twolineshloka
{केचित्क्रोधसमाविष्टा मदान्धा निरवग्रहाः}
{रथान्हयान्पदातींश्च ममृदुः शतशो रणे}


\twolineshloka
{तथा हया हयारोहैस्ताडिताः प्रासतोमरैः}
{तेन तेनाभ्यवर्तन्त कुर्वन्तो व्याकुला दिशः}


\twolineshloka
{रथिनो रथिभिः सार्धं कुलपुत्रास्तनुत्यजः}
{परां शक्तिं समास्थाय चक्रुः कर्माण्यभीतवत्}


\twolineshloka
{स्वयंवर इवामर्दे प्रजह्रुरितरेतरम्}
{प्रार्थयाना यशो राजन्स्वर्गं वा युद्धशालिनः}


\twolineshloka
{तस्मिंस्तथा वर्तमाने संग्रामे रोमहर्षणे}
{धार्तराष्ट्रं महत्सैन्यं प्रययौ विमुखीकृतम्}


\chapter{अध्यायः ९४}
\twolineshloka
{सञ्जय उवाच}
{}


\twolineshloka
{स्वसैन्यं निहतं दृष्ट्वा राजा दुर्योधनः स्वयम्}
{अभ्यधावत्सुसंक्रुद्धो भीमसेनमरिन्दमम्}


\twolineshloka
{प्रगृह्य सुमहच्चापमिन्द्राशनिसमस्वनम्}
{महता शरवर्षेण पाण्डवं समवाकिरत्}


\twolineshloka
{अर्धचन्द्रं च संधाय सुतीक्ष्णं लोमवापिनम्}
{भीमसेनस्य चिच्छेद चापं क्रोधसमन्वितः}


\twolineshloka
{तदन्तरं च संप्रेक्ष्य त्वरमामो महारथः}
{प्रसंदधे शितं बाणं गिरीणामपि दारणम्}


\twolineshloka
{तेनोरसि महाराज भीमसेनमताडयत्}
{स गाढविद्धो व्यथितः सृक्विणी परिसंलिहन्}


\twolineshloka
{समाललम्बे तेजस्वी ध्वजं हेमपरिष्कृतम्}
{तथा विमनसं दृष्ट्वा भीमसेनं घटोत्कचः}


\twolineshloka
{क्रोधेनाभिप्रजज्वाल दिधक्षन्निव पावकः}
{अभिमन्युमुखाश्चापि पाण्डवानां महारथाः}


\twolineshloka
{समभ्यधावन्क्रोशन्तो राजानं जातसंभ्रमाः}
{संप्रेक्ष्य तान्संपततः संक्रुद्धाञ्जातसंभ्रमान्}


\twolineshloka
{भारद्वाजोऽब्रवीद्वाक्यं तावकानां महारथान्}
{क्षिप्रं गच्छत भद्रं वो राजानं परिरक्षत}


\twolineshloka
{संशयं परमं प्राप्तं मञ्जन्तं व्यसनार्णवे}
{एते क्रुद्धा महेष्वासाः पाण्डवानां महारथाः}


\twolineshloka
{भीमसेनं पुरस्कृत्य दुर्योधनमुपाद्रवन्}
{नानाविधानि शस्त्राणि विसृजन्तो महारथाः}


\twolineshloka
{नदन्तो भैरवान्नादांस्त्रासयन्तश्च भूमिपान्}
{तदाऽऽचार्यवचः श्रुत्वा सौमदत्तिपुरोगमाः}


\twolineshloka
{तावकाः समवर्तन्त पाम्डवानामनीकिनीम्}
{कृपो भूरिश्रवाः शल्यो द्रोणपुत्रो विविंशतिः}


\twolineshloka
{चित्रसेनो विकर्णश्च सैन्धवोऽथ बृहद्बलः}
{आवन्त्यौ च महेष्वासौ कौरवं पर्यवारयन्}


\twolineshloka
{ते विंशतिपदं गत्वा संप्रहारं प्रचक्रिरे}
{पाण़्डवा धार्तराष्ट्राश्च परस्परजिघांसवःक}


\twolineshloka
{एवमुक्त्वा महाबाहुर्महद्विष्फार्य कार्मुकम्}
{भारद्वाजस्ततो भीमं षड्विंशत्या समार्पयत्}


\twolineshloka
{भूयश्चैनं महाबाहुः शरैः शीघ्रमवाकिरत्}
{पर्वतं वारिधाराभिः प्रावृषीव वलाहकः}


\twolineshloka
{तं प्रत्यविध्यद्दशभिर्भीमसेनः शिलीमुखैः}
{त्वरमाणो महेष्वासः सव्ये पार्श्वे महाबलः}


\twolineshloka
{स गाढविद्धो व्यथितो वयोवृद्धश्च भारत}
{प्रनष्टसंज्ञः सहसा रथोपस्थ उपाविशत्}


\twolineshloka
{गुरुं प्रव्यथितं दृष्ट्वा राजा दुर्योधनः स्वयम्}
{द्रौणायनिश्च संक्रुद्धौ भीमसेनमभिद्रुतौ}


\twolineshloka
{तावापतन्तौ संप्रेक्ष्य कालान्तकयमोपमौ}
{भीमसेनो महाबाहुर्गदामादाय सत्वरम्}


\threelineshloka
{अवप्लुत्य रथात्तूर्णं तस्थौ गिरिरिवाचलः}
{समुद्यम्य गदां गुर्वी यमदण्डोपमां रणे}
{}


\twolineshloka
{तमुद्यतगदं दृष्ट्वा कैलासमिव शृङ्गिणम्}
{कौरवो द्रोणपुत्रश्च सहितावभ्यधावताम्}


\twolineshloka
{तावापतन्तौ सहितौ त्वरितौ बलिनां वरौ}
{अभ्यधावत वेगेन त्वरमाणो वृकोदरः}


\twolineshloka
{तमापतन्तं संप्रेक्ष्य संक्रुद्धं भीमदर्शनम्}
{समभ्यधावंस्त्वरिताः कौरवाणां महारथाः}


\twolineshloka
{भारद्वाजमुखाः सर्वे भीमसेनजिघांसया}
{नानाविधानि शस्त्राणि भीमस्योरस्यपातयन्}


\twolineshloka
{सहिताः पाण्डवं सर्वे पीडयन्तः समन्ततः}
{तं दृष्ट्वा संशयं प्राप्तं पीड्यमानं महारथम्}


\twolineshloka
{अभिमन्युप्रभृतयः पाण़्डवानां महारथाःक}
{अभ्यदावन्परीप्सन्तः प्राणांस्त्यक्त्वा सुदुस्त्यजान्}


\twolineshloka
{अनूपाधिपतिः शूरो भीमस्य दयितः सखा}
{नीलो नीलाम्बुदप्रख्यः संक्रुद्धो द्रौणिमभ्यात्}


\twolineshloka
{स्पर्धते हि महेष्वासो नित्यं द्रोणसुतेन सः}
{स विष्फार्य महच्चापं द्रौणिं विव्याध पत्रिणा}


\twolineshloka
{यथा शक्रो महाराज पुरा विव्याध दानवम्}
{विप्रचित्तिं दुराधर्षं देवतानां भयंकरम्}


\twolineshloka
{येन लोकत्रयं क्रोधात्रासितं स्वेन तेजसाक}
{6-94-32b`स रुद्रेणजितः पूर्वं निहतो मातरिश्वना' तथा नीलेन निर्भिन्नः सम्यङ्भुक्तेन पत्रिणा}


\twolineshloka
{संजातरुधिरोत्पीडो द्रौणिः क्रोधसमन्वितः}
{स विष्फार्य धनुश्चित्रमिन्द्राशनिसमस्वनम्}


\twolineshloka
{दध्रे नीलविनाशाय मतिं मतिमतां वरः}
{ततः संधाय विमलान्भल्लान्कर्मारमार्जितान्}


\twolineshloka
{जघान चतुरो वाहान्पातयामास च ध्वजम्}
{`सूतं चैकेन भल्लेन रथनीडादपाहरत्'सप्तमेन च भल्लेन नीलं विव्याध वक्षसि}


\twolineshloka
{स गाढविद्धो व्यथितो रथोपस्थ उपाविशत्}
{मोहितं वीक्ष्य राजानं नीलमश्मचयोपमम्}


\fourlineindentedshloka
{घटोत्कचोऽभिसंक्रुद्धो ज्ञातिभिः परिवारितः}
{अभिदुद्राव वेगेन द्रौणिमाहवशोभिनम्}
{तथेतरे चाभ्यधावन्राक्षसा युद्धदुर्मदाः ॥ 6-94-38a` भीमसेनोमहाबाहुर्नीलं नीलाञ्जनप्रभम्}
{'आरोप्य स्वरथं वीरो दुर्योधनमपाद्रवत्}


% Check verse!
घटोत्कचं च संप्रेक्ष्य राक्षसं घोरदर्शनम् ॥अभ्यधावत तेजस्वी भारज्वाजात्मजस्त्वरन्
\threelineshloka
{निजघान च संक्रुद्धो राक्षसान्भीमदर्शनान्}
{योऽभवन्नग्रतः क्रुद्धा राक्षसस्य पुरःसराः}
{विमुखांश्चैव तान्दृष्ट्वा द्रौणिचापच्युतैः शरैः}


\twolineshloka
{अक्रुद्ध्यत महाकायो भैमसेनिर्घटोत्कचः}
{प्रादुश्चक्रे ततो मायां घोररूपां सुदारुणाम्}


\twolineshloka
{मोहयन्समरे द्रौणिं मायावी राक्षसाधिपः}
{ततस्ते तावकाः सर्वे मायया विमुखीकृताःक}


\twolineshloka
{अन्योन्यं समपश्यन्त निकृत्ता मेदिनीतले}
{विचेष्टमानाः कृपणाः शोणितेन परिप्लुताः}


\twolineshloka
{द्रोणं दुर्योधनं शल्यमश्वत्थामानमेव च}
{प्रायशश्च महेष्वासा ये प्रधानाः स्म कौरवाः}


\twolineshloka
{विध्वस्ता रथिनः सर्वे गजाश्च विनिपातिताः}
{हयाश्चैव हयारोहाः सन्निकृत्ताः सहस्रशः}


\twolineshloka
{तद्दृष्ट्वा तावकं सैन्यं विद्रुतं शिबिरं प्रति}
{मम प्राक्रोशतो राजंस्तथा देवव्रतस्य च}


\twolineshloka
{युध्यध्वं मा पलायध्वं मायैषा राक्षसी रणे}
{घटोत्कचप्रमुक्तेति नातिष्ठन्त विमोहिताः}


\twolineshloka
{नैव ते श्रद्दधुर्भीता वदतोरावयोर्वचः}
{तांश्च प्रद्रवतो दृष्ट्वा जयं प्राप्ताश्च पाण्डवाः}


\twolineshloka
{घटोत्कचेन सहिताः सिंहनादान्प्रचक्रिरे}
{शङ्खदुन्दुभिनिर्घोषैः समन्तान्नेदिरे भृशम्}


\twolineshloka
{एवं तव बलं सर्वं हैडिम्बेन महात्मना}
{सूर्यास्तमनवेलायां प्रभग्नं विद्रुतं दिशः}


\chapter{अध्यायः ९५}
\twolineshloka
{सञ्जय उवाच}
{}


\threelineshloka
{तस्मिन्महति संक्रन्दे राजा दुर्योधनस्तदा}
{`पराजयं राक्षसेन नामृष्यत परंतपः}
{'गाङ्गेयमुपसंगम्य विनयेनाभिवाद्य च}


\twolineshloka
{तस्य सर्वं यथावृत्तमाख्यातुमुपचक्रमे}
{घटोत्कचस्य विजयमात्मनश्च पराजयम्}


\twolineshloka
{कथयामास दुर्धर्षो विनिःश्वस्य पुनःपुनः}
{अब्रवीच्च तदा राजन्भीष्मं कुरुपितामहम्}


\twolineshloka
{भवन्तं समुपाश्रित्य द्रोणं चैव पितामह}
{पाण्डवैर्विग्रहो घोरः समारब्धो मया प्रभो}


\twolineshloka
{एकादश समाख्याता अक्षौहिण्यश्च या मम}
{निदेशे तव तिष्ठन्ति मया सार्धं परंतप}


\twolineshloka
{सोऽहं भरतशार्दूल भीमसेनपुरोगमैः}
{घटोत्कचं समाश्रित्य पाण्डवैर्युधि निर्जितः}


\twolineshloka
{तन्मे दहति गात्राणि शुष्कवृक्षमिवानलः}
{तदिच्छामि महाभाग त्वत्प्रसादात्परंतप}


\threelineshloka
{राक्षसापशदं हन्तुं स्वयमेव पितामह}
{त्वां समाश्रित्य दुर्धर्षं तन्मे कर्तुं त्वमर्हसि ॥सञ्जय उवाच}
{}


\twolineshloka
{एतच्छ्रुत्वा तु वचनं राज्ञो भरतसत्तम}
{दुर्योधनमिदं वाक्यं भीष्मः शान्तनवोऽब्रवीत्}


\twolineshloka
{शृणु राजन्मम वचो यत्त्वां वक्ष्यामि कौरव}
{यथा त्वया महाराज वर्तितव्यं परंतप}


\twolineshloka
{आत्मा रक्ष्यस्त्वया तात सर्वावस्थास्वरिन्दम}
{धर्मराजेन संग्रामस्त्वया कार्यः सदाऽनघ}


\twolineshloka
{अर्जुनेन यमाभ्यां वा भीमसेनेन वा पुनः}
{अन्यैर्वा पृथिवीपालै राजा राजानमृच्छति}


\threelineshloka
{न तु कार्यस्त्वया राजन्हैडिम्बेन दुरात्मा}
{अहं द्रोणः कृपो द्रौणिः कृतवर्मा च सात्वतः}
{शल्यश्च वृषसेनश्च विकर्णश्च महारथः}


\twolineshloka
{शेषाश्च भ्रातरस्त्वन्ये दुःशासनपुरोगमाः}
{त्वदर्थे प्रतियोत्स्यामो राक्षसं तं महाबलम्}


\twolineshloka
{रौद्रे तस्मिन्राक्षसेन्द्रे यदि ते हृच्छयो महान्}
{अयमागच्छतु रणं तस्य युद्धाय दुर्मतेः}


\twolineshloka
{भगदत्तो महीपालः पुरन्दरसमो युधि}
{एतावदुक्त्वा राजानं भगदत्तमथाब्रवीत्}


\twolineshloka
{समक्षं पार्थिवेन्द्रस्य वाक्यं वाक्यविशारदः}
{गच्छ शीघ्रं महाराज हैडिम्बं युद्धदुर्मदम्}


\twolineshloka
{वारयस्व रणे यत्तो मिषतां सर्वधन्विनाम्}
{राक्षसं क्रूरकर्माणं यथेन्द्रस्तारकं पुरा}


\twolineshloka
{तव दिव्यानि चास्त्राणि विक्रमश्च परंतप}
{समागमश्च बहुभिः पुराऽभूदमरैः सह}


\threelineshloka
{त्वं तस्य नृपशार्दूल प्रतियोद्धा महाहवे}
{स्वबलेनावृतो गच्छ जहि राक्षसपुङ्गवम् ॥सञ्जय उवाच}
{}


\twolineshloka
{एतच्छ्रुत्वा तु वचनं भीष्मस्य पृतनापतेः}
{प्रययौ सिंहनादेन परानभिमुखको द्रुतम्}


\twolineshloka
{तमापतन्तं संप्रेक्ष्य गर्जन्तमिव तोयदम्}
{अभ्यवर्तन्त संक्रुद्धाः पाण्डवानां महारथाः}


\twolineshloka
{भीमसेनोऽभिमन्युश्च राक्षसश्च घटोत्कचः}
{द्रौपदेयाः सत्यधृतिः क्षत्रदेवश्च भारत}


\twolineshloka
{चेदिपो वसुदानश्च दशार्णाधिपतिस्तथा}
{सुप्रतीकेन तांश्चापि भगदत्तोऽप्युपाद्रवत्}


\twolineshloka
{ततः समभवद्युद्धं घोररूपं भयानकम्}
{पाण्डूनां भगदत्तेन यमराष्ट्रविवर्धनम्}


\twolineshloka
{प्रयुक्ता रथिभिर्बाणा भीमवेगाः सुतेजनाः}
{ते निपेतुर्महाराज नागेषु च रथेषु च}


\twolineshloka
{प्रभिन्नाश्च महानागा विनीता हस्तिसादिभिः}
{परस्परं समासाद्य संनिपेतुरभीतवत्}


\twolineshloka
{मदान्धा रोषसंरब्धा विषाणाग्रैर्महाहवे}
{बिभिदुर्दन्तमुसलैः समासाद्य परस्परम्}


\twolineshloka
{हयाश्च चामरापीडाः प्रासपाणिभिरास्थिताः}
{चोदिताः सादिभिः क्षिप्रं निजघ्रुरितरेतरम्}


\twolineshloka
{पादाताश्च पदात्योघैस्ताडिताः शक्तितोमरैः}
{न्यपतन्त तदा भूमौ शतशोऽथ सहस्रशः}


\twolineshloka
{रथिनश्च रथाकंश्चित्रान्कर्णिनालीकतोमरैः}
{निहत्य समरे वीरान्सिंहनादान्विनेदिरे}


\twolineshloka
{तस्मिंस्तथा वर्तमाने संग्रामे रोमहर्षणे}
{भगतत्तो महेष्वासो भीमसेनमथाद्रवत्}


\twolineshloka
{कुञ्जरेण प्रभिन्नेन सप्तधा स्रवता मदम्}
{पर्वतेन यथा तोयं स्रवमाणेन सर्वशः}


\twolineshloka
{किरञ्छरसहस्राणि सुप्रतीकशिरोगतः}
{ऐरावतस्थो मघवान्वारिधारा इवानघ}


\twolineshloka
{स भीमं शरधाराभिस्ताजयामास पार्थिवः}
{पर्वतं वारिधाराभिस्तपान्ते जलदो यथा}


\twolineshloka
{भीमसेनस्तु संक्रुद्धः पादरक्षान्परःशतान्}
{निजघान महेष्वासः संरब्धः शरवृष्टिभिः}


\twolineshloka
{तान्दृष्ट्वा निहतान्क्रुद्धो भगदत्तः प्रतापवान्}
{चोदयामास नागेन्द्रं भीमसेनरथं प्रति}


\twolineshloka
{स नागः प्रेषितस्तेन बाणो ज्याचोदितो यथा}
{अभ्यधावत वेगेन भीमसेनपुरोगमाः}


\twolineshloka
{तमापतन्तं संप्रेक्ष्य पाण्डवानां महारथाः}
{अभ्यवर्तन्त वेगेन भीमसेनपुरोगमाः}


\twolineshloka
{केकयाश्चाभिमन्युश्च द्रौपदेयाश्च सर्वशः}
{दशार्णाधिपतिः शूरः क्षत्रदेवश्च मारिष}


\twolineshloka
{चेदिपो धृष्टकेतुश्च संरब्धाः सर्व एव ते}
{उत्तमास्त्राणि दिव्यानि दर्शयन्तो महाबलाः}


\twolineshloka
{तमेकं कुञ्जरं क्रुद्धाः समन्तात्पर्यवारयन्}
{स विद्धो बहुभिर्बाणैर्व्यरोचत महाद्विपः}


\twolineshloka
{संजातरुधिरोत्पीडो धातुचित्र इवाद्रिराट्}
{दशार्णाधिपतिश्चाऽपि गजं भूमिधरोपमम्}


\twolineshloka
{समास्थितोऽभिदुद्राव भगदत्तस्य वारणम्}
{तमापतन्तं समरे जगं जगपतिः स च}


\twolineshloka
{दधार सुप्रतीकोऽपि वेलेव मकरालयम्}
{वारितं प्रेक्ष्य नागेन्द्रं दशार्णस्य महात्मनः}


\twolineshloka
{साधुसाध्विति सैन्यानि त्वदीयान्यभ्यपूजयन्}
{ततः प्राग्ज्योतिषः क्रुद्धस्तोमरान्वै चतुर्दश}


\twolineshloka
{प्राहिणोत्तस्य नागस्य प्रमुखे नृपसत्तम}
{तस्य कुम्भपरित्राणां शातकुम्भपरिष्कृतम्}


\twolineshloka
{विदार्य प्राविशन्क्षिप्रं वल्मीकमिव पन्नगाः}
{स गाढविद्धो व्यथितो नागो भरतसत्तम}


\twolineshloka
{उपावृत्तमदः क्षिप्रमभ्यवर्तत वेगिनः}
{स प्रदुद्राव वेगेन प्रणदन्भैरवं रवम्}


\twolineshloka
{संमर्दमानः स्वबलं वायुर्वृक्षानिवौजसा}
{तस्मिन्पराजिते नागे पाण्डवानां महारथाः}


\twolineshloka
{सिंहनादं विनद्योच्चैर्युद्धायैवावतस्थिरे}
{ततो भीमं पुरस्कृत्य भगदत्तमुपाद्रवन्}


\twolineshloka
{किरन्तो विविधान्बाणाञ्शस्त्राणि विविधानि च}
{तेषामापततां राजन्संक्रुद्धानाममर्षिणाम्}


\twolineshloka
{श्रुत्वा स निनदंक घोरममर्षाद्गतसाध्वसः}
{भगदत्तो महेष्वासः स्वनागं प्रत्यचोदयत्}


\twolineshloka
{अङ्कुशाङ्गुष्ठनुदितः स गजप्रवरो युधि}
{तस्मिन्क्षणे समभवत्सांवर्तक इवानलः}


\twolineshloka
{रथसङ्घांस्तथा नागान्हयांश्च हयसादिनः}
{पादातांश्च सुसंक्रुद्धः शतशोऽथ सहस्रशःक}


\twolineshloka
{अमृद्गात्समरे नागः संप्रधावंस्ततस्ततः}
{तेन संलोड्यमानं तु पाण्डवानां बलं महत्}


\twolineshloka
{संचुकोच महाराज चर्मेवाग्नौ समाहितम्}
{भग्नं तु स्वबलं दृष्ट्वा भगदत्तेन धीमता}


\twolineshloka
{घटोत्कचोऽथ संक्रुद्धो भगदत्तमुपाद्रवत्}
{विकटः परुषो राजन्दीप्तास्यो दीप्तलोचनः}


\twolineshloka
{रूपं विभीषणं कृत्वा रोषेण प्रज्वलन्निव}
{जग्राह विमलं शूलं गिरीणामपि दारणम्}


\twolineshloka
{नागं जिघांसुः सहसा चिक्षेप च महाबलः}
{स विस्फुलिङ्गमालाभिः समन्तात्परिवेष्टितम्}


\twolineshloka
{तमापतन्तं सहसा दृष्ट्वा प्राग्ज्योतिषो नृपः}
{चिक्षेप रुचिरं तीक्ष्णमर्धचन्द्रं सुदारुणम्}


\twolineshloka
{चिच्छेद तन्महच्छूलं तेन बाणेन वेगवान्}
{उत्पपात द्विधा च्छिन्नं शूलं हेमपरिष्कृतम्}


\twolineshloka
{महाशनिर्यथा भ्रष्टा शक्रमुक्ता नभोगता}
{शूलं निपतितं दृष्ट्वा द्विधा कृत्तं च पार्थिवः}


\twolineshloka
{रुक्मदण्डां महाशक्तिं जग्राहाग्निशिखोपमान्}
{चिक्षेप तां राक्षसस्य तिष्ठितिष्ठेति चाब्रवीत्}


\twolineshloka
{तामापतन्तीं संप्रेक्ष्य वियत्स्थामशनीमिव}
{उत्पत्य राक्षसस्तूर्णं जग्राह च ननाद च}


\twolineshloka
{बभञ्ज चैनां त्वरितो जानुन्यारोप्य भारत}
{पश्यतः पार्थिवेन्द्रस्य तदद्भुतमिवाभवत्}


\twolineshloka
{तदवेक्ष्य कृतं कर्म राक्षसेन बलीयसा}
{दिवि देवाः सगन्धर्वा मुनयश्चापि विस्मिताः}


\twolineshloka
{पाण्डवाश्च महाराज भीमसेनपुरोगमाः}
{साधुसाध्विति नादेन पृथिवीमन्वनादयन्}


\twolineshloka
{तं तु श्रुत्वा महानादं प्रहृष्टानां महात्मनाम्}
{नामृष्यत महेष्वासो भगदत्तः प्रतापवान्}


\twolineshloka
{स विष्फार्य महच्चापमिन्द्राशानिसमप्रभम्}
{अभिदुद्राव वेगेन पाण्डवानां महारथान्}


\twolineshloka
{विसृजन्विमलांस्तीक्ष्णान्नाराचाञ्ज्वलनप्रभान्}
{भीममेकेन विव्याध राक्षसं नवभिः शरैः}


\twolineshloka
{अभिमन्युं त्रिभिश्चैव केकयान्पञ्च पञ्चभिः}
{पूर्णायतविसृष्टेन शरेणानतपर्वणा}


\twolineshloka
{बिभेद दक्षिणं बाहुं क्षत्रेदवस्य चाहवे}
{पपात सहसा तस्य सशरं धनुरुत्तमम्}


\twolineshloka
{द्रौपदेयांस्ततः पञ्च पञ्चभिः समताडयत्}
{भीमसेनस्य च क्रोधान्निजघान तुरंगमान्}


\twolineshloka
{ध्वजं केसरिणं चास्य चिच्छेद विशिखैस्त्रिभिः}
{निर्बिभेद त्रिभिश्चान्यैः सारथिं चास्य पत्रिभिः}


\twolineshloka
{स गाढविद्धो व्यथितो रथोपस्थ उपाविशत्}
{विशोको भरतश्रेष्ठ भगदत्तेन संयुगे}


\twolineshloka
{ततो भीमो महाबाहुर्विरथो रथिनां वरः}
{गदां प्रगृह्य वेगेन प्रचस्कन्द रथोत्तमात्}


\twolineshloka
{तमुद्यतगदं दृष्ट्वा सशृङ्गमिव पर्वतम्}
{तावकानां भयं घोरं समपद्यत भारत}


\twolineshloka
{एतस्मिन्नेव काले तु पाण्डवः कृष्णसारथिः}
{आजगाम महाराज निघ्नञ्शत्रून्समन्ततः}


\twolineshloka
{यत्र तौ पुरुषव्याघ्रौ पितापुत्रौ महाबलौ}
{प्राग्ज्योतिषेण संयुक्तौ भीमसेनघटोत्कचौ}


\twolineshloka
{दृष्ट्वा च पाण्डवो भ्रातॄन्युध्यमानान्महारथान्}
{त्वरितो भरतश्रेष्ठ तत्रायुध्यत्किरञ्छरान्}


\twolineshloka
{ततो दुर्योधनो राजा त्वरमाणो महारथः}
{सेनामचोदयत्क्षिप्रं रथनागाश्वसंकुलाम्}


\twolineshloka
{तामापतन्तीं सहसा कौरवाणां महाचमूम्}
{अभिदुद्राव वेगेन पाण्डवः श्वेतवाहनः}


\twolineshloka
{भगदत्तश्च समरे तेन नागेन भारत}
{विमृद्गन्पाण्डवबलं युधिष्ठिरमुपाद्रवत्}


\twolineshloka
{तदासीत्सुमहद्युद्धं भगदत्तस्य मारिष}
{पाञ्चालैः पाण्डवेयैश्च केकयैश्चोद्यतायुधैः}


\twolineshloka
{भीमसेनोऽपि समरे तावुभौ केशवार्जिनौ}
{अश्रावयद्यथावृत्तमिरावद्वधतुत्तमम्}


\chapter{अध्यायः ९६}
\twolineshloka
{सञ्जय उवाच}
{}


\twolineshloka
{पुत्रं विनिहतं श्रुत्वा इरावन्तं धऩञ्जयः}
{दुःखेन महताऽऽविष्टो निःश्वसन्पन्नगो यथा}


\twolineshloka
{अब्रवीत्समरे राजन्वासुदेवमिदं वचः}
{इदं नूनं महाप्राज्ञो विदुरो दृष्टवान्पुरा}


\twolineshloka
{कुरूणां पाण्डवानां च क्षयं घोरं महामतिःक}
{स ततो निवारितवान्धृतराष्ट्रं जनेश्वरम्}


\twolineshloka
{अन्ये च बहवो वीराः संग्रामे मधुसूदन}
{निहताः कौरवैः सङ्ख्ये तथाऽस्माभिश्च कौरवाः}


\twolineshloka
{अर्थहेतोर्नरश्रेष्ठ क्रियते कर्म कुत्सितम्}
{धिगर्थान्यत्कृते ह्येवं क्रियते ज्ञातिसंक्षयः}


\twolineshloka
{अधनस्य मृतं श्रेयो न च ज्ञातिवधाद्वनम्}
{किं नु प्राप्स्यामहे कृष्ण हत्वा ज्ञातीन्समागतान्}


\twolineshloka
{दुर्योधनापराधेन शकुनेः सौबलस्य च}
{क्षत्रिया निधनं यान्ति कर्णदुर्मन्त्रितेन च}


\twolineshloka
{इदानीं च विजानामि सुकृतं मधुसूदन}
{कृतं राज्ञा महाबाहो याचता त सुयोधनम्}


\twolineshloka
{राज्यार्धं पञ्च वा ग्रामान्नाकार्षीत्स च दुर्मतिःक}
{दृष्ट्वा हि क्षत्रियाञ्शूराञ्शयानान्धरणीतले}


\twolineshloka
{निन्दामि भृशमात्मानं धिगस्तु क्षत्रिजीविकाम्}
{अशक्तमिति मामेते ज्ञास्यन्ते क्षत्रिया रणे}


\twolineshloka
{एतदर्थं मया युद्धं रोचितं मधुसूदन}
{संचोदय हयाञ्शीघ्रं धार्तराष्ट्रचमूं प्रति}


\threelineshloka
{प्रतरिष्ये महापारं भुजाभ्यां समरोदधिम्}
{नायं क्लीबायितुं कालो विद्यते माधव क्वचित् ॥सञ्जय उवाच}
{}


\twolineshloka
{एवमुक्तस्तु पार्थेन केशवः परवीरहा}
{चोदयामास तानश्वान्पाण्डुरान्वातरंहसः}


\twolineshloka
{अथ शब्दो महानासीत्तव सैन्यस्य भारत}
{मारुतोद्धूतवेगस्य सागरस्येव पर्वणि}


\twolineshloka
{अपराह्णे महाराज संग्रामः समपद्यत}
{पर्जन्यसमनिर्घोषो भीष्मस्य सह पाण्डवैः}


\twolineshloka
{ततो राजंस्तव सुता भीमसेनमुपाद्रवन्}
{परिवार्य रणे द्रोणं वसवो वासवं यथा}


\twolineshloka
{ततः शान्तनवो भीष्मः कृपश्च रथिनां वरः}
{भगदत्तः सुशर्मा च धनंजयमुपाद्रवन्}


\twolineshloka
{हार्दिक्यो बाह्लिकश्चैव सात्यकिं समभिद्रुतौ}
{अम्बष्ठकस्तु नृपतिरभिमन्युमवस्थितः}


\twolineshloka
{शेषास्त्वन्ये महाराज शेषानेव महारथान्}
{ततः प्रववृते युद्धं घोररूपं भयावहम्}


\twolineshloka
{`भीमसेनस्तु संप्रेक्ष्य पुत्रांस्तव विशांपते'}
{प्रजज्वाल रणे क्रुद्धो हविषा हव्यवाडिव}


\twolineshloka
{पुत्रास्तु तव कौन्तेयं छादयांचक्रिरे शरैः}
{प्रावृषीव महाराज जलदा इव पर्वतम्}


\twolineshloka
{स च्छाद्यमानो बहुधा पुत्रैस्तव विशांपते}
{सृक्विणी संलिहन्वीरः शार्दूल इव दर्पितः}


\twolineshloka
{व्यूढोरस्कस्ततो भीमः पोथयामास पार्थिवम्}
{क्षुरप्रेण सुतीक्ष्णेन सुमुक्तेन महारणे}


\threelineshloka
{ताडयामास संक्रुद्धः सोऽभवद्व्यथितेन्द्रियः}
{अपरेण तु भल्लेन पीतेन निशितेन तु}
{अपातयत्कुण्डलितं सिंहः क्षुद्रमृगं यथा}


\twolineshloka
{ततः सुनिशितान्बाणान्भीमसेनः शिलाशितान्}
{स सप्त संदधे हन्तुं पुत्रास्ते भरतर्षभ}


\twolineshloka
{प्रेषिता भिमसेनेन शरास्ते दृढधन्वना}
{अपातयन्त पुत्रांस्ते रथेभ्यः सुमहारथान्}


\twolineshloka
{अनाधृष्टिं कुण्डभेदिं विराजं दीप्तलोचनम्}
{दीर्घबाहुं सुबाहुं च तथैव मकरध्वजम्}


\twolineshloka
{प्रपतन्तिस्म वीरास्ते विरेजुर्भरतर्षभ}
{वसन्ते पुष्पशबलाः किंशुकाः पतिता इव}


\twolineshloka
{ततः प्रदुद्रुवुः शेषास्तव पुत्रा महाहवे}
{तं कालमिव मन्यन्तो भीमसेनं महाबलम्}


\twolineshloka
{द्रोणस्तु समरे वीरं निर्दहनक्तं सुतांस्तव}
{यथाद्रिं वारिधाराभिः समन्ताद्व्यकिरच्छरैः}


\twolineshloka
{तत्राद्भुतमपश्याम कुन्तीपुत्रस्य पौरुषम्}
{द्रोणेन वार्यमाणोऽपि निजघ्ने यत्सुतांस्तव}


\twolineshloka
{यथा गोवृषभो वर्षं संधारयति स्वात्पतत्}
{भीमस्तथा द्रोणमुक्तं शरवर्षमदीधरत्}


\twolineshloka
{अद्भुतं च महाराज तत्र चक्रे वृकोदरः}
{यत्पुत्रांस्तेऽवधीत्सङ्ख्ये द्रोणं चैव न्यवारयत्}


\twolineshloka
{पुत्रेषु तव वीरेषु चिक्रीडार्जुनपूर्वजः}
{मृगेष्विव महाराज चरन्व्याघ्रो महाबलः}


\twolineshloka
{यथा हि पशुमध्यस्थो द्रावयेत पशून्वृकः}
{वृकोदरस्तव सुतांस्तथा व्यद्रावयद्रणे}


\twolineshloka
{गाङ्गेयो भगदत्तश्च गोतमश्च महारथाः}
{पाण्डवं रभसं युद्धे वारयामासुरर्जुनम्}


% Check verse!
अस्त्रैरस्त्राणि संवार्य तेषां सोऽतिरथो रणेप्रवीरांस्तव सैन्येषु प्रेषयामास मृत्यवे
\twolineshloka
{अभिमन्युस्तु राजानमम्बष्ठं लोकविश्रुतम्}
{विरथं रथिनां श्रेष्ठं कारयामास सायकैः}


\twolineshloka
{विरथो वध्यमानस्तु सौभद्रेण यशस्विना}
{अवप्लुत्य रथात्तूर्णमम्बष्ठो वसुधाधिपः}


\twolineshloka
{असिं चिक्षेप समरे सौभद्रस्य महात्मनः}
{आरुरोह रथं चैव हार्दिक्यस्य महाबलः}


\twolineshloka
{आपतन्तं तु निस्त्रिंशं युद्धमार्गविशारदः}
{लाघवाद्व्यंसयामास सौभद्रः परवीरहा}


\twolineshloka
{व्यंसितं वीक्ष्य निस्त्रिंशं सौभद्रेण रणे तदा}
{साधुसाध्विति सैन्यानां प्रणादोऽभूद्विशांपते}


\twolineshloka
{धृष्टद्युम्नमुखास्त्वन्ये तव सैन्यमयोधयन्}
{तथैव तावकाः सर्वे पाण्डुसैन्यमयोधयन्}


\twolineshloka
{तत्राक्रन्दो महानासीत्तव तेषां च भारत}
{निघ्नतां दृढमन्योन्यं कुर्वतां कर्म दुष्करम्}


\twolineshloka
{अन्योन्यं हि रणे शूराः केशेष्वाक्षिष्य मानिनः}
{नखदन्तैरयुध्यन्त मुष्टिभिर्जानुभिस्तथा}


\twolineshloka
{तलैश्चैवाथ निस्त्रिंशैर्बाहुभिश्च सुसंस्थितैः}
{विवरं प्राप्य चान्योन्यमनयन्यमसादनम्}


\twolineshloka
{न्यहनच्च पिता पुत्रं पुत्रश्च पितरं तथा}
{व्याकुलीकृतसंकल्पा युयुधुस्तत्र मानवाः}


\twolineshloka
{रणे चारूणि चापानि हेमपृष्ठानि मारिष}
{हतानामपविद्धानि कलापाश्च महाधनाः}


\twolineshloka
{जातरूपमयैः पुङ्खै राजतैर्निशिताः शराः}
{तैलधौता व्यराजन्त निर्मुक्तभुजगोपमाः}


\twolineshloka
{खङ्गाश्च दान्तत्सरवो जातरूपपरिष्कृताः}
{चर्माणि चापविद्धानि रुक्मचित्राणि धन्विनां}


\twolineshloka
{सुवर्णाविकृताः प्रासाः प्रभग्ना हेमभूषिताः}
{जातरूपमया यष्ट्यः शक्त्यश्च कनकोज्ज्वलाः}


\twolineshloka
{अवस्कन्दाश्च पतिता मुसलानि गुरूणि च}
{परिघाः पट्टसाश्चैव भिण्डिपालाश्च मारिष}


\twolineshloka
{पतिता विविधाश्चापाश्चित्रा हेमपरिष्कृताः}
{कुथा बहुविधाकाराश्चामरा व्यजनानि च}


\twolineshloka
{नानाविधानि शस्त्राणि प्रगृह्य पतिता नराः}
{जीवन्त इव दृश्यनते गतसत्त्वा महारथाः}


\twolineshloka
{गजाविमथितैर्गात्रैर्मुसलैर्भिन्नमस्तकाः}
{गजावाजिरथक्षुण्णाः शेरते स्म नराः क्षितौ}


\twolineshloka
{तथैवाश्वनृनागानां शरीरैर्विबभौ तदा}
{संछन्ना वसुधा राजन्पर्वतैरिव शातितैः}


\twolineshloka
{समरे पतितैश्चैव शक्त्यृष्टिशरतोमरैः}
{निस्त्रिंशैः पट्टसैः प्रासैरवस्कन्दैः परश्वथैः}


\twolineshloka
{परिघैर्भिण्डिपालैश्च शतघ्नीभिश्च मारिष}
{शरीरैः शस्त्रनिर्भिन्नैः समास्तीर्यत मेदिनी}


\twolineshloka
{विशब्दैरल्पशब्दैश्च शोणितौगपरिप्लुतैः}
{गतासुभिरमित्रघ्न विबभौ निचिता मही}


\twolineshloka
{सतलत्रैः सकेयूरैर्बाहुभिश्चन्दनोक्षितैः}
{हस्तिहस्तोपमैश्छिन्नैरूरुभिश्च तरस्विनाम्}


\twolineshloka
{बद्धचूजामणिवरैः शिरोभिश्च सकुण्डलैः}
{पातितै ऋषभाक्षाणां बभौ भारत मेदिनी}


\twolineshloka
{कवचैः शोणितादिग्धैर्विप्रकीर्णैश्च काञ्चनैः}
{रराज सुभृशं भूमिः शान्तार्चिर्भिरिवानलैः}


\twolineshloka
{विप्रविद्धैः कलापैश्च पतितैश्च शरासनैः}
{विप्रकीर्णैः शरैश्चैव रुक्मपुङ्खैः समन्ततः}


\twolineshloka
{रथैश्च सर्वतो भग्नैः किङ्किणीजालभूषितैः}
{वाजिभिश्च हतैर्बाणैः स्रस्तजिह्वैः सशोणितैः}


\twolineshloka
{अनुकर्षैः पताकाभिरुपासङ्गैर्ध्वजैरपि}
{प्रवीराणां महाशङ्खैर्विप्रकीर्णैश्च पाण्डुरैः}


\twolineshloka
{स्रस्तहस्तैश्च मातङ्गैः शयानैर्विबभौ मही}
{नानारूपेरलंकारैः प्रमदेवाभ्यलङ्कृता}


\twolineshloka
{दन्तिभिश्चापरैस्तत्र सप्राणैर्गाढवेदनैः}
{करैः शब्दं विमुञ्चद्भिः शीकरं च मुहुर्मुहुः}


\twolineshloka
{विबभौ तद्रणस्थानं स्यन्दमानैरिवाचलैः}
{नानारागैः कम्बलैश्च परिस्तोमेश्च दन्तिनाम्}


\twolineshloka
{वैदूर्यमणिदण्डैश्च पतितैरङ्कुशैः शुभैः}
{घण्टाभिश्च गजेन्द्राणां पतिताभिः समन्ततः}


\twolineshloka
{विपाटितविचित्राभिः कुथाभिरङ्कुशैस्तथा}
{ग्रैवेयैश्चित्ररूपैश्च रुक्मकक्ष्याभिरेव च}


\twolineshloka
{यन्त्रैश्च बहुधा च्छिन्नैस्तोमरैश्चापि काञ्चनैः}
{`रराज सुभृशं भूमिस्तत्रतत्र विशांपते'अश्वानां रेणुकपिलै रुक्मच्छन्नैरुरश्छदैः}


\twolineshloka
{सादिनां च भुजैश्छिन्नैः पतितैः साङ्गदैस्तथा}
{प्रासैश्च विमलैस्तीक्ष्णैर्विमलाभिस्तथर्ष्टिभिः}


\twolineshloka
{उष्णीषैश्च तथा चित्रैर्विप्रविद्धैस्ततस्ततः}
{विचित्रैर्बाणवर्षैश्च जातरूपपरिष्कृतैः}


\twolineshloka
{अश्वास्तरपरिस्तोमै राङ्कवैर्मृदितैस्तथा}
{नरेन्द्रचूडामणिभिर्विचित्रैश्च महाधनैः}


\twolineshloka
{छत्रैस्तथापविद्धैश्च चामरैर्व्यजनैरपि}
{पद्मेन्दुद्युतिभिश्चैव वदनैश्चारुकुण्डलैः}


\twolineshloka
{क्लृप्तश्मश्रुभिरत्यर्थं वीराणां समलङ्कृतैः}
{अपविद्धैर्महाराज सुवर्णाज्ज्वलकुण्डलैः}


\twolineshloka
{ग्रहनक्षत्रशबला द्यौरिवासीद्वसुंधरा}
{एवमेते महासेने मृदिते तत्र भारत}


\twolineshloka
{परस्परं समासाद्य तव तेषां च संयुगे}
{तेषु श्रान्तेषु भग्नेषु मृदितेषु च भारत}


\twolineshloka
{रात्रिः समभवत्तत्र नापश्याम ततोऽनुगान्}
{ततोऽवहारं सैकन्यानां प्रचुक्रुः कुरुपाण्डवाः}


\threelineshloka
{घोरे निशामुखे रौद्रे वर्तमाने महाभये}
{अवहारं ततः कृत्वा सहिताः कुरुपाणड्वाः}
{न्यविशन्त निशाकाले गत्वा स्वशिबिरं तदा}


\chapter{अध्यायः ९७}
\twolineshloka
{सञ्जय उवाच}
{}


\twolineshloka
{ततो दुर्योधनो राजा शकुनिश्चापि सौबलः}
{दुःशासनश्च पुत्रस्ते सूतपुत्रश्च दुर्जयः}


\twolineshloka
{समागम्य महाराज मन्त्रं चक्रुर्विवक्षितम्}
{कथं पाण्डुसुताः सङ्ख्ये जेतव्याः सगणा इति}


\twolineshloka
{ततो दुर्योधनो राजा सर्वांस्तानाह मन्त्रिणः}
{सूतपुत्रं समाभाष्य सौबलं च महाबलम्}


\twolineshloka
{द्रोणो भीष्म कृपः शल्यः सौमदत्तिश्च संयुगे}
{न पार्थान्प्रति बाधन्ते न जाने किंनु कारणम्}


\twolineshloka
{अवध्यमानास्ते चापि क्षपयन्ति बलं मम}
{सोऽस्मि क्षीणबलः कर्ण क्षीणशस्त्रश्च संयुगे}


\twolineshloka
{`द्रोणस्य प्रमुखे वीरा हतास्ते भ्रातरो मम}
{भीमसेनेन राधेय मम चैव च पश्यतः '}


\threelineshloka
{निकृतः पाण्डवैः शूरैरवध्यैर्दैवतैरपि}
{सोऽहं संशयमापन्नः प्रकरिष्ये कथं रणम् ॥सञ्जय उवाच}
{}


\twolineshloka
{तमब्रवीन्महाराज सूतपुत्रो नराधिपम्}
{मा शोच भरतश्रेष्ठ करिष्येऽहं प्रियं तव}


\twolineshloka
{भीष्मः शान्तनवस्तूर्णमपयातु महारणात्}
{निवृत्ते युधि गाङ्गेये न्यस्तशस्त्रे च भारत}


\twolineshloka
{अहं पार्थान्हनिष्यामि सहितान्सर्वसोमकैः}
{पश्यतो युधि भीष्मस्य शपे सत्येन ते नृप}


\twolineshloka
{पाण्डवेषु दयां नित्यं स हि भीष्मः करोति वै}
{अशक्तश्च रणे भीष्मो जेतुमेतान्महारथान्}


\twolineshloka
{अभिमानी रणे भीष्मो नित्यं चापि रणप्रियः}
{स कथं पाण्डवान्युद्धे जेष्यते तात संगतान्}


\twolineshloka
{स त्वं शीघ्रमितो गत्वा भीष्मस्य शिबिरं प्रति}
{अनुमान्य गुरुं वृद्धं शस्त्रं न्यासय भारत}


\twolineshloka
{न्यस्तशस्त्रे ततो भीष्मे निहतान्पश्य पाण्डवान्}
{मयैकेन रणे राजन्ससुहृद्गणबान्धवान्}


\twolineshloka
{एवमुक्तस्तु कर्णेन पुत्रो दुर्योधनस्तव}
{अब्रवीद्भ्रातरं तत्र दुःशासनमिदं वचः}


\twolineshloka
{अनुयात्रं यथा सर्वं सज्जीभवति सर्वशः}
{दुःशासन तथा क्षिप्रं सर्वमेवोपपादय}


\twolineshloka
{एवमुक्त्वा ततो राजन्कर्णमाह जनेश्वरः}
{अनुमान्य रणे भीष्ममेषोऽहं द्विपदां वरम्}


\twolineshloka
{आगमिष्ये ततः क्षिप्रं त्वत्सकाशमरिन्दम}
{अपक्रान्ते ततो भीष्मे प्रहरिष्यसि संयुगे}


\twolineshloka
{निष्पपात ततस्तूर्णं पुत्रस्तव विशांपते}
{सहितो भ्रातृभिस्तैस्तु देवैरिव शतक्रतुः}


\twolineshloka
{ततस्तं नृपशार्दूलं शार्दूलसमविक्रमम्}
{आरोपयद्धयं तूर्णं भ्राता दुःशासनस्तदा}


\twolineshloka
{अङ्गदी बद्धमुकुटो हस्ताभरणवान्नृप}
{धार्तराष्ट्रो महाराज विबभौ स पथि व्रजन्}


\twolineshloka
{भण्डीपुष्पनिकाशेन तपनीयनिभेन च}
{अनुलिप्तः परार्द्ध्येन चन्दनेन सुगन्धिना}


\twolineshloka
{अरजोम्बरसंवीतः सिंहखेलगतिर्नृप}
{शुशुभे विमलार्चिष्मान्नभसीव दिवाकरः}


\threelineshloka
{तं प्रयान्तं नरव्याघ्रं भीष्मस्य शिबिरं प्रति}
{अनुजग्मुर्महेष्वासाः सर्वलोकस्य धन्विनः}
{भ्रातरश्च महेष्वासास्त्रिदशा इव वासवम्}


\twolineshloka
{हयानन्ये समारुह्य गजानन्ये च भारत}
{रथानन्ये नरश्रेष्ठं परिवव्रुः समन्ततः}


\twolineshloka
{`पदातयश्च त्वरिता नखरप्रासयोधिनः}
{परिवव्रुर्महेष्वासं धार्तराष्ट्रं महारथम् ॥'}


\twolineshloka
{आत्तशस्त्राश्च सुहृदो रक्षणार्थं महीपतेः}
{प्रादुर्बभूवुः सहिताः शक्रस्येवामरा दिवि}


\threelineshloka
{स पूज्यमानः कुरुभिः कौरवाणां महाबलः}
{प्रययौ सदनं राजा गाङ्गेयस्य यशस्विनः}
{अन्वीयमानः सततं सोदरैः परिवारितः}


\twolineshloka
{दक्षिणं दक्षिणः काले संभृत्य स्वभुजं तदा}
{हस्तिहस्तोपमं सौम्यं सर्वशत्रुनिबर्हणम्}


\twolineshloka
{प्रगृह्णन्नञ्जलीन्नॄणामुद्यतान्सर्वतो दिशः}
{शुश्राव मधुरा वाचो नानादेशनिवासिनाम्}


\twolineshloka
{संस्तूयमानः सूतैश्च मागधैश्च महायशाः}
{पूजयानश्च तान्सर्वान्सर्वलोकेश्वरेश्वरः}


\twolineshloka
{प्रदीपैः काञ्चनैस्तत्र गन्धतैलावसेचितैः}
{परिवव्रुर्महाराजं प्रज्वलद्भिः समन्ततः}


\twolineshloka
{स तैः परिवृतो राजा प्रदीपैः काञ्चनैर्ज्वलन्}
{शुशुभे चन्द्रमायुक्तो दीप्तैरिव महाग्रहैः}


\twolineshloka
{कञ्चुकोष्णीषिणस्तत्र वेत्रझर्झरपाणयः}
{प्रोत्सारयन्तः शनकैस्तं जनं सर्वतो दिशम्}


\twolineshloka
{संप्राप्य तु ततो राजा भीष्मस्य सदनं शुभम्}
{अवतीर्य हयाच्चापि भीष्मं प्राप्य जनेश्वरः}


\threelineshloka
{अभिवाद्य ततो भीष्मं निषण्णः परमासने}
{काञ्चने सर्वतोभद्रे स्पर्द्ध्यास्तरणसंवृते}
{उवाच प्राञ्जलिर्भीष्मं बाष्पकण्ठोऽश्रुलोचनः}


\threelineshloka
{त्वां वयं हि समाश्रित्य संयुगे शत्रुसूदन}
{उत्सहेम रणे जेतुं सेन्द्रानपि सुरासुरान्}
{किमु पाण्डुसुतान्वीरान्ससुहृद्गणबान्धवान्}


\twolineshloka
{तस्मादर्हसि गाङ्गेय कृपां कर्तुं मयि प्रभो}
{जहि पाणडुसुतान्वारान्महेन्द्र इव दानवान्}


\twolineshloka
{पूर्वमुक्तं महाबाहो हनिष्यामि ससोमकान्}
{पाञ्चालान्केकयैः सार्धं करूपांश्चेति भारत}


\twolineshloka
{त्वद्वचः सत्यमेवास्तु जहि पार्थान्समागतान्}
{सोमकांश्च महेष्वासान्सत्यवाग्भव भारत}


\twolineshloka
{दयया यदि वा राजन्द्वेष्यभावान्मम प्रभो}
{मन्दभाग्यतया वापि मम रक्षसि पाण्डवान्}


\threelineshloka
{अनुजानीहि समरे कर्णमाहवशोभिनम्}
{स जेष्यति रणे पार्थान्ससुहृद्गणबान्धवान् ॥सञ्जय उवाच}
{}


\twolineshloka
{स एवमुक्त्वा नृपतिः पुत्रो दुर्योधनस्तव}
{नोवाच वचनं किंचिद्भीष्मं सत्यपराक्रमम्}


\chapter{अध्यायः ९८}
\twolineshloka
{सञ्जय उवाच}
{}


\twolineshloka
{वाक्शल्यैस्तव पुत्रेण सोऽतिविद्धो महामनाः}
{नोवाच दुःखोपहतो ह्यप्रियं प्रियमण्वपि}


\twolineshloka
{स ध्यात्वा सुचिरं कालं दुःखरोषसमन्वितः}
{श्वसन्निव महानागः प्रणुन्नो वाक्शलाकया}


\twolineshloka
{उद्वृत्य चक्षुषी लोपान्निर्दहन्निव भारत}
{सदेवासुरगन्धर्वं लोकं काल इवापरः}


\twolineshloka
{अब्रवीत्तव पुत्रं च सामपूर्वमिदं वचः}
{किं त्वं दुर्योधनैवं मां वाक्शल्यैरपकृन्तसि}


\twolineshloka
{घटमानं यथाशक्तिं कुर्वाणं च तव प्रियम्}
{जुह्वानं समरे प्राणांस्तव वै प्रियकाम्यया}


\twolineshloka
{यदा तु पाण्डवः शूरः खाण्डवेऽग्निमतर्पयत्}
{पराजित्य रणे शक्रं पर्याप्तं तन्निदर्शनम्}


\twolineshloka
{यदा च त्वां महाबाहो गन्धर्वैर्हृतमोजसा}
{अमोचयत्पाण्डुसुतः पर्याप्तं तन्निदर्शनम्}


\twolineshloka
{द्रवमाणेषु शूरेषु सोदरेषु तव प्रभो}
{सूतपुत्रे च राधेये पर्याप्तं तन्निदर्शनम्}


\twolineshloka
{यच्च नः सहितान्सर्वान्विराटनगरे तदा}
{एक एवाजयत्पार्थः पर्याप्तं तन्निदर्शनम्}


\twolineshloka
{द्रोणं च युधि संरब्धं मां च निर्जित्य संयुगे}
{वासांसि स समादत्त पर्याप्तं तन्निदर्शनम्}


\twolineshloka
{तथा द्रौणिं महेष्वासं शारद्वतमथापि च}
{गोग्रहे जितवान्पूर्वं पर्याप्तं तन्निदर्शनम्}


\twolineshloka
{विजित्य च यदा कर्णं सदा पुरुषमानिनम्}
{उत्तरायै ददौ वस्त्रं पर्याप्तं तन्निदर्शनम्}


\twolineshloka
{निवातकवचान्युद्धे वासवेनापि दुर्जयान्}
{जितवान्समरे पार्थः पर्याप्तं तन्निदर्शनम्}


\twolineshloka
{को हि शक्तो रणे जेतुं पाण्डवं रभसं तदा}
{यस्य गोप्ता जगद्गोप्ता शङ्खचक्रगदाधरः}


\twolineshloka
{वासुदेवोऽनन्तशक्तिः सृष्टिसंहारकारकः}
{सर्वेश्वरो देवदेवः परमात्मा सनातनः}


\twolineshloka
{उक्तोऽस्ति बहुशो राजन्नारदाद्यैर्महर्षिभिः}
{त्वं तु मोहान्न जानीषे वाच्यावाच्यं सुयोधन}


\twolineshloka
{मुमूर्षुर्हि नरः सर्वान्वृक्षान्पश्यति काञ्चनान्}
{तथा त्वमपि गान्धारे विपरीतानि पश्यसि}


\threelineshloka
{स्वयं वैरं महत्कृत्वा पाण्डवैः सह सृञ्जयैः}
{युद्ध्यस्व तानद्य रणे पश्यामः पुरुषो भव}
{`अशक्याः पाण्डवा जेतुं देवैरपि सवासवैः ॥'}


\twolineshloka
{अहं तु सोमकान्सर्वान्पाञ्चालांश्च समागतान्}
{निहनिष्ये नरव्याघ्र वर्जयित्वा शिखण्डिनम्}


\twolineshloka
{तैर्वाऽहं निहतः सङ्ख्ये गमिष्ये यमसादनम्}
{तान्वा निहत्य समरे प्रीतिं दास्याम्यहं तव}


\twolineshloka
{पूर्व हि स्त्री समुत्पन्ना शिखण्डी राजवेश्मनि}
{वरदानात्पुमाञ्जातः सैषा वे स्त्री शिखण्डिनी}


\twolineshloka
{तमहं न हनिष्यामि प्राणत्यागेऽपि भारत}
{याऽसौ प्राङ्वर्मिता धात्रा सैषा वै स्त्री शिखण्डिनी}


\threelineshloka
{सुखं स्वपिहि गान्धारे श्वोऽस्मि कर्ता महारणम्}
{यं जनाः कथयिष्यन्ति यावत्स्थास्यति मेदिनी ॥सञ्जय उवाच}
{}


\twolineshloka
{एवमुक्तस्तव सुतो निर्जगाम जनेश्वर}
{अभिवाद्य गुरुं मूर्ध्ना प्रययौ स्वं निवेशनम्}


\twolineshloka
{आगम्य तु ततो राजा विसृज्य च महाजनम्}
{प्रविवेश ततस्तूर्णं क्षयं शत्रुक्षयंकरः}


\twolineshloka
{प्रहृष्टः स निशां तां च गमयामास पार्थिवः}
{प्रभातायां च शर्वर्यां प्रातरुत्थाय तान्नृपः}


\twolineshloka
{राज्ञः समाज्ञापयत सेनां योजयतेति ह}
{अद्य भीष्मो रणे क्रद्धो निहनिष्यति सोमकान्}


\twolineshloka
{दुर्योधनस्य तच्छ्रुत्वा रात्रौ विलपितं बहु}
{मन्यमानः स तं राजन्प्रत्यादेशमिवात्मनः}


\twolineshloka
{निर्वेदं परमं गत्वा विनिन्द्य परवश्यताम्}
{दीर्घं दध्यौ शान्तनवो योद्धुकामोऽर्जुनं रणे}


\twolineshloka
{इङ्गितेन तु तज्ज्ञात्वा गाङ्गेयेन विचिन्तितम्}
{दुर्योधनो महाराज दुःशासनमचोदयत्}


\twolineshloka
{दुःशासन रथास्तूर्णं युज्यन्तां भीष्मरक्षिणः}
{द्वाविंशतिमनीकानि सर्वाण्येवाभिचोदय}


\twolineshloka
{अयं हि समनुप्राप्तो वर्षपूगाभिचिन्तितः}
{पाण्डवानां ससैन्यानां वधो राज्यस्य चागमः}


\twolineshloka
{तत्र कार्यमहं मन्ये भीष्मस्यैवाभिरक्षणम्}
{स नो गुप्तः सहायः स्याद्धन्यात्पार्थांश्च संयुगे}


\twolineshloka
{अब्रवीद्धि विशुद्धात्मा नाहं हन्यां शिखण्डिनम्}
{स्त्रीपूर्वको ह्यसौ राजंस्तस्माद्वर्ज्यो मया रणे}


\twolineshloka
{लोकस्तद्वेद यदहं पितुः प्रियचिकीर्षया}
{राज्यं स्फीतं महाबाहो स्त्रियश्च त्यक्तवान्पुरा}


\twolineshloka
{नैव चाहं स्त्रियं जातु न स्त्रीपूर्वं कथंचन}
{हन्यां युधि नरश्रेष्ठ सत्यमेतद्ब्रवीमि ते}


\twolineshloka
{अयं स्त्रीपूर्वको राजञ्छिखण्डी यदि ते श्रुतः}
{उद्योगे कथितं सर्वं यथा जाता शिखण्डिनी}


\twolineshloka
{कन्या भूत्वा पुमाञ्जातः स च योत्स्यति भारत}
{तस्याहं प्रमुखे बाणान्न मुञ्चेयं कथंचन}


\twolineshloka
{युद्धे हि क्षत्रियांस्तात पाण्डवानां जयैषिणः}
{सर्वानन्यान्हनिष्यामि संप्राप्तान्रणमूर्धनि}


\twolineshloka
{एवं मां भरतश्रेष्ठ गाङ्गेयः प्राह शास्त्रवित्}
{तत्र सर्वात्मना मन्ये गाङ्गेयस्यैव पालनम्}


\twolineshloka
{अरक्ष्यमाणं हि वृको हन्यात्सिंहं महाहवे}
{मा वृकेणेव गाङ्गेयं घातयेम शिखण्डिना}


\twolineshloka
{मातुलः शकुनिः शल्यः कृपो द्रोणो विविंशतिः}
{यत्ता रक्षन्तु गाङ्गेयं तस्मिन्गुप्ते ध्रुवो जयः}


\twolineshloka
{एतच्छ्रुत्वा तु ते सर्वे दुर्योधनवचस्तदा}
{सर्वतो रथवंशेन गाङ्गेयं पर्यवारयन्}


\twolineshloka
{पुत्राश्च तव गाङ्गेयं परिवार्य ययुर्मुदा}
{कम्पयन्तो भुवं द्यां च क्षोभयन्तश्च पाण्डवान्}


\twolineshloka
{ते रथैः सुप्रसंयुक्तैर्दन्तिभिश्च महारथाः}
{परिवार्य रणे भीष्मं दंशिताः समवस्थिताः}


\twolineshloka
{यथा देवासुरे युद्धे त्रिदशा वज्रधारिणम्}
{सर्वे ते स्म व्यतिष्ठन्त रक्षन्तस्तं महारथम्}


\threelineshloka
{ततो दुर्योधनो राजा पुनर्भ्रातरमब्रवीत्}
{सव्यं चक्रं युधामन्युरुत्तमौजाश्च दक्षिणम्}
{गोप्तारावर्जुनस्यैतावर्जुनोऽपि शिखण्डिनः}


\threelineshloka
{रक्ष्यमाणः स पार्थेन तथास्माभिर्विवर्जितः}
{यथा भीष्मं न नो हन्याद्दुःशासन तथा कुरु ॥सञ्जय उवाच}
{}


\twolineshloka
{भ्रातुस्तद्वचनं श्रुत्वा पुत्रो दुःशासनस्तव}
{भीष्मं प्रमुखतः कृत्वा प्रययौ सह सेनया}


\twolineshloka
{भीष्मं तु रथवंशेन दृष्ट्वा समभिसंवृतम्}
{अर्जुनो रथिनां श्रेष्ठो धृष्टद्युम्नमुवाच ह}


\twolineshloka
{शिखण्डिनं नरव्याघ्रं भीष्मस्य प्रमुखे नृप}
{स्थापयस्वाद्य पाञ्चाल्य तस्य गोप्ताऽहमित्युत}


\chapter{अध्यायः ९९}
\twolineshloka
{सञ्जय उवाच}
{}


\twolineshloka
{ततः शान्तनवो भीष्मो निर्ययौ सह सेनया}
{व्यूहं चाव्यूहत महत्सर्वतोभद्रमात्मनः}


\twolineshloka
{कृपश्च कृतवर्मा च शैब्यश्चैव महारथः}
{शकुनिः सैन्धवश्चैव काम्भोजश्च सुदक्षिणः}


\twolineshloka
{भीष्मेण सहिताः सर्वे पुत्रैश्च तव भारत}
{अग्रतः सर्वसैन्यानां व्यूहस्य प्रमुखे स्थिताः}


\twolineshloka
{द्रोणो भूरिश्रवाः शल्यो भगदत्तश्च मारिष}
{दक्षिणं पक्षमाश्रित्य स्थिता व्यूहस्य दंशिताः}


\twolineshloka
{अश्वत्थामा सोमदत्तश्चावन्त्यौ च महारथौ}
{महत्या सेनया युक्ता वामं पक्षमपालयन्}


\twolineshloka
{दुर्योधनो महाराज त्रिगर्तैः सर्वतो वृतः}
{व्यूहमध्ये स्थितो राजन्पाण्डवान्प्रति भारत}


\twolineshloka
{अलम्बुसो रथश्रेष्ठः श्रुतायुश्च महारथः}
{पृष्ठतः सर्वसैन्यानां स्थितौ व्यूहस्य दंशितौ}


\twolineshloka
{एवं च तं तदा व्यूहं कृत्वा भारत तावकाः}
{सन्नद्धाः समदृश्यन्त प्रतपन्त इवाग्नयः}


\twolineshloka
{ततो युधिष्ठिरो राजा भीमसेनश्च पाण्डवः}
{नकुलः सहदेवश्च माद्रीपुत्रावुभावपि}


\twolineshloka
{अग्रतः सर्वसैन्यानां स्थिता व्यूहस्य दंशिताः}
{धृष्टद्युम्नो विराटश्च सात्यकिश्च महारथः}


\twolineshloka
{स्थिताः सैन्येन महता परानीकविनाशनाः}
{शिखण्डी विजयश्चैव राक्षसश्च घटोत्कचः}


\twolineshloka
{चेकितानो महाबाहुः कुन्तिभोजश्च वीर्यवान्}
{स्थिता रणे महाराज महत्या सेनया वृताः}


\twolineshloka
{अभिमन्युर्महेष्वासो द्रुपदश्च महाबलः}
{युयुधानो महेष्वासो युधामन्युश्च वीर्यवान्}


\twolineshloka
{केकया भ्रातरश्चैव स्थिता युद्धाय दंशिताः}
{एवं तेऽपि महाव्यूहं प्रतिव्यूह्य सुदुर्जयम्}


\twolineshloka
{पाण्डवाः समरे शूराः स्थिता युद्धाय दंशिताः}
{तावकास्तु रणे यत्ताः सहसेना नराधिपाः}


\twolineshloka
{अभ्युद्ययू रणे पार्थान्भीष्मं कृत्वाऽग्रतो नृप}
{तथैव पाण्डवा राजन्भीमसेनपरोगमाः}


\twolineshloka
{भीष्मं योद्धुमभीप्सन्तः संग्रामे विजयैषिणः}
{क्ष्वेलाः किलकिलाः शङ्खान्क्रकचान्गोविषाणिकाः}


\twolineshloka
{भेरीमृदङ्गपणवान्नादयन्तश्च पुष्करान्}
{पाण्डवा अभ्यवर्तन्त नदन्तो भैरवान्रवान्}


\twolineshloka
{भेरीमृद्गशङ्खानां दुन्दुभीनां च निःस्वनैः}
{उत्कृष्टसिंहनादैश्च वल्गितैश्च पृथग्विधैः}


\twolineshloka
{वयं प्रतिनदन्तस्तानगच्छाम त्वरान्विताः}
{सहसैवाभिसंक्रद्धास्तदासीत्तुमुलं महत्}


\twolineshloka
{ततोऽन्योन्यं प्रधावन्तः संप्रहारं प्रचक्रिरे}
{ततः शब्देन महता प्रचकम्पे वसुंधरा}


\twolineshloka
{पक्षिणश्च महाघोरं व्याहरन्तो विबभ्रमुः}
{सप्रभश्चोदितः सूर्यो निष्प्रभः समपद्यत}


\twolineshloka
{ववुश्च वातास्तुमुलाः शंसन्तः सुमहद्भयम्}
{घोराश्च घोरनिर्ह्रादाः शिवास्तत्र ववाशिरे}


\twolineshloka
{वेदयन्त्यो महाराज महद्वैशसमागतम्}
{दिशः प्रज्वलिता राजन्पांसुवर्षं पपात च}


\twolineshloka
{रुधिरेण समुन्मिश्रमस्थिवर्षं पपात च}
{रुदतां वाहनानां च नेत्रेभ्यः प्रापतञ्जलम्}


\twolineshloka
{सुस्रुवुश्च शकृन्मूत्रं प्रध्यायन्तो विशांपते}
{अन्तर्हिता महानादाः श्रूयन्ते भरतर्षभ}


\twolineshloka
{रक्षसां पुरुषादानां नदतां भैरवान्रवान्}
{संपतन्तश्च दृश्यन्ते गोमायबलवायसाः}


\threelineshloka
{श्वानश्च विविधैर्नादैर्भषन्तस्तत्र मारिष}
{ज्वलिताश्च महोल्का वै समाहत्य दिवाकरम्}
{निपेतुः सहसा भूमौ वेदयन्त्यो महद्भयम्}


\twolineshloka
{महान्त्यनीकानि महासमुच्छ्रयेततस्तयोः पाण्डवधार्तराष्ट्रयोः}
{चकम्पिरे शङ्खमृदङ्गनिःस्वनैःप्रकम्पितानीव वनानि वायुना}


\twolineshloka
{नरेन्द्रनागाश्वसमाकुलाना-मभ्यायतीनामशिवे मुहूर्ते}
{बभूव घोषस्तुमुलश्चमूनांवातोद्धुतानामिव सागराणाम्}


\chapter{अध्यायः १००}
\twolineshloka
{सञ्जय उवाच}
{}


\twolineshloka
{अभिमन्यू रथोदारः पिशङ्गैस्तुरगोत्तमैः}
{अभिदुद्राव तेजस्वी दुर्योधनबलं महत्}


\twolineshloka
{विकिरञ्शरवर्षाणि वारिधारा इवाम्बुदः}
{न शेकुः समरे क्रुद्धं सौभद्रमरिसूदनम्}


\twolineshloka
{शस्त्रौघिणं गाहमानं सेनासागरमक्षयम्}
{निवारयितुमप्याजौ त्वदीयाः कुरुनन्दन}


\twolineshloka
{तेन मुक्ता रणे राजञ्शराः शत्रुनिबर्हणाः}
{क्षत्रियाननयञ्शूरान्प्रेतराजनिवेशनम्}


\twolineshloka
{यमदण्डोपमान्घोराञ्ज्वलिताशीविषोपमान्}
{सौभद्रः समरे क्रुद्धः प्रेषयामास सायकान्}


\twolineshloka
{सरथान्रथिनस्तूर्णं हयांश्चैव ससादिनः}
{गजारोहांश्च सगजान्दारयामास फाल्गुनिः}


\twolineshloka
{तस्य तत्कुर्वतः कर्म महत्सङ्ख्ये महीभृतः}
{पूजयांचक्रिरे हृष्टाः प्रशशंसुश्च फाल्गुनिम्}


\twolineshloka
{तान्यनीकानि सौभद्रो द्रावयामास भारत}
{तूलराशीनिवाकाशे मारुतः सर्वतो दिशम्}


\twolineshloka
{तेन विद्राव्यमाणानि तव सैन्यानि भारत}
{त्रतारं नाध्यगच्छन्त पङ्क्ते मग्ना इव द्वीपाः}


\twolineshloka
{विद्राव्य सर्वसैन्यानि तावकानि नरोत्तम}
{अभिमन्युः स्थितो राजन्विधूमोऽग्निरिव ज्वलन्}


\twolineshloka
{न चैनं तावका राजन्विषेहुररिघातिनम्}
{प्रदीप्तं पावकं यद्वत्पतङ्गाः कालचोदिताः}


\twolineshloka
{प्रहरन्सर्वशत्रुभ्यः पाण्डवानां महारथः}
{अदृश्यत महेष्वासः सवज्र इव वासवः}


\twolineshloka
{हेमपृष्ठं धनुश्चास्य ददृशे विचरद्दिशः}
{तोयदेषु यथा राजन्राजमाना शतह्रदा}


\twolineshloka
{शराश्च निशिताः पीता निश्चरन्ति स्म संयुगे}
{वनात्फुल्लद्रुमाद्राजन्भ्रमराणामिव व्रजाः}


\twolineshloka
{तथैव चरतस्तस्य सौभद्रस्य महात्मनः}
{रथेन काञ्चनाङ्गेन ददृशुर्नान्तरं जनाः}


\twolineshloka
{मोहयित्वा कृपं द्रोणं द्रौणिं च सबृहद्बलम्}
{सैन्धवं च महेष्वासो व्यचरल्लघु सुष्ठु च}


\twolineshloka
{मण्डलीकृतमेवास्य धनुः पश्याम भारत}
{सूर्यमण्डलसंकाशं दहतस्तव वाहिनीम्}


\twolineshloka
{तं दृष्ट्वा क्षत्रियाः शुराः प्रतपन्तं तरस्विनम्}
{द्विफल्गुनमिमं लोकं मेनिरे तस्य कर्मभिः}


\twolineshloka
{तेनार्दिता महाराज भारती सा महाचमूः}
{व्यभ्रमत्तत्रतत्रैव योषिन्मदवशादिव}


\twolineshloka
{द्रावयित्वा महासैन्यं कम्पयित्वा महारथान्}
{नन्दयामास सुहृदो मयं जित्वेव वासवः}


\twolineshloka
{तेन निद्राव्यमाणानि तव सैन्यानि संयुगे}
{चक्रुरार्तस्वनं घोरं पर्जन्यनिनदोपमम्}


\twolineshloka
{तं श्रुत्वा निनदं घोरं तव सैन्यस्य भारत}
{मारुतोद्धूतवेगस्य सागरस्येव पर्वणि}


\twolineshloka
{दुर्योधनस्तदा राजन्नार्श्यशृङ्गिमभाषत}
{एष कार्ष्णिर्महाबाहो द्वितीय इव फल्गुनः}


\twolineshloka
{चमूं द्रावयते क्रोधाद्वृत्रो देवचमूमिव}
{तस्य चान्यत्र पश्यामि संयुगे भेषजं महत्}


\twolineshloka
{ऋते त्वां राक्षसश्रेष्ठं सर्वविद्यासु पारगम्}
{स गत्वा त्वरितं वीरं जहि सौभद्रमाहवे}


\twolineshloka
{वयं पार्थं हनिष्यामो भीष्मद्रोणपुरोगमाः}
{स एवमुक्तो बलवान्राक्षसेन्द्रः प्रतापवान्}


\twolineshloka
{प्रययौ समरे तूर्णं तव पुत्रस्य शासनात्}
{नर्दमानो महानादं प्रावृषीव बलाहकः}


\twolineshloka
{तस्य शब्देन महता पाण्डवानां बलं महत्}
{प्राचलत्सर्वतो राजन्वातोद्धूत इवार्णवः}


\twolineshloka
{बहवश्च महाराज तस्य नादेन भीषिताः}
{प्रियान्प्राणान्परित्यज्य निपेतुर्धरणीतले}


\twolineshloka
{कार्ष्णिश्चापि मुदा युक्तः प्रगृह्य सशरं धनुः}
{नृत्यन्निव रथोपस्थे तद्रक्षः समुपाद्रवत्}


\twolineshloka
{ततः स राक्षसः क्रुद्धः संप्राप्यैवार्जुनं रणे}
{नातिदूरे स्थितां तस्य द्रावयामास वै चमूम्}


\twolineshloka
{तां वध्यमानां च तथा पाण्डवानां महाचमूम्}
{प्रत्यद्ययौ रणे रक्षो देवसेनां यथा बलः}


\twolineshloka
{विमर्दः सुमहानासीत्तस्य सैन्यस्य मारिष}
{रक्षसा घोररूपेण वध्यमानस्य संयुगे}


\twolineshloka
{ततः शरसहस्त्रैस्तां पाण्डवानां महाचमूम्}
{व्यद्रावयद्रणे रक्षो दर्शयतत्स्वपराक्रमम्}


\twolineshloka
{सा वध्यमाना च तथा पाण्डवानामनीकिनी}
{रक्षसा घोररूपेण प्रदुद्राव रणे भयात्}


\twolineshloka
{प्रमृद्य च रणे सेनां पद्मिनीं वारणो यथा}
{ततोऽभिद्रद्राव रणे द्रौपदेयान्महाबलान्}


\twolineshloka
{ते तु क्रुद्धा महेष्वासा द्रौपदेयाः प्रहारिणः}
{राक्षसं दुद्रुवुः सङ्ख्ये ग्रहाः पञ्च रविं यथा}


\twolineshloka
{वीर्यवद्भिस्ततस्तैस्तु पीडितो राक्षसोत्तमः}
{यथा युगक्षये घोरे चन्द्रमाः पञ्चभिर्ग्रहैः}


\twolineshloka
{प्रतिविन्ध्यस्ततो रक्षो बिभेद निशितैः शरैः}
{सर्वपारशवैस्तूर्णमकुण्ठाग्रैर्महाबलः}


\twolineshloka
{स तैर्भिन्नतनुत्राणः शुशुभे राक्षसोत्तमः}
{मरीचिभिरिवार्कस्य संस्यूतो जलदो महान्}


\twolineshloka
{विषक्तैः सशरैश्चापि तपनीयपरिच्छदैः}
{आर्श्यशृङ्गिर्बभौ राजन्दीप्तशृङ्ग इवाचलः}


\threelineshloka
{ततस्ते भ्रातरः पञ्च राक्षसेन्द्रं महाहवे}
{विव्यधुर्निशितैर्बाणैस्तपनीयविभूषितैः}
{}


\twolineshloka
{स निर्भिन्नः शरैर्घोरैर्भुजगैः कोपितैरिव}
{अलम्बुसो भृशं राजन्नागेन्द्र इव चुक्रुधे}


\twolineshloka
{सोऽतिविद्धो महाराज मुहूर्तमथ मारिष}
{प्रविवेश तमो दीर्घं पीडितस्तैर्महारथैः}


\twolineshloka
{प्रतिलभ्य ततः संज्ञां क्रोधेन द्विगुणीकृतः}
{चिच्छेद सायकैस्तेषां ध्वजांश्चैव धनूंषि च}


\twolineshloka
{एकैकं पञ्चभिर्बाणैराजघान स्मयन्निव}
{अलम्बुसो रथोपस्थे नृत्यन्निव महारथः}


\twolineshloka
{त्वरमाणः सुसंबद्धो हयांस्तेषां महात्मनाम्}
{जघान राक्षसः क्रुद्धः सारथींश्च सहस्रशः}


\twolineshloka
{बिभेद च सुसंरब्धः पुनश्चैनान्सुतांशितैः}
{शरैर्बहुविधाकारैः शतशोऽथ सहस्रशः}


\twolineshloka
{विरथांश्च महेष्वासान्कृत्वा तत्र स राक्षसः}
{अभिदुद्राव वेगेन हन्तुकामो निशाचरः}


\twolineshloka
{तानर्दितान्रणे तेन राक्षसेन दुरात्मना}
{दृष्ट्वाऽर्जुनसुतः सङ्ख्ये राक्षसं समुपाद्रवत्}


\twolineshloka
{तयोः समभवद्युद्धं वृत्रवासवयोरिव}
{ददृशुस्तावकाः सर्वे पाण्डवाश्च परस्परम्}


\twolineshloka
{तौ समेतौ महायुद्धे क्रोधदीप्तौ परस्परम्}
{महाबलौ महाराज क्रोधसंरक्तलोचनौ}


\threelineshloka
{परस्परमवेक्षेतां कालानलसमौ युधि}
{`आशीविषाविव क्रुद्धौ नेत्राभ्यामितरेतरम्}
{'तयोः समागमो घोरो बभूव कटुकोदयः}


% Check verse!
यथा देवासुरे युद्धे शक्रशम्बरयोः पुरा
\chapter{अध्यायः १०१}
\twolineshloka
{धृतराष्ट्र उवाच}
{}


\twolineshloka
{आर्जुनं समरे शूरं विनिघ्नन्तं महारथान्}
{अलम्बुसः कथं युद्धे प्रत्ययुध्यत सञ्जय}


\twolineshloka
{आर्श्यशृङ्गिं कथं चैव सौभद्रः परवीरहा}
{तन्ममाचक्ष्व तत्त्वेन यथावृत्तं स्म संयुगे}


\twolineshloka
{धनंजयश्च किं चक्रे मम सैन्येषु संयुगे}
{भीमो वा रथिनां श्रेष्ठो राक्षसो वा घटोत्कचः}


\threelineshloka
{नकुलः सहदेवो वा सात्यकिर्वा महारथः}
{एतदाचक्ष्व मे सत्यं कुशलो ह्यसि सञ्जय ॥सञ्जय उवाच}
{}


\twolineshloka
{हन्ते तेऽहं प्रवक्ष्यामि संग्रामं रोमहर्षणम्}
{यथाऽभूद्राक्षसेन्द्रस्य सौभद्रस्य च मारिष}


\twolineshloka
{अर्जुनश्च यथा सङ्ख्ये भीमसेनश्च पाण्डवः}
{नकुलः सहदेवश्च रणे चक्रुः पराक्रमम्}


\twolineshloka
{तथैव तावकाः सर्वे भीष्मद्रोणपुरःसराः}
{अद्भुतानि विचिन्राणि चक्रुः कर्माण्यभीतवत्}


\twolineshloka
{अलम्बुसस्तु समरे अभिमन्युं महारथम्}
{विनद्य सुमहानादं तर्जयित्वा मुहुर्मुहुः}


\twolineshloka
{अभिदुद्राव वेगेन तिष्ठतिष्ठेति चाब्रवीत्}
{अभिमन्युश्च वेगेन सिंहवद्विनदन्मुहुः}


\twolineshloka
{आर्श्यशृङ्गिं महेष्वासि पितुरत्यन्तवैरिणम्}
{ततः समीपतुः सङ्ख्ये त्वरितौ नरराक्षसौ}


\twolineshloka
{रथाभ्यां रथिनौ श्रेष्ठौ यथा वै देवदानवौ}
{मायावा राक्षसश्रेष्ठो दिव्यास्त्रश्चैव फाल्गुनिः}


\twolineshloka
{ततः कर्ष्णिर्महाराज निशितैः सायकैस्त्रिभिः}
{आर्श्यशृङ्गिं रणे विद्ध्वा पुनर्विव्याव पञ्चभिः}


\twolineshloka
{अलम्बुरोऽपि संक्रुद्धः कार्ष्णि नवभिराशुगैः}
{हृदि विव्याध वेगेन तोत्रैरिव महाद्विपम्}


\twolineshloka
{ततः शरसहस्रेण क्षिप्रकारी निशाचरः}
{अर्जुनास सुतं सङ्ख्ये पीडयामास भारत}


\twolineshloka
{अभिमन्युस्ततः क्रुद्धो नवभिर्नतपर्वभिः}
{बिभेद निशिसैर्बाणै राक्षसेन्द्रं महोरसि}


\twolineshloka
{ते तस्य विविशुस्तूर्णं कायं निर्भिद्य मर्मसु}
{स तैर्विभिन्नसर्वाङ्गः शुशुभे राक्षसोत्तमः}


\twolineshloka
{पुष्पितैः किंशुकै राजन्संस्तीर्ण इव पर्वतः}
{स संदधानश्च शरान्हेमपुङ्खान्महाबलः}


\twolineshloka
{विबभौ राक्षसश्रेष्ठः सज्वाल इव पर्वतः}
{ततः क्रुद्धो महाराज आर्श्यशृङ्गिरमर्षणः}


\twolineshloka
{महेन्द्रप्रतिमं कार्ष्णि छादयामास पत्रिभिः}
{तेन ते विशिखा मुक्ता यमदण्डोपमाः शिताः}


\twolineshloka
{अभिमन्युं विनिर्भिद्य प्राविशन्त धरातलम्}
{तथैवार्जुनिना मुक्ताः शराः कनकभूषणाः}


\twolineshloka
{अलम्बुसं विनिर्भिद्य प्राविशन्त धरातलम्}
{सौभद्रस्तु रणे रक्षः शरैः सन्नतपर्वभिः}


\twolineshloka
{चक्रे विमुखमासाद्य बलं शक्र इवाहवे}
{विमुखं च रणे रक्षो वध्यमानं रणेऽरिणा}


\twolineshloka
{प्रादुश्चक्रे महामायां तामसीमरिघातिनीम्}
{ततस्ते तमसा सर्वे वृताश्चासन्महीपते}


\twolineshloka
{नाभिमन्युमपश्यन्त नैव स्वान्न परान्रणे}
{अभिमन्युश्च तदृष्ट्वा घोररूपं महत्तमः}


\twolineshloka
{प्रादुश्चक्रेऽस्त्रमत्सुग्रं भास्करं कुरुनन्दनः}
{ततः प्रकाशमभवज्जगत्सर्वं महीपते}


\twolineshloka
{तां चाभिजघ्निवान्मायां राक्षसस्य दुरात्मनः}
{संक्रुद्धश्च महावीर्यो राक्षसेन्द्रं नरोत्तमः}


\twolineshloka
{छादयामास समरे शरैः सन्नतपर्वभिः}
{बह्वीस्तथाऽन्या मायाश्च प्रयुक्तास्तेन रक्षसा}


\twolineshloka
{सर्वास्त्रविदमेयात्मा वारयामास फाल्गुनिः}
{हतमायं ततो रक्षो वध्यमानं च सायकैः}


\twolineshloka
{रथं तत्रैव संत्यज्य प्राद्रवन्महतो भयात्}
{तस्मिन्विनिर्जिते तूर्णं कूटयोधिनि राक्षसे}


\twolineshloka
{आर्जुनिः समरे सैन्यं तावकं संममर्द ह}
{मदान्धो वन्यनागेन्द्रः सपद्मां पद्मिनीमिव}


\twolineshloka
{ततः शान्तनवो भीष्मः सैन्यं दृष्ट्वाऽभिविद्रुतम्}
{महता शरवर्षेण सौभद्रं पर्यवारयत्}


\twolineshloka
{कोष्ठीकृत्य च तं वीरं धार्तराष्ट्रा महारथः}
{एवं सुबहवो युद्धे ततक्षुः सायकैर्दृढम्}


\twolineshloka
{स तेषां रथिनां वीरः पितुस्तुल्यपराक्रमः}
{सदृशो वासुदेवस्य विक्रमेण बलेन च}


\twolineshloka
{उभयोः सदृशं कर्म स पितुर्मातुलस्य च}
{रणे बहुविधं चक्रे सर्वशस्त्रभृतां वरः}


\twolineshloka
{ततो धनञ्जयो वीरो विनिघ्नंस्तव सैनिकान्}
{आससाद रणे भीष्मं पुत्रप्रेप्सुरमर्षणः}


\twolineshloka
{तथैव समरे राजन्पिता देवव्रतस्तव}
{आससाद रणे पार्थं स्वर्भानुरिव भास्करम्}


\twolineshloka
{ततः सरथनागाश्वाः पुत्रास्तव जनेश्वर}
{परिवव्रू रणे भीष्मं जुगुपुश्च समन्ततः}


\twolineshloka
{तथैव पाण्डवा राजन्परिवार्य धनञ्जयम्}
{रणाय महते युक्ता दंशिता भरतर्षभ}


\twolineshloka
{शारद्वतस्ततो राजन्भीष्मस्य प्रमुखे स्थितम्}
{अर्जुनं पञ्चविंशत्या सायकानां समाचिनोत्}


\twolineshloka
{प्रत्युद्गम्याथ विव्याध सात्यकिस्तं शितैः शरैः}
{पाण्डवप्रियकामार्थं शार्दूल इव कुञ्जरम्}


\twolineshloka
{गौतमोऽपि त्वरायुक्तो माधवं नवभिः शरैः}
{हृदि विव्याध संक्रुद्धः कङ्कपत्रपरिच्छदैः}


\twolineshloka
{शैनेयोऽपि ततः क्रुद्धश्चापमानम्य वेगवान्}
{गौतमान्तकरं तूर्णं समाधत्त शिलीमुखम्}


\twolineshloka
{तमापतन्तं वेगेन शक्राशनिसमद्युतिम्}
{द्विधा चिच्छेद संक्रुद्धो द्रौणिः परमकोपनः}


\twolineshloka
{समुत्सृज्याथ शैनेयो गौतमं रथिनां वरः}
{अभ्यद्रवद्रणे द्रौणिं राहुः खे शशिनं यथा}


\twolineshloka
{तस्य द्रोणसुतश्चापं द्विधा चिच्छेद भारत}
{अथैनं छिन्नधन्वानं ताडयामास सायकैः}


\twolineshloka
{सोऽन्यत्कार्मुकमादाय शत्रुघ्नं भारसाधनम्}
{द्रौणिं षष्ट्या महाराज बाह्वोरुरसि चार्पयत्}


\twolineshloka
{स विद्धो व्यथितश्चैव मुहूर्तं कश्मलायुतः}
{निषसाद रथोपस्थे ध्वजयष्टिं समाश्रितः}


\twolineshloka
{प्रतिलभ्य ततः संज्ञां द्रोणपुत्रः प्रतापवान्}
{वार्ष्णेयं समरे क्रुद्धो नाराचेन समार्पयत्}


\twolineshloka
{शैनेयं स तु निर्भिद्य प्राविशद्धरणीतलम्}
{वसन्तकाले बलवान्बिलं सर्पशिशुर्यथा}


\twolineshloka
{अथापरेण भल्लेन माधवस्य ध्वजोत्तमम्}
{चिच्छेद समरे द्रौणिः सिंहनादं मुमोच ह}


\twolineshloka
{पुनश्चैनं शरैर्घोरैश्छादयामास भारत}
{निदाघान्ते महाराज यथा मेघो दिवाकरम्}


\twolineshloka
{सात्यकोऽपि महाराज शरजालं निहत्य तत्}
{द्रौणिमभ्याकिरत्तूर्णं शरजालैरनेकधा}


\twolineshloka
{तापयामास च द्रौणिं शैनेयः परवारहा}
{विमुक्तो मेघजालेन यथैव तपनस्तथा}


\twolineshloka
{शराणां च सहस्रेण पुनरेव समुद्यतः}
{सात्यकिश्छादयामास ननाद च महाबलः}


\twolineshloka
{दृष्ट्वा पुत्रं च तं ग्रस्तं राहुणेव निशाकरम्}
{अभ्यद्रवत शैनेयं भारद्वाजः प्रतापवान्}


\twolineshloka
{विव्याध च सुतीक्ष्णेन पृषत्केन महामृधे}
{परीप्सन्खसुतं राजन्वार्ष्णेयेनाभिपीडितम्}


\twolineshloka
{सात्यकिस्तु रणे हित्वा गुरुपुत्रं महारथम्}
{द्रोणं विव्याध विंशत्या सर्वपारशवैः शरैः}


\twolineshloka
{तदन्तरममेयात्मा कौन्तेयः शत्रुतापनः}
{अभ्यद्रवद्रणे क्रुद्धो द्रोणं प्रति महारथः}


\twolineshloka
{ततो द्रोणश्च पार्थश्च समेयातां महामृधे}
{यथा बुधश्च शुक्रश्च महाराज नभस्तले}


\chapter{अध्यायः १०२}
\twolineshloka
{धृतराष्ट्र उवाच}
{}


\twolineshloka
{कथं द्रोणो महेष्वासः पाण्डवश्च धनञ्जयः}
{समीयतू रणे यत्तौ तावुभौ पुरुषर्षभौ}


\twolineshloka
{प्रियो हि पाण्डवो नित्यं भारद्वाजस्य धीमतः}
{आचार्यश्च रणे नित्यं प्रियः पार्थस्य सञ्जय}


\threelineshloka
{तावुभौ रथिनो सङ्ख्ये हृष्टौ सिंहाविवोत्कटौ}
{कथं समीयतुर्यत्तौ भारद्वाजधनञ्जयौ ॥सञ्जय उवाच}
{}


\twolineshloka
{न द्रोणः समरे पार्थं जानीते प्रियमात्मनः}
{क्षत्रधर्मं पुरस्कृत्य पार्थो वा गुरुमाहवे}


\twolineshloka
{न क्षत्रिया रणे राजन्वर्जयन्ति परस्परम्}
{निर्मर्यादं हि युध्यन्ते पितृभिर्भ्रातृभिः सह}


\twolineshloka
{रणे भारत पार्थेन द्रोणो विद्धस्त्रिभिः शरैः}
{नाचिन्तयच्च तान्बाणान्पार्थचापच्युतान्युधि}


\twolineshloka
{शरवृष्ट्या पुनः पार्थश्छादयामास तं रणे}
{स प्रजज्वाल रोषेण गहनेऽग्निरिवोर्जितः}


\twolineshloka
{ततोऽर्जुनं रणे द्रोणः शरैः सन्नतपर्वभिः}
{छादयामास राजेन्द्र नचिरादेव भारत}


\twolineshloka
{ततो दुर्योधनो राजा सुशर्माणमचोदयत्}
{द्रोणस्य समरे राजन्पार्ष्णिग्रहणकारणात्}


\twolineshloka
{त्रिगर्तराडपि क्रुद्धो भृशमायम्य कार्मुकम्}
{छादयामास समरे पार्थं बाणैरयोमुखैः}


\twolineshloka
{ताभ्यां मुक्ताः शरा राजन्नन्तरिक्षे विरेजिरे}
{हंसा इव महाराज शरत्काले नभस्तले}


\twolineshloka
{ते शराः प्राप्य कौन्तेयं समन्ताद्विविशुः प्रभो}
{फलभारनतं यद्वत्स्वादुवृक्षं विहंगमाः}


\twolineshloka
{अर्जुनस्तु रणे नादं विनद्य रथिनां वरः}
{त्रिगर्तराजं समरे सपुत्रं विव्यधे शरैः}


\threelineshloka
{ते वध्यमानाः पार्थेन कालेनेव युगक्षये}
{पार्थमेवाभ्यवर्तन्त मरणे कृतनिश्चयाः}
{मुमुचुः शरवृष्टिं च पाण्डवस्य रथं प्रति}


\twolineshloka
{शरवृष्टिं ततस्तां तु शरवर्षैः समन्ततः}
{प्रतिजग्राह राजेन्द्र तोयवृष्टिमिवाचलः}


\twolineshloka
{तत्राद्भुतमपश्याम बीभत्सोर्हस्तलाघवम्}
{विमुक्तां बहुभिर्योधैः शस्त्रवृष्टिं दुरासदाम्}


\twolineshloka
{यदेको वारयामास मारुतोऽभ्रगणानिव}
{कर्मणा तेन पार्थस्य तुतुषुर्देवदानवाः}


\twolineshloka
{अथ क्रुद्धो रणे पार्थस्त्रिगर्तान्प्रति भारत}
{मुमोचास्त्रं महाराज वायव्यं पृतनामुखे}


\twolineshloka
{प्रादुरासीत्ततो वायुः क्षोभयाणो नभस्तलम्}
{पातयन्वै तरुगणान्विनिघ्नंश्चैव सैनिकान्}


\twolineshloka
{ततो द्रोणोऽभिवीक्ष्यैव वायव्यास्त्रं सुदारुणम्}
{शैलमन्यन्महाराज घोरमस्त्रं मुमोच ह}


\twolineshloka
{द्रोणेन युधि निर्मुक्ते तस्मिन्नस्त्रे नराधिप}
{प्रशशाम ततो वयुः प्रसन्नाश्च दिशो दश}


\twolineshloka
{ततः पाण्डुसुतो वीरस्त्रिगर्तस्य रथव्रजान्}
{निरुत्साहान्रणे चक्रे विमुखान्विपराक्रमान्}


\twolineshloka
{ततो दुर्योधनश्चैव कृपश्च रथिनां वरः}
{अश्वत्थामा तथा शल्यः काम्भोजश्च सुदक्षिणः}


\twolineshloka
{विन्दानुविन्दावावन्त्यौ बाह्लिकः सह बाह्लिकैः}
{महता रथवंशेन पार्थस्यावारयन्दिशः}


\twolineshloka
{तथैव भगदत्तश्च श्रतायुश्च महाबलः}
{गजानीकेन भीमस्य ताववारयतां दिशः}


\twolineshloka
{भूरिश्रवाः शलश्चैव सौबलश्च विशांपते}
{शरौघैर्विमलैस्तीक्ष्णैर्माद्रीपुत्राववारयन्}


\twolineshloka
{भीष्मस्तु संहतः सङ्ख्ये धार्तराष्ट्रैः ससैनिकैः}
{युधिष्ठिरं समासाद्य सर्वतः पर्यवारयत्}


\twolineshloka
{आपतन्तं गजानीकं दृष्ट्वा पार्थो वृकोदरः}
{लोलिहन्सृक्किणी वीरो मृगराडिव कानने}


\twolineshloka
{भीमस्तु रथिनां श्रेष्ठो गदां गृह्य महाहवे}
{अवप्लुत्य रथात्तूर्णं तव सैन्यान्यभीषयत्}


\twolineshloka
{तमुद्वीक्ष्य गदाहस्तं ततस्ते गजसादिनः}
{परिवव्रू रणे यत्ता भीमसेनं समन्ततः}


\twolineshloka
{गजमध्यमनुप्राप्तः पाण्डवः स व्यराजत}
{मेघजालस्य महतो यथा मध्यगतो रविः}


\twolineshloka
{व्यधमत्स गजानीकं गदया पाण्डवर्षभः}
{महाभ्रजालमतुलं मातरिश्वेव सन्ततम्}


\twolineshloka
{ते वध्यमाना बलिना भीमसेनेन दन्तिनः}
{आर्तनादं रणे चक्रुर्गर्जन्तो जलदा इव}


\twolineshloka
{बहुधा दारितश्चैव विषाणैस्तत्र दन्तिभिः}
{फुल्लाशोकनिभः पार्थः शुशुभे रणमूर्धनि}


\twolineshloka
{`सादिनां शस्त्रवृष्टिं च व्यधमद्गदया ततः}
{'वायुवेगसमायुक्तो व्यचरत्पाण्डवो युधि}


\twolineshloka
{विषाणोल्लिखितैर्गात्रौर्विषाणाभिहतो भृशम्}
{विषाणे दन्तिनं गृह्य निर्विषाणमथाकरोत्}


\twolineshloka
{विषाणेन च तेनैव कुम्भोऽभ्याहत्य दन्तिनम्}
{पातयामास समरे दण्डहस्त इवान्तकः}


\twolineshloka
{शोणिताक्तां गदां बिभ्रन्मेदोभञ्जाकृतच्छविः}
{कृताभ्यङ्गः शोणितेन रुद्रवत्प्रत्यदृश्यत}


\twolineshloka
{एवं के वध्यमानाश्च हतशेषा महागजाः}
{प्राद्रवन्त दिशो राजन्विमृद्गन्तः स्वकं बलम्}


\twolineshloka
{द्रवद्भिस्तैर्महानागैः समन्ताद्भरतर्षभ}
{दुर्योधनबलं सर्वं पुनरासीत्पराङ्भुखम्}


\chapter{अध्यायः १०३}
\twolineshloka
{सञ्जय उवाच}
{}


\twolineshloka
{मध्यंदिनो महाराज संग्रामः समपद्यत}
{लोकक्षयकरो रौद्रो भीष्मस्य सह सोमकैः}


\twolineshloka
{गाङ्गेयो रथिनां श्रेष्ठः पाण्डवानामनीकिनीम्}
{व्यधमन्निशितैर्बाणैः शतशोऽथ सहस्रशः}


\twolineshloka
{संममर्द च तत्सैन्यं पिता देवव्रतस्तव}
{मर्दयेच्च यथा राजन्सिंहः प्राप्य मृगव्रजम्}


\twolineshloka
{धृष्टद्युम्नः शिखण्डी च विराटो द्रुपदस्तथा}
{भीष्ममासाद्य समरे शरैर्जघ्नुर्महारथम्}


\twolineshloka
{धृष्टद्युम्नं ततो विद्ध्वा विराटं च शरैस्त्रिभिः}
{द्रुपदस्य च नाराचं प्रेषयामास भारत}


\twolineshloka
{तेन विद्धा महेष्वासा भीष्मेणामित्रकर्शिना}
{चुक्रुधुः समरे राजन्पादस्पृष्टा इवोरगाः}


\twolineshloka
{शिखण्डी तं च विव्याध भरतानां पितामहम्}
{स्त्रीमयं मनसा ध्यात्वा नास्मै प्राहरदच्युतः}


\twolineshloka
{धृष्टद्युम्नस्तु समरे क्रोधेनाग्निरिव ज्वलन्}
{पितामहं त्रिभिर्बाणैर्बाह्वोरुरसि चार्पयत्}


\twolineshloka
{द्रुपदः पञ्चविंशत्या विराटो दशमिः शरैः}
{शिखण्डी पञ्चविंशत्या भीष्मं विव्याध सायकैः}


\twolineshloka
{सोऽतिविद्धो महाराज शोणितौघपरिप्लुतः}
{वसन्ते पुष्पशबलो रक्ताशोक इवाबभौ}


\twolineshloka
{तान्प्रत्यविध्यद्गाङ्गेयस्त्रिभिस्त्रिभिरजिह्मगैः}
{द्रुपदस्य च भल्लेन धनुश्चिच्छेद मारिष}


\twolineshloka
{सोऽन्यत्कार्मुकमादाय भीष्मं विव्याध पञ्चभिः}
{सारथिं च त्रिभिर्बाणैः सुशितै रणमूर्धनि}


\twolineshloka
{तथा भीमो महाराज द्रौपद्याः पञ्च चात्मजाः}
{केकया भ्रातरः पञ्च सात्यकिश्चैव सात्वतः}


\twolineshloka
{अभ्यद्रवन्त गाङ्गेयं युधिष्ठिरसमाज्ञया}
{प्रति रक्षणकार्यार्थं धृष्टद्युम्नमुखान्रणे}


\twolineshloka
{तथैव तावकाः सर्वे भीष्मरक्षार्थमुद्यताः}
{प्रत्युद्ययुः पाण्डुसेनां सहसैन्या नराधिप}


\twolineshloka
{तत्रासीत्सुमहद्युद्धं तव तेषां च संकुलम्}
{नराश्वरथनागानां यमराष्ट्रविवर्धनम्}


\twolineshloka
{रथी रथिनमासाद्य प्राहिणोद्यमसादनम्}
{तथेतरान्समासाद्य नरनागाश्वसादिनः}


\twolineshloka
{अनयन्परलोकाय शरैः सन्नतपर्वभिः}
{शरैश्च विविधैर्घोरैस्तत्रतत्र विशांपते}


\twolineshloka
{रथास्तु रथिभिर्हीना हतसारथयस्तथा}
{विप्रद्रुताश्वाः समरे दिशो जग्मुः समन्ततः}


\twolineshloka
{मृद्गन्तस्ते नरान्राजन्हयांश्च सुबहून्रणे}
{वातायमाना दृश्यन्ते गन्धर्वनगरोपमाः}


\twolineshloka
{रथिनश्च रथैर्हीना वर्मिणस्तेजसा युताः}
{कुण्डलोष्णीषिणः सर्वे निष्काङ्गदविभूषणाः}


\twolineshloka
{देवपुत्रसमाः सर्वे शौर्ये शक्रसमा युधि}
{ऋद्ध्या वैश्रवणां चाति नयेन च बृहस्पतिम्}


\twolineshloka
{सर्वलोकेश्वराः शूरास्तत्रतत्र विशांपते}
{विप्रद्रुता व्यदृश्यन्त प्राकृता इव मानवाः}


\twolineshloka
{दन्तिनश्च नरश्रेष्ठ हीनाः परमसादिभिः}
{मृद्गन्तः स्वान्यनीकानि निपेतुः सर्वशब्दगाः}


\twolineshloka
{चर्मभिश्चामरैश्चित्रैः पताकाभिश्च मारिष}
{छत्रैः सितैर्हेमदण्डैश्चामरैश्च समन्ततः}


\twolineshloka
{विशीर्णैर्विप्रधावन्तो दृश्यन्ते स्म दिशो दश}
{नवमेघप्रतीकाशा जलदोपमनिःस्वनाः}


\twolineshloka
{तथैव दन्तिभिर्हीना गजारोहा विशांपते}
{प्रधावन्तोऽन्वदृश्यन्त तव तेषां च संकुले}


\twolineshloka
{नानादेशसमुत्थांश्च तुरगान्हेमभीषितान्}
{वातायमानानद्राक्षं शतशोऽथ सहस्रशः}


\twolineshloka
{अश्वारोहान्हतैरश्वैर्गृहीतासन्समन्ततः}
{द्रवमाणानपश्याम द्राव्यमाणांश्च संयुगे}


\threelineshloka
{गजो गजं समासाद्य द्रवमाणं महाहवे}
{ययौ प्रमृद्य तरसा पादातान्वाजिनस्तथा}
{तथैव च रथान्राजन्प्रममर्द रणे गजः}


\twolineshloka
{रथाश्चैव समासाद्य पतितांस्तुरगान्भुवि}
{व्यमृद्गन्समरे राजंस्तुरगांश्च नरान्रमे}


\twolineshloka
{एवं ते बहुधा राजन्प्रत्यमृद्गन्परस्परम्}
{`दृश्यन्तेस्म महाबाहो तत्रतत्र महाबलाः ॥'}


\twolineshloka
{तस्मिन्रौद्रे तथा युद्धे वर्तमाने महाभये}
{प्रावर्तत नदी घोरा शोमितान्त्रतरङ्गिणी}


\twolineshloka
{अस्थिसङ्घातसंबाधा केशशैवलाद्वला}
{रथह्रदा शरावर्ता हयमीना दुरासदा}


\twolineshloka
{शीर्षोपलसमाकीर्णा हस्तिग्राहसमाकुला}
{कवचोष्णीषफेनौघा धनुर्वेगासिकच्छपा}


\threelineshloka
{` शङ्खनक्रौघसंकीर्णा छत्रकूर्मरथोडुपा}
{'पताकाध्वजवृक्षाढ्या मर्त्यकूलापहारिणी}
{क्रव्यादहंससंकीर्णा यमराष्ट्रविवर्धनी}


\twolineshloka
{तां नदीं क्षत्रियाः शूरा रथनागहयप्लवैः}
{प्रतेरुर्बहवो राजन्भयं त्यक्त्वा महारथाः}


\twolineshloka
{अपोवाह रणे भीरून्कश्मलेनाभिसंवृतान्}
{यथा वैतरणी प्रेतान्प्रेतराजपुरं प्रति}


\twolineshloka
{प्राक्रोशन्क्षत्रियास्तत्र दृष्ट्वा तद्वैशसं महत्}
{दुर्योधनापराधेन गच्छन्ति क्षत्रियाः क्षयम्}


\twolineshloka
{गुणवत्सु कथं द्वेषं धृतराष्ट्रो जनेश्वरः}
{कृतवान्पाण्डुपुत्रेषु पापात्मा लोभमोहितः}


\twolineshloka
{एवं बहुविधा वाचः श्रूयन्ते स्म परस्परम्}
{पाण्डवस्तवसंयुक्ताः पुत्राणां ते सुदारुणाः}


\twolineshloka
{ता निशम्य ततो वाचः सर्वयोधैरुदाहृताः}
{आगस्कृत्सर्वलोकस्य पुत्रो दुर्योधनस्तव}


\twolineshloka
{भीष्मं द्रोणं कृपं चैव शल्यं चोवाच भारत}
{युध्यध्वमनहंकराः किं चिरं कुरुथेति च}


\twolineshloka
{`इति दुर्योधनोत्सृष्टाः सर्वे युयुधिरे नृपाः'ततः प्रववृते युद्धं कुरूणां पाण्डवैः सह}
{अक्षद्यूतकृतं राजन्सुघोरं वैशसं तदा}


\twolineshloka
{यत्पुरा न निगृह्णासि वार्यमाणो महात्मभिः}
{वैचित्रवीर्य तस्येदं फलं पश्य सुदारुणम्}


\twolineshloka
{न हि पाण्डुसुता राजन्ससैन्याः सपदानुगाः}
{रक्षन्ति समरे प्राणान्कौरवा वापि संयुगे}


\twolineshloka
{एतस्मात्कारणाद्धोरो वर्तते स्वजनक्षयः}
{दैवाद्वा पुरुषव्याघ्र तव चापनयान्नृप ॥ ॥ इतिश्रामन्महाभारते भीष्मपर्वणि भीष्मवधपर्वणि नवमदिवसयुद्धेत्र्यधिकशततमोऽध्यायः}


\chapter{अध्यायः १०४}
\twolineshloka
{सञ्जय उवाच}
{}


\twolineshloka
{अर्जुनस्तान्नरव्याघ्रः सुशर्मानुचरान्नृपान्}
{अनयत्प्रेतराजस्य सदनं सायकैः शितैः}


\twolineshloka
{सुशर्मापि ततो बाणैः पार्थं विव्याध संयुगे}
{वासुदेवं च सप्तत्या पार्थं च नवभिः पुनः}


\twolineshloka
{तं निवार्य शरौघेण सक्रुसूनुर्महारथः}
{सुशर्मणो रणे योधान्प्राहिणोद्यमसादनम्}


\twolineshloka
{ते वध्यमानाः पार्थेन कालेनेव युगक्षये}
{व्यद्रवन्त रणे राजन्भये जाते महारथाः}


\twolineshloka
{उत्सृज्य तुरगान्केचिद्रथान्केचिच्च मारिष}
{गजानन्ये समुत्सृज्य प्राद्रवन्त दिशो दश}


\twolineshloka
{अपरे तु तदादाय वाजिनागरथान्रणे}
{त्वरया परया युक्ताः प्राद्रवन्त विशांपते}


\twolineshloka
{पादाताश्चापि शस्त्राणि समुत्सृज्य महारणे}
{निरपेक्षा व्यधावन्त तेनतेन स्म भारत}


\twolineshloka
{वार्यमाणाः सुबहुशस्त्रैगर्तेन सुशर्मणा}
{तथान्यैः पार्थिवश्रेष्ठैर्न व्यतिष्ठन्त संयुगे}


\twolineshloka
{तद्बलं प्रद्रुतं दृष्ट्वा पुत्रो दुर्योधनस्तव}
{पुरस्कृत्य रणे भीष्मं सर्वसैन्यपुरस्कृतः}


\twolineshloka
{सर्वोद्योगेन महता धनञ्जयमुपाद्रवत्}
{त्रिगर्ताधिपतेरर्थे जीवितस्य विशांपते}


\twolineshloka
{स एकः समरे तस्थौ किरन्बहुविधाञ्शरान्}
{भ्रातृभिः सहितः सर्वैः शेषा हि प्रद्रुता नराः}


\twolineshloka
{तथैव पाण्डवा राजन्सर्वोद्योगेन दंशिताः}
{प्रययुः फल्गुनार्थाय यत्र भीष्मो व्यतिष्ठत}


\twolineshloka
{ज्ञायमाना रणे वीर्यं घोरं गाण्डीवधन्वनः}
{हाहाकारकृतोत्साहा भीष्मं जग्मुः समन्ततः}


\twolineshloka
{ततस्तालध्वजः शूरः पाण्डवानां वरूथिनीम्}
{छादयामास समरे शरैः सन्नतपर्वभिः}


\twolineshloka
{एकीभूतास्ततः सर्वे कुरवः सह पाण्डवैः}
{अयुध्यन्त महाराज मध्यं प्राप्ते दिवाकरे}


\twolineshloka
{सात्यकिः कृतवर्माणं विद्ध्वा पञ्चभिराशुगैः}
{अतिष्ठदाहवे शूरः किरन्बाणान्सहस्रशः}


\twolineshloka
{तथैव द्रुपदो राजा द्रोणं विद्ध्वा शितैः शरैः}
{पुनर्विव्याध सप्तत्या सारथिं चास्य पञ्चभिः}


\twolineshloka
{भीमसेनस्तु राजानं बाह्लीकं प्रपितामहम्}
{विद्ध्वा नदन्महानादं शार्दूल इव कानने}


\threelineshloka
{आर्जुनिश्चित्रसेनेन विद्धो बहुभिरशुगैः}
{अतिष्ठदाहवे शूरः किरन्बाणान्सहस्रशः}
{चित्रसेनं त्रिभिर्बाणैर्विव्याध समरे भृशम्}


\twolineshloka
{समागतौ तौ तु रणे महामात्रौ व्यरोचताम्}
{यथा दिवि महाघोरौ राजन्बुधशनैश्चरौ}


\twolineshloka
{तस्याश्वांश्चतुरो हत्वा सूतं च नवभिः शरैः}
{ननाद बलवान्नादं सौभद्रः परवीरहा}


\twolineshloka
{हताश्वात्तु रथात्तूर्णं सोऽवप्लुत्य महारथः}
{आरुरोह रथं तूर्णं दुर्मुखस्य विशांपते}


\twolineshloka
{द्रोणश्च द्रुपदं भित्त्वा शरैः सन्नतपर्वभिः}
{सारथिं चास्य विव्याध त्वरमाणः पराक्रमी}


\twolineshloka
{पीड्यमानस्ततो राजा द्रुपदो वाहिनीमुखे}
{अपायाज्जवनैरश्वैः पूर्ववैरमनुस्मरन्}


\twolineshloka
{भीमसेनस्तु राजानं मुहूर्तादिव बाह्लिकम्}
{व्यश्वसूतरथं चक्रे सर्वसैन्यस्य पश्यतः}


% Check verse!
ससंभ्रमो महाराज संशयं परमं गतः
\twolineshloka
{अवप्लुत्य ततो वाहाद्बाह्लीकः पुरुषोत्तमः}
{आरुरह रथं तूर्णं लक्ष्मणस्य महारणे}


\twolineshloka
{सात्यकिः कृतवर्माणं वारयित्वा महारणे}
{शरैर्बहुविधै राजन्नाससाद पितामहम्}


\twolineshloka
{स विद्ध्वा भारतं षष्ट्या निशितै रोमवाहिभिः}
{नृत्यन्निव रथोपस्थे विधुन्वानो महद्धनुः}


\twolineshloka
{तस्यायसीं महाशक्तिं चिक्षेपाथ पितामहः}
{हेमचित्रां महावेगां नागकन्योपमां शुभाम्}


\twolineshloka
{तामापतन्तीं सहसा मृत्युकल्पां सुदुर्जयाम्}
{व्यंसयामास वार्ष्णेयो लाघवेन महायशाः}


\twolineshloka
{अनासाद्य तु वार्ष्णेयं शक्तिः परमदारुणा}
{न्यपतद्धरणीपृष्ठे महोल्केव महाप्रभा}


\twolineshloka
{वार्ष्णेयस्तु ततो राजन्स्वां शक्तिं कनकप्रभाम्}
{वेगवद्गृह्य चिक्षेप पितामहरथं प्रति}


\twolineshloka
{वार्ष्णेयभुजवेगेन प्रणुन्ना सा महाहवे}
{अभिदुद्राव वेगेन कालरात्रिर्यथा नरम्}


\twolineshloka
{तामापतन्तीं सहसा द्विधा चिच्छेद भारतः}
{क्षुरप्राभ्यां सुतीक्ष्णाभ्यांसा व्यशीर्यत मेदिनीम्}


\twolineshloka
{छित्त्वा शक्तिं तु गाङ्गयः सात्यकिं नवभिः शरैः}
{आजघानोरसि क्रुद्धः प्रहसञ्छत्रुकर्शनः}


\twolineshloka
{ततः सरथनागाश्वाः पाण्डवाः पाण्डुपूर्वज}
{परिवब्रू रणे भीष्मं माधवत्राणकारणात्}


\twolineshloka
{ततः प्रववृते युद्धं तुमुलं रोमहर्षणम्}
{पाण्डवानां कुरूणां च समरे विजयैषिणाम्}


\chapter{अध्यायः १०५}
\twolineshloka
{सञ्जय उवाच}
{}


\twolineshloka
{दृष्ट्वा भीष्मं रणे युद्धं पाण्डवैरभिसंवृतम्}
{यथा मेघैर्महाराज तपान्ते दिवि भास्करम्}


\twolineshloka
{दुर्योधनो महाराज दुःशासनमभाषत}
{एष शूरो महेष्वासो भीष्मः शूरनिषूदनः}


\twolineshloka
{छादितः पाण्डवैः शूरैः समन्ताद्भरतर्षभ}
{तस्य कार्यं त्वया वीर रक्षणं सुमहात्मनः}


\twolineshloka
{रक्ष्यमाणो हि समरे भीष्मोऽस्माकं पितामहः}
{निहन्यात्समरे यत्तान्पाञ्चालान्पाण्डवैः सह}


\twolineshloka
{तत्र कार्यतमं मन्ये भीष्मस्यैवाभिरक्षणम्}
{गोप्ता ह्येष महेष्वासो भीष्मोऽस्माकं महाव्रतः}


\threelineshloka
{स भवान्सर्वसैन्येन परिवार्य पितामहम्}
{समरे कर्म कुर्वाणं दुष्करं परिरक्षतु ॥सञ्जय उवाच}
{}


\twolineshloka
{स एवमुक्तः समरे पुत्रो दुःशासनस्तव}
{परिवार्य स्थितो भीष्मं सैन्येन महता वृतः}


\twolineshloka
{ततः शतसहस्राणां हयानां सुबलात्मजः}
{विमलप्रासहस्तानामृष्टितोमरधारिणाम्}


\twolineshloka
{दर्पितानां सुवेगानां बलस्थानां पताकिनाम्}
{शिक्षितैर्युद्धकुशलैरुपेतानां नरोत्तमैः}


\twolineshloka
{नकुलं सहदेवं च धर्मराजं च पाण्डवम्}
{न्यवारयन्नरश्रेष्ठान्परिवार्य समन्ततः}


\twolineshloka
{ततो दुर्योधनो राजा शूराणां हयसादिनाम्}
{अयुतं प्रेषयामास पाण्डवानां निवारणे}


\twolineshloka
{तैः प्रविष्टैर्महावेगैर्गरुत्मद्भिरिवाहवे}
{खुराहता धरा राजंश्चकम्पे च ननाद च}


\twolineshloka
{खुरशब्दश्च सुमहान्वाजिनां शुश्रुवे तदा}
{महावंशवनस्येव दह्यमानस्य पर्वते}


\twolineshloka
{उत्पतद्भिश्च तैस्तत्र समुद्धूतं महद्रजः}
{दिवाकररथं प्राप्य च्छादयामास भास्करम्}


\twolineshloka
{वेगवद्भिर्हयैस्तैस्तु क्षोभिता पाण्डवीक चमूः}
{निपतद्भिर्महावेगैर्हंसैरिव महत्सरः}


\twolineshloka
{हेषतां चैव शब्देन न प्राज्ञायत किंचन}
{`अन्तर्दधे महाञ्शब्दस्तेन शब्देन मोहितः ॥'}


\twolineshloka
{ततो युधिष्ठिरो राजा माद्रीपुत्रौ च पाण्डवौ}
{प्रत्यघ्नंस्तरसा वेगं समरे हयसादिनाम्}


\twolineshloka
{उद्वृत्तस्य महाराज प्रावृट्कालेऽतिपूर्यतः}
{पौर्णमास्यामम्बुवेगं यथा वेला महोदधेः}


\twolineshloka
{ततस्ते रथिनो राजञ्शरैः सन्नतपर्वभिः}
{न्यकृन्तन्नुत्तमाङ्गानि शरेण हयसादिनाम्}


\twolineshloka
{ते निपेतुर्महाराज निहता दृढधन्विभिः}
{नागैरिव महानागा यथावद्गिरिगह्वरे}


\twolineshloka
{तेऽपि प्रासैः सुनिशितैः शरैः सन्नतपर्वभिः}
{न्यकृन्तन्नुत्तमाङ्गानि विचरन्तो दिशो दश}


\twolineshloka
{अभ्याहता हयारोहा ऋष्टिभिर्भरतर्षभ}
{अत्यजन्नुत्तमाङ्गानि फलानीव महाद्रुमाः}


\twolineshloka
{समादिनो हया राजंस्तत्रतत्र निषूदिताः}
{पतिताः पात्यमानाश्च प्रत्यदृश्यन्त सर्वशः}


\twolineshloka
{वध्यमाना हयाश्चैव प्राद्रवन्त भयार्दिताः}
{यथा सिंहं समासाद्य मृगाः प्राणपरायणाः}


\twolineshloka
{पाण्डवाश्च महाराज जित्वा शत्रून्महामृधे}
{दध्मुः शङ्खांश्च भेरीश्च ताडयामासुराहवे}


\twolineshloka
{ततो दुर्योधनो दीनो दृष्ट्वा सैन्यं पराजितम्}
{अब्रवीद्रतश्रेष्ठ मद्रराजमिदं वचः}


\twolineshloka
{एष पाण्डुसुतो ज्येष्ठो यमाभ्यां सहितो रणे}
{पश्यतां वो महाबाहो सेनां द्रावयति प्रभो}


\threelineshloka
{तं वारय महाबाहो वेलेव मकरालयम्}
{त्वं हि संश्रूयसेऽत्यर्थमसह्यबलविक्रमः ॥सञ्जय उवाच}
{}


\twolineshloka
{पुत्रस्य तव तद्वाक्यं श्रुत्वा शल्यः प्रतापवान्}
{स ययौ रथवंशेन यत्र राजा युधिष्ठिरः}


\twolineshloka
{तदापतद्वै सहसा शल्यस्य सुमहद्बलम्}
{महौघवेगं समरे वारयामास पाण्डवः}


\threelineshloka
{मद्रराजं च समरे धर्मराजो महारथः}
{दशभिः सायकैस्तूर्णमाजघान स्तनान्तरे}
{नकुलः सहदेवश्च तं सप्तभिरजिह्नगैः}


\twolineshloka
{मद्रराजोऽपि तान्सर्वानाजघान त्रिभिस्त्रिभिः}
{युधिष्ठिरं पुनः षष्ट्या विव्याध निशितैः शरैः}


\twolineshloka
{माद्रीपुत्रौ च संभ्रान्तौ द्वाभ्यां द्वाभ्यामताडयत्}
{ततो भीमो महाबहुर्दृष्ट्वा राजनमाहवे}


\twolineshloka
{मद्रराजवशं प्राप्तं मृत्योरास्यगतं यथा}
{अभ्यपद्यत संग्रामे युधिष्ठिरममित्रजित्}


\twolineshloka
{`आपतन्नेव भीमस्तु मद्रराजमताडयत्}
{'सर्वपारशवैस्तीक्ष्णैर्नाराचैर्मर्मभेदिभिः}


\twolineshloka
{ततो भीष्मश्च द्रोणश्च सैन्येन महता वृतौ}
{राजानमभ्यपद्येतामञ्जसा शरवर्षिणौ}


\twolineshloka
{ततो युद्धं महाघोरं प्रावर्तत सुदारुणम्}
{अपरां दिशमास्थाय द्योतमाने दिवाकरे}


\chapter{अध्यायः १०६}
\twolineshloka
{सञ्जय उवाच}
{}


\twolineshloka
{ततः पिता तव क्रुद्धो निशितैः सायकोत्तमैः}
{आजघान रणे पार्थान्सहसेनान्समन्ततः}


\twolineshloka
{भीमं द्वादशभिर्विद्ध्वा सात्यकिं नवभिः शरैः}
{नकुलं च त्रिभिर्विद्ध्वा सहदेवं च सप्तभिः}


\twolineshloka
{युधिष्ठिरं द्वादशभिर्बाह्वोरुरसि चार्पयत्}
{धृष्टद्युम्नं ततो विद्ध्वा ननाद सुमहाबलः}


\threelineshloka
{तं द्वादशाख्यैर्नकुलो माधवश्च त्रिभिः शरैः}
{धृष्टद्युम्नश्च सप्तत्या भीमसेनश्च सप्तभिः}
{युधिष्ठिरो द्वादशभिः प्रत्यविध्यत्पितामहम्}


\twolineshloka
{द्रोणस्तु सात्यकिं विद्ध्वा भीमसेनमविध्यत}
{एकैकं पञ्चभिर्बाणैर्यमदण्डोपमैः शितैः}


\twolineshloka
{तौ च तं प्रत्यविध्येतां त्रिभिस्त्रिभिरजिह्मगैः}
{तोत्रैरिव महानागं द्रोणं ब्राह्मणपुङ्गवम्}


\threelineshloka
{सौवीरा कितवाः प्राच्याः प्रतीच्योदीच्यमालवाः}
{अभीषाहाः शूरसेनाः शिबयोऽथ वसातयः}
{संग्रमे नाजहुर्भीष्मं वध्यमानाः शितैः शरैः}


\threelineshloka
{तथैवान्ये महीपाला नानादेशसमागताः}
{पाण्डवानभ्यवर्तन्त विविधायुधपाणयः}
{तथैव पाण्डवा राजन्परिवव्रुः पितामहम्}


\twolineshloka
{स समन्तात्परिवृतो रथौघैरपराजितः}
{गहनेऽग्निरिवोत्सृष्टः प्रजज्वाल दहन्परान्}


\twolineshloka
{रथाग्न्यगारश्चापार्चिरसिशक्तिगजेन्धनः}
{शरस्फुलिङ्गो भीष्माग्निर्ददाह क्षत्रियर्षभान्}


\twolineshloka
{` यथा हि सुमहानग्निः कक्षे चरसि सानिलः}
{'तथा भीष्मो महाराज दिव्यमस्त्रमुदीरयन्}


\twolineshloka
{सुवर्णपुङ्खैरिषुभिर्गार्ध्रपक्षैः सुतेजनैः}
{कर्णिनालीकनाराचैश्छादयामास तद्बलम्}


\twolineshloka
{अपातयद्ध्वजांश्चैव रथिनश्च शितैः शरैः}
{मुण्डतालवनानीव चकार स रथव्रजान्}


\twolineshloka
{निर्मनुष्यान्रथान्राजन्गजानश्वांश्च संयुगे}
{अकरोत्स महाबाहुः सर्वशस्त्रभृतां वरः}


\twolineshloka
{तस्य ज्यातलनिर्घोषं विस्फूर्जितमिवाशनेः}
{निशम्य सर्वभूतानि समकम्पन्त भारत}


\twolineshloka
{अमोघा ह्यपत्नबाणाः पितुस्ते भरतर्षभ}
{नासज्जन्त तनुत्रेषु भीष्मचापच्युताः शराः}


\twolineshloka
{हतवीरान्रथान्राजन्संयुक्ताञ्जवनैर्हयैः}
{अपश्याम महाराज ह्रियमाणान्रणाजिरे}


\threelineshloka
{चेदिकाशिकरूषाणां सहस्राणि चतुर्दश}
{महारथाः समाख्याताः कलपुत्रास्तनुत्यजः}
{अपरावर्तिनः सर्वे सुवर्णविकृतध्वजाः}


\twolineshloka
{संग्रामे भीष्ममासाद्य व्यादितास्यमिवान्तकम्}
{निमग्नाः परलोकाय सवाजिरथकुञ्जराः}


\twolineshloka
{भग्नाक्षोपस्करान्कांश्चिद्भग्नचक्रांश्च भारत}
{अपश्याम महाराज शतशोऽथ सहस्रशः}


\twolineshloka
{सवरूथै रथैर्भग्नै रथिभिश्च निपातितैः}
{शरैः सुकवचैश्छिन्नैः पट्टसैश्च विशांपते}


\twolineshloka
{गदाभिर्भिण्डिपालैश्च निशितैश्च शिलीमुखैः}
{अनुकर्षैरुपासङ्गैश्चक्रैर्भग्नैश्च मारिष}


\threelineshloka
{बाहुभिः कार्मुकैः खङ्गैः शिरोभिश्च सकुण्डलैः}
{तलत्रैरङ्गुलित्रैश्च ध्वजैश्च विनिपातितैः}
{चापैश्च बहुधा च्छिन्नैः समास्तीर्यत मेदिनी}


\twolineshloka
{गजारोहा गजान्राजन्हयांश्च हयसादिनः}
{अभिपेतुर्द्रुतं तत्र शतशोऽथ सहस्रशः}


\twolineshloka
{यतमानाश्च ते वीरा द्रवमाणान्महारथान्}
{नाशक्नुवन्वारयितुं भीष्मबाणप्रपीडितान्}


\twolineshloka
{महेन्द्रसमवीर्येण वध्यमाना महाचमूः}
{अभज्यत महाराज न च द्वौ समधावताम्}


\twolineshloka
{आविद्धरथनागाश्वं पतितध्वजसंकुलम्}
{अनीकं पाण्डुपुत्राणां हाहाभूतमचेतनम्}


\twolineshloka
{जघानात्र पिता पुत्रं पुत्रश्च पितरं तथ}
{प्रियं सखाय चाक्रन्दे सखा दैवबलात्कृतः}


\twolineshloka
{विमुच्य कवचानन्ये पाण्डुपुत्रस्य सैनिकाः}
{प्रकीर्य केशान्धावन्तः प्रत्यदृश्यन्त सर्वशः}


\twolineshloka
{तद्गोकुलमिवोद्भ्रान्तमुद्भान्तरथकूबरम्}
{ददृशे पाण्डुपुत्रस्य सैन्यमार्तस्वरं तदा}


\threelineshloka
{प्रभज्यमानं सैन्यं तु दृष्ट्वा यादवनन्दनः}
{उवाच पार्थं बीभत्सुं निगृह्य रथमुत्तमम् ॥श्रीभगवानुवाच}
{}


\twolineshloka
{अयं स कालः संप्राप्तः पार्थ यः काङ्क्षितस्तव}
{प्रहरास्मै नरव्याघ्र भीष्मायाहवशोभिने}


\twolineshloka
{यत्पुरा कथितं वीर त्वया राज्ञां समागमे}
{विराटनगरे तात सञ्जयस्य समीपतः}


\twolineshloka
{भीष्मद्रोणमुखान्सर्वान्धार्तराष्ट्रस्य सैनिकान्}
{सानुबन्धान्हनिष्यामि ये मां योत्स्यन्ति सङ्गरे}


\threelineshloka
{इति तत्कुरु कौन्तेय सत्यं वाक्यमरिन्दम}
{क्षत्रधर्ममनुस्मृत्य युध्यस्व विगतज्वरः ॥सञ्जय उवाच}
{}


\twolineshloka
{इत्युक्तो वासुदेवेन तिर्यग्दृष्टिरधोमुखः}
{अकाम इव बीभत्सुरिदं वचनमब्रवीत्}


\twolineshloka
{अवध्यानां वधं कृत्वा राज्यं वा नरकोत्तरम्}
{दुःखानि वनवासे वा किं नु मे सुकृतं भवेत्}


\threelineshloka
{चोदयाश्वान्यतो भीष्मः करिष्ये वचनं तव}
{पातयिष्यामि दुर्धर्षं भीष्मं कुरुपितामहम् ॥सञ्जय उवाच}
{}


\threelineshloka
{स चाश्वान्रजतप्रख्यांश्चोदयामास माधवः}
{यतो भीष्मस्ततो राजन्दुष्प्रेक्ष्यो रश्मिवानिव}
{}


\twolineshloka
{ततस्तत्पुनरावृत्तं युधिष्ठिरबलं महत्}
{दृष्ट्वा पार्थं महाबाहुं भीष्मायोद्यतमाहवे}


\twolineshloka
{ततो भीष्मः कुरुश्रेष्ठः सिंहवद्विनदन्मुहुः}
{धनंजयरथं शीघ्रं शरवर्षैरवाकिरत्}


\twolineshloka
{क्षणेन स रथस्तस्य सहयः सहसारथिः}
{शरवर्षेण महता न प्राज्ञायत भारत}


\twolineshloka
{वासुदेवस्त्वसंभ्रान्तो धैर्यमास्थाय सत्वरः}
{चोदयामास तानश्वान्विनुन्नान्भीष्मसायकैः}


\twolineshloka
{ततः पार्थो धनुर्गृह्य दिव्यं जलदनिःस्वनम्}
{पातयामास भीष्मस्य धनुश्छित्त्वा शितैः शरैः}


\twolineshloka
{स च्छिन्नधन्वा कौरव्यः पुनरन्यन्महद्धनुः}
{निमेषान्तरमात्रेण सज्यं चक्रे पिता तव}


\threelineshloka
{चकर्ष च ततो दोर्भ्यां धनुर्जलदनिःस्वनम्}
{अथास्य तदपि क्रुद्धश्चिच्छेद धनुरर्जुनः}
{तस्य तत्पूजयामास लाघवं शन्तनोः सुतः}


\twolineshloka
{गाङ्गेयस्त्वब्रवीत्पार्थं धन्विश्रेष्ठमरिंदम}
{साधुसाधु महाबाहो साधु कुन्तीसुतेति च}


\twolineshloka
{समाभाष्यैवमपरं प्रगृह्य रुचिरं धनुः}
{मुमोच समरे भीष्मः शरान्पार्थरथं प्रति}


\twolineshloka
{अदर्शयद्वसुदेवो हययाने परं बलम्}
{मोघान्कुर्वञ्शरांस्तस्य मण्डलानि निदर्शयन्}


\twolineshloka
{शुशुभाते नरव्याघ्रौ तौ भीष्मशरविक्षतौ}
{गोवृषाविव संरब्धौ विषाणोल्लिखिताङ्कितौ}


\twolineshloka
{वासुदेवस्तु संप्रेक्ष्य पार्थस्य मृदुयुद्धताम्}
{भीष्मं च शरवर्षाणि सृजन्तमनिशं युधि}


\twolineshloka
{प्रतपन्तमिवादित्यं मध्यमासाद्य सेनयोः}
{वरान्वरान्विनिघ्नन्तं पाण्डुपुत्रस्य सैनिकान्}


\twolineshloka
{युगान्तमिव कुर्वाणं भीष्मं यौधिष्ठिरे बले}
{नामृष्यत महाबाहुर्माधवः परवीरहा}


\twolineshloka
{उत्सृज्य रजतप्रख्यान्हयान्पार्थस्य मारिष}
{वासुदेवस्ततो योगी प्रचस्कन्द महारथात्}


\twolineshloka
{अभिदुद्राव भीष्मं स भुजप्रहरणो बली}
{प्रतोदपाणिस्तेजस्वी सिंहवद्विनदन्मुहुः}


\twolineshloka
{दारयन्निव पद्भ्यां स जगतीं जगदीश्वरः}
{क्रोधताम्रेक्षणः कृष्णो जिघांसुरमितद्युतिः}


\twolineshloka
{ग्रसन्निव च तेजांसि तावकानां महाहवे}
{दृष्ट्वा माधवमाक्रन्दे भीष्मायोद्यतमन्तिके}


\twolineshloka
{हतो भीष्मो हतो भीष्म इति तत्रस्म सैनिकाः}
{क्रोशन्तः प्राद्रवन्सर्वे वासुदेवभयातुराः}


\twolineshloka
{पीतकौशेयसंवीतो मणिश्यामो जनार्दनः}
{शुशुभे विद्रवन्भीष्मं विद्युन्माली यथाम्बुदः}


\twolineshloka
{स सिंह इव मातङ्गं यथर्षभ इवर्षभम्}
{अभिदुद्राव वेगेन विनदन्यादवर्षभः}


\threelineshloka
{तमापतन्तं संप्रेक्ष्य पुण्डरीकाक्षमाहवे}
{असंभ्रमं रणे भीष्मो विचकर्ष महद्धनुः}
{उवाच चैव गोविन्दमसंभ्रान्तेन चेतसा}


\twolineshloka
{एह्येहि पुण्डरीकाक्ष देवदेव नमोस्तु ते}
{मामद्य सात्वतश्रेष्ठ तापयस्त महाहवे}


\twolineshloka
{त्वया हि देव संग्रामे हतस्यापि ममानघ}
{श्रेय एव परं कृष्ण लोके भवति सर्वतः}


\threelineshloka
{संभावितोऽस्मि गोविन्द त्रैलोक्येनाद्य संयुगे}
{प्रहरस्व यथेष्टं वै दासोऽस्मि तव चानघ ॥सञ्जय उवाच}
{}


\twolineshloka
{अन्वगेव ततः पार्थः समभिद्रुत्य केशवम्}
{निजग्राह महाबाहुर्बाहुभ्यां परिगृह्य वै}


\twolineshloka
{निगृह्यमाणः पार्थेन कृष्णो राजीवलोचनः}
{जगामैवैनमादाय वेगेन पुरुषोत्तमः}


\twolineshloka
{पार्थस्तु विष्टभ्य बलाच्चरणौ परवीरहा}
{निजग्राह हृषीकेशं कथंचिद्दशमे पदे}


\twolineshloka
{तत एवमुवाचार्तः क्रोधपर्याकुलेक्षणम्}
{निःश्वसन्तं यथा नागमर्जुनः प्रणयात्सखा}


\twolineshloka
{निवर्तस्व महाबाहो नानृतं कर्तुमर्हसि}
{यत्त्वया कथितं पूर्वं न योत्स्यामीति केशव}


\twolineshloka
{मिथ्यावादीति लोकास्त्वां कथयिष्यन्ति माधव}
{ममैष भारः सर्वो हि हनिष्यामि पितामहम्}


\twolineshloka
{शपे केशव शस्त्रेण सत्येन सुकृतेन च}
{अन्तं यथा गमिष्यामि शत्रूणां शत्रुसूदन}


\twolineshloka
{अद्यैव पश्य दुर्धर्षं पात्यमानं महारथम्}
{तारापतिमिवापूर्णमन्तकाले यदृच्छया}


\threelineshloka
{माधवस्तु वचः श्रुत्वा फल्गुनस्य महात्मनः}
{`अभवत्परमप्रीतो दृष्ट्वा पार्थस्य विक्रमम्'}
{न किंचिदुक्त्वा सक्रोध आरुरोह रथं पुनः}


\twolineshloka
{तौ रथस्थौ नरव्याघ्रौ भीष्मः शान्तनवः पुनः}
{ववर्ष शरवर्षेण मेघो वृष्ट्या यथाऽचलौ}


\twolineshloka
{प्राणानादत्त योधानां पिता देवव्रतस्तव}
{गभस्तिभिरिवादित्यस्तेजांसि शिशिरात्यये}


\twolineshloka
{यथा कुरूणां सैन्यानि बभञ्जुर्युधि पाण्डवाः}
{तथा पाण्डवसैन्यानि बभञ्ज युधि ते पिता}


\twolineshloka
{हतविद्रुतसैन्यास्तु निरुत्साहा विचेतसः}
{निरीक्षितुं न शेकुस्ते भीष्ममप्रतिमं रणे}


\twolineshloka
{मध्यं गतमिवादित्यं प्रतपन्तं स्वतेजसा}
{ते वध्यमाना भीष्मेण शतशोऽथ सहस्रशः}


\twolineshloka
{कुर्वाणं समरे कर्माण्यतिमानुषविक्रमम्}
{वीक्षांचक्रुर्महाराज पाण्डवा भयपीडिताः}


\threelineshloka
{तथा पाण्डवसैन्यानि द्राव्यमाणानि भारत}
{त्रातारं नाध्यगच्छन्त गावः पङ्कगता इव}
{पिपीलिका इव क्षुण्णा दुर्बला बलिना रणे}


\twolineshloka
{तथैव योधा राजेन्द्र भीष्मेणामित्रघातिना}
{समरे मृदिताः सर्वे पाण्डवाः सह सृञ्जयैः}


\twolineshloka
{महारथं भारत दुष्प्रकम्पंशरौघिणं प्रतपन्तं नरेन्द्रान्}
{भीष्मं न शेकुः प्रतिवीक्षितुं तेशरार्चिषं सूर्यमिवातपन्तम्}


\twolineshloka
{विमृद्गतस्तस्य तु पाण्डुसेना-मस्तं जगामाथ सहस्ररश्मिः}
{ततो हि भीष्मः सबलान्ससैन्या-न्न्यवारयत्पाण्डुसुताञ्शरौघैः}


\threelineshloka
{जघान चैतान्सुभृशं महाबलोमहाव्रतः पाण्डुसुतान्महास्त्रैः}
{रणे करूणाधिपचेदिपैर्बलै-र्वृतान्सदा चक्रधरस्य पश्यतः}
{ततो बलानां श्रमकर्शितानांमनोऽवहारं प्रति संबभूव}


\chapter{अध्यायः १०७}
\twolineshloka
{सञ्जय उवाच}
{}


\twolineshloka
{युध्यतामेव तेषां तु भास्करेऽस्तमुपागते}
{सन्ध्या समभवद्धोरा नापश्याम ततो रणम्}


\twolineshloka
{ततो युधिष्ठिरो राजा सन्ध्यां संदृश्य भारत}
{वध्यमानं च भीष्मेण त्यक्तास्त्रं भयविह्वलम्}


\twolineshloka
{स्वसैन्यं च परावृत्तं पलायनपरायणम्}
{भीष्मं च युधि संरब्धं पीडयन्तं महारथम्}


\threelineshloka
{सोमकांश्च जितान्दृष्ट्वा निरुत्साहान्महारथान्}
{` निशामुखं च संप्रेक्ष्य घोररूपं भयानकम्}
{'चिन्तयित्वा ततो राज्ञामपहारमकारयत्}


\twolineshloka
{ततोऽपहारं सैन्यानां चक्रे राजा युधिष्ठिरः}
{तथैव तव सैन्यानामपहारे ह्यभूत्तदा}


\twolineshloka
{ततोऽपहारं सैन्यानां कृत्वा तत्र महारथाः}
{न्यविशन्त कुरुश्रेष्ठ संग्रामे क्षतविक्षताः}


\twolineshloka
{भीष्मस्य समरे कर्म चिन्तयानास्तु पाण्डवाः}
{नालभन्त तदा शान्तिं भीष्मबाणप्रपीडिताः}


\twolineshloka
{भीष्मोऽपि समरे जित्वा पाण्डवान्सह सृञ्जयान्}
{पूज्यमानस्तव सुतैर्वन्द्यमानश्च भारत}


\twolineshloka
{न्यविशत्कुरुभिः सार्धं हृष्टरूपैः समन्ततः}
{ततो रात्रिः समभवत्सर्वभूतप्रमोहिनी}


\twolineshloka
{तस्मिन्रात्रिमुखे घोरे पाण्डवा वृष्णिभिः सह}
{सृञ्जयाश्च दुराधर्षा मन्त्राय समुपाविशन्}


\twolineshloka
{आत्मनिःश्रेयसं सर्वे प्राप्तकालं महाबलाः}
{मन्त्रयामासुरव्यग्रा मन्त्रनिश्चयकोविदाः}


\twolineshloka
{ततो युधिष्ठिरो राजा मन्त्रयित्वा चिरं नृप}
{वासुदेवं समुद्वीक्ष्य वचनं चेदमाददे}


\twolineshloka
{कृष्ण पश्य माहात्मानं भीष्मं भीमपराक्रमम्}
{गजं नलवनानीव विमृद्गन्तं बलं मम}


\twolineshloka
{न चैवैनं महात्मानमुत्सहामो निरीक्षितुम्}
{लेलिह्यमानं सैन्येषु प्रवृद्धमिव पावकम्}


\twolineshloka
{यथा घोरो महानागस्तक्षको वै विषोल्बणः}
{तथा भीष्मो रणे क्रुद्धस्तीक्ष्णशस्त्रः प्रतापवान्}


\twolineshloka
{गृहीतचापः समरे प्रमुञ्चन्निशिताञ्छरान्}
{शक्यो जेतुं यमः क्रुद्धो वज्रपाणिश्च देवराट्}


\twolineshloka
{वरुणः पाशभृच्चापि सगदो वा धनेश्वरः}
{न तु भीष्मः सुसंक्रुद्धः शक्यो जेतुं महाहवे}


\twolineshloka
{सोऽहमेवं गते कृष्ण निमग्नः शोकसागरे}
{आत्मनो बुद्धिदौर्बल्याद्भीष्ममासाद्य संयुगे}


\twolineshloka
{वनं यास्यामि दुर्धर्ष श्रेयो वै तत्र मे गतम्}
{न युद्धं रोचते कृष्ण हन्ति भीष्मो हि नः सदा}


\twolineshloka
{यथा प्रज्वलितं वह्निं पतङ्गः समभिद्रवन्}
{एकतो मृत्युमभ्येति तथाऽहं भीष्ममेयिवान्}


\twolineshloka
{क्षयं नीतोऽस्मि वार्ष्णेय राज्यहेतोः पराक्रमी}
{भ्रातरश्चैव मे शुराः सायकैर्भृशपीडिताः}


\twolineshloka
{मत्कृते भ्रातृसौहार्दाद्राज्यभ्रष्टा वनं गताः}
{परिक्लिष्टा तथा कृष्णा मत्कृते मधुसूदन}


\twolineshloka
{जीवितं बहु मन्येऽहं जीवितं ह्यद्य दुर्लभम्}
{जीवितस्याद्य शेषेण चरिष्ये धर्ममुत्तमम्}


\threelineshloka
{यदि तेऽहमनुग्राह्यो भ्रातृभिः स केशव}
{स्वधर्मस्याविरोधेन हितं व्याहर केशव ॥सञ्जय उवाच}
{}


\threelineshloka
{एवं श्रुत्वा वचस्तस्य कारुण्याद्बहुविस्तरम्}
{प्रत्युवाच ततः कृष्णः सान्त्वयानो युधिष्ठिरम् ॥श्रीभगवानुवाच}
{}


\twolineshloka
{धर्मपुत्र विषादं त्वं मा कृथाः सत्यसंगर}
{यस्य ते भ्रातरः शूरा दुर्जयाः सत्रुसूदनाः}


\twolineshloka
{अर्जुनो भीमसेनश्च वाय्वग्निसमतेजसौ}
{माद्रीपुत्रौ च विक्रान्तौ त्रिदशानामिविश्वरौ}


\twolineshloka
{मां वा नियुङ्क्ष्व सौहार्दाद्योत्स्ये भीष्मेण पाण्डव}
{त्वप्रयुक्तो महाराज किं न कुर्यां महाहवे}


\twolineshloka
{हनिष्यामि रणे भीष्ममाहूय पुरुषर्षभम्}
{पश्यतां धार्तराष्ट्राणां यदि नेच्छति फल्गुनः}


\twolineshloka
{यदि भीष्मे हते वीरे जयं पश्यसि पण्डव}
{हन्तास्म्येकरथेनाद्य कुरुवृद्धं पितामहम्}


\twolineshloka
{पश्य मे विक्रमं राजन्महेन्द्रस्येव संयुगे}
{विमुञ्चन्तं महास्त्रामि पातयिष्यामि तं रथात्}


\twolineshloka
{यः शत्रुः पाण्डुपुत्राणां मच्छत्रुः स न संशयः}
{मदर्था भवदर्था ये ये मदीयास्तवैव ते}


\twolineshloka
{तव भ्राता मम सखा संबन्धी शिष्य एव च}
{मांसान्युत्कृत्य दास्यामि फल्गुनार्थे महीपते}


\twolineshloka
{एष चापि नरव्याघ्रो मत्कृते जीवितं त्यजेत्}
{एष नः समयस्तात तारयेम परस्परम्}


\twolineshloka
{स मां नियुङ्क्ष्व राजेन्द्र यावत्सज्जो भवाम्यहम्}
{प्रतिज्ञातमुपप्लाव्ये यत्तत्पार्थेन पूर्वतः}


\twolineshloka
{पातयिष्यामि गाङ्गेयमित्युलूकस्य संनिधौ}
{परिरक्ष्यमिदं तावद्वचः पार्थस्य धीमतः}


\twolineshloka
{अनुज्ञातेन पार्थेन मया कार्यं न संशयः}
{अथवा फल्गुनस्यैष भारः परिमितो रणे}


\twolineshloka
{स हनिष्यति संग्रामे भीष्मं परपुरंजयम्}
{अशक्यमपि कुर्याद्धि रणे पार्थः समुद्यतः}


\twolineshloka
{त्रिदशान्वा समुद्युक्तान्सहितान्दैत्यदानवैः}
{निहन्त्यादर्जुनः सङ्ख्ये किमु भीष्मं नराधिप}


\threelineshloka
{विपरीतो महावीर्यो गतसत्वोऽल्पजीवनः}
{भीष्मः शान्तनवो नूनं कर्तव्यं नावबुध्यते ॥युधिष्ठिर उवाच}
{}


\twolineshloka
{एवमेतन्महाबाहो यथा वदसि माधव}
{सर्वे ह्येते न पर्याप्तास्तव वेगविधारणे}


\twolineshloka
{नियतं समावाप्स्यामि सर्वमेतद्यथेप्सितम्}
{यस्य मे पुरुषव्याघ्र भवान्पक्षे व्यवस्थितः}


\twolineshloka
{सेन्द्रानपि रणे देवाञ्जयेयं जयतां वर}
{त्वया नाथेन गोविन्द किमु भीष्मं महारथम्}


\twolineshloka
{न तु त्वामनृतं कर्तुमुत्सहे स्वात्मगौरवात्}
{अयुध्यमानाः सहाय्यं यथोक्तं कुरु माधव}


\twolineshloka
{समयस्तु कृतः कश्चिन्मम भीष्मेण संयुगे}
{मन्त्रयिष्ये तवार्थाय न तु योत्स्ये कथंचन}


\twolineshloka
{दुर्योधनार्थं योत्स्यामि सत्यमेतदिति प्रभो}
{स हि राज्यस्य मे दाता मन्त्रस्यैव च माधव}


\twolineshloka
{तस्माद्देवव्रतं भूयो वधोपायार्थमात्मनः}
{भवता सहिताः सर्वे प्रयाम मधुसूदन}


\twolineshloka
{तद्वयं सहिता गत्वा भीष्ममाशु नरोत्तमम्}
{रुचिते तव पृच्छामि मन्त्रं वार्ष्णेय माचिरम्}


\twolineshloka
{स वक्ष्यति हितं वाक्यं सत्यमस्माज्जनार्दन}
{यथा च वक्ष्यते कृष्ण तथा कर्तास्मि संयुगे}


\twolineshloka
{स नो जयस्य दाता स्यान्मन्त्रस्य च दृढव्रतः}
{बालाः पित्रा विहीनाश्च तेन संवर्धिता वयम्}


\threelineshloka
{तं चेत्पितामहं वृद्धं हन्तुमिच्छामि माधव}
{पितुः पितरमिष्टं च धिगस्तु क्षत्रजीविकाम् ॥सञ्जय उवाच}
{}


\twolineshloka
{ततोऽब्रवीन्महाराज वार्ष्णेयः कुरुनन्दनम्}
{रोचते मे महाप्राज्ञ राजेन्द्र तव भाषितम्}


\twolineshloka
{देवव्रतः कृती भीष्मः प्रेक्षितेनापि निर्दहेत्}
{गम्यतां स्ववधोपायं प्रष्टुं सागरगासुतम्}


\twolineshloka
{वक्तमर्हति सत्यं स त्वया पृष्टो विशेषतः}
{ते वयं तत्र गच्छामः प्रष्टुं कुरुपितामहम्}


\threelineshloka
{गत्वा शान्तनवं वृद्धं मन्त्रं पृच्छाम भारत}
{स नो दास्यति मन्त्रं यं तेन योत्स्यामहे परान् ॥सञ्जय उवाच}
{}


\twolineshloka
{एवमामन्त्र्य ते वीराः पाण्डवाः पाण्डुपूर्वज}
{जग्मुस्ते सहिताः सर्वे वासुदेवश्च वीर्यवान्}


\twolineshloka
{विमुक्तशस्त्रकवचा भीष्मस्य सदनं प्रति}
{प्रविश्य च तदा भीष्मं शिरोभिः प्रणिपेदिरे}


\twolineshloka
{पूजयन्तो महाराज पाण्डवा भरतर्षभम्}
{प्रणम्य शिरसा चैनं भीष्मं शरणमभ्ययुः}


\twolineshloka
{तानुवाच महाबाहुर्भीष्मः कुरुपितामहः}
{स्वागतं तव वार्ष्णेय स्वागतं ते धनञ्जय}


\twolineshloka
{स्वागतं धर्मपुत्राय भीमाय यमयोस्तथा}
{किं वा कार्यं करोम्यद्य युष्माकं प्रीतिवर्धनं}


\threelineshloka
{`युद्धादन्यत्र हे वत्साः प्रीयन्तां मा विशङ्कथ}
{'सर्वात्मनापि कर्तास्मि यदपि स्यात्सदुष्करम्}
{तथा ब्रुवाणं गाङ्गेयं प्रीतियुक्तं पुनःपुनः}


\twolineshloka
{उवाच राजा दीनात्मा प्रीतियुक्तमिदं वचः}
{कथं जयेम सर्वज्ञ कथं राज्यं लभेमहि}


\twolineshloka
{प्रजानां संशयो न स्यात्कथं तन्मे बद प्रभो}
{भवान्हि नो वधोपायं ब्रवीतु स्वयमात्मनः}


\twolineshloka
{भवन्तं समरे वीर विषहेम कथं वयम्}
{न हि ते सूक्ष्ममप्यस्ति रन्ध्रं कुरुपितामह}


\twolineshloka
{मण्डलेनैव धनुषा दृश्यसे संयुगे सदा}
{आददानं संदधानं विकर्षन्तं धनुर्न च}


\twolineshloka
{पश्यामस्त्वां महाबाहो रथे सूर्यमिवापरम्}
{रथाश्वनरनागानां हन्तारं परवीरहन्}


\twolineshloka
{कोऽथवोत्सहते जेतुं वां पुमान्भरतर्षभ}
{वर्षता शरवर्षाणि महान्ति पुरुषर्षभ}


\twolineshloka
{क्षयं निता हि पृतना संयुगे महती मम}
{यथा युधि जयेम त्वां यथा राज्यं भृशं मम}


\threelineshloka
{मम सैन्यस्य च क्षेमं तन्मे ब्रूहि पितामह}
{6-107-69xसञ्जयउवाच}
{ततोऽब्रवीच्छान्तनवः पाण्डवान्पाण्डुपूर्वज}


\twolineshloka
{न कथंचन कौन्तेय मयि जीवति संयुगे}
{जयो भवति सर्वज्ञ सत्यमेतद्ब्रवीमि ते}


\twolineshloka
{निर्जिते मयि युद्धेन रणे जेष्यथ पाण्डवाः}
{क्षिप्रं मयि प्रहरत यदीच्छथ रणे जयम्}


\fourlineindentedshloka
{अनुजानामि वः पार्थाः प्रहरध्वं यथासुखम्}
{एवं हि सुकृतं मन्ये भवतां विदितो ह्यहम्}
{हते मयि हतं सर्वं तस्मादेवं विधीयताम् ॥युधिष्ठिर उवाच}
{}


\twolineshloka
{ब्रूहि तस्मादुपायं नो यथा युद्धे जयेमहि}
{भवन्तं समरे क्रुद्धं दण्डहस्तमिवान्तकम्}


\threelineshloka
{शक्यो वज्रधरो जेतुं वरुणोऽथ यमस्तथा}
{न भवान्समरे शक्यः सेन्द्रैरपि सुरासुरैः ॥भीष्म उवाच}
{}


% Check verse!
सत्यमेतन्महाबाहो यथा वदसि पाण्डव
\threelineshloka
{नाहं जेतुं रणे शक्यः सेन्द्रैरपि सुरासुरैः}
{आत्तशस्त्रो रणे यत्तो गृहीतवरकार्मुकः}
{ततो मां न्यस्तशस्त्रं तु एते हन्युर्महारथाः}


\twolineshloka
{निक्षिप्तशस्त्रे पतिते विमुक्तकवचध्वजे}
{द्रवमाणे च भीते च तवास्मीति च वादिनि}


\twolineshloka
{`स्त्रीजिते स्त्रीप्रधाने च स्त्रीप्रधायिनि धर्मज'स्त्रियां स्त्रीनामधेये च विकले चैकपुत्रिणि}
{अप्रसूते च षण्डे च न युद्धं रोचते मम}


\twolineshloka
{इमं मे शृणु राजेन्द्र संकल्पं पूर्वचिन्तितम्}
{असंकल्पध्वजं दृष्ट्वा न युध्येयं कदाचन}


\twolineshloka
{य एष द्रौपदो राजंस्तव सैन्ये महारथः}
{शिखण्डी समरामर्षी शूरश्च समितिंजयः}


\twolineshloka
{यथाऽभवच्च स्त्रीपूर्वं पश्चात्पुंस्त्वं समागतः}
{जानन्ति च भवन्तोऽपि सर्वमेतद्यथातथम्}


\twolineshloka
{अर्जुनः समरे शूरः पुरस्कृत्य शिखण्डीनम्}
{मामेव विशिखैस्तीक्ष्णैरभिद्रवतु दंशितः}


\twolineshloka
{असंकल्पध्वजे तस्मिन्स्त्रीपूर्वे च विशेषतः}
{न प्रहर्तुमभीप्सामि गृहीतेषु कथंचन}


\twolineshloka
{तदन्तरं समासाद्य पाण्डवो मां धनञ्जयः}
{शरैर्घातयतु क्षिप्रं समन्ताद्भरतर्षभ}


\twolineshloka
{न तं पश्यामि लोकेष मां हन्याद्यः समुद्यतम्}
{ऋते कृष्णान्महाभागात्पाण्डवाद्वा धनञ्जयात्}


\threelineshloka
{पार्षतं तु पुरोधाय क्लीबमद्य ममाग्रतः}
{आत्तशस्त्रो रणे यत्तो गृहीतवरकार्मुकः}
{मां पातयतु बीभत्सुरेवं तव जयो ध्रुवम्}


\threelineshloka
{एतत्कुरुष्व कौन्तेय यथोक्तं मम सुव्रत}
{ततो जेष्वसि संग्रामे धार्तराष्ट्रान्समागतान् ॥सञ्जय उवाच}
{}


\twolineshloka
{तेऽनुज्ञातास्ततः पार्था जग्मुः स्वशिबिरं प्रति}
{अभिवाद्य महात्मानं भीष्मं कुरुपितामहम्}


\twolineshloka
{तथोक्तवति गाङ्गेये परलोकाय दीक्षिते}
{अर्जुनो दुःखसंतप्तः सव्रीडमिदमब्रवीत्}


\twolineshloka
{गुरुणा कुरुवृद्धेन कृतप्रज्ञेन धीमता}
{पितामहेन संग्रामे कथं योद्धास्मि माधव}


\twolineshloka
{क्रीडता हि मया बाल्ये वासुदेव महामनाः}
{पांसुरूषितगात्रेण महात्मा परुषीकृतः}


\twolineshloka
{यस्याहमधिरुह्याङ्कं वालः किल गदाग्रज}
{तातेत्यवोचं पितरं पितुः पाण्डोर्महात्मनः}


\twolineshloka
{नाहं तातस्तव पितुस्तातोऽस्मि तव भारत}
{इति मामब्रवीद्बाल्ये यः स वध्यः कथं मया}


\twolineshloka
{कामं वध्यतु सैन्यं मे नाहं योत्स्ये महात्मना}
{जयो वास्तु वधो वा मे कथं वा कृष्ण मन्यसे}


\threelineshloka
{` कथमस्माद्विधः कृष्ण जानन्धर्मं सनातनम्}
{न्यस्तशस्त्रे च वृद्धे च प्रहरेद्धि पितामहे ॥'श्रीवासुदेव उवाच}
{}


\twolineshloka
{प्रतिज्ञाय वधं जिष्णो पुरा भीष्मस्य संयुगे}
{क्षत्रधर्मे स्थितः पार्थ कथं नैनं हनिष्यसि}


\twolineshloka
{पातयैनं रथात्पार्थ क्षत्रियं युद्धदुर्मदम्}
{नाहत्वा युधि गाङ्गेयं विजयस्ते भविष्यति}


\twolineshloka
{दृष्टमेतत्पुरा देवैर्भविष्यत्यवशस्य ते}
{यद्दृष्टं हि पुरा पार्थ तत्तथा न तदन्यथा}


\twolineshloka
{न हि भीष्मं दुराधर्षं व्यात्ताननमिवान्तकम्}
{त्वदन्यः शक्नुयाद्योद्धुमपि वज्रधरः स्वयम्}


\twolineshloka
{जहि भीष्मं स्थिरो भूत्वा शृणु चेदं वचो मम}
{यथोवाच पुरा शक्रं महाबुद्धिर्बृहस्पतिः}


\twolineshloka
{ज्यायांसमपि चेद्वृद्धं गुणैरपि समन्वितम्}
{आततायिनमायान्तं हन्याद्धातकमात्मनः}


\threelineshloka
{शाश्वतोऽयं स्थितो धर्मः क्षत्रियाणां धनञ्जय}
{योद्धव्यं रक्षितव्यं च यष्टव्यं चानसूयुभिः ॥अर्जुन उवाच}
{}


\twolineshloka
{शिखण्डी निधनं कृष्ण भीष्मस्य भविता ध्रुवम्}
{दृष्ट्वैव हि सदा भीष्मः पाञ्चाल्यं विनिवर्तते}


\twolineshloka
{ते वयं प्रमुखे तस्य पुरस्कृत्य शिखण्डिनम्}
{गाङ्गेयं पातयिष्याम उपायेनेति मे मतिः}


\twolineshloka
{अहमन्यान्महेष्वासान्वारयिष्यामि सायकैः}
{शिखण्ड्यपि युधां श्रेष्ठं भीष्ममेवाभियोधयेत्}


\twolineshloka
{श्रुतं हि कुरुमुख्यस्य नाहं हन्यां शिखण्डिनम्}
{कन्या ह्येषा पुरा भूत्वा पुरुषः समपद्यत}


\twolineshloka
{`अर्जुनस्य वचः श्रुत्वा भीष्मस्य वधसंयुतम्}
{जहृषुर्हृष्टरोमाणः सकृष्णाः पाण्डवास्तदा ॥'}


\threelineshloka
{इत्येवं निश्चयं कृत्वा पाण्डवाः सहमाधवाः}
{अनुमान्य महात्मानं प्रययुर्हृष्टमानसाः}
{शयनानि यथा स्वानि भेजिरे पुरुषर्षभाः}


\chapter{अध्यायः १०८}
\twolineshloka
{धृतराष्ट्र उवाच}
{}


\threelineshloka
{कथं शिखण्डी गाङ्गेयमभ्यवर्तत संयुगे}
{पाण्डवाश्च कथं भीष्मं तन्ममाचक्ष्व सञ्जय ॥सञ्जय उवाच}
{}


\twolineshloka
{ततः प्रभाते विमले सूर्यस्योदयनं प्रति}
{ताड्यमानासु भेरीषु मृदङ्गेष्वानकेषु च}


\twolineshloka
{ध्मायमानेषु शङ्खेषु पाण्डरेषु समन्ततः}
{शिखण्डिनं पुरस्कृत्य निर्याप्ताः पाण्डवा युधि}


\twolineshloka
{कृत्वा व्यूहं महाराज सर्वशत्रुनिबर्हणम्}
{शिखण्डी सर्वसैन्यानामग्र आसीद्विशांपते}


\twolineshloka
{चक्ररक्षौ ततस्तस्य भीमसेनधनंजयौ}
{पृष्ठतो द्रौपदेयाश्च सौभद्रश्चैव वीर्यवान्}


\twolineshloka
{सात्यकिश्चेकितानश्च तेषां गोप्ता महारथः}
{धृष्टद्युम्नस्ततः पश्चात्पाञ्चालैरभिरक्षितः}


\twolineshloka
{ततो युधिष्ठिरो राजा यमाभ्यां सहितः प्रभुः}
{प्रययौ सिंहनादेन नादयन्भरतर्षभ}


\twolineshloka
{विराटस्तु ततः पश्चात्स्वेन सैन्येन संवृत्तः}
{द्रुपदश्च महाबाहो ततः पञ्चादुपाद्रवत्}


\twolineshloka
{केकया भ्रातरः पञ्च धृष्टकेतुश्च वीर्यवान्}
{जघनं पालयामासुः पाण्डवेयश्च राक्षसः}


\twolineshloka
{एवं व्यूह्य महासैन्यं पाण्डवास्तव वाहिनीम्}
{अभ्यद्रवन्त संग्रामे त्यक्त्वा जीवितमात्मनः}


\twolineshloka
{तथैव कुरवो राजन्भीष्मं कृत्वा महारथम्}
{अग्रतः सर्वसैन्यानां प्रययुः पाण्डवान्प्रति}


\twolineshloka
{पुत्रैस्तव दुराधर्षो रक्षितः सुमहाबलैः}
{ततो द्रोणो महेष्वासः पुत्रश्चास्य महाबलः}


\twolineshloka
{भगदत्तस्ततः पश्चाद्गजानीकेन संवृतः}
{कृपश्च कृतवर्मा च भगदत्तमनुव्रतौ}


\twolineshloka
{काम्भोजराजो बलवांस्ततः पश्चात्सुदक्षिणः}
{मागधश्च जयत्सेनः सौबलश्च बृहद्बलः}


\twolineshloka
{तथैवान्ये महेष्वासाः सुशर्मप्रमुखा नृपाः}
{जघनं पालयामासुस्तव सैन्यस्य भारत}


\twolineshloka
{दिवसेदिवसे प्राप्ते भीष्मः शान्तनवो युधि}
{आसुरानकरोद्व्यूहान्पैशाचानथ राक्षसान्}


\twolineshloka
{ततः प्रववृते युद्धं तव तेषां च भारत}
{अन्योन्यं निघ्नतां राजन्यमराष्ट्रविवर्धनम्}


\twolineshloka
{अर्जुनप्रमुखाः पार्थाः पुरस्कृत्य शिखण्डिनम्}
{भीष्मं युद्धेऽभ्यवर्तन्त किरन्तो विविधाञ्शरान्}


\twolineshloka
{तत्र भारत भीमेन ताडितास्तावकाः शरैः}
{रुधिरौघपरिक्लिन्नाः परलोकं ययुस्तदा}


\twolineshloka
{नकुलः सहदेवश्च सात्यकिश्च महारथः}
{तव सैनयं समासाद्य पीडयामासुरोजसा}


\twolineshloka
{ते वध्यामानाः समरे तावका भरतर्षभ}
{नाशक्नुवन्वारयितुं पाण्डवानां महद्बलम्}


\twolineshloka
{ततस्तु तावकं सैन्यं वध्यमानं समन्ततः}
{संप्राद्रवद्दशः दिशः काल्यमानं महारथैः}


\threelineshloka
{त्रातारं नाध्यगच्छन्त तावका भरतर्षभ}
{वध्यमानाः शितैर्बाणैः पाण्डवैः सह सृञ्जयैः ॥धृतराष्ट्र उवाच}
{}


\twolineshloka
{पीड्यमानं बलं दृष्ट्वा पार्थैर्भीष्मः पराक्रमी}
{यदकार्षीद्रणे क्रुद्धस्तन्ममाचक्ष्व सञ्जय}


\threelineshloka
{कथं वा पाण्डवा युद्धे प्रत्युद्याताः परंतपाः}
{निघ्नन्तो मामकान्वीरांस्तन्ममाचक्ष्व सञ्जय ॥सञ्जय उवाच}
{}


\twolineshloka
{आचक्षे ते महाराज यदकार्षीत्पिता तव}
{पीडिते तव पुत्रस्य सैन्ये पाण्डव सृञ्जयैः}


\twolineshloka
{प्रंहृष्टमनसः शूराः पाण्डवाः पाण्डुपूर्वज}
{अभ्यवर्तन्त निघ्नन्तस्तव पुत्रस्य वाहिनीम्}


\twolineshloka
{तं विनाशं मनुष्येन्द्र नरवारणवाजिनाम्}
{नामृष्यत तदा भीष्मः सैन्यघातं रणे परैः}


\twolineshloka
{स पाणडवान्महेष्वासः पञ्चालांश्चैव सृञ्जयान्}
{नाराचैर्वत्सदन्तैश्च शितैरञ्जलिकैस्तथा}


\twolineshloka
{अभ्यवर्षत दुर्धर्षस्त्यक्त्वा जीवितमात्मनः}
{स पाण्डवानां प्रवरान्पञ्च राजन्महारथान्}


\twolineshloka
{आत्तशस्त्रो रणे यत्नाद्वारयामास सायकैः}
{नानाशस्त्रास्त्रवर्षैस्तान्वीर्यामर्षप्रवेरितैः}


\twolineshloka
{निजघ्ने समरे क्रुद्धो हस्त्यश्वं चामितं बहु}
{रथिनोऽपातयद्राजन्रथेभ्यः पुरुषर्षभ}


% Check verse!
सादिनश्चाश्वपृष्ठेभ्यः पादातांश्च समागतान्
\twolineshloka
{गजारोहान्गजेभ्यश्च परेषां जयकारिणः}
{तमेकं समरे भीष्मं त्वरमाणं महारथम्}


\twolineshloka
{पाण्डवाः समवर्तन्त वज्रहस्तमिवासुराः}
{शक्राशनिसमस्पर्शान्विमुञ्चन्निशिताञ्छरान्}


\twolineshloka
{दिक्ष्वदृश्यत सर्वासु घोरं सन्धारयन्वपुः}
{मण्डलीभूतमेवास्य नित्यं धनुरदृश्यत}


% Check verse!
संग्रामे युध्यमानस्य शक्रचापोपमं महत् ॥तदृष्ट्वा समरे कर्म पुत्रास्तव विशांपते
\twolineshloka
{विस्मायं परमं गत्वा पितामहमपूजयन्}
{पार्था विमनसो भूत्वा प्रैक्षन्त पितरं तव}


\twolineshloka
{युध्यमानं रणे शूरं विप्रचित्तिमिवामराः}
{न चैनं वारयामासुर्व्यात्ताननमिवान्तकम्}


\twolineshloka
{दशमेऽहनि संप्राप्ते रथानीकं शिखण्डिनः}
{अदहन्निशितैर्बाणैः कृष्णवर्त्मेव काननम्}


\twolineshloka
{तं शिखण्डी त्रिभिर्बाणैरभ्यविध्यत्सनान्तरे}
{आशीविषमिव क्रुद्धं कालसृष्टमिवान्तकम्}


\twolineshloka
{स तेनातिभृशं विद्धः प्रेक्ष्य भीष्मः शिखण्डिनम्}
{पुनर्नालोकयत्क्रुद्धः प्रहसन्निदमब्रवीत्}


\twolineshloka
{कामं प्रहर वा मा वा न त्वां योत्स्ये कथंचन}
{यैव हि त्वं कृता धात्रा सैव हि त्वं शिखण्डिनी}


\twolineshloka
{तस्य तद्वचनं श्रुत्वा शिखण्डी क्रोधमूर्च्छितः}
{उवाच भीष्मं समरे सृक्विणी परिलेलिहन्}


\twolineshloka
{जानामि त्वां महाबाहो क्षत्रियाणां भयंकरम्}
{मया श्रुतं च ते युद्धं जामदग्न्येन वै सह}


\twolineshloka
{दिव्यश्च ते प्रभावोऽयं मया च बहुशः श्रुतः}
{जनन्नपि प्रभावं ते योत्स्येऽद्याहं त्वया सह}


% Check verse!
पाण्डवानां प्रियं कुर्वन्नात्मनश्च नरोत्तम ॥अद्य त्वां योधयिष्यामि रणे पुरुषसत्तम
\twolineshloka
{ध्रुवं च त्वां हनिष्यामि शपे सत्येन तेऽग्रतः}
{एतच्छ्रुत्वा च मद्वाक्यं यत्कृत्यं तत्समाचर}


\threelineshloka
{कामं युध्यस्व वा मा वा न मे जीवन्प्रमोक्ष्यसे}
{सुदृष्टः क्रियतां भीष्म लोकोऽयं समितिंजय ॥सञ्जय उवाच}
{}


\twolineshloka
{एवमुक्त्वा ततो भीष्मं पञ्चभिर्नतपर्वभिः}
{अविध्यत रणे भीष्मं प्रतुदन्वाक्यसायकैः}


\twolineshloka
{तस्य तद्वचनं श्रुत्वा सव्यसाची महारथः}
{कालोऽयमिति संचिन्त्य शिखण्डिनमचोदयत्}


\twolineshloka
{अहं त्वामनुयास्यामि परान्विद्रावयञ्शरैः}
{अभिद्रव सुसंरब्धो भीष्मं भीमपराक्रमम्}


\twolineshloka
{न हि ते संयुगे पीडां शक्ताः कर्तुं महाबलाः}
{तस्मादद्य महाबाहो यत्नाद्भीष्यमभिद्रव}


\twolineshloka
{अहत्वा समरे भीष्म यदि यास्यसि मारिष}
{अवहास्योऽस्य लोकस्य भविष्यसि मया सह}


\twolineshloka
{नावहास्या यथा वीर भवेम परमाहवे}
{तथा कुरु रणे यत्नं साधयस्व पितामहम्}


\twolineshloka
{कुरूंश्च सहितान्सर्वान्यतमानान्महारथान्}
{अहमावारयिष्यामि सावयस्व पितामहम्}


\twolineshloka
{द्रोणं च द्रोणपुत्रं च कृपं चाथ सुयोधनम्}
{चित्रसेनं विकर्णं च सैन्धवं च यजद्रथम्}


\twolineshloka
{विन्दानुविन्दावावन्त्यौ काम्भोजं च सुदक्षिणम्}
{भगदत्तं तथा शूरं मागदं च महाबलम्}


\twolineshloka
{सौमदत्तिं तथा शूरमार्श्यशृङ्गिं च राक्षसम्}
{त्रिगर्तराजं च रणे सह सर्वैर्महारथैः}


\threelineshloka
{अहमावारयिष्यामि वेलेव मकरालयम्}
{कुरूंश्च सहितान्सर्वान्युध्यमानान्महाबलान्}
{निवारयिष्यामि रणे साधयस्व पितामहम्}


\chapter{अध्यायः १०९}
\twolineshloka
{धृतराष्ट्र उवाच}
{}


\twolineshloka
{कथं शिखण्डी गाङ्गेयमभ्यधावत्पितामहम्}
{पाञ्चाल्यः समरे क्रुद्धो धर्मात्मानं यतव्रतम्}


\twolineshloka
{केऽरक्षन्पाण्डवानीके शिखण्डिनमुदायुधाः}
{त्वरमाणास्त्वराकाले जिगीषन्तो महारथाः}


\twolineshloka
{कथं शान्तनवो भीष्मः स तस्मिन्दशमेऽहनि}
{अयुध्यत महावीर्यः पाण्डवैः सह सृञ्जयैः}


\threelineshloka
{न मृष्यामि रणे भीष्मं प्रत्युद्यातं शिखण्डिना}
{कच्चिन्न रथमद्गोऽस्व धनुर्वाऽशीर्यतास्थतः ॥सञ्जय उवाच}
{}


\threelineshloka
{नाशीर्यत धनुश्चास्य रथभङ्गो न चाप्यभूत्}
{युध्यमानस्य संग्रामे भीष्मस्य भरतर्षभ}
{निघ्नतः समरे शत्रून्शरैः सन्नतपर्वभिः}


\twolineshloka
{अनेकशतसाहस्रावकानां महारथाः}
{तथा दन्तिगणा राजन्हायश्चैव सुसञ्जिताः}


% Check verse!
अभ्यवर्तन्त युद्धाय पुरस्कृत्य पितामहम्
\twolineshloka
{यथाप्रतिज्ञं कौरव्य स चापि समिर्तिजयः}
{पार्यानामकरोद्भीष्मः सततं समिति क्षयम्}


\twolineshloka
{युध्यमानं महेष्वासं विनिघ्नन्तं पराञ्शरैः}
{पाञ्चालाः पाण्डवैः सार्ध सर्वतः पर्यवारयन्}


\twolineshloka
{दशमेऽहनि संप्राप्ते ततस्तां रिपुवाहिनीम्}
{कीर्यमाणां शितैर्बाणैः शतशोऽथ सहस्रशः}


\twolineshloka
{न हि भीष्मं महेष्वासं पाण्डवाः पाण्डुपूर्वज}
{अशक्नुवन्रणे जेतुं पाशहस्तमिवानतकम्}


\twolineshloka
{अथोपायान्महाराज सव्यसाची धनञ्जयः}
{त्रासयन्रथिनः सर्वान्बीभत्सुरपराजितः}


\twolineshloka
{सिंहवद्विनदन्नुच्चैर्धनुर्ज्यां विक्षिपन्मुहुः}
{शरौघान्विसृजन्पार्थो व्यचरत्कालवद्रणे}


\twolineshloka
{तस्य शब्देन वित्रस्तास्तावका भारतर्षभ}
{सिंहस्येव मृगा राजन्व्यद्रवन्त महाभयात्}


\twolineshloka
{जयन्तं पाण्डवं दृष्ट्वा त्वत्सैन्यं चाभिपीडितम्}
{दुर्योधनस्ततो भीष्ममब्रवीद्भृशपीडितः}


\twolineshloka
{एष पाण्डुसुतस्तात श्वेताश्वः कृष्णसारथिः}
{दहते मामकान्सर्वान्कृष्णवर्त्मेन काननम्}


\twolineshloka
{पश्य सैन्यानि गाङ्गेय द्रवमाणानि सर्वशः}
{पण्डवेन युधां श्रेष्ठ काल्यमानानि संयुगे}


\twolineshloka
{यथा पशुगणान्पालः संकालयति कानने}
{तथेदं मामकं सैन्यं काल्यते शत्रुतापन}


\twolineshloka
{धनञ्जयशरैर्भग्नं द्रवमाणं ततस्ततः}
{भीमोऽप्येवं दुराधर्पो विद्रावयति मे बलम्}


\twolineshloka
{सात्यकिश्चेकितानश्च माद्रीपुत्रौ च पाण्डवौ}
{अभिमन्युः सुविक्रान्तो वाहिनीं द्रवते मम}


\twolineshloka
{धृष्टद्युम्नस्तथा शूरो राक्षसश्च घटोत्कचः}
{व्यद्रावयेतां सहसा सैन्यं मम महारणे}


\twolineshloka
{वध्यमानस्य सैन्यस्य सर्वैरेतैर्महारथैः}
{नान्यां गतिं प्रपश्यामि स्थाने युद्धे च भारत}


\threelineshloka
{ऋते त्वां पुरुषव्याघ्र देवतुल्यपराक्रमम्}
{पर्याप्तस्तु भवाञ्शीघ्रं पीडितानां गतिर्भव ॥सञ्जय उवाच}
{}


\twolineshloka
{एवमुक्तो महाराज पिता देवव्रतस्तव}
{चिन्तयित्वा मुहूर्तं तु कृत्वा निश्चयमात्मनः}


\twolineshloka
{तव संधारयन्पुत्रमब्रवीच्छंतनोः सुतः}
{दुर्योधन विजानीहि स्थिरो भूत्वा विशांपते}


\twolineshloka
{`पातयिष्ये रिपूनन्यान्पाण्डवान्प्रतिपालयन्}
{प्रतिज्ञातो जयो ह्यद्य पाण्डवानां महात्मनाम् ॥'}


\twolineshloka
{पूर्वकालं तव मया प्रतिज्ञानं महाबल}
{हत्वा दशसहस्राणि क्षत्रियाणां महात्मनाम्}


\twolineshloka
{संग्रामादपयास्यामि ह्येतत्कर्म समाहितम्}
{इति तत्कृतवांश्चाहं यथोक्तं भरतर्षभ}


\twolineshloka
{अद्य चापि महत्कर्म प्रकरिष्ये यथाबलम्}
{अहं वाऽद्य हतः शेष्ये हनिष्ये वाऽद्य पाण्डवान्}


\twolineshloka
{`अशक्याः पाण्डवा जेतुं देवैरपि सवासवैः}
{'किं पुनर्मर्त्यधर्मेण क्षत्रियेण महाबलाः}


\threelineshloka
{अद्य ते पुरुषव्याघ्र प्रतरिष्ये ऋणं तव}
{भर्तृपिण्डकृतं राजन्निहताः पृतनामुखे ॥सञ्जय उवाच}
{}


\twolineshloka
{इत्युक्त्वा भरतश्रेष्ठ क्षत्रियान्प्रवपञ्छरैः}
{आससाद दुराधर्षः पाण्डवानामनीकिनीम्}


\twolineshloka
{अनीकमध्ये तिष्ठन्तं गाङ्गेयं भारतर्षभ}
{आशीविषमिव क्रुद्धं पाण्डवाः प्रत्यवारयन्}


\twolineshloka
{दशमेऽहनि भीष्मस्तु दर्शयञ्शक्तिमात्मनः}
{राजञ्शतसहस्राणि सोऽवधीत्कुरुनन्दन}


\twolineshloka
{पाञ्चालानां च ये श्रेष्ठा राजपुत्रा महारथाः}
{तेषामादत्त तेजांसि जलं सूर्य इवांशुभिः}


\twolineshloka
{हत्वा दशसहस्राणि कुञ्जराणां तरस्विनाम्}
{सारोहाणां महाराज हयानां चायुतं तथा}


\twolineshloka
{पूर्णे शतसहस्रे द्वे पादातानां नरोत्तमः}
{प्रजज्वाल रणे भीष्मो विधूम इव पावकः}


\twolineshloka
{न चैवं पाण्डवेयानां केचिच्छेकुर्निरीक्षितुम्}
{उत्तरं मार्गमास्थाय तपन्तमिव भास्करम्}


\twolineshloka
{ते पाण्डवेवाः संरब्धा महेष्वासेन पीडिताः}
{वधायाभ्यद्रवन्भीष्मं सृञ्जयाश्च महारथाः}


\twolineshloka
{स वध्यमानो बहुभिर्भीष्मः शान्तनवस्तथा}
{अवकीर्णो महेष्वासैः शैलो मेघैरिवावृतः}


\twolineshloka
{पुत्रास्तु तव गाङ्गेयं समन्तात्पर्यवारयन्}
{महत्या सेनया सार्धं ततो युद्धमवर्तत}


\chapter{अध्यायः ११०}
\twolineshloka
{सञ्जय उवाच}
{}


\twolineshloka
{अर्जुनस्तु रणे राजन्दृष्ट्वा भीष्मस्य विक्रमम्}
{शिखण्डिनमथोवाच समभ्योहि पितामहम्}


\threelineshloka
{न चापि भीस्त्वया कार्या भीष्मादद्य कथंचन}
{अहमेनं शरैस्तीक्ष्णैः पातयिष्ये रथोत्तमात् ॥सञ्जय उवच}
{}


\twolineshloka
{एवमुक्तस्तु पार्थेन शिखण्डी भरतर्षभ}
{अभ्यद्रवत गाङ्गेयं श्रुत्वा पार्थस्य भाषितम्}


\twolineshloka
{धृष्टद्युम्नस्तथा राजन्सौभद्रश्च महारथः}
{हृष्टावाद्रवतां भीष्मं श्रुत्वा पार्थस्य भाषितम्}


\threelineshloka
{विराटद्रुपदौ वृद्धौ कुन्तिभोजस्य दंशितः}
{अभ्यद्रवत गाङ्गेयं पुत्रस्य तव पश्यतः}
{नकुलः सहदेवश्च धर्मराजश्च वीर्यवान्}


\twolineshloka
{तथेतराणि सैन्यानि सर्वाम्येव विशांपते}
{समाद्रवन्त गाङ्गेयं श्रुत्वा पार्थस्य भाषितम्}


\twolineshloka
{प्रत्युद्ययुस्तावकाश्च समेतांस्तान्महारथान्}
{यथाशक्ति यथोत्साहं तन्मे निगदतः शृणु}


\twolineshloka
{चित्रसेनो महाराज चेकितानं समभ्यायात्}
{भीष्मप्रेप्सुं रणे यान्तं वृषं व्याघ्रशिशुर्यथा}


\twolineshloka
{धृष्टद्युम्नं महाराज भीष्मान्तिकमुपागतम्}
{त्वरमाणं रणे यत्तं कृतवर्मा न्यवरायत्}


\twolineshloka
{भीमसेनं सुसंक्रुद्धं गाङ्गेयस्य वधैषिणम्}
{त्वरमाणो महाराज सौमदत्तिर्न्यवारयत्}


\twolineshloka
{तथैव नकुलं शूरं किरन्तं सायकान्बहून्}
{विकर्णो वारयामास इच्छन्भीष्मस्य जीवितम्}


\twolineshloka
{सहदेवं तथा राजन्यान्तं भीष्मरथं प्रति}
{वारयामास संक्रुद्धः कृपः शारद्वतो युधि}


\twolineshloka
{राक्षसं क्रूरकर्माणं भैमसेनिं महाबलम्}
{भीष्मस्य निधनप्रेप्सुं दुर्मुखोऽभ्यज्रवद्बली}


\twolineshloka
{सात्यकिं समरे क्रुद्धमार्स्यशृङ्गिवारयत्}
{`भीष्मस्य वधमिच्छन्तं पाण्डवप्रीतिकाम्यया'}


\twolineshloka
{अभिमन्युं महाराज यान्तं भीष्मरथं प्रति}
{सुदक्षिणो महाराज काम्भोज प्रत्यवारयत्}


\twolineshloka
{विराटद्रुपदौ वृद्धौ समेतावरिमर्दनौ}
{अश्वत्थामा ततः क्रुद्धो वारयामास भारत}


\twolineshloka
{तथा पाण्डुसुतं ज्येष्ठं भीष्मस्य वधकाङ्क्षिणम्}
{भारद्वाजो रणे यत्तो धर्मपुत्रमवारयत्}


\twolineshloka
{अर्जुनं रभसं युद्धे पुरस्कृत्य शिखण्डिनम्}
{भीष्मप्रेप्सुं महाराज भासयन्तं दिशो दश}


\twolineshloka
{दुःशासनो महेष्वासो वारयामास संयुगे}
{अन्ये च तावका योधाः पाण्डवानां महारथान्}


\twolineshloka
{भीष्मस्याभिमुखान्यातान्वारयामासुराहवे}
{धृष्टद्युम्नस्तु सैन्यानि प्राक्रोशंस्तु पुनः पुनः}


\twolineshloka
{अभ्यद्रवत संरब्धो भीष्ममेकं महारथः}
{एषोऽर्जुनो रणे भीष्मं प्रयाति कुरुनन्दनः}


\twolineshloka
{अभ्यद्रवत माभैष्ट भीष्मो हि प्राप्स्यते न वः}
{अर्जुनं समरे योद्धुं नोत्सहेतापि वासवः}


\twolineshloka
{किमु भीष्मो रणे वीरा गतसत्वोऽल्पजीवितः}
{इति सेनापतेः श्रुत्वा पाण्डवानां महारथाः}


\twolineshloka
{अभ्यद्रवन्त संहृष्टा गाङ्गेयस्य रथं प्रति}
{आगच्छतस्तान्समरे वार्योघानचला इव}


\twolineshloka
{अवारयन्त संहृष्टास्तावकाः पुरुषर्षभाः}
{दुःशासनो महाराज भयं त्यक्त्वा महारथः}


\twolineshloka
{भीष्मस्य जीविताकाङ्क्षी धनञ्जयमुपाद्रवत्}
{6-110-26bतथैवपाण्डवाः शूरा गाङ्गेयस्य रथं प्रति}


\twolineshloka
{अभ्यद्रवन्त संग्रामे तव पुत्रान्महारथाः}
{तत्राद्भुतमपश्याम चित्ररूपं विशांपते}


\twolineshloka
{दुःशासनरथं प्राप्य यत्पार्थो नात्यवर्तत}
{यथा वारयते वेला क्षुब्धतोयं महार्णवम्}


\twolineshloka
{तथैव पाण्डवं क्रुद्धं तव पुत्रो न्यवारयत्}
{उभौ तौ रथिनां श्रेष्ठावुभौ भारतदुर्जयौ}


\twolineshloka
{उभौ चन्द्रार्कसदृशौ कान्त्या दीप्त्या च भारत}
{तथा तौ जातसंरम्भावन्योन्यवधकाङ्क्षिणौ}


\twolineshloka
{समीयतुर्महासङ्ख्ये मयशक्रौ यथा पुरा}
{दुःशासनो महाराज पाण्डवं विशिखैस्त्रिभिः}


\twolineshloka
{वासुदेवं च विंशत्या ताडयामास संयुगे}
{ततोऽर्जुनो जातमन्युर्वार्ष्णेयं वीक्ष्य पीडितम्}


\twolineshloka
{दुःशासनं शतेनाजौ नाराचानां समार्पयत्}
{ते तस्य कवचं भित्त्वा पपुः शोणितमाहवे}


\threelineshloka
{` यथैव पन्नगा राजंस्तटाकं तृषितास्तथा}
{'दुःशासनस्त्रिभिः क्रुद्धः पार्थं विव्याध पत्रिभिः}
{ललाटे भरतश्रेष्ठ शरैः सन्नतपर्वभिः}


\twolineshloka
{ललाटस्थैस्तु तैर्बाणैः शुशुभे पाण्डवो रणे}
{यथा मेरुर्महाराज शृङ्गैरत्यर्थमुच्छ्रितैः}


\twolineshloka
{सोऽतिविद्धो महेष्वासः पुत्रेण तव धन्विना}
{व्यराजत रणे पार्थः किंशुकः पुष्पवानिव}


\twolineshloka
{दुःशासनं ततः क्रुद्धः पीडयामास पाण्डवः}
{पर्वणीव सुसंक्रुद्धो राहुः पूर्णं निशाकरम्}


\twolineshloka
{पीड्यमानो बलवता पुत्रस्तव विशांपते}
{विव्याध समरे पार्थं कङ्कपत्रैः शिलाशितैः}


\twolineshloka
{तस्य पार्थो धनुश्छित्वा रथं चास्य त्रिभिः शरैः}
{आजघान ततः पश्चात्पुत्रं ते निशितैः शरैः}


\twolineshloka
{सोऽन्यत्कार्मुकमादाय भीमस्य प्रमुखे स्थितः}
{अर्जुनं पञ्चाविंशत्या बाह्वोरुरसि चार्पयत्}


\twolineshloka
{तस्य क्रुद्धो महाराज पाण्डवः शत्रुतापनः}
{अप्रैषीद्विशिखान्घोरान्यमदण्डोपमान्बहून्}


\twolineshloka
{अप्राप्तानेव तान्बाणांश्चिच्छेद तनयस्तव}
{यतमानस्य पार्थस्य तदद्भुतमिवाभवत्}


\twolineshloka
{पार्थं च निशितैर्बाणैरविध्यत्तनयस्तव}
{ततः क्रुद्धो रणे पार्थः शरान्संधाय कार्मुके}


\twolineshloka
{प्रेषयामास समरे स्वर्णपुङ्खाञ्छिलाशितान्}
{न्यमञ्जंस्ते महाराज तस्य काये महात्मनः}


\twolineshloka
{यथा हंसा महाराज तटाकं प्राप्य भारत}
{पीडितश्चैव पुत्रस्ते पाण्डवेन महात्मना}


\twolineshloka
{हित्वा पार्थं रणे तूर्णं भीष्मस्य रथमाव्रजत्}
{अगाधे मञ्जतस्तस्य द्वीपो भीष्मोऽभवत्तदा}


\twolineshloka
{प्रतिलभ्य ततः संज्ञां पुत्रस्तव विशांपते}
{अवारयत्ततः शूरो भूय एव पराक्रमी}


\twolineshloka
{शरैः सुनिशितैः पार्थं यथा वृत्रं पुरन्दरः}
{निर्बिभेद महाकायो विव्यथे नैव चार्जुनः}


\chapter{अध्यायः १११}
\twolineshloka
{सञ्जय उवाच}
{}


\twolineshloka
{सात्यकिं दंशितं युद्धे भीष्मायाभ्युद्यतं रणे}
{आर्श्यशृङ्गिर्महेष्वासो वारयामास संयुगे}


\twolineshloka
{माधवस्तु सुसंक्रुद्धो राक्षसं नवभिः शरैः}
{आजघान रणे राजन्प्रहसन्निव भारत}


\twolineshloka
{तथैव राक्षसो राजन्माधवं नवभिः शरैः}
{अर्दयामास राजेन्द्र संक्रुद्धः शिनिपुङ्गवम्}


\twolineshloka
{शैनेयः शरसङ्घं तु प्रेषयामास संयुगे}
{राक्षसाय सुसंक्रुद्धो माधवः परवीरहा}


\twolineshloka
{ततो रक्षो महाबाहुं सात्यकिं सत्यविक्रमम्}
{विव्याध विशिखैस्तीक्ष्णैः सिंहनाद ननाद च}


\twolineshloka
{माधवस्तु भृशं विद्धो राक्षसेन रणे तदा}
{धैर्यमालम्ब्य तेजस्वी जहास च ननाद च}


\twolineshloka
{भगदत्तस्ततः क्रुद्धो माधवं निशितैः शरैः}
{ताडयामास समरे तोत्रैरिव महागजम्}


\twolineshloka
{विहाय राक्षसं युद्धे शैनेयो रथिनां वरः}
{प्राग्ज्योतिषाय चिक्षेप शरान्संनतपर्वणः}


\twolineshloka
{तस्य प्राग्ज्योतिषो राजा माधवस्य महद्धनुः}
{चिच्छेद शतधारेण भल्लेन कृतहस्तवत्}


\twolineshloka
{अथान्यद्धनुरादाय वेगवत्परवीरहा}
{भगदत्तं रणे क्रुद्धं विव्याध निशितैः शरैः}


\twolineshloka
{सोऽतिविद्धो महेष्वासः सृक्किणी परिसंलिहन्}
{शक्तिं कनकवैडूर्यभूषितामायसीं दृढाम्}


\twolineshloka
{यमदण्डोपमां घोरां चिक्षेप परमाहवे}
{तामापतन्तीं सहसा तस्य बाहुबलेरिताम्}


\twolineshloka
{सात्यकिः समरे राजन्द्विधा चिच्छेद सायकैः}
{ततः पपात सहसा सहोल्केव हतप्रभा}


\twolineshloka
{शक्तिं विनिहतां दृष्ट्वा पुत्रस्तव विशांपते}
{महता रथवंशेन वारयामास माधवम्}


\twolineshloka
{तथा परिवृतं दृष्ट्वा वार्ष्णेयानां महारथम्}
{दुर्योधनो भृशं क्रुद्धो भ्रातॄन्सर्वानुवाच ह}


\twolineshloka
{तथा कुरुत कौरव्या यथाः वः सात्यको युधि}
{न जीवन्प्रतिनिर्याति महतोऽस्माद्रथव्रजात्}


\twolineshloka
{तस्मिन्हते हतं मन्ये पाण्डवानां महद्बलम्}
{तथेति च वचस्तस्य परिगृह्य महारथाः}


\threelineshloka
{शैनेयं योधयामासर्भीष्मायाभ्युद्यतं रणे}
{`अभिमन्युं तदाऽऽयान्तं भीष्मस्याभ्युद्यतं वधे}
{'काम्भोजराजो बलवान्वारयामास संयुगे}


\twolineshloka
{आर्जुनिं नृपतिर्विद्ध्वा शरैः सन्नतपर्वभिः}
{पुनरेव चतुःषष्ट्या राजन्विव्याध तं नृप}


\twolineshloka
{सुदक्षिणस्तु समरे पुनर्विव्याध तं पञ्चभिः}
{सारथिं चास्य नवभिरिच्छन्भीष्मस्य जीवितम्}


\twolineshloka
{तद्युद्धमासीत्सुमहत्तयोस्तत्र समागमे}
{यदाभ्यधावद्गाङ्गेयं शिखण्डी शत्रुकर्शनः}


\twolineshloka
{विराटद्रुपदौ वृद्धौ वारयनतौ महाचमूम्}
{भीष्मं च युधि संरब्धावाद्रवन्तौ महारथौ}


\twolineshloka
{अश्वत्थामा रणे क्रुद्धः समियाद्रथसत्तमः}
{ततः प्रववृते युद्धं तयोस्तस्य च भारत}


\twolineshloka
{विराटो दशभिर्भल्लैराजघान परंतप}
{यतमानं महेष्वासं द्रौणिमाहवशोभिनम्}


\twolineshloka
{द्रुपदश्च त्रिभिर्बाणैर्विव्याध निशितैस्तदा}
{गुरुपुत्रं समासाद्य प्रहरन्तौ महाबलौ}


\twolineshloka
{अश्वत्थामा ततस्तौ तु विव्याध बहुभिः शरैः}
{विराटद्रुपदौ वीरौ भीष्मं प्रति समुद्यतौ}


\twolineshloka
{तत्राद्भुतमपश्याम वृद्धयोश्चरितं महत्}
{यद्द्रौणिसायकान्घोरान्प्रत्यवारयतां युधि}


\twolineshloka
{सहदेवं तथाऽऽयान्तं कृपः शारद्वतोऽभ्ययात्}
{यथा नागो वने नागं मत्तो मत्तमुपाद्रवत्}


\twolineshloka
{कृपश्च समरे शूरो माद्रीपुत्रं महारथम्}
{आजघान शरैस्तूर्णं सप्तत्या रुक्मभूषणैः}


\twolineshloka
{तस्य माद्रीसुतश्चापं द्विधा चिच्छेद सायकैः}
{अथैनं छिन्नधन्वानं विव्याध नवभिः शरैः}


\twolineshloka
{सोऽन्यत्कार्मुकमादाय समरे भारसाधनम्}
{माद्रीपुत्रं सुसंहृष्टो दशभिर्निशितैः शरैः}


\twolineshloka
{आजघानोरसि क्रुद्ध इच्छन्भीष्मस्य जीवितम्}
{तथैव पाण्डवो राजञ्छारद्वतममर्षणम्}


\twolineshloka
{आजघानोरसि क्रुद्धो भीष्मस्य वधकाङ्क्षया}
{तयोर्युद्धं समभवद्धोररूपं भयावहम्}


\twolineshloka
{नकुलं तु रणे क्रुद्धो विकर्णः शत्रुतापनः}
{विव्याध सायकैः षष्ट्या रक्षन्भीष्मं महाबलम्}


\twolineshloka
{नकुलोऽपि भृशं विद्धस्तव पुत्रेण धीमता}
{विकर्णं सप्तसप्तत्या निर्बिभेद शिलीमुखैः}


\twolineshloka
{तत्र तौ नरशार्दूलौ भीष्महेतोः परंतपौ}
{अन्योन्यं जघ्नुतुर्वीरौ गोष्ठे गोवृषभाविव}


\twolineshloka
{घटोत्कचं रणे यान्तं निघ्नन्तं तव वाहिनीम्}
{दुर्मुखः समरे प्रायाद्भीष्यहेतोः पराक्रमी}


\twolineshloka
{हैडिम्बस्तु रणे राजन्दुर्मुखं शत्रुतापनम्}
{आजघानोरसि क्रुद्धः शरेणानतपर्वणा}


\twolineshloka
{भीमसेनसुतं चापि दुर्मुखः सुमुखैः शरैः}
{षष्ट्या वीरो नदन्हृष्टो विव्याध रणमूर्धनि}


\twolineshloka
{धृष्टद्युम्नं तथाऽऽयान्तं भीष्मस्य वधकाङ्क्षिणम्}
{हार्दिक्यो वारयामास रथश्रेष्ठं महारथः}


\twolineshloka
{हार्दिक्यः पार्षतं चापि विद्ध्वा पञ्चभिरायसैः}
{पुनः पञ्चाशता तूर्णं तिष्ठतिष्ठेति चाब्रवीत्}


\twolineshloka
{आजघान महाबाहुः पार्षतं तं महारथम्}
{तं चैव पार्षतो राजन्हार्दिक्यं नवभिः शरैः}


\twolineshloka
{विव्याध निशितैस्तीक्ष्णैः कङ्कपत्रैरजिह्नगैः}
{तयोः समभवद्युद्धं भीष्महेतोर्महाहवे}


\twolineshloka
{अन्योन्यातिशये युक्तं यथा वृत्रमहेन्द्रयोः}
{भीमसेनं तथाऽऽयान्तं भीष्मं प्रति महारथम्}


\twolineshloka
{भूरिश्रवाभ्ययात्तूर्णं तिष्ठतिष्ठेति चाब्रवीत्}
{सौमदत्तिरथो भीममाजघान स्तनान्तरे}


\twolineshloka
{नाराचेन सुतीक्ष्णेन रुक्मपुङ्खेन संयुगे}
{उरःस्थेन बभौ तेन भीमसेनः प्रतापवान्}


\twolineshloka
{स्कन्दशक्त्या यथा क्रौञ्चः पुरा नृपतिसत्तम}
{तौ शरान्सूर्यसंकाशान्कर्मारपरिमार्जितान्}


\twolineshloka
{अन्योन्यस्य रणे क्रुद्धौ चिक्षिपते नरर्षभौ}
{भीमो भीष्मवधाकाङ्क्षी सौमदत्तिं महारथम्}


\twolineshloka
{तथा भीष्मजये गृध्नुः सौमदत्तिस्तू पाण्डवम्}
{कृतप्रतिकृते यत्तौ योधयामासतू रणे}


\twolineshloka
{युधिष्ठिरं तु कौन्तेयं महत्या सेनया वृत्तम्}
{भीष्माभिमुखमायान्तं भारद्वाजो न्यवारयत्}


\twolineshloka
{द्रोणस्य रथनिर्घोषं पर्जन्यनिनदोपमम्}
{श्रुत्वा प्रभद्रका राजन्समकम्पन्त मारिष}


\twolineshloka
{सा सेना महती राजन्पाण्डुपुत्रस्य संयुगे}
{द्रोणेन वारिता यत्ता न चचाल पदात्पदम्}


\twolineshloka
{चेकितान रणे यत्तं भीष्मं प्रति जनेश्वर}
{चित्रसेनस्तव सुतः क्रुद्धरूपमवारयत्}


\twolineshloka
{भीष्महेतोः पराक्रान्तश्चित्रसेनः पराक्रमी}
{चेकितानं परं शक्त्या योधयामास भारत}


\twolineshloka
{तथैव चेकितानोऽपि चित्रसेनमवारयत्}
{तद्युद्धमासीत्सुमहत्तयोस्तत्र समागमे}


\twolineshloka
{अर्जुनो वार्यमाणस्तु बहुशस्तत्र भारत}
{विमुखीकृत्य पुत्रं ते सेनां तव ममर्द ह}


\twolineshloka
{दुःशासनोऽपि परया शक्त्या पार्थमवारयत्}
{कथं भीष्मं न नो हन्यादिति निश्चित्य भारत}


\twolineshloka
{सा वध्यमाना समरे पुत्रस्य तव वाहिनी}
{लोड्यते रथिभिः श्रेष्ठैस्तत्रतत्रैव भारत}


\chapter{अध्यायः ११२}
\twolineshloka
{सञ्जय उवाच}
{}


\twolineshloka
{अथ वीरो महेष्वासो मत्तवारणविक्रमः}
{समादाय महच्चापं मत्तवारणवारणम्}


\twolineshloka
{विधुन्वानो नरश्रेष्ठो द्रावयाणो वरूथिनीम्}
{पृतनां पाण्डवेयानां गाहमानो महाबलः}


\threelineshloka
{निमित्तानि निमित्तज्ञः सर्वतो वीक्ष्य वीर्यवान्}
{प्रतपन्तमनीकानि द्रोणः पुत्रमभाषत ॥द्रोण उवाच}
{}


\twolineshloka
{अयं हि दिवसस्तात यत्र पार्थो महाबलः}
{जिघांसुः समरे भीष्मं परं यत्नं करिष्यति}


\twolineshloka
{उत्पतन्ति हि मे बाणा धनुः प्रस्फुरतीव च}
{योगमस्त्राणि नेच्छन्ति क्रूरं मे वर्तते मनः}


\twolineshloka
{दिक्ष्वशान्तानि घोराणि व्याहरन्ति मृगद्विजाः}
{नीचैर्गृध्रा निलीयन्ते भारतानां चमूं प्रति}


\twolineshloka
{नष्टप्रभ इवादित्यः सर्वतो लोहिता दिशः}
{रसते व्यथते भूमिः कम्पतीव च सर्वशः}


\twolineshloka
{कङ्कगृध्रा बलाकाश्च व्याहरन्ति मुहुर्मुहुः}
{शिवाश्चैवाशिवा घोरा वेदयन्त्यो महद्भयम्}


\twolineshloka
{पपात महती चोल्का मध्येनादित्यमण्डलात्}
{सकबन्धश्च परिघो भानुमावृत्य तिष्ठति}


\twolineshloka
{परिवेषस्तथा घोरश्चन्द्रभास्करयोरभूत्}
{वेदयानो भयं घोरं राज्ञां देहावकर्तनम्}


\twolineshloka
{देवतायतनस्थाश्च कौरवेन्द्रस्य देवताः}
{कम्पन्ते च हसन्ते च नृत्यन्ति च रुदन्ति च}


\twolineshloka
{अपसव्यं ग्रहाश्चक्रुरलक्ष्माणं दिवाकरम्}
{अवाक्शिराश्च भगवानुपातिष्ठत चन्द्रकमाः}


\twolineshloka
{वपूंषि च नरेन्द्राणां विगताभानि लक्षये}
{धार्तराष्ट्रस्य सैन्येषु न च भ्राजन्ति दंशिताःक}


\twolineshloka
{सेनयोरुभयोश्चापि समन्ताच्छ्रूयते महान्}
{पाञ्चजन्यस्य निर्घोषो गाण्डीवस्य च निःस्वनः}


\twolineshloka
{ध्रुवमास्थाय बीभत्सुरुत्तमास्त्राणि संयुगे}
{अपास्यान्यान्रणे योधानभ्येष्यति पितामहम्}


\twolineshloka
{हृष्यन्ति रोमकूपाणि सीदतीव च मे मनः}
{चिन्तयित्वा महाबाहो भीष्मार्जुनसमागमम्}


\twolineshloka
{तं चेह निकृतिप्रज्ञं पाञ्चाल्यं पापचेतसम्}
{पुरस्कृत्य रणे पार्थो भीष्मस्यायोधनं गतः}


\twolineshloka
{अब्रवीच्च पुरा भीष्मो नाहं हन्यां शिखण्डिनम्}
{स्त्री ह्येषा विहिता धात्रा दैवाच्च स पुनः पुमान्}


\twolineshloka
{असंकल्पध्वजश्चैव याज्ञसेनिर्महाबलः}
{न चामङ्गलिके तस्मिन्प्रहरेदापगासुतः}


\twolineshloka
{एतद्विचिन्तयानस्य प्रज्ञा सीदति मे भृशम्}
{अभ्युद्यतो रणे पार्थः कुरुवृद्धमुपाद्रवत्}


\twolineshloka
{युधिष्ठिरस्य च क्रोधो भीष्मश्चार्जुनसंगतः}
{मम चास्त्रसमारम्भः प्रजानामशिवं ध्रुवम्}


\twolineshloka
{मनस्वी बलवाञ्शूरः कृतास्त्रोऽलघुविक्रमः}
{दूरपाती दृढेषुश्च निमित्तज्ञश्च पाण्डवः}


\twolineshloka
{अजेयः समरे चापि देवैरपि सवासवैः}
{बलवान्बुद्धिमांश्चैव जितक्लेशो युधां वरः}


\threelineshloka
{विजयी च रणे नित्यं भैरवास्त्रश्च पाण्डवः}
{6-112-24b`अभ्युद्यतं रणे दृष्ट्वा भैरवास्त्रं च पाण्डवम्}
{'तस्य मार्गं परिहरन्द्रुतं गच्छ यतव्रत}


\twolineshloka
{पश्याद्यैतन्महाघोरे संयुगे वैशसं महत्}
{हेमचित्राणि शूरणां महान्ति च शुभानि च}


\twolineshloka
{कवचान्यवदीर्यन्ते शरैः सन्नतपर्वभिः}
{छिद्यन्ते च ध्वजाग्राणि तोमराश्च धनूंषि च}


\twolineshloka
{प्रासाश्च विमलास्तीक्ष्णाः शक्त्यश्च कनकोज्ज्वलाः}
{वैजयन्त्यश्च नागानां संक्रुद्धेन किरीटिना}


\twolineshloka
{नायं संरक्षितुं कालः प्राणान्पुत्रोपजीविभिः}
{याहि स्वर्गं पुरस्कृत्य यशसे विजयाय च}


\twolineshloka
{रथनागहयावर्तां महाघोरां सुदुर्गमाम्}
{रथेन संग्रामनदीं तरत्येष कपिध्वजः}


\twolineshloka
{ब्रह्मण्यता दमो दानं तपश्च चरितं महत्}
{इहैव दृश्यते पार्थे भ्राता यस्य धनञ्जयः}


\twolineshloka
{भीमसेनश्च बलवान्माद्रीपुत्रौ च पाण्डवौ}
{वासुदेवश्च वार्ष्णेयो यस्य नाथो व्यवस्थितः}


\twolineshloka
{तस्यैष मन्युप्रभवो धार्तराष्ट्रस्य दुर्मतेः}
{तपोभावितदेहस्य कोपो दहति वाहिनीम्}


\twolineshloka
{एष संदृश्यते पार्थो वासुदेवव्यपाश्रयः}
{दारयन्सर्वसैन्यानि धार्तराष्ट्राणि सर्वशः}


\twolineshloka
{एतदालोक्यते सैन्यं क्षोभ्यमाणं किरीटिना}
{महोर्मिनद्धं सुमहत्तिमिनेव महाजलम्}


\twolineshloka
{हाहाकिलकिलाशब्दाः श्रूयन्ते च चमूमुखे}
{याहि पाञ्चालदायादमहं यास्ये युधिष्ठिरम्}


\twolineshloka
{दुर्गमं ह्यन्तरं राज्ञो व्यूहस्यामिततेजसः}
{समुद्रकुक्षिप्रतिमं सर्वतोऽतिरथैः स्थितैः}


\twolineshloka
{सात्यकिश्चाभिमन्युश्च धृष्टद्युम्नवृकोदरौ}
{पर्यरक्षन्त राजानं यमौ च मनुजेश्वरम्}


\twolineshloka
{उपेन्द्रसदृशश्यामो महाशाल इवोद्गतः}
{एष गच्छत्यनीकाग्रे द्वितीय इव फल्गुनः}


\twolineshloka
{उत्तमास्त्राणि चाधत्स्व गृहीत्वा च महद्धनुः}
{पार्षतं याहि राजानं युध्यस्व च वृकोदरम्}


\twolineshloka
{को हि नेच्छेत्प्रियं पुत्रं जीवन्तं शाश्वतीः समाः}
{क्षत्रधर्मं तु संप्रेक्ष्य ततस्त्वां नियुनज्म्यहम्}


\threelineshloka
{एष चातिरणे भीष्मो दहते वै महाचमूम्}
{युद्धेषु सदृशस्तात यमस्य वरुणस्य च ॥सञ्जय उवाच}
{}


\twolineshloka
{पुत्रं समनुशास्यैवं भारद्वाजः प्रतापवान्}
{महारणे महाराज धर्मराजमयोधयत्}


\chapter{अध्यायः ११३}
\twolineshloka
{सञ्जय उवाच}
{}


\twolineshloka
{भगदत्तः कृपः शल्यः कृतवर्ता तथैव च}
{विन्दानुविन्दावावन्त्यौ सैन्धवश्च जयद्रथः}


\twolineshloka
{चित्रसेनो विकर्णश्च तथा दुर्मर्षणो युवा}
{दशैते तावका योधा भीमसेनमयोधयन्}


\twolineshloka
{महत्या सेनया युक्ता नानादेशसमुत्थया}
{भीष्मस्य समरे राजन्प्रार्थयाना महद्यशः}


\twolineshloka
{शल्यस्तु नवभिर्बाणैर्भीमसेनमताडयत्}
{कृतवर्मा त्रिभिर्भाणैः कृपश्च नवभिः शरैः}


\twolineshloka
{चित्रसेनो विकर्णश्च भगदत्तश्च मारिष}
{दशभिर्दशभिर्बाणैर्भीमसेनमताडयन्}


\twolineshloka
{सैन्धवश्च त्रिभिर्बाणैर्भीमसेनमडायत्}
{विन्दानुविन्दावावन्त्यौ पञ्चभिः पञ्चभिः शरैः}


\twolineshloka
{दुर्मर्षणस्तु विंशत्या पाण्डवं निशितैः शरैः}
{स तान्सर्वान्महाराज राजमानान्पृथक्पृथक्}


\twolineshloka
{प्रवीरान्सर्वलोकस्य धार्तराष्ट्रान्महारथान्}
{जघान समरे वीरः पाण्डवः परवीरहा}


\twolineshloka
{सप्तभिः शल्यमाविध्यत्कृतवर्माणमष्टभिः}
{कृपस्य सशरं चापं मध्ये चिच्छेद भारत}


\twolineshloka
{अथैनं च्छिन्नधन्वानं पुनर्विव्याध सप्तभिः}
{विन्दानुविन्दौ च तथा त्रिभिस्त्रिभिरताडयत्}


\twolineshloka
{दुर्मर्षणं च विंशत्या चित्रसेनं च पञ्चभिः}
{विकर्णं दशभिर्बाणैः पञ्चभिश्च जयद्रथम्}


\twolineshloka
{विद्ध्वा भीमोऽनदद्धृष्टः सैन्धवं च पुनस्त्रिभिः}
{अथान्यद्धनुरादाय गौतमो रथिनां वरः}


\twolineshloka
{भीमं विव्याध संरब्धो दशभिर्निशितैः शरैः}
{स विद्धो दशभिर्बाणैस्तोत्रैरिव महाद्विपः}


\twolineshloka
{ततः क्रुद्धो महाराज भीमसेनः प्रतापवान्}
{गौतमं ताडयामास शरैर्बहुभिराहवे}


\twolineshloka
{सैन्धवस्य तथाऽश्वांश्च सारथिं च त्रिभिः शरैः}
{प्राहिणोन्मृत्युलोकाय कालान्तकसमद्युतिः}


\twolineshloka
{हताश्वात्तु रथात्तूर्णमवप्लुत्य महारथः}
{शरांश्चिक्षेप निशितान्भीमसेनस्य संयुगे}


\twolineshloka
{तस्य भीमो धनुर्मध्ये द्वाभ्यां चिच्छेद मारिष}
{भल्लाभ्यां भरतश्रेष्ठ सैन्धवस्य महात्मनः}


\twolineshloka
{स च्छिन्नधन्वा विरथो हताश्वो हतसारथिः}
{चित्रसेनरथं राजन्नारुरोह त्वरान्वितः}


\twolineshloka
{अत्यद्भुतं रणे कर्म कृतवांस्तत्र पाण्डवः}
{महारथाञ्शरैर्विद्ध्वा वारयित्वा च मारिष}


\twolineshloka
{विरथं सैन्धवं चक्रे सर्वलोकस्य पश्यतः}
{तदा न ममृषे शल्यो भीमसेस्य विक्रमम्}


\twolineshloka
{स संधाय शरांस्तीक्ष्णान्करमारपरिमार्जितान्}
{भीमं विव्याध समरे तिष्ठितिष्ठेति चाब्रवीत्}


\twolineshloka
{कृपश्च कृतवर्माच भगदत्तश्च वीर्यवान्}
{विन्दानुविन्दावावन्त्यौ चित्रसेनश्च संयुगे}


\twolineshloka
{दुर्मर्षणो विकर्णश्च सिन्धुराजश्च वीर्यवान्}
{भीमं ते विव्यधुस्तूर्णं शल्यहेतोररिन्दमाः}


\twolineshloka
{स च तान्प्रति विव्याध पञ्चभिः पञ्चभिः शरैःक}
{शल्यं विव्याध सप्तत्या पुनश्च दशभिः शरैः}


\twolineshloka
{तं शल्यो नवभिर्भित्त्वा पुनर्विव्याध पञ्चभिः}
{सारथिं चास्य भल्लेन गाढं विव्याध मर्मणि}


\twolineshloka
{विशोकं प्रेक्ष्य निर्भिन्नं भीमसेनः प्रतापवान्}
{मद्रराजं त्रिभिर्बाणैर्बाह्वोरुरुसि चार्पयत्}


\twolineshloka
{तथेतरान्महेष्वासांस्त्रिभिस्त्रिभिरजिह्मगैः}
{ताजयामास समरे सिंहवाद्विननाद च}


\twolineshloka
{ते हि यत्ता महेष्वासाः पाण्डवं युद्धकोविदम्}
{त्रिभिस्त्रिभिरकुण्ठाग्रैर्भृशं मर्मस्वताडयन्}


\twolineshloka
{सोऽतिविद्धो महेष्वासो भीमसेनो न विव्यथे}
{पर्वतो वारिधाराभिर्वर्षमाणैरिवाम्बुदैः}


\twolineshloka
{स तु क्रोधसमाविष्टः पाण्डवानां महारथः}
{मद्रेश्वरं त्रिभिर्बाणैर्भृशं विद्ध्वा महायशाः}


\twolineshloka
{कृपं च नवभिर्बाणैर्भृशं विद्ध्वा समन्ततः}
{प्रग्ज्योतिषं शतैराजौ राजन्विव्याध सायकैः}


\twolineshloka
{ततस्तु सशरं चापं सात्वतस्य महात्मनः}
{क्षुरप्रेण सुतीक्ष्णेन चिच्छेद कृतहस्तवत्}


\twolineshloka
{तथान्यद्धनुरादाय कृतवर्मा वृकोदरम्}
{आजघान भ्रुवोर्मध्ये नाराचेन परंतपः}


\twolineshloka
{भीमस्तु समरे विद्ध्वा शल्यं नवभिरायसैः}
{भगदत्तं त्रिभिश्चैव कृतवर्माणमष्टभिः}


\twolineshloka
{द्वाभ्यां द्वाभ्यां तु विव्याध गौतमप्रभृतीन्रथान्}
{तेऽपि तं समरे राजन्विव्यधुर्निशितैः शरैः}


\twolineshloka
{स तथा पीड्यमानोऽपि सर्वशस्त्रैर्महारथैः}
{मत्वा तृणेन तांस्तुल्यान्विचचार गतव्यथः}


\twolineshloka
{ते चापि रथिनां श्रेष्ठा भीमाय निशिताञ्शरान्}
{प्रेषयामासुरव्याग्राः शतशोऽथ सहस्रशः}


\twolineshloka
{तस्य शक्तिं महावेगां भगदत्तो महारथः}
{चिक्षेप समरे वीरः स्वर्णदण्डां महमते}


\twolineshloka
{तोमरं सैन्धवो राजा पट्टसं च महाभुजः}
{शतघ्नीं च कृपो राजञ्छरं शल्यश्च संयुगे}


\twolineshloka
{अथेतरे महेष्वासाः पञ्च पञ्च शिलीमुखान्}
{भीमसेनं समुद्दिश्य प्रेषयमासुरोजसा}


\twolineshloka
{तोमरं च द्विधा चक्रे क्षुरप्रेणानिलात्मजः}
{पट्टसं च त्रिभिर्बाणैश्चिच्छेद तिलकाण्डवत्}


\twolineshloka
{स बिभेद शतघ्नीं च नवभिः कङ्कपत्रिभिः}
{मद्रराजप्रयुक्तं च शरं छित्त्वा महारथः}


\twolineshloka
{शक्तिं चिच्छेद सहसा भगदत्तेरितां रणे}
{तथेतराञ्शरान्घोराञ्शरैः सन्नतपर्वभिः}


\twolineshloka
{भीमसेनो रणश्लाघी त्रिधैकैकं समाच्छिनत्}
{तांश्च सर्वान्महेष्वासांस्त्रिभिस्त्रिभिरताडयत्}


\twolineshloka
{ततो धनञ्जयस्तत्र वर्तमाने महारणे}
{आजगाम रथेनाजौ भीमं दृष्ट्वा महारथम्}


\twolineshloka
{निघ्नन्तं समरे शत्रून्योधयानं च सायकैः}
{तौ तु तत्र महात्मानौ समेतौ वीक्ष्य पाण्डवौ}


\twolineshloka
{न शशंसुर्जयं तत्र तावकाः पुरुषर्षभाः}
{अथार्जुनो रणे भीमं योधयन्तं महारथान्}


\twolineshloka
{भीष्मस्य निधनाकाङ्क्षी पुरस्कृत्य शिखण्डिनम्}
{आससाद रणे वीरांस्तावकान्दश भारत}


\twolineshloka
{ये स्म भीमं रणे राजन्योधयन्तो व्यवस्थिताः}
{बीभत्सुस्तानथाविध्यद्भीमस्य प्रियकाम्यया}


\twolineshloka
{ततो दुर्योधनो राजा सुशर्माणमचोदयत्}
{अर्जुनस्य वधार्थाय भीमसेनस्य चोभयोः}


\twolineshloka
{सुशर्मन्गच्छ शीघ्रं त्वं बलौघैः परिवारितः}
{जहि पाण्डुसुतावेतौ धनञ्जयवृकोदरौ}


\twolineshloka
{तच्छ्रुत्वा वचनं तस्य त्रैगर्तः प्रस्थलाधिपः}
{अभिद्रुत्य रणे भीममर्जुनं चैव धन्विनौ}


\twolineshloka
{रथैरनेकसाहस्रैः समन्तात्पर्यवारयत्}
{ततः प्रववृते युद्धमर्जुनस्य परैः सह}


\chapter{अध्यायः ११४}
\twolineshloka
{सञ्जय उवाच}
{}


\twolineshloka
{अर्जुनस्तु रणे शल्यं यतमानं महारथम्}
{छादयामास समरे शरैः सन्नतपर्वभिः}


\twolineshloka
{सुशर्माणं कृपं चव त्रिभीस्त्रिभिरविध्यत}
{प्राग्ज्योतिषं च समरे सैन्धवं च जयद्रथम्}


\twolineshloka
{चित्रसेनं विकर्णं च कृतवर्माणमेव च}
{दुर्मर्षणं च राजेन्द्र ह्यावन्त्यौ च महारथौ}


\twolineshloka
{एकैकं त्रिभिरानर्च्छत्कङ्कबर्हिणवाजितैः}
{शरैरतिरथो युद्धे पीडयन्वाहिनीं तव}


\twolineshloka
{जयद्रथो रणे पार्थं विद्ध्वा भारत सायकैः}
{भीमं विव्याध तरसा चित्रसेनरथे स्थितः}


% Check verse!
शल्यश्च समरे जिष्णुं कृपश्च रथिनां वरः ॥विव्याधाते महाराज बहुधा मर्ममेदिभिः
\twolineshloka
{चित्रसेनादयश्चैव पुत्रास्तव विशांपते}
{पञ्चाभिः पञ्चभिस्तूर्णं सुंयुगे निशितैः शरैः}


\twolineshloka
{आजघ्नुरर्जुनं सङ्ख्ये भीमसेनं च मारिष}
{तौ तत्र रथिनां श्रेष्ठौ कौन्तेयौ भरतर्षभौ}


\twolineshloka
{अपीडयेतां समरे त्रिगर्तानां महद्बलम्}
{सुशर्माणि रणे पार्थं शरैर्नवभिराशुगैः}


\twolineshloka
{ननाद बलवन्नादं त्रासयानो महद्बलम्}
{अन्ये च रथिनः शूरा भीमसेनधनञ्जयौ}


\twolineshloka
{विव्यधुर्निशितैर्बाणै रुक्मपुङ्खैरजिह्मगैः}
{तेषां च रथिनां मध्ये कौन्तेयौ भरतर्षभौ}


\twolineshloka
{क्रीडमानौ रथोदारौ चित्ररूपौ व्यदृश्यताम्}
{आमिषेप्सू गवां मध्ये सिंहाविव मदोत्कटौ}


\twolineshloka
{छित्त्वा धनूंषि शूराणां शरांश्च बहुधा रणे}
{पातयामासतुर्वीरौ शिरांसि शतशो नृणाम्}


\twolineshloka
{रथाश्च बहवो भग्ना हयाश्च शतशो हताः}
{गजाश्च सगजारोहाः पेतुरुर्व्यां महाहवे}


\twolineshloka
{रथिनः सादिनश्चापि तत्रतत्र निषूदिताः}
{दृश्यन्ते बहवो राजन्वेषमानाः समन्ततः}


\twolineshloka
{हतैर्गजपदात्योघैर्वाजिभिश्च निषूदितैः}
{रथैश्च बहुधा भग्नैः समास्तीर्यत मेदिनी}


\twolineshloka
{छत्रैश्च बहुधा च्छिन्नैर्ध्वजैश्च विनिपातितैः}
{अङ्कुशैरपविद्धैश्च परिस्तोमैश्च भारत}


\twolineshloka
{केयूरैरङ्गदैर्हारै राङ्कवैर्मृदितैस्तथा}
{उष्णीषैर्ऋष्टिभिश्चैव चामरव्यजनैरपि}


\twolineshloka
{तत्रतत्रापविद्धैश्च बाहुभिश्चन्दनोक्षितैः}
{ऊरुभिश्च नरेन्द्राणां समास्तीर्यत मेदिनी}


\twolineshloka
{तत्राद्भुतमपश्याम रणे पार्थस्य विक्रमम्}
{शरैः संवार्य तान्वीरान्यज्जघान महाबलः}


\twolineshloka
{पुत्रस्तु तव तं दृष्ट्वा भीमार्जुनपराक्रमम्}
{गाङ्गेयस्य रथाभ्याशमुपजग्मे महाबलः}


\twolineshloka
{कृपश्च कृतवर्मा च सैन्धवश्च जयद्रथः}
{विन्दानुविन्दावावन्त्यौ नाजहुः संयुगं तदा}


\twolineshloka
{ततो भीमो महेष्वासः फल्गुनश्च महारथः}
{कौरवाणां चमूं घोरां भृशं दुद्रुवतू रणे}


\twolineshloka
{ततो बर्हिणवाजानामयुतान्यर्बुदानि च}
{धनञ्जयरथे तूर्णं पातयन्ति स्म भूमिपाः}


\twolineshloka
{ततस्तञ्शरजालेन सन्निवार्य महारथान्}
{पार्थः समन्तात्समरे प्रेषयामास मृत्यवे}


\twolineshloka
{शल्यस्तु समरे जिष्णुं क्रीडन्निव महारथः}
{आजघानोरसि क्रुद्धो भल्लैः सन्नतपर्वभिः}


\twolineshloka
{तस्य पार्थो धनुश्छित्त्वा हस्तावापं च पञ्चभिः}
{अथैनं सायकैस्तीक्ष्णैर्भृशं विव्याध मर्मणि}


\twolineshloka
{अथान्यद्धनुरादाय समरे भारसाधनम्}
{मद्रेश्वरो रणे जिष्णुं ताडयामास रोषितः}


\twolineshloka
{त्रिभिः शरैर्महाराज वासुदेवं कच पञ्चभिः}
{भीमसेनं च नवभिर्बाह्वोरुरसि चार्पयत्}


\twolineshloka
{ततो द्रोणो महाराज मागधश्च महारथः}
{दुर्योधनसमादिष्टौ तं देशमुपजग्मतुः}


\twolineshloka
{यत्र पार्थो महाराज भीमसेनश्च पाण्डवः}
{कौरव्यस्य महासेनां जघ्नुतुः सुमहारथौ}


\twolineshloka
{जयत्सेनस्तु समरे भीमं भीमायुधं युधि}
{विव्याध निशितैर्बाणैरष्टभिर्भरतर्षभ}


\twolineshloka
{तं भीमो दशभिर्विद्धा पुनर्वीव्याध पञ्चभिः}
{सारथिं चास्य भल्लेन रथनीडादपातयत्}


\twolineshloka
{उद्भ्रांतैस्तुरगैः सोऽथ द्रवमाणैः समन्ततः}
{मागधोऽपसृतो राजा सर्वसैन्यस्य पश्यतः}


\twolineshloka
{द्रोणश्च विवरं दृष्ट्वा भीमसेनं शिलीमुखैः}
{विव्याध बाणैर्निशितैः पञ्चषष्टिभिरायसैः}


\twolineshloka
{तं भीमः समरश्लाघी गुरुं पितृसमं रणे}
{विव्याध पञ्चभिर्भल्लैस्तथा षष्ट्या च भारत}


\twolineshloka
{अर्जुनस्तु सुशर्माणं विद्ध्वा बहुभिरायसैः}
{व्यधमत्तस्य तत्सैन्यं महाभ्राणि यथाऽनिलः}


\twolineshloka
{ततो भीष्मश्च राज च कौसल्यश्च बृहद्बलः}
{समवर्तन्त संक्रुद्धा भीमसेनधनञ्जयौ}


\twolineshloka
{तथैव पाण्डवाः शूरा धृष्टद्युम्नश्च पार्षतः}
{अभ्यद्रवन्रणे भीष्मं व्यादितास्यमिवान्तकम्}


\twolineshloka
{शिखण्डी तु समासाद्य भरतानां पितामहम्}
{अभ्यद्रवत संहृष्टो भयं त्यक्त्वा महारथात्}


\twolineshloka
{युधिष्ठिरमुखाः पार्थाः पुरस्कृत्य शिखण्डिनम्}
{अयोधयन्रणे भीष्मं सहिताः सर्वसृञ्जयैः}


\twolineshloka
{तथैव तावकाः सर्वे पुरस्कृत्य यतव्रतम्}
{शिखण्डिप्रमुखान्पार्थान्योधयन्ति स्म संयुगे}


\twolineshloka
{ततः प्रववृते युद्धं कौरवाणां भयावहम्}
{तत्र पाण्डुसुतैः सार्धं भीष्मस्य विजयं प्रति}


\twolineshloka
{तावकानां जये भीष्मो ग्लह आसीद्विशांपते}
{तत्र हि द्यूतमासक्तं विजयायेतराय वा}


\twolineshloka
{धृष्टद्युम्नस्तु राजेन्द्र सर्वसैन्यान्यचोदयत्}
{अभ्यद्रवत गाङ्गेयं मा भैष्ट रथसत्तमाः}


\twolineshloka
{सेनापतिवचः श्रुत्वा पाण्डवानां वरूथिनी}
{भीष्मं समभ्ययात्तूर्णं प्राणांस्त्यक्त्वा महाहवे}


\twolineshloka
{भीष्मोऽपि रथिनां श्रेष्ठः प्रतिजग्राह तां चमूम्}
{आपतन्तीं महाराज वेलामिव महोदधिः}


\chapter{अध्यायः ११५}
\twolineshloka
{धृतराष्ट्र उवाच}
{}


\twolineshloka
{कथं शान्तनवो भीष्मो दशमेऽहनि सञ्जय}
{अयुध्यत महावीर्यः पाण्डवैः सह सृज्जयैः}


\threelineshloka
{कुरवश्च कथं युद्धे पाण्डवान्प्रत्यवारयन्}
{आचक्ष्व मे महायुद्धं भीष्मस्याहवशोभिनः ॥सञ्जय उवाच}
{}


\twolineshloka
{कुरवः पाण्डवैः सार्धं यदयुध्यन्त भारत}
{यथा च तदभूद्युद्धं तत्तु वक्ष्यामि सांप्रतम्}


\twolineshloka
{गमिताः परलोकाय परमास्त्रैः किरीटिना}
{अहन्यहनि संप्राप्ते तावकानां महारथाः}


\twolineshloka
{यथाप्रतिज्ञं कौरव्यः स चापि समितिंजयः}
{पार्थानामकरोद्भीष्मः सततं समिति क्षयम्}


\twolineshloka
{कुरुभिः सहितं भीष्मं युध्यमानं परंतप}
{अर्जुनं च सपाञ्चाल्यं दृष्ट्वा संशयिता जनाः}


\twolineshloka
{दशमेऽहनि तस्मिंस्तु भीष्मार्जुनसमागमे}
{अवर्तत महारौद्रः सततं समिति क्षयः}


\twolineshloka
{तस्मिन्नयुतशो राजन्भूयशश्च परंतपः}
{भीष्मः शान्तनवो योधाञ्जघान परामास्त्रवित्}


\twolineshloka
{येषामज्ञातकल्पानि नामगोत्राणि पार्थिव}
{ते हतास्तत्र भीष्मेण शूराः सर्वेऽनिवर्तिनः}


\twolineshloka
{दशाहानि तथा तप्त्वा भीष्मः पाण्डववाहिनीम्}
{निरविद्यत धर्मात्मा जीविते स परंतपः}


\twolineshloka
{स क्षिप्रं वधमन्विच्छन्नात्मनोऽभिसुखं रणे}
{न हन्यां मानवश्रेष्ठान्संग्रामेऽभिमुखानिति}


\threelineshloka
{चिन्तयित्वा महाबाहुः पिता देवव्रतस्तव}
{अभ्याशस्थं महाराज पाण्डवं वाक्यमब्रवीत् ॥भीष्म उवाच}
{}


\twolineshloka
{युधिष्ठिर महाप्राज्ञ सर्वशास्त्रविशारद}
{शृणुष्व वचनं तात धर्म्यं स्वर्ग्यं च जल्पतः}


\twolineshloka
{निर्विष्णोऽस्मि भृशं तात देहेनानेन भारत}
{घ्नतश्च मे गतः कालः सुबहून्प्राणिनो रणे}


\threelineshloka
{तस्मात्पार्थं पुरोधाय पाञ्चालान्सृज्जयांस्तथा}
{मद्वधे क्रियतां यत्नो मम चेदिच्छसि प्रियम् ॥सञ्जय उवाच}
{}


\twolineshloka
{तस्य तन्मतमाज्ञाय पाण्डवः सत्यदर्शनः}
{भीष्मं प्रति ययौ राजा संग्रामे सह सृञ्जयैः}


\twolineshloka
{धृष्टद्युम्नस्ततो राजन्पाण्डवश्च युधिष्ठिरः}
{श्रुत्वा भीष्मस्य तां वाचं चोदयामासतुर्बलम्}


\twolineshloka
{अभिद्रवध्यं युद्ध्यध्वं भीष्मं जयत संयुगे}
{रक्षिताः सत्यसन्धेन जिष्णुना रिपुजिष्णुना}


\twolineshloka
{अयं चापि महेष्वासः पार्षतो वाहिनीपतिः}
{भीमसेनश्च समरे पालयिष्यति वो ध्रुवम्}


\twolineshloka
{मा वो भीष्माद्भयं किंचिदस्त्वद्य युधि सृञ्जयाः}
{ध्रुवं भीष्मं विजेष्यामः पुरस्कृत्य शिखण्डिनम्}


\twolineshloka
{ते तथा समयं कृत्वा दशमेऽहनि पाण्डवाः}
{ब्रह्मलोकपरा भूत्वा संजग्मुः क्रोधमूर्च्छिताः}


\twolineshloka
{शिखण्डिनं पुरस्कृत्य पाण्डवं च धनञ्जयम्}
{भीष्मस्य पातने यत्नं परमं ते समास्थिताः}


\twolineshloka
{ततस्त्व सुतादिष्टा नानाजनपदेश्वराः}
{द्रोणेन सहपुत्रेण सहसेना महाबलाः}


\twolineshloka
{दुःशासनश्च बलवान्सह सर्वैः सहोदरैः}
{भीष्मं समरमध्यस्थं पालयांचक्रिरे तदा}


\twolineshloka
{ततस्तु तावकाः शूराः पुरस्कृत्य महाव्रतम्}
{शिखण्डिप्रमुखान्पार्थान्योधयन्ति स्म संयुगे}


\twolineshloka
{चेदिभिस्तु सपाञ्चालैः सहितो वानरध्वजः}
{ययौ शान्तनवं भीष्मं पुरस्कृत्य शिखण्डिनम्}


\twolineshloka
{द्रोणपुत्रं शिनेर्नप्ता धृष्टकेतुस्तु पौरवम्}
{अभिमन्युः सहामात्यं दुर्योधमयोधयत्}


\twolineshloka
{विराटस्तु सहानीकः सहसेनं जयद्रथम्}
{वृद्धक्षत्रस्य दायादमाससाद परंतप}


\twolineshloka
{मद्रराजं महेष्वासं सहसैन्यं युधिष्ठिरः}
{भीमसेनोऽभिगुप्तस्तु नागानीकमुपाद्रवत्}


\twolineshloka
{अप्रधृष्यमनावार्यं सर्वशस्त्रभृतां वरम्}
{द्रोणं प्रति ययौ यत्तः पाञ्चाल्यः सह सोदरैः}


\twolineshloka
{कर्णिकारध्वजं चैव सिंहकैतुररिंदमः}
{प्रत्युज्जगाम सौभद्रं राजपुत्रो बृहद्बलः}


\twolineshloka
{शिखण्डिनं च पुत्रास्ते पाण्डवं च धनञ्जयम्}
{राजभिः समरे पार्थमभिपेतुर्जिघांसवः}


\twolineshloka
{तस्मिन्नतिमहाभीमे सेनयोर्वै पराक्रमे}
{संप्रधावत्स्वनीकेषु मेदिनी समकम्पत}


\twolineshloka
{तान्यनीकान्यनीकेषु समयुध्यन्त भारत}
{तावकानां परेषां च दृष्ट्वा शान्तनवं रणे}


\twolineshloka
{ततस्तेषां प्रयततामन्योन्यमभिधावताम्}
{प्रादुरासीन्महाशब्दो दिक्षु सर्वासु भारत}


\twolineshloka
{शङ्खदुन्दुभिघोषश्च वारणानां च बृंहितैः}
{सिंहनादश्च सैन्यानां दारुणः समपद्यत}


\twolineshloka
{सा च सर्वनरेन्द्राणां चन्द्रार्कसदृशी प्रभा}
{हाराङ्गदकिरीटेषु निष्प्रभा समपद्यत}


\twolineshloka
{रजोमेघास्तु संजज्ञुः शस्त्रविद्युद्भिरावृताः}
{धनुषां चापि निर्घोषो दारुणः समपद्यत}


\twolineshloka
{बाणशङ्खप्रणादाश्च भेरीणां च महास्वनाः}
{रथघोषश्च संजज्ञे सेनयोरुभयोरपि}


\twolineshloka
{प्रासशक्त्यृष्टिसङ्घैश्च बाणौघैश्च समाकुलम्}
{निष्प्रकाशमिवाकाशं सेनयोः समपद्यत}


\twolineshloka
{अन्योन्यं रथिनः पेतुर्वाजिनश्च महाहवे}
{कुञ्जरान्कुञ्जरा जघ्नुः पादातांश्च पदातयः}


\twolineshloka
{तत्रासीत्सुमहद्युद्धं कुरूणां पाण्डवैः सह}
{भीष्महेतोर्नरव्याघ्र श्येनयोरामिषे यथा}


\twolineshloka
{तयोः समागमो घोरो बभूव युधि भारत}
{अन्योन्यस्य वधार्थाय जिगीषूणां महाहवे}


\chapter{अध्यायः ११६}
\twolineshloka
{सञ्जय उवाच}
{}


\twolineshloka
{अभिमन्युर्महाराज तव पुत्रमयोधयत्}
{महत्या सेनया युक्तं भीष्महेतोः पराक्रमी}


\twolineshloka
{दुर्योधनो रणे कार्षिं नवभिर्नतपर्वभिः}
{आजघानोरसि क्रुद्धः पुनश्चैनं त्रिभिः शरैः}


\twolineshloka
{तस्य शक्तिं रणे कार्ष्णिर्मृत्योर्जिह्वामिवायसीम्}
{प्रेषयामास संक्रुद्धो दुर्योधनरथं प्रति}


\twolineshloka
{तामापतन्तीं सहसा घोररूपां विशांपते}
{द्विधा चिच्छेद ते पुत्रः क्षुरप्रेण महारथः}


\twolineshloka
{तां शक्तिं पतितां दृष्ट्वा कार्ष्णिः परमकोपनः}
{दुर्योधनं त्रिभिर्बाणैर्बाह्वोरुरसि चार्पयत्}


\twolineshloka
{पुनश्चैनं शरैर्घोरैराजघान स्तनान्तरे}
{दशभिर्भरतश्रेष्ठ दुर्योधनममर्षणम्}


\twolineshloka
{तद्युद्धमभवद्धोरं चित्ररूपं च भारत}
{ईक्षितृप्रीतितजननं सर्वपार्थिवपूजितम्}


\twolineshloka
{भीष्मस्य निधनार्थाय पार्थस्य विजयाय च}
{युयुधाते रणे वीरौ सौभद्रकुरुपुङ्गवौ}


\twolineshloka
{सात्यकिं रभसं युद्धे द्रौणिर्ब्राह्मणपुङ्गवः}
{आजघानोरसि क्रुद्धो नाराचेन परंतपः}


\twolineshloka
{शैनेयोऽपि गुरोः पुत्रं सर्वमर्मसु भारत}
{अताडयदमेयात्मा नवभिः कङ्कपत्रिभिः}


\twolineshloka
{अश्वत्थामा तु समरे सात्यकिं नवभिः शरैः}
{त्रिंशता च पुनस्तूर्णं बाह्वोरुरसि चार्पयत्}


\twolineshloka
{सोऽतिविद्धो महेष्वासो द्रोणपुत्रेण सात्यकिः}
{द्रोणपुत्रं त्रिभिर्बाणैराजघान महायशाः}


\twolineshloka
{पौरवो धृष्टकेतुं च शरैराच्छाद्य संयुगे}
{बहुधा दारयांचक्रे महेष्वासं महारथः}


\twolineshloka
{तथैव पौरवं युद्धे धृष्टकेतुर्महारथः}
{त्रिंशता निशितैर्बाणैर्विव्याधाशु महाभुजः}


\twolineshloka
{पौरवस्तु धनुश्छित्त्वा धृष्टकेतोर्महारथः}
{ननाद बलवन्नादं विव्याध च शितैः शरैः}


\twolineshloka
{सोऽन्यत्कार्मुकमादाय पौरवं निशितैः शरैः}
{आजघान महाराज त्रिसप्तत्या शिलीमुखैः}


\twolineshloka
{तौ तु तत्र महेष्वासौ महामात्रौ महारथौ}
{महता शरवर्षेण परस्परमवर्षताम्}


\twolineshloka
{अन्योन्यस्य धनुश्छित्त्वा हयान्हत्वा च भारत}
{विरथावसियुद्धाय समीयतुरमर्षणौ}


\twolineshloka
{आर्षभे चर्मणी चित्रे शतचन्द्रपुरस्कृते}
{तारकाशतचित्रे च निस्त्रिंशौ सुमहाप्रभौ}


\twolineshloka
{प्रगृह्य विमलौ राजंस्तावन्योन्यमभिद्रुतौ}
{वासितासंगमे यत्तौ सिंहाविव महावने}


\twolineshloka
{मण्डलानि विचित्राणि गतप्रत्यागतानि च}
{चेरतुर्दर्शयन्तौ च प्रार्थयन्तौ परस्परम्}


\twolineshloka
{पौरवो धृष्टकेतुं तु शङ्खदेशे महासिना}
{ताडयामास संक्रुद्धस्तिष्ठतिष्ठेति चाब्रवीत्}


\twolineshloka
{चेदिराजोऽपि समरे पौरवं पुरुषर्षभम्}
{आजघान शिताग्रेण जत्रुदेशे महासिना}


\twolineshloka
{तावन्योन्यं महाराज समासाद्य महाहवे}
{अन्योन्यवेगाभिहतौ निपेततुररिंदमौ}


\twolineshloka
{ततः स्वरथमारोप्य पौरवं तनयस्तव}
{जयत्सेनो रथेनाजावपोवाह रणाजिरात्}


\twolineshloka
{धृष्टकेतु तु समरे माद्रीपुत्रः प्रतापवान्}
{अपोवाह रणे क्रुद्धः सहदेवः पराक्रमी}


\twolineshloka
{चित्रसेनः सुशर्माणं विद्ध्वा बहुभिरायसैः}
{पुनर्विव्याध तं षष्ट्या पुनश्च नवभि शरैः}


\twolineshloka
{सुशर्मा तु रणे क्रुद्धस्तव पुत्रं विशांपते}
{दशमिर्दशभिश्चैव विव्याध निशितैः शरैः}


\twolineshloka
{चित्रसेनश्च तं राजंस्त्रिंशता नतपर्वभिः}
{आजघान रमे क्रुद्धः स च तं प्रत्यविध्यत}


\twolineshloka
{भीष्मस्य समरे राजन्यशो मानं च वर्धयन्}
{सौभद्रो राजपुत्रं तु बृहद्बलमयोधयत्}


\twolineshloka
{पार्थहेतोः पराक्रान्तो भीष्मस्य निधनं प्रति}
{आर्जुनिं कोसलेन्द्रस्तु विद्ध्वा पञ्चभिरायसैः}


\twolineshloka
{पुनर्विव्याध विंशत्या शरैः सन्नतपर्वभिः}
{सौभद्रः कोसलेन्द्रं तु विव्याधाष्टभिरायसैः}


\twolineshloka
{नाकम्पयत संग्रामे विव्याध च पुनः शरैः}
{कौसल्यस्य धनुश्चापि पुनश्चिच्छेद फाल्गुनिः}


\twolineshloka
{आजघान शरैश्चापि त्रिंशता कङ्कपत्रिभिः}
{सोऽन्यत्कार्मुकमादाय राजपुत्रो बृहद्बलः}


\twolineshloka
{फाल्गुनिं समरे क्रुद्धो विव्याध बहुभिः शरैः}
{तयोर्युद्धं समभवद्भीष्महेतोः परंतप}


\twolineshloka
{संरब्धयोर्महाराज समरे चित्रयोधिनोः}
{यथा देवासुरे युद्धे बलिवासवयोरभूत्}


\twolineshloka
{भीमसेनो गजानीकं योधयन्बह्वशोभत}
{यथा शक्रो वज्रपाणिर्दारयन्पर्वतोत्तमान्}


\twolineshloka
{ते वध्यमाना भीमेन मात्गा गिरिसन्निभाः}
{निपेतुरुर्व्यां सहिता नादयन्तो वसुधराम्}


\twolineshloka
{गिरिमात्रा हि ते नागा भिन्नाञ्जनचयोपमाः}
{विरेजुर्वसुधां प्राप्ता विकीर्णा इव पर्वताः}


\twolineshloka
{युधिष्ठिरो महेष्वासो मद्रराजानमाहवे}
{महत्या सेनया गुप्तं पीडयामास संगतम्}


\twolineshloka
{मद्रेश्वरश्च समरे धर्मपुत्रं महारथम्}
{पीडयामास संरब्धो भीष्महेतोः पराक्रमी}


\twolineshloka
{विराटं सैन्धवो राजा विद्ध्वा सन्नतपर्वभिः}
{नवभिः सायकैस्तीक्ष्णैस्त्रिंशता पुनरार्पयत्}


\twolineshloka
{विराटश्च महाराज सैन्धवं वाहिनीपतिः}
{त्रिंशद्भिर्निशितैर्बाणैराजघान स्तनान्तरे}


\twolineshloka
{चित्रकार्मुकनिस्त्रिंशौ चित्रवर्मायुधध्वजौ}
{रेजतुश्चित्ररूपौ तौ संग्रामे मत्स्यसैन्धवौ}


\twolineshloka
{द्रोणः पाञ्चालपुत्रेण समागम्य महारणे}
{महासमुदयं चक्रे शरैः सन्नतपर्वभिः}


\twolineshloka
{ततो द्रोणो महाराज पार्षतस्य महद्धनुः}
{छित्त्वा पञ्चाशतेषूंश्च पार्षतं प्रति सृष्टवान्}


\twolineshloka
{सोऽन्यत्कार्मुकमादाय पार्षतः परवीरहा}
{द्रोणस्य मिषतो युद्धे प्रेषयामास सायकान्}


\twolineshloka
{ताञ्छराञ्छरघातेन चिच्छेद स महारथः}
{द्रोणो द्रुपदपुत्राय प्राहिणोत्पञ्च सायकान्}


\twolineshloka
{ततः क्रुद्धो महाराज पार्षतः परवीरहा}
{द्रोणाय चिक्षेप गदां यमदण्डोपमां रणे}


\twolineshloka
{तामापतन्तीं सहसा हेमपट्टविभूषिताम्}
{शरैः पञ्चाशता द्रोणो वारयामास संयुगे}


\twolineshloka
{सा छिन्ना बहुधा राजन्द्रोगचापच्युतैः शरैः}
{चूर्णीकृता विशीर्यन्ती पपात वसुधातले}


\twolineshloka
{गदां विनिहतां दृष्ट्वा पार्षतः शत्रुतापनः}
{द्रोणाय शक्तिं चिक्षेप सर्वपारशवीं शुभाम्}


\twolineshloka
{तां द्रोणो नवभिर्बाणैश्चिच्छेद युधि भारत}
{पार्षतं च महेष्वासं पीजयामास संयुगे}


\twolineshloka
{एवमेतन्महायुद्धं द्रोणपार्षतयोरभूत्}
{भीष्मं प्रति महाराज घोररूपं भयानकम्}


\twolineshloka
{अर्जुनः प्राप्य गाङ्गेयं पीडयन्निशितैः शरैः}
{अभ्यद्रवत संयत्तो वने मत्तमिव द्विपम्}


\twolineshloka
{प्रत्युद्ययौ च तं राजा भगदत्तः प्रतापवान्}
{त्रिधा भिन्नेन नागेन मदान्धेन महाबलः}


\twolineshloka
{तमापतन्तं सहसा महेन्द्रगजसन्निभम्}
{परं यत्नं समस्थाय बीभत्सुः प्रत्यपद्यत}


\twolineshloka
{ततो गजगतो राजा भगदत्तः प्रतापवान्}
{अर्जुनं शरवर्षेण वारयामास संयुगे}


\twolineshloka
{अर्जुनस्तु ततो नागमायान्तं रजतोपमैः}
{विमलैरायसैस्तीक्ष्णैरविध्यत महारणे}


\twolineshloka
{शिखण्डिनं च कौन्तेयो याहियाहीत्यचोदयत्}
{भीष्मं प्रति महाराज जह्येनमिति चाब्रवीत्}


\twolineshloka
{प्राग्ज्योतिषस्ततो हित्वा पाण्डवं पाण्डुपूर्वज}
{प्रययौ त्वरितो राजन्द्रुपदस्य रथं प्रति}


\threelineshloka
{ततोऽर्जुनो महाराज भीष्ममभ्यद्रवद्द्रुतम्}
{शिखण्डिनं पुरस्कृत्य ततो युद्धमवर्तत}
{}


\twolineshloka
{ततस्ते तावकाः शूराः पाण्डवं रभसं युधि}
{समभ्यधावन्क्रोशन्तस्तदद्भुतमिवाभवत्}


\twolineshloka
{नानाविधान्यनीकानि पुत्राणां ते जनाधिप}
{अर्जुनो व्यधमत्काले दिवीवाभ्राणि मारुतः}


\twolineshloka
{शिखण्डी तु समासाद्य भरतानां पितामहम्}
{इषुभिस्तूर्णमव्यग्रो बहुभिः स समाचिनोत्}


\twolineshloka
{रथाग्न्यगारश्चापार्चिरशिशक्तिगदेन्धनः}
{शरसङ्घमहाज्वालः क्षत्रियान्समरेऽदहत्}


\twolineshloka
{यथाऽग्निः सुमहानिद्धः कक्षे चरति सानिलः}
{तथा जज्वाल भीष्मोऽपि दिव्यान्यस्त्राण्युदीरयन्}


\twolineshloka
{सोमकांश्च रणे भीष्मो जघ्ने पार्थपदानुगान्}
{न्यवारयत तत्सैन्यं पाण्डवस्य महारथः}


\twolineshloka
{सुवर्णपुङ्खैरिषुभिः शितैः सन्नतपर्वभिः}
{नादयन्स दिशो भीष्मः प्रदिशश्च महाहवे}


\twolineshloka
{पातयन्रथिनो राजन्हयांश्च सह सादिभिः}
{मुण्डतालवनानीव चकार स रथव्रजान्}


\twolineshloka
{निर्मनुष्यान्रथान्राजन्गजानश्वांश्च संयुगे}
{चकार समरे भीष्मः सर्वशस्त्रभृतां वरः}


\twolineshloka
{तस्य ज्यातलनिर्घोषं विस्फूर्जितमिवाशनेः}
{निशम्य सर्वतो राजन्समकम्पन्त सैनिकाः}


\twolineshloka
{अघोमा न्यपतन्बाणाः पितुस्ते मनुजेश्वर}
{नासञ्जन्त शरीरेषु भीष्मचापच्युताः शराः}


\twolineshloka
{निर्मनुष्यान्रथान्राजन्सुयुक्ताञ्जवनैर्हयैः}
{वातायमानानद्राक्षं ध्रियमाणान्विशांपते}


\twolineshloka
{चेदिकाशिकरूशानां सहस्राणि चतुर्दश}
{महारथाः समाख्याताः कुलपुत्रास्तनुत्यजः}


\twolineshloka
{अपरावर्तिनः शूराः सुवर्णविकृतध्वजाः}
{संग्रामे भीष्ममासाद्य सवाजिरथकुञ्जराः}


\twolineshloka
{जग्मुस्ते परलोकाय व्यादितास्यमिवान्तकम्}
{न तत्रासीद्रणे राजन्सोमकानां महारथः}


\twolineshloka
{यः संप्राप्य रणे भीष्मं जीविते स्म मनो दधे}
{तांश्च सर्वान्रणे योधान्प्रेतराजपुरं प्रति}


\twolineshloka
{नीतानमन्यन्त जना दृष्ट्वा भीष्मस्य विक्रमम्}
{न कश्चिदेनं समरे प्रत्युद्याति महारथः}


\twolineshloka
{ऋते पाण्डुसुतं वीरं श्वेताश्वं कृष्णसारथिम्}
{शिखण्डिनं च समरे पाञ्चाल्यममितौजसम्}


\chapter{अध्यायः ११७}
\twolineshloka
{सञ्जय उवाच}
{}


\twolineshloka
{शिखण्डी तु रणे भीष्ममासाद्य पुरुषर्षभम्}
{दशभिर्निशितैर्भल्लैराजघान स्तनान्तरे}


\twolineshloka
{शिखण्डिनं तु गाङ्गेयः क्रोधदीप्तेन चक्षुषा}
{संप्रेक्षत कटाक्षेण निर्दहन्निव भारत}


\twolineshloka
{स्त्रीत्वं तस्य स्मरन्राजन्सर्वलोकस्य पश्यतः}
{नाजघान रणे भीष्मः स च तन्नावबुद्धवान्}


\twolineshloka
{अर्जुनस्तु महाराज शिखण्डिनमभाषत}
{अभित्वरस्वास्य वधे शिखण्डिन्रथसत्तम}


\twolineshloka
{किं ते विवक्षया वीर जहि भीष्मं महारथम्}
{न ह्यन्यमनुपश्यामि कंचिद्यौधिष्ठिरे बले}


\twolineshloka
{यः शक्तः समरे भीष्मं योधयेत पितामहम्}
{ऋते त्वां पुरुषव्याघ्र सत्यमेतद्ब्रवीमि ते}


\twolineshloka
{एवमुक्तस्तु पार्थेन शिखण्डी भरतर्षभ}
{शरैर्नानाविधैस्तूर्णं पितामहमुपाद्रवत्}


\twolineshloka
{अचिन्तयित्वा तान्बाणान्पिता देवव्रतस्तव}
{अर्जुनं समरे क्रुद्धं वारयामास सायकैः}


\twolineshloka
{तथैव च चमूं सर्वां पाण्डवानां महारथः}
{अप्रैषीत्स शरैस्तीक्ष्णैः परलोकाय मारिष}


\twolineshloka
{तथैव पाण्डवा राजन्सैन्येन महता वृताः}
{भीष्मं संछादयामासुर्मेघा इव दिवाकरम्}


\twolineshloka
{स समन्तात्परिवृतो भीष्मो हि भरतर्षभ}
{निर्ददाह रणे शूरान्वने वह्निरिव ज्वलन्}


\twolineshloka
{तत्राद्भुतमपश्याम तव पुत्रस्य पौरुषम्}
{अयोधयच्च यत्पार्थं जुगोप च पितामहम्}


\twolineshloka
{कर्मणा तेन समरे तव पुत्रस्य धन्विनः}
{दुःशासनस्य तुतुषुः सर्वे लोका महात्मनः}


\twolineshloka
{यदेकः समरे पार्थान्सानुगान्समयोधयत्}
{न चैनं पाण्डवा युद्धे वारयामासुरुल्बणम्}


\twolineshloka
{दुःशासनेन समरे रथिनो विरथीकृताः}
{सादिनश्च महेष्वासा हस्तिनश्च महाबलाः}


\twolineshloka
{विनिर्भिन्नाः शरैस्तीक्ष्णैर्निपेतुर्वसुधातले}
{शरातुरास्तथैवान्ये दन्तिनो विद्रुता दिशः}


\twolineshloka
{यथाग्निरिन्धनं प्राप्य ज्वलेद्दीप्तार्चिरुल्बणम्}
{तथा जज्वाल पुत्रस्ते पाण्डुसेनां विनिर्दहन्}


\twolineshloka
{तं भारतमहामात्रं पाण्डवानां महारथः}
{जेतुं नोत्सहते कश्चिन्नाभ्युद्यातुं कथंचन}


\twolineshloka
{ऋते महेन्द्रतनयाच्छ्वेताश्वात्कृष्णसारथेः}
{स हि तं समरे राजन्निर्जित्य विजयोऽर्जुनः}


\twolineshloka
{भीष्ममेवाभिदुद्राव सर्वसैन्यस्य पश्यतः}
{विजितस्तव पुत्रोऽपि भीष्मबाहुव्यपाश्रयः}


\twolineshloka
{पुनः पुनः समाश्वस्य प्रायुध्यत मदोत्कटः}
{अर्जुनस्तु रणे राजन्योधयन्संव्यराजत}


\twolineshloka
{शिखण्डी तु रणे राजन्विव्याधैव पितामहम्}
{शरैरशनिसंस्पर्शैस्तथा सर्पविषोपमैः}


\twolineshloka
{न च स्म ते रुजं चक्रुः पितुस्तव जनेश्वर}
{स्मय्रमानस्तु गाङ्गेयस्तान्बाणाञ्जगृहे तदा}


\twolineshloka
{उष्णार्तो हि नरो यद्वज्जलधाराः प्रतीच्छति}
{तथा जग्राह गाङ्गेयः शरधाराः शिखण्डिनः}


\twolineshloka
{तं क्षत्रिया महाराज ददृशुर्घोरमाहवे}
{भीष्मं दहन्तं सैन्यानि पाण्डवानां महात्मनां}


\twolineshloka
{ततोऽब्रवीत्तव सुतः सर्वसैन्यानि मारिष}
{अभिद्रवत संग्रामे फल्गुनं सर्वतो रणे}


\twolineshloka
{भीष्मो वः समरे सर्वान्पायिष्यति धर्मवित्}
{तद्भयं सुमहत्त्यक्त्वा पाण्डवान्प्रति युध्यत}


\twolineshloka
{एष तालेन महता भीष्मस्तिष्ठति पालयन्}
{सर्वेषां धार्तराष्ट्राणां समरे शर्म वर्म च}


\twolineshloka
{देवा अपि समुद्युक्ता नालं भीष्मं समासितुम्}
{किमु पार्था महात्मानं मर्त्यभूता महाबलाः}


\twolineshloka
{तस्माद्द्रवत मा योधाः फल्गुनं प्राप्य संयुगे}
{अहमद्य रणे यत्तो योधयिष्यामि पाण्डवम्}


\twolineshloka
{सहितः सर्वतो यत्तैर्भवद्भिर्वसुधाधिपैः}
{तच्छ्रुत्वा तु वचो राजंस्तव पुत्रस्य धन्विनः}


\twolineshloka
{अर्जुनं प्रति संयत्ता बलवन्तो महाबलाः}
{ते विदेहाः कलिङ्गाश्च दासेरकगणाश्च ह}


\twolineshloka
{अभिपेतुर्निषादाश्च सौवीराश्च महारणे}
{बाह्लीका दरदाश्चैव प्रतीच्योदीच्यमालवाः}


\twolineshloka
{अभीषाहाः शूरसेनाः शिबयोऽथ वसातयः}
{साल्वः शकास्त्रिगर्ताश्च अम्बष्ठाः केकयौः सह}


\twolineshloka
{अभिपेतू रणे पार्थं पतङ्गा इव पावकम्}
{स तान्सर्वान्सहानीकान्महाराज महारथान्}


\twolineshloka
{दिव्यान्यस्त्राणि संचिन्त्य प्रसंधाय धनञ्जयः}
{स तैरस्त्रैर्महावेगैर्ददाह सुमहाबलः}


\twolineshloka
{शरप्रतापैर्बीभत्सुः पतङ्गानिव पावकः}
{तस्य बाणसहस्राणि सृजतो दृढधन्विनः}


\twolineshloka
{दीप्यमानमिवाकाशे गाण्डीवं समदृश्यत}
{ते शरार्ता महाराज विप्रकीर्णमहाध्वजाः}


\twolineshloka
{नाभ्यवर्तन्त राजानः सहिता वानरध्वजम्}
{सध्वजा रथिनः पेतुर्हयारोहा हयैः सह}


\twolineshloka
{सगजाश्च गजारोहाः किरीटिशरताडिताः}
{ततोऽर्जुनभुजोत्सृष्टैरावृताऽऽसीद्वसुंधरा}


\twolineshloka
{विद्रुतं दिक्षु सर्वासु शरैर्बलमदृश्यत}
{अथ पार्थो महाराज द्रावयित्वा वरूथिनीम्}


\twolineshloka
{दुःशासनाय सुबहून्प्रेषयामास सायकान्}
{ते तु भित्त्वा तव सुतं दुःशासनमयोमुखाः}


\twolineshloka
{धरणीं विविशुः सर्वे वल्मीकमिव पन्नगाः}
{हयांश्चास्य ततो जघ्ने सारथिं च न्यपातयत्}


\twolineshloka
{विविंशतिं च विंशत्या विरथीकृतवान्प्रभुः}
{आजघान भृशं चैव पञ्चभिर्नतपर्वभिः}


\twolineshloka
{कृपं विकर्णं शल्यं च विद्ध्वा बहुभिरायसैः}
{चकार विरथांश्चैव कौन्तेयः श्वेतवाहनः}


\twolineshloka
{एवं ते विरथाः सर्वे कृपः शल्यश्च मारिष}
{दुःशासनो विकर्णश्च तथैव च विविंशतिः}


\twolineshloka
{संप्राद्रवन्त समरे निर्जिताः सव्यसाचिना}
{पूर्वाह्णे भरतश्रेष्ठ पराजित्य महारथान्}


\twolineshloka
{प्रजज्वाल रणे पार्थो विधूम इव पावकः}
{तथैव शरवर्षेण भास्करो रश्मिवानिव}


\twolineshloka
{अन्यानपि महाराज तापयामास पार्थिवान्}
{पराङ्मुखीकृत्य तथा शरवर्षैर्महारथान्}


\twolineshloka
{प्रावर्तयत संग्रामे शोणितोदां महानदीम्}
{मध्येन कुरुसैन्यानां पाण्डवानां च भारत}


\twolineshloka
{तस्मिन्नतिमहाभीमे राजन्वीरवरक्षये}
{भीष्मं प्रति पराक्रान्ताः पाण्डवाः सह सृञ्जयैः}


\twolineshloka
{ते पराक्रान्तमालोक्य युधि राजन्पितामहम्}
{न न्यवर्तन्त ते पुत्रा ब्रह्मलोकपुरस्कृताः}


\twolineshloka
{इच्छन्तो निधनं युद्धे स्वर्गलोकपरायणाः}
{पाण्डवानभ्यवर्तन्त तावका युद्धदुर्मदाः}


\twolineshloka
{पाण्डवाश्च महाराज स्मरन्तो विविधान्बहून्}
{क्लेशान्कृतान्सपुत्रेण त्वया पूर्वं नराधिप}


\twolineshloka
{भयं त्यक्त्वा रणे शूराः स्वर्गलोकपुरस्कृताः}
{तावकांस्तव पुत्रांश्च योधयन्ति स्म हृष्टवत्}


\twolineshloka
{गजाश्च रथसङ्घाश्च बहुधा रथिभिर्हताः}
{रथाश्च निहता नागैर्हयाश्चैव पदातिभिः}


\twolineshloka
{अन्तरा च्छिद्यमानानि शरीराणि शिरांसि च}
{निपेतुर्दिक्षु सर्वासु गजाश्वरथयोधिनाम्}


\twolineshloka
{छन्नमायोधनं राजन्कुण्डलाङ्गदधारिभिः}
{पतितैः पात्यमानैश्च राजपुत्रैर्महारथैः}


\twolineshloka
{रथनेमिनिकृत्तैश्च गजैश्चैवावपोथितैः}
{पादातश्चाप्यधावन्त साश्वाश्च हययोधिनः}


\twolineshloka
{गजाश्च रथयोधाश्च परिपेतुः समन्ततः}
{विकीर्णाश्च रथा भूमौ भग्रचक्रयुगध्वजाः}


\twolineshloka
{तद्गजाश्वरथौघानां रुधिरेण समुक्षितम}
{छन्नमायोधनं रेजे रक्ताभ्रमिव शारदम्}


\twolineshloka
{श्वानः काकाश्च गृध्राश्च वृका गोमायुभिः सह}
{प्रणेर्दुर्भक्ष्यमासाद्य विकृताश्च सगद्विजाः}


\twolineshloka
{ववुर्बहुविधश्चैव दिक्षु सर्वासु मारुताः}
{दृश्यमानेषु रक्षःसु भूतेषु च नदत्सु च}


\twolineshloka
{काञ्चनानि च दामानि पताकाश्च महाधनाः}
{धूयमाना व्यदृश्यन्त सहसा मारुतेरिताः}


\twolineshloka
{श्वेतच्छत्रसहस्राणि सध्वजाश्च महारथाः}
{विकीर्णाः समदृश्यन्त शतशोऽथ सहस्रशः}


\twolineshloka
{सपताकाश्च मातङ्गा दिशो जग्मुः शरातुराः}
{क्षत्रियाश्च मनुष्येन्द्र गदाशक्तिधनुर्धराः}


\threelineshloka
{समन्ततश्च दृश्यन्ते पतिता धरणीतले}
{6-117-67b`इषुभिस्ताड्यमानाश्च नाराचैश्च सहस्रशः}
{' पेतुरार्तस्वरं कृत्वा तत्रतत्र महागजाः}


\twolineshloka
{सेनापतिस्तु समरे प्राह सेनां महारथः}
{अभ्यद्रवत गाङ्गेयं सैनिकाः किं कृतेन वः}


\twolineshloka
{सेनापतिवचः श्रुत्वा सोमकाः सह सृञ्जयैः}
{अभ्यद्रवन्त गोङ्गेयं शस्त्रवृष्ट्या समन्ततः}


\twolineshloka
{ततो भीष्मो महाराज दिव्यमस्त्रमुदीरयन्}
{अभ्यधावत कौन्तेयं मिषतां सर्वधन्विनाम्}


\twolineshloka
{तं शिखण्डी रणे यान्तमभ्यद्रवत दंशितः}
{ततः समाहरद्भीष्मस्तदस्त्रं पावकोपमम्}


\twolineshloka
{एतस्मिन्नेव काले तु कौन्तेयः श्वेतवाहनः}
{निजघ्ने तावकं सैन्यं मोहयित्वा पितामहम्}


\chapter{अध्यायः ११८}
\twolineshloka
{सञ्जय उवाच}
{}


\twolineshloka
{असंव्यूढेष्वनीकेषु भूयिष्ठमनुवर्तिषु}
{ब्रह्मलोकपराः सर्वे समपद्यन्त भारत}


\twolineshloka
{च ह्यनीकमनीकेषु समसञ्जत संकले}
{रथा न रथिभिः सार्धं पादाता न पदातिभिः}


\twolineshloka
{अश्वा नाश्वैरयुध्यन्त गजा न गजयोधिभिः}
{उन्मत्तवन्महाराज युध्यन्ते तत्र भारत}


\twolineshloka
{महान्व्यतिकरो रौद्रः सेनयोः समपद्यत}
{नरनागगणेष्वेवं विकीर्णेषु च सर्वशः}


\twolineshloka
{क्षये तस्मिन्महारौद्रे निर्विशेषमजायत}
{ततः शल्यः कृपश्चैव चित्रसेनश्च भारत}


\twolineshloka
{दुःशासनो विकर्णश्च रथानास्थाय भास्वरान्}
{पाण्डवानां रणे शूरा ध्वजिनीं समकम्पयन्}


\twolineshloka
{सा वध्यमाना समरे पाण्डुसेना महात्मभिः}
{त्रातारं नाध्यगच्छद्वै मञ्जमानेव नौर्जंले}


\twolineshloka
{यथा हि शैशिरः कालो गवां मर्माणि कृन्तति}
{तथा पाण्डुसुतानां वै भीष्मो मर्माणि कृन्तति}


\twolineshloka
{तथैव तव सैन्यस्य पार्थेन च महात्मना}
{नवमेघप्रतीकाशाः पातिता बहुधा गजाः}


\twolineshloka
{मृद्यमानाश्च दृश्यन्ते पार्थेन नरयूथपाः}
{इषुभिस्ताड्यमानाश्च नाराचैश्च सहस्रशः}


\twolineshloka
{पेतुरार्तस्वरं घोरं कृत्वा तत्र महागजाः}
{आनद्धाभरणैः कायैर्निहतानां महात्मनाम्}


\twolineshloka
{छन्नमायोधनं रेजे शिरोभिश्च सकुण्डलैः}
{6-118-12bतस्मिन्नेवमहाराज महावीरवरक्षये}


\twolineshloka
{भीष्मे च युधि विक्रान्ते पाण्डवे च धनञ्जये}
{ते पराक्रान्तमालोक्य राजन्युधि पितामहम्}


\twolineshloka
{अभ्यवर्तन्त ते पुत्राः सर्वे सैन्यपुरस्कृताः}
{इच्छन्तो निधनं युद्धे स्वर्गं कृत्वा परायणम्}


\twolineshloka
{पाण्डवानभ्यवर्तन्त तस्मिन्वीरवरक्षये}
{पाण्डवापि महाराज स्मरन्तो विविधान्बहून्}


\twolineshloka
{क्लेशान्कृतान्सपुत्रेण त्वया पूर्वं नराधिप}
{भयं त्यक्त्वा रणे शूरा ब्रह्मलोकाय तत्पराः}


\twolineshloka
{तावकांस्तव पुत्रांश्च योधयन्ति प्रहृष्टवत्}
{सेनापतिस्तु समरे प्राह सेनां महारथः}


\twolineshloka
{अभिद्रवत गाङ्गेयं सोमकाः सृञ्जयैः सह}
{सेनापतिवचः श्रुत्वा सोमकाः सृञ्जयाश्च ते}


\twolineshloka
{अभ्यद्रवन्त गाङ्गेयं शरवृष्ट्या समाहताः}
{वध्यमानस्ततो राजन्पिता शान्तनवस्तव}


\twolineshloka
{अमर्षवशमापन्नो योधयामास सृञ्जयान्}
{तस्य कीर्तिमतस्तात पुरा रामेण धीमता}


\twolineshloka
{संप्रदत्ताऽस्त्रशिक्षा वै परानीकविनाशनी}
{स तां विद्यामधिष्ठाय कुर्वन्तपरबलक्षयम्}


\twolineshloka
{अहन्यहनि पार्थानां वृद्धः कुरुपितामहः}
{भीष्मो दशसहस्राणि जघान परवीरहा}


\twolineshloka
{तस्मिंस्तु दशमे प्राप्ते दिवसे भरतर्षभ}
{एको भीष्मो हि मत्स्येषु पाञ्चालेषु च संयुगे}


\twolineshloka
{गजाश्वममितं हत्वा हत्वा सप्त महारथान्}
{हत्वा पञ्चसहस्राणि रथानां प्रतिपामहः}


\twolineshloka
{नराणां च महायुद्धे सहस्राणि चतुर्दश}
{दन्तिनां च सहस्राणि हयानामयुतं पुनः}


\twolineshloka
{शिक्षबलेन न्यहनत्पिता तव विशांपते}
{ततः सर्वमहीपानां क्षपयित्वा वरूथिनीम्}


\twolineshloka
{विराटस्य प्रियो भ्राता शतानीको निपातितः}
{शतानीकं च समरे हत्वा भीष्मः प्रतापवान्}


\twolineshloka
{सहस्राणि महाराज राज्ञां भल्लैरपातयत्}
{उद्विग्नाः समरे योधा विक्रोशन्ति धनञ्जयम्}


\twolineshloka
{ये च केचन पार्थानामभियाता धनञ्जयम्}
{राजानो भीष्ममासाद्य गतास्ते यमसादनम्}


\twolineshloka
{किरन्दश दिशो भीष्मः शरजालैः समन्ततः}
{अतीत्य सेनां पार्थानामवतस्थे चमूमुखे}


\twolineshloka
{स कृत्वा सुमहत्कर्म तस्मिन्वै दशमेऽहनि}
{सेनयोरन्तरे तिष्ठन्प्रगृहीतशरासनः}


\twolineshloka
{न चैनं पार्थिवाः केचिच्छक्ता राजन्निरीक्षितुम्}
{मध्यं प्राप्तं यथा ग्रीष्मे तपन्तं भास्करं दिवि}


\twolineshloka
{यथा दैत्यचमूं शक्रस्तापयामास संयुगे}
{तथा भीष्मः पाण्डवेयांस्तापयामास भारत}


\twolineshloka
{तथा चैनं पराक्रान्तमालोक्य मधुसूदनः}
{उवाच देवकीपुत्रः प्रीयमाणो धनञ्जयम्}


\twolineshloka
{एष शान्तनवो भीष्मः सेनयोरन्तरे स्थितः}
{सन्निहत्य बलादेनं विजयस्ते भविष्यति}


\twolineshloka
{बलात्संस्तम्भयस्वैनं यत्रैषा भिद्यते चमूः}
{न हि भीष्मशरानन्यः सोढुमुत्सहते विभो}


\twolineshloka
{ततस्तस्मिन्क्षणे राजंश्चोदितो वानरध्वजः}
{सध्वजं सरथं साश्वं भीष्ममन्तर्दधे शरैः}


\twolineshloka
{स चापि कुरुमुख्यानामृषभः पाण्डवेरितान्}
{सरव्रतैः शरव्रातान्बहुधा विदुधाव तान्}


\twolineshloka
{ततः पाञ्चालराजश्च धृष्टकेतुश्च वीर्यवान्}
{पाण्डवो भीमसेनश्च धृष्टद्युम्नश्च पार्षतः}


\twolineshloka
{यमौ च चेकितानश्च केकयाः पञ्च चैव ह}
{सात्यकिश्च महाबाहुः सौभद्रोऽथ घटोत्कचः}


\twolineshloka
{द्रौपदेयाः शिखण्डी च कुन्तिभोजश्च वीर्यवान्}
{विराटश्च महाराज पाण्डवेया महाबलाः}


\twolineshloka
{एते चान्ये च बहवः पीडिता भीष्मसायकैः}
{समुद्धृताः फल्गुनेन निमग्नाः शोकसागरे}


\twolineshloka
{ततः शिखण्डी वेगेन प्रगृह्य परमायुधम्}
{भीष्ममेवाभिदुद्राव रक्ष्यमाणः किरीटिना}


\twolineshloka
{ततोऽस्यानुचरान्हत्वा सर्वान्रणविभागवित्}
{भीष्ममेवाभिदुद्राव बीभत्सुरपराजितः}


\twolineshloka
{सात्यकिश्चेकितानश्च धृष्टद्युम्नश्च पार्षतः}
{विराटो द्रुपदश्चैव माद्रीपुत्रौ च पाण्डवौ}


\twolineshloka
{दुद्रुवुर्भीष्ममेवाजौ रक्षिता दृढधन्वना}
{अभिमन्युश्च समरे द्रौपद्याः पञ्च चात्मजाः}


\twolineshloka
{ददृशुः समरे भीष्मं समुद्यतमहायुधम्}
{ते सर्वे दृढधन्वानः संयुगेष्वपलायिनः}


\twolineshloka
{बहुधा भीष्ममानर्च्छुर्मार्गणैः क्षतमार्गणैः}
{विधूय तान्बाणगणान्ये मुक्ताः पार्थिवोत्तमैः}


\twolineshloka
{पाण्डवानामदीनात्मा व्यगाहत वरूथिनीम्}
{चक्रे शरविघातं च क्रीडन्निव पितामहः}


\twolineshloka
{नाभिसंधत्त पाञ्चाल्ये स्मयमानो मुहुर्मुहुः}
{स्त्रीत्वं तस्यानुसंस्मृत्य भीष्मो बाणाञ्शिखण्डिने}


\twolineshloka
{जघान द्रुपदानीके रथान्सप्त महारथः}
{ततः किलकिलाशब्दः क्षणेन समभूत्तदा}


\twolineshloka
{मत्स्यपाञ्चालचेदीनां तमेकमभिधावताम्}
{ते नराश्वरथव्रातैर्मार्गणैश्च परंतप}


\twolineshloka
{तमेकं छादयामासुर्मेघा इव दिवाकरम्}
{भीष्मं भागीरथीपुत्रं प्रतपन्तं रणे रिपून्}


\twolineshloka
{ततस्तस्य च तेषां च युद्धे देवासुरोपमे}
{किरीटी भीष्ममानर्च्छ पुरस्कृत्य शिखण्डिनम्}


\chapter{अध्यायः ११९}
\twolineshloka
{सञ्जय उवाच}
{}


\twolineshloka
{एवं ते पाण्डवाः सर्वे पुरस्कृत्य शिखण्डिनम्}
{विव्यधुः समरे भीष्मं परिवार्य समन्ततः}


\twolineshloka
{शतघ्नीभिः सुघोराभिः परिघैश्च परश्वथैः}
{मुद्गरैर्मुसलैः प्रासैः क्षेपणीयैश्च सर्वशः}


\twolineshloka
{शरैः कनकपुङ्खैश्च शक्तितोमरकम्पनैः}
{नाराचैर्वत्सदन्तैश्च भुशुण्डीभिश्च सर्वशः}


\twolineshloka
{अताडयन्रणे भीष्मं सहिताः सर्वसृञ्जयाः}
{स विशीर्णतनुत्रामाः पीडितो बहुभिस्तदा}


\twolineshloka
{न विव्यथे तदा भीष्मो भिद्यमानेषु मर्मसु}
{संदीप्तशरचापाग्निरस्त्रप्रसृतमारुतः}


\twolineshloka
{नेमिनिर्ह्रादसंतापो महास्त्रोदयपावकः}
{चित्रचापमहाज्वालो वीरक्षयमहेन्धनः}


\twolineshloka
{युगान्तागनिसमप्रख्यः परेषां समपद्यत}
{निवृत्य रथसङ्घानामन्तरेण विनिःसृतः}


\twolineshloka
{दृश्यते स्म नेरन्द्राणां पुनर्मध्यगतश्चरन्}
{ततः पाञ्चालराजं च धृष्टकेतुमतीत्य च}


\twolineshloka
{पाण्डवानीकिनीमध्यमाससाद विशांपते}
{ततः सात्यकिभीमौ च पाण्डवं च धनञ्जयम्}


\twolineshloka
{द्रुपदं च विराटं च धृष्टद्युम्नं च पार्षतम्}
{भीमघोषैर्महावेगैर्मर्मावरणभेदिभिः}


\twolineshloka
{षडेतान्निशितैर्भीष्मः प्रविव्याधोत्तकैः शरैः}
{तस्य ते निशितान्बाणात्सन्निवार्य महारथाः}


\twolineshloka
{दशभिर्दशभिर्भीष्ममर्दयामासुरोजसा}
{शिखण्डी तु महाबाणान्यान्मुमोच महारथः}


\twolineshloka
{न चक्रुस्ते रुजं तस्य स्वर्मपुङ्खाः शिलाशिताः}
{ततः किरीटी संरब्धो भीष्ममेवाभ्यधावत}


\twolineshloka
{शिखण्डिनं पुरस्कृत्य धनुश्चास्य समाच्छिनत्}
{भीष्मस्य धनुषश्छेदं नामृष्यन्त महारथाः}


\twolineshloka
{द्रोणश्च कृतवर्मा च सैन्धवश्च जयद्रथः}
{भूरिश्रवाः शलः शल्यो भगदत्तस्तथैव च}


\twolineshloka
{सप्तैते परमक्रुद्धाः किरीटिनमभिद्रुताः}
{तत्र शस्त्राणि दिव्यानि दर्शयन्तो महारथाः}


\twolineshloka
{अभिपेतुर्भृशं क्रुद्धाश्चादयन्तश्च पाण्डवम्}
{तेषामापततां शब्दः शुश्रुवे फल्गुनं प्रति}


\twolineshloka
{उद्धूतानां यथा शब्दः समुद्राणां युगक्षये}
{घ्नत त्वरत गृह्णीत विद्ध्यध्वमवकर्तत}


\twolineshloka
{इत्यासीत्तुमुलः शब्दः फल्गुनस्य रथं प्रति}
{तं शब्दं तुमुलं श्रुत्वा पाण्डवानां महारथाः}


\twolineshloka
{अभ्यधावन्परीप्सन्तः फल्गुनं भरतर्षभ}
{सात्यकिर्भीमसेनश्च धृष्टद्युम्नश्च पार्षतः}


\twolineshloka
{विराटद्रुषदौ चोभौ राक्षसश्च घटोत्कचः}
{अभिमन्युश्च संक्रुद्धः सप्तैते क्रोधमूर्च्छिताः}


\twolineshloka
{समभ्यधावंस्त्वरिताश्चित्रकार्मुकधारिणः}
{तेषां समभवद्युद्धं तुमुलं रोमहर्षणम्}


\twolineshloka
{संग्रमे भरतश्रेष्ठ देवानां दानवैरिव}
{शिखण्डी तु रणे श्रेष्ठो रक्ष्ममाणः किरीटिना}


\twolineshloka
{अविध्यद्दशभिर्भीषअमं छिन्नधन्वानमाहवे}
{सारथिं दशभिश्चास्य ध्वजं चैकेन चिच्छिदे}


\twolineshloka
{सोऽन्यत्कार्मुकमादाय गाङ्गेयो वेगवत्तरम्}
{तदप्यस्य शितैर्बाणैस्त्रिभिश्चिच्छेद फल्गुनः}


\twolineshloka
{एवं स पाण्डवः क्रुद्ध आत्तमात्तं पुनः पुनः}
{धनुश्चिच्छेद भीष्मस्य सव्यसाची परंतपः}


\twolineshloka
{स च्छिन्नधन्वा संक्रुद्धाः सृक्किणी परिसंलिहन्}
{शक्तिं जग्राह तरसा गिरीणामपि दारणीम्}


\twolineshloka
{तां च चिक्षेप संक्रुद्धः फल्गुनस्य रथं प्रति}
{तामापतन्तीं संप्रेक्ष्य ज्वलन्तीमशनीमिव}


\twolineshloka
{समादत्त शितान्भल्लान्पञ्च पाण्डवनन्दनः}
{तस्य चिच्छेद तां शक्तिं पञ्चधा पञ्चभिः शरैः}


\twolineshloka
{संक्रुद्धो भरतश्रेष्ठ भीष्मवाहुप्रवेरिताम्}
{सा पपात तथा च्छिन्ना संक्रुद्धेन किरीटिना}


\threelineshloka
{मेघबृन्दपरिभ्रष्टा विच्छिन्नेव शतह्रदा}
{छिन्नां तां शक्तिमालोक्य भीष्मः क्रोधसमन्वितः}
{}


\twolineshloka
{अचिन्तयद्रणे वीरो बुद्ध्या परपुरंजयः}
{शक्तोऽहं धनुषैकेन निहन्तुं सर्वपाण्डवान्}


\threelineshloka
{यद्येषां न भवेद्गोप्ता विष्वक्सेनो महाबलः}
{`अजय्यश्चैव सर्वेषां लोकानामिति मे मतिः}
{'कारणद्वयमास्थाय नाहं योत्स्यामि पाण्डवान्}


\twolineshloka
{अवध्यत्वाच्च पाण्डूनां स्त्रीभावाच्च शिखण्डिनः}
{पित्रा तुष्टेन मे पूर्वं यदा कालीमुदवहम्}


\twolineshloka
{स्वच्छन्दमरणं दत्तमवध्यत्वं रणे तथा}
{तस्मान्मृत्युमहं मन्ये प्राप्तकालमिवात्मनः}


\twolineshloka
{एवं ज्ञात्वा व्यवसितं भीष्मस्यामिततेजसः}
{ऋषयो वसवश्चैव वियत्स्था भीष्ममब्रुवन्}


\twolineshloka
{यत्ते व्यवसितं तात तदस्माकमपि प्रियम्}
{तत्कुरुष्व महाराज युद्धे बुद्धिं निवर्तय}


\twolineshloka
{अस्य वाक्यस्य निधने प्रादुसासीच्छिवोऽनिलः}
{अनुलोमः सुगन्धी च पृषतैश्च समन्वितः}


\twolineshloka
{देवदुन्दुभयश्चैव संप्रणेदुर्महास्वनाः}
{पपात पुष्पवृष्टिश्च भीष्मस्योपरि मारिष}


\twolineshloka
{न च ताः शुश्रुवे कश्चित्तेषां संवदतां गिरः}
{ऋते भीष्मं महाबाहुं मां चापि मुनितेजसा}


\twolineshloka
{संभ्रमश्च महानासीत्रिदशानां विशांपते}
{पतिष्यति रथाद्भीष्मे सर्वलोकप्रिये तदा}


\twolineshloka
{इति देवगणानां च वाक्यं श्रुत्वा महातपाः}
{ततः शान्तनवो भीष्मो बीभत्सुं नाभ्यवर्तत}


\twolineshloka
{भिद्यमानः शितैर्बाणैः सर्वावरणभेदिभिः}
{शिखण्डी तु महाराज भरतानां पितामहम्}


\twolineshloka
{आजघानोरसि क्रुद्धो नवभिर्निशितैः शरैः}
{स तेनाभिहतः सङ्ख्ये भीष्मः कुरुपितामहः}


\twolineshloka
{नाकम्पत महाराज क्षितिकम्पे यथाऽचलः}
{ततः प्रहस्य बीभत्सुर्व्याक्षिपन्गाण्डिवं धनुः}


\twolineshloka
{गाङ्गेयं पञ्चविंशत्या क्षुद्रकाणां समार्पयत्}
{पुनः पुनः शतैरेनं त्वरमाणो धनञ्जयः}


\twolineshloka
{सर्वगात्रेषु संक्रुद्धस्तथा मर्मस्वताडयत्}
{एवमन्यैरपि भृशं विद्ध्यमानः सहस्रशः}


\twolineshloka
{तानप्याशु शरैर्भीष्मः प्रविव्याध महारथः}
{तैश्च मुक्ताञ्छरान्भीष्मो युधि सत्यपराक्रमः}


\twolineshloka
{निवारयामास शरैः समं सन्नतपर्वभिः}
{शिखण्डी तु रणे बाणान्यान्मुमोच महारथः}


\twolineshloka
{न चक्रुस्ते रुजं तस्य रुक्मपुङ्खाः शिलासिताः}
{ततः किरीटी संक्रुद्धो भीष्ममेवाभ्यवर्तत}


\twolineshloka
{शिखण्डिनं पुरस्कृत्य धनुश्चास्य समाच्छिनत्}
{अथैनं नवभिर्विद्ध्वा ध्वजमेकेन चिच्छिदे}


\twolineshloka
{सारथिं विशिखैश्चास्य दशभिः समकम्पयत्}
{सोऽन्यत्कार्मुकमादाय गाङ्गेयो बलवत्तरम्}


\twolineshloka
{तदप्यस्य शितैर्भल्लैस्त्रिधा त्रिभिरघातयत्}
{निमेषार्धेन कौन्तेय आत्तमात्तं महारणे}


\twolineshloka
{एवमस्य धनूंष्याजौ चिच्छेद सुबहून्यथ}
{ततः शान्तनवो भीष्मो बीभत्सुं नात्यवर्तत}


\twolineshloka
{अथैनं पञ्चविंशत्या क्षुद्रकाणां समार्पयत्}
{सोऽतिविद्धो महेष्वासो दुःशासनमभाषत}


\twolineshloka
{एष पार्थो रणे क्रुद्धः पाण्डवानां महारथः}
{शरैरनेकसाहस्त्रैर्मामेवाभ्यपतद्रणे}


\twolineshloka
{न चैष समरे शक्यो जेतुं वज्रभृता अपि}
{न चापि सहिता वीरा देवदानवराक्षसाः}


\twolineshloka
{मां चापि शक्ता निर्जेतुं किमु मर्त्या महारथाः}
{6-119-58b`ऋतेऽर्जुनं सुसंक्रुद्धमेतत्सत्यं ब्रवीमि ते ॥'}


\twolineshloka
{एवं तयोः संवदतोः फल्गुनो निशितैः शरैः}
{शिखण्डिनं पुरस्कृत्य भीष्मं विव्याध संयुगे}


\twolineshloka
{ततो दुःशासनं भीष्मः स्मयमान इवाब्रवीत्}
{अर्दितो निशितैर्बाणैर्भृशं गाण्डीवधन्वना}


\twolineshloka
{वज्राशनिसमस्पर्शाः सुपुङ्खाः सुप्रतेजनाः}
{सुमुक्ता अव्यवच्छिन्ना नेमे बाणाःशिखण्डिनः}


\twolineshloka
{निकृन्तमाना मर्माणि दृढावरणभेदिनः}
{सुसला इव मे ध्नन्ति नेमे बाणः शिखण्डिनः}


\twolineshloka
{वज्रदण्डसमस्पर्शा वज्रवेगदुरासदाः}
{मम प्राणानारुजन्ति नेमे बाणाः शिखण्डिनः}


\twolineshloka
{नाशयन्तीव मे प्राणान्यमदूता इवागताः}
{गदापरिघसंस्पर्शा नेमे बाणाः शिखण्डिनः}


\twolineshloka
{भुजगा इव संक्रुद्धा लेलिहाना विषोल्बणाः}
{समाविशन्ति मर्माणि नेमे बाणाः शिखण्डिनः}


\twolineshloka
{अर्जुनस्य इमे बाणा नेमे बाणाः शिखण्डिनः}
{कृन्तन्ति मम गात्राणि माघमां सेगवा इव}


\twolineshloka
{सर्वे ह्यपि न मे दुःखं कुर्युरन्ये नराधिपाःक}
{वीरं गाण्डीवधन्वानमृते जिष्णुं कपिध्वजम्}


\twolineshloka
{इति ब्रुवञ्छान्तनवो दिधक्षुरिव पाण्डवम्}
{शक्तिं भीष्मः स पार्थाय ततश्चिक्षेप भारत}


\twolineshloka
{तामस्य विशिखैश्छित्त्वा त्रिधा त्रिभिरपातयत्}
{पश्यतां कुरुवीराणां सर्वेषां तत्र भारत}


\twolineshloka
{चर्माथादत्त गाङ्गेयो जातरूपपरिष्कृतम्}
{खङ्गं चान्यतरप्रेप्सुर्मृत्योरग्रे जयाय वा}


\twolineshloka
{तस्य तच्छतधा चर्म व्यधमत्सायकैस्तथा}
{रथादनवरूढस्य तदद्भुतमिवाभवत्}


\twolineshloka
{ततो युधिष्ठिरो राजा स्वान्यनीकान्यचोदयत्}
{अभिद्रवत गाङ्गेयं मा वोऽस्तु भयमण्वपि}


\twolineshloka
{अथ ते तोमरैः प्रासैर्बाणौघैश्च समन्ततः}
{पट्टसैश्च सुनिस्त्रिंशैर्नाराचैश्च तथा शितैः}


\twolineshloka
{वत्सदन्तैश्च भल्लैश्च तमेकमभिदुद्रुवुः}
{सिंहनादस्ततो घोरः पाण्डवानामभूत्तदा}


\twolineshloka
{तथैव तव पुत्राश्च नेदुर्भीष्मजयैषिणः}
{तमेकमभ्यरक्षन्त सिंहनादांश्च चक्रिरे}


\twolineshloka
{तत्रासीत्तुमुलं युद्धं तावकानां परैः सह}
{दशमेऽहनि राजेन्द्र भीष्मार्जुनसमागमे}


\twolineshloka
{आसीद्धोर इवावर्तो मुहूर्तमुदधेरिव}
{सैन्यानां युध्यमानानां निघ्नतामितरेतरम्}


\twolineshloka
{असौम्यरूपा पृथिवी शोणिताक्ताऽभवत्तदा}
{समं च विषमं चैव न प्राज्ञायत किंचन}


\twolineshloka
{योधानामयुतं हत्वा तस्मिन्स दशमेऽहनि}
{अतिष्ठदाहवे भीष्मो भिद्यमानेषु मर्मसु}


\twolineshloka
{ततः सेनामुखे तस्मिन्स्थितः पार्थो धनुर्धरः}
{मध्येन कुरुसैन्यानां द्रावयामास वाहिनीम्}


\twolineshloka
{`तत्राद्भुतमपश्याम पाण्डवानां पराक्रमम्}
{द्रावयामासुरिषुभिः सर्वान्भीष्मपदानुगान् ॥'}


\twolineshloka
{वयं श्वेतहयाद्भीताः कुन्तीपुत्राद्धनञ्जयात्}
{पीड्यमानाः शितैः शस्त्रैः प्राद्रवाम रणे तदा}


\twolineshloka
{`पाण्डवैः पञ्चभिः सार्धं सात्यकेन च धन्विना}
{धृष्टद्युम्नसुखैः सर्वैः पाञ्चालैश्च समन्ततःक}


\threelineshloka
{भिद्यमानाः शरैस्तीक्ष्णैः सर्वे कार्ष्णिपुरोगमैः}
{द्रोणद्रौणिकृतैः सार्धं सर्वे शल्यशलादयः}
{तावकाः समरे राजञ्जहुर्भीष्मं महामृधे ॥'}


\twolineshloka
{सौवीराः कितवाः प्राच्याःप्रतीच्योदीच्यमालवाः}
{अभीषाहाः शूरसेनाः शिबयोऽथ वसातयः}


\threelineshloka
{साल्वाश्रयास्त्रिगर्ताश्च अम्बष्ठाः केकयैः सह}
{द्वादशैते जनपदाः शरार्ता व्रणपीडिताः}
{संग्रामे प्रजहुर्भीष्मं युध्यमानाः किरीटिना}


\twolineshloka
{ततस्तमेकं बहवः परिवार्य समन्ततः}
{परिकाल्य कुरून्सर्वाञ्शरर्षैरवाकिरन्}


\twolineshloka
{निपातयत गृह्णीत युध्यध्वमवकृन्तत}
{इत्यासीत्तुमुलः शब्दो राजन्भीष्मरथं प्रति}


\twolineshloka
{निहत्य समरे राजञ्शतशोऽथ सहस्रशः}
{न तस्यासीदनिर्भिन्नं गात्रे द्व्यङ्गुलमन्तरम्}


\twolineshloka
{एवंभूतस्तव पिता शरैर्विशकलीकृतः}
{शिताग्रैः फल्गुनेनाजौ प्राक्शिराः प्रापतद्रथात्}


\twolineshloka
{किंचिच्छेषे दिनकरे पुत्राणां तव पश्यताम्}
{हाहेति दिवि देवानां पार्थिवानां च भारत}


\twolineshloka
{पतमाने रथाद्भीष्मे बभूव सुमहास्वनः}
{संपतन्तमभिप्रेक्ष्य महात्मानं पितामहम्}


\twolineshloka
{सह भीष्मेण सर्वेषां प्रापतन्हृदयानि नः}
{स पपात महाबाहुर्वसुधामनुनादयन्}


\twolineshloka
{इन्द्रध्वज इवोत्सृष्टः केतुः सर्वधनुष्मताम्}
{धरणीं न स पस्पर्श शरसङ्घैः समावृतः}


\twolineshloka
{शलतल्पे महेष्वासं शयानं पुरुषर्षभम्}
{रथात्प्रपतितं चैनं दिव्यो भावः समाविशत्}


\twolineshloka
{अभ्यवर्षच्च पर्जन्यः प्राकम्पत च मेदिनी}
{पतन्स ददृशे चापि दक्षिणेन दिवाकरम्}


\twolineshloka
{संज्ञां चोपालभद्वीरः कालं संचिन्त्य भारत}
{अन्तरिक्षे च शुश्राव दिव्या वाचः समन्ततः}


\twolineshloka
{कथं महात्मा गाङ्गेयः सर्वशस्त्रभृतां वरः}
{कालकर्ता नरव्याघ्रः संप्राप्ते दक्षिणायने}


\twolineshloka
{स्थितोऽस्मीति च गाङ्गेयस्तच्छ्रुत्वा वाक्यमब्रवीत्}
{धारयिष्याम्यहं प्राणान्पतितिऽपि महीतले}


\twolineshloka
{उत्तरायणमन्विच्छन्सुगतिप्रतिकाङ्क्षया}
{तस्य तन्मतमाज्ञाय गह्गा हिमवतः सुताः}


\twolineshloka
{महर्षीन्हंसरूपेण प्रेषयामास तत्र वै}
{ततस्तं प्रति ते हंसास्त्वरिता मानसौकसः}


\twolineshloka
{आजग्मुः सहिता द्रष्टुं भीष्मं कुरुपितामहम्}
{यत्र शेते नरश्रेष्ठः शरतल्पे पितामहः}


\twolineshloka
{ते तु भीष्मं समासाद्य ऋषयो हंसरूपिणः}
{अपश्यञ्छरतल्पस्थं भीष्मं कुरुकुलोद्वहम्}


\twolineshloka
{ते तं दृष्ट्वा महात्मानं कृत्वा चापि प्रदक्षिणम्}
{गाङ्गेयं भरतश्रेष्ठं दक्षिणेन च भास्करम्}


\twolineshloka
{इतरेतरमामन्त्र्य प्राहुस्तत्र मनीषिणः}
{भीष्मः कथं महात्मा सन्संस्थाता दक्षिणायने}


\twolineshloka
{इत्युक्त्वा प्रस्थिता हंसा दक्षिणामभितो दिशम्}
{संप्रेक्ष्य वै महाबुद्धिश्चिन्तयित्वा च भारत}


\twolineshloka
{तानब्रवीच्छान्तनवो नाहं गन्ता कथंचन}
{दक्षिणावर्त आदित्ये एतन्मे मनसि स्थितम्}


\twolineshloka
{गमिष्यामि स्वकं स्थानमासीद्यन्मे पुरातनम्}
{उदगायन आदित्ये हंसाः सत्यं ब्रवीमि वः}


\twolineshloka
{धारयिष्याम्यहं प्राणानुत्तरायणकाङ्क्षया}
{प्राणानां च समुत्सर्ग ऐश्वर्यं नियतं मम}


\twolineshloka
{तस्मात्प्राणान्धारयिष्ये मुमूर्षुरुदगायने}
{यश्च दत्तो वरो मह्यं पित्रा तेन महात्मना}


\twolineshloka
{छन्दतस्ते भवेन्मृत्युरिति तत्सत्यमस्तु मे}
{धारयिष्ये ततः प्राणानुत्सर्गे नियते सति}


\twolineshloka
{इत्युक्त्वा तांस्तदा हंसान्स शेते शरतल्पगः}
{एवं कुरूणां पतिते शृङ्गे भीष्मे महौजसि}


\twolineshloka
{पाण्डवाः सृञ्जयाश्चैव सिंहनादं प्रचक्रिरे}
{तस्मिन्हते महासत्वे भरतानां पितामहे}


\twolineshloka
{न किंचित्प्रत्यपद्यन्त पुत्रास्ते भरतर्षभ}
{संमोहश्चैव तुमुलः करूणामभवत्तदा}


\twolineshloka
{कृपदुर्योधनमुखा निःश्वस्य रुरुदुस्ततः}
{विषादाच्च चिरं कालमतिष्ठन्विगतेन्द्रियाः}


\twolineshloka
{दध्युश्चैव महाराज न युद्धे दधिरे मनः}
{ऊरुग्राहगृहीताश्च नाभ्यधावन्त पाण्डवान्}


\threelineshloka
{अवध्ये शन्तनोः पुत्रे हते भीष्मे महौजसि}
{`दुःखार्तास्ते ततो राजन्कुरूणां पतयोऽभवन्}
{'अभाव सहसा राजन्कुरुराजस्य तर्कितः}


\twolineshloka
{हतप्रवीरास्तु वयं निकृत्ताश्च शितैः शरैः}
{कर्तव्यं नाभिजानीमो निर्जिताः सव्यसाचिना}


\twolineshloka
{पाणडवाश्च जयं लब्ध्वा परत्र च परां गतिम्}
{सर्वे दध्युर्महाशङ्खञ्शूराः परिघवाहवः}


\twolineshloka
{सोमकाश्च सपाञ्चालाः प्राहृष्यन्त जनेश्वर}
{ततस्तूर्यसहस्रेषु नदत्सु स महाबलः}


\twolineshloka
{आस्फोटयामास भृशं भीमसेनो ननाद च}
{सेनयोरुभयोश्चापि गाङ्गेये निहते विभौ}


\twolineshloka
{संन्यस्य विराः शस्त्राणि प्राध्यायन्त समन्ततः}
{प्राक्रोशन्प्राद्रवंश्चान्ये जग्मुर्मोहं तथाऽपरे}


\twolineshloka
{क्षत्रं चान्येऽभ्यनिन्दन्त भीष्मं चान्येऽभ्यपूजयन्}
{ऋषयः पितरश्चैव प्रशशसुर्महाव्रतम्}


\twolineshloka
{भरतानां च ये पूर्वे ते चैनं प्रशशंसिरे}
{महोपनिषदं चैव योगमास्थाय विर्यवान्}


% Check verse!
जपञ्शान्तनवो धीमान्कालाकाङ्क्षी स्थितोऽभवत्
\chapter{अध्यायः १२०}
\twolineshloka
{धृतराष्ट्र उवाच}
{}


\twolineshloka
{कथमासंस्तदा योधा हीना भीष्मेण सञ्जय}
{बलिना देवकल्पेन कौमारब्रह्मचारिणा}


\twolineshloka
{तदैव निहतान्मन्ये कुरूनन्यांश्च पाण्डवैः}
{न प्राहरद्यदा भीष्मो घृणित्वाद्द्रुपदात्मजम्}


\twolineshloka
{ततो दुःखतरं मन्ये किमन्यत्प्रभविष्यति}
{अद्यैव निहतं श्रुत्वा पितरं मम दुर्मतेः}


\twolineshloka
{अश्मसारमयं नूनं हृदयं मम सञ्जय}
{श्रुत्वा विनिहतं भीष्मं शतधा यन्न दीर्यते}


\twolineshloka
{यदन्यन्निहतेनाजौ भीष्मेण जयमिच्छता}
{चेष्टितं कुरुसिंहेन तन्मे कथय सुव्रत}


\twolineshloka
{पुनःपुनर्न मृष्यामि हतं देवव्रतं रणे}
{न हतो जामदग्न्येन दिव्यैरस्त्रैश्च यः पुरा}


\threelineshloka
{स हतो द्रौपदेयेन पाञ्चाल्येन शिखण्डिना}
{सञ्जय उवाच}
{सायाह्ने न्यपतद्भूमौ धार्तराष्ट्रान्विषादयन्}


\twolineshloka
{पाञ्चालान्हर्षयंश्चैव भीष्मः कुरुपितामहः}
{स शेते शरतल्पस्थो मेदिनीमस्पृशंस्तदा}


\twolineshloka
{भीष्मो रथात्प्रपतितः संछिन्नो बहुभिः शरैः}
{हाहेति तुमुलः शब्दो भूतानां समपद्यत}


\twolineshloka
{सीमावृक्षे निपतिते कुरूणां समितिंजये}
{सेनयोरुभयो राजन्क्षत्रियान्भयमाविशत्}


\twolineshloka
{भीष्मं शान्तनवं दृष्ट्वा विशीर्णकवचध्वजम्}
{कुरवः पर्यवर्तन्त पाण्डवाश्च विशांपते}


\twolineshloka
{खं तमःसंवृतमभूदासीद्भानुर्गतप्रभः}
{ररास पृथिवी चैव भीष्मे शान्तनवे हते}


\twolineshloka
{अयं ब्रह्मविदां श्रेष्ठो गतिर्ब्रह्मविदां सदा}
{इत्यभाषन्त भूतानि शयानं पुरुषर्षभम्}


\twolineshloka
{अयं पितरमाज्ञाय कामार्तं शन्तनुं पुरा}
{ऊर्ध्वरेतसमात्मानं चकार पुरुषर्षभः}


\twolineshloka
{इति स्म शरतल्पस्थं भरतानां पितामहम्}
{ऋषयस्त्वभ्यभाषन्त सहिताः सिद्धचारणैः}


\twolineshloka
{हते शान्तनवे भीष्मे भरतानां पितामहे}
{न किंचित्प्रत्यपद्यन्त पुत्रास्तव हि मारिष}


\twolineshloka
{विषण्णवदनाश्चासन्हतश्रीकाश्च भारत}
{अतिष्ठन्व्रीडिताश्चैव ह्रिया युक्ता ह्यधोमुखाः}


\twolineshloka
{पाण्डवाश्च जयं लब्ध्वा संग्रामशिरसि स्थिताः}
{सर्वे दध्युर्महाशङ्खान्हेमजालपरिष्कृतान्}


\twolineshloka
{हर्षात्तूर्यसहस्रेषु वाद्यमानेषु चानघ}
{अपश्याम महाराज भीमसेनं महाबलम्}


\twolineshloka
{विक्रीडमानं कौन्तेयं हर्षेण महता युतम्}
{निहत्य तरसा शत्रुं महाबलसमन्वितम्}


\twolineshloka
{संमोहश्चापि तुमुलः कुरूणामभवत्ततः}
{कर्णदुर्योधनौ चापि निःश्वसेतां मुहुर्मुहुः}


\twolineshloka
{तथा निपातिते भीष्मे कौरवाणां पितामहे}
{हाहाभूतमभूत्सर्वं निर्मर्यादमवर्तत}


\twolineshloka
{दृष्ट्वा च पतितं भीष्मं पुत्रो दुःशासनस्तव}
{उत्तमं जवमास्थाय द्रोणानीकभुपाद्रवत्}


\twolineshloka
{भ्रात्रा प्रस्थापितो वीरः स्वेनानीकेन दंशितः}
{प्रययौ पुरुषव्याघ्रः स्वसैन्यमभिहर्षयन्}


\twolineshloka
{तमायान्तमभिप्रेक्ष्य कुरवः पर्यवारयन्}
{दुःशासनं महाराज किमयं वक्ष्यतीति च}


\twolineshloka
{ततो द्रोणाय निहतं भीष्ममाचष्ट कौरवः}
{द्रोणस्तदाऽप्रियं श्रुत्वा मुमोह भरतर्षभ}


\twolineshloka
{स संज्ञामुपलभ्याशु भारद्वाजः प्रतापवान्}
{निवारयामास तदा स्वान्यनीकानि मारिष}


\twolineshloka
{विनिवृत्तान्कुरून्दृष्ट्वा पाण्डवापि स्वसैनिकान्}
{रथैः शीघ्राश्वसंयुक्तैः समन्तात्पर्यवारयन्}


\twolineshloka
{निवृत्तेषु च सैन्येषु पारम्पर्येण सर्वशः}
{निर्मुक्तकवचाः सर्वे भीष्ममीयुर्नराधिपाः}


\twolineshloka
{व्युपरम्य ततो युद्धाद्योधाः शतसहस्रशः}
{उपतस्थुर्महात्मानं प्रजापतिमिवामराः}


\twolineshloka
{ते तु भीष्मं समासाद्य शयानं भरतर्षभम्}
{अभिवाद्यावतिष्ठन्त पाण्डवाः कुरुभिः सह}


\twolineshloka
{अथ पाण्डून्कुरूंश्चैव प्रणिपत्याग्रतः स्थितान्}
{अभ्यभाषत धर्मात्मा भीष्मः शान्तनवस्तदा}


\twolineshloka
{स्वागतं वो महाभागाः स्वागतं वो महारथाः}
{तुष्यामि दर्शनाच्चाहं युष्माकममरोपमाः}


\threelineshloka
{अभिनन्द्य स तानेवं शिरसा लम्बताऽब्रवीत्}
{`परपार्श्वे तव सुतान्स्थितानुद्वीक्ष्य भारत}
{'शिरो मे लम्बतेऽत्यर्थमुपधानं प्रदीयताम्}


\twolineshloka
{ततो नृपाः समाजह्रुस्तनूनि च मृदूनि च}
{उपधानानि मुख्यानि नैच्छत्तानि पितामहः}


\twolineshloka
{अथाब्रवीन्नरव्याघ्रः प्रसहन्निव तान्नृपान्}
{नैतानि वीरशय्यासु युक्तरूपाणि पार्थिवाः}


\twolineshloka
{ततो वीक्ष्य नरश्रेष्ठमभ्यभाषत पाण्डवम्}
{धनञ्जयं दीर्घबाहुं सर्वलोकमहारथम्}


\threelineshloka
{धनञ्जय महाबाहो शिरो मे तात लम्बते}
{दीयतामुपधानं वै यद्युक्तमिह मन्यसे ॥सञ्जय उवाच}
{}


\twolineshloka
{समारोप्य महच्चापमभिवाद्य पितामहम्}
{नेत्राभ्यामश्रुपूर्णाभ्यामिदं वचनमब्रवीत्}


\twolineshloka
{आज्ञापय कुरुश्रेष्ठ सर्वशस्त्रभृतां वर}
{प्रेष्योऽहं तव दुर्धर्ष क्रियतां किं पितामह}


\twolineshloka
{तमब्रवीच्छान्तनवः शिरो मे तात लम्बते}
{6-120-41b`दीयतामुपधानंमे यद्युक्तमिह मन्यसे' उपधानं कुरुश्रेष्ठ उपधेहि ममार्जुन}


\twolineshloka
{वीरशय्यानुरूपं वै शीघ्रं वीर प्रयच्छ मे}
{त्वं हि पार्थ समर्थो वै श्रेष्ठः सर्वधनुष्मताम्}


\twolineshloka
{क्षत्रधर्मस्य वेत्ता च बुद्धिसत्वगुणान्वितः}
{फल्गुनोऽपि तथेत्युक्त्वा व्यवसायं परंजयः}


\twolineshloka
{गृह्यानुमन्त्र्य गाण्डीवं शरान्सन्नतपर्वणः}
{अनुमान्य महात्मानं भरतानां महारथम्}


\twolineshloka
{त्रिभिस्तीक्ष्णैर्महावेगैरुदगृह्णाच्छिरः शरैः}
{अभिप्राये तु विदिते धर्मात्मा स्वयसाचिना}


\twolineshloka
{अतुष्यद्भरतश्रेष्ठो भीष्मो धर्मार्थतत्त्ववित्}
{उपधानेन दत्तेन प्रत्यनन्दद्धनञ्जयम्}


\twolineshloka
{प्राह सर्वान्समुद्वीक्ष्य भरतान्भारतं प्रति}
{कुन्तीपुत्रं युधां श्रेष्ठं सुहृदां प्रीतिवर्धनम्}


\twolineshloka
{शयनस्यानुरूपं मे पाण्डवोपहितं त्वया}
{यद्यन्यथा प्रपद्येथाः शपेयं त्वामहं पुरा}


\twolineshloka
{एवमेव महाबाहो धर्मेषु परितिष्ठता}
{स्वप्तव्यं क्षत्रियेणाजौ शरतल्पगतेन वै}


\twolineshloka
{एवमुक्त्वा तु बीभत्सुं सर्वांस्तानब्रवीद्वचः}
{राज्ञश्च राजपुत्रांश्च पाण्डवैरभिसंवृतान्}


\twolineshloka
{पश्यध्वमुपधानं मे पाणडवेनाभिसन्धितम्}
{शयेयमस्यां शय्यायां यावदावर्तनं रवेः}


\twolineshloka
{ये तदा धारयिष्यन्ति ते च प्रेक्ष्यन्ति मां नृपाः}
{दिशं वैश्रवणाक्रान्तां यदा गन्ता दिवाकरः}


\twolineshloka
{नूनं सप्ताश्वयुक्तेन रथेनोत्तमतेजसा}
{रक्ष्येऽहं वै मम प्राणान्सुहृदः सुप्रियानिव}


\fourlineindentedshloka
{परिखाः खन्यतामत्र ममावसदने नृपाः}
{उपासिष्ये विवस्वन्तमेवं शरशताचितः}
{उपारमध्वं संग्रामाद्वैरमुत्सृज्य पार्थिवाः ॥स़ञ्जय उवाच}
{}


\twolineshloka
{उपातिष्ठन्नथो वैद्याः शल्योद्धरणकोविदाः}
{सर्वोपकरणैर्युक्ताः कुशलैः साधुशिक्षिताः}


\twolineshloka
{तान्दृष्ट्वा जाह्नवीपुत्रः प्रोवाच तनयं तव}
{दत्तदेया विसृज्यन्तां पूजयित्वा चिकित्सकाः}


\twolineshloka
{एवंगते मयेदानीं वैद्यैः कार्यमिहास्ति किम्}
{क्षत्रधर्मे प्रशस्तां हि प्राप्तोऽस्मि परमां गतिम्}


\twolineshloka
{नैष धर्मो महीपालाः शरतल्पगतस्य मे}
{एभिरेव शरैश्चाहं दग्धव्योऽग्नौ नराधिपाः}


\twolineshloka
{तच्छ्रुत्वा वचनं तस्य पुत्रो दुर्योधनस्तव}
{वैद्यान्विसर्जयामास पूजयित्वा यथार्हतः}


\twolineshloka
{ततस्ते विस्मयं जग्मुर्नानाजनपदेश्वराः}
{स्थितिं धर्मे परां दृष्ट्वा भीष्मस्यामिततेजसः}


\twolineshloka
{उपधानं ततो दत्त्वा पितुस्ते मनुजेश्वराः}
{सहिताः पाण्डवाः सर्वे कुरवश्च महारथाः}


% Check verse!
उपगम्य महात्मानं शयानं शयने शुभे
\twolineshloka
{तेऽभिवाद्य ततो भीष्मं कृत्वा च त्रिःप्रदक्षिणम्}
{विधाय रक्षां भीष्मस्य सर्व एव समन्ततः}


\twolineshloka
{वीराः स्वशिबिराण्येव ध्यायन्तः परमातुराः}
{निवेशायाभ्युपागच्छन्सायाह्ने रुधिरोक्षिताः}


\twolineshloka
{निविष्टान्पाण्डवांश्चैव प्रीयमाणान्महारथान्}
{भीष्मस्य पतनं दृष्ट्वा उपगम्य महाबलः}


\twolineshloka
{उवाच माधवः काले धर्मपुत्रं युधिष्ठिरम्}
{दिष्ट्या जयसि कौरव्य दिष्ट्या भीष्मो निपातितः}


\twolineshloka
{अवध्यो मानुषैरेव सत्यसन्धो महारथः}
{अथवा दैवतैः सार्धं सर्वशास्त्रस्य पारगः}


\threelineshloka
{त्वां तु चक्षुर्हणं प्राप्य दग्धो घोरेण चक्षुष}
{सञ्जय उवाच}
{एवमुक्तो धर्मराजः प्रत्युवाच जनार्दनम्}


\twolineshloka
{तव प्रसादाद्विजयः क्रोधात्तव पराजयः}
{त्वं हि नः शरणं कृष्ण भक्तानामभयंकरः}


\twolineshloka
{अनाश्चर्यो जयस्तेषां येषां त्वमसि केशव}
{रक्षिता समरे नित्यं नित्यं चापि हिते रतः}


\fourlineindentedshloka
{सर्वथा त्वां समासाद्य नाश्चर्यमिति मे मतिः}
{सञ्जय उवाच}
{एवमुक्तः प्रत्युवाच स्मयमानो जनार्दनः}
{तवैवैतद्युक्तरूपं वचनं पार्थिवोत्तम}


\chapter{अध्यायः १२१}
\twolineshloka
{सञ्जय उवाच}
{}


\twolineshloka
{व्युष्टायां तु महाराज शर्वर्यां सर्वपार्थिवाः}
{पाण्डवा धार्तराष्ट्राश्च उपातिष्ठन्पितामहम्}


\twolineshloka
{तं वीरशयने वीरं शयानं कुरुसत्तम}
{अभिवाद्योपतस्थुर्वै क्षत्रियाः क्षत्रियर्षभम्}


\twolineshloka
{कन्याश्चन्दनचूर्णैश्च लाजैर्माल्यैश्च सर्वशः}
{अवाकिरञ्छान्तनवं तत्र गत्वा सहस्रशः}


\twolineshloka
{स्त्रियो वृद्धास्तथा बालाः प्रेक्षकाश्च पृथग्जनाः}
{समभ्ययुः शान्तनवं भूतानीव तमोनुदम्}


\twolineshloka
{तूर्याणि शतसङ्ख्यानि तथैव नटनर्तकाः}
{शिल्पिनश्च तथाऽऽजग्मुः कुरुवृद्धं पितामहम्}


\twolineshloka
{उपागम्य च राजेन्द्र सन्नहान्विप्रमुच्य ते}
{आयुधानि च निक्षिप्य सहिताः कुरुपाण्डवाः}


\twolineshloka
{अन्वासन्त दुराधर्षं देवव्रतमरिंदमम्}
{अन्योन्यं प्रीतिमन्तस्ते यथापूर्वं यथावयः}


\twolineshloka
{सा पार्थिवशताकीर्णा समितिर्भीष्मशोभिता}
{शुशुभे भारती दीप्ता दिवीवादित्यमण्डलम्}


\twolineshloka
{विबभौ च नृपाणां सा गङ्गासुतमुपासताम्}
{देवानामिव देवेशं पितामहमुपासताम्}


\twolineshloka
{भीष्मस्तु वेदनां धैर्यान्निगृह्य भरतर्षभ}
{अभितप्तः शरैश्चैव नातिहृष्टमनाऽब्रवीत्}


\twolineshloka
{शराभितप्तकायो हि शस्त्रसंपातमूर्च्छितः}
{पानीयमित संप्रेक्ष्य राज्ञस्तान्प्रत्यभाषत}


\twolineshloka
{ततस्ते क्षत्रिया राजन्नुपाजह्रुः समन्ततः}
{भक्ष्यानुच्चावचान्राजन्वारिकुम्भांश्च शीतलान्}


\twolineshloka
{उपानीतं तु पानीयं दृष्ट्वा शान्तनवोऽब्रवीत्}
{न मेऽद्य सेवितुं योग्या भोगाः केवलमानुषाः}


\threelineshloka
{अपक्रान्तो मनुष्येभ्यः शरशय्यां गतो ह्यहम्}
{प्रतीक्षमाणस्तिष्ठामि निवृत्तिं शशिसूर्ययोः ॥` सञ्जय उवाच}
{}


\threelineshloka
{एवमुक्त्वा तदोवाच भीष्मः शरशतैश्चितः}
{पयः पास्यामि गोपाला गोमयं न तु गोमयम्}
{गोपयेनाग्निवर्णेन गोमयं न तु गोमयम् ॥'}


\twolineshloka
{एवमुक्त्वा शान्तनवो निन्दन्वाक्येन पार्थिवान्}
{अर्जुनं द्रष्टुमिच्छामीत्यभ्यभाषत भारत}


\twolineshloka
{अथोपेत्य महाबाहुरभिवाद्य पितामहम्}
{अतिष्ठत्प्राञ्जलिः प्रह्वः किं करोमीति चाब्रवीत्}


\twolineshloka
{तं दृष्ट्वा पाण्डवं राजन्नभिवाद्याग्रतः स्थितम्}
{अभ्यभाषत धर्मात्मा भीष्मः प्रीतो धनंजयम्}


\twolineshloka
{दह्यतीव शरीर मे संवृतस्य तवेषुभिः}
{मर्माणि परिदूयन्ते मुखं च परिशुष्यति}


\threelineshloka
{वेदनार्तशरीरस्य प्रयच्छापो ममार्जुन}
{त्वं हि शक्तो महेष्वास दातुमापो यथाविधि ॥सञ्जय उवाच}
{}


\twolineshloka
{अर्जुनस्तु तथेत्युक्त्वा रथमारुह्य वीर्यवान्}
{अधिज्यं बलवत्कृत्वा गाण्डीवं व्याक्षिपद्धनुः}


\twolineshloka
{तस्य ज्यातलनिर्घोषं विस्फूर्जितमिवाशनेः}
{वित्रेसुः सर्वभूतानि सर्वे श्रुत्वा च पार्थिवाः}


\twolineshloka
{ततः प्रदक्षिणं कृत्वा रथेन रथिनां वरः}
{शयानं भरतश्रेष्ठं सर्वशस्त्रभृतां वरम्}


\twolineshloka
{संधाय च शरं दीप्तमभिमन्त्र्य स पाण्डवः}
{पर्जन्यास्त्रेण संयोज्य सर्वलोकस्य पश्यतः}


\twolineshloka
{अविध्यत्पृथिवीं पार्थः पार्श्वे भीष्मस्य दक्षिणे}
{उत्पपात ततो धारा वारिणो विमला शुभा}


\twolineshloka
{शीतस्यामृतकल्पस्य दिव्यगन्धरसस्य च}
{अतर्पयत्ततः पार्थः शीतया जलधारया}


\twolineshloka
{भीष्मं कुरूणामृषभं दिव्यं दिव्यपराक्रमम्}
{कर्मणा तेन पार्थस्य शक्रस्येव विकुर्वतः}


\twolineshloka
{विस्मयं परमं जग्मुस्ततस्ते वसुधाधिपाः}
{तत्कर्म प्रेक्ष्य बीभत्सोरतिमानुषविक्रमम्}


\twolineshloka
{संप्रावेपन्त कुरवो गावः शीतार्दिता इव}
{विस्मयाच्चोत्तरीयाणि व्याविध्यन्सर्वतो नृपाः}


\twolineshloka
{शङ्खदुन्दुभिनिर्घोषस्तुमुलः सर्वतोऽभवत्}
{तृप्तः शान्तनवश्चापि राजन्बीभत्सुमब्रवीत्}


\twolineshloka
{सर्वपार्थिववीराणां सन्निधौ पूजयन्निव}
{नैतच्चित्रं महाबाहो त्वयि कौरवनन्दन}


\twolineshloka
{कथितो नारदेनासि पूर्वर्षिरमितद्युते}
{वासुदेवसहायस्त्वं महत्कर्म करिष्यसि}


\twolineshloka
{यन्नोत्सहति देवेन्द्रः सह देवैरपि ध्रुवम्}
{विदुस्त्वां निधनं पार्थ सर्वक्षत्रस्य तद्विदः}


% Check verse!
धनुर्धराणामेकस्त्वं पृथिव्यां प्रवरो नृषु
\twolineshloka
{मनुष्या जगति श्रेष्ठाः पक्षिणां पतगेश्वरः}
{सरितं सागरः श्रेष्ठो गौर्वरिष्ठा चतुष्पदाम्}


\twolineshloka
{आदित्यस्तेजसां श्रेष्ठो गिरीणां हिमवान्वरः}
{जातीनां ब्राह्मणः श्रेष्ठः श्रेष्ठस्त्वमसि धन्विनां}


\twolineshloka
{न वै श्रुतं धार्तराष्ट्रेण वाक्यंमयोच्यमानं विदुरेण चैव}
{द्रोणेन रामेण जनार्दनेनमुहुर्मुहुः सञ्जयेनापि चोक्तम्}


\threelineshloka
{परीतबुद्धिर्हि विसंज्ञकल्पोदुर्योधनो न च तच्छ्रद्दधाति}
{स शेष्यते वै निहतश्चिरायशास्त्रातिगो भीमबलाभिभूतः ॥सञ्जय उवाच}
{}


\twolineshloka
{एतच्छ्रुत्वा तद्वचः कौरवेन्द्रोदुर्योधनो दीनमना बभूव}
{तमब्रविच्छान्तनवोऽभिवीक्ष्यनिबोध राजन्भव वीतमन्युः}


\twolineshloka
{दृष्टं दुर्योधनैतत्ते यथा पार्थेन धीमता}
{जलस्य धारा जनिता शीतस्यामृतगन्धिनः}


\twolineshloka
{एतस्य कर्ता लोकेऽस्मिन्नान्यः कश्चन विद्यते}
{आग्नेयं वारुणं सौम्यं वायव्यमथ वैष्णवम्}


\twolineshloka
{ऐन्द्रं पाशुपतं ब्रह्मं पारमेष्ठ्यं प्रजापतेः}
{धातुस्त्वष्टुश्च सवितुर्वैवस्वतमथापि वा}


\twolineshloka
{सर्वस्मिन्मानुषे लोके वेत्त्येको हि धनञ्जयः}
{कृष्णो वा देवकीपुत्रो नान्यो वेदेह कश्चन}


\twolineshloka
{अशक्यः पाण्डवस्तात युद्धे जेतुं कथंचन}
{अमानुषाणि कर्माणि यस्यैतानि महात्मनः}


\twolineshloka
{तेन सत्ववता सङ्ख्ये शूरेणाहवशोभिना}
{जिष्णुना समरे राजन्संधिर्भवतु मा चिरम्}


\twolineshloka
{यावत्कृष्णो महाबाहुः स्वाधीनः कुरुसत्तम}
{तावत्पार्थेन शूरेण संधिस्ते तात युज्यताम्}


\twolineshloka
{यावन्न ते चमूः सर्वाः शरैः सन्नतपर्वभिः}
{नाशयत्यर्जुनस्तावत्संधिस्ते तात य्रुज्यताम्}


\twolineshloka
{यावत्तिष्ठन्ति समरे हतशेषाः सहोदराः}
{नृपाश्च बहवो राजंस्तावत्संधिः प्रयुज्यताम्}


\twolineshloka
{न निर्दहति ते यावत्क्रोधदीप्तेक्षणश्चमूम्}
{युधिष्ठिरो रणे तावत्संधिस्ते तात युज्यताम्}


\twolineshloka
{नकुलः सहदेवश्च भीमसेनश्च पाण्डवः}
{यावच्चमूं महाराज नाशयन्ति न सर्वशः}


\twolineshloka
{तावत्ते पाण्डवैर्वीरैः सौहार्दं मम रोचते}
{युद्धं मदन्तमेवास्तु तात संशाम्य पाण्डवैः}


\twolineshloka
{एतत्ते रोचतां वाक्यं यदुक्तोऽसि मयाऽनघ}
{एतत्क्षममहं मन्ये तव चैव कुलस्य च}


\twolineshloka
{त्यक्त्वा मन्युं व्युपशाम्यस्व पार्थैःपर्याप्तमेतद्यत्कृतं फल्गुनेन}
{भीष्मस्यान्तादस्तु वः सौहृदं चजीवन्तु शेषाः साधु राजन्प्रसीद}


\twolineshloka
{राज्यस्यार्दं दीयतां पाण्डवाना-मिन्द्रप्रस्थं धरमराजोऽभियातु}
{मा मित्रध्रुक्पार्थिवानां जघन्यःपापां कीर्तिं प्राप्स्यसे कौरवेन्द्र}


\twolineshloka
{ममावसानाच्छान्तिरस्तु प्रजानांसंगच्छन्तां पाण्डवाः प्रीतिमन्तः}
{पिता पुत्रं मातुलं भागिनेयोभ्राता चैव भ्रातरं प्रैतु राजन्}


\threelineshloka
{न चेदेवं प्राप्तकालं वचो मेमोहाविष्टः प्रतिपत्स्यस्यबुद्ध्या}
{तप्स्यस्यन्ते एतदन्ताः स्थ सर्वेसत्यामेतां भारतीमीरयामि ॥सञ्जय उवाच}
{}


\twolineshloka
{एतद्वाक्यं सौहृदादापगेयोमध्ये राज्ञां भारतं श्रावयित्वा}
{तूष्णीमासीच्छल्यसंतप्तमर्मायोज्यात्मानं वेदनां संनियम्य}


\twolineshloka
{धर्मार्थसहितं वाक्यं श्रुत्वा हितमनामयम्}
{नारोचयत पुत्रस्ते मुमूर्षुरिव भेषजम्}


\chapter{अध्यायः १२२}
\twolineshloka
{सञ्जय उवाच}
{}


\twolineshloka
{ततस्ते पार्थिवाः सर्वे जग्मुः स्वानालयान्पुनः}
{तूष्णींभूते महाराज भीष्मे शन्तनुनन्दने}


\twolineshloka
{श्रुत्वा तु निहतं भीष्मं राधेयः पुरुषर्षभः}
{ईषदागतसंत्रासस्त्वरयोपजगाम ह}


\twolineshloka
{स ददर्श महात्मानं शरतल्पगतं तदा}
{जन्मशय्यागतं वीरं कार्तिकेयमिव प्रभुम्}


\twolineshloka
{निमीलिताक्षं तं वीरं साश्रुकण्ठस्तदा वृषा}
{अभ्येत्य पादयोस्तस्य निपपात महाद्युतिः}


\twolineshloka
{राधेयोऽहं कुरुश्रेष्ठ नित्यमक्षिगतस्तव}
{द्वेभ्योऽत्यन्तमनागाः सन्निति चैनमुवाच ह}


\twolineshloka
{तच्छ्रुत्वा कुरुवृद्धो हि शरैः संवृतलोचनः}
{रहितं धिष्ण्यमालोक्य समुत्सार्य च रक्षिणः}


\twolineshloka
{पितेव पुत्रं गाङ्गेयः परिरभ्यैकपाणिना}
{शनैरुद्वीक्ष्य सस्नेहमिदं वचनमब्रवीत्}


\twolineshloka
{न विप्रियं ममैवेह यत्स्पर्धेथा मया सह}
{यदि मां नाधिगच्छेथा न ते श्रेयो ध्रुवं भवेत्}


\twolineshloka
{कौन्तेयस्त्वं न राधेयो न तवाधिरथः पिता}
{सूर्यजस्त्वं महाबाहो विदितो नारदान्मया}


\twolineshloka
{कृष्णद्वैपायनाच्चैव तेजसा च न संशयः}
{न च द्वेषोस्ति मे तात त्वयि सत्यं ब्रवीमि ते}


\twolineshloka
{तेजोवधनिमित्तं तु परुषं त्वाहमब्रुवम्}
{अकस्मात्पाण्डवान्हि त्वं द्विषसीति मतिर्मम}


\twolineshloka
{येनासि बहुशो राज्ञा चोदितः सूतनन्दन}
{जातोऽसि धर्मलोपेन ततस्ते बुद्धिरीदृशी}


\twolineshloka
{नीचाश्रयान्मत्सरेण द्वेषिणी गुमिनामपि}
{तेनासि बहुशो रूक्षं श्रावितः कुरुसंसदि}


\twolineshloka
{जानामि समरे वीर्यं शत्रुभिर्दुःसहं भुवि}
{ब्रह्मण्यतां च शौर्यं च दाने च परमां स्थितिम्}


\twolineshloka
{न त्वया सदृशः कश्चित्पुरुषेष्वमरोपम}
{कुलभेदभयाच्चाहं सदा परुषमुक्तवान्}


\twolineshloka
{इष्वस्त्रे भारसंधाने लाघवेऽस्त्रबले तथा}
{सदृशः फल्गुनेनासि कृष्णेन च महात्मना}


\twolineshloka
{कर्ण काशिपुरं गत्वा त्वयैकेन धनुष्मता}
{कन्यार्थे कुरुराजस्य राजानो मृदिता युधि}


\twolineshloka
{तथा च बलवान्राजा जरासन्धो दुरासदः}
{समरे समरश्लाघिन्न त्वया सदृशोऽभवत्}


\twolineshloka
{ब्रह्मण्यः सत्त्वयोधी च तेजसा च बलेन च}
{देवगर्भसमः सङ्ख्ये मनुष्यैरधिको युधि}


\twolineshloka
{व्यपनीतोऽद्य मन्युर्मे यस्त्वां प्रति पुरा कृतः}
{दैवं पुरुषकारेण न शक्यमतिवर्तितुम्}


\twolineshloka
{सोदर्याः पाण्डवा वीरा भ्रातरस्तेऽरिसूदन}
{संगच्छ तैर्महाबाहो मम चेदिच्छसि प्रियम्}


\threelineshloka
{मया भवतु निर्वृत्तं वैरमादित्यनन्दन}
{पृथिव्यां सर्वराजानो भवन्त्वद्य निरामयाः ॥कर्ण उवाच}
{}


\twolineshloka
{जानाम्येव महाबाहो सर्वमेतन्न संशयः}
{यथा वदसि मे भीष्म कौन्तेयोऽहं न सूतजः}


\twolineshloka
{अवकीर्णस्त्वहं कुन्त्या सूतेन च विवर्धितः}
{भुक्त्वा दुर्योधनैश्वर्यं न मिथ्या कर्तुमुत्सहे}


\twolineshloka
{वसुदेवसुतो यद्वत्पाण्डवाय दृढव्रतः}
{वमु चैव शरीरं च पुत्रदारं तथा यशः}


\twolineshloka
{सर्वं दुर्योधनस्यार्थे त्यक्तं मे भूरिदक्षिण}
{मा चैतद्व्याधिमरणं क्षत्रं स्यादिति कौरव}


\twolineshloka
{कोपिताः पाण्डवा नित्यं मयाऽश्रित्य सुयोधनम्}
{अवश्यभावी ह्यर्थोऽयं यो न शक्यो निवर्तितुम्}


\twolineshloka
{दैवं पुरुषकारेण को विवर्तितुमुत्सहेत्}
{पृथिवीक्षयसंशीनि निमित्तानि पितामह}


\twolineshloka
{भवद्भिरुपलब्धानि कथितानि च संसदि}
{पाण्डवा वासुदेवश्च विदिता मम सर्वशः}


\twolineshloka
{अजेयाः पुरुषैरन्यैरिति तंश्चोत्सहामहे}
{विजयिष्ये रणे पाण्डूनिति मे निश्चितः मनः}


\twolineshloka
{न च शक्यमवस्रष्टुं वैरमेतत्सुदारुणम्}
{धनंजयेन योत्स्येऽहं स्वधर्मप्रीतमानसः}


\twolineshloka
{अनुजानीष्व मां तात युद्धाय कृतनिश्चयम्}
{अनुज्ञातस्त्वया वीर युध्येयमिति मे मतिः}


\threelineshloka
{दुरुक्तं विप्रतीपं वा रभसाच्चापलात्तथा}
{यन्मयेह कृतं किंचित्तन्मे त्वं क्षन्तुमर्हसि ॥भीष्म उवाच}
{}


\twolineshloka
{न चेच्छक्यमवस्रष्टुं वैरमेतत्सुदारुणम्}
{अनुजानामि कर्ण त्वां युध्यस्व स्वर्गकाम्यया}


\twolineshloka
{निर्मन्युर्गतसंरम्भः कृतकर्मा रणे स्म ह}
{यथाशक्ति यथोत्साहं सतां वृत्तेषु वृत्तवान्}


\twolineshloka
{अहं त्वामनुजानामि यदिच्छसि तदाप्नुहि}
{क्षत्रधर्मजिताँल्लोकानवाप्स्यसि धनंजयात्}


\twolineshloka
{युध्यस्व निरहंकारो बलवीर्यव्यपाश्रयः}
{धर्म्याद्धि युद्धादधिकं क्षत्रियस्य न विद्यते}


\threelineshloka
{प्रशमे हि कृतो यत्नः सुमहान्सुचिरं मया}
{न चैव शकितः कर्तुं यतो धर्मस्ततो जयः ॥सञ्जय उवाच}
{}


\threelineshloka
{इत्युक्तवति गाङ्गेये अभिवाद्योपमन्त्र्य च}
{राधेयो रथमारुह्य प्रायात्तव सुतं प्रति ॥` वैशंपायन उवाच}
{}


\twolineshloka
{इत्येतद्बहुवृत्तान्तं भीष्मपर्वाखिलं मया}
{शृण्वते ते महाराज प्रोक्तं पापहरं शुभम्}


\twolineshloka
{यः श्रावयेत्सदा राजन्ब्राह्मणान्वेदपारगान्}
{श्रद्धावन्तश्च ये चापि श्रोष्यन्ति मनुजा भुवि}


\twolineshloka
{विधूय सर्वपापानि विहायान्ते कलेवरम्}
{प्रयान्ति तत्पदं विष्णोर्यत्प्राप्य न निवर्तते}


\twolineshloka
{तस्मात्सर्वप्रयत्नेन भारतं भरतर्षभ}
{शृणुयात्सिद्धिमन्विच्छन्निह वामुत्र मानवः}


\twolineshloka
{भोजनं भोजयेद्विप्रान्गन्धमाल्यैरलङ्कृतान्}
{भीष्मपर्वणि राजेन्द्र दद्यात्पानीयमुत्तमम् ॥'}


