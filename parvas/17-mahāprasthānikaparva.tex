\part{महाप्रस्थानिकपर्व}
\chapter{अध्यायः १}
\threelineshloka
{श्रीवेदव्यासाय नमः}
{नारायणं नमस्कृत्यि नरं चैव नरोत्तमम्}
{देवीं सरस्वतीं व्यासं ततो जयमुदीरयेत्}


\fourlineindentedshloka
{जनमेजय उवाच}
{एवं वृष्ण्यन्धककुले श्रुत्वा मौसलमाहवम्}
{पाण्डवाः किमकुर्वन्त तथा कृष्णे दिवं गते ॥वैशम्पायन उवाच}
{}


\twolineshloka
{श्रुत्वैवं कौरवो राजा वृष्णीनां कदनं महत्}
{प्रस्थाने मतिमाधाय वाक्यमर्जुनमब्रवीत्}


\twolineshloka
{कालः पचति भतानि सर्वाण्येव महामते}
{कालपाशमहं मन्ये त्वमपि द्रष्टुमर्हसि}


\twolineshloka
{इत्युक्तः स तु कौन्तेयः कालः काल इति ब्रुवन्}
{अन्वपद्यत तद्वाक्यं भ्रातुर्ज्येष्ठस्य धीमतः}


\twolineshloka
{अर्जुनस्य मतं ज्ञात्वा भीमसेनो यमौ तथा}
{अन्वपद्यन्त तद्वाक्यं यदुक्तं सव्यसाचिना}


\twolineshloka
{ततो युयुत्सुमानाय्य प्रव्रजन्धर्मकाम्यया}
{राज्यं परिददौ सर्वं वैश्यापुत्रे युधिष्ठिरः}


\twolineshloka
{अभिंषिच्य स्वराज्ये च राजानं च परिक्षितम्}
{दुःखार्तश्चाब्रवीद्राजा सुभद्रां पाण्डवाग्रजः}


\twolineshloka
{एष पुत्रस्य पुत्रस्ते कुरुराजो भविष्यति}
{यदूनां परिशेषश्च वज्रो राजा कृतश्च ह}


\twolineshloka
{परिक्षिद्धास्तिनपुरे शक्रप्रस्थे च यादवः}
{वज्रो राजा त्वया रक्ष्यो मा चाधर्मे मनः कृथाः}


\twolineshloka
{इत्युक्त्वा धर्मराजाः स वासुदेवस्य धीमतः}
{मातुलस्य च वृद्धस्य रामदीनां तथैव च}


\twolineshloka
{भ्रातृभिः सह धर्मात्मा कृत्वोदकमतन्द्रितः}
{श्राद्धान्यद्दिश्य सर्वेषां चकार विदिवत्तदा}


\threelineshloka
{द्वैपायनं नारदं च मार्कण्डेयं तपोधनम्}
{भारद्वाजं याज्ञवल्क्यं हरिमुद्दिश्य यत्नवान्}
{अभोजयत्स्वादु भोज्यं कीर्तयित्वा च शार्ङ्गिणं}


\twolineshloka
{ददौ रत्नानि वासांसि ग्रामानश्वान्रथांस्तथा}
{स्त्रियश्च द्विजमुख्येभ्यस्तदा शतसहस्रशः}


\threelineshloka
{कृपमभ्यर्च्य च गुरुमथ पौरपुरस्कृतम्}
{`आहूय भरतश्रेष्ठ संनिवेश्यासने तदा}
{'शिष्यं परिक्षितं तस्मै तदौ भरतसत्तमः}


\twolineshloka
{ततस्तु प्रकृतीः सर्वाः समानाय्य युधिष्ठिरः}
{सर्वमाचष्ट राजर्षिकीर्षितमथात्मनः}


\twolineshloka
{ते श्रुत्वैव वचस्तस्य पौरजानपदा जनाः}
{भृशमुद्विग्रमनसो नाभ्यनन्दन्त तद्वचः}


\twolineshloka
{नैवं कर्तव्यमिति ते तदोचुस्तं जनाधिपम्}
{न च राजा तथाऽकार्षीत्कालपर्यायधर्मिवित्}


\twolineshloka
{ततोऽनुमान्य धर्मात्मा पौरजानपदं जनम्}
{गमनाय मतिं चक्रे कृष्णस्य गमनादपि}


\twolineshloka
{`वर्तमाने विवादे तु वास्तुविक्रयिणं प्रति}
{धनेच्छा युगपत्प्राप्ता क्षेत्रतस्वामिभूभृताम्}


\twolineshloka
{प्राप्तं कलियुगं ज्ञात्वा सहदेवो हसन्निव}
{राज्ञस्तु कथयामास धर्मो नष्टस्तु सङ्करः}


\threelineshloka
{श्रुत्वा तु दुर्मना राजा पर्याप्तं जीवनं मम}
{'इति स्म राजा कौरव्यो धर्मपुत्रो युधिष्ठिरः}
{उत्सृज्याभरणान्यङ्गाज्जगृहे वल्कलान्युत}


\twolineshloka
{भीमार्जुनयमाश्चैव द्रौपदी च यशस्विनी}
{तथैव जगृहुः सर्वे वल्कलानि नराधिप}


\twolineshloka
{विधिवत्कारयित्वेष्टिं नैष्ठिकीं भरतर्षभ}
{समुत्सृज्याप्सु सर्वेऽग्नीन्प्रतस्थुर्नरपुङ्गवाः}


\twolineshloka
{ततः प्ररुरुदुः सर्वाः स्त्रिंयो दृष्ट्वा नरोत्तमान्}
{प्रस्थितान्द्रौपदीषष्ठान्पुरा द्यूतजितान्यथा}


\threelineshloka
{हर्षोऽभवच्च सर्वेषां भ्रातॄणां गमनं प्रति}
{युधिष्ठिरमतं ज्ञात्वा वृष्णिक्षयमवेक्ष्य च}
{}


\threelineshloka
{भ्रातरः पञ्च कृष्णा च षष्ठी श्वा चैव सप्तमः}
{आत्मना सप्तमो राजा निर्ययौ गजसाह्वयात्}
{पौरैरनुगतो दूरं सर्वैरन्तःपुरैस्तथा}


\twolineshloka
{न चैनमशकत्कश्चिन्निवर्तस्वेति भाषितुम्}
{न्यवर्तन्त ततः सर्वे नरा नगरवासिनः}


\twolineshloka
{कृपप्रभृतयश्चैव युयुत्सुं पर्यवारयन्}
{विवेश गङ्गां कौरव्य उलूपी भुजगात्मजा}


\twolineshloka
{चित्राङ्गदा ययौ चापि मणलूरपुरं प्रति}
{शिष्टाः परिक्षितं त्वन्या मातरः पर्यवारयन्}


\twolineshloka
{पाण्डवाश्च महात्मानो द्रौपदी च यशस्विनी}
{कृतोपवासाः कौरव्य प्रययुः प्राङ्मुखास्ततः}


\twolineshloka
{योगयुक्ता महात्मानस्त्यागधर्ममुपेयुषः}
{अभिजग्मुर्बहून्देसान्सरितः सागरांस्तथा}


\twolineshloka
{युधिष्ठिरो ययावग्रे भीमस्तु तदनन्तरम्}
{अर्जुनस्तस्य चान्वेव यमौ चापि यथाक्रमम}


\threelineshloka
{पृष्ठतस्तु वरारोहा श्यामा पद्मदलेक्षणा}
{द्रौपदी योषितांश्रेष्ठा ययौ भरतसत्तम}
{}


\twolineshloka
{श्वा चैवानुयायावेकः प्रस्थितान्पाण्डवान्वनम्}
{क्रमेणि ते ययुर्वीरा लौहित्यं सलिलार्णवम्}


\twolineshloka
{गाण्डीवं तु धनुर्दिव्यं न मुमोच धनंजयः}
{रत्नलोभान्महाराज ते चाक्षय्ये महेषुधी}


\twolineshloka
{अग्निं ते ददृशुस्तत्र स्थितं शैलमिवाग्रतः}
{मार्गमावृत्य तिष्ठन्तं साक्षात्पुरुषविग्रहम्}


\twolineshloka
{ततो देवः स सप्तार्चिः पाण्डवानिदमब्रवीत्}
{भोभो पाण्डुसुता वीराः पावकं मां निबोधत}


\twolineshloka
{युधिष्ठिर महाबाहो भीमसेन परंतप}
{अर्जुनाश्विसुतौ वीरौ निबोधत वचो मम}


\twolineshloka
{अहमग्निः कुरुश्रेष्ठा मया दग्धं च खाण्डवम्}
{अर्जुनस्य प्रभावेन तथा नारायणस्य च}


\twolineshloka
{अयं वः फल्गुनो भ्राता गाण्डीवं परमायुधम्}
{परित्यज्य वने यातु नानेनार्थोस्ति कश्चन}


\twolineshloka
{चक्ररत्नं तु यत्कृष्णे स्थितमासीन्महात्मनि}
{गतं तच्च पुनर्हस्ते कालेनैष्यति तस्य ह}


\twolineshloka
{वरुणादाहृतं पूर्वं मयैतत्पार्थकारणात्}
{गाण्डीवं धनुषां श्रेष्ठं वरुणायैव दीयताम्}


\twolineshloka
{ततस्ते भ्रातरः सर्वे धनंजयमचोदयन्}
{स जले प्राक्षिपच्चैतत्तथाऽक्षय्ये महेषुधी}


\twolineshloka
{ततोऽग्निर्भरतश्रेष्ठ तत्रैवान्तरधीयत}
{ययुस्च पाण्डवा वीरास्ततस्ते दक्षिणामुखाः}


\twolineshloka
{ततस्ते तूत्तरेणैव तीरेण लवणांभसः}
{जग्मुर्भरतशार्दूल दिशं दक्षिणपश्चिमाम्}


\twolineshloka
{ततः पुनः समावृत्ताः पश्चिमां दिशमेव ते}
{ददृशुर्द्वारकां चापि सागरेण परिप्लुताम्}


\twolineshloka
{उदीचीं पुनरावृत्त्य ययुर्भरतसत्तमाः}
{प्रादक्षिण्यं चिकीर्षन्तः पृथिव्या योगधर्मिणः}


\chapter{अध्यायः २}
\twolineshloka
{ततस्ते नियतात्मान उदीचीं दिशमास्थिताः}
{ददृशुर्योगयुक्ताश्च हिमवन्तं महागिरिम्}


\twolineshloka
{तं चाप्यतिक्रमन्तस्ते ददृशुर्वालुकार्णवम्}
{अवैक्षन्त महाशैलं मेरुं शिखरिणां वरम्}


\twolineshloka
{तेषां तु गच्छतां शीघ्रं सर्वेषां योगधार्मिणाम्}
{याज्ञसेनी भ्रष्टयोगा निपपात महीतले}


\twolineshloka
{तां तु प्रपतितां दृष्ट्वा भीमसेनो महाबलः}
{उवाच धर्मराजानं याज्ञसेनीमवेक्ष्य ह}


\threelineshloka
{नाधर्मश्चरितः कश्चिद्राजपुत्र्या परंतप}
{कारणं किंनु तद्ब्रूहि यत्कृष्णा पतिता भुवि ॥युधिष्ठिर उवाच}
{}


\threelineshloka
{पक्षपातो महानस्या विशेषेण धनंजये}
{तस्यैतत्फलमद्यैषा भुङ्क्ते पुरुषसत्तम ॥वैशम्पायन उवाच}
{}


\twolineshloka
{एवमुक्त्वाऽनवेक्ष्यैनां ययौ भरतसत्तमः}
{समाधाय मनो धीमान्धर्मात्मा पुरुषर्षभः}


\twolineshloka
{सहदेवस्ततो विद्वान्निपपात महीतले}
{तं चापि पतितं दृष्ट्वा भीमो राजानमब्रवीत्}


\threelineshloka
{योऽयमस्मासु सर्वेषु शुश्रूषुरनहंकृतः}
{सोयं माद्रवतीपुत्रः कस्मान्निपतितो भुवि ॥युधिष्ठिर उवाच}
{}


\threelineshloka
{आत्मनः सदृशं प्राज्ञं नैषोऽमन्यत कञ्चन}
{तेन दोषेण पतितो विद्वानेष नृपात्मजः ॥वैशम्पायन उवाच}
{}


\twolineshloka
{इत्युक्त्वा तं समुत्सृज्य सहदेवं ययौ तदा}
{भ्रातृभिः सह कौन्तेयः शुना चैव युधिष्ठिरः}


\twolineshloka
{कृष्णं निपतितां दृष्ट्वा सहदेवं च पाण्डवम्}
{आर्तो बन्धुप्रियः शूरो नकुलो निपपात ह}


\twolineshloka
{तस्मिन्निपतिते वीरे नकुले चारुदर्शने}
{पुनरेव तदा भीमो राजानमिदमब्रवीत्}


\twolineshloka
{योऽयमक्षतधर्मात्मा भ्राता वचनकारकः}
{रूपेणाप्रतिमो लोके नकुलः पतितो भुवि}


\twolineshloka
{इत्युक्तो भीमसेनेन प्रत्युवाच युधिष्ठिरः}
{नकुलं प्रति धर्मात्मा सर्वबुद्धिमतांवरः}


\twolineshloka
{रूपेणि मत्समो नास्ति कश्चिदित्यस्य दर्शनम्}
{अधिकश्चाहमेवैक इत्यस्य मनसि स्थितम्}


\twolineshloka
{नकुलः पतितस्तस्मादागच्छ त्वं वृकोदर}
{यस्य यद्विहितं वीर सोऽवश्यं तदुपाश्नुते}


\twolineshloka
{तांस्तु प्रपतितान्दृष्ट्वा पाण्डवः श्वेतवाहनः}
{पपात शोकसंतप्तस्ततोनु परवीरहा}


\twolineshloka
{तस्मिंस्तु पुरुषव्याघ्रे पतिते शक्रतेजसि}
{म्रियमाणे दुराधर्षे भीमो राजानमब्रवीत्}


\threelineshloka
{अनृतं न स्मराम्यस्य स्वैरेष्वपि महात्मनः}
{अथ कस्य विकारोऽयं येनायं पतितो भुवि ॥युधिष्ठिर उवाच}
{}


\twolineshloka
{एकोहं निर्दहेयं वै शत्रूनित्यर्जुनोऽब्रवीत्}
{न च तत्कृतवानेष शूरमानी ततोपतत्}


\threelineshloka
{अवमेने धनुर्ग्राहानेष सर्वांश्च फल्गुनः}
{तथा चैतन्न तु तथा कर्तव्यं भूतिमिच्छता ॥वैशम्पायन उवाच}
{}


\twolineshloka
{इत्युक्त्वा प्रस्थितो राजा भीमोथ निपपात ह}
{पतितश्चाब्रवीद्भीमो धर्मराजं युधिष्ठिरम्}


\threelineshloka
{भोभो राजन्नवेक्षस्व पतितोहं प्रियस्तव}
{किंनिमित्तं च पतनं ब्रूहि मे यदि वेत्थ ह ॥युधिष्ठिर उवाच}
{}


\twolineshloka
{अतिभुक्तं च भवता प्राणेन च विकत्थसे}
{अनवेक्ष्य परं पार्थ तेनासि पतितः क्षितौ}


\twolineshloka
{इत्युक्त्वा तं महाबाहुर्जगामानवलोकयन्}
{श्वाप्येकोनुययौ यस्ते बहुशः कीर्तितो मया}


\chapter{अध्यायः ३}
\twolineshloka
{ततः सन्नादयञ्शक्रो दिवं भूमिं च सर्वशः}
{रथेनोपययौ पार्थमारोहेत्यब्रवीच्च तम्}


\twolineshloka
{स्वभ्रातॄन्पतितान्दृष्ट्वा धर्मराजो युधिष्ठिरः}
{अब्रवीच्छोकसंतप्तः सहस्राक्षमिदं वचः}


\twolineshloka
{भ्रातरः पतिता मेऽत्र गच्छेयुस्ते मया सह}
{न विना भ्रातृभिः स्वर्गमिच्छे गन्तुं सुरेश्वर}


\threelineshloka
{सुकुमारी सुखार्हा च राजपुत्री पुरंदर}
{साऽस्माभिः सह गच्छेत तद्भवाननुमन्यताम् ॥शक्र उवाच}
{}


\twolineshloka
{भ्रातॄन्द्रक्ष्यसि स्वर्गे त्वमग्रतस्त्रिदिवं गतान्}
{कृष्णया सहितान्सर्वान्मा शुचो भरतर्षभ}


\threelineshloka
{निक्षिप्य मानुषं देहं गतास्ते भरतर्षभ}
{अनेन त्वं शरीरेण स्वर्गं गन्ता न संशयः ॥युधिष्ठिर उवाच}
{}


\threelineshloka
{अयं श्वा भूतभव्येश भक्तो मां नित्यमेव ह}
{स गच्छेत मया सार्धमानृशंस्या हि मे मतिः ॥शक्र उवाच}
{}


\threelineshloka
{अमर्त्यत्वं मत्समत्वं च राज-ञ्श्रियं कृत्स्नां महतीं चैव सिद्धिम्}
{संप्राप्तोद्य स्वर्गसुखानि च त्वंत्यज श्वानं नात्र नृशंसमस्ति ॥युधिष्ठिर उवाच}
{}


\threelineshloka
{अनार्यमार्येणि सहस्रनेत्रशक्यं कर्तुं दुष्करमेतदार्य}
{मा मे श्रिया सङ्गमनं तयाऽस्तुयस्याः कृते भक्तजनं त्यजेयम् ॥इन्द्र उवाच}
{}


\fourlineindentedshloka
{स्वर्गे लोके श्ववतां नास्ति धिष्ण्य-मिष्टापूर्तं क्रोधवशा हरन्ति}
{ततो विचार्यि क्रियतां धर्मराजत्यज श्वानं नात्र नृशंसमस्ति}
{युधिष्ठिर उवाच}
{}


\twolineshloka
{भक्तत्यागं प्राहुरत्यन्तपापंतुल्यं लोके ब्रह्मवध्याकृतेन}
{तस्मान्नाहं जातु कथञ्चनाद्यत्यक्ष्याम्येनं स्वसुखार्थी महेन्द्रः}


\threelineshloka
{भीतं भक्तं नान्यदस्तीति चार्तंप्राप्तं क्षीणं रक्षणे प्राणलिप्सुम्}
{प्राणत्यागादप्यहं नैव मोक्तुंयतेयं वै नित्यमेतद्व्रतं मे ॥इन्द्र उवाच}
{}


\twolineshloka
{शुना दृष्टं क्रोधवशा हरन्तियद्दत्तमिष्टं विवृतमथो हुतं च}
{तस्माच्छुनस्त्यागमिमं कुरुष्वशुनस्त्यागात्प्राप्स्यसे देवलोकम्}


\threelineshloka
{त्यक्त्वा भ्रातॄन्दयितां चापि कृष्णांप्राप्तो लोकः कर्मणा स्वेन वीर}
{श्वानं चैनं न त्यजसे कथं नुत्यागं कृत्स्नं चास्थितो मुह्यसेऽद्य ॥युधिष्ठिर उवाच}
{}


\twolineshloka
{न विद्यते संधिरथापि विग्रहोमृतैर्मर्त्यैरिति लोकेषु निष्ठा}
{न ते मया जीवयितुं हि शक्या-स्ततस्त्यागस्तेषु कृतो न जीवताम्}


\threelineshloka
{प्रतिप्रदानं शरणागतस्यस्त्रिया वधो ब्राह्मणस्वापहारः}
{मित्रद्रोहस्तानि चत्वारि शक्रभक्तत्यागश्चैव समो मतो मे ॥वैशम्पायन उवाच}
{}


\twolineshloka
{तद्धर्मराजस्य वचो निशम्यधर्मस्वरूपी भगवानुवाच}
{युधिष्ठिरं प्रीतियुक्तो नरेन्द्रंश्लक्ष्णैर्वाक्यैः संस्तवसंप्रयुक्तैः}


\twolineshloka
{अभिजातोसि राजेन्द्र पितुर्वृत्तेन मेधया}
{अनुक्रोशेनि चानेन सर्वभूतेषु भारत}


\twolineshloka
{पुरा द्वैतवने चासि मया पुत्र परीक्षितः}
{पानीयार्थे पराक्रान्ता यत्र ते भ्रातरो हताः}


\twolineshloka
{भीमार्जुनौ परित्यज्य यत्र त्वं भ्रातरावुभौ}
{मात्रोः साम्यमभीप्सन्वै नकुलं जीवमिच्छसि}


\twolineshloka
{अयं श्वा भक्त इत्येवं त्यक्तो देवरथस्त्वया}
{तस्मात्स्वर्गे न ते तुल्यः कश्चिदस्ति नराधिपः}


\threelineshloka
{अतस्तवाक्षया लोकाः स्वशरीरेण भारत}
{प्राप्तोसि भरतश्रेष्ठ दिव्यां गतिमनुत्तमाम् ॥वैशम्पायन उवाच}
{}


\twolineshloka
{ततो धर्मश्च शक्रश्च मरुतश्चाश्विनावपि}
{देवा देवर्षयश्चैव रथमारोप्य पाण्डवम्}


\twolineshloka
{प्रययुः स्वैर्विमानैस्ते सिद्धाः कामविहारिणः}
{सर्वे विरजसः पुण्याः पुण्यवाग्बुद्धिकर्मिणः}


\twolineshloka
{स तं रथं समास्थाय राजा कुरुकुलोद्वहः}
{ऊर्ध्वमाचक्रमे शीघ्रं तेजसाऽऽवृत्यरोदसी}


\twolineshloka
{ततो देवनिकाय्यस्थो नारदः सर्वलोकवित्}
{उवाचोच्चैस्तदा वाक्यं बृहद्वादी बृहत्तपाः}


\twolineshloka
{येऽपि राजर्षयः पूर्वे ते चापि समुपस्थिताः}
{कीति प्रच्छाद्य तेषां वै कुरुराजोऽधितिष्ठति}


\twolineshloka
{लोकानावृत्य यशसा तेजसा वृत्तसंपदा}
{स्वशरीरेण सम्प्राप्तं नान्यं शुश्रम पाण्डवात्}


\twolineshloka
{तेजांसि यानि दृष्टानि भूमिष्ठेन त्वया विभो}
{वेश्मानि भुवि देवानां पश्यामूनि सहस्रशः}


\twolineshloka
{नारदस्य वचः श्रुत्वा राजा वचनमब्रवीत्}
{भ्रातॄनपश्यन्धर्मात्मा स्वपक्षांश्चैव पार्थिवान्}


\twolineshloka
{शुभं वा यदि वा पापं भ्रातॄणां स्थानमद्य मे}
{तदेव प्राप्तुमिच्छामि लोकान्यान्न कामये}


\twolineshloka
{राज्ञस्तु वचनं श्रुत्वा देवराजः पुरंदरः}
{आनृशंस्यसमायुक्तं प्रत्युवाच युधिष्ठिरम्}


\twolineshloka
{स्थानेऽस्मिन्वस राजेन्द्र कर्मभिर्निर्जिते शुभैः}
{किं त्वं मानुष्यकं स्नेहमद्यापि परिकर्षसि}


\twolineshloka
{सिद्धिं प्राप्तोसि परमां यता नान्यः पुमान्क्वचित्}
{नैव ते भ्रातरः स्थानं सम्प्राप्ताः कुरुनन्दन}


\twolineshloka
{अद्यापि मानुषो भावः स्पृशते त्वां नराधिप}
{स्वर्गोऽयं पश्य देवर्षीन्सिद्धांश्च त्रिदिवालयान्}


\twolineshloka
{युधिष्ठिरस्तु देवेन्द्रमेवंवादिनमीश्वरम्}
{पुनरेवाब्रवीद्धीमानिदं वचनमर्थवत्}


\twolineshloka
{तैर्विना नोत्सहे वस्तुमिह दैत्यनिबर्हण}
{गन्तुमिच्छामि तत्राहं यत्र मे भ्रातरो गताः}


\twolineshloka
{यत्र सा बृहती श्यामा बुद्धिसत्त्वगुणान्विता}
{द्रौपदी योषितांश्रेष्ठा यत्र चैव सुता मम}


