\part{आश्रमवासिकपर्व}
\chapter{अध्यायः १}
\threelineshloka
{श्रीवेदव्यासाय नमः}
{नारायणं नमस्कृत्य नरं चैव नरोत्तमम्}
{देवीं सरस्वतीं व्यासं ततो जयमुदीरयेत्}


\threelineshloka
{जनमेजय उवाच}
{प्राप्य पैतामहं राज्यं मम पूर्वपितामहाः}
{कथमासन्महाराजे धृतराष्ट्रे महात्मनि}


\twolineshloka
{स तु राजा हतामात्यो हतपुत्रो निराश्रयः}
{कथमासीद्धतैश्वर्यो गान्धारी च तपस्विनी}


\threelineshloka
{कियन्तं चैव कालं ते मम पूर्वपितामहाः}
{स्थिता राज्ये महात्मानस्तन्मे व्याख्यातुमर्हसि ॥वैशम्पायन उवाच}
{}


\twolineshloka
{प्राप्य राज्यं महात्मानः पाण्डवा हतशत्रवः}
{धृतराष्ट्रं पुरस्कृत्य पृथिवीं पर्यपालयन्}


\twolineshloka
{धृतराष्ट्रमुपातिष्ठद्विदुरः संजयस्तथा}
{वैश्यापुत्रश्च मेधावी युयुत्सुः कुरुसत्तम}


\twolineshloka
{पाण्डवाः सर्वकार्येषु पर्यपृच्चन्त तं नृपम्}
{चक्रुस्तेनाभ्यनुज्ञाता वर्षाणि दश पञ्च च}


\twolineshloka
{सदाऽभिगम्य ते वीराः पर्युपासन्त तं नृपम्}
{पादाभिवादनं कृत्वा धर्मिराजमते स्थिताः}


\twolineshloka
{ते मूर्ध्नि समुपाघ्राताः सर्वकार्याणि चक्रिरे}
{कुन्तिभोजसुता चैव गान्धारीमन्ववर्तत}


\twolineshloka
{द्रौपदी च सुभद्रा च याश्चान्याः पाण्डवस्त्रियः}
{समां वृत्तिमवर्तन्त तयोः श्वश्र्वोर्यथाविधि}


\twolineshloka
{शयनानि महार्हाणि वासांस्याभरणानि च}
{राजार्हाणि च सर्वाणि भक्ष्यभोज्यान्यनेकशः}


\twolineshloka
{युधिष्ठिरो महाराज धृतराष्ट्रेऽभ्युपाहरत्}
{तथैव कुन्ती गान्धार्यां गुरुवृत्तिमवर्तत}


\twolineshloka
{विदुरः संजयश्चैव युयुत्सुश्चैव कौरव}
{उपासते स्म तं वृद्धं हतपुत्रं जनाधिपम्}


\twolineshloka
{श्यालो द्रोणस्य यश्चासीद्दयितो ब्राह्ममो महान्}
{स च तस्मिन्महेष्वासः कृपः समभवत्तदा}


\twolineshloka
{व्यासश्च भगवान्नित्यमासांचक्रे नृपेण ह}
{कथाः कुर्वन्पुराणर्षिर्देवर्षिपितृरक्षसाम्}


\twolineshloka
{धर्मयुक्तानि कार्याणि व्यवहारान्वितानि च}
{धृतराष्ट्राभ्यनुज्ञातो विदुरस्तान्यकारयत्}


\twolineshloka
{सामन्तेभ्यः प्रियाण्यस्य कार्याणि सुबहून्यपि}
{प्राप्यन्तेऽर्थेः सुलघुभिः सुनयाद्विदुरस्य वै}


\twolineshloka
{अकरोद्बन्धमोक्षं च वध्यानां मोक्षणं तथा}
{न च धर्मसुतो राजा कदाचित्किञ्चिदब्रवीत्}


\twolineshloka
{विहारयात्रासु पुनः कुरुराजो युधिष्ठिरः}
{सर्वान्कामानुपस्थाप्य धृतराष्ट्रे न्यवेदयत्}


\twolineshloka
{आरालिकाः सूपकारा रागषाडविकास्तथा}
{उपातिष्ठन्त राजानं धृतराष्ट्रं यथापुरम्}


\twolineshloka
{वासांसि च महार्हाणि माल्यानि विविधानि च}
{उपाजह्रुर्यथाकालं धृतराष्ट्रस्य पाण्डवाः}


\twolineshloka
{मैरेयमधुमांसानि पानकानि लघूनि च}
{चित्रान्भक्ष्यविकारांश्च चक्रुस्तस्य यथा पुरा}


\twolineshloka
{ये चापि पृथिवीपालाः समाजग्मुस्ततस्ततः}
{उपातिष्ठन्त ते सर्वे कौरवेन्द्रं यथापुरा}


\twolineshloka
{कुन्ती च द्रौपदी चैव सात्वती च यशस्विनी}
{उलूपी नागकन्या च देवी चित्राङ्गदा तथा}


\threelineshloka
{धृष्टकेतोश्च भगिनी जरासन्धसुता तथा}
{एताश्चान्याश्च बह्व्यो वै योषितः पुरुषर्षभ}
{किंकराः पर्युपातिष्ठन्सर्वाः सुबलजां तथा}


\twolineshloka
{यथा पुत्रवियुक्तोऽयं न किञ्चिद्दुःखमाप्नुयात्}
{इति तानन्वशाद्भातॄन्नित्यमेव युधिष्ठिरः}


\twolineshloka
{एवं ते धर्मराजस्य श्रुत्वा वचनमर्थवत्}
{सविशेषमवर्तन्ति भीममेकं तदा विना}


\twolineshloka
{न हि तत्तस्य वीरस्य हृदयादपसर्पति}
{धृतराष्ट्रस्य दुर्बुद्ध्या यद्वृत्तं द्यूतमण्डले}


\chapter{अध्यायः २}
\twolineshloka
{एवं सम्पूजितो राजा पाण्डवैरंबिकासुतः}
{विजहार यथापूर्वमृषिभिः पर्युपस्थितः}


\twolineshloka
{ब्रह्मदेयाग्रहारांश्च प्रददौ स कुरूद्वहः}
{तच्च कुन्तीसुतो राजा सर्वमेवान्वमोदत}


\threelineshloka
{आनृशंस्यपरो राजा प्रीयमाणो युधिष्ठिरः}
{उवाच स तदा भ्रातॄनमात्यांश्च महीपतिः}
{मया चैव भवद्भिश्च मान्य एष नराधिपः}


\twolineshloka
{निदेशे धृतराष्ट्रस्य यस्तिष्ठति स मे सुहृत्}
{विपरीतश्च मे शत्रुर्नियम्यश्च भवेन्नरः}


\twolineshloka
{परिवृत्तेषु चाहःसु पुत्राणां श्राद्धकर्मणि}
{ददौ च राजा वित्तानि यावदस्य चिकीर्षितम्}


\twolineshloka
{ततः स राजा कौरव्यो धृतराष्ट्रो महामनाः}
{ब्राह्मणेभ्यो यथार्हेभ्यो ददौ वित्तान्यनेकशः}


\twolineshloka
{धर्मिराजश्च भीमश्च सव्यसाची यमावपि}
{तत्सर्वमन्ववर्तन्त धृतराष्ट्रव्यपेक्षया}


\twolineshloka
{कथं नु राजा वृद्धः स पुत्रपौत्रवधार्दितः}
{शोकमस्मत्कृतं प्राप्यि न म्रियेतेति चिन्त्य ते}


\twolineshloka
{यावद्धि कुरुवीरस्य जीवत्पुत्रस्य वै सुखम्}
{तावत्सुखमवाप्नोति भोगांश्चैव व्यवस्थितान्}


\twolineshloka
{ततस्ते सहिताः पञ्च भ्रातरः पाण्डुनन्दनाः}
{तथाशीलाः समातस्थुर्धृतराष्ट्रस्य शासने}


\twolineshloka
{धृतराष्ट्रश्च तान्सर्वान्विनीतान्नियमे स्थितान्}
{शिष्यवृत्तिं समापन्नान्गुरुवत्प्रत्यपद्यत}


\twolineshloka
{गान्धारी चैव पुत्राणां विविधैः श्राद्धकर्मभिः}
{आनृण्यमगमत्कामान्विप्रेभ्यः प्रतिपाद्य सा}


\twolineshloka
{एवं धर्मभृतांश्रेष्ठो धर्मराजो युधिष्ठिरः}
{भ्रातृभिः सहितो धीमान्पूजयामासं तं नृपम्}


\twolineshloka
{स राजा सुमहातेजा वृद्धः कुरुकुलोद्वहः}
{न ददर्श तदा किञ्चिदप्रियं पाण्डुनन्दने}


\twolineshloka
{वर्तमानेषु सद्वृत्तिं पाण्डवेषु महात्मसु}
{प्रीतिमानभवद्राजा धृतराष्ट्रोंऽबिकासुतिः}


\twolineshloka
{सौबलेयी च गान्धारी पुत्रशोकमपास्य तम्}
{सदैव प्रीतिमत्यासीत्तनयेषु निजेष्विव}


\twolineshloka
{प्रियाण्येव तु कौरव्यो नाप्रियाणि कुरूद्वहः}
{वैचित्रवीर्य नृपतौ समाचरत सर्वदा}


\twolineshloka
{यद्यद्ब्रूते च किञ्चित्स धृतराष्ट्रो जनाधिपः}
{गुरु वा लघु वा कार्यं गान्धारी च तपस्विनी}


\twolineshloka
{तं स राजा महाराज पाण्डवानां धुरंधरः}
{पूजयित्वा वचस्तत्तदकार्षीत्परवीरहा}


\twolineshloka
{तेन तस्याभवत्प्रीतो वृत्तेन स नराधिपः}
{अन्वतप्यत संस्मृत्य पुत्रं तं मन्दचेतसम्}


\twolineshloka
{सदा च प्रातरुत्थाय कृतजप्यः शुचिर्नृपः}
{आशास्ते पाण्डुपुत्राणां समरेष्वपराजयम्}


\twolineshloka
{ब्राह्मणान्स्वस्तिवाच्याथ हुत्वा चैव हुताशनम्}
{आयूंषि पाण्डुपुत्राणामाशंसत नराधिपः}


\threelineshloka
{न तां प्रीतिं परामाप पुत्रेभ्यः स तदा पुरा}
{यां प्रीतिं पाण्डुपुत्रेभ्यः सदाऽवाप नराधिपः}
{}


\twolineshloka
{ब्राह्मणानां यथा वृत्तः क्षत्रियाणां यथाविधः}
{तथा विट्शूद्रसङ्घानामभवत्स प्रियस्तदा}


\twolineshloka
{यच्च किञ्चित्तदा पापं धृतराष्ट्रसुतैः कृतम्}
{अकृत्वा हृदि तत्पापं तं नृपं सोन्ववर्तत}


\twolineshloka
{यः कश्चिदपि यत्किञ्चित्प्रमादादंबिकासुते}
{कुरुते द्वेष्यतामेति स कौन्तेयस्य धीमतः}


\twolineshloka
{न राज्ञो धृतराष्ट्रस्य न च दुर्योधनस्य वै}
{उवाच दुष्कृतं कश्चिद्युधिष्ठिरभयान्नरः}


\twolineshloka
{धृत्या तुष्टो नरेन्द्रः स गान्धारी विदुरस्तथा}
{अजातशत्रोर्वृत्तेन न तु भीमस्य पार्थिवः}


\twolineshloka
{अन्ववर्तत भीमोपि निश्चितो धर्मजं नृपम्}
{धृतराष्ट्रं च सम्प्रेक्ष्य सदा भवति दुर्मनाः}


\twolineshloka
{राजानमनुवर्तन्त धर्मपुत्रममित्रहा}
{अन्ववर्तत संक्रुद्धो हृदयेन पराङ्मुखः}


\chapter{अध्यायः ३}
\twolineshloka
{युधिष्ठिरस्य नृपतेर्दुर्योधनपितुस्तदा}
{नान्तरं दद्दशू राज्ये पुरुषाः प्रणयं प्रति}


\twolineshloka
{यदा तु कौरवो राजा पुत्रं सस्मार बालिशम्}
{तदा भीमं हृदा राजन्नपध्याति स पार्थिवः}


\twolineshloka
{कौरवश्चैव भीमश्च हृदाऽन्योन्यमवर्तताम्}
{ध्यायन्तौ श्लक्ष्णया वाचा त्वन्योन्यमभितिष्टितां}


\twolineshloka
{अप्रकाशं व्यलीकानि चकारास्य वृकोदरः}
{आज्ञां प्रत्यहरच्चापि कृतकैः पुरुषैः सदा}


\twolineshloka
{स्मरन्दुर्मन्त्रितं तस्य वृत्तान्यप्यस्य कानिचित्}
{`असकृच्चाप्युवाचेदं हतास्ते मन्दचेतसः ॥'}


\twolineshloka
{अथ भीमः सुहृन्मध्ये बाहुशब्दं तथाऽकरोत्}
{संश्रवे धृतराष्ट्रस्य गान्धार्याश्चाप्यमर्षणः}


\twolineshloka
{स्मृत्वा दुर्योधनं शत्रुं कर्णदुःखासनावपि}
{प्रोवाचेदं सुसंरब्धो भीमः सपरुषं वचः}


\twolineshloka
{अन्धस्य नृपतेः पुत्रा मया परिघवाहुना}
{नीता लोकममुं सर्वे नानाशस्त्रास्त्रयोधिनः}


\twolineshloka
{इमौ तौ परिघप्रख्यौ भुजौ मम दुरासदौ}
{ययोरन्तरमासाद्य धार्तराष्ट्राः क्षयं गताः}


\twolineshloka
{ताविमौ चन्दनेनाकौ वन्दनीयौ च मे भुजौ}
{याभ्यां दुर्योधनो नीतः क्षयं ससुतबान्धवः}


\twolineshloka
{एतास्चान्याश्च विविधाः शल्यभुता नराधिपः}
{वृकोदरस्य ता वाचः श्रुत्वा निर्वेदमागमत्}


\threelineshloka
{सा च बुद्धिमती देवी कालपर्यायवेदिनी}
{गान्धारी सर्वधर्मज्ञा तान्यलीकानि शुश्रुवे}
{`कुन्तीं वर्या तु संवीक्ष्य शापे नास्याकरोन्मति'}


\twolineshloka
{ततः पञ्चदशे वर्षे समतीते नराधिपः}
{राजा निर्वेदमापेदे भीमवाग्बाणपीडितः}


\twolineshloka
{`सुखासक्तं कृच्छ्रपरं ज्ञात्वा चैव युधिष्ठिरम्}
{तपोयोगात्तपस्तप्तुं मनश्चक्रे महामतिः ॥'}


\twolineshloka
{नान्वबुध्यत तद्राजा कुन्तीपुत्रो युधिष्ठिरः}
{श्वेताश्वोवाऽथकुन्ती वा द्रौपदी वा यशस्विनी}


\twolineshloka
{माद्रीपुत्रौ च भीमस्य मतं तावन्ववर्तताम्}
{राज्ञस्तु चित्तं रक्षन्तौ नोचतुः स्वयमप्रियम्}


\twolineshloka
{ततः समानयामास धृतराष्ट्रः सुहृज्जनम्}
{बाष्पसंदिग्धमत्यर्थमिदमाह च तान्भृशम्}


\twolineshloka
{विदितं भवतामेतद्यथा वृत्तः कुरुक्षयः}
{ममापराधात्तत्सर्वमिति ज्ञेयं तु कौरवाः}


\threelineshloka
{योऽहं दुष्टमतिं मन्दो ज्ञातीनां भयवर्धनम्}
{दुर्योधनं कौरवाणामाधइपत्येऽभ्यषेचयम्}
{यच्चाहं वासुदेवस्य नाश्रौषं वाक्यमर्थवत्}


\twolineshloka
{वध्यतां साध्वयं पापः सामात्य इति दुर्मतिः}
{पुत्रस्नेहाभिभूतस्तु हितमुक्तो मनीषिभिः}


\threelineshloka
{विदुरेणाथ भीष्मेण द्रोणेन च कृपेण च}
{पदेपदे भगवता व्यासेन च महात्मना}
{संजयेनाथ गान्धार्या तदिदं तप्यते मया}


\twolineshloka
{यच्चाहं पाण्डुपुत्रेषु गुणवत्सु महात्मसु}
{न न्यस्तवाञ्श्रियं दीप्तां पितृपैतामहीमिमाम्}


\twolineshloka
{विनाशं पश्यमानो हि सर्वराज्ञां गदाग्रजः}
{एतच्छ्रेयस्तु परमममन्यत जनार्दनः}


\twolineshloka
{सोहमेतान्यलीकानि दुर्वृत्तान्यात्मनस्तदा}
{हृदये शल्यभूतानि धारयामि सहस्रशः}


\twolineshloka
{विशेषतस्तु दह्यामि वर्षे पञ्चदशेऽद्य वै}
{अस्य पापस्य शुद्ध्यर्थं नियतोस्मि सुदुर्मतिः}


\twolineshloka
{चतुर्थे नियते काले कदाचिदपि चाष्टमे}
{तृष्णाविनयनं भुञ्जे गान्धारी वेद तन्मम}


\twolineshloka
{करोत्याहारमिति मां सर्वः परिजनः सदा}
{युधिष्ठिरभयादेति भृशं तप्यति पाण्डवः}


\twolineshloka
{भूमौ शये जप्यपरो दर्भष्वजिनसंवृतः}
{नियमव्यपदेशेन गान्धारी च यशस्विनी}


\twolineshloka
{अहं पुत्रशतं वीरं सङ्ग्रामेष्वपलायिनम्}
{नानुतप्ये हतं तत्र क्षत्रधर्मं हि तं विदुः}


\twolineshloka
{इत्युक्त्वा धर्मराजानमभ्यभाषत कौरवः}
{भद्रं ते यादवीमातर्वचश्चेदं निबोध मे}


\twolineshloka
{सुखमध्युषितः पुत्र त्वया सुपरिपालितः}
{मया दानानि दत्तानि श्राद्धानि च पुनःपुनः}


\twolineshloka
{प्रकृष्टं च मया पुत्र पुण्यं चीर्णं यथाबलम्}
{गान्धारी हतपुत्रेयं धैर्येणोदीक्षते च माम्}


\twolineshloka
{द्रौपद्या ह्यपकर्तारस्तव चैश्वर्यहारिणः}
{समतीता नृशंसास्ते स्वधर्मेण हता युधि}


\twolineshloka
{न तेषु प्रतिकर्तव्यं पश्यामि कुरुनन्दन}
{सर्वे शस्त्रजिताँल्लोकान्गतास्तेऽभिमुखं हताः}


\twolineshloka
{आत्मनस्तु हितं पुण्यं प्रतिकर्तव्यमद्य ते}
{गान्धार्याश्चैव राजेन्द्र तदनुज्ञातुमर्हसि}


\twolineshloka
{त्वं तु शस्त्रभृतां श्रेष्ठः सततं धर्मवत्सलः}
{राजा गुरुः प्राणभृतां तस्मादेतद्ब्रवीम्यहम्}


\twolineshloka
{अनुज्ञातस्त्वया वीर संश्रयेयं वनान्यहम्}
{चीरवल्कलभृद्राजन्गान्धार्या सहितोऽनया}


\threelineshloka
{तवाशिषः प्रयुञ्जानो भविष्यामि वनेचरः}
{उचितं नः कुले तात सर्वेषां भरतर्षभ}
{पुत्रेष्वैश्चर्यमाधाय वयसोन्ते वनं नृप}


\twolineshloka
{तत्राहं वायुभक्षो वा निराहारोपि वा वसन्}
{पत्न्या सहानया वीर चरिष्यामि तपः परम्}


\twolineshloka
{त्वं चापि फलभाक्तात तपसः पार्थिवो ह्यसि}
{फलभाजो हि राजानः कल्याणस्येतरस्य वा}


\chapter{अध्यायः ४}
\twolineshloka
{न मां प्रीणयते राज्यं त्वय्येवं दुःखिते नृप}
{दिङ्मामस्तु सुदुर्बुद्धिं राज्यसक्तं प्रमादिनम्}


\twolineshloka
{गृहे वसन्तं दुःखार्तमुपवासकृशं भृशम्}
{यताहारं क्षितिशयं नाविदं भ्रातृभिः सह}


\twolineshloka
{अहोस्मि वञ्चितो मूढो भवता गूढबुद्धिना}
{विस्वासयित्वा पूर्वं मां यदिदं दुःखमश्नुथा}


\twolineshloka
{किं मे राज्येन भोगैर्वा किं यज्ञैः किं सुखेन वा}
{यस्य मे त्वं महीपाल दुखान्येतान्यवाप्तवान्}


\twolineshloka
{पीडितं चापि जानामि राज्यमात्मानमेव च}
{अनेन वचसा तेऽद्य दुःखितस्य जनेश्वरः}


\twolineshloka
{भवान्पिता भवान्माता भवान्नः परमो गुरुः}
{भवता विप्रहीणा वै क्वनु तिष्ठामहे वयम्}


\twolineshloka
{औरसो भवतः पुत्रो युयुत्सुर्नृपसत्तम}
{अस्तु राजा महाराज यमन्यं मन्यते भवान्}


\twolineshloka
{अहं वनं गमिष्यामि भवान्राजा प्रशास्त्विदम्}
{नमामयशसा दग्धं भूयस्त्वं दग्धुमर्हसि}


\twolineshloka
{नाहं राजा भवान्राजा भवता परवानहम्}
{कथं गुरुं त्वां धर्मज्ञमनुज्ञातुमिहोत्सहे}


\twolineshloka
{न मन्युर्हृदि नः कश्चित्सुयोधनकृतेऽनघ}
{भवितव्यं तथा तद्धि वयं चान्ये च मोहिताः}


\twolineshloka
{वयं पुत्रा हि भवोत यथा दुर्योधनादयः}
{गान्धारी चैव कुन्ती च निर्विशेषे मते मम}


\twolineshloka
{स मां त्वं यदि राजेन्द्र परित्यज्य गमिष्यसि}
{पृष्ठतस्त्वनुयास्यामि सत्यमात्मानमालभे}


\twolineshloka
{इयं हि वसुसंपूर्णा मही सागरमेखला}
{भवता विप्रहीणस्य न मे प्रीतिकरी भवेत्}


\twolineshloka
{भवदीयमिदं सर्वं शिरसा त्वां प्रसादये}
{त्वदधीनाः स्म राजेन्द्र व्येतु ते मानसो ज्वरः}


\threelineshloka
{भवितव्यमनुप्राप्तो मन्ये त्वं वसुधाधिप}
{दिष्ट्या शुश्रूषमाणस्त्वां मोक्षिष्ये मनसो ज्वरं ॥धृतराष्ट्र उवाच}
{}


\twolineshloka
{तापस्ये मे मनस्तात वर्तते कुरुनन्दन}
{उचितं च कुलेऽस्माकमरण्यगमनं प्रभो}


\threelineshloka
{चिरमध्युषितः पुत्र चिरं शुश्रूषितस्त्वया}
{वृद्धं मामप्यनुज्ञातुमर्हसि त्वं नराधिप ॥वैशम्पायन उवाच}
{}


\twolineshloka
{इत्युक्त्वा धर्मराजानं वेपमानं कृताञ्जलिम्}
{उवाच विदुरं राजा धृतराष्ट्रोंऽबिकासुतः}


\twolineshloka
{संजयं च महात्मानं कृपं चापि महारथम्}
{अनुनेतुमिहेच्छामि भवद्भिर्वसुधाधिपम्}


\twolineshloka
{म्लायते मे मनो हीदं मुखं च परिशुष्यति}
{वयसा च प्रकृष्टेन वाग्व्यायामेन चैव ह}


\twolineshloka
{इत्युक्त्वा स तु धर्मात्मा वृद्धो राजा कुरूद्वहः}
{गान्धारीं शिश्रिये धीमान्सहसैव गतासुवत्}


\threelineshloka
{तं तु दृष्ट्वा समासीनं विसंज्ञमिव कौरवम्}
{आर्तिं राजाऽगमत्तीव्रां कौन्तेयः परवीरहा ॥युधिष्ठिर उवाच}
{}


\twolineshloka
{यस्य नागसहस्रेण शतसङ्ख्येन वै बलम्}
{सोयं नारीं व्यपाश्रित्य शेते राजा गतासुवत्}


\twolineshloka
{आयसी प्रतिमा येन भीमसेनस्य सा पुरा}
{चूर्णीकृता बलवता सोबलामाश्रितः स्त्रियम्}


\twolineshloka
{धिगस्तु मामधर्मज्ञं धिग्बुद्धिं धिक्च मे श्रुतम्}
{यत्कृते पृथिवीपालः शेतेऽयमतथोचितः}


\threelineshloka
{अहमप्युपवत्स्यामि यथैवायं गुरुर्मम}
{यदि राजा न भुङ्क्तेऽयं गान्धारी च यशस्विनी ॥वैशम्पायन उवाच}
{}


\twolineshloka
{ततोस्य पाणिना राजञ्जलशीतेन पाण्डवः}
{उरो मुखं च शनकैः पर्यमार्जत धर्मवित्}


\twolineshloka
{तेन रत्नौषधिमता पुण्येन च सुगन्धिना}
{पाणिस्पर्शेन राज्ञः स राजा संज्ञामवाप ह}


\twolineshloka
{स्पृशन्तं पाणिना भूयः परिष्यज्य च पाण्डवम्}
{`उवाच राजा धर्मज्ञो धृतराष्ट्रः शुभं वचः ॥'}


\threelineshloka
{जीवामीवातिसंस्पर्शात्तव राजीवलोचन}
{मूर्धानं च तवाघ्रातुमिच्छामि मनुजाधिप}
{पाणिभ्यां हि परिस्प्रष्टुं प्राणानां हितमात्मनि}


\twolineshloka
{अष्टमो ह्यद्य कालोऽयमाहारस्य कृतस्य मे}
{येनाहं कुरुशार्दूल शक्नोमि न विचेष्टितुम्}


\twolineshloka
{व्यायामश्चायमत्यर्थं कृतस्त्वामभियाचता}
{ततो ग्लानमनास्तान नष्टसंज्ञ इवाभवम्}


\threelineshloka
{तवामृतसुखस्पर्शं हस्तस्पर्शमिमं प्रभो}
{लब्ध्वा संजीवितोस्मीति मन्ये कुरुकुलोद्वह ॥वैशम्पायन उवाच}
{}


\twolineshloka
{एवमुक्तस्तु कौन्तेयः पित्रा ज्येष्ठेन भारत}
{पस्पर्श सर्वगात्रेषु स्नेहार्द्रस्तं शनैस्तदा}


\twolineshloka
{उपलभ्य ततः प्राणान्धृतराष्ट्रो महीपतिः}
{बाहुभ्यां सम्परिष्वज्य मूर्ध्न्याजिघ्रत पाण्डवम्}


\twolineshloka
{विदुरादयश्च ते सर्वे रुरुदुर्दुःखिता भृशम्}
{अतिदुःखात्तु राजानं नोचुः किञ्चन पाण्डवम्}


\twolineshloka
{गान्धारी त्वेव धर्मज्ञा मनसोद्वहती भृशम्}
{दुःखान्यधारयद्राजन्मैवमित्येव चाब्रवीत्}


\twolineshloka
{इतरास्तु स्त्रियः सर्वाः कुन्त्या सह सुदःखिताः}
{नेत्रैरागतविक्लेदैः परिवार्य स्थिताऽभवन्}


\twolineshloka
{अथाब्रवीत्पुनर्वाक्यं धृतराष्ट्रो युधिष्ठिरम्}
{अनुजानीहि मां राजंस्तापस्ये भरतर्षभ}


\twolineshloka
{ग्लायते मे मनस्तात भूयोभूयः प्रजल्पतः}
{न मामतः परं पुत्र परिक्लेष्टुमिहार्हसि}


\twolineshloka
{तस्मिंस्तु कौरवेन्द्रे तं तथा ब्रुवति पाण्डवम्}
{सर्वेषामवरोधानामार्तनादो महानभूत्}


% Check verse!
दृष्ट्वा कृशं विवर्णं च राजानमतथोचितम् ॥उपवासपरिश्रान्तं त्वगस्थिपरिवारितम्
\twolineshloka
{धर्मपुत्रः स्वपितरं परिष्वज्य महाप्रभुम्}
{शोकजं बाष्पमुत्सृज्यि पुनर्वचनमब्रवीत्}


\twolineshloka
{न कामये नरश्रेष्ठ जीवितं पृथिवीं तथा}
{यथा तव प्रियं राजंश्चिकीर्षामि परंतप}


\twolineshloka
{यदि चाहमनुग्राह्यो भवतो दयितोऽपि वा}
{क्रियतां तावदाहारस्ततो वेत्स्याम्यहं परम्}


\twolineshloka
{ततोऽब्रवीन्महातेजा धृतराष्ट्रो युधिष्ठिरम्}
{अनुज्ञातस्त्वया पुत्र भुञ्जीयामिति कामये}


\twolineshloka
{इति ब्रुवति राजेन्द्रे धृतराष्ट्रे युधिष्ठिरम्}
{ऋषिः सत्यवतीपुत्रो व्यासोऽभ्योत्य वचोऽब्रवीत्}


\chapter{अध्यायः ५}
\twolineshloka
{युधिष्ठिर महाबाहो ********** कुरुनन्दनः}
{धृतराष्ट्रो महातेजास्तत्कुरुष्वाविचारयन्}


\twolineshloka
{अयं हि वृद्धो नृपतिर्हतपुत्रो विशेषतः}
{नेदं कृच्छ्रं चिरतरं सहेदिति मतिर्मम}


\twolineshloka
{गान्धारी च महाभागा प्राज्ञा करुणवेदिनी}
{पुत्रशोकं महाराज धैर्येणोद्वहते भृशम्}


\twolineshloka
{अहमप्येतदेव त्वां ब्रवीमि कुरु मे वचः}
{अनुज्ञां लभतां राजा मां वृथेह मरिष्यति}


\fourlineindentedshloka
{`स्वस्थो भवत्वयं धीमान्वनेषु मधुगन्धिषु}
{'राजर्षीणां पुराणानामनुयातु गतिं नृपः}
{राजर्षीणां हिसर्वेषामन्ते वनमुपाश्रयः ॥ वैशम्पायन उवाच}
{}


\twolineshloka
{इत्युक्तः स तदा राजा व्यासेनाद्भुतकर्मणा}
{प्रत्युवाच महातेजा धर्मराजो महामुनिम्}


\twolineshloka
{भगवानेव नो मान्यो भगवानेव नो गुरुः}
{भगवानस्य राज्यस्य कुलस्य च परायणम्}


\threelineshloka
{अहं तु पुत्रो भगवन्पिता राजा गुरुश्च मे}
{निदेशवती च पितुः पुत्रो भवति धर्मतः ॥वैशम्पायन उवाच}
{}


\twolineshloka
{इत्युक्तः स तु तं प्राह व्यासो वेदविदांवरः}
{युधिष्ठिरं महातेजाः पुनरेव महाकविः}


\twolineshloka
{एवमेतन्महाभाग यथा वदसिं भारत}
{राजाऽयं वृद्धतां प्राप्तः प्रमाणे परमे स्थितः}


\twolineshloka
{सोयं मयाऽभ्यनुज्ञातस्त्वथा च पृथिवीपतिः}
{करोतु स्वमभिप्रायं मा स्म विघ्नकरो भव}


\twolineshloka
{एष एव परो धर्मो राजर्षीणां युधिष्ठिर}
{समरे वा भवेन्मृत्युर्वने वा विधिपूर्वकम्}


\twolineshloka
{पित्रा तु तव राजेन्द्र पाण्डुना पृथिवीक्षिता}
{शिष्यभूतेन राजाऽयं गुरुवत्पर्युपासितः}


\twolineshloka
{क्रतुभिर्दक्षिणावद्भी रत्नपर्वतशोभितैः}
{महद्भिरिष्टं गौर्भुक्ता प्रजाश्च परिपालिताः}


\twolineshloka
{पुत्रसंस्यं च विपुलं राज्यं विप्रोषिते त्वयि}
{त्रयोदशसमा भुक्तं दत्तं च विविधं वसु}


\twolineshloka
{त्वया चायं नरव्याघ्र गुरुशुश्रूषयाऽनघ}
{आराधितः स भृत्येन गान्धारी च यशस्विनी}


\threelineshloka
{अनुजानीहि पितरं समयोऽस्म तपोविधौ}
{न मन्युर्विद्यते चास्य सुसूक्ष्मोऽपि युधिष्ठिर ॥वैशम्पायन उवाच}
{}


\twolineshloka
{एतावदुक्त्वा वचनमनुमान्य च पार्थिवम्}
{तथाऽस्त्विति च तेनोक्तः कौतेयेन ययौ वनम्}


\twolineshloka
{गते भगवति व्यासे राजा पाण्डुसुतस्तदा}
{प्रोवाच पितरं वृद्धं मन्दंमन्दमिवानतः}


\twolineshloka
{यदाह भगवान्व्यासो यच्चापि भवतो मतम्}
{यथाऽऽह च महेष्वासः कृपो विदुर एव च}


\twolineshloka
{युयुत्सुः संजयश्चैव तत्कर्तास्म्यहमञ्जसा}
{सर्व एव हि मान्या मे कुलस्य हि हितैषिणः}


\twolineshloka
{इदं तु याचे नृपते त्वामहं शिरसा नतः}
{क्रियतां तावदाहारस्ततो गच्छाश्रमं प्रति}


\chapter{अध्यायः ६}
\twolineshloka
{ततो राज्ञाऽभ्यनुज्ञातो धृतराष्ट्रो महामनाः}
{ययौ स्वभवनं राजा गान्धार्याऽनुगतस्तदा}


\twolineshloka
{मन्दप्राणगतिर्धामान्कृच्छ्रादिव समुद्वहन्}
{पदानि स महीपालो जीर्णो गजपतिर्यथा}


\twolineshloka
{तमन्वगच्छद्विदुरो विद्वान्सूतश्च संजयः}
{स चापि परमेष्वासः कृपः सारद्वतस्तथा}


\twolineshloka
{स प्रविश्य गृहं राजा कृतपूर्वाह्णिकक्रियः}
{तर्पयित्वा द्विजश्रेष्ठानाहारमकरोत्तदा}


\twolineshloka
{गान्धारी चैव धर्मज्ञा कुन्त्या सह मनस्विनी}
{वधूभिरुपचारेणि पूजिताऽभुङ्क्त भारत}


\twolineshloka
{कृताहारं कृताहाराः सर्वे ते विदुरादयः}
{पाण्डवाश्च कुरुश्रेष्ठमुपातिष्ठन्त तं नृपम्}


\twolineshloka
{ततोऽब्रवीन्महाराजः कुन्तीपुत्रमुपह्वरे}
{निषण्णं पाणिना पृष्ठे संस्पृशन्नंबिकासुतः}


\twolineshloka
{अप्रमादस्त्वया कार्यः सर्वथा कुरुनन्दन}
{अष्टाङ्गे राजशार्दूल राज्ये धर्मपुरस्कृते}


\twolineshloka
{तत्तु शक्यं महाराज रक्षितुं पाण्डुनन्दन}
{राज्यं धर्मेण कौन्तेय विद्वानसि निबोध तत्}


\twolineshloka
{विद्यावृद्धान्सदैव त्वमुपासीथा युधिष्ठिर}
{शृणुयास्ते च यद्ब्रूयुः कुर्याश्चैवाविचारयन्}


\twolineshloka
{प्रातरुस्थाय तान्राजन्पूजयित्वा यथाविधि}
{कृत्यकाले समुत्पन्ने पृच्छेथाः कार्यमात्मनः}


\twolineshloka
{ते तु सम्मानिता राजंस्त्वया लोकहितार्थिना}
{प्रवक्ष्यन्ति हितं तात सर्वथा तव भारत}


\twolineshloka
{इन्द्रियाणि च सर्वाणि वाजिवत्परिपालय}
{हितायैव भविष्यन्ति रक्षितं द्रविणं यथा}


\twolineshloka
{अमात्यानुपधातीतान्पितृपैतामहाञ्शुचीन्}
{दान्तान्कर्मसु सर्वेषु मुख्यान्मुख्येषु योजयेः}


\twolineshloka
{चारयेथाश्च सततं चारैरविदितः परैः}
{परीक्षितैर्बहुविधैः स्वराष्ट्रेषु परेषु च}


\twolineshloka
{पुरं च ते सुगुप्तं स्याद्दृढप्राकारतोरणम्}
{अट्टाट्टालकसम्बाधं षट्पदं सर्वतो दिशम्}


\twolineshloka
{तस्य द्वाराणि सर्वाणि पर्याप्तानि बृहन्ति च}
{सर्वतः सुविभक्तानि यन्त्रैरारक्षितानि च}


\threelineshloka
{पुरुषैरलमर्थस्ते विदितैः कुलशीलतः}
{आत्मा च रक्ष्यः सततं भोजनादिषु भारत}
{विहाराहारकालेषु माल्यशय्यासनेषु च}


\twolineshloka
{स्त्रियश्च ते सुगुप्ताः स्युर्वृद्धैराप्तैरधिष्ठिताः}
{शीलवद्भिः कुलीनैश्च विद्वद्भिश्च युधिष्ठिर}


\twolineshloka
{मन्त्रिणिश्चैव कुर्वीथा द्विजान्विद्याविशारदान्}
{विनीतांश्च कुलीनांश्च धर्मार्थकुशलानृजून्}


\twolineshloka
{तैः सार्धं मन्त्रयेथास्त्वं नात्यर्थं बहुभिःक सह}
{समस्तैरपि च व्यस्तैर्व्यपदेशेन केनचित्}


\twolineshloka
{सुसंवृतं मन्त्रगृहं स्थलं चारुह्य मन्त्रयेः}
{अरण्ये निःशलाके वा न च रात्रौ कथञ्चन}


\twolineshloka
{वानराः पक्षिणश्चैव ये मनुष्यानुकारिणः}
{सर्वे मन्त्रगृहे वर्ज्या ये चापि जडपङ्गवः}


\twolineshloka
{मन्त्रभेदे हि ये दोषा भन्ति पृथिवीक्षिताम्}
{न ते शक्याः समाधातुं कथञ्चिदिति मे मतिः}


\twolineshloka
{दोषांश्च मन्त्रभेदस्य ब्रूयास्त्वं मन्त्रिमण्डले}
{अभेदे च गुणा राजन्पुनःपुनररिंदम}


\twolineshloka
{पौरजानपदानां च शौचशौचे युधिष्ठिर}
{यथा स्याद्विदितं राजंस्तथा कार्यं कुरूद्वह}


\twolineshloka
{व्यवहारश्च ते राजन्नित्यमाप्तैरधिष्ठितः}
{योज्यस्तुष्टैर्हितै राजन्नित्यं चारैरनुष्ठितः}


\twolineshloka
{परिमाणं विदित्वा च दण्डं दण्ड्येषु भारत}
{प्रणयेयुर्यथान्यायं पुरुषास्ते युधिष्ठिर}


\twolineshloka
{आदानरुचयश्चैव परदाराभिमर्शिनः}
{उग्रदण्डिप्रधानाश्च मिथ्याव्याहारिणस्तथा}


\threelineshloka
{आक्रोष्टारश्च लुब्धाश्च हर्तारः साहसप्रियाः}
{सभाविहारभेत्तारो वर्णानां च प्रदूषकाः}
{हिरण्यदण्ड्या वध्याश्च कर्तव्या देशकालतः}


\twolineshloka
{`अवरोधभूमौ भृत्यैश्च पानं सह विवर्जयेत्}
{आक्रोशन्त्यनुमत्तास्ते कलत्रं वाऽपि गृह्णते}


\twolineshloka
{जिघांसन्त्यपि शस्त्रेण नष्टाः क्रीडन्ति चोत्कटाः}
{नानाक्षेपा व्याहरन्ति गम्यागम्यं न जानते}


\twolineshloka
{अतिपानेनि राजाऽपि सर्वं कोशं विनाशयेत्}
{वितरेद्गायकेभ्यश्च वृथा च द्रव्यसञ्चयम्}


\twolineshloka
{शब्दमात्मनि दोषांश्च पिबेदेकश्च जायया}
{युक्त्या प्रकाशमयति सुवीर्यस्यि विवृद्धये ॥'}


\twolineshloka
{प्रातरेव हि पश्यथा ये कुर्युः प्रियकर्म ते}
{अलङ्कारमथो भोज्यमत ऊर्ध्वं समाचरेः}


\twolineshloka
{पश्येथाश्चि ततो योधान्सदा त्वं प्रतिहर्षयन्}
{दूतानां च चराणां च प्रदोषस्ते सदा भवेत्}


\twolineshloka
{सदा चापररात्रान्ते भवेत्कार्यार्थनिर्णयः}
{मध्यरात्रे विहारस्ते मध्याह्ने च सदा भवेत्}


\twolineshloka
{सर्वे त्वौपयिकाः कालाः कार्याणां भरतर्षभ}
{तथैवालङ्कृतः काले तिष्ठिथा भूरिदक्षिणः}


\twolineshloka
{`न विद्राव्य च तिष्ठेत परिहार्य विभूषणम्}
{प्रयोज्यं सर्वदैवेह मङ्गल्यं पापनाशनम् ॥'}


\threelineshloka
{चक्रवत्तात कार्याणां पर्यायो दृश्यते सदा}
{कोशस्य निचये यत्नं कुर्वीथा न्यायतः सदा}
{विविधस्य महाराज विपरीतं विवर्जयेः}


\twolineshloka
{चारैर्विदित्वा शत्रूंश्च ये राज्ञामन्तरैषिणः}
{तानाप्तैः पुरुषैर्दूराद्धातयेथा नराधिप}


\twolineshloka
{कर्म दृष्ट्वाऽथ भृत्यांस्त्वं वरयेथाः कुरूद्वह}
{कारयेथाश्च कर्माणि युक्तायुक्तैरधिष्ठितैः}


\twolineshloka
{सेनाप्राणेता च भवेत्तव तात दृढव्रतः}
{शूरः क्लेशसहश्चैव हितो भक्तश्च पूरुषः}


\twolineshloka
{सर्वे जनपदाश्चैव तव कर्माणि पाण्डव}
{गोवद्रासभवच्चैव कुर्युर्ये व्यवहारिणः}


\twolineshloka
{स्वरन्ध्रं पररन्ध्रं च स्वेषु चैव परेषु च}
{उपलक्षयितव्यं ते नित्यमेव युधिष्ठिर}


\twolineshloka
{देशजाश्चैव पुरुषा विक्रान्ताः स्वेषु कर्मसु}
{यात्राभिरनुरूपाभिरनुग्राह्या हितास्त्वया}


\twolineshloka
{गुणार्थिनां गुणः कार्यो विदुषा वै जनाधिप}
{अविचार्याश्च ते ते स्युरचला इव नित्यशः}


\chapter{अध्यायः ७}
\twolineshloka
{मण्डलानि च बुध्येथाः परेषामात्मनस्तथा}
{उदासीनगुणानां च मध्यस्थानां च भारत}


\twolineshloka
{चतुर्णां शत्रुजातानां सर्वेषामाततायिनाम्}
{मित्रं चामित्रमित्रं च बोद्धव्यं तेऽरिकर्शन}


\twolineshloka
{अथ नानाजनपदा दुर्गाणि विविधानि च}
{बलानि च कुरुश्रेष्ठ भवत्येषां यथेच्छकम्}


\twolineshloka
{ते च द्वादश कौन्तेय राजानो विविधात्मकाः}
{मन्त्रिप्रधानाश्च गुणाः षष्टिर्द्वादश च प्रभो}


\twolineshloka
{एतन्मण्डलमित्याहुराचार्या नीतिकोविदाः}
{अत्र षाड्गुण्यमाचत्तं युधिष्ठिर निबोध तत्}


\twolineshloka
{वृद्धिक्षयौ च विज्ञेयौ स्थानं च कुरुसत्तम}
{द्विसप्तत्यां महाबाहो ततः षाड्गुण्यजा गुणाः}


\twolineshloka
{यथा स्वपक्षो बलवान्परपक्षस्तथा बलः}
{विगृह्य शत्रून्कौन्तेय यायात्क्षितिपतिस्तदा}


\twolineshloka
{यदा परे च बलिनः स्वपक्षश्चैव दुर्बलः}
{सार्धं विद्वांस्तदा क्षीणः परैः सन्धिं मसाश्रयेत्}


\twolineshloka
{द्रव्याणां सञ्चयश्चैव कर्तव्यः सुमहांस्तथा}
{यदा समर्था यानाय नचिरेणैव भारत}


\twolineshloka
{तदा सर्वं विधेयं स्यात्स्थानेन च विचारयेत्}
{भूमिरल्पफला देया विपरीतस्य भारत}


\twolineshloka
{हिरण्यरूप्यभूयिष्ठं मित्रं क्षीणमकोशभृत्}
{विपरीतं निगृह्णीयात्स्वयं सन्धिविशारदः}


\twolineshloka
{सन्ध्यर्थं राजपुत्रं वा लिप्सेथा भरतर्षभ}
{विपरीतं न तच्छ्रेयः पुत्र कस्यांचिदापदि}


\twolineshloka
{तस्याः प्रमोक्षे यत्नं च कुर्याः सोपायमन्त्रवित्}
{प्रकृतीनां च राजेन्द्र राजा दीनान्विभावयेत्}


\twolineshloka
{क्रमेणि युगपद्बुध्वा व्यसनानां बलाबलम्}
{पीडनं स्तंभनं चैव कोशभङ्गस्तथैव च}


\twolineshloka
{कार्यं यत्नेन शत्रूणां स्वराज्यं रक्षता स्वयम्}
{न च हिंस्योऽभ्युपगतः सामन्तो वृद्धिमिच्छता}


\twolineshloka
{कौन्तेय तद्धितं ते स्यात्पृथिवीं विजिगीषतः}
{गुणानां भेदने योगमीप्सेथाः सह मन्त्रिभिः}


\threelineshloka
{साधुसङ्ग्रहणाच्चैव पापनिग्रहणात्तथा}
{आत्मसात्करणे नित्यं पालनानि गृहे तथा}
{दुर्बलाश्चैव सततं नान्वेष्टव्या बलीयसा}


\threelineshloka
{तिष्ठेथा राजशार्दूल वैतसीं वृत्तिमास्थितः}
{यद्येनमभियायाच्च बलवान्दुर्बल नृपः}
{}


\twolineshloka
{सामादिभिरुपायैस्तं क्रमेणि विनिवर्तयेः}
{अशक्नुवंश्च युद्धाय निष्पतेत्सह मन्त्रिभिः}


\threelineshloka
{कोशेन पौरैर्दण्डेन ये चास्य प्रियकारिणः}
{असम्भवे तु सर्वस्य यथा मुख्येन निष्पतेत्}
{क्रमेणानेन मोक्षः स्याच्छरीरं प्रति केवलम्}


\chapter{अध्यायः ८}
\twolineshloka
{सन्धिविग्रहमप्यत्र पश्येथा राजसत्तम}
{द्वियोनिं विविधोपायं बहुकल्पं युधिष्ठिर}


\twolineshloka
{कौरव्य पर्युपासीथाः स्थित्वा द्वैविध्यमात्मनः}
{तुष्टपुष्टजनः शत्रुरर्थवानिति च स्मरेत्}


\twolineshloka
{पर्युपासनकाले तु विपरीतं विधीयते}
{आमर्दकाले राजेन्द्र व्यवसायस्ततोऽपरः}


\twolineshloka
{व्यसनं भेदनं चैव शत्रूणां कारयेत्ततः}
{कर्षणं भीषणं चैव युद्धे चैव बहुक्षयम्}


\twolineshloka
{प्रयास्यमानो नृपतिस्त्रिविधं परिचिन्तयेत्}
{आत्मनश्चैव शत्रोश्च शक्तिं शास्त्रविशारदः}


\twolineshloka
{उत्साहप्रभुशक्तिभ्यां मन्त्रशक्त्या च भारत}
{उपपन्नो नृपो यायाद्विपरीतं च वर्जयेत्}


\twolineshloka
{आददीत बलं राजा मौलं मित्रबलं तथा}
{अटवीबलं भृतं चैव तता श्रेणीबलं प्रभो}


\threelineshloka
{`मित्रामित्रबलं राजन्न्यायाद्वृद्ध्युदये धृतः}
{'तत्र मित्रबलं राजन्मौलं चैव विशिष्यते}
{श्रेणीबलं भृतं चैव तुल्ये एवेति मे मतिः}


\twolineshloka
{तथाऽऽचारबलं चैव परस्परसमं नृप}
{विज्ञेयं बलकालेषु राज्ञा काल उपस्थिते}


\twolineshloka
{आपदश्चापि बोद्धव्या बहुरूपा नराधिप}
{भवन्ति राज्ञा कौरव्य यास्ताः पृथगतः शृणु}


\twolineshloka
{विकल्पा बहुधा राजन्नापदां पाण्डुनन्दन}
{सामादिभिरुपन्यस्य गमयेत्तान्नृपः सदा}


\twolineshloka
{यात्रां गच्छेद्बलैर्युक्तो राजा षड्भिः परंतप}
{युक्तश्च देशकालाभ्यां बलैरात्मगुणैस्तथा}


\twolineshloka
{हृष्टपुष्टबलो गच्छेद्राजा वृद्ध्युदये रतः}
{अकृशश्चाप्यथो यायादनृतावपि पाण्डव}


\twolineshloka
{तूणाश्मानं वाजिरथप्रवाहांध्वजद्रुमैः संवृतकूलरोधसम्}
{पदातिनागैर्बहुकर्दमां नदींसपत्नानाशे नृपतिः प्रयोजयेत्}


\twolineshloka
{अथोपपत्त्या शकटं पद्मवज्रं च भारत}
{उशना वेद यच्छास्त्रं तत्रैतद्विहितं विभो}


\twolineshloka
{चारयित्वा परबलं कृत्वा स्वबलदर्शनम्}
{स्वभूमौ योजयेद्युद्धं परभूमौ तथैव च}


\twolineshloka
{बलं प्रसादयेद्राजा निक्षिपेद्बलनो नरान्}
{ज्ञात्वा स्वविषयं तत्र सामादिभिरुपक्रमेत्}


\twolineshloka
{सर्वथैव महाराज शरीरं धारयेदिह}
{प्रेत्य चेह च कर्तव्यमात्मनिःश्रेयसं परम्}


\twolineshloka
{एवं कुर्वञ्शुभा वाचो लोकेऽस्मिञ्शृणु ते नृप}
{प्रेत्य स्वर्गमवाप्नोति प्रजा धर्मेणि पालयन्}


\twolineshloka
{एवं त्वया कुरुश्रेष्ठ वर्तितव्यं प्रजासु वै}
{उभयोर्लोकयोस्तात प्राप्तये नित्यमेव हि}


\twolineshloka
{भीष्मेण सर्वमुक्तोसि कृष्णेन विदुरेण च}
{मयाऽप्यवश्यं वक्तव्यं प्रीत्या ते नृपसत्तम}


\twolineshloka
{एतत्सर्वं यथान्यायं कुर्वीथा भूरिदक्षिण}
{प्रियस्तथा प्रजानां त्वं स्वर्गे सुखमवाप्स्यसि}


\twolineshloka
{अश्वमेधसहस्रेण यो यजेत्पृथिवीपतिः}
{पालयेद्वाऽपि धर्मेणि प्रजास्तुल्यं फलं लभेत्}


\chapter{अध्यायः ९}
\twolineshloka
{एवमेतत्करिष्यामि यथाऽऽत्थ पृथिवीपते}
{भूयश्चैवानुशास्योऽहं भवता पार्थिवर्षभ}


\twolineshloka
{भीष्मे स्वर्गमनुप्राप्तो गते च मधुसूदने}
{विदुरे संजये चैव कोऽन्यो मां वक्तुमर्हति}


\threelineshloka
{यत्तु मामनुशास्तीह भवानद्य हिते स्थितः}
{कर्तास्मि तन्महीपाल निर्वृतो भव पार्थिव ॥वैशम्पायन उवाच}
{}


\twolineshloka
{एवमुक्तः स राजर्षिर्धर्मराजेन धीमता}
{कौन्तेयं समनुज्ञातुमियेष भरतर्षभ}


\twolineshloka
{पुत्र विश्राम्यतां तावन्ममापि बलवाञ्श्रमः}
{इत्युक्त्वा प्राविशद्राजा गान्धार्या भवनं तदा}


\twolineshloka
{तमासनगतं देवी गान्धारी धर्मचारिणी}
{उवाच काले कालज्ञा प्रजापतिसमं पतिम्}


\threelineshloka
{अनुज्ञातः स्वयं तेन व्यासेनि त्वं महर्षिणा}
{युधिष्ठिरस्यानुमते कदाऽरण्यं गमिष्यसि ॥धृतराष्ट्र उवाच}
{}


\twolineshloka
{गान्धार्यहमनुज्ञातः स्वयं पित्रा महात्मना}
{युधिष्ठिरस्यानुमते गन्तास्मि नचिराद्वनम्}


\fourlineindentedshloka
{अहं हि तावत्सर्वेषां तेषां दुर्द्यूतदेविनाम्}
{पुत्राणां दातुमिच्छामि प्रेत्यभावानुगं वसु}
{सर्वप्रकृतितिसान्निध्यं कारयित्वा स्ववेश्मनि ॥वैशम्पायन उवाच}
{}


\twolineshloka
{इत्युक्त्वा धर्मराजाय प्रेषयामास वै तदा}
{स च तद्वचनात्सर्वं समानिन्ये महीपतिः}


\twolineshloka
{ततः प्रतीतमनसो ब्राह्मणाः कुरुजाङ्गलाः}
{क्षत्रियाश्चैव वैश्याश्च शूद्राश्चैव समाययुः}


\twolineshloka
{ततो निष्क्रम्य नृपतिस्तस्मादन्तःकपुरात्तदा}
{दद्दशे तं जनं सर्वं सर्वाश्च प्रकृतीस्तथा}


\twolineshloka
{समवेतांश्च तान्सर्वान्पौराञ्जानपदांस्तथा}
{तानागतानभिप्रेक्ष्य समस्तं च सुहृञ्जनम्}


\twolineshloka
{ब्राह्मणांश्च महीपाल नानादेशसमागतान्}
{उवाच मतिमान्राजा धृतराष्ट्रोऽम्बिकासुतः}


\twolineshloka
{भवन्तः कुरुवश्चैव चिरकालं सहोषिताः}
{परस्परस्य सुहृदः परस्परहिते रताः}


\twolineshloka
{यदिदानीमहं ब्रूयामस्मिन्काल उपस्थिते}
{तथा भवद्भिः कर्तव्यमविचार्य वचो मम}


\twolineshloka
{अरण्यगमने बुद्धिर्गान्धारीसहितस्य मे}
{व्यासस्यानुमते राज्ञस्तथा कुन्तीसुतस्य मे}


% Check verse!
भवन्तोप्यनुजानन्तु मान्या वोऽभूद्विचारणा
\twolineshloka
{अस्माकं भवतां चैव येयं प्रीतिर्हि शाश्वती}
{न च साऽन्येषु देशेषु राज्ञामिति मतिर्मम}


\twolineshloka
{शान्तोस्मि वयसाऽनेन तथा पुत्रविनाकृतः}
{उपवासकृशश्चास्मि गान्धारीसहितोऽनघाः}


\twolineshloka
{युधिष्ठिरगते राज्ये प्राप्तश्चास्मि सुखं महत्}
{मन्ये दुर्योधनैश्वर्याद्विशिष्टं बहुभिर्गुणैः}


\twolineshloka
{मम चान्धस्य वृद्धस्य हतपुत्रस्य का गतिः}
{ऋते वनं महाभागास्तन्माऽनुज्ञातुमर्हथ}


\twolineshloka
{तस्य तद्वचनं श्रुत्वा सर्वे ते कुरुजाङ्गलाः}
{बाप्पसंदिग्धया वाचा रुरुदुर्भरतर्षभ}


\twolineshloka
{तानविब्रुवतः किञ्चित्सर्वाञ्शोकपरायणान्}
{पुनरेव महातेजा धृतराष्ट्रोऽब्रवीदिदम्}


\chapter{अध्यायः १०}
\threelineshloka
{शान्तनुः पालयामास यथावद्वसुधामिमाम्}
{तथा विचित्रवीर्यश्च भीष्मेण परिपालितः}
{पालयामास नस्तातो विदितं वो न संशयः}


\twolineshloka
{यथा च पाण्डुर्भाता मे दयितो भवतामभूत्}
{स चापि पालयामास यथावत्तच्च वेत्थ ह}


\twolineshloka
{`अनन्तरं हि पितरमनुज्ञातो युधिष्ठिरः}
{नात्र किञ्चिन्मृषा जातु भाषतेति मतिर्मम ॥'}


\twolineshloka
{मया च भवतां सम्यक् शुश्रूषा या कृताऽनघाः}
{असम्यग्वा महाभागास्तत्क्षन्तव्यमतन्द्रितैः}


\twolineshloka
{यदा दुर्योधनेनेदं भुक्तं राज्यमकण्टकम्}
{अपि तत्र न वो मन्दो दुर्बुद्धिरपराद्धवान्}


\threelineshloka
{तस्यापराधाद्दुर्बुद्धेरभिमानान्महीक्षिताम्}
{विमर्दः सुमहानासीदनयात्स्वकृतादथ}
{`घातिताः कौरवेयाश्च पृथिवी च विनाशिता ॥'}


\twolineshloka
{तन्मया साधु वाऽपीदं यदि वाऽसाधु वै कृतम्}
{तद्वो हृदि न कर्तव्यं मया बद्धोऽयमञ्जलिः}


\twolineshloka
{वृद्धोऽयं हतपुत्रोऽयं दुःखितोऽयं नराधिपः}
{पूर्वराज्ञां च पुत्रोऽयमिति कृत्वाऽनुजानथ}


\twolineshloka
{इयं च कृपणा वृद्धा हतपुत्रा तपस्विनी}
{गान्धारी पुत्रशोकार्ता तुल्यं याचति वो मया}


\twolineshloka
{हतपुत्राविमौ वृद्धौ विदित्वा दुःखितौ तथा}
{अनुजानीत भद्रं वो व्रजाव शरणं च वः}


\threelineshloka
{अयं च कौरवो राजा कुन्तीपुत्रो युधिष्ठिरः}
{सर्वैर्भवद्भिर्द्रव्यः समेषु विषमेषु च}
{न जातु विषमं चैव गमिष्यति कदाचन}


\twolineshloka
{चत्वारः सचिवा यस्य भ्रातरो विपुलौजसः}
{लोककपालसमा ह्येते सर्वधर्मार्थदर्शिनः}


\twolineshloka
{`चतुर्णां लोकपालानां मध्ये विपरिवर्तते}
{'ब्रह्मेव भगवानेष सर्वभूतजगत्पतिः}


\twolineshloka
{`एवमेव महाबाहुर्भीमार्जुनयमैर्वृतः}
{'युधिष्ठिरो महातेजा भवतः पालयिष्यति}


\threelineshloka
{अवश्यमेव वक्तव्यमिति कृत्वा ब्रवीमि वः}
{एष न्यासो मया दत्तः सर्वेषां वो युधिष्ठिरः}
{भवन्तोऽस्य च वीरस्य न्यासभूताः कृता मया}


\twolineshloka
{यदेव तैः कृतं किञ्चिद्व्यलीकं वः सुतैर्मम}
{यदन्येनि मदीयेन तदनुज्ञातुमर्हथ}


\twolineshloka
{भवद्भिर्न हि मे मन्युः कृतपूर्वः कथञ्चन}
{अत्यन्तगुरुभक्तानामेषोऽञ्जलिरिदं नमः}


\twolineshloka
{तेषामस्थिरबुद्धीनां लुब्धानां कामचारिणाम्}
{कृते याचेऽद्य वः सर्वान्गान्धारीसहितोऽनघाः}


\twolineshloka
{इत्युक्तांस्तेन ते सर्वे पौरजानपदा जनाः}
{नोचुर्बाष्पकलाः किञ्चिद्वीक्षांचक्रुः परस्परम्}


\chapter{अध्यायः ११}
\twolineshloka
{एवमुक्तास्तु ते तेन पौरजानपदा जनाः}
{वृद्धेन राज्ञा कौरव्य नष्टसंज्ञा इवाभवन्}


\twolineshloka
{तूष्णींभूतांस्ततस्तांस्तु वाष्पकण्ठान्महीपतिः}
{धृतराष्ट्रो महीपालः पुनरेवाभ्यभाषत}


\twolineshloka
{वृद्धं च हतपुत्रं च धर्मपत्न्या सहानया}
{विलपन्तं बहुविधं कृपणं चैव सत्तमाः}


\twolineshloka
{पित्रा स्वयमनुज्ञातं कृष्णद्वैपायनेन वै}
{वनवासाय धर्मज्ञा धर्मजेन नृपेण ह}


\threelineshloka
{सोहं पुनःपुनर्याचे शिरसाऽवनतोऽनघाः}
{गान्धार्या सहितं तन्मां समनुज्ञातुमर्हथ ॥वैशम्पायन उवाच}
{}


\twolineshloka
{तच्छ्रुत्वा कुरुराजस्य वाक्यानि करुणानि ते}
{रुरुदुः सर्वशो राजन्समेताः कुरुजाङ्गलाः}


\twolineshloka
{उत्तरीयैः करैश्चापि संछाद्य वदनानि ते}
{रुरुदुः शोकसंतप्ता मुहूर्तं पितृमातृवत्}


\twolineshloka
{हृदयैः शून्यभूतैस्ते धृतराष्ट्रप्रवासजम्}
{दुःखं संधारयन्तो हि नष्टसंज्ञा हवाभवन्}


\twolineshloka
{ते विनीय तमायासं धृतराष्ट्रवियोगजम्}
{शनैः शनैस्तदाऽन्योन्यमब्रुवन्स्वमतान्युत}


\twolineshloka
{ततः सञ्चिन्त्य ते सर्वे वाक्यान्यथ समासतः}
{एकस्मिन्ब्राह्मणे कार्यमावेस्योचुर्नराधिपम्}


\twolineshloka
{ततः स्वाचरणो विप्रः सम्मतोऽर्थविशारदः}
{सम्भाव्यो बह्वृचो राजन्वक्तुं समुपचक्रमे}


\twolineshloka
{अनुमान्य महाराजं सदः समनुभाष्य च}
{विप्रः प्रगल्भो मेधावी स राजानमुवाच ह}


\twolineshloka
{राजन्वाक्यं जनस्यास्य मयि सर्वं समर्पितम्}
{वक्ष्यामि तदहं वीर तज्जुषस्व नपाधिप}


\twolineshloka
{यथा वदसि राजेन्द्र सर्वमेतत्तथा विभो}
{नात्र मिथ्या वचः किञ्चित्सुहृत्त्वं नः परस्परम्}


\twolineshloka
{न जात्वस्य च वंशस्य राज्ञां कश्चित्कदाचन}
{राजाऽऽसीद्यः प्रजापालः प्रजानामप्रियोऽभवत्}


\twolineshloka
{पितृवन्मातृवच्चैव भवन्तः पालयन्ति नः}
{न च दुर्योधनः किञ्चिदयुक्तं कृतवान्नृपः}


\threelineshloka
{`प्रियाणि कुर्वन्सर्वेषामनुवृत्त्यर्थमुद्यतः}
{'यथा ब्रवीति धर्मात्मा मुनिः सत्यवतीसुतः}
{तथा कुरु महाराज स हि नः परमो गुरुः}


\twolineshloka
{त्यक्ता वयं तु भवता दुःखशोकपरायणाः}
{भविष्यामश्चिरं राजन्भवद्गुणशतैर्हृताः}


\threelineshloka
{यथा शन्तनुना गुप्ता राज्ञा चित्राङ्गदेन च}
{भीष्मवीर्योपगूढेन पित्रा तव च पार्थिव}
{भवद्बुद्धियुजा चैवि पाण्डुना पृथिवीक्षिता}


\threelineshloka
{तथा दुर्योधनेनापि राज्ञा सुपरिपालिताः}
{न स्वल्पमपि पुत्रस्ते व्यलीकं कृतवान्नृप}
{}


\twolineshloka
{पितरीव सुविश्वस्तास्तस्मिन्नपि नराधिपे}
{वयसा स्म यथा सम्यग्भवतो विदितं तथा}


\twolineshloka
{तथा वर्षसहस्राणि कुन्तीपुत्रेण धीमता}
{पाल्यमाना धृतिमता सुखं विन्दामहे नृप}


\twolineshloka
{राजर्षीणां पुराणानां भवतां पुण्यकर्मणाम्}
{कुरुसंवरणादीनां भरतस्य च धीमतः}


\twolineshloka
{वृत्तं समनुयात्येष धर्मात्मा भूरिदक्षिणः}
{नात्र वाच्यं महाराज सुसूक्ष्ममपि विद्यते}


\twolineshloka
{उषिताः स्म सुखं नित्यं भवता परिपालिताः}
{सुसूक्ष्मं च व्यलीकं ते सपुत्रस्य न विद्यते}


\twolineshloka
{यत्तु ज्ञातिविमर्देऽस्मिन्नात्थ दुर्योधनं प्रति}
{भन्तमनुनेष्यामि तत्रापि कुरुनन्दन}


\twolineshloka
{न तद्दुर्योधनकृतं न च तद्भवता कृतम्}
{न कर्णसौबलाभ्यां च कुरवो यत्क्षयं गताः}


\twolineshloka
{दैवं तत्तु विजानीमो यन्न शक्यं प्रबाधितुम्}
{दैवं पुरुषकारेणि न शक्यमपि बाधितुम्}


\twolineshloka
{अक्षौहिण्यो महाराज दशाष्टौ च समागताः}
{अष्टादशाहेन हताः कुरुभिर्योधपुङ्गवैः}


\twolineshloka
{भीष्मद्रोणकृपाद्यैश्च कर्णेन च महात्मना}
{युयुधानेन वीरेण धृष्टद्युम्नेन चाहवे}


\twolineshloka
{चतुर्भिः पाण्डुपुत्रैश्च भीमार्जुनयमैस्तथा}
{न च क्षयोऽयं नृपते क्रते दैवबलादभूत्}


\twolineshloka
{अवश्यमेव सङ्ग्रामे क्षत्रियेण विशेषतः}
{कर्तव्यं निधनं काले मर्तव्यं क्षत्रबन्धुना}


\twolineshloka
{तैरियं पुरुषव्याघ्रैर्विद्याबाहुबलान्वितैः}
{पृथिवी निहता सर्वा सहया सरथद्विपा}


\twolineshloka
{न स राज्ञां वधे सूनुः कारणं ते महात्मनाम्}
{न भवान्न च ते भृत्या न कर्णो न च सौबलः}


\twolineshloka
{यद्विशस्ताः कुरुश्रेष्ठ राजानश्च सहस्रशः}
{सर्वं दैवकृतं विद्धि कोत्र किं वक्तुमर्हति}


\twolineshloka
{गुरुर्मतो भवानस्य कृत्स्नस्य जगतः प्रभुः}
{धर्मात्मानमतस्तुभ्यमनुजानीमहे सुतम्}


\threelineshloka
{लभतां वीरलोकं स ससहायो नराधिपः}
{द्विजाग्र्यैः समनुज्ञातस्त्रिदिवे मोदतां सुखम्}
{}


\twolineshloka
{प्राप्स्यते च भवान्पुण्यं धर्मे च सततं स्थितः}
{वेद धर्मं महाबाहो लौक्यं वैदिकमेव च}


\twolineshloka
{दृष्टापदानाश्चास्माभिः पाण्डवाः पुरुषर्षभाः}
{समर्थास्त्रिदिवस्यापि पालने किं पुनः क्षितेः}


\twolineshloka
{अनुवर्त्स्यन्ति वा धीमन्समेषु विषमेषु च}
{प्रजाः कुरुकुलश्रेष्ठ पाण्डवाञ्शीलभूषणान्}


\twolineshloka
{ब्रह्मदेयाग्रहारांश्च पारिबर्हांश्च पार्थिवः}
{पूर्वराजातिसर्गांश्च पालयत्येव पाण्डवः}


\twolineshloka
{दीर्घदर्शीं मृदुर्दान्तः सदा वैश्रवणो यथा}
{अक्षुद्रसचिवश्चायं कुन्तीपुत्रो महामनाः}


\twolineshloka
{अप्यमित्रे दयावांश्च शुचिश्च भरतर्षभः}
{ऋजु पश्यति मेधावी पुत्रवत्पाति नः सदा}


\twolineshloka
{विप्रियं च जनस्यास्य संसर्गाद्धर्मजस्य वै}
{न करिष्यन्ति राजर्षे तथा भीमार्जुनादयः}


\twolineshloka
{मन्दा मृदुषु कौरव्य तीक्ष्णेष्वाशीविषोपमाः}
{वीर्यवन्तो महात्मानः पौराणां च हिते रताः}


\twolineshloka
{न कुन्ती न च पाञ्चाली न चोलूपी न सात्वती}
{अस्मिञ्जने करिष्यन्ति प्रतिकूलानि कर्हिचित्}


\twolineshloka
{भवत्कृतमिमं स्नेहं युधिष्ठिरविवर्धितम्}
{न पृष्ठतः करिष्यन्ति पौरा जानपदा जनाः}


\twolineshloka
{अधर्मिष्ठानपि सतः कुन्तीपुत्रा महारथाः}
{मानवान्पालयिष्यन्ति भूत्वा धर्मपरायणाः}


\threelineshloka
{स राजन्मानसं दुःखमपनीय युधिष्ठिरात्}
{कुरु कार्याणि धर्म्याणि नमस्ते पुरुषर्षभ ॥वैशम्पायन उवाच}
{}


\twolineshloka
{तस्य तद्वचनं धर्म्यमनुमान्य गुणोत्तरम्}
{साधुसाध्विति सर्वः स जनः प्रतिगृहीतवान्}


\twolineshloka
{धृतराष्ट्राश्च तद्वाक्यमभिपूज्य पुनःपुनः}
{विसर्जयामास तदा प्रकृतीस्तु शनैःशनैः}


\twolineshloka
{स तैः सम्पूजितो राजा शिवेनावेक्षितस्तथा}
{प्राञ्जलिः पूजयामास तं जनं भरतर्षभ}


\twolineshloka
{ततो विवेश भवनं गान्धार्या सहितो निजम्}
{आगतायां च शर्वर्यां सुखं शेते नराधिपः}


\chapter{अध्यायः १२}
\twolineshloka
{ततो रजन्यां व्युष्टायां धृतराष्ट्रोंऽबिकासुतः}
{विदुरं प्रेषयामास युधिष्ठिरनिवेशनम्}


\twolineshloka
{स गत्वा राजवचनादुवाचाच्युतमीश्वरम्}
{युधिष्ठि महातेजाः सर्वबुद्धिमतांवरः}


\twolineshloka
{धृतराष्ट्रो महाराजो वनवासाय दीक्षितः}
{गमिष्यति वनं राजन्नागतां कार्तिकीमिमाम्}


\twolineshloka
{स त्वां कुरुकुलश्रेष्ठ किञ्चिर्थमभीप्सति}
{श्राद्धमिच्छति दातुं स गाङ्गेयस्य महात्मनः}


\threelineshloka
{द्रोणस्य सोमदत्तस्य बाह्लीकस्य च धीमतः}
{पुत्राणां चैव सर्वेषां ये चान्ये सुहृदो हताः}
{यदि चाप्यनुजानीषे सैन्धवापशदस्य च}


\twolineshloka
{एतच्छ्रुत्वा तु वचनं विदुरस्य युधिष्ठिरः}
{हृष्टःक सम्पूजयामास गुडोकेशश्च पाण्डवः}


\twolineshloka
{न च भीमो द्दढक्रोधस्तद्वचो जगृहे तदा}
{विदुरस्य महातेजा दुर्योधनकृतं स्मरन्}


\twolineshloka
{अभिप्रायं विदित्वा तु भीमतसेनस्य फल्गुनः}
{किरीटी किञ्चिदानम्य तमुवाच नरर्षभम्}


\twolineshloka
{भीम राजा पिता वृद्धो वनवासाय दीक्षितः}
{दातुमिच्छति सर्वेषां सुहृदामौर्ध्वदेहिकम्}


\twolineshloka
{भवता निर्जितं वित्तं दातुमिच्छति कौरवः}
{भीष्मादीनां महाबाहो तदनुज्ञातुमर्हसि}


\twolineshloka
{दिष्ट्या त्वद्य महाबाहो धृतराष्ट्रः प्रयाचते}
{याचितो यः पुराऽस्माभिः पश्य कालस्य पर्ययम्}


\twolineshloka
{योसौ पृथिव्याः कृत्स्नाया भर्ता भूत्वा नराधिपः}
{परैर्विनिहतामात्यो वनं गन्तुमभीप्सति}


\twolineshloka
{मा तेऽन्यत्पुरुषव्याघ्र दानाद्भवतु दर्शनम्}
{अयशस्यमतोऽन्यत्स्यादधर्मश्च महाभुजः}


\twolineshloka
{राजानमुपतिष्ठस्व ज्येष्ठं भ्रातरमीश्वरम्}
{अर्हस्त्वमसि दातुं वै नादातुं भरतर्षभ}


\twolineshloka
{एवं ब्रुवाणं बीभत्सुं धर्मराजोऽप्यपूजयत्}
{भीमसेनस्तु सक्रोधमुवाच विजयं तदा}


\twolineshloka
{वयं भीष्मस्य दास्यामः प्रेतकार्यं तु फल्गुन}
{सोमदत्तस्य नृपतेर्भूरिश्रवस एव च}


\twolineshloka
{बाह्लीकस्य च राजर्षेर्द्रोणस्य च महात्मनः}
{अन्येषां चैव सुहृदां कुन्ती कर्णाय दास्यति}


\twolineshloka
{श्राद्धानि पुरुषव्याघ्र मा प्रदात्कौरवो नृपः}
{इति मे वर्तते बुद्धिर्मा वो नन्दन्तु शत्रवः}


\twolineshloka
{कष्टात्कष्टतरं यान्तु सर्वे दुर्योधनादयः}
{यैरियं पृथिवी कृत्स्ना घातिता कुलपांसनैः}


\twolineshloka
{कुतस्त्वमद्य विस्मृत्य वैरं द्वादशवार्षिकम्}
{अज्ञातवासगमनं द्रौपदीशोकवर्धनम्}


\threelineshloka
{क्व तदा धृतराष्ट्रस्य स्नेहोऽस्मद्गोचरो गतः}
{कृष्णाजिनोपसंवीतो हृताभरणभूषणः}
{सार्धं पाञ्चालपुत्र्या त्वं राजानमुपजग्मिवान्}


\threelineshloka
{क्व तदा द्रोणभिष्मौ तौ सोमदत्तोपि वाऽभवत्}
{यत्र द्वादश वर्षाणि वने वन्येन जीवथ}
{न तदा त्वां पिता ज्येष्ठः पितृत्वेनाभिवीक्षते}


\twolineshloka
{किं ते तद्विस्मृतं पार्थक यदेष कुलपांसनः}
{दुर्बुद्धिर्विदुरं प्राह द्यूते किं जितमित्युत}


\twolineshloka
{तमेवंवादिनं राजा कुन्तीपुत्रो युधिष्ठिरः}
{उवाच वचनं धीमाञ्जोषमास्स्वेति भर्त्सयन्}


\chapter{अध्यायः १३}
\twolineshloka
{भीम ज्येष्ठो गुरुर्मे त्वं नातोऽन्यद्वमुक्तुमुत्सहे}
{धृताराष्ट्रस्तु राजर्षिः सर्वथा मानमर्हति}


\twolineshloka
{न स्मरन्त्यपराद्धानि स्मरन्ति सुकृतान्यपि}
{असंभिन्नार्थमर्यादाः साधवः पुरुषोत्तमाः}


\twolineshloka
{इति तस्य वचः श्रुत्वा फल्गुनस्य महात्मनः}
{विदुरं प्राह धर्मात्मा कुन्तीपुत्रो युधिष्ठिरः}


\twolineshloka
{इदं मद्वचनात्क्षत्तः कौरवं ब्रूहि पार्थिवम्}
{यावदिच्छति पुत्राणां दातुं तावद्ददाम्यहम्}


\threelineshloka
{भीष्मादीनां च सर्वेषां सुहृदामुपकारिणाम्}
{मम कोशादिति विभो मा भूद्भीमः सुदुर्मनाः ॥वैशम्पायन उवाच}
{}


\twolineshloka
{इत्युक्त्वा धर्मराजस्तमर्जुनं प्रत्यपूजयत्}
{भीमसेनः कटाक्षेण वीक्षांचक्रे धनंजयम्}


\twolineshloka
{ततः स विदुरं धीमान्वाक्यमाह युधिष्ठिरः}
{न भीमसेन कोपं स नृपतिः कर्तुमर्हति}


\twolineshloka
{परिक्लिष्टो हि भीमस्तु हिमवृष्ट्यातपादिभिः}
{दुःखैर्बहुविधैर्धीमानरण्ये विदितं तव}


% Check verse!
किंतु मद्वचनाद्ब्रूहि राजानं भरतर्षभम् ॥यद्यदिच्छसि यावच्च गृह्यतां मद्गृहादिति
\twolineshloka
{यन्मात्सर्यमयं भीमः करोति भृशदुःखितः}
{न तन्मनसि कर्तव्यमिति वाच्यः स पार्थिवः}


\twolineshloka
{यन्ममास्ति धनं किञ्चिदर्जुनस्य च वेश्मनि}
{तस्य स्वामी महाराज इति वाच्यः स पार्थिवः}


\twolineshloka
{ददातु राजा विप्रेभ्यो यथेष्टं क्रियतां व्ययः}
{पुत्राणां सुहृदां चैव गच्छत्वानृण्यमद्य सः}


\twolineshloka
{इदं चापि शरीरं मे तवायत्तं जनाधिप}
{धनानि चेति विद्धि त्वं क्षत्तर्नास्त्यत्र संशयः}


\chapter{अध्यायः १४}
\twolineshloka
{एवमुक्तस्तु राज्ञा स विदुरो बुद्धिसत्तमः}
{धृतराष्ट्रमुपेत्यैव वाक्यमाह महार्थवत्}


\twolineshloka
{उक्तो युधिष्ठिरो राजा भवद्वचनमादितः}
{स च संश्रुत्य वाक्यं ते प्रशशंस महाद्युतिः}


\twolineshloka
{बीभत्सुश्च महातेजा निवेदयति ते गृहान्}
{वसु तस्य गृहे यच्च प्राणानपि च केवलान्}


\twolineshloka
{धर्मराजश्च पुत्रस्ते राज्यं प्राणान्धनानि च}
{अनुजानाति राजर्षे यच्चान्यदपि किञ्चन}


\twolineshloka
{भीमस्तु सर्वदुःखानि संस्मृत्य बहुलान्युत}
{कृच्छ्रादिव महाबाहुरनुजज्ञे विनिःश्वसन्}


\twolineshloka
{र राजन्धर्मशीलेन राज्ञा बीभत्सुना तथा}
{अनुनीतो महाबाहुः सौहृदे स्थापितोपि च}


\twolineshloka
{न च मन्युस्त्वया कार्य इति त्वां प्राह धर्मराट्}
{संस्मृत्य भीमस्तद्वैरं यदन्यायवदाचरत्}


\twolineshloka
{एवंप्रायो हि धर्मोऽयं क्षत्रियाणां नराधिप}
{शुद्धे क्षत्रियधर्मे न निरतोऽयं वृकोदरः}


\twolineshloka
{वृकोदरकृते चाहमर्जुनश्च पुनः पुनः}
{प्रसीद याचे नृपते भवान्प्रभुरिहास्ति यत्}


\twolineshloka
{तद्ददातु भवान्वित्तं यावदिच्छसि पार्थिवः}
{त्वमीश्वरो नो राज्यस्य प्राणानामपि भारत}


\twolineshloka
{ब्रह्मदेयाग्रहारांश्च पुत्राणामौर्ध्वदेहिकम्}
{इतो रत्नानि गाश्चैव दासीदासमजाविकम्}


\twolineshloka
{अर्चयित्वा कुरुश्रेष्ठो ब्राह्मणेभ्यः प्रयच्छतु}
{दीनान्धकृपणेभ्यश्च तत्रतत्र नृपाज्ञया}


\twolineshloka
{बह्वन्नरसपानाढ्याः सभा विदुर कारय}
{गवां निपानान्यन्यच्च विविधं पुण्यकं कुरु}


\twolineshloka
{इति मामब्रवीद्राजा पार्थश्चैव धनंजयः}
{यदत्रानन्तरं कार्यं तद्भवम्ववक्तुमर्हति}


\twolineshloka
{इत्युक्ते विदुरेणाथ धृतराष्ट्रोऽभिनन्द्य तान्}
{मनश्चक्रे महादाने कार्तिक्यां जनमेजय}


\chapter{अध्यायः १५}
\twolineshloka
{विदुरेणैवमुक्तस्तु धृतराष्ट्रो जनाधिपः}
{प्रीतिमानभवद्राजन्राज्ञो जिष्णोश्च कर्मणा}


\twolineshloka
{ततोऽभिरूपान्भीष्माय ब्राह्मणानृषिसत्तमान्}
{पुत्रार्थे सुहृदां चैव स समीक्ष्य सहस्रशः}


\twolineshloka
{कारयित्वाऽन्नपानानि यानान्याच्छादनानि च}
{सुवर्णमणिरत्नानि दासीदासपरिच्छदान्}


\twolineshloka
{कंबलानि च रत्नानि ग्रामान्क्षेत्रं तथा धनम्}
{सालङ्कारान्गजानश्वान्कन्याश्चैव वरस्त्रियः}


\twolineshloka
{आदिश्यादिश्य सर्वेभ्यो ददौ स नृपसत्तमः}
{द्रोणं संकीर्त्य भीष्मं च सोमदत्तं च बाह्लिकम्}


\twolineshloka
{दुर्योधनं च राजानं पुत्रांश्चैव पृथक्पृथक्}
{जयद्रथपुरोगांश्च सुहृदश्चापि सर्वशः}


\twolineshloka
{स श्राद्धयज्ञो ववृते बहुगोधनदक्षिणः}
{अनेकधनरत्नौघो युधिष्ठिरमते तदा}


\twolineshloka
{अनिशं यत्र पुरुषा गणका लेखकास्तदा}
{युधिष्ठिरस्य वचनादपृच्छन्त स्म तं नृपम्}


\twolineshloka
{आज्ञापय किमेतेभ्यः प्रदेयं दीयतामिति}
{तदुपस्थितमेवात्र वचनान्ते ददुस्तदा}


\twolineshloka
{शते देये दशशतं सहस्रं चायुतं तथा}
{दीयते वचनाद्राज्ञः कुन्तीपुत्रस्य धीमतः}


\twolineshloka
{एवं स वसुधाराभिर्वर्षमाणो नृपांबुदः}
{तप्रयामास विप्रांस्तान्वर्षन्भूमिमिवांबुदः}


\twolineshloka
{ततोऽन्तन्तरमेवात्र सर्ववर्णान्महामते}
{अन्नपानरसौघेण प्लावयामास पार्थिवः}


\twolineshloka
{सवस्त्रधनरत्नौघो मृदङ्गनिनदस्वनः}
{गवाश्वमकरावर्तो नानारत्नमहाकरः}


\twolineshloka
{ग्रामाग्रहारद्वीपाढ्यो मणिहेमजलार्णवः}
{जगत्संप्लावयामास धृतराष्ट्रोडुपोद्धतः}


\twolineshloka
{एवं स पुत्रपौत्राणां पितॄणामात्मनस्तथा}
{गान्धार्याश्च महाराज प्रददावौर्द्वदेहिकम्}


\twolineshloka
{परिश्रान्तो यदासीत्स ददद्दानान्यनेकशः}
{निवर्तयामास तदा दानयज्ञं नराधिपः}


\twolineshloka
{एवं स राजा कौरव्यश्चक्रे दानमहाक्रतुम्}
{नटनर्तकलास्याढ्यं बह्वन्नरसदक्षिणम्}


\twolineshloka
{दशाहमेवं दानानि दत्त्वा राजांऽबिकासुतः}
{बभूव पुत्रपौत्राणामनृणो भरतर्षभ}


\chapter{अध्यायः १६}
\twolineshloka
{ततः प्रभाते राजा स धृतराष्ट्रोंऽबिकासुतः}
{आनाय्य पाण्डवान्वीरान्वनषासे कृतक्षणः}


\twolineshloka
{गान्धारीसहितो धीमानभ्यनन्दद्यथाविधि}
{कार्तिक्यां कारयित्वेष्टिं ब्राह्मणैर्वेदपारगैः}


\twolineshloka
{अग्निहोत्रं पुरस्कृत्य वल्कलाजिनसंवृतः}
{वधूजनवृतो राजा निर्ययौ भवनात्ततः}


\twolineshloka
{ततः स्त्रियः कौरवपाण्डवानांयाञ्चापराः कौरवराजवंश्याः}
{तासां नादः प्रादुरासीत्तदानींवैचित्रिवीर्ये नृपतौ प्रयाते}


\twolineshloka
{ततो लाजैः सुमनोभिश्च राजाविचित्राभिस्तद्गृहं पूजयित्वा}
{संयोज्याश्वैर्भृत्यवर्गं च सर्वंततः समुत्सृज्य ययौ नरेन्द्रः}


\twolineshloka
{ततो राजा प्राञ्जलिर्वेपमानोयुधिष्ठिरः सस्वरं बाष्पकण्ठः}
{विनद्योच्चैर्गां महाराज साधोक्व यास्यसीत्यपतत्ताति भूमौ}


\twolineshloka
{तथाऽर्जुनस्तीव्रदुःखाभितप्तोमुहुर्मुहुर्निः श्वसन्भारताग्र्यः}
{युधिष्ठिरं मैवमित्येवमुक्त्वानिगृह्याथो दीनतरो बभूव}


\twolineshloka
{वृकोदरः फल्गुनश्चैव वीरौमाद्रीपुत्रौ विदुरः संजयश्च}
{वैश्यापुत्रः सहितो गौतमेनधौम्यो विप्राश्चान्वयुर्बाष्पकण्ठाः}


\twolineshloka
{कुन्ती गान्धारीं बद्धनेत्रां व्रजन्तींस्कन्धासक्तं हस्तमथोद्वहन्ती}
{राजा गान्धार्याः स्कन्धदेशेऽवसज्यपाणिं ययौ धृतराष्ट्रः प्रतीतः}


\twolineshloka
{तथा कृष्णा द्रौपदी यादवी चबालापत्या चोत्तरा कौरवी च}
{चित्राङ्गदा याश्च काश्चित्स्त्रियोऽन्याःसार्धं राज्ञा प्रस्थितास्ता वधूभिः}


\twolineshloka
{तासां नादो रुदतीनां तदाऽऽसी-द्राजन्दुःखात्कुररीणामिवोच्चैः}
{ततो निष्पेतुर्ब्राह्मणक्षत्रियाणांविट्शूद्राणां चैव भार्याः समन्तात्}


\twolineshloka
{तन्निर्याणे दुःखितः पौरवर्गोगजाह्वये चैव बभूव राजन्}
{यथापूर्वं गच्छतां पाण्डवानांद्यूते राजन्कौरवाणां सभायाम्}


\twolineshloka
{या नापश्यच्चन्द्रमा नैव सूर्यो रामाः काश्चित्ताः स्म तस्मिन्नरेन्द्रे}
{महावनं गच्छति कौरवेन्द्रेशोकेनार्ता राजमार्गं प्रपेदुः}


\chapter{अध्यायः १७}
\twolineshloka
{वैशम्पायन उवाच}
{}


\twolineshloka
{ततः प्रासादहर्म्येषु वसुधायां च पार्थिव}
{नारीणां च नराणां च निःस्वनः सुमहानभूत्}


\twolineshloka
{स राजा राजमार्गेण नृनारीसंकुलेन च}
{कथञ्चिन्निर्ययौ धीमान्वेपमानः कृताञ्जलिः}


\twolineshloka
{स वर्धमानद्वारेणि निर्ययौ गजसाह्वयात्}
{विसर्जयामास च तं जनौघं स मुहुर्मुहुः}


\twolineshloka
{वनं गन्तुं च विदुरो राज्ञा सह कृतक्षणः}
{संजयश्च महामात्रः सूतो गावल्गणिस्तथा}


\twolineshloka
{ककृपं निवर्तयामास युयुत्सुं च महारथम्}
{धृताराष्ट्रो महीपालः परिदाप्य युधिष्ठिरे}


\twolineshloka
{निवृत्ते पौरवर्गे च राजा सान्तःपुरस्तदा}
{धृतराष्ट्राभ्यनुज्ञातो निवर्तितुमियेष ह}


% Check verse!
सोब्रवीन्मातरं कुन्तीमुपेत्य भरतर्षभ ॥अहं राजानमन्विष्ये भवती विनिवर्तताम्
\twolineshloka
{वधूपरिवृता राज्ञि नगरं गन्तुमर्हसि}
{राजा यात्वेष धर्मात्मा तपसे कृतनिश्चयः}


\threelineshloka
{इत्युक्ता धर्मराजेनि बाष्पव्याकुललोचना}
{जगामैव तदा कुन्ती गान्धारीं परिगृह्य ह ॥कुन्त्युवाच}
{}


\twolineshloka
{सहदेवे महाराज माऽप्रसादं कृथाः क्वचित्}
{एष मामनुरक्तो हि राजंस्त्वां चैव सर्वदा}


\twolineshloka
{कर्णं स्मरेथाः सततं सङ्ग्रामेष्वपलायिनम्}
{अवकीर्णो हि स मया वीरो दुष्प्रज्ञया तदा}


\twolineshloka
{आयसं हृदयं नूनं मन्दाया मम पुत्रक}
{यत्सूर्यजमपश्यन्त्याः शतधा न विदीर्यते}


\twolineshloka
{एवं गते तु किं शक्यं मया कर्तुमरिंदम}
{मम दोषोऽयमत्यर्थं ख्यापितो यन्न सूर्यजः}


\twolineshloka
{तन्निमित्तं महाबाहो दानं दद्यास्त्वमुत्तमम्}
{सदैव भ्रातृभिः सार्धं सूर्यजस्यारिमर्दन}


\twolineshloka
{द्रौपद्याश्च प्रिये नित्यं स्थातव्यमरिकर्शन}
{भीमसेनोऽर्जुनश्चैव नकुलश्च कुरूद्वह}


\fourlineindentedshloka
{समाधेयास्त्वया राजंस्त्वय्यद्य कुलधूर्गता}
{श्वश्रूश्वशुरयोः पादाञ्शुश्रूषन्ती वने त्वहम्}
{गान्धारीसहिता वत्स्ये तापसी मलपङ्किनी ॥वैशम्पायन उवाच}
{}


\twolineshloka
{एवमुक्तः स धर्मात्मा भ्रातृभिः सहितो वशी}
{विषादमगमद्धीमान्न च किञ्चिदुवाच ह}


\twolineshloka
{मुहूर्तमिव तु ध्यात्वा धर्मराजो युधिष्ठिरः}
{उवाच मतारं दीनश्चिन्ताशोकपरायणः}


\twolineshloka
{किमिदं ते व्यवसितं नैवं त्वं वक्तुमर्हसि}
{न त्वामभ्यनुजानामि प्रसादं कर्तुमर्हसि}


\twolineshloka
{व्यचोदयः पुराऽस्माकमुत्साहं शुभदर्शने}
{विदुलाया वचोभिस्त्वं नास्मान्संत्यक्तुमर्हसि}


\twolineshloka
{निहत्य पृथिवीपालान्राज्यं प्राप्तमिदं मया}
{तव प्रज्ञामुपश्रुत्य वासुदेवान्नरर्षभात्}


\twolineshloka
{क्व सा बुद्धिरियं चाद्य भवत्या या श्रुता मया}
{क्षत्रधर्मे स्थितं त्यक्त्वा न प्रयातुमिहार्हसि}


\twolineshloka
{अस्मानुत्सृज्य राज्यं च स्नुषाहीना यशस्विनि}
{कथं वत्स्यसि दुर्गेषु वनेष्वद्य प्रसीद मे}


\twolineshloka
{इति बाष्पकला वाचः कुन्ती पुत्रस्य शृण्वती}
{जगामैवाश्रुपूर्णाक्षी भीमस्तामिदमब्रवीत्}


\twolineshloka
{यदा राज्यमिदं कुन्ति भोक्तव्यं पुत्रनिर्जितम्}
{प्राप्तव्या राजधर्माश्च तदेयं ते कुतो मतिः}


\twolineshloka
{किं वयं कारिताः पूर्वं भवत्या पृथिवीक्षयम्}
{कस्य हेतोः परित्यज्य वनं गन्तुमभीप्ससि}


\twolineshloka
{वनाच्चापि किमानीता भवत्या बालका वयम्}
{दुःखशोकसमाविष्टौ माद्रीपुत्राविमौ तथा}


\twolineshloka
{प्रसीद मातर्मा गास्त्वं वनमद्य यशस्विनि}
{श्रियं यौधिष्ठिरीं मातर्भुङ्क्ष्व पार्थबलार्जिताम्}


\twolineshloka
{इति सा निश्चितैवाशु वनवासकृततक्षणा}
{लालप्यतां बहुविधं पुत्राणां नाकरोद्वचः}


\twolineshloka
{द्रौपदी चान्वयाच्छ्वश्रूं विषष्णवदना तदा}
{वनवासाय गच्छन्तीं रुदती भद्रया सह}


\twolineshloka
{सा पुत्रान्रुदतः सर्वान्मुहुर्मुहुरवेक्षती}
{जगामैव महाप्राज्ञा वनाय कृतनिश्चया}


\twolineshloka
{अन्वयुः पाण्डवास्तां तु सभृत्यान्तःपुरास्तथा}
{ततः प्रमृज्य साऽश्रूणि पुत्रान्वचनमब्रवीत्}


\chapter{अध्यायः १८}
\twolineshloka
{एवमेतन्महाबाहो यथा वदसि पाण्डवः}
{कृतमुद्धरणं पूर्वं मया वः सीदतां नृपाः}


\twolineshloka
{द्यूतापहृतराज्यानां पतितानां सुखादपि}
{ज्ञातिभिः परिभूतानां कृतमुद्धरणं मया}


\twolineshloka
{कथं पाण्डोर्न नश्येत संततिः पुरुषर्षभाः}
{यशश्च वो न नश्येत इति चोद्धरणं कृतम्}


\twolineshloka
{यूयमिन्द्रसमाः सर्वे देवतुल्यपराक्रमाः}
{मा परेषां मुखप्रेक्षाः स्थेत्येवं तत्कृतं मया}


\twolineshloka
{कथं धर्मभृतां श्रेष्ठो राजा त्वं वासवोपमः}
{पुनर्वने न दुःखी स्या इति चोद्धरणं कृतम्}


\twolineshloka
{नागायुतसमप्राणः ख्यातविक्रमपौरुषः}
{नायं भीमोत्ययं गच्छेदिति चोद्धरणं कृतम्}


\twolineshloka
{भीमसेनादवरजस्तथाऽयं वासवोपमः}
{विजयो नावसीदेत इति चोद्धरणं कृतम्}


\twolineshloka
{नकुलः सहदेवश्च तथेमौ गुरुवर्तिनौ}
{क्षुधा कथं न सीदेतामिति चोद्धरणं कृतम्}


\twolineshloka
{इयं च बृहती श्यामा तथाऽत्यायतलोचना}
{वृथा सभातले क्लिष्टा मा भूदिति च तत्कृतं}


\twolineshloka
{अपेक्षतामेव वो भीम वेपन्तीं कदलीमिव}
{स्त्रीधर्मिणीमरिष्टाङ्गीं तथा द्यूतपराजिताम्}


\twolineshloka
{दुःशासनो यदा मौर्ख्याद्दासीवत्पर्यकर्षत}
{तदैव विदितं मह्यं पराभूतमिदं कुलम्}


\twolineshloka
{निषण्णाः कुरवश्चैव तदा मे श्वशुरादयः}
{सा दैवं नाथमिच्छन्ती व्यलपत्कुररी यथा}


\twolineshloka
{केशपक्षे परामृष्टा पापेन् हतबुद्धिना}
{यदा दुःशासनेनैषा तदा मुह्याम्यहं नृपाः}


\twolineshloka
{युष्मत्तेजोविवृद्ध्यर्तं मया ह्युद्धरणं कृतम्}
{तदानीं विदुलावाक्यैरिति तद्वित्त पुत्रकाः}


\twolineshloka
{कथं न राजवंशोऽयं नश्येत्प्राप्य सुतान्मम}
{पाण्डोरिति मया पुत्रस्तस्मादुद्धरणं कृतम्}


\twolineshloka
{न तस्य पुत्राः पौत्रा वा क्षतवंशस्य पार्थिव}
{लभ्ते सुकृताँल्लोकान्यस्माद्वंशः प्रणश्यति}


\twolineshloka
{भुक्तं राज्यफलं पुत्रा भर्तुर्मे विपुलं पुरा}
{महादानानि दत्तानि पीतः सोमो यथाविधि}


\twolineshloka
{नाहमात्मफलार्थं वै वासुदेवमचूचुदम्}
{विदुलायाः प्रलापैस्तैः पालनार्थं च तत्कृतम्}


\twolineshloka
{नाहं राज्यफलं पुत्राः कामये पुत्रनिर्जितम्}
{पतिलोकानहं पुण्यान्कामये तपसा पिभो}


\twolineshloka
{श्वश्रूश्वशुरयोः कृत्वा शुश्रूषां वनवासिनोः}
{तपसा शोषयिष्यामि युधिष्ठिर कलेवरम्}


\twolineshloka
{निवर्तस्व कुरुश्रेष्ठ भीमसेनादिभिः सह}
{धर्मे ते धीयतां बुद्धिर्मनस्तु महदस्तु च}


\chapter{अध्यायः १९}
\twolineshloka
{कुन्त्यास्तु वचनं श्रुत्वा पाण्डवा राजसत्तम}
{व्रीडिताः संन्यवर्तन्त पाञ्चाल्या सह भारत}


\twolineshloka
{ततः शब्दो महानेव सर्वेषामभवत्तदा}
{अन्तःपुराणां रुदतां दृष्ट्वा कुन्तीं तथा गताम्}


\twolineshloka
{प्रदक्षिणमथावृत्त्य राजानं पाण्डवास्तदा}
{अभिवाद्य न्यवर्तन्त पृथां तामनिवर्त्य वै}


\twolineshloka
{ततोऽब्रवीन्महातेजा धृतराष्ट्रोंऽबिकासुतः}
{गान्धारीं विदुरं चैव समाभाष्यावगृह्य च}


\twolineshloka
{युधिष्ठिरस्य जननी देवी साधु निवर्त्यताम्}
{यथा युधिष्ठिरः प्राह तत्सर्वं सत्यमेव हि}


\twolineshloka
{पुत्रैश्वर्यं महदिदमपास्य च महाफलम्}
{काऽनुगच्छेद्वनं दुर्गं पुत्रानुत्सृज्य मूढवत्}


\twolineshloka
{राज्यस्थया तपस्तप्तुं कर्तुं दानव्रतं महत्}
{अनया शक्यमेवाद्य श्रूयतां च वचो मम}


\twolineshloka
{गान्धारि परितुष्टोस्मि वध्वाः शुश्रूषणेन वै}
{तस्मात्त्वमेनां धर्मज्ञे समनुज्ञातुमर्हसि}


\twolineshloka
{इत्युक्ता सौबलेयी तु राज्ञा कुन्तीमुवाच ह}
{तत्सर्वं राजवचनं स्वं च वाक्यं विशेषवत्}


\twolineshloka
{न च सा वनवासाय देवी कृतमतिं तदा}
{शक्नोत्युपावर्तयितुं कुन्तीं धर्मपरां सतीम्}


\twolineshloka
{तस्यास्तां तु स्थितिं ज्ञात्वा व्यवसायं कुरुस्त्रियः}
{निवृत्तांश्च कुरुश्रेष्ठान्दृष्ट्वा प्ररुरुदुस्तदा}


\twolineshloka
{उपावृत्तेषु पार्थेषु सर्वास्वेव वधूषु च}
{ययौ राजा महाप्राज्ञो धृतराष्ट्रो वनं तदा}


\twolineshloka
{पाण्डवाश्चातिदीनास्ते दुःखशोकपरायणाः}
{यानैः स्त्रीसहिताः सर्वे पुरं प्रविविशुस्तदा}


\twolineshloka
{तदहृष्टमनानन्दं गतोत्सवमिवाभवत्}
{नगरं हास्तिनपुरं सस्त्रीवृद्धकुमारकम्}


\twolineshloka
{सर्वे चासन्निरुत्साहाः पाण्डवा जातमन्यवः}
{कुन्त्या हीनाः सुदुःखार्ता वत्सा इव विनाकृताः}


\twolineshloka
{धृतराष्ट्रस्तु तेनाह्ना गत्वा सुमहदन्तरम्}
{ततो भागीरथीतीरे निवासमकरोत्प्रभुः}


\threelineshloka
{प्रादुष्कृता यथान्यायमग्नयो वेदपारगैः}
{व्यराजन्त द्विजश्रेष्ठैस्तत्रतत्र तपोवने}
{प्रादुष्कृताग्निरभवत्स च वृद्धो नराधिपः}


\twolineshloka
{स राजाऽग्नीन्पर्युपास्य हुत्वा च विधिवत्तदा}
{सन्ध्यागतं सहस्रांशुमुपातिष्ठत भारत}


\twolineshloka
{विदुरः संजयश्चैव राज्ञः शय्यां कुशैस्ततः}
{चक्रतुः कुरुवीरस्य गान्धार्याश्चाविदूरतः}


\twolineshloka
{गान्धार्याः सन्निकर्षे तु निषसाद कुशे सुखम्}
{युधिष्ठिरस्य जननी कुन्ती साधुव्रते स्थिता}


\twolineshloka
{तेषां संश्रवणे चापि निषेदुर्विदुरादयः}
{याजकाश्च यथोद्देशं द्विजा ये चानुयायिनः}


\twolineshloka
{प्राधीतद्विजमुख्या सा संप्रज्वलितपावका}
{बभूव तेषां रजनी ब्राह्मीव प्रीतिवर्धिनी}


\threelineshloka
{ततो रात्र्यां व्यतीतायां कृतपूर्वाह्णिकक्रियाः}
{हुत्वाऽग्निं विधिवत्सर्वे प्रययुस्ते यथाक्रमम्}
{उदङ्मुखा निरीक्षन्त उपवासपरायणाः}


\twolineshloka
{स तेषामतिदुःखोऽभून्निवासः प्रथमेऽहनि}
{शोचतां वदतां चापि पौरजानपदैर्जनैः}


\chapter{अध्यायः २०}
\twolineshloka
{ततो भागीरथीतीरे मेध्ये पुण्यजनोचिते}
{निवासमकरोद्राजा विदुरस्य मते स्थितः}


\twolineshloka
{तत्रैनं पर्युपातिष्ठन्ब्राह्मणा वनवासिनः}
{क्षत्रविट्शूद्रसङ्खाश्च बहवो भरतर्षभ}


\twolineshloka
{स तैः परिवृतो राजा कथाभिः परिनन्द्य तान्}
{अनुजज्ञे स शिष्यान्वै विधिवत्प्रतिपूज्य च}


\twolineshloka
{सायाह्ने स महीपालस्ततो गङ्गामुपेत्य च}
{चकार विधिवच्छौचं गान्धारी च यशस्विनी}


\twolineshloka
{ते चैवान्ये पृथक् सर्वे तीर्थेष्वाप्लुत्य भारत}
{चक्रुः सर्वाः क्रियास्तत्र पुरुषा विदुरादयः}


\twolineshloka
{कृतशौचं ततो वृद्धं श्वशुरं कुन्तिभोजजा}
{गान्धारीं च पृथा राजन्गङ्गातीर्थमुपानयत्}


\twolineshloka
{राज्ञस्तु याजकैस्तत्र कृतो वेदीपरिस्तरः}
{जुहाव तत्र वह्निं स नृपतिः सत्यसङ्गरः}


\twolineshloka
{ततो भागीरथीतीरात्कुरुक्षेत्रं जगाम सः}
{सानुगो नृपतिर्वृद्धो नियतः संयतेन्द्रियः}


\twolineshloka
{तत्राश्रमपदं धीमानभिगम्य स पार्थिवः}
{आससादाथ राजर्षिं शतयूपं मनीषिणम्}


\twolineshloka
{स हि राजा महानासीत्केकयेषु परंपतः}
{स्वपुत्रं मनुजैश्वर्ये निवेश्य वनमाविशत्}


\twolineshloka
{तेनासौ सहितो राजा ययौ व्यासाश्रमं प्रति}
{तत्रैनं विधिवद्राजन्प्रत्यगृह्णात्कुरूद्वहम्}


\twolineshloka
{स दीक्षां तत्र संप्राप्य राजा कौरवनन्दनः}
{शतयूपाश्रमे तस्मिन्निवासमकरोत्तदा}


\twolineshloka
{तस्मै सर्वं विधिं राज्ञे राजाऽऽचख्यौ महामतिः}
{आरण्यकं महाराज व्यासस्यानुमते तदा}


\twolineshloka
{एवं च तपसा राजन्धृतराष्ट्रो महामनाः}
{योजयामास चात्मानं तांश्चाप्यनुचरांस्तदा}


\twolineshloka
{तथैव देवी गान्धारी वल्कलाजिनधारिणी}
{कुन्त्या सह महाराजि समानव्रतचारिणी}


\twolineshloka
{कर्मणा मनसा वाचा चक्षुषा चैव ते नृप}
{सन्नियम्येन्द्रियग्राममास्थिताः परमं तपः}


\twolineshloka
{त्वगस्थिभूतः परिशुष्कमांसोजटाजिनी वल्कलसंवृताङ्गः}
{स पार्तिवस्तत्र तपश्चचारमहर्षिवत्तीव्रमपेतमोहः}


\twolineshloka
{क्षत्ता च धर्मार्थविदग्र्यबुद्धिःससंजयस्तं नृपतिं सदारम्}
{उपाचरद्धोरतपोर्जितात्मातदा कृशो वल्कलचीरवासाः}


\chapter{अध्यायः २१}
\twolineshloka
{ततस्तत्र मुनिश्रेष्ठा राजानं द्रष्टुमभ्ययुः}
{नारदः पर्वतश्चैव देवलश्च महातपाः}


\twolineshloka
{द्वैपायनः सशिष्यश्च सिद्धाश्चान्ये मनीषिणः}
{शतयूपश्च राजर्षिर्वृद्धः परमधार्मिकः}


\twolineshloka
{तेषां कुन्ती महाराज पूजां चक्रे यथाविधि}
{ते चापि तुतुषुस्तस्यास्तापसाः परिचर्यया}


\twolineshloka
{तत्र धर्म्याः कथास्तात चक्रुस्ते परमर्षयः}
{रमयन्तो महात्मानं धृतराष्ट्रं जनाधिपम्}


\threelineshloka
{कथान्तरे तु कस्मिंश्चिद्देवर्षिर्नारदस्ततः}
{कथामिमामकथयत्सर्वप्रत्यक्षदर्शिवान् ॥नारद उवाच}
{}


\twolineshloka
{पुरा प्रजापतिसमो राजाऽऽसीदकुतोभयः}
{सहस्रचित्यि इत्युक्तः शतयूपपितामहः}


\twolineshloka
{स पुत्रे राज्यमासज्ज्यि ज्येष्ठे परमधार्मिके}
{सहस्रचित्यो धर्मात्मा प्रविवेश वनं नृपः}


\twolineshloka
{स गत्वा तपसः पारं दीप्तस्य वसुधाधिपः}
{पुरंदरस्य संस्थानं प्रतिपेदे महाद्युतिः}


\twolineshloka
{दृष्टपूर्वः स बहुशो राजन्सम्पतता मया}
{महेन्द्रसदने राजा तपसा दग्धकिल्बिषः}


\twolineshloka
{तथा शैलालयो राजा भगत्तपितामहः}
{तपोबलेनैव नृपो महेन्द्रसदनं गतः}


\twolineshloka
{तथा पृषध्रो राजाऽऽसीद्राजन्वज्रधरोपमः}
{स चापि तपसा लेभे नाकपृष्ठमितो गतः}


\twolineshloka
{अस्मिन्नरण्ये नृपते मांधातुरपि चात्मजः}
{पुरुकुत्सो नृपः सिद्धिं महतीं समवाप्तवान्}


\twolineshloka
{भार्या समभवद्यस्य नर्मदा सरितां वरा}
{सोस्मिन्नरण्ये नृपतिस्तपस्तप्त्वा दिवं गतः}


\twolineshloka
{शशलोमा च राजाऽऽसीद्राजन्परमधार्मिकः}
{सम्यगस्मिन्वने तप्त्वा ततो दिवमवाप्तवान्}


\twolineshloka
{द्वैपायनप्रसादाच्च त्वमपीदं तपोवनम्}
{राजन्नवाप्य दुष्प्रापां सिद्धिमग्र्यां गमिष्यसि}


\twolineshloka
{त्वं चापि राजशार्दूल तपसोन्ते श्रिया वृतः}
{गान्धारीसहितो गन्ता गतिं तेषां महात्मनां}


\twolineshloka
{पाण्डुः स्मरति ते नित्यं बलहन्तुः समीपगः}
{त्वां सदैव महाराज श्रेयसा स च योक्ष्यति}


\twolineshloka
{तव शुश्रूषया चैव गान्धार्याश्च यशस्विनी}
{भर्तुः सलोकतां कुन्ती गमिष्यति वधूस्तव}


\twolineshloka
{युधिष्ठिरस्य जननी स हि धर्मः सनातनः}
{वयमेतत्प्रपश्यामो नृपते दिव्यवक्षुषाः}


\threelineshloka
{प्रवेक्ष्यति महात्मानं विदुरश्च युधिष्ठिरम्}
{संजयस्तदनुध्यानादितः स्वर्गमवाप्स्यति ॥वैशम्पायन उवाच}
{}


\twolineshloka
{एतच्छ्रुत्वा कौरवेन्द्रो महात्मासार्धं पत्न्या प्रीतिमान्सम्बभूव}
{विद्वान्वाक्यं नारदस्य प्रशस्यचक्रे पूजां चातुलां नारदाय}


\twolineshloka
{ततः सर्वे नारदं विप्रसङ्घाःसम्पूजयामासुरतीव राजन्}
{राज्ञः प्रीत्या धृतराष्ट्रस्य ते वैपुनःपुनः सम्प्रहृष्टास्तदानीम्}


\chapter{अध्यायः २२}
\twolineshloka
{नारदस्य तु तद्वाक्यं शशंसुर्द्विजसत्तमाः}
{शतयूपस्तु राजर्षिर्नारदं वाक्यमब्रवीत्}


\twolineshloka
{अहो भगवता श्रद्धा कुरुराजस्य वर्धिता}
{सर्वस्य च जनस्यास्य मम चैव महाद्युते}


\twolineshloka
{अस्ति काचिद्विवक्षा तु तां मे निगदतः शृणु}
{धृतराष्ट्रं प्रति नृपं देवर्षे लोकपूजित}


\twolineshloka
{सर्ववृत्तान्ततत्त्वज्ञो भवान्दिव्येन चक्षुषा}
{युक्तः पश्यसि विप्रर्षे गतयो विविधा नृणाम्}


\twolineshloka
{उक्तवान्नृपतीनां त्वं महेन्द्रस्य सलोकताम्}
{न त्वस्य नृपतेर्लोकाः कथितास्ते महामुने}


\twolineshloka
{स्थानमप्यस्य नृपतेः श्रोतुमिच्छाम्यहं विभो}
{त्वत्तः कीदृक्कदा चेति तन्ममाख्याहि तत्त्वतः}


\threelineshloka
{इत्युक्तो नारदस्तेन वाक्यं सर्वमनोनुगम्}
{व्याजहार सभामध्ये दिव्यदर्शी महातपाः ॥नारद उवाच}
{}


\twolineshloka
{यदृच्छया शक्रसदो गत्वा शक्रं शचीपतिम्}
{दृष्टवानस्मि राजर्षे तत्र पाण्डुं नराधिपम्}


\twolineshloka
{तत्रेयं धृतराष्ट्रस्य कथा समभवन्नप}
{तपसो दुष्करस्यास्य यदयं तपते नृपः}


\twolineshloka
{तत्राहमिदमश्रौषं शक्रस्य वदतः स्वयम्}
{वर्षाणि त्रीणि शिष्टानि राज्ञोस्य परमायुषः}


\twolineshloka
{ततः कुबेरभवनं गान्धारीसहितो नृपः}
{प्रयाता धृतराष्ट्रोऽयं राजराजाभिसत्कृतः}


\twolineshloka
{कामगेन विमानेन दिव्याभरणभूषितः}
{ऋषिपुत्रो महाभागस्तपसा दग्धकिल्बिषः}


\twolineshloka
{सञ्चरिष्यति लोकांश्च देवगन्धर्वरक्षसाम्}
{स्वच्छन्देनेति धर्मात्मा व्यासस्य तु तपोबलात्}


\threelineshloka
{देवगुह्यमिदं प्रीत्या मया वः कथितं महत्}
{भवन्तो हि श्रुतधनास्तपसा दग्धकिल्बिषाः ॥वैशम्पायन उवाच}
{}


\twolineshloka
{इति ते तस्य तच्छ्रुत्वा देवर्षेर्मधुरं वचः}
{सर्वे सुमनसः प्रीता बभूवुः स च पार्थिवः}


\twolineshloka
{एवं कथाभिरन्वास्य धृतराष्ट्रं मनीषिणः}
{विप्रजग्मुर्यथाकामं ते सिद्धगतिमास्थिताः}


\chapter{अध्यायः २३}
\twolineshloka
{वनं गते कौरवेन्द्रे दुःखशोकसमन्विताः}
{बभूवुः पाण्डवा राजन्मातृशोकेनि चान्विताः}


\twolineshloka
{तथा पौरजनः सर्वः शोचन्नास्ते जनाधिपम्}
{कुर्वाणाश्च कथास्तत्र ब्राह्मणा नृपतिं प्रति}


\twolineshloka
{कथं न राजा वृद्धः स वने वसति निर्जने}
{गान्धारी च महाभागा सा च कुन्ती पृथा कथम्}


\twolineshloka
{सुखार्हः स हि राजर्षिरसुखी तद्वनं महत्}
{किमवस्थः समासाद्य प्रज्ञाचक्षुर्महामनाः}


\twolineshloka
{सुदुष्करं कृतवती कुन्ती पुत्रानपास्य सा}
{राज्यश्रियं परित्यज्य वनं सा समरोचयत्}


\twolineshloka
{विदुरः किमवस्थश्च भ्रातृशुश्रूषुरात्मवान्}
{स च गावल्गणिर्धीमान्भर्तृपिण्डानुपालकः}


\twolineshloka
{आकुमारं च पौरास्ते चिन्ताशोकसमाहताः}
{तत्रतत्र कथाश्चक्रुः समासाद्य परस्परम्}


\twolineshloka
{पाण्डवाश्चैव ते सर्वे भृशं शोकपरायणाः}
{शोचन्तो मातरं वृद्धामूषुर्नातिचिरं पुरे}


\twolineshloka
{तथैव वृद्धं पितरं हतपुत्रं जनेश्वरम्}
{गान्धारीं च महाभागां विदुरं च महामतिम्}


\twolineshloka
{नैषां बभूव सम्प्रीतिस्तान्विचिन्तयतां तदा}
{न राज्ये न च नारीषु न वेदाध्ययनेषु च}


\twolineshloka
{परं निर्वेदमगमंश्चिन्तयन्ता नराधिपम्}
{तं च ज्ञातिवधं घोरं संस्मरन्तः पुनःपुनः}


\twolineshloka
{अभिमन्योश्च बालस्य विनाशं रणमूर्धनि}
{कर्णस्य च महाबाहोः सङ्ग्रामेष्वपलायिनः}


\twolineshloka
{तथैव द्रौपदेयानामन्येषां सुहृदामपि}
{वधं संस्मृत्य ते वीरा नातिप्रमनसोऽभवन्}


\twolineshloka
{हतप्रवीरां पृथिवीं हृतरत्नां च भारत}
{सदैव चिन्तयन्तस्ते न शमं चोपलेभिरे}


\twolineshloka
{द्रौपदी हतपुत्रा च सुभद्रा चैव भामिनी}
{नातिप्रीतियुते देव्यौ तदास्तामप्रहृष्टवत्}


\twolineshloka
{वैराट्यास्तनयं दृष्ट्वा पितरं ते परिक्षितम्}
{धारयन्ति स्म ते प्राणांस्तव पूर्वपितामहाः}


\chapter{अध्यायः २४}
\twolineshloka
{वैशम्पायन उवाच}
{}


\twolineshloka
{एवं ते पुरुषव्याघ्राः पाण्डवा मातृनन्दनाः}
{स्मरन्तो मातरं वीरा बभूवुर्भशदुःखिताः}


\twolineshloka
{ये राजकार्येषु पुरा व्यासक्ता नित्यशोऽभवन्}
{ते राजकार्याणि तदा नाकार्षुः सर्वतः पुरे}


\twolineshloka
{प्रविष्टा इव शोकेन नाभ्यनन्दन्त किञ्चन}
{सम्भाष्यमाणा अपि ते न किञ्चित्प्रत्यपूजयन्}


\twolineshloka
{ते स्म वीरा दुराधर्षा गांभीर्ये सागरोपमाः}
{शोकोपहतविज्ञाना नष्टसंज्ञा इवाभवन्}


\twolineshloka
{अचिन्तयंश्च जननीं ततस्ते पाण्डुनन्दनाः}
{कथं नु वृद्धमिथुनं वहत्यतिकृशा पृथा}


\twolineshloka
{कथं च स महीपालो हतपुत्रो निराश्रयः}
{पत्न्या सह वसत्येको वने श्वापदसेविते}


\twolineshloka
{सा च देवी महाभागा गान्धारी हतबान्धवा}
{पतिमन्धं कथं वृद्धमन्वेति विजने वने}


\twolineshloka
{एवं तेषां कथयतामौत्सुक्यमभवत्तदा}
{गमने चाभवद्बुद्धिर्धृतराष्ट्रदिदृक्षया}


\threelineshloka
{सहदेवस्तु राजानं प्रणिपत्येदमब्रवीत्}
{अहो मे भवतो दृष्टं हृदयं गमनं प्रति}
{}


\twolineshloka
{न हि त्वां गौरवेणाहमशकं वक्तुमञ्जसा}
{गमनं प्रति राजेन्द्र तदिदं समुपस्थितम्}


\twolineshloka
{दिष्ट्या द्रक्ष्यामि तां कुन्तीं वर्तयन्तीं तपस्विनीम्}
{जटिलां तापसीं वृद्धां कुशकाशपरिक्षताम्}


\twolineshloka
{प्रासादहर्म्यसंवृद्धामत्यन्तसुखभागिनीम्}
{कदानुजननीं श्रान्तां द्रक्ष्यामि भृशदुःखिताम्}


\twolineshloka
{अनित्याः खलु मर्त्यानां गतयो भरतर्षभ}
{कुन्ती राजसुता यत्र वसत्यसुखिता वने}


\threelineshloka
{सहदेववचः श्रुत्वा द्रौपदी योषितां वरा}
{उवाच देवी राजानमभिपूज्याभिनन्द्य च}
{}


\twolineshloka
{कदा द्रक्ष्यामि तां देवीं यदि जीवति सा पृथा}
{जीवन्त्या ह्यद्य मे प्रीतिर्भविष्यति जनाधिप}


\twolineshloka
{एषा तेऽस्तु मतिर्नित्यं धर्मे ते रमतां मनः}
{योऽद्य त्वमस्मान्राजेन्द्र श्रेयसा योजयिष्यसि}


\twolineshloka
{अग्रपादस्थितं चेमं विद्धि राजन्वधूजनम्}
{काङ्क्षन्तं दर्शनं कुन्त्या गान्धार्याः श्वसुरस्च च}


\twolineshloka
{इत्युक्तः स नृपो देव्या द्रौपद्या भरतर्षभ}
{सेनाध्यक्षान्समानाय्य सर्वानिदमुवाच ह}


\twolineshloka
{निर्यातयत मे सेनां प्रभूतरथकुञ्जराम्}
{द्रक्ष्यामि वनसंस्थं च धृताष्ट्रं महीपतिम्}


\twolineshloka
{स्त्र्यध्यक्षांश्चाब्रवीद्राजा यानानि विविधानि मे}
{सज्जीक्रियन्तां सर्वाणि शिबिकाश्च सहस्रशः}


\twolineshloka
{शकटापणवेशाश्च कोशः शिल्पिन एव च}
{निर्यान्तु कोशपालाश्च कुरुक्षेत्राश्रमं प्रति}


\twolineshloka
{यश्च पौरजनः कश्चिद्द्रष्टुमिच्छति पार्थिवम्}
{अनावृतः सुविहितः स च यातु सुरक्षितः}


\twolineshloka
{सूदाः पौरोगवाश्चैव सर्वं चैव महानसम्}
{विविधं भक्ष्यभोज्यं च शकटैरुह्यतां मम}


\threelineshloka
{प्रयाणं घुष्यतां चैव श्वोभूत इति माचिरम्}
{क्रियतां पथि चाप्यद्य वेश्मानि विविधानि च}
{}


\twolineshloka
{एवमाज्ञाप्य राजा स भ्रातृभिः सह पाण्डवः}
{श्वोभूते निर्ययौ राजन्सस्त्रीवृद्धपुरःसरः}


\twolineshloka
{स बहिर्दिवसानेव जनौघं परिपालयन्}
{न्यवसन्नृपतिः पञ्च ततोऽगच्छद्वनं प्रति}


\chapter{अध्यायः २५}
\twolineshloka
{आज्ञापयामास ततः सेनां भरतसत्तमः}
{अर्जुनप्रमुखैर्गुप्तां लोकपालोपमैर्नरैः}


\twolineshloka
{योगोयोग इति प्रीत्या ततः शब्दो महानभूत्}
{क्रोशतां सादिनां तत्र युज्यतां युज्यतामिति}


\twolineshloka
{केचिद्यानैर्नरा जग्मुः केचिदश्वैर्महाजवैः}
{काञ्चनैश्च रथैः केचिज्ज्वलितज्वलनोपमैः}


\twolineshloka
{गजेन्द्रैश्च तथैवान्ये केचिदुष्ट्रैर्नराधिप}
{पदातयस्तथैवान्ये नखरप्रासयोधिनः}


\twolineshloka
{पौरजानपदाश्चैव यानैर्बहुविधैस्तथा}
{अन्वयुः कुरुराजानं धृतराष्ट्रं दिदक्षवः}


\twolineshloka
{स चापि राजवचनादाचार्यो गौतमः कृपः}
{सेनामादाय सेनानीः प्रययावाश्रमं प्रति}


\twolineshloka
{ततो द्विजैः परिवृतः कुरुराजो युधिष्ठिरः}
{संस्तूयमानो बहुभिः सूतमागधबन्दिभिः}


\twolineshloka
{पाण्डुरेणातपत्रेण ध्रियमाणेन मूर्धनि}
{रतानीकेन महता निर्जगाम कुरूद्वहः}


\twolineshloka
{गजैश्चाचलसंकाशैर्भीमकर्मा वृकोदरः}
{सज्जयन्त्रायुधोपेतैः प्रययौ पवनात्मजः}


\twolineshloka
{माद्रीपुत्रावपि तथा हयारोहौ सुसंवृतौ}
{जग्मतुः शीघ्रगमनौ सन्नद्धकवचध्वजौ}


\twolineshloka
{अर्जुनश्च महातेजा रथेनादित्यवर्चसा}
{वशी श्वेतैर्हयैर्युक्तैर्दिव्येनान्वगमन्नृपम्}


\twolineshloka
{द्रौपदीप्रमुखाश्चापि स्त्रीसङ्घाः शिबिकागताः}
{स्त्र्यध्यक्षगुप्ताः प्रययुर्विसृजन्तोऽमितं वसु}


\twolineshloka
{समृद्धरथहस्त्यश्वं वेणुवीणानुनादितम्}
{शुशुभे पाण्डवं सैन्यं तत्तदा भरतर्षभ}


\twolineshloka
{नदीतीरेषु रम्येषु सरःसु च विशाम्पते}
{वासान्कृत्वा क्रमेणाथ जग्मुस्ते कुरुपुङ्गवाः}


\twolineshloka
{युयुत्सुश्च महातेजा धौम्यश्चैवि पुरोहितः}
{युधिष्ठिरस्य वचनात्पुरगुप्तिं प्रचक्रतुः}


\twolineshloka
{ततो युधिष्ठिरो राजा कुरुक्षेत्रमवातरत्}
{क्रमोणोत्तीर्य यमुनां नदीं परमपाविनीम्}


\twolineshloka
{स ददर्शाश्रमं दूराद्राजर्षेस्तस्य धीमतः}
{शतयूपस्य कौरव्य धृतराष्ट्रस्य चैव ह}


\twolineshloka
{ततः प्रमुदितः सर्वो जनस्तद्वनमञ्जसा}
{विवेश सुमहानादैरापूर्य भरतर्षभ}


\chapter{अध्यायः २६}
\twolineshloka
{ततस्ते पाण्डवा दूरादवतीर्य पदातयः}
{अभिजग्मुर्नरपतेराश्रमं विनयानताः}


\twolineshloka
{स च योधजनः सर्वो ये च राष्ट्रनिवासिनः}
{स्त्रियश्च कुरुमुख्यानां पद्भिरेवान्वयुस्तदा}


\twolineshloka
{आश्रमं ते ततो जग्मुर्धृतराष्ट्रस्य पाण्डवाः}
{शून्यं मृगगणाकीर्णं कदलीवनशोभितम्}


\twolineshloka
{ततस्तत्र समाजग्मुस्तापसा नियतव्रताः}
{पाण्डवानागतान्द्रष्टुं कौतूहलसमन्विताः}


\twolineshloka
{तानपृच्छत्ततो राजा क्वासौ कौरववंशभृत्}
{पिता ज्येष्ठो गतोऽस्माकमिति बाष्पपरिप्लुतः}


\twolineshloka
{ते तमूचुस्ततो वाक्यं यमुनामवगाहितुम्}
{पुष्पाणामुदकुंभस्य चार्थे गत इति प्रभो}


\twolineshloka
{तैराख्यातेन मार्गेण ततस्ते जग्मुरञ्जसा}
{ददृशुश्चाविदूरे तान्सर्वानथ पदातयः}


\twolineshloka
{ततस्ते सत्वरा जग्मुः वितुर्दर्शनकाङ्क्षिणः}
{सहदेवस्तु वेगेन प्राधावद्यत्र सा पृथा}


\twolineshloka
{सुस्वरं रुरुदे धीमान्मातुः पादावुपस्पृशन्}
{सा च बाष्पाकुलमुखी ददर्शक दयितं सुतम्}


\twolineshloka
{बाहुभ्यां सम्परिष्वज्य समुन्नाम्य च पुत्रकम्}
{गान्धार्याः कथयामास सहद्रेवमुपस्थितम्}


\twolineshloka
{अनन्तरं च राजानं भीमसेनमथार्जुनम्}
{नकुलं च पृथा दृष्ट्वा त्वरमाणोपचक्रमे}


\threelineshloka
{सा ह्यग्रे गच्छति तयोर्दपत्योर्हतपुत्रयोः}
{कर्षन्ती तौ ततस्ते तां दृष्ट्वा संन्यपतन्भुवि}
{`तयोस्तु पादयो राजन्न्यपतन्हतपुत्रयोः ॥'}


\twolineshloka
{राजा तान्स्वरयोगेन स्पर्शेन च महामनाः}
{प्रत्यभिज्ञाय मेधावी समाश्वासयत प्रभुः}


\twolineshloka
{ततस्ते बाष्पमुत्सृज्य गान्धारीसहितं नृपम्}
{उपतस्थुर्महात्मानो मातरं च यथाविधि}


\twolineshloka
{सर्वेषां तोयकलशाञ्जगृहुस्ते स्वयं तदा}
{पाण्डवा लब्धसंज्ञास्ते मात्रा चाश्वासिताः पुनः}


\twolineshloka
{तथा नार्यो नृसिंहानां सोऽवरोधजनस्तदा}
{पौरजानपदाश्चैव ददृशुस्तं जनाधिपम्}


\twolineshloka
{निवेदयामास तदा जनं तन्नामगोत्रतः}
{युधिष्ठिरो नरपतिः स चैनं प्रत्यपूजयत्}


\twolineshloka
{स तैः परिवृतो मेने हर्षबाष्पाविलेक्षणः}
{राजाऽऽत्मानं गृहगतं पुरेवि गजसाह्वये}


\twolineshloka
{अभिवादितो वधूभिश्च कृष्णाद्याभिः स पार्थिवः}
{गान्धार्या सहितो धीमान्कुन्त्या च प्रत्यनन्दत}


\twolineshloka
{ततश्चाश्रममागच्छत्सिद्धचारणसेवितम्}
{दिदृक्षुभिः समाकीर्णं नभस्तारागणैरिव}


\chapter{अध्यायः २७}
\twolineshloka
{स तैः सह नरव्याध्रैर्भ्रातृभिर्भरतर्षभ}
{राजा रुचिरपद्माक्षैरासांचक्रे तदाश्रमे}


\twolineshloka
{तापसैश्च महाभागैर्नानादेशसमागतैः}
{द्रष्टुं कुरुपतेः पुत्रान्पाण्डवान्पृथुवक्षसः}


\twolineshloka
{ते ब्रुवञ्ज्ञातुमिच्छामः कतमोऽत्र युधिष्ठिरः}
{भीमार्जुनौ यमौ चैव द्रौपदी च यशस्विनी}


\threelineshloka
{तानाचख्यौ तदा सूतस्तेषां नाम प्रधानतः}
{संजयो द्रौपदीं चैव सर्वाश्चान्या कुरुस्त्रियः ॥सञ्जय उवाच}
{}


\twolineshloka
{य एष जांबूनदशुद्धगौर-स्तनुर्महासिंह इव प्रवृद्धः}
{प्रचण्डघोणः पृथुदीर्घनेत्र-स्ताम्रायताक्षः कुरुराज एषः}


\twolineshloka
{अयं पुनर्मत्तगजेन्द्रगामीप्रतप्तचामीकरशुद्धगौरः}
{पृथ्वायतांसः पृथुदीर्घबाहु-र्वृकोदरः पश्यत पश्यतेमम्}


\twolineshloka
{यस्त्वेष पार्श्वेऽस्य महाधनुष्मा-ञ्श्यामो युवा वारणयूथपाभः}
{सिंहोन्नतांसो गजखेलगामीपद्मायताक्षोऽर्जुन एष वीरः}


\twolineshloka
{कुन्तीसमीपे पुरुषोत्तमौ तुयमाविमौ विष्णुमहेन्द्रकल्पौ}
{मनुष्यलोके सकले समोस्तिययोर्न रूपे न बले न शीले}


\twolineshloka
{इयं पुनः पद्मदलायताक्षीमध्यं वयः किञ्चिदिव स्पशन्ती}
{नीलोत्पलाभा पुरदेवतेवकृष्णा स्थिता मूर्तिमतीव लक्ष्मीः}


\twolineshloka
{अस्यास्तु पार्श्वे कनकोपमत्व-गेषा स्थिता मूर्तिमतीव गौरी}
{सत्ये स्थिता सा भगिनी द्विजाग्र्या-श्चक्रायुधस्याप्रतिमस्य तस्य}


\twolineshloka
{इयं च जांबूनदशुद्धगौरीपार्थस्य भार्या भुजगेन्द्रकन्या}
{चित्राङ्गदा चैव नरेन्द्रकन्यायैषा सवर्णार्द्रमधूकपुष्पैः}


\twolineshloka
{इयं स्वसा राजचमूपतेश्चप्रवृद्धनीलोत्पलदामवर्णा}
{पस्पर्ध कृष्णेन सदा नृपो योवृकोदरस्यैष परिग्रहोऽग्र्यः}


\twolineshloka
{इयं च राज्ञो मगधाधिपस्यसुता जरासंध इति श्रुतस्य}
{यवीयसो माद्रवतीसुतस्यभार्या मता चम्पकदामगौरी}


\twolineshloka
{इन्दीवरश्यामतनुः स्थिता तुयैषा पराऽऽसन्नतरोर्लतेव}
{भार्या मता माद्रवतीसुतस्यज्येष्ठस्य सेयं कमलायताक्षी}


\twolineshloka
{इयं तु निष्टप्तसुवर्णगौरीराज्ञो विराटस्य सुता सपुत्रा}
{भार्याऽभिमन्योर्निहतो रणे योद्रोणादिभिस्तैर्विरथो रथस्थैः}


\twolineshloka
{एतास्तु सीमन्तशिरोरुहा याःशुक्लोत्तरीया नरराजपत्न्यः}
{राज्ञोस्य वृद्धस्य परंशताख्याःस्नुषा नृवीराहतपुत्रनाथाः}


\fourlineindentedshloka
{एकता यथामुख्यमुदाहृता वोब्राह्मण्यभावादृजुबुद्धिसत्वात्}
{सर्वा भवद्भिः परिपृच्छ्यमानानरेन्द्रपत्न्यः सुविशुद्धसत्वाः}
{`सर्वे भवन्तोपि तपोबलाग्र्याःक्षान्ताश्च दान्ताश्च कुलोद्भवाश्च ॥' वैशम्पायन उवाच}
{}


\twolineshloka
{एवं स राजा कुरुवृद्धवर्यःसमागतस्तैर्नरदेवपुत्रैः}
{पप्रच्छ सर्वं कुशलं तदानींगतेषु सर्वेष्वथ तापसेषु}


\twolineshloka
{योधेषु चाप्याश्रममण्डलं त-न्मत्वा निविष्टेषु विमुच्य पत्रम्}
{स्त्रीवृद्धबाले च सुसंनिविष्टेयथार्हतस्तान्कुशलान्यपृच्छत्}


\chapter{अध्यायः २८}
\twolineshloka
{युधिष्ठिर महाबाहो कच्चित्वं कुशली ह्यसि}
{सहितो भ्रातृभिः सर्वैः पौरजानपदैस्तथा}


\twolineshloka
{ये च त्वामनुजीवन्ति कच्चित्तेऽपि निरामयाः}
{सचिवा भृत्यवर्गाश्च गुरवश्चैव ते नृप}


\twolineshloka
{कच्चित्तेऽपि निरातङ्का वसन्ति विषये तव}
{कच्चिद्वर्तसि पौराणीं वृत्तिं राजर्षिसेविताम्}


\threelineshloka
{कच्चिन्न्यायाननुच्छिद्य कोशस्तेऽभिप्रपूर्यते}
{अरिमध्यस्थमित्रेषु वर्तसे चानुरूपतः}
{ब्राह्मणानग्रहारैर्वा यथावदनुपश्यसि}


\twolineshloka
{कच्चित्ते परितुष्यन्ति शीलेनि भरतर्षभ}
{शत्रवोपि कुतः पौरा भृत्या वा स्वजनोपि वा}


\twolineshloka
{कच्चिद्यजसकि राजेन्द्र श्रद्धावान्पितृदेवताः}
{अतिथीनन्नपानेनि कच्चिदर्चसि भारत}


\twolineshloka
{कच्चिन्नयपथे विप्राः स्वकर्मनिरतास्तव}
{क्षत्रिया वैश्यवर्णा वा शूद्रा वाऽपि कुटुम्बिनः}


\twolineshloka
{कच्चित्स्त्रीबालवृद्धं ते न शोचति न याचते}
{जामयः पूजिताः कच्चित्तव गेहे नरर्षभ}


\threelineshloka
{कच्चिद्राजर्षिवंशोऽयं त्वामासाद्य महीपतिम्}
{यथोचितं महाराज यशसा नावसीदति ॥वैशण्पायन उवाच}
{}


\threelineshloka
{इत्येवादिनं तं स न्यायवित्प्रत्यभाषत}
{कुशलप्रश्नसंयुक्तं कुशलो वाक्यमब्रवीत् ॥युधिष्ठिर उवाच}
{}


% Check verse!
अपि ते वर्धते राजंस्पपो मन्दः श्रमश्च ते
\twolineshloka
{अपि मे जननी चेयं शुश्रूषुर्विगतक्लमा}
{अथास्याः सफलो राजन्वनवासो भविष्यति}


\twolineshloka
{इयं च माता ज्येष्ठा मे शीतवाताध्वकर्शिता}
{घोरेण तपसा युक्ता देवी कच्चिन्न शोचति}


\twolineshloka
{हतान्पुत्रान्महावीर्यान्क्षत्रधर्मपरायणान्}
{नापध्यायति वा कच्चिदस्मान्पापकृतः सदा}


\threelineshloka
{क्व चासौ विदुरो राजन्नेमं पश्यामहे वयम्}
{संजयः कुशली चायं कच्चिन्नु तपसि स्थिरः ॥वैशम्पायन उवाच}
{}


\twolineshloka
{इत्युक्तः प्रत्युवाचैनं धृतराष्ट्रो जनाधिपम्}
{कुशली विदुरः पुत्र तपो घोरं समाश्रितः}


\twolineshloka
{वायुभक्षो निराहारः कृशो धमनिसंततः}
{कजाचिद्दृश्यते विप्रैः शून्येऽस्मिन्कानने क्वचित्}


\twolineshloka
{इत्येवं ब्रुवतस्तस्य जटी वीटामुखः कृशः}
{दिग्वासा मलदिग्धाङ्गो वनरेणुसमुक्षितः}


\twolineshloka
{दूरादालक्षितः क्षत्ता तत्राख्यातो महीपतेः}
{`कविदुरस्त्वेष धर्मात्मा जनं दृष्ट्वा निवर्तते ॥'}


\threelineshloka
{निवर्तमानं सहसा जनं दृष्ट्वाऽऽश्रमं प्रति}
{तमन्वधावन्नृपतिरेक एव युधिष्ठिरः}
{प्रविशन्तं वनं घोरं लक्ष्यालक्ष्यं क्वचित्क्वचित्}


\twolineshloka
{भोभो विदुर राजाऽहं दयितस्ते युधिष्ठिरः}
{इति ब्रुवन्नरपतिस्तं यत्नादभ्यधावत}


\twolineshloka
{ततो विविक्त एकान्ते तस्थौ बुद्धिमतां वरः}
{विदुरो वृक्षमाश्रित्य कञ्चित्तत्र वनान्तरे}


\twolineshloka
{तं राजा क्षीणभूयिष्ठमाकृतीमात्रसूचितम्}
{अभिजज्ञे महाबुद्धिं महाबुद्धिर्युधिष्ठिरः}


\twolineshloka
{युधिष्ठिरोऽहमस्मीति वाक्यमुक्त्वाऽग्रतः स्थितः}
{विदुरस्याश्रमे राजा स च प्रत्याह संज्ञया}


\twolineshloka
{ततः सोऽनिमिषो भूत्वा राजानं तमुदैक्षत}
{संयोज्य विदुरस्तस्मिन्दृष्टिं दृष्ट्या समाहितः}


\twolineshloka
{विवेश विदुरो धीमान्गात्रैर्गात्राणि चैव ह}
{प्राणान्प्राणेषु च दधदिन्द्रियाणीन्द्रियेषु च}


\twolineshloka
{स योगबलमास्थाय विवेश नृपतेस्तनुम्}
{विदुरो धर्मराजस्य तेजसा प्रज्वलन्निव}


\twolineshloka
{विदुरस्य शरीरं तु तथैव स्तब्धलोचनम्}
{वृक्षाश्रितं तदा राजा ददर्श गतचेतनम्}


\twolineshloka
{बलवन्तं तथाऽऽत्मानं मेने बहुगुणं तदा}
{धर्मराजो महातेजास्तच्च सस्मार पाण्डवः}


\twolineshloka
{पौराणमात्मनः सर्वं भविष्यं च विशाम्पते}
{योगधर्मं महातेजा व्यासेन कथिनं यथा}


\twolineshloka
{धर्मराजश्च तत्रैनं सञ्चस्कारयिषुस्तदा}
{दग्धुकामोऽभवद्विद्वानथ वागभ्यभाषत}


\twolineshloka
{भोभो राजन्न दग्धव्यमेतद्विदुरसंज्ञकम्}
{कलेबरमिहैवं ते धर्म एष सनातनः}


\twolineshloka
{लोको वैकर्तनो नाम भविष्यत्यस्य भारत}
{यतिधर्ममवाप्तोसौ नैष शोच्यः परंतप}


\twolineshloka
{इत्युक्तो धर्मराजः स विनिवृत्य ततः पुनः}
{राज्ञो वैचित्रवीर्यस्य तत्सर्वं प्रत्यवेदयत्}


\twolineshloka
{ततः स राजा धृतिमान्स च सर्वो जनस्तदा}
{भीमसेनादयश्चैव परं विस्मियमागताः}


\twolineshloka
{तच्छ्रुत्वा प्रीतिमान्राजा भूत्वा धर्मजमब्रवीत्}
{आपो मूलं फलं चैव ममेदं प्रतिगृह्यताम्}


\twolineshloka
{यदन्नो हि नरो राजंस्तदन्नोऽस्यातिथिः स्मृतः}
{इत्युक्तः स तथेत्येवं प्राह धर्मात्मजो नृपम्}


\threelineshloka
{फलं मूलं च बुभुजे राज्ञा दत्तं सहानुजः}
{ततस्ते वृक्षमूलेषु कृतवासपरिग्रहाः}
{तां रात्रिमवसन्सर्वे फलमूलजलाशनाः}


\chapter{अध्यायः २९}
\twolineshloka
{ततस्तु राजन्नेतेषामाश्रमे पुण्यकर्मणाम्}
{शिवा नक्षत्रसम्पन्ना सा व्यतीयाय शर्वरी}


\twolineshloka
{ततस्तत्र कथाश्चासंस्तेषां धर्मार्थलक्षणाः}
{विचित्रपदसञ्चारा नानाश्रुतिभिरन्विताः}


\twolineshloka
{पाण्डवास्त्वभितो मातुर्धरण्यां सुषुपुस्तदा}
{उत्सृज्य तु महार्हाणि शयनानि नराधिप}


\twolineshloka
{यदाहारोऽभकवद्राजा धृतराष्ट्रो महामनाः}
{तदाहारा नृवींरास्ते न्यवसंस्तां निशां तदा}


\twolineshloka
{व्यतीतायां तु शर्वर्यां कृतपौर्वाह्णिकक्रियः}
{भ्रातृभिः सहितो राजा ददर्शाश्रममण्डलम्}


\twolineshloka
{सान्तः पुरपरीवारः सभृत्यः सपुरोहितः}
{यथासुखं यथोद्देशं धृतराष्ट्राभ्यनुज्ञया}


\twolineshloka
{ददर्श तत्र वेदीश्च सम्प्रज्वलितपावकाः}
{कृताभिषेकैर्मुनिभिर्हुताग्निभिरुपस्थिताः}


\twolineshloka
{वानेयपुष्पनिकरैराज्यधूमोद्गमैरपि}
{ब्राह्मेण वपुषा युक्ता युक्ता मुनिगणस्य ताः}


\twolineshloka
{मृगयूथैरनुद्विग्नैस्तत्रतत्र समाश्रितैः}
{अशङ्कितैः पक्षिगणैः प्रगीतैरिव च प्रभो}


\twolineshloka
{केकाभिर्नीलकण्ठानां दात्यूहानां च कूजितैः}
{कोकिलानां कुहुरवैः सुखैः श्रुतिमनोहरैः}


\twolineshloka
{प्राधीतद्विजघोषैश्च क्वचित्क्वचिदलङ्कृतम्}
{फलमूलसमाहारैर्महद्भिश्चोपशोभितम्}


\twolineshloka
{ततः स राजा प्रददौ तापसार्थमुपाहृतान्}
{कलशान्काञ्चनान्राजंस्तथैवौदुम्बरानपि}


\twolineshloka
{अजिनानि प्रवेणीश्च स्रुक्स्रुवं च महीपतिः}
{कमण्डलूंश्चि स्थालीश्च पिठराणि च भारत}


\twolineshloka
{भाजनानि च लौहानि पात्रीश्च विविधा नृप}
{यद्यदिच्छति यावच्च यदन्यदपि काङ्क्षितम्}


\twolineshloka
{एवं स राजा धर्मात्मा परीत्याश्रममण्डलम्}
{वसु विश्राण्य तत्सर्वं पुनरायान्महीपतिः}


\twolineshloka
{कृताह्निकं च राजानं धृतराष्ट्रं महीपतिम्}
{ददर्शासीनमव्यग्रं गान्धारीसहितं तदा}


\twolineshloka
{मातरं चाविदूरस्थां शिष्यवत्प्रणतां स्थिताम्}
{कुन्तीं ददर्श धर्मात्मा शिष्टाचारसमन्विताम्}


\twolineshloka
{स तमभ्यर्च्य राजानं नाम संश्राव्य चात्मनः}
{निषीदेत्यभ्यनुज्ञातो वृस्यामुपविवेश ह}


\twolineshloka
{भीमसेनादयश्चैव पाण्डव भरतर्षभ}
{अभिवाद्योपसङ्गृह्य निषेदुः पार्थिवाज्ञया}


\twolineshloka
{स तैः परिवृतो राजा शुशुभेऽतीव कौरवः}
{बिभ्रद्ब्राह्मीं श्रियं दीप्तां देवैरिव बृहस्पतिः}


\twolineshloka
{तथा तेषूपविष्टेषु समाजग्मुर्महर्षयः}
{शतयूपप्रभृतयः कुरुक्षेत्रनिवासिनः}


\twolineshloka
{व्यासश्च भगवान्विप्रो देवर्षिगणसेवितः}
{वृतः शिष्यैर्महातेजा दर्शयामास पार्थिवम्}


\twolineshloka
{ततः स राजा कौरव्यः कुन्तीपुत्रश्च धर्मराट्}
{भीमसेनादयश्चैव प्रत्युत्थायाभ्यवादयन्}


\twolineshloka
{समागतस्ततो व्यासः शतयूपादिभिर्वृतः}
{धृतराष्ट्रं महीपालमास्यतामित्यभाषत}


\twolineshloka
{वरं तु विष्टरं कौश्यं कृष्णाजिनकुशोत्तरम्}
{प्रतिपेदे तदा व्यासस्तदर्थमुपकल्पितम्}


\twolineshloka
{ते च सर्वे द्विजश्रेष्ठा विष्टरेषु समन्ततः}
{द्वैपायनाभ्यनुज्ञाता निषेदुर्विपुलौजसः}


\chapter{अध्यायः ३०}
\twolineshloka
{ततः समुपविष्टेषु पाण्डवेषु महात्मसु}
{व्यासः सत्यवतीपुत्रः प्रोवाचामन्त्र्य पार्थिवम्}


\twolineshloka
{धृतराष्ट्र महाबाहो कच्चित्ते वर्तते तपः}
{कच्चिन्मनस्ते प्रीणाति वनवासे नराधिप}


\twolineshloka
{कच्चिद्धृदि न ते शोको राजन्पुत्रविनाशजः}
{कच्चिज्ज्ञानानि सर्वाणि सुप्रसन्नानि तेऽनघ}


\twolineshloka
{कच्चिद्बुद्धिं दृढां कृत्वा चरस्यारण्यकं विधिम्}
{कच्चिद्वधूश्च गान्धारी न शोकेनाभिभूयते}


\twolineshloka
{महाप्रज्ञा बुद्धिमती देवी धर्मार्थदर्शिनी}
{आगमापायतत्त्वज्ञा कच्चिदेषा न शोचति}


\twolineshloka
{कच्चित्कुन्ती च राजंस्त्वां शुश्रूषत्यनहङ्कृता}
{या परित्यज्य स्वं पुत्रं गुरुशुश्रूषणे रता}


\twolineshloka
{कच्चिद्धर्मसुतो राजा त्वया प्रत्यभिनन्दितः}
{भीमार्जुनयमाश्चैव कच्चिदेतेऽपि सान्तिताः}


\twolineshloka
{कच्चिन्नन्दसि दृष्ट्रैतान्कच्चिते निर्मलं मनः}
{कच्चिच्च शुद्धभावोसि जातज्ञानो नराधिप}


\twolineshloka
{एतद्धि त्रितयं श्रेष्ठं सर्वभूतेषु भारत}
{निर्वैरता महाराज सत्यमक्रोध एव च}


\twolineshloka
{कच्चित्ते न च मोहोस्ति वनवासेन भारत}
{स्वदत्ते वन्यमन्नं वा उपवासोपि वा भवेत्}


\twolineshloka
{विदितं चापि राजेन्द्र विदुरस्य महात्मनः}
{गमनं विधिनाऽनेन धर्मस्य सुमहात्मनः}


\twolineshloka
{माण्डव्यशापाद्धि स वै धर्मो विदुरतां गतः}
{महाबुद्धिर्महायोगी महात्मा सुमहामनाः}


\twolineshloka
{बृहस्पतिर्वा देवेषु शुक्रो वाऽप्यसुरेषु च}
{न तथा बुद्धिसम्पन्नो यथा स पुरुषर्षभः}


\twolineshloka
{तपोबलव्ययं कृत्वा सुचिरात्सम्भृतं तदा}
{माण्डव्येनर्षिणा धर्मो ह्यभिभूतः सनातनः}


\twolineshloka
{नियोगाद्ब्रह्मणः पूर्वं मया स्वेन बलेन च}
{वैचित्रवीर्यके क्षेत्रे जातः स सुमहामतिः}


\twolineshloka
{भ्राता तव महाराज देवदेवः सनातनः}
{धारणान्मनसा ध्यानाद्य धर्म कवयो विदुः}


\twolineshloka
{सत्येन संवर्धयति यो दमेन शमेन च}
{अहिंसया च दानेन तपसा च सनातनः}


\twolineshloka
{येन योगबलाज्जातः कुरुराजो युधिष्ठिरः}
{धर्म इत्येष नृपते प्राज्ञेनामितबुद्धिना}


\twolineshloka
{यथा वह्निर्यथा वायुर्यथाऽऽपः पृथिवी यथा}
{यथाऽऽकाशं तथा धर्म इह चामुत्र च स्थितः}


\twolineshloka
{सर्वगश्चैव राजेन्द्र सर्वं व्याप्य चराचरम्}
{दृश्यते देवदेवैः स सिद्धैर्निर्मुक्तकल्मषैः}


\twolineshloka
{यो हि धर्मः स विदुरो विदुरो यः स पाण्डवः}
{स एष राजन्दृश्यस्ते माण्डवः प्रेष्यवत्स्थितः}


\twolineshloka
{प्रविष्टः स महात्मानं भ्राता ते बुद्धिसम्मतः}
{दृष्ट्वा महात्मा कौन्तेयं महायोगबलान्वितः}


\twolineshloka
{त्वां चापि श्रेयसा योक्ष्ये नचिराद्भरतर्षभ}
{संशयच्छेदनार्थाय प्राप्तं मां विद्धि पुत्रक}


\twolineshloka
{न कृतं यैः पुरा कैश्चित्कर्म लोके महर्षिभिः}
{आश्चर्यभूतं तपसः फलं तद्दर्शयामि वः}


\twolineshloka
{किमिच्छसि महीपाल मत्तः प्राप्तुमभीप्सितम्}
{द्रष्ट्रुं स्प्रष्टुमथ श्रोतुं तत्कर्तास्मि तवानघ}


\chapter{अध्यायः ३१}
\twolineshloka
{वनवासं गते विप्र धृतराष्ट्रे महीपतौ}
{सभार्ये नृपशार्दूले वध्वा कुन्त्या समन्विते}


\twolineshloka
{विदुरे चापि धर्मज्ञे धर्मराजं व्यपश्रिते}
{वसत्सु पाण्डुपुत्रेषु सर्वेष्वाश्रममण्डले}


\twolineshloka
{यत्तदाश्चर्यमिति वै करिष्यामीत्युवाच ह}
{व्यासः परमतेजस्वी महर्षिस्तद्वदस्व मे}


\twolineshloka
{वनवासे च कौरव्याः कियन्तं कालमच्युतः}
{युधिष्ठिरो नरपतिर्न्यवसत्सजनस्तदा}


\threelineshloka
{किमाहाराश्च ते तत्र ससैन्या न्यवसन्प्रभो}
{सान्तःपुरा महात्मान एतदिच्छामि वेदितुम् ॥वैशम्पायन उवाच}
{}


\twolineshloka
{`वनवासगतं राजन्धृतराष्ट्रं महीपतिम्}
{युधिष्ठिरोऽभ्ययाद्द्रष्टुं ससैन्यो भ्रातृभिः सह}


\twolineshloka
{प्रथमे दिवसे चैषामापो मूलं फलं तथा}
{भोजनं भूमिशय्या च तत्रासीद्भरतर्षभ}


\twolineshloka
{तेऽनुज्ञातास्तदा राजन्कुरुराजेन पाण्डवाः}
{विविधान्यन्नपानानि शय्याश्चैवाभजन्वने}


\twolineshloka
{मासमेकं विजह्रुस्ते ससैन्यान्तःपुरो वने}
{अथ तत्रागमद्व्यासो यथोक्तं ते मयाऽनघ}


\twolineshloka
{तथा च तेषां सर्वेषां कथाभिर्नृपसन्निधौ}
{व्यासन्वास्य तं राजन्नाजग्मुर्मुनयोऽपरे}


\twolineshloka
{नारदः पर्वतश्चैव देवलश्च महातपाः}
{विश्वावसुस्तुंबुरुश्च चित्रसेनश्च भारत}


\twolineshloka
{तेषामपि यथान्यायं पूजां चक्रे महातपाः}
{धृतराष्ट्राभ्यनुज्ञातः कुरराजो युधिष्ठिरः}


\twolineshloka
{निषेदुस्ते ततः सर्वे पूजां प्राप्य युधिष्ठिरात्}
{आसनेषु च पुण्येषु बर्हिणेषु वरेषउ च}


\twolineshloka
{तेषु तत्रोपविष्टेषु स तु राजा महामतिः}
{पाण्डुपुत्रैः परिवृतो निषसाद कुरूद्वह}


\twolineshloka
{गान्धारी चैव कुन्ती च द्रौपदी सात्वती तथा}
{स्त्रियश्चान्यास्तथाऽन्याभिः सहोपविविशुस्ततः}


\twolineshloka
{तेषां तत्र कथा दिव्या धर्मिष्ठाश्चाभवन्नृप}
{राजर्षीणां पुराणानां देवासुरविभिश्रिताः}


\twolineshloka
{ततः कथान्ते व्यासस्तं प्रज्ञाचक्षुषमीश्वरम्}
{प्रोवाच वदतांश्रेष्ठः पुनरेव स तद्वचः}


\twolineshloka
{प्रीयमाणो महातेजाः सर्ववेदविदांवरः}
{विदितं मम राजेन्द्र यत्ते हृदि विवक्षितम्}


\twolineshloka
{दह्यमानस्य शोकेन तव पुत्रकृतेन वै}
{गान्धार्याश्चैव यद्दुःखं हृदि तिष्टति नित्यदा}


\threelineshloka
{कुन्त्याश्च यन्महाराज द्रौपद्याश्च हृदि स्थितम्}
{यच्च धारयते तीव्रं दुःखं पुत्रविनाशजम्}
{सुभद्रा कृष्णभगिनी तच्चापि विदितं मम}


\twolineshloka
{श्रुत्वा समागममिमं सर्वेषां वस्ततो नृप}
{संशयच्छेदनार्थाय प्राप्तः कौरवनन्दन}


\twolineshloka
{इमे च देवगन्धर्वाः सर्वे चेमे महर्षयः}
{पश्यन्तु तपसो वीर्यमद्य मे चिरसम्भृतम्}


\twolineshloka
{तदुच्यतां महाप्राज्ञ कं कामं प्रददामि ते}
{प्रवणोस्मि वरं दातुं पश्य मे तपसो बलम्}


\twolineshloka
{एवमुक्तः स राजेन्द्रो व्यासेनामितबुद्धिना}
{मुहूर्तमिव संचिन्त्य वचनायोपचक्रमे}


\twolineshloka
{धन्योस्म्यनुगृहीतश्च सफलं जीवितं च मे}
{यन्मे समागमोऽद्येह भवद्भिः सह साधुभिः}


\twolineshloka
{अद्य चाप्यवगच्छामि गतिमिष्टामिहात्मनः}
{ब्रह्मकल्पैर्भवद्भिर्यत्समेतोऽहं तपोधनाः}


\twolineshloka
{दर्शनादेव भवतां पूतोऽहं नात्र संशयः}
{विद्यते न भयं चापि परलोके ममानघाः}


\twolineshloka
{किंतु तस्य सुदुर्बुद्धेर्मन्दस्यापनयैर्भृशम्}
{दूयते मे मनो नित्यं स्मरतः पुत्रगृद्धिनः}


\twolineshloka
{अपापाः पाण्डवा येन निकृताः पापबुद्धिना}
{घातिता पृथिवी येन सहया सनरद्विपा}


\twolineshloka
{राजानश्च महात्मानो नानाजनपदेश्वराः}
{आगम्य मम पुत्रार्थे सर्वे मृत्युवशं गताः}


\twolineshloka
{ये ते पितॄश्चं दारांश्च प्राणांश्च मनसः प्रियान्}
{परित्यज्य गताः शूराः प्रेतराजनिवेशनम्}


\twolineshloka
{का नु तेषां गतिर्ब्रह्मन्मित्रार्थे ये हता मृधे}
{तथैव पुत्रपौत्राणां मम ये निहता युधि}


\twolineshloka
{दूयते मे मनोऽभीक्ष्णं घातयित्वा महाबलम्}
{भीष्मं शान्तनवं वृद्धं द्रोणं च द्विजसत्तमम्}


\twolineshloka
{मम पुत्रेण मूढेन पापेनाकृतबुद्धिना}
{क्षयं नीतं कुलं दीप्तं पृथिवीराज्यमिच्छता}


\threelineshloka
{`एते मदर्थे पुत्रांश्च दारांश्च मनसः प्रियान्}
{भोगांश्च विविधांस्तात इष्टापुर्तांस्तथैव च}
{परित्यज्य गताः शूराः प्रेतराजनिवेशनम् ॥'}


\twolineshloka
{एतत्सर्वमनुस्मृत्य दह्यमानो दिवानिशम्}
{न शान्तिमधिगच्छामि दुःखशोकसमाहतः}


\twolineshloka
{इति मे चिन्तयानस्य पितः शान्तिर्न विद्यते ॥वैशम्पायन उवाच}
{}


\twolineshloka
{तच्छ्रुत्वा विविधं तस्य राजर्षेः परिदेवितम्}
{पुनर्नवीकृतः शोको गान्धार्या जनमेजय}


\twolineshloka
{कुन्त्या द्रुपदपुत्र्याश्च सुभद्रायास्तथैव च}
{तासां च वरनारीणां वधूनां कौरवस्य ह}


\twolineshloka
{पुत्रशोकसमाविष्टा गान्धारी त्विदमब्रवीत्}
{श्वशुरं बद्धनयना देवी प्राञ्जलिरुत्थिता}


\twolineshloka
{`लोकान्तरगतान्पुत्रानयं काङ्क्षति मानद}
{तच्चास्य मासं ज्ञातं भगवंस्तपसा त्वया ॥'}


\twolineshloka
{षोडशेमानि वर्षाणि गतानि मुनिपुङ्गव}
{अस्य राज्ञोहतान्पुत्राञ्शोचतो न शमो विभो}


\twolineshloka
{पुत्रशोकसमाविष्टो निःश्वसन्ह्येष भूमिपः}
{न शेते वसतीः सर्वा धृतराष्ट्रो महामुने}


\twolineshloka
{लोकानन्यान्समर्थोसि स्रष्टुं सर्वांस्तपोबलात्}
{किमु लोकान्तरगतान्राज्ञो दर्शयितुं सुतान्}


\twolineshloka
{इयं च द्रौपदी कृष्णा हतज्ञातिसुता भृशम्}
{शोचत्यतीव सर्वासां स्नुषाणां दयिता स्नुषा}


\twolineshloka
{तथा कृष्णस्य भगिनी सुभद्रा भद्रभाषिणी}
{सौभद्रकवधसंतप्ता भृशं शोचति भामिनी}


\twolineshloka
{इयं च भूरिश्रवसो भार्या परमसम्मता}
{भर्तृव्यसनशोकार्ता भृशं शोचति भामिनी}


\twolineshloka
{यस्यास्तु श्वशुरो धीमान्बाह्लिकः स कुरूद्वहः}
{निहतः सोमदत्तश्च पित्रा सह महारणे}


\twolineshloka
{श्रीमतोस्य महाबुद्धेः सङ्ग्रामेष्वपलायिनः}
{पुत्रस्य ते पुत्रशतं निहतं यद्रणाजिरे}


\threelineshloka
{तस्य भार्याशतमिदं दुःखशोकसमाहतम्}
{पुनःपुनर्वर्धयानं शोकं रोज्ञो ममैव च}
{हताहारेण महता मामुपास्ते महामुने}


\twolineshloka
{ये च शूरा महात्मानः श्वशुरा मे महारथाः}
{सोमदत्तप्रभृतयः का नु तेषां गतिः प्रभो}


\twolineshloka
{तव प्रसादाद्भगवन्विशोकोऽयं महीपतिः}
{कुर्यात्कालमहं चेयं वधूस्तव महामुने}


\twolineshloka
{इत्युक्तवत्यां गान्धार्यां कुन्ती व्रतकृशानना}
{प्रच्चन्नजातं पुत्रं तु सस्मारादित्यसन्निभम्}


\twolineshloka
{तामृषिर्वरदो व्यासो दूरश्रवणदर्शनः}
{अपश्यद्दुःखितां देवीं मातरं सव्यसाचिनः}


\twolineshloka
{तामुवाच ततो व्यासो यत्ते कार्यं विवक्षितम्}
{तद्बूहि त्वं महाभागे यत्ते मनसि वर्तते}


\twolineshloka
{श्वशुराय तत कुन्ती प्रणम्य शिरसा तदा}
{उवाच वाक्यं सव्रीडा विवृण्वाना पुरातनम्}


\chapter{अध्यायः ३२}
\twolineshloka
{भगवञ्श्वशुरो मेऽसि दैवतस्यापि दैवतम्}
{स मे देवातिदेवस्त्वं शृणु सत्यां गिरं मम}


\twolineshloka
{तपस्वी कोपनो विप्रो दुर्वासा नाम मे पितुः}
{भिक्षामुपागतो भोक्तुं तमहं पर्यतोषयम्}


\twolineshloka
{शौचने त्वागसस्त्यागैः शुद्धेन मनसा तथा}
{कोपस्थानेष्वपि महत्स्वकुप्यन्न कदाचन}


\twolineshloka
{स प्रीतो वरदो मेऽभूत्कृतकृत्यो महामुनिः}
{अवश्यं ते ग्रहीतव्यमिति मां सोब्रवीद्वचः}


\twolineshloka
{ततः शापभयाद्विप्रमवोचं पुनरेव तम्}
{एवमस्त्विति च प्राह पुनरेव स मे द्विजः}


\twolineshloka
{धर्मस्य जननी भद्रे भवित्री त्वं शुभानने}
{वशे स्थास्यन्ति ते देवा यांस्त्वमावाहयिष्यसि}


\twolineshloka
{इत्युक्त्वाऽन्तर्हितो विप्रस्ततोऽहं विस्मिताऽभवम्}
{न च सर्वास्ववस्थासु स्मृतिर्मे विप्रणश्यति}


\twolineshloka
{अथ हर्म्यतलस्थाऽहं रविमुद्यन्तमीक्षती}
{संस्मृत्य तदृषेर्वाक्यं स्पृहयन्ती दिवाकरम्}


\twolineshloka
{स्थिताऽहं बालभावेन तत्र दोषमबुद्ध्यती}
{अथ देवः सहस्रांशुर्मत्समीपगतोऽभवत्}


\twolineshloka
{द्विधा कृत्वाऽऽत्मनो देहं भूमौ च गगनेऽपि च}
{तताप लोकानेकेन द्वितीयेनागमत्स माम्}


\twolineshloka
{स मामुवाच वेपन्तीं वरं मत्तो वृणीष्व ह}
{गम्यतामिति तं चाहं प्रणम्य सिरसाऽवदम्}


\twolineshloka
{स मामुवाच तिग्मांशुर्वृथाऽऽह्वानं न मे क्षमम्}
{धक्ष्यामि त्वां च विप्रं च येन दत्तो वरस्तव}


\twolineshloka
{तमहं रक्षती विप्रं शापादनपकारिणम्}
{पुत्रो मे त्वत्समो देव भवेदिति ततोऽब्रवम्}


\twolineshloka
{ततो मां तेजसाऽऽविश्य मोहयित्वा च भानुमान्}
{उवाच भविता पुत्रस्तवेत्यभ्यगमद्दिवम्}


\twolineshloka
{ततोऽहमन्तर्भवने पितुश्चित्तानुरक्षिणी}
{गूढोत्पन्नं सुतं बालं जले कर्णमवासृजम्}


\twolineshloka
{नूनं तस्यैव देवस्य प्रसादात्पुनरेव तु}
{कन्याऽहमभवं विप्र यथा प्राह स मामृषिः}


\twolineshloka
{स मया सूढया पुत्रो ज्ञायमानोऽप्युपेक्षितः}
{तन्मां दहति विप्रर्षे यथा सुविदितं तव}


\twolineshloka
{यदि पापमपापं वा यदेतद्विवृतं मया}
{तं द्रष्टुमिच्छामि भगवन्व्यपनेतुं त्वमर्हसि}


\twolineshloka
{यच्चास्य राज्ञो विदितं हृदिस्थं भवतोऽनघ}
{तं चायं लभतां काममद्यैव मुनिसत्तम}


\twolineshloka
{इत्युक्तः प्रत्युवाचेदं व्यासो वेदविदांवरः}
{साधु सर्वमिदं भाव्यमेवमेतद्यथाऽऽत्थ माम्}


\twolineshloka
{अपराधश्च ते नास्ति कन्याभावं गता ह्यसि}
{देवाश्चैश्वर्यवन्तो वै शरीराण्याविशन्ति वै}


\twolineshloka
{सन्ति देवनिकायाश्च संकल्पाज्जनयन्ति ये}
{वाचा दृष्ट्या तथा स्पर्शात्संघर्षेणेति पञ्चधा}


\twolineshloka
{मनुष्यधर्मो दैवेन धर्मेण हि न दुष्यति}
{इति कुन्ति विजानीहि व्येतु ते मानसो ज्वरः}


\twolineshloka
{सर्वं बलवतां पथ्यं सर्वं बलवतां शुचि}
{सर्वं बलवतां धर्मः सर्वं बलवतां स्वकम्}


\chapter{अध्यायः ३३}
\twolineshloka
{भद्रे द्रक्ष्यसि गान्धारि पुत्रान्भ्रातॄन्स्वकान्गणान्}
{वधूश्च पतिभिः सार्धं निशि सुप्तोस्थिता इव}


\twolineshloka
{कर्णं द्रक्ष्यति कुन्ती च सौभद्रं चापि यादवी}
{द्रौपदी पञ्चपुत्रांश्च पितॄन्भ्रातॄंस्तथैव च}


\twolineshloka
{पूर्वमेवैष हृदये व्यवसायोऽभकवन्मम}
{यदाऽस्मि चोदितो राज्ञा भवत्या पृथयैव च}


\twolineshloka
{न ते शोच्या महात्मानः सर्व एव नरर्षभाः}
{क्षत्रधर्मपराः सन्तस्तथा हि निधनं गताः}


\twolineshloka
{भवितव्यमवश्यं तत्सुरकार्यमनिन्दिते}
{अवतेरुस्ततः सर्वे देवा भागैर्महीतलम्}


\twolineshloka
{गन्धर्वाप्सरसश्चैव पिशाचा गुह्यराक्षसाः}
{तथा पुण्यजनाश्चैव सिद्धा देवर्षयोपि च}


\twolineshloka
{देवाश्च दानवाश्चैव तथा देवर्षयोऽमलाः}
{ते एते निधनं प्राप्ताः कुरुक्षेत्रे रणाजिरे}


\twolineshloka
{गन्धर्वराजो यो धीमान्धृतराष्ट्र इति श्रुतः}
{स एव मानुषे लोके धृतराष्ट्रः पतिस्तव}


\twolineshloka
{पाण्डुं मरुद्गणाद्विद्धि विशिष्टतममच्युतम्}
{धर्मस्यांशोऽभवत्क्षत्ता राजा चैव युधिष्ठिरः}


\twolineshloka
{कलिं दुर्योधनं विद्धि शकुनिं द्वापरं नृपम्}
{दुःशासनादीन्विद्धि त्वं राक्षसान्शुभदर्शने}


\threelineshloka
{मरुद्गणाद्भीमसेनं बलवन्तमरिंदमम्}
{विद्धिं त्वं तु नरमृषिमिमं पार्थं धनंजयम्}
{नारायणं हृषीकेशमश्विनौ यमजौ तथा}


\twolineshloka
{द्विधा कृत्वाऽऽत्मनो देहमादित्यं तपतां वरम्}
{लोकांश्च तापयानं वै कर्णं विद्धि पृथासुतम्}


\twolineshloka
{यः स वैरार्थमुद्भूतः संघर्षजननस्कतथा}
{तं कर्णं विद्धि कल्याणि भास्करं शुभदर्शने}


\twolineshloka
{यश्च पाण्डवदायादो हतः षङ्मिर्महारथैः}
{स सोम इह सौभद्रो योगादेवाभवद्द्विधा}


\twolineshloka
{द्रौपद्या सह संभूतं धृष्टद्युम्नं च पावकात्}
{अग्रेर्भागं शुभं विद्धि राक्षसं तु शिखण्डिनम्}


\twolineshloka
{द्रोणं बृहस्पतेर्भागं विद्धि द्रौणिं च रुद्रजम्}
{गाङ्गेयो वसुवीर्येण देवो मानुषतां गतः}


\twolineshloka
{एवमेते महाप्रज्ञे देवा मानुष्यमेत्य हि}
{ततः पुनर्गताः स्वर्गं कृते कर्मणि शोभने}


\twolineshloka
{यच्च वै हृदि सर्वेषां दुःखमेतच्चिरं स्थितम्}
{तदद्य व्यपनेष्यामि परलोककृताद्भयात्}


\threelineshloka
{सर्वे भवन्तो गच्छन्तु नदीं भागीरथीं प्रति}
{तत्र द्रक्ष्यथ तान्सर्वान्ये हतास्तत्र संयुगे ॥वैशम्पायन उवाच}
{}


\twolineshloka
{इति व्यासस्य वचनं श्रुत्वा सर्वा जनस्तदा}
{महता सिंहनादेन गङ्गामभिमुखो ययौ}


\twolineshloka
{धृतराष्ट्रश्च सामात्यः प्रययौ सह पाण्डवैः}
{सहितो मुनिशार्दूलैर्गन्धर्वैश्च समागतैः}


\twolineshloka
{ततो गङ्गां समासाद्य क्रमेण स जनार्णवः}
{निवासमकरोत्सर्वो यथाप्रीति यथासुखम्}


\twolineshloka
{राजा च पाण्डवैः सार्धमिष्टे देशे सहानुगः}
{निवासमकरोद्धीमान्सस्त्रीवृद्धपुरःसरः}


\twolineshloka
{जगाम तदहस्चापि तेषां वर्षशतं यथा}
{निशां प्रतीक्षमाणानां दिदृक्षूणां मृतान्नृपान्}


\twolineshloka
{अथ पुण्यं गिरिवरमस्तमभ्यगमद्रविः}
{ततः कृताभिषेकास्ते नैशं कर्म समाचरन्}


\chapter{अध्यायः ३४}
\twolineshloka
{ततो निशायां प्राप्तायां कृतसायाह्निकक्रियाः}
{व्यासमभ्यगमन्सर्वे ये तत्रासन्समागताः}


\twolineshloka
{धृतराष्ट्रस्तु धर्मात्मा पाण्डवैः सहितस्तदा}
{शुचिरेकमनाः सार्धमृषिभिस्तैरुपाविशत्}


\twolineshloka
{गान्धार्या सह नार्यस्तु सहिताः समुपाविशन्}
{पौरजानपदश्चापि जनः सर्वो यथा वयः}


\twolineshloka
{ततो व्यासो महातेजाः पुण्यं भागीरथीजलम्}
{अवतीर्याजुहावाथ सँर्वाल्लोकान्महामुनिः}


\threelineshloka
{पाण्डवानां च ये योधाः कौरवाणां च सर्वशः}
{राजानश्च महाभागा नानादेशनिवासिनः}
{`प्रतीक्ष्य तस्थुस्ते सर्वे तेषामागमनं प्रति ॥'}


\twolineshloka
{ततः सुतुमुलः शब्दो जलान्ते जनमेजय}
{प्रादुरासीद्यथायोगं कुरुपाण्डवसेनयोः}


\twolineshloka
{ततस्ते पार्थिवाः सर्वे भीष्मद्रोणपुरोगमाः}
{ससैन्याः सलिलात्तस्मात्समुत्तस्थुः सहस्रशः}


\twolineshloka
{विराटद्रुपदौ चैव सहपुत्रौ ससैनिकौ}
{द्रौपदेयाश्च सौभद्रो राक्षसश्च घटोत्कचः}


\twolineshloka
{कर्णदुर्योधनौ चैव शकुनिश्च महारथः}
{दुःशासनादयश्चैव धार्तराष्ट्रा महाबलाः}


\twolineshloka
{जारासन्धिर्भगदत्तो जलसन्धश्च वीर्यवान्}
{भूरिश्रवाः शलः शल्यो वृषसेनश्च सानुजः}


\twolineshloka
{लक्ष्मणो राजपुत्रश्च धृष्टद्युम्नस्सहात्मजः}
{शिखण्डिपुत्राः सर्वे च धृष्टकेतुश्च सानुज}


\twolineshloka
{अचलो वृषकश्चैव राक्षसश्चाप्यलायुधः}
{बाह्लिकः सोमदत्तश्च चेकितानश्च पार्थिवः}


\twolineshloka
{एते चान्ये च बहवो बहुत्वाद्ये न कीर्तिताः}
{सर्वे भासुरदेहास्ते समुत्तस्थुर्जलात्ततः}


\twolineshloka
{यस्य् वीरस्य यो वेषो यो ध्वजो यच्च वाहनम्}
{यद्वर्म यत्प्रहरणं तेन तेन स दृश्यते}


\twolineshloka
{दिव्यांबरधराः सर्वे सर्वे भ्राजिष्णुकुण्डलाः}
{निर्वैरा निरहङ्कारा विगतक्रोधमत्सराः}


\twolineshloka
{गन्धर्वैरुपगीयन्तः स्तूयमानाश्च बन्दिभिः}
{दिव्यमाल्यस्रजोपेतास्तथा दिव्याप्सरोवृताः}


\twolineshloka
{धृतराष्ट्रस्य च तदा दिव्यं चक्षुर्नराधिप}
{मुनिः सत्यवतीपुत्रः प्रीतः प्रादात्तपोबलात्}


\twolineshloka
{दिव्यज्ञानबलोपेता गान्धारी च यशस्विनी}
{ददर्श पुत्रांस्तान्सर्वान्ये चान्येऽपि मृघे हताः}


\twolineshloka
{तदद्भुतमचिन्त्यं च सुमहद्रोमहर्षमम्}
{विस्मितः स जनः सर्वो ददर्शानिमिषेक्षणः}


\twolineshloka
{तदुत्सवमहोदग्रं हृष्टनारीनराकुलम्}
{आश्चर्यभूतं ददृशे चित्रं पटगतं यथा}


\twolineshloka
{धृतराष्ट्रस्तु तान्सर्वान्पश्यन्दिव्येन चक्षुषा}
{मुमुदे भरतश्रेष्ठ प्रसादात्तस्य वै मुनेः}


\chapter{अध्यायः ३५}
\twolineshloka
{ततस्ते पुरषश्रेष्ठाः समाजग्मुः परस्परम्}
{विगतक्रोधमात्सर्याः सर्वे विगतकल्मषाः}


\twolineshloka
{विधिं परममास्थाय ब्रह्मर्षिविहितं शुभम्}
{संहृष्टमनसः सर्वे देवलोक इवामराः}


\twolineshloka
{पुत्रः पित्रा च मात्रा च भार्याश्च पतिभिः सह}
{भ्रात्रा भ्राता सखा चैव सख्या राजन्समागताः}


\twolineshloka
{पाण्डवास्तु महेष्वासं कर्णं सौभद्रमेव च}
{सम्प्रहर्षात्समाजग्मुर्द्रौपदेयांश्चक सर्वशः}


\twolineshloka
{ततस्ते प्रीयमाणा वै कर्णेन सह पाण्डवाः}
{समेत्य पृथिवीपाल सौहृदे च स्थिताऽभवन्}


\twolineshloka
{परस्परं समागम्य योधास्ते भरतर्षभ}
{मुनेः प्रसादात्ते ह्येवं क्षत्रिया नष्टमन्यवः}


\twolineshloka
{असौहृदं परित्यज्य सौहृदे पर्यवस्थिताः}
{एवं समागताः सर्वे गुरुभिर्बान्धवैः सह}


\twolineshloka
{पुत्रैश्च पुरुषव्याघ्राः कुरवोऽन्ये च पार्थिवाः}
{तां रात्रिमकिलामेवं विहृत्य प्रीतमानसाः}


\twolineshloka
{मेनिरे परितोषेण नृपाः स्वर्गसदो यथा}
{नात्र शोको भयं त्रासो नारतिर्नायशोऽभवत्}


\threelineshloka
{परस्परं समागम्य योधानां भरतर्षभ}
{समागतास्ताः पितृभिर्भ्रातृभिः पतिभिः सुतैः}
{मुदं परमिकां प्राप्य नार्यो दुःखमथात्यजन्}


\twolineshloka
{`देवलोकं गता ये च ये च ब्रह्मसदो गताः}
{ये चापि वारुणं लोकं ये च गोलोकमाश्रिताः}


\twolineshloka
{तथा वैवस्वतं लोकं ये च यक्षानुपागताः}
{राक्षसांश्च पिशाचांश्च कुरूंश्चापि तथोत्तरान्}


\twolineshloka
{विचित्राश्च गतीरन्ये ये प्राप्ताः कर्मभिर्नराः}
{सर्वे ते तद्वयोरूपवेषास्तत्र समभ्ययुः ॥'}


\twolineshloka
{एकां रात्रिं विहृत्यैवं ते वीरास्ताश्च योषितः}
{आमन्त्र्यान्योन्यमाश्लिष्य ततो जग्मुर्यथागतं}


\twolineshloka
{ततो विसर्जयामास लोकांस्तान्मुनिपुङ्गवः}
{क्षणेनान्तर्हिताश्चैव प्रेक्षतामेव तेऽभवन्}


\twolineshloka
{अवगाह्य महात्मानः पुण्यां भागीरथीं नदीम्}
{सरथाः सध्वजाश्चैव स्वानि वेश्मानि भेजिरे}


\twolineshloka
{देवलोकं ययुः केचित्केचिद्ब्रह्मसदस्तथा}
{केचिच्च वारुणं लोकं केचित्कौबेरमाप्नुवन्}


\twolineshloka
{ततो वैवस्वतं लोकं केचिच्चैवाप्नुवन्नृपाः}
{राक्षसानां पिशाचानां केचिच्चाप्युत्तरान्कुरून्}


\twolineshloka
{विचित्रगतयः सर्वे यानवाप्यामरैः सह}
{आजग्मुस्ते महात्मानः सवाहाः सपदानुगाः}


\fourlineindentedshloka
{गतेषु तेषु सर्वेषु सलिलस्थो महामुनिः}
{धर्मसीलो महातेजाः कुरूणां हितकृतदा}
{ततः प्रोवाच ताः सर्वाः क्षत्रिया निहतेश्वराः}
{}


\twolineshloka
{यायाः पतिकृताँल्लोकानिच्छनति परमस्त्रियः}
{ता जाह्नवीजलं क्षिप्रमवगाहन्त्वतन्द्रिताः}


\twolineshloka
{ततस्तस्य वचः श्रुत्वा श्रद्दधाना वराङ्गनाः}
{श्वशुरं समनुज्ञाप्य विविशुर्जाह्नवीजलम्}


\twolineshloka
{विमुक्ता मानुषैर्देहैस्ततस्ता भर्तृभिः सह}
{समाजग्मुस्तदा साध्व्यः सर्वा एवं विशाम्पते}


\twolineshloka
{एवं क्रमेणि सर्वास्ताः शीलवत्यः पतिव्रताः}
{प्रविश्य तोयं निर्मुक्ता जग्मुर्भर्तृसलोकताम्}


\twolineshloka
{दिव्यरूपसमायुक्ता दिव्याभरणभूषिताः}
{दिव्यमाल्यांबरधरा यथाऽऽसां पतयस्तथा}


\twolineshloka
{ताः शीलगुणसम्पन्ना विमानस्था गतक्लुमाः}
{सर्वाः सर्वगुणोपेताः स्वस्थानं प्रतिपेदिरे}


\twolineshloka
{यस्य यस्य तु यः कामस्तस्मिन्काले बभूव ह}
{तंतं विसृष्टवान्व्यासो वरदो धर्मवत्सलः}


\twolineshloka
{तच्छ्रुत्वा नरदेवानां पुनरागमनं नराः}
{जहृषुर्मुदिताश्चासन्नानादेशगता अपि}


\twolineshloka
{`ते नष्टभयसङ्कल्पा नरा विगतकल्मषाः}
{बभूवुः पौरवाःसर्वे तद्दृष्ट्वाऽऽश्चर्यमुत्तमम् ॥'}


\threelineshloka
{प्रियैः समागमं तेषां यः सम्यक् शृणुयान्नरः}
{प्रियाणि लभते नित्यमिह च प्रेत्य चैव सः}
{इष्टबान्धवसंयोगमनायासमनामयम्}


\twolineshloka
{यश्चैतच्छ्रावयेद्विद्वान्विदुषो धर्मवित्तमः}
{स यशः प्राप्नुयाल्लोके परत्र च शुभां गतिम्}


\twolineshloka
{स्वाध्याययुक्ता मनुजास्तपोयुक्ताश्च भारत}
{साध्वाचारा दमोपेता दाननिर्धूतकल्मषाः}


\threelineshloka
{ऋजवः शुचयः शान्ता हिंसानृतविवर्जिताः}
{आस्तिकाः श्रद्दधानाश्च धृतिमन्तश्च मानवाः}
{श्रुत्वाऽऽश्चर्यमिदं पर्व ह्यवाप्स्यन्ति परां गतिम्}


\twolineshloka
{`पुनस्ते दर्शनं प्राप्ताः पुनश्च परिकीर्तिताः}
{पुनःपुनः प्रयच्छन्ति शृण्वतामभयं सदा ॥'}


\chapter{अध्यायः ३६}
\twolineshloka
{एतच्छ्रुत्वा नृपो विद्वान्हृष्टोऽभूज्जनमेजयः}
{पितामहानां सर्वेषां गमनागमनं तदा}


\twolineshloka
{अब्रवीच्च मुदा युक्तः पुनरागमनं प्रति}
{कथं न त्यक्तदेहानां पुनस्तद्रूपदर्शनम्}


\threelineshloka
{इत्युक्तः स द्विजश्रेष्ठो व्यासशिष्यः प्रतापवान्}
{प्रोवाच वदतांश्रेष्ठस्तं नृपं जनमेजयम् ॥वैशम्पायन उवाच}
{}


\twolineshloka
{अविप्रणाशः सर्वेषां कर्मणामिति निश्चयः}
{कर्मजानि शरीराणि शरीराकृतयस्तथा}


\twolineshloka
{महाभूतानि नित्यानि भूताधिपतिसंश्रयात्}
{तेषां च नित्यसंवासो न विनाशो वियुज्यताम्}


\twolineshloka
{अनाशया कृतं कर्म तस्य चेष्टः फलागमः}
{आत्मा चैभिः समायुक्तः सुखदुःखमुपाश्नुते}


\twolineshloka
{अविनाश्यस्तथा नित्यं क्षेत्रज्ञ इति निश्चयः}
{भूतानामात्मभावो यो ध्रुवोसौ संविजानताम्}


\twolineshloka
{यावन्न क्षीयते कर्म तावत्तस्य स्वरूपता}
{क्षीणकर्मा नरो लोके रूपान्यत्वमुपाश्नुते}


\twolineshloka
{नानाभूतास्तथैकत्वं शरीरं प्राप्य संहताः}
{भवन्ति ते तथा नित्याः पृथग्भावं विजानताम्}


\twolineshloka
{अश्वमेधश्रुतिश्चेयमश्वसंज्ञपनं प्रति}
{लोकान्तरगता नित्यं प्राणा नित्या हि वाजिनः}


\twolineshloka
{अहं हितं वदाम्येतत्प्रियं चेत्तव पार्थिव}
{देवयाना हि पन्थानः श्रुतास्ते यज्ञसंस्तरे}


\twolineshloka
{सुकृतो यत्र यज्ञस्ते तत्र देवा हितास्तव}
{यदा समन्विता देवाः पशूनां गमनेश्वराः}


\twolineshloka
{गमिमन्तश्च तेनेष्ट्वा नान्ये नित्या भवन्त्युत}
{नित्येऽस्मिन्पञ्चके वर्गे नित्ये चात्मनि पूरुषः}


\twolineshloka
{अस्य नाशं समायोगं यः पश्यति वृथामतिः}
{वियोगे शोचतेऽत्यर्थं स बाल इति मे मतिः}


\twolineshloka
{वियोगे दोषदर्शी यः संयोगं स विसर्जयेत्}
{असङ्गे सङ्गमो नास्ति दुःखं भावि वियोगजम}


\twolineshloka
{परापरज्ञस्त्वपरो नाभिमानादुदीक्षितः}
{अपरज्ञः परां बुद्धिं स्पृष्ट्वा मोहाद्विमुच्यते}


\twolineshloka
{अदर्शनादापतितः पुनश्चादर्शनं गतः}
{नाहं तं वेद्मि नासौ मां न च मेऽस्ति विरागता}


\threelineshloka
{येनयेन शरीरेणि करोत्ययमनीश्वरः}
{तेनतेन शरीरेण तदवश्यमुपाश्नुते}
{मानसं मनसाऽऽप्नोति शरीरं च शरीरवान्}


\chapter{अध्यायः ३७}
\twolineshloka
{अदृष्ट्वा तु नृपः पुत्रान्दर्शनं प्रतिलब्धवान्}
{ऋषेः प्रसादात्पुत्राणां स्वरूपाणां कुरूद्वह}


\threelineshloka
{स राजा राजधर्मांश्च ब्रह्मोपनिषदं तथा}
{अवाप्तवान्नरश्रेष्ठो बुद्धिनिश्चयमेव च}
{}


\threelineshloka
{विदुरश्च महाप्राज्ञो ययौ सिद्धिं तपोबलात्}
{धृतराष्ट्रः समासाद्य व्यासं चैव तपस्विनम् ॥जनमेजय उवाच}
{}


\twolineshloka
{ममापि वरदो व्यासो दर्शयेत्पितरं यदि}
{तद्रूपवेषवयसं श्रद्दध्यां सर्वमेव तत्}


\threelineshloka
{प्रियं मे स्यात्कृतार्थश्च स्यामहं कृतनिश्चयः}
{प्रसादादृषिमुख्यस्य मम कामः समृद्ध्यताम् ॥सौतिरुवाच}
{}


\twolineshloka
{इत्युक्तवचने तस्मिन्नृपे व्यासः प्रतापवान्}
{प्रसादमकरोद्धीमानानयच्च परिक्षितम्}


\twolineshloka
{ततस्तद्रूपवयसमागतं नृपतिं दिवः}
{श्रीमन्तं पितरं राजा ददर्श स परीक्षितम्}


\twolineshloka
{शमीकं च महात्मानं पुत्रं तं चास्य शृङ्गिणम्}
{अमात्या ये च निहता राज्ञस्तांश्च ददर्श ह}


\twolineshloka
{ततः सोवभृथे राजा मुदितो जनमेजयः}
{पितरं स्नापयामास स्वयं सस्नौ च पार्थिवः}


\threelineshloka
{`परीक्षिदपि तत्रैव बभूव स तिरोहितः}
{'स्नात्वा स नृपतिर्विप्रमास्तीकमिदमब्रवीत्}
{यायावरकुलोत्पन्नं जरत्कारुसुतं तदा}


\threelineshloka
{आस्तीक विविधाश्चर्यो यज्ञेऽयमिति मे मतिः}
{यदद्यायं पिता प्राप्तो मम शोकप्रणाशनः ॥आस्तीक उवाच}
{}


\twolineshloka
{ऋषिर्द्वैपायनो यत्र पुराणस्तपसो निधिः}
{यज्ञे कुरुकुलश्रेष्ठ तस्य लोकावुभौ जितौ}


\twolineshloka
{श्रुतं विचित्रमाख्यानं त्वया पाण्डवनन्दन}
{सर्पाश्च भस्मसान्नीता गताश्च पदवीं पितुः}


\twolineshloka
{कथंचित्तक्षको मुक्तः सत्यत्वात्तव पार्थिव}
{ऋषयः पूजिताः सर्वे गतिर्दृष्टा महात्मनः}


\twolineshloka
{प्राप्तः सुविपुलो धर्मः श्रुत्वा पापविनाशनम्}
{विमुक्तो हृदयग्रन्थिरुदारजनदर्शनात्}


\threelineshloka
{ये च पक्षधरा धर्मे सद्वृत्तरुचयश्च ये}
{यान्दृष्ट्वा हीयते पापं तेभ्यः कार्या नमस्क्रिया ॥सौतिरुवाच}
{}


\twolineshloka
{एतच्छ्रुत्वा द्विजश्रेष्ठात्स राजा जनमेजयः}
{पूजयामास तमृषिमनुमान्य पुनःपुनः}


\twolineshloka
{पप्रच्छ तमृषि चापि वैशम्पायनमच्युतम्}
{कथावशेषं धर्मज्ञो वनवासस्य सत्तम}


\chapter{अध्यायः ३८}
\threelineshloka
{दृष्ट्वा पुत्रांस्तथा पौत्रान्सानुबन्धाञ्जनाधिपः}
{धृतराष्ट्रः किमकरोद्राजा चैव युधिष्ठिरः ॥वैशम्पायन उवाच}
{}


\twolineshloka
{तद्दृष्ट्वा महदाश्चर्यं पुत्राणां दर्शनं पुनः}
{वीतशोकः स राजर्षिः पुनराश्रममागमत्}


\twolineshloka
{इतरस्तु जनः सर्वस्ते चैव परमर्षयः}
{प्रतिजग्मुर्यथाकामं धृताराष्ट्राभ्यनुज्ञया}


\twolineshloka
{पाण्डवास्तु महात्मानो लघुभूयिष्ठसैनिकाः}
{अनुजग्मुर्महात्मानं सदारास्तं महीपतिम्}


\twolineshloka
{तमाश्रमगतं धीमान्ब्रह्मर्षिर्लोकपूजितः}
{द्वैपायनोऽभ्युपागम्य राजानमिदम्नब्रवीत्}


\twolineshloka
{धृताष्ट्र महाबाहो शृणु कौरवनन्दन}
{श्रुतास्ते ज्ञानवृद्धानामृषीणां पुण्यकर्मणाम्}


\twolineshloka
{अद्धाभिजनवृद्धानां वेदवेदाङ्गवेदिनाम्}
{धर्मज्ञानां पुराणानां वदतां विविधाः कथाः}


\twolineshloka
{मा स्म शोके मनः कार्षीर्दिष्टे न व्यथते बुधः}
{श्रुतं देवरहस्यं ते नारदाद्देवदर्शनात्}


\twolineshloka
{गतास्ते क्षत्रधर्मेण शस्त्रपूतां गतिं शुभाम्}
{यथा दृष्टास्त्वया पुत्रास्तथा कामविहारिणः}


\twolineshloka
{युधिष्ठिरः स्वयं धीमान्भवन्तमनुरुध्यते}
{सहितो भ्रातृभिः सर्वैःइ सदारः ससुहृज्जनः}


\twolineshloka
{विसर्जयैनं यात्वेष स्वराज्यमनुशासताम्}
{मासः समधिकस्तेषामतीतो वसतां वने}


\twolineshloka
{एतद्धि नित्यं यत्नेन पदं रक्ष्यं नराधिप}
{बहुप्रत्यर्थिकं ह्येतद्राज्यं नाम कुरूद्वह}


\twolineshloka
{इत्युक्तः कौरवो राजा व्यासेनामितबुद्धिना}
{युधिष्ठिरमथाहूय वाग्मी वचनमब्रवीत्}


\twolineshloka
{अजातशत्रो भद्रं ते शृणु मे भ्रातृभिः सह}
{त्वत्प्रसादान्महीपाल शोको नास्मान्प्रबाधते}


\twolineshloka
{रमे चाहं त्वया पुत्र पुरेव गजसाहये}
{नाथेनानुगतो विद्वन्प्रियेषु परिवर्तिना}


\twolineshloka
{प्राप्तं पुत्रफलं त्वत्तः प्रीतिर्मे परमा त्वयि}
{न मे मन्युर्महाबाहो गम्यतां मा चिरं कृथाः}


\twolineshloka
{भवन्तं चेह संप्रेक्ष्य तपो मे परिहीयते}
{उपयुक्तं शरीरं च त्वां दृष्ट्वा धारितं पुनः}


\twolineshloka
{मातरौ ते तथैवेमे शीर्णपर्णकृताशने}
{मम तुल्यव्रते पुत्र न चिरं वर्तयिष्यतः}


\twolineshloka
{दुर्योधनप्रभृतयो दृष्टा लोकान्तरं गताः}
{व्यासस्य तपसो वीर्याद्भवतश्च समागमात्}


\twolineshloka
{प्रयोजनं चिरं वृत्तं जीवितस्य ममानघ}
{उग्रं तपः समास्थास्येत्वमनुज्ञातुमर्हसि}


\twolineshloka
{त्वय्यद्य पिण्डः कीर्तिश्च कुलं चेदं प्रतिष्ठितम्}
{श्वोवाऽद्य वा महाबाहो गम्यतां माचिरं कृथाः}


\threelineshloka
{राजनीतिः सुबहुशः श्रुता ते भरतर्षभ}
{संदेष्टव्यं न पश्यामि कृतमेतावता विभो ॥वैशम्पायन उवाच}
{}


\twolineshloka
{इत्युक्तवचनं तं तु नृपो राजानमब्रवीत्}
{न मामर्हसि धर्मज्ञ परित्यक्तुमनागसम्}


\twolineshloka
{कामं गच्छन्तु मे सर्वे भ्रातरोऽनुचरास्तथा}
{भवन्तमहमन्विष्ये मातरौ च यतव्रतः}


\twolineshloka
{तमुवाचाथ गान्धारी मैवं पुत्र शृणुष्व च}
{त्वय्यधीनं कुरुकुलं पिण्डश्च श्वशुरस्य मे}


\threelineshloka
{गम्यतां पुत्र पर्याप्तमेतावत्पूजिता वयम्}
{राजा यदाह तत्कार्यं त्वया पुत्र पितुर्वचः ॥वैशम्पायन उवाच}
{}


\twolineshloka
{इत्युक्तः स तु गान्धार्या कुन्तीमिदमभाषत}
{स्नेहबाष्पाकुले नेत्रे परिमृज्य विनीतवत्}


\twolineshloka
{विसर्जयति मां राजा गान्धारी च यशस्विनी}
{भवत्यां बद्धचित्तस्तु कथं यास्यामि दुःखितः}


\twolineshloka
{न चोत्सहे तपोविघ्नं कर्तुं ते धर्मचारिणि}
{तपसो हि परं नास्ति तपसा विन्दते महत्}


\twolineshloka
{ममापि न तथा राज्ञि राज्ये बुद्धिर्यथा पुरा}
{तपस्येवानुरक्तं मे मनः सर्वात्मना तथा}


\twolineshloka
{शून्येयं च मही कृत्स्ना न मे प्रीतिकरी शुभे}
{बान्धवा नः परिक्षीणा बलं नो न यथा पुरा}


\twolineshloka
{पाञ्चालाः सुभृशं क्षीणाः कन्यामात्रावशेषिताः}
{न तेषां कुलकर्तारं कञ्चित्पश्याम्यहं शुभे}


\threelineshloka
{सर्वे हि भस्मासान्नीतास्ते द्रोणेन रणाजिरे}
{अवशिष्टाश्च निहता द्रोणपुत्रेण वै निशि}
{चेदयश्चैव मत्स्याश्च दृष्टपूर्वास्तथैव नः}


\twolineshloka
{केवलं वृष्णिचक्रं च वासुदेवपरिग्रहात्}
{यद््दृष्ट्वा स्थातुमिच्छामि धर्मार्थं नार्थहेतुतः}


\twolineshloka
{शिवेन पश्य नः सर्वान्दुर्लभं तव दर्शनम्}
{भविष्यत्यंब राजा हि तीव्रं चारप्स्यते तपः}


\twolineshloka
{एतच्छ्रुत्वा महाबाहुः सहदेवो युधांपतिः}
{युधिष्ठिरमुवाचेदं बाष्पव्याकुललोचनः}


\twolineshloka
{नोत्सहेऽहं परित्यक्तुं मातरं भरतर्षभ}
{प्रतियातु भवान्क्षिप्रं तपस्तप्स्याम्यहं वने}


\twolineshloka
{इहैव शोषयिष्यामि तपसेदं कलेवरम्}
{पादशुश्रूषणेरक्तो राज्ञो मात्रोस्तथाऽनयोः}


\twolineshloka
{तमुवाच ततः कुन्ती परिष्वज्य महाभुजम्}
{गम्यतां पुत्र मैवं त्वं वोचः कुरु वचो मम}


\twolineshloka
{आगमा वः शिवाः सन्तु स्वस्था भवत पुत्रकाः}
{उपरोधो भवेदेवमस्माकं तपसः कृते}


\twolineshloka
{त्वत्स्नेहपाशबद्धा च हीयेयं तपसः परात्}
{तस्मात्पुत्रक गच्छ त्वं शिष्टमल्पं च नः प्रभो}


\twolineshloka
{एवं संस्तंभितं वाक्यैः कुन्त्या बहुविधैर्मनः}
{सहदेवस्य राजेन्द्र राज्ञश्चैव विशेषतः}


\threelineshloka
{ते मात्रा समनुज्ञाता राज्ञा च कुरुपुङ्गवाः}
{अभिवाद्य कुरुश्रेष्ठमामन्त्रयितुमारभन् ॥युधिष्ठिर उवाच}
{}


\twolineshloka
{राज्यं प्रति गमिष्यामः शिवेन प्रतिनन्दितः}
{अनुज्ञातास्त्वया राजन्गमिष्यामो विकल्मषाः}


\twolineshloka
{एवमुक्तः स राजर्षिर्धर्मराज्ञा महात्मना}
{अनुयज्ञे जयाशीर्भिः पूजयित्वा युधिष्ठिरम्}


\twolineshloka
{भीमं च बलिनां श्रेष्ठं सान्त्वयामास पार्थिवः}
{स चास्य सम्यङ्मेधावी प्रत्यपद्यत वीर्यवान्}


\twolineshloka
{अर्जुनं च समाश्लिष्य यंमौ च भरतर्षभौ}
{अनुयज्ञे स कौरव्यः परिष्वज्याभिनन्द्य च}


\twolineshloka
{गान्धार्या चाभ्यनुज्ञाताः कृतपादाभिवादनाः}
{जनन्या समुपाघ्राताः परिष्वक्ताश्च ते नृपम्}


\twolineshloka
{चक्रुः प्रदक्षिणं सर्वे वत्सा इव निवारणे}
{पुनः पुनर्निरीक्षन्तः प्रचक्रुस्ते प्रदक्षिणम्}


\twolineshloka
{द्रौपदीप्रमुखाश्चैव सर्वाः कौरवयोषितः}
{न्यायतः श्वशुरे वृत्तिं प्रयुज्य प्रययुस्ततः}


% Check verse!
श्वश्रूभ्यां समनुज्ञाताः परिष्वज्याभिनन्दिताः ॥संदिष्टाश्चेतिकर्तव्यं प्रययुर्भर्तृभिः सह
\twolineshloka
{ततः प्रजज्ञे निनदः सूतानां युज्यतामिति}
{उष्ट्राणां क्रोशतां चापि हयानां हेषतामपि}


\twolineshloka
{ततो युधिष्ठिरो राजा सदारः सहसैनिकः}
{नगरं हास्तिनपुरं पुनरायात्सबान्धवः}


\chapter{अध्यायः ३९}
\twolineshloka
{तेषु चोपनिवृत्तेषु पाण्डवेषु यदृच्छया}
{देवर्षिर्नारदो राजन्नाजगाम युधिष्ठिरम्}


\twolineshloka
{तमभ्यर्च्य महाबाहुः कुरुराजो युधिष्ठिरः}
{आसीनं परिविश्वस्तं प्रोवाच वदतांवरः}


\twolineshloka
{चिरात्तु नानुपश्यामि भगवन्तमुपस्थितम्}
{कच्चित्ते कुशलं विप्र शुभं वा प्रत्युपस्थितम्}


\fourlineindentedshloka
{के देशाः परिदृष्टास्ते किं च कार्यं करोमि ते}
{तद्ब्रूहि द्विजमुख्यस्त्वं ह्यस्माकं च प्रियोऽतिथिः}
{नारद उवाच}
{}


\threelineshloka
{चिरदृष्टोसि मेऽत्येवमागतोऽहं तपोवनात्}
{परिदृष्टानि तीर्थानि गङ्गा चैव मया नृप ॥युधिष्ठिर उवाच}
{}


\twolineshloka
{वदन्ति पुरुषा मेऽद्य गङ्गातीरनिवासिनः}
{धृतराष्ट्रं महात्मानमास्थितं परमं तपः}


\twolineshloka
{अपि दृष्टस्त्वया तत्र कुशली स कुरूद्वहः}
{गान्धारी च पृथा चैव सूतपुत्रश्च संजयः}


\threelineshloka
{कथं च वर्तते चाद्य पिता मम स पार्थिवः}
{श्रोतुमिच्छामि भगवन्यदि दृष्टस्त्वया नृपः ॥नारद उवाच}
{}


\twolineshloka
{स्थिरीभूय महाराज शृणु वृत्तं यथातथम्}
{यथाश्रुतं च दृष्टं च मया तस्मिंस्तपोवने}


\twolineshloka
{वनवासनिवृत्तेषु भवत्सु कुरुनन्दन}
{कुरुक्षेत्रात्पिता तुभ्य गङ्गाद्वारं ययौ नृप}


\twolineshloka
{गान्धार्या सहितो धीमान्वध्वा कुन्त्या समन्वितः}
{संजयेन च सूतेन साग्निहोत्रः सयाजकः}


\twolineshloka
{आतस्थे स तपस्तीव्रं पिता तव तपोधनः}
{अग्निं मुखे समाधाय वायुभक्षोऽभवन्मुनिः}


\twolineshloka
{वने स मुनिभिः सर्वैः पूज्यमानो महातपाः}
{त्वगस्थिमात्रशेषः स षण्मासानभवन्नृपः}


\twolineshloka
{गान्धारी तु जलाहारी कुन्ती मासोपवासिनी}
{संजयः षष्ठभुक्तेन वर्तयामास भारत}


\twolineshloka
{अग्नींस्तु याजकास्तत्र जुहुवुर्विधिवत्प्रभो}
{दृश्यतोऽदृश्यतश्चैव वने तस्मिन्नृपस्य वै}


\twolineshloka
{अनिकेतोथ राजा स बभूव वनगोचरः}
{ते चापि सहिते देव्यौ संजयश्च तमन्वयुः}


\twolineshloka
{संजयो नृपतेर्नेता समेषु विषमेषु च}
{गान्धार्याश्च पृथा चैव चक्षुरासीदनिन्दिता}


\twolineshloka
{ततः कदाचिद्गङ्गायाः कच्छे स नृपसत्तमः}
{गङ्गायामाप्लुतो धीमानाश्रमाभिमुखो ययौ}


\twolineshloka
{अथ वायुः समुद्भुतो दावाग्निरभवन्महान्}
{ददाह तद्वनं सर्वं परिगृह्य समन्ततः}


\twolineshloka
{दह्यत्सु मृगयूथेषु द्विजिह्वेषु समन्ततः}
{वराहाणां च यूथेषु संश्रयत्सु जलाशयान्}


\threelineshloka
{सम्प्रदीप्ते वने तस्मिन्प्राप्ते व्यसन उत्तमे}
{निराहारतया राजा मन्दप्राणविचेष्टितः}
{असमर्थोऽपसरणे सुकृशे मातरौ च ते}


\twolineshloka
{ततः स नृपतिर्दृष्ट्वा वह्निमायान्तमन्तिकात्}
{इदमाह ततः सूतं संजयं जयतांवरः}


\twolineshloka
{गच्छ संजय यत्राग्निर्न त्वां दहति कर्हिचित्}
{वयमत्राग्निना युक्ता गमिष्यामः परां गतिम्}


\twolineshloka
{तमुवाच किलोद्विग्नः संजयो वदतांवरः}
{राजन्मृत्युरनिष्टोऽयं भविता ते वृथाऽग्निना}


\twolineshloka
{न चोपायं प्रपश्यामि मोक्षणे जातवेदसः}
{यदत्रानन्तरं कार्यं तद्भवान्वक्तुमर्हति}


\twolineshloka
{इत्युक्तः संजयेनेदं पुनराह स पार्थिवः}
{नैष मृत्युरनिष्टो नो निःसृतानां गृहात्स्वयम्}


\twolineshloka
{जलमग्निस्तथा वायुरथवाऽपि विकर्षणम्}
{तापसानां प्रशस्यं ते गच्च संजय माचिरम्}


\twolineshloka
{इत्युक्त्वा संजयं राजा समाधाय मनस्तथा}
{प्राङ्मुखः सह गान्धार्या कुन्त्या चोपाविशत्तदा}


\twolineshloka
{संजयस्तं तथा दृष्ट्वा प्रदक्षिणमथाकरोत्}
{उवाच चैनं मेधावी युङ्क्ष्वात्मानमिति प्रभो}


\twolineshloka
{ऋषिपुत्रो मनीषी स राज चक्रेऽस्य तद्वचः}
{सन्निरुध्येन्द्रियग्राममासीत्काष्ठोपमस्तदा}


\twolineshloka
{गान्धारी च महाभागा जननी च पृथा तव}
{दावाग्निना समायुक्ते स च राजा पिता तव}


\twolineshloka
{संजयस्तु महाप्राज्ञस्तस्माद्दावादमुच्यत}
{गङ्गाकूले मया दृष्टस्तापसैः परिवारितः}


\twolineshloka
{स तानामन्त्र्य तेजस्वी निवेद्यैतच्च सर्वशः}
{प्रययौ संजयो धीमान्हिमवन्तं महीधरम्}


\twolineshloka
{एवं स निधनं प्राप्तः कुरुराजो महामनाः}
{गान्धारी च पृथा चैव जनन्यौ ते विशाम्पते}


\twolineshloka
{यदृच्छयाऽनुव्रजता मया राज्ञः कलेवरम्}
{तयोश्च देव्योरुभयोर्मया दृष्टानि भारत}


\twolineshloka
{ततस्तपोवने तस्मिन्समाजग्मुस्तपोधनाः}
{श्रुत्वा राज्ञस्तदा निष्ठां न त्वशोचन्गतीश्च ते}


\twolineshloka
{तत्राश्रौषमहं सर्वमेतत्पुरुषसत्तम}
{यथा च नृपतिर्दग्धो देव्यौ ते चेति पाण्डव}


\threelineshloka
{न शोचितव्यं राजेन्द्र स्वर्गस्थः पृथिवीपतिः}
{प्राप्तवानग्निसंयोगं गान्धारी जननी च ते ॥वैशम्पायन उवाच}
{}


\twolineshloka
{एतच्छ्रुत्वा च सर्वेषां पाण्डवानां महात्मनाम्}
{निर्याणं धृतराष्ट्रस्य शोकः समभवन्महान्}


\twolineshloka
{अन्तःपुराणां च तदा महानार्तस्वरोऽभवत्}
{पौराणां च महाराज श्रुत्वा राज्ञस्तदा गतिम्}


\twolineshloka
{अहो धिगिति राजा तु विक्रुश्य भृशदुःखितः}
{ऊर्ध्वबाहुः स्मरन्मातुः प्ररुरोद युधिष्ठिरः}


\twolineshloka
{भीमसेनपुरोगाश्च भ्रातरः सर्व एव ते}
{`रुरुदुर्दुःखसंतप्ता वर्णयन्तः पृथां तदा}


\twolineshloka
{अन्तःपुरेषु च तदा सुमहान्रुदितस्वनः}
{प्रादुरासीन्महाराज पृथां श्रुत्वा तथागताम्}


\twolineshloka
{तं च वृद्धं तथा दग्धं हतपुत्रं नराधिपम्}
{अन्वशोचन्त ते सर्वे गान्धारीं च तपस्विनीम्}


\twolineshloka
{तस्मिन्नुपरते शब्दे मुहूर्तादिव भारत}
{निगृह्य बाष्पं धैर्येण धर्मराजोऽब्रवीदिदम्}


\chapter{अध्यायः ४०}
\twolineshloka
{तथा महात्मनस्तस्य तपस्युग्रे च तस्थुषः}
{अनाथस्येव निधनं तिष्ठत्स्वस्मासु बन्धुषु}


\twolineshloka
{दुर्विज्ञेया गतिर्ब्रह्मन्पुरुषाणां मतिर्मम}
{यत्र वैचित्रवीर्योसौ दग्ध एवं वनाग्निना}


\twolineshloka
{यस्य पुत्रशतं श्रीमदभवद्बाहुशालिनः}
{नागायुतबलो राजा स दग्धो हि दवाग्निना}


\twolineshloka
{यं पुरा पर्यवीजन्त तालवृन्तैर्वरस्त्रियः}
{तं गृध्राः पर्यवीजन्त दावाग्निपरिकालितम्}


\twolineshloka
{सूतमागधसङ्घैश्च शयानो यः प्रबोध्यते}
{धरण्यां स नृपः शेते विकृष्टो गृध्रवायसैः}


\twolineshloka
{न च शोचामि गान्धारीं हतपुत्रां यशस्विनीम्}
{पतिलोकमनुप्राप्तां तथा भर्तृव्रते स्थिताम्}


\twolineshloka
{पृथामेव च शोचामि या पुत्रैश्वर्यमृद्धिमत्}
{उत्सृज्य् सुमहद्दीप्तं वनवासमरोचयत्}


\twolineshloka
{धिग्राज्यमिदमस्माकं धिग्बलं धिक्पराक्रमम्}
{क्षत्रधर्मं च धिग्यस्मान्मृता जीवामहे वयम्}


\twolineshloka
{सुसूक्ष्मा किल लोकस्य गतिर्द्विजवरोत्तम}
{यत्समुत्सृज्य राज्यं सा वनवासमरोचयत्}


\twolineshloka
{युधिष्ठिरस्य जननी भीमस्य विजयस्य च}
{अनाथवत्कथं दग्धा इति मुह्यामि चिन्तयन्}


\twolineshloka
{वृथा संतर्पितो वह्निः खाण्डवे सव्यसाचिना}
{उपकारमजानन्स कृतघ्न इति मे मतिः}


\twolineshloka
{यत्रादहत्स भगवान्मातरं सव्यसाचिनः}
{कृत्वा यो ब्राह्मणच्छद्म भिक्षार्थी समुपागतः}


\twolineshloka
{धिगग्निं धिक् च पार्थस्य विश्रुतां सत्यसन्धताम्}
{इदं कष्टतरं चान्यद्भगवन्प्रतिभाति मे}


\twolineshloka
{वृथाऽग्निना समायोगो यदभूत्पृथिवीपतेः}
{तथा तपस्विनस्तस्य राजर्षेः कौरवस्य ह}


% Check verse!
कथमेवंविधो मृत्युः प्रशास्य पृथिवीमिमाम्
\twolineshloka
{तिष्ठत्सु मन्त्रपूतेषु तस्याग्निषु महावने}
{वृथाऽग्निना समायुक्तो निष्ठां प्राप्तः पिता मम}


\twolineshloka
{मन्ये पृथा वेपमाना कृशा धमनिसंतता}
{हा तात धर्मराजेति मामाक्रन्दन्महाभये}


\twolineshloka
{भीम पर्याप्नुहि भयादिति चैवाभिवाशती}
{समन्ततः परिक्षिप्ता माताऽभून्मे दवाग्निना}


\twolineshloka
{सहदेवः प्रियस्तस्याः पुत्रेभ्योधिक एव तु}
{न चैनां मोक्षयामास वीरो माद्रवतीसुतः}


\twolineshloka
{तच्छ्रुत्वा रुरुदुः सर्वे समालिङ्ग्य परस्परम्}
{पाण्डवाःक पञ्च दुःखार्ता भूतानीव युगक्षये}


\twolineshloka
{तेषां तु पुरुषेन्द्राणां रुदतां रुदितस्वनः}
{प्रासादाभोगसंरुद्धे अन्वरौत्सीत्स रोदसी}


\chapter{अध्यायः ४१}
\twolineshloka
{नासौ वृथाऽग्निना दग्धो यथा तत्र श्रुतं मया}
{वैचित्रवीर्यो नृपतिर्न ते शोच्यो नराधिप}


\twolineshloka
{वनं प्रविशतानेन वायुभक्षेण धीमता}
{अग्नयः कारयित्वेष्टिमुत्सृष्टा इति नः श्रुतम्}


\twolineshloka
{याजकास्तु ततस्तस्य तानग्नीन्निर्जने वने}
{समुत्सृज्य यथाकामं जग्मुर्भरतसत्तम}


\twolineshloka
{स विवृद्धस्तदा वह्निर्वने तस्मिन्नभूत्किल}
{तेन तद्वनमादीप्तमिति ते तापसाऽब्रुवन्}


\twolineshloka
{स राजा जाह्नवीतीरे यथा ते कथितं मया}
{तेनाग्निना समायुक्तः स्वेनैव भरतर्षभ}


\twolineshloka
{एवमावेदयामासुर्मुनयस्ते ममानघ}
{ये ते भागीरथीतीरे मया दृष्टा युधिष्ठिर}


\twolineshloka
{एवं स्वेनाग्निना राजा समायुक्तो महीपते}
{मा शोचिथास्त्वं नृपतिं गतः स परमां गतिम्}


\twolineshloka
{गुरुशुश्रूषया चैव जननी ते जनाधिप}
{प्राप्ता सुमहतीं सिद्धिमिति मे नात्र संशयः}


\threelineshloka
{कर्तुमर्हसि राजेन्द्र तेषां त्वमुदकक्रियाम्}
{भ्रातृभिः सहितः सर्वैरेतदत्र विधीयताम् ॥वैशम्पायन उवाच}
{}


\twolineshloka
{ततः स पृथिवीपालः पाण्डवानां धुरंधरः}
{निर्ययौ सहसोदर्यः सदारश्च नरर्षभः}


\twolineshloka
{पौरजानपदाश्चैव राजभक्तिपुरस्कृताः}
{गङ्गां प्रजग्मुरभितो वाससैकेन संवृताः}


\twolineshloka
{ततोऽवगाह्य सलिले सर्वे तु कुरुपुङ्गवाः}
{युयुत्सुमग्रतः कृत्वा ददुस्तोयं महात्मने}


\twolineshloka
{गान्धार्याश्च पृथायाश्च विधिवन्नामगोत्रतः}
{शाचं निवर्तयन्तस्ते तत्रोषुर्नगराद्बहिः}


\threelineshloka
{प्रेषयामास स नरान्विधिज्ञानाप्तकारिणः}
{गङ्गाद्वारं कुरुश्रेष्ठो यत्र दग्धोऽभवन्नृपः}
{}


\twolineshloka
{तत्रैव तेषां तुल्यानि गङ्गाद्वारेऽन्वशात्तदा}
{कर्तव्यानीति पुरुषान्दत्तदेयान्महीपतिः}


\twolineshloka
{द्वादशेऽहनि तेभ्यः स कृतशौचो नराधिपः}
{ददौ श्राद्धानि विधिवद्दक्षइणावन्ति पाण्डवः}


\twolineshloka
{धृतराष्ट्रं समुद्दिश्य ददौ स पृथिवीपतिः}
{सुवर्णं रजतं गाश्च शय्याश्च सुमहाधनाः}


\twolineshloka
{गान्धार्याश्चैव तेजस्वी पृथायाश्च पृथक्पृथक्}
{सङ्कीर्त्य नामनी राजा ददौ दानमनुत्तमम्}


\twolineshloka
{यो यदिच्छति यावच्च तावत्स लभते द्विजः}
{शयनं भोजनं यानं मणिरत्नमथो धनम्}


\twolineshloka
{यानमाच्छादनं भोगान्दासीश्च समलङ्कृताः}
{ददौ राजा समुद्दिश्य तयोर्मात्रोर्महीपतिः}


\twolineshloka
{ततः स पृथिवीपालो दत्त्वा श्राद्धान्यनेकशः}
{प्रविवेश पुनर्धीमान्नगरं वारणाह्वयम्}


\twolineshloka
{ते चापि राजवचनात्पुरुषा ये गताऽभवन्}
{सङ्कल्प्य तेषां कुल्यानि पुनः प्रत्यागमंस्ततः}


\twolineshloka
{माल्यैर्गन्धैश्च विविधैरर्चयित्वा यथाविधि}
{कुल्यानि तेषां संयोज्य तदाचख्युर्महीपतेः}


\twolineshloka
{समाश्वास्य तु राजानं धर्मात्मानं युधिष्ठिरम्}
{नारदोप्यगमत्प्रीतः परमर्षिर्यथोप्सितम्}


\twolineshloka
{एवं वर्षाण्यतीतानि धृतराष्ट्रस्य धीमतः}
{वनवासे तथा त्रीणि नगरे दशपञ्च च}


\twolineshloka
{हतपुत्रस्य सङ्ग्रामे दानानि ददतः सदा}
{ज्ञातिसम्बन्धिमित्राणां भ्रातॄणां स्वजनस्य च}


\twolineshloka
{युधिष्ठिरस्तु नृपतिर्नातिप्रीतमनास्तदा}
{धारयामास तद्राज्यं निहतज्ञातिबान्धवः}


