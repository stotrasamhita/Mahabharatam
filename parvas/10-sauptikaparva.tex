\part{सौप्तिकपर्व}
\chapter{अध्यायः १}
\threelineshloka
{श्रीवेदव्यासाय नमः}
{नारायणं नमस्कृत्य नरं चैव नरोत्तमम्}
{देवीं सरस्वतीं व्यासं ततो जयमुदीरयेत्}


\threelineshloka
{सञ्जय उवाच}
{ततस्ते सहिताः सर्वे प्रयाता दक्षिणामुखाः}
{उपास्तमयवेलायां शिबिराभ्याशमागताः}


\twolineshloka
{विमुच्य वाहांस्त्वरिता भीताः समनुबोधनात्}
{गहनं देशमासाद्य प्रच्छन्ना न्यविशन्त ते}


\threelineshloka
{सेनानिवेशमभितो नातिदूरमवस्थिताः}
{[निकृत्ता निशितैः शस्त्रैः समन्तात्क्षतविक्षताः}
{]दीर्घमुष्णं च निःश्वस्य पाण्डवानन्वचिन्तयन्}


\twolineshloka
{श्रुत्वा च निनदं घोरं पाण्डवानां जयैषिणाम्}
{अनुसारभयाद्भीताः प्राङ्मुखाः प्राद्रवन्पुनः}


\twolineshloka
{ते मुहूर्तात्ततो गत्वा श्रान्तवाहा पिपासिताः}
{नामृष्यन्त महेष्वासाः क्रोधामर्षवशङ्गताः}


\twolineshloka
{राज्ञो वधेन सन्तप्ता मुहूर्तं समवस्थिताः ॥धृतराष्ट्र उवाच}
{}


\twolineshloka
{अश्रद्धेयमिदं कर्म कृतं मीमेन सञ्जय}
{यत्स नागायुतप्राणः पुत्रो मम निपातितः}


\twolineshloka
{अवध्यः सर्वभूतानां वज्रसंहननो युवा}
{पाण्डवैः समरे पुत्रो निहतो मम निपातितः}


\twolineshloka
{न दिष्टमभ्यतिक्रान्तुं शक्यं गावल्गणे नरैः}
{यत्समेत्य रणे पार्थैः पुत्रो मम निपातितः}


\twolineshloka
{अद्रिसारमयं नूनं हृदयं मम सञ्जय}
{हतं पुत्रशतं श्रुत्वा यन्न दीर्णं सहस्रधा}


\twolineshloka
{कथं हि वृद्धमिथुनं हतपुत्रं भविष्यति}
{न ह्यहं पाण्डुपुत्रस्य विषये वस्तुमुत्सहे}


\twolineshloka
{कथं राज्ञः पिता भूत्वा स्वयं राजा च सञ्जय}
{प्रेष्यभूतः प्रवर्तेयं पाण्डवेयस्य शासनात्}


\threelineshloka
{प्रभुज्य पृथिवीं सर्वां स्थिता मूर्धनि सञ्जय}
{कथमद्य भविष्यामि श्रोतुं शक्ष्यामि सञ्जय}
{}


\twolineshloka
{येन पुत्रशतं पूर्णमेकेन निहतं मम ॥कृतं सत्यं वचस्तस्य विदुरस्य महात्मनः}
{}


\twolineshloka
{अकुर्वता वचस्तस्य मम पुत्रेण सञ्जय ॥अधर्मेण हते तात पुत्रे दुर्योधने मम}
{}


\twolineshloka
{कृतवर्मा कृपो द्रौणिः किमकुर्वत सञ्जय ॥सञ्जय उवाच}
{}


\twolineshloka
{गत्वा तु तावका राजन्नातिदूरं मनस्विनः}
{अपश्यन्त वनं घोरं नानाद्रुमलतावृतम्}


\twolineshloka
{ते मुहूर्तं तु विश्रम्य लब्धतोयैर्हयोत्तमैः}
{सूर्यास्तमयवेलायां कौरवेयस्य शासनात्}


\twolineshloka
{नानामृगगणैर्जुष्टं नानापक्षिभिराकुलम्}
{नानाद्रुमलताच्छन्नं नानाव्यालनिषेवितम्}


\twolineshloka
{नानातोयसमाकीर्णैस्तटाकैरुपशोभितम्}
{पद्मिनीशतसञ्छन्नं नीलोत्पलसमायुतम्}


\twolineshloka
{प्रविश्य तद्वनं घोरं वीक्षमाणाः समन्ततः}
{शाखासहस्रसञ्छन्नं न्यग्रोधं ददृशुस्ततः}


\twolineshloka
{उपेत्य तु तदा राजन्न्यग्रोधं ते महारथाः}
{ददृशुर्द्विपदां श्रेष्ठाः श्रेष्ठं तं वै वनस्पतिम्}


\twolineshloka
{तेऽवतीर्य रथेभ्यश्च विप्रमुच्य च वाजिनः}
{उपस्पृश्य यथान्यायं सन्ध्यामन्वासत प्रभो}


\twolineshloka
{ततोऽस्तं पर्वतश्रेष्ठमनुप्राप्ते दिवाकरे}
{सर्वस्य जगतो धात्री शर्वरी प्रत्यपद्यत}


\twolineshloka
{ग्रहनक्षत्रताराभिः प्रकीर्णाभिरलङ्कृतम्}
{नभोंऽशुकमिवाभाति प्रेक्षणीयं समन्ततः}


\twolineshloka
{इच्छया ते प्रवल्गन्ति ये सत्वा रात्रिचारिणः}
{दिवाचराश्च ये सत्वास्ते निद्रावशमागताः}


\twolineshloka
{रात्रिञ्चराणां सत्वानां निनादोऽभूत्सुदारुणः}
{क्रव्यादाश्च प्रमुदिता घोरा प्राप्ता च शर्वरी}


\twolineshloka
{तस्मिन्रात्रिमुखे घोरे दुःखशोकसमन्विताः}
{कृतवर्मा कृपो द्रौणिरुपोपविविशुः समम्}


\twolineshloka
{उपोपविष्टाः शोचन्तो न्यग्रोधस्य समीपतः}
{तमेवार्थमतिक्रान्तं कुरुपाण्डवयोः क्षयम्}


\twolineshloka
{निद्रया च परीताङ्गा निषेदुर्धरणीतले}
{श्रमेण सुदृढं युक्ता विक्षता विविधैः शरैः}


\twolineshloka
{ततो निद्रावशं प्राप्तौ कृपभोजौ महाबालौ}
{सुखोचितावदुःखार्हौ निषण्णौ धरणीतले}


\twolineshloka
{तौ तु सुप्तौ महाराज तस्मिन्देशे महारथौ}
{[महार्हशयनोपेतौ भूमावेव ह्यनाथवत् ॥]}


\twolineshloka
{क्रोधामर्षवशं प्राप्तो द्रोणपुत्रस्तु भारत}
{न वै स्म स जगामाथ निद्रां सर्प इव श्वसन्}


\twolineshloka
{न लेभे स तु निद्रां वै दह्यमानोऽतिमन्युना}
{वीक्षाञ्चक्रे महाबाहुस्तद्वनं घोरदर्शनम्}


\twolineshloka
{वीक्षमाणो वनोद्देशं नानासत्वैर्निषेवितम्}
{अपश्यत महाबाहुर्न्यग्रोधं वायसावृतम्}


\twolineshloka
{तत्र काकसहस्राणि तां निशां पर्यणामयन्}
{सुखं स्वपन्तः कौरव्य पृथक्पृथगुपाश्रयाः}


\twolineshloka
{सुप्तेषु तेषु काकेषु विस्रब्धेषु समन्ततः}
{सोऽपश्यत्सहसा यान्तमुलूकं घोरदर्शनम्}


\twolineshloka
{महास्वनं महाकायं हर्यक्षं बभ्रुपिङ्गलम्}
{सुतीक्ष्णघोणानखरं सुपर्णमिव वेगितम्}


\twolineshloka
{सोऽथ शब्दं मृदुं कृत्वा लीयमान इवाण़्डजः}
{न्यग्रोधस्य ततः साखां पातयामास भारत}


\twolineshloka
{सन्निपत्य तु शाखायां न्यग्रोधस्य विहङ्गमः}
{सुप्ताञ्जघान विस्रब्धान्वायसान्वायसान्तकः}


\twolineshloka
{केषाञ्चिदच्छिनत्पक्षाञ्शिरांसि च चकर्त ह}
{चरणांश्चैव केषाञ्चिद्बभञ्ज चरणायुधः}


% Check verse!
क्षणेनाघ्नत्स बलवान्येऽस्य दृष्टिपथे स्थिताः
\twolineshloka
{तेषां शरीरावयवैः शरीरैश्च विशाम्पते}
{न्यग्रोधमण्डलं सर्वं सञ्छन्नं पर्वतोपमम्}


\twolineshloka
{तांस्तु हत्वा ततः काकान्कौशिको मुदितोऽभवत्}
{प्रतिकृत्य यथाकामं शत्रूणां शत्रुसूदनः}


\twolineshloka
{तद्दृष्ट्वा तादृशं कर्म कौशिकेन कृतं निशि}
{तद्भावे कृतसङ्कल्पो द्रौणिरेकोऽन्वचिन्तयत्}


\twolineshloka
{उपदेशः कृतोऽनेन पक्षिणा मम संयुगे}
{शत्रूणां क्षपणं युक्तं प्राप्तः कालश्च मे मतः}


\twolineshloka
{नाद्य शक्यं मया हन्तुं पाण्डवा जितकाशिनः}
{बलवन्तः कृतोत्साहा लब्धलक्षाः प्रहारिणः}


\twolineshloka
{राज्ञः सकाशे तेषां च प्रतिज्ञातो वधो मया}
{पतङ्गाग्निसमां वृत्तिमास्यायात्मविनाशिनीम्}


\twolineshloka
{न्यायतो युध्यमानस्य प्राणत्यागो न संशयः}
{छद्मना तु भवेत्सिद्विः शत्रूणां च क्षयो महान्}


\twolineshloka
{तत्र सशयितादर्थाद्योऽर्थो निःसंशयो भवेत्}
{तं जना बहुमन्यन्ते येऽर्थशास्त्रविशारदाः}


\twolineshloka
{यच्चाप्यत्र भवेत्कार्यं गर्हितं लोकनिन्दितम्}
{कर्तव्यं तन्मनुष्येण क्षत्रधर्मेण वर्तता}


\twolineshloka
{निन्दितानि च कर्माणि कुत्सितानि पदेपदे}
{सोपधानि कृतान्येव पाण्डवैरकृतात्मभिः}


\twolineshloka
{अस्मिन्नर्थे पुरा गीताः श्रूयन्ते धर्मवित्तमैः}
{श्लोका न्यायमवेक्षद्भिस्तत्त्वार्थास्तत्त्वदर्शिभिः}


\twolineshloka
{परिश्रान्ते विदीर्णे वा भुञ्जाने वाऽपि शत्रुभिः}
{प्रस्थाने वा प्रवेशे वा प्रहर्तव्यं रिपोर्बलम्}


\twolineshloka
{निद्रार्तमर्धरात्रे च तथा नष्टप्रणायकम्}
{भिन्नयोधं बलं यच्च द्विधा युक्तं च यद्भवेत्}


\twolineshloka
{इत्येवं निश्चयं चक्रे सुप्तानां निशि मारणे}
{पाण्डूनां सह पाञ्चालैर्द्रोणपुत्रः प्रतापवान्}


\twolineshloka
{स क्रूरां मतिमास्थाय विनिश्चित्य मुहुर्मुहुः}
{सुप्तौ प्राबोधयत्तौ तु मातुलं भोजमेव च}


\twolineshloka
{तौ प्रबुद्धौ महात्मानौ कृपभोजौ महाबालौ}
{नोत्तरं प्रतिपद्येतां तत्र युक्तं हिया वृतौ}


% Check verse!
स मुहूर्तमिव ध्यात्वा तावुभौ वाक्यमब्रवीत्
\twolineshloka
{हतो दुर्योधनो राजा एकवीरो मबाहलः}
{यस्यार्थे वैरमस्माभिरासक्तं पाण्डवैः सह}


\twolineshloka
{एकाकी बहुभिः क्षुद्रैराहवे शुद्धविक्रमः}
{पातितो भीमसेनेन एकादशचमूपतिः}


\twolineshloka
{वृकोदरेण क्षुद्रेण सुनृशंसमिदं कृतम्}
{मूर्धाभिषिक्तस्य शिरः पादेन परिमृद्गता}


\twolineshloka
{विनर्दन्ति च पाञ्चालाः क्ष्वेलन्ति च हसन्ति च}
{धमन्ति शङ्खाञ्शतशो हृष्टा घ्नन्ति च दुन्दुभीन्}


\twolineshloka
{वादित्रघोषस्तुमुलो विमिश्रः शङ्खनिःस्वनैः}
{अनिलेनेरितो घोरो दिशः पूरयतीव ह}


\twolineshloka
{अश्वानां हेषमाणानां गजानां चैव बृंहताम्}
{सिंहनादश्च शूराणां श्रूयते सुमहानयम्}


\twolineshloka
{दिशं प्राचीं समाश्रित्य हृष्टानां गच्छतां भृशम्}
{रथनेमिस्वनाश्चैव श्रूयन्ते रोमहर्षणाः}


\twolineshloka
{पाण्डवैर्धार्तराष्ट्राणां यदिदं कदनं कृतम्}
{वयमेव त्रयः शिष्टा अस्मिन्महति वैशसे}


\twolineshloka
{केचिन्नागशतप्राणाः केचित्सर्वास्त्रकोविदाः}
{निहताः पाण्डवैर्यस्मिन्मन्ये कालस्य पर्ययम्}


\twolineshloka
{एवमेतेन भाव्यं हि नूनं कार्येण तत्त्वतः}
{यथा ह्यस्येदृशी निष्ठा कृते यत्नेऽपि दुष्करे}


\twolineshloka
{भवतोस्तु यदि प्रज्ञा न मोहादपचीयते}
{व्यसनेऽस्मिन्महत्यर्थे यन्नः श्रेयस्तदुच्यताम्}


\chapter{अध्यायः २}
\twolineshloka
{कृप उवाच}
{}


\twolineshloka
{श्रुतं ते वचनं सर्वं हेतुयुक्तं मया विभो}
{ममापि तु वचः किञ्चिच्छृणुष्वाद्य महाभुज}


\twolineshloka
{आबद्धा मानुषाः सर्वे निबद्धाः कर्मणोर्द्वयोः}
{दैवे पुरुषकारे च परं ताभ्यां न विद्यते}


\twolineshloka
{न हि दैवेन सिध्यन्ति कार्याण्येकेन सत्तम}
{न चापि कर्मणैकेन द्वाभ्यां सिद्धिस्तु योगतः}


\twolineshloka
{ताभ्यामुभाभ्यां सर्वार्था निबद्धा अधमोत्तमाः}
{प्रवृत्ताश्चैव दृश्यन्ते निवृत्ताश्चैव सर्वशः}


\twolineshloka
{पर्जन्यः पर्वते वर्षन्किन्नु साधयते फलम्}
{कृष्टे क्षेत्रे तथा वर्षन्किं न साधयते फलम्}


\twolineshloka
{उत्थानं चापि दैवस्य ह्यनुत्थानं च दैवतम्}
{व्यर्थं भवति सर्वत्र पूर्वस्तत्र विनिश्चयः}


\twolineshloka
{सुवृष्टे च यथा देवे सम्यक् क्षेत्रे च कर्षिते}
{बीजं महागुणं भूयात्तथा सिद्धिर्हि मानुषी}


\twolineshloka
{तयोर्दैवं तु दुश्चिन्त्यं स्ववशेनैव वर्तते}
{प्राज्ञाः पुरुषकारे तु वर्तन्ते देवमास्थिताः}


\twolineshloka
{ताभ्यां सर्वे हि कार्यार्था मनुष्याणां नरर्षभ}
{विचेष्टन्तः स्म दृश्यन्ते निवृत्तास्तु तथैव च}


\twolineshloka
{कृतः पुरुषकारश्च सोऽपि दैवेन सिध्यति}
{तथास्य कर्मणः कर्तुरभिनिर्वर्तते फलम्}


\twolineshloka
{उत्थानं च मनुष्याणां दक्षाणां दैववर्जितम्}
{अभलं दृश्यते लोके सम्यगप्युपपादितम्}


\twolineshloka
{तत्रालसा मनुष्याणां ये भवन्त्यमनस्विनः}
{उत्थानं ते विगर्हन्ति प्राज्ञानां तन्न रोचते}


\twolineshloka
{प्रायशो हि कृतं कर्म नाफलं दृश्यते भुवि}
{अकृत्वा च पुनर्दुःखं कर्म पश्येन्महाफलम्}


\twolineshloka
{चष्टामकुर्वंल्लभते यदि किञ्चिद्यदृच्छया}
{यो वा न लभते कृत्वा दुर्दर्शौ तावुभावपि}


\twolineshloka
{शक्नोति जीवितुं दक्षो नालसः सुखमेधते}
{दृश्यन्ते जीवलोकेऽस्मिन्दक्षाः प्रायो हितैषिणः}


\twolineshloka
{यदि दक्षः समारम्भात्कर्मणो नाश्नुते फलम्}
{नास्य वाच्यं भवेत्किञ्चिल्लब्धव्यं वाऽधिगच्छति}


\twolineshloka
{नाकृत्वा कर्म लोके हि फलं विन्दति कर्हिचित्}
{स तु वक्तव्यतां याति द्वेष्यो भवति भूयशः}


\twolineshloka
{एवमेतदनादृत्य वर्तते यस्त्वतोऽन्यथा}
{स करोत्यात्मनोऽनर्थानेष बुद्धिमतां नयः}


\twolineshloka
{हीनं पुरुषकारेण यदि दैवेन वा पुनः}
{कारणाभ्यामथैताभ्यामुत्थानमफलं भवेत्}


% Check verse!
हीनं पुरुषकारेण कर्म त्विह न सिध्यति
\twolineshloka
{दैवतेभ्यो नमस्कृत्य यस्त्वर्थान्सम्यगीहते}
{दक्षो दाक्षिण्यसम्पन्नो न स मोघैर्विहन्यते}


\twolineshloka
{सम्यगीहा पुनरियं या बृद्धानुपसेवते}
{आपृच्छति च यच्छ्रेयः करोति च हितं वचः}


\twolineshloka
{उत्थायोत्थाय हि सदा प्रष्टव्या वृद्धसम्मताः}
{ते स्म योगे परं मूलं तन्मूला सिद्धिरुच्यते}


\twolineshloka
{वृद्धानां वचनं श्रुत्वा योऽभ्युत्थानं प्रयोजयेत्}
{उत्थानस्य फलं सम्यक्तदा स लभतेऽचिरात्}


\twolineshloka
{रागात्क्रोधाद्भयाल्लोभाद्योऽर्थानीहेत मानवः}
{अनीशश्चावमानी च स शीघ्रं भ्रश्यते श्रियः}


\twolineshloka
{सोयं दुर्योधनेनार्थो लुब्धेनादीर्घदर्शिना}
{असमर्थः समारब्धो मूढत्वादविचिनन्तितः}


\twolineshloka
{हितबुद्धीननादृत्य सम्मन्त्र्यासाधुभिः सह}
{वार्यमाणोऽकरोद्वैरं पाण्डवैर्गुणवत्तरैः}


\twolineshloka
{पूर्वमप्यतिदुःशीलो न धैर्यं कर्तुमर्हति}
{तपत्यर्थे विपन्ने हि मित्राणां न कृतं वचः}


\twolineshloka
{अनुवर्तामहे यत्तु तं वयं पापपूरुषम्}
{अस्मानप्यनयस्तस्मात्प्राप्तोऽयं दारुणो महान्}


\twolineshloka
{अेन तु ममाद्यापि व्यसनेनोपतापिता}
{बुद्धिश्चिन्तयते किञ्चित्स्वं श्रेयो नावबुध्यते}


\twolineshloka
{मुह्यता तु मनुष्येण प्रष्टव्याः सुहृदो जनाः}
{तत्रास्य बुद्धिर्विनयस्तत्र श्रेयश्च पश्यति}


\twolineshloka
{ततोऽस्य मूलं कार्याणां बुद्ध्या निश्चित्य वै बुधाः'}
{तेऽत्र पृष्टा यथा ब्रूयुस्तत्कर्तव्यं तथा भवेत्}


\twolineshloka
{ते वयं धृतराष्ट्रं च गान्धारीं च यशस्विनीम्}
{उपपृच्छामहे गत्वा विदुरं च महामतिम्}


\twolineshloka
{ते पृष्टास्तु वदेयुर्यच्छ्रेयो नः समनन्तरम्}
{तदस्माभिः पुनः कार्यमिति मे नैष्ठिकी मतिः}


% Check verse!
अनारम्भात्तु कार्याणां नार्थः सम्पद्यते क्वचित्
\twolineshloka
{कृते पुरुषकारे तु येषां कार्यं न सिध्यति}
{दैवेनोपहतास्ते तु नात्र कार्या विचारणा}


\chapter{अध्यायः ३}
\twolineshloka
{सञ्जय उवाच}
{}


\twolineshloka
{कृपस्य वचनं श्रुत्वा धर्मार्थसहितं शुभम्}
{अश्वत्थामा महाराज दुःखशोकसमन्वितः}


\twolineshloka
{दह्यमानस्तु शोकेन प्रदीप्तेनाग्निना यथा}
{क्रूरं मनस्ततः कृत्वा तावभौ प्रत्यभाषत}


\twolineshloka
{पुरुषेपुरुषे बुद्धिर्याया भवति शोभना}
{तुष्यन्ति च पृथक्सर्वे प्रज्ञया ते स्वयास्वया}


\twolineshloka
{सर्वो हि मन्यते लोक आत्मानं बुद्धिमत्तरम्}
{सर्वस्यात्मा बहुमतः सर्वोत्मानं प्रशंसति}


\twolineshloka
{सर्वस्य हि स्वका प्रज्ञा साधुवादे प्रतिष्ठिता}
{परबुद्धिं च निन्दन्ति स्वां प्रशंसन्ति चासकृत्}


\twolineshloka
{कारणान्तरयोगेन येषां संवदते मतिः}
{तेऽन्योन्येन च तुष्यन्ति बहुमन्यन्ति चासकृत्}


\twolineshloka
{तस्यैव तु मनुष्यस्य सासा बुद्धिस्तदातदा}
{कालयोगे विपर्यासं प्राप्यान्योन्यं विपद्यते}


\twolineshloka
{अनित्यत्वात्तु चित्तानां मनुष्याणां विशेषतः}
{चित्तवैक्लब्यमासाद्य सासा बुद्धिः प्रजायते}


\twolineshloka
{यथा हि वैद्यः कुशलो ज्ञात्वा व्याधिं यथाविधि}
{भैषज्यं कुरुते योगात्प्रशमार्थमिति प्रभो}


% Check verse!
एवं कार्यस्य योगात्प्रशमार्थमिति प्रभो ॥प्रज्ञया च स्वया युक्त्या तां च गृह्णन्ति वै बुधाः
\twolineshloka
{अन्यया यौवने बाल्ये बुद्ध्या भवति मोहितः}
{मध्येऽन्यया जरायां तु सोन्यां रोचयते मतिम्}


\twolineshloka
{व्यसन वा महाघोरं समृद्धिं चापि तादृशीम्}
{अवाप्य पुरुषो भोज कुरुते बुद्धिवैकृतम्}


\twolineshloka
{एकस्मिन्नेव पुरुषे सासा बुद्धिस्तदातदा}
{भवत्यनित्या प्रज्ञा हि सा तस्यैव न रोचते}


\twolineshloka
{निश्चित्य तु यथाप्रज्ञं यां मतिं साधु पश्यति}
{तया प्रकुरुते भावं सा तस्योद्योगकारिका}


\twolineshloka
{सर्वो हि पुरुषो भोज साध्वेतदिति निश्चितः}
{कर्तुमारभते प्रीतिं मरणादिषु कर्मसु}


\twolineshloka
{सर्वे हि युक्तां विज्ञाय प्रज्ञां वापि स्वकां नराः}
{चेष्टन्ते विविधां चेष्टां हितमित्येव जानते}


\twolineshloka
{उपजाता व्यसनजा येयमद्य मतिर्मम}
{युवयोस्तां प्रवक्ष्यामि सर्वेषां शोकनाशिनीम्}


\twolineshloka
{प्रजापतिः प्रजाः सृष्ट्वा कर्म तासु विधाय च}
{वर्णेवर्णे समाधत्त ह्येकैकं गुणवत्तरम्}


\twolineshloka
{ब्राह्मणे वेदमग्र्यं तु क्षत्रिये तेज उत्तमम्}
{दाक्ष्यं वैश्ये च शूद्रे च सर्ववर्णानुकूलताम्}


\twolineshloka
{अदान्तो ब्राह्मणोऽसाधुर्निस्तेजाः क्षत्रियो मृतः}
{अदक्षो निन्द्यते वैश्यः शूद्रश्च प्रतिकूलवान्}


\twolineshloka
{सोऽस्मि जातः कुले श्रेष्ठे ब्राह्मणैरभिपूजिते}
{मन्दभाग्यतयाऽस्म्येतं क्षत्रधर्ममनुष्ठितः}


\twolineshloka
{क्षत्रधर्मं विदित्वाऽहं यदि ब्राह्मण्यमाश्रितः}
{प्रकरिष्ये महत्कर्म न मे तत्साधुसम्मतम्}


\twolineshloka
{धारयित्वा धनुर्दिव्यं दिव्यान्यस्त्राणि चाहवे}
{पितरं निहतं दृष्ट्वा किन्नु वक्ष्यामि संसदि}


\twolineshloka
{सोऽहमद्य यथाकामं क्षत्रधर्ममवाप्य च}
{गन्ताऽस्मि पदवीं राज्ञः पितुश्चापि महात्मनः}


\threelineshloka
{अद्य स्वप्स्यन्ति पाञ्चाला विश्वस्ता जितकाशिनः}
{विमुक्तयुग्यकवचा हर्षेण च समन्विताः}
{वयं जिता मताश्चैषां श्रान्ता व्यायामकर्शिताः}


\twolineshloka
{तेषां निशि प्रसुप्तानां सुस्थानां शिबिरे स्वके}
{अवस्कन्दं करिष्यामि शिबिरस्याद्य दुष्करम्}


\twolineshloka
{तानवस्कन्द्य शिबिरे प्रेतभूतानचेतसः}
{सूदयिष्यामि विक्रम्य मघवानिव दानवान्}


\twolineshloka
{अद्य तान्सहितान्सर्वान्धृष्टद्युम्नपुरोगमान्}
{सूदयिष्यामि विक्रम्य कक्षं दीप्त इवानलः}


% Check verse!
निहत्य चैव पाञ्चालाञ्शान्तिं लब्धाऽस्मि सत्तम
\twolineshloka
{पाञ्चालेषु चरिष्यामि सूदयन्नद्य संयुगे}
{पिनाकपाणिः सङ्क्रुद्धः स्वयं रुद्रः पशुष्विव}


\threelineshloka
{अद्याहं सर्वपाञ्चालान्निकृत्या च निकृष्य च}
{अर्दयिष्यामि संहृष्टो रणे पाण्डुसुतांस्तथा}
{`सूदयिष्यामि सङ्क्रुद्धः पशूनिव पिनाकधृत्'}


\twolineshloka
{अद्याहं सर्वपाञ्चालैः कृत्वा भूमिं शरीरिणीम्}
{प्रहृत्यैकेन शस्त्रेण भविष्याम्यनृणः पितुः}


\twolineshloka
{दुर्योधनस्य कर्णस्य भीष्मसैन्धवयोरपि}
{गमिष्यामि निशावेलां पदवीमद्य दुर्गमाम्}


\twolineshloka
{अद्य पाञ्चालराजस्य धृष्टद्युम्नस्य वै निशि}
{विरात्रे प्रमथिष्यामि पशोरिव शिरो बलात्}


\twolineshloka
{अद्य पाञ्चालपाण्डूनां शयितानां शिरो निशि}
{खङ्गेन निशितेनाजौ प्रमथिष्यामि गौतम}


\twolineshloka
{अद्य पाञ्चालसेनां तां निहत्य निशि सौप्तिके}
{कृतकृत्यः सुखी चैव भविष्यामि महामते}


\chapter{अध्यायः ४}
\twolineshloka
{कृप उवाच}
{}


\twolineshloka
{दिष्ट्या ते प्रतिकर्तव्ये मतिर्जातेयमच्युत}
{न त्वां वारयितुं शक्तो वज्रपाणिरपि स्वयम्}


\twolineshloka
{अनुयास्यावहे त्वां तु प्रभाते सहितावुभौ}
{अद्य रात्रौ विश्रमस्व विमुक्तकवचध्वजः}


\twolineshloka
{अहं त्वामनुयास्यामि कृतवर्मा च सात्वतः}
{परानभिमुखं यान्तं रथावास्थाय दंशितौ}


\twolineshloka
{आवाभ्यां सहितः शत्रूञ्श्वो निहन्ता समागमे}
{विक्रम्य रथिनां श्रेष्ठ पाञ्चालान्सपदानुगान्}


\twolineshloka
{शक्तस्त्वमसि विक्रम्य विश्रमस्व निशामिमाम्}
{चिरं ते जाग्रतस्तात स्वप तावन्निशामिमाम्}


\twolineshloka
{विश्रान्तश्च विनिद्रश्च स्वस्थचित्तश्च मानद}
{समेत्य समरे शत्रून्वधिष्यसि न संशयः}


\twolineshloka
{न हि त्वां रथिनां श्रेष्ठं प्रगृहीतवरायुधम्}
{जेतुमुत्सहते कश्चिदपि देवेषु पावकिः}


\twolineshloka
{कृपेण सहितं यान्तं गुप्तं च कृतवर्मणा}
{को द्रौणिं युधि संरब्धं योधयेदपि देवराट्}


\twolineshloka
{ते वयं निशि विश्रान्ता विनिद्रा विगतज्वराः}
{प्रभातायां रजन्यां वै निहनिष्याम शात्रवान्}


\twolineshloka
{तव ह्यस्त्राणि दिव्यानि मम चैव न संशयः}
{सात्वतोपि महेष्वासो नित्यं युद्धेषु कोविदः}


\twolineshloka
{ते वयं सहितास्तात सर्वाञ्शत्रून्समागतान्}
{प्रसह्य समरे हत्वा प्रीतिं प्राप्स्याम पुष्कलाम्}


\twolineshloka
{विश्रमस्व त्वमव्यग्रः स्वप चेमां निशां सुखम्}
{अहं च कृतवर्मा च प्रभाते त्वां नरोत्तमम्}


\twolineshloka
{अनुयास्याव सहितौ धन्विनौ परतापनौ}
{रथिनं त्वरया यान्तं रथमास्थाय दंशितौ}


\twolineshloka
{स गत्वा शिबिरं तेषां नाम विश्राव्य चाहवे}
{ततः कर्ताऽसि शत्रूणां युध्यतां कदनं महत्}


\twolineshloka
{कृत्वा च कदनं तेषां प्रभाते विमलेऽहनि}
{विहरस्व यथा शक्रः सूदयित्वा महासुरान्}


\twolineshloka
{त्वं हि शक्तो रणे जेतुं पाञ्चालानां वरूथिनीम्}
{दैत्यसेनामिव क्रुद्धः सर्वदानवसूदनः}


\twolineshloka
{मया त्वां सहितं सङ्ख्ये गुप्तं च कृतवर्मणा}
{न सहेत विभुः साक्षाद्वज्रपाणिरपि स्वयम्}


\twolineshloka
{न चाहं समरे तात कृतवर्मा न चैव हि}
{अनिर्जित्य रणे पाण्डूनपयास्यामि कर्हिचित्}


\twolineshloka
{हत्वा च समरे क्षुद्रान्पाञ्चालान्पाण्डुभिः सह}
{निवर्तिष्यामहे सर्वे हता वा स्वर्गगा वयम्}


\twolineshloka
{सर्वोपायैः सहायास्ते प्रभाते वयमाहवे}
{सत्यमेतन्महाबाहो प्रब्रवीमि तवानघ}


\twolineshloka
{एवमुक्तस्ततो द्रौणिक्रमातुलेन हितं वचः}
{अब्रवीन्मातुलं राजन्क्रोधादुद्वृत्य लोचने}


\twolineshloka
{आतुरस्य कुतो निद्रा नरस्यामर्षितस्य च}
{अर्थांश्चिन्तयतश्चापि कामयानस्य वा पुनः}


\twolineshloka
{तदिदं समनुप्राप्तं पश्य मेऽद्य चतुष्टयम्}
{यस्य भागश्चतुर्थो मे स्वप्नमह्नाय नाशयेत्}


\twolineshloka
{किं नाम दुःखं लोकेऽस्मिन्पितुर्वधमनुस्मरन्}
{हृदयं निर्दहन्मेऽद्य रात्र्यहानि न शाम्यति}


\twolineshloka
{यथा च निहतः पापैः पिता मम विशेषतः}
{प्रत्यक्षमपि ते सर्वं तन्मे मर्माणि कृन्तति}


\twolineshloka
{कथं हि मादृशो लोके मुहूर्तमपि जीवति}
{द्रोणहन्तेति यद्वाचः पाञ्चालानां शृणोम्यहम्}


\twolineshloka
{धृष्टद्युम्नमहत्वा तु नादं जीवितुमुत्सहे}
{स मे पितुर्वधाद्वध्यः पाञ्चाला ये च सङ्गताः}


\twolineshloka
{विलापो भग्नसक्थस्य यस्तु राज्ञो मया श्रुतः}
{स पुनर्हृदयं कस्य क्रूरस्यापि न निर्दहेत्}


\twolineshloka
{कस्य ह्यकरुणस्यापि नेत्राभ्यामश्रु नाव्रजेत्}
{नृपतेर्भग्नसक्थस्य श्रुत्वा तादृग्वचः पुनः}


\twolineshloka
{यश्चायं मित्रपक्षो मे मयि जीवति निर्जितः}
{शोकं मे वर्धयत्येष वारिवेग इवार्णवम्}


% Check verse!
एकाग्नमनसो मेऽद्य कुतो निद्रा कुतः सुखम्
\twolineshloka
{वासुदेवार्जुनाभ्यां च तानहं परिरक्षितान्}
{अविषह्यतमान्मन्ये महेन्द्रेणापि सत्तम}


\twolineshloka
{न चापि शक्तः संयन्तुमस्मात्कार्यात्कथञ्चन}
{तं न पश्यामि लोकेऽस्मिन्यो मां कोपान्निवर्तयेत्}


% Check verse!
इति मे निश्चिता बुद्धिरेषा साधुमता मम
\twolineshloka
{वादिकैः कथ्यमानस्तु मित्राणां मे पराभवः}
{पाण्डवानां च विजयटो हृदयं दहतीव मे}


\twolineshloka
{अहं तु कदनं कृत्वा शत्रूणामद्य सौप्तिके}
{ततो विश्रमिता चैव स्वप्ता च विगतज्वरः}


\chapter{अध्यायः ५}
\twolineshloka
{कृप उवाच}
{}


\twolineshloka
{शुश्रूषुरपि दुर्मेधाः पुरुषोऽनियतेन्द्रियः}
{नालं वेदयितुं कृत्स्नौ धर्मार्थाविति मे मतिः}


\twolineshloka
{तथैव तावन्मेधावी विनयं यो न शिक्षते}
{न च किञ्चन जानाति सोऽपि धर्मार्थनिश्चयम्}


\twolineshloka
{[चिरं ह्यपि जडः शूरः पण्डितं पर्युपास्य ह}
{न स धर्मान्विजानाति दर्वी सूपरसानिव}


\twolineshloka
{मुहूर्तमपि तं प्राज्ञः पण्डितं पर्युपास्य हि}
{क्षिप्रं धर्मान्विजानाति जिह्वा सूपरसानिव}


\twolineshloka
{शुश्रूषुस्त्वेव मेधावी पुरुषो नियतेन्द्रियः}
{जानीयादागमान्सर्वान्ग्राह्यं च न विरोधयेत्}


\twolineshloka
{अनयस्त्ववमानी यो दुरात्मा पापपूरुषः}
{दिष्टमुत्सृज्य कल्याणं करोति बहुपातकम्}


\twolineshloka
{नाथवन्तं तु सुहृदः प्रतिषेधन्ति पातकात्}
{निवर्तते तु लक्ष्मीवान्नालक्ष्मीवान्निवर्तते}


\twolineshloka
{यथा ह्युच्चावचैर्वाक्यैः क्षिप्तचित्तो नियम्यते}
{तथैव सुहृदा शक्यो नशक्यस्त्ववसीदति}


\twolineshloka
{तथैव सुहृदोऽप्राज्ञान्कुर्वाणान्कर्म पापकम्}
{प्राज्ञाः सम्प्रतिषेधन्ति यथाशक्ति पुनःपुनः}


\twolineshloka
{स कल्याणे मनः कृत्वा नियम्यात्मानमात्मना}
{कुरु मे वचनं तात येन पश्चान्न तप्स्यसे}


\twolineshloka
{न वधः पूज्यते लोके सुप्तानामिह धर्मतः}
{तथैव न्यस्तशस्त्राणां विमुक्तरथवाजिनाम्}


\twolineshloka
{ये व ब्रूयुस्तवास्मीति ये च स्युः शरणागताः}
{विमुक्तमूर्धजा ये च ये चापि हतवाहनाः}


\twolineshloka
{अद्य स्वप्स्यन्ति पाञ्चाला विमुक्तकवचा विभो}
{विश्वस्ता रजनीं सर्वे प्रेता इव विचेतसः}


\twolineshloka
{यस्तेषां तदवस्थानां द्रुह्येत पुरुषोऽनृजुः}
{व्यक्तं स नरके मज्जेदगाधे विपुलेऽप्लुवे}


\twolineshloka
{सर्वास्त्रविदुषां लोके श्रेष्ठस्त्वमसि विश्रुतः}
{न च ते जातु लोकेऽस्मिन्सुसूक्ष्ममपि किल्बिषम्}


\twolineshloka
{त्वं पुनः सूर्यसङ्काशः श्वोभूत उदिते रवौ}
{प्रकाशे सर्वभूतानां विजेता युधि शात्रवान्}


\threelineshloka
{असम्भावितरूपं हि त्वयि कर्म विगर्हितम्}
{शुक्ले रक्तमिव न्यस्तं भवेदिति मतिर्मम ॥अश्वत्थामोवाच}
{}


\twolineshloka
{एवमेव यथाऽऽत्थ त्वमनुशाससि मातुल}
{तैस्तु पूर्वमयं सेतुः समन्ताद्विह्वलीकृतः}


\twolineshloka
{प्रत्यक्षं भूमिपालानां भवतां चापि सन्निधौ}
{न्यस्तशस्त्रो मम पिता धृष्टद्युम्नेन पातितः}


\twolineshloka
{कर्णश्च पतिते चक्रे उत्थास्यन्रथिनां वरः}
{उत्तमे व्यसने मग्नो हतो गाण्डीवधन्वना}


\twolineshloka
{तथा शान्तनवो भीष्मो न्यस्तशस्त्रो निरायुधः}
{शिखण्डिनं पुरस्कृत्य हतो गाण्डीवधन्वना}


\twolineshloka
{भूरिश्रवा महेष्वासस्तथा प्रायगतो रणे}
{क्रोशतां बूमिपालानां युयुधानेन पातितः}


\twolineshloka
{दुर्योधनश्च भीमेन समेत्य गदया मृधे}
{पश्यतां भूमिपालानामधर्मेण निपातितः}


\twolineshloka
{एकाकी बहुभिस्तत्र परिवार्य महारथैः}
{अधर्मेण नरव्याघ्रो भीमसेनेन पातितः}


\twolineshloka
{विलापो भग्नसक्थस्य यो मे राज्ञः परिश्रुतः}
{वादिकानां कथयतां स मे मर्माणि कृन्तति}


\twolineshloka
{एवं चाधार्मिकाः पापाः पाञ्चाला भिन्नसतवः}
{तानेवं भिन्नमर्यादान्किं भवान्न विगर्हति}


\twolineshloka
{पितृहन्तॄनहं हत्वा पाञ्चालान्निशि सौप्तिके}
{कामं कीटः पतङ्गो वा जन्म प्राप्य भवामि वै}


\twolineshloka
{त्वरे चाहमनेनाद्य यदिदं मे चिकीर्षितम्}
{तस्य मे त्वरमाणस्य कुतो निद्रा कुतः सुखम्}


\threelineshloka
{न स जातः पुमाँल्लोके कश्चिन्न स भविष्यति}
{यो मे व्यावर्तयेदेतां वधे तेषां कृतां मतिम् ॥सञ्जय उवाच}
{}


\twolineshloka
{एवमुक्त्वा महाराज द्रोणपुत्रः प्रतापवान्}
{एकान्ते योजयित्वाऽश्वान्प्रायादभिमुखः परान्}


\twolineshloka
{तमब्रूतां महात्मानौ भोजशारद्वतावुभौ}
{किमयं स्यन्दनो युक्तः किं च कार्यं चिकीर्षितम्}


\twolineshloka
{एकसामर्थप्रयातौ स्वस्त्वया सह नरर्षभ}
{समदुःखसुखौ चापि तस्माच्छंसितुमर्हसि}


\twolineshloka
{अश्वत्थामा तुं सङ्क्रुद्धः पितुर्वधमनुस्मरन्}
{ताभ्यां तथ्यं तथाऽऽचख्यौ यदस्यात्मचिकीर्षितम्}


\twolineshloka
{हत्वा शतसहस्राणि योधानां निशितैः शरैः}
{न्यस्तशस्त्रो मम पिता धृष्टद्युम्नेन पातितः}


\twolineshloka
{तं तथैव वधिष्यामि न्यस्तवर्माणमद्य वै}
{पुत्रं पाञ्चालराजस्य पापं पापेन कर्मणा}


\twolineshloka
{तथा विनिहतः पापः पाञ्चाल्यः पशुवन्मया}
{शस्त्रेण विजिताँल्लोकान्नाप्नुयादिति मे मतिः}


\twolineshloka
{क्षिप्रं सन्नद्वकवचौ सखङ्गावात्तकार्मुकौ}
{मामेवाद्य प्रतीक्षेतां रथवर्यौ परन्तपौ}


\twolineshloka
{इत्युक्त्वा रथमास्थाय प्रायादभिमुखः परान्}
{तमन्वगात्कृपो राजन्कृतवर्मा च सात्वतः}


\twolineshloka
{ते प्रयाता व्यरोचन्त परानभिमुखास्त्रयः}
{हूयमाना यथा यज्ञे समिद्धा हव्यवाहनाः}


\twolineshloka
{ययुश्च शिबिरं तेषां सम्प्रसुप्तजनं विभो}
{द्वारदेशमनुप्राप्य द्रौणिस्तस्थौ महारथः}


\chapter{अध्यायः ६}
\twolineshloka
{धृतराष्ट्र उवाच}
{}


\threelineshloka
{द्वारदेशे ततो द्रौणिमवस्थितमवेक्ष्य तौ}
{अकुर्वतां भोजकृपौ किं सञ्जय वदस्व मे ॥स़ञ्जय उवाच}
{}


\twolineshloka
{कृतवर्माणमामन्त्र्य कृपं च स महारथः}
{द्रौणिर्मन्युपरीतात्मा शिबिरद्वारमासदत्}


\twolineshloka
{तत्र भूतं महाकायं चन्द्रार्कसदृशद्युतिम्}
{सोऽपश्यद्द्वारमावृत्य तिष्ठन्तं रोमहर्षणम्}


\twolineshloka
{वसानं चर्म वैयाघ्रं वसारुधिरविस्रवम्}
{कृष्णाजिनोत्तरासङ्गं नागयज्ञोपवीतिनम्}


\twolineshloka
{बाहुभिः स्वायतैर्भीमैर्नानाप्रहरणोद्यतैः}
{बद्धाङ्गदमहासर्पं ज्वालामालाकुलाननम्}


\twolineshloka
{दंष्ट्राकरालवदनं व्यादितास्यं भयानकम्}
{नयनानां सहस्रैश्च विचित्रैरभिभूषितम्}


\twolineshloka
{नैव तस्य वपुः शक्यं प्रवक्तुं वेष एव च}
{सर्वथा तु तदालक्ष्य स्फुटेयुरपि पर्वताः}


\twolineshloka
{तस्यास्यनासिकाभ्यां च श्रवणाभ्यां च सर्वशः}
{तेभ्यश्चाक्षिसहस्रेभ्यः प्रादुरासन्महार्चिषः}


\twolineshloka
{तथा तेजोमरीचिभ्यः शङ्खचक्रगदाधराः}
{प्रादुरासन्हृषीकेशाः शतशोऽथ सहस्रशः}


\twolineshloka
{तदत्यद्भुतमालोक्य भूतं लोकभयङ्करम्}
{द्रौणिरव्यथितो दिव्यैरस्त्रवर्षैरवाकिरत्}


\twolineshloka
{द्रौणिमुक्ताञ्छरांस्तांस्तु तद्भूतं महदग्रसत्}
{उदधेरिव वार्योघान्पावको बडबामुखः}


\twolineshloka
{अश्वत्थामा तु सम्प्रेक्ष्य शरौघांस्तान्निरर्थकान्}
{रथशक्तिं मुमोचास्मै दीप्तामग्निशिखामिव}


\twolineshloka
{सा तमाहत्य दीप्ताग्रा रथशक्तिरदीर्यत}
{युगान्ते सूर्यमाहत्य महोल्केव दिवश्च्युता}


% Check verse!
अथ हेमत्सरुं दिव्यं खङ्गमाकाशवर्चसम् ॥कोशात्समुद्बबर्हाशु बिलाद्दीप्तमिवोरगम्
\twolineshloka
{ततः खङ्गवरं धीमान्भूताय प्राहिणोत्तदा}
{स तदासाद्य भूतं वै विलयं तूलवद्ययौ}


\twolineshloka
{ततः स कुपितो द्रौणिरिन्द्रकेतुनिभां गदाम्}
{ज्वलन्तीं प्राहिणोत्तस्मै भूतं तामपि चाग्रसत्}


\twolineshloka
{ततः सर्वायुधाभावे वीक्षमाणस्ततस्ततः}
{अपश्यत्कृतमाकाशमनाकाशं जनार्दनैः}


\twolineshloka
{तदद्भुततमं दृष्ट्वा द्रोणपुत्रो निरायुधः}
{अचिन्तयत्सुसन्त्रस्तः कृपभोजवचः स्मरन्}


\twolineshloka
{ब्रुवतामप्रियं पथ्यं सुहृदां न शृमोति यः}
{स शोचत्यापदं प्राप्य थाऽहमवमत्य तौ}


\twolineshloka
{शास्त्रदृष्टानविद्वान्यः समतीत्य जिघांसति}
{स पथः प्रच्युतो धर्म्यात्कुपथे प्रतिहन्यते}


\twolineshloka
{गोब्राह्मणनृपस्त्रीषु सख्युर्मातुर्गुरोस्तथा}
{वृद्धबालजडान्धेषु सुप्तभीतोत्थितेषु च}


\twolineshloka
{मत्तोन्मत्तप्रमत्तेषु न शस्त्राणि च मातयेत्}
{इत्येवं गुरुभिः पूर्वमुपदिष्टं नृणां सदा}


\twolineshloka
{सोऽहमुत्क्रम्य पन्थानं शास्त्रदृष्टं सनातनम्}
{अमार्गेणैवमारभ्य घोरामापदमागतः}


\twolineshloka
{तां चापदं घोरतरां प्रवदन्ति मनीषिणः}
{यदुद्यम्य महत्कृत्यं भयादपि निवर्तते}


\twolineshloka
{अशक्यं चैव कः कर्तुं शक्तः शक्तिबलादिह}
{न हि दैवाद्गरीयो वै मानुष्यं किञ्चिदिष्यते}


\twolineshloka
{मानुष्यं कुर्वतः कर्म यदि दैवान्न सिध्यति}
{स पथः प्रच्युतो धर्म्याद्विपदं प्रतिपद्यते}


\twolineshloka
{प्रतिज्ञानं ह्यविज्ञानं प्रवदन्ति मनीषिणः}
{यदारभ्य क्रियां काञ्चिद्भयादिह निवर्तते}


\twolineshloka
{तदिदं दुष्प्रणीतेन भयं मा समुपस्थितम्}
{न हि द्रोणसुतः सङ्ख्ये निवर्तेत कथञ्चन}


\twolineshloka
{इदं च सुमहद्भूतं दैवदण्डमिवोद्यतम्}
{न चैतदभिजानामि चिन्तयन्नपि सर्वथा}


\twolineshloka
{ध्रुवं येयमधर्मेण प्रहिता कलुषा मतिः}
{तस्याः फलमिदं घोरं प्रतिघाताय कल्पते}


\twolineshloka
{तदिदं दैवविहितं मम सङ्ख्ये निवर्तनम्}
{नान्यत्र दैवादुद्यन्तुमिह शक्यं कथञ्चन}


\twolineshloka
{सोऽहमद्य महादेवं प्रपद्ये शरणं विभुम्}
{दैवदण्डमिमं घोरं स हि मे नाशयिष्यति}


\twolineshloka
{कपर्दिनं प्रपद्येऽहं देवदेवमुमापतिम्}
{कपालमालिनं रुद्रं भगनेत्रहरं हरम्}


\twolineshloka
{स हि देवोऽत्यगाद्देवांस्तपसा विक्रमेण च}
{तस्माच्छरणमभ्येष्ये गिरिशं शूलपाणिनम्}


\chapter{अध्यायः ७}
\twolineshloka
{सञ्जय उवाच}
{}


\threelineshloka
{एवं सञ्चिन्तयित्वा तु द्रोणपुत्रो विशाम्पते}
{अवतीर्य रथोपस्थाद्दध्यौ स प्रयतः स्थितः ॥द्रौणिरुवाच}
{}


\twolineshloka
{उग्रं स्थाणुं शिवं रुद्रं शर्वमीशानमीश्वरम्}
{गिरिशं वरदं देवं भवभावनमव्ययम्}


\twolineshloka
{शितिकण्ठमजं रुद्रं दक्षक्रतुहरं हरम्}
{विश्वरूपं विरूपाक्षं बहुरूपमुमापतिम्}


\twolineshloka
{श्मशानवासिनं दृप्तं महागणपतिं विभुम्}
{खट्वाङ्गधारिणं मुण्डं जटिलं ब्रह्मचारिणम्}


\twolineshloka
{मनसा ह्यनुचिन्त्यैनं दुष्करेणाल्पचेतसा}
{अद्य भूतोपहारेण यक्ष्ये त्रिपुराघातिनम्}


\twolineshloka
{xxतुतं स्तुत्यं स्तूयमानममोघं कृत्तिवाससम्}
{विलोहितं नीलकण्ठमसह्यं दुर्निवारणम्}


\twolineshloka
{शुक्रं विश्वसृजं ब्रह्म ब्रह्मचारिणमेव च}
{व्रतवन्तं तपोनिष्ठमनन्तं तपतां गतिम्}


\twolineshloka
{बहुरूपं गणाध्यक्षं त्र्यक्षं पारिषदप्रियम्}
{धनाध्यक्षप्रियसखं गौरीहृदयवल्लभम्}


\twolineshloka
{कुमारपितरं पिङ्गं गोवृषोत्तमवाहनम्}
{कुत्तिवाससमत्युग्रमातोषणतत्परम्}


\twolineshloka
{परं परेभ्यः परमं परं यस्मान्न विद्यते}
{इष्वस्त्रोत्तमभर्तारं दिगन्तं देशरक्षिणम्}


\twolineshloka
{हिरण्यकवचं देवं चन्द्रमौलिविभूषणम्}
{प्रपद्ये शरणं देवं परमेण समाधिना}


\twolineshloka
{इमां चेदापदं घोरां तराम्यद्य सुदुस्तराम्}
{सर्वभूतोपहारेण यक्ष्येऽहं शुचिना शुचिम्}


\twolineshloka
{इति तस्य व्यवसितं ज्ञात्वा योगात्सुकर्मणः}
{पुरस्तात्काञ्चनी वेदी प्रादुरासीन्महात्मनः}


\twolineshloka
{तस्यां वेद्यां तदा राजंश्चित्रभानुरजायत}
{स दिशो विदिशः खं च ज्वालाभिरिव पूरयन्}


\twolineshloka
{दीप्तास्यनयनाश्चात्र नैकपादशिरोधराः}
{[रत्नचित्राङ्गदधराः समुद्यतकरास्तथाः ॥]}


\twolineshloka
{द्विपाः शैलप्रतीकाशाः प्रादुरासन्महागणाः}
{श्ववराहोष्ट्ररूपाश्च हयगोमायुगोमुखाः}


\twolineshloka
{ऋक्षमार्जारवदना व्याघ्रद्वीपिमुखास्तथा}
{काकवक्त्राः प्लुवमुखाः शुकवक्त्रास्तथैव च}


\twolineshloka
{महाजगरवक्त्राश्च हंसवक्त्राः शितप्रभाः}
{दार्वाघाटमुखाश्चापि चाषवक्त्राश्च भारत}


\twolineshloka
{कूर्मनक्रमुखाश्चैव शिंशुमारमुखास्तथा}
{महामकरवक्त्राश्च तिमिवक्त्रास्तथैव च}


\twolineshloka
{हरिवक्त्राः क्रौञ्चमुखाः कपोताभमुखास्तथा}
{पारावतमुखाश्चैव मद्गुवक्त्रास्तथैव च}


\twolineshloka
{पाणिकर्णाः सहस्राक्षास्तथैव च महोदराः}
{निर्मांसाः काकवक्त्राश्च श्येनवक्त्राश्च भारत}


\twolineshloka
{तथैवाशिरसो राजन्नृक्षवक्त्राश्च भीषणाः}
{प्रदीप्तनेत्रजिह्वाश्च ज्वालावर्णास्तथैव च}


\twolineshloka
{ज्वालाकेशाश्च राजेन्द्र ज्वलद्रोमचतुर्भुजाः}
{मेषवक्त्रास्तथैवान्ये तथा छागमुखा नृप}


\twolineshloka
{शङ्खाभाः शङ्खवक्त्राश्च शङ्खवर्णास्तथैव च}
{शङ्खमालापरिकराः शह्खध्वनिसमस्वनाः}


\twolineshloka
{जटाधराः प़ञ्चशिखास्तथा मुण्डाः कृशोदराः}
{चतुर्दंष्ट्राश्चतुर्जिह्वाः शङ्कुकर्णाः किरीटिनः}


\twolineshloka
{मौञ्जीधराश्च राजेन्द्र तथा कुञ्चितमूर्धजाः}
{उष्णीषिणः कुण्डलिनश्चारुवक्त्राः स्वलङ्कृताः}


\twolineshloka
{पद्मोत्पलापीडधरास्तथा मुकुटधारिणः}
{महात्म्येन च संयुक्ताः शतशोऽथ सहस्रशः}


\twolineshloka
{शतघ्नीचक्रहस्ताश्च तथा मुसलपाणयः}
{भुशुण्डीपाशहस्ताश्च दण़्डहस्ताश्च भारत}


\twolineshloka
{पृष्ठेषु बद्धेषुधयश्चित्रबाणोत्कटास्तथा}
{सध्वजाः सपताकाश्च सघण्टाः सपरश्वथाः}


\twolineshloka
{महापाशोद्यतकरास्तथा लगुडपाणयः}
{स्थूणाहस्ताः खङ्गहस्ताः सर्पोच्छ्रितकिरीटिनः}


\threelineshloka
{महासर्पाङ्गदधराश्चित्राभरणधारिणः}
{रजोध्वजाः पङ्कदिग्धाः सर्वे चित्राम्बरस्रजः}
{नीलाङ्गाः पिङ्गलाङ्गाश्च मुण्डवक्तास्तथैव च}


\twolineshloka
{भेरीशङ्खमृदङ्गांश्च झर्झरानकगोमुखान्}
{अवादयन्पारिषदाः प्रहृष्टाः कनकप्रभाः}


\twolineshloka
{गायमानास्तथैवान्ये संहृष्टाः पुरुषर्षभाः}
{लङ्घयन्तः प्लुवन्तश्च वल्गन्तश्च महारथाः}


\twolineshloka
{धावन्तो जवनाश्चण्डाः पावकोद्वूतमूर्धजाः}
{मत्ता इव महानागा विनदन्तो मुहुर्मुहुः}


\twolineshloka
{सुभीमा घोररूपाश्च शूलपट्टसपाणयः}
{नानाविरागवसनाश्चित्रमाल्यानुलेपनाः}


\twolineshloka
{रत्नचित्राङ्गदधराः समुद्यतकरास्तथा}
{हन्तारो द्विषतां शूराः प्रसह्यासह्यविक्रमाः}


\twolineshloka
{पातारोऽसृग्वसाज्यानां मांसान्त्रकृतभोजनाः}
{चूडालाः कर्णिकाराश्च प्रहृष्टाः पिठरोदराः}


\twolineshloka
{अतिहस्वातिदीर्घाश्च प्रलम्बाश्चातिभैरवाः}
{विकटाः काललंबोष्ठा बृहच्छेफाण्डपिण्डिकाः}


\twolineshloka
{महार्हनानाविकटा मुण्डाश्च जटिलाः परे}
{सार्केन्दुग्रहनक्षत्रां द्यां कुर्युस्ते महीतले}


\twolineshloka
{उत्सहेरंश्च ये हन्तुं भूतग्रामं चतुर्विधम्}
{ये च वीतभया नित्यं हरस्य भ्रुकुटीसहाः}


\twolineshloka
{कामकारकरा नित्यं त्रैलोक्यस्येश्वरेश्वराः}
{नित्यानन्दप्रमुदिता वागीशा वीतमत्सराः}


\twolineshloka
{प्राप्याष्टगुणमैश्वर्यं ये न यास्यन्ति वै स्मयम्}
{येषां विस्मयते नित्यं भगवान्कर्मभिर्हरः}


\twolineshloka
{मनोवाक्कर्मभिर्युक्तैर्नित्यमाराधितश्च यैः}
{मनोवाक्कर्मभिर्भक्तान्पाति पुत्रानिवौरसान्}


\twolineshloka
{पिबन्तोऽसृग्वसाश्चान्ये क्रुद्धा ब्रह्मद्विषां सदा}
{चतुर्विधात्मकं सोमं ये पिबन्ति च सर्वदा}


\twolineshloka
{श्रुतेन ब्रह्मचर्येण तपसा च दमेन च}
{ये समाराध्य शूलाङ्कं भवसायुज्यमागताः}


\twolineshloka
{यैरात्मभूतैर्भगवान्पार्वत्या च महेश्वरः}
{महाभूतगणैर्भुङ्क्ते भूतभव्यभवत्प्रभुः}


\twolineshloka
{नानावादित्रहसितक्ष्वेडितोत्कृष्टगर्जितैः}
{सन्त्रासयन्तस्ते विश्वमश्वत्थामानमभ्ययुः}


\twolineshloka
{संस्तुवन्तो महादेवं भाः कुर्वाणाः सुवर्चसः}
{विवर्धयिषवो द्रौणेर्महिमानं महात्मनः}


\threelineshloka
{जिज्ञासमानास्तत्तेजः सौप्तिकं च दिदृक्षवः}
{भीमोग्रपरिघालातशूलपट्टसपाणयः}
{घोररूपाः समाजग्मुर्भूतसङ्घाः सहस्रशः}


\twolineshloka
{जनयेयुर्भयं ये स्म त्रैलोक्यस्यापि दर्शनात्}
{न च तान्प्रेक्षमाणोऽपि व्यथामुपजगाम ह}


\twolineshloka
{अथ द्रौणिर्धनुष्पाणिर्बद्धगोधाङ्गुलित्रवान्}
{स्वयमेवात्मनात्मानमुपहारमुपाहरत्}


\twolineshloka
{धनूंषि समिधस्तत्र पवित्राणि सिताः शराः}
{हविरात्मवतश्चात्मा तस्मिन्भारत कर्मणि}


\twolineshloka
{ततः सौम्येन मन्त्रेण द्रोणपुत्रः प्रतापवान्}
{उपहारं महामन्युरथात्मानमुपाहरत्}


\threelineshloka
{तं रुद्रं रौद्रकर्माणं रौद्रैः कर्मभिरच्युतम्}
{अभिष्टुत्य महात्मानमित्युवाच कृताञ्जलिः ॥द्रौणिरुवाच}
{}


\twolineshloka
{इममात्मानमद्याहं जातमाङ्गिरसे कुले}
{स्वग्नौ जुहोमि भगवन्प्रतिगृह्णीष्व मां बलिम्}


\twolineshloka
{भवद्भक्त्या महादेव पस्मेण समाधिना}
{अस्यामापदि विश्वात्मन्नुपाकुर्मी तवाग्रतः}


\twolineshloka
{त्वयि सर्वाणि भूतानि सर्वभूतेषु चासि वै}
{गुणानां हि प्रधानानां कैवल्यं त्वयि तिष्ठति}


\twolineshloka
{सर्वभूताश्रय विभो हविर्भूतमवस्थितम्}
{प्रतिगृहाण मां देव यद्यशक्याः परे मया}


\twolineshloka
{इत्युक्त्वा द्रौणिरास्थाय तां देवीं दीप्तपावकाम्}
{सन्त्यज्यात्मानमारुह्य कृष्णवर्त्मन्युपाविशत्}


\twolineshloka
{तमूर्ध्वबाहुं निश्चेष्टं दृष्ट्वा हविरुपस्थितम्}
{अब्रवीद्भगवान्साक्षान्महादेवो हसन्निव}


\twolineshloka
{सत्यशौचार्जवत्यागैस्तपसा नियमेन च}
{क्षान्त्या भक्त्या च धृत्या च कर्मणा मनसा गिरा}


\twolineshloka
{यथावदहमाराद्वः कृष्णेनाक्लिष्टकर्मणा}
{तस्मादिष्टतमः कृष्णादन्यो मम न विद्यते}


\twolineshloka
{कुर्वता तात सम्मानं त्वां च जिज्ञासता मया}
{पाञ्चालाः सर्वदा गुप्ता मायाश्च बहुशः कृताः}


\twolineshloka
{कृतस्तस्यैव सम्मानं पाञ्चालान्रक्षता मया}
{अभिभूतास्तु कालेन नैषामद्यास्ति जीवितम्}


\twolineshloka
{एवमुक्त्वा महात्मानं भगवानात्मनस्तनुम्}
{आविवेश ददौ चास्मै विमलं खङ्गमुत्तमम्}


\twolineshloka
{अथाविष्टो भगवता भूयो जज्वाल तेजसा}
{वेगवांश्चाभवद्युद्धे देवसृष्टेन तेजसा}


\threelineshloka
{तं दृष्ट्वा तानि भूतानि रक्षांसि च समाद्रवन्}
{अभितः शिबिरं यान्तं द्रोणपुत्रं महारथम्}
{देवदेवं हरं स्थाउं यान्तं साक्षादिवेश्वरम्}


\chapter{अध्यायः ८}
\twolineshloka
{धृतराष्ट्र उवाच}
{}


\twolineshloka
{तथा प्रयाते शिबिरं द्रोणपुत्रे महारथे}
{कच्चित्कृपश्च भोजश्च भयार्तौ न व्यवर्तताम्}


\twolineshloka
{कच्चिन्न वारितौ क्षुद्रौ रक्षिभिर्नोपलक्षितौ}
{असह्यमिति मन्वानौ न निवृत्तौ महारथौ}


\twolineshloka
{कच्चिदुन्मथ्य शिबिरं हत्वा सोमकपाण्डवान्}
{`कृता प्रतिज्ञा सफला कच्चित्सञ्जय सा निशि'}


\twolineshloka
{दुर्योधनस्य पदवीं कच्चित्परमिकां रणे}
{`गत्वातिष्ठदसौ द्रौणिः कृत्वा कर्म सुदुष्करम्}


\twolineshloka
{धृष्टद्युम्नशिखण्डिभ्यां द्रौपद्याश्च सुतैः किल}
{सञ्छन्ना मेदिनी सुप्तैर्निहतैः पाण्डुसैनिकैः}


\twolineshloka
{पाञ्चालैर्वा विनिहतैः शयानै रुधिरोक्षितैः}
{कच्चिन्महीतलं छन्नं तन्ममाचक्ष्व सञ्जय'}


\threelineshloka
{[पाञ्चालैर्निहतौ वीरौ कच्चित्तु स्वपतां क्षितौ}
{कच्चित्ताभ्यां कृतं कर्म तन्ममाचक्ष्व सञ्जय ॥]सञ्जय उवाच}
{}


\twolineshloka
{तस्मिन्प्रयाते शिबिरं द्रोणपुत्रे महात्मनि}
{कृपश्च कृतवर्मा च द्रौणिमेवाभ्यवर्तताम्}


\twolineshloka
{अश्वत्थामा तु तौ दृष्ट्वा यत्नवन्तौ महारथौ}
{प्रहृष्टः शनकै राजन्निदं वचनमब्रवीत्}


\twolineshloka
{यत्तौ भवन्तौ पर्याप्तौ सर्वक्षत्रस्य नाशने}
{किम्पुनर्योधशेषस्य प्रसुप्तस्य विशेषतः}


\threelineshloka
{अहं प्रवेक्ष्ये शिबिरं चरिष्यामि च कालवत्}
{यथा न कश्चिदपि वां जीवन्मुच्येत मानवः}
{तथा भवद्ध्यां कार्यं स्यादिति मे निश्चिता मतिः}


\twolineshloka
{इत्युक्त्वा प्राविशद्द्रौणिः पार्थानां शिबिरं महत्}
{अद्वारेणाभ्यवस्कन्द्य विहाय भयमात्मनः}


\threelineshloka
{स प्रविश्य महाबाहुरुद्देशज्ञश्च तस्य ह}
{`द्रौणिः परमसङ्क्रुद्धस्तेजसा प्रज्वलन्निव}
{ततः पर्यचरत्सर्वं सम्प्रसुप्तं जनं निशि'}


% Check verse!
धृष्टद्युम्नस्य निलयं शनकैरभ्युपागमत्
\twolineshloka
{ते तु कृत्वा महत्कर्म श्रान्ताश्च बलवद्रणे}
{प्रसुप्ता वै सुविश्वस्ताः स्वसैन्यपरिवारिताः}


\twolineshloka
{अथ प्रविश्य तद्वेश्म धृष्टद्युम्नस्य भारत}
{पाञ्चाल्यं शयने द्रौणिरपश्यत्सुप्तमन्तिकात्}


\twolineshloka
{क्षौमावदाते महति स्पर्द्व्यास्तरणसंवृते}
{माल्यप्रवरसंयुक्ते धूपैश्चूर्णैश्च वासिते}


\twolineshloka
{तं शयानं महात्मानं विस्रब्धमकुतोभयम्}
{अपोथयत पादेन शयनस्थं महीपते}


\twolineshloka
{सम्बुध्य चरणस्पर्शादुत्थाय रणदुर्मदः}
{अभ्यजानादमेयात्मा द्रोणपुत्रं महारथम्}


\twolineshloka
{तमुत्पतन्तं शयनादश्वत्थामा महाबलः}
{केशेष्वालभ्य पाणिभ्यां निष्पिपेष महीतले}


\twolineshloka
{स बलात्तेन निष्पिष्टः साध्वसेन च भारत}
{अभ्याक्रान्तश्च निद्रान्धो नाशकच्चेष्टितुं तदा}


\twolineshloka
{`निष्पिष्य तु ततो भूमौ पाञ्चाल्यं द्रौणिरञ्जसा}
{धनुषो ज्यां विमुच्याशु क्रूरबुद्धिरमर्षणः}


\twolineshloka
{तस्य कण्ठेऽथ बद्ध्वा तां त्वरितः क्रोधमूर्च्छितः}
{'द्रौणिः क्रूरं मनः कृत्वा पाञ्चाल्यमवधीत्तदा}


\twolineshloka
{तमाक्रम्य पदा राजन्कण्ठे चोरसि पादयोः}
{तदन्तं विस्फुरन्तं च पशुमारममारयत्}


\twolineshloka
{`स वार्यमाणस्तरसा बलाद्बलवता बली'}
{तुदन्नखैस्तु स द्रौणिं नातिव्यक्तमुदाहरत्}


\twolineshloka
{आचार्यपुत्र शस्त्रेण जहि मां मा चिरं कृथाः}
{त्वत्कृते सुकृतां लोकान्गच्छेयं द्विपदांवर}


\twolineshloka
{एवमुक्त्वा तु वचनं विरराम परन्तपः}
{सुतः पाञ्चालराजस्य आक्रान्तो बलिना भृशम्}


\threelineshloka
{तस्याव्यक्तां तु तां वचं संश्रुत्य द्रौणिरब्रवीत्}
{आचार्यघातिनां लोका न सन्ति कुलपांसन}
{तस्माच्छस्त्रेण निधनं न त्वमर्हसि दुर्मते}


\twolineshloka
{`नृशंसेनातिवृत्तेन त्वया मे निहतः पिता}
{तस्मात्त्वमपि वध्यश्च नृशंसेन नृशंसकृत् ॥'}


\twolineshloka
{एवं ब्रुवाणस्तं वीरं सिंहो मत्तमिव द्विपम्}
{मर्मस्वभ्यवधीत्क्रुद्धः पादघातैः सुदारुणैः}


\twolineshloka
{तस्य वीरस्य शब्देन मार्यमाणस्य वेश्मनि}
{अबुध्यन्त महाराज स्त्रियो ये चास्य रक्षिणः}


\twolineshloka
{ते दृष्ट्वा धर्षयन्तं तमतिमानुषविक्रमम्}
{भूतमित्यध्यवस्यन्तो न स्म प्रव्याहरन्भयात्}


\twolineshloka
{तं तु तेनाभ्युपायेन गमयित्वा यमक्षयम्}
{अध्यतिष्ठत तेजस्वी रथं प्राप्य सुदर्शनम्}


\twolineshloka
{स तस्य भवनाद्राजन्निष्क्रम्यानादयन्दिशः}
{रथेन शिबिरं प्रायाज्जिघांसुर्द्विषतो बली}


\twolineshloka
{अपक्रान्ते ततस्तस्मिन्द्रोणपुत्रे महारथे}
{सहितैः रक्षिभिः सर्वैः प्राणेदुर्योषितस्तदा}


\twolineshloka
{राजानं निहतं दृष्ट्वा भृशं शोकपरायणाः}
{व्याक्रोशन्क्षत्रियाः सर्वे धृष्टद्युम्नस्य भारत}


\twolineshloka
{तासां तु तेन शब्देन समीपे क्षत्रियर्षभाः}
{सम्भ्रान्ताः समनह्यन्त किमेतदिति चाब्रुवन्}


\twolineshloka
{स्त्रियस्तु राजन्वित्रस्ता भारद्वाजं निरीक्ष्य ताः}
{अब्रुन्दीनकण्ठेन क्षिप्रमाद्रवतेति वै}


\twolineshloka
{राक्षसो वा मनुष्यो वा नैनं जानीम कोन्वयम्}
{हत्वा पाञ्चालराजानं रथमारुह्य तिष्ठति}


\twolineshloka
{ततस्ते योधमुख्यास्तं सहसा पर्यवारयन्}
{स तानापततः सर्वान्रुद्रास्त्रेण व्यपोथयत्}


\twolineshloka
{धृष्टद्युम्नं च हत्वा स तांश्चैवास्य पदानुगान्}
{अपश्यच्छयने सुप्तमुत्तमौजसमन्तिके}


\twolineshloka
{तमप्याक्रम्य पादेन कण्ठे चोरसि तेजसा}
{तथैव मारयामास विनर्दन्तमरिन्दमम्}


\twolineshloka
{युधामन्युश्च विक्रान्तो मत्वा तं राक्षसं स्म सः}
{गदामुद्यम्य वेगेन हृदि द्रौणिमताडयत्}


\twolineshloka
{`गदाप्रहाराभिहतो नाचलद्द्रौणिराहवे'}
{तमभिद्रुत्य वेगेन क्षितौ चैनमपातयत्}


\twolineshloka
{विस्फुरन्तं च पशुवत्तथैवैनममारयत्}
{तथा स वीरो हत्वा तं ततोऽन्यान्समुपाद्रवत्}


\threelineshloka
{संसुप्तानेव राजेन्द्र तत्र तत्र महारथान्}
{पाञ्चालवीरानाक्रम्य क्रुद्धो न्यहनदन्तिके}
{स्फुरतो वेपमानांश्च शमितेव पशून्मखे}


\twolineshloka
{ततो निस्त्रिंशमादाय जघानान्यान्पृथक्पृथक्}
{भागशो विचरन्मार्गानसियुद्वविशारदः}


\twolineshloka
{तथैव गुल्मे सम्प्रेक्ष्य शयानान्मध्यगौल्मिकान्}
{श्रान्तान्व्यस्तायुधान्सर्वानसिनैव व्यपोथयत्}


\twolineshloka
{योधानश्वान्द्विपांश्चैव प्राच्छिनत्स वरासिना}
{रुधिरोक्षितसर्वाङ्गः कालसृष्ट इवान्तकः}


\twolineshloka
{विस्फुरद्भिश्च तैर्द्रौणिर्निस्त्रिंशस्योद्यमेन च}
{अवक्षेपेण चैवासेस्त्रिधा रक्तोक्षितोऽभवत्}


\twolineshloka
{तस्य लोहितरक्तस्य दीप्तखङ्गस्य युध्यतः}
{अमानुष इवाकारो बभौ परमभीषणः}


\twolineshloka
{ये त्वजाग्रन्त कौरव्य तेऽपि शब्देन मोहिताः}
{वीक्षमाणास्तु ते तत्र द्रौणिं दृष्ट्वा प्रविव्यथुः}


\twolineshloka
{तद्रूपं तस्य ते दृष्ट्वा क्षत्रियाः शत्रुकर्शनम्}
{राक्षसं मन्यमानास्तं नयनानि न्यमीलयन्}


\twolineshloka
{स घोररूपो व्यचरत्कालवच्छिबिरे तदा}
{अपश्यद्द्रौपदीपुत्रानवशिष्टांश्च सोमकान्}


\threelineshloka
{तेन शब्देन वित्रस्ता धनुर्हस्ता महारथाः}
{धृष्टद्युम्नं हतं श्रुत्वा द्रौपदेया विशाम्पते}
{अवाकिरञ्शरव्रातैर्भारद्वाजमभीतवत्}


\twolineshloka
{ततस्तेन निनादेन सम्प्रबुद्धाः प्रभद्रकाः}
{शिलीमुखैः शिखण्डी च द्रोणपुत्रं समार्दयन्}


\twolineshloka
{भारद्वाजः स तान्दृष्ट्वा शरवर्षाणि वर्षतः}
{ननाद बलवन्नादं जिघांसुस्तान्महारथान्}


\twolineshloka
{ततः परमसङ्क्रुद्धः पितुर्वधमनुस्मरन्}
{अवरुह्य रथोपस्थात्त्वरमाणोऽभिदुद्रुवे}


\threelineshloka
{सहस्रचन्द्रविमलं गृहीत्वा चर्म संयुगे}
{खङ्गं च विमलं दिव्यं जातरूपपरिष्कृतम्}
{द्रौपदेयानभिद्रुत्य खङ्गेन व्यधमद्बली}


\twolineshloka
{ततः स नरशार्दूलः प्रतिविन्ध्यं महाहवे}
{कुक्षिदेशेऽवधीद्राजन्स हतो न्यपतद्भुवि}


\twolineshloka
{प्रासेन विद्ध्वा द्रौणिं तु सुतसोमः प्रतापवान्}
{पुनश्चासिं समुद्यम्य द्रोणपुत्रमुपाद्रवत्}


\twolineshloka
{सुतसोमस्य सासिं तं बाहुं छित्त्वा नरर्षभ}
{पुनरप्याहनत्पार्श्वे स भिन्नहृदयोऽपतत्}


\twolineshloka
{नाकुलिस्तु शतानीको रथचक्रेण वीर्यवान्}
{दोर्भ्यामुत्क्षिप्य वेगेन वक्षस्येनमताडयत्}


\twolineshloka
{अताडयच्छतानीकं मुक्तचक्रं द्विजस्तु सः}
{स विह्वलो ययौ भूमिं ततोऽस्यापाहरच्छिरः}


\twolineshloka
{श्रुतकर्मा तु परिघं घोरं गृह्य दुरासदम्}
{अताडयत्समुद्यम्य वेगेन द्रौणिमुत्स्मयन्}


\twolineshloka
{स तु तं श्रुतकर्माणमास्ये जघ्ने वरासिना}
{स हतो न्यपतद्भूमौ विमूर्धा विकृताननः}


\twolineshloka
{तेन शब्देन वीरस्तु श्रुतकीर्तिरबुध्यत}
{अश्वत्थामानमासाद्य शरवर्षैरवाकिरत्}


\twolineshloka
{`शरैराच्छादितस्तेन द्रोणपुत्रो महारथः}
{अदृश्यत महाराज श्वाविच्छललतो यथा ॥'}


\twolineshloka
{तस्यापि शरवर्षाणि चर्मणा प्रतिवार्य सः}
{सकुण्डलं शिरः कायाद्वाजमानमपाहरत्}


\twolineshloka
{ततो भीष्मनिहन्तारं सह सर्वैः प्रभद्रकैः}
{आहनत्सर्वतो वीरं नानाप्रहरणैर्बलात्}


% Check verse!
शिलीमुखेन चाप्येनं भ्रुवोर्मध्ये समार्पयत्
\twolineshloka
{स तु क्रोधसमाविष्टो द्रोणपुत्रो महाबलः}
{शिखण्डिनं समासाद्य द्विधा चिच्छेद सोसिना}


\twolineshloka
{शिखण़्डिनं ततो हत्वा क्रोधाविष्टः परन्तपः}
{प्रभद्रकगणान्सर्वानभिदुद्राव वेगवान्}


\threelineshloka
{यच्च शिष्टं विराटस्य बलं तु भृशमाद्रवत्}
{द्रुपदस्य च पुत्राणां पौत्राणां सुहृदामपि}
{चकार कदनं घोरं दृष्ट्वा तत्र महाबलः}


\twolineshloka
{अन्यानन्यांश्च पुरुषानभिसृत्याभिसृत्य च}
{न्यकृन्तदसिना द्रौणिरसिमार्गविशारदः}


\twolineshloka
{कालीं रक्तास्यनयनां रक्तमाल्यानुलेपनाम्}
{रक्ताम्बरधरां घोरां पाशहस्तां कुटुम्बिनीम्}


\twolineshloka
{ददृशुः कालरात्रिं ते स्मयमानामिव स्थिताम्}
{नराश्वकुञ्जरान्पाशैर्बद्धा घोरैः प्रतस्थुषीम्}


\twolineshloka
{वहन्तीं विविधान्प्रेतान्पाशबद्वान्विमूर्धजान्}
{तथैव च सदा राजन्न्यस्तशस्त्रान्महारथान्}


\twolineshloka
{स्वप्ने सुप्तान्नयन्तीं तां रात्रिष्वन्यासु मारिष}
{ददृशुर्योधमुख्यास्ते घ्नन्तं द्रौणिं च नित्यदा}


\twolineshloka
{यतः प्रभृति सङ्ग्रामः कुरुपाण्डवसेनयोः}
{ततः प्रभृति तां कन्यामपश्यन्द्रौणिमेव च}


\twolineshloka
{तांस्तु दैवहतान्पूर्वं पश्चाद्द्रौणिर्व्यपातयत्}
{त्रासयन्सर्वभूतानि विनदन्भैरवान्रवान्}


\twolineshloka
{तदनुस्मृत्य ते वीरा दर्शनं पूर्वकालिकम्}
{इदं तदित्यमन्यन्त दैवेनोपनिपीडिताः}


\twolineshloka
{ततस्तेन निनादेन प्रत्यबुध्यन्त धन्विनः}
{शिबिरे पाण्डवेयानां शतशोऽथ सहस्रशः}


\twolineshloka
{सोऽच्छिनत्कस्यचित्पदौ जघनं चैव कस्यचित्}
{कांश्चिद्बिभेदपार्श्वेषु कालसृष्ट इवान्तकः}


\twolineshloka
{अत्युग्रप्रतिपिष्टैश्च नदद्भिश्च भृशोत्कटैः}
{गजाश्वमथितैश्चान्यैर्मही कीर्णाऽभवत्प्रभो}


\twolineshloka
{क्रोशतां किमिदं कोऽयं कः शब्दः किन्नु किं कृतम्}
{पाञ्चालानां तथा द्रौणिरन्तकः समपद्यत}


\twolineshloka
{अपेतशस्त्रसन्नाहान्सन्नद्वान्पाण्डुसृञ्जयान्}
{प्राहिणोन्मृत्युलोकाय द्रौणिः प्रहरतां वरः}


\twolineshloka
{ततस्तच्छस्त्रवित्रस्ता भयादभ्यपतन्नराः}
{निद्रान्धा नष्टसंज्ञाश्च तत्रतत्र निपेतिरे}


\twolineshloka
{ऊरुस्तम्भगृहीताश्च कश्मलाभिहतौजसः}
{विनदन्तो भृशं त्रस्ता निरैक्षन्त परस्परम्}


\twolineshloka
{ततो रथं पुनर्द्रौणिरास्थितो भीमदर्शनः}
{धनुष्पाणिः शरैरन्यान्प्रैषयद्वै यमक्षयम्}


\twolineshloka
{पुनरुत्पततश्चापि दूरादपि नरोत्तमान्}
{शूरान्सम्पततश्चान्यान्कालरात्र्यै न्यवेदयत्}


\twolineshloka
{तथैव स्यन्दनाग्रेण प्रमथन्स व्यरोचत}
{शरवर्षैश्च विविधैरवर्षच्छात्रवांस्ततः}


\twolineshloka
{पुनश्च सुविचित्रेण शतचन्द्रेण चर्मणा}
{तेन चाकाशवर्णेन तथाचरत सोऽसिना}


\twolineshloka
{तथा स शिबिरं तेषां द्रौणिराहवदुर्मदः}
{व्यक्षोभयत राजेन्द्र महाह््दमिव द्विपः}


\twolineshloka
{उत्पेतुस्तेन शब्देन योधा राजन्विचेतसः}
{निद्रार्ताश्च भयार्ताश्च व्यधावन्त ततस्ततः}


\twolineshloka
{विस्वरं चुक्रुशुश्चान्ये बह्वबद्वं तथाऽवदन्}
{न च स्म प्रत्यपद्यन्त शस्त्राणि वसनानि च}


\threelineshloka
{विमुक्तकेशाश्चाप्यन्ये नाभ्यजानन्परस्परम्}
{उत्पतन्तोऽपतञ्श्रान्ताः केचित्तत्राभ्रमंस्तदा}
{पुरीषमसृजन्केचित्केचिन्मूत्रं प्रसुस्रुवुः}


\twolineshloka
{बन्धनानि च राजेन्द्र सञ्छिद्य तुरगा द्विपाः}
{समं पर्यपतंश्चान्ये कुर्वन्तो महदाकुलम्}


\twolineshloka
{तत्र केचिन्नरा भीता व्यलीयन्त महीतले}
{तथैव तान्निपतितानपिंषन्गजवाजिनः}


\twolineshloka
{तस्मिंस्तथा वर्तमाने रक्षांसि पुरुषर्षभ}
{हृष्टानि व्यनदन्नुच्चैर्मुदा युक्तानि सत्तम}


\twolineshloka
{स शब्दः प्रेरितो राजन्भूतसङ्घैर्मुदा युतैः}
{अपूरयद्दिशः सर्वा दिवं चातिमहान्स्वनः}


\twolineshloka
{तेषामार्तरवं श्रुत्वा वित्रस्ता गजवाजिनः}
{मुक्ताः पर्यपतन्राजन्मृद्गन्तः शिबिरे जनम्}


\twolineshloka
{तैस्तत्र परिधावद्भिश्चरणोदीरितं रजः}
{अकरोच्छिबिरे तेषां रजन्यां द्विगुणं तमः}


\twolineshloka
{तस्मिंस्तमसि सञ्जाते प्रमूढाः सर्वतो जनाः}
{नाजानन्पितरः पुत्रान्भ्रातॄन्भ्रातर एव च}


\twolineshloka
{गजो राजानतिक्रम्य निर्मनुष्या हया हयान्}
{अताडयंस्तथाऽभञ्जंस्तथाऽमृद्गंश्च भारत}


\twolineshloka
{ते भग्नाः प्रपतन्ति स्म मृद्रन्तश्च परस्परम्}
{न्यपातयंस्तथा चान्यान्पातयित्वा तदाऽपिषन्}


\twolineshloka
{विचेतसः सनिद्राश्च तमसा चावृता नराः}
{जघ्रुः स्वानेव तत्राथ कालेनैव प्रचोदिताः}


\twolineshloka
{त्यक्त्वाद्वाराणि च द्वास्थास्तथागुल्मानिगौल्मिकाः}
{प्राद्रवन्त यथाशक्ति कांदिशीका विचेतसः}


\twolineshloka
{विप्रनष्टाश्च तेऽन्योन्यं नाजानन्तस्तथा विभो}
{क्रोशन्तस्तात पुत्रेति दैवोपहतचेतसः}


\twolineshloka
{पलायतां दिशस्तेषां स्वानप्युत्सृज्य बान्धवान्}
{गोत्रनामभिरन्योन्यमाक्रन्दन्त ततो जनाः}


\twolineshloka
{हाहाकारं च कुर्वाणाः पृथिव्यां शेरते परे}
{तान्बुद्धा रणमध्येऽसौ द्रोणपुत्रो व्यपोथयत्}


\twolineshloka
{तत्रापरे वध्यमाना मुहुर्महुरचेतसः}
{शिबिरान्निष्पतन्ति स्म क्षत्रिया भयपीडिताः}


\twolineshloka
{तांस्तु निष्पतितांस्त्रस्ताञ्शिबिराज्जीवितैषिणः}
{कृतवर्मा कृपश्चैव द्वारदेशे निजघ्नतुः}


\twolineshloka
{विसस्तयन्त्रकवचान्मुक्तकेशान्कृताञ्जलीन्}
{वेपमानान्क्षितौ भीतान्त्रैव कांश्चिदमुच्यताम्}


\twolineshloka
{नामुच्यत तयोः कश्चिन्निष्क्रान्तः शिबिराद्बहिः ॥कृपश्चैव महाराज हार्दिक्यश्चैव दुर्मतिः}
{}


\twolineshloka
{भूयश्चैव चिकीर्षन्तौ द्रौणपुत्रस्य तौ प्रियम्}
{त्रिषु देशेषु ददतुः शिबिरस्य हुताशनम्}


\twolineshloka
{ततः प्रकाशे शिबिरे खङ्गेन पितृनन्दनः}
{अश्वत्थामा महाराज व्यचरत्कृतहस्तवत्}


\twolineshloka
{कांश्चिदापततो वीरानपरांश्चैव धावतः}
{व्ययोजयत खङ्गेन प्राणैर्द्विजवरोत्तमः}


\twolineshloka
{कांश्चिद्योधान्स खङ्गेन मध्ये सञ्छिद्य वीर्यवान्}
{अशातयद्द्रोणपुत्रः संरब्धस्तिलकाण्डवत्}


\twolineshloka
{निनदद्भिर्भृशायस्तैर्नराश्वद्विरदोत्तमैः}
{पतितैरभवत्कीर्णा मेदिनी भरतर्षभ}


\twolineshloka
{मानुषाणां सहस्रेषु हतेषु पतितेषु च}
{उदतिष्ठन्कबन्धानि बहून्युत्थाय चापतन्}


\twolineshloka
{सायुधान्साङ्गदान्बाहून्विचकर्त शिरांसि च}
{हस्तिहस्तोपमानूरून्हस्तान्पादांश्च भारत}


\twolineshloka
{पृष्ठच्छिन्नान्पार्श्वच्छिन्नाञ्शिरश्छिन्नांस्तथापरान्}
{स महात्माकरोद्द्रौणिः कांश्चिच्चापि पराङ्मुखान्}


\twolineshloka
{मध्यदेशे नरानन्यांश्चिच्छेदान्यांश्च कर्णतः}
{अंसदेशे निहत्यान्यान्काये प्रावेशयच्छिरः}


\twolineshloka
{एवं हि बहुभिः शस्त्रैर्घ्नतोऽपि बलवत्तरान्}
{तमसा रजनी घोरा बभौ दारुणदर्शना}


\twolineshloka
{किञ्चित्प्राणैश्च पुरुषैर्हतैश्चान्यैः सहस्रशः}
{बहुना च गजाश्वेन भूतभूद्भीमदर्शना}


\threelineshloka
{यक्षरक्षःसमाकीर्णे रथाश्वद्विपदारुणे}
{क्रुद्धेन द्रोणपुत्रेण सञ्छिन्नाः प्रापतन्भुवि}
{भ्रातॄनन्ये पितॄनन्ये पुत्रानन्ये विचुक्रुशुः}


\threelineshloka
{केचिदूचुर्न तत्क्रुद्धैर्धार्तराष्ट्रैः कृतं रणे}
{यत्कृतं नः प्रसुप्तानां रक्षोभिः क्रूग्कर्मभिः}
{असान्निध्याद्वि पार्थानामिदं वः कदनं कृतम्}


\threelineshloka
{न चासुरैर्न गन्धर्वैर्न यक्षैर्न च राक्षसैः}
{शक्यो विजेतुं कौन्तेयो नेता यस्य जनार्दनः}
{ब्रह्मण्यः मत्यवाग्दान्तः सर्वभूतानुकम्पकः}


\twolineshloka
{न च सुप्तं प्रमत्तं वा न्यस्तशस्त्रं कृताञ्जलिम्}
{धावन्तं मुक्तकेशं वा हन्ति पार्थो धनञ्जयः}


\twolineshloka
{तदिदं नः कृतं घोरं रक्षोभिः क्रूरकर्मभिः}
{इति लालप्यमानाः स्म शेरते बहवो जनाः}


\twolineshloka
{स्तनतां च मनुष्याणामपरेषां च कूजताम्}
{ततो मुहूर्तात्प्राशाम्यत्स शब्दस्तुमुलो महान्}


\twolineshloka
{शोणितव्यतिषिक्तायां वसुधायां च भूमिप}
{तद्रजस्तुमुलं घोरं क्षणेनान्तरधीयत}


\twolineshloka
{स चेष्टमानानुद्विग्नान्निरुत्साहान्सहस्रशः}
{न्यपातयन्नरान्क्रुद्वः पशून्पशुपतिर्यथा}


\twolineshloka
{अन्योन्यं सम्परिष्वज्य शयानाञ्जीवतोऽपरान्}
{संलीनान्युध्यमानांश्च सर्वान्द्रौणिरपोथयत्}


\twolineshloka
{ब्रह्ममाना हुताशेन वध्यमानाश्च तेन ते}
{परस्परं तदा योधाननयद्यमसादनम्}


\twolineshloka
{तस्या रजन्यास्त्वर्धेन पाण्डवानां महद्बलम्}
{गमयामास राजेन्द्र द्रौणिर्यमनिवेशनम्}


\twolineshloka
{निशाचराणां सत्वानां रात्रिः सा हर्षवर्धिनी}
{आसीन्नरगजाश्वानां रौद्री क्षयकरी भृशम्}


\twolineshloka
{तत्रादृश्यन्त रक्षांसि पिशाचाश्च पृथग्विधाः}
{खादन्तो नरमांसानि पिबन्तः शोणितानि च}


\twolineshloka
{करालाः पिङ्गला रौद्राः शैलदन्ता रजस्वलाः}
{जटिला भीमवक्त्राश्च पञ्चपादा महोदराः}


\twolineshloka
{पञ्चादङ्गुलयो रूक्षा विरूपा भैरवस्वनाः}
{गजाननाश्च हस्वाश्च नीलवर्णा बिभीषणाः}


\twolineshloka
{सपुत्रदाराः सुक्रूराः सुदुर्दर्शाः सुनिर्घृणाः}
{विविधानि च रूपाणि तत्रादृश्यन्त रक्षसाम्}


\twolineshloka
{पीत्वा च शोणितं हृष्टाः प्रानृत्यन्गणशोऽपरे}
{इदं परमिदं मेध्यमिदं स्वाद्विति चाब्रुवन्}


\twolineshloka
{मेदोमज्जास्थिरक्तानां मांसानां च भृशाशिताः}
{परे मांसानि खादन्तः क्रव्यादा मांसजीविनः}


\twolineshloka
{वसाश्चैवापरे पीत्वा पर्यधावन्विकुक्षिकाः}
{नानावक्त्रास्तथा रौद्राः क्रव्यादाः पिशिताशनाः}


\twolineshloka
{अयुतानि च तत्रासन्प्रयुतान्यर्बुदानि च}
{रक्षसां घोररूपाणां महतां क्रुरकर्मणाम्}


\threelineshloka
{मुदितानां वितृप्तानां तस्मिन्महति वैशसे}
{समेतानि बहून्यासन्भूतानि च जनाधिप}
{}


\twolineshloka
{एवंविधा हि सा रात्रिः सोमकानां जनक्षये}
{प्रसुप्तानां प्रमत्तानामासीत्सुभृशदारुणा}


\threelineshloka
{असंशयं हि कालस्य पर्यायो दुरतिक्रमः}
{तादृशा निहता यत्र कृत्वाऽस्माकं जनक्षयम् ॥धृतराष्ट्र उवाच}
{}


\twolineshloka
{प्रागेव सुमहत्कर्म द्रौणिरेतन्महारथः}
{नाकरोदीदृशं कस्मान्मत्पुत्रविजये धृतः}


\threelineshloka
{अथ कस््माद्वते क्षत्रे कर्मेदं कृतवानसौ}
{द्रोणपुत्रो महात्मा स तन्मे शंसितुमर्हसि ॥सञ्जय उवाच}
{}


\threelineshloka
{तेषां नूनं भयान्नासौ कृतवान्कुरुनन्दन}
{असान्निध्याद्धि पार्थानां केशवस्य च धीमतः}
{सात्यकेश्चापि कर्मेदं द्रोमपुत्रेण साधितम्}


\twolineshloka
{को हि तेषां समक्षं तान्हन््यादपि मरुत्पतिः}
{एतदीदृशकं वृत्तं राजन्सुप्तजने विभो}


\twolineshloka
{ततो जनक्षयं कृत्वा पाण्डवानां महात्ययम्}
{प्रत्यूषकाले शिबिरात्प्रतिगन्तुमियेष सः}


\twolineshloka
{नृशोणितावसिक्तस्य द्रौणेरासीदसित्सरुः}
{पाणिना सह संश्लिष्ट एकीभूत इव प्रभो}


\twolineshloka
{स निःशेषानरीन्कृत्वा विरराम निशाक्षये}
{युगान्ते सर्वभूतानि भस्म कृत्वेव पावकः}


\twolineshloka
{यथाप्रतिज्ञं तत्कर्म कृत्वा द्रौणायनिः प्रभो}
{दुर्गमां पदवीं गच्छन्पितुरासीद्गतज्वरः}


\twolineshloka
{यथैव संसुप्तजने शिबिरे प्राविशन्निशि}
{तथैव हत्वा निःशब्दो निश्चक्राम नरर्षभः}


\twolineshloka
{निष्क्रम्य शिबिरात्तस्मात्ताभ्यां सङ्गम्य वीर्यवान्}
{आचख्यौ कर्म तत्सर्वं हृष्टः संहर्षयन्विभो}


\twolineshloka
{तावथाचख्यतुस्तस्मै प्रियं प्रियकरौ तदा}
{पाञ्चालान्सृञ्जयांश्चैव विनिकृत्तान्सहस्रशः}


\twolineshloka
{प्रीत्या चोच्चैरुद्रक्रोशंस्तथैवास्फोटयंस्तलान्}
{दिष्ट्यादिष्ट्येति चान्योन्यं समेत्योचुर्महारथाः}


\twolineshloka
{पर्यष्वजत्ततो द्रौणिस्ताभ्यां सम्प्रतिनन्दितः}
{इदं हर्षात्तु सुमहदाददे वाक्यमुत्तमम्}


\twolineshloka
{पाञ्चाला निहताः सर्वे द्रौपदेयाश्च सर्वशः}
{सोमका मत्स्यशेषाश्च सर्वे विनिहता मया}


\twolineshloka
{इदानीं कृतकृत्याः स्म याम तत्रैव मा चिरम्}
{यदि जीवति नो राजा तस्मै शंसामहे प्रियम्}


\chapter{अध्यायः ९}
\twolineshloka
{सञ्जय उवाच}
{}


\twolineshloka
{ते हत्वा सर्वपाञ्चालान्द्रौपदेयांश्च सर्वशः}
{आगच्छन्सहितास्तत्र यत्र दुर्योधनो हतः}


\twolineshloka
{गत्वा चैनमपश्यन्त किञ्चित्प्राणं जनाधिपम्}
{ततो रथेभ्यः प्रस्कन्द्य परिवव्रुस्तवात्मजम्}


\twolineshloka
{तं भग्रसक्थं राजेन्द्र कृच्छ्रप्राणमचेतसम्}
{वमन्तं रुधिरं वक्त्रादपश्यन्वसुधातले}


\twolineshloka
{वृतं समन्ताद्बहुभिः श्वापदैर्घोरदर्शनैः}
{सालावृकगणैश्चैव भक्षयिष्यद्भिरन्तिकात्}


\twolineshloka
{निरायन्तं कृच्छ्रात्ताञ्श्वापदांश्च चिखादिषून्}
{विवेष्टमानमुरुभ्यां सुभृशं गाढवेदनम्}


\threelineshloka
{तं शयानं तथा दृष्ट्वा भूमौ सुरुधिरोक्षितम्}
{हतशिष्टास्त्रयो वीराः शोकार्ताः पर्यदेवयन्}
{अश्वत्थामा कृपश्चैव कृतवर्मा च सात्वतः}


\twolineshloka
{तैस्त्रिभिः शोणितादिग्धौर्निः श्वसद्भिर्महारथैः}
{शुशुभे स वृतो राजा वेदी त्रिभिरिवाग्निभिः}


\twolineshloka
{ते तं शयानं सम्प्रेक्ष्य राजानमतथोचितम्}
{अविषह्येन दुःखेन ततस्ते रुरुदुस्त्रयः}


\threelineshloka
{ततस्तु रुधिरं हस्तैर्मुखान्निर्मृज्य तस्य हि}
{रणे राज्ञः शयानस्य कृपः सम्पर्यदेवयत् ॥कृप उवाच}
{}


\twolineshloka
{न दैवस्यातिभारोऽस्ति यदयं रुधिरोक्षितः}
{एकादशचमूभर्ता शेते दुर्योधनो हतः}


\twolineshloka
{पश्य चामीकराभस्य चामीकरविभूषिताम्}
{गदां गदाप्रियस्येमां समीपे पतितां भुवि}


\twolineshloka
{इयमेनं गदा शूरं न जहाति रणेरणे}
{स्वर्गायापि व्रजन्तं हि न जहाति यशस्विनम्}


\twolineshloka
{पश्येमां सह वीरेण जाम्बूनदविभूषिताम्}
{शयानां शयने हर्म्ये भार्यां प्रीतिमतीमिव}


\twolineshloka
{योऽयं मूर्धाभिषिक्तानामग्रे याति परन्तपः}
{स हतो ग्रसते पांसून्पश्य कालसय पर्ययम्}


\twolineshloka
{येनाजौ निहता भूमौ शेरते क्षत्रियर्षभाः}
{स भूमौ निहतः शेते कुरुराजः परैरयम्}


\twolineshloka
{भयान्नमन्ति राजानो यस्य स्म शतसङ्घशः}
{स वीरशयने शेते क्रव्याद्भिः परिवारितः}


\threelineshloka
{यमुपासन्नृपाः पूर्वमर्थहेतोर्महीपतिम्}
{उपात्पते च तं ह्यद्य क्रव्यादा मांसगृद्धिनः ॥सञ्जय उवाच}
{}


\twolineshloka
{तं शयानं कुरुश्रेष्ठं ततो भरतसत्तमम्}
{अश्वत्थामा समालिङ्ग्य करुणं पर्यदेवयत्}


\twolineshloka
{आहुस्त्वां राजशार्दूल मुख्यं सर्वधनुष्मताम्}
{धनाध्यक्षोपमं युद्धे शिष्यं सङ्कर्षणस्य च}


\twolineshloka
{कथं विवरमद्राक्षीद्भीमसेनस्तवानघ}
{बलिनं कृतिनो नित्यं सूदः पापात्मवान्नृप}


\twolineshloka
{कालो नूनं महाराज लोकेऽस्मिन्बलवत्तरः}
{पश्यामो निहतं त्वां च भीमसेनेन संयुगे}


\twolineshloka
{कथं त्वां सर्वधर्मज्ञं क्षुद्रः पापो वृकोदरः}
{निकृत्या हतवान्मन्दो नून कालो दुरत्ययः}


\twolineshloka
{द्वन्द्वयुद्धे ह्यधर्मेण समाहूयौजसा मृधे}
{गदया भीमसेनेन निर्भग्ने सक्थिनी तव}


\twolineshloka
{अधर्मेण हतस्याजौ मृद्यमानं पदा शिरः}
{य उपेक्षितवान्क्षुद्रं धिक्तमस्तु--युधिष्ठिरम्}


\twolineshloka
{युद्धेष्वपवदिष्यन्ति योधा नूनं वृकोदरम्}
{यावत्स्थास्यन्ति भूतानि निकृत्या ह्यसि पातितः}


\twolineshloka
{ननु रामोऽब्रवीद्राजंस्त्वां सदा यदुनन्दनः}
{दुर्योधनसमो नास्ति गदायामिति वीर्यवान्}


\twolineshloka
{श्लाघते त्वां हि वार्ष्णेयो राजन्संसत्सु भारत}
{स शिष्यो मम कौरव्यो गदायुद्ध इति प्रभो}


\twolineshloka
{यां गतिं क्षत्रियस्याहुः प्रशस्तां परमर्षयः}
{हतस्याभिमुखस्याजौ प्राप्तस्त्वमसि तां गतिम्}


\twolineshloka
{दुर्योधन न शोचामि त्वामहं पुरुषर्षभ}
{हतपुत्रौ तु शोचामि गान्धारीं पिरं च ते}


\twolineshloka
{द्वावनाथौ कृतौ वीर त्वया नाथेन वर्धितौ}
{भिक्षुकौ विचरिष्येते शोचन्तौ पृथिवीमिमाम््}


\twolineshloka
{घिगस्तु कृष्णं वार्ष्णेयमर्जुनं चापि दुर्मतिम्}
{धर्मज्ञमानिमौ यौ त्वां वध्यमानमुपेक्षताम्}


\twolineshloka
{पाण्डवाश्चापि ते सर्वे किं वक्ष्यन्ति नराधिप}
{कथं दुर्योधनोऽस्माभिर्हत इत्यनपत्रपाः}


\twolineshloka
{धन्यस्त्वमसि गान्धारे यस्वमायोधने हतः}
{प्रयातोऽभिमुखः शत्रून्धर्मेण पुरुषर्षभ}


\twolineshloka
{हतपुत्रा हि गान्धारी निहतज्ञातिबान्धवा}
{प्रज्ञाचक्षुश्च दुर्धर्षः कां दशां प्रतिपत्स्यते}


\twolineshloka
{धिगस्तु कृतवर्माणं मां कृपं च महारथम्}
{ये वयं न गताः स्वर्गं त्वां पुरस्कृत्य पार्थिवम्}


\twolineshloka
{दातारं सर्वकामानां रक्षितारं प्रजाहितम्}
{यद्वयं नानुगच्छाम त्वां धिगस्मान्नराधमान्}


\twolineshloka
{कृपस्य तव वीर्येण मम चैव पितुश्च मे}
{सभृत्यानां नरव्याघ्र रत्नवन्ति गृहाणि च}


\twolineshloka
{तव प्रसादादस्माभिः समित्रैः सह बान्धवैः}
{अवाप्ताः क्रतवो मुख्या बहवो भूरिदक्षिणाः}


\twolineshloka
{कुतश्चापीदृशं पापाः प्रवर्तिष्यामहे वयम्}
{यादृशेन पुरस्कृत्य त्वं गतः सर्वपार्थिवान्}


\threelineshloka
{वयमेव त्रयो राजन्गच्छन्तं परमां गतिम्}
{यद्वै त्वां नानुगच्छाभस्तेन तप्स्यामहे वयम्}
{}


\twolineshloka
{तत्स्वर्गहीना हीनार्थाः स्मरन्तः सुकृतस्य ते}
{किं नाम तद्भवेत्कर्म येन त्वां न व्रजाम वै}


\twolineshloka
{दुःखं नून कुरुश्रेष्ठ चरिष्याम महीमिमाम्}
{हीनानां नस्त्वया राजन्कुतः शान्तिः कुतः सुखं}


\twolineshloka
{गत्वैव तु महाराज समेत्य च महारथान्}
{यथाज्येष्ठं यथाश्रेष्ठं पूजयेर्वचनान्मम}


\twolineshloka
{आचार्यं पूजयित्वा च केतुं सर्वधनुष्मताम्}
{हतं मयाऽद्य शंशेथा दृष्टद्युम्नं नराधिप}


\twolineshloka
{परिष्वजेथा राजानं बाह्लिकं सुमहारथम्}
{सैन्धवं सोमदत्तं च भूरिश्रवसमेव च}


\threelineshloka
{तथा पूर्वगतानन्यान्स्वर्गे पार्थिवसत्मान्}
{अस्मद्वाक्यात्परिष्वज्य सम्पृच्छेस््वमनामयम् ॥सञ्जय उवाच}
{}


\twolineshloka
{इत्येवमुक््वा राजानं भग्नसक्थमचेतसम्}
{अश्वत्थामा लघुप्राणं पुनर्वचनमब्रवीत्}


\twolineshloka
{दुर्योधन जीवसि चेद्वाक्यं श्रोत्रमुखं शृणु}
{सप् पाण्डवतः शिष्टा धार्तराष्ट्रास्त्रयो वयम्}


\twolineshloka
{ते चैव भ्रातरः पञ्च वासुदेवोऽथ सात्यकिः}
{अहं च कृतवर्मा च कृपः शारद्वतस्तथा}


\twolineshloka
{द्रौपदेया हताः सर्वे धृष्टद्युम्नस्य चात्मजाः}
{पञ्चाला निहताः सर्वे मत्स्यशेषं च भारत}


\twolineshloka
{कृते प्रतिकृतं पश्य हतपुत्रा हि पाण्डवाः}
{सौप्तिके शिबिरं तेषां हतं सनरवाहनम्}


\twolineshloka
{मया च पापकर्माऽसौ धृष्टद्युम्नो महीपते}
{प्रविश्य शिबिरं रात्रौ पशुमारेण मारितः}


\twolineshloka
{दुर्योधनस्तु तां वाचं निशम्य मनसः प्रियाम्}
{प्रतिलभ्य पुनश्चेत इदं वचनमब्रवीत्}


\twolineshloka
{न मेऽकरोत्द्गाङ्गेयो न कर्णो न च ते पिता}
{यत्त्वया कृपभोजाभ्यां सहितेनाद्य मे कृतम्}


\twolineshloka
{स च सेनापतिः क्षुद्रो हतः सार्धं शिखण्डिना}
{तेन मन्ये मघवता सममात्मानमद्य वै}


\twolineshloka
{स्वस्ति प्राप्नुत भद्रं वः स्वर्गे नः सङ्गमः पुनः}
{इत्येवमुक्त्वा पुत्रस्ते कुरुराजो महामनाः}


\twolineshloka
{प्राणानुपासृजद्वीरः सुहृदां दुःखमादधत्}
{अपाक्रामद्दिवं पुण्यां शरीरं क्षितिमाविशत्}


\twolineshloka
{एवं ते निधनं यातः पुत्रो दुर्योधनो नृप}
{अग्रे यात्वा रणे शूरः पश्राद्विनिहतः परैः}


\twolineshloka
{तथैव ते परिष्वक्ताः परिष्वज्य च ते नृपम्}
{पुनः पुनः प्रेक्षमाणाः स्वकानारुरुहू रथान्}


\twolineshloka
{इत्येवं द्रोणपुत्रस्य निशम्य करुणां गिरम्}
{प्रत्यूषकाले शोकार्ताः प्राद्रवन्नगरं प्रति}


\twolineshloka
{एवमेष क्षयो वृत्तः कुरुपाण्डवसेनयोः}
{घोरो विशसनो रौद्रो राजन्दुर्मन्त्रिते तव}


\threelineshloka
{तव पुत्रे गते स्वर्गं शोकार्स्य ममानघ}
{ऋषिदत्तं प्रनष्टं तद्दिव्यदर्शित्वमद्य वै ॥वैशम्पायन उवाच}
{}


\twolineshloka
{इति श्रुत्वा स नृपतिर्ज्ञातिपुत्रवधं तदा}
{निःश्वस्य दीर्घमुष्णं च ततश्चिन्तापरोऽभवत्}


\chapter{अध्यायः १०}
\twolineshloka
{वैशम्पायन उवाच}
{}


\threelineshloka
{तस्यां रात्र्यां व्यतीतायां धृष्टद्युम्नस्य सारथिः}
{गत्वा शशंस पाण्डुभ्यः सौप्तिके कदनं कृतम् ॥सूत उवाच}
{}


\twolineshloka
{द्रौपदेया हता राजन्द्रुपदस्वात्मजैः सह}
{प्रमत्ता निशि विश्वस्ताः स्वपन्तः शिबिरे स्वके}


\twolineshloka
{गौतमेन नृशंसेन भोजेन कृतवर्मणा}
{अश्वत्थान्ना च पापेन हतं वः शिबिरं निशि}


\twolineshloka
{एतैर्नरगजाश्वानां प्रासशक्तिपरश्वयैः}
{सहस्रामि विकृन्द्भिर्निः शेषं शिबिरं कृतम्}


\threelineshloka
{`xxxxxxविहगफलभारनतस्य ह}
{'xxxxxxxमहतो वनस्येव परश्वथैः}
{xxxxxxxx सुमहाञ्शब्दो बलस्य तव भारत}


\twolineshloka
{महमेकोऽबशिष्टस्तु तस्मात्सैन्यामहीपते}
{मुक्तः कथञ्चिद्धर्मात्मन्व्यग्राच्च कृतवर्मणः}


\twolineshloka
{तच्छ्रुत्वा वाक्यमशिवं कुन्तीपुत्रो युधिष्ठिरः}
{पपात मह्यां धर्मात्मा पुत्रशोकसमन्वितः}


\twolineshloka
{पतन्तं तमतिक्रम्य परिजग्राह सात्यकिः}
{भीमसेनोऽर्जुनश्चैव माद्रीपुत्रौ च पाण्डवौ}


\twolineshloka
{लब्धचेतास्तु कौन्तेयः शोकविह्वलया गिरा}
{जित्वा शत्रूञ्जितः पश्चात्पर्यदेवयदार्तवत्}


\threelineshloka
{`अगम्या गतिरर्थानां कर्मणामीश्वरस्य च'}
{दुर्विदा गतिरर्थानामपि ये दिव्यचक्षुषः}
{जीयमाना जयन्त्यन्ये जयमाना वयं जिताः}


\twolineshloka
{हत्वा भ्रातॄन्वयस्यांश्च पितॄन्पुत्रान्सुहृद्गणान्}
{बन्धूनमात्यान्पौत्रांश्च जित्वा सर्वाञ्जिता वयम्}


\twolineshloka
{अनर्थो ह्यर्थसङ्काशस्तथाऽनर्थोऽर्थदर्शनः}
{जयोऽयमजयाकारो जयस्तस्मात्पराजयः}


\twolineshloka
{यज्जित्वा तप्यते पश्चादापन्न इव दुर्मतिः}
{कथं मन्येत विजयं ततो जिततरः परैः}


\twolineshloka
{येषामर्थाय पापं स्याद्विजयस्य सुहृद्वधैः}
{निर्जितैरप्रमत्तैर्हि विजिता जितकाशिनः}


\twolineshloka
{कर्णिनालीकदंष्ट्रस्य स्वङ्गजिह्वस्य संयुगे}
{चापव्यात्तास्यरौद्रस्य ज्यातलस्वननादिनः}


\twolineshloka
{क्रुद्धस्य नरसिंहस्य सङ्ग्रामेष्वपलायिनः}
{ये व्यमुञ्चन्त कर्णस्य प्रमादात्त इमे हताः}


\twolineshloka
{रथहदं शरवर्षोर्मिमन्तंरत्नाचितं वाहनयोधबृन्दम्}
{शक्त्यृष्टिमीनध्वजनागनक्रंशरासनावर्तमहेषुफेनम्}


\twolineshloka
{सङ्ग्रामचन्द्रोदयवेगवेलंद्रोणार्णवं ज्यातलनेमिघोषम्}
{ये तेरुरुच्चावचशस्त्रनौभि--स्ते राजपुत्रा निहताः प्रमादात्}


\twolineshloka
{न हि प्रमादात्परमस्ति कश्च--द्वधो नराणामिह जीवलोके}
{प्रमत्तमर्था हि नरं समन्ता--त्त्यजन्त्यनर्थाश्च समाविशन्ति}


\twolineshloka
{ध्वजोत्तमाग्रोच्छ्रितधूमकेतुंशरार्चिषं दीप्तमहापताकम्}
{महाधनुर्ज्यातलनेमिघोषंतनुत्रनानाविधशस्त्रहोमम्}


\twolineshloka
{महाचमूकक्षदवाभिपन्नंमहाहवे भीष्ममहादवाग्निम्}
{ये तेरुरुच्चावचशस्त्रवेगै--स्ते राजपुत्रा निहताः प्रमादात्}


\twolineshloka
{न हि प्रमत्तेन नरेण शक्य--माप्तुं वसु श्रीर्विपुलं यशो वा}
{पश्याप्रमादेन निहत्य शत्रू--न्सर्वान्महेन्द्रं सुखमेधमानम्}


\threelineshloka
{इन्द्रोपमान्पार्थिवपुत्रपौत्रा--न्पश्याविशेषेण हतान्प्रमादात्}
{तीर्त्वा समुद्रं जणिजः समृद्वा}
{मग्नाः कुनद्यामिव सीदमानाः}


\twolineshloka
{अमर्षितैर्ये निहता नरेन्द्रानिःसंशयं ते त्रिदिवं प्रपन्नाः}
{कृष्णां तु शोचामि कथं नु साध्वीशोकार्णवं सा विषहिष्यतीति}


\twolineshloka
{भातृंश्च पुत्रांश्च हतान्निशम्यपाञ्चालसराजं पितरं च वृद्वम्}
{ध्रुवं विसञ्ज्ञा पतिता पृथिव्यांसा शोष्यते शोककृशाङ्गयष्टिः}


\twolineshloka
{तच्छोकजं दुःखमपारयन्तीकथं भविष्यत्युचिता सुखानाम्}
{रोरूयते ज्ञातिवधाभितप्ताप्रदह्वमानेव हुताशनेन}


\twolineshloka
{इत्येवमार्तः परिदेवयन्सराजाज कुरूणां नकुलं बभाषे}
{गच्छान्यैनामिह मन्दभाग्यांसमातृपक्षामिति राजपुत्रीम्}


\twolineshloka
{माद्रीसुतस्तत्परिगृह्य वाक्यंधर्मेण धर्मप्रथमस्य राज्ञः}
{ययौ रथेनालयमाशुदेव्याःपाञ्चालराजस्य च यत्र दाराः}


\twolineshloka
{प्रस्थाप्य माद्रीसुतमाजमीढःशोकार्दितस्तैः सहितः सुहृद्भिः}
{रोरूयमाणः प्रययौ सुताना--मायोधनं भूतगणानुकीर्णम्}


\twolineshloka
{स तत्प्रविश्याशिवमुग्ररूपंददर्श पुत्रान्सुहृदः सखींश्च}
{भूमौ शयानान्रुधिरार्द्रगात्रा--न्विभिन्नदेहान्प्रहृतोत्तमाङ्गान्}


\twolineshloka
{स तांस्तु दृष्ट्वा भृशमार्तरूपोयुधिष्ठिरो धर्मभृतां वरिष्ठः}
{उच्चैः प्रचुक्रोश च कौरवाग्र्यःपपात चोर्व्यां सगणो विसंज्ञः}


\chapter{अध्यायः ११}
\twolineshloka
{वैशम्पायन उवाच}
{}


\twolineshloka
{स दृष्ट्वा निहतान्सङ्ख्ये पुत्रान्पात्रौन्सखींस्तथा}
{महादुःखपरीतात्मा बभूव जनमेजय}


\twolineshloka
{ततस्तस्य महाञ्शोकः प्रादुरासीन्महात्मनः}
{स्मरतः पुत्रपौत्रांस्तान्भ्रातॄन्सुहृद एव च}


\twolineshloka
{तमश्रुपरिपूर्णाक्षं वेपमानमचेतसम्}
{सुहृदो भृशसंविग्नाः सांत्वयाञ्चक्रिरे तदा}


\twolineshloka
{`कृत्वा तु विधिवत्तेषां पुत्राणाममितौजसाम्}
{प्रेतकार्याणि सर्वेषां बभूव भृशदुःखितः'}


\twolineshloka
{तस्मिन्मुहूर्ते जवनैर्वाजिभिर्हेममालिभिः}
{नकुलः कृष्णया सार्धमुपायात्परमार्तया}


\twolineshloka
{उपप्लाव्यं गता सा तु श्रुत्वा सुमहदप्रियम्}
{तदा विनाशं सर्वेषां पुत्राणां व्यथितेन्द्रिया}


\twolineshloka
{कम्पमानेव कदली वातेनाभिसमीरिता}
{कृष्णा राजानमासाद्य शोकार्ता न्यपतद्भुवि}


\twolineshloka
{न बभौ वदनं तस्या रुदन्त्याः शोककर्शितम्}
{फुल्लपद्मपलाशाक्ष्या मेघावृत इवोडुराट्}


\twolineshloka
{ततस्तां पतितां दृष्ट्वा संरम्भी सत्यविक्रमः}
{वाहुभ्यां परिजग्राह समुत्पत्य वृकोदरः}


\twolineshloka
{सा समाश्वासिता तेन भीमसेनेन भामिनी}
{रुदती पाण्डवज्येष्ठमिदं वचनमब्रवीत्}


\twolineshloka
{दिष्ट्या राजन्नवाप्येमामखिलां भोक्ष्यसे महीम्}
{आत्मजान्क्षत्रधर्मेण सम्प्रदाय यमाय वै}


\twolineshloka
{दिष्ट्या सर्वास्त्रकुशलं मत्तमातङ्गामिनम्}
{अवाप्य पृथिवीं कृत्स्नां सौभद्रं न स्मरिष्यसि}


\twolineshloka
{आत्मजान्क्षत्रधर्मेण श्रुत्वा शूरान्निपातितान्}
{स्थितो राज्ये मया सार्धं विहरन्न स्मरिष्यसि}


\twolineshloka
{प्रसुप्तानां वधं श्रुत्वा द्रौणिना पापकर्मणा}
{शोको मां दहते गाढो हुताशन इवाश्रयम्}


\twolineshloka
{तस्य पापकृतो द्रौणेर्न चेदद्य दुरात्मनः}
{हियते सानुबन्धस्य युधि विक्रम्य जीवितम्}


\twolineshloka
{इहैव प्रायमासिष्ये तन्निबोधत पाण्डवाः}
{न चेत्फलमवाप्नोति द्रौणिः पापस्य कर्मणः}


\twolineshloka
{एवमक्त्वा ततः कृष्णा पाण्डवं प्रत्युपाविशत्}
{युधिष्ठिरं याज्ञसेनी धर्मराजं यशस्विनी}


\twolineshloka
{दृष्ट्वोपविष्टां राजा तु पाण्डवो महिषीं प्रियाम्}
{प्रत्युवाच स धर्मात्मा द्रौपदीं चारुदर्शनाम्}


\twolineshloka
{क्षत्रधर्मेण धर्मज्ञे प्राप्तास्ते निधनं शुभे}
{पुत्रास्ते भ्रातरश्चैव तान्न शोचितुमर्हसि}


\threelineshloka
{स कल्याणि वनं दुर्गं दूरं द्रौणिरितो गतः}
{तस्य त्वं पातनं सङ्ख्ये कथं ज्ञास्यसि शोभने ॥द्रौपद्युवाच}
{}


\threelineshloka
{द्रोणपुत्रस्य सहजो मणिः शिरसि मे श्रुतः}
{निहत्य सङ्ख्ये तं पापं पश्येयं मणिमाहृतम्}
{द्रौणेः शिरस उत्कृत्य जीवेयमिति मे मतिः}


\twolineshloka
{इत्युक्त्वा पाण्डवं कृष्णा राजानं चारुदर्शना}
{भीमसेनकरे स्पृष्ट्वा कुपिता वाक्यमब्रवीत्}


\twolineshloka
{त्रातुमर्हसि मां भीम क्षत्रधर्ममनुस्मरन्}
{जहि तं पापकर्माणं शम्बरं सघवानिव}


\twolineshloka
{न हि ते विक्रमे तुल्यः पुमानस्तीह कश्चन}
{श्रुतं तत्सर्वलोकेषु परमव्यसने तथा}


\twolineshloka
{द्वीपोऽभूस्त्वं हि पार्थानां नगरे वारणावते}
{हिडिम्बदर्शने चैव तथां त्वमभवो गतिः}


\twolineshloka
{तथा विराटनगरे कीचकेन भृशार्दिताम्}
{मामप्युद्वृतवान्कृच्छ्रात्पौलोमीं मघवानिव}


\threelineshloka
{यथैतान्यकृथाः पार्थ महाकर्माणि वै पुरा}
{तथा द्रौणिममित्रघ्न विनिहत्य सुखी भव ॥वैशम्पायन उवाच}
{}


\twolineshloka
{तस्या बहुविधं दुःखं निशम्य परिदेवितम्}
{न चामर्षत कौन्तेयो भीमसेनो महाबलः}


\twolineshloka
{स काञ्चनविचित्राङ्गमारुरोह महारथम्}
{आदाय रुचिरं चित्रं समार्गणगुणं धनुः}


\twolineshloka
{नकुलं सारथिं कृत्वा द्रोणपुत्रवधे धृतः}
{विस्फार्य सशरं चापं तूर्णमश्वानचोदयत्}


\twolineshloka
{ते हयाः पुरुषव्याघ्र दीप्यमानाः स्वतेजसा}
{वहन्तः सहसा जग्मुर्हरयः शीघ्रगामिनः}


\twolineshloka
{शिबिरात्स्वाद्गृहीत्वा स रथस्य पदमच्युतः}
{`द्रोणपुत्रगतेनाशु ययौ मार्गेण भारत'}


\chapter{अध्यायः १२}
\twolineshloka
{वैशम्पायन उवाच}
{}


\twolineshloka
{तस्मिन्प्रयाते दुर्धर्षे यदूनामृषभस्ततः}
{अब्रवीत्पुण्डरीकाक्षः कुन्तीपुत्रं युधिष्ठिरम्}


\twolineshloka
{एष पाण़्डवे ते भ्राता पुत्रशोकपरायणः}
{जिघांसुर्द्रौणिमाक्रन्दे एक एवाभिधावति}


\twolineshloka
{भीमः प्रियस्ते सर्वेभ्यो भ्रातृभ्यो भरतर्षभ}
{तं कृच्छ्रगतमद्य त्वं कस्मान्नाभ्युपपद्यसे}


\twolineshloka
{यत्तदाचष्ट पुत्राय द्रोणः परपुरञ्जयः}
{अस्त्रं ब्रह्मशिरो नाम दहेत पृथिवीमपि}


\twolineshloka
{तन्महात्मा महाभागः केतुः सर्वधनुष्मताम्}
{प्रत्यपायदाचार्यः प्रीयमाणो धनञ्जयम्}


\twolineshloka
{तं पुत्रोऽप्येक एवैनमन्वयाचदमर्षणः}
{ततः प्रोवाच पुत्राय नातिहृष्टमना इव}


\twolineshloka
{विदितं चापलं ह्यासीदात्मजस्य दुरात्मनः}
{सर्वधर्मविदाचार्यः सोऽन्वशासत्सुतं ततः}


\twolineshloka
{परमापद्गतेनापि न स्म तात त्वया रणे}
{इदमस्त्रं प्रयोक्तव्यं मानुषेषु विशेषतः}


\twolineshloka
{इत्युक्त्वान्गुरुः पुत्रं द्रोणः पश्चाथोक्तवान्}
{न त्वं जातु सतां मार्गे स्थातेति पुरुषर्षभ}


\twolineshloka
{स तदाज्ञाय दुष्टात्मा पितुर्वचनमप्रियम्}
{निराशः सर्वकल्याणैः शोकात्पर्यचरन्महीम्}


\twolineshloka
{ततस्तदा कुरुश्रेष्ठ वनस्थे त्वयि भारत}
{अवसद्द्वारकामेत्य वृष्णिभिः परमार्चितः}


\twolineshloka
{स कदाचित्समुद्रान्ते वसन्द्वारवतीमनु}
{एक एकं समागम्य मामुवाच हसन्निव}


\twolineshloka
{यत्तदुग्रं तपः कृष्ण चरन्नमितविक्रमः}
{अगस्त्याद्भारताचार्यः प्रत्यपद्यत मे पिता}


\twolineshloka
{अस्त्रं ब्रह्मशिरो नाम देवगन्धर्वपूजितम्}
{तदद्य मयि दाशार्ह यथा पितरि मे तथा}


\twolineshloka
{अस्मत्तस्दुपादाय दिव्यमस्त्रं यदूत्तम}
{ममाप्यस््रं प्रयच्छ त्वं चक्रं रिपुहणं रणे}


\twolineshloka
{स राजन्प्रीयमाणेन मयाप्युक्तः कृताञ्जलिः}
{याचमानः प्रयत्नेन मत्तोऽस्त्रं भरतर्षभ}


\twolineshloka
{देवदानवगन्धर्वमनुष्यपतगोरगाः}
{न समा मम वीर्यस्य शतांशेनापि पिण्डिताः}


\twolineshloka
{इदं धनुरियं शक्तिरिदं चक्रमियं गदा}
{यद्यदिच्छसि चेदस््रं मत्तस्तत्तद्ददामि ते}


\twolineshloka
{यच्छक्नोषि समुद्यन्तुं प्रयोक्तुमपि वा रणे}
{तद्गृहाण विनाऽस्त्रेण यन्मे दातुमभीप्ससि}


\twolineshloka
{स सुनाभं सहस्रारं वज्रनाभमयस्मयम्}
{वव्रे चक्रं महाभागो मत्तः स्पर्धन्मया सह}


\twolineshloka
{गृहाण चक्रमित्युक्तो मया तु तदनन्तरम्}
{जग्राहोत्पत्य सहसा चक्रं सव्येन पाणिना}


\twolineshloka
{न चैनमशकत्स्थानात्सञ्चालयितुमप्युत}
{अथैनं दक्षिणेनापि ग्रहीतुमुपचक्रमे}


% Check verse!
सर्वयत्नेन तेनापि गृह्य नैनमकम्पयत्
\threelineshloka
{ततः सर्वबलेनापि यदैनं न शशाक ह}
{उद्यन्तुं वा चालयितुं द्रौणिः परमदुर्मनाः}
{कृत्वा यत्नं परिश्रान्तः स न्यवर्तत भारत}


\twolineshloka
{निवृत्मनसं तस्मादभिप्रायाद्विचेतसम्}
{अहमामन्त्र्य संविग्नमश्वत्थामानमब्रुवम्}


\twolineshloka
{यः स दैवमनुष्येषु प्रमाणं परमं गतः}
{गाण्डीवधन्वा श्वेताश्चः कपिप्रवरकेतनः}


\twolineshloka
{यः साक्षाद्देवदेवेशं शितिकण्ठमुमापतिम्}
{द्वन्द्वयुद्धे पुरा जिष्णुस्तोषयामास शङ्करम्}


\twolineshloka
{यस्मात्प्रियतरो नास्ति ममान्यः पुरुषो भुवि}
{नादेयं यस्य मे किञ्चिदपि प्राणान्महात्मनः}


\twolineshloka
{तेनापि सुहृदा ब्रह्मन्पार्थेनाक्लिष्टकर्मणा}
{नोक्पूर्वमिदं वाक्यं यस्त्वं मामभिभाषसे}


\twolineshloka
{ब्रह्मचर्यं महद्धोरं तीर््वा द्वादशवार्षिकम्}
{हिमवत्पार्श्वमास्थाय यो मया तपसाऽऽर्जितः}


\twolineshloka
{समानव्रतचारिण्यां रुक्मिण्यां योऽन्वजायत}
{सनत्कुमारस्तेजस्वी प्रद्युम्नो नाम मे सुतः}


\twolineshloka
{तेनाप्येतन्महद्दिव्यं चक्रमप्रतिमं रणे}
{न प्रार्थितमभून्मूढ तदितं प्रार्थितं त्वया}


\twolineshloka
{रामेणातिबलेनैतन्नोक्तपूर्वं कदाचन}
{न गदेन साम्बेन यदिदं प्रार्थितं त्वया}


\twolineshloka
{द्वारकावासिभिश्चान्यैर्वृष्ण्यन्धकमहारथैः}
{नोक्तपूर्वमिदं क्षुद्रं तदिदं प्रार्थितं त्वया}


\twolineshloka
{भारताचार्यपुत्रस्त्वं मानितः सर्वयादवैः}
{चक्रेण रथिनां श्रेष्ठ कं नु तात युयुत्ससे}


\twolineshloka
{एवमुक्तो मया द्रौणिर्मामिदं प्रत्युवाच ह}
{प्रयुज्य भवते पूजां योत्स्ये कृष्ण त्वया सह}


\twolineshloka
{प्रार्थितं ते मया चक्रं देवदानवपूजितम्}
{अजेयः स्यामिति विभो सत्यमेद्ब्रवीमि ते}


\twolineshloka
{सोऽहं तद्दुर्लभं चक्रमनवाप्यैव केशव}
{प्रतियास्यामि गोविन्द शिवेनाभिवदस्व माम्}


\twolineshloka
{एतत्सुनाभं भोजानामृषभेण त्वया धृतम्}
{चक्रमप्रतिचक्रेण भुवि नान्योऽभिपद्यते}


\twolineshloka
{एतावदुक्त्वा द्रौणिर्मां युग्यानश्वान्धनानि च}
{आदायोपययौ काले रत्नानि वविधानि च}


\twolineshloka
{स संरम्भी दुरात्मा च चपलः क्रूर एव च}
{वेद चास््रं ब्रह्मशिरस्तस्माद्रक्ष्यो वृकोदरः}


\chapter{अध्यायः १३}
\twolineshloka
{वैशम्पायन उवाच}
{}


\twolineshloka
{एवमुक्त्वा कुरुश्रेष्ठं सर्वयादवनन्दनः}
{सर्वायुधवरोपेतमारुरोह रथोत्तमम्}


\twolineshloka
{युक्तं परमकाम्भोजैस्तुरगैर्हेममालिभिः}
{उदितादित्यसङ्काशं सर्वरत्नविभूषितम्}


\twolineshloka
{दक्षिणे ह्यावहच्छैब्यः सुग्रीवः सव्यतो धुरम्}
{णर्ष्णिवाहौ तु तस्यास्तां मेघपुष्पबलाहकौ}


\twolineshloka
{विश्वकर्मकृता दिव्या रत्नधातुविभूषिता}
{उच्छ्रिता च रथे तस्मिन्ध्वजयष्टिरदृश्यत}


% Check verse!
वैनतेयः स्थितस्तस्यां प्रभामण्डलरश्मिवान् ॥तस्य सत्यवतः केतुर्भुजगारिरदृश्यत
\twolineshloka
{अथारोहद्धृषीकेशः केतुः सर्वधनुष्मताम्}
{अर्जुनः स च धर्मात्मा कुरुराजो युधिष्ठिरः}


\twolineshloka
{अशोभेतां महात्मानौ दाशार्हमभितः स्थितौ}
{रथस्थं शार्ङ्गधन्वानमश्विनाविव वासवम्}


\twolineshloka
{उभावारोप्य दाशार्हः स्यन्दनं लोकपूजितम्}
{प्रतोदेन जवोपेतान्परमाश्वानचोदयत्}


\twolineshloka
{ते हयाः सहसोत्पेतुर्गृहीत्वा स्यन्दनोत्तमम्}
{आस्थितं पाण्डवेयाभ्यां यदूनामृषभेण च}


\twolineshloka
{वहतां शार्ङ्गधन्वानमश्वानां शीघ्रगामिनाम्}
{प्रादुरासीन्महाञ्शब्दः पक्षिणां पततामिव}


\twolineshloka
{ते समर्था महाबाहुं क्षणेन भरतर्षभ}
{भीमसेनं महेष्वासमनुसस्रुः सुवेगिताः}


\twolineshloka
{क्रोधदीप्तं तु कौन्तेयं द्विषदर्थे समुद्यतम्}
{नाशक्नुवन्वारयितुं समेत्यापि महारथाः}


\threelineshloka
{स तेषामग्रतः शूरः श्रीमतां दृढधन्विनाम्}
{ययौ भागीरथीतीरं हरिभिर्भृशवेगितैः}
{यत्र स श्रूयते द्रौणिः पुत्रहन्ता दुरात्मवान्}


\twolineshloka
{स ददर्श महात्मानमुदकान्ते यशस्विनम्}
{कृष्णद्वैपायनं व्यासमासीनमृषिभिः सह}


\twolineshloka
{तं चैव क्रूरकर्माणं घृताक्तं कुशचीरिणम्}
{रजसा ध्वस्तमासीनं ददर्श द्रौणिमन्तिके}


\twolineshloka
{तमभ्यधावत्कौन्तेयः प्रगृह्य सशरं धनुः}
{भीमसेनो महाबाहुस्तिष्ठतिष्ठेति चाब्रवीत्}


\threelineshloka
{तं दृष्ट्वा भीमकर्माणं प्रगृहीतशरासनम्}
{भ्रातरौ पृष्ठतश्चास्य जनार्दनरथे स्थितौ}
{व्यथितात्माऽभवद्द्रौणिः प्राप्तं चेदममन्यत}


\twolineshloka
{स तद्दिव्यमदीनात्मा परमास््रमचिन्तयत्}
{जग्राह च शरैषीकां द्रौणिः सव्येन पाणिना}


\threelineshloka
{स तामापदमासाद्य दिव्यमस्त्रमुदैरयत्}
{अमृष्यमाणस्ताञ्छूरान्दिव्यायुधधरान्स्थितान्}
{अपाण्डवायेति रुषा वाचमुत्सृज्य दारुणम्}


\twolineshloka
{इत्युक्त्वा राजशार्दूल द्रोणपुत्रः प्रतापवान्}
{सर्वलोकप्रमोहार्थं तदस््रं प्रमुमोच ह}


\twolineshloka
{ततस्तस्यामिषीकायां पावकः समजायत}
{प्रधक्ष्यन्निव लोकांस्त्रीन्कालान्कयमोपमः}


\chapter{अध्यायः १४}
\twolineshloka
{वैशम्पायन उवाच}
{}


\twolineshloka
{इङ्गितेनैव दाशार्हस्तस्याभिप्रायमादितः}
{द्रौणेर्बुद्धा महाबाहुरर्जुनं प्रत्यभाषत}


\twolineshloka
{अर्जुनार्जुन यद्दिव्यमस्त्रं ते हृदि वर्तते}
{द्रोणोपदिष्टं तस्यायं कालः सम्प्रति पाण्डव}


\twolineshloka
{भ्रातॄणामात्मनश्चैव परित्राणाय भारत}
{विसृजैतत्त्वमप्याजावस्त्रमस्त्रनिवारणम्}


\twolineshloka
{केशवेनैवमुक्तोऽथ पाण्डवः परवीरहा}
{अवातरद्रथात्तूर्णं प्रगृह्य सशरं धनुः}


\twolineshloka
{पूर्वमाचार्यपुत्राय ततोऽनन्तरमात्मने}
{भ्रातृभ्यश्चैव सर्वेभ्यः स्वस्तीत्युक्त्वा परन्तपः}


\twolineshloka
{देवताभ्यो नमस्कृत्य गुरुभ्यश्चैव सर्वशः}
{उत्ससर्ज शिवं ध्यायन्नस्त्रमस्त्रेण शाम्यताम्}


\twolineshloka
{ततस्तदस््रं सहसा सृष्टं गाण्डीवधन्वना}
{प्रजज्वाल महार्चिष्मद्युगान्तानलसन्निभम्}


\twolineshloka
{तथैव द्रोणपुत्रस्य तदस्त्रं तिग्मतेजसः}
{प्रजज्वाल महाज्वालं तेजोमण्डलसंवृतम्}


\twolineshloka
{निर्घाता बहवश्चासन्पेतुरुल्काः सहस्रशः}
{महद्भयं च भूतानां सर्वेषां समजायत}


\twolineshloka
{सशब्दमभवद्व्योम ज्वालामालाकुलं भृशम्}
{चचाल च मही कृत्स्ना सपर्वतवनद्रुमा}


\twolineshloka
{तावस्त्रतेजसा लोकांस्त्रासयन्तौ ततः स्थितौ}
{महर्षी सहितौ तत्र दर्शयामासतुस्तदा}


\twolineshloka
{नारदः सर्वधर्मात्मा भरतानां पितामहः}
{उभौ शमयितुं वीरौ भारद्वाजधनञ्जयौ}


\twolineshloka
{तौ मुनी सर्वधर्मज्ञौ सर्वभूतहितैपिणौ}
{दीप्तयोरस््रयोर्मध्ये स्थितौ परमतेजसौ}


\twolineshloka
{तदन्तरमनाधृष्यावुपगम्य यशस्विनौ}
{आस्तामृषिवरौ तत्र ज्वलिताविव पावकौ}


\threelineshloka
{प्राणभृद्भिरनाधृष्यौ देवदानवसम्मतौ}
{अस्त्रतेजः शमयितुं लोकानां हितकाम्यया ॥ऋषी ऊचतुः}
{}


\threelineshloka
{महास्त्रविदुषः पूर्वे येऽप्यतीता महारथाः}
{नैतदस््रं मनुष्येषु तैः प्रयुक्तं कथञ्चन}
{किमिदं साहसं वीरौ कृतवन्तौ महात्ययम्}


\chapter{अध्यायः १५}
\twolineshloka
{वैशम्पायन उवाच}
{}


\threelineshloka
{दृष्टैव मुनिशार्दूलौ तावग्निसमतेजसौ}
{`गाण्डीवधन्वा सञ्चिन्त्य प्राप्तकालं महारथः}
{'सञ्जहार शरं दिव्यं त्वरमाणो धनञ्जयः}


\twolineshloka
{उवाच वचनं श्रेष्ठस्तावृषी प्राञ्जलिस्तदा}
{प्रमुक्तमस्त्रमस्त्रेण शाम्यतामिति वै मया}


\twolineshloka
{संहृते परमास्त्रेऽस्मिन्सर्वानस्मानशेषतः}
{पापकर्मा ध्रुवं द्रौणिः प्रधक्ष्यत्यस्त्रतेजसा}


\twolineshloka
{यदत्र हितमस्माकं लोकानां चैव सर्वथा}
{भवन्तौ देवसङ्काशौ तथा सम्मन्तुमर्हतः}


\twolineshloka
{इत्युक्त्वा सञ्जहारास्त्रं पुनरेव धनञ्जयः}
{संहारो दुष्करस्स्य देवैरपि हि संयुगे}


\twolineshloka
{विसृष्टस्य रणे तस्य परमास्त्रस्य सङ्ग्रहे}
{अशक्तः पाण्डवादन्यः साक्षादपि शतक्रतुः}


\twolineshloka
{ब्रह्मतेजोद्भवं तद्धि विसृष्टमकृतात्मना}
{न शक्यमावर्तयितुं ब्रह्मचर्यव्रतादृते}


\twolineshloka
{अचीर्णब्रह्मचर्यो यः सृष्ट्वाऽऽवर्यते पुनः}
{तदस्त्रं सानुबन्धस्य मूर्धानं तस्य कृन्तति}


\twolineshloka
{ब्रह्मचारी व्रती चापि दुरवापमवाप्य तत्}
{परमव्यसनार्तोऽपि नार्जुनोऽस्त्रं व्यमुञ्चत}


\twolineshloka
{सत्यव्रतधरः शूरो ब्रह्मचारी च पाण़्डवः}
{गुरुवर्ती च तेनास्त्रं सञ्जहारार्जुनः पुनः}


\twolineshloka
{द्रौणिरप्यथ सम्प्रेभ्य सोऽन्तरा तावृषी स्थितौ}
{न शशाक पुनर्घोरमस्त्रं संहर्तुमोजसा}


\twolineshloka
{अशक्तः प्रतिसंहारे परमास्त्रस्य संयुगे}
{द्रौणिर्दीनमना राजन्द्वैपायनमभाषत}


\twolineshloka
{उत्तमव्यसनार्तेन प्राणत्राणमभीप्सुना}
{मयैतदस्त्रमुत्सृष्टं भीमसेनभयान्मुने}


\twolineshloka
{अधर्मश्च कृतोऽनेन धार्तराष्ट्रं जिघांसता}
{मिथ्याचारेण भगवन्भीमसेनेन संयुगे}


\twolineshloka
{अतः सृष्टमिदं ब्रह्मन्मयाऽस्त्रमकृतात्मना}
{तस्य भूयोऽपि संहारं कर्तुं नाहमिहोत्सहे}


\twolineshloka
{निसृष्टं हि मया दिव्यमेतदस्त्रं दुरासदम्}
{अपाण्डवायेति मुने वह्नितेजोऽनुमन्त्र्य वै}


\twolineshloka
{तदिदं पाण्डवेयानामन्तायैवाभिसंहितम्}
{अद्य पाण्डुसुतान्सर्वाञ्जीविताद्वंशयिष्यति}


\twolineshloka
{कृतं पापमिदं ब्रह्मन्रोषाविष्टेन चेतसा}
{वधमाशास्य पार्थानां मयास्त्रं सृजता रणे}


\threelineshloka
{व्यास उवाच}
{अस्त्रं ब्रह्मशिरस्तात विद्वान्पार्थो धनञ्जयः}
{उत्सृष्टवानहिंसार्थं न रोषेण तवाहवे}


\twolineshloka
{अस्त्रमस्त्रेण तु रमे तव संशमयिष्यता}
{विसृष्टमर्जुनेनेदं पुनश्च प्रतिसंहृतम्}


\twolineshloka
{ब्रह्मास्त्रमप्यवाप्यैतदुपदेशात्पितुस्तव}
{क्षत्रधर्मान्महाबाहुर्नाकम्पत धनञ्जयः}


\twolineshloka
{एवं धृतिमतः साधोः सर्वास्त्रविदुषः सतः}
{सभ्रातृबन्धोः कस्मात्त्वं वधमस्य चिकीर्षसि}


\twolineshloka
{अस्त्रं ब्रह्मशिरो यत्र परमास्त्रेण वध्यते}
{समा द्वादश पर्जन्यस्तद्राष्ट्रं नाभिवर्षति}


\twolineshloka
{एतदर्थं महाबाहुः शक्तिमानपि पाण्डवः}
{न विहन्यात्तदस्त्रं तु प्रजाहितचिकीर्षया}


\twolineshloka
{पाण्डवास्त्वं च राष्ट्रं च सदा संरक्ष्यमेव नः}
{तस्मात्संहर दिव्यं त्वमस्त्रमेन्महाभुज}


\threelineshloka
{अरोषस्तव चैवास्तु पार्थाः सन्तु निरामयाः}
{न ह्यधर्मेण राजर्षिः पाण्डवो जेतुमिच्छति}
{}


\threelineshloka
{मणिं चैव प्रयच्छाद्य यस्ते शिरसि तिष्ठति}
{एतदादाय ते प्रामान्प्रतिदास्यन्ति पाण्डवाः ॥द्रौणिरुवाच}
{}


\twolineshloka
{पाण्डवैर्यानि रत्नानि यच्चान्यत्कौरवैर्धनम्}
{अवाप्तमिह तेभ्योऽयं मणिर्मम विशिष्यते}


\twolineshloka
{यमाबध्य भयं नास्ति शस्त्रव्याधिक्षुधाश्रयम्}
{देवेभ्यो दानवेभ्यो वा नागेभ्यो वा कथञ्चन}


\twolineshloka
{न च रक्षोगणभयं न तस्करभयं तथा}
{एवंवीर्यो मणिरयं न मे त्याज्यः कथञ्चन}


\fourlineindentedshloka
{यत्तु मे भगवानाह तन्मे कार्यमनन्तरम्}
{अयं मणिरयं चाहमिषीका तु पतिष्यति}
{गर्भेषु पाण्डुपुत्राणामुत्रायास्तथोदरे ॥वैशम्पायन उवाच}
{}


% Check verse!
प्राह द्रोणसुतं तत्र व्यासः परमदुर्मनाः
\twolineshloka
{एवं कुरु न चान्यत्र बुद्धिः कार्या कथञ्चन}
{गर्भेषु पाण्डवेयानां विसृज्यैतदुपारम}


\twolineshloka
{`तमुवाच हृषीकेशः पाण्डवानां हिते मतः}
{भविष्यमेकमुत्सृज्य गर्भेष्वस्त्रं निपात्यताम्}


\twolineshloka
{अहमेनं ददाम्येषां पिण्डदं कीर्तिवर्धनम्}
{राजर्षिं पुण्यकर्माणमनेकक्रतुयाजिनम्}


\twolineshloka
{एवं कुरु न चान्या ते बुद्धिः कार्या कथञ्चन}
{आगर्भात्पाण्डवेयानां कृत्वा पातं विनङ्क्ष्यति'}


\twolineshloka
{एवं ब्रुवाणं गोविन्दं वृषभं सर्वसात्वताम्}
{द्रौणिः परमसङ्क्रुद्धः प्रत्युवाचेदमुत्तरम्}


\twolineshloka
{नैतदेवं यदात्थ त्वं पक्षपातेन केशव}
{वचनात्पुण्डरीकाक्ष तव मद्वाक्यमन्यथा}


\threelineshloka
{पतिष्यत्येतदस्त्रं वै गर्भे तस्या मयोद्यतम्}
{विराटदुहितुः कृष्ण यं त्वं रक्षितुमर्हसि ॥वासुदेव उवाच}
{}


\twolineshloka
{अमोघः परमास्त्रस्य पातस्त्वद्य भविष्यति}
{`अभिमन्योः सृजैषीकां गर्भस्थः शाम्यतां शिशुः}


\twolineshloka
{अहमेनं मृतं जातं जीवयिष्यामि बालकम्'}
{स तु गर्भो मृतो जातो दीर्घमायुरवाप्स्यति}


\twolineshloka
{`इत्युक्तः प्रत्युवाचैनं द्रोणपुत्रः स्मयन्निव}
{यद्यस्त्रदग्धं गोविन्द जीवयस्येवमस्त्विति'}


\twolineshloka
{ततः परममस्त्रं तु द्रौणिरुद्यतमाहवे}
{द्वैपायनमनादृत्य गर्भेषु प्रमुमोच ह}


\chapter{अध्यायः १६}
\twolineshloka
{वैशम्पायन उवाच}
{}


\twolineshloka
{तदाज्ञाय हृषीकेशो विकृष्टं पापकर्मणा}
{हृष्यमाण इदं वाक्यं द्रौणिं प्रत्यब्रवीत्तदा}


\twolineshloka
{विराटस्य सुतां पूर्वं स्नुषां गाण्डीवधन्वनः}
{उपप्लाव्यगतां दृष्ट्वा ऋतवाग्ब्रह्मणोऽब्रवीत्}


\twolineshloka
{परिक्षीणेषु कुरुषु पुत्रस्तव भविष्यति}
{एतदस्य परिक्षित्त्वं गर्भस्थस्य भविष्यति}


\twolineshloka
{तस्य तद्वचनं साधोः सत्यमेद्भविष्यति}
{परिक्षिद्भविता ह्येषां पुनर्वंशकरः सुतः}


\twolineshloka
{त्वां तु कापुरुषं पापं विंदुः सर्वे मनीषिणः}
{असकृत्पापकर्माणं बालजीवितघातकम्}


\twolineshloka
{तस्मात्त्वमस्य पापस्य कर्मणः फलमाप्नुहि}
{त्रीणि वर्षसहस्राणि चरिष्यसि महीमिमाम्}


\twolineshloka
{अप्राप्नुवन्क्वचित्काञ्चित्संविदं जातु केनचित्}
{निर्जनानसहायस्त्वं देशान्प्रविचरिष्यसि}


\threelineshloka
{भवित्री न हि ते क्षुद्र जनमध्येषु संस्थितिः}
{पूयशोणिगन्धी च दुर्गकान्तारसंश्रयः}
{}


\twolineshloka
{विचरिष्यसि पापात्मंश्चिरमेको वसुन्धराम् ॥वयः प्राप्य परिक्षित्तुं देवव्रतमवाप्य च}
{}


\twolineshloka
{कृपाच्छारद्वताच्छूरः सर्वास्त्राण्युपपत्स्यते ॥विदित्वा परमास्त्राणि क्षत्रधर्मव्रते स्थितः}
{}


\twolineshloka
{षष्टिं वर्षामि धर्मात्मा वसुधां पालयिष्यति ॥इतश्चोर्ध्वं महाबाहुः कुरुराजो भविष्यति}
{}


\twolineshloka
{परिक्षिन्नाम नृपतिर्मिषतस्ते सुदुर्मते ॥[अहं तं जीवयिष्यामि दग्धं शस्त्राग्नितेजसा]}
{}


\threelineshloka
{पश्य मे तपसो वीर्यं सत्यस्य च नराधम ॥व्यास उवाच}
{यस्मादनादृत्य कृतं त्वयाऽस्मान्कर्म दारुणम्}
{ब्राह्मणस्य सतश्चेदं वृत्तमन्यायवर्तिनः}


\threelineshloka
{तस्माद्यद्देवकीपुत्र उक्वानुत्तमं वचः}
{आलोकात्तव तद्भावि क्षुद्रकर्मन्व्रजेति ह ॥अश्वत्थामोवाच}
{}


\threelineshloka
{सहैव भवता ब्रह्मन्स्थास्यामि पुरुषेष्विह}
{सत्यवागस्तु भगवानयं च पुरुषोत्तमः ॥वैशम्पायन उवाच}
{}


\twolineshloka
{प्रदायाथ मणिं द्रौणिः पाण्डवानां महात्मनाम्}
{जगाम विमनास्तेषां सर्वेषां पश्यतां वनम्}


\twolineshloka
{पाण्डवाश्च सदाशार्हास्तानृषीनभिवाद्य च}
{कृष्णद्वैपायनं चैव नारदं चैव पर्वतम्}


\threelineshloka
{द्रोणपुत्रस्य सहजं मणिमादाय सत्वराः}
{द्रौपदीमभ्यधावन्त प्रायोपेतां मनस्विनीम् ॥वैशम्पयन उवाच}
{}


\twolineshloka
{ततस्ते पुरुषव्याघ्राः सदश्वैरनिलोपमैः}
{अभ्ययुः सहदाशार्हाः शिबिरं पुनरेव हि}


\twolineshloka
{अवतीर्य रथेभ्यस्तु त्वरमाणा महारथाः}
{ददृशुर्द्रौपदीं हृष्टामार्तामार्ततराः स्वयम्}


\twolineshloka
{तामुपेत्य निरानन्दां दुःखशोकसमन्विताम्}
{परिवार्य व्यतिष्ठन्त पाण्डवाः सहकेशवाः}


\twolineshloka
{ततो राज्ञाऽभ्यनुज्ञातो भीमसेनो महाबलः}
{प्रददौ तं मणिं दिव्यं वचनं चेदमब्रवीत्}


\twolineshloka
{अयं भद्रे तव मणिः पुत्रहन्ता जितश्च ते}
{उत्तिष्ठ शोकमुत्सृज्य क्षात्रधर्ममनुस्मर}


\twolineshloka
{प्रयाणे वासुदेवस्य शमार्थवसितेक्षणे}
{यान्युक्तानि त्वया भीरु वाक्यानि मधुघातिनि}


\twolineshloka
{नैव मे पतयः सन्ति न पुत्रा भ्रातरो न च}
{न वै त्वमिति गोविन्द शममिच्छति राजनि}


\twolineshloka
{उक्तवत्यसि तीव्राणि वाक्यानि पुरुषोत्तमम्}
{क्षत्रधर्मानुरूपाणि तानि संस्मर्तुमर्हसि}


\twolineshloka
{हतो दुर्योधनः पापो राज्यस्य परिपन्थिकः}
{दुःशासनस्य रुधिरं पीतं विस्फुरतो मया}


\twolineshloka
{वैरस्य गतमानृण्यं न स्म वाच्या विवक्षताम्}
{जित्वा मुक्तो द्रोणपुत्रो ब्राह्मण्याद्गौरवेण च}


\threelineshloka
{यशोऽस्य पतितं देवि शरीरं त्ववशेषितम्}
{वियोजितश्च मणिना भ्रंशितश्चायुधं भुवि ॥द्रौपद्युवाच}
{}


\twolineshloka
{केवलानृण्यमाप्ताऽस्मि गुरुपुत्रो गुरुर्मम}
{शिरस्येतं मणिं राजा ग्रहीतुमनघोऽर्हति}


\twolineshloka
{तं गृहीत्वा ततो राजा शिरस्येवाकरोत्तदा}
{गुरोरुच्छेषमित्येव द्रौपद्या वचनादपि}


\twolineshloka
{ततो दिव्यं मणिवरं शिरसा धारयन्प्रभुः}
{शुशुभे स तदा राजा सचन्द्र इव पर्वतः}


\twolineshloka
{उत्तस्थौ पुत्रशोकार्ता ततः कृष्णा मनस्विनी}
{कृष्णं चापि महाबाहुः परिपप्रच्छ धर्मराट्}


\chapter{अध्यायः १७}
\twolineshloka
{वैशम्पायन उवाच}
{}


\twolineshloka
{हतेषु सर्वसैन्येषु सौप्तिकै तै रथैस्त्रिभिः}
{शोचन्युधिष्ठिरो राजा दाशार्हमिदमब्रवीत्}


\twolineshloka
{कथं नु कृष्ण पापेन क्षुद्रेण शठबुद्धिना}
{द्रौणिना निहताः सर्वे मम पुत्रा महारथाः}


\twolineshloka
{तथा कृतास््रविक्रान्ताः सङ्ग्रामेष्वपलायिनः}
{द्रुपदस्यात्मजाश्चैव द्रोणपुत्रेण पातिताः}


\twolineshloka
{यस्य द्रोणो महेष्वासो न प्रादादाहवे मुखम्}
{निजघ्ने रथिनां श्रेष्ठं धृष्टद्युम्नं कथं नु सः}


\threelineshloka
{किन्नु तेन कृतं कर्म तथायुक्तं नरर्षभ}
{यदेकः समरे सर्वानवधीन्नो गुरोः सुतः ॥श्रीभगवानुवाच}
{}


\twolineshloka
{नूनं स देवदेवानामीश्वरेश्वरमव्ययम्}
{जगाम शरणं द्रौणिरेकस्तेनावधीद्बहून्}


\twolineshloka
{प्रसन्नो हि महादेवो दद्यादमरतामपि}
{वीर्यं च गिरिशो दद्याद्येनेन्द्रमपि शातयेत्}


\twolineshloka
{वेदाहं हि महादेवं तत्त्वेन भरतर्षभ}
{यानि चास्यपुराणानि कर्माणि विविधानि च}


\twolineshloka
{आदिरेष हि भूतानां मध्यमन्तश्च भारत}
{विचेष्टते जगच्चेदं सर्वमस्यैव कर्मणा}


\twolineshloka
{एवं सिसृक्षुर्भूतानि ददर्श प्रथमं विभुः}
{पितामहोऽब्रवीच्चैनं भूतानि सृज माचिरम्}


\twolineshloka
{हरिकेशस्तथेत्युक््वा दीर्घदर्शी तदा प्रभुः}
{दीर्घकालं तपस्तेपे मग्नोऽम्भसि महातपाः}


\twolineshloka
{सुमहान्तं ततः कालं प्रतीक्ष्यैनं पितामहः}
{स्रष्टारं सर्वभूतानां ससर्ज मनसाऽपरम्}


\twolineshloka
{सोऽब्रवीद्वातरं दृष्ट्वा गिरिशं सुप्तमम्भसि}
{यदि मे नाग्रजोऽस्त्यन्यस्ततः स्रक्ष्याम्यहं प्रजाः}


\twolineshloka
{तमब्रवीत्पिता नास्ति त्वदन्यः पुरुषोऽग्रजः}
{स्थाणुरेष जले मग्नो विस्रब्धः कुरु वै प्रजाः}


\twolineshloka
{भूतान्यन्वसृजत्सप्त दक्षः क्षिप्रं प्रजापतिः}
{यैरिमं व्यकरोत्सर्वं भूतग्रामं चतुर्विधम्}


\twolineshloka
{ताः सृष्टमात्राः क्षुधिताः प्रजाः सर्वाः प्रजापतिम्}
{बिभक्षयिवो राजन्सहसा प्राद्रवंस्तदा}


\twolineshloka
{स भक्ष्यमाणस्त्राणार्थी पितामहमुपाद्रवत्}
{आभ्यो मां भगवांस्त्रातु वृत्तिरासां विधीयताम्}


\twolineshloka
{ततस्ताभ्यो ददावन्नमोषधीः स्थावराणि च}
{जङ्गमानि च भूतानि दुर्बलानि बलीयसाम्}


\twolineshloka
{विहितान्नाः प्रजास्तास्तु जग्मुस्तुष्टा यथागतम्}
{ततो ववृधिरे राजन्प्रीतिमत्यः स्वयोनिषु}


\twolineshloka
{भूतग्रामे विवृद्वे तु सृष्टे देवासुरे तदा}
{उदतिष्ठज्जलाज्ज्येष्ठः प्रजाश्चेमा ददर्श सः}


\twolineshloka
{बहुरूपाः प्रजाः सृष्टा विवृद्धाश्च स्वतेजसा}
{चुक्रोध बलवद्दृष्ट्वा लिङ्गं स्वं चाप्यविध्यत}


\twolineshloka
{तत्प्रविद्धं तथा भूमौ तथैव प्रत्यतिष्ठत}
{तमुवाचाव्ययो ब्रह्मा वचोभिः शमयन्निव}


\twolineshloka
{किं कृतं सलिले शर्व चिरकालस्थितेन ते}
{किमर्थं चेदमुत्पाद्य लिङ्गं भूमौ प्रवेशितम्}


\twolineshloka
{सोऽब्रवीज्जातसंरम्भस्तथा लोकगुरुर्गुरुम्}
{प्रजाः सृष्टाः परेणेमाः किं करिष्याम्यनेन वै}


\twolineshloka
{प्रजाः सृष्टाः परेणेमाः प्रजार्थं मे पितामह}
{ओषध्यः परिवर्तेरन्यथैवं सततं प्रजाः}


\twolineshloka
{एवमुक्त्वा स सक्रोधो जगाम विमना भवः}
{गिरेर्मुञ्जवतः पादं तपस्तप्तुं महातपाः}


\chapter{अध्यायः १८}
\threelineshloka
{जनमेजय उवाच}
{हते दुर्योधने चैव हते सैन्ये च सर्वशः}
{धृतराष्ट्रो महाराज श्रुत्वा किमकरोन्मुने}


