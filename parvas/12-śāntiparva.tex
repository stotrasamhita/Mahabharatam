\part{शान्तिपर्व}
\chapter{अध्यायः १}
\threelineshloka
{श्रीवेदव्यासाय नमः}
{नारायणं नमस्कृत्य नरं चैव नरोत्तमम्}
{देवीं सरस्वतीं व्यासं ततो जयमुदीरयेत्}


\threelineshloka
{वैशंपायन उवाच}
{कृत्वोदकं ते सुहृदां सर्वेषां पाण्डुनन्दनाः}
{विदुरो धृतराष्ट्रश्च सर्वाश्च भरतस्त्रियः}


\twolineshloka
{तत्र ते सुमहात्मानो न्यवसन्कुरुनन्दनाः}
{शौचं निर्वर्तयिष्यन्तो मासच्चात्रं बहिः पुरात्}


\twolineshloka
{कृतोदकं तु राजानं धर्मात्मानं युधिष्ठिरम्}
{अभिजग्मुर्महात्मानः सिद्धा ब्रह्मर्षिसत्तमाः}


\twolineshloka
{द्वैपायनो नारदश्च देवलश्च महानृषिः}
{देवस्थानश्च कण्वश्च तेषां शिष्याश्च सत्तमाः}


\twolineshloka
{अन्ये च वेदविद्वांसः कृतप्रज्ञा द्विजातयः}
{गृहस्थाः स्नातकाः सन्तो ददृशुः कुरुसत्तमम्}


\twolineshloka
{तेऽभिगम्य महात्मानं पूजिताश्च यथाविधि}
{आसनेषु महार्हेषु विविशुः परमर्षयः}


\twolineshloka
{प्रतिगृह्य ततः पूजां तत्कालसदृशीं तदा}
{पर्युपांसन्यथान्यायं परिवार्य युधिष्ठिरम्}


\twolineshloka
{पुण्ये भागीरथीतीरे शोकव्याकुलचेतसम्}
{आश्वासयन्तो राजेन्द्रं विप्राः शतसहस्रशः}


\threelineshloka
{नारदस्त्वव्रवीत्काले धर्मपुत्रं युधिष्ठिरम्}
{संभाष्य मुनिभिः सार्धं कृष्णद्वैपायनादिभिः ॥नारद उवाच}
{}


\twolineshloka
{पार्थस्य बाहुवीर्येण प्रसादान्माधवस्य च}
{जिता सेयं मही कृत्स्ना धर्मेण च युधिष्ठिर}


\twolineshloka
{दिष्ट्या मुक्ताः स्थ संग्रामादस्माल्लोकभयंकरात्}
{क्षत्रधर्मरतश्चासि कच्चिन्मोदसि पाण्डव}


\threelineshloka
{कच्चिच्च निहतामित्रः प्रीणासि सुहृदो नृप}
{कच्चिच्छ्रियमिमां प्राप्य न त्वां शोकः प्रबाधते ॥युधिष्ठिर उवाच}
{}


\twolineshloka
{विजितेयं मही कृत्स्ना कृष्णबाहुबलाश्रयात्}
{ब्राह्मणानां प्रसादेन भीमार्जुनबलेन च}


\twolineshloka
{इदं तु मे महद्दुःखं वर्तते हृदि नित्यदा}
{कृत्वा ज्ञातिक्षयमिमं महान्तं घोरदर्शनम्}


\twolineshloka
{सौभद्रं द्रौपदेयांश्च घातयित्वा सुतान्प्रियान्}
{जयोऽयमजयाकारो भगवन्प्रतिभाति मे}


\twolineshloka
{किंनु वक्ष्यति वार्ष्णेयी वधूर्मे मधुसूदनम्}
{द्वारकावासिनी कृष्णमितः प्रतिगतं हरिम्}


\twolineshloka
{द्रौपदी हतपुत्रेयं कृपणा हतबान्धवा}
{अस्मत्प्रियहिते युक्ता भूयः पीडयतीव माम्}


\twolineshloka
{इदमन्यच्च भगवन्यत्त्वां वक्ष्यामि नारद}
{मन्त्रसंवरणेनास्मि कुन्त्या दुःखेन योजितः}


\twolineshloka
{यः स नागायुतप्राणो लोकेऽप्रतिरथो रणे}
{सिंहविक्रान्तगामी च जितकाशी यतव्रतः}


\twolineshloka
{आश्रयो धार्तराष्ट्राणां मानी तीक्ष्णपराक्रमः}
{अमर्षी नित्यसंरम्भी क्षेप्ताऽस्माकं रणेरणे}


\twolineshloka
{शीघ्रास्त्रश्चित्रयोधी च कृती चाद्भुतविक्रमः}
{गूढोत्पन्नः सुतः कुन्त्या भ्राताऽस्माकमसौ किल}


\twolineshloka
{तोयकर्मणि तं कुन्ती कथयामास मे तदा}
{पुत्रं सर्वगुणोयेतं कर्णं त्यक्तं जले पुरा}


\twolineshloka
{मञ्जूषायां समाधाय गङ्गास्रोतस्यमज्जयत्}
{यं सूतपुत्रं लोकोऽयं राधेयं चाभ्यमन्यत}


\threelineshloka
{स सूर्यपुत्रः कुन्त्या यै भ्राताऽस्माकं च मातृतः}
{अजानता मया सड्ख्ये राज्यलुब्धेन घातितः}
{तन्मे दहति गात्राणि तूलराशिमिवानलः}


\twolineshloka
{न हि तं वेद पार्थोऽपि भ्रातरं श्वेतवाहनः}
{नाहं न भीमो न यमौ स त्वस्मान्वेद तत्वतः}


\twolineshloka
{गता किल पृथा तस्य सकाशमिति नः श्चुतम्}
{अस्माकं शमकामा वै त्वं च पुत्रो ममेत्यथ}


\twolineshloka
{पृथाया न कृतः कामस्तेन चापि महात्मना}
{अतीवानुचितं मातरवोच इति सोऽब्रवीत्}


\twolineshloka
{न हि शक्ष्यामि संत्यक्तुमहं दुर्योधनं रणे}
{अनार्यत्वं नृशंसत्वं कृतघ्नत्वं च मे भवेत्}


\twolineshloka
{युधिष्ठिरेण सन्धिं हि यदि कुर्यां मते तव}
{भीतो रणे श्वेतवाहादिति मां मंस्यते जनः}


\twolineshloka
{सोऽहं निर्जित्य समरे विजयं सहकेशवम्}
{संधास्ये धर्मपुत्रेण पश्चादिति च सोऽब्रवीत्}


\twolineshloka
{तमवोचत्किल पृथा पुनः पृथुलवक्षसम्}
{चतुर्णामभयं देहि कामं युध्यस्व फल्गुनम्}


\twolineshloka
{सोऽब्रवीन्मातरं धीमान्वेपमानां कृताञ्जलिः}
{प्राप्तान्विषह्यांश्चतुरो न हनिष्यामि ते सुतान्}


\twolineshloka
{पञ्चैव हि सुता देवि भविष्यन्ति तव ध्रुवाः}
{सार्जुना वा हते कर्णे सकर्णा वा हतेऽर्जुने}


\twolineshloka
{तं पुत्रगृद्धिनी भूयो माता पुत्रमथाब्रवीत्}
{भ्रातॄणां स्वस्ति कुर्वीथा येषां स्वस्ति चिकीर्षसि}


\twolineshloka
{एवमुक्त्वा किल पृथा विसृज्योपययौ गृहान्}
{सोऽर्जुनेन हतो वीरो भ्रात्रा भ्राता सहोदरः}


\twolineshloka
{न चैव निःसृतो मन्त्रः पृथायास्तस्य वा मुने}
{अथ शूरो महेष्वासः पार्थेनाजौ निपातितः}


\twolineshloka
{अहं त्वज्ञासिषं पश्चात्स्वसोदर्यं द्विजोत्तम}
{पूर्वजं भ्रातरं कर्णं पृथाया वचनात्प्रभो}


\twolineshloka
{तेन मे दूयते तीव्रं हृदयं भ्रातृघातिनः}
{कर्णार्जुनसहायोऽहं जयेयमपि वासवम्}


\twolineshloka
{सभायां क्लिश्यमानस्य धार्तराष्ट्रैर्दुरात्मभिः}
{सहसोत्पतितः क्रोधः कर्णं दृष्ट्वा प्रशाम्यति}


\twolineshloka
{यदा ह्यस्य गिरो रूक्षाः श्रृणोमि कटुकोदयाः}
{सभायां गदतो द्यूते दुर्योधनहितैषिणः}


\twolineshloka
{तदा नश्यति मे रोषः पादौ तस्य निरीक्ष्य ह}
{कुन्त्या हि सदृशौ पादौ कर्णस्येति मतिर्मम}


\twolineshloka
{सादृश्यहेतुमन्विच्छन्पृथायास्तस्य चैव ह}
{कारणं नाधिगच्छामि कथंचिदपि चिन्तयन्}


\twolineshloka
{कथं नु तस्य संग्रामे पृथिवी चक्रमग्रसत्}
{कथं नु शप्तो भ्राता मे तत्त्वं वक्तुमिहार्हसि}


\twolineshloka
{श्रोतुमिच्छामि भगवंस्त्वत्तः सर्वं यथातथम्}
{भवान्हि सर्वविद्विद्वाँल्लोके वेद कृताकृतम्}


\chapter{अध्यायः २}
\twolineshloka
{वैशंपायन उवाच}
{}


\threelineshloka
{स एवमुक्तस्तु तदा नारदो वदतांवरः}
{कथयामास तत्सर्वं यथा शप्तः स सूतजः ॥नारद उवाच}
{}


\twolineshloka
{एवमेतन्महाबाहो यथा वदसि भारत}
{न कर्णार्जुनयोः किंचिदविषह्यं भवेद्रणे}


\twolineshloka
{गुह्यमेतत्तु देवानां कथयिष्यामि ते नृप}
{तन्निबोध महाबाहो यथावृत्तमिदं पुरा}


\twolineshloka
{क्षत्रं स्वर्गं कथं गच्छेच्छस्त्रपूतमिति प्रभो}
{संघर्षजननस्तस्मात्कन्यागर्भो विसर्जितः}


\twolineshloka
{स बालस्तेजसा युक्तः सूतपुत्रत्वमागतः}
{चकाराङ्गिरसां श्रेष्ठे धनुर्वेदं गुरौ तव}


\twolineshloka
{स बलं भीमसेनस्य फल्गुनस्यास्त्रलाघवम्}
{बुद्धिं च तव राजेन्द्र यमयोर्विनयं तथा}


\twolineshloka
{सख्यं च वासुदेवेन बाल्ये गाण्डीवधन्वनः}
{राजानामनुरागं च चिन्तयानो व्यदह्यत}


\twolineshloka
{स सख्यमगमद्बाल्ये राज्ञा दुर्योधनेन च}
{युष्माभिर्नित्यसंघृष्टो दैवाच्चापि स्वभावतः}


\twolineshloka
{विद्याधिकमथालक्ष्य धनुर्वेदे धनञ्जयम्}
{द्रोणं रहस्युपागम्य कर्णो वचनमब्रवीत्}


\twolineshloka
{ब्रह्मास्त्रं वेत्तुमिच्छामि सरहस्यनिवर्तनम्}
{अर्जुनेन समो युद्धे भवेयमिति मे मतिः}


\twolineshloka
{समः पुत्रेषु च स्नेहः शिष्येषु च तव ध्रुवम्}
{त्वत्प्रसादान्न मा ब्रूयुरकृतास्त्रं विचक्षणाः}


\twolineshloka
{द्रोणस्तथोक्तः कर्णेन सापेक्षः फल्गुनं प्रति}
{दौरात्म्यं चैव कर्णस्य विदित्वा तमुवाच ह}


\twolineshloka
{ब्रह्मास्त्रं ब्राह्मणो विद्याद्यथावच्चरितव्रतः}
{क्षत्रियो वा तपस्वी यो नान्यो विद्यात्कथंचन}


\twolineshloka
{इत्युक्तोऽङ्गिरसां श्रेष्ठमामन्त्र्य प्रतिपूज्य च}
{जगाम सहसा राजन्महेन्द्रं पर्वतं प्रति}


\twolineshloka
{स तु राममुपागम्य शिरसाऽभिप्रणम्य च}
{ब्राह्मणो भार्गवोऽस्मीति गौरवेणाभ्यवन्दत}


\twolineshloka
{रामस्तं प्रतिजग्राह पृष्ट्वा गोत्रादि सर्वशः}
{उष्यतां स्वागतं चेति प्रीतिमांश्चाभवद्भृशम्}


\twolineshloka
{तत्र कर्णस्य वसतो महेन्द्रे स्वर्गसंमिते}
{गन्धर्वै राक्षसैर्यक्षैर्देवैश्चासीत्समागमः}


\twolineshloka
{स तत्रेष्वस्त्रमकरोद्भृगुश्रेष्ठाद्यथाविधि}
{प्रियश्चाभवदत्यर्थं देवदानवरक्षसाम्}


\twolineshloka
{स कदाचित्समुद्रान्ते विचरन्नाश्रमान्तिके}
{एकः खङ्गधनुष्पाणिः परिचक्राम सूतजः}


\twolineshloka
{सोऽग्निहोत्रप्रसक्तस्य कस्यचिद्ब्रह्मवादिनः}
{जघानाज्ञानतः पार्थ होमधेनुं यदृच्छया}


\twolineshloka
{तदज्ञानकृतं मत्वा ब्राह्मणाय न्यवेदयत्}
{कर्णः प्रसादयंश्चैनमिदमित्यब्रवीद्वचः}


\twolineshloka
{अबुद्धिपूर्वं भगवन्धेनुरेषा हता तव}
{मया तत्र प्रसादं मे कुरुष्वेति पुनः पुनः}


\twolineshloka
{तं स विप्रोऽब्रवीत्क्रुद्धो वाचा निर्भर्त्सयन्निव}
{दुराचार वधार्हस्त्वं फलं प्राप्स्यसि दुर्मते}


\twolineshloka
{येन विस्पर्धसे नित्यं यदर्थं घटसेऽनिशम्}
{युध्यतस्तेन ते पाप भूमिश्चक्रं ग्रसिष्यति}


\twolineshloka
{ततश्चक्रे महीग्रस्ते मूर्धानं ते विचेष्टतः}
{पातयिष्यति विक्रम्य शत्रुर्गच्छ नराधम}


\twolineshloka
{यथेयं गौर्हता मूढ प्रमत्तस्य त्वया मम}
{प्रमत्तस्यैव मे वाचा शिरस्ते पातयिष्यति}


\twolineshloka
{शप्तः प्रसादयामास कर्णस्तं द्विजसत्तमम्}
{गोभिर्धनैश्च रत्नैश्च स चैनं पुनरब्रवीत्}


\twolineshloka
{नेदमव्याहृतं कुर्याद्ब्रह्मलोकेऽपि केवलम्}
{गच्छ वा तिष्ठ वा यद्वा कार्यं यत्तत्समाचर}


\twolineshloka
{इत्युक्तो ब्राह्मणेनाथ कर्णो दैन्यादधोमुखः}
{राममभ्यागमद्भीतस्तदेव मनसा स्मरन्}


\chapter{अध्यायः ३}
\twolineshloka
{नारद उवाच}
{}


\twolineshloka
{कर्णस्य बाहुवीर्येण प्रश्रयेण दमेन च}
{तुतोष भृगुशार्दूलो गुरुशुश्रूषया तथा}


\twolineshloka
{ततस्तस्मै महातेजा ब्रह्मास्त्रं सनिवर्तनम्}
{प्रोवाच सुमहाप्रज्ञः स तपस्वी तपस्विने}


\twolineshloka
{विदितास्त्रस्ततः कर्णो रममाणोऽऽश्रमे भृगोः}
{चकार वै धनुर्वेदे यत्नमद्भुतविक्रमः}


\twolineshloka
{ततः कदाचिद्रामस्तु चरन्नाश्रममन्तिकात्}
{कर्णेन सहितो धीमानुपवासेन कर्शितः}


\twolineshloka
{सुष्वाप जामदग्न्यस्तु विस्रम्भोत्पन्नसौहृदः}
{तस्योत्सङ्गे समाधाय शिरः क्लान्तमना गुरुः}


\twolineshloka
{अथ क्रिमिः श्लेष्ममयो मांसशोणितभोजनः}
{दारुणो दारुणाकारः कर्णस्याभ्याशमागतः}


\twolineshloka
{स तस्योरुमथासाद्य बिभेद रुधिराशनः}
{न चैनमशकत्क्षेप्तुं वक्तुं वाऽपि गुरोर्भयात्}


\twolineshloka
{स दश्यमानोऽपि तथा कृमिणा तेन भारत}
{गुरोः प्रबोधनाकाङ्क्षी तमुपैक्षत सूर्यजः}


\twolineshloka
{कर्णस्तु वेदनां धैर्यादसह्यां विनिगृह्य ताम्}
{अकम्पयन्नव्यथयन्धारयामास भार्गवम्}


\twolineshloka
{यदा स रुधिरेणाङ्गे परिस्पृष्टोऽभवद्गुरुः}
{तदाऽबुध्यत तेजस्वी संरब्धश्चैनमब्रवीत्}


\twolineshloka
{अहोऽस्म्यशुचितां प्राप्तः किमिदं च कृतं त्वया}
{कथयस्व भयं त्यक्त्वा याथातथ्यमिदं मम}


\twolineshloka
{तस्य कर्णस्तदाचष्ट कृमिणा परिभक्षणम्}
{ददर्श रामस्तं चापि कृमिं सूकरसंस्थितम्}


\twolineshloka
{अष्टपादं तीक्ष्णदंष्ट्रं सूचीभिः परिसंवृतम्}
{रोमभिः सन्निरुद्धाङ्गमलर्कं नाम नामतः}


\twolineshloka
{स दृष्टमात्रो रामेण किमिः प्राणानवासृजत्}
{तस्मिन्नेवासृजि क्लिन्नस्तदद्भुतमिवाभवत्}


\twolineshloka
{ततोऽन्तरिक्षे ददृशे विश्वरूपः करालवान्}
{राक्षसो लोहितग्रीवः कृष्णाङ्गो मेघवाहनः}


\twolineshloka
{स रामं प्राञ्जलिर्भूत्वा बभाषे पूर्णमानसः}
{स्वस्ति ते भृगुशार्दूल गमिष्येऽहं यथागतम्}


\twolineshloka
{मोक्षितो नरकादस्माद्भवता मुनिसत्तम}
{भद्रं च तेऽस्तु सिद्धिश्च प्रियं मे भवता कृतम्}


\twolineshloka
{तमुवाच महाबाहुर्जामदग्न्यः प्रतापवान्}
{कस्त्वं कस्माच्च नरकं प्रतिपन्नो ब्रवीहि तत्}


\twolineshloka
{सोऽब्रवीदहमासं प्राग्दंशो नाम महासुरः}
{पुरा देवयुगे तात भृगोस्तु सवया इव}


\twolineshloka
{सोऽहं भृगोः सुदयितां भार्यामपहरं बलात्}
{महर्षेरभिशापेन क्रिमिभूतोऽपतं भुवि}


\twolineshloka
{अब्रवीद्धि स मां क्रुद्धस्तव पूर्वपितामहः}
{मूत्र श्लेष्माशनः पाय निरयं प्रतिपत्स्यसे}


\twolineshloka
{शापस्यान्तो भवेद्ब्रह्मन्नित्येवं तमथाब्रवम्}
{भविता भार्गवाद्रामादिंति मामब्रवीद्भृगुः}


\twolineshloka
{सोऽहमेनां गतिं प्राप्तो यथा नकुशलस्तथा}
{त्वया साधो समागम्य विमुक्तः पापयोनितः}


\twolineshloka
{एवमुक्त्वा नमस्कृत्य ययौ रामं महासुरः}
{रामः कर्णं तु सक्रोधमिदं वचनमब्रवीत्}


\twolineshloka
{अतिदुःखमिदं मूढ न जातु ब्राह्मणः सहेत्}
{क्षत्रियस्येव ते धैर्यं कामया सत्यमुच्यताम्}


\twolineshloka
{तमुवाच ततः कर्णः शापाद्भीतः प्रसादयन्}
{ब्रह्मक्षत्रान्तरे जातं सूतं मां विद्धि भार्गव}


\twolineshloka
{राधेयः कर्ण इति मां प्रवदन्ति जना भुवि}
{प्रसादं कुरु मे ब्रह्मन्नस्त्रलुब्धस्य भार्गव}


\twolineshloka
{पिता गुरुर्न संदेहो वेदविद्याप्रदः प्रभुः}
{अतो भार्गव इत्युक्तं मया गोत्रं तवान्तिके}


\twolineshloka
{तमुवाच भृगुश्रेष्ठः सरोपः प्रदहन्निव}
{भूमौ निपतितं दीनं वेपमानं कृताञ्जलिम्}


\twolineshloka
{यस्मान्मिथ्याविचीर्णोऽहमस्त्रलोभादिह त्वया}
{तस्मादेतन्न ते मूढ ब्रह्मास्त्रं प्रतिभास्यति}


\twolineshloka
{अन्यत्र वधकालात्ते सदृशे न समीयुपः}
{अब्राह्मणे न हि ब्रह्म चिरं तिष्ठेत्कदाचन}


\twolineshloka
{गच्छेदानीं न ते स्थानमनृतस्येह विद्यते}
{न त्वया सदृशो युद्धे भविता क्षत्रियो भुवि}


\twolineshloka
{एवमुक्तः स रामेण न्यायेनोपजगामह}
{दुर्योधनमुपागम्य कृतास्त्रोऽस्मीति चाब्रवीत्}


\chapter{अध्यायः ४}
\twolineshloka
{नारद उवाच}
{}


\twolineshloka
{कर्णस्तु समवाप्यैवमस्त्रं भार्गवनन्दनात्}
{दुर्योधनेन सहितो मुमुदे भरतर्षभ}


\twolineshloka
{ततः कदाचिदाजातः समाजग्मुः स्वयंवरे}
{कलिङ्गविषये राजन्राज्ञश्चित्राङ्गदस्य च}


\twolineshloka
{श्रीमद्राजपुरं नाम नगरं तत्र भारत}
{राजानः शतशस्तत्र कन्यार्थे समुपागमन्}


\twolineshloka
{श्रुत्वा दुर्योधनस्तत्र समेतान्सर्वपार्थिवान्}
{रथेन काञ्चनाङ्गेन कर्णेन सहितो ययौ}


\twolineshloka
{ततः स्वयंवरे तस्मिन्नानादेश्या महारथाः}
{समाजग्मुर्नृपतयः कन्यार्थे नृपसत्तम}


\twolineshloka
{शिशुपालो जरासन्धो भीष्मको वक्र एव च}
{कपोतरोमा नीलश्च रुक्मी च दृढविक्रमः}


\twolineshloka
{सृगालश्च महाराजः स्त्रीराज्याधिपतिश्च यः}
{विशोकः शतधन्वा च भोजो वीरश्च नामतः}


\twolineshloka
{एते चान्ये च बहवो दक्षिणां दिशमाश्रिताः}
{म्लेच्छाश्चार्याश्च राजानः प्राच्योदीच्यास्तथैव च}


\twolineshloka
{काञ्चनाङ्गदिनः सर्वे शुद्धजाम्बूनदप्रभाः}
{सर्वे भास्वरदेहाश्च व्याघ्रा इव बलोत्कटाः}


\twolineshloka
{ततः समुपविष्टेषु तेषु राजसु भारत}
{विवेश रङ्गं सा कन्या धात्रीवर्षवरान्विता}


\twolineshloka
{ततः संश्राव्यमाणेषु राज्ञां नामसु भारत}
{अत्यक्रामद्धार्तराष्ट्रं सा कन्या वरवर्णिनी}


\twolineshloka
{दुर्योधनस्तु कौरव्यो नामर्षयत लङ्घनम्}
{प्रत्यपेधच्च तां कन्यामसत्कृत्य नराधिपान्}


\twolineshloka
{स वीर्यमदमत्तत्वाद्भीष्मद्रोणावुपाश्रितः}
{रथमारोप्य तां कन्यामाजुहाव नराधिपान्}


\twolineshloka
{तमन्वगाद्रथी खङ्गी बद्धगोधाङ्गुलित्रवान्}
{कर्णः शस्त्रभृतां श्रेष्ठः पृष्ठतः पुरुषर्षभ}


\threelineshloka
{ततो विमर्दः सुमहान्राज्ञामासीद्युयुत्सताम्}
{सन्नह्यतां तनुत्राणि रथान्योजयतामपि}
{}


\twolineshloka
{तेऽभ्यधावन्त संक्रुद्धाः कर्णदुर्योधनावुभौ}
{शरवर्षाणि मुञ्चन्तो मेघाः पर्वतयोरिव}


\twolineshloka
{कर्णस्तेषामापततामेकैकेन शरेण ह}
{धनूंषि च शरव्रातान्पातयामास भूतले}


\twolineshloka
{ततो विधनुषः कांश्चित्कांश्चिदुद्यतकार्मुकान्}
{कांश्चिदुत्सृजतो बाणान्रथशक्तिगदास्तथा}


\twolineshloka
{लाघवाव्द्याकुलीकृत्य कर्णः प्रहरतां वरः}
{हतसूतांश्च भूयिष्ठान्स विजिग्ये नराधिपान्}


\twolineshloka
{ते स्वयं वाहयन्तोऽश्वान्याहि याहीति वादिनः}
{व्यपेयुस्ते रणं हित्वा राजानो भग्नमानसाः}


\twolineshloka
{दुर्योधनस्तु कर्णेन पाल्यमानोऽभ्ययात्तदा}
{हृष्टः कन्यामुपादाय नगरं नागसाह्वयम्}


\chapter{अध्यायः ५}
\twolineshloka
{नारद उवाच}
{}


\twolineshloka
{आदित्कृतबलं कर्णं दृष्ट्वा राजा स मागधः}
{आह्वयद्द्वैरथेनाजौ जरासन्धो महीपतिः}


\twolineshloka
{तयोः समभवद्युद्धं दिव्यास्त्रविदुषोर्द्वयोः}
{युधि नानाप्रहरणैरन्योन्यमभिवर्षतोः}


\twolineshloka
{क्षीणबाणौ विधनुषौ भग्नखङ्गौ महीं गतौ}
{बाहुभिः समसज्जेतामुभावतिबलान्वितौ}


\twolineshloka
{[बाहुकण्टकयुद्धेन तस्य कर्णोऽथ युध्यतः}
{]विभेदं संधिं देहस्य जरया श्लेषितस्य हि}


\twolineshloka
{स विकारं शरीरस्य दृष्ट्वा नृपतिरात्मनः}
{प्रीतोऽस्मीत्यब्रवीत्कर्णं वैरमुत्सृज्य दूरतः}


\twolineshloka
{प्रीत्या ददौ स कर्णाय मालिनीं नगरीमनु}
{अङ्गेषु नरशार्दूल स राजाऽऽसीत्सपत्नजित्}


\twolineshloka
{पालयामास वर्णांस्तु कर्णः परबलार्दनः}
{दुर्योधनस्यानुमते तवापि विदितं तथा}


\twolineshloka
{एवं शस्त्रप्रतापेन प्रथितः सोऽभवत्क्षितौ}
{त्वद्धितार्थं सुरेन्द्रेण भिक्षितो वर्मकुण्डले}


\twolineshloka
{स दिव्ये सहजे प्रादात्कुण्डले परमार्चिते}
{सहजं कवचं चापि मोहितो देवमायया}


\twolineshloka
{विमुक्तः कुण्डलाभ्यां च सहजेन च वर्मणा}
{निहतो विजयेनाजौ वासुदेवस्य पश्यतः}


\twolineshloka
{ब्राह्मणस्यापि शापेन रामस्य च महात्मनः}
{कुन्त्याश्च वरदानेन मायया च शतक्रतोः}


\threelineshloka
{भीष्मावमानात्सङ्ख्यायां रथानामर्धकीर्तनात्}
{शल्यतेजोवधाच्चापि वासुदेवनयेन च}
{}


\twolineshloka
{रुद्रस्य देवराजस्य यमस्य वरुणस्य च}
{कुबेरद्रोणयोश्चैव कृपस्य च महात्मनः}


\twolineshloka
{अस्त्राणि दिव्यान्यादाय युधि गाण्डीवधन्वना}
{हतो वैकर्तनः कर्णो दिवाकरसमद्युतिः}


\twolineshloka
{एवं शप्तस्तव भ्राता बहुभिश्चापि वञ्चितः}
{न शोच्यः पुरुषव्याघ्र युद्धे हि निधनं गतः}


\chapter{अध्यायः ६}
\twolineshloka
{वैशंपायन उवाच}
{}


\twolineshloka
{एतावदुक्त्वा देवर्षिर्विरराम स नारदः}
{युधिष्ठिरस्तु राजर्षिर्दध्यौ शोकपरिप्लुतः}


\twolineshloka
{तं दीनमनसं वीरमधोवदनमातुरम्}
{निःश्वसन्तं यथा नागं पर्यश्रुनयनं तथा}


\twolineshloka
{कुन्ती शोकपरीताङ्गी दुःखोपहतचेतना}
{अब्रवीन्मधुराभाषा काले वचनमर्थवत्}


\twolineshloka
{युधिष्ठिर महाबाहो नैनं शोचितुमर्हसि}
{जहि शोकं महाप्राज्ञ श्रृणु चेदं वचो मम}


\twolineshloka
{याचितः स मया पूर्वं भ्राता ज्ञापयितुं तव}
{भास्करेण च देवेन पित्रा धर्मभृतां वरः}


\twolineshloka
{यद्वाच्यं हितकामेन सुहृदां भूतिमिच्छता}
{तथा दिवाकरेणोक्तः स्वप्नान्ते मम चाग्रतः}


\twolineshloka
{न चैनमशकद्भानुरहं वा स्नेहकारणैः}
{पुरा प्रत्यनुनेतुं वा नेतुं वाऽप्येकतां त्वया}


\twolineshloka
{ततः कालपरीतः स वैरस्योद्धरणे रतः}
{प्रतीपकारी युष्माकमिति चोपेक्षितो मया}


\twolineshloka
{इत्युक्तो धर्मराजस्तु मात्रा बाष्पाकुलेक्षणः}
{उवाच वाक्यं धर्मात्मा शोकव्याकुललोचनः}


% Check verse!
भवत्या गूढमन्त्रत्वाद्वञ्चिताः स्म तदा भृशम्
\twolineshloka
{शशाप च महातेजाः सर्वलोकेषु योषितः}
{न गुह्यं धारयिष्यन्तीत्येवं दुःखसमन्वितः}


\twolineshloka
{स राजा पुत्रपौत्राणां संबन्धिसुहृदां तदा}
{स्मरन्नुद्विग्नहृदयो बभूवोद्विग्नचेतनः}


\twolineshloka
{ततः शोकपरीतात्मा सधूम इव पावकः}
{निर्वेदमगमद्धीमान्राज्ये संतापपीडितः}


\chapter{अध्यायः ७}
\twolineshloka
{वैशंपायन उवाच}
{}


\twolineshloka
{युधिष्ठिरस्तु धर्मात्मा शोकव्याकुलचेतनः}
{शुशोच दुःखसंतप्तः स्मृत्वा कर्णं महारथम्}


\threelineshloka
{आविष्टो दुःखशोकाभ्यां निःश्वसंश्च पुनः पुनः}
{दृष्ट्वार्जुनमुवाचेदं वचनं शोककर्शितः ॥युधिष्ठिर उवाच}
{}


\twolineshloka
{यद्भैक्ष्यमाचरिष्याम वृष्ण्यन्धकपुरे वयम्}
{ज्ञातीन्निष्पुरुषान्कृत्वा नेमां प्राप्स्याम दुर्गतिम्}


\twolineshloka
{अमित्रा नः समृद्धार्था वृत्तार्थाः कुरवः किल}
{आत्मानमात्मना हत्वा किं धर्मफलमाप्नुमः}


\twolineshloka
{धिगस्तु क्षात्रमाचारं धिगस्तु बलमौरसम्}
{धिगस्तु चार्थं येनेमामापदं गमिता वयम्}


\twolineshloka
{साधु क्षमा दमः शौचमविरोधो विमत्सरः}
{अहिंसा सत्यवचनं नित्यानि वनचारिणाम्}


\twolineshloka
{वयं तु लोभान्मोहाच्च दम्भं मानं च संश्रिताः}
{इमामवस्थां संप्राप्ता राज्यक्लेशबुभुक्षया}


\twolineshloka
{त्रैलोक्यस्यापि राज्येन नास्मान्कश्चित्प्रहर्षयेत्}
{बान्धवान्निहतान्दृष्ट्वा पृथिव्यामामिषैषिणः}


\twolineshloka
{ते वयं पृथिवीहेतोरवध्यान्पृथिवीतले}
{संपरित्यज्य जीवामो हीनार्था हतबान्धवाः}


\twolineshloka
{आमिषे गृध्यमानानामशुभं वै शुनामिव}
{आमिषं चैव नो नष्टमामिषस्य च भोजिनाम्}


\twolineshloka
{न पृथिव्या सकलया न सुवर्णस्य राशिभिः}
{न गजाश्वेन सर्वेण ते त्याज्या य इमे हताः}


\twolineshloka
{काममन्युपरीतास्ते क्रोधामर्षसमन्विताः}
{मृत्युयानं समारुह्य गता वैवस्वतक्षयम्}


\twolineshloka
{बहुकल्याणमिच्छन्त ईहन्ते पितरः सुतान्}
{तपसा ब्रह्मचर्येण वन्दनेन तितिक्षया}


\twolineshloka
{उपवासैस्तथेज्याभिर्व्रतकौतुकमङ्गलैः}
{लभन्ते मातरो गर्भांस्तान्मासान्दश बिभ्रति}


\threelineshloka
{यदि स्वस्ति प्रजायन्ते जाता जीवन्ति वा यदि}
{संभाघिता जातबला विदध्युर्यदि नः सुखम्}
{इह चामुत्र चैवेति कृपणाः फलहेतवः}


\twolineshloka
{तासामयं समुद्योगो निर्वृत्तः केवलोऽफलः}
{यदासां निहताः पुत्रा युवानो मृष्टकुण्डलाः}


\twolineshloka
{अभुक्त्वा पार्थिवान्भोगानृणान्यनपहाय च}
{पितृभ्यो देवताभ्यश्च गता वैयस्वतक्षयम्}


\twolineshloka
{यदैषामम्ब पितरौ जातकर्मकराविह}
{संजातबालरूपेषु तदैव निहता नृषाः}


\twolineshloka
{संयुक्ताः काममन्युभ्यां क्रोधामर्षसमन्विताः}
{न ते जयफलं किंचिद्भोक्तारो जातु कर्हिचित्}


\twolineshloka
{पाञ्चालानां कुरूणां च हता एव हि ये हताः}
{न सकामा वयं ते च न चास्माभिर्न तैर्जितम्}


\threelineshloka
{न तैर्भुक्तेयमवनिर्न नार्यो गीतवादितम्}
{नामात्यसुहृदां वाक्यं न च श्रुतवतां श्रुतम्}
{न रत्नानि परार्ध्यानि न भूर्न द्रविणागमः}


\threelineshloka
{न च धर्म्यानिमाँल्लोकान्प्रपद्याम स्वकर्मभिः}
{वयमेवास्य लोकस्य विनाशे कारणं स्मृताः}
{धृतराष्ट्रस्य पुत्रेण निकृतिप्रीतिसंयुताः}


\twolineshloka
{सदैव निकृतिप्रज्ञो द्वेष्टा विद्वेषजीवनः}
{मिथ्यावृत्तश्च सततमस्मास्वनपराधिषु}


\twolineshloka
{ऋद्धिमस्मासु तां दृष्ट्वा विवर्णो हरिणः कृशः}
{धृतराष्ट्रश्च नृपतिः सौबलेन निवेदितः}


\twolineshloka
{तं पिता पुत्रगृध्नुत्वादनुमेनेऽनये स्थितम्}
{अनपेक्ष्यैव पितरं गाङ्गेयं विदुरं तथा}


\twolineshloka
{असंशयं त्वयं राजा यथैवाहं तथा गतः}
{अनियम्याशुचिं लुब्धं पुत्रं कामवशानुगम्}


\threelineshloka
{यशसः पतितो दीप्ताद्धातयित्वा सहोदरान्}
{इमौ हि वृद्धौ शोकाग्नौ प्रक्षिप्य स सुयोधनः}
{अस्मत्प्रद्वेषसंतप्तः पापबुद्धिः सदैव ह}


\twolineshloka
{को हि बन्धुः कुलीनः संस्तथा ब्रूयात्सुहृज्जने}
{यथाऽसाववदद्वाक्यं युयुत्सुः कृष्णसन्निधौ}


\twolineshloka
{आत्मनो हि वयं दोषाद्विनष्टाः शाश्वतीः समाः}
{प्रदहन्तो दिशः सर्वा भास्वरा इव तेजसा}


\twolineshloka
{सोऽस्माकं वैरपुरुषो दुर्मतिः प्रग्रहं गतः}
{दुर्योधनकृते ह्येतत्कुलं नो विनिपातितम्}


% Check verse!
अवध्यानां वधं कृत्वा लोके प्राप्ताः स्म वाच्यतां
\twolineshloka
{कुलस्यास्यान्तकरणं दुर्मतिं पापपूरुषम्}
{राजा राष्ट्रेश्वरं कृत्वा धृतराष्ट्रोऽद्य शोचति}


\twolineshloka
{हताः शूराः कृतं पापं विषयोऽसौ विनाशितः}
{हत्वा नो विगतो मन्युः शोको मां दारयत्ययम्}


\threelineshloka
{धनञ्जय कृतं पापं कल्याणेनोपहन्यते}
{[ख्यापनेनानुतापेन दानेन तपसाऽपि वा}
{निवृत्त्या तीर्थगमनाच्छुतिस्मृतिजपेन वा ॥]}


\threelineshloka
{त्यागवांश्च पुनः पापं नालं कर्तुमिति श्रुतिः}
{त्यागवाञ्जन्ममरणे नाप्नोतीति श्रुतिर्यतः}
{प्राप्तवर्त्मा कृतमतिर्ब्रह्म संपद्यते तदा}


\twolineshloka
{स धनञ्जय निर्द्वन्द्वो मुनिर्ज्ञानसमन्वितः}
{वनमामन्त्र्य वः सर्वान्गमिष्यामि परंतप}


\twolineshloka
{न हि कृत्स्नतमो धर्मः शक्यः प्राप्तुमिति श्रुतिः}
{परिग्रहवता तन्मे प्रत्यक्षमरिसूदन}


\twolineshloka
{मया निसृष्टं पापं हि परिग्रहमभीप्सता}
{जन्मक्षयनिमित्तं च प्राप्तुं शक्यमिति श्रुतिः}


\fourlineindentedshloka
{स परिग्रहमुत्सृज्य कृत्स्नं राज्यं सुखानि च}
{गमिष्यामि विनिर्मुक्तो विशोको निर्ममः क्वचित्}
{प्रशासध्वमिमामुर्वी क्षेमां निहतकण्टकाम्}
{न ममार्थोऽस्ति राज्येन भोगैर्वा कुरुनन्दन}


\twolineshloka
{एतावदुक्त्वा वचनं कुरुराजो युधिष्ठिरः}
{उपारमत्ततः पार्थः कनीयान्प्रत्यभाषत}


\chapter{अध्यायः ८}
\twolineshloka
{वैशंपायन उवाच}
{}


\twolineshloka
{अथार्जुन उवाचेदमधिक्षिप्त इवाक्षमी}
{अभिनीततरं वाक्यं दृढवादपराक्रमः}


\threelineshloka
{दर्शयन्नैन्द्रिरात्मानमुग्रमुग्रपराक्रमः}
{स्मयमानो महातेजाः सृक्किणी परिसंलिहन् ॥अर्जुन उवाच}
{}


\twolineshloka
{अहो दुःखमहो कृच्छ्रमहो वैक्लव्यमुत्तमम्}
{यत्कृत्वाऽमानुषं कर्म त्यजेथाः श्रियमुत्तमाम्}


\twolineshloka
{शत्रून्हत्वा महीं लब्ध्वा स्वधर्मेणोपपादिताम्}
{हतामित्रः कथं स त्वं त्यजेथा बुद्धिलाघवात्}


\twolineshloka
{क्लीबस्य हि कुतो राज्यं दीर्घसूत्रस्य वा पुनः}
{किमर्थं च महीपालानवधीः क्रोधमूर्च्छितः}


\threelineshloka
{यो ह्याजिजीविषेद्भैक्षं कर्मणा नैव कस्य चित्}
{समारम्भान्बुभूषेत हतस्वस्तिरकिञ्चनः}
{सर्वलोकेषु विख्यातो न पुत्रपशुसंहितः}


\twolineshloka
{कापालीं नृप पापिष्ठां वृत्तिमासाद्य जीवतः}
{संत्यज्य राज्यमृद्धं ते लोकोऽयं किं वदिष्यति}


\twolineshloka
{सर्वारम्भान्समुत्सृज्य हतस्वस्तिरकिञ्चनः}
{कस्मादाशंससे भैक्षं चर्तुं प्राकृतवत्प्रभो}


\twolineshloka
{कस्माद्राजकुले जातो जित्वा कृत्स्नां वसुंधराम्}
{धर्मार्थावखिलौ हित्वा वनं मौढ्यात्प्रतिष्ठसे}


\twolineshloka
{यदीमानि हवींषीह विमथिष्यन्त्यसाधवः}
{भवता विप्रहीणत्वात्प्राप्तं त्वामेव किल्विषम्}


\twolineshloka
{आकिञ्चन्यं मुनीनां च इति वै नहुषोऽब्रवीत्}
{कृत्वा नृशंसं ह्यधने धिगस्त्वधनतामिह}


\twolineshloka
{आश्वस्तन्यमृषीणां हि विद्यते वेद तद्भवान्}
{यं त्विमं धर्ममित्याहुर्धनादेष प्रवर्तते}


\twolineshloka
{धर्मं स हरते तस्य धनं हरति यस्य यः}
{ह्रियमाणे धने राजन्वयं कस्य क्षमेमहि}


\twolineshloka
{अभिशस्तं प्रपश्यन्ति दरिद्रं पार्श्वतः स्थितम्}
{दारिद्र्यं पातकं लोके कस्तच्छंसितुमर्हति}


\twolineshloka
{पतितः शोच्यते राजन्निर्धनश्चापि शोच्यते}
{विशेषं नाधिगच्छामि पतितस्याधनस्य च}


\twolineshloka
{अर्थेभ्यो हि विवृद्धेभ्यः संभृतेभ्यस्ततस्ततः}
{क्रियाः सर्वाः प्रवर्तन्ते पर्वतेभ्य इवापगाः}


\twolineshloka
{अर्थाद्धर्मश्च कामश्च स्वर्गश्चैव नराधिप}
{प्राणयात्राऽपि लोकस्य विना ह्यर्थं न सिध्द्यति}


\twolineshloka
{अर्थेन हि विहीनस्य पुरुषस्याल्पमेधसः}
{विच्छिद्यन्ते क्रियाः सर्वा ग्रीष्मे कुसरितो यथा}


\twolineshloka
{यस्यार्थास्तस्य मित्राणि यस्यार्थास्तस्य बान्धवाः}
{यस्यार्थाः स पुमाँल्लोके यस्यार्थाः स च पण्डितः}


\twolineshloka
{अधनेनार्थकामेन नार्थः शक्यो विधित्सितुम्}
{अर्थैरर्था निबध्यन्ते गजैरिव महागजाः}


\twolineshloka
{धर्मः कामश्च स्वर्गश्च हर्षः क्रोधः श्रुतं दमः}
{अर्थादेतानि सर्वाणि प्रवर्तन्ते नराधिप}


\twolineshloka
{धनात्कुलं प्रभवति धनाद्धर्मः प्रवर्धते}
{नाधनस्यास्त्ययं लोको न परः पुरुषोत्तम}


\twolineshloka
{नाधनो धर्मकृत्यानि यथावदनुतिष्ठति}
{धनाद्धि धर्मः स्रवति शैलादभिनदी यथा}


\twolineshloka
{यः कृशार्थः कृशगवः कृशभृत्यः कृशांतिथिः}
{स वै राजन्कृशो नाम न शरीरकृशः कृशः}


\twolineshloka
{अवेक्षस्व यथान्यायं पश्य देवासुरं यथा}
{राजन्किमन्यज्ज्ञातीनां वधाद्गृध्यन्ति देवताः}


\twolineshloka
{न चेद्धर्तव्यमन्यस्य कथं तद्धर्ममारभेत्}
{एतावानेव वेदेषु निश्चयः कविभिः कृतः}


\twolineshloka
{अध्येतव्या त्रयी नित्यं भवितव्यं विपश्चिता}
{सर्वथा धनमाहार्यं यष्टव्यं चापि यत्नतः}


\twolineshloka
{द्रोहाद्देवैरवाप्तानि दिवि स्थानानि सर्वशः}
{द्रोहात्किमन्यज्ज्ञातीनां गृध्यन्ते येन देवताः}


\twolineshloka
{इति देवा व्यवसिता वेदवादाश्च शाश्वताः}
{अधीयन्तेऽध्यापयन्ते यजन्ते याजयन्ति च}


\twolineshloka
{कृत्स्नं तदेव तच्छ्रेयो यदप्याददतेऽन्यतः}
{न पश्यामोऽनपकृतं धनं किंचित्क्वचिद्वयम्}


\twolineshloka
{एवमेव हि राजानो यजन्ति पृथिवीमिमाम्}
{जित्वा ममेयं ब्रुवते पुत्रा इव पितृर्धनम्}


% Check verse!
राजर्षयोऽपि ते स्वर्ग्या धर्मो ह्येषां निरुच्यते
\twolineshloka
{यथैव पूर्णादुदधेः स्यन्दन्त्यापो दिशो दश}
{एवं राजकुलाद्वित्तं पृथिवीं प्रतितिष्ठति}


\twolineshloka
{आसीदियं दिलीपस्य नृगस्य नहुषस्य च}
{अंबरीपस्य मान्धातुः पृथिवी सा त्वयि स्थिता}


\twolineshloka
{स त्वां द्रव्यमयो यज्ञः संप्राप्तः सर्वदक्षिणः}
{तं चेन्न यजसे राजन्प्राप्तस्त्वं राज्यकिल्बिषम्}


\twolineshloka
{येषां राजाऽश्वमेधेन यजते दक्षिणावता}
{उपेत्य तस्यावभृथे पूताः सर्वे भवन्ति ते}


\twolineshloka
{विश्वरूपो महादेवः सर्वमेधे महामखे}
{जुहाव सर्वभूतानि तथैवात्मानमात्मना}


\twolineshloka
{शाश्वतोऽयं भूतिपथो नास्त्यन्तमनुशुश्रुम}
{महाजनपथं गन्ता मा राजन्कुपथं गमः}


\chapter{अध्यायः ९}
\twolineshloka
{युधिष्ठिर उवाच}
{}


\twolineshloka
{मुहूर्तं तावदेकाग्रो मनः श्रोत्रेऽन्तरात्मनि}
{धारयन्नपि ते श्रुत्वा रोचते वचनं मम}


\twolineshloka
{सार्थगम्यमहं मार्गं न जातु त्वत्कृते पुनः}
{गच्छेयं तं गमिष्यामि हित्वा ग्राम्यसुखान्युत}


\twolineshloka
{क्षेम्यश्चैकाकिना गम्यः पन्था कोस्तीति पृच्छ माम्}
{अथवा नेच्छसि प्रष्टुमपृच्छन्नपि मे शृणु}


\twolineshloka
{हित्वा ग्राम्यसुखाचारं तप्यमानो महत्तपः}
{अरण्ये फलमूलाशी चरिष्यामि मृगैः सह}


\twolineshloka
{जुह्वानोऽग्निं यथाकालमुभौ कालावुपस्पृशन्}
{कृशः परिमिताहारश्चर्मचीरजटाधरः}


\twolineshloka
{शीतवातातपसहः क्षुत्पिपासाश्रमक्षमः}
{तपसा विधिदृष्टेन शरीरमुपशोषयन्}


\threelineshloka
{मनःकर्णसुखा नित्यं श्रृण्वन्नुच्चावचा गिरः}
{मुदितानामरण्येषु नदतां मृगपक्षिणाम्}
{}


% Check verse!
आजिघ्रन्पेशलान्गन्धान्फुल्लानां वृक्षवीरुधाम् ॥नानारूपान्वने पश्यन्रमणीयान्वनौकसः
\twolineshloka
{वानप्रस्थजनस्यापि दर्शनं कूलवासिनः}
{नाप्रियाण्याचरिष्यामि किंपुनर्ग्रामवासिनाम्}


\twolineshloka
{एकान्तशीली विमृशन्पक्वापक्वेन वर्तयन्}
{पितॄन्देवांश्च वन्येन वाग्भिरद्भिश्च तर्पयन्}


\twolineshloka
{एवमारण्यशास्त्राणामुग्रमुग्रतरं विधिम्}
{सेवमानः प्रतीक्षिष्ये देहस्यास्य समापनम्}


\twolineshloka
{अथवैकोऽहमेकाहमेकैकस्मिन्वनस्पतौ}
{चरन्भैक्षं मुनिर्मुण्डः क्षपयिष्ये कलेवरम्}


\twolineshloka
{पांसुभिः समभिच्छन्नः शून्यागारप्रतिश्रयः}
{वृक्षमूलनिकेतो वा त्यक्तसर्वप्रियाप्रियः}


\twolineshloka
{न शोचन्न प्रहृष्यंश्च तुल्यनिन्दात्मसंस्तुतिः}
{निराशीर्निर्ममो भूत्वा निर्द्वन्द्वो निष्परिग्रहः}


\twolineshloka
{आत्मारामः प्रसन्नात्मा जडान्धबधिराकृतिः}
{अकुर्वाणः परैः कांचित्संविदं जातु कैरपि}


\twolineshloka
{जङ्गमाजङ्गमान्सर्वानविहिंसंश्चतुर्विधान्}
{प्रजाः सर्वाः स्वधर्मस्थाः समः प्राणभृतः प्रति}


\twolineshloka
{न चाप्यवहसन्कंचिन्न कुर्वन्भ्रुकुटीः क्वचित्}
{प्रसन्नवदनो नित्यं सर्वेन्द्रियसुसंयतः}


\twolineshloka
{अपृच्छन्कस्यचिन्मार्गं प्रव्रजन्नेव केनचित्}
{न देशं न दिशं कांचिद्गन्तुमिच्छन्विशेषतः}


\twolineshloka
{गमने निरपेक्षश्च पश्चादनवलोकयन्}
{ऋजुः प्रणिहितो गच्छन्स्त्रीसंस्थापरिवर्जकः}


\twolineshloka
{स्वभावस्तु प्रयात्यग्रे प्रभवन्त्यशनान्यपि}
{द्वन्द्वानि च विरुद्धानि तानि सर्वाण्यचिन्तयन्}


\twolineshloka
{अल्पं वा स्वादु वा भोज्यं पूर्वालाभेन जातुचित्}
{अन्येष्वपि चरँल्लाभमलाभे सप्त पूरयन्}


\twolineshloka
{विधूमे न्यस्तमुसले व्यङ्गारे भुक्तवज्जने}
{अतीतपात्रसंचारे काले विगतभिक्षुके}


\twolineshloka
{एककालं चरन्भैक्षं त्रीनथ द्वे च पञ्च वा}
{स्नेहपाशं विमुच्याहं चरिष्यामि महीमिमाम्}


\twolineshloka
{अलाभे सति वा लाभे समदर्शी महातपाः}
{न जिजीविषुवत्किंचिन्न मुमूर्षुवदाचरन्}


\threelineshloka
{जीवितं मरणं चैव नाभिनन्दन्न च द्विपन्}
{वास्यैकं तक्षतो बाहुं चन्दनेनैकपुक्षतः}
{नाकल्याणं न कल्याणं चिन्तयन्नुभयोस्तयोः}


\twolineshloka
{याः काश्चिज्जीवता शक्याः कर्तुमभ्युदयक्रियाः}
{सर्वास्ताः समभित्यज्य निमेषादिव्यवस्थितः}


\twolineshloka
{तेषु नित्यमसक्तश्च त्यक्तसर्वेन्द्रियक्रियः}
{अपरित्यक्तसंकल्पः सुनिर्णिक्तात्मकल्मपः}


\twolineshloka
{विमुक्तः सर्वसङ्गेभ्यो व्यतीतः सर्ववागुराः}
{न वशे कस्यचित्तिष्ठन्सधर्मा मातरिश्वनः}


\twolineshloka
{वीतरागश्चरन्नेवं तुष्टिं प्राप्स्यामि मानसीम्}
{तृष्णया हि महत्पापमज्ञानादस्मि कारितः}


\twolineshloka
{कुशलाकुशलान्येके कृत्वा कर्माणि मानवाः}
{कार्यकारणसंश्लिष्टं स्वजनं नाम ब्रिभ्रति}


\twolineshloka
{आयुपोऽन्ते प्रहायेदं क्षीणप्राणं कलेवरम्}
{प्रतिगृह्णाति तत्पापं कर्तुः कर्मफलं हि तत्}


\twolineshloka
{एवं संसारचक्रेऽस्मिन्व्याविद्धे रथचक्रवत्}
{समेति भूतग्रामोऽयं भूतग्रामेण कार्यवान्}


\twolineshloka
{जन्ममृत्युजराव्याधिवेदनाभिरभिद्रुतम्}
{अपारमिव चास्वस्थं संसारं त्यजतः सुखम्}


\twolineshloka
{दिवः पतत्सु देवेषु स्थानेभ्यश्च महर्षिषु}
{को हि नाम भवेनार्थी भवेत्कारणतत्त्ववित्}


\twolineshloka
{कृत्वा हि विविधं कर्म तत्तद्विविधलक्षणम्}
{पार्थिवैर्नृपतिः स्वल्पैः कारणैरेव बध्यते}


\twolineshloka
{तस्मात्प्रज्ञामुतमिदं चिरान्मां प्रत्युपस्थितम्}
{तत्प्राप्य प्रार्थये स्थानमव्ययं शाश्वतं ध्रुवम्}


\threelineshloka
{एतया संततं धृत्या चरन्नेवंप्रकारया}
{जन्ममृत्युजराव्याधिवेदनाभिरभिद्रुतम्}
{देहं संस्थापयिष्यामि निर्भयं मार्गमास्थितः}


\chapter{अध्यायः १०}
\twolineshloka
{भीम उवाच}
{}


\twolineshloka
{श्रोत्रियस्येव ते राजन्मन्दकस्याविपश्चितः}
{अनुवाकहता बुद्धिर्नैषा तत्त्वार्थदर्शिनी}


\twolineshloka
{आलस्ये कृतचित्तस्य राजधर्मानसूयतः}
{विनाशे धार्तराष्ट्राणां किं फलं भरतर्षभ}


\twolineshloka
{क्षमाऽनुकम्पा कारुण्यमानृशंस्यं न विद्यते}
{क्षात्रमाचरतो मार्गमपि बन्धोस्त्वदन्तरे}


\twolineshloka
{यदीमां भवतो बुद्धिं विद्याम वयमीदृशीम्}
{शस्त्रं नैव ग्रहीष्यामो न वधिष्याम कंचन}


\twolineshloka
{भैक्षमेवाचरिष्याम शरीरस्याविमोक्षणात्}
{न चेदं दारुणं युद्धमभविष्यन्महीक्षिताम्}


\twolineshloka
{प्राणस्यान्नमिदं सर्वमिति वै कवयो विदुः}
{स्थावरं जङ्गमं चैव सर्वं प्राणस्य भोजनम्}


\twolineshloka
{आददानस्य चेद्राज्यं ये केचित्परिपन्थिनः}
{हन्तव्यास्त इति प्राज्ञाः क्षत्रधर्मविदो विदुः}


\twolineshloka
{ते सदोषा हताऽस्माभिरन्नस्य परिपन्थिनः}
{तान्हत्वा भुङ्क्ष्व धर्मेण युधिष्ठिर महीमिमाम्}


\twolineshloka
{यथा हि पुरुषः खात्वा कूपमप्राप्य चोदकम्}
{पङ्कदिग्धो निवर्तेत् कर्मेदं नस्तथोपमम्}


\twolineshloka
{यथाऽऽरुह्य महावृक्षमपहृत्य ततो मधु}
{अप्राश्य निधनं गच्छेत्कर्मेदं नस्तथोपमम्}


\twolineshloka
{यथा महान्तमध्वानमाशया पुरुषः पतन्}
{स निराशो निवर्तेत कर्मैतन्नस्तथोपमम्}


\twolineshloka
{यथा शत्रून्घातयित्वा पुरुषः कुरुनन्दन}
{आत्मानं घातयेत्पश्चात्कर्मेदं नस्तथोपमम्}


\twolineshloka
{यथान्नं क्षुधितो लब्ध्वा न भुञ्जीयाद्यदृच्छया}
{कामी च कामिनीं लब्ध्वा कर्मेदं नस्तथोपमम्}


\twolineshloka
{वयमेवात्र गर्ह्या हि यद्वयं मन्दचेतसम्}
{त्वां राजन्ननुगच्छामो ज्येष्ठोऽयमिति भारत}


\twolineshloka
{वयं हि बाहुबलिनः कृतविद्या मनस्विनः}
{क्लीबस्य वाक्ये तिष्ठामो यथैवाशक्तयस्तथा}


\twolineshloka
{अगतीकगतीनस्मान्नष्टार्थनर्थसिद्धये}
{कथं वै नानुपश्येयुर्जनाः पश्यत यादृशम्}


\twolineshloka
{आपत्काले हि संन्यासः कर्तव्य इति शिष्यते}
{जरयाऽभिपरीतेन शत्रुभिर्व्यंसितेन वा}


\twolineshloka
{तस्मादिह कृतप्रज्ञास्त्यागं न परिचक्षते}
{धर्मव्यतिक्रमं चैव मन्यन्ते सूक्ष्मदर्शिनः}


\twolineshloka
{कथं तस्मात्समुत्पन्नास्तन्निष्ठास्तदुपाश्रयाः}
{तदेव निन्दां भाषेयुर्धाता तत्र न गर्ह्यते}


\twolineshloka
{श्रिया विहीनैरधनैर्नास्तिकैः संप्रवर्तितम्}
{वेदवादस्य विज्ञानं सत्याभासमिवानृत्}


\threelineshloka
{शक्यं तु मौनमास्थाय बिभ्रताऽऽत्मानमात्मना}
{धर्मच्छझ समास्थाय च्यवितुं न तु जीवितुम्}
{}


\twolineshloka
{शक्यं पुनररण्येषु सुखमेकेन जीवितुम्}
{अबिभ्रता पुत्रपौत्रान्देवर्षीनतिथीन्पितॄन्}


\twolineshloka
{नेमे मृगाः स्वर्गजितो न वराहा न पक्षिणः}
{अथान्येन प्रकारेण पुण्यमाहुर्न ते जनाः}


\twolineshloka
{यदि संन्यासतः सिद्धिं राजा कश्चिदवाप्नुयात्}
{पर्वताश्च दुमाश्चैव क्षिप्तं सिद्धिमवाप्नुयुः}


\twolineshloka
{एते हि नित्यसंन्यासा दृश्यन्ते निरुपद्रवाः}
{अपरिग्रहवन्तश्च सततं ब्रह्मचारिणः}


\twolineshloka
{अथ चेदात्मभाग्येषु नान्येषां सिद्धिमश्र्नुते}
{तस्मात्कर्मैव कर्तव्यं नास्ति सिद्धिरकर्मणः}


\twolineshloka
{औदकाः सृष्टयश्चैव जन्तवः सिद्धिमाप्नुयुः}
{तेषामात्मैव भर्तव्यो नान्यः कश्चन विद्यते}


\twolineshloka
{अवेक्षस्व यथा स्वैः कर्मभिर्व्यापृतं जगत्}
{तस्मात्कर्मैव कर्तव्यं न्प्रस्ति सिद्धिरकर्मणः}


\chapter{अध्यायः ११}
\twolineshloka
{अर्जुन उवाच}
{}


\twolineshloka
{अत्रैवोदाहरन्तीममितिहासं पुरातनम्}
{तापसैः सह संवादं शक्रस्य भरतर्षभ}


\threelineshloka
{`शक्यं पुनररण्येषु सुखमेतेन जीवितुम्}
{'केचिद्गृहान्परित्यज्य वनमभ्यागमन्द्विजाः}
{अजातश्मश्रवो मन्दाः कुले जाताः प्रवव्रजुः}


\twolineshloka
{धर्मोऽयमिति मन्वानाः समृद्धा धर्मचारिणः}
{त्यक्त्वा भ्रातॄन्पितॄंश्चैव तानिन्द्रोऽन्वकृपायत}


\twolineshloka
{स तान्बभाषे मघवान्पक्षीभूत्वा हिरण्मयः}
{सुदुष्करं मनुष्यैस्तु यत्कृतं विघसाशिभिः}


\threelineshloka
{पुण्यं भवति कर्मैषां प्रशस्तं चैव जीवितम्}
{सिद्धार्थास्ते गतिं मुख्यां प्राप्ताः कर्मपरायणाः ॥ऋषय ऊचुः}
{}


\threelineshloka
{अहो बतायं शकुनिर्विघसाशान्प्रशंसति}
{अस्मान्नूनमयं शास्ति वयं च विघसाशिनः ॥शकुनिरुवाच}
{}


\threelineshloka
{नाहं युष्मान्प्रशंसामि पङ्कदिग्धान्रजस्वलान्}
{उच्छिष्टभोजिनो मन्दानन्ये वै विघसाशिनः ॥ऋषय ऊचुः}
{}


\threelineshloka
{इदं श्रेयः परमिति वयमेवममंस्महि}
{शकुने ब्रूहि यच्छ्रेयो वयं ते श्रद्दधामहे ॥शकुनिरुवाच}
{}


\threelineshloka
{यदि मां नाभिशङ्कध्वं विभज्यात्मानमात्मना}
{ततोऽहं वः प्रवक्ष्यामि याथातथ्यं हितं वचः ॥ऋषय ऊचुः}
{}


\threelineshloka
{शृणुमस्ते वचस्तात पन्थानो विदितास्तव}
{नियोगे चैव धर्मात्मन्स्थातुमिच्छाम शाधि नः ॥शकुनिरुवाच}
{}


\twolineshloka
{चतुष्पदां गौः प्रवरा लोहानां काञ्चनं वरम्}
{शब्दानां प्रवरो मन्त्रो ब्राह्मणो द्विपदां वरः}


\twolineshloka
{मन्त्रोऽयं जातकर्मादिर्ब्राह्मणस्य विधीयते}
{जीवतोऽपि यथाकालं श्मशाननिधनान्तकः}


\threelineshloka
{कर्माणि वैदिकान्यस्य स्वर्ग्यः पन्थास्त्वनुत्तमः}
{अथ सर्वाणि कर्माणि मन्त्रसिद्धानि चक्षते}
{आम्नायदृढवादीनि तथा सिद्धिरिहेष्यते}


\twolineshloka
{मासार्धमासा ऋतव आदित्यशशितारकम्}
{ग्रसन्ते कर्म भूतानि तदिदं कर्मशंसिनाम्}


% Check verse!
सिद्धिक्षेत्रमिदं पुण्यमयमेवाश्रमो महान्
\twolineshloka
{अथ ये कर्म निन्दन्तो मनुष्याः कापथं गताः}
{मूढानामर्थहीनानां तेषामेनस्तु विद्यते}


\twolineshloka
{देववंशान्पितृवंशान्ब्रह्मवंशांश्च शाश्चतान्}
{संत्यज्य मूढा वर्तन्ते ततो यान्त्यशुचीन्पथः}


\twolineshloka
{एतद्वोऽस्तु तपोयुक्तं ददामीत्यृषिचोदितम्}
{तस्मात्तदध्यावसतस्तपस्वित्वमिहोच्यते}


\twolineshloka
{देववंशान्ब्रह्मवंशान्पितृवंशांश्च शाश्वतान्}
{संविभज्य गुरोश्चर्यां तद्वै दुष्करमुच्यते}


\twolineshloka
{देवा वै दुष्करं कृत्वा विभूतिं परमां गताः}
{तस्माद्गार्हस्थ्यमुद्वोढुं दुष्करं प्रब्रवीमि वः}


\twolineshloka
{तपः श्रेष्ठं प्रजानां हि मूलमेतन्न संशयः}
{कुटुम्वविधिनाऽनेन यस्मिन्सर्वं प्रतिष्ठितम्}


\twolineshloka
{एतद्विदुस्तपो विप्रा द्वन्द्वातीता विमत्सराः}
{एतस्माद्वनमध्ये तु लोकेषु तप उच्यते}


\twolineshloka
{दुराधर्षं पदं चैव गच्छन्ति विघसाशिनः}
{सायंप्रातर्विभज्यान्नं स्वकुटुम्बे यथाविधि}


\twolineshloka
{दत्त्वाऽतिथिभ्यो देवेभ्यः पितृभ्यः स्वजनाय च}
{अवशिष्टानि येऽश्नन्ति तानाहुर्विघसाशिनः}


\twolineshloka
{तस्मात्स्वधर्ममास्थाय सुव्रताः सत्यवादिनः}
{लोकस्य गुरवो भूत्वा ते भवन्त्यनुपस्कृताः}


\threelineshloka
{त्रिदिवं प्राप्य शक्रस्य स्वर्गलोके विमत्सराः}
{वसन्ति शाश्वतान्वर्षाञ्जना दुष्करकारिणः ॥अर्जुन उवाच}
{}


\twolineshloka
{ततस्ते तद्वचः श्रुत्वा धर्मार्थसहितं हितम्}
{उत्सृज्य नास्तिकमतिं गार्हस्थ्यं धर्ममाश्रिताः}


\twolineshloka
{तस्मात्त्वमपि सर्वज्ञ धैर्यमालम्ब्य शाश्वतम्}
{प्रशाधि पृथिवीं कृत्स्नां हतामित्रां नरोत्तम}


\chapter{अध्यायः १२}
\twolineshloka
{वैशंपायन उवाच}
{}


\twolineshloka
{अर्जुनस्य वचः श्रुत्वा नकुलो वाक्यमब्रवीत्}
{राजानमभिसंप्रेक्ष्य सर्वधर्मभृतां वरम्}


\threelineshloka
{अनुरुध्य महाप्राज्ञो भ्रातुश्चित्तमरिंदम}
{व्यूढोरस्को महाबाहुस्ताम्रास्यो मितभाषिता ॥नकुल उवाच}
{}


\twolineshloka
{विशाखयूपे देवानां सर्वेषामग्नयश्चिताः}
{तस्माद्विद्वि महाराज देवाः कर्मफले स्थिताः}


\twolineshloka
{अनास्तिकानां भूतानां प्राणदाः पितरश्च ये}
{तेऽपि कर्मैव कुर्वन्ति विधिं पश्यस्व पार्थिव}


\twolineshloka
{वेदवादापविद्धास्तु तान्विद्धि भृशनास्तिकान्}
{न हि वेदोक्तमुत्सृज्य विप्रः सर्वेषु कर्मसु}


\twolineshloka
{देवयानेन नाकस्य पृष्ठमाप्नोति भारत}
{अत्याश्रमानयं सर्वानित्याहुर्वेदमिश्चयाः}


\twolineshloka
{ब्राह्मणाः श्रुतिसंपन्नास्तान्निबोध नराधिप}
{वित्तानि धर्मलब्धानि क्रतुमुख्येष्ववासृजन्}


% Check verse!
कृतात्मा स महाराज स वै त्यागी स्मृतो नरः
\twolineshloka
{अनवेक्ष्य सुखादानं तथैवोर्ध्वं प्रतिष्ठितः}
{आत्मत्यागी महाराज स त्यागी तापसो मतः}


\twolineshloka
{अनिकेतः परिपतन्वृक्षमूलाश्रयो मुनिः}
{अयाचकः सदायोगी स त्यागी पार्थ भिक्षुकः}


\twolineshloka
{क्रोधहर्षावनादृत्य पैशुन्यं च विशेषतः}
{विप्रो वेदानधीते यः स त्यागी गुरुपूजकः}


\threelineshloka
{आश्रमांस्तुलया सर्वान्धृतानाहुर्मनीषिणः}
{एकतश्च त्रयो राजन्गृहस्थाश्रम एकतः ॥ 12-12-13aसमीक्ष्य तुलयापार्थ कामं स्वर्गं च भारत}
{अयं पन्था महर्षीणाप्रियं लोकविदां गतिः}


\twolineshloka
{इति यः कुरुते भावं स त्यागी भरतर्षभ}
{नरः परित्यज्य गृहान्वनमेति विमूढवत्}


\twolineshloka
{यदा कामान्समीक्षेत धर्मवैतंसिको नरः}
{अथैनं मृत्युपाशेन कण्ठे बध्नाति सृत्युराट्}


\twolineshloka
{अभिमानकृतं कर्म नैतत्फलवदुच्यते}
{त्यागयुक्तं महाराज सर्वमेव महाफलम्}


\twolineshloka
{शमो दमस्तथा धैर्यं सत्यं शौचमथार्जवम्}
{यज्ञो धृतिश्च धर्मश्च नित्यमार्षो विधिः स्मृतः}


\twolineshloka
{पितृदेवातिथिकृते समारम्भोऽत्र शस्यते}
{अत्रैव हि महाराज त्रिवर्गः केवलं फलम्}


\twolineshloka
{एतस्मिन्वर्तमानस्य विधौ विप्रनिषेविते}
{त्यागिनः प्रसृतस्येह नोच्छित्तिर्विद्यते क्वचित्}


\twolineshloka
{असृजद्धि प्रजा राजन्प्रजापतिरकल्मषः}
{मां यक्ष्यन्तीति धर्मात्मा यज्ञैर्विविधदक्षिणैः}


\twolineshloka
{वीरुधश्चैव वृक्षांश्च यज्ञार्थं वै तथौषधीः}
{पशूंश्चैव तथा मेध्यान्यज्ञार्थानि हवींषि च}


\twolineshloka
{गृहस्थाश्रमिणस्तच्च यज्ञकर्माविरोधितम्}
{तस्माद्गार्हस्थ्यमेवेह दुष्करं दुर्लभं तथा}


\twolineshloka
{तत्संप्राप्य गृहस्था ये पशुधान्यधनान्विताः}
{न यजन्ते महाराज शाश्वतं तेषु किल्विषम्}


\twolineshloka
{स्वाध्याययज्ञा ऋषयो ज्ञानयज्ञास्तथा परे}
{अथापरे महायज्ञान्मनस्येव वितन्वते}


\twolineshloka
{`इदमन्यन्महाराज विद्वद्भिः कथितं मम}
{भूमिरग्निश्च वायुश्च न चापो न दिवाकरः}


\twolineshloka
{नक्षत्राणि न चन्द्रश्च न दिशः काल एव च}
{शब्दः स्पर्शश्च रूपं च न गन्धो न रसः क्वचित्}


\twolineshloka
{न च सन्ति प्रमाणानि यैः प्रमेयं प्रसाध्यते}
{प्रत्यक्षमनुमानं न नोपमानमथागमः}


\twolineshloka
{नार्थापत्तिर्न चैतिह्यं न दृष्टान्तो न संशयः}
{न क्वचिन्निर्णयो राजन्न धर्माधर्म एव च}


\threelineshloka
{तिर्यक्व स्थावरं चैव न देवा न च मानुषाः}
{वर्णाश्रमविभागाश्च न च कर्ता न कामकृत्}
{न चार्थश्च विभूतिश्च न चार्थस्य विचेष्टितम्}


\twolineshloka
{तमोभूतमिदं सर्वमनालोकं जगन्नृप}
{न चात्मा विद्यमानोपि मनसा योगमिच्छति}


\threelineshloka
{अचेतनं मनस्त्वासीदात्मा एव सचेतनः}
{ईश्वरश्चेतनस्त्वेकस्तेनेदं गहनीकृतम्}
{मन्त्राश्च चेतना राजन्न च देहेन योजिताः}


\twolineshloka
{तेन विश्वसृजो नाम ऋषयो मन्त्रदेवताः}
{चैतन्यमीश्वरात्प्राप्य ब्रह्माण्डं तैर्विनिर्मितम्}


\twolineshloka
{इष्ट्वा विश्वसृजं यज्ञं निर्मितः प्रपितामहः}
{सृष्टिस्तेन समारब्धा प्रसादादीश्वरस्य च}


\twolineshloka
{चैतन्यमीश्वरस्येदं सचेतनमिदं जगत्}
{योगेन च समाविष्टं जगत्कृत्स्नं च शंभुना}


\twolineshloka
{धर्मश्चार्थश्च कामश्च उक्तो मोक्षश्च संक्षये}
{ब्रह्मणः परमेशस्य ईश्वरेण यदृच्छया}


\twolineshloka
{अज्ञो जन्तुरनीशश्च भाजनं सुखदुःखयोः}
{ईश्वरप्रेरितो गच्छेत्स्वर्गं वा श्वभ्रमेव वा}


\threelineshloka
{प्रधानं पुरुषः चैव आत्मानं सर्वदेहिनाम्}
{मनसा विषयैश्चैव चेतनेन प्रचोदिताः}
{सुखदुःखेन युज्यन्ते कर्मभिश्च प्रचोदिताः}


\twolineshloka
{वर्णाश्रमविभागाश्च ईश्वरेण प्रवर्तिताः}
{सदेवासुरगन्धर्वं येनेदं निर्मितं जगत्}


\twolineshloka
{त्वं चान्ये च महाराज ईश्वरस्य वशे स्थिताः}
{जीवन्ते च म्रियन्ते च न स्वतन्त्राः कथंचन}


\twolineshloka
{हित्वाहित्वा च भूतानि हत्वा सर्वमिदं जगत्}
{यजन्ते कर्मणा देवा न स पापेन लिप्यते}


\twolineshloka
{हिंसात्मकानि कर्माणि सर्वेषां गृहमेधिनाम्}
{देवतानामृषीणां च ते च यान्ति परां गतिम्}


\threelineshloka
{पातिताः शत्रवः पूर्व सर्वत्र वसुधाधिपैः}
{प्रजानां हितकामैश्च आत्मनश्च हितैषिभिः}
{}


\twolineshloka
{यदि तत्र भवेत्पापं कथं ते स्वर्गमास्थिताः}
{न प्राप्ता नरकं राजन्वेष्टिताः पापकर्मभिः ॥'}


\twolineshloka
{एवं मनःसमाधानं मार्गमातिष्ठतो नृप}
{द्विजातेर्ब्रह्मभूतस्य स्पृहयन्ति दिवौकसः}


\twolineshloka
{स रत्नानि विचित्राणि संहृतानि ततस्ततः}
{मखेष्वनभिसन्त्यज्य नास्तिक्यमभिजल्पसि}


\twolineshloka
{कुटुम्बमास्थिते त्यागं न पश्यामि नराधिप}
{राजसूयाश्वमेधेषु सर्वमेधेषु वा पुनः}


\threelineshloka
{ये चान्ये क्रतवस्तात ब्राह्मणैरभिपूजिताः}
{तैर्यजस्व महीपाल शक्रो देवपतिर्यथा}
{}


\twolineshloka
{राज्ञः प्रमाददोषेण दस्युभिः परिमुष्यताम्}
{अशरण्यः प्रजानां यः स राजा कलिरुच्यते}


\twolineshloka
{अश्वान्गाश्चैव दासीश्च करेणूश्च स्वलंकृताः}
{ग्रामाञ्जनपदांश्चैव क्षेत्राणि च गृहाणि च}


\twolineshloka
{अप्रदाय द्विजातिभ्यो मात्सर्याविष्टचेतसः}
{वयं ते राजकलयो भविष्याम विशांपते}


\twolineshloka
{अदातारोऽशरण्याश्च राजकिल्विषभागिनः}
{दुःखानामेव भोक्तारो न सुखानां कदाचन}


\twolineshloka
{अनिष्ट्वा च महायज्ञैरकृत्वा च पितृस्वधाम्}
{तीर्थेष्वनभिसंप्लुत्य प्रव्रजिष्यसि चेत्प्रभो}


\twolineshloka
{छिन्नाभ्रमिव गन्तासि विलयं मारुतेरितम्}
{लोकयोरुभयोर्भ्रष्टो ह्यन्तराले व्यवस्थितः}


\twolineshloka
{अन्तर्बहिश्च यत्किंचिन्मनोव्यासङ्गकारकम्}
{परित्यज्य भवेत्त्यागी न हित्वा प्रतितिष्ठति}


\twolineshloka
{एतस्मिन्वर्तमानस्य विधौ विप्रनिषेविते}
{ब्राह्मणस्य महाराज नोच्छित्तिर्विद्यते क्वचित्}


\twolineshloka
{निहत्य शत्रूंस्तरसा समृद्धाञ्शक्रो यथा दैत्यबलानि सङ्ख्ये}
{कः पार्थ शोचेन्निरतः स्वधर्मे पूर्वैः स्मृते पार्थिवशिष्टजुष्टे}


\twolineshloka
{क्षात्रेण धर्मेण पराक्रमेण जित्वा महीं मन्त्रविद्भ्यः प्रदाय}
{नाकस्य पृष्ठेऽसि नरेन्द्र गन्ता न शोचितव्यं भवताऽद्य पार्थ}


\chapter{अध्यायः १३}
\twolineshloka
{सहदेव उवाच}
{}


\twolineshloka
{न बाह्यं द्रव्यमुत्सृज्य सिद्धिर्भवति भारत}
{शारीरं द्रव्यमुत्सृज्य सिद्धिर्भवति वा न वा}


\twolineshloka
{बाह्यद्रव्यविमुक्तस्य शारीरेष्वनुगृध्यतः}
{यो धर्मो यत्सुखं वा स्याद्द्विषतां तत्तथाऽस्तु नः}


\twolineshloka
{शारीरंद्रव्यमुत्सृज्य पृथिवीमनुशासतः}
{यो धर्मो यत्सुखं वा स्यात्सुहृदां तत्तथाऽस्तु नः}


\twolineshloka
{द्व्यक्षरस्तु भवेन्मृत्युरुयक्षरं ब्रह्म शाश्वतम्}
{ममेति द्व्यक्षरो मृत्युर्न ममेति च शाश्वतम्}


\twolineshloka
{ब्रह्ममृत्यू ततो राजन्नात्मन्येव समाश्रितौ}
{अदृश्यमानौ भूतानि योजयेतामसंशयम्}


\twolineshloka
{अविनाशोऽस्य सत्वस्य नियतो यदि भारत}
{हित्वा शरीरं भूतानां न हिंसा प्रतिपत्स्यते}


\twolineshloka
{अथापि च सहोत्पत्तिः सत्वस्य प्रलयस्तथा}
{नष्टे शरीरे नष्टः स्याद्वृथा च स्यात्क्रियापथः}


\twolineshloka
{तस्मादेकान्तमुत्सृज्य पूर्वैः पूर्वतरैश्च यः}
{पन्था निषेवितः सद्भिः स निषेव्यो विजानता}


\twolineshloka
{`स्वायंभुवेन मनुना तथाऽन्यैश्तक्रवर्तिभिः}
{यद्ययं ह्यधमः पन्थाः कस्मात्तैस्तैर्निषेवितः}


\twolineshloka
{कृतत्रेतादियुक्तानि गुणवन्ति च भारत}
{युगानि बहुशस्तैश्च भुक्तेयमवनी नृप ॥'}


\twolineshloka
{लब्ध्वाऽपि पृथिवीं कृत्स्नां सहस्थावरजङ्गमाम्}
{न भुङ्क्ते यो नृपः सम्यक्किंफलं तस्य जीविते}


\twolineshloka
{अथवा वसतो राजन्वने वन्येन जीवतः}
{द्रव्येषु यस्य ममता मृत्योरास्ये स वर्तते}


\twolineshloka
{बाह्यान्तराणां भूतानां स्वभावं पश्य भारत}
{ये तु पश्यन्ति तद्भूतं मुच्यन्ते ते महाभयात्}


\twolineshloka
{भवान्पिता भवान्माता भवान्भ्राता भवान्गुरुः}
{दुःखप्रलापानार्तस्य तन्मे त्वं क्षन्तुमर्हसि}


\twolineshloka
{तथ्यं वा यदि वाऽतथ्यं यन्मयैतत्प्रभाषितम्}
{तद्विद्वि पृथिवीपाल भक्त्या भरतसत्तम}


\chapter{अध्यायः १४}
\twolineshloka
{वैशंपायन उवाच}
{}


\twolineshloka
{अव्याहरति कौन्तेये धर्मराजे युधिष्ठिरे}
{भ्रातृणां ब्रुवतां तांस्तान्विविधान्वेदनिश्चयान्}


\twolineshloka
{महाभिजनसंपन्ना श्रीमत्यायतलोचना}
{अभ्यभाषत राजानं द्रौपदी योषितां वरा}


\twolineshloka
{आसीनमृषभं राज्ञां भ्रातृभिः परिवारितम्}
{सिंहशार्दूलसदृशैर्वारणैरिव यूथपम्}


\twolineshloka
{अभिमानवती नित्यं विशेषेण युधिष्ठिरे}
{लालिता सततं राज्ञा धर्माधर्मनिदर्शिनी}


\threelineshloka
{आमन्त्रयित्वा सुश्रोणी साम्ना परमवल्गुना}
{भर्तारमभिसंप्रेक्ष्य ततो वचनमब्रवीत् ॥द्रौपद्युवाच}
{}


\twolineshloka
{इमे ते भ्रातरः पार्थ शुष्यन्ते स्तोकका इव}
{वावाश्यमानास्तिष्ठन्ति न चैनानभिनन्दसे}


\twolineshloka
{नन्दयैतान्महाराज मत्तानिव महाद्विपान्}
{उपपन्नेन वाक्येन सततं दुःखभागिनः}


\twolineshloka
{कथं द्वैतवने राजन्पूर्वमुक्त्वा तथा वचः}
{भ्रातॄनेतान्स्म सहिताञ्शीतवातातपार्दितान्}


\twolineshloka
{वयं दुर्योधनं हत्वा मृधे भोक्ष्याम मेदिनीम्}
{संपूर्णां सर्वकामानामाहवे विजयैषिणः}


\twolineshloka
{नृवीरांश्च रथान्हत्वा निहत्य च महागजान्}
{संस्तीर्य च रथैर्भूमिं ससादिभिररिंदमाः}


\twolineshloka
{यजेभ विविधैर्यज्ञैः समृद्धैराप्तदक्षिणैः}
{वनवासकृतं दुःखं भविष्यति सुखाय नः}


\twolineshloka
{इत्येतानेवमुक्त्वा त्वं स्वयं धर्मभृतां वर}
{कथमद्य पुनर्वीर विनिहंसि मनांसि नः}


\twolineshloka
{न क्लीबो वसुधां भुङ्क्ते न क्लीबो धनमश्नुते}
{न क्लीबस्य गृहे पुत्रा मत्स्याः पङ्क इवासते}


\twolineshloka
{नादण्डः क्षत्रियो भाति नादण्डो भूमिमश्नुते}
{नादण्डस्य प्रजा राज्ञः सुखं विन्दन्ति भारत}


\twolineshloka
{`सदेवासुरगन्घर्वैरप्सरोभिर्विभूषितम्}
{रक्षोभिर्गुह्यकैर्नागैर्मनुष्यैश्च विभूषितम्}


\twolineshloka
{त्रिवर्गेण च संपूर्णं त्रिवर्गस्यागमेन च}
{दण्डेनाभ्याहृतं सर्वं जगद्भोगाय कल्पते}


\twolineshloka
{स्वायंभुवं महीपाल आगमं शृणु शाश्वतम्}
{विप्राणां विदितश्चायं तव चैव विशांपते}


\threelineshloka
{अराजके हि लोकेऽस्मिन्सर्वतो विद्रुते भयात्}
{रक्षार्थमस्य लोकस्य राजानमसृजत्प्रभुः}
{महाकायं महावीर्यं पालने जगतः क्षमम्}


\twolineshloka
{अनिलाग्नियमार्काणामिन्द्रस्य वरुणस्य च}
{चन्द्रवित्तेशयोश्चैव मात्रा निर्हृत्य शाश्वतीः}


\twolineshloka
{यस्मादेषां सुरेन्द्राणां संभवत्यंशतो नृपः}
{तस्मादभिभवत्येष सर्वभूतानि तेजसा}


\twolineshloka
{तपत्यादित्यवच्चैव चक्षूंषि च मनांसि च}
{च चैनं भुवि शक्नोति कश्चिदप्यभिवीक्षितुम्}


\twolineshloka
{सोऽग्निर्भवति वायुश्च सोऽर्कः सोमश्च धर्मराट्}
{स कुबेरः स वरुणः स महेन्द्रः प्रतापवान्}


\twolineshloka
{पितामहस्य देवस्य विष्णोः शर्वस्य चैव हि}
{ऋषीणां चैव सर्वेषां तस्मिंस्तेजः प्रतिष्ठितम्}


\twolineshloka
{बालोऽपि नावमन्तव्यो मनुष्य इति भूमिपः}
{महती देवता ह्येषा नररूपेण तिष्ठति}


\twolineshloka
{एकमेव दहत्यग्निर्नरं दुरुपसर्पिणम्}
{कुलं दहति राजाग्निः सपशुद्रव्यसंचयम्}


\twolineshloka
{धृतराष्ट्रकुलं दग्धं क्रोधोद्भूतेन वह्निना}
{प्रत्यक्षमेतल्लोकस्य संशयो हि न विद्यते}


\twolineshloka
{कुलजो वृत्तसंपन्नो धार्मिकश्च महीपतिः}
{प्रजानां पालने युक्तः पूज्यते दैवतैरपि}


\twolineshloka
{कार्यं योऽवेक्ष्य शक्तिं च देशकालौ च तत्वतः}
{कुरुते धर्मसिद्ध्यर्थं वैश्वरूप्यं पुनः पुनः}


\twolineshloka
{तस्य प्रसादे पझा श्रीर्विजयश्च पराक्रमे}
{मृत्युश्च वसति क्रोधे सर्वतेजोमयो हि सः}


\twolineshloka
{तं यस्तु द्वेष्टि संमोहात्स विनश्यति मानवः}
{तस्य ह्याशुविनाशाय राजाऽपि कुरुते मनः}


\twolineshloka
{तस्माद्धर्मं यदिष्टेषु स व्यवस्यति पार्थिवः}
{अनिष्टं चाप्यनिष्टेषु तद्धर्मं न विचालयेत्}


% Check verse!
तस्यार्थे सर्वभूतानां गोप्तारं धर्ममात्मजम् ॥ब्रह्मतेजोमयं दण्डमसृजत्पूर्वमीश्वरः
\twolineshloka
{तस्य सर्वाणि भूतानि स्थावराणि चराणि च}
{भयाद्भोगाय कल्पन्ते धर्मान्न विचलन्ति च}


\twolineshloka
{देशकालौ च शक्तिं च कार्यं चावेक्ष्य तत्वतः}
{यथार्हतः संप्रणयेन्नरेष्वन्यायवर्तिषु}


\twolineshloka
{स राजा पुरुषो दण्डः स नेता शासिता च सः}
{वर्णानामाश्रमाणां च धर्मप्रभुरथाव्ययः}


\twolineshloka
{दण्डः शास्ति प्रजाः सर्वा दण्ड एवाभिरक्षति}
{दण्डः सुप्तेषु जागर्ति दण्डं धर्मं विदुर्बुधाः}


\twolineshloka
{सुसमीक्ष्य धृतो दण्डः सर्वा रञ्जयति प्रजाः}
{असमीक्ष्य प्रणीतस्तु विनाशयति सर्वशः}


\twolineshloka
{यदि न प्रणयेद्राजा दण्डं दण्ड्येष्वतन्द्रितः}
{जले मत्स्यानिवाभक्ष्यन्दुर्बलान्बलवत्तराः}


\twolineshloka
{काकोऽद्याच्च पुरोडाशं श्वा चैवावलिहेद्धविः}
{स्वामित्वं न क्वचिच्च स्यात्प्रपद्येताधरोत्तरम्}


\twolineshloka
{सर्वो दण्डजितो लोको दुर्लभस्तु शुचिर्नरः}
{दण्डस्य हि भयात्सर्वं जगद्भोगाय कल्पते}


\twolineshloka
{देवदानवगन्धर्वा रक्षांसि पतगोरगाः}
{तेऽपि भोगाय कल्पन्ते दण्डेनैवाभिपीडिताः}


\twolineshloka
{दूष्येयुः सर्ववर्णाश्च भिद्येरन्सर्वसेतवः}
{सर्वलोकप्रकोपश्च भवेद्दण्डस्य विभ्रमात्}


\twolineshloka
{यत्र श्यामो लोहिताक्षो दण्डश्चरति पापहा}
{प्रजास्तत्र न मुह्यन्ति नेता चेत्साधु पश्यति}


\twolineshloka
{आहुस्तस्य प्रणेतारं राजानं सत्यवादिनम्}
{समीक्ष्यकारिणं प्राज्ञं धर्मकामार्थकोविदम्}


\twolineshloka
{तं राजा प्रणयन्सम्यक्स्वर्गायाभिप्रवर्तते}
{कामात्मविषयी क्षुद्रो दण्डेनैव निहन्यते}


\twolineshloka
{दण्डो हि सुमहातेजा दुर्धरश्चाकृतात्मभिः}
{धर्माद्विचलितं हन्ति नृपमेव सबान्धवम्}


\twolineshloka
{ततो दुर्गं च राष्ट्रं च लोकं च सचराचरम्}
{अन्तरिक्षगतांश्चैव मुनीन्देवांश्च हिंसति}


\twolineshloka
{सोऽसहायेन मूढेन लुब्धेनाकृतबुद्धिना}
{अशक्यो न्यायतो नेतुं विषयांश्चैव सेवता}


\twolineshloka
{शुचिना सत्यसन्धेन नीतिशास्त्रानुसारिणा}
{दण्डः प्रणेतुं शक्यो हि सुसहायेन धीमता}


\twolineshloka
{स्वराष्ट्रे न्यायवर्ती स्याद्भृशदण्डश्च शत्रुषु}
{सुहृत्स्वजिह्मः स्निग्धेषु ब्राह्मणेषु क्षमान्वितः}


\twolineshloka
{एवंवृत्तस्य राज्ञस्तु शिलोञ्छेनापि जीवतः}
{विस्तीर्येत यशो लोके तैलबिन्दुरिवाम्भसि}


\twolineshloka
{अतस्तु विपरीतस्य नृपतेरकृतात्मनः}
{संक्षिप्येन यशो लोके घृतबिन्दुरिवाम्भसि}


\twolineshloka
{देवदेवेन रुद्रेण ब्रह्मणा च महीपते}
{विष्णुना चैव देवेन शक्रेण च महात्मना}


\threelineshloka
{लोकपालैश्च भूतैश्च पाण्डवैश्च महात्मभिः}
{धर्माद्विचलिता राजन्धार्तराष्ट्रा निपातिताः}
{अधार्मिका दुराचाराः ससैन्या विनिपातिताः}


\twolineshloka
{तान्निहत्य न दोषस्ते स्वल्पोऽपि जगतीपते}
{छलेन मायया वाऽथ क्षत्रधर्मेण वा नृप ॥'}


\twolineshloka
{मित्रता सर्वभूतेषु दानमध्ययनं तपः}
{ब्राह्मणस्यैव धर्मः स्यान्न राज्ञो राजसत्तम}


\twolineshloka
{असतां प्रतिषेधश्च सतां च परिपालनम्}
{एष राज्ञां परो धर्मः समरे चापलायनम्}


\twolineshloka
{यस्मिन्क्षमा च क्रोधश्च दानादाने भयाभये}
{निग्रहानुग्रहौ चोभौ स वै धर्मविदुच्यते}


\twolineshloka
{न श्रुतेन न दानेन न सांत्वेन न चेज्यया}
{त्वयेयं पृथिवी लब्धा न संकोचेन चाप्युत}


\twolineshloka
{यत्तद्बलममित्राणां तथा वीरसमुद्यतम्}
{हस्त्यश्वरथसंपन्नं त्रिभिरङ्गैरनुत्तमम्}


\twolineshloka
{रक्षितं द्रोणकर्णाभ्यामश्वत्थाम्ना कृपेण च}
{तत्त्वया निहतं वीर तस्माद्भुङ्क्ष्व वसुंधराम्}


\twolineshloka
{जम्बूद्वीपो महाराज नानाजनपदैर्युतः}
{त्वया पुरुषशार्दूल दण्डेन मृदितः प्रभो}


\twolineshloka
{जम्बूद्वीपेन सदृशः क्रौञ्चद्वीपो नराधिप}
{अपरेण महामेरोर्दण्डेन मृदितस्त्वया}


\twolineshloka
{क्रौञ्चद्वीपेन सदृशः शाकद्वीपो नराधिप}
{पूर्वेण तु महामेरोर्दण्डेन मृदितस्त्वया}


\twolineshloka
{उत्तरेण महामेरोः शाकद्वीपेन संमितः}
{भद्राश्वः पुरुषव्याघ्र दण्डेन मृदितस्त्वया}


\twolineshloka
{द्वीपाश्च सान्तरद्वीपा नानाजनपदाश्रयाः}
{विगाह्य सागरं वीर दण्डेन मृदितास्त्वया}


\twolineshloka
{एतान्यप्रतिमेयानि कृत्वा कर्माणि भारत}
{न प्रीयसे महाराज पूज्यमानो द्विजातिभिः}


\twolineshloka
{स त्वं भ्रातॄनिमान्दृष्ट्वा प्रतिनन्दस्व भारत}
{ऋषभानिव संमत्तान्गजेन्द्रान्गर्जितानिव}


\twolineshloka
{अमरप्रतिमाः सर्वे शत्रुसाहाः परंतपाः}
{एकैकोऽपि सुखायैषां मम स्यादिति मे मतिः}


\twolineshloka
{किं पुनः पुरुषव्याघ्राः पतयो मे नरर्षभाः}
{समस्तानीन्द्रियाणीव शरीरस्य विचेष्टने}


\twolineshloka
{अनृतं नाब्रवीच्छ्वश्रूः सर्वज्ञा सर्वदर्शिनी}
{युधिष्ठिरस्त्वां पाञ्चालि सुखे धास्यत्यनुत्तमे}


\twolineshloka
{इत्वा राजसहस्राणि बहून्याशुपराक्रमः}
{तद्व्यर्थं संप्रपश्यामि मोहात्तव जनाधिप}


\twolineshloka
{येषामुन्मत्तको ज्येष्ठः सर्वे तेऽप्यनुसारिणः}
{तवोन्मादान्महाराजसोन्मादाः सर्वपाण्डवाः}


\twolineshloka
{यदि हि स्युरनुन्मत्ता भ्रातरस्ते नराधिप}
{बद्ध्वा त्वां नास्तिकैः सार्धं प्रशासेयुर्वसुंधराम्}


\twolineshloka
{कुरुते मूढ एवं हि यः श्रेयो नाधिगच्छति}
{धूपैरञ्जनयोगैश्च नस्यकर्मभिरेव च}


\twolineshloka
{`उन्मत्तिरपनेतव्या तव राजन्यदृच्छया}
{'भेषजैः स चिकित्स्यः स्याद्य उन्मार्गेण गच्छति}


\twolineshloka
{साहं सर्वाधमा लोके स्त्रीणां भरतसत्तम}
{तथा विनिकृता पुत्रैर्याऽहमिच्छामि जीवितुम्}


\twolineshloka
{धृतराष्ट्रसुता राजन्नित्यमुत्पथगामिनः}
{तादृशानां वधे दोषं नाहं पश्यामि कर्हिचित्}


% Check verse!
इमांश्चोशनसा गीताञ्श्लोकाञ्श्रृणु नराधिप
\twolineshloka
{आत्महन्ताऽर्थहन्ता च बन्धुहन्ता विषप्रदः}
{अकारणेन हन्ता च यश्च भार्यां परामृशेत्}


\twolineshloka
{निर्दोषं वधमेतेषां षण्णामप्याततायिनाम्}
{ब्रह्मा प्रोवाच भगवान्भार्गवाय महात्मने}


\twolineshloka
{ब्रह्मक्षत्रविशां राजन्सत्पथे वर्तिनामपि}
{प्रसह्यागारमागम्य हन्तारं गरदं तथा}


\twolineshloka
{अभक्ष्यापेयदातारमग्निदं च निशातयेत्}
{मार्ग एष महीपानां गोब्राह्मणवधेषु च}


\twolineshloka
{केशग्रहे च नारीणामपि युध्येत्पितामहम्}
{ब्रह्माणं देवदेवेशं किं पुनः पापकारिणम्}


\twolineshloka
{गोब्राह्मणार्थे व्यसने च राज्ञां राष्ट्रोपमर्दे स्वशरीरहेतोः}
{स्त्रीणां च विक्रुष्टरुतानि श्रुत्वा विप्रोऽपि युध्येत महाप्रभावः}


\twolineshloka
{धर्माद्विचलितं विप्रं निहन्यादाततायिनम्}
{तस्यान्यत्र वधं विद्वान्मनसाऽपि न चिन्तयेत्}


\twolineshloka
{गोब्राह्मणवधे वृत्तं मन्त्रत्राणार्थमेव च}
{निहन्यात्क्षत्रियो विप्रं स्वकुटुम्बस्य चाप्तये}


\twolineshloka
{तस्करेण नृशंसेन धर्मात्प्रचलितेन च}
{क्षत्रबन्धुः परं शक्त्या युध्येद्विप्रेण संयुगे}


\twolineshloka
{आततायिनमायान्तमपि वेदान्तपारगम्}
{जिघांसन्तं जिघांसीयान्न तेन भ्रूणहा भवेत्}


\twolineshloka
{ब्राह्मणः क्षत्रियो वैश्यः शूद्रो वाऽप्यन्त्यजोपि वा}
{न हन्याद्ब्राह्मणं शान्तं तृणेनापि कदाचन}


\twolineshloka
{ब्राह्मणायावगुर्याद्यः स्पृष्ट्वा गुरुतरं महत्}
{वर्षाणां त्रिशतं पापः प्रतिष्ठां नाधिगच्छति}


\twolineshloka
{सहस्राणि च वर्षाणि निहत्य नरके पतेत्}
{तस्मान्नैवावगुर्याद्धि नैव शस्त्रं निपातयेत्}


\twolineshloka
{शोणितं यावतः पांसून्गृह्णातीति हि धारणा}
{तावतीः स समाः पापो नरके परिवर्तते}


\twolineshloka
{त्वगस्थिभेदं विप्रस्य यः कुर्यात्कारयेत वा}
{ब्रह्महा स तु विज्ञेयः प्रायश्चित्ती नराधमः}


\twolineshloka
{श्रोत्रियं ब्राह्मणं हत्वा तथाऽऽत्रेयीं च ब्राह्मणीम्}
{चतुर्विशतिवर्षाणि चरेद्ब्रह्महणो व्रतम्}


\twolineshloka
{द्विगुणां ब्रह्महत्येयं सर्वैः प्रोक्ता महर्षिभिः}
{प्रायश्चित्तमकुर्वाणं कृताङ्कं विप्रवासयेत्}


\twolineshloka
{ब्राह्मणं क्षत्रियं वैश्यं शूद्रं वा घातयेन्नृपः}
{ब्रह्मघ्नं तस्करं चैव माभूदेवं चरिष्यति}


\twolineshloka
{छित्त्वा हस्तौ च पादौ च नासिकोष्ठौ च भूपतिः}
{ब्रह्मघ्नं चोत्तमं पापं नेत्रोद्धारेण योजयेत्}


\twolineshloka
{शूद्रस्यैव स्मृतो दण्डस्तद्वद्राजन्यवैश्ययोः}
{प्रायश्चित्तमकुर्वाणं ब्राह्मणं तु प्रवासयेत्}


\twolineshloka
{क्षत्रियं वैश्यशूद्रौ च शस्त्रेणैव च घातयेत्}
{ब्रह्मघ्नान्ब्राह्मणात्राजा कृताङ्कान्विप्रवासयेत्}


\twolineshloka
{विकलेन्द्रियांस्त्रिवर्णांश्च चण्डालैः सह वासयेत्}
{तैश्च यः संपिबेत्कश्चित्स पिबन्ब्रह्महा भवेत्}


% Check verse!
प्रेतानां न च देयानि पिण्डदानानि केनचित्
\twolineshloka
{कृष्णवर्णा विरूपा च निर्णीता लम्बमूर्धजा}
{दुनोत्यदृष्ट्वा कर्तारं ब्रह्महत्येति तां विदुः}


\twolineshloka
{ब्रह्मघ्नेन पिबन्तश्च विप्रा देशाः पुराणि च}
{अचिरादेव पीड्यन्ते दुर्भिक्षव्याधितस्करैः}


\twolineshloka
{ब्राह्मणं पापकर्माणं विप्राणामाततायिनम्}
{क्षत्रियं वैश्यशूद्रौ च नेत्रोद्धारेण योजयेत्}


\twolineshloka
{दुर्बलानां बलं राजा बलिनो ये च साधवः}
{बलिनां दुर्बलानां च पापानां मृत्त्युरिष्यते}


\twolineshloka
{सदोषमपि यो हन्यादश्राव्य जगतीपते}
{दुर्बलं बलवन्तं वा स पराजयमर्हति}


\twolineshloka
{राजाज्ञां प्राड्विवाकं च नेच्छेद्यच्चापि निष्पतेत्}
{साक्षिणं साधुवाक्यं च जितं तमपि निर्दिशेत्}


\twolineshloka
{बन्धनान्निष्पतेद्यच्च प्रतिभूर्न ददाति च}
{कुलजश्च धनाढ्यश्च स पराजयमर्हति}


\twolineshloka
{राजाज्ञया समाहूतो यो न गच्छेत्सभां नरः}
{बलवन्तमुपाश्रित्य सायुधः स पराजितः}


\twolineshloka
{तं दण्डेन विनिर्जित्य महासाहसिकं नरम्}
{वियुक्तदेहसर्वस्वं परलोकं विसर्जयेत्}


\twolineshloka
{मृतस्यापि न देयानि पिण्डदानानि केनचित्}
{दत्त्वा दण्डं प्रयच्छेत मध्यमं पूर्वसाहयम्}


\twolineshloka
{कुलस्त्रीव्यभिचारं च राष्ट्रस्य च विमर्दनम्}
{ब्रह्महत्यां च चौर्यं च राजद्रोहं च पञ्चमम्}


\twolineshloka
{युद्धादन्यत्र हिंसायां सुरापस्य च कीर्तने}
{महान्तं गुरुतल्पे च मित्रद्रोहे च पातकम्}


\twolineshloka
{न कथंचिदुपेक्षेत महासाहसिकं नरम्}
{सर्वस्वमपहृत्याशु ततः प्राणैर्वियोजयेत्}


\twolineshloka
{त्रिषु वर्णेषु यो दण्डः प्रणीतो ब्रह्मणा पुरा}
{महासाहसिकं विप्रं कृताङ्कं विप्रवासयेत्}


\twolineshloka
{साहस्रो वा भवेद्दण्डः काञ्चनो देहनिष्क्रिया}
{चतुर्णामपि वर्णानामेवमाहोशना कविः}


\threelineshloka
{नारीणां बालवृद्धानां गोपतेश्च महामतिः}
{पापानां दुर्विनीतानां प्राणान्तं च बृहस्पतिः}
{दण्डमाह महाभाग सर्वेषामाततायिनाम्}


\threelineshloka
{सर्वेषां पापबुद्धीनां पापकर्मैव क्वुर्वताम्}
{धृतराष्ट्रस्य पुत्राणां दण्डो निर्दोष इष्यते}
{सौबलस्य च दुर्बुद्धेः कर्णस्य च दुरात्मनः}


\threelineshloka
{पश्यतां चैव शूराणां याऽहं द्यूते सभां तदा}
{रजस्वला समानीता भवतां पश्यतां नृप}
{वाससैकेन संवीता तव दोषेण भूपते}


\twolineshloka
{माभूद्धर्मविलोपस्ते धृतराष्ट्रकुलक्षयात्}
{क्रोधाग्निना तु दग्धं च सपशुद्रव्यसंचयम्}


\twolineshloka
{साऽहमेवंविधं दुःखं संप्राप्ता तव हेतुना}
{आदित्यस्य प्रसादेन न च प्राणैर्वियोजिता}


\twolineshloka
{रक्षिता देवदेवेन जगतः कालहेतुना}
{दिवाकरेण देवेन विवस्त्रा न कृता तदा'}


\twolineshloka
{एतेषां यतमानानां न मेऽद्य वचनं मृषा}
{त्वं तु सर्वां महीं त्यक्त्वा कुरुषे स्वयमापदम्}


\twolineshloka
{यथाऽऽस्तां संमतौ राज्ञां पृथिव्यां राजसत्तम्}
{मांधाता चाम्बरीषश्च तथा राजन्विराजसे}


\twolineshloka
{प्रशाधि पृथिवीं देवीं प्रजा धर्मेण पालयन्}
{सपर्वतवनद्वीपां मा राजन्विमना भव}


\twolineshloka
{यजस्व विविधैर्यज्ञैर्युध्यस्वारीन्प्रयच्छ च}
{धनानि भोगान्वासांसि द्विजातिभ्यो नृपोत्तम}


\chapter{अध्यायः १५}
\twolineshloka
{वैशंपायन उवाच}
{}


\threelineshloka
{याज्ञसेन्या वचः श्रुत्वा पुनरेवार्जुनोऽब्रवीत्}
{अनुमान्य महाबाहुं ज्येष्ठं भ्रातरमीश्वरम् ॥अर्जुन उवाच}
{}


\twolineshloka
{दण्डः शास्ति प्रजाः सर्वा दण्ड एवाभिरक्षति}
{दण़्डः सुप्तेषु जागर्ति दण्डं धर्मं विदुर्बुधाः}


\twolineshloka
{दण्डः संरक्षते धर्मं तथैवार्यं जनाधिप}
{कामं संरक्षते दण्डस्त्रिवर्गो दण्ड उच्यते}


\twolineshloka
{दण्डेन रक्ष्यते धान्यं धनं दण्डेन रक्ष्यते}
{एतद्विद्वानुपादाय स्वभावं पश्य लौकिकम्}


\twolineshloka
{राजदण्डभयादेके नराः पापं न कुर्वते}
{यमदण्डभयादेके परलोकभयादपि}


\twolineshloka
{परस्परभयादेके पापाः पापं न कुर्वते}
{एवं सांसिद्धिके लोके सर्वं दण्डे प्रतिष्ठितम्}


\twolineshloka
{दण्डस्यैव भयादेके न खादन्ति परस्परम्}
{अन्धेतमसि मज्जेयुर्यदि दण्डो न पालयेत्}


\twolineshloka
{यस्माददान्तान्दमयत्यशिष्टान्दण्डयत्यपि}
{दमनाद्दण्डनाच्चैव तस्माद्दण्डं विदुर्बुधाः}


\twolineshloka
{वाचि दण्डो ब्राह्मणानां क्षत्रियाणां भुजार्पणम्}
{धनदण्डाः स्मृता वैश्या निर्दण्डः शूद्र उच्यते}


\twolineshloka
{असंमोहाय मर्त्यानामर्थसंरक्षणाय च}
{मर्यादा स्थापिता लोके दण्डसंज्ञा विशांपते}


\twolineshloka
{यत्र श्यामो लोहिताक्षो दण्डश्चरति सूद्यतः}
{प्रजास्तत्र न मुह्यन्ति नेता चेत्साधु पश्यति}


\twolineshloka
{ब्रह्मचारी गृहस्थश्च वानप्रस्थस्च भिक्षुकः}
{दण्डस्यैव भयादेते मनुष्या वर्त्मनि स्थिताः}


\twolineshloka
{नाभीतो यजते राजन्नाभीतो दातुमिच्छति}
{नाभीतः पुरुषः कश्चित्समये स्थातुमिच्छति}


\twolineshloka
{नाच्छित्त्वा परमर्माणि नाकृत्वा कर्म दुष्करम्}
{नाहत्वा मत्स्यघातीव प्राप्नोति महतीं श्रियम्}


\threelineshloka
{नाघ्नतः कीर्तिरस्तीह न वित्तं न पुनः प्रजाः}
{इन्द्रो वृत्रवधेनैव महेन्द्रः समपद्यत}
{`माहेन्द्रं च गृहं लेभे लोकानां चेश्वरोऽभवत् ॥'}


\twolineshloka
{य एव देवा हन्तारस्ताँल्लोकोऽर्चयते भृशम्}
{हन्तारुद्रस्तथास्कन्दः शक्रोऽग्निर्वरुणो यमः}


\twolineshloka
{हन्ता कालस्तथा वायुर्मृत्युर्वैश्रवणो रविः}
{वसवो मरुतः साध्या विश्वेदेवाश्च भारत}


\twolineshloka
{एतान्देवान्नमस्यन्ति प्रतापप्रणता जनाः}
{न ब्रह्माणं न धातारं न पूषाणं कथंचन}


\twolineshloka
{मध्यस्थान्सर्वभूतेषु दान्ताञ्शमपरायणान्}
{यजन्ते मानवाः केचित्प्रशान्तान्सर्वकर्मसु}


\twolineshloka
{न हि पश्यामि जीवन्तं लोके कंचिदर्हिसया}
{सत्वैः सत्वा हि जीवन्ति दुर्बलैर्बलवत्तराः}


\twolineshloka
{नकुलो मूषिकानत्ति बिडालो नकुलं तथा}
{बिडालमत्ति श्वा राजञ्श्वानं व्यालमृगस्तथा}


\twolineshloka
{तानत्ति पुरुषः सर्वान्पश्य धर्मो यथा गतः}
{प्राणस्यान्नमिदं सर्वं जङ्गमं स्थावरं च यत्}


\twolineshloka
{विधानं दैवविहितं तत्र विद्वान्न मुह्यति}
{यथा सृष्टोऽसि राजेन्द्र तथा भवितुमर्हसि}


\twolineshloka
{विनीतक्रोधहर्षा हि मन्दा वनमुपाश्रिताः}
{विना वधं न कुर्वन्ति तापसाः प्राणयापनम्}


\twolineshloka
{उदके बहवः प्राणाः पृथिव्यां च फलेषु च}
{न च कश्चिन्न तान्हन्ति किमन्यत्प्राणयापनम्}


\twolineshloka
{सूक्ष्मयोनीनि भूतानि तर्कगम्यानि कानिचित्}
{पक्ष्मणोऽपि निपातेन येषां स्यात्स्कन्धपर्ययः}


\twolineshloka
{ग्रामान्निष्क्रम्य मुनयो विगतक्रोधमत्सराः}
{वने कुटुम्बधर्माणो दृश्यन्ते परिमोहिताः}


\twolineshloka
{भूमिं भित्त्वौषधीश्छित्त्वा वृक्षादीनण्डजान्पशून्}
{मनुष्यास्तन्वये यज्ञांस्ते स्वर्गं प्राप्नुवन्ति च}


\twolineshloka
{दण्डनीत्यां प्रणीतायां सर्वे सिध्यन्त्युपक्रमाः}
{कौन्तेय सर्वभूतानां तत्र मे नास्ति संशयः}


\twolineshloka
{दण्डश्चेन्न भवेल्लोके विनश्येयुरिमाः प्रजाः}
{जले मत्स्यानिवाभक्ष्यन्दुर्बलान्बलवत्तराः}


\twolineshloka
{सत्यं बतेदं ब्रह्मणा पूर्वमुक्तं दण्डः प्रजा रक्षति साधुनीतः}
{पश्याग्नयः पूतिमांसस्य भीताः सन्तर्जिता दण्डभयाज्ज्वलन्ति}


\twolineshloka
{अन्धंतम इवेदं स्यान्न प्रज्ञायेत किंचन}
{दण्डश्चेन्न भवेल्लोके विभजन्साध्वसाधुनी}


\twolineshloka
{येऽपि संभिन्नमर्यादा नास्तिका वेदनिन्दकाः}
{तेऽपि भोगाय कल्पन्ते दण्डेनाशु निपीडिताः}


\twolineshloka
{सर्वो दण्डजितो लोको दुर्लभो हि शुचिर्जनः}
{दण्डस्य हि भयाद्भीतो भोगायैव प्रकल्पते}


\twolineshloka
{चातुर्वर्ण्यप्रमोदाय सुनीतिकरणाय च}
{दण्डो विधात्रा विहितो धर्मार्थौं भुवि रक्षितुम्}


\twolineshloka
{यदि दण्डान्न विभ्येयुर्वयांसि श्वापदानि च}
{अद्युः पशून्मनुष्यांश्च यज्ञार्थानि हवींषि च}


\twolineshloka
{न ब्रह्मचार्यधीयीत न काल्यं दुहते च गौः}
{न कन्योद्वहनं गच्छेद्यदि दण्डो न पालयेत्}


\twolineshloka
{विष्वग्लोपः प्रवर्तेत भिद्येरन्सर्वसेतवः}
{ममत्वं न प्रजानीयुर्यदि दण्डो न पालयेत्}


\twolineshloka
{न संवत्सरसत्राणि तिष्ठेयुरकुतोभयाः}
{विधिवद्दक्षिणावन्ति यदि दण्डो न पालयेत्}


\twolineshloka
{चरेयुर्नाश्रमे धर्मं यथोक्तं विधिमाश्रिताः}
{न विद्यां प्राप्नुयात्कश्चिद्यदि दण्डो न पालयेत्}


\twolineshloka
{न चोष्ट्रा न बलीवर्दा नाश्वाश्वतरगर्दभाः}
{न विद्यां प्राप्नुर्यानानि यदि दण्डो न पालयेत्}


\twolineshloka
{न प्रेष्या वचनं कुर्युर्न बालो जातु कर्हिचित्}
{तिष्ठेत्पितुर्मते धर्मे यदि दण्डो न पालयेत्}


\twolineshloka
{दण्डे स्थिताः प्रजाः सर्वा भयं दण्डे विदुर्बुधाः}
{दण्डे स्वर्गो मनुष्याणां लोकोऽयं च प्रतिष्ठितः}


\twolineshloka
{न तत्र कूटं पापं वा वञ्चना वाऽपि दृश्यते}
{यत्र दण्डः सुविहितश्चरत्यरिविनाशनः}


\twolineshloka
{हविः श्वा प्रलिहेद्दृष्ट्वा दण्डश्चेन्नोद्यतो भवेत्}
{हरेत्काकः पुरोडाशं यदि दण्डो न पालयेत्}


\twolineshloka
{यदीदं धर्मतो राज्यं विहितं यद्यधर्मतः}
{कार्यस्तत्र न शोको वै भुङ्क्ष्व भोगान्यजस्व च}


\twolineshloka
{सुखेन धर्मं श्रीमन्तश्चरन्ति शुचिवाससः}
{संवसन्तः प्रियैर्दारैर्भुञ्जानाश्चान्नमुत्तमम्}


\twolineshloka
{अर्थे सर्वे समारम्भाः समायत्ता न संशयः}
{स च दण्डे समायत्तः पश्य दण्डस्य गौस्वम्}


\twolineshloka
{लोकयात्रार्थमेवेह धर्मप्रवचनं कृतम्}
{अहिंसाऽसाधुहिंसेति श्रेयान्धर्मपरिग्रहः}


\twolineshloka
{नात्यन्तं गुणवत्किंचिन्न चाप्यत्यन्तनिर्गुणम्}
{उभयं सर्वकार्येषु दृश्यते साध्वसाधु च}


\twolineshloka
{पशूनां वृषणं छित्त्वा ततो भिन्दन्ति नस्सु तान्}
{वहन्ति बहवो भारान्बध्नन्ति दमयन्ति च}


\twolineshloka
{एवं पर्याकुले लोके वितथैर्जर्झरीकृते}
{तैस्तैर्न्यायैर्महाराज पुराणं धर्ममाचर}


\twolineshloka
{यज देहि प्रजा रक्ष धर्मं समनुपालय}
{अमित्राञ्जहि कौन्तेय मित्राणि परिपालय}


\twolineshloka
{मा च ते निघ्नतः शत्रून्मन्युर्भवतु पार्थिव}
{न तत्र किल्विषं किंचिद्धन्तुर्भवति भारत}


\twolineshloka
{आततायी हि यो हन्यादाततायिनमागतम्}
{न तेन भ्रूणहा स स्यान्मन्युस्तं मन्युमार्च्छति}


\twolineshloka
{अवध्यः सर्वभूतानामन्तरात्मा न संशयः}
{अवध्ये चात्मनि कथं वध्यो भवति कस्यचित्}


\twolineshloka
{यथा हि पुरुषः शालां पुनः संप्रविशेन्नवाम्}
{एव जीवः शरीराणि तानितानि प्रपद्यते}


\twolineshloka
{देहान्पुराणानुत्सृज्य नवान्संप्रतिपद्यते}
{एवं मृत्युमुखं प्राहुर्जना ये तत्त्वदर्शिनः}


\chapter{अध्यायः १६}
\twolineshloka
{वैशंपायन उवाच}
{}


\twolineshloka
{अर्जुनस्य वचः श्रुत्वा भीमसेनोऽत्यमर्षणः}
{धैर्यमास्थाय तं ज्येष्ठं भ्राता भ्रातरमब्रवीत्}


\twolineshloka
{राजन्विदितधर्मोऽसि न तेऽस्त्यविदितं भुवि}
{उपशिक्षाम ते वृत्तं सदैव न च शक्नुमः}


\twolineshloka
{न वक्ष्यामि न वक्ष्यामीत्येवं मे मनसि स्थितम्}
{अतिदुःखात्तु वक्ष्यामि तन्निबोध जनाधिप}


\twolineshloka
{भवतः संप्रमोहेन सर्वं संशयितं कृतम्}
{विक्लबत्वं च नः प्राप्तमबलत्वं तथैव च}


\twolineshloka
{कथं हि राजा लोकस्य सर्वशास्त्रविशारदः}
{मोहमापद्यसे दैन्याद्यथा कापुरुषस्तथा}


\twolineshloka
{आगतिश्च गतिश्चैव लोकस्य विदिता तव}
{आयत्यां च तदात्वे च न तेऽस्त्यविदितं प्रभो}


\twolineshloka
{एवं गते महाराज राज्यं प्रति जनाधिप}
{हेतुमात्रं तु वक्ष्यामि तमिहैकमनाः श्रृणु}


\twolineshloka
{द्विविधो जायते व्याधिः शारीरो मानसस्तथा}
{परस्परं तयोर्जन्म निर्द्वन्द्वं नोपलभ्यते}


\twolineshloka
{शारीराज्जायते व्याधिर्मानसो नात्र संशयः}
{मानसाज्जायते व्याधिः शारीर इति निश्चयः}


\twolineshloka
{शारीरमानसे दुःखे योऽतीते त्वनुशोचति}
{दुःखेन लभते दुःखं द्वावनर्थौ च विन्दति}


\twolineshloka
{शीतोष्णे चैव वायुश्च त्रयः शारीरजा गुणाः}
{तेषां गुणानां साम्यं यत्तदाहुः स्वस्थलक्षणम्}


\threelineshloka
{तेषामन्यतमोद्रेके विधानमुपदिश्यते}
{उष्णेन बाध्यते शीतं शीतेनोष्णं प्रबाध्यते}
{`उभाभ्यां बाध्यते वायुर्विधानमिदमुच्यते ॥'}


\twolineshloka
{सत्वं रजस्तय इति मानसाः स्युस्त्रयो गुणाः}
{तेषां गुणानां साम्यं यत्तदाहुः स्वस्थलक्षणम्}


\threelineshloka
{तेषामन्यतमोद्रेके विधानमुपदिश्यते}
{हर्षेण बाध्यते शोको हर्षः शोकेन बाध्यते}
{`उभाभ्यां बाध्यते मोहो विधानमिदमुच्यते ॥'}


\twolineshloka
{कश्चित्सुखे वर्तमानो दुःखस्य स्मर्तुमिच्छति}
{कश्चिद्दुःखे वर्तमानः सुखस्य स्मर्तुमिच्छति}


\threelineshloka
{स त्वं न दुःखी दुःखस्य न सुखी च सुखस्य च}
{नादुःखी दुःखभागस्य नासुखी च सुखस्य च}
{स्मर्तुमर्हसि कौरव्य दिष्टं हि बलवत्तरम्}


\threelineshloka
{अथवा ते स्वभावोऽयं येन पार्थिव तुष्यसे}
{दृष्ट्वा सभागतां कृष्णामेकवस्त्रां रजस्वलाम्}
{मिषतां पाण्डुपुत्राणां न तस्य स्मर्तुमर्हसि}


\twolineshloka
{प्रव्राजनं च नगरादजिनैश्च विवासनम्}
{महारण्यनिवासश्च न तस्य स्मर्तुमर्हसि}


\twolineshloka
{जटासुरात्परिक्लेशं चित्रसेनेन चाहवम्}
{सैन्धवाच्च परिक्लेशं कथं विस्मृतवानसि}


\twolineshloka
{पुनरज्ञातचर्यायां कीचकेन पदा वधम्}
{द्रौपद्या राजपुत्र्यांश्च कथं विस्मृतवानसि}


\twolineshloka
{यच्च ते द्रोणभीष्माभ्यां युद्धमासीदरिंदम्}
{मनसैकेन योद्धव्यं तत्ते युद्धमुपस्थितम्}


\twolineshloka
{यत्र नास्ति शरैः कार्यं न मित्रैर्न च बन्धुभिः}
{आत्मनैकेन योद्धव्यं तत्ते युद्धमुपस्थितम्}


\twolineshloka
{तस्मिन्ननिर्जिते युद्धे प्राणान्यदि विमोक्ष्यसे}
{अन्यं देहं समास्थाय ततस्तैरिह योत्स्यसे}


\twolineshloka
{`यो ह्यनाढ्यः स पतितस्तदुच्छिष्टं तदल्पकम्}
{बह्वपथ्यं बलवतो न किंचित्रायते बलम् ॥'}


\twolineshloka
{तस्मादद्यैव गन्तव्यं युध्यस्व भरतर्षभ}
{परमव्यक्तरूपस्य व्यक्तं त्यक्त्वा स्वकर्मभिः}


\twolineshloka
{तस्मिन्ननिर्जिते युद्धे कामवस्थां गमिष्यसि}
{एतज्जित्वा महाराज कृतकृत्यो भविष्यसि}


\twolineshloka
{एतां बुद्धिं विनिश्चित्य भूतानामागतिं गतिम्}
{पितृपैतामहे वृत्ते शाधि राज्यं यथोचितम्}


\twolineshloka
{दिष्ट्या दुर्योधनः पापो निहतः सानुगो युधि}
{द्रौपद्याः केशपक्षस्य दिष्ट्या ते पदवीं गताः}


\twolineshloka
{यजस्व वाजिमेधेन विधिवद्दक्षिणावता}
{वयं ते किंकराः पार्थ वासुदेवश्च वीर्यवान्}


\chapter{अध्यायः १७}
\twolineshloka
{युधिष्ठिर उवाच}
{}


\twolineshloka
{असंतोषः प्रमादश्च मदो रागोऽप्रशान्तता}
{बलं मोहोऽभिमानश्चाप्युद्वेगश्चैव सर्वशः}


\twolineshloka
{एभिः पाप्मभिराविष्टो राज्यं त्वमभिकाङ्क्षसे}
{निरामिषो विनिर्मुक्तः प्रशान्तः सुसुखी भव}


\twolineshloka
{य इमामखिलां भूमिं शिष्यादेको महीपतिः}
{तस्याप्युदरमेकं वै किमिदं त्वं प्रशंससि}


\twolineshloka
{नाह्ना पूरयितुं शक्यां न मासैर्भरतर्षभ}
{अपूर्यां पूरयन्निच्छामायुषाऽपि न शक्नुयात्}


\twolineshloka
{यथेद्धः प्रज्वलत्यग्निरसमिद्धः प्रशाम्यति}
{अल्पाहारतयाग्निं त्वं शमयौदर्यमुत्थितम्}


\twolineshloka
{आत्मोदरकृतेऽप्राज्ञः करोति विशसं बहु}
{जयोदरं पृथिव्या ते श्रेयो निर्जितया जितम्}


\twolineshloka
{मानुषान्कामभोगांस्त्वमैश्वर्यं च प्रशंससि}
{अभोगिनोऽबलाश्चैव यान्ति स्थानमनुत्तमम्}


\twolineshloka
{योगः क्षेमश्च राष्ट्रस्य धर्माधर्मौ त्वयि स्थितौ}
{मुच्यस्व महतो भारात्त्यागमेवाभिसंश्रय}


\twolineshloka
{एकोदरकृते व्याघ्रः करोति विशसं बहु}
{तमन्येऽप्युपजीवन्ति मन्दवेगतरा मृगाः}


\twolineshloka
{विषयान्प्रतिसंगृह्य संन्यासे कुरुते मतिम्}
{न च तुष्यन्ति राजानः पश्य बुद्ध्यन्तरं यथा}


\twolineshloka
{पत्राहारैरश्मकुट्टैर्दन्तोलूखलिकैस्तथा}
{अब्भक्षैर्वायुभक्षैश्च तेरयं नरको जितः}


\twolineshloka
{यस्त्विमां वसुधां कृत्स्नां प्रशासेदखिलां नृपः}
{तुल्याश्मकाञ्चनो यश्च स कृतार्थो न पार्थिवः}


\twolineshloka
{संकल्पेषु निरारम्भो निराशीर्निर्ममो भव}
{अशोकं स्थानमातिष्ठ इह चामुत्र चाव्ययम्}


\twolineshloka
{निरामिषा न शोचन्ति शोचन्ति त्वामिषैषिणः}
{परित्यज्यामिषं सर्वं मृषावादात्प्रमोक्ष्यसे}


\twolineshloka
{पन्थानौ पितृयानश्च देवयानश्च विश्रुतौ}
{ईजानाः पितृयानेन देवयानेन मोक्षिणः}


\twolineshloka
{तपसा ब्रह्मर्येण स्वाध्यायेन महर्षयः}
{विमुच्य देहांस्ते यान्ति मृत्योरविषयं गताः}


\twolineshloka
{आमिषं बन्धनं लोके कर्मेहोक्तं तथाऽऽमिषम्}
{ताभ्यां विमुक्तः पापाभ्यां पदमाप्नोति तत्परम्}


\twolineshloka
{अपि गाथां पुरा गीतां जनकेन वदन्त्युत}
{निर्द्वन्द्वेन विमुक्तेन मोक्षं समनुपश्यता}


\twolineshloka
{अनन्तं बत मे वित्तं यस्य मे नास्ति किंचन}
{मिथिलायां प्रदीप्तायां न मे किंचित्प्रदह्यते}


\twolineshloka
{प्रज्ञाप्रासादमारुह्य न शोचेच्छोचतो जनान्}
{जगतीस्थोऽथवाऽद्रिस्थो मन्दबुद्धिर्नचेक्षते}


\twolineshloka
{दृश्यं पश्यति यः पश्यन्स चक्षुष्मान्स बुद्धिमान्}
{अज्ञातानां च विज्ञानात्संबोधाद्रुद्धिरुच्यते}


\twolineshloka
{यस्तु मानं विजानाति बहुमानमियात्स वै}
{ब्रह्मभावप्रभूतानां वैद्यानां भावितात्मनाम्}


\twolineshloka
{यदा भूतपृथग्भावमेकस्थमनुपश्यति}
{तत एव च विस्तारं ब्रह्म संपद्यते तदा}


\twolineshloka
{ते जनास्तां गतिं यान्ति नाविद्वांसोऽल्पचेतसः}
{नाबुद्धयो नातपसः सर्वं बुद्धौ प्रतिष्ठितम्}


\chapter{अध्यायः १८}
\twolineshloka
{वैशंपायन उवाच}
{}


\threelineshloka
{तूष्णींभूतं तु राजानं पुनरेवार्जुनोऽब्रवीत्}
{संतप्तः शोकदुःखाभ्यां राजवाक्शल्यपीडितः ॥अर्जुन उवाच}
{}


\twolineshloka
{कथयन्ति पुरावृत्तमितिहासमिमं जनाः}
{विदेहराज्ञः संवादं भार्यया सह भारत}


\twolineshloka
{उत्सृज्य राज्यं भिक्षार्थं कृतबुद्धिं नरेश्वरम्}
{विदेहराजमहीषी दुःखिता प्रत्यभाषत}


\twolineshloka
{धनान्यपत्यं मित्राणि रत्नानि विविधानि च}
{पन्थानं पावनं हित्वा जनको मौढ्यमास्थितः}


\twolineshloka
{तं ददर्श प्रिया भार्या भैक्षवृत्तिमकिंचनम्}
{धानामुष्टिमुपासीनं निरीहं गतमत्सरम्}


\twolineshloka
{तमुवाच समामत्य भर्तारमकुतोभयम्}
{क्रुद्धा मनस्विनी भार्या विविक्ते हेतुमद्वचः}


\twolineshloka
{कथमुत्सृज्य राज्यं स्वं धनधान्यसमन्वितम्}
{कापालीं वृत्तिमास्थाय धान्यमुष्टिमुपाससे}


\twolineshloka
{प्रतिज्ञा तेऽन्यथा राजन्विचेष्टा चान्यथा तव}
{यद्राज्यं महदुत्सृज्य स्वल्पे लुभ्यसि पार्थिव}


\twolineshloka
{नैतेनातिथयो राजन्देवर्षिपितरस्तथा}
{अद्य शक्यास्त्वया भर्तुं मोघस्तेऽयं परिश्रमः}


\twolineshloka
{देवतातिथिभिश्चैव पितृभिश्चैव पार्थिव}
{सर्वैरेतैः परित्यक्तः परिव्रजसि निष्क्रियः}


\twolineshloka
{यस्त्वं त्रैविद्यवृद्धानां ब्राह्मणानां सहस्रशः}
{भर्ता भूत्वा च लोकस्य सोऽद्यान्यैर्भूतिमिच्छसि}


\twolineshloka
{श्रियं हित्वा प्रदीप्तां त्वं श्ववत्संप्रति वीक्ष्यसे}
{अपुत्रा जननी तेऽद्य कौसल्या चापतिस्त्वया}


\twolineshloka
{आश्रिता धर्मकामास्त्वां क्षत्रियाः पर्युपासते}
{त्वदाशामभिकाङ्क्षन्तः कृपणाः फलहेतुकाः}


\twolineshloka
{तांश्च त्वं विफलान्कृत्वा कं नु लोकं गमिष्यसि}
{राजन्संशयिते मोक्षे परतन्त्रेषु देहिषु}


\twolineshloka
{नैव तेऽस्ति परो लोको नापरः पापकर्मणः}
{धर्म्यान्दारान्परित्यज्य यस्त्वमिच्छसि जीवितुम्}


\twolineshloka
{स्रजो गन्धानलंकारान्वासांसि विविधानि च}
{किमर्थमभिसंत्यज्य परिव्रजसि निष्क्रियः}


\twolineshloka
{निपानं सर्वभूतानां भूत्वा त्वं पावनं महत}
{आढ्यो वनस्पतिर्भूत्वा सोन्यांस्त्वं पर्युपाससे}


\twolineshloka
{खादन्ति हस्तिनं न्यासे क्रव्यादा बहवोऽप्युत}
{बहवः कृमयश्चैव किं पुनस्त्वामनर्थकम्}


\twolineshloka
{य इमां कुण्डिकां भिन्द्यान्त्रिविष्टब्धं च यो हरेत्}
{वासश्चापि हरेत्तस्मिन्कथं ते मानसं भवेत्}


\twolineshloka
{यस्त्वं सर्वं समुत्सृज्य धानामुष्टिमनुग्रहः}
{यदनेन कृतं सर्वं किमिदं मम दीयते}


\twolineshloka
{धानामुष्टेरिहार्थश्चेत्प्रतिज्ञा ते विनश्यति}
{का वाऽहं तव को मे त्वं कश्च ते मय्यनुग्रहः}


\twolineshloka
{प्रशाधि पृथिवीं राजन्यत्र तेऽनुग्रहो भवेत्}
{प्रासादे शयनं यानं वासांस्याभरणानि च}


\twolineshloka
{श्रियां निराशैरधनेस्त्यक्तमित्रैरकिंचनैः}
{सौखिकैः संभृतो योऽर्थः स संत्यजति किंनु तं}


\twolineshloka
{योऽत्यन्तं प्रतिगृह्णीयाद्यश्च दद्यात्सदैव हि}
{तयोस्त्वमन्तरं विद्धि श्रेयांस्ताभ्यां क उच्यते}


\twolineshloka
{सदैव याचमानेषु तथा दम्भान्वितेषु च}
{एतेषु दक्षिणा दत्ता दावाग्राविव दुर्हुतम्}


\twolineshloka
{जातवेदा यथा राजन्नादग्ध्वैवोपशाम्यति}
{सदैव याचमानो वै तथा शाम्यति न द्विजः}


\twolineshloka
{सतां वै ददतोऽन्नं च लोकेऽस्मिन्प्रकृतिर्ध्रुवा}
{न चेद्राजा भवेद्दाता कुतः स्युर्मोक्षकाङ्क्षिणः}


\twolineshloka
{अन्नाद्गृहस्था लोकेऽस्मिन्भिक्षवस्तत एव च}
{अन्नात्प्राणः प्रभवति अन्नदः प्राणदो भवेत्}


\twolineshloka
{गृहस्थेभ्योऽपि निर्मुक्ता गृहस्थानेव संश्रिताः}
{प्रभवं च प्रतिष्ठां च दान्ता विन्दन्त आसते}


\twolineshloka
{त्यागान्न भिक्षुकं विन्द्यान्न मौढ्यान्न च याचनात्}
{ऋजुस्तु योऽर्थं त्यजति तं मुक्तं विद्धि भिक्षुकम्}


\twolineshloka
{असक्तः शक्तवद्गच्छन्निः सङ्गो मुक्तबन्धनः}
{समः शत्रौ च मित्रे च स वै मुक्तो महीपते}


\twolineshloka
{परिव्रजन्ति दानार्थं मुण्डाः काषायवाससः}
{सिता बहुविधैः पाशैः संचिन्वन्तो वृथामिषम्}


\twolineshloka
{त्रयीं च नामवार्तां च त्यक्त्वा पुत्रान्व्रजन्ति ये}
{त्रिविष्टब्धं च वासश्च प्रतिगृह्णन्त्यबुद्धयः}


\twolineshloka
{अनिष्कषाये काषायमीहार्थमिति विद्धि तम््}
{धर्मध्वजानां मुण्डानां वृत्त्यर्थमिति मे मतिः}


\twolineshloka
{काषायैरजिनैश्चीरैर्नग्नान्मुण्डाञ्जटाधरान्}
{बिभ्रत्साधून्महाराज जय लोकाञ्जितेन्द्रियः}


\threelineshloka
{अग्न्याधेयानि गुर्वर्थं क्रतूनपि सुदक्षिणान्}
{ददात्यहरहः पूर्वं को नु धर्मरतस्ततः ॥अर्जुन उवाच}
{}


\twolineshloka
{तत्त्वज्ञो जनको राजा लोकेऽस्मिन्निति गीयते}
{सोऽप्यासीन्मोहसंपन्नो मा मोहवशमन्वगाः}


\twolineshloka
{एवं धर्ममनुक्रान्ता सदा दानतपः पराः}
{आनृशंस्यगुणोपेताः कामक्रोधविवर्जिताः}


\twolineshloka
{प्रजानां पालने युक्ता दममुत्तममास्थिताः}
{इष्ट्वा लोकानवाप्स्यामो गुरुवृद्धोपचायिनः}


\twolineshloka
{देवतातिथिभूतानां निर्वपन्तो यथाविधि}
{स्थानमिष्टमवाप्स्यामो ब्रह्मण्याः सत्यवादिनः}


\chapter{अध्यायः १९}
\twolineshloka
{युधिष्ठिर उवाच}
{}


\twolineshloka
{वेदाहं तात शास्त्राणि अपराणि पराणि च}
{उभयं वेदवचनं कुरु कर्म त्यजेति च}


\twolineshloka
{आकुलानि च शास्त्राणि हेतुभिश्चिन्तितानि च}
{निश्चयश्चैव यो मन्त्रे वेदाहं तं यथाविधि}


\twolineshloka
{त्वं तु केवलशास्त्रज्ञो वीरव्रतसमन्वितः}
{शास्त्रार्थं तत्त्वतो गन्तुं न समर्थः कथंचन}


\twolineshloka
{शास्त्रार्थतत्त्वदर्शी यो धर्मनिश्चयकोविदः}
{तेनाप्येवं न वाच्योऽयं यदि धर्मं प्रपश्यसि}


\twolineshloka
{भ्रातृसौहृदमास्थाय यदुक्तं वचनं त्वया}
{न्याय्यं युक्तं च कौन्तेय प्रीतोऽहं तेन तेऽर्जुन}


\twolineshloka
{`महेश्वरसमं सत्वं ब्रह्मणा चैव यत्समम्}
{वासुदेवसमं चैव न भूतं न भविष्यति}


% Check verse!
तथा त्वं योधमुख्येषु सत्वं परममिष्यते
\twolineshloka
{बलमिन्द्रे च वायौ च बलं यच्च जनार्दने}
{तद्वलं भीमसेने च त्वयि चार्जुना विद्यते}


\twolineshloka
{त्वत्समश्चित्रयोधी च दूरपाती च पाण्डव}
{दिव्यास्त्रबलसंपन्नः को वाऽन्यस्त्वत्समो नरः}


\twolineshloka
{युद्धधर्मेषु सर्वेषु क्रियाणां नैपुणेषु च}
{न त्वया सदृशः कश्चिन्त्रिषु लोकेषु विद्यते}


\threelineshloka
{धार्मिकं धर्मयुक्तं च निःशेषं ज्ञायते मया}
{धर्मसूक्ष्मं तु यद्वाच्यं तत्र दुष्प्रतरं त्वया}
{धनञ्जय न मे बुद्धिमतिशङ्कितुमर्हसि}


\twolineshloka
{युद्धशास्त्रविदेव त्वं न वृद्धाः सेवितास्त्वया}
{समासविस्तरविदां न तेषां वेत्सि निश्चयम्}


\twolineshloka
{तपस्त्यागो विधिरिति निश्चयस्तात धीमताम्}
{परस्परं ज्याय एषामिति नः श्रेयसी मतिः}


\twolineshloka
{यत्त्वेतन्मन्यसे पार्थ न ज्यायोऽस्ति धनादिति}
{तत्र ते वर्तयिष्यामि यथा नैतत्प्रधानतः}


\twolineshloka
{तपः स्वाध्यायशीला हि दृश्यन्ते धार्मिका जनाः}
{ऋषयस्तपसा युक्ता येषां लोकाः सनातनाः}


\twolineshloka
{अजातश्मश्रवो धीरास्तथाऽन्ये वनवासिनः}
{अरुणाः केतवश्चैव स्वाध्यायेन दिवं गताः}


\twolineshloka
{उत्तरेण तु पन्थानमार्या विषयनिग्रहात्}
{अबुद्धिजं तमस्त्यक्त्वा लोकांस्त्यागवतां गताः}


\twolineshloka
{दक्षिणेन तु पन्थानं यं भास्वन्तं प्रचक्षते}
{एते क्रियावतां लोका ये श्मशानानि भेजिरे}


\twolineshloka
{अनिर्देश्या गतिः सा तु यां प्रपश्यन्ति मोक्षिणः}
{तस्मात्त्यागः प्रधानेष्टः स तु दुःखं प्रवेदितुम्}


\twolineshloka
{अनुस्मृत्य तु शास्त्राणि कवयः समवस्थिताः}
{अपीह स्यादपीह स्यात्सारासारदिदृक्षया}


\twolineshloka
{वेदवादानतिक्रम्य शास्त्राण्यारण्यकानि च}
{विपाट्य कदलीस्तम्भं सारं ददृशिरे न ते}


\twolineshloka
{अथैकान्तव्युदासेन शरीरे पाञ्चभौतिके}
{इच्छाद्वेषसमायुक्तमात्मानं प्राहुरिङ्गितैः}


\twolineshloka
{अग्राह्यं चक्षुषा सत्वमनिर्देश्यं च तद्गिरा}
{कर्महेतुपुरस्कारं भूतेषु पस्विर्तते}


\twolineshloka
{कल्याणगोचरं कृत्वा मानं तृष्णां निगृह्य च}
{कर्मसंततिमुत्सृज्य स्यान्निरालम्बनः सुखी}


\twolineshloka
{अस्मिन्नेवं सूक्ष्मगम्ये मार्गे सद्भिर्निषेविते}
{कथमर्थमनर्थाढ्यमर्जुन त्वं प्रशंससि}


\twolineshloka
{पूर्वशास्त्रविदोऽप्येवं जनाः पश्यन्ति भारत}
{क्रियासु निरता नित्यं दाने यज्ञे च कर्मणि}


\twolineshloka
{भवन्ति सुदुरावर्ता हेतुमन्तोऽपि पण्डिताः}
{दृढपूर्वे स्मृता मूढा नैतदस्तीति वादिनः}


\twolineshloka
{अनृतस्यावमन्तारो वक्तारो जनसंसदि}
{चरन्ति वसुधां कृत्स्नां वावदूका बहुश्रुताः}


\twolineshloka
{पार्थ यन्न विजानीमः कस्ताञ्ज्ञातुमिहार्हति}
{एवं प्राज्ञाः श्रुताश्चापि महान्तः शास्त्रवित्तमाः}


\twolineshloka
{तपसा महदाप्नोति बुद्ध्या वै विन्दते महत्}
{त्यागेन सुखमाप्नोति सदा कौन्तेय धर्मवित्}


\chapter{अध्यायः २०}
\twolineshloka
{वैशंपायन उवाच}
{}


\threelineshloka
{अस्मिन्वाक्यान्तरे वक्ता देवस्थानो महातपाः}
{अभिनीततरं वाक्यमित्युवाच युधिष्ठिरम् ॥देवस्थान उवाच}
{}


\twolineshloka
{यद्वचः फल्गुनेनोक्तं न ज्यायोऽस्ति धनादिति}
{अत्र ते वर्तयिष्यामि तदेकान्तमनाः शृणु}


\twolineshloka
{अजातशत्रो धर्मेण कृत्स्ना ते वसुधा जिता}
{तां जित्वा च वृथा राजन्न परित्यक्तुमर्हसि}


\threelineshloka
{चतुष्पदी हि निःश्रेणी ब्रह्मण्येषा प्रतिष्ठिता}
{तां क्रमेण महाबाहो यथावज्जय पार्थिव}
{तस्मात्पार्थ महायज्ञैर्यजस्व बहुदक्षिणैः}


\twolineshloka
{स्वाध्याययज्ञा ऋषयो ज्ञानयज्ञास्तथाऽपरे}
{कर्मनिष्ठाश्च बुद्ध्यर्थास्तपोनिष्ठाश्च पार्थिव}


% Check verse!
वैखानसानां कौन्तेय वचनं श्रूयते यथा
\twolineshloka
{ईहेत धनहेतोर्यस्तस्यानीहा गरीयसी}
{भूयान्दोषो हि वर्धेत यस्तत्कर्म समाश्रयेत्}


\twolineshloka
{कृत्स्नं च धनसंहारं कुर्वन्ति विधिकारणात्}
{आत्मना तृपितो बुद्ध्या भ्रूणहत्यां न बुध्यते}


\twolineshloka
{अनर्हते यद्ददाति न ददाति यदर्हते}
{अर्हानर्हापरिज्ञानाद्दानधर्मोऽपि दुष्करः}


\twolineshloka
{यज्ञाय सृष्टानि धनानि धात्रायज्ञादिष्टः पुरुषो रक्षिता च}
{तस्मात्सर्वं यज्ञ एवोपयोज्यंधनं ततोऽनन्तर एव कामः}


\twolineshloka
{यज्ञैरिन्द्रो विविधै रत्नवद्भिर्देवान्सर्वानभ्ययाद्भूरितेजाः}
{तेनेन्द्रत्वं प्राप्य विभ्राजतेऽसौतस्माद्यज्ञे सर्वमेवोपयोज्यम्}


\twolineshloka
{महादेवः सर्वयज्ञे महात्माहुत्वाऽऽत्मानं देवदेवो बभूव}
{विश्वाँल्लोकान्व्याप्य विष्टभ्य कीर्त्याविराजते द्युतिमान्कृत्तिवासाः}


\twolineshloka
{आविक्षितः पार्थिवोऽसौ मरुत्तोवृद्ध्या शक्रं योऽजयद्देवराजम्}
{यज्ञे यस्य श्रीः स्वयं सन्निविष्टायस्मिन्भाण्डं काञ्चनं सर्वमासीत्}


\twolineshloka
{हरिश्चन्द्रः पार्थिवेन्द्रः श्रुतस्तेयज्ञैरिष्ट्वा पुण्यभाग्वीतशोकः}
{ऋद्ध्या शक्रं योऽजयन्मानुषः संस्तस्माद्यज्ञे सर्वमेवोपयोज्यम्}


\chapter{अध्यायः २१}
\twolineshloka
{देवस्थान उवाच}
{}


\twolineshloka
{अत्रैवोदाहरन्तीममितिहासं पुरातनम्}
{इन्द्रेण समये पृष्टो यदुवाच बृहस्पतिः}


\twolineshloka
{संतोषो वै स्वर्गसमः संतोषः परमं सुखम्}
{तुष्टेर्न किंचित्परतः सा सम्यक्प्रतितिष्ठति}


\twolineshloka
{यदा संहरते कामान्कूर्मोऽङ्गानीव सर्वशः}
{तदाऽऽत्मज्योतिरचिरात्स्वात्मन्येव प्रसीदति}


\twolineshloka
{न विभेति यदा चायं यदा चास्मान्न बिभ्यति}
{कामद्वेषौ च जयति तदाऽऽत्मानं च पश्यति}


\twolineshloka
{यदाऽसौ सर्वभूतानां न द्रुह्यति न काङ्क्षति}
{कर्मणा मनसा वाचा ब्रह्म संपद्यते तदा}


\twolineshloka
{एवं कौन्तेय भूतानि तंतं धर्मं तथातथा}
{तदाऽऽत्मना प्रपश्यन्ति तस्माद्वुध्यस्व भारत}


\twolineshloka
{अन्ये साम प्रशंसन्ति व्यायाममपरे जनाः}
{नैकं न चापरं केचिदुभयं च तथाऽपरे}


\twolineshloka
{यज्ञमेके प्रशंसन्ति संन्यासमपरे जनाः}
{`नैकं न चापरं केचिदुभयं च तथाऽपरे ॥'}


\twolineshloka
{दानमेके प्रशंसन्ति केचिच्चैव प्रतिग्रहम्}
{केचित्सर्वं परित्यज्य तूष्णीं ध्यायन्त आसते}


\twolineshloka
{राज्यमेके प्रशंसन्ति प्रजानां परिपालनम्}
{हत्वा छित्त्वा च भित्त्वा च केचिदेकान्तशीलिनः}


\twolineshloka
{एतत्सर्वं समालोक्य बुधानामेव निश्चयः}
{अद्रोहेणैव भूतानां यो धर्मः स सतां मतः}


\twolineshloka
{अद्रोहः सत्यवचनं संविभागो दया दमः}
{प्रजनं स्वेषु दारेषु मार्दवं हीरचापलम्}


\twolineshloka
{एवं धर्मं प्रधानेष्टं मनुः स्वायंभुवोऽब्रवीत्}
{तस्मादेतत्प्रयत्नेन कौन्तेय प्रतिपालय}


\twolineshloka
{यो हि राज्ये स्थितः शश्वद्वशी तुल्यप्रियाप्रियः}
{क्षत्रियो यज्ञशिष्टाशी राजा शास्त्रार्थतत्त्ववित्}


\twolineshloka
{असाधुनिग्रहरतः साधूनां प्रग्रहे रतः}
{धर्मवर्त्मनि संस्थाप्य प्रजा वर्तेत धर्मतः}


\twolineshloka
{पुत्रसंक्रामितश्रीश्च वने वन्येन वर्तयेत्}
{विधानमाश्रमाणां वै कुर्यात्कर्माण्यतन्द्रितः}


\twolineshloka
{य एवं वर्तते राजन्स राजा धर्मनिश्चितः}
{तस्यायं च परश्चैव लोकः स्यात्संफलोदयः}


% Check verse!
निर्वाणं हि सुदुष्प्राप्यं बहुविघ्नं च मे मतम्
\twolineshloka
{एवं धर्ममनुक्रान्ताः सत्यदानतपः पराः}
{आनृशंस्यगुणैर्युक्ताः कामक्रोधविवर्जिताः}


\twolineshloka
{प्रजानां पालने युक्ता धर्ममुत्तममास्थिताः}
{गोब्राह्मणार्थे युध्यन्तः प्राप्ता गतिमनुत्तमाम्}


\threelineshloka
{एवं रुद्राः सवसवस्तथाऽऽदित्याः परंतप}
{साध्या राजर्षिसङ्घाश्च धर्ममेतं समाश्रिताः}
{अप्रमत्तास्ततः स्वर्गं प्राप्ताः पुण्यैः स्वकर्मभिः}


\chapter{अध्यायः २२}
\twolineshloka
{वैशंपायन उवाच}
{}


\twolineshloka
{अस्मिन्नेवान्तरे वाक्यं पुनरेवार्जुनोऽब्रवीत्}
{निर्विण्णमनसं ज्येष्ठमिदं भ्रातरमच्युतम्}


\twolineshloka
{क्षत्रधर्मेण धर्मज्ञ प्राप्य राज्यं सुदुर्लभम्}
{जित्वा चारीन्नरश्रेष्ठ तप्यते किं भृशं भवान्}


\twolineshloka
{क्षत्रियाणां महाराज संग्रामे निधनं मतम्}
{विशिष्टं बहुभिर्यज्ञैः क्षत्रधर्ममनुस्मर}


\twolineshloka
{ब्राह्मणानां तपस्त्यागः प्रेत्य धर्मविधिः स्मृतः}
{क्षत्रियाणां च निधनं संग्रामे विहितं प्रभो}


\twolineshloka
{क्षात्रधर्मो महारौद्रः शस्त्रनित्य इति स्मृतः}
{वधश्च भरतश्रेष्ठ काले शस्त्रेण संयुगे}


\twolineshloka
{ब्राह्मणस्यापि चेद्राजन्क्षत्रधर्मेण वर्ततः}
{प्रशस्तं जीवितं लोके क्षत्रं हि ब्रह्मसंभवम्}


\twolineshloka
{न त्यागो न पुनर्यज्ञो न तपो मनुजेश्वर}
{क्षत्रियस्य विधीयन्ते न परस्वोपजीवनम्}


\twolineshloka
{स भवान्सर्वधर्मज्ञो धर्मात्मा भरतर्षभ}
{राजा मनीषी निपुणो लोके दृष्टपरावरः}


\twolineshloka
{त्यक्त्वा संतापजं शोकं दंशितो भव कर्मणि}
{क्षत्रियस्य विशेषेण हृदयं वज्रसन्निभम्}


\twolineshloka
{जित्वाऽरीन्क्षत्रधर्मेण प्राप्य राज्यमकण्टकम्}
{विजितात्मा मनुष्येन्द्र यज्ञदानपरो भव}


\twolineshloka
{इन्द्रो वै ब्रह्मणः पुत्रः क्षत्रियः कर्मणाऽभवत्}
{ज्ञातीनां पापवृत्तीनां जघान नवतीर्नव}


\twolineshloka
{तच्चास्य कर्म पूज्यं च प्रशस्यं च विशांपते}
{तेनेन्द्रत्वं समापेदे देवानामिति नः श्रुतम्}


\twolineshloka
{स त्वं यज्ञैर्महाराज यजस्व बहुदक्षिणैः}
{यथैवेन्द्रो मनुष्येन्द्र चिराय विगतज्वरः}


\twolineshloka
{मा त्वमेवं गते किंचिच्छोचेथाः क्षत्रियर्षभ}
{गतास्ते क्षत्रधर्मेण शस्त्रपूताः परां गतिम्}


\twolineshloka
{भवितव्यं तथा तच्च यद्वृत्तं भरतर्षभ}
{दिष्टं हि राजशार्दूल न शक्यमतिवर्तितुम्}


\chapter{अध्यायः २३}
\twolineshloka
{वैशंपायन उवाच}
{}


\threelineshloka
{एवमुक्तस्तु कौन्तेयो गुडाकेशेन भारत}
{नोवाच किंचित्कौरव्यस्ततो द्वैपायनोऽब्रवीत् ॥व्यास उवाच}
{}


\twolineshloka
{बीभत्सोर्वचनं सौम्य सत्यमेतद्युधिष्ठिर}
{शास्त्रदृष्टः परो धर्मः स्मृतो गार्हस्थ्य आश्रमः}


\twolineshloka
{स्वधर्मं चर धर्मज्ञ यथाशास्त्रं यथाविधि}
{न हि गार्हस्थ्यमुत्सृज्य तवारण्यं विधीयते}


\twolineshloka
{गृहस्थं हि सदा देवाः पितरोऽतिथयस्तथा}
{भृत्याश्चैवोपजीवन्ति तान्भरस्व महीपते}


\twolineshloka
{वयांसि पशवश्चैव भूतानि च जनाधिप}
{गृहस्थैरेव धार्यन्ते तस्माच्छ्रेष्ठो गृहाश्रमी}


\twolineshloka
{सोऽयं चतुर्णामेतेषामाश्रमाणां दुराचरः}
{तं चराद्य विधिं पार्थ दुश्चरं दुर्बलेन्द्रियैः}


\twolineshloka
{वेदज्ञानं च ते कृत्स्नं तपश्चाचरितं महत्}
{पितृपैतामहं राज्यं धुर्यवद्वोद्दुमर्हसि}


\threelineshloka
{तपो यज्ञस्तथा विद्या भैक्ष्यमिन्द्रियसंयमः}
{ध्यानं विद्या समुत्थानं संतोषश्च श्रियं प्रति}
{तथा ह्येकान्तशीलत्वं तुष्टिर्दानं च शक्तितः}


\twolineshloka
{ब्राह्मणानां महाराज चेष्टा संसिद्धिकारिका}
{क्षत्रियाणां तु वक्ष्यामि तवापि विदितं पुनः}


\twolineshloka
{यज्ञो विद्या समुत्थानमसंतोषः श्रियं प्रति}
{दण्डधारणमुग्रत्वं प्रजानां परिपालनम्}


\twolineshloka
{वेदज्ञानं तथा कृत्स्नं तपः सुचरितं तथा}
{द्रविणोपार्जनं भूरि पात्रे च प्रतिपादनम्}


\twolineshloka
{एतानि राज्ञां कर्माणि सुकृतानि विशांपते}
{इमं लोकममुं चैव साधयन्तीति नः श्रुतम्}


\twolineshloka
{एषां ज्यायस्तु कौन्तेय दण्डधारणमुच्यते}
{बलं हि क्षत्रिये नित्यं बले दण्डः समाहितः}


\twolineshloka
{एताश्चेष्टाः क्षत्रियाणां राजन्संसिद्धिकारिकाः}
{अपि गाथामिमां चापि बृहस्पतिरगायत}


\twolineshloka
{भूमिरेतौ निगिरति सर्पो बिलशयानिव}
{राजानं चाविरोद्धारं ब्राह्मणं चाप्रवासिनम्}


\threelineshloka
{सुद्युम्नश्चापि राजर्षिः श्रूयते दण्डधारणात्}
{प्राप्तवान्परमां सिद्धिं दक्षः प्राचेतसो यथा ॥युधिष्ठिर उवाच}
{}


\threelineshloka
{भगवन्कर्मणा केन सुद्युम्नो वसुधाधिपः}
{संसिद्धिं परमां प्राप्तः श्रोतुमिच्छामि तं नृपम् ॥व्यास उवाच}
{}


\twolineshloka
{अत्राप्युदाहरन्तीममितिहासं पुरातनम्}
{शङ्खश्च लिखितश्चास्तां भ्रातरौ संशितव्रतौ}


\twolineshloka
{तयोरावसथावास्तां रमणीयौ पृथक्पृथक्}
{नित्यपुष्पफलैर्वृक्षैरुपेतौ बाहुदामनु}


\twolineshloka
{ततः कदाचिल्लिखितः शङ्खस्याश्रममागतः}
{यदृच्छयाऽथ शङ्खोपि निष्क्रान्तोऽभवदाश्रमात्}


\twolineshloka
{सोऽभिगम्याश्रमं भ्रातुश्चंक्रमँल्लिखितस्तदा}
{फलानि शातयामास सम्यक्परिणतान्युत}


\twolineshloka
{तान्युपादाय विस्रब्धो भक्षयामास स द्विजः}
{तस्मिंश्च भक्षयत्येव शङ्खोऽप्याश्रममागतः}


\twolineshloka
{भक्षयन्तं तु तं दृष्ट्वा शङ्खो भ्रातरमब्रवीत्}
{कुतः फलान्यवाप्तानि हेतुना केन खादसि}


\twolineshloka
{सोऽब्रवीद्धातरं ज्येष्ठमुपसृत्याभिवाद्य च}
{इत एव गृहीतानि मयेति प्रहसन्निव}


\twolineshloka
{तमब्रतीत्तथा शङ्खस्तीव्ररोषसमन्वितः}
{स्तेयं त्वया कृतमिदं फलान्याददता स्वयम्}


\twolineshloka
{गच्छ राजानमासाद्य स्वकर्म कथयस्व वै}
{अदत्तादानमेवं हि कृतं पार्थिवसत्तम}


\twolineshloka
{स्तेनं मां त्वं विदित्वा च स्वधर्ममनुपालय}
{शीघ्रं धारय चोरस्य मम दण्डं नराधिप}


\twolineshloka
{इत्युक्तस्तस्य वचनात्सुद्युम्न स नराधिपम्}
{अभ्यगच्छन्महाबाहो लिखितः संशितव्रतः}


\twolineshloka
{सुद्युम्नस्त्वन्तपालेभ्यः श्रुत्वा लिखितमागतम्}
{अभ्यगच्छत्सहामात्यः पद्भ्यामेव जनेश्वरः}


\twolineshloka
{तमब्रवीत्समागम्य स राजा धर्मवित्तमम्}
{किमागमनमाचक्ष्व भगवन्कृतमेव तत्}


\twolineshloka
{एवमुक्तः स विप्रर्षिः सुद्युम्नमिदमब्रवीत्}
{प्रतिश्रुत्य करिष्येति श्रुत्वा तत्कर्तुमर्हसि}


\threelineshloka
{अनिसृष्टानि गुरुणा फलानि मनुजर्षभ}
{भक्षितानि महाराज तत्र मां शाधि माचिरम् ॥सुद्युम्न उवाच}
{}


\twolineshloka
{प्रमाणं चेन्मतो राजा भवतो दण्डधारणे}
{अनुज्ञायामपि तथा हेतुः स्याद्ब्राह्मणर्षभ}


\threelineshloka
{स भवानभ्यनुज्ञातः शुचिकर्मा महाव्रतः}
{ब्रूहि कामानतोऽन्यांस्त्वं करिष्यामि हि ते वचः ॥व्यास उवाच}
{}


\twolineshloka
{संछन्द्यमानो ब्रह्मर्षिः पार्थिवेन महात्मना}
{नान्यं स वरयामास तस्माद्दण्डादृते वरम्}


\twolineshloka
{ततः स पृथिवीपालो लिखितस्य महात्मनः}
{करौ प्रच्छेदयामास धृतदण्डो जगाम सः}


\threelineshloka
{स गत्वा भ्रातरं शङ्खमार्तरूपोऽब्रवीदिदम्}
{धृतदण्डस्य दुर्बुद्धेर्भवांस्तत्क्षन्तुमर्हति ॥शङ्ख उवाच}
{}


\threelineshloka
{न कुप्ये तव धर्मज्ञ न त्वं दूषयसे मम}
{`सुनिर्मलं कुलं ब्रह्मन्नस्मिञ्जगति विश्रुतम्}
{'धर्मस्तु ते व्यतिक्रान्तस्ततस्ते निष्कृतिः कृता}


\twolineshloka
{त्वं गत्वा बाहुदां शीघ्रं तर्पयस्व यथाविधि}
{देवानृषीन्पितृंश्चैव मा चाधर्मे मनः कृथाः}


\twolineshloka
{`ब्रह्महत्यां सुरापानं स्तेयं गुर्वङ्गनागमम्}
{'महान्ति पातकान्याहुः संयोगं चैव तैः सह}


\twolineshloka
{न स्तेयसदृशं ब्रह्मन्महापातकमस्ति हि}
{जगत्यस्मिन्महाभाग ब्रह्महत्यासमं हि तत्}


\twolineshloka
{सर्वपातकिनां ब्रह्मन्दण्डः शारीर उच्यते}
{तस्करस्य विशेषेण नान्यो दण्डो विधीयते}


\twolineshloka
{ब्राह्मणः क्षत्रियो वाऽपि वैश्यः शूद्रोऽथवा द्विज}
{सर्वे कामकृते पापे हन्तव्या न विचारणा}


\twolineshloka
{राजभिर्धृतदण्डा वै कृत्वा पापानि मानवाः}
{निर्मलाः स्वर्गमायान्ति सन्तः सुकृतिनो यथा}


% Check verse!
उद्धृतं नः कुलं ब्रह्मन्नाज्ञादण्डे धृते त्वयि ॥'
\twolineshloka
{तस्य तद्वचनं श्रुत्वा शङ्खस्य लिखितस्तदा}
{अवगाह्यापगां पुण्यामुदकार्थं प्रचक्रमे}


\twolineshloka
{प्रादुरास्तां ततस्तस्य करौ जलजसन्निभौ}
{ततः स विस्मितो भ्रातुर्दर्शयामास तौ करौ}


\threelineshloka
{ततस्तमब्रवीच्छङ्खस्तपसेदं कृतं मया}
{मा च तेऽव विशङ्का भूद्दैवमत्र विधीयते ॥लिखित उवाच}
{}


\threelineshloka
{किंतु नाहं त्वया पूतः पूर्वमेव महाद्युते}
{यस्य ते तपसो वीर्यमीदृशं द्विजसत्तम ॥शङ्ख उवाच}
{}


\threelineshloka
{एवमेतन्मया कार्यं नाहं दण्डधरस्तव}
{स च पूतो नरपतिस्त्वं चापि पितृभिः सह ॥व्यास उवाच}
{}


\twolineshloka
{स राजा पाण्डवश्रेष्ठ श्रेयान्वै तेन कर्मणा}
{प्राप्तवान्परमां सिद्धिं दक्षः प्राचेतसो यथा}


\twolineshloka
{एष धर्मः क्षत्रियाणां प्रजानां परिपालनम्}
{उत्पथेभ्यो महाराज मा स्म शोके मनः कृथाः}


\twolineshloka
{भ्रातुरस्य हितं वाक्यं शृणु धर्मज्ञसत्तम}
{दण्ड एव हि राजेन्द्र क्षत्रधर्मो न मुण्डनम्}


\chapter{अध्यायः २४}
\twolineshloka
{वैशंपायन उवाच}
{}


\twolineshloka
{पुनरेव महर्षिस्तं कृष्णद्वैपायनोऽर्थवत्}
{अजातशत्रुं कौन्तेयमिदं वचनमब्रवीत्}


\twolineshloka
{अरण्ये वसतां तात भ्रातॄणां ते मनिस्विनाम्}
{मनोरथा महाराज ये तत्रासन्युधिष्ठिर}


\twolineshloka
{तानि मे भरतश्रेष्ठ प्राप्नुवन्तु महारथाः}
{प्रशाधि पृथिवीं पार्थ ययातिरिव नाहुषः}


\twolineshloka
{अरण्ये दुःखवसतिरनुभूता तपस्विभिः}
{दुःखस्यान्ते नरव्याघ्र सुखान्यनुभवन्तु वै}


\twolineshloka
{धर्ममर्थं च कामं च भ्रातृभिः सह भारत}
{अनुभूय ततः पश्चात्प्रस्थाताऽसि विशांपते}


\twolineshloka
{अर्थिनां च पितृणां च देवतानां च भारत}
{आनृण्यं गच्छ कौन्तेय ततः स्वर्गं गमिष्यसि}


\twolineshloka
{सर्वमेधाश्चमेधाभ्यां यजस्व कुरुनन्दन}
{ततः पश्चान्महाराज गमिष्यसि परां गतिम्}


\twolineshloka
{भ्रातृंश्च सर्वान्क्रतुभिः संयोज्य बहुदक्षिणैः}
{संप्राप्तः कीर्तिमतुलां पाण्डवेय गमिष्यसि}


\twolineshloka
{विझस्ते पुरुषव्याघ्र वचनं कुरुसत्तम}
{शृणुष्वैवं यथा कुर्वन्न धर्माच्च्यवसे नृप}


\twolineshloka
{आददानस्य विजयं विग्रहं च युधिष्ठिर}
{समानधर्मकुशलाः स्थापयन्ति नरेश्वर}


\twolineshloka
{`प्रत्यक्षमनुमानं च उपमानं तथाऽऽगमः}
{अर्थापत्तिस्तथैतिह्यं संशयो निर्णयस्तथा}


\twolineshloka
{आकार इङ्गितं चैव गतिश्चेष्टा च भारत}
{प्रतिज्ञा चैव हेतुश्च दृष्टान्तोपनयस्तथा}


\twolineshloka
{उक्तिर्निगमनं तेषां प्रमेयं च प्रयोजनम्}
{एतानि साधनान्याहुर्बहुवर्गप्रसिद्धये}


\threelineshloka
{प्रत्यक्षमनुमानं च सर्वेषां योनिरुच्यते}
{प्रमाणज्ञो हि शक्नोति दण्डयोनौ विचक्षणः}
{अप्रमाणवता नीतो दण्डो हन्यान्महीपतिम्}


\twolineshloka
{देशकालप्रतीक्षी यो दस्यून्मर्षयते नृपः}
{शास्त्रजां बुद्धिमास्थाय युज्यते नैनसा हि सः}


\twolineshloka
{आदाय बलिषङ्भागं यो राष्ट्रं नाभिरक्षति}
{प्रतिगृह्णाति तत्पापं चतुर्थांशेन भूमिपः}


\threelineshloka
{निबोध च यथाऽऽतिष्ठन्धर्मान्न च्यवते नृपः}
{निग्रहाद्धर्मशास्त्राणामनुरुद्ध्यन्नपेतभीः}
{कामक्रोधावनादृत्य पितेव समदर्शनः}


\twolineshloka
{दैवेनाभ्याहतो राजा कर्मकाले महाद्युते}
{न साधयति यत्कर्म न तत्राहुरतिक्रमम्}


\twolineshloka
{तरसा बुद्धिपूर्वं वा निग्राह्या एव शत्रवः}
{पापैः सह न संदध्याद्राज्यं पुण्यं च कारयेत्}


\twolineshloka
{शूराश्चार्याश्च सत्कार्या विद्वांसश्च युधिष्ठिर}
{गोमिनो धनिनश्चैव परिपाल्या विशेषतः}


\twolineshloka
{व्यवाहरेषु धर्मेषु योक्तव्याश्च बहुश्रुताः}
{`प्रमाणज्ञा महीपाल न्यायशास्त्रावलम्बिनः}


\twolineshloka
{वेदार्थतत्त्वविद्राजंस्तर्कशास्त्रबहुश्रुताः}
{मन्त्रे च व्यवहारे च नियोक्तव्या विजानता}


\twolineshloka
{तर्कशास्त्रकृता बुद्धिर्धर्मशास्त्रकृता च या}
{दण्डनीतिकृता चैव त्रैलोक्यमपि साधयेत्}


\twolineshloka
{नियोज्या वेदतत्त्वज्ञा यज्ञकर्मसु पार्तिव}
{वेदज्ञा ये च शास्त्रज्ञास्ते च राजन्सुबुद्धयः}


\twolineshloka
{आन्वीक्षकीत्रयीवार्तादण्डनीतिषु पारगाः}
{ते तु सर्वत्र योक्तव्यास्ते च बुद्धेः परं गताः ॥ '}


% Check verse!
गुणयुक्तेऽपि नैकस्मिन्विश्वसेत विचक्षणः
\twolineshloka
{अरक्षिता दुर्विनीतो मानी स्तब्धोऽभ्यसूयकः}
{एनसा युज्यते राजा दुर्दान्त इति चोच्यते}


\twolineshloka
{ये रक्ष्यमाणा हीयन्ते दैवेनृभ्याहता नृप}
{तस्करैश्चापि हीयन्ते सर्वं तद्राजकिल्विषम्}


\twolineshloka
{सुमन्त्रिते सुनीते च सर्वतश्चोपपादिते}
{पौरुषे कर्मणि कृते नास्त्यधर्मो युधिष्ठिर}


\twolineshloka
{विच्छिद्यन्ते समारब्धाः सिद्ध्यन्ते चापि दैवतः}
{कृते पुरुषकारे तु नैनः स्पृशति पार्थिवम्}


\twolineshloka
{अत्र ते राजशार्दूल वर्तयिष्ये कथामिमाम्}
{यद्वृत्तं पूर्वराजर्षेर्हयग्रीवस्य पाण्डव}


\twolineshloka
{शत्रून्हत्वा हतस्याजौ शूरस्याक्लिष्टकर्मणः}
{असहायस्य संग्रामे निर्जितस्य युधिष्ठिर}


\twolineshloka
{यत्कर्म वै निग्रहे शात्रवाणांयोगश्चाग्र्यः पालने मानवानाम्}
{कृत्वा कर्म प्राप्य कीर्ति स युद्धाद्वाजिग्रीवो मोदते स्वर्गलोके}


\twolineshloka
{संत्यक्तात्मा समरेष्वाततायीशस्त्रैश्छिन्नो दस्युभिर्वध्यमानः}
{अश्वग्रीवः कर्मशीलो महात्मासंसिद्धार्थो मोदते स्वर्गलोके}


\twolineshloka
{धनुर्यूपो रशना ज्या शरः स्रुक्स्रुवः खङ्गो रुधिरं यत्र चाज्यम्}
{रथो वेदी कामजो युद्धमग्निश्चातुर्होत्रं चतुरो वाजिमुख्याः}


\twolineshloka
{हुत्वा तस्मिन्यज्ञवह्नावथारीन्पापान्मुक्तो राजसिंहस्तरस्वी}
{प्राणान्हुत्वा चावभृथे रणे सवाजिग्रीवो मोदते देवलोके}


\twolineshloka
{राष्ट्रं रक्षन्बुद्धिपूर्वं नयेनसंत्यक्तात्मा यज्ञशीलो महात्मा}
{सर्वांल्लोकान्व्याप्य कीर्त्या मनस्वीवाजिग्रीवो मोदते देवलोके}


\twolineshloka
{दैवीं सिद्धिं मानुषीं दण्डनीतिंयोगन्यासैः पालयित्वा महीं च}
{तस्माद्राजा धर्मशीलो महात्मावाजिग्रीवो मोदते देवलोके}


\twolineshloka
{विद्वांस्त्यागी श्रद्दधानः कृतज्ञस्त्यक्त्वा लोकं मानुषं कर्म कृत्वा}
{मेधाविनां विदुषां संमतानांतनुत्यजां लोकमाक्रम्य राजा}


\twolineshloka
{सम्यग्वेदान्प्राप्य शास्त्राण्यधीत्यसम्यग्राज्यं पालयित्वा महात्मा}
{चातुर्वर्ण्यं स्थापयित्वा स्वधर्मेवाजिग्रीवो मोदते देवलोके}


\twolineshloka
{जित्वा संग्रामान्पालयित्वा प्रजाश्चसोमं पीत्वा तर्पयित्वा द्विजाग्र्यान्}
{युक्त्या दण्डं धारयित्वा प्रजानांयुद्धे क्षीणे मोदते देवलोके}


\twolineshloka
{वृत्तं यस्य श्लाघनीयं मनुष्याःसन्तो विद्वांसोऽर्हयन्त्यर्हणीयम्}
{स्वर्गं जित्वा वीरलोकानवाप्यसिद्धिं प्राप्तः पुण्यकीर्तिर्महात्मा}


\chapter{अध्यायः २५}
\twolineshloka
{वैशंपायन उवाच}
{}


\threelineshloka
{द्वैपायनवचः श्रुत्वा कुपिते च धनञ्जये}
{व्यासमामन्त्र्य कौन्तेयः प्रत्युवाच युधिष्ठिरः ॥युधिष्ठिर उवाच}
{}


\twolineshloka
{न पार्थिवमिदं राज्यं न भोगाश्च पृथग्विधाः}
{प्रीणयन्ति मनो मेऽद्य शोको मां दारयत्ययम्}


\twolineshloka
{श्रुत्वा वीरविहीनानामपुत्राणां च योषिताम्}
{परिदेवयमानानां न शान्तिं मनसा लभे}


\threelineshloka
{इत्युक्तः प्रत्युवाचेदं व्यासो योगविदांवरः}
{युधिष्ठिरं महाप्राज्ञो धर्मज्ञो वेदपारगः ॥व्यास उवाच}
{}


\twolineshloka
{न कर्मणा लभ्यते चेज्यया वानाप्यस्ति दाता पुरुषस्य कश्चित्}
{पर्याययोगाद्विहितं विधात्राकालेन सर्वं लभते मनुष्यः}


\twolineshloka
{न बुद्धिशक्त्याऽऽध्ययनेन शक्यंप्राप्तुं विशेषं मनुजैरकाले}
{मूर्खोऽपि चाप्नोति कदाचिदर्थान्कालो हि सर्वं पुरुषस्य दाता}


\twolineshloka
{न भूरकालेषु फलं ददातिशिल्पानि मन्त्राश्च तथौषधानि}
{तान्येव कालेन समाहितानिसिद्ध्यन्ति वर्धन्ति च भूतिकाले}


\twolineshloka
{कालेन शीघ्राः प्रवहन्ति वाताःकालेन वृष्टिर्जलदानुपैति}
{कालेन पझोत्पलवञ्जलं चकालेन पुष्यन्ति वनेषु वृक्षाः}


\twolineshloka
{कालेन कृष्णाश्च सिताश्च रात्र्यःकालेन चन्द्रः परिपूर्णबिम्बः}
{नाकालतः पुष्पफलं द्रुमाणांनाकालवेगाः सरितो वहन्ति}


\twolineshloka
{नाकालमत्ताः खगपन्नगाश्चमृगद्विपाः शैलमृगाश्च लोके}
{नाकालतः स्त्रीषु भवन्ति गर्भानायन्त्यकाले शिशिरोष्णवर्षाः}


\twolineshloka
{नाकालतो म्रियते जायते वानाकालतो व्याहरते च बालः}
{नाकालतो यौवनमभ्युपैतिनाकालतो रोहति बीजमुप्तम्}


\twolineshloka
{नाकालतो भानुरुपैति योगंनाकालतोऽस्तं गिरिमभ्युपैति}
{नाकालतो वर्धते हीयते चचन्द्रः समुद्रोऽपि महोर्मिमाली}


\twolineshloka
{अत्राप्युदाहरन्तीममितिहासं पुरातनम्}
{गीतं राज्ञा सेनजिता दुःखार्तेन युधिष्ठिर}


\twolineshloka
{सर्वानेवैष पर्यायो मर्त्यान्स्पृशति दुःसहः}
{कालेन परिपक्वा हि म्रियन्ते सर्वपार्थिवाः}


\twolineshloka
{घ्नन्ति चान्यान्नरान्राजंस्तानप्यन्ये तथा नराः}
{संज्ञैषा लौकिकी राजन्न हिनस्ति न हन्यते}


\twolineshloka
{हन्तीति मन्यते कश्चिन्न हन्तीत्यपि चापरः}
{स्वभावतस्तु नियतौ भूतानां प्रभवाप्ययौ}


\twolineshloka
{नष्टे धने वा दारे वा पुत्रे पितरि वा मृते}
{अहो दुःखमिति ध्यायन्दुःखस्यापचितिं चरेत्}


\twolineshloka
{स किं शोचसि मूढः सञ्शोच्यान्किमनुशोचसि}
{पश्य दुःखेषु दुःखानि भयेषु च भयान्यपि}


\twolineshloka
{आत्माऽपि चायं न मम सर्वाऽपि पथिवी मम}
{यथा मम तथाऽन्येषामिति पश्यन्न मुह्यति}


\twolineshloka
{शोकस्थानसहस्राणि हर्षस्थानशतानि च}
{दिवसेदिवसे मूढमाविशन्ति न पण्डितम्}


\twolineshloka
{एवमेतानि कालेन प्रियद्वेष्याणि भागशः}
{जीवेषु परिवर्तन्ते दुःखानि च सुखानि च}


\twolineshloka
{दुःखमेवास्ति न सुखं तस्मात्तदुपलभ्यते}
{तृष्णार्तिप्रभवं दुःखं दुःखार्तिप्रभवं सुखम्}


\twolineshloka
{सुखस्यानन्तरं दुःखं दुःखस्यानन्तरं सुखम्}
{न नित्यं लभते दुःखं न नित्यं लभते सुखम्}


\twolineshloka
{सुखमेव हि दुःखान्तं कदाचिद्दुःखतः सुखम्}
{तस्मादेतद्द्वयं जह्याद्य इच्छेच्छाश्वतं सुखम्}


\threelineshloka
{सुखान्तप्रभवं दुःखं दुःखान्तप्रभवं सुखम्}
{यन्निमित्तो भवेच्छोकस्तापो वा दुःखमूर्च्छितः}
{आयासो वाऽपि यन्मूलस्तदेकाङ्गमपि त्यजेत्}


\twolineshloka
{सुखं वा यदि वा दुःखं प्रियं वा यदि वाऽप्रियम्}
{प्राप्तं प्राप्तमुपासीत हृदयेनापराजितः}


\twolineshloka
{ईषदप्यङ्ग दाराणां पुत्राणामाचरन्प्रियम्}
{ततो ज्ञास्यसि कः कस्य केन वा कथमेव च}


\twolineshloka
{ये च मूढतमा लोके ये च बुद्धेः परं गताः}
{त एव सुखमेधन्ते मध्यमः क्लिश्यते जनः}


\twolineshloka
{इत्यन्नवीन्महाप्राज्ञो युधिष्ठिर स सेनजित्}
{परावरज्ञो लोकस्य धर्मवित्सुखदुःखवित्}


\twolineshloka
{परदुःखेन दुःखी यो न स जातु सुखी भवेत्}
{दुःखानां हि क्षयो नास्ति जायते ह्यपरात्परम्}


\twolineshloka
{सुखं च दुःखं च भवाभवौ चलाभालाभौ मरणं जीवितं च}
{पर्यायतः सर्वमवाप्तुवन्तितस्मान्न मुह्येन्न च संप्रहृष्येत्}


\twolineshloka
{दीक्षां राज्ञां संयुगे धर्ममाहुर्योगं राज्ये दण्डनीतिं च सम्यक्}
{वित्तत्यागं दक्षिणां चैव यज्ञेसम्यग्दानं पावनानीति विद्यात्}


\twolineshloka
{रक्षन्राज्यं बुद्धिपूर्वं नयेनसंत्यक्तात्मा यज्ञशीलो महात्मा}
{सर्वाल्लोकान्धर्मदृष्ट्यावलोकन्नूध्वं देहान्मोदते देवलोके}


\twolineshloka
{जित्वा संग्रामान्पालयित्वा च राष्ट्रंसोमं पीत्वा वर्धयित्वा प्रजाश्च}
{युक्त्या दण्डं धारयित्वा प्रजानांपश्चात्क्षीणायुर्मोदते देवलोके}


\threelineshloka
{`यजन्ति यज्ञान्विजयन्ति राज्यंरक्षन्ति राष्ट्राणि प्रियाणि चैषाम्}
{'सम्यग्वेदान्प्राप्य शास्त्राण्यधीत्यसम्यग्राज्यं पालयित्वा च राजा}
{चातुर्वर्ण्यं स्थापयित्वा स्वधर्मेपूतात्मा वै मोदते देवलोके}


\twolineshloka
{यस्य वृत्तं नमस्यन्ति स्वर्गस्थस्यापि मानवाः}
{पौरजानपदामात्याः स राजा राजसत्तमः}


\chapter{अध्यायः २६}
\twolineshloka
{युधिष्ठिर उवाच}
{}


\twolineshloka
{अभिमन्यौ हते बाले द्रौपद्यास्तनयेषु च}
{धृष्टद्युम्ने विराटे च द्रुपदे च महीपतौ}


\twolineshloka
{वृषसेने च धर्मज्ञे धृष्टकेतौ तु पार्थिवे}
{तथाऽन्येषु नरेन्द्रेषु नानादेश्येषु संयुगे}


\twolineshloka
{न च मुञ्चति मां शोको ज्ञातिघातिनमातुरम्}
{राज्यकामुकमत्युग्रं स्ववंशोच्छेदकारिणम्}


\twolineshloka
{यस्याङ्के क्रीडमानेन मया विपरिवर्तितम्}
{स मया राज्यलुब्धेन गाङ्गेयो युधि पातितः}


\twolineshloka
{यदा ह्येनं विघूर्णन्तं मदर्थं पार्थसायकैः}
{तक्ष्यमाणं यथा वज्रैः प्रेक्षमाणं शिखण्डिनम्}


\twolineshloka
{जीर्णसिंहमिव प्राज्ञं नरसिंहं पितामहम्}
{कीर्यमाणं शरैर्दीप्तैर्दृष्ट्वा मे व्यथितं मनः}


\twolineshloka
{प्राड्भुखं सीदमानं च रथात्पररथारुजम्}
{घूर्णमानं यथा शैलं तदा मे कश्मलोऽभवत्}


\twolineshloka
{यः सवाणधनुष्पाणिर्योधयामास भार्गवम्}
{बहून्यहानि कौरव्यः कुरुक्षेत्रे महामृधे}


\twolineshloka
{समेतं पार्थिवं क्षत्रं वाराणस्यां नदीसुतः}
{कन्यार्थमाह्वयद्वीरो रथेनैकेन संयुगे}


\twolineshloka
{येन चोग्रायुधो राजा चक्रवर्ती दुरासदः}
{दग्धश्चास्त्रप्रतापेन स मया युधि पातितः}


\twolineshloka
{स्वयं मृत्युं रक्षमाणः पाञ्चाल्यं यः शिखण्डिनम्}
{न बाणैः पातयामास सोऽर्जुनेन निपातितः}


\twolineshloka
{यदैनं पतितं भूमावपश्यं रुधिरोक्षितम्}
{तदैवाविशदन्युग्रो ज्वरो मां मुनिसत्तम}


\threelineshloka
{येन संवर्धिता बाला येन स्म परिरक्षिताः}
{स मया राज्यलुब्धेन पापेन गुरुधातिना}
{अल्पकालस्य राज्यस्य कृते मूढेन पातितः}


\twolineshloka
{आचार्यश्च महेष्वासः सर्वपार्थिवपूजितः}
{अभिगम्य रणे मिथ्या पापेनोक्तः सुतं प्रति}


\twolineshloka
{तन्मे दहति गात्राणि यन्मां गुरुरभाषत}
{सत्यमाख्याहि राजंस्त्वं यदि जीवति मे सुतः}


\twolineshloka
{सत्यमामर्शयन्विप्रो मयि तत्परिपृष्टवान्}
{कुञ्जरं चान्तरं कृत्वा मिथ्योपचरितो मया}


\twolineshloka
{सुभृशं राज्यलुब्धेन पापेन गुरुघातिना}
{सत्यकञ्चुकमुन्मुच्य मया स गुरुराहवे}


\twolineshloka
{अश्वत्थामा हत इति निरुक्तः कुञ्जरे हते}
{काँल्लोकांस्तु गमिष्यामि कृत्वा कर्म सुदुष्करम्}


\twolineshloka
{अघातयं च यत्कर्णं समरेष्वपलायिनम्}
{ज्येष्ठभ्रातरमत्युग्रः को मत्तः पापकृत्तमः}


\twolineshloka
{अभिमन्युं च यद्वालं जातं सिंहमिवाद्रिषु}
{प्रावेशयमहं लुब्धो वाहिनीं द्रोणपालिताम्}


\twolineshloka
{तदाप्रभृति वीभत्सुं न शक्नोमि निरीक्षितुम्}
{कृष्णं च पुण्डरीकाक्षं किल्विषी भ्रूणहा यथा}


\twolineshloka
{द्रौपदीं चाप्यदुःखार्हां पञ्चपुत्रैर्विनाकृताम्}
{शोचामि पृथिवीं हीनां पञ्चभिः पर्वतैरिव}


\twolineshloka
{सोऽहमागस्करः पापः पृथिवीनाशकारकः}
{आसीन एवमेवेदं शोषयिष्ये कलेवरम्}


\twolineshloka
{प्रायोपविष्टं जानीध्वमथ मां गुरुघातिनम्}
{जातिष्वन्यास्वपि यथा न भवेयं कुलान्तकृत्}


\twolineshloka
{न भोक्ष्ये न च पानीयमुपयोक्ष्ये कथंचन}
{शोषयिष्ये प्रियान्प्राणानिहस्थोऽहं तपोधनाः}


\threelineshloka
{यथेष्टं गम्यतां काममनुजाने प्रसाद्य वः}
{सर्वे मामनुजानीत त्यक्ष्यामीदं कलेवरम् ॥वैशंपायन उवचा}
{}


\threelineshloka
{तमेवंवादिनं पार्थं बन्धुशोकेन विह्वलम्}
{मैवमित्यब्रवीद्व्यासो निगृह्य मुनिसत्तमः ॥व्यास उवाच}
{}


\twolineshloka
{अतिवेलं महाराज न शोकं कर्तुमर्हसि}
{पुनरुक्तं तु वक्ष्यामि दिष्टमेतदिति प्रभो}


\twolineshloka
{संयोगा विप्रयोगाश्च जातानां प्राणिनां ध्रुवम्}
{बुद्धुदा इव तोयेषु भवन्ति न भवन्ति च}


\twolineshloka
{सर्वे क्षयान्ता निचयाः पतनान्ताः समुच्छ्रयाः}
{संयोगा विप्रयोगान्ता मरणान्तं हि जीवितम्}


\twolineshloka
{सुखं दुःखान्तमालस्यं दाक्ष्यं दुःखं सुखोदयम्}
{भूतिः श्रीर्ह्रीर्धृतिः कीर्तिर्दक्षे वसति नालसे}


\twolineshloka
{नालं सुखाय सुहृदो नाले दुःखाय शत्रवः}
{न च प्रज्ञालमर्थेभ्यो न सुखेभ्योऽप्यलं धनम्}


\twolineshloka
{यथा सृष्टोऽसि कौन्तेय धात्रा कर्मसु तत्क्ररु}
{अत एव हि सिद्धिस्ते नेशस्त्वं ह्यात्मनो नृप}


\chapter{अध्यायः २७}
\twolineshloka
{वैशंपायन उवाच}
{}


\threelineshloka
{ज्ञातिशोकाभितप्तस्य प्राणानिष्टांस्त्यजिष्यतः}
{ज्येष्ठस्य पाण्डुपुत्रस्य व्यासः शोकमपानुदत् ॥व्यास उवाच}
{}


\twolineshloka
{अत्राप्युदाहरन्तीमभितिहासं पुरातनम्}
{अश्मगीतं नरव्याघ्र तन्निबोध युधिष्ठिर}


\threelineshloka
{अश्मानं ब्राह्मणं प्राज्ञं वैदेहो जनको नृपः}
{संशयं परिपप्रच्छ दुःखशोकसमन्वितः ॥जनक उवाच}
{}


\threelineshloka
{आगमे यदि वाऽपाये ज्ञातीनां द्रविणस्य च}
{नरेण प्रतिपत्तव्यं कल्याणं कथमिच्छता ॥अश्मोवाच}
{}


\twolineshloka
{उत्पन्नमिममात्मानं नरस्यानन्तरं ततः}
{तानितान्यनुवर्तन्ते दुःखानि च सुखानि च}


\twolineshloka
{तेषामन्यतरापत्तौ यद्यदेवोपसेवते}
{तदस्य चेतनामाशु हरत्यभ्रमिवानिलः}


\twolineshloka
{अभिजातोऽस्मि सिद्धोऽस्मि नास्मि केवलमानुषः}
{इत्येभिर्हेतुभिस्तस्य त्रिभिश्चित्तं प्रसिच्यते}


\twolineshloka
{संप्रसक्तमना भोगान्विसृज्य पितृसंचितान्}
{परिक्षीणः परस्वानामादानं साधु मन्यते}


\twolineshloka
{तमतिक्रान्तमर्यादमाददानमसांप्रतम्}
{प्रतिषेधन्ति राजानो लुब्धा मृगमिवेषुभिः}


\twolineshloka
{ये च विंशतिवर्षा वा त्रिंशद्वर्षाश्च मानवाः}
{परेण ते वर्षशतान्न भविष्यन्ति पार्थिव}


\twolineshloka
{तेषां परमदुःखानां बुद्ध्या भैषज्यमाचरेत्}
{सर्वप्राणभृतां वृत्तं प्रेक्षमाणस्ततस्ततः}


\twolineshloka
{मानसानां पुनर्योनिर्दुःखानां चित्तविभ्रमः}
{अनिष्टोपनिपातो वा तृतीयं नोपपद्यते}


\twolineshloka
{एवमेतानि दुःखानि तानि तानीह मानवम्}
{विविधान्युपवर्तन्ते तथा संस्पर्शजान्यपि}


\twolineshloka
{जरामृत्यू हि भूतानां खादितारौ वृकाविव}
{बलिनां दुर्बलानां च ह्रस्वानां महतामपि}


\twolineshloka
{न कश्चिज्जात्वतिक्रामेज्जरामृत्यू हि मानवः}
{अपि सागरपर्यन्तां विजित्येमां वसुंधराम्}


\twolineshloka
{सुखं वा यदि वा दुःखं भूतानां पर्युपस्थितम्}
{प्राप्तव्यमवशैः सर्वं परिहारो न विद्यते}


\twolineshloka
{पूर्वे वयसि मध्ये वाऽप्युत्तरे वा नराधिप}
{अवर्जनीयास्तेऽर्था वै काङ्क्षिता ये ततोऽन्यथा}


\twolineshloka
{अप्रियैः सह संयोगो विप्रयोगश्च सुप्रियैः}
{अर्थानर्थौ सुखं दुःखं विधानमनुवर्तते}


\twolineshloka
{प्रादुर्भावश्च भूतानां देहत्यागस्तथैव च}
{प्राप्तिव्यायामयोगश्च सर्वमेतत्प्रतिष्ठितम्}


\twolineshloka
{गन्धवर्णरसस्पर्शा निवर्तन्ते स्वभावतः}
{तथैव सुखदुःखानि विधानमनुवर्तते}


\twolineshloka
{आसनं शयनं यानमुत्थानं पानभोजनम्}
{नियतं सर्वभूतानां कालेनैव भवत्युत}


\twolineshloka
{वैद्याश्चाप्यातुराः सन्ति बलवन्तश्च दुर्बलाः}
{स्त्रीमन्तश्चापरे षण्ढा विचित्रः कालपर्ययः}


\twolineshloka
{कुले जन्म तथा वीर्यमारोग्यं रूपमेव च}
{सौभाग्यमुपभोगश्च भवितव्येन लभ्यते}


\twolineshloka
{सन्ति पुत्राः सुबहवो दरिद्राणामनिच्छताम्}
{नास्ति पुत्रः समृद्धानां विचित्रं विधिचेष्टितम्}


\twolineshloka
{व्याधिरग्निर्जलं शस्त्रं बुभुक्षाश्चापदो विषम}
{ज्वरश्च मरणं जन्तोरुच्चाच्च पतनं तथा}


\threelineshloka
{निर्याणे यस्य यद्दिष्टं तेन गच्छति सेतुना}
{दृश्यते नाप्यतिक्रामन्न निष्क्रान्तोऽथवा पुनः}
{दृश्यते चाप्यतिक्रामन्न निग्राह्योऽथवा पुनः}


\twolineshloka
{दृश्यते हि युवैवेह विनश्यन्वसुमान्नरः}
{दरिद्रश्च परिक्लिष्टः शतवर्षो जरान्वितः}


\twolineshloka
{अकिञ्चनाश्च दृश्यन्ते पुरुषाश्चिरजीविनः}
{समृद्धे च कुले जाता विनश्यन्ति पतङ्गवत्}


\twolineshloka
{प्रायेण श्रीमतां लोके भोक्तुं शक्तिर्न विद्यते}
{दरिद्राणां तु भूयिष्ठं काष्ठमश्मा हि जीर्यते}


\twolineshloka
{अहमेतत्करोमीति मन्यते कालनोदितः}
{यद्यदिष्टमसंतोषाहुरात्मा पापमाचरेत्}


\twolineshloka
{मृगयाक्षाः स्त्रियः पानं प्रसङ्गा निन्दिता बुधैः}
{दृश्यन्ते पुरुषाश्चात्र संप्रयुक्ता बहुश्रुताः}


\twolineshloka
{इति कालेन सर्वार्थानीप्सितानीप्सितानिह}
{स्पृशन्ति सर्वभूतानि निमित्तं नोपलभ्यते}


\twolineshloka
{वायुमाकाशमग्निं च चन्द्रादित्यावहः क्षपे}
{ज्योतींषि सरितः शैलान्कः करोति बिभतिं च}


\twolineshloka
{शीतमुष्णं तथा वर्षं कालेन परिवर्तते}
{एवमेव मनुष्याणां सुखदुःखे नरर्षभ}


\twolineshloka
{नौषधानि न शस्त्राणि न होमा न पुनर्जपाः}
{त्रायन्ते मृत्युनोपेतं जरया चापि मानवम्}


\twolineshloka
{यथा काष्ठं च काष्ठं च समेयातां महोदधौ}
{समेत्य च व्यपेयातां तद्वद्भूतसमागमः}


\twolineshloka
{ये च निष्परुषैरुक्तगीतवाद्यैरुपस्थिताः}
{ये चानाथाः परान्नादाः कालस्तेषु समक्रियः}


\twolineshloka
{मातापितृसहस्राणि पुत्रदारशतानि च}
{संसारेष्वनुभूतानि कस्य ते कस्य वा वयम्}


\twolineshloka
{नैवास्य कश्चिद्भविता नायं भवति कस्यचित्}
{पथि संगतमेवेदं दारबन्धुसुहृज्जनैः}


\twolineshloka
{क्वासे क्व च गमिष्यामि कोऽन्वहं किमिहास्थितः}
{कस्मात्किमनुशोचेयमित्येवं स्थापयेन्मनः}


\twolineshloka
{अनित्ये प्रियसंवासे संसारे चक्रवद्गतौ}
{पथि संगतमेवैतद्धाता माता पिता सखा}


\twolineshloka
{न दृष्टपूर्वं प्रत्यक्षं परलोकं विदुर्बुधाः}
{आगमांस्त्वनतिक्रम्य श्रद्धातव्यं बुभूषता}


\twolineshloka
{कुर्वीत पितृदैवत्यं धर्म्याणि च समाचरेत्}
{यजेच्च विद्वान्विधिवत्रिवर्गं चाप्युपाचरेत्}


\twolineshloka
{सन्निमज्जेज्जगदिदं गम्भीरे कालसागरे}
{जरामृत्युमहाग्राहे न कश्चिदवबुध्यते}


\twolineshloka
{आयुर्वेदमधीयानाः केवलं सपरिग्रहाः}
{दृश्यन्ते बहवो वैद्या व्याधिभिः समबिप्लुताः}


\twolineshloka
{ते पिबन्तः कषायांश्च सर्पीषि विविधानि च}
{न मृत्युमतिवर्तन्ते वेलामिव महोदधिः}


\twolineshloka
{रसायनविदश्चैव सुप्रयुक्तरसायनाः}
{दृश्यन्ते जरया भग्ना नागा नागैरिवोत्तमैः}


\twolineshloka
{तथैव तपसोपेताः स्वाध्यायाध्ययने रताः}
{दातारो यज्ञशीलाश्च न तरन्ति जरान्तकौ}


\twolineshloka
{न ह्यहानि निवर्तन्ते न मासा न पुनः समाः}
{जातानां सर्वभूतानां न पुनर्वै समागमः}


\twolineshloka
{सोऽयं विपुलमध्वानं कालेन ध्रुवमध्रुवः}
{स्रोतसैव समभ्येति सर्वभूतनिषेवितम्}


\twolineshloka
{देहो वा जीविताद्व्येति देही वाऽप्येति देहतः}
{पथि संगतमेवेदं दारैरन्यैश्च बन्धुभिः}


\twolineshloka
{नायमत्यन्तसंवासो लभ्यते जातु केनचित्}
{अपि स्वेन शरीरेण किमुतान्येन केनचित्}


\twolineshloka
{क्वनु तेऽद्य पिता राजन्क्वनु तेऽद्य पितामहाः}
{न त्वं पश्यसि तानद्य न त्वां पश्यन्ति तेऽनघ}


\twolineshloka
{न चैव पुरुषो द्रष्टा स्वर्गस्य नरकस्य च}
{आगमस्तु सतां चक्षुर्नृपते तमिहाचर}


\twolineshloka
{चरितब्रह्मचर्यो हि प्रजायेत यजेत च}
{पितृदेवमनुष्याणामानृण्यादनसूयकः}


\twolineshloka
{स यज्ञशीलः प्रजने निविष्टःप्राग्ब्रह्मचारी प्रविभक्तभैक्षः}
{आराधयेत्स्वर्गमिमं च लोकंपरं च मुक्त्वा हृदयव्यलीकम्}


\twolineshloka
{सम्यक्स्वधर्मं चरतो नृपस्यद्रव्याणि चाभ्याहरतो यथावत्}
{प्रवृद्धचक्रस्य यशोऽभिवर्धतेसर्वेषु लोकेषु चराचरेषु}


\twolineshloka
{इत्येवमाकर्ण्य विदेहराजोवाक्यं समग्रं परिपूर्णहेतु}
{अश्मानमामन्त्र्य विशुद्धबुद्धिर्ययौ गृहं स्वं प्रति शान्तशोकः}


\twolineshloka
{तथा त्वमप्यद्य विमुच्य शोकमुत्तिष्ठ शक्रोपम हर्षमेहि}
{क्षात्रेण धर्मेण मही जिता तेतां भुङ्क्ष्व कुन्तीसुत मावमंस्थाः}


\chapter{अध्यायः २८}
\twolineshloka
{वैशंपायन उवाच}
{}


\threelineshloka
{अव्याहरति राजेन्द्रे धर्मपुत्रे युधिष्ठिरे}
{गुडाकेशो हृषीकेशमभ्यभाषत पाण्डवः ॥अर्जुन उवाच}
{}


\twolineshloka
{ज्ञातिशोकाभिसंतप्तो धर्मपुत्रः परंतपः}
{एष शोकार्णवे मग्नस्तमाश्वासय माधव}


\threelineshloka
{सर्वे स्म ते संशयिताः पुनरेव जनार्दन}
{अस्य शोकं महाप्राज्ञ प्रणाशयितुमर्हसि ॥वैशंपायन उवाच}
{}


\twolineshloka
{एवमुक्तस्तु गोविन्दो विजयेन महात्मना}
{पर्यवर्तत राजानं पुण्डरीकेक्षणोऽच्युतः}


\twolineshloka
{अनतिक्रमणीयो हि धर्मराजस्य केशवः}
{बाल्यात्प्रभृति गोविंदः प्रीत्या चाभ्यधिकोर्जुनात्}


\twolineshloka
{संप्रगृह्य महाबाहुर्भुजं चन्दनभूषितम्}
{शैलस्तम्भोपमं शौरिरुवाचाभिविनोदयन्}


\threelineshloka
{शुशुभे वदनं तस्य सुदंष्ट्रं चारुलोचनम्}
{व्याकोचमिव विस्पष्टं पझं सूर्यविबोधितम् ॥वासुदेव उवाच}
{}


\twolineshloka
{मा कृथाः पुरुषव्याघ्र शोकं त्वं गात्रशोषणम्}
{न हि ते सुलभा भूयो ये हताऽस्मिन्रणाजिरे}


\twolineshloka
{स्वप्नलब्धा यथा लाभा वितथाः प्रतिबोधने}
{तथा ते क्षत्रिया राजन्ये व्यतीता महारणे}


\twolineshloka
{सर्वे ह्यभिमुखाः शूरा निहता रणशोभिनः}
{नैषां कश्चित्पृष्ठतो वा पलायन्वा निपातितः}


\twolineshloka
{सर्वे त्यक्त्वाऽऽत्मनः प्राणान्युद्ध्वा वीरा महामृधे}
{शस्त्रपूता दिवं प्राप्ता न ताञ्छोचितुमर्हसि}


\threelineshloka
{क्षत्रधर्मरताः शूरा वेदवेदाङ्गपारगाः}
{प्राप्ता वीरगतिं पुण्यां तान्न शोचितुमर्हसि}
{मृतान्महानुभावांस्त्वं श्रुत्वैव पृथिवीपतीन्}


\threelineshloka
{अत्रैवोदाहरन्तीममितिहासं पुरातनम्}
{सृञ्जयं पुत्रशोकार्तं यथाऽयं नारदोऽब्रवीत् ॥नारद उवाच}
{}


\twolineshloka
{सुखदुःखैरहं त्वं च प्रजाः सर्वाश्च सृञ्जय}
{अविमुक्ता मरिष्यामस्तत्र का परिदेवना}


\twolineshloka
{महाभाग्यं पुरा राज्ञां कीर्त्यमानं मया शृणु}
{गच्छावधानं नृपते ततो दुःखं प्रहास्यसि}


\threelineshloka
{मृतान्महानुभावांस्त्वं श्रुत्वैव पृथिवीपतीन्}
{शममानय संतापं शृणु विस्तरशश्च मे}
{क्रूरग्रहाभिशमनमायुर्वर्धनमुत्तमम्}


\twolineshloka
{अग्रिमाणां क्षितिभुजामुदारं च मनोहरम्}
{आविक्षितं मरुत्तं च मृतं सृञ्जय शुश्रुम}


\twolineshloka
{यस्य सेन्द्राः सवरुणा बृहस्पतिपुरोगमाः}
{देवा विश्वसृजो राज्ञो यज्ञमीयुर्महात्मनः}


\twolineshloka
{यः स्पर्धामानयच्छक्रं देवराजं पुरंदरम्}
{शक्रप्रियैषी यं विद्वान्प्रत्याचष्ट बृहस्पतिः}


% Check verse!
संवर्तो याजयामास यं पीडार्थं बृहस्पतेः
\twolineshloka
{यस्मिन्प्रशासति महीं नृपतौ राजसत्तम}
{अकृष्टपच्या पृथिवी विबभौ सस्यमालिनी}


\twolineshloka
{आविक्षितस्य वै सत्रे विश्वेदेवाः सभासदः}
{मरुतः परिवेष्टारः साध्याश्चासन्महात्मनः}


\twolineshloka
{मरुद्गण मरुत्तस्य यत्सोममपिबंस्ततः}
{देवान्मनुष्यान्गन्धर्वानत्यरिच्यन्त दक्षिणाः}


\twolineshloka
{स चेन्ममार सृञ्जय चतुर्भद्रतरस्त्वया}
{पुत्रात्पुण्यतरश्चैव मा पुत्रमनुतप्यथाः}


\twolineshloka
{सुहोत्रं च द्वतिथिनं मृतं सृञ्जय शुश्रुम}
{यस्मे हिरण्यं ववृषे मघवा परिवत्सरम्}


\twolineshloka
{सत्यनामा वसुमती यं प्राप्यासीञ्जनाधिपम्}
{हिरण्यमवहन्नद्यस्तस्मिञ्जनपदेश्वरे}


\twolineshloka
{मत्स्यान्कर्कटकान्नक्रान्मकराञ्छिंशुकानपि}
{नदीष्ववासृजद्राजन्मघवा लोकपूजितः}


\twolineshloka
{हैरण्यान्पातितान्दृष्ट्वा मत्स्यान्मकरकच्छपान्}
{सहस्रशोऽथ शतशस्ततोऽस्मयत वैतिथिः}


\twolineshloka
{तद्धिरण्यमपर्यन्तमावृतं कुरुजाङ्गले}
{ईजानो वितते यज्ञे ब्राह्मणेभ्यः समार्पयत्}


\threelineshloka
{स चेन्ममार सृञ्जय चतुर्भद्रतरस्त्वया}
{पुत्रात्पुण्यतरश्चैव मा पुत्रमनुतप्यथाः}
{अदक्षिणमयज्वानं श्वैत्य संशाम्य माशुचः}


\twolineshloka
{अङ्गं बृहद्रथं चैव मृतं सृञ्जय शुश्रुम}
{यः सहस्रं सहस्राणां श्वेतानश्वानवासृजात्}


\twolineshloka
{सहस्रं च सहस्राणां कन्या हमपरिष्कृताः}
{ईजानो वितते यज्ञे दक्षिणामत्यकालयत्}


\twolineshloka
{यः सहस्रं सहस्राणां गजानां पझमालिनाम्}
{ईजानो वितते यज्ञे दक्षिणामत्यकालयत्}


\twolineshloka
{शतं शतसहस्राणि वृषाणां हेममालिनाम्}
{गवां सहस्रानुचरं दक्षिणामत्यकालयत्}


\twolineshloka
{अङ्गस्य यजमानस्य तदा विष्णुपदे गिरौ}
{अमाद्यदिन्द्रः सोमेन दक्षिणाभिर्द्विजातयः}


\twolineshloka
{यस्य यज्ञेषु राजेन्द्र शतसङ्ख्येषु वै पुरा}
{देवान्मनुष्यान्गन्धर्वानत्यरिच्यन्त दक्षिणाः}


\twolineshloka
{न जातो जनिता नान्यः पुमान्यः संप्रदास्यति}
{यदङ्गः प्रददौ वित्तं सोमसंस्थासु सप्तसु}


\twolineshloka
{स चेन्ममार सृञ्जय चतुर्भद्रतरस्त्वया}
{पुत्रात्पुण्यतरश्चैव मा पुत्रमनुतप्यथाः}


\twolineshloka
{शिबिमौशीनरं चैव मृतं सृञ्जय शुश्रुम}
{य इमां पृथिवीं सर्वां चर्मवत्समवेष्टयत्}


\twolineshloka
{महता रथघोषेण पृथिवीमनुनादयन्}
{एकच्छत्रां महीं चक्रे जैत्रेणैकरथेन यः}


\twolineshloka
{यावदस्य गवाश्वं स्यादारण्यैः पशुभिः सह}
{तावतीः प्रददौ गाः स शिबिरौशीनरोऽध्वरे}


\threelineshloka
{न वोढारं धुरं तस्य कंचिन्मेने प्रजापतिः}
{न भूतं न भविष्यं च सर्वराजसु सृञ्जय}
{अन्यत्रौशीनराच्छैब्याद्राजर्षेरिन्द्रविक्रमात्}


\twolineshloka
{स चेन्ममार सृञ्जय चतुर्भद्रतरस्त्वया}
{पुत्रात्पुण्यतरश्चैव मा पुत्रमनुतप्यथाः}


% Check verse!
अदक्षिणमयज्वानं पुत्रं संस्मृत्य मा शुचः
\twolineshloka
{भरतं चैव दौष्यन्तिं मृतं सृञ्जय शुश्रुम}
{शाकुन्तलं महात्मानं भूरिद्रविणतेजसम्}


\twolineshloka
{योऽबध्नात्रिशतं चाश्वान्देवेभ्यो यमुनामनु}
{सरस्वतीं विंशतिं च गङ्गामनु चतुर्दश}


\twolineshloka
{अश्वमेधसहस्रेण राजसूयशतेन च}
{इष्टवान्स महातेजा दौष्यन्तिर्भरतः पुरा}


\twolineshloka
{भरतस्य महत्कर्म सवराजसु पार्थिवाः}
{स्वं मर्त्या इव बाहुभ्यां नानुगन्तुमशक्नुवन्}


\twolineshloka
{परं सहस्राद्योऽबध्नाद्धयान्वेदीर्वितत्य च}
{सहस्रं यत्र पझानां कण्वाय भरतो ददौ}


\twolineshloka
{स चेन्ममार सृञ्जय चतुर्भद्रतरस्त्वया}
{पुत्रात्पुण्यतरश्चैव मा पुत्रमनुतप्यथाः}


\twolineshloka
{रामं दाशरथिं चैव मृतं सृञ्जय शुश्रुम}
{योऽन्वकम्पत वै नित्यं प्रजाः पुत्रानिवौरसान्}


\twolineshloka
{नाधनो यस्य विषये नानर्थः कस्यचिद्भवेत्}
{सर्वस्यासीत्पितृसमो रामो राज्यं यदन्वशात्}


\twolineshloka
{कालवर्षी च पर्जन्यः सस्यानि समपादयत्}
{नित्यं सुभिक्षमेवासीद्रामे राज्यं प्रशासति}


\twolineshloka
{प्राणिनो नाप्सु मञ्जन्ति नानर्थे पावकोऽदहत्}
{न व्यालतो भयं चासीद्रामे राज्यं प्रशासति}


\twolineshloka
{आसन्वर्षसहस्रिण्यस्तथा वर्षसहस्रकाः}
{अरोगाः सर्वसिद्धार्था रामे राज्यं प्रशासति}


\twolineshloka
{नान्योन्येन विवादोऽभूत्स्त्रीणामपि कुतो नृणाम्}
{धर्मनित्याः प्रजाश्चासन्रामे राज्यं प्रशासति}


\twolineshloka
{संतुष्टाः सर्वसिद्धार्था निर्भयाः स्वैरचारिणः}
{नराः सत्यव्रताश्चासन्रामे राज्यं प्रशासति}


\twolineshloka
{नित्यपुष्पफलाश्चैव पादपा निरुपद्रवाः}
{सर्वा द्रोणदुघा गावो रामे राज्यं प्रशासति}


\twolineshloka
{स चतुर्दश वर्षाणि वने प्रोष्य महातपाः}
{दशाश्वमेधाञ्जारूथ्यानाजहार निरर्गलान्}


\twolineshloka
{युवा श्यामो लोहिताक्षो मातङ्ग इव यूथपः}
{आजानुबाहुः सुमुखः सिंहस्कन्धो महाभुजः}


\twolineshloka
{दशवर्षसहस्राणि दशवर्षशतानि च}
{अयोध्याधिपतिर्भूत्वा रामो राज्यमकारयत्}


\threelineshloka
{स चेन्ममार सृञ्जय चतुर्भद्रतरस्त्वया}
{पुत्रात्पुण्यतरश्चैव मा पुत्रमनुतप्यथाः}
{अयज्वानमदक्षिण्यं मा पुत्रमनुतप्यथाः}


\twolineshloka
{भगीरथं च राजानं मृतं सृञ्जय शुश्रुम्}
{यस्येन्द्रो वितते यज्ञे सोमं पीत्वा मदोत्कटः}


\twolineshloka
{असुराणां सहस्राणि बहूनि सुरसत्तमः}
{अजयद्वाहुवीर्येण भगवान्पाकशासनः}


\twolineshloka
{यः सहस्रं सहस्राणां कन्या हेमविभूषिताः}
{ईजानो वितते यज्ञे दक्षिणामत्यकालयत्}


\twolineshloka
{सर्वा रथगताः कन्या रथाः सर्वे चतुर्युजः}
{शतंशतं रथे नागाः पझिनो हेममालिनः}


\twolineshloka
{सहस्रमश्वा एकैकं हस्तिनं पृष्ठतोऽन्वयुः}
{गवां सहस्रमश्वेऽश्वे सहस्रं गव्यजाविकम्}


\twolineshloka
{उपह्वरे निवसतो यस्याङ्के निषसाद ह}
{गङ्गा भागीरथी तस्मादुर्वशी चाभवत्पुरा}


\twolineshloka
{भूरिदक्षिणमिक्ष्वाकुं यजमानं भगीरथम्}
{त्रिलोकपथगा गङ्गा दुहितृत्वमुपेयुषी}


\twolineshloka
{स चेन्ममार सृञ्जय चतुर्भद्रतरस्त्वया}
{पुत्रात्पुण्यतरश्चैव मा पुत्रमनुतप्यथाः}


\twolineshloka
{दिलीपं च महात्मानं मृतं सृञ्जय शुश्रुम}
{प्रस्य कर्माणि भूरीणि कथयन्ति द्विजातयः}


\twolineshloka
{य इमां वसुसंपूर्णां वसुधां वसुधाधिपः}
{ददौ तस्मिन्महायज्ञे ब्राह्मणेभ्यः समाहितः}


\twolineshloka
{यस्येह यजमानस्य यज्ञेयज्ञे पुरोहितः}
{सहस्रं वारणान्हैमान्दक्षिणामत्यकालयत्}


\twolineshloka
{यस्य यज्ञे महानासीद्यूपः श्रीमान्हिरण्मयः}
{ते देवां कर्म कुर्वाणाः शक्रज्येष्ठा उपासत}


\twolineshloka
{चषाले यस्य सौवर्णे तस्मिन्यूपे हिरण्मये}
{ननृतुर्देवगन्धर्वाः षट््सहस्राणि सप्तधा}


\twolineshloka
{अवादयत्तत्र वीणां मध्ये विश्वावसुः स्वयम्}
{सर्वभूतान्यमन्यन्त मम वादयतीत्ययम्}


\twolineshloka
{एतद्राज्ञो दिलीपस्य राजानो नानुचक्रिरे}
{यस्येभा हेमसंछन्नाः पथि मत्ताः स्म शेरते}


\twolineshloka
{राजानं शतधन्वानं दिलीपं सत्यवादिनम्}
{येऽपश्यन्सुमहात्मानं तेऽपि स्वर्गजितो नराः}


\twolineshloka
{त्रयः शब्दा न जीर्यन्ते दिलीपस्य निवेशने}
{स्वाध्यायशब्दः ज्याशब्दः शब्दो वै दीयतामिति}


\twolineshloka
{स चेन्ममार सृञ्जय चतुर्भद्रतरस्त्वया}
{पुत्रात्पुण्यतरश्चैव मा पुत्रमनुतप्यथाः}


\twolineshloka
{मान्धातारं यौवनाश्वं मृतं सृञ्जय शुश्रुम}
{यं देवा मरुतो गर्भं पितुः पार्श्वादपाहरन्}


\twolineshloka
{समृद्धो युवनाश्वस्य जठरे यो महात्मनः}
{पृषदाज्योद्भवः श्रीमांस्त्रिलोकविजयी नृपः}


\twolineshloka
{यं दृष्ट्वा पितुरुत्सङ्गे शयानं देवरूपिणम्}
{अन्योन्यमब्रुवन्देवाः कमयं धास्यतीति वै}


\twolineshloka
{मामेव धास्यतीत्येवमिन्द्रोऽथाभ्युपपद्यत}
{मांधातेति ततस्तस्य नाम चक्रे शतक्रतुः}


\twolineshloka
{ततस्तु पयसो धारां पुष्टिहेतोर्महात्मनः}
{तस्यास्ये यौवनाश्वस्य पाणिरिन्द्रस्य चास्रवत्}


\twolineshloka
{तं पिवन्पाणिमिन्द्रस्य शतमह्ना व्यवर्धत}
{स आसीद्द्वादशसमो द्वादशाहेन पार्थिवः}


\twolineshloka
{तमिमं पृथिवी सर्वा एकाह्ना समपद्यत}
{धर्मात्मानं महात्मानं शूरमिन्द्रसमं युधि}


\twolineshloka
{यश्चाङ्गारं तु नृपतिं मरुत्तमसितं गयम्}
{अङ्गं बृहद्रथं चैव मान्धाता समरेऽजयत्}


\twolineshloka
{यौवनाश्वो यदाङ्गारं समरे प्रत्ययुध्यत}
{विस्फारैर्धनुषो देवा द्यौरभेदीति मेनिरे}


\twolineshloka
{यत्र सूर्य उदेति स्म यत्र च प्रतितिष्ठति}
{सर्वं तद्यौवनाश्वस्य मान्धातुः क्षेत्रमुच्यते}


\twolineshloka
{अश्वमेधशतेनेष्ट्वा राजसूयशतेन च}
{अददद्रोहितान्मत्स्यान्ब्राह्मणेभ्यो विशांपते}


\twolineshloka
{हैरण्यान्यो जनोत्सेधानायतान्दशयोजनम्}
{अतिरिक्तान्द्विजातिभ्यो व्यभजंस्त्वितरे जनाः}


\twolineshloka
{स चेन्ममार सृञ्जय चतुर्भद्रतरस्त्वया}
{पुत्रात्पुण्यतरश्चैव मा पुत्रमनुतप्यथाः}


\twolineshloka
{ययातिं नाहुषं चैव मृतं सृञ्जय शुश्रुम}
{य इमां पृथिवीं कृत्स्नां विजित्य सहसागराम्}


\twolineshloka
{शम्यापातेनाभ्यतीयाद्वेदीभिश्चित्रयन्महीम्}
{ईजानः क्रतुभिर्मुख्यैः पर्यगच्छद्वसुन्धराम्}


\twolineshloka
{इष्ट्वा क्रतुसहस्रेण वाजपेयशतेन च}
{तर्पयामास विप्रेन्द्रांस्त्रिभिः काञ्चनपर्वतैः}


\twolineshloka
{व्यूढेनासुरयुद्धेन हत्वा दैतेयदानवान्}
{व्यभजत्पृथिवीं कृत्स्नां ययातिर्नहुषात्मजः}


\twolineshloka
{अन्त्येषु पुत्रान्निक्षिप्य यदुद्रुह्युपुरोगमान्}
{पुरुं राज्येऽभिषिच्याथ सदारः प्राविशद्वनम्}


\twolineshloka
{स चेन्ममार सृञ्जय चतुर्भद्रतरस्त्वया}
{पुत्रात्पुण्यतरश्चैव मा पुत्रमनुतप्यथाः}


\twolineshloka
{अम्बरीषं च नाभागं मृतं सृञ्जय शुश्रुम}
{यं प्रजा वव्रिरे पुण्यं गोप्तारं नृपसत्तमम्}


\twolineshloka
{यः सहस्रं सहस्राणां राज्ञामयुतयाजिनाम्}
{ईजानो वितते यज्ञे ब्राह्मणेभ्यस्त्वमन्यत}


\twolineshloka
{नैतत्पूर्वे जनाश्चक्रुर्न करिष्यन्ति चापरे}
{इत्यम्बरीषं नाभागिमन्वमोदन्त दक्षिणाः}


\twolineshloka
{शतं राजसहस्राणि शतं राजशतानि च}
{सर्वेऽश्वमेधैरीजानास्तेऽन्वयुर्दक्षिणायनम्}


\twolineshloka
{स चेन्ममार सृञ्जय चतुर्भद्रतरस्त्वया}
{पुत्रात्पुण्यतरश्चैव मा पुत्रमनुतप्यथाः}


\twolineshloka
{शशबिन्दुं चैत्ररथं मृतं शुश्रुम सृञ्जय}
{यस्य भार्यासहस्राणां शतमासीन्महात्मनः}


\twolineshloka
{सहस्रं तु सहस्राणां यस्यासञ्शाशबिन्दवाः}
{हिरण्यकवचाः सर्वे सर्वे चोत्तमधन्विनः}


\twolineshloka
{शतं कन्या राजपुत्रमेकैकं पृथगन्वयुः}
{कन्यांकन्यां शतं नागा नागंनागं शतं रथाः}


\twolineshloka
{रथेरथे शतं चाश्वा देशजा हेममालिनः}
{अश्वेअश्वे शतं गावो गांगां तद्वदजाविकम्}


\twolineshloka
{एतद्धनमपर्यन्तमश्वमेधे महामखे}
{शशबिन्दुर्महाराज ब्राह्मणेभ्यो ह्यमन्यत}


\twolineshloka
{स चेन्ममार सृञ्जय चतुर्भद्रतरस्त्वया}
{पुत्रात्पुण्यतरश्चैव मा पुत्रमनुतप्यथाः}


\twolineshloka
{गयं चाधूर्तरजसं मृतं शुश्रुम सृञ्जय}
{यः स वर्षशतं राजा हुतशिष्टाशनोऽभवत्}


\twolineshloka
{यस्मै वह्निर्वरान्प्रादात्ततो वव्रे वरान्गयः}
{ददतो मे क्षयो मा भूद्धर्मे श्रद्धा च वर्धताम्}


\twolineshloka
{मनो मे रभतां सत्ये त्वत्प्रसादाद्धुताशन}
{लेभे च कामांस्तान्सर्वान्पावकादिति नः श्रुतम्}


\twolineshloka
{दर्शेन पूर्णमासेन चातुर्मास्यैः पुनः पुनः}
{अयजद्धयमेधेन सहस्रं परिवत्सरान्}


\twolineshloka
{शतं गवां सहस्राणि शतमश्वशतानि च}
{उत्थायोत्थाय वै प्रादात्सहस्रं परिवत्सरान्}


\twolineshloka
{तर्पयामास सोमेन देवान्वित्तैर्द्विजानपि}
{पितॄन्स्वधाभिः कामैश्च स्त्रियः स्वाः पुरुषर्षभ}


\twolineshloka
{सौवर्णां पृथिवीं कृत्वा दशव्यामां द्विरायताम्}
{दक्षिणामददद्राजा वाजिमेधे महाक्रतौ}


\twolineshloka
{यावत्यः सिकता राजन्गङ्गायां पुरुषर्षभ}
{तावतीरेव गाः प्रादादाधूर्तरजसो गयः}


\twolineshloka
{स चेन्ममार सृञ्जय चतुर्भद्रतरस्त्वया}
{पुत्रात्पुण्यतरश्चैव मा पुत्रमनुतप्यथाः}


\twolineshloka
{रन्तिदेवं च सांकृत्यं मृतं सृञ्जय शुश्रुम}
{सम्यगाराध्य यः शक्राद्वरं लेभे महातपाः}


\twolineshloka
{अन्नं च नो बहु भवेदतिथींश्च लभेमहि}
{श्रद्धा च नो मा व्यगमन्मा च याचिष्म कंचन}


\twolineshloka
{उपातिष्ठन्त पशवः स्वयं तं संशितव्रतम्}
{ग्राम्यारण्या महात्मानं रन्तिदेवं यशस्विनम्}


\twolineshloka
{महानदी चर्मराशेरुत्क्लेदात्ससृजे यतः}
{ततश्चर्मण्वतीत्येवं विख्याता सा महानदी}


\twolineshloka
{ब्राह्मणेभ्यो ददौ निष्कान्सदसि प्रतते नृपः}
{तुभ्यंतुभ्यं निष्कमिति यदा क्रोशन्ति वै द्विजाः}


% Check verse!
सहस्रं तुभ्यमित्युक्त्वा ब्राह्मणान्संप्रपद्य ते
\threelineshloka
{अन्वाहार्योपकरणं द्रव्योपकरणं च यत्}
{घटाः पात्र्यः कटाहानि स्थाल्यश्च पिठराणि च}
{नासीत्किंचिदसौवर्णं रन्तिदेवस्य धीमतः}


\twolineshloka
{सांकृते रन्तिदेवस्य यां रात्रिमवसन्गृहे}
{आलभ्यन्त शतं गावः सहस्राणि च विंशतिः}


\twolineshloka
{तत्र स्म सूदाः क्रोशन्ति सुमृष्टमणिकुण्डलाः}
{सूपं भूयिष्ठमश्नीध्वं नाद्य मांसं यथा पुरा}


\twolineshloka
{स चेन्ममार सृञ्जय चतुर्भद्रतरस्त्वया}
{पुत्रात्पुण्यतरश्चैव मा पुत्रमनुतप्यथाः}


\twolineshloka
{सगरं च महात्मानं मृतं शुश्रुम सृञ्जय}
{ऐक्ष्वाकं पुरुषव्याघ्रमतिमानुषविक्रमम्}


\twolineshloka
{षष्टिः पुत्रसहस्राणि यं यान्तमनुजग्मिरे}
{नक्षत्रराजं वर्षान्ते व्यभ्रे ज्योतिर्गणा इव}


\twolineshloka
{एकच्छत्रा मही यस्य प्रतापादभवत्पुरा}
{योऽश्वमेधसहस्रेण तर्पयामास देवताः}


\twolineshloka
{यः प्रादात्कनकस्तम्भं प्रासादं सर्वकाञ्चनम्}
{पूर्णं पझदलाक्षीणां स्त्रीणां शयनसंकुलम्}


\twolineshloka
{द्विजातिभ्योऽनुरूपेभ्यः कामांश्च विविधान्बहून्}
{यस्यादेशेन तद्वित्तं व्यभजन्त द्विजातयः}


\twolineshloka
{खानयामास यः कोपात्पृथिवीं सागराङ्किताम्}
{यस्य नाम्ना समुद्रश्च सागरत्वमुपागतः}


\twolineshloka
{स चेन्ममार सृञ्जय चतुर्भद्रतरस्त्वया}
{पुत्रात्पुण्यतरश्चैव मा पुत्रमनुतप्यथाः}


\twolineshloka
{राजानं च पृथुं वैन्यं मृतं शुश्रुम सृञ्जय}
{यमभ्यषिञ्चन्संभूयः महारण्ये महर्षयः}


\twolineshloka
{प्रथयिष्यति वै लोकान्पृथुरित्येव शब्दितः}
{क्षताद्यो वै त्रायतीति स तस्मात्क्षत्रियः स्मृतः}


\twolineshloka
{पृथुं वैन्यं प्रजा दृष्ट्वा रक्तास्मेति यदब्रुवन्}
{ततो राजेति नामास्य अनुरागादजायत}


\twolineshloka
{अकृष्टपच्या पृथिवी पुटकेपुटके मधु}
{सर्वा द्रोणदुघा गावो वैन्यस्यासन्प्रशासतः}


\twolineshloka
{अरोगाः सर्वसिद्धार्था मनुष्या अकुतोभयाः}
{यथाऽभिकाममवसन्क्षेत्रेषु च गृहेषु च}


\twolineshloka
{आपस्तस्तम्भिरे चास्य समुद्रमभियास्यतः}
{शैलाश्चापाद्व्यदीर्यन्त ध्वजभङ्गश्च नाभवत्}


\twolineshloka
{हैरण्यांस्त्रिनरोत्सेधान्पर्वतानेकविंशतिम्}
{ब्राह्मणेभ्यो ददौ राजा योश्वमेधे महामखे}


\twolineshloka
{स चेन्ममार सृञ्जय चतुर्भद्रतरस्त्वया}
{पुत्रात्पुण्यतरश्चैव मा पुत्रमनुतप्यथाः}


\threelineshloka
{किं वा तूष्णीं ध्यायसे सृञ्जय त्वंन मे राजन्वाचिममां शृणोपि}
{न चेन्मोघं विप्रलप्तं ममेदंपथ्यं मुमूर्षोरिव सुप्रयुक्तम् ॥सृञ्जय उवाच}
{}


\twolineshloka
{शृणोमि ते नारद वाचमेनांविचित्रार्थां स्रजमिव पुण्यगन्धाम्}
{राजर्षीणां पुण्यकृतां महात्मनांकीर्त्या युक्तानां शोकनिर्नाशनार्थाम्}


\twolineshloka
{न ते मोघं विप्रलप्तं महर्षेदृष्ट्वैवाहं नारद त्वां विशोकः}
{शुश्रूषे ते वचनं ब्रह्मवादिन्न ते तृप्याम्यमृतस्येव पानात्}


\twolineshloka
{अमोघदर्शिन्मम चेत्प्रसादंसंतापदग्धस्य विभो प्रकुर्याः}
{सुतस्य संजीवनमद्य मे स्यात्तव प्रसादात्सुतसङ्गमाप्नुयाम्}


\threelineshloka
{नारद उवाच}
{यस्ते पुत्रः शयितोयं विजातःस्वर्णष्ठीवी यमदात्पर्वतस्ते}
{पुनस्तं ते पुत्रमहं ददामिहिरण्यनाभं वर्षसबस्रिणं च}


\chapter{अध्यायः २९}
\twolineshloka
{युधिष्ठिर उवाच}
{}


\twolineshloka
{स कथं काञ्चनष्ठीवी सृञ्जयस्य सुतोऽभवत्}
{पर्वतेन किमथे वा दत्तस्तेन ममार च}


\twolineshloka
{यदा वर्षसहस्रायुस्तदा भवति मानवः}
{कथमप्राप्तकौमारः सृञ्जयस्य सुतो मृतः}


\threelineshloka
{उताहो नाममात्रं वै सुवर्णष्ठीविनोऽभवत्}
{कथं वा काञ्चनष्ठीवीत्येतदिच्छामि वेदितुम् ॥श्रीकृष्ण उवाच}
{}


\twolineshloka
{अत्र ते वर्णयिष्यामि यथावृत्तं जनेश्वर}
{नारदः पर्वतश्चैव द्वावृवी लोकसत्तमौ}


\twolineshloka
{मातुलो भागिनेयश्च देवलोकादिहागतौ}
{विहर्तुकामौ संप्रीत्या मानुषेषु पुरा विभो}


\twolineshloka
{हविःपवित्रभोज्येन देवभोज्येन चैव हि}
{नारदो मातुलस्तत्र भागिनेयश्च पर्वतः}


\twolineshloka
{तावुभौ तपसोपेताववनीतलचारिणौ}
{भुञ्जानौ मानुषान्भोगान्यथावत्पर्यधावताम्}


\twolineshloka
{प्रीतिमन्तौ मुदा युक्तौ समयं चैव चक्रतुः}
{यो भवेद्धृदि संकल्पः शुभो वा यदि वाऽशुभः}


\twolineshloka
{अन्योन्यस्य च आख्येयो मृषा शापोऽन्यथा भवेत्}
{तौ तथेति प्रतिज्ञाय महर्षी लोकपूजितौ}


\twolineshloka
{सृञ्जयं श्वैत्यमभ्येत्य राजानमिदमूचतुः}
{आवां भवति वत्स्यावः कंचित्कालं हिताय ते}


\twolineshloka
{यथावत्पृथिवीपाल आवयोः प्रगुणीभव}
{तथेति कृत्वा राजा तौ सत्कृत्योपचचार ह}


\twolineshloka
{ततः कदाचित्तौ राजा महात्मानौ तपोधनौ}
{अब्रवीत्परमप्रीतः सुतेयं देवरूपिणी}


\threelineshloka
{एकैव मम कन्यैषा युवां परिचरिष्यति}
{दर्शनीयानवद्याङ्गी शीलवृत्तसमाहिता}
{सुकुमारी कुमारी च पझकिञ्जल्कसुप्रभा}


\twolineshloka
{परमं सौम्यमित्युक्तं ताभ्यां राजा शशास ताम्}
{कन्ये विप्रावुपचर देववत्पितृवच्च ह}


\twolineshloka
{सा तु कन्या तथेत्युक्त्वा पितरं धर्मचारिणी}
{यथानिदेशं राज्ञस्तौ सत्कृत्योपचचार ह}


\twolineshloka
{तस्यास्तेनोपचारेण रूपेणाप्रतिमेन च}
{नारदं हृच्छयस्तूर्णं सहसैवाभ्यपद्यत}


\twolineshloka
{ववृधे हि ततस्तस्य हृदि कामो महात्मनः}
{यथा शुक्लस्य पक्षस्य प्रवृत्तौ चन्द्रमाः शनैः}


\twolineshloka
{न च तं भागिनेयाय पर्वताय महात्मने}
{शशंस मन्मथं तीव्रं व्रीडमानः स धर्मवित्}


\twolineshloka
{तपसा चेङ्गितैश्चैव पर्वतोऽथ बुबोध तम्}
{कामार्तं नारदं क्रुद्धः शशापैनं ततो भृशम्}


\twolineshloka
{कृत्वा समयमव्यग्रो भवान्वै सहितो मया}
{यो भवेद्धृदि संकल्पः शुभो वा यदि वाऽशुभः}


\twolineshloka
{अन्योन्यस्य स आख्येय इति तद्वै मृषा कृतम्}
{भवता वचनं ब्रह्मंस्तस्मादेष शपाम्यहम्}


\twolineshloka
{न हि कामं प्रवर्तन्तं भवानाचष्ट मे पुरा}
{सुकुमार्यां कुमार्यां ते तस्मान्नैष क्षमाम्यहम्}


\twolineshloka
{ब्रह्मचारी गुरुर्यस्मात्तपस्वी ब्राह्मणश्च सन्}
{अकार्षीः समयभ्रंशमावाभ्यां यः कृतो मिथः}


% Check verse!
शप्स्ये तस्मात्सुसंक्रुद्धो भवन्तं तं निबोध मे
\threelineshloka
{सुकुमारी च ते भार्या भविष्यति न संशयः}
{वानरत्वं च ते कन्या विवाहात्प्रभृति प्रभो}
{संद्रक्ष्यन्ति नराश्चान्ये स्वरूपेण विनाकृतम्}


\twolineshloka
{स तद्वाक्यं तु विज्ञाय नारदः पर्वतं तथा}
{अशपत्तमपि क्रोधाद्भागिनेयं स मातुलः}


\twolineshloka
{तपसा ब्रह्मचर्येण सत्येन च दमेन च}
{युक्तोऽपि नित्यधर्मश्च न वै स्वर्गमवाप्स्यसि}


\twolineshloka
{तौ तु शावा भृशं क्रुद्धौ परस्परममर्षणौ}
{प्रतिजग्मतुहृन्योन्यं क्रुद्धाविव गजोत्तमौ}


\twolineshloka
{पर्वतः पृथिवीं कृत्स्नां विचचार महामतिः}
{पूज्यमानो यथान्यांयं तेजसा स्वेन भारत}


\twolineshloka
{अथ तामलभत्कन्यां नारदः सृञ्जयात्मजाम्}
{धर्मेण विप्रप्रवरः सुकुमारीमनिन्दिताम्}


\twolineshloka
{सा तु कन्या यथाशापं नारदं तं ददर्श ह}
{पाणिग्रहणमन्त्राणां नियोगादेव नारदम्}


\twolineshloka
{सुकुमारी च देवर्षि वानरप्रतिमाननम्}
{नैवावमन्यत तदा प्रीतिमत्येव चाभवत्}


\twolineshloka
{उपतस्थे च भर्तारं न चान्यं मनसाऽप्यगात्}
{देवं मुनीं वा यक्षं वा पतित्वे पतिवत्सला}


\twolineshloka
{ततः कदाचिद्भगवान्पर्वतोऽनुचचार ह}
{वनं विरहितं किंचित्तत्रापश्यत्स नारदम्}


\twolineshloka
{ततोऽभिवाद्य प्रोवाच नारदं पर्वतस्तदा}
{भवान्प्रसादं कुरुतात्स्वर्गादेशाय ये प्रभो}


\twolineshloka
{तमुवाच ततो दृष्ट्वा पर्वतं नारदस्तथा}
{कृताञ्जलिमुपासीनं दीनं दीनतरः स्वयम्}


\twolineshloka
{त्वयाऽहं प्रथमं शप्तो वानरस्त्वं भविष्यसि}
{इत्युक्तेन मया पश्चाच्छप्तस्तवमपि मत्सरात्}


\twolineshloka
{अद्यप्रभृति वै वासं स्वर्गे नावाप्स्यसीति ह}
{तव नैतद्विसदृशं पुत्रस्थाने हि मे भवान्}


% Check verse!
निवर्तयेतां तौ शापावन्योन्येन तदा मुनी
\twolineshloka
{श्रीसमृद्धं तदा दृष्ट्वा नारदं देवरूपिणम्}
{सुकुमारी प्रदुद्राव परपुंसविशङ्कया}


\twolineshloka
{तां पर्वतस्ततो दृष्ट्वा प्रद्रवन्तीमनिन्दिताम्}
{अब्रवीत्तव भर्तैष नात्र कार्या विचारणा}


\twolineshloka
{ऋषिः परमधर्मात्मा नारदो भगवान्प्रभुः}
{तवैवाभेद्यहृदयो मा ते भूदत्र संशयः}


\fourlineindentedshloka
{सानुनीता बहुविधं पर्वतेन महात्मना}
{शापदोषं च तं भर्तुः श्रुत्वा प्रकृतिमागता}
{पर्वतोऽथ ययौ स्वर्गं नारदोऽभ्यगमद्गृहान् ॥वासुदेव उवाच}
{}


\twolineshloka
{प्रत्यक्षकर्ता सर्वस्य नारदो भगवानृषिः}
{एष वक्ष्यति ते पृष्टो यथावृत्तं नरोत्तम}


\chapter{अध्यायः ३०}
\twolineshloka
{वैशंपायन उवाच}
{}


\twolineshloka
{ततो राजा पाण्डुसुतो नारदं प्रत्यभाषत}
{भगवञ्छ्रोतुमिच्छामि सुवर्णष्ठीविसंभवम्}


\threelineshloka
{एवमुक्तस्तु स मुनिर्धर्मराजेन नारदः}
{आचचक्षे यथावृत्तं सुवर्णष्ठीविनं प्रति ॥नारद उवाच}
{}


\twolineshloka
{एवमेतन्महाबाहो यथाऽयं केशवोऽब्रवीत्}
{कार्यस्यास्य तु यच्छेषं तत्ते वक्ष्यामि पृच्छतः}


\twolineshloka
{अहं च पर्वतश्चैव स्वस्रीयो मे महाप्नुनिः}
{वस्तुकामावभिगतौ सृञ्जयं जयतां वरम्}


\twolineshloka
{तत्रावां पूजितौ तेन विधिदृष्टेन कर्मणा}
{सर्वकामैः सुविहितौ निवसावोऽस्य वेश्मनि}


\twolineshloka
{व्यतिक्रान्तासु वर्षासु समये गमनस्य च}
{पर्वतो मामुवाचेदं काले वचनमर्थवत्}


\twolineshloka
{आवामस्य नरेन्द्रस्य गृहे परमपूजितौ}
{उषितौ समये ब्रह्मंस्तद्विचिन्तय सांप्रतम्}


\twolineshloka
{ततोऽहमब्रुवं राजन्पर्वतं सुभदर्शनम्}
{सर्वमेतत्त्वयि विभो भागिनेयोपपद्यते}


\twolineshloka
{वरेण च्छन्द्यतां राजा लभतां यद्यदिच्छति}
{आवयोस्तपसा सिद्धिं प्राप्नोतु यदि मन्यसे}


\twolineshloka
{तत आहूय राजानं सृञ्जयं जयतां वरम्}
{पर्वतोऽनुमतो वाक्यमुवाच कुरुपुङ्गव}


\twolineshloka
{प्रीतौ स्वो नृप सत्कारैर्भवदार्जवसंभृतैः}
{आवाभ्यामभ्यनुज्ञातो वरं नृवर चिन्तय}


\threelineshloka
{देवानामविहिंसायां न भवेन्मानुषे क्षमम्}
{तद्गृहाण महाराज पूजार्हो नौ मतो भवान् ॥सृञ्जय उवाच}
{}


\twolineshloka
{प्रीतौ भवन्तौ यदि मे कृतमेतावता मम}
{एष एव परो लाभो निर्वृत्तो मे महाफलः}


\twolineshloka
{तमेवंवादिनं भूयः पर्वतः प्रत्यभाषत}
{शृणु राजन्सुसंकल्पं यत्ते हृदि चिरं स्थितम्}


\twolineshloka
{अभीप्ससि सुतं वीरं वीर्यवन्तं दृढव्रतम्}
{आयुष्मतं महाभागं देवराजसमद्युतिम्}


\twolineshloka
{भविष्यत्येष ते कामो न त्वायुष्मान्भविष्यति}
{देवराजाभिभूत्यर्थं संकल्पो ह्ये ते हृदि}


\twolineshloka
{सुवर्णष्ठीवनाच्चैव स्वर्णष्ठीवी भविष्यति}
{रक्ष्यश्च देवराजात्स देवराजसमद्युतिः}


\twolineshloka
{तच्छ्रुत्वा सृञ्जयो वाक्यं पर्वतस्य महात्मनः}
{प्रसादयामास तदा नैतदेवं भवेदिति}


\twolineshloka
{आयुष्मान्मे भवेत्पुत्रो भवतोस्तपस मुन}
{न च तं पर्वतः किंचिदुवाचेन्द्रव्यपेक्षया}


\twolineshloka
{तमहं नृपतिं दीनमब्रवं पुनरेव च}
{स्मर्तव्योऽस्मि महाराज दर्शयिष्यामि ते सुतम्}


\twolineshloka
{अहं ते दयितं पुत्रं प्रेतराजवशं गतम्}
{पुनर्दास्यामि तद्रूपं मा शुचः पृथिवीपते}


\twolineshloka
{एवमुक्त्वा तु नृपतिं प्रयातौ स्वो यथेप्सितम्}
{सृञ्जयश्च यथाकामं प्रविवेश स्वमन्दिरम्}


\twolineshloka
{सृञ्जयस्याथ राजर्षेः कस्मिंश्चित्कालपर्यये}
{जज्ञे पुत्रो महावीर्यस्तेजसा प्रज्वलन्निव}


\twolineshloka
{ववृधे स यथाकालं सरसीव महोत्पलम्}
{बभूव काञ्चनष्ठीवी यथार्थं नाम तस्य तत्}


\twolineshloka
{तदद्भुततमं लोके पप्रथे कुरुसत्तम}
{बुबुधे तच्च देवेन्द्रो वरदानं मनीषिणोः}


\twolineshloka
{ततः स्वाभिभवाद्भीतो बृहस्पतिमते स्थितः}
{कुमारस्यान्तरप्रेक्षी नित्यमेवाभ्यवर्तत}


\twolineshloka
{चोदयामास तद्वज्रं दिव्यास्रं मूर्तिमत्स्थ्रितम्}
{व्याघ्रो भूत्वा जहीमं त्वं राजपुत्रमिति प्रभो}


\twolineshloka
{प्रवृद्धः किल वीर्येण मामेषोऽभिभविष्यति}
{सृञ्जयस्य सुतो वज्र यथैनं पर्वतोऽब्रवीत्}


\twolineshloka
{एवमुक्तस्तु शक्रेण वज्रः परपुरंजयः}
{कुमारमन्तरप्रेक्षी नित्यमेवान्वपद्यत}


\twolineshloka
{सृञ्जयोऽपि सुतं प्राप्य देवराजसमद्युतिम्}
{हृष्टः सान्तः पुरो राजा वननित्यो बभूव ह}


\twolineshloka
{ततो भागीरथीतीरे कदाचिन्निर्जने वने}
{धात्रीद्वितीयो बालः स क्रीडार्थं पर्यधावत}


\twolineshloka
{पञ्चवर्षकदेशीयो बालो नागेन्द्रविक्रमः}
{सहसोत्पतितं व्याघ्रमाससाद महाबलम्}


\twolineshloka
{स बालस्तेन निष्पिष्टो वेपमानो नृपात्मजः}
{व्यसुः पपात मेदिन्यां ततो धात्री विचुक्रुशे}


\twolineshloka
{हत्वा तु राजपुत्रं स तत्रैवान्तरधीयत}
{शार्दूलो देवराजस्य माययान्तर्हितस्तदा}


\twolineshloka
{धात्र्यास्तु निनदं श्रुत्वा रुदत्याः परमार्तवत्}
{अभ्यधावत तं देशं स्वयमेव महीपतिः}


\twolineshloka
{स ददर्श शयानं तं गतासुं पीतशोणितम्}
{कुमारं विगतानन्दं निशाकरमिव च्युतम्}


\twolineshloka
{स तमुत्सङ्गमारोप्य परिपीडितवक्षसम्}
{पुत्रं रुधिरसंसिक्तं पर्यदेवयदातुरः}


\twolineshloka
{ततस्ता मातरस्तस्य रुदत्यः शोककर्शिताः}
{अभ्यधावन्त तं देशं यत्र राजा स सृञ्जयः}


\twolineshloka
{ततः स राजा सस्मार मामेव गतमानसः}
{तदाऽहं चिन्तनं ज्ञात्वा गतवांस्तस्य दर्शनम्}


\twolineshloka
{मयैतानि च वाक्यानि श्रावितः शोकलालसः}
{यानि ते यदुवीरेण कथितानि महीपते}


\twolineshloka
{संजीवितश्चापि पुनर्वासवानुमते तदा}
{भवितव्यं तथा तच्च न तच्छक्यमतोऽन्यथा}


\twolineshloka
{तत ऊर्ध्वं कुमारस्तु स्वर्णष्ठीवी महायशाः}
{चित्तं प्रसादयामास पितृर्मातुश्च वीर्यवान्}


\twolineshloka
{कारयामास राज्यं च पितरि स्वर्गते नृप}
{वर्षाणां शतमेकं च सहस्रं भीमविक्रमः}


\twolineshloka
{तत ईजे महायज्ञैर्बहुभिर्भूरिदक्षिणैः}
{तर्पयामास देवांश्च पितॄंश्चैव महाद्युतिः}


\twolineshloka
{उत्पाद्य च बहून्पुत्रान्कुलसतानकारिणः}
{कालेन महता राजन्कालधर्ममुपेयिवान्}


\twolineshloka
{स त्वं राजेन्द्र संजातं शोकमेकं निवर्तय}
{यथा त्वं केशवः प्राह व्यामश्च सुमहातपाः}


\twolineshloka
{पितृपैतामहं राज्यमास्थाय धुरमुद्वह}
{इष्ट्वा पुण्यैर्महायज्ञैरिष्टं लोकमवाप्स्यसि}


\chapter{अध्यायः ३१}
\twolineshloka
{वैशंपायन उवाच}
{}


\threelineshloka
{तूष्णींभूतं तु राजानं शोचमानं युधिष्ठिरम्}
{तपस्वी धर्मतत्त्वज्ञः कृष्णद्वैपायनोऽब्रवीत् ॥व्यास उवाच}
{}


\twolineshloka
{प्रजानां पालनं धर्मो राज्ञां राजीवलोचन}
{धर्मः प्रमाणं लोकस्य नित्यं धर्मोऽनुवर्त्यताम्}


\twolineshloka
{अनुतिष्ठस्व तद्राजन्पितृपैतामहं पदम्}
{ब्राह्मणेषु तु यो धर्मः स नित्यो वेदनिश्चितः}


\twolineshloka
{तत्प्रमाणं प्रमाणानां शाश्वतं भरतर्षभ}
{तस्य धर्मस्य कृत्स्नस्य क्षत्रियः परिरक्षिता}


\twolineshloka
{तथा यः प्रतिहन्त्यस्य शासनं विषये नरः}
{स बाहुभ्यां विनिग्राह्यो लोकयात्राविघातकः}


\twolineshloka
{प्रमाणमप्रमाणं यः कुर्यान्मोहवशं गतः}
{भृत्यो वा यदि वा पुत्रस्तपस्वी वाऽथ कश्चन}


\twolineshloka
{पापान्सर्वैरुपायैस्तान्नियच्छेच्छातयीत वा}
{अतोऽन्यथा वर्तमानो राजा प्राप्नोति किल्विषम्}


\twolineshloka
{धर्मं विनश्यमानं हि यो न रक्षेत्स धर्महा}
{ते त्वया धर्महन्तारो निहताः सपदानुगाः}


\threelineshloka
{स्वधर्मे वर्तमानस्त्वं किंनु शोचसि पाण्डव}
{राजा हि हन्याद्दद्याच्च प्रजा रक्षेच्च धर्मतः ॥युधिष्ठिर उवाच}
{}


\twolineshloka
{न तेऽतिशङ्के वचनं यद्ब्रवीषि तपोधन}
{अपरोक्षो हि ते धर्मः सर्वधर्मविदां वर}


\threelineshloka
{मया त्ववध्या बहवो घातिता राज्यकारणात्}
{तानि कर्माणि मे ब्रह्मन्दहन्ति च पचन्ति च ॥व्यास उवाच}
{}


\twolineshloka
{ईश्वरो वा भवेत्कर्ता पुरुषो वाऽपि भारत}
{हठो वा वर्तते लोके कर्मजं वा फलं स्मृतम्}


\twolineshloka
{ईश्वरेण नियुक्तो हि साध्वसाधु च भारत}
{कुरुते पुरुषः कर्म फलमीश्वरगामि तत्}


\twolineshloka
{यथाहि पुरुषश्छिन्द्याद्वृक्षं परशुना वने}
{छेत्तुरेव भवेत्पापं परशोर्न कथंचन}


\twolineshloka
{अथवा तदुपादानात्प्राप्नुयात्कर्मणः फलम्}
{दण्डशस्त्रकृतं पापं पुरुषे तन्न विद्यते}


\twolineshloka
{न चैतदिष्टं कौन्तेय यदन्येन कृतं फलम्}
{प्राप्नुयादिति तस्माच्च ईश्वरे तन्निवेशय}


\twolineshloka
{अथापि पुरुषः कर्ता कर्मणोः शुभपापयोः}
{न परो विद्यते तस्मादेवमप्यशुभं कुतः}


\twolineshloka
{न हि कश्चित्क्वचिद्राजन्दिष्टं प्रतिनिवर्तते}
{दण्डशस्त्रकृतं पापं पुरुषे तन्न विद्यते}


\twolineshloka
{यदि वा मन्यसे राजन्हतमेकं प्रतिष्ठितम्}
{एवमप्यशुभं कर्म न भूतं न भविष्यति}


\twolineshloka
{अथाभिपत्तिर्लोकस्य कर्तव्या पुण्यपापयोः}
{अभिपन्नमिदं लोके राज्ञामुद्यतदण्डनम्}


\twolineshloka
{तथापि लोके कर्माणि समावर्तन्ति भारत}
{शुभाशुभफलं चैते प्राप्नुवन्तीति मे मतिः}


\twolineshloka
{एवं पश्य शुभादेशं कर्मणस्तत्फलं ध्रुवम्}
{त्यज त्वं राजशार्दूल मैवं शोके मनः कृथा}


\twolineshloka
{स्वधर्मे वर्तमानस्य सापवादेऽपि भारत}
{एवमात्मपरित्यागस्तव राजन्न शोभनः}


\twolineshloka
{विहितानि हि कौन्तेय प्रायश्चित्तानि कर्मणाम्}
{शरीरवांस्तानि कुर्यादशरीरः पराभवेत्}


\threelineshloka
{तद्राजञ्जीवमानस्त्वं प्रायश्चित्तं करिष्यसि}
{प्रायश्चित्तमकृत्वा तु प्रेत्य तप्ताऽसि भारत ॥युधिष्ठिर उवाच}
{}


\twolineshloka
{हताः पुत्राश्च पौत्राश्च भ्रातरः पितरस्तथा}
{श्वशुरा गुरवश्चैव मातुलाश्च पितामहाः}


\twolineshloka
{क्षत्रियाश्च महात्मानः संबन्धिसुहृदस्तथा}
{वयस्या भागिनेयाश्च ज्ञातयश्च पितामह}


\twolineshloka
{बहवश्च मनुष्येन्द्रा नानादेशसमागताः}
{घातिता राज्यलुब्धेन मयैकेन पितामह}


\twolineshloka
{तांस्तादृशानहं हत्वा धर्मनित्यान्महीक्षितः}
{असकृत्सोमपान्वीरान्क्रिं प्राप्स्यामि तपोधन}


\twolineshloka
{दह्याम्यनिशमद्यापि चिन्तयानः पुन पुनः}
{हीनां पार्थिवसिंहैस्तैः श्रीमद्भिः पृथिवीमिमाम्}


\twolineshloka
{दृष्ट्वा ज्ञातिवधं घोरं हतांश्च शतशः परान्}
{कोटिशश्च नरानन्यान्परितप्ये पितामह}


\twolineshloka
{का नु तासां वरस्त्रीणामवस्थाऽद्य भविष्यति}
{विहीनानां तु तनयैः पतिभिर्भ्रातृभिस्तथा}


\twolineshloka
{अस्मानन्तकरान्घोरान्पाण्डवान्वृष्णिसंहतान्}
{आक्रोशन्त्यः कृशा दीनाः प्रपतिष्यन्ति भूतले}


\twolineshloka
{अपश्यन्त्यः पितॄन्भ्रातॄन्पतीन्पुत्रांश्च योषितः}
{त्यक्त्वा प्राणान्स्त्रियः सर्वागमिष्यन्ति यमक्षयम्}


\twolineshloka
{वत्सलत्वाद्द्विजश्रेष्ठ तत्र ये नास्ति संसयः}
{व्यक्तं सौक्ष्म्याच्च धर्मस्य प्राप्स्यामः स्त्रीवधं वयम्}


\twolineshloka
{ते वयं सुहृदो हत्वा कृत्वा पापमनन्तकम्}
{नकरे निपतिष्यामो ह्यधः शिरस एव ह}


\twolineshloka
{शरीराणि विमोक्ष्यामस्तपसोग्रेण सत्तम}
{आश्रमाणां विशेषं त्वमथाचक्ष्व पितामह}


\chapter{अध्यायः ३२}
\twolineshloka
{वैशंपायन उवाच}
{}


\threelineshloka
{युधिष्ठिरस्य तद्वाक्यं श्रुत्वा द्वैपायनस्तदा}
{परीक्ष्य निपुणं बुद्ध्या ऋषिः प्रोवाच पाण्डवम् ॥व्यास उवाच}
{}


\twolineshloka
{मा विषादं कृथा राजन्क्षत्रधर्ममनुस्मरन्}
{स्वधर्मेण हता ह्येते क्षत्रियाः क्षत्रियर्षभ}


\twolineshloka
{काङ्क्षमाणाः श्रियं कृत्स्नां पृथिव्यां च मबद्यशः}
{कृतान्तविधिसंयुक्ताः कालेन निधनं गताः}


\twolineshloka
{नृ त्वं हन्ता न भीमोऽयं नार्जुनो न यमावपि}
{कालः पर्यायधर्मेण प्राणानादत्त देहिनाम्}


\twolineshloka
{न तस्य मातापितरौ नानुग्राह्यो हि कश्चन}
{कर्मसाक्षी प्रजानां यस्तेन कालेन संहृताः}


\twolineshloka
{हेतुमात्रमिदं तस्य विहितं भरतर्षभ}
{यद्धन्ति भूतैर्भूतानि तदस्मै रूपमैश्वरम्}


\twolineshloka
{कर्म मूर्त्यात्मकं विद्धि साक्षिणं शुभपापयोः}
{सुखदुःखगुणोदर्कं कालं कालफलप्रदम्}


\twolineshloka
{तेषामपि महाबाहो कर्माणि परिचिन्तय}
{विनाशहेतुकानि त्वं यैस्तै कालवशं गताः}


\twolineshloka
{आत्मनश्च विजानीहि नियतव्रतशीलताम्}
{यदा त्वमीदृशं कर्म विधिनाऽऽक्रम्य कारितः}


\twolineshloka
{त्वष्ट्रेव विहितं यन्त्रं यथा चेष्टयितुर्वशे}
{कर्मणा कालयुक्तेन तथेदं भ्राम्यते जगत्}


\twolineshloka
{पुरुषस्य हि दृष्ट्वेमामुत्पत्तिमनिमित्ततः}
{यदृच्छया विनाशं च शोकहर्षावनर्थकौ}


\twolineshloka
{व्यलीकमपि यत्त्वत्र चित्तवैतंसिकं तव}
{तदर्थमिष्यते राजन्प्रायश्चित्तं तदाचर}


\twolineshloka
{इदं तु श्रूयते पार्थ युद्धे देवासुरे पुरा}
{असुरा भ्रातरो ज्येष्ठा देवाश्चापि यवीयसः}


\twolineshloka
{तेषामपि श्रीनिमित्तं महानासीत्समुच्छ्रयः}
{युद्धं वर्षसहस्राणि द्वात्रिंशदभवत्किल}


\twolineshloka
{एकार्णवां महीं कृत्वा रुधिरेण परिप्लुताम्}
{जघ्नुर्दैत्यांस्तथा देवास्त्रिदिवं चाभिलेभिरे}


\twolineshloka
{तथैव पृथिवीं लब्ध्वा ब्राह्मणा वेदपारगाः}
{संश्रिता दानवानां वै साह्यार्थं दर्पमोहिताः}


\twolineshloka
{शालावृका इति ख्यातास्त्रिषु लोकेषु भारत}
{अष्टाशीतिसहस्राणि ते चापि विबुर्धैर्हताः}


\twolineshloka
{धर्मव्युच्छित्तिमिच्छतो येऽधर्मस्य प्रवर्तकाः}
{हन्तव्यास्ते दुरात्मानो देवैर्दैत्या इवोल्वणाः}


\twolineshloka
{एकं हत्वा यदि कुले शिष्टानां स्यादनामयम्}
{कुलं हत्वा च राष्ट्रे च न तद्वृत्तोपघातकम्}


\twolineshloka
{अधर्मरूपो धर्मो हि कश्चिदस्ति नराधिप}
{धर्मरूपो ह्यधर्मश्च तच्च ज्ञेयं विपश्चिता}


\twolineshloka
{तस्मात्संस्तम्भयात्मानं श्रुतवानसि पाण्डव}
{देवैः पूर्वगतं मार्गमनुयातोऽसि भारत}


\twolineshloka
{न हीदृशा गमिष्यन्ति नरकं पाण्डवर्षभ}
{भ्रातॄनाश्वासयैतांस्त्वं सुहृदश्च परंतप}


\twolineshloka
{यो हि पापसमारम्भे कार्ये तद्भावभावितः}
{कुर्वन्नपि तथैव स्यात्कृत्वा च निरपत्रपः}


\twolineshloka
{तस्मिंस्तत्कलुषं सर्वं समस्तमिति शब्दितम्}
{प्रायश्चित्तं न तस्यास्ति ह्रासो वा पापकर्मणः}


\twolineshloka
{त्वं तु शुक्लाभिजातीयः परदोषेण कारितः}
{अनिच्छमानः कर्मेदं कृत्वा च परितप्यसे}


\twolineshloka
{अश्वमेधो महायज्ञः प्रायश्चित्तमुदाहृतम्}
{तमाहर महाराज विपाप्मैवं भविष्यसि}


\twolineshloka
{मरुद्भिः सह जित्वाऽरीन्भगवान्पाकशासनः}
{एकैकं क्रतुमाहृत्य शतकृत्वः शतक्रतुः}


\twolineshloka
{धूतपाप्मा जितस्वर्गो लोकान्प्राप्य सुखोदयान्}
{मरुद्गणैर्वृतः शक्रः शुशुभे भासयन्दिशः}


\twolineshloka
{स्वर्गे लोके महीयन्तमप्सरोभिः शचीपतिम्}
{ऋषयः पर्युपासन्ते देवाश्च विबुधेश्वरम्}


\twolineshloka
{सेयं त्वामनुसंप्राप्ता विक्रमेण वसुंधरा}
{निर्जिताश्च महीपाला विक्रमेण त्वयाऽनध}


\twolineshloka
{तेषां पुराणि राष्ट्राणि गत्वा राजन्सुहृद्वॄतः}
{भ्रातॄन्पुत्रांश्च पौत्रांश्च स्वेस्वे राज्येऽभिषेचय}


\twolineshloka
{बालानपि च गर्भस्थान्सांत्वेन समुदाचरन्}
{रञ्जयन्प्रकृतीः सर्वाः परिपाहि वसुंधराम्}


\twolineshloka
{कुमारो नास्ति येषां च कन्यास्तत्राभिषेचय}
{कामाशयो हि स्त्रीवर्गः शोकमेवं प्रहास्यसि}


\twolineshloka
{एवमाश्वासनं कृत्वा सर्वराष्ट्रेषु भारत}
{यजस्व वाजिमेधेन यथेन्द्रो विजयी पुरा}


\twolineshloka
{अशोच्यास्ते महात्मानः क्षत्रियाः क्षत्रियर्षभ}
{स्वकर्मभिर्गता नाशं कृतान्तबलमोहिताः}


\twolineshloka
{अवाप्तः क्षत्रधर्मस्ते राज्यं प्राप्तमकण्टकम्}
{रक्ष स्वधर्मं कौन्तेय श्रेयान्यः प्रेत्यभाविकः}


\chapter{अध्यायः ३३}
\twolineshloka
{युधिष्ठिर उवाच}
{}


\threelineshloka
{कानि कृत्वेह कर्माणि प्रायश्चित्तीयते नरः}
{किं कृत्वा मुच्यते तत्र तन्मे ब्रूहि पितामह ॥व्यास उवाच}
{}


\twolineshloka
{अकुर्वन्विहितं कर्म प्रतिषिद्धानि चाचरन्}
{प्रायश्चित्तीयते ह्येवं नरो मिथ्याऽनुवर्तयन्}


\twolineshloka
{सूर्येणाभ्युदितो यश्च ब्रह्मचारी भवत्युत}
{तथा सूर्याभिनिर्मुक्तः कुनखी श्यावदन्नपि}


\twolineshloka
{परिवित्तिः परिवेत्ता ब्रह्मेज्यायाश्च दूषकः}
{दिधिषूपतिस्तथा यः स्यादग्रेदिधिषुरेव च}


\twolineshloka
{अवकीर्णी भवेद्यश्च द्विजातिवधकस्तथा}
{अतीर्थे ब्राह्मणस्त्यागी तीर्थे चाप्रतिपादकः}


\twolineshloka
{ग्रामयाजी च कौन्तेय मांसस्य परिविक्रयी}
{यश्चाग्नीनपविध्येत तथैव ब्रह्मविक्रयी}


\twolineshloka
{शूद्रस्त्रीवधको यश्च पूर्वः पूर्वस्तु गर्हितः}
{वृथा पशुसमालम्भी वनदाहस्य कारकः}


\twolineshloka
{अनृतेनोपवर्ती च प्रतिषेद्धा गुरोस्तथा}
{`स्वदत्तस्यापहर्ता च परदत्तनिरोधकः}


\twolineshloka
{वाग्दत्तं च मनोदत्तं धारादत्तं च यो हरेत्}
{पाकभेदेन भोक्ता च भुञ्जानस्याप्यनादरः}


\twolineshloka
{स्वजनैः कलहं चैव आश्रितानामरक्षणम्}
{'एतान्येनांसि सर्वाणि व्युत्क्रान्तसमयश्च यः}


\twolineshloka
{अकार्याणि तु वक्ष्यामि यानि तानि निबोध मे}
{लोकवेदविरुद्धानि तान्येकाग्रमनाः शृणु}


\twolineshloka
{स्वधर्मस्य परित्यागः परधर्मस्य च क्रिया}
{अयाज्ययाजनं चैव तथाऽभक्ष्यस्य भक्षणम्}


\twolineshloka
{शरणागतसंत्यागो भृत्यस्याभरणं तथा}
{रसानां विक्रयश्चापि तिर्यग्योनिवधस्तथा}


\twolineshloka
{आधानादीनि कर्माणि शक्तिमान्न करोति यः}
{अप्रयच्छंश्च सर्वाणि नित्यदेयानि भारत}


\twolineshloka
{दक्षिणानामदानं च ब्राह्मणस्वाभिमर्शनम्}
{सर्वाण्येतान्यकार्याणि प्राहुर्धर्मविदो जनाः}


\twolineshloka
{पित्रा विवदते पुत्रो यश्च स्याद्गुरुतल्पगः}
{अप्रजायन्नरव्याघ्र भवत्यधार्मिको नरः}


% Check verse!
उक्तान्येतानि कर्माणि विस्तरेणेतरेण च
\threelineshloka
{यानि कुर्वन्निकुर्वंश्च प्रायश्चित्तीयते नरः}
{एतान्येव तु कर्माणि क्रियमाणानि मानवैः}
{येषुयेषु निमित्तेषु न लिप्यन्तेऽथ ताञ्शृणु}


\twolineshloka
{प्रगृह्य शस्त्रमायान्तमपि वेदान्तगं रणे}
{जिघांसन्तं जिघांसीयान्न तेन ब्रह्महा भवेत्}


\twolineshloka
{इति चाप्यत्र कौन्तेय मन्त्रो वेदेषु पठ्यते}
{वेदप्रमाणविहितं धर्मं च प्रब्रवीमि ते}


\twolineshloka
{अपेतं ब्राह्मणं वृत्ताद्यो हन्यादाततायिनम्}
{न तेन ब्रह्महा स स्यान्मन्युस्तन्मन्युमृच्छति}


\twolineshloka
{प्राणात्यये तथा ज्ञानादाचरन्मदिरामपि}
{आदेशितो धर्मपरैः पुनः संस्कारमर्हति}


\twolineshloka
{एतत्ते सर्वमाख्यातं कौन्तेयाभक्ष्यभक्षणम्}
{प्रायश्चित्तविधानेन सर्वमेतेन शुद्ध्यति}


\twolineshloka
{गुरुतल्पं हि गुर्वर्थं न दूषयति मानवम्}
{उद्दालकः श्वेतकेतुं जनयामास शिष्यतः}


\twolineshloka
{स्तेयं कुर्वंश्च गुर्वर्थमापत्सु न निषिध्यते}
{बहुशः कामकारेण न चेद्यः संप्रवर्तते}


\twolineshloka
{अन्यत्र ब्राह्मणस्वेभ्य आददानो न दुष्यति}
{स्वयमप्राशिता यश्च न स पापेन लिप्यते}


\twolineshloka
{प्राणत्राणेऽनृतं वाच्यमात्मनो वा परस्य च}
{गुर्वर्थे स्त्रीषु चैव स्याद्विवाहकरणेषु च}


\twolineshloka
{नावर्तते व्रतं स्वप्ने शुक्रमोक्षे कथंचन}
{आज्यहोमः समिद्धेऽग्नौ प्रायश्चित्तं विधीयते}


\twolineshloka
{पारिवित्त्यं तु पतिते नास्ति प्रव्रजिते तथा}
{भिक्षिते पारदार्यं च तद्धर्मस्य न दूषकम्}


\twolineshloka
{वृथा पशुसमालम्भं नैव कुर्यान्न कारयेत्}
{भ्रनुग्रहः पशूनां हि संस्कारो विधिनोदितः}


\twolineshloka
{अनर्हे ब्राह्मणे दत्तमज्ञानात्तन्न दूषकम्}
{सत्काराणां तथा तीर्थे नित्यं वा प्रतिपादनम्}


\twolineshloka
{स्त्रियास्तथापचारिण्या निष्कृतिः स्याददूषिका}
{अपि सा पूयते तेन न तु भर्ता प्रदुष्यति}


\twolineshloka
{तत्त्वं ज्ञात्वा तु सोमस्य विक्रयः स्याददोषवान्}
{असमर्थस्य भृत्यस्य विसर्गः स्याददोषवान्}


\twolineshloka
{वनदाहो गवामर्थे क्रियमाणो न दूषकः}
{उक्तान्येतानि कर्माणि यानि कुर्वन्न दुष्यति}


\twolineshloka
{प्रायश्चित्तानि वक्ष्यामि विस्तरेणैव भारत}
{`यानि कृत्वा नरः पूतो भविष्यति नराधिप ॥'}


\chapter{अध्यायः ३४}
\twolineshloka
{व्यास उवाच}
{}


\twolineshloka
{तपसा कर्मणा चैव प्रदानेन च भारत}
{पुनाति पापं पुरुषः पूतश्चेन्न प्रवर्तते}


\twolineshloka
{एककालं तु भुञ्जानश्चरन्भैक्षं स्वकर्मकृत्}
{कपालपाणिः खट्वाङ्गी ब्रह्मचारी सदोत्थितः}


\twolineshloka
{अनसूयुरधः शायी कर्म लोके प्रकाशयन्}
{पूर्णैर्द्वादशभिर्वर्षैर्ब्रह्महा विप्रमुच्यते}


\twolineshloka
{लक्ष्यः शस्त्रभृतां वा स्याद्विदुषामिच्छयाऽऽत्मनः}
{प्रास्येदात्मानमग्नौ वा समिद्धे त्रिरवाक््शिराः}


\twolineshloka
{जपन्वाऽन्यतमं वेदं योजनानां शतं व्रजेत्}
{सर्वस्वं वा वेदविदे ब्राह्मणायोपपादयेत्}


\twolineshloka
{धनं वा जीवनायालं गृहं वा सपरिच्छदम्}
{मुच्यते ब्रह्महत्याया गोप्ता गोब्राह्मणस्य च}


\twolineshloka
{षङ्गिर्वर्षैः कृच्छ्रभोजी ब्रह्महा पूयते नरः}
{मासेमासे समश्नंस्तु त्रिभिर्वर्षैः प्रमुच्यते}


\twolineshloka
{संवत्सरेण मासाशी पूयते नात्र संशयः}
{तथैवोपवसन्राजन्स्वल्पेनापि प्रपूयते}


\threelineshloka
{क्रतुना चाश्वमेधेन पूयते नात्र संशयः}
{ये चाप्यवभृथस्नाताः केचिदेवंविधा नराः}
{ते सर्वे धूतपाप्मानो भवन्तीति परा श्रुतिः}


% Check verse!
ब्राह्मणार्थे हतो युद्धे मुच्यते ब्रह्महत्यया
\twolineshloka
{गवां शतसहस्रं तु पात्रेभ्यः प्रतिपादयेत्}
{ब्रह्महा विप्रमुच्येत सर्वपापेभ्य एव च}


\twolineshloka
{कपिलानां सहस्राणि यो दद्यात्पञ्चविंशतिम्}
{दोग्ध्रीणां स च पापेभ्यः सर्वेभ्यो विप्रमुच्यते}


\twolineshloka
{गोसहस्रं सवत्सानां दोग्ध्रीणां प्राणसंशये}
{साधुभ्यो वै दरिद्रेभ्यो दत्त्वा मुच्येत किल्विषात्}


\twolineshloka
{शतं वै यस्तु काम्भोजान्ब्राह्मणेभ्यः प्रयच्छति}
{नियतेभ्यो महीपाल स च पापात्प्रमुच्यते}


\twolineshloka
{मनोरथं तु यो दद्यादेकस्मा अपि भारत}
{न कीर्तयेत दत्त्वा यः स च पापात्प्रमुच्यते}


\twolineshloka
{सुरापानं सकृत्कृत्वा योऽग्निवर्णां सुरां पिबेत्}
{स पावयत्यथात्मानमिह लोके परत्र च}


\twolineshloka
{मरुप्रपातं प्रपतञ्ज्वलनं वा समाविशन्}
{महाप्रस्थानमातिष्ठन्मुच्यते सर्वकिल्बिषैः}


\twolineshloka
{बृहस्पतिसवेनेष्ट्वा सुरापो ब्राह्मणः पुनः}
{समितिं ब्राह्मणो गच्छेदिति वै ब्रह्मणः श्रुतिः}


\twolineshloka
{भूमिप्रदानं कुर्याद्यः सुरां पीत्वा विमत्सरः}
{पुनर्न च पिबेद्राजन्संस्कृतः स च शुध्यति}


\twolineshloka
{गुरुतल्पी शिलां तप्तामायसीमभिसंविशेत्}
{अवकृत्यात्मनः शेफं प्रव्रजेदूर्ध्वदर्शनः}


\twolineshloka
{शरीरस्य विमोक्षेण मुच्यते कर्मणोऽशुभात्}
{कर्मभ्यो विप्रमुच्यन्ते यताः संवत्सरं स्त्रियः}


\twolineshloka
{महाव्रतं चरेद्यस्तु दद्यात्सर्वस्वमेव तु}
{गुर्वर्थे वा हतो युद्धे स मुच्येत्कर्मणोऽशुभात्}


\twolineshloka
{अनृतेनोपवर्ती चेत्प्रतिरोद्धा गुरोस्तथा}
{उपाहृत्य प्रियं तस्मै तस्मात्पापात्प्रमुच्यते}


\twolineshloka
{अवकीर्णनिमित्तं तु ब्रह्महत्याव्रतं चरेत्}
{गोचर्मवासाः षण्मासांस्तथा मुच्येत किल्बिषात्}


\twolineshloka
{परदारोपसेवी तु परस्यापहरन्वसु}
{संवत्सरं व्रती भूत्वा तथा मुच्येत किल्बिषात्}


\twolineshloka
{धनं तु यस्यापहरेत्तस्मै दद्यात्समं वसु}
{विविधेनाभ्युपायेन तदा मुच्येत किल्बिषात्}


\twolineshloka
{कृच्छ्राद्द्वादशरात्रेण संयतात्मा व्रते स्थितः}
{परिवेत्ता भवेत्पूतः परिवित्तिस्तथैव च}


\twolineshloka
{निवेश्यं तु पुनस्तेन भवेत्तारयता पितॄन्}
{न तु स्त्रिया भवेद्दोषो न तु सा तेन लिप्यते}


\twolineshloka
{भोजनं ह्यन्तराशुद्धं चातुर्मास्ये विधीयते}
{स्त्रियस्तेन प्रशुध्यन्ति इति धर्मविदो विदुः}


\twolineshloka
{स्त्रियस्त्वाशङ्किताः पापे नोपगम्या विजानता}
{रजसा ता विशुध्यन्ते भस्मना भाजनं यथा}


\twolineshloka
{पादजोच्छिष्टकांस्यं यद्गवा घ्रातमथापि वा}
{गण्डूषोच्छिष्टमपि वा विशुध्येद्दशभिस्तु तत्}


\twolineshloka
{चतुष्पात्सकलो धर्मो ब्राह्मणस्य विधीयते}
{पादोन इष्टो राजन्ये तथा धर्मो विधीयते}


\twolineshloka
{तथा वैश्ये च शूद्रे च पादः पादो विधीयते}
{विद्यादेवंविधनैषां गुरुलाघवनिश्चयम्}


\twolineshloka
{तिर्यग्योनिवधं कृत्वा द्रुमांश्छित्वोत्तरान्बहून्}
{त्रिरात्रं वायुभक्षः स्यात्कर्म च प्रथयन्नरः}


\twolineshloka
{अगम्यागमने राजन्प्रायश्चित्तं विधीयते}
{आर्द्रवस्त्रेण षण्मासान्विभाव्यं भस्मशायिना}


\twolineshloka
{एवमेव तु सर्वेषामकार्याणां विधिर्भवेत्}
{ब्रह्मणोक्तेन विधिना दृष्टान्तागमहेतुभिः}


\twolineshloka
{सावित्रीमप्यधीयानः शुचौ देशे मिताशनः}
{अहिंसो मन्दकं जल्पान्मुच्यते सर्वकिल्बिषात्}


\twolineshloka
{अहः सु सततं तिष्ठेदभ्याकाशं निशाः स्वपन्}
{त्रिरह्नि त्रिर्निशायां च सवासा जलमाविशेत्}


\twolineshloka
{स्त्रीशूद्रपतितांश्चापि नाभिभाषेद्ब्रतान्वितः}
{पापान्यज्ञानतः कृत्वा मुच्येदेवंव्रतो द्विजः}


\twolineshloka
{शुभाशुभफलं प्रेत्य लभते भूतसाक्षिकम्}
{अतिरिच्येत्तयोर्यस्तु तत्कर्ता लभते फलम्}


\twolineshloka
{तस्माद्दानेन तपसा कर्मणा च फलं शुभम्}
{वर्धयेदशुभं कृत्वा यथा स्यादतिरेकवान्}


\twolineshloka
{कुर्याच्छुभानि कर्माणि निमित्ते पापकर्मणाम्}
{दद्यान्नित्यं च वित्तानि तथा मुच्येत किल्बिषात्}


\twolineshloka
{अनुरूपं हि पापस्य प्रायश्चित्तमुदाहृतम्}
{महापातकवर्जं तु प्रायश्चित्तं विधीयते}


\twolineshloka
{भक्ष्याभक्ष्येषु चान्येषु वाच्यावाच्ये तथैव च}
{अज्ञानज्ञानयो राजन्विहितान्यनुजानतः}


\twolineshloka
{जानता तु कृतं पापं गुरु सर्वं भवत्युत}
{अज्ञानात्स्खलिते दोषे प्रायश्चित्तं विधीयते}


\twolineshloka
{शक्यते विधिना पापं यथोक्तेन व्यपोहितुम्}
{आस्तिके श्रद्दधाने च विधिरेष विधीयते}


\twolineshloka
{नास्तिकाश्रद्दधानेषु पुरुषेषु कदाचन}
{दम्भद्वेषप्रधानेषु विधिरेष न शिष्यते}


\twolineshloka
{शिष्टाचारश्च दिष्टश्च धर्मो धर्मभूतां वर}
{सेवितव्यो नरव्याघ्र प्रेत्येह च हितेप्सुना}


\twolineshloka
{स राजन्मोक्ष्यते पापात्तेन पूर्णेन हेतुना}
{त्राणार्थं वा वधे तेषामथवा नृपकर्मणा}


\threelineshloka
{अथवा ते घृणा काचित्प्रायश्चित्तं चरिष्यसि}
{मा चैवानार्यजुष्टेन मृत्युना निधनं गमः ॥वैशंपायन उवाच}
{}


\twolineshloka
{एवमुक्तो भगवता धर्मराजो युधिष्ठिरः}
{चिन्तयित्वा मुहूर्तेन प्रत्युवाच तपोधनम्}


\chapter{अध्यायः ३५}
\twolineshloka
{युधिष्ठिर उवाच}
{}


\threelineshloka
{किं भक्ष्यं चाप्यभक्ष्यं च किंच देयं प्रशस्यते}
{किंच पात्रमपात्रं वा तन्मे ब्रूहि पितामह ॥व्यास उवाच}
{}


\twolineshloka
{अत्राप्युदाहरन्तीममितिहासं पुरातनम्}
{सिद्धानां चैव संवादं मनोश्चैव प्रजापतेः}


\twolineshloka
{ऋषयस्तु व्रतपराः समागम्य पुरा विभुम्}
{धर्मं पप्रच्छुरासीनमादिकाले प्रजापतिम्}


\twolineshloka
{कथमन्त्रं कथं दानं गम्यागम्याः कथं स्त्रियः}
{कार्याकार्यं च यत्सर्वं शंस वै त्वं प्रजापते}


\twolineshloka
{तैरेवमुक्तो भगवान्मनुः स्वायंभुवोऽब्रवीत्}
{शुश्रूषध्वं यथावृत्तं धर्मं व्याससमासतः}


\twolineshloka
{अनादेशे जपो होम उपवासस्तथैव च}
{आत्मज्ञानं पुण्यनद्यो यत्र प्रायश्च तत्पराः}


\twolineshloka
{अनादिष्टं तथैतानि पुण्यानि धरणीभृतः}
{सुवर्णप्राशनमपि रत्नादिस्नानमेव च}


\twolineshloka
{देवस्थानाभिगमनमाज्यप्राशनमेव च}
{एतानि मेध्यं पुरुषं कुर्वन्त्याशु न संशयः}


\twolineshloka
{न गर्वेण भवेत्प्राज्ञः कदाचिदपि मानवः}
{दीर्घमायुरथेच्छन्हि त्रिरात्रं चोष्णपो भवेत्}


\twolineshloka
{अदत्तस्यानुपादानं दानमध्ययनं तपः}
{अहिंसा सत्यमक्रोधं क्षमा धर्मस्य लक्षणम्}


\twolineshloka
{स एव धर्मः सोऽधर्मो देशकाले प्रतिष्ठितः}
{आदानमनृतं हिंसा धर्मो ह्यात्यन्तिकः स्मृतः}


\twolineshloka
{द्विविधौ चाप्युभावेतौ धर्माधर्मौ विजानताम्}
{अप्रवृत्तिः प्रवृत्तिश्च द्वैविध्यं लोकवेदयोः}


\threelineshloka
{अप्रवृत्तेरमर्त्यत्वं मर्त्यत्वं कर्मणः फलम्}
{अशुभस्याशुभं विद्याच्छुभस्य शुभमेव च}
{एतयोश्चोभयोः स्यातां शुभाशुभतया तथा}


\twolineshloka
{दैवं च दैवसंयुक्तं प्राणश्च प्रलयस्तथा}
{अप्रेक्षापूर्वकरणादशुभानां शुभं फलम्}


\twolineshloka
{ऊर्ध्वं भवति संदेहादिहादिष्टार्थमेव च}
{अप्रेक्षापूर्वकरणात्प्रायश्चित्तं विधीयते}


\threelineshloka
{क्रोधमोहकृते चैव दृष्टान्तागमहेतुभिः}
{शरीराणामुपक्लेशो मनसश्च प्रियाप्रिये}
{तदौषधैश्च मन्त्रैश्च प्रायश्चित्तैश्च शाम्यति}


\twolineshloka
{उपवासेनैकरात्रं दण्डोत्सर्गे नराधिपः}
{विशुद्ध्येदात्मशुद्ध्यर्थं त्रिरात्रं तु पुरोहितः}


\twolineshloka
{क्षयं शोकं प्रकुर्वाणो न म्रियेत यदा नरः}
{शस्त्रादिभिरुपाविष्टस्त्रिरात्रं तत्र निर्दिशेत्}


\twolineshloka
{जातिश्रेण्यधिवासानां कुलधर्मांश्च शाश्वतान्}
{वर्जयन्ति च ये धर्मं तेषां धर्मो न विद्यते}


\twolineshloka
{दश वा वेदशास्त्रज्ञास्त्रयो वा धर्मपाठकाः}
{यद्ब्रूयुः कार्य उत्पन्ने स धर्मो धर्मसंशये}


\twolineshloka
{अनुष्णा मृत्तिका चैव तथा क्षुद्रपिपीलिकाः}
{श्लेष्मातकस्तथा विप्रैरभक्ष्यं विषमेव च}


\twolineshloka
{अभक्ष्या ब्राह्मणैर्मत्स्याः शकलैर्ये विवर्जिताः}
{चतुष्पात्कच्छपादन्यो मण्डूका जलजाश्च ये}


\twolineshloka
{भासा हंसाः सुपर्णाश्च चक्रवाकाः प्लवा बकाः}
{काको मद्रुश्च गृध्रश्च श्येनोलूकस्तथैव च}


\twolineshloka
{क्रव्यादा दंष्ट्रिणः सर्वे चतुष्पात्पक्षिणश्च ये}
{येषां चोभयतो दन्ताश्चतुर्दंष्ट्राश्च सर्वशः}


\twolineshloka
{एडकाश्च मृगोष्ट्राणां सूकराणां गवामपि}
{मानुषीणां खरीणां च न पिबेद्ब्राह्मणः पयः}


\twolineshloka
{प्रेतान्नं सूतकान्नं च यच्च किंचिदनिर्दशम्}
{अभोज्यं चाप्यपेयं च धेनोर्दुग्धमनिर्दशम्}


\twolineshloka
{राजान्नं तेज आदत्ते शूद्रान्नं ब्रह्मवर्चसम्}
{आयुः सुवर्णकारान्नमवीरायाश्च योषितः}


\twolineshloka
{विष्ठा वार्धुषिकस्यान्नं गणिकान्नमथेन्द्रियम्}
{मृष्यन्ति ये चोपपतिं स्त्रीजितान्नं च सर्वशः}


\twolineshloka
{दीक्षितस्य कदर्यस्य क्रतुविक्रयिकस्य च}
{तक्ष्णश्चर्मावकर्तुश्च पुंश्चल्या रजकस्य च}


\twolineshloka
{चिकित्सकस्य यच्चान्नमभोज्यं रक्षिणस्तथा}
{गणग्रामाभिशस्तानां रङ्गस्त्रीजीविनां तथा}


\twolineshloka
{परिवित्तीनामपुंसां च बन्दिद्यूतविदां तथा}
{वामहस्ताहृतं चान्नं शुष्कं पर्युषितं च यत्}


\threelineshloka
{सुरानुगतमुच्छिष्टमभोज्यं शेषितं च यत्}
{पिष्टमांसेक्षुशाकानामाविकाजापयस्तथा}
{सक्तु धाना करम्भाश्च नोपभोग्याश्चिरस्थिताः}


\twolineshloka
{पायसं कृसरं मांसमपूपाश्च वृथा कृताः}
{अपेयाश्चाप्यभक्ष्याश्च ब्राह्मणैर्गृहमेधिभिः}


\twolineshloka
{देवानृषीन्मनुष्यांश्च पितॄन्गृह्याश्च देवताः}
{पूजयित्वा ततः पश्चाद्गृहस्थो भोक्तुमर्हति}


\twolineshloka
{यथा प्रव्रजितो भिक्षुस्तथैव स्वे गृहे वसेत्}
{एवंवृत्तः प्रियैर्दारैः संबसन्धर्ममाप्नुयात्}


\twolineshloka
{न दद्याद्यशसे दानं न भयान्नोपकारिणे}
{न नृत्यगीतशीलेषु हासकेषु च धार्मिकः}


\twolineshloka
{न मत्ते चैव नोन्मत्ते न स्तेने न च कुत्सके}
{न वाग्घीने विवर्णे वा नाङ्गहीने न वामने}


\twolineshloka
{न दुर्जने दौष्कुले वा व्रतैर्यो वा न संस्कृतः}
{न श्रोत्रियमृते दानं ब्राह्मणे ब्रह्मवर्जिते}


\twolineshloka
{असम्यच्कैव यद्दत्तमसम्यक् च प्रतिग्रहः}
{उभयं स्यादनर्थाय दातुरादातुरेव च}


\twolineshloka
{यथा खदिरमालम्ब्य शिलां वाप्यर्णवं तरन्}
{मञ्जेत मञ्जतस्तद्वद्दाता यश्च प्रतिग्रही}


\twolineshloka
{काष्ठैरार्द्रैर्यथा वह्निरुपस्तीर्णो न दीप्यते}
{तपःस्वाध्यायचारित्रैरेवं हीनः प्रतिग्रही}


\twolineshloka
{कपाले यद्वदापः स्युः श्वदृतौ च यथा पयः}
{आश्रयस्थानदोषेण वृत्तहीने तथा श्रुतम्}


\twolineshloka
{निर्मन्त्रो निर्वृतो यः स्यादशास्त्रज्ञोऽनसूयकः}
{अनुक्रोशात्प्रदातव्यं हीनेष्वव्रतिकेषु च}


\twolineshloka
{न वै देयमनुक्रोशाद्दीनायापगुणाय तु}
{आप्ताचरित इत्येव धर्म इत्येव वा पुनः}


\twolineshloka
{निष्कारणं स्मृतं दत्तं ब्राह्मणे ब्रह्मवर्जिते}
{न फलेत्पात्रदोषेण न चात्रास्ति विचारणा}


\twolineshloka
{यथा दारुमयो हस्ती यथा चर्ममयो मृगः}
{ब्राह्मणश्चानधीयानस्त्रयस्ते नामधारकाः}


\twolineshloka
{यथा षण्ढोऽफलः स्त्रीषु यथा गौर्गवि चाफला}
{शकुनिर्वाप्यपक्षः स्यान्निर्मन्त्रो ब्राह्मणस्तथा}


\twolineshloka
{ग्रामस्थानं यथा शून्यं यथा कूपश्च निर्जलः}
{यथा हुतमनग्नौ च तथैव स्यान्निराकृतौ}


\twolineshloka
{देवतानां पितॄणां च हव्यकव्यविनाशकः}
{सर्वथाऽर्थहरो मूर्खो न लोकान्प्राप्तुमर्हति}


\twolineshloka
{एतत्ते कथितं सर्वं यथावृत्तं युधिष्ठिर}
{समासेन महद्ध्येतच्छ्रोतव्यं नरतर्षभ}


\chapter{अध्यायः ३६}
\twolineshloka
{युधिष्ठिर उवाच}
{}


\twolineshloka
{श्रोतुमिच्छामि भगवन्विस्तरेण महामुने}
{राजधर्मान्द्विजश्रेष्ठ चातुर्वर्ण्यस्य चाखिलान्}


\twolineshloka
{आपत्सु च यथा नीतिः प्रणेतव्या द्विजोत्तम}
{धर्म्यमालम्ब्य पन्थानं विजयेयं कथं महीम्}


\twolineshloka
{प्रायश्चित्तकथा ह्येषा भक्ष्याभक्ष्यसमन्विता}
{कौतूहलानुप्रवणा हर्षं जनयतीव मे}


\threelineshloka
{धर्मचर्या च राज्यं च नित्यमेव विरुध्यते}
{एवं मुह्यति मे चेतश्चिन्तयानस्य नित्यशः ॥वैशंपायन उवाच}
{}


\twolineshloka
{तमुवाच महाराज व्यासो वेदविदां वरः}
{नारदं समभिप्रेक्ष्य सर्वं जानन्पुरातनम्}


\twolineshloka
{श्रोतुमिच्छसि चेद्धर्मं निखिलेन नराधिप}
{प्रेहि भीष्मं महाबाहो वृद्धं कुरुपितामहम्}


\twolineshloka
{स ते धर्मरहस्येषु संशयान्मनसि स्थितान्}
{छेत्ता भागीरथीपुत्रः सर्वज्ञः सर्वधर्मवित्}


\twolineshloka
{जनयामास यं देवी दिव्या त्रिपथगा नदी}
{साक्षाद्ददर्श यो देवान्सर्वानिन्द्रपुरोगमान्}


\twolineshloka
{बृहस्पतिपुरोगांस्तु देवर्षीनसकृत्प्रभुः}
{तोषयित्वोपचारेण राजनीतिमधीतवान्}


\twolineshloka
{उशना वेद यच्छास्त्रं देवासुरगुरुर्द्विजः}
{स च धर्मं सवैयाख्यं प्राप्तवान्कुरुसत्तमः}


\twolineshloka
{भार्गवाच्च्यवनाच्चापि वेदानङ्गोपबृंहितान्}
{प्रतिपेदे महाबुद्धिर्वसिष्ठाच्चरितव्रतः}


\twolineshloka
{पितामहसुतं ज्येष्ठं कुमारं दीप्ततेजसम्}
{अध्यात्मगतितत्त्वज्ञमुपाशिक्षत यः पुरा}


\twolineshloka
{मार्कण्डेयमुखात्कृत्स्नं यतिधर्ममवाप्तवान्}
{रामादस्त्राणि शक्राच्च प्राप्तवान्पुरुपर्षभः}


\twolineshloka
{मृत्युरात्मेच्छया यस्य जातस्य मनुजेष्वपि}
{तथाऽनपत्यस्य सतः पुण्यलोकादिविश्रुताः}


\twolineshloka
{यस्य ब्रह्मर्षयः पुण्या नित्यमासन्स भसादः}
{यस्य नाविदितं किंचिज्ज्ञानं ज्ञेयेषु दृश्यते}


\twolineshloka
{स ते वक्ष्यति धर्मज्ञः सूक्ष्मधर्मार्थतत्त्ववित्}
{तमभ्येहि पुरा प्राणान्स विमुञ्चति धर्मवित्}


\threelineshloka
{एवमुक्तस्तु कौन्तेयो दीर्घप्रज्ञो महामतिः}
{उवाच वदतां श्रेष्ठं व्यासं सत्यवतीसुतम् ॥युधिष्ठिर उवाच}
{}


\twolineshloka
{वैशसं सुमहत्कृत्वा ज्ञातीनां रोमहर्षणम्}
{आगस्कृत्सर्वलोकस्य पृथिवीनाशकारकः}


\threelineshloka
{घातयित्वा तमेवाजौ छलेनाजिह्नयोधिनम्}
{उपसंप्रष्टुमर्हामि तमहं केन हेतुना ॥वैशंपायन उवाच}
{}


\threelineshloka
{ततस्तं नृपतिश्रेष्ठं चातुर्वर्ण्यहितेप्सया}
{पुनरेव महाबाहुर्यदुश्रेष्ठोऽब्रवीद्वचः ॥वासुदेव उवाच}
{}


\twolineshloka
{नेदानीमतिनिर्बन्धं शोके त्वं कर्तुमर्हसि}
{यदाह भगवान्व्यासस्तत्कुरुष्व नृपोत्तम}


\twolineshloka
{ब्राह्मणास्त्वां महाबाहो भ्रातरश्च महौजसः}
{पर्जन्यमिव घर्मान्ते नाथमाना उपासते}


\twolineshloka
{हतशिष्टाश्च राजानः कृत्स्नं चैव समागतम्}
{चातुर्वर्ण्यं महाराज राष्ट्रं ते कुरुजाङ्गलम्}


\twolineshloka
{प्रियार्थमपि चैतेषां ब्राह्मणानां महात्मनाम्}
{नियोगादस्य च गुरोर्व्यासस्यामिततेजसः}


\threelineshloka
{सुहृदामस्मदादीनां द्रौपद्यांश्च परंतप}
{कुरु प्रियममित्रघ्न लोकस्य च हितं कुरु ॥वैशंपायन उवाच}
{}


\twolineshloka
{एवमुक्तः स कृष्णेन राजा राजीवलोचनः}
{हितार्थं सर्वलोकस्य समुत्तस्थौ महामनाः}


\twolineshloka
{सोऽनुनीतो नरव्याघ्र विष्टरश्रवसा स्वयम्}
{द्वैपायनेन च तथा देवस्थानेन जिष्णुना}


\twolineshloka
{एतैश्चान्यैश्च बहुभिरनुनीतो युधिष्ठिरः}
{व्यजहान्मानसं दुःखं संतापं च महायशाः}


\twolineshloka
{श्रुतवाक्यः श्रुतनिधिः श्रुतश्राव्यविशारदः}
{व्यवस्य मनसा शान्तिमगच्छत्पाण्डुनन्दनः}


\twolineshloka
{स तैः परिवृतो राजा नक्षत्रैरिव चन्द्रमाः}
{धृतराष्ट्रं पुरस्कृत्य स्वपुरं प्रविवेश ह}


\twolineshloka
{प्रविविक्षुः स धर्मज्ञः कुन्तीपुत्रो युधिष्ठिरः}
{अर्चयामास देवांश्च ब्राह्मणांश्च सहस्रशः}


\twolineshloka
{ततो नवं रथं शुभ्रं कम्बलाजिनसंवृतम्}
{युक्तं षोडशभिस्त्वश्चैः पाण़्डुरैः शुभलक्षणैः}


\twolineshloka
{मन्त्रैरभ्यर्चितं पण्यैः स्तूयमानश्च बन्दिभिः}
{आरुरोह यथा देवः सोमोऽम्रतमयं यथम्}


\twolineshloka
{जग्राह रश्मीन्कौन्तेयो भीमो भीमपराक्रमः}
{अर्जुनः पाण्डुरं छत्रं धारयामास भानुमत्}


\twolineshloka
{ध्रियमाणं च तच्छत्रं पाण्डुरं राजमूर्धनि}
{शुशुभे तारकाराजः सिताभ्र इव चाम्बरे}


\twolineshloka
{चामरव्यजने त्वस्य वीरौ जगृहतुस्तदा}
{चन्द्ररश्मिप्रये शुभ्रे माद्रीपुत्रावलंकृते}


\twolineshloka
{ते पञ्च रथमास्थाय भ्रातरः समलंकृताः}
{भूतानीव समस्तानि राजन्ददृशिरे तदा}


\twolineshloka
{आस्थाय तु रथं शुभ्रं युक्तमश्वैर्मनोजवैः}
{अन्वयात्पृष्ठतो राजन्युयुत्सुः पाण्डवाग्रजम्}


\twolineshloka
{रथं हेममयं शुभ्रं शैव्यसुग्रीवयाजितम्}
{सह सात्यकिना कृष्णः समास्थायान्वयात्कुरून्}


\twolineshloka
{नरयानेन तु ज्येष्ठः पिता पार्थस्य भारत}
{अग्रतो धर्मराजस्य गान्धारीसहितो ययौ}


\twolineshloka
{कुरुस्त्रियश्च ताः सर्वाः कुन्ती कृष्णा च माधवी}
{यानैरुच्चावचैर्जग्मुर्विदुरेण पुरस्कृताः}


\twolineshloka
{ततो रथाश्च बहुला नागाश्वसमलंकृताः}
{पादाताश्च हयाश्चैव पृष्ठतः समनुव्रजन्}


\twolineshloka
{ततो वैतालिकैः सूतैर्मागधैश्च सुभाषितैः}
{स्तूयमानो ययौ राजा नगरं नागसाह्वयम्}


\twolineshloka
{तत्प्रयाणं माहबाहोर्बभूवाप्रतिमं भुवि}
{आकुलाकुलमुत्क्रुष्टं हृष्टपुष्टजनाकुलम्}


\twolineshloka
{अभियाने तु पार्थस्य नरैर्नगरवासिभिः}
{नगरं राजमार्गाश्च यथावत्समलंकृताः}


\twolineshloka
{पाण्डुरेण च माल्येन पताकाभिश्च मेदिनी}
{संस्कृतो राजमार्गोऽभूद्धूपनैश्च प्रधूपितः}


\twolineshloka
{अथ चूर्णैश्च गन्धानां नानापुष्पप्रियङ्गुभिः}
{माल्यदामभिरासक्तै राजवेश्माभिसंवृतम्}


\twolineshloka
{कुम्भाश्च नगरद्वारि वारिपूर्णा नवा दृढाः}
{सिताः सुमनसो गौराः स्थापितास्तत्र तत्र ह}


\twolineshloka
{तथा स्वलंकृतं द्वारं नगरं पाण्डुनन्दनः}
{स्तूयमानः शुभैर्वाक्यैः प्रविवेश सुहृद्वृतः}


\chapter{अध्यायः ३७}
\twolineshloka
{वैशंपायन उवाच}
{}


\twolineshloka
{प्रवेशने तु पार्थानां जनानां पुरवासिनाम्}
{दिदृक्षूणां सहस्राणि समाजग्मुः सहस्रशः}


\twolineshloka
{स राजमार्गः शुशुभे समलंकृतचत्वरः}
{यथा चन्द्रोदये राजन्वर्धमानो महोदधिः}


\twolineshloka
{गृहाणि राजमार्गेषु रत्नवन्ति महान्ति च}
{प्राकम्पन्तीव भारेण स्त्रीणां पूर्णानि भारत}


\twolineshloka
{ताः शनैरिव सव्रीडं प्रशशंसुर्युधिष्ठिरम्}
{भीमसेनार्जुनौ चैव माद्रीपुत्रौ च पाण्डवौ}


\twolineshloka
{धन्या त्वमसि पाञ्चालि या त्वं पुरुषसत्तमान्}
{उपतिष्ठसि कल्याणि महर्षिमिव गौतमी}


\twolineshloka
{तव कर्माण्यमोघानि व्रतचर्या च भामिनि}
{इति कृष्णां महाराज प्रशशंसुस्तदा स्त्रियः}


\twolineshloka
{प्रशंसावचनैस्तासां मिथः शब्दैश्च भारत}
{प्रीतिजैश्च तदा शब्दैः पुरमासीत्समाकुलम्}


\twolineshloka
{तमतीत्य यथायुक्तं राजमार्गं युधिष्ठिरः}
{अलंकृतं शोभमानमुपापाद्राजवेश्म ह}


\twolineshloka
{ततः प्रकृतयः सर्वाः पौरा जानपदास्तदा}
{ऊचुः कर्णसुखा वाचः समुपेत्य ततस्ततः}


\twolineshloka
{दिष्ट्या जयसि राजेन्द्र शत्रूञ्छत्रुनिषूदन}
{दिष्ट्या राज्यं पुनः प्राप्तं धर्मेण च बलेन च}


\twolineshloka
{भव नस्त्वं महाराज राजेह शरदां शतम्}
{प्रजाः पालय धर्मेण यथेन्द्रस्त्रिदिवं तथा}


\twolineshloka
{एवं राजकुलद्वारि मङ्गलैरभिपूजितः}
{आशीर्वादान्द्विजैरुक्तान्प्रतिगृह्य समन्ततः}


\twolineshloka
{प्रविश्य भवनं राजा देवराजगृहोपमम्}
{श्रुत्वा विजयसंयुक्तं रथात्पश्चादवातरत्}


\twolineshloka
{प्रविश्याभ्यन्तरं श्रीमान्दैवतान्यभिगम्य च}
{पूजयामास रत्नैश्च गन्धमाल्यैश्च सर्वशः}


\twolineshloka
{निश्चक्राम ततः श्रीमान्पुनरेव महायशाः}
{ददर्श ब्राह्मणांश्चैव सोऽभिरूपानवस्थितान्}


\twolineshloka
{स संवृतस्तदा विप्रैराशीर्वादविवक्षुभिः}
{शुशुभे विमलश्चन्द्रस्तारागणवृतो यथा}


\threelineshloka
{तांस्तु वै पूजयामास कौन्तेयो विधिवद्द्विजान्}
{सुमनोमोदकै रत्नैर्हिरण्येन च भूरिणा}
{गोभिर्वस्त्रैश्च राजेन्द्र विविधैश्च किमिच्छकैः}


\twolineshloka
{धौम्यं गुरुं पुरस्कृत्य ज्येष्ठं पितरमेव च}
{`प्रविवेश सभां राजा सुधर्मां वासवो यथा ॥'}


\twolineshloka
{ततः पुण्याहघोषोऽभूद्दिवं स्तब्ध्वेव भारत}
{सुहृदां प्रीतिजननः पुण्यः श्रुतिसुखावहः}


\twolineshloka
{हंसवन्नेदुषां राजन्द्विजानां तत्र भारती}
{शुश्रुवे वेदविदुषां पुष्कलार्थपदाक्षरा}


\twolineshloka
{ततो दुन्दुभिनिर्घोषः शङ्खानां च मनोरमः}
{जयं प्रवदतां तत्र स्वनः प्रादुरभून्नृप}


\twolineshloka
{निःशब्दे च स्थिते तत्र ततो विप्रजने पुनः}
{राजानं ब्राह्मणच्छझा चार्वाको राक्षसोऽब्रवीत्}


\twolineshloka
{तत्र दुर्योधनसखा भिक्षुरूपेण संवृतः}
{साङ्ख्यः शिखी त्रिदण्डी च धृष्टो विगतसाध्वसः}


\twolineshloka
{वृतः सर्वैस्तथा विप्रैराशीर्वादविवक्षुभिः}
{परस्सहस्रै राजेन्द्र तपोनियमसंस्थितैः}


\threelineshloka
{सुदुष्टः पापमाशंसुः पाण्डवानां महात्मनाम्}
{अनामन्त्र्यैव तान्विप्रांस्तमुवाच महीपतिम् ॥चार्वाक उवाच}
{}


\twolineshloka
{इमे प्राहुर्द्विजाः सर्वे समारोप्य वचो मयि}
{धिग्भवन्तं कुनृपतिं ज्ञातिघातिनमस्तु वै}


\twolineshloka
{किं ते राज्येन कौन्तेय कृत्वेमं ज्ञातिसंक्षयम्}
{घातयित्वा गुरूंश्चैव मृतं श्रेयो न जीवितम्}


\twolineshloka
{इति ते वै द्विजाः श्रुत्वा तस्य दुष्टस्य रक्षसः}
{विव्यथुश्चुक्रुशुश्चैव तस्य वाक्यप्रधर्षिताः}


\threelineshloka
{ततस्ते ब्राह्मणाः सर्वे स च राजा युधिष्ठिरः}
{व्रीडिताः परमोद्विग्रास्तूष्णीमासन्विशांपते ॥युधिष्ठिर उवाच}
{}


\threelineshloka
{प्रसीदन्तु भवन्तो मे प्रणतस्याभियाचतः}
{प्रत्यासन्नव्यसनिनं न मां धिक्कर्तुमर्हथ ॥वैशंपायन उवाच}
{}


\twolineshloka
{ततो राजन्ब्राह्माणास्ते सर्व एव विशांपते}
{ऊचुर्नैष द्विजोऽस्माकमन्यस्तु तव पार्थिव}


\threelineshloka
{जज्ञुश्चैनं महात्मानस्ततस्तं ज्ञानचक्षुषा}
{ब्राह्मणा वेदविद्वांसस्तपोभिर्विमलीकृताः ॥ब्राह्मणा ऊचुः}
{}


\twolineshloka
{एष दुर्योधनसखा विश्रुतो ब्रह्मराक्षसः}
{परिव्राजकरूपेण हितं तस्य चिकीर्षति}


\threelineshloka
{न वयं ब्रूम धर्मात्मन्व्येतु ते भयमीदृशम्}
{उपतिष्ठतु कल्याणं भवन्तं भ्रातृभिः सह ॥वैशंपायन उवाच}
{}


\twolineshloka
{ततस्ते ब्राह्मणाः सर्वे हुंकारैः क्रोधमूर्च्छिताः}
{निर्भर्त्सयन्तः शुचयो निजघ्नुः पापराक्षसम्}


\twolineshloka
{स पपात विनिर्दग्धस्तेजसा ब्रह्मवादिनाम्}
{महेन्द्राशनिनिर्दग्धः पादपोऽङ्कुरवानिव}


\twolineshloka
{पूजिताश्च ययुर्विप्रा राजानमभिनन्द्य तम्}
{राजा च हर्षमापेदे पाण्डवः ससुहृज्जनः}


\chapter{अध्यायः ३८}
\twolineshloka
{वैशंपायन उवाच}
{}


\threelineshloka
{ततस्तत्र तु राजानं तिष्ठन्ते भ्रातृभिः सह}
{उवाच देवकीपुत्रः सर्वदर्शी जनार्दनः ॥वासुदेव उवाच}
{}


\twolineshloka
{ब्राह्मणास्तात लोकेऽस्मिन्नर्चनीयाः सदा मम}
{एते भूमिचरा देवा वाग्विषाः सुप्रसादकाः}


\twolineshloka
{पुरा कृतयुगे राजंश्चार्वाको नाम राक्षसः}
{तपस्तेपे महाबाहो बदर्यां बहुवार्षिकम्}


\twolineshloka
{वरेण च्छन्द्यमानश्च ब्रह्मणा च पुनः पुनः}
{अभयं सर्वभूतेभ्यो वरयामास भारत}


\twolineshloka
{द्विजावमानादन्यत्र प्रादाद्वरमनुत्तमम्}
{अभयं सर्वभूतेभ्यो ददौ तस्मै जगत्पतिः}


\twolineshloka
{स तु लब्धवरः पापो देवानमितविक्रमः}
{राक्षसस्तापयामास तीव्रकर्मा महाबलः}


\twolineshloka
{ततो देवाः समेताश्च ब्रह्माणमिदमब्रुवन्}
{वधाय रक्षसस्तस्य बलविप्रकृतास्तदा}


\twolineshloka
{तानुवाच ततो देवो विहितं तत्र वै मया}
{यथाऽस्य भविता मृत्युरचिरणेति भारत}


\twolineshloka
{राजा दुर्योधनो नाम सखाऽस्य भविता नृषु}
{तस्य स्नेहावबद्धोऽसौ ब्राह्मणानवमंस्यते}


\twolineshloka
{तत्रैनं रुषिता विप्रा विप्रकारप्रधर्षिताः}
{धक्ष्यन्ति वाग्बलाः पापं ततो नाशं गमिष्यति}


\twolineshloka
{स एष निहतः शेते ब्रह्मदण्डेन राक्षसः}
{चार्वाको नृपतिश्रेष्ठ मा शुचो भरतर्षभ}


\twolineshloka
{हतास्ते क्षत्रधर्मेण ज्ञातयस्तव पार्थिव}
{स्वर्गताश्च महात्मानो वीराः क्षत्रियपुङ्गवाः}


\twolineshloka
{स त्वमातिष्ठ कार्याणि मा ते भूद्वुद्धिरन्यथा}
{शत्रूञ्जहि प्रजा रक्ष द्विजांश्च परिपूजय}


\chapter{अध्यायः ३९}
\twolineshloka
{वैशंपायन उवाच}
{}


\twolineshloka
{ततः कुन्तीसुतो राजा गतमन्युर्गतज्वरः}
{काञ्चने प्राड्भुखो हृष्टो न्यषीदत्परमासने}


\twolineshloka
{तमेवाभिमुखौ पीठे प्रदीप्ते काञ्चने शुभे}
{सात्यकिर्वासुदेवश्च निषीदतुररिंदमौ}


\twolineshloka
{मध्ये कृत्वा तु राजानं भीमसेनार्जुनावुभौ}
{निषीदतुर्महात्मानौ श्लक्ष्णयोर्मणिपीठयोः}


\twolineshloka
{दान्ते शय्यासने शुभ्रे जाम्बूनदविभूषिते}
{पृथाऽपि सहदेवेन सहास्ते नकुलेन च}


\twolineshloka
{सुधर्मा विदुरो धौम्यो धृतराष्ट्रश्च कौरवः}
{निषेदुर्ज्वलनाकारेष्वासनेषु पृथक्पृथक्}


\twolineshloka
{युयुत्सुः सञ्जयश्चैव गान्धारी च यशस्विनी}
{धृतराष्ट्रो यतो राजा ततः सर्वे समाविशन्}


\twolineshloka
{तत्रोपविष्टो धर्मात्मा श्वेताः सुमनसोऽस्पृशत्}
{स्वस्तिकानक्षतान्भूमिं सुवर्णं रजतं मणीन्}


\twolineshloka
{ततः प्रकृतयः सर्वाः पुरस्कृत्य पुरोहितम्}
{ददृशुर्धर्मराजानमादाय बहुमङ्गलम्}


\twolineshloka
{पृथिवीं च सुवर्णं च रत्नानि विविधानि च}
{आभिषेचनिकं भाण्डं सर्वसंभारसंभृतम्}


\twolineshloka
{काञ्चनौदुम्बरास्तत्र राजताः पृथिवीमयाः}
{पूर्णकुम्भाः सुमनसो लाजा बर्हीषि गोरसाः}


\twolineshloka
{शमीपिप्पलपालाशसमिधो मधुसर्पिषी}
{स्रुव औदुम्बरः शङ्खस्तथा हेमविभूषितः}


\twolineshloka
{दाशार्हेणाभ्यनुज्ञातस्तत्र धौम्यः पुरोहितः}
{प्रागुदक्प्रवणे वेदीं लक्षणेनोपलिख्य च}


\twolineshloka
{व्याघ्रचर्मोत्तरे शुक्ले सर्वतोभद्र आसने}
{दृढपादप्रतिष्ठाने हुताशनसमत्विपि}


\twolineshloka
{उपवेश्य महात्मानं कृष्णां च द्रुपदात्मजाम्}
{जुहाव पावकं धीमान्विधिमन्त्रपुरस्कृतम्}


\threelineshloka
{तत उत्थाय दाशार्हः शङ्खमादाय पूरितम्}
{अभ्यषिञ्चत्पतिं पृथ्व्याः कुन्तीपुत्रं युधिष्ठिरम्}
{धृतराष्ट्रश्च राजर्षिः सर्वाः प्रकृतयस्तथा}


\twolineshloka
{अनुज्ञातोऽथ कृष्णेन भ्रातृभिः सह पाण्डवः}
{पाञ्चजन्याभिषिक्तश्च राजाऽमृतमुखोऽभवत्}


% Check verse!
ततोऽनुवादयामासुः पणवानकदुन्दुभीन्
\twolineshloka
{धर्मराजोऽपि तत्सर्वं प्रतिजग्राह धर्मतः}
{पूजयामास तांश्चापि विधिवद्भूरिदक्षिणः}


\twolineshloka
{ततो निष्कसहस्रेण ब्राह्मणान्स्वस्त्यवाचयन्}
{वेदाध्ययनसंपन्नान्धृतिशीलसमन्वितान्}


\twolineshloka
{ते प्रीता ब्राह्मणा राजन्स्वस्त्यूचुर्जयमेव च}
{हंसा इव च नर्दन्तः प्रशशंसुर्युधिष्ठिरम्}


\twolineshloka
{युधिष्ठिर महाबाहो दिष्ट्या जयसि पाण्डव}
{दिष्ट्या स्वधर्मं प्राप्तोऽसि विक्रमेण महाद्युते}


\twolineshloka
{दिष्ट्या गाण्डीवधन्वा च भीमसेनश्च पाण्डवः}
{त्वं चापि कुशली राजन्माद्रीपुत्रौ च पाण्डवौ}


\twolineshloka
{मुक्ता वीरक्षयात्तस्मात्संग्रामाद्विजितद्विषः}
{क्षिप्रमुत्तरकार्याणि कुरु सर्वाणि भारत}


\twolineshloka
{ततः प्रीत्याऽर्चितः सद्भिर्धर्मराजो युधिष्ठिरः}
{प्रतिपेदे महद्राज्यं सुहृद्भिः सह भारत}


\chapter{अध्यायः ४०}
\twolineshloka
{वैशंपायन उवाच}
{}


\twolineshloka
{प्रकृतीनां च तद्वाक्यं देशकालोपबृंहितम्}
{श्रुत्वा युधिष्ठिरो राजाऽथोत्तरं प्रत्यभाषत}


\twolineshloka
{धन्याः पाण्डुसुता नूनं येषां ब्राह्मणपुङ्गवाः}
{तथ्यान्वाप्यथवाऽतथ्यान्गुणानाहुः समागताः}


\twolineshloka
{अनुग्राह्या वयं नूनं भवतामिति मे मतिः}
{यदेवं गुणसंपन्नानस्मान्ब्रूथ विमत्सराः}


\twolineshloka
{धृतराष्ट्रो महाराजः पिता नो दैवतं परम्}
{शासनेऽस्य प्रिये चैव स्थेयं मत्प्रियकाङ्क्षिभिः}


\twolineshloka
{एतदर्थं हि जीवामि कृत्वा ज्ञातिवधं महत्}
{अस्य शुश्रूषणं कार्यं मया नित्यमतन्द्रिणा}


\twolineshloka
{यदि चाहमनुग्राह्यो भवतां सुहृदां तथा}
{धृतरा यथापूर्वं वृत्तिं वर्तितुमर्हथ}


\twolineshloka
{एष नाथो हि जगतो भवतां च मया सह}
{अस्य प्रसादे पृथिवी पाण्डवाः सर्व एव च}


\twolineshloka
{एतन्मनसि कर्तव्यं भवद्भिर्वचनं मम}
{अनुज्ञाप्याथ तान्राजा यथेष्टं गम्यतामिति}


\twolineshloka
{पौरजानपदान्सर्वान्विसृज्य कुरुनन्दनः}
{यौवराज्येन कौन्तेयं भीमसेनमयोजयत्}


\twolineshloka
{मन्त्रे च निश्चये चैव षाङ्गुण्यस्य च चिन्तने}
{विदुरं बुद्धिसंपन्नं प्रीतिमान्स समादिशत्}


\twolineshloka
{कृताकृतपरिज्ञाने तथाऽऽयव्ययचिन्तने}
{संजयं योजयामास वृद्धं सर्वगुणैर्युतम्}


\twolineshloka
{बलस्य परिमाणे च भक्तवेतनयोस्तथा}
{नकुलं व्यादिशद्राजा कर्मणां चान्ववेक्षणे}


\twolineshloka
{परचक्रोपरोधे च दृप्तानां चावमर्दने}
{युधिष्ठिरो महाराज फल्गुनं व्यादिदेश ह}


\twolineshloka
{द्विजानां देवकार्येषु कार्येष्वन्येषु चैव ह}
{धौम्यं पुरोधसां श्रेष्ठं नित्यमेव समादिशत्}


\twolineshloka
{सहदेवं समीपस्थं नित्यमेव समादिशत्}
{तने गोप्यो हि नृपतिः सर्वावस्थो विशांपते}


\twolineshloka
{यान्यानमन्यद्योग्यांश्च येषु येष्विह कर्मसु}
{तांस्तांस्तेष्वेव युयुजे प्रीयमाणो महीपतिः}


\twolineshloka
{विदुरं संजयं चैव युयुत्सुं च महामतिम्}
{अब्रवीत्परवीरघ्नो धर्मात्मा धर्मवत्सलः}


\twolineshloka
{उत्थायोत्थाय तत्कार्यमस्य राज्ञः पितुर्मम}
{सर्वं भवद्भिः कर्तव्यमप्रमत्तैर्यथा मम}


\twolineshloka
{पौरजानपदानां च यानि कार्याणि नित्यशः}
{राजानं समनुज्ञाप्य तानि कार्याणि धर्मतः}


\chapter{अध्यायः ४१}
\twolineshloka
{वैशंपायन उवाच}
{}


\twolineshloka
{ततो युधिष्ठिरो राजा ज्ञातीनां ये हता युधि}
{श्राद्धानि कारयामास तेषां पृथगुदारधीः}


\threelineshloka
{धृतराष्ट्रो ददौ राजा पुत्राणामौर्ध्वदेहिकम्}
{सर्वकामगुणोपेतमन्नं गाश्च धनानि च}
{रत्नानि च विचित्राणि महार्हाणि महायशाः}


\twolineshloka
{युधिष्ठिरस्तु द्रोणस्य कर्णस्य च महात्मनः}
{धृष्टद्युम्नाभिमन्युभ्यां हेडिम्बस्य च रक्षसः}


\twolineshloka
{विराटप्रभृतीनां च सुहृदामुपकारिणाम्}
{द्रुपदद्रौपदेयानां द्रौपद्या सहितो ददौ}


\twolineshloka
{ब्राह्मणानां सहस्राणि पृथगेकैकमुद्दिशन्}
{धनै रत्नैश्च गोभिश्च वस्त्रैश्च समतर्पयत्}


\twolineshloka
{ये चान्ये पृथिवीपाला येषां नास्ति सुहृज्जनः}
{उद्दिश्योद्दिश्य तेषां च चक्रे राजौर्ध्वदेहिकम्}


\twolineshloka
{सभाः प्रपाश्च विविधास्तटाकानि च पाण्डवः}
{सुहृदां कारयामास सर्वेषामौर्ध्वदेहिकम्}


\twolineshloka
{स तेषामनृणो भूत्वा गत्वा लोकेष्ववाच्यताम्}
{कृतकृत्योऽभवद्राजा प्रजा धर्मेण पालयन्}


\twolineshloka
{धृतराष्ट्रं यथापूर्वं गान्धारीं विदुरं तथा}
{सर्वांश्च कौरवान्मान्यान्भृत्यांश्च समपूजयत्}


% Check verse!
याश्च तत्र स्त्रियः काश्चिद्धतवीरा हतात्मजाः ॥सर्वास्ताः कौरवो राजा संपूज्यापालायद्धृणी
\twolineshloka
{दीनान्धकृपणानां च गृहाच्छादनभोजनैः}
{आनृशंस्यपरो राजा चकारानुग्रहं प्रभुः}


\twolineshloka
{स विजित्य महीं कृत्स्नामानृण्यं प्राप्य वैरिषु}
{निःसपत्नः सुखी राजा विजहार युधिष्ठिरः}


\chapter{अध्यायः ४२}
\twolineshloka
{वैशंपायन उवाच}
{}


\twolineshloka
{अभिषिक्तो महाप्राज्ञो राज्यं प्राप्य युधिष्ठिरः}
{दाशार्हं पुण्डरीकाक्षमुवाच प्राञ्जलिः शुचिः}


\twolineshloka
{तव कृष्ण प्रसादेन नयेन न बलेन च}
{बुद्ध्या च यदुशार्दूल तथा विक्रमणेन च}


\twolineshloka
{पुनः प्राप्तमिदं राज्यं पितृपैतामहं मया}
{नमस्ते पुण्डरीकाक्ष पुनः पुनररिंदम}


\twolineshloka
{त्वामेकमाहुः पुरुषं त्वामाहुः सात्वतां पतिम्}
{नामभिस्त्वां बहुविधैः स्तुवन्ति प्रयता द्विजाः}


\twolineshloka
{विश्वकर्मन्नमस्तेऽस्तु विश्वात्मन्विश्वसंभव}
{विष्णो जिष्णो हरे कृष्ण वैकुण्ठ पुरुषोत्तम}


\twolineshloka
{अदित्याः सप्तधा त्वं तु पुराणो गर्भतां गतः}
{पृश्निगर्भस्त्वमेवैकस्त्रियुगं त्वां वदन्त्यपि}


\twolineshloka
{शुचिश्रवा हृषीकेशो घृतार्चिर्हंस उच्यते}
{त्रिचक्षुः शंभुरेकस्त्वं विभुर्दामोदरोऽपि च}


\twolineshloka
{वराहोऽग्निर्बृहद्भानुर्वृषभस्तार्क्ष्यलक्षणः}
{अनीकसाहः पुरुषः शिपिविष्ट उरुक्रमः}


\twolineshloka
{वरिष्ठ उग्रसेनानीः सत्यो वाजसनिर्गुहः}
{अच्युतश्च्यावनोऽरीणां संस्कृतो विकृतिर्वृषः}


\twolineshloka
{कृष्णधर्मस्त्वमेवादिर्वृषदर्भो वृषाकपिः}
{सिन्धुर्विधूर्मिस्त्रिककुप् त्रिधामा त्रिवृदच्युतः}


\twolineshloka
{सम्राड् विराट् स्वराट् चैव स्वराड््भूतमयो भवः}
{विभूर्भूरतिभूः कृष्णः कृष्णवर्त्मा त्वमेव च}


\twolineshloka
{स्विष्टकृद्भिषजावर्तः कपिलस्त्वं च वामनः}
{यज्ञो ध्रुवः पतङ्गश्च जयत्सेनस्त्वमुच्यसे}


\twolineshloka
{शिखण्डी नहुषो बभ्रुर्दिविस्पृक् त्वं पुनर्वसुः}
{सुबभ्रू रुक्मयज्ञश्च सुषेणो दुन्दुभिस्तथा}


\twolineshloka
{गभस्तिनेमिः श्रीपझः पुष्करः शुष्मधारणः}
{ऋभुर्विभुः सर्वसूक्ष्मस्त्व धरित्री च पठ्यसे}


\twolineshloka
{अम्भोनिधिस्त्वं ब्रह्मा त्वं पवित्रं धाम धामवित्}
{हिरण्यगर्भः पुरुषः स्वधा स्वाहा च केशवः}


\threelineshloka
{योनिस्त्वमस्य प्रलयश्च कृष्णत्वमेवेदं सृजसि विश्वमग्रे}
{विश्वं चेदं त्वद्वशे विश्वयोनेनमोस्तु ते शार्ङ्गचक्रासिपाणे ॥वैशंपायन उवाच}
{}


\twolineshloka
{एवं स्तुतो धर्मराजेन कृष्णःसभामध्ये प्रीतिमान्पुष्कराक्षः}
{तमभ्यनन्दद्भारतं पुष्कलाभिर्वाग्भिर्ज्येष्ठं पाण्डवं यादवाग्र्यः}


\twolineshloka
{`एतन्नामशतं विष्णोर्धर्मराजेन कीर्तितम्}
{यः पठेच्छृणुयाद्वापि सर्वपापैः प्रमुच्यते ॥'}


\chapter{अध्यायः ४३}
\twolineshloka
{वैशंपायन उवाच}
{}


\twolineshloka
{ततो विसर्जयामास सर्वास्ताः प्रकृतीर्नृपः}
{विविशुश्चाभ्यनुज्ञाता यथास्वानि गृहाणि ते}


\twolineshloka
{ततो युधिष्ठिरो राजा भीमं भीमपराक्रमम्}
{सान्त्वयन्नब्रवीच्छ्रीमानर्जुनं यमजौ तथा}


\twolineshloka
{शत्रुभिर्विविधैः शस्त्रैः क्षतदेहा महारणे}
{श्रान्ता भवन्तः सुभृशं तापिताः शोकमन्युभिः}


\twolineshloka
{अरण्ये दुःखवसतिर्मत्कृते भरतर्षभाः}
{भवद्भिरनुभूता हि यथा कापुरुषैस्तथा}


\twolineshloka
{यथासुखं यथाजोषं जयोऽयमनुभूयताम्}
{विश्रान्ताँल्लब्धविश्वासाञ्श्वः समेताऽस्मि वः पुनः}


\twolineshloka
{ततो दुर्योधनगृहं प्रासादैरुपशोभितम्}
{बहुरत्नसमाकीर्णं दासीदाससमाकुलम्}


\twolineshloka
{धृतराष्ट्राभ्यनुज्ञातं भ्रात्रा दत्तं वृकोदरः}
{प्रतिपेदे महाबाहुर्मन्दिरं मघवानिव}


\twolineshloka
{यथा दुर्योधनगृहं तथा दुःशासनस्य तु}
{प्रासादभालासंयुक्तं हेमतोरणभूषितम्}


\twolineshloka
{दासीदाससुसंपूर्णं प्रभूतधनधान्यवत्}
{प्रतिपेदे महाबाहुरर्जुनो राजशासनात्}


\twolineshloka
{दुर्मर्षणस्य भवनं दुःशासनगृहाद्वरम्}
{कुबेरभवनप्रख्यं मणिहेमविभूषितम्}


\twolineshloka
{नकुलाय वरार्हाय कर्शिताय महावने}
{ददौ प्रीतो महाराज धर्मपुत्रो युधिष्ठिरः}


\twolineshloka
{दुर्मुखस्य च वेश्माग्र्यं श्रीमत्कनकभूषणम्}
{पूर्णपझदलाक्षीणां स्त्रीणां शयनसंकुलम्}


\twolineshloka
{प्रददौ सहदेवाय संततं प्रियकारिणे}
{मुमुदे तच्च लब्ध्वाऽसौ कैलासं धनदो यथा}


\twolineshloka
{युयुत्सुर्विदुरश्चैव सञ्जयश्च विशांपते}
{सुधर्मा चैव धौम्यश्च यथा स्वाञ्जग्मुरालयान्}


\twolineshloka
{सह सात्यकिना शौरिरर्जुनस्य निवेशनम्}
{विवेश पुरुषव्याघ्रो व्याघ्रो गिरिगुहामिव}


\twolineshloka
{तत्र भक्ष्यान्नपानैस्ते मुदिताः सुसुखोषिताः}
{सुखप्रबद्धा राजानमुपतस्थुर्युधिष्ठिरम्}


\chapter{अध्यायः ४४}
\twolineshloka
{जनमेजय उवाच}
{}


\twolineshloka
{प्राप्य राज्यं महाबाहुर्धर्मपुत्रो युधिष्ठिरः}
{यदन्यदकरोद्विप्र तन्मे वक्तुमिहार्हसि}


\threelineshloka
{भगवान्वा हृषीकेशस्त्रैलोक्यस्य परो गुरुः}
{ऋषे यदकरोद्वीरस्तच्च व्याख्यातुमर्हसि ॥वैशंपायन उवाच}
{}


\twolineshloka
{शृणु तत्त्वेन राजेन्द्र कीर्त्यमानं मयाऽनघ}
{वासुदेवं पुरस्कृत्य यदकुर्वत पाण्डवाः}


\twolineshloka
{प्राप्य राज्यं महाराज कुन्तीपुत्रो युधिष्ठिरः}
{वर्णान्संस्थापयामास नयेन विनयेन च}


\twolineshloka
{ब्राह्मणानां सहस्रं च स्नातकानां महात्मनाम्}
{सहस्रनिष्कैरेकैकं तर्पयामास पाण्डवः}


\twolineshloka
{तथाऽनुजीविनो भृत्यान्संश्रितानतिथीनपि}
{कामैः संतर्पयामास कृपणांस्तार्किकानपि}


\twolineshloka
{पुरोहिताय धौम्याय प्रादादयुतशः स गाः}
{धनं सुवर्णं रजतं वासांसि विविधान्यपि}


\twolineshloka
{कृपाय च महाराज पितृवत्तमतर्पयत्}
{विदुराय च राजाऽसौ पूजां चक्रे यतव्रतः}


\twolineshloka
{भक्ष्यान्नपानैर्विविधैर्वासोभिः शयनासनैः}
{सर्वान्संतोपयामास संश्रितान्ददतां वरः}


\twolineshloka
{लब्धप्रशमनं कृत्वा स राजा राजसत्तम}
{युयुत्सोर्धार्तराष्ट्रस्य पूजां चक्रे महायेशाः}


\twolineshloka
{धृतराष्ट्राय तद्राज्यं गान्धार्यै विदुराय च}
{निवेद्य सुस्थवद्राजा सुखमास्ते युधिष्ठिरः}


\twolineshloka
{तथा सर्वं स नगरं प्रसाद्य भरतर्षभ}
{वासुदेवं महात्मानमभ्यगच्छत्कृताञ्जलिः}


\twolineshloka
{ततो महति पर्यङ्के मणिकाञ्चनभूषिते}
{ददर्श कृष्णमासीनं नीलं मेराविवाम्बुदम्}


\twolineshloka
{जाज्वल्यमानं वपुषा दिव्याभरणभूषितम्}
{पीतकौशेयवसनं हेम्नेवोपगत मणिम्}


\twolineshloka
{कौस्तुभेनोरसिस्थेन मणिनाऽभिविराजितम्}
{उद्यतेवोदयं शैलं सूर्येणाभिविराजितम्}


% Check verse!
नौपम्यं विद्यते तस्य त्रिषु लोकेषु किंचन
\twolineshloka
{सोऽभिगम्य महात्मानं विष्णुं पुरुषसत्तमम्}
{उवाच मधुरं राजा स्मितपूर्वमिदं तदा}


\twolineshloka
{सुखेन ते निशा कच्चिद्व्युष्टा बुद्धिमतां वर}
{कच्चिज्ज्ञानानि सर्वाणि प्रसन्नानि तवाच्युत}


\twolineshloka
{तथैवोपश्रिता देवी बुद्धिर्बुद्धिमतां वर}
{वयं राज्यमनुप्राप्ताः पृथिवी च वशे स्थिता}


\twolineshloka
{तव प्रसादाद्भगवंस्त्रिलोकगतिविक्रम}
{जयं प्राप्ता यशश्चाग्र्यं न च धर्मच्युता वयम्}


\twolineshloka
{तं तथा भाषमाणं तु धर्मराजमरिंदमम्}
{नोवाच भगवान्किंचिद्ध्यानमेवान्वपद्यत}


\chapter{अध्यायः ४५}
\twolineshloka
{युधिष्ठिर उवाच}
{}


\twolineshloka
{किमिदं परमाश्चर्यं ध्यायस्यमितविक्रम}
{कच्चिल्लोकत्रयस्यास्य स्वस्ति लोकपरायण}


\threelineshloka
{`इन्द्रियाणि मनश्चैव बुद्धौ संवेशितानि ते'}
{चतुर्थं ध्यानमार्गं त्वमालम्ब्य पुरुषर्षभ}
{अपक्रान्तो यतो जीवस्तेन मे विस्मितं मनः}


\twolineshloka
{निगृहीतो हि वायुस्ते पञ्चकर्मा शरीरगः}
{इन्द्रियाणि च सर्वाणि मनसि स्थापितानि ते}


\twolineshloka
{वाक्च सत्वं च गोविन्द बुद्धौ संवेशितानि ते}
{सर्वे चैव गुणा देवाः क्षेत्रज्ञे ते निवेशिताः}


\twolineshloka
{नेङ्गन्ति तव रोमाणि स्थिरा बुद्धिस्तथा मनः}
{काष्ठकुड्यशिलाभूतो निरीहश्चासि माधव}


\twolineshloka
{यथा दीपो निवातस्थो निरिङ्गो ज्वलतेऽच्युत}
{तथाऽसि भगवन्देन निश्चलो योगनिश्चयात्}


\twolineshloka
{यदि श्रोतुमिहार्हामि न रहस्यं च ते यदि}
{छिन्धि मे संशयं देव प्रपन्नायाभियाचते}


\twolineshloka
{त्वं हि कर्ता विकर्ता च त्वं क्षरश्चाक्षरश्च ह}
{अनादिनिधनो ह्याद्यस्त्वमेकः पुरुषोत्तम}


\twolineshloka
{त्वं प्रपन्नाय भक्ताय शिरसा प्रणताय च}
{ध्यानस्यास्य यथातत्त्वं ब्रूहि धर्मभृतां वर}


\threelineshloka
{ततः स्वगोचरे न्यस्य मनोबुद्धीन्द्रियाणि च}
{स्मितपूर्वमुवाचेदं भगवान्वासवानुजः ॥वासुदेव उवाच}
{}


\twolineshloka
{शरतल्पगतो भीष्मः शाम्यन्निव हुताशनः}
{मां ध्याति पुरुषव्याघ्रस्ततो मे तद्गतं मनः}


\twolineshloka
{यस्य ज्यातलनिर्घोषं विस्फूर्जितमिवाशनेः}
{न सहेद्देवराजोऽपि तमस्मि मनसा गतः}


\twolineshloka
{येनाभिजित्य तरसा समस्तं राजमण्डलम्}
{ऊढास्तिस्रः पुरा कन्यास्तमस्मि मनसा गतः}


\twolineshloka
{त्रयोविंशतिरात्रं यो योधयामास भार्गवम्}
{न च रामेण निस्तीर्णस्तमस्मि मनसा गतः}


\twolineshloka
{यं गङ्गा गर्भविधिना धारयामास भारतम्}
{वसिष्ठशिष्यं तं तात गतोऽस्मि मनसा नृप}


\twolineshloka
{दिव्यास्राणि महातेजा यो धारयति बुद्धिमान्}
{साङ्गांश्च चतुरो वेदांस्तमस्मि मनसा गतः}


\twolineshloka
{रामस्य दयितं शिष्यं जामदग्न्यस्य पाण्डव}
{आधारं सर्व्रविद्यानां तमस्मि मनसा गतः}


\twolineshloka
{`एकीकृत्येन्द्रियग्रामं मनः संयम्य मेधया}
{शरणं मामुपागच्छत्ततो मे तद्गतं मनः ॥ '}


\twolineshloka
{स हि भूतं भविष्यच्च भवच्च भरतर्षभ}
{वेत्ति धर्मविदां श्रेष्ठस्तमस्मि मनसा गतः}


\twolineshloka
{तस्मिन्हि पुरुषव्याघ्रे शान्ते भीष्मे महात्मनि}
{भविष्यति मही पार्थ नष्टचन्द्रेव शर्वरी}


\twolineshloka
{तद्युधिष्ठिर गाङ्गेयं भीष्मं भीमपराक्रमम्}
{अभिगम्योपसंगृह्य पृच्छ यत्ते मनोगतम्}


\twolineshloka
{चातुर्विद्यं चातुर्होत्रं चातुराश्रम्यमेव च}
{राजधर्मांश्चि निखिलान्पृच्छैनं पृथिवीपते}


\twolineshloka
{तस्मिन्नस्तमिते भीष्मे कौरवाणां धुंरधरे}
{ज्ञानान्यल्पीभविष्यन्ति तस्मात्त्वां चोदयाम्यहम्}


\twolineshloka
{तच्छ्रुत्वा वासुदेवस्य तथ्यं वचनमुत्तमम्}
{साश्रुकण्ठः स धर्मज्ञो जनार्दनमुवाच ह}


\twolineshloka
{यद्भवानाह भीष्मस्य प्रभावं प्रति माधव}
{तथा तन्नात्र संदेहो विद्यते मम माधव}


\twolineshloka
{महाभाग्यं च भीष्मस्य प्रभावश्च महाद्युते}
{श्रुतं मया कथयतां ब्राह्मणानां महात्मनाम्}


\twolineshloka
{भवांश्च कर्ता लोकानां यद्ब्रवीत्यरिसूदन}
{तथा तद्रनभिध्येयं वाक्यं यादवनन्दन}


\twolineshloka
{यदि त्वनुग्रहवती बुद्धिस्ते मयि माधव}
{त्वामग्रतः पुरस्कृत्य भीष्मं यास्यामहे वयम्}


\twolineshloka
{आवृत्ते भगवत्यर्के स हि लोकान्गमिष्यति}
{त्वद्दर्शनं महाबाहो तस्मादर्हति कौरवः}


\threelineshloka
{तव ह्याद्यस्य देवस्य क्षरस्यैवाक्षरस्य च}
{दर्शनं त्वस्य लाभः स्यात्त्वं हि ब्रह्ममयो निधिः ॥वैशंपायन उवाच}
{}


\twolineshloka
{श्रुत्वैवं धर्मराजस्य वचनं मधुसूदनः}
{पार्श्वस्थं सात्यकिं प्राह रथो मे युज्यतामिति}


\twolineshloka
{सात्यकिस्त्वाशु निष्क्रम्य केशवस्य समीपतः}
{दारुकं प्राह कृष्णस्य युज्यतां रथ इत्युत}


\twolineshloka
{स सात्यकेराशु वचो निशम्यरथोत्तमं काञ्चनभूषिताङ्गम्}
{मसारगल्वर्कमयैर्विभङ्गैर्विभूषितं हेमनिबद्धचक्रम्}


\twolineshloka
{दिवाकरांशुप्रभमाशुगामिनंविचित्रनानामणिभूषितान्तरम्}
{नवोदितं सूर्यमिव प्रतापिनंविचित्रतार्क्ष्यध्वजिनं पताकिनम्}


\twolineshloka
{सुग्रीवशैब्यप्रप्नुखैर्वराश्वैर्मनोजवैः काञ्चनभूषिताङ्गैः}
{संयुक्तमावेदयदच्युतायकृताञ्जलिर्दारुको राजसिंह}


\chapter{अध्यायः ४६}
\twolineshloka
{जनमेजय उवाच}
{}


\threelineshloka
{शरतल्पे शयानस्तु भरतानां पितामहः}
{कथमुत्सृष्टवान्देहं कं च योगमधारयत् ॥वैशंपायन उवाच}
{}


\twolineshloka
{शृणुष्वावहितो राजञ्शुचिर्भूत्वा समाहितः}
{भीष्मस्य कुरुशार्दूल देहोत्सर्गं महात्मनः}


\twolineshloka
{प्रवृत्तमात्रे त्वयनमुत्तरेण दिवाकरे}
{`शुक्लपक्षस्य चाष्टभ्यां माघमासस्य पार्थिव}


\twolineshloka
{प्राजापत्ये च नक्षत्रे मध्यं प्राप्ते दिवाकरे}
{'समावेशयदात्मानमात्मत्येव समाहितः}


\twolineshloka
{विकीर्णांशुरिवादित्यो भीष्मः शरशतैश्चितः}
{शुशुभे परया लक्ष्म्या वृतो ब्राह्मणसत्तमैः}


\twolineshloka
{व्यासेन देवश्रवसा नारदेन सुरर्षिणा}
{देवस्थानेन वात्स्येन तथाऽश्मकसुमन्तुना}


\twolineshloka
{तथा जैमिनिना चैव पैलेन च महात्मना}
{शाण्डिल्यदेवलाभ्यां च मैत्रेयेण च धीमता}


\twolineshloka
{असितेन वसिष्ठेन कौशिकेन महात्मना}
{हारितलोमशाभ्यां च तथाऽऽत्रेयेण धीमता}


\twolineshloka
{[बृहस्पतिश्च शुक्रश्च च्यवनश्च महामुनिः}
{सनत्कुमारः कपिलो चाल्मीकिस्तुम्बुरुः कुरुः}


\twolineshloka
{मौद्गल्यो भार्गवो रामस्तृणबिन्दुर्महामुनिः}
{पिप्पलादोऽथ वायुश्च सवर्तः पुलहः कचः}


\twolineshloka
{काश्यपश्च पुलस्त्यश्च क्रतुर्दक्षः पराशरः}
{मरीचिरङ्गिराः काश्यो गौतमो गालवो मुनिः}


\threelineshloka
{धौम्यो विभाण्डो माण्डव्योधौम्रः कृष्णानुभौतिकः}
{उलूकः परमो विप्रो मार्कण्डेयो महामुनिः}
{भास्करिः पूरणः कृष्णः सूतः परमधार्मिकः}


\twolineshloka
{एतैश्चान्यैर्मुनिगणैर्महाभागैर्महात्मभिः}
{श्रद्धादमशमोपेतैर्वृतश्चन्द्र इव ग्रहैः}


\twolineshloka
{भीष्मस्तु पुरुषव्याघ्रः कर्मणा मनसा गिरा}
{शरतल्पगतः कृष्णं प्रदध्यौ प्राञ्जलिः शुचिः}


\threelineshloka
{स्वरेण हृष्टपुष्टेन तुष्टाव मधुसूदनम्}
{योगेश्वरं पझनाभं विष्णुं जिष्णुं जगत्पतिम्}
{`अनादिनिधनं विष्णुमात्मयोनिं सनातनम् ॥'}


\twolineshloka
{कृताञ्जलिपुटो भूत्वा वाग्विदां प्रवरः प्रभुः ॥भीष्मः परमधर्मात्मा वासुदेवमथास्तुवत् ॥भीष्म उवाच}
{}


\twolineshloka
{आरिराधयिषुः कृष्णं वाचं जिगदिषामि याम्}
{तया व्याससमासिन्या प्रीयतां पुरुषोत्तमः}


\twolineshloka
{शुचिं शुचिपदं हंसं तत्परं परमेष्ठिनम्}
{युक्त्वा सर्वात्मनाऽऽत्मानं तं प्रपद्ये प्रजापतिम्}


\twolineshloka
{अनाद्यन्तं परं ब्रह्म न देवा नर्षयो विदुः}
{एकोऽयं वेद भगवान्धाता नारायणो हरिः}


\twolineshloka
{नारायणादृषिगणास्तथा सिद्धमहोरगाः}
{देवा देवर्षयश्चैव यं विदुर्दुःखभेषजम्}


\twolineshloka
{देवदानवगन्धर्वा यक्षराक्षसपन्नगाः}
{यं न जानन्ति को ह्येष कुतो वा भगवानिति}


\twolineshloka
{`यमाहुर्जगतः कोशं यस्मिंश्च निहिताः प्रजाः}
{यस्मिँल्लोकाः स्फुरन्त्येते जाले शकुनयो यथा ॥'}


\twolineshloka
{यस्मिन्विश्वानि भूतानि तिष्ठन्ति च विशन्ति च}
{गुणभूतानि भूतेशे सूत्रे मणिगणा इव}


% Check verse!
`यं च विश्वस्य कर्तारं जगतस्तस्थुषां पतिम्वदन्ति जगतोऽध्यक्षमध्यात्मपरिचिन्तकाः ॥ '
\twolineshloka
{यस्मिन्नित्ये तते तन्तौ दृढे स्रगिव तिष्ठति}
{सदसद्ग्रथितं विश्वं विश्वाङ्गे विश्वकर्मणि}


\twolineshloka
{हरिं सहस्रशिरसं सहस्रचरणेक्षणम्}
{सहस्रबाहुमकुटं सहस्रवदनोज्ज्वलम्}


\threelineshloka
{प्राहुर्नारायणं देवं यं विश्वस्य परायणम्}
{अणीयसामणीयांसं स्थविष्ठं च स्थवीयसाम्}
{गरीयसां गरिष्ठं च श्रेष्ठं च श्रेयसामपि}


\twolineshloka
{चं वाकेष्वनुवाकेषु निषत्सूपनिषत्सु च}
{गृणन्ति सत्यकर्माणं सत्यं सत्येषु सामसु}


\twolineshloka
{चतुर्भिश्चतुरात्मानं सत्वस्थं सात्वतां पतिम्}
{यं दिव्यैर्देवमर्चन्ति गुह्यैः परमनामभिः}


\twolineshloka
{[यस्मिन्नित्यं तपस्तप्तं यदङ्गेष्वनुतिष्ठति}
{सर्वात्मा सर्ववित्सर्वः सर्वज्ञः सर्वभावनः ॥ ]}


\twolineshloka
{यं देवं देवकी देवी वसुदेवादजीजनत्}
{भौमस्य ब्रह्मणो गुप्त्यै दीप्तमग्निमिवारणिः}


\twolineshloka
{यमनन्यो व्यपेताशीरात्मानं वीतकल्मषम्}
{इष्ट्वानन्त्याय गोविन्दं पश्यत्यात्मानमात्मनि}


\threelineshloka
{`अप्रतर्क्यमविज्ञेयं हरिं नारायणं विभुम्}
{'अतिवाय्विन्द्रकर्माणमतिसूर्याग्नितेजसम्}
{अतिबुद्धीन्द्रियात्मानं तं प्रपद्ये प्रजापतिम्}


\twolineshloka
{पुराणे पुरुषं प्रोक्तं ब्रह्मप्रोक्तं युगादिषु}
{क्षये संकर्षणं प्रोक्तं तमुपास्यमुपास्महे}


\twolineshloka
{यमेकं बहुधात्मानं प्रादुर्भूतमधोक्षजम्}
{नान्यभक्ताः क्रियावन्तो यजन्ते सर्वकामदम्}


\twolineshloka
{ऋतमेकाक्षरं ब्रह्म यत्तत्सदसतः परम्}
{अनादिमध्यपर्यन्तं न देवा नर्षयो विदुः}


\twolineshloka
{यं सुरासुरगन्धर्वाः सिद्धा ऋषिमहोरगाः}
{प्रयता नित्यमर्चन्ति परमं सुखभेषजम्}


\twolineshloka
{अनादिनिधनं देवमात्मयोनिं सनातनम्}
{अप्रेक्ष्यमनभिज्ञेयं हरिं नारायणं प्रभुम् ॥अथ भीष्मस्तवराजः}


\twolineshloka
{हिरण्यवर्णं यं गर्भमदितिर्दैत्यनाशनम्}
{एकं द्वादशधा जज्ञे तस्मै सूर्यात्मने नमः}


\twolineshloka
{शुक्ले देवान्पितॄन्कृष्णे तर्पयत्यमृतेन यः}
{यश्च राजा द्विजातीनां तस्मै सोमात्मने नमः}


\twolineshloka
{`हुताशनमुखैर्देवैर्धार्यते सकलं जगत्}
{हविः प्रथमभोक्ता यस्तस्मै होत्रात्मने नमः ॥ '}


\twolineshloka
{महतस्तमसः पारे पुरुषं ह्यतितेजसम्}
{यं ज्ञात्वा मृत्युमत्येति तस्मै ज्ञेयात्मने नमः}


\twolineshloka
{यं बृहन्तं बृहत्युक्थे यमग्नौ यं महाध्वरे}
{यं विप्रसङ्घा गायन्ति तस्मै वेदात्मने नमः}


\twolineshloka
{पादाङ्गं संधिपर्वाणं स्वरव्यञ्जनभूषितम्}
{यमाहुरक्षरं विप्रास्तस्मै वागात्मने नमः}


\twolineshloka
{[यज्ञाङ्गो यो वराहो वै भूत्वा गामुज्जहारह}
{लोकत्रयहितार्थाय तस्मै वीर्यात्मने नमः ॥]}


\twolineshloka
{ऋग्यजुःसामधामानं दशार्धहविराकृतिम्}
{यं सप्ततन्तुं तन्वन्ति तस्मै यज्ञात्मने नमः}


\twolineshloka
{[चतुर्भिश्च चतुर्भिश्च द्वाभ्यां पञ्चभिरेव च}
{हूयते च पुनर्द्वाभ्यां तस्मै होमात्मने नमः ॥]}


\twolineshloka
{यः सुपर्णो यजुर्नाम च्छन्दोगात्रस्त्रिवृच्छिराः}
{रथन्तरबृहत्पक्षस्तस्मै स्तोत्रात्मने नमः}


\twolineshloka
{यः सहस्रसवे सत्रे जज्ञे विश्वसृजामृषिः}
{हिरण्यपक्षः शकुनिस्तस्मै तार्क्ष्यात्मने नमः}


\twolineshloka
{यश्चिनोति सतां सेतुमृतेनामृतयोनिना}
{धर्मार्थव्यवहाराङ्गैस्तस्मै सत्यात्मने नमः}


\twolineshloka
{यं पृथग्धर्मचरणाः पृथग्धर्मफलैषिणः}
{पृथग्धर्मैः समर्चन्ति तस्मै धर्मात्मने नमः}


\twolineshloka
{[यतः सर्वे प्रसूयन्ते ह्यनङ्गात्माङ्गदेहिनः}
{उन्मादः सर्वभूतानां तस्मै कामात्मने नमः ॥]}


\twolineshloka
{यं तं व्यक्तस्थमव्यक्तं विचिन्वन्ति महर्षयः}
{क्षेत्रे क्षेत्रज्ञमासीनं तस्मै क्षेत्रात्मने नमः}


\twolineshloka
{यं दृगात्मानमात्मस्थं वृतं षोडशभिर्गुणैः}
{प्राहुः सप्तदशंसाङ्ख्यास्तस्मै साङ्ख्यात्मने नमः}


\twolineshloka
{यं विनिद्रा जितश्वासाः संतुष्टाः संयतेन्द्रियाः}
{ज्योतिः पश्यन्ति युञ्जानास्तस्मै योगात्मेन नमः}


\twolineshloka
{अपुण्यपुण्योपरमे यं पुनर्भवनिर्भयाः}
{शान्ताः संन्यासिनो यान्ति तस्मै मोक्षात्मने नमः}


\twolineshloka
{यस्याग्रिरास्यं द्यौर्मूर्धा खं नाभिश्चरणौ क्षितिः}
{सूर्यश्चक्षुर्दिशः श्रोत्रं तस्मै लोकात्मने नमः}


\twolineshloka
{युगेष्वावर्तते योंऽशैर्मासर्त्वयनहायनैः}
{सर्गप्रलययोः कर्ता तस्मै कालात्मने नमः}


\twolineshloka
{योऽसौ युगसहस्रान्ते प्रदीप्तार्चिर्विभावसुः}
{संभक्षयति भूतानि तस्मै घोरात्मने नमः}


\twolineshloka
{संभक्ष्य सर्वभूतानि कृत्वा चैकार्णवं जगत्}
{बालः स्वपिति यश्चैकस्तस्मै मायात्मने नमः}


\twolineshloka
{सहस्रशिरसे तस्मै पुरुषायामितात्मने}
{चतुःसमुद्रपयसि योगनिद्रात्मने नमः}


\twolineshloka
{अजस्य नाभावध्येकं यस्मिन्विश्वं प्रतिष्ठितम्}
{पुष्करं पुष्कराक्षस्य तस्मै पझात्मने नमः}


\twolineshloka
{यस्य केशेषु जीमूता नद्यः सर्वाङ्गसंधिषु}
{कुक्षौ समुद्राश्चत्वारस्तस्मै तोयात्मने नमः}


\twolineshloka
{यस्मात्सर्गाः प्रवर्तन्ते सर्गप्रलयविक्रियाः}
{यस्मिंश्चैव प्रलीयन्ते तस्मै हेत्वात्मने नमः}


\twolineshloka
{[यो निषण्णो भवेद्रात्रौ दिवा भवति विष्ठितः}
{इष्टानिष्टस्य च द्रष्टा तस्मै द्रष्ट्रात्मने नमः ॥]}


\twolineshloka
{अकार्यः सर्वकार्येषु धर्मकार्यार्थमुद्यतः}
{वैकुण्ठस्य हि तद्रूपं तस्मै कार्यात्मने नमः}


\twolineshloka
{ब्रह्म वक्तं भुजौ क्षत्रं कृत्स्नमूरूदरं विशः}
{पादौ यस्याश्रिताः शूद्रास्तस्मैवर्णात्मने नमः}


\twolineshloka
{अन्नपानेन्धनमयो रसप्राणविवर्धनः}
{यो धारयति भूतानि तस्मै प्राणात्मने नमः}


\twolineshloka
{[प्राणानां धारणार्थाय योऽन्नं भुङ्क्ते चतुर्विधम्}
{अन्तर्भूतः पचत्यग्निस्तस्मै पाकात्मने नमः ॥ ]}


\twolineshloka
{विषये वर्तमानानां यं तं वैषयिकैर्गुणैः}
{प्राहुर्विषयगोप्तारं तस्मै गोप्त्रात्मने नमः}


\twolineshloka
{अप्रमेयशरीराय सर्वतो बुद्धिचक्षुषे}
{अपारपरिमाणाय तस्मै दिव्यात्मने नमः}


\twolineshloka
{परः कालात्परो यज्ञात्परः सदसतश्च यः}
{अनादिरादिर्विश्वस्य तस्मै विश्वात्मने नमः}


\twolineshloka
{वैद्युतो जाठरश्चैव पावकः शुचिरेव च}
{दहनः सर्वभक्षाणां तस्मै वह्न्यात्मने नमः}


\twolineshloka
{रसातलगतः श्रीमार्ननन्तो भगवान्प्रभुः}
{जगद्धारयते योऽसौ तस्मै शेषात्मने नमः}


\twolineshloka
{ज्वलनार्केन्दुताराणां ज्योतिषां दिव्यमूर्तिनाम्}
{यस्तेजयति तेजांसि तस्मै तेजात्मने नमः}


\twolineshloka
{आत्मज्ञानमिदं ज्ञानं ज्ञात्वा पञ्चस्ववस्थितम्}
{यं ज्ञानेनाधिगच्छन्ति तस्मै ज्ञानात्मने नमः}


\twolineshloka
{साङ्ख्यैर्योगैर्विनिश्चित्य साध्यैश्च परमर्षिभिः}
{यस्य तु ज्ञायते तत्वं तस्मै गुह्यात्मने नमः}


\twolineshloka
{जटिने दण्डिने नित्यं लम्बोदरशरीरिणे}
{कमण्डलुनिषङ्गाय तस्मै ब्रह्मात्मने नमः}


\twolineshloka
{[शूलिने त्रिदशेशाय त्र्यम्बकाय महात्मने}
{भस्मदिग्धोर्ध्वलिङ्गाय तस्मा रुद्रात्मने नमः}


\twolineshloka
{चन्द्रार्धकृतशीर्षाय व्यालयज्ञोपवीतिने}
{पिनाकशूलहस्ताय तस्मै उग्रात्मने नमः ॥]}


\twolineshloka
{यो जातो वसुदेवेन देवक्यां यदुनन्दनः}
{शङ्खचक्रगदापाणिर्वासुदेवात्मने नमः}


\twolineshloka
{शिरःकपालमालाय व्याघ्रचर्मनिवासिने}
{भस्मदिग्धशरीराय तस्मै रुद्रात्मने नमः}


\twolineshloka
{यो मोहयति भूतानि सर्वपाशानुबन्धनैः}
{सर्वस्य रक्षणार्थाय तस्मै मोहात्मने नमः}


\twolineshloka
{चैतन्यं सर्वतो नित्यं सर्वप्राणिहृदि स्थितम्}
{सर्वातीततरं सूक्ष्मं तस्मै सूक्ष्मात्मने नमः}


\twolineshloka
{पञ्चभूतात्मभूताय भूतादिनिधनाय च}
{अक्रोधद्रोहमोहाय तस्मै शान्तात्मने नमः}


\twolineshloka
{यस्मिन्सर्वं यतः सर्वं यः सर्वं सर्वतश्च यः}
{यश्च सर्वमयो देवस्तस्मै सर्वात्मने नमः}


\twolineshloka
{यः शेते क्षीरपर्यङ्के दिव्यनागविभूषिते}
{फणासहस्ररचिते तस्मै निद्रात्मने नमः}


\twolineshloka
{विश्वे च मरुतश्चैव रुद्रादित्याश्विनावपि}
{वसवः सिद्धसाध्याश्च तस्मै देवात्मने नमः}


\twolineshloka
{अव्यक्तं बुद्ध्यहंकारो मनोबुद्धीन्द्रियाणि च}
{तन्मात्राणि विशेषाश्च तस्मै तत्वात्मने नमः}


\twolineshloka
{भूतं भव्यं भविष्यच्च भूतादिप्रभवाव्ययः}
{योऽग्रजः सर्वभूतानां तस्मै भूतात्मने नमः}


\twolineshloka
{यं हि सूक्ष्मं विचिन्वन्ति परं सूक्ष्मविदो जनाः}
{सूक्ष्मात्सूक्ष्मं च यद्ब्रह्म तस्मै सूक्ष्मात्मने नमः}


\twolineshloka
{मत्स्यो भूत्वा विरिञ्चाय येन वेदाः समाहृताः}
{रसातलगतः शीघ्रं तस्मै मत्स्यात्मने नमः}


\twolineshloka
{मन्दराद्रिर्धृतो येन प्राप्ते ह्यमृतमन्थने}
{अतिकर्कशदेहाय तस्मै कूर्मात्मने नमः}


\twolineshloka
{वाराहं रूपमास्थाय महीं सवनपर्वताम्}
{उद्धरत्येकदंष्ट्रेण तस्मै क्रोडात्मने नमः}


\twolineshloka
{नारसिंहवपुः कृत्वा सर्वलोकभयंकरम्}
{हिरण्यकशिपुं जघ्ने तस्मै सिंहात्मने नमः}


\twolineshloka
{पिङ्गेक्षणसटं यस्य रूपं दंष्ट्रानखैर्युतम्}
{दानवेन्द्रान्तकरणं तस्मै दृप्तात्मने नमः}


\threelineshloka
{यं न देवा न गन्धर्वा न दैत्या न च दानवाः}
{तत्वतो हि विजानन्ति तस्मै सूक्ष्मात्मने नमः ॥]वामनं रूपमास्थाय बलिं संयम्य मायया}
{त्रैलोक्यं क्रान्तवान्यस्तु तस्मै क्रान्तात्मने नमः}


\twolineshloka
{जमदग्निसुतो भूत्वा रामः शस्त्रभृतां वरः}
{महीं निःक्षत्रियां चक्रे तस्मै रामात्मने नमः}


\twolineshloka
{त्रिःसप्तकृत्वो यश्चैको धर्मे व्युत्क्रान्तिगौरवात्}
{जघान क्षत्रियान्सङ्ख्ये तस्मै क्रोधात्मने नमः}


\twolineshloka
{[विभज्य पञ्चधाऽऽत्मानं वायुर्भूत्वा शरीरगः}
{यश्चेष्टयति भूतानि तस्मै वाय्वात्मने नमः ॥]}


\twolineshloka
{रामो दशिरथिर्भूत्वा पुलस्त्यकुलनन्दनम्}
{जघान रावणं सङ्ख्ये तस्मै क्षत्रात्मने नमः}


\twolineshloka
{यो हली मुसली श्रीमान्नीलाम्बरधरः स्थितः}
{रामाय रौहिणेयाय तस्मै भोगात्मने नमः}


\twolineshloka
{शङ्खिने चक्रिणे नित्यं शार्ङ्गिणे पीतवाससे}
{वनमालाधरायैव तस्मै कृष्णात्मने नमः}


\twolineshloka
{वसुदेवसुतः श्रीमान्क्रीडितो नन्दगोकुले}
{कंसस्य निधनार्थाय तस्मै क्रीडात्मने नमः}


\twolineshloka
{वासुदेवत्वमागम्य यदोर्वंशसमुद्भवः}
{भूभारहरणं चक्रे तस्मै कृष्णात्मने नमः}


\twolineshloka
{सारथ्यमर्जुनस्याजौ कुर्वन्गीतामृतं ददौ}
{लोकत्रयोपकाराय तस्मै ब्रह्मात्मने नमः}


\twolineshloka
{दानवांस्तु वशे कृत्वा पुनर्बुद्धत्वमागतः}
{सर्गस्य रक्षणार्थाय तस्मै बुद्धात्मने नमः}


\twolineshloka
{हनिष्यति कलौ प्राप्ते म्लेच्छांस्तुरगवाहनः}
{धर्मसंस्थापको यस्तु तस्मै कल्क्यात्मने नमः}


\twolineshloka
{तारान्वये कालनेमिं हत्वा दानवपुङ्गवम्}
{ददौ राज्यं महेन्द्राय तस्मै साङ्ख्यात्मने नमः}


\twolineshloka
{यः सर्वप्राणिनां देहे साक्षिभूतो ह्यवस्थितः}
{अक्षरः क्षरमाणानां तस्मै साक्ष्यात्मने नमः}


\twolineshloka
{नमोस्तु ते महादेव नमस्ते भक्तवत्सल}
{सुब्रह्मण्य नमस्तेऽस्तु प्रसीद परमेश्वर}


\twolineshloka
{अव्यक्तव्यक्तरूपेण व्याप्तं सर्वं त्वया विभो}
{नारायणं सहस्राक्षं सर्वलोकमहेश्वरम्}


\twolineshloka
{हिरण्यनाभ यज्ञाङ्गममृतं विश्वतोमुखम्}
{सर्वदा सर्वकार्येषु नास्ति तेषाममङ्गलम्}


\twolineshloka
{येषां हृदिस्थो देवेशो मङ्गलायतनं हरिः}
{मङ्गल भगवान्विष्णुर्मङ्गलं मधुसूदनः}


\twolineshloka
{मङ्गलं पुण़्डरीकाक्षो मङ्गलं गरुडध्वजः}
{विश्वकर्मन्नमस्तेऽस्तु विश्वात्मन्विश्वसंभव}


\twolineshloka
{अपवर्गस्थभूतानां पञ्चानां परमास्थित}
{नमस्ते त्रिषु लोकेषु वमस्ते परतस्त्रिषु}


\twolineshloka
{नमस्ते दिक्षु सर्वासु त्वं हि सर्वपरायणम्}
{नमस्ते भगवन्विष्णो त्येकानां प्रभवाव्यय}


\twolineshloka
{त्वं हि कर्ता हृषीकेशः संहर्ता चापराजितः}
{तेन पश्यामि ते दिव्यान्भावान्हि त्रिषुवर्त्मसु}


\threelineshloka
{तच्च पश्यामि तत्वेन यत्ते रूपं सनातनम्}
{दिवं ते शिरसा व्याप्तं पद्भ्यां देवी वसुंधरा}
{विक्रमेण त्रयो लोकाः पुरुषोऽसि सनातनः}


\twolineshloka
{[दिशो भुजा रविश्चक्षुर्वीर्ये शुक्रः प्रतिष्ठितः}
{सप्तमार्गा निरुद्धास्ते वायोरमिततेजसः ॥]}


\twolineshloka
{व्यक्ताव्यक्तस्वरूपेण व्याप्तं सर्वं त्वया विभो}
{अव्यक्तं ब्राह्मणं रूपं व्यक्तमेतच्चराचरम्}


\twolineshloka
{अतसीपुष्पसंकाशं पीतवाससमच्युतम्}
{ये नमस्यन्ति गोविन्दं न तेषां विद्यते भयम्}


\twolineshloka
{[एकोऽपि कृष्णस्य कृतः प्रणामोदशाश्वमेधावभृथेन तुल्यः}
{दशाश्वमेधी पुनरेति जन्मकृष्णप्रणामी न पुनर्भवाय}


\twolineshloka
{कृष्णव्रताः कृष्णमनुस्मरन्तोरात्रौ च कृष्णं पुनरुत्थिता ये}
{ते कृष्णदेहाः प्रविशन्ति कृष्णमाज्यं यथा मन्त्रहुतं हुताशे}


\twolineshloka
{नमो नरकसंत्रासरक्षामण्डलकारिणे}
{संसारनिम्नगावर्ततरिकाष्ठाय विष्णवे}


\twolineshloka
{नमो ब्रह्मण्यदेवाय गोब्राह्मणहिताय च}
{जगद्धिताय कृष्णाय गोविन्दाय नमोनमः}


\twolineshloka
{प्राणकान्तारपाथेयं संसारोच्छेदभेषजम्}
{दुःखशोकपरित्राणं हरिरित्यक्षरद्वयम् ॥]}


\twolineshloka
{नारायणपरं ब्रह्म नारायणपरं तपः}
{नारायणपरं सत्यं नारायणपरं परम्}


\twolineshloka
{यथा विष्णुमयं सत्यं यथा विष्णुमयं हविः}
{तथा विष्णुमयं सर्वं पाप्मा ने नश्यतां तथा}


\twolineshloka
{तस्य यज्ञवराहस्य विष्णोरमिततेजसः}
{प्रणामं येऽपि कुर्वन्ति तेषामपि नमोनमः}


\twolineshloka
{त्वां प्रपन्नाय भक्ताय गतिमिष्टां जिगीषवे}
{यच्छ्रेयः पुण्डरीकाक्ष तद्ध्यायस्व सुरोत्तम}


\threelineshloka
{इति विद्यातपोयोनिरयोनिर्विष्णुरीडितः}
{वाग्यज्ञेनार्चितो देवः प्रीयतां मे जनार्दनः ॥वैशंपायन उवाच}
{}


\twolineshloka
{एतावदुक्त्वा वचनं भीष्मस्तद्रतमानसः}
{नम इत्येव कृष्णाय प्रणाममकरोत्तदा}


\twolineshloka
{तस्मिन्नुपरते वाक्ये ततस्ते ब्रह्मवादिनः}
{भीष्मं वाग्भिर्वाष्पगलास्तमानर्चुर्महाद्युतिम्}


\twolineshloka
{तेऽस्तुवन्तश्च विप्रेन्द्राः केशवं पुरुषोत्तमम्}
{भीष्मं च शनकैः सर्वे प्रशशंसुः पुनः पुनः}


\twolineshloka
{अधिगम्य तु योगेन भक्तिं भीष्मस्य माधवः}
{त्रैलोक्यदर्शनं ज्ञानं दिव्यं दत्त्वा ययौ हरिः}


\twolineshloka
{विदित्वा भक्तियोगं तं भीष्मस्य पुरुषोत्तमः}
{सहसोत्थाय तं हृष्टो यानमेवान्वपद्यत}


\twolineshloka
{केशवः सात्यकिश्चैव रथेनकेन जग्मतुः}
{अपरेण महात्मानौ युधिष्ठिरधनञ्जयौ}


\twolineshloka
{भीमसेनो यमौ चोभौ रथमेकं समास्थिताः}
{कृपो युयुत्सुः सूतश्च सञ्जयश्चापरं रथम्}


\twolineshloka
{ते रथैर्नगराकारैः प्रयाताः पुरुषर्षभाः}
{नेमिघोषेण महता कम्पयन्ते वसुंधराम्}


\twolineshloka
{ततो गिरः पुरुषवरस्तवेरितांद्विजेरिताः पथिषु मनाक् स शुश्रुवे}
{कृताञ्जलिं प्रणतमथापरं जनंस केशिहा मुदितमनास्थनन्दत}


\twolineshloka
{इति स्मरन्पठति च शार्ङ्गधन्वनःशृणोतु वा यदुकुलनन्दनस्तवम्}
{स चक्रभृत्प्रतिहतसर्वाकिल्विषोजनार्दनं प्रविशति देहसंक्षये}


\twolineshloka
{यं योगिनः प्राणवियोगकालेयत्नेन चित्ते विनिवेशयन्ति}
{स तं पुरस्ताद्धरिमीक्षमाणःप्राणाञ्जहौ प्राप्तफलो हि भीष्मः}


\twolineshloka
{स्ववराजः समाप्तोऽयं विष्णोरद्भुतकर्मणः}
{गाङ्गेयेन पुरा गीतो महापातकनाशनः}


\twolineshloka
{इमं नरः स्तवराजं मुमुक्षुःपठञ्शुचिः कलुषितकल्मषापहम्}
{अतीत्य लोकान्मलिनः समामतान्पदं स गच्छत्यमृतं महात्मनः}


\chapter{अध्यायः ४७}
\twolineshloka
{वैशंपायन उवाच}
{}


\twolineshloka
{ततः स च हृषीकेशः स च राजा युधिष्ठिरः}
{कृपादयश्च ते सर्वे चत्वारः पाण्डवाश्च ते}


\twolineshloka
{रथैस्तैर्नगरप्रख्यैः पताकाध्वजशोभितैः}
{ययुराशु कुरुक्षेत्रं वाजिभिः शीघ्रगामिभिः}


\twolineshloka
{तेऽवतीर्य कुरुक्षेत्रे केशमढज्जास्थिसंकुले}
{देहन्यासः कृंतो यत्र क्षत्रियैस्तैर्महारथैः}


\twolineshloka
{गजाश्वदेहास्थिचयैः पर्वतैरिव संचितम्}
{नरशीर्षकपालैश्च हंसैरिव च सर्वशः}


\twolineshloka
{चितासहस्रैर्निचितं वर्मशस्त्रसमाकुलम्}
{आपानभूमिं कालस्य तदा भुक्तोज्झितामिव}


\twolineshloka
{भूतसङ्घानुचरितं रक्षोगणनिषेवितम्}
{पश्यन्तस्ते कुरुक्षेत्रं ययुराशु महारथाः}


\twolineshloka
{गच्छन्नेव महाबाहुः सर्वं यादवनन्दनः}
{युधिष्ठिराय प्रोवाच जामदग्न्यस्य विक्रमम्}


\twolineshloka
{अमी रामह्रदाः पञ्च दृश्यन्ते पार्थ दूरतः}
{येषु संतर्पयामास पितॄन्क्षत्रियशोणितैः}


\threelineshloka
{त्रिःसप्तकृत्वो वसुधां कृत्वा निःक्षत्रियां प्रभुः}
{इहेदानीं ततो रामः कर्मणो विरराम ह ॥युधिष्ठिर उवाच}
{}


\twolineshloka
{त्रिःसप्तकृत्वः पृथिवी कृता निःक्षत्रिया पुरा}
{रामेणेति यदात्थ त्वमत्र मे संशयो महान्}


\twolineshloka
{क्षत्रबीजं यथा दग्धं रामेण यदुपुङ्गव}
{कथं भूयः समुत्यत्तिः क्षत्रस्यामितविक्रम}


\twolineshloka
{महात्मना भगवता रामेण यदुपुङ्गव}
{कथमुत्सादित्तं क्षत्रं कथमृद्धिगतं पुनः}


\twolineshloka
{महता रथयुद्धेन कोटिशः क्षत्रिया हताः}
{तथाऽभूच्च मही कीर्णा क्षत्रियैर्वदतां वर}


\twolineshloka
{किमर्थं भार्गवेणेदं क्षत्रमुत्सादितं पुरा}
{रामेण यदुशार्दूल कुरुक्षेत्रे महात्मना}


\threelineshloka
{एतन्मे छिन्धि वार्ष्णेय संशयं तार्क्ष्यकेतन}
{आगमो हि परः कृष्ण त्वत्तो नो वासवानुज ॥वैशंपायन उवाच}
{}


\twolineshloka
{ततो व्रजन्नेव गदाग्रजः प्रभुःशशंस तस्मै निखिलेन तत्त्वतः}
{युधिष्ठिरायाप्रतिमौजसे तदायथाऽभवत्क्षत्रियसंकुला मही}


\chapter{अध्यायः ४८}
\twolineshloka
{वासुदेव उवाच}
{}


\twolineshloka
{शृणु कौन्तेय रामस्य प्रभावो यो मया श्रुतः}
{महर्षीणां कथयतां कारणं तस्य जन्म च}


\twolineshloka
{यथा च जामदग्न्येन कोटिशः क्षत्रिया हताः}
{उद्भूता राजवंशेषु ये भूयो भारते हताः}


\twolineshloka
{जह्नोरजस्तु तनयो बलाकाश्चस्तु तत्सुतः}
{कुशिको नाम धर्मज्ञस्तस्य पुत्रो महीपते}


\twolineshloka
{अग्र्यं तपः समातिष्ठत्सहस्राक्षसमो भुवि}
{पुत्रं लभेयमजितं त्रिलोकेश्वरमित्युत}


\twolineshloka
{तमुग्रतपसं दृष्ट्वा सहस्राक्षः पुरंदरः}
{समर्थं पुत्रजनने स्वयमेवैत्य भारत}


\twolineshloka
{पुत्रत्वमगमद्राजंस्तस्य लोकेश्वरेश्वरः}
{गाधिर्नामाभवत्पुत्रः कौशिकः पाकशासनः}


\twolineshloka
{तस्य कन्याऽभवद्राजन्नाम्ना सत्यवती प्रभो}
{तां गाधिर्भृगुपुत्राय ऋचीकाय ददौ प्रभुः}


\twolineshloka
{ततस्तया हि कौन्तेय भार्गवः कुरुनदनः}
{पुत्रार्थं श्रपयामास चरुं गाधेस्तथैव च}


\twolineshloka
{आहूय चाह तां भार्यामृचीको भार्गवस्तदा}
{उपयोज्यश्चरुरयं त्वया मात्राऽप्ययं तव}


\twolineshloka
{तस्या जनिष्यते पुत्रो दीप्तिमान्क्षत्रियर्षभः}
{अजय्यः क्षत्रियैर्लोके क्षत्रियर्षभसूदनः}


\twolineshloka
{तवापि पुत्रं कल्याणि धृतिमन्तं शमात्मकम्}
{तपोन्वितं द्विजश्रेष्ठं चरुरेष विधास्यति}


\twolineshloka
{इत्येवमुक्त्वा तां भार्यां सर्चीको भृगुनन्दनः}
{तपस्यभिरतः श्रीमाञ्जगामारण्यमेव हि}


\twolineshloka
{एतस्मिन्नेव काले तु तीर्थयात्रापरो नृपः}
{गाधिः सदारः संप्राप्त ऋचीकस्याश्रमं प्रति}


\twolineshloka
{चरुद्वयं गृहीत्वा तु राजन्सत्यवती तदा}
{भर्त्रा दत्तं प्रसन्नेन मात्रे हृष्टा न्यवेदयत्}


\twolineshloka
{माता तु तस्याः कौन्तेय दुहित्रे स्वं चरुं ददौ}
{तस्याश्ररुमथाज्ञातमात्मसंस्थं चकार ह}


\twolineshloka
{अथ सत्यवती गर्भं क्षत्रियान्तकरं तदा}
{धारयामास दीप्तेन वपुषा घोरदर्शनम्}


\twolineshloka
{तामृचीकस्तदा दृष्ट्वा ध्यानयोगेन भारत}
{अब्रवीद्भृगुशार्दूलः स्वां भार्यां देवरूपिणीम्}


\threelineshloka
{मात्राऽसि व्यंसिता भद्रे चरुव्यत्यासहेतुना}
{तस्माज्जनिष्यते पुत्रः क्रूरकर्माऽत्यमर्षणः}
{`जनयिष्यति माता ते ब्रह्मभूतं तपोधनम् ॥'}


\twolineshloka
{विश्वं हि ब्रह्म सुमहच्चरौ तव समाहितम्}
{क्षत्रवीर्यं च सकलं तव मात्रे समर्पितम्}


\twolineshloka
{विपर्ययेण ते भद्रे नैतदेवं भविष्यति}
{मातुस्ते ब्राह्मणो भूयात्तव च क्षत्रियः सुतः}


\twolineshloka
{सैवमुक्ता महाभागा भर्त्रा सत्यवती तदा}
{पपात शिरसा तस्मै वेपन्ती चाब्रवीदिदम्}


\threelineshloka
{नार्होऽसि भगवन्नद्य वक्तुमेवंविधं वचः}
{ब्राह्मणापशदं पुत्रं प्राप्स्यसीति हि मां प्रभो ॥ऋचीक उवाच}
{}


\threelineshloka
{नैष संकल्पितः कामो मया भद्रे तथा त्वयि}
{उग्रकर्मा भवेत्पुत्रश्चरुव्यत्यासहेतुना ॥सत्यवत्युवाच}
{}


\threelineshloka
{इच्छँल्लोकानपि मुने सृजेथाः किं पुनः सुतम्}
{शमात्मकमृजुं पुत्रं दातुमर्हसि मे प्रभो ॥ऋचीक उवाच}
{}


\twolineshloka
{नोक्तपूर्वं मया भद्रे स्वैरेष्वप्यनृतं वचः}
{किमुताग्निं समाधाय मन्त्रवच्चरुसाधने}


\threelineshloka
{[दृष्टमेतत्पुरा भद्रे ज्ञातं च तपसा मया}
{ब्रह्मभूतं हि सकलं पितुस्तव कुलं भवेत् ॥]सत्यवत्युवाच}
{}


\threelineshloka
{काममेवं भवेत्पौत्रो मामैवं तनयः प्रभो}
{शमात्मकमृजुं पुत्रं लभेयं जपतां वर ॥ऋचीक उवाच}
{}


\threelineshloka
{पुत्रे नास्ति विशेषो मे पौत्रे च वरवर्णिनि}
{यथा त्वयोक्तं वचनं तथा भद्रे भविष्यति ॥वासुदेव उवाच}
{}


\twolineshloka
{ततः सत्यवती पुत्रं जनयामास भार्गवम्}
{तपस्यभिरतं शान्तं जमदग्निं यतव्रतम्}


\twolineshloka
{विश्वामित्रं च दायादं गाधिः कुशिकनन्दनः}
{प्राप ब्रह्मर्षिसमितं विश्वेन ब्रह्मणा युतम्}


\twolineshloka
{ऋचीको जनयामास जमदग्निं तपोनिधिम्}
{सोऽपि पुत्रं ह्यजनयज्जमदग्निः सुदारुणम्}


\twolineshloka
{सर्वविद्यान्तगं श्रेष्ठं धनुर्वेदस्य पारगम्}
{रामं क्षत्रियहन्तारं प्रदीप्तमिव पावकम्}


\twolineshloka
{[तेषयित्वा महादेवं पर्वते गन्धमादने}
{अस्त्राणि वरयामास परशुं चातितेजसम्}


\twolineshloka
{स तेनाकुण्ठधारेण ज्वलितानलवर्चसा}
{कुठारेणाप्रमेयेण लोकेष्वप्रतिमोऽभवत् ॥]}


\twolineshloka
{एतस्मिन्नेव काले तु कृतवीर्यात्मजो बली}
{अर्जुनो नाम तेजस्वी क्षत्रियो हैहयाधिपः}


\twolineshloka
{दत्तात्रेयप्रसादेन राजा बाहुसहस्रवान्}
{चक्रवर्ती महातेजा विप्राणामाश्वमेधिके}


\twolineshloka
{ददौ स पृथिवीं सर्वां सप्तद्वीपां सपर्वताम्}
{सबाह्वस्त्रबलेनाजौ जित्वा परमधर्मवित्}


\twolineshloka
{तृषितेन च कौन्तेय भिक्षितश्चित्रभानुना}
{सहस्त्रबाहुर्विक्रान्तः प्रादाद्भिक्षामथाग्नये}


\twolineshloka
{ग्रामान्पुराणि राष्ट्राणि घोषांश्चैव तु वीर्यवान्}
{जज्वाल तस्य वाणेद्धचित्रभानुर्दिधक्षया}


\twolineshloka
{स तस्य पुरुषेन्द्रस्य प्रभावेण महौजसः}
{ददाह कार्तवीर्यस्य शैलानपि धरामपि}


\twolineshloka
{स शून्यमाश्रमारण्यमापवस्य महात्मनः}
{ददाह पवनेनेद्धश्चित्रभानुः सहैहयः}


\twolineshloka
{आपवस्तं ततो रोषाच्छशापार्जुनमच्युत}
{दग्धे श्रमे महाबाहो कार्तवीर्येण वीर्यवान्}


\twolineshloka
{त्वया न वर्जितं यस्मान्ममेदं हि महद्वनम्}
{दग्धं तस्माद्रणे रामो बाहूंस्ते च्छेत्स्यतेऽर्जुन}


\threelineshloka
{अर्जुनस्तु महातेजा बली नित्यं शमात्मकः}
{ब्रह्मण्यश्च शरण्यश्च दाता शूरश्च भारत}
{नाचिन्तयत्तदा शापं तेन दत्तं महात्मना}


\twolineshloka
{तस्य पुत्राः सुबलिनः शापेनासन्पितुर्वधे}
{निमित्तमवलिप्ता वै नृशंसाश्चैव नित्यदा}


\twolineshloka
{जमदग्नेस्तु धेन्वास्ते वत्समानिन्युरच्युत}
{अज्ञातं कार्तवीर्यस्य हैहयेन्द्रस्य धीमतः}


\twolineshloka
{तन्निमित्तमभूद्युद्धं जामदग्नेर्महात्मनः}
{ततोऽर्जुनस्य बाहून्स चिच्छेद रुषितोऽनघ}


\twolineshloka
{तं भ्रमन्तं ततो वत्सं जामदग्न्यः स्वमाश्रमम्}
{प्रत्यानयत राजेन्द्र तेषामन्तः पुरात्प्रभुः}


\threelineshloka
{अर्जुनस्य सुतास्ते तु संभूयाबुद्धयस्तदा}
{गत्वाऽऽश्रममसंबुद्धा जमदग्नेर्महात्मनः}
{अपातयन्त भल्लाग्रैः शिरः कायान्नराधिप}


\twolineshloka
{समित्कुशार्थं रामस्य निर्यातस्य यशस्विनः}
{`प्रत्यक्षं राममातुश्च तथैवाश्रमवासिनाम्}


\twolineshloka
{श्रुत्वा रामस्तमर्थं च क्रुद्धः कालानलोपमः}
{धनुर्वेदेऽद्वितीयो हि दिव्यास्त्रैः समलंकृतः}


\threelineshloka
{चन्द्रबिम्बार्धसंकाशं परशुं गृह्य भार्गवः}
{'ततः पितृवधामर्षाद्रामः परममन्युमान्}
{निःक्षत्रियां प्रतिश्रुत्य महीं शस्त्रमगृह्णत}


\twolineshloka
{ततः स भृगुशार्दूलः कार्तवीर्यस्य वीर्यवान्}
{विक्रम्य निजघानाशु पुत्रान्पौत्रांश्च सर्वशः}


\twolineshloka
{स हैहयसहस्राणि हत्वा परममन्युमान्}
{महीं सागरपर्यन्तां चकार रुधिरोक्षिताम्}


\twolineshloka
{स तथा सुमहातेजाः कृत्वा निःक्षत्रियां महीम्}
{कृपया परयाऽऽविष्टो वनमेव जगाम ह}


\twolineshloka
{ततो वर्षसहस्रेषु समतीतेषु केषुचित्}
{कोपं संप्राप्तवांस्तत्र प्रकृत्या कोपनः प्रभुः}


\twolineshloka
{विश्वामित्रस्य पौत्रस्तु रैभ्यपुत्रो महातपाः}
{परावसुर्महाराज क्षिप्त्वाऽऽह जनसंसदि}


\twolineshloka
{ये ते ययातिपतने यज्ञे सन्तः समागताः}
{प्रतर्दनप्रभृतयो राम किं क्षत्रिया न ते}


\twolineshloka
{मिथ्याप्रतिज्ञो राम त्वं कत्थसे जनसंसदि}
{भयात्क्षत्रियवीराणां पर्वतं समुपाश्रितः}


\twolineshloka
{सा पुनः क्षत्रियशतैः पृथिवी सर्वतः स्तृता}
{परावसोर्वचः श्रुत्वा शस्त्रं जग्राह भार्गवः}


\twolineshloka
{ततो ये क्षत्रिया राजञ्शतशस्तेन वर्जिताः}
{ते विवृद्धा महावीर्याः पृथिवीपतयोऽभवन्}


\twolineshloka
{स पुनस्ताञ्जघानाशु बालानपि नराधिप}
{गर्भस्थैस्तु मही व्याप्ता पुनरेवाभवत्तदा}


\twolineshloka
{जातंजातं स गर्भं तु पुनरेव जघान ह}
{अरक्षंश्च सुतान्कांश्चित्तदा क्षत्रिययोषितः}


\twolineshloka
{त्रिःसप्तकृत्वः पृथिवीं कृत्वा निःक्षत्रियां प्रभुः}
{दक्षिणामश्वमेधान्ते कश्यपायाददत्ततः}


\twolineshloka
{स क्षत्रियाणां शेषार्थं करेणोद्दिश्य कश्यपः}
{स्रुक्प्रग्रहवता राजंस्ततो वाक्यमथाब्रवीत्}


\twolineshloka
{गच्छ पारं समुद्रस्य दक्षिणस्य महामुने}
{न ते मद्विषये राम वस्तव्यमिह कर्हिचित्}


\twolineshloka
{`पृथिवी दक्षिणा दत्ता वाजिमेधे मम त्वया}
{पुनरस्याः पृथिव्या हि दत्त्वा दातुमनीश्वरः ॥'}


\twolineshloka
{ततः शूर्पाकरं देशं सागरस्तस्य निर्ममे}
{संत्रासाज्जामदग्न्यस्य सोऽपरान्तमहीतलम्}


\twolineshloka
{कश्यपस्तां महाराज प्रतिगृह्य वसुंधराम्}
{कृत्वा ब्राह्मणसंस्थां वै प्रविष्टः सुमहद्वनम्}


\twolineshloka
{ततः शूद्राश्च वैश्याश्च यथा स्वैरप्रचारिणः}
{अवर्तन्त द्विजाग्र्याणां दारेषु भरतर्षभ}


\twolineshloka
{अराजके जीवलोके दुर्बला बलवत्तरैः}
{वध्यन्ते न हि वित्तेषु प्रभुत्वं कस्यचित्तदा}


\twolineshloka
{`ब्राह्मणाः क्षत्रिया वैश्याः शृद्राश्चोत्पथगामिनः}
{परस्परं समाश्रित्य घातयन्त्यपथस्थिताः}


\twolineshloka
{स्वधर्मं ब्राह्मणास्त्यक्त्वा पाषण़्डत्वं समाश्रिताः}
{चौरिकानृतमायाश्च सर्वे चैव प्रकुर्वते}


\twolineshloka
{स्वधर्मस्थान्द्विजान्हत्वा तथाऽऽश्रमनिवासिनः}
{वैश्याः सत्पथसंस्थाश्च शूद्रा ये चैव धार्मिकाः}


\twolineshloka
{तान्सर्वान्घातयन्ति स्म दुराचाराः सुनिर्भयाः}
{यज्ञाध्ययनशीलांश्च आश्रमस्थांस्तपस्विनिः}


\twolineshloka
{गोबालवृद्धनारीणां नाशं कुर्वन्ति चापरे}
{आन्वीक्षकी त्रयी वार्ता न च नीतिः प्रवर्तते}


\twolineshloka
{व्रात्यतां समनुप्राप्ता बहवो हि द्विजातयः}
{अधरोत्तरापचारेण म्लेच्छभूताश्च सर्वशः ॥ '}


\threelineshloka
{ततः कालेन पृथिवी पीड्यमाना दुरात्मभिः}
{विपर्ययेण तेनाशु प्रविवेश पसातलम्}
{अरक्ष्यमाणा विधिवत्क्षत्रियैर्धर्मरक्षिभिः}


\threelineshloka
{तां दृष्ट्वा द्रवतीं तत्र संत्रासात्स महामनाः}
{ऊरुणा धारयामास कश्यपः पृथिवीं ततः}
{निमज्जन्तीं ततो राजंस्तेनोर्वीति मही स्मृता}


\threelineshloka
{रक्षणार्थं समुद्दिश्य ययाचे पृथिवी तदा}
{प्रसाद्य कश्यपं देवी क्षत्रियान्बाहुशालिनः ॥पृथिव्युवाच}
{}


\twolineshloka
{सन्ति ब्रह्मन्मया गुप्ताः स्त्रीषु क्षत्रियपुङ्गवाः}
{हैहयानां कुले जातास्ते संरक्षन्तु मां मुने}


\twolineshloka
{अस्ति पौरवदायादो विदूरथसुतः प्रभो}
{ऋक्षैः संवर्धितो विप्र ऋक्षवत्यथ पर्वते}


\twolineshloka
{तथाऽनुकम्पमानेन यज्वनाथामितौजसा}
{पराशरेण दायादः सौदासस्याभिरक्षितः}


\twolineshloka
{सर्वकर्माणि कुरुते शूद्रवत्तस्य स द्विजः}
{सर्वकर्मेत्यभिख्यातः स मां रक्षतु पार्थिवः}


\twolineshloka
{शिबिपुत्रो महातेजा गोपतिर्नाम नामतः}
{वने संवर्धितो गोभिः सोऽभिरक्षतु मां मुने}


\twolineshloka
{प्रतर्दनस्य पुत्रस्तु वत्सो नाम महाबलः}
{वत्सैः संवर्धितो गोष्ठे स मां रक्षतु पार्थिवः}


\twolineshloka
{दधिवाहनपुत्रस्तु पौत्रो दिविरथस्य च}
{अङ्गः स गौतमेनासीद्गङ्गाकूलेऽभिरक्षितः}


\twolineshloka
{बृहद्रथो महातेजा भूरिभूतिपरिष्कृतः}
{गोलाङ्गूलैर्महाभागो गृध्रकूटेऽभिरक्षितः}


\twolineshloka
{मरुत्तस्यान्ववाये च रक्षिताः क्षत्रियात्मजाः}
{मरुत्पतिसमा वीर्ये समुद्रेणाभिरक्षिताः}


\twolineshloka
{एते क्षत्रियदायादास्तत्रतत्र परिश्रुताः}
{व्योकारहेमकारादिजातिं नित्यं समाश्रिताः}


\twolineshloka
{यदि मामभिरक्षन्ति ततः स्थास्यामि निश्चला}
{एतेषां पितरश्चैव तथैव च पितामहाः}


\twolineshloka
{मदर्थं निहता युद्धे रामेणाक्लिष्टकर्मणा}
{तेषामपचितिश्चैव मया कार्या महामुने}


\threelineshloka
{न ह्यहं कामये नित्यमतिक्रान्तेन रक्षणम्}
{वर्तमानेन वर्तेयं तत्क्षिप्रं संविधीयताम् ॥वासुदेव उवाच}
{}


\twolineshloka
{ततः पृथिव्या निर्दिष्टांस्तान्समानीय कश्यपः}
{अभ्यषिञ्चन्महीपालान्क्षत्रियान्वीर्यसंमतान्}


\threelineshloka
{तेषां पुत्राश्च पौत्राश्च येषां वंशाः प्रतिष्ठिताः}
{एवमेतत्पुरावृत्तं यन्मां पृच्छसि पाण़्डव ॥वैशंपायन उवाच}
{}


\twolineshloka
{एवं ब्रुवंस्तं च यदुप्रवीरोयुधिष्ठिरं धर्मभृतां वरिष्ठम्}
{रथेन तेनाशु ययौ यथाऽर्कोविशन्प्रभाभिर्भगवांस्त्रिलोकीम्}


\chapter{अध्यायः ४९}
\twolineshloka
{वैशंपायन उवाच}
{}


\twolineshloka
{ततो रामस्य तत्कर्म श्रुत्वा राजा युधिष्ठिरः}
{विस्मयं परमं गत्वा प्रत्युवाच जनार्दनम्}


\twolineshloka
{अहो रामस्य वार्ष्णेय शक्रस्येव महात्मनः}
{विक्रमो वसुधा येन क्रोधान्निःक्षत्रिया कृता}


\twolineshloka
{गोभिः समुद्रेण तथा गोलाङ्गूलर्क्षवानरैः}
{गुप्ता रामभयोद्विग्नाः क्षत्रियाणां कुलोद्वहाः}


\twolineshloka
{अहो धन्यो नृलोकोऽयं समाग्याश्च नरा भुवि}
{यत्र कर्मेदृशं धर्म्यं द्विजाग्र्यैः कृतमच्युत}


\twolineshloka
{कथयन्तौ कथां तात तावच्युतयुधिष्ठिरौ}
{जग्मतुर्यत्र गाङ्गेयः शरतल्पगतः प्रभुः}


\twolineshloka
{ततस्ते ददृशुर्भीष्मं शरप्रस्तरशायिनम्}
{स्वरश्मिमालासंवीतं सायंसूर्यसमप्रभम्}


\twolineshloka
{उपास्यमानं मुनिभिर्देवैरिव शतक्रतुम्}
{देशे परमधर्मिष्ठे नदीमोघवतीमनु}


\twolineshloka
{दूरादेव तमालोक्य कृष्णो राजा च धर्मजः}
{चत्वारः पाण्डवाश्चैव ते च शारद्वतादयः}


\twolineshloka
{अवम्कन्द्याथ वाहेभ्यः संयम्य प्रचलं मनः}
{एकीकृत्येन्द्रियग्राममुपतम्थुर्महामुनीन्}


\twolineshloka
{अभिवाद्य तु गोविन्दः सात्यकिस्ते च पार्थिवाः}
{व्यासादीस्तानृपीन्पश्चाद्गाङ्गेयमुपतस्थिरे}


\twolineshloka
{तपोवृद्धि ततः पृष्ट्वा गाङ्गेयं यदुपुङ्गवः}
{परिवार्य ततः सर्वे निपेदुः पुरुषर्षभाः}


\twolineshloka
{ततो निशाम्य गाङ्गेयं शाम्यमानमिवानलम्}
{किंचिद्दीनमना भीष्ममितिहोवाच केशवः}


\twolineshloka
{कच्चिज्ज्ञानानि सर्वाणि प्रसन्नानि यथापुरम्}
{कच्चिन्न व्याकुला चैव बुद्धिस्ते वदतां वर}


\twolineshloka
{शराभिघातदुःखार्तं कच्चिद्गात्रं न दूयते}
{मानसादपि दुःखाद्धि शारीरं बलवत्तरम्}


\twolineshloka
{वरदानात्पितुः कामं छन्दमृत्युरसि प्रभो}
{शन्तनोर्धर्मनित्यस्य न त्वेतदिह कारणम्}


\twolineshloka
{सुमूक्ष्मोऽपि तु देहे वै शल्यो जनयते रुजम्}
{किंपुनः शरसंघातैश्चितस्य तव पार्थिव}


\twolineshloka
{कामं नैतत्तवाख्येयं प्राणिनां प्रभवाप्ययौ}
{भवानुपदिशेच्छ्रेयो देवानामपि भारत}


\twolineshloka
{यच्च भूतं भविष्यं च भवच्च पुरुषर्षभ}
{सर्वं तज्ज्ञानवृद्धस्य तव पाणाविवाहितम्}


\twolineshloka
{संसारस्येह भूतानां धर्मस्य च फलोदयः}
{विदितस्ते महाप्राज्ञ त्वं हि धर्ममयो निधिः}


\twolineshloka
{त्व, हि राज्ये स्थितं स्फीते समग्राङ्गमरोगिणम्}
{स्त्रीसहस्रैः परिवृतं पश्यामीवोर्ध्वरेतसम्}


\twolineshloka
{ऋते शान्तनवाद्भीष्मात्रिषु लोकेषु पार्थिवम्}
{सत्यधर्मान्महावीर्याच्छूराद्धर्मैकतत्परात्}


\twolineshloka
{मृत्युमावार्य तपसा शरसंस्तरशायिनः}
{त्रिवर्गप्रभवं कंचिन्न च तातानुशुश्रुम}


\twolineshloka
{सत्ये तपसि दाने च यज्ञाधिकरणे तथा}
{धनुर्वेदे च वेदे च नित्यं चैवान्ववेक्षणे}


\twolineshloka
{अनृशंसं शुचिं दान्तं सर्वभूतहिते रतम्}
{महारथं त्वत्सदृशं न कंचिदनुशुश्रुम}


\twolineshloka
{त्वं हि देवान्सगन्धर्वानसुरान्यक्षराक्षसान्}
{शक्तस्त्वेकरथेनैव विजेतुं नात्र संशयः}


\twolineshloka
{स त्वं भीष्म महाबाहो वसूनां वासवोपमः}
{नित्यं विप्रैः समाख्यातो नवमोऽनवमो गुणैः}


\twolineshloka
{अहं च त्वाऽभिजानामि स्वयं पुरुषसत्तम}
{त्रिदशेष्वपि विख्यातस्त्वं शक्त्या पुरुषोत्तमः}


\twolineshloka
{मनुष्येषु मनुष्येन्द्र न दृष्टो न च मे श्रुतः}
{भवतो हि गुणैस्तुल्यः पृथिव्यां पुरुषः क्वचित्}


% Check verse!
त्वं हि सर्वगुणै राजन्देवानप्यतिरिच्यसे
\twolineshloka
{तपसा हि भवाञ्शक्तः स्रष्टुं लोकांश्चराचरान्}
{किंपुनश्चात्मनो लोकानुत्तमानुत्तमैर्गुणैः}


\twolineshloka
{तदस्य तप्यमानस्य ज्ञातीनां संक्षयेन वै}
{ज्येष्ठस्य पाण्डुपुत्रस्य शोकं भीष्म व्यपानुद}


\twolineshloka
{ये हि धर्माः समाख्याताश्चातुर्वर्ण्यस्य भारत}
{चातुराश्रम्यसंयुक्ताः सर्वे ते विदितास्तव}


\twolineshloka
{चातुर्विद्ये च ये प्रोक्ताश्चातुर्होत्रे च भारत}
{योगे साङ्ख्ये च नियता ये च धर्माः सनातनाः}


\twolineshloka
{चातुर्वर्ण्यस्य यश्चोक्तो धर्मो न स्म विरुध्यते}
{सेव्यमानः सवैयाख्यो गाङ्गेय विदितस्तव}


\threelineshloka
{प्रतिलोमप्रसूतानां म्लेच्छानां चैव यः स्मृतः}
{देशजातिकुलानां च जानीषे धर्मलक्षणम्}
{}


\twolineshloka
{वेदोक्तो यश्च शिष्टोक्तः सदैव विदितस्तव}
{`प्रवृत्तश्च निवृत्तश्च स चापि विदितस्तव ॥'}


\twolineshloka
{इतिहासपुराणार्थाः कार्त्स्न्येन विदितास्तव}
{धर्मशास्त्रं च सकलं नित्यं मनसि ते स्थितम्}


\twolineshloka
{ये च केचन लोकेऽस्मिन्नर्थाः संशयकारकाः}
{तेषां छेत्ता नास्ति लोके त्वदन्यः पुरुषर्षभः}


\twolineshloka
{स पाण्डवेयस्य मनःसमुत्थितंनरेन्द्र शोकं व्यपकर्ष मेधया}
{भवद्विधा ह्युत्तमबुद्धिर्विस्तराविमुह्यमानस्य जनस्य शान्तये}


\chapter{अध्यायः ५०}
\twolineshloka
{वैशंपायन उवाच}
{}


\threelineshloka
{श्रुत्वा तु वचनं भीष्मो वासुदेवस्य धीमतः}
{किंचिदुन्नाम्य वदनं प्राञ्जलिर्वाक्यमब्रवीत् ॥भीष्म उवाच}
{}


\twolineshloka
{नमस्ते भगवन्कृष्ण लोकानां प्रभवाप्यय}
{त्वं हि कर्ता हृषीकेश संहर्ता चापराजितः}


\twolineshloka
{विश्वकर्मन्नमस्तेऽस्तु विश्वात्मन्विश्वसंभव}
{अपवर्गस्थ भूतानां पञ्चानां परतः स्थित}


\twolineshloka
{नमस्ते त्रिषु लोकेषु नमस्ते परतस्त्रिषु}
{योगेश्वर नमस्तेऽस्तु त्वं हि सर्वपरायणः}


\twolineshloka
{मत्संश्रितं यदात्थ त्वं वचः पुरुषसत्तम}
{तेन पश्यामि ते दिव्यान्भावांस्त्रिषु च वर्त्मसु}


\twolineshloka
{तच्च पश्यामि तत्वेन यत्ते रूपं सनातनम्}
{सप्त मार्गा निरुद्धास्ते वायोरमिततेजसः}


\twolineshloka
{दिवं ते शिरसा व्याप्तं पभ्द्यां देवी वसुंधरा}
{दिशो भुजा रविश्चक्षुर्वीर्ये शुक्रः प्रतिष्ठितः}


\twolineshloka
{अतसीपुष्पसंकाशं पीतवाससमच्युतम्}
{वपुर्ह्यनुमिमीमस्ते मेघस्येव सविद्युतः}


\threelineshloka
{त्वां प्रपन्नाय भक्ताय गतिमिष्टां जिगीषवे}
{यच्छ्रेयः पुण्डरीकाक्ष तद्ध्यायस्व सुरोत्तम ॥वासुदेव उवाच}
{}


\twolineshloka
{यतः खलु परा भक्तिर्मयि ते पुरुषर्षभ}
{ततो मया वपुर्दिव्यं तव राजन्प्रदर्शितम्}


\twolineshloka
{न ह्यभक्ताय राजेन्द्र भक्तायानृजवे न च}
{दर्शयाम्यहमात्मानं न चादान्ताय भारत}


\twolineshloka
{भवांस्तु मम भक्तश्च नित्यं चार्जवमास्थितः}
{दमे तपसि सत्ये च दाने च निरतः शुचिः}


\twolineshloka
{अर्हस्त्वं भीष्म मां द्रष्टुं तपसा स्वेन पार्थिव}
{तव ह्युपस्थिता लोका येभ्यो नावर्तते पुनः}


\twolineshloka
{पञ्चाशतं षट् च कुरुप्रवीरशेषं दिनानां तव जीवितस्य}
{ततः शुभैः कर्मफलोदयैस्त्वंसमेष्यसे भीष्म विमुच्य देहम्}


\twolineshloka
{एते हि देवा वसवो विमानान्यास्थाय सर्वे ज्वलिताग्निकल्पाः}
{अन्तर्हितास्त्वां प्रतिपालयन्तिकाष्ठां प्रपद्यन्तमुदक्पतङ्गाम्}


\twolineshloka
{व्यावृत्तमात्रे भगवत्युदीचींसूर्ये जगत्कालवशं प्रपन्ने}
{गन्तासि लोकान्पुरुषप्रवीरनावर्तते यानुपलभ्य विद्वान्}


\twolineshloka
{अमुं च लोकं त्वयि भीष्म यातेज्ञानानि सर्वाणि पराभविष्यन्}
{अतस्तु सर्वे तव सन्निकर्षंसमागता धर्मविवेचनाय}


\twolineshloka
{तज्ज्ञातिशोकोपहतश्रुतायसत्याभिसन्धाय युधिष्ठिराय}
{प्रब्रूषि धर्मार्थसमाधियुक्तंसत्यं वचोऽस्यापनुदेच्छुचं यत्}


\chapter{अध्यायः ५१}
\twolineshloka
{वैशंपायन उवाच}
{}


\twolineshloka
{ततः कृष्णस्य तद्वाक्यं धर्मार्थसहितं हितम्}
{श्रुत्वा शान्तनवः कृष्णं प्रत्युवाच कृताञ्जलिः}


\twolineshloka
{लोकनाथ महाबाहो शिव नारायणाच्युत}
{तव वाक्यमुपश्रुत्य हर्षेणास्मि परिप्लुतः}


\twolineshloka
{किंचाहमभिधास्यामि वाक्पते तव सन्निधौ}
{यदा वाचोगतं सर्वं तव वाचि समाहितम्}


\twolineshloka
{यच्च किंचित्कृतं लोके कर्तव्यं क्रियते च यत्}
{त्वत्तस्तन्निः सृतं देव लोके बुद्धिमतो हिते}


\twolineshloka
{कथयेद्देवलोकं यो देवराजसमीपतः}
{धर्मकामार्थमोक्षाणां सोऽर्थं ब्रूयात्तवाग्रतः}


\twolineshloka
{शराभितापाद्व्यथितं मनो मे मधुसूदन}
{गात्राणि चावसीदन्ति न च बुद्धिः प्रसीदति}


\twolineshloka
{न च मे प्रतिभा काचिदस्ति किंचित्प्रभाषितुम्}
{पीड्यमानस्य गोविन्द विपानलसमैः शरैः}


\twolineshloka
{बलं मे प्रजहातीव प्राणाः सत्वरयन्ति च}
{मर्माणि परितप्यन्ति भ्रान्तचित्तस्तथा ह्यहम्}


\twolineshloka
{दौर्बल्यात्सज्जते वाङ्भे स कथं वक्तुमुत्सहे}
{साधु मे त्वं प्रसीदस्व दाशार्हकुलवर्धन}


% Check verse!
तत्क्षमस्व महाबाहो न ब्रूयां किंचिदच्युत ॥त्वत्सन्निधौ च सीदेद्धि वाचस्पतिरपि ब्रुवन्
\twolineshloka
{न दिशः संप्रजानामि नाकाशं न च मेदिनीम्}
{केवलं तव वीर्येण तिष्ठामि मधुसूदन}


\twolineshloka
{स्वयमेव भवांस्तस्माद्धर्मराजस्य यद्धितम्}
{तद्ब्रवीत्वाशु सर्वेषामागमानां त्वमागमः}


\threelineshloka
{कथं त्वयि स्थिते कृष्णे शाश्वते लोककर्तरि}
{प्रब्रूयान्मद्विधः कश्चिद्गुरौ शिष्य इव स्थिते ॥वासुदेव उवाच}
{}


\twolineshloka
{उपपन्नमिदं वाक्यं कौरवाणां धुरन्धरे}
{महावीर्ये महासत्वे स्थिरे सर्वार्थदर्शिनि}


\twolineshloka
{यच्च मामात्थ गाङ्गेय बाणघातरुजं प्रति}
{गृहाणात्र वरं भीष्म मत्प्रसादकृतं प्रभो}


\twolineshloka
{न ते ग्लानिर्न ते मूर्च्छा न तापो न च ते रुजा}
{प्रभविष्यन्ति गाङ्गेय क्षुत्पिपासे न चाप्युत}


\twolineshloka
{ज्ञानानि च समग्राणि प्रतिभास्यन्ति तेऽनघ}
{न च ते क्वचिदासत्तिर्बुद्धेः प्रादुर्भविष्यति}


\twolineshloka
{सत्वस्थं च मनो नित्यं तव भीष्म भविष्यति}
{रजस्तमोभ्यां निर्मुक्तं घनैर्मुक्त इवोडुराट्}


\twolineshloka
{यद्यच्च धर्मसंयुक्तमर्थयुक्तमथापि च}
{चिन्तयिष्यसि तत्राग्र्या बुद्धिस्तव भविष्यति}


\twolineshloka
{इमं च राजशार्दूल भूतग्रामं चतुर्विधम्}
{चक्षुर्दिव्यं समाश्रित्य द्रक्ष्यस्यमितविक्रम}


\threelineshloka
{चतुर्विधं प्रजाजालं संयुक्तो ज्ञानचक्षुषा}
{भीष्म द्रक्ष्यसि तत्त्वेन जले मीन इवामले ॥वैशंपायन उवाच}
{}


\twolineshloka
{ततस्ते व्याससहिताः सर्व एव महर्षयः}
{ऋग्यजुःसामसहितैर्वचोभिः कृष्णमार्चयन्}


\twolineshloka
{ततः सर्वार्तवं दिव्यं पुष्पवर्षं न भस्तलात्}
{पपात यत्र वार्ष्णेयः सगाङ्गेयः सपाण्डवः}


\twolineshloka
{वादित्राणि च सर्वाणि जगुश्चाप्सरसां गणाः}
{न चाहितमनिष्टं च किंचित्तत्र व्यदृश्यत}


\twolineshloka
{ववौ शिवः सुखो वायुः सर्वगन्धवहः शुचिः}
{शान्तायां दिशिशन्ताश्च प्रावदन्मृगपक्षिणः}


\twolineshloka
{ततो मुहूर्ताद्भगवान्सहस्रांशुर्दिवाकरः}
{दहन्वनमिवैकान्ते प्रतीच्यां प्रत्यदृश्यत}


\twolineshloka
{ततो महर्षयः सर्वे समुत्थाय जनार्दनम्}
{भीष्ममामन्त्रयांचक्रू राजानं च युधिष्ठिरम्}


\twolineshloka
{ततः प्रणाममकरोत्केशवः सहपाण्डवः}
{सात्यकिः स़ञ्जयश्चैव स च शारद्वतः कृपः}


\twolineshloka
{ततस्ते धर्मनिरताः सम्यक् तैरभिपूजिताः}
{श्वः समेष्याम इत्युक्त्वा यथेष्टं त्वरिता ययुः}


\twolineshloka
{तथैवामन्त्र्य गाङ्गेयं केशवः पाण्डवास्तथा}
{प्रदक्षिणमुपावृत्य रथानारुरुहुः शुभान्}


\twolineshloka
{ततो रथैः काञ्चनचित्रकूबरैर्महीधराभैः समदैश्च दन्तिभिः}
{हयैः सुपर्णैरिव चाशुगामिभिःपदातिभिश्चात्तशरासनादिभिः}


\twolineshloka
{ययौ रथानां पुरतो हि सा चमूस्तथैव पश्चादतिमात्रसारिणी}
{पुरश्च पश्चाच्च यथा महानदीतमृक्षवन्तं गिरिमेत्य नर्मदा}


\threelineshloka
{ततः पुरस्ताद्भगवान्निशाकरः}
{समुत्थितस्तामभिहर्षयंश्चमूम्}
{दिवाकरापीतरसा महौषधीःपुनः स्वकेनैव गुणेन योजयन्}


\twolineshloka
{ततः पुरं सुरपुरसंमितद्युतिप्रविश्य ते यदुवृषपाण्डवास्तदा}
{यथोचितान्भवनवरान्समाविशन्श्रमान्विता मृगपतयो गुहा इव}


\chapter{अध्यायः ५२}
\twolineshloka
{वैशंपायन उवाच}
{}


\twolineshloka
{ततः प्रविश्य भवनं प्रविश्ये मधुसूदनः}
{याममात्रावशेषायां यामिन्यां प्रत्यबुध्यत}


\twolineshloka
{स ध्यानपथमाविश्य सर्वज्ञानानि माधवः}
{अवलोक्य ततः पश्चाद्दध्यौ ब्रह्म सनातनम्}


\twolineshloka
{सूताः स्तुतिपुराणज्ञा रक्तकण्ठाः सुशिक्षिताः}
{अस्तुवन्विश्वकर्माणं वासुदेवं प्रजापतिम्}


\twolineshloka
{पठन्ति पाणिस्वनिकास्तथा गायन्ति गायनाः}
{शङ्खानथ मृदङ्गांश्च प्रवाद्यन्ति सहस्रशः}


\twolineshloka
{वीणापणववेणूनां स्वनश्चातिमनोरमः}
{सहास इव विस्तीर्णः शुश्रुवे तस्य वेश्मनि}


\twolineshloka
{ततो युधिष्ठिरस्यापि राज्ञो मङ्गलसंहिताः}
{उच्चेरुर्मधुरा वाचो गीतवादित्रबृंहिताः}


\twolineshloka
{तत उत्थाय दाशार्हः स्नातः प्राञ्जलिरच्युतः}
{जप्त्वा गुह्यं महाबाहुरग्नीनाश्रित्य तस्थिवान्}


\twolineshloka
{ततः सहस्रं विप्राणां चतुर्वेदविदां तथा}
{गवां सहस्रेणैकैकं वाचयामास माधवः}


\twolineshloka
{मङ्गलालम्भनं कृत्वा आत्मानमवलोक्य च}
{आदर्शे विमले कृष्णस्ततः सात्यकिमब्रवीत्}


\twolineshloka
{गच्छ शैनेय जानीहि गत्वा राजनिवेशनम्}
{अपि सञ्जो महातेजा भीष्मं द्रष्टुं युधिष्ठिरः}


\twolineshloka
{ततः कृष्णस्य वचनात्सात्यंकिस्त्वरितो ययौ}
{उपगम्य च राजानं युधिष्ठिरमभाषत}


\twolineshloka
{युक्तो रथवरो राजन्वासुदेवस्य धीमतः}
{समीपमापगेयस्य प्रयास्यति जनार्दनः}


\twolineshloka
{भवत्प्रतीक्षः कृष्णोऽसौ धर्मराज महाद्युते}
{यदत्रानन्तरं कृत्यं तद्भवान्कर्तुमर्हति}


\twolineshloka
{एवमुक्तः प्रत्युवाच धर्मपुत्रो युधिष्ठिरः}
{युज्यतां मे रथवरः फल्गुनाप्रतिमद्युते}


\twolineshloka
{न सैनिकैश्च यातव्यं यास्यामो वयमेव हि}
{न च पीडयितव्यो मे भीष्मो धर्मभृतां वरः}


\fourlineindentedshloka
{अतः पुरःसराश्चापि निवर्तन्तु धनञ्जय}
{अद्यप्रभृति गाङ्गेयः परं गुह्यं प्रवक्ष्यति}
{अतो नेच्छामि कौन्तेय पृथग्जनसमागमम् ॥वैशंपायन उवाच}
{}


\twolineshloka
{स तद्वाक्यमथाज्ञाय कुन्तीपुत्रो धनञ्जयः}
{युक्तं रथवरं तस्मा आचचक्षे नरर्षभः}


\twolineshloka
{ततो युधिष्ठिरो राजा यमौ भीमार्जुनावपि}
{भूतानीव समस्तानि ययुः कृष्णनिवेशनम्}


\twolineshloka
{आगच्छत्स्वथ कृष्णोऽपि पाण्डवेषु महात्मसु}
{शैनेयसहितो धीमान्रथमेवान्वपद्यत}


\twolineshloka
{रथस्थाः संविदं कृत्वा सुखां पृष्ट्वा च शर्वरीम्}
{मेघघोषै रथवरैः प्रययुस्ते नरर्षभाः}


\twolineshloka
{बलाहकं मेघपुष्पं शैब्यं सुग्रीवमेवच}
{दारुकश्चोदयामास वासुदेवस्य वाजिनः}


\twolineshloka
{ते हया वासुदेवस्य दारुकेण प्रचोदिताः}
{गां खुराग्रैस्तथा राजँल्लिखन्तः प्रययुस्तदा}


\twolineshloka
{ते ग्रसन्त इवाकाशं वेगवन्तो महाबलाः}
{क्षेत्रं धर्मस्य कृत्स्नस्य कुरुक्षेत्रमवातरन्}


\twolineshloka
{ततो ययुर्यत्र भीष्मः शरतल्पगतः प्रभुः}
{आस्ते महर्षिभिः सार्ध्रं ब्रह्मा देवगणैर्यथा}


\threelineshloka
{ततोऽवतीर्य गोविन्दो रथात्स च युधिष्ठिरः}
{भीमो गाण्डीवधन्वा च यमौ सात्यकिरेव च}
{ऋषीनभ्यर्चयामासुः करानुद्यम्य दक्षिणान्}


\twolineshloka
{स तैः परिवृतो राजा नक्षत्रैरिव चन्द्रमाः}
{अभ्याजगाम गाङ्गेयं ब्रह्माणमिव वासवः}


\twolineshloka
{शरतल्पे शयानं तमादित्यं पतितं यथा}
{स ददर्श महाबाहुं भयाच्चागतसाध्वसः}


\chapter{अध्यायः ५३}
\twolineshloka
{जनमेजय उवाच}
{}


\twolineshloka
{धर्मात्मनि महावीर्ये सत्यसन्धे जितात्मनि}
{देवव्रते महाभागे शरतल्पगतेऽच्युते}


\twolineshloka
{शयाने वीरशयने भीष्मे शन्तनुनन्दने}
{गाङ्गेये पुरुषव्याघ्रे पाण्डवैः पर्युपासिते}


\threelineshloka
{काः कथाः समवर्तन्त तस्मिन्वीरसमागमे}
{हतेषु सर्वसैन्येषु तन्मे शंस महामुने ॥वैशंपायन उवाच}
{}


\twolineshloka
{शरतल्पगते भीष्मे कौरवाणां पितामहे}
{आजग्मुर्ऋषयः सिद्धा नारदप्रमुखा नृप}


\twolineshloka
{हतशिष्टाश्च राजानो युधिष्ठिरपुरोगमाः}
{धृतराष्ट्रश्च कृष्णश्च भीमार्जुनयमास्तथा}


\twolineshloka
{तेऽभिगम्य महात्मानो भरतानां पितामहम्}
{अन्वशोचन्त गाङ्गेयमादित्यं पतितं यथा}


\twolineshloka
{मुहूर्तमिव च ध्यात्वा नारदो देवदर्शनः}
{उवाच पाण्डवान्सर्वान्हतशिष्टांश्च पार्थिवान्}


\twolineshloka
{प्राप्तकालं समाचक्षे भीष्मोऽयमनुयुज्यताम्}
{अस्तमेति हि गाङ्गेयो भानुमानिव भारत}


\twolineshloka
{अयं प्राणानुत्सिसृक्षुस्तं सर्वेऽभ्यनुपृच्छत}
{कृत्स्नान्हि विविधान्धर्मांश्चातुर्वर्ण्यस्य वेत्त्ययम्}


\threelineshloka
{एष वृद्धः पराँल्लोकान्संप्राप्नोति तनुं त्यजन्}
{तं शीघ्रमनुयुञ्जीध्वं संशयान्मनसि स्थितान् ॥वैशंपायन उवाच}
{}


\twolineshloka
{एवमुक्ते नारदेन भीष्ममीयुर्नराधिपाः}
{प्रष्टुं चाशक्रुवन्तस्ते वीक्षांचक्रुः परस्परम्}


\twolineshloka
{अथोवाच हृषीकेशं पाण्डुपुत्रो युधिष्ठिरः}
{नान्यस्त्वद्देवकीपुत्र शक्तः प्रष्टुं पितामहम्}


\twolineshloka
{प्रव्याहर यदुश्रेष्ठ त्वमग्रे मधुसूदन}
{त्वं हि नस्तात सर्वेषां सर्वधर्मविदुत्तमः}


\twolineshloka
{एवमुक्तः पाण्डवेन भगवान्केशवस्तदा}
{अभिगम्य दुराधर्षं प्रव्याहारयदच्युतः}


\twolineshloka
{कच्चित्सुखेन रजनी व्युष्टा ते राजसत्तम}
{विस्पष्टलक्षणा बुद्धिः कच्चिच्चोपस्थिता तव}


\threelineshloka
{कच्चिज्ज्ञानानि सर्वाणि प्रतिभान्ति च तेऽनघ}
{न ग्लायते च हृदयं न च ते व्याकुलं मनः ॥भीष्म उवाच}
{}


\twolineshloka
{दाहो मोहः श्रमश्चैव क्लमो ग्लानिस्तथा रुजा}
{तव प्रसादाद्वार्ष्णेय सद्यो व्यपगतानि मे}


\twolineshloka
{यच्च भूतं भविष्यच्च भवच्च परमद्युते}
{तत्सर्वमनुपश्यामि पाणौ फलमिवाहितम्}


\twolineshloka
{वेदोक्ताश्चैव ये धर्मा वेदान्ताधिगताश्च ये}
{तान्सर्वान्संप्रपश्यामि वरदानात्तवाच्युत}


\twolineshloka
{शिष्टैश्च धर्मो यः प्रोक्तः स च मे हृदि वर्तते}
{देशजातिकुलानां च धर्मज्ञोऽस्मि जनार्दन}


\twolineshloka
{चतुर्ष्वाश्रमधर्मेषु योऽर्थः स च हृदि स्थितः}
{राजधर्मांश्च सकलानवगच्छामि केशव}


\twolineshloka
{यच्च यत्र च वक्तव्यं तद्वक्ष्यामि जनार्दन}
{तव प्रसादाद्धि शुभा मनो मे बुद्धिराविशत्}


\twolineshloka
{युवेवास्मि समावृत्तस्त्वदनुध्यानबृंहितः}
{वक्तुं श्रेयः समर्थोऽस्मि त्वत्प्रसादाज्जनार्दन}


\threelineshloka
{स्वयं किमर्थं तु भवाञ्श्रेयो न प्राह पाण्डवम्}
{किं ते विवक्षितं चात्र तदाशु वद माधव ॥वासुदेव उवाच}
{}


\twolineshloka
{यशसः श्रेयसश्चैव मूल मां विद्धि कौरव}
{मत्तः सर्वेऽभिनिर्वृत्ता भावाः सदसदात्मकाः}


\twolineshloka
{शीतांशुश्चन्द्र इत्युक्ते लोके को विस्मयिष्यति}
{तथैव यशसा पूर्णे मयि को विस्मयिष्यति}


\twolineshloka
{आधेयं तु मया भूयो यशस्तव महाद्युते}
{ततो मे विपुला बुद्धिस्त्वयि भीष्म समाहिता}


\twolineshloka
{यावद्धि पृथिवीपाल पृथ्वीयं स्थास्यति ध्रुवा}
{तावत्तवाक्षया कीर्तिर्लोकाननुचरिष्यति}


\twolineshloka
{यच्च त्वं वक्ष्यसे भीष्म पाण्डवायानुपृच्छते}
{वेदप्रवाद इव ते स्थास्यते वसुधातले}


\twolineshloka
{यश्चैतेन प्रमाणेन योक्ष्यत्यात्मानमात्मना}
{स फलं सर्वपुण्यानां प्रेत्य चानुभविष्यति}


\twolineshloka
{एतस्मात्कारणाद्भीष्म मतिर्दिव्या मया हि ते}
{दत्ता यशो विप्रथयेत्कथं भूयस्तवेति ह}


\twolineshloka
{यावद्धि प्रथते लोके पुरुषस्य यशो भुवि}
{तावत्तस्याक्षया कीर्तिर्भवतीति विनिश्चिता}


\twolineshloka
{राजानो हतशिष्टास्त्वां राजन्नभित आसते}
{धर्माननुयुयुक्षन्तस्तेभ्यः प्रब्रूहि भारत}


\twolineshloka
{भवान्हि वयसा वृद्धः श्रुताचारसमन्वितः}
{कुशलो राजधर्माणां सर्वेषामपराश्च ये}


\twolineshloka
{जन्मप्रभृति ते किंचिद्वॄजिनं न ददर्श ह}
{ज्ञातारं सर्वधर्माणां त्वां विदुः सर्वपार्थिवाः}


\twolineshloka
{तेभ्यः पितेव पुत्रेभ्यो राजन्ब्रूहि परं नयम्}
{ऋषयश्चैव देवाश्च त्वया नित्यमुपासिताः}


\twolineshloka
{तस्माद्वक्तव्यमेवेदं त्वयाऽवश्यमशेषतः}
{धर्मं शुश्रूषमाणेभ्यः पृष्टेन न सता पुनः}


\twolineshloka
{वक्तव्यं विदुषा चेति धर्ममाहुर्मनीषिणः}
{अप्रतिब्रुवतः कष्टो दोषो हि भविता प्रभो}


\twolineshloka
{तस्मात्पुत्रैश्च पौत्रेश्च धर्मान्पृष्टान्सनातनान्}
{विद्वञ्जिज्ञासमानेभ्यः प्रब्रूहि भरतर्षभ}


\chapter{अध्यायः ५४}
\twolineshloka
{वैशंपायन उवाच}
{}


\twolineshloka
{अथाव्रवीन्महातेजा वाक्यं कौरवनन्दनः}
{हन्त धर्मान्प्रवक्ष्यामि दृढे वाङ्भनसी मम}


\threelineshloka
{तव प्रसादाद्गोविन्द भूतात्मा ह्यसि शाश्वतः}
{युधिष्ठिरस्तु धर्मात्मा मां धर्माननुपृच्छतु}
{एवं प्रीतो भविष्यामि धर्मान्वक्ष्यामि चाखिलान्}


\twolineshloka
{यस्मिन्राजर्षभे जाते धर्मात्मनि महात्मनि}
{अहृष्यन्नृषयः सर्वे स मां पृच्छतु पाण्डवः}


\twolineshloka
{सर्वेषां दीप्तयशसां कुरूणां धर्मचारिणाम्}
{यस्य नास्ति समः कश्चित्स मां पृच्छतु पाण्डवः}


\twolineshloka
{धृतिर्दमो ब्रह्मचर्यं क्षमा धर्मश्च नित्यदा}
{यस्मिन्नोजश्च तेजश्च स मां पृच्छतु पाण्डवः}


\twolineshloka
{संबन्धिनोऽतिथीन्भृत्यान्संश्रितांश्चैव यो भृशम्}
{संमानयति सत्कृत्य स मां पृच्छतु पाण़्डवः}


\twolineshloka
{सत्यं दानं तपः शौर्यं शान्तिर्दाक्ष्यमसंभ्रमः}
{यस्मिन्नेतानि सर्वाणि स मां पृच्छतु पाण्डवः}


\twolineshloka
{यो न कामान्न संरम्भान्न भयान्नार्थकारणात्}
{कुर्यादधर्मं धर्मात्मा स मां पृच्छतु पाण्डवः}


\twolineshloka
{सत्यनित्यः क्षमानित्यो ज्ञाननित्योऽतिथिप्रियः}
{यो ददाति सतां नित्यं स मां पृच्छतु पाण्डवः}


\threelineshloka
{इज्याध्ययननित्यश्च धर्मे च निरतः सदा}
{क्षान्तः श्रुतरहस्यश्च स मां पृच्छतु पाण्डवः ॥वासुदेव उवाच}
{}


\twolineshloka
{लज्जया परयोपेतो धर्मराजो युधिष्ठिरः}
{अभिशापभयाद्भीतो भवन्तं नोपसर्पति}


\twolineshloka
{लोकस्य कदनं कृत्वा लोकनाथो विशांपते}
{अभिशापभयाद्भीतो भवन्तं नोपसर्पति}


\threelineshloka
{पूज्यान्मान्यांश्च भक्तांश्च गुरून्संबन्धिबान्धवान्}
{अर्घार्हानिषुभिर्भित्त्वा भवन्तं नोपसर्पति ॥भीष्म उवाच}
{}


\twolineshloka
{ब्राह्मणानां यथा धर्मो दानमध्ययनं तपः}
{क्षत्रियाणां तथा कृष्ण समरे देहपातनम्}


\twolineshloka
{पितॄन्पितामहान्भ्रातॄन्गुरून्संबन्धिबान्धवान्}
{मिथ्याप्रवृत्तान्यः सख्ये निहन्याद्धर्म एव सः}


\twolineshloka
{समयत्यागिनो लुब्धान्गुरूनपि च केशव}
{निहन्ति समरे पापान्क्षत्रियो यः स धर्मवित्}


\twolineshloka
{यो लोभान्न समीक्षेत धर्मसेतुं सनातनम्}
{निहन्ति यस्तं समरे क्षत्रियो वै स धर्मवित्}


\twolineshloka
{लोहितोदां केशतृणां गजशैलां ध्वजद्रुमाम्}
{महीं करोति युद्धेषु क्षत्रियो यः स धर्मवित्}


\threelineshloka
{आहूतेन रणे नित्यं योद्धव्यं क्षत्रबन्धुना}
{धर्म्यं स्वर्ग्यं च लोक्यं च युद्धं हि मनुरब्रवीत् ॥वैशंपायन उवाच}
{}


\twolineshloka
{एवमुक्तस्तु भीष्मेण धर्मपुत्रो युधिष्ठिरः}
{विनीतवदुपागम्य तस्थै संदर्शनेऽग्रतः}


\twolineshloka
{अथास्य पादौ जग्राह भीष्मश्चापि ननन्द तम्}
{मूर्ध्निं चैनमुपाघ्राय निषीदेत्यब्रवीत्तदा}


\twolineshloka
{तमुवाचाथ गाङ्गेयो वृषभः सर्वधन्विनाम्}
{मां पृच्छ तात विस्रब्धं मा भैस्त्वं कुरुसत्तम}


\chapter{अध्यायः ५५}
\twolineshloka
{वैशंपायन उवाच}
{}


\twolineshloka
{प्रणिपत्य हृषीकेशमभिवाद्य पितामहम्}
{अनुमान्य गुरून्सर्वान्पर्यपृच्छद्युधिष्ठिरः}


\twolineshloka
{राज्ञां वै परमो धर्म इति धर्मविदो विदुः}
{महान्तमेतं भारं च मन्ये तद्ब्रूहि पार्थिव}


\twolineshloka
{राजधर्मान्विशेषेण कथयस्य पितामह}
{सर्वस्य जीवलोकस्य राजधर्मः परायणम्}


\twolineshloka
{त्रिवर्गो हि समासक्तो राजधर्मेषु कौरव}
{मोक्षधर्मश्च विस्पष्टः सकलोऽत्र समाहितः}


\twolineshloka
{यथा हि रश्मयोऽश्वस्य द्विरदस्याङ्कुशो यथा}
{नरेन्द्रधर्मो लोकस्य तथा प्रग्रहणं स्मृतम्}


\twolineshloka
{अत्र वै संप्रमूढे तु धर्मे राजर्षिसेविते}
{लोकस्य संस्था न भवेत्सर्वं च व्याकुलीभवेत्}


\twolineshloka
{उदयन्हि यथा सूर्यो नाशयत्यशुभं तमः}
{राजधर्मस्तथा लोक्यामाक्षिपत्यशुभां गतिम्}


\twolineshloka
{तदग्रे राजधर्मान्हि मदर्थे त्वं पितामह}
{प्रब्रूहि भरतश्रेष्ठ त्वं हि धर्मभूतां वरः}


\threelineshloka
{आगमश्च परस्त्वत्तः सर्वेषां नः परन्तप}
{भवन्तं हि परं बुद्धौ वासुदेवोऽभिमन्यते ॥भीष्म उवाच}
{}


\twolineshloka
{नमो धर्माय महते नमः कृष्णाय वेधसे}
{ब्राह्मणेभ्यो नमस्कृत्य धर्मान्वक्ष्यामि शाश्वतान्}


\twolineshloka
{शृणु कार्त्स्न्येन मत्तस्त्वं राजधर्मान्युधिष्ठिर}
{निरुच्यमानान्नियतो यच्चान्यदपि वाञ्छसि}


\twolineshloka
{आदावेव कुरुश्रेष्ठ राज्ञा रञ्जनमिच्छता}
{देवतानां द्विजानां च वर्तितव्यं यथाविधि}


\twolineshloka
{दैवतान्यर्चयित्वा हि ब्राह्मणांश्च कुरूद्वह}
{आनृण्यं याति धर्मस्य लोकेन च समर्च्यते}


\twolineshloka
{उत्थानेन सदा पुत्र प्रयतेथा युधिष्ठिर}
{न ह्युत्थानमृते दैवं राज्ञामर्थं प्रसाधयेत्}


\twolineshloka
{साधारणं द्वयं ह्येतद्दैवमुत्थानमेव च}
{पौरुषं हि परं मन्ये दैवं निश्चित्य मुह्यते}


\twolineshloka
{विपन्ने च समारम्भे सन्तापं मा स्म वै कृथाः}
{घटेतैवं सदाऽऽत्मानं राज्ञामेष परो नयः}


\twolineshloka
{न हि सत्यादृते किंचिद्राज्ञां वै सिद्धिकारकम्}
{सत्ये हि राजा निरतः प्रेत्य चेह च नन्दति}


\twolineshloka
{ऋषीणामपि राजेन्द्र सत्यमेव परं धनम्}
{तथा राज्ञां परं सत्यान्नान्यद्विश्वासकारणम्}


\twolineshloka
{गुणवाञ्शीलवान्दान्तो मृदुदण्डो जितेन्द्रियः}
{सुदर्शः स्थूललक्ष्यश्च न भ्रश्येत सदा श्रियः}


\threelineshloka
{आर्जवं सर्वकार्येषु श्रयेथाः कुरुनन्दन}
{पुनर्नयविचारेण त्रयीसंवरणेन च}
{`आर्जवेन समायुक्ता मोदन्ते ऋषयो दिवि ॥'}


\twolineshloka
{मृदुर्हि राजा सततं लङ्घ्यो भवति सर्वशः}
{तीक्ष्णाच्चोद्विजते लोकस्तस्मादुभयमाचरेत्}


\twolineshloka
{अदण्ड्याश्चैव ते पुत्र विप्राः स्युर्ददतां वर}
{भूतमेतत्परं लोके ब्राह्मणा नाम पाण्डव}


\twolineshloka
{मनुना चात्र राजेन्द्र गीतौ श्लोकौ महात्मना}
{धर्मेषु स्वेषु कौरव्य हृदि तौ कर्तुमर्हसि}


\twolineshloka
{अभ्द्योऽग्निर्ब्रह्मतः क्षत्रमश्मनो लोहमुत्थितम्}
{तेषां सर्वत्रगं तेजः स्वासु योनिषु शाम्यति}


\twolineshloka
{अयो हन्ति यदाऽश्मानमग्निरापो निहन्ति च}
{ब्रह्म च क्षत्रियो द्वेष्टि तदा सीदन्ति ते त्रयः}


\twolineshloka
{एवं कृत्वा महाराज नमस्या एव ते द्विजाः}
{भौमं ब्रह्म द्विजश्रेष्ठा धारयन्ति शमान्विताः}


\twolineshloka
{एवं चैव नरव्याघ्र लोकयात्राविघातकाः}
{निग्राह्या एव बाहुभ्यां ब्राह्मणास्ते नरेश्वर}


\twolineshloka
{श्लोकौ चोशनसा गीतौ पुरा तात महर्षिणा}
{तौ निवोध महाराज त्वमेकाग्रमना नृप}


\twolineshloka
{उद्यम्य शस्त्रमायान्तमपि वेदान्तगं रणे}
{निगृह्णीयात्स्वधर्मेण धर्मापेक्षी नराधिपः}


\twolineshloka
{विनश्यमानं धर्मं हि योऽभिरक्षेत्स धर्मवित्}
{न तेन धर्महा स स्यान्मन्युस्तन्मन्युमृच्छति}


\twolineshloka
{एवं चैव नरश्रेष्ठ रक्ष्या एव द्विजातयः}
{सापराधानपि हि तान्विषयान्ते समुत्सृजेत्}


\twolineshloka
{अभिशस्तमपि ह्येषां पीडयेन्न विशांपते}
{ब्रह्मघ्ने गुरुतल्पे च भ्रूणहत्ये तथैव च}


\twolineshloka
{राजद्विष्टे च विप्रस्य विषयान्ते विवासनम्}
{विधीयते न शारीरं भयमेषां कदाचन}


\twolineshloka
{दयिताश्च नरास्ते स्युर्भक्तिमन्तो द्विजेषु ये}
{न शोकः परमा तुष्टी राज्ञां भवति संचयात्}


\twolineshloka
{दुर्गेषु च महाराज षट््सु ये शास्त्रनिश्चिताः}
{सर्वदुर्गेषु मन्यन्ते नरदुर्गं सुदुर्गमम्}


\twolineshloka
{तस्मान्नित्यं दया कार्या चातुर्वर्ण्ये विपश्चिता}
{धर्मात्मा सत्यवाक्चैव राजा रञ्जयति प्रजाः}


\twolineshloka
{न च क्षान्तेन ते नित्यं भाव्यं पुरुषसत्तम}
{अधर्मो हि मृद् राजा क्षमावानिव कुञ्जरः}


\twolineshloka
{वार्हस्पत्ये च शास्त्रे च श्लोकोऽयं नियतः प्रभो}
{अस्मिन्नर्थे निगदितस्तन्मे निगदतः शृणु}


\twolineshloka
{क्षममाणं नृपं नित्यं नीचः परिभवेञ्जनः}
{हस्तियन्ता गजस्येव शिर एवारुरुक्षति}


\twolineshloka
{तस्मान्नैव मृदुर्नित्यं तीक्ष्णो वाऽपि भवेन्नृपः}
{वसन्तेऽर्क इव श्रीमान्न शीतो न च घर्मदः}


\twolineshloka
{प्रत्यक्षेणानुमानेन तथौपम्यागमैरपि}
{परीक्ष्यास्ते महाराज स्वे परे चैव नित्यशः}


\twolineshloka
{व्यसनानि च सर्वाणि त्यजेथा भूरिदक्षिण}
{नचैतानि प्रयुञ्जीथाः सङ्गं तु परिवर्जय}


\twolineshloka
{व्यसनी यस्तु लोकेऽस्मिन्परिभूतो भवत्युत}
{उद्वेजयति लोकं च योऽतिद्वेषी महीपतिः}


\twolineshloka
{भवितव्यं सदा राज्ञा गर्भिणीसहधर्मिणा}
{कारणं च महाराज शृणु येनेदमुच्यते}


\twolineshloka
{यथा हि गर्भिणी हित्वा स्वं प्रियं मनसोऽनुगम्}
{गर्भस्य हितमाधत्ते तथा राज्ञाऽप्यसंशयम्}


\twolineshloka
{वर्तितव्यं कुरुश्रेष्ठ सदा धर्मानुवर्तिना}
{स्वं प्रियं तु परित्यज्य यद्यल्लोकहितं भवेत्}


\twolineshloka
{न सन्त्याज्यं च ते धैर्यं कदाचिदपि पाण्डव}
{धीरस्य स्पष्टदण्डस्य न ह्याज्ञा प्रतिहन्यते}


\twolineshloka
{परिहासश्च भृत्यैस्ते नात्यर्थं वदतां वर}
{कर्तव्यो राजशार्दूल दोषमत्र हि मे शृणु}


\twolineshloka
{अवमन्यन्ति भर्तारं सहर्षमुपजीविनः}
{स्वे स्थाने न च तिष्ठन्ति लङ्घ्यन्ति च तद्वचः}


\twolineshloka
{प्रेष्यमाणा विकल्पन्ते गुह्यं चाप्यनुयुञ्जते}
{अयाच्यं चैव याचन्ते भोज्यान्याहारयन्ति च}


\twolineshloka
{क्रुथ्यन्ति परिदीप्तन्ति भूमिपायाधितिष्ठते}
{उत्कोचैर्वञ्चनाभिश्च कार्याणि घ्नन्ति चास्य ते}


\twolineshloka
{जर्झरं चास्य विषयं कुर्वन्ति प्रतिरूपकैः}
{स्त्रीरक्षिभिश्च सज्जन्ते तुल्यवेषा भवन्ति च}


\twolineshloka
{वान्तं निष्ठीवनं चैव कुर्वते चास्य सन्निधौ}
{निर्लज्जा राजशार्दूल व्याहरन्ति च तद्वचः}


\twolineshloka
{हयं वा दन्तिनं वाऽपि रथं वा नृपसंमतम्}
{अधिरोहन्त्यवज्ञाय सहर्षाः पार्थिवे मृदौ}


\twolineshloka
{इदं ते दुष्करं राजन्निदं ते दुर्विचेष्टितम्}
{इत्येवं सुहृदो नाम ब्रुवते परिपद्गताः}


\twolineshloka
{क्रुद्धे चास्मिन्हसन्त्येव न च हृष्यन्ति पूजिताः}
{सङ्घर्षशीलाश्च तदा भवन्त्यन्योन्यकारणात्}


\twolineshloka
{विस्रंसयन्ति मन्त्रं च विवृण्वन्ति च दुष्कृतम्}
{लीलया चैव कुर्वन्ति सावज्ञास्तस्य शासनम्}


\twolineshloka
{अलङ्काराणि भोज्यं च तथा स्नानानुलेपने}
{हेलयाना नरव्याघ्र स्वस्थास्तस्योपभुञ्जये}


\twolineshloka
{निन्दन्ते स्वानधीकारान्सन्त्यजन्ते च भारत}
{न वृत्त्या परितृष्यन्ति राजदेयं हरन्ति च}


\twolineshloka
{क्रीडितुं तेन चेच्छन्ति ससूत्रेणेव पक्षिणा}
{अस्मत्प्रणेयो राजेति लोकांश्चैव वदन्त्युत}


\twolineshloka
{एते चैवापरे चैव दोषाः प्रादुर्भवन्त्युत}
{नृपतौ मार्दवोपेते हर्षुले च युधिष्ठिर}


\chapter{अध्यायः ५६}
\twolineshloka
{भीष्म उवाच}
{}


\twolineshloka
{नित्योद्युक्तेन वै राज्ञा भवितव्यं युधिष्ठिर}
{प्रशस्यते न राजा हि नारीवोद्यमवर्जितः}


\twolineshloka
{भगवानुशना चाह श्लोकमत्र विशांपते}
{तदिहैकमना राजन्गदतस्तं निबोध मे}


\twolineshloka
{द्वाविमौ ग्रसते भूमिः सर्पो बिलशयानिव}
{राजानं चाविरोद्धारं ब्राह्मणं चाप्रवासिनम्}


\twolineshloka
{तदेतन्नरशार्दूल हृदि त्वं कर्तुमर्हसि}
{सन्धेयानभिसन्धत्स्व विरोध्यांश्च विरोधय}


\twolineshloka
{सप्ताङ्गस्य च राज्यस्य विपरीतं य आचरेत्}
{गुरुर्वा यदि वा मित्रं प्रतिहन्तव्य एव सः}


\twolineshloka
{मरुत्तेन हि राज्ञा वै गीतः श्लोकः पुरातनः}
{राज्याधिकारे राजेन्द्र बृहस्पतिमतः पुरा}


\twolineshloka
{गुरोरप्यवलिप्तस्य कार्याकार्यमजानतः}
{उत्पथं प्रतिपन्नस्य परित्यागो विधीयते}


\twolineshloka
{बाहोः पुत्रेण राज्ञा च सगरेण च धीमता}
{असमञ्जः सुतो ज्येष्ठस्त्यक्तः पौरहितैषिणा}


\twolineshloka
{असमञ्जः सरय्वां स पौराणां बालकान्नृप}
{न्यमज्जयदतः पित्रा निर्भर्त्स्य स विवासितः}


\twolineshloka
{ऋषिणोद्दालकेनापि श्वेतकेतुर्महातपाः}
{मिथ्या विप्रानुपचरन्सन्त्यक्तो दयितः सुतः}


\twolineshloka
{लोकरञ्जनमेवात्र राज्ञां धर्मः सनातनः}
{सत्यस्य रक्षणं चैव व्यवहारस्य चार्जवम्}


\twolineshloka
{न हिंस्यात्परवित्तानि देयं काले च दापयेत्}
{विक्रान्तः सत्यवाक्क्षान्तो नृपो न चलते पथः}


\twolineshloka
{गुप्तमन्त्रो जितक्रोधः शास्त्रार्थकृतनिश्चयः}
{धर्मे चार्थे च कामे च मोक्षे च सततं रतः}


\twolineshloka
{त्रय्या संवृतमन्त्रश्च राजा भवितुर्महति}
{वृजिनं च नरेन्द्राणां नान्यच्चारक्षणात्परम्}


\twolineshloka
{चातुर्वर्ण्यस्य धर्माश्च रक्षितव्या महीक्षिता}
{धर्मसंकररक्षा च राज्ञां धर्मः सनातनः}


\twolineshloka
{न विश्वसेच्च नृपतिर्न चात्यर्थं च विश्वसेत्}
{षाङ्गुण्यगुणदोषांश्च नित्यं बुद्ध्याऽवलोकयेत्}


\twolineshloka
{अच्छिद्रदर्शी नृपतिर्नित्यमेव प्रशस्यते}
{त्रिवर्गे विदितार्थश्च युक्ताचारपथश्च यः}


\twolineshloka
{कोशस्योपार्जनरतिर्यमवैश्रवणोपमः}
{वेत्ता च दशवर्गस्य स्थानवृद्धिक्षयात्मनः}


\twolineshloka
{अभृतानां भवेद्भर्ता भृतानामन्ववेक्षकः}
{नृपतिः सुभुखश्च स्यात्स्मितपूर्वाभिभाषिता}


\twolineshloka
{उपासिता ---- जिततन्द्रिरलोलुपः}
{सतां वृत्ते स्थितमतिः सतां ह्याचारदर्शनः}


\twolineshloka
{न चाददीत वित्तानि सतां हस्तात्कदाचन}
{असभ्द्यश्च समादाय सभ्द्यस्तु प्रतिपादयेत्}


\twolineshloka
{स्वयं प्रहर्ता दाता च वश्यात्मा वश्यसाधनः}
{काले दाता च भोक्ता च शुद्धाचारस्तथैव च}


\twolineshloka
{शूरान्भक्तानसंहार्यान्कुले जातानरोगिणः}
{शिष्टाञ्शिष्टाभिसंबन्धान्मानिनोऽनवमानिनः}


\threelineshloka
{विद्याविदो लोकविदः परलोकान्ववेक्षकान्}
{धर्मे च निरतान्साधूनचलानचलानिव}
{}


\twolineshloka
{सहायान्सततं कुर्याद्राजा भूतिपरिष्कृतान्}
{तैश्च तुल्यो भवेद्भोगैश्छत्रमात्राज्ञयाऽधिकः}


\twolineshloka
{प्रत्यक्षा च परोक्षा च वृत्तिश्चास्य भवेत्समा}
{एवं कुर्वन्नरेन्द्रो हि न खेदमिह विन्दति}


\twolineshloka
{सर्वाभिशङ्की नृपतिर्यश्च सर्वहरो भवेत्}
{स क्षिप्रमनृर्जुर्लुब्धः स्वजनेनैव बाध्यते}


\twolineshloka
{शुचिस्तु पृथिवीपालो लोकस्यानुग्रहे रतः}
{न पतत्यरिभिर्ग्रस्तः पतितश्चाधितिष्ठति}


\twolineshloka
{अक्रोधनो ह्यव्यसनी मृदुदण्डो जितेन्द्रियः}
{राजा भवति भूतानां विश्वास्यो हिमवानिव}


\twolineshloka
{प्राज्ञो न्यायगुणोपेतः पररन्ध्रेषु लालसः}
{सुदर्शः सर्ववर्णानां नयापनयवित्तथा}


\twolineshloka
{क्षिप्रकारी जितक्रोधः सुप्रसादो महामनाः}
{अरोगप्रकृतिर्युक्तः क्रियावानविकत्थनः}


\twolineshloka
{आरब्धान्येव कार्याणि न पर्यवसितान्यपि}
{यस्य राज्ञः प्रदृश्यन्ते स राजा राजसत्तमः}


\twolineshloka
{पुत्रा इव पितुर्गेहे विषये यस्य मानवाः}
{निर्भया विचरिष्यन्ति स राजा राजसत्तमः}


\twolineshloka
{अगूढविभवा यस्य पौरा राष्ट्रनिवासिनः}
{नयापनयवेत्तारः स राजा राजसत्तमः}


\twolineshloka
{स्वधर्मनिरता यस्य जना विषयवासिनः}
{असङ्घातरता दान्ताः पाल्यमाना यथाविधि}


\twolineshloka
{वश्या यत्ता विनीताश्च न च सङ्घर्षशीलिनः}
{विषये दानरुचयो नरा यस्य स पार्थिवः}


\twolineshloka
{न यस्य कूटं कपटं न माया न च मत्सरः}
{विषये भूमिपालस्य तस्य धर्मः सनातनः}


\twolineshloka
{यः सत्करोति ज्ञानानि श्रेयान्परहिते रतः}
{सतां वर्त्मानुगस्त्यागी स राजा स्वर्गमर्हति}


\twolineshloka
{यस्य चाराश्च मन्त्राश्च नित्यं चैव कृताकृताः}
{न ज्ञायन्ते हि रिपुभिः स राजा राज्यमर्हति}


\twolineshloka
{श्लोकद्वयं पुरा गीतं भार्गवेण महात्मना}
{आख्याते राजचरिते नृपतिं प्रति भारत}


\twolineshloka
{राजानं प्रथमं विन्देत्ततो भार्यां ततो धनम्}
{राजन्यसति लोकेऽस्मिन्कुतो भार्या कुतो धनं}


\twolineshloka
{तद्राज्ये राज्यकामानां नान्यो धर्मः सनातनः}
{ऋते रक्षां तु विस्पष्टां रक्षा लोकस्य धारिणी}


\twolineshloka
{प्राचेतसेन मनुना श्लोकौ चेमावुदाहृतौ}
{राजधर्मेषु राजेन्द्र ताविहैकमनाः शृणु}


\twolineshloka
{षडेतान्पुरुषो जह्याद्भिन्नां नावमिवार्णवे}
{अप्रवक्तारमाचार्यमनधीयानमृत्विजम्}


\twolineshloka
{अरक्षितारं राजानं भार्यां चाप्रियवादिनीम्}
{ग्रामकामं च गोपालं वनकामं च नापितम्}


\chapter{अध्यायः ५७}
\twolineshloka
{भीष्म उवाच}
{}


\twolineshloka
{एतत्ते राजधर्माणां नवनीतं युधिष्ठिर}
{बृहस्पतिर्हि भगवान्नान्यं धर्मं प्रशंसति}


\twolineshloka
{विशालाक्षश्च भगवान्काव्यश्चैव महातपाः}
{सहस्राक्षो महेन्द्रश्च तथा प्राचेतसो मनुः}


\twolineshloka
{भरद्वाजश्च भगवांस्तथा गौरशिरा मुनिः}
{राजशास्त्रप्रणेतारो ब्राह्मणा ब्रह्मवादिनः}


\twolineshloka
{रक्षामेव प्रशंसन्ति धर्मं धर्मभृतां वर}
{राज्ञां राजीवताम्राक्ष साधनं चात्र मे शृणु}


\twolineshloka
{चारश्च प्रणिधिश्चैव काले दानममत्सरः}
{युक्त्या दानं न चादानमयोगेन युधिष्ठिर}


\twolineshloka
{सतां संग्रहणं शौर्यं दाक्ष्यं सत्यं प्रजाहितम्}
{अनार्जवैरार्जवैश्च शत्रुपक्षाविवर्धनम्}


\twolineshloka
{केतनानां च जीर्णानामवेक्षा चैव सीदताम्}
{द्विविधस्य च दण्डस्य प्रयोगः कालचोदितः}


\twolineshloka
{साधूनामपरित्यागः कुलीनानां च धारणम्}
{निचयश्च निचेयानां सेवा बुद्धिमतामपि}


\twolineshloka
{बलानां हर्षणं नित्यं प्रजानामन्ववेक्षणम्}
{कार्येष्वखेदः कोशस्य तथैव च विवर्धनम्}


\twolineshloka
{पुरगुप्तिरविश्वासः पौरसंघातभेदनम्}
{अरिमध्यस्थमित्राणां यथावच्चान्ववेक्षणम्}


\twolineshloka
{उपजापश्च भृत्यानामात्मनः पुरदर्शनम्}
{अविश्वासः स्वयं चैव परस्याश्वासनं तथा}


\twolineshloka
{नीतिवर्त्मानुसारेण नित्यमुत्थानमेव च}
{रिपूणामनवज्ञानं नित्यं चानार्यवर्जनम्}


\twolineshloka
{उत्थानं हि नरेन्द्राणां बृहस्पतिरभाषत}
{राजधर्मस्य यन्मूलं श्लोकांश्चात्र निबोध मे}


\twolineshloka
{उत्थानेनामृतं लब्धमुत्थानेनासुरा हताः}
{उत्थानेन महेन्द्रेण श्रैष्ठ्यं प्राप्तं दिवीह च}


\twolineshloka
{उत्थानधीरः पुरुषो वाग्धीरानधितिष्ठति}
{उत्थानवीरान्वाग्वीरा रमयन्त उपासते}


\twolineshloka
{उत्थानहीनो राजा हि बुद्धिमानपि नित्यशः}
{प्रधर्षणीयः शत्रूणां भुजङ्ग इव निर्विषः}


\twolineshloka
{न च शत्रुरवज्ञेयो दुर्बलोऽपि बलीयसा}
{अल्पोऽपि हि दहत्यग्निर्विषमल्पं हिनस्ति च}


\twolineshloka
{एकाङ्गेनापि संभूतः शत्रुर्दुर्गमुपाश्रितः}
{सर्वं तापयते देशमपि राज्ञः समृद्धिनः}


\twolineshloka
{राज्ञो रहस्यं यद्वाक्यं जयार्थे लोकसंग्रहः}
{हृदि यच्चास्य जिह्नं स्यात्कारणार्थं च यद्भवेत्}


\twolineshloka
{यच्चास्य कार्यं वृजिनं मार्दवेनैव धार्यते}
{रञ्जनार्थं च लोकस्य धर्मिष्ठामाचरेत्क्रियाम्}


\twolineshloka
{राज्यं हि सुमहत्तत्र दुर्धार्यमकृतात्मभिः}
{न शक्यं मृदुना वोदुमाघातस्थानमुल्वणम्}


\twolineshloka
{राज्यं सर्वामिषं नित्यमार्जवेनैव धार्यते}
{तस्मान्मिश्रेण सततं वर्तितव्यं युधिष्ठिर}


\twolineshloka
{यद्यप्यस्य विपक्तिः स्याद्रक्षमाणस्य वै प्रजाः}
{सोप्यस्य विपुलो धर्म एवं वृत्ता हि भूमिपाः}


\threelineshloka
{एष ते राजधर्माणां लेशः समनुवर्णितः}
{भूयस्ते यत्र संदेहस्तद्ब्रूहि कुरुसत्तम ॥वैशंपायन उवाच}
{}


\twolineshloka
{ततो व्यासश्च भगवान्देवस्थानोऽश्म एव च}
{वासुदेवः कृपश्चैव सात्यकिः सञ्जयस्तथा}


\twolineshloka
{साधुसाध्विति संहृष्टा घुष्यमाणैरिवाननैः}
{अस्तुवंश्च नरव्याघ्रं भीष्मं धर्मभृतां वरम्}


\twolineshloka
{ततो दीनमना भीष्ममुवाच कुरुनन्दनः}
{नेत्राभ्यामश्रुपूर्णाभ्यां पादौ तस्य शनैः स्पृशन्}


\twolineshloka
{श्व इदानीं स्वसंदेहं प्रवक्ष्यामि पितामह}
{उपैति सविता ह्यस्तं रसमापीय पार्थिवम्}


\threelineshloka
{ततो द्विजातीनभिवाद्य केशवः}
{कृपश्च ते चैव युधिष्ठिरादयः}
{प्रदक्षिणीकृत्य महानदीसुतंततो रथानारुरुहुर्मुदान्विताः}


\threelineshloka
{दृषद्वतीं चाप्यवगाह्य सुव्रताः}
{कृतोदकार्थाः कृतजप्यमङ्गलाः}
{उपास्य संध्यां विधिवत्परंतपास्ततः पुरं ते विविशुर्गजाह्वयम्}


\chapter{अध्यायः ५८}
\twolineshloka
{वैशंपायन उवाच}
{}


\twolineshloka
{ततः कल्यं समुत्थाय कृतपूर्वाह्णिकक्रियाः}
{ययुस्ते नगराकारैः रथैः पाण्डवयादवाः}


\twolineshloka
{प्रतिपद्य कुरुक्षेत्रं भीष्ममासाद्य चानघम्}
{सुखां च रजनीं पृष्ट्वा गाङ्गेयं रथिनां वरम्}


\twolineshloka
{व्यासादीनभिवाद्यर्षीन्सर्वैस्तैश्चाभिनन्दिताः}
{निषेदुरभितो भीष्मं परिवार्य समन्ततः}


\threelineshloka
{ततो राजा महातेजा धर्मपुत्रो युधिष्ठिरः}
{अब्रवीत्प्राञ्जलिर्भीष्मं प्रतिपूज्य यथाविधि ॥युधिष्ठिर उवाच}
{}


\twolineshloka
{य एष राजन्राजेति शब्दश्चरति भारत}
{कथमेष समुत्पन्नस्तन्मे ब्रूहि पितामह}


\twolineshloka
{तुल्यपाणिभुजग्रीवस्तुल्यबुद्धीन्द्रियात्मकः}
{तुल्यदुःखसुखात्मा च तुल्यपृष्ठमुखोदरः}


\twolineshloka
{तुल्यशुक्रास्थिमज्जा च तुल्यमांसासृगेव च}
{नेःश्वासोच्छ्वासतुल्यश्च तुल्यप्राणशरीरवान्}


\twolineshloka
{समानजन्ममरणः समः सर्वैर्गुणैर्नृणाम्}
{विशिष्टबुद्धीञ्शूरांश्च कथमेकोऽधितिष्ठति}


\twolineshloka
{कथमेको महीं कृत्स्नां शूरवीरार्यसंकुलम्}
{रक्षत्यपि च लोकस्य प्रसादमभिवाञ्छति}


\twolineshloka
{एकस्य तु प्रसादेन कृत्स्नो लोकः प्रसीदति}
{व्याकुले चाकुलः सर्वो भवतीति विनिश्चयः}


\twolineshloka
{एतदिच्छाम्यहं श्रोतुं त्वत्तो हि भरतर्षभ}
{कृत्स्नं तन्मे यथातत्त्वं प्रब्रूहि वदतां वर}


\threelineshloka
{नैतत्कारणमत्यल्पं भविष्यति विशांपते}
{यदेकस्मिञ्जगत्सर्वं देववद्याति सन्नतिम् ॥भीष्म उवाच}
{}


\twolineshloka
{नियतस्त्वं नरव्याघ्र शृणु सर्वमशेषतः}
{यथा राज्यं समुत्पन्नमादौ कृतयुगेऽभवत्}


\twolineshloka
{नैव राज्यं न राजाऽऽसीन्न च दण्डो न दाण्डिकः}
{धर्मेणैव प्रजाः सर्वा रक्षन्ति स्म परस्परम्}


\twolineshloka
{पाल्यमानास्तथाऽन्योन्यं नरा धर्मेण भारत}
{दैन्यं परमुपाजग्मुस्ततस्तान्मोह आविशत्}


\twolineshloka
{ते मोहवशमापन्ना मनुजा मनुजर्षभ}
{प्रतिपत्तिविमोहाच्च धर्मस्तेषामनीनशत्}


\twolineshloka
{नष्टायां प्रतिपत्तौ च मोहवश्या नरास्तदा}
{लोभस्य वशमापन्नाः सर्वे भरतसत्तम}


\twolineshloka
{अप्राप्तस्याभिमर्शं तु कुर्वन्तो मनुजास्ततः}
{कामो नामापरस्तत्र प्रत्यपद्यत वै प्रभो}


\twolineshloka
{तांस्तु कामवशं प्राप्तान्रागो नामाभिसंस्पृशत्}
{रक्ताश्च नाभ्यजानन्त कार्याकार्ये युधिष्ठिर}


\twolineshloka
{अगम्यागमनं चैव वाच्यावाच्यं तथैव च}
{भक्ष्याभक्ष्यं चराजेन्द्र दोषादोषं च नात्यजन्}


\twolineshloka
{विप्लुते नरलोकेऽस्मिंस्ततो ब्रह्म ननाश ह}
{नाशाच्च ब्रह्मणो राजन्धर्मो नाशमथागमत्}


\twolineshloka
{नष्टे ब्रह्मणि धर्मे च देवास्त्रासमथागमन्}
{ते त्रस्ता नरशार्दूल ब्रह्माणं शरणं ययुः}


\twolineshloka
{प्रपद्य भगवन्तं ते देवं लोकपितामहम्}
{ऊचुः प्राञ्जलयः सर्वे दुःखवेगसमाहताः}


\twolineshloka
{भगवन्नरलोकस्थं ग्रस्तं ब्रह्म सनातनम्}
{लोभमोहादिभिर्भावैस्ततो नो भयमाविशत्}


\twolineshloka
{ब्रह्मणश्च प्रणाशेन धर्मो व्यनशदीश्चर}
{ततस्तु समतां याता मर्त्यैस्त्रिभुवनेश्वराः}


\twolineshloka
{अधोर्भिवर्षास्तु वयं भौमास्तूर्ध्वप्रवर्षिणः}
{क्रियाव्युपरमात्तेषां ततोऽगच्छाम संशयम्}


\twolineshloka
{अत्र निःश्रेयसं यन्नस्तद्ध्यायस्व पितामह}
{त्वत्प्रसादात्समुत्थोसौ प्रभावो नो भवत्वयम्}


\twolineshloka
{तानुवाच सुरान्सर्वान्स्वयंभूर्भगवांस्ततः}
{श्रेयोऽहं चिन्तयिष्यामिव्येतु वो भीः सुरोत्तमाः}


\twolineshloka
{ततोऽध्यायसहस्राणां शतं चक्रे स्वबुद्धिजम्}
{यत्र धर्मस्तथैवार्थः कामश्चैवानुवर्णितः}


\twolineshloka
{त्रिवर्ग इति विख्यातो गण एव स्वयंभुवा}
{चतुर्थो मोक्ष इत्येव पृथगर्थः पृथग्गुणः}


\twolineshloka
{मोक्षस्यास्ति त्रिवर्गोऽन्यः प्रोक्तः सत्वं रजस्तमः}
{स्थानं वृद्धिः क्षयश्चैव त्रिवर्गश्चैव दण्डजः}


\twolineshloka
{आत्मादेशश्च कालश्चाप्युपायाः कृत्यमेव च}
{सहायाः कारणं चैव षड्वर्गो नीतिजः स्मृतः}


\twolineshloka
{त्रयी चान्वीक्षिकी चैव वार्ता च भरतर्षभ}
{दण्डनीतिश्च विपुला विद्यास्तत्र निदर्शिताः}


\twolineshloka
{अमात्यलिप्सा प्रणिधी राजपुत्रस्य लक्षणम्}
{चारश्च विविधोपायः प्रणिधिश्च पृथग्विधः}


\twolineshloka
{सामभेदः प्रदानं च ततो दण्डश्च पार्थिव}
{उपेक्षा पञ्चमी चात्र कार्त्स्न्येन समुदाहृता}


\twolineshloka
{मन्त्रश्च वर्णितः कृत्स्नो मन्त्रभेदार्थ एव च}
{विभ्रमश्चैव मन्त्रस्य सिद्ध्यसिद्ध्योश्च यत्फलम्}


\twolineshloka
{संधिश्च त्रिविधाभिख्यो हीनो मध्यस्तथोत्तमः}
{भयसत्कारवित्ताख्यं कार्त्स्न्येन परिवर्णितम्}


\twolineshloka
{यात्राकालाश्च चत्वारस्त्रिवर्गस्य च विस्तरः}
{विजयो धर्मयुक्तश्च तथार्थविजयश्च ह}


\twolineshloka
{आसुरश्चैव विजयः कार्त्स्न्येन परिवर्णितः}
{लक्षणं पञ्चवर्गस्य त्रिविधं चात्र वर्णितम्}


\twolineshloka
{प्रकाशश्चाप्रकाशश्च दण़्डोऽथ परिशब्दितः}
{प्रकाशोऽष्टविधस्तत्र गुह्यश्च बहुविस्तरः}


\threelineshloka
{रथा नागा हयाश्चैव पादाताश्चैव पाण्डव}
{विष्टिर्नावश्चराश्चैव देशिका इति चाष्टमः}
{अङ्गान्येतानि कौरव्य प्रकाशानि बलस्य तु}


\twolineshloka
{जङ्गमाजङ्गमाश्चोक्ताश्चूर्णयोगा विषादयः}
{स्पर्शे चाभ्यवहार्ये चाप्युपांशुर्विविधः स्मृतः}


\twolineshloka
{`क्रीडापूर्वे रणे द्यूते विस्रम्भणसमन्वितम्}
{उक्तं कैतव्यमित्येतदुपायो नवमो बुधैः}


\twolineshloka
{उपेक्षा सर्वकार्येषु कर्मणां करणेषु च}
{अनिष्टानां समुत्थाने त्रिवर्गो नश्यते यया}


\twolineshloka
{इन्द्रजालादिका माया वाजीवनकुशीलवैः}
{सुनिमित्तैदुर्निमित्तैरुत्पातैश्च समन्वितम्}


\twolineshloka
{डम्भो लिङ्गं समाश्रित्य शत्रुवर्गे प्रयुज्यते}
{शाठ्यं निश्चेष्टता प्रोक्ता चित्तदोषप्रदूषिका ॥'}


\threelineshloka
{अरिर्मित्र उदासीन इत्येतेऽप्यनुवर्णिताः}
{कृत्स्ना मार्गगुणाश्चैव तथा भूमिगुणाश्च ह}
{आत्मरक्षणमाश्वासः स्पर्शानां चान्ववेक्षणम्}


\twolineshloka
{कल्पना विविधाश्चापि नृनागरथवाजिनाम्}
{व्यूहाश्च विविधाभिख्या विचित्रं युद्धकौशलम्}


\twolineshloka
{उत्पाताश्च निपाताश्च सुयुद्धं सुपलायितम्}
{शस्त्राणां पालनं ज्ञानं तथैव भरतर्षभ}


\twolineshloka
{बलव्यसनयुक्तं च तथैव बलहर्षणम्}
{पीडा चापदकालश्च भयकालश्च पाण्डव}


\twolineshloka
{तथाख्यातविधानं च योगः संचार एव च}
{चोरैराटविकैश्चोग्रैः परराष्ट्रस्य पीडनम्}


\twolineshloka
{अग्निदैर्गरदैश्चेव प्रतिरूपककारकैः}
{श्रेणिमुख्योपजापेन वीरुधश्छेदनेन च}


\twolineshloka
{दूषणेन च नागानामातङ्कजननेन च}
{आराधनेन भक्तस्य पत्युश्चोपग्रहेण च}


\twolineshloka
{सप्ताङ्गस्य च राज्यस्य ह्रासवृद्धिसमीक्षणम्}
{दूतसामर्थ्ययोगश्च राष्ट्रस्य च विवर्धनम्}


\twolineshloka
{अरिमध्यस्थमित्राणां सम्यक्चोक्तं प्रपञ्चनम्}
{अवमर्दः प्रतीघातस्तथैव च बलीयसाम्}


\twolineshloka
{व्यवहारः सुसूक्ष्मश्च तथा कण्टकशोधनम्}
{श्रमो व्यायामयोगश्च योगद्रव्यस्य सञ्चयः}


\twolineshloka
{अभृतानां च भरणं भृतानां चान्ववेक्षणम्}
{अन्तकाले प्रदानं च व्यसने चाप्रसङ्गिता}


\twolineshloka
{तथा राजगुणाश्चैव सेनापतिगुणाश्च ह}
{करणस्य च कर्तुश्च गुणदोषास्तथैव च}


\twolineshloka
{दुष्टेङ्गितं च विविधं वृत्तिश्चैवानुवर्तिनाम्}
{शङ्कितत्वं च सर्वस्य प्रमादस्य च वर्जनम्}


\twolineshloka
{अलब्धलिप्सा लब्धस्य तथैव च विवर्धनम्}
{प्रदानं च विवृद्धस्य पात्रेभ्यो विधिवत्तथा}


\twolineshloka
{विसर्गोऽर्थस्य धर्मार्थमर्थार्थं कामहेतुकम्}
{चतुर्थं व्यसनाघाते तथैवात्रानुवर्णितम्}


\twolineshloka
{क्रोधजानि ततोग्राणि कामजानि तथैव च}
{दशोक्तानि कुरुश्रेष्ठ व्यसनान्यत्र चैव ह}


\twolineshloka
{मृगयाक्षास्तथा पानं स्त्रियश्च भरतर्षभ}
{कामजान्याहुराचार्याः प्रोक्तानीह स्वयंभुवा}


\twolineshloka
{वाक्पारुष्यं तथोग्रत्व दण्डपारुष्यमेव च}
{आत्मनो निग्रहस्त्यागो ह्यर्थदूषणमेव च}


\twolineshloka
{यन्त्राणि विविधान्येव क्रियास्तेषां च वर्णिताः}
{अवमर्दः प्रतीघातः केतनानां च भञ्जनम्}


\twolineshloka
{चैत्यद्रुमावमर्दश्च रोधः कर्मान्तनाशनम्}
{अपस्करोऽथ वमनं तथोपास्या च वर्णिता}


\twolineshloka
{पणवानकशङ्खानां भेरीणां च युधिष्ठिर}
{उपार्जनं च द्रव्याणां परमर्म च तानि षट्}


\twolineshloka
{लब्धस्य च प्रशमनं सतां चैवाभिपूजनम्}
{विद्वद्भिरेकीभावश्च जपहोमविधिज्ञता}


\twolineshloka
{मङ्गलालम्भनं चैव शरीरस्य प्रतिक्रिया}
{आहारयोजनं चैव नित्यमास्तिक्यमेव च}


\twolineshloka
{एकेन च यथोत्थेयं सत्यत्वं मधुरा गिरः}
{उत्तमानां समाजानां क्रियाः केतनजास्तथा}


\twolineshloka
{प्रत्यक्षाश्च परोक्षाश्च सर्वाधिकरणेष्वथ}
{वृत्तेर्भरतशार्दूल नित्यं चैवान्ववेक्षणम्}


\twolineshloka
{अदण्ड्यत्वं च विप्राणां युक्त्या दण़्डनिपातनम्}
{अनुजीवि स्वजातिभ्यो गुणेभ्यश्च समुद्भवः}


\twolineshloka
{रक्षणं चैव पौराणां राष्ट्रस्य च विवर्धनम्}
{मण्डलस्था च या चिन्ता राजन्द्वादशराजिका}


\twolineshloka
{द्विसप्ततिमतिश्चैव प्रोक्ता या च स्वयंभुवा}
{देशजातिकुलानां च धर्माः समनुवर्णिताः}


\twolineshloka
{धर्मश्चार्थश्च कामश्च मोक्षश्चात्रानुवर्णिताः}
{उपायाश्चार्थलिप्सा च विविधा भूरिदक्षिणः}


\twolineshloka
{मूलकर्मक्रिया चात्र मायायोगश्च वर्णितः}
{दूषणं स्रोतसां चैव वर्णितं च स्थिराम्भसाम्}


\twolineshloka
{यैर्यैरुपायैर्लोकस्तु न चलेदार्यवर्त्मनः}
{तत्सर्वं राजशार्दूल नीतिशास्त्रेऽभिवर्णितम्}


\twolineshloka
{एतत्कृत्वा सुभं शास्त्रं ततः स भगवान्प्रभुः}
{देवानुवाच संहृष्टः सर्वाञ्छक्रपुरोगमान्}


\twolineshloka
{उपकाराय लोकस्य त्रिवर्गस्थापनाय च}
{नवनीतं सरस्वत्या बुद्धिरेषा प्रभाषिता}


\twolineshloka
{दण्डेन सहिता ह्येषा लोकरक्षणकारिका}
{निग्रहानुग्रहरता लोकाननुचरिष्यति}


\twolineshloka
{दण्डेन नीयते चेदं दण्डं नयति वा पुनः}
{दण्डनीतिरितिख्याता त्रीँल्लोकानवपत्स्यते}


\threelineshloka
{पाङ्गुण्यगुणसारैषा स्थास्यत्यग्रे महात्मसु}
{धर्मार्थकाममोक्षाश्च सकला ह्यत्र शब्दिताः ॥भीष्म उवाच}
{}


\twolineshloka
{ततस्तां भगवान्नीतिं पूर्वं जग्राह शंकरः}
{बहुरूपो विशालाक्षः शिवः स्थाणुरुमापतिः}


\twolineshloka
{अनादिनिधनो देवश्चैतन्यादिसमन्वितः}
{ज्ञानानि च वशे यस्य तारकादीन्यशेषतः}


\twolineshloka
{अणिमादिगुणोपेतमैश्वर्यं न च कृत्रिमम्}
{तुष्ट्यर्थं ब्रह्मणः पुत्रो ललाटादुत्थितः प्रभुः}


\twolineshloka
{अरुदत्सस्वनं घोरं जगतः प्रभुरव्ययः}
{जायमानः पिता पुत्रे पुत्रः पितरि चैव हि}


\twolineshloka
{बुद्धिं विश्वसृजे दत्त्वा ब्रह्माण्डं येन निर्मितम्}
{यस्मिन्हिरण्मयो हंसः शकुनिः समपद्यत}


\threelineshloka
{कर्ता सर्वस्य लोकस्य ब्रह्मा लोकपितामहः}
{स देवः सर्वभूतानां महादेवः सनातनः}
{असंख्यातसहस्राणां रुद्राणां स्थानमव्ययम्}


\twolineshloka
{युगानामायुषो ह्रासं विज्ञाय भगवाञ्शिवः}
{संचिक्षेप ततः शास्त्रं महास्त्रं ब्रह्मणा कृतम्}


\twolineshloka
{वैशालाक्षमिति प्रोक्तं तदिन्द्रः प्रत्यपद्यत}
{दशाध्यायसहस्राणि सुब्रह्मण्यो महातपाः}


\twolineshloka
{मघवानपि तच्छास्त्रं देवात्प्राप्य महेश्वरात्}
{प्रजानां हितमन्विच्छन्संचिक्षेप पुरंदरः}


\threelineshloka
{सहस्त्रैः पञ्चभिस्तात यदुक्तं बाहुदन्तकम्}
{अध्यायानां सहस्त्रैस्तु त्रिभिरेव बृहस्पतिः}
{संचिक्षेपेश्वरो बुद्ध्या बार्हस्पत्यं यदुच्यते}


\twolineshloka
{अध्यायानां सहस्रेण काव्यः संक्षेपमब्रवीत्}
{तच्छास्त्रममितप्रज्ञो योगाचार्यो महायशाः}


\twolineshloka
{एवं लोकानुरोधेन शास्त्रमेतन्महर्षिभिः}
{संक्षिप्तमायुर्विज्ञाय लोकानां ह्रासि पाण्डव}


\twolineshloka
{अथ देवाः समागम्य विष्णुमूचुः प्रजापतिम्}
{एको योऽर्हति मर्त्येभ्यः श्रैष्ठ्य वै तं समादिश}


\twolineshloka
{ततः संचिन्त्य भगवान्देवो नारायणः प्रभुः}
{तैजसं वै विरजसं सोऽसृजन्मानसं सुतम्}


\twolineshloka
{विरजास्तु महाभागः प्रभुत्वं भुवि नैच्छत}
{न्यासायैवाभवद्वुद्धिः प्रणीता तस्य पाण्डव}


\twolineshloka
{कीर्तिमांस्तस्य पुत्रोऽभूत्सोऽपि मर्त्याधिकोऽभवत्}
{कर्दमस्तस्य तु सुतः सोऽप्यतप्यन्महत्तपः}


\twolineshloka
{प्रजापतेः कर्दमस्य त्वनङ्गो नाम वीर्यवान्}
{प्रजा रक्षयिता साधुर्दण्डनीतिविशारदः}


\twolineshloka
{अनङ्गपुत्रोऽतिबलो नीतिमानभिगम्य वै}
{प्रतिपेदे महाराज्यमथेन्द्रियवशोऽभवत्}


\twolineshloka
{`प्राप्य नारीं महाभागां रूपिणीं काममोहितः}
{सौभाग्येन च संपन्नां गुणैश्चानुत्तमां सतीम्'}


\twolineshloka
{मृत्योस्तु दुहिता राजन्सुनीथा नाम नामतः}
{प्रख्याता त्रिषु लोकेषु या सा वेनमजीजनत्}


\twolineshloka
{तं प्रजासु विधर्माणं रागद्वेषवशानुगम्}
{मन्त्रपूतैः कुशैर्जघ्नुर्ऋषयो ब्रह्मवादिनः}


\twolineshloka
{ममन्थुर्दक्षिणं चोरुमृषयस्तस्य भारत}
{ततोऽस्य विकृतो जज्ञे ह्रस्वकः पुरुषोऽशुचिः}


\twolineshloka
{दग्धस्थूणाप्रतीकाशो रक्ताक्षः कृष्णमूर्धजः}
{निषीदेत्येवमूचुस्तमृषयो ब्रह्मवादिनः}


\twolineshloka
{तस्मान्निषादाः संभूताः क्रूराः शैलवनाश्रयाः}
{ये चान्ये विन्ध्यनिलया म्लेच्छाः शतसहस्रशः}


\twolineshloka
{भूयोऽस्य दक्षिणं पाणिं ममन्थुस्ते महर्षयः}
{ततः पुरुष उत्पन्नो रुपेणेन्द्र इवापरः}


\twolineshloka
{कवची बद्धनिस्त्रिंशः सशरः सशरासनः}
{वेदवेदाङ्गविच्चैव धनुर्वेदे च पारगः}


\twolineshloka
{तं दण्डनीतिः सकला श्रिता राजन्नरोत्तमम्}
{ततस्तु प्राञ्जलिर्वैन्यो महर्षीस्तानुवाच ह}


\twolineshloka
{सुसूक्ष्मा मे समुत्पन्ना बुद्धिर्धरर्मार्थदर्शिनी}
{अनया किं मया कार्यं तन्मे तत्त्वेन शंसत}


\twolineshloka
{यन्मां भवन्तो वक्ष्यन्ति कार्यमर्थसमन्वितम्}
{तदहं वः करिष्यामि नात्र कार्या विचारणा}


\twolineshloka
{तमूचुस्तत्र देवास्ते ते चैव परमर्षयः}
{नियतो यत्र धर्मो वै तमशङ्कः समाचर}


\twolineshloka
{प्रियाप्रिये परित्यज्य समः सर्वेषु जन्तुषु}
{कामं क्रोधं च लोभं च मानं चोत्सृज्य दूरतः}


\twolineshloka
{यश्च धर्मात्प्रविचलेल्लोके कश्चन मानवः}
{निग्राह्यस्ते स्वबाहुभ्यां शश्वद्धर्ममवेक्षता}


\twolineshloka
{प्रतिज्ञां चाधिरोहस्व मनसा कर्मणा गिरा}
{पालयिष्याम्यहं भौमं ब्रह्म इत्येव चासकृत्}


\twolineshloka
{यश्चात्र धर्म इत्युक्तो दण्डनीतिव्यपाश्रयः}
{तमशङ्कः करिष्यामि स्ववशो न कदाचन}


\twolineshloka
{अदण्ड्या मे द्विजाश्चेति प्रतिजानीष्व चाभिभो}
{लोकं च संकरात्कृत्स्नं त्राताऽस्मीति परंतप}


\twolineshloka
{वैन्यस्ततस्तानुवाच देवानृषिपुरोगमान्}
{ब्राह्मणा मे सहायाश्चेदेवमस्तु सुरर्षभाः}


\twolineshloka
{एवमस्त्विति वैन्यस्तु तैरुक्तो ब्रह्मवादिभिः}
{पुरोधाश्चाभवत्तस्य शुक्रो ब्रह्ममयो निधिः}


\twolineshloka
{मन्त्रिणो वालखिल्याश्च सारस्वत्यो गणस्तथा}
{महर्षिर्भगवान्गर्गस्तस्य सांवत्सरोऽभवत्}


\twolineshloka
{आत्मनाऽष्टम इत्येव श्रुतिरेषां परा नृषु}
{उत्पन्नौ बन्दिनौ चास्य तत्पूर्वौ सूतमागधौ}


\twolineshloka
{तयो प्रीतो ददौ राजा पृथुर्वैन्यः प्रतापवान्}
{अनूपदेशं सूताय मगधं मागधाय च}


\twolineshloka
{समतां वसुधायाश्च स सम्यगुदपादयत्}
{वैषम्यं हि परं भूमेरिति नः परमा श्रुतिः}


\twolineshloka
{मन्वन्तरेषु सर्वेषु विषमा जायते मही}
{उज्जहार ततो वैन्यः शिलाजालान्समन्ततः}


\threelineshloka
{धनुष्कोट्या महाराज तेन शैला विमर्दिताः}
{स विष्णुना च देवेन शक्रेण विबुधैः सह}
{ऋषिभिश्च प्रजापाल्ये ब्रह्मणा चाभिषेचितः}


\twolineshloka
{तं साक्षात्पृथिवी भेजे रत्नान्यादाय पाण़्डव}
{सागरः सरितां भर्ता हिमवांश्चाचलोत्तमः}


\twolineshloka
{शक्रश्च धनमक्षय्यं प्रादात्तस्मै युधिष्ठिर}
{रुक्मं चापि महामेरुः स्वयं कनकपर्वतः}


\twolineshloka
{यक्षराक्षसभर्ता च भगवान्नरवाहनः}
{धर्मे चार्थे च कामे च समर्थं प्रददौ धनम्}


\twolineshloka
{हया रथाश्च नागाश्च कोटिशः पुरुषास्तथा}
{प्रादुर्बभूवुर्वैन्यस्य चिन्तयानस्य पाण्डव}


\threelineshloka
{न जरा न च दुर्भिक्षं नाधयो व्याधयः कुतः}
{सरीसृपेभ्यः स्तेनेभ्यो न चान्येभ्यः कदाचन}
{भयमासीत्ततस्तस्य पृथिवी सस्यमालिनी}


\twolineshloka
{आपस्तस्तम्भिरे चास्य समुद्रमभियास्यतः}
{पर्वताश्च ददुर्मार्गं ध्वजभङ्गश्च नाभवत्}


\twolineshloka
{तेनेयं पृथिवी दुग्धा सस्यानि दश सप्त च}
{यक्षराक्षसनागानामीप्सितं यस्य यस्य यत्}


\twolineshloka
{तेन धर्मोत्तरश्चायं कृतो लोको महात्मना}
{रञ्जिताश्च प्रजाः सर्वास्तेन राजेति शब्द्यते}


\twolineshloka
{ब्राह्मणानां क्षतत्राणात्ततः क्षत्रिय उच्यते}
{प्रथिता धर्मतश्चेयं पृथिवी साधुभिः स्मृता}


\twolineshloka
{स्थापनं चाकरोद्विष्णुः स्वयमेव सनातनः}
{नातिवर्तिष्यते कश्चिद्राजंस्त्वामिति भारत}


\twolineshloka
{ततः स भगवान्विष्णुराविवेश च पार्थिवम्}
{देववन्नरदेवानां नमतीदं जगत्ततः}


\twolineshloka
{दण्डनीत्या च सततं रक्षितारं नरेश्वरम्}
{नाधर्षयेत्तथा कश्चिच्चारनिष्पन्ददर्शनात्}


\twolineshloka
{शुभं हि कर्म राजेन्द्र शुभत्वायोपकल्पते}
{आत्मना करणैश्चैव समस्येह महीक्षितः}


% Check verse!
को हेतुर्यद्वशे तिष्ठेल्लोको दैवादृते गुणात्
\twolineshloka
{विष्णोर्ललाटात्कमलं सौवर्णमभवत्तदा}
{श्रीः संभूता ततस्तस्मिन्देवी धर्मस्य पाण्डव}


\twolineshloka
{श्रियः सकाशादर्थश्च जातो धर्मस्य पाण्डव}
{अथ धर्मस्तथैवार्थः श्रीश्च राज्ये प्रतिष्ठिता}


\twolineshloka
{सुकृतस्य क्षयाच्चैव स्वर्लोकादेत्य मेदिनीम्}
{पार्थिवो जायते तात दण्डनीतिविशारदः}


\twolineshloka
{माहात्म्येन च संयुक्तो वैष्णवेन नरो भुवि}
{बुद्ध्या भवति संयुक्तो माहात्म्यं चाधिगच्छति}


\twolineshloka
{स्थापितं च ततो देवैर्न कश्चिदतिवर्तते}
{तिष्ठत्येकस्य च वशे तं चैवानुविधीयते}


\twolineshloka
{शुभं हि कर्म राजेन्द्र शुभत्वायोपकल्पते}
{तुल्यस्यैकस्य येनायं लोको वचसि तिष्ठति}


\twolineshloka
{योऽस्य वै मुखमद्राक्षीत्सोऽस्य सर्वो वशानुगः}
{सुभगं चार्थवन्तं च रूपवन्तं च पश्यति}


\twolineshloka
{महत्त्वात्तस्य दण्डस्य नीतिर्विस्पष्टलक्षणा}
{नयश्चारश्च विपुलो येन सर्वमिदं ततम्}


\twolineshloka
{आगमश्च पुराणानां महर्षीणां च संभवः}
{तीर्थवंशश्च वंशश्च क्षत्रियाणां युधिष्ठिर}


\twolineshloka
{सकलं चातुराश्रम्यं चातुर्होत्रं तथैव च}
{चातुर्वर्ण्यं तथैवात्र चातुर्विद्यं च कीर्तिम्}


\twolineshloka
{इतिहासोपवेदाश्च न्यायः कृत्स्नश्च वर्णितः}
{तपो ज्ञानमर्हिसा च सत्यं दानममत्सरः}


\twolineshloka
{वृद्धोपसेवा दानं च शौचमुत्थानमेव च}
{सर्वभूतानुकम्पा च सर्वमत्रोपवर्णितम्}


\twolineshloka
{भुवि वाचोगतं यच्च तच्च सर्वं समर्पितम्}
{तस्मिन्पैतामहे शास्त्रे पाण्डवेय न संशयः}


\twolineshloka
{ततो जगति राजेन्द्र सततं शब्दितं बुधैः}
{देवाश्च नरदेवाश्च तुल्या इति विशांपते}


\twolineshloka
{एतत्ते सर्वमाख्यातं महत्त्वं प्रति राजसु}
{कार्त्स्न्येन भरतश्रेष्ठ किमन्यदिह वर्तते}


\chapter{अध्यायः ५९}
\twolineshloka
{वैशंपायन उवाच}
{}


\twolineshloka
{ततः पुनः स गाङ्गेयमभिवाद्य पितामहम्}
{प्राञ्जलिर्नियतो भूत्वा पर्यपृच्छद्युधिष्ठिरः}


\twolineshloka
{के धर्माः सर्ववर्णानां चातुर्वर्ण्यस्य के पृथक्}
{चातुर्वर्ण्याश्रमाणां च राजधर्माश्च के मताः}


\twolineshloka
{केन वै वर्धते राष्ट्रं राजा केन विवर्धते}
{केन पौराश्च भृत्याश्च वर्धन्ते भरतर्षभ}


\twolineshloka
{कोशं दण्डं च दुर्गं च सहायान्मन्त्रिणस्तथा}
{ऋत्विक्पुरोहिताचार्यान्कीदृशान्वर्जयेन्नृपः}


\twolineshloka
{केषु विश्वसितव्यं स्याद्राज्ञा कस्यांचिदापदि}
{कुतो वाऽऽत्मा दृढं रक्ष्यस्तन्मे ब्रूहि पितामह}


\twolineshloka
{`द्वैधीभावे च भृत्यानां शपथः कीदृशो भवेत्}
{अधर्मस्य फलं यच्च शपथस्य विलङ्घने}


\threelineshloka
{सर्वमेतद्यथातत्वं व्यवहारं च तादृशम्}
{समासव्यासयोगेन कथयस्व पितामह ॥'भीष्म उवाच}
{}


\twolineshloka
{नमो धर्माय महते नमः कृष्णाय वेधसे}
{ब्राह्मणेभ्यो नमस्कृत्य धर्मान्वक्ष्यामि शाश्वतान्}


\twolineshloka
{अक्रोधः सत्यवचनं संविभागश्च सर्वशः}
{प्रजनं स्वेषु दारेषु शौचमद्रोह एव च}


\twolineshloka
{आर्जवं भृत्यभरणं त एते सार्ववर्णिकाः}
{ब्राह्मणस्य तु यो धर्मस्तं ते वक्ष्यामि केवलम्}


\twolineshloka
{दममेव महाराज धर्ममाहुः पुरातनम्}
{स्वाध्यायोऽध्यापनं चैव तत्र कर्म समाप्यते}


\twolineshloka
{तं चेद्द्विजमुपागच्छेद्वर्तमानं स्वकर्मणि}
{अकुर्वाणं विकर्माणि शान्तं प्रज्ञानतर्पितम्}


\twolineshloka
{कुर्वीतापत्यसंतानमथो दद्याद्यजेत च}
{संविभज्य च भोक्तव्यं धनं सद्भिरीतीष्यते}


\twolineshloka
{परिनिष्ठितकार्यस्तु स्वाध्यायेनैव वै द्विजः}
{कुर्यादन्यन्न वा कुर्यान्मैत्रो ब्राह्मण उच्यते}


\twolineshloka
{क्षत्रियस्यापि यो धर्मस्तं ते वक्ष्यामि भारत}
{दद्याद्राजा न याचेत यजेत न च याजयेत्}


\twolineshloka
{नाध्यापयेदधीयीत प्रजाश्च परिपालयेत्}
{नित्योद्युक्तो दस्युवदे रणे कुर्यात्पराक्रमम्}


\twolineshloka
{ये तु क्रतुभिरीजानाः श्रुतवन्तश्च भूमिपाः}
{य एवाहवजेतारस्त एषां लोकजित्तमाः}


\twolineshloka
{अविक्षतेन देहेन समराद्यो निवर्तते}
{क्षत्रियो नास्य तत्कर्म प्रशंसन्ति पुराविदः}


\twolineshloka
{एवं हि क्षत्रबन्धूनां धर्ममाहुः प्रधानतः}
{नास्य कृत्यतमं किंचिदन्यद्दस्युनिबर्हणात्}


\twolineshloka
{दानमध्ययनं यज्ञो राज्ञां क्षेमो विधीयते}
{तस्माद्राज्ञा विशेषेण योद्धव्यं धर्ममीप्सता}


\twolineshloka
{स्वेषु धर्मेष्ववस्थाप्यः प्रजाः सर्वा महीपतिः}
{धर्मेण सर्वकृत्यानि शमनिष्ठानि कारयेत्}


\twolineshloka
{परिनिष्ठितकार्यस्तु नृपतिः परिपालनात्}
{कुर्यादन्यन्न वा कुर्यादैन्द्रो राजन्य उच्यते}


\twolineshloka
{वैश्यस्यापि हि यो धर्मस्तं ते वक्ष्यामि शाश्वतम्}
{दानमध्ययनं यज्ञः शौचेन धनसंचयः}


\twolineshloka
{पितृवत्पालयेद्वैश्यो युक्तः सर्वान्पशूनिह}
{विकर्म तद्भवेदन्यत्कर्म यत्स समाचरेत्}


\twolineshloka
{रक्षया स हि तेषां वै महत्सुखमवाप्नुयात्}
{प्रजापतिर्हि वैश्याय सृष्ट्वा परिददौ पशून्}


\twolineshloka
{ब्राह्मणाय च राज्ञे च सर्वाः परिददे प्रजाः}
{तस्य वृत्तिं प्रवक्ष्यामि यच्च तस्योपजीवनम्}


\twolineshloka
{पण्णामेकां पिबेद्धेनुं शताच्च मिथुनं हरेत्}
{लये च सप्तमो भागस्तथा शृङ्गे कला खुरे}


% Check verse!
सस्यानां सर्वबीजानामेषा सांवत्सरी भृतिः
\twolineshloka
{न च वैश्यस्य कामः स्यान्न रक्षेयं पशूनिति}
{वैश्ये चेच्छति नान्येन रक्षितव्याः कथंचन}


\twolineshloka
{शूद्रस्यापि हि यो धर्मस्तं ते वक्ष्यामि भारत}
{प्रजापतिर्हि वर्णानां दासं शूद्रमकल्पयत्}


\twolineshloka
{तस्माच्छूद्रस्य वर्णानां परिचर्या विधीयते}
{तेषां शुश्रूषणाच्चैव महत्सुखमवाप्नुयात्}


\twolineshloka
{शूद्र एतान्परिचरेत्रीन्वर्णाननसूयकः}
{संचयांश्च न कुर्वीत जातु शूद्रः कथंचन}


\twolineshloka
{पापीयान्हि धनं लब्ध्वा वशे कुर्याद्गरीयसः}
{राज्ञा वा समनुज्ञातः कामं कुर्वीत धार्मिकः}


\twolineshloka
{तस्य वृत्तिं प्रवक्ष्यामि यच्च तस्योपजीवनम्}
{अवश्यं भरणीयो हि वर्णानां शूद्र उच्यते}


\twolineshloka
{छत्रं वेष्टनमौशीरमुपानद्व्यजनानि च}
{यातयामानि देयानि शूद्राय परिचारिणे}


\twolineshloka
{अधार्याणि विशीर्णानि वसनानि द्विजातिभिः}
{शूद्रायैव प्रदेयानि तस्य धर्मधनं हि तत्}


\twolineshloka
{यंच कश्चिद्द्विजातीनां शूद्रः शुश्रूषुराव्रजेत्}
{कल्प्यां तेन तु तस्याहुर्वृत्तिं धर्मविदो जनाः}


\twolineshloka
{देयः पिण्डोऽनपत्याय भर्तव्यौ वृद्धदुर्बलौ}
{शूद्रेण तु न हातव्यो भर्ता कस्यांचिदापदि}


\twolineshloka
{अतिरेकेण भर्तव्यो भर्ता द्रव्यपरिक्षये}
{न हि स्वमस्ति शूद्रस्य भर्तृहार्यधनो हि सः}


\twolineshloka
{उक्तस्त्रयाणां वर्णानां यज्ञस्त्रय्येव भारत}
{स्वाहाकारवषट््कारौ मन्त्रः शूद्रे न विद्यते}


\twolineshloka
{तस्माच्छूद्रः पाकयज्ञैयजेताव्रतवान्स्वयम्}
{पूर्णपात्रमयीमाहुः पाकयज्ञस्य दक्षिणाम्}


\twolineshloka
{शूद्रः पैजवनो नाम सहस्राणां शतं ददौ}
{ऐन्द्राग्नेन विधानेन दक्षिणामिति नः श्रुतम्}


% Check verse!
यतो हि सर्ववर्णानां यज्ञस्तस्यैव भारत
\threelineshloka
{अग्रे सर्वेषु यज्ञेषु श्रद्धायज्ञो विधीयते}
{दैवतं हि महच्छ्रद्धा पवित्रं यजतां च यत्}
{दैवतं हि परं विप्राः स्वेनस्वेन परस्परम्}


\twolineshloka
{अयजन्निह सत्रैस्ते तैस्तैः कामैः समाहिताः}
{संसृष्टा ब्राह्मणैरेव त्रिषु वर्णेषु सृष्टयः}


\twolineshloka
{देवानामपि ये देवा यद्ब्रूयुस्ते परं हितम्}
{तस्माद्वर्णैः सर्वयज्ञाः संसृज्यन्ते न काम्यया}


\threelineshloka
{ऋग्यजुः सामवित्पूज्यो नित्यं स्याद्देववद्द्विजः}
{अनृग्यजुरसामा च प्राजापत्य उपद्रवः}
{यज्ञो मनीषया तात सर्ववर्णेषु भारत}


\twolineshloka
{नास्य यज्ञकृतो देवा ईहन्ते नेतरे जनाः}
{ततः सर्वेषु वर्णेषु श्रद्धायज्ञो विधीयते}


\twolineshloka
{स्वं दैवतं ब्राह्मणः स्वेन नित्यंपरान्वर्णानयजन्नैवमासीत्}
{अधरो वितानस्त्वथ तत्र सृष्टोन ब्राह्मणस्त्रिषु वर्णेषु राजन्}


\twolineshloka
{तस्माद्वर्णा ऋजवो ज्ञातिवर्णाःसंसृज्यन्ते तस्य विकार एव}
{एकं साम यजुरेकमृगेकाविप्रश्वैको निश्चये तेषु सृष्टः}


\twolineshloka
{अत्र गाथा यज्ञगीताः कीर्तयन्ति पुराविदः}
{वैखानसानां राजेन्द्र मुनीनां यष्टुमिच्छताम्}


\twolineshloka
{उदितेऽनुदिते वाऽपि श्रद्दधानो जितेन्द्रियः}
{वह्निं जुहोति धर्मेण श्रद्धा वै कारणं महत्}


\twolineshloka
{यत्स्कन्नमस्य तत्पूर्वं यदस्कन्नं तदुत्तरम्}
{बहूनि यज्ञरूपाणि नानाकर्मफलानि च}


\twolineshloka
{तानि यः संप्रजानाति ज्ञाननिश्चयनिश्चितः}
{द्विजातिः श्रद्धयोपेतः स यष्टुं पुरुषोऽर्हति}


\twolineshloka
{स्तेनो वा यदि वा पापो यदि वा पापकृत्तमः}
{यष्टुमिच्छति यज्ञं यः साधुमेव वदन्ति तम्}


\twolineshloka
{ऋषयस्तं प्रशंसन्ति साधु चैतदसंशयम्}
{सर्वथा सर्वदा वर्णैर्यष्टव्यमिति निर्णयः}


\threelineshloka
{न हि यज्ञसमं किंचित्रिषु लोकेषु विद्यते}
{तस्माद्यष्टव्यमित्याहुः पुरुषेणानसूयता}
{श्रद्धापवित्रमाश्रित्य यथाशक्ति यथेच्छया}


\chapter{अध्यायः ६०}
\twolineshloka
{भीष्म उवाच}
{}


\twolineshloka
{आश्रमाणां महाबाहो शृणु सत्यपराक्रम}
{चतुर्णामपि नामानि कर्माणि च युधिष्ठिर}


\twolineshloka
{वानप्रस्थं भैक्षचर्यं गार्हस्थ्यं च महाश्रमम्}
{ब्रह्मचर्याश्रमं प्राहुश्चतुर्थं ब्रह्मणेरितम्}


\twolineshloka
{चूडाकरणसंस्कारं द्विजातित्वमवाप्य च}
{आधानादीनि कर्माणि प्राप्य वेदानधीत्य च}


\twolineshloka
{सदारो वाऽप्यदारो वा विनीतः संयतेन्द्रियः}
{वानप्रस्थाश्रमं गच्छेत्कृतकृत्यो गृहाश्रमात्}


\twolineshloka
{तत्रारण्यकशास्त्राणि समधीत्य स धर्मवित्}
{ऊर्ध्वरेताः प्रजा हित्वा गच्छत्यक्षरसात्मताम्}


\twolineshloka
{एतान्येव निमित्तानि मुनीनामूर्ध्वरेतसाम्}
{कर्तव्यानीह विप्रेण राजन्नादौ विपश्चिता}


\twolineshloka
{चरितब्रह्मचर्यस्य ब्राह्मणस्य विशांपते}
{भैक्षचर्यास्वधीकारः प्रशस्तो देहमोक्षणे}


\twolineshloka
{यत्रास्तमितशायी स्यान्निरग्निरनिकेतनः}
{यथोपलब्धजीवी स्यान्मुनिर्दान्तो जितेन्द्रियः}


\twolineshloka
{निराशीर्निर्नमस्कारो निर्भोगो निर्विकारवान्}
{विप्रः क्षेमाश्रमं प्राप्तो गच्छत्यक्षरसात्मताम्}


\twolineshloka
{अधीत्य वेदान्कृतसर्वकृत्यःसंतानमुत्पाद्य सुखानि भुक्त्वा}
{समाहितः प्रचरेद्दुश्चरं तंगार्हस्थ्यधर्मं मुनिधर्मजुष्टम्}


\twolineshloka
{स्वदारतुष्टस्त्वृतुकालगामीनियोगसेवी न शठो न जिह्नः}
{मिताशनो देवरतः कृतज्ञःसत्यो मृदुश्चानृशंसः क्षमावान्}


\twolineshloka
{दान्तो विधेयो हव्यकव्याप्रमत्तोह्यन्नस्य दाता सततं द्विजेभ्यः}
{अमत्सरी सर्वलिङ्गप्रदातावैताननित्यश्च गृहाश्रमी स्यात्}


\twolineshloka
{अथात्र नारायणगीतमाहुर्महर्षयस्तात महानुभावाः}
{महार्थमत्यन्ततपः प्रयुक्तंतदुच्यमानं हि मया निबोध}


\twolineshloka
{सत्यार्जवं चातिथिपूजनं चधर्मस्तथार्थश्च रतिः स्वदारैः}
{निषेवितव्यानि सुखानि लोकेह्यस्मिन्परे चैव मतं ममैतत्}


\twolineshloka
{भरणं पुत्रदाराणां वेदानां चानुपालनम्}
{सेवतामाश्रमं श्रेष्ठं वदन्ति परमर्षयः}


\twolineshloka
{एवं हि यो ब्राह्मणो यज्ञशीलोगार्हस्थ्यमध्यावसते यथावत्}
{गृहस्थवृत्तिं प्रतिगाह्य सम्यक्स्वर्गे विशुद्धं फलमश्नुते सः}


\twolineshloka
{तस्य देहं परित्यज्य इष्टकामाक्षया मताः}
{आनन्त्यायोपकल्पन्ते सर्वतोक्षिशिरोमुखाः}


\twolineshloka
{वसन्नेको जपन्नेकः सर्वान्वेदान्युधिष्ठिर}
{एकस्मिन्नेव चाचार्ये शुश्रूषुर्मलपङ्कवान्}


\threelineshloka
{ब्रह्मचारी व्रती नित्यं नित्यं दीक्षापरो वशी}
{`गुरुच्छायानुगो नित्यमधीयानः सुयन्त्रितः}
{'अविचाल्यव्रतोपेतं कृत्यं कुर्वन्वसेत्सदा}


\twolineshloka
{शुश्रूषां सततं कुर्वन्गुरोः संप्रणमेत च}
{षट््कर्मस्वनिवृत्तश्च न प्रवृत्तश्च सर्वशः}


\twolineshloka
{नाचरत्यधिकारेण सेवेत द्विषतो न च}
{एषोऽऽश्रमपदस्तात ब्रह्मचारिण इष्यते}


\chapter{अध्यायः ६१}
\twolineshloka
{युधिष्ठिर उवाच}
{}


\threelineshloka
{पुनः शिवान्महोदर्कानहिंस्रांल्लोकसंमतान्}
{ब्रूहि धर्मान्सुखोपायान्मद्विधानां सुखावहान् ॥भीष्म उवाच}
{}


\twolineshloka
{ब्राह्मणस्य तु चत्वारस्त्वाश्रमा विहिताः प्रभो}
{वर्णास्तान्नानुवर्तन्ते त्रयो भारतसत्तम}


\twolineshloka
{उक्तानि कर्माणि बहूनि राजन्स्वर्ग्याणि राजन्यपरायणानि}
{शास्त्रस्य सर्वस्य विधौ स्मृतानिक्षात्रे हि सर्वं विहितं यथावत्}


\threelineshloka
{क्षात्राणि वैश्यानि च सेवमानःशौद्राणि कर्माणि च ब्राह्मणः सत्}
{अस्मिँल्लोके निन्दितो मन्दचेताः}
{परे च लोके निरयं प्रयाति}


\twolineshloka
{या संज्ञा विहिता लोके दासे शुनि वृके पशौ}
{विकर्मणि स्थिते विप्रे तां संज्ञां कुरु पाण्डव}


\twolineshloka
{षट््कर्मसंप्रवृत्तस्य आश्रमेषु चतुर्ष्वपि}
{सर्वधर्मोपपन्नस्य तद्भूतस्य कृतात्मनः}


\twolineshloka
{ब्राह्मणस्य विशुद्धस्य तपस्यभिरतस्य च}
{निराशिषो वदान्यस्य लोका ह्यक्षरसंज्ञिताः}


\twolineshloka
{यो यस्मिन्कुरुते कर्म यादृशं येन यत्र च}
{तादृशं तादृशेनैव सगुणं प्रतिपद्यते}


\twolineshloka
{वृद्ध्या कृषिवणिक्त्वेन जीवसंजीवनेन च}
{वेत्तुमर्हसि राजेन्द्र स्वाध्यात्मगुणितेन च}


\twolineshloka
{कालसंचोदितः काले कालपर्यायनिश्चितः}
{उत्तमाधममध्यानि कर्माणि कुरुतेऽवशः}


\twolineshloka
{अन्तवन्ति प्रदानानि परं श्रेयस्कराणि च}
{स्वकर्मविहितो लोको ह्यक्षरः सर्वतोमुखः}


\chapter{अध्यायः ६२}
\twolineshloka
{भीष्म उवाच}
{}


\twolineshloka
{ज्याकर्षणं शत्रुनिवर्हणं चकृपिर्वणिज्या पशुपालनं च}
{शुश्रूषणं चापि तथाऽर्थहेतोरकार्यमेतत्परमं द्विजस्य}


\twolineshloka
{सेव्यं तु ब्रह्म षट््कर्म गृहस्थेन मनीषिणा}
{कृतकृत्यस्य चारण्ये वासो विप्रस्य शस्यते}


\twolineshloka
{राजप्रेष्यं कृषिधनं जीवनं च वणिज्यया}
{कौटिल्यं कौलटेयं च ब्राह्मणस्य विगर्हितम्}


\threelineshloka
{शूद्रो राजन्भवति ब्रह्मबन्धुर्दुश्चारित्रो यश्च धर्मादपेतः}
{वृपलीपतिः पिशुनो नर्तनश्च}
{ग्रामप्रेष्यो यश्च भवेद्विकर्मा}


\threelineshloka
{`एवंविधो ब्राह्मणः कौरवेन्द्रवृत्तापेतो यो भवेन्मन्दचेताः}
{'जपन्वेदानजपंश्चापि राजन्समः शूद्रैर्दासवच्चापि भोज्यः}
{एते सर्वे शूद्रसमा भवन्तिराजन्नेतान्वर्जयेद्देवकृत्ये}


\threelineshloka
{निर्मर्यादे वाक्छठे क्रूरवृत्तौ}
{हिंसाकामे त्यक्तवृत्तस्वधर्मो}
{हव्यं कव्यं यानि चान्यानि राजन्देयान्यदेयानि भवन्ति तस्मिन्}


\twolineshloka
{तस्माद्धर्मो विहितो ब्राह्मणस्यदमः शौचं चार्जवं चापि राजन्}
{तथा विप्रस्याश्रमाः सर्व एवपुरा राजन्ब्रह्मणा संनिसृष्टाः}


\twolineshloka
{यः स्याद्दान्तः सोमपाश्चार्यशीलःसानुक्रोशः सर्वसहो निराशीः}
{ऋजुर्मृदुरनृशंसः क्षमावान्स वै विप्रो नेतरः पापकर्मा}


\twolineshloka
{विप्रं वैश्यं राजपुत्रं च राजन्लोकाः सर्वे संश्रिता धर्मकामाः}
{तस्माद्वर्णाञ्जातिधर्मेषु सक्ताञ्जेतुं विष्णुर्नेच्छति पाण्डुपुत्र}


\twolineshloka
{लोकश्चायं सर्वलोकस्य न स्याच्चातुर्वर्ण्यं वेदवादाश्च न स्युः}
{सर्वाश्चेज्याः सर्वलोकक्रियाश्चसद्यः सर्वे चाश्रमाश्चैव न स्युः}


\twolineshloka
{यच्च त्रयाणां वर्णानामिच्छेदाश्रमसेवनम्}
{कर्तुमाश्रमदृष्टांश्च धर्मास्ताञ्शृणु पाण्डव}


\twolineshloka
{शुश्रूषोः कृतकार्यस्य कृतसंतानकर्मणः}
{अभ्यनुज्ञाप्य राजानं शूद्रस्य जगतीपते}


\twolineshloka
{अल्पान्तरगतस्यापि देशधर्मगतस्य वा}
{आश्रमा विहिताः सर्वे वर्जयित्वा निराशिषम्}


\twolineshloka
{भैक्षचर्यां नचैवाहुस्तस्य तद्धर्मवादिनः}
{तथा वैश्यस्य राजेन्द्र राजपुत्रस्य चैव हि}


\twolineshloka
{कृतकृत्यो वयोतीतो राज्ञः कृतपरिश्रमः}
{वैश्यो गच्छेदनुज्ञातो नृपेणाश्रमसंश्रयम्}


\twolineshloka
{वेदानधीत्य धर्मेण राजशास्त्राणि चानघ}
{संतानादीनि कर्माणि कृत्वा सोमं निषेव्य च}


\twolineshloka
{पालयित्वा प्रजाः सर्वा धर्मेण वदतांवर}
{राजसूयाश्वमेधादीन्मखानन्यांस्तथैव च}


\twolineshloka
{आनयित्वा यथान्यायं विप्रेभ्यो दत्तदक्षिणः}
{संग्रामे विजयं प्राप्य तथाऽल्पं यदि वा बहु}


\twolineshloka
{स्थापयित्वा प्रजापालं पुत्रं राज्ये च पाण्डव}
{अन्यगोत्रं प्रशस्तं वा क्षत्रियं क्षत्रियर्षभ}


\twolineshloka
{अर्चयित्वा पितॄञ्श्राद्धैः पितृयज्ञैर्यथाविधि}
{देवान्यज्ञैर्ऋषीन्वेदैरर्चयित्वा तु यत्नतः}


\twolineshloka
{अन्तकाले च संप्राप्ते य इच्छेदाश्रमान्तरम्}
{सोनुपूर्व्याश्रमान्राजन्गत्वा सिद्धिमवाप्नुयात्}


\twolineshloka
{राजर्षित्वेन राजेन्द्र भैक्ष्यचर्याद्यसेवया}
{अपेतगृहधर्मापि चरेज्जीवितकाम्यया}


\twolineshloka
{न चैतन्नैष्ठिकं कर्म त्रयाणां भूरिदक्षिण}
{चतुर्णां राजशार्दूल प्राहुराश्रमवासिनाम्}


\twolineshloka
{बाह्वायत्तं क्षत्रियैर्मानवानांलोकश्रेष्ठं धर्ममासेवमानैः}
{सर्वे धर्माः सोपधर्मास्त्रयाणांराज्ञो धर्मं नीतिशास्त्रे शृणोमि}


\twolineshloka
{यथा राजन्हस्तिपदे पदानिसंलीयन्ते सर्वसत्वोद्भवानि}
{एवं धर्मान्राजधर्मेषु सर्वान्सर्वावस्थान्संप्रलीनान्निबोध}


\twolineshloka
{अल्पाश्रयानल्पफलान्वदन्तिधर्मानन्यान्धर्मविदो मनुष्याः}
{महाश्रयं बहुकल्याणरूपंक्षात्रं धर्मं नेतरं प्राहुरार्याः}


\twolineshloka
{सर्वे धर्मा राजधर्मप्रधानाःसर्वे वर्णाः पाल्यमाना भवन्ति}
{सर्वस्त्यागो राजधर्मेषु राजंस्त्यागं धर्मं चाहुरग्र्यं पुराणम्}


\twolineshloka
{मज्जेत्रयी दण्डनीतौ हतायांसर्वे धर्माः प्रक्षयेयुर्विरुद्धाः}
{सर्वे धर्माश्चाश्रमाणां हताः स्युःक्षात्रे नष्टे राजधर्मे पुराणे}


\twolineshloka
{सर्वे भोगा राजधर्मेषु दृष्टाःसर्वा दीक्षा राजधर्मेषु चोक्ताः}
{सर्वा विद्या राजधर्मेषु युक्ताःसर्वे लोका राजधर्मे प्रविष्टाः}


\threelineshloka
{`सर्वे धर्मा राजधर्मेषु दृष्टाःसर्वे भोगा राजधर्मेषु राजन्}
{'सर्वे योगा राजधर्मेषु चोक्ताःसर्वे धर्मा राजधर्मे प्रविष्टाः}
{तस्माद्धर्मो राजधर्माद्विशिष्टोनान्यो लोके विद्यतेऽजातशत्रो}


\twolineshloka
{सर्वाण्येतानि कर्माणि क्षात्रे भरतसत्तम}
{भवन्ति जीवलोकाश्च क्षत्रधर्मे प्रतिष्ठिताः}


\twolineshloka
{यथा जीवाः प्राकृतैर्वध्यमानाधर्मश्रुतीनामुपपीडनाय}
{एवं धर्मा राजधर्मैर्वियुक्ताःसंचिन्वन्तो नाद्रियन्ते स्वधर्मम्}


\chapter{अध्यायः ६३}
\twolineshloka
{भीष्म उवाच}
{}


\twolineshloka
{चातुराश्रम्यधर्माश्च यतिधर्माश्च पाण्डव}
{लोकवेदोत्तराश्चैव क्षात्रधर्मे समाहिताः}


\twolineshloka
{सर्वाण्येतानि कर्माणि क्षात्रे भरतसत्तम}
{निराशिषो जीवलोकाः क्षत्रधर्मे व्यवस्थिते}


\twolineshloka
{अप्रत्यक्षं बहुफलं धर्ममाश्रमवासिनाम्}
{प्ररूपयन्ति तद्भावमागमैरेव शाश्वतम्}


\twolineshloka
{अपरे वचनैः पुण्यैर्वादिनो लोकनिश्चये}
{अनिश्चयज्ञा धर्माणामदृष्टान्ते परे रताः}


\twolineshloka
{प्रत्यक्षं फलभूयिष्ठमात्मसाक्षिकमच्छलम्}
{सर्वलोकहितं धर्मं क्षत्रियेषु प्रतिष्ठितम्}


\threelineshloka
{धर्माश्रमेऽध्यवसिनां ब्राह्मणानां युधिष्ठिर}
{यथा त्रयाणां वर्णानां संख्यातोपश्रुतिः पुरा}
{राजधर्मेष्वनुमता लोकाः सुचरितैः सह}


\threelineshloka
{उदाहृतं ते राजेन्द्र यथा विष्णुं महौजसम्}
{सर्वभूतेश्वरं देवं ब्राह्मं नारायणं पुरा}
{जग्मुः सुबहुशः शूरा राजानो दण्डनीतये}


\twolineshloka
{एकैकमात्मनः कर्म तुलयित्वाश्रमं पुरा}
{जानः पर्युपासन्त दृष्टान्तवचने स्थिताः}


\twolineshloka
{साध्या देवा वसवश्चाश्विनौ चरुद्राश्च विश्वे मरुतां गणाश्च}
{सृष्टाः पुरा ह्यादिदेवेन देवाःक्षात्रे धर्मे वर्तयन्ते च सिद्धाः}


\twolineshloka
{अत्र ते वर्तयिष्यामि धर्ममर्थविनिश्चये}
{निर्मर्यादे वर्तमाने दानवैकार्णवे पुरा}


\twolineshloka
{बभूव राजा राजेन्द्र मान्धाता नाम वीर्यवान्}
{पुरा वसुमतीपालो यज्ञं चक्रे दिदृक्षया}


\twolineshloka
{अनादिमध्यनिधनं देवं नारायणं प्रभुम्}
{स राजा राजशार्दूल मान्धाता परमेश्वरम्}


\twolineshloka
{जगाम शिरसा पादौ यज्ञे विष्णोर्महात्मनः}
{दर्शयामास तं विष्णू रूपमास्थाय वासवम्}


\fourlineindentedshloka
{स पार्थिवैर्वृतः सद्भिरर्चयामास तं प्रभुम्}
{तस्य पार्थिवसङ्घस्य तस्य चैव महात्मनः}
{संवादोऽयं महानासीद्विष्णुं प्रति महाद्युतिम् ॥इन्द्र उवाच}
{}


\twolineshloka
{किमिष्यसे धर्मभूतां वरिष्ठयं द्रष्टुकामोऽसि तमप्रमेयम्}
{अनन्तमायामितमन्त्रवीर्यंनारायणं ह्यादिदेवं पुराणम्}


% Check verse!
नासौ देवो विश्वरूपो मयाऽपिशक्यो द्रष्टुं ब्रह्मणा वाऽपि साक्षात्येऽन्ये कामास्तव राजन्हृदिस्थादास्ये चैतांस्त्वं हि मर्त्येषु राजा
\threelineshloka
{सत्ये स्थितो धर्मपरो जितेन्द्रियःशूरो दृढप्रीतिरतः सुराणाम्}
{बुद्ध्या भक्त्या चोत्तमः श्रद्धया चततस्तेऽहं दझि वरान्यथेष्टम् ॥मान्धातोवाच}
{}


\twolineshloka
{असंशयं भगवन्नादिदेवंवक्ष्यामि त्वाऽहं शिरसा संप्रसाद्य}
{त्यक्त्वा कामान्धर्मकामो ह्यरण्यमिच्छे गन्तुं सत्पथं साधुजुष्टम्}


\threelineshloka
{क्षात्राद्धर्माद्विपुलादप्रमेयाश्लोकाः प्राप्ताः स्थापितं स्वं यशश्च}
{धर्मो योऽसावादिदेवात्प्रवृत्तोलोकश्रेष्ठं तं न जानामि कर्तुम् ॥इन्द्र उवाच}
{}


\twolineshloka
{असैनिका धर्मपराश्च धर्मेपरां गतिं न नयन्ते ह्ययुक्तम्}
{क्षात्रो धर्मो ह्यादिदेवात्प्रवृत्तःपश्चादन्ये शेषभूताश्च धर्माः}


\twolineshloka
{शेषाः सृष्टा ह्यन्तवन्तो ह्यनन्ताःसप्रस्थानाः क्षात्रधर्मा विशिष्टाः}
{अस्मिन्धर्मे सर्वधर्माः प्रविष्टाःक्षात्रं धर्मं श्रेष्ठतमं वदन्ति}


\twolineshloka
{कर्मणा वै पुरा देवा ऋषयश्चामितौजसः}
{त्राताः सर्वे प्रसह्यारीन्क्षत्रधर्मेण विष्णुना}


\twolineshloka
{यदि ह्यसौ भगवन्नाहनिष्यद्रिपू सर्वानसुरानप्रमेयः}
{न च ब्रह्मा नैव लोकादिकर्तासन्तो धर्माश्चादिधर्माश्च न स्युः}


\twolineshloka
{इमामुर्वी नाजयद्विक्रमेणदेवश्रेष्ठः सासुरामादिदेवः}
{चातुर्वर्ण्यं चातुराश्रम्यधर्माःसर्वे न स्युर्ब्राह्मणानां विनाशात्}


\twolineshloka
{नष्टा धर्माः शतधा शाश्वतास्तेक्षात्रेण धर्मेण पुनः प्रवृद्धाः}
{युगेयुगे ह्यादिधर्माः प्रवृत्तालोकज्येष्ठं क्षात्रधर्मं वदन्ति}


\twolineshloka
{आत्मत्यागः सर्वभूतानुकम्पालोकज्ञानं पालनं मोक्षणं च}
{विषण्णानां मोक्षणं पीडितानांक्षात्रे धर्मे विद्यते पार्थिवानाम्}


\twolineshloka
{निर्मर्यादाः काममन्युप्रवृत्ताभीता राज्ञो नाधिगच्छन्ति पापम्}
{शिष्टाश्चान्ये सर्वधर्मोपपन्नाःसाध्वाचाराः साधुधर्मं वदन्ति}


\twolineshloka
{पुत्रवत्पाल्यमानानि धर्मलिङ्गानि पार्थिवैः}
{लोके भूतानि सर्वाणि चरन्ते नात्र संशयः}


\twolineshloka
{सर्वधर्मपरं क्षात्रं लोकश्रेष्ठं सनातनम्}
{शश्वदक्षरपर्यन्तमक्षरं सर्वतोमुखम्}


\chapter{अध्यायः ६४}
\twolineshloka
{इन्द्र उवाच}
{}


\threelineshloka
{एवंवीर्यः सर्वधर्मोपपन्नःक्षात्रः श्रेष्ठः सर्वधर्मेषु धर्मः}
{पाल्यो युष्माभिर्लोकपालैरुदारैर्विपर्यये स्यादभवः प्रजानाम्}
{}


\twolineshloka
{भूसस्कारं धर्मसंस्कारयोग्यंदीक्षाचर्यां पालनं च प्रजानाम्}
{विद्याद्राज्ञः सर्वभूतानुकम्पादेहत्यागं चाहवे धर्ममग्र्यम्}


\twolineshloka
{त्यागं श्रेष्ठं मुनयो वै वदन्तिसर्वश्रेष्ठं यच्छरीरं त्यजन्ति}
{नित्यं व्यक्तं राजधर्मेषु सर्वेप्रत्यक्षं ते भूमिपाला यथैते}


\threelineshloka
{बहुश्रुत्या गुरुशुश्रूषया वापरस्पराः संहननाद्वदन्ति}
{नित्यं धर्मं क्षत्रियो ब्रह्मचारी}
{चरेदेको ह्याश्रमं धर्मकामः}


\twolineshloka
{सामान्यार्थे व्यवहारे प्रवृत्तेप्रियाप्रिये वर्जयन्नैव यत्नात्}
{चातुर्वर्ण्यं स्थापनात्पालनाच्चतैस्तैर्योगैर्नियमैरौषधैश्च}


\twolineshloka
{सर्वोद्योगैराश्रमं धर्ममाहुःक्षात्रं श्रेष्ठं सर्वधर्मोपपन्नम्}
{वंस्वं धर्मं येन चरन्ति वर्णास्तांस्तान्धर्मानन्यथार्थान्वदन्ति}


\twolineshloka
{नेर्मर्यादान्नित्यमर्थे निविष्टानाहुस्तान्वै पशुभूतान्मनुष्यान्}
{यथा नीतिं गमयत्यर्थयोगाच्छ्रेयस्तस्मादाश्रमात्क्षत्रधर्मः}


\twolineshloka
{त्रैविद्यानां या गतिर्ब्राह्मणानांये चैवोक्ताः स्वाश्रमा ब्राह्मणानाम्}
{एतत्कर्म क्षत्रियस्याहुरग्र्यमन्यत्कुर्वञ्छूद्रवच्छस्त्रवध्यः}


\twolineshloka
{चातुराश्रम्यधर्माश्च वेदवादाश्च पार्थिव}
{ब्राह्मणेनानुगन्तव्या नान्यो विद्यात्कदाचन}


\twolineshloka
{अन्यथा वर्तमानस्य न सा वृत्तिः प्रकल्प्यते}
{कर्मणा वर्धते धर्मो यथा धर्मस्तथैव सः}


\twolineshloka
{यो विकर्मस्थितो विप्रो न स सन्मानमर्हति}
{कर्म स्वमप्रयुञ्जानमविश्वास्यं हि तं विदुः}


\threelineshloka
{एते वर्णाः सर्वधर्मैश्च हीनाउत्क्रष्टव्याः क्षत्रियैरेव धर्माः}
{तस्माच्छ्रेष्ठा राजधर्मा न चान्येवीर्यश्रेष्ठा राजधर्मा मता मे ॥मान्धातोवाच}
{}


\twolineshloka
{यवनाः किराता गान्धाराश्चीनाः शबरबर्बराः}
{शकास्तुषाराः कङ्काश्च पल्लवाश्चान्ध्रमद्रकाः}


\twolineshloka
{उष्ट्राः पुलिन्दा आरट्टाः काचा म्लेच्छाश्च सर्वशः}
{ब्रह्मक्षत्रप्रसूताश्च वैश्याः शूद्राश्च मानवाः}


\twolineshloka
{कथं धर्मांश्चरिष्यन्ति सर्वे विषयवासिनः}
{मद्विधैश्च कथं स्थाप्याः सर्वे वै दस्युजीविनः}


\threelineshloka
{एतदिच्छाम्यहं श्रोतुं मघवंस्तद्ब्रवीहि मे}
{त्वं बन्धुभूतो ह्यस्माकं क्षत्रियाणां सुरेश्वर ॥इन्द्र उवाच}
{}


\twolineshloka
{मातापित्रोर्हि शुश्रूषा कर्तव्या सर्वदस्युभिः}
{आचार्यगुरुशुश्रूषा तथैवाश्रमवासिनाम्}


\twolineshloka
{भूमिपानां च शुश्रूषा कर्तव्या सर्वदस्युभिः}
{देशधर्मक्रियाश्चैव तेषां धर्मो विधीयते}


\twolineshloka
{पितृयज्ञास्तथा कूपाः प्रपाश्च शयनानि च}
{दानानि च यथाकालं दातव्यानि द्विजातिषु}


\twolineshloka
{अहिंसा सत्यमक्रोधो वृत्तिदायानुपालनम्}
{भरणं पुत्रदाराणां शौचमद्रोह एव च}


\twolineshloka
{दक्षिणा सर्वयज्ञानां दातव्या धर्ममिच्छता}
{पाकयज्ञा महार्थाश्च दातव्याः सर्वदस्युभिः}


\threelineshloka
{एतान्येवंप्रकाराणि विहितानि पुराऽनघ}
{सर्वलोकस्य कर्माणि कर्तव्यानीह पार्थिव ॥मान्धातोवाच}
{}


\threelineshloka
{दृश्यन्ते मानुषे लोके सर्ववर्णेषु दस्यवः}
{लिङ्गान्तरे वर्तमाना आश्रमेषु तथैव च ॥इन्द्र उवाच}
{}


\twolineshloka
{विनष्टायां दण्डनीत्यां राजधर्मे विनाकृते}
{संप्रमुह्यन्ति भूतानि राजदौरात्म्यतोऽनघ}


\twolineshloka
{असङ्ख्याता भविष्यन्ति भिक्षवो लिङ्गिनस्तथा}
{आश्रमाणां विकल्पाश्च वृत्तेऽस्मिन्वैकृते युगे}


\twolineshloka
{अशृण्वानाः पुराणानां धर्माणां शतशो नराः}
{उत्पथं प्रतिपत्स्यन्ते काममन्युसमीरिताः}


\twolineshloka
{यदा निवर्त्यते पापो दण्डनीत्या महात्मभिः}
{तदा धर्मो न चलते संभूतः शाश्वतः पुरा}


\twolineshloka
{सर्वलोकगुरुं चैव राजानं योऽवमन्यते}
{न तस्य दत्तं न कृतं न श्रुतं फलति क्वचित्}


\twolineshloka
{मानुषाणामधिपतिं देवभूतं महाद्युतिम्}
{देवाश्च बहुमन्यन्ते धर्मकामं नरेश्वरम्}


\twolineshloka
{प्रजापतिर्हि भगवान्यः सर्वमसृजज्जगत्}
{स प्रवृत्तिनिवृत्त्यर्थं धर्माणां क्षत्रमिच्छति}


\threelineshloka
{प्रवृत्तस्य हि धर्मस्य बुद्ध्या यः स्मरते गतिम्}
{स मे मान्यश्च पूज्यश्च स च क्षत्रे प्रतिष्ठितः ॥भीष्म उवाच}
{}


\twolineshloka
{एवमुक्त्वा स भगवान्मरुद्गणवृतः प्रभुः}
{जगाम भवनं विष्णुरक्षरं शाश्वतं परम्}


\twolineshloka
{एवं प्रवर्तिते धर्मे पुरा सुचरितेऽनघ}
{कः क्षत्रमतिवर्तेत चेतनावान्बहुश्रुतः}


\twolineshloka
{अन्यायेन प्रवृत्तानि निवृत्तानि तथैव च}
{अन्तरा विलयं यान्ति यथा पथि विचक्षुषः}


\twolineshloka
{आदौ प्रवर्तिते चक्रे तथैवादिपरायणे}
{वर्तस्व पुरुषव्याघ्र संविजानामि तेऽनघ}


\chapter{अध्यायः ६५}
\twolineshloka
{युधिष्ठिर उवाच}
{}


\threelineshloka
{श्रुता मे कथिताः पूर्वं चत्वारो मानवाश्रमाः}
{व्याख्यानमेषामाचक्ष्व पृच्छतो मे पितामह ॥भीष्म उवाच}
{}


\twolineshloka
{विदिताः सर्व एवेह धर्मास्तव युधिष्ठिर}
{यथा मम महाबाहो विदिताः साधुसंमताः}


\threelineshloka
{यत्तु लिङ्गान्तरगतं पृच्छसे मां युधिष्ठिर}
{धर्मं धर्मभृतां श्रेष्ठ तन्निबोध नराधिप}
{}


\twolineshloka
{सर्वाण्येतानि कौन्तेय विद्यन्ते भरतर्षभ}
{साध्वाचारप्रवृत्तानां चातुराश्रम्यकर्मणाम्}


\twolineshloka
{अकामद्वेषसंयुक्तो दण़्डनीत्या युधिष्ठिर}
{समदर्शी च भूतेषु भैक्ष्याश्रमपदं भवेत्}


\twolineshloka
{वेत्ति दानं विसर्गं च विग्रहानुग्रहौ तथा}
{यथोक्तवृत्तो धीरश्च क्षमाश्रमपदं भवेत्}


\twolineshloka
{अर्हान्पूजयतो नित्यं संविभागेन पाण्डव}
{सर्वतस्तस्य कौन्तेय भैक्ष्याश्रमपदं भवेत्}


\twolineshloka
{ज्ञातिसंबन्धिमित्राणि व्यापन्नानि युधिष्ठिर}
{समभ्युद्धरमाणस्य दीक्षाश्रमपदं भवेत्}


\twolineshloka
{लोकमुख्येषु सत्कारं लिङ्गिमुख्येषु चासकृत्}
{कुर्वतस्तस्य कौन्तेय वन्याश्रमपदं भवेत्}


\twolineshloka
{आह्निकं पितृयज्ञांश्च भूतयज्ञान्समानुषान्}
{कुर्वतः पार्थ विपुलान्वन्याश्रमपदं भवेत्}


\twolineshloka
{संविभागेन भूतानामतिथीनां तथाऽर्चनात्}
{देवयज्ञैश्च राजेन्द्र वन्याश्रमपदं भवेत्}


\twolineshloka
{मर्दनं परराष्ट्राणां शिष्टार्थं सत्यविक्रम}
{कुर्वतः पुरुषव्याघ्र वन्याश्रमपदं भवेत्}


\twolineshloka
{पालनात्सर्वभूतानां स्वराष्ट्रपरिपालनात्}
{दीक्षा बहुविधा राजन्सत्याश्रमपदं भवेत्}


\twolineshloka
{वेदाध्ययननित्यत्वं क्षमाऽथाचार्यपूजनम्}
{तथोपाध्यायशुश्रूषा ब्रह्माश्रमपदं भवेत्}


\twolineshloka
{आह्निकाञ्जपमानस्य देवान्पूजयतः सदा}
{धर्मेण पुरुषव्याघ्र धर्माश्रमपदं भवेत्}


\twolineshloka
{मृत्युर्वा रक्षणं वेति यस्य राज्ञो विनिश्चयः}
{प्राणद्यूते व्यवस्थाप्य ब्रह्माश्रमपदं भवेत्}


\twolineshloka
{अजिह्नमशठं मार्गं वर्तमानस्य भारत}
{सर्वदा सर्वभूतेषु ब्रह्माश्रमपदं भवेत्}


\twolineshloka
{वानप्रस्थेषु विप्रेषु त्रैविद्येषु च भारत}
{प्रयच्छतोऽर्थान्विपुलान्वन्याश्रमपदं भवेत्}


\twolineshloka
{सर्वभूतेष्वनुक्रोशं कुर्वतस्तव भारत}
{आनृशंस्ये प्रवृत्तस्य नियतः पुण्यसंचयः}


\twolineshloka
{बालवृद्धेषु कौन्तेय सर्वावस्थं युधिष्ठिर}
{अनुक्रोशक्रिया पार्थ धर्म एष सनातनः}


\twolineshloka
{बलात्कृतेषु भूतेषु परित्राणं कुरूद्वह}
{शरणागतेषु कौरव्य परं कारुण्यमाचर}


\twolineshloka
{चराचराणां भूतानां रक्षणं चापि सर्वशः}
{यथार्हपूजां च तथा कुर्वन्गार्हस्थ्यमावसेत्}


\twolineshloka
{ज्येष्ठानुज्येष्ठपत्नीनां भ्रातॄणां पुत्रनप्तृणाम्}
{निग्रहानुग्रहौ पार्थ गार्हस्थ्यममितं तपः}


\twolineshloka
{साधूनामर्चनीयानां पूजासु विदितात्मनाम्}
{पालनं पुरुषव्याघ्र गृहाश्रमपदं भवेत्}


\twolineshloka
{आश्रमस्थानि भूतानि यस्य वेश्मनि भारत}
{भुञ्जते विपुलं भोज्यं तद्गार्हस्थ्यं युधिष्ठिर}


\twolineshloka
{यः स्थितः पुरुषो धर्मे धात्रा सृष्टे यथार्थवत्}
{आश्रमाणां हि सर्वेषां फलं प्राप्नोत्यनामयम्}


\twolineshloka
{यस्मिन्न नश्यन्ति गुणाः कौन्तेय पुरुषे सदा}
{आश्रमस्थं तमप्याहुर्नरश्रेष्ठं युधिष्ठिर}


\twolineshloka
{स्थानमानं कुलेमानं वयोमानं तथैव च}
{कुर्वन्वसति सर्वेषु ह्याश्रमेषु युधिष्ठिर}


\twolineshloka
{देशधर्मांश्च कौन्तेय कुलधर्मास्तथैव च}
{पालयन्पुरुषव्याघ्र राजा सर्वाश्रमी भवेत्}


\twolineshloka
{काले विभूतिं भूतानामुपहारांस्तथैव च}
{अर्हयन्पुरुषव्याघ्र साधूनामाश्रमे वसेत्}


\twolineshloka
{देशधर्मगतश्चापि यो धर्मं प्रत्यवेक्षते}
{सर्वलोकस्य कौन्तेय राजा भवति सोश्रमी}


\twolineshloka
{ये धर्मकुशला लोके धर्मं कुर्वन्ति भारत}
{पालिता यस्य विषये पादांशस्तस्य भूपतेः}


\twolineshloka
{धर्मारामान्धर्मपरान्ये न रक्षन्ति मानवान्}
{पार्थिवाः पुरुषव्याघ्र तेषां पापं हरन्ति ते}


\twolineshloka
{ये च रक्षासहायाः स्युः पार्थिवानां युधिष्ठिर}
{ते चैवांशहराः सर्वे धर्मे परकृतेऽनघ}


\twolineshloka
{सर्वाश्रमपदेऽप्याहुर्गार्हस्थ्यं दीप्तनिर्णयम्}
{पावनं पुरुषव्याघ्र यद्वयं पर्युपास्महे}


\twolineshloka
{आत्मोपमस्तु भूतेषु यो वै भवति मानवः}
{न्यस्तदण्डो जितक्रोधः प्रेत्येह लभते सुखम्}


\twolineshloka
{धर्मोच्छ्रिता सत्यजला शीलयष्टिर्दमध्वजा}
{त्यागवाताध्वगा शीघ्रा नौस्तया सन्तरिष्यति}


\twolineshloka
{यदा सर्वत्र निर्मुक्तः कामो नास्य हृदि स्थितः}
{यदा सत्यान्वितो वृत्तैस्तदा ब्रह्म सम श्नुते}


\twolineshloka
{सुप्रसन्नस्तु भावेन योगेन च नराधिप}
{धर्मं पुरुषशार्दूल प्राप्स्यसे पालने रतः}


\twolineshloka
{वेदाध्ययनशीलानां विप्राणां साधुकर्मणाम्}
{पालने यत्नमातिष्ठ सर्वलोकस्य चानघ}


\twolineshloka
{वने चरन्ति ये धर्ममाश्रमेषु च भारत}
{रक्षणात्तच्छतगुणं धर्मं प्राप्नोति पार्थिवः}


\twolineshloka
{एष ते विविधो धर्मः पाण्डवश्रेष्ठ कीर्तितः}
{युधिष्ठिर त्वमेनं वै पूर्वं दृष्टं सनातनम्}


\twolineshloka
{चातुराश्रम्यमैकाग्र्यं चातुर्वर्ण्यं च पाण्डवं}
{धर्मं पुरुषशार्दूल प्राप्स्यसे पालने रतः}


\chapter{अध्यायः ६६}
\twolineshloka
{युधिष्ठिर उवाच}
{}


\threelineshloka
{चातुराश्रम्यमुक्तं ते चातुर्वण्यं तथैव च}
{राष्ट्रस्य यत्कृत्यतमं तन्मे ब्रूहि पितामह ॥भीष्म उवाच}
{}


\twolineshloka
{राष्ट्रस्य यत्कृत्यतमं राज्ञ एवाभिषेचनम्}
{अनिन्द्रमबलं राष्ट्रं दस्यवोऽभिभवन्त्युत}


\twolineshloka
{अराजकेषु राष्ट्रेषु धर्मो न व्यवतिष्ठते}
{परस्परं च खादन्ति सर्वथा धिगराजकम्}


\twolineshloka
{इन्द्रमेव प्रणमते यद्राजानमिति श्रुतिः}
{यथैवेन्द्रस्तथा राजा संपूज्यो भूतिमिच्छता}


\twolineshloka
{नाराजकेषु राष्ट्रेषु वस्तव्यमिति वैदिकम्}
{नाराजकेषु राष्ट्रेषु हव्यं वहति पावकः}


\twolineshloka
{अथ चेदधिवेर्तेत राज्यार्थी बलवत्तरः}
{अराजकाणि राष्ट्राणि हतवीराणि वा पुनः}


\twolineshloka
{प्रत्युद्गम्याभिपूज्यः स्यादेतदत्र सुमन्त्रितम्}
{न हि राज्यात्पापतरमस्ति किंचिदराजकात्}


\twolineshloka
{स चेत्समनुपश्येत समग्रं कुशलं भवेत्}
{बलवान्हि प्रकुपितः कुर्यान्निः शेषतामपि}


\twolineshloka
{भूयांसं लभते क्लेशं या गौर्भवति दुर्दुहा}
{अथ या सुदुहा राजन्नैव तां वितुदन्त्यपि}


\twolineshloka
{यदतप्तं प्रणमते न तत्संतापयन्त्युत}
{यत्स्वयं नमते दारु न तत्संनामयन्त्यपि}


\twolineshloka
{एतयोपमया धीरः सन्नमेत बलीयसे}
{इन्द्राय स प्रणमते नमते यो बलीयसे}


\twolineshloka
{तस्माद्राजैव कर्तव्यः सततं भूतिमिच्छता}
{न धनार्थो न दारार्थस्तेषां येषामराजकम्}


\twolineshloka
{प्रीयते हि हरन्पापः परवित्तमराजके}
{यदाऽस्य तद्धरन्त्यन्ये तदा राजानमिच्छति}


\twolineshloka
{पापा ह्यपि तदा क्षेमं न लभन्ते कदाचन}
{एकस्य हि द्वौ हरतो द्वयोश्च बहवोऽपरे}


\twolineshloka
{अदासः क्रियते दासो ह्रियन्ते च बलात्स्त्रियः}
{एतस्मात्कारणाद्देवाः प्रजापालान्प्रचक्रिरे}


\twolineshloka
{राजा चेन्न भवेल्लोके पृथिव्या दण्डधारकः}
{जले मत्स्यानिवाभक्ष्यन्दुर्बलं बलवत्तराः}


\twolineshloka
{अराजकाः प्रजाः पूर्वं विनेशुरिति नः श्रुतम्}
{परस्परं भक्षयन्तो मत्स्या इव जले कृशान्}


\twolineshloka
{समेत्य तास्ततश्चक्रुः समयानिति नः श्रुतम्}
{वाक्शूरो दण्डपरुषो यश्च स्यात्पारदारिकः}


\threelineshloka
{यश्च नः समयं भिन्द्यात्त्याज्या नस्तादृशा इति}
{विश्वासार्थं च सर्वेषां वर्णानामविशेषतः}
{तास्तथा समयं कृत्वा समयेनावतस्थिरे}


\twolineshloka
{सहितास्तास्तदा जग्मुरसुखार्ताः पितामहम्}
{अनीश्वरा विनश्यामो भगवन्नीश्वरं दिश}


\threelineshloka
{यं पूजयेम संभूय यश्च नः प्रतिपालयेत्}
{ताभ्यो मनुं व्यादिदेश मनुर्नाभिननन्द ताः ॥मनुरुवाच}
{}


\threelineshloka
{बिभेमि कर्मणः पापाद्राज्यं हि भृशदुष्करम्}
{विशेषतो मनुष्येषु मिथ्यावृत्तेषु नित्यदा ॥भीष्म उवाच}
{}


\twolineshloka
{तमब्रुवन्प्रजा मा भैर्विधास्यामो धनं तव}
{पशूनामथ पञ्चांशं धरण्यस्य तथैव च}


\twolineshloka
{धान्यस्य दशमं भागं दास्यामः कोशवर्धनम्}
{कन्यां शुल्के चारुरूपां विवाहेषूद्यतासु च}


\twolineshloka
{मुख्येन शस्त्रपत्रेण ये मनुष्याः प्रधानतः}
{भवन्तं तेऽनुयास्यन्ति महेन्द्रमिव देवताः}


\twolineshloka
{स त्वं जातबलो राजन्दुष्प्रधर्षः प्रतापवान्}
{सुखे धास्यसि नः सर्वान्कुबेर इव नैर्ऋतान्}


\twolineshloka
{यं च धर्मं चरिष्यन्ति प्रजा राज्ञा सुरक्षिताः}
{चतुर्थं तस्य धर्मस्य त्वत्संस्थं नो भविष्यति}


\twolineshloka
{तेन धर्मेण महता सुखं लब्धेन भावितः}
{पाह्यस्मान्सर्वतो राजन्देवानिव शतक्रतुः}


\twolineshloka
{विजयाय हि निर्याहि प्रतपत्रश्मिवानिव}
{मानं विधम शत्रूणां धर्मं जनय नः सदा}


\twolineshloka
{स निर्ययौ महातेजा बलेन महता वृतः}
{महाभिजनसंपन्नस्तेजसा प्रज्वलन्निव}


\threelineshloka
{तस्य दृष्ट्वा महत्वं ते महेन्द्रस्येव देवताः}
{अपतत्रसिरे सर्वे स्वधर्मे च ददुर्मनः}
{`वर्णिनश्चाश्रमाश्चैव म्लेच्छाः सर्वे च दस्यवः ॥'}


\twolineshloka
{ततो महीं परिययौ पर्जन्य इव वृष्टिमान्}
{शमयन्सर्वतः पापान्स्वकर्मसु च योजयन्}


\twolineshloka
{एवं ये भूतिमिच्छेयुः पृथिव्यां मानवाः क्वचित्}
{कुर्यू राजानमेवाग्रे प्रजानुग्रहकारणात्}


\twolineshloka
{नमस्येरंश्च तं भक्त्या शिष्या इव गुरुं सदा}
{देवा इव च देवेन्द्रं नरा राजानमन्तिकात्}


\twolineshloka
{सत्कृतं स्वजनेनेह परोऽपि बहुमन्यते}
{स्वजनेन त्ववज्ञातं परे परिभवन्त्युत}


\twolineshloka
{राज्ञः परैः परिभवः सर्वेषामसुखावहः}
{तस्माच्छत्रं च पत्रं च वासांस्याभरणानि च}


\twolineshloka
{भोजनान्यथ पानानि राज्ञे दद्युर्गृहाणि च}
{आसनानि च शय्याश्च सर्वोपकरणानि च}


\twolineshloka
{गोप्ता चास्य दुराधर्षः स्मितपूर्वाभिभाषिता}
{आभाषितश्च मधुरं प्रत्याभाषेत मानवान्}


\twolineshloka
{कृतज्ञो दृढभक्तिः स्यात्संविभागी जितेन्द्रियः}
{ईक्षितः प्रतिवीक्षेत मृदु वल्गु च चर्जु च}


\chapter{अध्यायः ६७}
\twolineshloka
{युधिष्ठिर उवाच}
{}


\threelineshloka
{किमाहुर्दैवतं विप्रा राजानं भरतर्षभ}
{मनुष्याणामधिपतिं तन्मे ब्रूहि पितामह ॥भीष्म उवाच}
{}


\twolineshloka
{अत्राप्युदाहरन्तीममितिहासं पुरातनम्}
{बृहस्पतिं वसुमना यथा पप्रच्छ भारत}


\twolineshloka
{राजा वसुमना नाम कौसल्यो धीमतां वरः}
{महर्षि किल पप्रच्छ कृतप्रज्ञं बृहस्पतिम्}


\twolineshloka
{सर्वं वैनयिकं कृत्वा विनयज्ञो बृहस्पतिम्}
{दक्षिणानन्तरो भूत्वा प्रणम्य विधिपूर्वकम्}


\twolineshloka
{विधिं पप्रच्छ राजस्य सर्वलोकहिते रतः}
{प्रजानां सुखमन्विच्छन्धर्मशीलं बृहस्पतिम्}


\twolineshloka
{केन भूतानि वर्धन्ते क्षयं गच्छन्ति केन वा}
{कमर्चन्तो महाप्राज्ञ सुखमव्ययमाप्नुयुः ॥`एतन्मे शंस देवर्षे धर्मकामार्थसंशयम्'}


\threelineshloka
{एवं पृष्टो महाप्राज्ञः कौसल्येनामितौजसा}
{राजसत्कारमव्यग्रो राज्यस्य च विवर्धनम्}
{दण्डनीतिं समाश्रित्य शशंसास्मै बृहस्पतिः}


\twolineshloka
{राजमूलो महाप्राज्ञ धर्मो लोकस्य लक्ष्यते}
{प्रजा राजभयादेव न खादन्ति परस्परम्}


\twolineshloka
{राजा ह्येवाखिलं लोकं समुदीर्णं समुत्सुकम्}
{प्रसादयति धर्मेण प्रसाद्य च विराजते}


\twolineshloka
{यथा ह्यनुदये राजन्भूतानि शशिसूर्ययोः}
{अन्धे तमसि मज्जेयुरपश्यन्तः परस्परम्}


\twolineshloka
{यथा ह्यनुदके मत्स्या निराक्रन्दे विहङ्गमाः}
{विहरेयुर्यथाकामं विहिंसन्तः पुनः पुनः}


\twolineshloka
{विमथ्यातिक्रमेरंश्च विषह्यापि परस्परम्}
{अभावमचिरेणैव गच्छेयुर्नात्र संशयः}


\twolineshloka
{एवमेव विना राज्ञा विनश्येयुरिमाः प्रजाः}
{अन्धे तमसि मज्जेयुरगोपाः पशवो यथा}


\twolineshloka
{हरेयुर्बलवन्तोऽपि दुर्बलानां परिग्रहान्}
{हन्युर्व्यायच्छमानांश्च यदि राजा न पालयेत्}


\threelineshloka
{ममेदमिति लोकेऽस्मिन्न भवेत्स्वपरिग्रहः}
{न दारा न च पुत्रः स्यान्न धनं न परिग्रहः}
{विष्वग्लोपः प्रवर्तेत यदि राजा न पालयेत्}


\twolineshloka
{यानं वस्त्रमलङ्कारान्रत्नानि विविधानि च}
{हरेयुः सहसा पापा यदि राजा न पालयेत्}


\twolineshloka
{पतेद्बहुविधं शस्त्रं बहुधा धर्मचारिषु}
{अधर्मः प्रगृहीतः स्याद्यदि राजा न पालयेत्}


\twolineshloka
{मातरं पितरं वृद्धमाचार्यमतिथिं गुरुम्}
{क्लिश्नीयुरपि हिंस्युर्वा यदि राजा न पालयेत्}


\threelineshloka
{`अन्यांश्च क्रोशतो हिंस्युर्लोकोऽयं दस्युवद्भवेत्}
{'वधबन्धपरिक्लेशो नित्यमर्थवतां भवेत्}
{ममत्वं च न विन्देयुर्यदि राजा न पालयेत्}


\twolineshloka
{अन्ताश्चाकाल एव स्युर्लोकोऽयं दस्युसाद्भवेत्}
{पतेद्बहुविधं राज्यं यदि राजा न पालयेत्}


\twolineshloka
{न योनिदोषो वर्तेत न कृषिर्न वणिक्पथः}
{मज्जेद्धर्मस्त्रयी न स्याद्यदि राजा न पालयेत्}


\twolineshloka
{न यज्ञाः संप्रवर्तेयुर्विधिवत्स्वाप्तदक्षिणाः}
{न विवाहाः समाजो वा यदि राजा न पालयेत्}


\twolineshloka
{न वृषाः संप्रवर्तेरन्न मथ्येरंश्च गर्गराः}
{घोषाः प्रणाशं गच्छेयुर्यदि राजा न पालयेत्}


\twolineshloka
{त्रस्तमुद्विग्नहृदयं हाहाभूतमचेतनम्}
{क्षणेन विनशेत्सर्वं यदि राजा न पालयेत्}


\twolineshloka
{न संवत्सरसत्राणि तिष्ठेयुरकुतोभयाः}
{विधिवद्दक्षिणावन्ति यदि राजा न पालयेत्}


\twolineshloka
{ब्राह्मणाश्चतुरो वेदान्नाधीयीरंस्तपस्विनः}
{विद्यास्नाता व्रतस्नाता यदि राजा न पालयेत्}


\twolineshloka
{न भवेद्धर्मसंसेवी मोहविप्रहतो जनः}
{हर्ता स्वच्छेन्द्रियो गच्छेद्यदि राजा न पालयेत्}


\twolineshloka
{हस्ताद्धस्तं परिमुषेद्भिद्येरन्सर्वसेतवः}
{भयार्तं विद्रवेत्सर्वं यदि राजा न पालयेत्}


\twolineshloka
{अनयाः संप्रवपर्तेरन्भवेद्वै वर्णसङ्करः}
{दुर्भिक्षमाविशेद्राष्ट्रं यदि राजा न पालयेत्}


\twolineshloka
{विवृत्य हि यथाकामं गृहद्वाराणि शेरते}
{मनुष्या रक्षिता राज्ञा समन्तादकुतोभयाः}


\twolineshloka
{नाक्रुष्टं सहते कश्चित्कुतो वा हस्तलाघवम्}
{यदि राजा न सम्यग्गां रक्षयत्यपि धार्मिकः}


\twolineshloka
{स्त्रियश्चापुरुषा मार्गं सर्वालङ्कारभूषिताः}
{निर्भयाः प्रतिपद्यन्ते यदि रक्षति भूमिपः}


\twolineshloka
{धर्ममेव प्रपद्यन्ते न हिंसन्ति परस्परम्}
{अनुगृह्णन्ति चान्योन्यं यदा रक्षति भूमिपः}


\twolineshloka
{यजन्ते च महायज्ञैस्त्रयो वर्णाः पृथग्विधैः}
{युक्ताश्चाधीयते विद्यां यदा रक्षति भूमिपः}


\twolineshloka
{वार्तामूलो ह्ययं लोकस्तया वै धार्यते सदा}
{तत्सर्वं वर्तते सम्यग्यदा रक्षति भूमिपः}


\twolineshloka
{यदा राजा धुरं श्रेष्ठामादाय वहति प्रजाः}
{महता बलयोगेन तदा लोकः प्रसीदयि}


\twolineshloka
{यस्याभावेन भूतानामभावः स्यात्समन्ततः}
{भावे च भावो नित्यं स्यात्कस्तं न प्रतिपूजयेत्}


\twolineshloka
{तस्य यो वहते भारं सर्वलोकसुखावहम्}
{तिष्ठन्प्रियहिते राज्ञ उभौ लोकाविमौ जयेत्}


\twolineshloka
{यस्तस्य पुरुषः पापं मनसाऽप्यनुचिन्तयेत्}
{असंशयमिह क्लिष्टः प्रेत्यापि नरकं व्रजेत्}


\twolineshloka
{न हि जात्ववमन्तव्यो मनुष्य इति भूमिपः}
{महती देवता ह्येषा नररूपेण तिष्ठति}


\twolineshloka
{कुरुते पञ्चरूपाणि कालयुक्तानि यः सदा}
{भवत्यग्निस्तथाऽऽदित्यो मृत्युर्वैश्रवणो यमः}


\twolineshloka
{यदा ह्यासीदतः पापान्दहत्युग्रेण तेजसा}
{मिथ्योपचरितो राजा तदा भवति पावकः}


\twolineshloka
{यदा पश्यति चारेण सर्वभूतानि भूमिपः}
{क्षेमं च कृत्वा व्रजति तदा भवति भास्करः}


\twolineshloka
{अशुचींश्च यदा क्रुद्धः क्षिणोति शतशो नरान्}
{सपुत्रपौत्रान्सामात्यांस्तदा भवति सोन्तकः}


\twolineshloka
{यदा त्वधार्मिकान्सर्वांस्तीक्ष्णैर्दण़्डैर्नियच्छति}
{धार्मिकांश्चानुगृह्णाति भवत्यथ यमस्तदा}


\twolineshloka
{यदा तु धनधाराभिस्तर्पयत्युपकारिणः}
{आच्छिनत्ति च रत्नानि विविधान्यपकारिणां}


\twolineshloka
{श्रियं ददाति कस्मैचित्कस्माच्चिदपकर्षति}
{तदा वैश्रवणो राजा लोके भवति भूमिपः}


\twolineshloka
{नास्यापवादे स्थातव्यं दक्षेणाक्लिष्टकर्मणा}
{धर्ममाकाङ्क्षता लोके ईश्वरस्यानसूयता}


\twolineshloka
{न हि राज्ञः प्रतीपानि कुर्वन्सुखमवाप्नुयात्}
{पुत्रो भ्राता वयस्यो वा यद्यप्यात्मसमो भवेत्}


\twolineshloka
{कुर्यात्कृष्णगतिः शेषं ज्वलितोऽनिलसारथिः}
{न तु राज्ञाऽभिपन्नस्य शेषं क्वचन विद्यते}


\twolineshloka
{तस्य सर्वाणि रक्ष्याणि दूरतः परिवर्जयेत्}
{मृत्योरिव जुगुप्सेत राजस्वहरणान्नरः}


\twolineshloka
{वध्येतभिमृशन्सद्यो मृगः कूटमिव स्पृशन्}
{आत्मस्वमिव संरक्षेद्राजस्वमिह बुद्धिमान्}


\twolineshloka
{महान्तं नरकं घोरमप्रतिष्ठमचेतसः}
{पतन्ति चिररात्राय राजवित्तापहारिणः}


\twolineshloka
{राजा भोजो विराट् सम्राट् क्षत्रियो भूपतिर्नृपः}
{य एभिः स्तूयते शब्दैः कस्तं नार्चितुमर्हति}


\twolineshloka
{तस्माद्बुभूषुर्नियतो जितात्मा संयतेन्द्रियः}
{मेधावी धृतिमान्दक्षः संश्रयेत् महीपतिम्}


\twolineshloka
{कृतज्ञं प्राज्ञमक्षुद्रं दृढभक्तिं जितेन्द्रियम्}
{धर्मनित्यं स्थितं स्थाने मन्त्रिणं पूजयेन्नृपः}


\twolineshloka
{दृढभक्तिं कृतप्रज्ञं धर्मज्ञं संयतेन्द्रियम्}
{शूरमक्षुद्रकर्माणं प्रसिद्धं जनमाश्रयेत्}


\twolineshloka
{राजा प्रगल्भं पुरुषं करोतिराजा भृशं बृंहयते मनुष्यम्}
{राजाभिपन्नस्य कुतः सुखानिराजाऽभ्युपेतं सुखिनं करोति}


\twolineshloka
{`राजा प्रजानां प्रथमं शरीरंप्रजाश्च राज्ञोऽप्रतिमं शरीरम्}
{राज्ञा विहीना न भवन्ति देशादेशैर्विहीना न नृपा भवन्ति ॥'}


\twolineshloka
{राजा प्रजानां हृदयं गरीयोगतिः प्रतिष्ठा सुखमुत्तमं च}
{समाश्रिता लोकमिमं परं चजयन्ति सम्यक्पुरुषा नरेन्द्र}


\threelineshloka
{नराधिपश्चाप्यनुशिष्य मेदिनींदमेन सत्येन च सौहृदेन}
{महद्भिरिष्ट्वा क्रतुभिर्महायशास्त्रिविष्टपे स्थानमुपैति शाश्वतम् ॥भीष्म उवाच}
{}


\twolineshloka
{स एवमुक्तोऽङ्गिरसा कौसल्यो राजसत्तमः}
{प्रयत्नात्कृतवान्वीरः प्रजापालनमुत्तमम्}


\chapter{अध्यायः ६८}
\twolineshloka
{युधिष्ठिर उवाच}
{}


\twolineshloka
{पार्थिवेन विशेषेण किं कार्यमवशिष्यते}
{कथं रक्ष्यो जनपदः कथं वध्याश्च शत्रवः}


\threelineshloka
{कथं चारान्प्रयुञ्जीत वर्णान्विश्वासयेत्कथम्}
{कथं भृत्यान्कथं दारान्कथं पुत्रांश्च भारत ॥भीष्म उवाच}
{}


\twolineshloka
{राजवृत्तं महाराज शृणुष्वावहितोऽखिलम्}
{यत्कार्यं पार्थिवेनादौ पार्थिवप्रकृतेन वा}


\twolineshloka
{आत्मा जेयः सदा राज्ञा ततो जेयाश्च शत्रवः}
{अजितात्मा नरपतिर्विजयेत कथं रिपून्}


\twolineshloka
{एतावानात्मविजयः पञ्चवर्गविनिग्रहः}
{जितेन्द्रियो नरपतिर्बाधितुं शक्नुयाद्रिपुन्}


\twolineshloka
{न्यसेत गुल्मान्दुर्गेषु सन्धौ च कुरुनन्दन}
{नगरोपवने चैव पुरोद्याने तथैव च}


\twolineshloka
{संस्थानेषु च सर्वेषु पुटेषु नगरस्य च}
{मध्ये च नरशार्दूल तथा राजनिवेशने}


\twolineshloka
{प्रणिर्धीश्च ततः कुर्याज्जडान्धबधिराकृतीन्}
{पुंसः परीक्षितान्प्राज्ञान्क्षुत्पिपासाश्रमक्षमान्}


\twolineshloka
{अमात्येषु च सर्वेषु मित्रेषु त्रिविधेषु च}
{पुत्रेषु च महाराज प्रणिदध्यात्समाहितः}


\twolineshloka
{पुरे जनपदे चैव तथा सामन्तराजसु}
{यथा न विद्युरन्योन्यं प्रणिधेयास्तथा हि ते}


\twolineshloka
{चारांश्च विद्यात्प्रहितान्परेण भरतर्षभ}
{आपणेषु विहारेषु समवायेषु वीथिषु}


\twolineshloka
{आरामेषु तथोद्याने पण्डितानां समागमे}
{वेशेषु चत्वरे चैव सभास्वावसथेषु च}


\twolineshloka
{एवं विहन्याच्चारेण परचारं विचक्षणः}
{चारे च विहते सर्वं हतं भवति भारत}


\twolineshloka
{यदा तु हीनं नृपतिर्विद्यादात्मानमात्मना}
{अमात्यैः सह संमन्त्र्य कुर्यात्सन्धिं बलीयसा}


\twolineshloka
{`विद्वांसः क्षत्रिया वैश्या ब्राह्मणाश्च बहुश्रुताः}
{दण्डनीतौ तु निष्पन्ना मन्त्रिणः पृथिवीपते}


\threelineshloka
{प्रष्टव्यो ब्राह्मणः पूर्वं नीतिशास्त्रस्य तत्ववित्}
{पश्चात्पृच्छेत भूपालः क्षत्रियं नीतिकोविदम्}
{वैश्यशूद्रौ तथा भूयः शास्त्रज्ञौ हितकारिणौ ॥ '}


\twolineshloka
{अज्ञायमाने हीनत्वे सन्धिं कुर्यात्परेण वै}
{लिप्सुर्वा कंचिदेवार्थं त्वरमाणो विचक्षणः}


\twolineshloka
{गुणवन्तो महोत्साहा धर्मज्ञाः साधवश्च ये}
{सन्दधीत नृपस्तैश्च राष्ट्रं धर्मेण पालयन्}


\twolineshloka
{उच्छिद्यमानमात्मानं ज्ञात्वा राजा महामतिः}
{पूर्वापकारिणो हन्याल्लोकद्विष्टांश्च सर्वशः}


\twolineshloka
{यो नोपकर्तुं शक्नोति नापकर्तुं महीपतिः}
{न शक्यरूपश्चोद्धर्तुमुपेक्ष्यस्तादृशो भवेत्}


\twolineshloka
{यात्रां यायादविज्ञातमनाक्रन्दमनन्तरम्}
{व्यासक्तं च प्रमत्तं च दुर्बलं च विचक्षणः}


\twolineshloka
{यात्रामाज्ञापयेद्वीरः कल्यः पुष्टबलः सुखी}
{पूर्वं कृत्वा विधानं च यात्रायां नगरे तथा}


\twolineshloka
{न च पश्यो भवेदस्य नृपो यश्चातिवीर्यवान्}
{हीनश्च बलवीर्याभ्यां कर्षयंस्तत्परो वसेत्}


\twolineshloka
{राष्ट्रं च पीडयेत्तस्य शस्त्राग्निविषमूर्च्छनैः}
{अमात्यवल्लभानां च विवादांस्तस्य कारयेत्}


\twolineshloka
{वर्जनीयं सदा युद्धं राज्यकामेन धीमता}
{उपायैस्त्रिभिरादानमर्थस्याह बृहस्पतिः}


\twolineshloka
{सान्त्वेन तु प्रदानेन भेदेन च नराधिप}
{यमर्थं शक्नुयात्प्राप्तुं तेन तुष्येत पण्डितः}


\twolineshloka
{आददीत बलिं चापि प्रजाभ्यः कुरुनन्दन}
{षङ्भागममितप्रज्ञस्तासामेवाभिगुप्तये}


\twolineshloka
{दशाधर्मगतेभ्यो यद्वसु बह्वल्पमेव वा}
{तन्नाददीत सहसा पौराणां रक्षणाय वै}


\twolineshloka
{यथा पुत्रास्तथा पौरा द्रष्टव्यास्ते न संशयः}
{भक्तिश्चैषु न कर्तव्या व्यवहारप्रदर्शने}


\twolineshloka
{श्रोतुं चैव न्यसेद्राजा प्राज्ञान्सर्वार्थदर्शिनः}
{व्यवहारेषु सततं तत्र राज्यं प्रतिष्ठितम्}


\twolineshloka
{आकरे लवणे शुल्के तरे नागबले तथा}
{न्यसेदमात्यान्नृपतिः स्वाप्तान्वा पुरुषान्हितान्}


\twolineshloka
{सम्यग्दण्डधरो नित्यं राजा धर्ममवाप्नुयात्}
{नृपस्य सततं दण्डः सम्यग्धर्मः प्रशस्यते}


\twolineshloka
{वेदवेदाङ्गवित्प्राज्ञः सुतपस्वी नृपो भवेत्}
{दानशीलश्च सततं यज्ञशीलश्च भारत}


\twolineshloka
{एते गुणाः समस्ताः स्युर्नृपस्य सततं स्थिराः}
{व्यवहारस्य लोपेन कुतः स्वर्गः कुतो यशः}


\twolineshloka
{यदा तु पीडितो राजा भवेद्राज्ञा बलीयसा}
{[तदाभिसंश्रयेद्दुर्गं बुद्धिमान्पृथिवीपतिः}


\twolineshloka
{विधावाक्रम्य मित्राणि विधानमुपकल्पयेत्}
{सामभेदान्विरोधार्थं विधानमुपकल्पयेत् ॥]}


\threelineshloka
{`त्रिधा तु कृत्वा मित्राणि विधानमुपकल्पयेत्}
{'घोषान्न्यसेत मार्गेषु ग्रामानुत्थापयेदपि}
{प्रवेशयेच्च तान्सर्वाञ्शाखानगरकेष्वपि}


\twolineshloka
{ये गुप्ताश्चैव दुर्गाश्च देशास्तेषु प्रवेशयेत्}
{धनिनो बलमुख्यांश्च सान्त्वयित्वा पुनः पुनः}


\twolineshloka
{सस्याभिहारं कुर्वीत स्वयमेव नराधिपः}
{असंभवे प्रवेशस्य दाहयेदग्निना भृशम्}


\twolineshloka
{क्षेत्रस्थेषु च सस्येषु शत्रोरुपजपेन्नरान्}
{विनाशयेद्वा तत्सर्वं बलेनाथ स्वकेन वै}


\twolineshloka
{नदीमार्गेषु च तथा संक्रमानवसादयेत्}
{जलं विस्रावयेत्सर्वमविस्राव्यं च दूषयेत्}


\twolineshloka
{तदात्वेनायतीभिश्च निवसेद्भूम्यनन्तरम्}
{प्रतिघातं परस्याजौ युद्धकालेऽप्युपस्थिते}


\twolineshloka
{दुर्गाणां चाभितो राजा मूलच्छेदं प्रकारयेत्}
{सर्वेषां क्षुद्रवृक्षाणां चैत्यवृक्षान्विवर्जयेत्}


\twolineshloka
{प्रवृद्धानां च वृक्षाणां शाखां प्रच्छेदयेत्तथा}
{चैत्यानां सर्वथा त्याज्यमपि पत्रस्य पातनम्}


\threelineshloka
{`देवानामाश्रयाश्चैत्या यक्षराक्षसभोगिनाम्}
{पिशाचपन्नगानां च गन्धर्वाप्सरसामपि}
{रौद्राणां चैव भूतानां तस्मात्तान्परिवर्जयेत्}


\twolineshloka
{श्रूयते हि निकुम्भेन सौदासस्य बलं हतम्}
{महेश्वरगणेशेन वाराणस्यां नराधिप ॥ '}


\twolineshloka
{प्रगण्डीः कारयेत्सम्यगाकाशजननीस्तदा}
{आपूरयेच्च परिखां स्थाणुनक्रझषाकुलाम्}


\twolineshloka
{सङ्कटद्वारकाणि स्युरुच्छ्वासार्थं पुरस्य च}
{तेषां च द्वारवद्गुप्तिः कार्या सर्वात्मना भवेत्}


\twolineshloka
{द्वारेषु च गुरूण्येव यन्त्राणि स्थापयेत्सदा}
{आरापयेच्छतघ्नीश्च स्वाधीनानि च कारयेत्}


\twolineshloka
{काष्ठानि चाभिहार्याणि तथा कूपांश्च खानयेत्}
{संशोधयेत्तथा कूपान्कृतान्पूर्वपयोर्थिभिः}


\twolineshloka
{तृणच्छन्नानि वेश्मानि पङ्केनाथ प्रलेपयेत्}
{निर्हरेच्च तृणं मासि चैत्रे वह्निभयात्पुरा}


\twolineshloka
{नक्तमेव च भक्तानि पाचयेत नराधिपः}
{न दिवा ज्वालयेदग्निं वर्जयित्वाऽऽग्निहोत्रिकम्}


\twolineshloka
{`यथासंभवशैलानि चैष्टकानि च कारयेत्}
{मृण्मयानि च कुर्वीत ज्ञात्वा देशं बलाबलम् ॥ '}


\twolineshloka
{कर्मारारिष्टशालासु ज्वलेदग्निः सुरक्षितः}
{गृहाणि च प्रवेश्यान्तर्विधेयः स्याद्धुताशनः}


\twolineshloka
{महादण्डश्च तस्य स्याद्यस्याग्निर्वै दिवा भवेत्}
{प्रघोषयेदथैवं च रक्षणार्थं पुरस्य च}


\twolineshloka
{भिक्षुकांश्चाक्रिकांश्चैव क्लीबोन्मत्तान्कुशीलवान्}
{बाह्यान्कुर्यान्नरश्रेष्ठ दोषाय स्युर्हि ते शठाः}


\twolineshloka
{चत्वरेष्वथ तीर्थेषु सभास्वावसथेषु च}
{यथार्थवर्णं प्रणिधिं कुर्यात्सर्वत्र पार्थिवः}


\twolineshloka
{विशालान्राजमार्गांश्च कारयेत नराधिपः}
{प्रपाश्च विपणीश्चैव यथोद्देशं समादिशेत्}


\twolineshloka
{भाण्डागारायुधागारान्धान्यागारांश्च सर्वशः}
{अश्वागारान्गजागारान्बलाधिकरणानि च}


\twolineshloka
{परिखाश्चैव कौरव्य प्रतोलीसङ्कटानि च}
{न जातु कश्चित्पश्येत गुह्यमेतद्युधिष्ठिर}


\twolineshloka
{अर्थसंनिचयं कुर्याद्राजा परबलार्दितः}
{तैलं मधु घृतं सस्यमौषधानि च सर्वशः}


\twolineshloka
{अङ्गारकुशमुञ्जानां पलाशशरवर्णिनाम्}
{यवसेन्धनदिग्धानां कारयेत च संचयान्}


\twolineshloka
{आयुधानां च सर्वेषां शक्त्यृष्टिप्रासचर्मणाम्}
{संचयानेवमादीनां कारयेत नराधिपः}


\twolineshloka
{औषधानि च सर्वाणि मूलानि च फलानि च}
{चतुर्विधांश्च वैद्यान्वै संगृह्णीयाद्विशेषठः}


\twolineshloka
{नटांश्च नर्तकांश्चैव मल्लान्मायाविनस्तथा}
{शोभयेयुः पुरवरं मोदयेयुश्च सर्वशः}


\twolineshloka
{यतः शङ्का भवेच्चापि भृत्यतोऽथापि मन्त्रितः}
{पौरेभ्यो नृपतेर्वापि स्वाधीनान्कारयेत तान्}


\twolineshloka
{कृते कर्माणि राजा तान्पूजयेद्धनसंचयैः}
{माननेन यथार्हेण सान्त्वेन विविधेन च}


\twolineshloka
{निर्वेदयित्वा तु परं हत्वा वा कुरुनन्दन}
{गतानृण्यो भवेद्राजा यथा शास्त्रे निदर्शितम्}


\twolineshloka
{राज्ञा सप्तैव रक्ष्याणि तानि चैव निबोध मे}
{आत्माऽमात्याश्च कोशाश्च दण्डो मित्राणि चैव हि}


\twolineshloka
{तथा जनपदाश्चैव पुरं च कुरुनन्दन}
{एतत्सप्तात्मकं राज्यं परिपाल्यं प्रयत्नतः}


\twolineshloka
{षाङ्गुण्यं च त्रिवर्गं च त्रिवर्गपरमं तथा}
{यो वेत्ति पुरुषव्याघ्र स भुङ्क्ते पृथिवीमिमाम्}


\twolineshloka
{षाङ्गुण्यमिति यत्प्रोक्तं तन्निबोध युधिष्ठिर}
{सन्धायासनमित्येव यात्रासन्धानमेव च}


\twolineshloka
{विगृह्यासनमित्येव यात्रां संपरिगृह्य च}
{द्वैधीभावस्तथाऽन्येषां संश्रयोऽथ परस्य च}


\twolineshloka
{त्रिवर्गश्चापि यः प्रोक्तस्तमिहैकमनाः शृणु}
{क्षयः स्थानं च वृद्धिश्च त्रिवर्गः परमस्तथा}


\threelineshloka
{`धर्मश्चार्थश्च कामश्च त्रिवर्गो वै सनातनः}
{मन्त्रश्चैव प्रभावश्च उत्साहश्चैव तान्त्रिकः}
{शक्तित्रयं समाख्यातं त्रिवर्गस्य च तत्परम्}


\threelineshloka
{कार्यं च कारणं चैव कर्ता च परिकीर्तितः}
{एतत्परतरं विद्यान्त्रिवर्गादपि भारत}
{सर्वेपां च क्षये राजन्यस्त्रिवर्गः सनातनः}


\twolineshloka
{सत्वं रजस्तमश्चैव त्रिवर्गकरणं स्मृतम्}
{तेनात्यन्तविमुक्तश्च मुक्तः पुरुष उच्यते}


\threelineshloka
{कार्यस्य सर्वथा नाशो मोक्ष इत्यभिधीयते}
{तेन मोक्षपरश्चैव देवदेवः पितामहः}
{तुष्ट्यर्थस्य त्रिवर्गस्य रक्षामाह पितामहः}


\twolineshloka
{जगतो लौकिकी यात्रा यत्र नित्यं प्रतिष्ठिता}
{'धर्मोऽर्थश्चैव कामश्च सेवितव्याः स्वकालतः}


\twolineshloka
{`सेवा धर्मस्य कर्तव्या सततं भूरिवत्सरैः}
{पुरुषैर्नरशार्दूल तन्मूलाः सर्वथा क्रियाः ॥'}


\threelineshloka
{धर्मेण च महीपालश्चिरं रक्षति मेदिनीम्}
{`यः कश्चिद्धार्मिको राजा स विपन्नोऽपि भूपतिः}
{अर्थकामविहीनोऽपि चिरं पालयते महीम् ॥'}


\twolineshloka
{अस्मिन्नर्थे हि द्वौ श्लोकौ गीतावङ्गिरसा स्वयम्}
{यादवीपुत्र भद्रं ते श्रोतुमर्हसि तावपि}


\twolineshloka
{कृत्वा कार्याणि धर्मेण सम्यक्संपाल्य मेदिनीम्}
{पालयित्वा तथा पौरान्परत्र सुखमेधते}


\twolineshloka
{किं तस्य तपसा राज्ञः किंच तस्याध्वरैः कृतैः}
{सुपालिताः प्रजा यस्य सर्वधर्मकृदेव सः}


\twolineshloka
{श्लोकाश्चोशनसा गीतास्तान्निबोध युधिष्ठिर}
{दण्डनीतेश्च यन्मूलं त्रिवर्गस्य च भूपते}


\twolineshloka
{भार्गवाङ्गिरसं कर्म षोडशाङ्गं च यद्बलम्}
{विषं माया च दैवं च पौरुषं चात्मसिद्धये}


\twolineshloka
{प्रागुदक्प्रवणं दुर्गं समासाद्य महीपतिः}
{त्रिवर्गत्रयसंपूर्णमुपादाय तमुद्वहेत्}


\twolineshloka
{षट््पञ्च च विनिर्जित्य दश चाष्टौ च भूपतिः}
{त्रिवर्गैर्दशभिर्युक्तः सुरैरपि न जीर्यते}


\threelineshloka
{न बुद्धिं परिगृह्णीत स्त्रीणां मूर्खजनस्य च}
{दैवोपहतबुद्धीनां ये च वेदैर्विवर्जिताः}
{न तेषां शृणुयाद्राजा बुद्धिस्तेषां पराङ्भुखी}


\twolineshloka
{स्त्रीप्रधानानि राज्यानि विद्वद्भिर्वर्जितानि च}
{मूर्खामात्याग्नितप्तानि शुष्यन्ते जलबिन्दुवत्}


\twolineshloka
{विद्वांसः प्रथिता ये च ये चाप्ताः सर्वकर्मसु}
{बुद्धेषु दृष्टकर्माणि तेषां च शृणुयान्नृपः}


\twolineshloka
{दैवं पुरुषकारं च त्रिवर्गं च समाश्रितः}
{दैवतानि च विप्रांश्च प्रणम्य विजयी भवेत् ॥'}


\chapter{अध्यायः ६९}
\twolineshloka
{युधिष्ठिर उवाच}
{}


\threelineshloka
{दण़्डनीतिश्च राजा च समस्तौ तावुभावपि}
{तत्र किं कुर्वतः सिद्धिस्तन्मे ब्रूहि पितामह ॥भीष्म उवाच}
{}


\twolineshloka
{माहात्म्यं दण्डनीत्यास्तु साध्यं शब्दैः सहेतुकैः}
{शृणु मे शंसतो राजन्यथावदिह भारत}


\twolineshloka
{दण्डनीतिः स्वधर्मेषु चातुर्वर्ण्यं नियच्छति}
{प्रयुक्ता स्वामिना सम्यगधर्मेभ्यो नियच्छति}


\twolineshloka
{चातुर्वर्ण्ये स्वधर्मस्थे मर्यादानामसंकरे}
{दण़्डनीतिकृते क्षेमे प्रजानामकुतोभये}


\twolineshloka
{सोमे प्रयत्नं कुर्वन्ति वयो वर्णा यथाविधि}
{तस्मादेव मनुष्याणां सुखं विद्धि समाहितम्}


\twolineshloka
{कालो वा कारणं राज्ञो राजा वा कालकारणम्}
{इति ते संशयो माभूद्राजा कालस्य कारणम्}


\twolineshloka
{दण्डनीत्यां यदा राजा सम्यक्कार्त्स्न्येन वर्तते}
{तदा कृतयुगं नाम कालः श्रेष्ठः प्रवर्तते}


\twolineshloka
{भवेत्कृतयुगे धर्मो नाधर्मो विद्यते क्वचित्}
{सर्वेषामेव वर्णानां नाधर्मे रमते मनः}


\twolineshloka
{योगक्षेमाः प्रवर्तन्ते प्रजानां नात्र संशयः}
{वैदिकानि च कर्माणि भवन्त्यपि गुणान्युत}


\twolineshloka
{ऋतवश्च सुखाः सर्वे भवन्त्युत निरामयाः}
{प्रसीदन्ति नराणां च स्वरवर्णमनांसि च}


\twolineshloka
{व्याधयो न भवन्त्यत्र नाल्पायुर्दृश्यते नरः}
{विधवा न भवन्त्यत्र नृशंसो नात्र जायते}


\twolineshloka
{अकृष्टपच्या पृथिवी भवन्त्योषधयस्तथा}
{त्वक्पत्रफलमूलानि वीर्यवन्ति भवन्ति च}


\twolineshloka
{नाधर्मो विद्यते तत्र धर्म एव तु केवलम्}
{इति कार्तयुगानेतान्गुणान्विद्धि युधिष्ठिर}


\twolineshloka
{दण्डनीत्या यदा राजा त्रीनंशाननुवर्तते}
{चतुर्थमंशमुत्सृज्य तदा त्रेता प्रवर्तते}


\twolineshloka
{अधर्मस्य चतुर्थांशस्त्रीनंशाननुवर्तते}
{कृष्टपच्यैव पृथिवी भवन्त्योषधयस्तथा}


\twolineshloka
{अर्धं त्यक्त्वा यदा राजा नीत्यर्धमनुवर्तते}
{ततस्तु द्वापरं नाम स कालः संप्रवर्तते}


\twolineshloka
{अशुभस्य यदा त्वर्धं द्वावंशावनुवर्तते}
{कुष्टपच्यैव पृथिवी भवत्यर्धफला तथा}


\twolineshloka
{दण्डनीतिं परित्यज्य यदा कार्त्स्न्येन भूमिपः}
{प्रजाः क्लिश्नात्ययोगेन प्रवर्तेत तदा कलिः}


\twolineshloka
{कलावधर्मो भूयिष्ठो धर्मो भवति न क्वचित्}
{सर्वेषामेव वर्णानां स्वधर्माच्च्यवते मनः}


\twolineshloka
{शूद्रा भैक्षेण जीवन्ति ब्राह्मणाः परिचर्यया}
{योगक्षेमस्य नाशश्च वर्तते वर्णसंकरः}


\twolineshloka
{वैदिकानि च कर्माणि भवन्ति विगुणान्युत}
{ऋतवो न सुखाः सर्वे भवन्त्यामयिनस्तथा}


\twolineshloka
{ह्रसन्त च मनुष्याणां स्वरवर्णमनांस्युत}
{व्याधयश्च भवन्त्यत्र म्रियन्ते चाशतायुषः}


\twolineshloka
{विधवाश्च भवन्त्यत्र नृशंसा जायते प्रजा}
{क्वचिद्वर्षति पर्जन्यः क्वचित्सस्यं प्ररोहति}


\twolineshloka
{रसाः सर्वे क्षयं यान्ति यदा नेच्छति भूमिपः}
{प्रजाः संरक्षितुं सम्यग्दण़्डनीतिसमाहितः}


\twolineshloka
{राजा कृतयुगस्रष्टा त्रेताया द्वापरस्य च}
{युगस्य च चतुर्थस्य राजा भवति कारणम्}


\twolineshloka
{कृतस्य करणाद्राजा स्वर्गमत्यन्तमश्नुते}
{त्रेतायाः करणाद्राजा स्वर्गं नात्यन्तमश्नुते}


\twolineshloka
{प्रवर्तनाद्द्वापरस्य यथाभागमुपाश्नुते}
{कलेः प्रवर्तनाद्राजा पापमत्यन्तमश्नुते}


\twolineshloka
{ततो वसति दुष्कर्मा नरके शाश्वतीः समाः}
{प्रजानां कल्मषे मग्नोऽकीर्ति पापं च विन्दति}


\threelineshloka
{दण्डनीतिं पुरस्कृत्य क्षत्रियेण विजानता}
{लिप्सितव्यमलभ्यं च लब्धं रक्ष्यं च भारत}
{`योगक्षेमाः प्रवर्तन्ते प्रजानां नात्र संशयः ॥'}


\twolineshloka
{लोकस्य सीमन्तकरी मर्यादा लोकपावनी}
{सम्यङ्गीता दण्डनीतिर्यथा माता यथा पिता}


\twolineshloka
{यस्यां भवन्ति भूतानि तद्विद्धि भरतर्षभ}
{एष एव परो धर्मो यद्राजा दण्डनीतिमान्}


\twolineshloka
{तस्मात्कौरव्य धर्मेण प्रजाः पालय नीतिमान्}
{एवं वृत्तः प्रजा रक्षन्स्वर्गं जेतासि दुर्जयम्}


\chapter{अध्यायः ७०}
\twolineshloka
{युधिष्ठिर उवाच}
{}


\threelineshloka
{क्तेन वृत्तेन वृत्तक्ष वर्तमानो महीपतिः}
{सुखेनार्यान्तुसुखोदर्कानिह च प्रेत्य चाश्नुते ॥भीष्म उवाच}
{}


\twolineshloka
{इत्थं गुणानां षट््त्रिंशी षट््त्रिंशद्गुणसंयुता}
{यान्गुणांस्तु गुणोपेतः कुर्वन्गुणमवाप्नुयात्}


\twolineshloka
{चरेद्धर्मानकटुको मुञ्चेत्स्नेहं न चास्तिकः}
{अनृशंसश्चरेदर्थं चरेत्काममनुद्धतः}


\twolineshloka
{प्रियं ब्रूयादकृपणः शूरः स्यादविकत्थनः}
{दाता नापात्रवर्षी स्यात्प्रगल्भः स्यादनिष्ठुरः}


\twolineshloka
{संदधीत न चानार्यैर्विगृह्णीयाच्च शत्रुभिः}
{नानाप्तैश्चारयेच्चारं कुर्यात्कार्यमपीडया}


\twolineshloka
{अर्थं ब्रूयान्न चासत्सु गुणान्ब्रूयान्न चात्मनः}
{आदद्यान्न च साधुभ्यो नासत्पुरुषमाश्रयेत्}


\twolineshloka
{नापरीक्ष्य नयेद्दण्डं न च मन्त्रं प्रकाशयेत्}
{विसृजेन्न च लुब्धेभ्यो विश्वसेन्नापकारिषु}


\twolineshloka
{अनीर्षुर्गुप्तदारः स्याच्चोक्षः स्यादघृणी नृपः}
{स्त्रियः सेवेत नात्यर्थं मृष्टं भुञ्जीत नाहितम्}


\twolineshloka
{अस्तब्धः पूजयन्मान्यान्गुरून्सेवेदमायया}
{अर्चेद्देवानदम्भेन श्रियमिच्छेदकृत्सिताम्}


\twolineshloka
{सेवेत प्रणयं हित्वा दक्षः स्यान्न त्वकालवित्}
{सान्त्वयेन्न च मोक्षाय अनुगृह्णन्न चाक्षिपेत्}


\twolineshloka
{प्रहरेन्न त्वविज्ञाय हत्वा शत्रून्न शोचयेत्}
{क्रोधं कुर्यान्न चाकस्मान्मृदुः स्यान्नापकारिषु}


\twolineshloka
{एवं चरस्व राज्यस्थो यदि श्रेय इहेच्छसि}
{अतोऽन्यथा नरपतिर्भयमृच्छत्यनुत्तमम्}


\threelineshloka
{इति सर्वान्गुणानेतान्यथोक्तान्योऽनुवर्तते}
{अनुभूयेह भद्राणि प्रेत्य स्वर्गे महीयते ॥वैशंपायन उवाच}
{}


\twolineshloka
{इदं वचः शान्तनवस्य शुश्रुवान्युधिष्ठिरः पार्थिवमुख्यसंवृतः}
{तदा ववन्दे च पितामहं नृपोयथोक्तमेतच्च चकार बुद्धिमान्}


\chapter{अध्यायः ७१}
\twolineshloka
{युधिष्ठिर उवाच}
{}


\threelineshloka
{कथं राजा प्रजा रक्षन्नाधिबन्धेन युज्यते}
{धर्मे च नापराध्नोति तन्मे ब्रूहि पितामह ॥भीष्म उवाच}
{}


\twolineshloka
{समासेनैव ते राजन्धर्मान्वक्ष्यामि शाश्वतान्}
{विस्तरेणैव धर्माणां न जात्वन्तमवाप्नुयात्}


\twolineshloka
{धर्मनिष्ठाञ्श्रुतवतो देवव्रतसमाहितान्}
{अर्चितान्वासयेथास्त्वं गृहे गुणवतो द्विजान्}


\twolineshloka
{प्रत्युत्थायोपसंगृह्य चरणावभिवाद्य च}
{अथ सर्वाणि कुर्वीथाः कार्याणि सपुरोहितः}


\twolineshloka
{धर्मकार्याणि निर्वर्त्य मङ्गलानि प्रयुज्य च}
{ब्राह्मणान्वाचयेथास्त्वमर्थसिद्धिजयाशिषः}


\twolineshloka
{आर्जवेन च संपन्नो धृत्या बुद्ध्या च भारत}
{धर्मार्थौ प्रतिगृह्णीयात्कामक्रोधौ च वर्जयेत्}


\twolineshloka
{कामक्रोधौ पुरस्कृत्य योऽर्थं राजाऽनुतिष्ठति}
{न स धर्मं न चाप्यर्थं प्रतिगृह्णाति बालिशः}


\twolineshloka
{मा स्म लुब्धांश्च मूर्खांश्च कामार्थे च प्रयूयुजः}
{अलुब्धान्बुद्धिसंपन्नान्सर्वकर्मसु योजयेत्}


\twolineshloka
{मूर्खो ह्यधिकृतोऽर्थेषु कार्याणामविशारदः}
{प्रजाः क्लिश्नात्ययोगेन कामक्रोधसमन्वितः}


\twolineshloka
{बलिषष्ठेन शुल्केन दण्डेनाथापराधिनाम्}
{शास्त्रानीतेन लिप्सेथा वेतनेन धनागमम्}


\twolineshloka
{दापयित्वा करं धर्म्यं राष्ट्रं नीत्या यथाविधि}
{तथैतं कल्पयेद्राजा योगक्षेममतन्द्रितः}


\twolineshloka
{गोपायितारं दातारं धर्मनित्यमतन्द्रितम्}
{अकामद्वेषसंयुक्तमनुरज्यन्ति मानवाः}


\twolineshloka
{तस्माद्धर्मेण लाभेन लिप्सेथास्त्वं धनागमम्}
{धर्मार्थावध्रुवौ तस्य यो न शास्त्रपरो भवेत्}


\twolineshloka
{अपशास्त्रधनो राजा संचयं नाधिगच्छति}
{अस्थाने चास्य तद्वित्तं सर्वमेव विनश्यति}


\twolineshloka
{अर्थमूलोऽपि हिंसां च कुरुते स्वयमात्मनः}
{करैरशास्त्रदृष्टैर्हि मोहात्संपीडयन्प्रजाः}


\twolineshloka
{ऊधश्छिन्द्यात्तु यो धेन्वाः क्षीरार्थी न लभेत्पयः}
{एवं राष्ट्रमयोगेन पीडितं न विवर्धते}


\threelineshloka
{`यवसोदकमादाय सान्त्वेन विनयेन च}
{'यो हि दोग्ध्रीमुपास्ते च स नित्यं विन्दते पयः}
{एवं राष्ट्रमुपायेन भुञ्जानो लभते फलम्}


\twolineshloka
{अथ राष्ट्रमुपायेन भुज्यमानं सुरक्षितम्}
{जनयत्यतुलां नित्यं कोशवृद्धिं युधिष्ठिर}


\twolineshloka
{दोग्ध्री धान्यं हिरण्यं च मही राजा सुरक्षिता}
{नित्यं स्वेभ्यः परेभ्यश्च तृप्ता माता यथा पयः}


\twolineshloka
{मालाकारोपमो राजन्भव माऽऽङ्गारिकोपमः}
{तथायुक्तश्चिरं राज्यं भोक्तुं शक्ष्यसि पालयन्}


\twolineshloka
{परचक्राभियानेन यदि ते स्याद्धनक्षयः}
{अथ साम्नैव लिप्सेथा धनमब्राह्मणेषु यत्}


\twolineshloka
{मा स्म ते ब्राह्मणं दृष्ट्वा धनस्थं प्रचलेन्मनः}
{अन्त्यायामप्यवस्थायां किमु स्फीतस्य भारत}


\twolineshloka
{धनानि तेभ्यो दद्यास्त्वं यथाशक्ति यथार्हतः}
{सान्त्वयन्परिरक्षंश्च स्वर्गमाप्स्यसि दुर्जयम्}


\twolineshloka
{एवं धर्म्येण वृत्तेन प्रजास्त्वं परिपालय}
{स्वर्ग्यं पुण्यं यशो नित्यं प्राप्स्यसे कुरुनन्दन}


\twolineshloka
{धर्मेण व्यवहारेण प्रजाः पालय पाण्डव}
{युधिष्ठिर यथायुक्तो नाधिबन्धेन योक्ष्यसे}


\twolineshloka
{एष एव परो धर्मो यद्राजा रक्षति प्रजाः}
{भूतानां हि यदा धर्मो रक्षणं परमा दया}


\twolineshloka
{तस्मादेवं परं धर्मं मन्यन्ते धर्मकोविदाः}
{यो राजा रक्षणे युक्तो भूतेषु कुरुते दयाम्}


\twolineshloka
{यदह्ना कुरुते पापमरक्षन्भयतः प्रजाः}
{राजा वर्षसहस्रेण तस्यान्तमधिगच्छति}


\twolineshloka
{यदह्ना कुरुते धर्मं प्रजा धर्मेण पालयन्}
{दशवर्षसहस्राणि तस्य भुङ्क्ते फलं दिवि}


\twolineshloka
{स्विष्टिः स्वधीतिः सुतपा लोकाञ्जयति यावतः}
{क्षणेन तानवाप्नोति प्रजा धर्मेण पालयन्}


\twolineshloka
{एवं धर्मं प्रयत्नेन कौन्तेय परिपालय}
{ततः पुण्यफलं लब्ध्वा नानुबन्धेन योक्ष्यसे}


\twolineshloka
{स्वर्गलोके सुमहतीं श्रियं प्राप्स्यसि पाण्डव}
{असंभवश्च धर्माणामीदृशानामराजसु}


\threelineshloka
{तस्माद्राजैव नान्योऽस्ति यो धर्मफलमाप्नुयात्}
{स राज्यं धृतिमान्प्राप्य धर्मेण परिपालय}
{इन्द्रं तर्पय सोमेन कामैश्च सुहृदो जनान्}


\chapter{अध्यायः ७२}
\twolineshloka
{युधिष्ठिर उवाच}
{}


\threelineshloka
{`कीदृशो ब्राह्मणो राजा कार्याकार्यविचारणे}
{क्षमः कर्तुं समर्थो वा तन्मे ब्रूहि पितामह ॥'भीष्म उवाच}
{}


\twolineshloka
{य एव तु सतो रक्षेदसतश्च निवर्तयेत्}
{स एव राज्ञा कर्तव्यो राजन्राजपुरोहितः}


\threelineshloka
{अत्राप्युदाहरन्तीममितिहासं पुरातनम्}
{पुरूरवस ऐलस्य संवादं मातरिश्वना ॥पुरूरवा उवाच}
{}


\threelineshloka
{कुतः स्विद्ब्राह्मणो जातो वर्णाश्चापि कुतस्त्रयः}
{कस्माच्च भवति श्रेयांस्तन्मे व्याख्यातुमर्हसि ॥मातरिश्वोवाच}
{}


\twolineshloka
{ब्राह्मणो मुखतः सृष्टो ब्रह्मणो राजसत्तम}
{बाहुभ्यां क्षत्रियः सृष्ट ऊरुभ्यां वैश्य एव च}


\twolineshloka
{वर्णानां परिचर्यार्यं त्रयाणां भरतर्षभ}
{वर्णश्चतुर्थः पश्चात्तु पभ्द्यां शूद्रो विनिर्मितः}


\twolineshloka
{ब्राह्मणो जायमानो हि पृथिव्यामनुजायते}
{ईश्वरः सर्वभूतानां धर्मकोशस्य गुप्तये}


\twolineshloka
{`सर्वस्वं ब्राह्मणस्येदं यत्किंचिदिह दृश्यते}
{धर्मयुक्तं प्रशस्तं च जगत्यस्मिन्नृपात्मज ॥'}


\twolineshloka
{ततः पृथिव्या यन्तारं क्षत्रियं दण्डधारणे}
{द्वितीयं वर्णमकरोत्प्रजानामनुगुप्तये}


\threelineshloka
{वैश्यस्तु धनधान्येन त्रीन्वर्णान्बिभृयादिमान्}
{शूद्रो ह्येतान्परिचरेदिति ब्रह्मानुशासनम् ॥ऐल उवाच}
{}


\threelineshloka
{द्विजस्य क्षत्रबन्धोर्वा कस्येयं पृथिवी भवेत्}
{धर्मतः सह वित्तेन सम्यग्वायो प्रचक्ष्व मे ॥वायुरुवाच}
{}


\twolineshloka
{विप्रस्य सर्वमेवैतद्यत्किंचिज्जगतीगतम्}
{`धनं धान्यं हिरण्यं च स्त्रियो रत्नानि वाहनम्}


\twolineshloka
{मङ्गलं च प्रशस्तं च यच्चान्यदपि विद्यते}
{'ज्येष्ठेनाभिजनेनेह तद्धर्मकुशला विदुः}


\twolineshloka
{स्वमेव ब्राह्मणो भुङ्क्ते स्वं वस्ते स्वं ददाति च}
{गुरुर्हि सर्ववर्णानां ज्येष्ठः श्रेष्ठश्च वै द्विजः}


\threelineshloka
{पत्यभावे यथैव स्त्री देवरं कुरुते पतिम्}
{`आनन्तर्यात्तथा क्षत्रं पृथिवी कुरुते पतिम्}
{'एष ते प्रथमः कल्प आपद्यन्यो भवेदतः}


\twolineshloka
{यदि स्वर्गे परं स्थानं धर्मतः परिमार्गसि}
{यत्किंचिज्जयसे भूमिं ब्राह्मणाय निवेदय}


\twolineshloka
{श्रुतवृत्तोपपन्नाय धर्मज्ञाय तपस्विने}
{स्वधर्मपरितृप्ताय यो न वित्तपरो भवेत्}


\twolineshloka
{यो राजानं नयेद्बुद्ध्या सर्वतः परिपूर्णया}
{ब्राह्मणो हि कुले जातः कृतप्रज्ञो विनीतवान्}


\twolineshloka
{श्रेयो नयति राजानं ब्रुवंश्चित्रां सरस्वतीम्}
{राजा चरति यं धर्मं ब्राह्मणेन निदर्शितम्}


\twolineshloka
{शुश्रूषुरनहंवादी क्षत्रधर्मव्रते स्थितः}
{तावता सत्कृतः प्राज्ञश्चिरं यशसि तिष्ठति}


\twolineshloka
{तस्य धर्मस्य सर्वस्य भागी राजपुरोहितः}
{एवमेव प्रजाः सर्वा राजानमभिसंश्रिताः}


\twolineshloka
{`ब्राह्मणं च सविद्वांसं राजशास्त्रविपश्चितम्}
{'सम्यग्वृत्ताः स्वधर्मस्था न कुतश्चिद्भयान्विताः}


\twolineshloka
{राष्ट्रे चरन्ति यं धर्मं राज्ञा साध्वभिरक्षिताः}
{चतुर्थं तस्य धर्मस्य राजा भागं तु विन्दति}


\twolineshloka
{देवा मनुष्याः पितरो गन्धर्वोरगराक्षसाः}
{यज्ञमेवोपजीवन्ति नास्ति यष्टा ह्यराजके}


\twolineshloka
{इतो दत्तेन जीवन्ति देवताः पितरस्तथा}
{राजन्येवास्य धर्मस्य योगक्षेमः प्रतिष्ठितः}


\threelineshloka
{छायायामप्सु वायौ च सुखमुष्णेऽधिगच्छति}
{अग्नौ वाससि सूर्ये च सुखं शीतेऽधिगच्छति}
{शब्दे स्पर्शे रसे रूपे गन्धे च रमते मनः}


\twolineshloka
{तेषु भोगेषु सर्वेषु न भीतो लभते सुखम्}
{अभयस्य हि यो दाता तस्यैव सुमहत्फलम्}


\threelineshloka
{न हि प्राणसमं दानं त्रिषु लोकेषु विद्यते}
{इन्द्रो राजा यमो राजा धर्मो राजा तथैव च}
{राजा बिभर्ति भूतानि राज्ञा सर्वमिदं धृतम्}


\chapter{अध्यायः ७३}
\twolineshloka
{` युधिष्ठिर उवाच}
{}


\threelineshloka
{राज्ञा पुरोहितः कार्यः कीदृशो वर्णतो भवेत्}
{पुरोधा यादृशः कार्यः कथयस्व पितामह ॥भीष्म उवाच}
{}


\twolineshloka
{गौरो वा लोहितो वाऽपि श्यामो वा नीरुजः सुखी}
{अक्रोधनो ह्यचपलः सर्वतश्च जितेन्द्रियः ॥'}


\twolineshloka
{राज्ञा पुरोहितः कार्यो भवेद्विद्वान्बहुश्रुतः}
{उभौ समीक्ष्य धर्मार्थावप्रमेयावनन्तरम्}


\twolineshloka
{धर्मात्मा मन्त्रविद्येषां राज्ञां राजन्पुरोहितः}
{`तेषामर्थश्च कामश्च धर्मश्चेति विनिश्चयः}


\twolineshloka
{श्लोकाश्चोशनसा गीतास्तान्निबोध युधिष्ठिर}
{उच्छिष्टः स भवेद्राजा यस्य नास्ति पुरोहितः}


\twolineshloka
{रक्षसामसुराणां च पिशाचोरगपक्षिणाम्}
{शत्रूणां च भवेद्वध्यो यस्य नास्ति पुरोहितः}


\twolineshloka
{ब्रह्मत्वं सर्वयज्ञेषु कुर्वीताथर्वणो द्विजः}
{राज्ञश्चाथर्ववेदेन सर्वकर्माणि कारयेत्}


\twolineshloka
{ब्रूयाद्गर्ह्याणि सततं महोत्पातान्यघानि च}
{इष्टमङ्गलयुक्तानि तथाऽन्तः पुरिकाणि च}


\twolineshloka
{गीतनृत्ताधिकारेषु संमतेषु महीपतेः}
{कर्तव्यं करणीयं वै वैश्वदेवबलिस्तथा}


\twolineshloka
{पक्षसंधिषु कुर्वीत महाशान्तिं पुरोहितः}
{रौद्रहोमसहस्रं च स्वस्य राज्ञः प्रियं हितम्}


\twolineshloka
{राज्ञः पापमलाः सप्त यानृच्छति पुरोहितः}
{अमात्याश्च कुकर्माणो मन्त्रिणश्चाप्युपेक्षकाः}


\twolineshloka
{चौर्यमव्यवहारश्च व्यवहारोपसेविनाम्}
{अदण्ड्यदण्डनं चैव दण्ड्यानां चाप्यदण्डनम्}


\twolineshloka
{हिंसा चान्यत्र संग्रामाद्राज्ञश्च मल उच्यते}
{कुभृत्यैस्तु प्रजानाशः सप्तमस्तु महामलः}


\twolineshloka
{रौद्रैर्होमैर्महाशान्त्या घृतकम्बलकर्मणा}
{भृग्वङ्गिरोविधिज्ञो वै पुरोधा निर्णुदे मलात्}


\twolineshloka
{एतान्हित्वा दिवं याति राजा सप्त महामलान्}
{सामात्यः सपुरोधाश्च प्रजानां पालने रतः}


\twolineshloka
{एतस्मिन्नेव कौरव्य पौरोहित्ये महामते}
{श्लोकानाह महेन्द्रस्य गुरुर्देवो बृहस्पतिः}


\twolineshloka
{तान्निबोध महीपाल महाभाग हिताञ्शुभान्}
{ऋग्वेदे सामवेदे च यजुर्वेदे च वाजिनाम्}


\twolineshloka
{न निर्दिष्टानि कर्माणि त्रिषु स्थानेषु भूभृताम्}
{शान्तिकं पौष्टिकं चैव अरिष्टानां च शातनम्}


\twolineshloka
{शप्तास्ते याज्ञवल्क्येन यज्ञानां हितमीहता}
{ब्रह्मिष्ठानां वरिष्ठेन ब्रह्मणः संमते विभोः}


\threelineshloka
{बह्वृचं सामगं चैव वाजिनं च विवर्जयेत्}
{बह्वृचो राष्ट्रनाशाय राजनाशाय सामगः}
{अध्वर्युर्बलनाशाय प्रोक्तो वाजसनेयकः}


\twolineshloka
{अब्राह्मणेषु वर्णेषु मन्त्रान्वाजसनेयकान्}
{शान्तिके पौष्टिके चैव नित्यं कर्मणि वर्जयेत्}


\twolineshloka
{ब्राह्मणस्य महीपस्य सर्वथा न विरोधिनः}
{वेदाश्चत्वार इत्येते ब्राह्मणा ये च तद्विदुः}


\twolineshloka
{पौरोहित्ये प्रमाणं तु ब्राह्मणश्च महीपतेः}
{जात्या न क्षत्रियः प्रोक्तः क्षतत्राणं करोति यः}


\twolineshloka
{चातुर्वर्ण्यबहिष्ठोऽपि स एव क्षत्रियः स्मृतः}
{भार्गवाङ्गिरसैर्मन्त्रैस्तेषां कर्म विधीयते}


\twolineshloka
{वैतानं कर्म यच्चैव गृह्यकर्म च यत्स्मृतम्}
{द्विजातीनां त्रयाणां तु सर्वकर्म विधीयते}


\twolineshloka
{राजधर्मप्रवृत्तानां हितार्थं त्रीमि कारयेत्}
{शान्तिकं पौष्टिकं चैव तथाऽभिचरणं च यत्}


\twolineshloka
{अग्निष्टोममुखैर्यज्ञैर्दूषिता भूपकर्मभिः}
{न सम्यक्फलमृच्छन्ति ये यजन्ति द्विजातयः}


\twolineshloka
{पौरोहित्यं तु कुर्वाणा नाशं यास्यन्ति भूभृताम्}
{यज्ञकर्माणि कुर्वाणा ऋत्विजस्तु विरोधिनः}


\twolineshloka
{ब्रह्मक्षत्रविशः सर्वे पौरोहित्ये विवर्जिताः}
{तदभावे च पारक्यं निर्दिष्टं राजकर्मसु}


\twolineshloka
{ऋषिणा याज्ञवल्क्येन तत्तथा न तदन्यथा}
{भार्गवाङ्गिरसां वेदे कृतविद्यः षडङ्गवित्}


\twolineshloka
{यज्ञकर्मविधिज्ञस्तु विधिज्ञः पौष्टिकेषु च}
{अष्टादशविकल्पानां विधिज्ञः शान्तिकर्मणाम्}


\twolineshloka
{सर्वरोगविहीनश्च संयतः संयतेन्द्रियः}
{श्वित्रकुष्ठक्षयक्षीणैर्ग्रहापस्मारदूषितैः}


\twolineshloka
{अशस्तैर्वातदुष्टैश्च दूरस्थैः संवदेन्नृपः}
{रोगिणं ऋत्विजं चैव वर्जयेच्च पुरोहितम्}


\twolineshloka
{नचान्यस्य कृतं येन पौरोहित्यं कदाचन}
{यस्य याज्यो मृतश्चैव भ्रष्टः प्रव्रजितो यथा}


\twolineshloka
{युद्धे पराजितश्चैव सर्वांस्तान्वर्जयेन्नृपः}
{नक्षत्रस्यानुकूल्येन यः संजातो नरेश्वरः}


\twolineshloka
{राजशास्त्रविनीतश्च श्रेयान्राज्ञः पुरोहितः}
{अधन्यानां निमित्तानामुत्पातानामथार्थवित्}


\twolineshloka
{शत्रुपक्षक्षयज्ञश्च श्रेयान्राज्ञः पुरोहितः}
{वाजिनं तदभावे च चरकाध्वर्यवानपि}


\twolineshloka
{बह्वृचं सामगं चैव नीतिशास्त्रकृतश्रमान्}
{कृतिनोऽथर्वणो वेदे स्थापयेत्तु पुरोहितान्}


\twolineshloka
{हिंसालिङ्गा हि निर्दिष्टा मन्त्रा वैतानिकैर्द्विजैः}
{न तानुच्चारयेत्प्राज्ञः क्षात्रधर्मविरोधिनः}


\twolineshloka
{प्रजागुणाः पुरोधाश्च पुरोहितगुणाः प्रजाः}
{'राजा वै सगुणो येषां कुशलं तेषु सर्वशः}


\twolineshloka
{उभौ प्रजा वर्धयतो देवान्पूर्वापरान्पितॄन्}
{यौ भवेतां स्थितौ धर्मे श्रद्धेयौ सुतपस्विनौ}


\twolineshloka
{परस्परस्य सुहृदौ विहितौ समचेतसौ}
{ब्रह्मक्षत्रस्य समानात्प्रजा सुखमवाप्नुयात्}


\twolineshloka
{विमाननात्तयोरेव प्रजा नश्येयुरेव हि}
{ब्रह्मक्षत्रं हि सर्वासां प्रजानां मूलमुच्यते}


\threelineshloka
{अत्राप्युदाहरन्तीममितिहासं पुरातनम्}
{ऐलकश्यपसंवादं तं निबोध युधिष्ठिर ॥ऐल उवाच}
{}


\threelineshloka
{यदा हि ब्रह्म प्रजहाति क्षत्रंक्षत्रं यदा वा प्रजहाति ब्रह्म}
{अन्वग्बलं कतमेऽस्मिन्भजन्तेतथा बलं कतमेऽस्मिन्ध्रियन्ते ॥कश्यप उवाच}
{}


\twolineshloka
{द्विधा हि राष्ट्रं भवति क्षत्रियस्यब्रह्म क्षत्रं यत्र विरुध्यतीह}
{अन्वग्बलं दस्यवस्तद्भजन्तेतथा वर्णं तत्र विदन्ति सन्तः}


\twolineshloka
{नैषां ब्रह्म च वर्धते नोत पुत्रान गर्गरो मथ्यते नो जयन्ते}
{नैषां पुत्रा देवमधीयते चयदा ब्रह्म क्षत्रियाः संत्यजन्ति}


\twolineshloka
{नैषामर्थो वर्धते जातु गेहेनाधीयते तत्प्रजा नो यजन्ते}
{अपध्वस्ता दस्युभूता भवन्तिये ब्राह्मणान्क्षत्रियाः संत्यजन्ति}


\twolineshloka
{एतौ हि नित्यं संयुक्तावितरेतरधारणे}
{क्षत्रं वै ब्रह्मणो योनिर्योनिः क्षत्रस्य वै द्विजः}


\twolineshloka
{उभावेतौ नित्यमभिप्रपन्नौसंप्रापतुर्महतीं संप्रतिष्ठाम्}
{तयोः सन्धिर्भिद्यते चेत्पुराणस्ततः सर्वं भवति हि संप्रमूढम्}


\twolineshloka
{नात्र पारं लभते पारगामीमहोदधौ नौरिव संप्रभिन्ना}
{चातुर्वण्यं भवति हि संप्रमूढंप्रजास्ततः क्षयसंस्था भवन्ति}


\twolineshloka
{ब्रह्मवृक्षो रक्ष्यमाणो मधु हेम च वर्षति}
{अरक्ष्यमाणः सततमश्रु पापं च वर्षति}


\twolineshloka
{अब्रह्मचारी चरणादपेतोयदा ब्रह्म ब्रह्मणि त्राणमिच्छेत्}
{आश्चर्यतो वर्षति तत्र देवस्तत्राभीक्ष्णं दुष्प्रभाश्चाविशन्ति}


\twolineshloka
{स्त्रियं हत्वा ब्राह्मणं वाऽपि पापःसभायां यत्र लभते साधुवादम्}
{राज्ञः सकाशे न विभेति चापिततो भयं विद्यते क्षत्रियस्य}


\threelineshloka
{पापैः पापे क्रियमाणेऽतिवेलंततो रुद्रो जायते देव एषः}
{पापैः पापाः संजनयन्ति रुद्रंततः सर्वान्साध्वसाधून्हिनस्ति ॥ऐल उवाच}
{}


\threelineshloka
{कुतो रुद्रः कीदृशो वाऽपि रुद्रःसत्वैः सत्वं दृश्यते वध्यमानम्}
{एतत्सर्वं कश्यप मे प्रचक्ष्वयतो रुद्रो जायते देव एषः ॥कश्यप उवाच}
{}


\threelineshloka
{आत्मा रुद्रो हृदये मानवानांस्वं स्वं देहं परदेहं च हन्ति}
{वातोत्पातैः सदृशं रुद्रमाहुर्देवं जीमूतैः सदृशं रूपमस्य ॥ऐल उवाच}
{}


\threelineshloka
{न वै वातः परिवृणोति कश्चिन्न जीमूतो वर्षति तत्र देवः}
{तथा युक्तो दृश्यते मानुषेषुकामद्वेषाद्वध्यते मुह्यते च ॥कश्यप उवाच}
{}


\threelineshloka
{यथैकगेहाज्जातवेदाः प्रदीप्तःकृत्स्नं ग्रामं दहते च त्वरावान्}
{विमोहनं कुरुते देव एपततः सर्वं स्पृश्यते पुण्यपापैः ॥ऐल उवाच}
{}


\threelineshloka
{यदि दण्डः स्पृशतेऽपुण्यपापंपापं पापे क्रियमाणे विशेषात्}
{कस्य हेतोः सुकृतं नाम कुर्याद्दुष्कृतं वा कस्य हेतोर्न कुर्यात् ॥कश्यप उवाच}
{}


\threelineshloka
{असंत्यागात्पापकृतामपापांस्तुल्यो दण्डः स्पृशते मिश्रभावात्}
{शुष्केणार्द्रं दह्यते मिश्रभावान्न मिश्रः स्यात्पापकृद्भिः कथंचित् ॥ऐल उवाच}
{}


\threelineshloka
{साध्वसाधून्धारयतीह भूमिःसाध्वसाधूंस्तापयतीह सूर्यः}
{साध्वसाधूंश्चापि वातीह वायुरापस्तथा साध्वसाधून्वहन्ति ॥कश्यप उवाच}
{}


\twolineshloka
{एवमस्मिन्वर्तते लोक एषनामुत्रैवं वर्तते राजपुत्र}
{प्रेत्यैतयोरन्तरावान्विशेषोयो वै पुण्यं चरते यश्च पापम्}


\twolineshloka
{पुण्यस्य लोको मधुमान्घृतार्चिर्हिरण्यज्योतिरमृतस्य नाभिः}
{तत्र प्रेत्य मोदते ब्रह्मचारीन तत्र मृत्युर्न जरा नोत दुःखम्}


\twolineshloka
{पापस्य लोको निरयोऽप्रकाशोनित्यं दुःखं शोकभूयिष्ठमेव}
{तत्रात्मानं शोचति पापकर्मावह्वीः समाः प्रतपन्नप्रतिष्ठः}


\twolineshloka
{मिथोभेदाद्ब्राह्मणक्षत्रियाणांप्रजा दुःखं दुःसहं चाविशन्ति}
{एवं ज्ञात्वा कार्य एवेह विद्वान्पुरोहितो नैकविद्यो नृपेण}


\twolineshloka
{तं चैव लब्ध्वाभिषिञ्चेत्तथा धर्मो विधीयते}
{अग्रं हि ब्राह्मणः प्रोक्तं सर्वस्यैवेह धर्मतः}


\twolineshloka
{पूर्वं हि ब्रह्मणः सृष्टिरिति ब्रह्मविदो विदुः}
{ज्येष्ठेनाभिजनेनास्य प्राप्तं पूर्वं यदुत्तमम्}


\twolineshloka
{तस्मान्मान्यश्च पूज्यश्च ब्राह्मणः प्रसृताग्रभुक्}
{सर्वं श्रेष्ठं विशिष्टं च निवेद्यं तस्य धीमतः}


\threelineshloka
{अवश्यमेतत्कर्तव्यं राज्ञा बलवताऽपि हि}
{ब्रह्म वर्धयति क्षत्रं क्षत्रतो ब्रह्म वर्धते}
{राज्ञः सर्वस्य चान्यस्य स्वामी राज्ञः पुरोहितः}


\chapter{अध्यायः ७४}
\twolineshloka
{`युधिष्ठिर उवाच}
{}


\twolineshloka
{ब्रह्मक्षत्रस्य सामर्थ्यं कथितं ते पितामह}
{पुरोहितप्रभावश्च लक्षणं च पुरोधसः}


\fourlineindentedshloka
{इदानीं श्रोतुमिच्छामि ब्रह्मक्षत्रविनिर्णयम्}
{ब्रह्मक्षत्रं हि सर्वस्य कारणं जगतः परम्}
{योगक्षेमो हि राष्ट्रस्य ताभ्यामायत्त एव च ॥' भीष्म उवाच}
{}


\twolineshloka
{योगक्षेमो हि राष्ट्रस्य राजन्यायत्त उच्यते}
{योगक्षेमो हि राज्ञो हि समायत्तः पुरोहिते}


\twolineshloka
{यत्रादृष्टं भयं ब्रह्म प्रजानां शमयत्युत}
{दृष्टं च राजा बाहुभ्यां तद्राज्यं सुखमेधते}


\twolineshloka
{अत्राप्युदाहरन्तीममितिहासं पुरातनम्}
{मुचुकुन्दस्य संवादं राज्ञो वैश्रवणस्य च}


\twolineshloka
{मुचुकुन्दो विजित्येमां पृथिवीं पृथिवीपतिः}
{जिज्ञासमानः स बलमभ्ययादलकाधिपम्}


\twolineshloka
{ततो वैश्रवणो राजा राक्षसानसृजत्तदा}
{ते बलान्यवमृद्गन्त मुचुकुन्दस्य नैर्ऋताः}


\twolineshloka
{स हन्यमाने सैन्ये स्वे मुचुकुन्दो नराधिपः}
{गर्हयामास विद्वांसं पुरोहितमरिन्दमः}


\twolineshloka
{तत उग्रं तपस्तप्त्वा वसिष्ठो धर्मवित्तमः}
{रक्षांस्युपावधीत्तत्र पन्थानं चाप्यविन्दत}


\threelineshloka
{ततो वैश्रवणो राजा मुचुकुन्दमगर्हयत्}
{वध्यमानेषु सैन्येषु वचनं चेदमब्रवीत् ॥धनद उवाच}
{}


\twolineshloka
{बलवन्तस्त्वया पूर्वे राजानः सपुरोहिताः}
{न चैवं समवर्तन्त यथा त्वमिव वर्तसे}


\twolineshloka
{ते खल्वपि कृतास्त्राश्च बलवन्तश्च भूमिपाः}
{आगम्य पर्युपासन्ते मामीशं सुखदुःखयोः}


\twolineshloka
{यद्यस्ति बाहुवीर्यं ते तद्दर्शयितुमर्हसि}
{किं ब्राह्मणबलेन त्वमतिमात्रं प्रवर्तसे}


\twolineshloka
{मुचुकुन्दस्ततः क्रुद्धः प्रत्युवाच धनेश्वरम्}
{न्यायपूर्वमसंलब्धमसंभ्रान्तमिदं वचः}


\twolineshloka
{ब्रह्मक्षत्रमिदं सृष्टमेकयोनि स्वयंभुवा}
{पृथग्बलविधानं च तल्लोकं परिपालयेत्}


\twolineshloka
{तपोमन्त्रबलं नित्यं ब्राह्मणेषु प्रतिष्ठितम्}
{अस्रबाहुबलं नित्यं क्षत्रियेषु प्रतिष्ठितम्}


\twolineshloka
{ताभ्यां संभूय कर्तव्यं प्रजानां परिपालनम्}
{तथा च मां प्रवर्तन्तं किं गर्हस्यलकाधिप}


\twolineshloka
{ततोऽब्रवीद्वैश्रवणो राजानं सपुरोहितम्}
{नाहं राज्यमनिर्दिष्टं कस्मैचिद्विदधाम्युत}


\threelineshloka
{नाच्छिन्दे वाऽप्यनिर्दिष्टमिति जानीहि पार्थिव}
{प्रशाधि पृथिवीं कृत्स्नां मद्दत्तामखिलामिमाम्}
{[एवमुक्तः प्रत्युवाच मुचुकुन्दो महीपतिः ॥]}


\threelineshloka
{नाहं राज्यं भवद्दत्तं भोक्तुमिच्छामि पार्थिव}
{बाहुवीर्यार्जितं राज्यमश्नीयामिति कामये ॥भीष्म उवाच}
{}


\twolineshloka
{ततो वैश्रवणो राजा विस्मयं परमं ययौ}
{क्षत्रधर्मे स्थितं दृष्ट्वा मुचुकुन्दमरिन्दमम्}


\twolineshloka
{ततो राजा मुचुकुन्दः सोन्वशासद्वसुंधराम्}
{बाहुवीर्यार्जितां सम्यक्क्षत्रधर्ममनुव्रतः}


\twolineshloka
{एवं यो ब्रह्मविद्राजा ब्रह्मपूर्वं प्रवर्तते}
{स भुङ्क्ते विजितामुवीं यशश्च महदश्नुते}


\twolineshloka
{नित्योदकी ब्राह्मणः स्यान्नित्यशस्त्रश्च क्षत्रियः}
{तयोर्हि सर्वमायत्तं यत्किंचिज्जगतीगतम्}


\twolineshloka
{यशश्च तेजश्व महीं च कृत्स्नांप्राप्नोति राजन्विपुलां च कीर्तिम्}
{प्रधानधर्मं नृपते नियच्छतथा च धर्मस्य चतुर्थमंशम्}


\chapter{अध्यायः ७५}
\twolineshloka
{युधिष्ठिर उवाच}
{}


\threelineshloka
{यया वृत्त्या महीपालो विवर्धयति मानवान्}
{पुण्यांश्च लोकाञ्जयति तन्मे ब्रूहि पितामह ॥भीष्म उवाच}
{}


\twolineshloka
{दानशीलो भवेद्राजा यज्ञशीलश्च भारत}
{उपवासतपः शीलः प्रजानां पालने रतः}


\twolineshloka
{सर्वाश्चैव प्रजा नित्यं राजा धर्मेण पालयन्}
{उत्थानेन प्रदानेन पूजयेच्चापि धार्मिकान्}


\twolineshloka
{राज्ञा हि पूजितो धर्मस्ततः सर्वत्र पूज्यते}
{यद्यदाचरते राजा तत्प्रजानां स्म रोचते}


\twolineshloka
{नित्यमुद्यतदण्डश्च भवेन्मृत्युरिवारिषु}
{निहन्यात्सर्वतो दस्यून्न राज्ञो दस्युषु क्षमा}


\twolineshloka
{यं हि धर्मं चरन्तीह प्रजा राज्ञा सुरक्षिताः}
{चतुर्थं तस्य धर्मस्य राजा भागं च विन्दति}


\twolineshloka
{यदधीते यद्ददाति यञ्जुहोति यदर्चति}
{राजा चतुर्थभाक्तस्य प्रजा धर्मेण पालयन्}


\twolineshloka
{यद्राष्ट्रोऽकुशलं किंचिद्राज्ञो रक्षयतः प्रजाः}
{चतुर्थं तस्य पापस्य राजा भारत विन्दति}


\twolineshloka
{अप्याहुः सर्वमेवेति भूयोऽर्धमिति निश्चयः}
{कर्मणा पृथिवीपाल नृशंसोऽनृतवागपि}


\threelineshloka
{तादृशात्किल्बिपाद्राजा शृणु येन प्रमुच्यते}
{प्रत्याहर्तुमशक्यं स्याद्धनं चोरैर्हृतं यदि}
{तत्स्वकोशात्प्रदेयं स्यादशक्तेनोपजीवतः}


\twolineshloka
{सर्ववर्णैः सदा रक्ष्यं ब्रह्मस्वं ब्राह्मणा यथा}
{न स्थेयं विषये तेन योऽपकुर्याद्द्विजातिषु}


\twolineshloka
{ब्रह्मस्वे रक्ष्यमाणे तु सर्वं भवति रक्षितम्}
{तेषां प्रसादे निर्वृत्ते कृतकृत्यो भवेन्नृपः}


\twolineshloka
{पर्जन्यमिव भूतानि महाद्रुममिव द्विजाः}
{नरास्तमुपजीवन्ति नृपं सर्वार्थसाधकम्}


\threelineshloka
{न हि कामात्मना राज्ञा सततं शठबुद्धिना}
{नृशंनेनातिलुब्धेन शक्यं पालयितुं प्रजाः ॥युधिष्ठिर उवाच}
{}


\twolineshloka
{नाहं राज्यसुखान्वेषी राज्यमिच्छाम्यपि क्षणम्}
{धर्मार्थं रोचये राज्यं धर्मश्चात्र न विद्यते}


\twolineshloka
{तदल मम राज्येन यत्र धर्मो न विद्यते}
{वनमेव गमिष्यामि तस्माद्धर्मचिकीर्षया}


\threelineshloka
{तत्र मेध्येष्वरण्येषु न्यस्तदण्डो जितेन्द्रियः}
{धर्ममाराधयिष्यामि मुनिर्मूलफलाशनः ॥भीष्म उवाच}
{}


\twolineshloka
{वेदाहं तव या बुद्धिरानृशंस्येऽगुणैव सा}
{न च नित्यानृशंसेन शक्यं राज्यमुपासितुम्}


\twolineshloka
{सदैव त्वां मृदुप्रज्ञमत्यार्यमतिधार्मिकम्}
{क्लीबं धर्मघृणायुक्तं न लोको बहुमन्यते}


\twolineshloka
{राजधर्ममवेक्षस्व पितृपैतामहोचितम्}
{नैतद्राज्ञामथो वृत्तं यथा त्वं स्थातुमिच्छसि}


\twolineshloka
{न हि वैक्लव्यसंसृष्टमानृशंस्यमिहास्थितः}
{प्रजापालनसंभूतं प्राप्तो धर्मफलं ह्यसि}


\twolineshloka
{न ह्येतामाशिषं पाण्डुर्न च कुन्त्यभ्यभाषत}
{`विचित्रवीर्यो धर्मात्मा चित्रवीर्यो नराधिपः}


\twolineshloka
{शन्तनुश्च महीपालः सर्वक्षत्रस्य पूजितः}
{'तवैतां प्राज्ञतां तात यथा चरसि मेधया}


\twolineshloka
{शौर्यं बलं च सत्यं च पिता तव सदाऽब्रवीत्}
{महत्त्वं बलमौदार्यं भवतः कुन्त्ययाचत}


\twolineshloka
{नित्यं स्वाहा स्वधा नित्यं चोभे मानुषदैवते}
{पुत्रेष्वाशासते नित्यं पितरो दैवतानि च}


\twolineshloka
{दानमध्ययनं यज्ञं प्रजानां परिपालनम्}
{धर्म्यमेतदधर्म्यं वा जन्मनैवाभ्यजायथाः}


\twolineshloka
{कुले धुरि च युक्तानां वहतां भारमीदृशम्}
{सीदतामपि कौन्तेय कीर्तिर्न परिहीयते}


\twolineshloka
{समन्ततो विनीतो यो वहत्यस्खलितो हि सः}
{निर्दोषकर्मवचनात्सिद्धिः कर्मण एव सा}


\twolineshloka
{नैकान्ते विनिपातेऽपि विहरेदिह कश्चन}
{धर्मी गृही वा राजा वा ब्रह्मचार्यथवा द्विजः}


\twolineshloka
{अल्पं हि सारभूयिष्ठं यत्कर्मोदारमेव तत्}
{कृतमेवाकृताच्छ्रेयो न पापीयोऽस्य कर्मणः}


\twolineshloka
{यदा कुलीनो धर्मज्ञः प्राप्नोत्यैश्वर्यमुत्तमम्}
{योगक्षेमस्तदा राज्ञः कुशलायैव कल्पते}


\twolineshloka
{दानेनान्यं बलेनान्यमन्यं सूनृतया गिरा}
{सर्वतः प्रतिगृह्णीयाद्राज्यं प्राप्येह धार्मिकः}


\threelineshloka
{यं हि वैद्याः कुले जाता ह्यवृत्तिभयपीडिताः}
{प्राप्य तृप्ताः प्रतिष्ठन्ति धर्मः कोऽभ्यधिकस्ततः ॥युधिष्ठिर उवाच}
{}


\threelineshloka
{किं न्वतः परमं स्वर्ग्यं का ततः प्रीतिरुत्तमा}
{किं ततः परमैश्वर्यं ब्रूहि मे यदि पश्यसि ॥भीष्म उवाच}
{}


\twolineshloka
{यस्मिन्भयार्दिताः सन्तः क्षेमं विन्दन्त्यपि क्षणम्}
{स स्वर्गजित्तमोऽस्माकं सत्यमेतद्ब्रवीमि ते}


\twolineshloka
{त्वमेव प्रीतिमांस्तस्मात्कुरूणां कुरुसत्तम}
{भव राजा जय स्वर्गं सतो रक्षाऽसतो जहि}


\twolineshloka
{अनु त्वां तात जीवन्तु सुहृदः साधुभिः सह}
{पर्जन्यमिव भूतानि स्वादुद्रुममिव द्विजाः}


\twolineshloka
{धृष्टं शूरं प्रहर्तारमनृशंसं जितेन्द्रियम्}
{वत्सलं संविभक्तारमुपजीवन्तु बान्धवाः}


\chapter{अध्यायः ७६}
\twolineshloka
{युधिष्ठिर उवाच}
{}


\threelineshloka
{स्वकर्मण्यपरे युक्तास्तथैवान्ये विकर्मणि}
{तेषां विशेषमाचक्ष्व ब्राह्मणानां पितामह ॥भीष्म उवाच}
{}


\twolineshloka
{विद्यालक्षणसंपन्नाः सर्वत्राम्नायदर्शिनः}
{एते ब्रह्मसमा राजन्ब्राह्मणाः परिकीर्तिताः}


\twolineshloka
{ऋत्विगाचार्यसंपन्नाः स्वेषु कर्मस्ववस्थिताः}
{एते देवसमा राजन्ब्राह्मणानां भवन्त्युत}


\twolineshloka
{`गोऽजाविमहिषाणां च बडवानां च पोषकाः}
{वृत्त्यर्थं प्रतिपद्यन्ते तान्वैश्यान्संप्रचक्षते}


\twolineshloka
{ऐश्वर्यकामा ये चापि सामिपा वाऽपि भारत}
{निग्रहानुग्रहरतांस्तान्द्विजान्क्षत्रियान्विदुः}


\twolineshloka
{अश्वारोहा गजारोहा रथिनोऽथ पदातयः}
{एते वैश्यसमा राजन्ब्राह्मणानां भवन्त्युत}


\twolineshloka
{जन्मकर्मविहीना ये कदर्या ब्रह्मबन्धवः}
{एते शूद्रसमा राजन्ब्राह्मणानां भवन्त्युत}


\twolineshloka
{अश्रोत्रियाः सर्वे एते सर्वे चानाहिताग्नयः}
{तान्सर्वान्धार्मिको राजा बलिं विष्टिं च कारयेत्}


\twolineshloka
{आह्वायका देवलका नाक्षत्रा ग्रामयाजकाः}
{एते ब्राह्मणचाण्डाला महापथिकपञ्चमाः}


\twolineshloka
{[ऋत्विक्पुरोहितो मन्त्री दूतो वार्तानुकर्षकः}
{एते क्षत्रसमा राजन्ब्राह्मणानां भवन्त्युत ॥]}


\twolineshloka
{`म्लेच्छदेशाश्च ये केचित्पापैरध्युषिता नरैः}
{गत्वा तु ब्राह्मणस्तांश्च चण्डालः प्रेत्य चेह च}


\twolineshloka
{व्रात्यान्म्लेच्छांश्च शूद्रांश्च याजयित्वा द्विजाधमः}
{अकीर्तिमिह संप्राप्य नरकं प्रतिपद्यते}


\twolineshloka
{महावृन्दसमुद्राभ्यां पर्यायेणैकविंशतिम्}
{ब्राह्मणो ऋग्यजुः साम्नां मूढः कृत्वा तु विप्लवम्}


\twolineshloka
{कल्पमेकं कृमिस्थोऽथ नानाविष्ठासु जायते}
{व्रात्ये म्लेच्छे तथा शूद्रे तस्करे पत्तितेऽशुचौ}


\twolineshloka
{कुदेशे च सुरापे च ब्रह्मघ्ने वृषलीपतौ}
{अनधीतेषु सर्वत्र भुञ्जाने यत्र तत्र वा}


\twolineshloka
{वालस्त्रीवृद्धहन्तुश्च मातापित्रोर्गुरोस्तथा}
{मित्रद्रुहि कृतघ्ने च गोघ्ने चैव कथंचन}


\twolineshloka
{पुत्रघातिनि शत्रौ च न मन्त्राद्याजयेद्द्विजः}
{स तेषां विप्लवः प्रोक्तो मन्त्रविद्भिः सनातनैः}


\twolineshloka
{यदि विप्रो विदेशस्थस्तीर्थयात्रां गतोऽपि वा}
{यदि भीतः प्रपन्नो वा कुदेशं शौचवर्जितम्}


\threelineshloka
{सुसयतः शुचिर्भुत्वा मन्त्रानुच्चारयेद्द्विजः}
{आर्तश्चोच्चारयेन्मन्त्रमार्तत्राणपरोऽथवा}
{हीनेष्वपि प्रयुञ्जानो नासौ विप्लावकः स्मृतः}


\twolineshloka
{क्रूरकर्मा विकर्मा वा कर्मभिर्वञ्चितोऽथवा}
{तत्त्ववित्तरते पापं शीलवान्सयतेन्द्रियः ॥ '}


\twolineshloka
{एतेभ्यो बलिमादद्याद्धीनकोशो महीपतिः}
{ऋते ब्रह्मसमेभ्यश्च देवकल्पेभ्य एव च}


\threelineshloka
{अब्राह्मणानां वित्तस्य स्वामी राजेति नः श्रुतिः}
{ब्राह्मणानां च येकेचिद्विकर्मस्था इति श्रुतिः}
{`प्रागुक्तांश्चाप्यनुक्तांश्च सर्वास्तान्दापयेत्करान्'}


\twolineshloka
{विकर्मस्थाश्च नोपेक्ष्या विप्रा राज्ञा कथंचन}
{नियम्याः संविभज्याश्च धर्मानुग्रहकाम्यया}


\twolineshloka
{यस्य स्म विषये राज्ञः स्तेनो भवति वै द्विजः}
{राज्ञ एवापराधं तं मन्यन्ते तद्विदो जनाः}


\twolineshloka
{अवृत्त्या यो भवेत्स्तेनो वेदवित्स्नातकस्तथा}
{राजन्स राज्ञा भर्तव्य इति वेदविदो विदुः}


\twolineshloka
{स चेन्नापि निवर्तेत कृतवृत्तिः परन्तप}
{ततो निर्वासनीयः स्यात्तस्माद्देशात्सबान्धवः}


\twolineshloka
{`यज्ञः श्रुतमपैशुन्यमर्हिसाऽतिथिपूजनम्}
{दमः सत्यं तपो दानमेतद्ब्राह्मणलक्षणम् ॥'}


\chapter{अध्यायः ७७}
\twolineshloka
{युधिष्ठिर उवाच}
{}


\threelineshloka
{केषां प्रभवते राजा वित्तस्य भरतर्षभ}
{कया च वृत्त्या वर्तेत तन्मे ब्रूहि पितामह ॥भीष्म उवाच}
{}


\twolineshloka
{अब्राह्मणानां वित्तस्य स्वामी राजेति वैदिकम्}
{ब्राह्मणानां च ये केचिद्विकर्मस्था भवन्त्युत}


\twolineshloka
{विकर्मस्थाश्च नोपेक्ष्या विप्रा राज्ञा कथंचन}
{इति राज्ञां पुरावृत्तमभिजल्पन्ति साधवः}


\twolineshloka
{यस्य स्म विषये राज्ञः स्तेनो भवति वै द्विजः}
{राज्ञ एवापराधं तं मन्यन्ते किल्विषं नृप}


\twolineshloka
{अभिशस्तमिवात्मानं मन्यन्ते तेन कर्मणा}
{तस्माद्राजर्षयः सर्वे ब्राह्मणानन्वपालयन्}


\twolineshloka
{अत्राप्युदाहरन्तीममितिहासं पुरातनम्}
{गीतं केकयराजेन हियमाणेन रक्षसा}


\threelineshloka
{केकयानामधिपतिं रक्षो जग्राह दारुणम्}
{स्वाध्यायेनान्वितं राजन्नरण्ये संशितव्रतम् ॥राजोवाच}
{}


\twolineshloka
{न मे स्तेनो जनपदे न कदर्यो न मद्यपः}
{नानाहिताग्निर्नायज्वा मा ममान्तरमाविशः}


\twolineshloka
{न च मे ब्राह्मणोऽविद्वान्नाव्रती नाप्यसोमपः}
{द्विजातिर्विषये मह्यं मा ममान्तरमाविशः}


\twolineshloka
{नानाप्तदक्षिणैर्यज्ञैर्यजन्ते विषये मम}
{नाधीते चाव्रती कश्चिन्मा ममान्तरमाविशः}


\twolineshloka
{अध्यापयन्त्यधीयन्ते यजन्ते याजयन्ति च}
{ददति प्रतिगृह्णन्ति षट््सु कर्मस्ववस्थिताः}


\twolineshloka
{पूजिताः संविभक्ताश्च मृदवः सत्यवादिनः}
{ब्राह्मणा मे स्वकर्मस्था मा ममान्तरमाविशः}


\twolineshloka
{न याचन्ते प्रयच्छन्ति सत्यधर्मविशारदाः}
{नाध्यापयन्त्यधीयन्ते यजन्ते याजयन्ति न}


\twolineshloka
{ब्राह्मणान्परिरक्षन्ति सङ्ग्रामेष्वपलायिनः}
{क्षत्रिया मे स्वकर्मस्था मा ममान्तरमाविशः}


\twolineshloka
{कृषिगोरक्षवाणिज्यमुपजीवन्त्यमायया}
{अप्रमत्ताः क्रियावन्तः सुवृत्ताः सत्यवादिनः}


\twolineshloka
{संविभागं दमं शौचं सौहृदं च व्यपाश्रिताः}
{मम वैश्याः स्वकर्मस्था मा ममान्तरमाविशः}


\twolineshloka
{त्रीन्वर्णानुपतिष्ठन्ते यथावदनसूयकाः}
{मम शूद्राः स्वकर्मस्था मा ममान्तरमाविशः}


\twolineshloka
{कृपणानाथवृद्धानां दुर्बलातुरयोषिताम्}
{संविभक्ताऽस्मि सर्वेषां मा ममान्तरमाविशः}


\twolineshloka
{कुलानुरूपधर्माणां प्रस्थितानां यथाविधि}
{अव्युच्छेत्ताऽस्मि सर्वेषां मा ममान्तरमाविशः}


\twolineshloka
{तपस्विनो मे विषये पूजिताः परिपालिताः}
{संविभक्ताश्च सत्कृत्य मा ममान्तरमाविशः}


\twolineshloka
{नासंविभज्य भोक्ताऽस्मि न विशामि परस्त्रियम्}
{स्वतन्त्रो जातु न क्रीडे मा ममान्तरमाविशः}


\twolineshloka
{नाब्रह्मचारी भिक्षावान्भिक्षुर्वा ब्रह्मचर्यवान्}
{अनृत्विजा हुतं नास्ति मा ममान्तरमाविशः}


\twolineshloka
{`कृतं राज्यं मया सर्वं राज्यस्थेनापि कार्यवत्}
{नाहं व्युत्क्रामितः सत्यान्मा ममान्तरमाविशः ॥ '}


\twolineshloka
{नावजानाम्यहं वैद्यान्न वृद्धान्न तपस्विनः}
{राष्ट्रे स्वपति जागर्मि मा ममान्तरमाविशः}


\twolineshloka
{`शुक्लकर्मास्मि सर्वत्र न दुर्गतिभयं मम}
{धर्मचारी गृहस्थश्च मा ममान्तरमाविशः ॥'}


\twolineshloka
{वेदाध्ययनसंपन्नस्तपस्वी सत्यधर्मवित्}
{स्वामी सर्वस्य राष्ट्रस्य धीमान्मम पुरोहितः}


\twolineshloka
{दानेन दिव्यानभिवाञ्छामि लोकान्सत्येनाथ ब्राह्मणानां च गुप्त्या}
{शुश्रूषया चापि गुरूनुपैमिन मे भयं विद्यते राक्षसेभ्यः}


\twolineshloka
{न मे राष्ट्रे विधवा ब्रह्मबन्धुर्न ब्राह्मणः कितवो नोत चोरः}
{नायाज्ययाजी न च पापकर्मान मे भयं विद्यते राक्षसेभ्यः}


\twolineshloka
{न मे शस्त्रैरनिर्भिन्नं गात्रे व्द्यङ्गुलमन्तरम्}
{धर्मार्थं युध्यमानस्य मा ममान्तरमाविशः}


\threelineshloka
{गोब्राह्मणेभ्यो यज्ञेभ्यो नित्यं स्वस्त्ययनं मम}
{आशासते जना राष्ट्रे मा ममान्तरमाविशः ॥राक्षस उवाच}
{}


\twolineshloka
{`नारीणां व्यभिचाराच्च अन्यायाच्च महीक्षिताम्}
{विप्राणां कर्मदोषाच्च प्रजानां जायते भयम्}


\twolineshloka
{अवृष्टिर्मारको दोषः सततं क्षुद्भयानि च}
{विग्रहश्च सदा तस्मिन्देशे भवति दारुणः}


\twolineshloka
{यक्षरक्षःपिशाचेभ्यो नासुरेभ्यः कथंचन}
{भयमुत्पद्यते तत्र यत्र विप्राः सुसंयताः}


\twolineshloka
{गन्धर्वाप्सरसः सिद्धाः पन्नगाश्च सरीसृपाः}
{मानवान्न जिघांसन्ति यत्र नार्यः पतिव्रताः}


\twolineshloka
{ब्राह्मणः क्षत्रिया वैश्या यत्र शूद्राश्च धार्मिकाः}
{नाऽनावृष्टिभयं तत्र न दुर्भिक्षं न विभ्रमः}


\twolineshloka
{धार्मिको यत्र भूपालो न तत्रास्ति पराभवः}
{उत्पाता न च दृश्यन्ते न दिव्या न च मानुषाः}


\twolineshloka
{यस्मात्सर्वास्ववस्थासु धर्ममेवान्ववेक्षसे}
{तस्मात्प्राप्नुहि कैकेय गृहं स्वस्ति व्रजाम्यहम्}


\twolineshloka
{येषां गोब्राह्मणा रक्ष्याः प्रजा रक्ष्याश्च केकय}
{न रक्षोऽभ्यो भयं तेषां कुत एव तु पातकम्}


\threelineshloka
{येषां पुरोगमा विप्रा येषां ब्रह्म परं बलम्}
{सुरक्षितास्तथा विप्रास्ते वै स्वर्गजितो नृपाः ॥भीष्म उवाच}
{}


\twolineshloka
{तस्माद्द्विजातीन्रक्षेत ते हि रक्षन्ति रक्षिताः}
{आशीरेषां भवेद्राजन्राज्ञां सम्यक्प्रवर्तताम्}


\twolineshloka
{तस्माद्राज्ञा विशेषेण विकर्मस्था द्विजातयः}
{नियम्याः संविभज्याश्च प्रजानुग्रहकारणात्}


\twolineshloka
{एवं यो वर्तते राजा पौरजानपदेष्विह}
{अनुभूयेह भद्राणि प्राप्नोतीन्द्रसलोकताम्}


\chapter{अध्यायः ७८}
\twolineshloka
{युधिष्ठिर उवाच}
{}


\threelineshloka
{व्याख्याता राजधर्मेण वृत्तिरापत्सु भारत}
{कथंचिद्वैश्यधर्मेण जीवेद्वा ब्राह्मणो न वा ॥भीष्म उवाच}
{}


\threelineshloka
{अशक्तः क्षत्रधर्मेण वैश्यधर्मेण वर्तयेत्}
{कृषिं गोरक्ष्यमास्थाय व्यसने वृत्तिसंक्षये ॥युधिष्ठिर उवाच}
{}


\threelineshloka
{कानि पण्यानि विक्रीणन्स्वर्गलोकान्न हीयते}
{ब्राह्मणो वैश्यधर्मेण वर्तयन्भरतर्षभ ॥भीष्म उवाच}
{}


\twolineshloka
{सुरालवणमित्येतत्तिलान्केसरिणः पशून्}
{वृषभान्मधु मांसं च कृतान्नं च युधिष्ठिर}


\twolineshloka
{सर्वास्ववस्थास्वेतानि ब्राह्मणः परिवर्जयेत्}
{एतेषां विक्रयात्तात ब्राह्मणो नरके पतेत्}


\twolineshloka
{अजोऽग्निर्वरुणो मेषः सूर्योऽश्वः पृथिवी विराट्}
{धेनुर्यज्ञश्च सोमश्च न विक्रेयाः कथंचन}


\twolineshloka
{पक्वेनामस्य निमयं न प्रशंसन्ति साधवः}
{निमयेत्पक्वमामेन भोजनार्थाय भारत}


\twolineshloka
{वयं सिद्धमशिष्यामो भवान्साधयतामिदम्}
{एवं संवीक्ष्य समयं नाधर्मोऽस्ति कथंचन}


\twolineshloka
{अत्र ते वर्तयिष्यामि यथा कर्मः सनातनः}
{व्यवहारप्रवृत्तानां तन्निबोध युधिष्ठिर}


\twolineshloka
{भवतेऽहं ददानीदं भवानेतत्प्रयच्छतु}
{उचितो वर्तते धर्मो न बलात्संप्रवर्तते}


\threelineshloka
{इत्येवं संप्रवर्तन्ते व्यवहाराः पुरातनः}
{ऋषीणामितरेषां च साधु चैतदसंशयम् ॥युधिष्ठिर उवाच}
{}


\twolineshloka
{अथ तात यदा सर्वाः शस्त्रमाददते प्रजाः}
{व्युत्क्रमन्ते स्वधर्मेभ्यः क्षत्रस्य क्षीयते बलम्}


\threelineshloka
{तदा त्राता तु को नु स्यात्को धर्मः किं परायणम्}
{एतं मे संशयं ब्रूहि विस्तरेण पितामह ॥भीष्म उवाच}
{}


\twolineshloka
{दानेन तपसा यज्ञैदद्रोहेण दमेन च}
{ब्राह्मणप्रमुखा वर्णाः क्षेममिच्छेयुरात्मनः}


\twolineshloka
{तेषां ये वेदबलिनस्त उत्थाय समन्ततः}
{राज्ञो बलं वर्धयेयुर्महेन्द्रस्येव देवताः}


\twolineshloka
{राज्ञो हि क्षीयमाणस्य ब्रह्मैवाहुः परायणम्}
{तस्माद्ब्राह्मबलेनैव समुत्थेयं विजानता}


\twolineshloka
{यदा तु विजयी राजा क्षेमं राष्ट्रेऽभिसन्दधेत्}
{तदा वर्णा यथाधर्मं निविशेयुः स्वकर्मसु}


\threelineshloka
{उन्मर्यादे प्रवृत्ते तु दस्युभिः संकरे कृते}
{सर्वे वर्णा न दुष्येयुः शस्त्रवन्तो युधिष्ठिर ॥युधिष्ठिर उवाच}
{}


\threelineshloka
{अथ चेत्सर्वतः क्षत्रं प्रदुष्येद्ब्राह्मणं प्रति}
{कस्तत्र ब्राह्मणांस्त्राता को धर्मः किं परायणम् ॥भीष्म उवाच}
{}


\twolineshloka
{तपसा ब्रह्मचर्येण शस्त्रेण च बलेन च}
{अमायया मायया च नियन्तव्यं तदा भवेत्}


\twolineshloka
{क्षत्रियस्यातिवृत्तस्य ब्राह्मणेषु विशेषतः}
{ब्रह्मैव संनियन्तृ स्यात्क्षत्रं हि ब्रह्मसंभवम्}


\twolineshloka
{अभ्द्योऽग्निर्ब्रह्मतः क्षत्रमश्मनो लोहमुत्थितम्}
{तेषां सर्वत्रगं तेजः स्वस्वयोनिषु शाम्यति}


\twolineshloka
{यदा छिनत्त्ययोऽश्मानमग्निश्चापोऽभिहन्ति च}
{क्षत्रं च ब्राह्मणं द्वेष्टि तदा शाम्यन्ति ते त्रयः}


\threelineshloka
{तस्माद्ब्रह्मणि शाम्यन्ति क्षत्रियाणां युधिष्ठिर}
{समुदीर्णान्यजेयानि तेजांसि च बलानि च ॥युधिष्ठिर उवाच}
{}


\twolineshloka
{ब्रह्मवीर्ये मृदूभूते क्षत्रवीर्ये च दुर्बले}
{दुष्टेषु सर्ववर्णेषु ब्राह्मणान्प्रति भारत}


\threelineshloka
{ये तत्र युद्धं कुर्वन्ति त्यक्त्वा जीवितमात्मनः}
{`ब्राह्मणान्परिरक्षन्ति तेषां लोका भवन्ति के ॥भीष्म उवाच}
{}


\threelineshloka
{ब्राह्मणान्परिरक्षन्तो धर्ममात्मानमेव च}
{मनस्विनो मन्युमन्तः पुण्याँल्लोकान्व्रजन्त्यमी}
{ब्राह्मणार्थं हि सर्वेषां शस्त्रग्रहणभिष्यते}


\twolineshloka
{अतिस्विष्टमधीतानां लोकानतितपस्विनाम्}
{अनाशकाग्न्याहितानां शूरा यान्ति परां गतिम्}


\twolineshloka
{ब्राह्मणस्त्रिषु वर्णेषु शस्त्रं गृह्णन्न दुष्यति}
{एष एवात्मनस्त्यागो नान्यं धर्मं विदुर्जना}


\twolineshloka
{तेभ्यो नमश्च भद्रं च ये शरीराणि जुह्वति}
{ब्रह्मद्विपो जिघांसन्तस्तेषां नोऽस्तु सलोकता}


\threelineshloka
{ब्रह्मलोकजितः स्वर्ग्यान्वीरांस्तान्मनुरव्रवीत्}
{यथाऽश्वमेधावभृथे स्नाताः पूता भवन्त्युत}
{दुष्कृतः सुकृतश्चैव तथा शस्त्रहता रणे}


\twolineshloka
{भवत्यधर्मो धर्मो हि धर्मोऽधर्मो भवत्युत}
{कारणाद्देशकालस्य देशः कालः स तादृशः}


\twolineshloka
{मैत्राः क्रूराणि कुर्वन्तो जयन्ति स्वर्गमुत्तमम्}
{धर्म्याः पापानि कुर्वाणा गच्छन्ति परमां गतिम्}


\threelineshloka
{ब्राह्मणस्त्रिषु कालेषु शस्त्रं गृह्णन्न दुष्यति}
{आत्मत्राणे दस्युदोषे सर्वस्वहरणे तथा ॥युधिष्ठिर उवाच}
{}


\twolineshloka
{अभ्युत्थिते दस्युबले क्षत्रार्थे वर्णसङ्करे}
{संप्रमूढेषु वर्णेषु यदन्योऽभिभवेद्वली}


\threelineshloka
{ब्राह्मणो यदि वा वैश्यः शूद्रो वा राजसत्तम}
{दस्युभ्यो यः प्रजा रक्षेद्दण्डं धर्मेण धारयेत् ॥भीष्म उवाच}
{}


\twolineshloka
{कार्यं कुर्यान्न वा कुर्यात्स वार्यो वा भवेन्न वा}
{न स्म शस्त्रं गृहीतव्यमन्यत्र क्षत्रबन्धुतः}


\twolineshloka
{अपारे यो भवेत्पारमप्लवेः यः प्लवो भवेत्}
{शूद्रो वा यदि वाऽप्यन्यः सर्वथा मानमर्हति}


\twolineshloka
{यमाश्रित्य नरा राजन्वर्तयेयुर्यथासुखम्}
{अनाथास्तप्यमानाश्च दस्युभिः परिपीडिताः}


\twolineshloka
{तमेव पूजयेयुस्ते प्रीत्या स्वमिव बान्धवम्}
{यहद्ध्यभीष्टं कौरव्य कर्ता सन्मानमर्हति}


\twolineshloka
{किमनडुहा यो न वहेत्किं धेन्वा वाऽप्यदुग्धया}
{बन्ध्यया भार्यया कोऽर्थः कोऽर्थो राज्ञाऽप्यरक्षता}


\twolineshloka
{यथा दारुमयो हस्ती यथा चर्ममयो मृगः}
{यथा ह्यदक्षः पुरुषः पथि क्षेत्रं यथोपरम्}


\twolineshloka
{यथा विप्रोऽनधीयानो राजा यश्च न रक्षिता}
{मेघो न वर्षते यश्च सर्व एव निरर्थकाः}


\twolineshloka
{नित्यं यस्तु सतो रक्षेदसतश्च निवर्तयेत्}
{स एव राजा कर्तव्यस्तेन सर्वमिदं धृतम्}


\chapter{अध्यायः ७९}
\twolineshloka
{युधिष्ठिर उवाच}
{}


\threelineshloka
{क्व समर्थाः कथंशीला ऋत्विजः स्युः पितामह}
{कथंविधाश्च राजेन्द्र तद्ब्रूहि वदतां वर ॥भीष्म उवाच}
{}


\twolineshloka
{प्रतिकर्मपरा राजन्वृत्तिरस्य विधीयते}
{छन्दः सामादि विज्ञाय द्विजानां श्रुतमेवच}


\twolineshloka
{ये त्वेकरतयो नित्यं धीराश्च प्रियवादिनः}
{परस्परस्य सुहृदः समन्तात्समदर्शिनः}


\twolineshloka
{आनृशंस्यं सत्यवाक्यमहिंसा दम आर्जवम्}
{अद्रोहोऽनभिमानश्च ह्रीस्तितिक्षा दमः शमः}


\threelineshloka
{`यस्मिन्नेतानि दृश्यन्ते स पुरोहित उच्यते}
{'धीमान्सत्यधृतिर्दान्तो भूतानामविहिंसकः}
{अकामद्वेषसंयुक्तस्त्रिभिः शुक्लैः समन्वितः}


\threelineshloka
{अहिंसको ज्ञानतृप्तः स ब्रह्मासनमर्हति}
{एते महर्त्विजस्तात सर्वे मान्या यथार्हतः ॥युधिष्ठिर उवाच}
{}


\twolineshloka
{यदिदं वेदवचनं दक्षिणासु विधीयते}
{इदं देयमिदं देयं न क्वचिव्द्यवतिष्ठते}


\twolineshloka
{देयं प्रतिधनं शास्त्रमापद्धर्मा न शास्त्रतः}
{आज्ञा शास्त्रस्य घोरे यं न शक्तिं समवेक्षते}


\threelineshloka
{श्रद्धामालम्ब्य यष्टव्यमित्येषा वैदिकी श्रुतिः}
{मिथ्योपेतस्य यज्ञस्य किमु श्रद्धा करिष्यति ॥भीष्म उवाच}
{}


\twolineshloka
{न वेदानां परिभवान्न शाठ्येन न मायया}
{कश्चिन्महदवाप्नोति मा ते भूद्बुद्धिरीदृशी}


\twolineshloka
{यज्ञाङ्गं दक्षिणा तात मन्त्राणां परिबृंहणम्}
{न मन्त्रा दक्षिणाहीनास्तारयन्ति कथंचन}


\twolineshloka
{शक्तिस्तु पूर्णपात्रेण संमिता नावमा भवेत्}
{अवश्यं तात यष्टव्यं त्रिभिर्वर्णैर्थथाबलम्}


\twolineshloka
{सोमो राजा ब्राह्मणानामित्येषा वैदिकी श्रुतिः}
{तं च विक्रेतुमिच्छन्ति न तथा वृत्तिरिष्यते}


\twolineshloka
{तेन क्रीतेन धर्मेण ततो यज्ञः प्रतायते}
{इत्येवं धर्ममाख्यातमृषिभिर्धर्मकोविदैः}


\twolineshloka
{पुमान्यज्ञश्च सोमश्च न्यायवृत्तो यदा भवेत्}
{अन्यायवृत्तः पुरुषो न परस्य न चात्मनः}


\twolineshloka
{शरीरं यज्ञपात्राणि इत्येषा श्रूयते श्रुतिः}
{तानि सम्यक्प्रणीतानि ब्राह्मणानां महात्मनाम्}


\twolineshloka
{तपो यज्ञादपि श्रेष्ठमित्येषा परमा श्रुतिः}
{तत्ते तपः प्रवक्ष्यामि विद्वंस्तदपि मे शृणु}


\twolineshloka
{अहिंसा सत्यवचनमानृशंस्यं दमो घृणा}
{एतत्तपो विदुर्धीरा न शरीरस्य शोषणम्}


\twolineshloka
{अप्रामाण्यं च वेदानां शास्त्राणां चातिलङ्घनम्}
{अव्यवस्था च सर्वत्र तद्वै नाशनमात्मनः}


\threelineshloka
{निबोध दशहोतॄणां विधानं पार्थ यादृशम्}
{चित्तिः स्रुक् चित्तमाज्यं च पवित्रं ज्ञानमुत्तमम्}
{`न शाठ्यं न च जिह्यत्वं कालो देशश्च ते दश ॥'}


\twolineshloka
{सर्वं दिह्नं मृत्युपदमार्जवं ब्रह्मणः पदम्}
{एतावाञ्ज्ञानविषयः किं प्रलापः करिष्यति}


\chapter{अध्यायः ८०}
\twolineshloka
{युधिष्ठिर उवाच}
{}


\twolineshloka
{यदप्यल्पतरं कर्म तदप्येकेन दुष्करम्}
{पुरुषेणासहायेन किमु राज्यं पितामह}


\threelineshloka
{किंशीलः किंसमाचारो राज्ञो यः सचिवो भवेत्}
{कीदृशे विश्वसेद्राजा कीदृशे न च विश्वसेत् ॥भीष्म उवाच}
{}


\twolineshloka
{चतुर्विधानि मित्राणि राज्ञां राजन्भवन्त्युत}
{सहार्थो भजः निश्च सहजः कृत्रिमस्तथा}


\twolineshloka
{धर्मात्मा पञ्चमं मित्रं स तु नैकस्य न द्वयोः}
{यतो धर्मस्ततो वा स्यान्मध्यस्थो वा ततो भवेत्}


\threelineshloka
{यो यस्यार्थो न रोचेत न तं तस्य प्रकाशयेत}
{`मित्राणां प्रकृतिर्नास्ति त्वमित्राणां च भारत}
{उपकाराद्भवेन्मित्रमपकाराद्भवेदरिः}


\twolineshloka
{यस्यैव हि मनुष्यस्य नरो मरणमृच्छति}
{तस्य पर्यागते काले पुनर्जीवितुमिच्छति}


\threelineshloka
{धर्माधर्मेण राजानश्चरन्ति विजिगीषवः}
{चतुर्णां मध्यमौ श्रेष्ठौ नित्यं शङ्क्यौ तथाऽपरौ}
{सर्वे नित्यं शङ्कितव्याः प्रत्यक्षं कार्यमात्मनः}


\twolineshloka
{न हि राज्ञा प्रमादो वै कर्तव्यो मित्ररक्षणे}
{प्रमादिनं हि राजानं लोकाः परिभवन्त्युत}


\twolineshloka
{असाधुः साधुतामेति साधुर्भवति दारुणः}
{अरिश्च मित्रं भवति मित्रं चापि प्रदुष्यति}


\twolineshloka
{अनित्यचित्तः पुरुषस्तस्मिन्को जातु विश्वसेत्}
{तस्मात्प्रधानं यत्कार्यं प्रत्यक्षं तत्समाचरेत्}


\twolineshloka
{एकान्तेन हि विश्वासः कृत्स्नो कर्मार्थनाशकः}
{अविश्वासश्च सर्वत्र मृत्युर्नापि विशिष्यते}


\twolineshloka
{अकालमृत्युर्विश्वासोऽविश्वसन्हि विपद्यते}
{यस्मिन्करोति विश्वासमिच्छतस्तस्य जीवति}


\twolineshloka
{तस्माद्विश्वसितव्यं च शङ्कितव्यं च केषुचित्}
{एषा नीतिगतिस्तात लक्ष्मीश्चैषा सनातनी}


\twolineshloka
{यं मन्येत ममाभावादिममर्थागमः स्पृशेत्}
{नित्यं तस्माच्छङ्कितव्यममित्रं तं विदुर्बुधाः}


\twolineshloka
{यस्य क्षेत्रादप्युदकं क्षेत्रमन्यस्य गच्छति}
{न तत्रानिच्छतस्तस्य भिद्येरन्सर्वसेतवः}


\twolineshloka
{तथैवात्युदकाद्भीतस्तस्य भेदनमिच्छति}
{यमेवंलक्षणं विद्यात्तममित्रं विदुर्बुधाः}


\twolineshloka
{यस्तु वृद्ध्या न तप्येत क्षये दीनतरो भवेत्}
{एतदुत्तममित्रस्य निमित्तमभिचक्षते}


\twolineshloka
{यन्मन्येत ममाभावादस्याभावो भवेदिति}
{तस्मिन्कुर्वीत विश्वासं यथा पितरी वै तथा}


\twolineshloka
{तं शक्त्या वर्तमानं च सर्वतः परिबृंहयेत्}
{नित्यं क्षताद्वारयति यो धर्मेष्वपि कर्मसु}


\twolineshloka
{क्षताद्भीतं विजानीयादुत्तमं मित्रलक्षणम्}
{ये यस्य क्षयमिच्छन्ति ते तस्य रिपवः स्मृताः}


\twolineshloka
{व्यसनान्नित्यभीतो यः समृद्ध्या यो न दुष्यति}
{यत्स्यादेवंविधं मित्रं तदात्मसममुच्यते}


\twolineshloka
{रूपवर्णस्वरोपेतस्तितिक्षुरनसूयकः}
{कुलीनः शीलसंपन्नः स ते स्यात्प्रत्यनन्तरः}


\twolineshloka
{मेधावी स्मृतिमान्दक्षः प्रकृत्या चानृशंस्यवान्}
{यो मानितोऽमानितो वा न सन्तुष्येत्कथंचन}


\twolineshloka
{ऋत्विग्वा यदि वाऽऽचार्यः सखा वाऽत्यंतसत्कृतः}
{गृहे वसेदमात्यस्ते स स्यात्परमपूजितः}


\twolineshloka
{संविद्याः परमं मित्रं प्रकृतिं चार्थधर्मयोः}
{विश्वासस्ते भवेत्तत्र यथा पितरि वै तथा}


\twolineshloka
{नैव द्वौ न त्रयः कार्या न मृष्येरन्परस्परम्}
{एकार्थे हेतुभूतानां भेदो भवति सर्वदा}


\twolineshloka
{कीर्तिप्रधानो यस्त स्याद्यश्च स्यात्समये स्थितः}
{समर्थान्यश्च न द्वेष्टि नानर्थान्कुरुते च यः}


\twolineshloka
{यो न कामाद्भयाल्लोभात्क्रोधाद्वा धर्ममुत्सृजेत्}
{दक्षः पर्याप्तवचनः स ते स्यात्प्रत्यनन्तरः}


\twolineshloka
{कुलीनः शीलसंपन्नस्तितिक्षुरविकत्थनः}
{शूरश्चार्यश्च विद्वांश्च प्रतिपत्तिविशारदः}


\twolineshloka
{एते ह्यमात्याः कर्तव्याः सर्वकर्मस्ववस्थिताः}
{पूजिताः संबिभक्ताश्च सुसहायाः स्वनुष्ठिताः}


\twolineshloka
{कृत्स्नप्रेते विनिक्षिप्ताः प्रतिरूपेषु कर्मसु}
{युक्ता महत्सु कार्येषु श्रेयांस्युत्पादयन्त्युत}


\twolineshloka
{एते कर्माणि कुर्वन्ति स्पर्धमाना मिथः सदा}
{अनुतिष्ठन्ति चैवार्थमाचक्षाणाः परस्परम्}


\twolineshloka
{ज्ञातिभ्यो बिभियाश्चैव मृत्योरिव यतस्तदा}
{उपराजेव राजर्धि ज्ञातिर्न सहते सदा}


\twolineshloka
{ऋजोर्मृदोर्वदान्यस्य ह्रीमतः सत्यवादिनः}
{नान्यो ज्ञातेर्महाबाहो विनाशमभिनन्दति}


\twolineshloka
{अज्ञातयोऽप्यसुखदा ज्ञातयोऽपि सुखावहाः}
{अज्ञातिमन्तं पुरुषं परे चाभिभवन्त्युत}


\twolineshloka
{निकृतस्य नरैरन्यैर्ज्ञातिरेव परायणम्}
{नान्यो निकारं सहते ज्ञातिर्ज्ञातेः कदाचन}


\twolineshloka
{आत्मानमेव जानाति निकृतं बान्धवैपरि}
{तेषु सन्ति गुणाश्चैव नैर्गुण्यं चैव लक्ष्यते}


\twolineshloka
{नाज्ञातिरनुगृह्णाति नाज्ञातिर्वृद्धिमश्नुते}
{उभयं ज्ञातिवर्गेषु दृश्यते साध्वसाधु च}


\twolineshloka
{संमानयेत्पूजयेच्च वाचा नित्यं च कर्मणा}
{कुर्याच्च प्रियमेतेभ्यो नाप्रियं किंचिद चरेत्}


% Check verse!
विश्वस्तवदविश्वस्तस्तेषु वर्तेत सर्वदान हि दोषो गुणो वेति निरूप्यस्तेषु दृश्यते
\twolineshloka
{अस्यैवं वर्तमानस्य पुरुषस्याप्रमादिनः}
{अमित्राः संप्रसीदन्ति ततो मित्रं भवन्त्यपि}


\twolineshloka
{य एवं वर्तते नित्यं ज्ञातिसंबन्धिमण्डले}
{मित्रेष्वमित्रे मध्यस्थे चिरं यशसि तिष्ठति}


\chapter{अध्यायः ८१}
\twolineshloka
{युधिष्ठिर उवाच}
{}


\threelineshloka
{एवमग्राह्यके तस्मिञ्ज्ञातिसंबन्धिमण्डले}
{मित्रेष्वमित्रेष्वपि च कथं भावो विभाव्यते ॥भीष्म उवाच}
{}


\threelineshloka
{अत्राप्युदाहरन्तीममितिहासं पुरातनम्}
{संवादं वासुदेवस्य महर्षेर्नारदस्य च ॥वासुदेव उवाच}
{}


\twolineshloka
{नासुहृत्परर्म मन्त्रं नारदार्हति वेदितुम्}
{अपण्डितो वाऽपि सुहृत्पण्डितो वाप्यनात्मवान्}


\twolineshloka
{स ते सौहृदमास्थाय किंचिद्वक्ष्यामि नारद}
{कृत्स्नां बुद्धिं च ते प्रेक्ष्य संपृच्छे त्रिदिवङ्गम}


\twolineshloka
{दास्यमैश्वर्यवादेन ज्ञातीनां वै करोम्यहम्}
{अर्धंभोक्ताऽस्मि भोगानां वाग्दुरुक्तानि च क्षमे}


\twolineshloka
{अरणीमग्निकामो वा मथ्नाति दहृयं मम}
{वाचा दुरुक्तं देवर्षे तन्मां दहति नित्यदा}


\twolineshloka
{बलं सङ्कर्षणे नित्यं सौकुमार्यं पुनर्गदे}
{रूपेण मत्तः प्रद्युम्नः सोऽसहायोऽस्मि नारद}


\twolineshloka
{अन्ये हि सुमहाभागा बलवन्तो दुरासदाः}
{नित्योत्थानेन संपन्ना नारदान्धकवृष्णयः}


\twolineshloka
{यस्य न स्युर्न वै स स्याद्यस्य स्युः कृत्स्नमेव तत्}
{द्वयोरेनं प्रचरतोर्वृणोम्येकरतं न च}


\twolineshloka
{स्यातां यस्याहुकाक्रूरौ किं नु दुःखतरं ततः}
{यस्य चापि न तौ स्यातां किं नु दुःखतरं ततः}


\twolineshloka
{सोऽहं कित्नवमातेव द्वयोरपि महामुने}
{नैकस्य जयमाशंसे द्वितीयस्य पराजयम्}


\threelineshloka
{ममैवं क्लिश्यमानस्य नारदोभयदर्शनात्}
{वक्तुमर्हसि यच्छ्रेयो ज्ञातीनामात्मनस्तथा ॥नारद उवाच}
{}


\twolineshloka
{आपदो द्विविधाः कृष्ण बाह्याश्चाम्यन्तराश्च ह}
{प्रादुर्भवन्ति वार्ष्णेय स्वकृता यदि वाऽन्यतः}


\twolineshloka
{सेयमाभ्यन्तरा तुभ्यमापत्कृच्छ्रा स्वकर्मजा}
{अक्रूरभोजप्रभवा सर्वे ह्येते तदन्वयः}


\twolineshloka
{अर्थहेतोर्हि कामाद्वा वीरबीभत्सयाऽपि वा}
{आत्मना प्राप्तमैश्वर्यमन्यत्र प्रतिपादितम्}


\twolineshloka
{कृतमूलमिदानीं तद्राजशब्दसहायवत्}
{न शक्यं पुनरादातुं वान्तमन्नमिव स्वयम्}


\twolineshloka
{बभ्रूग्रसेनतो राज्यं नाप्नुं शक्यं कथंचन}
{ज्ञातिभेदभयात्कृष्ण त्वया चापि विशेषतः}


\twolineshloka
{तच्च सिध्येत्प्रयत्नेन कृत्वा कर्म सुदुष्करम्}
{महाक्षयं व्ययो वा स्याद्विनाशो वा पुनर्भवेत्}


\threelineshloka
{अनायसेन शस्त्रेण मृदुना हृदयच्छिदा}
{जिह्वामुद्धर सर्वेषां परिमृदज्यानुमृज्य च ॥वासुदेव उवाच}
{}


\threelineshloka
{अनायसं मुने शस्त्रं मृदु विद्यामहं कथम्}
{येनैषामुद्धरे जिह्वां परिमृज्यानुमृज्य च ॥नारद उवाच}
{}


\twolineshloka
{शक्त्याऽन्नदानं सततं तितिक्षाऽऽर्जवमार्दवम्}
{यथार्हप्रतिपूजा च शस्त्रमेतदनायसम्}


\twolineshloka
{ज्ञातीनां वक्तुकामानां कटुकानि लधूनि च}
{गिरा त्वं हृदयं वाचं शमयस्य मनांसि च}


\twolineshloka
{नामहापुरुषः कश्चिन्नानात्मा नासहायवान्}
{महतीं धुरमादाय समुद्यम्योरसा वहेत्}


\twolineshloka
{सर्व एव गुरुं भारमनङ्वान्वहते समे}
{दुर्गे प्रतीतः सुगवो भारं वहति दुर्वहम्}


\twolineshloka
{भेदाद्विनाशः सङ्घानां सङ्घमुख्योऽसि केशव}
{यथा त्वां प्राप्य नोत्सीदेदयं सङ्घस्तथा कुरु}


\twolineshloka
{नान्यत्र बुद्धिक्षान्तिभ्यां नान्यत्रेन्द्रियनिग्रहात्}
{नान्यत्र धनसन्त्यागाद्गुणः प्राज्ञेऽवतिष्ठते}


\twolineshloka
{धन्यं यशस्यमायुष्वं स्वपक्षोद्भावनं सदा}
{ज्ञातीनामविनाशः स्याद्यथा कृष्ण तथा कुरु}


\twolineshloka
{आयत्यां च तदात्वे च न तेऽस्त्यविदितं प्रभो}
{षाङ्गुण्यस्य विधानेन यात्रा यानविधौ तथा}


\twolineshloka
{यादवाः कुकुरा भोजाः सर्वे चान्धकवृष्णयः}
{त्वय्यायत्ता महाबाहो लोका लोकेश्वराश्च ये}


\twolineshloka
{उपासन्ते हि त्वद्बुद्धिमृषयश्चापि माधव}
{त्वं गुरुः सर्वभूतानां जानीषे त्वं परां गतिम्}


% Check verse!
त्वामासाद्य यदुश्रेष्ठमेधन्ते वादवाः सुखम्
\chapter{अध्यायः ८२}
\twolineshloka
{भीष्म उवाच}
{}


\twolineshloka
{एषा प्रथमतो वृत्तिर्द्वितीयां शृणु भारत}
{यः कश्चिद्वेदयेदर्थं राज्ञा रक्ष्यः स मानवः}


\twolineshloka
{ह्रियमाणममात्येन भृत्यो वा यदि वा भृतः}
{यो राजकोशं नश्यन्तमाचक्षीत युधिष्ठिर}


\twolineshloka
{श्रोतव्यमस्य च रहो रक्ष्यश्चामात्यतो भवेत्}
{अमात्या ह्यपहर्तारो भूयिष्ठं घ्नन्ति भारत}


\twolineshloka
{राजकोशस्य गोप्तारं राजकोशविलोपकाः}
{समेत्य सर्वे बाधन्ते स विनश्यत्यरक्षितः}


\twolineshloka
{अत्राप्युदाहरन्तीममितिहासं पुरातनम्}
{मुनिः कालकवृक्षीयः कौसल्यं यदुवाच ह}


\twolineshloka
{कोसलानामाधिपत्यं संप्राप्तं क्षेमदर्शिनम्}
{मुनिः कालकवृक्षीय आजगोमेति नः श्रुतम्}


\twolineshloka
{स काकं पञ्जरे बद्ध्वा विषयं क्षेमदर्शिनः}
{सर्वं पर्यचरद्युक्तः प्रवृत्त्यर्थी पुनः पुनः}


\twolineshloka
{अधीये वायसीं विद्यां शंसन्ति मम वायसाः}
{अनागतमतीतं च यच्च संप्रति वर्तते}


\twolineshloka
{इति राष्ट्रे परिपतन्बहुभिः पुरुषैः सह}
{सर्वेषां राजयुक्तानां दुष्कृतं परिदृष्टवान्}


\twolineshloka
{स बुद्ध्वा तस्य राष्ट्रस्य व्यवसायं हि सर्वशः}
{राजयुक्तापचारांश्च सर्वान्बुद्ध्वा ततस्ततः}


\twolineshloka
{ततः स काकमादाय राजानं द्रष्टुमागमत्}
{सर्वज्ञोऽस्मीति वचनं ब्रुवाणः संशितव्रतः}


\twolineshloka
{स स्म कौसल्यमागम्य राजामात्यमलंकृतम्}
{प्राह काकस्य वचनादमुत्रेदं त्वया कृतम्}


\twolineshloka
{असौ चासौ च जानीते राजकेशस्त्वया हृतः}
{एवमाख्याति काकोऽयं तच्छीघ्रमनुगम्यताम्}


\twolineshloka
{तथाऽन्यानपि स प्राह राजकोशहरांस्तदा}
{न चास्य वचनं किंचिदनृतं श्रूयते क्वचित्}


\twolineshloka
{तेन विप्रकृताः सर्वे राजयुक्ताः कुरूद्वह}
{तमतिक्रम्य सुप्तं तु निशि काकमपोथयन्}


\twolineshloka
{वायसं तु विनिर्भिन्नं दृष्ट्वा वाणेन पञ्जरे}
{पूर्वाह्णे ब्राह्मणो वाक्यं क्षेमदर्शिनमब्रवीत्}


\twolineshloka
{राजंस्त्वामभयं याचे प्रभुं प्राणधनेश्वरम्}
{अनुज्ञातस्त्वया ब्रूयां वचनं भवतो हितम्}


\twolineshloka
{मित्रार्थमभिसंतप्तो भक्त्या सर्वात्मनाऽऽगतः}
{ह्रियन्ते हि महार्थाश्च पुरुषे विक्रमत्यपि}


\twolineshloka
{संबुबोधयिषुर्मित्रं सदश्वमिव सारथिः}
{अतिमन्युप्रसक्तो हि प्रसह्य हितकारणात्}


\twolineshloka
{तथाविधस्य सुहृदा क्षन्तव्यं संविजानता}
{ऐश्वर्यमिच्छता नित्यं पुरुषेण बुभूषता}


\twolineshloka
{तं राजा प्रत्युवाचेदं यत्किंचिन्मां भवान्वदेत्}
{कस्मादहं न क्षमेयमाकाङ्क्षन्नात्मनो हितम्}


\threelineshloka
{ब्राह्मण प्रतिजाने ते प्रब्रूहि यदिहेच्छसि}
{करिष्यामि हि ते वाक्यं यन्मां विप्र प्रवक्ष्यसि ॥मुनिरुवाच}
{}


\twolineshloka
{विद्वान्नयानपायांश्च भयाख्यातॄन्भयानि च}
{भक्त्या वृत्तिं समाख्यातुं भवतोऽन्तिकमागतः}


\twolineshloka
{प्रागेवोक्तं तु दोषोऽयमाचार्यैर्नृपसेवनम्}
{अगतेः कुगतिर्ह्येषा या राज्ञा सहजीविका}


\twolineshloka
{आशीविषैश्च तस्याहुः संगमं यस्य राजभिः}
{बहुमित्रांश्च राजानो बह्वभित्रास्तथैव च}


\twolineshloka
{तेभ्यः सर्वेभ्य एवाहुर्भयं राजोपजीविनाम्}
{तथाऽस्य राजतो राजन्मुहुर्तादागतं भयम्}


\twolineshloka
{नैकान्तेनाप्रमादो हि शक्यः कर्तुं महीपतौ}
{न तु प्रमादः कर्तव्यः कथंचिद्भूतिमिच्छता}


\twolineshloka
{प्रमादात्स्खलते बुद्धिः स्खलतो नास्ति जीवितम्}
{अग्निं दीप्तमिवासीदेद्राजानप्नुपशिक्षितः}


\twolineshloka
{आशीविषमिव क्रुद्धं प्रभुं प्राणधनेश्वरम्}
{यत्नेनोपचरेन्नित्यं नाहमस्मीति मानवः}


\twolineshloka
{दुर्व्याहृताच्छङ्कमानो दुःस्थिताद्दुरनुष्ठितात्}
{दुरासदाद्दुर्वृजिनादिङ्गिताद्ध्यायितादपि}


\twolineshloka
{देवतेव हि सर्वार्थान्कुर्याद्राजा प्रसादितः}
{वैश्वानर इव क्रुद्धः समूलमपि निर्दहेत्}


\twolineshloka
{इति राजन्यमः प्राह वर्तते च तथैव तत्}
{अथ भूयांसमेवार्थं करिष्यामि पुनः पुनः}


\twolineshloka
{ददात्यत्मद्विधोऽऽमात्यो बुद्धिसाहाय्यमापदि}
{वायसस्त्वेष मे राजन्नन्तकायाभिसंहितः}


\twolineshloka
{न च मेऽत्र भवान्गर्ह्यो न च येषां भवान्प्रियः}
{हिताहितांस्तु बुद्ध्येथा मापरोक्षमतिर्भव}


\twolineshloka
{ये त्वादानपरा एव वसन्ति भवतो गृहे}
{अभूतिकामा भूतानां तादृशैर्मेऽभिसंहितम्}


\twolineshloka
{यो वा भवद्विनाशेन राज्यमिच्छत्यनन्तरम्}
{आन्तरैराभेसंधाय राजन्सिद्ध्यति नान्यथा}


\twolineshloka
{तेषामहं भयाद्राजन्गमिष्याम्यन्यमाश्रमम्}
{तैर्हि मे संधितो बाणः काके निपतितः प्रभो}


\twolineshloka
{छझना मम काकश्च गमितो यमसादनम्}
{दृष्टं ह्येतन्मया राजंस्तपोदीर्घेन चक्षुषा}


\twolineshloka
{बहुनक्रझषग्राहां तिमिंगिलगणैर्युताम्}
{काकेन वालिशेनेमामतार्षमहमापगाम्}


\threelineshloka
{स्थाण्वश्मकण्टकवर्तीं सिंह व्याघ्रसमाकुलाम्}
{दुरासदां दुष्प्रसहां गुहां हैमवतीमिव}
{}


\twolineshloka
{अग्निना तामसं दुर्गं नौभिराप्यं च गम्यते}
{राजदुर्गावतरणे नोपायं पण्डिता विदुः}


\twolineshloka
{गहनं भवतो राज्यमन्धकारं तमोन्वितम्}
{नेह विश्वसितुं शक्यं भवताऽपि कुतो मया}


\twolineshloka
{अतो नायं शुभो वासस्तुल्ये सदसती इह}
{वधो ह्येवात्र सुकृते दुष्कृते न च संशयः}


\twolineshloka
{न्यायतो दुष्कृते घातः सुकृते न कथनम्}
{नेह युक्तं स्थिरं स्थातुं जवेनैवाव्रजेद्वुधः}


\twolineshloka
{सीता नाम नदी राजन्प्लवो यस्यां निमज्जति}
{तयोपमामिमां मन्ये वागुरां सर्वधातिनीम्}


\twolineshloka
{मधुप्रपातो हि भवान्भोजनं विषसंयुतम्}
{असतामिव ते भावो वर्तते न सतामिव}


% Check verse!
आशीविषैः परिवृतः कूपस्त्वमसि पार्थिव
\threelineshloka
{दुर्गतीर्था बृहत्कूला कावेरी चोरसंयुता}
{नदी मधुरपानीया यथा राजंस्तथा भवान्}
{श्वगृध्रगोमायुयुतो राजहंससमो ह्यसि}


\twolineshloka
{यथाऽऽश्रित्य महावृक्षं कक्षः संवर्धते महान्}
{ततस्तं संवृणोत्येव तमतीत्य च वर्धते}


\twolineshloka
{तेनैवोग्रेन्धनेनैनं दावो दहति दारुणः}
{तथोपमा ह्यमात्यास्ते राजंस्तान्परिशोधय}


\twolineshloka
{त्वया चैव कृता राजन्भवता परिपालिताः}
{भवन्तं पर्यवज्ञाय जिघांसन्ति भवत्प्रियम्}


\threelineshloka
{उषितं शङ्कमानेन प्रमादं परिरक्षता}
{अन्तः सर्प इवागारे वीरपत्न्या इवालये}
{शीलं जिज्ञासमानेन राज्ञः साहसजीविनः}


\twolineshloka
{कच्चिज्जितेन्द्रियो राजा कच्चिदस्यान्तरा जिताः}
{कच्चिदेषां प्रियो राजा कच्चिद्राज्ञः प्रियाः प्रजाः}


\twolineshloka
{विजिज्ञासुरिह प्राप्तस्तवाहं राजसत्तम}
{तस्य मे रोचते राजन्क्षुधितस्येव भोजनम्}


\threelineshloka
{अमात्या मे न रोचन्ते वितृष्णस्य यथोदकम्}
{भवतोऽर्थकृदित्येवं मयि ते दोषमादधन्}
{विद्यते कारणं नान्यदिति मे नात्र संशयः}


\threelineshloka
{न हि तेषामहं द्रोग्धा तत्तेषां द्रोहवद्गतम्}
{अरेर्हि दुर्हृदाद्भेयं भग्नपृष्ठादिवोरगात् ॥राजोवाच}
{}


\twolineshloka
{भूयसा परिहारेण सत्कारेण च भूयसा}
{पूजितो ब्राह्मणश्रेष्ठ भूयो वस गृहे मम}


\twolineshloka
{ये त्वां ब्राह्मण नेच्छन्ति ते न वत्स्यन्ति मे गृहे}
{भवतैव हि तज्ज्ञेयं यत्तदेषामनन्तरम्}


\threelineshloka
{यथा स्यात्सुधृतो दण्डो यथा च सुकृतं कृतम्}
{तथा समीक्ष्य भगवञ्श्रेयसे विनियुङ्क्ष्व माम् ॥मुनिरुवाच}
{}


\threelineshloka
{अदर्शयन्निमं दोषमेकैकं दुर्बलं कुरु}
{ततः कारणमाज्ञाय पुरुषंपुरुषं जहि}
{एकदोषा हि बहवो मृद्गीयुरपि कण्टकान्}


\twolineshloka
{`अर्थे सर्वं जगद्वद्धमर्थेनैव निबध्यते}
{अर्थे दर्पो मनुष्याणां तस्मादर्थं विरोचय}


\twolineshloka
{एकेनैकस्य दोषेण तद्विरुद्धं प्रचोदय}
{स तस्य दोषानुद्भाव्य तस्यार्थं ग्राहयिष्यति}


\twolineshloka
{सामपूर्वं च केषांचिद्भेदेन च परस्परम्}
{वैरं कारय भूपाल पश्चाद्दण्डं प्रचोदय}


\twolineshloka
{बिल्वेन च यथा बिल्वमाकारं छाद्य बुद्धिमान्}
{अशुद्धं सचिवं राजन्नशुद्धेनैव नाशय ॥'}


% Check verse!
मन्त्रभेदभयाद्राजंस्तस्मादेतद्ब्रवीमि ते
\twolineshloka
{वयं तु ब्राह्मणा नाम मृदुदण्डाः कृपालवः}
{स्वस्ति चेच्छाम भवतः परेषां च यथाऽऽत्मनः}


\threelineshloka
{राजन्नात्मानमाचक्षे संबन्धी भवतो ह्यहम्}
{मुनिः कालकवृक्षीय इत्येवमभिसंज्ञितः}
{पितुः सखा च भवतः संमतः सत्यसंगरः}


\twolineshloka
{व्यापन्ने भवतो राज्ये राजन्पितरि संस्थिते}
{सर्वकामान्परित्यज्य तपस्तप्तं तदा मया}


% Check verse!
स्नेहात्त्वां तु ब्रवीम्येतन्मा भूयो विभ्रो दिति
\threelineshloka
{उभे दृष्ट्वा दुःखसुखे राज्यं प्राप्य यदृच्छया}
{राज्येनामात्यसंस्थेन कथं राजन्प्रमाद्यसि ॥भीष्म उवाच}
{}


\twolineshloka
{ततो राजकुले नान्दी संजज्ञे भूयसा पुनः}
{पुरोहितकुले चैव संप्राप्ते ब्राह्मणर्षभे}


\twolineshloka
{एकच्छत्रां महीं कृत्वा कौसल्याय यशस्विने}
{मुनिः कालकवृक्षीय ईजे क्रतुभिरुत्तमैः}


\twolineshloka
{हितं तद्वचनं श्रुत्वा कौसल्योऽप्यजयन्महीम्}
{तथा च कृतवान्राजा यथोक्तं तेन भारत}


\chapter{अध्यायः ८३}
\twolineshloka
{युधिष्ठिर उवाच}
{}


\threelineshloka
{सभासदः सहायाश्च सुहृदश्च विशांपते}
{परिच्छदास्तथाऽमात्याः कीदृशाः स्युः पितामह ॥भीष्म उवाच}
{}


\twolineshloka
{ह्रीनिषेवास्तथा दान्ताः सत्यार्जवसमन्विताः}
{शक्ताः कथयितुं सम्यक्ते तव स्युः सभासदः}


\twolineshloka
{अमात्याश्चातिशूराश्च ब्रह्मण्याश्च बहुश्रुताः}
{सुसंतृष्टाश्च कौन्तेय महोत्साहाश्च कर्मसु}


% Check verse!
एतान्सहायाँल्लिप्सेथाः सर्वास्वापत्सु भारत
\threelineshloka
{कुलीनः पूजितो नित्यं न हि शक्तिं निगूहति}
{प्रसन्नमप्रसन्नं वा पीडितं हतमेव वा}
{आवर्तयति भूयिष्ठं तदेव ह्यनुपालितम्}


\twolineshloka
{कुलीना देशजाः प्राज्ञा रूपवन्तो बहुश्रुताः}
{प्रगल्भाश्चानुरक्ताश्च ते तव स्युः परिच्छदाः}


\twolineshloka
{दौष्कुलेयाश्च लुब्धाश्च नृशंसा निरपत्रपाः}
{ते त्वां तात निषेवेयुर्यावदार्द्रकपाणयः}


\threelineshloka
{कुलीनाञ्शीलसंपन्नानिङ्गितज्ञाननिष्ठुरान्}
{देशकालविधानज्ञान्भर्तृकार्यहितैपिणः}
{नित्यमर्थेषु सर्वेषु राजा कुर्वीत मन्त्रिणः}


\twolineshloka
{अर्थमानार्घसत्कारैर्भोगैरुच्चावचैः प्रियैः}
{यानर्थभाजो मन्येथास्तेते स्युः सुखभागिनः}


\twolineshloka
{अभिन्नवृत्ता विद्वांसः सद्वॄत्ताश्चरितव्रताः}
{नत्वां नित्यार्थिनो जह्युरक्षुद्राः सत्यवादिनः}


\twolineshloka
{अनार्या ये न जानन्ति समयं मन्दचेतसः}
{तेभ्यः परिजुगुप्सेथा ये चापि समयच्युताः}


\twolineshloka
{नैकमिच्छेद्गणं हित्वा स्याच्चेदन्यतरग्रहः}
{यस्त्वेको बहुभिः श्रेयान्कामं तेन गणं त्यजेत्}


\twolineshloka
{श्रेयसो लक्षणं चैतद्विक्रमो यस्य दृश्यते}
{कीर्तिप्रधानो यश्च स्यात्समये यश्च तिष्ठति}


\twolineshloka
{समर्थान्पूजयेद्यश्च नास्पर्ध्यैः स्पर्धते च यः}
{न च कामाद्भयात्क्रोधाल्लोभाद्वा धर्ममुत्सृजेत्}


\twolineshloka
{अमानी अत्यवाक्शक्तो जितात्मा मानसंयुतः}
{स ते मन्त्रसहायः स्यात्सर्वावस्थापरीक्षितः}


\twolineshloka
{कुलीनः कुलसंपन्नस्तितिक्षुर्दश आत्मवान्}
{शूरः कृतज्ञः सत्यश्च श्रेयसः पार्थ लक्षणम्}


\twolineshloka
{तस्यैवं वर्तमानस्य पुरुषस्य विजानतः}
{अमित्राः संप्रसीदन्ति तथा मित्रीभवन्त्यपि}


\twolineshloka
{अत ऊर्ध्वममात्यानां परीक्षेत गुणागुणम्}
{संयतात्मा कृतप्रज्ञो भूतिकामश्च भूमिपः}


\twolineshloka
{संबन्धिपुरुषैराप्तैरभिजातैः स्वदेशजैः}
{अहार्यैरव्यभीचारैः सर्वशः सुपरीक्षितैः}


\twolineshloka
{यौनाः श्रौतास्तथा मौलास्तथैवाप्यनहंकृताः}
{कर्तव्या भूतिकामेन पुरुषेण बुभूपता}


\twolineshloka
{एषां वैनयिकी बुद्धिः प्रकृतिश्चैव शोभना}
{तेजो धैर्यं क्षमा शौचमनुरागः स्थितिर्धृतिः}


\twolineshloka
{परीक्ष्य च गुणान्नित्यं प्रौढभावान्धुरंधरान्}
{पञ्चोपधाव्यतीतांश्च कुर्याद्राजाऽर्थकारिणः}


\twolineshloka
{पर्याप्तवचनान्वीरान्प्रतिपत्तिविशारदान्}
{कुलीनान्सत्वसंपन्नानिङ्गितज्ञाननिष्ठुरान्}


\twolineshloka
{देशकालविधानज्ञान्भर्तृकार्यहितैषिणः}
{नित्यमर्थेषु सर्वेषु राजन्कुर्वीत मन्त्रिणः}


\twolineshloka
{हीनतेजोभिसंसृष्टो नैव जातु व्यवस्यति}
{अवश्यं जनयत्येव सर्वकर्मसु संशयम्}


\twolineshloka
{एवमल्पश्रुतो मन्त्री कल्याणाभिजनोऽप्युत}
{धर्मार्थकामसंयुक्तो नालं मन्त्रं परीक्षितुम्}


\twolineshloka
{तथैवानभिजातोऽपि काममस्तु बहुश्रुतः}
{अनायक इवाचक्षुर्मुह्यत्यूह्येषु कर्मसु}


\twolineshloka
{यो वाऽप्यस्थिरसंकल्पो बुद्धिमानागतागमः}
{12-83-28bउपायज्ञोऽपिनालं स कर्म प्रापयितुं चिरम्}


\twolineshloka
{केवलात्पुनरादानात्कर्मणो नोपपद्यते}
{परामर्शो विशेषणामश्रुतस्येह दुर्मतेः}


\twolineshloka
{मन्त्रिण्यननुरक्ते तु विश्वासो नोपपद्यते}
{तस्मादननुरक्ताय नैव मन्त्रं प्रकाशयेत्}


\twolineshloka
{व्यथयेद्धि स राजानं मन्त्रिभिः सहितोऽनृजुः}
{मारुतोपहितच्छिद्रैः प्रविश्याग्निरिव द्रुमम्}


\twolineshloka
{संक्रुद्धश्चैकदा स्वामी स्थानाच्चैवापकर्षति}
{वाचा क्षिपति संरब्धः पुनः पश्चात्प्रसीदति}


\twolineshloka
{तानितान्यनुरक्तेन शक्यानि हि तितिक्षितुम्}
{मन्त्रिणां च भवेत्क्रोधो विस्फूर्जितमिवाशनेः}


\twolineshloka
{यस्तु संहरते तानि भर्तुः प्रियचिकीर्षया}
{समानसुखदुःखं तं पृच्छेदर्थेषु मानवम्}


\twolineshloka
{अनृजुस्त्वनुरक्तोऽपि संपन्नश्चेतरैर्गुणैः}
{राज्ञः प्रज्ञानयुक्तोऽपि न मन्त्रं श्रोतुमर्हति}


\twolineshloka
{योऽमित्रैः सह संबद्धो न परान्बहुमन्यते}
{असुहृत्तादृशो ज्ञेयो न मन्त्रं श्रोतुमर्हति}


\twolineshloka
{अविद्वानशुचिः स्तब्धः शत्रुसेवी विकत्थनः}
{असुहृत्क्रोधनो लुब्धो न मन्त्रं श्रोतुमर्हति}


\twolineshloka
{आगन्तुश्चानुरक्तोऽपि काममस्तु बहुश्रुतः}
{सत्कृतः संविभक्तो वा न मन्त्रं श्रोतुमर्हति}


\twolineshloka
{विधर्मतो विप्रकृतः पिता यस्याभवत्पुरा}
{सत्कृतः स्थापितः सोऽपि न मन्त्रं श्रोतुमर्हति}


\twolineshloka
{यः स्वल्पेनापि कार्येण सुहृदाक्षारितो भवेत्}
{पुनरन्यैर्गुणैर्युक्तो न मन्त्रं श्रोतुमर्हति}


\twolineshloka
{कृतप्रज्ञश्च मेधावी बुधो जानपदः शुचिः}
{सर्वकर्मसु यः शुद्धः स मन्त्रं श्रोतुमर्हति}


\twolineshloka
{ज्ञानविज्ञानसंपन्नः प्रकृतिज्ञः परात्मनोः}
{सुहृदात्मसमो राज्ञः स मन्त्रं श्रोतुमर्हति}


\twolineshloka
{सत्यवाक्शीलसंपन्नो गन्भीरः सत्रपो मृदुः}
{पितृपैतामहो यः स्यात्स मन्त्रं श्रोतुमर्हति}


\twolineshloka
{संतुष्टः संमतः सद्भिः शौटीरो द्वेष्यपापकः}
{मन्त्रवित्कालविच्छूरः स मन्त्रं श्रोतुमर्हति}


\twolineshloka
{सर्वलोकमिमं शक्तः सान्त्वेन कुरुते वशम्}
{तस्मै मन्त्रः प्रयोक्तव्यो दण्डमाधित्सता नृप}


\twolineshloka
{पौरजानपदा यस्मिन्विश्वासं धर्मतो गताः}
{योद्धा नयविपश्चिच्च स मन्त्रं श्रोतुमर्हति}


\twolineshloka
{तस्मात्सर्वैर्गुणैरेतैरुपपन्नाः सुपूजिताः}
{मन्त्रिणः प्रकृतिज्ञाः स्युख्यवरा महदीप्सवः}


\twolineshloka
{स्वासु प्रकृतिषु च्छिद्रं लक्षयेरन्परस्य च}
{मन्त्रिणां मन्त्रमूलं हि राज्ञो राष्ट्रं विवर्धते}


\twolineshloka
{नास्य च्छिद्रं परः पश्येच्छिद्रेषु परमन्वियात्}
{गूहेत्कूर्म इवाङगानि रक्षेद्विवरमात्मनः}


\twolineshloka
{मन्त्रग्राहा हि राजस्य मन्त्रिणो ये मनीषिणः}
{मन्त्रसंहननो राजा मन्त्राङ्गानीतरे जन्गः}


\twolineshloka
{राज्यं प्रणिधिमूलं हि मन्त्रसारं प्रचक्षते}
{स्वामिनं त्वनुवर्न्तते वृत्त्यर्थमिह मन्त्रिणः}


\twolineshloka
{संविनीयमदक्रोधौ मानमीर्ष्यां च निर्वृताः}
{नित्यं पञ्चोपधातीतैर्मन्त्रयेत्सह मन्त्रिभिः}


\twolineshloka
{तेषां त्रयाणां त्रिविधं विमर्शंविबुध्य चित्तं विनिवेश्य तत्र}
{स्वनिश्चयं तं परनिश्चयं चनिदर्शयेदुत्तरमन्त्रकाले}


\twolineshloka
{धर्मार्थकामज्ञमुपेत्य पृच्छेद्युक्तो गुरुं ब्राह्मणमुत्तरार्थम्}
{निष्ठा कृता तेन यदा सहः स्यात्तं मन्त्रमार्गं प्रणयेदसक्तः}


\twolineshloka
{एवं सदा मन्त्रयितत्र्यमाहुर्ये मन्त्रतत्त्वार्थविनिश्चयज्ञाः}
{तस्मात्तमेवं प्रणयेत्सदैवमन्त्रं प्रजासंग्रहणे समर्थम्}


\twolineshloka
{न वामनाः कुब्जकृशा न खञ्जानान्धा जडाः स्त्री च नपुंसकाश्च}
{न चात्र तिर्यक्च पुरो न पश्चान्नोर्ध्वं न चाधः प्रपरेत्कथंचित्}


\twolineshloka
{आरुह्य वा वेश्म तथैव शून्यंस्थलं प्रकाशं कुशकाशहीनम्}
{वागङ्गदोषान्परिहृत्य सर्वान्संमन्त्रयेत्कार्यमहीनकालम्}


\chapter{अध्यायः ८४}
\twolineshloka
{भीष्म उवाच}
{}


\threelineshloka
{अत्राप्युदाहरन्तीममितिहासं पुरातनम्}
{बृहस्पतेश्च संवादं शक्रस्य च युधिष्ठिर ॥शक्र उवाच}
{}


\threelineshloka
{किंस्विदेकपढं ब्रह्मन्पुरुषः सम्यगाचरन्}
{प्रमाणं सर्वभूतानां यशश्चैवाप्नुयान्महत् ॥बृहस्पति उवाच}
{}


\twolineshloka
{सान्त्वमेकपदं शक्र पुरुषः सम्यगाचरन्}
{प्रमाणं सर्वभूतानां यशश्चैवाप्नुयान्महत्}


\twolineshloka
{एतदेकपदं शक्र सर्वलोकसुखावहम्}
{आचरन्सर्वभूतेषु प्रियो भवति सर्वदा}


\twolineshloka
{यो हि नाभाषते किंचित्सर्वदा भुकुटीमुखः}
{द्वेष्यो भवति भूतानां स सान्त्वमिह नाचरन्}


\twolineshloka
{यस्तु सर्वमभिप्रेक्ष्य पूर्वमेवाभिभाषते}
{स्मितपूर्वाभिभाषी च तस्य लोकः प्रसीदति}


\twolineshloka
{दानमेव हि सर्वत्र सान्त्वेनानभिजल्पितम्}
{न प्रीणयति भूतानि निर्व्यञ्जनमिवाशनम्}


\twolineshloka
{आददन्नपि भूतानां मधुरामीरयन्गिरम्}
{सर्वलोकमिमं शक्र सान्त्वेव कुरुते वशे}


\twolineshloka
{तस्मात्सान्त्वं प्रयोक्तव्यं दण्डमाधित्सताऽपि हि}
{प्रीतिं च जनयत्येवं न चास्योद्विजते जनः}


\threelineshloka
{सुकृतस्य हि सान्त्वस्य श्लक्ष्णस्य मधुरस्य च}
{सम्यगासेव्यमानस्य तुल्यं जातु न विद्यते ॥भीष्म उवाच}
{}


\twolineshloka
{इत्युक्तः कृतवान्सर्वं यथा शक्रः पुरोधसा}
{तथा त्वमपि कौन्तेय सम्यगेतत्समाचर}


\chapter{अध्यायः ८५}
\twolineshloka
{युधिष्ठिर उवाच}
{}


\threelineshloka
{कथंस्विदिह राजेन्द्र पालयन्पार्थिवः प्रजाः}
{प्रैति धर्मं विशेषेण कीर्तिमाप्नोति शाश्वतीम् ॥भीष्म उवाच}
{}


\threelineshloka
{व्यवहारेण शुद्धेन प्रजापालनतत्परः}
{प्राप्य धर्मं च कीर्ति च लोकानाप्नोत्यसौ शुचिः ॥युधिष्ठिर उवाच}
{}


\twolineshloka
{कीदृशव्यवहारं तु कैश्च व्यवहरेन्नृपः}
{एतत्पृष्टो महाप्राज्ञ यथावद्वक्तुमर्हसि}


\threelineshloka
{ये चैव पूर्वकथिता गुणास्ते पुरुषं प्रति}
{नैकस्मिन्पुरुषे ह्येते विद्यन्त इति मे मतिः ॥भीष्म उवाच}
{}


\twolineshloka
{एवमेतन्महाप्राज्ञ यथा वदसि बुद्धिमन्}
{दुर्लभः पुरुषः कश्चिदेभिर्युक्तो गुणैः शुभैः}


\twolineshloka
{किंतु संक्षेपतः शीलं प्रयत्नेनेह दुर्लभम्}
{वक्ष्यामि तु यथाऽमात्यान्यादृशांश्च करिष्यसि}


\twolineshloka
{चतुरो ब्राह्मणान्वैद्यान्प्रगल्भान्स्नातकाञ्शुचीन्}
{क्षत्रियान्दश चाष्टौ च बलिनः शस्त्रपाणिनः}


\twolineshloka
{वैश्यान्वित्तेन संपन्नानेकविंशतिसङ्ख्यया}
{त्रींश्च शूद्रान्विनीतांश्च शुचीन्कर्मणि पूर्वके}


\twolineshloka
{अष्टाभिश्च गुणैर्युक्तं सूतं पौराणिकं तथा}
{पञ्चाशद्वर्षवयसं प्रगल्भमनसूयकम्}


\twolineshloka
{श्रुतिस्मृतिसमायुक्तं विनीतं समदर्शिनम्}
{कार्ये विवदमानानां शक्तमर्थेष्वलोलुपम्}


\twolineshloka
{वर्जितं चैव व्यसनैः सुघोरैः सप्तभिर्भृशम्}
{अष्टानां मन्त्रिणां मध्ये मन्त्रं राजोपधारयेत्}


\twolineshloka
{ततः संप्रेषयेद्राष्ट्रे राष्ट्रीयाय च दर्शयेत्}
{अनेन व्यवहारेण द्रष्टव्यास्ते प्रजाः सदा}


\twolineshloka
{न चापि गूढं द्रव्यं ते ग्राह्यं कार्योपघातकम्}
{कार्ये खलु विपन्ने त्वां यो धर्मस्तं च पीडयेत्}


\twolineshloka
{विद्रवेच्चैव राष्ट्रं ते श्येनात्पक्षिगणा इव}
{परिस्रवेच्च सततं नौर्विशीर्णेव सागरे}


\twolineshloka
{प्रजाः पालयतोऽसम्यगधर्मेणेह भूपतेः}
{हार्दं भयं संभवति स्वर्गश्चस्य विरुध्यते}


\twolineshloka
{अथ यो धर्मतः पाति राजाऽमात्योऽथवाऽऽत्मजः}
{धर्मासने सन्नियुक्तो धर्ममूले नरर्षभ}


\threelineshloka
{`स्वर्गं याति महीपालो नियुक्तैः सचिवैः सह}
{'कार्येष्वधिकृताः सम्यगकुर्वन्तो नृपानुगाः}
{आत्मानं पुरतः कृत्वा यान्त्यधः सह पार्थिवाः}


\twolineshloka
{बलात्कृतानां वलिभिः कृपणं बहुजल्पताम्}
{नाथो वै भूमिपो नित्यमनाथानां नृणां भवेत्}


\twolineshloka
{ततः साक्षिबलं साधु द्वैधवादकृतं भवेत्}
{असाक्षिकमनाथं वा परीक्ष्यं तद्विशेषतः}


\twolineshloka
{अपराधानुरूपं च दण्डं पापेषु धारयेत्}
{वियोजयेद्धनैर्ऋद्धानधनानथ बन्धनैः}


\twolineshloka
{विनयेच्चापि दुर्वृत्तान्प्रहारैरपि पार्थिवः}
{सान्त्वेनोपप्रदानेन शिष्टांश्च परिपालयेत्}


\twolineshloka
{राज्ञो वधं चिकीर्षेद्यस्तस्य चित्रो वधो भवेत्}
{आदीपकस्य स्तेनस्य वर्णसंकरिकस्य च}


\twolineshloka
{सम्यक्प्रणयतो दण्डं भूमिपस्य विशांपते}
{युक्तस्य वा नास्त्यधर्मो धर्म एव हि शाश्वतः}


\twolineshloka
{कामकारेण दण्डं तु यः कुर्यादविचक्षणः}
{स इहाकीर्तिसंयुक्तो मृतो नरकमृच्छति}


\twolineshloka
{न परस्य प्रवादेन परेषां दण्डमर्पयेत्}
{आगमानुगमं कृत्वा बध्नीयान्मोक्षयीत वा}


\twolineshloka
{न तु हन्यान्नृपो जातु दूतं कस्यांचिदापदि}
{दूतस्य हन्ता निरयमाविशेत्सचिवैः सह}


\twolineshloka
{यथोक्तवादिनं दूतं क्षत्रधर्मरतो नृपः}
{यो हन्यात्पितरस्तस्य भ्रूणहत्यामवाप्नुयुः}


\twolineshloka
{कुलीनः शीलसंपन्नो वाग्मी दक्षः प्रियंवदः}
{यथोक्तवादीस्मृतिमान्दूतः स्यात्सप्तभिर्गुणैः}


\twolineshloka
{एतैरेव गुणैर्युक्तः प्रतीहारोऽस्य रक्षिता}
{शिरोरक्षश्च भवति गुणैरेतैः समन्वितः}


\twolineshloka
{धर्मशास्त्रार्थतत्त्वज्ञः सांधिविग्रहिको भवेत्}
{मतिमान्धृतिमान्ह्रीमान्रहस्यविनिगूहिता}


\twolineshloka
{कुलीनः सत्वसंपन्नः शुक्लोऽमात्यः प्रशस्यते}
{एतैरेव गुणैर्युक्तस्तथा सेनापतिर्भवेत्}


\twolineshloka
{व्यूहयन्त्रायुधानां च तत्त्वज्ञो विक्रमान्वितः}
{वर्षशीतोष्णवातानां सहिष्णुः पररन्ध्रवित्}


\twolineshloka
{विश्वासयेत्परांश्चैव विश्वसेच्च न कस्यचित्}
{पुत्रेष्वपि हि राजेन्द्र विश्वासो न प्रशस्यते}


\twolineshloka
{एतच्छास्त्रार्थतत्त्वं तु मयाऽऽख्यातं तवानघ}
{अविश्वासो नरेन्द्राणां गुह्यं परममुच्यते}


\chapter{अध्यायः ८६}
\twolineshloka
{युधिष्ठिर उवाच}
{}


\threelineshloka
{कथंविधं पुरं राजा स्वयमावस्तुमर्हति}
{कृतं वा कारयित्वा वा तन्मे ब्रूहि पितामह ॥भीष्म उवाच}
{}


\twolineshloka
{वस्तव्यं यत्र कौन्तेय सपुत्रज्ञातिबन्धुना}
{न्याय्यं च परिप्रष्टुं वृत्तिं गुप्तिं च भारत}


\twolineshloka
{तस्मात्ते र्तयिष्यामि दुर्गकर्म विशेषतः}
{श्रुत्वा तथा विधातव्यमनुष्ठेयं च यत्नतः}


\twolineshloka
{षङ्विधं दुर्गमास्थाय पुराण्यथ निवेशयेत्}
{सर्वसंपत्प्रधानं च बाहुल्यं चापि संभवेत्}


\twolineshloka
{धन्वदुर्गं महीदुर्गं गिरिदुर्गं तथैव च}
{मनुष्यदुर्गं मृद्दुर्गं वनदुर्गं च तानि षट्}


\twolineshloka
{यत्पुरं दुर्गसंपन्नं धान्यायुधसमन्वितम्}
{दृढप्राकारपरिखं हस्त्यश्वरथसंकुलम्}


\twolineshloka
{विद्वांसः शिल्पिनो यत्र निचयाश्च सुसंचिताः}
{धार्मिकश्च जनो यत्र दाक्ष्यमुत्तममास्थितः}


\twolineshloka
{ऊर्जस्विनरनागाश्वं चत्वरापणशोभितम्}
{प्रसिद्धव्यवहारं च प्रशान्तमकुतोभयम्}


\twolineshloka
{सुप्रभं सानुनादं च सुप्रशस्तनिवेशनम्}
{शूराढ्यं प्राज्ञसंपूर्णं ब्रह्मघोषानुनादितम्}


\twolineshloka
{समाजोत्सवसंपन्नं सदापूजितदैवतम्}
{वश्यामात्यबलो राजा तत्पुरं स्वयमाविशेत्}


\twolineshloka
{तत्र कोशं बलं मित्रं व्यवहारं च वर्धयेत्}
{पुरे जनपदे चैव सर्वदोषान्निवर्तयेत्}


\twolineshloka
{भाण्डागारायुधागारं प्रयत्नेनाभिवर्धयेत्}
{निचयान्वर्धयेत्सर्वांस्तथा यन्त्रकटंकटान्}


\twolineshloka
{काष्ठलोहतुषाङ्गारदारुशृङ्गास्थिवैणवान्}
{मज्जास्नेहवसाक्षौद्रमौषधग्राममेव च}


\twolineshloka
{शणं सर्जरसं धान्यमायुधानि शरांस्तथा}
{चर्म स्नायुं तथा वेत्रं मुञ्जवल्वजदंध्वनान्}


\twolineshloka
{आशयाश्चोदपानाश्च प्रभूतसलिलाकराः}
{निरोद्धव्याः सदा राज्ञा क्षीरिणश्च महीरुहाः}


\twolineshloka
{सत्कृताश्च प्रयत्नेन आचार्यर्त्विक्पुरोहिताः}
{महेष्वासाः स्थपतयः सांवत्सरचिकित्सकाः}


\twolineshloka
{प्राज्ञा मेधाविनो दान्ता दक्षाः शूरा बहुश्रुताः}
{कुलीनाः सत्वसंपन्ना युक्ताः सर्वेषु कर्मसु}


\twolineshloka
{पूजयेद्धार्मिकान्राजा निगृह्णीयादधार्मेकान्}
{नियुञ्ज्याच्च प्रयत्नेन सर्ववर्णान्स्वकर्मसु}


\twolineshloka
{बाह्यमाभ्यन्तरं चैव पौरजानपदं तथा}
{चारैः सुविदितं कृत्वा ततः कर्म प्रयोजयेत्}


\twolineshloka
{चरान्मन्त्रं च कोशं च दण्डं चैव विशेषतः}
{अनुतिष्ठेत्स्वयं राजा सर्वं ह्यत्र प्रतिष्ठितम्}


\twolineshloka
{उदासीनारिमित्राणां सर्वमेव चिकीर्षितम्}
{पुरे जनपदे चैव ज्ञातव्यं चारचक्षुषा}


\twolineshloka
{ततस्तेषां विधातव्यं सर्वमेवाप्रमादतः}
{भक्तान्पूजयता नित्यं द्विषतश्च निगृह्णता}


\twolineshloka
{यष्टव्यं क्रतुभिर्नित्यं दातव्यं चाप्यपीडया}
{प्रजानां रक्षणं कार्यं न कार्यं धर्मबाधकम्}


\twolineshloka
{कृपणानाथवृद्धानां विधवानां च योषिताम्}
{योगक्षेमं च वृत्तिं च नित्यमेव प्रकल्पयेत्}


\twolineshloka
{आश्रमेषु यथाकालं चैलभाजनभोजनम्}
{सदैवोपहरेद्राजा सत्कृयाभ्यर्च्य मान्य च}


\twolineshloka
{आत्मानं सर्वकार्याणि तापसे राष्ट्रमेव च}
{निवेदयेत्प्रयत्नेन तिष्ठेत्प्रह्वश्च सर्वदा}


\twolineshloka
{` ते कस्यांचिदवस्थायां शरणं शरणार्थिने}
{राज्ञे दद्युर्यथाकामं तापसाः शंसितव्रताः ॥'}


\twolineshloka
{सर्वार्थत्यागिनं राजा कुले जातं बहुश्रुतम्}
{पूजयेत्तादृशं दृष्ट्वा शयनासनभोजनैः}


\twolineshloka
{तस्मिन्कुर्वीत विश्वासं राजा कस्यांचिदापदि}
{तापसेषु हि विश्वासमपि कुर्वन्ति दस्यवः}


\twolineshloka
{तस्मिन्निधीनादधीत पुनः प्रत्याददीत च}
{न चाप्यभीक्ष्णं सेवेत भृशं वा प्रतिपूजयेत्}


\twolineshloka
{अन्यः कार्यः स्वराष्ट्रेषु परराष्ट्रेषु चापरः}
{अटवीषु परः कार्यः सामन्तनगरेष्वपि}


\twolineshloka
{तेषु सत्कारमानाभ्यां संविभागांश्च कारयेत्}
{परराष्ट्राटवीस्थेषु यथा स्वविषये तथा}


\twolineshloka
{ते कस्यांचिदवस्थायां शरणं शरणार्थिने}
{राज्ञे दद्युर्थथाकामं तापसाः संशितव्रताः}


\twolineshloka
{एष ते लक्षणोद्देशः संक्षेपेण प्रकीर्तितः}
{यादृशे नगरे राजा स्वयमावस्तुमर्हति}


\chapter{अध्यायः ८७}
\twolineshloka
{युधिष्ठिर उवाच}
{}


\threelineshloka
{राष्ट्रगुप्तिं च मे राजन्राष्ट्रस्यैव तु संग्रहम्}
{सम्यग्जिज्ञासमानाय प्रब्रूहि भरतर्षभ ॥भीष्म उवाच}
{}


\twolineshloka
{राष्ट्रगुप्तिं च ते सम्यग्राष्ट्रस्यैव तु संग्रहम्}
{हन्त सर्वं प्रवक्ष्यामि तत्त्वमेकमनाः शृणु}


\twolineshloka
{ग्रामस्याधिपतिः कार्यो दशग्रामपतिस्तथा}
{विंशतित्रिंशतीशं च सहस्रस्य च कारयेत्}


\twolineshloka
{ग्रामेयान्ग्रामदोषांश्च ग्रामिकः प्रतिभावेयेत्}
{तानाचक्षीत दशिने दशिको विंशिने पुनः}


\twolineshloka
{विंशाधिपस्तु तत्सर्वं वृत्तं जानपदे जने}
{ग्रामाणां शतपालाय सर्वमेव निवेदयेत्}


\twolineshloka
{यानि ग्राम्याणि भोज्यानि ग्रामिकस्तान्युपाश्निया}
{दशपस्तेन भर्तव्यस्तेनापि द्विगुणाधिपः}


\threelineshloka
{ग्रामं ग्रामशताध्यक्षो भोक्तुमर्हति सत्कुरः}
{महान्तं भरतश्रेष्ठ सुस्फीतं जनसंकुलम्}
{तत्र ह्यनेकपायत्तं राज्ञो भवति भारत}


\twolineshloka
{शाखानगरमर्हस्तु सहस्रपतिरुत्तमः}
{धान्यहैरण्यभोगेन भोक्तुं राष्ट्रीयसंगतः}


\twolineshloka
{तेषां संग्रामकृत्यं स्याद्वामकृत्यं च तेषु यत्}
{धर्मज्ञः सचिवः कश्चित्तत्तत्पश्येदतन्द्रितः}


\twolineshloka
{नगरेनगरे वा स्यादेकः सर्वार्थचिन्तकः}
{उच्चैः स्थाने घोररूपो नक्षत्राणामिव ग्रहः}


\twolineshloka
{भवेत्स तान्परिक्रामेत्सर्वानेव सभासदः}
{तेषां वृत्तिं परिणयेत्कश्चिद्राष्ट्रेषु तच्चरः}


\twolineshloka
{जिघांसवः पापकामाः परस्वादायिनः शठाः}
{रक्षाभ्यधिकृता नाम तेभ्यो रक्षेदिमाः प्रजाः}


\twolineshloka
{विक्रयं क्रयमध्वानं भक्तं च सपरिव्ययम्}
{योगक्षेमं च संप्रेक्ष्य वणिजां कारयेत्करान्}


\twolineshloka
{उत्पत्तिं दानवृत्तिं च शिल्पं संप्रेक्ष्य चासकृत्}
{शिल्पं प्रति करानेवं शिल्पिनः प्रति कारयेत्}


\twolineshloka
{उच्चावचकरन्यायाः पूर्वराज्ञां युधिष्ठिर}
{यथायथा न सीदेरंस्तथा कुर्यान्महीपतिः}


\twolineshloka
{फलं कर्म च संप्रेक्ष्य ततः सर्वं प्रकल्पयेत्}
{फलं कर्म च निर्हेतु न कश्चित्संप्रवर्तते}


\twolineshloka
{यथा राजा च कर्ता च स्यातां कर्मणि भागिनौ}
{संवेक्ष्य तु तथा राज्ञा प्रणेयाः सततं कराः}


\twolineshloka
{नोच्छिद्याहात्मनो मूलं परेषां चापि तृष्णया}
{ईहाद्वाराणि संरुध्य राजा संवृतदर्शनः}


\twolineshloka
{प्रद्विषन्ति परिख्यातं राजानमतिखादिनम्}
{प्रद्विष्टस्य कुतः श्रेयो संवृतो लभते श्रियम्}


\twolineshloka
{वत्सौपम्येन दोग्धव्यं राष्ट्रमक्षीणबुद्धिना}
{भृतो वत्सो जातबलः षीडां सहति भारत}


\twolineshloka
{न कर्म कुरुते वत्सो भृशं दुग्धो युधिष्ठिर}
{राष्ट्रमप्यातिदुग्धं हि न कर्म कुरुते महत्}


\twolineshloka
{यो राष्ट्रमनुगृह्णाति परिरक्षन्स्वयं नृपः}
{संजातमुपजीवन्स लभते सुमहत्फलम्}


\twolineshloka
{आपदर्थं च निचयात्राजानो हि चिचिन्वते}
{राष्ट्रं च कोशभूतं स्यात्कोशो वेश्मगतस्तथा}


\twolineshloka
{पौरजानपदान्सर्वान्संश्रितोषाश्रितांस्तथा}
{यथाशक्त्यनुकम्पेत सर्वान्स्वल्पधनानपि}


\twolineshloka
{बाह्यं जनं भेदयित्वा भोक्तव्यो मध्यमः सुखम्}
{एवं नास्य प्रकुप्यन्ति जनाः सुखितदुः खिताः}


\twolineshloka
{प्रामेव तु धनादानमनुभाष्य ततः पुनः}
{सन्निपत्य स्वविषये भयं राष्ट्रे प्रदर्शयेत्}


\twolineshloka
{इयमापत्समुत्पन्ना परचक्रभयं महत्}
{अपि चान्ताय कल्पन्ते वेणोरिव फलागमाः}


\threelineshloka
{अरयो मे समुत्थाय बहुभिर्दस्युभिः सह}
{इदमात्मवधायैव राष्ट्रमिच्छन्ति बाधितुम्}
{}


\twolineshloka
{अस्यामापदि घोरायां संप्राप्ते दारुणे भये}
{परित्राणाय भवतः प्रार्थयिष्ये धनानि वः}


\twolineshloka
{प्रतिदास्ये च भवतां सर्वं चाहं भयक्षये}
{नारयः प्रतिदास्यन्ति यद्धरेयुर्बलादितः}


\twolineshloka
{कलत्रमादितः कृत्वा सर्वं वो विनशेदिति}
{शरीरपुत्रदारार्थमर्थसंचय इष्यते}


\twolineshloka
{नन्दामि वः प्रभावेण पुत्राणामिव चोदये}
{यखाशक्त्युपगृह्णामि राष्ट्रस्यापीडया च वः}


\twolineshloka
{आपत्स्वेव निवोढव्यं भवद्भिः संगतैरिह}
{न वः प्रियतरं कार्यं धनं कस्यांचिदापदि}


\twolineshloka
{इति वाचा मधुरया श्लक्ष्णया सोपचारया}
{स्वरश्मीनभ्यवसृजेद्योगमाधाय कालवित्}


\twolineshloka
{प्रचारं भृत्यभरणं व्ययं संग्रामतो भयम्}
{योगक्षेणं च संप्रेक्ष्य गोमिनः कारयेत्करम्}


\twolineshloka
{उपेक्षिता हि नश्येयुर्गोमिनोऽरण्यवासिनः}
{तस्मात्तेषु विशेषेण मृदुपूर्वं समाचरेत्}


\twolineshloka
{सान्त्वनं रक्षणं दानमवस्था चाप्यभीक्ष्णशः}
{गोमिनां पार्थ कर्तव्यः संविभागः प्रियाणि च}


\twolineshloka
{अजस्रमुपयोक्तव्यं फलं गोमिषु भारत}
{प्रभावयन्ति राष्ट्रं च व्यवहारं कृषिं तथा}


\twolineshloka
{तस्माद्गोमिषु यत्नेन प्रीतिं कुर्याद्विचक्षणः}
{दयावानप्रमत्तश्च करान्संप्रणयन्मृदून्}


\twolineshloka
{सर्वत्र क्षेमचरणं सुलभं नाम गोमिषु}
{न ह्यतः सदृशं किंचिद्धनमस्ति युधिष्ठिर}


\chapter{अध्यायः ८८}
\twolineshloka
{युधिष्ठिर उवाच}
{}


\threelineshloka
{यदा राजा समर्थोऽपि कोशार्थी स्यान्महामते}
{कथं प्रवर्तेत करस्तन्मे ब्रूहि पितामह ॥भीष्म उवाच}
{}


\twolineshloka
{यथादेशं यथाकालं यथाबुद्धि यथाबलम्}
{अनुशिष्यात्प्रजा राजा धर्मार्थी तद्धिते रतः}


\twolineshloka
{यथा तासां च मन्येत श्रेय आत्मन एव च}
{तथा धर्माणि सर्वाणि राजा राष्ट्रेषु वर्तयेत्}


\twolineshloka
{मधुदोहं दुहेद्राष्ट्रं भ्रमरान्न प्रपातयेत्}
{वत्सापेक्षी दुहेच्चैव स्तनांश्च न विकुट्टयेत्}


\twolineshloka
{जलौकावत्पिबेद्राष्ट्रं मृदुनैव नराधिपः}
{व्याघ्रीव च हरेत्पुत्रान्संदशेन्न च पीडयेत्}


\twolineshloka
{यथा शल्यकवानाखुः पदं धूनयते सदा}
{अतीक्ष्णेनाभ्युपायेन तथा राष्ट्रं समापिबेत्}


\twolineshloka
{अल्पेनाल्पेन देयेन वर्धमानं प्रदापयेत्}
{ततो भूयस्ततो भूयः क्रमवृद्धिं समाचरेत्}


\twolineshloka
{दमयन्निव दम्यानि शश्वद्भारं विवर्धयेत्}
{मृदुपूर्वं प्रयत्नेन पाशानभ्यवहारयेत्}


\twolineshloka
{सकृत्पाशावकीर्णास्ते न भविष्यन्ति दुर्दमाः}
{उचितेनैव भोक्तव्यास्ते भविष्यन्त्ययत्नतः}


\twolineshloka
{तस्मात्सर्वसमारम्भो दुर्लभः पुरुषं प्रति}
{यथा मुख्यान्सान्त्वयित्वा भोक्तव्या इतरे जनाः}


\twolineshloka
{ततस्तान्भेदयित्वा तु परस्परविवक्षितान्}
{भुञ्जीत सान्त्वयंश्चैव यथासुखमयत्नतः}


\twolineshloka
{न चास्थाने न चाकाले करांस्तेभ्यो निपातयेत्}
{आनुपूर्व्येण सान्त्वेन यथाकालं यथाविधि}


\twolineshloka
{उपायान्प्रब्रवीम्येतान्न मे माया विवक्षिता}
{अनुपायेन दमयन्प्रकोपयति वाजिनः}


\twolineshloka
{पानागारनिवोशाश्च वेश्याः प्रापणिकास्तथा}
{कुशीलवाः सकितवा ये चान्ये केचिदीदृशाः}


\twolineshloka
{नियम्याः सर्व एवैते ये राष्ट्रस्योपघातकाः}
{एते राष्ट्रेऽभितिष्ठन्तो बाधन्ते भद्रिकाः प्रजाः}


\twolineshloka
{न केनचिद्याचितव्यः कश्चित्किंचिदनापदि}
{इति व्यवस्था भूतानां पुरस्तान्मनुना कृता}


\twolineshloka
{सर्वे तथाऽनुजीवेयुर्न कुर्युः कर्म चेदिह}
{सर्व एव इमे लोका न भवेयुरसंशयम्}


\twolineshloka
{प्रभुर्नियमने राजा य एतान्न नियच्छति}
{भुङ्क्ते स तस्य पापस्य चतुर्भागमिति श्रुतिः}


\twolineshloka
{भोक्ता तस्य तु पापस्य सुकृतस्य यथातथा}
{नियन्तव्याः सदा राज्ञा पापा ये स्युर्नराधिप}


\threelineshloka
{कृतपापस्त्वसौ राजा य एतान्न नियच्छति}
{तथा कृतस्य धर्मस्य चतुर्भागमुपाश्नुते}
{}


\twolineshloka
{स्थानान्येतानि संयम्य प्रसङ्गो भूतिनाशनः}
{कामे प्रसक्तः पुरुषः किमकार्यं विवर्जयेत्}


\twolineshloka
{मद्यमांसपरस्वानि तथा दारधनानि च}
{आहरेद्रागवशगस्तथा शास्त्रं प्रदर्शयेत्}


\twolineshloka
{आपद्येव तु याचन्ते येषां नास्ति परिग्रहः}
{दातव्यं धर्मतस्तेभ्यस्त्वनुक्रोशाद्भयान्न तु}


\twolineshloka
{मा ते राष्ट्रे याचनका भवेयुर्मा च दस्यवः}
{उपादातार एवैते नैते भूतस्य भावकाः}


\twolineshloka
{ये भूतान्यनुगृह्णन्ति वर्धयन्ति च ये प्रजाः}
{तेते राष्ट्रेषु वर्तन्तां मा भूतानां प्रबाधकाः}


\twolineshloka
{दण्ड्यास्ते च महाराज धनादानप्रयोजकाः}
{प्रयोगं कारयेथास्ते यथा दद्युः करांस्तथा}


\twolineshloka
{कृषिगोरक्ष्यवाणिज्यं यच्चान्यत्किंचिदीदृशम्}
{पुरुषैः कारयेत्कर्म बहुभिः कर्मभेदतः}


\twolineshloka
{नरश्चेत्कृषिगोरक्ष्यं वाणिज्यं चाप्यनुष्ठितः}
{संशयं लभते किंचित्तेन राजा विगर्ह्यते}


\twolineshloka
{धनिनः पूजयेन्नित्यं पानाच्छादनभोजनैः}
{वक्तव्याश्चानुगृह्णीध्वं प्रजाः सह मयेति वै}


\twolineshloka
{अङ्गमेतन्महद्राज्ये धनिनो नाम भारत}
{ककुदं सर्वभूतानां धनस्थो नात्र संशयः}


\twolineshloka
{प्राज्ञः शूरो धनस्थश्च स्वामी धार्मिक एव च}
{तपस्वी सत्यवादी च बुद्धिमांश्चापि रक्षति}


\twolineshloka
{तस्मात्सर्वेषु भूतेषु प्रीतिमान्भव पार्थिव}
{सत्यमार्जवमक्रोधमानृशंस्यं च पालय}


\twolineshloka
{एवं दण्डं च कोशं च मित्रं भूमिं च लप्स्यसि}
{सत्यार्जवपरो राजन्मित्रकोशबलान्वितः}


\chapter{अध्यायः ८९}
\twolineshloka
{भीष्म उवाच}
{}


\twolineshloka
{वनस्पतीभक्ष्यफलान्न च्छिन्द्युर्विषये तव}
{ब्राह्मणानां मूलफलं धर्ममाहुर्मनीषिणः}


\twolineshloka
{ब्राह्मणेभ्योऽतिरिक्तं च भुञ्जीरन्नितरे जनाः}
{न ब्राह्मणोपरोधेन हरेदन्यः कथंचन}


\twolineshloka
{विप्रश्चेत्त्यागमातिष्ठेदाख्याया वृत्तिकर्शितः}
{परिकल्यास्य वृत्तिः स्यात्सदारस्य नराधिप}


\twolineshloka
{स चेन्नोपनिवर्तेत वाच्यो ब्राह्मणसंसदि}
{कस्मिन्निदानीं मर्यादामयं लोकः करिष्यति}


\twolineshloka
{असंशयं निवर्तेत न चेत्त्यक्ष्यत्यतः परम्}
{पूर्वं परोक्षं वक्तव्यमेतत्कौन्तेय शाश्वतम्}


\twolineshloka
{आहुरेतज्जना ब्रह्मन्न चैतच्छ्रद्दधाम्यहम्}
{निमन्त्र्यश्च भवेद्भोगैरवृत्त्या च तदा चरेत्}


\twolineshloka
{कृषिगोरक्ष्यवाणिज्यं लोकानामिह जीवनम्}
{ऊर्ध्वं चैव त्रयी विद्या सा भूतान्भावयत्युत}


\twolineshloka
{तस्यां प्रपतमानायां ये स्युस्तत्परिपन्थिनः}
{दस्यवस्तद्वधायेह ब्रह्मा क्षत्रमथासृजत्}


\twolineshloka
{शत्रूञ्जय प्रजा रक्ष यजस्व क्रतुभिर्नृप}
{युध्यस्व समरे वीरो भूत्वा कौरवनन्दन}


\twolineshloka
{संरक्ष्यान्रक्षते राजा स राजा राजसत्तमः}
{ये केचित्तान्न रक्षन्ति तैरर्थो नास्ति कश्चन}


\twolineshloka
{सदैव राज्ञा योद्धव्यं सर्वलोकाद्युधिष्ठिर}
{तस्याद्धेतोर्हि भुञ्जीत मनुष्यानेव मानवः}


\twolineshloka
{आन्तरेभ्यः परान्रक्षन्परेभ्यः पुनरान्तरान्}
{परान्परेभ्यः स्वान्खेभ्यः सर्वान्पालय नित्यदा}


\twolineshloka
{आत्मानं सर्वतो रक्षन्राजन्रक्षस्व मेदिनीम्}
{आत्ममूलमिदं सर्वमाहुर्वै विदुषो जनाः}


\twolineshloka
{किं छिद्रं कोनु सङ्गो मे किंवाऽस्त्यविनिपातितम्}
{कुतो मामाश्रयेद्दोष इति नित्यं विचिन्तयेत्}


\twolineshloka
{अतीतदिवसे वृत्तं प्रशंसन्ति न वा पुनः}
{गुप्तैश्चारैरनुमतैः पृथिवीमनुसारयेत्}


\twolineshloka
{जानीत यदि मे वृत्तं प्रशंसन्ति न वा पुनः}
{कच्चिद्रोचेज्जनपदे कच्चिद्राष्ट्रे च मे वशः}


\twolineshloka
{धर्मज्ञानां धृतिमतां संग्रामेष्वपलायिनाम्}
{राष्ट्रे तु येऽनुजीवन्ति ये तु राज्ञोऽनुजीविनः}


\twolineshloka
{अमात्यानां च सर्वेषां मध्यस्थानां च सर्वशः}
{ये च त्वाऽभिप्रशंसेयुर्निन्देयुरथवा पुनः}


\fourlineindentedshloka
{सर्वान्सुपरिणीतांस्तान्कारयेथा युधिष्ठिर}
{एकान्तेन हि सर्वेषां न शक्यं तात रोचितुम्}
{मित्रामित्रमथो मध्यं सर्वभूतेषु भारत ॥युधिष्ठिर उवाच}
{}


\threelineshloka
{तुल्यबाहुबलानां च तुल्यानां च गुणैरपि}
{कथं स्यादधिकः कश्चित्स च भुञ्जीत मानवान् ॥भीष्म उवाच}
{}


\twolineshloka
{यच्चरा ह्यचरानद्युरदंष्ट्रान्दंष्ट्रिणस्तथा}
{आशीविषा इव क्रुद्धा भुजङ्गान्भुजगा इव}


\twolineshloka
{एतेभ्यश्चाप्रमत्तः स्यात्सदा शत्रोर्युधिष्ठिर}
{भारुण्डसदृशा ह्येते निपतन्ति प्रमादतः}


\twolineshloka
{कच्चित्ते वणिजो राष्ट्रे नोद्विजन्ति करार्दिताः}
{क्रीणन्तो बहुनाऽल्पेन कान्तारकृतविश्रमाः}


\twolineshloka
{कच्चित्कृषिकरा राष्ट्रं न जहत्यतिपीडिताः}
{ये बहन्ति धुरं राज्ञां ते भरन्तीतरानपि}


\twolineshloka
{`आलस्येन हृतः पादः पादः पाषण़्डमाश्रितः}
{राजानं सेवते पादः पादः कृषिमुपाश्रितः}


\twolineshloka
{एकपादं त्रयः पादा भक्षयन्ति दिनेदिने}
{तस्मात्सर्वप्रयत्नेन पादं रक्ष युधिष्ठिर ॥ '}


\twolineshloka
{इतो दत्तेन जीवन्ति देवाः पितृगणास्तथा}
{मानुषोरगरक्षांसि वयांसि पशवस्तथा}


\twolineshloka
{एषा ते राष्ट्रवृत्तिश्च राज्ञां गुप्तिश्च भारत}
{प्रोक्तोद्दिश्यैतमेवार्थं भूयो वक्ष्यामि पा--व}


\chapter{अध्यायः ९०}
\twolineshloka
{भीष्म उवाच}
{}


\twolineshloka
{आनङ्गिराः क्षत्रधर्मानुचध्यो ब्रह्मवित्तमः}
{मान्धात्रे यौवनाश्चाय प्रीतिमानभ्यभाषत}


\threelineshloka
{स यथाऽनुशशासैजमुचध्यो ब्रह्मवित्तमः}
{तत्तेऽहं संप्रवक्ष्यामि निखिलेन युधिष्ठिर ॥उचथ्य उवाच}
{}


\twolineshloka
{धर्माम राजा भवति न कामकरणाय तु}
{मान्यातरभिजानीहि राजा लोकस्य रक्षिता}


\twolineshloka
{राजा चरति चेद्धर्मं देवत्वायैव कल्पते}
{स चेदधर्मं चरति नरकायैव गच्छति}


\twolineshloka
{धर्मे तिष्ठन्ति भूतानि धर्मो राजनि तिष्ठाति}
{तं राजा साधु यः शास्ति स राजा श्रियमश्नुते}


\twolineshloka
{राजापराधान्मान्धातर्लक्ष्मीवान्पाप उच्यते}
{देवाश्च गर्हां गच्छन्ति धर्मो नास्तीति चोच्यते}


\twolineshloka
{अधर्मे वर्तमानानामर्थसिद्धिः प्रदृश्यते}
{तदेव मङ्गलं लोकः सर्वः समनुवर्तते}


\twolineshloka
{उच्छिद्यते धर्मवृत्तमधर्मो वर्तते महान्}
{भयमाहुर्दिवारात्रं यदा पापा न वार्यते}


\twolineshloka
{इदं मम इदं नेति साधूनां तात धर्मतः}
{न वै व्यवस्था भवति यदा पापो न वार्यते}


\twolineshloka
{नैव भार्या न पशवो न क्षेत्रं न निवेशनम्}
{संदृश्येत मनुष्याणां यदा पापबलं भवेत्}


\twolineshloka
{देवाः पूजां न जानन्ति न स्वधां पितरस्तदा}
{न पूज्यन्ते ह्यतिथयो यदा पापो न वार्यते}


\twolineshloka
{न वेदानधिगच्छन्ति व्रतवन्तो द्विजातयः}
{न यज्ञांस्तन्वते विप्रा यदा पापो न वार्यते}


\twolineshloka
{वध्यानामिव सत्वानां मनो भवति विह्वलम्}
{मनुष्याणां महाराज यदा पापो न वार्यते}


\twolineshloka
{उभौ लोकावभिप्रेक्ष्य राजानमसृजंस्तथा}
{मुनयोऽथ महद्भूतमयं धर्मो भविष्यति}


\twolineshloka
{यस्मिन्धर्मो विराजेत तं राजानं प्रचक्षते}
{यस्मिन्विलीयते धर्मस्तं देवा वृषलं विदुः}


\twolineshloka
{वृषो हि भगवान्धर्मो यस्तस्य कुरुते लयम्}
{वृषलं तं वीवदुर्देवास्तस्माद्धर्मं न लोपयेत्}


\twolineshloka
{धर्मे वर्धति वर्धन्ति सर्वभूतानि सर्वदा}
{तस्मिन्ह्रसति हीयन्ते तस्माद्धर्मं विवर्धयेत्}


\twolineshloka
{धनानि स्पौति धर्मो हि धारणाद्वेति निश्चयः}
{मानवान मनुष्येन्द्र स सीमान्तकरः स्मृतः}


\twolineshloka
{प्रभवार्थंमहि भूतानां धर्मः सृष्टः स्वयंभुवा}
{तस्मात्प्रवर्धयेद्धर्मं प्रजानुग्रहकारणात्}


\twolineshloka
{तस्माद्धि राजशार्दूल धर्मः श्रेष्ठतरः स्मृतः}
{स राजायः प्रजाः शास्ति साधुकृत्पुरुषर्षभ}


\twolineshloka
{कामक्रोधावनादृत्य धर्ममेवानुपालयेत्}
{धर्मः श्रेयस्करतमो राज्ञां भरतसत्तम}


\twolineshloka
{धर्मस्य ब्राह्मणे योनिस्तस्मात्तान्पूजयेत्सदा}
{ब्राह्मणानां च मान्धातः कुर्यात्कामानमत्सरी}


\twolineshloka
{तेषां ह्यकामकरणाद्राज्ञः संजायते भयम्}
{मित्राणि न च वर्धन्ते तथाऽमित्रीभवन्त्यपि}


\twolineshloka
{ब्राह्मणानां सदासूयन्बाल्याद्वैरोचनिर्बलिः}
{अथास्माच्छ्रीरपाक्रामद्याऽस्मिन्नासीत्प्रतापिनी}


\twolineshloka
{ततस्तस्मादपाक्रम्य साऽगच्छत्पाकशासनम्}
{अथ सोऽन्वतपत्पश्चाच्छ्रियं दृष्ट्वा पुरंदरे}


\twolineshloka
{एतत्फलमसूयाया अभिमानस्य चाभिभो}
{तस्माद्बुध्यस्व मान्धातर्मा त्वां जह्यात्प्रतापिनी}


\twolineshloka
{दर्पोनाम श्रियः पुत्रो जज्ञेऽधर्मादिति श्रुतिः}
{तेन देवासुरा राजन्नीताः सुबहवोऽव्ययम्}


\twolineshloka
{राजर्षयश्च बहवस्तथा बुध्यस्व पार्थिव}
{राजा भवति तं जित्वा दासस्तेन पराजितः}


\twolineshloka
{स यथा दर्पसहितमधर्मं नानुसेवते}
{तथा वर्तस्व मान्धातश्चिरं चेत्स्थातुमिच्छसि}


\twolineshloka
{मत्तात्प्रमत्तात्पौगण्डादुन्मत्ताच्च विशेषतः}
{निन्दिताच्चासदाचाराद्दुर्हृदां चापि सेवनात्}


\twolineshloka
{निगृहीतादमात्याच्च स्त्रीभ्यश्चैव विशेषतः}
{पर्वताद्विषमाद्दुर्गाद्धस्तिनोऽश्वात्सरीसृपात्}


\twolineshloka
{एतेभ्योऽनित्ययुक्तः स्यान्नक्तं चर्यां च वर्जयेत्}
{अत्याशां चाभिमानं च दम्भं क्रोधं च वर्जयेत्}


\twolineshloka
{अविज्ञातासु च स्त्रीषु क्लीबासु स्वैरिणीषु च}
{परभार्यासु कन्यासु नाचरेन्मैथुनं नृप}


\twolineshloka
{कुलेषु पापरक्षांसि जायन्ते वर्णसंकरात्}
{अपुमांसोऽङ्गहीनाश्च स्थूलजिह्वा विचेतसः}


\twolineshloka
{एते चान्ये च जायन्ते यदा राजा प्रमाद्यति}
{तस्माद्राज्ञा विशेषेणं वर्तितव्यं प्रजाहिते}


\twolineshloka
{क्षत्रियस्य प्रमत्तस्य दोषः संजायते महान्}
{अधर्माः संप्रवर्धन्ते प्रजासंकरकारकाः}


\twolineshloka
{अशीते विद्यते शीतं शीते शीतं न विद्यते}
{अवृष्टिरतिवृष्टिश्च व्याधिश्चाप्याविशेत्प्रजाः}


\twolineshloka
{नक्षत्राण्युपतिष्ठन्ति ग्रहा घोरास्तथागते}
{उत्पाताश्चात्र दृश्यन्ते बहवो राजनाशनाः}


\twolineshloka
{अरक्षितात्मा यो राजा प्रजाश्चापि न रक्षति}
{प्रजाश्च तस्य क्षीयन्ते ततः सोऽनु विनश्यति}


\twolineshloka
{द्वावाददाते ह्येकस्य द्वयोः सुबहवोऽपरे}
{कुमार्यः संप्रलुप्यन्ते तदाहुर्नृपदूषणम्}


\twolineshloka
{ममैतदिति नैतच्च मनुष्येष्ववतिष्ठति}
{त्यक्त्वा धर्मं यदा राजा प्रमादमनुतिष्ठित}


\chapter{अध्यायः ९१}
\twolineshloka
{उचथ्य उवाच}
{}


\twolineshloka
{कालवर्षी च पर्जन्यो धर्मचारी च पार्थिवः}
{संपद्यदेषा भवति सा बिभर्ति सुखं प्रजाः}


\twolineshloka
{यो न जानाति निर्हर्तुं वस्त्राणां रजको मलम्}
{रत्नानि वा शोधयितुं यथा नास्ति तथैव सः}


\twolineshloka
{एवमेतद्द्विजेन्द्राणां क्षत्रियाणां विशां तथा}
{शूद्रश्चतुर्थो वर्णानां नानाकर्मस्ववस्थितः}


\twolineshloka
{कर्म शूद्रे कृषिर्वैश्ये दण्डनीतिश्च राजनि}
{ब्रह्मचर्यं तपो मन्त्राः सत्यं चापि द्विजातिषु}


\twolineshloka
{तेषां यः क्षत्रियो वेद पात्राणामिव शोधनम्}
{शीलदोषान्विनिर्हर्तुं स पिता स प्रजापतिः}


\twolineshloka
{कृतं त्रेता द्वापरश्च कलिश्च भरतर्षभ}
{राजवृत्तानि सर्वाणि राजैव युगमुच्यते}


\twolineshloka
{चातुर्वर्ण्यं तथा वेदाश्चातुराश्रम्यमेव च}
{सर्वमेतत्प्रमुह्येत यदा राजा प्रमाद्यति}


\twolineshloka
{अग्नित्रेता त्रयी विद्या यज्ञाश्च सहदक्षिणाः}
{सर्व एव प्रमुह्यन्ते यदा राजा प्रमाद्यति}


\twolineshloka
{राजैव कर्ता भूतानां राजैव च विनाशकः}
{धर्मात्मा यः स कर्ता स्यादधर्मात्मा विनाशकः}


\twolineshloka
{राज्ञो भार्याश्च पुत्राश्च बान्धवाः सुहृदस्तथा}
{समेत्य सर्वे शोचन्ति यदा राजा प्रमाद्यति}


\twolineshloka
{हस्तिनोऽश्वाश्च गावश्चाप्युष्ट्राश्वतरगर्दभाः}
{अधर्मवृत्ते नृपतौ सर्वे सीदन्ति जन्तवः}


\twolineshloka
{[दुर्बलार्थं बलं सृष्टं धात्रा मान्धातरुच्यते}
{अबलं तु महद्भूतं यस्मिन्सर्वं प्रतिष्ठितम्}


\twolineshloka
{यश्च भूतं संभजते ये च भूतास्तदन्वयाः}
{अधर्मस्थे हि नृपतौ सर्वे शोचन्ति पार्थिव ॥]}


\twolineshloka
{दुर्बलस्य च यच्चक्षुर्मुनेराशीविषस्य च}
{अविषह्यतमं मन्ये मा स्म दुर्बलमासदः}


\twolineshloka
{दुर्बलांस्तात मन्येथा नित्यमेवाविमानि तान्}
{मा त्वां दुर्बलचक्षूंषि प्रदहेयुः सबान्धवम्}


\twolineshloka
{न हि दुर्बलदग्धस्य कुले किंचित्प्ररोहति}
{आमूलं निर्दहन्त्येव मा स्म दुर्बलमासद}


\twolineshloka
{अबलं वै बलाच्छ्रेयो यच्चातिबलवद्बलम्}
{बलस्याबलदग्धस्य न किंचितवशिष्यते}


\twolineshloka
{विमानितो हतः क्लिष्टस्त्रातारं चेन्न विन्दन्ते}
{अमानुषकृतस्तत्र दण्डो हन्ति नराधिपम्}


\twolineshloka
{मा स्म तात बलस्थस्त्वं भुञ्जीथा दुर्बलं जनम्}
{मा त्वां दुर्बलचक्षूंषि दहन्त्वग्निरिवाशयम्}


\twolineshloka
{यानि मिथ्याभिशस्तानां पतन्त्यश्रूणि रोदताम्}
{तानि पुत्रान्पशून्घ्नन्ति तेषां मिथ्याभिशंसिनां}


\twolineshloka
{यदि नात्मनि पुत्रेषु न चेत्पौत्रेषु नप्नृषु}
{न हि पापं कृतं कर्म सद्यः फलति गौरिब}


\twolineshloka
{यत्राबलो वध्यमानस्त्रातारं नाधिगच्छति}
{महान्दैवकृतस्तत्र दण्डः पतति दारुणः}


\twolineshloka
{युक्ता यदा जानपदा भिक्षन्ते ब्राह्मणा इव}
{अभीक्ष्णं भिक्षुरूपेण राजानं घ्नन्ति तादृशाः}


\twolineshloka
{राज्ञो यदा जनपदे बहवो राजपूरुषाः}
{अनयेनोपवर्तन्ते तद्राज्ञः किल्बिषं महत्}


\twolineshloka
{यदा युक्त्या नयेदर्थान्कामादर्थवशेन वा}
{कृपणं याचमानानां तद्राज्ञो वैशसं महत्}


\twolineshloka
{महान्वृक्षो जायते वर्धते चतं चैव भूतानि समाश्रयन्ति}
{यदा वृक्षश्छिद्यते दह्यते चतदाश्रया अनिकेता भवन्ति}


\twolineshloka
{यदा राष्ट्रे धर्ममग्र्यं चरन्तिसंस्कारं का राजगुणं ब्रुवाणाः}
{तैश्चाधर्मश्चरितो धर्ममोहात्तूप जह्यात्सुकृतं दुष्कृतं च}


\twolineshloka
{यत्र पापा ज्ञायमानाश्चरन्तिसभां कलिर्विन्दते तत्र राज्ञः}
{यत्र राजा शास्ति नरान्न शक्त्यान तद्राज्यं वर्धते भूमिपस्य}


\twolineshloka
{यश्चामात्यान्मानयित्वा यथा हिमन्त्रे च युद्धे च नृपोऽनुयुञ्ज्यात्}
{बिबर्धते तस्य राष्ट्रं नृपस्यभुङ्क्ते महीं चाप्यखिलां चिराय}


\twolineshloka
{अत्रापि सुकृतं कर्म वाचं चैव सुभाषिताम्}
{समीक्ष्य पूजयन्राजा धर्मं प्राप्नोत्यनुत्तमम्}


\twolineshloka
{संविभज्य यदा भुङ्क्ते नचान्यानवमन्यते}
{निहन्ति बलिनं दृप्तं स राज्ञो धर्म उच्यते}


\twolineshloka
{त्रायते हि यदा सर्वं वाचा कायेन कर्मणा}
{पुत्रस्यापि न मृष्येच्च स राज्ञो धर्म उच्यते}


\twolineshloka
{संविभज्य यदा भुङ्क्ते नृपतिर्दुर्बलान्नरान्}
{तदा भवन्ति बलिनः स राज्ञो धर्म उच्यते}


\twolineshloka
{यदा रक्षति राष्ट्राणि यदा दस्यूनपोहति}
{यदा जयति संग्रामे स राज्ञो धर्म उच्यते}


\twolineshloka
{पापमाचरतो यत्र कर्मणा व्याहृतेन वा}
{प्रियस्यापि न मृष्येत स राज्ञो धर्म उच्यते}


\twolineshloka
{यदा सारणिकान्राजा पुत्रवत्परिरक्षति}
{भिनत्ति न च मर्यादां स राज्ञो धर्म उच्यते}


\twolineshloka
{यदाप्तदक्षिणैर्यज्ञैर्यजते श्रद्धयाऽन्वितः}
{कामद्वेषावनादृत्य स राज्ञो धर्म उच्यते}


\twolineshloka
{कृपणानाथवृद्धानां यदाऽश्रु परिमार्जति}
{हर्षं संजनयन्नॄणां स राज्ञो धर्म उच्यते}


\twolineshloka
{विवर्धयति मित्राणि तथाऽरींश्चापि कर्षति}
{संपूजयति साधूंश्च स राज्ञो धर्म उच्यते}


\twolineshloka
{सत्यं पालयति प्रीत्या नित्यं भूमिं प्रयच्छति}
{पूजयेदतिथीन्भृत्यान्स राज्ञो धर्म उच्यते}


\twolineshloka
{निग्रहानुग्रहौ चोभौ यत्र स्यातां प्रतिष्ठितौ}
{अस्मिँल्लोके परे चैव राजा स प्राप्नुते फलम्}


\twolineshloka
{यमो राजा धार्मिकाणां मान्धातः परमेश्वरः}
{संयच्छन्यमवत्प्राणानसंयच्छंस्तु पावकः}


\twolineshloka
{ऋत्विक्पुरोहिताचार्यान्सत्कृत्यानवमत्य च}
{यदा सम्यक्प्रगृह्णाति स राज्ञो धर्म उच्यते}


\twolineshloka
{यमो यच्छति भूतानि सर्वाण्येवाविशेषतः}
{तथा राज्ञाऽनुकर्तव्यं यन्तव्या विधिवत्प्रजाः}


\twolineshloka
{सहस्राक्षेण राजा हि सर्वथैवोपमीयते}
{स पश्यति च यं धर्म स धर्मः पुरुषर्षभ}


\twolineshloka
{अप्रमादेन शिक्षेथाः क्षमां बुद्धिं धृतिं मतिम्}
{भूतानां तत्त्वजिज्ञासा साध्वसाधु च सर्वदा}


\twolineshloka
{संग्रहः सर्वभूतानां दानं च मधुरा च वाक्}
{पौरजानपदाश्चैव गोप्तव्याः स्वप्रजा यथा}


\twolineshloka
{न जात्वदक्षो नृपतिः प्रजाः शक्नोति रक्षितुम्}
{भारो हि सुमहांस्तात राज्यं नाम सुदुर्वहम्}


\twolineshloka
{तद्दण्डविन्नृपः प्राज्ञः शूरः शक्नोति रक्षितुम्}
{न हि शक्यमदण्डेन क्लीबेनाबुद्धिनाऽपि वा}


\twolineshloka
{अभिरूपैः कुले जातैर्दक्षैर्भक्तैर्बहुश्रुतैः}
{सर्वं बुद्ध्या परीक्षेथास्तापसाश्रमिणामपि}


\twolineshloka
{अतस्त्वं सर्वभूतानां धर्मं वेत्स्यसि वै परम्}
{स्वदेशे परदेशे वा न ते धर्मो विनङ्क्ष्यति}


\twolineshloka
{धर्मेचार्थे च कामे च धर्म एवोत्तरो भवेत्}
{अस्मिंल्लोके परे चैव धर्मात्मा सुखमेधते}


\twolineshloka
{त्यजन्ति दारान्पुत्रांश्च मनुष्याः परिपूजिताः}
{संग्रहश्चैव भूतानां दानं च मधुरा च वाक्}


\twolineshloka
{अप्रमादश्च शौचं च राज्ञो भूतिकरं महत्}
{एतेभ्यश्चैव मान्धातः सततं मा प्रमादिथाः}


\twolineshloka
{अप्रमत्तो भवेद्राजा छिद्रदर्शी परात्मनोः}
{नास्य च्छिद्रं परः पश्येच्छिद्रेषु परमन्वियात्}


\twolineshloka
{एतद्वॄत्तं वासवस्य यमस्य वरुणस्य च}
{राजर्षीणां च सर्वेषां तत्त्वमप्यनुपालय}


\twolineshloka
{तत्कुरुष्व महाराज वृत्तं राजर्षिसेवितम्}
{आतिष्ठ दिव्यं पन्थानमह्नाय पुरुषर्षभ}


\threelineshloka
{धर्मवृत्तं हि राजानं प्रेत्य चेह च भारत}
{देवर्षिपितृगन्धर्वाः कीर्तयन्ति महौजसः ॥भीष्म उवाच}
{}


\twolineshloka
{स एवमुक्तो मान्धाता तेनोचथ्येन भारत}
{कृतवानविशङ्कश्च एकः प्राप च मेदिनीम्}


\twolineshloka
{भवानपि तथा सम्यङ्भान्धातेव महीपते}
{धर्मं कृत्वा महीं रक्ष स्वर्गे स्थानमवाप्स्यसि}


\chapter{अध्यायः ९२}
\twolineshloka
{युधिष्ठिर उवाच}
{}


\threelineshloka
{कथं धर्मे स्थातुमिच्छन्राजा वर्तेत भारत}
{पृच्छामि त्वां कुरुश्रेष्ठ तन्मे ब्रूहि पितामह ॥भीष्म उवाच}
{}


\twolineshloka
{अत्राप्युदाहरन्तीममितिहासं पुरातनम्}
{गीतं दृष्टार्थतत्त्वेन वामदेवेन धीमता}


\twolineshloka
{राजा वसुमना नाम कौसल्यो बलवाञ्शुचिः}
{महर्षि परिपप्रच्छ वामदेवं तपस्विनम्}


\twolineshloka
{धर्मार्थसहितैर्वाक्यैर्भगवन्ननुशाधि माम्}
{येन वृत्तेन वै तिष्ठन्न च्यवेयं स्वधर्मतः}


\threelineshloka
{तमब्रवीद्धामदेवस्तेजस्वी तपतां वरः}
{हेमवर्णं सुखासीनं ययातिमिव नाहुषम् ॥वामदेव उवाच}
{}


\twolineshloka
{धर्ममेवानुवर्तस्व न धर्माद्विद्यते परम्}
{धर्मे स्थिता हि राजानो जयन्ति पृथिवीमिनाम्}


\twolineshloka
{अर्थसिद्धेः परं धर्मं मन्यते यो महीपतिः}
{वृद्ध्यां च कुरुते बृद्धिं स धर्मेण विराजते}


\twolineshloka
{अधर्मदर्शी यो राजा बलादेव प्रवर्तते}
{क्षिप्रमेवापयातोऽस्मादुभौ प्रथममध्यमौ}


\twolineshloka
{असत्पापिष्ठसचिवो वध्यो लोकस्य धर्महा}
{सहैव परिवारेण क्षिप्रमेवावसीदति}


\twolineshloka
{अर्थानामननुष्ठाता कामचारी विकत्थनः}
{अपि सर्वां महीं लब्ध्वा क्षिप्रमेव विनश्यति}


\twolineshloka
{अथाददानः कल्याणमनसूयुर्जितेन्द्रियः}
{वर्धते मतिमान्राजा स्रोतोभिरिव सागरः}


\twolineshloka
{न पूर्णोऽस्मीति मन्येत धर्मतः कामतोऽर्थतः}
{बुद्धितो मन्त्रतश्चापि सततं वसुधाधिप}


\twolineshloka
{एतेष्वेव हि सर्वेषु लोकयात्रा प्रतिष्ठिता}
{एतानि शृण्वँल्लभते यशः कीर्ति श्रियं प्रजाः}


\twolineshloka
{एवं यो धर्मसंरम्भी धर्मार्थपरिचिन्तकः}
{अर्थान्परीक्ष्यारभते स ध्रुवं महदश्नुते}


\twolineshloka
{अदाता ह्यनभिस्नेहो दण़्डेनावर्तयन्प्रजाः}
{साहसप्रकृती राजा क्षिप्रमेव विनश्यति}


\twolineshloka
{अथ पापं कृतं बुद्ध्या न च पश्यत्यबुद्धिमान्}
{अकीर्त्याऽभिसमायुक्तो भूयो नरकमश्नुते}


\twolineshloka
{ततो न याचितुर्दातुः शुक्लस्य रसवेदिनः}
{व्यसनं स्वमिवोत्पन्नं विजिघांसन्ति मानवाः}


\twolineshloka
{यस्य नास्ति गुरुर्धर्मे न चान्यानपि पृच्छति}
{सुखतन्त्रोऽर्थलाभेषु न चिरं सुखमश्नुते}


\twolineshloka
{गुरुप्रधानो धर्मेषु स्वयमर्थानवेक्षिता}
{धर्मप्रधानो लाभेषु स चिरं सुखमश्नुते}


\chapter{अध्यायः ९३}
\twolineshloka
{वामदेव उवाच}
{}


\twolineshloka
{यत्राधर्मं प्रणयते दुर्बले बलवत्तरः}
{तां वृत्तिमुपजीवन्ति ये भवन्ति तदन्वयाः}


\twolineshloka
{राजानमनुवर्तन्ते तं पापाभिप्रवर्तकम्}
{अविनीतमनुष्यं तत्क्षिप्रं राष्ट्रं विनश्यति}


\twolineshloka
{यद्वॄत्तमुपजीवन्ति प्रकृतिस्थस्य मानवाः}
{तदेव विषमस्थस्य स्वजनोऽपि न मृष्यते}


\twolineshloka
{साहसप्रवृत्ते र्यत्र किंचिदुल्वणमाचरेत्}
{अशास्त्रलक्षणो राजा क्षिप्रमेव विनश्यति}


\twolineshloka
{सद्वॄत्ताचरितां वृत्तिं क्षत्रियो नानुवर्तते}
{जितानामजितानां च क्षत्रधर्मादपैति सः}


\twolineshloka
{द्विषन्तं कृतकल्याणं गृहीत्वा नृपतिं रणे}
{यो न नयते द्वेषात्क्षत्रधर्मादपैति सः}


\twolineshloka
{शक्तः आत्सुसुखो राजा कुर्यात्तारणमापदि}
{प्रियो अति भूतानां न च विभ्रश्यते श्रियः}


\twolineshloka
{अप्रियं यस्य कुर्वीत भूयस्तस्य प्रियं चरेत्}
{अचिरेण प्रियः स स्याद्योऽप्रिये प्रियमाचरेत्}


\twolineshloka
{मृषावादं परिहरेत्कुर्यात्प्रियमयाचितः}
{न कामान्न च संरम्भान्न द्वेषाद्धर्ममुत्सृजेत्}


\twolineshloka
{`अमाययैव वर्तेत न च सत्यं त्यजेद्बुधः}
{दमं धर्मं च शीलं च क्षत्रधर्मं प्रजाहितम् ॥'}


\twolineshloka
{नापत्रपेत प्रश्नेषु नाभिभाविगिरं सृजेत्}
{न त्वरेत न चासूयेत्तथा संगृह्यते परः}


\twolineshloka
{प्रिये नातिभृशं हृष्येदप्रिये न च संज्वरेत्}
{न तप्येदर्थकृच्छ्रेषु प्रजाहितमनुस्मरन्}


\twolineshloka
{यः प्रियं कुरुते नित्यं गुणतो वसुधाधिपः}
{तस्य कर्माणि सिद्ध्यन्ति न च संत्यज्यते श्रिया}


\twolineshloka
{निवृत्तं प्रतिकूलेभ्यो वर्तमानमनुप्रिये}
{भक्तं भजेत नृपतिस्तद्वै वृत्तं सतामिह}


\twolineshloka
{अप्रकीर्णेन्द्रियग्राममत्यन्तानुगतं शुचिम्}
{शक्तं चैवानुरक्तं च युञ्ज्यान्महति कर्मणि}


\threelineshloka
{`श्रेयसो लक्षणं चैतद्विक्रमो यत्र दृश्यते}
{कीर्तिप्रधानो यश्च स्यात्समये यश्च तिष्ठति}
{समर्थान्पूजयेद्यश्च न च स्पर्धेत यश्चतैः}


\twolineshloka
{एवमेतैर्गुणैर्युक्तो योऽनुरज्यति भूमिपम्}
{भर्तुरर्थेष्वप्रमत्तं नियुञ्ज्यादर्थकर्मणि}


\twolineshloka
{मूढमैन्द्रियकं लुब्धमनार्यचरितं शठम्}
{अनतीतोपधं हिंस्रं दुर्बुद्धिमबहुश्रुतम्}


\twolineshloka
{त्यक्तोपात्तं मद्यरतं द्यूतस्त्रीमृगयापरम्}
{कार्ये महति युञ्जानो हीयते नृपतिः श्रिया}


\twolineshloka
{रक्षितात्मा च यो राजा रक्ष्यान्यश्चानुरक्षति}
{प्रजाश्च तस्य वर्धन्ते सुखं च महदश्नुते}


\twolineshloka
{ये केचिद्भूमिपतयः सर्वांस्तानन्ववेक्षयेत्}
{सुहृद्भिरनभिख्यातैस्तेन राजा न रिष्यते}


\twolineshloka
{अपकृत्य बलस्थस्व दूरस्थोऽस्मीति नाश्वसेत्}
{श्येनाभिपतनैरेते निपतन्ति प्रमाद्यतः}


\twolineshloka
{दृढमूलस्त्वदुष्टात्मा विदित्वा बलमात्मनः}
{अबलानभियुञ्जीत न तु ये बलवत्तराः}


\twolineshloka
{विक्रमेण महीं लब्ध्वा प्रजा धर्मेण पालयेत्}
{आहवे निधनं कुर्याद्राजा धर्मपरायणः}


\twolineshloka
{मरणान्तमिदं सर्वं नेह किंचिदनामयम्}
{तस्माद्धर्मे स्थितो राजा प्रजा धर्मेम पालयेत्}


\twolineshloka
{रक्षाधिकरणं युद्धं तथा धर्मानुशासनम्}
{मन्त्रचिन्ता सुखं काले पञ्चभिर्वर्धते मही}


\twolineshloka
{एतानि यस्य गुप्तानि स राजा राजसत्तम}
{सततं वर्तमानोऽत्र राजा भुङ््क्ते महीमिमाम्}


\twolineshloka
{नैतान्येकेन शक्यानि सातत्येनानुवीक्षितुम्}
{एतेष्वाप्तान्प्रतिष्ठाप्य राजा भुङ्क्ते चिरं महीम्}


\twolineshloka
{दातारं संविभक्तारं मार्दवोपगतं शुचिम्}
{असंत्यक्तमनुष्यं च तं जनाः कुर्वते नृपम्}


\twolineshloka
{यस्तु नैःश्रेयसं श्रुत्वा ज्ञानं तत्प्रतिपद्यते}
{आत्मनो मतमुत्सृज्य तं लोकोऽनुविधीयते}


\twolineshloka
{योऽर्थकामस्य वचनं प्रातिकूल्यान्न मृष्यते}
{शृणोति प्रतिकूलानि सर्वदा विमना इव}


\twolineshloka
{अग्राम्यचरितां वृत्तिं यो न सेवेत नित्यदा}
{जितानामजितानां च क्षत्रधर्मादपैति सः}


\threelineshloka
{[निगृहीतादमात्याच्च स्त्रीभ्यश्चैव विशेषतः}
{पर्वताद्विषमाद्दुर्गाद्धस्तिनोऽश्वात्सरीसृपात्}
{एतेभ्यो नित्ययुक्तः सन्रक्षेदात्मानमेव तु ॥]}


\twolineshloka
{मुख्यानमात्यान्यो हित्वा निहीनान्कुरुते प्रियान्}
{स वै व्यसनमासाद्य साधुमार्गं न विन्दति}


\twolineshloka
{यः कल्याणगुणाञ्ज्ञातीन्प्रद्वेषान्नो बुभूषति}
{अदृढात्मा दृढक्रोधः नास्यार्थो वसतेऽन्तिके}


\twolineshloka
{अथ यो गुणसंपन्नान्हृदयस्य प्रियानपि}
{प्रियेण कुरुते वश्यांश्चिरं यशसि तिष्ठति}


\twolineshloka
{नाकाले प्रणयेदर्थान्नाप्रिये जातु संज्वरेत्}
{प्रिये नातिभृशं तुष्येद्युञ्जीतारोग्यकर्मणि}


\twolineshloka
{के वाऽनुरक्ता राजानः के भयात्समुपाश्रिताः}
{मध्यस्थदोषाः के चैषामिति नित्यं विचिन्तयेत्}


\twolineshloka
{न जातु बलवान्भूत्वा दुर्बले विश्वसेत्क्वचित्}
{भारुण्डसदृशा ह्येते निपतन्ति प्रमाद्यतः}


\twolineshloka
{अपि सर्वगुणैर्युक्तं भर्तारं प्रियवादिन}
{अभिद्रुह्यति पापात्मा न तस्माद्विश्वसेज्जनान्}


\twolineshloka
{एतद्राजोपनिषदं ययातिः स्माह नाहुषः}
{मनुष्यविषये युक्तो हन्ति शत्रून्सवासवान्}


\chapter{अध्यायः ९४}
\twolineshloka
{वामदेव उवाच}
{}


\twolineshloka
{अयुद्धेनैव विजयं वर्धयेद्वसुधाधिपः}
{जघन्यमाहुर्विजयं युद्धेन च नराधिप}


\twolineshloka
{न चाप्यलब्धं लिप्सेत मूले नातिदृढे सति}
{न हि दुर्बलमूलस्य राज्ञो लाभो विवर्धते}


\twolineshloka
{यस्य स्फीतो जनपदः संपन्नप्रियराजकः}
{संतुष्टः पुष्टसचिवो दृढमूलः स पार्थिवः}


\twolineshloka
{यस्य योधाः सुसंतुष्टाः स्वनुरक्ताः सुपूजिताः}
{अल्पेनापि स दण्डेन महीं जयति पार्थिवः}


\twolineshloka
{`दण्डो हि बलवान्यत्र तत्र साम प्रयुज्यते}
{प्रदानं सामपूर्वं च भेदमूलं प्रशस्यते}


\twolineshloka
{त्रयाणां विफलं कर्म यदा पश्येत भूमिपः}
{रन्ध्रं ज्ञात्वा ततो दण्डं प्रयुञ्जीताविचारयन्}


\twolineshloka
{अभिभूतो यदा शत्रुः शत्रुभिर्बलवत्तरैः}
{उपेक्षा तत्र कर्तव्या वध्यता बलिनां बलम्}


\threelineshloka
{दुर्बलो हि महीपालो यदा भवति भारत}
{उपेक्षा तत्र कर्तव्या चतुर्णामविरोधीनि}
{उपायः पञ्चमः सोऽपि सर्वेषां बलवत्तरः}


\twolineshloka
{भार्गवेण च गीतानां श्लोकानां कोसलाधिप}
{विज्ञाय तत्वं तत्वज्ञ तत्वतस्तत्करिष्यति}


\twolineshloka
{यदि रक्षःपिशाचेन हन्यते यत्रकुत्रचित्}
{उपेक्षा तत्र कर्तव्या वाच्यतां बलिनां बलम्}


\twolineshloka
{दुर्बलोऽपि महीपाल शत्रूणां शत्रुमुद्धरेत्}
{पादलग्नं करस्थेन कण्टकेनैव कण्टकम्}


\twolineshloka
{शठानां उचिवानां च म्लेच्छानां च महीपते}
{एष उक्त उपायानामुपेक्षा बलवत्तम}


\twolineshloka
{अश्मना नाशयेल्लोहं लोहेनाश्मानमेव तु}
{बिल्वानि वा परैर्बिल्वैर्म्लेच्छैर्म्लेच्छान्प्रसादयेत्}


\threelineshloka
{दासानां च प्रदृप्तानामेतदेव हि कारयेत्}
{चण्डालम्लेच्छजातीनां दण्डेनैव निवारणम्}
{शठानां दुर्विनीतैश्च पूर्वमुक्तं समाचरेत्}


\twolineshloka
{अन्त्याः शठाश्च सचिवास्तथा कुब्राह्मणादयः}
{उपायैः पञ्चभिः साध्याश्चतुर्वर्गविरोधिनः}


\twolineshloka
{पौरजानपदा यस्य स्वनुरक्ता अपीडिताः}
{राष्ट्रकर्मकरा ह्येते राष्ट्रस्य च विरोधिनः}


\threelineshloka
{दुर्विनीता विनीताश्च सर्वे साध्याः प्रयत्नतः}
{चण्डालम्लेच्छजात्याश्च पाषण्डाश्च विकर्मिणः}
{बलिनश्चाश्रमाश्चैव तथा गायकनर्तकाः ॥'}


\twolineshloka
{पौरजानपदा यस्य भूतेषु च दयालवः}
{सधना धान्यवन्तश्च दृढमूलः स पार्थिवः}


\twolineshloka
{प्रतापकालमधिकं यदा मन्येत चात्मनः}
{तदा लिप्सेत मेधावी परभूमिधनान्युत}


\twolineshloka
{भोगेषूदयमानस्य भूतेषु च दयावतः}
{वर्धते त्वरमाणस्य विषयो रक्षितात्मनः}


\twolineshloka
{तक्षेदात्मानमेवं स वनं परशुना यथा}
{यः सम्यग्वर्तमानेषु स्वेषु मिथ्या प्रवर्तते}


\twolineshloka
{नैव द्विषन्तो हीयन्ते राज्ञो नित्यमनिघ्नतः}
{क्रोधं निहन्तुं यो वेद तस्य द्वेष्टा न विद्यते}


\twolineshloka
{यदार्यजनविद्विष्टं कर्म तन्नाचरेद्बुधः}
{यत्कल्याणमभिध्यायेत्तत्रात्मानं नियोजयेत्}


\twolineshloka
{नैवमन्येऽवजानन्ति नात्मना परितप्यते}
{कृत्यशेषेण यो राजा सुखान्यनुबुभूषति}


\threelineshloka
{इदं वृत्तं मनुष्येषु वर्तते यो महीपतिः}
{उभौ लोकौ विनिर्जित्य विजये संप्रतिष्ठते ॥भीष्म उवाच}
{}


\twolineshloka
{इत्युक्तो वामदेवेन सर्वं तत्कृतवान्नृपः}
{तथा कुर्वंस्त्वमप्येतौ लोकौजेता न संशयः}


\chapter{अध्यायः ९५}
\twolineshloka
{युधिष्ठिर उवाच}
{}


\threelineshloka
{अथ यो विजिगीषेत क्षत्रियः क्षत्रियं युधि}
{कस्तस्य विजये धर्मो ह्येतं पृष्टो ब्रवीहि मे ॥भीष्म उवाच}
{}


\twolineshloka
{ससहायोऽसहायो वा राष्ट्रमागम्य भूमिपः}
{ब्रूयादहं यो राजेति रक्षिष्यामि च वः सदा}


\twolineshloka
{मम धर्मबलिं दत्त किंवा मां प्रतिपत्स्यथ}
{ते चोक्तमागतं तत्र घृणीयुः कुशलं भवेत्}


\twolineshloka
{ते चेदक्षत्रियाः सन्तो विरुध्येरन्कथंचन}
{सर्वोपायैर्नियन्तव्या विकर्मस्था नराधिप}


\threelineshloka
{अशस्त्रं क्षत्रियं मत्वा शस्त्रं गृह्णात्यथापरः}
{त्राणायाप्यसमर्थं तं मन्यमानमतीव च ॥युधिष्ठिर उवाच}
{}


\threelineshloka
{अथः यः क्षत्रियो राजा क्षत्रियं प्रत्युपाव्रजेत्}
{कथं संप्रतियोद्धव्यस्तन्मे ब्रूहि पितामह ॥भीष्म उवाच}
{}


\twolineshloka
{नासन्नह्यो नाकवचो योद्धव्यः क्षत्रियो रणे}
{एक एकेन भाव्यश्च विसृजेति क्षिपामि च}


\twolineshloka
{स चेत्सन्नद्ध आगच्छेत्सन्नद्धव्यं ततो भवेत्}
{स चेत्ससैन्य आगच्छेत्ससैन्यस्तमथाह्वयेत्}


\twolineshloka
{स चेन्निकृत्या युध्येत निकृत्या प्रतियोधयेत्}
{अथ चेद्धर्मतो युध्येद्धर्मेणैव निवारयेत्}


\twolineshloka
{नाश्वेन रथिनं यायादुदियाद्रथिनं रथी}
{व्यसने न प्रहर्तव्यं न भीताय जिताय च}


\twolineshloka
{नेषुर्लिप्तो न कर्णी स्यादसतामेतदायुधम्}
{यथार्थमेव योद्धव्यं न क्रुद्ध्येत जिघांसतः}


\twolineshloka
{`नास्त्येकस्य गजो युद्धे गजश्चेकस्य विद्यते}
{न पदातिर्गजं युध्येन्न गतेन पदातिनम्}


\threelineshloka
{हस्तिना योधयेन्नागं कदाचिच्छिक्षितो हयः}
{दिव्यास्त्रबलसंपन्नः कामं युध्येत सर्वदा}
{नागे भूमौ समे चैव रथेनाश्वेन वा पुनः}


\twolineshloka
{रामरावणयोर्युद्धे हरयो वै पदातयः}
{लक्ष्मणश्च महाभागस्तथा राजन्विभीषणः}


\twolineshloka
{रावणस्यान्तकाले च रथेनैन्द्रेण राधवः}
{निजघान दुराचारं रावणं पापकारिणम्}


\twolineshloka
{दिव्यास्त्रबलसंपन्ने सर्वमेतद्विधीयते}
{देवासुरेषु सर्वेषु दृष्टमेतत्पुरातनैः ॥'}


\fourlineindentedshloka
{[साधूनां तु यदा भेदात्साधुश्चेद्व्यसनी भवेत्}
{]निष्प्राणो नाभिहन्तव्यो नानपत्यः कथंचन}
{भग्नशस्त्रो विपन्नश्च कृत्तज्यो हतवाहनः}
{}


\twolineshloka
{चिकित्स्यः स्यात्स्वविषये प्राप्यो वा स्वगृहे भवेत्}
{निर्व्रणश्च स भोक्तव्य एष धर्मः सनातनः}


\twolineshloka
{तस्माद्धर्मेण योद्धव्यमिति स्वायंभुवोऽब्रवात्}
{सत्सु नित्यः सतां धर्मस्तमास्थाय न नाशयेत्}


\twolineshloka
{यो वै जयत्यधर्मेण क्षत्रियो धर्मसंगरः}
{आत्मानमात्मना हन्ति पापो निकृतिजीवनः}


\twolineshloka
{कर्म चैतदसाधूनां साधून्योऽसाधुना जयेत्}
{धर्मेण निधनं श्रेयो न जयः पापकर्मणा}


\twolineshloka
{नाधर्मश्चरितो राजन्सद्यः फलति गौरिव}
{मूलानि च प्रशाखाश्च दहन्समधिगच्छते}


\twolineshloka
{पापेन कर्मणा वित्तं लब्ध्वा पापः प्रहृष्यति}
{स वर्धमानस्तेनैव पापः पापे प्रसज्जति}


\threelineshloka
{न धर्मोऽस्तीति मन्वानः शुचीनवहसन्निव}
{अश्रद्दधानश्च भवेद्विनाशमुपगच्छतिस बद्धो वारुणैः पाशैरमर्त्यैरवमन्यते}
{}


\twolineshloka
{महादृतिरिवाध्मातः स्वकृतेनैव वर्धते}
{ततः समूलो ह्रियते नदीकूलादिव द्रुमा}


\twolineshloka
{अथैनमभिनिन्दन्ति भिन्नं कुम्भमिवाश्यानि}
{तस्माद्धर्मेण विजयं कोशं लिप्सेत भूमिपः}


\chapter{अध्यायः ९६}
\twolineshloka
{भीष्म उवाच}
{}


\twolineshloka
{नाधर्मेण महीं जेतुं लिप्सेत जगतीपतिः}
{अधर्मविजयं लब्ध्वा को नु मन्येत भूमिपः}


\twolineshloka
{अधर्मयुक्तो विजयो ह्यध्रुवोऽस्वर्ग्य एव च}
{पातयत्येव राजानं महीं च भरतर्षभ}


\twolineshloka
{विशीर्णकवचं चैव तवास्मीति च वादिनम्}
{कृताञ्जलिं न्यस्तशस्त्रं गृहीत्वा न विहिंसयेत्}


\twolineshloka
{बलेन विजितो यश्च न तं युध्येत भूमिपः}
{संवत्सरं विप्रणयेत्तस्माज्जातः पुनर्भवेत्}


\twolineshloka
{नार्वाक्संवत्सरात्कन्या प्रष्टव्या विक्रमाहृता}
{एवमेव धनं सर्वं यच्चान्यत्सहसा हृतम्}


\twolineshloka
{न तु वध्ये धनं तिष्ठेत्पिबेयुर्ब्राह्मणाः पयः}
{युञ्जीरन्नप्यनडुहः क्षन्तव्यं वा पुनर्भवेत्}


\twolineshloka
{राज्ञा राजैव योद्धव्यस्तथा धर्मो विधीयते}
{नान्यो राजानमभ्यस्येदराजन्यः कथंचन}


\twolineshloka
{अनीकयोः संहतयोर्यदीयाद्ब्राह्मणोऽन्तरा}
{शान्तिमिच्छन्नुभयतो न योद्धव्यं तदा भवेत्}


\threelineshloka
{मर्यादां शाश्वतीं भिन्द्याद्ब्राह्मणं योऽभिलङ्घयेत्}
{अथ चेल्लङ्घयेदेव मर्यादां क्षत्रियब्रुवः}
{असङ्ख्येपतदूर्ध्वं स्यादनादेयश्च संसदि}


\twolineshloka
{यस्तु धर्मविलोपेन मर्यादाभेदनेन च}
{तां वृत्तिं नानुवर्तेत विजिगीषुर्महीपतिः}


% Check verse!
धर्मलब्धाद्धि विजयाल्लाभः कोऽभ्यधिको भवेत्
\twolineshloka
{सहसा न्याय्यभूतानि क्षिप्रमेव प्रसादयेत्}
{सान्त्वेन भोगदानेन स राज्ञां परमो नयः}


\twolineshloka
{भुज्यमाना ह्यभोगेन स्वराष्ट्रादभितापिताः}
{अमित्रान्पर्युपासीरन्व्यसनौघप्रतीक्षिणः}


\twolineshloka
{अमित्रोपग्रहं चास्य ते कुर्युः क्षिप्रमापदि}
{संतुष्टाः सर्वतो राजन्राजव्यसनकाङ्क्षिणः}


\twolineshloka
{नामित्रो विनिकर्तव्यो नातिच्छेद्यः कथंचन}
{जीवितं ह्यप्यतिच्छिन्नः संत्यजेदेकदा नरः}


\twolineshloka
{अल्पेनापि च संयुक्तस्तुष्यते नापराधितः}
{शुद्धं जीवितमेवापि तादृशो बहुमन्यते}


\twolineshloka
{यस्य स्फीतो जनपदः संपन्नः प्रियराजकः}
{संतुष्टभृत्यसचिवो दृढमूलः स पार्थिवः}


\twolineshloka
{ऋत्विक्पुरोहिताचार्या ये चान्ये श्रुतसत्तमाः}
{पूजार्हाः पूजिता यस्य स वै लोकविदुच्यते}


\twolineshloka
{एतेनैव च वृत्तेन महीं प्राप सुरोत्तमः}
{अन्येऽपि चैव विजयं विजिगीषन्ति पार्थिवाः}


\twolineshloka
{भूमिवर्जं धनं राजा जित्वा राजन्महाहवे}
{अपि चान्नोषधीः शश्वदाजहार प्रतर्दनः}


\twolineshloka
{अग्रिहोत्राग्निशेषं च हविर्भोजनमेव च}
{आजहार दिवोदासस्ततो विप्रकृतोऽभवत्}


\twolineshloka
{सराजकानि राष्ट्राणि नाभागो दक्षिणां ददौ}
{अन्यत्र श्रोत्रियस्वाच्च तापसार्थाच्च भारत}


\twolineshloka
{उच्चावचानि वित्तानि धर्मज्ञानां युधिष्ठिर}
{आसन्राज्ञां पुराणानां सर्वं तन्मम रोचते}


\twolineshloka
{सर्वविद्यातिरेकेण जयमिच्छेन्महीपतिः}
{न मायया न दम्भेन य इच्छेद्भूतिमात्मनः}


\chapter{अध्यायः ९७}
\twolineshloka
{युधिष्ठिर उवाच}
{}


\twolineshloka
{क्षत्रधर्माद्धि पापीयान्न धर्मोऽस्ति नराधिप}
{अपयाने च युद्धे च राजा हन्ति महाजनम्}


\threelineshloka
{अथ स्म कर्मणा केन लोकाञ्जयति पार्थिवः}
{विद्वञ्जिज्ञासमानाय प्रब्रूहि भरतर्षभ ॥भीष्म उवाच}
{}


\twolineshloka
{निग्रहेण च पापानां साधूनां संग्रहेण च}
{यज्ञैर्दानैश्च राजानो भवन्ति शुचयोऽमलाः}


\twolineshloka
{उपरुन्धन्ति राजानो भूतानि विजयार्थिनः}
{त एव विजयं प्राप्य वर्धयन्ति पुनः प्रजाः}


\twolineshloka
{अपविध्यन्ति पापानि दानयज्ञतपोबलैः}
{अनुग्रहेण भूतानां पुण्यमेषां विवर्धते}


\twolineshloka
{यथैव क्षेत्रनिर्याता निर्यातं क्षेत्रमेव च}
{हिनस्ति धान्यकक्षं च न च धान्यं विनश्यति}


\twolineshloka
{एवं शस्त्राणि मुञ्चन्तो घ्नन्ति वध्याननेकधा}
{तस्यैषा निष्कृतिर्दृष्टा भूतानां भावनं पुनः}


\twolineshloka
{यो भूतानि सदाऽनर्थाद्वधात्क्लेशाच्च रक्षति}
{दस्युभ्यः प्राणदानात्स धनदः सुखदो विराट्}


\twolineshloka
{स सर्वयज्ञारीजानो राजाऽथाभयदक्षिणैः}
{अनुभूयेह भद्राणि प्राप्नोतीन्द्रसलोकताम्}


\twolineshloka
{ब्राह्मणार्थे समुत्पन्ने योऽभिनिष्पत्य युध्यति}
{आत्मानं यूपमुत्सृज्य स यज्ञोऽनन्तदक्षिणः}


\twolineshloka
{अभीतो विकिरञ्शत्रून्प्रतिगृह्य शरांस्तथा}
{न तस्मान्त्रिदशाः श्रेयो भुवि पश्यन्ति किंचन}


\twolineshloka
{तस्य शस्त्राणि यावन्ति त्वचं भिन्दन्ति संयुगे}
{तावतः सोऽश्नुते लोकान्सर्वकामदुहोऽक्षयान्}


\twolineshloka
{यदस्य रुधिरं गात्रादाहवे संप्रवर्तते}
{सह तेनैव स्रावेण सर्वपापैः प्रमुच्यते}


\twolineshloka
{यानि दुःखानि सहते प्राणानामतिपातने}
{न तपोऽस्ति ततो भूय इति धर्मविदो विदुः}


\twolineshloka
{पृष्ठतो भीरवः सङ्ख्ये वर्तन्ते धर्मपूरुषाः}
{शूराच्छरणमिच्छन्तः पर्जन्यादिव जीवनम्}


\twolineshloka
{यदि शूरं तथा क्षेमे प्रतीक्षेरन्यथा भये}
{प्रतिरूपं जनाः कुर्युर्न च तद्वर्तते तथा}


\twolineshloka
{यदि ते कृतमाज्ञाय नमस्कुर्युः सदैव तम्}
{युक्तं न्याय्यं च कुर्युस्ते न च तद्वर्तते तथा}


\twolineshloka
{पुरुषाणां समानानां दृश्यते महदन्तरम्}
{संग्रामेऽनीकवेलायामुत्कृष्टेषु पतत्सु च}


\twolineshloka
{पतत्यभिमुखं शूरः परान्भीरुः पलायते}
{आस्थाय स्वर्ग्यमध्वानं सहायान्विषमे त्यजन्}


% Check verse!
मा स्म तांस्तादृशांस्तात जनिष्टाऽधर्मपूरुषान्
\twolineshloka
{ये सहायान्रणे हित्वा स्वस्तिमन्तो गृहान्ययुः}
{अस्वस्ति तेभ्यः कुर्वन्ति देवा इन्द्रपुरोगमाः}


\threelineshloka
{त्यागेन यः सहायानां स्वान्प्राणांस्त्रातुमिच्छति}
{तं हन्युः काष्ठलोहैर्वा दहेयुर्वा कटाग्निना}
{पशुवन्मारयेयुर्वा क्षत्रिया ये स्युरीदृशाः}


\twolineshloka
{अधर्मः क्षत्रियस्यैष यच्छय्यामरणं भवेत्}
{विसृजञ्श्लेष्मपित्तानि कृपणं परिदेवयन्}


\twolineshloka
{अविक्षतेन देहेन प्रलयं योऽधिगच्छति}
{क्षत्रियो नास्य तत्कर्म प्रशंसन्ति पुराविदः}


\twolineshloka
{न गृहे मरणं तात क्षत्रियाणां प्रशस्यते}
{शौण्डीराणामशौण्डीर्यमधर्मं कृपणं च तत्}


\twolineshloka
{इदं कृच्छ्रमहो दुःखं पापीय इति निष्टनन्}
{प्रतिध्वस्तमुखः पूतिरमात्याननुशोचयन्}


\twolineshloka
{अरोगाणां स्पृहयते मुहुर्मृत्युमपीच्छति}
{वीरो दृप्तो मनस्वी च नेदृशं मृत्युमर्हति}


\twolineshloka
{रणेषु कदनं कृत्वा सुहृद्भिः प्रतिपूजित}
{तीक्ष्णैः शस्त्रैरभिक्लिष्टः क्षत्रियो मुत्युमर्हति}


\twolineshloka
{शूरो हि सत्वमन्युभ्यामाविष्टो युध्यते मशम्}
{कृत्यमानानि गात्राणि परैर्नैवावबुध्यते}


\twolineshloka
{स सङ्ख्ये निधनं प्राप्य प्रशस्तं लोकपूजितम्}
{स्वधर्मं विपुलं प्राप्य शक्रस्यैति सलोकताम्}


\twolineshloka
{सर्वोपायै रणमुखमातिष्ठंस्त्यक्तजीवितः}
{प्राप्नोतीन्द्रस्य सालोक्यं शूरः पृष्ठमदर्शयन्}


\twolineshloka
{यत्रयत्र हतः शूरः शत्रुभिः परिवारितः}
{अक्षयांल्लभते लोकान्यदि दैन्यं न सेवते}


\chapter{अध्यायः ९८}
\twolineshloka
{युधिष्ठिर उवाच}
{}


\threelineshloka
{के लोका युध्यमानानां शूराणामनिवर्तिनाम्}
{भवन्ति निधनं प्राप्य तन्मे ब्रूहि पितामह ॥भीष्म उवाच}
{}


\twolineshloka
{अत्राप्युदाहरन्तीममितिहासं पुरातनम्}
{अम्बरीपस्य संवादमिन्द्रस्य च युधिष्ठिर}


\twolineshloka
{अम्बरीषो हि नाभागः स्वर्गं गत्वा सुदुर्लभम्}
{ददर्श सुर--कस्थं शक्रेण सचिवैः सह}


\twolineshloka
{सर्वतेजोमयं दिव्यं विमानवरमास्थितम्}
{उ---गच्छन्तं स्थानं सेनापतिं शुभम्}


\fourlineindentedshloka
{स दष्ट्वापरि गच्छन्तं सेनापतिमुदारधीः}
{`शूरस्थानमनुप्राप्तं सुदेवं नाम नामतः}
{'ऋद्धिं दृष्ट्वा सुदेवस्य विस्मितः प्राह वासवम् ॥अम्बरीप उवाच}
{}


\twolineshloka
{सागरान्तां महीं कृत्स्नामनुशास्य यथाविधि}
{चातुर्वर्ण्ये यथाशास्त्रं प्रवृत्तो धर्मकाम्यया}


\twolineshloka
{ब्रह्मचर्येण घोरेण गुर्वाचारेण सेवया}
{वेदानधीत्य धर्मेण राजशास्त्रं च केवलम्}


\twolineshloka
{अतिथीनन्नपानेन पितॄंश्च स्वधया तथा}
{ऋषीन्स्वाध्यायदीक्षाभिर्देवान्यज्ञैरनुत्तमैः}


\twolineshloka
{क्षत्रध--स्थितो भूत्वा यथाशास्त्रं यथाविधि}
{उदीक्षमाणः पृतनां जयामि युधि वासव}


\threelineshloka
{दे--ज सुदेवोऽयं मम सेनापतिः पुरा}
{आसाद्योधः प्रशान्तात्मा सोऽयं कस्मादतीव माम्}
{विमानं सूर्यसङ्काशमास्थितो मोदते दिवि}


\twolineshloka
{अनेन ऋतुभिर्मुख्यैर्नेष्टं नापि द्विजातयः}
{तर्पिता विधिवच्छक्र सोऽयं कस्मादतीत्य माम्}


\fourlineindentedshloka
{ऐश्वर्यमीदृशं प्राप्तः सर्वदेवैः सुदुर्लभम्}
{इन्द्र उवाच}
{`यदनेन कृतं कर्म प्रत्यक्षं ते महीपते}
{पुरा पालयतः सम्यक्पृथिवीं धर्मतो नृप}


\twolineshloka
{शत्रवो निर्जिताः सर्वे ये तवाहितकारिणः}
{संयमो वियमश्चैव सुयमश्च नहाबलः}


\twolineshloka
{राक्षसा दुर्जया लोके त्रयस्ते युद्धदुर्मदाः}
{पुत्रास्ते शतशृङ्गस्य राक्षसस्य महीपतेः}


\twolineshloka
{तथा तस्मिञ्शुभे काले तव यज्ञं वितन्वतः}
{अश्वमेधं महायागं देवानां हितकाम्यया}


\threelineshloka
{तस्य ते खलु विघ्नार्थमागता राक्षसास्त्रयः}
{कोटीशतपरीवारां राक्षसानां महाचमूम्}
{परिगृह्य ततः सर्वाः प्रजा वन्दीकृतास्तव}


\twolineshloka
{विह्वलाश्च प्रजाः सर्वाः सर्वे च तव सैनिकाः}
{निराकृतस्तु यच्चासीत्सुदेवः सैन्यनायकः}


\twolineshloka
{तत्रामात्यवचः श्रुत्वा निरस्तः सर्वकर्मसु}
{श्रुत्वा तेषां वचो भूयः सोपधं वसुधाधिपः}


\twolineshloka
{सर्वसैन्यसमायुक्तः सुदेवः प्रेरितस्त्वया}
{साक्षसानां वधार्थाय दुर्जयानां नराधिप}


\twolineshloka
{नाजित्वा राक्षसीं सेनां पुनरागमनं तव}
{बन्दीमोक्षमकृत्वा च न चागमनमिष्यते}


\twolineshloka
{सुदेवस्तद्वचः श्रुत्वा प्रस्थानमकरोन्नृप}
{संप्राप्तश्च स तं देशं यत्र बन्दीकृताः प्रजाः}


\twolineshloka
{पश्यति स्म महाघोरां राक्षसानां महाचमूम्}
{दृष्ट्वा सुचिन्तयामास सुदेवो वाहिनीपतिः}


\threelineshloka
{नेयं शक्या चमूर्जेतुमपि सेन्द्रैः सुरासुरैः}
{नाम्बरीषः कलामेकामेषां क्षपयितुं क्षमः}
{दिव्यास्त्रबलभूयिष्ठः किमहं पुनरीदृशः}


\twolineshloka
{ततः सेनां पुनः सर्वां प्रेषयामास पार्थिव}
{यत्र त्वं सचिवैः सर्वैर्मन्त्रिभिः सोपधैर्नृप}


\twolineshloka
{ततो रुद्रं महादेवं प्रपन्नो जगतः पतिम्}
{श्मशाननिलयं देवं तुष्टाव वृषभध्वजम्}


% Check verse!
स्तुत्वा शस्त्रं समादाय स्वशिरश्छेत्तुमुद्यतः
\twolineshloka
{कारुण्याद्देवदेवेन गृहीतस्तस्य दक्षिणः}
{स पाणिः सह शस्त्रेण दृष्ट्वा चेदमुवाच ह}


\twolineshloka
{किमिदं साहसं पुत्र कुर्तकामो वदस्व मे}
{स उवाच महादेवं शिरसा त्ववनीं गतः}


\threelineshloka
{भगवन्वाहिनीमेनां राक्षसानां सुरेश्वर}
{अशक्तोऽहं रणे जेतुं तस्मात्त्यक्ष्यामि जीवितम्}
{गतिर्भव महादेव ममार्तस्य जगत्पते}


\twolineshloka
{नागन्तव्यमजित्वा च मामाह जगतीपतिः}
{अम्बरीषो महादेव क्षारितः सचिवैः सह}


\twolineshloka
{तमुवाच महादेवः सुदेवं पतितं क्षितौ}
{अधोमुखं महात्मानं सत्वानां हितकाम्यया}


\twolineshloka
{धनुर्वेदं समाहूय सगणं सहविग्रहम्}
{रथनागाश्वकलिलं दिव्यास्त्रसमलंकृतम्}


\threelineshloka
{रथं च सुमहाभागं येन तन्त्रिपुरं हतम्}
{धनुः पिनाकं खङ्गं च रौद्रमस्त्रं च शंकरः}
{निजघानासुरान्सर्वान्येन देवस्त्रियम्बकः}


\twolineshloka
{उवाच च महादेवः सुदेवं वाहिनीपतिम्}
{रथादस्मात्सुदेव त्वं दुर्जयः स सुरासुरैः}


\twolineshloka
{मायया मोहितो भूमौ न पदं कर्तुमर्हसि}
{रथस्थस्त्रिदशान्सर्वाञ्जेष्यसि त्वं सदानवान्}


\twolineshloka
{राक्षसाश्च पिशाचाश्च न शक्ता द्रष्टुमीदृशम्}
{रथं सूर्यसहस्राभं किमु योद्धुं त्वया सह}


\threelineshloka
{स जित्वा राक्षसान्सर्वान्कृत्वा बन्दीविमोक्षणम्}
{घातयित्वा च तान्सर्वान्बाहुयुद्धे त्वयं हतः}
{वियमं प्राप्य भूपाल वियमश्च निपातितः ॥'}


\twolineshloka
{तस्य विक्रमतस्तात सुदेवस्य बभूव ह}
{संग्रामयज्ञः सुमहान्यश्चान्यो युध्यते नरः}


\threelineshloka
{सन्नद्धो दीक्षितः सर्वो योधः प्राप्य चमूमुखम्}
{युद्धयज्ञाधिकारस्थो भवतीति विनिश्चयः ॥अम्बरीष उवाच}
{}


\threelineshloka
{कानि यज्ञे हवींष्यस्मिन्किमाज्यं का च दक्षिणा}
{ऋत्विजश्चात्र क्रे प्रोक्तास्तन्मे ब्रूहि शतक्रतो ॥इन्द्र उवाच}
{}


\twolineshloka
{ऋत्विजः कुञ्जरास्तत्र वाजिनोऽध्वर्यवस्तथा}
{हवींषि परमांसानि रुधिरं त्वाज्यमुच्यते}


\twolineshloka
{शृगालगृध्रकाकोलाः सदस्यास्तत्र पन्त्रिणः}
{आज्यशेषं पिबन्त्येते हविः प्राश्नन्ति चाध्वरे}


\twolineshloka
{प्रासतोमरसंघाताः खङ्गशक्तिपरश्वथाः}
{ज्वलन्तो निशिताः पीताः स्रुचस्तस्याथ सत्रिणः}


\twolineshloka
{चापवेगायतस्तीक्ष्णः परकायावभेदनः}
{ऋजुः सुनिशितः पीतः सायकश्च स्रुवो म--}


\twolineshloka
{द्वीपिचर्मावनद्धश्च नागदन्तकृतत्सरुः}
{हस्तिहस्तहरः खङ्गः स्फयो भवेत्तस्य संयुगे}


\twolineshloka
{ज्वलितैर्निशितैः प्रासशक्त्यृष्टिसपरश्वथैः}
{शैक्यायसमयैस्तीक्ष्णैरभिघातो भवेद्वसु}


\threelineshloka
{[सङ्ख्यासमयविस्तीर्णमभिजातोद्भवं बहु}
{]आवेधाद्यच्च रुधिरं संग्रामे स्रवते भुवि}
{साऽस्य पूर्णाहुतिर्होत्रैः समृद्धा सर्वकामधुक्}


\twolineshloka
{छिन्धि भिन्धीति यः शब्दः श्रूयते वाहिनीमुखे}
{सामानि सामगास्तस्य गायन्ति यमसादने}


% Check verse!
हविर्धानं तु तस्याहुः परेषां वाहिनीमुखम्
\twolineshloka
{कुञ्जराणां हयानां च वर्मिणां च समुच्चय}
{अग्निः श्येनचितो नाम यज्ञे तस्य विधीयते}


\twolineshloka
{उत्तिष्ठते कबन्धोऽत्र सहस्रे पतिते तु यः}
{स यूपस्तस्य शूरस्य खादिरोऽष्टाश्रिरुच्यते}


\twolineshloka
{इडोपहूताः क्रोशन्ति कुञ्जरास्त्वङ्कुशेरिताः}
{ज्याघुष्टतलतालेन वषट््कारेण पार्थिव}


\threelineshloka
{उद्गाता तत्र संग्रामे त्रिसामा दुन्दुभिर्नृप}
{ब्रह्मस्वे ह्रियमाणे तु त्यक्त्वा युद्धे प्रियां तनुम्}
{आत्मानं यूपमुच्छ्रित्य स यज्ञोऽनन्तदक्षिणः}


\twolineshloka
{भर्तुरर्थे च यः शूरो निष्क्रामेद्वाहिनीमुखात्}
{न भयाद्विनिवर्तेत तस्य लोका यथा मम}


\twolineshloka
{द्वीपिचर्मावृतैः खङ्गैर्बाहुभिः परिघोपमैः}
{यस्य वेदिरुपस्तीर्णा तस्य लोका यथा मम}


\twolineshloka
{यस्तु नापेक्षते कंचित्सहायं विषमे स्थितः}
{विगाह्य वाहिनीमध्यं तस्य लोका यथा मम}


\twolineshloka
{यस्य शोणितसंघट्टा भेरीमण्डूककच्छपा}
{वीरास्थिशर्करा दुर्गा मांसशोणितकर्दमा}


\twolineshloka
{असिचर्मप्लवा घोरा केशशैवलशाद्वला}
{अश्वनागरथैश्चैव संछिन्नैः कृतसंक्रमा}


\twolineshloka
{पताकाध्वजवानीरा हतवाहनवारणा}
{शोणितोदकसंपूर्णा दुस्तरा पारगैर्नरैः}


\twolineshloka
{रहतनागमहानक्रा परलोकवहाऽशिवा}
{ऋष्टिखङ्गमहामीना गृध्रकङ्कबलप्लवा}


\twolineshloka
{पुरुषादानुचरिता भीरूणां कश्मलावहा}
{नदी योधस्य संग्रामे तदस्यावभृथं नृप}


\twolineshloka
{वेदिर्यस्य त्वमित्राणां शिरोभिर्व्यवकीर्यते}
{अश्वस्कन्धैर्गजस्कन्धैस्तस्य लोका यथा मम}


\twolineshloka
{पत्नी शालाकृता यस्य परेषां वाहिनीमुखम्}
{हविर्धानं स्ववाहिन्यास्तदस्याहुर्मनीषिणः}


\twolineshloka
{सदस्या दक्षिणा योधा आग्नीध्रश्चोत्तरां दिशम्}
{शत्रुसेना अलत्रस्य सर्वलोकानदूरतः}


\twolineshloka
{यस्य भयतो व्यूहे भवत्याकाशमग्रतः}
{सास्य वेदिस्तदा यज्ञैर्नित्यं व्यूहास्त्रयोऽग्नयः}


\twolineshloka
{यस्तु योधः परावृत्तः संत्रस्तो हन्यते परैः}
{अप्रतिष्ठः स नरकं याति नास्त्यत्र संशयः}


\twolineshloka
{यस्य शोणितवेगेण वेदिः स्यात्संपरिप्लुता}
{केशमांसास्थिसंपूर्णा स गच्छेत्परमां गतिम्}


\twolineshloka
{यस्तु सेनापतिं हत्वा तद्यानमधिरोहति}
{स विष्णुविक्रमक्रामी बृहस्पतिसमः प्रभुः}


\twolineshloka
{नायकं तत्कुमारं वा यो वा स्यात्तत्र पूजितः}
{जीवग्राहं प्रगृह्णाति तस्य लोका यथा मम}


\twolineshloka
{आहवे तु हतं शूरं न शोचेत कथंचन}
{अशोच्यो हि हतः शूरः स्वर्गलोके महीयते}


\twolineshloka
{न ह्यन्नं नोदकं तस्य न स्नानं नाप्यशौचकम्}
{हतस्य कर्तुमिच्छन्ति तस्य लोकाञ्शृणुष्व मे}


\twolineshloka
{वराप्सरः सहस्राणि शूरमत्योधने हतम्}
{त्वरमाणानि धावन्ति मम भर्ता भवेदिति}


\twolineshloka
{एतत्तपश्च पुण्यं च धर्मश्चैव सनातनः}
{चत्वारश्चाश्रमास्तस्य यो युद्धे न पलायते}


\twolineshloka
{वृद्धबालौ न हन्तव्यौ न च स्त्री नैव पृष्ठतः}
{तृणपूर्णमुखश्चैव तवास्मीति च यो वदेत्}


\twolineshloka
{अहं वृत्रं बलं पाकं महाकायं विरोचनम्}
{दुरावारं च नमुचिं शतमायं च शम्बरम्}


\twolineshloka
{विप्रचित्तिं च दैतेयं दनोः पुत्रांश्च सर्वशः}
{प्रह्वादं च निहत्याजौ ततो देवाधिपोऽभवम्}


\twolineshloka
{इत्येतच्छक्रवचनं निशम्य प्रतिपूज्य च}
{योधानामात्मनः सिद्धिमम्बरीषोऽभिपन्नवान्}


\chapter{अध्यायः ९९}
\twolineshloka
{भीष्म उवाच}
{}


\twolineshloka
{अत्राप्युदाहरन्तीम-तिहासं पुरातनम्}
{प्रतर्दनो मैथिलश्च संग्रामं यत्र चक्रतुः}


\twolineshloka
{यज्ञोपवीती संग्रामे जनको मिथिलाधिपः}
{योधानुद्धर्षयामास तन्निबोध युधिष्ठिर}


\twolineshloka
{जनको मैथिलो राजा महात्मा सर्वतत्त्ववित्}
{योधानां दर्शयामास स्वर्गं नरकमेव च}


\twolineshloka
{अभीरूणामिमे लोका भास्वन्तो हन्त पश्यत}
{पूर्णा गन्धर्वकन्याभिः सर्वकामदुहोऽक्षयाः}


\twolineshloka
{इमे पलायमानानां नरकाः प्रत्युपस्थिताः}
{अकीर्तिः शाश्वती चैव यतितव्यमनन्तरम्}


\twolineshloka
{तान्दृष्ट्वाऽरीन्विजयत भूत्वा संत्यागबुद्धयः}
{नरकस्याप्रतिष्ठस्य मा भूत वशवर्तिनः}


\twolineshloka
{त्यागमूलं हि शूराणां स्वर्गद्वारमनुत्तमम्}
{इत्युक्तास्ते नृपतिना योधाः परपुरंजय}


\twolineshloka
{अजयन्त रणे शत्रून्हर्षयन्तो नरेश्वरम्}
{तस्मात्त्यक्तात्मना नित्यं स्थातव्यं रणमूर्धनि}


\twolineshloka
{गजानां रथिनो मध्ये रथानामनुसादिनः}
{सादिनामन्तरे स्थाप्यं पादातमपि दंशितम्}


\twolineshloka
{य एवं व्यूहते राजा स नित्यं जयते रिपून्}
{तस्मदितद्विधातव्यं नित्यमेव युधिष्ठिर}


\twolineshloka
{स्वर्गे सुकृतमिच्छन्तः सुयुद्धेनातिमन्यवः}
{क्षोभयेयुरनीकानि सागरं मकरा यथा}


\twolineshloka
{हर्षयेयुर्विषण्णांश्च व्यवस्थाप्य परस्परम्}
{तेषां च भूमिं रक्षेयुर्भग्नान्नात्यनुसारयेत्}


\twolineshloka
{पुनरावर्तमानानां निराशानां च जीविते}
{वेगः सुदुःसहो राजंस्तस्मान्नात्यनुसारयेत्}


\twolineshloka
{न हि प्रहर्तुमिच्छन्ति शूराः प्रद्रवतो भयात्}
{तस्मात्पलायमानानां कुर्यान्नात्यनुसारणम्}


\twolineshloka
{चराणामचरा ह्यन्नमदंष्ट्रा दंष्ट्रिणामपि}
{अपाणयः पाणिमतामन्नं शूरस्य कातराः}


\twolineshloka
{समानपृष्ठोदरपाणिपादाःपश्चाच्छरं भीरवोऽनुव्रजन्ति}
{अतो भयार्ताः प्रणिपत्य भूयःकृत्वाञ्जलीनुपतिष्ठन्ति शूरान्}


\twolineshloka
{शूरबाहुषु लोकोऽयं लम्बते पुत्रवत्सद}
{तस्मात्सर्वेषु लोकेषु शूरः संमानमर्हति}


\twolineshloka
{न हि शौर्यात्परं किंचिन्त्रिषु लोकेषु विद्यते}
{शूरः सर्वं पालयति सर्वं शूरे प्रतिष्ठितम्}


\chapter{अध्यायः १००}
\twolineshloka
{युधिष्ठिर उवाच}
{}


\threelineshloka
{यथा जयार्थिनः सेनां नयन्ति भरतर्षभ}
{ईषद्धर्मं प्रपीड्यापि तन्मे ब्रूहि पितामह ॥भीष्म उवाच}
{}


\twolineshloka
{सन्त्येव हि स्थिता धर्म उपपत्त्या तथा परे}
{साध्वाचारतया केचित्तथैवौपयिकादपि}


\twolineshloka
{उपायधर्मान्वक्ष्यामि संसिद्धानर्थसिद्धये}
{निर्मर्यादा दस्यवस्तु भवन्ति परिपन्थिनः}


\twolineshloka
{तेषां प्रतिविघातार्थं प्रवक्ष्याम्यथ नैगमम्}
{कार्याणां संप्रसिद्ध्यर्थं तानुपायान्निबोध मे}


\twolineshloka
{उभे प्रज्ञे वेदितव्ये ऋज्वी वक्रा च भारत}
{जानन्वक्रां न सेवेत प्रतिबाधेत चागताम्}


\twolineshloka
{अमित्रा एव राजानं भेदेनोपचरन्त्युत}
{तां राजा विकृतिं जानन्यथाऽमित्रान्प्रबाधते}


\twolineshloka
{गजानां पार्थ वर्माणि गोवृषाजगराणि च}
{शल्यकण्टकलोहानि तनुत्राणि मतानि च}


\twolineshloka
{शातपीतानि शस्त्राणि सन्नाहाः पीतलोहकाः}
{नानारञ्जनरक्ताः स्युः पताकाः केतवश्च ह}


\twolineshloka
{ऋष्टयस्तोमराः खङ्गा निशिताश्च परश्वथाः}
{फलकान्यथ चर्माणि प्रतिकल्प्यान्यनेकशः}


\twolineshloka
{अभिनीतानि शस्त्राणि योधाश्च कृतनिश्चयाः}
{चैत्रे वा मार्गशीर्षे वा सेनायोगः प्रशस्यते}


\twolineshloka
{पक्वसस्या हि पृथिवी भवत्यम्बुमती तदा}
{नैवातिशीतो नात्युष्णः कालो भवति भारत}


\twolineshloka
{तस्मात्तदा योजयेत परेषां व्यसनेऽथवा}
{एते हि योगाः सेनायाः प्रशस्ताः परबाधने}


\twolineshloka
{जलवांस्तृणवान्मार्गः समो गम्यः प्रशस्यते}
{चारैः सुविदिताभ्यासः कुशलैर्वनगोचरैः}


\twolineshloka
{न ह्यरण्यानि शक्यन्ते गन्तुं मृगगणैरिव}
{तस्मात्सेनासु तानेव योजयन्ति जयार्थिनः}


\twolineshloka
{[अग्रतः पुरुषानीकं शक्तं चापि कुलोद्भवम्}
{]आवासस्तोयवान्मार्गः पर्याकाशः प्रशस्यते}


\twolineshloka
{पोषामपसर्पाणां प्रतिघातस्तथा भवेत्}
{आकाशं हि वनाभ्याशे मन्यन्ते गुणवत्तरम्}


\twolineshloka
{बहुभिर्गुणजातैश्च ये युद्धकुशला जनाः}
{[उपन्यासो भवेत्तत्र बलानां नातिदूरतः ॥]}


\twolineshloka
{उपन्यासोऽपसर्पाणां पदातीनां च गूहनम्}
{हतशत्रुप्रतीघातमापदर्थं परायणम्}


\twolineshloka
{सप्तर्पीन्पृष्ठतः कृत्वा युध्येयुरचला इव}
{अनेन विधिना शत्रूञ्जिगीषेतापि दुर्जयान्}


\twolineshloka
{यतो वायुर्यतः सूर्यो यतः सोमस्ततो जयः}
{पूर्वंपूर्वं ज्याय एषां सन्निपाते युधिष्ठिर}


\twolineshloka
{अकर्द -मनुदकाममर्यादामलोष्टकाम्}
{अश्वभूमिं प्रशंसन्ति ये युद्धकुशला जनाः}


\twolineshloka
{समा निरुदकाकाशा रथभूमिः प्रशस्यते}
{नीचद्रुमा महाकक्षा सोदका हस्तियोधिनाम्}


\twolineshloka
{बहुदुर्गा महाकक्षा वेणुवेत्रतिरस्कृता}
{पदातीनां क्षमा भूमिः पर्वतोपवनानि च}


\twolineshloka
{पदातिबहुला सेना दृढा भवति भारत}
{रथाश्वबहुला सेना सुदिनेषु प्रशस्यते}


\twolineshloka
{पदातिनागबहुला प्रावृट््काले प्रशस्यते}
{गुणानेतान्प्रसंख्याय देशकालौ प्रयोजयेत्}


\twolineshloka
{एवं संचिन्त्य यो याति तिथिनक्षत्रपूजितः}
{विजयं लभते नित्यं सेनां सम्यक्प्रयोजयन्}


\threelineshloka
{प्रसुप्तांस्तृषिताञ्श्रान्तान्प्रकीर्णान्नाभिघातयेत्}
{मोक्षे प्रयाणे चलने पानभोजनकालयोः}
{अतिक्षिप्तान्व्यतिक्षिप्तान्निहतान्प्रतनूकृतान्}


\twolineshloka
{अविस्रब्धान्कृतारम्भानुपन्यासात्प्रतापितान्}
{बहिश्वरानुपन्यासान्कृतवेश्मानुसारिणः}


\twolineshloka
{पारम्पर्यागते द्वारे ये केचिदनुवर्तिनः}
{परिचर्यापरोद्धारो ये च केचन वल्गिनः}


\twolineshloka
{अनीकं ये विभिदन्ति भिन्नं संस्थापयन्ति च}
{समानाशनपानास्ते कार्या द्विगुणवेतनाः}


\threelineshloka
{`जातिगोत्रं च विज्ञाय कर्म चानुत्तमं शुभम्}
{समानदेहरक्षार्थे कार्या द्विगुणवेतनाः}
{त्रिगुणं चतुर्गुणं चैव वेतनं तेषु कारयेत् ॥'}


\twolineshloka
{दशाधिपतयः कार्याः शताधिपतयस्तथा}
{ततः सहस्राधिपतिं कुर्याच्छूरमतन्द्रितम्}


\twolineshloka
{यथा मुख्यान्सन्निपात्य वक्तव्याः संशयामहे}
{यथा जयार्थं संग्रामे न जह्याम परस्परम्}


\twolineshloka
{इहैव ते निवर्तन्तां ये च केचन भीरवः}
{न घातयेयुः प्रदरं कुर्वाणास्तुमुले सति}


% Check verse!
[न सन्निपाते प्रदरं वधं वा कुर्युरीदृशाः ॥]आत्मानं च स्वपक्षं च पालयन्हन्ति संयुगे
\twolineshloka
{अर्थनाशो वधोऽकीर्तिरयशश्च पलायने}
{अमनोज्ञाऽसुखा वाचः पुरुषस्य पलायतः}


\threelineshloka
{प्रतिध्वस्तोष्ठदन्तस्य न्यस्तसर्वायुधस्य च}
{`हित्वा पलायमानस्य सहायान्प्राणसंशये}
{'अमित्रैरवरुद्धस्य द्विषतामस्तु नः सदा}


\twolineshloka
{मनुष्यापसदा ह्येते ये भवन्ति पराङ्भुखाः}
{राशिवर्धनमात्रास्ते नैव ते प्रेत्य नो इह}


\twolineshloka
{अमित्रा हृष्टमनसः प्रत्युद्यान्ति पलायिनम्}
{जयिनस्तु नरास्तात मङ्गलैर्वन्दनेन च}


\twolineshloka
{यस्य स्म व्यसने राजन्ननुमोदन्ति शत्रवः}
{तदसह्यतरं दुःखं मन्यन्ते मरणादपि}


\twolineshloka
{श्रियं जानीत धर्मस्य मूलं सर्वसुखस्य च}
{या भीरूणां पराख्यातिः शूरस्तामधिगच्छति}


\twolineshloka
{ते वयं स्वर्गमिच्छन्तः संग्रामे त्यक्तजीविताः}
{जयन्तो वध्यमाना वा प्राप्नुयाम च सद्गतिम्}


\twolineshloka
{एवं संशप्तशपथाः समभित्यक्तजीविताः}
{अमित्रवाहिनीं वीराः प्रतिगाहन्त्यभीरवः}


\twolineshloka
{अग्रतः पुरुषाऽनीकमसिचर्मवतां भवेत्}
{पृष्ठतः शकटानीकं कलत्रं मध्यतस्तथा}


\twolineshloka
{परेषां प्रतिघातार्थं पदातीनां च गूहनम्}
{अपि तस्मिन्पुरे वृद्धा भवेयुर्ये पुरोगमाः}


\twolineshloka
{ये पुरस्तादभिमताः सत्ववन्तो मनस्विनः}
{ते पूर्वमभिवर्तेरंश्चैतानेवेतरे जनाः}


\twolineshloka
{अपि चोद्धर्षणं कार्यं भीरूणामपि यत्नतः}
{स्कन्धदर्शनमात्रात्तु तिष्ठेयुर्वा समीपतः}


\twolineshloka
{संहतान्योधयेदल्पान्कामं विस्तारयेद्बहून्}
{सूचीमुखमनीकं स्यादल्पानां बहुभिः सह}


\twolineshloka
{संप्रयुक्ते निकृष्टे वा सत्यं वा यदि वाऽनृतम्}
{प्रगृह्य बाहून्क्रोशेत हन्त भग्नाः परे इति}


\twolineshloka
{आगतं मे मित्रबलं प्रहरध्वमभीतवत्}
{सत्ववन्तो निधावेयुः कुर्वन्तो भैरवान्रवान्}


\twolineshloka
{क्ष्वेडाः किलकिलाशब्दाः क्रकचा गोविषाणिकाः}
{भेरीमृदङ्गपणवान्नादयेयुश्च जर्झरान्}


\chapter{अध्यायः १०१}
\twolineshloka
{युधिष्ठिर उवाच}
{}


\threelineshloka
{किंशीलाः किंसमुत्थानाः कथंरूपाश्च भारत}
{किंसन्नाहाः कथंशस्त्रा जनाः स्युः संयुगे नृपाः ॥भीष्म उवाच}
{}


\twolineshloka
{यथाचरितमेवात्र शस्त्रं पत्रं विधीयते}
{आचाराद्धीह पुरुषस्तथा कर्मसु वर्तते}


\twolineshloka
{गान्धाराः सिन्धुसौवीरा नखरप्रासयोधिनः}
{अभीरवः सुबलिनस्तद्वलं सर्वपारगम्}


\twolineshloka
{सर्वशस्त्रेषु कुशलाः सत्ववन्तो ह्युशीनराः}
{प्राच्या मातङ्गयुद्धेषु कुशलाः कूटयोधिनः}


\threelineshloka
{तथा यवनकाम्भोजा मधुरामभितश्च ये}
{एतेऽश्वयुद्धकुशला दाक्षिणात्याऽसिचर्मिणः}
{सर्वत्र शूरा जायन्ते महासत्वा महाबलाः}


\twolineshloka
{आवन्तिका महाशूराश्चतुरङ्गे च मालवाः}
{एकोऽपि हि सहस्रस्य तिष्ठत्यभिमुखो रणे}


% Check verse!
प्रायो देशाः समुद्दिष्टा लक्षणानि तु मे शृणु
\threelineshloka
{सिंहशार्दूलवाङ्गेत्राः सिंहशार्दूलगामिनः}
{पारावतकुलिङ्गाक्षाः सर्वे शूराः प्रमाथिनः}
{}


\twolineshloka
{मृगस्वरा द्वीपिनेत्रा ऋषभाक्षास्तथा परे}
{प्रमाथिनश्च मन्द्राश्च क्रोधनाः किङ्किणीस्वनाः}


\twolineshloka
{मेघस्वनाः क्रूरमुखाः केचिच्च कलनिस्वनाः}
{जिह्मनासाग्रजिह्वाश्च दूरगा दूरपातिनः}


\twolineshloka
{बिडालकुब्जाः स्तब्धाक्षास्तनुकेशास्तनुत्वचः}
{शीघ्राश्चपलचित्ताश्च ते भवन्ति दुरासदाः}


\twolineshloka
{गौरा निमीलिताः केचिन्मृदुप्रकृतयस्तथा}
{तुरङ्गगतिनिर्घोषास्ते नराः पारयिष्णवः}


\twolineshloka
{सुसंहताः प्रतनवो व्यूढोरस्काः सुसंस्थिताः}
{प्रवादितेषु कुप्यन्ति हृष्यन्ति कलद्देषु च}


\twolineshloka
{गम्भीराक्षा निसृष्टाक्षाः पिङ्गाक्षा भ्रुकुटीमुखाः}
{नकुलाक्षास्तथा चैव सर्वे शूरास्तनुत्यजः}


\twolineshloka
{जिह्नाक्षाः प्रललाटाश्च निर्मांसहनवोऽव्यथाः}
{वक्रबाह्वङ्गुलीसक्थाः कृशा धमनिसंतताः}


\twolineshloka
{प्रविशन्ति च वेगेन सांपराये ह्युपस्थिते}
{वारणा इव संमत्तास्ते भवन्ति दुरासदाः}


\twolineshloka
{दीप्तस्फुटितकेशान्ताः स्थूलपार्श्वहनूमुखाः}
{उन्नतांसाः पृथुग्रीवा विकटाः स्थूलपिण्डिकाः}


\twolineshloka
{उद्बन्धा इव सुग्रीवा विनताविहगा इव}
{पिण्डशीर्षातिवक्राश्च पृषदंशमुखास्तथा}


\twolineshloka
{अग्रस्वरा मन्युमन्तो युद्धेष्वारावसारिणः}
{अधर्मज्ञाऽवलिप्ताश्च घोरा रौद्रप्रदर्शनाः}


\twolineshloka
{त्यक्तात्मानः सर्व एते उदग्रा ह्यनिवर्तिनः}
{पुरस्कार्याः सदा सैन्ये हन्यन्ते घ्नन्ति चापि ते}


\twolineshloka
{अधार्मिका भिन्नवृत्ताः सान्त्वेनैषां पराभवः}
{एवमेव प्रदूष्यन्ते राज्ञोऽप्येते ह्यभीक्ष्णशः}


\chapter{अध्यायः १०२}
\twolineshloka
{युधिष्ठिर उवाच}
{}


\threelineshloka
{जयिन्याः कानि रूपाणि भवन्ति भरतर्षभ}
{पृतनायाः प्रशस्तानि तानि चेच्छामि वेदितुम् ॥भीष्म उवाच}
{}


\twolineshloka
{जयिन्या यानि रूपाणि भवन्ति भरतर्षभ}
{पृतनायाः प्रशस्तानि तानि वक्ष्यामि सर्वशः}


\twolineshloka
{दैवे पूर्वं प्रकुपिते मानुषे कालचोदिते}
{तद्विद्व अऽनुपस्यन्ति ज्ञानदीर्घेण चक्षुषा}


\twolineshloka
{प्रायश्चित्तविधिं चात्र जपहोमांश्च तद्विदः}
{मङ्गल नि च कुर्वन्ति शमयन्त्यहितानि च}


\twolineshloka
{उदीणानसो योधा वाहनानि च भारत}
{यस्यां भवन्ति सेनायां ध्रुवं तस्या जयो भवेत्}


\twolineshloka
{अन्वेव वायवो यान्ति तथैवेन्द्रधनूंषि च}
{अनुप्लवन्ते मेघाश्च तथाऽऽदित्यस्य रश्मयः}


\twolineshloka
{गोमायवश्चातुलोमबला गृध्राश्च सर्वशः}
{अर्हयेयुर्यदा सेनां तदा सिद्धिरनुत्तमा}


\twolineshloka
{प्रसन्नभाः पावकश्चोर्ध्वरश्मिःप्रदक्षिणावर्तशिखो विधूमः}
{पुण्या गन्धाश्चाहुतीनां भवन्तिजयस्यैतद्भाविनो रूपमाहुः}


\twolineshloka
{गम्भीरशब्दाश्च महास्वनाश्चशङ्ख्याश्च भेर्यश्च नदन्ति यत्र}
{युयुत्सवश्चाप्रतीपा भवन्तिजयस्यैतद्भाविनो रूपमाहुः}


\twolineshloka
{इष्टा मृगाः पृष्ठतो वामतश्चसंप्रस्थितानां च गमिष्यतां च}
{जिघांसतां दक्षिणाः सिद्धिमाहुर्ये त्वग्रतस्ते प्रतिषेधयन्ति}


\twolineshloka
{माङ्गल्यशब्दाञ्शकुना वदन्तिहंसाः क्रौञ्चाः शतपत्राश्च चाषाः}
{हृष्टा योधाः सत्ववन्तो भवन्तिजयस्यैतद्भाविनो रूपमाहुः}


\twolineshloka
{शस्त्रैर्यन्त्रैः कवचैः केतुभिश्चसुभानुभिर्मुखवर्णैश्च यूनाम्}
{भ्राजिष्मती दुष्प्रतिवीक्षणीयायेषां चमूस्तेऽभिभवन्ति शत्रून्}


\twolineshloka
{शुश्रूषवश्चानभिमानिनश्चपरस्परं सौहृदमास्थिताश्च}
{येषां योधाः शौर्यमनुष्ठिताश्चजयस्यैतद्भाविनो रुपमाहुः}


\twolineshloka
{शब्दाः स्पर्शास्तथा गन्धा विचरन्ति मनः प्रियाः}
{धैर्यं चाविशते योधान्विजयस्य मुखं च तत्}


\twolineshloka
{शब्दो वामः प्रस्थितस्य दक्षिणः प्रविविक्षतः}
{पश्चात्संसाधयत्यर्थं पुरस्ताच्च निषेधति}


\twolineshloka
{संहत्य महतीं सेनां चतुरङ्गां युधिष्ठिर}
{साम्नैव वर्तयेः पूर्वं प्रसतेथास्ततो युधि}


\twolineshloka
{जघन्य एष विजयो यद्युद्धे सामभाषणम्}
{यादृच्छिको युधि जयो दैवेनेति विचारणम्}


\twolineshloka
{आपगेव महावेगा त्रस्ता इव महामृगाः}
{दुर्निवार्यतमा चैव प्रभग्ना महती चमूः}


\twolineshloka
{भग्ना इत्येव भज्यन्ते विद्वांसोऽपि न कारणम्}
{उदारसारा महती रुरुसंघोपमा चमूः}


\twolineshloka
{परस्परज्ञाः संहृष्टास्त्यक्तप्राणाः सुनिश्चिताः}
{अपि पञ्चत्रतै शूरा निघ्नन्ति परवाहिनीम्}


\twolineshloka
{अपि वा पञ्चषट््सप्तसंहिताः कृतनिश्चयाः}
{क्ललीनाः पूजिताः सम्यग्विजयन्तीह शात्रवान्}


\twolineshloka
{सन्निपातो न मन्तव्यः शक्ये सति कथंचन}
{सान्त्वभेदप्रदानानां युद्धमुत्तरमुच्यते}


\twolineshloka
{संसर्पेण हि सेनाया भयं भीरून्प्रबाधते}
{वज्रादिय प्रज्वलितादियं स्वित्क्षपयिष्यति}


\twolineshloka
{अमिप्रयातां समितिं ज्ञात्वा ये प्रतियान्त्यथ}
{तेषां सन्दन्ति गात्राणि योधानां विवयस्य च}


\twolineshloka
{विषयो व्यथते राजन्सर्वः सस्थाणुजङ्गमः}
{अस्त्रप्रतापतप्तानां मज्जाः सीदन्ति देहिनाम्}


\twolineshloka
{तेषां सान्त्वं क्रूरमिश्रं प्रणेतव्यं पुनः पुनः}
{संपीड्यमाना हि परैर्योगमायान्ति सर्वतः}


\twolineshloka
{आन्तराणां च भेदार्थं चरानभ्यवचारयेत्}
{यश्च तस्मात्परो राजा तेन संधिः प्रशस्यते}


\twolineshloka
{न हि तस्यान्यथा पीडा शक्या कर्तुं तथाविधा}
{यथा सार्धममित्रेण सर्वतः प्रतिबाधनम्}


\twolineshloka
{क्षमा वै साधुमायाति न ह्यसाधून्क्षमा सदा}
{क्षमायाश्चाक्षमायाश्च पार्थ विद्धि प्रयोजनम्}


\twolineshloka
{विजित्य क्षममाणस्य यशो राज्ञो विवर्धते}
{महापराधे ह्यप्यस्मिन्विश्वसन्त्यपि शत्रवः}


\twolineshloka
{मन्यन्ते कर्षयित्वा तु क्षमा साध्वीति शाम्बराः}
{असंतप्तं तु यद्दारु प्रत्येति प्रकृतिं पुनः}


\twolineshloka
{नैतत्प्रशंसन्त्याचार्या न चैतत्साधु दर्शनम्}
{अक्रोधेनाविनाशेन नियन्तव्याः स्वपुत्रवत्}


\twolineshloka
{द्वेष्यो भवति भूतानामुग्रो राजा युधिष्ठिर}
{मृदुमप्यवमन्यन्ते तस्मादुभयभाग्भवेत्}


\twolineshloka
{प्रहरिष्यन्प्रियं ब्रूयात्प्रहरन्नपि भारत}
{प्रहृत्य च प्रियं ब्रूयाच्छोचन्निव रुदन्निव}


\twolineshloka
{न मे प्रिया ये स्म हताः संप्रहृष्टाः परेऽपि च}
{न च कत्थनमेवाग्र्यमुच्यमानं पुनः पुनः}


\twolineshloka
{अहो जीवितमाकाङ्क्षेन्नेदृशो वधमर्हति}
{सुदुर्लभाः सुपुरुषाः संग्रामेष्वपलायिनः}


\twolineshloka
{कृतं ममाप्रियं तेन येनायं निहतो मृधे}
{इति वाचा वदन्हन्तृन्पूजयेत रहोगतः}


\twolineshloka
{हन्तॄणां च हतानां च पूजां कुर्याद्यथार्थतः}
{क्रोशेद्बाहुं प्रगृह्यापि चिकीर्षञ्जनसंग्रहम्}


\twolineshloka
{एवं सर्वास्ववस्थासु सान्त्वपूर्वं समाचरेत्}
{प्रियो भवति भूतानां धर्मज्ञो वीतभीर्नृपः}


\twolineshloka
{विश्वासं चात्र गच्छन्ति सर्वभूतानि भारत}
{विश्वस्तः शक्यते भोक्तुं यथाकालं समुत्थितः}


\twolineshloka
{तस्माद्विश्वासयेद्राजा सर्वभूतान्यमायया}
{सर्वतः परिरक्षेच्च यो महीं भोक्तुमिच्छति}


\chapter{अध्यायः १०३}
\twolineshloka
{युधिष्ठिर उवाच}
{}


\threelineshloka
{कथं मृदौ कथं तीक्ष्णे महापक्षे च भारत}
{अरौ वर्तेत नृपतिस्तन्मे ब्रूहि पितामह ॥भीष्म उवाच}
{}


\twolineshloka
{अत्राप्युदाहरन्तीममितिहासं पुरातनम्}
{बृहस्पतेश्च संवादमिन्द्रस्य च युधिष्ठिर}


\twolineshloka
{बृहस्पतिं देवपतिरभिवाद्य कृताञ्जलिः}
{उपसंगम्य पप्रच्छ वासवः परवीरहा}


\twolineshloka
{अहितेषु कथं ब्रह्मन्प्रवर्तेयमतन्द्रितः}
{असमुछिद्य चैवैतान्नियच्छेयमुपायतः}


\twolineshloka
{सेनयोर्थ्यतिषङ्गे च जयः साधारणो भवेत्}
{किं कुर्वाणं न मां जह्याज्ज्वलिता श्रीः प्रतापिनी}


\twolineshloka
{ततो र्मार्थकामानां कुशलः प्रतिभानवान्}
{राजधर्मविधानज्ञः प्रत्युवाच पुरंदरम्}


\twolineshloka
{न जा कलहेनेच्छेन्नियन्तुमपकारिणः}
{बालैर सवितं ह्येतद्यदमर्षो यदक्षमा}


% Check verse!
न शत्रुर्विवृतः कार्यो वधमस्याभिकाङ्क्षता
\twolineshloka
{क्रोधं भयं च हर्षं च नियम्य स्वयमात्मनि}
{अमित्र पसेवेत विश्वस्तवदविश्वसन्}


\twolineshloka
{प्रियमेव वदेन्नित्यं नाप्रियं किंचिदाचरेत्}
{विरमेच्छुष्कवैरेभ्यः कर्णजापं च वर्जयेत्}


\threelineshloka
{यथा वैतंसिको युक्तो द्विजानां सदृशस्वरः}
{तान्द्विजान्कुरुते वश्यांस्तथायुक्तो महीपतिः}
{वशं चोपनयेच्छत्रून्निहन्याच्च पुंरदर}


\twolineshloka
{न नित्यं परिभूयारीन्सुखं स्वपिति वासव}
{जागर्त्येव हि दुष्टात्मा संकरेऽग्निरिवोत्थितः}


\twolineshloka
{न सन्निपातः कर्तव्यः सामान्ये विजये सति}
{विश्वास्यैवोपसंनम्यो वशे कृत्वा रिपुः प्रभो}


\threelineshloka
{संप्रधार्य सहामात्यैर्मन्त्रविद्भिर्महात्मभिः}
{उपेक्ष्यमाणो विज्ञातो हृदयेनापराजितः}
{अथास्य प्रहरेत्काले विधेर्वित्तलितो यदा}


% Check verse!
दण्डं च दूषयेदस्य पुरुषैराप्तकारिभिः
\twolineshloka
{आदिमध्यावसानज्ञान्प्रच्छन्नं च विचारयेत्}
{बलानि दूषयेदस्य जानन्नेव प्रमाणतः}


\twolineshloka
{भेदेनोपप्रदानेन संसृजेदौषधैस्तथा}
{न त्वेव खलु संसर्गं रोचयेदरिभिः सह}


\twolineshloka
{दीर्घकालमपीक्षेत्त निग्राह्या एव शत्रवः}
{कालकाङ्गी च युक्तः सन्नुपासीत शचीपते}


\threelineshloka
{तथा प्रियं च वक्तव्यं यथा विस्रम्भमाप्नुयात्}
{न सद्योऽरीन्विहन्याच्च द्रष्टव्यो विजयो ध्रुवः}
{भूयः शल्यं घटयति नवं च कुरुते व्रणम्}


\twolineshloka
{प्राप्ते च प्रहरेत्काले न च संवर्तते पुनः}
{हन्तुकामस्य देवेन्द्र पुरुषस्य रिपून्प्रति}


\twolineshloka
{यं हि कालो व्यतिक्रामेत्पुरुषं कालकाङ्क्षिणम्}
{दुर्लभः स पुनस्तेन कालः कर्म चिकीर्षता}


\twolineshloka
{औजस्यं जनयेदेव संगृह्णन्साधुसंमतम्}
{कालेन साधयेत्कृत्यमप्राप्तो न हि पीडयेत्}


\twolineshloka
{विहाय कामं क्रोधं च तथाऽहंकारमेव च}
{युक्तो विवरमन्विच्छेदहितानां सदा नृपः}


\twolineshloka
{मार्दवं दण्ड आलस्यं प्रमादश्च सुरोत्तम}
{मायाः सुविहिताः शक्र शातयन्त्यविचक्षणम्}


\twolineshloka
{निहत्यैतानि चत्वारि मायां प्रतिविधाय च}
{ततः शक्नोति शत्रूणां प्रहर्तुमविचारयन्}


\twolineshloka
{यदेवैतेन शक्येत गुह्यं कर्तुं तदाऽऽचरेत्}
{यच्छन्ति सतिवा गुह्यं मिथो विश्रावयन्त्यपि}


\twolineshloka
{अशक्यमिति कृत्वा वा ततोऽन्यैः संविदं चरेत्}
{ब्रह्मदण्डमदृष्टेषु दृष्टेषु चतुरङ्गिणीम्}


\twolineshloka
{भेदं च प्रथमं विद्यात्तूष्णीं दण्डं तथैव च}
{काले प्रयोजयेद्राजा तस्मिंस्तस्मिंस्तदातदा}


\twolineshloka
{प्रणिपातं च गच्छेत काले शत्रोर्बलीयसः}
{युक्तोऽस्य वधमन्विच्छेदप्रमत्तः प्रमाद्यतः}


\twolineshloka
{प्रणिपातेन दानेन वाचा मधुरया ब्रुवन्}
{अमित्रमुपसेवेत न च जातु विशङ्कयेत्}


\twolineshloka
{स्थानानि शङ्कितानां च नित्यमेव विवर्जयेत्}
{न च तेष्वाश्वसेद्राजा जाग्रतीह निराकृताः}


\twolineshloka
{न ह्यतो दुष्करं कर्म किंचिदस्ति सुरोत्तम}
{यथा विविधवृत्तानामैश्वर्यममराधिप}


\twolineshloka
{तथा विविधशीलानामपि संभव उच्यते}
{प्रयतेद्योगमास्थाय मित्रामित्रानधारयन्}


\twolineshloka
{मृदुमप्यवमन्यन्ते तीक्ष्णादुद्विजते जनः}
{मातीक्ष्णो मा मृदुर्भूस्त्वं तीक्ष्णो भव मृदुर्भव}


\twolineshloka
{यथा वप्रे वेगवति सर्वतः संप्लतोदके}
{नित्यं विचरणाद्वाधस्तथा राज्यं प्रमाद्यतः}


\twolineshloka
{न बहूनुपरुध्येत यौगपद्येन शात्रवान्}
{साम्ना दानेन भेदेन दण्डेन च पुरंदर}


\twolineshloka
{एकैकमेषां निष्पिष्य शिष्टेषु निपुणं चरेत्}
{न तु शक्तोऽपि मेधावी सर्वानेवाचरेद्बुधः}


\twolineshloka
{यदा स्यान्महती सेना हयनागरथाकुला}
{पदातियन्त्रबहुला अनुरक्ता षडङ्गिनी}


\twolineshloka
{यदा बहुविधां वृद्धिं मन्येत प्रतियोगतः}
{तदा विवृत्य प्रहरेद्दस्यूनामविचारयन्}


\twolineshloka
{न साम दण्डोपनिषत्प्रशस्यतेन मार्दवं शत्रुषु यात्रिकं सदा}
{न सस्यघातो न च संकरक्रियान चापि भूयः प्रकृतेर्विचारणा}


\twolineshloka
{मायाविभेदानुपसर्जनानिवाचं तथैव प्रथमं प्रयोगात्}
{आप्तैर्मनुष्यैरुपचारयेतपुरेषु राष्ट्रेषु च संप्रयुक्तान्}


% Check verse!
पुराऽपि चैताननुसृत्य भूमिपाःपुरेषु भोगानखिलाञ्जयन्तिपुरेषु नीतिं विहितां यथाविधिप्रयोजयन्तो बलवृत्रसूदन
\twolineshloka
{प्रदाय गूढानि वसूनि नामप्रच्छिद्य भोगानपहाय च स्वान्}
{दुष्टाः स्वदोषैरिति कीर्तयित्वापुरेषु राष्ट्रेषु च योजयन्ति}


\threelineshloka
{तथैव चान्यैरपि शास्त्रवेदिभिःस्वलंकृतैः शास्त्रविधानलिङ्गितैः}
{सुशिक्षितैर्भाष्यकथाविशारदैःपरेषु कृत्यामुपधारयेच्च ॥इन्द्र उवाच}
{}


\threelineshloka
{कानि लिङ्गानि दुष्टस्य भवन्ति द्विजसत्तम}
{कथं दुष्टं विजानीयादेतत्पुष्टो ब्रवीहि मे ॥बृहस्पतिरुवाच}
{}


\twolineshloka
{परोक्षमगुणानाह सद्रुणानभ्यसूयति}
{परैर्वा कीर्त्यमानेषु तूष्णीमास्ते पराङ्भुखः}


\twolineshloka
{तूष्णींभावेऽपि विज्ञेयं न चेद्भवति कारणम्}
{विश्वासं चोष्ठसंदंशं शिरसश्च प्रकम्पनम्}


\twolineshloka
{करोत्यभीक्ष्णं संसृष्टमसंसृष्टश्च भाषते}
{अदृष्टवद्विकुरुते दृष्ट्वा वा नाभिभाषते}


\twolineshloka
{पृथगेत्य समश्नाति नेदमद्य यथाविधि}
{आसने शयने याने भावा लक्ष्या विशेषतः}


\twolineshloka
{आर्तिरार्ते प्रिये प्रीतिरेतावन्मित्रलक्षणम्}
{विपरीतं तु बोद्धव्यमरिलक्षणमेव तत्}


\twolineshloka
{एतान्येव यथोक्तानि बुध्येथास्त्रिदशाधिप}
{पुरुषाणां प्रदुष्टानां स्वभावो बलवत्तरः}


\twolineshloka
{इति दुष्टस्य विज्ञानमुक्तं ते सुतसत्तम}
{निशाम्य शास्त्रतत्त्वार्थं यथावदमरेश्वरः ॥भीष्म उवाच}


\twolineshloka
{स तद्वचः शत्रुनिबर्हणे रतस्तथा चकारावितथं बृहस्पतेः}
{चचार काले विजयाय चारिहावशं च शत्रूननयत्पुरंदरः}


\chapter{अध्यायः १०४}
\twolineshloka
{युधिष्ठिर उवाच}
{}


\threelineshloka
{धार्मिकोऽर्थानसंप्राप्य राजामात्यैः प्रबाधितः}
{च्युतः कोशाच्च दण्डाच्च सुखमिच्छन्कथं चरेत् ॥भीष्म उवाच}
{}


\twolineshloka
{अत्रायं क्षेमदर्शीय इतिहासोऽनुगीयते}
{तत्तेऽह संप्रवक्ष्यामि तन्निबोध युधिष्ठिर}


\threelineshloka
{क्षेमदर्शी नृपसुतो यत्र क्षीणबलः पुरा}
{मुनिं कालकवृक्षीयमाजमामेति नः श्रुतम्}
{तं पप्रच्छानुसंगृह्य कृच्छ्रामापदमास्थितः}


\twolineshloka
{अर्थेषु मागी पुरुष ईहमानः पुनः पुनः}
{अलब्ध्वा मद्विधो राज्यं ब्रह्मन्किं कर्तुमर्हति}


\twolineshloka
{अन्यत्र मरणाद्दैन्यादन्यत्र परसंश्रयात्}
{क्षुद्रादन्यत्र चाचारात्तन्ममाचक्ष्व सत्तम}


\twolineshloka
{व्याधिनां चाभिपन्नस्य मानसेनेतरेण वा}
{बहुश्रुतः कृतप्रज्ञस्त्वद्विधः शरणं भवेत्}


\twolineshloka
{निर्विद्य हि नरः कामान्नियम्य सुखमेधते}
{त्वक्त्वा प्रीतिं च शोकं च लब्ध्वा बुद्धिमयं वसु}


\twolineshloka
{सुखमर्थाश्रयं येषामनुशोचामि तानहम्}
{मम ह्यर्थाः सुबहवो नष्टाः स्वप्नगता इव}


\twolineshloka
{दुष्करं बत कुर्वन्ति महतोऽर्थांस्त्यजन्ति ये}
{वयं त्वेतान्परित्यक्तुमसतोऽपि न शक्नुमः}


\twolineshloka
{इमामवस्थां संप्राप्तं दीनमार्तं श्रिया च्युतम्}
{यदन्यत्सुखमस्तीह तद्ब्रह्मन्ननुशाधि माम्}


\twolineshloka
{कौसल्येनैवमुक्तस्तु राजपुत्रेण धीमता}
{मुनिः कालकवृक्षीयः प्रत्युवाच महाद्युतिः}


\twolineshloka
{पुरस्तादेव ते बुद्धिरियं कार्या विजानतः}
{अनित्यं सर्वमेवैतदहं च मम चास्ति यत्}


\twolineshloka
{यत्किंचिन्मन्यसेऽस्तीति सर्वं नास्तीति विद्धि तत्}
{एवं न व्यथते प्राज्ञः कृच्छ्रामप्यापदं गतः}


\twolineshloka
{यद्धि भूतं भविष्यच्च ध्रुवं तन्न भविष्यति}
{एवं विदितवेद्यस्त्वमनर्थेभ्यः प्रमोक्ष्यसे}


\twolineshloka
{ये च पूर्वसमारम्भा ये च पूर्वतरे परे}
{सर्वं नास्तीति ते चैव तज्ज्ञात्वा को नु संज्वरेत्}


\twolineshloka
{भूत्वा च न भवत्येतदभूत्वा च भविष्यति}
{शोके न ह्यस्ति सामर्थ्यं शोचेत स कथं नरः}


\twolineshloka
{क्वनु तेऽद्य पिता राजन्क्वनु तेऽद्य पितामहः}
{न त्वं पश्यसि तानद्य न त्वां पश्यन्ति तेऽपि वा}


\twolineshloka
{आत्मनोऽध्रुवतां पश्यंस्तांस्त्वं किमनुशोचसि}
{बुद्ध्या चैवानुबुद्ध्यस्व ध्रुवं हि न च विद्यते}


\twolineshloka
{अहं च त्वं च नृपते सुहृदः शत्रवश्च ते}
{अवश्यं न भविष्यामः सर्वं च न भविष्यति}


\twolineshloka
{ये तु विंशतिवर्षा वै त्रिंशद्वर्षाश्च मानवाः}
{अर्वागेव हि ते सर्वे मरिष्यन्ति शरच्छतात्}


\twolineshloka
{अपि चेन्महतो वित्तान्न प्रमुच्यते पूरुषः}
{नैतन्ममेति तन्मत्वा कुर्वीत प्रियमात्मनः}


\twolineshloka
{अनागतं यन्न ममेति विद्यादतिक्रान्तं यन्न ममेति विद्यात्}
{दिष्टं बलीय इति मन्यमानास्ते पण्डितास्तत्सतां वृत्तिमाहुः}


\twolineshloka
{अनाढ्याश्चापि जीवन्ति राज्यं चाप्यनुशासते}
{बुद्धिपौरुषसंपन्नास्त्वया तुल्याधिका जनाः}


\threelineshloka
{न च त्वमिव शोचन्ति तस्मात्त्वमपि मा शुचः}
{किं न त्वं तैर्नरैः श्रेयांस्तुल्यो वा बुद्धिपौरुषैः ॥राजोवाच}
{}


\twolineshloka
{यादृच्छिकं सर्वमासीत्तद्राज्यमिति चिन्तये}
{ह्रियते सर्वमेवेदं कालेन महता द्विज}


\threelineshloka
{तस्यैव ह्रियमाणस्य स्रोतसेव तपोधन}
{फलमेतत्प्रपश्यामि यथालब्धेन वर्तयन् ॥मुनिरुवाच}
{}


\twolineshloka
{अनागतमतीतं च याथातथ्यविनिश्चयात्}
{नानुशोचेत कौसल्य सर्वार्थेषु तथा भव}


\twolineshloka
{अवाप्यान्कामयन्नर्थान्नानवाप्यान्कदाचन}
{प्रत्युत्पन्नाननुभवन्मा शुचस्त्वमनागतान्}


\twolineshloka
{यथालब्धोपपन्नार्थैस्तथा कौसल्य रंस्यसे}
{कच्चिच्छुद्धस्वभावेन श्रिया हीनो न शोचसि}


\twolineshloka
{पुरस्ताद्भूतपूर्वत्वाद्धीनभोग्यो हि दुर्मतिः}
{धातारं गर्हते नित्यं लब्धार्थश्च न मृष्यते}


\twolineshloka
{अनर्हानपि चैवान्यान्मन्यते श्रीमतो जनान्}
{एतस्मात्कारणादेतद्दुःखं भूयोऽनुवर्तते}


\twolineshloka
{ईर्ष्याभिमानसंपन्ना राजन्पुरुषमानिनः}
{कच्चित्त्वं न तथा प्राज्ञ मत्सरी कोसलाधिप}


\threelineshloka
{सहस्व श्रियमन्येषां यद्यपि त्वयि नास्ति सा}
{अन्यत्रापि सतीं लक्ष्मीं कुशला भुञ्जते नराः}
{अभिनिष्यन्दते देही श्रीभूतश्च द्विषज्जनात्}


\twolineshloka
{श्रियं च पुत्रपौत्रं च मनुष्या धर्मचारिणः}
{त्यागधर्मविदो धीराः स्वयमेव त्यजन्त्युत}


\twolineshloka
{`त्यक्तं स्वायंभुवे वंशे शुभेन भरतेन च}
{नानारत्नसमाकीर्णं राज्यं स्फीतमिति श्रुतम्}


\threelineshloka
{तथाऽन्यैर्भूमिपालैश्च त्यक्तं राज्यं महोदयम्}
{त्यक्त्वा राज्यानि ते सर्वे वने वन्यफलाशिनः}
{गताश्च तपसः पारं दुःखस्यान्तं च भूमिपा}


\twolineshloka
{बहुसंकुसुकं दृष्ट्वा विधित्सासाधनेन च}
{तथान्ये संत्यजन्त्येव मत्वा परमदुर्लभम्}


\twolineshloka
{त्वं पुनः प्राज्ञरूपः सन्कृपणं परितप्यसे}
{अकाम्यान्कामयानोऽर्थान्पराधीनानुपद्रवान्}


\twolineshloka
{तां बुद्धिमनुविज्ञाय त्वमेवैनान्परित्यज}
{अनर्थाश्चार्थरूपेण ह्यर्थाश्चानर्थरूपिणः}


\twolineshloka
{अर्थायैव हि केषांचिद्धननाशा भवन्त्युत}
{अनित्यं तत्सुखं मत्वा श्रियमन्ये न लिप्सते}


\twolineshloka
{रममाणः श्रिया कश्चिन्नान्यच्छ्रेयोऽभिमन्यते}
{तथा तस्येहमानस्य संरम्भोऽपि विनश्यति}


\twolineshloka
{कृच्छ्राल्लब्धमभिप्रेतं यथा कौसल्य नश्यति}
{तदा निर्विद्यते सोऽर्थात्परिभग्नक्रमो नरः}


% Check verse!
`अनित्यां तां श्रियं मत्वा श्रियं वा कः परीप्सति ॥'
\twolineshloka
{धर्ममेकेऽभिपद्यन्ते कल्याणाभिजना नराः}
{परत्र सुखमिच्छन्तो निर्विद्येयुश्च लौकिकात्}


\twolineshloka
{जीवितं संत्यजन्त्येके धनलोभपरा नराः}
{न जीवितार्थं मन्यन्ते पुरुषा हि धनादृते}


\twolineshloka
{पश्य चैषां कृपणतां पश्य चैषामबुद्धिताम्}
{अध्रुवे जीविते मोहादर्थतृष्णामुपाश्रिताः}


\twolineshloka
{संचये च विनाशान्ते मरणान्ते च जीविते}
{संयोगे च वियोगान्ते कोनु विप्रणयेन्मनः}


\twolineshloka
{धनं वा पुरुषो राजन्पुरुषं वा पुनर्धनम्}
{अवश्यं प्रजहात्येव तद्विद्वान्कोनु संज्वरेत्}


\twolineshloka
{अन्यत्रोपनता ह्यापत्पुरुषं तोषयत्युत}
{तेन शान्तिं न लभते नाहमेवेति कारणात् ॥'}


\threelineshloka
{अन्येषामपि नश्यन्ति सुहृदश्च धनानि च}
{पश्य बुद्ध्या मनुष्याणां तुल्यामापदमात्मनः}
{नियच्छ यच्छ संयच्छ इन्द्रियाणि मनस्तथा}


% Check verse!
प्रतिषेद्धा न चाप्येषु दुर्बलेष्वहितेषु च
\twolineshloka
{प्राप्तिसृष्टेषु भावेषु व्यपकृष्टेष्वसंभवे}
{प्रज्ञानतृप्तो विक्रान्तस्त्वद्विधो नानुशोचति}


\twolineshloka
{अल्पमिच्छन्नचपलो मृदुर्दान्तः सुसंस्थितः}
{ब्रह्मचर्योपपन्नश्च त्वद्विधो नैव मुह्यति}


\twolineshloka
{न त्वेव जाल्मीं कापालीं वृत्तिमेषितुमर्हसि}
{नृशंसवृत्तिं पापिष्ठां दुःखां कापुरुषोचिताम्}


\twolineshloka
{अपि मूलफलाहारो रमस्वैको महावने}
{वाग्यतः संगृहीतात्मा सर्वभूतदयान्वितः}


\twolineshloka
{सदृशं पण्डितस्यैतदीषादन्तेन हस्तिना}
{यदेको रमतेऽरण्ये यच्चाप्यल्पेन तुष्यति}


\twolineshloka
{महाह्रदः संक्षुभित आत्मनैव प्रसीदति}
{`एवं नरः स्वत्मानैव कृतप्रज्ञः प्रसीदति ॥'}


\threelineshloka
{एतदेवं गतस्याहं सुखं पश्यामि केवलम्}
{असंभवे श्रियो राजन्हीनस्य सचिवादिभिः}
{दैवे प्रतिनिविष्टे च किं श्रेयो मन्यते भवान्}


\chapter{अध्यायः १०५}
\twolineshloka
{मुनिरुवाच}
{}


\twolineshloka
{अथ चेत्पौरुषं किंचित्क्षत्रियात्मनि पश्यसि}
{ब्रवीम्यहं तु ते नीतिं राज्यस्य प्रतिपत्तये}


\twolineshloka
{तां चच्छक्ष्यस्यनुष्ठातुं कर्म चैव करिष्यसि}
{शृणु सर्वमशेषेण यत्ते वक्ष्यामि तत्त्वतः}


\fourlineindentedshloka
{आचरिष्यसि चेत्कर्म महतोऽर्थानवाप्स्यसि}
{राज्यं वा राज्यमन्त्रं वा महतीं वा पुनः श्रियम्}
{यद्येतद्रोचते राजन्पुनर्ब्रूहि ब्रवीमि ते ॥राजोवाच}
{}


\threelineshloka
{ब्रवीतु भगवान्नीतिमभिपन्नोऽस्म्यधीहि भो}
{अमोघ एव मेऽद्यास्तु त्वया सह समागमः ॥मुनिरुवाच}
{}


\threelineshloka
{हित्वा मानं च दम्भं च क्रोधं हर्ष भयं तथा}
{प्रत्यमित्राणि सेवस्व प्रणिपत्य कृताञ्जलिः}
{तमुत्तमेन शौचेन कर्मणा चावधारय}


\twolineshloka
{दातुमर्हति ते वित्तं वैदेहः सत्यविक्रमः}
{प्रमाणं सर्वभूतेषु प्रग्रहं च गमिष्यसि}


\threelineshloka
{ततः सहायान्सोत्साहाँल्लप्स्यसेऽव्यसनाञ्शुचीन्}
{वर्तमानः स्वशास्त्रे वै संयतात्मा जितेन्द्रियः}
{अभ्युद्धरति चात्मानं प्रसादयति च प्रजाः}


\twolineshloka
{तेनैव त्वं धृतिमता श्रीमता चापि सत्कृतः}
{प्रमाणं सर्वभूतेषु गत्वा च ग्रहणं महत्}


\threelineshloka
{ततः सुहृद्बलं लब्ध्वा मन्त्रयित्वा सुमन्त्रितम्}
{सान्त्वेन भेदयित्वाऽरीन्बिल्वं बिल्वेन शातय}
{परैर्वा संविदं कृत्वा बलमप्यस्य घातय}


\twolineshloka
{अलभ्या ये शुभा भावाः स्त्रियश्चाच्छादनानि च}
{शय्यासनानि यानानि महार्हाणि गृहाणि च}


\twolineshloka
{पक्षिणो मृगजातानि रसगन्धाः फलानि च}
{तेष्वेव सज्जयेथास्त्वं यथा नश्येत्स्वयं परः}


\threelineshloka
{यद्येवं प्रतिषेद्धव्यो यद्युपेक्षणमर्हति}
{`सदैव राजशार्दूल विदुषा हितमिच्छता}
{'न जातु विवृतः कार्यः शत्रुः सुनयमिच्छता}


\twolineshloka
{वसस्व पुरमामित्रं विषये मित्रसंमतः}
{भजस्व श्वेतकाकीयैर्मित्रधर्ममनथैकैः}


\twolineshloka
{आरम्भांश्चास्य महतो दुष्करान्संप्रयोजय}
{नदीबन्धविभेदांश्च बलवद्भिर्विरुध्यताम्}


\twolineshloka
{उद्यानानि महार्हाणि शयनान्यासनानि च}
{प्रीतिभोगमुखेनैव कोशमस्य विरोचय}


\twolineshloka
{यज्ञदाने प्रशंसास्मै ब्राह्मणाननुवर्तय}
{ते त्वा प्रियं करिष्यन्ति तच्छेत्स्यन्ति वृका इव}


\twolineshloka
{असंशयं पुण्यशीलाः प्राप्नोति परमां गतिम्}
{त्रिविष्टपे पुण्यतमं स्थानं प्राप्नोति शाश्वतम्}


\twolineshloka
{कोशक्षये त्वमित्राणां वशं कौसल्य गच्छति}
{उभयत्र प्रयुक्तस्य धर्मे चाधर्म एव च}


\twolineshloka
{फलार्थमूलमुच्छिद्यात्तेन नन्दन्ति शत्रवः}
{न चास्मै मानुषं कर्म दैवप्रस्योपवर्णय}


\twolineshloka
{असंशयं दैवपरः क्षिप्रमेव विनश्यति}
{याजयैनं विश्वजिता सर्वस्वेन वियुज्यताम्}


\twolineshloka
{ततो गच्छत्यसिद्धार्थः पीडयानो महाजनम्}
{त्यागधर्मविदं पुण्यं कंचिदस्योपवर्णय}


\threelineshloka
{अपि त्यागं बुभूषेत कच्चिद्गच्छेदनामयम्}
{सिद्धेनौषधियोगेन सर्वशत्रुविनाशिना}
{गजानश्वान्मनुष्यांश्च कृतकैरुपघातय}


\twolineshloka
{एते चान्ये च बहवो दम्भयोगाः सुचिन्तिताः}
{शक्या विपहता कर्तुं न क्लीबेन नृपात्मज}


\chapter{अध्यायः १०६}
\twolineshloka
{राजोवाच}
{}


\twolineshloka
{न निकृत्या न दम्भेन ब्रह्मन्निच्छामि जीवितुम्}
{नाधर्मयुक्तानिच्छेयमर्थान्सुमहतोऽप्यहम्}


\twolineshloka
{पुरस्तादेव भगवन्मयैतदपवर्जितम्}
{येन पापं न शङ्केत यद्वा कृत्स्नं हितं भवेत्}


\threelineshloka
{आनृशंस्येन धर्मेण लोके ह्यस्मिञ्जिजीविषुः}
{नाहमेतदलं कर्तुं नैतन्मय्युपपद्यते ॥मुनिरुवाच}
{}


\twolineshloka
{उपपन्नस्त्वमेतेन यथा क्षत्रिय भाषसे}
{प्रकृत्या ह्युपपन्नोऽसि बुद्ध्या चाद्भुतदर्शनः}


\twolineshloka
{उभयोरेव साह्यार्थे यतिष्ये तव तस्य च}
{संश्लेषं वा करिष्यामि शाश्वतं ह्यनपायिनम्}


\twolineshloka
{त्वादृशं हि कुले जातभनृशंसं बहुश्रुतम्}
{अमात्यं को न कुर्वीत राज्यप्रणयकोविदम्}


\twolineshloka
{यस्त्वं प्रव्राजितो राज्याद्व्यसनं चोत्तमं गतः}
{आनृशंस्येन वृत्तेन क्षत्रियेच्छसि जीवितुम्}


\threelineshloka
{आगन्ता मद्गृहं तात वैदेहः सत्यसंगरः}
{अथाहं तं नियोक्ष्यामि तत्करिष्यत्यसंशयम् ॥भीष्म उवाच}
{}


\twolineshloka
{तत आहूय वैदेहं मुनिर्वचनमब्रवीत्}
{अयं राजकुले जातो विदिताभ्यन्तरो मम}


\twolineshloka
{आदर्श इव शुद्धात्मा शारदश्चन्द्रमा यथा}
{नास्मिन्पश्यामि वृजिनं सर्वतो मे परीक्षितः}


\twolineshloka
{तेन ते संधिरेवास्तु विश्वसास्मिन्यथा मयि}
{न राज्यमनमात्येन शक्यं शास्तुममित्रहन्}


\twolineshloka
{अमात्य शुद्ध एव स्याद्बुद्धिसंपन्न एव वा}
{तस्माच्चैव भयं राज्ञः पश्य राज्यस्य योजनम्}


\threelineshloka
{धर्मात्मनां क्वचिल्लोके नान्यास्ति गतिरीदृशी}
{तदा राजपुत्रोऽयं सतां मार्गमनुष्ठितः}
{असंगृहीतस्त्वेवैष त्वया धर्मपुरोगमः}


% Check verse!
संसेव्यमानः शत्रूंस्ते गृह्णीयान्महतो गणान्
\twolineshloka
{यद्ययं प्रतियुद्ध्येत स्वकर्म क्षत्रियस्य तत्}
{जिगीषमाणस्त्वां युद्धे पितृपैतामहे पदे}


\twolineshloka
{त्वं अपि प्रतियुद्ध्येथा विजिगीषुर्व्रते स्थितः}
{अयुद्ध्वैव नियोगान्मे वशे कुरु हिते स्थितः}


\twolineshloka
{स त्वं धर्ममवेक्षस्व हित्वा लोभमसांप्रतम्}
{न च कामान्न च द्रोहात्स्वधर्मं हातुमर्हसि}


\twolineshloka
{नैव नित्यं जयस्तात नैव नित्यं पराजयः}
{तस्माज्जयश्च भोक्तव्यो भोक्तव्यश्च पराजयः}


\twolineshloka
{आत्मन्यपि च संदृश्यावृभौ जयपराजयौ}
{निःशेषकारिणां तात निःशेषकरणाद्भयम्}


\twolineshloka
{इत्युक्तः प्रत्युवाचेदं वचनं ब्राह्मणर्षभम्}
{प्रतिपूज्याभिसत्कृत्य पूजार्हमनुमान्य च}


\twolineshloka
{यथा ब्रूयान्महाप्राज्ञो यथा ब्रूयान्महाश्रुतः}
{श्रेयस्कामो यथा ब्रूयादुभयोरेव तत्क्षमम्}


\twolineshloka
{यद्यद्वचनमुक्तोऽस्मि करिष्यामि च तत्तथा}
{एतद्धि परमं श्रेयो न मेऽत्रास्ति विचारणा}


\twolineshloka
{ततः कौसल्यमाहूय मैथिलो वाक्यमब्रवीत्}
{धर्मतो बुद्धितश्चैव बलेन च जितं मया}


\twolineshloka
{अहं त्वया चात्मगुणैर्जितः पार्थिवसत्तम}
{आत्मानमनवज्ञाय जितवद्वर्ततां भवान्}


\twolineshloka
{नावमन्यामि ते बुद्धिं नावमन्ये च पौरुषम्}
{नावमन्ये जयामीति जितवद्वर्ततां भवान्}


\twolineshloka
{यथावत्पूजितो राजन्गृहं गन्तासि मे गृहात्}
{ततः संपूज्य तौ विप्रं विश्वस्तौ जग्मतुर्गृहान्}


\twolineshloka
{वैदेहस्त्वथ कौसल्यं प्रवेश्य गृहमञ्जसा}
{प्राद्यार्ध्यमधुपर्कैस्तं पूजार्हं प्रत्यपूजयत्}


\twolineshloka
{ददौ दुहितरं चास्मै रत्नानि विविधानि च}
{एष राज्ञां परो धर्मः समौ जयपराजयौ}


\chapter{अध्यायः १०७}
\twolineshloka
{युधिष्ठिर उवाच}
{}


\twolineshloka
{ब्राह्मणक्षत्रियविशां शूद्राणां च परंतप}
{धर्मवृत्तं च वित्तं च वृत्त्युपायाः फलानि च}


\twolineshloka
{राज्ञां वृत्तं च कोशं च कोशसंचयनं जयः}
{अमात्यगुणवृत्तिश्च प्रकृतीनां च वर्धनम्}


\twolineshloka
{षाङ्गुण्यगुणकल्पश्च सेनानीतिस्तथैव च}
{दुष्टस्य च परिज्ञानमदुष्टस्य च लक्षणम्}


\twolineshloka
{समहीनाधिकानां च यथावल्लक्षणं च यत्}
{मध्यमस्य च तुष्ट्यर्थं यथा स्थेयं विवर्धता}


\twolineshloka
{क्षीणग्रहणवृत्तिश्च यथा धर्मं प्रकीर्तितम्}
{लघुनाऽदेशरूपेण ग्रन्थयोगेन भारत}


\twolineshloka
{विजिगीषोस्तथा वृत्तमुक्तं चैव तथैव ते}
{गणानां वृत्तिमिच्छामि श्रोतुं मतिमतां वर}


\twolineshloka
{यथा गणाः प्रवर्धन्ते न भिद्यन्ते च भारत}
{अरींश्च विजिगीषन्ते सुहृदः प्राप्नुवन्ति च}


\twolineshloka
{भेदमूलो विनाशो हि गणानामुपलक्षये}
{मन्त्रसंवरणं दुःखं बहूनामिति मे मतिः}


\threelineshloka
{एतदिच्छाम्यहं श्रोतुं निखिलेन परंतप}
{यथा च ते न भिद्येरंस्तच्च मे वद भारत ॥भीष्म उवाच}
{}


\twolineshloka
{गणानां च कुलानां च राज्ञां भरतसत्तम}
{वैरसंदीपनावेतौ लोभामर्षौ नराधिप}


\twolineshloka
{लोभमेको हि वृणुते ततोऽमर्षमनन्तरम्}
{ततो ह्यमर्षसंयुक्तावन्योन्यजनिताशयौ}


\twolineshloka
{चारमन्त्रबलादानैः सामदानविभेदनैः}
{क्षयव्ययभयोपायैः प्रकर्षन्तीतरेतरम्}


\twolineshloka
{तत्रादानेन भिद्यन्ते गणाः संघातवृत्तयः}
{भिन्ना विमनसः सर्वे गच्छन्त्यरिवशं भयात्}


\twolineshloka
{भेदे गणा विनश्युर्हि भिन्नास्तु सुजयाः परैः}
{तस्मात्संघातयोगेन प्रयतेरन्गणाः सदा}


\twolineshloka
{अर्थाश्चैवाधिगम्यन्ते संघातबलपौरुषैः}
{ब्राह्माश्च मैत्रीं कुर्वन्ति तेषु संघातवृत्तिषु}


\twolineshloka
{ज्ञानवृद्धाः प्रशंसन्ति शुश्रूषन्तः परस्परम्}
{विनिवृत्ताभिसंधानाः सुखमेधन्ति सर्वशः}


\twolineshloka
{धर्मिष्ठान्व्यवहारांश्च स्थापयन्तश्च शास्त्रतः}
{यथावत्प्रतिपश्यन्तो विवर्धन्ते गणोत्तमाः}


\twolineshloka
{पुत्रान्भ्रातृन्निगृह्णन्तो विनयन्तश्च तान्सदा}
{विनीतांश्च प्रगृह्णन्तो विवर्धन्ते गणोत्तमाः}


\twolineshloka
{चारमन्त्रविधानेषु कोशसंनिचयेषु च}
{नित्ययुक्ता महाबाहो वर्धन्ते सर्वतो गणाः}


\twolineshloka
{प्राज्ञांश्चारान्महोत्साहान्कर्मसु स्थिरपौरुषान्}
{मानयन्तः सदा युक्ता विवर्धन्ते गणा नृप}


\twolineshloka
{द्रव्यवन्तश्च शूराश्च शस्त्रज्ञाः शास्त्रपारगाः}
{कृच्छ्रास्वापत्सु संमूढान्गणाः संतारयन्ति ते}


\twolineshloka
{क्रोधो भेदो भयं दण्डः कर्षणं निग्रहो वधः}
{नयत्यरिवशं सद्यो गणान्भरतसत्तम}


\twolineshloka
{तस्मान्मानयितव्यास्ते गणमुख्याः प्रधानतः}
{लोकयात्रा समायत्ता भूयसी तेषु पार्थिव}


\twolineshloka
{मन्त्रगुप्तिः प्रधानेषु चारश्चामित्रकर्शण}
{न गणाः कृत्स्नशो मन्त्रं श्रोतुमर्हन्ति भारत}


% Check verse!
गणमुख्यैस्तु संभूय कार्यं गणहितं मिथः
\twolineshloka
{पृथग्गणस्य भिन्नस्य विततस्य ततोऽन्यथा}
{अर्थाः प्रत्यवसीदन्ति तथाऽनर्था भवन्ति व}


\twolineshloka
{तेषामन्योन्यभिन्नानां स्वशक्तिमनुतिष्ठताम्}
{निग्रहः पण्डितैः कार्यः क्षिप्रमेव प्रधानतः}


\twolineshloka
{कुलेषु कलहा जाताः कुलवृद्धैरुपेक्षिताः}
{गोत्रस्य नाशं कुर्वन्ति गणभेदस्य कारकम्}


\twolineshloka
{आभ्यन्तरं भयं रक्ष्यमसारं बाह्यतो भयम्}
{आभ्यन्तरं भयं राजन्सद्यो मूलानि कृन्तति}


\twolineshloka
{अकस्मात्क्रोधमोहाभ्यां लोभाद्वाऽपि स्वभावजात्}
{अन्योन्यं नाभिभाषन्ते तत्पराभवलक्षणम्}


\twolineshloka
{जात्या च सदृशाः सर्वे कुलेन सदृशास्तथ}
{न चोद्योगेन बुद्ध्या वा रूपद्रव्येण वा पुनः}


\twolineshloka
{भेदाच्चैव प्रदानाच्च नाम्यन्ते रिपुभिर्गणाः}
{तस्मात्संघातमेवाहुर्गणानां शरणं महत्}


\chapter{अध्यायः १०८}
% Check verse!
युधिष्ठिर उवाच
\twolineshloka
{महानयं धर्मपथो बहुशाखश्च भारत}
{किंस्विदेवेह धर्माणामनुष्ठेयतमं मतम्}


\threelineshloka
{किं कार्यं सर्वभूतानां गरीयो भवतो मतम्}
{यथाऽहं परमं धर्ममिह च प्रेत्य चाप्नुयाम् ॥भीष्म उवाच}
{}


\twolineshloka
{मातापित्रोर्गुरूणां च पूजा बहुमता मम}
{अत्र वर्तन्नरो लोकान्यशश्च महदश्नुते}


\twolineshloka
{यदेते ह्यनुजानीयुः कर्म तात सुपूजिताः}
{धर्मं धर्मविरुद्धं वा तत्कर्तव्यं युधिष्ठिर}


\twolineshloka
{न तैरभ्यननुज्ञातो धर्ममन्यं समाचरेत्}
{यं मे तेऽभ्यनुजानीयुः स धर्म इति निश्चयः}


\twolineshloka
{एत एव त्रयो लोका एत एवाश्रमास्त्रयः}
{एत एव त्रयो वेदा एत एव त्रयोऽग्नयः}


\twolineshloka
{पितावै गार्हपत्योऽग्निर्माताऽग्निर्दक्षणः स्मृतः}
{गुरु वनीयस्तु साऽग्नित्रेता गरीयसी}


\twolineshloka
{त्रिष्वप्रमाद्यन्नेतेषु त्रील्लोँकानपि जेष्यसि}
{पितृवृत्त्या त्विमं लोकं मातृवृत्त्या तथा परम्}


\twolineshloka
{ब्रह्मलोकं गुरोर्वृत्त्या नियमेन तरिष्यसि}
{स--गेतेषु वर्तस्व त्रिषु लोकेषु भारत}


\twolineshloka
{यशः प्राप्स्यसि भद्रं ते धर्मं च सुमहाफलम्}
{नैतानतिशयीथास्त्वं नात्यश्नीथा न दूषयेः}


\twolineshloka
{हियं परिचरेश्चैव तद्वै सुकृतमुत्तमम्}
{कीर्ति पुण्यं यशो लोकान्प्राप्स्यसे त्वं जनाधिप}


\twolineshloka
{सर्वे तस्यादृता लोका यस्यैते त्रय आदृताः}
{अनादृतास्तु यस्यैते सर्वास्तस्याफलाः क्रियाः}


\twolineshloka
{न चायं न परो लोको न यशस्तस्य भारत}
{अमानिता नित्यमेव यस्यैते गुरवस्त्रयः}


\twolineshloka
{न चास्मिन्न परे लोके यशस्तस्य प्रकाशते}
{यच्चान्यदपि कल्याणं पारत्रं समुदाहृतम्}


\threelineshloka
{तेभ्य एव हि यत्सर्वं कृत्यं यन्निसृजाम्यहम्}
{तदासीन्मे शतगुणं सहस्रगुणमेव च}
{तस्मान्मे संप्रकाशन्ते त्रयो लोका युधिष्ठिर}


\twolineshloka
{दशैव तु सदाऽऽचार्यः श्रोत्रियानधितिष्ठति}
{दशाचार्यानुपाध्याय उपाध्यायान्पिता दश}


\twolineshloka
{पितॄन्दश तु मातैका सर्वां वा पृथिवीमपि}
{गुरुत्वेनाभिभवति नास्ति मातृसमो गुरुः}


\twolineshloka
{गुरुर्गरीयान्पितृतो मातृतश्चेति मे मतिः}
{उभौ हि मातापितरौ जन्मन्येवोपयुज्यतः}


\twolineshloka
{शरीरमेतौ सृजतः पिता माता च भारत}
{आचार्यशिष्टा या जातिः सासम्यगजरामरा}


% Check verse!
अवध्या हि सदा माता पिता चाप्युपचारिणौ
\twolineshloka
{न स दुष्यति तत्कृत्वा न च ते दूषयन्ति तम्}
{धर्माय यतमानानां विदुर्देवाः सहर्षिभिः}


\twolineshloka
{य आवृणोत्यवितथेन कर्मणाऋतं ब्रुवन्नमृतं संप्रयच्छन्}
{तं मन्येथाः पितरं मातरं चतस्मै न द्रुह्येत्कृतमस्य जानन्}


\threelineshloka
{विद्यां श्रुत्वा ये गुरुं नाद्रियन्तेप्रत्युत्पन्ना मनसा कर्मणा वा}
{तेषां पापं भ्रूणहत्याविशिष्टंनान्यस्तेभ्यः पापकृदस्ति लोके}
{यथैव ते गुरुभिर्भावनीयास्तथैव तेषां गुरवोऽभ्यर्चनीयाः}


\twolineshloka
{तस्मात्पूजयितव्याश्च संविभज्याश्च यत्नतः}
{गुरवोऽर्चयितव्याश्च पुराणं धर्ममिच्छता}


\twolineshloka
{येन प्रीणन्ति पितरस्तेन प्रीतः प्रजापतिः}
{प्रीणाति जननीयेन पृथिवी तेन पूजिता}


\twolineshloka
{येन प्रीणात्युपाध्यायस्तेन स्याद्ब्रह्म पूजितम्}
{मातृतः पितृतश्चैव तस्मात्पूज्यतमो गुरुः}


\twolineshloka
{ऋषयश्च हि देवाश्च प्रीयन्ते पितृभिः सह}
{पूज्यमानेषु गुरुषु तस्मात्पूज्यतमो गुरुः}


\twolineshloka
{केनचिन्न च वृत्तेन ह्यवज्ञेयो गुरुर्भवेत्}
{न च माता न च पितो तादृशो यादृशो गुरुः}


\twolineshloka
{न तेऽवमानमर्हन्ति न तेषां दूषयेत्कृतम्}
{गुरूणामेव सत्कारं विदुर्देवाः सहर्षिभिः}


\twolineshloka
{उपाध्यायं पितरं मातरं चये विद्रुह्यन्ते मनसा कर्मणा वा}
{तेषां पापं भ्रूणहत्याविशिष्टंतस्मान्नान्यः पापकृदस्ति लोके}


\twolineshloka
{भृतो भर्तारं यो न विभिर्ति पुत्रःस्वयोनिजः पितरं मातरं च}
{तस्य पापं भ्रूणहत्याविषिष्टंतस्मान्नान्यः पापकृदस्ति लोके}


\twolineshloka
{मित्रद्रुहः कृतघ्नस्य स्त्रीघ्नस्य पिशुनस्य च}
{चतुर्णामपि चैतेषां निष्कृतिं नानुशुश्रुम्}


\twolineshloka
{एतत्सर्वं मनुनिर्देशदृष्टंयत्कर्तव्यं पुरुषेणेह किंचित्}
{एतच्छ्रेयो नान्यदस्माद्विशिष्टंसर्वान्धर्माननुसृत्यैतदुक्तम्}


\chapter{अध्यायः १०९}
\twolineshloka
{युधिष्ठिर उवाच}
{}


\twolineshloka
{कथं धर्मे स्थातुमिच्छन्नरो वर्तेत भारत}
{तत्त्वं जिज्ञासमानाय प्रब्रूहि भरतर्षभ}


\twolineshloka
{सत्यं चैवानृतं चोभे लोकानावृत्य तिष्ठतः}
{तयोः किमाचरेद्राजन्पुरुषो धर्मनिश्चितः}


\threelineshloka
{किंस्वित्सत्यं किमनृतं किंस्विद्धर्म्यं सनातनम्}
{कस्मिन्काले वदेत्सत्यं कस्मिन्वाऽप्यनृतं वदेत् ॥भीष्म उवाच}
{}


\twolineshloka
{सत्यस्य वचनं साधु न सत्याद्विद्यते परम्}
{यत्तु लोके सुदुर्ज्ञेयं तत्ते वक्ष्यामि भारत}


\twolineshloka
{भवेत्सत्यं न वक्तव्यं वक्तव्यमनृतं भवेत्}
{यत्रानृतं भवेत्सत्यं सत्यं वाऽप्यनृतं भवेत्}


\twolineshloka
{तादृशो वर्धते पापो यत्र सत्यमनिश्चितम्}
{सत्यानृते विनिश्चित्य ततो भवति धर्मवि}


\twolineshloka
{अप्यनार्योऽकृतप्रज्ञः पुरुषोऽप्यतिदारुणः}
{सुमहत्प्राप्नुयात्पुण्यं बलाकोऽन्धवधादिव}


\twolineshloka
{किमाश्चर्यं च यन्मूढो धर्मकामोऽप्यधर्मवित्}
{सुमहत्प्राप्नुयात्पुण्यं गङ्गायामिव कौशिकः}


\twolineshloka
{तादृशोऽयमनुप्रश्नो यत्र धर्मः सुदुर्विदः}
{दुष्कारं चापि संख्यातुं तर्केणात्र व्यवस्यति}


\twolineshloka
{प्रभवार्थाय भूतानां धर्मप्रवचनं कृतम्}
{यः स्यात्प्रभवसंयुक्तः स धर्म इति निश्चयः}


\twolineshloka
{`अहिंसा सत्यमक्रोधस्तपो दानं दमो मतिः}
{अनसूयाऽप्यसामर्थ्यमनीर्ष्या शीलमेव च}


\twolineshloka
{एष धर्मः कुरुश्रेष्ठ कथितं परमेष्ठिना}
{ब्रह्मणा देवदेवेन अयं चैव सनातनः}


\twolineshloka
{अस्मिन्धर्मे स्थितो राजन्नरो भद्राणि पश्यति}
{श्रौतो वधात्मको धर्म अहिंसापरमार्थिकः ॥'}


\twolineshloka
{धारणाद्धर्ममित्याहुर्धर्मेण विधृताः प्रजाः}
{यः स्याद्धारणसंयुक्तः स धर्म इति निश्चः}


\twolineshloka
{अहिंसार्थाय भूतानां धर्मप्रवचनं कृतम्}
{यः स्यादहिंसासंयुक्तः स धर्म इति निश्चयः}


\twolineshloka
{श्रुतिं धर्मं वदन्त्यन्ये मानान्याहुः परे जनाः}
{न च तं स्वभ्यसूयामो न हि सर्वं विधीयते}


\twolineshloka
{येऽन्यायेन जिहीर्षन्तो धनमिच्छन्ति कर्हिचित्}
{तेभ्यस्तु न तदाख्येयं स धर्म इति निश्चयः}


\twolineshloka
{अकूजनेन चेन्मोक्षो नावकूजेत्कथंचन}
{अवश्यं कूजितव्यं वा शङ्केरन्वाऽप्यकूजनात्}


\twolineshloka
{`येऽन्ये वाऽप्यनृतं कुर्युः कुर्यादेव विचारणम्}
{श्रेयस्तत्रानृतं वक्तुं सत्यादिति विचारितम्}


\twolineshloka
{अक्षयाद्यो वधं राजन्कुर्यादेवाविचारयन्}
{अबुध्वाऽनुशये दोषं श्रेयस्तच्चानृतं भवेत्}


\twolineshloka
{न स्तेनः सह संबन्धान्मुच्यते शपथादपि}
{'श्रेयस्तत्रानृतं वक्तुं सत्यादिति हि धारणा}


\threelineshloka
{यः पापैः सह संबन्धान्मुच्यते शपथादपि}
{न च तेभ्यो धनं दद्याच्छक्ये सति कथंचन}
{पापेभ्यो हि धनं दत्तं दातारमपि पीडयेत्}


\threelineshloka
{स्वशरीरोपरोधेन धनमादातुमिच्छतः}
{सत्यसंप्रतिपत्त्यर्थं यद्ब्रूयुः साक्षिणः क्वचित्}
{अनुक्त्वा तत्र तद्वाच्यं सर्वे तेऽनृतवादिनः}


\twolineshloka
{प्राणात्यये विवाहे च वक्तव्यमनृतं भवेत्}
{अर्थस्य रक्षणार्थाय परेषां धर्मकारणात्}


\twolineshloka
{परेषां सिद्धिमाकाङ्क्षन्न च स्याद्धर्मभिक्षुकः}
{प्रतिश्रुत्य न दातव्यं श्वः कार्यस्तु बलात्कृतः}


\twolineshloka
{यः कश्चिद्धर्मसमयात्प्रच्युतो धर्मजीवनः}
{दण्डेनैव स हन्तव्यस्तं पन्थानं समाश्रितः}


\twolineshloka
{च्युतः सदैव धर्मेभ्यो धनवान्धर्ममाश्रितः}
{कथं स्वधर्ममुत्सृज्य तमिच्छेदुपजीवितुम्}


\twolineshloka
{सर्वोपायैर्नियन्तव्यः पापो निकृतिजीवनः}
{धनमित्येव पापानां सर्वेषामिह निश्चयः}


\threelineshloka
{अविवाह्या ह्यसंभोज्या निकृत्या निरयं गताः}
{च्युता देवमनुष्येभ्यो यथा प्रेतास्तथैव ते}
{[निर्यज्ञास्तपसा हीना मा स्म तैः सह संगमः ॥]}


\twolineshloka
{धनादानाद्दुःखतरं जीविता धिक्प्रयोजनम्}
{इदं ते रोचतां धर्म इति वाच्यं प्रयत्नतः}


\twolineshloka
{न कश्चिदस्ति पापानां धर्म इत्येष निश्चयः}
{तथाविधं च यो हन्यान्न स पापेन लिप्यते}


\twolineshloka
{स्वकर्मणा हतं हन्ति हत एव स हन्यते}
{तेषु यः समयं कश्चित्कुर्वीत हतबुद्धिषु}


\twolineshloka
{यथा काकास्तथैव श्वा तथैवोपधिजीवनः}
{ऊर्ध्वं देहविमोक्षान्ते भवन्त्येतासु योनिषु}


\twolineshloka
{यस्मिन्यथा वर्तति यो मनुष्यस्तस्मिंस्तथा वर्तितव्यं स धर्मः}
{मायाचारो मायया बाधितव्यःसाध्वाचारः साधुनैवाभ्युपेयः}


\chapter{अध्यायः ११०}
\twolineshloka
{युधिष्ठिर उवाच}
{}


\threelineshloka
{क्लिश्यमानेषु भूतेषु तैस्तैर्भावैः पृथक्पृथक्}
{दुर्गाण्यतितरेद्येन तन्मे ब्रूहि पितामह ॥भीष्म उवाच}
{}


\twolineshloka
{आश्रमेषु यथोक्तेषु यथोक्तं ये द्विजातयः}
{वर्तन्ते संयतात्मानो दुर्गाण्यतितरन्ति ते}


\twolineshloka
{ये दम्भान्नाचरन्ति स्म येषां वृत्तिश्च संयता}
{विषयांश्च निगृह्णन्ति दुर्गाण्यतितरन्ति ते}


\twolineshloka
{प्रत्याहुर्नोच्यमाना ये न हिंसन्ति च हिंसिताः}
{प्रयच्छन्ति न याचन्ते दुर्गाण्यतितरन्ति ते}


\twolineshloka
{वासयन्त्यतिथीन्नित्यं नित्यं ये चानसूयकाः}
{नित्यं स्वाध्यायशीलाश्च दुर्गाण्यतितरन्ति ते}


\twolineshloka
{मातापित्रोश्च ये वृत्तिं वर्तन्ते धर्मकोविदाः}
{वर्जयन्ति दिवास्वप्नं दुर्गाण्यतितरन्ति ते}


\twolineshloka
{ये वा पापं न कुर्वन्ति कर्मणा मनसा गिरा}
{निक्षिप्तदण्डा भूतेषु दुर्गाण्यतितरन्ति ते}


\twolineshloka
{ये न लोभान्नयन्त्यर्थान्राजानो रजसाऽन्विताः}
{विषयान्परिरक्षन्ति दुर्गाण्यतितरन्ति ते}


\twolineshloka
{स्वेषु दारेषु वर्तन्ते न्यायलब्धेष्वृतावृतौ}
{अग्निहोत्रपराः सन्तो दुर्गाण्यतितरन्ति ते}


\twolineshloka
{आहवेषु च ये शूरास्त्यक्त्वा मृत्युकृतं भयम्}
{धर्मेण जयमिच्छन्ति दुर्गाण्यतितरन्ति ते}


\twolineshloka
{ये वदन्तीह सत्यानि प्राणत्यागेऽप्युपस्थिते}
{प्रमाणभूता भूतानां दुर्गाण्यतितरन्ति ते}


\threelineshloka
{कर्माण्यकुत्सनार्थानि येषां वाचश्च सूनृताः}
{येषामर्थाश्च साध्वर्था दुर्गाण्यतितरन्ति ते}
{}


\twolineshloka
{अनध्यायेषु ये विप्राः स्वाध्यायं नैव कुर्वते}
{तपोनिष्ठाः सुतपसो दुर्गाण्यतितरन्ति ते}


\twolineshloka
{ये तपश्च तपस्यन्ति कौमारब्रह्मचारिणः}
{विद्या वेदव्रतस्नाता दुर्गाण्यतितरन्ति ते}


\twolineshloka
{ये च संशान्तरजसः संशान्ततमसश्च ये}
{सत्वे स्थिता महाभागा दुर्गाण्यतितरन्ति ते}


\twolineshloka
{येषां न कश्चित्रसति न त्रसन्ति हि कस्यचित्}
{येषामात्मसमो लोको दुर्गाण्यतितरन्ति ते}


\twolineshloka
{परश्रिया न तप्यन्ति ये सन्तः पुरुषर्षभाः}
{ग्राम्यादन्नान्निवृत्ताश्च दुर्गाण्यतितरन्ति ते}


\twolineshloka
{सर्वान्देवान्नमस्यन्ति सर्वधर्मांश्च शृण्वते}
{ये श्रद्दधानाः शान्ताश्च दुर्गाण्यतितरन्ति ते}


\twolineshloka
{ये न मानित्वमिच्छन्ति मानयन्ति च ये परान्}
{मान्यमानान्नमस्यन्ति दुर्गाण्यतितरन्ति ते}


\twolineshloka
{ये च श्राद्धानि कुर्वन्ति तिथ्यांतिथ्यां प्रजार्थिनः}
{सुविशुद्धेन मनसा दुर्गाण्यतितरन्ति ते}


\twolineshloka
{ये क्रोधं संनियच्छन्ति क्रुद्धान्संशमयन्ति च}
{न च रुष्यन्ति भृत्यानां दुर्गाण्यतितरन्ति ते}


\twolineshloka
{मधु मांसं स्त्रियो नित्यं वर्जयन्तीह मानवाः}
{जन्मप्रभृति मद्यं च दुर्गाण्यतितरन्ति ते}


\twolineshloka
{यात्रार्थं भोजनं येषां संतानार्थं च मैथुनम्}
{वाक् सत्यवचनार्थं च दुर्गाण्यतितरन्ति ते}


\twolineshloka
{ईश्वरं सर्वभूतानां जगतः प्रभवाप्ययम्}
{भक्ता नारायणं देवं दुर्गाण्यतितरन्ति ते}


\twolineshloka
{य एष पझरक्ताक्षः पीतवासा महाभुजः}
{सुहृद्धाता च मित्रं च संबन्धी च तवाच्युत}


\twolineshloka
{य इमान्सकलाँल्लोकांश्चर्मवत्परिवेष्टयेत्}
{इच्छन्प्रभुरचिन्त्यात्मा गोविन्दः पुरुषोत्तमः}


\twolineshloka
{स्थितः प्रियहिते नित्यं स एष पुरुषोत्तमः}
{राजंस्तव यदुश्रेष्ठो वैकुण्ठः पुरुषर्षभः}


\twolineshloka
{य एनं संश्रयन्तीह भक्त्या नारायणं हरिम्}
{ते तरन्तीह दुर्गाणि न चात्रास्ति विचारणा}


\threelineshloka
{` अस्मिन्नर्पितकर्माणः सर्वभावेन भारत}
{कृष्णे कमलपत्राक्षे दुर्गाण्यतितरन्ति ते}
{}


\twolineshloka
{लोकरक्षार्थमुत्पन्नमदित्यां कश्यपात्मजम्}
{देवमिन्द्रं नमस्यन्ति दुर्गाण्यतितरन्ति ते}


\twolineshloka
{ब्रह्माणं लोककर्तारं ये नमस्यन्ति सत्पतिम्}
{यष्टव्यं क्रतुभिर्देवं दुर्गाण्यतितरन्ति ते}


\threelineshloka
{यं विष्णुरिन्द्रः शंभुश्च ब्रह्मा लोकपितामहः}
{स्तुवन्ति विविधैः स्तोत्रैर्देवदेवं महेश्वरम्}
{समर्चयन्ति ये शश्वद्दुर्गाण्यतितरन्ति ते ॥'}


\twolineshloka
{दुर्गातितरणं ये च पठन्ति श्रावयन्ति च}
{कथयन्ति च विप्रेभ्यो दुर्गाण्यतितरन्ति ते}


\twolineshloka
{इति कृत्यसमुद्देशः कीर्तितस्ते मयाऽनघ}
{तरते येन दुर्गाणि परत्रेह च मानवः}


\chapter{अध्यायः १११}
\twolineshloka
{युधिष्ठिर उवाच}
{}


\threelineshloka
{असौम्याः सौम्यरूपेण सौम्याश्चासौम्यरूपिणः}
{तादृशान्पुरुषांस्तात कथं विद्यामहे वयम् ॥भीष्म उवाच}
{}


\twolineshloka
{अत्राप्युदाहरन्तीममितिहासं पुरातनम्}
{व्याघ्रगोमायुसंवादं तं निबोध युधिष्ठिर}


\twolineshloka
{पुरिकायां पुरि पुरा श्रीमत्यां पौरिको नृपः}
{पररिंसापरः क्रूरो बभूव पुरुषाधमः}


\twolineshloka
{स त्वायुषि परिक्षीणे जगामानीप्सितां गतिम्}
{गोमायुत्वं च संप्राप्तो दूषितः पूर्वकर्मणा}


\twolineshloka
{संस्मृत्य पूर्वजातिं स्वां निर्वेदं परमं गतः}
{न भक्षयन्ति मांसानि परैरुपहृतान्यपि}


\twolineshloka
{अहिंसा सर्वभूतेषु सत्यवाक् सुदृढव्रतः}
{चकार च यथाकालमाहारं पतितैः फलैः}


\twolineshloka
{`पर्णहारः कदाचिच्च नियमव्रतवानपि}
{कदा चदुदकेनापि वर्तयन्न तु यन्त्रितः ॥'}


\twolineshloka
{श्मशाने तस्य चावासो गोमायोः संमतोऽभवत्}
{जन्मभूम्यनुरोधाच्च नान्यं वा समरोचयत्}


\twolineshloka
{तस्य शौचममृष्यन्तस्ते सर्वे सहजातयः}
{चालयन्ति स्म तां बुद्धिं वचनैः प्रश्रयोत्तरैः}


\twolineshloka
{वसन्पितृवने रौद्रे शौचं लम्भितुमिच्छसि}
{इयं विप्रतिपत्तिस्ते यदा त्वं पिशिताशनः}


\twolineshloka
{तत्समानो भवात्समाभिर्भक्ष्यं दास्यामहे वयम्}
{भुङ्क्ष्व शौचं परित्यज्य यद्धि भुक्तं तदस्ति ते}


\twolineshloka
{इति तेषां वचः श्रुत्वा प्रत्युवाच समाहितः}
{मधुरैः प्रश्रितैर्वाक्यैर्हेतुमद्भिरनिष्ठुरैः}


\twolineshloka
{अप्रमाणा प्रसूतिर्मे शीलतः क्रियते कुलम्}
{प्रार्थयामि च तत्कर्म येन विस्तीर्यते यशः}


\twolineshloka
{श्मशाने यदि मे वासः समाधिर्मे निशाम्यताम्}
{आत्मा फलति कर्माणि नाश्रमो धर्मलक्षणम्}


\twolineshloka
{आश्रमे यो द्विजं हन्याद्दानं दद्यादनाश्रमे}
{किंतु तत्पातकं न स्यात्तद्वा दानं वृथा भवेत्}


\twolineshloka
{भवन्तः स्वार्थलोभेन केवलं भक्षणे रताः}
{अनुबन्धेषु ये दोषास्तान्न पश्यन्ति मोहिताः}


\twolineshloka
{अप्रत्ययकृतां गर्ह्यामर्थापनयदूषिताम्}
{इह चामुत्र चानिष्टां तस्माद्वृत्तिं न रोचये}


\threelineshloka
{तं शुचिं पण्डितं मत्वा शार्दूलः ख्यातविक्रमः}
{कृत्वाऽऽत्मसदृशीं पूजां साचिव्येऽवरयत्स्वयम् ॥शार्दूल उवाच}
{}


\twolineshloka
{सौम्य विज्ञातरूपस्त्वं गच्छ यात्रां मया सह}
{व्रियन्तामीप्सिता भोगाः परिहार्याश्च पुष्कलाः}


\threelineshloka
{तीक्ष्णा इति वयं ख्याता भवन्तं ज्ञापयामहे}
{मृदुपूर्वं प्रशाधि त्वं श्रेयश्चाधिगमिष्यसि ॥भीष्म उवाच}
{}


\twolineshloka
{अथं संपूज्य तद्वाक्यं मृगेन्द्रस्य महात्मनः}
{गोमायुः प्रश्रितं वाक्यं बभाषे किंचिदानतः}


\twolineshloka
{सदृशं मृगराजैतत्तव वाक्यं मदन्तरे}
{यत्सहायान्मृगयसे धर्मार्थकुशलाञ्शुचीन्}


\twolineshloka
{न शक्यं ह्यनमात्येन महत्त्वमनुशासितुम्}
{दुष्टामात्येन वा वीर शरीरपरिपन्थिना}


\twolineshloka
{सहायाननुरक्तांश्च नयज्ञानुपसंहितान्}
{परस्परमसंतुष्टान्विजिगीषूनलोलुपात्}


\twolineshloka
{अनतीतोपदान्प्राज्ञान्हिते युक्तान्मनस्विनः}
{पूजयेथा महाभाग यथा भ्रातृन्यथा पितॄन्}


\twolineshloka
{न त्वेव मम संतोषाद्रोचतेऽन्यन्मृगाधिप}
{न कामये सुखान्भोगानैश्वर्यं वा त्वदाश्रयम्}


\twolineshloka
{न योक्ष्यति हि मे शीलं तव भृत्यैः पुरातनैः}
{ते त्यां विभेदयिष्यन्ति दुःखशीला मदन्तरे}


\twolineshloka
{संश्रयः श्लाघनीयस्त्वमन्येषामपि भास्वताम्}
{कृतात्मा सुमहाभागः पापकेष्वप्यदारुणः}


\twolineshloka
{दीर्घदर्शी महोत्साहः स्थूललक्षो महाबलः}
{कृते चामोघकर्ताऽसि भाग्यैश्च समलंकृतः}


\twolineshloka
{किंतु स्वेनास्मि संतुष्टो दुःखा वृत्तिरनुष्ठिता}
{सेवायां चापि नाभिज्ञः स्वच्छन्देन वनेचरः}


\twolineshloka
{प्राज्ञोपक्रोशदोषाश्च सर्वे संश्रयवासिनाम्}
{वनचर्या तु निःसङ्गा निर्भया विरवग्रहा}


\twolineshloka
{नृपेण हियमाणस्य यत्तिष्ठति भयं हृदि}
{न तत्तिष्ठति तुष्टानां वने मूलफलाशिनाम्}


\twolineshloka
{पानीयं वा निरायासं स्वाद्वन्नं वा गुणोत्तरम्}
{विचार्य खलु पश्यामि तत्सुखं यत्र निर्वृत्तिः}


\twolineshloka
{अपराधैर्न तावन्तो भृत्याः शिष्टा नराधिपैः}
{अपजातैर्यथा भृत्या दूषिताः निधनं गताः}


\twolineshloka
{यदि वा तन्ममा कार्यं मृगेन्द्र यदि मन्यसे}
{समयं कृतमिच्छामि वर्तितव्यं यथाविधि}


\twolineshloka
{मदीया माननीयास्ते श्रोतव्यं च हितं वचः}
{कल्पिता या च मे वृत्तिः सा भवेत्त्वयि सुस्थिरा}


\twolineshloka
{न ------- सचिवैः सह कर्हिचित्}
{तीतिमन्तः परीप्सन्तो वृथा ब्रूयुः परे मयि}


\twolineshloka
{एक एकेन संगम्य रहो ब्रूयां हितं वचः}
{नच ते शातिकार्येषु प्रष्टत्र्योऽस्मि हिताहिते}


\threelineshloka
{--- --- पश्चाच्च न हिंस्याः सचिवास्त्वया}
{मदीयानां च कुपितो मा त्वं दण्डं निपातयेः ॥भीष्म उवाच}
{}


\twolineshloka
{एवमस्त्विति तेनासौ मृगेन्द्रेणाभिपूजितः}
{प्राप्तवान्मतिसाचिव्यं गोमायुर्व्याघ्रचोदितः}


\twolineshloka
{तं तथा सत्कृतं दृष्ट्वा युज्यमानं च कर्मसु}
{प्राद्विषन्कृतसंघाताः पूर्वभृत्या मुहुर्मुहुः}


\twolineshloka
{मित्रबुद्ध्या च गोमायुं सान्त्वयित्वा प्रवेश्य च}
{दोषेषु समयान्नेतुमिच्छन्त्यशुभबुद्धयः}


\twolineshloka
{अन्यथा ह्युषिताः पूर्वं परद्रव्यापहारिणः}
{अशक्ताः किंचिदाहर्तुं द्रव्यं गोमायुयन्त्रिताः}


\twolineshloka
{व्युत्थानं चात्र काङ्क्षद्भिः कथाभिः प्रतिलोभ्यते}
{धनेन महता चैव बुद्धिरस्य विलोभ्यते}


\twolineshloka
{न चापि स महाप्राज्ञस्तस्माद्वैर्याच्चचाल ह}
{अथास्य समयं कृत्वा विनाशाय स्थिताः परे}


\twolineshloka
{ईप्सितं तु मृगेन्द्रस्य मांसं यत्तत्र संस्कृतम्}
{अपनीय स्वयं तद्धि तैर्न्यस्तं तस्य वेश्मनि}


\twolineshloka
{यदर्थं चाप्यपहृतं येन तच्चैव मन्त्रितम्}
{तस्य तद्विदितं सर्वं कारणार्थं च मर्षितम्}


\twolineshloka
{समयोऽयं कृतस्तेन साचिव्यमुपगच्छता}
{नोपघातस्त्वया कार्यो राजन्मैत्रीमिहेच्छता`इति तस्य च मन्त्रस्य स्थित्यर्थं तदुपेक्षितम्}


\twolineshloka
{[क्षुधितस्य मृगेन्द्रस्य भोक्तुमभ्युत्थितस्य च]भोजने चोपहर्तव्ये तन्मांसं नह्यदृश्यत}
{मृगराजेन चाज्ञप्तं मृग्यतां चोर इत्युत}


\twolineshloka
{कृतकैश्चापि तन्मांसं मृगेन्द्राय निवेदितम्}
{सचिवेनापनीतं ते विदुषा प्राज्ञमानिना}


\twolineshloka
{सरोषस्त्वध शार्दूलः श्रुत्वा गोमायुचापलम्}
{बभूवामर्षितो राजा वधं चास्य व्यरोचयत्}


\threelineshloka
{छिद्रं तु तस्य तदृष्ट्वा प्रोचुस्ते पूर्वमन्त्रिणः}
{सर्वेषामेव सोऽस्माकं वृत्तिभङ्गे प्रवर्तते}
{निश्चित्यैवं पुनस्तस्य ते तत्कर्मण्यवर्तयन्}


\twolineshloka
{इदं तस्येदृशं कर्म किं तेन न कृतं भवेत्}
{श्रुतश्च स्वामिना पूर्वं यादृशो नैव तादृशः}


\twolineshloka
{वाङ्भात्रेणैव धर्मिष्ठः स्वभावेन तु दारुणः}
{धर्मच्छझा ह्ययं पापो वृथाचारपरिग्रहः}


\twolineshloka
{कार्यार्थं भोजनाद्येषु व्रतेषु कृतवाञ्श्रमम्}
{यदि विप्रत्ययो ह्येष तदिदं दर्शयाम् ते}


% Check verse!
तन्मांसं तैश्च गोमायोस्तत्क्षणादाशु ढौकितम्
\twolineshloka
{मांसापनयनं श्रुत्वा व्याघ्रस्तेषां च तद्वचः}
{आज्ञापयामास तदा गोमायुर्वध्यतामिति}


\twolineshloka
{गोमायोर्व्यसनं श्रुत्वा शार्दूलजननी ततः}
{मृगराजं हितैर्वाक्यैः संबोधयितुमागमत्}


\twolineshloka
{पुत्र नैतत्त्वया ग्राह्यं कपटारम्भसंयुतम्}
{कर्म संघर्षजैर्दोषैर्दुष्येताशुचिभिः सुचिः}


\twolineshloka
{नोच्छ्रितं सहते कश्चित्प्रक्रिया वैरकारिका}
{शुचेरपि हि युक्तस्य दोष एव निपात्यते}


% Check verse!
[मुनेरपि वनस्थस्य स्वानि कर्माणि कुर्वतःउत्पाद्यन्ते त्रयः पक्षा मित्रोदासीनशत्रवः ॥]
\threelineshloka
{लुब्धानां शुचयो द्वेष्याः कातराणां तरस्विनः}
{मूर्खाणां पण्डिता द्वेष्या दरिद्राणां महाधनाः}
{अधार्मिकाणां धर्मिष्ठा विरूपाणां सुरूपिणः}


\twolineshloka
{बह पण्डिता मूर्खा लुब्धा मायोपजीविनः}
{आहुंर्दोषमदोषस्य बृहस्पतिमतेरपि}


\twolineshloka
{सुन्य्रस्तं ते गृहे मांसं यदद्यापहृतं तव}
{नेचते दीयमानं च साधु तावद्विधीयताम्}


\twolineshloka
{असयाः सत्यसंकाशाः सत्याश्चासत्यदर्शनाः}
{दृश्यन्ते विविधा भावास्तेषु युक्तं परीक्षणम्}


\twolineshloka
{तलवद्दृश्यते व्योम खद्योतो हव्यवाडिव}
{न चैवास्ति तलं व्योम्नि खद्योते न हुताशनः}


\twolineshloka
{तस्मात्प्रत्यक्षदृष्टोऽपि युक्तो ह्यर्थः परीक्षितुम्}
{परीक्ष्य ज्ञापयन्नर्थान्न पश्चात्परितप्यते}


\twolineshloka
{न दुष्करमिदं पुत्रं यत्प्रभुर्घातयेत्परम्}
{श्लाघनीया यशस्या च लोके प्रभवतां क्षमा}


\twolineshloka
{स्थापितोऽयं त्वया पुत्र सामन्तेष्वपि विश्रुतः}
{दुःखेनासाद्यते पात्रं धार्यतामेष ते सुहृत्}


\twolineshloka
{दूषितं परदोषैर्हि गृह्णीते योऽन्यथा शुचिम्}
{स्वयं संदूषितामात्यः क्षिप्रमेव विनश्यति}


\twolineshloka
{एतस्मादरिसंघाताद्गोमायोः कश्चिदागतः}
{धर्मात्मा तेन चाख्यातं यथैतत्कपटं कृतम्}


\twolineshloka
{ततो विज्ञातचारित्रः सत्कृत्य स विमोक्षितः}
{परिष्वक्तश्च सस्नेहं मृगेन्द्रेण पुनः पुनः}


\twolineshloka
{अनुज्ञाय मृगेन्द्रं तु गोमायुर्नीतिशास्त्रवित्}
{तेनामर्षेण संतप्तः प्रायमासितुमैच्छत}


\twolineshloka
{गोमायुं तु स शार्दूलः स्नेहात्प्रसृतलोचनः}
{न्यवारयत्स धर्मिष्ठं पूजया प्रतिपूजयन्}


\twolineshloka
{तं स गोमायुरालोक्य स्नेहादागतसंभ्रमः}
{बभाषे प्रणतो वाक्यं बाष्पगद्गदया गिरा}


\twolineshloka
{पूजितोऽहं त्वया पूर्वं पश्चाच्चैव विमानितः}
{परेषामास्पदं नीतो वस्तुं नार्हाम्यहं त्वयि}


\twolineshloka
{असंतुष्टाश्च्युताः स्थानान्मानात्प्रत्यवरोपिताः}
{स्वयं चोपद्रुता भृत्या ये चाप्युपहिताः परैः}


\twolineshloka
{परिक्षीणाश्च लुब्धाश्च क्रुद्धा भीताः प्रतारिताः}
{हृतस्वा मानिनो ये च त्यक्तोपात्ता महेप्सवः}


\twolineshloka
{संलालिताश्च ये केचिद्व्यसनौघप्रतीक्षिणः}
{अन्तर्हिताः सोहपृतास्ते सर्वेऽपरसाधनाः}


\twolineshloka
{अवमानेन युक्तस्य स्थापितस्य च मे पुनः}
{कथं यास्यसि विश्वासमहमेष्वामि वा कथम्}


\twolineshloka
{समर्थ इति संगृह्य स्थापयित्वा परीक्षितः}
{कृतं च समयं भित्त्वा त्वयाऽहमवमानितः}


\twolineshloka
{प्रथमं यः समाख्यातः शीलवानिति संसदि}
{न वाच्यं तस्य वैगुण्यं प्रतिज्ञां परिरक्षता}


\twolineshloka
{एवं चावमतस्येह विश्वासं मे न यास्यसि}
{त्वयि चापेतविश्वासे ममोद्वेगो भविष्यति}


\twolineshloka
{शङ्कितस्त्वमहं भीतः परे च्छिद्रानुसारिणः}
{अस्त्रिग्धाश्चैव दुस्तोषाः कर्म चैतद्बहुच्छलम्}


\twolineshloka
{दुःखेन श्लिष्यते भिन्नं श्लिष्टं दुःखेन भिद्यते}
{भिन्नश्लिष्टे तु या प्रीतिर्न सा स्नेहेन वर्धते}


\twolineshloka
{कश्चित्तव हिते भर्तुर्दृश्यते न परात्मनः}
{कार्यापेक्षा हि वर्न्तते भावस्त्रिग्धाः सुदुर्लभाः}


\twolineshloka
{सुदुःखं पुरुषज्ञानं चित्तं ह्येषां चलाचलम्}
{समर्थो वाप्यशङ्को वा शतेष्वेकोऽधिगम्यते}


\threelineshloka
{अकस्मात्प्रक्रिया नॄणामकस्माच्चापकर्षणम्}
{शुभाशुभे महत्त्वं च प्रहर्तुं बुद्धिलाघवम् ॥भीष्म उवाच}
{}


\twolineshloka
{एवंविधं सान्त्वमुक्त्वा धर्मकामार्थहेतुमत्}
{प्रसादयित्वा राजानं गोमायुर्वनमभ्यगात्}


\twolineshloka
{अगृह्यानुनयं तस्य मृगेन्द्रस्य च बुद्धिमान्}
{गोमायुः प्रायमासीनस्त्यक्त्वा देहं दिवं ययौ}


\chapter{अध्यायः ११२}
\twolineshloka
{युधिष्ठिर उवाच}
{}


\threelineshloka
{किं पार्थिवेन कर्तव्यं किंच कृत्वा सुखी भवेत्}
{तन्ममाचक्ष्व तत्त्वेन सर्वधर्मभृतां वर ॥भीष्म उवाच}
{}


\twolineshloka
{हन्त तेऽहं प्रवक्ष्यामि शृणु कार्यैकनिश्चयम्}
{यथा राज्ञेह कर्तव्यं यच्च कृत्वा सुखी भवेत्}


\twolineshloka
{नचैवं वर्तितव्यं स्म यथेदमनुशुश्रुम्}
{उष्ट्रस्य तु महद्वृत्तं तन्निबोध युधिष्ठिर}


\twolineshloka
{जातिस्मरो महानुष्ट्रः प्रजापतिकुलोद्भवः}
{तपः सुमहदातिष्ठदरण्ये संशितव्रतः}


\threelineshloka
{तपसस्तस्य चान्तेऽथ प्रीतिमानभवद्विभुः}
{वरेण च्छन्दयामास ततश्चैनं पितामहः ॥उष्ट्र उवाच}
{}


\twolineshloka
{भगवंस्त्वत्प्रसादान्मे दीर्घा ग्रीवा भवेदियम्}
{योजनानां शतं साग्रमिच्छेयं चारितुं विभो}


\twolineshloka
{एवमस्विति चोक्तः स वरदेन महात्मना}
{प्रतिलभ्य वरं श्रेष्ठं ययावुष्ट्रः स्वकं वनम्}


\twolineshloka
{स चकार तदाऽऽलस्यं वरदानात्सुदुर्मतिः}
{न चैच्छच्चिरतुं गन्तुं दुरात्मा कालमोहितः}


\twolineshloka
{स कदाचित्प्रसार्यैव तां ग्रीवां शतयोजनाम्}
{चचार श्रान्तहृदयो वातश्चागात्ततो महान्}


\twolineshloka
{स गुहायां शिरोग्रीवां निधाय पशुरात्मनः}
{आस्ते वर्षमथाभ्यागात्सुमहत्प्लावयज्जगत्}


\twolineshloka
{अथ शीतपरीताङ्गो जम्बुकः क्षुच्छ्रमान्वितः}
{सदारस्तां गुहामाशु प्रविवेश जलार्दितः}


% Check verse!
स दृष्ट्वा मांसजीवी तु सुभृशं क्षुच्छ्रमान्वितः ॥अभक्षयत्ततो ग्रीवामुष्ट्रस्य भरतर्षभ
\twolineshloka
{यदा त्वबुध्यतात्मानं भक्ष्यमाणं स वै पशु}
{तदा संकोचने यत्नमकरोद्भृशदुःखितः}


\twolineshloka
{यावदूर्ध्वमधश्चैव ग्रीवां संक्षिपते पशुः}
{तावत्तेन सदारेण जम्बुकेन स भक्षितः}


\twolineshloka
{स हत्वा भक्षयित्वा च तमुष्ट्रं जम्बुकस्तदा}
{विगते वातवर्पे तु निश्चक्राम गुहोदरात्}


\twolineshloka
{एवं दुर्बुद्धिना प्राप्तमुष्ट्रेण निधनं तदा}
{आलस्यस्य क्रमात्पश्य महान्तं दोषमागतम्}


\twolineshloka
{त्वमप्येवंविधं हित्वा योगेन नियतेन्द्रियः}
{वर्तस्व बुद्धिमूलं तु विजयं मनुरब्रवीत्}


\twolineshloka
{बुद्धिश्रेष्ठानि कर्माणि बाहुमध्यानि भारत}
{तानि जङ्घाजघन्यानि भारप्रत्यवराणि च}


\threelineshloka
{राज्यं तिष्ठति दक्षस्य संगृहीतेन्द्रियस्य च}
{[आर्तस्य बुद्धिमूलं हि विजयं मनुरब्रवीत्}
{]गुप्तं मन्त्रं श्रुतवतः सुसहायस्य चानघ}


\threelineshloka
{`असहायवतो ह्यर्था न तिष्ठन्ति कदाचन}
{'परीक्षितसहायस्य तिष्ठन्तीह युधिष्ठिर}
{सहाययुक्तेन मही कृत्स्ना शक्या प्रशासितुम्}


\twolineshloka
{इदं हि सद्भिः कथितं विधिज्ञैःपुरा महेन्द्रप्रतिमप्रभावः}
{मयाऽपि चोक्तं तव शास्त्रदृष्ट्यात्वमप्रमत्तः प्रचरस्व राजन्}


\chapter{अध्यायः ११३}
\twolineshloka
{युधिष्ठिर उवाच}
{}


\threelineshloka
{राजा राज्यमनुप्राप्य दुर्बलो भरतर्षभ}
{अमित्रस्यातिवृद्धस्य कथं तिष्ठेदसाधनः ॥भीष्म उवाच}
{}


\twolineshloka
{अत्राप्युदाहरन्तीममितिहासं पुरातनम्}
{सरितां चैव संवादं सागरस्य च भारत}


\threelineshloka
{सुरारिनिलयः शश्वत्सागरः सरितां पतिः}
{पप्रच्छ सरितः सर्वाः संशयं जातमात्मनः ॥सागर उवाच}
{}


\twolineshloka
{समूलशाखान्पश्यामि निहतान्क्वापि नो द्रुमान्}
{युष्माभिरिह पूर्णाभिरन्यांस्तत्र न वेतसान्}


\twolineshloka
{अ-पकायश्चाल्पसारो वेतसः कूलजश्च यः}
{अज्ञया वा नानीतः किं च वा तेन वः कृतम्}


\twolineshloka
{तदहं श्रोतुमिच्छामि सर्वासामेव वो मतम्}
{यथा चेमानि कूलानि हित्वा नायाति वेतसः}


\threelineshloka
{तत्र प्राह नदी गङ्गा वाक्यमुत्तरमर्थवत्}
{हेतुमद्ग्राहकं चैव सागरं सरितां पतिम् ॥गङ्गोवाच}
{}


\twolineshloka
{तिष्ठन्त्येते यथास्थानं नगा ह्येकनिकेतनाः}
{ततस्त्यजन्ति तत्स्थानं प्रातिलोम्यान्न वेतसः}


\twolineshloka
{वेतसो वेगमायान्तं दृष्ट्वा नमति नापरे}
{स च वेगे ह्यतिक्रान्ते स्थानमापद्यते पुनः}


\twolineshloka
{कालज्ञः समयज्ञश्च सदावश्यश्च नो द्रुमः}
{अनुलोमवृत्तितस्तब्धस्तेन त्वां नैति वेतसः}


\threelineshloka
{मारुतोदकवेगेन ये नमन्त्युन्नमन्ति च}
{ओषध्यः पादपा गुल्मा न ते यान्ति पराभवम् ॥भीष्म उवाच}
{}


\twolineshloka
{यो हि शत्रोर्विवृद्धस्य प्रभोर्बन्धविनाशने}
{पूर्वं न सहते वेगं क्षिप्रमेव विनश्यति}


\twolineshloka
{सारासारं बलं वीर्यमात्मनो द्विषतश्च यः}
{जानन्विचरति प्राज्ञो न स याति पराभवम्}


\twolineshloka
{एवमेव यदा विद्वान्मन्यते विपुलं बलम्}
{संश्रयेद्वैतसीं वृत्तिमेतत्प्रज्ञानलक्षणम्}


\chapter{अध्यायः ११४}
\twolineshloka
{युधिष्ठिर उवाच}
{}


\threelineshloka
{विद्वान्मूढप्रगल्भेन मृदुस्तीक्ष्णेन भारत}
{आक्रुश्यमानः सदसि कथं कुर्यादरिंदम् ॥भीष्म उवाच}
{}


\twolineshloka
{श्रूयतां पृथिवीपाल यथैऽषोर्थोऽवगम्यते}
{सदा सचेताः सहते नरस्येहाल्पचेतसः}


\twolineshloka
{आक्रुश्य दूष्यमाणश्च सुकृतं तस्य विन्दति}
{दुष्कृतं चात्मनो मर्षी तस्मिन्नेव प्रमार्जति}


\twolineshloka
{गर्हितं तमुपेक्षेत वाश्यमानमिवातुरम्}
{लोके विद्वेषमापन्नो निष्फलं प्रतिपद्यते}


\threelineshloka
{इति संश्लाघते नित्यं तेन पापेन कर्मणा}
{इदमुक्तो मया कश्चित्सर्वतो जनसंसदि}
{स तत्र व्रीडितः शुष्को मृतकल्पोऽवतिष्ठते}


\twolineshloka
{श्लाघन्नश्लाघनीयेन कर्मणा निरपत्रपः}
{उपेक्षितव्यो दान्तेन तादृशः पुरुषाधमः}


% Check verse!
यद्यद्ब्रूयादल्पमतिस्तत्तदस्य सहेत्तदा
\twolineshloka
{प्रकृत्या हि प्रशंसन्वा निन्दन्वा किं करिष्यति}
{वने काक इवाबुद्धिर्वाश्यमानो निरर्थकम्}


\twolineshloka
{यदि वाग्भिः प्रयोगः स्यात्प्रयोज्यः पापकर्मणा}
{वागेवार्थो भवेत्तस्य न ह्येवार्थो जिघांसतः}


\twolineshloka
{निषेकं वै परस्यासावाचष्टे वृत्तचेष्टया}
{मयूर इव कौपीनं नृत्यं संदर्शयन्निव}


\twolineshloka
{यस्यावाच्यं न लोकेऽस्मिन्नाकार्यं चापि किंचन}
{वाचं तेन न संदध्याच्छुचिः संश्लिष्टकर्मणा}


\twolineshloka
{प्रत्यक्षं गुणवादी यः परोक्षं तु विनिन्दकः}
{स मानवः श्ववल्लोके नष्टलोकपरायणः}


\twolineshloka
{तादृग्दिनशतं चापि यद्ददाति जुहोति च}
{परोक्षेणापवादेन तं नाशयति तत्क्षणात्}


\twolineshloka
{तस्मात्प्राज्ञो नरः सद्यस्तादृशं पापचेतसम्}
{वर्जयेन्मतिमान्वर्ज्यं सारमेयामिषं यथा}


\twolineshloka
{परिवादं ब्रुवाणो हि दुरात्मा वै महाजने}
{प्रकाशयति दोषान्स्वान्सर्पः फणमिवोन्नतम्}


\twolineshloka
{तं स्वकर्मणि कुर्वाणं प्रतिकर्तुं य इच्छति}
{भस्मकूट इवाबुद्धिः खरो रजसि मज्जति}


\twolineshloka
{मनुष्यसालावृकमप्रशान्तंजनापवादे सततं निविष्टम्}
{मातङ्गमुन्मत्तमिवोन्नदन्तंत्यजेत तं श्वानमिवातिरौद्रम्}


\twolineshloka
{अनार्यजुष्टे पथि वर्तमानंदमादपेतं विनयाच्च पापम्}
{अरिव्रतं नित्यमभूतिकामंधिगस्तु तं पापमतिं मनुष्यम्}


\twolineshloka
{प्रत्युच्यमानस्त्वथ भूय एवनिशाम्य माभूस्त्वमथार्तरूपः}
{उच्चस्य नीचेन हि संप्रयोगंविगर्हयन्ति स्थिरबुद्धयो ये}


\twolineshloka
{क्रुद्धो दशेद्वाऽपि च ताडयेद्वास पांसुभिर्वा विकिरेत्तुषैर्वा}
{विवृत्य दन्तांश्च विभीषयेद्वासिद्धं हि मूढे कुपिते नृशंसे}


\twolineshloka
{विगर्हणां नाऽपि दुरात्मना कृतांसहेत यः संसदि दुर्जनानाम्}
{पठेदिदं चापि निदर्शनं सदान वाङ्भयं स लभति किंचिदप्रियम्}


\chapter{अध्यायः ११५}
\twolineshloka
{युधिष्ठिर उवाच}
{}


\twolineshloka
{पितामह महाप्राज्ञ संशयो मे महानयम्}
{संछेत्तव्यस्त्वया राजन्भवान्कुलकरो हि नः}


\twolineshloka
{पुरुषाणामयं तात दुर्वृत्तानां दुरात्मनाम्}
{कथितो वाक्यसंचारस्ततो विज्ञापयामि ते}


\twolineshloka
{यद्धितं राज्यतन्त्रस्य कुलस्य च सुखोदयम्}
{अयत्यां च तदात्वे च क्षेमवृद्धिकरं च तत्}


\twolineshloka
{पुत्रपौत्राभिरामं च राष्ट्रवृद्धिकरं च यत्}
{अन्नपाने शरीरे च हितं यत्तद्ब्रवीहि मे}


\twolineshloka
{अभिषिक्तो हि यो राजा राज्यस्थो मित्रसंवृत}
{समुहृत्समुपेतो वा स कथं रञ्जयेत्प्रजाः}


\twolineshloka
{यो ह्यसत्प्रग्रहरतिः स्नेहरागबलात्कृतः}
{इन्द्रियाणामनीशत्वादसज्जनबुभूषकः}


\twolineshloka
{तस्य भृत्या विमुखतां यान्ति सर्वे कुलोद्गताः}
{न च भृत्यबलैरर्थैः स राजा संप्रयुज्यते}


\twolineshloka
{एतन्मे चिन्तयानस्य राजधर्मान्दिवानिशम्}
{बृहस्पतिसमो बुद्ध्या भवाञ्शंसितुमर्हति}


\twolineshloka
{शासता पुरुषव्याघ्र त्वं नः कुलहिते रतः}
{क्षत्ता चैको महाप्राज्ञो यो नः शंसति सर्वदा}


\twolineshloka
{त्वत्तः कुलहितं वाक्यं श्रुत्वा राज्यहितोदयम्}
{अमृतस्याव्ययस्येव तृप्तः स्वप्स्याम्यहं सुखम्}


\twolineshloka
{कीदृशाः सन्निकर्षस्था भृत्याः सर्वगुणान्विताः}
{कीदृशैः किं कुलीनैर्वा सह यात्रा विधीयते}


\threelineshloka
{न ह्येको भृत्यरहितो राजा भवति रक्षिता}
{राज्यं चेदं जनः सर्वस्तत्कुलीनः प्रशासति ॥भीष्म उवाच}
{}


% Check verse!
न च प्रशास्तुं राज्यं हि शक्यमेकेन भारत
\twolineshloka
{असहायवता तात नैवार्थाः केचिदप्युत}
{लब्धुं लब्धा ह्यपि सदा रक्षितुं भरतर्षभ}


\twolineshloka
{यस्य भृत्यजनः सर्वो ज्ञानविज्ञानकोविदः}
{हितैषी कुलजः स्निग्धः स राज्यफलमश्नुते}


\twolineshloka
{मन्त्रिणो यस्य कुलजा असंहार्याः सहोषिताः}
{नृपतेर्मतिमाप्सन्ते सत्पथज्ञानकोविदाः}


\twolineshloka
{अनागतविधातारः कालज्ञानविशारदाः}
{अतिक्रान्तमशोचन्तः स राज्यफलमश्नुते}


\twolineshloka
{समदुःखसुखा यस्य सहायाः प्रियकारिणः}
{अर्थचिन्तापराः सभ्याः स राज्यफलमश्नुते}


\twolineshloka
{यस्य नार्तो जनपदः सन्निकर्षगतः सदा}
{अक्षुद्रः सत्पथालम्बी स राजा राज्यभाग्भवेत्}


\twolineshloka
{कोशोऽक्षपटलं यस्य कोशवृद्धिकरैर्नरैः}
{आप्तैस्तुष्टैश्च पृष्टैश्च धार्यते स नृपोत्तमः}


\twolineshloka
{कोष्ठागारमसंहार्यैराप्तैः संचयतत्परैः}
{पात्रभूतैरलुब्धैश्च पाल्यमानं गुणी भवेत्}


\twolineshloka
{व्यवहारश्च नगरे यस्य धर्मफलोदयः}
{दृश्यते शङ्खलिखितः स धर्मफलभाङ् नृपः}


\twolineshloka
{संगृहीतमनुष्यश्च यो राजा राजधर्मवित्}
{षङ््भागं परिगृह्णाति स धर्मफलमश्नुते}


\chapter{अध्यायः ११६}
\twolineshloka
{`युधिष्ठिर उवाच}
{}


\threelineshloka
{न सन्ति कुलजा यत्र सहायाः पार्थिवस्य तु}
{अकुलीनाश्च कर्तव्या न वा भरतसत्तम् ॥'भीष्म उवाच}
{}


\twolineshloka
{अत्राप्युदाहरन्तीममितिहासं पुरातनम्}
{निदर्शनं परं लोके सज्जनाचरितं सदा}


\twolineshloka
{अस्यैवार्थस्य सदृशं यच्छ्रुतं मे तपोवने}
{जामदग्न्यस्य रामस्य यदुक्तमृषिसत्तमैः}


\twolineshloka
{वने महति कस्मिंश्चिदमनुष्यनिषेविते}
{ऋषिर्मूलफलाहारो नियतो नियतेन्द्रियः}


\twolineshloka
{दीक्षादमपरिश्रान्तः स्वाध्यायपरमः शुचिः}
{उपवासविशुद्धात्मा सततं सत्पथे स्थितः}


\twolineshloka
{तस्य संदृश्य सद्भावमुपविष्टस्य धीमतः}
{सर्वे सत्वाः समीपस्था भवन्ति वनचारिणः}


\twolineshloka
{सिंहा व्याघ्राः सशरभा मत्ताश्चैव महागजाः}
{द्वीपिनः खङ्गभल्लूका ये चान्ये भीमदर्शनाः}


\twolineshloka
{ते सुखप्रश्नदाः सर्वे भवन्ति क्षतजाशनाः}
{तस्यर्षेः शिष्यवच्चैव चित्तज्ञाः प्रियकारिणः}


\twolineshloka
{उक्त्वा च ते सुखप्रश्नं सर्वे यान्ति यथासुखम्}
{ग्राम्यस्त्वेकः पशुस्तत्र नाजहात्स महामुनिम्}


\twolineshloka
{भक्तोऽनुरक्तः सततमुपवासकृशोऽबलः}
{फलमूलोत्तराहारः शान्तः शिष्टाकृतिर्यथा}


\twolineshloka
{तस्यर्षेरुपविष्टस्य पादमूले महामतेः}
{मनुष्यवद्गतो भावं स्नेहबद्धोऽभवद्भृशम्}


\twolineshloka
{ततोऽभ्ययान्महारौद्रो द्वीपी क्षतजभोजनः}
{श्वार्थमत्यर्थमुद्धुष्टः क्रूरः कालइवान्तकः}


\twolineshloka
{लेलिह्यमानस्तृषितः पुच्छास्फोटनतत्परः}
{व्यादितास्यः क्षुधाः भुग्नः प्रार्थयानस्तदामिषम्}


\twolineshloka
{दृष्ट्वा तं क्रूरमायान्तं जीवितार्थी नराधिप}
{प्रोवाच श्वा मुनिं तत्र तच्छृणुष्व विशांपते}


\threelineshloka
{श्वशत्रुर्भगवन्नेष द्वीपी मां हन्तुमिच्छति}
{त्वत्प्रसादाद्भयं न स्यादस्मान्मम महामुने}
{[तथा कुरु महाबाहो सर्वज्ञस्त्वं न संशयः}


\threelineshloka
{स मुनिस्तस्य विज्ञाय भावज्ञो भयकारणम्}
{रुतज्ञः सर्वसत्वानां तमैश्वर्यसमन्वितः ॥]मुनिरुवाच}
{}


\twolineshloka
{न भयं द्वीपिनः कार्यं मृत्युतस्ते कथंचन}
{एष श्वरूपरहितो द्वीपी भवसि पुत्रक}


\twolineshloka
{ततः श्वा द्वीपितां नीतो जाम्बूनदनिभाकृति}
{चित्राङ्गो विस्फुरद्दंष्ट्रो वने वसति निर्भयः}


\twolineshloka
{तं दृष्ट्वा संमुखे द्वीपी आत्मनः सदृशं पशुम्}
{अविरुद्धस्ततस्तस्य क्षणेन समपद्यत}


\twolineshloka
{ततोऽभ्ययान्महारौद्रो व्यादितास्यः क्षुधान्विः}
{द्वीपिनं लेलिहन्वक्रं व्याघ्रो रुधिरलालसः}


\twolineshloka
{व्याघ्रं दृष्ट्वा क्षुधा भुग्नं दंष्ट्रिणं वनचारिणम्}
{द्वीपि जीवितरक्षार्थमृषिं शरणमेयिवान्}


\twolineshloka
{ततः संवासजं स्नेहमृषिणा कुर्वता तदा}
{स द्वीपी व्याघ्रतां नीतो रिपुभ्यो बलवत्तरः}


\twolineshloka
{ततोऽदृष्ट्वा स शार्दूलो नाभ्यघ्नत्तं विशांपते}
{स तु श्चा व्याघ्रतां प्राप्य बलवान्पिशिताशनः}


\threelineshloka
{न मूलफलभोगेषु स्पृहामप्यकरोत्तदा}
{यथा मृगपतिर्नित्यं प्रकाङ्क्षति वनौकसः}
{तथैव स महाराज व्याघ्रः समभवत्तदा}


\chapter{अध्यायः ११७}
\twolineshloka
{भीष्म उवाच}
{}


\twolineshloka
{व्याघ्रश्वोटजमूलस्थस्तृप्तः सुप्तो हतैर्मृगैः}
{नागभागात्तमुद्देशं मत्तो मेघ इवोत्थितः}


\twolineshloka
{प्र--करटः प्रांशुः पझी विततकुम्भकः}
{सु--णो महाकायो मेघगम्भीरनिः स्वनः}


\twolineshloka
{तं दृष्ट्वा कुञ्जरं मत्तमायान्तं बलगर्वितम्}
{व्या हस्तिभयान्त्रस्तस्तमृषिं शरणं ययौ}


\twolineshloka
{ततोऽनयत्कुञ्जरत्वं व्याघ्रं तमृषिसत्तमः}
{महमेघोपमं दृष्ट्वा स भीतो ह्यभवद्गजः}


\twolineshloka
{ततः कमलषण़्डानि सल्लकीगहनानि च}
{व्य--रत्स मुदायुक्तः पझरेणुविभूषितः}


\twolineshloka
{कदचिद्दममाणस्य हस्तिनः संमुखं तदा}
{ऋषेस्तस्योटस्थस्य कालोऽगच्छद्दिवानिशम्}


\twolineshloka
{अथाजगाम तं देशं केसरी केसरारुणः}
{गिन्विन्दरजो भीमः सिंहो नागकुलान्तकः}


\twolineshloka
{तं--- सिंहमायान्तं नागः सिंहभयार्दितः}
{ऋषिं शरणमापेदे वेपमानो भयातुरः}


\twolineshloka
{स त-ः सिंहतां नीतो गजेन्द्रो मुनिना तदा}
{तं च नागणयत्सिंहं तुल्यजातिसमन्वयात्}


\twolineshloka
{दृष्ट्वा च सोऽभवत्सिंहो वन्यो हिंसन्नवाग्बलः}
{स चाश्रपेऽवसत्सिंहस्तस्मिन्नेव सुखी वने}


\twolineshloka
{न चान्ये क्षुद्रपशवस्तपोवनसमीपतः}
{व्यदृश्यन्त तदा त्रस्ता जीविताकाङ्क्षिणस्तथा}


\twolineshloka
{कदाचित्कालयोगेन सर्वप्राणिविहिंसकः}
{बलवान्क्षतजाहारो नानासत्वभयंकरः}


\twolineshloka
{अष्टपादर्ध्वनयनः शरभो वनगोचरः}
{तं सिंहं हन्तुमागच्छन्मुनेस्तस्य निवेशने}


\twolineshloka
{`तं दृष्ट्वा शरभं यान्तं सिंहः परभयान्वितः}
{ऋषिं शरणमापेदे वेपमानः कृताञ्जलिः ॥'}


\threelineshloka
{सं मुनिः शरभं चक्रे बलोत्कटमरिंदम}
{ततः स शरभो वन्यो मुनेः शरभमग्रतः}
{दृष्ट्वा बलिनमत्युग्रं द्रुतं संप्राद्रवद्वनम्}


\twolineshloka
{स एवं शरभस्थाने न्यस्तो वै मुनिना तदा}
{मुनेः पार्श्वगतो नित्यं शरभः सुखमाप्तवान्}


\twolineshloka
{ततः शरभसंत्रस्ताः सर्वे मृगगणा वनात्}
{दिशः संप्राद्रवत्राजन्भयाज्जीवितकाङ्क्षिणः}


\twolineshloka
{शरभोऽप्यतिसंहृष्टो नित्यं प्राणिवधे रतः}
{फलमूलाशनं कर्तुं नैच्छत्स पिशिताशनः}


\twolineshloka
{ततः क्षुद्रसमाचारो बलेन च समन्वितः}
{इयेष तं मुनिं हन्तुमकृतज्ञः कृतान्वयः}


% Check verse!
`चिन्तयामास च तदा शरभः श्वानपूर्वकः
\twolineshloka
{अस्य प्रभावात्संप्राप्तो वाङ्भात्रेणैव केवलम्}
{शरभत्वं सुदुष्प्रापं सर्वभूतभयंकरम्}


\twolineshloka
{अन्येऽप्यत्र भयत्रस्ताः सन्ति सत्वा भयार्दिताः}
{मुनिमाश्रित्य जीवन्तो मृगाः पक्षिगणास्तथा}


\twolineshloka
{तेषामपि कदाचिच्च शरभत्वं प्रयच्छति}
{सर्वसत्वोत्तमं लोके बलं यत्र प्रतिष्ठितम्}


% Check verse!
पक्षिणामप्ययं दद्यात्कदाचिद्गारुडं बलम्
\twolineshloka
{यावदन्यस्य संप्रीतः कारुण्यं तु समाश्रितः}
{न ददाति बलं तुष्टः सत्वस्यान्यस्य कस्यचित्}


\twolineshloka
{तावदेनमहं विप्रं वधिष्यामि च शीघ्रतः}
{स्थातुं मया शक्यमिह मुनिघातान्न संशयः ॥'}


\twolineshloka
{ततस्तेन तपःशक्त्या विदितो ज्ञानचक्षुषा}
{विज्ञाय च मुनिः प्राज्ञस्ततः शापं प्रयुक्तवान्}


\twolineshloka
{`अहमग्निप्रभो नाम मुनिर्भृगुकुलान्वयः}
{मनसा निर्दहेयं च जगत्संधारयामि च}


\threelineshloka
{मम वश्यं जगत्सर्वं देवा यच्च चराचरम्}
{सन्ति देवाश्च ये भीताः स्वधर्मं न त्यजन्ति ये}
{स्वधर्माच्चलितान्सर्वान्वाङ्भात्रेणापि निर्दहे}


\twolineshloka
{किमङ्ग त्वं मया नीतः शरभत्वमनामयम्}
{क्रूरः स सर्वभूतेषु हीनश्चाशुचिरेव च ॥'}


\twolineshloka
{श्वा त्वं द्वीपित्वमापन्नो द्वीपी व्याघ्रत्वमागतः}
{व्याघ्रान्नागो मदपटुर्नागः सिंहत्वमागतः}


\twolineshloka
{सिंहस्त्वं बलमापन्नो भूयः शरभतां गतः}
{मया स्नेहपरीतेन विसृष्टो न कुलान्वयः}


\twolineshloka
{यस्मादेवमपापं मां पाप हिंसितुमिच्छसि}
{तस्मात्स्वयोनिमापन्नः पुनः श्वानो भविष्यसि}


\twolineshloka
{ततो मुनिजनद्वेष्टा दुष्टात्मा प्राकृतोऽबुधः}
{ऋषिणा शरभः शप्तस्तद्रूपं पुनराप्तवान्}


\chapter{अध्यायः ११८}
\twolineshloka
{भीष्म उवाच}
{}


\twolineshloka
{स श्वा प्रकृतिमापन्नः परं दैन्यमुपागमत्}
{ऋषिणा हुंकृतः पापस्तपोवनबहिष्कृतः}


\twolineshloka
{एवं राज्ञा मतिमता विदित्वा शीलशौचताम्}
{आर्जवं प्रकृतिं सत्वं श्रुतं वृत्तं कुलं दमम्}


\twolineshloka
{अनुक्रोशं बलं वीर्यं प्रभावं प्रशमं क्षमाम्}
{भृत्यायेमन्त्रिणो योग्यास्तत्र स्थाप्याः सुरक्षिताः}


\twolineshloka
{नाषरीक्ष्य महीपालः प्रकर्तुं भृत्यमर्हति}
{अकुलीननराकीर्णो न राजा सुखमेधते}


\twolineshloka
{कुलजः प्राकृतो राजंस्तत्कुलीनतया सदा}
{न पापे कुरुते बुद्धिं निन्द्यमानोऽप्यनागसि}


\twolineshloka
{अकुलीनस्तु पुरुषः प्राकृतः साधुसंक्षयात्}
{दुर्लभैश्वर्यतां प्राप्तो निन्दितः शत्रुतां व्रजेत्}


\twolineshloka
{`काकः श्वानोऽकुलीनश्च बिडालः सर्प एव च}
{अकुलीना च या नारी तुल्यास्ते परिकीर्तिताः}


\twolineshloka
{लोकपालाः सदोद्विग्नाः पश्यन्त्यकुलजान्यथा}
{नारीं वा पुरुषं वाऽथ शीलं तत्रापि कारणम्}


\twolineshloka
{दुष्कुलीना च या स्त्री स्याद्दुष्कुलीनश्च यः पुम}
{अहिंसाशीलसंयोगाद्धर्मश्चाऽऽकुलतां व्रजेत्}


\twolineshloka
{धर्मं प्रति महाराज श्लोकानाह बृहस्पतिः}
{शृणु सर्वान्महीपाल हृदि तांश्च करिष्यसि}


\twolineshloka
{असितं सितकर्माणं यथा दान्तं तपस्विनम्}
{वृत्तस्थमपि चण्डालं तं देवा ब्राह्मणं विदुः}


\twolineshloka
{यदि घातयते कश्चित्पापसत्वं प्रजाहितः}
{सर्वसत्वहितार्थाय न तेनासौ विहिंसकः}


\twolineshloka
{द्वीपिनं शरभं सिंहं व्याघ्रं कुञ्जरमेव च}
{महिषं च वराहं च सूकरं श्वानपन्नगान्}


\twolineshloka
{गोब्राह्मणहितार्थाय बालस्त्रीरक्षणाय च}
{वृद्धातुरपरित्राणे यो हिनस्ति स धर्मवित्}


\twolineshloka
{ब्राह्मणः पापकर्मा च म्लेच्छो वा धार्मिकः शु वः}
{श्रेयांस्तत्र भवेन्म्लेच्छो ब्राह्मणः पापकृत्तमः}


\twolineshloka
{दुष्कुलीनः कुलीनो वा यः कश्चिच्छीलवान्नरः}
{प्रकृतिं तस्य विज्ञाय स्थिरां वा यदि वाऽस्थिराम्}


\twolineshloka
{शीलं वाऽनुत्तमं कर्म कुर्याद्राजा समाहितः}
{नियुञ्जीत महीपालो दुर्वृत्तं पापकर्मसु ॥'}


\twolineshloka
{कुलीनं शिक्षितं प्राज्ञं ज्ञानविज्ञानकोविदम्}
{सर्वशास्त्रार्थतत्त्वज्ञं सहिष्णुं देशजं तथा}


\twolineshloka
{कृतज्ञं बलवन्तं च क्षान्तं दान्तं जितेन्द्रियम्}
{अलुब्धं लब्धसंतुष्टं स्वामिमित्रबुभूषकम्}


\twolineshloka
{सचिवं देशकालज्ञं सर्वसंग्रहणे रतम्}
{संस्कृतं युक्तवचनं हितैषिणमतन्द्रितम्}


\twolineshloka
{युक्ताचारं स्वविषये संधिविग्रहकोविदम्}
{शस्तं त्रिवर्गवेत्तारं पौरजानपदप्रियम्}


\twolineshloka
{सेनाव्यूहनतत्त्वज्ञं बलहर्षणकोविदम्}
{इङ्गिताकरातत्त्वज्ञं यात्रासेनाविशारदम्}


\twolineshloka
{हस्तिशिक्षाश्वतत्त्वज्ञमहंकारविवर्जितम्}
{प्रगल्भं दक्षिणं दान्तं बलिनं युक्तमन्त्रिणम्}


\twolineshloka
{चौक्षं चौक्षजनाकीर्णं सुवेषं सुखदर्शनम्}
{नायकं नीतिकुशलं गुणैः षङ्भिः समन्वितम्}


\twolineshloka
{अस्तब्धं प्रश्रितं श्लक्ष्णं मृदुवादिनमेव च}
{धीरं महर्द्धि च देशकालोपपादकम्}


\twolineshloka
{सचि यः प्रकुरुते न चैनमवमन्यते}
{तस्य विस्तीर्यते राज्यं ज्योत्स्ना ग्रहपतेरिव}


\twolineshloka
{एतैरेव गुणैर्युक्तो राजा शास्त्रविशारदः}
{एष्टव्यो धर्मपरमः प्रजापालनतत्परः}


\twolineshloka
{धीरो मर्षी शुचिः शीघ्रः काले पुरुषकारवित्}
{शुश्रूषुः श्रुतवाञ्श्रोता ऊहापोहविशारदः}


\twolineshloka
{मेधावी धारणायुक्तो यथान्यायोपपादकः}
{दान्तः सदा प्रियाभाषी क्षमावांश्च विपर्यये}


\twolineshloka
{नातिच्छेत्ता स्वयंकारी श्रद्धालुः सुखदर्शनः}
{आर्तहस्तप्रदो नित्यमाप्तामात्यो नये रतः}


\twolineshloka
{नाहंवादी ननिर्द्वन्द्वो नयत्किंचनकारकः}
{कृते कर्मण्यमोघानां कर्ता भृत्यजनप्रियः}


\twolineshloka
{संगृहीतजनोऽस्तब्धः प्रसन्नवदनः सदा}
{त्राता भृत्यजनापेक्षी न क्रोधी सुमहामनाः}


\twolineshloka
{युक्तदण्डो न निर्दण्डो धर्मकार्यानुशासनः}
{चारनेत्रः प्रजावेक्षी धर्मार्थकुशलः सदा}


\twolineshloka
{राजा गुणशताकीर्ण एष्टव्यस्तादृशो भवेत्}
{योधाश्चैव मनुष्येन्द्र सर्वैर्गुणगणैर्वृताः}


\twolineshloka
{अन्वेष्टव्याः सुपुरुषाः सहाया राज्यधारणे}
{न विमानयितव्यास्ते राज्ञा वृद्धिमभीप्सता}


\twolineshloka
{योधाः समरशौण्डीराः कृतज्ञाः शास्त्रकोविदाः}
{धर्मशास्त्रसमायुक्ताः पदातिजनसंवृताः}


\twolineshloka
{अर्थमानविवृद्धाश्च रथचर्याविशारदाः}
{इष्वस्त्रकुशला यस्य तस्येयं नृपतेर्मही}


\twolineshloka
{`ज्ञातीनामनवज्ञानं भृत्येष्वशठता तथा}
{नैपुणं चार्थचर्यासु यस्यैते तस्य सा मही}


\twolineshloka
{आलस्यं चैव निद्रा च व्यसनान्यतिहास्यता}
{यस्तैतानि न विद्यन्ते तस्यैव सुचिरं मही}


\twolineshloka
{वृद्धसेवी महोत्साहो वर्णानां चैव रक्षिता}
{धर्मचर्याः सदा यस्य तस्येयं सुचिरं मही}


\twolineshloka
{नीतिवर्त्मानुसरणं नित्यमुत्थानमेव च}
{रिपूणामनवज्ञानं तस्येयं सुचिरं मही}


% Check verse!
उत्थानं चैव दैवं च तयोर्नानात्वमेव च ॥मनुना वर्णितं पूर्वं वक्ष्ये शृणु तदेव हि
\twolineshloka
{उत्थानं हि नरेन्द्राणां बृहस्पतिरभाषत}
{नयानयविधानज्ञः सदा भव कुरूद्वह}


\twolineshloka
{दुर्हृदां छिद्रदर्शी यः सुहृदामुपकारवान्}
{विशेषविच्च भृत्यानां स राज्यफलमश्नुते ॥ '}


\twolineshloka
{सर्वसंग्रहणे युक्तो नृपो भवति यः सदा}
{उत्थानशीलो मन्त्राढ्यः स राजा राजसत्तमः}


\twolineshloka
{शक्या चाश्वसहस्रेण वीरारोहेण भारत}
{संगृहीतमनुष्येण कृत्स्ना जेतुं वसुंधरा}


\chapter{अध्यायः ११९}
\twolineshloka
{भीष्म उवाच}
{}


\twolineshloka
{एवं गुणयुतान्भृत्यान्स्वेस्वे स्थाने नराधिपः}
{नियोजयति कृत्येषु स राज्यफलमश्नुते}


\twolineshloka
{न वा स्वस्थानमुत्क्रम्य प्रमाणमपि सत्कृतम्}
{आरोप्य चापि स्वस्थानमुत्क्रम्यान्यत्प्रपद्यते}


\twolineshloka
{स्वजातिगुणसंपन्नाः स्वेषु धर्मेष्ववस्थिताः}
{प्रकर्तव्या ह्यमात्यास्तु नास्थाने प्रक्रिया क्षमा}


\twolineshloka
{अनुरूपाणि कर्माणि भृत्येभ्यो यः प्रयच्छति}
{स भृत्यगुणसंपन्नं राजा फलमुपाश्नुते}


\twolineshloka
{शरभः शरभस्थाने सिंहः सिंह इवोत्थितः}
{व्याघ्रो व्याघ्र इव स्थाप्यो द्वीपी द्वीपी यथा तथा}


\twolineshloka
{कर्मस्विहानुरूपेषु न्यस्या भृत्या यथाविधि}
{प्रतिलोमं न भृत्यास्ते स्थाप्याः कर्मफलैषिणा}


\twolineshloka
{यः प्रमाणमतिक्रम्य प्रतिलोमं नराधिपः}
{भृत्यान्स्थापयतेऽबुद्धिर्न स रञ्जयते प्रजाः}


\twolineshloka
{न बालिशा न च क्षुद्रा नाप्राज्ञा नाजितेन्द्रियाः}
{नाकुलीना जनाः पार्श्वे स्थाप्या राज्ञा गुणैषिणा}


\twolineshloka
{साधवः कुलजाः शूरा ज्ञानवन्तोऽनसूयकाः}
{अक्षुद्राः शुचयो दक्षाः स्युर्नराः पारिपार्श्वकाः}


\twolineshloka
{उद्भूतास्तत्पराः शान्ताश्चौक्षाः प्रकृतिजाः शुभाः}
{स्वेस्वे स्थानेऽनुपाकृष्टास्ते स्यू राज्ञो बहिश्चराः}


\twolineshloka
{सिंहस्य सततं पार्श्वे सिंह एव जनो भवेत्}
{असिंहः सिंहसहितः सिंहवल्लभते फलम्}


\twolineshloka
{यस्तु सिंहः श्वभिः कीर्णः सिंहकर्मफले रतः}
{न स सिंहफलं भोक्तुं शक्तः श्वभिरुपासितः}


\twolineshloka
{एव मेतैर्मनुष्येन्द्र शूरैः प्राज्ञैर्बहुश्रुतैः}
{कुलीनैः सह शक्येत कृत्स्ना जेतुं वसुंधरा}


\twolineshloka
{नावैद्यो नानृजुः पार्श्वे नाप्राज्ञो ना महायशाः}
{संग्राह्यो वसुधापालैर्भृत्यो भृत्यवतां वर}


\twolineshloka
{वाणवद्विसृता यान्ति स्वामिकार्यपरा नराः}
{ये भृत्याः पार्थिवहितास्तेषु सान्त्वं सदा चरेत्}


\twolineshloka
{कोशश्च सततं रक्ष्यो यत्नमास्थाय राजभिः}
{कोशमूला हि राजानः कोशवृद्धिकरो भवेत्}


\threelineshloka
{कोष्ठागारं च ते नित्यं स्फीतं धान्यैः सुसंचितैः}
{सदा त्वं सत्सु संन्यस्तधनधान्यपको भव}
{}


\twolineshloka
{नित्ययुक्ताश्च ते भृत्या भवन्तु रणकोविदाः}
{वाजिनां च प्रयोगेषु वैशारद्यमिहेष्यते}


\twolineshloka
{ज्ञातिबन्धुजनावेक्षी मित्रसंबन्धिसत्कृतः}
{पौरकार्यहितान्वेक्षई भव कौरवनन्दन}


\twolineshloka
{एषा ते नैष्ठिकी बुद्धिः प्रज्ञा चाभिहिता मषा}
{श्वाते निदर्शनं तात किं भूयः श्रोतुमिच्छसि}


\chapter{अध्यायः १२०}
\twolineshloka
{युधिष्ठिर उवाच}
{}


\twolineshloka
{राजवृत्तान्यनेकानि त्वया प्रोक्तानि भारत}
{पूर्वैः पूर्वनियुक्तानि राजधर्मार्थवेदिभिः}


\threelineshloka
{तदेव विस्तरेणोक्तं पूर्ववृत्तं सतां मतम्}
{प्रणेयं राजधर्माणां प्रब्रूहि भरतर्षभ ॥भीष्म उवाच}
{}


\twolineshloka
{रक्षणं सर्वभूतानामिति क्षात्रं परं मतम्}
{तद्यथा रक्षणं कुर्यात्तथा शृणु महीपते}


\twolineshloka
{यथा बर्हाणि चित्राणि बिभर्ति भुजगाशनः}
{तथा बहुविधं राजा रूपं कुर्वीत् धर्मवित्}


\twolineshloka
{तैक्ष्ण्यं जिह्नत्वमादानं सत्यमार्जवमेव च}
{मध्यस्थाः सत्वमातिष्ठंस्तथा वै सुखमृच्छति}


\twolineshloka
{यस्मिन्नर्थे यथैव स्यात्तदूर्णं रूपमादिशेत्}
{बहुरूपस्य राज्ञो हि सूक्ष्मोऽप्यर्थो न सीदति}


\twolineshloka
{नित्यं रक्षितमन्त्रः स्याद्यथा मूकः शरच्छिखी}
{श्लक्ष्णाक्षरगतः श्रीमान्भवेच्छास्त्रविशारदः}


\threelineshloka
{आयव्ययेषु युक्तः स्याज्जलस्रवणेष्विव}
{शैलाद्वर्षोदकानीव द्विजान्सिद्धान्समाश्रयेत्}
{आत्मार्थं हि सदा राजा कुर्याद्धर्मध्वजोत्तमम्}


\twolineshloka
{नित्यमुद्यतदण्डः स्यादाचारे चाप्रमादवान्}
{लोके चायव्ययौ दृष्ट्वा वृक्षाद्वृक्षमिवाव्रजेत्}


\twolineshloka
{आज्ञावान्स्यात्स्वयूथ्येषु भौमानि चरणैः किरन्}
{जतिपक्षः परिस्पन्देत्प्रेक्षेद्वैकल्यमात्मनः}


\twolineshloka
{दोषान्विवृणुयाच्छत्रोः परपक्षांश्च सूदयेत्}
{क ननेष्विव पुष्पाणि बहिरर्थान्समाचरेत्}


\twolineshloka
{उच्छ्रितानाश्रयेत्स्फूतान्नरेन्द्रानचलोपमान्}
{श्रयेच्छायामिव ज्ञातिं गुप्तं शरणमाश्रयेत्}


\twolineshloka
{प्रावृषीवासितग्रीवो माद्येत निशि निर्जने}
{मायूरेण गुणेनैव स्त्रीभिरारक्षितश्चरेत्}


\twolineshloka
{न जह्याच्च तनुत्राणं रक्षेदात्मानमात्मना}
{चारभूमिष्विव ततान्पाशांश्च परिवर्जयेत्}


\threelineshloka
{प्रणयेद्वाऽपि तां भूमिं प्रणश्येद्ग्रहणे पुनः}
{`एवं मयूरधर्मेण वर्तयन्सततं नरः}
{'हन्यात्क्रुद्धानतिविषांस्ताञ्जिह्मगतयोऽहिताः}


\threelineshloka
{नासूयेच्चावगर्ह्याणि सन्निवासान्निवासयेत्}
{सदा बर्हिसमं कामं प्रशस्तं कृतमाचरेत्}
{सर्वतश्चाददेत्प्रज्ञां पतङ्गं गहनेष्विव}


\twolineshloka
{एवं मयूरवद्राजा स्वराज्यं परिपालयेत्}
{आत्मबुद्धिकरीं नीतिं विदधीत विचक्षणः}


\twolineshloka
{आत्मसंयमनं बुद्ध्या परबुद्ध्या विचारणाम्}
{बुद्ध्या चात्मगुणप्राप्तिरेतच्छास्त्रनिदर्शनम्}


\twolineshloka
{परं विश्वासयेत्साम्ना स्वशक्तिं चोपलक्षयेत्}
{आत्मनः परिमर्शेन बुद्धिं बुद्ध्या विचारयेत्}


\twolineshloka
{सान्त्वयोगमतिः प्राज्ञः कार्याकार्यप्रयोजनकः}
{निगूढबुद्धेर्धीरस्य वक्तव्ये वक्ष्यते तथा}


\twolineshloka
{संनिकृष्टां कथां प्राज्ञो यदि बुद्ध्या बृहस्पतिः}
{स्वभावमेष्यते तप्तं कृष्णायसमिवोदके}


\twolineshloka
{अनुयुञ्जीत सत्यानि सर्वाण्येव महीपतिः}
{आगमैरुपदिष्टानि स्वस्य चैव परस्य च}


\twolineshloka
{मृदुं क्रूरं तथा प्राज्ञं शूरं चार्थविधानवित्}
{स्वकर्मणि नियुञ्जीत ये चान्ये वचनाधिकाः}


\twolineshloka
{अप्यदृष्टानि युक्तानि स्वानुरूपेषु कर्मसु}
{सर्वांस्ताननुवर्तेत स्वरांस्तन्त्रीरिवायताः}


\twolineshloka
{धर्माणामविरोधेन सर्वेषां प्रियमाचरेत्}
{ममायमिति राजा यः सपर्वत इवाचलः}


\twolineshloka
{व्यवहारं समाधाय सूर्यो रश्मीनिवायतान्}
{धर्ममेवाभिरक्षेत कृत्वा तुल्ये प्रियाप्रिये}


\twolineshloka
{कुलप्रकृतिदेशानां धर्मज्ञान्मृदुभाषिणः}
{मध्ये वयसि निर्दोषान्हिते युक्ताञ्जितक्लमान्}


\twolineshloka
{अलुब्धाञ्शिक्षितान्दान्तान्धर्मेषु परिनिष्ठितान्}
{स्थापयेत्सर्वकार्येषु राजा सर्वार्थरक्षिणः}


\twolineshloka
{एतेन च प्रकारेण कृत्यानामागतिं गतिम्}
{युक्त्या समनुतिष्ठेन तुष्टश्चारैः पुरस्कृतः}


\twolineshloka
{अमोघक्रोधहर्शस्य स्वयं कृत्याऽनुदर्शिनाः}
{आत्मप्रत्ययकोशस्य वसुदैव वसुंधरा}


\twolineshloka
{व्यक्तश्चानुग्रहो यस्य यथोक्तश्चापि निग्रहः}
{गुप्तात्मा गुप्तराष्ट्रस्य स राजा राजधर्मवित्}


\twolineshloka
{नित्यं राष्ट्रमवेक्षेत गोभिः सूर्य इवातपन्}
{चारांश्चानुचरान्विद्यात्तथा बुद्ध्या स्वयं चरेत्}


\twolineshloka
{कालप्राप्तमुपादद्यान्नार्थं राजा प्रसूचयेत्}
{अहन्यहनि संदुह्यान्महीं गामिव बुद्धिमान्}


\twolineshloka
{यथाक्रमेण पुष्पेभ्यश्चिनोति मधु षट््पदः}
{तथा द्रव्यमुपादाय राजा कुर्वीत संचयम्}


\twolineshloka
{यद्धि गुप्तावशिष्टं स्यात्तद्वित्तं धर्मकामयोः}
{संचयान्न विसर्गी स्याद्राजा शास्त्रविदात्मवान्}


\twolineshloka
{नार्थमल्पं परिभवेन्नावमन्येत शात्रवान्}
{बुद्ध्याऽनुबुद्ध्या चात्मानं न चाबुद्धेषु विश्वसेत्}


\twolineshloka
{धृतिर्दाक्ष्यं संयमो बुद्धिरात्माधैर्यं शौर्यं देशकालाप्रमादः}
{अल्पस्य वा महतो वा विवृद्धौधनस्यैतान्यष्ट समिंधनानि}


\twolineshloka
{अग्निस्तोको वर्धतेऽप्याज्यसिक्तोबीजं चैकं बहुसहस्रमेति}
{क्षयोदयौ विपुलौ सन्नियम्यौतस्मादल्पं नावमन्येत वित्तम्}


\threelineshloka
{बालोऽप्यबालः स्थविरो रिपुर्यः}
{सदा प्रमत्तं पुरुषं निहन्यात्}
{कालेनान्यस्तस्य मूलं हरेत्कालज्ञानं पार्थिवाना वरिष्ठम्}


\twolineshloka
{हरेत्कीर्ति धर्ममस्योपरुन्ध्यादर्थे विघ्नं वीर्यमस्योपहन्यात्}
{रिपुर्द्वेष्टा दुर्बलो वा बली वातस्माच्छत्रोर्नैव बिभ्येद्यथात्मा}


\twolineshloka
{क्षयं शत्रोः संचयं पालनं वाउभावर्थौ सहितौ धर्मकामौ}
{ततश्चान्यन्मतिमान्संदधीततस्माद्राजा बुद्धिमन्तं श्रयेत}


\twolineshloka
{बुद्धिर्दीप्ता बलवन्तं हिनस्तिबलं बुद्ध्या पाल्यते वर्धमानम्}
{शत्रुर्बुद्ध्या सीदते पीड्यमानोबुद्धिपूर्वं कर्म यत्तत्प्रशस्तम्}


\twolineshloka
{सर्वान्कामान्कामयानो हि धीरःसत्वेनाल्पेनाप्नुते हीनदोषः}
{यश्चात्मानं प्रार्थयतेऽर्थ्यमानैःश्रेयः पात्रं पूरयते च नाल्पम्}


\twolineshloka
{तस्माद्राजा प्रगृहीतः प्रजासुमूलं लक्ष्म्याः सर्वशो ह्याददीत}
{दीर्घं कालं ह्यपि संपीड्यमानोव्युष्यात्संपद्व्यवसायेन शक्त्या}


\twolineshloka
{विद्या तपो वा विपुलं धनं वासर्वं ह्येतद्व्यवसायेन शक्यम्}
{ब्रह्मायत्तं निवसति देहवत्सुतस्माद्विद्याद्व्यवसायं प्रभूतम्}


\twolineshloka
{यत्रासते मतिमन्तो मनस्विनःशक्रो विष्णुर्यत्र सरस्वती च}
{वसन्ति भूतानि च यत्र नित्यंतस्माद्विद्वान्नावमन्येत देहम्}


\twolineshloka
{लुब्धं हन्यात्संप्रदानाद्धि नित्यंलुब्धस्तृप्तिं परवित्तस्य नैति}
{सर्वो लुब्धः सर्वगुणोपभोगोयोऽर्थैर्हीनो धर्मकामौ जहाति}


\twolineshloka
{धनं भोगं पुत्रदारं समृद्धिंसर्वं लुब्धः प्रार्थयते परेषाम्}
{लुब्धे दोषाः संभवन्तीह सर्वेतस्माद्राजा न प्रगृह्णीत् लुब्धम्}


\twolineshloka
{संदर्शनेन पुरुषं जघन्यमपि चोदयेत्}
{आरम्भान्द्विषतां प्राज्ञः सर्वार्थांश्च प्रसूदयेत्}


\twolineshloka
{धर्मान्वितेषु विज्ञाता मन्त्रगुप्तिंश्च पाण्डव}
{आप्तो राजन्कुलीनश्च पर्याप्तो राष्ट्रसंग्रहे}


\twolineshloka
{विविप्रयुक्तान्नरदेवधर्मानुक्तान्समासेन निबोध बुद्ध्या}
{इमान्विदध्यादनुसृत्य यो वैराजा महीं पालयितुं स शक्तः}


\twolineshloka
{सुनीतिजं यस्य विधानजं सुखंधर्मप्रणीतं विधिवत्प्रसिद्ध्यति}
{न निन्द्यते तस्य गतिर्महीपतेस विन्दते राज्यसुखं ह्यनुत्तमम्}


\twolineshloka
{धनैर्विशिष्टान्मतिशीलपूजितान्गुणोपपन्नान्युधि दृष्टविक्रमान्}
{गुणेषु युक्तानचिरादिवात्मवांस्ततोऽभिसंधाय निहन्ति शात्रवान्}


\twolineshloka
{पश्येदुपायान्विविधेषु कर्मसुन चानुपायेन मतिं निवेशयेत्}
{श्रियं विशिष्टां विपुलं यशो धनंन दोषदर्शी पुरुषः समश्नुते}


\twolineshloka
{प्रीतिप्रवृत्तिं विनिवर्तनं चसुहृसु विज्ञाय विचार्य चोभयोः}
{यदेव मित्रं गुरुभारमावहेत्तदेव सुस्निग्धमुदाहरेद्बुधः}


\twolineshloka
{एतान्मयोक्तांश्चर राजधर्मान्नृणां च गुप्तौ मतिमादधत्स्व}
{अवाप्स्यसे पुण्यफलं सुखेनसर्वो हि लोको नृप धर्ममूलः}


\chapter{अध्यायः १२१}
\twolineshloka
{युधिष्ठिर उवाच}
{}


\twolineshloka
{अयं पितामहेनोक्तो राजधर्मः सनातनः}
{कीदृशश्च महादण्डो दण्डे सर्वं प्रतिष्ठितम्}


\twolineshloka
{देवतानामृषीणां च पितॄणां च महात्मनाम्}
{यक्षरक्षः पिशाचानां मर्त्यानां च विशेषतः}


\twolineshloka
{सर्वेषां प्राणिनां लोके तिर्यक्ष्वपि निवासिनाम्}
{सर्वस्यापि महातेजा दण्डः श्रेयानिति प्रभो}


\threelineshloka
{इत्येतदुक्तं भवता दण्डे वै सचराचरम्}
{पश्यतां लोक आयत्तं ससुरासुरमानुषम्}
{एतदिच्छाम्यहं श्रोतुं तत्त्वेन भरतर्षभ}


\twolineshloka
{को दण्डः कीदृशो दण्डः किंरूपः किंपरायणः}
{किमात्मकः कथंभूतः कतिमूर्तिः कथं प्रभुः}


\twolineshloka
{जागर्ति च कथं दण्डः प्रजासु विहितात्मकः}
{कश्च पूर्वापरमिदं जागर्ति प्रतिपालयन्}


\threelineshloka
{कश्च विज्ञायते पूर्वः को वरो दण्डसंज्ञितः}
{किंसंस्थश्चाभवद्दण्डः का चास्य गतिरिष्यते ॥भीष्म उवाच}
{}


\twolineshloka
{शृणु कौरव्य यो दण्डो व्यवहारो यथा च सः}
{यस्मिन्हि सर्वमायत्तं स धर्म इति केवलः}


\twolineshloka
{धर्मस्यार्थे महाराज व्यवहार इतीष्यते}
{तस्य लोपः कथं न स्याल्लोकेष्विह महात्मनः}


\twolineshloka
{इत्यर्थं व्यवाहरस्य व्यवहारत्वमिष्यते}
{अपि चैतत्पुरा राजन्मनुना प्रोक्तमादितः}


\twolineshloka
{सुप्रणीतेन दण्डेन प्रियाप्रियसमात्मना}
{प्रजा रक्षति यः सम्यग्धर्म एव स केवलः}


\twolineshloka
{यथोक्तमेतद्वचनं प्रागेव मनुना पुरा}
{यन्मयोक्तं वसिष्ठेन ब्रह्मणो वचनं महत्}


\twolineshloka
{प्रागिदं वचनं प्रोक्तमतः प्राग्वचनं विदुः}
{व्यवहारस्य चाख्यानाद्व्यवहार इहोच्यते}


\twolineshloka
{दण्डान्त्रिवर्गः सततं सुप्रणीतात्प्रवर्तते}
{दैवं हि परमो दण्डो रूपतोऽग्निरिवोत्थितः}


\twolineshloka
{नीलोत्पलदलश्यामश्चतुर्दंष्ट्रश्चतुर्भुजः}
{अष्टपान्नैकनयनः शङ्कुकर्णोर्ध्वरोमवान्}


\twolineshloka
{जटी द्विजिह्वस्ताम्रास्यो मृगराजतनुच्छदः}
{एतद्रूपं बिभर्त्युग्रं दण्डो नित्यं दुरासदः}


\twolineshloka
{असिर्धनुर्गदा शक्तिस्त्रिशूलं मुद्गरः शरः}
{मुसलं परशुश्चक्रं प्रासदण्डर्ष्टितोमराः}


\twolineshloka
{सर्वप्रहरणीयानि यानि यानीह कानिचित्}
{दण्ड एव स सर्वात्मा लोके चरति मूर्तिमान्}


\twolineshloka
{भिन्दंश्छिन्दन्रुजन्कृन्तन्दारयन्पाटयंस्तथा}
{घातयन्नभिधावंश्च दण्ड एव चरत्युत}


\twolineshloka
{असिर्विशसनो धर्मस्तीक्ष्णवर्मा दुरासदः}
{श्रीगर्भो विजयः शास्ता व्यवहारः प्रजागरः}


\twolineshloka
{शास्त्रं ब्राह्मणमन्त्राश्च शास्ता प्रवचनं परम्}
{धर्मपालोऽक्षरो गोपः सत्यगो नित्यगो गृहः}


\twolineshloka
{असङ्गो रुद्रतनयो मनुर्ज्येष्ठः शिवंकरः}
{नामान्येतानि दण्डस्य कीर्तितानि युधिष्ठिर}


\twolineshloka
{दण्डो हि भगवान्विष्णुर्यज्ञो नारायणः प्रभुः}
{शश्वद्रुपं महद्बिभ्रन्महात्पुरुष उच्यते}


\twolineshloka
{तथोक्ता ब्रह्मकन्येति लक्ष्मीर्नीतिः सरस्वती}
{दण्डनीतिर्जगद्धात्री दण्डो हि बहुविग्रहः}


\twolineshloka
{अर्थानर्थौ सुखं दुःखं धर्माधर्मौ बलाबले}
{दौर्भाग्यं भागधेयं च पुण्यापुण्ये गुणागुणौ}


\twolineshloka
{कामाकामावृतुर्मासः शर्वरी दिवसः क्षणः}
{अप्रमादः प्रमादश्च हर्षशोकौ शमो दमः}


\twolineshloka
{दैवं पुरुषकारश्च मोक्षामोक्षौ भयाभये}
{हिंसाहिंसे तपो यज्ञः संयमोऽथ विषामृते}


\twolineshloka
{अन्तश्चादिश्च मध्यं च कृतानां च प्रपञ्चनम्}
{मदः प्रमोदो दर्पश्च दम्भो धैर्यं नयानयौ}


\twolineshloka
{अशक्तिः शक्तिरित्येवं मानस्तम्भौ व्ययाव्ययौ}
{विनयश्च विसर्गश्च कालाकालौ च कौरव}


\twolineshloka
{अनृतं चाज्ञता सत्यं श्रद्धाश्रद्धे तथैव च}
{क्लीबता व्यवसायश्च लाभालाभौ जयाजयौ}


\twolineshloka
{तीक्ष्णता मृदुतात्युग्रमागमानागमौ तथा}
{विरोधश्चाविरोधश्च कार्याकार्ये बलाबले}


\twolineshloka
{असूया चानसूया च धर्माधर्मौ तथैव च}
{अपत्रपानपत्रपे ह्रीश्च संपद्विपच्च ह}


\twolineshloka
{तेजः कर्माणि पाण्डित्यं वाक्शक्तिर्बुद्धितत्वता}
{एवं दण्डस्य लोकेऽस्मिञ्जागर्ति बहुरूपता}


\twolineshloka
{न स्याद्यदीह दण्डो वै प्रमथेयुः परस्परम्}
{भयाद्दण्डस्य नान्योन्यं घ्नन्ति चैव युधिष्ठिर}


\twolineshloka
{दण्डेन रक्ष्यमाणा हि राजन्नहरहः प्रजाः}
{राजानं वर्धयन्तीह तस्माद्दण्डः परायणम्}


\twolineshloka
{व्यवस्थापयते नित्यमिमं लोकं नरेश्वर}
{सत्ये व्यवस्थितो धर्मो ब्राह्मणेष्ववतिष्ठते}


\twolineshloka
{धर्मयुक्ता द्विजश्रेष्ठा देवयुक्ता भवन्ति च}
{बभूव यज्ञो देवेभ्यो यज्ञः प्रीणाति देवताः}


\twolineshloka
{प्रीताश्च देवता लोकमिन्द्रे प्रतिददत्युत}
{अत्रं ददाति शक्रश्चाप्यनुगृह्णन्निमाः प्रजाः}


\twolineshloka
{प्राणाश्च सर्वभूतानां नित्यमन्ने प्रतिष्ठिताः}
{तस्मात्प्रजाः प्रतिष्ठन्ते दण्डो जागर्ति तासु च}


\twolineshloka
{एवंप्नयोजनश्चैव दण्डः क्षत्रियतां गतः}
{रक्षप्रजाः स जागर्ति नित्यं स्ववहितोक्षरः}


\twolineshloka
{ईश्वरः पुरुषः प्राणः सत्वं वृत्तं प्रजापतिः}
{भूतात्मा जीव इत्येवं नामभिः प्रोच्यतेऽष्टभिः}


\twolineshloka
{अददद्दण्डमेवास्मै धृतमैश्वर्यमेव च}
{बले नयश्च संयुक्तः सदा पञ्चविधात्मकः}


\twolineshloka
{कुलं बहुधनामात्याः प्रज्ञा प्रोक्ता बलानि तु}
{आहार्यमष्टकैर्द्रव्यैर्बलमन्यद्युधिष्ठिर}


\twolineshloka
{हस्तिनोऽश्वा रथाः पत्तिर्नावो विष्टिस्तथैव च}
{देशिकाश्चादिकाश्चैव तदष्टाङ्गं बलं स्मृतम्}


\twolineshloka
{अथ चाङ्गस्य युक्तस्य रथिनो हस्तियायिनः}
{अश्वारोहाः पदाताश्च मन्त्रिणो रसदाश्च ये}


\twolineshloka
{भिक्षुकाः प्राड्विवाकाश्च मौहूर्ता दैवचिन्तकाः}
{कोशो मित्राणि धान्यं च सर्वोपकरणानि च}


\twolineshloka
{सप्तप्रकृति चाष्टाङ्गं शरीरमिह तं विदुः}
{राज्यस्य दण्डमेवाङ्गं दण्डः प्रभव एव च}


\twolineshloka
{ईश्वरेण प्रसन्नेन कारणात्क्षत्रियस्य च}
{दण्डो दत्तः सदा गोप्ता दण्डो हीदं सनातनम्}


\twolineshloka
{राजा पूज्यतमो नान्यो यथा धर्मः प्रदर्शितः}
{ब्रह्मणा लोकरक्षार्थं स्वधर्मस्थापनाय च}


\twolineshloka
{भर्तृप्रत्यय उत्पन्नो व्यवहारस्तथाविधः}
{तस्याद्य सहितो दृष्टो भर्तृप्रत्ययलक्षणः}


\twolineshloka
{व्यवहारस्तु वेदात्मा वेदप्रत्यय उच्यते}
{मौलश्च नरशार्दूल शास्त्रोक्तश्च तथाऽपरः}


\twolineshloka
{उक्तो यश्चापि दण्डोऽसौ भर्तृप्रत्ययलक्षणः}
{ज्ञेयो नः स नरेन्द्रस्थो दण्डः प्रत्यय एव च}


\twolineshloka
{दण्डप्रत्ययदृष्टोऽपि व्यवहारात्मकः स्मृतः}
{व्यवहारः स्मृतो यश्च स वेदविषयात्मकः}


\twolineshloka
{यश्च वेदप्रसूतात्मा स धर्मो गुणदर्शनः}
{धर्मप्रत्यय उद्दिष्टो यश्च धर्मः कृतात्मभिः}


\twolineshloka
{व्यवहारः प्रजागोप्ता ब्रह्मदृष्टो युधिष्ठिर}
{त्रीन्धारयति लोकान्वै सत्यात्मा भूतिवर्धनः}


\twolineshloka
{यश्च दण्डः स दृष्टो नो व्यवहारः सनातनः}
{व्यवहारश्च दृष्टो यः स वेद इति नः श्रुतिः}


\twolineshloka
{यश्च वेदः स वै धर्मो यश्च धर्मः स सत्पथः}
{ब्रह्मा पितामहः पूर्वं भगवांश्च प्रजापतिः}


\twolineshloka
{लोकानां स हि सर्वेषां ससुरासुररक्षसाम्}
{स मनुष्योरगवतां कर्ता चैव स भूतकृत्}


\twolineshloka
{ततो नो व्यवहारोऽयं भर्तृप्रत्ययलक्षणः}
{तस्मादिदमवोचाम व्यवहारनिदर्शनम्}


\twolineshloka
{माता पिता च भ्राता च भार्या चैव पुरोहितः}
{नादण्ड्यो विद्यते राज्ञो यः स्वधर्मेण तिष्ठति}


\chapter{अध्यायः १२२}
\twolineshloka
{भीष्म उवाच}
{}


\twolineshloka
{अत्राप्युदाहरन्तीममितिहासं पुरातनम्}
{अङ्गेषु राजा द्युतिमान्वसुहोम इति श्रुतः}


\threelineshloka
{स राजा धर्मविन्नित्यं सह पत्न्या महातपाः}
{मुञ्जपृष्ठं जगामाथ देवर्षिगणसेवितम्}
{}


\twolineshloka
{तत्र शृङ्गे हिमवतो वसतिं समुपागमत्}
{यत्र मुञ्जवटे रामो जटाहरणमादिशत्}


\twolineshloka
{तदादि च महाप्राज्ञः ऋषिभिः संशितव्रतैः}
{मुञ्जपृष्ठ इति प्रोक्तः स देशो रुद्रसेवितः}


\twolineshloka
{स तत्र बहुभिर्युक्तस्तदा श्रुतिमयैर्गुणैः}
{ब्राह्मणानामनुमतो देवर्षिसदृशोऽभवत्}


\twolineshloka
{तं कदाचिददीनात्मा सखा शक्रस्य मानितः}
{अभ्यगच्छन्महीपालो मान्धाता शत्रुकर्शनः}


\twolineshloka
{सोपसृत्य तु मान्धाता वसुहोमं नराधिपम्}
{दृष्ट्वा प्रकृष्टतपसं विनयेनोपतिष्ठते}


\twolineshloka
{वसुहोमोऽपि राज्ञो वै गामर्ध्यं च न्यवेदयत्}
{सप्ताङ्गस्य तु राज्यस्य पप्रच्छ कुशलाव्ययौ}


\twolineshloka
{सद्भिराचरितं पूर्वं यथावदनुयायिनम्}
{अब्रवीद्वसुहोमस्तं राजन्किं करवाणि ते}


\twolineshloka
{सोऽब्रवीत्परमप्रीतो मान्धाता राजसत्तमः}
{वसुहोमं महाप्राज्ञमासीनं कुरुनन्दन}


\twolineshloka
{बृहस्पतेर्मतं राजन्नघीतं सकलं त्वया}
{तथैवौशनसं शास्त्रं विज्ञातं ते नरोत्तम}


\twolineshloka
{तदहं श्रोतुमिच्छामि दण्ड उत्पद्यते कथम्}
{किं वाऽस्य पूर्वं जागर्ति किं वा परममुच्यते}


\threelineshloka
{कथं क्षत्रियसंस्थश्च दण्डः संप्रत्यवस्थितः}
{ब्रूहि मे तद्यथातत्वं ददाम्याचार्यवेतनम् ॥वसुहोम उवाच}
{}


\twolineshloka
{शृणु राजन्यथा दण्डः संभूतो लोकसंग्रहः}
{प्रजाविनयरक्षार्थं धर्मस्यात्मा सनातनः}


\twolineshloka
{ब्रह्मा यियक्षुर्भगवान्सर्वलोकपितामहः}
{ऋत्विजं नात्मनस्तुल्यं ददर्शेति हि नः श्रुतम्}


\twolineshloka
{स गर्भं भगवान्देवो बहुवर्षाण्यधारयत्}
{अथ पूर्णे सहस्रे तु स गर्भः क्षुवतोऽपतत्}


\twolineshloka
{स क्षुपो नाम संभूतः प्रजापतिररिंदम्}
{ऋत्विगासीन्महाराज यज्ञे तस्य महात्मनः}


\twolineshloka
{तस्मिन्प्रवृत्ते सत्रे तु ब्रह्मणः पार्थिवर्षभ}
{इष्टरूपप्रचारत्वाद्दण्डः सोऽन्तर्हितोऽभवत्}


\twolineshloka
{तस्मिन्नन्तर्हिते चापि प्रजानां संकरोऽभवत्}
{नैव कार्यं न वा कार्यं भोज्याभोज्यं न विद्यते}


\twolineshloka
{पेयापेये कुतः सिद्धिर्हिसन्ति च परस्परम्}
{गम्यागम्यं तदा नासीत्स्वं परस्वं च वै समम्}


\twolineshloka
{परस्परं विलुम्पन्ति सारमेया यथाऽऽमिषम्}
{अबलान्बलिनो जघ्नुर्निर्मर्यादं प्रवर्तते}


\twolineshloka
{ततः पितामहो विष्णुं भगवन्तं सनातनम्}
{संपूज्य वरदं देवं महादेवमथाब्रवीत्}


\twolineshloka
{अत्र त्वमनुकम्पां वै कर्तुमर्हसि शङ्कर}
{अयं विष्णुः सखा तुभ्यं धर्मस्य परिरक्षणे}


\twolineshloka
{`त्वं हि सर्वविधानज्ञः सत्वानां त्वं गतिः परा}
{'संकरो न भवेदत्र यथा तद्वै विधीयताम्}


\twolineshloka
{ततः स भगवान्ध्यात्वा तदा शूलवरायुधः}
{` देवदेवो महादेवः कारणं जगतः परम्}


\threelineshloka
{ब्रह्मविष्ण्विन्द्रसहितः सर्वैश्च ससुरासुरैः}
{लोकसन्धारणार्थं च लोकसंकरनाशनम्}
{'आत्मानमात्माना दण्डं ससृजे देवसत्तमः}


\twolineshloka
{तस्माच्च धर्मचरणान्नीतिं देवीं सरस्वतीम्}
{असृजद्दण्डनीतिं वै त्रिषु लोकेषु विश्रुताम्}


\twolineshloka
{`यथाऽसौ नीयते दण्डः सततं पापकारिषु}
{दण्डस्य नयनात्सा हि दण्डनीतिरिहोच्यते ॥'}


\twolineshloka
{भूयः स भगवान्ध्यात्वा चिरं शूलघरः प्रभुः}
{`असृजत्सर्वशास्त्राणि महादेवो महेश्वरः}


\twolineshloka
{दण्डनीतेः प्रयोगार्थं प्रमाणानि च सर्वशः}
{विद्याश्चतस्रः कूटस्थास्तासां भेदविकल्पनाः}


\twolineshloka
{अङ्गानि वेदाश्चत्वारो मीमांसा न्यायविस्तरः}
{पुराणं धर्मशास्त्रं च विद्या ह्येताश्चतुर्दश}


\twolineshloka
{आयुर्वेदो धनुर्वेदो गान्धर्वश्चेति ते त्रयः}
{अर्थशास्त्रं चतुर्थं तु विद्या ह्यष्टादशैव तु}


\twolineshloka
{दश चाष्टौ च विख्याता एता धर्मस्य संहिताः}
{एतासामेव विद्यानां व्यासमाह महेश्वरः}


\twolineshloka
{शतानि त्रीणि शास्त्राणां महातन्त्राणि सप्ततिम्}
{व्यास एव तु विद्यानां महादेवेन कीर्तितः}


\twolineshloka
{तन्त्रं पाशुपतं नाम पञ्चरात्रं च विश्रुतम्}
{योगशास्त्रं च साङ्ख्यं च तन्त्रं लोकायतं तथा}


\twolineshloka
{तन्त्रं ब्रह्मतुला नाम तर्कविद्या दिवौकसाम्}
{सुखदुःखार्थजिज्ञासा कारणं चेति विश्रुतम्}


\twolineshloka
{तर्कविद्यास्तथा चाष्टौ स चोक्तो न्यायविस्तरः}
{दश चाष्टौ च विज्ञेयाः पौराणा यज्ञसंहिताः}


\twolineshloka
{पुराणाश्च प्रणीताश्च तावदेवेह संहिताः}
{धर्मशास्त्राणि तद्वच्च एकार्थानि च नान्यथा}


\twolineshloka
{एकार्थानि पुराणानि वेदाश्चैकार्यसंहिताः}
{नानार्थानि च सर्वाणि ततः शास्त्राणि शंकरः}


\twolineshloka
{प्रोवाच भगवान्देवः कालज्ञानानि यानि च}
{चतुःषष्टिप्रमाणानि आयुर्वेदं च सोत्तरम्}


\twolineshloka
{अष्टादशविकल्पां तां दण्डनीतिं च शाश्वतीम्}
{गान्धर्वमितिहासं च नानाविस्तरमुक्तवान्}


\twolineshloka
{इत्येताः शंकरप्रोक्ता विद्याः शब्दार्थसंहिताः}
{पुनर्भेदसहस्रं तु तासामेव तु विस्तरः}


\twolineshloka
{ऋषिभिर्देवगन्धर्वैः सविकल्पः सविस्तरः}
{शश्वदभ्यस्यते लोके वेद एव तु सर्वशः}


\twolineshloka
{वेदाश्चतस्रः संक्षिप्ता वेदवादाश्च ते स्मृताः}
{एतासां पारगो यश्च स चोक्तो वेदपारगः}


\twolineshloka
{वेदानां पारगो रुद्रो विष्णुरिन्द्रो बृहस्पतिः}
{शक्रः स्वायंभुवश्चैव मनुः परमधर्मवित्}


\threelineshloka
{ब्रह्मा च परमो देवः सदा सर्वैः सुरासुरैः}
{सर्वस्यानुग्रहाच्चैव व्यासो वै वेदपारगः ॥भीष्म उवाच}
{}


\threelineshloka
{अहं शान्तनवो भीष्मः प्रसादान्माधवस्य च}
{शंकरस्य प्रसादाच्च ब्रह्मणश्च कुरूद्वह}
{वेदपारग इत्युक्तो याज्ञवल्क्यश्च सर्वशः}


\twolineshloka
{कल्पेकल्पे महाभागैर्ऋषिभिस्तत्त्वदर्शिभिः}
{ऋषिपुत्रैर्ऋषिगणैर्भिद्यन्ते मिश्रकैरपि}


\twolineshloka
{शिवेन ब्रह्मणा चैव विष्णुना च विकल्पिताः}
{आदिकल्पे पुनश्चैव भिद्यन्ते साधुभिः पुनः}


\twolineshloka
{इदानीमपि विद्वद्भिर्भिद्यन्ते च विकल्पकैः}
{पूर्वजन्मानुसारेण बहुधेयं सरस्वती}


\twolineshloka
{भूयः स भगवान्ध्यात्वा चिरं शूलवरायुधः}
{'तस्यतस्य निकायस्य चकारैकैकमीश्वरम्}


\twolineshloka
{देवानामीश्वरं चक्रे दैवं दशशतेक्षणम्}
{यमं वैवस्वतं चापि पितॄणामकरोत्पतिम्}


\twolineshloka
{अपां राज्ये सुराणां च विदधे वरुणं प्रभुम्}
{धनानां राक्षसानां च कुवेरमपि चेश्वरम्}


\twolineshloka
{पर्वतानां पतिं मेरुं सरितां च महोदधिम्}
{मृत्युं प्राणेश्वरमथो तेजसां च हुताशनम्}


\twolineshloka
{रुद्राणामपि चेशानं गोप्तारं विदधे प्रभुः}
{महात्मानं महादेवं विशालाक्षं सनातनम्}


\twolineshloka
{`दश चैकश्च ये रुद्रास्तस्यैते मूर्तिसंभवाः}
{नानारूपधरो देवः स एव भगवाञ्शिवः ॥'}


\twolineshloka
{वसिष्ठमीशं विप्राणां वसूनां जातवेदसम्}
{तेजसां भास्करं चक्रे नक्षत्राणां निशाकरम्}


\twolineshloka
{वीरुधां वसुमन्तं च भूतानां च प्रभुं वरम्}
{कुमारं द्वादशभुजं स्कन्दं राजानमादिशत्}


\twolineshloka
{कालं सर्वेशमकरोत्संहारविनयात्मकम्}
{मृत्योश्चतुर्विभागस्य दुःखस्य च सुखस्य च}


\twolineshloka
{ईश्वरो देवदेवस्तु राजराजो नराधिपः}
{सर्वेषामेव रुद्राणां शूलपाणिरिति श्रुतिः}


\twolineshloka
{`ईश्वरश्चेतनः कर्ता पुरुषः कारणं शिवः}
{विष्णुर्ब्रह्मा शशी सूर्यः शक्रो देवाश्च सान्वयाः}


\twolineshloka
{सृजते ग्रसते चैतत्तमोभूतमिदं यथा}
{अप्रज्ञातं जगत्सर्वं यदा ह्येको महेश्वरः ॥ '}


\twolineshloka
{तमेनं ब्रह्मणः पुत्रमनुजातं क्षुपं ददौ}
{प्रजानामधिपं श्रेष्ठं सर्वधर्मभृतामपि}


\twolineshloka
{महादेवस्ततस्तस्मिन्वृत्ते यज्ञे समाहितः}
{दण़्डं धर्मस्य गोप्तारं विष्णवे सत्कृतं ददौ}


\twolineshloka
{विष्णुरङ्गिरसे प्रादादङ्गिरा मुनिसत्तमः}
{प्रादादिन्द्रमरीचिभ्या मरीचिर्भृगवे ददौ}


\twolineshloka
{भृगुर्ददावृषिभ्यस्तु दण्डं धर्मसमाहितम्}
{ऋषयो लोकपालेभ्यो लोकपालाः क्षुपाय च}


\twolineshloka
{क्षुपस्तु मनवे प्रादादादित्यतनयाय च}
{पुत्रेभ्यः श्राद्धदेवस्तु सूक्ष्मधर्मार्थकारणात्}


\twolineshloka
{विभज्य दण़्डः कर्तव्यो दण्डे तु नयमिच्छता}
{दुर्वाचा निग्रहो दण्डो हिरण्यं बाह्यतः क्रिया}


\twolineshloka
{व्यङ्गत्वं च शरीरस्य वधो वाऽनल्पकारणात्}
{शरीरपीडा कार्या तु स्वदेशाच्च विवासनम्}


\twolineshloka
{तं ददौ सूर्यपुत्राय मनवे रक्षणात्मकम्}
{आनुपूर्व्याच्च दण्डोऽयं प्रजा जागर्ति पालयन्}


\twolineshloka
{इन्द्रो जागर्ति भगवानिन्द्रादग्निर्विभावसुः}
{अग्नेर्जागर्ति वरुणो वरुणाच्च प्रजापतिः}


\twolineshloka
{प्रजापतेस्ततो धर्मो जागर्ति विनयात्मकः}
{धर्माच्च ब्रह्मणः पुत्रो व्यवसायः सनातनः}


\twolineshloka
{व्यवसायात्ततस्तेजो जागर्ति परिपालयत्}
{ओषध्यस्तेजसस्तस्मादोषधीभ्यश्च पर्वताः}


\twolineshloka
{पर्वतेभ्यश्च जागर्ति रसो रसगुणात्तथा}
{जागर्ति निर्ऋतिर्देवी ज्योतींषि निर्ऋतीमनु}


\twolineshloka
{वेदाः प्रतिष्ठा ज्योतिर्भ्यस्ततो हयशिराः प्रभुः}
{ब्रह्मा पितामहस्तस्माज्जागर्ति प्रभुरव्ययः}


\twolineshloka
{पितामहान्महादेवो जागर्ति भगवाञ्शिवः}
{विश्वेदेवाः शिवाच्चापि विश्वेभ्य ऋषयस्तथा}


\twolineshloka
{ऋषिभ्यो भगवान्सोमः सोमाद्देवाः सनातनाः}
{देवेभ्यो ब्राह्मणा लोके जाग्रतीत्युपधारय}


\twolineshloka
{ब्राह्मणेभ्यश्च राजन्या लोकान्रक्षन्ति धर्मताः}
{स्थावरं जङ्गमं चैव क्षत्रियेभ्यः सनातनम्}


\twolineshloka
{प्रजा जाग्रति लोकेऽस्मिन्दण्डो जागर्ति तासु च}
{सर्वसंक्षेपको दण़्डः पितामहसुतः प्रभुः}


\twolineshloka
{जागर्ति कालः पूर्वं च मध्ये चान्ते च भारत}
{ईशः सर्वस्य कालो हि महादेवः प्रजापतिः}


\twolineshloka
{देवदेवः शिवः सर्वो जागर्ति सततं प्रभुः}
{कपर्दी शंकरो रुद्रो भवः स्थाणुरुमापतिः}


\threelineshloka
{इत्येष दण्डो व्याख्यातस्तथौषध्यस्तथापरे}
{भूमिपालो यथान्यायं वर्तेतानेन धर्मवित् ॥भीष्म उवाच}
{}


\twolineshloka
{इतीदं वसुहोमस्य योऽऽत्मवाञ्शृणुयान्मतम्}
{श्रुत्वा सम्यक्प्रवर्तेत स लोकानाप्नुयान्नृपः}


\twolineshloka
{इति ते सर्वमाख्यातं यो दण्डो मनुजर्षभ}
{नियन्ता सर्वलोकस्य धर्माक्रान्तस्य भारत}


\twolineshloka
{`वसुहोमाच्छ्रुतं राज्ञा मान्धात्रा भूभृता पुरा}
{मयापि कथितं राजन्नाख्यानं प्रथितं मया ॥'}


\chapter{अध्यायः १२३}
\twolineshloka
{युधिष्ठिर उवाच}
{}


\twolineshloka
{तात धर्मार्थकामानां श्रोतुमिच्छामि निश्चयम्}
{लोकयात्रा हि कार्त्स्न्येन त्रिष्वेतेषु प्रतिष्ठिता}


\threelineshloka
{धर्मार्थकामाः किंमूलाः प्रभवः प्रलयश्च कः}
{अन्योन्यं चानुषज्जन्ते वर्तन्ते च पृथक्पृथक् ॥भीष्म उवाच}
{}


\twolineshloka
{य एते स्युः सुमनसो लोकसंस्थार्थनिश्चये}
{कामप्रभवसंस्थासु सज्जन्ते च त्रयस्तदा}


\twolineshloka
{धर्ममूलोऽर्थ इत्युक्तः कामोऽर्थफलमुच्यते}
{संकल्पमूलास्ते सर्वे संकल्पो विषयात्मकः}


\twolineshloka
{विषयाश्चैव कार्त्स्न्येन सर्व आहारसिद्धये}
{मूलमेतन्त्रिवर्गस्य निवृत्तिर्मोक्ष उच्यते}


\twolineshloka
{धर्मः शरीरसंगुप्तिर्धर्मार्थश्चार्थ इष्यते}
{कामो रतिफलश्चात्र सर्वे रतिफलाः स्मृताः}


\twolineshloka
{सन्निकृष्टांश्चरेदेतान्न चैतान्मनसा त्यजेत्}
{विमुक्तस्तपसा सर्वान्धर्मादीन्कामनैष्ठिकान्}


\twolineshloka
{श्रेष्ठबुद्धिस्त्रिवर्गस्य उदयं प्राप्नुयात्क्षणात्}
{[कर्मणा बुद्धिपूर्वेण भवत्यर्थो न वा पुनः ॥]}


\threelineshloka
{अर्थार्थमन्यद्भवति विपरीतमथापरम्}
{अनर्थार्थमवाप्यार्थमन्यत्राद्योपकारकम्}
{]बुद्ध्या बुद्ध इहार्थेन तदह्ना तु निकृष्टया}


\twolineshloka
{अपध्यानमलो धर्मो मलोऽर्थस्य निगूहनम्}
{संप्रमोहमलः कामो भूयस्तद्गुणवर्धितः}


\twolineshloka
{अत्राप्युदाहरन्तीममितिहासं पुरातनम्}
{अरिष्टस्य च संवादं कामन्दस्य च भारत}


\twolineshloka
{कामन्दमृषिमासीनमभिवाद्य नराधिपः}
{आङ्गोरिष्ठोऽथ पप्रच्छ कृत्वा समयमव्ययम्}


\twolineshloka
{यः पापं कुरुते राजा काममोहबलात्कृतः}
{प्रत्यासन्नस्य तस्यर्षे किं स्यात्पापप्रणाशनम्}


\threelineshloka
{अधर्मं धर्म इति च यो मोहादाचरेन्नरः}
{तं चापि प्रथितं लोके कथं राजा निवर्तयेत् ॥कामन्द उवाच}
{}


\twolineshloka
{यो धर्मार्थौ परित्यज्य काममेवानुवर्तते}
{स धर्मार्थपरित्यागात्प्रज्ञानाशमिहार्च्छति}


\twolineshloka
{प्रज्ञानाशात्मको मोहस्तथा धर्मार्थनाशकः}
{तस्मान्नास्तिकता चैव दुराचारश्च जायते}


\twolineshloka
{दुराचारान्यदा राजा प्रदुष्टान्न नियच्छति}
{तस्मादुद्विजते लोकः सर्पाद्वेश्मगतादिव}


\twolineshloka
{तं प्रजा नानुरज्यन्ते न विप्रा न च साधवः}
{ततः संक्षयमाप्नोति तथा वध्यत्वमेव च}


\twolineshloka
{अपध्वस्तस्त्ववमतो दुःखं जीवति जीवितम्}
{जीवते यदपध्वस्तः शुद्धं मरणमेव तत्}


\twolineshloka
{अत्रैतदाहुराचार्याः पापस्य परिवर्तनम्}
{सेवितव्या त्रयी विद्या सत्कारो ब्राह्मणेषु च}


\twolineshloka
{महामना भवेद्धर्मे विवहेच्च महाकुले}
{ब्राह्मणांश्चापि सेवेत क्षमायुक्तान्मनस्विनः}


\twolineshloka
{जपेदुदकशीलः स्यात्सुमुखो न च नास्तिकः}
{धर्मान्वितान्संप्रविशेद्बहिः प्लुत्यैव दुष्कृतम्}


\twolineshloka
{प्रसादयेन्मधुरया वाचा वाऽप्यथ कर्मणा}
{इत्यस्तीति वदेन्नित्यं परेषां कीर्तयन्गुणान्}


\twolineshloka
{अपापो ह्येवमाचारः क्षिप्रं बहुमतो भवेत्}
{पापान्यपि हि कृच्छ्राणि शमयेन्नात्र संशयः}


\twolineshloka
{गुरवो हि परं धर्मं यं ब्रूयुस्तं तथा कुरु}
{गुरूणां हि प्रसादाद्वै श्रेयः परमवाप्स्यसि}


\chapter{अध्यायः १२४}
\twolineshloka
{युधिष्ठिर उवाच}
{}


\twolineshloka
{इमे जना मनुष्येन्द्र प्रशंसन्ति सदा भुवि}
{धर्मस्य शीलमेवादौ ततो मे संशयो महान्}


\twolineshloka
{यदि तच्छक्यमस्माभिर्ज्ञातुं धर्मभूतां वर}
{श्रोतुमिच्छामि तत्सर्वं यथैतदुपलभ्यते}


\threelineshloka
{कथं तत्प्राप्यते शीलं श्रोतुमिच्छामि भारत}
{किंलक्षणं च तत्प्रोक्तं ब्रूहि मे वदतां वर ॥भीष्म उवाच}
{}


\twolineshloka
{पुरा दुर्योधनेनेह धृतराष्ट्राय मानद}
{आख्यातं तप्यमानेन श्रियं दृष्ट्वा तवागताम्}


\twolineshloka
{इन्द्रप्रस्थे महाराज तव सभ्रातृकस्य ह}
{सभायां चापहसनं तत्सर्वं शृणु भारत}


\twolineshloka
{भवतस्तां सभां दृष्ट्वा समृद्धिं चाप्यनुत्तमाम्}
{दुर्योधनस्तदा दीनः सर्वं पित्रे न्यवेदयत्}


\threelineshloka
{श्रुत्वा हि धृतराष्ट्रश्च दुर्योधनवचस्तदा}
{अब्रवीत्कर्णसहितं दुर्योधनमिदं वचः ॥धृतराष्ट्र उवाच}
{}


\twolineshloka
{किमर्थं तप्यसे पुत्र श्रोतुमिच्छामि तत्त्वतः}
{श्रुत्वा त्वामनुनेष्यामि यदि सम्यग्भविष्यति}


\twolineshloka
{यदा त्वां महदैश्वर्यं प्राप्तं परपुरंजय}
{किंकरा भ्रातरः सर्वे मित्रसंबन्धिबान्धवाः}


\threelineshloka
{आच्छादयसि प्रावारानश्नासि पिशितौदनम्}
{आजानेया वहन्ति त्वां कस्माच्छेचसि पुत्रक ॥दुर्योधन उवाच}
{}


\twolineshloka
{दश तात सहस्राणि स्नातकानां महात्मनाम्}
{भुञ्जते रुक्मपात्रीभिर्युधिष्ठिरनिवेशने}


\twolineshloka
{दृष्ट्वा च तां सभां दिव्यपुष्पफलान्विताम्}
{अश्वांस्तित्तिरकल्माषान्रत्नानि विविधानि च}


\threelineshloka
{दृष्ट्वा तां पाण्डवेयानामृद्धिमिन्द्रोपमां शुभाम्}
{अमित्राणां सुमहतीमनुशोचामि मानद ॥धृतराष्ट्र उवाच}
{}


\twolineshloka
{यदीच्छसि श्रियं तात यादृशी सा युधिष्ठिरे}
{विशिष्टां वा नरश्रेष्ठ शीलवान्भव पुत्रक}


\twolineshloka
{शीलेन हि त्रयो लोकाः शक्या जेतुं न संशयः}
{न हि किंचिदसाध्यंवै लोके शीलवतां सताम्}


\twolineshloka
{एकरात्रेण मान्धाता त्र्यहेण जनमेजयः}
{सप्तरात्रेण नाभागः पृथिवीं प्रतिपेदिवान्}


\threelineshloka
{एते हि पार्थिवाः सर्वे शीलवन्तो यशोन्विताः}
{ततस्तेषां गुणक्रीता वसुधा स्वयमागता ॥दुर्योधन उवाच}
{}


\threelineshloka
{कथं तत्प्राप्यते शीलं श्रोतुमिच्छामि भारत}
{येन शीलेन संप्राप्ताः क्षिप्रमेव वसुंधराम् ॥धृतराष्ट्र उवाच}
{}


\twolineshloka
{अत्राप्युदाहरन्तीममितिहासं पुरातनम्}
{नारदेन पुरा वृत्तं शीलमाश्रित्य भारत}


\twolineshloka
{प्रा--देन हृतं राज्यं महेन्द्रस्य महात्मनः}
{श---माश्रित्य दैत्येन त्रैलोक्यं च वशे कृतम्}


\twolineshloka
{त--बृहस्पतिं शक्रः प्राञ्जलिः समुपस्थितः}
{तमुवाच महाप्राज्ञः श्रेय इच्छामि वेदितुम्}


\twolineshloka
{ततो बृहस्पतिस्तस्मै ज्ञानं नैश्रेयसं परम्}
{कथयामास भगवान्देवेन्द्राय कुरूद्वह}


\threelineshloka
{एतावच्छ्रेय इत्येव बृहस्पतिरभाषत}
{इन्द्रस्तु भूयः पप्रच्छ को विशेषो भवेदिति ॥बृहस्पतिरुवाच}
{}


\twolineshloka
{विशेषोऽस्ति महांस्तात भार्गवस्य महात्मनः}
{तत्रागमय भद्रं ते भूय एव सुरोत्तम}


\twolineshloka
{आत्मनस्तु ततः श्रेयो भार्गवः सुमहायशाः}
{ज्ञानमागमयत्प्रीत्या पुनः स परमद्युतिः}


\twolineshloka
{तेनापि समनुज्ञातो भार्गवेण महात्मना}
{श्रेयोऽस्तीति परं भूयः शुक्रमाह शतक्रतुः}


\twolineshloka
{भार्गवस्त्वाह सर्वज्ञः प्रह्लादस्य महात्मनः}
{ज्ञानमस्ति विशेषेणेत्युक्तो हृष्टश्च सोऽभवत्}


\twolineshloka
{स तत्र ब्राह्मणो भूत्वा प्रह्लादं पाकशासनः}
{स्तुत्वा प्रोवाच मेधावी श्रेय इच्छामि वेदितुम्}


\twolineshloka
{प्रह्लादस्त्वब्रवीद्विप्रं क्षणो नास्ति द्विजोत्तम}
{त्रैलोक्यराज्यसक्तस्य ततो नोपदिशामि ते}


\twolineshloka
{ब्राह्मणस्त्वब्रवीद्राजन्यस्मिन्काले क्षणो भवेत्}
{तदोपादेष्टुमिच्छामि यदि कार्यान्तरं भवेत्}


\twolineshloka
{ततः प्रीतोऽभवद्राजा प्रह्वादो ब्रह्मवादिनः}
{तथेत्युक्त्वा ददौ काले ज्ञानतत्त्वं द्विजे तदा}


\twolineshloka
{ब्राह्मणोऽपि यथान्यायं गुरुवृत्तिमनुत्तमाम्}
{चकार सर्वभावेन यद्यच्च मनसेच्छति}


\fourlineindentedshloka
{पृष्टश्च तेन बहुशः प्राप्तं कथमरिंदम्}
{त्रैलोक्यराज्यं धर्मज्ञ कारणं तद्ब्रवीहि मे}
{[प्रह्लादोऽपि महाराज ब्राह्मणं वाक्यमब्रवीत् ॥] प्रह्लाद उवाच}
{}


\twolineshloka
{नासूयामि द्विजान्विप्र राजास्मीति कथंचन}
{काम्यानि वदतां तेषां संयच्छामि वहामि च}


\twolineshloka
{ते विस्रब्धाः प्रभाषन्ते संयच्छन्ति च मां सदा}
{तेषां कार्यपथे युक्तं शुश्रूषुमनहंकृतम्}


\twolineshloka
{धर्मात्मानं जितक्रोधं नियतं संयतेन्द्रियम्}
{समासिञ्चन्ति शास्त्रज्ञाः क्षौद्रं मध्विव मक्षिकाः}


\twolineshloka
{सोऽहं वागग्रविद्यानां रसानामवलेहिता}
{स्वजात्यानधितिष्ठामि नक्षत्राणीव चन्द्रमाः}


\twolineshloka
{एतत्पृथिव्याममृतमेतच्चक्षुरनुत्तमम्}
{यद्ब्राह्मणमुखे हव्यमेतच्छ्रुत्वा प्रवर्तते}


\twolineshloka
{एतावच्छेय इत्याह प्रह्लादो ब्रह्मवादिनम्}
{शुश्रूषितस्तेन तदा दैत्येन्द्रो वाक्यमब्रवीत्}


\twolineshloka
{यथावद्गुरुवृत्त्या ते प्रीतोऽस्मि द्विजसत्तम}
{वरं वृणीष्व भद्रं ते प्रदाताऽस्मि न संशयः}


\threelineshloka
{कृतमित्येव दैत्येन्द्रमुवाच द्विजसत्तमः}
{प्रह्लादस्त्वब्रवीत्प्रीतो गृह्यतां वर इत्युत ॥ब्राह्मण उवाच}
{}


\twolineshloka
{यदि राजन्प्रसन्नस्त्वं मम चेदिच्छसि प्रियम्}
{भवतः शीलमिच्छामि प्राप्नुमेष वरो मम}


\twolineshloka
{ततः प्रीतस्तु दैत्येन्द्रो भयमस्याभवन्महत्}
{वरे प्रदिष्टे विप्रेण नाल्पचेतायमित्युत}


\twolineshloka
{एवमस्त्विति स प्राह प्रह्लादो विस्मितस्तदा}
{उपाकृत्य तु विप्राय वरं दुःखान्वितोऽभवत्}


\twolineshloka
{दत्ते वरे गते विप्रे चिन्ताऽसीन्महती तदा}
{प्रह्लादस्य महाराज निश्चयं न च जग्मिवान्}


\twolineshloka
{तस्य चिन्तयतस्तावच्छायाभूतं महाद्युतेः}
{तेजोविग्रहवत्तात शरीरमजहात्तदा}


\twolineshloka
{तमपृच्छन्महाराजः प्रह्लादः को भवानिति}
{प्रत्याह तं तु शीलोस्मि त्यक्तो गच्छाम्यहं त्वया}


\threelineshloka
{तस्मिन्द्विजोत्तमे राजन्वत्स्याम्यहमरिंदम}
{योऽसौ शिष्यत्वमागम्य त्वयि नित्यं समाहितः}
{इत्युक्त्वाऽन्तर्हितं तद्वै शक्रं चान्वाविशत्प्रभो}


\twolineshloka
{तस्मिंस्तेजसि याते तु तादृग्रूपस्ततोपरः}
{शरीरान्निः सृतस्तस्य को भवानिति सोब्रवीत्}


\twolineshloka
{धर्मं प्रह्लाद मां विद्धि यत्रासौ द्विजसत्तमः}
{तत्र यास्यामि दैत्येन्द्र यतः शीलं ततो ह्यहम्}


\twolineshloka
{ततोऽपरो महाराज प्रज्वलन्निव तेजसा}
{शरीरान्निः सृतस्तस्य प्रह्लादस्य महात्मनः}


\twolineshloka
{को भवानिति पृष्टश्च तमाह स महाद्युतिः}
{सत्यं विद्ध्यसुरेन्द्राद्य प्रयास्ये धर्ममन्वहम्}


\twolineshloka
{तस्मिन्ननुगते धर्मं पुरुषे पुरुषोऽपरः}
{निश्चक्राम ततस्तस्मात्पृष्टश्चाह महातपाः}


\twolineshloka
{वृत्तं प्रह्लाद मां विद्धि यतः सत्यं ततो ह्यहम्}
{तस्मिन्गते महाश्वेता शरीरात्तस्य निर्ययौ}


\twolineshloka
{पृष्टश्चाह बलं विद्धि यतो वृत्तमहं ततः}
{इत्युक्त्वा प्रययौ तत्र यतो वृत्तं नराधिप}


\twolineshloka
{ततः प्रभामयी देवी शरीरात्तस्य निर्ययौ}
{तामपृच्छत्स दैत्येन्द्रः सा श्रीरित्येनमब्रवीत्}


\twolineshloka
{उषिताऽस्मि सुखं नित्यं त्वयि सत्यपराक्रम}
{त्वया युक्ता गमिष्यामि बलं ह्यनुगता ह्यहम्}


\twolineshloka
{ततो भयं प्रादुरासीत्प्रह्लादस्य महात्मनः}
{अपृच्छत च तां भूयः क्व यासि कमलालये}


\threelineshloka
{त्वं हि सत्यव्रता देवी लोकस्य परमेश्वरी}
{कश्चासौ ब्राह्मणश्रेष्ठस्तत्त्वमिच्छामि वेदितुम् ॥श्रीरुवाच}
{}


\twolineshloka
{स शक्तो ब्रह्मचारी यस्त्वत्तश्चैवोपशिक्षितः}
{त्रैलोक्ये ते यदश्वर्यं तत्तेनापहृतं प्रभो}


\twolineshloka
{शीलेन हि त्रयो लोकास्त्वया धर्मज्ञ निर्जिताः}
{तद्विज्ञाय सुरेन्द्रेण तव शीलं हृतं प्रभो}


\threelineshloka
{धर्मः सत्यं तथा वृत्तं बलं चैव तथाऽप्यहम्}
{शीलमूला महाप्राज्ञ सदा नास्त्यत्र संशयः ॥भीष्म उवाच}
{}


\twolineshloka
{एवमुक्त्वा गता श्रीस्तु ते च सर्वे युधिष्ठिर}
{दुर्योधनस्तु पितरं भूय एवाब्रवीत्तदा}


\threelineshloka
{शीलस्य तत्त्वमिच्छामि वेत्तुं कौरवनन्दन}
{प्राप्यते च यथा शीलं तं चोपायं ब्रवीहि मे ॥धृतराष्ट्र उवाच}
{}


\twolineshloka
{सोपायं पूर्वमुद्दिष्टं प्रह्लादेन महात्मना}
{संक्षेपतस्तु शीलस्य शृणु प्राप्तिं नरेश्वर}


\twolineshloka
{अद्रोहः सर्वभूतेषु कर्मणा मनसा गिरा}
{अनुग्रहश्च दानं च शीलमेतत्प्रशस्यते}


\twolineshloka
{यदन्येषां हितं न स्यादात्मनः कर्म पौरुषम्}
{अपत्रपेत वा येन न तत्कुर्यात्कथंचन}


\twolineshloka
{तत्तु कर्म तथा कुर्याद्येन श्लाध्येत संसदि}
{शीलं समासेनैतत्ते कथितं कुरुसत्तम}


\threelineshloka
{यद्यप्यशीला नृपते प्राप्नुवन्ति श्रियं क्वचित्}
{न भुञ्जते चिरं तात समूलाश्च पतन्ति ते ॥धृतराष्ट्र उवाच}
{}


\fourlineindentedshloka
{एतद्विदित्वा तत्त्वेन शीलवान्भव पुत्रक}
{यदीच्छसि श्रियं तात सुविशिष्टां युधिष्ठिरात्}
{`अधिकां चापि राजेन्द्र ततस्त्वं शीलवान्भवा ॥' भीष्म उवाच}
{}


\twolineshloka
{एतत्कथितवान्पुत्रे धृतराष्ट्रो महीपतिः}
{एतत्कुरुष्व कौन्तेय ततः प्राप्स्यसि तत्फलम्}


\chapter{अध्यायः १२५}
\twolineshloka
{युधिष्ठिर उवाच}
{}


\twolineshloka
{शीलं प्रधानं पुरुषे कथितं ते पितामह}
{कथमाशा समुत्पन्ना का च सा तद्वदस्व मे}


\twolineshloka
{संशयो मे महानेष समुत्पन्नः पितामह}
{छेत्ता च तस्य नान्योऽस्ति त्वत्तः परपुरंजय}


\twolineshloka
{पितामहाशा महती ममासीद्धि सुयोधने}
{प्राप्ते युद्धे तु तद्युक्तं तत्कर्ताऽयमिति प्रभो}


\twolineshloka
{सर्वस्याशा सुमहती पुरुषस्योपजायते}
{स्यां विहन्यमानायां दुःखो मृत्युर्न संशयः}


\twolineshloka
{ऽहं हताशो दुर्बुद्धिः कृतस्तेन दुरात्मना}
{र्तराष्ट्रेण राजेन्द्र पश्य मन्दात्मतां मम}


\twolineshloka
{आशां बृहत्तरीं मन्ये पर्वतादपि सद्रुमात्}
{आकाशादपि वा राजन्नप्रमेयाऽथवा पुनः}


\threelineshloka
{एषा चैव कुरुश्रेष्ठ दुर्विचिन्त्या सुदुर्लभा}
{दुर्लभत्वाच्च पश्यामि किमन्यद्दुर्लभं ततः ॥भीष्म उवाच}
{}


\twolineshloka
{अत्र ते वर्तयिष्यामि युधिष्ठिर निबोध मे}
{इतिहासं सुमित्रस्य निर्वृत्तमृषभस्य च}


\twolineshloka
{सुमित्रो नाम राजर्षिर्हैहयो मृगयां गतः}
{ससार च मृगं विद्ध्वा बाणेनानतपर्वणा}


\twolineshloka
{स मृगो बाणमादाय ययावतिपराक्रमः}
{स च राजा बली तूर्णं ससार मृगमन्तिकात्}


\twolineshloka
{ततो निम्नं स्थलं चैव समृगोऽद्रवदाशुगः}
{मुहूर्तमिव राजेन्द्र समेन स पथाऽगमत्}


\twolineshloka
{ततः स राजा तारुण्यादौरसन बलेन च}
{चचार बाणासनभृत्सखङ्गो हंसवत्तदा}


\twolineshloka
{ततो नदान्नदीश्चैव पल्वलानि वनानि च}
{अतिक्रम्याभ्यतिक्रम्य ससारैको वनेचरः}


\twolineshloka
{स तु तावन्मृगो राजन्नासाद्यासाद्य पार्थिवम्}
{पुनरभ्येति जवनो जवेन महता ततः}


\twolineshloka
{स तस्य बाणैर्बहुभिः समभ्यस्तो वनेचरः}
{प्रक्रीडन्निव राजेन्द्र पुनरभ्येति चान्तिकम्}


\twolineshloka
{पुनश्च जवमास्थाय जवनो मृगयूथपः}
{[अतीत्यातीत्य राजेन्द्र पुनरभ्येति चान्तिकम् ॥]}


\twolineshloka
{तस्य मर्मच्छिदं घोरं सुमित्रोऽमित्रकर्शनः}
{समादाय शरं श्रेष्ठं कार्मुकान्निरवासयत्}


\twolineshloka
{[ततो गव्यूतिमात्रेण मृगयूथपयूथपः}
{]तस्य बाणपथं मुक्त्वा तस्थिवान्प्रहसन्निव}


\twolineshloka
{तस्मिन्निपतिते बाणे भूमौ ज्वलिततेजसि}
{प्रविवेश मृगोऽरण्यं मृगं राजाऽप्यभिद्रवत्}


\chapter{अध्यायः १२६}
\twolineshloka
{भीष्म उवाच}
{}


\twolineshloka
{प्रविश्य च महारण्यं तापसानामथाश्रमम्}
{आससाद ततो राजा श्रान्तश्चोपाविशत्तदा}


\twolineshloka
{तं कार्मुकधरं दृष्ट्वा श्रमार्तं क्षुधितं तदा}
{समेत्य ऋषयस्तस्मै पूजां चक्रुर्यथाविधि}


\twolineshloka
{स पूजामृषिभिर्दत्तां प्रतिगृह्य नराधिपः}
{अपृच्छत्तापसान्सर्वांस्तपोवृद्धिमनुत्तमाम्}


\twolineshloka
{ते तस्य राज्ञो वचनं प्रतिगृह्य तपोधनाः}
{ऋषयो राजशार्दूलमपृच्छंस्तत्प्रयोजनम्}


\twolineshloka
{केन भद्रमुखार्थेन तपोवनमुपागतः}
{पदातिर्बद्धनिस्त्रिंशो धन्वी बाणी नरेश्वर}


\twolineshloka
{एतदिच्छामहे श्रोतुं कुतः प्राप्तोऽसि मानद}
{कस्मिन्कुले तु जातस्त्वं किंनामा चासि ब्रूहि नः}


\twolineshloka
{ततः स राजा सर्वेभ्यो द्विजेभ्यः पुरुषर्षभ}
{आचख्यौ तद्यथावृत्तं परिचर्यां च भारत}


\twolineshloka
{हैहयानां कुले जातः सुमित्रोऽमित्रकर्शनः}
{चरामि मृगयूथानि निघ्रन्बाणैः सहस्रशः}


\twolineshloka
{बलेन महता ब्रह्मन्सामात्यः सावरोधकः}
{मृगस्तु विद्धो बाणेन मया सरति शल्यवान्}


\twolineshloka
{तं द्रवन्तमनुप्राप्तो वनमेतद्यदृच्छया}
{भवत्सकाशं नष्टश्रीर्हताशः श्रमकर्शितः}


\twolineshloka
{किंनु दुःखमतोऽन्यद्वै यदहं श्रमकर्शितः}
{भवतामाश्रमं प्राप्तो हताशो भ्रष्टलक्षणः}


\twolineshloka
{न राजलक्षणत्यागः पुनरस्य तपोधनाः}
{दुःखं करोति मे तीव्रं यथाऽऽशा विहता मम}


\twolineshloka
{हिमवान्वा महाशैलः समुद्रो वा महोदधिः}
{महत्त्वान्नान्वपद्येतां रोदस्योरन्तरं यथा}


\twolineshloka
{आशायास्तपसि श्रेष्ठास्तथा नान्तमहं गतः}
{भवतां विदितं सर्वं सर्वज्ञा हि तपोधनाः}


\twolineshloka
{भवन्तः सुमहाभागास्तस्मात्पृच्छामि संशयम्}
{आशावान्पुरुषो यः स्यादन्तरिक्षमथापि वा}


\twolineshloka
{किंनु ज्यायस्तरं लोके महत्त्वे प्रतिभाति वः}
{एतदिच्छामि तत्त्वेन श्रोतुं किमिह दुर्लभम्}


\twolineshloka
{यदि गुह्यं न वो नित्यं तदा प्रब्रूत मा चिरम्}
{न हि गुह्यतमं श्रोतुमिच्छामि द्विजपुङ्गवाः}


\twolineshloka
{भवत्तपोविघातो वा येन स्याद्विरमे ततः}
{यदि वाऽस्ति कथायोगो योऽयं प्रश्नो मयेरितः}


\twolineshloka
{एतत्कारणसामर्थ्यं श्रोतुमिच्छामि तत्त्वतः}
{भवन्तोऽपि तपोनित्या ब्रूयुरेतत्समाहिताः}


\chapter{अध्यायः १२७}
\twolineshloka
{भीष्म उवाच}
{}


\twolineshloka
{ततस्तेषां समेतानामृषीणामृषिसत्तमः}
{ऋषभो नाम विप्रर्षिर्विस्मयन्निदमब्रवीत्}


\twolineshloka
{पुराऽहं राजशार्दूल तीर्थान्यनुचरन्प्रभो}
{समासादितवान्दिव्यं नरनारायणाश्रमम्}


\twolineshloka
{यत्र सा बदरी रम्या सरो वैहायसं तथा}
{यत्र चाश्वशिरा राजन्वेदान्पठति शाश्वतान्}


\twolineshloka
{तस्मिन्सरसि कृत्वाऽहं विधिवत्तर्पणं पुरा}
{पितृणां देवतानां च ततोश्रममियां तदा}


\twolineshloka
{रमाते यत्र तौ नित्यं नरनारायणावृषी}
{अदूरादाश्रमात्किंचिद्वासार्थमगमं तदा}


\twolineshloka
{अत्र चीराजिनधरं कृशमुच्चमतीव च}
{अद्राक्षमृषिमायान्तं तनुं नाम तपोनिधिम्}


\twolineshloka
{अन्यैर्नरैर्महाबाहो वपुषाऽप्रतिमं तदा}
{कृशता चापि राजर्षे न दृष्टा तादृशी मया}


\twolineshloka
{शरीरमपि राजेन्द्र तनु कानिष्ठिकासमम्}
{ग्रीवा बाहू तथा पादौ केशाश्चाद्भुतदर्शनाः}


\twolineshloka
{शिरः कायानुरूपं च कर्णौ नेत्रे तथैव च}
{तस्य वाक्चैव चेष्टा च सामान्ये राजसत्तम}


\twolineshloka
{दृष्ट्वाऽहं तं कृशं विप्रं भीतः परमदुर्मनाः}
{पादौ तस्याभिवाद्याथ स्थितः प्राञ्जलिरग्रतः}


\twolineshloka
{निवेद्य नामगोत्रे च तथा कार्यं नरर्षभ}
{प्रदिष्टे चासने तेन शनैरहमुपाविशम्}


\twolineshloka
{ततः स कथयामास धर्मार्थसहिताः कथाः}
{ऋषिमध्ये महाराज तत्र धर्मभृतां वरः}


\twolineshloka
{त--स्तु कथयत्येव राजा राजीवलोचनः}
{उपयाज्जवनैरश्वैः सबलः सावरोधनः}


\twolineshloka
{स्मरम्पुत्रमरण्ये वै नष्टं परमदुर्मनाः}
{भूविद्युम्नपिता श्रीमान्वीरद्युम्नो महायशाः}


\twolineshloka
{इह द्रक्ष्यामि तं पुत्रं द्रक्ष्यामीहेति भारत}
{एवमाशाकृशो राजा चरन्वनमिदं पुरा}


\twolineshloka
{दुर्लभः स मया द्रष्टुं भूय एव च धार्मिक}
{एकः पुत्रो महारण्ये नष्ट इत्यसकृत्तदा}


\twolineshloka
{न स शक्यो मया द्रष्टुमाशा च महती मम}
{तया परीतगात्रोऽहं मुमूर्षुर्नात्र संशयः}


\twolineshloka
{एतच्छ्रुत्वा तु भगवांस्तनुर्मुनिवरोत्तमः}
{अवाक््शिरा ध्यानपरो मुहूर्तमिव तस्थिवान्}


\twolineshloka
{तमनुध्यान्तमालक्ष्य राजा परमदुर्मनाः}
{उवाच वाक्यं दीनात्मा मन्दमन्दमिवासकृत्}


\threelineshloka
{दुर्लभं किंनु देवर्षे आशायाश्चैव किं महत्}
{ब्रवीतु भगवानेतद्यदि गुह्यं न चेत्तदा ॥तनुरुवाच}
{}


\threelineshloka
{महर्षिर्भगवांस्तेन पूर्वमासीद्विमानितः}
{बालिशां बुद्धिमास्थाय मन्दभाग्यतयाऽऽत्मनः}
{अर्थयन्कुशलं राजन्काञ्चनं वल्कलानि च}


\twolineshloka
{[अवज्ञापूर्वकेनापि न संपादितवांस्ततः}
{]निर्विण्णः स तु विप्रर्षिर्निराशः समपद्यत}


\twolineshloka
{एवमुक्तोऽभिवाद्याथ तमृषिं लोकपूजितम्}
{श्रान्तोऽवसीदद्धर्मात्मा यथा त्वं नरसत्तम}


\twolineshloka
{अर्ध्यं ततः समानीय पाद्यं चैव महायशाः}
{आरण्येनैव विधिना राज्ञे सर्वं न्यवेदयत्}


\twolineshloka
{ततस्तमृषयः सर्वे परिवार्य नरर्षभम्}
{उपाविशन्पुरस्कृत्य सप्तर्षय इव ध्रुवम्}


\twolineshloka
{अपृच्छंश्चैव तत्रैनं राजानमपराजितम्}
{प्रयोजनमिदं सर्वमाश्रमस्य प्रवेशने}


\chapter{अध्यायः १२८}
\twolineshloka
{राजोवचा}
{}


\twolineshloka
{वीरद्युम्न इति ख्यातो राजाऽहं दिक्षु विश्रुतः}
{भूरिद्युम्नं सुतं नष्टमन्वेष्टुं वनमागतः}


\threelineshloka
{एकः पुत्रः स विप्राग्र्य बाल एव च सोऽनघः}
{न दृश्यते वने चास्मिंस्तमन्वेष्टुं चराम्यहम् ॥ऋषभ उवाच}
{}


\twolineshloka
{इत्युक्ते तेन वचने राज्ञा मुनिरधोमुखः}
{तूष्णीमेवाभवत्तत्र न च प्रत्युक्तवान्नृपम्}


\twolineshloka
{स हि तेन पुरा विप्रो राज्ञा नात्यर्थमानितः}
{आशाकृतश्च राजेन्द्र तपो दीर्घं समाश्रितः}


\twolineshloka
{प्रतिग्रहमहं राज्ञां न करिष्ये कथंचन}
{अन्येषां चैव वर्णानामिति कृत्वा धियं तदा}


\fourlineindentedshloka
{आशा हि पुरुषं बालमालापयति तस्थूषी}
{तामहं व्यपनेष्यामि इति कृत्वा व्यवस्थितः}
{[वीरद्युम्नस्तु तं भूयः पप्रच्छ मुनिसत्तमम् ॥] राजोवाच}
{}


\threelineshloka
{आशायाः किंच वृत्तं वै किंचेह भुवि दुर्लभम्}
{ब्रवीतु भगवानेतत्त्वं हि धर्मार्थदर्शिवान् ॥ऋषभ उवाच}
{}


\threelineshloka
{ततः संस्मृत्य तत्सर्वं स्मारयिष्यन्निवाब्रवीत्}
{राजानं भगवान्विप्रस्ततः कृशतनुस्तदा ॥ऋषिरुवाच}
{}


\threelineshloka
{कृशत्वेन समं राजन्नाशाया विद्यते नृप}
{तस्या वै दुर्लभत्वाच्च प्रार्थितः पार्थिवो मया ॥राजोवाच}
{}


\twolineshloka
{कृशाकृशे मया ब्रह्मन्गृहीते वचनात्तव}
{दुर्लभत्वं च तस्यैव वेदवाक्यमिवाद्विजे}


\twolineshloka
{संशयस्तु महाप्राज्ञ संजातो हृदये मम}
{तन्मुने मम तत्त्वेन वक्तुमर्हसि पृच्छतः}


\threelineshloka
{त्वत्तः कृशतरं किंनु ब्रवीतु भगवानिदम्}
{यदि गुह्यं न ते विप्र लोके किंचेह दुर्लभम् ॥कृश उवाच}
{}


\twolineshloka
{दुर्लभोऽप्यथवा नास्ति योऽर्थी धृतिमवाप्नुयात्}
{स दुर्लभतरस्तात योऽर्थिनं नावमन्यते}


\twolineshloka
{सत्कृत्य नोपक्रियते परं शक्त्या यथार्थतः}
{या सक्ता सर्वभूतेषु साऽऽशा कृशतरी मया}


\twolineshloka
{कृतघ्नेषु च या सक्ता नृशंसेष्वलसेषु च}
{अपकारिषु चासक्ता साऽऽशा कृशतरी मया}


\twolineshloka
{एकपुत्रः पिता पुत्रे नष्टे वा प्रोषितेऽपि वा}
{प्रवृत्तिं यो न जानाति साऽऽशा कृशतरी मता}


\twolineshloka
{प्रसवे चैव नारीणां वृद्धानां पुत्रकारिता}
{तथा नरेन्द्र धनिनां साऽशा कृशतरी मता}


\twolineshloka
{प्रदानकाङ्क्षिणीनां च कन्यानां वयसि स्थिते}
{श्रुत्वा कथास्तथायुक्ताः साऽऽशा कृशतरी मता}


\threelineshloka
{एतच्छ्रुत्वा ततो राजन्स राजा सावरोधनः}
{संस्पृश्य पादौ शिरसा निपपात द्विजर्षभे ॥राजोवाच}
{}


\fourlineindentedshloka
{प्रसादये त्वां भगवन्पुत्रेणेच्छामि संगमम्}
{यदेतदुक्तं भवता संप्रति द्विजसत्तम}
{वृणीष्व च वरान्विप्र यानिच्छसि यथाविधि}
{अब्रवीच्चैव तद्वाक्यं राजा राजीवलोचनः}


% Check verse!
सत्यमेतत्त्वया विप्र यथोक्तं नान्यथा मृषा
\twolineshloka
{ततः प्रहस्य भगवांस्तनुर्धर्मभृतां वरः}
{पुत्रमस्यानयत्क्षिप्रं तपसा च श्रुतेन च}


\twolineshloka
{स समानीय तत्पुत्रं तमुपालभ्य पार्थिवम्}
{आत्मानं दर्शयामास धर्मं धर्मभृतांवरः}


\twolineshloka
{स दर्शयित्वा चात्मानं दिव्यमद्भुतदर्शनम्}
{विपाप्मा विगतक्रोधश्चचार वनमन्तिकात्}


\threelineshloka
{एतदॄष्टं मया राजंस्त्वत्तश्च वचनं श्रुतम्}
{आशामपनय त्वाशु ततः कृषतरीमिमाम् ॥भीष्म उवाच}
{}


\twolineshloka
{स तथोक्तस्तदा राजन्नृषभेण महात्मना}
{सुमित्रोऽपानयत्क्षिप्रमाशां कृशतरीं ततः}


\twolineshloka
{एवं त्वमपि कौन्तेय श्रुत्वा वाणीमिमां मम}
{स्थिरो भव महाराज हिमवानिव निश्चलः}


\twolineshloka
{त्वं हि श्रुत्वा च पृष्ट्वा च कृच्छ्रेष्वर्थेषु तेष्विह}
{श्रुत्वा मम महाराज न संतप्तुमिहार्हसि}


\chapter{अध्यायः १२९}
\twolineshloka
{युधिष्ठिर उवाच}
{}


\twolineshloka
{नामृतस्येव पर्याप्तिर्ममास्ति ब्रुवति त्वयि}
{[यथाहि स्वात्मवृत्तिस्थस्तथा तृप्तोऽस्मि भारत ॥]}


\threelineshloka
{तस्मात्कथय भूयोऽपि त्वं ममेह पितामह}
{[न हि तृप्तिमहं यामि पिबन्धर्मामृतं हि ते ॥]भीष्म उवाच}
{}


\twolineshloka
{अत्राप्युदाहरन्तीममितिहासं पुरातनम्}
{गौतमस्य च संवादं यमस्य च महात्मनः}


\twolineshloka
{पारियात्रं गिरिं प्राप्य गौतमस्याश्रमो महान्}
{वसते गौतमो यत्र तपसा दग्धकिल्विषः}


\twolineshloka
{पृष्टिं वर्षसहस्राणि सोऽतप्यद्गौतमस्तपः}
{तमुग्रतपसा युक्तं भावितं सुमहामुनिम्}


\twolineshloka
{उपयातो नरव्याघ्र लोकपालो यमस्तदा}
{तमपश्यत्सुतपसमृषिं वै गौतमं तदा}


\twolineshloka
{स तं विदित्वा ब्रह्मर्षियेममागतमोजसा}
{प्राञ्जलिः प्रणतो भूत्वा उपसृप्तस्तपोधनः}


\threelineshloka
{तं धर्मराजो दृष्ट्वैव नमस्कृत्य द्विजोत्तमम्}
{न्यमन्त्रयत धर्मेण क्रियतां किमिति ब्रुवन् ॥गौतम उवाच}
{}


\threelineshloka
{मातापितृभ्यामानृण्यं किं कृत्वा समवाप्नुयात्}
{कथं च लोकानाप्नोति पुरुषो दुर्लभाञ्शुभान् ॥यम उवाच}
{}


\twolineshloka
{तपः शौचवता नित्यं सत्यधर्मरतेन च}
{मातापित्रोरहरहः पूजनं कार्यमञ्जसा}


\twolineshloka
{अश्वमेधैश्च यष्टव्यं बहुभिः स्वाप्तदक्षिणैः}
{तेन लोकानवाप्नोति पुरुषोऽद्भुतदर्शनान्}


\chapter{अध्यायः १३०}
\twolineshloka
{युधिष्ठिर उवाच}
{}


\twolineshloka
{त्रैः प्रहीयमाणस्य बह्वमित्रस्य का गतिः}
{-- संक्षीणकोशस्य बलहीनस्य भारत}


\twolineshloka
{--मात्यसहायस्य श्रुतमन्त्रस्य सर्वतः}
{ज्यात्प्रच्यवमानस्य गतिमन्यामपश्यतः}


\twolineshloka
{परचक्राभियातस्य परराष्ट्राणि मृद्गतः}
{विग्रहे वर्तमानस्य दुर्बलस्य बलीयसा}


\threelineshloka
{असंविहितराष्ट्रस्य देशकालावजानतः}
{अप्राप्यं च भवेत्सान्त्वं भेदो वाऽप्यतिपीडनात्}
{}


\threelineshloka
{जीवितं त्वर्थहेतोर्वा तत्र किं सुकृतं भवेत् ॥भीष्म उवाच}
{गुह्यां मा धर्ममप्राक्षीरतीव भरतर्षभ}
{}


\twolineshloka
{प्रवक्तुं नोत्सहे पृष्टो धर्ममेतं युधिष्ठिर ॥धर्मो ह्यणीयान्वचनाद्वुद्धेश्च भरतर्षभ}
{श्रुत्वौपम्यं सदाचारैः साधुर्भवति स क्वचित्}


\twolineshloka
{कर्मणा बुद्धिपूर्वेण भवत्याढ्ये न वा पुनः}
{तादृशोऽयमनुप्रश्नस्तद्ध्यायस्व स्वया धिया}


\twolineshloka
{उपायं धर्मबहुलं यात्रार्थं शृणु भारत}
{नाहमेतादृशे धर्मे बुभूषे धर्मकारणात्}


\twolineshloka
{दुःखादान इह ह्येष स्यात्तु पश्चात्क्षमो मम}
{अभिगम्य मतीनां हि सर्वासामेव निश्चयम्}


\twolineshloka
{यथायथा हि पुरुषो नित्यं शास्त्रमवेक्षते}
{तथातथा विजानाति विज्ञानं चास्य रोचते}


\twolineshloka
{अविज्ञानादयोगो हि पुरुषस्योपजायते}
{विज्ञानादपि योगश्च योगो भूतिकरः परः}


\twolineshloka
{अशङ्कमानो वचनमनसूयुरिदं शृणु}
{राज्ञः कोशक्षयादेव जायते बलसंक्षयः}


\twolineshloka
{कोशं संजनयेद्राजा नित्यमेभ्यो यथाबलम्}
{कालं प्राप्यानुगृह्णीयादेष धर्मोऽत्र सांप्रतम्}


\twolineshloka
{उपायधर्मं प्राप्यैनं पूर्वैराचरितं जनैः}
{अन्यो धर्मः समर्थानामापत्स्वल्पश्च भारत}


\twolineshloka
{प्रकार्यं प्रोच्यते धर्मो वृत्तिर्धर्मे गरीयसी}
{धर्मं प्राप्य यथान्यायं न बलीयान्निषीदति}


\twolineshloka
{यस्माद्धर्मस्योपचितिरेकान्तेन न विद्यते}
{तस्मादापद्यधर्मोऽपि श्रूयते धर्मलक्षणः}


\twolineshloka
{अधर्मो जायते तस्मिन्निति वै कवयो विदुः}
{अनन्तरं क्षत्रियस्य तत्र किं विचिकित्स्यते}


\twolineshloka
{यथास्य धर्मो न ग्लायेन्नेयाच्छत्रुवशं यथा}
{तत्कर्तव्यमिहेत्याहुर्नात्मानमवसादयेत्}


\twolineshloka
{सर्वात्मनैव धर्मस्य न परस्य न चात्मनः}
{सर्वोपायैरुज्जिहीर्षेदात्मानमिति निश्चयः}


\twolineshloka
{तत्र धर्मविदस्तात निश्चयो धर्मनैपुणैः}
{उद्यमं जीवनं क्षात्रे बाहुवीर्यादिति श्रुतिः}


\twolineshloka
{क्षत्रियो वृत्तिसंरोधे कस्य नादातुमर्हति}
{अन्यत्र तापसस्वाच्च श्रोत्रियस्वाच्च भारत}


\twolineshloka
{यथा वै ब्राह्मणः सीदन्नयाज्यमपि याजयेत्}
{अभोज्यमपि चाश्नीयात्तत्रेदं नात्र संशयः}


\twolineshloka
{पीडितस्य किमद्वारमुत्पथेनार्दितस्य च}
{अद्वारतः प्रद्रवति यथा भवति पीडितः}


\twolineshloka
{तस्य कोशबलग्लान्यां सर्वलोकपराभवः}
{भैक्षचर्या न विहिता न च विट््शूद्रजीविका}


\twolineshloka
{स्वधर्मानन्तरावृत्तिर्याऽन्यामनुपजीवतः}
{जहतः प्रथमं कल्पमनुकल्पेन जीवनम्}


\twolineshloka
{आपद्गतेन धर्माणामन्यायेनोपजीवनम्}
{अपि ह्येतद्ब्राह्मणेषु दृष्टं वृत्तिपरिक्षये}


\twolineshloka
{क्षत्रिये संशयः कस्मादित्येत्निश्चितं सदा}
{आददीत विशिष्टेभ्यो नावसीदेत्कथंचन}


\twolineshloka
{आर्तानां रक्षितारं च प्रजानां क्षत्रियं विदुः}
{तस्मात्संरक्षता कार्यमादानं क्षत्रबन्धुना}


\twolineshloka
{अन्यत्रापि विहिंसाया वृत्तिर्नेहास्ति कस्यचित्}
{अप्यरण्यसमुत्थस्य एकस्य चरतो मुनेः}


\twolineshloka
{न शङ्खलिखितां वृत्तिं शक्यमास्थाय जीवितुम्}
{विशेषतः कुरुश्रेष्ठ प्रजापालनमीप्सता}


\twolineshloka
{परस्पराभिहरणं राज्ञा राष्ट्रेण चापदि}
{नित्यमेव हि कर्तव्यमेष धर्मः सनातनः}


\twolineshloka
{राजा राष्ट्रं यथापत्सु द्रव्यौधैः परिरक्षति}
{राष्ट्रेण राजा व्यसने परिरक्ष्यस्तथा भवेत्}


\twolineshloka
{कोशं दण़्डं बलं मित्रं यदन्यदपि संचितम्}
{न कुर्वीतान्तरं राष्ट्रे राजा परिगतः क्षुधा}


\twolineshloka
{बीजं भक्तेन संपाद्यमिति धर्मविदो विदुः}
{अत्रैतच्छम्बरस्याहुर्महामायस्य दर्शनम्}


\twolineshloka
{धिक्तस्य जीवितं राज्ञो राष्ट्रं यस्यावसीदति}
{अवृत्त्यान्यमनुष्योऽपि यो वैदेशिक इत्यपि}


\twolineshloka
{राज्ञः कोशबलं मूलं कोशमूलं पुनर्बलम्}
{तन्मूलं सर्वधर्माणां धर्ममूलाः पुनः प्रजाः}


\twolineshloka
{नान्यानपीडयित्वेह कोशः शक्यः कुतो बलम्}
{तदर्थं पीडयित्वा च न दोषं प्राप्नुमर्हति}


\twolineshloka
{अकार्यमपि कार्यार्थं क्रियते यज्ञकर्मसु}
{एतस्मिन्कारणे राजा न दोषं प्राप्नुमर्हति}


\threelineshloka
{अर्थार्थमन्यद्भवति विपरीतमथापरम्}
{अनर्थार्थमथाप्यन्यत्तत्सर्वं ह्यर्थकारणम्}
{एवं बुद्ध्या संप्रपश्येन्मेधावी कार्यनिश्चयम्}


\twolineshloka
{यज्ञार्थमन्यद्भवति यज्ञोऽन्यार्थस्तथाः परः}
{यज्ञस्वार्थार्थमेवान्यत्तत्सर्वं यज्ञसाधकम्}


% Check verse!
उपमामत्र वक्ष्यामि धर्मतत्त्वप्रकाशिनीम्
\twolineshloka
{यूपं छिन्दन्ति यज्ञार्थं तत्र ये परिपन्थिः}
{द्रुमाः केचन सामन्ता ध्रुवं छिन्दन्ति तानपि}


\threelineshloka
{ते चापि निपतन्तोऽन्यान्निघ्नन्त्यपि वनस्पतीन्}
{एवं कोशस्य महतो ये नराः परिपन्थिनः}
{तानहत्वा न पश्यामि सिद्धिमत्र परंतप}


\twolineshloka
{धनेन जयते लोकमिमं चामुं च भारत}
{सत्यं च धर्मवचनं यथा नास्त्यधनस्तथा}


\twolineshloka
{सर्वोपायैराददीत धनं यज्ञप्रयोजनम्}
{न तुल्यदोषः स्यादेवं कार्याकार्येषु भारत}


\twolineshloka
{नोभौ सभवतो राजन्कथंचिदपि भारत}
{न ह्यरण्येषु पश्यामि धनवृद्धानहं क्वचित्}


\twolineshloka
{यदिदं दृश्यते वित्तं पृथिव्यामिह किंचन}
{ममेदं स्यान्ममेदं स्यादित्येवं मन्यते जनः}


\twolineshloka
{न च राज्ञः समो धर्मः कश्चिदस्ति कथंचन}
{धर्मः संशब्दितो राज्ञामापदर्थस्ततोऽन्यथा}


\twolineshloka
{ज्ञानेन कर्मणा चान्ये तपन्त्यन्ये तपस्विनः}
{बुद्ध्या दाक्ष्येण चैवान्ये चिन्वन्ति धनसंचयान्}


\twolineshloka
{अधनं दुर्बलं प्राहुर्धनेन बलवान्भवेत्}
{सर्वं बलवतः प्राप्यं सर्वं तरति कोशवान्}


\twolineshloka
{कोशो धर्मश्च कामश्च परलोकस्तथा ह्ययम्}
{तं धर्मेण विलिप्सेत नाधर्मेण कदाचन}


\chapter{अध्यायः १३१}
\twolineshloka
{युधिष्ठिर उवाच}
{}


\twolineshloka
{क्षीणस्य दीर्घसूत्रस्य सानुक्रोशस्य बन्धुषु}
{परिशङ्कितमुख्यस्य दुष्टमन्त्रस्य भारत}


\twolineshloka
{विरक्तराज्यपौरस्य निर्द्रव्यनिचयस्य च}
{असंभावितमित्रस्य भिन्नामात्यस्य सर्वतः}


\threelineshloka
{परचक्राभियातस्य दुर्बलस्य बलीयसा}
{आपन्नचेतसो ब्रूहि किं कार्यमवशिष्यते ॥भीष्म उवाच}
{}


\twolineshloka
{बाह्यश्चेद्विजिगीषुः स्याद्धर्मार्थकुशलः शुचिः}
{जवेन सन्धिं कुर्वीत पूर्वं पूर्वं विमोक्षयेत्}


\twolineshloka
{[योऽधर्मविजिगीषुः स्याद्बलवान्पापनिश्चयः}
{]आत्मनः सन्निरोधेन सन्धिं तेनापि रोचयेत्}


\twolineshloka
{अपास्य राजधानीं वा तरेदन्येन वाऽऽपदम्}
{तद्भावभावो द्रव्याणि जीवन्पुनरुपार्जयेत्}


\twolineshloka
{यास्तु कोशबलत्यागाच्छक्यास्तरितुमापदः}
{कस्तत्राधिकमात्मानं संत्यजेदर्थधर्मवित्}


\threelineshloka
{अपराधाज्जुगुप्सेत का सपत्नधने दया}
{न त्वेवात्मा प्रदातव्यः शक्ये सति कथंचन ॥युधिष्ठिर उवाच}
{}


\threelineshloka
{आभ्यन्तरे च कुपिते बाह्ये चोपनिपीडिते}
{क्षीणे कोशे श्रुते मन्त्रे किं कार्यमवशिष्यते ॥भीष्म उवाच}
{}


\twolineshloka
{क्षिप्रं वा सन्धिकामः स्यात्क्षिप्रं वा तीक्ष्णविक्रम}
{यदाऽपनयनं क्षिप्रमेतद्वै सांपरायिकम्}


\twolineshloka
{अनुरक्तेन पुष्टेन हृष्टेन जगतीपतिः}
{अल्पेनापि स्वसैन्येन भूमिं जयति भूमिपः}


\twolineshloka
{हतो वा दिवमारोहेद्धत्वा च सुखमावहेत्}
{युद्धे हि संत्यजन्प्राणाञ्शक्रस्यैति सलोकताम्}


\twolineshloka
{सर्वलोकागसं कृत्वा मृदुत्वं गन्तुमेव च}
{विश्वासाद्विनयं कुर्यात्संजह्याद्वाऽप्युपानहौ}


\twolineshloka
{अपचिक्रमिषुः क्षिप्रं सेनां स्वां परिसान्त्वयन्}
{विलङ्घयित्वा सत्रेण ततः स्वयमुपक्रमेत्}


\chapter{अध्यायः १३२}
\twolineshloka
{युधिष्ठिर उवाच}
{}


\twolineshloka
{हीने परमके धर्मे सर्वलोकातिलङ्घने}
{सर्वस्मिन्दस्युसाद्भूते पृथिव्यामुपजीवने}


\threelineshloka
{केन स्विद्ब्राह्मणो जीवेज्जघन्ये काल आगते}
{असंत्यजन्पुत्रपौत्राननुक्रोशात्पितामह ॥भीष्म उवाच}
{}


\twolineshloka
{विज्ञानबलमास्थाय जीवितव्यं तथा गते}
{सर्वं साध्वर्थमेवेदमसाध्वर्थं न किंचन}


\twolineshloka
{असाधुभ्योऽर्थमादाय साधुभ्यो यः प्रयच्छति}
{आत्मानं संक्रमं कृत्वा कृच्छ्रधर्मकृदेव सः}


\twolineshloka
{अरोषेणात्मनो राजन्राज्ये स्थितिमकोपयन्}
{अदत्तमप्याददीत् भ्रातुर्वित्तं ममेति वा}


\twolineshloka
{विज्ञानबलपूतो यो वर्तते निन्दितेष्वपि}
{वृत्तिविज्ञानवान्धीरः कस्तं वक्तुमिहार्हति}


\twolineshloka
{येषां बहुकृता बुद्धिस्तेषामन्या न रोचते}
{यजसा ते प्रवर्तन्ते बलवन्तो युधिष्ठिर}


\twolineshloka
{यदैव प्रकृतं शास्त्रं जनस्तदनुवर्तते}
{यदैवमध्यासेवन्ते मेध्रावी वाऽप्यथोत्तरम्}


\twolineshloka
{ऋत्विक्पुरोहिताचार्यान्सत्कृतानभिसत्कृतान्}
{न ब्राह्मणान्घातयीत दोषान्प्राप्नोति घातयन्}


\twolineshloka
{एतत्प्रमाणं लोकस्य चक्षुरेतत्सातनम्}
{तत्प्रमाणोऽवगाहेत तेन तत्साध्वसाधु वा}


\twolineshloka
{हवो ग्रामवास्तव्या दोषान्ब्रूयुः परस्परम्}
{न तेषां वचनाद्राजा सत्कुर्याद्धातयीत वा}


\twolineshloka
{न वाच्यः परिवादो वै न श्रोतव्यः कथंचन}
{कर्णौ तत्र पिधातव्यौ गन्तव्यं वा ततोऽन्यतः}


\twolineshloka
{न सतां शीलमेतद्वै परिवादो न पैशुनम्}
{गुणानामेव वक्तारः सन्तो नित्यं युधिष्ठिर}


\twolineshloka
{यथा समधुरौ दम्यौ सुदान्तौ साधुवाहिनौ}
{धुरमुद्यम्य वहतस्तथा वर्तेत वै नृपः}


\twolineshloka
{यथायथाऽस्य बहवः सहायाः स्युस्तथा चरेत्}
{आचारमेव मन्यन्ते गरीयो धर्मलक्षणम्}


\twolineshloka
{अपरे नैवमिच्छन्ति ये शङ्खलिखिंतप्रियाः}
{अर्थे क्षीणेऽथवा लुब्धास्ते ब्रूयुर्वाक्यमीदृशम्}


\twolineshloka
{आर्षमष्यत्र पश्यन्ति विकर्मस्थस्य पातनम्}
{न चार्षात्सदृशं किंचित्प्रमाणं दृश्यते क्वचित्}


\twolineshloka
{देवा ह्यपि विकर्मस्थं घातयन्ति नराधमम्}
{व्याजेन विन्दन्वित्तं हि धर्मतः परिहीयते}


\twolineshloka
{सर्वतः सत्कृतः सद्भिर्भूतिप्रवरकारणैः}
{हृदयेनाभ्यनुज्ञातो यो धर्मस्तं व्यवस्यति}


\twolineshloka
{यश्चतुर्गुणसंपन्नं धर्मं वेद स धर्मवित्}
{अहेरिव हि धर्मस्य पदं दुःखं गवेषितुम्}


\twolineshloka
{यथा मृगस्य विद्धस्य मृगव्याधः पदं नयेत्}
{लक्षेद्रुधिरपातेन तथा धर्मपदं नयेत्}


\twolineshloka
{यथा सम्यग्वितेन पथा गन्तव्यमप्युत}
{राजर्षीणां वृत्तमेतदेवं गच्छ युधिष्ठिर}


\chapter{अध्यायः १३३}
\twolineshloka
{भीष्म उवाच}
{}


\twolineshloka
{स्वराष्ट्रात्परराष्ट्राच्च कोशं संजनयेन्नृपः}
{कोशाद्धि धर्मः कौन्तेय राज्यमूलं प्रवर्तते}


\twolineshloka
{तस्मात्संजनयेत्कोशं सत्कृत्य परिपालयेत्}
{परिपाल्यानुगृह्णीयादेव धर्मः सनातनः}


\twolineshloka
{स कोशः शुद्धभावेन न नृशंसेन जायते}
{मध्यमं पदमास्थाय कोशसंग्रहणं चरेत्}


\twolineshloka
{अबलस्य कुतः कोशो ह्यकोशस्य कुतो बलम्}
{अबलस्य कुतो राज्यमराज्ये श्रीर्भवेत्कुतः}


\twolineshloka
{उच्चैर्वृत्तेः श्रियो हानिर्यथैव मरणं तथा}
{तस्मात्कोशं बलं मित्रमथ राजा विवर्धयेत्}


\twolineshloka
{हीनकोशं हि राजानमवमन्यन्ति शत्रवः}
{न चास्याल्पे तुष्यन्ति कर्मणाऽप्युत्सहन्ति च}


\twolineshloka
{श्रियो हि कारणाद्राजा सत्क्रियां लभते पराम्}
{साऽस्य गूहति पापानि वासो गुह्यमिव स्त्रियाः}


\threelineshloka
{ऋद्धिमस्यानुतप्यन्ते पुरा विप्रकृता नराः}
{सालावृका इवाजस्नं जिघांसूनेव विन्दति}
{ईदृशस्य कुतो राज्यं सुखं भरतसत्तम}


\twolineshloka
{उद्यच्छेदेव न ग्लायेदुद्यमो ह्येव पौरुषम्}
{अप्यपर्वणि भज्येत न नमेतेह कस्यचित्}


\twolineshloka
{अप्यरण्यं समाश्रित्य चरेन्मृगगणैः सह}
{न त्वेवोद्रिक्तमर्यादैर्दस्युभिः सहितश्चरेत्}


\twolineshloka
{दस्यूनां सुलभा सेना रौद्रकर्मसु भारत}
{एकान्ततो ह्यमर्यादात्सर्वोऽप्युद्विजते जनः}


% Check verse!
दस्यवोऽप्यभिशङ्कन्ते निरनुक्रोशकारिणः
\twolineshloka
{स्थापयेदेव मर्यादां जनचित्तप्रसादिनीम्}
{अल्पाप्यर्थेषु मर्यादा लोके भवति पूजिता}


\twolineshloka
{नायं लोकोऽस्ति न पर इति व्यवसितो जनः}
{नालं गन्तुमिहाश्वासं नास्तिक्यभयशङ्कितैः}


\twolineshloka
{यथा सद्भिः परादानमहिंसा दस्युभिस्तथा}
{अनुरज्यन्ति भूतानि समर्यादेषु दस्युषु}


\twolineshloka
{अयुध्यमानस्यादानं दारामर्शः कृतघ्नता}
{ब्रह्मवित्तस्य चादानं निःशेषकरणं तथा}


\twolineshloka
{स्त्रिया मोषः पथिस्थानं साधुष्वेव विगर्हितम्}
{सदोष एव भवति दस्युरेतानि वर्जयेत्}


\twolineshloka
{अभिसंदधते ये च विनाशायास्य भारत}
{सशेषमेवोपलभ्य कुर्वन्तीति विनिश्चयः}


\twolineshloka
{तस्मात्सशेषं कर्तव्यं स्वाधीनमपि दस्युभिः}
{न बलस्थोऽहमस्मीति नृशंसानि समाचरेत्}


\twolineshloka
{सशेषकारिणस्तत्र शेषं पश्यन्ति सर्वशः}
{निःशेषकारिणो नित्यं निःशेषकरणाद्भयम्}


\chapter{अध्यायः १३४}
\twolineshloka
{भीष्म उवाच}
{}


\threelineshloka
{`अत्राप्युदाहरन्तीममितिहासं पुरातनम्}
{'अत्र कामन्दवचनं कीर्तयन्ति पुराविदः}
{प्रत्यक्षावेव धर्मार्थौ क्षत्रियस्य विजानतः}


\twolineshloka
{तौ तु न व्यवधातव्यौ परोक्षा धर्मयातना}
{अधर्मो धर्म इत्येतद्यथा वृक्षफलं तथा}


\twolineshloka
{धर्माधर्मफले जातु ददर्शेह न कश्चन}
{वुभूषेद्बलमेवैतत्सर्वं बलवतो वशे}


\twolineshloka
{श्रियं बलममात्यांश्च बलवानिह विन्दति}
{यो ह्यनाढ्यः स पतितस्तदुच्छिष्टं यदल्पकम्}


\twolineshloka
{बह्वपथ्ये बलवति न किंचित्क्रियते भयात्}
{उभौ सत्याधिकारौ तौ त्रायेते महतो भयात्}


\twolineshloka
{अति धर्माद्बलं मन्ये बलाद्धर्मः प्रवर्तते}
{बले प्रतिष्ठितो धर्मो धरण्यामिव जंगमः}


\twolineshloka
{धूमो वायोरिव वशे बलं धर्मोऽनुवर्तते}
{अनीश्वरो बलं धर्मो द्रुमं वल्लीव संश्रिता}


\twolineshloka
{वशे बलवतां धर्मः सुखं भोगवतामिव}
{नास्त्यसाध्यं बलवतां सर्वं बलवता जितम्}


\twolineshloka
{दुराचारः क्षीणबलः परिमाणं न गच्छति}
{अथ तस्मादुद्विजते सर्वो लोको वृकादिव}


\twolineshloka
{अपध्वस्तो ह्यवमतो दुःखं जीवति जीवितम्}
{जीवितं यदधिक्षिप्तं यथैव मरणं तथा}


\twolineshloka
{यदेवमाहुः पापेन चारित्रेण विवक्षितम्}
{सुभृशं तप्यते तेन वाक््शल्येन परिक्षतः}


\twolineshloka
{अत्रैतदाहुराचार्याः पापस्य परिमोक्षणे}
{त्रयीं विद्यामुपासीत तथोपासीत वै द्विजान्}


\twolineshloka
{प्रसादयेन्मधुरया वाचा चाप्यथ कर्मणा}
{महामनाश्चैव भवेद्विवहेच्च महाकुले}


\twolineshloka
{इत्यस्तीति वदेदेव परेषां कीर्तयेद्गुणान्}
{जपेदुदकशीलः स्यात्पेशलो नातिजल्पकः}


\twolineshloka
{ब्रह्म क्षत्रं संप्रविशेद्बहु कृत्वा सुदुष्करम्}
{उच्यमानो हि लोकेन बहु तत्तदचिन्तयन्}


\twolineshloka
{उपप्राप्यैवमाचारं क्षिप्रं बहुमतो भवेत्}
{सुखं च वित्तं भुञ्जीत वृत्तेनैकेन गोपयेत्}


\threelineshloka
{`अपि तेभ्यो मृगान्हत्वा नयेच्च सततं वने}
{यस्मिन्न प्रतिगृह्णन्ति दस्युभोजनशङ्कया}
{'लोके च लभते पूजां परत्रेह महत्फलम्}


\chapter{अध्यायः १३५}
\twolineshloka
{भीष्म उवाच}
{}


\twolineshloka
{अत्राप्युदाहन्तीममितिहासं पुरातनम्}
{यथा दस्युः समर्यादो दस्युत्वात्सिद्धिमाप्तवान्}


\twolineshloka
{प्रहर्ता मतिमाञ्शूरः श्रुतवान्सुनृशंसवान्}
{अक्षन्नाश्रमिणां धर्मं ब्रह्मण्यो गुरुपूजकः}


\twolineshloka
{अनिषाद्यां क्षत्रियाज्जातः क्षत्रधर्मानुपालकः}
{कापच्यो नाम नैषादिर्दस्युत्वात्सिद्धिमाप्तवान्}


\twolineshloka
{अरण्ये सायं पूर्वाह्णे मृगयूथप्रकोपिता}
{वेधिज्ञो मृगजातीनां नैषादानां च कोविदः}


\twolineshloka
{सर्वकाननदेशज्ञः पारियात्रचरः सदा}
{धर्मज्ञः सर्ववर्णानाममोघेषुर्दृढायुधः}


\twolineshloka
{अप्यनेकशतां सेनामेक एव जिगाय सः}
{स वृद्धावम्बपितरौ महारण्येऽभ्यपूजयत्}


\twolineshloka
{मधुमांसैर्मूलफलैरन्नैरुच्चावचैरपि}
{सत्कृत्य भोजयामास सम्यक्परिचचार ह}


\twolineshloka
{आरण्यकान्प्रव्रजितान्ब्राह्मणान्परिपालयन्}
{अपि तेभ्यो मृगान्हत्वा निनाय सततं वने}


\twolineshloka
{येऽस्मान्न प्रतिगृह्णन्ति दस्युभोजनशङ्कया}
{तेषामासज्य गेहेषु कल्य एव स गच्छति}


\threelineshloka
{तं बहूनि सहस्राणि ग्रामणीत्वेऽभिवव्रिरे}
{निर्मर्यादानि दस्यूनां निरनुक्रोशवर्तिनाम् ॥दस्यव ऊचुः}
{}


\twolineshloka
{मुहूर्तदेशकालज्ञः प्राज्ञः शूरो दृढव्रतः}
{ग्रामणीर्भव नो मुख्यः सर्वेपामेव संमतः}


\threelineshloka
{यथायथा वक्ष्यसि नः करिष्यामस्तथातथा}
{पालयास्मान्यथान्यायं यथा माता यथा पिता ॥कापच्य उवाच}
{}


\twolineshloka
{मा वधीस्त्वं स्त्रियं भीरुं मा शिशुं मा तपस्विनम्}
{नायुध्यमानो हन्तव्यो न च ग्राह्या बलात्स्त्रियः}


\twolineshloka
{सर्वथा स्त्री न हन्तव्या सर्वसत्वेषु बुध्यत}
{नित्यं गोब्राह्मणे स्वस्ति योद्धव्यं च तदर्थतः}


\twolineshloka
{सत्यं च नापि हर्तव्यं सारविघ्नं च मा कृथाः}
{पूज्यन्ते यत्र देवाश्च पितरोऽतिथयश्च ह}


\twolineshloka
{सर्वभूतेष्वपि वरो ब्राह्मणो मोक्षमर्हति}
{कार्या चापचितिस्तेषां सर्वस्वेनापि भावयेत्}


\twolineshloka
{यस्य त्वेते संप्रदुष्टास्तस्य विद्यात्पराभवम्}
{न तस्य त्रिषु लोकेषु त्राता भवति कस्चन}


\twolineshloka
{यो ब्राह्मणान्परिभवेद्विनाशं चापि रोचयेत्}
{सूर्योदय इव ध्वान्ते ध्रुवं तस्य पराभवः}


\twolineshloka
{इहैव फलमासीनः प्रत्याकाङ्क्षेत सर्वशः}
{येये नो न प्रदास्यन्ति तांस्तांस्तेनाभियास्यसि}


\twolineshloka
{शिष्ट्यर्थं विहितो दण्डो न वधार्थं विधीयते}
{ये च शिष्टान्प्रबाधन्ते धर्मस्तेषां वधः स्मृतः}


\twolineshloka
{ये च राष्ट्रोपरोधेन वृद्धिं कुर्वन्ति केचन}
{तानेवानुम्रियेरंस्ते कुणपं कृमयो यथा}


\fourlineindentedshloka
{ये पुनर्धर्मशास्त्रेण वर्तेरन्निह दस्यवः}
{अपि ते दस्यवो भूत्वा क्षिप्रं सिद्धिमवाप्नुयुः}
{भीष्म उवाच}
{}


\twolineshloka
{ते सर्वमेवानुचक्रुः कापच्यस्यानुशासनम्}
{वृद्धिं च लेभिरे सर्वे पापेभ्यश्चाप्युपारमन्}


\twolineshloka
{कापच्यः कर्मणा तेन महतीं सिद्धिमाप्तवान्}
{साधूनामाचरन्क्षेमं दस्यून्पापान्निवर्तयन्}


\twolineshloka
{इदं कापच्यचरितं यो नित्यमनुचिन्तयेत्}
{नारण्येभ्योऽपि भूतेभ्यो भयमृच्छेत्कथंचन}


\twolineshloka
{न भयं तस्य मर्त्येभ्यो नामर्त्येभ्यः कथंचन}
{न सतो नासतो राजन्स ह्यरण्येषु गोपतिः}


\chapter{अध्यायः १३६}
\twolineshloka
{भीष्म उवाच}
{}


\twolineshloka
{अत्र गाथा ब्रह्मगीताः कीर्तयन्ति पुराविदः}
{येन मार्गेण राजानः कोशं संजनयन्त्युत}


\twolineshloka
{न धनं यज्ञशीलानां हार्यं देवस्वमेव च}
{दस्यूनां निष्क्रियाणां च क्षत्रियो हर्तुमर्हति}


\twolineshloka
{इमाः प्रजाः क्षत्रियाणां रक्ष्या हन्याश्च भारत}
{थनं हि क्षत्रियस्येह द्वितीयस्य न विद्यते}


\twolineshloka
{तदस्य स्याद्बलार्थं वा धनं यज्ञार्थमेव वा}
{अभोज्याश्चौषधीश्छित्त्वा भोज्या एव पचन्त्युत}


\twolineshloka
{यो वै न देवान्न पितृन्न मर्त्यान्हविषाऽर्चति}
{अनर्थकं धनं तत्र प्राहुर्धमेविदो जनाः}


\twolineshloka
{हरेत्तद्द्रविणं राजन्धार्मिकः पृथिवीतिः}
{न हि न प्रीणयेल्लोकान्न लोके गर्हते नृपम्}


\twolineshloka
{असाधुभ्योऽर्थमादाय साधुभ्यो यः प्रयच्छति}
{आत्मानं संक्रमं कृत्वा मन्ये धर्मविदेव सः}


\twolineshloka
{[तथातथा जयेल्लोकाञ्शक्त्या चैव यथायथा}
{]औद्भिदा जन्तवो यद्वच्छ्रुत्वा वाजो यथातथा}


\twolineshloka
{अनिमित्तात्संभवन्ति तथा यज्ञः प्रजायते}
{यथैव दंशमशकं यथा कीटपिपीलिकम्}


% Check verse!
सैव वृत्तिर्हि यज्ञेषु यथा धर्मो विधीयते
\twolineshloka
{यथा ह्यकस्माद्भवति भूमौ पांसुस्तृणोलपम्}
{तथैवेह भवेद्धर्मः सूक्ष्मः सूक्ष्मतरः स्मृतः}


\chapter{अध्यायः १३७}
\twolineshloka
{भीष्म उवाच}
{}


\twolineshloka
{अनागतविधाता च प्रत्युत्पन्नमतिश्च यः}
{द्वावेतौ सुखमेधेते दीर्घसूत्री विनश्यति}


\twolineshloka
{अत्रैव चेदमव्यग्रं शृण्वाख्यानमनुत्तमम्}
{द्रीर्घसूत्रमुपाश्रित्य कार्याकार्यविनिश्चये}


\twolineshloka
{नातिगाधे जलस्थाने सुहृदः कुशलास्त्रयः}
{प्रभूतमत्स्ये कौन्तेय बभूवुः सहचारिणः}


\twolineshloka
{तत्रैकः प्राप्तकालज्ञो दीर्घदर्शी तथाऽपरः}
{दीर्घसूत्रश्च तत्रैकस्त्रयाणां जलचारिणाम्}


\twolineshloka
{क्रदाचित्तज्जलस्थानं मत्स्यबन्धाः समन्ततः}
{स्रावयामासुरथो निम्नेषु विविधैर्मुखैः}


\twolineshloka
{क्षीयमाणं तद्बुद्ध्वा जलस्थानं भयागमे}
{अब्रवीद्दीर्घदर्शी तु तावुभौ सुहृदौ तदा}


\twolineshloka
{ज्ञयमापत्समुत्पन्ना सर्वेषां सलिलौकसाम्}
{शीघ्रमन्यत्र गच्छामः पन्था यावन्न शुष्यति}


\twolineshloka
{अनागतमनर्थं हि सुनयैर्यः प्रबाधते}
{स न संशयमाप्नोति तथाऽन्यत्र व्रजामहे}


\twolineshloka
{शीर्घसूत्रस्तु यस्तत्र सोऽब्रवीत्सम्यगुष्यताम्}
{न तु कार्या त्वरा तावदिति मे निश्चिता मतिः}


\twolineshloka
{अथ संप्रतिपत्तिज्ञस्त्वब्रवीद्दीर्घदर्शिनम्}
{प्राप्ते काले न मे किंचिन्न्यायतः परिहास्यते}


\twolineshloka
{एवप्नुक्तो निराक्रामद्दीर्घदर्शी महामतिः}
{अगाम स्रोतसैकेन गम्भीरं सलिलाशयम्}


\twolineshloka
{ततः प्रसृततोयं तं प्रसमीक्ष्य जलाशयम्}
{बबन्धुर्विविधैर्योगैर्मत्स्यान्मत्स्योपजीविनः}


\twolineshloka
{विलोड्यमाने तस्मिंस्तु स्रुततोये जलाशये}
{अगच्छद्ग्रहणं तत्र दीर्घसूत्रः सहापरै----}


\twolineshloka
{उद्दानं क्रियमाणं तु मत्स्यानां ------}
{प्रविश्यान्तरमन्येषामग्रसत्प्रति-------}


\twolineshloka
{ग्रस्तमेव तदुद्दानं गृहीत्वा----सः}
{सर्वानेव च तांस्तत्र ते वि--- इति}


\twolineshloka
{ततः प्रक्षाल्यमानेषु मत्स्येषु विपुले जले}
{त्वक्त्वा रज्जुं प्रमुक्तोसौ शीघ्रं संप्रतिपत्तिमान्}


\twolineshloka
{दीर्घसूत्रस्तु मन्दात्मा हीनेयुद्धिरचेतनः}
{मरणं प्राप्तवान्मूढो यथैवोपहतेन्द्रियः}


\twolineshloka
{एवं प्राप्ततमं कालं यो मोहान्नावबुध्यते}
{स विनश्यति वै क्षिप्रं दीर्घसूत्रो यथा झषः}


\twolineshloka
{आदौ न कुरुते श्रेयः कुशलोऽस्मीति यः पुमान्}
{स हि संशयमाप्नोति यथा संप्रतिपत्तिमान्}


\twolineshloka
{अनागतविधाता च प्रत्युत्पन्नमतिश्च यः}
{द्वावेतौ सुखमेधेते दीर्घसूत्री विनश्यति}


\twolineshloka
{काष्ठा कला मुहूर्ताश्च दिनरात्र्यः क्षणा लवाः}
{मासाः पक्षाः षडृतवः कालः संवत्सराणि च}


\twolineshloka
{पृथिवी देश इत्युक्तः स च कालो न दृश्यते}
{अभिप्रेतार्थसिद्ध्यर्थं दूरतो न्यायतस्तथा}


\twolineshloka
{एतौ धर्मार्थशास्त्रेषु मोक्षशास्त्रेषु चर्षिभिः}
{प्रधानाविति निर्दिष्टौ कामे चाभिमतौ नृणाम्}


\twolineshloka
{परीक्ष्यकारी युक्तश्च स सम्यगुपपादयेत्}
{देशकालावभिप्रेतौ तोभ्यां फलमवाप्नुयात्}


\chapter{अध्यायः १३८}
\twolineshloka
{युधिष्ठिर उवाच}
{}


\twolineshloka
{सर्वत्र बुद्धिः कथिता श्रेष्ठा ते भरतर्षभ}
{अनागता तथोत्पन्ना दीर्घसूत्रा विनाशिनी}


\twolineshloka
{तदिच्छामि परां बुद्धिं श्रोतुं ते भरतर्षभ}
{यथा राजा न मुह्येत शत्रुभिः परिपीडितः}


\twolineshloka
{------याज्ञं सर्वशास्त्रविशारदम्}
{पृच्छ------- तन्मे व्याख्यातुमर्हसि}


\twolineshloka
{शत्रुभिबहु------था मुच्येत पार्थिवः}
{एतदिच्छाम्य------र्वमेव यथाविधि}


\twolineshloka
{विषमस्थं हि राजा----त्रवः परिपन्थिनः}
{बहवोऽप्येकमुद्धर्तुं यतन्ते पूर्वतापिताः}


\twolineshloka
{सर्वतः प्रार्थ्यमानेन दर्बलेन महाबलैः}
{एकेनैवासहायेन शक्यं स्थातुं भवेत्कथम्}


\twolineshloka
{कथं मित्रमरिं चापि विन्देत भरतर्षभ}
{चेष्टितव्यं कथं चात्र शत्रोर्मित्रस्य चान्तरे}


\twolineshloka
{अजातलक्षणे राजन्नमित्रे मित्रतां गते}
{कथं नु पुरुषः कुर्यात्कृत्वा किं वा सुखी भवेत्}


\twolineshloka
{विग्रहं केन वा कुर्यात्संधिं वा केन योजयेत्}
{कथं वा शत्रुमध्यस्थो वर्तेत बलवानपि}


\twolineshloka
{एतद्धै सर्वकृत्यानां परं कृत्यं नराधिप}
{नैतस्य कश्चिद्वक्ताऽस्ति श्रोता वाऽपि सुदुर्लभः}


\threelineshloka
{ऋते पितामहाद्भीष्मात्सत्यसंधाज्जितेन्द्रियात्}
{तदन्वीक्ष्य महाभाग सर्वमेतद्ब्रवीहि मे ॥भीष्म उवाच}
{}


\twolineshloka
{त्वद्युक्तोऽयमनुप्रश्नो युधिष्ठिर गुणोदयः}
{शृणु मे पुत्र कार्त्स्न्येन गुह्यमापत्सु भारत}


\twolineshloka
{अमित्रो मित्रतां याति मित्रं चापि प्रदुष्यति}
{सामर्थ्ययोगात्कार्याणामनित्या हि सदा गतिः}


\twolineshloka
{तस्माद्विश्वसितव्यं च विग्रहं च समाचरेत्}
{देशं कालं च विज्ञाय कार्याकार्यविनिश्चये}


\twolineshloka
{संधातव्यं बुधैर्नित्यं व्यवस्य च हितार्थिभिः}
{अमित्रैरपि संधेयं प्राणा रक्ष्या हि भारत}


\twolineshloka
{यो ह्यमित्रैर्नरैर्नित्यं न संदध्यादपण्डितः}
{न सोर्थं प्राप्नुयात्किंचित्फलान्यपि च भारत}


\twolineshloka
{यस्त्वमित्रेण संधत्ते मित्रेण च विरुध्यते}
{अर्थयुक्तिं समालोक्य सुमहद्विन्दते फलम्}


\twolineshloka
{अत्राप्युदाहरन्तीममितिहासं पुरातनम्}
{मार्जारस्य च संवादं न्यग्रोधे मूषिकस्य च}


\twolineshloka
{वने महति कस्मिंश्चिन्न्यग्रोधः सुमहानभूत्}
{लताजालपरिच्छन्नो नानाद्विजगणायुतः}


\twolineshloka
{स्कन्धवान्मेघसंकाशः शीतच्छायो मनोरमः}
{अरण्यमभितो जातस्तरुर्व्यालमृगायुतः}


\twolineshloka
{तस्य मूलमुपाश्रित्य कृत्वा शतमुखं बिलम्}
{वसति स्म महाप्राज्ञः पलितो नाम मूषिकः}


\twolineshloka
{शाखां तस्य समाश्रित्य वसति स्म सुखं तदा}
{लोमशो नाम मार्जारः सर्वसत्वावसादकः}


\twolineshloka
{तत्र त्वागत्य चण्डालो ह्यरण्यकृतकेतनः}
{युयोज यन्त्रमुन्माथं नित्यमस्तंगते रवौ}


\twolineshloka
{तत्र स्नायुमयान्पाशान्यथावत्संविधाय सः}
{गृहं गत्वा सुखं शेते प्रभातामेति शर्वरीम्}


\twolineshloka
{तत्र स्म नित्यं बध्यन्ते नक्तं बहुविधा मृगाः}
{कदाचिदत्र मार्जारः संप्रवृत्तो व्यबध्यत}


\twolineshloka
{तस्मिन्बद्धे महाप्राणे शत्रौ नित्याततायिनि}
{तं कालं पलितो ज्ञात्वा प्रचचार सुनिर्भयः}


\twolineshloka
{तेनानुचरता तस्मिन्वते विश्वस्तचारिणा}
{भक्ष्यं मृगयमाणे नचिराद्दृष्टमामिषम्}


\twolineshloka
{स तमुन्माथमारुह्य तदामिषमभक्षयत्}
{तस्योपरि सपत्नस्य बद्धस्य मनसा हसन्}


\twolineshloka
{आमिषे तु प्रसक्तः स कदाचितवलोकयन्}
{अपश्यदपरं घोरमात्मनो रिपुमागतम्}


\twolineshloka
{शरप्रसूनसंकाशं महीविवरशायिनम्}
{सकुलं हरिकं नाम चपलं ताम्रलोचनम्}


\twolineshloka
{तन मूषिकगन्धेन त्वरमाण उपागतम्}
{सक्ष्यार्थं लेलिहन्वक्रं भूमावूर्ध्वमुखः स्थितः}


\twolineshloka
{--ाखागतमरिं चान्यमपश्यत्कोटरालयम्}
{सूलकं चन्द्रकं नाम वक्रतुण्डं दुरासदम्}


\twolineshloka
{पतस्य विषयं तस्य नकुलोलूकयोस्तथा}
{यथास्यासीदियं चिन्ता तत्प्राप्तस्य महद्भयम्}


\twolineshloka
{पपद्यस्यां सुकष्टायां मरणे समुपस्थिते}
{मन्ताद्भय उत्पन्ने कथं कार्यं मनीषिणा}


\twolineshloka
{तथा सर्वतो रुद्धः सर्वत्र भयकर्शितः}
{भवद्भयसंत्रस्तश्चक्रे च परमां मतिम्}


\twolineshloka
{आपद्विनाशभूयिष्ठं शङ्कनीयं हि जीवितम्}
{मन्तात्संशयः सोऽयं तस्मादापदुपस्थिता}


\twolineshloka
{गतं हि सहसा भूमिं नकुलो मामवाप्नुयात्}
{उलूकश्चेह तिष्ठन्तं मार्जारः पाशसंक्षयात्}


\twolineshloka
{न त्वेवास्मद्विधः प्राज्ञः संमोहं गन्तुमर्हति}
{रिष्ये जीविते यत्नं यावदुच्छ्वासनिग्रहात्}


\twolineshloka
{न हि बुद्ध्याऽन्वितः प्राज्ञो नीतिशास्त्रविशारदः}
{निमज्जत्यापदं प्राप्य महतोऽर्थानवाप्य ह}


\twolineshloka
{न त्वन्यामिह मार्जाराद्गतिं पश्यामि सांप्रतम्}
{विषमस्थो ह्ययं शत्रुः कृत्यं चास्य महन्मया}


\twolineshloka
{जीवितार्थी कथं त्वद्य शत्रुभिः प्रार्थितस्त्रिभिः}
{प्राणहेतोरिमं मित्रं मार्जारं संश्रयामि वै}


\twolineshloka
{नीतिशास्त्रं समाश्रित्य हितमस्योपवर्णये}
{येनेमं शत्रुसंघातं मतिपूर्वेण वञ्चये}


\twolineshloka
{अयमत्यन्तशत्रुर्मे वैषम्यं परमं गतः}
{मूढो ग्राहयितुं स्वार्थं संगत्या यदि शक्यते}


% Check verse!
कदाजिद्व्यसनं प्राप्य संधिं कुर्यान्मया सह
\twolineshloka
{बलिना सन्निकृष्टस्य शत्रोरपि परिग्रहः}
{कार्य इत्याहुराचार्या विषमे जीवितार्थिना}


\twolineshloka
{श्रेष्ठो हि पण्डितः शत्रुर्न च मित्रमपण्डितः}
{अमित्रे खलु मार्जारे जीवितं संप्रतिष्ठितम्}


\twolineshloka
{ततोऽस्मै संप्रवक्ष्यामि हेतुमात्माभिरक्षणे}
{अपीजानीमयं शत्रुः संगत्या पण्डितो भवेत्}


% Check verse!
एवं विचिन्तयामास मूषिकः शत्रुचेष्टितम्
\twolineshloka
{ततोऽर्थगतितत्त्वज्ञः संधिविग्रहकालवित्}
{सान्त्वपूर्वमिदं वाक्यं मार्जारं मूषिकोऽब्रवीत्}


\twolineshloka
{सौहृदेनाभिभाषे त्वां कच्चिन्मार्जार जीवसे}
{जीवितं हि तवेच्छामि श्रेयः साधारणं हि नौ}


\twolineshloka
{न ते सौम्य भयं कार्यं जीविष्यसि यथा पुरा}
{अहं त्वामुद्धरिष्यामि प्राणाञ्जह्यां हि ते कृते}


\twolineshloka
{अस्ति कश्चिदुपायोऽत्र पुष्कल प्रतिभाति मे}
{येन शक्यस्त्वया मोक्षः प्राप्नुं श्रेयस्तथा मया}


\twolineshloka
{मयाऽप्युपायो दृष्टोऽयं विचार्य मतिमात्मनः}
{आत्मार्थं च त्वदर्थं च श्रेयः साधारणां हि नौ}


\twolineshloka
{इदं हि नकुलोलूकं पापबुद्ध्या हि संस्थितम्}
{न धर्षयति मार्जार तेन ते स्वस्ति सांप्रतम्}


\twolineshloka
{कूजंश्चपलनेत्रोऽयं कौशिको मां निरीक्षते}
{नगशाखाग्रगः पापस्तस्याहं भृशमुद्विजे}


\twolineshloka
{सतां साप्तपदं मैत्रं स सखा मेऽसि पण्डितः}
{साहाय्यकं करिष्यामि नास्ति ते प्राणतो भयम्}


\twolineshloka
{न हि शक्तोऽसि मार्जार पाशं छेत्तुं मया विना}
{अहं छेत्स्यामि पाशांस्ते यदि मां त्वं न हिंससि}


\twolineshloka
{त्वमाश्रितो द्रुमस्याग्रं मूलं त्वहमुपाश्रितः}
{चिरोषितावुभावावां वृक्षेऽस्मिन्विदितं च ते}


\twolineshloka
{यस्मिन्नाश्वासते कश्चिद्यश्च नाश्वसिति क्वचित्}
{न तौ धीराः प्रशंसन्ति नित्यमुद्विग्रमानसौ}


\twolineshloka
{तस्माद्विवर्धतां प्रीतिर्नित्यं संगतमस्तु नौ}
{कालातीतमिहार्थं हि न प्रशंसन्ति पण्डिताः}


\twolineshloka
{अर्थयुक्तिमिमां तत्र यथाभूतां निशामय}
{तव जीवितमिच्छामि त्वं ममेच्छसि जीवितम्}


\twolineshloka
{कश्चित्तरति काष्ठेन सुगम्भीरां महानदीम्}
{स तारयति तत्काष्ठं स च काष्ठेन तार्यते}


\twolineshloka
{ईदृशो नौ क्रियायोगो भविष्यति सुविस्तरः}
{अहं त्वां तारयिष्यामि मां च त्वं तारयिष्यसि}


\twolineshloka
{एवमुक्त्वा तु पलितस्तमर्थमुभयोर्हितम्}
{हेतुमद्ग्रहणीयं च कालापेक्षी व्यतिष्ठत}


\twolineshloka
{अथ सुव्याहृतं श्रुत्वा तस्य शत्रोर्विचक्षणः}
{हेतुमद््ग्रहणीयार्थं मार्जारो वाक्यमब्रवीत्}


\twolineshloka
{बुद्धिमान्वाक्यसंपन्नस्तद्वाक्यमनुवर्तयन्}
{स्वामवस्थां प्रतीक्ष्यैनं साम्नैव प्रत्यपूजयत्}


\twolineshloka
{ततस्तीक्ष्णाग्रदशनो मणिवैदूर्यलोचनः}
{मूषिकं मन्दमुद्वीक्ष्य मार्जारो लोमशोऽब्रबीत्}


\twolineshloka
{नन्दामि सौम्य भद्रं ते यो मां जीवितुमिच्छसि}
{श्रेयश्च यदि जानीषे क्रियतां मा विचारय}


\twolineshloka
{अहं हि भृशमापन्नस्त्वमापन्नतरो मया}
{द्वयोरापन्नयोः सन्धिः क्रियतां मा चिराय च}


\twolineshloka
{विधत्स्व प्राप्तकालं यत्कार्थं सिध्यतु चावयोः}
{मयि कृच्छ्राद्विनिर्मुक्ते न विनङ्क्ष्यति ते कृतम्}


\twolineshloka
{न्यस्तमानोस्मि भक्तोस्मि शिष्यस्त्वद्धितकृत्तथा}
{तथा निदेशवर्ती च भवन्तं शरणं गतः}


\twolineshloka
{इत्येवमुक्तः पलितो मार्जारं वशमागतम्}
{वाक्यं हितमुवाचेदमभिजातार्थमर्थवित्}


\twolineshloka
{उदारं यद्भवानाह नैतच्चित्रं भवद्विधे}
{विहितो यस्तु मार्गो मे हितार्थं शृणु तं मम}


\twolineshloka
{अहं त्वाऽनुप्रवेक्ष्यामि नकुलान्मे महद्भयम्}
{त्रायस्व मां मा वधीश्च शक्तोऽस्मि तव रक्षणे}


\twolineshloka
{उलूकाच्चैव मां रक्ष क्षुद्रः प्रार्थयते हि माम्}
{अहं छेत्स्यामि ते पाशान्सखे सत्येने ते शपे}


\twolineshloka
{तद्वचः संगतं श्रुत्वा लोमशो युक्तमर्थवत्}
{हर्षादुद्वीक्ष्य पलितं स्वागतेनाभ्यपूजयत्}


\twolineshloka
{तं संपूज्याथ पलितं मार्जारः सौहृदे स्थितम्}
{स विचिन्त्याब्रवीद्धीरः प्रीतस्त्वरित एव च}


\twolineshloka
{क्षिप्रमागच्छ भद्रं ते त्वं मे प्राणसमः सखा}
{तव प्राज्ञप्रसादाद्धि प्रियं प्राप्स्यामि जीवितम्}


\twolineshloka
{यद्यदेवंगतेनाद्य शक्यं कर्तुं मया तव}
{तदाज्ञापस्य कर्तास्मि सिद्धिरेवास्तु नौ सखे}


\twolineshloka
{अस्मात्ते संशयान्मुक्तः समित्रगणबान्धवः}
{सर्वकार्याणि कर्ताऽहं प्रियाणि च हितानि च}


\twolineshloka
{मुक्तश्च व्यसनादस्मात्सौम्याहमपि नाम ते}
{प्रीतिमुत्पादयेयं न प्रतिकर्तुं च शक्नुयाम्}


\threelineshloka
{प्रत्युपकुर्वन्बह्वपि न भाति पूर्वोपकारिणा तुल्यः}
{एकः करोति हि कृते निष्कारणमेव कुरुतेऽन्यः ॥भीष्म उवाच}
{}


\twolineshloka
{एवमाश्वासितो बिद्वान्मार्जारेण स मूषिकः}
{प्रविवेश सुविस्रब्धः सम्यगङ्गीचकार ह}


\twolineshloka
{ग्राहयित्वा तु तं स्वार्थं मार्जारं मूषिकस्तथा}
{मार्जारोरसि विस्रब्धः सुष्वाप पितृमातृवत्}


\twolineshloka
{निलीनं तस्य गात्रेषु मार्जारस्याथ मूषिकम्}
{दृष्ट्वा तौ नकुलोलूकौ निराशौ प्रत्यपद्यताम्}


\twolineshloka
{तथैव तौ सुसंत्रस्तौ दृढमागततन्द्रितौ}
{दृष्ट्वा तयोः परां प्रीतिं विस्मयं परमं गतौ}


\twolineshloka
{बलिनौ मतिमन्तौ च सुवृत्तौ चाप्युपासितौ}
{अशक्तौ तु नयात्तस्मात्संप्रधर्षयितुं बलात्}


\twolineshloka
{कार्यार्थं कृतसंधी तौ दृष्ट्वा मार्जारमूषिकौ}
{उलूकनकुलौ तूर्णं जग्मतुस्तौ स्वमालयम्}


\twolineshloka
{लीनः स तस्य गात्रेषु पलितो देशकालवित्}
{चिच्छेद पाशान्नृपते कालाकाङ्क्षी शनैः शनैः}


\twolineshloka
{अथ बन्धपरिक्लिष्टो मार्जारो वीक्ष्य मूषिकम्}
{छिन्दन्तं वै तदा पाशानत्वरन्तं त्वरान्वितः}


\twolineshloka
{तमत्वरन्तं पलितं पाशानां छेदने तदा}
{संचोदयितुमारेभे मार्जारो मूषिकं ततः}


\twolineshloka
{किं सौम्य नातित्वरसे किं कृतार्थोऽवमन्यसे}
{छिन्धि पाशानमित्रघ्न पुरा श्वपच एति सः}


\twolineshloka
{इत्युक्तस्त्वरताऽनेन मतिमान्पलितोऽब्रवीत्}
{मार्जारमकृतप्रज्ञं तथ्यमात्महितं वचः}


\twolineshloka
{तूष्णीं भव न ते सौम्य त्वरा कार्या न संभ्रमः}
{वयमेवात्र कालज्ञा न कालः परिहास्यते}


\twolineshloka
{अकाले कृत्यमारब्धं कर्तुर्नार्थाय कल्पते}
{तदेव काल आरब्धं महतेऽर्थाय कल्पते}


\twolineshloka
{अकाले विप्रमुक्तान्मे त्वत्त एव भयं भवेत्}
{तस्मात्कालं प्रतीक्षस्व किमिति त्वरसे सखे}


\twolineshloka
{यावत्पश्यामि चण्डालमायान्तं शस्त्रपाणिनम्}
{ततश्छेत्स्यामि ते पाशान्प्राप्ते साधारणे भये}


\twolineshloka
{तस्मिन्काले प्रमुक्तस्त्वं तरुमेवाधिरोक्ष्यसे}
{न हि ते जीवितादन्यत्किंचित्कृत्यं भविष्यति}


\twolineshloka
{तस्मिन्कालेऽपि च तता दिवाकीर्तिभयार्दितः}
{मम न ग्रहणे शक्तः पलायनपरायणः ॥'}


\twolineshloka
{ततो भवत्यपक्रान्ते त्रस्ते भीते च लुब्धकात्}
{अहं बिलं प्रवेक्ष्यामि भवाञ्शाखां गमिष्यति}


\twolineshloka
{एवमुक्तस्तु मार्जारो मूषिकेणात्मनो हितम्}
{वचनं वाक्यतत्त्वज्ञो जीवितार्थी महामतिः}


\twolineshloka
{अथात्मकृत्ये त्वरितः सम्यक्प्रार्थितमाचरन्}
{उवाच लोमशो वाक्यं मूषिकं चिरकारिणाम्}


% Check verse!
नह्येवं मित्रकार्याणि प्रीत्या कुर्वन्ति साधवः
\threelineshloka
{यथा त्वं मोक्षितः कृच्छ्रात्त्वरमाणेन वै मया}
{तथा हि त्वरमाणेन त्वया कार्यमिदं मम}
{यत्नं कुरु महाप्राज्ञ यथा स्वस्त्यावयोर्भवेत्}


\twolineshloka
{अथवा पूर्ववैरं त्वं स्मरन्कालं जिहीर्षसि}
{पश्य दुष्कृतकर्मंस्त्वं व्यक्तमायुःक्षयो मम}


\twolineshloka
{यदि किंचिन्मयाऽज्ञानात्पुरस्ताद्दुष्कृतं कृतम्}
{न तन्मनसि कर्तव्यं क्षामये त्वां प्रसीद मे}


\twolineshloka
{तमेवंवादिनं प्राज्ञं शास्त्रविद्बुद्धिसत्तमः}
{उवाचेदं वचः श्रेष्ठं मार्जारं मूषिकस्तदा}


\twolineshloka
{श्रुतं मे तव मार्जार स्वमर्थं परिगृह्णतः}
{ममापि त्वं विजानासि स्वमर्थं परिगृह्णतः}


\twolineshloka
{यन्मित्रं भीतवत्साध्यं यस्मिन्मित्रे भयं हितम्}
{आरक्षितं ततः कार्यं पाणिः सर्पमुखादिव}


\twolineshloka
{कृत्वा बलवता संधिमात्मानं यो न रक्षति}
{अपथ्यमिव तद्भुक्तं तस्यार्थाय कल्पते}


\twolineshloka
{कश्चित्कस्यचिन्मित्रं न कश्चित्कस्यचिद्रिपुः}
{अर्थतस्तु निबध्यन्ते मित्राणि रिपवस्तथा}


\threelineshloka
{अर्थैरर्था निबध्यन्ते गजैरिव महागजाः}
{न च कश्चित्कृते कार्ये कर्तारं समवेक्षते}
{तस्मात्सर्वाणि कार्याणि सावशेषाणि कारयेत्}


\twolineshloka
{तस्मिन्कालेऽपि च भवान्दिवाकीर्तिभयार्दितः}
{मम न ग्रहणे शक्तः पलायपरायणः}


\twolineshloka
{छिन्नं तु तन्तुबाहुल्यं तन्तुरेकोऽवशेषितः}
{छेत्स्याम्यहं तमप्याशु निर्वृतो भव लोमश}


\twolineshloka
{तयोः संवदतोरेवं तथैवापन्नयोर्द्वयोः}
{क्षयं जगाम सा रात्रिर्लोमशं त्वागमद्भयम्}


\twolineshloka
{ततः प्रभातसमये विकटः कृष्णपिङ्गलः}
{स्थूलस्फिग्विकृतो रूक्षः श्वयूथपरिवारितः}


\twolineshloka
{शङ्कुकर्णो महावक्रः खनित्री घोरदर्शनः}
{परिघो नाम चण्डालः शस्त्रपाणिरदृश्यत}


\twolineshloka
{तं दृष्ट्वा यमदूताभं मार्जारस्त्रस्तचेतनः}
{उवाच पलितं भीतः किमिदानीं करिष्यसि}


\twolineshloka
{तथैव च सुसंत्रस्तौ तं दृष्ट्वा घोरसंकुलम्}
{क्षणेन नकुलोलूकौ नैराश्यमुपजग्मतुः}


\twolineshloka
{बलिनौ मतिमन्तौ च संघातं चाप्युपागतौ}
{अशक्तौ सुनयात्तस्मात्संप्रधर्षयितुं बलात्}


\twolineshloka
{कार्यार्थे कृतसंधी तौ दृष्ट्वा मार्जारमूषिकौ}
{उलूकनकुलौ तूर्णं जग्मतुः स्वंस्वमालयम्}


\twolineshloka
{ततश्चिच्छेद तं तन्तुं मार्जारस्य स मूषिकः}
{विप्रमुक्तोऽथ मार्जारस्तमेवाभ्यपतद्दुमम्}


\twolineshloka
{स तस्मात्संभ्रमान्मुक्तो मुक्तो घोरेण सत्रुणा}
{बिलं विवेश पलितः शाखां लेभे स लोमशः}


\threelineshloka
{उन्माथमप्युपादाय चण्डालो वीक्ष्य सर्वशः}
{विहताशः क्षणेनैव तस्माद्देशादपाक्रमत्}
{जगाम स स्वभवनं चण्डालो भरतर्षभ}


\twolineshloka
{ततस्तस्माद्भान्मुक्तो दुर्लभं प्राप्य जीवितम्}
{बिलस्थं पादपाग्रस्थः पलितं लोमशोऽब्रवीत्}


\twolineshloka
{अकृत्वा संविदं कांचित्सहसा त्वमपस्रुतः}
{कृतज्ञः कृतकल्याणः कच्चिन्मां नाभिशङ्कसे}


\twolineshloka
{गत्वा च मम विश्वासं दत्त्वा च मम जीवितम्}
{मित्रोपभोगसमये किं हि मां नोपसर्पसि}


\twolineshloka
{कृत्वा हि पूर्वं मित्राणि यः पश्चान्नानुतिष्ठति}
{न स मित्राणि लभते कृच्छ्रात्स्वापत्सु दुर्मतिः}


\twolineshloka
{सत्कृतोऽहं त्वया मित्र सामर्थ्यादात्मानः सखे}
{स मां मित्रत्वमापन्नमुपभोक्तुं त्वमर्हसि}


\twolineshloka
{यानि मे सन्ति मित्राणि ये च मे सन्ति बान्धवाः}
{सर्वे त्वां पूजयिष्यन्ति शिष्या गुरुमिव प्रियम्}


\twolineshloka
{अहं च पूजयिष्ये त्वां समित्रगणबान्धवम्}
{जीवितस्य प्रदातारं कृतज्ञः को न पूजयेत्}


\twolineshloka
{ईश्वरो मे भवानस्तु शरीरस्य गृहस्य च}
{अर्थानां चैव सर्वेषामनुशास्ता च मे भव}


\twolineshloka
{अमात्यो मे भव प्राज्ञ पितेवेह प्रशाधि माम्}
{न तेऽस्ति भयमस्मत्तो जीवितेनात्मनः शपे}


\twolineshloka
{बुद्ध्या त्वमुशना साक्षाद्बलेनाधिकृता वयम्}
{त्वं मन्त्रबलयुक्तो हि दद्या विजयमेव मे}


\twolineshloka
{एवमुक्तः परं सान्त्वं मार्जारेण स मूषिकः}
{उवाच परमार्थज्ञः श्लक्ष्णमात्महितं वचः}


\twolineshloka
{यद्भवानाह तत्सर्वं मया ते लोमश श्रुतम्}
{ममापि तावद्ब्रुवतः शृणु यत्प्रतिभाति मे}


\twolineshloka
{वेदितव्यानि मित्राणि बोद्धव्याश्चापि शत्रवः}
{एतत्सुसूक्ष्मं लोकेऽस्मिन्दृश्यते प्राज्ञसंमतैः}


\twolineshloka
{शत्रुरूपा हि सुहृदो मित्ररूपाश्च शत्रवः}
{सान्त्वितास्ते न बुध्यन्ते रागलोभवशं गताः}


\twolineshloka
{येषां सौम्यानि मित्राणि क्रोधनाश्चैव शत्रवः}
{सान्त्वितास्ते न बुध्यन्ते रागलोभवशंगताः}


\twolineshloka
{नास्ति जात्या रिपुर्नाम मित्रं नाम न विद्यते}
{सामर्थ्ययोगाज्जायन्ते मित्राणि रिपवस्तथा}


\twolineshloka
{यो यस्मिञ्जीवति स्वार्थे पश्येत्पीडां न जीवति}
{स तस्य मित्रं तावत्स्यःद्यावन्न स्याद्विपर्ययः}


\twolineshloka
{नास्ति मैत्री स्थिरा नाम न च ध्रुवमसौहृदम्}
{अर्थयुक्त्याऽनुजायन्ते मित्राणि रिपवस्तथा}


\twolineshloka
{मित्रं च शत्रुतामेति कस्मिंश्चित्कालपर्यये}
{शत्रुश्च मित्रतामेति स्वार्थो हि बलवत्तरः}


\twolineshloka
{यो विश्वसिति मित्रेषु न विश्वसिति शत्रुषु}
{अर्थयुक्तिमविज्ञाय चलितं तस्य जीवितम्}


\threelineshloka
{मित्रे वा यदि वा शत्रौ तस्यापि चलिता मतिः}
{न विश्वसेदविश्वस्ते विश्वस्ते नातिविश्वसेत्}
{विश्वासाद्भयमुत्पन्नमपि मूलानि कृन्तति}


\twolineshloka
{अर्थयुक्त्या हि जायन्ते पिता माता सुतस्तथा}
{मातुला भागिनेयाश्च तथा संबन्धिबान्धवाः}


\twolineshloka
{पुत्रं हि मातापितरौ त्यजतः पतितं प्रियम्}
{लोको रक्षति चात्मानं पश्य स्वार्थस्य सारताम्}


\twolineshloka
{सामान्या निष्कृतिः प्राज्ञ यो मोक्षात्समन्तरम्}
{कृत्यं मृगयते कर्तुं सुखोपायमसंशयम्}


\twolineshloka
{अस्मिन्निलय एवं त्वं न्यग्रोधादवतारितः}
{पूर्वं निविष्टमुन्माथं चपलन्वान्न बुद्धवान्}


\twolineshloka
{आत्मनश्चपलो नास्ति कुतोऽन्येषां भविष्यति}
{तस्मात्सर्वाणि कार्याणि चपलो हन्त्यसंशयम्}


\twolineshloka
{ब्रवीषि मधुरं यच्च प्रियो मेऽद्य भवानिति}
{तन्मिथ्याकारणं सर्वं विस्तरेणापि मे शृणु}


\twolineshloka
{कारणात्प्रियतामेति द्वेष्यो भवति कारणात्}
{अर्थार्थी जीवलोकोऽयं न कश्चित्कस्यचित्प्रियः}


\twolineshloka
{सख्यं सोदर्ययोर्भ्रात्रोदर्पंत्योर्वा परस्परम्}
{कस्यचिन्नाभिजानामि प्रीतिं निष्कारणामिह}


\twolineshloka
{यद्यपि भ्रातरः क्रुद्धा भार्या वा कारणान्तरे}
{स्वभावतस्ते प्रीयन्ते नेतरः प्राकृतो जनः}


\twolineshloka
{प्रियो भवति दानेन प्रियवादेन चापरः}
{मन्त्रहोमजपैरन्यः कार्यार्थे प्रीयते जनः}


\twolineshloka
{उत्पन्ना कारणात्प्रीतिरासीन्नौ कारणान्तरे}
{प्रध्वस्ते कारणस्थाने सा प्रीतिर्नाभिवर्तते}


\twolineshloka
{किंनु तत्कारणं मन्ये येनाहं भवतः प्रियः}
{अन्यत्राभ्यवहारार्थात्तत्रापि च बुधा वयम्}


\threelineshloka
{कालो हेतुं विकुरुते स्वार्थस्तमनुवर्तते}
{स्वार्धं प्राज्ञोऽभिजानाति प्राज्ञं लोकोऽनुवर्तते}
{न त्वीदृशं त्वया वाच्यं विद्यते स्वार्थपण्डितः}


\twolineshloka
{न कालो हि समर्थस्य स्नेहहेतुरयं तव}
{तस्मान्नाहं चले स्वार्थात्सुस्थितः संधिविग्रहे}


\threelineshloka
{अभ्राणामिव रुपाणि विकुर्वन्ति पदेपदे}
{अद्यैव हि रिपुर्भूत्वा पुनरद्यैव मे सुहृत्}
{पुनश्च रिपुरद्यैव युक्तीनां पश्य चापलम्}


\twolineshloka
{आसीन्मैत्री तु तावन्नौ यावद्धेतुरभूत्पुरा}
{सागता सह तेनैव कालयुक्तेन हेतुना}


\twolineshloka
{त्वं हि मेऽत्यन्ततः शत्रुः सामर्थ्यान्मित्रतां गतः}
{तत्कृत्यमभिनिर्वर्त्य प्रकृतिः शत्रुतां गता}


\twolineshloka
{सोऽहमेवं प्रणीतानि ज्ञात्वा शास्त्राणि तत्त्वतः}
{प्राविशेयं कथं पाशं त्वत्कृते तद्ब्रवीहि मे}


\twolineshloka
{त्वद्वीर्येण विमुक्तोऽहं मद्वीर्येण तथा भवान्}
{अन्योन्यानुग्रहे वृत्ते नास्ति भूयः समागमः}


\twolineshloka
{त्वं हि सौम्य कृतार्थोऽद्य निवृत्तार्थास्तथा वयम्}
{न तेऽस्त्यद्य मया कृत्यं किंचिदन्यत्र भक्षणात्}


\twolineshloka
{अहमन्नं भवान्भोक्ता दुर्बलोऽहं भवान्बली}
{नावयोर्विद्यते संधिर्वियुक्ते विषमे बले}


\twolineshloka
{स मन्येऽहं तव प्रज्ञां यन्मोक्षात्प्रत्यनन्तरम्}
{भक्ष्यं मृगयते नूनं सुखोपायमसंशयम्}


\twolineshloka
{--र्थी ह्येव सुव्यक्तो विमुक्तः प्रसृतः क्षुधा}
{शास्त्रजां मतिमास्थाय प्रातराशमिहेच्छसि}


\twolineshloka
{जानामि क्षुधितं च त्वामाहारसमयश्च ते}
{स त्वं मामभिसंधाय भक्ष्यं मृगयसे पुनः}


\twolineshloka
{किंचात्र पुत्रदारार्थं यद्वाणीं सृजसे मयि}
{शुश्रूषां यतसे कर्तुं सखे मम तत्क्षमम्}


\twolineshloka
{त्वया मां सहितं दृष्ट्वा प्रिया भार्या सुताश्च ये}
{कस्मात्ते मां न खादेयुः स्पृष्टवा प्रणयिनि त्वयि}


\twolineshloka
{नाहं त्वया समेष्यामि वृत्ते हेतुसमागमे}
{शिवं ध्यायस्व मेऽत्रस्थः सुकृतं स्मरसे यदि}


\twolineshloka
{शत्रोरन्नाद्यभूतः सन्क्लिष्टस्य क्षुधितस्य च}
{भक्ष्यं मृगयमाणस्य कः प्राज्ञो विषयं व्रजेत्}


\twolineshloka
{स्वस्ति तेऽस्तु गमिष्यामि दूरादपि तवोद्विजे}
{[विश्वस्तं वा प्रमत्तं वा एतदेव कृतं भवेत् ॥]}


\twolineshloka
{नाहं त्वया समेष्यामि निवृत्तो भव लोमश}
{बलवत्सन्निकर्षो हि न कदाचित्प्रशस्यते}


\twolineshloka
{यदि त्वं सुकृतं वेत्सि तत्सख्यमनुसारय}
{प्रशान्तादपि हि प्राज्ञाद्भेतव्यं बलिनः सदा}


\twolineshloka
{यदि त्वर्थेन ते कार्यं ब्रूहि किं करवाणि ते}
{कामं सर्वं प्रदास्यामि न त्वात्मानं कथंचन}


\twolineshloka
{आत्मार्थे संततिस्त्याज्या राज्यं रत्नं धनानि च}
{अपि सर्वस्वमुत्सृज्य रक्षेदात्मानमात्मवान्}


\twolineshloka
{ऐश्वर्यधनरत्नानां प्रत्यमित्रेऽपि वर्तताम्}
{दृष्टा हि पुनरावृत्तिर्जीवतामिति नः श्रुतम्}


\twolineshloka
{न त्वात्मनः संप्रदानं धनरत्नवदिष्यते}
{आत्मा हि सर्वदा रक्ष्यो दारैरपि धनैरपि}


\twolineshloka
{आत्मरक्षणतन्त्राणां सुपरीक्षितकारिणाम्}
{आपदो नोपपद्यन्ते पुरुषाणां स्वदोषजाः}


\twolineshloka
{शत्रुं सम्यगविज्ञातो विप्रियो ह्यबलीयसा}
{`शङ्कनीयः स सर्वत्र प्रियमप्याचरन्सदा}


\twolineshloka
{कुलजानां सुमित्राणां धार्मिकाणां महात्मनाम्}
{'न तेषां चाल्यते बुद्धिः शास्त्रार्थकृतिश्चया}


\twolineshloka
{इत्यभिव्यक्तमेवासौ पलितेनापहासितः}
{मार्जारो व्रीडितो भूत्वा मूषिकं वाक्यमब्रवीत्}


\twolineshloka
{सत्यं शपे त्वयाऽहं वै मित्रद्रोहो विगर्हितः}
{संमन्येऽहं तव प्रज्ञां यस्त्वं मम हिते रतः}


\twolineshloka
{उक्तवानर्थतत्त्वेन मया संभिन्नदर्शनः}
{न तु मामन्यथा साधो त्वं ग्रहीतुमिहार्हसि}


\twolineshloka
{प्राणप्रदानजं त्वत्तो मयि सौहृदमागतम्}
{धर्मज्ञोऽस्मि गुणज्ञोऽस्मि कृतज्ञोस्मि विशेषतः}


\twolineshloka
{मित्रेषु वत्सलश्चास्मि त्वद्भक्तश्च विशेषतः}
{त्वं मामेवंगते साधो न वाचयितुमर्हसि}


\threelineshloka
{त्वया हि वाच्यमानोऽहं जह्यां प्राणान्सबान्धवः}
{धिक्शब्दो हि बुधैर्दृष्टो मद्विधेषु मनस्विषु}
{पतनं धर्मतत्त्वज्ञ न मे शङ्कितुमर्हसि}


\twolineshloka
{इति संस्तूयमानोऽपि मार्जारेण स मूषिकः}
{मनसा भावगम्भीरं मार्जारमिदमब्रवीत्}


\twolineshloka
{साधुर्भवान्कृतार्थोऽस्मि प्रिये च न च विश्वसे}
{संस्तवैर्वा धनौघैर्वा नाहं शक्यः पुनस्त्वया}


\twolineshloka
{न ह्यमित्रवशं यान्ति प्राज्ञा निष्कारणं सखे}
{अस्मिन्नर्थे च गाथे द्वे निबोधोशनसा कृते}


\twolineshloka
{शत्रुसाधारणे कृत्ये कृत्वा सन्धिं बलीयसा}
{समाहितश्चरेद्बुद्ध्या कृतार्थश्च न विश्वसेत्}


\twolineshloka
{न विश्वसेदवश्वस्ते विश्वस्ते नातिविश्चसेत्}
{नित्यं विश्वासयेदन्यान्परेषां तु न विश्वसेत्}


\twolineshloka
{तस्मात्सर्वास्ववस्थासु रक्षेज्जीवितमात्मनः}
{द्रव्याणि संततिश्चैव सर्वं भवति जीवताम्}


\twolineshloka
{संक्षेपो नीतिशास्त्राणामविश्वासः परो मतः}
{नृषु तस्मादविश्वासः पुष्कलं हितमात्मनः}


\twolineshloka
{वध्यन्ते न ह्यविश्वस्ताः शत्रुर्भिर्दुर्बला अपि}
{विश्वस्तास्तेषु वध्यन्ते बलवन्तोऽपि शत्रुभिः}


\twolineshloka
{त्वद्विधेभ्यो मया ह्यात्मा रक्ष्यो मार्जार सर्वदा}
{रक्ष त्वमपि चात्मानं चण्डालाज्जातिकिल्बिषात्}


\twolineshloka
{स तस्य ब्रुवतस्त्वेवं संत्रासाज्जातसाध्वसः}
{कथां हित्वा जवेनाशु मार्जारः प्रययौ ततः}


\twolineshloka
{ततः शास्त्रार्थतत्त्वज्ञो बुद्धिसामर्थ्यमात्मनः}
{विश्राव्य पलितः प्राज्ञो बिलमन्यज्जगाम ह}


\twolineshloka
{एवं प्रज्ञावता बुद्ध्या दुर्बलेन महाबलाः}
{एकेन बहवोऽमित्राः पलितेनाभिसंधिताः}


\twolineshloka
{अरिणापि समर्थेन सन्धिं कुर्वीत पण्डितः}
{मूषिकश्च बिडालश्च मुक्तावन्योन्यसंश्रयात्}


\twolineshloka
{इत्येवं क्षत्रधर्मस्य मया मार्गो निदर्शितः}
{विस्तरेण महाराज संक्षेपमपि मे शृणु}


\twolineshloka
{अन्योन्यं कृतवैरौ तु चक्रतुः प्रीतिमुत्तमाम्}
{अन्योन्यमभिसंधातुं संबभूव तयोर्मतिः}


\twolineshloka
{तत्र प्राज्ञोऽभिसंधत्ते सम्यग्बुद्धिबलाश्रयात्}
{अभिसंधीयते प्राज्ञः प्रमादादपि वा बुधैः}


\twolineshloka
{तस्मादभीतवद्भीतो विश्वस्तवदविश्वसेत्}
{न ह्यप्रमत्तश्चलति चलितोऽवा न नश्यति}


\twolineshloka
{काले हि रिपुणा संधिः काले मित्रेण विग्रहः}
{कार्य इत्येव तत्वज्ञाः प्राहुर्नित्यं नराधिप}


\twolineshloka
{एतज्ज्ञात्वा महाराज शास्त्रार्थमभिगम्य च}
{अभियुक्तोऽप्रमत्तश्च प्राग्भयाद्भीतवच्चरेत्}


\twolineshloka
{भीतवत्संहितः कार्यः प्रतिसंधिस्तथैव च}
{भयादुत्पद्यते बुद्धिरप्रमत्ताभियोगजा}


\twolineshloka
{न भयं जायते राजन्भीतस्यानागते भये}
{अभीतस्य च विस्रम्भात्सुमहज्जायते भयम्}


\twolineshloka
{न भीरुरिति चात्यन्तं मन्त्रो देयः कथंचन}
{अविज्ञानाद्धि विज्ञाने गच्छेदास्पददर्शनाम्}


\twolineshloka
{तस्मादभीतवद्भीतो विश्वस्तवदविश्वसन्}
{कार्याणां गुरुतां ज्ञात्वा नादृतं किंचिदाचरेत्}


\twolineshloka
{एवमेतन्मया प्रोक्तमितिहासं युधिष्ठिर}
{श्रुत्वा त्वं सुहृदां मध्ये यथावत्समुदाचर}


\twolineshloka
{उपलभ्य मतिं चाग्र्यामरिमित्रान्तरं तथा}
{संधिविग्रहकालौ च मोक्षोपायं तथाऽऽपदि}


\twolineshloka
{शत्रुसाधारणे कृत्ये कृत्वा सन्धिं बलीयसा}
{समागतश्चरेद्बुद्ध्या कृतार्थो न च विश्वसेत्}


\twolineshloka
{अविरुद्धां त्रिवर्गेण नीतिमेतां महीपते}
{अभ्युत्तिष्ठ श्रुतात्तस्माद्भूयः संरञ्जयन्प्रजाः}


\twolineshloka
{ब्राह्मणैश्चापि ते सार्धं यात्रा भवतु पाण्डव}
{ब्राह्मणाद्धि परं श्रेयो दिवि चेह च भारत}


\twolineshloka
{एते धर्मस्य वेत्तारः कृतज्ञाः सततं प्रभो}
{पूजिताः शुभकर्तारः पूजयैनाञ्जाधिप}


\twolineshloka
{राज्यं श्रेयः परं राजन्यशश्च महदाप्स्यसे}
{कुलस्य संततिं चैव यथान्यायं यथाक्रमम्}


\twolineshloka
{श्रुतं च ते भारत संधिविग्रहंविभावितं बुद्धिविशेषकारितम्}
{तथा त्ववेक्ष्य क्षितिपेन सर्वदानिषेवितव्यं नृप शत्रुमण्डलम्}


\chapter{अध्यायः १३९}
\twolineshloka
{युधिष्ठिर उवाच}
{}


\twolineshloka
{उक्तो मन्त्रो महाबाहो विश्वासो नास्ति शत्रुषु}
{कथं हि राजा वर्तेत यदि सर्वत्र नाश्वसेत्}


\twolineshloka
{विश्वासाद्धि परं राजन्राज्ञामुत्पद्यते भयम्}
{कथं हि नाश्वसन्राजा शत्रूञ्जयति पार्थिवः}


\threelineshloka
{एतन्मे संशयं छिन्धि मनो मे संप्रमुह्यति}
{अविश्वासे कथामेतामुपाश्रित्य पितामह ॥भीष्म उवाच}
{}


\twolineshloka
{शृणुष्व राजन्यो वृत्तो ब्रह्मदत्तनिवेशने}
{पूजन्या सह संवादो ब्रह्मदत्तस्य भूपतेः}


\threelineshloka
{काम्पिल्ये ब्रह्मदत्तस्य त्वन्तः पुरनिवासिनी}
{पूजी नाम शकुनिर्दीर्घकालं सहोपिता}
{}


\twolineshloka
{रुदज्ञा सर्वभूतानां यथा वै जीवजीवकः}
{सर्वज्ञा सर्वतत्त्वज्ञा तिर्यग्योनिं गताऽपि स}


\twolineshloka
{अभिप्रजाता सा तत्र पुत्रमेकं सुवर्चसम्}
{समकालं च राज्ञोऽपि देव्यां पुत्रो व्यजायत}


\twolineshloka
{तयोरर्थे कृतज्ञा तु खेचरी पूजनी सदा}
{समद्रतीरं सा गत्वा आजहार फलद्वयम्}


\twolineshloka
{अष्ट्यर्थं च स्वपुत्रस्य राजपुत्रस्य चैव ह}
{लमेकं सुतायादाद्राजपुत्राय चापरम्}


\twolineshloka
{---मृतास्यादसदृशं बलतेजोभिवर्धनम्}
{[आदायादाय सैवाशु तयोः प्रादात्पुनः पुनः ॥]}


\twolineshloka
{ततोऽगच्छत्परां वृद्धिं राजपुत्रः फलाशनात्}
{ततः स धात्र्या कक्षेण उह्यमानो नृपात्मजः}


\twolineshloka
{ददर्श तं पक्षिसुतं बाल्यादागत्य बालकः}
{ततो बाल्याच्च यत्नेन तेनाक्रीडत पक्षिणा}


\twolineshloka
{शून्ये च तमुपादाय पक्षिणं समजातकम्}
{हत्वा ततः स राजेन्द्र धात्र्या हस्तमुपागतः}


\twolineshloka
{अथ सा पूजनी राजन्नागमत्फलहारिणी}
{अपश्यन्निहतं पुत्रं तेन बालेन भूतले}


\twolineshloka
{बाष्पपूर्णमुखी दीना दृष्ट्वा तं पतितं सुतम्}
{पूजनी दुःखसंतप्ता रुदन्ती वाक्यमब्रवीत्}


\twolineshloka
{क्षत्रिये संगतं नास्ति न प्रीतिर्न च सौहृदम्}
{कारणे सान्त्वयन्त्येते कृतार्थाः संत्यजन्ति च}


\twolineshloka
{क्षत्रियेषु न विश्वासः कार्यः सर्वापकारिषु}
{अपकृत्यापि सततं सान्त्वयन्ति निरर्थकम्}


\twolineshloka
{इयमस्य करोम्यद्य सदृशीं वैरयातनाम्}
{कृतघ्नस्य नृशंसस्य भृशं विश्वासघातिनः}


\twolineshloka
{सहसंजातवृद्धस्य तथैव सहभोजिनः}
{शरण्यस्य वधश्चैव त्रिविधं तस्य किल्विषम्}


\twolineshloka
{इत्युक्त्वा चरणाभ्यां तु नेत्रे नृपसुतस्य सा}
{हृत्वा स्वस्था तत इदं पूजनी वाक्यमब्रवीत्}


\twolineshloka
{इच्छयेह कृतं पापं सद्य एवोपसर्पति}
{कृतं प्रतिकृतं येषां न नश्यति शुभाशुभम्}


\twolineshloka
{पापं कर्म कृतं किंचिद्यदि तस्मिन्न दृश्यते}
{निपात्यतेऽस्य पुत्रेषु पौत्रेष्वपि च नप्नृषु}


\twolineshloka
{ब्रह्मदत्तः सुतं दृष्ट्वा पूजन्या हृतलोचनम्}
{कृतप्रतिकृतं मत्वा पूजनीमिदमब्रवीत्}


\threelineshloka
{अस्ति वै कृतमस्माभिरस्ति प्रतिकृतं त्वया}
{उभयं तत्समीभूतं वस पूजनि मा गमः ॥पूजन्युवाच}
{}


\twolineshloka
{सकृत्कृतापराधस्य तत्रैव परिलम्बतः}
{न तद्बुधाः प्रशंसन्ति श्रेयस्तत्रापसर्पणम्}


\twolineshloka
{सान्त्वे प्रयुक्ते विवृते वैरे चैव न विश्वसेत्}
{क्षिप्रं स हन्यते मूढो न हि वैरं प्रशाम्यति}


\twolineshloka
{अन्योन्यकृतवैराणां पुत्रपौत्रं नियच्छति}
{पुत्रपौत्रविनाशे च परलोकं नियच्छति}


\twolineshloka
{सर्वेषां कृतवैराणामविश्वासः सुखावहः}
{एकान्ततो न विश्वासः कार्यो विश्वासघातके}


\threelineshloka
{न विश्वसेदविश्वस्ते विश्वस्ते नातिविश्वसेत्}
{विश्वासाद्भयमुत्पन्नमपि मूलं निकृन्तति}
{कामं विश्वासयेदन्यान्परेषां च न विश्वसेत्}


\twolineshloka
{माता पिता बान्धवानां वरिष्ठौभार्या क्षेत्रं बीजमात्रं तु पुत्रः}
{भ्राता शत्रुः क्लिन्नपाणिर्वयस्यआत्मा ह्येकः सुखदुःखस्य भोक्ता}


\twolineshloka
{अन्योन्यकृतवैराणां न संधिरुपपद्यते}
{स च हेतुरतिक्रान्तो यदर्थमहमावसम्}


\twolineshloka
{पूजितस्यार्थमानाभ्यां सान्त्वं पूर्वापकारिणः}
{हृदयं भवत्यविश्वस्तं कर्म त्रासयते बलात्}


\twolineshloka
{पूर्वं संमानना यत्र पश्चाच्चैव विमानना}
{जह्यात्स सत्ववान्वासं संमानितविमानितः}


\threelineshloka
{उषिताऽस्मि तवागारे दीर्घकालमहिंसिता}
{तदिदं वैरमुत्पन्नं सुखमास्ख व्रजाम्यहम् ॥ब्रह्मदत्त उवाच}
{}


\threelineshloka
{यः कृते प्रतिकुर्याद्वै न स तत्रापराध्नुयात्}
{अनृणस्तेन भवति वस पूजनि मागमः ॥पूजन्युवाच}
{}


\threelineshloka
{न कृतस्य तु कर्तुश्च सख्यं संधीयते पुनः}
{हृदयं तत्र जानाति कर्तुश्चैव कृतस्य च ॥ब्रह्मदत्त उवाच}
{}


\threelineshloka
{कृतस्य चैव कर्तुश्च सख्यं संधीयते पुनः}
{वैरस्योपशमो दृष्टः पापं नोपाश्नुते पुनः ॥पूजन्युवाच}
{}


\twolineshloka
{नास्ति वैरमतिक्रान्तं सान्त्वितोऽस्मीति नाश्वसेत्}
{विश्वासाद्बध्यते लोकस्तस्माच्छ्रेयोप्यदर्शनम्}


\threelineshloka
{तरसा ये न शक्यन्ते शस्त्रैः सुनिसितैरपि}
{साम्ना तेऽपि निगृह्यते गजा इव करेणुभिः ॥ब्रह्मदत्त उवाच}
{}


\twolineshloka
{संवासाज्जायते स्नेहो जीवितान्तकरेष्वपि}
{अन्योन्यस्य हि विश्वासः श्वानश्वपचयोरिव}


\threelineshloka
{अन्योन्यकृतवैराणां संवासान्मृदुतां गतम्}
{नैव तिष्ठति तद्वैरं पुष्करस्थमिवोदकम् ॥पूजन्युवाच}
{}


\twolineshloka
{वैरं पञ्चसमुत्थानं तच्च बुध्यन्ति पण्डिताः}
{स्त्रीकृतं वास्तुजं वाग्जं स्वसपत्नापराधजम्}


\twolineshloka
{तत्र दाता न हन्तव्यः क्षत्रियेण विशेषतः}
{प्रकाशं वाऽप्रकाशं वा बुद्ध्वा दोषबलाबलम्}


\twolineshloka
{कृतवैरे न विश्वासः कार्यस्त्विह सुहृद्यपि}
{प्रच्छन्नं तिष्ठते वैरं गूढोऽग्निरिव दारुषु}


\twolineshloka
{न वित्तेन न पारुष्यैर्न च सान्त्वेन च श्रुतैः}
{वैराग्निः शाम्यते राजन्निमग्नोऽग्निरिवार्णवे}


\twolineshloka
{न हि वैराग्निरुद्धूतः कर्म चाप्यपराधजम्}
{शाम्यत्यदग्ध्वा नृपते विना ह्येकतरक्षयात्}


\twolineshloka
{सत्कृतस्यार्थामनाभ्यां तत्र पूर्वापकारिणः}
{नैव शान्तिर्न विश्वासः कर्मणा जायते बलात्}


\threelineshloka
{नैवापकारे कस्मिंश्चिदहं त्वयि तथा भवान्}
{उषितावाऽपि चक्रितं नेदानीं विश्वसाम्यहम् ॥ब्रह्मदत्त उवाच}
{}


% Check verse!
कालेन क्रियते कार्यं तथैव विविधाः क्रियाःकालेनैव प्रवर्तन्ते कः कस्येत्यपराध्यति
\twolineshloka
{तुल्यं चोभे प्रवर्तेते मरणं जन्म चैव हि}
{कार्यते चैव कालेन तन्निमित्तं न जीवति}


\twolineshloka
{बध्यन्ते युगपत्केचिदेकैकं चापरे तथा}
{कालो दहति भूतानि संप्राप्तोऽग्निरिवेन्धनम्}


\twolineshloka
{नाहं प्रमाणं नैव त्वभन्योन्यं कारणं शुभे}
{कालो नित्यमुपादत्ते सुखं दुःखं च देहिनाम्}


\threelineshloka
{एवं वसेह सन्नेहा यथाकाममहिंसिता}
{यत्कृतं तत्तु मे क्षान्तं त्वं च वै क्षम पूजनि ॥पूजन्युवाच}
{}


\twolineshloka
{यदि कालः प्रमाणं ते न वैरं कस्यचिद्भवेत्}
{कस्मादपचितिं यान्ति बान्धवा बान्धवे हते}


\twolineshloka
{कस्माद्देवासुराः सर्वे अन्योन्यमभिजघ्निरे}
{यदि कालेन निर्याणं सुखदुःखे भवाभवौ}


\twolineshloka
{भिषजो भैषजं कर्तुं कस्मादिच्छन्ति रोगिणः}
{यदि कालेन पच्यन्ते भेषजैः किं प्रयोजनम्}


\twolineshloka
{प्रलापः सुमहान्कस्मात्क्रियते शोकमूर्च्छितैः}
{यदि कालः प्रमाणं ते कस्माद्धर्मोऽस्ति कर्तृषु}


\twolineshloka
{तव पुत्रो ममापत्यं हतवान्हिंसितो मया}
{अनन्तरं त्वयाहं च बाधितव्या महीपते}


\twolineshloka
{अहं हि पुत्रशोकेन कृतपापा तवात्मजे}
{तथा त्वया प्रहर्तव्यं मयि तत्त्वं च मे शृणु}


\twolineshloka
{भक्षार्थं क्रीडनार्थं च नरा वाञ्छन्ति पक्षिणः}
{तृतीयो नास्ति संयोगो वधबन्धादृते क्षमः}


\twolineshloka
{वधबन्धभयादेके मोक्षतन्त्रमुपाश्रिताः}
{जनीमरणजं दुःखं प्राहुर्वेदविदो जनाः}


\twolineshloka
{सर्वस्य दयिताः प्राणाः सर्वस्य दयिताः सुताः}
{दुःखादुद्विजते सर्वं सर्वस्य सुखमीप्सितम्}


\twolineshloka
{दुःखं जरा ब्रह्मदत्त दुःखमर्थविपर्ययः}
{दुःखं चानिष्टसंवासो दुःखमिष्टवियोजनम्}


\twolineshloka
{वैरबन्धकृतं दुःखं स्त्रीकृतं सह तथा}
{दुःखं दुःखेन सततं विवर्धति---- धिप}


\twolineshloka
{न दुःखं परदुःखे वै केचिदाहुर----यः}
{यो दुःखं नाभिजानाति स जल्पति माहाजने}


\twolineshloka
{यस्तु शोचति दुःखार्ताः स कथं वक्तमुत्सहेत}
{रसज्ञः सर्वदुःखस्य यथाऽऽत्मनि तथा परे}


\twolineshloka
{`भिन्ना श्लिष्टा न सज्यन्ते शस्त्रैः सुनिशितैरपि}
{'साम्ना तेऽपि निगृह्यन्ते गजा इव करेणुभिः}


\twolineshloka
{यत्कृतं ते मया राजंस्त्वया च मम यत्कृतम्}
{न तद्वर्षशतैः शक्यं व्यपोहितुमरिंदम्}


\twolineshloka
{आवयोः कृतमन्योन्यं तस्य संधिर्न विद्यते}
{स्मृत्वास्मृत्वा हि ते पुत्रं नवं वैरं भविष्यति}


\twolineshloka
{वैरमन्तिकमासाद्य यः प्रीतिं कर्तुमिच्छति}
{मृण्मयस्येव भग्नस्य तस्य संधिर्न विद्यते}


\twolineshloka
{निश्चयः स्वार्थशास्त्रेषु न विश्वासः सुखोदयः}
{उशना चैव गाथे द्वे प्रह्लादायाब्रवीत्पुरा}


\twolineshloka
{ये वैरिणः श्रद्दधते सत्ये सत्येतरेऽपि वा}
{वध्यन्ते श्रद्दधाना हि मधु शुष्कतृणैरिव}


\twolineshloka
{न हि वैराणि शाम्यन्ति कुलेष्वादशमाद्युगात्}
{आख्यातारश्च विद्यन्ते कुले चेज्जायते पुमान्}


\twolineshloka
{उपगृह्य तु वैराणि सान्त्वयन्ति नराधिपाः}
{अथैनं प्रतिहिंसन्ति पूर्णं घटमिवाश्मनि}


\threelineshloka
{सदा न विश्वसेद्राजन्पापं कृत्वेह कस्यचित्}
{अपकृत्य परेषां हि विश्वासाद्दुःखमश्नुते ॥ब्रह्मदत्त उवाच}
{}


\threelineshloka
{नाविश्वासाच्चिनोत्यर्थमीहते चापि किंचन}
{भयात्त्वेकतरं मित्रं कृतकृत्या भवत्विह ॥पूजन्युवाच}
{}


\twolineshloka
{यस्येह व्रणिनौ पादौ पभ्द्यां च परिधावतः}
{क्षिण्येते तस्य तौ पादौ सुगुप्तमपि धावतः}


\twolineshloka
{नेत्राभ्यां सरुजाभ्यां यः प्रतिवातमुदीक्षते}
{तस्य वायुरुजाऽत्यर्थं नेत्रयोर्भवति ध्रुवम्}


\twolineshloka
{दुष्टं पन्थानमासाद्य यो मोहादभिपद्यते}
{आत्मनो बलमज्ञात्वा तदन्तं तस्य जीवितम्}


\twolineshloka
{यस्तु वर्षमविज्ञाय क्षेत्रं कर्षति कर्षकः}
{हीनः पुरुषकारेण तस्य वै नाप्नुते फलम्}


\twolineshloka
{यस्तु तिक्तं कषायं वा स्वादु वा मधुरं हितम्}
{आहारं कुरुते नित्यं सोऽमृतत्वाय कल्पते}


\twolineshloka
{पथ्यं मुक्त्वा तु यो मोहाद्दुष्टमश्नाति भोजनम्}
{परिणाममविज्ञाय तदन्तं तस्य जीवितम्}


\twolineshloka
{दैवं पुरुषकारश्च स्थितावन्योन्यसंश्रयात्}
{उदात्तं कर्म वै तत्र दैवं क्लीबा उपासते}


\twolineshloka
{कर्म चात्महितं कार्यं तीक्ष्णं वा यदि वा मृदु}
{ग्रस्यतेऽकर्मशीलस्तु सदाऽनर्थैरकिंचनः}


\twolineshloka
{तस्मात्संशयितव्येऽर्थे कार्य एव पराक्रमः}
{सर्वस्वमपि संत्यज्य कार्यमात्महितं नरैः}


\twolineshloka
{विद्या शौचं च दाक्ष्यं च बलं शौर्यं च पञ्चमम्}
{मित्राणि सहजान्याहुर्वर्तयन्तीह यैर्बुधाः}


\twolineshloka
{निवेशनं च कुप्यं च क्षेत्रं भार्यां सुहृज्जनम्}
{एतान्युपचितान्याहुः सर्वत्र लभते पुमान्}


\twolineshloka
{सर्वत्र रमते प्राज्ञः सर्वत्र च विरोचते}
{न विभीषयते किंचिद्भीषितो न बिभेति च}


\twolineshloka
{नित्यं बुद्धिमतोऽप्यर्थः स्वल्पकोऽपि विवर्धते}
{दाक्ष्येण कुर्वतां कर्मं संयमात्प्रतितिष्ठति}


\twolineshloka
{गृहस्नेहावबद्धानां नराणामल्पमेधसाम्}
{कुस्त्री खादति मांसानि माघमां सेगवा इव}


\twolineshloka
{गृहं क्षेत्राणि मित्राणि स्वदेश इति चापरे}
{इत्येवमवसीदन्ति नरा बुद्धिविपर्यये}


\twolineshloka
{उत्पथाच्च विमानाच्च देशाद्दुर्भिक्षपीडितात्}
{अन्यत्र वसतिं गच्छेद्वसेद्वा नित्यमानितः}


\twolineshloka
{तस्मादन्यत्र यास्यामि वस्तुं नाहमिहोत्सहे}
{कृतमेतदनाहार्यं तव पुत्रे च पार्थिव}


\twolineshloka
{कुभार्यां च कुपुत्रं च कुराजानं कुसौहृदम्}
{कुसंबन्धं कुदेशं च दूरतः परिवर्जयेत्}


\twolineshloka
{कुमित्रे नास्ति विश्वासः कुभार्यायां कुतो रतिः}
{कुराज्ये निर्वृतिर्नास्ति कुदेशे नास्ति जीविका}


\twolineshloka
{कुपुत्रे सौहृदं नास्ति नित्यमस्थिरसौहृदम्}
{अवमानः कुंसबन्धे भवत्यर्थविपर्यये}


\twolineshloka
{सा भार्या या प्रियं ब्रूते स पुत्रो यत्र निर्वृतिः}
{तन्मित्रं यत्र विश्वासः स देशो यत्र जीवति}


\twolineshloka
{यत्र नास्ति बलात्कारः स राजा तीव्रशासनः}
{स च यौनाभिसंबन्धो यः सतोऽपि बुभूषति}


\twolineshloka
{भार्या देशोऽथ मित्राणि पुत्रसंबन्धिबान्धवाः}
{एते सर्वे गुणवति धर्मनेत्रे महीपतौ}


\twolineshloka
{अधर्मज्ञस्य विषये प्रजा नश्यन्ति निग्रहात्}
{राजा मूलं त्रिवर्गस्य अप्रमत्तोऽनुपालयन्}


\twolineshloka
{बलिषङ्भागमुद्धृत्य फलं समुपयोजयेत्}
{न रक्षति प्रजाः सम्यग्यः स पार्थिवतस्करः}


\twolineshloka
{दत्वाऽभयं यः स्वयमेव राजान तत्प्रमाणं कुरुतेऽर्थलोभात्}
{स सर्वलोकादुपलभ्य पापमधर्मबुद्धिर्निरयं प्रयाति}


\twolineshloka
{दत्त्वाऽभयं स्वयं राजा प्रमाणं कुरुते यदि}
{स सर्वं सुखमाप्नोति प्रजा धर्मेण पालयन्}


\twolineshloka
{पिता भ्राता गुरुः शास्ता वह्निर्वैश्रवणो यमः}
{सप्त राज्ञो गुणानेतान्मनुराह प्रजापतिः}


\twolineshloka
{पिता हि राजा लोकस्य प्रजानां योऽनुकम्पिता}
{तस्मिन्मिथ्यापनीते हि तिर्यग्भवति मानवः}


\twolineshloka
{संभावयति मातेव दीनमप्युपपद्यते}
{दहत्यग्निरिवानिष्टान्यमयत्यहितांस्तदा}


\twolineshloka
{इष्टेषु विसृजन्नर्थान्कुबेर इव कामदः}
{गुरुर्धर्मोपदे--- गोप्ता च परिपालनात्}


\twolineshloka
{यस्तु रञ्जयते ---- पौरजानपदान्गुणैः}
{न तस्य भ्रश्यते---ज्यं गुणधर्मानुपालनात्}


\twolineshloka
{यः सम्यक्प्रति---ह्णाति पौरजानपदार्चनम्}
{स सुखं प्रेक्षते राजा इह लोके परत्र च}


\twolineshloka
{नित्योद्विग्नाः प्रजा यस्य करुभारप्रपीडिताः}
{अनर्थैर्विप्रलुप्यन्ते स गच्छति पराभवम्}


\twolineshloka
{प्रजा यस्य विवर्धन्ते सरसीव महोत्पलम्}
{स राजा सर्वसुखदः स्वर्गलोके महीयते}


\threelineshloka
{बलिना विग्रहो राजन्न कदाचित्प्रशस्यते}
{बलिना विग्रही तस्य कुतो राज्यं कुतः सुखम् ॥भीष्म उवाच}
{}


\twolineshloka
{सैवमुक्त्वा शकुनिका ब्रह्मदत्तं नराधिपम्}
{राजानं समनुज्ञाप्य जगामाभीप्सितां दिशम्}


\twolineshloka
{एतत्ते ब्रह्मदत्तस्य पूजन्या सह भाषितम्}
{मयोक्तं भरतश्रेष्ठ किमन्यच्छ्रोतुमिच्छसि}


\chapter{अध्यायः १४०}
\twolineshloka
{युधिष्ठिर उवाच}
{}


\threelineshloka
{युगक्षयात्परिक्षीणे धर्मे लोके च भारत}
{दस्युभिः पीड्यमाने च कथं स्थेयं पितामह ॥भीष्म उवाच}
{}


\twolineshloka
{हन्त ते वर्तयिष्यामि नीतिमापत्सु भारत}
{उत्सृज्यापि घृणां काले यथा वर्तेत भूमिपः}


\twolineshloka
{अत्राप्युदाहरन्तीममितिहासं पुरातनम्}
{भारद्वाजस्य संवादं राज्ञः शत्रुंतपस्य च}


\twolineshloka
{राजात्शत्रुंतपो नाम सौवीरेषु महारथः}
{भारद्वाजमुपागम्य पप्रच्छार्थविनिश्चयम्}


\twolineshloka
{अलब्धस्य कथं लिप्सा लब्धं केन विवर्धते}
{वधितं पाल्यते केन पालितं प्रणयेत्कथम्}


\twolineshloka
{तस्मै विनिश्चितार्थाय परिपृष्टोऽर्थिश्चयम्}
{उवाच मतिमान्वाक्यमिदं हेतुमदुत्तमम्}


\twolineshloka
{नित्यमुद्यतदण्डः स्यान्नित्यं विवृतपौरुषः}
{अच्छिद्रश्छिद्रदर्शी च परेषां विवरानुगः}


\twolineshloka
{नित्यमुद्यतदण्डस्य भृशमुद्विजते नरः}
{तस्मात्सर्वाणि भूतानि दण्डेनैव प्रसाधयेत्}


\twolineshloka
{एवमेव प्रशंसन्ति बुधा ये तत्त्वदर्शिनः}
{तस्माच्चतुष्टये तस्मिन्प्रधानो दण्ड उच्यते}


\twolineshloka
{छिन्नमूले त्वधिष्ठाने सर्वे तज्जीविनो हताः}
{कथं हि शाखास्तिष्ठेयुश्छिन्नमूले वनस्पतौ}


\twolineshloka
{मूलमेवादितश्छिन्द्यादरिपक्षस्य पण्डितः}
{ततः सहायान्पक्षं च सर्वमेवानुशातयेत्}


\twolineshloka
{सुमन्त्रितं सुविक्रान्तं सुयुद्धं सुपलायितम्}
{आपदागमकाले तु कुर्वीत न विचारयेत्}


\twolineshloka
{वाङ्भात्रेण विनीतः स्याद्धृदयेन यथा क्षुरः}
{श्लक्ष्णपूर्वाभिभाषी च कामक्रोधौ विवर्जयेत्}


\twolineshloka
{सपत्नसहितो राज्ये कृत्वा सन्धिं न विश्वसेत्}
{उपक्रामेत्ततः शीघ्रं कृतकार्यो विचक्षणः}


\twolineshloka
{शत्रुं च मित्रं पूर्वेण सान्त्वेनैवानुसान्त्वयेत्}
{नित्यशश्चोद्विजेत्तस्मात्सर्पाद्वेश्मगतादिव}


\twolineshloka
{यस्य बुद्धिं परिभवेत्तमतीतेन सान्त्वयेत्}
{अनागतेन दुष्प्रज्ञं प्रत्युत्पन्नेन पण्डितम्}


\twolineshloka
{अञ्जलिं शपथं सान्त्वं शिरसा पादवन्दनम्}
{अश्रुप्रपातनं चैव कर्तव्यं भूतिमिच्छता}


\twolineshloka
{वहेदमित्रं स्कन्धेन यावदर्थस्य लम्भनम्}
{अथैनमागते काले भिन्द्याद्धटमिवाश्मनि}


\twolineshloka
{मुहूर्तमपि राजेन्द्र तिन्दुकालातवज्ज्वलेत्}
{मा तुषाग्निरिवानर्चिर्धूमायेत चिरं नरः}


\threelineshloka
{नानार्थिकोऽर्थसंबन्धं कृतघ्ने न समाचरेत्}
{अर्थी तु शक्यते भोक्तुं कृतकार्योऽवमन्यते}
{तस्मात्सर्वाणि कार्याणि सावशेषाणि कारयेत्}


\twolineshloka
{कोकिलस्य वराहस्य मेरोः शून्यस्य वेस्मनः}
{व्यालस्य भक्तचित्तस्य यच्छ्रेयस्तत्समाचरेत्}


\twolineshloka
{उत्थायोत्थाय गच्छेच्च नित्ययुक्तो रिपोर्गृहम्}
{कुशलं चास्य पृच्छेत यद्यप्यकुशलं भवेत्}


\twolineshloka
{नालसाः प्राप्नुवन्त्यर्थान्न क्लीबा नातिमानिनः}
{न च लोकरवाद्भीता न वै शश्वत्प्रतीक्षिणः}


\twolineshloka
{नास्य च्छिद्रं परो विद्याद्विद्याच्छिद्रं परस्य तु}
{गूहेत्कूर्म इवाङ्गानि रक्षेद्विवरमात्मनः}


\twolineshloka
{बकवच्चिन्तयेदर्थान्सिंहवच्च पराक्रमेत्}
{वृकवच्चावलुम्पेत शरवच्च विनिष्पतेत्}


\twolineshloka
{पानमक्षास्तथा नार्यो मृगया गीतवादितम्}
{एतानि युक्त्या सेवेत प्रसङ्गो ह्यत्र दोषवान्}


\twolineshloka
{कुर्यात्तृणमयं चापं शयीत मृगशायिकाम्}
{अन्धः स्यादन्धवेलायां बाधिर्यमपि संश्रयेत्}


\twolineshloka
{देशकालं समासाद्य विक्रमेत विचक्षणः}
{देशकालव्यतीतो हि विक्रमो निष्फलो भवेत्}


\twolineshloka
{कालाकालौ संप्रधार्य बलाबलमथात्मनः}
{परस्य च बलं ज्ञात्वा तथाऽऽत्मानं नियोजयेत्}


\twolineshloka
{दण्डेनोपनतं शत्रुं यो राजा न नियच्छति}
{स मृत्युमुपगूहेत् गर्भमश्वतरी यथा}


\twolineshloka
{सुपुष्पितः स्यादफलः फलवान्स्याद्दुरारुहः}
{आमः स्यात्पक्वसंकाशो न च शीर्येत कस्यचित्}


\twolineshloka
{आशां कालवतीं कुर्यात्तां च विघ्नेन योजयेत्}
{विघ्नं निमित्ततो ब्रूयान्निमित्तं चापि हेतुमत्}


\twolineshloka
{भीतवत्संविधातव्यं यावद्भयमनागतम्}
{आगतं तु भयं दृष्ट्वा प्रहर्तव्यमभीतवत्}


\twolineshloka
{न साहसमनारुह्य नरो भद्राणि पश्यति}
{संशयं पुनरारुह्य यदि जीवति पश्यति}


\twolineshloka
{अनागतं विजानीयात्त्यजेद्भयमुपस्थितम्}
{पुनर्बुद्धिक्षयात्किंचिदनिवृत्तिं निशामयेत्}


\twolineshloka
{प्रत्युपस्थितकालस्य सुखस्य परिवर्जनम्}
{अनागतसुखाशा च नैव बुद्धिमतां नयः}


\twolineshloka
{योऽरिणा सह संधाय विश्वस्तः स्वपते सुखम्}
{स वृक्षाग्रे प्रसुप्तो वा पतितः प्रतिबुध्यते}


\twolineshloka
{कर्मणा येन केनेह मृदुना दारुणेन वा}
{उद्धरेद्दीनमात्मानं समर्थो धर्ममाचरेत्}


\twolineshloka
{ये सपत्नाः सपत्नानां सर्वांस्ताननुवर्तयेत्}
{आत्मनश्चापि बोद्धव्याश्चाराः सुमहिताः परैः}


\twolineshloka
{चारः सुविहितः कार्य आत्मनोऽथ परस्य च}
{पाषण़्डांस्तापसादींश्च परराष्ट्रे प्रवेशयेत्}


\twolineshloka
{उद्यानेषु विहारेषु प्रपास्वावसथेषु च}
{पानागारे प्रवेशेषु तीर्थेषु च सभासु च}


\twolineshloka
{धर्माभिचारिणः पापाश्चौरा लोकस्य कण्टकाः}
{समागच्छन्ति तान्बुद्ध्वा नियच्छेच्छमयीत च}


\twolineshloka
{न विश्वसेदविश्वस्ते विश्वस्ते नातिविश्वसेत्}
{विश्वासाद्भयमभ्येति नापरीक्ष्य च विश्वसेत्}


\twolineshloka
{विश्वासयित्वा तु परं तत्त्वभूतेन हेतुना}
{अथास्य प्रहरेत्काले किंचिद्विचलिते पदे}


\twolineshloka
{अशङ्क्यमपि शङ्केत नित्यं शङ्केत शङ्कितान}
{भयं ह्यशङ्किताज्जातं समूलमपि कृन्तति}


\twolineshloka
{अवधानेन मौनेन काषायेण जटाजिनैः}
{विश्वासयित्वा द्वेष्टारमवलुम्पेद्यथा वृकः}


\twolineshloka
{पुत्रो वा यदि वा भ्राता पिता वा यदि वा सुहृद}
{अर्थस्य विघ्नं कुर्वाणा हन्तव्या भूतिमिच्छता}


\twolineshloka
{गुरोरप्यवलिप्तस्य कार्याकार्यमजानतः}
{उत्पथं प्रतिपन्नस्य कार्यं भवति शासनम्}


\threelineshloka
{प्रत्युत्थानाभिवादाभ्यां संप्रदानेन केनचित्}
{प्रपूजयन्निघाती स्यात्तीक्ष्णतुण्ड इव द्विजः}
{}


\twolineshloka
{नाच्छित्त्वा परमर्माणि नाकृत्वा कर्म दारुणम्}
{नाहत्वा मत्स्यघातीव प्राप्नोति परमां श्रियम्}


\twolineshloka
{नास्ति जात्या रिपुर्नाम मित्रं वाऽपि न विद्यते}
{सामर्थ्ययोगाज्जायन्ते मित्राणि रिपवस्तथा}


\twolineshloka
{न प्रमुञ्चेत दायादं वदन्तं करुणं बहु}
{दुःखं तत्र न कर्तव्यं हन्यात्पूर्वापकारिणम्}


\twolineshloka
{संग्रहानुग्रहे यत्नः सदा कार्योऽनसूयता}
{निग्रहश्चापि यत्नेन कर्तव्यो हितमिच्छता}


\twolineshloka
{प्रहरिष्यन्प्रियं ब्रूयात्प्रहृत्यापि प्रियोत्तरम्}
{असिनाऽपि शिरश्छित्त्वा शोचेत च रुदेत च}


\twolineshloka
{निमन्त्रयीत सान्त्वेन संमानेन तितिक्षया}
{आशाकरणमित्येतत्कर्तव्यं भूतिमिच्छता}


\threelineshloka
{न शुष्कवैरं कुर्वीत बाहुभ्यां न नदीं तरेत्}
{अनर्थकमनायुष्यं गोविषाणस्य भक्षणम्}
{दन्ताश्च परिमृद्यन्ते रसश्चापि न लभ्यते}


\twolineshloka
{त्रिवर्गे त्रिविधा पीडा अनुबन्धस्तथैव च}
{अनुबन्धं तथा ज्ञात्वा पीडां च परिवर्जयेत्}


\twolineshloka
{ऋणशेषं चाग्निशेषं शत्रुशेषं तथैव च}
{पुनः पुनः प्रवर्धन्ते तस्माच्छेषं न कारयेत्}


\twolineshloka
{ऋणशेषां विवर्धन्ते परिभूताश्च शत्रवः}
{आवहन्त्यनयं तीव्रं व्याधयश्चाप्युपेक्षिताः}


\twolineshloka
{नाम---क्कृत्यकारी स्यादप्रमत्तः सदा भवेत्}
{कष्यकोपि हि दुश्छिन्नो विकारं कुरुते चिरम्}


\twolineshloka
{वधेम च मनुष्याणां मार्गाणां दूषणेन च}
{आकाराणां विनाशैश्च परराष्ट्रं विनाशयेत्}


\twolineshloka
{गृध्रदृष्टिर्बकालीनः श्वचेष्टः सिंहविक्रमः}
{अनुद्विग्रः काकशङ्की भुजङ्गचरितं चरेत्}


\twolineshloka
{शूरमञ्जलिपातेन भीरुं भेदेन भेदयेत्}
{लुब्धमर्थप्रदानेन समं तुल्येन विग्रहः}


\twolineshloka
{श्रेणीमुख्योपजापेषु वल्लभानुनयेषु च}
{अमात्यान्परिरक्षेत भेदसंघातयोरपि}


\twolineshloka
{मृदुरित्यवजानन्ति तीक्ष्ण इत्युद्विजन्ति च}
{तीक्ष्णकाले भवेत्तीक्ष्णो मृदुकाले मृदुर्भवेत्}


\twolineshloka
{मृदुनैव मृदुं हन्ति मृदुना हन्ति दारुणम्}
{नासाध्यं मृदुना किंचित्तस्मात्तीक्ष्णतरो मृदुः}


\twolineshloka
{काले मृदुर्यो भवति काले भवति दारुणः}
{स साधयति कृत्यानि शत्रुं चाप्यधितिष्ठति}


\twolineshloka
{पण्डितेन विरुद्धस्तु दूरस्थोऽस्मीति नाश्वसेत्}
{दीर्घौ बुद्धिमतो बाहू याभ्यां हिंसति हिंसितः}


\twolineshloka
{न तत्तरेद्यस्य न पारमुत्तरेन्न तद्धरेद्यत्पुनराहरेत्परः}
{न तत्खनेद्यस्य न मूलमुद्धरेन्न तं हन्याद्यस्य शिरो न पातयेत्}


\threelineshloka
{इतीदमुक्तं वृजिनाभिसंहितंन चैतदेवं पुरुषः समाचरेत्}
{परप्रयुक्तस्तु कथं विभावयेदतो मयोक्तं भवतो हितार्थिना ॥भीष्म उवाच}
{}


\twolineshloka
{यथावदुक्तं वचनं हितार्थिनानिशम्य विप्रेण सुवीरराष्ट्रपः}
{तथाऽकरोद्वाक्यमदीनचेतनःश्रियं च दीप्तां बुभुजे सबान्धवः}


\chapter{अध्यायः १४१}
\twolineshloka
{युधिष्ठिर उवाच}
{}


\twolineshloka
{हीने परमके धर्मे सर्वलोकविलङ्घिते}
{अधर्मे धर्मतां नीते धर्मे चाधर्मतां गते}


\twolineshloka
{मर्यादासु प्रभिन्नासु क्षुभिते लोकिश्चये}
{राजभिः पीडिते लोके चोरैर्वाऽपि विशांपते}


\twolineshloka
{सर्वाश्रमेषु मूढेषु कर्मसूपहतेषु च}
{कामाल्लोभाच्च मोहाच्च भयं पश्यत्सु भारत}


\twolineshloka
{अविश्वस्तेषु सर्वेषु नित्यं भीतेषु भारत}
{नित्यं च हन्यमानेषु वञ्चयत्सु परस्परम्}


\twolineshloka
{प्रदीप्तेषु च देशेषु ब्राह्मण्ये चातिपीडिते}
{अवर्षति च पर्जन्ये मिथो भेदे समुत्थिते}


\twolineshloka
{सर्वस्मिन्दस्युसाद्भूते पृथिव्यामुपजीवने}
{केनस्विद्ब्राह्मणो जीवेज्जघन्ये काल आगते}


\twolineshloka
{अतितिक्षुः पुत्रपौत्राननुक्रोशान्नराधिप}
{कतमापदि वर्तेत तन्मे ब्रूहि पितामह}


\threelineshloka
{कथं च राजा वर्तेत लोके कलुषतां गते}
{कथमर्थाच्च धर्माच्च न हीयेत परंतप ॥भीष्म उवाच}
{}


\twolineshloka
{राजमूला महाबाहो योगक्षेमसुवृष्टयः}
{प्रजासु व्याधयश्चैव मरणं च भयानि च}


\twolineshloka
{कृतं त्रेता द्वापरं च कलिश्च भरतर्षभ}
{राजमूला इति मतिर्मम नास्त्यत्र संशयः}


\twolineshloka
{तस्मिंस्त्वभ्यागते काले प्रजानां दोषकारके}
{विज्ञानबलमास्थाय जीवितव्यं भवेत्तदा}


\twolineshloka
{अधाप्युदाहरन्तीममितिहासं पुरातनम्}
{विश्वामित्रस्य संवादं चण्डालस्य च पक्कणे}


\twolineshloka
{त्रेताद्वापरयोः संधौ पुरा दैवव्यतिक्रमात्}
{अनावृष्टिरभूद्धोरा लोके द्वादशवार्षिकी}


\twolineshloka
{प्रजानामतिवृद्धानां युगान्ते समुपस्थिते}
{त्रेतायां मोक्षसमये द्वापरप्रतिपादने}


\twolineshloka
{न ववर्ष सहस्राक्षः प्रतिलोमोऽभवद्गुरुः}
{जगाम दक्षिणं मार्गं सोमो व्यावृत्तमण्डलः}


\twolineshloka
{नावश्यायोऽपि रात्र्यन्ते कुत एवाभ्रराजयः}
{नद्यः संक्षिप्ततोयौघाः किंचिदन्तर्गताऽभवन्}


\twolineshloka
{सरांसि सरितश्चैव कूपाः प्रस्रवणानि च}
{हतत्विषो न लक्ष्यन्ते निसर्गात्पूर्वकारितात्}


\twolineshloka
{भूमिः शुष्कजलस्थाना विनिवृत्तसभाप्रपा}
{निवृत्तयज्ञस्वाध्याया निर्वषट््कारमङ्गला}


\twolineshloka
{उत्सन्नकृषिगोरक्षा निवृत्तविपणापणा}
{निवृत्तपूर्वसमया संप्रनष्टमहोत्सवा}


\twolineshloka
{अस्थिकङ्कालसंकीर्णा हाहाभूतनराकुला}
{शून्यभूयिष्ठनगरा दग्धग्रामनिवेशना}


\twolineshloka
{क्वचिच्चोरैः क्वचिच्छूरैः क्वचिद्राजभिरातुरैः}
{परस्परभयाच्चैव शून्यभूयिष्ठनिर्जना}


\twolineshloka
{गतदैवतसंस्थाना वृद्धबालविनाकृता}
{गोजाविमहिषीहीना परस्परपराहता}


\twolineshloka
{हतविप्रा हतारक्षा प्रनष्टोत्सवसंचया}
{शवभूतनरप्राया बभूव वसुधा तदा}


\twolineshloka
{तस्मिन्प्रतिभये काले क्षीणधर्मे युधिष्ठिर}
{बभूवुः क्षुधिता मर्त्याः खादमानाः परस्परम्}


\twolineshloka
{ऋषयो नियमांस्त्यक्त्वा परित्यक्ताग्निदेवताः}
{आश्रमान्संपरित्यज्य पर्यधावन्नितस्ततः}


\twolineshloka
{विश्वामित्रोऽथ भगवान्महर्षिरनिकेतनः}
{क्षुधा परिगतो धीमान्समन्तात्पर्यधावत्}


\twolineshloka
{त्यक्त्वा दारांश्च पुत्रांश्च कस्मिंश्च जनसंसदि}
{भक्ष्याभक्ष्यसमो भ्रूत्वा निरग्निरनिकेतनः}


\twolineshloka
{स कदाचित्परिपतञ्श्वपचानां निकेतनम्}
{हिंस्राणां प्राणिघातानामाससाद वने क्वचित्}


\twolineshloka
{विभिन्नकलशाकीर्णं श्रमांसेन च भूषितम्}
{वराहखरभग्नास्थिकपालघटसंकुलम्}


\twolineshloka
{मृतचेलपरिस्तीर्णं निर्माल्यकृतभूषणम्}
{सर्पनिर्मोकमालाभिः कृतचिह्नकुटीमुखम्}


\twolineshloka
{कुक्कुटाराबहुलं गर्दभध्वनिनादितम्}
{उद्धोषद्भिः खरैर्वाक्यैः कलहद्भिः परस्परम्}


\twolineshloka
{उलूकपक्षिध्वनिभिर्देवतायतनैर्वृतम्}
{लोहघण्टापरिष्कारं श्वयूथपरिवारितम्}


\twolineshloka
{तत्प्रविश्य क्षुधाविष्टो गाधिपुत्रो महानृपिः}
{आहारान्वेषणे युक्तः परं यत्नं समास्थितः}


\twolineshloka
{न च क्वचिदविन्दत्स भिक्षमाणोऽपि कौशिकः}
{मांसमन्नं फलं मूलमन्यद्वा तत्र किंचन}


\twolineshloka
{अहो कृच्छ्रं मया प्राप्तमिति निश्चित्य कौशिकः}
{पपात भूमौ दौर्बल्यात्तस्मिंश्चण्डालपक्कणे}


\twolineshloka
{स चिन्तयामास मुनिः किंनु मे सुकृतं भवेत्}
{कथं वृथा न मृत्युः स्यादिति पार्थिवसत्तम}


\twolineshloka
{स ददर्श श्वमांसस्य कुतन्त्रीं पतितां मुनिः}
{चण्डालस्य गृहे राजन्सद्यः शस्त्रहतस्य वै}


\twolineshloka
{स चिन्तयामास तदा स्तेयं कार्यमितो मया}
{न--दानीमुपायो मे विद्यते प्राणधारणे}


\threelineshloka
{---सु विहितं स्तेयं विशिष्टसमहीनतः}
{परस्परं भवेत्पूर्वमास्थेयमिति निश्चयः}
{}


\twolineshloka
{हीनादादेयमादौ स्यात्समानात्तदनन्तरम्}
{असंभवे त्वाददीत विशिष्टादपि धार्मिकात्}


\twolineshloka
{सोऽहमन्तावसायीनां हराम्येनां प्रतिग्रहात्}
{न स्तेयदोषं पश्यामि हरिष्याम्येतदामिषम्}


\twolineshloka
{एतां बुद्धिं समास्थाय विश्वामित्रो महामुनिः}
{तस्मिन्देशे सुसुष्वाप पतितो यत्र भारत}


\twolineshloka
{स विगाढां निशां दृष्ट्वा सुप्ते चण्डालपक्कणे}
{शनैरुत्थाय भगवान्प्रविवेश कुटीमुखम्}


\twolineshloka
{स सुप्त एव चण्डालः श्लेष्मापिहितलोचनः}
{परिभिन्नस्वरो रूक्षः प्रोवाचाप्रियदर्शनः}


\twolineshloka
{कः कुतन्त्रीं घट्टयति सुप्ते चण्डालपक्कणे}
{जागर्मि नैव सुप्तोऽस्मि हतोऽसीति च दारुणः}


\twolineshloka
{विश्वामित्रोऽहमित्येव सहसा तमुवाच ह}
{सहसाऽभ्यागतं भूयः सोद्वेगस्तेन कर्मणा}


\twolineshloka
{विश्वामित्रोऽहमायुष्मन्नागतोऽहं बुभुक्षितः}
{मा वधीर्मम सद्बुद्धे यदि सम्यक्प्रपश्यसि}


\twolineshloka
{चण्डालस्तद्वचः श्रुत्वा महर्षेर्भावितात्मनः}
{शयनादुपसंभ्रान्त उद्ययौ प्रति तं ततः}


\twolineshloka
{स विसृज्याश्रु नेत्राभ्यां बहुमानात्कृताञ्जलिः}
{उवाच कौशिकं रात्रौ ब्रह्मन्किं ते चिकीर्षितम्}


\twolineshloka
{विश्वामित्रस्तु मातङ्गमुवाच परिसान्त्वयन्}
{क्षुधितोऽन्तर्गतप्राणो हरिष्यामि श्वजाघनीम्}


\twolineshloka
{क्षुधितः कलुषं यातो नास्ति ह्रीरशनार्थिनः}
{क्षुच्च मां दूषयत्यत्र हरिष्यामि श्वजाघनीम्}


\threelineshloka
{अवसीदन्ति मे प्राणाः स्मृतिर्मे नश्यति क्षुधा}
{दुर्बलो नष्टसंज्ञश्च भक्ष्याभक्ष्यविवर्जितः}
{सोधर्मं बुध्यमानोऽपि हरिष्यामि श्वजाघनीम्}


\twolineshloka
{यदा भैक्षं न विन्दामि युष्माकमहमालये}
{तदा बुद्धिः कृता पापे हरिष्यामि श्वजाघनीम्}


\twolineshloka
{अग्निर्मुखं पुरोधाश्च देवानां शुचिषाङ्विभुः}
{यथा च सर्वभुग्ब्रह्मा तथा मां विद्धि धर्मतः}


\twolineshloka
{तमुवाच स चण्डालो महर्षे शृणु मे वचः}
{श्रुत्वा तथा तमातिष्ठ यथा धर्मो न हीयते}


% Check verse!
धर्मं तवापि विप्रर्षे शृणु यत्ते ब्रवीम्यहम्
\twolineshloka
{मृगाणामधमं श्वानं प्रवदन्ति मनीषिणः}
{तस्याप्यधम उद्देशः शरीरस्य तु जाघी}


\twolineshloka
{नेदं सम्यग्व्यवसितं महर्षे कर्म गर्हितम्}
{चण्डालस्वस्य हरणमभक्ष्यस्य विशेषतः}


\twolineshloka
{साध्वन्यमनुपश्य त्वमुपायं प्राणधारणे}
{श्वमांसलोभात्तपसो नाशस्ते स्यान्महामुने}


\twolineshloka
{जानता विहितो मार्गो न कार्यो धर्मसंकरः}
{मा स्म धर्मं परित्याक्षीस्त्वं हि धर्मविदुत्तमः}


\twolineshloka
{विश्वामित्रस्ततो राजन्नित्युक्तो भरतर्षभ}
{क्षुधार्तः प्रत्युवाचेदं पुनरेव महामुनिः}


\twolineshloka
{निराहारस्य सुमहान्मम कालोऽभिधावतः}
{न विद्यतेऽप्युपायश्च कश्चिन्मे प्राणधारणे}


\twolineshloka
{येनकेन विशेषेण कर्मणा येनकेनचित्}
{उज्जिहीर्षे सीदमानः समर्थो धर्ममाचरेत्}


\twolineshloka
{ऐन्द्रो धर्मः क्षत्रियाणां ब्राह्मणानामथाग्निकः}
{ब्रह्मवह्निर्मम बलं भोक्ष्यामि शमयन्क्षुधाम्}


\twolineshloka
{यथायथैव जीवेद्धि तत्कर्तव्यमहेलया}
{जीवितं मरणाच्छ्रेयो जीवन्धर्ममवाप्नुयात्}


\twolineshloka
{सोऽहं जीवितमाकाङ्क्षन्नभक्ष्यस्यापि भक्षणम्}
{व्यवस्ये बुद्धिपूर्वं वै तद्भवाननुमन्यताम्}


\threelineshloka
{जीवन्धर्मं चरिष्यामि प्रणोत्स्याम्यशुभानि तु}
{तपोभिर्विद्यया चैव ज्योतींषीव महत्तमः ॥श्वपच उवाच}
{}


\threelineshloka
{नैतत्खादन्प्राप्स्यसे प्राणमद्यनायुर्दीर्घं नामृतस्येव तृप्तिम्}
{भिक्षामन्यां भिक्ष मा ते मनोस्तुश्वभक्षणे श्वा ह्यभक्ष्यो द्विजानाम् ॥विश्वामित्र उवाच}
{}


\threelineshloka
{न दुर्भिक्षे सुलभं मांसमन्यच्छ्वपाकमन्ये न च मेऽस्ति वित्तम}
{क्षुघार्तश्चाहमगतिर्निराशःश्वजाघनीं ष़ड््सात्साधु मन्ये ॥श्वपच उवाच}
{}


\fourlineindentedshloka
{पञ्च पञ्चनखा भक्ष्या ब्रह्मक्षत्रस्य वै विशः}
{`शल्यकः श्वाविधो गोधा शशः कूर्मश्च पञ्चमः}
{'यदि शास्त्रं प्रमाणं ते माऽभक्ष्ये वै मनः कृथाः ॥विश्वामित्र उवाच}
{}


\threelineshloka
{अगस्त्येनासुरो जग्धो वातापिः क्षुधितेन वै}
{अहमापद्गतः क्षुब्धो भक्षयिष्ये श्वजाघनीम् ॥श्वपच उवाच}
{}


\threelineshloka
{भिक्षामन्यामाहरेति न च कर्तुमिहार्हसि}
{न नूनं कार्यमेतद्वै हर कामं श्वजाघनीम् ॥विश्वामित्र उवाच}
{}


\threelineshloka
{शिष्टा वै कारणं धर्मे तद्वृत्तमनुवर्तये}
{परां मेध्याशनादेनां भक्ष्यां मन्ये श्वजाघनीम् ॥श्वपच उवाच}
{}


\threelineshloka
{असद्भिर्यः समाचीर्णो न स धर्मः सनातनः}
{अकार्यमिह कार्यं वा मा छलेनाशुभं कृथाः ॥विश्वामित्र उवाच}
{}


\threelineshloka
{न पातकं नावमतमृषिः सन्कर्तुमर्हति}
{समौ च श्वमृगौ मन्ये तस्माद्भोक्ष्ये श्वजाघनीम् ॥श्वपच उवाच}
{}


\threelineshloka
{यद्ब्राह्मणार्थे कृतमर्थिनेनतेनर्षिणा तदभक्ष्यं न कामात्}
{स वै धर्मो यत्र न पापमस्तिसर्वैरुपायैर्गुरवो हि रक्ष्याः ॥विश्वामित्र उवाच}
{}


\threelineshloka
{मित्रं च मे ब्राह्मणस्यायमात्माप्रियश्च मे पूज्यतमश्च लोके}
{तद्भोक्तुकामोऽहमिमां जिहीर्षेनृशंसानामीदृशानां न विभ्ये ॥श्वपच उवाच}
{}


\threelineshloka
{कामं नरा जीवितं संत्यजन्तिन चाभक्ष्ये क्वचित्कुर्वन्ति बुद्धिम्}
{सर्वांश्च कामान्प्राप्नुवन्तीति विद्धिस्वर्गे निवासात्सहते क्षुधां वै ॥विश्वामित्र उवाच}
{}


\twolineshloka
{स्थाने भवेत्स यशः प्रेत्यभावेनिःसंशयः कर्मणां वै विनाशः}
{अहं पुनर्व्रतनित्यः शमात्मामूलं रक्ष्यं भक्षयिष्याम्यभक्ष्यम्}


\threelineshloka
{बुद्ध्यात्मके व्यक्तमस्तीति सृष्टोमोक्षात्मके त्वं यथा शिष्टचक्षुः}
{यद्यप्येतत्संशयाच्च त्रपामिनाहं भविष्यामि यथा न माया ॥श्वपच उवाच}
{}


\threelineshloka
{गोपनीयमिदं दुःखमिति मे निश्चिता मतिः}
{दुष्कृतं ब्राह्मणं सन्तं यस्त्वामहमुपालभे ॥विश्वामित्र उवाच}
{}


\threelineshloka
{पिबन्त्येवोदकं गावो मण्डूकेषु रुवत्स्वपि}
{न तेऽधिकारो धर्मेऽस्ति वा भूरात्मप्रशंसकः ॥श्वपच उवाच}
{}


\threelineshloka
{सुहृद्भूत्वाऽनुशोचे त्वां कृपा हि त्वयि मे द्विज}
{तदिदं श्रेय आधत्स्व मा लोभे चेत आदधाः ॥विश्वामित्र उवाच}
{}


\threelineshloka
{सृहृन्मे त्वं सुखेप्सुश्चेदापदो मां समुद्धर}
{जामऽहं धर्मतोऽऽत्मानमुत्सृजेमां श्वजाघनीम् ॥श्वपच उवाच}
{}


\threelineshloka
{नैवोत्सहे भवतो दातुमेतांनोपेक्षितुं ह्रियमाणं स्वमन्नम्}
{उभौ स्यावः श्वमलेनानुलिप्तौदाता चाहं ब्राह्मणस्त्वं प्रतीच्छम् ॥विश्वामित्र उवाच}
{}


\threelineshloka
{अद्याहमेतद्वॄजिनं कर्म कृत्वाजीवंश्चरिष्यामि महापवित्रम्}
{संपूतात्मा धर्ममेवाभिपत्स्येयदेतयोर्गुरु तद्वै ब्रवीहि ॥श्वपच उवाच}
{}


\threelineshloka
{आत्मैव साक्षी किल धर्मकृत्येत्वमेव जानासि यदत्र दुष्कृतम्}
{यो ह्याद्रियाद्भक्ष्यमिति श्वमांसंमन्ये न तस्यास्ति विवर्जनीयम् ॥विश्वामित्र उवाच}
{}


\threelineshloka
{उपधानैः साधते नापि दोषःकार्ये सिद्धे मित्र नात्रापवादः}
{अस्मिन्नहिंसा नानृते वाक्यलेशोभक्ष्यक्रिया यत्र न तद्गरीयः ॥श्वपच उवाच}
{}


\threelineshloka
{यद्येष हेतुस्तव खादने स्यान्न ते वेदः कारणं नार्यधर्मः}
{तस्माद्भक्ष्ये भक्षणे वा द्विजेन्द्रदोषं न पश्यामि यथेदमत्र ॥विश्वामित्र उवाच}
{}


\threelineshloka
{न पातकं भक्षमाणस्य दृष्टंसुरां तु पीत्वा पततीति शब्दः}
{अन्योन्यकार्याणि यथा तथैवन लेपमात्रेण कृतं हिनस्ति ॥श्वपच उवाच}
{}


\twolineshloka
{`पादौ मूलं समभवद्वृन्ताकं शिर उच्यते}
{शेफात्तु गृञ्जरं जातं पलाण्डुस्त्वण्डसंभवः}


\twolineshloka
{श्वरोमजः शैव्यशाको लशुनं द्विजसंभवम्}
{चुक्किनामा पर्णशाकः कर्णादजनि भूसुर ॥'}


\threelineshloka
{अस्थानतो हीनतः कुत्सिताद्वातद्विद्वांसं बाधते साधु वृत्तम्}
{श्वानं पुनर्यो लभतेऽभिषङ्गात्तेनापि दण्डः सहितव्य एव ॥भीष्म उवाच}
{}


\twolineshloka
{एवमुक्त्वा निववृते मातङ्गः कौशिकं तदा}
{विश्वामित्रो जहारैव कृतबुद्धिः श्वजाघनीम्}


\twolineshloka
{ततो जग्राह स श्वाङ्गं जीवितार्थी महामुनिः}
{सदारस्तामुपाहृत्य वने भोक्तुमियेप सः}


\twolineshloka
{अथास्य बुद्धिरभवद्विधिनाऽहं श्वजाघनीम्}
{भक्षयामि यथाकालं पूर्वं संतर्प्य देवताः}


\twolineshloka
{ततोऽग्निमुपसंहृत्य ब्राह्मेण विधिना मुनिः}
{ऐन्द्राग्नेयेन विधिना चरुं श्रपयत स्वयम्}


\twolineshloka
{ततः समारभत्कर्म दैवं पित्र्यं च भारत}
{आहूय देवानिन्द्रादीन्भागंभागं विधिक्रमात्}


\twolineshloka
{एतस्मिन्नेव काले तु प्रववर्ष स वासवः}
{संजीवयन्प्रजाः सर्वा जनयामास चौषधीः}


\twolineshloka
{विश्वामित्रोऽपि भगवांस्तपसा दग्धकिल्चिषः}
{कालेन महता सिद्धिमवाप परमाद्भुताम्}


\twolineshloka
{स संहृत्य च तत्कर्म अनास्वाद्य च तद्धविः}
{तोषयामास देवांश्च पितॄंश्च द्विजसत्तमः}


\twolineshloka
{एवं विद्वानदीनात्मा व्यसनस्थो जिजीविषुः}
{सर्वोपायैरुपायज्ञो दीनमात्मानमुद्धरेत्}


\twolineshloka
{एतां बुद्धिं समास्थाय जीवितव्यं सदा भवेत्}
{जीवन्पुण्यमवाप्नोति पुरुषो भद्रमश्नुते}


\twolineshloka
{तस्मात्कौन्तेय विदुषा धर्माधर्मविनिश्चये}
{बुद्धिमास्थाय लोकेऽस्मिन्वर्तितव्यं कृतात्मना}


\chapter{अध्यायः १४२}
\twolineshloka
{युधिष्ठिर उवाच}
{}


\twolineshloka
{यदिदं घोरमुद्दिष्टमश्रद्धेयमिवानृतम्}
{अस्तिस्विद्दस्युमर्यादा यामयं परिवर्जयेत्}


\threelineshloka
{संमुह्यामि विपीदामि धर्मो मे शिथिलीकृतः}
{उद्यमं नाधिगच्छामि कुतश्चित्परिचिन्तयन् ॥भीष्म उवाच}
{}


\twolineshloka
{नैतच्छ्रुत्वागमादेव तव धर्मानुशासनम्}
{प्रज्ञासमभिहारोऽयं कविभिः संभृतं मधु}


\twolineshloka
{बाह्याः प्रतिविधातव्याः प्रज्ञा राज्ञा ततस्ततः}
{बहुशाखेन धर्मेण यत्रैषा संप्रसिध्यते}


\twolineshloka
{बुद्धिं संजनयेद्राज्ञां धर्ममाचरतां सदा}
{जयो भवति कौरव्य सदा तद्वृद्धिरेव च}


\twolineshloka
{बुद्धिश्रेष्ठा हि राजानो जयन्ति विजयैषिणः}
{धर्मः प्रतिविधातव्यो बुद्ध्या राज्ञा ततस्ततः}


\twolineshloka
{नैकशाखेन धर्मेण राज्ञो धर्मो विधीयते}
{दुर्बलस्य कुतः प्रज्ञा पुरस्तादनुदाहृता}


\twolineshloka
{अद्वैधज्ञः प्रतिद्वैधे संशयं प्राप्नुमर्हति}
{बुद्धिद्वैधं विधातव्यं पुरस्तादेव भारत}


\twolineshloka
{पार्श्वतः कारणं राज्ञो विषूच्यस्त्वापगा इव}
{जनास्तूच्चरितं धर्मं विजान्त्यन्यथाऽन्यथा}


\twolineshloka
{सम्यग्विज्ञानिनः केचिन्मिथ्याविज्ञानिनः परे}
{तद्वै यथायथं बुद्ध्वा ज्ञानमाददते सताम्}


\twolineshloka
{परिमुष्णन्ति शास्त्राणि धर्मस्य परिपन्थिनः}
{वैषम्यमर्थविद्यानां निरर्थाः ख्यापयन्ति ते}


\twolineshloka
{आजिजीविषवो विद्यां यशः कामौ समन्तनः}
{ते सर्वे नृप पापिष्ठा धर्मस्य परिपन्थिनः}


\twolineshloka
{अपक्वमतयो मन्दा न जानन्ति यथातथम्}
{तथा ह्यशास्त्रकुशलाः सर्वत्रायुक्तिनिष्ठिताः}


\twolineshloka
{परिमुष्णन्ति शास्त्राणि शास्त्रदोषानुदर्शिनः}
{विज्ञानमथ विद्यानां न सम्यगिति मे मतिः}


\twolineshloka
{निन्दया परविद्यानां स्वविद्यां ख्यापयन्ति च}
{वागास्तिक्यानुनीताश्च दुग्धविद्याफला इव}


\twolineshloka
{तान्विद्यावणिजो विद्धि राक्षसानिव भारत}
{व्याजेन कृत्स्नो विहितो धर्मस्ते परिहास्यते}


\twolineshloka
{न धर्मवचनं वाचा नैव बुद्ध्येति नः श्रुतम्}
{इति बार्हस्पत्यविज्ञानं प्रोवाच मघवा स्वयम्}


\twolineshloka
{न त्यव वचनं किंचिदनिमित्तादिहोच्यते}
{स्वविनीतेन शास्त्रेण ह्यविद्यः स्यादथापरः}


\twolineshloka
{लोकयात्रामिहैके तु धर्मं प्राहुर्मनीषिणः}
{समुद्दिष्टं सतां धर्मं स्वयमूहेत् पण्डितः}


\twolineshloka
{अमर्षाच्छास्त्रसंमोहादनिमित्तादिहोच्यते}
{शास्त्रं प्राज्ञस्य वदतः समूहे यात्यदर्शनम्}


\twolineshloka
{आगमागतया बुद्ध्या वचनेन प्रशस्यते}
{अज्ञानाज्ज्ञानहेतुत्वाद्वचनं साधु मन्यते}


\twolineshloka
{अनुपागतमेवेदं शास्त्रमेवमपार्थकम्}
{दैतेयानुशना प्राह संशयच्छेदनं पुरा}


\twolineshloka
{ज्ञानमप्यपदिश्यं हि यथा नास्ति तथैव तत्}
{तेन संच्छिन्नमूलेन कस्तोषयितुमिच्छति}


\twolineshloka
{पुनर्व्यवसितं यो वा नेदं वाक्यमुपाश्नुते}
{उग्रायैव हि सृष्टोऽसि कर्मणे तत्त्वमीक्षसे}


\twolineshloka
{अग्रे मामन्ववेक्षस्व राजन्योऽयं बुभूषते}
{यथा प्रमुच्यते त्वन्यो यदर्थं न प्रमोदते}


\twolineshloka
{अजोऽश्वः क्षत्रमित्येतत्सदृशं ब्रह्मणा कृतम्}
{तस्मान्नतैक्ष्ण्याद्भूतानां यात्रा काचित्प्रसिद्ध्यति}


\twolineshloka
{यस्त्ववध्यवधे दोषः स वध्यस्यावधे स्मृतः}
{एषा ह्येव तु मर्यादा यामयं परिवर्जयेत्}


\twolineshloka
{तस्मात्तीक्ष्णः प्रजा राजा स्वधर्मे स्थापयत्युत}
{अन्योन्यं भक्षयन्तो हि प्रचरेयुर्वृका इव}


\twolineshloka
{यस्य दस्युगणा राष्ट्रे ध्वाङ्क्षा मत्स्याञ्जलादिव}
{विहरन्ति परस्वानि स वै क्षत्रियपांसनः}


\twolineshloka
{कुलीनान्सचिवान्कृत्वा वेदविद्यासमन्वितान्}
{प्रशाधि पृथिवीं राजन्प्रजा धर्मेण पालयन्}


\twolineshloka
{विहीनजन्मकर्माणि यः प्रगृह्णाति भूमिपः}
{उभयस्याविशेषज्ञस्तद्वै क्षत्रं नपुंसकम्}


\twolineshloka
{नैवोग्रं नैव चानुग्रं धर्मेणेह प्रशस्यते}
{उभयं न व्यतिक्रामेदुग्रो भूत्वा मृदुर्भव}


\twolineshloka
{कष्टः क्षत्रियधर्मोऽयं सौहृदं त्वयि मे स्थितम्}
{उग्रकर्मणि सृष्टोऽसि तस्माद्राज्यं प्रशाधि वै}


\fourlineindentedshloka
{`अरुष्टः कस्यचिद्राजन्नेवमेव समाचर}
{'अशिष्टनिग्रहो नित्यं शिष्टस्य परिपालनम्}
{एवं शुक्रोऽब्रवीद्धीमानापत्सु भरतर्षभ ॥युधिष्ठिर उवाच}
{}


\threelineshloka
{अस्ति चेदिह मर्यादा यामन्यो नातिलङ्घयेत्}
{पृच्छामि त्वां सतां श्रेष्ठ तन्मे ब्रूहि पितामह ॥भीष्म उवाच}
{}


\twolineshloka
{ब्राह्मणानेव सेवेत विद्याबृद्धांस्तपस्विनः}
{श्रुतचारित्रवृत्ताढ्यान्पवित्रं ह्येतदुत्तमम्}


\twolineshloka
{शुश्रूषा तु महाराज सान्त्वं विप्रेषु नित्यदा}
{क्रुद्धैर्हि विप्रैः कर्माणि कृतानि बहुधा नृप}


\twolineshloka
{तेषां प्रीत्या यशो मुख्यमप्रीत्या परमं भयम्}
{प्रीत्या ह्यमृतवद्विप्राः क्रुद्धाश्चैव यथोरगाः}


\chapter{अध्यायः १४३}
\twolineshloka
{युधिष्ठिर उवाच}
{}


\threelineshloka
{पितामह महाप्राज्ञ सर्वशास्त्रविशारद}
{शरणागतं पालयतो यो धर्मस्तं ब्रवीहि मे ॥भीष्म उवाच}
{}


\twolineshloka
{महान्धर्मो महाराज शरणागतपालने}
{अर्हः प्रष्टुं भवांश्चैनं प्रश्नं भरतसत्तम}


\twolineshloka
{शिबिप्रभृतयो राजन्राजानः शरणं गतान्}
{परिपाल्य महात्मानः संसिद्धिं परमां गताः}


\threelineshloka
{श्रूयते च कपोतेन शत्रुः शरणमागतः}
{पूजितश्च यथान्यायं स्वैश्च मांसैर्निमन्त्रितः ॥युधिष्ठिर उवाच}
{}


\threelineshloka
{कथं कपोतेन पुरा शत्रुः शरणमागतः}
{स्वमांसं भोजितः कां च गतिं लेभे स भारत ॥भीष्म उवाच}
{}


\twolineshloka
{शृणु राजन्कथां दिव्यां सर्वपापप्रणाशिनीम्}
{नृपतेर्मुचुकुन्दस्य कथितां भार्गवेण वै}


\twolineshloka
{इममर्थं पुरा पार्थ मुचुकुन्दो नराधिपः}
{भार्गवं परिपप्रच्छ प्रणतः पुरुषर्षभ}


\threelineshloka
{तस्मै शुश्रूषमाणाय भार्गवोऽकथयत्कथाम्}
{इमां यथा कपोतेन सिद्धिः प्राप्ता नराधिप ॥उशनोवाच}
{}


\twolineshloka
{धर्मिश्चयसंयुक्तां कामार्थसहितां कथाम्}
{शृणुष्वावहितो राजन्गदतो मे महाभुजः}


\twolineshloka
{कश्चित्क्षुद्रसमाचारः पृथिव्यां कालसंमितः}
{चचार पृथिवीपाल घोरः शकुनिलुब्धकः}


\twolineshloka
{काकोल इव कृष्णाङ्गो रूक्षः पापसमाहितः}
{यवमध्यः कृशग्रीवो ह्रस्वपादो महाहनुः}


\twolineshloka
{नैव तस्य सुहृत्कश्चिन्न संबन्धी न बान्धवाः}
{बान्धवैः संपरित्यक्तस्तेन रौद्रेण कर्मणा}


\twolineshloka
{नरः पापसमाचारस्त्यक्तव्यो दूरतो बुधैः}
{आत्मानं यो न संधत्ते सोन्यस्य स्यात्कथं हितः}


\twolineshloka
{ये नृशंसा दुरात्मानः प्राणिप्राणहरा नराः}
{उद्वेजनीया भूतानां व्याला इव भवन्ति ते}


\twolineshloka
{स वै क्षारकमादाय वने हत्वा च पक्षिणः}
{चकार विक्रयं तेषां पतङ्गानां जनाधिपः}


\twolineshloka
{एवं तु वर्तमानस्य तस्य वृत्तिं दुरात्मनः}
{अगमत्सुमहान्कालो न चाधर्ममबुध्यत}


\twolineshloka
{तस्य भार्यासहायस्य रममाणस्य शाश्वतम्}
{दैवयोगविमूढस्य नान्या वृत्तिररोचत}


\twolineshloka
{ततः कदाचित्तस्याथ वनस्थस्य समन्ततः}
{पातयन्निव वृक्षांस्तान्सुमहान्वातसंभ्रमः}


\twolineshloka
{मेघसंकुलमाकाशं विद्युन्मण्डलमण्डितम्}
{संछन्नस्तु मुहूर्तेन नौसार्थैरिव सागरः}


\twolineshloka
{वारिधारासमूहेन संप्रहृष्टः शतक्रतुः}
{क्षणेन पूरयामास सलिलेन वसुंधराम्}


\twolineshloka
{ततो धाराकुले लोके संभ्रमन्नष्टचेतनः}
{शीतार्तस्तद्वनं सर्वमाकुलेनान्तरात्मना}


\twolineshloka
{नैव निम्नं स्थलं वाऽपि सोऽविन्दत विहंगहा}
{पूरितो हि जलौघेनन तस्य मार्गो वनस्य तु}


\twolineshloka
{पक्षिणं वर्षवेगेन हता लीनास्तु पादपात्}
{मृगसिहवराहाश्च ये चान्ये तत्र पक्षिणः}


\twolineshloka
{महता वातवर्षेण त्रासितास्ते वनौकसः}
{भयार्ताश्च क्षुधार्ताश्च बभ्रमुः सहिता वने}


\twolineshloka
{स तु शीतहतैर्गात्रैर्जगामैव न तस्थिवान्}
{ददर्श पतितां भूमौ कपोतीं शीतविह्वलाम्}


\twolineshloka
{दृष्टवाऽऽर्तोपि हि पापात्मा स तां पञ्जरकेऽक्षिपत्}
{स्वयं दुःखाभिभूतोऽपि दुःखमेवाकरोत्परे}


\twolineshloka
{पापात्मा पापकारित्वत्पापमेव चकार सः}
{सोऽपश्यत्तरुषण्डेषु मेघनीलं वनस्पतिम्}


\twolineshloka
{सेव्यमानं विहंगौघैश्छायावासफलार्थिभिः}
{धात्रा परोपकाराय स साधुरिव निर्मितः}


\threelineshloka
{अथाभवत्क्षणेनैव वियद्विमलतारकम्}
{महत्सर इवोत्फुल्लं कुमुदच्छुरितोदकम्}
{कुसुमाकारताराढ्यमाकाशं निर्मलं बहु}


\twolineshloka
{घनैर्मुक्तं नभो दृष्ट्वा लुब्धकः शीतविह्वलः}
{दिशो विलोकयामास वेलां च सुदुरात्मवान्}


\twolineshloka
{दूरे ग्रामनिवेशश्च तस्मात्स्थानादिति प्रभो}
{कृतबुद्धिर्द्रुमे तस्मिन्वस्तुं तां रजनीं ततः}


\twolineshloka
{साञ्जलिः प्रणतिं कृत्वा वाक्यमाह वनस्पतिम्}
{शरणं यामि यान्यस्मिन्दैवतानीति भारत}


\twolineshloka
{स शिलायां शिरः कृत्वा पर्णान्यास्तीर्य भूतले}
{दुःखेन महताऽऽविष्टस्ततः सुष्वाप पक्षिहा}


\chapter{अध्यायः १४४}
\twolineshloka
{भीष्म उवाच}
{}


\twolineshloka
{अथ वृक्षस्य शाखायां विहंगः ससुहृज्जनः}
{दीर्घकालोषितो राजंस्तत्र चित्रतनूरुहः}


\twolineshloka
{तस्य कल्यगता भार्या चरितुं नाभ्यवर्तत}
{प्राप्तां च रजनीं दृष्ट्वा स पक्षी पर्यतप्यत}


\twolineshloka
{वातवर्षं महच्चासीन्न चागच्छति मे प्रिया}
{किंनु तत्कारणं येन साऽद्यापि न निवर्तते}


\twolineshloka
{अपि स्वस्ति भवेत्तस्याः प्रियाया मम कानने ॥तया विरहितं हीदं शून्यमद्य गृहं मम}
{}


\twolineshloka
{पुत्रपौत्रवधूभृत्यैराकीर्णमपि सर्वतः}
{भार्याहीनं गृहस्थस्य शून्यमेव गृहं भवेत्}


\twolineshloka
{न गृहं गृहमित्याहुर्गृहिणी गृहमुच्यते}
{गृहं तु गृहिणीहीनमरण्यसदृशं मतम्}


\twolineshloka
{यदि सा रक्तेत्रान्ता चित्राङ्गी मधुरस्वरा}
{अद्य नाभ्येति मे कान्ता न कार्यं जीवितेन मे}


\twolineshloka
{न भुङ्क्ते मय्यभुक्ते या नास्नाते स्नाति सुव्रता}
{नातिष्ठत्युपतिष्ठेन शेते च शयिते मयि}


\twolineshloka
{हृष्टे भवति सा हृष्टा दुःखिते मयि दुःखिता}
{प्रोपिते दीनवदना क्रुद्धे च प्रियवादिनी}


\twolineshloka
{पतिधर्मव्रता साध्वी प्राणेभ्योऽपि गरीयसी}
{यस्य स्यात्तादृशी भार्या धन्यः स पुरुषो भुवि}


\twolineshloka
{सा हि श्रान्तं क्षुधार्तं च जानीते मां तपस्विनी}
{अनुरक्ता स्थिरा चैव भक्ता स्निग्धा यशस्विनी}


\twolineshloka
{वृक्षमूलेऽपि दयिता यस्य तिष्ठति तद्गृहम्}
{प्रासादोपि तया हीनः कान्तार इति निश्चितम्}


\twolineshloka
{धर्मार्थकामकालेषु भार्या पुंसः सहायिनी}
{विदेशगमने चास्य सैव विश्वासकारिका}


\twolineshloka
{भार्या हि परमो ह्यर्थः पुरुषस्येह पट्यते}
{असहायस्य लोकेऽस्मिँल्लोकयात्रासहायिनी}


\twolineshloka
{तथा रोगाभिभूतस्य नित्यं कृच्छ्रगतस्य च}
{नास्ति भार्यासमं मित्रं नरस्यार्तस्य भेषजम्}


\twolineshloka
{नास्ति भार्यासमो बन्धुर्नास्ति भार्यासमा गतिः}
{नास्ति भार्यासमो लोके सहायो धर्मसंग्रहे}


\threelineshloka
{यस्य भार्या गृहे नास्ति साध्वी च प्रियवादिनी}
{अरण्यं तेन गन्तव्यं यथाऽरण्यं तथा गृहम् ॥भीष्म उवाच}
{}


\threelineshloka
{एवं विलपतस्तस्य द्विजस्यार्तस्य वै तदा}
{गृहीता शकुनिघ्नेन भार्या शुश्राव भारतीम् ॥कपोत्युवाच}
{}


\twolineshloka
{अहोऽतीव सुभाग्याऽहं यस्या मे दयितः पतिः}
{असतो वा सतो वाऽपि गुणानेवं प्रभाषते}


\threelineshloka
{सा हि स्त्रीत्यवगन्तव्या यस्य भर्ता तु तुष्यति}
{तुष्टे भर्तरि नारीणां तुष्टाः स्युः सर्वदेवताः}
{अग्निसाक्षिकमप्येतद्भर्ता हि शरणं परम्}


\twolineshloka
{दावाग्निनेव निर्दग्धा सपुष्पस्तबका लता}
{भस्मीभवति सा नारी यस्या भर्ता न तुष्यति}


\twolineshloka
{इति संचिन्त्य दुःखार्ता भर्तारं दुःखितं तदा}
{कपोती लुब्धकेनापि गृहीता वाक्यमब्रवीत्}


\twolineshloka
{हन्त वक्ष्यामि ते श्रेयः श्रुत्वा तु कुरु तत्तथा}
{शरणागतसंत्राता भव कान्त विशेषतः}


\twolineshloka
{एष शाकुनिकः शेते तव वासं समाश्रितः}
{शीतार्तश्च क्षुधार्तश्च पूजामस्मै समाचर}


\twolineshloka
{यो हि कश्चिद्द्विजं हन्याद्गां वा लोकस्य मातरम्}
{शरणागतं च यो हन्यात्तुल्यं तेषां च पातकम्}


\twolineshloka
{अस्माकं विहिता वृत्तिः कापोती जातिधर्मतः}
{सा न्याय्याऽऽत्मवता नित्यं त्वद्विधेनानुवर्तितुं}


\twolineshloka
{यस्तु धर्मं यथाशक्ति गृहस्थो ह्यनुवर्तते}
{स प्रेत्य लभते लोकानक्षयानिति शुश्रुम्}


\threelineshloka
{स त्वं संतानवानद्य पुत्रवानपि च द्विज}
{त्वं स्वदेहे दयां त्यक्त्वा धर्मार्थौ परिगृह्य य}
{पूजामस्मै प्रयुङ्क्ष्व त्वं प्रीयेतास्य मनो यथा}


\twolineshloka
{शरीरे मा च संतापं कुर्वीथास्त्वं विहगंम}
{शरीरयात्रावृत्त्यर्थमन्यान्दारानुपैष्यसि}


\twolineshloka
{इति सा शकुनी वाक्यं पञ्जरस्था तपस्विनी}
{अतिदुःखान्विता प्रोक्त्वा भर्तारं समुदैक्षत}


\chapter{अध्यायः १४५}
\twolineshloka
{भीष्म उवाच}
{}


\twolineshloka
{सपत्न्या वचनं श्रुत्वा धर्मयुक्तिसमन्वितम्}
{हर्षेण महता युक्तो वाक्यं व्याकुललोचनः}


\twolineshloka
{तं वै शाकुनिकं दृष्ट्वा विधिदृष्टेन कर्मणा}
{स पक्षी पूजयामास यत्नात्तं पक्षिजीविनम्}


\twolineshloka
{उवाच स्वागतं तेऽद्य ब्रूहि किं करवाणि ते}
{सतांपश्च न कर्तव्यः स्वगृहे वर्तते भवान्}


\twolineshloka
{तद्ब्रवीतु भवान्क्षिप्रं किं करोमि किमिच्छसि}
{प्रणयेन ब्रवीमि त्वां त्वं हि नः शरणागतः}


\twolineshloka
{अरावप्युचितं कार्यमातिथ्यं गृहमागते}
{छेत्तुमप्यागते छायां नोपसंहरते द्रुमः}


\twolineshloka
{शरणागतस्य कर्तव्यमातिथ्यं हि प्रयत्नतः}
{पञ्चयज्ञप्रवृत्तेन गृहस्थेन विशेषतः}


\twolineshloka
{पञ्चयज्ञांस्तु यो मोहान्न करोति गृहाश्रमी}
{तस्य नायं न च परो लोको भवति धर्मतः}


\twolineshloka
{तद्ब्रूहि मां सुविस्रब्धो यत्त्वं वाचा वदिष्यसि}
{तत्करिष्याम्यहं सर्वं मा त्वं शोके मनः कृथाः}


\twolineshloka
{तस्य तद्वचनं श्रुत्वा शकुनेर्लुब्धकोऽब्रवीत्}
{बाधते खलु मां शीतं संत्राणं हि विधीयताम्}


\twolineshloka
{एवमुक्तस्तनः पक्षी पर्णान्यास्तीर्य भूतले}
{यथा शुष्काणि यत्नेन ज्वलनार्थं द्रुतं ययौ}


\twolineshloka
{स---वाऽङ्गारकर्मान्तं गृहीत्वाऽग्निमथागमत्}
{तथा शुष्केषु पर्णेषु पावकं सोऽप्यदीपयत्}


\twolineshloka
{स--प्तं महत्कृत्वा तमाह शरणागतम्}
{----- सुविस्रब्धः स्वगात्राण्यकुतोभयः}


\twolineshloka
{-- तथोक्तस्तथेत्युक्त्वा लुब्धो गात्राण्यतापयत्}
{अग्निप्रत्यागतप्राणस्ततः प्राह विहंगमम्}


\twolineshloka
{हर्षेण महताऽऽविष्टो वाक्यं व्याकुललोचनः}
{तथेमं शकुनिं दृष्ट्वा विधिदृष्टेन कर्मणा}


\twolineshloka
{दत्तमाहारमिच्छामि त्वया क्षुद्बाधते हि माम्}
{स तद्वचः प्रतिश्रुत्य वाक्यमाह विहंगमः}


\twolineshloka
{न मेऽस्ति विभवो येन नाशयेयं क्षुधां तव}
{उत्पन्नेन हि जीवामो वयं नित्यं वनौकसः}


\twolineshloka
{संचयो नास्ति चास्माकं मुनीनामिव कानने}
{इत्युक्त्वा तं तदा तत्र विवर्णवदनोऽभवत्}


\twolineshloka
{कथं नु खलु कर्तव्यमिति चिन्तापरस्तदा}
{बभूव भरतश्रेष्ठ गर्हयन्वृत्तिमात्मनः}


\twolineshloka
{मुहूर्ताल्लब्धसंज्ञस्तु स पक्षी पक्षिघातिनम्}
{उवाच तर्पयिष्ये त्वां मुहूर्तं प्रतिपालय}


\twolineshloka
{इत्युक्त्वा शुष्कपर्णैस्तु समुज्ज्वाल्य हुताशनम्}
{हर्षेण महताऽऽविष्टः कपोतः पुनरब्रबीत्}


\twolineshloka
{ऋषीणां देवतानां च पितृणां च महात्मनाम्}
{युतः पूर्वं मया धर्मो महानतिथिपूजने}


\twolineshloka
{कुरुष्वानुग्रहं सौम्य सत्यमेतद्ब्रबीमि ते}
{निश्चिता खलु मे बुद्धिरतिथिप्रतिपूजने}


\twolineshloka
{ततः कृतप्रतिज्ञो वै स पक्षी प्रहसन्निव}
{तमग्निं त्रिः परिक्रम्य प्रविवेश महामतिः}


\twolineshloka
{अग्निमध्ये प्रविष्टं तु लुब्धो दृष्ट्वा च पक्षिणम्}
{चिन्तयामास मनसा किमिदं वै मया कृतम्}


\twolineshloka
{अहो मम नृशंसस्य गर्हितस्य स्वकर्मणा}
{अधर्मः सुमहान्धोरो भविष्यति न संशयः}


\twolineshloka
{एवं बहुविधं भूरि विललाप स लुब्धकः}
{गर्हयन्स्वानि कर्माणि द्विजं दृष्ट्वा तथाऽऽगतम्}


\chapter{अध्यायः १४६}
\twolineshloka
{भीष्म उवाच}
{}


\twolineshloka
{ततः स लुब्धकः पश्यन्क्षुधयाऽपि परिप्लुतः}
{कपोतमग्निपतितं वाक्यं पुनरुवाच ह}


\threelineshloka
{किमीदृशं नृशंसेन मया कृतमबुद्धिना}
{भविष्यति हि मे नित्यं पातकं भुवि जीवतः}
{स विनिन्दंस्तथाऽऽत्मानं पुनः पुनरुवाच ह}


\twolineshloka
{धिङ्भामस्तु सुदुर्बुद्धिं सदा निकृतिनिश्चयम्}
{शुभं कर्म परित्यज्य सोऽहं शकुनिलुब्धकः}


\twolineshloka
{नृशंसस्य ममाद्यायं प्रत्यादेशो न संशयः}
{दत्तः स्वमांसं दहता कपोतेन महात्मना}


\twolineshloka
{सोहं त्यक्ष्ये प्रियान्प्राणान्पुत्रान्दारान्विसृज्य च}
{उपदिष्टो हि मे धर्मः कपोतेनात्र धर्मिणा}


\twolineshloka
{अद्यप्रभुति देहं स्वं सर्वभोगैर्विवर्जितम्}
{यथा स्वल्पं सरो ग्रीष्मे शोषयिष्याम्महं तथा}


\twolineshloka
{क्षुत्पिपासातपसहः कृशो ध्रमनिसन्ततः}
{उपवासैर्बहुविधैश्चरिष्ये पारलौकिकम्}


\twolineshloka
{अहो देहप्रदानेन दर्शिताऽतिथिपूजना}
{तस्माद्धर्मं चरिष्यामि धर्मो हि परमा गतिः}


\twolineshloka
{दृष्टो धर्मो हि धर्मिष्ठे यादृशो विहगोत्तमे}
{एवमुक्त्वा विनिश्चित्य रौद्रकर्मा स लुब्धकः}


% Check verse!
महाप्रस्थानमाश्रित्य प्रययौ संशितव्रतः
\twolineshloka
{ततो यष्टिं शलाकां च धारकं पञ्जरं तथा}
{तां च बद्धां कपोतीं स प्रमुच्य विससर्ज ह}


\chapter{अध्यायः १४७}
\twolineshloka
{भीष्म उवाच}
{}


\twolineshloka
{ततो गतो शाकुनिके कपोती प्राह दुःखिता}
{संस्मृत्य सा च भर्तारं रुदती शोककर्शिता}


\twolineshloka
{नाहं ते विप्रियं कान्त कदाचिदपि संस्मरे}
{सर्वाऽपि विधवा नारी बहुपुत्राऽपि शोचते}


\twolineshloka
{शोच्या भवति बन्धूनां पतिहीना तपस्विनी}
{लालिताऽहं त्वया नित्यं बहुमानाच्च पूजिता}


\twolineshloka
{वचनैर्मधुरैः स्निग्धैरसंक्लिष्टमनोहरैः}
{कन्दरेषु च शैलानां नदीनां निर्झरेषु च}


\threelineshloka
{द्रुमाग्रेषु च रम्येषु रमिताऽहं त्वया सह}
{आकाशगमने चैव विहृताऽहं त्वया सुखम्}
{रमामि स्म पुरा कान्त तन्मे नास्त्यद्य मे प्रिय}


\twolineshloka
{मितं ददाति हि पिता मितं भ्राता मितं सुतः}
{अमितस्य हि दातारं भर्तारं का न पूजयेत्}


\twolineshloka
{नास्ति भर्तृसमो नाथो नास्ति भर्तृसमं सुखम्}
{विसृज्य धनसर्वस्वं भर्ता वै शरणं स्त्रियाः}


\twolineshloka
{न कार्यमिह मे नाथ जीवितेन त्वया विना}
{पतिहीना तु का नारी सती जीवितुमुत्सहेत्}


\twolineshloka
{एवं विलप्य बहुधा करुणं सा सुदुःखिता}
{पतिव्रता संप्रदीप्तं प्रविवेश हुताशनम्}


\twolineshloka
{ततश्चित्राङ्गदधरं भर्तारं साऽन्वपद्यत}
{विमानस्थं सुकृतिभिः पूज्यमानं महात्मभिः}


% Check verse!
चित्रमाल्याम्बरधरं सर्वाभरणभूषितम् ॥विमानशतकोटीभिरावृतं पुण्यकर्मभिः
\twolineshloka
{ततः स्वर्गं गतः पक्षी विमानवरमास्थितः}
{कर्मणा पूजितस्तत्र रेमे स सह भार्यया}


\chapter{अध्यायः १४८}
\twolineshloka
{भीष्म उवाच}
{}


\twolineshloka
{वमानस्थौ तु तौ राजंल्लुब्धकः ग्वे ददर्श ह}
{दृष्ट्वा तौ दंपती राजन्व्यचिन्तयत तां गतिम्}


\twolineshloka
{कीदृशेनेह तपसा गच्छेयं परमां गतिम्}
{इति बुद्ध्या विनिश्चित्य गमनायोपचक्रमे}


\twolineshloka
{महाप्रस्थानमाश्रित्य लुब्धकः पक्षिजीवकः}
{निश्चष्टो मरुदाहारो निर्ममः स्वर्गकाङ्क्षया}


\twolineshloka
{ततोऽपश्यत्सुविस्तीर्णं हृद्यं पद्माभिभूषितम्}
{नानापक्षिगणाकीर्णं सरः शीतजलं शिवम्}


% Check verse!
पिपांसार्तोऽपि तदृष्ट्वा तृप्तः स्यान्नात्र संशयः
\twolineshloka
{उपवासकृशोऽत्यर्थं स तु पार्थिव लुब्धकः}
{उपसृत्य तु तद्धृष्टः श्वापदाध्युपितं वनम्}


\twolineshloka
{महान्तं निश्चयं कृत्वा लुब्धकः प्रविवेश ह}
{प्रविशन्नेव स वनं निगृहीतः स कण्टकैः}


\twolineshloka
{स कण्टकैर्विभिन्नाङ्गो लोहितार्द्रीकृतच्छविः}
{वभ्राम तस्मिन्विजने नानामृगसमाकुले}


\twolineshloka
{ततो द्रुमाणां महतां पवनेन वने तदा}
{उदतिष्ठत संघर्षान्सुमहान्हव्यवाहनः}


\twolineshloka
{तद्वनं वृक्षसंकीर्णं लताविटपसंकुलम्}
{ददाह पावकः क्रुद्धो युगान्ताग्निसमप्रभः}


\twolineshloka
{स ज्वलैः पवनोद्भूतैर्विस्फुलिङ्गैः समन्ततः}
{ददाह तद्वनं घोरं मृगपक्षिसमाकुलम्}


\twolineshloka
{ततः स देहमोक्षार्थं संप्रहृष्टेन चेतसा}
{अभ्यधावत वर्धन्तं पावकं लुब्धकस्तदा}


\twolineshloka
{ततस्तेनाग्निना दग्धो लुब्धको नष्टकल्मषः}
{जगाम परमां सिद्धिं ततो भरतसत्तम}


\twolineshloka
{ततः स्वर्गस्थमात्मानमपश्यद्विगतज्वरः}
{यक्षगन्धर्वसिद्धानां मध्ये भ्राजन्तमिन्द्रवत्}


\twolineshloka
{एवं खलु कपोतश्च कपोती च पतिव्रता}
{लुब्धकेन सह स्वर्गं गताः पुण्येन कर्मणा}


\twolineshloka
{याऽन्या चैवंविधा नारी भर्तारमनुवर्तते}
{विराजते हि सा क्षिप्रं कपोतीव दिवि स्थिता}


\twolineshloka
{एवमेतत्पुरावृत्तं लुब्धकस्य महात्मनः}
{कपोतस्य च धर्मिष्ठा गतिः पुण्येन कर्मणा}


\twolineshloka
{यश्चेदं शृणुयान्नित्यं यश्चेदं परिकीर्तयेत्}
{नाशुभं विद्यते तस्य मनसाऽपि प्रमादतः}


\twolineshloka
{युधिष्ठिर महानेष धर्मो धर्मभृतां वर}
{गोघ्नेष्वपि भवेदस्मिन्निष्कृतिः पापकर्मणः}


\threelineshloka
{न निष्कृतिर्भवेत्तस्य यो हन्याच्छरणागतम्}
{इतिहासमिमं श्रुत्वा पुण्यं पापप्रणाशनम्}
{न दुर्गतिप्रवाप्नोति स्वर्गलोकं च गच्छति}


\chapter{अध्यायः १४९}
\twolineshloka
{युधिष्ठिर उवाच}
{}


\threelineshloka
{अबुद्धिपूर्वं यत्पापं कुर्याद्भरतसत्तम}
{मुच्यते स कथं तस्मादेनसस्तद्ब्रवीहि मे ॥भीष्म उवाच}
{}


\twolineshloka
{अत्र ते वर्तयिष्यामि पुराणमृषिसंस्तुतम्}
{इन्द्रोतः शौनको विप्रो यदाह जनमेजयम्}


\twolineshloka
{आसीद्राजा महावीर्यः पारिक्षिज्जनमेजयः}
{अबुद्धिजा ब्रह्महत्या तमागच्छन्महीपतिम्}


\twolineshloka
{ब्राह्मणाः सर्व एवैनं तत्यजुः सपुरोहिताः}
{स जगाम वनं राजा दह्यमानो दिवानिशम्}


\twolineshloka
{प्रजाभिः स परित्यक्तश्चकार कुशलं महत्}
{अतिवेलं तपस्तेपे दह्यमानः स मन्युना}


\twolineshloka
{ब्रह्महत्यापनोदार्थमपृच्छद्ब्राह्मणान्बहून्}
{पर्यटन्पृथिवीं कृत्स्नां देशेदेशे नराधिपः}


\twolineshloka
{तत्रेतिहासं वक्ष्यामि धर्मस्यास्योपवृंहणम्}
{दह्यमानः पापकृत्या जगाम जनमेजयः}


\twolineshloka
{चरिष्यमाण इन्द्रोतं शौनकं संशितव्रतम्}
{समासाद्योपजग्राह पादयोः परिपीडयन्}


\twolineshloka
{ततो भीतो महाप्राज्ञो जगर्हे सुभृशं तदा}
{कर्ता पापस्य महतो भ्रूणहा किमिहागतः}


\twolineshloka
{किं तवास्मासु कर्तव्यं मा मां द्राक्षीः कथंचन}
{गच्छगच्छ न ते स्थानं प्रीणात्यस्मानिति ब्रुवन्}


\twolineshloka
{रुधिरस्येव ते गन्धः शवस्येव च दर्शनम्}
{अशिवः शिवसंकाशो मृतो जीवन्निवाटसि}


\twolineshloka
{अन्तर्भृत्युरशुद्धात्मा पापमेवानुचिन्तयन्}
{प्रबुध्यसे प्रस्वपिपि वर्तसे परमे सुखे}


\twolineshloka
{मोघं ते जीवितं राजन्परिक्लिष्टं च जीवसि}
{पापायैव हि सृष्टोऽसि कर्मणेह यवीयसे}


\twolineshloka
{बहुकल्याणमिच्छन्त ईहन्ते पितरः सुतान्}
{तपसा दैवतेज्याभिर्वन्दनेन तितिक्षया}


\twolineshloka
{पितृवंशमिमं पश्य त्वत्कृते निधनं गतम्}
{निरर्थाः सर्व एवैषामाशाबन्धास्त्वदाश्रयाः}


\twolineshloka
{यान्पूजयन्तो विन्दन्ति स्वर्गमायुर्यशः प्रजाः}
{तेषु ते संततं द्वेषो ब्राह्मणेषु निरर्थकः}


\twolineshloka
{इमं लोकं विमुच्य त्वमवाङ्भूर्धा पतिष्यसि}
{अशाश्वतीः शाश्वतीश्च समाः पापेन कर्मणा}


\twolineshloka
{स्वाद्यमानो जन्तुशतैस्तीक्ष्णदंष्ट्रैरयोमुखैः}
{ततश्च पुनरावृत्तः पापयोनिं गमिष्यसि}


\twolineshloka
{यदिदं मन्यसे राजन्नायमस्ति कुतः परः}
{प्रतिस्मारयितारस्त्वां यमदूता यमक्षये}


\chapter{अध्यायः १५०}
\twolineshloka
{भीष्म उवाच}
{}


\twolineshloka
{एवमुक्तः प्रत्युवाच तं मुनिं जनमेजयः}
{गर्ह्यं भवान्गर्हयते निन्द्यं निन्दति मां पुनः}


\twolineshloka
{धिक्कार्यं मां धिक्कुरुते तस्मात्त्वाऽहं प्रसादये}
{सर्वं हीदं स्वकृतं मे ज्वलाम्यग्नाविवाहितम्}


\twolineshloka
{स्वकर्माण्यभिसंधाय नाभिनन्दति मे मनः}
{प्राप्तं घोरं भयं नूनं मया वैवस्वतादपि}


\twolineshloka
{तत्तु शल्यमनिर्हृत्य कथं शक्ष्यामि जीवितुम्}
{सर्वं मन्युं विनीय त्वमभि मां वद शौनक}


\threelineshloka
{[महानासं ब्राह्मणानां भूयो वक्ष्यामि सांप्रतम्}
{]`गन्ता गतिं ब्राह्मणानां भविष्याम्यर्थवान्पुनः}
{'अस्तु शेषं कुलस्यास्य मा पराभूदिदं कुलम्}


\twolineshloka
{न हि नो ब्रह्मशप्तानां शेषं भवितुमर्हति}
{स्तुतीरलभमानानां संविदं वेद निश्चयात्}


\twolineshloka
{निन्दमानः स्वमात्मानं भूयो वक्ष्यामि सांप्रतम्}
{भूयश्चैवाभिमज्जन्ति निर्धर्मा निर्जला इव}


\twolineshloka
{न ह्ययज्ञा अमुं लोकं प्राप्नुवन्ति कथंचन}
{अवाक्च प्रपतिष्यन्ति पुलिन्दशवरा इव}


\threelineshloka
{अविज्ञायैव मे प्रज्ञां बालस्येव स पण्डितः}
{ब्रह्मन्पितेव पुत्रस्य प्रीतिमान्भव शौनक ॥शौनक उवाच}
{}


\twolineshloka
{किमाश्चर्यं यतः प्राज्ञो बहुकुर्यादसांप्रतम्}
{इति वै पण्डितो भूत्वा भूतानां को नु तप्यते}


\twolineshloka
{प्रज्ञाप्रासादमारुह्य अशोच्यः शोचते जनान्}
{जगतीस्थानिवाद्रिस्थः प्रज्ञया प्रतिपत्स्यति}


\twolineshloka
{न चोपलभते कश्चिन्न चाश्चर्याणि पश्यति}
{निर्विण्णात्मा परोक्षो वा धिक्कृतः सर्वसाधुषु}


\twolineshloka
{विदित्वा भवतो वीर्यं माहात्म्यं चैव चागमे}
{कुरुष्वेह यथाशान्ति ब्रह्मा शरणमस्तु ते}


\threelineshloka
{तद्वै वारित्रकं तात ब्राह्मणानामकुप्यताम्}
{अथवा तप्यसे पापे धर्मं चेदनुपश्यसि ॥जनमेजय उवाच}
{}


\threelineshloka
{अनुतप्ये च पापेन न चाधर्मं चराम्यहम्}
{बुभूषेद्भजमानं च प्रीतिमान्भव शौनक ॥शौनक उवाच}
{}


\twolineshloka
{छित्त्वा दम्भं च मानं च प्रीतिमिच्छामि ते नृप}
{सर्वभूतहिते तिष्ठ धर्मं चैव प्रतिस्मरन्}


\twolineshloka
{न भयान्न च कार्पण्यान्न लोभात्त्वामुपाह्वये}
{तां मे दैवीं गिरं सत्यां शृणु त्वं ब्राह्मणैः सह}


\twolineshloka
{सोऽहं न केनचिच्चार्थी त्वां च धर्मादुपाह्वये}
{क्रोशतां सर्वभूतानां हाहाधिगिति जल्पताम्}


\twolineshloka
{वक्ष्यन्ति मामधर्मज्ञं त्यक्ष्यन्ति सुहृदो जनाः}
{ता वाचः सुहृदः श्रुत्वा संज्वरिष्यन्ति मे भृशं}


\twolineshloka
{केचिदेव महाप्राज्ञाः प्रतिज्ञास्यन्ति कार्यताम्}
{जानीहि मत्कृतं तात ब्राह्मणान्प्रति भारत}


\threelineshloka
{यथा ते सत्कृताः क्षेमं लभेरंस्त्वं तथा कुरु}
{प्रतिजानीहि चाद्रोहं ब्राह्मणानां नराधिप ॥जनमेजय उवाच}
{}


\twolineshloka
{नैव वाचा न मनसा पुनर्जातु न कर्मणा}
{द्रोग्धाऽस्मि ब्राह्मणान्विप्र चरणावेव ते स्पृशे}


\chapter{अध्यायः १५१}
\twolineshloka
{शौनक उवाच}
{}


\twolineshloka
{तस्मात्तेऽहं प्रवक्ष्यामि धर्ममावृतचेतसे}
{श्रीमन्महाबलस्तुष्टः स्वयं धर्ममवेक्षसे}


\twolineshloka
{पुरस्ताद्दारुणे भूत्वा सुचित्रतरमेव तत्}
{अनुगृह्णाति भूतानि स्वेन वृत्तेन पार्थिवः}


\threelineshloka
{कृत्स्ने नूनं सदसती इति लोको व्यवस्यति}
{यत्र त्वं तादृशो भूत्वा धर्ममेवानुपश्यसि}
{}


\twolineshloka
{दर्पं हित्वा पुनश्चापि भोगांश्च तप आस्थितः}
{इत्येतदभिभूतानामद्भुतं जनमेजय}


\twolineshloka
{योऽदुर्बलो भवेद्दाता कृपणो वा तपोधनः}
{अनाश्चर्यं तदित्याहुर्नातिदूरेण वर्तते}


\twolineshloka
{तप एव हि कार्पण्यं समग्रमसमीक्षितम्}
{तच्चेत्समीक्षयैव स्याद्भवेत्तस्मिंस्तपो गुणः}


\twolineshloka
{यज्ञो दानं दया वेदाः सत्यं च पृथिवीपते}
{पञ्चैतानि पवित्राणि षष्ठं सुचरितं तपः}


\twolineshloka
{तदेव राज्ञां परमं पवित्रं जनमेजय}
{तेन सम्यग्गृहीतेन श्रेयांसं धर्ममाप्स्यसि}


\twolineshloka
{पुण्यदेशाभिगमनं पवित्रं परमं स्मृतम्}
{अत्राप्युदाहरन्तीमां गाथां गीतां ययातिना}


\twolineshloka
{यो मर्त्यः प्रतिपद्येत आयुर्जीवेन वा पुनः}
{यज्ञमेकं ततः कृत्वा तत्संन्यस्य तपश्चरेत्}


\twolineshloka
{पुण्यमाहुः कुरुक्षेत्रं कुरुक्षेत्रात्सरस्वतीम्}
{सरस्वत्याश्च तीर्थानि तीर्थेभ्यश्च पृथूदकम्}


\twolineshloka
{यत्रावगाह्य स्थित्वा च नैनं श्वोमरणं तपेत्}
{महासरः पुष्कराणि प्रभासोत्तरमानसे}


\threelineshloka
{कालोदकं च गन्तासि लब्धायुर्जीविते पुनः}
{सरस्वतीदृषद्वत्योः सेवमानोऽनुसंज्वरेत्}
{स्वाध्यायशील एतेषु सर्वेष्वेवमुपस्पृशेत्}


% Check verse!
त्यागधर्मं पवित्राणां संन्यासं मनुरब्रवीत्
\threelineshloka
{अत्राप्युदाहरन्तीमाः गाथाः सत्यवता कृताः}
{यथा कुमारः सत्यो वै नैव पुण्यो न पापकृत}
{न ह्यस्ति सर्वभूतेषु दुःखमस्मिन्कुतः सुखम्}


\threelineshloka
{एवं प्रकृतिभूतानां सर्वसंसर्गयायिनाम्}
{त्यजतां जीवितं प्रायो निवृत्ते पुण्यपापके}
{}


% Check verse!
यत्त्वेव राज्ञो ज्यायिष्ठं कार्याणां तद्ब्रवीमि ते
\twolineshloka
{बलेन संविभागैश्च जय स्वर्गं पुनीष्व च}
{यस्यैव बलमोजश्च स धर्मस्य प्रभुर्नरः}


\twolineshloka
{ब्राह्मणानां सुखार्थं त्वं पर्येहि पृथिवीमिमाम्}
{यथैवैतान्पुरा क्षेप्सीस्तथैवैतान्प्रसादय}


\threelineshloka
{अपि धिक््क्रियमाणोऽपि तर्ज्यमानोऽप्यनेकधा}
{आत्मनो दर्शनं विद्वान्नाहर्ताऽस्मीति मा क्रुधः}
{घटमानः स्वकार्येषु कुरु निःश्रेयसं परम्}


\twolineshloka
{हिमाग्निघोरसदृशो राजा भवति कश्चन}
{लाङ्गलाशिकल्पो वा भवेदन्यः परंतपः}


\twolineshloka
{न विशेषेण गन्तव्यमचिकित्सेन वा पुनः}
{न जातु नाहमस्मीति प्रसक्तव्यमसाधुषु}


\twolineshloka
{विकर्मणा तप्यमानः पापात्पापः प्रमुच्यते}
{नैतत्कुर्या पुनरिति द्वितीयात्परिमुच्यते}


\twolineshloka
{चरिष्ये धर्ममेवेति तृतीयात्परिमुच्यते}
{शुचिस्तीर्थान्यनुचरन्बहुत्वात्परिमुच्यते}


\twolineshloka
{कल्याणमनुकर्तव्यं पुरुषेण बुभूषता}
{ये सुगन्धीनि सेवन्ते तथागन्धा भवन्ति ते}


\twolineshloka
{ये दुर्गन्धीनि सेवन्ते तथागन्धा भवन्ति ये}
{तपश्चर्यापरः सत्यं पापाद्विपरिमुच्यते}


\twolineshloka
{संवत्सरमुपास्याग्निमभिशस्तः प्रमुच्यते}
{त्रीणि वर्षाण्युपास्याग्निं भ्रूणहा विप्रमुच्यते}


\twolineshloka
{महासरः पुष्कराणि प्रभासोत्तरमानसे}
{अभ्येत्य योजनशतं भ्रूणहा विप्रमुच्यते}


\twolineshloka
{यावतः प्राणिनो हन्यात्तज्जातीयांस्तु तावतः}
{प्रमीयमाणानुन्मोच्य प्राणिहा विप्रमुच्यते}


\twolineshloka
{अपि चाप्सु निमज्जेत जपंस्त्रिरघमर्षणम्}
{यथाऽश्वमेधावभृथस्तथा तन्मनुरब्रवीत्}


\twolineshloka
{क्षिप्रं प्रणुदते पापं सत्कारं लभते तथा}
{अपि चैनं प्रसीदन्ति भूतानि जडमूकवत्}


\twolineshloka
{बृहस्पतिं देवगुरुं सुरासुराःसमेत्य सर्वे नृपते त्वयुज्जत}
{धर्मे फलं हेतुकृते महर्षेतथेतरस्मिन्नरके पापलोक्ये}


\threelineshloka
{उभे तु यस्य सुकृते भवेतांकिं तत्तयोस्तत्र जयोत्तरं स्यात्}
{आचक्ष्व तत्कर्मफलं महर्षेकथं पापं नुदते धर्मशीलः ॥बृहस्पतिरुवाच}
{}


\twolineshloka
{कृत्वा पापं पूर्वमबुद्धिपूर्वंपुण्यानि चेत्कुरुते बुद्धिपूर्वम्}
{स तत्पापं नुदते कर्मशीलोवासो यथा मलिनं क्षारयुक्त्या}


\twolineshloka
{पापं कृत्वा हि मन्येत नाहमस्तीति पुरुषः}
{चिकीर्षेदेव कल्याणं श्रद्दधानोऽनसूयकः}


\twolineshloka
{छिद्राणि वसनस्येव साधुना संवृणोति सः}
{यः पापं पुरुषः कृत्वा कल्याणमभिपद्यते}


\threelineshloka
{आदित्यः पुनरुद्यन्वा तमः सर्वं व्यपोहति}
{कल्याणमाचरन्नेवं सर्वपापं व्यपोहति ॥भीष्म उवाच}
{}


\twolineshloka
{एवमुक्त्वा तु राजानमिन्द्रोतो जनमेजयम्}
{याजयामास विधिवद्वाजिमेधेन शौनकः}


\twolineshloka
{ततः स राजा व्यपनीतकल्मषःश्रिया युतः प्रज्वलितोऽनुरूपया}
{विवेश राज्यं स्वममित्रकर्शनोयथा दिवं पूर्णवपुर्निशाकरः}


\chapter{अध्यायः १५२}
\twolineshloka
{युधिष्ठिर उवाच}
{}


\threelineshloka
{कच्चित्पितामहेनासीच्छ्रुवं वा दृष्टमेव च}
{कच्चिन्मर्त्यो मृतो राजन्पुनरुज्जीवितोऽभवत् ॥भीष्म उवाच}
{}


\twolineshloka
{शृणु पार्थ यथावृत्तमितिहासं पुरातनम्}
{गृध्रजम्बुकसंवादं यो वृत्तो वैदिशे पुरे}


\twolineshloka
{कस्यचिद्ब्राह्मणस्यासीद्दुःखलब्धः सुतो मृतः}
{बाल एव विशालाक्षो बालग्रहनिपीडितः}


\twolineshloka
{दुःखिताः केचिदादाय बालमप्राप्तयोवनम्}
{कुलसर्वस्वभूतं वै रुदन्तः शोककर्शिताः}


\twolineshloka
{बालं मृतं गृहीत्वाऽथ श्मशानाभिमुखाः स्थिताः}
{अङ्गेनाङ्गं समाक्रस्य रुरुदुर्भृशदुःखिताः}


\twolineshloka
{शोचन्तस्तस्य पूर्वोक्तान्भाषितांश्चासकृत्पुनः}
{तं बालं भूतले क्षिप्य प्रतिगन्तुं न शक्नुयुः}


\twolineshloka
{तेषां रुदितशब्देन गृध्रोऽभ्येत्य वचोऽब्रवीत्}
{प्रेतात्मकमिमंकाले त्यक्त्वा गच्छत माचिरम्}


\twolineshloka
{इह पुंसां सहस्राणि स्त्रीसहस्राणि चैव ह}
{समानीतानि कालेन हित्वा वै यान्ति बान्धवाः}


\twolineshloka
{संपश्यत जगत्सर्वं सुखदुःखैरधिष्ठितम्}
{संयोगो विप्रयोगश्च पर्यायेणोपलभ्यते}


\twolineshloka
{गृहीत्वा ये च गच्छन्ति येऽनुयान्ति च तान्मृतान्}
{तेऽप्यायुषः प्रमाणेन स्वेन गच्छन्ति जन्तवः}


\twolineshloka
{अलं स्थित्वा श्मशानेऽस्मिन्गृध्रगोमायुसंकुले}
{कङ्कालबहुले घोरे सर्वप्राणिभयंकरे}


\twolineshloka
{न पुनर्जीवितः कश्चित्कालधर्ममुपागतः}
{प्रियो वा यदि वा द्वेष्यः प्राणिनां गतिरीदृशी}


\twolineshloka
{सर्वेण खलु मर्तव्यं मर्त्यलोके प्रसूयता}
{कृतान्तविहिते मार्गे मृतं को जीवयिष्यति}


\twolineshloka
{दिशान्तोपचितो यावदस्तं गच्छति भास्करः}
{गम्यतां स्वमधिष्ठानं सुतस्नेहं विसृज्य वै}


\twolineshloka
{ततो गृध्रवचः श्रुत्वा विक्रोशन्तस्तदा नृप}
{बान्धवास्तेऽभ्यगच्छन्त पुत्रमुत्सृज्य भूतले}


\twolineshloka
{विनिश्चित्याथ च तदा विक्रोशन्तस्ततस्ततः}
{[मृत इत्येव गच्छन्तो निराशास्तस्य दर्शने ॥]}


\twolineshloka
{निश्चितार्थाश्च ते सर्वे संत्यजन्तः स्वमात्मजम्}
{निराशा जीविते तस्य मार्गमावृत्य धिष्ठिताः}


\twolineshloka
{ध्वाङ्क्षपक्षसवर्णस्तु बिलान्निःसृत्य जम्बुकः}
{गच्छमानान्स्म तानाह निर्घृणाः खलु मानुषाः}


\twolineshloka
{आदित्योऽयं स्थितो मूढाः स्नेहं कुरुत मा भयम्}
{बहुरूपो मुहूर्ताच्च जीवेदपि च बालकः}


\twolineshloka
{दर्भान्भूमौ विनिक्षिप्य पुत्रस्नेहविनाकृताः}
{श्मशाने सुतमुत्सृज्य कस्माद्गच्छत निर्घृणाः}


\twolineshloka
{न वोऽस्त्यस्मिन्सुते स्नेहो बाले मधुरभाषिणि}
{यस्य भाषितमात्रेण प्रसादमधिगच्छत}


\twolineshloka
{न पश्यध्वं सुतस्नेहो यादृशः पशुपक्षिणाम्}
{न तेषां धारयित्वा तान्कश्चिदस्ति फलागमः}


\twolineshloka
{चतुष्पात्पक्षिकीटानां प्राणिनां स्नेहसङ्गिनाम्}
{परलोकगतिस्थानां मुनियज्ञक्रियामिव}


\twolineshloka
{तेषां पुत्राभिरामाणामिह लोके परत्र च}
{न गुणो दृश्यते कश्चित्प्रजाः संधारयन्ति च}


\twolineshloka
{अपश्यतां प्रियान्पुत्रान्येषां शोको न तिष्ठति}
{न ते पोषणसंप्रीता मातापितर एव हि}


\twolineshloka
{मानुषाणां कुतः स्नेहो येषां शोको न विद्यते}
{इमं कुलकरं पुत्रं त्यक्त्वा क्व नु गमिष्यथ}


\twolineshloka
{चिरं मुञ्चत बाष्पं च चिरं स्नेहेनन पश्यत}
{एवंविधानि हीष्टानि दुस्त्यजानि विशेषतः}


\twolineshloka
{क्षीणस्याथामिशस्तस्य श्मशानाभिमुखस्य च}
{बान्धवा यत्र तिष्ठति तत्रान्यो नाधितिष्ठति}


\twolineshloka
{सर्वस्य दयिताः प्राणाः सर्वः स्नेहं च विन्दति}
{तिर्यग्योनिष्वपि संतां स्नेहं पश्यत यादृशम्}


\threelineshloka
{त्यक्त्वा कथं गच्छथेमं पद्मलोलायतेक्षणम्}
{यथा नवोद्वाहकृतं स्नानमाल्यविभूषितम्}
{}


\threelineshloka
{जम्बुकस्य वचः श्रुत्वा कृपणं परिदेवतः}
{न्यवर्तन्त तदा सर्वे बालार्थं ते स्म मानुषाः ॥गृध्र उवाच}
{}


\twolineshloka
{अहो बत नृशंसेन जम्बुकेनाल्पमेधसा}
{क्षुद्रेणोक्ता हीनसत्वा मानुषाः किं निवर्तथ}


\twolineshloka
{पञ्चभूतपरित्यक्तं शुष्कं काष्ठत्वमागतम्}
{कस्माच्छोचथ निश्चेष्टमात्मानं किं न शोचथ}


\twolineshloka
{तपः कुरुत वै तीव्रं मुच्यध्वं येन किल्बिषात्}
{तपसा लभ्यते सर्वं विलापः किं करिष्यति}


\twolineshloka
{अनिष्टानि न भाग्यानि जानीत स्वंस्वमात्मना}
{येन गच्छति बालोऽयं दत्त्वा शोकमनन्तकम्}


\twolineshloka
{धनं गावः सुवर्णं च मणिरत्नमथापि च}
{अपत्यं च तपोमूलं तपो योगाच्च लभ्यते}


\twolineshloka
{यथा कृता च भूतेषु प्राप्यते सुखदुःखिता}
{गृहीत्वा जायते जन्तुर्दुःखानि च सुखानि च}


\twolineshloka
{न कर्मणा पितुः पुत्रः पिता व्रा पुत्रकर्मणा}
{मार्गेणान्येन गच्छन्ति बद्धाः सुकृतदुष्कृतैः}


\twolineshloka
{धर्मं चरत यत्नेन तथाऽधर्मान्निवर्तत}
{वर्तध्वं च यथाकालं दैवतेषु द्विजेषु च}


\twolineshloka
{शोकं त्यजत दैन्यं च सुतस्नेहान्निवर्तत}
{त्यज्यतामयमाक्रोशस्ततः शीघ्रं निवर्तत}


\twolineshloka
{यत्करोति शुभं कर्म तथा कर्म सुदारुणम्}
{तत्कर्तैव समश्नाति बान्धवानां किमत्र ह}


\twolineshloka
{इह त्यक्त्वा न तिष्ठन्ति बान्धवा बान्धवं प्रियम्}
{स्नेहमुत्सृज्य गच्छन्ति बाष्पपूर्णाविलेक्षणाः}


\twolineshloka
{प्राज्ञो वा यदि वा मूर्खः सधनो निर्धनोऽपि वा}
{सर्वः कालवशं याति शुभाशुभसमन्वितः}


\twolineshloka
{किं करिष्यथ शोचित्वा मृतं किमनुशोचथ}
{सर्वस्य हि प्रभुः कालो धर्मतः समदर्शनः}


\threelineshloka
{यौवनस्थांश्च बालांश्च बृद्धान्गर्भगतानपि}
{सर्वानाविशते मृत्युरेवंभूतमिदं जगत् ॥जम्बुक उवाच}
{}


\twolineshloka
{अहो मन्दीकृतः स्नेहो गृध्रेणेहाल्पबुद्धिना}
{पुत्रस्नेहाभिभूतानां युष्माकं शोचतां भृशम्}


\threelineshloka
{समैः सम्यक्प्रयुक्तैश्च वचनैर्हेतुदर्शनैः}
{`सर्वमेतत्प्रपद्याशु कुरुध्वं वा विचारणां}
{'यद्गच्छथ जलस्थानं स्नेहमुत्सृज्य दुस्त्यजम्}


\twolineshloka
{अहो पुत्रवियोगेन मृतशून्योपसेवनात्}
{क्रोशतां वा भृशं दुःखं विवत्सानां गवामिव}


\twolineshloka
{अद्य शोकं विजानामि मानुषाणां महीतले}
{स्नेहं हि कारणं कृत्वा ममाप्यश्रूण्यथापतन्}


\twolineshloka
{यत्नो हि सततं कार्यस्ततो दैवेन सिद्ध्यति}
{दैवं पुरुषकारश्च कृतान्तेनोपपद्यते}


\twolineshloka
{अनिर्वेदः सदा कार्यो निर्वेदाद्धि कुतः सुखम्}
{प्रयत्नात्प्राप्यते ह्यर्थः कस्माद्गच्छथ निर्दयम्}


\twolineshloka
{आत्ममांसोपवृत्तं च शरीरार्धमर्यी तनुम्}
{पितॄणां वंशकर्तारं वने त्यक्त्वा क्व यास्यथ}


\threelineshloka
{अथवाऽस्तं गते सूर्ये संध्याकाल उपस्थिते}
{ततो नेष्यश्च वा पुत्रमिहस्था वा भविष्यथ ॥गृध्र उवाच}
{}


\twolineshloka
{अद्य वर्षसहस्रं मे साग्रं जातस्य मानुषाः}
{न च पश्यामि जीवन्तं मृतं स्त्रीपुंनपुंसकम्}


\twolineshloka
{मृता गर्भेषु जायन्ते जातमात्रा म्रियन्ति च}
{चंक्रमन्तो म्रियन्ते च यौवनस्थास्तथा परे}


\twolineshloka
{अनित्यानीह भाग्यानि चतुष्पात्पक्षिणामपि}
{जङ्गमाजङ्गमानां च ह्यायुरग्रेऽवतिष्ठते}


\twolineshloka
{इष्टदारवियुक्ताश्च पुत्रशोकान्वितास्तथा}
{दह्यमानाः स्म शोकेन गृहं गच्छन्ति नित्यशः}


\twolineshloka
{अनिष्टानां सहस्राणि तथेष्टानां शतानि च}
{उत्सृज्येह प्रयाता वै बान्धवा भृशदुःखिताः}


\twolineshloka
{त्यज्यतामेष निस्तेजाः शून्यः काष्ठत्वमागतः}
{अन्यदेहविषक्तं हि शिशुं काष्ठमुपासथ}


\twolineshloka
{त्यक्तजीवस्य वै बाष्पं कस्माद्धित्वा न गच्छत}
{निरर्थको ह्ययं स्नेहो निष्फलश्च परिश्रमः}


\twolineshloka
{न च क्षुर्भ्यां न कर्णाभ्यां च शृणोति स पश्यति}
{कस्मादेनं सप्नुत्सृज्य न गृहान्गच्छताशु वै}


\twolineshloka
{मोक्षधर्माश्रितैर्वाक्यैर्हेतुमद्भिः सुनिष्ठुरैः}
{भयोक्ता गच्छत क्षिप्रं स्वं स्वमेव निवेशनम्}


\twolineshloka
{प्रज्ञाविज्ञानयुक्तेन बुद्धिसंज्ञाप्रदायिना}
{वच्चं श्राविता नूनं मानुषाः संनिवर्तथ}


\fourlineindentedshloka
{[शोको द्विगुणतां याति दृष्ट्वा स्मृत्वा च चेष्टितम्}
{इत्येतद्वचनं श्रुत्वा सन्निवृत्तास्तु मानुषाः}
{अपश्यत्तं तदा सुप्तं द्रुतमागत्य जम्बुकः ॥] जम्बुक उवाच}
{}


\twolineshloka
{इमं कनकवर्णाभं भूषणैः समलंकृतम्}
{गृध्रवाक्यात्कथं पुत्रं त्यक्ष्यध्वं पितृपिण्डदम्}


\twolineshloka
{न स्नेहस्य च विच्छेदो विलापरुदितस्य च}
{मृतस्यास्य परित्यागात्तापो वै भविता ध्रुवम्}


\twolineshloka
{श्रूयते शम्बुके शूद्रे हते ब्राह्मणदारकः}
{जीवितो धर्ममासाद्य रामात्सत्यपराक्रमात्}


\twolineshloka
{तथा श्वैत्यस्य राजर्षेर्बालो दिष्टान्तमागतः}
{मुनिना धर्मनिष्ठेन मृतः संजीवितः पुनः}


\twolineshloka
{तथा कश्चिद्भवेत्सिद्धो मुनिर्वा देवतापि वा}
{कृपणानामनुक्रोशं कुर्याद्वो रुदतामिह}


\threelineshloka
{इत्युक्तास्ते न्यवर्तन्त शोकार्ताः पुत्रवत्सलाः}
{अङ्के शिरः समाधाय रुरुदुर्बहुविस्तरम्}
{तेषां रुदितशब्देन गृध्रोऽभ्येत्य वचोऽब्रवीत्}


\twolineshloka
{अश्रुपातपरिक्लिन्नः पाणिस्पर्शप्रपीडितः}
{धर्मराजप्रयोगाच्च दीर्घनिद्रां प्रवेशितः}


\twolineshloka
{`तपसाऽपि हि संयुक्तो जनः कालेन हन्यते}
{सर्वस्नेहावसक्तानामिदं हि स्नेहवर्तनम् ॥'}


\twolineshloka
{बालवृद्धसहस्राणि सदा संत्यज्य बान्धवाः}
{दिनानि चैव रात्रीश्च दुःखं तिष्ठन्ति भूतले}


\twolineshloka
{अलं निर्बन्धमागत्य शोकस्य परिवारणम्}
{अप्रत्ययं कुतो ह्यस्य पुनरद्येह जीवितम्}


\twolineshloka
{`नैष जम्बुकवाक्येन पुनः प्राप्स्यति जीवितम्}
{'मृतस्योत्सृष्टदेहस्य पुनर्देहो न विद्यते}


\twolineshloka
{नैव मूर्तिप्रदानेन जम्बुकस्य शतैरपि}
{न स जीवयितुं शक्यो बालो वर्षशतैरपि}


\twolineshloka
{अथ रुद्रः कुमारो वा ब्रह्मा वा विष्णुरेव च}
{वरमस्मै प्रयच्छन्ति ततो जीवेदयं शिशुः}


\twolineshloka
{नैव बाष्पविमोक्षेण न वा श्वासकृतेन च}
{न दीर्घरुदितेनायं पुनर्जीवं गमिष्यति}


\twolineshloka
{अहं च क्रोष्टुकश्चैव यूयं ये चास्य बान्धवाः}
{धर्माधर्मौ गृहीत्वेह सर्वे वर्तामहेऽध्वनि}


\twolineshloka
{अप्रियं परुषं चापि परद्रोहं परस्त्रियम्}
{अधर्ममनृतं चैव दूरात्प्राज्ञो विवर्जयेत्}


\twolineshloka
{धर्मं सत्यं श्रुतं न्याय्यं महतीं प्राणिनां दयाम्}
{अजिह्नत्वमशाठ्यं च यत्नतः परिमार्गत}


\twolineshloka
{मातरं पितरं वाऽपि बान्धवान्सुहृदस्तथा}
{जीवतो ये न पश्यति तेषां धर्मविपर्ययः}


\twolineshloka
{यो न पश्यति चक्षुर्भ्यां नेङ्गते च कथंचन}
{तस्य निष्ठावसानान्ते रुदन्ताः किं करिष्यथ}


\threelineshloka
{इत्युक्तास्ते सुतं त्यक्त्वा भूमौ शोकपरिप्लुताः}
{दह्यमानाः सुतस्नेहात्प्रययुर्बान्धवा गृहम् ॥जम्बुक उवाच}
{}


\twolineshloka
{दारुणो मर्त्यलोकोऽयं सर्वप्राणिविनाशनः}
{इष्टबन्धुवियोगश्च तथेहाल्पं च जीवितम्}


\threelineshloka
{बह्वलीकमसत्यं चाप्यतिवादाप्रियंवदम्}
{इमं प्रेक्ष्य पुनर्भावं दुःखशोकविवर्धनम्}
{न मे मानुषलोकोऽयं मुहूर्तमपि रोचते}


\twolineshloka
{अहो धिग्गृध्रवाक्येन यथैवाबुद्धयस्तथा}
{कथं गच्छथ निःस्नेहाः सुतस्नेहं विसृज्य च}


\twolineshloka
{प्रदीप्ताः पुत्रशोकेन सन्निवर्तथ मानुषाः}
{श्रुत्वा गृध्रस्य वचनं पापस्येहाकृतात्मनः}


\twolineshloka
{सुखस्यानन्तरं दुःखं दुःखस्यानन्तरं सुखम्}
{सुखदुःखावृते लोके नास्ति सौख्यमनन्तकम्}


\twolineshloka
{इमं क्षितितले त्यक्त्वा बालं रूपसमन्वितम्}
{कुलशोभाकरं मूढाः पुत्रं त्यक्त्वा क्व यास्यथ}


\twolineshloka
{रूपयौवनसंपन्नं द्योतमानमिव श्रिया}
{जीवन्तमेव पश्यामि मनसा नात्र संशयः}


\twolineshloka
{विनाशेनास्य न हि वै सुखं प्राप्स्यथ मानुषाः}
{पुत्रशोकाभितप्तानां मृतमप्यद्य वः क्षमम्}


\threelineshloka
{दुःखसंभावनं कृत्वा धारयित्वा सुखं स्वयम्}
{त्यक्त्वा गमिष्यथ क्वाद्य समुत्सृज्याल्पबुद्धिवत् ॥भीष्म उवाच}
{}


\twolineshloka
{तथा धर्मविरोधेन प्रियमिथ्याभिधायिनाम्}
{श्मशानवासिना नित्यं रात्रिं मृगयता नृप}


\threelineshloka
{ततो मध्यस्थतां नीता वचनैरमृतोपमैः}
{जम्बुकेन स्वकार्यार्थं बान्धवास्तत्र वारिताः ॥गृध्र उवाच}
{}


\twolineshloka
{अयं प्रेतसमाकीर्णो यक्षराक्षससेवितः}
{दारुणः काननोद्देशः कौशिकैरभिनादितः}


\twolineshloka
{भीमः सुघोरश्च तथा नीलमेघसमप्रभः}
{अस्मिञ्शवं परित्यज्य प्रेतकार्याण्युपासत}


\twolineshloka
{भानुर्यावन्न यात्यस्तं यावच्च विमला दिशः}
{तावदेनं परित्यज्य प्रेतकार्याण्युपासत}


\twolineshloka
{नदन्ति परुषं श्येनाः शिवाः क्रोशन्ति दारुणम्}
{मृगेन्द्राः प्रतिनर्दन्ति रविरस्तं च गच्छति}


\twolineshloka
{चिता धूमेन नीलेन संरज्यन्ते च पादपाः}
{श्मशाने च निराहाराः प्रतिनर्दन्ति देवताः}


\twolineshloka
{सर्वे विकृतदेहाश्चाप्यस्मिन्देशे सुदारुणे}
{युष्मान्प्रधर्षयिष्यन्ति विकृता मांसभोजिनः}


\twolineshloka
{क्रूरश्चायं वनोद्देशो भयमद्य भविष्यति}
{त्यज्यतां काष्ठभूतोऽयं मुच्यतां जाम्बुकं वचः}


\threelineshloka
{यदि जम्बुकवाक्यानि निष्फलान्यनृतानि च}
{श्रोष्यथ भ्रष्टविज्ञानास्ततः सर्वे विनङ्क्ष्यथ ॥जम्बुक उवाच}
{}


\twolineshloka
{स्थीयतां वो न भेतव्यं यावत्तपति भास्करः}
{तावदस्मिन्सुते स्नेहादनिर्वेदेन वर्तत}


\twolineshloka
{स्वैरं रुदन्तो विस्रब्धाश्चिरं स्नेहेन पश्यत}
{`दारुणेऽस्मिन्वनोद्देशे भयं वो न भविष्यति}


\twolineshloka
{अयं सौम्यो वनोद्देशः पितृणां निधनाकरः}
{'स्थीयतां यावदादित्यः किंवः क्रव्यादभाषितैः}


\threelineshloka
{यदि गृध्रस्य वाक्यानि तीव्राणि रभसानि च}
{गृह्णीत मोहितात्मानः सुतो वो न भविष्यति ॥भीष्म उवाच}
{}


\twolineshloka
{गृध्रो नास्तमितेऽभ्येति तिष्ठेन्नक्तं च जम्बुकः}
{मृतस्य तं परिजनमूचतुस्तौ क्षुधान्वितौ}


\twolineshloka
{स्वकार्यबद्धकक्षौ तौ राजन्गृध्रोऽथ जम्बुकः}
{क्षुत्पिपासापरिश्रान्तौ शास्त्रमालम्ब्य जल्पतः}


\twolineshloka
{तयोर्विज्ञानविदुषोर्द्वयोर्मृगपतत्रिणोः}
{वाक्यैरमृतकल्पैस्तैः प्रतिष्ठन्ते व्रजन्ति च}


\twolineshloka
{शोकदैन्यसमाविष्टा रुदन्तस्तस्थिरे तदा}
{स्वकार्यकुशलाभ्यां ते संभ्राम्यन्ते ह नैपुणात्}


\twolineshloka
{तथा तयोर्विवदतोर्विज्ञानविदुषोर्द्वयोः}
{बान्धवानां स्थितानां चाप्युपातिष्ठत शंकरः}


\twolineshloka
{देव्या प्रणोदितो देवः कारुण्यार्द्रीकृतेक्षणः}
{ततस्तानाह मनुजान्वरदोऽस्मीति शंकरः}


\threelineshloka
{ते प्रत्यूचुरिदं वाक्य दुःखिताः प्रणताः स्थिताः}
{एकपुत्रविहीनानां सर्वेष्नां जीवितार्थिनाम्}
{पुत्रस्य नो जीवदानाज्जीवितं दातुमर्हसि}


\twolineshloka
{एवमुक्तः स भगवान्वारिपूर्णेन पाणिना}
{जीवं तस्मै कुमाराय प्रादाद्वर्षशतानि वै}


\twolineshloka
{तथा गोमायुगृध्राभ्यां प्राददत्क्षुद्विनाशनम्}
{वरं पिनाकी भगवान्सर्वभूतहिते रतः}


\twolineshloka
{ततः प्रणम्य ते देवं श्रेयोहर्षसमन्विताः}
{कृतकृत्याः सुसंहृष्टाः प्रातिष्ठन्त तदा विभो}


\twolineshloka
{अनिर्वेदेन दीर्घेण निश्चयेन ध्रुवेण च}
{देवदेवप्रसादाच्च क्षिप्रं फलमवाप्यते}


\twolineshloka
{पश्य दैवस्य संयोगं बान्धवाना च निश्चयम्}
{कृपणानां तु रुदतां कृतमश्रुप्रमार्जनम्}


\twolineshloka
{पश्य चाल्पेन कालेन निश्चयाध्वेषणेन च}
{प्रसादं शंकरात्प्राप्य दुःखिताः सुखमाप्नुवन्}


\twolineshloka
{ते विस्मिताः प्रहृष्टाश्च पुत्रसंजीवनात्पुनः}
{बभूवुर्भरतश्रेष्ठ प्रसादाच्छंकरस्य वै}


\twolineshloka
{ततस्ते त्वरिता राजंस्त्यक्त्वा शोकं शिशूद्भवम्}
{विविशुः पुत्रमादाय नगरं हृष्टमानसाः}


% Check verse!
एषा बुद्धिः समस्तानां चातुर्वर्ण्येन दर्शिता
\twolineshloka
{धर्मार्थमोक्षसंयुक्तमितिहासं पुरातनम्}
{श्रुत्वा मनुष्यः सततमिहामुत्र प्रमोदते}


\chapter{अध्यायः १५३}
\twolineshloka
{युधिष्ठिर उवाच}
{}


\twolineshloka
{बलिनः प्रत्यमित्रस्य नित्यमासन्नवर्तिनः}
{उण्कारापकाराभ्यां समर्थस्योद्यतस्य च}


\twolineshloka
{मोहाद्विकत्थनामात्रैरसारोऽल्पबलो लघुः}
{वाग्भिरप्रतिरूपाभिरभिद्रुह्य पितामह}


\threelineshloka
{आत्मनो बलमास्थाय कथं वर्तेत मानवः}
{आगच्छतोऽतिक्रुद्धस्य तस्योद्धरणकाम्यया ॥भीष्म उवाच}
{}


\twolineshloka
{अत्राप्युदाहन्तीममितिहासं पुरातनम्}
{संवादं भरतश्रेष्ठ शाल्मलेः पवनस्य च}


\twolineshloka
{हिमवन्तं समासाद्य महानासीद्वनस्पतिः}
{वर्षपूगाभिसंवृद्धः शाखामूलपलाशवान्}


\twolineshloka
{तत्र स्म मत्तमातङ्गा घर्मार्ताः श्रमकर्शिताः}
{विश्राम्यन्ति महाबाहो तथाऽन्या मृगजातयः}


\twolineshloka
{नल्वमांत्रपरीणाहो घनच्छायो वनस्पतिः}
{शुकशारिकसंघुष्टः पुष्पवान्फलवानपि}


\twolineshloka
{सार्थका वणिजश्चापि तापसाश्च वनौकसः}
{वसन्ति तत्र मार्गस्थाः सुरम्ये नगसत्तमे}


\twolineshloka
{तस्य ता विपुलाः शाखा दृष्ट्वा स्कन्धं च सर्वशः}
{अभिगम्याब्रवीदेनं नारदो भरतर्षभ}


\twolineshloka
{अहो नु रमणीयस्त्वमहो चासि मनोहरः}
{प्रीयामहे त्वया नित्यं तरुप्रवर शाल्मके}


\twolineshloka
{सदैव शकुनास्तात मृगाश्चाथ तथा गजाः}
{वसन्ति तव संहृष्टा मनोहरतरास्तथा}


\twolineshloka
{तव शाखा महाशाख स्कन्धांश्च विपुलांस्तथा}
{न वै प्रभग्नान्पश्यामि मारुतेन कथंचन}


\twolineshloka
{किंनु ते पवनस्तात प्रीतिमानथवा सुहृत्}
{त्वां रक्षति सदा येन वनेऽत्र पवनो ध्रुवम्}


\twolineshloka
{भगवान्पवनः स्थानाद्वृक्षानुच्चावचानपि}
{पर्वतानां च शिखराण्याचालयति वेगवान्}


\twolineshloka
{शोषयत्येव पातालं वहन्गन्धवहः शुचिः}
{सरांसि सरितश्चैव सागरांश्च तथैव च}


\twolineshloka
{संरक्षति त्वां पवनः सखित्वेन न संशयः}
{तस्मात्त्वं बहुशाखोऽपि पर्णवान्पुष्पवानपि}


\twolineshloka
{इदं च रमणीयं ते प्रतिभाति वनस्पते}
{य इमे विहगास्तात रमन्ते मुदितास्त्वयि}


\twolineshloka
{एषां पृथक्समस्तानां श्रूयते मधुरस्वरः}
{पुष्पसंमोदने काले वाशन्ते हरियूथपाः}


\threelineshloka
{तथेमे गर्जिता नागाः स्वयूथगणशोभिताः}
{घर्मार्तास्त्वां समासाद्य सुखं विन्दति शाल्मके}
{}


\twolineshloka
{तथैव मृगजातीभिरन्याभिरभिशोभसे}
{तथा सार्थाधिवासैश्च शोभसे मेरुवद्द्रुम्}


\twolineshloka
{ब्राह्मणैश्च तपः सिद्धैस्तापसैः श्रमणैस्तथा}
{त्रिविष्टपसमं मन्ये तवायतनमेव हि}


\chapter{अध्यायः १५४}
\twolineshloka
{नारद उवाच}
{}


\twolineshloka
{बन्धुत्वादथवा सख्याच्छाल्मले नात्र संशयः}
{कस्मात्त्वां रक्षते नित्यं भीमः सर्वत्रगोऽनिलः}


\twolineshloka
{तद्भावं परमं वायोः शाल्मके त्वमुपागतः}
{तवाहमस्मीति सदा येन रक्षति मारुतः}


\twolineshloka
{न तं पश्याम्यहं वृक्षं पर्वतं वेश्म चेदृशम्}
{यं न वायुबलाद्भग्नं पृथिव्यामिति मे मतिः}


\threelineshloka
{त्वं पुनः कारणैर्नूनं रक्ष्यसे शाल्मले यथा}
{वायुना सपरीवारस्तेन तिष्ठस्यसंशयम् ॥शाल्मलिरुवाच}
{}


\twolineshloka
{न मे वायुः सखा ब्रह्मन्न बन्धुर्मम नारद}
{चिरं मे प्रीयते नैव येन मां रक्षतेऽनिलः}


\twolineshloka
{मम तेजोबलं भीमं वायोरपि हि नारद}
{कलामष्टादशीं प्राणैर्न मे प्राप्नोति मारुतः}


\twolineshloka
{आगच्छन्परुषो वायुर्मया विष्टम्भितो बलात्}
{भञ्जन्द्रुमान्पर्वतांश्च यच्चान्यत्स्थाणुजङ्गमम्}


\threelineshloka
{स मया बहुशो भग्नः प्रभञ्जन्वै प्रभञ्जनः}
{तस्मान्न विभ्ये देवर्षे क्रुद्धादपि समीरणात् ॥नारद उवाच}
{}


\twolineshloka
{शाल्मले विपरीतं ते दर्शनं नात्र संशयः}
{न हि वायोर्बले नास्ति भूतं तुल्यबलं क्वचित्}


\twolineshloka
{इन्द्रो यमो वैश्रवणो वरुणश्च जलेश्वरः}
{नैतेऽपि तुल्या मरुतः किं पुनस्त्वं वनस्पते}


\twolineshloka
{यश्च कश्चिदपि प्राणी चेष्टते शाल्मले भुवि}
{सर्वत्र भगवान्वायुश्चेष्टाप्राणकरः प्रभुः}


\twolineshloka
{एष चेष्टयते सम्यक्प्राणिनः सम्यगायतः}
{असम्यगायतो भूयश्चेष्टते विकृतं नृषु}


\twolineshloka
{स त्वमेवंविधं वायुं सर्वसत्वभृतां वरम्}
{न पूजयसि पूज्यन्तं किमन्यद्वुद्धिलाघवात्}


\twolineshloka
{असारश्चापि दुर्मेधाः केवलं बहु भाषसे}
{क्रोधादिभिरवच्छन्नो मिथ्या वदसि शाल्मले}


\twolineshloka
{मम रोषः समुत्पन्नस्त्वय्येवं संप्रभाषति}
{ब्रवीम्येष स्वयं वायोस्तव दुर्भाषितं बहु}


\threelineshloka
{चन्दनैः स्यन्दनैः शालैः सरलैर्देवदारुभिः}
{वेतसैर्धन्वनैश्चापि ये चान्ये बलवत्तराः}
{तैश्चापि नैवं दुर्बुद्धे क्षिप्तो वायुः कृतात्मभिः}


\twolineshloka
{तेऽपि जानन्ति वायोश्च बलमात्मन एव च}
{तस्मात्ते नावमन्यन्ते श्वसनं तरुसत्तमाः}


\twolineshloka
{त्वं तु मोहान्न जानीपे वायोर्बलमनन्तकम्}
{एवं तस्माद्गमिष्यामि सकाशं मातरिश्वनः}


\chapter{अध्यायः १५५}
\twolineshloka
{भीष्म उवाच}
{}


\threelineshloka
{एवमुक्त्वा तु राजेन्द्र शाल्मलिं ब्रह्मवित्तमः}
{नारदः पवने सर्वं शाल्मलेर्वाक्यमब्रवीत् ॥नारद उवाच}
{}


\twolineshloka
{हिमवत्पृष्ठजः कश्चिच्छाल्मलिः परिवारवान्}
{बृहन्मूलो बृहच्छायः स त्वां वायोऽवमन्यते}


\twolineshloka
{बहुव्याक्षेपयुक्तानि त्वामाह वचनानि सः}
{न युक्तानि मया वायो तानि वक्तुं तवाग्रतः}


\threelineshloka
{जानामि त्वामहं वायो सर्वप्राणभृतां वरम्}
{वरिष्ठं च गरिष्ठं च सर्वलोकेश्वरं प्रभुम् ॥भीष्म उवाच}
{}


\threelineshloka
{एतत्तु वचनं श्रुत्वा नारदस्य समीरणः}
{शाल्मलिं तमुपागम्य क्रुद्धो वचनमब्रवीत् ॥वायुरुवाच}
{}


\twolineshloka
{शाल्मले नारदो गच्छंस्त्वयोक्तो मद्विगर्हणम्}
{अहं वायुः प्रभावं ते दर्शयाम्यात्मनो बलम्}


\twolineshloka
{नाहं त्वां नाभिजानामि विदितश्चासि मे द्रुम्}
{पितामहः प्रजासर्गे त्वयि विश्रान्तवान्प्रभुः}


\twolineshloka
{तस्य विश्रमणादेव प्रसादौ मत्कृतस्तव}
{अभूत्तस्य प्रसादात्त्वां न भज्यामि द्रुमाधम्}


\threelineshloka
{यन्मां त्वमवजानीषे यथाऽन्यं प्राकृतं तथा}
{दर्शयाम्येष चात्मानं यथा मां नावमन्यसे ॥भीष्म उवाच}
{}


\twolineshloka
{एवमुक्तस्ततः प्राह शाल्मलिः प्रहसन्निव}
{पवन त्वं वने क्रुद्धो दर्शयात्मानमात्मना}


\threelineshloka
{मयि वै मुच्यतां क्रोधः किं मे क्रुद्धः करिष्यसि}
{न ते बिभेमि पवन यद्यपि त्वं स्वयं प्रभुः}
{बलाधिकोऽहं त्वत्तश्च न भीः कार्या मया तव}


\twolineshloka
{ये तु बुद्ध्या हि बलिनस्ते भवन्ति बलीयसः}
{प्राणमात्रबला ये वै नैव ते बलिनो मताः}


\twolineshloka
{इत्येवमुक्तः पवनः श्व इत्येवाब्रवीद्वचः}
{दर्शयिष्यामि ते तेजस्ततो रात्रिरुपागमत्}


\twolineshloka
{अथ निश्चित्य मनसा शाल्मलिर्वैरधारणम्}
{पश्यमानस्तदात्मानमसमं मातरिश्वना}


\twolineshloka
{नारदे यन्मया प्रोक्तं वचनं प्रति तन्मृषा}
{असमर्थो ह्यहं वायोर्बलेन बलवान्हि सः}


\twolineshloka
{मारुतो बल्यान्नित्यं यथा वै नारदोऽब्रवीत्}
{अहं तु दुर्बलोऽन्येभ्यो वृक्षेभ्यो नात्र संशयः}


\twolineshloka
{किं तु बुद्ध्या समो नास्ति मम कश्चिद्वनस्पतिः}
{तदहं बुद्धिमास्थाय भयं त्यक्ष्ये समीरणात्}


\twolineshloka
{यदि तां बुद्धिमास्थाय तिष्ठेयुः पर्णिनो वने}
{अरिष्टाः स्युः सदा क्रुद्धात्पवनान्नात्र संशयः}


\threelineshloka
{ते तु बाला न जानन्ति यथा नैतान्समीरणः}
{समीरयेत संक्रुद्धो यथा जानाम्यहं तथा ॥भीष्म उवाच}
{}


\twolineshloka
{ततो निश्चित्य मनसा शाल्मलिः क्षुभितस्तदा}
{शाखाः स्कन्धान्प्रशाखाश्च स्वयमेव व्यशातयत्}


\twolineshloka
{स परित्यज्य शाखाश्च पत्राणि कुसुमानि च}
{प्रभाते वायुमायान्तं प्रेक्षते स्म वनस्पतिः}


\twolineshloka
{ततः क्रुद्धः श्वसन्वायुः पातयन्वै महाद्रुमान्}
{आजगामाथ तं देशमास्ते यत्र स शाल्मलिः}


\threelineshloka
{तं हीनपर्णं पतिताग्रशाखंनिशीर्णपुष्पं प्रसमीक्ष्य वायुः}
{उवाच वाक्यं स्ययमान एवंमुदायुतः शाल्मलिं रुग्णशाखम् ॥वायुरुवाच}
{}


\twolineshloka
{अहमप्येवमेव त्वां कुर्यां वै शाल्मले रुपा}
{आत्मना यत्कृतं कृच्छ्रं शाखानामपकर्पणम्}


\threelineshloka
{हीनपुष्पाग्रशाखस्त्वं शीर्णाङ्कुरपलाशकः}
{आत्मदुर्मन्त्रितेनेह मद्वीर्यवशगः कृतः ॥भीष्म उवाच}
{}


\twolineshloka
{एतच्छ्रुत्वा वचो वायोः शाल्मलिर्व्रीडितस्तदा}
{अतप्यत वचः स्मृत्वा नारदो यत्तदाऽब्रवीत्}


\twolineshloka
{एवं हि राजशार्दूल दुर्बलः सन्वलीयसा}
{वैरमासञ्जते बालस्तप्यते शाल्मलिर्यथा}


\twolineshloka
{तस्माद्वैरं न कुर्वीत दुर्बलो बलवत्तरैः}
{शोचेद्धि वैरं कुर्वाणो यथा वै शाल्मलिस्तथा}


\twolineshloka
{न हि वैरं महात्मानो विवृण्वन्त्यपकारिषु}
{शनैः शनैर्महाराज दर्शयन्ति स्म ते बलम्}


\twolineshloka
{वैरं न कुर्वीत नरो दुर्बुद्धिर्बुद्धिजीविना}
{बुद्धिर्वुद्धिमतो याति तूलष्विव हुताशनः}


\twolineshloka
{न हि बुद्ध्या समं किंचिद्विद्यते पुरुषे नृप}
{तथा बलेन राजेन्द्र न समोऽस्तीह कश्चन}


\twolineshloka
{तस्मात्क्षमेत बालाय जडान्धवधिराय च}
{बलाधिकाय राजेन्द्र तद्दृष्टं त्वयि शत्रुहन्}


\twolineshloka
{अक्षौहिण्यो दशैका च सप्त चैव महाद्युते}
{बलेन न समा राजन्नर्जुनस्य महात्मनः}


\twolineshloka
{निहताश्चैव भग्नाश्च पाण्डवेन यशस्विना}
{चरता बलमास्थाय पाकशासनिना मृधे}


\twolineshloka
{उक्ताश्च ते राजधर्मा आपद्धर्माश्च भारत}
{विस्तरेण महाराज किं भूयः प्रव्रवीमि ते}


\chapter{अध्यायः १५६}
\twolineshloka
{युधिष्ठिर उवाच}
{}


\threelineshloka
{पापस्य यदधिष्ठानं यतः पापं प्रवर्तते}
{एतदिच्छाम्यहं श्रोतुं तत्त्वेन भरतर्षभ ॥भीष्म उवाच}
{}


\twolineshloka
{पापस्य यदधिष्ठानं तच्छृणुष्व नराधिप}
{एको लोभो महाग्राहो लोभात्पापं प्रवर्तते}


\twolineshloka
{अतः पापमधर्मश्च तथा दुःखमनुत्तमम्}
{निकृत्या मूलमेतद्धि येन पापकृतो जनाः}


\twolineshloka
{लोभात्क्रोधः प्रभवति लोभात्कामः प्रवर्तते}
{लोभान्मोहश्च माया च मानस्तम्भः परासुता}


\twolineshloka
{अक्षमा ह्रीपरित्यागः श्रीनाशो धर्मसंक्षयः}
{अभिध्याऽप्रख्यता चैव सर्वं लोभात्प्रवर्तते}


\twolineshloka
{अत्यागश्च कुतर्कश्च विकर्मसु च याः क्रियः}
{कुलविद्यामदश्चैव रूपैश्वर्यमदस्तथा}


\twolineshloka
{सर्वभूतेष्वभिद्रोहः सर्वभूतेष्वसत्कृतिः}
{सर्वभूतेष्वविश्वासः सर्वभूतेष्वनार्जवम्}


\twolineshloka
{हरणं परवित्तानां परदाराभिमर्शनम्}
{वाग्वेगो मनसो वेगो निन्दावेगस्तथैव च}


\twolineshloka
{उपस्थोदरयोर्वेगो मृत्युवेगश्च दारुणः}
{ईर्ष्यावेगश्च बलवान्मिथ्यावेगश्च दुर्जयः}


\threelineshloka
{रसवेगश्च दुर्वार्यः श्रोत्रवेगश्च दुःसह}
{कुत्सा विकत्था मात्सर्यं पापं दुष्कर्मकारिता}
{साहसानां च सर्वेषामकार्याणां क्रियास्तथा}


\twolineshloka
{आतौ बाल्ये च कौमारे यौवने चापि मानवाः}
{न त्यजन्त्यात्मकर्मैकं यन्न जीर्यति जीर्यतः}


\twolineshloka
{यो न पूरयितुं शक्यो लोभः प्रीत्या कथंचन}
{नित्यं गम्भीरतोयाभिरापगाभिरिवोदधिः}


\twolineshloka
{न प्रहृष्यति यो लोभैः कामैर्यश्च न तृप्यति}
{यो न देवैर्न गन्धर्वैर्नासुरैर्न महोरगैः}


\twolineshloka
{ज्ञायते नृप तत्त्वेन सर्वैर्भूतगणैस्तथा}
{स लोभः सह मोहेन विजेतव्यो जितात्मना}


\twolineshloka
{दम्भो द्रोहश्च निन्दा च पैशून्यं मत्सरस्तथा}
{भवन्त्येतानि कौरव्य लुब्धानामकृतात्मनाम्}


\twolineshloka
{सुमहान्त्यपि शास्त्राणि धारयन्तो बहुश्रुताः}
{छेत्तारः संशयानां च क्लिश्यन्तीहाल्पबुद्धयः}


\twolineshloka
{द्वेपक्रोधप्रसक्ताश्च शिष्टाचारबहिष्कृताः}
{अन्तःक्षुरा वाङ्भधुराः कृपाश्छन्नास्तृणैरिव}


\threelineshloka
{धर्मवैतंसिकाः क्षुद्रा मुष्णन्ति ध्वजिनो जगत्}
{कुर्वते च बहून्मार्गांस्तान्हेतुबलमाश्रिताः}
{सर्वमार्गान्विलुम्पन्ति लोभज्ञानेष्ववस्थिताः}


\twolineshloka
{धर्मस्य ह्रियमाणस्य लोभग्रस्तैर्दुरात्मभिः}
{याया विक्रियते संस्था ततः साऽपि प्रपद्यते}


\twolineshloka
{दर्पः बोधो मदः स्वप्नो हर्षः शोकोऽभिमानिता}
{एत हि कौरव्य दृश्यन्ते लुब्धबुद्धिषु}


\twolineshloka
{एताना टान्बुद्धस्व नित्यं लोभसमन्वितान्}
{शिष्टांस्तु परिपृच्छेथा यान्वक्ष्यामि शुचिव्रतान्}


\twolineshloka
{येष्वावृत्तिभयं नास्ति परलोकभयं न च}
{नामिपेषु प्रसङ्गोऽस्ति न प्रियेष्वप्रियेषु च}


\twolineshloka
{शिष्टाचारः प्रियो येषु दमो येषु प्रतिष्ठितः}
{सुखं दुःखं समं येषां सत्यं येषां परायणम्}


\twolineshloka
{दातारो न ग्रहीतारो दयावन्तस्नथैव च}
{पितृदेवातिथेयाश्च नित्योद्युक्तास्तथैव च}


\twolineshloka
{सर्वोपकारिणो वीराः सर्वधर्मानुपालकाः}
{सर्वभूतहिताश्चैव सर्वदेयाश्च भारत}


\twolineshloka
{न ते चालयितुं शक्या धर्मव्याहारकारिणः}
{न तेषां भिद्यते वृत्तं यत्पुरा साधुभिः कृतम्}


\twolineshloka
{न त्रासिनो न चपला न रौद्राः सत्पथे स्थिताः}
{ते सेव्याः साधुभिर्नित्यमसाधूंश्च विवर्जयेत्}


\twolineshloka
{कामक्रोधव्यपेता ये निर्ममा निरहंकृताः}
{सुव्रताः स्थिरमर्यादास्तानुपास्व च पृच्छ च}


% Check verse!
न वागर्थं यशोर्थं वा धर्मस्तेपां युधिष्ठिर ॥अवश्यं कार्य इत्येव शरीरस्य क्रियास्तथा
\twolineshloka
{त भयं क्रोधचापल्ये न शोकस्तेषु विद्यते}
{न धर्मध्वजिनश्चैव न गुह्यं किंचिदास्थिताः}


\twolineshloka
{येष्वलोभस्तथाऽमोहो ये च सत्यार्जवे स्थिताः}
{तेषु कौन्तेय रज्येथा येषां न भ्रश्यते पुनः}


\twolineshloka
{ये न हृष्यन्ति लाभेषु नालाभेषु व्यथन्ति च}
{निर्ममा निरहंकाराः सत्वस्थाः समदर्शिनः}


\twolineshloka
{लाभालाभौ सुखदुःखे च तातप्रियाप्रिये मरणं जीवितं च}
{समानि येषां स्थिरविक्रमाणांबुभुत्सतां सत्यपथे स्थितानाम्}


\twolineshloka
{धर्मप्रियांस्तान्सुमहानुभावान्दान्तोऽप्रमत्तश्च समर्चयेथाः}
{दैवात्सर्वे गुणवन्तो भवन्तिशुभाशुभे वाक्प्रलापास्तथाऽन्ये}


\chapter{अध्यायः १५७}
\twolineshloka
{युधिष्ठिर उवाच}
{}


\threelineshloka
{अनर्थानामधिष्ठानमुक्तो लोभः पितामह}
{अज्ञानमपि कौरव्य श्रोतुमिच्छामि तत्त्वतः ॥भीष्म उवाच}
{}


\twolineshloka
{करोति पापं योऽज्ञानान्नात्मनो वेत्ति च क्षमम्}
{द्विषते साधुवृत्तांश्च स लोकस्यैति वाच्यताम्}


\threelineshloka
{अज्ञानान्निरयं याति तथा ज्ञानेन दुर्गतिम्}
{अज्ञानात्क्लेशमाप्नोति तथाऽऽपत्सु निमज्जति ॥युधिष्ठिर उवाच}
{}


\twolineshloka
{प्रजानामप्रवृत्तिं च ज्ञानवृद्धिक्षयोदयान्}
{मूलं स्थानं गतिं कालं कारणं हेतुमेव च}


\threelineshloka
{श्रोतुमिच्छामि तत्त्वेन यथावदिह पार्थिव}
{अज्ञानप्रसवं हीदं यद्दुःखमुपलभ्यते ॥भीष्म उवाच}
{}


\twolineshloka
{रागो द्वेषस्तथा मोहो हर्षः शोकोऽभिमानिता}
{कामः क्रोधश्च दर्पश्च तन्द्री चालस्यमेव च}


\twolineshloka
{इच्छा द्वेषस्तथा तापः परवृद्ध्युपतापिता}
{अज्ञानमेन्निर्दिष्टं पापानां चैव याः क्रियाः}


\twolineshloka
{एतस्य वा प्रवृत्तेश्च वृद्ध्यादीन्यांश्च पृच्छसि}
{विस्तरेण महाराज शृणु तच्च विशेषतः}


\twolineshloka
{उभावेतौ समफलौ समदोषौ च भारत}
{अज्ञानं चातिलोभश्चाप्येवं जानीहि पार्थिव}


\twolineshloka
{लोभप्रभवमज्ञानं वृद्धं भूयः प्रवर्धते}
{स्थाने स्थानं क्षयेत्क्षीणमुपैति विविधां गतिम्}


\twolineshloka
{मूलं लोभस्य मोहो वै कालात्मगतिरेव च}
{[छिन्ने भिन्ने तथा लोभे कारणं काल एव च ॥]}


\twolineshloka
{तस्याज्ञानाद्धि लोभो हि कामात्मा गतिरेव च}
{सर्वे दोषास्तथा लोभात्तस्माल्लोभं विवर्जयेत्}


\threelineshloka
{जनको युवनाश्वश्च पृषदश्वः प्रसेनजित्}
{लोभक्षयाद्दिवं प्राप्तास्तथैवान्ये नराधिपाः}
{`छिन्ने छिन्ने तथा लोभे दिवं प्राप्ता जनाधिपाः ॥'}


\twolineshloka
{प्रत्यक्षं तु कुरुश्रेष्ठ त्यज लोभमिहात्मना}
{त्यक्त्वा लोभं सुखी लोके प्रेत्य चेह च मोदते}


\chapter{अध्यायः १५८}
\twolineshloka
{युधिष्ठिर उवाच}
{}


\twolineshloka
{स्वाध्यायकृतयत्नस्य ब्राह्मणस्य विशेषतः}
{धर्मकामस्य धर्मात्मन्किंनु श्रेय इहोच्यते}


\twolineshloka
{बहुधा दर्शने लोके श्रेयो यदिह मन्यसे}
{अस्मिँल्लोके परे चैव तन्मे ब्रूहि पितामह}


\twolineshloka
{महानयं धर्मपथो बहुशाखश्च भारत}
{किंस्विदेवेह धर्माणामनुष्ठेयतमं मतम्}


\threelineshloka
{धर्मस्य महतो राजन्बहुशाखस्य तत्त्वतः}
{यन्मूलं परमं तात तत्सर्वं ब्रूह्यतन्द्रितः ॥भीष्म उवाच}
{}


\twolineshloka
{हन्त ते कथयिष्यामि येन श्रेयो ह्यवाप्स्यसि}
{पीत्वाऽमृतमिव प्राज्ञो येन तृप्तो भविष्यसि}


\twolineshloka
{धर्मस्य विधयो नैके तेते प्रोक्ता महर्षिभिः}
{स्वंस्वं विज्ञानमाश्रित्य दमस्तेषां परायणम्}


\twolineshloka
{दमं निःश्रेयसं प्राहुर्वृद्धा निश्चितदर्शिनः}
{ब्राह्मणस्य विशेषेण दमो धर्मः सनातनः}


\twolineshloka
{नादान्तस्य क्रियासिद्धिर्यथावदुपलभ्यते}
{दमो दानं तथा यज्ञानधीतं चातिवर्तते}


\twolineshloka
{दमस्तेजो वर्धयति पवित्रं च दमः परम्}
{विपाप्मा तेजसा युक्तः पुरुषो विन्दते महत्}


\twolineshloka
{दमेन सदृशं धर्मं नान्यं लोकेषु शुश्रुम्}
{दमो हि परमो लोके प्रशस्तः सर्वधर्मिणाम्}


\twolineshloka
{प्रेत्य चात्र मनुष्येन्द्र परमं विन्दते सुखम्}
{दमेन हि सदा युक्तो महान्तं धर्ममश्नुते}


\twolineshloka
{सुखं दान्तः प्रस्वपिति सुखं च प्रतिबुध्यते}
{सुखं पर्येति लोकांश्च मनश्चास्य प्रसीदति}


\twolineshloka
{अदान्तः पुरुषः क्लेशमभीक्ष्णं प्रतिपद्यते}
{अनर्थांश्च बहूनन्यान्प्रसृजत्यात्मदोषजान्}


\twolineshloka
{आश्रमेषु चतुर्ष्वाहुर्दममेवोत्तमं व्रतम्}
{दमलिङ्गानि वक्ष्यामि येषां समुदयो दमः}


\twolineshloka
{क्षमा धृतिरार्हेसा च समता सत्यमार्जवम्}
{इन्द्रियाभिजयो दाक्ष्यं मार्दवं ह्रीरचापलम्}


\twolineshloka
{अकार्पण्यमसंरम्भः संतोषः प्रियवादिता}
{अविहसाऽनसूया चाप्येषां समुदयो दमः}


\twolineshloka
{गुरुपूजा च कौरव्य दया भूतेष्वपैशुनम्}
{जनवादमृषावादस्तुतिनिन्दाविसर्जनम्}


\twolineshloka
{कामं क्राधं च लोभं च दर्पं स्तम्भं विकत्थनम्}
{रोषमीर्ष्यावमानं च नैव दान्तो निषेवते}


\twolineshloka
{अनिन्दितो ह्यकामात्मा नाल्पेष्वर्थ्यनसूयकः}
{समुद्रकल्पः स नरो न कथंचन पूर्यते}


\twolineshloka
{अहं त्वयि मम त्वं च मयि ते तेषु चाप्यहम्}
{पूर्वसंबन्धिसंयोगं नैतद्दान्तो निषेवते}


\twolineshloka
{सर्वा ग्राम्यास्तथाऽऽरण्या याश्च लोके प्रवृत्तयः}
{निन्दां चैव प्रशंसां च यो नाश्रयति मुच्यते}


\twolineshloka
{मैत्रोऽथ शीलसंपन्नः प्रसन्नात्मात्मविच्च यः}
{मुक्तस्य विविधैः सङ्गैस्तस्य प्रेत्य फलं महत्}


\twolineshloka
{सुवृत्तः शीलसंपन्नः प्रसन्नात्माऽऽत्मविद्वुधः}
{प्राप्येह लोके सत्कारं सुगतिं प्रतिपद्यते}


\twolineshloka
{कर्म यच्छुभमेवेह सद्भिराचरितं च यत्}
{तदेव ज्ञानयुक्तस्य मुनेर्वर्त्म न हीयते}


\twolineshloka
{निष्क्रम्य वनमास्थाय ज्ञानयुक्तो जितेन्द्रियः}
{कालाकाङ्गी चरन्नेवं ब्रह्मभूयाय कल्पते}


\twolineshloka
{अभयं यस्य भूतेभ्यो भूतानामभयं यतः}
{तस्य देहाद्विमुक्तस्य भयं नास्ति कुतश्चन}


\twolineshloka
{अवाचिनोति कर्माणि न च संप्रचिनोति ह}
{समः सर्वेषु भूतेषु मैत्रायणगतिं चरेत्}


\twolineshloka
{शकुनीनामिवाकाशे मत्स्यानांमिव चोदके}
{यथा गतिर्न दृश्येत तथा तस्य स संशयः}


\twolineshloka
{गृहानुत्सृज्य यो राज्मोक्षमेवाभिपद्यते}
{लोकास्तेजोमयास्तस्य कल्पन्ते शाश्वतीः समाः}


\twolineshloka
{संन्यस्य सर्वकर्माणि संन्यस्य विधिवत्तपः}
{संन्यस्य विविधा विद्याः सर्वं संन्यस्य चैव ह}


\twolineshloka
{कामे शुचिरनावृत्तः प्रसन्नात्माऽऽत्मविच्छुचिः}
{प्राप्येह लोके सत्कारं स्वर्गं समभिपद्यते}


\twolineshloka
{यच्च पैतामहं स्थानं ब्रह्मराशिसमुद्भवम्}
{गुहायां निहितं नित्यं तद्दमेनाभिगम्यते}


\twolineshloka
{ज्ञानारामस्य बुद्धस्य सर्वभूतानुरोधिनः}
{नावृत्तिभयमस्तीह परलोकभयं कुतः}


\twolineshloka
{एक एव दमे दोषो द्वितीयो नोपपद्यते}
{यदेनं दमसंयुक्तमशक्तं मन्यते जनः}


\twolineshloka
{एकोऽस्य सुमहाप्राज्ञ दोषः स्यात्सुमहान्गुणः}
{क्षमया विपुला लोका दुर्लभा हि सहिष्णुता}


\threelineshloka
{दान्तस्य किमरण्येन तथाऽदान्तस्य भारत}
{यत्रैव निवसेद्दान्तस्तदरण्यं स चाश्रमः ॥वैशंपायन उवाच}
{}


\twolineshloka
{एतद्भीष्मस्य वचनं श्रुत्वा राजा युधिष्ठिरः}
{अमृतेनेव संतृप्तः प्रहृष्टः समपद्यत}


\twolineshloka
{पुनश्च परिपप्रच्छ भीष्मं धर्मभृतां वरम्}
{ततः प्रीतः स चोवाच तस्मै सर्वं कुरूद्वहः}


\chapter{अध्यायः १५९}
\twolineshloka
{भीष्म उवाच}
{}


\twolineshloka
{सर्वमेतत्तपोमूलं कवयः परिचक्षते}
{न ह्यतप्ततपा मूढः क्रियाफलमवाप्नुते}


\twolineshloka
{प्रजापतिरिदं सर्वं तपसैवासृजत्प्रभुः}
{तथैव वेदानृपयस्तपसा प्रतिपेदिरे}


\twolineshloka
{तपसैव ससर्जान्नं फलमूलानि यानि च}
{त्रीँल्लोकांस्तपसा सिद्धाः पश्यन्ति सुसमाहिताः}


\twolineshloka
{औषधान्यगदादीनि तिस्त्रो विद्याश्च संस्कृताः}
{तषसैव हि सिद्ध्यन्ति तपोमूलं हि साधनम्}


\threelineshloka
{यद्दुरापं दुराराध्यं दुराधर्षं दुरुत्सहम्}
{तत्सर्वं तपसा शक्यं तपो हि दुरतिक्रमम्}
{ऐश्वर्यमृषयः प्राप्तास्तपसैव न संशयः}


\twolineshloka
{सुरापोऽसंमतादायी भ्रूणहा गुरुतल्पगः}
{तपसैव सुतप्तेन नरः पापात्प्रमुच्यते}


\twolineshloka
{तपसो बहुरूपस्य तैस्तैर्द्वारैः प्रवर्ततः}
{निवृत्त्या वर्तमानस्य तपो नानशनात्परम्}


\twolineshloka
{अहिंसा सत्यवचनं दानमिन्द्रियनिग्रहः}
{एतेभ्यो हि महाराज तपो नानशनात्परम्}


\twolineshloka
{न दुष्करतरं दानान्नाति मातरमाश्रमः}
{त्रैविद्येभ्यः परं नास्ति संन्यासान्नापरं तपः}


\twolineshloka
{इन्द्रियाणीह रक्षन्ति विप्रर्षिपितृदेवताः}
{तस्मादर्थे च धर्मे च तपो नानशनात्परम्}


\twolineshloka
{ऋषयः पितरो देवा मनुष्या मृगपक्षिणः}
{यानि चान्यानि भूतानि स्यावराणि चराणि च}


\twolineshloka
{तपः परायणाः सर्वे सिध्यन्ति तपसा च ते}
{इत्येवं तपसा देवा महत्त्वं प्रतिपेदिरे}


\twolineshloka
{इमानीष्टविभागानि फलानि तपसः सदा}
{तपसा शक्यते प्राप्नुं देवत्वमपि निश्चयः}


\chapter{अध्यायः १६०}
\twolineshloka
{युधिष्ठिर उवाच}
{}


\twolineshloka
{सत्यं धर्मं प्रशंसन्ति विप्रर्षिपितृदेवताः}
{सत्यमिच्छाम्यहं ज्ञातुं तन्मे ब्रूहि पितामह}


\threelineshloka
{सत्यं किंलक्षणं राजन्कथं वा तदवाप्यते}
{सत्यं प्राप्य भवेत्किंच कथं चैव तदुच्यताम् ॥भीष्म उवाच}
{}


\twolineshloka
{चातुर्वर्ण्यस्य धर्माणां संकरो न प्रशस्यते}
{धर्मः साधारणः सत्यं सर्ववर्णेषु भारत}


\twolineshloka
{सत्यं सत्सु सदा धर्मः सत्यं धर्मः सनातनः}
{सत्यमेव नमस्येत सत्यं हि परमा गतिः}


\twolineshloka
{सत्यं धर्मस्तपोयोगः सत्यं ब्रह्म सनातनम्}
{सत्यं यज्ञः परः प्रोक्तः सर्वं सत्ये प्रतिष्ठितम्}


\twolineshloka
{रूपं यदिह सत्यस्य यथावदनुपूर्वशः}
{लक्षणं च प्रवक्ष्यामि सत्यस्येह पराक्रमम्}


\twolineshloka
{प्राप्यते च यथा सत्यं तच्च वेत्तुमिहार्हसि}
{सत्यं त्रयोदशावधं सर्वलोकेषु भारत}


\twolineshloka
{सत्यं च समता चैव दमश्चैव न संशयः}
{अमात्सर्यं क्षमा चैव ह्रीस्तितिक्षाऽनसूयता}


\twolineshloka
{त्यागो ध्यानमथार्यत्वं धृतिश्च सततं दया}
{अहिंसा चैव राजेन्द्र सत्याकारास्त्रयोदश}


\twolineshloka
{सत्यं नामाव्ययं नित्यमविकारि तथैव च}
{सर्वधर्माविरुद्धं च योगेनैतदवाप्यते}


\twolineshloka
{आत्मनीष्टे तथाऽनिष्टे रिपौ च समता तथा}
{इच्छाद्वेषं क्षयं प्राप्य कामक्रोधक्षयं तथा}


\twolineshloka
{दमी नान्यस्पृहा नित्यं गाम्भीर्यं धैर्यमेव च}
{अशाठ्यं क्रोधदमनं ज्ञानेनैतदवाप्यते}


\twolineshloka
{अमात्सर्यं बुधाः प्राहुर्दाने धर्मे च संयमः}
{अवस्थितेन नित्यं च सत्येनामत्सरी भवेत्}


\twolineshloka
{अक्षमायाः क्षमायाश्च प्रियाणीहाप्रियाणि च}
{क्षमते स तः साधुस्ततः प्राप्नोति सत्यताम्}


\twolineshloka
{कल्याणं रुते बाढं धीमान्न ग्लायते क्वचित्}
{प्रशान्तवाङ्भना नित्यं ह्रीस्तु धर्मादवाप्यते}


\twolineshloka
{धर्मार्थहेतोः क्षमते तितिक्षा धर्म उत्तमः}
{लोकसंग्रहणार्थं वै सा तु धैर्येण लभ्यते}


\threelineshloka
{`अनसूया तु गाम्भीर्यं दानेनैतदवाप्यते}
{'त्यक्तस्नेहस्य यस्त्यागो विषयाणां तथैव च}
{रागद्वेषप्रहीणस्य त्यागो भवति नान्यथा}


\threelineshloka
{`ध्यानं च शाठ्यमित्युक्तं मौनेनैतदवाप्यते}
{'आर्यता नाम भूतानां यः करोति प्रयत्नतः}
{शुभं कर्म निराकारो वीतरागस्तथैव च}


% Check verse!
धृतिर्नाम सुखे दुःखे यया नाप्नोति विक्रियाम् ॥तां भजेत सदा प्राज्ञो य इच्छेद्भूतिमात्मनः
\twolineshloka
{सर्वथा क्षमिणा भाव्यं तथा सत्यपरेण च}
{वीतहर्षभयक्रोधो धृतिमाप्नोति पण्डितः}


\twolineshloka
{अद्रोहः सर्वभूतेषु कर्मणा मनसा गिरा}
{अनुग्रहश्च दानं च सतां धर्मः सनातनः}


\twolineshloka
{एते त्रयोदशाकाराः पृथक्सत्यैकलक्षणाः}
{भजन्ते सत्यमेवेह बृंहयन्ते च भारत}


\twolineshloka
{नान्तः शक्यो गुणानां च वक्तुं सत्यस्य पार्थिव}
{अतः सत्यं प्रशंसन्ति विप्राः सपितृदेवताः}


\twolineshloka
{नास्ति सत्यात्परो धर्मो नानृतात्पातकं परम्}
{स्थितिर्हि सत्यं धर्मस्य तस्मात्सत्यं न लोपयेत्}


\twolineshloka
{उपैति सत्याद्दानं हि तथा यज्ञाः सदक्षिणाः}
{त्रेताग्निहोत्रं वेदाश्च ये चान्ये धर्मनिश्चयाः}


\twolineshloka
{अश्वमेधसहस्त्रं च सत्यं च तुलया धृतम्}
{अश्वमेधसहस्राद्धि सत्यमेव विशिष्यते}


\chapter{अध्यायः १६१}
\twolineshloka
{युधिष्ठिर उवाच}
{}


\twolineshloka
{यतः प्रभवति क्रोधः कामो वा भरतर्षभ}
{शोकमोहौ विधित्सा च परासुत्वं तथा मदः}


\threelineshloka
{लोभो मात्सर्यमीर्ष्या च कुत्साऽसूया कृपा भयम्}
{एतत्सर्वं महाप्राज्ञ याथातथ्येन मे वद ॥भीष्म उवाच}
{}


\twolineshloka
{त्रयोदशैतेऽतिबलाः शत्रवः प्राणिनां स्मृताः}
{उपासते महाराज समन्तात्पुरुषानिह}


\twolineshloka
{एते प्रमत्तं पुरुषमप्रमत्तास्तुदन्ति च}
{वृका इव विलुम्पन्ति दृष्ट्वेव पुरुषेतरान्}


\twolineshloka
{एभ्यः प्रवर्तते दुःखमेभ्यः पापं प्रवर्तते}
{इति मर्त्यो विजानीयात्सततं पुरुषर्षभ}


\threelineshloka
{एतेषामुदयं स्थानं क्षयं च पृथिवीपते}
{हन्त ते कथयिष्यामि क्रोधस्योत्पत्तिमादितः}
{यथातत्त्वं क्षितिपते तन्मे निगदतः शृणु}


\twolineshloka
{लोभात्क्रोधः प्रभवति परदोषैरुदीर्यते}
{क्षमया तिष्ठते राजन्क्षमया विनिवर्तते}


\twolineshloka
{संकल्पाज्जायते कामः सेव्यमानो विवर्धते}
{यदा प्राज्ञो विरमते तदा सद्यः प्रणश्यति}


\threelineshloka
{[परामूया क्रोधलोभावन्तरा प्रतिमुच्यते}
{दयया सर्वभूतानां निर्वेदाद्विनिवर्तते}
{]अवद्यदर्शनादेति तत्त्वज्ञानाच्च नश्यति}


\twolineshloka
{अज्ञानप्रभवो मोहः पापाभ्यासात्प्रवर्तते}
{यदा प्राज्ञेषु रमते तदा सद्यः प्रणश्यति}


\twolineshloka
{विरुद्धानीह शास्त्राणि ये पश्यन्ति कुरूद्वह}
{विधित्सा जायते तेषां तत्त्वज्ञानान्निवर्तते}


\twolineshloka
{प्रीतेः शोकः प्रभवति वियोगात्तस्य देहिनः}
{यदा निरर्थकं वेत्ति तदा सद्यः प्रणश्यति}


\twolineshloka
{परासुता क्रोधलोभादभ्यासाच्च प्रवर्तते}
{दयया सर्वभूतानां निर्वेदात्सा निवर्तते}


\twolineshloka
{सत्यत्यागात्तु मात्सर्यमहितानां च सेवया}
{एतत्तु क्षीयते तात साधूनामुपसेवनात्}


\twolineshloka
{कुलाञ्ज्ञानात्तथैश्वर्यान्मदो भवति देहिनाम्}
{एभिरेव तु विज्ञातैर्मदः सद्यः प्रणश्यति}


\twolineshloka
{ईर्ष्या कामात्प्रभवति संहर्षाच्चैव जायते}
{इतरेषां तु सत्वानां प्रज्ञया सा प्रणश्यति}


\twolineshloka
{विभ्रमाल्लोकबाह्यानां द्वेष्यैर्वाक्यैरसंमतैः}
{कुत्सा संजायते राजँल्लोकान्प्रेक्ष्याभिशाम्यति}


\twolineshloka
{प्रतिकर्तुं न शक्ता ये बलस्थायापकारिणे}
{असूया जायते तीव्रा कारुण्याद्विनिवर्तते}


\twolineshloka
{कृपणान्सततं दृष्ट्वा ततः संजायते कृपा}
{धर्मनिष्ठां यदा वेत्ति तदा शाम्यति सा कृपा}


\twolineshloka
{अज्ञानप्रभवो लोभो भूतानां दृश्यते सदा}
{अस्थिरत्वं च भोगानां दृष्ट्वा ज्ञात्वा निवर्तते}


\threelineshloka
{एतान्येव जितान्याहुः प्रशान्तेन त्रयोदश}
{एते हि धार्तराष्ट्राणां सर्वे दोषास्त्रयोदश}
{त्वया सत्यार्थिना नित्यं विजिता जेष्यता चते}


\chapter{अध्यायः १६२}
\twolineshloka
{युधिष्ठिर उवाच}
{}


\twolineshloka
{आनृशंस्यं विजानामि दर्शनेन सतां सदा}
{नृशंसान्न विजानामि तेषां कर्म च भारत}


\twolineshloka
{कण्टकान्कूपमग्निं च वर्जयन्ति यथा नराः}
{तथा नृशंसकर्माणं वर्जयन्ति नरा नरम्}


\threelineshloka
{नृशंसो दह्यते व्यक्तं प्रेत्य चेह च भारत}
{तस्मात्त्वं ब्रूहि कौरव्य तस्य धर्मविनिश्चयम् ॥भीष्म उवाच}
{}


% Check verse!
स्पृहाऽस्यान्तर्गता चैव विदितार्था च कर्मणाम्आक्रोष्टा क्रुश्यते चैव बन्धिता बध्यते स च
\twolineshloka
{दत्तानुकीर्तिर्विषमः क्षुद्रो नैकृतिकः शठः}
{असंभोगी च मानी च तथा सङ्गी विकत्थनः}


\twolineshloka
{सर्वातिशङ्की पुरुषो बलीशः कृपणोऽथवा}
{वर्गप्रशंसीं सततमाश्रमद्वेपसंकरी}


\twolineshloka
{हिंसाविकारी सततमविशेषगुणागुणः}
{बह्वलीको मनस्वी च लुब्धोऽत्यर्थं नृशंसकृत्}


\twolineshloka
{धर्मशीलं गुणोपेतं पाप इत्यवगच्छति}
{आत्मशीलोपमानेन न विश्वसिति कस्यचित्}


\twolineshloka
{परेषां यत्र दोपः स्यात्तद्गुह्यं संप्रकाशयेत्}
{समानेष्वेव दोपेषु वृत्त्यर्थमुपघातयेत्}


\twolineshloka
{तथोपकारिणं चैव मन्यते वञ्चितं परम्}
{दत्त्वाऽपि च धनं काले संतपत्युपकारिणे}


\twolineshloka
{भक्ष्यं पेयमथालेह्यं यच्चान्यत्साधु भोजनम्}
{प्रेक्षमाणेषु योऽश्नीयान्नृशंसमिति तं वदेत्}


\twolineshloka
{ब्राह्मणेभ्यः प्रदायाग्रं यः सुहृद्भिः सहाश्नुते}
{स प्रेत्य लभते स्वर्गमिह चानन्त्यमश्नुते}


\twolineshloka
{एष ते भरतश्रेष्ठ नृशंसः परिकीर्तितः}
{सदा विवर्जनीयो हि पुरुषेण बुभूषता}


\chapter{अध्यायः १६३}
\twolineshloka
{भीष्म उवाच}
{}


\twolineshloka
{कृतार्थी यक्ष्यमाणश्च सर्ववेदान्तगश्च यः}
{आचार्यपितृकायार्थं स्वाध्यायार्थमथापि च}


\twolineshloka
{एते वै साधवो दृष्टा ब्राह्मणाः धर्मभिक्षवः}
{निःस्वेभ्यो देयमेतेभ्यो दानं विद्या च भारत}


\twolineshloka
{अन्यत्र दक्षिणादानं देयं भरतसत्तम}
{अन्येभ्योऽपि वहिर्वेदि न कृतान्नं विधीयते}


\twolineshloka
{सर्वरत्नानि राजा हि यथार्हं प्रतिपादयेत्}
{ब्राह्मणायैव यज्ञाश्च सहान्नाः सहदक्षिणाः}


\threelineshloka
{अन्येभ्यो विमलाचारा यजन्ते गुणतः सदा}
{यस्य त्रैवार्पिकं भक्तं पर्याप्तं भृत्यवृत्तये}
{अधिकं चापि विद्येत स सोमं पातुमर्हति}


\twolineshloka
{यज्ञश्चेत्प्रतिरुद्धः स्यादंशेनैकेन यज्वनः}
{ब्राह्मणस्य विशेषेण धार्मिके सति राजनि}


\twolineshloka
{यो वैश्यः स्याद्बहुपशुर्हीनक्रतुरसोमपः}
{कुटुम्बात्तस्य तद्वित्तं यज्ञार्थं पार्थिवो हरेत्}


\twolineshloka
{आहरेद्द्रुह्यतः किंचित्कामं शूद्रस्य वेश्मनि}
{न हि वेश्मनि शूद्रस्य किंचिदस्ति परिग्रहः}


\twolineshloka
{योऽनाहिताग्निः शतगुरयज्वा त्त सहस्रगुः}
{तयोरपि कुटुम्बाभ्यामाहरेदविचारयन्}


\twolineshloka
{अदातृभ्यो हरेद्वित्तं विख्याप्य नृपतिः सदा}
{तथैवाचरतो धर्मो नृपतेः स्यादथाखिलः}


\twolineshloka
{तथैव सप्तमे भक्ते भक्तानि पडनश्नतः}
{अश्वस्तनविभागेन हर्तव्यं हीनकर्मणः}


\threelineshloka
{खलात्क्षेत्रात्तथागाराद्यतो वाऽप्युपपद्यते}
{आख्यातव्यं नृपस्यैतत्पृच्छतोऽपृच्छतोपि वा}
{न तस्मै धारयेद्दण्डं राजा धर्मेण धर्मवित्}


\twolineshloka
{क्षत्रियस्य तु बालिश्याद्ब्राह्मणः क्लिश्यते क्षुधा}
{श्रुतशीले समाज्ञाय वृत्तिमस्य प्रकल्पयेत्}


\twolineshloka
{अथैनं परिरक्षेत पिता पुत्रमिवौरसम्}
{}


\twolineshloka
{इष्टिं वैश्वानरीं नित्यं निर्वपेदब्दपर्यये}
{अविकल्पः पुरा धर्मो धर्मवादैस्तु केवलः}


\twolineshloka
{विश्वैर्देवैश्च साध्यैश्च ब्राह्मणैश्च महर्षिभिः}
{आपत्सु गरणाद्भीतैर्लिङ्गः प्रतिनिधीकृतः}


\twolineshloka
{प्रभुः प्रथमकल्पस्य योऽनुकल्पेन वर्तते}
{स नाप्नोति फलं तस्य प्रेत्य चेह च दुर्मतिः}


\threelineshloka
{न ब्राह्मणो वेदयीत किंचिद्राजनि धर्मवित्}
{अविद्यावेदनाद्विद्यात्स्ववीर्यं वीर्यवत्तरम्}
{तस्माद्राज्ञः सदा तेजो दुःसहं ब्रह्मवादिनाम्}


\twolineshloka
{मन्ता शास्ता विधाता च ब्राह्मणो देव उच्यते}
{तस्मिन्नाकुशलं ब्रूयान्न शुष्कामीरयेद्गिरम्}


\twolineshloka
{क्षत्रियो बाहुवीर्येण तरेदापदमात्मनः}
{धनैर्वैश्यश्च शुद्रश्च मन्त्रैर्होमैश्च वै द्विजः}


\twolineshloka
{नैव कन्या न युवतिर्नामन्त्रज्ञो न बालिशः}
{परिवेष्टाऽग्निहोत्रस्य भवेन्नासंस्कृतस्तथा}


\twolineshloka
{नरके निपतन्त्येते जुह्वानाः सवनस्य तत्}
{तस्माद्वैतानकुशलो होता स्याद्वेदपारगः}


\twolineshloka
{प्राजापत्यमदत्त्वा च अग्न्याधेयस्य दक्षिणाम्}
{अनाहिताग्निरिति सं प्रोच्यते धर्मदर्शिभिः}


\twolineshloka
{पुण्यान्यन्यानि कुर्वीत श्रद्दधानो जितेन्द्रियः}
{अनाप्तदक्षिणैर्यज्ञैर्न यजेत कथंचन}


\twolineshloka
{प्रजाः पंशूश्च स्वर्गं च हन्ति यज्ञो ह्यदक्षिणः}
{इन्द्रियाणि यशः कीर्तिभायुश्चाप्यवकृन्तति}


\twolineshloka
{उदक्यामासते ये च द्विजाः केचिदनग्नयः}
{कुलं चाश्रोत्रियं येषां सर्वे ते शूद्रकर्मिणः}


\twolineshloka
{उदपानोदके ग्रामे ब्राह्मणो वृपलीयतिः}
{अपित्वा द्वादश समाः शूद्रकर्मैव गच्छति}


\threelineshloka
{अभार्यी शयने विभ्रच्छूद्रं वृद्धं च वै द्विजः}
{अब्राह्मणं भन्यमानस्तृणेष्वासीत पृष्ठतः}
{तथा संशुध्यते राजञ्शृणु चात्र वचो मम}


\twolineshloka
{यदेकरात्रेण करोति पापंकृष्णं वर्णं ब्राह्मणः सेवमानः}
{स्थानासनाभ्यां विहरन्वती सत्रिभिर्वर्षैः शमयेदात्मपापम्}


\twolineshloka
{न नर्मयुक्तमतृतं हिनस्तिन स्त्रीषु राजन्न विवाहकाले}
{प्राणात्यये सर्वधनापहारेपञ्चानृतान्याहुरपातकानि}


\twolineshloka
{श्रद्दधानः शुभां विद्यां हीनादपि समाप्नुयात्}
{सुवर्णमपि चामेध्यादाददीताविचारयन्}


\twolineshloka
{स्त्रीरत्नं दुष्कुलाच्चापि विषादप्यमृतं पिबेत्}
{अदूष्या हि स्त्रियो रत्नमाप इत्येव धर्मतः}


\twolineshloka
{गोब्राह्मणहितार्थं च वर्णानां संकरेषु च}
{वैश्यो गृह्णीत शस्त्राणि परित्राणार्थमात्मनः}


\twolineshloka
{सुरापो ब्रह्महा चैव गुरुतल्पगतस्तथा}
{अचिरेण महाराज पतितो वै भवत्युत}


\twolineshloka
{सुवर्णहरणं स्तैन्यं विप्रस्वं चेति पातकम्}
{विहारो मद्यपानं च अगम्यागमनं तथा}


\twolineshloka
{पतितैः संप्रयोगश्च ब्राह्मणीयोनितस्तथा}
{अनिर्देश्यानि मन्यन्ते प्राणान्तानीति धारणा}


\twolineshloka
{संवत्सरेण पतति पतितेन सहाचरन्}
{याजनाध्यापनाद्दानान्न तु यानासनाशनात्}


\twolineshloka
{एतानि हित्वातोऽन्यानि निर्देश्यानीति धारणा}
{निर्देश्यकेन विधिना कालेनाव्यसनी भवेत्}


\twolineshloka
{अनुत्तीर्य न होतव्यं प्रेतकर्मण्युपाश्रिते}
{त्रिषु त्वेतेषु पूर्वेषु न कुर्वीत विचारणम्}


\twolineshloka
{अमात्यान्वा गुरून्वापि जह्याद्धर्मेण धार्मिकः}
{प्रायश्चित्तान्यकुर्वाणा नैते कुर्वन्ति संविदम्}


\twolineshloka
{अधर्मकारी धर्मेण तपसा हन्ति किल्विषम्}
{ब्राह्मणायावगुर्येत स्पृष्टे गुरुतरं भवेत्}


\threelineshloka
{अस्तेनं स्तेन इत्युक्त्वा द्विगुणं पापमाप्नुयात्}
{त्रिभागं ब्रह्महत्यायाः कन्यां प्राप्नोति दुष्यति}
{यस्तु दूषयिता तस्याः शेषं प्राप्नोति पाप्मनः}


\twolineshloka
{ब्राह्मणानवगर्ह्येह स्पृष्ट्वा गुरुतरं भवेत्}
{वर्षाणां हि शतं पापः प्रतिष्ठां नाधिगच्छति}


\twolineshloka
{सहस्रं चैव वर्षाणां निपत्य नरकं वसेत्}
{तस्मान्नैवावगुर्याद्धि नैव जातु निपातयेत्}


\twolineshloka
{शोणितं यावतः पांसून्संगृह्णीयाद्द्विजक्षतात्}
{तावतीः स समा राजन्नरके प्रतिपद्यते}


\twolineshloka
{भ्रूणहाऽऽहवमध्ये तु शुध्यते शस्त्रपाततः}
{आत्मानं जुहुयादग्नौ समिद्धे तेन शुध्यते}


% Check verse!
सुरापो वारुणीमुष्णां पीत्वा पापाद्विमुच्यते
\twolineshloka
{तया स काये निर्दग्धे मृत्युं वा प्राप्य शुध्यति}
{लोकांश्च लभते विप्रो नान्यथा लभते हि सः}


\twolineshloka
{गुरुतल्पमधिष्ठाय दुरात्मा पापचेतनः}
{शिलां ज्वलन्तीमासाद्य मृत्युना सोभिशुध्यति}


\twolineshloka
{अधवा शिश्नवृषणावादायाञ्जलिना स्वयम्}
{नैर्ऋतीं दिशमास्थाय निपतेत्सत्वजिह्मगः}


% Check verse!
ब्राह्मणार्थेऽपि वा प्राणान्संत्यजंस्तेन शुध्यति
\twolineshloka
{अश्वमेधेन वाऽपीष्ट्वा अथवा गोसवेन वा}
{मरुत्सोमेन वा सम्यगिह प्रेत्य च पूज्यते}


\twolineshloka
{तथैव द्वादशसमाः कापोतं धर्ममाचरेत्}
{एककालं चरेद्भैक्षं स्वकर्मोदाहरञ्जने}


\twolineshloka
{एवं वा तपसा युक्तो ब्रह्महा सवनी भवेत्}
{एवं गर्भमविज्ञातमात्रेयीं वा निपातयेत्}


\twolineshloka
{द्विगुणा ब्रह्महत्या वै आत्रेयीहिंसने भवेत्}
{सुरापी नियताहारो ब्रह्मचारी क्षपाचरः}


\twolineshloka
{ऊर्ध्वं त्रिभ्योऽपि वर्षेभ्यो यजेताग्निष्टुता परम्}
{ऋषभैकसहस्रं वा गा दत्त्वा शौचमाप्नुयात्}


\twolineshloka
{वैश्यं दत्त्वा तु वर्षे द्वे ऋषभैकशतं च गाः}
{शूद्रं हत्वाऽब्दमेवैकमृषभं च शतं च गाः}


\twolineshloka
{श्ववराहखरान्हत्वा शौद्रमेव व्रतं चरेत्}
{मार्जारचाषमण्डूकान्काकं व्यालं च मूषिकम्}


\threelineshloka
{उक्तः पशुवधे दोषो राजन्प्राणिनिपातनात्}
{`अनस्थिकेषु गोमूत्रं पानमेकं प्रचक्षते}
{'प्रायश्चित्तान्यथान्यानि प्रवक्ष्याम्यनुपूर्वशः}


\twolineshloka
{तल्पे वाऽन्यस्य चौर्ये च पृथक् संवत्सरं चरेत्}
{त्रीणि श्रोत्रियभार्यायां परदारे च द्वे स्मृते}


\twolineshloka
{काले चतुर्थे भुञ्जानो ब्रह्मचारी व्रती भवेत्}
{स्थानासनाभ्यां विहरेत्रिरह्नाऽभ्युपयन्नपः}


\twolineshloka
{`ऐवमेव चरन्राजंस्तस्मात्पापात्प्रमुच्यते}
{'एवमेव निराकर्ता यश्चाग्नीनपविध्यति}


\twolineshloka
{त्यजत्यकारणे यश्च पितरं मातरं गुरुम्}
{पतितः स्यात्स कौरव्य यथा धर्मेषु निश्चयः}


\twolineshloka
{ग्रासाच्छादनयानं च शयनं ह्यासनं तथा}
{`ब्रह्मचारी द्विजेभ्यश्च दत्त्वा पापात्प्रमुच्यते ॥'}


\threelineshloka
{भार्यायां व्यभिचारिण्यां निरुद्धायां विशेषतः}
{यत्पुंसः परदारेषु तदेनां चारयेद्व्रतम्}
{}


\twolineshloka
{श्रेयांसं शयने हित्वा पापीयांसं समृच्छति}
{श्वभिस्तमर्दयेद्राजा संस्थाने बहुविस्तरे}


\twolineshloka
{पुमांसं बन्धयेत्पाशैः शयने तप्त आयसे}
{अप्यादधीत दारूणि तत्र दह्येत पापकृत}


\twolineshloka
{एव दण्डो महाराज स्त्रीणां भर्तृव्यतिक्रमे}
{संवत्सारोऽभिशस्तस्य दुष्टस्य द्विगुणो भवेत्}


\twolineshloka
{द्वे तस्य त्रीणि वर्षाणि चत्वारि सहसेविनः}
{कुमारः पञ्चवर्षाणि चरेद्भैक्षं मुनिव्रतः}


\twolineshloka
{परिवित्तिः परिवेत्ता या चैव परिविद्यते}
{पाणिग्राहस्त्वधर्मेण सर्वे ते पतिताः स्मृताः}


\twolineshloka
{चरेयुः सर्व एवैते वीरहा यद्व्रतं चरेत्}
{चान्द्रायणं चरेन्मासं कृच्छ्रं वा पापशुद्धये}


\threelineshloka
{परिवेत्ता प्रयच्छेता तां स्नुषां परिवित्तये}
{ज्येष्ठेन त्वभ्यनुज्ञातो यवीयाप्यनन्तरम्}
{एनसो मोक्षमाप्नोति तौ च सा चैव धर्मतः}


\twolineshloka
{अमानुषीषु गोवर्जमनादिष्टं न दुष्यति}
{अधिष्ठातारमत्तारं पशूनां पुरुषं विदुः}


\twolineshloka
{परिधायोर्ध्ववालं तु पात्रमादाय मृन्मयम्}
{चरेत्सप्तगृहान्भैक्षं स्वकर्म परिकीर्तयन्}


\twolineshloka
{तथैव लब्धभोजी स्याद्द्वादशाहात्स शुध्यति}
{चरेत्संवत्सरं चापि तद्व्रतं येन कृन्तति}


\twolineshloka
{भवेत्तु मानुषेष्वेवं प्रायश्चिमनुत्तमम्}
{दानं वा दानशक्तेषु सवेर्मतत्प्रकल्पयेत्}


\twolineshloka
{अनास्तिकेषु गोमात्रं दानमेकं प्रचक्षते}
{श्ववराहमनुष्याणां कुक्कुटस्य खरस्य च}


\twolineshloka
{मांसं मूत्रं पुरीषं च प्राश्य संस्कारमर्हति}
{ब्राह्मणस्तु सुरापस्य गन्धमादाय सोमपाः}


\twolineshloka
{अपख्यहं पिबेदुष्णाः संयतात्मा जितेन्द्रियः}
{अपः पीत्वा तु स पुनर्वायुभक्षो भवेन्त्र्यहम्}


\twolineshloka
{एवमेतत्समुद्दिष्टं प्रायश्चित्तिषेवणम्}
{ब्राह्मणस्य विशेषेण यदज्ञानेन जायते}


\chapter{अध्यायः १६४}
\twolineshloka
{वैशंपायन उवाच}
{}


\threelineshloka
{कथान्तरमथासाद्य स्वङ्गयुद्धविशारदः}
{नकुलः शरतल्पस्थमिदमाह पितामहाम् ॥नकुल उवाच}
{}


\twolineshloka
{धनुः प्रहरणं श्रेष्ठमितिवादः पितामह}
{मतस्तु मम धर्मज्ञः खङ्ग एव सुसंशितः}


\twolineshloka
{छिन्ने च कार्मुके राजन्प्रक्षीणेषु शरेषु च}
{खङ्गेन शक्यते योद्धुमात्मानं परिरक्षितुम्}


\twolineshloka
{शरासनधरांश्चैव गदाशक्तिधरांस्तदा}
{एकः खङ्गधरो वीरः समर्थः प्रतिबाधितुम्}


\twolineshloka
{अत्र मे संशयश्चैव कौतूहलमतीव च}
{किंस्वित्प्रहरणं श्रेष्ठं सर्वयुद्धेषु पार्थिव}


\threelineshloka
{कथं चोत्पादितः खङ्गः कस्मै चार्थाय केन वा}
{पूर्वाचार्यं च खङ्गस्य प्रव्रवीहि पितामह ॥वैशंपायन उवाच}
{}


\twolineshloka
{तस्य तद्वचनं श्रुत्वा माद्रीपुत्रस्य धीमतः}
{स्वरकौशलसंयुक्तं सूक्ष्मचित्रार्थवत्सुखम्}


\twolineshloka
{ततस्तस्योत्तरं वाक्यं स्वरवर्णोपपादितम्}
{शिक्षया चोपपन्नाय द्रोणशिष्याय पृच्छते}


\threelineshloka
{उवाच सर्वधर्मज्ञो धनुर्वेदस्य पारगः}
{शरतल्पगतो भीष्मो नकुलाय महात्मने ॥भीष्म उवाच}
{}


\twolineshloka
{तत्त्वं शृणुष्व माद्रेय यथैतत्परिपृच्छसि}
{प्रबोधितोऽस्मि भवता सानुमानिव पर्वतः}


\twolineshloka
{सलिलैकार्णवं तात पुरा सर्वमभूदिदम्}
{अप्रज्ञातमनाकाशमनिर्देश्यमहीतलम्}


\twolineshloka
{तमस्संवृतमस्पर्शमतिगम्भीरदर्शनम्}
{निःशब्दं चाप्रमेयं च तत्र जज्ञे पितामहः}


\twolineshloka
{सोऽसृजद्वायुमग्निं च भास्करं चापि वीर्यवान्}
{आकाशममृजच्चोर्ध्वमध्नो भूमिं च नैर्ऋतिम्}


\threelineshloka
{ततः सचन्द्रतारं च नक्षत्राणि ग्रहांस्तथा}
{संवत्सरानहोरात्रानृतूनथ लवान्क्षणान्}
{}


\threelineshloka
{ततः शरीरं लोकस्थं स्थापयित्वा पितामहः}
{जनयामास भगवान्पुत्रानुत्तमतेजसः}
{}


\twolineshloka
{मरीचिं भृगुमत्रिं च पुलस्त्यं पुलहं क्रतुम्}
{वसिष्ठाङ्गिरसौ चोभौ भरद्वाजं तथैव च}


\twolineshloka
{प्रजापतिस्तथा दक्षः कन्याः षष्टिमजीजनात्}
{ताश्च ब्रह्मर्पीन्सर्वान्प्रजार्थं प्रतिपेदिरे}


\twolineshloka
{ताभ्यो विश्वानि भूतानि देवाः पितृगणास्तथा}
{गन्धर्वाप्सरसश्चैव रक्षासि विविधानि च}


\twolineshloka
{पतत्रिमृगमीनाश्च गावश्चैव महोरगाः}
{नानाकृतिबलाश्चान्ये जलक्षितिविचारिणः}


\twolineshloka
{उद्भिज्जाः स्वेदजाश्चैव साण्डजाश्च जरायुजाः}
{अज्ञे तात जगत्सर्वं तथा स्थावरजङ्गमम्}


\twolineshloka
{अतः सर्गमिमं कृत्वा सर्वलोकपितामहः}
{अश्वतं वेदपठितं धर्मं च जुजुपे पुनः}


\threelineshloka
{स्मन्धर्मे स्थिता देवाः सहाचार्यपुरोहिताः}
{---त्या वसवो रुद्राः ससाध्या मरुदश्विनः}
{}


\twolineshloka
{भृ---त्र्यङ्गिरसः सिद्धाः कश्यपश्च तपोधनाः}
{वष्ठगौतमागस्त्यास्तथा नारदपर्वतौ}


\twolineshloka
{क्रयो बालखिल्याश्च प्रभासाः सिकतास्तथा}
{घृगाच्या सोमवायव्या वैश्वानरमरीचिपाः}


\twolineshloka
{करूपाश्चैव हंसाश्च ऋषयो वाऽग्नियोनयः}
{---पप्रस्थाः पृश्नयश्च स्थिता ब्रह्मानुशासने}


\twolineshloka
{दानवेन्द्रास्त्वतिक्रम्य तत्पितामहशासनम्}
{धर्मस्यापनयं चक्रुः क्रोधलोभसमन्विताः}


\twolineshloka
{हिरण्यकशिपुश्चैव हिरण्याक्षो विरोचनः}
{शम्बरो विप्रचित्तिश्च प्रह्लादो नमुचिर्बलिः}


\twolineshloka
{एते चान्ये च बहवः सगणा दैत्यदानवाः}
{धर्मसेतुमतिक्रम्य रेमिरेऽधर्मनिश्चयाः}


\twolineshloka
{सर्वे तुल्याभिजातीया यथा देवास्तथा वयम्}
{इत्येवं हेतुमास्थाय स्पर्धमानाः सुरर्षिभिः}


\twolineshloka
{न प्रियं नाप्यनुक्रोशं चक्रुर्भूतेषु भारत}
{त्रीनुपायानतिक्रम्य दण्डेन रुरुधुः प्रजाः}


\twolineshloka
{न जग्मुः संविदं तैश्च दर्पादसुरसत्तमाः}
{अथ वै भगवान्ब्रह्मा सर्वलोकनमस्कृतः}


\twolineshloka
{तदा हिमवतः पृष्ठे सुरम्ये पद्मतारके}
{शतयोजनविस्तारे मणिमुक्ताचयाचिते}


\twolineshloka
{तस्मिन्गिरिवरे पुत्र पुष्पितद्रुमकानने}
{तस्थौ स विबुधश्रेष्ठो ब्रह्मा लोकार्थसिद्धये}


\twolineshloka
{ततो वर्षसहस्रान्ते वितानमकरोत्प्रभुः}
{विधिना कल्पदृष्टेन यथोक्तेनोपपादितम्}


\twolineshloka
{ऋषिभिर्यज्ञपटुभिर्यथावत्कर्मकर्तृभिः}
{मरुद्भिः परिसंकीर्णं दीप्यमानैश्च पावकैः}


\twolineshloka
{काञ्चनैर्यज्ञभाण्डैश्च भ्राजिष्णुभिरलंकृतम्}
{वृतं देवगणैश्चैव प्रबभौ यज्ञमण्डलम्}


\twolineshloka
{तथा ब्रह्मर्षिभिश्चैव सदस्यैरुपशोभितम्}
{अत्र घोरतमं वृत्तमृषीणां मे परिश्रुतम्}


\twolineshloka
{चन्द्रमा विमलं व्योम यथाऽभ्युदिततारकम्}
{विदार्याग्निं तथा भूतमुत्थितं श्रूयते तदा}


\twolineshloka
{लोनीत्पलसवर्णाभं तीक्ष्णदंष्ट्रं कृशोदरम्}
{प्रांशुमुद्धर्षणं चापि तथैव ह्यमितौजसम्}


\twolineshloka
{अस्मिन्नुत्पद्यमाने च प्रचचास वसुंधरा}
{महोर्मिकलिलावर्तश्रुक्षुभे स महोदधिः}


\threelineshloka
{पेतुश्चोत्का महोत्पाताः शाखाश्च मुमुचुर्द्रुमाः}
{अप्रसन्ना दिशः सर्वाः पवनश्चाशिवो ववौ}
{मुहुर्मुहुश्च भूतानि प्राव्यथन्त भयात्तथा}


\twolineshloka
{ततः स तुमुलं दृष्ट्वा तद्भूतं समुपस्थितम्}
{महर्षिसुरगन्धर्वानुवाचेदं पितामहः}


\twolineshloka
{मयैवं चिन्तितं भूतमसिर्नामैष वीर्यवान्}
{रक्षणार्थाय लोकस्य वधाय च सुरद्विषाम्}


\twolineshloka
{ततस्तद्रुपमुत्सृज्य बभौ निस्त्रिंश एव सः}
{विमलस्तीक्ष्णधारश्च कालान्तक इवोद्यतः}


\twolineshloka
{ततस्तं नीलकण्ठाय रुद्रायर्षभकेतवे}
{ब्रह्मा ददावसिं तीक्ष्णमधर्मप्रतिवारणम्}


\twolineshloka
{ततः स भगवान्रुद्रो ब्रह्मर्षिगणपूजितः}
{प्रगृह्मासिममेयात्मा रूपमन्यच्चकार ह}


\twolineshloka
{चतुर्बाहुः स्पृशन्मूर्ध्ना भूमिष्ठोऽपि दिशो दश}
{ऊर्ध्वदृष्टिर्महाबाहुर्मुखाज्ज्वालाः समुत्सृजन्}


\twolineshloka
{विकुर्वन्बहुधा वर्णान्नीलपाण्डुरलोहितान्}
{बिभ्रत्कृष्णाजिनं वासो हेमप्रवरतारकम्}


\twolineshloka
{नेत्रं चैकं ललाटस्थं भास्करप्रतिमं महत्}
{शुशुभाते सुविमले द्वे नेत्रे कृष्णपिङ्गले}


\twolineshloka
{ततो देवो महादेवः शूलपाणिर्भगाक्षिहा}
{संप्रगृह्य तु निस्त्रिंशं कालाग्निसमवर्चसम्}


\threelineshloka
{त्रिकूटं चर्म चोद्यम्य सविद्युतमिवाम्बुदम्}
{चचार विविधान्मार्गान्दानवान्तचिकीर्षया}
{विधुन्वन्नसिमाकाशे तथा युद्धचिकीर्षया}


\twolineshloka
{तस्य नादं विनदतो महाहासं च मुञ्चतः}
{बभौ प्रतिभयं रूपं तदा रुद्रस्य भारत}


\twolineshloka
{तद्रूपधारिणं रुद्रं रौद्रकर्मचिकीर्षया}
{निशाम्य दानवाः सर्वे हृष्टाः समभिदुद्रुवुः}


\twolineshloka
{अश्मभिश्चाम्यवर्षन्त प्रदीप्तैश्च तथोल्मुकैः}
{घोरैः प्रहरणैश्चान्यैः क्षुरधारैरयस्मयैः}


\twolineshloka
{ततस्तु दानवानीकं संप्रणेतृकमप्युत}
{खङ्गं दृष्ट्वा बलाधूतं प्रमुमोह चचाल च}


\twolineshloka
{चित्रं शीघ्रपदत्वाच्च चरन्तमसिपाणिनम्}
{तमेकमसुराः सर्वे सहस्रमिति मेनिरे}


\twolineshloka
{छिन्दन्भिन्दन्रुजन्कृन्तन्दारयन्प्रथमन्नपि}
{अचरद्वैरिसङ्घेषु दावाग्निरिव कक्षगः}


\twolineshloka
{असिवेगप्रभग्नास्ते छिन्नबाहूरुवक्षसः}
{उत्तमाङ्गप्रकृत्ताश्च पेतुरुर्व्यां महाबलाः}


\twolineshloka
{अपरे दानवा भग्नाः खङ्गधारावपीडिताः}
{अन्योन्यमभिनर्दन्तो दिशः संप्रतिपेदिरे}


\twolineshloka
{भूमिं केचित्प्रविविशुः पर्वतानपरे तथा}
{अपरे जग्मुराकासमपरेऽम्भः समाविशन्}


\twolineshloka
{तस्मिन्महति संवृत्ते समरे भृशदारुणे}
{बभूव भूः प्रतिभया मांसशोणितकर्दमा}


\twolineshloka
{दानवानां शरीरैश्च पतितैः शोणितोक्षितैः}
{समाकीर्णा महाबाहो शैलैरिव सकिंशुकैः}


\twolineshloka
{`रुधिरेण परिक्लिन्ना प्रबभौ वसुधा तदा}
{रक्तार्द्रवसना श्यामा नारीव मदविह्वला ॥'}


\twolineshloka
{स रुद्रो दानवान्हत्वा कृत्वा धर्मोत्तरं जगत्}
{रौद्रं रुपमथाक्षिप्य चक्रे रूपं शिवं शिवः}


\twolineshloka
{ततो महर्षयः सर्वे सर्वे देवगणास्तथा}
{जयेनाद्भुतकल्पेन देवदेवमथास्तुवन्}


\twolineshloka
{ततः स भगवान्रुद्रो दानवक्षतजोक्षितम्}
{असिं धर्मस्य गोप्तारं ददौ सत्कृत्य विष्णवे}


\twolineshloka
{विष्णुर्मरीचये प्रादान्मरीचिर्भार्गवाय तम्}
{महर्षिभ्यो ददौ खङ्गमृषयो वासवाय च}


\twolineshloka
{महेन्द्रो लोकपालेभ्यो लोकपाला तु पुत्रक}
{मनवे सूर्यपुत्राय ददुः खङ्गं सुविस्तरम्}


\twolineshloka
{ऊचुश्चैनं तथा वाक्यं मानुषाणां त्वमीश्वरः}
{असिना धर्मगर्भेण पालयस्व प्रजा इति}


\twolineshloka
{धर्मसेतुमतिक्रान्ताः स्थूलसूक्ष्मार्थकारणात्}
{विभज्य दण्डं रक्ष्याः स्युर्धर्मतो न यदृच्छया}


\twolineshloka
{दुर्वाचा निग्रहो दण्डो हिरण्यबहुलस्तथा}
{व्यङ्गता च शरीरस्य वधो नाल्पस्य करणात्}


\twolineshloka
{असेरेतानि रूपाणि दुर्वारादीनि नि दशेत्}
{असेरेवं प्रमाणानि परमाण्यभ्यतिक्रमात्}


\twolineshloka
{अभिषिच्याथ पुत्रं स्वं प्रजानामधिपं ततः}
{मनुः प्रजानां रक्षार्थं क्षुपाय प्रददावसिम्}


\twolineshloka
{क्षुपाज्जग्राह चेक्ष्वाकुरिक्ष्वाकोश्च पुरूरवाः}
{आयुश्च तस्माल्लोभे तं नहुषश्च ततो भुवि}


\twolineshloka
{ययातिर्नहुषाच्चापि पूरुस्तस्माच्च लब्धवान्}
{आधूर्तश्च गयस्तस्मात्ततो भूमिशयो नृपः}


\twolineshloka
{भरतश्चापि दौष्यन्तिर्लेभे भूमिशयादसिम्}
{तस्माल्लोभे च धर्मज्ञो राजन्नैलबिलस्तथा}


\twolineshloka
{ततस्त्वैलबिलाल्लेभे धुन्धुमारो नरेश्वरः}
{धुन्धुमाराच्च काम्भोजो मुचुकुन्दस्ततोऽभजत्}


\twolineshloka
{मुचुकुन्दान्मरुत्तश्च मरुत्तादपि रैवतः}
{रैवताद्युवनाश्वश्च युवनाश्वात्ततो रघुः}


\twolineshloka
{इक्ष्वाकुवंशजस्तस्माद्धरिणाश्वः प्रतापवान्}
{हरिणाश्वादसिं लेभे शुनकः शुनकादपि}


\twolineshloka
{उशीनरो वै धर्मात्मा तस्माद्भोजाः सयादवाः}
{यदुभ्यश्च शिबिर्लेभे शिबेश्चापि प्रतर्दनः}


\threelineshloka
{प्रतर्दनादष्टकश्च रुशदश्वोऽष्टकादपि}
{रुशदश्वाद्भरद्वाजो द्रोणस्तस्मात्कृपस्ततः}
{ततस्त्वं भ्रातृभिः सार्धं परमासिमवाप्तवान्}


\twolineshloka
{कृत्तिकास्तस्य नक्षत्रमसेरग्निश्च दैवतम्}
{रोहिण्यो गोत्रमस्याथ रुद्राश्च गुरुसत्तमाः}


\twolineshloka
{असेरष्टौ हि नामानि रहस्यानि निबोध मे}
{पाण्डवेय सदा यानि कीर्तयँल्लभते जयम्}


\twolineshloka
{असिर्विशसनः खङ्गस्तीक्ष्णचर्मा दुरासदः}
{श्रीगर्भो विजयश्चैव धर्मपालस्तथैव च}


\threelineshloka
{अग्र्यः प्रहरणानां च खङ्गो भुवि परिश्रुतः}
{महेश्वरप्रणीतश्च पुराणे निश्चयं गतः}
{`एतानि चैव नामानि पुराणे निश्चितानि वै ॥'}


\threelineshloka
{पृथुस्तूत्पादयामास धनुराद्यमरिंदमः}
{तेनेयं पृथिवी दुग्धा सस्यानि सुबहून्यपि}
{धर्मेण च यथापूर्वं वैन्येन परिरक्षिता}


\twolineshloka
{तदेतदार्षं माद्रेय प्रमाणं कर्तुमर्हसि}
{असेश्च पूजा कर्तव्या सदा युद्धविशारदैः}


\twolineshloka
{इत्येष प्रथमः कल्पो मया ते कथितः पुनः}
{एवमेवासिसर्गोऽयं यथावद्भरतर्षभ}


\twolineshloka
{सर्वथा तमिह श्रुत्वा स्वङ्गस्यागममुत्तमम्}
{लभते पुरुषः कीर्ति प्रेत्य चानन्त्यमश्नुते}


\chapter{अध्यायः १६५}
\twolineshloka
{वैशंपायन उवाच}
{}


\twolineshloka
{इत्युक्तवति भीष्मे तु तूष्णींभूते युधिष्ठिरः}
{पप्रच्छावसथं गत्वा भ्रातृन्विदुरपञ्चमान्}


\twolineshloka
{धर्मे चार्थे च कामे च लोकवृत्तिः समाहिता}
{तेषां गरीयान्कतमो मध्यमः को लघुश्च कः}


\twolineshloka
{कस्मिंश्चात्मा नियन्तव्यस्त्रिवर्गविजयाय वै}
{संपृष्टा नैष्ठिकं वाक्यं यथाबद्वक्तुमर्हथ}


\threelineshloka
{ततोऽर्थगतितत्त्वज्ञः प्रथमं प्रतिभानवान्}
{जगाद विदुरो वाक्यं धर्मशास्त्रमनुस्मरन् ॥विदुर उवाच}
{}


\twolineshloka
{बहुश्रुतं तपस्त्यागः श्रद्धा यज्ञक्रिया क्षमा}
{भावशुद्धिर्दया सत्यं संयमश्चात्मसंपदः}


\twolineshloka
{एतदेवाभिपद्यस्व मा ते भूच्चलितं मनः}
{एतन्मूलौ हि धर्मार्थावेतदेकपदं हितम्}


\twolineshloka
{धर्मेणैवर्षयस्तीर्णा धर्मे लोकाः प्रतिष्ठिताः}
{धर्मेण देवा दिवि च धर्मे चार्थः समाहितः}


\twolineshloka
{धर्मो राजन्गुणः श्रेष्ठो मध्यमो ह्यर्थ उच्यते}
{कामो यवीयानिति च प्रवदन्ति मनीषिणः}


\threelineshloka
{तस्माद्धर्मप्रधानेन भवितव्यं यतात्मना}
{तथा च सर्वभूतेषु वर्तितव्यं यतात्मना ॥वैशंपायन उवाच}
{}


\twolineshloka
{समाप्तवचने तस्मिन्भीमकर्मा धनञ्जयः}
{ततोऽर्थगतितत्त्वज्ञो जगौ वाक्यं प्रचोदितः}


\twolineshloka
{कर्मभूमिरियं राजन्निह वार्ता प्रशस्यते}
{कृषिर्वाणिज्यगोरक्षं शिल्पानि विविधानि च}


\twolineshloka
{अर्थ इत्येव सर्वेषां कर्मणामव्यतिक्रमः}
{निवृत्तेऽर्थे न वर्तेते धर्मकामाविति श्रुतिः}


\twolineshloka
{विषहेतार्थवान्धर्ममाराधयितुमुत्तमम्}
{कामं च चरितुं शक्तो दुष्प्रापमकृतात्मभिः}


\twolineshloka
{अर्थस्यावयवावेतौ धर्मकामाविति श्रुतिः}
{अर्थसिद्ध्या विनिर्वृत्तावृभावेतौ भविष्यतः}


\twolineshloka
{तद्गतार्थं हि पुरुषं विशिष्टतरयोनयः}
{ब्रह्माणमिव भूतानि सततं पर्युपासते}


\twolineshloka
{जटाजिनधरा दान्ताः पङ्कदिग्धा जितेन्द्रियाः}
{मुण्डा निस्तन्तवश्चापि वसन्त्यर्थार्थिनः पृथक्}


\twolineshloka
{काषायवसनाश्चान्ये श्मश्रुला हि सुसंयताः}
{विद्वांसश्चैव शान्ताश्च मुक्ताः सर्वपरिग्रहैः}


\threelineshloka
{`अर्थार्थिनः सन्ति नित्यं परितष्यन्ति कर्मभिः}
{'अर्थार्थिनः सन्ति केचिदपरे स्वर्गकाङ्क्षिणः}
{कुलप्रत्यागमाश्चैके स्वंस्वं धर्ममनुष्ठिताः}


\twolineshloka
{आस्तिका नास्तिकाश्चैव नियताः संयमे परे}
{अप्रज्ञानं तमोभूतं प्रज्ञानं तु प्रकाशिता}


\fourlineindentedshloka
{भृत्यान्भोगैर्द्विषो दण्डैर्यो योजयति सोऽर्थवान्}
{एतन्मतिमतांश्रेष्ठ मतं मम यथातथम्}
{अनयोस्तु विबोध त्वं वचनं शक्रकण्वयोः ॥वैशंपायन उवाच}
{}


\twolineshloka
{तथा धर्मार्थकुशलौ माद्रीपुत्रावनन्तरम्}
{नकुलः सहदेवश्च वाक्यमूचतुरुत्तमम्}


\twolineshloka
{आसीनश्च शयानश्च विचरन्नपि वा स्थितः}
{अर्थयोगं दृढं कुर्याद्योगैरुच्चावचैरपि}


\twolineshloka
{अस्मिंस्तु वै विनिर्वृत्ते दुर्लभे परमप्रिये}
{इह कामानवाप्नोति प्रत्यक्षं नात्र संशयः}


\twolineshloka
{योऽर्थो धर्मेण संयुक्तो धर्मो यश्चार्थसंयुतः}
{मध्विवामृतसंसृष्टं तस्मादेतौ मताविह}


\twolineshloka
{अनर्थस्य न कामोस्ति तथाऽर्थोऽधर्मिणः कुतः}
{तस्मादुद्विजले लोको धर्मार्थाभ्यां बहिष्कृतात्}


\twolineshloka
{तस्माद्धर्मप्रधानेन साध्योऽर्थः संयतात्मना}
{विश्वस्तेषु हि भूतेषु कल्पते सर्वमेव हि}


\threelineshloka
{धर्मं समाचरेत्पूर्वं ततोऽर्थं धर्मसंयुतम्}
{ततः कामं चरेत्पश्चात्सिद्धार्थस्य हि तत्फलम् ॥वैशंपायन उवाच}
{}


\twolineshloka
{विरेमतुस्तु तद्वाक्यमुक्त्वा तावश्विनोः सुतौ}
{भीमसेनस्ततो वाक्यमिदं वक्तुं प्रचक्रमे}


\twolineshloka
{नाकामः कामयत्यर्थं नाकामो धर्ममिच्छति}
{नाकामः कामयानोऽस्ति तस्मात्कामो विशिष्यते}


\twolineshloka
{कामेन युक्ता ऋषयस्तपस्येव समाहिताः}
{पलाशाः शाकमूलाशा वायुभक्षाः सुसंयताः}


\twolineshloka
{वेदोपवेदेष्वपरे युक्ताः स्वाध्यायपारगाः}
{श्राद्धे यज्ञक्रियायां च तथा दानप्रतिग्रहे}


\twolineshloka
{वणिजः कर्षका गोपाः कारवः शिल्पिनस्तथा}
{देशधर्मकृतश्चैव युक्ताः कामेन कर्मसु}


\twolineshloka
{समुद्रं वा विशन्त्यन्ये नराः कामेन संयुताः}
{कामो हि विविधाकारः सर्वं कामेन संततम्}


\twolineshloka
{नास्ति नासीन्न भविता भूतं काममृते परम्}
{एतत्सारं महाराज धर्मार्थावत्र संश्रितौ}


\twolineshloka
{नवनीतं यथा दध्नस्तथा कामोऽर्थधर्मतः}
{श्रेयस्तैलं न पिण्याको घृतं श्रेय उदश्वितः}


\threelineshloka
{श्रेयः पुष्पफलं काष्ठात्कामो धर्मार्थयोर्वरः}
{पुष्पतो मध्विव परं कामात्संजायते सुखम्}
{कामो धर्मार्थयोर्योनिः कामश्चाथ तदात्मकः}


\twolineshloka
{[नाकामतो ब्राह्मणाः स्वन्नमर्थान्नाकामतो ददति ब्राह्मणेभ्यः}
{नाकामतो विविधा लोकचेष्टातस्मात्कामः प्राक् त्रिवर्गस्य दृष्टः ॥]}


\twolineshloka
{सुचारुवेषाभिरलंकृताभिर्मदोत्कटाभिः प्रियवादिनीभिः}
{रमस्व योषिद्भिरुपेत्य कामंकामो हि राजन्परमाभिरामः}


\twolineshloka
{बुद्धिर्ममैषा परिखास्थितस्यमाभूद्विचारस्तव धर्मपुत्र}
{स्वात्संहितं सद्भिरफल्गुसारमसस्तवाक्यं परमानृशंसम्}


\twolineshloka
{वर्मार्थकामाः सममेव सेव्यायो ह्येकभक्तः स नरो जघन्यः}
{द्वयोस्तु सक्तं प्रवदन्ति मध्यमंस उत्तमो योऽभिरतस्त्रिवर्गे}


\twolineshloka
{प्राज्ञः सुहृच्चन्दनसारलिप्तोविचित्रमाल्याभरणैरुपेतः}
{ततो वचः संग्रहविस्तरेणप्रोक्त्वाऽथ वीरान्विरराम भीमः}


\twolineshloka
{ततो मुहूर्तादथ धर्मराजोवाक्यानि तेषामनुचिन्त्य सम्यक्}
{उवाच वाचाऽवितथं स्मयन्वैबहुश्रुतो धर्मभृतां वरिष्ठः}


\threelineshloka
{निःसंशयं निश्चितसर्वशास्त्राःसर्वे भवन्तो विदितप्रमाणाः}
{विज्ञातुकामस्य ममेह वाक्यमुक्तं यद्वो नैष्ठिकं तच्छ्रुतं मे}
{इहानुवंशं गदतो ममापिवाक्यं निबोधध्वमनन्यभावाः}


\twolineshloka
{यो वै न पापे निरतो न पुण्येनार्थे न धर्मे मनुजो न कामे}
{विमुक्तदोषः समफल्गुसारोविमुच्यते दुःखसुखात्स सिद्धः}


\twolineshloka
{भूतानि जातीमरणान्वितानिजराविकारैश्च समन्वितानि}
{भूयश्च तैस्तैरुपसेवितानिमोक्षं प्रशंसन्ति न तं च विद्मः}


\twolineshloka
{स्नेहेन बद्धस्य न सन्ति तानिचैवं स्वयंभूर्भगवानुवाच}
{बोधाय निर्वाणपरा भवन्तितस्मान्न कुर्यात्प्रियमप्रियं च}


\twolineshloka
{एतच्च मुख्यं न तु कामकारोयथा नियुक्तोऽस्मि तथा करोमि}
{भूतानि सर्वाणि विधिर्नियुङ्क्तेविधिर्बलीयानिति वित्त सर्वे}


\threelineshloka
{न कर्मणाऽप्नोत्यनवाप्यमर्थंयद्भावि तद्वै भवतीति विद्मः}
{त्रिवर्गहीनोऽपि हि विन्दतेऽर्थंतस्माददो लोकहिताय गुह्यम् ॥वैशंपायन उवाच}
{}


\twolineshloka
{तदग्र्यबुद्धेर्वचनं मनोनुगंसमस्तमाज्ञाय तथाहि हेतुमत्}
{तदा प्रणेदुश्च जहर्षिरे च तेकुरुप्रवीराय च चक्रिरेऽञ्जलिम्}


\twolineshloka
{सुचारुवर्णाक्षरशब्दभूषितांमनोनुगां निर्गतवाक्यकण्टकाम्}
{निशम्य तां पार्थिवभाषितां गिरंपार्थस्य सर्वे प्रणता बभूवुः}


\twolineshloka
{तथैव राजा प्रशशंस् वीर्यवान्पुनश्च पप्रच्छ सरिद्वरासुतम्}
{धर्मार्थकामेषु विनिश्चयज्ञंततः परं धर्ममहीनचेतसम्}


\chapter{अध्यायः १६६}
\twolineshloka
{युधिष्ठिर उवाच}
{}


\twolineshloka
{पितामह महाप्राज्ञ कुरूणां प्रीतिवर्धन}
{प्रश्नं कंचित्प्रवक्ष्यामि तन्मे व्याख्यातुमर्हसि}


\twolineshloka
{कीदृशा मानवाः सेव्याः कै प्रीति परमा भवेत्}
{आयत्यां च तदात्वे च के क्षमास्तान्वदस्व मे}


\twolineshloka
{न हि तत्र धनं स्फीतं न च संबन्धिबान्धवाः}
{तिष्ठन्ति यत्र सुहृदस्तिष्ठन्तीति मतिर्मम}


\threelineshloka
{दुर्लभो हि सुहृच्छ्रोता दुर्लभश्च हितः सुहृत्}
{एतद्धर्मभृतां श्रेष्ठ सर्वं व्याख्यातुमर्हसि ॥भीष्म उवाच}
{}


\twolineshloka
{सन्धेयान्पुरुषान्राजन्नसन्धेयांश्च तत्त्वतः}
{वदतो मे निबोध त्वं निखिलेन युधिष्ठिर}


\twolineshloka
{लुब्धः क्रूरस्त्यक्तधर्मा निकृतिः शठ एव च}
{क्षुद्रः पापसमाचारः सर्वशङ्की तथाऽलसः}


\twolineshloka
{दीर्घसूत्रोऽनृजुः क्रुष्टो गुरुदारप्रधर्पकः}
{व्यसने यः परित्यागी दुरात्मा निरपत्रपः}


\twolineshloka
{सर्वतः पापदर्शी च नास्तिको वेदनिन्दकः}
{संप्रकीर्णेन्दियो लोके यः कालनिरतश्चरेत्}


\twolineshloka
{असभ्यो लोकविद्विष्टः समये चानवस्थितः}
{पिशुनोऽथाकृतप्रज्ञो मत्सरी पापनिश्चयः}


\twolineshloka
{दुःशीलोऽथाकृतात्मा च नृशंसः कितवस्तथा}
{मित्रैरपकृतिर्नित्यमटतेऽर्थं धनेप्सया}


\twolineshloka
{ददतश्च यथाशक्ति यो न तुष्यति मन्दधीः}
{अधैर्यमपि यो युङ्क्ते सदा मित्रं नराधमः}


\twolineshloka
{अस्थानक्रोधनो यश्च अकस्माच्च विरज्यते}
{सुहृदश्चैव कल्याणानाशु त्यजति किल्बिपी}


\threelineshloka
{अल्पेऽप्यपकृते मूढे न संस्मरनि यत्कृतम्}
{कार्यसेवी च मित्रेषु मित्रद्वेषी नराधिप}
{}


\twolineshloka
{शत्रुर्मित्रमुखो यश्च जिह्नप्रेक्षी विलोचनः}
{न तुष्यति च कल्याणे यम्त्यजेत्तादृशं नरम्}


\twolineshloka
{पानपो द्वेषणः क्रोधी निर्घृणः परुपस्तथा}
{परोपतापी मित्रध्रुक् तथा प्राणिवधे रतः}


\twolineshloka
{कृतघ्नश्चाधमो लोके न सन्धेयः कथंचन}
{मित्रद्वेषी ह्यसंधेयः सन्धेयानपि मे शृणु}


\twolineshloka
{कुलीना वाक्यसंपन्ना ज्ञानविज्ञानकोविदाः}
{रूपवन्तो गुणोपेतास्तथाऽलुब्धा जितश्रमाः}


\twolineshloka
{सन्मित्राश्च कृतज्ञाश्च सर्वज्ञा लोभवर्जिताः}
{माधुर्यगुणसंपन्नाः सत्यसन्धा जितेन्द्रियाः}


\twolineshloka
{व्यायामशीलाः सततं भृत्यपुत्राः कुलोद्वहाः}
{दोषैः प्रमुक्ताः प्रथितास्ते ग्राह्याः पार्थिवैर्नराः}


\twolineshloka
{यथाशक्ति समाचाराः संप्रतुष्यन्ति हि प्रभो}
{नास्थाने क्रोधवन्तश्च न चाकस्माद्विरागिणः}


\threelineshloka
{विरक्ताश्च न दुष्यन्ति मनसाऽप्यर्थकोविदाः}
{आत्मानं पीडयित्वाऽपि सुहृत्कार्यपरायणाः}
{विरज्यन्ति न मित्रेभ्यो वासो रक्तमिवाविकम्}


\twolineshloka
{दोषांश्च लोभमोहादीनर्थेषु युवतीपु च}
{न दर्शयन्ति सुहृदो विश्वस्ता बन्धुवत्सलाः}


\threelineshloka
{लोष्टकाञ्चनतुल्यार्थाः सुहृत्सु दृढबुद्धयः}
{ये चरन्त्यनभीमाना निसृष्टार्थविभूषणाः}
{संगृह्णन्तः परिजनं स्वाम्यर्थपरमाः सदा}


\twolineshloka
{ईदृशैः पुरुषश्रेष्ठैर्यः सन्धिं कुरुते नृपः}
{तस्य विस्तीर्यते राज्यं ज्योत्स्ना ग्रहपतेरिव}


\threelineshloka
{सत्ववन्तो जितक्रोधा बलवन्तो रणे सदा}
{जन्मशीलगुणोपेताः सन्धेयाः पुरुषोत्तमाः}
{}


\fourlineindentedshloka
{ये च दोपसमायुक्ता नराः प्रोक्ता मयाऽन}
{तेषामप्यधमा राजन्कृतघ्ना मित्रघातकाः}
{त्यक्तव्यास्तु दुराचाराः सर्वेषामिति निश्चयः}
{}


\chapter{अध्यायः १६७}
\twolineshloka
{युधिष्ठिर उवाच}
{}


\threelineshloka
{विस्तरेणार्थसंबन्धं श्रोतुमिच्छामि तत्त्वतः}
{मित्रद्रोही कृतघ्नश्च यः प्रोक्तस्तं च मे वद ॥भीष्म उवाच}
{}


\twolineshloka
{हन्त ते वर्तयिष्येऽहमितिहासं पुरातनम्}
{उदीच्यां दिशि यद्वृत्तं म्लेच्छेषु मनुजाधिप}


\twolineshloka
{ब्राह्मणो मध्यदेशीयः कृष्णाङ्गो ब्रह्मवर्जितः}
{ग्रामं दस्युगणाकीर्णं प्राविशद्धनतृष्णया}


\twolineshloka
{तत्र दस्युर्धनयुतः सर्ववर्णविशेषवित्}
{ब्रह्मण्यः सत्यसन्धश्च दाने च निरतोऽभवत्}


\twolineshloka
{तस्य क्षयमुपागम्य ततो भिक्षामयाचत}
{प्रतिश्रयं च वासार्थं भिक्षां चैवाथ वार्षिकीम्}


\twolineshloka
{प्रादात्तस्मै स विप्राय वस्त्रं च सदशं नवम्}
{नारीं चापि वयोपेतां भर्त्रा विरहितां तथा}


\twolineshloka
{एतत्संप्राप्य हृष्टात्मा गौतमोऽथ द्विजस्तथा}
{तस्मिन्गृहवरे राजंस्तया रेमे स गौतमः}


\twolineshloka
{कुटुस्बार्थं च दस्योश्च साहाय्यं चाप्यथाकरोत्}
{सोऽवसद्वर्षमेकं वै समृद्धे शबरालये}


\threelineshloka
{बाणवेधे परं यत्नमकरोच्चैव गौतमः}
{चक्राङ्गान्स च नित्यं वै सर्वतो वनगोचरान्}
{जघान गौतमो राजन्यथा दस्युगणास्तथा}


\twolineshloka
{हिंसापटुर्घृणाहीनः सदा प्राणिवधे रतः}
{गौतमः सन्निकर्षेण दस्युभिः समतामियात्}


\twolineshloka
{तथा तु वसतस्तस्य दस्युग्रामे सुखं तदा}
{अगमन्बहवो मासा निघ्नतः पक्षिणो बहून्}


\twolineshloka
{ततः कदाचिदपरो द्विजस्तं देशमागतः}
{जटाचीराजिनधरः स्वाध्यायनिरतः शुचिः}


\threelineshloka
{विनीतो वेदशास्त्रेषु वेदान्तानां च पारगः}
{अदृश्यत ततस्तत्र सखा तस्यैव तु द्विजः}
{तं दस्युग्राममगमद्यत्रासौ गौतमोऽभवत्}


\twolineshloka
{स तु विप्रगृहान्वेपी शूद्रान्नपरिवर्जकः}
{ग्रामे दस्युसमाकीर्णे व्यचरत्सर्वतो द्विजः}


\twolineshloka
{ततः स गौतमगृहं प्रविवेश द्विजोत्तमः}
{गौतमश्चापि संप्राप्तस्तावन्योन्येन संगतौ}


\twolineshloka
{चक्राङ्गभारस्कन्धं तं धनुष्पाणिं धृतायुधम्}
{रुधिरेणावसिक्ताङ्गं गृहद्वारमुपागतम्}


\twolineshloka
{तं दृष्ट्वा पुरुषादाभमपध्वस्तं क्षमागतम्}
{अभिज्ञाय द्विजो व्रीडन्निदं वाक्यमथाब्रवीत्}


\twolineshloka
{किमिदं कुरुपे मोहाद्विप्रस्त्वं हि कुलोद्भवः}
{मध्यदेशपरिज्ञातो दस्युभावं गतः कथम्}


\twolineshloka
{पूर्वान्स्मर द्विज ज्ञातीन्प्रख्यातान्वेदपारगान्}
{येषां वंशेऽभिजातस्त्वमीदृशः कुलपांसनः}


\twolineshloka
{अवबुध्यात्मनाऽऽत्मानं सत्वं शीलं श्रुतं दमम्}
{अनुक्रोशं च संस्मृत्य त्यज वासमिमं द्विज}


\twolineshloka
{स एवमुक्तः सुहृदा तेन तत्र हितैपिणा}
{प्रत्युवाच ततो राजन्विनिश्वस्य तदाऽऽर्तबत्}


\twolineshloka
{निर्धनोऽस्मि द्विजश्रेष्ठ नापि वेदविदप्यहम्}
{वित्तार्थमिह संप्राप्तं विद्धि मां द्विजसत्तम}


\twolineshloka
{त्वद्दर्शनात्तु विप्रेन्द्र कृतार्थोऽस्म्यद्य वै द्विज}
{अध्वानं सह यास्यावः श्वो वसस्वाद्य शर्वरीम्}


\twolineshloka
{स तत्र न्यवसद्विप्रो घृणी किंचिदसंस्पृशन्}
{क्षुधितश्छन्द्यमानोऽपि भोजनं नाभ्यनन्दत}


\chapter{अध्यायः १६८}
\twolineshloka
{भीष्म उवाच}
{}


\twolineshloka
{तस्यां निशायां व्युष्टायां गते तस्मिन्द्विजोत्तमे}
{निष्क्रम्य गौतमोऽगच्छद्धनार्थी विचचार ह}


\twolineshloka
{सामुद्रिकान्सवणिजस्ततोऽपश्यत्स्थितान्पथि}
{स तेन सह सार्थेन प्रययौ सागरं प्रति}


\twolineshloka
{स तु सार्थो महाराज कस्मिंश्चिद्गिरिगह्वरे}
{मत्तेन द्विरदेनाथ निहतः प्रायशोऽभवत्}


\twolineshloka
{स कथंचिद्भयात्तस्माद्विमुक्तो प्रायशोऽभवत्}
{कांदिग्भूतो जीवितार्थी प्रदुद्रावोत्तरां दिशम्}


\twolineshloka
{ततस्तु स परिभ्रष्टः सार्थाद्देशात्तथाऽर्थतः}
{एकाकी व्यभ्रमत्तत्र वने कापुरुषो यथा}


\twolineshloka
{स पन्थानमथासाद्य समुद्राभिसरं तदा}
{आससाद वनं रम्यं महत्पुष्पितपादपम्}


\twolineshloka
{सर्वर्तुकैराम्रवणैः पुष्पितैरुपशोभितम्}
{नन्दनोद्देशसदृशं यक्षकिन्नरसेवितम्}


\threelineshloka
{सालतालघवाश्वत्थप्लक्षागुरुवनैस्तथा}
{चन्दनस्य च मुख्यस्य पादपैरुपशोभितम्}
{गिरिप्रस्थेषु रम्येषु सुखेषु सुखगन्धिषु}


\twolineshloka
{समन्ततो द्विजश्रेष्ठा वल्गु कूजन्ति तत्र वै}
{मनुष्यवदनाश्चान्ये भारुण्डा इति विश्रुताः}


\twolineshloka
{स तान्यतिमनोज्ञानि विहगानां रुतानि वै}
{शृण्वन्सुरमणीयानि विप्रोऽगच्छत गौतमः}


\threelineshloka
{ततोऽपश्यत्सुरम्येषु सुवर्णसिकताचिते}
{देशभागे समे चित्रे स्वर्गोद्देशसमप्रभे}
{श्रिया जुष्टं ददर्शाथ न्यग्रोधं च सुमण्डलम्}


\twolineshloka
{शाखाभिरनुरूपाभिः संवृतं छत्रसन्निभम्}
{तस्य मूलं च संसिक्तं वरचन्दनवारिणा}


\twolineshloka
{दिव्यपुष्पान्वितं श्रीमत्पितामहसदोपमम्}
{तं दृष्ट्वा गौतमः प्रीतो मनःकान्तमनुत्तमम्}


\twolineshloka
{मेध्यं सुरगृहप्रख्यं पुष्पितैः पादपैर्वृतम्}
{तमासाद्य मुदा युक्तस्तस्याधस्तादुपाविशत्}


\threelineshloka
{तत्रासीनस्य कौन्तेय गौतमस्य नराधिप}
{पुष्पाणि समुपस्पृश्य प्रववावनिलः शुभः}
{ह्लादयंस्तस्य गात्राणि गौतमस्य तदा नृप}


\twolineshloka
{स तु विप्रः परिश्रान्तिः स्पृष्टः पुण्येन वायुना}
{सुखमासाद्य सुष्वाप भास्करश्चास्तमभ्ययात्}


\twolineshloka
{ततोऽस्तं भास्करे याते सन्ध्याकाल उपस्थिते}
{आजगाम स्वभवनं ब्रह्मलोकात्खगोत्तमः}


\twolineshloka
{नाडीजङ्घ इति ख्यातो दयितो ब्रह्मणः सखा}
{बकराजो महाप्राज्ञः काश्यपस्यात्मसंभवः}


\twolineshloka
{राजधर्मेति विख्यातो बभूवाप्रतिभो भुवि}
{देवकन्यासुतः श्रीमान्विद्वान्देवसमप्रभः}


\twolineshloka
{मृष्टहाटकसंछन्नो भूषणैरर्कसन्निभैः}
{भूषितः सर्वगात्रेषु देवगर्भः श्रिया ज्वलन्}


\threelineshloka
{तमागतं खगं दृष्ट्वा गौतमो विस्मितोऽभवत्}
{क्षुत्पिपासापरीतात्मा हिंसार्थं चैनमैक्षत ॥राजधर्मोवाच}
{}


\twolineshloka
{स्वागतं भवतो विप्र दिष्ट्या प्राप्तोऽसि मे गृहान्}
{अस्तं च सविता यातः संध्येयं समुपस्थिता}


\twolineshloka
{मम त्वं निलयं प्राप्तः प्रियातिथिरनिन्दितः}
{पूजितो यास्यसि प्रातर्विधिदृष्टेन कर्मणा}


\chapter{अध्यायः १६९}
\twolineshloka
{भीष्म उवाच}
{}


\threelineshloka
{गिरं तां मधुरां श्रुत्वा गौतमो विस्मितस्तदा}
{कौतूहलान्वितो राजन्राजधर्माणमैक्षत ॥राजधर्मोवाच}
{}


\threelineshloka
{भोः काश्यपस्य पुत्रोऽहं माता दाक्षायणी मम}
{सती त्वं च गुणोपेतः स्वागतं ते द्विजोत्तम ॥भीष्म उवाच}
{}


\twolineshloka
{तस्मै दत्त्वा स सत्कारं विधिदृष्टेन कर्मणा}
{शालपुष्पमयीं दिव्यां बृसीं समुपकल्पयत्}


\twolineshloka
{भगीरथरथाक्रान्तदेशान्गङ्गानिषेवितान्}
{ये चरन्ति महामीनास्तांश्च तस्यान्वकल्पयत्}


\twolineshloka
{वह्निं चापि सुसंदीप्तं मीनांश्चापि सुपीवरान्}
{स गौतमायातिथये न्यवेदयत काश्यपिः}


\twolineshloka
{भुक्तवन्तं च तं विप्रं प्रीतात्मानं महामनाः}
{श्रमापनयनार्थं स पक्षाभ्यामभ्यवीजयत्}


\twolineshloka
{तनो विश्रान्तमासीनं गोत्रवृत्तमपृच्छत}
{सोऽब्रवीद्गौतमोऽस्मीति ब्राह्मणोस्मीत्युदाहरत्}


\twolineshloka
{तस्मै पर्णमयं दिव्यं दिव्यपुष्पाधिवासितम्}
{गन्धाढ्यं शयनं प्रादात्स शिश्ये तत्र वै सुखम्}


\twolineshloka
{अपोपविष्टं शयने गौतमं वाक्यवित्तमः}
{पप्रच्छ काश्यपिर्वाग्मी किमागमनमित्युत}


\twolineshloka
{ततोऽब्रवीद्गौतमस्तं दरिद्रोऽहं महामते}
{समुद्रगमनाकाङ्क्षी द्रव्यार्थमिति भारत}


\twolineshloka
{तं काश्यपोऽब्रवीत्प्रीत्या नोत्कण्ठां कर्तुमर्हसि}
{कृतकार्यो द्विजश्रेष्ठ सद्रव्यो यास्यसे गृहान्}


\twolineshloka
{चतुर्विधा ह्यर्थगतिर्बृहस्पतिमतं यथा}
{मित्रं विद्या हिरण्यं च बुद्धिश्चेति बुधेप्सिता}


\twolineshloka
{प्रादुर्भूतोऽस्मि ते मित्रं सुहृत्त्वं च महत्तरम्}
{सोऽहं तथा यतिष्यामि भविष्यसि यथार्थवान्}


\twolineshloka
{ततः प्रभातसमये सुखं पृष्ट्वाऽब्रवीदिदम्}
{गच्छ सौम्य पथाऽनेन कृतकृत्यो भविष्यसि}


\twolineshloka
{इतस्त्रियोजनं गत्वा राक्षसाधिपतिर्महान्}
{विरूपाक्ष इति ख्यातः सखा मम महाबलः}


\twolineshloka
{तं गच्छ द्विजमुख्य त्वं स मद्वाक्यप्रचोदितः}
{कामानभीप्सितांस्तुभ्यं दाता नास्त्यत्र संशयः}


\twolineshloka
{इत्युक्तः प्रययौ राजन्गौतमो विगतक्लमः}
{फलान्यमृतकल्पानि भक्षयानो यथेष्टतः}


\twolineshloka
{चन्दनागुरुमुख्यानि पत्रत्वचवनानि च}
{तस्मिन्पथि महाराज सेवमानो द्रुतं ययौ}


\twolineshloka
{ततो मनुव्रजं नाम नगरं शैलतोरणम्}
{शैलप्राकारवप्रं च शैलयन्त्रार्गलं तथा}


\twolineshloka
{विदितश्चाभवत्तस्य राक्षसेन्द्रस्य धीमतः}
{प्रहितः सुहृदा राजन्प्रीयमाणः प्रियातिथिः}


\twolineshloka
{ततः स राक्षसेन्द्रः स्वान्प्रेष्यानाह युधिष्ठिर}
{गौतमो नगरद्वाराच्छीघ्रमानीयतामिति}


\twolineshloka
{तस्मात्पुरवरात्तूर्णं पुरुषाः श्वेतवेष्टनाः}
{गौतमेत्यभिभाषन्तः पुरद्वारमुपागमन्}


\twolineshloka
{ते तमूचुर्महाराज राजप्रेष्यास्तदा द्विजम्}
{त्वरस्व तूर्णमागच्छ राजा त्वां द्रष्टुमिच्छति}


\twolineshloka
{राक्षसाधिपतिर्वीरो विरूपाक्ष इति श्रुतः}
{स त्वां त्वरति वै द्रष्टुं तत्क्षिप्रं संविधीयताम्}


\twolineshloka
{ततः स प्राद्रवद्विप्रो विस्मयाद्विगतक्लमः}
{गौतमः परमर्द्धि तां पश्यन्परमविस्मितः}


\twolineshloka
{तैरेव सहितो राज्ञो वेश्म तूर्णमुपाद्रवत्}
{दर्शनं राक्षसेन्द्रस्य काङ्क्षमाणो द्विजस्तदा}


\chapter{अध्यायः १७०}
\twolineshloka
{भीष्म उवाच}
{}


\twolineshloka
{ततः स विदितो राज्ञः प्रविश्य गृहमुत्तमम्}
{पूजितो राक्षसेन्द्रेण निपसादासनोत्तमे}


\twolineshloka
{पृष्टश्च गोत्रचरणं स्वाध्यायं ब्रह्मचारिकम्}
{पृष्टो राज्ञा स नाज्ञासीद्गोत्रमात्रमथाब्रवीत्}


\threelineshloka
{ब्रह्मवर्चमहीनस्य स्वाध्यायाद्विरतस्य च}
{गोत्रमात्रविदो राजा निवामं समपृच्छत ॥राक्षम उवाच}
{}


\threelineshloka
{क्व ते निवासः कल्याण किंगोत्रा ब्राह्मणी च ते}
{तत्त्वं ब्रूहि न भीः कार्या विश्रमस्व यथासुखम् ॥गौतम उवाच}
{}


\threelineshloka
{मध्यदेशप्रसूतोऽहं वासो मे शवरालये}
{शूद्री पुनर्भूर्भार्या मे सत्यमेतद्ब्रवीमि ते ॥भीष्म उवाच}
{}


\twolineshloka
{ततो राजा विममृशे कथं कार्यमिदं भवेत्}
{कथं वा सुकृतं मे स्यादिति वुद्ध्याऽन्वचिन्तयन्}


\twolineshloka
{अयं वै जन्मना विप्रः सुहृत्तम्य महात्मनः}
{संप्रेपितश्च तेनायं काश्यपेन महात्मना}


\twolineshloka
{तस्य प्रियं करिष्यामि स हि मामाश्रितः सदा}
{भ्राता मे बान्धवश्चासौ सखा चैव प्रियो मम}


\twolineshloka
{कार्तिक्यामद्य भोक्तारः सहस्रं मे द्विजोत्तमाः}
{तत्रायमपि भोक्ता तु देयमस्मै च मे धनम्}


\twolineshloka
{स चाद्य दिवसः पुण्यो ह्यतिथिश्चायमागतः}
{संकल्पितं चैव धनं किं विचार्यमतः परम्}


\twolineshloka
{ततः सहस्रं विप्राणां विदुषां समलंकृतम्}
{स्नातानामनुलिप्तानामहतक्षौमवाससाम्}


\twolineshloka
{तानागतान्द्विजश्रेष्ठान्विरूपाक्षो विशांपते}
{यथार्हं प्रतिजग्राह विधिदृष्टेन कर्मणा}


\twolineshloka
{बृस्यस्तेषां तु संन्यस्ता राक्षसेन्द्रस्य शासनात्}
{भूमौ वरकुशाः स्तीर्णाः प्रेष्यैर्भरतसत्तम}


\twolineshloka
{तासु ते पूजिता राज्ञा निपण्णा द्विजसत्तमाः}
{निलदर्भोदकेनाथ अर्चिता विधिवद्द्विजाः}


\threelineshloka
{विश्वेदेवाः सपितरः साग्नयश्चोपकल्पिताः}
{विलिप्ताः पुष्पवन्तश्च सुप्रचाराः सुपूजिताः}
{व्यराजन्त महाराज नक्षत्रपतयो यथा}


\twolineshloka
{ततो जाम्बूनदीः पात्रीर्वज्राङ्का विमलाः शुभाः}
{वरान्नपूर्णा विप्रेभ्यः प्रादान्मधुघृतप्लुताः}


\twolineshloka
{तस्य नित्यं मदापाढ्यां माध्यां च बहवो द्विजाः}
{ईप्सितं भोजनवरं लभन्ते सत्कृतं तथा}


\twolineshloka
{विशेषतस्तु कार्तिक्यां द्विजेभ्यः संप्रयच्छति}
{शरद्व्यपाये रत्नानि पौर्णमास्यामिति श्रुतिः}


\twolineshloka
{सुवर्णं रजतं चैव मणीनथ च मौक्तिकान्}
{वज्रान्महाधनांश्चैव वैदूर्याजिनराङ्कवान्}


\twolineshloka
{रत्नवन्ति च पात्राणि दक्षिणार्थं स भारत}
{दत्त्वा प्राह द्विजश्रेष्ठान्विरूपाक्षो महायशाः}


\threelineshloka
{गृह्णीत रत्नान्येतानि यथोत्साहं यथेष्टतः}
{येषुयेषु च भाण्डेषु भुक्तवन्तो द्विजोत्तमाः}
{तान्येवादाय गच्छध्वं स्ववेश्मानीति भारत}


\twolineshloka
{इत्युक्तवचने तस्मिन्राक्षसेन्द्रे महात्मनि}
{यथेष्टं तानि रत्नानि जगृहुर्ब्राह्मणवर्षभाः}


\twolineshloka
{ततो महार्हैस्तैस्तेन रत्नैरभ्यर्चिताः शुभैः}
{ब्राह्मणा मृष्टवसनाः सुप्रीताः समुदोऽभवन्}


\twolineshloka
{ततस्तान्राक्षसेन्द्रस्तु द्विजानाह पुनर्वचः}
{नाना देशागतान्राजा ब्राह्मणाननुमन्य वै}


\twolineshloka
{अद्यैकं दिवसं विप्रा न वोऽस्तीह भयं क्वचित्}
{राक्षसेभ्यः प्रमोदध्वमिष्टतो यात माचिरम्}


\twolineshloka
{ततः प्रदुद्रुवुः सर्वे विप्रसङ्घाः समन्ततः}
{गौतमोऽपि सुवर्णस्य भारमादाय सत्वरः}


\twolineshloka
{कृच्छ्रात्समुद्वहन्भारं न्यग्रोधं समुपागमत्}
{न्यषीदच्च परिश्रान्तः क्लान्तश्च क्षुधितश्च सः}


\twolineshloka
{ततस्तमभ्यगाद्राजन्राजधर्मा खगोत्तमः}
{स्वागतेनाभ्यनन्दच्च गौतमं मित्रवत्सलः}


\twolineshloka
{तस्य पक्षाग्रविक्षेपैः क्लमं व्यपनयद्बकः}
{पूजां चाप्यकरोद्धीमान्भोजनं च यथाविधि}


% Check verse!
ततस्तौ संविदं कृत्वा खगेन्द्रद्विजसत्तमौ
\twolineshloka
{गौतमश्चिन्तयामास रात्रौ तस्य समीपतः}
{हाटकस्याभिरूपस्य भारोऽयं सुमहान्मया}


\threelineshloka
{गृहीतो लोभमोहाभ्यां दूरं च गमनं मम}
{न चास्ति पथि भोक्तव्यं प्राणसंधारणं मम}
{किं कृत्वा सुकृतं हि स्यादिति चिन्तापरोऽभवत्}


\twolineshloka
{ततः स पथि भोक्तव्यं प्रेक्षमाणो न किंचन}
{कृतघ्नः पुरुषव्याघ्र मनसेदमचिन्तयत्}


\twolineshloka
{अयं बकपतिः पार्श्वे मांसराशिचितो महान्}
{इमं हत्वा गृहीत्वाऽस्य मांसं यास्य इति प्रभो}


\chapter{अध्यायः १७१}
\twolineshloka
{भीष्म उवाच}
{}


\twolineshloka
{अथ तत्र महार्चिष्माननलो वातसारथिः}
{तस्याविदूरे रक्षार्थं खगेन्द्रेण कृतोऽभवत्}


\twolineshloka
{स चापि पार्श्वे सुष्वाप विश्वस्तो बकराट् तदा}
{कृतघ्नस्तु स दुष्टात्मा तं जिघांसुरजागरीत्}


\twolineshloka
{ततोऽलातेन दीप्तेन स सुप्तं निजघान तम्}
{निहत्य च मुदा युक्तः सोऽनुबन्धं न दृष्टवान्}


\twolineshloka
{स तं विपक्षरोमाणं कृत्वाऽग्नावपचत्तदा}
{तं गृहीत्वा सुवर्णं च ययौ द्रुततरं द्विजः}


\twolineshloka
{`ततो दाक्षायणीपुत्रं नागतं तं तु भारत}
{विरूपाक्षश्चिन्तयन्वै हृदयेन विदूयता ॥'}


\threelineshloka
{ततोऽन्यस्मिन्गते चाह्नि विरूपाक्षोऽब्रवीत्सुतम्}
{न प्रेक्षे राजधर्माणमद्य पुत्र खगोत्तमम्}
{}


\twolineshloka
{स पूर्वसन्ध्यां ब्रह्माणं वन्दितुं याति सर्वदा}
{मां चादृष्ट्वा कदाचित्स न गच्छति गृहं खगः}


\twolineshloka
{द्विरात्रमुभयोः सन्ध्योर्नाभ्यगच्छन्ममालयम्}
{तस्मान्न शुद्ध्यते भावो मम स ज्ञायतां सुहृत्}


\twolineshloka
{स्वाध्यायेन वियुक्तो हि ब्रह्मवर्चसवर्जितः}
{स गतस्तत्र मे शङ्का हन्यात्तं स द्विजाधमः}


\threelineshloka
{दुराचारः स दुर्बुद्धिरिङ्गितैर्लक्षितो मया}
{निष्कृपो दारुणाकारो दुष्टो दस्युरिवाधमः}
{गौतमः स गतस्तत्र तेनोद्विग्नं मनो मम}


\twolineshloka
{पुत्र शीघ्रमितो गत्वा राजधर्मनिवेशनम्}
{ज्ञायतां स विशुद्धात्मा यदि जीवति वा चिरम्}


\twolineshloka
{स एवमुक्तस्त्वरितो रक्षोभिः सहितो ययौ}
{* न्यग्रोधे राजधर्माणमपश्यन्निहतं ततः}


\twolineshloka
{रुदित्वा बहुशस्तस्मै विलप्य च स राक्षसः}
{गतो रोषसमाविष्टो गौतमग्रहणाय वै}


\twolineshloka
{गृहीतो गौतमः पापो रक्षोभिः क्रोधमूर्च्छितैः}
{राजधर्मशरीरस्य कङ्कालश्चाप्यथो धृतः}


\twolineshloka
{मनुव्रजं तु नगरं यातुधानास्ततो गताः}
{क्रोधरक्तेक्षणा घोरा गौतमस्य वधे धृताः}


\twolineshloka
{पार्थिवस्वाग्रतो न्यस्तः कङ्कालो राजधर्मणः}
{तं दृष्ट्वा विमना राजा सामात्यः सगणोऽभवत्}


\twolineshloka
{आर्तरावो महानासीद्गृहे तस्य महात्मनः}
{समुत्थितः स्रीसङ्घस्य निहते काश्यपात्मजे}


\twolineshloka
{राजा चैवाब्रवीत्पुत्रं पापोऽयं वध्यतामिति ॥राक्षमा ऊचुः}
{}


\fourlineindentedshloka
{अस्य मांसं वयं सर्वे खादिष्यामः समागताः}
{पापकृत्पापकर्मा च पापात्मा पापमास्थितः}
{हन्तव्य एव पापात्मा कृतघ्नो नात्र संशयः ॥विरूपाक्ष उवाच}
{}


\threelineshloka
{कृतघ्नं पापकर्माणां न भक्षयितुमुत्सहे}
{दासेभ्यो दीयतामेप मित्रध्रुक्पुरुपाधमः ॥भीष्म उवाच}
{}


\twolineshloka
{दासाः सर्वे समाहूता यातुधानास्तथा परे}
{नेच्छन्ति स्म कृतघ्नं तं खादितुं पुरुषोत्तम}


\twolineshloka
{शिरोभिश्चागता भूमिं महाराज ततो बलात्}
{मानार्थं जातु निर्बन्धं किल्विषं दातुमर्हसि}


\twolineshloka
{यातुधाना नृपेणोक्ताः पापकर्मा विशस्यताम्}
{दह्यतां त्यज्यतां वाऽयं दर्शनादपनीयताम्}


\twolineshloka
{ततस्ते रुपिता दासाः शूलपट्टसपाणयः}
{खण्डशो विकृतं हत्वा क्रव्याद्भ्यो ह्यददुस्तदा}


\twolineshloka
{क्रव्यादास्त्वपि राजेन्द्र नेच्छन्ति पिशिताशनाः}
{मृतानपि हि क्रव्यादाः कृतघ्नान्नोपभुञ्जते}


\twolineshloka
{ब्रह्मस्वहरणे चोरे ब्रह्मघ्ने गुरुतल्पगे}
{निष्कृतिर्विहिता सद्भिः कृतघ्ने नास्ति निष्कृतिः}


\twolineshloka
{मित्रद्रुहं कृतघ्नं च नृशंसं च नराधमम्}
{क्रव्यादाः किमयश्चैव नोपभुञ्जन्ति वै सदा}


\chapter{अध्यायः १७२}
\twolineshloka
{भीष्म उवाच}
{}


\twolineshloka
{विद्वान्संस्कारयामास पार्थिवो राजधर्मणः}
{गन्धैर्बहुभिरव्यग्रो दाहयामास पूजितम्}


\twolineshloka
{तस्य देवस्य वचनादिन्द्रस्य बकराडिह}
{तेनैवामृतसिक्ताश्च पुनः संजीवितो बकः}


\threelineshloka
{राजधर्माऽपि तं प्राह सहस्राक्षमरिंदमम्}
{गौतमो ब्राह्मणः क्वाऽसौ मुच्यतां मत्प्रियः सखा ॥भीष्म उवाच}
{}


\twolineshloka
{तस्य वाक्यं समाज्ञाय कौशिकः सुरसत्तमः}
{गौतमं ह्यभ्यनुज्ञाप्य प्रीतोऽथ गमनोत्सुकः}


\twolineshloka
{प्रतीतः स गतः सौम्यो राजधर्मा स्वमालयम्}
{नृशंसो गौतमो मुक्तो मित्रध्रुक्पुरुषाधमः}


\twolineshloka
{सभाण्डोपस्करो यातः स तदा शबरालयम्}
{तत्रासौ शबरी देहे प्रसूतो निरयोपमे}


% Check verse!
एष शापो महांस्तत्र मुक्तः सुरगणैस्तदा
\twolineshloka
{दग्धे राक्षसराजेन खगराजे प्रतापिना}
{चितायाः पार्श्वतो दोग्ध्री सुरभिर्जीवयच्च तम्}


\twolineshloka
{तस्या वक्राच्च्युतः फेनो दुग्धमात्रस्तदाऽनघ}
{समीरणाहृतो यातश्चितां तां राजधर्मणः}


\threelineshloka
{देवराजस्ततः प्राप्तो विरूपाक्षपुरं तदा}
{विरूपाक्षोऽपि तं शक्रमयाचत पुनः पुनः}
{काश्यपश्य सुतो देव भ्राता मे जीवतामिति}


\twolineshloka
{विरूपाक्षमुवाचेदमीश्वरः पाकशासनः}
{ब्रह्मणा व्याहृतो रोषाद्राजधर्मा कदाचन}


\twolineshloka
{यस्मात्त्वं नागतो द्रष्टुं मम नित्यमिमां सभाम्}
{तस्माद्बको भवान्भावी धर्मशीलः परात्मवित्}


\twolineshloka
{आगमिष्यति ते वासं कदाचित्पापकर्मकृत्}
{शबरावासगो विप्रः कृतघ्नो वृषलीपतिः}


\threelineshloka
{यदा निहन्ता मोक्षस्ते तदा भावीत्युवाच तम्}
{तस्मादेष गतो लोकं ब्रह्मणः परमेष्ठिनः ॥भीष्म उवाच}
{}


% Check verse!
स चापि निरयं प्राप्तो दुष्कृतिः कुलपांसनः
\twolineshloka
{एतच्छ्रुत्वा सभामध्ये तद्वाक्यं नारदेरितम्}
{मयाऽपि तव राजेन्द्र यथावदनुवर्णितम्}


\twolineshloka
{ब्रह्मघ्ने च सुरापे च चोरे भ्रष्टव्रते तथा}
{निष्कृतिर्विहिता राजन्कृतघ्ने नास्ति निष्कृतिः}


\twolineshloka
{कुतः कृतघ्नस्य यशः कुतः स्थानं कुतः सुखम्}
{अश्रद्धेयः कृतघ्नो हि कृतघ्ने नास्ति निष्कृतिः}


\twolineshloka
{मित्रद्रोहो न कर्तव्यः पुरुषेण विशेषतः}
{मित्रध्रुङ्गिरयं घोरं नरकं प्रतिपद्यते}


\twolineshloka
{कृतज्ञमनसा भाव्यं मित्रभावेन चानघ}
{मित्रात्प्रभवते सर्वं मित्रं धन्यतरं स्मृतम्}


\twolineshloka
{अर्थाद्वा मित्रलाभाद्वा मित्रलाभो विशिष्यते}
{सुलभा मित्रतोऽर्थास्तु मित्रेण यतितुं क्षमम्}


\twolineshloka
{मित्रं चाभिमतं स्निग्धं फलं चापि सतां फलम्}
{सत्कारैः स्वजनोपेतः पूजयेत विचक्षणः}


\twolineshloka
{परित्याज्यो बुधैः पापः कदर्यः कुलपांसनः}
{मित्रद्रोही कुलाङ्गारः पापकर्मा कुलाधमः}


\threelineshloka
{एषा सज्जनसांनिध्ये प्रज्ञा प्रोक्ता मयाऽनघ}
{मित्रदुहि कृतघ्ने च किं भूयः श्रोतुमिच्छसि ॥वैशंपायन उवाच}
{}


\twolineshloka
{एतच्छ्रुत्वा ततो वाक्यं भीष्मेणोक्तं महात्मना}
{युधिष्ठिरः प्रीतमना बभूव जनमेजय}


\chapter{अध्यायः १७३}
\twolineshloka
{युधिष्ठिर उवाच}
{}


\threelineshloka
{धर्माः पितामहेनोक्ता राजधर्माश्रिताः शुभाः}
{धर्ममाश्रमिणां श्रेष्ठं वक्तुमर्हसि सत्तम ॥भीष्म उवाच}
{}


\twolineshloka
{सर्वत्र विहितो धर्मः स्वर्ग्यः सत्यफलोदयः}
{बहुद्वारस्य धर्मस्य नेहास्ति विफला क्रिया}


\twolineshloka
{यस्मिन्यस्मिंस्तु विषये योयो याति विनिश्चयम्}
{स तमेवाभिजानाति नान्यं भरतसत्तम्}


\twolineshloka
{यथायथा च पर्येति लोकतन्त्रमसारवत्}
{तथातथा विरागोऽत्र जायते नात्र संशयः}


\threelineshloka
{एवं व्यवसिते लोके बहुदोषे युधिष्ठिर}
{आत्ममोक्षनिमित्तं वै यतेत मतिमान्नरः ॥युधिष्ठिर उवाच}
{}


\threelineshloka
{नष्टे धने वा दारे वा पुत्रे पितरि वा मृते}
{यया बुद्ध्या नुदेच्छोक तन्मे ब्रूहि पितामह ॥भीष्म उवाच}
{}


\twolineshloka
{नष्टे धने वा दारे वा पुत्रे पितरि वा मृते}
{अहोदुःखमिति ध्यायञ्शोकस्यापचितिं चरेत्}


\twolineshloka
{अत्राप्युदाहरन्तीममिहासं पुरातनम्}
{यथा सेनजितं विप्रः कश्चिदेत्याब्रवीत्सुहृत्}


\twolineshloka
{पुत्रशोकाभिसंतप्तं राजानं शोकविह्वलम्}
{विषण्णमनसं दृष्ट्वा विप्रो वचनमब्रवीत्}


\twolineshloka
{किंनु मुह्यसि मूढस्त्वं शोच्यः किमनु शोचसि}
{यदा त्वामपि शोचंतः शोच्या यास्यन्ति तां गातिम्}


\threelineshloka
{त्वं चैवाहं च ये चान्ये त्वां राजन्पर्युपासते}
{सर्वे तत्र गमिष्यामो यत एवागता वयम् ॥सेनजिदुवाच}
{}


\threelineshloka
{का बुद्धिः किं तपो विप्र कः समाधिस्तपोधन}
{किं ज्ञानं किं श्रुतं वा ते यत्प्राप्य न विषीदसि ॥ब्राह्मण उवाच}
{}


\twolineshloka
{हृष्यन्तमवसीदन्तं सुखदुःखविपर्यये}
{आत्मानमनुशोचामि यो ममैष हृदि स्थितः}


\twolineshloka
{पश्य भूतानि दुःखेन व्यतिषिक्तानि सर्वशः}
{उत्तमाधममध्यानि तेषु तेष्विह कर्मसु}


\threelineshloka
{आत्माऽपि चायं न मम सर्वा वा पृथिवी मम}
{यथा मम तथाऽन्येषामिति मत्वा न मे व्यथा}
{एतां बुद्धिमहं प्राप्य न प्रहृष्ये न च व्यथे}


\twolineshloka
{यथा काष्ठं च काष्ठं च समेयातां महोदधौ}
{समेत्य च व्यपेयातां तद्वद्भूतसमागमः}


\twolineshloka
{एवं पुत्राश्च पौत्राश्च ज्ञातयो बान्धवास्तथा}
{तेषु स्नेहो न कर्तव्यो विप्रयोगो ध्रुवो हि तैः}


\twolineshloka
{अदर्शनादापतितः पुनश्चादर्शनं गतः}
{न त्वाऽसौ वेद न त्वंतं कस्मात्त्वमनुशोचसि}


\twolineshloka
{सुखान्तप्रभवं दुःखं दुःखान्तप्रभवं सुखम्}
{सुखात्संजायते दुःखं दुःखात्संजायते सुखम्}


\twolineshloka
{सुखस्यानन्तरं दुःखं दुःखस्यानन्तरं सुखम्}
{सुखदुःखे मनुष्याणां चक्रवत्परिवर्ततः}


\twolineshloka
{सुखात्त्वं दुःखमापन्नः पुनरापत्स्यसे सुखम्}
{न नित्यं लभते दुःखं न नित्यं लभते सुखम्}


\twolineshloka
{[शरीरमेवायतनं सुखस्यदुःखस्य चाप्यायतनं शरीरम्}
{यद्यच्छरीरेण करोति कर्मतेनैव देही समुपाश्नुते तत्}


\twolineshloka
{जीवितं च शरीरेण तेनैव सह जायते}
{उभे सह विवर्तेते उभे सह विनश्यतः}


\twolineshloka
{स्नेहपाशैर्बहुविधैराविष्टविषया जनाः}
{अकृतार्थाश्च सीदन्ते जलैः सैकतसेतवः}


\twolineshloka
{स्नेहेन तैलवत्सर्वं सर्गचक्रे निपीड्यते}
{तिलपीडैरिवाक्रम्य क्लेशैरज्ञानसंभवैः}


\twolineshloka
{संचिनोत्यशुभं कर्म कलत्रापेक्षया नरः}
{एकः क्लेशानवाप्नोति परत्रेह च मानवः}


\twolineshloka
{पुत्रदारकुटुम्बेषु प्रसक्ताः सर्वमानवाः}
{शोकपङ्कार्णवे मग्ना जीर्णा वनगजा इव}


\threelineshloka
{पुत्रनाशे वित्तनाशे ज्ञातिसंबन्धिनामपि}
{प्राप्यते सुमहद्दुःखं दावाग्निप्रतिमं विभो}
{दैवायत्तमिदं सर्वं सुखदुःखे भवाभवौ}


\twolineshloka
{असुहृत्ससुहृच्चापि सशत्रुर्मित्रवानपि}
{सप्रज्ञः प्रज्ञया हीनो दैवेन लभते सुखम् ॥]}


\twolineshloka
{नालं सुखाय सुहृदो नालं दुःखाय दुर्हृदः}
{न च प्रज्ञाऽलमर्थानां न सुखानामलं धनम्}


\twolineshloka
{न बुद्धिर्धनलाभाय न मौढ्यमसमृद्ध्ये}
{लोकपर्यायवृत्तान्तं प्राज्ञो जानाति नेतरः}


\twolineshloka
{बुद्धिमन्तं च शूरं च मूढं भीरुं जडं कविम्}
{दुर्बलं बलवन्तं च भागिनं भजते सुखम्}


\twolineshloka
{धेनुर्वत्सस्य गोपस्य स्वामिनस्तस्करस्य च}
{पयः पिबति यस्तस्या धेनुस्तस्येति निश्चयः}


\twolineshloka
{ये च मूढतमा लोके ये च बुद्धेः परं गताः}
{ते नराः सुखमेधन्ते क्लिश्यत्यन्तरितो जनः}


\twolineshloka
{अन्तेषु रेमिरे धीरा न ते मध्येषु रेमिरे}
{अन्तप्राप्तिं सुखं प्राहुर्दुःखमन्तरमन्तयोः}


\twolineshloka
{सुखं स्वपिति दुर्मेधाः स्वानि कर्माण्यचिन्तयन्}
{अविज्ञानेन महता कम्बलेनेव संवृतः}


\twolineshloka
{ये च बुद्धिं परां प्राप्ता द्वन्द्वातीता विमत्सराः}
{तान्नैवार्था न चानर्था व्यथयन्ति कदाचन}


\twolineshloka
{अथ ये बुद्धिमप्राप्ता व्यतिक्रान्ताश्च मूढताम्}
{तेऽतिवेलं प्रहृष्यन्ति संतापमुपयान्ति च}


\twolineshloka
{नित्यं प्रमुदिता मूढा दिवि देवगणा इव}
{अवलेपेन महता परितृप्ता विचेतसः}


\twolineshloka
{सुखं दुःखान्तमालस्यं दुःखं दाक्ष्यं सुखोदयम्}
{भूतिश्चैव श्रिया सार्धं दक्षे वसति नालसे}


\twolineshloka
{सुखं वा यदि वा दुःखं प्रियं वा यदि वाऽप्रियम्}
{प्राप्तं प्राप्तमुपासीत हृदयेनापराजितः}


\twolineshloka
{शोकस्थानसहस्राणि भयस्थानशतानि च}
{दिवसेदिवसे मूढमाविशन्ति न पण्डितम्}


\twolineshloka
{बुद्धिमन्तं कृतप्रज्ञं शुश्रूषुमनहंकृतम्}
{शान्तं जितेन्द्रियं चापि शोको न स्पृशते नरम्}


\threelineshloka
{एतां बुद्धिं समास्थाय शुद्धचित्तश्चरेद्बुधः}
{`शुक्लकृष्णगतिज्ञं तं देवासुरविनिर्गमम्}
{'उदयास्तमयज्ञं हि न शोकः स्प्रष्टुमर्हति}


\twolineshloka
{यन्निमत्तो भवेच्छोकस्रासो वा क्रोध एव वा}
{आयासो वा यतो मूलं तदेकाङ्गमपि त्यजेत्}


\twolineshloka
{यद्यत्त्यजति कामनां तत्सुखस्याभिपूर्यते}
{कामानुसारी पुरुषः कामाननुविनश्यति}


\twolineshloka
{यच्च कामसुखं लोके यच्च दिव्यं महत्सुखम्}
{तृष्णाक्षयसुखस्यैते नार्हतः षोडशीं कलाम्}


\twolineshloka
{पूर्वदेहकृतं कर्म शुभं वा यदि वाऽशुभम्}
{प्राज्ञं मूढं तथा शूरं भजते तादृशं नरम्}


\twolineshloka
{एवमव किलैतानि प्रियाण्येवाप्रियाणि च}
{जीवेषु परिवर्तन्ते दुःखानि च सुखानि च}


\twolineshloka
{एतां बुद्धिं समास्थाय नावसीदेद्गुणान्वितः}
{सर्वान्कामाञ्जुगुप्सेन कोपं कुर्वीत पृष्ठतः}


\twolineshloka
{वृत्त एव हृदि प्रौढो मृत्युरेप मनोभवः}
{क्रोधो नाम शरीरस्थो देहिनां प्रोच्यते बुधैः}


\twolineshloka
{यदा संहरते कामान्कूर्मोऽङ्गानीव सर्वशः}
{तदात्मज्योतिरात्मश्रीरात्मन्येव प्रसीदति}


\twolineshloka
{किंचिदेव ममत्वेन यदा भवति कल्पितम्}
{तदेव परितापाय नाशे संपद्यते तदा}


\twolineshloka
{न बिभेति यदा चायं यदा चास्मान्न विभ्यति}
{यदा नेच्छति न द्वेष्टि ब्रह्म संपद्यते तदा}


\twolineshloka
{उभे सत्यानृते त्यक्त्वा शोकानन्दौ भयाभये}
{प्रियाप्रिये परित्यज्य प्रशान्तात्मा भविष्यति}


\twolineshloka
{यदा न कुरुते धीरः सर्वभूतेषु पापकम्}
{कर्मणा मनसा वाचा ब्रह्म संपद्यते तदा}


\twolineshloka
{या दुस्त्यजा दुर्मतिभिर्या न जीर्यति जीर्यतः}
{योसौ प्राणान्तिको रोगस्तां तृष्णां त्यजतः सुखं}


\twolineshloka
{अत्र पिङ्गलया गीता गाथा शृणु नराधिप}
{यथा सा कृच्छ्रकालेऽपि लेभे शर्म सनातनम्}


\threelineshloka
{संकेते पिङ्गला वेश्या कान्तेनासीद्विनाकृता}
{अथ कृच्छ्रगता शान्तां बुद्धिमास्थापयत्तदा ॥पिङ्गलोवाच}
{}


\twolineshloka
{उन्मत्ताऽहमनुन्मत्तं कान्तमन्ववसं चिरम्}
{अन्तिके रमणं सन्तं नैनमध्यगमं पुरा}


\twolineshloka
{एकस्थूणं नवद्वारमपिधास्याम्यगारकम्}
{का ह्यकान्तमिहायान्तं कान्त इत्यभिमंस्यते}


\twolineshloka
{अकामाः कामरूपेण धूर्ताश्च नररूपिणः}
{न पुनर्वञ्चयिष्यन्ति प्रतिबुद्धाऽस्मि जागृमि}


\twolineshloka
{अनर्थोऽपि भवत्यर्थो दैवात्पूर्वकृतेन वा}
{संबुद्धाऽहं निराकारा नाहमद्याजितेन्द्रिया}


\threelineshloka
{सुखं निराशः स्वपिति नैराश्यं परमं सुखम्}
{आशामनाशां कृत्वा हि सुखं स्वपिति पिङ्गला ॥भीष्म उवाच}
{}


\twolineshloka
{एतैश्चान्यैश्च विप्रस्य हेतुमद्भिः प्रभाषितैः}
{पर्यवस्थापितो राजा सेनजिन्मुमुदे सुखम्}


\chapter{अध्यायः १७४}
\twolineshloka
{युधिष्ठिर उवाच}
{}


\threelineshloka
{अतिक्रामति कालेऽस्मिन्सर्वभूतक्षयावहे}
{किं श्रेयः प्रतिपद्येत तन्मे ब्रूहि पितामह ॥भीष्म उवाच}
{}


\twolineshloka
{अत्राप्युदाहरन्तीममितिहासं पुरातनम्}
{पितुः पुत्रेण संवादं तं निबोध युधिष्ठिर}


\twolineshloka
{द्विजातेः कस्यचित्पार्थ स्वाध्यायनिरतस्य वै}
{बभूव पुत्रो मेधावी मेधावीनाम नामतः}


\threelineshloka
{सोऽब्रवीत्पितरं पुत्रः स्वाध्यायकरणे रतम्}
{मोक्षधर्मार्थकुशलो लोकतन्त्रविचक्षणः ॥पुत्र उवाच}
{}


\threelineshloka
{धीरः किंस्वित्तात कुर्यात्प्रजानन्क्षिप्रं ह्यायुर्भ्रश्यते मानवानाम्}
{पितस्तदाचक्ष्व यथार्थयोगंममानुपूर्व्या येन धर्मं चरेयम् ॥पितोवाच}
{}


\threelineshloka
{वेदानधीत्य ब्रह्मचर्येण पुत्रपुत्रानिच्छेत्पावनार्थं पितृणाम्}
{अग्नीनाधाय विधिवच्चेष्टयज्ञोवनं प्रविश्याथ मुनिर्बुभूषेत् ॥पुत्र उवाच}
{}


\threelineshloka
{एवमभ्याहते लोके समन्तात्परिवारिते}
{अमोघासु पतन्तीषु किं धीर इव भाषसे ॥पितोवाच}
{}


\threelineshloka
{कथमभ्याहतो लोकः केन वा परिवारितः}
{अमोघाः काः पतन्तीह किंनु भीषयसीव माम् ॥पुत्र उवाच}
{}


\fourlineindentedshloka
{मृत्युनाभ्याहतो लोको जरया परिवारितः}
{अहोरात्राः पतन्त्येते ननु कस्मान्न बुध्यसे}
{अमोघा रात्रयश्चापि नित्यमायान्ति यान्ति च ॥पितोवाच}
{}


\threelineshloka
{यथाऽहमेतज्जानामि न मृत्युस्तिष्ठतीति ह}
{सोऽहं कथं प्रतीक्षिष्ये जालेनेवावृतश्चरन् ॥पुत्र उवाच}
{}


\twolineshloka
{रात्र्यांरात्र्यां व्यतीतायामायुरल्पतरं यदा}
{तदैव बन्ध्यं दिवसमिति विन्द्याद्विचक्षणः}


\twolineshloka
{गाधोदके मत्स्य इव सुखं विन्देत कस्तदा}
{अनवाप्तेषु कामेषु मृत्युरभ्योति मानवम्}


\twolineshloka
{पुष्पाणीव विचिन्वन्तमन्यत्र गतमानसम्}
{वृकीवोरणमासाद्य मृत्युरादाय गच्छति}


\twolineshloka
{अद्यैव कुरु यच्छ्रेयो मा त्वां कालोऽत्यगादयम्}
{अकृतेष्वेव कार्येषु मृत्युर्वै संप्रकर्षति}


\twolineshloka
{श्वः कार्यमद्य कुर्वीत पूर्वाह्णे चापराह्णिकम्}
{नहि प्रतीक्षते मृत्युः कृतमस्य न वा कृतम्}


\threelineshloka
{को हि जानाति कस्याद्य मृत्युकालो भविष्यति}
{युवैव धर्मशीलः स्यादनित्यं खलु जीवितम्}
{कृते धर्मे भवेत्कीर्तिरिह प्रेत्य च वै सुखम्}


\twolineshloka
{मोहेन हि समाविष्टः पुत्रदारार्थमुद्यतः}
{कृत्वा कार्यमकार्यं वा पुष्टिमेषां प्रयच्छति}


\twolineshloka
{तं पुत्रपशुसंपन्नं व्यासक्तमनसं नरम्}
{सुप्तं व्याघ्रो मृगमिव मृत्युरादाय गच्छति}


\twolineshloka
{संचिन्वानकमेवैनं कामानामवितृप्तकम्}
{व्याघ्रः पशुमिवादाय मृत्युरादाय गच्छति}


\twolineshloka
{इदं कृतमिदं कार्यमिदमन्यत्कृताकृतम्}
{एवमीहासुखासक्तं कृतान्तः कुरुते वशे}


\twolineshloka
{कृतानां फलमप्राप्तं कर्मणां कर्मसंज्ञितम्}
{क्षेत्रापणगृहासक्तं मृत्युरादाय गच्छति}


\twolineshloka
{दुर्बलं बलवन्तं च शूरं भीरुं जडं कविम्}
{अप्राप्तं सर्वकामार्थान्मृत्युरादाय गच्छति}


\twolineshloka
{नृत्युर्जरा च व्याधिश्च दुःखं चानेककारणम्}
{अनुषक्तं यदा देहे किं स्वस्थ इव तिष्ठसि}


\twolineshloka
{जातमेवान्तकोऽन्ताय जरा चान्वेति देहिनम्}
{अनुषक्ता द्वयेनैते भावाः स्थावरजङ्गमाः}


\twolineshloka
{अत्योर्वा मुखमेतद्वै या ग्रामे वसतो रतिः}
{वानामेष वै गोष्ठो यदरण्यमिति श्रुतिः}


\twolineshloka
{तेबन्धनी रज्जुरेषा या ग्रामे वसतो रवि}
{छेत्त्वेता सुकृतो यान्ति नैनां छिन्दन्ति दुष्कृतः}


\twolineshloka
{हिंसयति यो जन्तून्मनोवाक्कायहेतुभिः}
{जीवितार्थापनयनैः प्राणिभिर्न स हिंस्यते}


\twolineshloka
{न मृत्युसेनामायान्तीं जातु कश्चित्प्रबाधते}
{ऋते सत्यमसत्त्याज्यं सत्ये ह्यमृतमाश्रितम्}


\twolineshloka
{तस्मात्सत्यव्रताचारः सत्ययोगपरायणः}
{सत्यागमः सदा दान्तः सत्येनैवान्तकं जयेत्}


\twolineshloka
{अमृतं चैव मृत्युश्च द्वयं देहे प्रतिष्ठितम्}
{मृत्युरापद्यते मोहात्सत्येनापद्यतेऽमृतम्}


\twolineshloka
{सोऽहं ह्यहिंस्रः सत्यार्थी कामक्रोधबहिष्कृतः}
{समदुःखसुखः क्षेमी मृत्युंहास्याम्यमर्त्यवत्}


\twolineshloka
{शान्तियज्ञरतो दान्तो ब्रह्मयज्ञे स्थितो मुनिः}
{वाङ्भनः कर्मयज्ञश्च भविष्याम्युदगायने}


\twolineshloka
{पशुयज्ञैः कथं हिंस्रैर्मादृशो चष्टुमर्हति}
{अन्तवद्भिरिव प्राज्ञः क्षेत्रयज्ञैः पिशाचवत्}


\twolineshloka
{यस्य वाङ्भनसी स्यातां सम्यक्प्रणिहिते सदा}
{तपस्त्यागश्च सत्यं च स वै सर्वमवाप्नुयात्}


\twolineshloka
{नास्ति विद्यासमं चक्षुर्नास्ति सत्यसमं तपः}
{नास्ति रागसमंदुःखं नास्ति त्यागसमं सुखम्}


\twolineshloka
{आत्मन्येवात्मना जात आत्मनिष्ठोऽप्रजोपि वा}
{आत्मन्येव भविष्यामि न मां तारयति प्रजा}


\twolineshloka
{नैतादृशं ब्राह्मणस्यास्ति वित्तंयथैकता समता सत्यता च}
{शीलं स्थितिर्दण्डनिधानमार्जवंततस्ततश्चोपरभः क्रियाभ्यः}


\threelineshloka
{किं ते धनैर्बान्धवैर्वापि किं तेकिं ते दारैर्ब्राह्मण यो मरिष्यसि}
{आत्मानमन्विच्छ गुहां प्रविष्टंपितामहास्ते क्व गताः पिता च ॥भीष्म उवाच}
{}


\twolineshloka
{पुत्रस्यैतद्वचः श्रुत्वा यथाऽकार्षीत्पिता नृप}
{तथा त्वमपि वर्तस्व सत्यधर्मपरायणः}


\chapter{अध्यायः १७५}
\twolineshloka
{युधिष्ठिर उवाच}
{}


\threelineshloka
{धनिनश्चाधना ये च वर्तयन्ति स्वतन्त्रिणः}
{सुखदुःखागमस्तेषां कः कथं वा पितामह ॥भीष्म उवाच}
{}


\twolineshloka
{अत्राप्युदाहरन्तीममितिहासं पुरातनम्}
{शम्याकेन विमुक्तेन गीतं शान्तिगतेन च}


\twolineshloka
{अब्रवीन्मां पुरा कश्चिद्ब्राह्मणस्त्यागमाश्रितः}
{क्लिश्यमानः कुदारेण कुचेलेन बुभुक्षया}


\twolineshloka
{उत्पन्नमिह लोके वै जन्मप्रभृति मानवम्}
{विविधान्युपवर्तन्ते दुःखानि च सुखानि च}


\twolineshloka
{तयोरेकतरो मार्गो यदेनमुपसन्नयेत्}
{न सुखं प्राप्य संहृष्येन्नासुखं प्राप्य संज्वरेत्}


\twolineshloka
{न वै चरसि यच्छ्रेय आत्मनो वा न रंस्यसे}
{अकामात्माऽपि हि सदा धुरमुद्यम्य चैव ह}


\twolineshloka
{अकिञ्चनः परिपतन्सुखमास्वादयिष्यसि}
{अकिञ्चनः सुखं शेते समुत्तिष्ठति चैव ह}


\twolineshloka
{आकिञ्चन्यं सुखं लोके पथ्यं शिवमनामयम्}
{अनमित्रपथो ह्येष दुर्लभः सुलभः सताम्}


\twolineshloka
{अकिञ्चनस्य शुद्धस्य उपपन्नस्य सर्वतः}
{अवेक्षमाणस्त्रील्लोँकान्न तुल्यमिह लक्षये}


\twolineshloka
{आकिञ्चन्यं च राज्यं च तुलया समतोलयम्}
{अत्यरिच्यत दारिद्र्यं राज्यादपि गुणाधिकम्}


\twolineshloka
{आकिञ्चन्ये च राज्ये च विशेषः सुमहानयम्}
{नित्योद्विग्नो हि धनवान्मृत्योरास्यगतो यथा}


\twolineshloka
{नैवास्याग्निर्न चादित्यो न मृत्युर्न च दस्यवः}
{प्रभवन्ति धनं हर्तुमितरे स्युः कुतः पुनः}


\twolineshloka
{तं वै सदा कामचरमनुपस्तीर्णशायिनम्}
{बाहूपधानं शाम्यन्तं प्रशंसन्ति दिवौकसः}


\twolineshloka
{धनवान्क्रोधलोभाभ्यामाविष्टो नष्टचेतनः}
{तिर्यग्दृष्टिः शुष्कमुखः पापको भ्रुकुटीमुखः}


\twolineshloka
{निर्दशन्नधरोष्ठं च क्रुद्धो दारुणभाषिता}
{कस्तमिच्छेत्परिद्रष्टुं दातुमिच्छति चेन्महीम्}


\twolineshloka
{श्रिया ह्यभीक्ष्णं संवासो मोहयत्यविचक्षणम्}
{सा तस्य चित्तं हरति शारदाभ्रमिवानिलः}


\threelineshloka
{अथैनं रूपमानश्च धनपानश्च विन्दति}
{अभिजातोऽस्मि सिद्धोऽस्मि नास्मि केवलमानुषः}
{इत्येभिः कारणैस्तस्य त्रिभिश्चित्तं प्रमाद्यति}


\twolineshloka
{संप्रसक्तमना भोगान्विसृज्य पितृसंचितान्}
{परिक्षीणः परस्वानामादानं साधु मन्यते}


\twolineshloka
{तमतिक्रान्तमर्यादमाददानं ततस्ततः}
{प्रतिषेधन्ति राजानो लुब्धा मृगमिवेषुभिः}


\twolineshloka
{एवमेतानि दुःखानि तानि तानीह मानवम्}
{विविधान्युपवर्तन्ते गात्रसंस्पर्शजान्यपि}


\twolineshloka
{तेषां परमदुःखानां बुद्ध्या भैषज्यमाचरेत्}
{लोकधर्मं समाज्ञाय ध्रुवाणामध्रुवैः सह}


\twolineshloka
{नात्यक्त्वा सुखमाप्नोति नात्यक्त्वा विन्दते परम्}
{नात्यक्त्वा चाभयः शेते त्यक्त्वा सर्वं सुखी भवेत्}


\twolineshloka
{इत्येतद्धास्तिनपुरे ब्राह्मणेनोपवर्णितम्}
{शम्याकेन पुरा मह्यं तस्मात्त्यागः परो मतः}


\chapter{अध्यायः १७६}
\twolineshloka
{युधिष्ठिर उवाच}
{}


\threelineshloka
{ईहमानः समारम्भान्यदि नासादयेद्धनम्}
{धनतृष्णाभिभूतश्च किं कुर्वन्सुखमाप्नुयात् ॥भीष्म उवाच}
{}


\twolineshloka
{सर्वसाम्यमनायासः सत्यवाक्यं च भारत}
{निर्वेदश्चाविधित्सा च यस्य स्यात्स सखी नरः}


\twolineshloka
{एतान्येव पदान्याहुः पञ्च वृद्धाः प्रशान्तये}
{एष स्वर्गश्च धर्मश्च सुखं चानुत्तमं सताम्}


\twolineshloka
{अत्राप्युदाहरन्तीममितिहासं पुरातनम्}
{निर्वेदान्मङ्किना गीतं तन्निबोध युधिष्ठिर}


\twolineshloka
{ईहमानो धनं मङ्किर्भग्नेहश्च पुनः पुनः}
{केनचिद्धनलेशेन क्रीतवान्दम्यगोयुगम्}


\twolineshloka
{सुसंबद्धौ तु तौ दम्यौ दमनायाभिनिःसृतौ}
{आसीनमुष्ट्रं मध्येन सहसैवाभ्यधावताम्}


\twolineshloka
{तयोः संप्राप्तयोरुष्ट्रः स्कन्धदेशममर्पणः}
{उत्थायोत्क्षिप्य तौ दम्यौ पससार महाजवः}


\twolineshloka
{ह्रियमाणौ तु तौ दम्यौ तेनोष्ट्रेण प्रमाथिना}
{प्रियमाणौ च संप्रेक्ष्य मङ्किस्तत्राब्रवीदिदम्}


\twolineshloka
{न जात्वविहितं शक्यं दक्षेणाषीहितुं धनम्}
{युक्तेन श्रद्धया सम्यगीहां समनुतिष्ठता}


\twolineshloka
{पूर्वमर्यैर्विहीनस्य युक्तस्याप्युतिष्ठतः}
{इमं पश्यत संगत्या मम दैवमुपप्लवम्}


\twolineshloka
{उद्यम्योद्यम्य मे दम्यौ विषमेणैव गच्छतः}
{उत्क्षिप्य काकतालीयमुन्माथेनेव जम्बुकः}


\twolineshloka
{मणीवोष्ट्रस्य लम्वेते प्रियौ वत्सतरौ मम}
{शुद्वं हि दैवमेवेदं हठे नैवास्ति पौरुषम्}


\twolineshloka
{यदि वाऽप्युपपद्येत पौरुषं नाम कर्हिचित्}
{अन्विष्यमाणं तदपि दैवमेवावतिष्ठते}


\twolineshloka
{तस्मान्निर्वेद एवेह गन्तव्यः सुखभीप्सता}
{सुखं स्वपिति निर्विण्णो निराशश्चार्थसाधने}


\twolineshloka
{अहो सम्यक्शुकेनोक्तं सर्वतः परिमुच्यता}
{प्रतिष्ठता महारण्यं जनकस्य निवेशनात्}


\twolineshloka
{यः कामानाप्नुयात्सर्वान्यश्चैतान्केवलांस्त्यजेत्}
{प्रापणात्सर्वकामानां परित्यागो विशिष्यते}


\twolineshloka
{नान्तं सर्वविधित्सानां गतपूर्वोऽस्ति कश्चन}
{शरीरे जीविते चैव तृष्णा मर्त्यस्य वर्धते}


\twolineshloka
{निवर्तस्य विधित्साभ्यः शाम्य निर्विद्य कामुक}
{असकृच्चासि निकृतो न च निर्विद्यसे मनः}


\twolineshloka
{यदि नाहं विनाश्यस्ते यद्येवं रमसे मया}
{मा मां योजय लोभेन वृथाऽत्वं वित्तकामुक}


\twolineshloka
{संचितं संचितं द्रव्यं नष्टं तव पुनः पुनः}
{कदा तां मोक्ष्यसे मूढ धनेहां धनकामुक}


\threelineshloka
{अहो नु मम बालिश्यं योऽहं क्रीडनकस्तव}
{`क्लेशैर्नानाविधैर्नित्यं संयोजयसि निर्घृणः}
{'किं नैवं जातु पुरुषः परेषां प्रेष्यतामियात्}


\twolineshloka
{न पूर्वे नापरे जातु कामानामन्तमाप्नुवन्}
{त्यक्त्वा सर्वसमारम्भान्प्रतिबुद्धोऽस्मि जागृमि}


\twolineshloka
{नूनं मे हृदयं कामं वज्रसारमयं दृढम्}
{यदनर्थशताविष्टं शतधा न विदीर्यते}


\twolineshloka
{त्यजामि काम त्वां चैव यच्च किंचित्प्रियं तव}
{तवाहं प्रियमन्विच्छन्नात्मन्युपलभे सुखम्}


\twolineshloka
{काम जानामि ते मूलं संकल्पात्किल जायसे}
{न त्वां संकल्पयिष्यामि समूलो नभविष्यसि}


\twolineshloka
{ईहा धनस्य न सुखा लुब्ध्वा चिन्ता च भूयसी}
{लब्धनाशो यथा मृत्युर्लब्धं भवति वा न वा}


\twolineshloka
{परित्यागे न लभते ततो दुःखतरं नु किम्}
{न च तुष्यति लब्धेन भूय एव च मार्गति}


\twolineshloka
{अनुतर्पुल एवार्थः स्वादु गाङ्गभिवोदकम्}
{मद्विलापनमेततु प्रतिबुद्धोऽस्मि संत्यज}


\twolineshloka
{य इमं मामकं देहं भूतग्रामः समाश्रितः}
{स यात्वितो यथाकामं वसतां वा यथासुखम्}


\twolineshloka
{न युष्मास्विह मे प्रीतिः कामलोभानुसारिषु}
{तस्मादुत्सृज्य वः सर्वान्सत्वमेवाश्रयाम्यहम्}


\twolineshloka
{सर्व भूतान्यहं देहे पश्यन्मनसि चात्मनः}
{योगे बुद्धिं श्रुते सत्वं मनो ब्रह्मणि धारयन्}


\twolineshloka
{विहरिष्याम्यनासक्तः सुखी लोकान्निरामयः}
{यथा मां त्वं पुनर्नैवं दुःखेषु प्रणिधास्यसि}


\twolineshloka
{त्वया हि मे प्रणुन्नस्य गतिरन्या न विद्यते}
{तृष्णा शोकश्रमाणां हि त्वं काम प्रभवः सदा}


\twolineshloka
{धननाशेऽधिकं दुःखं मन्ये सर्वमहत्तरम्}
{ज्ञातयो ह्यवमन्यन्ते मित्राणि च धनाच्च्युतम्}


\twolineshloka
{अवज्ञानसहस्रैस्तु दोषाः कष्टतराऽधने}
{धने सुखकला या तु साऽपि दुःखैर्विधीयते}


\twolineshloka
{धनमस्येति पुरुषं पुरो निघ्नन्ति दस्यवः}
{क्लिश्यन्ति विविधैर्दण्डैर्नित्यमुद्वेजयन्ति च}


\twolineshloka
{धनलोलुपता दुःखमिति बुद्धं चिरान्मया}
{यद्यदालम्बसे कामं तत्तदेवानुरुध्यसे}


\twolineshloka
{अतत्त्वज्ञोऽसि बालश्च दुस्तोषोऽपूरणोऽनलः}
{नैव त्वं वेत्थ सुलभं नैव त्वं वेत्थ दुर्लभम्}


\twolineshloka
{पाताल इव दुष्पूरो मां दुःखैर्योक्तुमिच्छसि}
{नाहमद्य समावेष्टुं शक्यः काम पुनस्त्वया}


\twolineshloka
{निर्वेदमहमासाद्य द्रव्यनाशाद्यदृच्छया}
{निवृत्तिं परमां प्राप्य नाद्य कामान्विचिन्तये}


\twolineshloka
{अतिक्लेशान्सहामीह नाहं बुद्ध्याम्यबुद्धिमान्}
{निकृतो धननाशेन शये सर्वाङ्गविज्वरः}


\twolineshloka
{परित्यजामि काम त्वां हित्वा सर्वं मनोगतम्}
{न त्वं मया पुनः काम नस्योतेनेव रंस्यसे}


\twolineshloka
{क्षमिष्ये क्षिपमाणानां न हिंसिष्ये विहिंसितः}
{द्वेष्यमुक्तः प्रियं वक्ष्याम्यनादृत्य तदप्रियम्}


\twolineshloka
{तृप्तः स्वस्थेन्द्रियो नित्यं यथालब्धेन वर्तयन्}
{न सकामं करिष्यामि त्वामहं शत्रुमात्मनः}


\twolineshloka
{निर्वेदं निर्वृतिं तृप्तिं शान्तिं सत्यं दमं क्षमाम्}
{सर्वभूतदयां चैव विद्धि मां शरणागतम्}


\twolineshloka
{तस्मात्कामश्च लोभश्च तृष्णा कार्पण्यमेव च}
{त्यजन्तु मां प्रतिष्ठन्तं सत्वस्थो ह्यस्मि सांप्रतम्}


\twolineshloka
{प्रहाय कामं लोभं च क्रोधं पारुष्यमेव च}
{नाद्य लोभवशं प्राप्तो दुःखं प्राप्स्याम्यनात्मवान्}


\twolineshloka
{यद्यस्त्यजति कामानां तत्सुखस्याभिपूर्यते}
{कामस्य वशगो नित्यं दुःखमेव प्रपद्यते}


\twolineshloka
{कामानुबन्धं नुदते यत्किंचित्पुरुषो रजः}
{कामक्रोधोद्भवं दुःखमह्रीररतिरेव च}


\twolineshloka
{एष ब्रह्मप्रतिष्ठोऽहं ग्रीष्मे शीतमिव ह्रदम्}
{शाम्यामि परिनिर्वामि सुखमासे च केवलम्}


\twolineshloka
{यच्च कामसुखं लोके यच्च दिव्यं महत्सुखम्}
{तृष्णाक्षयसुखस्यैते नार्हतः षोडशीं कलाम्}


\twolineshloka
{आत्मना सप्तमं कामं हत्वा शत्रुमिवोत्तमम्}
{प्राप्यावध्यं ब्रह्मपुरं राजेव च वसाम्यहम्}


\twolineshloka
{एतां बुद्धिं समास्थाय मङ्किर्निर्वेदमागतः}
{सर्वान्कामान्परित्यज्य प्राप्य ब्रह्म महत्सुखम्}


\twolineshloka
{दम्यनाशकृते मङ्किरमृतत्वं किलागमत्}
{अच्छिनत्काममूलं स तेन प्राप परां गतिम्}


\twolineshloka
{अत्राप्युदाहरन्तीमं श्लोकं मोक्षोपसंहितम्}
{गीतं विदेहराजेन जनकेन प्रशाम्यता}


\twolineshloka
{अनन्तमिव मे वित्तं यस्य मे नास्ति किंचन}
{मिथिलायां प्रदीप्तायां न मे दह्यति किंचन}


\twolineshloka
{अत्रैवोदाहरन्तीमं बोध्यस्य पदसंचयम्}
{निर्वेदं प्रति विन्यस्तं तं निबोध युधिष्ठिर}


\twolineshloka
{बोध्यं शान्तमृषिं राजा नाहुषः पर्यपृच्छत}
{निर्वेदाच्छान्तिमापन्नं शास्त्रप्रज्ञानतर्पितम्}


\threelineshloka
{उपदेशं महाप्राज्ञ शमस्योपदिशस्व मे}
{कां बुद्धिं समनुप्राप्य शान्तश्चरसि निर्वृतः ॥बोध्य उवाच}
{}


\twolineshloka
{उपदेशेन वर्तामि नानुशास्मीह कंचन}
{लक्षणं तस्य वक्ष्येऽहं तत्स्वयं परिमृष्यताम्}


\threelineshloka
{पिङ्गला कुररः सर्पः सारङ्गान्वेषणं वने}
{इषुकारः कुमारी च षडेते गुरवो मम ॥[*भीष्म उवाच}
{}


\twolineshloka
{आशा बलवती राजन्नैराश्यं परमं सुखम्}
{आशां निराशां कृत्वा तु सुखं स्वपिति पिङ्गला}


\twolineshloka
{सामिषं कुररं दृष्ट्वा वध्यमानं निरामिषैः}
{आमिषस्य परित्यागात्कुररः सुखमेधते}


\twolineshloka
{गृहारम्भो हि दुःखाय न सुखाय कदाचन}
{सर्पः परकृतं वेश्म प्रविश्य सुखमेधते}


\twolineshloka
{सुखं जीवन्ति मुनयो भैक्ष्यवृत्तिं समाश्रिताः}
{अद्रोहणैव भूतानां सारङ्ग इव पक्षिणः}


\twolineshloka
{`अल्पेभ्यश्च महद्भ्यश्च शास्त्रेभ्यो मतिमान्नरः}
{सर्वतः सारमादद्यात्पुष्पेभ्य इव षट््पदः ॥'}


\twolineshloka
{इषुकारो नरः कश्चिदिपावासक्तमानसः}
{समीपेनापि गच्छन्तं राजानं नावबुद्धवान्}


\twolineshloka
{बहूनां कलहो नित्यं द्वयोः संकथनं ध्रुवम्}
{एकाकी विचरिष्यानि कुमारीशङ्खको यथा ॥]}


\chapter{अध्यायः १७७}
\twolineshloka
{युधिष्ठिर उवाच}
{}


\threelineshloka
{केन वृत्तेन वृत्तज्ञ वीतशोकश्चरेन्महीम्}
{किंच कुर्वन्नरो लोके प्राप्नोति गतिमुत्तमाम् ॥भीष्म उवाच}
{}


\twolineshloka
{अत्राप्युदाहरन्तीममितिहासं पुरातनम्}
{प्रह्लादस्य च संवादं मुनेराजगरस्य च}


\threelineshloka
{चरन्तं ब्राह्मणं कंचित्कल्यचित्तमनामयम्}
{पप्रच्छ राजा प्रह्लादो बुद्धिमान्प्राज्ञसत्तमः ॥प्रह्लाद उवाच}
{}


\twolineshloka
{स्वस्थः शक्तो मृदुर्दान्तो निर्विधित्सोऽनसूयकः}
{सुवाग्बहुमतो लोके प्राज्ञश्चरसि बालवत्}


\twolineshloka
{नैव प्रार्थयसे लाभं नालाभेष्वनुशोचसि}
{नित्यतृप्त इव ब्रह्मन्न किंचिदिव मन्यसे}


\twolineshloka
{स्रोतसा ह्रियमाणासु प्रजासु विमना इव}
{धर्मकामार्थकार्येषु कूटस्थ इव लक्ष्यसे}


\twolineshloka
{नानुतिष्ठसि धर्मार्थौ न कामे चापि वर्तसे}
{इन्द्रियार्थाननादृत्य मुक्तश्चरसि साक्षिवत्}


\threelineshloka
{का नु प्रज्ञा श्रुतं वा किं वृत्तिर्वा का नु ते मुने}
{क्षिप्रमाचक्ष्व मे ब्रह्मञ्श्रेयो यदिह मन्यसे ॥भीष्म उवाच}
{}


\twolineshloka
{अनुयुक्तः स मेधावी लोकधर्मविधानवित्}
{उवाच श्लक्ष्णया वाचा प्रह्लादमनपार्थया}


\twolineshloka
{पश्य प्रह्लाद भूतानामुत्पत्तिमनिमित्ततः}
{ह्रासं वृद्धिं विनाशं च न प्रहृष्ये न च व्यथे}


\twolineshloka
{स्वभावादेव संदृश्या वर्तमानाः प्रवृत्तयः}
{स्वभावनिरताः सर्वाः प्रतिपाद्या न केनचित्}


\twolineshloka
{पश्य प्रह्लाद संयोगान्विप्रयोगपरायणान्}
{संचयांश्च विनाशान्तान्न क्वचिद्विदधे मनः}


\threelineshloka
{अन्तवन्ति च भूतानि गुणयुक्तानि पश्यतः}
{उत्पत्तिनिधनज्ञस्य किं पर्यायेणोपलक्षये}
{}


\twolineshloka
{जलजानामपि ह्यन्तं पर्यायेणोपलक्षये}
{महतामपि कायानां सूक्ष्माणां च महोदधौ}


\twolineshloka
{जङ्गमस्थावराणां च भूतानामसुराधिप}
{पार्थिवानामपि व्यक्तं मृत्युं पश्यामि सर्वशः}


\twolineshloka
{अन्तरिक्षचराणां च दानवोत्तमपक्षिणाम्}
{उत्तिष्ठते यथाकालं मृत्युर्बलवतामपि}


\twolineshloka
{दिवि संचरमाणानि ह्रस्वानि च महान्ति च}
{ज्योतींष्यपि यथाकालं पतमानानि लक्षये}


\twolineshloka
{इति भूतानि संपश्यन्ननुषक्तानि मृत्युना}
{सर्वं सामान्यतो विद्वान्कृतकृत्यः सुखं स्वपे}


\twolineshloka
{सुमहान्तमपि ग्रासं ग्रसे लब्धं यदृच्छया}
{शये पुनरभुञ्जानो दिवसानि बहून्यपि}


\twolineshloka
{आशयन्त्यपि मामन्नं पुनर्बहुगुणं बहु}
{पुनरल्पं पुनस्तोकं पुनर्नैवोपपद्यते}


\twolineshloka
{कणं कदाचित्खादामि पिण्याकमपि च ग्रसे}
{भक्षये शालिमांसानि भक्षांश्चोच्चावचान्पुनः}


\twolineshloka
{शये कदाचित्पर्यङ्के भूमावपि पुनः शये}
{प्रासादे चापि मे शय्या कदाचिदुपपद्यते}


\twolineshloka
{धारयामि च चीराणि शाणक्षौमाजिनानि च}
{महार्हाणि च वासांसि धारयाम्यहमेकदा}


\twolineshloka
{न सन्निपतितं धर्म्यमुपभोगं यदृच्छया}
{प्रत्याचक्षे न चाप्येनमनुरुध्ये सुदुर्लभम्}


\twolineshloka
{अचलमनिधनं शिवं विशोकंशुचिमतुलं विदुषां मते प्रविष्टम्}
{अनभिमतमसेवितं विमूढैर्व्रतमिदमाजगरं शुचिश्चरामि}


\twolineshloka
{अचलितमतिरच्युतः स्वधर्मात्परिमितसंसरणः परावरज्ञः}
{विगतभयकषायलोभमोहोव्रतमिदमाजगरं शुचिश्चरामि}


\twolineshloka
{अनियतफलभक्ष्यभोज्यपेयंविधिपरिणामविभक्तदेशकालम्}
{हृदयसुखमसेवितं कदर्यैर्व्रतमिदमाजगरं सुचिश्चरामि}


\twolineshloka
{इदमिदमिति तृष्णयाऽभिभूतंजनमनवाप्तधनं विषीदमानम्}
{निपुणमनुनिशाम्य तत्त्वबुद्ध्याव्रतमिदमाजगरं शुचिश्चरामि}


\twolineshloka
{बहुविधमनुदृश्य चार्थहेतोःकृपणमिहार्यमनार्यमाश्रयं तम्}
{उपशमरुचिरात्मवान्प्रशान्तोव्रतमिदमाजगरं शुचिश्चरामि}


\twolineshloka
{सुखमसुखमलाभमर्थलाभंरतिमरतिं मरणं च जीवितं च}
{विधिनियतमवेक्ष्य तत्त्वतोऽहंव्रतमिदमाजगरं शुचिश्चरामि}


\twolineshloka
{अपगतभयरागमोहदर्पोधृतिमतिबुद्धिसमन्वितः प्रशान्तः}
{उपगतफलभोगिनो निशाम्यव्रतमिदमाजगरं शुचिश्चरामि}


\twolineshloka
{अनियतशयनासनः प्रकृत्यादमनियमव्रतसत्यशौचयुक्तः}
{अपगतफलसंचयः प्रहृष्टोव्रतमिदमाजगरं शुचिश्चरामि}


\twolineshloka
{अपगतमसुखार्थमीहनार्थैरुपगतबुद्धिरवेक्ष्य चात्मसंस्थम्}
{तृपितमनियतं मनो नियन्तुंव्रतमिदमाजगरं शुचिश्चरामि}


\twolineshloka
{न हृदयमनुरुध्यते मनो वाप्रियसुखदुर्लभतामनित्यतां च}
{तदुभयमुपलक्षयन्निवाहंव्रतमिदमाजगरं शुचिश्चरामि}


\twolineshloka
{बहु कथितमिदं हि बुद्धिमद्भिःकविभिरपि प्रथयद्भिरात्मकीर्तिम्}
{इदमिदमिति तत्रतत्र हन्तस्वपरमतैर्गहनं प्रतर्कयद्भिः}


\threelineshloka
{तदिदमनुनिशाम्य विप्रपातंपृथगभिपन्नमिहाबुधैर्मनुष्यैः}
{अनवसितमनन्तदोषपारंनृपु विहरामि विनीतदोषतृष्णः ॥भीष्म उवाच}
{}


\twolineshloka
{अजगरचरितं व्रतं महात्माय इह नरोऽनुचरेद्विनीतरागः}
{अपगतभयलोभमोहमन्युःस खलु सुखी विचरेदिमं विहारम्}


\chapter{अध्यायः १७८}
\twolineshloka
{युधिष्ठिर उवाच}
{}


\threelineshloka
{बान्धवाः कर्म वित्तं वा प्रज्ञा वेह पितामह}
{नरस्य का प्रतिष्ठा स्यादेतत्पृष्टो वदस्व मे ॥भीष्म उवाच}
{}


\twolineshloka
{प्रज्ञा प्रतिष्ठा भूतानां प्रज्ञा लाभः परो मतः}
{प्रज्ञा निःश्रेयसी लोके प्रज्ञा स्वर्गो मतः सताम्}


\twolineshloka
{प्रज्ञया प्रापितार्थो हि बलिरैश्वर्यसंक्षये}
{प्रह्लादो नमुचिर्मङ्किस्तस्याः किं विद्यते परम्}


\twolineshloka
{अत्राप्युदाहरन्तीममितिहासं पुरातनम्}
{इन्द्रकाश्यपसंवादं तन्निबोध युधिष्ठिर}


\twolineshloka
{वैश्यः कश्चिदृषिसुतं काश्यपं संशितव्रतम्}
{रथेन पातयामास श्रीमान्दृप्तस्तपस्विनम्}


\twolineshloka
{आर्तः स पतितः क्रुद्धस्त्यक्त्वाऽऽत्मानमथाब्रवीत्}
{मरिष्याम्यधनस्येह जीवितार्थो न विद्यते}


\twolineshloka
{तथा मुमूर्षमासीनमकूजन्तमचेतसम्}
{इन्द्रः सृगालरूपेण बभाषे क्षुब्धमानसम्}


\twolineshloka
{मनुष्ययोनिमिच्छन्ति सर्वभूतानि सर्वशः}
{मनुष्यत्वे च विप्रत्वं सर्व एवाभिनन्दति}


\twolineshloka
{मनुष्यो ब्राह्मणश्चासि श्रोत्रियश्चासि काश्यप}
{सुदुर्लभमवाप्यैतन्न दोषान्मर्तुमर्हसि}


\twolineshloka
{सर्वे लाभाः साभिमाना इति सत्यवती श्रुतिः}
{संतोषणीयरूपोऽसि लोभाद्यदभिमन्यसे}


\twolineshloka
{अहो सिद्धार्थता तेषां येषां सन्तीह पाणयः}
{[अतीव स्पृहये तेषां येषां सन्तीह पाणयः ॥]}


\twolineshloka
{पाणिमद्भ्यः स्पृहाऽस्माकं यथा तव धनस्य वै}
{न पाणिलाभादधिको लाभः कश्चन विद्यते}


\twolineshloka
{अपाणित्वाद्वयं ब्रह्मन्कण्टकं नोद्धरामहे}
{जन्तूनुच्चावचानङ्गे दशतो न कषाम वा}


\twolineshloka
{अथ येषां पुनः पाणी देवदत्तौ दशाङ्गुली}
{उद्धरन्ति कृमीनङ्गाद्दशतो निकषन्ति च}


\twolineshloka
{वर्षाहिमातपानां च परित्राणानि कुर्वते}
{चेलमन्नं सुखं शय्यां निवातं चोपभुञ्जते}


\twolineshloka
{अधिष्ठाय च गां लोके भुञ्जते वाहयन्ति च}
{उपायैर्बहुभिश्चैव वश्यानात्मनि कुर्वते}


\twolineshloka
{ये खल्वजिह्वाः कृपणा अल्पप्राणा अपाणयः}
{सहन्ते तानि दुःखानि दिष्ट्या त्वं न तथा मुने}


\twolineshloka
{दिष्ट्या त्वं न शृगालो वै न कृमिर्न च मूषकः}
{न सर्पो न च मण्डूको न चान्यः पापयोनिजः}


\twolineshloka
{एतावताऽपि लाभेन तोष्टुमर्हसि काश्यप}
{किं पुनर्योसि सत्वानां सर्वेषां ब्राह्मणोत्तमः}


\twolineshloka
{इमे मां कृमयोऽदन्ति येषामुद्धरणाय वै}
{नास्ति शक्तिरपाणित्वात्पश्यावस्थामिमां मम}


\twolineshloka
{अकार्यमिति चैवेमं नात्मानं संत्यजाम्यहम्}
{नातः पापीयसीं योनिं पतेयमपरामिति}


\twolineshloka
{मध्ये वै पापयोनीनां सृगालीयामहं गतः}
{पापीयस्यो बहुतरा इतोऽन्याः पापयोनयः}


\twolineshloka
{जात्यैवैके सुखितराः सन्त्यन्ये भृशदुःखिताः}
{नैकान्तं सुखमेवेह क्वचित्पश्यामि कस्यचित्}


\twolineshloka
{मनुष्या ह्याढ्यतां प्राप्य राज्यमिच्छन्त्यनन्तरम्}
{राज्याद्देवत्वमिच्छन्ति देवत्वादिन्द्रतामपि}


\twolineshloka
{भवेस्त्वं यद्यपि त्वाढ्यो न राजा न च दैवतम्}
{देवत्वं प्राप्य चेन्द्रत्वं नैव तुष्येस्तथा सति}


\twolineshloka
{न तृप्तिः प्रियलाभेऽस्ति तृष्णा नाद्भिः प्रशाम्यति}
{संप्रज्वलति सा भूयः समिद्भिरिव पावकः}


\twolineshloka
{अस्त्येव त्वयि शोकोऽपि हर्षश्चापि तथा त्वयि}
{सुखदुःखे तथा चोभे तत्र का परिदेवना}


\twolineshloka
{परिच्छिद्यैव कामानां सर्वेषां चैव कर्मणाम्}
{मूलं बुद्धीन्द्रियग्रामं शकुन्तानिव पञ्जरे}


\twolineshloka
{न द्वितीयस्य शिरसश्छेदनं विद्यते क्वचित्}
{न च पाणेस्तृतीयस्य यन्नास्ति न ततो भयम्}


\twolineshloka
{न खल्वप्यरसज्ञस्य कामः क्वचन जायते}
{संस्पर्शाद्दर्शनाद्वापि श्रवणाद्वापि जायते}


\twolineshloka
{न त्वं स्मरसि वारुण्या लट्वाकानां च पक्षिणाम्}
{ताभ्यां चाभ्यधिको भक्ष्यो न कश्चिद्विद्यते क्वचित्}


\twolineshloka
{यानि चान्यानि भूतेषु भक्ष्यभोज्यानि काश्यप}
{येषामभुक्तपूर्वाणि तेषामस्मृतिरेव ते}


\twolineshloka
{अप्राशनमसंस्पर्शमसंदर्शनमेव च}
{पुरुषस्यैष नियमो मन्ये श्रेयो न संशयः}


\twolineshloka
{पाणिमन्तो बलवन्तो धनवन्तो न संशयः}
{मनुष्या मानुषैरेव दासत्वमुपपादिताः}


\twolineshloka
{वधबन्धपरिक्लेशैः क्लिश्यन्ते च पुनः पुनः}
{ते खल्वपि रमन्ते च मोदन्ते च हसन्ति च}


\twolineshloka
{अपरे बाहुबलिनः कृतविद्या मनस्विनः}
{जुगुप्सितां च कृपणां पापवृत्तिमुपासते}


\twolineshloka
{उत्सहन्ते च ते वृत्तिमन्यामप्युपसेवितुम्}
{स्वकर्मणा तु नियतं भवितव्यं तु तत्तथा}


\twolineshloka
{न पुल्कसो न चण्डाल आत्मानं त्यक्तुमिच्छति}
{तया तुष्टः स्वया योन्या मायां पश्यस्व यादृशीम्}


\twolineshloka
{दृष्ट्वा कुणीन्पक्षहतान्मनुष्यानामयाविनः}
{सुसंपूर्णः स्वया योन्या लब्धलाभोसि काश्यप}


\twolineshloka
{यदि ब्राह्मणदेहस्ते निरातङ्को निरामयः}
{अङ्गानि च समग्राणि न च लोकेषु धिक्कृतः}


\twolineshloka
{न केनचित्प्रवादेन सत्येनैवापहारिणा}
{धर्मायोत्तिष्ठ विप्रर्षे नात्मानं त्यक्तुमर्हसि}


\twolineshloka
{यदि ब्रह्मञ्शृणोष्येतच्छ्रद्दधासि च मे वचः}
{वेदोक्तस्यैव धर्मस्य फलं मुख्यमवाप्स्यसि}


\twolineshloka
{स्वाध्यायमग्निसंस्कारमप्रमत्तोऽनुपालय}
{सत्यं दमं च दानं च स्पर्धिष्ठा मा च केनचित्}


\twolineshloka
{ये केचन स्वध्ययनाः प्राप्ता यजनयाजनम्}
{कथं ते चानुशोचेयुर्ध्यायेयुर्वाऽप्यशोभनम्}


\threelineshloka
{इच्छन्तस्ते विहाराय सुखं महदवाप्नुयुः}
{येऽनुजाताः सुनक्षत्रे सुतिथौ सुमुहूर्तके}
{यज्ञदानप्रजेहायां यतन्ते शक्तिपूर्वकम्}


\twolineshloka
{नक्षत्रेष्वासुरेष्वन्ये दुस्तिथौ दुर्मुहूर्तजाः}
{सम्पतन्त्यासुरीं योनिं यज्ञप्रसववर्जिताः}


\twolineshloka
{अहमासं पण्डितको हैतुको वेदनिन्दकः}
{आन्वीक्षिकीं तर्कविद्यामनुरक्तो निरर्थिकाम्}


\twolineshloka
{हेतुवादान्प्रवदिता वक्ता संसत्सु हेतुमत्}
{आक्रोष्टा चातिवक्ता च ब्रह्मवाक्येषु च द्विजान्}


\twolineshloka
{नास्तिकः सर्वशङ्की च मूर्खः पण्डितमानिकः}
{तस्येयं फलनिर्वृत्तिः सृगालत्वं मम द्विज}


\twolineshloka
{अपि जातु तथा तत्स्यादहोरात्रशतैरपि}
{यदहं मानुषीं योनिं सृगालः प्राप्नुयां पुनः}


\threelineshloka
{संतुष्टश्चाप्रमत्तश्च यज्ञदानतपोरतः}
{ज्ञेयं ज्ञाता भवेयं वै वर्ज्यं वर्जयिता तथा ॥भीष्म उवाच}
{}


\twolineshloka
{ततः स मुनिरुत्थाय काश्यपस्तमुवाच ह}
{अहो बतामि कुशलो बुद्धिमांश्चेति विस्मितः}


\twolineshloka
{समवैक्षत तं विप्रो ज्ञानदीर्घेण चक्षुषा}
{ददर्श चैनं देवानां देवमिन्द्रं शचीपतिम्}


\twolineshloka
{ततः संपूजयामास काश्यपो हरिवाहनम्}
{अनुज्ञातस्तु तेनाथ प्रविवेश स्वमालयम्}


\chapter{अध्यायः १७९}
\twolineshloka
{युधिष्ठिर उवाच}
{}


\threelineshloka
{यद्यस्ति दत्तमिष्टं वा तपस्तप्तं तथैव च}
{गुरूणां वाऽपि शुश्रूषा तन्मे ब्रूहि पितामह ॥भीष्म उवाच}
{}


\twolineshloka
{`यथाऽस्मिंश्च तथा तत्र जानीयां नृपसत्तम}
{दुष्कर्तारो यथा लोके यत्कुर्वन्ति तथा शृणु ॥'}


\twolineshloka
{आत्मनाऽनर्थयुक्तेन पापे निविशते मनः}
{स्वकर्म कलुषं कृत्वा दुःखे महति धीयते}


\twolineshloka
{दुर्भिक्षादेव दुर्भिक्षं क्लेशात्क्लेशं भयाद्भयम्}
{मृतेभ्यः प्रमृता यान्ति दरिद्राः पापकारिणः}


\twolineshloka
{उत्सवादुत्सवं यान्ति स्वर्गात्स्वर्गं सुखात्सुखम्}
{श्रद्दधानाश्च दान्ताश्च सत्वस्थाः शुभकारिणाः}


\twolineshloka
{व्यालकुञ्जरदुर्गेषु सर्पचोरभयेषु च}
{हस्तावापेन गच्छन्ति नास्तिकाः किमतः परम्}


\twolineshloka
{प्रियदेवातिथेयाश्च वदान्याः प्रियसाधवः}
{क्षेम्यमात्मवतां मार्गमास्थिता हस्तदक्षिणम्}


\twolineshloka
{पुलाका इव धान्येषु पुत्तिका इव पक्षिषु}
{तद्विधास्ते मनुष्येषु येषां धर्मो न कारणम्}


\twolineshloka
{सुशीघ्रमपि धावन्तं विधानमनुधावति}
{शेते सह शयानेन येनयेन यथाकृतम्}


\twolineshloka
{उपतिष्ठति तिष्ठन्तं गच्छन्तमनुगच्छति}
{करोति कुर्वतः कर्म छायेवाऽनुविधीयते}


\twolineshloka
{येनयेन यथा यद्यत्पुरा कर्म समार्जितम्}
{तत्तदेव नरो भुङ्क्ते नित्यं विहितमात्मना}


\twolineshloka
{स्वकर्मफलनिक्षेपं विधानपरिरक्षितम्}
{भूतग्राममिमं कालः समन्तात्परिकर्षति}


\twolineshloka
{अचोद्यमानानि यथा पुष्पाणि च फलानि च}
{स्वं कालं नातिवर्तन्ते तथा कर्म पुराकृतम्}


\twolineshloka
{संमानश्चावमानश्च लाभालाभौ क्षयोदयौ}
{प्रवृत्तानि विवर्तन्ते विद्यानान्ते पुनःपुनः}


\twolineshloka
{आत्मना विहितं दुःखमात्मना विहितं सुखम्}
{गर्भशय्यामुपादाय भुज्यते पौर्वदेहिकम्}


\twolineshloka
{बालो युवा च वृद्धश्च यत्करोति शुभाशुभम्}
{तस्यांतस्यामवस्थायां भुङ्क्ते जन्मनिजन्मनि}


\twolineshloka
{यथा धेनुसहस्रेषु वत्सो विन्दति मातरम्}
{तथा पूर्वकृतं कर्म कर्तारमनुगच्छति}


\twolineshloka
{संक्लिन्नमग्रतो वस्त्रं पश्चाच्छुध्यति वारिणा}
{`दुष्कर्मापि तथा पश्चात्पूयते पुण्यकर्मणा}


% Check verse!
तपसा तप्यते देहस्तपसा विन्दते महत् ॥'उपवासैः प्रतप्तानां दीर्घं सुखमनन्तरम्
\twolineshloka
{दीर्घकालेन तपसा सेवितेन तपोवने}
{धर्मनिर्धूतपापानां संसिद्ध्यन्ते मनोरथाः}


\twolineshloka
{शकुनीनामिवाकाशे मत्स्यानामिव चोदके}
{पदं यथा न दृश्येत तथा धर्मविदां गतिः}


\twolineshloka
{अलमन्यैरुपालम्भैः कीर्तितैश्च व्यतिक्रमैः}
{पेशलं चानुरूपं च कर्तव्यं हितमात्मनः}


\chapter{अध्यायः १८०}
\twolineshloka
{युधिष्ठिर उवाच}
{}


\twolineshloka
{कुतः सृष्टमिदं सर्वं जगत्स्थावरजङ्गमम्}
{प्रलये च किमभ्येति तन्मे ब्रूहि पितामह}


\twolineshloka
{ससागरः सगगनः सशैलः सबलाहकः}
{सभूमिः साग्निपवनो लोकोऽयं केन निर्मितः}


\twolineshloka
{कथं सृष्टानि भूतानि कथं वर्णविभक्तयः}
{शोचाशौचं कथं तेषां धर्माधर्मावथो कथम्}


\threelineshloka
{कीदृशो जीवतां जीवः क्व वा गच्छन्ति ये मृताः}
{अस्माल्लोकादमुं लोकं सर्वं शंसतु नो भवान् ॥भीष्म उवाच}
{}


\twolineshloka
{अत्राप्युदाहरन्तीममितिहासं पुरातनम्}
{भृगुणाऽभिहितं श्रेष्ठं भरद्वाजाय पृच्छते}


\twolineshloka
{कैलासशिखरे दृष्ट्वा दीप्यमानमिवौजसा}
{भृगुं महर्षिमासीनं भरद्वाजोऽन्वपृच्छत}


\twolineshloka
{ससागरः सगगनः शशैलः सवलाहकः}
{सभूमिः साग्निपवनो लोकोऽयं केन निर्मितः}


\threelineshloka
{कथं सृष्टानि भूतानि कथं वर्णविभक्तयः}
{शौचाशौचं कथं तेषां धर्माधर्मावथो कथम्}
{}


\twolineshloka
{कीदृशो जीवतां जीवः क्व वा गच्छन्ति ये मृताः}
{परलोकमिमं चापि सर्वं शंसितृमर्हसि}


\threelineshloka
{एवं स भगवान्पृष्टो भरद्वाजेन संशयम्}
{ब्रह्मर्पिर्ब्रह्मसंकाशः सर्वं तस्मै ततोऽब्रवीत् ॥भृगुरुवाच}
{}


\twolineshloka
{`नारायणो जगन्मूर्तिरन्तरात्मा सनातनः}
{कूटस्थोऽक्षर अव्यक्तो निर्लेपो व्यापकः प्रभुः}


\twolineshloka
{प्रकृतेः परतो नित्यमिन्द्रियैरप्यगोतरः}
{स सिसृक्षुः सहस्रांशादसृजत्पुरुषं प्रभुः ॥'}


\twolineshloka
{मानसो नाम यः पूर्वो विश्रुतो वै महर्षिभिः}
{अनादिनिधनो देवस्तथाऽभेद्योऽदजरामरः}


\twolineshloka
{अव्यक्त इति विख्यातः शाश्वतोऽथाक्षयोऽव्ययः}
{यतः सृष्टानि भूतानि तिष्ठन्ति च म्रियन्ति च}


\threelineshloka
{सोऽसृजत्प्रथमं देवो महान्तं नाम नामतः}
{महान्ससर्जाहंकारं स चापि भगवानथ}
{आकाशमिति विख्यातं सर्वभूतधरः प्रभुः}


\twolineshloka
{आकाशादभवद्वारि सलिलादग्निमारुतौ}
{अग्निमारुतसंयोगात्ततः समभवन्मही}


\twolineshloka
{ततस्ते गेमयं दिव्यं पद्मं सृष्टं स्वयंभुवा}
{तस्मात्पद्मात्समभवद्ब्रह्मा वेदमयो निधिः}


\twolineshloka
{अहंकार इति ख्यातः सर्वभूतात्मभूतकृत्}
{ब्रह्मा वै स महातेजा य एते पञ्चधातवः}


\twolineshloka
{शैलास्तस्यास्थिसंज्ञास्तु मेदो मांसं च मेदिनी}
{समुद्रास्तस्य रुधिरमाकाशमुदरं तथा}


\twolineshloka
{पवनश्चैव निःश्वासस्तेजोऽग्निर्निम्नगाः सिराः}
{दिवाकरश्च सोमश्च नयने तस्य विश्रुते}


\twolineshloka
{नभश्चोर्ध्वं शिरस्तस्य क्षितिः पादौ भुजौ दिशः}
{दुर्विज्ञेयो ह्यनन्तात्मा सिद्धैरपि न संशयः}


\twolineshloka
{स एष भगवान्विष्णुरनन्त इति विश्रुतः}
{सर्वभूतात्मभूतस्थो दुर्विज्ञेयोऽकृतात्माभः}


\threelineshloka
{अहंकारस्य यः स्रष्टा सर्वभूतोद्भवाय वै}
{यतः समभवद्विश्वं पृष्टोऽहं यदिह त्वया ॥भरद्वाज उवाच}
{}


\threelineshloka
{गगनस्य दिशां चैव भूतलस्यानिलस्य च}
{कान्यत्र परिमाणानि संशयं छिन्धि मेऽर्थितः ॥भृगुरुवाच}
{}


\twolineshloka
{अनन्तमेतदाकाशं सिद्धचारणसेवितम्}
{रम्यं नानाश्रयाकीर्णं यस्यान्तो नाधिगम्यते}


\twolineshloka
{ऊर्ध्वं गतेरधस्तात्तु चन्द्रादित्यौ न दृश्यतः}
{तत्र देवाः स्वयं दीप्ताः सूर्यभासोऽग्निवर्चसः}


\twolineshloka
{ते चाप्यन्तं न पश्यन्ति नभसः प्रथितौजसः}
{दुर्गमत्वादनन्तत्वादिति वै विद्धि मानद}


\twolineshloka
{उपर्युपरि तैर्देवैः प्रज्वलद्भिः स्वयंप्रभैः}
{निरुद्धमेतदाकाशमप्रमेयं सुरैरपि}


\twolineshloka
{पृथिव्यन्ते समुद्रास्तु समुद्रान्ते तमः स्मृतम्}
{तप्नसोऽन्ते जलं प्राहुर्जलस्यान्तेऽग्निरेव च}


\twolineshloka
{रसातलान्ते सलिलं जलान्ते पन्नगाधिपाः}
{तदन्ते पुनराकाशमाकाशान्ते पुनर्जलम्}


\twolineshloka
{एवमन्तं हि नभसः प्रमाणं सलिलस्य च}
{अग्निमारुतयोश्चैव दुर्ज्ञेयं दैवतैरपि}


\twolineshloka
{अग्निमारुततोयानां वर्णाः क्षितितलस्य च}
{आकाशसदृशा ह्येते भिद्यन्तेऽतत्वदर्शनात्}


\twolineshloka
{पठन्ति चैव मुनयः शास्त्रेषु विविधेषु च}
{त्रैलोक्यसागरे चैव प्रमाणं विहितं यथा}


\threelineshloka
{अदृश्यत्वादगम्यत्वात्कः प्रमाणमुदाहरेत्}
{सिद्धानां देवतानां च यदा परिमिता गतिः}
{तदा गौणमनन्तस्य नामानन्तेति विश्रुतम्}


\threelineshloka
{नामधेयानुरूपस्य मानसस्य महात्मनः}
{यदा तु र्दिव्यं यद्रूपं ह्रसते वर्धते पुनः}
{कोऽन्यस्तद्वेदितुं शक्तो योपि स्यात्तद्विधोऽपरः}


\threelineshloka
{ततः पुष्करतः सृष्टः सर्वज्ञो मूर्तिमान्प्रभुः}
{ब्रह्मा धर्ममयः पूर्वः प्रजापतिरनुत्तमः ॥भरद्वाज उवाच}
{}


\threelineshloka
{पुष्कराद्यदि संभूतो ज्येष्ठं भवति पुष्करम्}
{ब्रह्माणं पूर्वजं चाह भवान्संदेह एव मे ॥भृगुरुवाच}
{}


\twolineshloka
{मानसस्येह या मूर्तिर्ब्रह्मत्वं समुपागता}
{तस्यासनविधानार्थं पृथिवी पद्ममुच्यते}


\threelineshloka
{कर्णिकां तस्य पद्मस्य मेरुर्गगमुच्छ्रितः}
{तस्य मध्ये स्थितो लोकान्सृजते जगतः प्रभुः}
{मानसांश्च तथा देवान्भूतानि विविधानि च}


\chapter{अध्यायः १८१}
\twolineshloka
{भरद्वाज उवाच}
{}


\threelineshloka
{मेरुमध्ये स्थितो ब्रह्मा कथं स ससृजे प्रजाः}
{एतन्मे सर्वमाचक्ष्व याथातथ्येन पृच्छतः ॥भृगुरुवाच}
{}


\twolineshloka
{प्रजाविसर्गं पूर्वं स मानसो मनसाऽसृजत्}
{संरक्षणार्थं भूतानां सृष्टं प्रथमतो जलम्}


\twolineshloka
{यत्प्राणः सर्वभूतानां वर्धन्ते येन च प्रजाः}
{परित्यक्ताश्च नश्यन्ति तेनेदं सर्वमावृतम्}


\threelineshloka
{पृथिवी पर्वता मेघा मूर्तिमन्तश्च ये परे}
{सर्वं तद्वारुणं ज्ञेयमापस्तस्तम्भिरे हि ताः ॥भरद्वाज उवाच}
{}


\threelineshloka
{कथं सलिलमुत्पन्नं कथं चैवाग्निमारुतौ}
{कथं वा मेदिनी सृष्टेत्यत्र मे संशयो महान् ॥भृगुरुवाच}
{}


\twolineshloka
{ब्रह्मकल्पे पुरा ब्रह्मन्ब्रह्मर्षीणां समागमे}
{लोकसंभवसन्देहः समुत्पन्नो महात्मनाम्}


\twolineshloka
{तेऽतिष्ठन्ध्यानमालम्ब्य मौनमास्थाय निश्चलाः}
{त्यक्ताहाराः पवनपा दिव्यं वर्षशतं द्विजाः}


\twolineshloka
{तेषां ब्रह्ममयी वाणी सर्वेषां श्रोत्रमागमत्}
{दिव्या सरस्वती तत्र संबभूव नभस्तलात्}


\twolineshloka
{पुराऽस्तमितनिःशब्दमाकाशमचलोपमम्}
{नष्टचन्द्रार्कपवनं प्रसुप्तमिव संबभौ}


\twolineshloka
{ततः सलिलमुत्पन्नं तमसीवापरं तमः}
{तस्माच्च सलिलोत्पीडात्समजायत मारुतः}


\twolineshloka
{यथा भाजनमच्छिद्रं निःशब्दमिह लक्ष्यते}
{तच्चाम्भसा पूर्यमाणं सशब्दं कुरुतेऽनिलः}


\twolineshloka
{तथा सलिलसंरुद्धे नभसोन्ते निरन्तरे}
{भित्त्वाऽर्णवतलं वायुः समुत्पतति घोषवान्}


\twolineshloka
{स एष चरते वायुरर्णवोत्पीडसंभवः}
{आकाशस्थानमासाद्य प्रशान्तिं नाधिगच्छति}


\twolineshloka
{तस्मिन्वाय्वम्बुसंघर्षे दीप्ततेजा महाबलः}
{प्रादुर्बभूवोर्ध्वशिखः कृत्वा निस्तिमिरं नभः}


\twolineshloka
{अग्निः पवनसंयुक्तः स्वात्समुत्क्षिपते जलम्}
{सोऽग्निमारुतसंयोगाद्धनत्वमुपपद्यते}


\twolineshloka
{तस्याकाशान्निपतितः स्नेहस्तिष्ठति योऽपरः}
{स संघातत्वमापन्नो भूमित्वमनुगच्छति}


\twolineshloka
{रसानां सर्वगन्धानां स्नेहानां प्राणिनां तथा}
{भूमिर्योनिरिह ज्ञेया यस्यां सर्वं प्रसूयते}


\chapter{अध्यायः १८२}
\twolineshloka
{भरद्वाज उवाच}
{}


\twolineshloka
{त एते धातवः पञ्च ब्रह्मा यानसृजत्पुरा}
{आवृता यैरिमे लोका महाभूताभिसंज्ञिताः}


\threelineshloka
{यदाऽसृजत्सहस्राणि भूतानां स महामतिः}
{पञ्चानामेव भूतत्वं कथं समुपपद्यते ॥भृगुरुवाच}
{}


\twolineshloka
{अमितानां महाशब्दो भूतानां याति संभवम्}
{ततस्तेषां महाभूतशब्दोऽयमुपपद्यते}


\twolineshloka
{चेष्टा वायुः खमाकाशमूष्माऽग्निः सलिलं द्रवः}
{पृथिवी चात्र संघातः शरीरं पाञ्चभौतिकम्}


\threelineshloka
{इत्येतैः पञ्चभिर्भूतैर्युक्तं स्थावरजङ्गमम्}
{श्रोत्रं घ्राणं रमः स्पर्शो दृष्टिश्चेन्द्रियसंज्ञिताः ॥भरद्वाज उवाच}
{}


\twolineshloka
{पञ्चभिर्यदि भूतैस्तु यक्ताः स्थावरजङ्गमाः}
{स्थावराणां न दृश्यन्ते शरीरे पञ्च धातवः}


\twolineshloka
{अनूष्मणामचेष्टानां घनानां चैव तत्त्वतः}
{वृक्षा नोपलभ्यन्ते शरीरे पञ्च धातवः}


\twolineshloka
{न शृणुन्ते न पश्यन्ति न गन्धरसवेदिनः}
{न चस्पर्शं विजानन्ति ते कथं पाञ्चभौतिकाः}


\threelineshloka
{अद्रवत्वादनग्नित्वादभूतित्वादवायुतः}
{आकाशस्याप्रमेयत्वाद्वृक्षाणां नास्ति भौतिकम् ॥भृगुरुवाच}
{}


\twolineshloka
{घनानामपि वृक्षाणामाकाशोऽस्ति न संशयः}
{तेषां पुष्पफलव्यक्तिर्नित्यं समुपपद्यते}


\twolineshloka
{ऊष्मतो म्लायते वर्णं त्वक्फलं पुष्पमेव च}
{म्लायते शीर्यते चापि स्पर्शस्तेनात्र विद्यते}


\twolineshloka
{वाय्वग्न्यशनिनिष्पेषैः फलं पुष्पं विशीर्यते}
{श्रोत्रेण गृह्यते शब्दस्तस्माच्छृण्वन्ति पादपाः}


\twolineshloka
{वल्ली वेष्टयते वृक्षं सर्वतश्चैव गच्छति}
{न ह्यदृष्टेश्च मार्गोऽस्ति तस्मात्पश्यन्ति पादपाः}


\twolineshloka
{पुण्यापुण्यैस्तथा गन्धैर्धूपैश्च विविधैरपि}
{अरोगाः पुष्पिताः सन्ति तस्माज्जिघ्रन्ति पादपाः}


\twolineshloka
{पादैः सलिलपानाच्च व्याधीनां चापि दर्शनात्}
{व्याधिप्रतिक्रियत्वाच्च विद्यते रसनं द्रुमे}


\twolineshloka
{वक्रेणोत्पलनालेन यथोर्ध्वं जलमाददेत्}
{तथा पवनसंयुक्तः पादैः पिबति पादपः}


\twolineshloka
{सुखदुःखयोश्च ग्रहणाच्छिन्नस्य च विरोहणात्}
{जीवं पश्यामि वृक्षाणामचैतन्यं न विद्यते}


\twolineshloka
{तेन तज्जलमादत्तं जरयत्यग्निमारुतौ}
{आहारपरिणामाच्च स्नेहो वृद्धिश्च जायते}


\twolineshloka
{जङ्गमानां च सर्वेषां शरीरे पञ्च धातवः}
{प्रत्येकशः प्रभिद्यन्ते यैः शरीरं विचेष्टते}


\twolineshloka
{त्वक्च मांसं तथाऽस्थीनि मज्जा स्नायुश्च पञ्चमम्}
{इत्येतदिह संघातं शरीरे पृथिवीमयम्}


\twolineshloka
{तेजो ह्यग्निस्तथा क्रोधश्चक्षुरूष्मा तथैव च}
{अग्निर्जरयते यच्च पञ्चाग्नेयाः शरीरिणः}


\twolineshloka
{श्रोत्रं घ्राणं तथाऽऽस्यं च हृदयं कोष्ठमेव च}
{आकाशात्प्राणिनामेते शरीरे पञ्च धातवः}


\twolineshloka
{श्लेष्मा पित्तमथ स्वेदो वसा शोणितमेव च}
{इत्यापः पञ्चधा देहे भवन्ति प्राणिनां सदा}


\twolineshloka
{प्राणात्प्राणयते प्राणी व्यानाद्व्यायच्छते तथा}
{गच्छत्यपाने वाक्चैव समानने समः स्थितः}


\twolineshloka
{उदानादुच्छ्वसिति च प्रतिभेदाच्च भाषते}
{इत्येते वायवः पञ्च चेष्टयन्तीह देहिनम्}


\threelineshloka
{भूमेर्गन्धगुणान्वेत्ति रसं चाद्भ्यः शरीरवान्}
{ज्योतेः पश्यति रूपाणि स्पर्शं वेत्ति च वायुतः}
{`शब्दं शृणोति च तदैवाकाशात्तु शरीरवान्}


\twolineshloka
{गन्धः स्पर्शो रसो रूपं शब्दश्चात्र गुणाः स्मृताः}
{तस्य गन्धस्य वक्ष्यामि विस्तराभिहितान्गुणान्}


\threelineshloka
{इष्टश्चानिष्टगन्धश्च मधुरः कटुरेव च}
{निर्हारी संहतः स्निग्धो रूक्षो विशद एव च}
{एवं नवविधो ज्ञेयः पार्थिवो गन्धविस्तरः}


\twolineshloka
{ज्योतिः पश्यति चक्षुर्भ्यां स्पर्शं वेत्ति च वायुना}
{शब्दः स्पर्शश्च रूपं च रसश्चापि गुणाः स्मृताः}


\twolineshloka
{रसज्ञानं तु वक्ष्यामि तन्मे निगदतः शृणु}
{रसो बहुविधः प्रोक्तः सूरिभिः प्रथितात्मभिः}


\twolineshloka
{मधुरो लवणस्तिक्तः कषायोऽम्लः कटुस्तथा}
{एवं षङ्किधविस्तारो रसो वारिमयः स्मृतः}


\threelineshloka
{शब्दः स्पर्शश्च रूपं च त्रिगुणं ज्योतिरुच्यते}
{ज्योतिः पश्यति रूपाणि रूपं च बहुधा स्मृतम्}
{ह्रस्वो दीर्घस्तथा स्थूलश्चतुरस्रोऽणुवृत्तवान्}


\threelineshloka
{शुक्लः कृष्णस्तथा रक्तः पीतो नीलारुणस्तथा}
{कठिनश्चिक्कणः श्लक्ष्णः पिच्छिलो मृदुदारुणः}
{एवं द्वादशविस्तारो ज्योतीरूपगुणः स्मृतः}


\twolineshloka
{शब्दस्पर्शौ च विज्ञेयौ द्विगुणो वायुरित्युत}
{वायव्यस्तु गुणः स्पर्शः स्पर्शश्च बहुधा स्मृतः}


\threelineshloka
{उष्णः शीतः सुखो दुःखः स्निग्धो विशद एव च}
{तथा खरो मृदू रूक्षो लघुर्गुरुतरोऽपि च}
{एवं द्वादशधा स्पर्शो व्याव्यो गुण उच्यते}


\twolineshloka
{तत्रैकगुणमाकाशं शब्द इत्येव तत्स्मृतम्}
{तस्य शब्दस्य वक्ष्यामि विस्तरं विविधात्मकम्}


\twolineshloka
{षड््ज ऋषभगान्धारौ मध्यमो धैवतस्तथा}
{पञ्चमश्चापि विज्ञेयस्तथा चापि निषादवान्}


\twolineshloka
{एष सप्तविधः प्रोक्तः शब्द आकाशसंभवः}
{त्र्यैस्वर्येण तु सर्वत्र स्थितोऽपि पटहादिषु}


\threelineshloka
{[मृदङ्गभेरीशङ्खानां स्तनयित्नो रथस्य च}
{यः कश्चिच्छ्रूयते शब्दः प्राणिनो प्राणिनोऽपि वा}
{एतेषामेव सर्वेषां विषये संप्रकीर्तितः}


\twolineshloka
{एवं बहुविधाकारः शब्द आकाशसंभवः}
{आकाशजं शब्दमाहुरेभिर्वायुगुणैः सह ॥]}


\twolineshloka
{अव्याहतैश्चेतयने न वेत्ति विषमस्थितैः}
{आप्याय्यन्ते च ते नित्यं धातवस्तैस्तु धातुभिः}


\twolineshloka
{आपोऽग्निर्मारुतश्चैव नित्यं जाग्रति देहिषु}
{मूलमेते शरीरस्य व्याप्य प्राणानिह स्थिता}


\chapter{अध्यायः १८३}
\twolineshloka
{भरद्वाज उवाच}
{}


\threelineshloka
{पार्थिवं धातुमाश्रित्य शारीरोऽग्निः कथं भवेत्}
{अवकाशविशेषेण कथं वर्तयतेऽनिलः ॥भृगुरुवाच}
{}


\twolineshloka
{वायोर्गतिमहं ब्रह्मन्कीर्तयिष्यामि तेऽनघ}
{प्राणिनामनिलो देहान्यथा चेष्टयते बली}


\twolineshloka
{श्रितो मूर्धानमग्निस्तु शरीरं परिपालयन्}
{प्राणो मूर्धनि चाग्नौ च वर्तमानो विचेष्टने}


\twolineshloka
{स जन्तुः सर्वभूतात्मा पुरुषः स सनातनः}
{मनो बुद्धिरहंकारो भूतानि विषयाश्च सः}


\twolineshloka
{एवं त्विह स सर्वत्र प्राणेन परिपाल्यते}
{कोष्ठतस्तु समानेन स्वां स्वां गतिमुपाश्रितः}


\twolineshloka
{वस्तिमूलं गुदं चैव पावकं समुपाश्रितः}
{वहन्मूत्रं पुरीषं चाप्यपानः परिवर्तते}


\twolineshloka
{प्रयत्ने कर्मणि बले य एकस्त्रिषु वर्तते}
{उदान इति तं प्राहुरध्यात्मकुशला जनाः}


\twolineshloka
{संधिष्वपि च सर्वेषु सन्निविष्टस्तथाऽनिलः}
{शरीरेषु मनुष्याणां व्यान इत्युपदिश्यते}


\twolineshloka
{धातुष्वग्निस्तु विततः समानोऽग्निः समीरितः}
{रसान्धातूंश्च दोषांश्च वर्तयन्नवतिष्ठते}


\twolineshloka
{अपानप्राणयोर्मध्ये प्राणापानसमाहितः}
{समन्वितस्त्वधिष्ठानं सम्यक्पचति पावकः}


\twolineshloka
{आस्यं हि पायुसंयुक्तमन्ते स्याद्गुदसंज्ञितम्}
{स्रोतस्तस्मात्प्रजायन्ते सर्वस्रोतांसि देहिनाम्}


\twolineshloka
{प्राणानां सन्निपाताच्च सन्निपातः प्रजायते}
{ऊष्मा सोग्निरिति ज्ञेयो योऽन्नं पचति देहिनाम्}


\twolineshloka
{अग्निवेगवहः प्राणो गुदान्ते प्रतिहन्यते}
{स ऊर्ध्वमागम्य पुनः समुत्क्षिपति पावकम्}


\twolineshloka
{पक्वाशयस्त्वधो नाभ्या ऊर्ध्वमामाशयः स्मृतः}
{नाभिमध्ये शरीरस्य सर्वे प्राणाः समाश्रिताः}


\twolineshloka
{प्रसृता हृदयात्सर्वास्तिर्यगूर्ध्वमधस्तथा}
{वहन्त्यन्नरसान्नाड्यो दशप्राणप्रचोदिताः}


\twolineshloka
{एष मार्गोऽथ योगानां येन गच्छन्ति तत्पदम्}
{जितक्लमासना धीरा सूर्धन्यात्मानमादधन्}


\threelineshloka
{एवं सर्वेषु विहितः प्राणापानेषु देहिनाम्}
{तस्मिन्योऽवस्थितो नित्यमग्निः स्थाल्यामिवाहितः}
{}


\chapter{अध्यायः १८४}
\twolineshloka
{भरद्वाज उवाच}
{}


\twolineshloka
{यदि प्राणयते वायुर्वायुरेव विचेष्टते}
{श्वसित्याभाषते चैव तस्माज्जीवो निरर्थकः}


\twolineshloka
{यदूष्मभाव आग्नेयो वह्निना पच्यते यदि}
{अग्निर्जरयते चैतत्तस्माज्जीवो निरर्थकः}


\twolineshloka
{जन्तोः प्रमीयमाणस्य जीवो नैवोपलभ्यते}
{वायुरेव जहात्येनमूष्मभावश्च नश्यति}


\twolineshloka
{यदि वायुमयो जीवः संश्लेषो यदि वायुना}
{वायुमण्डलवद्दृश्येद्गच्छन्सह मरुद्गणैः}


\twolineshloka
{श्लेष्मं वा यदि वा जीवः सह तेन प्रणश्यति}
{महार्णववियुक्तत्वादन्यत्सलिलभाजनम्}


\twolineshloka
{यत्क्षिपेत्सलिलं कूपे प्रदीपं वा हुताशने}
{तन्नश्यत्युभयं तद्वज्जीवो वातानलात्मकः}


\twolineshloka
{पञ्च साधारणो ह्यस्मिञ्शरीरे जीवितं कुतः}
{तेषामन्यतरत्यागाच्चतुर्णां नास्ति संग्रहः}


\twolineshloka
{नश्यन्त्यापो ह्यनाधाराद्वायुरुच्छ्वासनिग्रहात्}
{नश्यते कोष्ठभेदात्खमग्निर्नश्यत्यभोजनात्}


\twolineshloka
{व्याधिप्राणपरिक्लेशैर्मेदिनी चैव शीर्यते}
{पीडितेऽन्यतमे ह्येषां संघातो याति पञ्चताम्}


\twolineshloka
{तस्मिन्पञ्चत्वमापन्ने जीवः किमनुधावति}
{किं वेदयति वा जीवः किं शृणोति ब्रवीति च}


\twolineshloka
{एषा गौः परलोकस्थं तारयिष्यति मामिति}
{यो दत्त्वा म्रियते जन्तुः सा गौः कं तारयिष्यति}


\twolineshloka
{गौश्च प्रतिग्रहीता च दाता चैव समं यदा}
{इहैव विलयं यान्ति कुतस्तेषां समागमः}


\twolineshloka
{विहगैरुपभुक्तस्य शैलाग्नात्पतितस्य च}
{अग्निना चोपयुक्तस्य कुतः संजीवनं पुनः}


\twolineshloka
{छिन्नस्य यदि वृक्षस्य न मूलं प्रतिरोहति}
{बीजान्यस्य प्ररोहन्ति मृतः क्व पुनरेष्यति}


\twolineshloka
{बीजमात्रं पुरा सृष्टं यदेतत्परिवर्तते}
{मृतामृताः प्रणश्यन्ति बीजाद्वीजं प्रवर्तते}


\chapter{अध्यायः १८५}
\twolineshloka
{भृगुरुवाच}
{}


\twolineshloka
{न प्रणाशोऽस्ति जीवानां दत्तस्य च कृतस्य च}
{याति देहान्तरं प्राणी शरीरं तु विशीर्यते}


\threelineshloka
{न शरीराश्रितो जीवस्तस्मिन्नष्टे प्रणश्यति}
{यथा समित्सु दग्धासु न प्रणश्यति पावकः ॥भरद्वाज उवाच}
{}


\twolineshloka
{अग्नेर्यथा समिद्धस्य यदि नाशो न विद्यते}
{इन्धनस्योपयोगान्ते स चाग्निर्नोपलभ्यते}


\threelineshloka
{नश्यतीत्येव जानामि शान्तमग्निमनिन्धनम्}
{मतिर्यस्य प्रमाणं वा संस्थानं वा न दृश्यते ॥भृगुरुवाच}
{}


\threelineshloka
{`जीवस्य चेन्धनाग्नेश्च सदा नाशो न विद्यते}
{'समिधामुपयोगान्ते सन्नेवाग्निर्न दृश्यते}
{आकाशानुगतत्वाद्धि दुर्ग्रहः स निराश्रयः}


\twolineshloka
{तथा शरीरसंत्यागे जीवो ह्याकाशमाश्रितः}
{न गृह्यते तु सूक्ष्मत्वाद्यथा ज्योतिरनिन्धनम्}


\twolineshloka
{प्राणान्धारयते योऽग्निः स जीव उपधार्यताम्}
{वायुसंधारणो ह्यग्निर्नश्यत्युच्छ्वासनिग्रहात्}


\twolineshloka
{तस्मिन्नष्टे शरीराग्नौ शरीरं तदचेतनम्}
{पतितं याति भूमित्वमयनं तस्य हि क्षितिः}


\threelineshloka
{जङ्गमानां हि सर्वेषां स्थावराणां तथैव च}
{आकाशं पवनोऽन्वेति ज्योतिस्तमनुगच्छति}
{तेषां त्रयाणामेकत्वं द्वयं भूमौ प्रतिष्ठितम्}


\threelineshloka
{यत्र खं तत्र पवनस्तत्राग्निर्यत्र मारुतः}
{अमूर्तयस्ते विज्ञेया आपो मूर्तास्तथा क्षितिः ॥भरद्वाज उवाच}
{}


\twolineshloka
{यद्यग्निमारुतौ भूमिः खमापश्च शरीरिषु}
{जीवः किंलक्षणस्तत्रेत्येतदाचक्ष्व मेऽनघ}


\twolineshloka
{पञ्चात्मके पञ्चरतौ पञ्चविज्ञानसंयुते}
{शरीरे प्राणिनां जीवं वेत्तुमिच्छामि यादृशम्}


\twolineshloka
{मांसशोणितसंघाते मेदः स्नाय्वस्थिसंचये}
{भिद्यमाने शरीरे तु जीवो नैवोपलभ्यते}


\twolineshloka
{यद्यजीवं शरीरं तु पञ्चभूतसमन्वितम्}
{शारीरे मानसे दुःखे कस्तां वेदयते रुजम्}


\twolineshloka
{शृणोति कथितं जीवः कर्णाभ्यां न शृणोति तत्}
{महर्षे मनसि व्यग्रे तस्माज्जीवो निरर्थकः}


\twolineshloka
{सर्वं पश्यति यद्दृश्यं मनोयुक्तेन चक्षुषा}
{मनसि व्याकुले तस्मिन्पश्यन्नपि न पश्यति}


\twolineshloka
{न पश्यति न चाघ्राति न शृणोति न भाषते}
{न च स्पर्शरसौ वेत्ति निद्रावशगतः पुनः}


\threelineshloka
{हृष्यति क्रुध्यते कोऽत्र शोचत्युद्विजते च कः}
{इच्छति ध्यायति द्वेष्टि वाचमीरयते च कः ॥भृगुरुवाच}
{}


\twolineshloka
{न पञ्चसाधारणमत्र किंचिच्छरीरमेकी वहतेऽन्तरात्मा}
{स वेत्ति गन्धांश्च रसाञ्श्रुतीश्चस्पर्शं च रूपं च गुणाश्च येऽन्ये}


\twolineshloka
{पञ्चात्मके पञ्चगुणप्रदर्शीस सर्वगात्रानुगतोऽन्तरात्मा}
{स वेत्ति दुःखानि सुखानि चात्रतद्विप्रयोगात्तु न वेत्ति देही}


\twolineshloka
{यदा न रूपं न स्पर्शो नोष्मभावश्च पञ्चके}
{तदा शान्ते शरीराग्नौ देहं त्यक्त्वा न नश्यति}


\twolineshloka
{अम्मयं सर्वमेवेदमापो मूर्तिः शरीरिणाम्}
{तत्रात्मा मानसो ब्रह्मा सर्व भूतेषु लोककृत्}


\twolineshloka
{[आत्मा क्षेत्रज्ञ इत्युक्तः संयुक्तः प्राकृतैर्गुणैः}
{तैरेव तु विनिर्मुक्तः परमात्मेत्युदाहृतः ॥]}


\twolineshloka
{आत्मानं तं विजानीहि सर्वलोकविपाचकम्}
{स तस्मिन्संश्रितो देहे ह्यब्बिन्दुरिव पुष्करे}


\twolineshloka
{क्षेत्रज्ञं तं विजानीहि नित्यं लोकहितात्मकम्}
{तमो रजश्च सत्त्वं च विद्धि जीवगुणानिमान्}


\twolineshloka
{सचेतनं जीवगुणं वदन्तिस चेष्टते चेष्टयते च सर्वम्}
{ततः परं क्षेत्रविदो वदन्तिप्रावर्तयद्यो भुवनानि सप्त}


\twolineshloka
{न जीवनाशोऽस्ति हि देहभेदेमिथ्यैतदाहुर्मुत इत्यबुद्धाः}
{जीवस्तु देहान्तरितः प्रयातिदशार्धतैवास्य शरीरभेदः}


\twolineshloka
{एवं सर्वेषु भूतेषु गूढश्चरति संवृतः}
{दृश्यते त्वग्र्यया बुद्ध्या सूक्ष्मया तत्त्वदर्शिभिः}


\twolineshloka
{तं पूर्वापररात्रेषु युञ्जानः सततं बुधः}
{लध्वाहारो विशुद्धात्मा पश्यत्यात्मानंमात्मनि}


\twolineshloka
{चित्तस्य हि प्रसादेन हित्वा कर्म शुभाशुभम्}
{प्रसन्नात्माऽत्मनि स्थित्वा सुखमव्ययमश्नुते}


\twolineshloka
{मानसोऽग्निः शरीरेषु जीव इत्यभिधीयते}
{सृष्टिः प्रजापतेरेषा भूताध्यात्मविनिश्चया}


\chapter{अध्यायः १८६}
\twolineshloka
{भृगुरुवाच}
{}


\twolineshloka
{असृजद्ब्राह्मणानेव पूर्वं ब्रह्मा प्रजापतीन्}
{आत्मतेजोभिनिर्वृत्तान्भास्कराग्निसमप्रभान्}


\twolineshloka
{ततः सत्यं च धर्मं च तपो ब्रह्म च शाश्वतम्}
{आचारं चैव शौचं च सर्गादौ विदधे प्रभुः}


\twolineshloka
{देवदानवगन्वर्गा दैत्यासुरमहोरगाः}
{यक्षराक्षसनागाश्च पिशाचा मनुजास्तथा}


\twolineshloka
{ब्राह्मणाः क्षत्रिया वैश्याः शूद्राश्च द्विजसत्तम}
{ये चान्ये भूतसङ्घानां संघातास्तांश्च निर्ममे}


\threelineshloka
{ब्राह्मणानां सितो वर्णः क्षत्रियाणां तु लोहितः}
{वैश्यानां पीतको वर्णः शूद्राणामसितस्तथा ॥भरद्वाज उवाच}
{}


\twolineshloka
{चातुर्वर्ण्यस्य वर्णेन यदि वर्णो विभज्यते}
{सर्वेषां खलु वर्णानां दृश्यते वर्णसंकरः}


\twolineshloka
{कामः क्रोधो भयं लोभः शोकश्चिन्ता क्षुधा श्रमः}
{सर्वेषां नः प्रभवति कस्माद्वर्णो विभज्यते}


\twolineshloka
{स्वेदमूत्रपुरीषाणि श्लेष्मा पित्तं सशोणितम्}
{तनुः क्षरति सर्वेषां कस्माद्वर्णो विभज्यते}


\threelineshloka
{जङ्गमानामसङ्ख्येयाः स्थावराणां च जातयः}
{तेषां विविधवर्णानां कुतो वर्णविनिश्चयः ॥भृगुरुवाच}
{}


\twolineshloka
{न विशेषोऽस्ति वर्णानां सर्वं ब्राह्ममिदं जगत्}
{ब्राह्मणाः पूर्वसृष्टा हि कर्मभिर्वर्णतां गताः}


\twolineshloka
{कामभोगप्रियास्तीक्ष्णाः क्रोधनाः प्रियसाहसाः}
{त्यक्तस्वधर्मा रक्ताङ्गास्ते द्विजाः क्षत्रतां गताः}


\twolineshloka
{गोषु वृत्तिं समाधाय पीताः कृष्युपजीविनः}
{स्वधर्मान्नानुतिष्ठन्ति ते द्विजा वैश्यतां गताः}


\twolineshloka
{हिंसानृतप्रिया लुब्धाः सर्वकर्मोपजीविनः}
{कृष्णाः शौचपरिभ्रष्टास्ते द्विजाः शूद्रता गताः}


\twolineshloka
{इत्येतैः कर्मभिर्व्यस्ता द्विजा वर्णान्तरं गताः}
{धर्मो यज्ञक्रिया चैषां नित्यं न प्रतिषिध्यते}


\twolineshloka
{इत्येते चतुरो वर्णा येषां ब्राह्मी सरस्वती}
{विहिता ब्रह्मणा पूर्वं लोभात्त्वज्ञानतां गताः}


\twolineshloka
{ब्राह्मणा ब्रह्मतन्त्रस्थास्तपस्तेषां न नश्यति}
{ब्रह्म धारयतां नित्यं व्रतानि नियमांस्तथा}


\twolineshloka
{ब्रह्म वैव परं सृष्टं ये तु जानन्ति ते द्विजाः}
{तेषां बहुविधास्त्वन्ये तत्रतत्र द्विजातयः}


\twolineshloka
{पिशाचा राक्षसाः प्रेता विविधा म्लेच्छजातयः}
{प्रनष्टज्ञानविज्ञानाः स्वच्छन्दाचारचेष्टिताः}


\twolineshloka
{प्रजा ब्राह्मणसंस्काराः स्वकर्मकृतनिश्चयाः}
{ऋषिभिः स्वेन तपसा सृज्यन्ते चापरे परैः}


\twolineshloka
{आदिदेवसमुद्भूता ब्रह्ममूलाक्षयाव्यया}
{सा सृष्टिर्मानसी नाम धर्मतन्त्रपरायणा}


\chapter{अध्यायः १८७}
\twolineshloka
{भरद्वाज उवाच}
{}


\threelineshloka
{ब्राह्मणः केन भवति क्षत्रियो वा द्विजोत्तम}
{वैश्यः शूद्रश्च विप्रर्षे तद्ब्रूहि वदतां वर ॥भृगुरुवाच}
{}


\twolineshloka
{जातकर्मादिभिर्यस्तु संस्कारैः संस्कृतः शुचिः}
{वेदाध्ययनसंपन्नः षट््सु कर्मस्ववस्थितः}


\twolineshloka
{शौचाचारस्थितः सम्यग्विघसाशी गुरुप्रियः}
{नित्यव्रती सत्यपरः स वै ब्राह्मण उच्यते}


\twolineshloka
{सत्यं दानमथाद्रोह आनृशंस्यं क्षमा धृणा}
{तपश्च दृश्यते यत्र स ब्राह्मण इति स्मृतः}


\twolineshloka
{क्षत्रजं सेवते कर्म देवाध्ययनसंगतः}
{दानादानरतिर्यस्तु स वै क्षत्रिय उच्यते}


\twolineshloka
{कृपिगोरक्षवाणिज्यं यो विशत्यनिशं शुचिः}
{वेदाध्ययनसंपन्नः स वैश्य इति संज्ञितः}


\twolineshloka
{सर्वभक्षरतिर्नित्यं सर्वकर्मकरोऽशुचिः}
{त्यक्तवेदस्त्वनाचारः स वै शूद्र इति स्मृतः}


\twolineshloka
{शूद्रे चैतद्भवेल्लक्ष्यं द्विजे तच्च न विद्यते}
{न वै शूद्रो भवेच्छ्रद्रो ब्राह्मणो न च ब्राह्मणः}


\twolineshloka
{सर्वोपायैस्तु लोभस्य क्रोधस्य च विनिग्रहः}
{एतत्पवित्रं ज्ञातव्यं तथा चैवात्मसंयमः}


% Check verse!
वार्यौ सर्वात्मना तौ हि श्रेयोघातार्थमुच्छ्रितौ
\twolineshloka
{नित्यं क्रोधाच्छ्रियं रक्षेत्तपो रक्षेच्च मत्सरात्}
{विद्यां मानापमानाभ्यामात्मानं तु प्रमादतः}


\twolineshloka
{यस्य सर्वे समारम्भा निराशाबन्धना द्विज}
{त्यागे यस्य हुतं सर्वं स त्यागी च स बुद्धिमान्}


\threelineshloka
{अहिंस्रः सर्वभूतानां मैत्रायणगतिश्चरेत्}
{परिग्रहान्परित्यज्य भवेद्बुद्ध्या जितेन्द्रियः}
{अचलं स्थानमातिष्ठेदिह चामुत्र चोभयोः}


\twolineshloka
{तपोनित्येन दान्तेन मुनिना संयतात्मना}
{अजितं जेतुकामेन भाव्यं सङ्गेष्वसङ्गिना}


\twolineshloka
{इन्द्रियैर्गृह्यते यद्यत्तत्तद्व्यक्तमिति स्थितिः}
{अव्यक्तमिति विज्ञेयं लिङ्गग्राह्यमतीन्द्रियम्}


\twolineshloka
{अविस्रम्भे न गन्तव्यं विस्रम्भे धारयेन्मनः}
{मनः प्राणे निगृह्णीयात्प्राणं ब्रह्मणि धारयेत्}


\twolineshloka
{निर्वेदादेव निर्वायान्न च किंचिद्विचिन्तयेत्}
{सुखं वै ब्राह्मणो ब्रह्म स वै तेनाधिगच्छति}


\twolineshloka
{शौचेन सततं युक्तः सदाचारसमन्वितः}
{सानुक्रोशश्च भूतेषु तद्द्विजातिषु लक्षणम्}


\chapter{अध्यायः १८८}
\twolineshloka
{भृगुरुवाच}
{}


\twolineshloka
{सत्यं ब्रह्म तपः सत्यं सत्यं सृजति च प्रजाः}
{सत्येन धार्यते लोकः स्वर्गं सत्येन गच्छति}


\twolineshloka
{अनृतं तमसो रूपं तमसा नीयते ह्यधः}
{तमोग्रस्ता न पश्यन्ति प्रकाशं तमसाऽऽवृतम्}


\twolineshloka
{स्वर्गः प्रकाश इत्याहुर्नरकं तम एव च}
{सत्यानृतात्तदुभयं प्राप्यते जगतीचरैः}


\twolineshloka
{तत्र त्वेवंविधा लोके वृत्तिः सत्यानृते भवेत्}
{धर्माधर्मौ प्रकाशश्च तमो दुःखं सुखं तथा}


\threelineshloka
{तत्र यत्सत्यं स धर्मो यो धर्मः स प्रकाशो यःप्रकाशस्तत्सुखमिति}
{तत्र यदनृतं सोऽधर्मो योऽधर्मस्तत्तमोयत्तमस्तद्दुःखमिति ॥अत्रोच्यते}
{}


\twolineshloka
{शारीरैर्मानसैर्दुःखैः सुखैश्चाप्यसुखोदयैः}
{लोकसृष्टिं प्रपश्यन्तो न मुह्यन्ति विचक्षणाः}


\twolineshloka
{तत्र दुःखविमोक्षार्थं प्रयतेत विचक्षणः}
{सुखं ह्यनित्यं भूतानामिह लोके परत्र च}


\twolineshloka
{राहुग्रस्तस्य सोमस्य यथा ज्योत्स्ना न भासते}
{तथा तमोभिभूतानां भूतानां भ्रश्यते सुखम्}


\threelineshloka
{तत्खलु द्विविधं सुखमुच्यते शारीरं मानसं च}
{इह खल्वमुष्मिंश्चलोके सर्वारम्भप्रवृत्तयः सुखार्थमभिधीयन्ते न ह्यतः परं त्रिवर्गफलंविशिष्टतरमस्ति स एष काम्यो गुणविशेषोधर्मार्थगुणारम्भस्तद्धेतुरस्योत्पत्तिः सुखप्रयोजनार्थ आरम्भः ॥भरद्वाज उवाच}
{}


% Check verse!
यदेतद्भवताऽभिहितं सुखानां परमार्थस्थितिरिति तन्न गृह्णीमो नह्येषामृषीणां तपसि स्थितानामप्राप्य एव काम्यो गुणविशेषो नचैनमभिलषन्ति च तपसि श्रूयते त्रिलोकेकृद्ब्रह्मा प्रभुरेकाकी तिष्ठति

ब्रह्मचारी न कामसुखेष्वात्मानमवदधाति

अपिच भगवान्विश्वेश्वर[उमापतिः] काममभिवर्तमानमनङ्गत्वेन नाशमनयत्

तस्माद्ब्रूमो न तुमहात्मभिः प्रतिगृहीतोऽयमर्थो तत्वेष तावद्विशिष्टो गुणगण इति

नैतद्भगवान्प्रत्येति भगवतोक्तं सुखानां परमार्थस्थितिरिति लोकप्रवादोहि द्विविधः फलोदयः सुकृतात्सुखमवाप्यते दुष्कृताद्दुःखमिति ॥भृगुरुवाच


% Check verse!
अत्रोच्यते

अनृतात्खलु तमः प्रादुर्भूतं ततस्तमोग्रस्ताअधर्ममेवानुवर्तन्ते न धर्मम्

क्रोधलोभमोहमदादिभिरवच्छन्ना नखल्वस्मिँल्लोके नामुत्र सुखमाप्नुवन्ति

विविधव्याधिव्रणरुजोपतापैरवकीर्यन्ते

वधबन्धननिरोधपरिक्लेशादिभिश्चक्षुत्पिपासाश्रमकृतैरुपतापैरुपतप्यन्ते

चण्डवातात्युष्णातिशीतकृतैश्चप्रतिभयैः शारीरैर्दुःखैरुपतप्यन्ते

बन्धुधनविननाशविप्रयोगकृतैश्चमानसैः शोकैरभिभूयन्ते जरामृत्युकृतैश्चान्यैरिति
\twolineshloka
{यदैतैः शारीरैर्मानसैर्दुःखैर्न स्पृश्यते तत्सुखं विद्यात्}
{न चैते दोषाः स्वर्गे प्रादुर्भवन्ति तत्र खलु भवन्ति}


\twolineshloka
{सुसुखः पवनः स्वर्गे गन्धश्च सुरभिस्तथा}
{क्षुत्पिपासाश्रमो नास्ति न जरा न च पातकम्}


\twolineshloka
{नित्यमेव सुखं स्वर्गे सुखं दुःखमिहोभयम्}
{नरके दुःखमेवाहुः सुखं तु परमं पदम्}


\twolineshloka
{पृथिवी सर्वभूतानां जनित्री तद्विधाः स्त्रियः}
{पुमान्प्रजापतिस्तत्र शुक्रं तेजोमयं विदुः}


\twolineshloka
{इत्येतल्लोकनिर्माणं ब्रह्मणा विहितं पुरा}
{प्रजा विपरिवर्न्तते स्वैः स्वैः कर्मभिरावृताः}


\chapter{अध्यायः १८९}
\twolineshloka
{भरद्वाज उवाच}
{}


\threelineshloka
{दानस्य किं फलं प्रोक्तं धर्मस्य चरितस्य च}
{तपसश्च सुतप्तस्य स्वाध्यायस्य हुतस्य वा ॥भृगुरुवाच}
{}


\twolineshloka
{हुतेन शाम्यते पापं स्वाध्यायैः शान्तिरुत्तमा}
{दानेन भोग इत्याहुस्तपसा सर्वमाप्नुयात्}


\twolineshloka
{दानं तु द्विविधं प्राहुः परत्रार्थमिहैव च}
{सद्भ्यो यद्दीयते किंचित्तत्परत्रोपतिष्ठते}


\threelineshloka
{असद्भ्यो दीयते यत्तु तद्दानमिह भुज्यते}
{यादृशं दीयते दानं तादृशं फलमुच्यते ॥भरद्वाज उवाच}
{}


\threelineshloka
{किं कस्य धर्माचरणं किं वा धर्मस्य लक्षणम्}
{धर्मः कतिविधो वाऽपि तद्भवान्वक्तुमर्हति ॥भृगुरुवाच}
{}


\threelineshloka
{स्वधर्माचरणे युक्ता ये भवन्ति मनीषिणः}
{तेषां स्वर्गफलावाप्तिर्योऽन्यथा स विमुह्यते ॥भरद्वाज उवाच}
{}


\threelineshloka
{यदेतच्चातुराश्रम्यं ब्रह्मर्षिविहितं पुरा}
{तस्य स्वंस्वं समाचारं यथावद्वक्तुमर्हसि ॥भृगुरुवाच}
{}


\fourlineindentedshloka
{पूर्वमेव भगवता ब्रह्मणा लोकहितमनुतिष्ठताधर्मसंरक्षणार्थमाश्रमाश्चत्वारोऽभिनिर्दिष्टाः}
{तत्र गुरुकुलवासमेवप्रथममाश्रममुदाहरन्ति}
{सम्यग्यत्र शौचसंस्कारनियमव्रतविनियतात्माउभेसन्ध्ये भास्कराग्निदैवतान्युपस्थाय विहाय निद्रालस्येगुरोरभिवादननवेदाभ्यासश्रवणपवित्रीकृतान्तरात्मा [त्रिषवणमुपस्पृश्यब्रह्मचर्याग्निपरिचरणगुरुशुश्रूषानित्यभिक्ष्यादिसर्वनिवेदितान्तरात्मा]गुरुवचनिर्देशानुष्ठानाप्रतिकूलो गुरुप्रसादलब्धस्वाध्यायतत्परःस्यात् ॥भवति चात्र श्लोकः}
{}


\twolineshloka
{गुरुं यस्तु समाराध्य द्विजो वेदमवाप्नुयात्}
{तस्य स्वर्गफलावाप्तिः शुध्यते चास्य मानसमिति}


% Check verse!
गार्हस्थ्यं खलु द्वितीयमाश्रमं वदन्ति

तस्य समुदाचारलक्षणंसर्वमनुव्याख्यास्यामः

समावृत्तानां सदाचाराणां सहधर्मचर्याफलार्थिनांगृहाश्रमो विधीयते

धर्मार्थकामावाप्त्यर्थंत्रिवर्गसाधनमपेक्ष्यागर्हितेन कर्मणा धनान्यादायस्वाध्यायोपलब्धप्रकर्षेण वा ब्रह्मर्षिनिर्मितेन वा

हव्यकव्यनियमाभ्यां दैवतपूजासमाधिप्रसादविध्युपलब्धेन धनेन गृहस्थोगार्हस्थ्यं वर्तयेत्

तद्धि सर्वाश्रमाणां मूलमुदाहरन्ति

गुरुकुलनिवासिनः परिव्राजका ये चान्येसंकल्पितव्रतनियमधर्मानुष्ठायिनस्तेपामष्यत एव भिक्षाबलिसंविभागाःप्रवर्तन्ते
\twolineshloka
{वानप्रस्थानां च द्रव्योपस्कार इति प्रायशः खल्वेते साधवःसाधुपथ्याशिनः स्वाध्यायप्रसङ्गिनस्तीर्थाभिगमनदेशदर्शनार्थं पृथिवींपर्यटन्ति तेषांप्रत्युत्थानाभिगमनाभिवादनानसूयवाक्प्रदानसुखशक्त्यासनसुखशयनाभ्यवहारसत्क्रियाचेति ॥भवति चात्र श्लोकः}
{}


\twolineshloka
{अतिथिर्यस्य भग्नाशो गृहात्प्रतिनिवर्तते}
{स तस्य दुष्कृतं दत्त्वा पुण्यमादाय गच्छति}


\twolineshloka
{अपि चात्र यज्ञक्रियाभिर्देवताः प्रीयन्ते निवापेन पितरोवेदविद्याभ्यासश्रवणधारणेन ऋषय अपत्योत्पादनेन प्रजापतिरिति ॥श्लोकौ चात्र भवतः}
{}


\twolineshloka
{वत्सलाः सर्वभूतानां वाच्याः श्रोत्रसुखा गिरः}
{परिवादापवादौ च पारुष्यं चात्र गर्हितम्}


\twolineshloka
{अवज्ञानमहंकारो दम्भश्चैव विगर्हितः}
{अहिंसा सत्यमक्रोधः सर्वाश्रमगतं तपः}


\twolineshloka
{अपि चात्रमाल्याभरणवस्त्राभ्यङ्गनित्योपभोगनृत्तगीतवादित्रश्रुतिसुखनयनाभिरामदर्शनानांप्राप्तिर्भक्ष्यभोज्यलेह्यपेयचोष्याणामभ्यवहार्याणां विविधानामुपभोगःस्वविहारसंतोषः कामसुखावाप्तिरिति ॥श्लोकौ चात्र भवतः}
{}


\twolineshloka
{त्रिवर्गगुणनिर्वृत्तिर्यस्य नित्यं गृहाश्रमे}
{स सुखान्यनुभूयेह शिष्टानां गतिमाप्नुयात्}


\twolineshloka
{उञ्छवृत्तिर्गृहस्थो यः स्वधर्माचरणे रतः}
{त्यक्तकामसुखारम्भः स्वर्गस्तस्य न दुर्लभः}


\chapter{अध्यायः १९०}
\twolineshloka
{भृगुरुवाच}
{}


\twolineshloka
{वानप्रस्थाः खल्वृषिधर्ममनुवर्तन्ते पुण्यानि तीर्थानिनदीप्रस्रवणान्युचरन्ति सुविभक्तेष्यरण्येषुमृगमहिषवराहशार्दूलसृमरगजाकीर्णेषु तपस्यन्तोऽनुसंचरन्तित्यक्तग्राम्यवस्त्राभ्यवहारोपभोगावन्यौषधिफलमूलपर्णपरिमितविचित्रनियताहाराः स्थानासनिनोभूमिपाषाणसिकताशर्करावालुकाभस्मशायिनः काशकुशचर्मवल्कलसंवृताङ्गाःकेशश्मश्रुनखरोमधारिणो नियतकालोपस्पर्शना अस्कन्नकालबहिहोमानुष्ठायिनःसमित्कुशकुसुमापहारार्चनसंमार्जनहोमान्तलब्धविश्रयाः शीतोष्ण[वर्ष]पवनविनिष्टप्तविभिन्नसर्वत्वचो विविधनियमयोगचर्यानुष्ठानहृत[परिशुष्क] मांसशोणितत्वगस्थिभूता धृतिपराःसत्वयोगाच्छरीराण्युद्वहन्ति ॥भवति चात्र श्लोकः}
{}


\twolineshloka
{यश्चैतां नियतश्चर्यां ब्रह्मर्षिविहितां चरेत्}
{स दहेदग्निवद्दोषाञ्जयेल्लोकांश्च दुर्जयान्}


\threelineshloka
{परिव्राजकानां पुनराचारः}
{तद्यथा विमुच्यधनकलत्रपरिबर्हणंसङ्गेष्वात्मनः स्नेहपाशानवधूय परिव्रजन्तिसमलोष्टाश्मकाञ्चनास्त्रिवर्गप्रवृत्तेष्वारम्भेष्वसक्तबुद्धयोऽरिमित्रोदासीनानांतुल्यदर्शनाः स्थावरजङ्गमानां जरायुजाण्डजस्वेदजोद्भिज्जानां भूतानांवाङ्भनः कर्मभिरभिद्रोहिणोऽनिकेताःपर्वतपुलिनवृक्षमूलदेवतायतनान्यननुचरन्तो वासार्थमुपेयुर्नगरं ग्रामंवा नगरे पञ्चरात्रिका ग्रामे चैकरात्रिकाः प्रविश्च च प्राणधारणार्थंद्विजातीनां भवनान्यसंकीर्णकर्मणामुपतिष्ठेयुःपात्रपतिताऽयाचितभैक्ष्याःकामक्रोधदर्पलोभमोहकार्पण्यदम्भपरिवादाभिमानहिंसानिवृत्ता इति ॥भवन्ति चात्र श्लोकाः}
{}


\twolineshloka
{अभयं सर्वभूतेभ्यो दत्त्वा यश्चरते मुनिः}
{न तस्य सर्वभूतेभ्यो भयमुत्पद्यते क्वचित्}


\twolineshloka
{कृत्वाऽग्निहोत्रं स्वशरीरसंस्थंशारीरमग्निं स्वमुखे जुहोति}
{यो भैक्षचर्योपगतैर्हविर्भिश्चिताग्निना प्राप्य स याति लोका}


\threelineshloka
{मोक्षाश्रमं यः कुरुते यथोक्तंशुचिः सुसंकल्पितबुद्धियुक्तः}
{अनिन्धनं ज्योतिरिव प्रशान्तंस ब्रह्मलोकं श्रयते द्विजातिः ॥भरद्वाज उवाच}
{}


\threelineshloka
{अस्माल्लोकात्परो लोकः श्रूयते नोपलभ्यते}
{तमहं ज्ञातुमिच्छामि तद्भवान्वक्तुमर्हति ॥भृगुरुवाच}
{}


\twolineshloka
{उत्तरे हिमवत्पार्श्वे पुण्ये सर्वगुणान्विते}
{पुण्यः क्षेम्यश्च काम्यश्च स परो लोक उच्यते}


\twolineshloka
{तत्र ह्यपापकर्माणः शुचयोऽत्यन्तनिर्मलाः}
{लोभमोहपरित्यक्ता मानवा निरुपद्रवाः}


\twolineshloka
{स स्वर्गसदृशो लोकस्तत्र ह्युक्ताः शुभा गुणाः}
{नात्र मृत्युः प्रभवति स्पृशन्ति व्याधयो न च}


\twolineshloka
{न लोभः परदारेषु स्वदारनिरतो जनः}
{न चान्योन्यवधस्तत्र द्रव्येषु च न विस्मयः}


\twolineshloka
{परोक्षधर्मो नैवास्ति संदेहो नापि जायते}
{कृतस्य तु फलं व्यक्तं प्रत्यक्षमुपलभ्यते}


\twolineshloka
{यानासनाशनोपेताः प्रासादभवनाश्रयाः}
{सर्वकामैर्वृताः केचिद्धेमाभरणभूषिताः}


\twolineshloka
{प्राणधारणमात्रं तु केषांचिदुपलभ्यते}
{श्रमेण महता केचित्कुर्वन्ति प्राणधारणम्}


\twolineshloka
{इह धर्मपराः केचित्केचिन्नैकृतिका नराः}
{सुखिताः दुःखिताः केचिन्निर्धना धनिनोऽपरे}


\twolineshloka
{इह श्रमो भयं मोहः क्षुधा निद्रा च जायते}
{लोभश्चार्थकृतो नॄणां येन मुह्यन्त्यपण्डिताः}


\twolineshloka
{इह अर्ता बहुविधा धर्माधर्मस्य कर्मणः}
{यस्तद्वेदोभयं प्राज्ञः पाप्मना न स लिप्यते}


\twolineshloka
{सोपधं कृतिः स्तेयं परिवादो ह्यसूयिता}
{परोपघातो हिंसा च पैशुन्यमनृतं तथा}


\twolineshloka
{एतानि सेवते यस्तु तपस्तस्य मितायते}
{यस्त्वेतान्नाचरेद्विद्वांस्तपस्तस्य प्रवर्धते}


\threelineshloka
{इह चिन्ता बहुविधा धर्माधर्मस्य कर्मणः}
{कर्मभूमिरियं लोके इह कृत्वा शुभाशुभम्}
{शुभैः शुभमवाप्नोति कर्ताऽशुभमथान्यथा}


\twolineshloka
{इह प्रजापतिः पूर्वं देवाः सर्षिगणास्तथा}
{इष्टेन तपसा पूता ब्रह्मलोकमुपाश्रिताः}


\twolineshloka
{उत्तरः पृथिवीभागः सर्वपुण्यतमः शुभः}
{इहत्यास्तत्र जायन्ते ये वै पुण्यकृतो जनाः}


\twolineshloka
{असत्कर्माणि कुर्वाणास्तिर्यग्योनिषु चापरे}
{क्षीणायुषस्तथा चान्ये नश्यन्ति पृथिवीतले}


\twolineshloka
{अन्योन्यभक्षणासक्ता लोभमोहसमन्विताः}
{इहैव परिवर्न्तते न ते यान्त्युत्तरां दिशम्}


\twolineshloka
{ये गुरून्पर्युपासन्ते नियता ब्रह्मचारिणः}
{पन्थानं सर्वलोकानां ते जानन्ति मनीषिणः}


\threelineshloka
{इत्युक्तोऽयं मया धर्मः संक्षेपाद्ब्रह्मनिर्मितः}
{धर्माधर्मौ हि लोकस्य यो वै वेत्ति स बुद्धिमान् ॥भीष्म उवाच}
{}


\twolineshloka
{इत्युक्तो भृगुणा राजन्भरद्वाजः प्रतापवान्}
{भृगुं परमधर्मात्मा विस्मितः प्रत्यपूजयत्}


\twolineshloka
{एष ते प्रभवो राजञ्जगतः संप्रकीर्तितः}
{निखिलेन महाप्राज्ञ किं भूयः श्रोतुमिच्छसि}


\chapter{अध्यायः १९१}
\twolineshloka
{युधिष्ठिर उवाच}
{}


\threelineshloka
{आचारस्य विधिं तात प्रोच्यमानं त्वयाऽनघ}
{श्रोतुमिच्छामि धर्मज्ञ सर्वज्ञो ह्यसि मे मतः ॥भीष्म उवाच}
{}


\twolineshloka
{दुराचारा दुर्विचेष्टा दुष्प्रज्ञाः प्रियसाहसाः}
{असन्तस्त्वभिविख्याताः सन्तश्चाचारलक्षणाः}


\threelineshloka
{पुरीषं यदि वा मूत्रं ये न कुवन्ति मानवाः}
{राजमार्गे गवां मध्ये धान्यमध्ये शिवालये}
{अग्न्यगारे तथा तीरे ये न कुर्वन्ति ते शुभाः}


\twolineshloka
{शौचमावश्यकं कृत्वा देवतानां च तर्पणम्}
{धर्ममाहुर्मनुष्याणामुपस्पृश्य नदीं तरेत्}


\twolineshloka
{सूर्यं सदोपतिष्ठेन न स्वपेद्भास्करोदये}
{सायंप्रातर्जपेत्सन्ध्यां तिष्ठन्पूर्वां तथेतराम्}


\twolineshloka
{पञ्चार्द्रो भोजनं भुञ्ज्यात्प्राद्भुखो मौनमास्थितः}
{न निन्द्यादन्नभक्ष्यांश्च स्वादुस्वादु च भक्षयेत्}


\twolineshloka
{नार्द्रपाणिः समुत्तिष्ठेन्नार्द्रपादः स्वपेन्निशि}
{देवर्षिर्नारदः प्राह एतदाचारलक्षणम्}


\twolineshloka
{शोचिष्केशमनड्वाहं देवगोष्ठं चतुष्पथम्}
{ब्राह्मणं धार्मिकं चैव नित्यं कुर्यात्प्रदक्षिणम्}


\twolineshloka
{अतिथीनां च सर्वेषां प्रेष्याणां स्वजनस्य च}
{सामात्यं भोजनं भृत्यैः पुरुषस्य प्रशस्यते}


\twolineshloka
{सायंप्रातर्मनुष्याणामशनं वेदनिर्मितम्}
{नान्तरा भोजनं दृष्टमुपवासी तथा भवेत्}


\twolineshloka
{होमकाले तथ्ना जुह्वन्नृतुकाले तथा व्रजन्}
{अनन्यस्त्रीजनः प्राज्ञो ब्रह्मचारी तथा भवेत्}


\twolineshloka
{अमृतं ब्राह्मणोच्छिष्टं जनन्या हृदयं कृतम्}
{तज्जनाः पर्युपासन्ते सत्यं सन्तः समासते}


\twolineshloka
{लोष्टमदीं तृणच्छेदी नखखादी तु यो नरः}
{नित्योच्छिष्टः संकसुको नेहायुर्विन्दते महत्}


\twolineshloka
{यजुषा संस्कृतं मांसं निवृत्तो मांसभक्षणात्}
{भक्षयेन्न वृथामांसं पृष्ठमांसं च वर्जयेत्}


\twolineshloka
{स्वदेशे परदेशे वा अतिर्थि नोपवासयेत्}
{काम्यकर्मफलं लब्ध्वा गुरूणामुपपादयेत्}


\twolineshloka
{गुरूणामासनं देयं कर्तव्यं चाभिवादनम्}
{गुरूनभ्यर्च्य युज्येत आयुषा यशसा श्रिया}


\twolineshloka
{नेक्षेतादित्यमुद्यन्तं न च नग्नां परस्त्रियम्}
{मैथुनं सततं धर्म्यं गुह्ये चैव समाचरेत्}


\twolineshloka
{तीर्थानां हृदयं तीर्थं शुचीनां हृदयं शुचिः}
{सर्वमार्यकृतं धर्म्यं वालसंस्पर्शनानि च}


\twolineshloka
{दर्शनेदर्शने नित्यं सुखप्रश्नमुदाहरेत्}
{सायं प्रातश्च विप्राणां प्रदिष्टमभिवादनम्}


\twolineshloka
{देवगोष्ठे गवां मध्ये ब्राह्मणानां क्रियापथे}
{स्वाध्याये भोजने चैव दक्षिणं पाणिमुद्धरेत्}


\threelineshloka
{सायं प्रातश्च विप्राणां पूजनं च यथाविधि}
{पण्यानां शोभते पण्यं कृषीणामृद्ध्यतां कृषिः}
{बहुकारं च सस्यानां वाह्ये वाहो गवां तथा}


\twolineshloka
{संपन्नं भोजने नित्यं पानीये तर्पणं तथा}
{सुशृतं पायसे ब्रूयाद्यवाग्वां कृसरे तथा}


\twolineshloka
{श्मश्रुकर्मणि संप्राप्ते क्षुते स्नानेऽथ भोजने}
{व्याधितानां च सर्वेषामायुष्ममभिनन्दनम्}


\twolineshloka
{प्रत्यादित्यं न मेहेत न पश्येदात्मनः शकृत्}
{सुतैः स्त्रिया च शयनं सह भोज्यं च वर्जयेत्}


\twolineshloka
{त्वंकारं नामधेयं च ज्येष्ठानां परिवर्जयेत्}
{अवराणां समानानामुभयं नैव दुष्यति}


\twolineshloka
{हृदयं पापवृत्तानां पापमाख्याति वैकृतम्}
{ज्ञानपूर्वं विनश्यन्ति गूहमाना महाजने}


\twolineshloka
{ज्ञानपूर्वकृतं पापं छादयन्त्यबहुश्रुताः}
{नैनं मनुष्याः पश्यन्ति पश्यन्त्येव दिवौकसः}


\threelineshloka
{पापेनापिहितं पापं पापमेवानुवर्तते}
{धर्मेणापिहितो धर्मो धर्ममेवानुवर्तते}
{धार्मिकेण कृतो धर्मो धर्ममेवानुवर्तते}


\twolineshloka
{पापं कृतं न स्मरतीह मूढोविवर्तमानस्य तदेति कर्तुः}
{राहुर्यथा चन्द्रमुपैति चापितथाऽबुधं पापमुपैति कर्म}


\twolineshloka
{आशया संचितं द्रव्यं दुःखेनैवोपभुज्यते}
{तद्बुधा न प्रशंसन्ति मरणं न प्रतीक्षते}


\twolineshloka
{मानसं सर्वभूतानां धर्ममाहुर्मनीषिणः}
{तस्मात्सर्वेषु भूतेषु मनसा शिवमाचरेत्}


\twolineshloka
{एक एव चरेद्धर्मं नास्ति धर्मे सहायता}
{केवलं विधिमासाद्य सहायः किं करिष्यति}


\twolineshloka
{धर्मो योनिर्मनुष्याणां देवानाममृतं दिवि}
{प्रेत्यभावे सुखं धर्माच्न्छश्वत्तैरुपभुज्यते}


\chapter{अध्यायः १९२}
\twolineshloka
{युधिष्ठिर उवाच}
{}


\twolineshloka
{अध्यात्मं नाम यदिदं पुरुषस्येह चिन्त्यते}
{यदध्यात्मं यथा चैतत्तन्मे ब्रूहि पितामह}


\threelineshloka
{कुतः सृष्टमिदं सर्वं ब्रह्मन्स्थावरजङ्गमम्}
{प्रलये च कमभ्येति तन्मे वक्तुमिहार्हसि ॥भीष्म उवाच}
{}


\twolineshloka
{अध्यात्ममिति मां पार्थ यदेतदनुपृच्छसि}
{तद्व्याख्यास्यामि ते तात श्रेयस्करतमं शुभम्}


\threelineshloka
{[सृष्टिप्रलयसंयुक्तमाचार्यैः परिदर्शितम्}
{]यज्ज्ञात्वा पुरुषो लेके प्रीतिं सौख्यं च विन्दति}
{फललाभश्च तस्य स्यात्सर्वभूतहितं च तत्}


\twolineshloka
{`आत्मानममलं राजन्नावृत्यैवं व्यवस्थितम्}
{तस्मिन्प्रकाशते नित्यं तमः सोमो यथैव तत्}


\twolineshloka
{तद्विद्वान्नष्टयाप्मैष ब्रह्मभूयाय कल्पते}
{अण्डावरणभूतानां पर्यन्तं हि यथा तमः ॥'}


\twolineshloka
{पृथिवी वायुराकाशमापो ज्योतिश्च पञ्चमम्}
{महाभूतानि भूतानां सर्वेषां प्रभवाप्ययौ}


\twolineshloka
{यतः सृष्टानि तत्रैव तानि यान्ति पुनःपुनः}
{महाभूतानि भूतेभ्यः सागरस्योर्मयो यथा}


\twolineshloka
{प्रसार्य च यथाङ्गानि कूर्मः संहरते पुनः}
{तद्वद्भूतानि भूतात्मा सृष्ट्वा संहरते पुनः}


\twolineshloka
{`स तेषां गुणसंघातः शरीरे भरतर्षभ}
{सततं प्रविलीयन्ते गुणास्ते प्रभवन्ति च}


\twolineshloka
{यद्विना नैव शृणुते न पश्यति न दीप्यते}
{यदधीनं यतस्तस्मादध्यात्ममिति कथ्यते}


\twolineshloka
{ज्ञानं तदेकरूपाख्यं नानाप्रज्ञान्वितं तदा}
{न तेवाचाऽनुरूपं स्याद्यया रासविवर्जितम्}


\threelineshloka
{आकाशात्खलु याज्येषु भवन्ति सुमहागुणाः}
{इति तन्मयमेवैतत्सर्वं स्थावरजङ्गमम्}
{}


\twolineshloka
{प्रलये च तमभ्येति तस्मादुत्सृज्यते पुनः}
{महाभूतेषु भूतात्मा सृष्ट्वा संहरते पुनः ॥'}


\twolineshloka
{महाभूतानि पञ्चैव सर्वभूतेषु भूतकृत्}
{अकरोत्तेषु वैषम्य तत्तु जीवो न पश्यति}


\twolineshloka
{शब्दः श्रोत्रं तथा खानि त्रयमाकाशसंभवम्}
{वायोः स्पर्शस्तथा चेष्टा त्वक्चैव त्रितयं स्मृतम्}


\twolineshloka
{रूपं चक्षुस्तथा पाकस्त्रिविधं तेज उच्यते}
{रसः क्लेदश्च जिह्वा च त्रयो जलगुणाः स्मृताः}


\twolineshloka
{घ्रेयं घ्राणं शरीरं च एते भूमिगुणास्त्रयः}
{महाभूतानि पञ्चैव षष्ठं च मन उच्यते}


\twolineshloka
{इन्द्रियाणि मनश्चैव विज्ञानान्यस्य भारत}
{सप्तमी बुद्धिरित्याहुः क्षेत्रज्ञः पुनरष्टमः}


\twolineshloka
{चक्षुरालोचनायैव संशयं कुरुते मनः}
{बुद्धिरध्यवसानाय क्षेत्रज्ञः साक्षिवत्स्थितः}


\twolineshloka
{`चिच्छक्त्याधिष्ठिता बुद्धिश्चेतनेत्यभिविश्रुता}
{चेतनानन्तरो जीवस्तदा वेत्ति च लक्ष्यते}


\twolineshloka
{नोत्सृजन्विसृजंश्चैव शरीरं दृश्यते तमः}
{तस्मिंश्चेतोपलब्धिः स्यात्तमो वा सारयन्त्युत}


\twolineshloka
{ऊर्ध्वं पादतलाभ्यां यदर्वाक्चोर्ध्वं च पश्यति}
{एतेन सर्वमेवेदं बिद्ध्यभिव्याप्तमन्तरम्}


\twolineshloka
{पुरुषैरिन्द्रियाणीह विजेतव्यानि कृत्स्नशः}
{तमो रजश्च सत्वं च तेऽपि भावास्तदाश्रिताः}


\twolineshloka
{एतां बुद्ध्वा नरो बुद्ध्या भूतानामागतिं गतिम्}
{समवेक्ष्य शनैश्चैव लभते शममुत्तमम्}


\twolineshloka
{गुणैर्नेनीयते बुद्धिर्बुद्धिरेवेन्द्रियाण्यपि}
{मनःषष्ठानि सर्वाणि बुद्ध्यभावे कुतो गुणाः}


\twolineshloka
{इति तन्मयमेवैतत्सर्वं स्थावरजङ्गमम्}
{प्रलीयते चोद्भवति तस्मान्निर्दिश्यते तथा}


\twolineshloka
{येन पश्यति तच्चक्षुः शृणोति श्रोत्रमुच्यते}
{जिघ्रति घ्राणमित्याहू रसं जानाति जिह्वया}


\twolineshloka
{त्वचा स्पर्शयते स्पर्शं बुद्धिर्विक्रियतेऽसकृत्}
{येन संकल्पयत्यर्थं किंचिद्भवति तन्मनः}


\twolineshloka
{अधिष्ठानानि बुद्धेर्हि पृथगर्थानि पञ्चधा}
{पञ्चेन्द्रियाणि यान्याहुस्तान्यदृश्योऽधितिष्ठति}


\twolineshloka
{पुरुषाधिष्ठिता बुद्धिस्त्रिषु भावेषु वर्तते}
{कदाचिल्लभते प्रीतिं कदाचिदनुशोचति}


\twolineshloka
{न सुखेन न दुःखेन कदाचिदपि वर्तते}
{एवं नराणां मनसि त्रिषु भावेष्ववस्थिता}


\twolineshloka
{सेयं भावात्मिका भावांस्त्रीनेतानतिवर्तते}
{सरितां सागरो भर्ता महावेलामिवोर्मिमान्}


\twolineshloka
{अविभागगता बुद्धिर्भावे मनसि वर्तते}
{प्रवर्तमानं तु रजस्तद्भावमनुवर्तते}


\twolineshloka
{इन्द्रियाणि हि सर्वाणि प्रवर्तयति सा तदा}
{ततः सत्वं तमो भावः प्रातियोगात्प्रवर्तते}


\twolineshloka
{प्रीतिः सत्वं रजः शोकस्तमो मोहस्तु ते त्रयः}
{येये च भावा लोकेऽस्मिन्सर्वेष्वेतेषु वै त्रिषु}


\twolineshloka
{इति बुद्धिगतिः सर्वा व्याख्याता तव भारत}
{इन्द्रियाणि च सर्वाणि विजेतव्यानि धीमता}


\twolineshloka
{सत्वं रजस्तमश्चैव प्राणिनां संश्रिताः सदा}
{त्रिविधा वेदना चैव सर्वसत्वेषु दृश्यते}


\threelineshloka
{सात्विकी राजसी चैव तामसी चेति भारत}
{सुखस्पर्शः सत्त्वगुणो दुःखस्पर्शो रजोगुणः}
{तमोगुणेन संयुक्तौ भवतो व्यावहारिकौ}


\twolineshloka
{तत्र यत्प्रीतिसंयुक्तं काये मनसि वा भवेत्}
{वर्तते सात्विको भाव इत्युपेक्षेत तत्तथा}


\twolineshloka
{अथ यद्दुःखसंयुक्तमप्रीतिकरमात्मनः}
{प्रवृत्तं रज इत्येव तन्न संरभ्य चिन्तयेत्}


\threelineshloka
{अथ यन्मोहसंयुक्तमव्यक्तविषयं भवेत्}
{अप्रतर्क्यमविज्ञेयं तमस्तदुपधारयेत्}
{}


\twolineshloka
{प्रहर्षः प्रीतिरानन्दः सुखं संशान्तचित्तता}
{कथंचिदभिवर्तन्त इत्येते सात्विका गुणा}


\twolineshloka
{अतुष्टिः परितापश्च शोको लोभस्तथाऽक्षमा}
{लिङ्गानि रजसस्तानि दृश्यन्ते हेत्वहेतुभिः}


\twolineshloka
{अभिमानस्तथा मोहः प्रमादः स्वप्नतन्द्रिता}
{कथंचिदभिवर्तन्ते विविधास्तामसा गुणाः}


\twolineshloka
{दूरगं बहुधागामि प्रार्थनासंशयात्मकम्}
{मनः सुनियतं यस्य स सुखी प्रेत्य चेह च}


\twolineshloka
{सत्वक्षेत्रज्ञयोरेतदन्तरं पश्य सूक्ष्मयोः}
{सृजते तु गुणानेक एको न सृजते गुणान्}


\twolineshloka
{मशकोदुम्बरौ वाऽपि संप्रयुक्तौ यथा सदा}
{अन्योन्यमेतौ स्यातां च संप्रयोगस्तथा तयोः}


\twolineshloka
{पृथग्भूतौ प्रकृत्या तौ संप्रयुक्तौ च सर्वदा}
{यथा मत्स्यो जलं चैव संप्रयुक्तौ तथैव तौ}


\twolineshloka
{न गुणा विदुरात्मानं स गुणान्वेति सर्वशः}
{परिद्रष्टा गुणानां स संसृष्टान्मन्यते तथा}


\twolineshloka
{इन्द्रियैस्तु प्रदीपार्थं कुरुते बुद्धिसप्तमैः}
{निर्विचेष्टैरजानद्भिः परमात्मा प्रदीपवत्}


\twolineshloka
{सृजते हि गुणान्सत्वं क्षेत्रज्ञः परिपश्यति}
{संप्रयोगस्तयोरेष सत्वक्षेत्रज्ञयोर्ध्रुवः}


\twolineshloka
{आश्रयो नास्ति सत्वस्य क्षेत्रज्ञस्य च कश्चन}
{सत्वं मनः संसृजते न गुणान्वै कदाचन}


\twolineshloka
{रश्मीस्तेषां स मनसा यदा सम्यङ्गियच्छति}
{तदा प्रकाशतेऽस्यात्मा घटे दीपो ज्वलन्निव}


\twolineshloka
{त्यक्त्वा यः प्राकृतं कर्म नित्यमात्मरतिर्मुनिः}
{सर्वभूतात्मभूस्तस्मात्स गच्छेदुत्तमां गतिम्}


\twolineshloka
{यथा वारिचरः पक्षी सलिलेन न लिप्यते}
{एवमेव कृतप्रज्ञो भूतेषु परिवर्तते}


\twolineshloka
{एवं स्वभावमेवैतत्स्वबुद्ध्या विहरेन्नरः}
{अशोचन्नप्रहृष्यंश्च चरेद्विगतमत्सरः}


\twolineshloka
{स्वभावसिद्ध्या युक्तस्तु स नित्यं सृजत गुणान्}
{ऊर्णनाभिर्यथा सूत्रं विज्ञेयास्तन्तुवद्गुणाः}


\twolineshloka
{प्रध्वस्ता न निवर्न्तते निवृत्तिर्नोपलभ्यते}
{प्रत्यक्षेण परोक्षं तदनुमानेन सिध्यति}


\twolineshloka
{एवमेकेऽध्यवस्यन्ति निवृत्तिरिति चापरे}
{उभयं संप्रधार्यैतद्व्यवस्येत यथामति}


\twolineshloka
{इतीमं हृदयग्रत्थिं बुद्धिभेदमयं दृढम्}
{विमुच्य सुखमासीत न शोचेच्छिन्नसंशयः}


\twolineshloka
{मलिनाः प्राप्नुयुः शुद्धिं यथा पूर्णां नदीं नराः}
{अवगाह्य सुविद्वांसो विद्धि ज्ञानमिदं तथा}


\twolineshloka
{महानद्या हि पारज्ञस्तप्यते न तदन्यथा}
{न तु तप्यति तत्त्वज्ञः फले ज्ञाते तरत्युत}


% Check verse!
एवं ये विदुराध्यात्मं केवलं ज्ञानमुत्तमम्
\twolineshloka
{एतां बुद्धा नरः सर्वां भूतानामागतिं गतिम्}
{अवेक्ष्य च शनैर्बुद्ध्या लभते शममुत्तमम्}


\twolineshloka
{त्रिवर्गो यस्य विदितः प्रेक्ष्य तं स विमुच्यते}
{अन्वीक्ष्य मनसा युक्तस्तत्त्वदर्शी निरुत्सुकः}


\twolineshloka
{न चात्मा शक्यते द्रष्टुमिन्द्रियेषु विभागशः}
{तत्रतत्र विसृष्टेषु दुर्वार्येष्वकृतात्मभिः}


\twolineshloka
{एतद्बुद्धा भवेद्बुद्धः किमन्यद्बुद्धिलक्षणम्}
{प्रतिगृह्य च निह्नोति ह्यन्यथा च प्रदृश्यते}


\twolineshloka
{न सर्पति च यं प्राहुः सर्वत्र प्रतिहन्यते}
{धूमेन चाप्रसन्नोऽग्निर्यथाऽर्कं न प्रवर्तयेत्}


\twolineshloka
{धिष्ण्याधिपे प्रसन्ने तु स्थितिमेतां निरीक्षते}
{अतिक्षूराच्च सूक्ष्मत्वात्प्रस्थानं न प्रकाशते}


\twolineshloka
{प्रपद्य तच्छ्रुताह्नानि चिन्मयं स्वीकृतं विना}
{विज्ञाय तद्धि मन्यन्ते कृतकृत्या मनीषिणः}


\twolineshloka
{न भवति विदुषां ततो भयंयदविदुषां सुमहद्भयं भवेत्}
{न हि गतिरधिकास्ति कस्यचित्सति हि गुणे प्रवदन्त्यतुल्यताम्}


\twolineshloka
{यः करोत्यनभिसन्धिपूर्वकंतच्च निर्णुदति यत्पुरा कृतम्}
{नाप्रियं तदुभयं कुतः प्रियंतस्य तज्जनयतीह सर्वतः}


\threelineshloka
{लोकमातुरमसूयते जनस्तस्य तज्जनयतीह सर्वतः}
{लोक आतुरजनान्विराविणस्तत्तदेव बहु पश्य शोचतः}
{तत्र पश्य कुशलानशोचतोये विदुस्तदुभयं पदं सताम्}


\chapter{अध्यायः १९३}
\twolineshloka
{भीष्म उवाच}
{}


\twolineshloka
{हन्त वक्ष्यामि ते पार्थ ध्यानयोगं चतुर्विधम्}
{यं ज्ञात्वा शाश्वतीं सिद्धिं गच्छन्तीह महर्षयः}


\twolineshloka
{यथा स्वनुष्ठितं ध्यानं तथा कुर्वन्ति योगिनः}
{महर्षयो ज्ञानतृप्ता निर्वाणगतमानसाः}


\twolineshloka
{नावर्तन्ते पुनः पार्थ मुक्ताः संसारदोषतः}
{अन्मदोषपरिक्षीणाः स्वभावे पर्युपस्थिताः}


\twolineshloka
{निर्द्वन्द्वा नित्यसत्वस्था विमुक्तिं नित्यमश्रिताः}
{असङ्गीन्यविवादीनि मनः शान्तिकराणि च}


\twolineshloka
{तत्र ध्यानेन संश्लिष्टमेकाग्रे धारयेन्मनः}
{तत्र च ध्यानसंरोधादथ ज्ञानी भवत्युत}


\threelineshloka
{चतुर्विधेषु भावेषु योऽर्थसक्तः सदैव हि}
{तज्ज्ञात्वा वास्तवं तेषामर्थेषु परिवर्तते}
{पिण्डीकृत्येन्द्रियग्राममासीनः काष्ठवन्मुनिः}


\twolineshloka
{शब्दं न विन्देच्छ्रोत्रेण त्वचा स्पर्शं न वेदयेत्}
{रूपं न चक्षुषा विद्याज्जिह्वया न रसांस्तथा}


\twolineshloka
{घ्रेयाण्यपि च सर्वाणि जह्याद्राणेन योगवित्}
{पञ्चवर्गप्रमाथीनि नेच्छेच्चैतानि वीर्यवान्}


\twolineshloka
{ततो मनसि संसृज्य पञ्चवर्ग विचक्षणः}
{समादध्यान्मनो भ्रान्तमिन्द्रियैः सह पञ्चमिः}


\twolineshloka
{विसंचारि निरालम्बं पञ्चद्वारं चलाचलम्}
{पूर्वं ध्यानपदे धीरः समादध्यान्मनो नरः}


\twolineshloka
{इन्द्रियाणि मनश्चैव यदा पिण्डीकरोत्ययम्}
{एव ध्यानपथः पूर्वो मया समनुवर्णितः}


\twolineshloka
{तस्य तत्पूर्वसंरुद्धमात्सषष्ठं मनोऽन्तरा}
{स्फुरिष्यति समुद्धान्तं विद्युदम्बुधरे यथा}


\twolineshloka
{जलबिन्दुर्यथा लोलः पर्णस्थः सर्वतश्चलः}
{एवमेवास्य तच्चित्तं भ्रमति ध्यानवर्त्मनि}


\twolineshloka
{समाहितं क्षणं किंचिद्ध्यानवर्त्मनि तिष्ठति}
{पुनर्वायुपथं प्राप्तं मनो भवति वायुवत्}


\twolineshloka
{अनिर्वेदो गतक्लेशो गततन्द्रीरमत्सरः}
{समादध्यात्पुनश्चेतो ध्यानेन ध्यानयोगवित्}


\twolineshloka
{विचारश्च विवेकश्च वितर्कश्चोपजायते}
{मुनेः समादधानस्य प्रथमं ध्यानमादितः}


\twolineshloka
{मनसा क्लिश्यमानस्तु समाधानं च कारयेत्}
{न निर्वेदं मुनिर्गच्छेत्कुर्यादेवात्मनो हितम्}


\twolineshloka
{पांसुभस्मकरीषाणां यथा वै राशयश्चिताः}
{सहसा वारिणा सिक्ता न यान्ति परिभावनम्}


\twolineshloka
{किंचित्स्निग्धं यथा च स्याच्छुष्कचूर्णमभावितम्}
{क्रमशस्तु शनैर्गच्छेत्सर्वं तत्परिभावनम्}


\twolineshloka
{एवमेवेन्द्रियग्रामं शनैः संपरिभावयेत्}
{संहरेत्क्रमशश्चैनं स सम्यक्प्रशमिष्यति}


\twolineshloka
{स्वयमेव मनश्चैवं पञ्चवर्गं च भारत}
{पूर्वं ध्यानपथे स्थाप्य नित्ययोगेन शाम्यति}


\twolineshloka
{न तत्पुरुषकारेण न च दैवेन केनचित्}
{सुखमेष्यति तत्तस्य न भवन्ति विपत्तयः}


\twolineshloka
{सुखेन तेन संयुक्तो रस्यते ध्यानकर्मणि}
{गच्छन्ति योगिनो ह्येवं निर्वाणं तन्निरामयम्}


\chapter{अध्यायः १९४}
\twolineshloka
{युधिष्ठिर उवाच}
{}


\twolineshloka
{चातुरात्रस्यनुक्तं ते राजधर्मास्तथैव च}
{नानाश्रयाश्च भगवन्नितिहासाः पृथग्विधाः}


\twolineshloka
{धुतास्त्वत्तः कथाश्चैव धर्मयुक्ता महामते}
{संदेहोऽस्ति तु कश्चिन्मे तद्भवान्वक्तुमर्हति}


\twolineshloka
{जापकानां फलावाप्तिं श्रोतुमिच्छामि भारत}
{किं फलं जपतामुक्तं क्व वा तिष्ठन्ति जापकाः}


\twolineshloka
{जपस्य च विधिं कृत्स्नं वक्तुमर्हसि सेऽनघ}
{जापका इति किंचैतत्साङ्ख्ययोगक्रियाविधिः}


\threelineshloka
{किं यज्ञविधिरेवैष किमेतज्जप्यमुच्यते}
{एतन्मे सर्वमाचक्ष्व सर्वज्ञो ह्यसि मे मतः ॥भीष्म उवाच}
{}


\twolineshloka
{अत्राप्युदाहरन्तीममितिहासं पुरातनम्}
{यमस्य यत्पुरा वृत्तं कालस्य ब्राह्मणस्य च}


\threelineshloka
{`इक्ष्वाकोश्चैव मृत्योश्च विवादे धर्मकारणात्}
{'संन्यास एव वेदान्ते वर्तते जपनं प्रति}
{वेदवादाङ्गनिर्वृत्ताः शान्ता ब्रह्मण्यवस्थिताः}


\twolineshloka
{साङ्ख्ययोगौ तु यावुक्तौ मुनिभिः समदर्शिभिः}
{मार्गौ तावप्युभावेतौ संश्रितौ न च संश्रितौ}


\twolineshloka
{यथा संश्रूयते राजन्कारणं चात्र वक्ष्यते}
{`क्रमेण चैव विहितो जपयज्ञविधिर्नृप}


\twolineshloka
{सालम्बनमिति ज्ञेयं जपयज्ञात्मकं शुभम्}
{'मनः समाधिरेवात्र तथेन्द्रियजयः स्मृतः}


\twolineshloka
{सत्यमग्निपरीचारो विविक्तानां च सेवनम्}
{ध्यानं तपो दमः क्षान्तिरनसूया मिताशनम्}


\twolineshloka
{विषयप्रतिसंहारो मितजल्पस्तथा शमः}
{एष प्रर्वाको धर्मो निवर्तकमथो शृणु}


\threelineshloka
{यथा निवर्तते धर्मो जपतो ब्रह्मचारिणः}
{`न जपो न च वै ध्यानं नेच्छा न द्वेषहर्षणौ}
{युज्यते नृपशार्दूल सुसंवेद्यं हि तत्किल}


\twolineshloka
{जपमावर्तयन्नित्यं जपन्वै ब्रह्मचारिकम्}
{तदर्थबुद्ध्या संयाति मनसा जापकः परम्}


\twolineshloka
{यथा संश्रूयते जापो येन वै जापको भवेत्}
{संहिताप्रणवेनैव सावित्री च परा मता}


\twolineshloka
{यदन्यदुचितं शुद्धं वेदस्मृत्युपपादितम्}
{'एतत्सर्वमशेषेण यथोक्तं परिवर्तयेत्}


\threelineshloka
{द्विविदं मार्गमासाद्य व्यक्ताव्यक्तमनामयम्}
{कुशोच्चयनिषष्णः सन्कुशहस्तः कुशैः शिख}
{कुशैः परिवृतस्तस्मिन्मध्ये छन्नः कुशैस्तथा}


\twolineshloka
{विषयेभ्यो नमस्कुर्याद्विषयान्न च भावयेत्}
{साम्यमुत्पाद्य मनसा मनस्येव मनो दधत्}


\twolineshloka
{तद्धिया ध्यायति ब्रह्म जपन्वै संहितां हिताम्}
{संन्यस्यत्यथवा तां वै समाधौ पर्यवस्थितः}


\twolineshloka
{ध्यानमुत्पादयत्यत्र संहिताबलसंश्रयात्}
{`अथाभिमतमन्त्रेण प्रणवाद्यं जपेत्कृती}


\threelineshloka
{यस्मिन्नेवाभिपतितं मनस्तत्र निवेशयेत्}
{समाधौ स हि मन्त्रे तु संहितां वा यथाविधि}
{'शुद्धात्मा तपसा दान्तो निवृत्तद्वेषकामवान्}


\twolineshloka
{अरागमोहो निर्द्वन्द्वो न शोचति न सज्जते}
{न कर्ता करणीयानां नाकार्याणामिति स्थितिः}


\twolineshloka
{न चाहंकारयोगेन मनः प्रस्थाफ्येत्क्वचित्}
{न चार्थग्रहणे युक्तो नावमानी न चाक्रियः}


\twolineshloka
{ध्यानक्रियापरो युक्तो ध्यानवान्ध्याननिश्चयः}
{ध्याने समाधिमुत्पाद्य तदपि त्यजति क्रमात्}


\twolineshloka
{स वै तस्यामवस्थायां सर्वत्यागकृतः सुखी}
{निरिच्छस्त्यजति प्राणान्ब्राह्मीं संश्रयते तनुम्}


\twolineshloka
{`निरालम्बो भवेत्स्मृत्वा मरणाय समाधिमान्}
{सर्वाल्लोँकान्समाक्रम्य क्रमात्प्राप्नोति वै परम् ॥'}


\twolineshloka
{अथवा नेच्छते तत्र ब्रह्मकायनिषेवणम्}
{उत्क्रामति च मार्गस्थो नैव क्वचन जायते}


\twolineshloka
{आत्मबुद्धिं समास्थाय शान्तीभूतो निरामयः}
{अमृतं विरजाः शुद्धमात्मानं प्रतिपद्यते}


\chapter{अध्यायः १९५}
\twolineshloka
{युधिष्ठिर उवाच}
{}


\threelineshloka
{गतीनामुत्तमा प्राप्तिः कथिता जापकेष्विह}
{एकैवैषा गतिस्तेषामुत यान्त्यपरामपि ॥भीष्म उवाच}
{}


\twolineshloka
{शृणु वहितो राजञ्जापकानां गतिं विभो}
{यथा गच्छन्ति निरयमनेकं पुरुषर्षभ}


% Check verse!
यथोक्तमेतत्पूर्वं यो नानुतिष्ठति जापकः ॥एकदेशक्रियश्चात्र निरयं स निगच्छति
\twolineshloka
{अवमानेन कुरुते न तुष्यति न शोचति}
{ईदृशो जापको याति निरयं नात्र संशयः}


\twolineshloka
{अहंकारकृतश्चैव सर्वे निरयग्रामिनः}
{परावमानी पुरुषो भविता निरयोपगः}


\twolineshloka
{अभिध्यापूर्वकं जप्यं कुरुते यश्च मोहितः}
{यत्रास्य रागः पतति तत्रतत्रोपपद्यते}


\twolineshloka
{अथैश्वर्यप्रसक्तः सञ्जापको यत्र रज्यते}
{स एव निरयस्तस्य नासौ तस्मात्प्रमुच्यते}


\twolineshloka
{रागेण जापको जप्यं कुरुते यश्च मोहितः}
{यत्रास्य रागः पतति तत्र तत्रोपजायते}


\twolineshloka
{दुर्बुद्धिरकृतप्रज्ञश्चले मनसि तिष्ठति}
{फलस्यापचितिं याति निरयं चाधिगच्छति}


\twolineshloka
{अकृतव्रज्ञको बालो मोहं गच्छति जापकः}
{स मोहान्निरयं याति यत्र गत्वाऽनुशोचति}


\twolineshloka
{दृढग्राही करोमीति जाप्यं जपति जापकः}
{न संपूर्णो न वा युक्तो निरयं सोऽधिगच्छति}


\twolineshloka
{अनिमित्तं परं यत्तदव्यक्तं ब्रह्मणि स्थितम्}
{तद्भूतो जापकः कस्मात्सशरीरमिहाविशेत्}


\twolineshloka
{दुष्प्रज्ञानेन निरया बहवः समुदाहृताः}
{प्रशस्तं जापकत्वं च दोषाश्चैते तदात्मकाः}


\chapter{अध्यायः १९६}
\twolineshloka
{युधिष्ठिर उवाच}
{}


\threelineshloka
{कीदृशं निरयं याति जापको वर्णयस्व मे}
{कौतूहलं हि राजन्मे तद्भवान्वक्तुमर्हति ॥भीष्म उवाच}
{}


\twolineshloka
{धर्मस्यांशप्रसूतोऽसि धर्मज्ञोऽसि स्वभावतः}
{धर्ममूलाश्रयं वाक्यं शृणुष्वावहितोऽनघ}


\twolineshloka
{अमूनि यानि स्थानानि देवानाममरात्मनाम्}
{नानासंस्थानवर्णानि नानारूपफलानि च}


\twolineshloka
{दिव्यानि कामचारीणि विमानानि सभास्तथा}
{आक्रीडा विविधा राजन्पद्मिन्यश्चामलोदकाः}


\twolineshloka
{चतुर्णां लोकपालानां शुक्रस्याथ बृहस्पतेः}
{मरुतां विश्वदेवानां साध्यानामश्विनोरपि}


\twolineshloka
{रुद्रादित्यवसूनां च तथाऽन्येषां दिवौकसाम्}
{एते वै निरयास्तात स्थानस्य परमात्मनः}


\twolineshloka
{अभयं चानिमित्तं च न च क्लेशभयावृतम्}
{द्वाभ्यां मुक्तं त्रिभिर्मुक्तमष्टाभिस्त्रिभिरेव च}


\twolineshloka
{चतुर्लक्षणवर्जं तु चतुष्कारणवर्जितम्}
{अप्रहर्षमनानन्दमशोकं विगतक्लमम्}


\twolineshloka
{कालः संपच्यते तत्र कालस्तत्र न वै प्रभुः}
{स कालस्य प्रभू राजन्सर्वस्यापि तथेश्वरः}


\threelineshloka
{`एतद्वै ब्रह्मणः स्थाने जापकस्य महात्मनः}
{तत्रस्थं परमात्मानं ध्यायन्वै सुसमाहितः}
{हिरण्यगर्भः सायुज्यं प्राप्नुयाद्वा नृपोत्तम ॥'}


\twolineshloka
{आत्मकेवलतां प्राप्तस्तत्र गत्वा न शोचति}
{ईदृशं परमं स्थानं निरयास्ते च तादृशाः}


\twolineshloka
{एते ते निरयाः प्रोक्ताः सर्व एव यथातथम्}
{तस्य स्थानवरस्येह सर्वे निरयसंज्ञिताः}


\chapter{अध्यायः १९७}
\twolineshloka
{युधिष्ठिर उवाच}
{}


\threelineshloka
{कालमृत्युयमानां ते इक्ष्वाकोर्ब्राह्मणस्य च}
{विवादो व्याहृतः पूर्वं तद्भवान्वक्तुमर्हति ॥भीष्म उवाच}
{}


\twolineshloka
{अत्राप्युदाहरन्तीममितिहासं पुरातनम्}
{इक्ष्वाकोः सूर्यपुत्रस्य यद्वृत्तं ब्राह्मणस्य च}


\threelineshloka
{कालस्य मृत्योश्च तथा यद्वृत्तं तन्निबोध मे}
{यथा स तेषां संवादो यस्मिन्स्थानेऽपि चाभवत्}
{`येनैव कारणेनात्र धर्मवादसमन्वितः ॥'}


\twolineshloka
{ब्राह्मणो जापकः कश्चिद्धर्मवृत्तो महायशाः}
{षडङ्गविन्महाप्राज्ञः पैप्पलादिः स कौशिकः}


\twolineshloka
{तस्यापरोक्षं विज्ञानं षडङ्गेषु वभूव ह}
{वेदेषु चैव निष्णातो हिमवत्पादसंश्रयः}


\twolineshloka
{सोयं ब्राह्मं पस्तेपे संहितां संयतो जपन्}
{तस्य वर्षसहस्रं तु नियमेन तथा गतम्}


\twolineshloka
{स देव्या दर्शितः साक्षात्प्रीतास्मीति तदा किल}
{जप्यमावर्तयंस्तूष्णीं न स तां किंचिदब्रवीत्}


\twolineshloka
{तस्यानुकम्पया देवी प्रीता समभवत्तदा}
{वेदमाता ततस्तस्य तज्जप्यं समपूजयत्}


\twolineshloka
{`चतुर्भिरक्षरैर्युक्ता सोमपानेऽक्षराष्टका}
{जगद्बीजसमायुक्ता चनुर्विशाक्षरात्मिका ॥'}


\twolineshloka
{समाप्य जप्यं तूत्थाय शिरसा पादयोस्तदा}
{पपात देव्या धर्मात्मा वचनं चेदमब्रवीत्}


\threelineshloka
{दिष्ट्या देवि प्रसन्ना त्वं दर्शनं चागता मम}
{यदि चापि प्रसन्नासि जप्ये मे रमतां मनः ॥सावित्र्युवाच}
{}


\twolineshloka
{किं प्रार्थयसि विप्रर्षे किं चेष्टं करवाणि ते}
{प्रब्रूहि जपतां श्रेष्ठ सर्वं तत्ते भविष्यति}


\twolineshloka
{इत्युक्तः स तदा देव्या विप्रः प्रोवाच धर्मवित्}
{जप्यं प्रति ममेच्छेयं वर्धत्विति पुनः पुनः}


\twolineshloka
{मनसश्च समाधिर्मे वर्धेताहरहः शुभे}
{तत्तथेति ततो देवी मधुरं प्रत्यभाषत}


\twolineshloka
{इदं चैवापरं प्राह देवी तत्प्रियकाम्यया}
{निरयं नैव याता त्वं यत्र याता द्विजर्षभाः}


\twolineshloka
{यास्यसि ब्रह्मणः स्थानमनिमित्तमतन्द्रितः}
{साधु ते भविता चैतद्यत्त्वयाऽहमिहार्थिता}


\fourlineindentedshloka
{नियतो जप चैकाग्रो धर्मस्त्वां समुपैष्यति}
{कालो मृत्युर्यमश्चैव समायास्यन्ति तेऽन्तिकम्}
{भविता च विवादोऽत्र तव तेषां च धर्मतः ॥भीष्म उवाच}
{}


% Check verse!
एवमुक्त्वा भगवती जगाम भवनं स्वकम्
\twolineshloka
{ब्राह्मणोऽपि जपन्नास्ते दिव्यं वर्षशतं तथा}
{सदा दान्तो जितक्रोधः सत्यसन्धोऽनसूयकः}


\threelineshloka
{समाप्ते नियमे तस्मिन्नथ विप्रस्य धीमतः}
{साक्षात्प्रीतस्तदा धर्मो दर्शयामास तं द्विजम् ॥धर्म उवाच}
{}


\twolineshloka
{द्विजाते पश्य मां धर्ममहं त्वां द्रष्टुमागतः}
{जप्यस्यास्य फलं यत्तत्प्रंप्राप्तं तच्च मे शृणु}


\twolineshloka
{जिता लोकास्त्वया सर्वे ये दिव्या ये च मानुषाः}
{देवानां निलयान्साधो सर्वानुत्क्रम्य यास्यसि}


\threelineshloka
{प्राणत्यागं कुरु मुने गच्छ लोकान्यथेप्सितान्}
{त्यक्त्वाऽऽत्मनः शरीरं च ततो लोकानवाप्स्यसि ॥ब्राह्मण उवाच}
{}


\threelineshloka
{कृतं लोकेन मे धर्म गच्छ त्वं च यथासुखम्}
{बहुदुःखमहं देहं नोत्सृजेयमहं विभो ॥धर्म उवाच}
{}


\twolineshloka
{`अचलं ते मनः कृत्वा त्यज देहं महामते}
{अनेन किं ते संयोगः कथं मोहं गमिष्यसि ॥'}


\threelineshloka
{अवश्यं भोः शरीरं ते त्यक्तव्यं मुनिपुङ्गव}
{स्वर्गमारोह भो विप्र किं वा वै रोचतेऽनघ ॥ब्राह्मण उवाच}
{}


\twolineshloka
{किमुक्तं धर्म किं नेति कस्मान्मां प्रोक्तवानसि}
{त्यज देहं द्विजेति त्वं ससंबुध्यात्र मे यदि}


\threelineshloka
{न रोचये स्वर्गवासं विना देहमहं विभो}
{गच्छ धर्म न मे श्रद्धा स्वर्गं गन्तुं विनाऽऽत्मना ॥धर्म उवाच}
{}


\threelineshloka
{अलं देहे मनः कृत्वा त्यक्त्वा देहं सुखी भव}
{गच्छ लोकानरजसो यत्र गत्वा न शोचसि ॥ब्राह्मण उवाच}
{}


\threelineshloka
{रमे जपन्महाभाग कृतं लोकैः सनातनैः}
{सशरीरेण गन्तव्यं मया स्वर्गं न चान्यथा ॥धर्म उवाच}
{}


\fourlineindentedshloka
{`एवं ते कायसंप्रातिर्वर्तते मुनिसत्तम}
{'यदि त्वं नेच्छसि त्यक्तुं शरीरं पश्य वै द्विज}
{एष कालस्तथा मृत्युर्यमश्च त्वामुपागताः ॥भीष्म उवाच}
{}


\threelineshloka
{अथ वैवस्वतः कालो मृत्युश्च त्रितयं विभो}
{ब्राह्मणं तं महाभागमुपगम्येदमब्रवन् ॥यम उवाच}
{}


\threelineshloka
{तपसोऽस्य सुतप्तस्य तथा सुचरितस्य च}
{फलप्राप्तिस्तव श्रेष्ठा यमोऽहं त्वामुपब्रुवे ॥काल उवाच}
{}


\threelineshloka
{यथा वदस्य जप्यस्य फलं प्राप्तस्त्वमुत्तमम्}
{कालस्ते स्वर्गमारोहुं कालोऽहं त्वामुपागतः ॥मृत्युरुवाच}
{}


\threelineshloka
{मृत्युं मां विद्धि धर्मज्ञ रूपिणं स्वयमागतम्}
{कालेन चोदितो विप्र त्वामितो नेतुमद्य वै ॥ब्राह्मण उवाच}
{}


\threelineshloka
{स्वागतं सूर्यपुत्राय कालाय च महात्मने}
{मृत्यवे चाथ धर्माय किं कार्यं करवाणि वः ॥भीष्म उवाच}
{}


\twolineshloka
{अर्ध्यं पाद्यं च दत्त्वा स तेभ्यस्तत्र समागम}
{अब्रवीत्परमप्रीतः स्वशक्त्या किं करोमि वः}


\threelineshloka
{`स्वकार्यनिर्भरा यूयं परोपद्रवहेतवः}
{भवन्तो लोकसामान्याः किमर्थं ब्रूत सत्तमाः ॥यम उवाच}
{}


\threelineshloka
{वयमप्येवमत्युग्रा धातुराज्ञापुरः सराः}
{चोदिता धावमाना वै कर्मभावमनुव्रताः ॥भीष्म उवाच}
{'}


\twolineshloka
{तस्मिन्नेवाथ काले तु तीर्थयात्रामुपागतः}
{इक्ष्वाकुरगमत्तत्र समेता यत्र ते विभो}


\twolineshloka
{सर्वानेव तु राजर्षिः संपूज्याथ प्रणम्य च}
{कुशलप्रश्नमकरोत्सर्वेषां राजसत्तमः}


\twolineshloka
{तस्मै सोऽथासनं दत्त्वा पाद्यमर्ध्यं तथैव च}
{अब्रवीद्ब्राह्मणो वाक्यं कृत्वा कुशलसंविदम्}


\threelineshloka
{स्वागतं ते महाराज ब्रूहि यद्यदिहेच्छसि}
{स्वशक्त्या किं करोमीह तद्भवान्प्रब्रवीतु माम ॥राजोवाच}
{}


\threelineshloka
{राजाऽहं ब्राह्मणश्च त्वं यदा षट््कर्मसंस्थितः}
{ददानि वसु किंचित्ते प्रार्थितं तद्वदस्व मे ॥ब्राह्मण उवाच}
{}


\twolineshloka
{द्विविधो ब्राह्मणो राजन्धर्मश्च द्विविधः स्मृतः}
{प्रवृत्तश्च निवृत्तश्च निवृत्तोऽह प्रतिग्रहात्}


\fourlineindentedshloka
{तेभ्यः प्रयच्छ दानानि ये प्रवृत्ता नराधिप}
{अहं न प्रतिगृह्णामि किमिष्टं किं ददामि ते}
{ब्रूहि त्वं नृपतिश्रेष्ठ तपसा साधयामि किम् ॥राजोवाच}
{}


\threelineshloka
{क्षत्रियोऽहं न जानामि देहीति वचनं क्वचित्}
{प्रयच्छ युद्धमित्येवंवादी चास्मि द्विजोत्तम ॥ब्राह्मण उवाच}
{}


\threelineshloka
{तुष्यसि त्वं स्वधर्मेण तथा तुष्टा वयं नृप}
{अन्योन्यस्योत्तरं नास्ति यदिष्टं तत्समाचर ॥राजोवाच}
{}


\threelineshloka
{स्वशक्त्याऽहं ददानीति त्वया पूर्वमुदाहृतम्}
{याचे त्वां दीयतां मह्यं जप्यस्यास्य फलं द्विज ॥ब्राह्मण उवाच}
{}


\threelineshloka
{युद्धं मम सदा वाणी याचतीति विकत्थसे}
{न च युद्धं मया सार्धं किमर्थं याचसे पुनः ॥राजोवाच}
{}


\threelineshloka
{वाग्वज्राब्राह्मणाः प्रोक्ताः क्षत्रिया बाहुजीविनः}
{वाग्युद्धं तदिदं तीव्रं मम विप्र त्वया सह ॥ब्राह्मण उवाच}
{}


\threelineshloka
{सेयमद्य प्रतिज्ञा मे स्वशक्त्या किं प्रदीयताम्}
{ब्रूहि दास्यामि राजेन्द्र विभवे सति माचिरम् ॥राजोवाच}
{}


\threelineshloka
{यत्तद्वर्षशतं पूर्णं जप्यं वै जपता त्वया}
{फलं प्राप्तं तत्प्रयच्छ मम दित्सुर्भवान्यदि ॥ब्राह्मण उवाच}
{}


\twolineshloka
{परमं गृह्यतां तस्य फलं यज्जपितं मया}
{अर्धं त्वमविचारेण फलं तस्य ह्यवाप्नुहि}


\threelineshloka
{अथवा सर्वमेवेह मामकं जापकं फलम्}
{राजन्प्राप्नुहि कामं त्वं यदि सर्वमिहेच्छसि ॥राजोवाच}
{}


\threelineshloka
{कृतं सर्वेण भद्रं ते जप्यं यद्याचितं मया}
{स्वस्ति तेऽस्तु गमिष्यामि किंच तस्य फलं वद ॥ब्राह्मण उवाच}
{}


\threelineshloka
{फलप्राप्तिं न जानामि दत्तं यज्जपितं मया}
{अयं धर्मश्च कालश्च यमो मृत्युश्च साक्षिणः ॥राजोवाच}
{}


\fourlineindentedshloka
{अज्ञातमस्य धर्मस्य फलं किं मे करिष्यति}
{फलं ब्रवीषि धर्मस्य न चेज्जप्यकृतस्य माम्}
{प्राप्नोतु तत्फलं विप्रो नाहमिच्छे ससंशयम् ॥ब्राह्मण उवाच}
{}


\twolineshloka
{नाददेऽपरदत्तं वै दत्तं वा चाफलं मया}
{वाक्यं प्रमाणं राजर्षे ममाद्य तव चैव हि}


\twolineshloka
{सकृदंशो निपतति सकृत्कन्या प्रदीतये}
{सकृदेव ददानीति त्रीण्येतानि सकृत्सकृत्}


\twolineshloka
{नाभिसंधिर्मया जप्ये कृतपूर्वः कदाचन}
{जप्यस्य राजशार्दूल कथं वेत्स्याम्यहं फलम्}


\twolineshloka
{ददस्वेति त्वया चोक्तं दत्तं वाचा फलं मया}
{न वाचं दूषयिष्यामि सत्यं रक्ष स्थिरो भव}


\twolineshloka
{अथैवं वदतो मेऽद्य वचनं न करिष्यसि}
{महानधर्मो भविता तव राजन्मृषा कृतः}


\twolineshloka
{न युक्ता तु मृषावाणी त्वया वक्तुमरिंदम}
{तथा मयाऽप्यभिहितं मिथ्या कर्तुं न शक्यते}


\twolineshloka
{संश्रुतं च मया पूर्वं ददानीत्यविचारितम्}
{तद्गृह्णीष्वाविचारेण यदि सत्ये स्थितो भवान्}


\twolineshloka
{इहागम्य हि मां राजञ्जाप्यं फलमयाचथाः}
{तन्मे निसृष्टं गृह्णीष्व भव सत्येस्थितोपि च}


\twolineshloka
{नायं लोकोऽस्ति न परो न च पूर्वान्स तारयेत्}
{कुत एवापरान्राजन्मृषावादपरायणः}


\twolineshloka
{न यज्ञाध्ययने दानं नियमास्तारयन्ति हि}
{यथा सत्यं परे लोके तथेह पुरुषर्षभ}


\twolineshloka
{तपांसि यानि चीर्णानि चरिष्यन्ति च यत्तपः}
{समाशतैः सहस्रैश्च तत्सत्यान्न विशिष्यते}


\twolineshloka
{सत्यमेकं परं ब्रह्म सत्यमेकं परं तपः}
{सत्यमेकं परो यज्ञः सत्यमेकं परं श्रुतम्}


\twolineshloka
{सत्यं वेदेषु जागर्ति फलं सत्ये परं स्मृतम्}
{तपो धर्मो दमश्चैव सर्वं सत्ये प्रतिष्ठितम्}


\twolineshloka
{सत्यं वेदास्तथाङ्गानि सत्यं यज्ञास्तथा विधिः}
{व्रतचर्या तथा सत्यमोंकारः सत्यमेव च}


\twolineshloka
{प्राणिनां जननं सत्यं सत्यं सन्ततिरेव च}
{सत्येन वायुरभ्येति सत्येन तपते रविः}


\twolineshloka
{सत्येन चाग्निर्दहति स्वर्गः सत्ये प्रतिष्ठितः}
{सत्यं यज्ञस्तपो वेदाः स्तोभा मन्त्राः सरस्वती}


\twolineshloka
{तुलामारोपितो धर्मः सत्यं चैवेति नः श्रुतम्}
{समां कक्षां धारयतो यः सत्यं ततोऽधिकम्}


\twolineshloka
{यतो धर्मस्ततः सत्यं सर्वं सत्येन वर्धते}
{किमर्थमनृतं कर्म कर्तुं राजंस्त्वमिच्छसि}


\twolineshloka
{सत्ये कुरु स्थिरं भावं मा राजन्ननृतं कृथाः}
{कस्मात्त्वमनृतं वाक्यं देहीति कुरुषेऽशुभम्}


\twolineshloka
{यदि जप्यफलं दत्तं मया नेच्छसि वै नृप}
{स्वधर्मेभ्यः परिभ्रष्टो लोकाननुचरिष्यसि}


\threelineshloka
{संश्रुत्य यो न दित्सेत याचित्वा यश्च नेच्छति}
{उभावानृतिकावेतो न मृपा कर्तुमर्हसि ॥राजोवाच}
{}


\threelineshloka
{योद्धव्यं रक्षितव्यं च क्षत्रधर्मः किल द्विज}
{दातारः क्षत्रियाः प्रोक्ता गृह्णीयां भवतः कथम् ॥ब्राह्मण उवाच}
{}


\threelineshloka
{न च्छन्दयामि ते राजन्नापि ते गृहमाव्रजम्}
{इहागम्य तु याचित्वा न गृह्णीषे पुनः कथम् ॥धर्म उवाच}
{}


\threelineshloka
{अविवादोऽस्तु युवयोर्वित्तं मां धर्ममागतम्}
{द्विजो दानफलैर्युक्तो राजा सत्यफलेन च ॥स्वर्ग उवाच}
{}


\threelineshloka
{स्वर्गं मां विद्धि राजेन्द्र रूपिणं स्वयमागतम्}
{अविवादोऽस्तु युवयोरुमौ तुल्यफलौ युवाम् ॥राजोवाच}
{}


\threelineshloka
{कृतं स्वर्गेण मे कार्यं गच्छ स्वर्ग यथागतम्}
{विप्रो यदीच्छते दातुं चीर्णं गृह्णातु मे फलम् ॥ब्राह्मण उवाच}
{}


\twolineshloka
{बाल्ये यदि स्मादज्ञानान्मया हस्तः प्रसारितः}
{निवृत्तलक्षणं धर्ममुपासे संहितां जपन्}


\fourlineindentedshloka
{निवृत्तं मां चिराद्राजन्विप्रलोभयसे कथम्}
{स्वेन कार्यं करिष्यामि त्वत्तो नेच्छे फलं नृप}
{तपःस्वाध्यायशीलोऽहं निवृत्तश्च प्रतिग्रहात् ॥राजोवाच}
{}


\twolineshloka
{यदि विप्र विसृष्टं ते जप्यस्य फलमुत्तमम्}
{आवयोर्यत्फलं किंचित्सहितं नौ तदस्त्विह}


\twolineshloka
{द्विजाः प्रतिग्रहे युक्ता दातारो राजवंशजाः}
{यदि धर्मः क्षुतो विप्र सहैव फलमस्तु नौ}


\threelineshloka
{मा वा भूत्सह भोज्यं नौ मदीयं फलमाप्नुहि}
{प्रतीच्छ मत्कृतं धर्मं यदि ते मय्यनुग्रहः ॥भीष्म उवाच}
{}


\twolineshloka
{ततो विकृतवैषौ द्वौ पुरुषौ समुपस्थितौ}
{गृहीत्वाऽन्योन्यमावेष्ठ्य कुचेलावूचतुर्वचः}


\twolineshloka
{न मे धारयसीत्येको धारयामीति चापरः}
{इहास्ति नौ विवादोऽयमयं राजाऽनुशासकः}


\threelineshloka
{सत्यं ब्रवीम्यहमिदं न मे धारयते भवान्}
{अनृतं वदसीह त्वमृणं ते धारयाम्यहम्}
{}


\threelineshloka
{तावुभौ सुभृशं तप्तौ राजानमिदमृचतुः}
{परीक्ष्यौ तु यथा स्याव नावामिह विगर्हितौ ॥विरूप उवाच}
{}


\threelineshloka
{घारयामि नरव्याघ्र विकृतस्येह गोः फलम्}
{ददतश्च न गृह्णाति विकृतो मे महीपते ॥विकृत उवाच}
{}


\threelineshloka
{न मे धारयते किंचिद्विरूपोऽयं नराधिप}
{मिथ्या ब्रवीत्ययं हि त्वां सत्याभासं नराधिप ॥राजोवाच}
{}


\threelineshloka
{विरूप किं धारयते भवानस्य ब्रवीतु मे}
{श्रुत्वा तथा करिष्येऽहमिति मे धीयते मनः ॥विरूप उवाच}
{}


\twolineshloka
{शृणुष्वावहितो राजन्यथैतद्धारयाम्यहम्}
{विकृतस्यास्य राजर्षे निखिलेन नराधिप}


\twolineshloka
{अनेन धर्मप्राप्त्यर्थं शुभा दत्ता पुराऽनघ}
{धेनुर्विप्राय राजर्षे तपःस्वाध्यायशीलिने}


\twolineshloka
{तस्याश्चायं मया राजन्फलमभ्येत्य याचितः}
{विकृतेन च मे दत्तं विशुद्धेनान्तरात्मना}


\twolineshloka
{ततो मे सुकृतं कर्म कृतमात्मविशुद्धये}
{गावौ च कपिले क्रीत्वा वत्सले बहुदोहने}


\twolineshloka
{ते चोञ्छवृत्तये राजन्मया समुपवर्जिते}
{यथाविधि यथाश्रद्धं तदस्याहं पुनः प्रभो}


\twolineshloka
{इहाद्यैव प्रयच्छामि गृहीत्वा द्विगुणां फलम्}
{एवं स्यात्पुरुषव्याघ्र कःशुद्धः कोऽत्र दोषवान्}


\twolineshloka
{एवं विवदमानौ स्वस्त्यामिहाभ्यागतौ नृप}
{कुरु धर्ममधर्मं वा विनये नौ समादध}


\threelineshloka
{यदि नेच्छति मे दानं यथा दत्तमनेन वै}
{भवानत्र स्थिरो भूत्वा मार्गे स्थापयिताऽद्य नौ ॥राजोवाच}
{}


\threelineshloka
{दीयमानं न गृह्णासि ऋणं कस्मात्त्वमद्य वै}
{यथैव तेऽभ्यनुज्ञातं यथा गृह्णीष्व माचिरम् ॥विकृत उवाच}
{}


\threelineshloka
{दीयतामित्यनेनोक्तं ददानीति तथा मया}
{नायं मे धारयत्यत्र गच्छतां यत्र वाञ्छति ॥राजोवाच}
{}


\threelineshloka
{ददतोऽस्य न गृह्णासि विषमं प्रतिभाति मे}
{दणड्यो हि त्वं मम मतो नास्त्यत्र खलु संशयः ॥विकृत उवाच}
{}


\threelineshloka
{मयाऽस्य दत्तं राजर्षे गृह्णीयां तत्कथं पुनः}
{को ममात्रापराधो मे दण्डमाज्ञापय प्रभो ॥विरूप उवाच}
{}


\threelineshloka
{दीयमानं यदि मया न गृह्णासि कथंचन}
{नियच्छति त्वां नृपतिरयं धर्मानुशासकः ॥विकृत उवाच}
{}


\threelineshloka
{स्वयं मया याचितेन दत्तं कथमिहाद्य तत्}
{गृह्णीयां गच्छतु भवानभ्यनुज्ञां ददानि ते ॥ब्राह्मण उवाच}
{}


\threelineshloka
{श्रुतमेतत्त्वया राजन्ननयोः कथितं द्वयोः}
{प्रतिज्ञातं मया यत्ते तद्गृहाणाविचारितम् ॥राजोवाच}
{}


\twolineshloka
{प्रस्तुतं सुमहत्कार्यमनयोर्गह्वरं यथा}
{जापकस्य दृढीकारः कथमेतद्भविष्यति}


\twolineshloka
{यदि तावन्न गृह्णामि जापकेनापवर्जितम्}
{कथं न लिप्येयमहं पापेन महताऽद्य वै}


\twolineshloka
{तौ चोवाच स राजर्षिः कृतकार्यौ गमिष्यथः}
{नेदानीं मामिहासाद्य राजधर्मो भवेन्मृषा}


\threelineshloka
{स्वधर्मः परिपाल्यस्तु राज्ञामिति विनिश्चयः}
{विप्रधर्मश्च गहनो मामनात्मानमाविशत् ॥ब्राह्मण उवाच}
{}


\threelineshloka
{गृहाण धारयेऽहं च याचितं संश्रुतं मया}
{न चेद्भहीष्यसे राजञ्शपिष्ये त्वां न संशयः ॥राजोवाच}
{}


\twolineshloka
{धिग्राजधर्मं यस्यायं कार्यस्येह विनिश्चयः}
{इत्यर्थं मे ग्रहीतव्यं कथं तुल्यं भवेदिति}


\threelineshloka
{एष पाणिरपूर्वं मे निक्षेपार्थं प्रसारितः}
{यन्मे धारयसे विप्र तदिदानीं प्रदीयताम् ॥ब्राह्मण उवाच}
{}


\threelineshloka
{संहितां जपता यावान्गुणः कश्चित्कृतो मया}
{तत्सर्वं प्रतिगृह्णीष्व यदि किंचिदिहास्ति मे ॥राजोवाच}
{}


\threelineshloka
{जलमेतन्निपतितं मम पाणौ द्विजोत्तम}
{सममस्तु सहैवास्तु प्रतिगृह्णातु वै भवान् ॥विरूप उवाच}
{}


\twolineshloka
{कामक्रोधौ विद्धि नौ त्वमावाभ्यां कारितो भवान्}
{`जिज्ञासमानौ युवयोर्मनोत्थं तु द्विजोत्तम ॥'}


\twolineshloka
{सहेति च यदुक्तं ते समा लोकास्तवास्य च}
{नायं धारयते किंचिज्जिज्ञासा त्वत्कृते कृता}


\twolineshloka
{कालो धर्मस्तथा मृत्युः कामक्रोधौ तथा युवाम्}
{सर्वमन्योन्यनिष्कर्षे निकृष्टं पश्यतस्तव}


\twolineshloka
{`सर्वेषामुपरिस्थानं ब्रह्मणो व्यक्तजन्मनः}
{युवयोः स्थानमूलं निर्द्वन्द्वममलात्मकम्}


\twolineshloka
{सर्वे गच्छाम यत्र स्वान्स्वाँल्लोकांश्च तथा वयम्}
{'गच्छ लोकाञ्जितान्स्वेन कर्मणा यत्र वाञ्छसि}


\twolineshloka
{` ततो धर्मयमाद्यास्ते वाक्यमूचुर्नपर्द्विजौ}
{अस्माकं यः स्मृतो मूर्धा ब्रह्मलोकमिति स्मृतं}


\threelineshloka
{तत्रस्थौ हि भवन्तौ हि युवाभ्यां निर्जिता वयम्}
{युवयोः काम आपन्नस्तत्काम्यमविशङ्कया ॥'भीष्म उवाच}
{}


\twolineshloka
{जापकानां फलावाप्तिर्मया ते संप्रदर्शिता}
{गतिः स्थानं च लोकाश्च जापकेन यथा जिताः}


\twolineshloka
{प्रयाति संहिताध्यायी ब्रह्माणं परमेष्ठिनम्}
{अथवाऽग्निं समायाति सूर्यमाविशतेऽपि वा}


\twolineshloka
{स तैजसेन भावेन यदि तत्र रमत्युत}
{गुणांस्तेषां समाधत्ते रागेण प्रतिमोहितः}


\twolineshloka
{एवं सोमे तथा वायौ भूम्याकाशशरीरगः}
{सरागस्तत्र वसति गुणांस्तेषां समाचरन्}


\twolineshloka
{अथ तत्र विरागी स परं गच्छत्यसंशयम्}
{परमव्ययमिच्छन्स तमेवाविशते पुनः}


\twolineshloka
{अमृताच्चामृतं प्राप्तः शान्तीभूतो निरात्मवान्}
{ब्रह्मभूतः स निर्द्वन्द्वः सुखी शान्तो निरामयः}


\twolineshloka
{ब्रह्मस्थानमनावर्तमेकमक्षरसंज्ञकम्}
{अदुःखमजरं शान्तं स्थानं तत्प्रतिपद्यते}


\twolineshloka
{चतुर्भिर्लक्षणैर्हीनं तथा पड्भिः सषोडशैः}
{पुरुषं तमतिक्रम्य आकाशं प्रतिपद्यते}


\twolineshloka
{अथ नेच्छति रागात्मा सर्वं तदधितिष्ठति}
{यच्च प्रार्थयते तच्च मनसा प्रतिपद्यते}


\twolineshloka
{अथवा चेक्षते लोकान्सर्वान्निरयसंज्ञितान्}
{निस्पृहः सर्वतो मुक्तस्तत्र वै रमते सुखम्}


\twolineshloka
{एवमेषा महाराज जापकस्य गतिर्यथा}
{एतत्ते सर्वमाख्यातं किं भूयः श्रोतुमर्हसि}


\chapter{अध्यायः १९८}
\twolineshloka
{युधिष्ठिर उवाच}
{}


\twolineshloka
{किमुत्तरं तदा तौ स्म चक्रतुस्तस्य भाषिते}
{ब्राह्मणो वाऽथवा राजा तन्मे ब्रूहि पितामह}


\threelineshloka
{अथवा तौ गतौ तत्र यदेतत्कीर्तितं त्वया}
{संवादो वा तयोः कोऽभूत्किं वा तौ तत्र चक्रतुः ॥भीष्म उवाच}
{}


\twolineshloka
{तथेत्येवं प्रतिश्रुत्य धर्मं संपूज्य जापकः}
{यमं कालं च मृत्युं च स्वर्गं संपूज्य चार्हतः}


\twolineshloka
{पूर्वं ये चापरे तत्र समेता ब्राह्मणर्षभाः}
{सर्वान्संपूज्य शिरसा राजानं सोऽब्रवीद्द्विजः}


\twolineshloka
{फलेनानेन संयुक्तो राजर्षे गच्छ मुख्यताम्}
{भवता चाभ्यनुज्ञातो जपेयं भूय एव ह}


\threelineshloka
{वरश्च मम पूर्वं हि दत्तो देव्या महाबल}
{श्रद्धा ते जपतो नित्यं भवत्विति विशांपते ॥राजोवाच}
{}


\threelineshloka
{यद्येवं सफला सिद्धिः श्रद्धा च जपितुं तव}
{गच्छ विप्र मया सार्धं जापकं फलमाप्नुहि ॥ब्राह्मण उवाच}
{}


\threelineshloka
{कृतः प्रयत्नः सुमहान्सर्वेषां सन्निधाविह}
{सह तुल्यफलावावां गच्छावो यत्र नौ गतिः ॥भीष्म उवाच}
{}


\twolineshloka
{व्यवसायं तयोस्तत्र विदित्वा त्रिदशेश्वरः}
{सह देवैरुपययौ लोकपालैस्तथैव च}


\twolineshloka
{साध्याश्च विश्वे मरुतो वाक्यानि सुमहान्ति च}
{नद्यः शैलाः समुद्राश्च तीर्थानि विविधानि च}


\twolineshloka
{तपांसि संयोगविधिर्वेदास्तोभाः सरस्वती}
{नारदः पर्वतश्चैव विश्वावसुर्हहाहुहूः}


\twolineshloka
{गन्धर्वश्चित्रसेनश्च परिवारगणैर्युतः}
{नागाः सिद्धाश्च मुनयो देवदेवः प्रजापतिः}


\threelineshloka
{`आजगाम च देवेशो ब्रह्मा वेदमयोऽव्ययः}
{'विष्णुः सहस्रशीर्षश्च देवोऽचिन्त्यः समागमत्}
{अवाद्यन्तान्तरिक्षे च भेर्यस्तूर्याणि वा विभो}


\twolineshloka
{पुष्पवर्षाणि दिव्यानि तत्र तेषां महात्मनाम्}
{ननृतुश्चाप्सरः सङ्घास्तत्रतत्र समन्ततः}


\threelineshloka
{अथ स्वर्गस्तथा रूपी ब्राह्मणं वाक्यमब्रवीत्}
{संसिद्धस्त्वं महाभाग त्वं च सिद्धस्तथा नृप ॥भीष्म उवाच}
{}


\twolineshloka
{अथ तौ सहितौ राजन्नन्योन्यस्य विधानतः}
{विषयप्रतिसंहारमुभावेव प्रचक्रतुः}


\twolineshloka
{प्राणापानौ तथोदानं समानं व्यानमेव च}
{एवं तौ मनसि स्थाप्य दधतुः प्राणयोर्मनः}


\twolineshloka
{उपस्थितकृतौ तौ च नासिकाग्रमधो भ्रुवोः}
{भ्रुकुट्याक्ष्णोश्च मनसा शनैर्धारयतस्तदा}


\twolineshloka
{निश्चेष्टाभ्यां शरीराभ्यां स्थिरदृष्टी समाहितौ}
{जितासनौ समाधाय र्मूर्धन्यात्मानमेव च}


\twolineshloka
{तालुदेशमथोद्दाल्य ब्राह्मणस्य महात्मनः}
{ज्योतिर्ज्वाला सुमहती जगाम त्रिदिवं तदा}


\twolineshloka
{हाहाकारस्तथा दिक्षु सर्वासु सुमहानभूत्}
{तज्ज्योतिः स्तूयमानं स्म ब्रह्माणं प्राविशत्तदा}


\twolineshloka
{ततः स्वागतमित्याह तत्तेजः प्रपितामहः}
{प्रादेशमात्रं पुरुषं प्रत्युद्गम्य विशांपते}


\twolineshloka
{भूयश्चैवापरं प्राह वचनं मधुरं स्मयन्}
{जापकैस्तुल्यफलता योगानां नात्र संशयः}


\twolineshloka
{योगस्य तावदेतेभ्यः प्रत्यक्षं फलदर्शनम्}
{जापकानां विशिष्टं तु प्रत्युत्थानं समाहितम्}


\twolineshloka
{उष्यतां मयि चेत्युक्त्वा व्याददे स ततो मुखम्}
{अथास्यं प्रविवेशास्य ब्राह्मणो विगतज्वरः}


\twolineshloka
{राजाऽप्येतेन विधिना भगवन्तं पितामहम्}
{यथैव द्विजशार्दूलस्तथैव प्राविशत्तदा}


% Check verse!
स्वयंभुवमथो देवा अभिवाद्य ततोऽब्रुवन्
\twolineshloka
{जापकार्थमयं यत्नो यदर्थं वयमागताः}
{कृतपूजाविमौ तुल्यौ त्वया तुल्यफलान्वितौ}


\threelineshloka
{योगजापकयोस्तुल्यं फलं सुमहदद्य वै}
{सर्वांल्लोकानतिक्रम्य गच्छेतां यत्र वाञ्छितम् ॥ब्रह्मोवाच}
{}


\twolineshloka
{महास्मृतिं पठेद्यस्तु तथैवानुस्मृतिं शुभाम्}
{तावप्येतेन विधिना गच्छेतां मत्सलोकताम्}


\fourlineindentedshloka
{यश्च योगे भवेद्भक्तः सोऽपि नास्त्यत्र संशयः}
{विधिनानेन देहान्ते मम लोकानवाप्नुयात्}
{साधये गम्यतां चैव यथा स्थानानि सिद्धये ॥भीष्म उवाच}
{}


\twolineshloka
{इत्युक्त्वा स तदा देवस्तत्रैवान्तरधीयत}
{आमन्त्र्य च ततो देवा ययुः स्वंस्वं निवेशनम्}


\twolineshloka
{ते च सर्वे महात्मानो धर्मं सत्कृत्य तत्र वै}
{पृष्ठतोऽनुययू राजन्सर्वे सुप्रीतचेतसः}


\twolineshloka
{एतत्फलं जापकानां गतिश्चैषा प्रकीर्तिता}
{यथाश्रुतं महाराज किं भूयः श्रोतुमिच्छसि}


\chapter{अध्यायः १९९}
\twolineshloka
{युधिष्ठिर उवाच}
{}


\threelineshloka
{किं फलं ज्ञानयोगस्य वेदानां नियमस्य च}
{भूतात्मा च कथं ज्ञेयस्तन्मे ब्रूहि पितामह ॥भीष्म उवाच}
{}


\twolineshloka
{अत्राप्युदाहरन्तीममितिहासं पुरातनम्}
{मनोः प्रजापतेर्वादं महर्षेश्च बृहस्पतेः}


\twolineshloka
{प्रजापतिं श्रेष्ठतमं प्रजानांदेवर्षिसङप्रवरो महर्षिः}
{बृहस्पतिः प्रश्नमिमं पुराणंप्रपच्छ शिष्योऽथ गुरुं प्रणम्य}


\twolineshloka
{यत्कारणं मन्त्रविधिः प्रवृत्तोज्ञाने फलं यत्प्रवदन्ति विप्राः}
{यन्मन्त्रशब्दैरकृतप्रकाशंतदुच्यतां मे भगवन्यथावत्}


\twolineshloka
{यत्स्तोत्रशास्त्रागममन्त्रिविद्भिर्यज्ञैरनेकैरथ गोप्रदानैः}
{फलं महद्भिर्यदुपास्यते चकिं तत्कथं वा भविता क्व वा तत्}


\twolineshloka
{मही महीजाः पवनोऽन्तरिक्षंजलौकसश्चैव जलं तथा द्यौः}
{दिवौकसश्चापि यतः प्रसूतास्तदुच्यतां मे भगवन्पुराणम्}


\twolineshloka
{ज्ञानं यतः प्रार्थयते नरो वैततस्तदर्था भवति प्रवृत्तिः}
{न चाप्यहं वेद परं पुराणंमिथ्याप्रवृत्तिं च कथं नु कुर्याम्}


\twolineshloka
{ऋक्सामसङ्घांश्च यजूंषि चाहंछन्दांसि नक्षत्रगतिं निरुक्तम्}
{अधीत्य च व्याकरणं सकल्पंशिक्षां च भूतप्रकृतिं न वेद्मि}


\twolineshloka
{स मे भवाञ्शंसतु सर्वमेतत्सामान्यशब्दैश्च विशेषणैश्च}
{स मे भवाञ्शंसतु तावदेतज्ज्ञाने फलं कर्मणि वा यदस्ति}


\threelineshloka
{यथा च देहाच्च्यवते शरीरीपुनः शरीरं च यथाऽभ्युपैति}
{मनुरुवाच}
{यद्यत्प्रियं यस्य सुखं तदाहुस्तदेव दुःखं प्रवदन्त्यनिष्टम्}


\twolineshloka
{इष्टं च मे स्यादितरच्च न स्यादेतत्कृते कर्मविधिः प्रवृत्तः}
{इष्टं त्वनिष्टं च न मां भजेतेत्येतत्कृते ज्ञानविधिः प्रवृत्तः}


% Check verse!
[कामात्मकाश्छन्दसि कर्मयोगाएभिर्विमुक्तः परमश्नुवीत्

नानाविधे कर्मपथे सुखार्थीनरः प्रवृत्तो निरयं प्रयाति

]बृहस्पतिरुवाच

इष्टं त्वनिष्टं च सुखासुखे चसाशीस्तपश्छन्दति कर्मभिश्च ॥मनुरुवाच


\twolineshloka
{एभिर्विमुक्तः परमाविवेशएतत्कृते कर्मविधिः प्रवृत्तः}
{[कामात्मकांश्छन्दति कर्मयोगएभिर्विमुक्तः परमाददीत ॥]}


\twolineshloka
{आत्मादिभिः कर्मभिरिध्यमानोधर्मे प्रवृत्तो द्युतिमान्सुखार्थी}
{परं हि तत्कर्मफलादपेतंनिराशिषो यत्पदमाप्नुवन्ति}


\twolineshloka
{प्रजाः सृष्टा मनसा कर्मणा चद्वावेवैतौ सत्पथौ लोकजुष्टौ}
{दृष्टं कर्माशाश्वतं चान्तवच्चमनस्त्यागे कारणं नान्यदस्ति}


\threelineshloka
{`कामात्मकौ छन्दसि कामभोगावेभिर्वियुक्तः परमश्नुवीत}
{नानाविधे कर्मफले सुखार्थीनरः प्रमत्तो न परं प्रयाति}
{फलं हि तत्कर्मफलादपेतंनिराशिषो ब्रह्म परं ह्युपेतम् ॥'}


\twolineshloka
{स्वेनात्मना चक्षुरिव प्रणेतानिशात्यये तमसा संवृतात्मा}
{ज्ञानं तु विज्ञानगुणेन युक्तंकर्माशुभं पश्यति वर्जनीयम्}


\twolineshloka
{सर्पान्कृशाग्राणि तथोदपानंत्वा मनुष्याः परिवर्जयन्ति}
{अज्ञामतस्तत्र पतन्ति मूढाज्ञाने फलं पश्य यथा विशिष्टम्}


\twolineshloka
{कृत्स्नस्तु मन्त्रो विधिवत्प्रयुक्तोयज्ञा यथोक्तास्त्विह दक्षिणाश्च}
{अन्नप्रदानं मनसः समाधिःपञ्चात्मकं कर्मफलं वदन्ति}


\twolineshloka
{गुणात्मकं कर्म वदन्ति वेदास्तस्मान्मन्त्रो मन्नपूर्वं हि कर्म}
{विधिर्विधेयं मनसोपपत्तिःफलस्य भोक्ता तु तथा शरीरी}


\threelineshloka
{शब्दाश्च रूपाणि रसाश्च पुण्याः}
{स्पर्शाश्च गन्धाश्च शुभास्तथैव}
{नरोऽत्र हि स्थानगतः प्रभुः स्यादेतत्फलं सिद्ध्यति कर्मणोऽस्य}


\twolineshloka
{यद्यच्छरीरेण करोति कर्मशरीरयुक्तः समुपाश्नुते तत्}
{शरीरमेवायतनं सुखस्यदुःखस्य चाप्यायतनं शरीरम्}


\twolineshloka
{वाचा तु यत्कर्म करोति किंचिद्वाचैव सर्वं समुपाश्नुते तत्}
{मनस्तु यत्कर्म करोति किंचिन्मनःस्थ एवायमुपाश्नुते तत्}


\twolineshloka
{यथायथा कर्मगुणां फलार्थीकरोत्ययं कर्मफले निविष्टः}
{तथातथाऽयं गुणसंप्रयुक्तःशुभाशुभं कर्मफलं भुनस्ति}


\twolineshloka
{मत्स्यो यथा स्रोत इवाभिपातीतथा कृतं पूर्वमुपैति कर्म}
{शुभे त्वसौ तुष्यति दुष्कृते तुन तुष्यते वै परमः शरीरी}


\twolineshloka
{यतो जगत्सर्वमिदं प्रसूतंज्ञात्वाऽऽत्मवन्तो ह्युपयान्ति शान्तिम्}
{यन्मन्त्रशब्देरकृतप्रकाशंतदुच्यमानं शणु मे परं यत्}


\twolineshloka
{रसैर्विमुक्तं विविधैश्च गन्धैरशब्दमस्पर्शमरूपवच्च}
{अग्राह्यमव्यक्तमवर्णमेकंपञ्चप्रकारान्ससृजे प्रजानाम्}


\twolineshloka
{न स्त्री पुमान्नापि नपुंसकं चन सन्न चासत्सदसच्च तन्न}
{पश्यन्ति तद्ब्रह्मविदो मनुष्यास्तदक्षरं न क्षरतीति विद्धि}


\chapter{अध्यायः २००}
\twolineshloka
{मनुरुवाच}
{}


\twolineshloka
{अक्षरात्खं ततो वायुस्ततो ज्योतिस्ततो जलम्}
{जलात्प्रसूता जगती जगत्या जायते जगत्}


\twolineshloka
{इमे शरीरैर्जलमेव गत्वाजलाच्च तेजः पवनान्तरिक्षम्}
{खाद्वै निवर्तन्ति न भाविनस्तेये भाविनस्ते परमाप्नुवन्ति}


\twolineshloka
{नोष्णं न शीतं मृदु नापि तीक्ष्णंनाम्लं कषायं मधुरं न तिक्तम्}
{न शब्दवन्नापि च गन्धवत्तन्न रूपवत्तत्परमस्वभावम्}


\twolineshloka
{स्पर्शं तनुर्वेद रसं च जिह्वाघ्राणं च गन्धाञ्श्रवणे च शब्दान्}
{रूपाणि चक्षुर्नच तत्परं यद्गृह्णन्त्यनध्यात्मविदो मनुष्याः}


\twolineshloka
{निवर्तयित्वा रसनां रसेभ्योघ्राणं च गन्धाच्छ्रवणे च शब्दात्}
{स्पर्शात्तनुं रूपगुणात्तु चक्षुस्ततः परं पश्यति तत्स्वभावम्}


\twolineshloka
{यतो गृहीत्वा हि करोति यच्चयस्मिंश्च यामारभते प्रवृत्तिम्}
{यस्मै च यद्येन च यश्च कर्तायत्कारणं तं स्वमुपेयमाहुः}


\twolineshloka
{यद्वाऽप्यभूद्व्यापकं साधकं चयन्मन्त्रवत्स्थास्यति चापि लोके}
{यः सर्वहेतुः परमार्थकारीतत्कारणं कार्यमतो यदन्यत्}


\twolineshloka
{यथा हि कश्चित्सुकृतैर्मनुष्यःशुभाशुभं प्राप्नुते चाविरोधात्}
{एवं शरीरेषु शुभाशुभेषुस्वकर्मभिर्ज्ञानमिदं निवद्धम्}


\threelineshloka
{यथा प्रदीप्तः पुरतः प्रदीपः}
{प्रकाशमन्यस्य करोति दीप्यन्}
{तथेह पञ्चेन्द्रियदीपवृक्षाज्ञानप्रदीप्ताः परवन्त एव}


\twolineshloka
{यथा च राज्ञो बहवो ह्यमात्याःपृथक् प्रमाणं प्रवदन्ति युक्ताः}
{तद्वच्छरीरेषु भवन्ति पञ्चज्ञानैकदेशाः परमः स तेभ्यः}


\twolineshloka
{यथार्चिषोऽग्नेः पवनस्य वेगामरीचयोऽर्कस्य नदीषु चापः}
{गच्छन्ति चायान्ति च संयताश्चतद्वच्छरीराणि शरीरिणां तु}


\twolineshloka
{यथा च कश्चित्परशुं गृहीत्वाधूमं न पश्येज्ज्वलनं च काष्ठे}
{तद्वच्छरीरोदरपाणिपादंछित्त्वा न पश्यन्ति ततो यदन्य}


\twolineshloka
{तान्येव काष्ठानि यथा विमथ्यधूमं च पश्येज्ज्वलनं च योगात्}
{तद्वत्सुबुद्धिः सममिन्द्रियार्थैर्बुद्धः परं पश्यति तत्स्वभावम्}


\twolineshloka
{यथात्मनोऽङ्गं पतितं पृथिव्यांस्वप्नान्तरे पश्यति चात्मनोऽन्यत्}
{श्रोत्रादियुक्तः सुमनाः सुबुद्धिर्लिङ्गात्तथा गच्छति लिङ्गमन्यत्}


\twolineshloka
{उत्पत्तिवृद्धिक्षयमसन्निपातैर्न युज्यतेऽसौ परमः शरीरी}
{अनेन लिङ्गेन तु लिङ्गमन्यद्गच्छत्यदृष्टः प्रतिसन्धियोगात्}


\twolineshloka
{न चक्षुषा पश्यति रूपमात्मनोन चापि संस्पर्शमुपैति किंचित्}
{न चापि तैः साधयते स्वकार्यंते तं न पश्यन्ति स पश्यते तान्}


\twolineshloka
{यथा समीपे ज्वलतोऽनलस्यसंतापजं रूपमुपैति कश्चित्}
{न चान्तरा रूपगुणं बिभर्तितथैव तद्दृश्यते रूपमस्य}


\twolineshloka
{तथा मनुष्यः परिमुच्य कायमदृश्यमन्यद्विशते शरीरम्}
{विसृज्य भूतेषु महत्सु देहंतदाश्रयं चैव बिभर्ति रूपम्}


\twolineshloka
{खं वायुमग्निं सलिलं तथोर्वीसमन्ततोऽभ्याविशते शरीरी}
{नान्याश्रयाः कर्मसु वर्तमानाःश्रोत्रादयः पञ्चगुणाञ्श्रयन्ते}


\twolineshloka
{श्रोत्रं नभो घ्राणमथो पृथिव्यास्तेजोमयं रूपमथो विपाकः}
{जलाश्रयं तेज उक्तं रसं चवाय्वात्मकः स्पर्शकृतो गुणश्च}


\twolineshloka
{महत्सु भूतेषु च सन्ति पञ्चपञ्चेन्द्रियार्थेषु तथेन्द्रियाणि}
{सर्वाणि चैतानि मनोनुगानिबुद्धिं मनोऽन्वेति मतिः स्वभावम्}


\twolineshloka
{शुभाशुभं कर्म कृतं यदस्यतदेव प्रेत्याददतेऽन्यदेहे}
{मनोऽनुवर्तन्ति परावराणिजलौकसः स्रोत इवानुकूलम्}


\twolineshloka
{चलं यथा दृष्टिपथं परैतिसूक्ष्मं महद्रूपमिवावभाति}
{तातप्यमानो न पतेच्च धीरःपरं तथा बुद्धिपथं परैति}


\chapter{अध्यायः २०१}
\twolineshloka
{मनुरुवाच}
{}


\twolineshloka
{यदिन्द्रियैस्तूपगतैः पुरस्तात्प्राप्तान्गुणान्संस्मरते चिराय}
{तेष्विन्द्रियेषूपहतेषु पश्चात्स बुद्धिरूपः परमः स्वभावः}


\twolineshloka
{य इन्द्रियार्थान्युगपत्समन्तान्नावेक्षते कृत्स्नशस्तुल्यकालम्}
{यथाक्रमं संचरते स विद्वांस्तस्मात्स एकः परमः शरीरी}


\threelineshloka
{रजस्तमः सत्वमथो तृतीयंगच्छत्यसौ ज्ञानगुणान्विरूपान्}
{`न तैर्निबद्धः स तु बध्नाति विश्वंन चानुयातीहागुणान्परात्मा}
{'तथेन्द्रियाण्याविशते शरीरीहुताशनं वायुरिवेन्धनस्थम्}


\twolineshloka
{न चक्षुषा पश्यति रूपमात्मनोन पश्यति स्पर्शनमिन्द्रियेन्द्रियम्}
{न श्रोत्रलिङ्गं श्रवणेन दर्शनंतथा कृतं पश्यति तद्विनश्यति}


\twolineshloka
{श्रोत्रादीनि न पश्यन्ति स्वंस्वमात्मानमात्मना}
{सर्वज्ञः सर्वदर्शी च क्षेत्रज्ञस्तानि पश्यति}


\twolineshloka
{यथा हिमवतः पार्श्वे पृष्ठं चन्द्रमसो यथा}
{न दृष्टपूर्वं मनुजैर्न च तन्नास्ति तावता}


\twolineshloka
{तद्वद्भूतेषु भूतात्मा सूक्ष्मो ज्ञानात्मवानसौ}
{अदृष्टपूर्वश्चक्षुर्म्यां न चासौ नास्ति तावता}


\twolineshloka
{पश्यन्नपि यथा लक्ष्म जनः सोमेन विन्दति}
{एवमस्ति न चोत्पन्नं न च तन्न परायणम्}


\twolineshloka
{रूपवन्तमरूपत्वादुदयास्तमने बुधाः}
{धिया समनुपश्यन्ति तद्गताः सवितुर्गतिम्}


\twolineshloka
{तथा बुद्धिप्रदीपेन दूरस्थं सुविपश्चितः}
{प्रत्यासन्नं निषीदन्ति ज्ञेयं ज्ञानाभिसंहितम्}


\twolineshloka
{न हि स्वल्वनुपायेन कश्चिदर्थोऽभिसिद्ध्यति}
{सूत्रजालैर्यथा मत्स्यान्बध्नन्ति जलजीविनः}


\twolineshloka
{मृगैर्मृगाणां ग्रहणं पक्षिणां पक्षिभिर्यथा}
{गजानां च गजैरेव ज्ञेयं ज्ञानेन गृह्यते}


\twolineshloka
{अहिरेव ह्यहेः पादान्पश्यतीति निदर्शनम्}
{तद्वन्मूर्तिषु मूर्तिस्थं ज्ञेयं ज्ञानेन पश्यति}


\twolineshloka
{नोत्सहन्ते यथा वेत्तुमिन्द्रियैरिन्द्रियाण्यपि}
{तथैवेह परा बुद्धिः परं बुद्ध्या न पश्यति}


\twolineshloka
{यथा चन्द्रो ह्यमावास्यामलिङ्गत्वान्न दृश्यते}
{न च नाशोऽस्य भवति तथा विद्धि शरीरिणम्}


\twolineshloka
{क्षीणकोशो ह्यमावास्यां चन्द्रमा न प्रकाशते}
{तद्वन्मूर्तिविमुक्तोऽसौ शरीरी नोपलभ्यते}


\twolineshloka
{यथा कोशान्तरं प्राप्य चन्द्रमा भ्राजते पुनः}
{तद्वल्लिङ्गान्तरं प्राप्य शरीरी भ्राजते पुनः}


\twolineshloka
{जन्म बुद्धिः क्षयश्चास्य प्रत्यक्षेणोपलभ्यते}
{सा तु चन्द्रमसो व्यक्तिर्न तु तस्य शरीरिणः}


\twolineshloka
{उत्पत्तिवृद्धिव्ययतो यथा स इति गृह्यते}
{चन्द्र एव त्वमावास्यां तथा भवति मूर्तिमान्}


\twolineshloka
{नाभिसर्पद्विमुञ्चद्वा शशिनं दृश्यते तमः}
{विसृजंश्चोपसर्पंश्च तद्वत्पश्य शरीरिणम्}


\twolineshloka
{यथा चन्द्रार्कसंयुक्तं तमस्तदुपलभ्यते}
{तद्वच्छरीरसंयुक्तं ज्ञानं तदुपलभ्यते}


\twolineshloka
{यथा चन्द्रार्कनिर्मुक्तः स राहुर्नोपलभ्यते}
{तद्वच्छरीरनिर्मुक्तः शरीरी नोपलभ्यते}


\twolineshloka
{यथा चन्द्रो ह्यमावास्यां नक्षत्रैर्युज्यते गतः}
{तद्वच्छरीरनिर्मुक्तः फलैर्युज्यति कर्मणः}


\chapter{अध्यायः २०२}
\twolineshloka
{मनुरुवाच}
{}


\twolineshloka
{यथा व्यक्तमिदं शेते स्वप्ने चरति चेतनम्}
{ज्ञानमिन्द्रियसंयुक्तं तद्वत्प्रेत्य भवाभवौ}


\twolineshloka
{यथाऽम्भसि प्रसन्ने तु रूपं पश्यति चक्षुषा}
{तद्वत्प्रसन्नेन्द्रियवाञ्ज्ञेयं ज्ञानेन पश्यति}


\twolineshloka
{स एव लुलिते तस्मिन्यथा रूपं न पश्यति}
{तथेन्द्रियाकुलीभावे ज्ञेयं ज्ञाने न पश्यति}


\twolineshloka
{अबुद्धिरज्ञानकृता अबुद्ध्या दूष्यते मनः}
{दुष्टस्य मनसः पञ्च संप्रदुष्यन्ति मानसाः}


\twolineshloka
{अज्ञानतृप्तो विषयेष्ववगाढो न पश्यति}
{स दृष्ट्वैव तु पूतात्मा विषयेभ्यो निवर्तते}


\twolineshloka
{तर्षच्छेदो न भवति पुरुषस्येह कल्मषात्}
{निवर्तते तदा तर्षः पापमन्तर्गतं यदा}


\twolineshloka
{`अन्तर्गतेन पापेन दह्यमानेन चेतसा}
{शुभाशुभविकारेण न स भूयोऽभिजायते ॥'}


\twolineshloka
{विषयेषु तु संसर्गाच्छाश्वतस्य तु संश्रयात्}
{मनसा चान्यथा काड्क्षन्परं न प्रतिपद्यते}


\twolineshloka
{ज्ञानमुत्पद्यते पुंसां क्षयात्पापस्य कर्मणः}
{अथादर्शतलप्रख्ये पश्यत्यात्मानमात्मनि}


\twolineshloka
{प्रसृतैरिन्द्रियैर्दुःखी तैरेव नियतैः सुखी}
{तस्मादिन्द्रियचोरेभ्यो यच्छेदात्मानमात्मना}


\twolineshloka
{इन्द्रियेभ्यो मनः पूर्वं बुद्धिः परतरा ततः}
{बुद्धेः परतरं ज्ञानं ज्ञानात्परतरं परम्}


\twolineshloka
{अव्यक्तात्प्रसृतं ज्ञानं ततो बुद्धिस्ततो मनः}
{मनः श्रोत्रादिभिर्युक्तं शब्दादीन्साधु पश्यति}


\threelineshloka
{यस्तांस्त्यजति शब्दादीन्सर्वाश्च व्यक्तयस्तथा}
{`प्रसृतानीन्द्रियाण्येव प्रतिसंहरति कूर्मवत्}
{'विमुञ्चत्याकृतिग्रामांस्तान्मुक्त्वाऽमृतमश्नुते}


\threelineshloka
{उद्यन्हि सविता यद्वत्सृजते रश्मिमण्डलम्}
{`दृश्यते मण्डलं तस्य न च दृश्येत मण्डली}
{तद्वद्देहस्तु संदृश्य आत्माऽदृश्यः परः सदा}


\twolineshloka
{ग्रस्तं ह्युद्गिरते नित्यमुद्गीथं वेत्ति नित्यशः}
{बाल्ये रथाभ्यां योगेन तत्वज्ञानं तु संमतम् ॥'}


\twolineshloka
{स एवास्तमुपागच्छंस्तदेवात्मनि यच्छति}
{`आदत्ते सर्वभूतानां रसभूतं विकासवान् ॥'}


\threelineshloka
{अन्तरात्मा तथा देहमाविश्येन्द्रियरश्मिभिः}
{प्राप्येन्द्रि गुणान्पञ्च सोऽस्तमावृत्त्य गच्छति}
{`रश्मिमण्ड हीनस्तु न चासौ नास्ति तावता ॥'}


\twolineshloka
{प्रणीतं कर्मणा मार्गं नीयमानः पुनः पुनः}
{प्राप्नोत्ययं कर्मफलं प्रवृत्तं धर्ममाप्तवान्}


\twolineshloka
{विषया विनिवर्तन्ते निराहारस्य देहिनः}
{रसवर्जं रसोऽप्यस्य परं दृष्ट्वा निवर्तते}


\twolineshloka
{बुद्धिः कर्मगुणैर्हीना यदा मनसि वर्तते}
{तदा संपद्यते ब्रह्म तत्रैव प्रलयं गतम्}


\twolineshloka
{अस्पर्शनमशृण्वानमनास्वादमदर्शनम्}
{अघ्राणमवितर्कं च सत्वं प्रविशते परम्}


\twolineshloka
{`अव्यक्तात्प्रसृतं ज्ञानं ततो बुद्धिस्ततो मनः}
{आत्मनः प्रसृता बुद्धिरव्यक्तं ज्ञानमुच्यते}


\twolineshloka
{तस्माद्बुद्धिः स्मृता तज्ज्ञैर्मनस्तस्मात्ततः स्मृतम्}
{तस्मादाकृतयः पञ्च मनः परममुच्यते}


\threelineshloka
{तस्मात्परतरा बुद्धिर्ज्ञानं तस्मात्परं स्मृतम्}
{ततः सूक्ष्मस्ततो ह्यात्मा तस्मात्परतरं न च}
{इन्द्रियाणि निरीक्षन्ते मनसैतानि सर्वशः ॥'}


\twolineshloka
{मनस्याकृतयो मग्ना मनस्त्वभिगतं मतिम्}
{मतिस्त्वभिगता ज्ञानं ज्ञानं चाभिगतं महत्}


\twolineshloka
{नेन्द्रियैर्मनसः सिद्धिर्न बुद्धिं बुध्यते मनः}
{न बुद्धिर्बुध्यतेऽव्यक्तं सूक्ष्मं त्वेतानि पश्यति}


\chapter{अध्यायः २०३}
\twolineshloka
{मनुरुवाच}
{}


\twolineshloka
{दुःखोपघाते शारीरे मानसे चाप्युपस्थिते}
{यस्मिन्न शक्यते कर्तुं यत्नस्तं नानुचिन्तयेत्}


\twolineshloka
{भैषज्यमेतद्दुःखस्य यदेतन्नानुचिन्तयेत्}
{चिन्त्यमानं हि चाभ्येति भूयश्चापि प्रवर्तते}


\twolineshloka
{प्रज्ञया मानसं दुःखं हन्याच्छारीरमौषधैः}
{एतद्विज्ञानसामर्थ्यं न बालैः समतामियात्}


\twolineshloka
{अनित्यं यौवनं रूपं जीवितं द्रव्यसंचयः}
{आरोग्यं प्रियसंवासो गृध्येत्तत्र न पण्डितः}


\twolineshloka
{न जानपदिकं दुःखमेकः शोचितुमर्हति}
{अशोचन्प्रतिकुर्वीत यदि पश्येदुपक्रमम्}


\twolineshloka
{सुखाद्बहुतरं दुःखं जीविते नास्ति संशयः}
{स्रिग्धस्य चेन्द्रियार्थेषु मोहान्मरणमप्रियम्}


\twolineshloka
{परित्यजति यो दुःखं सुखं वाऽप्युभयं नरः}
{अभ्येति ब्रह्म सोत्यन्तं न ते शोचन्ति पण्डिताः}


\twolineshloka
{दुःखमर्था हि युज्यन्ते पालने न च ते सुखम्}
{दुःखेन चाधिगम्यन्ते नाशमेषां न चिन्तयेत्}


\twolineshloka
{ज्ञानं ज्ञेयाभिर्निवृत्तं विद्धि ज्ञानगुणं मनः}
{प्रज्ञाकरणसंयुक्तं ततो बुद्धिः प्रवर्तते}


\twolineshloka
{यदा कर्मगुणोपेता बुद्धिर्मनसि वर्तते}
{तदा प्रज्ञायते ब्रह्म ध्यानयोगसमाधिना}


\twolineshloka
{सेयं गुणवती बुद्धिर्गुणेष्वेवाभिवर्तते}
{अपरादभिनिः स्रौति गिरेः शृङ्गादिवोदकम्}


\twolineshloka
{यदा निर्गुणमाप्नोति ध्यानं मनसि पूर्वजम्}
{तदा प्रज्ञायते ब्रह्म निकषं निकषे यथा}


\twolineshloka
{मनस्त्वसंहृतं बुद्ध्या हीन्द्रियार्थनिदर्शकम्}
{न समर्थं गुणापेक्षि निर्गुणस्य निदर्शने}


\twolineshloka
{सर्वाण्येतानि संवार्य द्वाराणि मनसि स्थितः}
{मनस्येकाग्रतां कृत्वा तत्परं प्रतिपद्यते}


\twolineshloka
{यथा महान्ति भूतानि निवर्तन्ते गुणक्षये}
{तथेन्द्रियाण्युपादाय बुद्धिर्मनसि वर्तते}


\twolineshloka
{यदा मनसि सा बुद्धिर्वर्ततेऽन्तरचारिणी}
{व्यवसायगुणोपेता तदा संपद्यते मनः}


\twolineshloka
{गुणवद्भिर्गुणोपेतं यदा ध्यानगतं मनः}
{तदा सर्वान्गुणान्हित्वा निर्गुणं प्रतिपद्यते}


\twolineshloka
{अव्यक्तस्येह विज्ञाने नास्ति तुल्यं निदर्शनम्}
{यत्र नास्ति पदन्यासः कस्तं विषयमाप्नुयात्}


\twolineshloka
{तपसा चानुमानेन गुणैर्जात्या श्रुतेन च}
{निनीषेत्परमं ब्रह्म विशुद्धेनान्तरात्मना}


\twolineshloka
{गुणहीनो हि तं मार्गं बहिः समनुवर्तते}
{गुणाभावात्प्रकृत्या वा निस्तर्क्यं ज्ञेयसंमितम्}


\twolineshloka
{नैर्गुण्याद्ब्रह्म चाप्नोति सगुणत्वान्निवर्तते}
{गुणप्रसारिणी बुद्धिर्हुताशन इवेन्धने}


\twolineshloka
{यदा पञ्च वियुक्तानि इन्द्रियाणि स्वकर्मभिः}
{तदा तत्परमं ब्रह्म संमुक्तं प्रकृतेः परम्}


\twolineshloka
{एवं प्रकृतितः सर्वे संभवन्ति शरीरिणः}
{निवर्तन्ते निवृत्तौ च स्वर्गे नैवोपयान्ति च}


\twolineshloka
{पुरुषप्रकृतिर्बुद्धिर्विशेषाश्चेन्द्रियाणि च}
{अहंकारोऽभिमानश्च संभूतो भूतसंज्ञकः}


\twolineshloka
{एतस्याद्या प्रवृत्तिस्तु प्रधानात्संप्रवर्तते}
{द्वितीया मिथुनव्यक्तिमविशेषान्नियच्छति}


\twolineshloka
{धर्मादुत्कृष्यते श्रेयस्तथा धर्मोऽष्यधर्मतः}
{रागवान्प्रकृतिं ह्येति विरक्तो ज्ञानवान्भवम्}


\chapter{अध्यायः २०४}
\twolineshloka
{मनुरुवाच}
{}


\twolineshloka
{`तदेव सततं मन्ये न शक्यमनुवर्णितुम्}
{यथा निदर्शनं वस्तु न शक्यमनुबोधितुम्}


\twolineshloka
{यथाहि सारं जानाति न कथंचन संस्थितम्}
{परकायच्छविस्तद्वद्देहेऽयं चेतनस्तथा}


\twolineshloka
{विना कायं न सा च्छाया तां विना कायमस्त्युत}
{तद्वदेव विना नास्ति प्रकृतेरिह वर्तनम्}


\twolineshloka
{इदं वै नास्ति नेदमस्ति परं विना}
{जीवात्मना त्वसौ छिन्नस्त्वेष चैव परात्मना}


\twolineshloka
{तत्तवेति विदुः केचिदतथ्यमिति चापरे}
{उभयं मे मतं विद्वन्मुक्तिहेतौ समाहितम्}


\twolineshloka
{विमुक्तैश्च मृगः सोऽपि दृश्यते संयतेन्द्रियः}
{सर्वेषां न हि दृश्यो हि तटिद्वत्स्फुरति ह्यसौ}


\twolineshloka
{ब्राह्मणस्य समादृश्यो वर्तते सोऽपि किं पुनः}
{विद्यते परमः शुद्धः साक्षिभूतो विभावसुः}


\twolineshloka
{श्रुतिरेषा ततो नित्या तस्मादेकः परो मतः}
{न प्रयोजनमुद्दिश्य चेष्टा तस्य महात्मनः}


\threelineshloka
{तादृशोस्त्विति मन्तव्यस्तथा सत्यं महात्मना}
{नानासंस्थेन भेदेन सदा गतिविभेदवत्}
{तस्य भेदः समाख्यातो भेदो ह्यस्ति तथाविधः}


\twolineshloka
{एवं विद्वन्विजानीहि परमात्मानमव्ययम्}
{तत्तद्गुणविशेषेण संज्ञानामनुसंयुतम्}


\twolineshloka
{सर्वेश्वरः सर्वमयः स च सर्वप्रवर्तकः}
{सर्वात्मकः सर्वशक्तिः सर्वकारणकारणम्}


\twolineshloka
{सर्वसाधारणः सर्वैरुपास्यश्च महात्मभिः}
{वासुदेवेति विख्यातस्तं विदित्वाऽश्नुतेऽमृतम् ॥'}


\twolineshloka
{यदा ते प़ञ्चभिः पञ्च युक्तानि मनसा सह}
{अथ तद्द्रक्ष्यते ब्रह्म मणौ सूत्रमिवार्पितम्}


\twolineshloka
{तदेव च यथा सूत्रं सुवर्णे वर्तते पुनः}
{मुक्तास्वथ प्रवालेषु मृन्मये राजते तथा}


\twolineshloka
{तद्वद्गोऽश्वमनुष्येषु तद्वद्धस्तिमृगादिषु}
{तद्वत्कीटपतङ्गेषु प्रसक्तात्मा स्वकर्मभिः}


\twolineshloka
{येनयेन शरीरेण यद्यत्कर्म करोत्ययम्}
{तेनतेन शरीरेण तत्तत्फलमुपाश्नुते}


\twolineshloka
{यथा ह्येकरसा भूमिरोषध्यर्थानुसारिणी}
{तथा कर्मानुगा बुद्धिरन्तरात्माऽनुदर्शिनी}


\twolineshloka
{ज्ञानपूर्वोद्भवा लिप्सा लिप्सापूर्वाभिसन्धिता}
{अभिसन्धिपूर्वकं कर्म कर्ममूलं ततः फलम्}


\twolineshloka
{फलं कर्मत्मकं विद्यात्कर्म ज्ञेयात्मकं तथा}
{ज्ञेयं ज्ञानात्मकं विद्याज्ज्ञानं सदसदात्मकम्}


\threelineshloka
{`तदेवमिष्यते ब्रह्म संख्यानाद्विनिभिद्यते}
{'ज्ञानानां च फलानां च ज्ञेयानां कर्मणां तथा}
{क्षयान्ते यत्फलं विद्याज्ज्ञानं ज्ञेयप्रतिष्ठितम्}


\twolineshloka
{महद्धि परमं भूतं युक्ताः पश्यन्ति योगिनः}
{अबुधास्तं न पश्यन्ति ह्यात्मस्थं गुणबुद्धयः}


\twolineshloka
{पृथिवीरुपतो रूपमपामिह महत्तरम्}
{अद्भ्यो महत्तरं तेजस्तेजसः पवनो महान्}


\twolineshloka
{पवनाच्च महद्व्योम तस्मात्परतरं मनः}
{मनसो महती बुद्धिर्बुद्धेः कालो महान्स्मृतः}


\twolineshloka
{कालात्स भगवान्विष्णुर्यस्य सर्वमिदं जगत्}
{नादिर्न मध्यं नैवान्तस्तस्य देवस्य विद्यते}


\twolineshloka
{अनादित्वादमध्यत्वादनन्तत्वाच्च सोऽव्ययः}
{अत्येति सर्वदुःखानि दुःखं ह्यन्तवदुच्यते}


\twolineshloka
{तद्ब्रह्म परमं प्रोक्तं तद्धाम परमं पदम्}
{तद्गत्वा कालविषयाद्विमुक्ता मोक्षमाश्रिताः}


\twolineshloka
{गुणेष्वेते प्रकाशन्ते निर्गुणत्वात्ततः परम्}
{निवृत्तिलक्षणो धर्मस्तथाऽऽनन्त्याय कल्पते}


\twolineshloka
{ऋचो यजूंषि सामानि शरीराणि व्यपाश्रिताः}
{जिह्वाग्रेषु प्रवर्तन्ते यत्नसाध्याविनाशिनः}


\twolineshloka
{न चैवमिष्यते ब्रह्म शरीराश्रयसंभवम्}
{न यत्नसाध्यं तद्ब्रह्म नादिमध्यं न चान्तवन्}


\twolineshloka
{ऋचामादिस्तथा साम्नां यजुषामादिरुच्यते}
{अन्तश्चादिमतां दृष्टो न त्वादिर्ब्रह्मणः स्मृतः}


\twolineshloka
{अनादित्वादनन्तत्वात्तदनन्तमथाव्ययम्}
{अव्ययत्वाच्च निर्दुःखं द्वन्द्वाभावस्ततः परम्}


\twolineshloka
{अदृष्टतोऽनुपायाच्च प्रतिसन्धेश्च कर्मणः}
{न तेन मर्त्याः पश्यन्ति येन गच्छान्त तत्पदम्}


\twolineshloka
{विषयेषु च संसर्गाच्छाश्वतस्य च संशयात्}
{मनसा चान्यदाकाङ्क्षन्परं न प्रतिपद्यते}


\twolineshloka
{गुणान्यदिह पश्यन्ति तदिच्छन्त्यपरे जनाः}
{परं नैवाभिकाङ्क्षन्ति निर्गुणत्वाद्गुणार्थिनः}


\twolineshloka
{गुणैर्यस्त्ववरैर्युक्तः कथं विद्याद्गुणानिमान्}
{अनुमानाद्धि गन्तव्यं गुणैरवयवैः परम्}


\twolineshloka
{सूक्ष्मेण मनसा विद्मो वाचा वक्तुं न शक्नुमः}
{मनो हि मनसा ग्राह्यं दर्शनेन च दर्शनम्}


\twolineshloka
{ज्ञानेन निर्मलीकृत्य बुद्धिं बुद्ध्या मनस्तथा}
{मनसा चेन्द्रियग्राममक्षरं प्रतिपद्यते}


\twolineshloka
{बुद्धिप्रहीणो मनसा समृद्धस्तथाऽनिराशीर्गुणतामुपैति}
{परं त्यजन्तीह विलोभ्यमानाहुताशनं वायुरिवेन्धनस्थम्}


\twolineshloka
{गुणादाने विप्रयोगे च तेषांमनः सदा विद्धि परावराभ्याम्}
{अनेनैव विधिना संप्रवृत्तोगुणादाने ब्रह्म शरीरमेति}


\twolineshloka
{अव्यक्तात्मा पुरुषोऽव्यक्तकर्मासोऽव्यक्तत्वं गच्छति ह्यन्तकाले}
{तैरेवायं चेन्द्रियैर्वर्धमानैर्ग्लायद्भिर्वा वर्ततेऽकामरूपः}


\twolineshloka
{सर्वैरयं चेन्द्रियैः संप्रयुक्तोदेहं प्राप्तः यञ्चभूताश्रयः स्यात्}
{न सामर्थ्याद्गच्छति कर्मणेहहीनस्तेन परमेणाव्ययेन}


\twolineshloka
{पृथ्व्या नरः पश्यति नान्तमस्याह्यन्तश्चास्या भविता चेति विद्धि}
{परं न यातीह विलोभ्यमानोयथा प्लवं वायुरिवार्णवस्थम्}


\twolineshloka
{दिवाकरो गुणमुपलभ्य निर्गुणोयथा भवेदपगतरश्मिमण्डलः}
{तथां ह्यसौ मुनिरिह निर्विशेषवान्स निर्गुणं प्रविशति ब्रह्म चाव्ययम्}


\twolineshloka
{अनागतं सुकृतवतां परां गतिंस्वयंभुवं प्रभवनिधानमव्ययम्}
{सनातनं यदमृतमव्ययं ध्रुवंनिचाय्य तत्परममृतत्वमश्नुते}


\chapter{अध्यायः २०५}
\twolineshloka
{* युधिष्ठिर उवाच}
{}


\threelineshloka
{पितामह महाप्राज्ञ दुःखशोकसमाकुले}
{संसारचक्रे लोकानां निर्वेदो नास्ति किंन्विदम् ॥भीष्म उवाच}
{}


\twolineshloka
{अत्राप्युदाहरन्तीममितिहासं पुरातनम्}
{निबन्धनस्य संवादं भोगवत्या नृपोत्तम}


\threelineshloka
{मुनिं निबन्धनं शुष्कं धमनीयाकृतिं तथा}
{निरारम्भं निरालम्बमसज्जन्तं च कर्मणि}
{पुत्रं दृष्ट्वाऽप्युवाचार्तं माता भोगवती तदा}


\threelineshloka
{उत्तिष्ठ मूढ किं शेषे निरपेक्षः सुहृज्जनैः}
{निरालम्बो धनोपाये पैतृकं तव किं धनम् ॥निबन्धन उवाच}
{}


\twolineshloka
{पैतृकं मे महन्मातः सर्वदुःखालयं त्विह}
{अस्त्येतत्तद्विधाताय यतिष्ये तत्र मा शुचः}


\twolineshloka
{इदं शरीरमत्युग्रं पित्रा दत्तमसंशयम्}
{तमेव पितरं गत्वा धनं तिष्ठति शाश्वतम्}


\twolineshloka
{कश्चिन्महति संसारे वर्तमान---}
{वनदुगमभिप्राप्तो महत्क्रव्या-----}


\twolineshloka
{सिंहव्याघ्रगजाकारैरतिघोरैर्महा-----}
{समन्तात्सुपरिक्षिप्तं स दृष्ट्वा व्यथते पुमान्}


\twolineshloka
{स तद्वनं ह्यनुचरन्विप्रधावन्नितस्ततः}
{वीक्षमाणो दिशः सर्वाः शरणार्थं प्रधावति}


\twolineshloka
{अथापश्यद्वनं रूढं समन्ताद्वागुरावृतम्}
{वनमध्ये च तत्रासीदुदपानः समावृतः}


\twolineshloka
{वल्लिभिस्तृणसंछिन्नैर्गूढाभिरभिसंवृतः}
{स पपात द्विजस्तत्र विजने सलिलाशये}


\twolineshloka
{विलग्नश्चाभवत्तस्मिँल्लतासन्तानसङ्कुले}
{बाहुभ्यां संपरिष्वक्तस्तया परमसत्वया}


\twolineshloka
{स तथा लम्बते तत्र ऊर्ध्वपादो ह्यधश्शिराः}
{अधस्तत्रैव जातश्च जम्बूवृक्षः सुदुस्तरः}


\twolineshloka
{कूपस्य तस्य वेलाया अपश्यत्सुमहाफलम्}
{वृक्षं बहुविधं व्योमं वल्लीपुष्पसमाकुलम्}


\twolineshloka
{नानारूपा मधुकरास्तस्मिन्वुक्षऽभवन्किल}
{तेषां मधूनां बहुधा धारा प्रववृते तदा}


\twolineshloka
{विलम्बमानः स पुमान्धारां पिबति सर्वदा}
{न तस्य तृष्णा विरता पीयमानस्य संकटे}


\twolineshloka
{परीप्सति च तां नित्यमतृप्तः स पुनः पुनः}
{एवं स वसते तत्र दुःखिदुःखी पुनः पुनः}


\twolineshloka
{मया तु तद्धनं देयं तव दास्यामि चेच्छसि}
{तस्य च प्रार्थितः सोथ दत्वा मुक्तिमवाप सः}


% Check verse!
सा च त्यक्त्वाऽर्थसंकल्पं जगाम परमां गतिम्
\threelineshloka
{एवं संसारचक्रस्य स्वरूपज्ञा नृपोत्तम}
{परं वैराग्यमागम्य गच्छन्ति परमं पदम् ॥युधिष्ठिर उवाच}
{}


\twolineshloka
{एवं संसारचक्रस्य स्वरूपं विदितं न मे}
{पैतृकं तु धनं प्रोक्तं किं तद्विद्वन्महात्मना}


\twolineshloka
{कान्तारमिति किं प्रोक्तं को हस्ती स तु कूपकः}
{किंसंज्ञिको महावृक्षो मधु वाऽपि पितामह}


\threelineshloka
{एवं मे संशयं विद्धि धनशब्दं किमुच्यते}
{कथं लब्धं धनं तेन तथा च किमिदं त्विह ॥भीष्म उवाच}
{}


\twolineshloka
{उपाख्यानमिदं सर्वं मोक्षविद्भिरुदाहृतम्}
{सुमतिं विन्दते येन बन्धनाशश्च भारत}


\twolineshloka
{एतदुक्तं हि कान्तारं महान्संसार एव सः}
{ये ते प्रतिष्ठिता व्याला व्याधयस्ते प्रकीर्तिताः}


\twolineshloka
{या सा नारी महाघोरा वर्णरूपविनाशिनी}
{तामाहुश्च जरां प्राज्ञाः परिष्वक्तं यया जगत्}


\twolineshloka
{यस्तत्र कूपे वसते महाहिः काल एव सः}
{यो वृक्षः स च मृत्युर्हि स्वकृतं तस्य तत्फलम्}


% Check verse!
ये तु कृष्णाः सिता राजन्मूषिका रात्र्यहानि वै
\twolineshloka
{द्विषट्कपदसंयुक्तो यो हस्ती षण्मुखाकृतिः}
{स च संवत्सरः प्रोक्तः पाशमासर्तवो मुखाः}


\twolineshloka
{एतत्संसारचक्रस्य स्वरूपं व्याहृतं मया}
{एवं लब्धधनं राजंस्तत्स्वरूपं विनाशय}


\twolineshloka
{एतज्ज्ञात्वा तु सा राजन्परं वैराग्यमास्थिता}
{यथोक्तविधिना भूयः परं पदमवाप सः}


\twolineshloka
{धत्ते धारयते चैव एतस्मात्कारणाद्धनम्}
{तद्गच्छ चामृतं शुद्धं हिरण्यममृतं तपः}


\twolineshloka
{तत्स्वरूपो महादेवः कृष्णो देवकिनन्दनः}
{तस्य प्रसादाद्दुःखस्य नाशं प्राप्स्यसि मानद}


% Check verse!
एकः कर्ता स कृष्णश्च ज्ञानिनां परमा गतिः
\twolineshloka
{इदमाश्रित्य देवेन्द्रो देवा रुद्रास्तथाऽश्विनौ}
{स्वेस्वे पदे विविशिरे भुक्तिमुक्तिविदो जनाः}


\twolineshloka
{भूतानामन्तरात्माऽसौ स नित्यपदसंवृतः}
{श्रूयतामस्य सद्भावः सम्यग्ज्ञानं यथा तव}


\twolineshloka
{भवेदेतन्निबोध त्वं नारदाय पुरा हरिः}
{दर्शयित्वाऽऽत्मनो रूपं यदवोचत्स्वयं विभुः}


\twolineshloka
{पुरा देवऋषिः श्रीमान्नारदः परमार्थवान्}
{चचार पृथिवीं कृत्स्नां तीर्थान्यनुचरन्प्रभुः}


\twolineshloka
{हिमवत्पादमाश्रित्य विचार्य च पुनःपुनः}
{स ददर्श ह्रदं तत्र पद्मोत्पलसमाकुलम्}


\twolineshloka
{ददर्श कन्यां तत्तीरे सर्वाभरणभूषिताम्}
{शोभमानां श्रिया राजन्क्रीडन्तीमुत्पलैस्तथा}


\twolineshloka
{सा महात्मानमालोक्य नारदेत्याह भामिनी}
{तस्याः समीपमासाद्य तस्थौ विस्मितमानसः}


\twolineshloka
{वीक्षमाणं तमाज्ञाय सा कन्या चारुवासिनी}
{विजजृम्भे महाभागा स्मयमाना पुनः पुनः}


\twolineshloka
{तस्मात्समभवद्वक्रात्पुरुषाकृतिसंयुतः}
{रत्नविन्दुचिताङ्गस्तु सर्वाभरणभूषितः}


\twolineshloka
{आदित्यसदृशाकारः शिरसा धारयन्मणिम्}
{पुनरेव तदाकारसदृशः समजायत}


\twolineshloka
{तृतीयस्तु महाराज विविधाभरणैर्युतः}
{प्रदक्षिणं तु तां कृत्वा विविधध्वनयस्तु ताम्}


% Check verse!
ततः सर्वेण विप्रर्षिः कन्यां पप्रच्छ तां शुभाम्
\threelineshloka
{का त्वं परमकल्याणि पद्मेन्दुसदृशानने}
{न जाने त्वां महादेवि ब्रूहि सत्यमनिन्दिते ॥कन्योवाच}
{}


\threelineshloka
{सावित्री नाम विप्रर्षे शृणु भद्रं तवास्तु वै}
{किं करिष्यामि तद्ब्रूहि तव यच्चेतसि स्थितम् ॥नारद उवाच}
{}


\twolineshloka
{अभिवादये त्वां सावित्रि कृतार्थोऽहमनिन्दिते}
{एतं मे संशयं देवि वक्तुमर्हसि शोभने}


\threelineshloka
{यस्तु वै प्रथमोत्पन्नः कोऽसौ स पुरुषाकृतिः}
{बिन्दवस्तु महादेवि मूर्ध्नि ज्योतिर्मयाकृतिः ॥कन्योवाच}
{}


\twolineshloka
{अग्रजः प्रथमोत्पन्नो यजुर्वेदस्तथाऽपरः}
{तृतीयः सामवेदस्तु संशयो व्येतु ते मुने}


\twolineshloka
{वेदाश्च बिन्दुसंयुक्ता यज्ञस्य फलसंश्रिताः}
{यत्तद्दृष्टं महज्ज्योतिर्ज्योतिरित्युच्यते बुधैः}


\threelineshloka
{ऋषे ज्ञेयं मया चाऽपीत्युक्त्वा चान्तरधीयत}
{ततः स विस्मयाविष्टो नारदः पुरुषर्षभ}
{ध्यानयुक्तः स तु चिरं न बुबोध महामतिः}


\twolineshloka
{ततः स्नात्वा महातेजा वाग्यतो नियतेन्द्रियः}
{तुष्टाव पुरुषव्याघ्रो जिज्ञासुश्च तदद्भुतम्}


\twolineshloka
{ततो वर्षशते पूर्णे भगवाँलोकभावनः}
{प्रादुश्चकार विश्वात्मा ऋषेः परमसौहृदात्}


\twolineshloka
{तमागतं जगन्नाथं सर्वकारणकारणम्}
{अखिलामरमौल्यङ्गरुक्मारुणपदद्वयम्}


\twolineshloka
{वैनतेयपदस्पर्शकिणशोभितजानुकम्}
{पीताम्बरलसत्काञ्चीदामबद्धकटीतटम्}


\twolineshloka
{श्रीवत्सवक्षसं चारुमणिकौस्तुभकन्धरम्}
{मन्दस्मितमुखाम्भोजं चलदायतलोचनम्}


\twolineshloka
{नम्रचापानुकरणनम्रभ्रूयुगशोभितम्}
{नानारत्नमणीवज्रस्फुरन्मकरकुण्डलम्}


\twolineshloka
{इन्द्रनीलनिभासं तं केयूरमकुटोज्ज्वलम्}
{देवैरिन्द्रपुरोगैश्च ऋषिसङघैरभिष्टुतम्}


% Check verse!
नारदो जयशब्देन ववन्दे शिरसा हरिम्
\twolineshloka
{ततः स भगवाञ्श्रीमान्मेघगम्भीरया गिरा}
{प्राहेशः सर्वभूतानां नारदं पतितं क्षितौ}


\twolineshloka
{भद्रमस्तु ऋषे तुभ्यं वरं वरय सुव्रत}
{यत्ते मनसि सुव्यक्तमस्ति च प्रददामि तन्}


\twolineshloka
{स चेमं जयशब्देन प्रसीदेत्यातुरो मुनिः}
{प्रोवाच हृदि संरूढं शङ्खचक्रगदाधरम्}


\twolineshloka
{विवक्षितं जगन्नाथ मया ज्ञातं त्वयाऽच्युत}
{तत्प्रसीद हृषीकेश श्रोतुमिच्छामि तद्धरे}


\twolineshloka
{ततः स्मयन्महाविष्णुरभ्यभाषत नारदम्}
{यद्दृष्टं मम रूपं तु वेदानां शिरसि त्वया}


\twolineshloka
{निर्द्वन्द्वा निरहंकाराः शुचयः शुद्धलोचनाः}
{तं मां पश्यन्ति सततं तान्पृच्छ यदिहेच्छसि}


\threelineshloka
{ये योगिनो महाप्राज्ञा मदंशा ये व्यवस्थिताः}
{तेषां प्रसादं देवर्षे मत्प्रसादमवैहि तत् ॥भीष्म उवाच}
{}


\twolineshloka
{इत्युक्त्वा स जगामाथ भगवान्भूतभावनः}
{तस्माद्व्रज हृषीकेशं कृष्णं देवकिनन्दनम्}


\twolineshloka
{एतमाराध्य गोविन्दं गता मुक्तिं महर्षयः}
{एष कर्ता विकर्ता च सर्वकारणकारणम्}


\twolineshloka
{मयाऽप्येतच्छ्रुतं राजन्नारदात्तु निबोध तत्}
{स्वयमेव समाचष्ट नारदो भगवान्मुनिः}


\twolineshloka
{समस्तसंसारविघातकारणंभजन्ति ये विष्णुमनन्यमानसाः}
{ते यान्ति सायुज्यमतीव दुर्लभमितीव नित्यं हृदि वर्णयन्ति ॥'}


\chapter{अध्यायः २०६}
\twolineshloka
{युधिष्ठिर उवाच}
{}


\twolineshloka
{पितामह महाप्राज्ञ पुण़्डरीकाक्षमच्युतम्}
{कर्तारमकृतं विष्णुं भूतानां प्रभवाप्ययम्}


\threelineshloka
{नारायणं हृषीकेशं गोविन्दमपराजितम्}
{तत्त्वेन भरतश्रेष्ठ श्रोतुमिच्छामि केशवम् ॥भीष्म उवाच}
{}


\twolineshloka
{श्रुतोऽयमर्थो रामस्य जामदग्न्यस्य जल्पतः}
{नारदस्य च देवर्षेः कृष्णद्वैपायनस्य च}


\twolineshloka
{असितो देवलस्तात वाल्मीकिश्च महातपाः}
{मार्कण्डेयश्च गोविन्दे कथयन्त्यद्भुतं महत्}


\twolineshloka
{`केशवस्य मया राजन्न शक्या वर्णितुं गुणाः}
{ईदृशोऽसौ हृषीकेशो वासुदेवः परात्परः ॥'}


\twolineshloka
{केशवो भरतश्रेष्ठ भगवानीश्वरः प्रभुः}
{पुरुषः सर्वमित्येव श्रूयते बहुधा विभुः}


\twolineshloka
{किंतु यानि विदुर्लोके ब्राह्मणाः शार्ङ्गधन्वनि}
{महात्मनि महाबाहो शृणु तानि युधिष्ठिर}


\twolineshloka
{यानि चाहुर्मनुष्येन्द्र ये पुराणविदो जनाः}
{श्रुत्वा सर्वाणि गोविन्दो कीर्तयिष्यामि तान्यहम्}


\twolineshloka
{महाभूतानि भूतात्मा महात्मा पुरुषोत्तमः}
{वायुर्ज्योतिस्तथा चापः खं च गां चान्वकल्पयत्}


\twolineshloka
{स सृष्ट्वा पृथिवीं चैव सर्वभूतेश्वरः प्रभुः}
{अप्स्वेव शयनं चक्रे महात्मा पुरुषोत्तमः}


\twolineshloka
{सर्वतेजोमयस्तस्मिञ्शयानः शयने शुभे}
{सोऽग्रजं सर्वभूतानां संकर्षणमचिन्तयत्}


\twolineshloka
{आश्रयं सर्वभूतानां मनसेतीह शुश्रुम्}
{स धारयति भूतानि उभे भूतभविष्यती}


\twolineshloka
{`प्रद्युम्नमसृजत्तस्मात्सर्वतेजः प्रकाशकम्}
{अनिरुद्धस्ततो जज्ञे सर्वशक्तिर्महाद्युतिः}


\threelineshloka
{अप्सु व्योमगतः श्रीमान्योगनिद्रामुपेयिवान्}
{तस्मात्संजज्ञिरे देवा ब्रह्मविष्णुमहेश्वराः}
{लयस्थित्यन्तकर्माणस्रयस्ते सुमहौजसः ॥'}


\twolineshloka
{ततस्तस्मिन्महाबाहौ प्रादुर्भूते महात्मनि}
{भास्करप्रतिमं दिव्यं नाभ्यां पद्ममजायत}


\twolineshloka
{स तत्र भगवान्देवः पुष्करे भ्राजयन्दिशः}
{ब्रह्मा समभवत्तात सर्वभूतपितामहः}


\twolineshloka
{तस्मिन्नपि महाबाहौ प्रादुर्भूते महात्मनि}
{तमसः पूर्वजो जज्ञे मधुर्नाम महासुरः}


\twolineshloka
{तमुग्रमुग्रकर्माणमुग्रां बुद्धिं समास्थितम्}
{ब्रह्मणोपचितिं कुर्वञ्जघान पुरुषोत्तमः}


\twolineshloka
{तस्य तात वधात्सर्वे देवदानवमानवाः}
{मधुसूदनमित्याहुर्ऋषभं सर्वसात्वताम्}


\twolineshloka
{ब्रह्माऽनुससृजे पुत्रान्मानसान्दक्षसप्तमान्}
{मरीचिमत्र्यङ्गिरसौ पुलस्त्यं पुलहं क्रतुम्}


\twolineshloka
{मरीचिः कश्यपं तात पुत्रमग्रजमग्रजः}
{मानसं जनयामास तैजसं ब्रह्मवित्तमम्}


\twolineshloka
{अङ्गुष्ठात्ससृजे ब्रह्मा मरीचेरपि पूर्वजम्}
{सोऽभवद्भरतश्रेष्ठ दक्षो नाम प्रजापतिः}


% Check verse!
तस्य पूर्वमजायन्त दश तिस्रश्च भारत ॥प्रजापतेर्दुहितरस्तासां ज्येष्ठाऽभवद्दितिः
\twolineshloka
{सर्वधर्मविशेषज्ञः पुण्यकीर्तिर्महायशाः}
{मारीचः कश्यपस्तात सर्वासामभवत्पतिः}


\twolineshloka
{उत्पाद्य तु महाभागस्तासामवरजा दश}
{ददौ धर्माय धर्मज्ञो दक्ष एव प्रजापतिः}


\twolineshloka
{धर्मस्य वसवः पुत्रा रुद्राश्चामिततेजसः}
{विश्वेदेवाश्च साध्याश्च मरुत्वन्तश्च भारत}


\twolineshloka
{अपराश्च यवीयस्यस्ताभ्योऽन्याः सप्तविंशतिः}
{सोमस्तासां महाभागः सर्वासामभवत्पतिः}


\twolineshloka
{इतरास्तु व्यजायन्त गन्धवोस्तुरगान्द्विजान्}
{गाश्च किंपुरुषान्मत्स्यानुद्भिज्जांश्च वनस्पतीन्}


\twolineshloka
{आदित्यानदितिर्जज्ञे देवश्रेष्ठान्महाबलान्}
{तेषां विष्णुर्वामनोऽभूद्गोविन्दश्चाभवत्प्रभुः}


\twolineshloka
{तस्य विक्रमणाच्चापि देवानां श्रीर्व्यवर्धत}
{दानवाश्च पराभूता दैतेयाश्चासुरी प्रजा}


\twolineshloka
{विप्रचित्तिप्रधानांश्च दानवानसृजद्दनुः}
{दितिस्तु सर्वानसुरान्महासत्वानजीजनत्}


\twolineshloka
{`ततः ससर्ज भगवान्मृत्युं लोकभयंकरम्}
{हर्तारं सर्व भूतानां ससर्ज च जनार्दनः}


\twolineshloka
{अहोरात्रं च कालं च यथर्तु मधुसूदनः}
{पूर्वाह्णं चापराह्णं च सर्वमेवान्वकल्पयत्}


\twolineshloka
{लब्ध्वापः सोऽसृजन्मेघांस्तथा स्थावरजङ्गमान्}
{पृथिवीं सोऽसृजद्विश्वां सहीतां भूरितेजसा}


\twolineshloka
{ततः कृष्णो महाभागः पुनरेव युधिष्ठिर}
{ब्राह्मणानां शतं श्रेष्ठं मुखादेवासृजत्प्रभुः}


\twolineshloka
{बाहुभ्यां क्षत्रियशतं वैश्यानामूरुतः शतम्}
{पद्भ्यां शूत्रशतं चैव कशेवो भरतर्षभ}


\twolineshloka
{स एवं चतुरो वर्णान्समुत्पाद्य महातपाः}
{अध्यक्षं सर्व भूतानां धातारमकरोत्स्वयम्}


\twolineshloka
{वेदविद्याविधातारं ब्रह्माणमतितद्युतिम्}
{भूतमातृगणाध्यक्षं विरूपाक्षं च सोऽसृजत्}


\twolineshloka
{शासितारं च पापानां पितृणां समवर्तिनम्}
{असृजत्सर्वभूतात्मा निधिपं च धनेश्वरम्}


\twolineshloka
{यादसामसृजन्नाथं वरुणं च जलेश्वरम्}
{वासवं सर्वदेवानामध्यक्षमकरोत्प्रभुः}


\twolineshloka
{यावद्यावदभूच्छ्रद्धा देहं धारयितुं नृणाम्}
{तावत्तावदजीवंस्ते नासीद्यमकृतं भयम्}


\twolineshloka
{न चैषां मैथुनो धर्मो बभूव भरतर्षभ}
{संकल्पादेव चैतेषां गर्भः समुपपद्यते}


\twolineshloka
{ततस्रेतायुगे काले संस्पर्शाज्जायते प्रजा}
{न ह्यभून्मैथुनो धर्मस्तेषामपि जनाधिप}


\twolineshloka
{द्वापरे मैथुनो धर्मः प्रजानामभवन्नृप}
{तथा कलियुगे राजन्द्वन्द्वमापेदिरे जनाः}


\twolineshloka
{एष भूतपतिस्तात स्वध्यक्षश्च तथोच्यते}
{निरपेक्षांश्च कौन्तेय कीर्तयिष्यामि तच्छृणु}


\twolineshloka
{दक्षिणापथजन्मानः सर्वे करभृतस्तव}
{आन्ध्राः पुलिन्दाः शवराश्चूचुपा मद्रकैः सह}


\twolineshloka
{उत्तरापथजन्मानः कीर्तयिष्यामि तानपि}
{ये तु काम्भोजगान्धाराः किराता बर्वरैः सह}


\twolineshloka
{एते पापकृतस्तात चरन्ति पृथिवीमिमाम्}
{बकश्वपाकगृध्राणां सधर्माणो नराधिप}


\twolineshloka
{नैते कृतयुगे तात चरन्ति पृथिवीमिमाम्}
{त्रेताप्रभुति वर्धन्ते ते जना भरतर्षभ}


\twolineshloka
{ततस्तस्मिन्महाघोरे संन्ध्याकाले युगान्तिके}
{राजानः समसज्जन्त समासाद्येतरेतम्}


\twolineshloka
{`ऐन्द्रं रूपं समास्थाय ह्यसुरेभ्यो चरन्महीम्}
{स एव भगवान्देवो वेदित्वं च गता मही}


\twolineshloka
{एवंभूते सृष्टिर्नारसिंहादयः क्रमात्}
{प्रादुर्भावाः स्मृता विष्णोर्जगतीरक्षणाय वै}


\twolineshloka
{एष कृष्णो महायोगी तत्तत्कार्यानुरूपणम्}
{हिरण्यकशिपुं दैत्यं हिरण्याक्षं तथैव च}


\twolineshloka
{रावणं च महादैत्यं हत्वासौ पुरुषोत्तमः}
{भूमेर्दुःखोपनाशार्थं ब्रह्मशक्रादिभिः स्तुतः}


\twolineshloka
{आत्मनोऽङ्गान्महातेजा उद्वबर्ह जनार्दनः}
{सितकृष्णौ महाराज केशौ हरिरुदारधीः}


\twolineshloka
{वसुदेवस्य देवक्यामेष जात इहोत्तमः}
{देहवानिह विश्वात्मा संबन्धी ते जनार्दनः}


\twolineshloka
{आविर्वभूव योगीन्द्रो मनोतीतो जगत्पतिः}
{अचिन्त्यः पुरुषव्याघ्र नैव केवलमानुषः}


\twolineshloka
{अव्यक्तादिविशेषान्तं परिमाणार्थसंयुतम्}
{क्रीडा हरेरिदं सर्वं क्षरमित्येव धार्यताम्}


\twolineshloka
{अक्षरं तत्परं नित्यं वैरूप्यं जगतो हरेः}
{तद्विद्धि रूपमतुलममृतत्वं भवज्जितम्}


\twolineshloka
{तदेव कृष्णो दाशार्हः श्रीमाञ्श्रीवत्सलक्षणः}
{न भूतसृष्टिसंस्थानं देहोऽस्य परमात्मनः}


\twolineshloka
{देहवानिह यो विष्णुरसौ मायामयो हरिः}
{आत्मनो लोकरक्षार्थं ध्याहि नित्यं सनातनम्}


\twolineshloka
{अङ्गानि चतुरे वेदा मीमांसा न्यायविस्तरः}
{इतिहासपुराणानि धर्माः स्वायंभुवादयः}


\twolineshloka
{य एनं प्रतिवर्तन्ते वेदान्तानि च सर्वशः}
{भक्तिहीना न तैर्यान्ति नित्यमेनं कथंचन}


\twolineshloka
{सर्वभूतेषु भूतात्मा तत्तद्बुद्धिं समास्थितः}
{तस्माद्बुद्धस्त्वमेवैनं ध्याहि नित्यमतन्द्रितः ॥'}


\twolineshloka
{एवमेष कुरुश्रेष्ठ प्रादुर्भावो महात्मनः}
{एवं देवर्षिराचष्ट नारदः सर्वलोकदृक्}


\twolineshloka
{नारदोऽप्यथ कृष्णस्य परं मेने नराधिप}
{शाश्वतत्वं महाबाहो यथावद्भरतर्षभ}


\twolineshloka
{एवमेव महाबाहुः केशवः सत्यविक्रमः}
{अचिन्त्यः पुण्डरीकाक्षो नैष केवलमानुषः}


\twolineshloka
{`एवंविधोऽसौ पुरुषः को वैनं वेत्ति सर्वदा}
{एतत्ते कथितं राजन्भूयः श्रोतुं किमिच्छसि'}


\chapter{अध्यायः २०७}
\twolineshloka
{युधिष्ठिर उवाच}
{}


\threelineshloka
{के पूर्वमासन्पतयः प्रजानां भरतर्षभ}
{के चर्षयो महाभागा दिक्षु प्रत्येकशः स्थिताः ॥भीष्म उवाच}
{}


\twolineshloka
{श्रूयतां भरतश्रेष्ठ यन्मां त्वं परिपृच्छसि}
{प्रजानां पतयो ये च दिक्षु ये चर्षयः स्मृताः}


\twolineshloka
{एकः स्वयंभूर्भगवानाद्यो ब्रह्मा सनातनः}
{ब्रह्मणः सप्त वै पुत्रा महात्मानः स्वयंभुवः}


\twolineshloka
{मरीचिरत्र्यङ्गिरसौ पुलस्त्यः पुलहः क्रतुः}
{वसिष्ठश्च महाभागः सदृशो वै स्वयंभुवा}


\twolineshloka
{सप्त ब्रह्मण इत्येते पुराणे निश्चयं गताः}
{अत ऊर्ध्वं प्रवक्ष्यामि सर्वानेव प्रजापतीन्}


\twolineshloka
{अत्रिवंशतमुत्पन्नो ब्रह्मयोनिः सनातनः}
{प्राचनवर्हिर्भगवांस्तस्मात्प्राचेतसो दश}


\twolineshloka
{दशानां तनयस्त्वेको दक्षो नाम प्रजापतिः}
{तस्य द्वे नामनी लोके दक्षः क इति चोच्यते}


\twolineshloka
{मरीचेः कश्यपः पुत्रस्तस्य द्वे नामनी स्मृते}
{अरिष्टनेमिरित्येके कश्यपेत्यपरे विदुः}


\twolineshloka
{अत्रेश्चैवौरसः श्रीमान्राजा सोमश्च वीर्यवान्}
{सहस्रं यश्च दिव्यानां युगानां पर्युपासिता}


\twolineshloka
{अर्यमा चैव भगवान्ये चास्य तनया विभो}
{एते प्रदेशाः कथिता भुवनानां प्रभावनाः}


\twolineshloka
{शशबिन्दोश्च भार्याणां सहस्राणि दशाच्युत}
{एकैकस्यां सहस्रं तु तनयानामभूत्तदा}


\twolineshloka
{एवं शतसहस्राणि दश तस्य महात्मनः}
{पुत्राणां च न ते संचिदिच्छन्त्यन्यं प्रजापतिम्}


\twolineshloka
{प्रजामाचक्षते विप्राः पुराणाः शाशबिन्दवीम्}
{स वृष्णिवंशप्रभवो महावंशः प्रजापतेः}


% Check verse!
एते प्रजानां पतयः समुद्दिष्टा यशस्विनः
\twolineshloka
{`शशबिन्दुस्तु राजर्षिर्महायोगी महामनाः}
{अध्यात्मवित्सहस्राणां भार्याणां दशमध्यगः}


\twolineshloka
{स योगी योगमापन्नस्ततः सायुच्यतां गतः}
{'अतः परं प्रवक्ष्यामि देवांस्त्रिभुवनेश्वरान्}


\twolineshloka
{भयोंऽशश्चार्यमा चैव मित्रोऽथ वरुणस्तथा}
{सविता चैव घाता च विवस्वांश्च महाबलः}


\twolineshloka
{त्वष्टा पूषा तथैवेन्द्रो द्वादशो विष्णुरुच्यते}
{इत्येते द्वादशादित्याः कश्यपस्यात्मसंभवाः}


\twolineshloka
{नासत्यश्चैव दस्रश्च स्मृतो द्वावश्विनावपि}
{मार्तण्डस्यात्मजावेतावात्मस्य प्रजापतेः}


% Check verse!
त्वष्टुश्चैवात्मजः श्रीमान्विश्वरूपो महायशाः
\twolineshloka
{अजैकपादहिर्बुध्न्यो विरूपाक्षोऽथ रैवतः}
{हरश्च बहुरूपश्च त्र्यम्बकश्च सुरेश्वरः}


\twolineshloka
{सावित्रश्च जयन्तश्च पिनाकी चापराजितः}
{`एकादशैते कथिता रुद्रास्त्रिभुवनेश्वराः ॥'}


\threelineshloka
{पूर्वमेव महाभागा वसवोऽष्टौ प्रकीर्तिताः}
{एत एवविधा देवा मनोरेव प्रजापतेः}
{ते च पूर्वं सुराश्चेति द्विविधाः पितरः स्मृताः}


\twolineshloka
{शीलयौवनयोस्त्वन्यस्तथाऽन्ये सिद्धसाध्ययोः}
{ऋभवो मरुतश्चैव देवानां चोदितो गणः}


\twolineshloka
{एवमेते समाम्नाता विश्वेदेवास्तथाऽश्विनौ}
{आदित्याः क्षत्रियास्तेषां विशश्च मरुतस्तथा}


\threelineshloka
{अश्विनौ तु स्मृतौ शूद्रौ तपस्युग्रे समास्थितौ}
{स्मृतास्त्वङ्गिरसो देवा ब्राह्मणा इति निश्चयः}
{इत्येतत्सर्वदेवानां चातुर्वर्ण्यं प्रकीर्तितम्}


\twolineshloka
{एतान्वै प्रातरुत्थाय देवान्यस्तु प्रकीर्तयेत्}
{स्वजादन्यकृताच्चैव सर्वपापात्प्रमुच्यते}


\twolineshloka
{यवक्रीतोऽथ रैभ्यश्च अर्वावसुपरावसू}
{औशिजश्चैव कक्षीवान्बलश्चाङ्गिरसः स्मृतः}


\twolineshloka
{ऋषिर्मेधातिथेः पुत्रः कण्वो बर्हिषदस्तथा}
{त्रैलोक्यभावनास्तात प्राच्यां सप्तर्षयस्तथा}


\twolineshloka
{उन्मुचो विमुचश्चैव स्वस्त्यात्रेयश्च वीर्यवान्}
{प्रमुचश्चेध्मवाहश्च भगवांश्च दृढव्रतः}


\twolineshloka
{मित्रावरुणयोः पुत्रस्तथाऽगस्त्यः प्रतापवान्}
{एते ब्रह्मर्षयो नित्यमास्थिता दक्षिणां दिशम्}


\twolineshloka
{उषङ्गुः कवषो धौम्यः परिव्याधश्च वीर्यवान्}
{एकतश्च द्वितश्चैव त्रितश्चैवं महर्षयः}


\twolineshloka
{अत्रेः पुत्रश्च दुर्वासास्तथा सारस्वतः प्रभुः}
{एते चैव महात्मानः पश्चिमामाश्रिता दिशम्}


\twolineshloka
{अत्रिश्चैव वसिष्ठश्च काश्यपश्च महानुषिः}
{गौतमोऽथ भरद्वाजो विश्वामित्रोऽथ कौशिकः}


\twolineshloka
{तथैव पुत्रो भगवानृचीकस्य महात्मनः}
{जमदग्निश्च सप्तैते उदीचीमाश्रिता दिशम्}


\twolineshloka
{एते प्रतिदिशं सर्वे कीर्तितास्तिग्मतेजसः}
{साक्षिभूता महात्मानो भुवनानां प्रभावनाः}


\twolineshloka
{एवमेते महात्मानः स्थिताः प्रत्येकशो दिशम्}
{एतेषां कीर्तनं कृत्वा सर्वपापात्प्रमुच्यते}


\twolineshloka
{यस्यांयस्यां दिशि ह्येते तां दिशं शरणं गतः}
{मुच्यते सर्वपापेभ्यः स्वस्तिमांश्च तथा भवेत्}


\chapter{अध्यायः २०८}
\twolineshloka
{युधिष्ठिर उवाच}
{}


\twolineshloka
{पितामह महाप्राज्ञ युधि सत्यपराक्रम}
{श्रोतुमिच्छामि कार्त्स्न्येन वृष्णमव्ययमीश्वरम्}


\twolineshloka
{यच्चास्य तेजः सुमहद्यच्च कर्म पुरा कृतम्}
{तन्मे सर्वं यथातत्त्वं ब्रूहि त्वं पुरुषर्षभ}


\threelineshloka
{तिर्यग्योनिगतो रूपं कथं धारितवान्प्रभुः}
{केन कार्यनिसर्गेण तमाख्याहि महाबल ॥भीष्म उवाच}
{}


\twolineshloka
{पुराऽहं मृगयां यातो मार्कण्डेयाश्रमे स्थितः}
{तत्रापश्यं मुनिगणान्समासीनान्सहस्रशः}


\threelineshloka
{ततस्ते मधुपर्केण पूजां चक्रुरथो मयि}
{प्रतिगृह्य च तां पूजां चक्रुरथो मयि}
{}


\twolineshloka
{कथैषा कथिता तत्र कश्यपेन महर्षिणा}
{मनः प्रह्वादिनीं दिव्यां तामिहैकमनाः शृणु}


\twolineshloka
{पुरा दानवमुख्या हि क्रोधलोभसमन्विताः}
{बलेन मत्ताः शतशो नरकाद्या महासुराः}


\twolineshloka
{तथैव चान्ये बहवो दानवा युद्धदुर्मदाः}
{न सहन्ते स्म देवानां समृद्धिं तामनुत्तमाम्}


\twolineshloka
{`नराकाद्या महाघोरा हिरण्याक्षमुपाश्रिताः}
{उद्योगं परमं चक्रुर्देवानां निग्रहे तदा}


\threelineshloka
{नियुतं वत्सराणां तु वायुभक्षोऽभवत्तदा}
{हिरण्याक्षो महारौद्रो लेभे देवात्पितामहात्}
{वरानचिन्त्यानतुलाञ्शतशोऽथ सहस्रशः ॥'}


\twolineshloka
{दानवैरर्द्यमानास्तु देवा देवर्षयस्तथा}
{न शर्म लेभिरे राजन्क्लिश्यमानास्ततस्ततः}


\threelineshloka
{पृथिवीमार्तरूपां ते समपश्यन्दिवौकसः}
{दानवैरभिसंकीर्णां घोररूपैर्महाबलैः}
{भारार्तामप्रहृष्टां च दुःखितां संनिमज्जतीम्}


\threelineshloka
{`गृहीत्वा पृथिवी देवी पाताले न्यवसत्तदा}
{ततस्त्रैलोक्यमखिलं निरोषधिगणान्वितम्}
{निःस्वाध्यायवषट्कारमभूत्सर्वं समन्ततः ॥'}


\twolineshloka
{अथादितेयाः संत्रस्ता ब्रह्माणमिदमब्रुवन्}
{कथं शक्ष्यामहे ब्रह्मन्दानवैरभिमर्दनम्}


\twolineshloka
{`हिरण्याक्षेण भगवन्गृहीतेयं वसुन्धरा}
{न शक्ष्यामो वयं तत्र प्रवेष्टुं जलदुर्गमम्}


\twolineshloka
{तानाह भगवान्ब्रह्मा मुनिरेव प्रसाद्यताम्}
{अगस्त्योऽसौ महातेजाः पातु तज्जलमञ्जसा}


\twolineshloka
{तथेति चोक्त्वा ते देवा मुनिमूचुर्मुदान्विताः}
{त्रायस्व लोकान्विप्रर्षे जलमेतत्क्षयं नय}


\twolineshloka
{तथेति चोक्त्वा भगवान्कालानलसमद्युतिः}
{ध्यायञ्जलादनिवहं स क्षणेन पपौ जलम्}


\threelineshloka
{शोषिते तु समुद्रे च देवाः सर्षिपुरोगमाः}
{ब्रह्माणं प्रणिपत्योचुर्मुनिना शोषितं जलम्}
{इति भूयः समाचक्ष्व किं करिष्यामहे विभो}


% Check verse!
स्वयंभूस्तानुवाचेदं निसृष्टोऽत्र विधिर्मया
\threelineshloka
{ते वरेणाभिसंपन्ना बलेन च मदेन च}
{नावबुद्ध्यन्ति संमूढा विष्णुमव्यक्तदर्शनम्}
{वराहरूपिणं देवमधृष्यममरैरपि}


\threelineshloka
{एष वेगेन गत्वा हि यत्र ते दानवाधमाः}
{अन्तर्भूमिगता घोरा निवसन्ति सहस्रशः}
{शमयिष्यति तच्छ्रुत्वा जहृषुः सुरसत्तमाः}


\twolineshloka
{ततो विष्णुर्महातेजा वाराहं रूपमास्थितः}
{अन्तर्भूमिं संप्रविश्य जगाम दितिजान्प्रति}


\twolineshloka
{दृष्ट्वा च सहिताः सर्वे दैत्याः सत्वममानुषम्}
{प्रसह्य तरसा सर्वे संतस्थुः कालमोहिताः}


\twolineshloka
{ततस्ते समभिद्रुत्य वराहं जगृहुः समम्}
{संक्रुद्धाश्च वराहं तं व्यकर्षन्त समन्ततः}


\twolineshloka
{दानवेन्द्रा महाकाया महावीर्यबलोच्छ्रिताः}
{नाशक्नुवंश्च किंचित्ते तस्य कर्तुं तदा विभो}


\twolineshloka
{ततोऽगच्छन्विस्मयं ते दानवेन्द्रा भयं तथा}
{संशयं गतमात्मानं मेनिरे च सहस्रशः}


\twolineshloka
{ततो देवाधिदेवः स योगात्मा योगसारथिः}
{योगमास्थाय भगवांस्तदा भरतसत्तम}


\twolineshloka
{विननाद महानादं क्षोभयन्दैत्यदानवान्}
{सन्नादिता येन लोकाः सर्वाश्चैव दिशो दश}


\twolineshloka
{तेन सन्नादशब्देन लोकानां क्षोभ आगमत्}
{संश्रान्ताश्च दिशः सर्वा देवाः शक्रपुरोगमाः}


\twolineshloka
{निर्विचेष्टं जगच्चापि बभूवातिभृशं तदा}
{स्थावरं जङ्गमं चैव तेन नादेन मोहितम्}


\twolineshloka
{ततस्ते दानवाः सर्वे तेन नादेन भीषिताः}
{पेतुर्गतासवश्चैव विष्णुतेजः प्रमोहिताः}


\threelineshloka
{`त्रस्तांश्च देवानालोक्य ब्रह्मा प्राह पितामहः}
{योगेश्वरोऽयं भगवान्वाराहं रूपमास्थितः}
{नर्दमानोऽत्र संयाति मा भैष्ट सुरसत्तमाः}


\twolineshloka
{एवमुक्त्वा ततो ब्रह्मा नमश्चक्रे पितामहः}
{देवता मुनयश्चैव विष्णुं वै मुक्तिहेतवे}


\threelineshloka
{ततो हरिर्महातेजा ब्रह्माणमभिनन्द्य च}
{'रसातलगतश्चापि वराहस्त्रिदशद्विषाम्}
{खुरैर्विदारयामास मांसमेदोस्थिसंचयान्}


\twolineshloka
{नादेन तेन महता सनातन इति स्मृतः}
{पद्मनाभो महायोगी भूतात्मा भूतभावनः}


\twolineshloka
{ततो देवगणाः सर्वे पितामहमुपाद्रवन्}
{तत्र गत्वा महात्मानमूचुश्चैव जगत्पतिम्}


\threelineshloka
{नादोऽयं कीदृशो देव नेतं विद्म वयं प्रभो}
{कोसौ हि कस्य वा नादो येन विह्वलितं जगत्}
{देवाश्च दानवाश्चैव मोहितास्तस्य तेजसा}


\threelineshloka
{एतस्मिन्नन्तरे विष्णुर्वाराहं रूपमास्थितः}
{उदतिष्ठन्महाबाहो स्तूयमानो महर्षिभिः ॥पितामह उवाच}
{}


\twolineshloka
{`दिव्यं------ युद्धमासीन्महात्मनोः}
{हिरण्याक्षस्य विष्णोश्च सर्वसंक्षोभकारणम्}


\threelineshloka
{जघान च हिरण्याक्षमन्तर्भूमिगतं हरिः}
{तदाकर्ण्य महातेजा ब्रह्मा मधुरमब्रवीत् ॥'पीतामह उवाच}
{}


\twolineshloka
{निहत्य दानवपतीन्महावर्ष्मा महाबलः}
{एष देवो महायोगी भूतात्मा भूतभावनः}


\twolineshloka
{सर्वभूतेश्वरो योगी मुनिरात्मा तथाऽऽत्मनः}
{स्थिरीभवत कृष्णोऽयं सर्वविध्नविनाशनः}


\twolineshloka
{कृत्वा कर्मातिसाध्वेतदशक्यममितप्रभः}
{समायातः स्वमात्मानं महाभागो महाद्युतिः}


\twolineshloka
{पद्मनाभो महायोगी पुराणपुरुषोत्तमः}
{न संतापो न भीः कार्या शोको वा सुरसत्तमैः}


\twolineshloka
{विधिरेष प्रभावश्च कालः संक्षयकारकः}
{लोकान्धारयता तेन नादो मुक्तो महात्मना}


\twolineshloka
{स एष हि महाबाहुः सर्वलोकनमस्कृतः}
{अच्युतः पुण़्डरीकाक्षः सर्वभूतादिरीश्वरः}


\chapter{अध्यायः २०९}
\twolineshloka
{`* युधिष्ठिर उवाच}
{}


\twolineshloka
{पितामह महाप्राज्ञ केशवस्य महात्मनः}
{वक्तुमर्हसि तत्त्वेन माहात्म्यं पुनरेव तु}


\threelineshloka
{न तृप्याम्यहमप्येनं पश्यञ्शृण्वंश्च भारत}
{एवं कृष्णं महाबाहो तस्मादेतद्ब्रवीहि मे ॥भीष्म उवाच}
{}


\twolineshloka
{शृणु राजन्कथामेतां वैष्णवीं पापनाशनीम्}
{नारदो मां पुरा प्राह यामहं ते वदामि ताम्}


\twolineshloka
{देवर्षिर्नारदः पूर्वं तत्वं वेत्स्यामि वै हरेः}
{इति संचिन्त्य मनसा दध्यौ ब्रह्म सनातनम्}


\twolineshloka
{हिमालये शुभे दिव्ये दिव्यं वर्षशतं किल}
{अनुच्छ्वसन्निराहारः संयतात्मा जितेन्द्रियः}


\twolineshloka
{ततोऽन्तरिक्षे वागासीत्तं मुनिप्रवरं प्रति}
{मेघगम्भीरनिर्घोषा दिव्या वाह्याऽशरीरिणी}


\twolineshloka
{किमर्थं त्वं समापन्नो ध्यानं मुनिवरोत्तम}
{अहं ददामि ते ज्ञानं धर्माद्यं वा वृणीष्व माम्}


\twolineshloka
{तच्छ्रुत्वा मुनिरालोच्य संभ्रमाविष्टमानसः}
{किंनु स्यादिति संचिन्त्य वाक्यमाहापरं प्रति}


\twolineshloka
{कस्त्वं भवानण्डं बिभेद मध्येसमास्थितो वाक्यमुदीरयन्माम्}
{न रूपमन्यत्तव दृश्यते वैईदृग्विधस्त्वं समधिष्ठितोऽसि}


\twolineshloka
{पुनस्तमाह स मुनिमनन्तोऽहं बृहत्तरः}
{न मां मूढा विजानन्ति ज्ञानिनो मां विदन्त्युत}


\twolineshloka
{तं प्रत्याह मुनिः श्रीमान्प्रणतो विनयान्वितः}
{भवन्तं ज्ञातुमिच्छामि तव तत्वं ब्रवीहि मे}


\threelineshloka
{तस्य तद्वचनं श्रुत्वा नारदं प्राह लोकपः}
{ज्ञानेन मां विजानीहि नान्यथा शक्तिरस्ति ते ॥नारद उवाच}
{}


\threelineshloka
{कीदृग्विधं तु तज्ज्ञानं येन जानामि ते तनुम्}
{अनन्त तन्मे ब्रूहि त्वं यद्यनुग्रहवानहम् ॥लोकपाल उवाच}
{}


\twolineshloka
{विकल्पहीनं विपुलं तस्य चूरं शिवं परम्}
{ज्ञानं तत्तेन जानासि साधनं प्रति ते मुने}


\threelineshloka
{अत्रावृत्य स्थितं ह्येतत्तच्छुद्धमितरन्मृषा}
{एतत्ते सर्वमाख्यातं संक्षेपान्मुनिसत्तम ॥नारद उवाच}
{}


\twolineshloka
{त्वमेव तव यत्तत्वं ब्रूहि लोकगुरो मम}
{भवन्तं ज्ञातुमिच्छामि कीदृग्भूतस्त्वमव्यय}


\twolineshloka
{ततः प्रहस्य भगवान्मेघगम्भीरया गिरा}
{प्राहेशः सर्वभूतानां न मे चास्यं श्रुतिर्न च}


\twolineshloka
{न घ्राणजिह्वे दृक्चैव त्वचा नास्ति तथा मुने}
{कथं वक्ष्यामि चात्मानमशरीरस्तथाप्यहम्}


\threelineshloka
{तज्ज्ञात्वा विस्मयाविष्टो मुनिराह प्रणम्य तम्}
{येन त्वं पूर्वमात्मानमनन्तोऽहं बृहत्तरः}
{शतोऽहमिति मां प्रीतः प्रोक्तवानसि तत्कथम्}


\twolineshloka
{पुनस्तमाह भगवांस्तवाप्यक्षाणि सन्ति वै}
{त्वमेनं ब्रूहि चात्मानं यदि शक्नोषि नारद}


\threelineshloka
{आत्मा यथा तव मुने विदितस्तु भविष्यति}
{मां च जानासि तेन त्वमेकं साधनमावयोः}
{इत्युक्त्वा भगवान्देवस्ततो नोवाच किंचन}


\twolineshloka
{नारदोऽप्युत्स्मयन्खिन्नः क्व गतोऽसाविति प्रभुः}
{स्थित्वा स दीर्घकालं च मुनिर्व्यामूढमानसः}


\twolineshloka
{आह मां भगवान्देवस्त्वनन्तोऽहं बृहत्तरः}
{तेनाहमिति सर्वस्य को वानन्तो बृहत्तरः}


\threelineshloka
{केयमुर्वी ह्यनन्ताख्या बृहती नूनमेव सा}
{यस्यां जानन्ति भूतानि विलीनानि ततस्ततः}
{एनां पृच्छामि तरुणीं सैषा नूनमुवाच माम्}


\twolineshloka
{इत्येवं स मुनिः श्रीमान्कृत्वा निश्चयमात्मनः}
{स भूतलं समाविश्य प्रणिपत्येदमब्रवीत्}


\twolineshloka
{आश्चर्यासि च धन्यासि वृहती त्वं वसुन्धरे}
{त्वामत्र वेत्तुमिच्छामि याग्दृभूताऽसि शोभने}


\twolineshloka
{तच्छ्रुत्वा धरणी देवी स्मयमानाऽब्रवीदिदम्}
{नाहं हि बृहती विप्र न चानन्ता च सत्तम}


\twolineshloka
{कारणं मम यो गन्धो गन्धात्मानं ब्रवीहि तम्}
{ततो मुनिस्तद्धि तत्वं प्रणिपत्येदमब्रवीत्}


% Check verse!
कारणं मे जलं मत्तो बृहत्तरतमं हि तत्
\twolineshloka
{स समुद्रं मुनिर्गत्वा प्रणिपत्येदमब्रवीत्}
{आश्चर्योसि च धन्योसि ह्यनन्तोसि बृहत्तरः}


\twolineshloka
{भवन्तं वेत्तुमिच्छामि कीदृग्भूतस्त्यमव्यय}
{तच्छ्रुत्वा सरितानाथः समुद्रो मुनिमब्रवीत्}


\twolineshloka
{कारणं मेऽत्र संपृच्छ रसात्मानं बृहत्तरम्}
{ततो बृहत्तरं विद्वंस्त्वं पृच्छ मुनिसत्तम}


\twolineshloka
{ततो मुनिर्यथायोगं जलं तत्वमवेक्ष्य तत्}
{जलात्मानं प्रणम्याह जलतत्वस्थितो मुनिः}


\twolineshloka
{आश्चर्योसि च धन्योसि ह्यनन्तोसि बृहत्तरः}
{भवन्तं श्रोतुमिच्छामिकीदृग्भूतस्त्वमव्यय}


\threelineshloka
{ततो रसात्म--मुनिमाह पुनः पुनः}
{ममापि कारणं पृच्छ तेजोरूपं विभावसुम्}
{नाहं बृहत्तरो ब्रह्मन्नाप्यनन्तश्च सत्तम्}


\twolineshloka
{ततोऽग्निं प्रणिपत्याह मुनिर्विस्मितमानसः}
{यज्ञात्मानं महावासं सर्वभूतनमस्कृतम्}


\twolineshloka
{आश्चर्योसि च धन्योसि ह्यनन्तश्च बृहत्तरः}
{भवन्तं वेत्तुमिच्छामि कीदृग्भूतस्त्वमव्यय}


\threelineshloka
{ततः प्रहस्य भगवान्मुनिं स्विष्टकृदब्रवीत्}
{नाहं बृहत्तरो ब्रह्मन्नाप्यनन्तश्च सत्तम}
{कारणं मम रूपं यत्तं पृच्छ मुनिसत्तम}


\twolineshloka
{ततो योगक्रमेणैव प्रतीतं तं प्रविश्य सः}
{रूपात्मानं प्रणम्याह नारदो वदतांवरः}


\twolineshloka
{आश्चर्योसि च धन्योसि ह्यनन्तोसि बृहत्तरः}
{भवन्तं वेत्तुमिच्छामि कीदृग्भूतस्त्वमव्यय}


\threelineshloka
{उत्स्मयित्वा तु रूपात्मा तं मुनिं प्रत्युवाच ह}
{वायुर्मे कारणं ब्रह्मंस्तं पृच्छ मुनिसत्तम}
{मत्तो बहुतरः श्रीमाननन्तश्च महाविलम्}


\twolineshloka
{स मारुतं प्रणम्याह भगवान्मुनिसत्तमः}
{योगसिद्धो महायोगी ज्ञानविज्ञानपारगः}


\twolineshloka
{आश्चर्योसि च धन्योसि ह्यनन्तोसि बृहत्तरः}
{भवन्तं वेत्तुमिच्छामि कीदृग्भूतस्त्वमव्यय}


\twolineshloka
{ततो वायुर्हि संप्राह नारदं मुनिसत्तमम्}
{कारणं पृच्छ भगवन्स्पर्शात्मानं ममाद्य वै}


\twolineshloka
{मत्तो बृहत्तरः श्रीमाननन्तश्च तथैव सः}
{ततोस्य वचनं श्रुत्वा स्पर्शात्मानमुवाच सः}


\twolineshloka
{आश्चर्योसि च धन्योसि ह्यनन्तोसि बृहत्तरः}
{भवन्तं वेत्तुमिच्छामि कीदृग्भूतस्त्वमव्यय}


\twolineshloka
{तस्य तद्वचनं श्रुत्वा स्पर्शात्मा मुनिमब्रवीत्}
{नाहं वृहत्तरो ब्रह्मन्नाप्यनन्तश्च सत्तम}


\twolineshloka
{कारणं मम चैवेममाकाशं च बृहत्तरम्}
{तं पृच्छ मुनिशार्दूल सर्वव्यापिनमव्ययम्}


\twolineshloka
{तच्छ्रुत्वा नारदः श्रीमान्वाक्यं वाक्यविशारदः}
{आकाशं समुपागम्य प्रणम्याह कृताञ्जलिः}


\twolineshloka
{आश्चर्योसि न धन्योसि ह्यनन्तोसि बृहत्तरः}
{भवन्तं वेत्तुमिच्छामि कीदृग्भूतस्त्वमव्यय}


\threelineshloka
{आकाशस्तमुवाचेदं प्रहसन्मुनिसत्तमम्}
{नाहं बृहत्तरो ब्रह्मञ्शब्दो वै कारणं मम}
{तं पृच्छ मुनिशार्दूल स वै मत्तो बृहत्तरः}


\threelineshloka
{ततो ह्याविश्य चाकाशं शब्दात्मानमुवाच ह}
{स्वरव्यञ्जनसंयुक्तं नानाहेतुविभूषितम्}
{वेदाख्यं परमं गुह्यं वेदकारणमच्युतम्}


\twolineshloka
{आश्चर्योसि च धन्योसि ह्यनन्तोसि बृहत्तरः}
{भवन्तं श्रोतुमिच्छामि कीदृग्भूतस्त्वमव्यय}


\twolineshloka
{वेदात्मा प्रत्युवाचेदं नारदं मुनिपुङ्गवम्}
{मया कारणभूतेन सर्ववेत्ता पितामहः}


\threelineshloka
{ब्रह्मणो बुद्धिसंस्थानमास्थितोऽहं महामुने}
{तस्माद्वृहत्तरो मत्तः पद्मयोनिर्महामतिः}
{तं पृच्छ मुनिशार्दूल सर्वकारणकारणम्}


\twolineshloka
{ब्रह्मलोकं ततो गत्वा नारदो मुनिपुङ्गवैः}
{सेव्यमानं महात्मानं लोकपालैर्मरुद्गणैः}


\twolineshloka
{समुद्रैश्च सरिद्भिश्च भूततत्वैः सभूधरैः}
{गन्धर्वैरप्सरोभिश्च ज्योतिषां च गणैस्तथा}


\twolineshloka
{स्तुतिस्तोमग्रहस्तोभैस्तथा वेदैर्मुनीश्वरैः}
{उपास्यमानं ब्रह्माणं लोकनाथं परात्परम्}


\twolineshloka
{हिरण्यगर्भं विश्वेशं चतुर्वक्रेण भूषितम्}
{प्रणम्य प्राञ्जलिः प्रह्वस्तमाह मुनिपुङ्गवः}


\twolineshloka
{आश्चर्योसि च धन्योसि ह्यनन्तोसि बृहत्तरः}
{भवन्तं वेत्तुमिच्छामि कीदृग्भूतस्त्वमव्यय}


\twolineshloka
{तच्छ्रुत्वा भगवान्ब्रह्मा सर्वलोकपितामहः}
{उत्स्मयन्मुनिमाहेदं कर्ममूलस्य लोपकम्}


\twolineshloka
{नाहं बृहत्तरो ब्रह्मन्नाप्यनन्तश्च सत्तम}
{लोकानां मम सर्वेषां नाथभूतो बृहत्तरः}


\twolineshloka
{नन्दगोपकुले गोपकुमारैः परिवारितः}
{समस्तजगतां गोप्ता गोपवेषेण संस्थितः}


\twolineshloka
{मद्रूपं च समास्थाय जगत्सृष्टिं करोति सः}
{ऐशानमास्थितः श्रीमान्हन्ति नित्यं हि पाति च}


\twolineshloka
{विष्णुः स्वरूपरूपोऽसौ कारणं स हरिर्मम}
{तं पृच्छ मुनिशार्दूल स चानन्तो बृहत्तरः}


\twolineshloka
{ततोऽवतीर्य भगवान्ब्रह्मलोकान्महामुनिः}
{नन्दगोपकुले विष्णुमेनं कृष्णं जगत्पतिम्}


\twolineshloka
{बालक्रीडनकासक्तं वत्सजालविभूषितम्}
{पाययित्वाथ बध्नन्तं धूलिधूम्राननं परम्}


\twolineshloka
{गाहमानैर्हसद्भिश्च नृत्यद्भिश्च समन्ततः}
{पाणिवादनकैश्चैव संवृतं वेणुवादकैः}


\threelineshloka
{प्रणिपत्याब्रवीदेनं नारदो भगवान्मुनिः}
{आश्चर्योसि च धन्योसि ह्यनन्तश्च बृहत्तरः}
{वेत्ताऽसि चाव्ययश्चासि वेत्तुमिच्छामि यादृशम्}


\twolineshloka
{ततः प्रहस्य भगवान्नारदं प्रत्युवाच ह}
{मत्तः परतरं नास्ति मत्तः सर्वं प्रतिष्ठितम्}


\twolineshloka
{मतो बृहत्तरं नान्यदहमेव बृहत्तरः}
{आकाशे च स्थितः पूर्वमुक्तवानहमेव ते}


\twolineshloka
{न मां वेत्ति जनः कश्चिन्माया मम दुरत्यया}
{भक्त्या त्वनन्यया युक्ता मां विजानन्ति योगिनः}


\twolineshloka
{प्रियोसि मम भक्तोसि मम तत्वं विलोकय}
{ददामि तव तज्ज्ञानं येन तत्वं प्रपश्यसि}


\twolineshloka
{अन्येषां चैव भक्तानां मम योगरतात्मनाम्}
{ददामि दिव्यं ज्ञानं च येन तत्वं प्रपश्यसि}


\twolineshloka
{अन्येषां चैव भक्तानां मम योगरतात्मनाम्}
{ददामि दिव्यं ज्ञानं च तेन ते यान्ति मत्पदम्}


\twolineshloka
{एवमुक्त्वा ययौ कृष्णो नन्दगोपगृहं हरिः ॥भीष्म उवाच}
{}


\twolineshloka
{एतत्ते कथितं राजन्विष्णुतत्वमनुत्तमम्}
{भजस्वैनं विशालाक्षं जपन्कृष्णेति सत्तम}


\twolineshloka
{मोहयन्मां तथा त्वां च शृणोत्येष मयेरितान्}
{धर्मात्मा च महाबाहो भक्तान्रक्षति नान्यथा}


\chapter{अध्यायः २१०}
\twolineshloka
{`युधिष्ठिर उवाच}
{}


\twolineshloka
{पितामह महाप्राज्ञ सर्वशास्त्रविशारद}
{प्रयाणकाले किं जप्यं मोक्षिभिस्तत्त्वचिन्तकैः}


\threelineshloka
{किंनु स्मरन्कुरुश्रेष्ठ मरणे समुपस्थिते}
{प्राप्नुयात्परमां सिद्धिं श्रोतुमिच्छामि तत्वतः ॥भीष्म उवाच}
{}


\twolineshloka
{त्वद्युक्तश्च हितः सूक्ष्म उक्तः प्रश्नस्त्वयाऽनघ}
{शृणुष्वावहितो राजन्नारदेन पुरा श्रुतम्}


\twolineshloka
{श्रीवत्साङ्कं जगद्बीजमनन्तं लोकसाक्षिणम्}
{पुरा नारायणं देवं नारदः पर्यपृच्छत}


\twolineshloka
{अक्षरं परमं ब्रह्म निर्गुणं तमसः परम्}
{आहुर्वैद्यं परं धाम ब्रह्मादिकमलोद्भवम्}


\twolineshloka
{भगवन्भूतभव्येश श्रद्दधानैर्जितेन्द्रियैः}
{कथं भक्तैर्विचिन्त्योसि योगिभिर्मोक्षकाङ्क्षिभिः}


\twolineshloka
{किंनु जप्यं जपेन्नित्यं काल्यमुत्थाय मानवः}
{स्मरेच्च म्रियमाणो वै विशेषेण महाद्युते}


% Check verse!
कथं युञ्जन्समाध्यायेद्ब्रूहि तत्वं सनातनम्
\twolineshloka
{श्रुत्वा च नारदोक्तं तु देवानामीश्वरः स्वयम्}
{प्रोवाच भगवान्विष्णुर्नारदं वरदः प्रभुः}


\twolineshloka
{हन्त ते कथयिष्यामि इमां दिव्यामनुस्मृतिम्}
{यामधीत्य प्रयाणे तु मद्भावयोपपद्यते}


\twolineshloka
{ओंकारमग्रतः कृत्वा मां नमस्कृत्य नारद}
{एकाग्रः प्रयतो भूत्वा इमं मन्त्रमुदीरयेत्}


% Check verse!
ओं नमो भगवते वासुदेवायेति
\threelineshloka
{इत्युक्तो नारदः प्राह प्राञ्जलिः प्रणतः स्थितः}
{सर्वदेवेश्वरं विष्णुं सर्वात्मानं हरिं प्रभुम् ॥नारद उवाच}
{}


\twolineshloka
{अव्ययं शाश्वतं देवं प्रभवं पुरुषोत्तमम्}
{प्रपद्ये प्राञ्जलिर्विष्णुमक्षरं परमं पदम्}


\twolineshloka
{पुराणं प्रभवं विष्णुमक्षयं लोकसाक्षिणम्}
{प्रपद्ये पुण्डरीकाक्षमीशं भक्तानुकम्पिनम्}


\twolineshloka
{लोकनाथं सहस्राक्षमद्भुतं परदं पदम्}
{भगवन्तं प्रपन्नोऽस्मि भूतभव्यभवत्प्रभुम्}


\twolineshloka
{स्रष्टारं सर्वलोकानामनन्तं सर्वतोमुखम्}
{पद्मनाभं हृषीकेशं प्रपद्ये सत्यमच्युतम्}


\twolineshloka
{हिरण्यगर्भममृतं भूगर्भं परतः परम्}
{प्रभोः प्रभुमनाद्यन्तं प्रपद्ये तं रविप्रभम्}


\twolineshloka
{सहस्रशीर्षं पुरुषं महर्षि तत्वभावनम्}
{प्रपद्ये सूक्ष्ममचलं वरेण्यमभयप्रदम्}


\twolineshloka
{नारायणं पुराणर्षि योगात्मानं सनातनम्}
{संस्थानं सर्वतत्वानां प्रपद्ये ध्रुवमीश्वरम्}


\twolineshloka
{यः प्रभुः सर्वभूतानां येन सर्वमिदं ततम्}
{परावरगुरुर्विष्णुः स मे देवः प्रसीदतु}


\twolineshloka
{यस्मादुत्पद्यते ब्रह्मा पद्मयोनिः सनातनः}
{ब्रह्मयोनिर्हि विश्वात्मा स मे विष्णुः प्रसीदतु}


\twolineshloka
{यः पुरा प्रलये प्राप्ते नष्टे स्थावरजङ्गमे}
{ब्रह्मादिषु प्रलीनेषु नष्टे लोकपरावरे}


\twolineshloka
{आभूतसंप्लवे चैव प्रलीनेऽप्राकृतो महान्}
{एकस्तिष्ठति विश्वात्मा स मे विष्णुः प्रसीदतु}


\twolineshloka
{चतुर्भिश्च चतुर्भिश्च द्वाभ्यां पञ्चभिरेव च}
{हुयते च पुनर्द्वाभ्यां स मे विष्णुः प्रसीदतु}


\twolineshloka
{पर्जन्यः पृथिवी सस्यं कालो धर्मः क्रियाक्रिये}
{गुणाकरः स मे बभ्रुर्वासुदेवः प्रसीदतु}


\twolineshloka
{अग्नीषोमार्कताराणां ब्रह्मरुद्रेन्द्रयोगिनाम्}
{यस्तेजयति तेजांसि स मे विष्णुः प्रसीदतु}


\twolineshloka
{योगावास नमस्तुभ्यं सर्वावास वरप्रद}
{यज्ञगर्भ हिरण्याङ्गं पञ्चयज्ञ नमोस्तु ते}


\twolineshloka
{चतुर्मूर्ते परं धाम लक्ष्म्यावास परार्चित}
{सर्वावास नमस्तेऽस्तु वासुदेव प्रधानकृत्}


\twolineshloka
{अजस्त्वनामयः पन्था ह्यमूर्तिर्विश्वमूर्तिधृत्}
{विकर्तः पञ्चकाज्ञ नमस्ते ज्ञानसागर}


\twolineshloka
{अव्यक्ताद्व्यक्तमुत्पन्नमव्यक्ताद्यः परोऽक्षरः}
{यस्मात्परतरं नास्ति तमस्मि शरणं गतः}


\twolineshloka
{न प्रधानो न च महान्पुरुषश्चेतनो ह्यजः}
{अनयोर्यः परतरस्तमस्मि शरणं गतः}


\twolineshloka
{चिन्तयन्तो हि यं नित्यं ब्रह्मेशानादयः प्रभुम्}
{निश्चयं नाधिगच्छन्ति तमस्मि शरणं गतः}


\twolineshloka
{जितेन्द्रिया महात्मानो ज्ञानध्यानपरायणाः}
{यं प्राप्य न निवर्तन्ते तमस्मि शरणं गतः}


\twolineshloka
{एकांशेन जगत्सर्वमवष्टभ्य विभुः स्थितः}
{अग्राह्यं निर्गुणं नित्यं तमस्मि शरणं गतः}


\twolineshloka
{सोमार्काग्निमयं तेजो या च तारमयी द्युतिः}
{दिवि संजायते योऽयं स महात्मा प्रसीदतु}


\twolineshloka
{गुणादिर्निर्गुणश्चाद्यो लक्ष्मीवांश्चेतनो ह्यजः}
{सूक्ष्मः सर्वगतो योगी स महात्मा प्रसीदतु}


\twolineshloka
{साङ्ख्ययोगाश्च ये चान्ये सिद्धाश्च परमर्षयः}
{यं विदित्वा विमुच्यन्ते स महात्मा प्रसीदतु}


\twolineshloka
{अव्यक्तः समधिष्ठाता अचिन्त्यः सदसत्परः}
{अस्थितिः प्रकृतिश्रेष्ठः स महात्मा प्रसीदतु}


\twolineshloka
{क्षेत्रज्ञः पञ्चधा भुङ्क्ते प्रकृतिं पञ्चभिर्मुखैः}
{महान्गुणांश्च यो भुङ्क्ते स महात्मा प्रसीदतु}


\twolineshloka
{सूर्यमध्ये स्थितः सोमस्तस्य मध्ये च या स्थिता}
{भूतबाह्या च या दीप्तिः स महात्मा प्रसीदतु}


\twolineshloka
{नमस्ते सर्वतः सर्वं सर्वतोक्षिशिरोमुख}
{निर्विकार नमस्तेऽस्तु साक्षी क्षेत्रध्रुवस्थितिः}


\twolineshloka
{अतीन्द्रिय नमस्तुभ्यं लिङ्गैर्व्यक्तैर्न मीयसे}
{ये च त्वां नाभिजानन्ति संसारे संसरन्ति ते}


\twolineshloka
{कामक्रोधविनिर्मुक्ता रागद्वेषविवर्जिताः}
{मान्यभक्ता विजानन्ति न पुनर्भवका द्विजाः}


\twolineshloka
{एकान्तिनो हि निर्द्वन्द्वा निराशीःकर्मकारिणः}
{ज्ञानाग्निदग्धकर्माणस्त्वां विशन्ति विचिन्तकाः}


\twolineshloka
{अशरीरं शरीरस्थं समं सर्वेषु देहिषु}
{पुण्यपापविनिर्मुक्ता भक्तास्त्वां प्राविशन्त्युत}


\twolineshloka
{अव्यक्तं बुद्ध्यहङ्कारमनोभूतेन्द्रियाणि च}
{त्वयि तानि च तेषु त्वं न तेषु त्वं न ते त्वयि}


\twolineshloka
{एकत्वान्यत्वनानात्वं ये विदुर्यान्ति ते परम्}
{समोसि सर्वभूतेषु न ते द्वेष्योस्ति न प्रियः}


\threelineshloka
{समत्वमभिकाङ्क्षेऽहं भक्त्या वै नान्यचेतसा}
{चराचरमिदं सर्वं भूतग्रामं चतुर्विधम्}
{त्वया त्वय्येव तत्प्रोतं सूत्रे मणिगणा इव}


\twolineshloka
{स्रष्टा भोक्तासि कूटस्थो ह्यतत्वं तत्वसंज्ञिकः}
{अकर्ता हेतुरचलः पृथगात्मन्यवस्थितः}


\twolineshloka
{न ते भूतेषु संयोगो भूततत्वगुणाधिकः}
{अहङ्कारेण बुद्ध्या वा न ते योगस्त्रिभिर्गुणैः}


\twolineshloka
{न मोक्षधर्मो वा न त्वं नारम्भो जन्म वा पुनः}
{जरामरणमोक्षार्थं त्वां प्रपन्नोस्मि सर्वग}


\twolineshloka
{ईश्वरोसि जगन्नाथ ततः परम उच्यसे}
{भक्तानां यद्धितं देव तद्ध्याहि त्रिदशेश्वर}


\twolineshloka
{विषयैरिन्द्रियैर्वाऽपि न मे भूयः समागमः}
{पृथिवीं यातु गन्धो वै रसं यातु जलं तथा}


\twolineshloka
{तेजो हुताशनं यातु स्पर्शो यातु च मारुतम्}
{श्रोत्रमाकाशमप्येतु मनो वैकारिकं पुनः}


\twolineshloka
{इन्द्रियाण्यपि संयान्तु स्वासुस्वासु च योनिषु}
{पृथिवी यातु सलिलमापोग्निमनलोऽनिलम्}


\twolineshloka
{वायुराकाशमप्येतु मनश्चाकाश एव च}
{अहंकारं मनो यातु मोहनं सर्वदेहिनाम्}


% Check verse!
अहंकारस्ततो बुद्धिं बुद्धिरव्यक्तमच्युत
\twolineshloka
{प्रधाने प्रकृतिं याते गुणसाम्ये व्यवस्थिते}
{वियोगः सर्वकरणैर्गुणैर्भूतैश्च मे भवेत्}


\twolineshloka
{निष्केवलं पदं तात काङ्क्षेऽहं परमं तव}
{एकीभावस्त्वया मेऽस्तु न मे जन्म भवेत्पुनः}


\twolineshloka
{त्वद्बुद्धिस्त्वद्गतप्राणस्त्वद्भक्तिस्त्वत्परायणः}
{त्वामेवाहं स्मरिष्यामि मरणे पर्युपस्थिते}


\twolineshloka
{पूर्वदेहकृता ये तु व्याधयः प्रविशन्तु माम्}
{अर्दयन्तु च दुःखानि ऋणं मे प्रविमुञ्चतु}


\twolineshloka
{अनुध्यातोऽसि देवेश न मे जन्म भवेत्पुनः}
{तस्माद्ब्रवीमि कर्माणि ऋणं मे न भवेदिति}


\threelineshloka
{नोपतिष्ठन्तु मां सर्वे व्याधयः पूर्वसंचिताः}
{अनृणो गन्तुमिच्छामि तद्विष्णोः परमं पदम् ॥श्रीभगवानुवाच}
{}


\twolineshloka
{अहं भगवतस्तस्य मम चासौ सनातनः}
{तस्याहं न प्रणश्यासि स च मे न प्रणश्यति}


\twolineshloka
{कर्मेन्द्रियामि संयम्य पञ्च बुद्धीन्द्रियाणि च}
{दशेन्द्रियाणि मनसि अहंकारे तथा मनः}


\twolineshloka
{अहंकारं तथा बुद्धौ बुद्धिमात्मनि योजयेत्}
{यतबुद्धीन्द्रियः पश्येद्बुद्ध्या बुद्ध्येत्परात्पम्}


\twolineshloka
{ममायमिति यस्याहं येन सर्वमिदं तततम्}
{ततो बुद्धेः परं बुद्ध्वा लभते न पुनर्भवम्}


\twolineshloka
{मरणे समनुप्राप्ते यश्चैवं मामनुस्मरेत्}
{अपि पापसमाचारः स याति परमां गतिम्}


\twolineshloka
{ओं नमो भगवते तस्मै देहिनां परमात्मने}
{नारायणाय भक्तानामेकनिष्ठाय शाश्वते}


\twolineshloka
{इमामनुस्मृतिं दिव्यां वैष्णवीं सुसमाहितः}
{स्वपन्विबुद्धश्च पठेद्यत्र तत्र समभ्यसेत्}


\twolineshloka
{पौर्णमास्याममावास्यां द्वादश्यां च विशेषतः}
{श्रावयेच्छ्रद्दधानांश्च मद्भक्तांश्च विशेषतः}


\twolineshloka
{यद्यहंकारमाश्रित्य यज्ञदानतपः क्रियाः}
{कुर्वंस्तत्फलमाप्नोति पुनरावर्तनं न तु}


\twolineshloka
{अभ्यर्चयन्पितॄन्देवान्पठञ्जुह्वन्यलिं ददत्}
{ज्वलन्नग्निं स्मरेद्यो मां स याति परमां गतिम्}


\twolineshloka
{यज्ञो दानं तपश्चैव पावनानि शरीरिणाम्}
{यज्ञं दानं तपस्तस्मात्कुर्यादाशीर्विवर्जितः}


\twolineshloka
{नम इत्येव यो ब्रूयान्मद्भक्तः श्रद्धयान्वितः}
{तस्याक्षयो भवेल्लोकः श्वपाकस्यापि नारद}


\twolineshloka
{किं पुनर्ये यजन्ते मां साधका विधिपूर्वकम्}
{श्रद्धावन्तो यतात्मानस्ते मां यान्ति मदाश्रिताः}


\threelineshloka
{कर्माण्याद्यन्तवन्तीह मद्भक्तोऽमृतमश्नुते}
{मामेव तस्माद्देवर्षे ध्याहि नित्यमतन्द्रितः}
{अवाप्स्यसि ततः सिद्धिं द्रक्ष्यस्येव पदं मम}


\twolineshloka
{अज्ञानिने च यो ज्ञानं दद्याद्धर्मोपदेशनम्}
{कृत्स्नां वा पृथिवीं दद्यात्तेन तुल्यं न तत्फलम्}


\twolineshloka
{तस्मात्प्रेदयं साधुभ्यो जन्मबन्धभयापहम्}
{एवं दत्त्वा नरश्रेष्ठ श्रेयो वीर्यं च विन्दति}


\threelineshloka
{अश्वमेधसहस्राणां सहस्रं यः समाचरेत्}
{नासौ पदमवाप्नोति मद्भक्तैर्यदवाप्यते ॥भीष्म उवाच}
{}


\twolineshloka
{एवं पृष्टः पुरा तेन नारदेन सुरर्षिणा}
{यदुवाच तदाऽसौ भो तदुक्तं तव सुव्रत}


\twolineshloka
{त्वमप्येकमना भूत्वा ध्याहि ज्ञेयं गुणातिगम्}
{भजस्व सर्वभावेन परमात्मानमव्ययम्}


\twolineshloka
{श्रुत्वैतन्नारदो वाक्यं दिव्यं नारायणेरितम्}
{अत्यन्तभक्तिमान्देव एकान्तत्वमुपेयिवान्}


\twolineshloka
{नारायणमृषिं देवं दशवर्षाण्यनन्यभाक्}
{इदं जपन्वै प्राप्नोति तद्विष्णोः परमं पदम्}


\twolineshloka
{किं तस्य बहुभिर्मन्त्रैर्भक्तिर्यस्य जनार्दने}
{नमो नारायणायेति मन्त्रः सर्वार्थसाधकः}


\twolineshloka
{इमां रहस्यां परमामनुस्मृतिमधीत्य बुद्धिं लभते च नैष्ठिकीम्}
{विहाय दुःखान्यवमुच्य सङ्कटात्स वीतरागो विगतज्वरः सुखी}


\chapter{अध्यायः २११}
\twolineshloka
{* युधिष्ठिर उवाच}
{}


\twolineshloka
{देवानुरमनुष्येषु ऋषिमुख्येषु वा पुनः}
{विष्णोस्तत्वं यथाख्यातं को विद्वाननुवेत्ति तत्}


\threelineshloka
{एतन्मे सर्वमाचक्ष्व न मे तृप्तिर्हि तत्वतः}
{वर्तते भरतश्रेष्ठ सर्वज्ञोऽसीति मे मतिः ॥भीष्म उवाच}
{}


\twolineshloka
{कारितोऽहं त्वया राजन्यदॄत्तं च पुरा मम}
{गरुडेन पुरा मह्यं संवादोऽभूभृतोत्तम्}


\threelineshloka
{पुराहं तप आस्थाय वासुदेवपरायणः}
{ध्यायन्स्तुवन्नमस्यंश्च यजमानस्तमेवच}
{गङ्गद्वीपे समासीनो दशवर्षाणि भारत}


\twolineshloka
{माता च मम ता देवी जननी लोकपावनी}
{समासीना समीपे मे रक्षणार्थं ममाच्युत}


\threelineshloka
{तस्मिन्कालेऽद्भुतः श्रीमान्सर्ववेदमयः प्रभुः}
{सुपर्णः पततांश्रेष्ठो मेरुमन्दरसन्निभः}
{आजगाम विशुद्धात्मा गङ्गां द्रष्टुं महायशाः}


\twolineshloka
{तमागतं महात्मानं प्रत्युद्गम्याहमर्थितः}
{प्रणिपत्य यथान्यायं कृताज्जलिरवस्थितः}


\twolineshloka
{सोऽपि देवो महाभागामभिनन्द्य च जाह्नवीम्}
{तथा च पूजितः श्रीमानुणेपाविशदासने}


\twolineshloka
{ततः कथान्तरे तं वै वचनं चेदमव्रवम्}
{वेदवेद महावीर्य वैनतेग महाबल}


\fourlineindentedshloka
{नारायणं हृषीकेशं सहमानोऽनिशं हरिम्}
{जानासि तं यथा वक्तुं यादृग्भूतो जनार्दनः}
{ममापि तस्य सद्भातं वक्तुमर्हसि सत्तम ॥गरुड उवाच}
{}


\twolineshloka
{शृणु भीष्म यथान्यायं पुरा त्वमिह सत्तमाः}
{अनेके पुनयः सिद्धा मानसोत्तरवासिनः ॥पगच्छुर्मा महाप्राज्ञा वासुदेवपरायणाः}


\twolineshloka
{पक्षीन्द्र वासुदेवस्य तत्वं वेत्सि परं पदम्}
{स्वसा सयो न तस्यास्ति सन्निकृष्टप्रियोपि च}


\twolineshloka
{तेषामहं वचः श्रुत्वा प्रणिपत्य महाहरिम्}
{अब्रवं च यथावृत्तं मम नारायणस्य च}


\twolineshloka
{शृणुध्वं मुनिशार्दूला हृत्वा सोममहं पुरा}
{आकाशे पतमानस्तु वाक्यं तत्र शृणोमि वै}


\twolineshloka
{साधुसाधु महाबाहो प्रीतोस्मि तव दर्शनात्}
{वृणीष्व वचनं मत्तः पक्षीन्द्र गरुडाधुना}


\twolineshloka
{त्वामहं भक्तितत्वज्ञो ब्रवै वचनमुत्तमम्}
{इत्याह स्म ध्रुवं तत्र मामाह भगवान्पुनः}


\twolineshloka
{ऋषिरस्मि महावीर्य न मां जानाति वा मयि}
{असूयति च मां मूढ तच्छ्रुत्वा गर्वमास्थितः}


\threelineshloka
{अहं देवनिकायानां मध्ये वचनमब्रवम्}
{ऋषे पूर्वं वरं मत्तस्त्वं वृणीष्व ततो ह्यहम्}
{वृणे त्वत्तो वरं पश्चादित्येवं मुनिसत्तमाः}


\threelineshloka
{तस्मात्त्वां भगवान्देवः श्रीमाञ्श्रीवत्सलक्षणः}
{अद्य पश्यति पक्षीन्द्र वाहनं भव मे सदा}
{वृणेऽहं वरमेतद्धि त्वत्तोऽद्य पतगेश्वर}


\twolineshloka
{तथेति तं वीक्ष्य मातामनहंकारमास्थितम्}
{जेतुकामो ह्यहं विष्णुं मायया मायिनं हरिम्}


\twolineshloka
{त्वत्तो ह्यहं वृणे त्वद्य वरं ऋषिवरोत्तम}
{तवोपरिष्टात्स्थास्यामि वरमेतत्प्रयच्छ मे}


\threelineshloka
{तथेति च हसन्प्राह हरिर्नारायणः प्रभुः}
{ध्वजं च मे भव सदा त्वमेव विहगेश्वर}
{उपरिष्टात्स्थितिस्तेऽस्तु मम पक्षीन्द्र सर्वदा}


\twolineshloka
{इत्युक्त्वा भगवान्देवः शङ्खचक्रगदाधरः}
{सहस्रचरणः श्रीमान्सहस्रादित्यसन्निभः}


\twolineshloka
{सहस्रशीर्षा पुरुषः सहस्रनयनो महान्}
{सहस्रमकुटोऽचिन्त्यः सहस्रवदनो विभुः}


\twolineshloka
{विद्युन्मालानिभैर्दिव्यैर्नानाभरणराजिभिः}
{क्वचित्संदृश्यमानस्तु चतुर्बाहुः क्वचिद्वरिः}


\twolineshloka
{क्वचिज्ज्योतिर्मयोचिन्त्यः क्वचित्स्कन्धे समाहितः}
{एवं मम जयन्देवस्तत्रैवान्तरधीयत}


\twolineshloka
{ततोऽहं विस्मयापन्नः कृत्वा कार्यमनुत्तमम्}
{अस्याविमुच्य जननीं मया सह मुनीश्वराः}


\twolineshloka
{अचिन्त्योऽयमहं भूयः कोऽसौ मामब्रवीत्पुरा}
{कीदृग्विधः स भगवानिति मत्वा तमास्थितः}


\twolineshloka
{अनन्तरं देवदेवं स्कन्धे मम समाश्रितम्}
{अद्राक्षं पुण्डरीकाक्षं वहमानोऽहमद्भुतम्}


\threelineshloka
{अवशस्तस्य भावेन यत्र यत्र स चेच्छति}
{विस्मयापन्नहृदयो ह्यहं किमिति चिन्तयन्}
{अन्तर्जलमहं सर्वं वहमानोऽगमं पुनः}


\twolineshloka
{सेन्द्रैर्देवैर्महाभागैर्ब्रह्माद्यैः कल्पजीविभिः}
{स्तूयमानो ह्यहमपि तैस्तैरभ्यर्चितः पृथक्}


\twolineshloka
{क्षीरोदस्योत्तरे कूले दिव्ये मणिमये शुभे}
{वैकर्णनाम सदनं हरेस्तस्य महात्मनः}


\twolineshloka
{दिव्यं तेजोमयं श्रीमदचिन्त्यममरैरपि}
{तेजोनिलमयैः स्तम्भैर्नानासंस्थानसंस्थितैः}


\twolineshloka
{विभूषितं हिरण्येन भास्वरेण समन्ततः}
{दिव्यं ज्योतिः समायुक्तं गीतवादित्रशोभितम्}


\threelineshloka
{शृणोमि शब्दं तत्राहं न पश्यामि शरीरिणम्}
{न च स्थलं न चान्यच्च पादयोस्तं समन्ततः}
{वेपमानो ह्यहं तत्र विष्ठितोऽहं कृताञ्जलिः}


\twolineshloka
{ततो ब्रह्मादयो देवा लोकपालास्तथैव च}
{सनन्दनाद्या मुनयस्तथाऽन्ये परजीविनः}


\twolineshloka
{प्राप्तास्तत्र सभाद्वारि देवगन्धर्वसत्तमाः}
{ब्रह्माणं परतः कृत्वा कृताञ्जलिपुटास्तदा}


\twolineshloka
{ततस्तदन्तरे तस्मिन्क्षीरोदार्णवशीकरैः}
{बोध्यमानो महाविष्णुराविर्भूत इवाबभौ}


\twolineshloka
{फणासहस्रमालाढ्यं शेषमव्यक्तसंस्थितम्}
{पश्याम्यहं मुदाऽऽकाशे यस्योपरि जनार्दनम्}


\twolineshloka
{दीर्घवृत्तैः समैः पीनैः केयूरवलयोज्ज्वलैः}
{चर्तुभिर्बाहुभिर्युक्तं------------}


\twolineshloka
{पिताम्बरेण संवीतं कौस्तुभेन विराजितम्}
{वक्षस्थलेन संयुक्तं पद्मयाऽधिष्ठितेन च}


\twolineshloka
{ईषदुन्मीलिताक्षं तं सर्वकारणकारणम्}
{क्षीरोदस्योपरि बभौ नीलाभ्रं परमं यथा}


\threelineshloka
{न कश्चिद्वदते कश्चिन्न व्याहरति कश्चन}
{ब्रह्मादिस्तम्बपर्यन्तं माशब्दमिति रोषितम्}
{भ्रुकुटीकुटिलाक्षास्ते नानाभूतगणाः स्थिताः}


\twolineshloka
{कृत्वा च प्रस्थितं तत्र जगतां हितकाम्यया}
{गच्छध्वमिति मामुक्त्वा गरुडेत्याह मां ततः}


\twolineshloka
{ततोऽहं प्रणिपत्याग्रे कृताञ्जलिरवस्थितः}
{आगच्छेति च मामुक्त्वा पूर्वोत्तरपथं गतः}


\threelineshloka
{अतीव मृदुभावेन गच्छन्निव स दृश्यते}
{अयुतं नियुतं चाहं प्रयुतं चार्बुदं तथा}
{पतमानोऽहमनिशं योजनानि ततस्ततः}


\twolineshloka
{ननु तत्वमहं भक्तो विष्ठितोस्मि प्रशास्तु नः}
{आगच्छ गरुडेत्येवं पुनराह स माधवः}


\twolineshloka
{ततो भूयो ह्यहं पातं पतमानो विहायसम्}
{आजगाम ततो घोरं शतकोटिसमावृतम्}


\twolineshloka
{तामसानीव भूतानि पर्वताभानि तत्र ह}
{समानानीव पद्मानि ततोऽहं भीत आस्थितः}


\twolineshloka
{ततो मां किंकरो घोरः शतयोजनमायतम्}
{निगृह्य पाणिना तस्माच्चिक्षेप च स लोष्टवत्}


\threelineshloka
{तत्तमोऽहमतिक्रम्य ह्यापं चैव विहायसम्}
{हुङ्कारघोपं तत्राहमशनीपातसन्निभान्}
{कर्णमूले ह्यशृण्वन्तस्ततो भूतैः समास्थितः}


\twolineshloka
{ततोऽहं देवदेवेश त्राहि मां पुष्करेक्षण}
{इत्यब्रवमहं तत्र ततो विष्णुरुवाच माम्}


\twolineshloka
{सुषिरस्य मुखे कश्चिन्मां चिक्षेप भयङ्करः}
{अतीतोऽहं क्षणादग्निमपश्यं वायुमण्डलम्}


\twolineshloka
{आकाशमिव संप्रेक्ष्य क्षेप्तुकाममुपागतः}
{तत्राहं दुःखितो भूतः क्रोशमानो ह्यवस्थितः}


\twolineshloka
{क्षणान्तरेण घोरेण क्रुद्धो हि परमात्मना}
{स्वपक्षराजिना दृष्ट्वा मां चिक्षेप भयङ्करः}


\twolineshloka
{-----गरुडकुलं सहस्रादित्यसन्निभम्}
{मां दृष्ट्वाऽप्यथ संस्थेऽथ ह्यल्पकालोऽतिदुर्बलः}


\twolineshloka
{अहो विहङ्गमः प्राप्त इति विस्मयमानसाः}
{मां दृष्ट्वोचुरहं तत्र पश्यामि गरुडध्वजम्}


\twolineshloka
{सहस्रयोजनायामं सहस्रादित्यवर्चसम्}
{सहस्रगरुडारूढं गरुडास्ते महाबलाः}


\twolineshloka
{अत्याश्चर्यमिमं देव वपुषाऽस्मत्कुलोद्भवः}
{स्वल्पप्राणः स्वल्पकायः कोसौ पक्षी इहागतः}


\twolineshloka
{तच्छ्रुत्वाऽहं नष्टगर्वो भीतो लज्जासमन्वितः}
{स्वयं बुद्ध्श्च संविग्नस्ततो ह्यशृणवं पुनः}


\threelineshloka
{आगच्छ गरुडेत्येव ततोऽहं यानमास्थितः}
{परार्ध्यं च ततो गत्वा योजनानां शतं पुनः}
{तत्रापश्यमहं यो वै ब्रह्माणं परमेष्ठिनम्}


\twolineshloka
{तत्रापि चापरं तत्र शतकोटिपितामहान्}
{पुनरेहीत्युवाचोच्चैर्भगवान्मधुसूदनः}


\twolineshloka
{महाकुलं ततोऽपश्यं प्रमाणानि तमव्ययम्}
{कपित्थफलसंकाशमन्धकारैः समाश्रितम्}


\twolineshloka
{तत्र स्थितो हरिः श्रीमानण्डमेकं बिभेद ह}
{महद्भूतं हि मां गृह्य दत्त्वा वै प्राक्षिपत्पुनः}


\twolineshloka
{तन्सध्ये सागरान्सप्त ब्रह्माणं च तथा सुरान्}
{पश्याम्यहं यथायोगं मातरं स्वकुलं तथा}


\twolineshloka
{एवं मयाऽनुभूतं हि तत्वान्वेषणकाङ्क्षिणा}
{शिबिकासदृशं मां वै पश्यध्वं मुनिसत्तमाः}


\twolineshloka
{इत्येवमब्रवं विप्रान्भीष्म यन्मे पुराऽभवत्}
{तत्ते सर्वं यथान्यायमुक्तवानस्मि सत्तम}


\twolineshloka
{योगिनस्तं प्रपश्यन्ति ज्ञानं दृष्ट्वा परं हरिम्}
{नान्यथा शक्यरूपोसौ ज्ञानगम्यः परः पुमान्}


\twolineshloka
{अनन्यया च भक्त्या च प्राप्तुं शक्यो महाहरिः ॥भीष्म उवाच}
{}


\twolineshloka
{इत्येवमुक्त्वा भगवान्सुपर्णः पक्षिराट् प्रभुः}
{आमन्त्र्य जननीं मे वै तत्रैवान्तरधीयत}


\twolineshloka
{तस्माद्राजेन्द्र सर्वात्मा वासुदेवः प्रधानकृत्}
{ज्ञानेन भक्त्या सुलभो नान्यथेति मतिर्मम ॥'}


\chapter{अध्यायः २१२}
\twolineshloka
{युधिष्ठिर उवाच}
{}


\twolineshloka
{योगं मे परमं तात मोक्षस्य वदभारत}
{तमहं तत्त्वतो ज्ञातुमिच्छामि वदतांवर}


\threelineshloka
{`भूयोपि ज्ञानसद्भावे स्थित्यर्थं त्वां ब्रवीम्यहम्}
{अचिन्त्यं वासुदेवाख्यं तस्मात्प्रब्रूहि सत्तम ॥'भीष्म उवाच}
{}


\twolineshloka
{अत्राप्युदाह न्तीममितिहासं पुरातनम्}
{संवादं मोक्षसंयुक्तं शिष्यस्य गुरुणा सह}


\twolineshloka
{कश्चिद्ब्राह्मणमासीनमाचार्यमृषिसत्तमम्}
{तेजोराशिं महात्मानं सत्यसन्धं जितेन्द्रियम्}


\twolineshloka
{शिष्यः परममेधावी श्रेयोर्थी सुसमाहितः}
{चरणावुपसंगृह्य स्थितः प्राञ्जलिरब्रवीत्}


\threelineshloka
{उपासनात्प्रसन्नोऽसि यदि वै भगवन्मम}
{संशयो मे महान्कश्चित्तं मे व्याख्यातुमर्हसि}
{कुतश्चाहं कुतश्च त्वं तत्सम्यग्ब्रूहि यत्परम्}


\twolineshloka
{कथं च सर्वभूतेषु समेषु द्विजसत्तम}
{सम्यग्वृत्ता निवर्तन्ते विपरीताः क्षयोदयाः}


\threelineshloka
{वेदेषु चापि यद्वाक्यं लौकिकं व्यापकं च यत्}
{एतद्विद्वन्यथातत्त्वं सर्वं व्याख्यातुमर्हसि ॥गुरुरुवाच}
{}


\twolineshloka
{शृणु शिष्य महाप्राज्ञ ब्रह्मगुह्यमिदं परम्}
{अध्यात्मं सर्वभूतानामागमानां च यद्वसु}


\twolineshloka
{वासुदेवः सर्वमिदं विश्वस्य ब्रह्मणो सुखम्}
{सत्यं दानं तपो यज्ञस्तितिक्षा दम आर्जवम्}


\twolineshloka
{पुरुषं सनातनं विष्णुं यं तं वेदविदो विदुः}
{सर्गप्रलयकर्तारमव्यक्तं ब्रह्म शाश्वतम्}


\twolineshloka
{तदिदं ब्रह्म वार्ष्णोयमितिहासं शृणुष्व मे}
{ब्राह्मणो ब्राह्मणैः श्राव्यो राजन्यः क्षत्रियैस्तता}


\twolineshloka
{[वैश्यो वैश्यैस्तथा श्राव्यः शूद्रः शूद्रैर्महामनाः}
{]माहात्म्यं देवदेवस्य विष्णोरमिततेजसः}


% Check verse!
अर्हस्त्वमसि कल्याणं वार्ष्णेयाध्यात्ममुत्तमम्
\twolineshloka
{`यमच्युतं परं नित्यं लिङ्गहीनं च निर्मलम्}
{निर्वाणममृतं श्रीमत्तद्विष्णोः परमं पदम्}


\twolineshloka
{भवे च भेदवद्भिन्नं प्रदानं गुणकारकम्}
{तस्मिन्न सज्यते नित्यं स एष पुरुषोऽपरः}


\twolineshloka
{पुरुषाधिष्ठितं नित्यं प्रधानं ब्रह्म कारणम्}
{कालस्वरूपं रूपेण विष्णुना प्रभविष्णुना}


\threelineshloka
{क्षोभ्यमाणं सृजत्येव नानाभूतानि भागशः}
{तद्दृष्ट्वा पुरुषोतत्वं साक्षीभूत्वा प्रवर्तते}
{तत्प्रविश्य यथायोगमभिन्नो भिन्नलक्षणः ॥'}


\twolineshloka
{कालचक्रमनाद्यन्तं भावाभावस्वलक्षणम्}
{त्रैलोक्ये सर्वभूतेषु चक्रवत्परिवर्तते}


\twolineshloka
{यत्तदक्षरमव्यक्तममृतं ब्रह्म शाश्वतम्}
{वदन्ति पुरुषव्याघ्र केशवं पुरुषर्षभम्}


\twolineshloka
{`तदक्षरमचिन्त्यं वै भिन्नरूपेण दृश्यते}
{पश्य कालाख्यमनिशं न चोष्णं नातिशीतलम्}


\twolineshloka
{न सन्त्येते गुणास्तस्मिंतथा तस्मात्प्रवर्तते}
{शीतलोऽयमनुप्राप्तः कालो ग्रीष्मस्तथैव च}


\twolineshloka
{वक्ष्यन्ति सर्वभूतानि ह्येते सूर्योदयं प्रति}
{आगच्छन्ति निवर्तन्ति स कालो गुणराशयः}


\threelineshloka
{न चैव प्रकृतिस्थेन कालयुक्तेन नित्यशः}
{गुणैः संभोगमरतिस्तत्वविज्ञानकोविदम्}
{पुरुषाधिष्ठिता नित्यं प्रकृतिः सूयते परा ॥'}


\twolineshloka
{पितॄन्देवानृषींश्चैव तथा वै यक्षराक्षसान्}
{नागासुरमनुष्यांश्च सृजते मनसाऽव्ययः}


\twolineshloka
{तथैव वेदशास्त्राणि लोकधर्मांश्च शाश्वतान्}
{प्रलये प्रकृतिं यातान्युगादौ सृजते पुनः}


\twolineshloka
{यथर्तुष्वृतुलिङ्गानि नानारूपाणि पर्यये}
{दृश्यन्ते तानि तान्येव तथा भावा युगादिषु}


\twolineshloka
{अथ यद्यद्यदा भावि कालयोगाद्युगादिषु}
{तत्तदुत्पद्यते ज्ञानं लोकयात्राविधानजम्}


\twolineshloka
{`श्रुतिरेषा समाख्याता तदर्थं कारणात्मना}
{अनाम्नायविधानाद्वै वेदा ह्यन्तर्हिता यथा ॥'}


\threelineshloka
{युगान्ते ह्यस्तभूतानि शास्त्राणि विविधानि च}
{सर्वसत्वविना द्वै जीवात्मनित्यया स्मृताः}
{अन्यस्मिन्नण्डसद्भावे वर्तमानानि नित्यशः}


\twolineshloka
{युगान्तेऽन्तर्हितान्वेदान्सेतिहासान्महर्षयः}
{लेभिरे तपसा पूर्वमनुज्ञाताः स्वयंभुवा}


\threelineshloka
{`नियोगाद्ब्रह्मणो विप्रा लोकतन्त्रप्रवर्तकाः}
{'वेदविद्भगवान्ब्रह्मा वेदाङ्गानि बृहस्पतिः}
{भार्गवो नीतिशास्त्रं तु जगाद जगतो हितम्}


\twolineshloka
{गान्धर्वं नारदो वेद भरद्वाजो धनुर्ग्रहम्}
{देवर्षिचरितं गर्गो कृष्णात्रेयश्चिकित्सितम्}


\threelineshloka
{`न्यायतन्त्रं हि कार्त्स्न्येन गौतमो वेद तत्वतः}
{वेदान्तकर्मायोगं च वेदविद्ब्रह्मविद्विभुः}
{द्वैपायनो निजग्राह शिल्पशास्त्रं भृगुः पुनः}


\twolineshloka
{न्यायतन्त्राण्यनेकानि तस्तैरुक्तानि वादिभिः}
{हेत्वागमसदाचारैर्यदुक्तं तदुपास्यते}


\twolineshloka
{अनाद्यं तत्परं ब्रह्म न देवा नर्षयो विदुः}
{एकस्तद्वेद भगवान्धाता नारायणः प्रभुः}


\threelineshloka
{नारायणादृषिगणास्तथा मुख्याः सुरासुराः}
{राजर्षयः पुराणाश्च परमं दुःखभेषडम्}
{`वक्ष्येऽहं तव यत्प्राप्तमृषेद्वैपोयनान्मया ॥'}


\twolineshloka
{पुरुषाधिष्ठितान्भावान्प्रकृतिः सूयते यदा}
{हेतुयुक्तमतः पूर्वं जगत्संपरिवर्तते}


\twolineshloka
{दीपादन्ये यथा दीपाः प्रवर्तन्ते सहस्रशः}
{प्रकृतिः सूयते सद्वदानन्त्यान्नापचीयते}


\twolineshloka
{अव्यक्तकर्मजा बुद्धिरहंकारं प्रसूयते}
{आकाशं चाप्यहंकाराद्वायुराकाशसंभवः}


\twolineshloka
{वायोस्तेजस्ततश्चाप अद्भ्योऽथ वसुधोद्गता}
{मूलप्रकृतयो ह्यष्टौ जगदेतास्ववस्थितम्}


\twolineshloka
{ज्ञानेन्द्रियाण्यतः पञ्च पञ्च कर्मेन्द्रियाण्यपि}
{विषयाः पञ्च चैकं च विकाराः षोडशं मनः}


\twolineshloka
{श्रोत्रं त्वक्चक्षुषी जिह्वा घ्राणं ज्ञानेन्द्रियाण्यश्च}
{पादौ पायुरुपस्थश्च हस्तौ वाक्कर्मणी अपि}


\twolineshloka
{शब्दः स्पर्शश्च रूपं च रसो गन्धस्तथैव च}
{विज्ञेयं व्यापकं चित्तं तेषु सर्वगतं मनः}


\twolineshloka
{`बुद्धीन्द्रियार्था इत्युक्ता दशसंसर्गयोनयः}
{सदसद्भावयोगे च मन इत्यभिधीयते}


\twolineshloka
{व्यवसायगुणा बुद्धिरहंकारोऽभिमानकः}
{न बीजं देहयोगे च कर्मबीजप्रवर्तनात् ॥'}


\twolineshloka
{रसज्ञाने तु जिह्वेयं व्याहृते वाक्यथैव च}
{इन्द्रियैर्विविधैर्युक्तं सर्वैर्व्यतं मनस्तथा}


\twolineshloka
{विद्यात्तु षोडशैतानि दैवतानि विभागशः}
{देहेषु ज्ञानकर्तारमुपासीनमुपासते}


\threelineshloka
{तत्र सोमगुणा जिह्वा गन्धस्तु पृथिवीगुणः}
{श्रोत्रे शब्दगुणे चैव चक्षुरग्नेर्गुणस्तथा}
{स्पर्शं वायुगुणं विद्यात्सर्वभूतेषु सर्वदा}


\twolineshloka
{मनः सत्वगुणं प्राहु सत्वमव्यक्तजं तथा}
{सर्वभूतात्मभूतस्थं तस्माद्बुद्ध्येत बुद्धिमान्}


\twolineshloka
{एते भावा जगत्सर्वं बहन्ति सचराचरम्}
{श्रिता विरजसं देवं यमाहुः परमं पदम्}


\twolineshloka
{नवद्वारं पुरं पुण्यमेतैर्भावैः स्मन्वितम्}
{व्याप्य शेते महानात्मा तस्मात्पुरुष उच्यते}


\twolineshloka
{अजरश्चामरश्चैव व्यक्ताव्यक्तोपदेशवान्}
{व्यापकः सगुणः सूक्ष्मः सर्वभूतगुणाश्रयः}


\twolineshloka
{यथा दीपः प्रकाशात्मा ह्रस्वो वा यदि वा महान्}
{ज्ञानात्मानं तथा विद्यात्पुरुषं सर्वजन्तुषु}


\twolineshloka
{श्रोत्रं वेदयते वेद्यं स शृणोति स पश्यति}
{कारणं तस्य देहोऽयं स कर्ता सर्वकर्मणाम्}


\twolineshloka
{अग्निर्दारुगतो यद्वद्भिन्ने दारौ न दृश्यते}
{तथैवात्मा शरीरस्थ ऋते योगान्न दृश्यते}


\twolineshloka
{अग्निर्यथा ह्युपायेन मथित्वा दारु दृश्यते}
{तथैवात्मा शरीरस्थो योगेनैवात्र दृश्यते}


\twolineshloka
{नदीष्वापो यथा युक्ता यथा सूर्ये मरीचयः}
{संतन्वाना यथा यान्ति तथा देहाः शरीरिणाम्}


\twolineshloka
{स्वप्नयोगे यथैवात्मा पञ्चेन्द्रियसमायुतः}
{देहमुत्सृज्य वै याति तथैवात्मोपलभ्यते}


\twolineshloka
{कर्मणा व्याप्यते सर्वं कर्मणैवोपपद्यते}
{कर्मणा नीयतेऽन्यत्र स्वकृतेन बलीयसा}


\twolineshloka
{स तु देहाद्यथा देहं त्यक्त्वाऽन्यं प्रतिपद्यते}
{तथा तं संप्रवक्ष्यामि भूतग्रामं स्वकर्मजम्}


\chapter{अध्यायः २१३}
\twolineshloka
{गुरुरुवाच}
{}


\threelineshloka
{चतुर्विधानि भूतानि स्थावराणि चराणि च}
{अव्यक्तप्रभवान्याहुरव्यक्तनिधनानि च}
{अव्यक्तलक्षणं विद्यादव्यक्तात्मात्मकं मनः}


\twolineshloka
{यथाऽश्वत्थकणीकायामन्तर्भूतो महाद्रुमः}
{निष्पन्नो दृश्यते व्यक्तमव्यक्तात्संभवस्तथा}


\twolineshloka
{`आत्मानमनुसंयाति बुद्धिरव्यक्तजा तथा}
{तामन्वेति मनो यद्वल्लोहवर्मणि सन्निधौ ॥'}


\twolineshloka
{अभिद्रवत्ययस्कान्तमयोनिश्चेतनं यथा}
{स्वभावहेतुजा भावा यद्वदन्यदपीदृशम्}


\twolineshloka
{तद्वदव्यक्तजा भावाः कर्तुः कारणलक्षणाः}
{अचेतनाश्चेतयितुः कारणादभिसंगताः}


\twolineshloka
{न भूर्न खं द्यौर्भूतानि नर्षयो न सुरासुराः}
{नान्यदासीदृते जीवमासेदुर्न तु संहतिम्}


\twolineshloka
{सर्वं नित्यं सर्वगतं मनोहेतुत्वलक्षणम्}
{अज्ञानकर्म निर्दिष्टमेतत्कारणलक्षणम्}


\twolineshloka
{तत्कारणेन संयुक्तं कार्यसंग्रहकारकम्}
{येनैतद्वर्तते चक्रमनादिनिधनं महत्}


\twolineshloka
{`येन स्वभावसद्भावं हेतुभूता सकारणा}
{एवं प्राकृतविस्तारो ह्याश्रित्य पुरुषं परम् ॥'}


\twolineshloka
{अव्यक्तनाभं व्यक्तारं विकारपरिमण्डलम्}
{क्षेत्रज्ञाधिष्ठितं चक्रं स्निग्धाक्षं वर्तते ध्रुवम्}


\twolineshloka
{स्निग्धत्वात्तिलवत्सर्वं चक्रेऽस्मिन्पीड्यते जगत्}
{तिलपीडैरिवाक्रम्य भोगैरज्ञानसंभवैः}


\twolineshloka
{`प्राणेनायं हि शान्ते तु विरोधात्प्रतिपालनम्}
{देहस्येषून्य आस्ते यः शुद्धोऽचिन्त्यः सनातनः}


\twolineshloka
{भ्रामयन्नेषतो याति कालचक्रसमन्वितः}
{भूतानि मोहयन्नित्यं चक्रस्य च रयं गतः}


\threelineshloka
{स्नेहद्रव्यसमायोगे क्षेत्रपाचं न वस्तुषु}
{तिलवत्पीडिते चक्रे ह्याधियन्त्रनिपीडिते}
{बहिश्चाधिष्ठिते यद्वज्ज्ञानिनां कर्मसंभवम्'}


\twolineshloka
{कर्म तत्कुरुते तर्षादहंकारपरिग्रहम्}
{कार्यकारणसंयोगे स हेतुरुपपादितः}


\threelineshloka
{`यथाऽऽकर्ण्य च तच्छिष्यस्तत्वज्ञानमनुत्तमम्}
{'नात्येति कारणं कार्यं न कार्यं कारणं तथा}
{कार्याण्यमूनि करणे कालो भवति हेतुमान्}


\twolineshloka
{हेतुयुक्ताः प्रकृतयो विकाराश्च परस्परम्}
{अन्योन्यमभिवर्तन्ते पुरुषाधिष्ठिताः सदा}


\twolineshloka
{सत्वरजस्तामसैर्भावैश्च्युतो हेतुबलान्वितः}
{क्षेत्रज्ञमेवानयाति पांसुर्वातेरितो यथा}


\twolineshloka
{न च तैः स्पृश्यते भावैर्न ते तेन महात्मना}
{सरजस्कोऽरजस्कश्च स वै वायुर्भवेद्यथा}


\twolineshloka
{तथैतदन्तरं विद्यात्सत्वक्षेत्रज्ञयोर्बुधः}
{अभ्यासात्स तथा युक्तो न गच्छेत्प्रकृतिं पुनः}


\twolineshloka
{संदेहमेतमुत्पन्नमच्छिनद्भगवानृषिः}
{तथा वार्तां समीक्षेत कृतलक्षणसंविदम्}


\twolineshloka
{बीजान्यग्न्युपदग्धानि नरो हन्ति यथा पुनः}
{ज्ञानदग्धैस्तथा क्लेशैर्नात्मा संपद्यते पुनः}


\chapter{अध्यायः २१४}
\twolineshloka
{गुरुरुवाच}
{}


\twolineshloka
{प्रवृत्तिलक्षणो धर्मो यथा समुपलभ्यते}
{तेषां विज्ञाननिष्ठानामन्यत्तत्वं न रोचते}


\twolineshloka
{दुर्लभा वेदविद्वांसो वेदोक्तेषु व्यवस्थिताः}
{प्रयोजनं महत्त्वात्तु मार्गमिच्छन्ति संस्तुतम्}


\threelineshloka
{`वेदस्य न विदुर्भावं ज्ञानमार्गप्रतिष्ठितम्}
{'सद्भिराचरितत्वात्तु वृत्तमेतदगर्हितम्}
{इयं सा बुद्धिरभ्येत्य यथा याति परां गतिम्}


\twolineshloka
{शरीरवानुपादत्ते मोहात्सर्वान्परिग्रहान्}
{कामक्रोधादिभिर्भावैर्युक्तो राजसतामसैः}


\twolineshloka
{नाशुद्धमाचरेत्तस्मादभीप्सन्देहयापनम्}
{कर्मणां विवरं कुर्वन्न लोकानाप्नुयाच्छुभान्}


\twolineshloka
{लोहयुक्तं तथा हेम विपक्वं न विराजते}
{तथाऽपक्वकषायाख्यं विज्ञानं न प्रकाशते}


\twolineshloka
{`केचिदात्मगुणं प्राप्तास्ते मुक्ताश्चक्रबन्धनात्}
{इतरे दुःखसन्द्वन्द्वास्तथा दुःखपरायणाः}


\twolineshloka
{शुकाकर्मानुरूपं ते जायमानाः पुनः पुनः}
{क्रोधलोभमदाविष्टा मूढान्तः करणाः सदा}


\twolineshloka
{यथा--- संछाया नास्ति नित्यतया परा}
{गुणानेव तथा चिन्त्या सन्त्येति च विदुर्बुधाः ॥'}


\twolineshloka
{यश्चाधर्मं चरेल्लोभात्कामक्रोधावनुप्लुवन्}
{धर्म्यं पन्थानमुत्क्रम्य सानुबन्धो विनश्यति}


\threelineshloka
{`अचलं ज्ञानमप्राप्य चलचित्तश्चलानियात्}
{'शब्दादीन्विषयांस्तस्मान्न संरागादुपप्लवेत्}
{क्रोधो हर्षो विषादश्च जायन्ते हि परस्परात्}


\twolineshloka
{`गुणाः कार्याः क्रोधहर्षौ सुखदुःखे प्रियाप्रिये}
{द्वन्द्वान्यथैवमादीनि विजयेच्चैव सर्ववित् ॥'}


\twolineshloka
{पञ्चभूतात्मके देहे सत्त्वराजसताभसे}
{कमभिष्टुवते चायं कं वा क्रोशति किं वदन्}


\twolineshloka
{स्पर्शरूपरसाद्येषु सङ्गं गच्छन्ति बालिशाः}
{नावगच्छन्ति विज्ञानादात्मानं पार्थिवं गुणम्}


\twolineshloka
{मृन्मयं शरणं यद्वन्मृदैव परिलिप्यते}
{पार्थिवोऽयं तथा देहो मृद्विकारान्न नश्यति}


\twolineshloka
{मधु तैलं पयः सर्पिर्मांसानि लवणं गुडः}
{धान्यानि फलमूलानि मृद्विकाराः सहाम्भसा}


\twolineshloka
{यद्वत्कान्तारमातिष्ठन्नौत्सुक्यं समनुव्रजेत्}
{ग्राम्यमाहारमादद्यादस्वाद्वपि हि यापनम्}


\twolineshloka
{तद्वत्संसारकान्तारमातिष्ठञ्श्रमतत्परः}
{यात्रार्थमद्यादाहारं व्याधितो भेषजं यथा}


\twolineshloka
{`भक्षणे श्वापदैर्मार्गादिति चारं करोति चेत्}
{एवं संसारमार्गेण यात्रार्थं विषयाणि च}


\twolineshloka
{न गच्छेद्भोगविज्ञानादुन्मार्गे पद्यते तदा}
{तस्माददुःखतो मार्गमास्थितस्तमनुस्मरेत्}


\twolineshloka
{नानापर्णफला वृक्षा बहवः सन्ति तत्र हि}
{भोक्तारो मुनयश्चैव तस्मात्परतरं वनम्}


\threelineshloka
{अनुमानैस्तथाशास्त्रैर्यशसा विक्रमेण च}
{'सत्यशौचार्जवत्यागैर्वर्चसा विक्रमेण च}
{क्षान्त्या धृत्या च बुद्ध्या च मनसा तपसैव च}


\twolineshloka
{भावान्सर्वान्यथावृत्तान्संवसेत यथाक्रमम्}
{शान्तिमिच्छन्नदीनात्मा संयच्छेदिन्द्रियाणि च}


\twolineshloka
{सत्त्वेन रजसा चैव तमसा चैव मोहिताः}
{चक्रवत्परिवर्तन्ते ह्यज्ञानाज्जन्तवो भृशम्}


\twolineshloka
{तस्मात्सम्यक्परीक्षेत दोषानज्ञानसंभवान्}
{अज्ञानप्रभवं नित्यमहंकारं परित्यजेत्}


\twolineshloka
{महाभूतानीन्द्रियाणि गुणाः सत्त्वं रजस्तमः}
{`देहमूलं विजानीहि नैतानि भगवानतः}


\twolineshloka
{उपायतः प्रवक्ष्यामि तं च मृत्युं दुरासदम्}
{त्रैलोक्यं सेश्वरं सर्वमहंकारे प्रतिष्ठितम्}


\twolineshloka
{यथेह नियतः कालो दर्शयत्यार्तवान्गुणान्}
{तद्वद्भतेष्वहंकारं विद्याद्भूतप्रवर्तकम्}


\twolineshloka
{संमोहकं तमो विद्यात्कृष्णमज्ञानसंभवम्}
{प्रकृतेर्गुणसंजातो महानहंक्रिया ततः}


\twolineshloka
{अहंकारात्पुनः पश्चाद्भूतग्राममुदाहृतम्}
{अव्यक्तस्य गुणेभ्यस्तु तद्गुणांश्च निबोध तान्}


\twolineshloka
{प्रीतिदुःखनिबद्धांश्च समस्तांस्त्रीनथो गुणान्}
{सत्त्वस्य रजसश्चैव तमसश्च निबोध तान्}


\twolineshloka
{प्रसादो हर्षजा प्रीतिरसंदेहो धृतिः स्मृतिः}
{एतान्सत्त्वगुणान्विद्यादिमान्राजसतामसान्}


\twolineshloka
{`असन्तोषोऽक्षमाऽधैर्यमतृप्तिर्विषयादिषु}
{राजसाश्च गुणा ह्येते तत्परं तामसाञ्शृणु}


\twolineshloka
{मोहस्तन्द्री तथा दुःखं निद्राऽऽलस्यं प्रमादता}
{विषादी दीर्घसूत्रश्च तत्तामसमुदाहृतम् ॥'}


\twolineshloka
{कामक्रोधौ प्रमादश्च लोभमोहौ भयं क्लमः}
{विषादशोकावरतिर्मानदर्पावनार्यता}


\twolineshloka
{दोषाणामेवमादीनां परीक्ष्य गुरुलाघवम्}
{विमृशेदात्मसंस्थानमेकैकमनुसंततम्}


\twolineshloka
{`यस्मिन्प्रतिष्ठितं चेदं यस्मिन्सज्ज्ञाननिर्गतिः}
{सर्वभूताधिकं नित्यमहंकारं विलोकयेत्}


\threelineshloka
{विलोकमानः स तदा स्वबुद्ध्या सूक्ष्मया पुनः}
{तदेव भाति तद्रूपमात्मना यत्सुनिर्मलम् ॥'शिष्य उवाच}
{}


% Check verse!
के दोषा मनसा त्यक्ताः के बुद्ध्या शिथिलीकृताःके पुनः पुनरायान्ति के मोहादचला इव
\threelineshloka
{केषां बलाबलं बुद्ध्या हेतुभिर्विमृशेद्बुधः}
{`एतन्मे सर्वमाचक्ष्व यथा विद्यामहं विभो}
{मह्यं शुश्रूषवे विद्वन्वक्ष्येतद्बुद्धिनिश्चितम्}


\threelineshloka
{शान्तत्वादपरान्ताच्च आरम्भादपि चैकतः}
{प्रोक्तो ह्यत्र यथा हेतुरेवमाहुर्मनीषिणः ॥गुरुरुवाच}
{'}


\threelineshloka
{दोषैर्मूलादवच्छिन्नैर्विशुद्धात्मा विमुच्यते}
{विनाशयति संभूतमयस्मयमयो यथा}
{तथा कृतात्मा सहजैर्दोषैर्नश्यति राजसैः}


\twolineshloka
{`सहजैरविशुद्धात्मा दोषैर्नश्यति तामसैः}
{'राजसं तामसं चैव शुद्धात्मा कालसंभवम्}


\twolineshloka
{`शमयेत्सत्त्वमास्थाय बुद्ध्या केवलया द्विजः}
{त्यजेच्च मनसा चेतः शुद्धात्मा बुद्धिमास्थितः'}


\twolineshloka
{तत्सर्वं देहिनां बीजं सत्त्वमात्मवतः समम्}
{तस्मादात्मवता वर्ज्यं रजश्च तम एव च}


\twolineshloka
{रजस्तमोभ्यां निर्मुक्तं सत्त्वं निर्मलतामियात्}
{`आहारान्वर्जयेन्नित्यं राजसांस्तामसानपि}


\twolineshloka
{ते ब्रह्म पुनरायान्ति न मोहादचला इव}
{'अथवा मन्त्रवद्ब्रूयुर्मांसादीनां यजुष्कृतम्}


\twolineshloka
{स वै हेतुरनादाने शुद्धधर्मानुपालने}
{रजसा कामयुक्तानि कार्याण्यपि समाप्नुते}


\threelineshloka
{अर्थयुक्तानि चात्यर्थं कामान्सर्वांश्च सेवते}
{तमसा लोभयुक्तानि क्रोधजानि च सेवते}
{हिंसाविहाराभिरतस्तन्द्रीनिद्रासमन्वितः}


\twolineshloka
{सत्वस्थः सात्विकान्भावाञ्शुद्धान्पश्यति संश्रितः}
{स देही विमलः श्रीमाञ्श्रद्धाविद्यासमन्वितः}


\chapter{अध्यायः २१५}
\twolineshloka
{गुरुरुवाच}
{}


\twolineshloka
{रजसा साध्यते मोहस्तमश्च भरतर्षभ}
{क्रोधलोभौ भयं दर्प एतेषां सादनाच्छुचिः}


\twolineshloka
{परमं परमात्मानं देवमक्षयमव्ययम्}
{विष्णुमव्यक्तसंस्थानं विदुस्तं देवसत्तमम्}


\twolineshloka
{तस्य मायापिनद्धाङ्गा ज्ञान भ्रष्टा निराशिषः}
{मानवा ज्ञानसंमोहात्ततः कामं प्रयान्ति वै}


\twolineshloka
{कामात्क्रोधमवाप्याथ लोभमोहौ च मानवाः}
{मानदर्पावहंकारमहंकारात्ततः क्रियाः}


\twolineshloka
{क्रियाभिः स्नेहसंबन्धः स्नेहाच्छोकमनन्तरम्}
{अथ दुःखसमारम्भा जराजन्मकृतक्षणाः}


\twolineshloka
{जन्मतो गर्भवासं तु शुक्रशोणितसंभवम्}
{पुरीषमूत्रविक्लेदं शोणितप्रभवाविलम्}


\threelineshloka
{तृष्णाभिभूतस्तैर्वद्धस्तानेवाभिपरिप्लवन्}
{` तथा नरकगर्तस्थस्तृष्णारज्जुभिराचितः}
{पुण्यपापप्रणुन्नाङ्गो जायते स यथा कृमिः}


\twolineshloka
{मशकैर्मत्कुणैर्दष्टस्तथा चित्रवधार्दितः}
{नानाव्याधिभिराकीर्णः कथंचिद्यौवनं गतः}


\twolineshloka
{कूर्मोत्सृजति भूयश्च रज्जुः स्वस्वमुखेप्सया}
{योषितं नरकं गृह्य जन्मकर्मवशानुगः}


\twolineshloka
{पुरक्षेत्रनिमित्तं यद्दुःखं वक्तुं न शक्यते}
{कस्तत्र निन्दकश्चैव नरके पच्यते भृशम्}


\twolineshloka
{वार्धक्यमनुलङ्घेत तत्र कर्मारभेत्पुनः}
{भगवान्संस्तुतः पश्चात्किं प्रवक्ष्यामि ते भृशम्'}


\threelineshloka
{संसारतन्त्रवाहिन्यस्तत्र बुद्ध्येत योषितः}
{प्रकृत्याः क्षेत्रभूतास्ता नराः क्षेत्रज्ञलक्षणाः}
{तस्मादेवाविशेषेण नरोऽतीयाद्विशेषतः}


\twolineshloka
{कृत्या ह्येता घोररूपा मोहयन्त्यविचक्षणान्}
{रजस्यन्तर्हिता मूर्तिरिन्द्रियाणां सनातनी}


\threelineshloka
{तस्मात्तर्षात्मकाद्रागाद्बीजाज्जायन्ति जन्तवः}
{स्वदेहजानस्वसंज्ञान्यद्वदङ्गात्कृमींस्त्यजेत्}
{स्वसंज्ञानस्वकांस्तद्वत्सुतसंज्ञान्कृमींस्त्यजेत्}


\twolineshloka
{शुक्रतो रसतश्चैव देहाज्जायन्ति जन्तवः}
{स्वभावात्कर्मयोगाद्वा तानुपेक्षेत बुद्धिमान्}


\twolineshloka
{रजस्तमसि पर्यस्तं सत्वं च रजसि स्थितम्}
{ज्ञानाधिष्ठानमज्ञानं बुद्ध्यंहंकारलक्षणम्}


\twolineshloka
{तद्बीजं देहिनामाद्दुस्तद्बीजं जीवसंज्ञितम्}
{कर्मणा कालयुक्तेन संसारपरिवर्तनम्}


\twolineshloka
{रमत्ययं यथा स्वप्ने मनसा देहवानिव}
{कर्मगर्भैर्गुणैर्देही गर्भे तदुपलभ्यते}


\twolineshloka
{कर्मणा बीजभूतेन चोद्यते यद्यदिन्द्रियम्}
{जायते तदहंकाराद्रागयुक्तेन चेतसा}


\twolineshloka
{शब्दरागाच्छ्रोत्रमस्य जायते भावितात्मनः}
{रूपरागात्तथा चक्षुर्घ्राणं गन्धजिघृक्षया}


\twolineshloka
{संस्पर्शेभ्यस्तथा वायुः प्राणापानव्यपाश्रयः}
{व्यानोदानौ समानश्च पञ्चधा देहयापनम्}


\twolineshloka
{संजातैर्जायते गात्रैः कर्मजैर्ब्रह्मणा वृतः}
{दुःखाद्यन्तैर्दुःखमध्यैर्नरः शारीरमानसैः}


\twolineshloka
{दुःखं विद्यादुपादानादभिमानाच्च वर्धते}
{त्यागात्तेभ्यो निरोधः स्यान्निरोधज्ञो विमुच्यते}


\twolineshloka
{इन्द्रियाणां रजस्येव प्रलयप्रभवावुभौ}
{परीक्ष्य संचरेद्विद्वान्यथावच्छास्त्रचक्षुषा}


\twolineshloka
{ज्ञानेन्द्रियाणीन्द्रियार्थान्नोपसर्पन्त्यतर्षुलम्}
{ज्ञानैश्च करणैर्देही न देहं पुननर्हति}


\chapter{अध्यायः २१६}
\twolineshloka
{गुरुरुवाच}
{}


\twolineshloka
{अत्रोपायं प्रवक्ष्यामि यथावच्छास्त्रचक्षुषा}
{तत्त्वज्ञानाच्चरन्राजन्प्राप्नुयात्परमां गतिम्}


\twolineshloka
{सर्वेषामेव भूतानां पुरुषः श्रेष्ठ उच्यते}
{पुरुषेभ्यो द्विजानाहुर्द्विजेभ्यो मन्त्रदर्शिनः}


\twolineshloka
{सर्वभूतात्मभूतास्ते सर्वज्ञाः सर्वदर्शिनः}
{ब्राह्मणा वेदशास्त्रज्ञास्तत्त्वार्थगतनिश्चयाः}


\twolineshloka
{नेत्रहीनो यथा ह्येकः कृच्छ्राणि लभतेऽध्वनि}
{ज्ञानहीनस्तथा लोके तस्माज्ज्ञानविदोऽधिकाः}


\twolineshloka
{तांस्तानुपासते धर्मान्धर्मकामा यथागमम्}
{न त्वेषामर्थसामान्यमन्तरेण गुणानिमान्}


\twolineshloka
{वाग्देहमनसां शौचं क्षमा सत्यं धृतिः स्मृतिः}
{सर्वधर्मेषु धर्मज्ञा ज्ञापयन्ति गुणाञ्छुभान्}


\twolineshloka
{यदिदं ब्रह्मणो रूपं ब्रह्मंचर्यमिति स्मृतम्}
{परं तत्सर्वधर्मेभ्यस्तेन यान्ति परां गतिम्}


\twolineshloka
{लिङ्गसंयोगहीनं यच्छब्दस्पर्शविवर्जितम्}
{श्रोत्रेण श्रवणं चैव चक्षुषा चैव दर्शनम्}


\twolineshloka
{वाक्संभाषाप्रवृत्तं यत्तन्मनः परिवर्जितम्}
{बुद्ध्या चाध्यवसीयीत ब्रह्मचर्यमकल्मषम्}


\twolineshloka
{सम्यग्वृत्तिर्ब्रह्मलोकं प्राप्नुयान्मध्यमः सुरान्}
{द्विजाग्र्यो जायते विद्वान्कन्यसीं वृत्तिमास्थितः}


\twolineshloka
{सुदुष्करं ब्रह्मचर्यमुपायं तत्र मे शृणु}
{संप्रदीप्तमुदीर्णं च निगृह्णीयाद्द्विजो मनः}


\twolineshloka
{योषितां न कथा श्राव्या न निरीक्ष्या निरम्बराः}
{कथंचिद्दर्शनादासां दुर्बलानां विशेद्रजः}


\twolineshloka
{रागोत्पन्नश्चेरत्कृच्छ्रमह्नस्त्रिः प्रविशेदपः}
{मग्नस्त्वप्स्वेव मनसा त्रिर्जपेदघमर्षणम्}


\twolineshloka
{पाप्मानं निर्दहेदेवमन्तर्भूतरजोमयम्}
{ज्ञानयुक्तेन मनसा संततेन विचक्षणः}


\twolineshloka
{कुणपामेध्यसंयुक्तं यद्वदच्छिद्रबन्धनम्}
{तद्वद्देहगतं विद्यादात्मानं देहबन्धनम्}


\twolineshloka
{`अमेध्यपूर्णं यद्भाण्डं श्लेष्मान्तकलिलावृतम्}
{नेच्छते वीक्षितुं भाण्डं कुतः स्प्रष्टुं प्रवर्तते}


\twolineshloka
{देहभाण्डं मलैः पूर्णं बहिः स्वेदजलावृतम्}
{बीभत्सं नरनारीणां ज्ञानिनां नरकं मतम्}


\twolineshloka
{छिद्रकुम्भो यथा स्रावं सृजते तद्गतं दृढम्}
{अन्तस्यं स्रंसते तद्वज्जलं देहेषु देहिनाम्}


\twolineshloka
{श्लेष्माश्रुमूत्रकलिलं पुरीषं शुक्लमेव च}
{कफजालविनिर्यासः सरसश्चित्त मुञ्चय ॥'}


\twolineshloka
{वातपित्तकफान्रक्तं त्वङ्भांसं स्नायुमस्थि च}
{मज्जां देहं सिराजालैस्तर्पयन्ति रसा नृणाम्}


\twolineshloka
{दश विद्याद्धमन्योऽत्र पञ्चेन्द्रियगुणावहाः}
{याभिः सूक्ष्माः प्रजायन्ते धमन्योऽन्याः सहस्रशः}


\twolineshloka
{एवमेताः सिरा नद्यो रसोदा देहसागरम्}
{तर्पयन्ति यथाकालमापगा इव सागरम्}


\twolineshloka
{मध्ये च हृदयस्यैका सिरा तत्र मनोवहा}
{शुक्रं संकल्पजं नॄणां सर्वगात्रैर्विमुञ्चति}


\twolineshloka
{सर्वगात्रप्रतायिन्यस्तस्या ह्यनुगताः सिराः}
{नेत्रयोः प्रतिपद्यन्ते वहन्त्यस्तैजसं गुणम्}


\twolineshloka
{पयस्यन्तर्हितं सर्पिर्यद्वन्निर्मथ्यते खजैः}
{शुक्रं निर्मथ्यते तद्वद्देहसंकल्पजैः खजैः}


\twolineshloka
{स्वप्नेऽप्येवं यथाऽभ्येति मनः संकल्पजं रजः}
{शुक्रमस्पर्शजं देहात्सृजन्त्यस्य मनोवहाः}


\twolineshloka
{महर्षिर्भगवानत्रिर्वेद तच्छ्रुक्रसंभवम्}
{नृबीजमिन्द्रदैवत्यं तस्मादिन्द्रियमुच्यते}


\twolineshloka
{ये वै शुक्रगतिं विद्युर्भूतसंकरकारिकाम्}
{विरागा दग्धदोषास्ते नाप्नुयुर्देहसंभवम्}


\twolineshloka
{गुणानां साम्यमागम्य मनसैव मनोवहम्}
{देहकर्म नुदन्प्राणानन्तकाले विमुच्यते}


\twolineshloka
{भविता मनसो ज्ञानं मन एव प्रजायते}
{ज्योतिष्मद्विरजो नित्यं मन्त्रसिद्धं महात्मनाम्}


\twolineshloka
{तस्मात्तदभिघाताय कर्म कुर्यादकल्मषम्}
{देहबीजं समुत्पन्नमस्मित्कर्मणि विद्यते}


\twolineshloka
{न स्मरेन्न प्रयुञ्जीत ज्ञानी तत्कर्म बुद्धिमान्}
{रजस्तमश्च हित्वेह न तिर्यग्गतिमाप्नुयात्}


\twolineshloka
{तरुणाधिगतं ज्ञानं जरादुर्बलतां गतम्}
{विपक्वबुद्धिः कालेन आदत्ते मानसं बलम्}


\twolineshloka
{`एव पुत्रकलत्रेषु ज्ञातिसंबन्धिबन्धुषु}
{आदत्ते हृदये कामं व्याध्यादिभिरभिप्लुतः}


\threelineshloka
{यतस्ततः परिपतन्नविन्दन्सुखमण्वपि}
{बहुदुःखसमापन्नः पश्चान्निर्वेदमास्थितः}
{ज्ञानवृक्षं समाश्रित्य पश्चान्निर्वृतिमश्नुते ॥'}


\twolineshloka
{सुदुर्गमिव पन्थानमतीत्य गुणबन्धनम्}
{यथा पश्येत्तथा दोषानतीत्यामृतमश्नुते}


\chapter{अध्यायः २१७}
\twolineshloka
{गुरुरुवाच}
{}


\twolineshloka
{दुरन्तेष्विन्द्रियार्थेषु सक्ताः सीदन्ति जन्तवः}
{ये त्वसक्ता महात्मानस्ते यान्ति परमां गतिम्}


\twolineshloka
{जन्ममृत्युजरादुःखैर्व्याधिभिर्मानसक्लमैः}
{दृष्ट्वैव संततं लोकं घटेन्मोक्षाय बुद्धिमान्}


\twolineshloka
{वाङ्भनोभ्यां शरीरेण शुचिः स्यादनहंकृतः}
{प्रशान्तो ज्ञानवान्भिक्षुर्निरपेक्षश्चरेत्सुखम्}


\twolineshloka
{`वशा मोक्षवतां पाशास्तासां रूपं प्रदर्शकम्}
{दुर्ग्रहं पश्यमानोऽपि मन्यते मोहितस्तदा}


\twolineshloka
{एवं पश्यन्तमात्मानमनुध्यातं हि बन्धुषु}
{अयथात्वेन जानामि भेदरूपेण संस्थितम् ॥'}


\twolineshloka
{अथवा मनसः सङ्गं पश्येद्भूतानुकम्पया}
{तत्राप्युपेक्षां कुर्वीत ज्ञात्वा कर्मफलं जगत्}


\twolineshloka
{यत्कृतं स्याच्छुभं कर्म पापं वा यदि वाऽश्नुते}
{तस्माच्छुभानि कर्माणि कुर्याद्वा बुद्धिकर्मभिः}


\twolineshloka
{अहिंसा सत्यवचनं सर्वभूतेषु चार्जवम्}
{क्षमा चैवाप्रमादश्च यस्यैते स सुखी भवेत्}


\twolineshloka
{`अनक्षसाध्यं तद्ब्रह्म निर्मलं जगतः परम्}
{स्वात्मप्रकाशमग्राह्यमहेतुकमचञ्चलम्}


\twolineshloka
{विवेकज्ञानवाचिस्थो ह्याशुरूपेण संस्थितः}
{वैकारिकात्प्रदृश्येतै गैरिके मधुधारवत् ॥'}


\twolineshloka
{यश्चैनं परमं धर्मं सर्वभूतसुखावहम्}
{दुःखान्निः सरणं वेद तत्त्वज्ञः स सुखी भवेत्}


\twolineshloka
{तस्मात्समाहितं बुद्ध्या मनो भूतेषु धारयेत्}
{नापथ्यायेन्न स्पृहयेन्नाबद्धं चिन्तयेदसत्}


\twolineshloka
{अथामोघप्रयत्नेन मनो ज्ञाने निवेशयेत्}
{सुवाचोऽथ प्रयोगेण मनोज्ञं संप्रवर्तते}


\twolineshloka
{विवेकयित्वा तद्वाक्यं धर्मसूक्ष्ममवेक्ष्य च}
{सत्यां वाचमहिंस्रां च वदेदनपवादिनीम्}


\twolineshloka
{कल्कापेतामपरुषामनृशंसामपैशुनीम्}
{ईदृगल्पं च वक्तव्यमविक्षिप्तेन चेतसा}


\twolineshloka
{वाक्यबन्धेन संरागविहाराद्व्याहरेद्यदि}
{बुद्ध्याऽप्यनुगृहीतेन मनसा कर्म तामसम्}


\threelineshloka
{रजोभूतैर्हि करणैः कर्मणि प्रतिपद्यते}
{स दुःखं प्राप्य लोकेऽस्मिन्नरकायोपपद्यते}
{तस्मान्मनोवाक्शरीरैराचरेद्वैर्यमात्मनः}


\threelineshloka
{प्रकीर्ण एव भारो हि यद्वद्धार्येत दस्युभिः}
{प्रतिलोमां दिशं बुद्ध्वा संसारमबुधास्तथा}
{`संसारमार्गमापन्नः प्रतिलोमं विवर्जयेत् ॥'}


\twolineshloka
{तामेव च यथा दस्यून्हत्वा गच्छेच्छिवां दिशम्}
{तथा रजस्तमः कर्माण्युत्सृज्य प्राप्नुयाच्छुभम्}


\twolineshloka
{निःसंदिग्धमनीहो वै मुक्तः सर्वपरिग्रहैः}
{विविक्तचारी लघ्वाशी तपस्वी नियतेन्द्रियः}


\twolineshloka
{ज्ञानदग्धपरिक्लेशः प्रयोगरतिरात्मवान्}
{निष्प्रचारेण मनसा परं तदधिगच्छति}


\threelineshloka
{धृतिमानात्मवान्बुद्धिं निगृह्णीयादसंशयम्}
{मनो बुद्ध्या निगृह्णीयाद्विषयान्मनसाऽऽत्मनः}
{`योजयित्वा मनस्तत्र निश्चलं परमात्मनि}


\twolineshloka
{योगाभिसन्धियुक्तस्य ब्रह्म तत्संप्रकाशते}
{ऐकान्त्यं तदिदं विद्धि सर्ववस्त्वन्तरस्थितिः}


\threelineshloka
{विशेषहीनं गृह्णन्ति विशेषां कारणात्मिकाम्}
{अथवा न प्रभुस्तत्र परमात्मनि वर्तितुम्}
{आगामित्तत्त्वं योगात्मा योगतन्त्रमुपक्रमेत् ॥'}


\twolineshloka
{निगृहीतेन्द्रियस्यास्य कुर्वाणस्य मनो वशे}
{देवतास्ताः प्रकाशन्ते हृष्टा यान्ति तमीश्वरम्}


\twolineshloka
{ताभिः संयुक्तमनसो ब्रह्म तत्संप्रकाशते}
{शनैश्चापगते सत्वे ब्रह्मभूयाय कल्पते}


\twolineshloka
{अथवा न प्रवर्तेत योगतन्त्रैरुपक्रमेत्}
{योगतन्त्रमयं तन्त्रं वृत्तिः स्यात्ततदाचरेत्}


\twolineshloka
{कणकुल्माषपिण्याकशाकयावकसक्तवः}
{तथा मूलफलं भैक्ष्यं पर्यायेणोपयोजयेत्}


\twolineshloka
{आहारनियमं चैव देशे काले च सात्विकः}
{तत्परीक्ष्यानुवर्तेत यत्प्रवृत्त्यनुवर्तकम्}


\twolineshloka
{प्रवृत्तं नोपरुन्धेत शनैरग्निमिवेन्धयेत्}
{ज्ञानैधितं तथा ज्ञानमर्कवत्संप्रकाशते}


\twolineshloka
{ज्ञानाधिष्ठानमज्ञानं त्रील्लोँकानधितिष्ठति}
{विज्ञानानुगतं ज्ञानमज्ञानेनापकृष्यते}


\twolineshloka
{पृथक्त्वात्संप्रयोगाच्च नासूयुर्वेद शाश्वतम्}
{स तयोरपवर्गज्ञो वीतरागो विमुच्यते}


\twolineshloka
{वयोतीतो जरामृत्यू जित्वा ब्रह्म सनातनम्}
{अमृतं तदवाप्नोति यत्तदक्षरमव्ययम्}


\chapter{अध्यायः २१८}
\twolineshloka
{गुरुरुवाच}
{}


\twolineshloka
{निष्कल्मषं ब्रह्मचर्यमिच्छताचरितुं सदा}
{निद्रा सर्वात्मना त्याज्या स्वप्नदोषमवेक्षता}


\twolineshloka
{स्वप्ने हि रजसा देही तमसा चाभिभूयते}
{देहान्तरमिवापन्नश्चरत्यपगतस्मृतिः}


\twolineshloka
{ज्ञानाभ्यासाज्जागरिता जिज्ञासार्थमनन्तरम्}
{विज्ञानाभिनिवेशात्तु स जागर्त्यनिशं सदा}


\twolineshloka
{अत्राह कोन्वयं भावः स्वप्ने विषयवानिव}
{प्रलीनैरिन्द्रियैर्देही वर्तते देहवानिव}


\twolineshloka
{अत्रोच्यते यथा ह्येतद्वेद योगेश्वरो हरिः}
{तथैतदुपपन्नार्थं वर्णयन्ति महर्षयः}


\twolineshloka
{इन्द्रियाणां श्रमात्स्वप्नमाहुः सर्वगतं मनः}
{`तन्मयानीन्द्रियाण्याहुस्तावद्गच्छन्ति तानि वै}


\twolineshloka
{अत्राहुस्त्रितयं नित्यमतथ्यमिति चेच्च न}
{प्रथमे वर्तमानोऽसौ त्रितयं चेति सर्वदा}


\twolineshloka
{नेतरावुपसंगम्य विजानाति कथंचन}
{स्वप्नावस्थागतो ह्येष स्वप्न इत्येव वेत्ति च}


\twolineshloka
{तदप्यसदृशं युक्त्या त्रितयं मोहलक्षणम्}
{यदात्मत्रितयान्मुक्तस्तदा जानात्यसत्कृतः ॥'}


\threelineshloka
{मनसस्त्वप्रलीनत्वात्तत्तदाहुर्निदर्शनम्}
{कार्ये चासक्तमनसः संकल्पो जाग्रतो ह्यपि}
{यद्वन्मनोरथैश्चर्यं स्वप्ने तद्वन्मनोगतम्}


\twolineshloka
{संसाराणामसंख्यानां कामात्मा तदवाप्नुयात्}
{मनस्यन्तर्हितं सर्वं वेद सोत्तमपूरुषः}


\twolineshloka
{गुणानामपि यद्येतत्कर्मणा चाप्युपस्थितम्}
{तत्तच्छंसन्ति भूतानि मनो यद्भावितं यथा}


\twolineshloka
{ततस्तमुपसर्पन्ति गुणा राजसतामसाः}
{सात्विका वा यथायोगमानन्तर्यफलोदयम्}


\twolineshloka
{ततः पश्यन्त्यसंबन्धान्वातपित्तकफोत्तरान्}
{रजस्तमोभवैर्भावैस्तदप्याहुर्दुरत्ययम्}


\twolineshloka
{प्रसन्नैरिन्द्रियैर्यद्यत्संकल्पयति मानसम्}
{तत्तत्स्वप्नेप्युपरते मनो बुद्धिर्निरीक्षते}


\twolineshloka
{व्यापकं सर्वभूतेषु वर्तते दीपवन्मनः}
{आत्मप्रभावात्तं विद्यात्सर्वा ह्यात्मनि देवताः}


\twolineshloka
{मनस्यन्तर्हितं द्वारं देहमास्थाय मानुषम्}
{यत्तत्सदसदव्यक्तं स्वपित्यस्मिन्निदर्शनम्}


\twolineshloka
{`व्यक्तभेदमतीतोऽसौ चिन्मात्रं परिदृश्यते}
{'सर्वभूतात्मभूतस्थं तमध्यात्मगुणं विदुः}


\twolineshloka
{लिप्सेन मनसा यश्च संकल्पादैश्वरं गुणम्}
{आत्मप्रसादात्तं विद्यात्सर्वा ह्यात्मनि देवताः}


\twolineshloka
{एवं हि तपसा युञ्ज्यादर्कवत्तमसः परम्}
{त्रैलोक्यप्रकृतिर्देही तमसोन्ते महेश्वरम्}


\twolineshloka
{तपो ह्यधिष्ठितं देवैस्तपोघ्नमसुरैस्तमः}
{एतद्देवासुरैर्गुप्तं तदाहुर्ज्ञानलक्षणम्}


\twolineshloka
{सत्त्वं रजस्तमश्चेति देवासुरगुणान्विदुः}
{सत्त्वं देवगुणं विद्यादितरावासुरौ गुणौ}


\twolineshloka
{`सत्त्वं मनस्तथा बुद्धिर्देवा इत्यभिशंब्दिताः}
{तैरेव हि वृतस्तस्माज्ज्ञात्वैवं परमं-----}


\twolineshloka
{निद्राविकल्पेन सतां---- विशति लोकवत्}
{स्वस्थो भवति गूढात्मा कलुषैः परिवर्जितः}


\threelineshloka
{निशादिका ये कथिता लोकानां कलुषा मताः}
{तैर्हीनं यत्पुरं शुद्धं बाह्याभ्यन्तरवर्तिनम्}
{सदानन्दमयं नित्यं भूत्वा तत्परमन्वियात्}


\threelineshloka
{एवमाख्यातमत्यर्थं ब्रह्मचर्यमकल्मषम्}
{सर्वसंयोगहीनं तद्विष्ण्वाख्यं परमं पदम्}
{अचिन्त्यमद्भुतं लोके ज्ञानेन परिवर्तते ॥'}


\twolineshloka
{ब्रह्म तत्परमं ज्ञानममृतं ज्योतिरक्षरम्}
{ये विदुर्भावितात्मानस्ते यान्ति परमां गतिम्}


\twolineshloka
{हेतुमच्छक्यमाख्यातुमेतावज्ज्ञानचक्षुषा}
{प्रत्याहारेण वा शक्यमव्यक्तं ब्रह्म वेदितुम्}


\chapter{अध्यायः २१९}
\twolineshloka
{गुरुरुवाच}
{}


\twolineshloka
{न स वेद परं ब्रह्म यो न वेद चतुष्टयम्}
{व्यक्ताव्यक्तं च यत्तत्त्वं संप्रोक्तं परमर्षिणा}


\twolineshloka
{व्यक्तं मृत्युमुखं विद्यादव्यक्तममृतं पदम्}
{निवृत्तिलक्षणं धर्ममृषिर्नारायणोऽब्रवीत्}


\twolineshloka
{तत्रैवावस्थितं सर्वं त्रैलोक्यं सचराचरम्}
{निवृत्तिलक्षणं धर्ममव्यक्तं ब्रह्म शाश्वतम्}


\twolineshloka
{प्रवृत्तिलक्षणं धर्मं प्रजापतिरतथाब्रवीत्}
{प्रवृत्तिः पुनरावृत्तिर्निवृत्तिः परमा गतिः}


\twolineshloka
{तां गतिं परमामेति निवृत्तिपरमो मुनिः}
{ज्ञानतत्त्वपरो नित्यं शुभाशुभनिदर्शकः}


\twolineshloka
{तदेवमेतौ विज्ञेयावव्यक्तपुरुषाबुभौ}
{अव्यक्तपुरुषाभ्यां तु यत्स्यादन्यन्महत्तरम्}


\twolineshloka
{तं विशेषमवेक्षेति विशेषेण विचक्षणः}
{अनाद्यन्तावुभावेतावलिङ्गौ चाप्युभावपि}


\twolineshloka
{उभौ नित्यावनुचरौ महद्भ्यश्च महत्तरौ}
{सामान्यमेतदुभयोरेवं ह्यन्यद्विशेषणम्}


\twolineshloka
{प्रकृत्या सर्गधर्मिण्या तथा त्रिगुणसत्वया}
{विपरीतमतो विद्यात्क्षेत्रज्ञस्य स्वलक्षणम्}


\twolineshloka
{प्रकृतेश्च विकाराणां द्रष्टारमगुणान्वितम्}
{`क्षेत्रज्ञमाहुर्जीवं तु कर्तारं गुणसंवृतम्}


\twolineshloka
{अग्राह्यं येन जानन्ति तज्ज्ञानं दंशितश्च तत्}
{तेनैव दंशितो नित्यं न गुणः परिभूयते}


\twolineshloka
{अग्राह्यौ पुरुषावेतावलिङ्गत्वादसङ्गिनौ}
{संयोगलक्षणोत्पत्तिः कर्मजा गृह्यते यथा}


\twolineshloka
{करणैः कर्मनिर्वृत्तैः कर्ता यद्यद्विचेष्टते}
{कीर्त्यते शब्दसंज्ञाभिः कोऽहमेषोप्यसाविति}


\threelineshloka
{`ममापि कायमिति च तदज्ञो नित्यसंवृतः}
{'उष्णीषवान्यथा वस्त्रैस्त्रिभिर्भवति संवृतः}
{संवृतोऽयं तथा देही सत्त्वराजसतामसैः}


\twolineshloka
{`भेदवस्तु त्वभेदेन जानाति स यदा पुमान्}
{तदा परं परात्माऽसौ भवत्येव निरञ्जनः}


\twolineshloka
{क्रियायोगे च भेदाख्ये बहु संक्षिप्यते क्वचित्}
{वसुरुद्रगणाद्येषु स्वानुभोगेन भोगतः}


\twolineshloka
{एवमेष परः सत्त्वो नानारूपेण संस्थितः}
{संक्षिप्तो दृश्यते पश्चादेकरूपेण विष्ठितः ॥'}


\threelineshloka
{तस्माच्चतुष्टयं वेद्यमेतैर्हेतुभिरावृतम्}
{तथासंज्ञो ह्ययं सम्यगन्तकाले न मुह्यति}
{`वायुर्विधो यथा भानुर्विप्रकाशं गमिष्यति ॥'}


\twolineshloka
{श्रियं दिव्यामभिप्रेप्सुर्वर्ष्मवान्मनसा शुचिः}
{शारीरैर्नियमैरुग्रैश्चरेन्निष्कल्मषं तपः}


\twolineshloka
{त्रैलोक्यं तपसा व्याप्तमन्तर्भूतेन भास्वता}
{सूर्यश्च चन्द्रमाश्चैव भासतस्तपसा दिवि}


\twolineshloka
{`अन्यच्च धर्मसाम्यं यत्तपस्तत्कीर्त्यते पुनः}
{'प्रकाशस्तपसो ज्ञानं लोके संशब्दितं तपः}


\twolineshloka
{रजस्तमोघ्नं यत्कर्म तपसस्तत्स्वलक्षणम्}
{`त्रितयं ह्येतदाख्यातं यद्यस्माद्भासितुं पुनः}


\twolineshloka
{स्वभासा भासयंश्चापि चन्द्रमा ह्यत्र वर्तते}
{सूर्ययोगे तु यः सन्धिस्तपः सर्वं प्रदीप्यते ॥'}


\twolineshloka
{ब्रह्मचर्यमहिंसा च शारीरं तप उच्यते}
{वाङ्भनोनियमः सम्यङ्भानसं तप उच्यते}


\twolineshloka
{विधिज्ञेभ्यो द्विजातिभ्यो ग्राह्यमन्नं विशिष्यते}
{आहारनियमेनास्य पाप्मा शाम्यति राजसः}


\twolineshloka
{वैमनस्यं च विषये यान्त्यस्य करणानि च}
{तस्मात्तन्मात्रमादद्याद्यावदत्र प्रयोजनम्}


\twolineshloka
{अन्तकाले बलोत्कर्षाच्छनैः कुर्यादनातुरः}
{एवं युक्तेन मनसा ज्ञानं यदुपपद्यते}


\twolineshloka
{रजोवर्ज्यो ह्ययं देही देहवाञ्छब्दवांश्चरेत्}
{कार्यैरव्याहतमतिर्वैराग्यात्प्रकृतौ स्थितः}


\twolineshloka
{आ देहादप्रमादाच्च देहान्ताद्विप्रमुच्यते}
{हेतुयुक्तः सदा सार्गो भूतानां प्रलयस्तथा}


\twolineshloka
{परप्रत्ययसर्गे तु नियमो नातिवर्तते}
{एवं तत्प्रभवां प्रज्ञामासते ये विषर्यये}


\twolineshloka
{धृत्या देहान्धारयन्तो बुद्धिसंक्षिप्तचेतसः}
{स्थानेभ्यो ध्वंसमानाश्च सूक्ष्मत्वात्तदुपासते}


\twolineshloka
{यथागमं च तत्सर्वं बुद्ध्या तन्नैव बुद्ध्यते}
{देहान्तं कश्चिदन्वास्ते भावितात्मा निराश्रयः}


\twolineshloka
{युक्तो धारणया कश्चित्सतः केचिदुपासते}
{अभ्यस्यन्ति परं देवं विद्यासंशब्दिताक्षरम्}


\twolineshloka
{अन्तकाले ह्युपासन्ते तपसा दग्धकिल्विषाः}
{सर्व एते महात्मानो गच्छन्ति परमां गतिम्}


\twolineshloka
{सूक्ष्मं विशेषणं तेषामवेक्षेच्छास्त्रचक्षुषा}
{देहं तु परमं विद्याद्विमुक्तमपरिग्रहम्}


\twolineshloka
{अन्तरिक्षादन्यतरं धारणासक्तमानसम्}
{मर्त्यलोकाद्विमुच्यन्ते विद्यासंसक्तचेतसः}


\twolineshloka
{ब्रह्मभूता विरजसस्ततो यान्ति परां गतिम्}
{एवमेकायनं धर्ममाहुर्वेदविदो जनाः}


\threelineshloka
{यथाज्ञानमुपासन्तः सर्वे यान्ति परां गतिम्}
{कषायवर्जितं----तेषामुत्पद्यतेऽमलम्}
{यान्ति तेऽपि----- कान्विशुध्यन्ति यथाबलं}


\twolineshloka
{भगवन्तमजं दिव्य विष्णुमव्यक्तसंज्ञितम्}
{भावेन यान्ति शुद्धा ये ज्ञानतृप्ता निराशिषः}


\twolineshloka
{ज्ञात्वाऽऽत्मस्थं------- न निवर्तन्ति तेऽव्ययाः}
{प्राप्य तत्परमं स्थानमोदन्तेऽक्षरमव्ययम्}


\twolineshloka
{एतावदेतद्विज्ञानमेतदस्ति च नास्ति च}
{तृष्णाबद्धं जगत्सर्वं चक्रवत्परिवर्तते}


\twolineshloka
{विसतन्तुर्यथैवायमन्तस्थः सर्वतो बिसे}
{तृष्णातन्तुरनाद्यन्तस्तथा देहगतः सदा}


\twolineshloka
{सूच्या सूत्रं यथा वस्त्रे संसारयति वायकः}
{तद्वत्संसारसूत्रं हि तृष्णासूच्या निबद्ध्यते}


\twolineshloka
{`इतस्ततः समाहृत्य रूपं निर्वर्तयिष्यति}
{'विकारं प्रकृतिं चैव पुरुषं च सनातनम्}


\threelineshloka
{यो यथावद्विजानाति स वितृष्णो विमुच्यते}
{याति नित्यं स सद्भावमात्मनो वै महद्भुवम् ॥भीष्म उवाच}
{}


\twolineshloka
{प्रकाशं भगवानेतदृषिर्नारायणोऽमृतम्}
{भूतानामनुकम्पार्थं जगाद जगतो हितम्}


\chapter{अध्यायः २२०}
\twolineshloka
{युधिष्ठिर उवाच}
{}


\threelineshloka
{केन वृत्तेन वृत्तज्ञो जनको मिथिलाधिपः}
{जगाम मोक्षं धर्मज्ञो भोगानुत्सृज्य बुद्धिमान् ॥भीष्म उवाच}
{}


\twolineshloka
{अत्राप्युदाहरन्तीममितिहासं पुरातनम्}
{येन वृत्तेन धर्मज्ञः स जगाम महत्सुखम्}


\twolineshloka
{जनको जनदेवस्तु मिथिलायां जनाधिप}
{और्ध्वदेहिकधर्माणामासीद्युक्तो विचिन्तने}


\twolineshloka
{तस्य स्म शतमाचार्या वसन्ति सततं गृहे}
{दर्शयन्तः पृथग्धर्मान्नानापाषण्डवादिनः}


\twolineshloka
{स तेषां प्रेत्यभावेन प्रेत्य गातौ विनिश्चये}
{आगमस्थः स भूयिष्ठमात्मतत्त्वेन तुष्यन्ति}


\twolineshloka
{तत्र पञ्चशिखो नाम कापिलेयो महामुनिः}
{परिधावन्महीं कृत्स्नां जगाम यलामथ}


\twolineshloka
{सर्वसंन्यासधर्माणां तत्त्वज्ञाननिश्चये}
{सुपर्यवसितार्थश्च निर्द्वन्द्वो नष्टगशयः}


\twolineshloka
{ऋषीणामाहुरेकं यं कामाद-----नृषु}
{शाश्वतं सुखमत्यन्तमन्वि---सुदुर्लभम्}


\twolineshloka
{यमाहुः कपिलं साङ्ख्याः परमर्षि प्रजापतिम्}
{समेत्य तेन रूपेण विस्मापयति हि स्वयम्}


\twolineshloka
{आसुरेः प्रथमं शिष्यं यमाहुश्चिरजीविनम्}
{पञ्चस्रोतसि यः सत्रमास्ते वर्षसहस्रिकम्}


\threelineshloka
{तमासीनं समागम्य कापिलं मण्डलं महत्}
{[पञ्चस्रोतसि निष्णातः पञ्चरात्रविशारदः}
{}


\twolineshloka
{पञ्चज्ञः पञ्चकृत्पञ्चगुणः पञ्चशिखः स्मृतः}
{]पुरुषावस्थमव्यक्तं परमार्थं न्यवेदयत्}


\twolineshloka
{इष्ट्वा सत्रेण संपृष्टो भूयश्च तपसाऽऽसुरिः}
{क्षेत्रक्षेत्रज्ञयोर्व्यक्तिं बुबुधे देवदर्शनात्}


\threelineshloka
{यत्तदेकाक्षरं ब्रह्म नानारूपं प्रदृश्यते}
{`बोधायनपरान्विप्रानृषिभावमुपागतः}
{'आसुरिर्मण्डले तस्मिन्प्रतिपेदे तदव्ययम्}


\twolineshloka
{तस्य पञ्चशिखः शिष्यो मानुष्याः पयसा भृतः}
{ब्राह्मणी कपिला नाम काचिदासीत्कुटुम्बिनी}


\twolineshloka
{तस्याः पुत्रत्वमागम्य स्त्रियाः स पिबति स्तनौ}
{ततः स कापिलेयत्वं लेभे बुद्धिं च नैष्ठिकीम्}


\twolineshloka
{एतन्मे भगवानाह कापिलेयस्य संभवम्}
{तस्य तत्कापिलेयत्वं सर्ववित्त्वमनुत्तमम्}


\twolineshloka
{सामान्यं जनकं ज्ञात्वा धर्मज्ञानामनुत्तमम्}
{उपेत्य शतमाचार्यान्मोहयामास हेतुभिः}


\twolineshloka
{`निराकरिष्णुस्तान्सर्वांस्तेषां हेतुगुणान्वहून्}
{श्रावयामास मतिमान्मुनिः पञ्चशिखो नृप ॥'}


\twolineshloka
{जनकस्त्वभिसंरक्तः कापिलेयानुदर्शनात्}
{उत्सृज्य शतमाचार्यान्पृष्ठतोऽनुजगाम तम्}


\twolineshloka
{तस्मै परमकल्याय प्रणताय च धर्मतः}
{अब्रवीत्परमं मोक्षं यतः साङ्ख्यं विधीयते}


\twolineshloka
{जातिनिर्वेदमुक्त्वा स कर्मनिर्वेदमब्रवीत्}
{कर्मनिर्वेदमुक्त्वा च सर्वनिर्वेदमब्रवीत्}


\twolineshloka
{यदर्थं धर्मसंसर्गः कर्मणां च फलोदयः}
{तमनाश्वासिकं मोहं विनाशि चलमध्रुवम्}


\twolineshloka
{दृश्यमाने विनाशे च प्रत्यक्षे लोकसाक्षिके}
{आगमात्परमस्तीति ब्रुवन्नपि पराजितः}


\twolineshloka
{आत्मना ह्यात्मनो नित्यं क्लेशमृत्युजरामयम्}
{आत्मानं मन्यते मोहात्तदसम्यक्परं मतम्}


\twolineshloka
{अथ चेदेवमप्यस्ति यल्लोके नोपपद्यते}
{अजरोऽयममृत्युश्च राजाऽसौ मन्यते तथा}


\twolineshloka
{अस्ति नास्तीति चाप्येतत्तस्मिन्नसति लक्षणे}
{किमधिष्ठाय तद्ब्रूयाल्लोकयात्राविनिश्चयम्}


\twolineshloka
{प्रत्यक्षं ह्येतयोर्मूलं कृतान्तैतिह्ययोरपि}
{प्रत्यक्षेणागमो भिन्नः कृतान्तो वा न कश्चन}


\twolineshloka
{यत्रतत्रानुमानेऽस्मिन्कृतं भावयतोऽपि च}
{नान्यो जीवः शरीरस्य नास्तिकानां मते स्मृतः}


\twolineshloka
{रेतो वटकणीकायां घृतपाकाधिवासनम्}
{जातिः स्मृतिरयस्कान्तः सूर्यकान्तोऽम्बुभक्षणम्}


\twolineshloka
{प्रेत्य भूताप्ययश्चैव देवताभ्युपयाचनम्}
{मृते कर्मनिवृत्तिश्च प्रमाणमिति निश्चयः}


\twolineshloka
{न त्वेते हेतवः सन्ति ये केचिन्मूर्तिसंस्थिताः}
{अमूर्तस्य हि मूर्तेन सामान्यं नोपपद्यते}


\twolineshloka
{अविद्याकर्मचेष्टानां केचिदाहुः पुनर्भवे}
{कारणं लोभमोहौ तु दोषाणां च निषेवणम्}


\twolineshloka
{अविद्यां क्षेत्रमाहुर्हि कर्मबीजं तथा कृतम्}
{तृष्णासंजननं स्नेह एष तेषां पुनर्भवः}


\twolineshloka
{तस्मिन्मूढे च जग्धे च देहे मरणधर्मिणि}
{अन्योऽसौ जायते प्रेतस्तदाहुस्तत्वमक्षयम्}


\twolineshloka
{यदा स्वरूपतश्चान्यो जातितः श्रुतितोऽर्थतः}
{कथमस्मिन्स इत्येवं संबोधः स्यादसंहितः}


\twolineshloka
{एवं सति च का प्रीतिर्दानविद्यातपोबलैः}
{यद्यदाचरितं कर्म सर्वमन्यत्प्रपद्यते}


\twolineshloka
{यदि ह्ययमिहैवान्यैः प्राकृतैर्दुःखितो भवेत्}
{सुखितो दुःखितैर्वाऽपि दृश्यो ह्यस्यविनिर्णयः}


\twolineshloka
{यदा हि मुसलैर्हन्युः शरीरं न पुनर्भवेत्}
{पृथग्ज्ञानं यदन्यच्च येनैतन्नोपपद्यते}


\twolineshloka
{ऋतुसंवत्सरौ तिथ्यः शीतोष्णेऽथ प्रियाप्रिये}
{यथाऽतीता न दृश्यन्ते तादृशः सत्वसंक्षयः}


\twolineshloka
{जरयाऽभिपरीतस्य मृत्युना न विनाशिना}
{दुर्बलं दुर्बलं पूर्वं गृहस्येव विनश्यति}


\twolineshloka
{इन्द्रियाणी मनो वायुः शोणितं मांसमस्थि च}
{आनुपूर्व्या विनश्यन्ति स्वं धातुमुपयान्ति च}


\twolineshloka
{लोकयात्राविधानं च दानधर्मफलागमः}
{तदर्थं वेदशब्दाश्च व्यवहाराश्च लौकिकाः}


\twolineshloka
{इति सम्यङ्भनस्येते बहवः सन्ति हेतवः}
{एतदासीन्ममास्तीति न कश्चित्प्रतिपद्यते}


\twolineshloka
{तेषां विमृशतामेवं तत्तत्समभिधावताम्}
{क्वचिन्निविशते बुद्धिस्तत्र जीर्यति वृक्षवत्}


\twolineshloka
{एवमर्थैरनर्थैश्च दुःखिताः सर्वजन्तवः}
{आगमैरपकृष्यन्ते हस्तिपैर्हस्तिनो यथा}


\twolineshloka
{`न जातु कामः कामानामुपभोगेन शाम्यति}
{हविषा कृष्णवर्त्मेव भूय एवाभिवर्धते ॥'}


\twolineshloka
{अर्थांस्तथाऽत्यन्तमुखावहांश्चलिप्सन्त एते बहवो विशुष्काः}
{महत्तरं दुःखमनुप्रपन्नाहित्वा सुखं मृत्युवशं प्रयान्ति}


\twolineshloka
{विनाशिनो ह्यध्नुवजीवितस्यकिं बन्धुभिर्मित्रपरिग्रहैश्च}
{विहाय यो गच्छति सर्वमेवक्षणेन गत्वा न निवर्तते च}


\twolineshloka
{भूव्योमतोयानलवायवोऽपिसदा शरीरं प्रतिपालयन्ति}
{इतीदमालक्ष्य रतिः कुतो भवेद्विनाशिनो ह्यस्य न कर्म विद्यते}


\twolineshloka
{इदमनुपधिवाक्यमच्छलंपरमनिरामयमात्मसाक्षिकम्}
{नरपतिरभिवीक्ष्य विस्मितःपुनरनुयोक्तुमिदं प्रचक्रमे}


\chapter{अध्यायः २२१}
\twolineshloka
{`*भीष्म उवाच}
{}


\twolineshloka
{जनको नरदेवस्तु ज्ञापितः परमर्षिणा}
{पुनरेवानुपप्रच्छ सांपराये भवाभवौ}


\twolineshloka
{भगवन्यदिदं प्रेत्य संज्ञा भवति कस्यचित्}
{एवं सति किमज्ञानं ज्ञानं वा किं करिष्यति}


\twolineshloka
{विवादादेव सिद्धोऽसौ कारणस्येव वेदना}
{चेतनो विद्यते ह्यत्र हैतुकं च मनोगतम्}


\twolineshloka
{आगमादेव सिद्धोऽसौ स्वताः सिद्धा इति श्रुतिः}
{वर्तते पृथगन्योन्यं न ह्यपःश्रित्य कर्मसु}


\threelineshloka
{चेतनो ह्यंशवस्तत्र स्वमूर्तं धारयन्त्यतः}
{स्वभावं पौरुषं कर्म ह्यात्मानं तमुपाश्रितम्}
{तमाश्रित्य प्रवर्तन्ते देहिनो देहबन्धनाः}


\twolineshloka
{गुणज्ञानमभिज्ञानं तस्य लिङ्गानुशब्दयत्}
{पृथिव्यादिषु भूतेषु तत्तदाहुर्निदर्शनम्}


\twolineshloka
{आत्माऽसौ वर्तते भिन्नस्तत्रतत्र समन्वितः}
{परमात्मा तथीवैको देवेऽस्मिन्निति वै श्रुतिः}


\twolineshloka
{आकाशं वायुरूष्भा च स्नेहो यच्चापि पार्थिवम्}
{यथा त्रिधा प्रवर्तन्ते तथाऽसौ पुरुषः स्मृतः}


\twolineshloka
{पपस्यन्तर्हितं यद्वत्तद्वद्व्याप्तं महात्मकम्}
{पूर्वं नैश्चर्ययोगेन तस्मादेतन्न शेपवान्}


\twolineshloka
{शब्दाः कालः क्रिया देहो ममैकस्वैव कल्पना}
{स्वभावं तन्मयं त्वेदं मायारूपं तु भेदवत्}


\twolineshloka
{नानाख्यं परं शुद्धं निर्विकल्पं परात्मकम्}
{लिङ्गादि देवमध्यास्ते ज्ञानं देवस्य तत्तथा}


\twolineshloka
{चिन्मयोऽयं हि नादाख्यः शब्दश्चासौ मनो महान्}
{गतिमानुत संधत्ते वर्णमत्तत्पदान्वितम्}


\twolineshloka
{कायो नास्ति च तेषां वै अवकाशस्तथा परम्}
{एतेनोढा इति चाख्याताः सर्वे ते धर्मदूषकाः}


\twolineshloka
{अवन्धनमविज्ञानाज्ज्ञानं तद्भुवमव्ययम्}
{नानाभेदविकल्पने येषामात्मा स्मृतः सदा}


\threelineshloka
{प्रकृतेरपरस्तेषां बहवोऽप्यात्मवादिनः}
{विरोधो ह्यात्मसन्मायां न तेषां सिद्ध एव हि}
{अन्यदा च गृहीतै-----वेदबाह्यास्ततः स्मृताः}


\twolineshloka
{एकानेकात्मकं तेषां प्रतिषेधो हि भेदनुत्}
{तस्माद्वेदस्य हृदयमद्वैध्यमिति विद्धि तत्}


\twolineshloka
{वेदादृष्टेरयं लोकः सर्वार्थेषु प्रवर्तते}
{तस्माच्च स्मृतयो जाताः सेतिहासाः पृथग्विधाः}


\twolineshloka
{न यन्न साध्यं तद्ब्रह्म नादिमध्यं न चान्तवत्}
{इन्द्रियाणि च भूरीणि परा च प्रकृतिर्मनः}


% Check verse!
आत्मा च परमः शुद्धः प्रोक्तोऽसौ परमः पुमान्
\threelineshloka
{उत्पत्तिलक्षणं चेदं विपरीतमथोभयोः}
{यो वेत्ति प्रकृतिं नित्यं तथा चैवात्मनस्तु ताम्}
{प्रदहत्येष कर्माख्यं दावोद्भूत इवानलः}


% Check verse!
चिन्मात्रपरमः शुद्धः सर्वाकृतिषु वर्तते
\threelineshloka
{आकाशकल्पं विमलं नानाशक्तिसमन्वितम्}
{तापनं सर्वभूतानां ज्योतिषां मध्यमस्थितिम्}
{दुःखमस्ति न निर्दुःखं तद्विद्वान्न च लिप्यति}


\twolineshloka
{असावश्नाति यद्वत्तद्वमरोऽश्नाति यन्मधु}
{एवमेव महानात्मा नात्मानमवबुध्यते}


\threelineshloka
{एवंभूतस्त्वमित्यत्र स्वाधितो बुद्ध्यते परम}
{बुधस्य बोधनं तत्र क्रियते सद्भिरित्युत}
{न बुधस्येति वै कश्चिन्न तथावच्छृणुष्व मे}


\twolineshloka
{शोकमस्य न गत्वा ते शास्त्राणां शास्त्रदस्यवः}
{लोकं निध्नन्ति संभिन्ना ज्ञातिनोत्र वदन्त्युत}


\twolineshloka
{एवं तस्य विभोः कृत्यं धातुरस्य महात्मनः}
{क्षमन्ति ते महात्मानः सर्वद्वन्द्वविवर्जिताः}


\twolineshloka
{अतोऽन्यथा महात्मानमन्यथा प्रतिपद्यते}
{किं तेन न कृतं पापं चोरेणात्मापहारिणा}


\twolineshloka
{तस्य संयोगयोगेन शुचिरप्यशुचिर्भवेत्}
{अशुचिश्च शुचिश्चापि ज्ञानाद्देहादयो यथा}


% Check verse!
दृश्यं न चैव दृष्टं स्याद्दृष्टं दृश्यं तु नैव च
\twolineshloka
{अतीतत्रितयाः सिद्धा ज्ञानरूपेण सर्वदा}
{एवं न प्रतिपद्यन्ते रागमोहमदान्विताः}


\twolineshloka
{वेदबाह्या दुरात्मानः संसारे दुःखभागिनः}
{आगमानुगतज्ञाना बुद्धियुक्ता भवन्ति ते}


\twolineshloka
{बुद्ध्या भवति बुद्ध्या त्वं यद्बुद्धं चात्मरूपवत्}
{तमस्यन्धे न संदेहात्परं यान्ति न संशयः}


\twolineshloka
{नित्यनैमित्तिकान्कृत्वा पापहानिमवाप्य च}
{शुद्धसत्वा महात्मानो ज्ञाननिर्धूतकल्मषाः}


\twolineshloka
{असक्ताः परिवर्तन्ते संसरन्त्यथ वायुवत्}
{न युज्यन्तेऽथवा क्लेशैरहंभावोद्भवैः सह}


\twolineshloka
{इतस्ततः समाहृत्य ज्ञानं निर्वर्णयन्त्युत}
{ज्ञानान्वितस्तमो हन्यादर्कवत्स महामतिः}


\twolineshloka
{एवमात्मानमन्वीक्ष्य नानादुःखसमन्वितम्}
{देहं पङ्कमले मग्नं निर्मलं परमार्थतः}


\threelineshloka
{तमेवं सर्वदुःखात्तु मोचयेत्परमात्मवान्}
{ब्रह्मचर्यव्रतोपेतः सर्वसङ्गबहिष्कृतः}
{लघ्वाहारो विशुद्धात्मा परं निर्वाणमृच्छति}


\twolineshloka
{इन्द्रियाणि मनो वायुः शोणितं मांसमस्थि च}
{आनुपूर्व्याद्विनश्यन्ति स्वं धातुमुपयान्ति च}


\twolineshloka
{कारणानुगतं कार्यं यदि तच्च विनश्यति}
{अलिङ्गस्य कथं लिङ्गं युज्यते तन्मृषा दृढम्}


\twolineshloka
{न त्वेव हेतवः सन्ति ये केचिन्मूर्तिसंस्थिताः}
{अमर्त्यस्य च मर्त्येन सामान्यं नोपपद्यते}


\twolineshloka
{लोकदृष्टो यथा जातेः स्वेदजः पुरुषः स्त्रियाम्}
{कृतानुस्मरणात्सिद्धो वेदगम्यः परः पुमान्}


% Check verse!
प्रत्यक्षानुगतो वेदो नामहेतुभिरिष्यते
\threelineshloka
{यथा शाखा हि वै शाखा तरोः संबध्यते तदा}
{श्रुत्या तथापरोप्यात्मा दृश्यते सोऽप्यलिङ्गवान्}
{अलिङ्गसाध्यं तद्ब्रह्म बहवः सन्ति हेतवः}


\twolineshloka
{लोकयात्राविधानं च दानधर्मफलागमः}
{तदर्थं वेदशब्दाश्च व्यवहाराश्च लौकिकाः}


\twolineshloka
{इति सम्यङ्भनस्येते बहवः सन्ति हेतवः}
{एतदस्तीदमस्तीति न किंचित्प्रतिदृश्यते}


\twolineshloka
{तेषां विमृशतामेवं तत्तत्समभिधावताम्}
{क्वचिन्निविशते बुद्धिस्तत्र जीर्यति वृक्षवत्}


\twolineshloka
{एवमर्थैरनर्थैश्च दुःखिताः सर्वजन्तवः}
{आगमैरपकृष्यन्ति हस्तिनो हस्तिपैर्यथा}


\twolineshloka
{न जातु कामः कामामामुपभोगेन शाम्यति}
{हविषा कृष्णवर्त्मेव भूय एवाभिर्वधते}


\twolineshloka
{अर्थांस्तथाऽत्यन्तदुःखाबहांश्चलिप्सन्त एके बहवो विशुष्काः}
{महत्तरं दुःखमभिप्रपन्नाहित्वा सुखं मृत्युवशं प्रयान्ति}


\twolineshloka
{विनाशिनो ह्यध्रुवजीवितस्यकिं बन्धुभिर्मन्त्रपरिग्रहैश्च}
{विहाय यो गच्छति सर्वमेवक्षणेन गत्वा न निवर्तते च}


\twolineshloka
{स्वं भूमितोयानलवायवो हिसदा शरीरं प्रतिपालयन्ति}
{इतीदमालक्ष्य कुतो रतिर्भवेद्विनाशिनो ह्यस्य न कर्म विद्यते}


\twolineshloka
{इदमनुपधिवाक्यमच्छलंपरमनिरामयमात्मसाक्षिकम्}
{नरपतिरनुवीक्ष्य विस्मितःपुनरनुयोक्तुमिदं प्रचक्रमे ॥'}


\chapter{अध्यायः २२२}
\twolineshloka
{भीष्म उवाच}
{}


\threelineshloka
{जनको नरदेवस्तु ज्ञापितः परमर्षिणा}
{पुनरेवानुपप्रच्छ सांपराये भवाभवौ ॥जनक उवाच}
{}


\twolineshloka
{भगवन्यदि न प्रेत्य संज्ञा भवति कस्यचित्}
{एवं सति किमज्ञानं ज्ञानं वा किं करिष्यति}


\twolineshloka
{सर्वमुच्छेदनिष्ठं स्यात्पश्य चैतद्द्विजोत्तम}
{अप्रमत्तः प्रमत्तो वा किं विशेषं करिष्यति}


\threelineshloka
{असंसर्गो हि भूतेषु संसर्गो वा विनाशिषु}
{कस्मै क्रियेत तत्वेन निश्चयः कोऽत्र तत्त्वतः ॥भीष्म उवाच}
{}


\twolineshloka
{तमसा हि प्रतिच्छन्नं विभ्रान्तमिव चातुरम्}
{पुनः प्रशमयन्वाक्यैः कविः पञ्चशिखोऽब्रवीत्}


\threelineshloka
{उच्छेदनिष्ठा नेहास्ति भा निष्ठा न विद्यते}
{अयं ह्यपि समाहारः शरीरेन्द्रियचेतसाम्}
{वर्तते पृथगन्योन्यमप्यपाश्रित्य कर्मसु}


\twolineshloka
{धावतः पञ्च तेषां तु खं वायुर्ज्योतिरम्बु भूः}
{ते स्वभावेन तिष्ठन्ति वियुज्यन्ते स्वभावतः}


\twolineshloka
{आकाशो वायुरूष्मा च स्नेहो यश्चापि पार्थिवः}
{एष पञ्चसमाहारः शरीरमपि नैकधा}


\twolineshloka
{`अहं वाच्यं द्विजानां यद्विशिष्टं बुद्धिरूपवत्}
{वाचामगोचरं नित्यं ज्ञेयमेवं भविष्यति}


\twolineshloka
{ज्ञानं ज्ञेयं तथा ज्ञानं त्रिविधं ज्ञानमुच्यते}
{'ज्ञानमूष्मा च वायुश्च त्रिविधः कर्मसंग्रहः}


\twolineshloka
{इन्द्रियाणीन्द्रियार्थाश्च स्वभावश्चेतना मनः}
{प्राणापानौ विकारश्च धातवश्चात्र निःसृताः}


\twolineshloka
{`प्राणादयस्तथा स्पर्शा न संबाधगतास्तथा}
{पुत्राधीनं भविष्येत चिन्मात्रः स परः पुमान्}


\twolineshloka
{श्रवणं स्पर्शनं जिह्वा दृष्टिर्नासा तथैव च}
{इन्द्रियाणीति पञ्चैते चित्तपूर्वगमा गुणाः}


\twolineshloka
{तत्र विज्ञानसंयुक्ता त्रिविधा चेतना ध्रुवा}
{सुखदुःखेति यामाहुरदुःखेत्यसुखेति च}


\twolineshloka
{शब्दः स्पर्शश्च रूपं च रसो गन्धश्च मूर्तयः}
{एते ह्यामरणात्पञ्च षङ्गुणा ज्ञानसिद्धये}


\twolineshloka
{तेषु कर्मविसर्गश्च सर्वतत्वार्थनिश्चयः}
{तमाहुः परमं शुक्रं `पारे च रजसः प्रभुम्}


\twolineshloka
{विरागाद्वर्तते तस्मिन्मतो रजसि नित्यगम्}
{तस्मिन्प्रसन्ने संपश्ये' द्वुद्धिरित्यव्ययं महत्}


\twolineshloka
{इमं गुणसमाहारमात्मभावेन पश्यतः}
{असम्यद्गर्शिनो दुःखमनन्तं नोपशाम्यति}


\threelineshloka
{`तस्मादेतेषु मेधावी न प्रसज्येत बुद्धिमान्}
{'अनात्मेति च यद्दृष्टं तन्नाहं न ममेत्यपि}
{वर्तते किमधिष्ठाना प्रसक्ता दुःखसंततिः}


\twolineshloka
{यत्र सम्यङ्भनो नाम त्यागमात्रमनुत्तमम्}
{शृणु यत्तव मोक्षाय भाष्यमाणं भविष्यति}


\twolineshloka
{त्याग एव हि सर्वेषां युक्तानामपि कर्मणाम्}
{नित्यदुःखविनीतानां श्लेषो दुःखवहो हतः}


\twolineshloka
{द्रव्यत्यागे तु कर्माणि भोगत्यागे व्रतान्यपि}
{सुखत्यागे तपोयोगं सर्वत्यागे समापना}


\twolineshloka
{तस्य मार्गोऽयमद्वैधः सर्वत्यागस्य दर्शितः}
{विप्रहाणाय दुःखस्य दुर्गतिस्त्वन्यथा भवेत्}


\twolineshloka
{`शेते जरामृत्युभयैर्विमुक्तःक्षीणे पुण्ये विगते च पापे}
{तपोनिमित्ते विगते च निष्ठेफले यथाऽऽकाशमलिङ्ग एव ॥'}


\twolineshloka
{पञ्चज्ञानेन्द्रियाण्युक्त्वा मनःषष्ठानि चेतसि}
{मनःषष्ठानि वक्ष्यामि पञ्चकर्मेन्द्रियाणि तु}


\twolineshloka
{हस्तौ कर्मेन्द्रियं ज्ञेयमथ पादौ गतीन्द्रियम्}
{प्रजनानन्दयोः शेफो निसर्गे पायुरिन्द्रियम्}


\twolineshloka
{वाक्च शब्दविशेषार्थं गतिं पञ्चान्वितां विदुः}
{एवमेकादशैतानि बुद्ध्या तूपहतं मनः}


\twolineshloka
{कर्णौ शब्दश्च चित्तं च त्रयः श्रवणसंग्रहे}
{तथा स्पर्शे तथा रूपे तथैव रसगन्धयोः}


\twolineshloka
{एवं पञ्चत्रिका ह्येते गुणास्तदुपलब्धये}
{येनायं त्रिविधो भावः पर्यायात्समुंपस्थितः}


\twolineshloka
{सात्विको राजसश्चापि तामसश्चापि ते त्रयः}
{त्रिविधा वेदना येषु प्रसूताः सर्वसाधनाः}


\twolineshloka
{प्रहर्षः प्रीतिरानन्दः सुखं संशान्तचित्तता}
{अकुतश्चित्कुतश्चिद्वा चिन्तितः सात्विको गुणः}


\twolineshloka
{अतुष्टिः परितापश्च शोको लोभस्तथाऽक्षमा}
{लिङ्गानि रजसस्तानि दृश्यन्ते हेत्वहेतुतः}


\twolineshloka
{अविवेकस्तथा मोहः प्रमादः स्वप्नतन्द्रिता}
{कथंचिदपि वर्तन्ते विविधास्तामसा गुणाः}


\twolineshloka
{तत्र यत्प्रीतिसंयुक्तं काये मनसि वा भवेत्}
{वर्तते सात्विको भाव इत्यपेक्षेत तत्तथा}


\twolineshloka
{यत्तु सन्तापसंयुक्तमप्रीतिकरमात्मनः}
{प्रवृत्तं रज इत्येवं ततस्तदपि चिन्तयेत्}


\twolineshloka
{अथ यन्मोहसंयुक्तं काये मनसि वा भवेत्}
{अप्रतर्क्यमविज्ञेयं तमस्तदुपधारयेत्}


\twolineshloka
{श्रोत्रं व्योमाश्रितं भूतं शब्दः श्रोत्रं समाश्रितः}
{नोभयं शब्दविज्ञाने विज्ञानस्तेतरस्य वा}


\twolineshloka
{एवं त्वक्चक्षुषी जिह्वा नासिका चेति पञ्चमी}
{स्पर्शे रूपे रसे गन्धे तानि चेतो मनश्च तत्}


\twolineshloka
{स्वकर्मयुगपद्भावो दशस्वेतेषु तिष्ठति}
{चित्तमेकादशं विद्धि बुद्धिर्द्वादशमी भवेत्}


\twolineshloka
{तेषामयुगपद्भाव उच्छेदो नास्ति तामसः}
{आस्थितो युगपद्भावे व्यवहारः स लौकिकः}


\twolineshloka
{इन्द्रियाण्युपसृत्यापि दृष्ट्वा पूर्वं श्रुतागमात्}
{चिन्तयन्ननुपर्येति त्रिभिरेवान्वितो गुणैः}


\twolineshloka
{यत्तमोपहतं चित्तमाशुसंचारमध्रुवम्}
{करोत्युपरमं काये तदाहुस्तामसं सुखम्}


\twolineshloka
{यद्यदागमसंयुक्तं न कृच्छ्रादुपशाम्यति}
{अथ तत्राप्युपादत्ते तमो व्यक्तमिवानृतम्}


\twolineshloka
{एवमेव प्रसङ्ख्यातः स्वकर्मप्रत्ययो गुणः}
{कथंचिद्वर्तते सम्यक्केषांचिद्वा निवर्तते}


\twolineshloka
{`अहमित्येष वै भावो नान्यत्र प्रतितिष्ठति}
{यस्य भावो दृढो नित्यं स वै विद्वांस्तथेतरः}


\twolineshloka
{देहधर्मस्तथा नित्यं सर्वभूतेषु वै दृढः}
{एतेनैवानुमानेन त्याज्यो धर्मस्तथा ह्यसौ}


\twolineshloka
{ज्ञानेन मुच्यते जन्तुर्धर्मात्मा ज्ञानवान्भवेत्}
{धर्मेण धार्यते लोकः सर्वं धर्मे प्रतिष्ठितम्}


\twolineshloka
{सर्वार्थजनकश्चैव धर्मः सर्वस्य कारणम्}
{सर्वो हि दृश्यते लोके न सर्वार्थः कथंचन}


\twolineshloka
{सर्वत्यागे कृते तस्मात्परमात्मा प्रसीदति}
{व्यक्तादव्यक्तमतुलं लोकेषु परिवर्तते ॥'}


\twolineshloka
{एतदाहुः समाहारं क्षेत्रमध्यात्मचिन्तकाः}
{स्थितो मनसि यो भावः स वै क्षेत्रज्ञ उच्यते}


\twolineshloka
{एवं सति क उच्छेदः शाश्वतो वा कथं भवेत्}
{स्वभावाद्वर्तमानेषु सर्वभूतेषु हेतुषु}


\twolineshloka
{यथार्णवगता नद्यो व्यक्तीर्जहति नाम च}
{नतु स्वतां नियच्छन्ति तादृशः सत्वसंक्षयः}


\twolineshloka
{एवं सति कुतः संज्ञा प्रेत्यभावे पुनर्भवेत्}
{प्रतिसंमिश्रिते जीवे गृह्यमाणे च सर्वतः}


\twolineshloka
{इमां च यो वेद विमोक्षबुद्धिमात्मानमन्विच्छति चाप्रमत्तः}
{न लिप्यते कर्मफलैरनिष्टैःपत्रं बिसस्येव जलेन सिक्तम्}


\twolineshloka
{दृढैर्हि पाशैर्बहुभिर्विमुक्तःप्रजानिमित्तैरपि दैवतैश्च}
{यदा ह्यसौ सुखदुःखे जहातिमुक्तस्तदाग्र्यां गतिमेत्यलिङ्गः}


\threelineshloka
{श्रुतिप्रमाणागममङ्गलैश्चशेते जरामृत्युभयादभीतः}
{क्षीणे च पुण्ये विगते च पापेततो निमित्ते च फले विनष्टे}
{अलेपमाकाशमलिङ्गमेवमास्थाय पश्यन्ति महत्यसक्ताः}


\twolineshloka
{यथोर्णनाभिः परिवर्तमानस्तन्तुक्षये तिष्ठति पात्यमानः}
{तथा विमुक्तः प्रजहाति दुःखंबिध्वंसते लोष्ठ इवाद्रिमृच्छन्}


\twolineshloka
{यथा रुरुः शृङ्गमथो पुराणंहित्वा त्वचं वाऽप्युरगो यथा च}
{विहाय गच्छत्यनवेक्षमाणस्तथा विमुक्तो विजहाति दुःखम्}


\twolineshloka
{द्रुमं यथावाऽप्युदकै पतन्तमुत्सृज्य पक्षी निपतत्यसक्तः}
{तथा ह्यसौ सुखदुःखे विहायमुक्तः पराद्धर्यां गतिमेत्यलिङ्गः}


\threelineshloka
{`इमान्स्वलोकाननुपश्य सर्वान्व्रजन्यथाऽऽकाशमिवाप्नुकामः}
{इमां हि गाथां प्रलपन्यथाऽस्तिसमस्तसङ्कल्पविशेषमुक्तः}
{अहं हि सर्वं किल सर्वभावेह्यहं तदन्तर्ह्यहमेव भोक्ता ॥'}


\threelineshloka
{अपिच भवति मैथिलेन गीतंनगरमुपाहितमग्निनाऽभिवीक्ष्य}
{न खलु मम तुषोऽपि दह्यतेऽत्रस्वयमिदमाह किल स्म भूमिपालः ॥भीष्म उवाच}
{}


\twolineshloka
{इदममृतपदं विदेहराजास्वयमिह पञ्चशिखेन भाष्यमाणम्}
{निखिलमभिसमीक्ष्य निश्चितार्थःपरमसुखी विजहार वीतशोकः}


\twolineshloka
{इमं हि यः पठति विमोक्षनिश्चयंमहीपते सततमवेक्षते तथा}
{उपद्रवान्नानुभवत्यदुःखितःप्रमुच्यते कपिलमिवैत्य मैथिलः}


\chapter{अध्यायः २२३}
\twolineshloka
{`* युधिष्ठिर उवाच}
{}


\threelineshloka
{किं कारणं महाप्राज्ञ दह्यमानश्च मैथिलः}
{मिथिलां नेह धर्मात्मा प्राह वीक्ष्य विदाहिताम् ॥भीष्म उवाच}
{}


\twolineshloka
{श्रृयतां नृपशार्दूल यदर्थं दीपिता पुरा}
{वह्निना दीपिता सा तु तन्मे शृणु महामते}


\twolineshloka
{जनको जनदेवस्तु कर्माण्याध्याय चात्मनि}
{सर्वभावमनुप्राप्य भावेन विचचार सः}


\twolineshloka
{यजन्ददंस्तथा जुह्वन्पालयन्पृथिवीमिमाम्}
{अध्यात्मविन्महाप्राज्ञस्तन्मयत्वेन निष्ठितः}


\twolineshloka
{स तस्य हृदि संकल्पं ज्ञातुमैच्छत्स्वयं प्रभुः}
{सर्वलोकाधिपस्तत्र द्विजरूपेण संयुतः}


\twolineshloka
{मिथिलायां महाबुद्धिर्व्यलीकं किंचिदाचरन्}
{स गृहीत्वा द्विजश्रेष्ठैर्नृपाय प्रतिवेदितः}


\twolineshloka
{अपराधं समुद्दिश्य तं राजा प्रत्यभाषत}
{न त्वां ब्राह्मण दण्डेन नियोक्ष्यामि कथंचन}


\twolineshloka
{मम राज्याद्विनिर्गच्छ यावत्सीमा भुवो मम}
{तच्छ्रुत्वा ब्राह्मणो गत्वा राजानं प्रत्युवाच ह}


\twolineshloka
{करिष्ये वचनं राजन्ब्रवीहि मम जानतः}
{का सीमा तव भूमेस्तु ब्रूहि धर्मं ममाद्य वै}


\twolineshloka
{तच्छ्रुत्वा मैथिलो राजा लज्जयावनताननः}
{नोवाच वचनं विप्रं तत्वबुद्ध्या समीक्ष्य तत्}


\twolineshloka
{पुनःपुनश्च तं विप्रश्चोदयामास सत्वरम्}
{ब्रूहि राजेन्द्र गच्छामि तव राज्या द्विवासितः}


\threelineshloka
{ततो नृपो विचार्यैवमाह ब्राह्मणपुङ्गवम्}
{आवासो वा न मेऽस्त्यत्र सर्वा वा पृथिवी मम}
{गच्छ वा तिष्ठ वा ब्रह्मन्निति मे निश्चिता मतिः}


\twolineshloka
{इत्युक्तः स तथा तेन मैथिलेन द्विजोत्तमः}
{अब्रवीत्तं महात्मानं राजानं मन्त्रिभिर्वृतम्}


\twolineshloka
{त्वमेवं पद्मनाभस्य नित्यं पक्षपदाहितः}
{अहो सिद्धार्थरूपोऽसि गमिष्ये स्वस्ति तेऽस्तु वै}


\twolineshloka
{इत्युक्त्वा प्रययौ विप्रस्तज्जिज्ञासुर्द्विजोत्तमान्}
{अदहच्चाग्निना तस्य मिथिलां भगवान्स्वयम्}


\twolineshloka
{प्रदीप्यमानां मिथिलां दृष्ट्वा राजा न कम्पितः}
{जनैः स परिपृष्टस्तु वाक्यमेतदुवाच ह}


\twolineshloka
{अनन्तं वत मे वित्तं भाव्यं मे नास्ति किंचन}
{मिथिलायां प्रदीप्तायां न मे किंचन दह्यते}


\twolineshloka
{तदस्य भाषमाणस्य श्रुत्वा श्रुत्वा हृदि स्थितम्}
{पुनः संजीवयामास मिथिलां तां द्विजोत्तमः}


\twolineshloka
{आत्मानं दर्शयामास वरं चास्नै दद्रौ पुनः}
{धर्मे तिष्ठस्व सद्भावो बुद्धिस्तेऽर्थे नराधिप}


\threelineshloka
{सत्ये तिष्ठस्व निर्विण्णः स्वस्ति तेऽस्तु व्रजाम्यहम्}
{इत्युक्त्वा भगवांश्चैनं तत्रैवान्तरधीयत}
{एतत्ते कथितं राजन्किं भूयः श्रोतुमिच्छसि ॥'}


\chapter{अध्यायः २२४}
\twolineshloka
{`युधिष्ठिर उवाच}
{}


\fourlineindentedshloka
{अस्ति कश्चिद्यदि विभो सदारो नियतो गृहे}
{अतीतसर्वसंसारः सर्वद्वन्द्वविवर्जितः}
{तं मे ब्रूहि महाप्राज्ञ दुर्लभः पुरुषो महान् ॥भीष्म उवाच}
{}


\twolineshloka
{शृणु राजन्यथावृत्तं यन्मां त्वं पृष्टवानसि}
{इतिहासमिमं शुद्धं संसारभयभेषजम्}


\twolineshloka
{देवलो नाम विप्रर्षिः सर्वशास्त्रार्थकोविदः}
{क्रियावान्धार्मिको नित्यं देवब्राह्मणपूजकः}


\threelineshloka
{सुता सुवर्चला नाम तस्य कल्याणलक्षणा}
{नातिह्रस्वा नातिकृशा नातिदीर्घा यशस्विनी}
{प्रदानसमयं प्राप्ता पिता तस्य ह्यचिन्तयत्}


\twolineshloka
{अस्याः पतिः कुतो वेति ब्राह्मणः श्रोत्रियः परः}
{विद्वान्विप्रो ह्यकुटुम्बः प्रियवादी महातपाः}


% Check verse!
इत्येवं चिन्तयानं तं रहस्याह सुवर्चला
\threelineshloka
{अन्धाय मां महाप्राज्ञ देह्यनन्धाय वै पितः}
{एवं स्मर सदा विद्वन्ममेदं प्रार्थितं मुने ॥पितोवाच}
{}


\twolineshloka
{न शक्यं प्रार्थितं वत्से त्वयाऽद्य प्रतिभाति मे}
{अन्धतानन्धता चेति विकारो मम जायते}


\twolineshloka
{उन्मत्तेवाशुभं वाक्यं भाषसे शुभलोचने ॥सुवर्चलोवाच}
{}


\twolineshloka
{नाहमुन्मत्तभूताऽद्य बुद्धिपूर्वं ब्रवीमि ते}
{विद्यते चेत्पतिस्तादृक्स मां भरति वेदवित्}


\threelineshloka
{येभ्यस्त्वं मन्यसे दातुं मामिहानय तान्द्विजान्}
{तादृशं तं पतिं तेषु वरयिष्ये यथातथम् ॥भीष्म उवाच}
{}


\twolineshloka
{तथेति चोक्त्वा तां कन्यामृषिः शिष्यानुवाच ह}
{ब्राह्मणान्वेदसंपन्नान्योनिगोत्रविशोधितान्}


\twolineshloka
{मातृतः पितृतः शुद्धाञ्शुद्धानाचारतः शुभान्}
{अरोगान्बुद्धिसंपन्नाञ्शीलसत्वगुणान्वितान्}


\threelineshloka
{असंकीर्णांश्च गोत्रेषु वेदव्रतसमन्वितान्}
{ब्राह्मणान्स्नातकाञ्शीघ्रं मातापितृसमन्वितान्}
{निवेष्टुकामान्कन्यां मे दृष्ट्वाऽऽनयत शिष्यकाः}


\twolineshloka
{तच्छ्रुत्वा त्वरिताः शिष्या ह्याश्रमेषु ततस्ततः}
{ग्रामेषु च ततो गत्वा ब्राह्मणेभ्यो न्यवेदयन्}


\twolineshloka
{ऋषेः प्रभावं मत्वा ते कन्यायाश्च द्विजोत्तमाः}
{अनेकमुनयो राजन्संप्राप्ता देवलाश्रमम्}


\twolineshloka
{अनुमान्य यथान्यायं मुनीन्मुनिकुमारकान्}
{अभ्यर्च्य विधिवत्तत्र कन्यामाह पिता महान्}


\twolineshloka
{एतेऽपि मुनयो वत्से स्वपुत्रैकमता इह}
{वेदवेदाङ्गसंपन्नाः कुलीनाः शीलसंमताः}


\twolineshloka
{येऽमी तेषु वरं भद्रे त्वमिच्छसि महाव्रतम्}
{तं कुमारं वृणीष्वाद्य तस्मै दास्याम्यहं शुभे}


\twolineshloka
{तथेति चोक्त्वा कल्याणी तप्तहेमनिभा तदा}
{सर्वलक्षणसंपन्ना वाक्यमाह यशस्विनी}


\twolineshloka
{विप्राणां समितीर्दृष्ट्वा प्रणिपत्य तपोधनान्}
{यद्यस्ति समितौ विप्रो ह्यन्धोऽनन्धः स मे वरः}


\twolineshloka
{तच्छ्रुत्वा मुनयस्तत्र वीक्षमाणाः परस्परम्}
{नोचुर्विप्रा महाभागाः कन्यां मत्वा ह्यवेदिकां}


\twolineshloka
{कुत्सयित्वा मुनिं तत्र मनसा मुनिसत्तमाः}
{यथागतं ययुः क्रुद्धा नानादेशनिवासिनः}


% Check verse!
कन्या च संस्थिता तत्र पितृवेश्मनि भामिनी
\twolineshloka
{ततः कदाचिद्ब्रह्मण्यो विद्वान्न्यायविशारदः}
{ऊहापोहविधानज्ञो ब्रह्मचर्यसमन्वितः}


\twolineshloka
{वेदविद्वेदतत्वज्ञः क्रियाकल्पविशारदः}
{आत्मतत्वविभागज्ञः पितृमान्गुणसागरः}


\twolineshloka
{श्वेतकेतुरिति ख्यातः श्रुत्वा वृत्तान्तमादरात्}
{कन्यार्थं देवलं चापि शीघ्रं तत्रागतोऽभवत्}


\twolineshloka
{उद्दालकसुतं दृष्ट्वा श्वेतकेतुं महाव्रतम्}
{यथान्यायं च संपूज्य देवलः प्रत्यभाषत}


\twolineshloka
{कन्ये एष महाभागे प्राप्तो ऋषिकुमारकः}
{वरयैनं महाप्राज्ञं वेदवेदाङ्गपारगम्}


\twolineshloka
{तच्छ्रुत्वा कुपिता कन्या ऋषिपुत्रमुदैक्षत}
{तां कन्यामाह विप्रर्षिः सोऽहं भद्रे समागतः}


\threelineshloka
{अन्धोऽहमत्र तत्वं हि तथा मन्ये च सर्वदा}
{विशालनयनं विद्धि तथा मां हीनसंशयम्}
{वृणीष्व मां वरारोहे भजे च त्वामनिन्दिते}


\twolineshloka
{येनेदं वीक्षते नित्यं वृणोति स्पृशतेऽथवा}
{घ्रायते वक्ति सततं येनेदं सार्यते पुनः}


\twolineshloka
{येनेदं मन्यते तत्वं येन बुध्यति वा पुनः}
{न चक्षुर्विद्यते ह्येतत्स वै भूतान्ध उच्यते}


\twolineshloka
{यस्मिन्प्रवर्तते चेदं पश्यञ्छृण्वन्स्पृशन्नपि}
{जिघ्रंश्च रसयंस्तद्वद्वर्तते येन चक्षुषा}


\twolineshloka
{तन्मे नास्ति ततो ह्यन्धो वृणु भद्रेऽद्य मामतः}
{लोकदृष्ट्या करोमीह नित्यनैमित्तिकादिकम्}


\twolineshloka
{आत्मदृष्ट्या च तत्सर्वं विलिप्यासि च नित्यशः}
{स्थितोऽहं निर्भरः शान्तः कार्यकारणभावनः}


\fourlineindentedshloka
{अविद्यया तरन्मृत्युं विद्यया तं तथाऽमृतम्}
{यथाप्राप्तं तु संदृश्य वसामीह विमत्सरः}
{क्रीते व्यवसितं भद्रे भर्ताऽहं ते वृणीष्व माम् ॥भीष्म उवाच}
{}


\threelineshloka
{ततः सुवर्चला दृष्ट्वा प्राह तं द्विजसत्तमम्}
{मनसाऽसि वृतो विद्वञ्शेषकर्ता पिता मम}
{वृणीष्व पितरं मह्यमेष वेदविधिक्रमः}


\twolineshloka
{तद्विज्ञाय पिता तस्या देवलो मुनिसत्तमः}
{श्वेतकेतुं च संपूज्य तथैवोद्दालकेन तम्}


\twolineshloka
{मुनीनामग्रतः कन्यां प्रददौ जलपूर्वकम्}
{उदाहरन्ति वै तत्र श्वेतकेतुं निरीक्ष्य तम्}


\twolineshloka
{हृत्पुण्डरीकनिलयः सर्वभूतात्मको हरिः}
{श्वेतकेतुस्वरूपेण स्थितोऽसौ मधुसूदनः}


\threelineshloka
{प्रीयतां माधवो देवः पत्नी चेयं सुता मम}
{प्रतिपादयामि ते कन्यां सहधर्मचरीं शुभाम्}
{इत्युक्त्वा प्रददौ तस्मै देवलो मुनिपुङ्गवः}


\twolineshloka
{प्रतिगृह्य च तां कन्यां श्वेतकेतुर्महायशाः}
{उपयम्य यथान्यायमत्र कृत्वा यथाविधि}


\twolineshloka
{समाप्य तन्त्रं मुनिभिर्वैवाहिकमनुत्तमम्}
{स गार्हस्थ्ये वसन्धीमान्भार्यां तामिदमब्रवीत्}


\twolineshloka
{यानि चोक्तानि वेदेषु तत्सर्वं कुरु शोभने}
{मया सह यथान्यायं सहधर्मचरी मम्}


\twolineshloka
{अहमित्येव भावेन स्थितोऽहं त्वं तथैव च}
{तस्मात्कर्माणि कुर्वीथाः कुर्यां ते च ततः परम्}


\twolineshloka
{न ममेति च भावेन ज्ञानाग्निनिलयेन च}
{अनन्तरं तथा कुर्यास्तानि कर्माणि भस्मसात्}


\threelineshloka
{एवं त्वया च कर्तव्यं सर्वदा दुर्भगा मया}
{यद्यदाचरति श्रेष्ठस्तत्तदेवेतरो जनः}
{तस्माल्लोकस्य सिद्ध्यर्थं कर्तव्यं चात्मसिद्धये}


\twolineshloka
{उक्त्वैवं स महाप्राज्ञः सर्वज्ञानैकभाजनः}
{पुत्रानुत्पाद्य तस्यां च यज्ञैः संतर्प्य देवताः}


\twolineshloka
{आत्मयोगपरो नित्यं निर्द्वन्द्वो निष्परिग्रहः}
{भार्यां तां सदृशीं प्राप्य बुद्धिं क्षेत्रज्ञयोरिव}


\twolineshloka
{लोकमन्यमनुप्राप्तौ भार्या भर्ता तथैव च}
{साक्षिभूतौ जगत्यस्मिंश्चरमाणौ मुदाऽन्वितौ}


\twolineshloka
{ततः कदाचिद्भर्तारं श्वेतकेतुं सुवर्चला}
{पप्रच्छ को भवानत्र ब्रूहि मे तद्द्विजोत्तम}


\twolineshloka
{तामाह भगवान्वाग्मी तया ज्ञातो न संशयः}
{द्विजोत्तमेति मामुक्त्वा पुनः कमनुपृच्छसि}


\twolineshloka
{सा तमाह महात्मानं पृच्छामि हृदि शायिनम्}
{तच्छ्रुत्वा प्रत्युवाचैनां स न वक्ष्यति भामिनि}


\twolineshloka
{नामगोत्रसमायुक्तमात्मानं मन्यसे यदि}
{तन्मिथ्यागोत्रसद्भावे वर्तते देहबन्धनम्}


\threelineshloka
{अहमित्येष भावोऽत्र त्वयि चापि समाहितः}
{त्वमप्यहमहं सर्वमहमित्येव वर्तते}
{नात्र तत्परमार्थं वै किमर्थमनुपृच्छसि}


\twolineshloka
{ततः प्रहस्य सा हृष्टा भर्तारं धर्मचारिणी}
{उवाच वचनं काले स्मयमाना तदा नृप}


\fourlineindentedshloka
{किमनेकप्रकारेण विरोधेन प्रयोजनम्}
{क्रियाकलापैर्ब्रह्मर्षे ज्ञाननष्टोऽसि सर्वदा}
{तन्मे ब्रूहि महाप्राज्ञ यथाऽहं त्वामनुव्रता ॥श्वेतकेतुरुवाच}
{}


\twolineshloka
{यद्यदाचरति श्रेष्ठस्तत्तदेवेतरो जनः}
{वर्तते तेन लोकोऽयं संकीर्णश्च भविष्यति}


\twolineshloka
{संकीर्णे च तथा धर्मे वर्णः संकरमेति च}
{संकरे च प्रवृत्ते तु मात्स्यो न्यायः प्रवर्तते}


\twolineshloka
{तदनिष्टं हरेर्भद्रे धातुरस्य महात्मनः}
{परमेश्वरसंक्रीडा लोकसृष्टिरियं शुभे}


\twolineshloka
{यावत्पासव उद्दिष्टास्तावत्योऽस्य विभूतयः}
{तावत्यश्चैव मायास्तु तावत्योऽस्याश्च शक्तयः}


\threelineshloka
{एवं सुगह्वरे युक्तो यत्र मे तद्भवाभवम्}
{छित्त्वा ज्ञानासिना गच्छेत्स विद्वान्स च मे प्रियः}
{सोऽहमेव न सन्देहः प्रतिज्ञा इति तस्य वै}


\twolineshloka
{ये मूढास्ते दुरात्मानो धर्मसंकरकारकाः}
{मर्यादाभेदका नीचा नरके यान्ति जन्तवः}


% Check verse!
आसुरीं योनिमापन्ना इति देवानुशासनम्
\threelineshloka
{भगवत्या तथा लोके रक्षितव्यं न संशयः}
{मर्यादालोकरक्षार्थमेवमस्ति तथा स्थितः ॥सुवर्चलोवाच}
{}


\threelineshloka
{शब्दः कोत्र इति ख्यातस्तथाऽर्थं च महामुने}
{आकृत्या पतयो ब्रूहि लक्षणेन पृथक्पृथक् ॥श्वेतकेतुरुवाच}
{}


\threelineshloka
{व्यत्ययेन च वर्णानां परिवादकृतो हि यः}
{स शब्द इति विज्ञेयस्तन्निपातोऽर्थ उच्यते ॥सुवर्चलोवाच}
{}


\threelineshloka
{शब्दार्थयोर्हि संबन्धस्त्वनयोरस्ति वा न वा}
{तन्मे ब्रूहि यथातत्वं शब्दस्यानेऽर्थ एव चेत् ॥श्वेतकेतुरुवाच}
{}


\threelineshloka
{शब्दार्थयोर्न चैवास्ति संबन्धोऽत्यन्त एव हि}
{पुष्करे च यथा तोयं तथाऽस्मीति च वेत्थ तत् ॥सुवर्चलोवाच}
{}


\threelineshloka
{अर्थे स्थितिर्हि शब्दस्य नान्यथा च स्थितिर्भवेत्}
{विद्यते चेन्महाप्राज्ञ विनाऽर्थं ब्रूहि सत्तम ॥श्वेतकेतुरुवाच}
{}


\threelineshloka
{ससंसर्गोऽतिमात्रस्तु वाचकत्वेन वर्तते}
{अस्ति चेद्वर्तते नित्यं विकारोच्चारणेन वै ॥सुवर्चलोवाच}
{}


\threelineshloka
{शब्दस्थानोत्र इत्युक्तस्तथाऽर्थ इति मे कृतः}
{अर्थः स्थितो न तिष्ठेच्च विरूढमिह भाषितम् ॥श्वेतकेतुरुवाच}
{}


\threelineshloka
{न विकूलोऽत्र कथितो नाकाशं हि विना जगत्}
{संबन्धस्तत्र नास्त्येव तद्वदित्येष मन्यताम् ॥सुवर्चलोवाच}
{}


\threelineshloka
{सदाऽहंकारशब्दोऽयं व्यक्तमात्मनि संश्रितः}
{न वाचस्तत्र वर्तन्ते इति मिथ्या भविष्यति ॥श्वेतकेतुरुवाच}
{}


\threelineshloka
{अहंशब्दो ह्यहंभावो नात्मभावे शुभव्रते}
{न वर्तन्ते परेऽचिन्त्ये वाचः सगुणलक्षणाः ॥सुवर्चलोवाच}
{}


\threelineshloka
{अहं गात्रैकतः श्यामा भावनपि तथैव च}
{तन्मे ब्रूहि यथान्यायमेवं चेन्मुनिसत्तम ॥श्वेतकेतुरुवाच}
{}


\twolineshloka
{मृण्मये हि घटे भावस्तादृग्भाव इहेष्यते}
{अयं भावः परेऽचिन्त्ये ह्यात्मभावो यथाच तत्}


\twolineshloka
{अहं त्वमेतदित्येव परे संकल्पना मया}
{तस्माद्वाचो न वर्तन्त इति नैव विरुध्यते}


\twolineshloka
{तस्माद्वामेन वर्तन्ते मनसा भीरु सर्वशः}
{यथाऽऽकाशगतं विश्वं संसक्तमिव लक्ष्यते}


\twolineshloka
{संसर्गे सति संबन्धात्तद्विकारं भविष्यति}
{अनाकाशगतं सर्वं विकारे च सदा गतम्}


\threelineshloka
{तद्ब्रह्म परमं शुद्धमनौषम्यं न शक्यते}
{न दृश्यते तथा तच्च दृश्यते च मतिर्मम ॥सुवर्चलोवाच}
{}


\threelineshloka
{निर्विकारं ह्यमूर्ति च निरयं सर्वगं तथा}
{दृश्यते च वियन्नित्यं दृगात्मा तेन दृश्यते ॥श्वेतकेतुरुवाच}
{}


\twolineshloka
{त्वचा स्पृशति वै वायुमाकाशस्थं पुनः पुनः}
{तत्स्थं गन्धं तथाघ्राति ज्योतिः पश्यति चक्षुषा}


\twolineshloka
{तमोरश्मिगणश्चैव मेघजालं तथैव च}
{वर्षं तारागणं चैव नाकाशं दृश्यते पुनः}


\threelineshloka
{आकाशस्याप्यथाकाशं सद्रूपमिति निश्चितम्}
{तदर्थे कल्पिता ह्येते तत्सत्यो विष्णुरेव च}
{यानि नामानि गौणानि ह्युपचारात्परात्मनि}


\twolineshloka
{न चक्षुषा न मनसा न चान्येन परो विभुः}
{चिन्त्यते सूक्ष्मया बुद्ध्या वाचा वक्तुं न शक्यते}


\twolineshloka
{एतत्प्रपञ्चमखिलं तस्मिन्सर्वं प्रतिष्ठितम्}
{महाघटोऽल्पकश्चैव यथा मह्यां प्रतिष्ठितौ}


\twolineshloka
{न च स्त्री न पुमांश्चैव यथैव न नपुंसकः}
{केवलज्ञानमात्रं तत्तस्मिन्सर्वं प्रतिष्ठितम्}


\twolineshloka
{भूमिसंस्थानयोगेन वस्तुसंस्थानयोगतः}
{रसभेदा यथा तोये प्रकृत्यामात्मनस्तथा}


\threelineshloka
{तद्वाक्यस्मरणान्नित्यं तृप्तिं वारि पिबन्निव}
{प्राप्नोति ज्ञानमखिलं तेन तत्सुखमेधते ॥सुवर्चलोवाच}
{}


\twolineshloka
{अनेन साध्यं किं स्याद्वै शब्देनेति मतिर्मम}
{वेदगम्यः परोऽचिन्त्य इति पौराणिका विदुः}


\threelineshloka
{निरर्थको यथा लोके तद्वत्स्यादिति मे मतिः}
{निरीक्ष्यैवं यथान्यायं वक्तुमर्हसि मेऽनघ ॥श्वेतकेतुरुवाच}
{}


\twolineshloka
{वेदगम्यं परं शुद्धमिति सत्या परा श्रुतिः}
{व्याहत्या नैतदित्याह व्युपलिङ्गे च वर्तते}


\twolineshloka
{निरर्थको न चैवास्ति शब्दो लौकिक उत्तमे}
{अनन्वयास्तथा शब्दा निरर्था इति लौकिकैः}


\twolineshloka
{गृह्यन्ते तद्वदित्येव न वर्तन्ते परात्मनि}
{अगोचरत्वं वचसां युक्तमेवं तथा शुभे}


\threelineshloka
{साधनस्योपदेशाच्च ह्युपायस्य च सूचनात्}
{उपलक्षणयोगेन व्यावृत्त्या च प्रदर्शनात्}
{वेदगम्यः परः शुद्ध इति मे धीयते मतिः}


\twolineshloka
{अध्यात्मध्यानसंभूतमभूतं----- वत्स्फुटम्}
{ज्ञानं विद्धि शुभाचारे तेन यान्ति परां गतिम्}


\twolineshloka
{यदि मे व्याहृतं गुह्यं श्रुतं न तु त्वया शुभे}
{तथ्यमित्येव वा शुद्धे ज्ञानं ज्ञानविलोचने}


\twolineshloka
{नानारूपवदस्यैवमैश्वर्यं दृश्यते शुभे}
{न वायुस्तं न सूर्यस्तं नाग्निस्तत्तत्परं पदम्}


\fourlineindentedshloka
{अनेन पूर्णमेतद्धि हृदि भूतमिहेष्यते}
{एतावदात्मविज्ञानमेतावद्यदहं स्मृतम्}
{आवयोर्न च सत्वे वै तस्मादज्ञानबन्धनम् ॥भीष्म उवाच}
{}


\twolineshloka
{एवं सुवर्चला हृष्टा प्रोक्ता भर्त्रा यथार्थवत्}
{परिचर्यमाणा ह्यनिशं तत्वबुद्धिसमन्विता}


\twolineshloka
{भर्ता च तामनुप्रेक्ष्य नित्यनैमित्तिकान्वितः}
{परमात्मनि गोविन्दे वासुदेवे महात्मनि}


\twolineshloka
{समाधाय च कर्माणि तन्मयत्वेन भावितः}
{कालेन महता राजन्प्राप्नोति परमां गतिम्}


\twolineshloka
{एतत्ते कथितं राजन्यस्मात्त्वं परिपृच्छसि}
{गार्हस्थ्यं च समास्थाय गतौ जायापती परम्'}


\chapter{अध्यायः २२५}
\twolineshloka
{युधिष्ठिर उवाच}
{}


\threelineshloka
{किं कुर्वन्सुखमाप्नोति किं कुर्वन्दुःखमाप्नुयात्}
{किं कुर्वन्निर्भयो लोके सिद्धश्चरति भारत ॥भीष्म उवाच}
{}


\twolineshloka
{दममेव प्रशंसन्ति वृद्धाः श्रुतिसमाधयः}
{सर्वेषामेव वर्णानां ब्राह्मणस्य विशेषतः}


\twolineshloka
{नादान्तस्य क्रियासिद्धिर्यथावदुपपद्यते}
{क्रिया तपश्च देवाश्च दमे सर्वं प्रतिष्ठितम्}


\twolineshloka
{दमस्तेजो वर्धयति पवित्रं दम उच्यते}
{विपाष्मा निर्भयो दान्तः पुरुषो विन्दते महत्}


\twolineshloka
{सुखं दान्तः प्रस्वपिति सुखं च प्रतिबुध्यते}
{सुखं लोके विपर्येति मनश्चास्य प्रसीदति}


\twolineshloka
{तेजो दमन ध्रियते तत्र तीक्ष्णोऽधिगच्छति}
{अमित्रांश्च बहून्नित्यं पृथगात्मनि पश्यति}


\twolineshloka
{क्रव्याद्भ्य इव भूतानामदान्तेभ्यः सदा भयम्}
{तेषां विप्रतिषेधार्थं राजा सृष्टः स्वयंभुवा}


\threelineshloka
{आश्रमेषु च सर्वेषु दम एव विशिष्यते}
{12-225-8b`धर्मः संरक्ष्यतेतैस्तु यतस्ते धर्मसेतवः}
{' यच्च तेषु फलं धर्म्यं भूयो दान्ते तदुच्यते}


\twolineshloka
{तेषां लिङ्गानि वक्ष्यानि येषां समुदयो दमः}
{अकार्पण्यमसंरम्भः संतोषः श्रद्दधानता}


\twolineshloka
{अक्रोध आर्जवं नित्यं नातिवादोऽभिमानिता}
{गुरुपूजाऽनसूया च दया भूतेष्वपैशुनम्}


\twolineshloka
{जनवादमृषावादस्तुतिनिन्दाविवर्जनम्}
{साधुकामांश्च स्पृहयेन्नायतिं प्रत्ययेषु च}


\twolineshloka
{अवैरकृत्सूपचारः समो निन्दाप्रशंसयोः}
{सुवृत्तः शीलसंपन्नः प्रसन्नात्माऽऽत्मवाञ्शुचिः}


\twolineshloka
{प्राप्य लोके च सत्कारं स्वर्गं वै प्रेत्य गच्छति}
{दुर्गमं सर्वभूतानां प्रापयन्मोदते सुखी}


\twolineshloka
{सर्वभूतहिते युक्तो न स्म यो द्विषते जनम्}
{महाह्रद इवाक्षोभ्यः प्राज्ञस्तृप्तः प्रसीदति}


\twolineshloka
{अभयं यस्य भूतेभ्यः सर्वेषामभयं यतः}
{नमस्यः सर्वभूतानां दान्तो भवति बुद्धिमान्}


\twolineshloka
{न हृष्यति महत्यर्थे व्यसने च न शोचति}
{सदाऽपरिमितप्रज्ञः स दान्तो द्विज उच्यते}


\twolineshloka
{कर्मभिः श्रुतसंपन्नः सद्भिराचारेतः शुचिः}
{सदैव दमसंयुक्तस्तस्य भुङ्क्ते महाफलम्}


\twolineshloka
{अनसूयाऽक्षमा शान्तिः संतोषः प्रियवादिता}
{सत्यं दानमनायासो नैष मार्गो दुरात्मनाम्}


\twolineshloka
{कामक्रोधौ च लोभश्च परस्येर्ष्या विकत्थना}
{`अतुष्टिरनृतं मोह एष मार्गो दुरात्मनाम् ॥'}


\threelineshloka
{कामक्रोधौ वशे कृत्वा ब्रह्मचारी जितेन्द्रियः}
{विक्रम्य घोरे तमसि ब्राह्मणः संशितव्रतः}
{कालाकाङ्क्षी चरेल्लोकान्निरपाय इवात्मवान्}


\chapter{अध्यायः २२६}
\twolineshloka
{युधिष्ठिर उवाच}
{}


\threelineshloka
{द्विजातयो व्रतोपेता यदिदं भुञ्जते हविः}
{अन्नं ब्राह्मणकामाय कथमेतत्पितामह ॥भीष्म उवाच}
{}


\threelineshloka
{अवेदोक्तव्रतोपेता भुञ्जानाः कार्यकारिणः}
{वेदोक्तेषु च भुञ्जाना व्रतलुब्धा युधिष्ठिर ॥युधिष्ठिर उवाच}
{}


\threelineshloka
{यदिदं तप इत्याहुरुपवासं पृथग्जनाः}
{एतत्तपो महाराज उताहो किं तपो भवेत् ॥भीष्म उवाच}
{}


\twolineshloka
{मासपक्षोपवासेन मन्यन्ते यत्तपो जनाः}
{आत्मतन्त्रोपधातस्तु न तपस्तत्सतां मतम्}


\twolineshloka
{त्यागश्च सन्नतिश्चैव शिष्यते तप उत्तमम्}
{सदोपवासी स भवेद्ब्रह्मचारी सदा भवेत्}


\twolineshloka
{मुनिश्च स्यात्सदा विप्रो दैवतं च सदा भवेत्}
{कुटुम्बिको धर्मपरः सदाऽस्वप्नश्च भारत}


\twolineshloka
{अमांसादी सदा च स्यात्पवित्री च सदा भवेत्}
{अमृताशी सदा च स्यान्न च स्याद्विषभोजनः}


\threelineshloka
{विधसाशी सदा च स्यात्सदा चैवातिथिप्रियः}
{[श्रद्दधानः सदा च स्याद्देवताद्विजपूजकः] ॥युधिष्ठिर उवाच}
{}


\threelineshloka
{कथं सदोपवासी स्याद्ब्रह्मचारी कथं भवेत्}
{विघसाशी कथं च स्यात्सदा चैवातिथिप्रयिः ॥भीष्म उवाच}
{}


\twolineshloka
{अन्तरा प्रातराशं च सायमाशं तथैव च}
{सदोपवासी स भवेद्यो न भुङ्क्तेऽन्तरा पुनः}


\twolineshloka
{भार्यां गच्छन्ब्रह्मचारी ऋतौ भवति ब्राह्मणः}
{ऋतवादी भवेन्नित्यं ज्ञाननित्यश्च यो नरः}


\twolineshloka
{न भक्षयेद्वृथा मांसममांसाशी भवत्यपि}
{दाननित्यः पवित्रीस्यादस्वप्नश्च दिवाऽस्वपन्}


\twolineshloka
{भृत्यातिथिषु यो भुङ्क्ते भुक्तवत्सु सदा नरः}
{अमृतं केवलं भुङ्क्ते इति विद्धि युधिष्ठिर}


\threelineshloka
{अभुक्तवत्सु भुञ्जानो विषमश्नाति वै द्विजः}
{अदत्त्वा योऽतिथिभ्योऽन्नं न भुङ्क्ते सोतिथिप्रियः}
{`अभुक्त्वा दैवतेभ्यश्च यो न भुङ्क्ते सदैवतम्'}


\twolineshloka
{देवताभ्यः पितृभ्यश्च भृत्येभ्योऽतिथिभिः सह}
{अवशिष्टं तु योऽश्नाति तमाहुर्विघसाशिनम्}


\twolineshloka
{तेषां लोका ह्यपर्यन्ताः सदने ब्रह्मणा सह}
{उपस्थिताश्चाप्सरोभिः परियान्ति दिवौकसः}


\twolineshloka
{देवताभिश्च ये सार्धं पितृभ्यश्चोपभुञ्जते}
{रमन्ते पुत्रपौत्रैश्च तेषां गतिरनुत्तमा}


\chapter{अध्यायः २२७}
\twolineshloka
{`* युधिष्ठिर उवाच}
{}


\twolineshloka
{केचिदाहुर्द्विधा लोके त्रिधा राजन्ननेकधा}
{न प्रत्ययो न चान्यच्च दृश्यते ब्रह्म नैव तत्}


\threelineshloka
{नानाविधानि शास्त्राणि उक्ताश्चैव पृथग्विधाः}
{किमधिष्ठाय तिष्ठामि तन्मे ब्रूहि पितामह ॥भीष्म उवाच}
{}


\twolineshloka
{स्वेस्वे युक्ता महात्मानः शास्त्रेषु प्रभविष्णवः}
{वर्तन्ते षण्डिता लोके को विद्वान्कश्च पण्डितः}


\twolineshloka
{सर्वेषां तत्वमज्ञाय यथारुचि तथा भवेत्}
{अस्मिन्नर्थे पुराभूतमितिहासं पुरातनम्}


\twolineshloka
{महाविवादसंयुक्तमृषीणां भावितात्मनाम्}
{हिमवत्पार्श्व आसीना ऋषयः संशितव्रताः}


\twolineshloka
{षण्णां तानि सहस्राणि ऋषीणां गणमाहितम्}
{तत्र केचिद्धुवं विश्वं सेश्वरं तु निरीश्वरम्}


\twolineshloka
{प्राकृतं कारणं नास्ति सर्वं नैवमिदं जगत्}
{अनेन चापरे विप्राः स्वभावं कर्म चापरे}


\twolineshloka
{पौरुषं कर्म दैवं च यत्स्वभावादिरेव तम्}
{नानाहेतुशतैर्युक्ता नानाशास्त्रप्रवर्तकाः}


\twolineshloka
{स्वभावाद्ब्राह्मणा राजञ्जिगीपन्तः परस्परम्}
{ततस्तु मूलमुद्भूतं वादिप्रत्यर्थिसंयुतम्}


\twolineshloka
{पात्रदण्डविघातं च वल्कलाजिनवाससाम्}
{एके मन्युसमापन्नास्ततः शान्ता द्विजोत्तमाः}


\twolineshloka
{वसिष्ठमब्रुवन्सर्वे त्वं नो ब्रूहि सनातनम्}
{नाहं जानामि विप्रेन्द्राःप्रत्युवाच स तान्प्रभुः}


\twolineshloka
{ते सर्वे सहिता विप्रा नारदं ऋषिमनुवन्}
{त्वं नो ब्रूहि महाभाग तत्वविच्च भवानसि}


\twolineshloka
{नाहं द्विजा विजानामि क्व हि गच्छाम संगताः}
{इति तानाह भगवांस्ततः प्राह च स द्विजान्}


\twolineshloka
{को विद्वानिह लोकेऽस्मिन्नमोहोऽमृतमद्भुतम्}
{तच्च ते शुश्रुवुर्वाक्यं ब्राह्मणा ह्यशरीरिणः}


% Check verse!
सनद्धाम द्विजा गत्वा पृच्छध्वं स च वक्ष्यति
\twolineshloka
{तमाह कश्चिद्विजवर्यसत्तमोविभाण्डको मण्डितवेदराशिः}
{कस्त्वं भवानर्थविभेदमध्येन दृश्यसे वाक्यमुदीरयंश्च}


% Check verse!
अथाहेदं तं भगवान्सनन्तंमहामुने विद्धि मां पण्डितोऽसिऋषिं पुराणं सततैकरूपंयमक्षयं वेदविदो वदन्ति
\twolineshloka
{पुनस्तमाहेदमसौ महात्मास्वरूपसंस्थं वद आह पार्थ}
{त्वमेकोऽस्मदृषिपुङ्गवाद्यनसत्स्वरूपमथवा पुनः किम्}


\twolineshloka
{अथाह गम्भीरतरानुवादंवाक्यं महात्मा ह्यशरीर आदिः}
{न ते मुने श्रोत्रमुखेऽपि चास्यंन पादहस्तौ प्रपदात्मके न}


\twolineshloka
{ब्रुवन्मुनीन्सत्यमथो निरीक्ष्यस्वमाह विद्वान्मनसा निगम्य}
{ऋषे कथं वाक्यमिदं ब्रवीषिन चास्य मन्ता न च विद्यते चेत्}


\twolineshloka
{न शुश्रुवुस्ततस्तत्तु प्रतिवाक्यं द्विजोत्तमाः}
{निरीक्षमाणा आकाशं प्रहसन्तस्ततस्ततः}


\twolineshloka
{आश्चर्यमिति मत्वा ते ययुर्हैमं महागिरिम्}
{सनत्कुमारसङ्काशं सगणा मुनिसत्तमाः}


\twolineshloka
{तं पर्वतं समारुह्य ददृशुर्ध्यानमाश्रिताः}
{कुमारं देवमर्हन्तं वेदपाराविवर्जितम्}


\twolineshloka
{ततः संवत्सरे पूर्णे प्रकृतिस्थं महामुनिम्}
{सनत्कुमारं राजेन्द्र प्रणिपत्य द्विजाः स्थिताः}


\threelineshloka
{आगतान्भगवानाह ज्ञाननिर्धूतकल्मषः}
{ज्ञातं मया मुनिगणा वाक्यं तदशरीरिणः}
{कार्यमद्य यथाकामं पृच्छध्वं मुनिपुङ्गवाः}


\twolineshloka
{तमब्रुवन्प्राञ्जलयो महामुनिंद्विजोत्तमं ज्ञाननिधिं सुनिर्मलम्}
{कथं वयं ज्ञाननिधिं वरेण्यंयक्ष्यामहे विश्वरूपं कुमार}


\twolineshloka
{प्रसीद नो भगवञ्ज्ञानलेशंमधुप्रयाताय सुखाय सन्तः}
{यत्तत्पदं विश्वरूपं महामुनेतत्र ब्रूहि किं तत्र महानुभाव}


\twolineshloka
{स तैर्वियुक्तो भगवान्महात्मायः सङ्गवान्सत्यवित्तच्छृणुष्व}
{अनेक साहस्रकलेषु चैवप्रसन्नधातुं च शुभाज्ञया सत्}


\twolineshloka
{यथाह पूर्वं युष्मासु ह्यशरीरी द्विजोत्तमाः}
{तथैव वाक्यं तत्सत्यमजानन्तश्च कीर्तितम्}


% Check verse!
शृणुध्वं परमं कारणमस्ति कथमवगम्यते

अहन्यहनि पाकविशेषो दृश्यतेतेन मिश्रं सर्वं मिश्रयते

यथा मण्डली दृशि सर्वेषामस्ति निदर्शनम्

अस्ति चक्षुष्मतामस्ति ज्ञाने स्वरूपं पश्यति

यथा दर्पणान्तंनिदर्शनम्
% Check verse!
स एव सर्वं विद्वान्न बिभेति न गच्छति कुत्राहं कस्य नाहं केनकेनेत्यवर्तमानो विजानाति
\threelineshloka
{स युगतो व्यापी}
{स पृथक्स्थितः}
{तदपरमार्थः}


\threelineshloka
{यथा वायुरेकः सन्बहुधेरितः}
{आश्रयविशेषो वा यस्याश्रयंयथावद्द्विजे मृगे व्याघ्रे च मनुजे वेणुंसश्रयो भिद्यते वायुरथैकः}
{आत्मा तथाऽसौ परमात्माऽसावन्य इव भाति}


% Check verse!
एवमात्मा स एव गच्छति सर्वमात्मा पश्यञ्शृणोति न च घ्राति नभाषते
\twolineshloka
{चक्रेऽस्य तं महात्मानं परितो दश रश्मयः}
{विनिष्क्रम्य यथा सूर्यमनुगच्छन्ति तं प्रभुम्}


\twolineshloka
{दिनेदिनेऽस्तमभ्येति पुनरुद्गच्छते दिशः}
{तावुभौ न रवौ चास्तां तथा वित्त शरीरिणम्}


\twolineshloka
{पतिते वित्त विप्रेन्द्रं भक्षणे चरणे परः}
{ऊर्ध्वमेकस्तथाऽधस्तादेकस्तिष्ठति चापरः}


\twolineshloka
{हिरण्यसदनं ज्ञेयं समेत्य परमं पदम्}
{आत्मना ह्यात्मदीपं तमात्मनि ह्यात्मपूरुषम्}


\twolineshloka
{संचितं संचितं पूर्वं भ्रमरो वर्तते भ्रमम्}
{योऽभिमानीव जानाति न मुह्यति न हीयते}


\twolineshloka
{न चक्षुषा पश्यति कश्चनैनंहृदा मनीषा पश्यति रुपमस्य}
{न शुक्लं न कृष्णं परमार्थभावंगुहाशयं ज्ञानदेवीकरस्थम्}


\twolineshloka
{ब्राह्मणस्य न सादृश्ये वर्तते सोऽपि किं पुनः}
{इज्यते यस्तु मन्त्रेण यजमानो द्विजोत्तमः}


\twolineshloka
{नैव धर्मी न चाधमीं द्वन्द्वातीतो विमत्सरः}
{ज्ञानतृप्तः सुखं शेते ह्यमृतात्मा न संशयः}


\twolineshloka
{एवमेव जगत्सृष्टिं कुरुते मायया प्रभुः}
{न जानाति विमूढात्मा कारणं चात्मनो ह्यसौ}


\fourlineindentedshloka
{ध्याता द्रष्टा तथा मन्ता बोद्धा दृष्टान्स एव सः}
{को विद्वान्परमात्मानमनन्तं लोकभावनम्}
{यत्तु शक्यं मया प्रोक्तं गच्छध्वं मुनिपुङ्गवाः ॥भीष्म उवाच}
{}


\twolineshloka
{एवं प्रणम्य विप्रेन्द्रा ज्ञानसागरसंभवम्}
{सनत्कुमारं संदृष्ट्वा जग्मुस्ते रुचिरं पुनः}


\twolineshloka
{तस्मात्त्वमपि कौन्तेय ज्ञानयोगपरो भव}
{ज्ञानमेवं महाराज सर्वदुःखविनाशनम्}


\twolineshloka
{इदं महादुःखसमाकराणांनृणां परित्राणविनिर्मितं पुरा}
{पुराणपुंसा ऋषिणा महात्मनामहामुनीनां प्रवरेण तद्भुवम् ॥'}


\chapter{अध्यायः २२८}
\twolineshloka
{` युधिष्ठिर उवाच}
{}


\fourlineindentedshloka
{यदिदं तप इत्याहुः किं तपः संप्रकीर्तितम्}
{उपवासमथान्यत्तु वेदाचारमथो नु किम्}
{शास्त्रं तपो महाप्राज्ञ तन्मे ब्रूहि पितामह ॥भीष्म उवाच}
{}


\threelineshloka
{पक्षमासोपवासादीन्मन्यन्ते वै तपोधनाः}
{वेदव्रतादीनि तप अपरे वेदपारगाः}
{वेदपारायणं चान्ये चाहुस्तत्वमथापरे}


\twolineshloka
{यथाविहितमाचारस्तपः सर्वं व्रतं गताः}
{आत्मविद्याविधानं यत्तत्तपः परिकीर्तितम्}


\twolineshloka
{त्यागस्तपस्तथा शान्तिस्तप इन्द्रियनिग्रहः}
{ब्रह्मचर्यं तपः प्रोक्तमाहुरेवं द्विजातयः}


% Check verse!
सदोपवासो यो विद्वान्ब्रह्मचारी सदा भवेत्
\twolineshloka
{यो मुनिश्च सदा धीमान्विघसाशी विमत्सरः}
{ततस्त्वनन्तमप्याहुर्यो नित्यमतिथिप्रियः}


\twolineshloka
{नान्तराशीस्ततो नित्यमुपवासी महाव्रतः}
{ऋतुगामी तथा प्रोक्तो विघसाशी स्मृतो बुधैः}


\twolineshloka
{भृत्यशेषं तु यो भुङ्क्ते यज्ञशेषं तथाऽमृतम्}
{एवं नानार्थसंयोगं तपः शश्वदुदाहृतम्}


\threelineshloka
{केषां लोका ह्यपर्यन्ताः सर्वे सत्यव्रते स्थिताः}
{येऽपि कर्ममयं प्राहुस्ते द्विजा ब्राह्मणाः स्मृताः}
{रमन्ते दिव्यभोगैश्च पूजिता ह्यप्सरोगणैः}


\twolineshloka
{ज्ञानात्मकं तपश्शब्दं ये वदन्ति विनिश्चिताः}
{ते ह्यन्तराऽऽत्मसद्भावं प्रपन्ना नृपसत्तम}


\threelineshloka
{एतत्ते नृपशार्दूल प्रोक्तं यत्पृष्ट्वानसि}
{यथा वस्तुनि संज्ञानि विविधानि भवन्त्युत ॥युधिष्ठिर उवाच}
{}


\twolineshloka
{पितामह महाप्राज्ञ राजाधीना नृपाः पुनः}
{अन्यानि च सहस्राणि नामानि विविधानि च}


\twolineshloka
{प्रतियोगीनि वै तेषां छन्नान्यस्तमितानि च}
{दृढं सर्वं प्राकृतकभिदं सर्वत्र पश्य वै}


\twolineshloka
{तस्माद्यथागतं राजन्यथारुचि नृणां भवेत्}
{अस्मिन्नर्थे पुरावृत्तं शृणु राजन्युधिष्ठिर}


\twolineshloka
{ब्राह्मणानां समूहे तु यदुवाच सुवर्चला}
{देवलस्य सुता विद्वन्सर्वलक्षणशोभिता}


\twolineshloka
{कन्या सुवर्चला नाम योगभावितचेतना}
{हेतुना केन जाता सा निर्द्वन्द्वा नष्टसंशया}


% Check verse!
साऽब्रवीत्पितरं विप्रं वरान्वेषणतत्परा
\threelineshloka
{अन्धाय मां महाप्राज्ञ देहि वीक्ष्य सुलोचनम्}
{एवं स्म च पितः शश्वन्मयेदं---मुने ॥पितोवाच}
{}


\fourlineindentedshloka
{न शक्यं प्रार्थितुं वत्से त्वयाऽद्य प्रतिभाति मे}
{अन्धताऽनन्धता चेति विचारो मम जायते}
{उन्मत्तेव सुते वाक्यं भाषसे पृथुलोचने ॥कन्योवाच}
{}


\twolineshloka
{नाहमुन्मत्तभूताऽऽद्य बुद्धिपूर्वं ब्रवीमि ते}
{विद्धि वैतादृशं लोके स मां भजति वेदवित्}


\twolineshloka
{यान्यांस्त्वं मन्यसे दातुं मां द्विजोत्तम तानिह}
{आनयान्यान्महाभाग ह्यहं द्रक्ष्यामि तेषु तम्}


\threelineshloka
{तथेति चोक्त्वा तां विप्रः प्रेषयामास शिष्यकान्}
{ऋषेः प्रभावं दृष्ट्वा ते कन्यायाश्च द्विजोत्तमाः}
{अनेकमुनयो राजन्संप्राप्ता देवलाश्रमम्}


\twolineshloka
{तानागतानथाभ्यर्च्यं कन्यामाह पिता महान्}
{यदीच्छसि वरं भद्रे तं विप्रं वरय स्वयम्}


\twolineshloka
{तथेचि चोक्त्वा कल्याणी तप्तहेमनिभानना}
{करसंमितमध्याङ्गी वाक्यमाह तपोधनाः}


\twolineshloka
{यद्यस्ति संमतो विप्रो ह्यन्धोऽनन्धः स मे वरः}
{नोचुर्विप्रा महाभागां प्रतिवाक्यं ययुश्च ते}


% Check verse!
कन्या च तिष्ठतामत्र पितुर्वेश्मनि भारत
\twolineshloka
{श्वेतकेतुः कहालस्य श्यालः परमधर्मवित्}
{श्रुत्वा ब्रह्मा तदागम्य कन्यामाह महीपते}


\threelineshloka
{सोहं भद्रे समावृत्तस्त्वयोक्तो यः पुरा द्विजः}
{विशालनयनं विद्धि मामन्धोऽहं वृणीष्व माम् ॥सुवर्चलोवाच}
{}


\threelineshloka
{कथं विशालनेत्रोऽसि कथं वा त्वमलोचनः}
{ब्रूहि पश्चादहं विद्वन्परीक्षे त्वां द्विजोत्तम ॥द्विज उवाच}
{}


\threelineshloka
{शब्दे स्पर्शे तथा रूपे रसे गन्धे सहेतुकम्}
{न मे प्रवर्तते चेतो न प्रत्यक्षं हि तेषु मे}
{अलोचनोऽहं तस्माद्धि न गतिर्विद्यते यतः}


\twolineshloka
{येन पश्यति सुश्रोणि भाषते स्पृशते पुनः}
{भुज्यते घ्रायते नित्यं शृणोति मनुते तथा}


\threelineshloka
{तच्चक्षुर्विद्यते मह्यं येन पश्यति वै स्फुटम्}
{सुलोचनोऽहं भद्रे वै पृच्छ वा किं वदामि ते}
{सर्वमस्मिन्न मे विद्या विद्वान्हि परमार्थतः}


\twolineshloka
{सा विशुद्धा ततो भूत्वा श्वेतकेतुं महामुनिम्}
{प्रणम्य पूजयामास तां भार्यां स च लब्धवान्}


\twolineshloka
{वैराग्यसंयुता कन्या तादृशं परिमुत्तमम्}
{प्राप्ता राजन्महाप्राज्ञ तस्मादर्थः पृथक्पृथक्}


% Check verse!
एतत्ते कथितं राजन्किं भूयः श्रोतुमिच्छसि ॥'
\chapter{अध्यायः २२९}
\twolineshloka
{युधिष्ठिर उवाच}
{}


\twolineshloka
{यदिदं कर्म लोकेऽस्मिञ्शुभं वा यदि वाऽशुभम्}
{पुरुषं योजयत्येव फलयोगेन भारत}


\threelineshloka
{कर्ता स्वित्तस्य पुरुष उताहो नेति संशयः}
{एतदिच्छामि तत्त्वेन त्वत्तः श्रोतुं पितामह ॥भीष्म उवाच}
{}


\twolineshloka
{अत्राप्युदाहरन्तीममितिहासं पुरातनम्}
{प्रह्लादस्य च संवादमिन्द्रस्य च युधिष्ठिर}


\twolineshloka
{असक्तं धूतपाप्मानं कुले जातं बहुश्रुतम्}
{अस्तब्धमनहंकारं सत्वस्थं संयतेन्द्रियम्}


\twolineshloka
{तुल्यनिन्दास्तुतिं दान्तं शून्यागारसमाकृतिम्}
{चराचराणां भूतानां विदितप्रभवाप्ययम्}


\twolineshloka
{अक्रुध्यन्तमहृष्यन्तमप्रियेषु प्रियेषु च}
{काञ्चने वाऽथ लोष्ठे वा उभयोः समदर्शनम्}


% Check verse!
आत्मनि श्रेयसि ज्ञाने धीरं निश्चितनिश्चयम् ॥परावरज्ञं भूतानां सर्वज्ञं सर्वदर्शनम्
\twolineshloka
{`अव्यक्तात्मनि गोविन्दे वासुदेवे महात्मनि}
{हृदयेन समाविष्टं सर्वभावप्रियंकरम्}


\twolineshloka
{भक्तं भागवतं नित्यं नारायणपरायणम्}
{ध्यायन्तं परमात्मानं हिरण्यकशिपोः सुतम् ॥'}


\twolineshloka
{शक्रः प्रह्लादमासीनमेकान्ते संयतेन्द्रियम्}
{बुभुत्समानस्तत्प्रज्ञामभिगम्येदमब्रवीत्}


\twolineshloka
{यैः कैश्चित्संमतो लोके गुणैः स्यात्पुरुषो नृषु}
{भवत्यनपगान्सर्वांस्तान्गुणाँल्लक्षयामहे}


\twolineshloka
{अथ ते लक्ष्यते बुद्धिः समा बालजनैरिह}
{आत्मानं मन्यमानः सञ्श्रेयः किमिह मन्यसे}


\twolineshloka
{बद्धः पाशैश्च्युतः स्थानाद्द्विषतां वशमागतः}
{श्रिया विहीनः प्रह्लाद शोचितव्ये न शोचसि}


\threelineshloka
{प्रज्ञालाभेन दैतेय उताहो धृतिमत्तथा}
{प्रह्लाद स्वस्थरूपोऽसि पश्यन्व्यसनमात्मनः ॥भीष्म उवाच}
{}


\threelineshloka
{इति संचोदितस्तेन धीरो निश्चितनिश्चयः}
{उवाच श्लक्ष्णया वाचा स्वां प्रज्ञामनुवर्णयन् ॥प्रह्लाद उवाच}
{}


\twolineshloka
{प्रवृत्तिं च निवृत्तिं च भूतानां यो न बुध्यते}
{तस्य स्तम्भो भवेद्बाल्यान्नास्ति स्तंभोऽनुपश्यतः}


\twolineshloka
{`गहनं सर्वभूतानां ध्येयं नित्यं सनातनम्}
{अनिग्रहमनौपम्यं सर्वाकारं परात्परम्}


\twolineshloka
{सर्वावरणसंभूतं तस्मादेतत्प्रवर्तते}
{तन्मया अपि संपश्य नानालक्षणलक्षिताः}


\twolineshloka
{स वै पाति जगत्स्रष्टा विष्णुरित्यभिशब्दितः}
{पुनर्दर्शति संप्राप्ते------सुरेश्चरः ॥'}


\twolineshloka
{स्वभावात्संप्रवर्तन्ते निवर्तन्ते तथैव च}
{सर्वे भावास्तथा भावाः पुरुषार्थो न विद्यते}


\twolineshloka
{पुरुषार्थस्य चाभावे नास्ति कश्चित्स्वकारकः}
{स्वयं च कुर्वतस्तस्य जातु मानो भवेदिह}


\twolineshloka
{यस्तु कर्तारमात्मानं मन्यते साध्वसाधु वा}
{तस्य दोषवती प्रज्ञा अतत्त्वज्ञेति मे मतिः}


\twolineshloka
{यदि स्यात्पुरुषः कर्ता शक्रात्मश्रेयसे ध्रुवम्}
{आरम्भास्तस्य सिद्ध्येयुर्न तु जातु पराभवेत्}


\twolineshloka
{अनिष्टस्य हि निर्वृत्तिरनिवृत्तिः प्रियस्य च}
{लक्ष्यते यतमानानां पुरुषार्थस्ततः कुतः}


\twolineshloka
{अनिष्टस्याभिनिर्वृत्तिमिष्टसंवृतिमेव च}
{अप्रयत्नेन पश्यामः केषां चित्तत्स्वभावतः}


\twolineshloka
{प्रतिरूपतराः केचिद्दृस्यन्ते बुद्धिमत्तराः}
{विरूपेभ्योऽल्पबुद्धिभ्यो लिप्समाना धनागमं}


\twolineshloka
{स्वभावप्रेरिताः सर्वे निविशन्ते गुणा यदा}
{शुभाशुभास्तदा तत्र तस्य किं मानकारणम्}


\twolineshloka
{स्वभावादेव तत्सर्वमिति मे निश्चिता मतिः}
{आत्मप्रतिष्ठा प्रज्ञा वा मम नास्ति ततोऽन्यथा}


\twolineshloka
{कर्मजं त्विह मन्यन्ते पलयोगं शुभाशुभम्}
{कर्मणां विषयं कृत्स्नमहं वक्ष्यामि तच्छृणु}


\twolineshloka
{यथा वेदयते कश्चिदोदनं पायसं ह्यदन्}
{एवं सर्वाणि कर्माणि स्वभावस्यैव लक्षणम्}


\twolineshloka
{विकारानेव यो वेद न वेद प्रकृतिं पराम्}
{तस्य स्तंभोऽभवेद्बाल्यान्नास्ति स्तंभोऽनुपश्यतः}


\twolineshloka
{स्वभावभाविनो भावान्सर्वानेवेह निश्चये}
{बुद्ध्यमानस्य दर्पो वा मानो वा किं करिष्यति}


\twolineshloka
{वेद धर्मविधिं कृत्स्नं भूतानां चाप्यनित्यताम्}
{तस्माच्छक्र न शोचारि सर्वं ह्येवेदमन्तवत्}


\twolineshloka
{निर्ममो निरहंकारो निरीहो मुक्तबन्धनः}
{स्वस्थो व्यपेतः पश्यामि भूतानां प्रभवाप्ययौ}


\twolineshloka
{कृतप्रज्ञस्य दान्तस्य वितृष्णाय निराशिषः}
{नायासो विद्यते शक्र पश्यतो योगवित्तया}


\twolineshloka
{प्रकृतौ च विकारे च न मे प्रीतिर्न च द्विषे}
{द्वेष्टारं च न पश्यामि यो ममाद्य विरुध्यति}


\threelineshloka
{नोर्ध्वं नावाङ्वतिर्यक्च न क्वचिच्छक्र कामये}
{न हि ज्ञेये न विज्ञाने नाज्ञाने विद्यतेऽन्तरम् ॥शक्र उवाच}
{}


\threelineshloka
{येनैषा लभ्यते प्रज्ञा येन शान्तिरवाप्यते}
{प्रब्रूहि तमुपायं मे सम्यक्प्रह्लाद पृच्छते ॥प्रह्लाद उवाच}
{}


\twolineshloka
{आर्जवेनाप्रमादेन प्रसादेनात्मवत्तया}
{गुरुशुश्रूषया शक्र पुरुषो लभते महत्}


\twolineshloka
{स्वभावाल्लभते प्रज्ञां शान्तिमेति स्वभावतः}
{स्वभावादेव तत्सर्वं यत्किंचिदनुपश्यति}


\twolineshloka
{`नैवान्तरं विजानाति श्रुत्वा गुरुमुखात्ततः}
{वाक्यं वाक्यार्थविज्ञानमालोक्य मनसा यतिः}


\twolineshloka
{विवेकप्रत्ययापन्नमात्मानमनुपश्यति}
{विरज्यति ततो भीत्या परमेश्वरमृच्छति}


\twolineshloka
{त्रातारं सर्वदुःखानां तत्सुखान्वेषणं ततः}
{करोति सद्भिः संसर्गमलं सन्तः सुखाय वै}


\twolineshloka
{सतां सकाशादाज्ञाय मार्गं लक्षणवत्तया}
{सर्वसङ्गविनिर्मुक्तः परमात्मानमृच्छति}


\twolineshloka
{विषयेच्छाकृतो धर्मं सरजस्को भयावहः}
{धर्महानिमवाप्नोति क्रमात्तेन नरः पुनः}


\twolineshloka
{भक्तिहीनो भवत्येव परमात्मनि चाच्युत}
{वाचके वाऽपि च स्थानं न हन्त्येव विमोचितः}


\twolineshloka
{सार्क्ष्ये चास्य रतिर्नित्यं संसारे च रतिर्भवेत्}
{तस्य नित्यमविज्ञानादात्माचैव न सिद्ध्यति}


\twolineshloka
{उन्मत्तवृत्तिर्भवति क्रमादेवं प्रवर्तते}
{आशौचं वर्धते नित्यं न शाम्यति कथंचन}


\twolineshloka
{विषये चान्वितस्यास्य मोक्षवाञ्छा न जायते}
{हेत्वाभासेषु संलीनः स्तौति वैषयिकान्गुणान्}


\threelineshloka
{न शास्त्राणि शृणोत्येव मानदर्पसमन्वितः}
{स्वतःसिद्धं न भोगस्तं स्वतः सिद्धं न वेत्ति च}
{चिद्रूपधारणं चैव परसृष्टिमथाव्ययम्}


\threelineshloka
{नानायोनिगतस्तेन भ्राम्यमाणः स्वकर्मभिः}
{तीर्णपारं न जानाति महामोहसमन्वितः}
{आचार्यसंश्रयाद्विद्याद्विनयं समुपागतः}


\twolineshloka
{अनुकूलेषु धर्मेषु चिनोत्येनं ततस्ततः}
{आचार्य इति च ख्यातस्तेनासौ बलवृत्रहन्}


\twolineshloka
{नियतेनैव सद्भावस्तेन जन्मान्तरादिषु}
{कर्मसञ्चयतूलौघः क्षिप्यते ज्ञानवायुना}


\twolineshloka
{एवं युक्तसमाचारः संसारविनिवर्तकः}
{अनुकूलवृत्तिं सततं छिनत्त्येव भृगुर्यथा}


\twolineshloka
{येन चायं समापन्नं वैतृष्ण्यं नाधिगच्छति}
{अभ्यन्तरः स्मृतः शक्र तत्साम्यं परिवर्जयेत्}


\twolineshloka
{प्रथमं तत्कृतेनैव कर्मणा परिमृच्छति}
{द्वितीयं स्वप्नयोगं च कर्मणा परिगच्छति}


\twolineshloka
{एतैरक्षैः समापन्नः प्रत्यक्षोऽसौ समास्थितः}
{सुपुप्त्याख्यस्तुरीयोसौ न च ह्यावरणान्वितः}


\twolineshloka
{लोकवृत्त्या तमीशानं यजञ्जुह्वन्यमी भवेत्}
{आत्मन्यायासयोगेन निष्क्रियं स परात्परम्}


\twolineshloka
{आयामे तां विजानाति मायैषा परमात्मनः}
{प्रातिभासिकसामान्याद्बुद्धेर्या संविदात्मिका}


\twolineshloka
{स्फुलिङ्गसत्त्वसदृशादग्निभावो यथा भवेत्}
{शिशूनामेवमज्ञानामात्मभावोऽन्यथा स्मृतः}


\twolineshloka
{साध्येऽप्यवस्तुभूताख्ये मित्रामित्रादयः कुतः}
{तदभावे तु शोकाद्या न वर्तन्ते सुरेश्वर}


\twolineshloka
{एवं बुध्यस्व भगवन्समबुद्धिं समन्वियात्}
{उपायमेतदाख्यातं मा वक्रं गच्छ देवप}


\twolineshloka
{ज्ञानेन पश्यते कर्म ज्ञानिनां न प्रवर्तकम्}
{यावदारब्धमस्येह तावन्नैवोपशाम्यति}


\twolineshloka
{तदन्ते तं प्रयात्येव न विद्वानिति मे मतिः}
{यदस्य वाचकं वक्ष्ये संस्मरे तद्भवेत्तदा}


\twolineshloka
{तेनतेन च भावेन अपायं तत्र पश्यति}
{स्थानभेदेषु वागेषा तालुसंस्था यथा तथा}


% Check verse!
तद्वदुद्धिगता ह्यर्था बुद्धिमात्मगतः सदा
\twolineshloka
{समस्तसंकल्पविशेषमुक्तंपरं पराणां परमं महात्मा}
{त्रय्यन्तविद्भिः परिगीयतेऽसौविष्णुर्विभुर्वास्ति गुणो न नित्यम्}


\threelineshloka
{वर्णेषु लोकेषु विशेषणेषुस वासुदेवो वसनान्महात्मा}
{टगुणानुरूपं स च कर्मरूपंददाति सर्वस्य समस्तरूपम्}
{न संदृशे तिष्ठति रुपमस्यन चक्षुषा पश्यति कश्चिदेनम्}


% Check verse!
भक्त्या च धृत्या च समाहितात्माज्ञानस्वरूपं परिपश्यतीह
\threelineshloka
{वदन्ति तन्मे भगवान्ददौ सस एव शेषं मघवान्महात्मा}
{एवं ममोपायमवैहि शक्रतस्माल्लोको नास्ति मह्यं सदैव ॥'भीष्म उवाच}
{}


\twolineshloka
{इत्युक्तो दैत्यपतिना शक्रो विस्मयमागमत्}
{प्रीतिमांश्च तदा राजंस्तद्वाक्यं प्रत्यपूजयत्}


\twolineshloka
{स तदाभ्यर्च्य दैत्येन्द्रं त्रैलोक्यपतिरीश्वरः}
{असुरेन्द्रमुपामन्त्र्य जगाम स्वं निवेशनम्}


\chapter{अध्यायः २३०}
\twolineshloka
{युधिष्ठिर उवाच}
{}


\threelineshloka
{यया बुद्ध्या महीपालो भ्रष्टश्रीर्विचरेन्महीम्}
{कालदण्डविनिष्पिष्टस्तन्मे ब्रूहि पितामह ॥भीष्म उवाच}
{}


\twolineshloka
{अत्राप्युदाहरन्तीममितिहासं पुरातनम्}
{वासवस्य च संवादं बलेर्वैरोचनस्य च}


\twolineshloka
{पितामहमुपागम्य प्रणिपत्य कृताञ्जलिः}
{सर्वानेवासुराञ्जित्वा बलिं पप्रच्छ वासवः}


\twolineshloka
{यस्य स्म ददतो वित्तं न कदाचन हीयते}
{तं बलिं नाधिगच्छामि ब्रह्मन्नाचक्ष्व मे बलिम्}


\twolineshloka
{स वायुर्वरुणश्चैव स रविः स च चन्द्रमाः}
{सोऽग्निस्तपति भूतानि जलं च स भवत्युत}


\twolineshloka
{तं बलिं नाधिगच्छामि ब्रह्मन्नाचक्ष्व मे बलिम्}
{स एव ह्यस्तमयते स स्म विद्योतते दिशः}


\threelineshloka
{स वर्षति स्म वर्षाणि यथाकालमतन्द्रितः}
{तं बलिं नाधिगच्छामि ब्रह्मन्नाचक्ष्व मे बलिम् ॥ब्रह्मोवाच}
{}


\twolineshloka
{नैतत्ते साधु मघवन्यदेनमनुपृच्छसि}
{पृष्टस्तु नानृतं ब्रूयात्तस्माद्वक्ष्यामि ते बलिम्}


\threelineshloka
{उष्ट्रेषु यदि वा गोषु खरेष्वश्वेषु वा पुनः}
{वरिष्ठो भविता जन्तुः शून्यागारे शचीपते ॥शक्र उवाच}
{}


\threelineshloka
{यदि स्म बलिना ब्रह्मञ्शून्यागारे समेयिवान्}
{हन्यामेनं न वा हन्यां तद्ब्रह्मन्ननुशाधि माम् ॥ब्रह्मोवाच}
{}


\threelineshloka
{मा स्म शक्र बलिं हिंसीर्न बलिर्वधमर्हति}
{न्यायस्तु शक्र प्रष्टव्यस्त्वया वासव काम्यया ॥भीष्म उवाच}
{}


\twolineshloka
{एवमुक्तो भगवता महेन्द्रः पृथिवीं तदा}
{चचारैरावतस्कन्धमधिरुह्य श्रिया वृतः}


\threelineshloka
{ततो ददर्श स बलिं खरवेषेण संवृतम्}
{यथाख्यातं भगवता शून्यागारकृतालयम् ॥शक्र उवाच}
{}


\twolineshloka
{खरयोनिमनुप्राप्तस्तुपभक्षोऽसि दानव}
{इदं ते योनिरसमा शोचस्याहो न शोचसि}


\twolineshloka
{अदृष्टं वत पश्यामि द्विषतां वशमागतम्}
{श्रिया विहीनं मित्रैश्च भ्रष्टैश्वर्यपराक्रमम्}


\twolineshloka
{यत्तद्यानसहस्रैस्त्वं ज्ञातिभिः परिवारितः}
{लोकान्प्रतापयन्सर्वान्यास्यस्मानवितर्कयन्}


\twolineshloka
{त्वन्मुखाश्चैव दैतेय व्यतिष्ठंस्तव शासने}
{अकृष्टपच्या पृथिवी तवैश्वर्ये बभूव ह}


\twolineshloka
{इदं च तेऽद्य व्यसनं शोचस्याहो न शोचसि}
{यदाऽतिष्ठः समुद्रस्य पूर्वकूले विलेलिखन्}


\twolineshloka
{ज्ञातिभ्यो विभजन्वित्तं तदासीत्ते मनः कथम्}
{यत्ते सहस्रसमिता ननुतुर्देवसोषितः}


\twolineshloka
{बहूनि वर्षपूगानि विहारे दीप्यतः श्रिया}
{सर्वाः पुष्करमालिन्यः सर्वाः काञ्चनसप्रभाः}


\twolineshloka
{कथमद्य तदा चैव मनस्ते दानवेश्वर}
{छत्रं तवासीत्सुमहत्सौवर्णं रत्नभूषितम्}


\twolineshloka
{ननृतुस्तत्र गन्धर्वाः षट््सहस्राणि सप्तचा}
{यूपस्तवासीत्सुमहान्यजतः सर्वकाञ्चनः}


\twolineshloka
{यत्राददः सहस्राणि अयुतानां गवां दश}
{अनन्तरं सहस्रेण तदाऽऽसीद्दैत्या का मतिः}


\twolineshloka
{यदा च पृथिवीं सर्वां यजमानोऽनुपर्यगाः}
{शम्याक्षेपेण विधिना तदाऽऽसीत्किंतु ते हृदि}


\threelineshloka
{न ते पश्यामि भृङ्गारं न च्छत्रं व्यजनं न च}
{ब्रह्मदत्तां च ते मालां न पश्याम्यसुराधिप ॥`भीष्म उवाच}
{}


\twolineshloka
{ततः प्रहस्य स बलिर्वासवेन समीरितम्}
{निशम्य मानगम्भीरं सुरराजमथाब्रवीत्}


\twolineshloka
{अहो हि तव बालिश्यमिह देवगणाधिप}
{अयुक्तं देवराजस्य तव कष्टमिदं वचः ॥'}


\twolineshloka
{न त्वं पश्यसि भृङ्गारं न च्छन्नं व्यजनं न च}
{ब्रह्मदत्तां च मे मालां न त्वं द्रक्ष्यसि वासव}


\twolineshloka
{गुहायां निहितानि त्वं मम रत्नानि पृच्छसि}
{यदा मे भविता कालस्तदा त्वं तानि द्रक्ष्यसि}


\twolineshloka
{`न जानीषे भवान्सिद्धिं शुभाङ्गस्वरूपरूपिणीम्}
{कालेन भविता सर्वो नात्र गच्छति वासव ॥'}


\twolineshloka
{न त्वेतदनुरूपं ते यशसो वा कुलस्य च}
{समृद्धार्थोऽसमृद्धार्थं यन्मां कत्थितुमिच्छसि}


\twolineshloka
{न हि दुःखेषु शोचन्ते न प्रहृष्यन्ति चर्द्धिषु}
{कृतप्रज्ञाः ज्ञानतृप्ताः क्षान्ताः सन्तो मनीषिणः}


\twolineshloka
{त्वं तु प्राकृतया बुद्ध्या पुरंदर विकत्थसे}
{यदाऽहमिव भावी स्यास्तदा नैवं वदिष्यसि}


\twolineshloka
{`ऐश्वर्यमदमत्तो मां स त्वं किंचिन्न बुद्ध्यसे}
{राज्याद्विनिपतानेन सोहं न त्वपराजितः ॥'}


\chapter{अध्यायः २३१}
\twolineshloka
{भीष्म उवाच}
{}


\twolineshloka
{पुनेरव तु सं शक्रः अहसन्निदमब्रवीत्}
{निश्वसन्तं यथा गायं प्रत्याहाराय भारत}


\twolineshloka
{यथज्ञानसहस्रेण ज्ञातिभिः परिवारितः}
{लोकान्यतापयन्तर्वान्यास्यस्मानवितर्कयन्}


\twolineshloka
{दृष्ट्वा सुकृष्णां चेमागयस्थामात्मनो बले}
{ज्ञातिमित्रपरित्यक्तः शोचस्याहो न शोचसि}


\threelineshloka
{प्रीतिं प्राप्यातुलां पूर्वं लोकांश्चात्मवशे स्थितान्}
{विनिपातमिमं चाद्य शोचस्याहो न शोचसि ॥बलिरुवाच}
{}


\twolineshloka
{`गर्वं हित्वा तथा मानं देवराज शृणुष्व मे}
{मया च त्वाऽनुसद्भावं पूर्वमाचरितं महत्}


\twolineshloka
{अवश्यकालपर्यायमात्मनः परिवर्तनम्}
{अविदँल्लोकमाहात्म्यं------- ॥'}


\twolineshloka
{अनित्यमुपलक्ष्येह कालपर्यायमात्मनः}
{तस्माच्छक्र न शोचामि सर्वं ह्येवेदमन्तवत्}


\twolineshloka
{अन्तवन्त इमे देहा भूतानां च सुराधिप}
{तेन शक्र न शोचामि नापराधादिदं मम}


\twolineshloka
{जीवितं च शरीरं च जात्या वै सह जायते}
{उभे सह विवर्धेते उभे सह विनश्यतः}


\twolineshloka
{न हीदृशमहंभावमवशः प्राप्य केवलम्}
{यदेवमभिजानामि का व्यथा मे विजानतः}


\twolineshloka
{भूतानां निधनं निष्ठा स्रोतसामिव सागरः}
{नैतत्सम्यग्विजानन्तो नरा मुह्यन्ति वज्रधृत्}


\twolineshloka
{ये त्वेवं नाभिजानन्ति रजोमोहपरायणाः}
{ते कृच्छ्रं प्राप्य सीदन्ति बुद्धिर्येषां प्रणश्यति}


\twolineshloka
{बुद्धिलाभात्तु पुरुषः सर्वं तुदति किल्विषम्}
{विपाप्मा लभते सत्वं सत्वस्थः संप्रसीदति}


\twolineshloka
{ततस्तु ये निवर्तन्ते जायन्ते वा पुनः पुनः}
{कृपणाः परितप्यन्ते तैरर्थैरभिचोदिताः}


\twolineshloka
{अर्थसिद्धिमनर्थं च जीवितं मरणं तथा}
{सुखं दुःखं फलं चैव न द्वेष्मि न च कामये}


\twolineshloka
{हतं हन्ति हतो ह्येव यो नरो हन्ति कंचन}
{उभौ तौ न विजानीतो यश्च हन्ति हतश्च यः}


\twolineshloka
{हत्वा जित्वा च मघवन्यः कश्चित्पुरुषायते}
{अकर्ता ह्येव भवति कर्ता ह्येव करोति तत्}


\twolineshloka
{को हि लोकस्य कुरुते विनाशप्रभवावुभौ}
{कृतं हि तत्कृतेनैव कर्ता तस्यापि चापरः}


\twolineshloka
{पृथिवी ज्योतिराकाशमापो वायुश्च पञ्चमः}
{एतद्योनीनि भूतानि तत्र का परिदेवना}


\twolineshloka
{महाविद्योऽल्पविद्यश्च बलवान्दुर्बलश्च यः}
{दर्शनीयो विरूपश्च सुभगो दुर्भगश्च यः}


\twolineshloka
{सर्वं कालः समादत्ते गम्भीरः स्वेन तेजसा}
{तस्मिन्कालवशं प्राप्ते का व्यथा मे विजानतः}


\twolineshloka
{दग्धमेवानुदहते हतमेवानुहन्यते}
{नश्यते नष्टमेवाग्रे लब्धव्यं लभते नरः}


\twolineshloka
{नास्य द्वीपः कुतः पारो नावारः संप्रदृश्यते}
{नान्तमस्य प्रपश्यामि विधेर्दिव्यस्य चिन्तयन्}


\twolineshloka
{यदि मे पश्यतः कालो भूतानि न विनाशयेत्}
{स्यान्मे हर्षश्च दर्पश्च क्रोधश्चैव शचीपते}


\twolineshloka
{तुषभक्षं तु मां ज्ञात्वा प्रविविक्तजने गृहे}
{बिभ्रतं गार्दभं रूपमागत्य परिगर्हसे}


\twolineshloka
{इच्छन्नहं विकुर्यां हि रूपाणि बहृधाऽऽत्मनः}
{विभीषणानि यानीक्ष्य पलायेथास्त्वमेव मे}


\twolineshloka
{कालः सर्वं समादत्ते कालः सर्वं प्रयच्छति}
{कालेन विहितं सर्वं मा कृथाः शक्र पौरुषम्}


\twolineshloka
{पुरा सर्वं प्रव्यथितं मयि क्रुद्धे पुरंदर}
{`विद्रवन्ति त्वया सार्धं सर्व एव दिबौकसः ॥'}


\twolineshloka
{अवैमि त्वस्य लोकस्य कर्मं शक्र सनातनम्}
{त्वमप्येवमवेक्षस्व माऽऽत्मना विस्मगं गमः}


\threelineshloka
{प्रभवश्च प्रभावश्च नात्मसंस्थः कदाचन}
{कौमारमेव ते चित्तं तथैवाद्य यथा पुरा}
{समवेक्षस्व मघवन्बुद्धिं विन्दस्व नैष्ठिकीम्}


\twolineshloka
{देवा मनुष्याः पितरो गन्धर्वोरगराक्षसाः}
{आसन्सर्वे मम वशे तत्सर्वं वेत्थ वासव}


\twolineshloka
{नमस्तस्यै दिशेऽप्यस्तु यस्यां वैरोचनो बलिः}
{इति मामभ्यपद्यन्त बुद्धिमात्सर्यमोहिताः}


\twolineshloka
{नाहं तदनुशोचामि नात्मभ्रंशं शचीपते}
{एवं मे निश्चिता बुद्धिः शास्तुस्तिष्ठाम्यहं वशे}


\twolineshloka
{दृश्यते हि कुले जातो दर्शनीयः प्रतापवान्}
{दुःखं जीवन्सहामात्यो भवितव्यं हि तत्तथा}


\twolineshloka
{दौष्कुलेयस्तथा मूढो दुर्जातः शक्र दृश्यते}
{सुखं जीवन्सहामात्यो भवितव्यं हि तत्तथा}


\twolineshloka
{कल्याणी रूपसंपन्ना दुर्भगा शक्र दृश्यते}
{अलक्षणा विरूपा च सुभगा दृश्यते परा}


\twolineshloka
{नैतदस्मत्कृतं शक्र नैतच्छक्र त्वया कृतम्}
{यत्तमेवं गतो वज्रिन्यच्चाप्येवं गता वयम्}


\twolineshloka
{न कर्म तव नान्येषां कुतो मम शतक्रतो}
{ऋद्धिर्वाऽप्यथवा नर्द्धिः पर्यायकृतमेव तत्}


\twolineshloka
{पश्यामि त्वां विराजन्तं देवराजमवस्थितम्}
{श्रीमन्तं द्युतिमन्तं च गर्जमानं ममोपरि}


\twolineshloka
{एवं नैव न चेत्कालो मामाक्रम्य स्थितो भवेत्}
{पातयेयमहं त्वाऽद्य सवज्रमपि मुष्टिना}


\twolineshloka
{न तु विक्रमकालोऽयं शान्तिकालोऽयमागतः}
{कालः स्थापयते सर्वं कालः पचति वै तथा}


\twolineshloka
{मां चेदभ्यागतः कालो दानवेश्वरमूर्जितम्}
{गर्जन्तं प्रतपन्तं च कमन्यं नागमिष्यति}


\twolineshloka
{द्वादशानां तु भवतामादित्यानां महात्मनाम्}
{तेजांस्येकेन सर्वेषां देवराज धृतानि मे}


\twolineshloka
{अहमेवोद्वहाम्यापो विसृजामि च वासव}
{तपामि चैव त्रैलोक्यं विद्योताम्यहमेव च}


\twolineshloka
{संरक्षामि विलुम्पामि ददाम्यहमथाददे}
{संयच्छामि नियच्छामि लोकेषु प्रभुरीश्वरः}


\twolineshloka
{तदद्य विनिवृत्तं मे प्रभुत्वममराधिप}
{कालसैन्यावगाढस्य सर्वं न प्रतिभाति मे}


\twolineshloka
{नाहं कर्ता न चैव त्वं नान्यः कर्ता शचीपते}
{पर्यायेण हि भुज्यन्ते लोकाः शक्र यदृच्छया}


\twolineshloka
{मासमासार्धवेश्मानमहोरात्राभिसंवृतम्}
{ऋतुद्वारं वायुमुखमायुर्वेदविदो जनाः}


\twolineshloka
{आहुः सर्वमिदं चिन्त्यं जनाः केचिन्मनीषया}
{`अनित्यपञ्चवर्षाणि षष्ठो दृश्यति देहिनाम् ॥'}


\twolineshloka
{अस्याः पञ्चैव चिन्तायाः पर्येष्यामि च पञ्चधा}
{`ततस्तानि न पश्यामि काले तमपि वृत्रहन् ॥'}


\twolineshloka
{गम्भीरं गहनं ब्रह्म महत्तोयार्णवं यथा}
{अनादिनिधनं चाहुरक्षरं क्षरमेव च}


\twolineshloka
{सत्त्वेषु लिङ्गमाविश्य निर्लिङ्गमपि तत्स्वयम्}
{मन्यन्ते ध्रुवमेवैनं ये जनास्तत्त्वदर्शिनः}


\twolineshloka
{`यमिन्द्रियाणि सर्वाणि नानुपश्यन्ति पञ्चधा}
{तं कालमिति जानीहि यस्य सर्वमिदं वशे ॥'}


\twolineshloka
{भूतानां तु विषर्यासं कुरुते भगवानिति}
{न ह्येतावद्भवेद्ग्राम्यं न यस्मात्प्रभवेत्पुनः}


\twolineshloka
{गतिं हि सर्वभूतानामगत्वा क्व गमिष्यति}
{यो धावता न हातव्यस्तिष्ठन्नपि न हीयते}


\twolineshloka
{तमिन्द्रियाणि सर्वाणि नानुपश्यन्ति पञ्चधा}
{आहुश्चैनं केचिदग्निं केचिदाहुः प्रजापतिम्}


\twolineshloka
{ऋतून्मासार्धमासांश्च दिवसांश्च क्षणांस्तथा}
{पूर्वाह्णमपराह्णं च मध्याह्नमपि चापरे}


\twolineshloka
{मुहूर्तमपि चैवाहुरेकं सन्तमनेकधा}
{तं कालमिति जानीहि यस्य सर्वमिदं वशे}


\twolineshloka
{बहूनीन्द्रसहस्राणि समतीतानि वासव}
{बलवीर्योपपन्नानि यथैव त्वं शचीपते}


\twolineshloka
{त्वामप्यतिबलं शक्र देवराजं बलोत्कटम्}
{प्राप्ते काले महावीर्यः कालः संशमयिष्यति}


\twolineshloka
{य इदं सर्वमादत्ते तस्माच्छक्र स्थिरो भव}
{मया त्वया च पूर्वैश्च न स शक्योऽतिवर्तितुम्}


\twolineshloka
{यामेतां प्राप्य जानीषे राज्यश्रियमनुत्तमाम्}
{स्थिता मयीति तन्मिथ्या नैषा ह्येकत्र तिष्ठति}


\twolineshloka
{स्थिता हीन्द्रसहस्रेषु त्वद्विशिष्टतमेष्वियम्}
{मां च लोला परित्यज्य त्वामगाद्विबुधाधिप}


\threelineshloka
{मैवं शक्र पुनः कार्षीः शान्तो भवितुमर्हसि}
{त्वामप्येवंविधं ज्ञात्वा क्षिप्रमन्यं गमिष्यति}
{`कालेन चोदिता शक्र मा ते गर्वः शतक्रतो}


\twolineshloka
{क्षमस्व कालयोगं तमागतं विद्धि देवप}
{निर्लज्जश्चैव कस्मात्त्वं देवराज विकत्थसे}


\twolineshloka
{सर्वासुराणामधिपः सर्वदेवभयंकरः}
{जितवान्ब्रह्मणो लोकं को विद्यादागतं गतिम् ॥'}


\chapter{अध्यायः २३२}
\twolineshloka
{भीष्म उवाच}
{}


\twolineshloka
{शतक्रतुरथापश्यद्बलेर्दीप्तां महात्मनः}
{स्वरूपिणीं शरीराद्धि निष्क्रामन्तीं तदा श्रियम्}


\twolineshloka
{तां दृष्ट्वा प्रभया दीप्तां भगवान्पाककशासनः}
{विस्मयोत्फुल्लनयनो बलिं पप्रच्छ वासवः}


\threelineshloka
{बले केयमपक्रान्ता रोचमाना शिखण्डिनी}
{त्वत्तः स्थिता सकेयूरा दीप्यमाना स्वतेजसा ॥बलिरुवाच}
{}


\threelineshloka
{न हीमामासुरीं वेद्मि न दैवीं च न मानुषीम्}
{त्वमेनां पृच्छ वा मा वा यथेष्टं कुरु वासव ॥शक्र उवाच}
{}


\twolineshloka
{का त्वं बलेरपक्रान्ता रोचमाना शिखण्डिनी}
{अजानतो ममाचक्ष्व नामधेयं शुचिस्मिते}


\threelineshloka
{का त्वं तिष्ठसि मामेवं दीप्यमाना स्वतेजसा}
{हित्वा दैत्यवरं सुभ्रु तन्ममाचक्ष्व पृच्छतः ॥श्रीरुवाच}
{}


\twolineshloka
{न मां विरोचनो वेद नायं वैरोचनो बलिः}
{आहुर्मां दुःसहेत्येवं विधित्सेति च मां विदुः}


\threelineshloka
{भूतिर्लक्ष्मीति मामाहुः श्रीरित्येव च वासव}
{त्वं मां शक्र न जानीषे सर्वे देवा न मां विदुः ॥शक्र उवाच}
{}


\threelineshloka
{किमिदं त्वं मम कृते उताहो बलिनः कृते}
{दुःसहे विजहास्येनं चिरसंवासिनी सती ॥श्रीरुवाच}
{}


\threelineshloka
{नो धाता न विधाता मां विदधाति कथंचन}
{कालस्तु शक्र पर्यागान्मैवं शक्रावमन्यथाः ॥शक्र उवाच}
{}


\threelineshloka
{कथं त्वया बलिस्त्यक्तः किमर्थं वा शिखण्डिनि}
{कथं च मां न जह्यास्त्वं तन्मे ब्रूहि शुचिस्मिते ॥श्रीरुवाच}
{}


\twolineshloka
{सत्ये स्थिताऽस्मि दाने च व्रते तपसि चैव हि}
{पराक्रमे च धर्मे च पराचीनस्ततो बलिः}


\twolineshloka
{ब्रह्मण्योऽयं पुरा भूत्वा सत्यवादी जितेन्द्रियः}
{अभ्यसूयन्ब्राह्मणान्वै उच्छिष्टश्चास्पृशद्धृतम्}


\twolineshloka
{यज्ञशीलः सदा भूत्वा मामेव यजते स्वयम्}
{ततः प्रहाय मूढात्मा कालेनोपनिपीडितः}


\threelineshloka
{अपक्रान्ता ततः शक्र त्वयि वत्स्यामि वासव}
{अप्रमत्तेन धार्याऽस्मि तपसा विक्रमेण च ॥शक्र उवाच}
{}


\threelineshloka
{कोऽस्ति देवमनुष्येषु सर्वभूतेषु वा पुमान्}
{यस्त्वामेको विषहितुं शक्नुयात्कमलालये ॥श्रीरुवाच}
{}


\threelineshloka
{नैव देवो न गन्धर्वो नासुरो नं च न्राक्षसः}
{यो मामेको विषहितुं शक्तः कश्चित्पुरंदर ॥शक्र उवाच}
{}


\threelineshloka
{तिष्ठेथा मयि नित्यं त्वं यथा तद्ब्रूहि मे शुभे}
{तत्करिष्यामि ते वाक्यमृतं तद्वक्तुमर्हसि ॥शक्र उवाच}
{}


\threelineshloka
{स्थास्यामि नित्यं देवेन्द्र यथा त्वयि निबोध तत्}
{विधिना वेददृष्टेन चतुर्धा विभजस्व माम् ॥शक्र उवाच}
{}


\twolineshloka
{अहं वै त्वां निधास्यामि यथाशक्ति यथाबलम्}
{न तु मेऽतिक्रमः स्याद्वै सदा लक्ष्मि तवान्तिके}


\threelineshloka
{भूमिरेव मनुष्येषु धारिणी भूतभाविनी}
{सा ते पादं तितिक्षेत समर्था हीति मे मतिः ॥श्रीरुवाच}
{}


\threelineshloka
{एष मे निहितः पादो योऽयं भूमौ प्रतिष्ठितः}
{द्वितीयं शक्र पादं मे तस्मात्सुनिहितं कुरु ॥शक्र उवाच}
{}


\threelineshloka
{आप एव मनुष्येषु द्रवन्त्यः परिधारणे}
{तास्ते पादं तितिक्षन्तामलमापस्तितिक्षितुम् ॥श्रीरुवाच}
{}


\threelineshloka
{एष मे निहितः पादो योऽयमप्सु प्रतिष्ठितः}
{तृतीयं शक्र पादं मे तस्मात्सुनिहितं कुरु ॥शक्र उवाच}
{}


\threelineshloka
{यस्मिन्वेदाश्च यज्ञाश्च यस्मिन्देवाः प्रतिष्ठिताः}
{तृतीयं पादमग्निस्ते सुधृतं धारयिष्यति ॥श्रीरुवाच}
{}


\threelineshloka
{एष मे निहितः पादो योऽयमग्नौ प्रतिष्ठितः}
{चतुर्थं शक्र पादं मे तस्मात्सुनिहितं कुरु ॥शक्र उवाच}
{}


\threelineshloka
{ये वै सन्तो मनुष्येषु ब्रह्मण्याः सत्यवादिनः}
{तेते पादं तितिक्षन्तामलं सन्तस्तितिक्षितुम् ॥श्रीरुवाच}
{}


\threelineshloka
{एष मे निहितः पादो योऽयं सत्सु प्रतिष्ठितः}
{एवं हि निहितां शक्र भूतेषु परिधत्स्व माम् ॥शक्र उवाच}
{}


\twolineshloka
{`भूमिशुद्धिं ततः कृत्वा अद्भिः संतर्पयन्ति ये}
{भूतानि च यजन्त्यग्नौ तेषां त्वमनपायिनी}


\twolineshloka
{ये क्रियाभिः सुरक्ताभिर्हेतुयुक्ताः समाहिताः}
{ज्ञानवन्तो विवत्सायां लब्धा माद्यन्ति योगिनः ॥'}


\threelineshloka
{भूतानामिह यो वै त्वां मया विनिहितां सतीम्}
{उपहन्यात्स मे द्वेष्यस्तथा शृण्वन्तु मे वचः ॥`भीष्म उवाच}
{}


\twolineshloka
{तथेति चोक्त्वा सा भ्रष्टा सर्वलोकनमस्कृता}
{वासवं पालयामास सा देवी कमलालया ॥'}


% Check verse!
ततस्त्यक्तः श्रिया राजा दैत्यानां बलिरब्रवीत्
\twolineshloka
{यावत्पुरस्तात्प्रतपेत्तावद्वै दक्षिणां दिशम्}
{पश्चिमां तावदेवापि तथोदीचीं दिवाकरः}


\twolineshloka
{तथा मध्यंदिने सूर्यो नास्तमेति यदा तदा}
{पुनर्देवासुरं युद्धं भावि जेताऽस्मि वस्तदा}


\threelineshloka
{सर्वलोकान्यदाऽऽदित्यो मध्यस्थस्तापयिष्यति}
{तदा देवासुरे युद्धे जेताऽहं त्वां शतक्रतो ॥शक्र उवाच}
{}


\twolineshloka
{ब्रह्मणाऽस्मि समादिष्टो न हन्तव्यो भवानिति}
{तेन तेऽहं बले वज्रं न विमुञ्जामि मूर्धनि}


% Check verse!
यथेष्टं गच्छ दैत्येन्द्र स्वस्ति तेऽस्तु महासुर
\twolineshloka
{आदित्यो नैव तपिता कदाचिन्मध्यतः स्थितः}
{स्थापितो ह्यस्य समयः पूर्वमेव स्वयंभुवा}


\fourlineindentedshloka
{अजस्रं परियात्येष सत्येनावतपन्प्रजाः}
{अयनं तस्य षण्मासा उत्तरं दक्षिणं तथा}
{येन संयाति लोकेषु शीतोष्णे विसृजन्रविः ॥भीष्म उवाच}
{}


\twolineshloka
{एवमुक्तस्तु दैत्येन्द्रो बलिरिन्द्रेण भारत}
{जगाम दक्षिणामाशामुदीचीं तु पुरंदरः}


\twolineshloka
{इत्येतद्बलिना गीतमनहंकारसंज्ञितम्}
{वाक्यं श्रुत्वा सहस्राक्षः स्वमेवारुरुहे तदा}


\chapter{अध्यायः २३३}
\twolineshloka
{`युधिष्ठिर उवाच}
{}


\threelineshloka
{व्यसनेषु निमग्नस्य किं श्रेयस्तद्ब्रवीहि मे}
{भूय एव महाबाहो स्थित्यर्थं तं ब्रवीहि मे ॥'भीष्म उवाच}
{}


\twolineshloka
{अत्रैवोदाहन्तीममितिहासं पुरातनम्}
{शतक्रतोश्च संवादं नमुचेश्च युधिष्ठिर}


\twolineshloka
{श्रिया विहीनमासीनमक्षोभ्यमिव सागरम्}
{भवाभवज्ञं भूतानामित्युवाच पुरंदरः}


\threelineshloka
{बद्धः पाशैश्च्युतः स्थानाद्द्विपतां वशमागतः}
{श्रिया विहीनो नमुचे शोचस्याहो न शोचसि ॥नमुचिरुवाच}
{}


\threelineshloka
{अनवाप्यं च शोकेन शरीरं चोपशुष्यति}
{अमित्राश्च प्रहृष्यन्ति शोके नास्ति सहायता}
{तस्माच्छक्र न शोचामि सर्वं ह्येवेदमन्तवत्}


\twolineshloka
{संतापाद्धश्यते रूपं संतापाद्धश्यते श्रियः}
{संतापाद्धश्यते चायुर्धर्मश्चैव सुरेश्वर}


\twolineshloka
{विनीय खलु तद्दुःखमागतं वै मनस्सुखम्}
{ध्यातव्यं मनसा हृद्यं कल्याणं संविजानता}


\twolineshloka
{यथायथा हि पुरुषः कल्याणे कुरुते मनः}
{तथैवास्य प्रसिध्यन्ति सर्वार्था नात्र संशयः}


\twolineshloka
{एकः शास्ता न द्वितीयोऽस्ति शास्तागर्भे शयानं पुरुषं शास्ति शास्ता}
{तेनानुयुक्तः प्रवणादिवोदकंयथा नियुक्तोऽस्मि तथा भवामि}


\twolineshloka
{भावाभावावजिनानन्गरीयोजानामि श्रेयो न तु तत्करोमि}
{आशासु हर्म्यासु हृदासु कुर्वन्यथा नियुक्तोऽस्मि तथा वहामि}


\twolineshloka
{यथायथाऽस्म प्राप्तव्यं प्राप्नोत्येव तथातथा}
{भवितव्यं यथा यच्च भवत्येव तथातथा}


\twolineshloka
{यत्रयत्रैव संयुक्तो धात्रा गर्भे पुनः पुनः}
{तत्रतत्रैव वसति न यत्र स्वयमिच्छति}


\twolineshloka
{भावो योऽयमनुप्राप्तो भवितव्यमिदं मम}
{इति यस्य सदा भावो न स शोचेत्कदाचन}


\twolineshloka
{पर्यायैर्हन्यमानानामभिषङ्गो न विद्यते}
{दुःखमेतत्तु यद्द्वेष्टा कर्ताऽहमिति मन्यते}


\twolineshloka
{ऋषींश्च देवांश्च महासुरांश्चत्रैविद्यवृद्धांश्च वने मुनींश्च}
{का नापदो नोपनमन्ति लोकेपरावरज्ञास्तु न संभ्रमन्ति}


\twolineshloka
{न पण्डितः क्रुद्ध्यति नाभिषज्यतेन चापि संसीदति न प्रहृष्यति}
{न चार्थकृच्छ्रव्यसनेषु शोचतेस्थितः प्रकृत्या हिमवानिवाचलः}


\twolineshloka
{यमर्थसिद्धिः परमा न हर्षयेत्तथैव काले व्यसनं न मोहयेत्}
{सुखं च दुःखं च तथैव मध्यमंनिषेवते यः स धुरधरो नरः}


\twolineshloka
{यांयामवस्थां पुरुषोऽधिगच्छेत्तस्यां रमेतापरितप्यमानः}
{एवं प्रवृद्धं प्रणुदन्मनोजंसंतापनीलं सकलं शरीरात्}


\twolineshloka
{न तत्सदः सत्परिषत्सभा चसा प्राप्य यां न कुरुते सदा भयम्}
{धर्मतत्त्वमवगाह्य बुद्धिमान्योऽभ्युपैति स धुरंधरः पुमान्}


\twolineshloka
{प्राज्ञस्य कर्माणि दुरन्वयानिन वै प्राज्ञो मुह्यति मोहकाले}
{स्थानाच्च्युतश्चेन्न मुमोह गौतमस्तावत्कृच्छ्रामापदं प्राप्य वृद्धः}


\threelineshloka
{न मन्त्रबलवीर्येण प्रज्ञया पौरुषेण च}
{[न शीलेन न वृत्तेन तथा नैवार्थसंपदा}
{]अलभ्यं लभते मर्त्यस्तत्र का परिदेवना}


\twolineshloka
{यदेवमभिजातस्य धातारो विदधुः पुरा}
{तदेवानुभविष्यामि किं मे मृत्युः करिष्यति}


\twolineshloka
{लब्धव्यान्येव लभते गन्तव्यान्येव गच्छति}
{प्राप्तव्यान्येव चाप्नोति दुःखानि च सुखानि च}


\twolineshloka
{एतद्विदित्वा कार्त्स्न्येन यो न मुह्यति मानवः}
{कुशली सर्वदुःखेषु स वै सर्वधनी नरः}


\chapter{अध्यायः २३४}
\twolineshloka
{युधिष्ठिर उवाच}
{}


\twolineshloka
{मग्नस्य व्यसने कृच्छ्रे किं श्रेयः पुरुषस्य हि}
{बन्धुनाशे महीपाल राज्यनाशेऽथवा पुनः}


\threelineshloka
{त्वं हि नः परमो वक्ता लोकेऽस्मिन्भरतर्षभ}
{एतद्भवन्तं पृच्छामि तन्मे त्वं वक्तुमर्हसि ॥भीष्म उवाच}
{}


\threelineshloka
{पुत्रदारैः सुखैश्चैव वियक्तस्य धनेन च}
{मग्नस्य व्यसने कृच्छ्रे धृतिः श्रेयस्करी नृप}
{धैर्येण युक्तस्य सतः शरीरं न विशीर्यते}


\twolineshloka
{[विशोकता सुखं धत्ते धत्ते चारोग्यमुत्तमम्}
{]आरोग्याच्च शरीरस्य स पुनर्विन्दते श्रियम्}


\twolineshloka
{यश्च प्राज्ञो नरस्तात सात्विकीं वृत्तिमास्थितः}
{तस्यैश्वर्यं च धैर्यं च व्यवसायश्च कर्मसु}


\twolineshloka
{अत्रैवोदाहरन्तीममितिहासं पुरातनम्}
{बलिवासवसंवादं पुनरेव युधिष्ठिर}


\twolineshloka
{वृत्ते देवासुरे युद्धे दैत्यदानवसंक्षये}
{विष्णुक्रान्तेषु लोकेषु देवराजे शतक्रतौ}


\twolineshloka
{इज्यमानेषु देवेषु चातुर्वर्ण्ये व्यवस्थिते}
{समृध्यमाने त्रैलोक्ये प्रीतियुक्ते स्वयंभुवि}


\twolineshloka
{रुद्रैर्वसुभिरादित्यैरश्विभ्यामपि चर्षिभिः}
{गन्धर्वैर्भुजगेन्द्रैश्च सिद्धैश्चान्यैर्वृतः प्रभुः}


\twolineshloka
{चतुर्दन्तं सुदान्तं च वारणेन्द्रं श्रिया वृतम्}
{आरुह्यैरावणं शक्रस्त्रैलोक्यमनुसंययौ}


\twolineshloka
{स कदाचित्समुद्रान्ते कस्मिंश्चिद्गिरिगह्वरे}
{बलिं वैरोचनिं वज्री ददर्शोपससर्प च}


\twolineshloka
{तमैरावतमूर्धस्थं प्रेक्ष्य देवगणैर्वृतम्}
{सुरेन्द्रमिन्द्रं दैत्येन्द्रो न शुशोच न विव्यथे}


\twolineshloka
{दृष्ट्वा तमविकारस्थं तिष्ठन्तं निर्भयं बलिम्}
{अधिरूढो द्विपश्रेष्ठमित्युवाच शतक्रतुः}


\twolineshloka
{दैत्य न व्यथसे शौर्यादथवा वृद्धसेवया}
{तपसा भावितत्वाद्वा सर्वथैतत्सुदुष्करम्}


\twolineshloka
{शत्रुभिर्वशमानीतो हीनः स्थानादनुत्तमात्}
{वैरोचने किमाश्रित्य शोचितव्ये न शोचसि}


\twolineshloka
{श्रैष्ट्यं प्राप्य स्वजातीनां भुक्त्वा भोगाननुत्तमान्}
{हृतस्वरत्नराज्यस्त्वं ब्रूहि कस्मान्न शोचसि}


\twolineshloka
{ईश्वरो हि पुरा भूत्वा पितृपैतामहे पदे}
{तत्त्वमद्य हृतं दृष्ट्वा सपत्नैः किं न शोचसि}


\twolineshloka
{बद्धश्च वारुणैः पाशैर्वज्रेण च समाहतः}
{हृतदारो हृतधनो ब्रूहि कस्मान्न शोचसि}


\twolineshloka
{नष्टश्रीर्विभवभ्रष्टो यन्न शोचसि दुष्करम्}
{त्रैलोक्यराज्यनाशे हि कोऽन्यो जीवितुमुत्सहेत्}


\twolineshloka
{एतच्चान्यच्च परुषं ब्रुवन्तं परिभूय तम्}
{श्रुत्वा दुःखमसंभ्रान्तो बलिर्वैरोचनोऽब्रवीत्}


\twolineshloka
{निगृहीते मयि भृशं शक्र किं कत्थितेन ते}
{वज्रमुद्यम्य तिष्ठन्तं पश्यामि त्वां पुरंदर}


\twolineshloka
{अशक्तः पूर्वमासीस्त्वं कथंचिच्छक्ततां गतः}
{कस्त्वदन्य इमां वाचं सुक्रूरां वक्तुमर्हति}


\twolineshloka
{यस्तु शत्रोर्वशस्थस्य शक्तोऽपि कुरुते दयाम्}
{हस्तप्राप्तस्य वीरस्य तं चैव पुरुषं विदुः}


\twolineshloka
{अनिश्चयो हि युद्धेषु द्वयोर्विवदमानयोः}
{एकः प्राप्नोति विजयमेकश्चैव पराजयम्}


\twolineshloka
{मा च ते भूत्स्वभावोऽयं मयि दानवपुङ्गवे}
{ईश्वरः सर्वभूतानां विक्रमेण जितो बलात्}


\twolineshloka
{नैतदस्मत्कृतं शक्र नैतच्छक्र कृतं त्वया}
{यत्त्वमेवं गतो वज्रिन्यद्वाऽऽप्येवं गता वयम्}


\twolineshloka
{अहमासं यथाऽद्य त्वं भविता त्वं यथा वयम्}
{मावमंस्था मया कर्म दुष्कृतं कृतमित्युत}


\twolineshloka
{सुखदुःखे हि पुरुषः पर्यायेणाधिगच्छति}
{पर्यायेणासि शक्रत्वं प्राप्तः शक्र न कर्मणा}


\twolineshloka
{कालः काले नयति मां त्वां च कालो नयत्ययम्}
{तेनाहं त्वं यथा नाद्य त्वं चापि न यथा वयम्}


\twolineshloka
{न मातृपितृशुश्रूषा न च दैवतपूजनम्}
{नान्यो गुणसमाचारः पुरुषस्य सुखावहः}


\twolineshloka
{न विद्या न तपो दानं न मित्राणि न बान्धवाः}
{शक्नुवन्ति परित्रातुं नरं कालेन पीडितम्}


\twolineshloka
{नागामिनमनर्थं हि प्रतिघातशतैरपि}
{शक्नुवन्ति प्रतिव्योढुमृते बुद्धिबलान्नराः}


\twolineshloka
{पर्यायैर्हन्यमानानां परित्राता न विद्यते}
{इदं तु दुःखं यच्छक्र कर्ताऽहमिति मन्यसे}


\twolineshloka
{यदि कर्ता भवेत्कर्ता न क्रियेत कदाचन}
{यस्मात्तु क्रियते कर्ता तस्मात्कर्ताऽप्यनीश्वरः}


\twolineshloka
{कालेन त्वाऽहमजयं कालेनाहं जितस्त्वया}
{गन्ता गतिमतां कालः कालः कलयति प्रजाः}


\twolineshloka
{इन्द्र प्राकृतया बुद्ध्या प्रलपन्नावबुद्ध्यसे}
{केचित्त्वां बहुमन्यन्ते श्रैष्ठ्यं प्राप्तं स्वकर्मणा}


\twolineshloka
{कथमस्मद्विधो नाम जानँल्लोकप्रवृत्तयः}
{कालेनाभ्याहतः शोचेन्मुह्येद्वाऽप्यथविभ्रमेत्}


\twolineshloka
{कथं कालपरीतस्य मम वा मद्विधस्य वा}
{बुद्धिर्व्यसनमासाद्य भिन्ना नौरिव सीदति}


\twolineshloka
{अहं च त्वं च ये चान्ये भविष्यन्ति सुराधिपाः}
{ते सर्वे शक्र यास्यन्ति मार्गमिन्द्रशतैर्गतम्}


\twolineshloka
{त्वामप्येवं सुदुर्धर्षं ज्वलन्तं परया श्रिया}
{काले परिणते कालः कलयिष्यति मामिव}


\twolineshloka
{बहूनीन्द्रसहस्राणि दैवतानां युगे युगे}
{अभ्यतीतानि कालेन कालो हि दुरतिक्रमः}


\twolineshloka
{इदं तु लब्ध्वा संस्थानमात्मानं बहु मन्यसे}
{सर्वभूतभवं देवं ब्रह्माणमिव शाश्वतम्}


\twolineshloka
{न चेदमचलं स्थानमनन्तं वाऽपि कस्यचित्}
{त्वं तु बालिशया बुद्ध्या ममेदमिति मन्यसे}


\twolineshloka
{अविश्वस्ते विश्वसिषि मन्यसे वाऽध्रुवे ध्रुवम्}
{नित्यं कालपरीतात्मा भवत्येवं सुरेश्वर}


\twolineshloka
{ममेयमिति मोहात्त्वं राजश्रियमभीप्ससि}
{नेयं तव न चास्माकं न चान्येषां स्थिरा मता}


\twolineshloka
{अतिक्रम्य बहूनन्यांस्त्वयि तावदियं गता}
{कंचित्कालमियं स्थित्वा त्वयि वासव चञ्चला}


\twolineshloka
{गौर्निवासमिवोत्सृज्य पुनरन्यं गमिष्यति}
{सुरेशा ये ह्यतिक्रान्तास्तान्न संख्यातुमुत्सहे}


\twolineshloka
{त्वत्तो बहुतराश्चान्ये भविष्यन्ति पुरंदर}
{सवृक्षौषधिगुल्मेयं ससरित्पर्वताकरा}


\twolineshloka
{तानिदानीं न पश्यामि यैर्भुक्तेयं पुरा मही}
{पृथुरैलो मयो भीमो नरकः शम्बरस्तथा}


% Check verse!
अश्वग्रीवः पुलोमा च स्वर्भानुरमितप्रभः]प्रह्लादो नमुचिर्दक्षो विप्रचित्तिर्विरोचनः
\twolineshloka
{ह्रीनिषेवः सुहोत्रश्च भूरिहा पुष्पवान्वृषः}
{सत्येषुर्ऋषभो बाहुः कपिलाश्वो निरूपकः}


\twolineshloka
{बाणः कार्तस्वरो वह्निर्विश्वदंष्ट्रोऽथ नैर्ऋतिः}
{संकोचोऽथ वरीताक्षो वराहाश्वो रुचिप्रभः}


\twolineshloka
{विश्वजित्प्रतिरूपश्च वृषाण्डो विष्करो मधुः}
{हिरण्यकशिपुश्चैव कैटभश्चैव दानवः}


\twolineshloka
{दैतेया दानवाश्चैव सर्वे ते नैर्ऋतैः सह}
{एते चान्ये च बहवः पूर्वे पूर्वतराश्च ये}


\twolineshloka
{दैत्येन्द्रा दानवेन्द्राश्च यांश्चान्याननुशुश्रुम्}
{बहवः पूर्वदैत्येन्द्राः संत्यज्य पृथिवीं गताः}


\twolineshloka
{कालेनाभ्याहताः सर्वे कालो हि बलवत्तरः}
{सर्वैः क्रतुशतैरिष्टं न त्वमेकः शतक्रतुः}


\twolineshloka
{सर्वे धर्मपराश्चासन्सर्वे सततसत्रिणः}
{अन्तरिक्षचराः सर्वे सर्वेऽभिमुखयोधिनः}


\threelineshloka
{सर्वे संहननोपेताः सर्वे परिघबाहवः}
{सर्वे मायाशतधराः सर्वे ते कामरूपिणः}
{सर्वे समरमासाद्य न श्रूयन्ते पराजिताः}


\twolineshloka
{सर्वे सत्यव्रतपराः सर्वे कामविहारिणः}
{सर्वे वेदव्रतपराः सर्वे चैव बहुश्रुताः}


\twolineshloka
{सर्वे संमतमैश्वर्यमीश्वराः प्रतिपेदिरे}
{न चैश्वर्यमदस्तेषां भूतपूर्वो महात्मनाम्}


\twolineshloka
{सर्वे यथार्हदातारः सर्वे विगतमत्सराः}
{सर्वे सर्वेषु भूतेषु यथावत्प्रतिपेदरे}


\twolineshloka
{सर्वे दाक्षायणीपुत्राः प्राजपत्या महाबलाः}
{ज्वलन्तः प्रतपन्तश्च कालेन प्रतिसंहृताः}


\twolineshloka
{त्वं चैवेमां यदा भुक्त्वा पृथिवीं त्यक्ष्यसे पुनः}
{न शक्ष्यसि तदा शक्र नियन्तुं शोकमात्मनः}


\twolineshloka
{मुञ्चेच्छां कामभोगेषु मुञ्जेमं श्रीभवं मदम्}
{एवं स्वराज्यनाशे त्वं शोकं संप्रसहिष्यसि}


\twolineshloka
{शोककाले शुचो मा त्वं हर्षकाले च मा हृषः}
{अतीतानागतं हित्वा प्रत्युत्पन्नेन वर्तय}


\twolineshloka
{मां चेदभ्यागतः कालः सदा युक्तमतन्द्रितः}
{क्षमस्व न चिरादिन्द्र त्वामप्युपगमिष्यति}


\twolineshloka
{त्रासयन्निव देवेन्द्र वाग्भिस्तक्षसि मामिह}
{संयते मयि ननं त्वमात्मानं बहु मन्यसे}


\twolineshloka
{कालः प्रथममायान्मां पञ्चात्त्वामनुधावति}
{तेन गर्जसि देवेन्द्र पूर्वं कालहते मयि}


\twolineshloka
{को हि स्थातुमलं लोके मम क्रुद्धस्य संयुगे}
{कालस्तु बलवान्प्राप्तस्तेन तिष्ठसि वासव}


\twolineshloka
{यत्तद्वर्षसहस्रान्तं तूर्णं भवितुमर्हति}
{यथा मे सर्वगात्राणि न सुस्थानि महौजसः}


\twolineshloka
{अहमैन्द्राच्च्युतः स्थानात्त्वमिन्द्रः प्रकृतो दिवि}
{सुचित्रे जीवलोकेऽस्मिन्नुपास्यः कालपर्ययः}


\twolineshloka
{किं हि कृत्वा स्वमिन्द्रोऽद्य किं वा कृत्वा वयं च्युताः}
{कालः कर्ता विकर्ता च सर्वमन्यदकारणम्}


\twolineshloka
{नाशं विनाशमैश्वर्यं सुखदुःखे भवाभवौ}
{विद्वान्प्राप्यैवमत्यर्थं न प्रहृष्येन्न च व्यथेत्}


\twolineshloka
{त्वमेव हीन्द्र वेत्थास्मान्वेदाहं त्वां च वासव}
{किं कत्थसे मां किंच त्वं कालेन निरपत्रयः}


\twolineshloka
{त्वमेव हि पुरा वेत्थ यत्तदा पौरुषं मम}
{समरेषु च विक्रान्तं पर्याप्तं तन्निदर्शनम्}


\twolineshloka
{आदित्याश्चैव रुद्राश्च साध्याश्च वसुभिः सह}
{मया विनिर्जिताः पूर्वं मरुतश्च शचीपते}


\twolineshloka
{त्वमेव शक्र जानासि देवासुरसमागमे}
{समेता विबुधा भग्नास्तरसा समरे मया}


\twolineshloka
{पर्वताश्चासकृत्क्षिप्ताः सवनाः सवनौकसः}
{सशृङ्गशिखरा भग्नाः समरे मूर्ध्निं ते मया}


\twolineshloka
{किं नु शक्यं मया कर्तुं कालो हि दुरतिक्रमः}
{न हि त्वां नोत्सहे हन्तुं सवज्रमपि मुष्टिना}


\twolineshloka
{न तु विक्रमकालोऽयं क्षमाकालोऽयमागतः}
{तेन त्वां मर्षये शक्र दुर्मर्षणतरस्त्वया}


\twolineshloka
{तं मां परिणते काले परीतं कालवह्निना}
{नियतं कालपाशेन बद्धं शक्र विकत्थसे}


\twolineshloka
{अयं स पुरुषः श्यामो लोकस्य दुरतिक्रमः}
{बद्ध्वा तिष्ठति मां रौद्रः पशुं रशनया यथा}


\twolineshloka
{लाभालाभौ सुखं दुःखं कामक्रोधौ भवाभवौ}
{वधो बन्धप्रमोक्षश्च सर्वं कालेन लभ्यते}


\twolineshloka
{नाहं कर्ता न कर्ता त्वं कर्ता यस्तु सदा प्रभुः}
{सोयं पचति कालो मां वृक्षे फलमिवागतम्}


\twolineshloka
{यान्येव पुरुषः कुर्वन्सुखैः कालेन युज्यते}
{पुनस्तान्येव कुर्वाणो दुःखैः कालेन युज्यते}


\twolineshloka
{न च कालेन कालज्ञः स्पृष्टः शोचितुमर्हति}
{तेन शक्र न शोचामि नास्ति शोके सहायता}


\threelineshloka
{यदा हि शोचतः शोको व्यसनं नाषकर्षति}
{सामर्थ्यं शोचतो नास्तीत्यतोऽहं नाद्य शोचिमि ॥भीष्म उवाच}
{}


\twolineshloka
{एवमुक्तः सहस्राक्षो भगवान्पाकशासनः}
{प्रतिसंहृत्य संरम्भमित्युवाच शतक्रतुः}


\twolineshloka
{सवज्रमुद्यतं बाहुं दृष्ट्वा पाशांश्च वारुणान्}
{कस्येह न व्यथेद्बुद्धिर्मृत्योरपि जिघांसतः}


\twolineshloka
{सा ते न व्यथते बुद्धिरचला तत्त्वदर्शिनी}
{व्रुवन्न व्यथसेऽद्य त्वं धैर्यात्सत्यपराक्रम}


\twolineshloka
{को हि विश्वासमर्थेषु शरीरे वा शरीरभृत्}
{कर्तुमुत्सहते लोके दृष्ट्वा संप्रस्थितं जगत्}


\twolineshloka
{अहमप्येवमेवैनं लोकं जानाम्यशाश्वतम्}
{कालाग्नावाहितं घोरे गुह्ये सततगेऽक्षरे}


\twolineshloka
{न चात्र परिहारोऽस्ति कालस्पृष्टस्य कस्यचित्}
{सूक्ष्माणां महतां चैव भूतानां परिपच्यताम्}


\twolineshloka
{अनीशस्याप्रमत्तस्य भूतानि पचतः सदा}
{अनिवृत्तस्य कालस्य क्षयं प्राप्तो न मुच्यते}


\twolineshloka
{अप्रमत्तः प्रमत्तेषु कालो जागर्ति देहिषु}
{प्रयत्नेनाप्यपक्रान्तो दृष्टपूर्वो न केनचित्}


\twolineshloka
{पुराणः शाश्वतो धर्मः सर्वप्राणभृतां समः}
{कालो न परिहार्यश्च न चास्यास्ति व्यतिक्रमः}


\twolineshloka
{अहोरात्रांश्च मासांश्च क्षणान्काष्ठा लवान्कलाः}
{संपीडयति यः कालो वृद्धिं वार्धुपिको यथा}


\twolineshloka
{इदमद्य करिष्यामि श्वः कर्ताऽस्मीति वादिनम्}
{कालो हरति संप्राप्तो नदीवेग इव द्रुमम्}


\twolineshloka
{इदानीं तावदेवासौ मया दृष्टः कथं मृतः}
{इति कालेन ह्रियतां प्रलापः श्रूयते नृणाम्}


\twolineshloka
{नश्यन्त्यर्थास्तथा भोगाः स्थानमैश्वर्यमेव च}
{जीवितं जीवलोकस्य कालेनागम्य नीयते}


\twolineshloka
{उच्छ्राया विनिपातान्ता भावोऽभावः स एव च}
{अनित्यमध्रुवं सर्वं व्यवसायो हि दुष्करः}


\twolineshloka
{सा ते न व्यथते बुद्धिरचला तत्त्वदर्शिनी}
{अहमासं पुरा चेति मनसाऽपि न बुद्ध्यते}


\twolineshloka
{कालेनाक्रम्य लोकेऽस्मिन्पच्यमाने बलीयसा}
{अज्येष्ठमकनिष्ठं च क्षिप्यमाणो न बुद्ध्यते}


\twolineshloka
{ईर्ष्याभिमानलोभेषु कामक्रोधभयेषु च}
{स्पृहामोहाभिमानेषु लोकः सक्तो विमुह्यति}


\twolineshloka
{भवांस्तु भावतत्त्वज्ञो विद्वाञ्ज्ञानतपोन्वितः}
{कालं पश्यति सुव्यक्तं पाणावामलकं यथा}


\twolineshloka
{कालचारित्रतत्त्वज्ञः सर्वशास्त्रविशारदः}
{वैरोचते कृतार्थोऽसि स्पृहणीयो विजानताम्}


\twolineshloka
{सर्वलोको ह्ययं मन्ये बुद्ध्या परिगतस्त्वया}
{विहरन्सर्वतोमुक्तो न क्वचिच्च विषीदसि}


\twolineshloka
{रजश्च हि तमश्च त्वां स्पृशते न जितेन्द्रियम्}
{निष्प्रीतिं नष्टसंतापमात्मानं त्वमुपाससे}


\twolineshloka
{सुहृदं सर्वभूतानां निर्वैरं शान्तमानसम्}
{दृष्ट्वा त्वां मम संजाता त्वय्यनुक्रोशिनी मतिः}


\twolineshloka
{नाहमेतादृशं बुद्धं हन्तुमिच्छामि बन्धनैः}
{आनृशंस्यं परो धर्मो ह्यनुक्रोशश्च मे त्वयि}


\twolineshloka
{मोक्ष्यन्ते वारुणाः पाशास्तवेमे कालपर्ययात्}
{प्रजानामुपचारेण स्वस्ति तेऽस्तु महासुर}


\twolineshloka
{यदा श्वश्रूं स्नुषा वृद्धां परिचारेण योक्ष्यते}
{पुत्रश्च पितरं मोहात्प्रेषयिष्यति कर्मसु}


\twolineshloka
{ब्राह्मणैः कारयिष्यन्ति वृषलाः पादधावनम्}
{शूद्राश्च ब्राह्मणीं भार्यामुपयास्यन्ति निर्भयाः}


\twolineshloka
{वियोनिषु विमोक्ष्यन्ति बीजानि पुरुषा यदा}
{संकरं कांस्यभाण्डैश्च बलिं चैव कुपात्रकैः}


\twolineshloka
{चातुर्वर्ण्यं यदा कृत्स्नममर्यादं भविष्यति}
{एकैकस्ते तदा पाशः क्रमशः परिमोक्ष्यते}


\threelineshloka
{अस्मत्तस्ते भयं नास्ति समयं प्रतिपालय}
{सुखी भव निराबाधः स्वस्थचेता निरामयः ॥भीष्म उवाच}
{}


\twolineshloka
{तमेवमुक्त्वा भगवाञ्छतक्रतुःप्रतिप्रयातो गजराजवाहनः}
{विजित्य सर्वानसुरान्सुराधिपोननन्द हर्षेण बभूव चैकराट्}


\twolineshloka
{महर्षयस्तुष्टुवुरञ्जसा च तंवृषाकर्पि सर्वचराचरेश्वरम्}
{हिमापहो हव्यमुवाह चाध्वरेतथाऽमृतं चार्पितमीश्वरोऽपि हि}


\twolineshloka
{द्विजोत्तमैः सर्वगतैरभिष्टुतोविदीप्ततेजाः शतमन्युरीश्वरः}
{प्रशान्तचेता मुदितः स्वमालयंत्रिविष्टपं प्राप्य मुमोद वासवः}


\chapter{अध्यायः २३५}
\twolineshloka
{युधिष्ठिर उवाच}
{}


\threelineshloka
{पूर्वरूपाणि मे राजन्पुरुषस्य भविष्यतः}
{पराभविष्यतश्चैव तन्मे ब्रूहि पितामह ॥भीष्म उवाच}
{}


\twolineshloka
{मन एव मनुष्यस्य पूर्वरूपाणि शंसति}
{भविष्यतश्च भद्रं ते तथैव न भविष्यतः}


\twolineshloka
{अत्राप्युदाहरन्तीममितिहासं पुरातनम्}
{श्रिया शक्रस्य संवादं तन्निबोध युधिष्ठिर}


\twolineshloka
{महतस्तपसो व्यष्ट्या पश्यँल्लोकौ परावरौ}
{सामान्यमृषिभिर्गत्वा ब्रह्मलोकनिवासिभिः}


\twolineshloka
{ब्रह्मेवामितदीप्तौजाः शान्तपाप्मा महातपाः}
{विचचार यथाकालं त्रिषु लोकेषु नारदः}


\twolineshloka
{कदाचित्प्रातरुत्थाय पिस्पृक्षुः सलिलं शुचि}
{ध्रुवद्वारभवां गङ्गां जगामावततार च}


\twolineshloka
{`मेरुपादोद्भवां गङ्गां नारायणपदच्युताम्}
{स वीक्षमाणो हृष्टात्मा तं देशमभिजग्मिवान्}


\twolineshloka
{यं--देवजवाकीर्णं सूक्ष्मकाञ्जनवालुकम्}
{गङ्गाद्वीपं समासाद्य नानावृक्षैरलङ्कृतम्}


\twolineshloka
{सालतालाश्वकर्णानां चन्दनानां च राजिभिः}
{मण्डितं विविधैः पुष्पैर्हंसकारण्डवायुतम्}


\twolineshloka
{नदीपुलिनमासाद्य स्नात्वा संतर्प्य देवताः}
{जजाप जप्यं धर्मात्मा तन्मयत्वेन भास्वता ॥'}


\twolineshloka
{सहस्रनयनश्चापि वज्री शम्बरपाकहा}
{तस्या देवर्षिजुष्टायास्तीरमभ्याजगाम ह}


\twolineshloka
{तावाप्लुत्य यतात्मानौ कृतजप्यौ समासतः}
{नद्याः पुलिनमासाद्य सूक्ष्मकाञ्चनवालुकम्}


\threelineshloka
{पुण्यकर्मभिराख्याता देवर्षिकथिताः कथाः}
{चक्रतुस्तौ तथाऽऽसीनौ महर्षिकथितास्तथा}
{पूर्ववृत्तव्यतीतानि कथयन्तौ समाहितौ}


\twolineshloka
{अद्य भास्करमुद्यन्तं रश्मिजालपुरस्कृतम्}
{पूर्णमण्डलमालोक्य तावुत्थायोपतस्थतुः}


\twolineshloka
{`विविक्ते पुण्यदेशे तु रममाणौ मुदा युतौ}
{ददृशातेऽन्तरिक्षे तौ सूर्यस्योदयनं प्रति}


\twolineshloka
{ज्योतिर्ज्वालसमाकीर्णं ज्योतिषां गणमण्डितम्}
{अभितस्तूदयन्तं तमर्कमर्कमिवापरम्}


\threelineshloka
{आकाशो ददृशे ज्योतिरुद्यतार्चिः समप्रभम्}
{`अर्कस्य तेजसा तुल्यं तद्भास्करसमप्रभम्}
{'तयोः समीपं तं प्राप्तं प्रत्यदृश्यत भारत}


\twolineshloka
{तत्सुपर्णार्करचितमास्थितं वैष्णवं पदम्}
{भाभिरप्रतिमं भाति त्रैलोक्यमवभासयत्}


\twolineshloka
{`दृष्ट्वा तौ तु विक्रान्तौ प्राञ्जली समुपास्थितौ}
{क्रमात्संप्रेक्ष्यमाणौ तौ विमानं दिव्यमद्भुतम्}


\twolineshloka
{तस्मिंस्तदा सतीं कान्तां लोककान्तां परां शुभाम्}
{धात्रीं लोकस्य रमणीं लोकमातरमच्युताम् ॥'}


\twolineshloka
{दिव्याभिरुपशोभाभिरप्सरोभिः पुरस्कृताम्}
{बृहतीमंशुमत्प्रख्यां बृहद्भानोरिवार्चिषम्}


\twolineshloka
{नक्षत्रकल्पाभरणां तारापङ्क्तिसमस्रजम्}
{श्रियं ददृशतुः पद्मां साक्षात्पद्मदलस्थिताम्}


\twolineshloka
{साऽवरुह्य विमानाग्रादङ्गनानामनुत्तमा}
{अभ्यागच्छन्त्रिलोकेशं शक्रं चर्षि च नारदम्}


\twolineshloka
{नारदानुगतः साक्षान्मघवांस्तामुपागमत्}
{कृताञ्जलिपुटो देवीं निवेद्यात्मानमात्मना}


\twolineshloka
{चक्रे चानुपमां पूजां तस्याश्चापि स सर्वंवित्}
{देवराजः श्रियं राजन्वाक्यं चेदमुवाच ह}


\threelineshloka
{का त्वं केन च कार्येण संप्राप्ता चारुहासिनि}
{कुतश्चागम्यते सुभ्रु गन्तव्यं क्व च ते शुभे ॥श्रीरुवाच}
{}


\twolineshloka
{पुण्येषु त्रिषु लोकेषु सर्वे स्थावरजङ्गमाः}
{ममात्मभावमिच्छन्तो यतन्ते परमात्मना}


\twolineshloka
{साऽहं वै पङ्कजे जाता सूर्यरश्मिप्रबोधिते}
{भूत्यर्तं सर्वभूतानां पद्मा श्रीः पद्ममालिनी}


\twolineshloka
{अहं लक्ष्मीरहं भूतिः श्रीश्चाहं बलसूदन}
{अहं श्रद्धा च मेधा च सन्नतिर्विजितिः स्थितिः}


\twolineshloka
{अहं धृतिरहं सिद्धिरहं संभूतिरेव च}
{अहं स्वाहा स्वधा चैव संस्तुतिर्नियतिः कृतिः}


\twolineshloka
{राज्ञां विजयमानानां सेनाग्नेषु ध्वजेषु च}
{निवसे धर्मशीलानां विषयेषु पुरेषु च}


\twolineshloka
{जितकाशिनि शूरे च संग्रामेष्वनिवर्तिनि}
{निवसामि मनुष्येन्द्रे सदैव बलसूदन}


\twolineshloka
{धर्मनित्ये महाबुद्धौ ब्रह्मण्ये सत्यवादिनि}
{प्रश्रिते दानशीले च सदैव निवसाम्यहम्}


\threelineshloka
{असुरेष्ववसं पूर्वं सत्यधर्मनिबन्धनात्}
{विपरीतांस्तु तान्बुद्ध्वा त्वयि वासमरोचयम् ॥शक्र उवाच}
{}


\threelineshloka
{कथं वृत्तेषु दैत्येषु त्वमवात्सीर्वरानाने}
{दृष्ट्वा च किमिहागास्त्वं हित्वा दैतेयदानवान् ॥श्रीरुवाच}
{}


\twolineshloka
{स्वधर्ममनुतिष्ठत्सु धैर्यादचलितेषु च}
{स्वर्गमार्गाभिरामेषु सत्वेषु निरता ह्यहम्}


\twolineshloka
{दानाध्ययनयज्ञेज्यापितृदैवतपूजनम्}
{गुरूणामतिथीनां च तेषां नित्यमवर्तत}


\twolineshloka
{सुसंमृष्टगुहाश्चासञ्जितस्त्रीका हुताग्नयः}
{गुरुशुश्रूषका दान्ता ब्रह्मण्याः सत्यवादिनः}


\twolineshloka
{श्रद्दधाना जितक्रोधा दानशीलाऽनसूयवः}
{भृतपुत्रा भृतामात्या भृतदारा ह्यनीर्षवः}


\twolineshloka
{अमर्षेण न चान्योन्यं स्पृहयन्ते कदाचन}
{न च जातूपतप्यन्ति धीराः परसमृद्धिभिः}


\twolineshloka
{दातारः संग्रहीतार आर्याः करुणवेदिनः}
{महाप्रसादा ऋजवो दृढभक्ता जितेन्द्रियाः}


\twolineshloka
{संतुष्टभृत्यसचिवाः कृतज्ञाः प्रियवादिनः}
{यथार्हमानार्थकरा ह्रीनिषेवा यतव्रताः}


\twolineshloka
{नित्यं पर्वसु सुस्नाताः स्वनुलिप्ताः स्वलङ्कृताः}
{उपवासतपः शीलाः प्रतीता ब्रह्मवादिनः}


\twolineshloka
{नैनानभ्युदियात्सूर्यो नैवास्वप्स्यन्प्रगेशयाः}
{रात्रौ दधि च सक्तूंश्च नित्यमेव व्यवर्जयन्}


\twolineshloka
{काल्यं घृतं तु चान्वीक्ष्य प्रयता ब्रह्मवादिनः}
{पङ्गल्यान्यपि चापश्यन्ब्रह्माणांश्चाप्यपूजयन्}


\twolineshloka
{सदा हि ददतां धर्म्यं सदाचाप्रतिगृह्णताम्}
{अर्धं च रात्र्याः स्वपतां दिवा चास्वपतां तथा}


\threelineshloka
{कृपणानाथवृद्धानां दुर्बलातुरयोषिताम्}
{दयां च संविभागं च नित्यमेवान्वमोदताम्}
{कालो यातः सुखेनैव धर्ममार्गे निवर्तताम्}


\twolineshloka
{त्रस्तं विषण्णमुद्विग्नं भयार्तं व्याधिपीडितम्}
{हृतस्वं व्यसनार्तं च नित्यमाश्वासयन्ति ते}


\twolineshloka
{धर्ममेवानुवर्तन्ते न हिंसन्ति परस्परम्}
{अनुकूलाश्च कार्येषु गुरुवृद्धोपसेविनः}


\twolineshloka
{पितॄन्देवातिथींश्चैव गुरूंश्चैवाभ्यपूजयन्}
{अवशेषाणि चाश्नन्ति नित्यं सत्यतपोधृताः}


\twolineshloka
{नैकेऽश्नन्ति सुसंपन्नं न गच्छन्ति परस्त्रियम्}
{सर्वभूतेष्ववर्तन्त यथाऽऽत्मनि दयां प्रति}


\twolineshloka
{नैवाकाशे न पशुषु नायोनौ च न पर्वसु}
{इन्द्रियस्य विसर्गं ते रोचयन्ति कदाचन}


\twolineshloka
{नित्यं दानं तथा दाक्ष्यमार्जवं चैव नित्यदा}
{उत्साहोऽथानहकारः परमं सौहृदं क्षमा}


\twolineshloka
{सत्यं दानं तपः शौचं कारुण्यं वागनिष्ठुरा}
{मित्रेषु चानभिद्रोहः सर्वं तेष्वभवत्प्रभो}


\twolineshloka
{निद्रा तन्द्रीरसंप्रीतिरसूयाऽथानवेक्षिता}
{अरतिश्च विषादश्च स्पृहा चाप्यविशन्न तान्}


\twolineshloka
{साऽहमेवंगुणेष्वेव दानवेष्ववसं पुरा}
{प्रजासर्गमुपादाय यावद्गुणविपर्ययम्}


\twolineshloka
{ततः कालविपर्यासे तेषां गुणविपर्ययात्}
{अपश्यं निर्गतं धर्मं कामक्रोधवशात्मनाम्}


\twolineshloka
{सभासदां च वृद्धानां सतां कथयतां कथाः}
{प्राहसन्नभ्यसूयंश्च सर्वबुद्धान्गुरून्परान्}


\twolineshloka
{युवानश्च समासीना वृद्धानपि गतान्सतः}
{नाभ्युत्थानाभिवादाभ्यां यथापूर्वमपूजयन्}


\twolineshloka
{वर्तयत्येव पितरि पुत्रः प्रभवते तथा}
{अमित्रभृत्यतां प्राप्य ख्यापयन्त्यनपत्रपाः}


\twolineshloka
{तथा धर्मादपेतेन कर्मणा गर्हितेन ये}
{महतः प्राप्नुवन्त्यर्थांस्तेषां तत्राभवत्स्पृहा}


\twolineshloka
{उच्चैश्चाभ्यवदन्रात्रौ नीचैस्तत्राग्निरज्वलत्}
{पुत्राः पितृनत्यचरन्नार्यश्चात्यचरन्पतीन्}


\twolineshloka
{मातरं पितरं वृद्धमाचार्यमतिथिं गुरुम्}
{गुरुत्वान्नाभ्यनन्दन्त कुमारान्नान्वपालयन्}


\twolineshloka
{भिक्षां बलिमदत्त्वा च स्वयमन्नानि भुञ्जते}
{अनिष्ट्वाऽसंविभज्याथ पितृदेवातिथीन्गुरून्}


\twolineshloka
{न शौचमन्वरुद्ध्यन्त तेषां सूदजनास्तथा}
{मनसा कर्मणा वाचा भक्ष्यमासीदनावृतम्}


\twolineshloka
{`बालानां प्रेक्षमाणानां भक्तान्यश्नन्ति मोहिताः}
{एको दासो भवेत्तेषां तेषां दासीद्वयं तथा}


\twolineshloka
{त्रिगवा दानवाः केचिच्चतुरोजास्तथा परे}
{षडश्वाः सप्तमातङ्गाः पञ्चमाहिषिकाः परे}


\twolineshloka
{रात्रौ दधि च सक्तूंश्च नित्यमेवाविवर्जिताः}
{अन्तर्दशाहे चाश्नन्ति गवां क्षिरं विचेतनाः}


\twolineshloka
{क्रमदोहं न कुर्वन्ति वत्सस्तन्यानि भुञ्जते}
{अनाथां कृपणां भार्यां घ्नन्ति नित्यं शपन्ति च}


\twolineshloka
{शूद्रान्नपुष्टा विप्रास्तु निर्लज्जाश्च भवन्त्युत}
{संकीर्णानि च धान्यानि नात्यवेक्षत्कुटुंबिनी}


\twolineshloka
{मार्जारकुक्कुटश्वानैः क्रीडां कुर्वन्ति मानवाः}
{गृहे कण्टकिनो वृक्षास्तथा निष्पाववल्लरी}


\twolineshloka
{यज्ञियाश्च तथा वृक्षास्तेषामासन्दुरात्मनाम्}
{कूपस्नानरता नित्यं पर्वमैथुनगामिनः}


\twolineshloka
{तिलानश्नन्ति रात्रौ च तैलाभ्यक्ताश्च शेरते}
{विभीतककरञ्जानां छायामूलनिवासिनः}


\twolineshloka
{करवीरं च ते पुष्पं धारयन्ति च मोहिताः}
{पद्मबिजानि खादन्ति पुष्पं जिघ्रन्ति मोहिताः}


\twolineshloka
{न भोक्ष्यन्ति तथा नित्यं दैत्याः कालेन मोहिताः}
{निन्दन्ति स्तवनं विष्णोस्तस्य नित्यद्विषो जनाः}


\twolineshloka
{होमधूमो न तत्रासीद्वेदघोषस्तथैव च}
{यज्ञाश्च न प्रवर्तन्ते यथापूर्वं गृहेगृहे}


\twolineshloka
{शिष्याचार्यक्रमो नासीत्पुत्रैरात्मपितुः पिता}
{विष्णुं ब्रह्मण्यदेवेशं हित्वा पाषण्डमाश्रिताः}


\threelineshloka
{हव्यकव्यविहीनाश्च ज्ञानाध्ययनवर्जिताः}
{देवस्वादानरुचयो ब्रह्मस्वरुचयस्तथा}
{स्तुतिमङ्गलहीनानि देवस्थानानि सर्वशः ॥'}


\twolineshloka
{विप्रकीर्णानि धान्यानि काकमूषिकभोजनम्}
{अपावृतं पयोतिष्ठदुच्छिष्टाश्चास्पृशन्धृतम्}


\twolineshloka
{कुद्दालं दात्रपिटकं प्रकीर्णं कांस्यभोजनम्}
{द्रव्योपकरणं सर्वं नान्ववैक्षत्कुटुम्बिनी}


\twolineshloka
{प्राकारागारविध्वंसान्न स्म ते प्रतिकुर्वते}
{`क्षुद्राः संस्कारहीनाश्च नार्यो ह्युदरपोषणाः}


\threelineshloka
{शौचाचारपरिभ्रष्टा निर्लज्जा भोगवञ्चिताः}
{उभाभ्यामेव पाणिभ्यां शिरः कण्डूयनान्विताः}
{गृहजालाभिसंस्थाना ह्यासंस्तत्र स्त्रियः पुनः}


\twolineshloka
{श्वश्रूश्वशुरयोर्मध्ये भर्तारं कृतकं यथा}
{प्रेक्षयन्ति च निर्लज्जा नार्यः कुलजलक्षणाः ॥'}


\threelineshloka
{नाद्रियन्ते पशून्बद्ध्वा यवसेनोदकेन च}
{बालानां प्रेक्षमाणानां स्वयं भक्ष्यमभक्षयन्}
{तथा भृत्यजनं सर्वमसंतर्प्य च दानवाः}


\twolineshloka
{पायसं कृसरं मांसमपूपानथ शष्कुलीः}
{अपाचयन्नात्मनोऽर्थे वृथा मांसान्यभक्षयन्}


\twolineshloka
{उत्सूर्यशायिनश्चासन्सर्वे चासन्प्रगेशयाः}
{आवृत्तकलहाश्चात्र दिवारात्रं गृहेगृहे}


\twolineshloka
{अनार्याश्चार्यमासीनं पर्युपासन्न तत्र ह}
{आश्रमस्थान्विकर्मस्थाः प्राद्विषन्त परस्परम्}


\threelineshloka
{संकराश्चाभ्यवर्तन्त न च शौचमवर्तत}
{ये च वेदविदो विप्रा विस्पष्टमनुचश्च ये}
{निरन्तरविशेषास्ते बहुमानावमानयोः}


\twolineshloka
{भावमाभरणं वेषं गतं स्थितमवेक्षितम्}
{असेवन्त भुजिष्या वै दुर्जनाचरितं विधिम्}


\twolineshloka
{स्त्रियः पुरुषवेषेण पुंसः स्त्रीवेषधारिणः}
{क्रीडारतिविहारेषु परां मुदमवाप्नुवन्}


\twolineshloka
{प्रभवद्भिः पुरा दायानर्हेभ्यः प्रतिपादितान्}
{नाभ्यवर्न्तत नास्तिक्याद्वर्तन्तः संभवेष्वपि}


\twolineshloka
{मित्रेणाभ्यर्थितं द्रव्यमर्थी संश्रयते क्वचित्}
{वालकोट्यग्रमात्रेण स्वार्थेनाघ्नत तद्वसु}


\twolineshloka
{परस्वादानरुचयो विपणव्यवहारिणः}
{अदृश्यन्तार्यवर्णषु शृद्राश्चापि तपोधनाः}


\twolineshloka
{अधीयतेऽव्रताः केचिद्वृथा व्रतमथापरे}
{अशुश्रूषुर्गुरोः शिष्यः कश्चिच्छिप्यसखो गुरुः}


\twolineshloka
{पिता चैव जनित्री च श्रान्तौ वृत्तोत्सवाविव}
{अप्रभुत्वे स्थितौ वृद्धावन्नं प्रार्थयतः सुतान्}


\twolineshloka
{तत्र वेदविदः प्राज्ञा गाम्भीर्ये सागरोपमाः}
{कृष्यादिष्वभवन्सक्ता मूर्खाः श्राद्धान्यभुञ्जत}


\twolineshloka
{प्रातः सायं च सुप्रश्नं कल्पनं प्रेषणक्रियाः}
{शिष्यानप्रहितास्तेषामकुर्वन्गुरवश्च ह}


\twolineshloka
{श्वश्रूश्वशुरयोरग्रे वधूः प्रेष्यानशासत}
{अन्वशासच्च भर्तारं समाह्वायाभिजल्पति}


\twolineshloka
{प्रयत्नेनापि चारक्षच्चित्तं पुत्रस्य वै पिता}
{व्यभजच्चापि संरम्भाद्दुःखवासं तथाऽवसत्}


\twolineshloka
{अग्निदाहेन चोरैर्वा राजभिर्वा हृतं धनम्}
{दृष्ट्वा द्वेषात्प्राहसन्त सुहृत्संभाविता ह्यपि}


\twolineshloka
{कृतघ्ना नास्तिकाः पापा गुरुदाराभिमर्शिनः}
{`श्वशुरानुगताः सर्वे ह्युत्सृज्य पितरौ सुताः}


\twolineshloka
{स्वकर्मणा च जातोऽहमित्येवंवादिनस्तथा}
{'अभक्ष्यभक्षणरता निर्मर्यादा हतत्विषः}


\twolineshloka
{तेष्वेवमादीनाचारानाचरत्सु विपर्यये}
{नाहं देवेन्द्र वत्स्यामि दानवेष्विति मे मतिः}


\twolineshloka
{तन्मां स्वयमनुप्राप्ताभिनन्द शचीपते}
{त्वयाऽर्चितां मां देवेश पुरो धास्यन्ति देवताः}


\twolineshloka
{यत्राहं तत्र मत्कान्ता मद्विशिष्टा मदर्पणाः}
{सप्तदेव्यो जयाष्टभ्यो वासमेष्यन्ति तेऽष्टधा}


\twolineshloka
{आशा श्रद्धा धृतिः क्षान्तिर्विजितिः सन्नतिः क्षमा}
{अष्टमी वृत्तिरेतासां पुरोगा पाकशासन}


\twolineshloka
{ताश्चाहं चासुरांस्त्यक्त्वा युष्मद्विषयमागताः}
{त्रिदशेषु निवत्स्यामो धर्मनिष्ठान्तरात्मसु}


\twolineshloka
{इत्युक्तवचनां देवीं प्रीत्यर्थं च ननन्दतुः}
{नारदश्चात्र देवर्षिर्वृत्रहन्ता च वासवः}


\twolineshloka
{ततोऽनलसखो वायुः प्रववौ देववर्त्मसु}
{इष्टगन्धः सुखस्पर्शः सर्वेन्द्रियसुखावहः}


\twolineshloka
{शुचौ चाभ्यर्थिते देशे त्रिदशाः प्रायशः स्थिताः}
{लक्ष्मीसहितमासीनं मघवन्तं दिदृक्षवः}


\threelineshloka
{ततो दिवं प्राप्य सहस्रलोचनः}
{स्त्रियोपपन्नः सुहृदा महर्षिणा}
{रथेन हर्यश्वयुजा सुरर्षभःसदः सुराणामभिसत्कृतो ययौ}


\twolineshloka
{अथेङ्गितं वज्रधरस्य नारदःश्रियश्च देव्या मनसा विचारयन्}
{श्रियै शशंसामरदृष्टपौरुषःशिवेन तत्रागमनं महर्षिभिः}


\twolineshloka
{ततोऽमृतं द्यौः प्रववर्ष भास्वतीपितामहस्यायतने स्वयंभुवः}
{अनाहता दुन्दुभयोऽथ नेदिरेतथा प्रसन्नाश्च दिशश्चकाशिरे}


\twolineshloka
{यथर्तु सस्येषु ववर्ष वासवोन धर्ममार्गाद्विचचाल कश्चन}
{अनेकरत्नाकरभूषणा च भूःसुघोषघोषाश्च दिवौकसां जये}


\twolineshloka
{क्रियाभिरामा मनुजा मनस्विनोबभुः शुभे पुण्यकृतां पथि स्थिताः}
{नरामराः किन्नरयक्षराक्षसाःसमृद्धिमन्तः सुमनस्विनोऽभवन्}


\twolineshloka
{न जात्वकाले कुसुमं कुतः फलंपपात वृक्षात्पवनेरितादपि}
{रसप्रदाः कामदुघाश्च धेनवोन दारुणा वाग्विचचार कस्यचित्}


\twolineshloka
{इमां सपर्यां सह सर्वकामदैःश्रियाश्च शक्रप्रमुखैश्च दैवतैः}
{पठन्ति ये विप्रसदः समागताःसमृद्धकामाः श्रियमाप्नुवन्ति ते}


\twolineshloka
{त्वया कुरूणां वर यत्प्रचोदितंभवाभवस्येह परं निदर्शनम्}
{तदद्य सर्वं परिकीर्तितं मयापरीक्ष्य तत्त्वं परिगन्तुमर्हसि}


\twolineshloka
{`संस्मृत्य बुद्धीन्द्रियगोचरातिगंस्वगोचरे सर्वकृतालयं तम्}
{हरिं महापाग्रहरं जनास्तेसंस्मृत्य संपूज्य विधूतपापाः}


\twolineshloka
{यमैश्च नित्यं नियमैश्च संयतास्तत्वं च विष्णोः परिपश्यमानाः}
{देवानुसारेण विमुक्तियोगंते गाहमानाः परमाप्नुवन्ति}


\twolineshloka
{एवं राजेन्द्र सततं जपहोमपरायणः}
{वासुदेवपरो नित्यं ज्ञानध्यानपरायणः}


\threelineshloka
{दानधर्मरतिर्नित्यं प्रजास्त्वं परिपालय}
{वासुदेवपरो नित्यं ज्ञानध्यानपरायणान्}
{विशेषेणार्चयेथास्त्वं सततं पर्युपास्व च ॥'}


\chapter{अध्यायः २३६}
\twolineshloka
{युधिष्ठिर उवाच}
{}


\threelineshloka
{किंशीलः किंसमाचारः किंविद्यः किंपराक्रमः}
{प्राप्नोति ब्रह्मणः स्थानं यत्परं प्रकृतेर्ध्रुवम् ॥भीष्म उवाच}
{}


\twolineshloka
{मोक्षधर्मेषु नियतो लध्वाहारो जितेन्द्रियः}
{प्राप्नोति ब्रह्मणः स्थानं तत्परं प्रकृतेर्ध्रुवम्}


\twolineshloka
{अत्राप्युदाहरन्तीममितिहासं पुरातनम्}
{जैगीषव्यस्य संवादमसितस्य च भारत}


\twolineshloka
{`महादेवान्तरे वृत्तं देव्याश्चैवान्तरे तथा}
{यथावच्छृणु राजेन्द्र ज्ञानदं पापनाशनम् ॥'}


\twolineshloka
{जैगीषव्यं नहाप्रज्ञं धर्नाणामागतागमम्}
{अक्रुध्यन्तमहृष्यन्तमसितो देवलोऽब्रवीत्}


\threelineshloka
{न प्रीमसे बन्द्यमानो निन्द्यमानो न कुप्यसे}
{का ते प्रज्ञा कुतश्चैषा किं ते तस्याः परायणम् ॥भीष्म उवाच}
{}


\threelineshloka
{इति तेनानुयुक्तः स तमृपाच महातपाः}
{महद्वाक्यप्रसंदिग्धं पुष्कलार्थपदं शुचि ॥जैगीषव्य उवाच}
{}


\twolineshloka
{या यतियौ परा निष्ठा या शान्तिः पुण्यकर्मणाम्}
{तां तेऽहं सं प्रवक्ष्यामि यां मां पृच्छसि वै द्विज}


\twolineshloka
{निन्वत्सु वा सप्ता नित्यं प्रशंसत्सु च देवल}
{निहवन्ति च ये तेषां समयं सुकृतं च यत्}


\twolineshloka
{उक्ताश्च न विवक्ष्यन्ति वक्तारमहिते हितम्}
{प्रतिहन्तुं न चेच्छन्ति हन्तारं वै मनीषिणः}


\twolineshloka
{नाप्राप्तमनुशोचन्ति प्राप्तकालानि कुर्वते}
{न चातीतानि शोचन्ति न चैतान्प्रतिजानते}


\twolineshloka
{संप्राप्तानां च पूज्यानां कामादर्थेषु देवल}
{यथोपपत्तिं कुर्वन्ति शक्तिमन्तो धृतव्रताः}


\twolineshloka
{पक्वविद्या महाप्राज्ञा जितक्रोधा जितेन्द्रियाः}
{मनसा कर्मणा वाचा नापराध्यन्ति कस्यचित्}


\twolineshloka
{अनीर्षवो न चोन्योन्यं विहिंसन्ति कदाचन}
{न च जातूपतप्यन्ते धीराः परसमृद्धिभिः}


\twolineshloka
{निन्दाप्रशंसे चात्यर्थं न वदन्ति परस्य च}
{न च निन्दाप्रशंसाभ्यां विक्रियन्ते कदाचन}


\twolineshloka
{सर्वतश्च प्रशान्ता ये सर्वभूतहिते रताः}
{न कुध्यन्ति स हृष्यन्ति नापराध्यन्ति कस्यचित्}


\twolineshloka
{विमुच्य हृदयग्रन्थिं चङ्कम्यन्ते यथासुखम्}
{न चैषां बान्धवाः सन्ति ये चान्येषां च बान्धवाः}


\twolineshloka
{अमित्राश्च न सन्त्येषां ये चामित्रा न कस्यचित्}
{य एवं कुर्वते मर्त्याः सुखं जीवन्ति सर्वदा}


\twolineshloka
{ये धर्मं चानुरुध्यन्ते धर्मज्ञा द्विजसत्तमाः}
{ये ह्यतो विच्युता मार्गात्ते हृष्यन्त्युद्विजन्ति च}


\twolineshloka
{आस्थितस्तमहं मार्गमसूयिष्यामि कं कथम्}
{निन्द्यमानः प्रशस्तो वा हृष्येयं केन हेतुना}


\twolineshloka
{यद्यदिच्छन्ति तत्तस्मादधिगच्छन्ति मानवाः}
{न मे निन्दाप्रशंसाभ्यां ह्रासवृद्धी भविष्यतः}


\twolineshloka
{अमृतस्येव संतृप्येदवमानस्य तत्त्ववित्}
{विषस्येवोद्विजेन्नित्यं संमानस्य विचक्षणः}


\twolineshloka
{अवज्ञातः सुखं शेते इह चामुत्र चोभयोः}
{विमुक्तः सर्वपापेभ्यो योऽवमन्ता स बुध्यते}


\twolineshloka
{परां गतिं च ये केचित्प्रार्थयन्ति मनीषिणः}
{एतद्व्रतं समाश्रित्य सुखमेधन्ति ते जनाः}


\twolineshloka
{सर्वतश्च समाहृत्य क्रतून्सर्वाञ्जितेन्द्रियः}
{प्राप्नोति ब्रह्मणः स्थानं यत्परं प्रकृतेर्ध्रुवम्}


\twolineshloka
{नास्य देवा न गन्धर्वा न पिशाचा न राक्षसाः}
{पदमन्ववरोहन्ति प्राप्तस्य परमां गतिम्}


\twolineshloka
{`एतच्छ्रुत्वा मुनेस्तस्य वचनं देवलस्तथा}
{तदधीनो भवच्छिष्यः सर्वद्वन्द्वविनिष्ठितः}


\twolineshloka
{अथान्यत्तु पुरावृतं जैगीषव्यस्य धीमतः}
{शृणु राजन्नवहितः सर्वज्ञानसमन्वितः}


\twolineshloka
{यमाहुः सर्वलोकेशं सर्वलोकनमस्कृतम्}
{अष्टमूर्ति जगन्मूर्तिमिष्टसंधिविभूषितम् ॥यं प्राप्ता न विषीदन्ति न शोचन्त्युद्विजन्ति च}


\twolineshloka
{यस्य स्वाभाविकी शक्तिरिदं विश्वं चराचरम्}
{याति सज्जति सर्वात्मा स देवः परमेश्वरः}


\twolineshloka
{मेरोरुत्तरपूर्वे तु सर्वरत्नविभूषिते}
{अचिन्त्ये विमले स्थाने सर्वर्तुकुसुमान्विते}


\twolineshloka
{वृक्षैश्च शोभिते नित्यं दिव्यवायुसमीरिते}
{नानाभूतगणैर्युक्तः सर्वदेवनमस्कृतः}


\twolineshloka
{तत्र विद्याधरगणा गन्धर्वाप्सरसां गणाः}
{लोकपालाः समुद्राश्च नद्यः शैलाः सरांसि च}


\twolineshloka
{ऋषयो वालखिल्याश्च यज्ञाः स्तोभाह्वयास्तथा}
{उपासांचक्रिरे देवं प्रजानां पतयस्तथा}


\twolineshloka
{तत्र रुद्रो महादेवो देव्या चैव सहोमया}
{आस्ते वृषध्वजः श्रीमान्सोमसूर्याग्निलोचनः}


\twolineshloka
{तत्रैवं देवमालोक्य देवी धात्री विभावरी}
{उमा देवी परेशानमपृच्छद्विनयान्विता}


\twolineshloka
{अर्थः कोऽथार्थशक्तिः का भगवन्ब्रूहि मेऽर्थितः}
{तयैवं परिपृष्टोऽसौ प्राह देवो महेश्वरः}


\twolineshloka
{अर्थोऽहमर्थशक्तिस्त्वं भोक्ताऽहं भोज्यमेव च}
{रूपं विद्धि महाभागे प्रकृतिस्त्वं परो ह्यहम्}


\threelineshloka
{अहं विष्णुरहं ब्रह्मा ह्यहं यज्ञस्तथैव च}
{आवयोर्न च भेदोऽस्ति परमार्थस्ततोऽबले}
{तथापि विद्मस्ते भेदं किं मां त्वं परिपृच्छसि}


\twolineshloka
{एवमुक्ता ततः प्राह ह्यधिकं ह्येतयोर्वद}
{श्रेष्ठं वेद महादेव नम इत्येव भामिनी}


\twolineshloka
{तदन्तरे स्थितो विद्वान्वसुरूपो महामुनिः}
{जैगीषव्यः स्मयन्प्राह ह्यर्थ इत्येव नादयन्}


\twolineshloka
{श्रेष्ठोन्योऽस्मान्महीपिण्डा तल्लीना शक्तिरापरा}
{मुद्रिकादिविशेषेण विस्तृता संभृतेति च}


\twolineshloka
{तच्छ्रुत्वा वचनं देवी कोसावित्यब्रवीद्रुषा}
{वाक्यमस्याद्य संभङ्क्त्वा प्रोक्तवानिति शंकरम्}


\twolineshloka
{तच्छ्रुत्वा निर्गतो धीमानाश्रमं स्वं महामुनिः}
{स्थानात्स्वर्गगणे विद्वान्योगैश्वर्यसमन्वितः}


\twolineshloka
{ततः प्रहस्य भगवान्सर्वपापहरो हरः}
{प्राह देवीं प्रशान्तात्मा जैगीषव्यो महामुनिः}


\twolineshloka
{भक्तो मम सखा चैव शिष्यश्चात्र महामुनिः}
{जैगीषव्य इति ख्यातः प्रोक्त्वासा निर्गतः शुभे}


\twolineshloka
{तच्छ्रुत्वा साऽथ संक्रुद्धा न न्याय्यं तेन वै कृतम्}
{विकृताऽहं त्वया देव मुनिना च तथाकृता}


\twolineshloka
{अत*ज्ञादयदेवेश मध्ये प्राप्तं न तच्छ्रुतम्}
{तच्छ्रुत्वा भगवानाह महादेवः पिनाकधृत्}


\twolineshloka
{निरपेक्षो मुनिर्योगी मामुपाश्रित्य संस्थितः}
{निर्द्वन्द्वः सततं धीमान्समरूपस्वभावधृत्}


% Check verse!
तस्मात्क्षमस्व तं देवि रक्षितव्यस्त्वया च सः
\twolineshloka
{इत्युक्ता प्राह सा देवी मुनेस्तस्य महात्मनः}
{निराशत्वमहं द्रष्टुमिच्छाम्यन्तकनाशन}


\twolineshloka
{तथेति चोक्त्वा तां देवो वृषमारुह्य सत्वरम्}
{देवगन्धर्वसङ्घैश्च स्तूयमानो जगत्पतिः}


\twolineshloka
{अजरामरशुद्धात्मा यत्रास्ते स महामुनिः}
{इतस्ततः समाहृत्य वीरसंघैर्महायशाः}


\threelineshloka
{देहप्रावरणार्थं वै संसरन्स तदा मुनिः}
{प्रत्युद्गम्य महादेवं यथार्हं प्रतिपूज्य च}
{पुनः स पूर्ववत्कथां सूच्या सूत्रेण सूचयत्}


\twolineshloka
{तमाह भगवाञ्शंभुः किं प्रदास्यामि ते मुने}
{वृणीष्व मत्तः सर्वं त्वं जैगीषव्य यदीच्छसि}


\threelineshloka
{नावलोकयमानस्तु देवदेवं महामुनिः}
{अनवाप्तं न पश्यामि त्वत्तो गोवृषभध्वज}
{कृतार्थः परिपूर्णोऽहं यत्ते कार्यं तु गम्यताम्}


\twolineshloka
{प्रहसंस्तु पुनः शर्वो वृणीष्वेति तमब्रवीत्}
{अवश्यं हि वरो प्रत्तः श्राव्यं वरमनुत्तमम्}


\twolineshloka
{जैगीषव्यस्तमाहेदं श्रोतव्यं च त्वया मम}
{सूचीमनु महादेव सूत्रं समनुगच्छतः}


\twolineshloka
{ततः प्रहस्य भगवान्गौरीमालोक्य शङ्करः}
{स्वस्थानं प्रययौ हृष्टः सर्वदेवनमस्कृतः}


\twolineshloka
{एतत्ते कथितं राजन्यस्मात्त्वं परिपृच्छसि}
{निर्द्वन्द्वा योगिनो नित्याः सर्वशस्ते स्वयंभुवः ॥'}


\chapter{अध्यायः २३७}
\twolineshloka
{युधिष्ठिर उवाच}
{}


\threelineshloka
{प्रियः सर्वस्य लोकस्य सर्वसत्वाभिनन्दितः}
{गुणैः तर्पैरुपेतश्च कोन्वस्ति भुवि मानवः ॥भीष्म उवाच}
{}


\threelineshloka
{अत्र ते वर्तयिष्यामि पृच्छतो भरतर्षभ}
{उग्रसेनस्य संवादं नारदे केशवस्य च ॥उग्रसेन उवाच}
{}


\threelineshloka
{यस्य संकल्पते लोको नारदस्य प्रकीर्तने}
{मन्ये स गुणसंपन्नो ब्रूहि तन्मम पृच्छतः ॥वासुदेव उवाच}
{}


\twolineshloka
{कुकुराधिप यान्मन्ये शृणु तान्मे विवक्षतः}
{नारदस्य गुणान्साधून्संक्षेपेण नराधिप}


\twolineshloka
{न चारित्रनिमित्तोऽस्याहंकारो देहपातनः}
{अभिन्नश्रुतचारित्रस्तस्मात्सर्वत्र पूजितः}


% Check verse!
अरतिः क्रोधचापल्ये भयं नैतानि नारदे ॥अदीर्घसूत्रः शूरश्च तस्मात्सर्वत्र पूजितः
\twolineshloka
{उपास्यो नारदो बाढं वाचि नास्य व्यतिक्रमः}
{कामतो यदि वा लोभात्तस्मात्सर्वत्र पूजितः}


\twolineshloka
{अध्यात्मविधितत्त्वज्ञः क्षान्तः शक्तो जितेन्द्रियः}
{ऋजुश्च सत्यवादी च तस्मात्सर्वत्र पूजितः}


\twolineshloka
{तेजसा यशसा बुद्ध्या ज्ञानेन विनयेन च}
{जन्मना तपसा वृद्धस्तस्मात्सर्वत्र पूजितः}


\twolineshloka
{सुशीलः सुखसंवेशः सुभोजः स्वादरः शुचिः}
{सुवाक्यश्चाप्यनीर्ष्यश्च तस्मात्सर्वत्र पूजितः}


\twolineshloka
{कल्याणं कुरुते बाढं पापमस्मिन्न विद्यते}
{न प्रीयते परानर्थैस्तस्मात्सर्वत्र पूजितः}


\twolineshloka
{वेदश्रूतिभिराख्यानैरर्थानभिजिगीषति}
{तितिक्षुरनवज्ञश्च तस्मात्सर्वत्र पूजितः}


\twolineshloka
{समत्वाच्च प्रियो नास्ति नाप्रियश्च कथंचन}
{मनोऽनुकूलवादी च तस्मात्सर्वत्र पूजितः}


\twolineshloka
{बहुश्रुतश्चित्रकथः पण्डितोऽनलसोऽशठः}
{अदीनोऽक्रोधनोऽलुब्धस्तस्मात्सर्वत्र पूजितः}


\twolineshloka
{नार्थे धने वा कामे वा भूतपूर्वोऽस्य विग्रहः}
{दोषाश्चास्य समुच्छिन्नास्तस्मात्सर्वत्र पूजितः}


\twolineshloka
{दृढभक्तिरनिन्द्यात्मा श्रुतवाननृशंसवान्}
{वीतसंमोहदोषश्च तस्मात्सर्वत्र पूजितः}


\twolineshloka
{असक्तः सर्वसङ्गेषु सक्तात्मेव च लक्ष्यते}
{अदीर्घसंशयो वाग्मी तस्मात्सर्वत्र पूजितः}


\twolineshloka
{समाधिर्नास्य कामार्थै नात्मानं स्तौति कर्हिचित्}
{अनीर्षुर्मृदुसंवादस्तस्मात्सर्वत्र पूजितः}


\threelineshloka
{`नाहंकारे मुक्तिरस्य चारित्रे बुद्धिरास्थिता}
{वेदार्थविद्विभागेन यज्ञविद्योगवित्कविः}
{भक्तिमान्य सदा विद्वांस्तस्मात्सर्वत्र पूजितः}


\twolineshloka
{त्रिगुणं गुणभोक्तारं पञ्चयज्ञात्मकं तथा}
{यथावत्स विजानाति तस्मात्सर्वत्र पूजितः}


\twolineshloka
{कल्याणं कुरुते बाढं पापमस्मिन्न विद्यते}
{न प्रीयते परानर्थैस्तस्मात्सर्वत्र पूज्यते ॥'}


\twolineshloka
{लोकस्य विविधं चित्तं प्रेक्षते चाप्यकुत्सयन्}
{संसर्गविद्याकुशलस्तस्मात्सर्वत्र पूजितः}


\twolineshloka
{नासूयत्यागमं कंचित्स्वनयेनोपजीवति}
{अबन्ध्यकालोऽवश्यात्मा तस्मात्सर्वत्र पूजितः}


\twolineshloka
{कृतश्रमः कृतप्रज्ञो न च तृप्तः समाधितः}
{नित्ययुक्तोऽप्रमत्तश्च तस्मात्सर्वत्र पूजितः}


\twolineshloka
{नापत्रपश्च युक्तश्च नियुक्तः श्रेयसे परैः}
{अभेत्ता परगुह्यानां तस्मात्सर्वत्र पूजितः}


\twolineshloka
{न हृष्यत्यर्थलाभेषु नालाभे तु व्यथत्यपि}
{स्थिरबुद्धिरसक्तात्मा तस्मात्सर्वत्र पूजितः}


\twolineshloka
{तं सर्वगुणसंपन्नं दक्षं शुचिमनामयम्}
{कालज्ञं च प्रियज्ञं च कः प्रियं न करिष्यति}


\twolineshloka
{`इत्युक्तः संप्रशस्यैनमुग्रसेनो गतो गृहात्}
{आस्ते कृष्णस्तथैकान्ते पर्यङ्के रत्नभूषिते}


\twolineshloka
{कदाचित्तत्र भगवान्प्रविवेश महामुनिः}
{तमभ्यर्च्य यथान्यायं तूष्णीमास्ते जनार्दनः}


\fourlineindentedshloka
{तं खिन्नमिव संलक्ष्य केशवं वाक्यमब्रवीत्}
{किमिदं केशव तव वैमनस्यं जनार्दन}
{अभूतपूर्वं गोविन्द तन्मे व्याख्यातुमर्हसि ॥श्रीवासुदेव उवाच}
{}


\twolineshloka
{नासुहृत्परमं मेऽद्य नापदोऽर्हति वेदितुम्}
{अपण्डितो वापि सुहृत्पण्डितो वाऽप्यनात्मवान्}


\twolineshloka
{स त्वं सुहृच्च विद्वांश्च जितात्मा श्रोतुमर्हसि}
{अप्येतद्धृदि यद्दुःखं तद्भवाञ्श्रोतुमर्हति}


\twolineshloka
{दास्यमैश्वर्यवादेन ज्ञातीनां च करोम्यहम्}
{द्विषन्ति सततं क्रुद्धा ज्ञातिसंबन्धिवान्धवाः}


\threelineshloka
{दिव्या अपि तथा भोगा दत्तास्तेषां मया पृथक्}
{तथाऽपि च द्विषन्तो मां वर्तन्ते च परस्परम् ॥नारद उवाच}
{}


\threelineshloka
{अनायसेन शस्त्रेण परिमृज्यानुमृज्य च}
{जिह्वामुद्धर चैतेषां न वक्ष्यन्ति ततः परम् ॥भगवानुवाच}
{}


\threelineshloka
{अनायसं कथं विन्द्यां शस्त्रं मुनिवरोत्तम}
{येनषामुद्धरे जिह्वां ब्रूहि तन्मे यथातथम् ॥नारद उवाच}
{}


\twolineshloka
{गोहिरण्यं च वासांसि रत्नाद्यं यद्धनं बहु}
{आस्ये प्रक्षिप चैतेषां शस्त्रमेतदनायसम्}


\twolineshloka
{सुहृत्संबन्धिमित्राणां गुरूणां स्वजनस्य च}
{आख्यातं शस्रमेतद्धि तेन च्छिन्धि पुनः पुनः}


\threelineshloka
{तवैश्वर्यप्रदानानि श्लाध्यमेषां वचांसि च}
{समर्थं त्वामभिज्ञाय प्रवदन्ति च ते नराः ॥भीष्म उवाच}
{}


\twolineshloka
{ततः प्रहस्य भगवान्संपूज्य च महामुनिम्}
{तथाऽकरोन्महातेजा मुनिवाक्येन चोदितः}


\twolineshloka
{एवंप्रभावो ब्रह्मर्षिर्नारदो मुनिसत्तमः}
{पृष्टवानसि यन्मां त्वं तदुक्तं राजसत्तम}


\twolineshloka
{सर्वधर्महिते युक्ताः सत्यधर्मपरायणाः}
{लोकप्रियत्वं गच्छन्ति ज्ञानविज्ञानकोविदाः ॥'}


\chapter{अध्यायः २३८}
\twolineshloka
{युधिष्ठिर उवाच}
{}


\twolineshloka
{आद्यन्तं सर्वभूतानां ज्ञातुमिच्छामि कौरव}
{ध्यानं कर्म च कालं च तथैवायुर्युगेयुगे}


\twolineshloka
{लोकतत्त्वं च कार्त्स्न्येन भूतानामागतिं गतिम्}
{सर्गश्च निधनं चैव कुत एतत्प्रवर्तते}


\twolineshloka
{`भेदकं भेदतत्वं च तथाऽन्येषां मतं तथा}
{अवस्थात्रितयं चैव यादृशं च पितामह ॥'}


\twolineshloka
{यदि तेऽनुग्रहे बुद्धिरस्मास्विह सतां वर}
{एतद्भवन्तं पृच्छामि तद्भवान्प्रब्रवीतु मे}


% Check verse!
पूर्वं हि कथितं श्रुत्वा भृगुभाषितमुत्तमम् ॥भरद्वाजस्य विप्रर्षेस्ततो मे बुद्धिरुत्तमा
\threelineshloka
{जाता परमधर्मिष्ठा दिव्यसंस्थानसंस्थिता}
{ततो भूयस्तु पृच्छामि तद्भवान्वक्तुमर्हति ॥भीष्म उवाच}
{}


\twolineshloka
{अत्र ते वर्तयिष्येऽहमितिहासं पुरातनम्}
{जगौ यद्भगवान्व्यासः पुत्राय परिपृच्छते}


\twolineshloka
{अधीत्य वेदानखिलान्साङ्गोपनिषदस्तथा}
{अन्विच्छन्नैष्ठिकं कर्म धर्मनैपुणदर्शनात्}


\threelineshloka
{कृष्णद्वैपायनं व्यासं पुत्रो वैयासकिः शुकः}
{पप्रच्छ संशयमिमं छिन्नधर्मार्थसंशयम् ॥श्रीशुक उवाच}
{}


\twolineshloka
{भूतग्रामस्य र्क्तारं कालज्ञाने च निश्चितम्}
{`ज्ञानं ब्रह्म च योगं च गवात्मकमिदं जगत्}


\twolineshloka
{त्रितये त्वेनमायाति तथा ह्येषोऽपि वा पुनः}
{केनैव च विभागः स्यात्तुरीयो लक्षणैर्विना}


\fourlineindentedshloka
{ज्ञानज्ञेयान्तरे कोसौ कोयं भावस्तु भेदवत्}
{यज्ज्ञानं लक्षणं चैव तेषां कर्तारमेव च}
{'ब्राह्मणस्य च यत्कृत्यं तद्भवान्वक्तुमर्हति ॥भीष्म उवाच}
{}


\twolineshloka
{तस्मै प्रोवाच तत्सर्वं पिता पुत्राय पृच्छते}
{अतीतानागते विद्वान्सर्वज्ञः सर्वधर्मवित्}


\twolineshloka
{`पृच्छतस्तव सत्पुत्र यथावत्कीर्तयाम्यहम्}
{शृणुष्वावहितो भूत्वा यथाऽऽवृतमिदं जगत्}


\twolineshloka
{कार्यादि कारणान्तं यत्कार्यान्तं कारणादिकम्}
{ज्ञानं तदुभयं वित्त्वा सत्यं च परमं शुभम्}


\twolineshloka
{ब्रह्मेति चाभिविख्यातं तद्वै पश्यन्ति सूरयः}
{ब्रह्मतेजोमयं भूतं भूतकारणमद्भुतम्}


\twolineshloka
{आसीदादौ ततस्त्वाहुः प्राधान्यमिति तद्विदः}
{त्रिगुणां तां महामायां वैष्णवीं प्रकृतिं विदुः}


\twolineshloka
{तदीदृशमनाद्यन्तमव्यक्तमजरं ध्रुवम्}
{अप्रतर्क्यमविज्ञेयं ब्रह्माग्रे वैकृतं च तत्}


\twolineshloka
{तद्वै प्रधानमुद्दिष्टं त्रिसूक्ष्मं त्रिगुणात्मकम्}
{सम्यग्योगगुणं स्वस्थं तदिच्छाक्षोभितं महत्}


\twolineshloka
{शक्तित्रयात्मिका तस्य प्रकृतिः कारणात्मिका}
{अस्वतन्त्रा च सततं विदधिष्ठानसंयुता}


\twolineshloka
{स्वभावाख्यं समापन्ना मोहविग्रहधारिणी}
{विविधस्यास्य जीवस्य भोगार्थं समुपागता}


\twolineshloka
{यथा संनिधिमात्रेण गन्धक्षोभाय जायते}
{मनस्तद्वदशेषस्य परात्पर इति स्मृतः}


\twolineshloka
{सृष्ट्वा प्रविश्य तत्तस्मिन्क्षोभयामास विष्ठितः}
{सात्विको राजसश्चैव तामसश्च त्रिधा महान्}


\twolineshloka
{प्रधानतत्वादुद्भूतो महत्वाच्च महान्स्मृतः}
{प्रधानतत्वमुद्भूतं महत्तत्वं समावृणोत्}


\twolineshloka
{कालात्मनाऽभिभूतं तत्कालोंऽशः परमात्मनः}
{पुरुषश्चाप्रमेयात्मा स एव इति गीयते}


% Check verse!
त्रिगुणोसौ महाज्ञातः प्रधान इति वै श्रुतिः
\twolineshloka
{सात्विको राजसश्चैव तामसश्च त्रिधात्मकः}
{त्रिविधोऽयमहङ्कारो महत्तत्वादजायत}


\twolineshloka
{तामसोऽसावहङ्कारो भूतादिरिति संज्ञितः}
{भूतानामादिभूतत्वाद्रक्ताहिस्तामसः स्मृतः}


\twolineshloka
{भूतादिः स विकुर्वाणः शिष्टं तन्मात्रकं ततः}
{ससर्ज शब्दं तन्मात्रमाकाशं शब्दलक्षणम्}


\twolineshloka
{शब्दलक्षणमाकाशं शब्दतन्मात्रमावृणोत्}
{तेन संपीड्यमानस्तु स्पर्शमात्रं ससर्ज ह}


\twolineshloka
{शब्दमात्रं तदाकाशं स्पर्शमात्रं समावृणोत्}
{ससर्ज वायुस्तेनासौ पीड्यमान इति श्रुतिः}


\twolineshloka
{स्पर्शमात्रं तदा वायू रूपमात्रं समावृणोत्}
{तेन संपीड्यमानस्तु ससर्जाग्निमिति श्रुतिः}


\twolineshloka
{रूपमात्रं ततो वह्निं समुत्सृज्य समावृणोत्}
{तेन संपीड्यमानस्तु रसमात्रं ससर्ज ह}


\twolineshloka
{रुपमात्रगतं तेजो रसमात्रं समावृणोत्}
{तेन संपीड्यमानस्तु ससर्जाम्भ इति श्रुतिः}


\twolineshloka
{रसमात्रात्मकं भूयो रसं तन्मात्रमावृणोत्}
{तेन संपीड्यमानस्तु गन्धं तन्मात्रकं ततः}


\twolineshloka
{ससर्ज गन्धं तन्मात्रमावृणोत्करकं ततः}
{तेन संपीड्यमानस्तु काठिन्यं च ससर्ज ह}


% Check verse!
पृथिवी जायते तस्माद्गन्धतन्मात्रजात्तथा
\twolineshloka
{अम्मयं सर्वमेवेदमापस्तस्तम्भिरे पुनः}
{भूतानीमानि जातानि पृथिव्यादीनि वै श्रुतिः}


\twolineshloka
{भूतानां मूर्तिरेवैषामन्नं चैषां मता बुधैः}
{तस्मिंस्तस्मिंस्तु तन्मात्रा तन्मात्रा इति ते स्मृताः}


\twolineshloka
{तैजसानीन्द्रियाण्याहुर्देवा वैकारिका दश}
{एकादशं मनश्चात्र देवा वैकारिकाः स्मृताः}


\threelineshloka
{एषामुद्धर्तकः कालो नानाभेदवदास्थितः}
{परमात्मा च भूतात्मा गुणभेदेन संस्थितः}
{एक एव त्रिधा भिन्नः करोति विविधाः क्रियाः}


\twolineshloka
{ब्रह्मा सृजति भूतानि पाति नारायणोऽव्ययः}
{रुद्रो हन्ति जगन्मूर्तिः काल एष क्रियाबुधः}


\twolineshloka
{कालोपि तन्मयोचिन्त्यस्त्रिगुणात्मा सनातनः}
{अव्यक्तोसावचिन्त्योसौ वर्तते भिन्नलक्षणः}


\threelineshloka
{कालात्मना त्विदं भिन्नमभिन्नं श्रूयते हि यत्}
{अनाद्यन्तमजं दिव्यमव्यक्तमजरं ध्रुवम्}
{'अप्रतर्क्यमविज्ञेयं ब्रह्माग्रे संप्रवर्तते}


\twolineshloka
{काष्ठा निमेषा दश पञ्च चैवत्रिंशत्तु काष्ठा गणयेत्कलां ताम्}
{त्रिंशत्कलश्चापि भवेन्मुहूर्तोभागः कलाया दशमश्च यः स्यात्}


\twolineshloka
{त्रिंशन्मुहूर्तं तु भवेदहश्चरात्रिश्च सङ्ख्या मुनिभिः प्रणीता}
{मासः स्मृतो रात्र्यहनी च त्रिंशुत्संवत्सरो द्वादशमास उक्तः}


% Check verse!
संवत्सरं द्वे अयने वदन्तिसङ्ख्याविदो दक्षिणमुत्तरं च
\twolineshloka
{पहोरात्रौ विभजते सूर्यो मानुषलौकिकौ}
{रात्रिः स्वप्नाय संयाति चेष्टायै कर्मणामहः}


\twolineshloka
{पित्र्ये रात्र्यहनी मासः प्रविभागस्तयोः पुनः}
{शुक्लोऽहः कर्मचेष्टायां कृष्णः स्वप्नाय शर्वरी}


\twolineshloka
{दैवे रात्र्यहनी ह्यब्दः प्रविभागस्तयोः पुनः}
{अहस्तत्रोदगयनं रात्रिः स्याद्दक्षिणायनम्}


\twolineshloka
{ये ते रात्र्यहनी पूर्वं कीर्तिते दैवलौकिके}
{तयोः सङ्ख्याय वर्षाग्रं ब्राह्ने वक्ष्याम्यहः क्षपे}


\twolineshloka
{तेषां संवत्सराग्नाणि प्रवक्ष्याम्यनुपूर्वशः}
{कृते त्रेतायुगे चैव द्वापरे च कलौ तथा}


\twolineshloka
{चत्वार्याहुः सहस्राणि वर्षाणां तत्कृतं युगम्}
{तस्य तावच्छती संध्या संध्यांशश्च तथाविधः}


\twolineshloka
{इतरेषु ससंध्येषु संध्यांशेषु ततस्त्रिषु}
{एकापायेन संयान्ति सहस्राणि शतानि च}


\twolineshloka
{एतानि शाश्वताँल्लोकान्धारयन्ति सनातनान्}
{एतद्ब्रह्मविदां तात विदितं ब्रह्म शाश्वतम्}


\twolineshloka
{चतुष्पात्सकलो धर्मः सत्यं चैव कृते युग}
{नाधर्मेणागमः कश्चिद्युगे तस्मिन्प्रवर्तते}


\threelineshloka
{इतरेष्वागमाद्धर्मः पादशस्त्ववरोप्यते}
{`सत्यं शौत्रं तथायुश्च धर्मश्चापैति पादशः}
{'चौर्यकानृतमायाभिरधर्मश्चोपचीयते}


\twolineshloka
{अरोगाः सर्वसिद्धार्थाश्चतुर्वर्षशतायुषः}
{कृते त्रेतायुगे त्वेषां पादशो ह्रसते वयः}


\twolineshloka
{वेदवादाश्चानुयुगं ह्रसन्तीतीह न श्रुतम्}
{आयूंषि चाशिषश्चैव वेदस्यैव च यत्फलम्}


\twolineshloka
{अन्ये कृतयुगे धर्मास्त्रेतायां द्वापरेऽपरे}
{अन्ये कलियुगे धर्मा यथाशक्ति कृता इव}


\twolineshloka
{तपः परं कृतयुगे त्रेतायां सत्यमुत्तमम्}
{द्वापरे यज्ञमेवाहुर्दानमेव कलौ युगे}


\twolineshloka
{एतां द्वादशसाहस्त्रीं युगाख्यां कवयो विदुः}
{सहस्रपरिवर्तं तद्ब्राह्मं दिवसमुच्यते}


\twolineshloka
{रात्रिस्तु तावती ब्राह्मी तदादौ विश्वमीश्वरः}
{प्रलयेत्मानमाविश्य सुप्त्वासोऽन्ते विबुध्यते}


\twolineshloka
{सहस्रयुगपर्यन्तमहर्यद्ब्रह्मणे विदुः}
{रात्रिं युगसहस्रां तां तेऽहोरात्रविदो जनाः}


\twolineshloka
{प्रतिबुद्धो विकुरुते ब्रह्माक्षय्यं क्षपाक्षये}
{सृजते च महद्भूतं तस्माद्व्यक्तात्मकं मनः}


\twolineshloka
{मनः सृष्टिं विकुरुते चोद्यमानं सिसृक्षया}
{आकाशं जायते तस्मात्तस्य शब्दे गुणो मतः}


\twolineshloka
{आकाशात्तु विकुर्वाणात्सर्वगन्धवहः शुचिः}
{बलवाञ्जायते वायुस्तस्य स्पर्शो गुणो मतः}


\twolineshloka
{वायोरपि विकुर्वाणाज्ज्योतिर्भवति भास्वरम्}
{रोचनं जनयेच्छुद्धं तद्रूपगुणमुच्यते}


\twolineshloka
{ज्योतिषोपि विकुर्वाणाद्भवन्त्यापो रसात्मिकाः}
{अद्भ्यो गन्धवहा भूमिः पूर्वेषां सृष्टिरुच्यते}


\twolineshloka
{`ब्रह्मतेजोमयं शुक्लं यस्य सर्वमिदं जगत्}
{एकस्य ब्रह्मभूतस्य द्वयं स्थावरजङ्गमम्}


\twolineshloka
{अहर्मुखे विवुद्धं तत्सृजते विद्यया जगत्}
{अग्र एव महद्भूतमाशु व्यक्तात्मकं मनः}


\twolineshloka
{अभिभूयेह चातिष्ठद्व्यसृदत्सप्त मानसान्}
{दूरगं बहुधागामि प्रार्थनासंशयात्मकम्}


\twolineshloka
{मनः सृष्टिं न कुरुते चोद्यमानं सिसृक्षया}
{आकाशोजायते तस्मात्तस्य शब्दो गुणो मतः}


\twolineshloka
{आकाशात्तु विकुर्वाणात्सर्वगन्धवहः शुचिः}
{बलवाञ्जायते वायुस्तस्य स्पर्शगुणं विदुः}


\twolineshloka
{वायोरपि विकुर्वाणाज्ज्योतिर्भूतं तमोनुदम्}
{रोचिष्णुर्जायते तत्र तद्रूपगुणमुच्यते}


\twolineshloka
{ज्योतिषोपि विकुर्वाणाद्भवन्त्यापो रसात्मिकाः}
{अद्भ्यो गन्धवहा भूमिः पूर्वेषां सृष्टिरुच्यते ॥'}


\twolineshloka
{गुणाः पूर्वस्य पूर्वस्य प्राप्नुवन्त्युत्तरोत्तरम्}
{तेषां यावद्गुणं यद्यत्तत्तावद्गुणकं स्मृतम्}


\twolineshloka
{उपलभ्याप्सु चेद्गन्धं केचिद्ब्रयुरनैपुणात्}
{पृथिव्यामेव तं विद्यादपां वायोश्च संश्रितम्}


\twolineshloka
{एते सप्तविधात्मानो नानावीर्याः पृथक्पृथक्}
{नाशक्नुवन्प्रजाः स्रष्टुभसमागम्य कृत्स्नशः}


\twolineshloka
{ते समेत्य महात्मानो ह्यन्योन्यमभिसंश्रिताः}
{शरीराश्रयणं प्राप्तास्ततः पुरुष उच्यते}


\twolineshloka
{श्रयणाच्छरीरी भवति मूर्तिमान्षोडशात्मकः}
{तमाविशन्ति भूतानि महान्ति सह कर्मणा}


\twolineshloka
{सर्वभूतान्युपादाय तपसश्चरणाय हि}
{आदिकर्ता महाभूतं तमेवाहुः प्रजापतिम्}


\twolineshloka
{स वै सृजति भूतानि स एव पुरुषः परः}
{अजो जनयते ब्रह्मा देवर्षिपितृमानवान्}


\threelineshloka
{लोकान्नदीः समुद्रांश्च दिशः शैलान्वनस्पतीन्}
{नरकिन्नररक्षांसि वयः पशुमृगोरगान्}
{अव्ययं च व्ययं चैव द्वयं स्थावरजङ्गमम्}


\twolineshloka
{तेषां ये यानि कर्माणि प्राक्सृष्ट्यां प्रतिपेदिरे}
{तान्येव प्रतिपाद्यन्ते सृज्यमानाः पुनः पुनः}


\twolineshloka
{हिंस्राहिंस्रे मृदुक्रूरे धर्माधर्मावृत्तानृते}
{तद्भाविताः प्रपद्यन्ते तस्मात्तत्तस्य रोचते}


\twolineshloka
{महाभूतेषु नानात्वमिन्द्रियार्थेषु मूर्तिषु}
{विनियोगं च भूतानां धातैव विदधात्युत}


\twolineshloka
{केचित्पुरुषकारं तु प्राहुः कर्मविदो जनाः}
{दैवमित्यपरे विप्राः स्वभावं भूतचिन्तकाः}


\twolineshloka
{पौरुषं कर्म दैवं च फलवृत्तिस्वभावतः}
{त्रय एतेऽपृथग्भूता अविवेकः कथंचन}


\twolineshloka
{एवमेतच्च दैवं च द्भूतं सृजते जगत्}
{कर्मस्था विषयं ब्रूयुः सत्वस्थाः समदर्शिनः}


\twolineshloka
{ततो निःश्रेयसं जन्तोस्तस्य मूलं शमो दमः}
{तेन सर्वानवाप्नोति यान्कामान्मनसेच्छति}


\twolineshloka
{तपसा तदवाप्नोति यद्भूतं सृजते जगत्}
{स तद्भूतश्च सर्वेषां भूतानां भवति प्रभुः}


\twolineshloka
{ऋषयस्तपसा वेदानध्यैषन्त दिवानिशम्}
{अनादिनिधना नित्या वागुत्सृष्टा स्वयंभुवा}


\twolineshloka
{ऋषीणां नामधेयानि याश्च वेदेषु सृष्टयः}
{नाम रूपं च भूतानां कर्मणां च प्रवर्तनम्}


\threelineshloka
{वेदशब्देभ्य एवादौ निर्मिमीते स ईश्वरः}
{नामधेयानि चर्षीणां याश्च वेदेषु सृष्टयः}
{शर्वर्यन्तेषु जातानामन्येभ्यो विदधात्यजः}


\twolineshloka
{नामभेदतपः कर्मयज्ञाख्या लोकसिद्धये}
{आत्मसिद्धिस्तु वेदेषु प्रोच्यते दशभिः क्रमैः}


\twolineshloka
{यदुक्तं वेदवादेषु गहनं वेददृष्टिभिः}
{तदन्तेषु यथायुक्तं क्रमयोगेन लक्ष्यते}


\twolineshloka
{कर्मजोऽयं पृथग्भावो द्वन्द्वयुक्तो हि देहिनः}
{आत्मसिद्धिस्तु विज्ञानाज्जहाति प्रायशो बलम्}


\twolineshloka
{द्वे ब्रह्मणी वेदितव्ये शब्दब्रह्म परं च यत्}
{शब्दब्रह्मणि निष्णातः परं ब्रह्माधिगच्छति}


\twolineshloka
{आलम्भयज्ञाः क्षत्राश्च हविर्यज्ञा विशः स्मृताः}
{परिचारयज्ञाः शूद्रास्तु तपोयज्ञा द्विजातयः}


\twolineshloka
{त्रेतायुगे विधिस्त्वेष यज्ञानां न कृते युगे}
{द्वापरे विप्लवं यान्ति यज्ञाः कलियुगे तथा}


\twolineshloka
{अपृथग्धर्मिणो मर्त्या ऋक्सामानि यजूंषि च}
{काम्या इष्टीः पृथक्दृष्ट्वा तपोभिस्तप एव च}


\twolineshloka
{त्रेतायां तु समस्ता ये प्रादुरासन्महाबलाः}
{संयन्तारः स्थावराणां जङ्गमानां च सर्वशः}


\twolineshloka
{त्रेतायां संहता वेदा यज्ञा वर्णास्तथैव च}
{संरोधादायुषस्त्वेते व्यस्यन्ते द्वापरे युगे}


\twolineshloka
{दृश्यन्ते न च दृश्यन्ते वेदाः कलियुगेऽखिलाः}
{उत्सीदन्ते सयज्ञाश्च केवला धर्मपीडिताः}


\twolineshloka
{कृते युगे यस्तु धर्मो ब्राह्मणेषु प्रदृश्यते}
{आत्मवत्सु तपोवत्सु श्रुतवत्सु प्रतिष्ठितः}


\twolineshloka
{स धर्मः प्रैति संयोगं यथाधर्मं युगेयुगे}
{विक्रियन्ते स्वधर्मस्था वेदवादा यथागमम्}


\twolineshloka
{यथा विश्वानि भूतानि वृष्ट्या भूयांसि प्रावृषि}
{सृज्यन्ते जङ्गमस्थानि तथा धर्मा युगेयुगे}


\twolineshloka
{यथर्तुष्वृतुलिङ्गानि नानारूपाणि पर्यये}
{दृश्यन्ते तानि तान्येव तथा ब्रह्महरादिषु}


\twolineshloka
{विहितं कालनानात्वमनादिनिधनं तथा}
{कीर्तितं यत्पुरस्तात्ते तत्सूते चाति च प्रजाः}


\twolineshloka
{ददाति भवनस्थानं भूतानां संयमो यमः}
{स्वभावेनैव वर्तन्ते द्वन्द्वयुक्तानि भूरिशः}


\twolineshloka
{सर्वकालक्रिया वेदाः कर्ता कार्यं क्रियाफलम्}
{प्रोक्तं ते पुत्र सर्वं वै यन्मां त्वं परिपृच्छसि}


\chapter{अध्यायः २३९}
\twolineshloka
{व्यास उवाच}
{}


\twolineshloka
{प्रत्याहारं तु वक्ष्यामि शर्वर्यादौ गतेऽहनि}
{यथेदं कुरुतेऽध्यात्मं सुसूक्ष्मं विश्वमीश्वरः}


\twolineshloka
{दिवि सूर्यस्तथा सप्त दहन्ति शिखिनोऽर्चिषः}
{सर्वमेतत्तदाऽर्चिर्भिः पूर्णं जाज्वल्यते जगत्}


\twolineshloka
{पृथिव्यां यानि भूतानि जङ्गमानि ध्रुवाणि च}
{तान्येवाग्रे प्रलीयन्ते भूमित्वमुपयान्ति च}


\twolineshloka
{ततः प्रलीने सर्वास्मिन्स्थावरे जङ्गमे तथा}
{अकाष्ठा निस्तृणा भूमिर्दृश्यते कूर्मपृष्ठवत्}


\twolineshloka
{भूमेरपि गुणं गन्धमाप आददते यदा}
{आत्तगन्धा तदा भूमिः प्रलयत्वाय कल्पते}


\twolineshloka
{आपस्तत्र प्रतिष्ठन्ते ऊर्मिमत्यो महास्वनाः}
{सर्वमेवेदमापूर्य तिष्ठन्ति च चरन्ति च}


\twolineshloka
{अपामपि गुणांस्तात ज्योतिराददते यदा}
{आपस्तदा त्वात्तगुणा ज्योतिः षूपरमन्ति वै}


\twolineshloka
{यदाऽऽदित्यं स्थितं मध्ये गूहन्ति शिखिनोऽर्चिषः}
{सर्वमेवेदमर्चिर्भिः पूर्णं जाज्वल्यते नमः}


\twolineshloka
{ज्योतिषोऽपि गुणं रूपं वायुराददते यदा}
{प्रशाम्यति ततो ज्योतिर्वायुर्दोधूयते महान्}


\twolineshloka
{ततस्तु मूलमासाद्य वायुः संभवमात्मनः}
{अधश्चोर्ध्वं च तिर्यक्च दोधवीति दिशो दश}


\twolineshloka
{वायोरपि गुणं स्पर्शमाकाशं ग्रसते यदा}
{प्रशाम्यति तदा वायुः खं तु तिष्ठति नानदन्}


\twolineshloka
{अरूपमरसस्पर्शमगन्धं न च मूर्तिमत्}
{सर्वलोकप्रणुदितं स्वं तु तिष्ठति नानदत्}


\threelineshloka
{आकाशस्य गुणं शब्दमभिव्यक्तात्मकं मनः}
{`ग्रसते च यदा सोऽपि शाम्यति प्रतिसंचरे}
{'मनसो व्यक्तमव्यक्तं ब्राह्मः संप्रतिसंचरः}


\twolineshloka
{तदाऽऽत्मगुणमाविश्य मनो ग्रसति चन्द्रमाः}
{मनस्युपरते चात्मा चन्द्रमस्युपतिष्ठते}


\twolineshloka
{तं तु कालेन महता संकल्पं कुरुते वशे}
{चित्तं ग्रसति संकल्पस्तच्च ज्ञानमनुत्तमम्}


\twolineshloka
{कालो गिरति विज्ञानं कालं बलमिति श्रुतिः}
{बलं कालो ग्रसति तु तं विद्वान्कुरुते वशे}


\threelineshloka
{[आकाशस्य तदा घोषं तं विद्वान्कुरुतेऽऽत्मनि}
{]तदव्यक्तं परं ब्रह्म तच्छाश्वतमनुत्तमम्}
{एवं सर्वाणि भूतानि ब्रह्मैव प्रतिसंहरेत्}


\twolineshloka
{यथावत्कीर्तितं सत्यमेवमेतदसंशयम्}
{बोध्यं विद्यामयं दृष्ट्वा योगिभिः परमात्मभिः}


\twolineshloka
{एवं विस्तारसंक्षेपौ ब्रह्माव्यक्ते पुनःपुनः}
{युगसाहस्रयोरादावहोरात्र्यास्तथैव च}


\chapter{अध्यायः २४०}
\twolineshloka
{व्यास उवाच}
{}


\twolineshloka
{भूतग्रामे नियुक्तं यत्तदेतत्कीर्तितं मया}
{ब्राह्मणस्य तु यत्कृत्यं तत्ते वक्ष्यामि सांप्रतम्}


\twolineshloka
{जातकर्मप्रभृत्यस्य कर्मणां दक्षिणावताम्}
{क्रिया स्यादासमावृत्तेराचार्ये वेदपारगे}


\twolineshloka
{अधीत्य वेदानखिलान्गुरुशुश्रूषणे रतः}
{गुरूणामनृणो भूत्वा समावर्तेत यज्ञवित्}


\twolineshloka
{आचार्येणाभ्यनुज्ञातश्चतुर्णामेकमाश्रमम्}
{आविमोक्षाच्छरीरस्य सोऽवतिष्ठेद्यथाविधि}


\twolineshloka
{प्रजासर्गण दारैश्च ब्रह्मचर्येण वा पुनः}
{वने गुरुसकाशे वा यतिधर्मेण वा पुनः}


\twolineshloka
{गृहस्थस्त्वेष धर्माणां सर्वेषां मूलमुच्यते}
{यत्र पक्वकषायो हि दान्तः सर्वत्र सिध्यति}


\twolineshloka
{प्रजावान्श्रोत्रियो यज्वा मुक्त एव ऋणैस्त्रिभिः}
{अथान्यानाश्रमान्पश्चात्पूतो गच्छेत कर्मभिः}


\twolineshloka
{यत्पृथिव्यां पुण्यतमं विद्यात्स्थानं तदावसेत्}
{यतेत तस्मिन्प्रामाण्यं गन्तुं यशसि चोत्तमे}


\twolineshloka
{तपसा वः सुमहता विद्यानां पारणेन वा}
{इज्यया वा प्रदानैर्वा विप्राणां वर्धते यशः}


\twolineshloka
{यावदस्य भवत्यस्मिन्कीर्तिर्लोके यशस्करी}
{तावत्पुण्यकृताँल्लोकाननन्तान्पुरुषोऽश्नुते}


\twolineshloka
{अध्यापयेदधीयीत याजयेत यजेत वा}
{न वृथा प्रतिगृह्णीयान्न च दद्यात्कथंचन}


\twolineshloka
{याज्यतः शिष्यतो वाऽपि कन्याया वा धनं महत्}
{यद्यागच्छेद्यजेद्दद्यान्नैकोऽश्नीयात्कथंचन}


\twolineshloka
{गृहमावसतो ह्यस्य नान्यत्तीर्थमुदाहृतम्}
{देवर्षिपितृगुर्वर्थं वृद्धातुरबुभुक्षताम्}


\twolineshloka
{अन्तर्हिताभितप्तानां यथाशक्ति बुभूषताम्}
{द्रव्याणामतिशक्त्याऽपि देयमेषां कृतादपि}


\threelineshloka
{अर्हतामनुरूपाणां नादेयं ह्यस्ति किंचन}
{उच्चैः श्रवसमप्यक्षं काश्यपाय महात्मने}
{दत्त्वा जगाम प्रह्लादो लोकान्देवैरभिष्टुतान्}


\twolineshloka
{अनुनीय तथा काव्यः सत्यसन्धो महाव्रतः}
{स्वैः प्राणैर्ब्राह्मणप्राणान्परित्राय दिवं गतः}


\twolineshloka
{रन्तिदेवश्च सांकृत्यो वसिष्ठाय महात्मने}
{अपः प्रदाय शीतोष्णा नाकपृष्ठे महीयते}


\twolineshloka
{आत्रेयश्चेन्द्रद्रुमये ह्यर्हते विविधं धनम्}
{दत्त्वा लोकान्ययौ धीमाननन्तान्स महीपतिः}


\twolineshloka
{शिबिरौशीनरोऽङ्गानि सुतं च प्रियमौरसम्}
{ब्राह्मणार्थमुपाकृत्य नाकपृष्ठमितो गतः}


\twolineshloka
{प्रतर्दनः काशिपतिः प्रदाय नयने स्वके}
{ब्राह्मणायातुलां कीर्तिमिह चामुत्र चाश्नुते}


\twolineshloka
{दिव्यमष्टशलाकं तु सौवर्णं परमर्द्धिमत्}
{छत्रं देवावृधो दत्त्वा सराष्ट्रोऽभ्यगमद्दिवम्}


\twolineshloka
{सांकृतिश्च तथाऽऽत्रेयः शिष्येभ्यो ब्रह्म निर्गुणम्}
{उपदिश्य महातेजा गतो लोकाननुत्तमान्}


\twolineshloka
{अम्बरीषो गवां दत्त्वा ब्राह्मणेभ्यः प्रतापवान्}
{अर्बुदानि दशैकं च सराष्ट्रोऽभ्यगमद्दिवम्}


\twolineshloka
{सावित्री कुण्डले दिव्ये शरीरं जनमेजयः}
{ब्राह्मणार्थे परित्यज्य जग्मतुर्लोकमुत्तमम्}


\twolineshloka
{सर्वरत्नं वृषादर्विर्युवनाश्वः प्रियाः स्त्रियः}
{रम्यमावसथं चैव दत्त्वामुं लोकमास्थितः}


\twolineshloka
{निमी राष्ट्रं च वैदेहो जामदग्न्यो वसुंधराम्}
{ब्राह्मणेभ्यो ददौ चापि गयश्चोर्वी सपत्तनाम्}


\twolineshloka
{अवर्षति च पर्जन्ये सर्वभूतानि भूतकृत}
{वसिष्ठो जीवयामास प्रजापतिरिव प्रजाः}


\twolineshloka
{करंधमस्य पुत्रस्तु मरुतो नृपतिस्तथा}
{कन्यामङ्गिरसे दत्त्वा दिवमाशु जगाम ह}


\twolineshloka
{ब्रह्मदत्तश्च पाञ्चाल्यो राजा बुद्धिमतां वरः}
{निधिं शङ्खं द्विजाग्र्येभ्यो दत्त्वा लोकानवाप्तवान्}


\twolineshloka
{राजा मित्रसहश्चापि वसिष्ठाय महात्मने}
{मदयन्तीं प्रियां दत्त्वा तया सह दिवं गतः}


\twolineshloka
{सहस्रजिच्च राजर्षिः प्राणानिंष्टान्महायशाः}
{ब्राह्मणार्थं परित्यज्य गतो लोकाननुत्तमान्}


\twolineshloka
{सर्वकामैश्च संपूर्णं दत्त्वा वेश्म हिरण्मयम्}
{मुद्गलाय गतः स्वर्गं शतद्युम्नो महीपतिः}


\twolineshloka
{नाम्ना च द्युतिमान्नाम साल्वराजः प्रतापवान्}
{दत्त्वा राज्यमृचीकाय गतो लोकाननुत्तमान्}


\twolineshloka
{लोमपादश्च राजर्षिः शान्तां दत्त्वा सुतां प्रभुः}
{ऋश्यशृङ्गाय विपुलैः सर्वकामैरयुज्यत}


\twolineshloka
{मदिराश्वश्च राजर्षिर्दत्त्वा कन्यां सुमध्यमाम्}
{हिरण्यहस्ताय गतो लोकान्देवैरभिष्टुतान्}


\twolineshloka
{दत्त्वा शतसहस्रं तु गवां राजा प्रसेनजित्}
{सवत्सानां महातेजा गतो लोकाननुत्तमान्}


\twolineshloka
{एते चान्ये च बहवो दानेन तपसैव च}
{महात्मानो गताः स्वर्गं शिष्टात्मानो जितेन्द्रियाः}


\twolineshloka
{तेषां प्रतिष्ठिता कीर्तिर्यावत्स्थास्यति मेदिनी}
{दानयज्ञप्रजासर्गैरेते हि दिवमाप्नुवन्}


\chapter{अध्यायः २४१}
\twolineshloka
{व्यास उवाच}
{}


\twolineshloka
{त्रयीं विद्यामवेक्षेत वेदेपूत्तमतां गतः}
{ऋक्सामवर्णाक्षरतो यजुषोऽथर्वणस्तथा}


\twolineshloka
{[तिष्ठत्येतेषु भगवान्षट््सु कर्मसु संस्थितः}
{]वेदवादेषु कुशला ह्यध्यात्मकुशलाश्च ये}


\twolineshloka
{सत्ववन्तो महाभागाः पश्यन्ति प्रभवाप्ययौ}
{एवं धर्मेण वर्तेत क्रियाः शिष्टवदाचरेत्}


\twolineshloka
{असंरोधेन भूतानां वृत्तिं लिप्सेत वै द्विजः}
{सद्भ्य आगतविज्ञानः शिष्टः शास्त्रविचक्षणः}


\twolineshloka
{स्वधर्मेण क्रिया लोके कुर्वाणः सोऽप्यसङ्करः}
{तिष्ठते तेषु गृहवान्षट््सु कर्मसु स द्विजः}


\threelineshloka
{पञ्चभिः सततं यज्ञैः श्रद्दधानो यजेत च}
{धृतिमानप्रमत्तश्च दान्तो धर्मविदात्मवान्}
{वीतहर्षमदक्रोधो ब्राह्मणो नावसीदति}


\twolineshloka
{दानमध्ययनं यज्ञस्तपो ह्रीरार्जवं दमः}
{एतैर्विवर्धते तेजः पाप्मानं चापकर्षति}


\twolineshloka
{धूतपाप्मा च मेधावी लघ्वाहारो जितेन्द्रियः}
{कामक्रोधौ वशे कृत्वा निनीषेद्ब्रह्मणः पदम्}


\twolineshloka
{अग्नींश्च ब्राह्मणांश्चार्चेद्देवताः प्रणमेत च}
{वर्जयेदुशतीं वाचं हिंसां चाधर्मसंहिताम्}


\twolineshloka
{एषा पूर्वतरा वृत्तिर्ब्राह्मणस्य विधीयते}
{ज्ञानागमेन कर्माणि कुर्वन्कर्मसु सिद्ध्यति}


\twolineshloka
{पञ्चेन्द्रियजलां घोरां लोभकूलां सुदुस्तराम्}
{मन्युपङ्कामनाधृष्यां नदीं तरति बुद्धिमान्}


% Check verse!
कालमभ्युद्यतं पश्येन्नित्यमत्यन्तमोहनम्
\twolineshloka
{महता विधिदृष्टेन बलिनाऽप्रतिघातिना}
{स्वभावस्रोतसा वृत्तमुह्यते सततं जगत्}


\twolineshloka
{कालोदकेन महता वर्षावर्तेन संततम्}
{मासोर्मिणर्तुवेगेन पक्षोलपतृणेन च}


\twolineshloka
{निर्मषोन्मेषफेनेन अहोरात्रजवेन च}
{कामग्राहेण घोरेण वेदयज्ञप्लवेन च}


\twolineshloka
{धर्मद्वीपेन भूतानां चार्थकामरवेण च}
{ऋतवाङ्भोक्षतीरेण विहिंसातरुवाहिना}


\twolineshloka
{युगह्रदौघमध्येन ब्रह्मप्रायभवेन च}
{धात्रा सृष्टानि भूतानि कृष्यन्ते यमसादनम्}


\twolineshloka
{एतत्प्रज्ञामयैर्धीरा निस्तरन्ति मनीषिणः}
{प्लवैरप्लववन्तो हि किं करिष्यन्त्यचेतसः}


\twolineshloka
{उपपन्नं हि यत्प्राज्ञो निस्तरेन्नेतरो जनः}
{दूरतो गुणदोषौ हि प्राज्ञः सर्वत्र पश्यति}


\twolineshloka
{संशयात्तु स कामात्मा चलचित्तोऽल्पचेतनः}
{अप्राज्ञो न तरत्येनं यो ह्यास्ते न स गच्छति}


\twolineshloka
{अप्लवो हि महादोषं मुह्यमानो न गच्छति}
{कामग्राहगृहीतस्य ज्ञानमप्यस्य न प्लवः}


\twolineshloka
{तस्मादुन्मज्जनस्यार्थे प्रयतेत विचक्षणः}
{एतदुन्मज्जनं तस्य यदयं ब्राह्मणो भवेत्}


\twolineshloka
{त्र्यवदाते कुले जातस्त्रिसंदेहस्त्रिकर्मकृत्}
{तस्मादुन्मज्जनं तिष्ठेत्प्रज्ञया निस्तरेद्यथा}


\twolineshloka
{संस्कृतस्य हि दान्तस्य नियतस्य यतात्मनः}
{प्राज्ञस्यानन्तरा सिद्धिरिह लोके परत्र च}


\twolineshloka
{वर्तेत तेषु गृहवानक्रुध्यन्ननसूयकः}
{पञ्चभिः सततं यज्ञैर्विघसाशी यजेत च}


\twolineshloka
{सतां धर्मेण वर्तेत क्रियां शिष्टवदाचरेत्}
{असंरोधेन लोकस्य वृत्तिं लिप्सेदगर्हिताम्}


\twolineshloka
{श्रुतविज्ञानतत्त्वज्ञः शिष्टाचारविचक्षणः}
{स्वधर्मेण क्रियावांश्च कर्मणा सोऽप्यसंकरः}


\twolineshloka
{क्रियावाञ्श्रद्दधानो कहि दान्तः प्रायोऽनसूयकः}
{धमार्धर्मविशेषज्ञः सर्वं तरति दुस्तरम्}


\twolineshloka
{धृतिमानप्रमत्तश्च दान्तो धर्मविदात्मवान्}
{वीतहर्षमदक्रोधो ब्राह्मणो नावसीदति}


\twolineshloka
{एषा पुरातनी वृत्तिर्ब्राह्मणस्य विधीयते}
{ज्ञानवृद्ध्यैव कर्माणि कुर्वन्सर्वत्र सिध्यति}


\twolineshloka
{अधर्मं धर्मकामो हि करोति ह्यविचक्षणः}
{धर्मं वा धर्मसंकाशं शोचन्निव करोति सः}


\twolineshloka
{धर्मं करोमीति करोत्यधर्ममधर्मकामश्च करोति धर्मम्}
{उभे बालः कर्मणी न प्रजानम्संजायते म्रियते चापि देही}


\chapter{अध्यायः २४२}
\twolineshloka
{व्यास उवाच}
{}


\twolineshloka
{अथ चेद्रोचयेदेतदुह्यते मनसा तथा}
{उन्मज्जंश्च निमज्जंश्च ज्ञानवान्प्लववान्भवेत्}


\twolineshloka
{प्रज्ञया निर्मितैर्धीरास्तारयन्त्यबुधान्प्लवैः}
{नाबुधास्तारयन्त्यन्यानात्मानं वा कथंचन}


\twolineshloka
{छिन्नदोषो मुनिर्योगयुक्तो युञ्जीत द्वादश}
{दशकर्मसुखानर्थानुपायापायनिष्क्रियः}


\twolineshloka
{चक्षुराचारसंग्राहैर्मनसा दर्शनेन च}
{यच्छेद्वाङ्भनसी बुद्ध्या य इच्छेज्ज्ञानमुत्तमम्}


\twolineshloka
{ज्ञानेन यच्छेदात्मानं य इच्छेच्छान्तिमात्मनः}
{एतेषां चेदनुद्रष्टा पुरुषोऽपि सुदारुणः}


\twolineshloka
{यदि वा सर्ववेदज्ञो यदि वाऽप्यनृचो द्विजः}
{यदि वा धार्मिको यज्वा यदि वा पापकृत्तमः}


\twolineshloka
{यदि वा पुरुषव्याघ्रो यदि वैक्लव्यधारणः}
{तरत्येवं महादुर्गं जरामरणसागरम्}


\twolineshloka
{एवं ह्येतेन योगेन युञ्जानो ह्येवमन्ततः}
{अपि जिज्ञासमानोऽपि शब्दब्रह्माऽतिवर्तते}


\twolineshloka
{धर्मोपस्थो ह्रीवरूथ उपायापायकूवरः}
{अपानाक्षः प्राणयुगः प्रज्ञायुर्जीववन्धनः}


\twolineshloka
{चेतनाबन्धुरश्चारुश्चाचारग्रहनेमिमान्}
{दर्शनस्पर्शनवहो घ्राणश्रवणवाहनः}


\twolineshloka
{प्रज्ञानाभिः सर्वतन्त्रप्रतोदो ज्ञानसारथिः}
{क्षेत्रज्ञाधिष्ठितो धीरः श्रद्धादमपुरः सरः}


\twolineshloka
{त्यागरश्म्यनुगः क्षेम्यः शौचगो ध्यानगोतरः}
{जीवयुक्तो रथो दिव्यो ब्रह्मलोके धिराजते}


\twolineshloka
{अथ संत्वरमाणस्य रथमेवं युयुक्षतः}
{अक्षरं गन्तुमनसो विधिं वक्ष्यामि शीघ्रगम्}


\twolineshloka
{सप्त यो धारणाः कृत्स्ना वाग्यतः प्रतिपद्यते}
{पृष्ठतः पार्श्वतश्चान्यास्तावत्यस्ताः प्रधारणाः}


\threelineshloka
{क्रमशः पार्थिवं यच्च वायव्यं खं तथा पयः}
{ज्योतिषो यत्तदैश्वर्यमहंकारस्य बुद्धितः}
{अव्यक्तस्य तथैश्वर्यं क्रमशः प्रतिपद्यते}


\twolineshloka
{विक्रमाश्चापि यस्यैते तथा युङ्क्ते स योगतः}
{तथाऽस्य योगयुक्तस्य सिद्धिमात्मनि पश्यतः}


\twolineshloka
{निर्मुच्यमानः सूक्ष्मत्वाद्रूपाणीमानि पश्यतः}
{शैशिरस्तु यथा धूमः सूक्ष्मः संश्रयते नभः}


\twolineshloka
{तथा देहाद्विमुक्तस्य पूर्वरूपं भवत्युत}
{अथ धूमस्य विरमेद्द्वितीयं रूपदर्शनम्}


\twolineshloka
{जलरूपमिवाकाशे तत्रैवात्मनि पश्यति}
{अपां व्यतिक्रमे चास्य वह्निरूपं प्रकाशते}


\twolineshloka
{तस्मिन्नुपरते चास्य वायव्यं सूक्ष्ममव्ययम्}
{रूपं प्रकाशते तस्य पीतवस्त्रवदव्ययम्}


\threelineshloka
{तस्मिन्नुपरते रुपमाकाशस्य प्रकाशते}
{तस्मिन्नुपरते चास्य बुद्धिरूपं प्रकाशते}
{ऊर्णारूपसवर्णस्य तस्य रूपं प्रकाशते}


\twolineshloka
{अथ श्वेतां गतिं गत्वा सोहङ्कारे प्रकाशते}
{सुशुक्लं चेतसः सौक्ष्म्यमप्युक्तं ब्राह्मणस्य वै}


\twolineshloka
{एतेष्वपि हि जातेषु फलजातानि मे शृणु}
{जातस्य पार्थिवैश्वर्यैः सृष्टिरिष्टा विधीयते}


\twolineshloka
{प्रजापतिरिवाक्षोभ्यः शरीरात्सृजते प्रजाः}
{अङ्गुल्यङ्गुष्ठमात्रेण हस्तपादेन वा तथा}


\threelineshloka
{पृथिवीं कम्पयत्येको गुणो वायोरिति श्रुतिः}
{आकाशभूतश्चाकाशे सवर्णत्वात्प्रकाशते}
{वर्णतो गृह्यते चाप्सु नापः पिबति चाशया}


\twolineshloka
{न चास्य तेजसां रूपं दृश्यते शाम्यते तथा}
{अहंकारेऽस्य विजिते पञ्चैते स्युर्वशानुगाः}


\twolineshloka
{षण्णामात्मनि बुद्धौ च जितायां प्रभवत्यथ}
{निर्दोषा प्रतिभा ह्येनं कृत्स्ना समभिवर्तते}


\twolineshloka
{तथैव व्यक्तमात्मानमव्यक्तं प्रतिपद्यते}
{यतो निःसरते लोको भवति व्यक्तसंज्ञकः}


\twolineshloka
{तत्राव्यक्तमयीं विद्यां शृणु त्वं विस्तरेण मे}
{तथा व्यक्तमयं चैव संख्यापूर्वं निबोध मे}


\twolineshloka
{पञ्चविंसतितत्त्वानि तुल्यान्युभयतः समम्}
{योगे साङ्ख्येऽपि च तथा विशेषं तत्र मे शृणु}


\twolineshloka
{प्रोक्तं तद्व्यक्तमित्येव जायते वर्धते च यत्}
{जीर्यते म्रियते चैव चतुर्भिर्लक्षणैर्युतम्}


\twolineshloka
{विपरीतमतो यत्तु तदव्यक्तमुदाहृतम्}
{द्वावात्मानौ च वेदेषु सिद्धान्तेष्वप्युदाहृतौ}


\twolineshloka
{चतुर्लक्षणजं त्वाद्यं चतुर्वर्गं प्रचक्षते}
{व्यक्तमव्यक्तजं चैव तथा बुद्धिरथेतरत् ॥सत्वं क्षेत्रज्ञ इत्येतद्द्वयमव्यक्तदर्शनम्}


\twolineshloka
{द्वावात्मानौ च वेदेषु विषयेष्वनुरज्यतः}
{विषयात्प्रतिसंहारः साङ्ख्यानां विद्धि लक्षणम्}


\twolineshloka
{निर्ममश्चानहंकारो निर्द्वन्द्वश्छिन्नसंशयः}
{नैव क्रुध्यति न द्वेष्टि नानृता भाषते गिरः}


\twolineshloka
{आक्रुष्टस्ताडितश्चैव मैत्रीयं ध्याति नाशुभम्}
{वाग्दण्डकर्ममनसां त्रयाणां च निवर्तकः}


\twolineshloka
{समः सर्वेषु भूतेषु ब्रह्माणमभिवर्तते}
{नैवेच्छति न चानिच्छो यात्रामात्रव्यवस्थितः}


\twolineshloka
{अलोलुपोऽव्यथो दान्तो नाकृतिर्न निराकृतिः}
{नास्येन्द्रियमनेकाग्रं नाविक्षिप्तमनोरथः}


\twolineshloka
{सर्वभूतसदृङ्भैत्रः समलोष्टाश्मकाञ्चनः}
{तुल्यप्रियाप्रियो धीरस्तुल्यनिन्दात्मसंस्तुतिः}


\twolineshloka
{अस्पृहः सर्वकामेभ्यो ब्रह्मचर्यदृढव्रतः}
{अहिंस्रः सर्वभूतानामीदृक्साङ्ख्यो विमुच्यते}


\twolineshloka
{यथा योगाद्विमुच्यन्ते कारणैर्यैर्निबोध तत्}
{योगैश्वर्यमतिक्रान्तो योऽतिक्रामति मुच्यते}


\twolineshloka
{इत्येषा भावजा बुद्धिः कथिता ते न संशयः}
{एवं भवति निर्द्वन्द्वो ब्रह्माणं चाधिगच्छति}


\chapter{अध्यायः २४३}
\twolineshloka
{व्यास उवाच}
{}


\threelineshloka
{अथ ज्ञानप्लवं धीरो गृहीत्वा शान्तिमात्मनः}
{उन्मज्जंश्च निमज्जंश्च विद्यामेवाभिसंश्रयेत् ॥शुक उवाच}
{}


\threelineshloka
{किं तज्ज्ञानमथो विद्या यथा निस्तरते द्वयम्}
{प्रवृत्तिलक्षणो धर्मो निवृत्तिरिति चैव हि ॥व्यास उवाच}
{}


\twolineshloka
{यस्तु पश्यन्स्वभावेन विनाभावमचेतनः}
{पुष्णाति स पुनः सर्वान्प्रज्ञया मुक्तहेतुकः}


\twolineshloka
{येषां चैकान्तभावेन स्वभावः कारणं मतम्}
{दूर्वातृणवृसीका ये ते लभन्ते न किंचन}


\twolineshloka
{येचैनं पक्षमाश्रित्य निवर्तन्त्यल्पमेधसः}
{स्वभावं कारणं ज्ञात्वा न श्रेयः प्राप्नुवन्ति ते}


\twolineshloka
{स्वभावो हि विनाशाय मोहकर्ममनोभवः}
{निरुक्तमेतयोरेतत्स्वभावपरिभावयोः}


\twolineshloka
{कृष्यादीनीह कर्माणि सस्यसंहरणानि च}
{प्रज्ञावद्भिः प्रक्लृप्तानि यानासनगृहाणि च}


\twolineshloka
{आक्रीडानां गृहाणां च गदानामगदस्य च}
{प्रज्ञावन्तः प्रवक्तारो ज्ञानवद्भिरनुष्ठिताः}


\twolineshloka
{प्रज्ञा संयोजयत्यर्थैः प्रज्ञा श्रेयोऽधिगच्छति}
{राजानो भुञ्जते राज्यं प्रज्ञया तुल्यलक्षणाः}


\twolineshloka
{परावरं तु भूतानां ज्ञानेनैवोपलभ्यते}
{विद्यया तात सृष्टानां विद्यैवेह परा गतिः}


\twolineshloka
{भूतानां जन्म सर्वेषां विविधानां चतुर्विधम्}
{जरायुजाण्डजोद्भिज्जस्वेदजं चोपलक्षयेत्}


\twolineshloka
{स्थावरेभ्यो विशिष्टानि जङ्गमान्युपधारयेत्}
{उपपन्नं हि यच्चेष्टा विशिष्येत विशेष्यया}


\twolineshloka
{आहुर्द्विबहुपादानि जङ्गमानि द्वयानि तु}
{बहुषाद्भ्यो विशिष्टानि द्विपादानि बहून्यपि}


\twolineshloka
{द्विपदानि द्वयान्याहुः पार्थिवानीतराणि च}
{पार्थिवानि विशिष्टानि तानि ह्यन्नानि भुञ्जते}


\twolineshloka
{पार्थिवानि द्वयान्याहुर्मध्यमान्युत्तमानि तु}
{मध्यमानि विशिष्टानि जातिधर्मोपधारणात्}


\twolineshloka
{मध्यमानि द्वयान्याहुर्धर्मज्ञानीतराणि च}
{धर्मज्ञानि विशिष्टानि कार्याकार्योपधारणात्}


\twolineshloka
{धर्मज्ञानि द्वयान्याहुर्वेदज्ञानीतराणि च}
{वेदज्ञानि विशिष्टानि वेदो ह्येषु प्रतिष्ठितः}


\twolineshloka
{वेदज्ञानि द्वयान्याहुः प्रवक्तृणीतराणि च}
{प्रवक्तॄणि विशिष्टानि सर्वधर्मोपधारणात्}


\twolineshloka
{विज्ञायन्ते हि यैर्वेदाः सधर्माः सक्रियाफलाः}
{सधर्मा निखिला वेदाः प्रवक्तृभ्यो विनिःसृताः}


\twolineshloka
{प्रवक्तॄणि द्वयान्याहुरात्मज्ञानीतराणि च}
{आत्मज्ञानि विशिष्टानि जन्माजन्मोपधारणात्}


\twolineshloka
{धर्मद्वयं हि यो वेद स सर्वज्ञः स सर्ववित्}
{सत्याशीः सत्यसंकल्पः सत्यः शुचिरथेश्वरः}


\twolineshloka
{धर्मज्ञानप्रतिष्ठं हि तं देवा ब्राह्मणं विदुः}
{शब्दब्रह्मणि निष्णातं परे च कृतनिश्चयम्}


\twolineshloka
{अन्तस्थं च बहिष्ठं च येऽधियज्ञाधिदैवतम्}
{जानन्ति तान्नमस्यामस्ते देवास्तात ते द्विजाः}


\twolineshloka
{तेषु विश्वमिदं भूतं साग्रं च जगदाहितम्}
{तेषां माहात्म्यभावस्य सदृशं नास्ति किंचन}


\twolineshloka
{आद्यन्तनिधनं चैव कर्म चातीत्य सर्वशः}
{चतुर्विधस्य भूतस्य सर्वस्येशाः स्वयंभुवः}


\chapter{अध्यायः २४४}
\twolineshloka
{व्यास उवाच}
{}


\twolineshloka
{एषा पूर्वतरा वृत्तिर्ब्राह्मणस्य विधीयते}
{ज्ञानवानेव कर्माणि कुर्वन्सर्वत्र सिद्ध्यति}


\twolineshloka
{तत्र चेन्न भवेदेवं संशयः कर्मनिश्चये}
{किंतु कर्मस्वभावोऽयं ज्ञानं कर्मेति वा पुनः}


\twolineshloka
{तत्र वेदविवित्सायां ज्ञानं चेत्पुरुषं प्रति}
{उपपत्त्युपलब्धिभ्यां वर्णयिष्यामि तच्छॄणु}


\twolineshloka
{पौरुषं कारणं केचिदाहुः कर्मसु मानवाः}
{दैवमेके प्रशंसन्ति स्वभावमपरे जनाः}


\twolineshloka
{पौरुषं कर्म दैवं च फलवृत्तिस्वभावतः}
{त्रयमेतत्पृथग्भूतमविवेकं तु केचन}


\twolineshloka
{एतदेवं च नैवं न च चोभे नानुभे तथा}
{कर्मस्था विषयं ब्रूयुः सत्वस्थाः समदर्शिनः}


\twolineshloka
{त्रेतायां द्वापरे चैव कलिजाश्च ससंशयाः}
{तपस्विनः प्रशान्ताश्च सत्वस्थाश्च कृते युगे}


\twolineshloka
{अपृथग्दर्शनाः सर्वे ऋक्सामसु यजुःषु च}
{कामद्वषौ पृथग्दृष्ट्वा तपः कृत उपासते}


\twolineshloka
{तपोधर्मेण संयुक्तस्तपोनित्यः सुसंशितः}
{तेन सर्वानवाप्नोति कामान्यान्मनसेच्छति}


\twolineshloka
{तपसा तदवाप्नोति यद्भूतं सृजते जगत्}
{तद्भूतश्च ततः सर्वभूतानां भवति प्रभुः}


\twolineshloka
{तदुक्तं वेदवादेषु गहनं वेददर्शिभिः}
{वेदान्तेषु पुनर्व्यक्तं क्रमयोगेन लक्ष्यते}


\twolineshloka
{आरम्भयज्ञाः क्षव्राश्च हविर्यज्ञा विशः स्मृताः}
{परिचारयज्ञाः शूद्राश्च जपयज्ञा द्विजातयः}


\twolineshloka
{परिनिष्ठितकार्यो हि स्वाध्यायेन द्विजो भवेत्}
{कुर्यादन्यन्न वा कुर्यान्मैत्रो ब्राह्मण उच्यते}


\twolineshloka
{त्रेतादौ सकला वेदा यज्ञा वर्णाश्रमास्तथा}
{संरोधादायुषस्त्वेते व्यस्यन्ते द्वापरे युगे}


\twolineshloka
{द्वापरे विप्लवं यान्ति वेदाः कलियुगे तथा}
{दृश्यन्ते नापि दृश्यन्ते कलेरन्ते पुनः किल}


\twolineshloka
{उत्सीदन्ति स्वधर्माश्च तत्राधर्मेण पीडिताः}
{गवां भूमेश्च ये चापामोषधीनां च ये रसाः}


\twolineshloka
{अधर्मान्तर्हिता वेदा वेदधर्मास्तथाऽऽश्रमाः}
{विक्रियन्ते स्वधर्माश्च स्थावराणि चराणि च}


\twolineshloka
{यथा सर्वाणि भूतानि वृष्टथा तृप्यन्ति प्रावृषि}
{सृजन्ते सर्वतोऽङ्गानि तथा वेदा युगेयुगे}


\twolineshloka
{विहितं कालनानात्वमनादिनिधनं च यत्}
{कीर्तितं यत्पुरस्तात्ते यतः संयान्ति च प्रजाः}


\twolineshloka
{यच्चेदं प्रभवः स्थानं भूतानां संयमो यमः}
{स्वभावेनैव वर्तन्ते द्वन्द्वसृष्टानि भूरिशः}


\twolineshloka
{सर्गः कालो धृतिर्वेदाः कर्ता कार्यं क्रियाफलम्}
{एतत्ते कथितं तात यन्मां त्वं परिपृच्छसि}


\chapter{अध्यायः २४५}
\twolineshloka
{भीष्म उवाच}
{}


\threelineshloka
{इत्युक्तोऽभिप्रशस्यैतत्परमर्षेस्तु शासनम्}
{मोक्षधर्मार्थसंयुक्तमिदं प्रष्टुं प्रचक्रमे ॥शुक उवाच}
{}


\twolineshloka
{प्रजावाञ्श्रोत्रियो यज्वा कृतप्रज्ञोऽनसूयकः}
{अनागतमनैतिह्यं कथं ब्रह्माधिगच्छति}


\twolineshloka
{तपसा ब्रह्मचर्येण सर्वत्यागेन मेधया}
{साङ्ख्ये वा यदि वा योग एतत्पृष्टो वदस्व मे}


\threelineshloka
{मनसश्चेन्द्रियाणां च यथैकाग्र्यमवाप्यते}
{येनोपायेन पुरुषैस्तत्त्वं व्याख्यातुमर्हसि ॥व्यास उवाच}
{}


\twolineshloka
{नान्यत्र विद्यातपसोर्नान्यत्रेन्द्रियनिग्रहात्}
{नान्यत्र लोभसंत्यागात्सिद्धिं विन्दति कश्चन}


\twolineshloka
{महाभूतानि सर्वाणि पूर्वसृष्टिः स्वयंभुवः}
{भूयिष्ठं प्राणभृत्काये निविष्टानि शरीरिषु}


\twolineshloka
{भूमेर्देहो जलास्त्रोतो ज्योतिषश्चक्षुषी स्मृते}
{प्राणापानाश्रयो वायुः स्वेष्वाकाशं शरीरिणाम्}


\twolineshloka
{क्रान्ते विष्णुर्बले शक्रः कोष्ठेऽग्निर्भोक्तुमिच्छति}
{कर्णयोः प्रदिशः श्रोत्रे जिह्वायां वाक् सरस्वती}


\twolineshloka
{कर्णौ त्वक्चक्षुषी जिह्वा नासिका चैव पञ्चमी}
{दश तानीन्द्रियोक्तानि द्वाराण्याहारसिद्धये}


\twolineshloka
{शब्दः स्पर्शस्तथा रूपं रसो गन्धश्च पञ्चमः}
{इन्द्रियाणि पृथक्स्वार्थान्मनसा दर्शयन्त्युत}


\twolineshloka
{इन्द्रियाणि मनो युङ्क्ते वश्यान्यन्तेव वाजिनः}
{मनश्चापि सदा भुक्ते भूतात्मा हृदयाश्रितः}


\twolineshloka
{इन्द्रियाणां तथैवैषां सर्वेषामीश्वरं मनः}
{नियमे च विसर्गे च भूतात्मा मानसस्तथा}


\twolineshloka
{इन्द्रियाणीन्द्रियार्थाश्च स्वभावश्चेतना मनः}
{प्राणापानौ च जीवश्च नित्यं देहेषु देहिनाम्}


\twolineshloka
{आश्रयो नास्ति सत्वस्य गुणः सत्त्वस्य चेतना}
{सत्वं हि तेजः सृजति न गुणान्वै कथंचन}


\twolineshloka
{एवं सप्तदशं देहे वृतं षोडशभिर्गुणैः}
{मनीषीमनसा विप्रः पश्यत्यात्मानमात्मनि}


\twolineshloka
{न ह्ययं चक्षुषा दृश्यो न च सर्वैरपीन्द्रियैः}
{मनसा दीपभूतेन महानात्मा प्रकाशते}


\twolineshloka
{अशब्दस्पर्शरूपं तदरसागन्धमव्ययम्}
{अशरीरं शरीरेषु निरीक्षते निरिन्द्रियम्}


\twolineshloka
{अव्यक्तं सर्वदेहेषु मर्त्येष्वमृतमाहितम्}
{योऽनुपश्यति स प्रेत्य कल्पते ब्रह्मभूयसे}


\twolineshloka
{विद्याभिजनसंपन्ने ब्राह्मणे गवि हस्तिनि}
{शुनि चैव श्वपाके च पण्डिताः समदर्शिनः}


\twolineshloka
{स हि सर्वेषु भूतेषु जङ्गमेषु ध्रुवेषु च}
{वसत्येको महानात्मा येन सर्वमिदं ततम्}


\twolineshloka
{सर्वभूतेषु चात्मानं सर्वभूतानि चात्मनि}
{यदा पश्यति भूतात्मा ब्रह्म संपद्यते तदा}


\twolineshloka
{यावानात्मनि मे ह्यात्मा तावानात्मा परात्मनि}
{य एवं सततं वेद सोऽमृतत्वाय कल्पते}


\twolineshloka
{सर्वभूतात्मभूतस्य सर्वभूतहितस्य च}
{देवाऽपि मार्गे मुह्यन्ति अपदस्य पदैषिणः}


\twolineshloka
{शकुन्तानामिवाकाशे मत्स्यानामिव चोदके}
{यथा गतिर्न दृश्येत तथा ज्ञानविदां गतिः}


\twolineshloka
{कालः पचति भूतानि सर्वाण्येवात्मनाऽऽत्मनि}
{यस्मिंस्तु पच्यते कालस्तं वेदेह न कश्चन}


\twolineshloka
{न स ऊर्ध्वं न तिर्यक्च नाधश्चरति यः पुनः}
{न मध्ये प्रतिगृह्णीते नैव किंचित्कुतश्चन}


\twolineshloka
{सर्वेऽन्तस्था इमे लोका बाह्यमेषां न किंचन}
{यः सहस्र समा गच्छेद्यथा बाणो गुणच्युतः}


\twolineshloka
{नैवान्तं कारणस्येयाद्यद्यपि स्यान्मनोजवः}
{तस्मात्सूक्ष्मात्सूक्ष्मतरं नास्ति स्थूलतरं ततः}


\twolineshloka
{सर्वतः पाणिपादं तत्सर्वतोक्षिशिरोमुखम्}
{सर्वतः श्रुतिमल्लोके सर्वमावृरत्य तिष्ठति}


\twolineshloka
{तदेवाणोरणुतरं तन्महद्भ्यो महत्तरम्}
{तदन्तः सर्वभूतानां ध्रुवं तिष्ठन्न दृश्यते}


\twolineshloka
{अक्षरं च क्षरं चैव द्वैधीभावोऽयमात्मनः}
{क्षरः सर्वेषु भूतेषु दिवि ह्यमृतमक्षरम्}


\twolineshloka
{नवद्वारं पुरं गत्वा हंसो हि नियतो वशी}
{ईशः सर्वस्य भूतस्य स्थावरस्य चरस्य च}


\twolineshloka
{हानिभङ्गविकल्पानां नवानां संचयेन च}
{शरीराणामजस्याहुर्हंसत्वं पारदर्शिनः}


\twolineshloka
{हंसोक्तं चाक्षरं चैव कूटस्थं यत्तदक्षरम्}
{तद्विद्वानक्षरं प्राप्य जहाति प्राणजन्मनी}


\chapter{अध्यायः २४६}
\twolineshloka
{व्यास उवाच}
{}


\twolineshloka
{पृच्छतस्तव सत्पुत्र यथावदिह तत्त्वतः}
{साङ्ख्यन्यायेन संयुक्तं यदेतत्कीर्तितं मया}


\twolineshloka
{योगकृत्यं तु ते कृत्स्नं वर्तयिष्यामि तच्छृणु}
{एकत्वं बुद्धिमनसोरिन्द्रियाणां च सर्वशः}


\twolineshloka
{आत्मनोऽव्यथिनस्तात् ज्ञानमेतदनुत्तमम्}
{तदेतदुपशान्तेन दान्तेनाध्यात्मशीलिना}


\twolineshloka
{आत्मारामेण बुद्धेन बोद्धव्यं शुचिकर्मणा}
{योगदोषान्समुच्छिन्द्यात्पञ्च यान्कवयो विदुः}


\twolineshloka
{कामं क्रोधं त्त लोभं च भयं स्वप्नं च पञ्चमम्}
{क्रोधं शमेन जयति कामं संकल्पवर्जनात्}


\twolineshloka
{सत्त्वसंसेवनाद्धीरो निद्रामुच्छेत्तुमर्हति}
{धृत्या शिश्नोदरं रक्षेत्पाणिपादं च चक्षुषा}


\twolineshloka
{चक्षुःश्रोत्रे च मनसा मनो वाचं च कर्मणा}
{अप्रमादाद्भयं जह्याल्लोभं प्राज्ञोपसेवनात्}


\twolineshloka
{एवमेतान्योगदोषाञ्चयेन्नित्यमतन्द्रितः}
{अग्नींश्च ब्राह्मणांश्चार्चेद्देवताः प्रणमेत च}


\twolineshloka
{वर्जयेदुशतीं वाचं हिंसायुक्तां मनोनुदाम्}
{ब्रह्म तेजोमयं शुक्रं यस्य सर्वमिदं ततम्}


\twolineshloka
{एतस्य सूत्रभूतस्य द्वयं स्थावरजङ्गमम्}
{ध्यानमध्ययनं दानं सत्यं ह्रीरार्जवं क्षमा}


\twolineshloka
{शोचमाहारसंशुद्धिरिन्द्रियाणां च निग्रहः}
{एतैर्विवर्धते तेजः पाप्मानं चापकर्षति}


\twolineshloka
{सिद्ध्यन्ति चास्य सर्वार्था विज्ञानं च प्रवर्धते}
{समः सर्वेषु भूतेषु लब्धालब्धेन वर्तयेत्}


\twolineshloka
{धूतपाप्मा तु तेजस्वी लघ्वाहारो जितेन्द्रियः}
{कामक्रोधौ वशे कृत्वा निनीषेद्ब्रह्मणः पदम्}


\twolineshloka
{मनसश्चेन्द्रियाणां च कृत्वैकाग्र्यं समाहितः}
{पूर्वरात्रेऽपरात्रे च धारयेन्मन आत्मनि}


\twolineshloka
{जन्तोः पञ्चेन्द्रियस्यास्य यदेकं छिद्रमिन्द्रियम्}
{ततोऽस्य स्रवते प्रज्ञा दृतेः पादादिवोदकम्}


\twolineshloka
{मनस्तु पूर्वमादद्यात्कुमीनमिव मत्स्यहा}
{ततः श्रोत्रं ततश्चक्षुर्जिह्वा घ्राणं च योगवित्}


\twolineshloka
{तत एतानि संयम्य मनसि स्थापयेद्यतिः}
{तथैवापो ह्यसंकल्पान्मनो ह्यात्मनि धारयेत्}


\twolineshloka
{पञ्चेन्द्रियाणि संधाय मनसि स्थापयेद्यतिः}
{यदैतान्यवतिष्ठन्ति मनःषष्ठानि चात्मनि}


\twolineshloka
{प्रसीदन्ति च संस्थाय तदा ब्रह्म प्रकाशते}
{विधूम इव सप्तार्चिरादित्य इव दीप्तिमान्}


\twolineshloka
{वैद्युतोऽग्निरिवाकाशे दृश्यतेऽऽत्मा तथाऽऽत्मनि}
{सर्वस्तत्र स सर्वत्र व्यापकत्वाच्च दृश्यते}


\twolineshloka
{तं पश्यन्ति महात्मानो ब्राह्मणा ये मनीषिणः}
{धृतिमन्तो महाप्राज्ञाः सर्वभूतहिते रताः}


\twolineshloka
{एवं परिमितं कालमाचरन्संशितव्रतः}
{आसीनो हि रहस्येको गच्छेदक्षरसाम्यताम्}


\twolineshloka
{विमोहो भ्रम आवर्तो घ्राणं श्रवणदर्शने}
{अद्भुतानि रसस्पर्शे शीतोष्णे मारुताकृतिः}


\twolineshloka
{प्रतिभामुपसर्गांश्चाप्युपसंगृह्य योगतः}
{तांस्तत्त्वविदनादृत्य आत्मन्येव निवर्तयेत्}


\twolineshloka
{कुर्यात्परिचयं योगे त्रैकाल्ये नियतो मुनिः}
{गिरिशृङ्गे तथा चैत्ये वृक्षाग्रेषु च योजयेत्}


\twolineshloka
{संनियम्येन्द्रियग्रामं कोष्ठे भाण्डमना इव}
{एकाग्रं चिन्तयेन्नित्यं योगान्नोद्वेजयेन्मनः}


\twolineshloka
{येनोपायेन शक्येत संनियन्तुं चलं मनः}
{तत्तद्युक्तो निषेवेत न चैव विचलेत्ततः}


\twolineshloka
{शून्या गिरि---श्वैव देवतायतनानि च}
{शून्यागारा---काग्रो निवासार्थमुपक्रमेत्}


\twolineshloka
{नाभिष्व---वाचा कर्मणा मनसाऽपि वा}
{उपे-----रो लब्धालब्धे समो भवेत्}


\twolineshloka
{यश्चैन-----न्देत यश्चैनमभिवादयेत्}
{समस्त-----भयोर्नाभिध्यायेच्छुभाशुभम्}


\twolineshloka
{न प्रहृ---भेषु नालाभेषु च चिन्तयेत्}
{समः स-------षु सधर्मा मातरिश्वनः}


\twolineshloka
{एवं सर्वात्मनः साधोः सर्वत्र समदर्शिनः}
{षण्मासान्नित्ययुक्तस्य शब्दब्रह्मातिवर्तते}


\twolineshloka
{वेदनार्ताः प्रजा दृष्ट्वा समलोष्टाश्मकाञ्चनः}
{एतस्मिन्निरतो मार्गे विरमेन्न च मोहितः}


\twolineshloka
{अपि वर्णावकृष्टस्तु नारी वा धर्मकाङ्क्षिणी}
{तावप्येतेन मार्गेण गच्छेतां परमां गतिम्}


\twolineshloka
{अजं पुराणमजरं सनातनंयदिन्द्रियैरुपलभेत निश्चलैः}
{अणोरणीयो महतो महत्तरंतदात्मना पश्यति युक्तमात्मवान्}


\twolineshloka
{इदं महर्षेर्वचनं महात्मनोयथावदुक्तं मनसाऽनुदृश्य च}
{अवेक्ष्य चेमां परमेष्ठिसाम्यतांप्रयान्ति यां भूतगतिं मनीषिणः}


\chapter{अध्यायः २४७}
\twolineshloka
{शुक उवाच}
{}


\twolineshloka
{यदिदं वेदवचनं कुरु कर्म त्यजेति च}
{कां दिशं विद्यया यान्ति कां च गच्छन्ति कर्मणा}


\threelineshloka
{एतद्वै श्रोतुमिच्छामि तद्भवान्प्रब्रवीतु मे}
{एतच्चान्योन्यवैरूप्ये वर्तेते प्रतिकूलतः ॥भीष्म उवाच}
{}


\twolineshloka
{इत्युक्तः प्रत्युवाचेदं पराशरसुतः सुतम्}
{कर्मविद्यामयावेतौ व्याख्यास्यामि क्षराक्षरौ}


\twolineshloka
{यां दिशं विद्यया यान्ति यां च गच्छन्ति कर्मणा}
{शृणुष्वैकमना वत्स गह्वरं ह्येतदन्तरम्}


\twolineshloka
{अस्ति धर्म इति ह्युक्त्वा नास्तीत्यत्रैव यो वदेत्}
{तस्य पक्षस्य सदृशमिदं मम भवेदथ}


\twolineshloka
{द्वाविमावथ पन्थानौ यत्र वेदाः प्रतिष्ठिताः}
{प्रवृत्तिलक्षणो धर्मो निवृत्तौ च व्यवस्थितः}


\twolineshloka
{कर्मणा बध्यते जन्तुर्विद्यया तु प्रमुच्यते}
{तस्मात्कर्म न कुर्वन्ति यतयः पारदर्शिनः}


\twolineshloka
{कर्मणा जायते प्रेत्य मूर्तिमान्षोडशात्मकः}
{विद्यया जायते नित्यमव्ययो ह्यक्षरात्मकः}


\twolineshloka
{कर्म त्वेके प्रशंसन्ति स्वल्पबुद्धितया नराः}
{तेन ते देहजालानि रमयन्त उपासते}


\twolineshloka
{ये स्म बुद्धिं परां प्राप्ता धमैर्नपुण्यदर्शिनः}
{न ते कर्म प्रशंसन्ति कूपं नद्यां पिबन्निव}


\twolineshloka
{कर्मणः फलमाप्नोति सुखदुःखे भवाभवौ}
{विद्यया तदवाप्नोति यत्र गत्वा न शोचति}


\twolineshloka
{यत्र गत्वा न म्रियते यत्र गत्वा न जायते}
{न जीर्यते यत्र गत्वा यत्र गत्वा न वर्धते}


\twolineshloka
{यत्र तद्ब्रह्म परममव्यक्तमचलं ध्रवम्}
{अव्याहतमनायासममृतं चावियोगि च}


\twolineshloka
{द्वन्द्वैर्न यत्र बाध्यन्ते मानसेन च कर्मणा}
{समाः सर्वत्र मैत्राश्च सर्वभूतहिते रताः}


\threelineshloka
{विद्यामयोऽन्यः पुरुषस्तात कर्ममयोऽपरः}
{विद्धि चन्द्रमसं दर्शे सूक्ष्मया कलया स्थितम्}
{`विद्यामयं तं पुरुषं नित्यं ज्ञानगुणात्मकम् ॥'}


\twolineshloka
{तदेतदृषिणा प्रोक्तं विस्तरेणानुमीयते}
{नवं तु शशिनं दृष्ट्वा वक्रतन्तुमिवाम्बरे}


\twolineshloka
{एकादशविकारात्मा कलासंभारसंभृतः}
{मृर्तिमानिति तं विद्धि तात कर्म गुणात्मकम्}


\twolineshloka
{`तस्मिन्यः संस्थितो ह्यग्निर्नित्यंस्थाल्यामिवाहितः}
{आत्मानं तं विजानीहि नित्यं त्यागजितात्मकं}


\twolineshloka
{देवो यः संश्रितस्तस्मिन्नब्विन्दुरिव पुष्करे}
{क्षेत्रज्ञं तं विजानीयान्नित्यं योगजितात्मकम्}


\twolineshloka
{तमोरजश्च सत्त्वं च विद्धि जीवगुणात्मकम्}
{जीवमात्मगुणं विद्यादात्मानं प-----नः}


% Check verse!
अचेतनं जीवगुणं वदन्तिस चेष्टते चेष्टयते च सर्वम्ततः परं क्षेत्रविदो वदन्तिप्राकल्पयद्यो भुवनानि सप्त
\chapter{अध्यायः २४८}
\twolineshloka
{शुक उवाच}
{}


\twolineshloka
{क्षरात्प्रभृति यः सर्गः सगुणानीन्द्रियाणि च}
{बुद्ध्यैश्वर्यातिसर्गोऽयं प्रधानश्चात्मनः श्रुतम्}


\twolineshloka
{भूय एव तु लोकेऽस्मिन्सद्वृतिं कालहेतुकीम्}
{यया सन्तः प्रवर्तन्ते तदिच्छाम्यनुवर्तितुम्}


\twolineshloka
{वेदे वचनमुक्तं तु कुरु कर्म त्यजेति च}
{कथमेतद्विजानीयां तच्च व्याख्यातुमर्हसि}


\threelineshloka
{लोकवृत्तान्ततत्वज्ञः पूतोऽहं गुरुशासनात्}
{कृत्वा बुद्धिं विमुक्तात्मा द्रक्ष्याम्यात्मानमव्ययं ॥व्यास उवाच}
{}


\twolineshloka
{एषा वै विहिता वृत्तिः पुरस्ताद्ब्रह्मणा स्वयम्}
{एषा पूर्वतरैः सद्भिराचीर्णा परमर्षिभिः}


\twolineshloka
{ब्रह्मचर्येण वै लोकाञ्जयन्ति परमर्षयः}
{आत्मनश्च हृदि श्रेयो ह्यन्विच्छन्मनसाऽऽत्मनि}


\twolineshloka
{वने मूलफलाशी च तप्यन्सुविपुलं तपः}
{पुण्यायतनचारी च भूतानामविहिंसक}


\twolineshloka
{विधूमे सन्नमुसले वानप्रस्थप्रतिश्रये}
{काले प्राप्ते चरन्भैक्षं कल्पते ब्रह्मभूयसे}


\threelineshloka
{निस्तुतिर्निर्नमस्कारः परित्यज्य शुभाशुभे}
{अरण्ये विचरैकाकी येनकेनचिदाशितः ॥शुक उवाच}
{}


\twolineshloka
{यदिदं वेदवचनं लोकवादे विरुध्यते}
{प्रमाणे चाप्रमाणे च विरुद्धे शास्त्रतः कुतः}


\threelineshloka
{इत्येतच्छ्रोतुमिच्छामि प्रमाणं तूभयं कथम्}
{कर्मणामविरोधेन कथमेतत्प्रवर्तते ॥भीष्म उवाच}
{}


\threelineshloka
{इत्युक्तः प्रत्युवाचेदं गन्धवत्याः सुतः सुतम्}
{ऋषिस्तत्पूजयन्वाक्यं पुत्रस्यामिततेजसः ॥व्यास उवाच}
{}


\twolineshloka
{ब्रह्मचारी गृहस्थश्च वानप्रस्थोऽथ भिक्षुकः}
{यथोक्तकारिणः सर्वे गच्छन्ति परमां गतिम्}


\twolineshloka
{एको वाऽप्याश्रमानेतान्योऽनुतिष्ठेद्यथाविधि}
{अकामद्वेषसंयुक्तः स परत्र महीयते}


\twolineshloka
{चतुष्पदी हि निःश्रेयणी ब्रह्मण्येषा प्रतिष्ठिता}
{एतामाश्रित्य निःश्रेणीं ब्रह्मलोके महीयते}


\twolineshloka
{आयुषस्तु चतुर्भागं ब्रह्मचार्यनसूयकः}
{गुरौ वा गुरुपुत्रे वा वसेद्धर्मार्थकोविदः}


\twolineshloka
{जघन्यशायी पूर्वं स्यादुत्थाय गुरुवेश्मनि}
{यच्च शिष्येण कर्तव्यं कार्यं दासेन वा पुनः}


\twolineshloka
{कृतमित्येव तत्सर्वं कृत्वा तिष्ठेत पार्श्वतः}
{किंकरः सर्वकारी स्यात्सर्वकर्मसु कोविदः}


\twolineshloka
{कर्मातिशेषेण गुरावध्येतव्यं बुभूषता}
{दक्षिणेनोपसादी स्यादाकूतो नुल्माश्रयेत्}


\twolineshloka
{शुचिर्दक्षो गुणोपेतो ब्रूयादिष्टमिवान्तरा}
{चक्षुष गुरुमव्यग्रो निरीक्षेत जितेन्द्रियः}


\twolineshloka
{नाभुक्तवति चाश्नीयादपीतवति नो पिबेत्}
{नातिष्ठति तथासीत नासुप्ते प्रस्वपेत च}


\twolineshloka
{उत्तानाभ्यां च पाणिभ्यां पादावस्य मृदु स्पृशेत्}
{दक्षिणं दक्षिणेनैव सव्यं सव्येन पीडयेत्}


\twolineshloka
{अभिवाद्य गुरुं ब्रूयादधीष्व भगवन्निति}
{इदं करिष्ये भगवन्निदं चापि कृतं मया}


\threelineshloka
{ब्रह्मंस्तदपि कर्ताऽस्मि यद्भावन्वक्ष्यते पुनः}
{इति सर्वमनुज्ञाप्य निवेद्य गुरवे पुनः}
{}


\twolineshloka
{कुर्यात्कृत्वा च तत्सर्वमाख्येयं गुरवे पुनः}
{यांस्तु गन्धान्रसान्वाऽपि ब्रह्मचारी न सेवते}


\twolineshloka
{सेवेत तान्समावृत्त इति धर्मेषु निश्चयः}
{ये केचिन्नियमेनोक्ता नियमा ब्रह्मचारिणः}


\twolineshloka
{तान्सर्वाननुगृह्णीयाद्भवेच्चानपगो गुरोः}
{स एवं गुरवे प्रीतिमुपहृत्य यथाबलम्}


\threelineshloka
{अग्राम्येणा मेष्वेवं शिष्यो वर्तते कर्मणा}
{वदव्रतोपवासेन चतुर्थे च-------}
{गुरवे दक्षिणां दत्त्वा समावतेद्यथातिधि----}


\twolineshloka
{धर्मलब्धैर्युतो दारैरग्नीनुत्पाद्य यनतः}
{द्वितीयमायुषो भागं गृहमेधी भवेद्व्रती}


\chapter{अध्यायः २४९}
\twolineshloka
{व्यास उवाच}
{}


\twolineshloka
{--तीयमायुषो भागं गृहमेधी गृहे वसेत्}
{धर्मलब्धैर्युतो दारैरग्नीनाहृत्य सुव्रतः}


\twolineshloka
{गृहस्थवृत्तयश्चैव चतस्रः कविभिः स्मृताः}
{कुमूलधान्यः प्रथमः कुम्भधान्यस्त्वनन्तरम्}


\twolineshloka
{अश्वस्तनोऽथ कापोतीमाश्रितो वृत्तिमाहरेत्}
{तेषां परः परो ज्यायान्धर्मतो लोकजित्तमः}


\twolineshloka
{षट््कर्मावर्तयत्येकस्त्रिभिरन्यः प्रवर्तते}
{द्वाभ्यामेकश्चतुर्थस्तु ब्रह्मसत्रे व्यवस्थितः}


\twolineshloka
{गृहमेधिव्रतान्यत्र महान्तीह प्रचक्षते}
{नात्मार्थे पाचयेदन्नं न वृथा घातयेत्पशून्}


\twolineshloka
{प्राणी वा यदि वाऽप्राणी संस्कारं यजुषाऽर्हति}
{न दिवा प्रस्वपेज्जातु न पूर्वापररात्रयोः}


\twolineshloka
{न भुञ्जीतान्तराकाले नानृतावाह्वयेत्स्त्रियम्}
{नास्यानश्नन्गृहे विप्रो वसेत्कश्चिदपूजितः}


\twolineshloka
{तथास्यातिथयः पूज्या हव्यकव्यवहाः सदा}
{वेदविद्याव्रतस्नाताः श्रोत्रिया वेदपारगाः}


\twolineshloka
{स्वकर्मजीविनो दान्ताः क्रियावन्तस्तपस्विनः}
{तेषां हव्यं च कव्यं चाप्यर्हणार्थं विधीयते}


\twolineshloka
{नखरैः संप्रयातस्य स्वकर्मव्यापकस्य च}
{अपविद्धाग्निहोत्रस्य गुरोर्वाऽलीकचारिणः}


\twolineshloka
{संविभागोऽत्र भूतानां सर्वेषामेव शिष्यते}
{तथैवापचमानेभ्यः प्रदेयं गृहमेधिना}


\twolineshloka
{विस्साशी भवेन्नित्यं नित्यं चामृतभोजनः}
{असुत --शेषं स्याद्भोजनं हविषा समम्}


\twolineshloka
{भृत्यशेष तु योऽश्नाति तमाहुर्विघसा शिनम्}
{विघसं भृत्यशेषं तु यज्ञशेषमथास्मृतय}


\twolineshloka
{स्वदारनिरतो दान्तो ह्यनसूयुर्जितेन्द्रियः}
{ऋत्विक्पुरोहिताचार्यर्मालुलातिथिनंश्रितैः}


\twolineshloka
{वृद्धबालातुरैर्वैद्यैर्ज्ञातिसंबन्धिरान्धवैः}
{मातापितृभ्यां जामीभिर्भ्रात्रा पुत्रेण भावया}


\twolineshloka
{दुहित्रा दासवर्गेण विवादं न समाचरत्}
{एतान्विमुच्य संवादान्सर्वपापैर्विमुच्यते}


\twolineshloka
{एतैर्जितस्तु जयति सर्वाल्लोँकान्न संशयः}
{आचार्यो ब्रह्मलोकेशः प्राजापत्ये पिता प्रभुः}


\twolineshloka
{अतिथिस्त्विन्द्रलोकस्य देवलोकस्य चर्त्विजः}
{जामयोऽप्सरसां लोके वैश्वदेवे तु ज्ञातयः}


\twolineshloka
{संबन्धिबान्धवा दिक्षु पृथिव्यां मातृमातुलौ}
{बृद्धबावातुरकृशास्त्वाकाशे प्रभविष्णवः}


\twolineshloka
{भ्राता ज्येष्ठः समः पित्रा भार्या पुत्रः स्वका तनुः}
{छाया स्वा दासवर्गश्च दुहिता कृपणं परम्}


\twolineshloka
{तस्मादेतैरधिक्षिप्तः सहेन्नित्यमसंज्वरः}
{गृहधर्मरतो विद्वान्धर्मनित्यो जितक्लमः}


\twolineshloka
{न चार्थबद्धः कर्माणि धर्मं वा किंचिदाचरेत्}
{गृहस्थवृत्तयस्तिस्रस्तासां निःश्रेयसं परम्}


\twolineshloka
{परंपरं तथैवाहुश्चातुराश्रम्यमेव तत्}
{ये चोक्ता नियमास्तेषां सर्वं कार्यं बुभूषता}


\twolineshloka
{कुम्भधान्यैरुच्छशिलैः कापोतीं चास्थितास्तथा}
{यस्मिंश्चैते वसन्त्यर्हास्तद्राष्ट्रमभिवर्धते}


\twolineshloka
{दश पूर्वान्दश परान्पुनाति च पितामहान्}
{गृहस्थवृत्तीश्चाप्येता वर्तयेद्यो गतव्यथः}


\twolineshloka
{स चक्रधरलोकानां सदृशीमाप्नुयाद्गतिम्}
{वितेन्द्रियाणामथवा गतिरेषा विधीयते}


\twolineshloka
{सर्वलोको गृहस्थानामुदारमनसां हितः}
{-- विमानसंयुक्तो वेददृष्टः -----}


\threelineshloka
{--लोको गृहस्थानां प्रतिष्ठा निवतात्मनान्}
{ब्रह्मणा विहिता श्रेणिरेषा पस्पाद्विधीयते}
{द्वितीयं क्रमशः प्राप्य स्वर्गलोके महीयते}


\twolineshloka
{अतः परं परममुदारमाश्रम्तृतीयमाहुस्त्यजतां कलेवरम्}
{वनौकसां गृहपतिनामनुत्तमंशृणुष्व संश्लिष्टशरीरकारिणाम्}


\chapter{अध्यायः २५०}
\twolineshloka
{भीष्म उवाच}
{}


\fourlineindentedshloka
{प्रोक्ता गृहस्थवृत्तिस्ते विहिता या मनीषिभिः}
{तदनन्तरमुक्तं यत्तन्निबोध युधिष्ठिर ॥`व्यासेन कथितं पूर्वं पुत्राय सुमहात्मने}
{' व्यास उवाच}
{}


\twolineshloka
{क्रय----त्ववधृयैनां तृतीयां वृत्तिमुत्तमाम्}
{संयोगव्रत----वानप्रस्थाश्रमौकसाम्}


\twolineshloka
{श्रूयतां पुत्र भद्रं ते सर्वलोकाश्रमात्मनाम्}
{प्रेक्षापूर्वं यदा पत्त्येद्वलीपलितमात्मना}


% Check verse!
अपत्यस्यैव चापण्यं वनमेव तदाऽऽश्रयेत्
\twolineshloka
{तृतीयमायुषो भागं वानप्रस्थाश्रमे वसेत्}
{तानेवाग्नीन्पस्त्विरेद्यजमानो दि--कसः}


\twolineshloka
{निय------नियताराहः षष्ठभक्तो ---त्तवान}
{तदाग्रहात्रं तो --- यज्ञाङ्गां व सर्वशः}


\twolineshloka
{अवैकृष्टं व्रीहियव नीवारे विघसानि च}
{हवींषि संप्रयच्छेत मखेष्वत्रापि पञ्चसु}


\twolineshloka
{वानप्रस्खाश्रमेऽप्येताश्चतस्रो वृत्तयः स्मृताः}
{सद्यः प्रक्षालकाः केचित्केचिन्मासिकसंचयाः}


\twolineshloka
{वार्षिकं संचयं केचित्केतिद्द्वादशवार्षिकम्}
{कुर्वन्त्यतिथिपूजार्थं यज्ञतन्त्रार्थमेव वा}


\twolineshloka
{अभ्रावकाशा वर्षासु हेमन्ते जलसंश्रयाः}
{ग्रीष्मे च पञ्चतपसः शश्वच्च मितभोजनाः}


\twolineshloka
{भूमौ विपरिवर्तन्ते तिष्ठन्ति प्रपदैरपि}
{स्थानासनैर्वर्तयन्ति स वनेष्वभिषिञ्चते}


\twolineshloka
{दन्तोलूखलिकाः केचिदश्मकुट्टास्तथा परे}
{शुक्लपक्षे पिबन्त्येके यवागूं क्वथितां सकृत्}


\twolineshloka
{कृष्णपक्षे पिबन्त्यन्ये भुञ्जते वा यथागतम्}
{मूलैरेके फलैरेके पुष्पैरेके दृढव्रताः}


\twolineshloka
{वर्तयन्ति यथान्यायं वैखानसमतं श्रिताः}
{एताश्चान्याश्च विविधा दीक्षास्तेषां मनीषिणाम्}


\threelineshloka
{चतुर्थश्चौपनिषदो धर्मः साधारणः स्मृतः}
{वानप्रस्थाद्गृहस्थाच्च ततोऽन्यः संप्रवर्तते}
{अस्मिन्नेव युगे तात वितैस्तत्वार्थदर्शिभिः}


\twolineshloka
{अगस्त्यः सप्तऋषयो मधुच्छन्दोऽधमर्षणः}
{सांकृतिः सुदिवातण्डिर्यथावासो कृतश्रमः}


\threelineshloka
{अहोवीर्यस्तथा काव्यस्ताण्ड्यो मेधातिथिदुऽ}
{बलवान्कर्णनिर्वाकः शून्यपालः कृतश्रम-----}
{एते धर्मे सुविद्वांसस्ततः स्वर्गमुसागमन}


\twolineshloka
{तात प्रत्यक्षधर्माणस्तथा काथवारा गणाः}
{ऋषीणामुग्रतपसां धर्मनैपुणेदर्शिनाम्}


\twolineshloka
{अन्ये चापरिमेयाश्च ब्राह्मणा वनमश्रितताः}
{वैखातसा वालखिल्याः सैकताच्च तथा परे}


\twolineshloka
{कर्मभिस्ते निरानन्दा धर्मनित्या जितेन्द्रियाः}
{गताः प्रत्यक्षधर्माणस्ते सर्वे वनमाश्रिताः}


\twolineshloka
{अनक्षत्रास्त्वनाधृष्या दृश्यते ज्योतिषां गणाः}
{जरया च परिद्यूना व्याधिना च प्रपीडिताः}


\twolineshloka
{चतुर्थे चायुषः शेषे वानप्रस्थाश्रमं त्यजेत्}
{साद्यस्कां संनिरुप्येष्टिं सर्ववेदसदक्षिणाम्}


\twolineshloka
{आत्मयाजी सोऽत्मरतिरात्मक्रीडात्मसंश्रयः}
{आत्मन्यग्नीन्समारोप्य त्यक्त्वा सर्वपरिग्रहान्}


\twolineshloka
{साद्यस्कांश्च यजेद्यज्ञानिष्टीश्चैवेह सर्वदा}
{यदैव याजिनां यज्ञादात्मनीज्या प्रवर्तते}


\twolineshloka
{त्रींश्चैवाग्नींस्त्यजेत्सम्यगात्मन्येवात्ममोक्षणात्}
{प्राणेभ्यो यजुषां पञ्च षट्् प्राश्नीयादकुत्सयन्}


\twolineshloka
{केशलोमनखान्वाप्य वानप्रस्थो मुनिस्ततः}
{आश्रमादाश्रमं पुण्यं पूतो गच्छति कर्मभिः}


\twolineshloka
{अभयं सर्वभूतेभ्यो दत्त्वा यः प्रव्रजेद्द्विजः}
{लोकास्तेजोमयास्तस्य प्रेत्य चानन्त्यमश्नुते}


\twolineshloka
{सुशीलवृत्तो व्यपनीतकल्मषोनचेह नामुत्र च कर्तुमीहते}
{अरोपमोहो गतसन्धिविग्रहोभवेदुदासीनवदात्मविन्नरः}


\twolineshloka
{यमेषु चैवानुगतेषु न व्यथेत्स्वशास्त्रमूत्राहुतिमन्त्रविक्रमः}
{भवेद्यथेष्टा गतिरात्मयाजिनोन संशयो धर्मपरे जितेन्द्रिये}


\twolineshloka
{ततः परं श्रेष्ठमतीव सद्गुणैरधिष्ठितं त्रीनधिवृत्तिमुत्तमम्}
{चतुर्थमुक्तं परमाश्रमं शृणुप्रकीर्त्यमानं परमं परायणम्}


\chapter{अध्यायः २५१}
\twolineshloka
{श्रीशुक उवाच}
{}


\threelineshloka
{वर्तमानस्तयेवात्र वानप्रस्थाश्रमे यथा}
{योक्तव्योऽऽत्मा कथं शक्त्या परं वै काङ्क्षता पदम् ॥व्यास उवाच}
{}


\twolineshloka
{प्राप्य संस्कारमेताभ्यामाश्रमाभ्यां ततः परम्}
{यत्कार्यं परमार्थार्थं तदिहैकमनाः शृणु}


\twolineshloka
{कषायं पाचयित्वाऽऽशु श्रेणिस्थानेषु च त्रिषु}
{प्रव्रजेच्च परं स्थानं पारिव्राज्यमनुत्तमम्}


\twolineshloka
{यद्भवानेवमभ्यस्य वर्ततां श्रूयतां तथा}
{एक एव चरेद्धर्मं सिद्ध्यर्थमसहायवान्}


\twolineshloka
{एकश्चरतिः यः पश्यन्न जहाति न हीयते}
{अनग्निरनिकेतश्च ग्राममन्नार्थमाश्रयेत्}


\twolineshloka
{अश्वस्तनविधाता स्यान्मुनिर्भावसमन्वितः}
{लघ्वाशी नियताहारः सकृदन्ननिषेविता}


\twolineshloka
{णलं वृक्षमलानि कुचेलमसदृ-------}
{उपक्षा सर्वभूतानामेतावद्भिक्षुलक्षणाम्}


\twolineshloka
{यस्मिन्वाचा प्राविशन्ति कूपे प्राप्ताः शिलाइव}
{न वक्तारं पुनर्यान्ति स कैवल्याश्रमे वसेत्}


\twolineshloka
{नैव पश्येन्न शृणुयादवाच्यं जातु कस्यचित्}
{ब्राह्मणानां विशेषेण नैव ब्रूयात्कथंचन}


\twolineshloka
{बद्ब्राह्मणस्य कुशलं तदेव सततं वदेत्}
{तूष्णीमासीत निन्दायां कुर्वन्भैषज्यमात्मनः}


\twolineshloka
{येन पूर्णमिवाकाशं भवत्येकेन सर्वदा}
{शून्यं येन जनाकीर्णं तं देवा ब्राह्मणं विदुः}


\twolineshloka
{येनकेन चिदाच्छन्नो येनकेनचिदाशितः}
{यत्र क्वचन शायी च तं देवा ब्राह्मणं विदुः}


\twolineshloka
{अहेरिव गणाद्भीतः सन्मानान्मरणादिव}
{कुणपादिव च स्त्रीभ्यस्तं देवा ब्राह्मणं विदुः}


\twolineshloka
{न कुद्ध्येन्न प्रहृष्येच्च मानितोऽमानितश्च यः}
{सर्वभूतेष्वभयदस्तं देवा ब्राह्मणं विदुः}


\twolineshloka
{नाभिनन्देत मरणं नाभिनन्देत जीवितम्}
{कालमेव प्रतीक्षेत निदेशं भृतको यथा}


\twolineshloka
{अनभ्याहतचित्तः स्यादनभ्याहतवाग्भवेत्}
{निर्मुक्तः सर्वपापेभ्यो निरमित्रस्य किं भयम्}


\twolineshloka
{अभयं सर्वभूतेभ्यो भूतानामभयं यतः}
{तस्य मोहाद्विमुक्तस्य भयं नास्ति कुतश्चन}


\twolineshloka
{यथा नागपदेऽन्यानि पदानि पदगामिनाम्}
{सर्वाण्येवापिलीयन्ते पदजातानि कौञ्जरे}


\twolineshloka
{एवं सर्वमहिंसायां धर्मार्थमभिधीयते}
{अमृतः स नित्यं भवति यो हिंसां न प्रपद्यते}


\twolineshloka
{अहिसंकः समः सत्यो धृतिमान्नियतेन्द्रियः}
{शरण्यः सर्वभूतानां गतिमाप्नोत्यनुत्तमाम्}


\twolineshloka
{एवं प्रज्ञानतृप्तस्य निर्भयस्य निराशिषः}
{न मृत्युरतिगो भावः स मृत्युं नाधिगच्छति}


\twolineshloka
{विमुक्तं सर्वसङ्गेभ्यो मुनिमाकाशवत्स्थितम्}
{अस्वमेकचरं शान्तं तं देवा ब्राह्मणं विदुः}


\twolineshloka
{जीवितं यस्य धर्मार्थं धर्मो हर्यर्थमेव च}
{अहोरात्राश्च पुण्यार्थं तं देवा ब्राह्मणं विदुः}


\twolineshloka
{निराशिषमनारम्भं निर्नमस्कारमस्तुतिम्}
{निर्मुक्ते बन्धनैः सर्वैस्तं देवा ब्राह्मणं विदुः}


\twolineshloka
{सर्वाणि भूतानि सुखे रमन्तेसर्वाणि दुःखस्य भृशं त्रसन्ते}
{तेषां भयोत्पादनजातखेदःकुर्यान्न कर्माणि हि श्रद्दधानः}


\twolineshloka
{दानं हि भूताभयदक्षिणायाःसर्वाणि दानान्यधितिष्ठतीह}
{तीक्ष्णां तनुं यः प्रथमं जहातिसोऽनन्तमाप्नोत्यभयं प्रजाभ्यः}


\twolineshloka
{स दत्तमास्येन हविर्जुहोतिलोकस्य नाभिर्जगतः प्रतिष्ठा}
{तस्याङ्गमङ्गानि कृताकृतं चवैश्वानरः सर्वमिदं प्रपेदे}


\twolineshloka
{प्रादेशमात्रे हृदि निःसृतं यत्तस्मिन्प्राणानात्मयाजी जुहोति}
{तस्याग्निहोत्रं हुतमात्मसंस्थंसर्वेषु लोकेषु सदैवतेषु}


\twolineshloka
{देवं त्रिधातुं त्रिवृतं सुपर्णये विद्युरग्र्यां परमात्मतां च}
{ते सर्वलोकेषु महीयमानादेवाः समर्था अमृतं वहन्ति}


\twolineshloka
{वेदांश्च वेद्यं तु विधिं च कृत्स्नमथो निरुक्तं परमार्थतां च}
{सर्वं शरीरात्मनि यः प्रवेदतस्य स्म देवाः स्पृहयन्ति नित्यम्}


\twolineshloka
{भूमावसक्तं दिवि चाप्रमेयंहिरण्मयं योऽण्डजमण्डमध्ये}
{पतत्रिणं पक्षिणमन्तरिक्षेयो वेद भोग्यात्मनि दीप्तरश्मिः}


\twolineshloka
{आवर्तमानमजरं विवर्तनंषण्णाभिकं द्वादशारं सुपर्व}
{यस्येदमास्योपरि याति विश्वंयत्कालचक्रं निहितं गुहायाम्}


\twolineshloka
{यः संप्रजानञ्जगतः शरीरंसर्वान्स लोकानधिगच्छतीह}
{तस्मिन्हितं तर्पयतीह देवांस्ते वै तृप्तास्तर्पयन्त्यास्यमस्य}


\twolineshloka
{तेजोमयो नित्यमयः पुराणोलोकाननन्तानभयानुपैति}
{भूतानि यस्मान्न त्रसन्ते कदाचित्स भूतानां न त्रसते कदाचित्}


\twolineshloka
{अगर्हणीयो न च गर्हतेऽन्यान्स वै विप्रः परमात्मानमीक्षेत्}
{विनीतमोहो व्यपनीतकल्मषोन चेह नामुत्र च सोऽन्नमर्च्छति}


\twolineshloka
{अरोषमोहः समलोष्टकाञ्चनःप्रहीणशोको गतसन्धिविग्रहः}
{अपेतनिन्दास्तुतिरप्रियाप्रियश्चरन्नुदासीनवदेष भिक्षुकः}


\chapter{अध्यायः २५२}
\twolineshloka
{व्यास उवाच}
{}


\twolineshloka
{प्रकृतेस्तु विकारा ये क्षेत्रज्ञस्तैरधिष्ठितः}
{न चैनं ते प्रजानन्ति स तु जानाति तानपि}


\twolineshloka
{तैश्चैवं कुरुते कार्यं मनःषष्ठैरिहेन्द्रियैः}
{सुदान्तैरिव संयन्ता दृढैः परमवाजिभिः}


\twolineshloka
{इन्द्रियेभ्यः परे ह्यर्था अर्थेभ्यः परमं मनः}
{मनसस्तु परा बुद्धिर्बुद्धेरात्मा महान्परः}


\twolineshloka
{महतः परमव्यक्तमव्यक्तात्पुरुषः परः}
{पुरुषान्न परं किंचित्सा काष्ठा सा परा गतिः}


\twolineshloka
{एवं सर्वेषु भूतेषु गूढोत्मा न प्रकाशते}
{दृश्यते त्वग्र्यया बुद्ध्या सूक्ष्मया सूक्ष्मदर्शिभिः}


\twolineshloka
{अन्तरात्मनि संलीय मनः षष्ठानि मेधया}
{इन्द्रियाणीन्द्रियार्थांश्च बहुचिन्त्यमचिन्तयन्}


\twolineshloka
{ध्यानोपरमणं कृत्वा विद्यासंपादितं मनः}
{अनिश्चरः प्रशान्तात्मा ततोर्च्छत्यमृतं पदम्}


\twolineshloka
{इन्द्रियाणां तु सर्वेषां पश्यात्मा चलितस्मृतिः}
{आत्मनः संप्रदानेन मर्त्यो मृत्युमुपाश्नुते}


\twolineshloka
{हित्वा तु सर्वसंकल्पान्सत्वे चित्तं निवेशयेत्}
{सत्वे चित्तं समावेश्य ततः कालंजरो भवेत्}


\twolineshloka
{चित्तप्रसादेन यतिर्जहातीह शुभाशुभम्}
{प्रसन्नात्मात्मनि स्थित्वा सुखमत्यन्तमश्नुते}


\twolineshloka
{लक्षणं तु प्रसादस्य यथा तृप्तः सुखं स्वपेत्}
{निवाते वा यथा दीपो दीप्यमानो न कम्पते}


\twolineshloka
{एवं पूर्वापरे रात्रौ युञ्जन्नात्मानमात्मनि}
{लघ्वाहारो विशुद्धात्मा पश्यत्यात्मानमात्मनि}


\twolineshloka
{रहस्यं सर्ववेदानामनैतिह्यमनागतम्}
{आत्मप्रत्ययिकं शास्त्रमिदं पुत्रानुशासनम्}


\twolineshloka
{धर्माख्यानेषु सर्वेषु चित्राख्यानेषु यद्वसु}
{दृश्यते ऋक्सहस्राणि निर्मथ्यामृतमुद्धृतम्}


\twolineshloka
{नवनीतं यथा दध्नः काष्ठादग्निर्यथैव च}
{तथैव विदुषां ज्ञानं पुत्रहेतोः समुद्धृतम्}


\twolineshloka
{स्नातकानामिदं शास्त्रं वाच्यं पुत्रानुशासनम्}
{तदितं नाप्रशान्ताय नादान्तायातपस्विने}


\twolineshloka
{नावेदविदुषे वाच्यं तथा नानुगताय च}
{नासूयकायानृजवे न चानिर्दिष्टकारिणे}


\twolineshloka
{न तर्कशास्त्रदग्धाय तथैव पिशुनाय च}
{श्लाघिने श्लाघनीयाय प्रशान्ताय तपस्विने}


\twolineshloka
{इदं प्रियाय पुत्राय शिष्यायानुगताय च}
{रहस्यधर्मं वक्तव्यं नान्यस्मै तु कथंचन}


\twolineshloka
{यद्यप्यस्य महीं दद्याद्रत्नपूर्णामिमां नरः}
{इदमेव ततः श्रेय इति मन्येत तत्त्ववित्}


\twolineshloka
{अतो गुह्यतरार्थं तदध्यात्ममतिमानुषम्}
{यत्तन्महर्षिभिर्जुष्टं वेदान्तेषु च गीयते}


% Check verse!
तत्तेऽहं संप्रवक्ष्यामि यन्मां त्वं परिपृच्छसि
\twolineshloka
{यच्च ते मनसि वर्तते परंयत्र चास्ति तव संशयः क्वचित्}
{श्रूयतामयमहं तवाग्रतःपुत्र किं हि कथयामि ते पुनः}


\chapter{अध्यायः २५३}
\twolineshloka
{शुक उवाच}
{}


\threelineshloka
{अध्यात्मं विस्तरेणेह पुनरेव वदस्व मे}
{यदध्यात्मं यथा वेद भगवन्नृषिसत्तम ॥व्यास उवाच}
{}


\twolineshloka
{अध्यात्मं यदिदं तात पुरुषस्येह विद्यते}
{तत्तेऽहं वर्तयिष्यामि तस्य व्याख्यामिमां शृणु}


\twolineshloka
{भूमिरापस्तथा ज्योतिर्वायुराकाश एव च}
{महाभूतानि भूतानां सागरस्योर्मयो यथा}


\twolineshloka
{प्रसार्येह यथाऽङ्गानि कूर्मः संहरते पुनः}
{तद्वन्महान्ति भूतानि यवीयःसु विकुर्वते}


\twolineshloka
{इति तन्मयमेवेदं सर्वं स्थावरजङ्गमम्}
{सर्गे च प्रलये चैव तस्मिन्निर्दिश्यते तथा}


\threelineshloka
{महाभूतानि पञ्चैव सर्वभूतेषु भूतकृत्}
{अकरोत्तात वैषम्यं यस्मिन्यदनुपश्यति ॥शुक उवाच}
{}


\threelineshloka
{अकरोद्यच्छरीरेषु कथं तदुपलक्षयेत्}
{इन्द्रियाणि गुणाः केचित्कथं तानुपलक्षयेत् ॥व्यास उवाच}
{}


\twolineshloka
{एतत्ते वर्तयिष्यामि यथावदनुपूर्वकः}
{शृणु तत्त्वमिहैकाग्रो यथा तत्त्वं यथा च तत्}


\twolineshloka
{शब्दः श्रोत्रं तथा खानि त्रयमाकाशसंभवम्}
{प्राणश्रेष्टा तथा स्पर्श एते वायुगुणास्त्रयः}


\twolineshloka
{रूपं चक्षुर्विपाकश्च त्रिधा ज्योतिर्विधीयते}
{रसोऽथ रसनं स्नेहो गुणास्त्वेते त्रयोऽम्भसः}


\twolineshloka
{घ्रेयं घ्राणं शरीरं च भूमेरेते गुणास्त्रयः}
{`श्रोत्रं त्वक्चक्षुषी जिह्वा नासिका चैव पञ्चमी}


\threelineshloka
{एतावानिन्द्रियग्रामो व्याख्यातः पाञ्चभौतिकः}
{वायोः स्पर्शो रसोऽद्भ्यश्च ज्योतिषो रुपमुच्यते}
{आकाशप्रभवः शब्दो गन्धो भूमिगुणः स्मृतः}


\twolineshloka
{मनो बुद्धिः स्वभावश्च त्रय एते मनोमयाः}
{न गुणानतिवर्तन्ते गुणेभ्यः परमागताः}


\twolineshloka
{यथा कूर्म इहाङ्गानि प्रसार्य विनियच्छति}
{एवमेवेन्द्रियग्रामं बुद्धिः सृष्ट्वा नियच्छति}


\twolineshloka
{यदूर्ध्वं पादतलयोरवाङ्भूर्ध्नश्च पश्यति}
{एतस्मिन्नेव कृत्ये तु वर्तते बुद्धिरुत्तमा}


\twolineshloka
{गुणान्नेनीयते बुद्धिर्बुद्धिरेवेन्द्रियाण्यपि}
{मनः षष्ठानि सर्वाणि बुद्ध्य भावे कृतो गुणाः}


\twolineshloka
{इन्द्रियाणि नरे पञ्च षष्ठं तु मन उच्यते}
{सप्तमीं बुद्धिमेवाहुः क्षेत्रज्ञं पुनरष्टमम्}


\twolineshloka
{चक्षुरालोचनायैव संशयं कुरुते मनः}
{बुद्धिरध्यवसानाय साक्षी क्षेत्रज्ञ उच्यते}


\twolineshloka
{रजस्तमश्च सत्वं च त्रय एते स्वयोनिजाः}
{समाः सर्वेषु भूतेषु तान्गुणानुपलक्षयेत्}


\twolineshloka
{तत्र यत्प्रीतिसंयुक्तं किंचिदात्मनि लक्षयेत्}
{प्रशान्तमिव संशुद्धं सत्वं तदुपधारयेत्}


\twolineshloka
{यत्तु संतापसंयुक्तं काये मनसि वा भवेत्}
{प्रवृत्तं रज इत्येवं तत्र चाप्युपलक्षयेत्}


\twolineshloka
{यत्तु संमोहसंयुक्तमव्यक्तविषयं भवेत्}
{अप्रतर्क्यमविज्ञेयं तमस्तदुपधार्यताम्}


\twolineshloka
{प्रहर्षः प्रीतिरानन्दः साम्यं स्वस्थात्मचित्तता}
{अकस्माद्यदि वा कस्माद्वर्तन्ते सात्विका गुणाः}


\twolineshloka
{अभिमानो मृषावादो लोभो मोहस्तथाऽक्षमा}
{लिङ्गानि रजसस्तानि वर्तन्ते हेत्वहेतुतः}


\twolineshloka
{तथा मोहः प्रमादश्च निद्रा तन्द्रा प्रबोधिता}
{कथंचिदभिवर्तन्ते विज्ञेयास्तामसा गुणाः}


\chapter{अध्यायः २५४}
\twolineshloka
{व्यास उवाच}
{}


\twolineshloka
{मनः प्रसृजते भावं बुद्धिरध्यवसायिनी}
{हृदयं प्रियाप्रिये वेद त्रिविधा कर्मवेदना}


\twolineshloka
{इन्द्रियेभ्यः परा ह्यर्था अर्थेभ्यः परमं मनः}
{मनसस्तु परा बुद्धिर्बुद्धेरात्मा परो मतः}


\twolineshloka
{बुद्धिरात्मा मनुष्यस्य बुद्धिरेवात्मनो गतिः}
{यदा विकुरुते भाव तदा भवति सा मनः}


\twolineshloka
{इन्द्रियाणां पृथग्भावाद्बुद्धिर्विक्रियतेऽसकृत्}
{शृण्वती भवति श्रोत्रं स्पृशती स्पर्श उच्यते}


\twolineshloka
{पश्यती भवते दृष्टी रसती रसनं भवेत्}
{जिघ्रती भवति घ्राणं बुद्धिर्विक्रियते पृथक्}


\twolineshloka
{इन्द्रियाणीति तान्याहुस्तेष्वदृश्योऽधितिष्ठति}
{तिष्ठती पुरुषे बुद्धिस्त्रिषु भावेषु वर्तते}


\twolineshloka
{कदाचिल्लभते प्रीतिं कदाचिदपि शोचति}
{न सुखेन न दुःखेन कदाचिदिह युज्यते}


\twolineshloka
{सेयं भावात्मिका भावांस्त्रीनेताननुवर्तते}
{सरितां सागरो भर्ता महावेलामिवोर्मिमान्}


\threelineshloka
{यदा प्रार्थयते किंचित्तदा भवति सा मनः}
{अधिष्ठानानि वै बुद्ध्यां पृथगेतानि संस्मरेत}
{इन्द्रियाण्येवमेतानि विजेतव्यानि कृत्स्नशः}


\threelineshloka
{सर्वाण्येवानुपूर्व्येण यद्यदाऽनुविधीयते}
{अविभागगता बुद्धिर्भावे मनसि वर्तते}
{12-254-10c` प्रवर्तमानंतु रजः सत्वमप्यनुवर्तते ॥'}


\twolineshloka
{ये चैव भावा वर्तन्ते सर्व एष्वेव ते त्रिषु}
{अन्वर्थाः संप्रवर्तन्ते रथनेमिमरा इव}


\twolineshloka
{प्रदीपार्थं मनः कुर्यादिन्द्रियैर्बुद्धिसत्तमैः}
{निश्चरद्भिर्यथायोगमुदासीनैर्यदृच्छया}


\twolineshloka
{एवं स्वभावमेवेदमिति विद्वान्न मुह्यति}
{अशोचन्नप्रहृष्यन्हि नित्यं विगतमत्सरः}


\twolineshloka
{न चात्मा शक्यते द्रष्टुमिन्द्रियैः कामगोचरैः}
{प्रवर्तमानैरनयैर्दुर्धर्षैरकृतात्मभिः}


\twolineshloka
{तेषां तु मनसा रश्मीन्यदा सम्यङ्क्तियच्छति}
{तदा प्रकाशतेऽस्यात्मा दीपदीप्ता यथाऽऽकृतिः}


\twolineshloka
{सर्वेषामेव भूतानां मनस्युपरते यथा}
{प्रकाशं भवते सर्वं तथेदमुपधार्यताम्}


\twolineshloka
{यथा वारिचरः पक्षी न लिप्यति जले चरन्}
{विमुक्तात्मा तथा योगी गुणदोषैर्न लिप्यते}


\twolineshloka
{एवमेव कृतप्रज्ञो न दोषैर्विषयांश्चरन्}
{असज्जमानः सर्वेषु कथंचन न लिप्यते}


\twolineshloka
{त्यक्त्वा पूर्वकृतं कर्म रतिर्यस्य सदाऽऽत्मनि}
{सर्वभूतात्मभूतस्य गुणवर्गेष्वसज्जतः}


\twolineshloka
{सत्वमात्मा प्रसरति गुणान्वाऽपि कदाचन}
{न गुणा विदुरात्मानं गुणान्वेद स सर्वदा}


\twolineshloka
{परिद्रष्टा गुणानां च परिस्रष्टा यथातथम्}
{क्षेतक्षेत्रज्ञयोरेतदन्तरं विद्धि सूक्ष्मयोः}


\twolineshloka
{सृजतेऽत्र गुणानेक एको न सृजते गुणान्}
{पृथग्भूतौ प्रकृत्या तौ संप्रयुक्तौ च सर्वदा}


\twolineshloka
{यथा मत्स्योऽद्भिरन्यः स्यात्संप्रयुक्तौ तथैव तौ}
{मशकोदुम्बरौ वाऽपि संप्रयुक्तौ यथा सह}


\twolineshloka
{इषीका वा यथा मुञ्जे पृथक्च सह चैव च}
{तथैव सहितावेतावन्योन्यस्मिन्प्रतिष्ठितौ}


\chapter{अध्यायः २५५}
\twolineshloka
{व्यास उवाच}
{}


\twolineshloka
{सृजते त्रिगुणान्सत्वं क्षेत्रज्ञस्त्वधितिष्ठति}
{गुणान्विक्रियते सर्वानुदासीनवदीश्वरः}


\twolineshloka
{स्वभावयुक्तं तत्सत्वं यदिमान्सृजते गुणान्}
{ऊर्णनाभिर्यथा सूत्रं सृजते तन्तुवद्गुणान्}


\twolineshloka
{प्रध्वस्ता न निवर्तन्ते प्रवृत्तिर्नोपलभ्यते}
{एवमेके व्यवस्यन्ति निवृत्तिरिति चापरे}


\twolineshloka
{उभयं संप्रधार्यैतदध्यवस्येद्यथामति}
{अनेनैव विधानेन भवेद्गर्भशयो महान्}


\twolineshloka
{अनादिनिधनं नित्यं तं बुद्ध्वा विचरेन्नरः}
{अक्रुध्यन्नप्रहृष्यंश्च नित्यं विगतमत्सरः}


\twolineshloka
{इत्येवं हृदयग्रन्थिं बुद्धिचिन्तामयं दृढम्}
{अतीत्य सुखमासीत अशोचंश्छिन्नसंशयः}


\twolineshloka
{ताम्येयुः प्रच्युताः पृथ्व्यां यथा पूर्णां नदीं नराः}
{अवगाढा ह्यविद्वांसो विद्धि लोकमिमं तथा}


\twolineshloka
{न तु ताम्यति वै विद्वान्स्थले चरति तत्त्ववित्}
{एवं यो विन्दतेऽऽत्मानं केवलं ज्ञानमात्मनः}


\twolineshloka
{एवं बुद्ध्वा नरः सर्वं भूतानामागतिं गतिम्}
{समवेक्ष्य च वैषम्यं लभते शममुत्तमम्}


\twolineshloka
{एतद्वै जन्मसामर्थ्यं ब्राह्मणस्य विशेषतः}
{आत्मज्ञानं शमश्चैव पर्याप्तं तत्परायणम्}


\twolineshloka
{एतद्बुद्ध्वा भवेद्बुद्धः किमन्यद्बुद्धलक्षणम्}
{विज्ञायैतद्विमुच्यन्ते कृतकृत्या मनीषिणः}


\twolineshloka
{न भवति विदुषां महद्भयंयदविदुषां सुमहद्भयं परत्र}
{न हि गतिरधिकाऽस्ति कस्यचिद्भवति हि या विदुषः सनातनी}


\twolineshloka
{लोकमातुरमसूयते जनस्तत्तदेव च निरीक्ष्य शोचते}
{तत्र पश्य कुशलानशोचतोये विदुस्तदुभयं कृताकृतम्}


% Check verse!
यत्करोत्यनभिसन्धिपूर्वकंतच्च निर्णुदति ---न प्रियं तदुभयं न चाप्रियतस्य तज्जनयतीह कुर्वतः
\chapter{अध्यायः २५६}
\twolineshloka
{शुक उवाच}
{}


\threelineshloka
{यस्माद्धर्मात्परो धर्मो विद्यदे नेह कश्चन}
{यो वि----- न्प्रब्रवीतु मे ॥व्यास उवाच}
{}


\twolineshloka
{धर्मं ति संप्रवक्ष्यामि पुराणमृषिसंस्तुतम्}
{विशिष्टं सर्वधर्मेभ्यस्तमिहैकमनाः शृणु}


\twolineshloka
{इन्द्रिया-----प्रमाथीनि बुद्ध्या संयम्य यत्नतः}
{सर्वतो--------------------}


\twolineshloka
{मनसश्चेन्द्रियाणा चाप्यकाग्र्य परमं तप}
{तज्ज्यायः सर्वधर्मेभ्यः स धर्मः पर उच्य}


\twolineshloka
{तानि सर्वाणि संधाय मनःषष्ठानि मेधया}
{आत्मतृप्त इवासीत बहुचिन्त्यमचिन्तयन्}


\twolineshloka
{गोचरेभ्यो निवृत्तानि यदा स्थास्यन्ति वेश्मनि}
{तदा त्वमात्मनाऽऽत्मानं परं द्रक्ष्यसि शाश्वतम्}


\twolineshloka
{सर्वात्मानं महात्मानं विधूममिव पावकम्}
{तं पश्यन्ति महात्मानो ब्राह्मणा ये मनीषिणः}


\twolineshloka
{यथा पुष्पफलोपेतो बहुशाखो महाद्रुमः}
{आत्मनो नाभिजानीते क्व मे पुष्पं क्व मे फलम्}


\twolineshloka
{एवमात्मा न जानीते क्व गभिष्ये कुतस्त्वहम्}
{अन्यो ह्यत्रान्तरात्माऽस्ति यः सर्वमनुपश्यति}


\twolineshloka
{ज्ञानदीपेन दीप्तेन पश्यत्यात्मानमात्मना}
{दृष्ट्वा त्वमात्मनाऽऽत्मानं निरात्मा भव सर्ववित्}


\twolineshloka
{विमुक्तः सर्वपापेभ्यो विमुक्तत्वगिवोरगः}
{परां बुद्धिमवाप्येह विपाप्मा विगतज्वरः}


\twolineshloka
{सर्वतः प्रवहां घोरां नदीं लोकप्रवाहिनीम्}
{पञ्चेन्द्रियग्राहवतीं मनःसंकल्परोधसम्}


\twolineshloka
{लोभमोहतृणच्छन्नां कामक्रोधसरीसृपाम्}
{सत्यतीर्थानृतक्षोभां क्रोधपङ्कां सरिद्वराम्}


\twolineshloka
{अव्यक्तप्रभवां शीघ्रां दुस्तरामकृतात्मभिः}
{प्रतरस्व नदीं बुद्ध्या कामग्राहसमाकुलाम्}


\twolineshloka
{------------ तालद््स्तराम्}
{आत्मकमा---------जिह्वावर्तां दुरासदाम्}


\twolineshloka
{यां तरन्ति कृतप्रज्ञा धृतिमन्तो मनीषिणः}
{तां तीर्णः सर्वतोमुक्तो विधूतात्माऽऽत्मविच्छुचिः}


\twolineshloka
{उत्तमां बुद्धिमास्थाय ब्रह्मभूयं भविष्यसि}
{संतीर्णः सर्वसंक्लेशान्प्रसन्नात्मा विकल्मषः}


\twolineshloka
{भूमिष्ठानीव भूतानि पर्वतस्थो निशामय}
{अक्रुध्यन्नप्रहृष्यंश्च अनृशंसमतिस्तथा}


\threelineshloka
{ततो द्रक्ष्यसि सर्वेषां भूतानां प्रभवाप्ययौ}
{एनं वै सर्वभूतेभ्यो विशिष्टं मेनिरे बुधाः}
{धर्मं धर्मभृतां श्रेष्ठा मुनयस्तत्त्वदर्शिनः}


\twolineshloka
{आत्मनो व्यापिनो ज्ञानमिदं पुत्रानुशासनम्}
{प्रयताय प्रवक्तव्यं हितायानुगताय च}


\twolineshloka
{आत्मज्ञानमिदं गुह्यं सर्वगुह्यतमं महत्}
{अब्रुवं यदहं तात आत्मसाक्षिकमञ्जसा}


\twolineshloka
{नैव स्त्री न पुमानेतन्नैव वेद नपुंसकम्}
{अदुःखमसुखं ब्रह्म भूतभव्यभवात्मकम्}


\twolineshloka
{नैतज्ज्ञात्वा पुमान्स्त्री वा पुनर्भवमवाप्नुते}
{स्वभावप्रतिपत्त्यर्थमेतद्धर्मं विधीयते}


\twolineshloka
{यथा मतानि सर्वाणि तथैतानि यथातथा}
{कथितानि मया पुत्र भवन्ति न भवन्ति च}


\twolineshloka
{तत्प्रीतियुक्तेन गुणान्वितेनपुत्रेण सत्पुत्र दमान्वितेन}
{पृष्टो हि संप्रीतिमना यथार्थंब्रूयात्सुतस्येह यदुक्तमेतत्}


\chapter{अध्यायः २५७}
\twolineshloka
{व्यास उवाच}
{}


\twolineshloka
{---गन्धान्रसान्नानुरुन्ध्यात्सखं वा--------कीर्ति च यशश्च नच्छ-}
{----- वै प्राचरः पश्यतो ब्राह्मणस्य}


\twolineshloka
{----नधीयीत शुश्रूषुर्ब्रह्मचर्यवान्}
{ऋवो यजूंषि सामानि वेदवेदाङ्गपारगः}


\twolineshloka
{ज्ञानी यः सर्वभूतानां सर्ववित्सर्वभूतवित्}
{नाकामो म्रियते जातु-------------}


\twolineshloka
{इष्टी------श्राप्य क्रतूश्चवाप्तदाक्षणान्}
{प्राप्नोति नैव ब्राह्मण्यमविज्ञानात्कथंचन}


\twolineshloka
{यदा चायं न बिभेति यदा चास्मान्न बियति}
{यदा नेच्छति न द्वेष्टि ब्रह्म संपद्यते तदा}


\twolineshloka
{यदा न कुरुते भावं सर्वभूतेषु पापकम्}
{कर्मणा मनसा वाचा ब्रह्म संपद्यते तदा}


\twolineshloka
{कामबन्धनमेवेदं नान्यदस्तीह बन्धनम्}
{कामबन्धनमुक्तो हि ब्रह्मभूयाय कल्पते}


\twolineshloka
{कामतो मुच्यमानस्तु धूमाभ्रादिव चन्द्रमाः}
{विरजाः कालमाकाङ्क्षन्धीरो धैर्येण वर्तते}


\twolineshloka
{आपूर्यमाणमचलप्रतिष्ठंसमुद्रमापः प्रविशन्ति यद्वत्}
{तद्वत्कामा यं प्रविशन्ति सर्वेस शान्तिमाप्नोति न कामकामः}


% Check verse!
क कामकान्तो न तु कामकामःस वै कामात्स्वर्गमुपैति देही
\twolineshloka
{वेदस्योपनिपद्दानं दानस्योपनिषद्दमः}
{दमस्योपनिपद्दानं दानस्योपनिपत्तपः}


\twolineshloka
{तपसोपनिपत्त्यागस्त्यागस्योपनिपत्सुखम्}
{सुखस्योपनिपत्स्वर्गः स्वर्गस्योपनिपच्छमः}


\twolineshloka
{क्लेदनं शोकमनसोः संतीर्णं तृष्णया सह}
{सत्वमृच्छति संतोपाच्छान्तिलक्षणमुत्तमम्}


\twolineshloka
{विशोको निर्ममः शांतः प्रशांतात्माऽत्मविच्छुचिः}
{पङ्गिर्लक्षणवानेतैः समग्रः पुनरेष्यति}


\twolineshloka
{पङ्भिः सत्वगुणोपेतैः प्राज्ञैरधिगतं त्रिभिः}
{ये विदुः प्रत्यगात्मानमिहस्थानमृतान्विदुः}


\twolineshloka
{अकृत्रिममसंहार्यं प्राकृतं निरुपस्कृतम्}
{अध्यात्मवित्कृतप्रज्ञः सुखमव्ययमश्नुते}


\twolineshloka
{निष्प्रचारं मनः कृत्वा प्रतिष्ठाप्य च सर्वशः}
{यामयं लभते तुष्टिं सा न शक्याऽऽत्मनोन्यथा}


\twolineshloka
{येन तृप्यत्यभुञ्जानो येन तृप्यत्यवित्तवान्}
{येनास्नेहो बलं धत्ते यस्तं वेद स वेदवित्}


\twolineshloka
{असङ्गो ह्यात्मनो द्वाराण्यपिधाय विचिन्तयन्}
{यो ह्यास्ते ब्राह्मणः शिष्टः स आत्मरतिरुच्यते}


\twolineshloka
{समाहितं परे तत्त्वे क्षीणकाममवस्थितम्}
{सर्वतः सुखमन्वेति वपुश्चान्द्रमसं यथा}


\twolineshloka
{अविशेषाणि भूतानि गुणांश्च जहतो मुनेः}
{सुखेनापोह्यते दुःखं भास्करेण तमो यथा}


\twolineshloka
{तमतिक्रान्तकर्माणमतिक्रान्तगुणक्षयम्}
{ब्राह्मणं विपयाश्लिष्टं जरामृत्यू न विन्दतः}


\twolineshloka
{स यदा सर्वतो मुक्तः समः पर्यवतिष्ठते}
{इन्द्रियाणीन्द्रियार्थांश्च शरीरस्थोऽतिवर्तते}


\twolineshloka
{कारणं परमं प्राप्य अतिक्रान्तस्य कार्यताम्}
{पुनरावर्तनं नास्ति संप्राप्तस्य परात्परम्}


\chapter{अध्यायः २५८}
\twolineshloka
{व्यास उवाच}
{}


\twolineshloka
{द्वन्द्वासि मोक्षजिज्ञासुरर्थधर्मानुतिष्ठतः}
{वक्रा गुणवता शिष्यः श्राव्यः पूर्वमिदं महत्}


\twolineshloka
{आकाशं मारुतो ज्योतिरा--पृथ्वी च पञ्चमी}
{भावाभावौ च कालश्च सर्वभूतेषु पञ्चसु}


\twolineshloka
{अन्तरात्मकमाकाशं तन्मयं श्रोत्रमिन्द्रियम्}
{तस्य शब्दं गुणं विद्यान्मुनिः शास्त्रविधानवित्}


\twolineshloka
{चरणं मारुतात्मेति प्राणापानौ च तन्मयौ}
{स्पर्शनं चेन्द्रियं विद्यात्तथा स्पर्शं च तन्मयम्}


\twolineshloka
{तापः पाकः प्रकाशश्च ज्योतिश्चक्षुश्च तन्मयम्}
{तस्य रूपं गुणं विद्यात्तमोनाशकमात्मवान्}


\twolineshloka
{प्रक्लेदो द्रवता स्नेह इत्यपामुपदिश्यते}
{[असृङ्भज्जा च यच्चान्यत्स्निग्धं विद्यात्तदात्मकम् ॥]रसनं चेन्द्रियं जिह्वा रसश्चापां गुणो मतः}


\twolineshloka
{संघातः पार्थिवो धातुरस्थिदन्तनखानि च}
{श्मश्चु रोम च केशाश्च सिरा स्नायु च चर्म च}


\twolineshloka
{इन्द्रियं घ्राणसंज्ञातं नासिकेत्यभिसंज्ञिता}
{गन्धश्चैवेन्द्रियार्थोऽयं विज्ञेयः पृथिवीमयः}


\twolineshloka
{उत्तरेषु गुणाः सर्वे सन्ति पूर्वेषु नोत्तराः}
{पञ्चानां भूतसङ्घानां संततिं मुनयो विदुः}


\twolineshloka
{मनो नवममेषां तु बुद्धिस्तु दशमी स्मृता}
{एकादशस्त्वन्तरात्मा स सर्वः पर उच्यते}


\twolineshloka
{व्यवसायात्मिका बुद्धिर्मनो व्याकरणात्मकम्}
{कर्मानुमानाद्विज्ञेयः स जीवः क्षेत्रसंज्ञकः}


\twolineshloka
{एभिः कालात्मकैर्भावैर्यः सर्वैः सर्वमन्वितम्}
{पश्यत्यकलुयं बुद्ध्या स मोहं नानुवर्तते}


\chapter{अध्यायः २५९}
\twolineshloka
{व्यास उवाच}
{}


\twolineshloka
{शरीराद्विप्रमुक्तं हि सूक्ष्मभूतं शरीरिणम्}
{कर्मभिः परिपश्यन्ति शास्त्रोक्तैः शास्त्रचेतसः}


\twolineshloka
{यथा मरीच्यः सहिताश्चरन्तिगच्छन्ति तिष्ठन्ति च दृश्यमानाः}
{देहैर्विमुक्तानि वरन्ति लोकांस्तथैव सत्वान्यतिमानुषाणि}


\twolineshloka
{प्रतिरूपं यथैवाप्सु तावत्सूर्यस्य लक्ष्यते}
{सत्ववांस्तु तथा सत्वं प्रतिरूपं स पश्यति}


\twolineshloka
{तानि सूक्ष्माणि सत्वानि विमुक्तानि शरीरतः}
{तेन तत्वेन तत्वज्ञाः पश्यन्ति नियतेन्द्रियाः}


\twolineshloka
{स्वपतां जाग्रतां चैष सर्वेषामात्मचिन्तितम्}
{प्रधानाद्वैधयुक्तानां दह्यते कर्मजं रजः}


\twolineshloka
{यथाऽहनि तथा रात्रौ यथा रात्रौ तथाऽहनि}
{वशे तिष्ठति सत्वात्मा सततं योगयोगिनाम्}


\twolineshloka
{तेषां नित्यं सदा नित्यो भूतात्मा सततं गुणैः}
{सप्तभिस्त्वन्वितः सूक्ष्मैश्चरिष्णुरजरामरः}


\twolineshloka
{मनोवुद्धिपराभूतः स्वदेहपरदेहवित्}
{स्वप्नेष्वपि भवत्येष विज्ञाता सुखदुःखयोः}


\twolineshloka
{तत्रापि लभते दुःखं तत्रापि लभते सुखम्}
{कामं क्रोधं च तत्रापि कृत्वा व्यसनमृच्छति}


\twolineshloka
{प्रीणितश्चापि भवति महतोऽर्थानवाप्य हि}
{करोति पुण्यं तत्रापि जाग्रन्निव च पश्यति}


\twolineshloka
{सदोष्मान्तर्गतश्चापि गर्भत्वं समुपेयिवान्}
{दश मासान्वसन्कुक्षौ नैषोऽन्नमिव जीर्यते}


\twolineshloka
{तमेतमतितेजोंशं भूतात्मानं हृदि स्थितम्}
{तमोरजोभ्यामाविष्टा नानुपश्यन्ति मूर्तिषु}


\threelineshloka
{योगशास्त्रपरा भूत्वा स्वमात्मानं परीप्सवः}
{`तमोरजोभ्यां निर्मुक्तास्तं प्रपश्यन्ति मूर्तिषु}
{'अनुच्छ्वासान्यमूर्तानि यानि वज्रोपमान्यपि}


\twolineshloka
{पृथग्भूतेषु सृष्टेषु चतुर्ष्वाश्रमकर्मसु}
{समाधौ योगमेवैतच्छाण्डिल्यः सममब्रवीत्}


\twolineshloka
{विदित्वा सप्तसूक्ष्माणि षडङ्गं च महेश्वरम्}
{प्रधानविनियोगज्ञः परं ब्रह्मानुपश्यति}


\chapter{अध्यायः २६०}
\twolineshloka
{व्यास उवाच}
{}


\twolineshloka
{हृदि कामद्रुमश्चित्रो मोहसंचयसंभवः}
{क्रोधमानमहास्कन्धो विवित्सापरिवेषणः}


\twolineshloka
{तस्य चाज्ञानमाधारः प्रमादः परिषेचनम्}
{सोऽभ्यसूयापलाशो हि पुरा दुष्कृतसारवान्}


\twolineshloka
{संमोहचिन्ताविटपः शोकशाखो भयाङ्कुरः}
{मोहनीभिः पिपासाभिर्लताभिरनुवेष्टितः}


\twolineshloka
{उपासते महावृक्षं सुलुब्धास्तत्फलेप्सवः}
{आयतैः संयुताः पाशैः फलदं परिवेष्ट्य तम्}


\twolineshloka
{यस्तान्पाशान्वशे कृत्वा तं वृक्षमपकर्षति}
{गतः स दुःखयोरन्तं जरामरणयोर्द्वयोः}


\twolineshloka
{संरोहत्यकृतप्रज्ञः ससत्वो हन्ति पादपम्}
{स तमेवाहतो हन्ति विषं ग्रस्तमिवातुरम्}


\twolineshloka
{तस्यानुगतमूलस्य मूलमुद्भियते बलात्}
{योगप्रसादात्कृतिना साम्येन परमासिना}


\twolineshloka
{एवं यो वेद कामस्य केवलं परिसर्पणम्}
{एतच्च कामशास्त्रस्य सुदुःखान्यतिवर्तते}


\twolineshloka
{शरीरं पुरमित्याहुः स्वामिनी बुद्धिरिष्यते}
{तत्र बुद्धेः शरीरस्थं मनो नामार्थचिन्तकम्}


\threelineshloka
{इन्द्रियाणि जनाः पौरास्तदर्थं तु परा कृतिः}
{तत्र द्वौ दारुणौ दोषौ तमो नाम रजस्तथा}
{तदर्थमुपजीवन्ति पौराः सह पुरेश्वरैः}


\twolineshloka
{अद्वारेण तमेवार्थं द्वौ दोषावुपजीवतः}
{तत्र बुद्धिर्हि दुर्धर्षा मनः साधर्म्यमुच्यते}


\twolineshloka
{पौराश्चापि मनस्तृप्तास्तेषामपि चला स्थितिः}
{यदर्थं बुद्धिरध्यास्ते सोऽनर्थः परिषीदति}


\twolineshloka
{`पौरमन्त्रवियुक्तायाः सोऽर्थः संसीदति क्रमात्'}
{यदर्थं पृथगध्यास्ते मनस्तत्परिषीदति}


\twolineshloka
{पृथग्भूतं मनो बुद्ध्या मनो भवति केवलम्}
{तत्रैनं विकृतं शून्यं रजः पर्यवतिष्ठते}


\twolineshloka
{तन्मनः कुरुते सख्यं रजसा सह संगतम्}
{तं चादाय जनं पौरं रजसे संप्रयच्छति}


\chapter{अध्यायः २६१}
\twolineshloka
{भीष्म उवाच}
{}


\twolineshloka
{भूतानां गुणसङ्ख्यानं भूयः पुत्र निशामय}
{द्वैपायनमुखाद्धष्टं श्लाघया परयाऽनघ}


\twolineshloka
{दीप्तानलनिभः प्राह भगवान्धूमवत्सलः}
{ततोऽहमपि वक्ष्यामि भूयः पुत्र निदर्शनम्}


\twolineshloka
{भूमेः स्थैर्यं गुरुत्वं च काठिन्यं प्रसवात्मता}
{गन्धो भारश्च शक्तिश्च संघातः स्थापना धृतिः}


\twolineshloka
{अपां शैत्यं रसः क्लेदो द्रवत्वं स्नेहसौम्यता}
{जिह्वाविस्यन्दनं चापि भौमानां श्रपणं तथा}


\twolineshloka
{अग्नेर्दुर्धर्षता ज्योतिस्तापः पाकः प्रकाशनम्}
{शौचं रागो लघुस्तैक्ष्ण्यं सततं चोर्ध्वभागिता}


\twolineshloka
{वायोरनियमस्पर्शो वादस्थानं स्वतन्त्रता}
{बलं शैध्यं च मोक्षं च कर्म चेष्टात्मता भवः}


\twolineshloka
{आकाशस्य गुणः शब्दो व्यापित्वं छिद्रताऽपि च}
{अनाश्रयमनालम्बमव्यक्तमविकारिता}


\twolineshloka
{अप्रतीघातिता चैव श्रोतत्वं विवराणि च}
{गुणाः पञ्चाशतं प्रोक्ताः पञ्चभूतविभाविताः}


\twolineshloka
{फलोपपत्तिर्व्यक्तिश्च विसर्गः कल्पना क्षमा}
{सदसच्चाशुता चैव मनसो नव वै गुणाः}


\threelineshloka
{इष्टानिष्टविपत्तिश्च व्यवसायः समाधिता}
{संशयः प्रतिपत्तिश्च बुद्धेः पञ्च गुणान्विदुः ॥युधिष्ठिर उवाच}
{}


\threelineshloka
{कथं पञ्चगुणा बुद्धिः कथं पञ्चेन्द्रिया गुणाः}
{एतन्मे सर्वमाचक्ष्व सूक्ष्मज्ञानं पितामह ॥भीष्म उवाच}
{}


\twolineshloka
{आहुः षष्टिं भूतगुणान्वैभूतविषक्तान्प्रकृतिविसृष्टान्}
{नित्यविषक्तांश्चाक्षरसृष्टान्पुत्र न नित्यं तदिह वदन्ति}


\twolineshloka
{तत्पुत्रचिन्ताकलिलं तदुक्तमनागतं वै तव संप्रतीह}
{भूतार्थवत्त्वं तदवाप्य सर्वंभूतप्रभावाद्भव शान्तबुद्धिः}


\chapter{अध्यायः २६२}
\twolineshloka
{युधिष्ठिरं उवाच}
{}


\twolineshloka
{य इमे पृथिवीपालाः शेरते पृथिवीतले}
{पृतनामध्य एते हि गतसत्त्वा महाबलाः}


\twolineshloka
{एकैकशो भीमबला नागायुतबलास्तथा}
{एते हि निहताः सङ्ख्ये तुल्यतेजीबलैर्नरैः}


\twolineshloka
{नैषां पश्यामि हन्तारं प्राणिनां संयुगे पुरा}
{विक्रमेणोपसंपन्नास्तेजोबलसमन्विताः}


\twolineshloka
{अथ चेमे महाप्राज्ञाः शेरते हि गतासवः}
{मृता इति च शब्दोऽयं वर्तत्येषु गतासुषु}


\twolineshloka
{इमे मृता नृपतयः प्रायशो भीमविक्रमाः}
{तत्र मे संशयो जातः कुतः संज्ञा मृता इति}


\threelineshloka
{कस्य मृत्युः कुतो मृत्युः केन मृत्युरिह प्रजाः}
{हरत्यमरसंकाश तन्मे ब्रूहि पितामह ॥भीष्म उवाच}
{}


\twolineshloka
{पुरा कृतयुगे तात राजा ह्यासीदकम्पनः}
{स शत्रुवशमापन्नः संग्रामे क्षीणवाहनः}


\twolineshloka
{तस्य पुत्रो हरिर्नाम नारायणसमो बले}
{स शत्रुभिर्हतः सङ्ख्ये सबलः सपदानुगः}


\twolineshloka
{स राजा शत्रुवशगः पुत्रशोकसमन्वितः}
{यदृच्छया शान्तिपरो ददर्श भुवि नारदम्}


\twolineshloka
{तस्मै स सर्वमाचष्ट यथावृत्तं जनेश्वरः}
{शत्रुभिर्ग्रहणं सङ्ख्ये पुत्रस्य मरणं तथा}


\threelineshloka
{तस्य तद्वचनं श्रुत्वा नारदोऽथ तपोधनः}
{आख्यानमिदमाचष्ट पुत्रशोकापहं तदा ॥नारद उवाच}
{}


\twolineshloka
{राजञ्शृणु महाख्यानं ममेदं बहुविस्तरम्}
{यथावृत्तं श्रुतं चैव मयाऽपि वसुधाधिप}


\twolineshloka
{प्रजाः सृष्ट्वा महातेजाः प्रजासर्गे पितामहः}
{अतीव वृद्धा बहुला नामृष्यत पुनः प्रजाः}


\twolineshloka
{न ह्यन्तरमभूत्किंचित्क्वचिज्जन्तुभिरच्युत}
{निरुच्छ्वासमिवोन्नद्धं त्रैलोक्यमभवन्नृप}


\twolineshloka
{तस्य चिन्ता समुत्पन्ना संहारं प्रति भूपते}
{चिन्तयन्नाध्यगच्छच्च संहारे हेतुकारणम्}


\twolineshloka
{तस्य रोपान्महाराज खेभ्योऽग्निरुदतिष्ठत}
{तेन सर्वा दिशो राजन्ददाह स पितामहः}


\twolineshloka
{ततो दिवं भुवं खं च जगच्च सचराचरम्}
{ददाह पावको राजन्भगवत्कोपसंभवः}


\twolineshloka
{तत्रादह्यन्त भूतानि जङ्गमानि ध्रुवाणि च}
{महता क्रोधवेगेन कुपिते प्रपितामहे}


\twolineshloka
{ततो हरो जटी स्थाणुर्देवोऽध्वरपतिः शिवः}
{जगाम शरणं देवो ब्रह्माणं परमेष्ठिनम्}


\twolineshloka
{तस्मिन्नभिगते स्थाणौ प्रजानां हितकाम्यया}
{अब्रवीद्वरदो देवो ज्वलन्निव तदा शिवम्}


\twolineshloka
{करवाण्यद्य कं कामं व्नरार्होऽसि मतो मम}
{कर्ता ह्यसि प्रियं शंभो तव यद्धृदि वर्तते}


\chapter{अध्यायः २६३}
\twolineshloka
{स्थाणुरुवाच}
{}


\twolineshloka
{प्रजासर्गनिमित्तं मे कार्यवत्तामिमां प्रभो}
{विद्धि सृष्टास्त्वया हीमा मा कुप्यासां पितामह}


\threelineshloka
{तव तेजोग्निना देव प्रजा दह्यन्ति सर्वशः}
{ता दृष्ट्वा मम कारुण्यं मा कुप्यासां जगत्प्रभो ॥प्रजापतिरुवाच}
{}


\twolineshloka
{न कुप्ये न च मे कामो नभवेयुः प्रजा इति}
{लाघवार्थं धरण्यास्तु ततः संहार इष्यते}


\twolineshloka
{इयं हि मां सदा देवी भारार्ता समचोदयत्}
{संहारार्थं महादेव भारेणाप्सु निमज्जती}


\threelineshloka
{यदाऽहं नाधिगच्छामि बुद्ध्या बहु विचारयन्}
{संहारमासां वृद्धानां ततो मां क्रोध आविशत् ॥स्थाणुरुवाच}
{}


\twolineshloka
{संहारात्त्वं निवर्तस्य मा क्रुधो विवुधेश्वर}
{मा प्रजाः स्थावरं चैव जङ्गमं च व्यनीनशः}


\twolineshloka
{पल्वलानि च सर्वाणि सर्वं चैव तृणीलपम्}
{स्थावरं जङ्गमं चैव भूतग्रामं चतुर्विधम्}


\twolineshloka
{अकाले भस्मसाद्भूतं जगत्सर्वमुपप्लुतम्}
{प्रसीद भगवन्साधो वर एष वृतो मया}


\twolineshloka
{नष्टा न पुनरेष्यन्ति प्रजा ह्येताः कथंचन}
{तस्मान्निवर्ततामेतत्तेन स्वेनेव तेजसा}


\twolineshloka
{उपायमन्यं संपश्य भूतानां हितकाम्यया}
{यथामी जन्तवः सर्वे न दह्येरन्पितामह}


\threelineshloka
{अभावं हि न गच्छेयुरुत्सन्नप्रजनाः प्रजाः}
{`पुत्रत्वेनानुसंकल्प्ये तदाऽहं तप्य दानवैः}
{'अधिदैवे नियुक्तोस्मि त्वया लोकहितेप्सुना}


\threelineshloka
{त्वद्भवं हि जगन्नाथ एतत्स्थावरजङ्गमम्}
{प्रसाद्य त्वां महादेव याचाम्यावृत्तिजाः प्रजाः ॥नारद उवाच}
{}


\twolineshloka
{श्रुत्वा तु वचनं देवः स्थाणोर्नियतवाङ्भनाः}
{तेजस्तत्सन्निजग्राह पुनरेवान्तरात्मनि}


\twolineshloka
{ततोऽग्निमुपसंगृह्य भगवाँल्लोकपूजितः}
{प्रवृत्तिं च निवृत्तिं च कल्पयामास वै प्रभुः}


\twolineshloka
{उपसंहरतस्तस्य तमग्निं रोषजं तदा}
{प्रादुर्बभूव विश्वेभ्यः खेभ्यो नारी महात्मनः}


\twolineshloka
{कृष्णरक्ताम्बरधरा कृष्णनेत्रतलान्तरा}
{दिव्यकुण्डलसंपन्ना दिव्याभरणभूषिता}


\twolineshloka
{सा विनिःसृत्य वै खेभ्यो दक्षिणामाश्रिता दिशम्}
{ददृशाते च तां कन्यां देवौ विश्वेश्वरावुभौ}


\twolineshloka
{तामाहूय तदा देवो लोकानामादिरीश्वरः}
{मृत्यो इति महीपाल जहि चेमाः प्रजा इति}


\twolineshloka
{त्वं हि संहारबुद्ध्या मे चिन्तिता रुषितेन च}
{तस्मात्संहर सर्वास्त्वं प्रजाः सजडपण्डिताः}


\twolineshloka
{अविशेषेण चैव त्वं प्रजाः संहर कामिनि}
{मम त्वं हि नियोगेन श्रेयः परमवाप्स्यसि}


\twolineshloka
{एवमुक्ता तु या देवी मृत्युः कमलमालिनी}
{प्रदध्यौ दुःखिता बाला साश्रुपातमतीव च}


\twolineshloka
{पाणिभ्यां चैव जग्राह तान्यश्रूणि जनेश्वरः}
{मानवानां हितार्थाय ययाचे पुनरेव ह}


\chapter{अध्यायः २६४}
\twolineshloka
{नारद उवाच}
{}


\twolineshloka
{विनीय दुःखमबला साऽऽत्मनैवायतेक्षणा}
{उवाच प्राञ्जलिर्भूत्वा तमेवावर्जिता तदा}


\twolineshloka
{त्वया सृष्टा कथं नारी मादृशी वदतां वर}
{रौद्रकर्माभिजायेत सर्वप्राणिभयंकरी}


\twolineshloka
{विभेम्यहमधर्मस्य धर्म्यमादिश कर्म मे}
{त्वं मां भीतामवेक्षस्व शिवेनेश्वरचक्षुषा}


\twolineshloka
{बालान्वृद्धान्वयस्थांश्च न हरेयमनागसः}
{प्राणिनः प्राणिनामीश नमस्तेऽस्तु प्रसीद मे}


\twolineshloka
{प्रियान्पुत्रान्वयस्यांश्च भ्रातॄन्मातॄः पितृनपि}
{अपध्यास्यन्ति यद्येवं मृतास्तेभ्यो विभेम्यहम्}


\twolineshloka
{कृपणाश्रुपरिक्लेदो दहेन्मां शाश्वतीः समाः}
{तेभ्योऽहं बलवद्भीता शरणं त्वामुपागता}


\twolineshloka
{यमस्य भवने देव पात्यन्ते पापकर्मिणः}
{प्रसादये त्वां वरद प्रसादं कुरु मे प्रभो}


\threelineshloka
{एतदिच्छाम्यहं कामं त्वत्तो लोकपितामह}
{इच्छेयं त्वत्प्रसादाच्च तपस्तप्तुं महेश्वर ॥पितामह उवाच}
{}


\twolineshloka
{त्वं हिं संहारबुद्ध्या मे चिन्तिता रुषितेन च}
{तस्मात्संहर सर्वास्त्वं प्रजा मा च विचारय}


\twolineshloka
{एतदेवमवश्यं हि भविता नैतदन्यथा}
{क्रियतामनवद्याङ्गि यथोक्तं मद्वचोऽनघे}


\twolineshloka
{एवमुक्ता महाबाहो मृत्युः परपुरंजय}
{न व्याजहार तस्थौ च प्रह्वा भगवदुन्मुखी}


\twolineshloka
{पुनः पुनरथोक्ता सा गतसत्त्वेव भामिनी}
{तूष्णीमासीत्ततो देवो लोकानामीश्वरेश्वरः}


\twolineshloka
{प्रससाद किल ब्रह्मा स्वयमेवात्मनाऽऽत्मनि}
{स्मयमानश्च लोकेशो लोकान्सर्वानवैक्षत}


\twolineshloka
{निवृत्तरोषे तस्मिंस्तु भगवत्यपराजिते}
{सा कन्याऽथ जगामास्य समीपादिति नःश्रुतम्}


\twolineshloka
{अपसृत्याप्रतिश्रुत्य प्रजासंहरणं तदा}
{त्वरमाणेव राजेन्द्र मृत्युर्धेनुकमभ्यगात्}


\twolineshloka
{सा तत्र परमं देवी तपोऽचरत दुश्चरम्}
{समा ह्येकपदे तस्थौ दशपद्मानि पञ्च च}


\twolineshloka
{तां तथा कुर्वतीं तत्र तपः परमदुश्चरम्}
{पुनरेव महातेजा ब्रह्मा वचनमब्रवीत्}


\twolineshloka
{कुरुष्व मे वचो मृत्यो तदनादृत्य सत्वरा}
{तथैवैकपदे तात पुनरन्यानि सप्त सा}


\twolineshloka
{तस्थौ पद्मानि षट् चैव पञ्च द्वे चैव मानद}
{भूयः पद्मायुतं तात मृगैः सह चचार सा}


\twolineshloka
{द्वे चायुते नरश्रेष्ठ वाय्वाहारा महामते}
{पुनरेव ततो राजन्मौनमातिष्ठदुत्तमम्}


\twolineshloka
{अप्सु वर्षसहस्राणि सप्त चैकं च पार्थिव}
{ततो जगाम सा कन्या कौशिकीं नृपसत्तम}


\twolineshloka
{तत्र वायुजलाहारा चचार नियमं पुनः}
{ततो ययौ महाभागा गङ्गां मेरुं च केवलम्}


\twolineshloka
{तस्थौ दार्विव निश्चेष्टा प्रजानां हितकाम्यया}
{ततो हिमवतो मूर्ध्निं यत्र देवाः समीजिरे}


\twolineshloka
{तत्राङ्गुष्ठेन राजेन्द्र निखर्वमचरत्तपः}
{तस्थौ पितामहं चैव तोषयामास यत्नतः}


\twolineshloka
{ततस्तामब्रवीत्तत्र लोकानां प्रपितामहः}
{किमिदं वर्तसे पुत्रि क्रियतां मम तद्वचः}


\twolineshloka
{ततोऽब्रवीत्पुनर्मृत्युर्भगवन्तं पितामहम्}
{न हरेयं प्रजा देव पुनस्त्वाऽहं प्रसादये}


\twolineshloka
{तामधर्मभयाद्भीतां पुनरेव प्रयाचतीम्}
{तदाऽब्रवीद्देवदेवो निगृह्येदं वचस्ततः}


\twolineshloka
{अधर्मो नास्ति ते मृत्यो संयच्छेमाः प्रजाः शुभे}
{मयाऽप्युक्तं मृषा भद्रे भविता नेह किंचन}


\twolineshloka
{धर्मः सनातनश्च त्वामिहैवानुप्रवेक्ष्यति}
{अहं च विबुधाश्चैव त्वद्धिते निरताः सदा}


\twolineshloka
{इमन्यं च ते कामं ददानि मनसेप्सितम्}
{न त्वां दोषेण यास्यन्ति व्याधिसंपीडिताः प्रजाः}


\twolineshloka
{पुरुषेषु च रूपेण पुरुषस्त्वं भविष्यसि}
{स्त्रीषु स्त्रीरूपिणी चैव तृतीयेषु नपुंसकम्}


\twolineshloka
{सैवमुक्ता महाराज कृताञ्जलिरुवाच ह}
{पुनरेव महात्मानं नेति देवेशमव्ययम्}


\twolineshloka
{तामब्रवीत्तदा देवो मृत्यो संहर मानवान्}
{अधर्मस्ते न भविता यथा ध्यास्याम्यहं शुभे}


\twolineshloka
{`त्वं हि शक्ता च युक्ता च पूर्वोत्पन्ना च भामिनि}
{अनुशिष्टा च निर्दोषा तस्मात्त्वं कुरु मे मतम्'}


\twolineshloka
{यानश्रुबिन्दून्पतितानपश्यंये पाणिभ्यां धारितास्ते पुरस्तात्}
{ते व्याधयो मानवान्घोररूपाःप्राप्ते काले पीडयिष्यन्ति मृत्यो}


\twolineshloka
{सर्वेषां त्वं प्राणिनामन्तकालेकामक्रोधौ सहितौ योजयेथाः}
{एवं धर्मस्त्वामुपैष्यत्यमोघोन चाधर्मं लप्स्यसे तुल्यवृत्तिः}


\twolineshloka
{एवं धर्मं पालयिष्यस्यथो त्वंन चात्मानं मञ्जयिष्यस्यधर्मे}
{तस्मात्कामं रोचयाभ्यागतं त्वंसा त्वं साधो संहरस्वेह जन्तून्}


\twolineshloka
{सा वै तदा मृत्युसंज्ञा कृतास्त्रीशापाद्भीता बाढमित्यब्रवीत्तम्}
{अथो प्राणान्प्राणिनामन्तकालेकामक्रोधौ प्राप्य नित्यं निहन्ति}


\twolineshloka
{मृत्योर्ये ते व्याधयश्चाश्चुपातामनुष्याणां युज्यते यैः शरीरम्}
{सर्वेषां वै प्राणिनां प्राणनान्तेतस्माच्छोकं मा कृथा बुद्ध्य बुद्ध्या}


\twolineshloka
{सर्वे देवाः प्राणिनां प्राणनान्तेगत्वा वृत्ताः सन्निवृत्तास्तथैव}
{एवं सर्वे मानवाः प्राणनान्तेगत्वा वृत्ता देववद्राजसिंह}


\twolineshloka
{वायुर्भीमो भीमनादो महौजाःसर्वेषां च प्राणिनां प्राणभूतः}
{अनावृत्तिर्देहिनां देहपातेतस्माद्वायुर्देवदेवो विशिष्टः}


\twolineshloka
{सर्वे देवा मर्त्यसंज्ञाविशिष्टाःसर्वे मर्त्या देवसंज्ञाविशिष्टाः}
{तस्मात्पुत्रं मा शुचो राजसिंहपुत्रः स्वर्गं प्राप्यते मोदते हे}


\twolineshloka
{एवं मृत्युर्देवसृष्टा प्रजानांप्राप्ते काले संहरन्ती यथावत्}
{तस्याश्चैव व्याधयस्तेऽश्रुपाताःप्राप्ते काले संहरन्तीह जन्तून्}


\chapter{अध्यायः २६५}
\twolineshloka
{युधिष्ठिर उवाच}
{}


\twolineshloka
{इमे वै मानसाः सर्वे धर्मं प्रति विशङ्किताः}
{कोऽयं धर्मः कुतो धर्मस्तन्मे ब्रूहि पितामह}


\threelineshloka
{धर्मस्त्वयमिहार्थः किममुत्रार्थोपि वा भवेत्}
{उभयार्थो हि वा धर्मस्तन्मे ब्रूहि पितामह ॥भीष्म उवाच}
{}


\twolineshloka
{सदाचारः स्मृतिर्वेदास्त्रिविधं धर्मलक्षणम्}
{चतुर्थमर्थमप्याहुः कवयो धर्मलक्षणम्}


\twolineshloka
{अविध्युक्तानि कर्माणि व्यवस्यन्त्युप्तमूषरे}
{लोकयात्रार्थमेवेह धर्मस्य नियमः कृतः}


\twolineshloka
{उभयत्र सुखोदर्क इह चैव परत्र च}
{अलब्ध्वा निपुणं धर्मं पापः पापे प्रसज्जति}


\threelineshloka
{न च पापकृतः पापान्मुच्यन्ते केचिदापदि}
{अपापवादी भवति यथा भवति धर्मवित्}
{धर्मस्य निष्ठा स्वाचारस्तमेवाश्रित्य चावसेत्}


\twolineshloka
{यथाधर्मसमाविष्टो धनं गृह्णाति तस्करः}
{रमते निर्हरस्तेनः परवित्तमराजके}


\twolineshloka
{यदास्य तद्धरन्त्यन्ये तदा राजानमिच्छति}
{तदा तेषां स्पृहयते ये वै तुष्टाः स्वकैर्धनैः}


\twolineshloka
{अभीतः शुचिरभ्येति राजद्वारमशङ्कितः}
{न हि दुश्चरितं किंचिदन्तरात्मनि पश्यति}


\twolineshloka
{सत्यस्य वचनं साधु न सत्याद्विद्यते परम्}
{सत्येन विधृतं सर्वं सर्वं सत्ये प्रतिष्ठितम्}


\twolineshloka
{अपि पापकृतो रौद्राः सत्यं कृत्वा मिथःकृतम्}
{अद्रोहमविसंवादं प्रवर्तन्ते तदाश्रयाः}


\twolineshloka
{ते चेन्मिथ्या धृतिं कुर्युर्विनश्येयुरसंशयम्}
{न हर्तव्यं परधनमिति धर्मविदो विदुः}


\twolineshloka
{मन्यन्ते बलवन्तस्तं दुर्बलैः संप्रवर्तितम्}
{यदा नियतिदौर्बल्यमथैषामेव रोचते}


\twolineshloka
{न ह्यत्यन्तं बलयुता भवन्ति सुखिनोपि वा}
{तस्मादनार्जवे बुद्धिर्न कार्या ते कदाचन}


\twolineshloka
{असाधुभ्योऽस्य न भयं न चौरेभ्यो न राजतः}
{अकिंचित्कस्यचित्कुर्वन्निर्भयः शुचिरावसेत्}


\twolineshloka
{सर्वतः शङ्कते स्तेनो मृगो ग्राममिवेयिवान्}
{बहुधाऽऽचरितं पापमन्यत्रैवानुपश्यति}


\twolineshloka
{मुदितः शुचिरभ्येति सर्वतो निर्भयः सदा}
{न हि दुश्चरितं किंचिदात्मनोऽन्येषु पश्यति}


\twolineshloka
{दातव्यमित्ययं धर्म उक्तो भूतहिते रतैः}
{तं मन्यन्ते धनयुताः कृपणैः संप्रवर्तितम्}


\twolineshloka
{यदा नियतिकार्पण्यमथैपामव रोचते}
{धनवन्तोपि नात्यन्तं भवन्ति सुखिनोपि वा}


\twolineshloka
{यदन्यैर्विहितं नेच्छेदात्मनः कर्म पूरुषः}
{न तत्परेषु कुर्वीत जानन्नप्रियमात्मनः}


\twolineshloka
{योऽन्यस्य स्यादुपपतिः स कं किं वक्तुमर्हति}
{यदन्यस्य ततः कुर्यान्न मृष्येदिति मे मतिः}


\twolineshloka
{जीवितुं यः स्वयं चेच्छेत्कथं सोऽन्यं प्रघातयेत्}
{यद्यदात्मन इच्छेत तत्परस्यापि चिन्तयेत्}


\twolineshloka
{अतिरिक्तः संविभजेद्भोगैरन्यानकिंचनान्}
{एतस्मात्कारणाद्धात्रा कुसीदं संप्रवर्तितम्}


\twolineshloka
{यस्मिंस्तु देवाः समये सन्तिष्ठेरंस्तथा भवेत्}
{अथ चेल्लोभसमये स्थितिर्धर्मोऽपि शोभना}


\twolineshloka
{सर्वं प्रियाभ्युपगतं पुण्यमाहुर्मनीषिणः}
{पश्यैतं लक्षणोद्देशं धर्माधर्मे युधिष्ठिर}


\twolineshloka
{लोकसंग्रहसंयुक्तं विधात्रा विहितं पुरा}
{सूक्ष्मधर्मार्थनियतं सतां चरितमुत्तमम्}


\twolineshloka
{धर्मलक्षणमाख्यातमेतत्ते कुरुसत्तम}
{तस्मादनार्जवे बुद्धिर्न ते कार्या कथंचन}


\chapter{अध्यायः २६६}
\twolineshloka
{युधिष्ठिर उवाच}
{}


\twolineshloka
{सूक्ष्मं साधु समादिष्टं भवता धर्मलक्षणम्}
{प्रतिभा त्यस्ति मे काचित्तां ब्रूयामनुमानतः}


\twolineshloka
{भूयांसो हृदये ये मे प्रश्नास्ते व्याहृतास्त्वया}
{इदं त्वन्यत्प्रवक्ष्यामि न राजन्निग्रहादिव}


\twolineshloka
{इमानि हि प्राणयन्ति सृदन्त्युत्तारयन्ति च}
{न धर्मः परिपाठेन शक्यो भारत वेदितुम्}


\twolineshloka
{अन्यो धर्मः समस्थस्य विषमस्थस्य चापरः}
{आपदस्तु कथं शक्याः परिपाठेन वेदितुम्}


\twolineshloka
{सदाचारो मचो धर्मः संतस्त्वाचारलक्षणाः}
{साध्यासाध्यं कथं शक्यं सदाचारो ह्यलक्षणः}


\twolineshloka
{दृश्यते धर्मरूपेण ह्यधर्मं प्राकृतश्चरन्}
{धर्मं चाधर्मरूपेण कश्चिदप्राकृतश्चरन्}


\twolineshloka
{पुनरस्य प्रमाणं हि निर्दिष्टं शास्त्रकोविदैः}
{वेदवादाश्चानुयुगं ह्रसन्तीतीह नः श्रुतम्}


\twolineshloka
{अन्ये कृतयुगे धर्मास्त्रेतायां द्वापरे परे}
{अन्ये कलियुगे धर्मा यथाशक्ति कृता इव}


\twolineshloka
{आम्नायवचनं सत्यमित्ययं लोकसंग्रहः}
{आम्नायेभ्यः पुनर्वेदाः प्रसृताः सर्वतोमुखाः}


\twolineshloka
{ते चेत्सर्वप्रमाणं वै प्रमाणं ह्यत्र विद्यते}
{प्रमाणं च प्रमाणेन विरुद्ध्येच्छास्त्रता कुतः}


\twolineshloka
{धर्मस्य क्रियमाणस्य बलवद्भिर्दुरात्मभिः}
{यदा विक्रियते संस्था ततः साऽपि प्रणश्यति}


\twolineshloka
{विद्मश्चैनं न वा विद्मः शक्यं वा वेदितुं न वा}
{अणीयान्क्षुरधाराया गरीयानपि पर्वतात्}


\twolineshloka
{गन्धर्वनगराकारः प्रथमं संप्रदृश्यते}
{अन्वीक्ष्यमाणः कविभिः पुनर्गच्छत्यदर्शनम्}


\twolineshloka
{निपानानीव गोभ्याशे क्षेत्रे कुल्ये च भारत}
{स्मृतो हि शाश्वतो धर्मो विप्रहीणो न दृश्यते}


\twolineshloka
{कामादन्ये भयादन्ये कारणैरपरैस्तथा}
{असन्तोऽपि वृथाचारं भजन्ते बहवोऽपरे}


\twolineshloka
{धर्मो भवति स क्षिप्रं विलोमस्तेष्वसाधुषु}
{अथैतानाहुरुन्मत्तानपि चावहसन्त्युत}


\twolineshloka
{महाजना ह्युपावृत्ता राजधर्मं समाश्रिताः}
{न हि सर्वहितः कश्चिदाचारः संप्रवर्तते}


\twolineshloka
{तेनैवान्यः प्रभवति सोऽपरं बाधते पुनः}
{दृश्यते चैव स पुनस्तुल्यरूपो यदृच्छया}


\twolineshloka
{येनैवान्यः प्रभवति सोऽपरानपि बाधते}
{आचाराणामनैकाग्र्यं सर्वेषामेव लक्षयेत्}


\twolineshloka
{चिराभिपन्नः कविभिः पूर्वं धर्म उदाहृतः}
{तेनाचारेण पूर्वेण संस्था भवति शाश्वती}


\chapter{अध्यायः २६७}
\twolineshloka
{भीष्म उवाच}
{}


\twolineshloka
{अत्राप्युदाहरन्तीममितिहासं पुरातनम्}
{तुलाधारस्य वाक्यानि धर्मे जाजलिना सह}


\twolineshloka
{वने वनचरः कश्चिज्जाजलिर्नाम वै द्विजः}
{सागरोद्देशमागम्य तपस्तेपे महातपाः}


\twolineshloka
{नियतो नियताहारश्चीराजिनजटाधरः}
{मलपङ्कधरो धीमान्बहून्वर्षगणान्मुनिः}


\twolineshloka
{स कदाचिन्महातेजा जलवासो महीपते}
{चचार लोकान्विप्रर्षिः प्रेक्षमाणो मनोजवः}


\twolineshloka
{स चिन्तयामास मुनिर्जलमध्ये कदाचन}
{विप्रेक्ष्य सागरान्तां वै महीं सवनकाननाम्}


\twolineshloka
{न मया सदृशोऽस्तीह लोके स्थावरजङ्गमे}
{अप्सु वैहायसं गच्छेन्मया योऽन्यः सहेति वै}


\twolineshloka
{स दृश्यमानो रक्षोभिर्जलमध्ये च भारत}
{आस्फोटयत्तदाऽऽकाशे धर्मः प्राप्तो मयेति वै}


\threelineshloka
{अब्रुवंश्च पिशाचास्तं नैवं त्वं वक्तुमर्हसि}
{तुलाधारो वणिग्धर्मा वाराणस्यां महायशाः}
{सोऽप्येवं नार्हते वक्तुं यथा त्वं द्विजसत्तम}


\twolineshloka
{इत्युक्तो जाजलिर्भूतैः प्रत्युवाच महातपाः}
{पश्येयं तमहं प्राज्ञं तुलाधारं यशस्विनम्}


\twolineshloka
{इति ब्रुवाणं तमृषिं रक्षांस्युत्थाय सागरात्}
{अब्रुवन्गच्छ पन्थानमास्थायेमं द्विजोत्तम्}


\threelineshloka
{इत्युक्तो जाजलिर्भूतैर्जगाम विमनास्तदा}
{वाराणस्यां तुलाधारं समासाद्याब्रवीदिदम् ॥युधिष्ठिर उवाच}
{}


\threelineshloka
{किं कृतं दुष्करं तात कर्म जाजलिना पुरा}
{येन सिद्धिं परां प्राप्तस्तन्मे व्याख्यातुमर्हसि ॥भीष्म उवाच}
{}


\twolineshloka
{अतीव तपसा युक्तो घोरेण स बभूव ह}
{नद्युपस्पर्शनपरः सायंप्रातर्महातपाः}


\twolineshloka
{अग्नीन्परिचरन्सम्यक्स्वाध्यायपरमो द्विजः}
{वानप्रस्थ विधानज्ञो जाजलिर्ज्वलितः श्रिया}


\twolineshloka
{वने तपस्यतिष्ठत्स न चाधर्ममवैक्षत}
{वर्षास्वाकाशशायी च हेमन्ते जलसंश्रयः}


\twolineshloka
{वातातपसहो ग्रीष्मे न चाधर्ममविन्दत}
{दुःखशय्याश्च विविधा भूमौ च परिवर्तनम्}


\twolineshloka
{ततः कदाचित्स मुनिर्वर्षास्वाकाशमास्थितः}
{अन्तरिक्षाज्जलं मूर्ध्नां प्रत्यगृह्णान्मुहुर्मुहुः}


\twolineshloka
{आप्लुतस्य जटाः क्लिन्ना बभूवुर्ग्रथिताः प्रभो}
{अरण्यगमनान्नित्यं मलिनोऽमलसंयुतः}


\twolineshloka
{स कदाचिन्निराहारो वायुभक्षो महातपाः}
{तस्थौ काष्ठवदव्यग्रो न चचाल च कर्हिचित्}


\twolineshloka
{तस्य स्म स्थाणुभूतस्य निर्विचेष्टस्य भारत}
{कुलिङ्गशकुनौ राजन्नीडं शिरसि चक्रतुः}


\twolineshloka
{स तौ दयावान्ब्रह्मर्षिरुपप्रैक्षत दंपती}
{कुर्वाणौ नीडकं तत्र जटासु तृणतन्तुभिः}


\twolineshloka
{यदा न स चलत्येव स्थाणुभूतो महातपाः}
{ततस्तौ सुखविश्वस्तौ सुखं तत्रोषतुस्तदा}


\twolineshloka
{अतीतास्वथ वर्षासु शरत्काल उपस्थिते}
{प्राजापत्येन विधिना विश्वासात्काममोहितौ}


\twolineshloka
{तत्रोत्पादयतां राजञ्शिरस्यण्डानि खेचरौ}
{तान्यबुध्यत तेजस्वी स विप्रः संशितव्रतः}


\twolineshloka
{बुद्ध्वा च स महातेजा न चचाल च जाजलिः}
{धर्मे कृतमना नित्यं नाधर्मं स त्वरोचयत्}


\twolineshloka
{अहन्यहनि चागत्य ततस्तौ तस्य मूर्धनि}
{आश्वासितौ निवसतः संप्रहृष्टौ तदा विभौ}


\twolineshloka
{अण्डेभ्यस्त्वथ पुष्टेभ्यः प्राजायन्त शकुन्तकाः}
{व्यवर्धन्त च तत्रैव न चाकम्पत जाजलिः}


\twolineshloka
{स रक्षमाणस्त्वण़्डानि कुलिङ्गानां धृतव्रतः}
{तथैव तस्थौ धर्मात्मा निर्विचेष्टः समाहितः}


\twolineshloka
{ततस्तु काले राजेन्द्र बभूवुस्तेऽथ पक्षिणः}
{बुबुधे तांस्तु स मुनिर्जातपक्षान्कुलिङ्गकान्}


\twolineshloka
{ततः कदाचित्तांस्तत्र पश्यन्पक्षीन्यतव्रतः}
{बभूव परमप्रीतस्तदा मतिमतां वरः}


\twolineshloka
{तथा तानभिसंवृद्धान्दृष्ट्वा चैवाप्तवान्मुदम्}
{शकुनौ निर्भयौ तत्र ऊषतुश्चात्मजैः सह}


\twolineshloka
{जातपक्षांश्च सोऽपश्यदुड्डीनान्पुनरागतान्}
{सायंसायं द्विजान्विप्रो न चाकम्पत जाजलिः}


\twolineshloka
{कदाचित्पुनरभ्येत्य पुनर्गच्छन्ति संततम्}
{त्यक्ता मातापितृभ्यां ते नचाकम्पत जाजलिः}


\twolineshloka
{तथा ते दिवसं चापि गत्वा सायं पुनर्नृप}
{उपावर्तन्त तत्रैव निवासार्थं शकुन्तकाः}


\twolineshloka
{कदाचिद्दिवसान्पञ्च समुत्पत्य विहगमाः}
{षष्ठेऽहनि समाजग्मुर्न चाकम्पत जाजलिः}


\twolineshloka
{क्रमेण च पुनः सर्वे दिवसान्सुबहूनथ}
{नोपावर्तन्त शकुना जातपक्षाश्च ते यदा}


\twolineshloka
{कदाचिन्मासमात्रेण समुत्पत्य विहंगमाः}
{नैवागच्छंस्ततो राजन्प्रातिष्ठत स जाजलिः}


\twolineshloka
{ततस्तेषु प्रलीनेषु जाजलिर्जातविस्मयः}
{सिद्धोस्मीति मतिं चक्रे ततस्तं मान आविशत्}


\twolineshloka
{स तथा निर्गतान्दृष्ट्वा शकुन्तान्नियतव्रतः}
{संभावितात्मा संभाव्य भृशं प्रीतस्तदाऽभवत्}


\twolineshloka
{स नद्यां समुपस्पृश्य तर्पयित्वा हुताशनम्}
{उदयन्तमथादित्यमभ्यागच्छन्महातपाः}


\twolineshloka
{संभाव्य चटकान्मूर्ध्नि जाजलिर्जपतांवरः}
{आस्फोटयत्तथाऽऽकाशे धर्मः प्राप्तो मयेति वै}


\twolineshloka
{अथान्तरिक्षे वागासीत्तां च शुश्राव जाजलिः}
{धर्मेण न समस्त्वं वै तुलाधारस्य जाजले}


\twolineshloka
{वाराणस्यां महाप्राज्ञस्तुलाधारः प्रतिष्ठितः}
{सोऽप्येवं नार्हते वक्तुं यथा त्वं भाषसे द्विज}


\twolineshloka
{सोमर्षवशमापन्नस्तुलाधारदिदृक्षया}
{पृथिवीमचरद्राजन्यत्रसायंगृहो मुनिः}


\twolineshloka
{कालेन महताऽगच्छत्स तु वाराणसीं पुरीम्}
{विक्रीणन्तं च पण्यानि तुलाधारं ददर्श सः}


\threelineshloka
{सोऽपि दृष्ट्वैव तं विप्रमायान्तं भाण्डजीवनः}
{समुत्थाय सुसंहृष्टः स्वागतेनाभ्यपूजयत् ॥तुलाधार उवाच}
{}


\twolineshloka
{आयानेवासि विदितो मम ब्रह्मन्न संशयः}
{ब्रवीमि यत्तु वचनं तच्छृणुष्व द्विजोत्तम}


\twolineshloka
{सागरानूपमाश्रित्य तपस्तप्तं त्वया महत्}
{न च धर्मस्य संज्ञां त्वं पुरा वेत्थ कथंचन}


\twolineshloka
{ततः सिद्धस्य तपसा तव विप्र शकुन्तकाः}
{क्षिप्रं शिरस्यजायन्त ते च संभावितास्त्वया}


\threelineshloka
{जातपक्षा यदा ते च गता संचरितुं ततः}
{मन्यमानस्ततो धर्मं चटकप्रभवं द्विज}
{खे वाचां त्वमथाश्रौषीर्मां प्रति द्विजसत्तम}


\twolineshloka
{अमर्षवशमापन्नस्ततः प्राप्तो भवानिह}
{करवाणि प्रियं किं ते तद्ब्रूहि द्विजसत्तम}


\chapter{अध्यायः २६८}
\twolineshloka
{भीष्म उवाच}
{}


\threelineshloka
{इत्युक्तः स तदा तेन तुलाधारेण धीमता}
{प्रोवाच वचनं धीमाञ्जाजलिर्जपतांवरः ॥जाजलिरुवाच}
{}


\twolineshloka
{विक्रीणतः सर्वरसान्सर्वगन्धांश्च वाणिज}
{वनस्पतीनोषधीश्च तेषां मूलफलानि च}


\threelineshloka
{अग्र्या सा नैष्ठिकी बुद्धिः कुतस्त्वामियमागता}
{एतदाचक्ष्व मे सर्वं निखिलेन महामते ॥भीष्म उवाच}
{}


\twolineshloka
{एवमुक्तस्तुलाधारो ब्राह्मणेन यशस्विना}
{उवाच धर्मसूक्ष्माणि वैश्यो धर्मार्थतत्त्ववित्}


\twolineshloka
{वेदाहं जाजले धर्मं सरहस्यं सनातनम्}
{सर्वभूतहितं मैत्रं पुराणं यं जना विदुः}


\twolineshloka
{अद्रोहेणैव भूतानामल्पद्रोहेण वा पुनः}
{या वृत्तिः स परो धर्मस्तेन जीवामि जाजले}


\twolineshloka
{परिच्छिन्नैः काष्ठतृणैर्मयेदं शरणं कृतम्}
{अलक्तं पद्मकं तुङ्गं गन्धांश्चोच्चावचांस्तथा}


\twolineshloka
{रसांश्च तांस्तान्विप्रर्षे मद्यवर्ज्यान्बहूनहम्}
{क्रीत्वा वै प्रतिविक्रीणे परहस्तादमायया}


\twolineshloka
{सर्वेषां यः सुहृन्नित्यं सर्वेषां च हिते रतः}
{कर्मणा मनसा वाचा स धर्मं वेद जाजले}


\threelineshloka
{नानुरुध्ये विरुध्ये वा न द्वेष्मि न च कामये}
{समोऽहं सर्वभूतेषु पश्य मे जाजले व्रतम्}
{तुला मे सर्वभूतेषु समा तिष्ठति जाजले}


\twolineshloka
{नाहं परेषां कृत्यानि प्रशंसामि न गर्हये}
{आकाशस्येव विप्रेन्द्र पश्यँल्लोकस्य चित्रताम्}


\twolineshloka
{`कृपा मे सर्वभूतेषु समा तिष्ठति जाजले}
{इष्टानिष्टनियुक्तस्य प्रियद्वेषौ बहिष्कृतौ ॥'}


\twolineshloka
{इति मां त्वं विजानीहि सर्वलोकस्य जाजले}
{समं मतिमतां श्रेष्ठ समलोष्टाश्मकाञ्चनम्}


\twolineshloka
{यथाऽन्धबधिरोत्मत्ता उच्छ्वासपरमाः सदा}
{देवैरपिहितद्वाराः सोपमा पश्यतो मम}


\twolineshloka
{यथा वृद्धातुरकृशा निस्पृहा विषयान्प्रति}
{तथाऽर्थकामभोगेषु ममापि विगता स्पृहा}


\twolineshloka
{यदा चायं न बिभेति यदा चास्मान्न बिभ्यति}
{यदा नेच्छति न द्वेष्टि तदा सिद्ध्यति वै द्विज}


\twolineshloka
{यदा न कुरुते भावं सर्वभूतेषु पापकम्}
{कर्मणा मनसा वाचा ब्रह्म संपद्यते तदा}


\twolineshloka
{न भूतो न भविष्योऽस्ति न च धर्मोस्ति कश्चन}
{योऽभयः सर्वभूतानां स प्राप्नोत्यभयं पदम्}


\twolineshloka
{यस्मादुद्विजते लोकः सर्वो मृत्युमुखादिव}
{वाक््क्रूराद्दण्डपरुषात्स प्राप्नोति महद्भयम्}


\twolineshloka
{यथावद्वर्तमानानां वृद्धानां पुत्रपौत्रिणाम्}
{अनुवर्तामहे वृत्तमहिंस्त्राणां महात्मनाम्}


\twolineshloka
{प्रनष्टः शाश्वतो धर्मः सदाचारेण मोहितः}
{तेन वैद्यस्तपस्वी वा बलवान्वा विमुह्यते}


\twolineshloka
{आचाराज्जाजले प्राज्ञः क्षिप्रं धर्ममवाप्नुयात्}
{एवं यः साधुभिर्दान्तश्चरेदद्रोहचेतसा}


\twolineshloka
{नद्यां चेह यथा काष्ठमुह्यमानं यदृच्छया}
{यदृच्छयैव काष्ठेन सन्धि गच्छेत केनचित्}


\twolineshloka
{तत्रापराणि दारूणि संसृज्यन्ते ततस्ततः}
{तृणकाष्ठकरीपाणि कदाचिन्न समीक्षया}


\twolineshloka
{यस्मान्नोद्विजते भूतं जातु किंचित्कथंचन}
{अभयं सर्वभूतेभ्यः स प्राप्नोति सदा मुने}


\twolineshloka
{यस्मादुद्विजते विद्वन्सर्वलोको वृकादिव}
{क्रोशतस्तीरमासाद्य यथा सर्वे जलेचराः}


\twolineshloka
{एवमेवायमाचारः प्रादुर्भूतो यतस्ततः}
{सहायवान्द्रव्यवान्यः सुभगोऽथ परस्तथा}


\twolineshloka
{ततस्तानेव कवयः शास्त्रेषु प्रवदन्त्युत}
{कीर्त्यर्थमल्पहृल्लेखाः पटवः कृत्स्ननिर्णयाः}


\twolineshloka
{तपोभिर्यज्ञदानैश्च वाक्यैः प्रज्ञाश्रितैस्तथा}
{प्राप्नोत्यभयदानस्य यद्यत्फलमिहाश्नुते}


\twolineshloka
{लोके यः सर्वभूतेभ्यो ददात्यभयदक्षिणाम्}
{स सत्ययज्ञैरीजानः प्राप्नोत्यभयदक्षिणाम्}


\threelineshloka
{न भूतानामहिंसाया ज्यायान्धर्मोऽस्ति कश्चन}
{यस्मान्नोद्विजते भूतं जातु किंचित्कथंचन}
{सोऽभयं सर्वभूतेभ्यः संप्राप्नोति महामुने}


\twolineshloka
{यस्मादुद्विजते लोकः सर्पाद्वेश्मगतादिव}
{न स धर्ममवाप्नोति इह लोके परत्र च}


\twolineshloka
{सर्वभूतात्मभूतस्य सर्वभूतानि पश्यतः}
{देवाऽपि मार्गे मुह्यन्ति ह्यपदस्य पदैपिणः}


\twolineshloka
{दानं भूताभयस्याहुः सर्वदानेभ्य उत्तमम्}
{ब्रवीमि ते सत्यमिदं श्रद्धत्स्व मम जाजले}


\twolineshloka
{स एव सुभगो भूत्वा पुनर्भवति दुर्भगः}
{व्यापत्तिं कर्मणां दृष्ट्वा जुगुप्सन्ति जनाः सदा}


\twolineshloka
{अकारणो हि नैवास्ति धर्मः सूक्ष्मो हि जाजले}
{भूतभव्यार्थमेवेह धर्मप्रवचनं कृतम्}


\twolineshloka
{सूक्ष्मत्वान्न स विज्ञातुं शक्यते बहुनिह्नवः}
{उपलभ्यान्तरा चान्यानाचारानवबुध्यते}


\threelineshloka
{ये च च्छिन्दन्ति वृषणान्ये च भिन्दन्ति नस्तकान्}
{वहन्ति महतो भारान्बध्नन्ति दमयन्ति च}
{हत्वा सत्वानि खादन्ति तान्कथं न विगर्हसे}


\twolineshloka
{मानुषा मानुपानेव दासभोगेन भुञ्जते}
{वधबन्धनिरोधेन कारयन्ति दिवानिशम्}


\twolineshloka
{आत्मनश्चापि जानाति यद्दुःखं वधबन्धने}
{पञ्चेन्द्रियेषु भूतेषु सर्वं वसति दैवतम्}


\twolineshloka
{आदित्यश्चन्द्रमा वायुर्ब्रह्मा प्राणः क्रतुर्यमः}
{तानि जीवानि विक्रीय का मृतेषु विचारणा}


\twolineshloka
{अजोऽग्निर्वरुणो मेपः सूर्योऽश्वः पृथिवी विराट्}
{धेनुर्वत्सश्च सोमो वै विक्रीयैतन्न सिध्यति}


% Check verse!
का तैले का धृते ब्रह्मन्मधुन्यप्स्वोषधीषु वा
\twolineshloka
{अदंशमशके देशे सुखसंवर्धितान्पशून्}
{तांश्च मातुः प्रियाञ्जानन्नाक्रम्य बहुधा नराः}


\twolineshloka
{बहुदंशाकुलान्देशान्नयन्ति बहुकर्दमान्}
{वाहसंपीडिता धुर्याः सीदन्त्यविधिना परे}


\twolineshloka
{न मन्ये भ्रूणहत्याऽपि विशिष्टा तेन कर्मणा}
{कृपिं साध्विति मन्यन्ते सा च वृत्तिः सुदारुणा}


\twolineshloka
{भूमिं भूमिशयांश्चैव हन्ति काष्ठैरयोमुखैः}
{तथैवानडुहो युक्तान्क्षुत्तृष्णाश्रमकर्शितान्}


\twolineshloka
{अध्न्या इति गवां नाम क एता हन्तुमर्हति}
{महच्चकाराकुशलं वृथा यो गां निहन्ति ह}


\threelineshloka
{ऋपयो यतयो ह्येतन्नहुपे प्रत्यवेदयन्}
{गां मातरं चाप्यवधीर्वृपभं च प्रजापतिम्}
{अकार्यं नहुपाकापींर्लप्स्यामस्त्वत्कृते व्यथाम्}


\twolineshloka
{शतं चैकं च रोगाणां सर्वभूतेष्वपातयन्}
{ऋपयस्ते महाभागाः प्रशस्तास्ते च जाजले}


\threelineshloka
{भ्रृणहं नहुषं त्वाहुर्न तं भोक्ष्यामहे वयम्}
{इत्युक्त्वा ते महात्मानः सर्वे तत्त्वार्थदर्शिनः}
{ऋषयो यतयः शान्तास्तपसा प्रत्येषधयन्}


\twolineshloka
{ईदृशानशिवान्घोरानाचारानिह जाजले}
{केवलाचरितत्वात्तु निपुणो नावबुध्यसे}


% Check verse!
कारणाद्धर्ममन्विच्छन्न लोकं विरसं चरेत्
\threelineshloka
{यो हन्याद्यश्च मां स्तौति तत्रापि शृणु जाजले}
{समौ तावपि मे स्थातां न हि मे स्तः प्रियाप्रिये}
{एतदीदृशकं धर्मं प्रशंसन्ति मनीषिणः}


\twolineshloka
{उपपत्त्या हि संपन्नो यतिभिश्चैव सेव्यते}
{सततं धर्मशीलैश्च निपुणेनोपलक्षितः}


\chapter{अध्यायः २६९}
\twolineshloka
{जाजलिरुवाच}
{}


\twolineshloka
{यथा प्रवर्तितो धर्मस्तुलां धारयता त्वया}
{स्वर्गद्वारं च वत्तिं च भूतानामवरोत्स्यते}


\twolineshloka
{कृष्या ह्यन्नं प्रभवति ततस्त्वमपि जिवसि}
{पशुभिश्चौषधीभिश्च मर्त्या जीवन्ति वाणिज}


\threelineshloka
{ततो यज्ञः प्रभवति नास्तिक्यमपि जल्पसि}
{न हि वर्तेदयं लोको वार्तामुत्सृज्य केवलाम् ॥तुलाधार उवाच}
{}


\twolineshloka
{वक्ष्यामि जाजले वृत्तिं नास्मि ब्राह्मण नास्तिकः}
{न यज्ञं च विनिन्दामि यज्ञवित्तु सुदुर्लभः}


\twolineshloka
{नमो ब्राह्मण यज्ञाय ये च यज्ञविदो जनाः}
{स्वयज्ञं ब्राह्मणा हित्वा क्षत्रयज्ञमनुष्ठिताः}


\twolineshloka
{लुब्धैर्वित्तपरैर्ब्रह्मन्नास्तिकैः संप्रवर्तितम्}
{वेदवादानभिज्ञानां सत्याभासमिवानृतम्}


\twolineshloka
{इदं देयमिदं देयमिति नान्यच्चिकीर्षति}
{अतः स्तैन्यं प्रभवति विकर्माणि च जाजले}


\threelineshloka
{यदेव सुकृतं हव्यं तेन तुष्यन्ति देवताः}
{नमस्कारेण हविषा स्वाध्यायैरौषधैस्तथा}
{पूजा स्याद्देवतानां हि यथाशास्त्रनिदर्शनम्}


\twolineshloka
{इष्टापूर्तादसाधूनां विदुषां जायते प्रजा}
{लुब्धेभ्यो जायते लुब्धः समेभ्यो जायते समः}


\twolineshloka
{यजमाना यथाऽऽत्मानमृत्विजश्च तथा प्रजाः}
{यज्ञात्प्रजा प्रभवति नभसोऽम्भ इवामलम्}


\twolineshloka
{अग्नौ प्रास्ताहुतिर्ब्रह्मन्नादित्यमुपगच्छति}
{आदित्याज्जायते वृष्टिर्वृष्टेरन्नं ततः प्रजाः}


\twolineshloka
{तस्मात्सुनिष्ठिताः पूर्वे सर्वान्कामांश्च लेभिरे}
{अकृष्टपच्या पृथिवी आशीर्भिर्वीररुधोऽभवन्}


\twolineshloka
{न ते यज्ञेष्वात्मसु वा फलं पश्यन्ति किंचन}
{शङ्कमानाः फलं यज्ञे ये यजेरन्कथंचन}


\twolineshloka
{जानन्तः सर्वथा साधु लुब्धा वित्तप्रयोजनाः}
{स स्म पापकृतां लोकान्गुच्छेदशुभकर्मणा}


\twolineshloka
{प्रमाणमप्रमाणेन यः कुर्यादशुभं नरः}
{पापात्मा सोऽकृतप्रज्ञः सदैवेह द्विजोत्तम}


\twolineshloka
{कर्तव्यमिति कर्तव्यं वेत्ति वै ब्राह्मणो भयम्}
{ब्रह्मैव वर्तते लोके नैव कर्तव्यतां पुनः}


\twolineshloka
{विगुणं च पुनः कर्म ज्याय इत्यनुशुश्रुम्}
{सर्वभूतोपकारश्च फलभावे च संयमः}


\twolineshloka
{सत्ययज्ञा दमयज्ञा अलुब्धाश्चात्मवृत्तयः}
{उत्पन्नत्यागिनः सर्वे जना आसन्नमत्सराः}


\twolineshloka
{क्षेत्रक्षेत्रज्ञतत्त्वज्ञाः स्वयज्ञपरिनिष्ठिताः}
{ब्राह्मं वेदमधीयन्तस्तोषयन्त्यपरानपि}


\twolineshloka
{अखिलं दैवतं सर्वं ब्रह्म ब्रह्मणि संश्रितम्}
{तृप्यन्ति तृप्यतो देवास्तृप्ताऽतृप्तस्य जाजले}


\twolineshloka
{यथा सर्वरसैस्तृप्तो नाभिनन्दति किंचन}
{तथा प्रज्ञानतृप्तस्य नित्यतृप्तिः सुखोदया}


\twolineshloka
{धर्माधारा धर्मसुखाः कृत्स्नव्यवसितास्तथा}
{अस्ति नस्तत्त्वतो भूय इति प्राज्ञस्त्ववेक्षते}


\twolineshloka
{ज्ञानविज्ञानिनः केचित्परं पारं तितीर्षवः}
{अतीव पुण्यदं पुण्यं पुण्याभिजनसंहितम्}


\twolineshloka
{यत्र गत्वा न शोचन्ति च च्यवन्ति व्यथन्ति च}
{ते तु तद्ब्रह्मणः स्थानं प्राप्नुवन्तीह सात्विकाः}


\twolineshloka
{नैव ते स्वर्गमिच्छन्ति न यजन्ति यशोधनाः}
{सतां वर्त्मानुवर्तन्ते यथाबलमहिंसया}


\twolineshloka
{वनस्पतीनोषधीश्च फलमूलानि ते विदुः}
{न चैतानृत्विजो लुब्धा याजयन्ति फलार्थिनः}


\twolineshloka
{स्वमेव चार्थं कुर्वाणा यज्ञं चक्रुः पुनर्द्विजाः}
{परिनिष्ठितकर्माणाः प्रजानुग्रहकाम्यया}


\threelineshloka
{तस्मात्तानृत्विजो लुब्धा याजयन्त्यशुभान्नरान्}
{प्रापयेयुः प्रजाः स्वर्गे स्वधर्माचरणेन वै}
{इति मे वर्तते बुद्धिः समा सर्वत्र जाजले}


\twolineshloka
{प्रयुञ्जते येन यज्ञे सदा प्राज्ञा द्विजर्षभाः}
{तेन ते देवयानेन पथा यान्ति महामुने}


\twolineshloka
{आवृत्तिस्तस्य चैकस्य नास्त्यावृत्तिर्मनीषिणाः}
{उभौ तौ देवयानेन गच्छतो जाजले यथा}


\twolineshloka
{स्वयं चैषामनडुहो युञ्जन्ति च वहन्ति च}
{स्वयमुस्राश्च दुह्यन्ते मनःसंकल्पसिद्धिभिः}


\twolineshloka
{स्वयं यूपानुपादाय यजन्ते स्वाप्तदक्षिणाः}
{यस्तथा भावितात्मा स्यात्स गामालब्धुमर्हति}


\twolineshloka
{ओषधीभिस्तथा ब्रह्मन्यजेरंस्ते न तादृशाः}
{श्रद्धया गां पुरस्कृत्य तदृतं प्रब्रवीमि ते}


\twolineshloka
{निराशिषमनारम्भं निर्नमस्कारमस्तुतिम्}
{अक्षीणं क्षीणकर्माणं तं देवा ब्राह्मणं विदुः}


\fourlineindentedshloka
{न श्रावयन्न च यजन्न ददद्ब्राह्मणेषु च}
{काम्यां वृत्तिं लिप्समानः कां गतिं याति जाजले}
{इदं तु दैवतं कृत्वा यथा यज्ञमवाप्नुयात् ॥जाजलिरुवाच}
{}


\twolineshloka
{[न वै मुनीनां शृणुमः स्म तत्त्वंपृच्छामि ते वाणिज कष्टमेतत्}
{पूर्वेपूर्वे चास्य नावेक्षमाणानातः परं तमृषयः स्थापयन्ति}


\fourlineindentedshloka
{यस्मिन्नेवात्मतीर्थेन पशवः प्राप्नुयुर्मखम्}
{]अथ स्म कर्मणा केन वाणिज प्राप्नुयात्सुखम्}
{शंस मे तन्महाप्राज्ञ भृशं वै श्रद्दधामि ते ॥तुलाधार उवाच}
{}


\threelineshloka
{उत यज्ञा उतायज्ञा मस्वं नार्हन्ति ते क्वचित्}
{आज्येन पयसा दध्ना सत्कृत्यामिक्षया त्वचा}
{बालैः शृङ्गेण पादेन संभरत्येव गौर्मखम्}


\twolineshloka
{पत्नीव्रतेन विधिना प्रकरोति नियोजयन्}
{इष्टं तु दैवतं कृत्वा यथा यज्ञमवाप्नुयात्}


\threelineshloka
{पुरोडाशो हि सर्वेषां पशूनां मेध्य उच्यते}
{सर्वा नद्यः सरस्वत्यः सर्वे पुण्याः शिलोच्चयाः}
{जाजले तीर्थमात्मेव मा स्म देशातिथिर्भव}


\threelineshloka
{एतानीदृशकान्धर्मानाचरन्निह जाजले}
{कारणैर्धर्ममन्विच्छन्स लोकानाप्नुते शुभान् ॥भीष्म उवाच}
{}


\twolineshloka
{एतानीदृशकान्धर्मांस्तुलाधारः प्रशंसति}
{उपपत्त्याऽभिसंपन्नान्नित्यं सद्भिर्निषेवितान्}


\chapter{अध्यायः २७०}
\twolineshloka
{तुलाधार उवाच}
{}


\twolineshloka
{सद्भिर्वा यदि वाऽसद्भिः पन्थानमिममाश्रितः}
{प्रत्यक्षं क्रियतां साधु ततो ज्ञास्यसि तद्यथा}


\twolineshloka
{एते शकुन्ता बहवः समन्ताद्विचरन्ति ह}
{तवोत्तमाङ्गे संभूताः श्येनाश्चान्याश्च जातयः}


\twolineshloka
{आहूयैनान्महाब्रह्मन्विशमानांस्ततस्ततः}
{पश्येमान्हस्तपादैश्च श्लिष्टान्देहेषु सर्वशः}


\threelineshloka
{संभावयन्ति पितरं त्वया संभाविताः स्वगाः}
{असंशयं पिता वै त्वं पुत्रानाह्वय जाजले ॥भीष्म उवाच}
{}


\threelineshloka
{ततो जाजलिना तेन समाहूताः पतत्रिणः}
{वाचमुच्चारयन्ति स्म धर्मस्य वचनात्किल ॥तुलाधार उवाच}
{}


\twolineshloka
{अहिंसादि कृतं कर्म इह चैव परत्र च}
{श्रद्धां निहन्ति वै ब्रह्मन्सा हता हन्ति तं नरम्}


\twolineshloka
{समानां श्रद्दधानानां संयतानां सुचेतसाम्}
{कुर्वतां यज्ञ इत्येव न यज्ञो जातु नेष्यते}


\twolineshloka
{श्रद्धा वै सात्विकी देवी सूर्यस्य दुहिता द्विज}
{सावित्री प्रसवित्री च हविर्वाङ्भनसी ततः}


\twolineshloka
{वाग्वृद्धं त्रायते श्रद्धा मनोवृद्धं च जाजले}
{श्रद्धावृद्धं वाङ्भनसी न यज्ञस्त्रातुमर्हति}


\twolineshloka
{अत्र गाथा ब्रह्मगीताः कीर्तयन्ति पुराविदः}
{शुचेरश्रद्दधानस्य श्रद्दधानस्यर चाशुचेः}


\twolineshloka
{देवा वित्तममन्यन्त सदृशं यज्ञकर्मणि}
{श्रोत्रियस्य कदर्यस्य वदान्यस्य च वार्धुषेः}


\twolineshloka
{मीमांसित्वोभयं देवाः सममन्नमकल्पयन्}
{प्रजापतिस्तानुवाच विषमं कृतमित्युत}


\twolineshloka
{श्रद्धापूतं वदान्यस्य हतमश्रद्धयेतरत्}
{भोज्यमन्नं वदान्यस्य कदर्यस्य न वार्धुषेः}


\twolineshloka
{अश्रद्दधान एवैको देवानां नार्हते हविः}
{तस्यैवान्नं न भोक्तव्यमिति धर्मविदो विदुः}


\twolineshloka
{अश्रद्धा परमं पापं श्रद्धा पापप्रमोचनी}
{जहाति पापं श्रद्धावान्सर्पो जीर्णामिव त्वचम्}


\twolineshloka
{ज्यायसी या पवित्राणां निवृत्तिः श्रद्धया सह}
{निवृत्तशीलदोषो यः श्रद्धावान्पूत एव सः}


\twolineshloka
{किं तस्य तपसा कार्यं किं वृत्तेन किमात्मना}
{श्रद्धामयोऽयं पुरुषो यो यच्छ्रद्धः स एव सः}


\twolineshloka
{इति धर्मः समाख्यातः सद्भिर्धर्मार्थदर्शिभिः}
{वयं जिज्ञासमानास्तु संप्राप्ता धर्मदर्शनात्}


\twolineshloka
{श्रद्धां कुरु महाप्राज्ञ ततः प्राप्स्यसि यत्परम् ॥`जाजलिरुवाच}
{}


\twolineshloka
{न वै मुनीनां शृणुमश्च तत्वंपृच्छामि ते वाणिज तत्वमेतत्}
{पूर्वे हि पूर्वेऽप्यनवेक्षमाणानातः परं ते ऋषयः स्थापयन्ति}


\fourlineindentedshloka
{यस्मिन्नेवानुतीर्थेन पशवः प्राप्नुयुः सुखम्}
{पत्नीव्रतेन विधिना प्रकरोति नियोजयन्}
{'श्रद्धावाञ्श्रद्दधानश्च धर्मश्चैव हि वाणिज ॥तुलाधार उवाच}
{}


\twolineshloka
{स्ववर्त्मनि स्थितश्चैव गरीयानेव जाजले ॥भीष्म उवाच}
{}


\threelineshloka
{ततोऽचिरेण कालेन तुलाधारः स एव च}
{दिवं गत्वा महाप्राज्ञौ विहरेतां यथासुखम्}
{स्वंस्वं स्थानमुपागम्य स्वकर्मफलनिर्मितम्}


\twolineshloka
{एवं बहुविधार्थं च तुलाधारेण भापितम्}
{सम्यक्चैवमुपालब्धो धर्मश्चोक्तः सनातनः}


\twolineshloka
{तस्य विख्यातवीर्यस्य श्रुत्वा वाक्यानि जाजलिः}
{तुलाधारस्य कौन्तेय शान्तिमेवान्वपद्यत}


\twolineshloka
{`समानां श्रद्दधानानां युक्तानां च यथाबलम्}
{कुर्वतां यज्ञ इत्येव नायज्ञो जातु नेष्यते ॥'}


\twolineshloka
{एवं बहुमतार्थं च तुलाधारेण भाषितम्}
{यथौपम्योपदेशेन किं भूयः श्रोतुमिच्छसि}


\chapter{अध्यायः २७१}
\twolineshloka
{`युधिष्ठिर उवाच}
{}


\threelineshloka
{शरीरमापदश्चैव न विदन्त्यविहिंसकाः}
{कथं यात्रा शरीरस्य निरारम्भस्य सेत्स्यते ॥भीष्म उवाच}
{}


\twolineshloka
{यथा शरीरं न म्लायेन्नैव मृत्युवशं भवेत्}
{तथा कर्मसु वर्तेत समर्थो धर्ममाचरेत् ॥'}


\twolineshloka
{अत्राप्युदाहरन्तीममितिहासं पुरातनम्}
{प्रजानामनुकम्पार्थं गीतं राज्ञा विचख्युना}


\twolineshloka
{छिन्नस्थूणं वृपं दृष्ट्वा विरावं च गवां भृशम्}
{गोगृहे यज्ञवाटे च प्रेक्षमाणः स पार्थिवः}


\twolineshloka
{स्वस्ति गोभ्योऽस्तु लोकेषु ततो निर्वचनं कृतम्}
{हिंसायां हि प्रवृत्तायामाशीरेषा तु कल्पिता}


\twolineshloka
{अव्यवस्थितमर्यादैर्विमूढैर्नास्तिकैर्नरैः}
{संशयात्मभिरव्यक्तैर्हिंसा समनुदर्शिता}


\twolineshloka
{सर्वकर्मग्वहिंसां हि धर्मात्मा मनुरब्रवीत्}
{कामकाराद्विहिंसन्ति बहिर्वेद्यां पशून्नराः}


\twolineshloka
{तस्मात्प्रमाणतः कार्यो धर्मः सूक्ष्मो विजानता}
{अहिंसा ह्येव सर्वेभ्यो धर्मेभ्यो ज्यायसी मता}


\twolineshloka
{उपोष्य संशितो भूत्वा हित्वा वेदकृतां शुचिः}
{आचार इत्यनाचारः कृपणाः फलहेतवः}


\twolineshloka
{यदि च्छिन्दन्ति वृक्षांश्च यूपांश्चोद्दिश्य मानवाः}
{वृथा मांसानि खादन्ति नैप धर्मः प्रशस्यते}


\threelineshloka
{सुरां मत्स्यान्मधु मांसमासवं कृसरौदनम्}
{धूर्तैः प्रवर्तितं ह्येतन्नैतद्वेदेषु कल्पितम्}
{कामान्मोहाच्च लोभाच्च लौल्यमेतत्प्रवर्तितम्}


\twolineshloka
{विष्णुमेवाभिजानन्ति सर्वयज्ञेषु ब्राह्मणाः}
{पायसैः सुमनोभिश्च तस्यैव यजनं स्मृतम्}


\fourlineindentedshloka
{यज्ञियाश्चैव ये वृक्षा वेदेषु परिकल्पिताः}
{यच्चापि किंचित्कर्तव्यमन्यच्चोक्षैः सुसंस्कृतम्}
{महासत्वैः शुद्धभावैः सर्वं देवार्हमेव तत् ॥युधिष्ठिर उवाच}
{}


\threelineshloka
{शरीरमापदश्चापि विवदन्त्यविहंसतः}
{कथं यात्रा शरीरस्य निरारम्भस्य सेत्स्यते ॥भीष्म उवाच}
{}


\twolineshloka
{यथा शरीरं न ग्लायेन्नेयान्मृत्युवशं यथा}
{तथा कर्मसु वर्तेत समर्थो धर्ममाचरेत्}


\chapter{अध्यायः २७२}
\twolineshloka
{युधिष्ठिर उवाच}
{}


\threelineshloka
{कथं कार्यं परीक्षेत शीघ्रं वाऽथ चिरेण वा}
{सर्वथा कार्यदुर्गेऽस्मिन्भवान्नः परमो गुरुः ॥भीष्म उवाच}
{}


\twolineshloka
{अत्राप्युदाहरन्तीममितिहासं पुरातनम्}
{चिरकारेस्तु यत्पूर्वं वृत्तमाङ्गिरसां कुले}


\threelineshloka
{`गौतमस्य सुता ह्यासन्वीयांश्चिरकारिकः}
{'चिरकारिक भद्रं ते भद्रं ते चिरकारिक}
{चिरकारी हि मेधावी नापराध्यति कर्मसु}


\twolineshloka
{चिरकारी महाप्राज्ञो गौतमस्याभवत्सुतः}
{चिरेण सर्वकार्याणि विमृश्यार्थान्प्रपद्यते}


\twolineshloka
{चिरं स चिन्तयत्यर्थांश्चिरं जाग्रच्चिरं स्वपन्}
{चिरं कार्याभिषत्तिं च चिरकारी तथोच्यते}


\twolineshloka
{अलसग्रहणं प्राप्तो दुर्मेधावीति चोच्यते}
{बुद्धिलाघवयुक्तेन जनेनादीर्घदर्शिना}


\twolineshloka
{व्यभिचारे तु कस्मिंश्चिद्व्यतिक्रम्यापरान्सुतान्}
{पित्रोक्तः कुपितेनाथ जहीमां जननीमिति}


\twolineshloka
{इत्युक्त्वा स तदा विप्रो गौतमो जपतां वरः}
{अविमृश्य महाभागो वनमेव जगाम सः}


\twolineshloka
{स तथेति चिरेणोक्त्वा स्वभावाच्चिरकारिकः}
{विमृश्य चिरकारित्वाच्चिन्तयामास वै चिरम्}


\twolineshloka
{पितुराज्ञां कथं कुर्यां न हन्यां मातरं कथम्}
{कथं धर्मच्छले नास्मिन्निमज्जेयमसाधुवत्}


\twolineshloka
{पितुराज्ञा परो धर्मः स्वधर्मो मातृरक्षणम्}
{अस्वतन्त्रं च पुत्रत्वं किंतु मां नानुपीडयेत्}


\twolineshloka
{स्त्रियं हत्वा मातरं च को हि जातु सुखी भवेत्}
{पितरं चाप्यवज्ञाय कः प्रतिष्ठामवाप्नुयात्}


\twolineshloka
{अनवज्ञा पितुर्युक्ता स्वधर्मो मातृरक्षणम्}
{युक्तक्षमावुभावेतौ नातिवर्तेतमां कथम्}


\twolineshloka
{पिता ह्यात्मानमाधत्ते जायायां जायतामिति}
{शीलचारित्रगोत्रस्य धारणार्थं कुलस्य च}


\twolineshloka
{सोऽहं मात्रा स्वयं पित्रा पुत्रत्वे प्रकृतः पुनः}
{विज्ञानं मे कथं न स्याद्द्वौ बुद्ध्ये चात्मसंभवम्}


\twolineshloka
{जातकर्मणि यत्प्राह पिता यच्चोपकर्मणि}
{पर्याप्तः स दृढीकारः पितुर्गौरवनिश्चये}


\twolineshloka
{गुरुरग्र्यः परो धर्मः पोषणाध्यापनान्वितः}
{पिता यदाह धर्मः स वेदेष्वपि सुनिश्चितः}


\twolineshloka
{प्रीतिमात्रं पितुः पुत्रः सर्वं पुत्रस्य वै पिता}
{शरीरादीनि देयानि पिता त्वेकः प्रयच्छति}


\twolineshloka
{तस्मात्पितुर्वचः कार्यं न विचार्यं कदाचन}
{पातकान्यपि पूयन्ते पितुःर शासनकारिणः}


\twolineshloka
{भाग्यभोगे प्रसवने सर्वलोकनिदर्शने}
{धात्र्याश्चैव समायोगे सीमन्तोन्नयने तथा}


\twolineshloka
{पिता धर्मः पिता स्वर्गः पिता हि परमं तपः}
{पितरि प्रीतिमापन्ने सर्वाः प्रीणन्ति देवताः}


\twolineshloka
{आशिषस्ता भजन्त्यनं पुरुषं प्राह यत्पिता}
{निष्कृतिः सर्वपापानां पिता यच्चाभिनन्दति}


\twolineshloka
{मुच्यते बन्धनात्पुरुषं फलं वृक्षान्प्रमुच्यते}
{क्लिश्यन्नपि सुतस्नेहैः पिता पुत्रं न मुञ्चति}


\twolineshloka
{एतद्विचिन्तितं तावत्पुत्रस्य पितृगौरवम्}
{पिता नाल्पतरं स्थानं चिन्तयिष्यामि मातरम्}


\twolineshloka
{यो ह्ययं मयि संघातो मर्त्यत्वे पाञ्चभौतिकः}
{अस्य मे जननी हेतुः पावकस्य यथाऽरणिः}


\twolineshloka
{माता देहारणिः पुंसां सर्वस्यार्तस्य निर्वृतिः}
{मातृलाभे सनाथत्वमनाथत्वं विपर्यये}


\twolineshloka
{न च शोचति नाप्येनं स्थाविर्यमपकर्षति}
{श्रिया हीनोऽपि यो गेहमम्बेति प्रतिपद्यते}


\twolineshloka
{पुत्रपौत्रोपपन्नोपि जननीं यः समाश्रितः}
{अपि वर्षशतस्यान्ते स द्विहायनवच्चरेत्}


\twolineshloka
{समर्थं वाऽसमर्थं वा कृशं वाप्यकृशं तथा}
{रक्षत्येव सुतं माता नान्यः पोष्टा विधानतः}


\twolineshloka
{तदा स वृद्धो भवति तदा भवति दुःखितः}
{तदा शून्यं जगत्तस्य यदा मात्रा वियुज्यते}


\twolineshloka
{नास्ति मातृसमा च्छाया नास्ति मातृसमा गतिः}
{नास्ति मातृसमं त्राणं नास्ति मातृसमा प्रिया}


\twolineshloka
{कुक्षौ संधारणाद्धात्री जननाज्जननी स्मृता}
{अङ्गानां वर्धनादम्बा वीरसूत्वेन वीरसूः}


\twolineshloka
{शिशोः शुश्रूषणाच्छुश्रूर्माता देहमनन्तरम्}
{चेतनावान्स को हन्याद्यस्य नासुषिरं शिरः}


\twolineshloka
{दंपत्योः प्राणसंश्लेषे योऽभिसन्धिः कृतः किल}
{तं माता च पिता चेति भूतार्थो मातरि स्थितः}


\twolineshloka
{माता जानाति यद्गोत्रं माता जानाति यस्य सः}
{मातुर्भरणमात्रेण प्रीतिः स्नेहः पितुः प्रजाः}


\twolineshloka
{पाणिबन्धं स्वयं कृत्वा सहधर्ममुपेत्य च}
{यदा यास्यन्ति पुरुषाः स्त्रियो नार्हन्ति याप्यतां}


\twolineshloka
{भरणाद्धि स्त्रियो भर्ता पालनाद्धि पतिस्तथा}
{गुणस्यास्य निवृत्तौ तु न भर्ता न पुनः पतिः}


\twolineshloka
{एवं स्त्री नापराघ्नोति नर एवापराध्यति}
{व्युच्चरंश्च महादोषं नर एवापराध्यति}


\twolineshloka
{स्त्रिया हि परमो भर्ता दैवतं परमं स्मृतम्}
{तस्मात्मना तु सदृशमात्मानं परमं ददौ}


\twolineshloka
{नापराधोऽस्ति नारीणां नर एवापराध्यति}
{सर्वकार्यापराध्यत्वान्नापराध्यन्ति चाङ्गनाः}


\twolineshloka
{यश्चनोक्तोऽथ निर्देशः स्त्रिया मैथुनवृद्धये}
{तस्य स्मारयतो व्यक्तमधर्मो नास्ति संशयः}


\twolineshloka
{एवं नारीं मातरं च गौरवे चाधिके स्थिताम्}
{अवध्यां तु विजानीयुः पशवोऽप्यविचक्षणाः}


\twolineshloka
{देवतानां समावायमेकस्थं पितरं विदुः}
{मर्त्यानां देवतानां च स्नेहादभ्येति मातरम्}


\twolineshloka
{एवं विमृशतस्तस्य विरकारितया बहु}
{दीर्घः कालो व्यतिक्रान्तस्ततोस्याभ्यागमत्पिता}


\twolineshloka
{मेधातिथिर्महाप्राज्ञो गौतमस्तपसि स्थितः}
{विमृश्य तेन कालेन पत्न्याः संस्थाव्यतिक्रमम्}


\twolineshloka
{सोऽब्रवीद्भृशसंतप्तो दुःखेनाश्रूणि वर्तयन्}
{श्रुतधैर्यप्रसादेन पश्चात्तापमुपागतः}


\twolineshloka
{आश्रमं मम संप्राप्तस्त्रिलोकेशः पुरंदरः}
{अतिथिव्रतमास्थाय ब्राह्मण्यं रूपमास्थितः}


\twolineshloka
{स मया सान्त्वितो वाग्भिः स्वागतेनाभिषूजितः}
{अर्ध्यं पाद्यं यथान्यायं मया च प्रतिपादितः}


\twolineshloka
{परवानस्मि चेत्युक्तः प्रणयिष्यति तेन च}
{अत्र चाकुशले जाते स्त्रिया नास्ति व्यतिक्रमः}


\twolineshloka
{एवं न स्त्री न चैवाहं नाध्वगस्त्रिदशेश्वरः}
{अपराध्यति धर्मस्य प्रमादस्त्वपराध्यति}


\twolineshloka
{ईर्ष्याजं व्यसनं प्राहुस्तेन चैवोर्ध्वरेतसः}
{ईर्ष्यया त्वहमाक्षिप्तो मग्नो दुष्कृतसागरे}


\twolineshloka
{हत्वा साध्वीं च नारीं च व्यसनित्वाच्च वासिताम्}
{भर्तव्यत्वेन भार्यां च को नु मां तारयिष्यति}


\twolineshloka
{अन्तरेण मयाऽऽज्ञप्तश्चिरकारीत्युदारधीः}
{यद्यद्य चिरकारी स्यात्स मां त्रायेत पातकात्}


\twolineshloka
{चिरकारिक भद्रं ते भद्रं ते चिरकारिक}
{यद्यद्य चिरकारी त्वं ततोऽसि चिरकारिकः}


\twolineshloka
{त्राहि मां मातरं चैव तपो यच्चार्जितं त्वया}
{आत्मानं पातकेभ्यश्च भवाद्य चिरकारिकः}


\twolineshloka
{सहजं चिरकारित्वमतिप्रज्ञतया तव}
{सफलं तत्तथा तेऽस्तु भवाद्य चिरकारिकः}


\twolineshloka
{चिरमांशसितो मात्रा चिरं गर्भेण धारितः}
{सफलं चिकारित्वं कुरु त्वं चिरकारिक}


\threelineshloka
{चिरायते च संतापाच्चिरं स्वपिति धारितः}
{आवयोश्चिरसंतापादवेक्ष्य चिरकारिक ॥भीष्म उवाच}
{}


\twolineshloka
{एवं स दुःखितो राजन्महर्षिर्गौतमस्तदा}
{चिरकारिं ददर्शाथ पुत्रं स्थितमथान्तिके}


\twolineshloka
{चिरकारी तु पितरं दृष्ट्वा परमदुःखितः}
{शस्त्रं त्यक्त्वा ततो मूर्ध्ना प्रसादायोपचक्रमे}


\twolineshloka
{गौतमस्तं ततो दृष्ट्वा शिरसा पतितं भुवि}
{पत्नीं चैव निराकारां परामभ्यागमन्मुदम्}


\twolineshloka
{न हि सा तेन संभेदं पत्नी नीता महात्मना}
{विजने चाश्रमस्थेन पुत्रश्चापि समाहितः}


\twolineshloka
{हन्या इति समादेशः शस्त्रपाणौ सुते स्थिते}
{विनीते प्रसवत्यर्थे विवासे चात्मकर्मसु}


\twolineshloka
{बुद्धिश्चासीत्सुतं दृष्ट्वा पितुश्चरणयोर्नतम्}
{शस्त्रग्रहणचापल्यं संवृणोति भयादिति}


\twolineshloka
{ततः पित्रा चिरं स्तुत्वा चिरं चाघ्राय मूर्धनि}
{चिरं दोर्भ्यां परिष्वज्य चिरं जीवेत्युदाहृतः}


\twolineshloka
{एवं स गौतमः पुत्रं प्रीतिहर्षगुणैर्युतः}
{अभिनन्द्य महाप्रज्ञ इदं वचनमब्रवीत्}


\twolineshloka
{चिरकारिक भद्रं ते चिरकारी चिरं भव}
{चिराय यदि ते सौम्य चिरमस्मि न दुःखितः}


\twolineshloka
{गाथाश्चाप्यब्रवीद्विद्वान्गौतमो मुनिसत्तमः}
{चिरकारिषु धीरेषु गुणोद्देशसमाश्रयाः}


\twolineshloka
{चिरेण मित्रं बध्नीयाच्चिरेण च कृतं त्यजेत्}
{चिरेण हि कृतं मित्रं चिरं धारणमर्हति}


\twolineshloka
{रागे दर्पे च माने च द्रोहे पापे च कर्मणि}
{अप्रिये चैव कर्तव्ये चिरकारी प्रशस्यते}


\twolineshloka
{बन्धूनां सुहृदां चैव भृत्यानां स्त्रीजनस्य च}
{अव्यक्तेष्वपराधेषु चिरकारी प्रशस्यते}


\twolineshloka
{एवं स गौतमस्तत्र प्रीतः पुत्रस्य भारत}
{कर्मणा तेन कौरव्य चिरकारितया तथा}


\twolineshloka
{एवं सर्वेषु कार्येषु विमृश्य पुरुषस्ततः}
{चिरेण निश्चयं कृत्वा चिरं न परितप्यते}


\twolineshloka
{चिरं धारयते रोषं चिरं कर्म नियच्छति}
{पश्चात्तापकरं कर्म न किंचिदुपपद्यते}


\twolineshloka
{चिरं वृद्धानुपासीत चिरमन्यांश्च पूजयेत्}
{चिरं धर्मं निषेवेत कुर्याच्चान्वेषणं चिरम्}


\twolineshloka
{चिरमन्वास्य विदुषश्चिरं शिष्टान्निषेव्य च}
{चिरं विनीय चात्मानं चिरं चात्यनवज्ञताम्}


\twolineshloka
{ब्रुवतश्च परस्यापि वाक्यं धर्मोपसंहितम्}
{चिरं पृष्टोऽपि च ब्रूयाच्चिरं न परितप्यते}


\twolineshloka
{उपास्य बहुलास्तस्मिन्नाश्रमो सुमहातपाः}
{समाः स्वर्गं गतो विप्रः पुत्रेण सहितस्तदा}


\chapter{अध्यायः २७३}
\twolineshloka
{युधिष्ठिर उवाच}
{}


\threelineshloka
{कथं राजा प्रजा रक्षेन्न च किंचित्प्रतापयेत्}
{पृच्छामि त्वां सतां श्रेष्ठ तन्मे ब्रूहि पितामह ॥भीष्म उवाच}
{}


\twolineshloka
{अत्राप्युदाहरन्तीममितिहासं पुरातनम्}
{द्युमत्सेनस्य संवादं राज्ञा सत्यवता सह}


\twolineshloka
{अव्याहृतं व्याजहार सत्यवानिति नः श्रुतम्}
{वधाय नीयमानेषु पितुरेवानुशासनात्}


\threelineshloka
{अधर्मतां याति धर्मो यात्यधर्मश्च धर्मताम्}
{वधो नाम भवेद्धर्मो नैतद्भवितुमर्हति ॥द्युमत्सेन उवाच}
{}


\twolineshloka
{अथ चेदवधो धर्मोऽधर्मः को जातुचिद्भवेत्}
{दस्यवश्चेन्न हन्येरन्सत्यवन्संकरो भवेत्}


\threelineshloka
{ममेदमिति नास्यैतत्प्रवर्तेत कलौ युगे}
{लोकयात्रा न चैव स्यादश्च चेद्वेत्थ शंस नः ॥सत्यवानुवाच}
{}


\twolineshloka
{सर्व एव त्रयो वर्णाः कार्या ब्राह्मणबन्धनाः}
{धर्मपाशनिबद्धानां नाल्पोऽप्यपचरिष्यति}


\twolineshloka
{यो यस्तेषामपचरेत्तमाचक्षीत वै द्विजः}
{अयं मे न शृणोतीति तस्मिन्राजा प्रधारयेत्}


\threelineshloka
{तत्त्वाभावेन यच्छास्त्रं तत्कुर्यान्नान्यथा वधः}
{असमीक्ष्यैव कर्माणि नीतिशास्त्रं यथाविधि}
{दस्यून्निहन्ति वै राजा भूयसो वाऽप्यनागसः}


\twolineshloka
{भार्या माता पिता पुत्रो हन्यन्ते पुरुषेण ते}
{परेणापकृतो राजा तस्मात्सम्यक्प्रधारयेत्}


\twolineshloka
{असाधुश्चैव पुरुषो लभते शीतमेकदा}
{साधोश्चापि ह्यसाधुभ्यः शोभना जायते प्रजाः}


\twolineshloka
{न मूलघातः कर्तव्यो नैष धर्मः सनातनः}
{अपि खल्ववधेनैव प्रायश्चित्तं विधीयते}


\twolineshloka
{उद्वे तेन बन्धेन विरूपकरणेन च}
{वधदण्डेन क्लिश्या न पुरोहितससदि}


\twolineshloka
{यदा पुरोहितं वा ते पर्येयुः शरणैपिणः}
{करिष्यामः पुनर्ब्रह्मन्न पापमिति वादिनः}


\twolineshloka
{तदा विसर्गमर्हाः स्युरितीदं धातृशासनम्}
{विभ्रद्दण्डाजिनं मुण्डो ब्राह्मणोऽर्हति शासनम्}


\threelineshloka
{गरीयांसो गरीयांसमपराधे पुनः पुनः}
{तदा विसर्गमर्हन्ति न यथा प्रथमे तथा ॥द्युमत्सेन उवाच}
{}


\threelineshloka
{यत्रयत्रैव शक्येरन्संयन्तुं समये प्रजाः}
{स तावान्प्रोच्यते धर्मो यावन्न प्रतिलङ्घ्यते}
{अहन्यमानेषु पुनः सर्वमेव पराभवेत्}


\twolineshloka
{पूर्वे पूर्वतरे चैव सुशास्या ह्यभवञ्जनाः}
{मृदवः सत्यभूयिष्ठा अल्पद्रोहाल्पमन्यवः}


\twolineshloka
{पुरा धिग्दण्ड एवासीद्वाग्दण्डस्तदनन्तरम्}
{आसीदादानदण्डोऽपि वधदण्डोऽद्य वर्तते}


% Check verse!
वधेनापि न शक्यन्ते नियन्तुमपरे जनाः
\twolineshloka
{नैव दस्युर्मनुष्याणां न देवानामिति श्रुतिः}
{न गन्धर्वपितृणां च कः कस्येह न कश्चन}


\threelineshloka
{पक्वं श्मशानादादत्ते पिशाचांश्चापि दैवतम्}
{तेषु यः समयं कश्चित्कुर्वीत हतबुद्धिषु ॥सत्यवानुवाच}
{}


\twolineshloka
{तान्न शक्नोषि चेत्साधून्परित्रातुमहिंसया}
{कस्यचिद्भूतभव्यस्य लोभेनान्तं तथा कुरु}


\twolineshloka
{राजानो लोकयात्रार्थं तप्यन्ते परमं तपः}
{तेऽपत्रपन्ति तादृग्भ्यस्तथावृत्ता भवन्ति च}


\twolineshloka
{वित्रास्यमानाः सुकृतो न कामाद्धन्ति दुष्कृतीन्}
{सुकृतेनैव राजानो भूयिष्ठं शासते प्रजाः}


\threelineshloka
{श्रेयसः श्रेयसोऽप्येवं वृत्तं लोकोऽनुवर्तते}
{सदैव हि गुरोर्वृत्तमनुवर्तन्ति मानवाः ॥द्युमत्सेन उवाच}
{}


\twolineshloka
{आत्मानमसमाधाय समाधित्सति यः परान्}
{विषयेष्विन्द्रियवशं मानवाः प्रहसन्ति तम्}


\twolineshloka
{यो राज्ञो दम्भमोहेन किंचित्कुर्यादसांप्रतम्}
{सर्वोपायैर्नियम्यः स तथा पापान्निवर्तते}


\twolineshloka
{आत्मैवादौ नियन्तव्यो दुष्कृंतं संनियच्छता}
{दण्डयेच्च महादण्डैरपि बन्धूननन्तरान्}


\twolineshloka
{`यो राजा लोभमोहेन किंचित्कुर्यादसांप्रतम्}
{सर्वोपायैर्नियम्यः स तथा पापान्निवर्तते ॥'}


\twolineshloka
{यत्र वै पापकृन्नीचो न महद्दुःखमर्च्छति}
{वर्धन्ते तत्र पापानि धर्मो ह्रसति च ध्नुवम्}


\threelineshloka
{इति कारुण्यशीलस्तु विद्वान्वै ब्राह्मणोऽन्वशात्}
{इति चैवानुशिष्टोऽस्मि पूर्वैस्तातपितामहैः}
{आश्वासयद्भिः सुभृशमनुक्रोशात्तथैव च}


\threelineshloka
{एतत्प्रथमकल्पेन राजा कृतयुगे जयेत्}
{पादोनेनापि धर्मेण गच्छेत्रेतायुगे तथा}
{द्वापरे तु द्विपादेन पादेन त्ववरे युगे}


\twolineshloka
{तथा कलियुगे प्राप्ते राज्ञो दुश्चरितेन ह}
{भवेत्कालविशेषेण कला धर्मस्य पोडशी}


\twolineshloka
{अथ प्रथमकल्पेन सत्यवन्संकरो भवेत्}
{आयुः शक्तिं च कालं च निर्दिश्य तप आदिशेत्}


\twolineshloka
{सत्याय हि यथा नेह जह्याद्धर्मफलं महत्}
{भूतानामनुकम्पार्थं मनुः स्वायंभुवोऽब्रवीत्}


\chapter{अध्यायः २७४}
\twolineshloka
{युधिष्ठिर उवाच}
{}


\twolineshloka
{अविरोधेन भूतानां त्यागः षाङ्गुण्यकारकः}
{यः स्यादुमयभाग्धर्मस्तन्मे ब्रूहि पितामह}


\threelineshloka
{गार्हस्थ्यस्य च धर्मस्य योगधर्मस्य चोभयोः}
{अदूरसंप्रस्थितयोः किंस्विच्छ्रेयः पितामह ॥भीष्म उवाच}
{}


\twolineshloka
{उभौ धर्मौ महाभागावृभौ परमदुश्चरौ}
{उभौ नहाफलौ तौ तु सद्भिराचरितावुभौ}


\twolineshloka
{अव ते वर्तयिष्यासि प्रामाण्यमुभयोस्तयोः}
{शुणुष्वैकमताः पार्थ च्छिन्नधर्मार्थसंशयम्}


\twolineshloka
{अत्राप्युदाहरन्तीममितिहासं पुरातनम्}
{कपिलस्य गोश्च संवादं तन्निबोध युधिष्ठिर}


\twolineshloka
{आम्नायमनुपश्यन्हि पुराणं शाश्वतं ध्रुवम्}
{नहुषः पूर्वमालेभे त्वष्टुर्गामिति नः श्रुतम्}


\twolineshloka
{तां नियुक्तामदीनात्मा सत्वस्थः संयमे रतः}
{ज्ञानवान्नियताहारो ददर्श कपिलस्तथा}


\twolineshloka
{स बुद्धिमुत्तमां प्राप्तो नैष्ठिकीमकुतोभयाम्}
{स्वरेण शिथिलां सत्यां वेदा 3 इत्यब्रवीत्सकृत्}


\twolineshloka
{तां गामृषिः स्यूमरश्मिः प्रविश्य यतिमब्रवीत्}
{हंहो वेदा 3 यदि मता धर्माः केनापरे मताः}


\twolineshloka
{तपस्विनो धृतिमतः श्रुतिविज्ञानचक्षुषः}
{सर्वमार्षं हि मन्यन्ते व्याहृतं विदितात्मनः}


\threelineshloka
{तस्यैवं गततृष्णस्य विज्वरस्य निराशिषः}
{का विवक्षाऽस्ति वेदेषु निरारम्भस्य सर्वतः ॥कपिल उवाच}
{}


\twolineshloka
{नाहं वेदान्विनिन्दामि न विवक्ष्यामि कर्हिचित्}
{पृथगाश्रमिणां कर्माण्येकार्थानीति नः श्रुतम्}


\twolineshloka
{गच्छत्येव परित्यागी वानप्रस्थश्च गच्छति}
{गृहस्थो ब्रह्मचारी च उभौ तावपि गच्छतः}


\twolineshloka
{देवयाना हि पन्थानश्चत्वारः शाश्वता मताः}
{नैषां ज्यायः कनीयस्त्वं फलेषूक्तं बलाबलम्}


\twolineshloka
{एवं विदित्वा सर्वार्थानारभेतेति वैदिकम्}
{नारभेतेति चान्यत्र नैष्ठिकी श्रूयते श्रुतिः}


\twolineshloka
{अनारम्भे ह्यदोषः स्यादारम्भे दोष उत्तमः}
{एवं स्थितस्य शास्त्रस्य दुर्विज्ञेयं बलाबलम्}


\threelineshloka
{यदत्र किंचित्प्रत्यक्षमहिंसायाः परं मतम्}
{ऋते त्वागमशास्त्रेभ्यो ब्रूहि तद्यदि पश्यसि ॥स्यूमरश्मिरुवाच}
{}


\twolineshloka
{स्वर्गकामो यजेतेति सततं श्रूयते श्रुतिः}
{फलं प्रकल्प्य पूर्वं हि ततो यज्ञः प्रतायते}


\twolineshloka
{अजश्चाश्वश्च मेषश्च र्गौश्च पक्षिगणाश्च ये}
{ग्राम्यारण्याश्चौषधयः प्राणस्यान्नमिति श्रुतिः}


\twolineshloka
{तथैवान्नं ह्यहरहः सायंप्रातर्निरूप्यते}
{पशवश्चाथ धान्यं च यज्ञस्याङ्गमिति श्रुतिः}


\twolineshloka
{एतानि सह यज्ञेन प्रजापतिरकल्पयत्}
{तेन प्रजापतिर्देवान्यज्ञेनायजत प्रभुः}


% Check verse!
तदन्योन्यवराः सर्वे प्राणिनः सप्तसप्त च
\twolineshloka
{`गौरजो मनुजः श्वा वा अश्वाश्वतरगर्दभाः}
{एते ग्राम्याः समाख्याताः पशवः सप्त साधुभिः}


\twolineshloka
{सिंहा व्याघ्रा वराहाश्च महिषा वारणास्तथा}
{हरिणः शललाश्चैव सप्तारण्यास्तथा स्मृताः ॥'}


% Check verse!
यज्ञेषूपाकृतं विश्वं प्राहुरुत्तमसंज्ञितम्
\twolineshloka
{एतच्चैवाभ्यनुज्ञातं पूर्वैः पूर्वतरैस्तथा}
{को जातु न विचिन्वीत विद्वान्स्वां शक्तिमात्मनः}


\twolineshloka
{पशवश्च मनुष्याश्च द्रुमाश्चौषधिभिः सह}
{स्वर्गमेवाभिकाङ्क्षन्ते न च स्वर्गोस्ति ते मखात्}


\twolineshloka
{ओषध्यः पशवो वृक्षा वीरुदाज्यं पयो दधि}
{हविर्भूमिर्दिशः श्रद्धा कालश्चैतानि द्वादश}


\twolineshloka
{ऋचो यजूंषि सामानि ऋत्विजश्चापि षोडश}
{अग्निर्ज्ञेयो गृहपतिः स सप्तदश उच्यते}


\twolineshloka
{अङ्गान्येतानि यज्ञस्य यज्ञो मूलमिति श्रुतिः}
{आज्येन पयसा दध्ना शकृताऽऽमिक्षया त्वचा}


\twolineshloka
{बालैः शृङ्गेण पादेन संभवत्येव गौर्मखम्}
{एवं प्रत्यकेशः सर्वं यद्यदस्य विधीयते}


\twolineshloka
{यज्ञं वहन्ति संभूय सहत्विंग्भिः सदक्षिणैः}
{संहृत्यैतानि सर्वाणि यज्ञं निर्वर्तयन्त्युत}


\twolineshloka
{यज्ञार्थानि हि सृष्टानि यथार्था श्रूयते श्रुतिः}
{एवं पूर्वतराः पूर्वे प्रवृत्ताश्चैव मानवाः}


\twolineshloka
{न हिनस्ति नारभते नाभिद्रुह्यति किंचन}
{यज्ञैर्यष्टव्यमित्येव यो यजत्यफलेप्सया}


\twolineshloka
{यज्ञाङ्गान्यपि चैतानि यथोक्तान्यपि सर्वशः}
{विधिना विधियुक्तानि तारयन्ति परस्परम्}


\twolineshloka
{आम्नायमार्षं पश्यामि यस्मिन्वेदाः प्रतिष्ठिताः}
{तं विद्वांसोऽनुपश्यन्ति ब्राह्मणस्यानुदर्शनात्}


\twolineshloka
{ब्राह्मणप्रभवो यज्ञो ब्राह्मणार्पण एव च}
{अनुयज्ञं जगत्सर्वं यज्ञश्चानुजगत्सदा}


\twolineshloka
{ओमिति ब्रह्मणो योनिर्नमः स्वाहा स्वधा वषट्}
{यस्यैतानि प्रयुज्यन्ते यथाशक्ति कृतान्यपि}


\twolineshloka
{न तस्य त्रिषु लोकेषु परलोकभयं विदुः}
{इति वेदा वदन्तीह सिद्धाश्च परमर्षयः}


\twolineshloka
{ऋचो यजूंहि सामानि स्तोत्राश्च विधिचोदिताः}
{यस्मिन्नेतानि सर्वाणि भवन्तीह स वै द्विजः}


\twolineshloka
{अग्न्याधेये यद्भवति यच्च सोमे सुते द्विज}
{यच्चेतरैर्महायज्ञैर्वेद तद्भगवांस्तथा}


\twolineshloka
{तस्माद्ब्रह्मन्यजेच्चैव याजयेच्चाविचारयन्}
{यजतो यज्ञविधिना प्रेत्य स्वर्गफलं महत्}


\twolineshloka
{नायं लोकोस्त्ययज्ञानां परश्चेति विनिश्चयः}
{वेदवादविदश्चैव प्रमाणमुभयं तदा}


\chapter{अध्यायः २७५}
\twolineshloka
{कपिल उवाच}
{}


\twolineshloka
{एतावदनुपश्यन्तो यतयो यान्ति मार्गगाः}
{नैषां सर्वेषु लोकेषु कश्चिदस्ति व्यतिक्रमः}


\twolineshloka
{निर्द्वन्द्वा निर्नमस्कारा निराशीर्बन्धना बुधाः}
{विमुक्ताः सर्वपापेभ्यश्चरन्ति शुचयोऽमलाः}


\twolineshloka
{अपवर्गेऽथ संत्यागे बुद्धौ च कृतनिश्चयाः}
{ब्रह्मिष्ठा ब्रह्मभूताश्च ब्रह्मण्येव कृतालयाः}


\threelineshloka
{येऽशोका नष्टरजसस्तेषां लोकाः सनातनाः}
{तेषां गतिं परां प्राप्य गार्हस्थ्ये किं प्रयोजनम् ॥स्यूमरश्मिरुवाच}
{}


\twolineshloka
{यद्येषां परमा निष्ठा यद्येषां परमा गतिः}
{गृहस्थानव्यपाश्रित्य नाश्रमोऽन्यः प्रवर्तते}


\twolineshloka
{यथा मातरमाश्रित्य सर्वे जीवन्ति जन्तवः}
{एवं गार्हस्थ्यमाश्रित्य वर्तन्त इतराश्रमाः}


\twolineshloka
{गृहस्थ एव यजते गृहस्थस्तप्यते तपः}
{गार्हस्थ्यमस्य धर्मस्य मूलं यत्किंचिदेव हि}


\twolineshloka
{प्रजनाद्यभिनिर्वृत्ताः सर्वे प्राणभृतो मुनेः}
{प्रजनं चाप्युतान्यत्र न कथंचन विद्यते}


\twolineshloka
{यास्तु स्युर्बहिरोषध्यो बहिरन्यास्तथाऽद्रिजाः}
{ओषधिभ्यो बहिर्यस्मात्प्राणी कश्चिन्न विद्यते}


\twolineshloka
{कस्यैषा वाग्भवेत्सत्या मोक्षो नास्ति गृहादिति}
{अश्रद्दधानैरप्राज्ञैः सूक्ष्मदर्शनवर्जितैः}


\twolineshloka
{निराशैरलसैः श्रान्तैस्तप्यमानैः स्वकर्मभिः}
{शमस्योपरमो दृष्टः प्रव्रज्यायामपण़्डितैः}


\twolineshloka
{त्रैलोक्यस्येव हेतुर्हि मर्यादा शाश्वती ध्रुवा}
{ब्राह्मणो नाम भगवाञ्जन्मप्रभृति पूज्यते}


\twolineshloka
{प्राग्गर्भाघानमन्त्रा हि प्रवर्तन्ते द्विजातिषु}
{अविश्वस्तेषु वर्तन्ते विश्वस्तेष्वपि संश्रिताः}


\twolineshloka
{दाहे पुनः संश्रयणे संश्रिते पात्रभोजने}
{दानं गवां पशूनां वा पिण्डानामप्सु मज्जनम्}


\twolineshloka
{अर्चिष्मन्तो बर्हिषदः क्रव्यादाः पितरस्तथा}
{मृतस्याप्यनुमन्यन्ते मन्त्रा मन्त्राश्च कारणम्}


\twolineshloka
{एवं क्रोशत्सु वेदेषु कुतो मोक्षोऽस्ति कस्यचित्}
{ऋणवन्तो यदा मर्त्याः पितृदेवद्विजातिषु}


\twolineshloka
{श्रिया विहीनैरलसैः पण्डितैश्च पलायितम्}
{वेदवादापरिज्ञानं सत्याभासमिवानृतम्}


\twolineshloka
{न वै पापैर्हियते कृष्यते वायो ब्राह्मणो यजते वेदशास्त्रैः}
{ऊर्ध्वं यजन्पशुभिः सार्धमेतिततः पुनस्तर्कयते न कामान्}


\threelineshloka
{न वेदानां परिभवान्न शाठ्येन न मायया}
{महत्प्राप्नोति पुरुषो ब्राह्मणो ब्रह्म विन्दति ॥कपिल उवाच}
{}


\twolineshloka
{दर्शश्च पूर्णमासश्च अग्निहोत्रं च धीमताम्}
{चातुर्मास्यानि चैवासंस्तेषु यज्ञः सनातनः}


\twolineshloka
{अनारम्भाः सुधृतयः शुचयो ब्रह्मसंज्ञिताः}
{ब्राह्मणा एव ते देवांस्तर्पन्त्यमृतैरिव}


\twolineshloka
{सर्वभूतात्मभूतस्य सर्वभूतानि पश्यतः}
{देवाऽपि मार्गे मुह्यन्ति ह्यपदस्य पदैषिणः}


\twolineshloka
{चतुर्द्वारं पुरुष चर्तुर्मुखंचतुर्मुखो नैनमुपैति निन्दा}
{बाहुभ्यां पद्भ्यामुदरादुपस्थात्तेषां द्वारं द्वारपालो बुभूषेत्}


\twolineshloka
{नाक्षैर्दीव्येन्नाददीतान्यवित्तंन वाऽयोनीयस्य शृतं प्रगृह्णात्}
{क्रुद्धो न चैव प्रहरेत धीमांस्तथास्य तत्पाणिपादं सुगुप्तम्}


\twolineshloka
{नाक्रोशमृच्छेन्न वृथा वदेच्चन पैशुनं जनवादं च कुर्यात्}
{सत्यव्रतो मितभाषोऽप्रमत्तस्तथाऽस्य वाग्द्वारमथो सुगुप्तम्}


\twolineshloka
{नानाशनः स्यान्न महाशनः स्यान्न लोलुपः साधुभिरागतः स्यात्}
{यात्रार्थमाहारमिहाददीततथाऽस्य स्याज्जाठरद्वारगुप्तिः}


\twolineshloka
{न वीरपत्नीं विहरेत नारींन चापि नारीमनृतावाह्वयीत}
{भार्याव्रतं ह्यात्मनि धारयीततथास्योपस्थद्वारगुप्तिर्भवेत्}


\twolineshloka
{द्वाराणि यस्य सर्वाणि सुगुप्तानि मनीषिणः}
{उपस्थमुदरं पाणी वाक्चतुर्थी स वै द्विजः}


\twolineshloka
{मोघान्यगुप्तद्वारस्य सर्वाण्येव भवन्त्युत}
{किं तस्य तपसा कार्यं किं यज्ञेन किमात्मना}


\twolineshloka
{अनुत्तरीयवसनमनुपस्तीर्णशायिनम्}
{बाहूपधानं शाम्यन्तं तं देवा ब्राह्मणं विदुः}


\twolineshloka
{द्वन्द्वारामेषु सर्वेषु य एको रमते मुनिः}
{परेषामननुध्यायंस्तं देवा ब्राह्मणं विदुः}


\twolineshloka
{येन सर्वमिदं बुद्धं प्रकृतिर्विकृतिश्च या}
{गतिज्ञः सर्वभूतानां तं देवा ब्राह्मणं विदुः}


\twolineshloka
{अभयं सर्वभूतेभ्यः सर्वेषामभयं यतः}
{सर्वभूतात्मभूतो यस्तं देवा ब्राह्मणं विदुः}


\twolineshloka
{नान्तरेणानुजानाति वेदानां यत्क्रियाफलम्}
{अविज्ञाय च तत्सर्वमन्यद्रोचयते फलम्}


\twolineshloka
{स्वकर्मभिः संश्रितानां तपो घोरत्वमागमत्}
{तं सदाचारमाश्चर्यं पुराणं शाश्वतं ध्रुवम्}


\twolineshloka
{अशक्नुवन्तश्चरितुं किंचिद्धर्मेषु सूत्रितम्}
{निरापद्धर्म आचारो ह्यप्रमादो पराभवः}


\twolineshloka
{फलवन्ति च कर्माणि व्युष्टिमन्ति ध्रवाणि च}
{विगुणानि च पश्यन्ति तथा नैकानि केन च}


\threelineshloka
{गुणाश्चात्र सुदुर्ज्ञेया ज्ञाताश्चात्र सुदुष्कराः}
{अनुष्ठिताश्चान्तवन्त इति त्वमनुपश्यसि ॥स्यूमरश्मिरुवाच}
{}


\threelineshloka
{यथा च वेदप्रामाण्यं त्यागश्च सफलो यथा}
{तौ पन्थानावुभौ व्यक्तौ भगवंस्तद्ब्रवीहि मे ॥कपिल उवाच}
{}


\threelineshloka
{प्रत्यक्षमिह पश्यन्ति भवन्तः सत्पथे स्थिताः}
{प्रत्यक्षं तु किमत्रास्ति यद्भवन्त उपासते ॥स्यूमरश्मिरुवाच}
{}


\twolineshloka
{स्यूमरश्मिरहं ब्रह्मञ्जिज्ञासार्थमिहागतः}
{श्रेयस्कामः प्रत्यवोचमार्जवान्न विवक्षया}


\threelineshloka
{इमं च संशयं घोरं भगवान्प्रब्रवीतु मे}
{प्रत्यक्षमिह पश्यन्तो भवन्तः सत्पथे स्थिताः}
{किमत्र प्रत्यक्षतमं भवन्तो यदुपासते}


\twolineshloka
{अन्यत्र तर्कशास्त्रेभ्य आगमार्थं यथागमम्}
{आगमो वेदवादास्तु तर्कशास्त्राणि चागमः}


\twolineshloka
{यथाश्रममुपासीत आगमस्तत्र सिध्यति}
{सिद्धिः प्रत्यक्षरूपा च दृश्यत्यागमनिश्ययात्}


\twolineshloka
{नौर्नावीव निबद्धा हि स्रोतसा सनिबन्धना}
{ह्रियमाणा कथ विप्र कुबुद्धींस्तारयिष्यति}


\threelineshloka
{एतद्ब्रवीतु भगवानुपपन्नोऽस्म्यधीहि भो}
{नैव त्यागी न संतुष्टो नाशोको न निरामयः}
{नानिर्विवित्सो नावृत्तो नापवृत्तोऽस्ति कश्चन}


\twolineshloka
{भवन्तोऽपि च हृष्यन्ति शोचन्ति च यथा वयम्}
{इन्द्रियार्थाश्च भवतां समानाः सर्वजन्तुषु}


\fourlineindentedshloka
{एवं चतुर्णां वर्णानामाश्रमाणां प्रवृत्तिषु}
{एकमालम्बमानानां निर्णये किं निरामयम्}
{`एतद्ब्रवीतु भगवानुपपन्नोस्य्यधीहि भो ॥' कपिल उवाच}
{}


\twolineshloka
{यद्यदाचरते शास्त्रमर्थ्यं सर्वप्रवृत्तिषु}
{यस्य यत्र ह्यनुष्ठानं तस्य तत्तु निरामयम्}


\twolineshloka
{सर्वं प्रापयति ज्ञानं ये ज्ञानं ह्यनुवर्तते}
{ज्ञानादपेत्य या वृत्तिः सा विनाशयति प्रजाः}


\twolineshloka
{भवन्तो ज्ञानिनो नित्यं सर्वतश्च निरमयाः}
{ऐकात्म्यं नाम कश्चिद्धि कदाचिदभिपद्यते}


\twolineshloka
{शास्त्रं ह्यबुद्ध्वा तत्त्वेन केचिद्वादबलाज्जनाः}
{कामद्वेषाभिभूतत्वादहंकारवशं गताः}


\twolineshloka
{याथातथ्यमविज्ञाय शास्त्राणां शास्त्रदस्यवः}
{ब्रह्मस्तेना निरारम्भा अपक्वमनसोऽशिवाः}


\twolineshloka
{नैर्गुण्यमेव पश्यन्ति न गुणाननुयुञ्जते}
{तेषां तमः शरीराणां तम एव परायणम्}


\threelineshloka
{यो यथाप्रकृतिर्जन्तुः प्रकृतेः स्याद्वशानुगः}
{तस्य द्वेषश्च कामश्च क्रोधो दम्भोऽनृतं मदः}
{नित्यमेवानुवर्तन्ते गुणाः प्रकृतिसंभवाः}


\threelineshloka
{ये तद्बुद्ध्वाऽनुपश्यन्तः संत्यजेयुः शुभाशुभम्}
{परां गतिमभीप्सन्तो यतयः संयमे रताः ॥स्यूमरश्मिरुवाच}
{}


\twolineshloka
{सर्वमेतत्त्वया ब्रह्मञ्शास्त्रतः परिकीर्तितम्}
{न ह्यविज्ञाय शास्त्रार्थं प्रवर्तन्ते प्रवृत्तयः}


\twolineshloka
{यः कश्चिन्न्याय्य आचारः सर्वं शास्त्रमिति श्रुतिः}
{यदन्याय्यमशास्त्रं तदित्येषा श्रूयते श्रुतिः}


\twolineshloka
{न प्रवृत्तिर्ऋते शास्त्रात्काचिदस्तीति निश्चयः}
{यदन्यद्वेदवादेभ्यस्तदशास्त्रमिति श्रुतिः}


\threelineshloka
{शास्त्रादपेतं पश्यन्ति बहवोऽत्यर्थमानिनः}
{शास्त्रदोषान्न पश्यन्ति इह चामुत्र चापरे}
{[इन्द्रियार्थाश्च भवतां समानाः सर्वजन्तुषु}


\twolineshloka
{एवं चतुर्णां वर्णानामाश्रमाणां प्रवृत्तिषु}
{एकमालम्बमानानां निर्णये सर्वतो दिशम्}


\twolineshloka
{आनन्त्यं वदमानेन शक्तेनावर्जितात्मना]}
{अविज्ञानहतप्रज्ञा हीनप्रज्ञास्तमोवृताः}


\twolineshloka
{शक्यं त्वेकेन युक्तेन कृतकृत्येन सर्वशः}
{पिण्डमात्रं व्यपाश्रित्य चरितुं सर्वतो दिशम्}


\threelineshloka
{`नात्यन्तं वन्दमानेन शक्तेन विजितात्मना}
{'वेदवादं व्यपाश्रित्य मोक्षोऽस्तीति प्रभापितुम्}
{अपेतन्यायशास्त्रेण सर्वलोकविगर्हिणा}


\twolineshloka
{इदं तु दुष्करं कर्म कुटुम्बमभिसंश्रितम्}
{दानमध्ययनं यज्ञः प्रजासंतानमार्जवम्}


\twolineshloka
{यद्येतदेवं कृत्वाऽपि न विमोक्षोऽस्ति कस्यचित्}
{धिक्कर्तारं च कार्यं च श्रमश्चायं निरर्थकः}


\twolineshloka
{नास्तिक्यमन्यथा च स्याद्वेदानां पृष्ठतः क्रिया}
{एतस्यानन्त्यमिच्छामि भगवञ्श्रोतुमञ्जसा}


\twolineshloka
{तत्त्वं वदस्व मे ब्रह्मन्नुपसन्नोस्म्यधीहि भोः}
{यथा ते विदितो मोक्षस्तथेच्छाम्युपशिक्षितुम्}


\chapter{अध्यायः २७६}
\twolineshloka
{कपिल उवाच}
{}


\twolineshloka
{वेदाः प्रमाणं लोकानां न वेदाः पृष्ठतः कृताः}
{द्वे ब्रह्मणी वेदितव्ये शब्दब्रह्म परं च यत्}


\twolineshloka
{शब्दब्रह्मणि निष्णातः परं ब्रह्माधिगच्छति}
{शरीरमेतत्कुरुते यद्वेदे कुरुते तनुम्}


\twolineshloka
{कृतशुद्धशरीरो हि पात्रं भवति ब्राह्मणः}
{आनन्त्यमनुचिन्त्येदं कर्मणां तद्ब्रबीमि ते}


\twolineshloka
{निरागममनैतिह्यमत्यक्षं लोकसाक्षिकम्}
{धर्म इत्येव ये यज्ञान्वितन्वन्ति निराशिषः}


\twolineshloka
{उत्पन्नत्यागिनो लुब्धाः कृपासूयाविवर्जिताः}
{धनिनामेष वै पन्थास्तीर्थेषु प्रतिपादनम्}


\twolineshloka
{अनाश्रिताः पापकृत्याः कदाचित्कर्मयोगिनः}
{मनः संकल्पसंसिद्धा विशुद्धज्ञाननिश्चयाः}


\twolineshloka
{अक्रुध्यन्तोऽनसूयन्तो निरहंकारमत्सराः}
{ज्ञाननिष्ठास्त्रिशुक्लाश्च सर्वभूतहिते रताः}


\twolineshloka
{आसन्गृहस्था भूयिष्ठा अपक्रान्ताः स्वकर्मसु}
{राजानश्च तथा युक्ता ब्राह्मणाश्च यथाविधि}


\twolineshloka
{समा ह्यार्जवसंपन्नाः संतुष्टा ज्ञाननिश्चयाः}
{प्रत्यक्षधर्माः शुचयः श्रद्दधानाः परावरे}


\twolineshloka
{पुरस्ताद्भावितात्मानो यथावच्चरितव्रताः}
{चरन्ति धर्मं कृच्छ्रेऽपि दुर्गे चैवापि संहताः}


\twolineshloka
{संहत्य धर्मं चरतां पुराऽऽसीत्सुखमेव तत्}
{तेषां नासीद्विधातव्यं प्रायश्चित्तं कथंचन}


\twolineshloka
{सत्यं हि धर्ममास्थाय दुराधर्षतमा मताः}
{न मात्रामनुरुध्यन्ते न धर्मच्छलमन्ततः}


% Check verse!
य एव प्रथमः कल्पस्तमेवात्र चरन्महान्
\twolineshloka
{अस्यां स्थितौ स्थितानां हि प्रायश्चित्तं न विद्यते}
{यदा तु दुर्बलात्मानः प्रायश्चित्तं तदा भवेत्}


\twolineshloka
{एत एवंविधाः प्राहुः पुराणा यज्ञवाहनाः}
{त्रैविद्यवृद्धाः शुचयो वृत्तवन्तो यशस्विनः}


\twolineshloka
{यजन्तोऽहरहर्यज्ञैर्निराशीर्बन्धना बुधाः}
{तेषा यज्ञाश्च वेदाश्च कर्माणि च यथागमम्}


\twolineshloka
{आगमाश्च यथाकामं संकल्पाश्च यथाव्रतम्}
{अपेतकामक्रोधानां दुश्चराचारकर्मणाम्}


\twolineshloka
{स्वकर्मभिः शंसितानां प्रकृत्या शंसितात्मनाम्}
{ऋजूनां शमनित्यानां स्वेषु कर्मसु वर्तताम्}


\twolineshloka
{सर्वमानन्त्यमेवासीदिति नः शाश्वताश्चुतिः}
{तेषामदीनसत्वानां दुश्चराचारकर्मणाम्}


\twolineshloka
{स्वकर्मभिः संसितानां तपो घोरत्वमागतम्}
{सं सदाचारमाश्चर्यं पुराणं शाश्वतं ध्रुवम्}


\twolineshloka
{अशक्नुवद्भिश्चरितुं किंचिद्धर्मेषु सूचितम्}
{निरापद्धर्म आचारो ह्यप्रमादोऽपराभवः}


\twolineshloka
{सर्ववर्णेषु यत्तेषु नासीत्कश्चिद्व्यतिक्रमः}
{धर्ममेकं चतुष्पादमाश्रितास्ते नरा विभो}


\twolineshloka
{तं सन्तो विधिवत्प्राप्य गच्छन्ति परमां गतिम्}
{गृहेभ्य एव निष्क्रम्य वनमन्ये समाश्रिताः}


\twolineshloka
{गृहमेवाभिसंश्रित्य ततोऽन्ये ब्रह्मचारिणः}
{`व्यस्तमेकं चतुर्धा तु ब्राह्मणा आश्रमं विदुः}


\twolineshloka
{सर्वे सर्वत्र तिष्ठन्तो गच्छन्ति परमां गतिम्}
{एव एवंविधाः प्राहुः पुराणा ब्रह्मचारिणः ॥'}


\threelineshloka
{त एते दिवि दृश्यन्ते ज्योतिर्भूता द्विजातयः}
{नक्षत्राणीव धिष्ण्येषु बहवस्तारकागणाः}
{आनन्त्यमुपसंप्राप्ताः संतोषादिति वैदिकम्}


\twolineshloka
{यद्यागच्छन्ति संसारं पुनर्योनिषु तादृशाः}
{न लिप्यन्ते पारकृत्यैः कदाचित्कर्मयोनितः}


\twolineshloka
{एवमेव ब्रह्मचारी शुश्रूषुर्घोरनिश्चयः}
{एवंयुक्तो ब्राह्मणः स्यादन्यो ब्राह्मणको भवेत्}


\twolineshloka
{कर्मैव पुरुषस्याह शुभं वा यदि वाऽशुभम्}
{एवं पक्वकषायाणामानन्त्येन श्रुतेन च}


\twolineshloka
{सर्वमानन्त्यमेवासीदिति नः शाश्वती श्रुतिः}
{तेषामपेततृष्णानां निर्णिक्तानां शुभात्मनाम्}


\twolineshloka
{चतुर्थ औपनिषदो धर्मः साधारणः स्मृतः}
{संसिद्धैः सेव्यते नित्यं ब्राह्मणैर्नियतात्मभिः}


\twolineshloka
{संतोषमूलस्त्यागात्मा ज्ञानाधिष्ठानमुच्यते}
{अपवर्गमतिर्नित्यो यतिधर्मः सनातनः}


\fourlineindentedshloka
{साधारणः केवलो वा यथाबलमुपास्यते}
{गच्छन्ते बलिनः क्षेमं दुर्बलोऽत्रावसीदति}
{ब्राह्मणः पदमन्विच्छन्संसारान्मुच्यते शुचिः ॥स्यूमरश्मिरुवाच}
{}


\twolineshloka
{ये भुञ्जते ये ददते यजन्तेऽधीयते च ये}
{मात्राभिर्धर्मलुब्धाभिर्ये वा त्यागं समाश्रिताः}


\threelineshloka
{एतेषां प्रेत्यभावे तु कतमः स्वर्गजित्तमः}
{एतदाचक्ष्व मे ब्रह्मन्याथातथ्येन पृच्छतः ॥कपिल उवाच}
{}


\threelineshloka
{परिग्रहाः शुभाः सर्वे गुणतोऽभ्युदयाश्च ते}
{न तु त्यागसुखं प्राप्ता एतत्त्वमपि पश्यसि ॥स्यूमरश्मिरुवाच}
{}


\twolineshloka
{भवन्तो ज्ञाननिष्ठा वै गृहस्थाः कर्मनिश्चयाः}
{आश्रमाणां च सर्वेषां निष्ठायामैक्यमुच्यते}


\threelineshloka
{एकत्वेन पृथक्त्वेन विशेषो नान्य उच्यते}
{तद्यथावद्यथान्यायं भगवान्प्रब्रवीतु मे ॥कपिल उवाच}
{}


\twolineshloka
{शरीरपक्तिः कर्माणि ज्ञानं तु परमा गतिः}
{पक्वे कषायविज्ञानं यथा ज्ञानं च तिष्ठति}


\twolineshloka
{आनृशंस्यं क्षमा शान्तिरहिंसा सत्यमार्जवम्}
{अद्रोहोऽनभिमानश्च ह्रीस्तितिक्षा शमस्तथा}


\twolineshloka
{पन्थानो ब्रह्मणस्त्वेत एतैः प्राप्नोति यत्परम्}
{तद्विद्वाननुबुद्ध्येत मनसा कर्मनिश्चयम्}


\twolineshloka
{यां विप्राः सर्वतः शान्ता विशुद्धा ज्ञाननिश्चयाः}
{गतिं गच्छन्ति संतुष्टास्तामाहुः परमां गतिम्}


\twolineshloka
{वेदांश्च वेदितव्यं च विदित्वा च यथास्थितिम्}
{एवं वेदविदित्याहुरतोऽन्यो वातरेचकः}


\twolineshloka
{सर्वं विदुर्वेदविदो वेदे सर्वं प्रतिष्ठितम्}
{वेदे हि निष्ठा सर्वस्य यद्यदस्ति च नास्ति च}


\twolineshloka
{एषैव निष्ठा सर्वत्र यत्तदस्ति च नास्ति च}
{एतदन्तं च मध्यं च सच्चाऽसच्च विजानतः}


\twolineshloka
{समाप्तं त्याग इत्येव शम इत्येव निश्चितम्}
{संतोष इत्यनुगतमपवर्गे प्रतिष्ठितम्}


\twolineshloka
{ऋतं सत्यं विदितं वेदितव्यंसर्वस्यात्मा स्थावरं जङ्गमं च}
{सर्वं सुखं यच्छिवमुत्तरं चब्रह्माव्यक्तं प्रभवश्चाव्ययं च}


\twolineshloka
{तेजः क्षमा शान्तिरनामयं शुभंतथाविधं व्योम सनातनं ध्रुवम्}
{एतैः शब्दैर्गम्यते बुद्धिनेत्रैस्तस्मै नमो ब्रह्मणे ब्राह्मणाय}


\chapter{अध्यायः २७७}
\twolineshloka
{युधिष्ठिर उवाच}
{}


\threelineshloka
{धर्ममर्थं च कामं च वेदाः शंसन्ति भारत}
{कस्य लाभो विशिष्टोऽत्र तन्मे ब्रूहि पितामह ॥भीष्म उवाच}
{}


\twolineshloka
{अत्र ते वर्तयिष्यामि इतिहासं पुरातनम्}
{कुण्डधारेण यत्प्रीत्या भक्तायोपकृतं पुरा}


\twolineshloka
{अधनो ब्राह्मणः कश्चित्कामाद्धनमवैक्षत}
{यज्ञार्थं सततोऽर्थार्थी तपोऽतप्यत दारुणम्}


\twolineshloka
{स निश्चयमथो कृत्वा पूजयामास देवताः}
{भक्त्या न चैवाध्यगच्छद्धनं संपूज्य देवताः}


\twolineshloka
{ततश्चिन्तामनुप्राप्तः कतमद्दैवतं तु तत्}
{यन्मे द्रुतं प्रसीदेत मानुषैरजडीकृतम्}


\twolineshloka
{सोऽथ सौम्येन मनसा देवानुचरमन्तिके}
{प्रत्यपश्यज्जलधरं कुण्डधारमवस्थितम्}


\twolineshloka
{दृष्ट्वैव तं महाबाहुं तस्य भक्तिरजायत}
{अयं मे धास्यति श्रेयो वपुरेतद्धि तादृशम्}


\twolineshloka
{संनिकृष्टश्च देवस्य न चान्यैर्मानुषैर्वृतः}
{एष मे दास्यति धनं प्रभूतं शीघ्रमेव च}


\twolineshloka
{ततो धूपैश्च गन्धैश्च माल्यैरुच्चावचैरपि}
{बलिभिर्विविधाभिश्च पूजयामास तं द्विजः}


\twolineshloka
{ततस्त्वल्पेन कालेन तुष्टो जलधरस्तदा}
{तस्योपकारनियतामिमां वाचमुवाच ह}


\twolineshloka
{ब्रह्मघ्ने च सुरापे च चोरे भग्नव्रते तथा}
{निष्कृतिर्विहिता सद्भिः कृतघ्ने नास्ति निष्कृतिः}


\twolineshloka
{आशायास्तनयोऽधर्मः क्रोधोऽसूयासुतः स्मृतः}
{लोभः पुत्रो निकृत्यास्तु कृतघ्नो नार्हति प्रजां}


\twolineshloka
{ततः स ब्राह्मणः स्वप्ने कुण्डधारस्य तेजसा}
{अपश्यत्सर्वभूतानि कुशेषु शयितस्तदा}


\twolineshloka
{शमेन तपसा चैव भक्त्या च निरुपस्कृतः}
{शुद्धात्मा ब्राह्मणो रात्रौ निदर्शनमपश्यत}


\twolineshloka
{माणिभद्रं स तत्रस्थं देवतानां महाद्युतिम्}
{अपश्यत महात्मानं व्यादिशन्तं युधिष्ठिर}


\twolineshloka
{तत्र देवाः प्रयच्छन्ति राज्यानि च धनानि च}
{शुभैः कर्मभिरारब्धाः प्रच्छिदन्त्यशुभेषु च}


\twolineshloka
{पश्यतामथ यक्षाणं कुण्डधारो महाद्युतिः}
{निष्पत्य पतितो भूमौ देवानां भरतर्षभ}


\threelineshloka
{ततस्तु देववचनान्मणिभद्रो महामनाः}
{उवाच पतितं भूमौ कुण्डधार किमिच्छसि ॥कुण्डधार उवाच}
{}


\twolineshloka
{यदि प्रसन्ना देवा मे भक्तोऽयं ब्राह्मणो मम}
{अस्यानुग्रहमिच्छामि कृतं किंचित्सुखोदयम्}


% Check verse!
ततस्तं माणिभद्रस्तु पुनर्वचनमब्रवीत्देवानामेव वचनात्कुण्डधारं माहद्युतिम्
\twolineshloka
{उत्तिष्ठोत्तिष्ठ भद्रं ते कृतकृत्यः सुखी भव}
{धनार्थी यदि विप्रोऽयं धनमस्मै प्रदीयताम्}


\twolineshloka
{यावद्धनं प्रार्थयते ब्राह्मणोऽयं सथा तव}
{देवानां शासनात्तावदसङ्ख्येयं ददाम्यहम्}


\threelineshloka
{विचार्य कुण्डधारस्तु मानुष्यं चलमध्रुवम्}
{तपसो मतिमाधत्त ब्राह्मणस्य यशस्विनः ॥कुण्डधार उवाच}
{}


\twolineshloka
{नाहं धनानि याचामि ब्राह्मणाय धनप्रद}
{अन्यमेवाहमिच्छामि भक्तायानुग्रहं कृतम्}


\twolineshloka
{पृथिवीं रत्नपूर्णां वा महद्वा रत्नसंचयम्}
{भक्ताय नाहमिच्छमि भवेदेष तु धार्मिकः}


\threelineshloka
{धर्मेऽस्य रमतां बुद्धिर्धर्मं चैवोपजीवतु}
{धर्मप्रधानो भवतु ममैषोऽनुग्रहो मतः ॥माणिभद्र उवाच}
{}


\threelineshloka
{सदा धर्मफलं राज्यं सुखानि विविधानि च}
{फलान्येवायमश्नातु कायक्लेशविवर्जितः ॥भीष्म उवाच}
{}


\threelineshloka
{ततस्तदेव बहुशः कुण्डधारो महायशाः}
{अभ्यासमकरोद्धर्मे ततस्तुष्टास्तु देवताः ॥माणिभद्र उवाच}
{}


\threelineshloka
{प्रीतास्ते देवताः सर्वा द्विजस्यास्य तथैव च}
{भविष्यत्येष धर्मात्मा धर्मे चाधरस्यते मतिः ॥भीष्म उवाच}
{}


\twolineshloka
{ततः प्रीतो जलधरः कृतकार्यो युधिष्ठिर}
{ईप्सितं मनसो लब्ध्वा वरमन्यैः सुदुर्लभम्}


\threelineshloka
{ततोऽपश्यत चीराणि सूक्ष्माणि द्विजसत्तमः}
{पार्श्वतोऽभ्याशतो न्यस्तान्यथ निर्वेदमागतः ॥ब्राह्मण उवाच}
{}


\threelineshloka
{अयं न सुकृतं वेत्ति को न्वन्यो वेत्स्यते कृतम्}
{गच्छामि वनमेवाहं परं धर्मेण जीवितुम् ॥भीष्म उवाच}
{}


\twolineshloka
{निर्वेदाद्देवतानां च प्रसादात्स द्विजोत्तमः}
{वनं प्रविश्य सुमहत्तप आरब्धवांस्तदा}


\twolineshloka
{देवतातिथिशेषेण फलमूलाशनो द्विजः}
{धर्मे चास्य महाराज दृढा बुद्धिरजायत}


\twolineshloka
{त्यक्त्वा मूलफलं सर्वं पर्णाहारोऽभवद्द्विजः}
{पर्णं त्यक्त्वा जलाहारः पुनरासीद्द्विजस्तदा}


\twolineshloka
{वायुभक्षस्ततः पश्चाद्बहून्वर्षगणानभूत्}
{न चास्य क्षीयते प्राणस्तदद्भुतमिवाभवत्}


\twolineshloka
{धर्मे च श्रद्दधानस्य तपस्युग्रे च वर्ततः}
{कालेन महता तस्य दिव्या दृष्टिरजायत}


\twolineshloka
{तस्य बुद्धिः प्रादुरासीद्यदि दद्यामहं धनम्}
{तुष्टः कस्यचिदेवेह मिथ्या वाङ्ग भवेन्मम}


\twolineshloka
{ततः प्रहृष्टवदनो भूय आरब्धवांस्तपः}
{भूयश्चाचिन्तयत्सिद्धो यत्परं सोऽभिमन्यते}


\twolineshloka
{यदि दद्यामहं राज्यं तुष्टो वै यस्य कस्यचित्}
{स भवेदचिराद्राजा न मिथ्या वाग्भवेन्मम}


\twolineshloka
{तस्य साक्षात्कुण्डधारो दर्शयामास भारत}
{ब्राह्मणस्य तपोयोगात्सौहृदेनाभिचोदितः}


\twolineshloka
{समागम्य स तेनाथ पूजां चक्रे यथाविधि}
{ब्राह्मणः कुण्डधारस्य विस्मितश्चाभवन्नृप}


\twolineshloka
{ततोऽब्रवीत्कुण्डधारो दिव्यं ते चक्षुरुत्तमम्}
{पश्य राज्ञां गतिं विप्र लोकांश्चैव तु चक्षुषा}


\threelineshloka
{ततो राजसहस्राणि मग्नानि निरये तदा}
{दूरादपश्यद्विप्रः स दिव्ययुक्तेन चक्षुषा ॥कुण्डधार उवाच}
{}


\twolineshloka
{मां पूजयित्वा भावेन यदि त्वं दुःखमाप्नुयाः}
{कृतं मया भवेत्किं ते कश्च तेऽनुग्रहो भवेत्}


\threelineshloka
{पश्यपश्य च भूयस्त्वं कामानिच्छेत्कथं नरः}
{स्वर्गद्वारं हि संरुद्धं मानुषेषु विशेषतः ॥भीष्म उवाच}
{}


\threelineshloka
{ततोऽपश्यत्स कामं च क्रोधं लोभं भयं मदम्}
{निद्रां तन्द्रीं तथाऽऽलस्यमावृत्त्य पुरुषान्स्थितान् ॥'कुण्डधार उवाच}
{}


\twolineshloka
{एतैर्लोकाः सुसंरुद्धा देवानां मानुषाद्भयम्}
{तथैव देववचनाद्विघ्नं कुर्वन्ति सर्वशः}


\threelineshloka
{न देवैरननुज्ञातः कश्चिद्भवति धार्मिकः}
{एष शक्तोस्मि तपसा दातुं राज्यं धनानि च ॥भीष्म उवाच}
{}


\twolineshloka
{ततः पपात शिरसा ब्राह्मणस्तोयधारिणे}
{उवाच चैनं धर्मात्मा महान्मेऽनुग्रहः कृतः}


\twolineshloka
{कामलोभानुबन्धेन पुरा ते यदसूयितम्}
{मया स्नेहमविज्ञाय तत्र मे क्षन्तुमर्हसि}


\twolineshloka
{क्षान्तमेव मयेत्युक्त्वा कुण्डधारो द्विजर्षभम्}
{संपरिष्वज्य बाहुभ्यां तत्रैवान्तरधीयत}


\twolineshloka
{ततः सर्वांस्तदा लोकान्ब्राह्मणोऽनुचचार ह}
{कुण्डधारप्रसादेन तपसा सिद्धिमागतः}


\twolineshloka
{विहायसा च गमनं तथा संकल्पितार्थता}
{धर्माच्छक्त्या तथा योगाद्या चैव परमा गतिः}


\twolineshloka
{देवता ब्राह्मणाः सन्तो यक्षा मानुषचारणाः}
{धार्मिकान्पूजयन्तीह न धनाढ्यान्न कामिनः}


\twolineshloka
{सुप्रसन्ना हि ते देवा यत्ते धर्मे रता मतिः}
{धने सुखकला काचिद्धर्मे तु परमं सुखम्}


\chapter{अध्यायः २७८}
\twolineshloka
{युधिष्ठिर उवाच}
{}


\threelineshloka
{बहूनां यज्ञतपसामेकार्थानां पितामह}
{धर्मार्थं न सुखार्थार्थं कथं यज्ञः समाहितः ॥भीष्म उवाच}
{}


\threelineshloka
{अत्र ते वर्तयिष्यामि नारदेनानुकीर्तितम्}
{उञ्छवृत्तेः पुरावृत्तं यज्ञार्थे ब्राह्मणस्य च ॥नारद उवाच}
{}


\twolineshloka
{राष्ट्रे धर्मोत्तरे श्रेष्ठे विदर्भेष्वभवद्द्विजः}
{उञ्छवृत्तिर्ऋषिः कश्चिद्यज्ञं यष्टुं समादधे}


\twolineshloka
{श्यामाकमशनं तत्र सूर्यपर्णी सुवर्चला}
{तिक्तं च विरसं शाकं तपसा स्वादुतां गतम्}


\twolineshloka
{उपगम्य वने पृथ्वीं सर्वभूताविहिंसया}
{अपि मूलफलैरिष्टो यज्ञः स्वर्ग्यः परंतप}


\twolineshloka
{तस्य भार्या व्रतकृशा शुचिः पुष्करमालिनी}
{यज्ञपत्नी समानीता सत्येनानुविधीयते}


\twolineshloka
{सा तु शापपरित्रस्ता तत्स्वभावानुर्तिनी}
{मायूरजीर्णपर्णानां वस्त्रं तस्याश्च वर्णितम्}


\twolineshloka
{अकामया कृतस्तत्र यज्ञो होत्रनुमार्गतः}
{शुकस्य पुनराजातिरवध्यानादधर्मवत्}


\twolineshloka
{तस्मिन्वने समीपस्थो मृगोऽभूत्सहचारिकः}
{वचोभिरब्रवीत्सत्यं त्वयेदं दुष्कृतं कृतम्}


\twolineshloka
{यदि मन्त्राङ्गहीनोऽयं यज्ञो भवति वैकृतः}
{मा भोःप्रक्षिप होत्रे त्वं गच्छ स्वर्गमतन्द्रितः}


\twolineshloka
{ततस्तु यज्ञे सावित्री साक्षात्तं संन्यमन्त्रयत्}
{निमन्त्रयन्ती प्रत्युक्ता न हन्यां सहवासिनम्}


\twolineshloka
{एवमुक्त्वा निवृत्ता सा प्रवृत्ता यज्ञपावकात्}
{किंनु दुश्चरितं यज्ञे दिदृक्षुः सा रसातलम्}


\twolineshloka
{स तु बद्धाञ्जलिं सत्यमयाचद्धरिणः पुनः}
{सत्येन स परिष्वज्य संदिष्टो गम्यतामिति}


\twolineshloka
{ततः स हरिणो गत्वा पदान्यष्टौ न्यवर्तत}
{साधु हिंसय मां सत्य हतो यास्यामि सद्गदितम्}


\twolineshloka
{पश्य ह्यप्सरसो दिव्या मया दत्तेन चक्षुषा}
{विमानानि विचित्राणि गन्धर्वाणां महात्मनाम्}


\twolineshloka
{ततः स सुचिरं दृष्ट्वा स्पृहालग्नेन चक्षुषा}
{मृगमालोक्य हिंसायां स्वर्गवासं समर्थयत्}


\twolineshloka
{स तु धर्मो मृगो भूत्वा बहुवर्षोषितो वने}
{तस्य निष्कृतिमाधत्त न त्वसौ यज्ञसंविधिः}


\twolineshloka
{तस्य तेनानुभावेन मृगहिंसात्मनस्तदा}
{तपो महत्समुच्छिन्नं तस्माद्धिंसा न यज्ञिया}


\twolineshloka
{ततस्तं भगवान्धर्मो यज्ञं याजयत स्वयम्}
{समाधानं च भार्याया लेभे स तपसा परम्}


\twolineshloka
{अहिंसा परो धर्मो हिंसाधर्मस्तथा हितः}
{सत्यं तेऽहं प्रवक्ष्यामि नो धर्मः सत्यवादिनाम्}


\chapter{अध्यायः २७९}
\twolineshloka
{युधिष्ठिर उवाच}
{}


\threelineshloka
{कथं भवति पापात्मा कथं धर्मं करोति वा}
{केन निर्वेदमादत्ते मोक्षं वा केन गच्छति ॥भीष्म उवाच}
{}


\twolineshloka
{विदिताः सर्वधर्मास्ते स्थित्यर्थमनुपृच्छसि}
{शृणु मोक्षं सनिर्वेदं पापं धर्मं च मूलतः}


\twolineshloka
{विज्ञानार्थं हि पञ्चानामिच्छापूर्वं प्रवर्तते}
{प्राप्यतां वर्तते कामो द्वेषो वा भरतर्षभ}


\twolineshloka
{ततस्तदर्थं यतते कर्म चारभते महत्}
{इष्टानां रूपगन्धानामभ्यासं च चिकीर्षति}


\twolineshloka
{ततो रागः प्रभवति द्वेषश्च तदनन्तरम्}
{ततो लोभः प्रभवति मोहश्च तदनन्तरम्}


\twolineshloka
{लोभमोहाभिभूतस्य रागद्वेषान्वितस्य च}
{न धर्मे जायते बुद्धिर्व्याजाद्धर्मं करोति च}


\twolineshloka
{व्याजेन चरते धर्ममर्थं व्याजेन रोचते}
{व्याजेन सिद्ध्यमानेषु धर्मेषु कुरुनन्दन}


\twolineshloka
{तत्रैव कुरुते बुद्धिं ततः पापं चिकीर्षति}
{सुहृद्भिर्वार्यमाणोऽपि पण़्डितैश्चापि भारत}


\twolineshloka
{उत्तरं न्यायसंबद्धं ब्रवीति विधिचोदितम्}
{अधर्मस्त्रिविधस्तस्य वर्धते रागमोहजः}


\twolineshloka
{पापं चिन्तयते कर्म प्रब्रवीति करोति च}
{तस्याधर्मप्रवृत्तस्य दोषान्पश्यन्ति साधवः}


\twolineshloka
{एकशीलाश्च मित्रत्वं भजन्ते पापकर्मिणः}
{स नेह सुखमाप्नोति कुत एव परत्र वै}


\twolineshloka
{एवं भवति पापात्मा धर्मात्मानं तु मे शृणु}
{यथा कुशलधर्मा स कुशलं प्रतिपद्यते}


\twolineshloka
{कुशलेनैव धर्मेण गतिमिष्टां प्रपद्यते}
{य एतान्प्रज्ञया दोषान्पूर्वमेवानुपश्यति}


\twolineshloka
{कुशलस्तु सुखार्थाय साधूंश्चाप्युपसेवते}
{तस्य साधुसमाचारादभ्यासाच्चैव वर्धते}


\twolineshloka
{प्राज्ञो धर्मे च रमते धर्मं चैवोपजीवति}
{सोऽथ धर्मादवाप्तेषु धनेषु कुरुनन्दन}


\twolineshloka
{तस्यैव सिञ्चते मूलं गुणान्पश्यति यत्र वै}
{धर्मात्मा भवति ह्येवं मित्रं च लभते शुभम्}


\twolineshloka
{स मित्रधनलाभात्तु प्रेत्य चेह च नन्दति}
{शब्दे स्पर्शे रसे रूपे तथा गन्धे च भारत}


\twolineshloka
{प्रभुत्वं लभते जन्तुर्धर्मस्यैतत्फलं विदुः}
{स तु धर्मफलं लब्ध्वा न तृष्यति युधिष्ठिर}


\threelineshloka
{धर्मे स्थितानां कौन्तेय सर्वभोगक्रियासु च}
{अतृप्यमाणो निर्वेदमादत्ते ज्ञानचक्षुषा}
{प्रज्ञाचक्षुर्यदा कामे दोषमेवानुपश्यति}


\twolineshloka
{शब्दे स्पर्शे तथा रूपे न च भावयते मनः}
{विमुच्यते तदा कामान्न च धर्मं विमुञ्चति}


\twolineshloka
{सर्वत्यागे च यतते दृष्ट्वा लोकं क्षयात्मकम्}
{ततो मोक्षाय यतते नानुपायादुपायतः}


\twolineshloka
{शनैर्निर्वेदमादत्ते पापं कर्म जहाति च}
{धर्मात्मा चैव भवति मोक्षं च लभते परम्}


\twolineshloka
{एतत्ते कथितं तात यन्मां त्वं परिपृच्छसि}
{पापं धर्मस्तथा मोक्षो निर्वेदश्चैव भारत}


\twolineshloka
{तस्माद्धर्मे प्रवर्तेथाः सर्वावस्थं युधिष्ठिर}
{धर्मे स्थितानां कौन्तेय सिद्धिर्भवति शाश्वती}


\chapter{अध्यायः २८०}
\twolineshloka
{युधिष्ठिर उवाच}
{}


\threelineshloka
{मोक्षः पितामहेनोक्त उपायान्नानुपायतः}
{तमुपायं यथान्यायं श्रोतुमिच्छामि भारत ॥भीष्म उवाच}
{}


\twolineshloka
{त्वय्येवैतन्महाप्राज्ञ युक्तं निपुणदर्शनम्}
{यदुपायेन सर्वार्थं नित्यं मृगयसेऽनघ}


\twolineshloka
{करणे घटस्य या बुद्धिर्घटोत्पत्तौ न सा मता}
{एवं धर्माभ्यपायेषु नान्यद्धर्मेषु कारणम्}


\twolineshloka
{पूर्वे समुद्रे यः पन्थाः स न गच्छति पश्चिमम्}
{एकः पन्था हि मोक्षस्य तन्मे विस्तरतः शृणु}


\twolineshloka
{क्षमया क्रोधमुच्छिन्द्यात्कामं संकल्पवर्जनात्}
{सत्वसंसेवानद्धीरो निद्रामुच्छेत्तुमर्हति}


\twolineshloka
{अप्रमादाद्भयं रक्षेच्छ्वासं क्षेत्रज्ञशीलनात्}
{इच्छां द्वेषं च कामं च धैर्येण विनिवर्तयेत्}


\twolineshloka
{भ्रमं संमोहमावर्तमभ्यासाद्विनिवर्तयेत्}
{निद्रां चाप्रतिभां चैव ज्ञानाभ्यासेन तत्त्ववित्}


\twolineshloka
{उपद्रवांस्तथा रोगान्हितजीर्णमिताशनात्}
{लोभं मोहं च संतोषाद्विषयांस्तत्त्वदर्शनात्}


\twolineshloka
{अनुक्रोशादधर्मं च जयेद्धर्ममवेक्षया}
{आयत्या च जयेदाशामर्थं सङ्गविवर्जनात्}


\twolineshloka
{अनित्यत्वेन च स्नेहं क्षुधं योगेन पण्डितः}
{कारुण्येनात्मनो मानं तृष्णां च परितोषतः}


\twolineshloka
{उत्थानेन जयेत्तन्द्रीं वितर्कं निश्चयाज्जयेत्}
{मौनेन बहुभाषां च शौर्येण च भयं जयेत्}


\twolineshloka
{यच्छेद्वाङ्भनसी बुद्ध्या तां यच्छेज्ज्ञानचक्षुषा}
{ज्ञानमात्मा महान्यच्छेत्तं यच्छेज्ज्ञानमात्मनः}


\twolineshloka
{तदेतदुपशान्तेन बोद्धव्यं शुचिकर्मणा}
{योगदोषान्समुच्छिद्यात्पञ्च यान्कवयो विदुः}


\twolineshloka
{कामं क्रोधं च लोभं च भयं स्वप्नं च पञ्चमम्}
{परित्यज्य निषेवेत तथेमान्योगसाधनान्}


\twolineshloka
{ध्यानमध्ययनं दानं सत्यं ह्रीरार्जवं क्षमा}
{शौचमाहारतः शुद्धिरिन्द्रियाणां च संयमः}


\twolineshloka
{एतैर्विवर्धते तेजः पाप्मानमपहन्ति च}
{सिध्यन्ति चास्य संकल्पा विज्ञानं च प्रवर्तते}


\twolineshloka
{धूतपापः स तेजस्वी लध्वाहारो जितेन्द्रियः}
{कामक्रोधौ वशे कृत्वा निनीषेद्ब्रह्मणः पदम्}


\twolineshloka
{अमूढत्वमसङ्गित्वं कामक्रोधविवर्जनम्}
{अदैन्यमनुदीर्णत्वमनुद्वेगो व्यवस्थितिः}


\twolineshloka
{एष मार्गो हि मोक्षस्य प्रसन्नो विमलः शुचिः}
{तथा वाक्कायमनसां नियमः कामतोऽन्यथा}


\chapter{अध्यायः २८१}
\twolineshloka
{भीष्म उवाच}
{}


\twolineshloka
{अत्रैवोदाहरन्तीममितिहासं पुरातनम्}
{नारदस्य च संवादं देवलस्यासितस्य च}


\threelineshloka
{आसीनं देवलं वृद्धं बुद्ध्वा बुद्धिमतां वरम्}
{नारदः परिपप्रच्छ भूतानां प्रभवाप्ययम् ॥नारद उवाच}
{}


\threelineshloka
{कुतः सृष्टमिदं विश्वं ब्रह्मन्स्थावरजङ्गमम्}
{प्रलये च कमभ्येति तद्भवान्प्रब्रवीतु मे ॥असित उवाच}
{}


\twolineshloka
{येभ्यः सृजति भूतानि कालो भावप्रचोदितः}
{महाभूतानि पञ्चेति तान्याहुर्भूतचिन्तकाः}


\twolineshloka
{तेभ्यः सृजति भूतानि काल आत्मप्रचोदितः}
{एतेभ्यो यः परं ब्रूयादसद्ब्रूयादसंशयम्}


\twolineshloka
{विद्धि नारद पञ्चैताञ्शाश्वतानचलान्ध्रुवान्}
{महतस्तेजसो राशीन्कालषष्ठान्स्वभावतः}


\twolineshloka
{आपश्चैवान्तरिक्षं च पृथिवी वायुपावकौ}
{असिद्धिः परमेतेभ्यो भूतेभ्यो मुक्तसंशयम्}


\twolineshloka
{नोपपत्त्या न वा युक्त्या त्वसद्ब्रूयादसंशयम्}
{वेत्थैतानभिनिर्वृत्तान्षडेते यस्य राशयः}


\twolineshloka
{पञ्चैव तानि कालश्च भावाभावौ च केवलौ}
{अष्टौ भूतानि भूतानां शाश्वतानि भवाव्ययौ}


\twolineshloka
{अभावभावितेष्वेव तेभ्यश्च प्रभवन्त्यपि}
{विनष्टोऽप्यनुतान्येव जन्तुर्भवति पञ्चधा}


\twolineshloka
{तस्य भूमिमयो देहः श्रोत्रमाकाशसंभवम्}
{सूर्याच्चक्षुरसुर्वायोरद्भ्यस्तु खलु शोणितम्}


\twolineshloka
{चक्षुषी नासिकाकर्णौ त्वक् जिह्वेति च प़ञ्चमी}
{इन्द्रियाणीन्द्रियार्थानां ज्ञानानि कवयो विदुः}


\twolineshloka
{दर्शनं श्रवणं घ्राणं स्पर्शनं रसनं तथा}
{उपपत्त्या गुणान्विद्धि पञ्च पञ्चसु धातुषु}


\twolineshloka
{रूपं गन्धो रसः स्पर्शः शब्दश्चैवाथ तद्गुणाः}
{इन्द्रियैरुपलभ्यन्ते पञ्चधा पञ्च पञ्चभिः}


\twolineshloka
{रूपं गन्धं रसं स्पर्शं शब्दं चैवाथ तद्गुणान्}
{इन्द्रियाणि न बुध्यन्ते क्षेत्रज्ञस्तैस्तु बुध्यते}


\twolineshloka
{चित्तमिन्द्रियसंघातात्परं तस्मात्परं मनः}
{मनसस्तु परा बुद्धिः क्षेत्रज्ञो बुद्धितः परः}


\threelineshloka
{पूर्वं चेतयते जन्तुरिन्द्रियैर्विषयान्पृथक्}
{विचार्य मनसा पश्चादथ बुद्ध्या व्यवस्यति}
{इन्द्रियैरुपसृष्टार्थान्मत्वा यस्त्वध्यवस्यति}


\twolineshloka
{चित्तमिन्द्रियसंघातं मनो बुद्धिस्तथाऽष्टमी}
{अष्टौ ज्ञानेन्द्रियाण्याहुरेतान्यध्यात्मचिन्तकाः}


\twolineshloka
{पाणिं पादं च पायुं च मेहनं पञ्चमं मुखम्}
{इति संशब्द्यमानानि शृणु कर्मेन्द्रियाण्यपि}


\twolineshloka
{जल्पनाभ्यवहारार्थं मुखमिन्द्रियमुच्यते}
{गमनेन्द्रियं तथा पादौ कर्मणः करणे करौ}


\twolineshloka
{पायूपस्थं विसर्गार्थमिन्द्रिये तुल्यकर्मणी}
{विसर्गे च पुरीषस्य विसर्गे चापि कामिके}


\twolineshloka
{मनः षष्ठान्यथैतानि वाचा सम्यग्यथागमम्}
{ज्ञानचेष्टेन्द्रियगुणाः सर्वेषां शब्दिता मया}


\twolineshloka
{इन्द्रियाणां स्वकर्मभ्यः श्रमादुपरमो यदा}
{भवतीन्द्रियसंन्यासादथ स्वपिति वै नरः}


\twolineshloka
{इन्द्रियाणां व्युपरमे मनोऽव्युपरतं यदि}
{सेवते विषयानेव तं विद्यात्स्वप्नदर्शनम्}


\twolineshloka
{सात्विकाश्चैव ये भावास्तथा राजसतामसाः}
{कर्मयुक्ताः प्रशंसन्ति सात्विकान्नेतरांस्तथा}


\twolineshloka
{आनन्दः कर्मणां सिद्धिः प्रतिपत्तिः परा गतिः}
{सात्विकस्य निमित्तानि भावान्संश्रयसे स्मृतिः}


\twolineshloka
{जन्तुष्वेकतमेष्वेवं भावं यो वा समास्थितः}
{भावयोरीप्सितं नित्यं प्रत्यक्षं गमनं द्वयोः}


\twolineshloka
{इन्द्रियाणि च भावाश्च गुणाः सप्तदश स्मृताः}
{तेषामष्टादशो देही यः शरीरे स शाश्वतः}


\twolineshloka
{अथवा सशरीरास्ते गुणाः सर्वे शरीरिणाम्}
{संश्रितास्तद्वियोगे हि सशरीरा न सन्ति ते}


\threelineshloka
{अथवा संविभागेन शरीरं पाञ्चभौतिकम्}
{एकश्च दश चाष्टौ च गुणाः सह शरीरिणाम्}
{ऊष्मणा सह विशो वा संघातः पाञ्चभौतिकः}


\twolineshloka
{महान्संधारयत्येतच्छरीरं वायुना सह}
{सत्यप्रभावयुक्तस्य निमित्तं देहभेदने}


\threelineshloka
{तथैवोत्पद्यते किंचित्पञ्चत्वं गच्छते तथा}
{पुण्यपापविनाशान्ते पुण्यपापसमीरितः}
{देहं विशति कालेन ततोऽयं कर्मसंभवम्}


\twolineshloka
{हित्वाहित्वा ह्ययं प्रैति देहाद्देहं कृताश्रयः}
{कालसंचोदितः क्षेत्री विशीर्णाद्वा गृहाद्गृहम्}


\twolineshloka
{तं तु नैवानुतप्यन्ते प्राज्ञा निश्चितनिश्चयाः}
{कृपणास्त्वनुतप्यन्ते जनाः संबन्धिमानिनः}


\twolineshloka
{न ह्ययं कस्यचित्कश्चिन्नास्य कश्चन विद्यते}
{भवत्येको ह्ययं नित्यं शरीरे सुखदुःखकृत्}


\twolineshloka
{नैव संजायते जन्तुर्न च जातु विपद्यते}
{याति देहमयं मुक्त्वा कदाचित्परमां गतिम्}


\twolineshloka
{पुण्यपापमयं देहं क्षपयन्कर्मसंक्षयात्}
{क्षीणदेहः पुनर्देही ब्रह्मत्वमुपगच्छति}


\twolineshloka
{पुण्यपापक्षयार्थं हि साङ्ख्यज्ञानं विधीयते}
{तत्क्षये हृदि पश्यन्ति ब्रह्मभावे परां गतिम्}


\chapter{अध्यायः २८२}
\twolineshloka
{युधिष्ठिर उवाच}
{}


\twolineshloka
{भ्रातरः पितरः पौत्रा ज्ञातयः सुहृदः सुताः}
{अर्थहेतोर्हताः क्रूरैरस्माभिः पापबुद्धिभिः}


\threelineshloka
{येयमर्थोद्भवा तृष्णा कथमेतां पितामह}
{निवर्तयेयं पापानि तृष्णया कारिता वयम् ॥भीष्म उवाच}
{}


\twolineshloka
{अत्राप्युदाहरन्तीममितिहासं पुरातनम्}
{गीतं विदेहराजेन माण्डव्यायानुपृच्छते}


\twolineshloka
{सुसुखं बत जीवामि यस्य मे नास्ति किंचन}
{मिथिलायां प्रदीप्तायां न मे दह्यति किंचन}


\twolineshloka
{अर्थाः खलु समृद्धा हि गाढं दुःखं विजानताम्}
{असमृद्धास्त्वपि सदा मोहयन्त्यविचक्षणान्}


\twolineshloka
{यच्च कामसुखं लोके यच्च दिव्यं महत्सुखम्}
{तृष्णाक्षयसुखस्यैते नार्हतः षोडशीं कलाम्}


\twolineshloka
{यथैव शृङ्गं गोः काले वर्धमानस्य वर्धते}
{तथैव तृष्णा वित्तेन वर्धमानेन वर्धते}


\twolineshloka
{किंचिदेव ममत्वेन यदा भवति कल्पितम्}
{तदेव परितापाय नाशे संपद्यते पुनः}


\twolineshloka
{न कामाननुरुध्येत दुःखं कामेषु वै रतिः}
{प्राप्यार्थमुपयुञ्जीत धर्मं कामान्विवर्जयेत्}


\twolineshloka
{विद्वान्सर्वेषु भूतेषु व्याघ्रमांसोपमो भवेत्}
{कृतकृत्यो विशुद्धात्मा सर्वं ज्यजति वै स्वयम्}


\twolineshloka
{उभे सत्यानृते त्यक्त्वा शोकानन्दौ प्रियाप्रिये}
{भयाभये च संत्यज्य भव शान्तो निरामयः}


\twolineshloka
{या दुस्त्यजा दुर्मतिभिर्या न जीर्यति जीर्यतः}
{योसौ प्राणान्तिको रोगस्तां तृष्णां त्यजतः सुखं}


\twolineshloka
{चारित्रमात्मनः पश्यंश्चन्द्रशुद्धमनामयम्}
{धर्मात्मा लभते कीर्ति प्रेत्य चेह यथासुखम्}


\twolineshloka
{राज्ञस्तद्वचनं श्रुत्वा प्रीतिमानभवद्द्विजः}
{पूजयित्वा च तद्वाक्यं माण्डव्यो मोक्षमाश्रितः}


\chapter{अध्यायः २८३}
\twolineshloka
{* युधिष्ठिर उवाच}
{}


\threelineshloka
{अतिक्रामति कालेऽस्मिन्सर्वभूतभयावहे}
{किं श्रेयः प्रतिपद्येत तन्मे ब्रूहि पितामह ॥भीष्म उवाच}
{}


\twolineshloka
{अत्राप्युदाहरन्तीममितिहासं पुरातनम्}
{पितुः पुत्रेण संवादं तं निबोध युधिष्ठिर}


\twolineshloka
{द्विजातेः कस्यजित्पार्थ स्वाध्यायनिरतस्य वै}
{पुत्रो बभूव मेधावी मेधावी नाम नामतः}


\threelineshloka
{सोऽब्रवीत्पितरं पुत्रः स्वाध्यायकरणे रतम्}
{मोक्षधर्मेष्वकुशलं मोक्षधर्मविचक्षणः ॥पुत्र उवाच}
{}


\threelineshloka
{धीरः किंस्वित्तात कुर्यात्प्रजानन्क्षिप्रं ह्यायुर्भ्रश्यते मानवानाम्}
{पितस्तथाऽऽख्याहि यथार्थयोगंममानुपूर्व्या येन धर्मं चरेयम् ॥पितोवाच}
{}


\threelineshloka
{अधीत्य वेदान्ब्रह्मचर्येषु पुत्रपुत्रानिच्छेत्पावनार्थं पितृणाम्}
{अग्नीनाधाय विधिवच्चेष्टयज्ञोवनं प्रविश्याथ मुनिर्बुभूषेत् ॥पुत्र उवाच}
{}


\threelineshloka
{एवमभ्याहते लोके सर्वतः परिवारिते}
{अमोधासु पतन्तीषु किं धीर इव भाषसे ॥पितोवाच}
{}


\threelineshloka
{कथमभ्याहतो लोकः केन वा परिवारितः}
{अमोघाः काः पतन्तीह किंनु भीषयसीव माम् ॥पुत्र उवाच}
{}


\twolineshloka
{मृत्युनाऽऽभ्याहतो लोको जस्या परिवारितः}
{अहोरात्राः पतन्तीमे तच्च कस्मान्न बुध्यसे}


\twolineshloka
{यदाहमेव जानामि न मृत्युस्तिष्ठतीति ह}
{सोहं कथं प्रतीक्षिप्ये ज्ञानेनापिहितश्चरन्}


\twolineshloka
{रात्र्यांरात्र्यां व्यतीतायामायुरल्पतरं यदा}
{गाधोदके मत्स्य इव सुखं विन्देत कस्तदा}


\twolineshloka
{`यामेकरात्रिं प्रथमां गर्भो विशति मातरम्}
{तामेव रात्रिं प्रस्वाप्य मरणाय विवर्तकः ॥'}


\twolineshloka
{पुष्पाणीव विचिन्वन्तमन्यत्र गतमानसम्}
{अनवाप्तेषु कामेषु मृत्युरभ्येति गानवम्}


\twolineshloka
{श्वः कार्यमद्य कुर्वीत पूर्वाङ्गे चापराहिकम्}
{न हि प्रतीक्षते मृत्युः कृतं वाऽस्य न वा कृतम्}


\twolineshloka
{अद्यैव कुरु यच्छ्रेयो मा त्वां कालोऽत्यगान्महान्}
{को हि जानाति कस्याद्य मृत्युकालो मविष्यति}


\twolineshloka
{अकृतेष्वेव कार्येषु मृत्युर्वै संप्रकर्षति}
{युवैव धर्मशीलः स्यादनिमित्तं हि जीवितम्}


\twolineshloka
{कृते धर्म भवेत्प्रीतिरिह प्रेत्य च शाश्वती}
{मोहेन हि समाविष्टः पुत्रदारार्तमुद्यतः}


\twolineshloka
{कृत्वा कार्यमकार्यं वा तुष्टिमेषां प्रयच्छति}
{तं पुत्रपशुसंपन्नं व्याभक्तमनसं नरम्}


\twolineshloka
{सप्तं व्यायं महौघो वा मृत्युरादाय गच्छति}
{संविन्वानकमेवैनं कामानामवितृप्तकम्}


\twolineshloka
{वृकीवोरपमासाद्य मुत्युरादाय गच्छति}
{इदं कृतमिदं कार्यमिदमन्यत्कृताकृतम्}


\twolineshloka
{एवमीहासमायुक्तं मृत्युरादाय गच्छति}
{कृतानां फलमप्राप्तं कार्याणां कर्मसङ्गिनाम्}


\twolineshloka
{क्षेत्रापणगृहासक्तं मृत्युरादाय गच्छति}
{दुर्बलं बलवन्तं च प्राज्ञं शूरं जडं कविम्}


\twolineshloka
{अप्राप्तसर्वकामार्थं मृत्युरादाय गच्छति}
{मृत्युर्जरा च व्याधिश्चदुःखं चानेककारणम्}


\twolineshloka
{असंत्याज्यं यदा मर्त्यैः किं स्वस्थ इव तिष्ठति}
{जातमेवान्तकोऽन्ताय जरा चाभ्येति देहिनम्}


\twolineshloka
{अनुषक्ता द्वयेनैते भावाः स्थावरजङ्गमाः}
{न मृत्युसेनामायान्तीं जातु कश्चित्प्रबाधते}


\twolineshloka
{बलात्सत्यमृते त्वेकं सत्ये ह्यमृतमाश्रितम्}
{मृत्योर्वा गृहमेतद्वै या ग्रामे वसतो रतिः}


\twolineshloka
{देवानामेषु वै गोष्ठो यदरण्यमिति श्रुतिः}
{निबन्धनी रज्जुरेषा या ग्रामे वसतो रतिः}


\twolineshloka
{छित्त्वैनां सुकृतो यान्ति नैनां छिन्दन्ति दुष्कृतः}
{यो न हिंसति सत्वानि मनोवाक्कर्महेतुभिः}


\twolineshloka
{जीवितार्थापनयनैः प्राणिभिर्न स बध्यते}
{तस्मात्सत्यव्रताचारः सत्यव्रतपरायणः}


\twolineshloka
{सत्यकामः समो दान्ताः सत्येनैवान्तकं जयेत्}
{अमृतं चैव मृत्युश्च द्वयं देहे प्रतिष्ठितम्}


\twolineshloka
{मृत्युरापद्यते मोहात्सत्येनापद्यतेऽमृतम्}
{सोहं सत्यमहिंसाथीं कामक्रोधबहिष्कृतः}


\twolineshloka
{समाश्रित्य सुखं क्षेमी मृत्युं हास्याम्यमृत्युवत्}
{शान्तियज्ञरतो दान्तो ब्रह्मयज्ञे स्थितो मुनिः}


\twolineshloka
{वाङ्भनः कर्मयज्ञश्च भविष्याम्युदगायने}
{पशुयज्ञैः कथं हिंस्रैर्मादृशो यष्टुमर्हति}


\twolineshloka
{अन्तवद्भिरुत प्राज्ञः क्षत्रयज्ञैः पिशाचवत्}
{आत्मन्येवात्मना जात आत्मनिष्ठोऽप्रजः पितः}


\twolineshloka
{आत्मयज्ञो भविष्यामि न मां तारयति प्रजा}
{यस्य वाङ्भनसी स्यातां सम्यक्प्रणिहिते सदा}


\twolineshloka
{तपस्त्यागश्च योगश्च स तैः सर्वमवाप्नुयात्}
{नास्ति विद्यासमं चक्षुर्नास्ति विद्यासमं फलम्}


% Check verse!
नास्ति रागसमं दुःखं नास्ति त्यागसमं सुखम्
\twolineshloka
{नैतादृशं ब्राह्मणस्यास्ति वित्तंयथैकता समता सत्यता च}
{शीले स्थितिर्दण्डविधानमार्जवंततस्ततश्चोपरमः क्रियाभ्यः}


\threelineshloka
{किं ते धनैर्बान्धवैर्वाऽपि किं तेकिं ते दारैब्राह्मण यो मरिष्यसि}
{आत्मानमन्विच्छ गृहा प्रविष्टंपितामहास्ते क्व गताः पिता च ॥भीष्म उवाच}
{}


\twolineshloka
{पुत्रस्यैतद्वचः श्रुत्वा तथाकार्षीत्पिता नृप}
{तथा त्वमपि राजेन्द्र सत्यधर्मपरो भव}


\chapter{अध्यायः २८४}
\twolineshloka
{युधिष्ठिर उवाच}
{}


\threelineshloka
{किंशीलः किंसमाचारः किंविद्यः किंपरायणः}
{प्राप्नोति ब्रह्मणः स्थानं यत्परं प्रकृतेर्ध्रुवम् ॥भीष्म उवाच}
{}


\twolineshloka
{मोक्षधर्मेषु निरतो लघ्वाहारो जितेन्द्रियः}
{प्राप्नोति परमं स्थानं यत्परं प्रकृतेर्ध्रुवम्}


\twolineshloka
{`अत्राप्युदाहरन्तीममितिहासं पुरातनम्}
{हारीतेन पुरा गीतं तं निबोध युधिष्ठिर}


\twolineshloka
{स्वगृहादभिनिःसृत्य लाभेऽलाभे समो मुनिः}
{समुपोढेषु कामेषु निरपेक्षः परिव्रजेत्}


\twolineshloka
{न चक्षुषा न मनसा न वाचा दूषयेत्परम्}
{न प्रत्यक्षं परोक्षं वा दूषणं व्याहरेत्क्वचित्}


\twolineshloka
{न हिंस्यात्सर्वभूतानि मैत्रायणगतिश्चरेत्}
{नेदं जीवितमासाद्य वैरं कुर्वीत केनचित्}


\twolineshloka
{अतिवादांस्तितिक्षेत नाभिमन्येत कंचन}
{क्रोध्यमानः प्रियं ब्रूयादाक्रुष्टः कुशलं वदेत्}


\twolineshloka
{प्रदक्षिणं च सव्यं च ग्राममध्ये च नाचरेत्}
{भैक्षचर्यामनापन्नो न गच्छेत्पूर्वकेतनम्}


\twolineshloka
{अवकीर्णः सुगुप्तश्च न वाचाऽप्यप्रियं चरेत्}
{मृदुः स्यादप्रतीकारो विस्रब्धः स्यादरोषणः}


\twolineshloka
{विधूमे न्यस्तमुसले व्यङ्गारे भुक्तवज्जने}
{अतीते पात्रसंचारे भिक्षां लिप्सेत वै मुनिः}


\twolineshloka
{प्राणयात्रिकमात्रः स्यान्मात्रालाभेष्वनादृतः}
{अलाभे न विहन्येत लाभश्चैनं न हर्षयेत्}


\twolineshloka
{लाभं साधारणां नेच्छेन्न भुञ्जीताभिपूजितः}
{अभिपूजितलाभं हि जुगुप्सेतैव तादृशः}


\twolineshloka
{न चान्नदोषान्निन्देत न गुणानभिपूजयेत्}
{शय्यासने विविक्ते च नित्यमेवाभिपूजयेत्}


\twolineshloka
{शून्यागारं वृक्षमूलमरण्यमथवा गुहाम्}
{अज्ञातचर्यां गत्वाऽन्यां ततोऽन्यत्रैव संविशेत्}


\twolineshloka
{अनुरोधविरोधाभ्यां समः स्यादचलो ध्रुवः}
{सुकृतं दुष्कृतं चोभे नानुरुध्येत कर्मणि}


\twolineshloka
{नित्यतृप्तः सुसंतृष्टः प्रसन्नवदनेन्द्रियः}
{विभीर्जप्यपरो मौनी वैराग्यं समुपाश्रितः}


\threelineshloka
{अभ्यस्तं भौतिकं पश्यन्भूतानामागतिं गतिम्}
{विस्मितः सर्वदर्शी च पक्वापक्वेन वर्तयन्}
{आत्मारामः प्रशान्तात्मा लघ्वाहारो जितेन्द्रियः}


\twolineshloka
{वाचो वेगं मनसः क्रोधवेगंहिंसावेगमुदरोपस्थवेगम्}
{एतान्वेगान्विनयेद्वै तपस्वीनिन्दा चास्य हृदयं नोपहन्यात्}


\twolineshloka
{मध्यस्थ एव तिष्ठेत प्रशंसानिन्दयोः समः}
{एतत्पवित्रं परमं परिव्राजक आश्रयेत्}


\twolineshloka
{महात्मा सर्वतो दान्तः सर्वत्रैवानपाश्रितः}
{अपूर्वचारकः सौम्यो ह्यनिकेतः समाहितः}


\twolineshloka
{वानप्रस्थगृहस्थाभ्यां न संसृज्येत कर्हिचित्}
{अज्ञातलिप्सं लिप्सेत न चैनं हर्ष आविशेत्}


\twolineshloka
{विजानतां मोक्ष एष श्रमः स्यादविजानताम्}
{मोक्षयानमिदं कृत्स्नं विदुषां हारितोऽब्रवीत्}


\twolineshloka
{अभयं सर्वभूतेभ्यो दत्त्वा यः प्रव्रजेद्गृहात्}
{लोकास्तेजोमयास्तस्य तथाऽनन्त्याय कल्पते}


\chapter{अध्यायः २८५}
\twolineshloka
{युधिष्ठिर उवाच}
{}


\twolineshloka
{धन्याधन्या इति जनाः सर्वेऽस्मान्प्रवदन्त्युत}
{न दुःखिततरः कश्चित्पुमानस्माभिरस्ति ह}


\twolineshloka
{लोकसंभावितैर्दुःखं यत्प्राप्तं कुरुसत्तम}
{प्राप्य जातिं मनुष्येषु देवैरपि पितामह}


\twolineshloka
{कदा वयं करिष्यामः संन्यासं दुःखभेषजम्}
{दुःखमेतच्छरीराणां धारणं कुरुसत्तम}


\twolineshloka
{विमुक्ताः सप्तदशभिर्हेतुभूतैश्च पञ्चभिः}
{इन्द्रियार्थैर्गुणैश्चैव अष्टाभिश्च पितामह}


\threelineshloka
{न गच्छन्ति पुनर्भावं मुनयः संशितव्रताः}
{कदा वयं गमिष्यामो राज्यं हित्वा परंतप ॥भीष्म उवाच}
{}


\twolineshloka
{नास्त्यनन्तं महाराज सर्वं सङ्ख्यानगोचरम्}
{पुनर्भावोपि संख्यातो नास्ति किंचिदिहाचलम्}


\twolineshloka
{न चापि गम्यते राजन्नैष दोषः प्रसङ्गतः}
{उद्योगादेव धर्मज्ञाः कालेनैव गमिष्यथ}


\twolineshloka
{नेशेऽयं सततं देही नृपते पुण्यपापयोः}
{तत एव समुत्थेन तमसा रुध्यतेऽपि च}


\twolineshloka
{यथाञ्जनमयो वायुः पुनर्मानःशिलं रजः}
{अनुप्रविश्य तद्वर्णो दृश्यते रञ्जयन्दिशः}


\twolineshloka
{तथा कर्मफलैर्देही रञ्जितस्तमसा वृतः}
{विवर्णो वर्णमाश्रित्य देहेषु परिवर्तते}


\twolineshloka
{ज्ञानेन हि यदा जन्तुरज्ञानप्रभवं तमः}
{व्यपोहति तदा ब्रह्म प्रकाशेत सनातनम्}


\twolineshloka
{अयत्नसाध्यं मुनयो वदन्तिचे चापि मुक्तास्तदुपासितव्याः}
{त्वया च लोकेन च सामरेणतस्मान्न शाम्यन्ति महर्षिसङ्घाः}


\twolineshloka
{अस्मिन्नर्थे पुरा गीतं शृणुष्वैकमना नृप}
{यथा दैत्येन वृत्रेण भ्रष्टैश्वर्येण चेष्टितम्}


\twolineshloka
{निर्जितेनासहायेन हृतराज्येन भारत}
{अशोचता शत्रुमध्ये बुद्धिमास्थाय केवलाम्}


\threelineshloka
{भ्रष्टैश्वर्यं पुरा वृत्रमुशना वाक्यमब्रवीत्}
{कच्चित्पराजितस्याद्य न व्यथा तेऽस्ति दानव ॥वृत्र उवाच}
{}


\twolineshloka
{सत्येन तपसा चैव विदित्वा संक्षयं ह्यहम्}
{न शोचामि न हृष्यामि भूतानामागतिं गतिम्}


\twolineshloka
{कालसंचोदिता जीवा मज्जन्ति नरकेऽवशाः}
{परिहृष्टानि सर्वाणि दिव्यान्याहुर्मनीषिणः}


\twolineshloka
{क्षपयित्वा तु तं कालं गणितं कालचोदिताः}
{सावशेषेण कालेन संधावन्ति पुनःपुनः}


\twolineshloka
{तिर्यग्योनिसहस्राणि गत्वा नरकमेव च}
{निर्गच्छन्त्यवशा जीवाः कालबन्धनबन्धनाः}


\twolineshloka
{एवं संसरमाणानि ह्यहं भूतानि दृष्टवान्}
{यथा कर्म तथा लाभ इति शास्त्रनिदर्शनम्}


\twolineshloka
{तिर्यग्गच्छन्ति नरकं मानुष्यं दैवमेव च}
{सुखदुःखे प्रिये द्वेष्ये चरित्वा पूर्वमेव च}


\twolineshloka
{कृतान्तविधिसंयुक्तः सर्वो लोकः प्रपद्यते}
{गतं गच्छन्ति चाध्वानं सर्वभूतानि सर्वदा}


\fourlineindentedshloka
{कालसङ्ख्यानसङ्ख्येयं सृष्टिस्थितिपरायणम्}
{तं भाषमाणं भगवानुशना प्रत्यभाषत}
{इमान्दुष्टप्रलापांस्त्वं तात कस्मात्प्रभाषते ॥वृत्र उवाच}
{}


\twolineshloka
{प्रत्यक्षमेतद्भवतस्तथाऽन्येषां मनीषिणाम्}
{मया यज्जयलुब्धेन पुरा तप्तं महत्तपः}


\twolineshloka
{गन्धानादाय भूतानां रसांश्च विविधानपि}
{अवर्धं त्रीन्समाक्रम्य लोकान्वै स्वेन तेजसा}


\twolineshloka
{ज्वालामालापरिक्षिप्तो वैहायसगतिस्तथा}
{अजेयः सर्वभूतानामासं नित्यमपेतभीः}


\twolineshloka
{ऐश्वर्यं तपसा प्राप्तं भ्रष्टं तच्च स्वकर्मभिः}
{धृतिमास्थाय भगवन्न शोचामि ततस्त्वहम्}


\twolineshloka
{युयुत्सता महेन्द्रेण पुरा सार्धं महात्मना}
{ततो मे भगवान्दृष्टो हरिर्नारायणः प्रभुः}


\twolineshloka
{वैकुण्ठः पुरुषोऽनन्तः शुक्लो विष्णुः सनातनः}
{मुञ्जकेशो हरिश्मश्रुः सर्वभूतपितामहः}


\twolineshloka
{नूनं तु तस्य तपसः सावशेषं ममास्ति वै}
{यदहं प्रष्टुमिच्छामि भवन्तं कर्मणः फलम्}


\twolineshloka
{ऐश्वर्यं वै महद्ब्रह्मन्वर्णे कस्मिन्प्रतिष्ठितम्}
{निवर्तते चापि पुनः कथमैश्वर्यमुत्तमम्}


\twolineshloka
{भवन्ति कस्माद्भूतानि प्रवर्तन्ते यथा पुनः}
{किं वा फलं परं प्राप्य जीवस्तिष्ठति शाश्वतः}


\twolineshloka
{केन वा कर्मणा शक्यमथ ज्ञानेन केन वा}
{ब्रह्मर्षे तत्फलं प्राप्तुं तन्मे व्याख्यातुमर्हसि}


\twolineshloka
{इतीदमुक्तः स मुनिस्तदानींप्रत्याह यत्तच्छृणु राजसिंह}
{मयोच्यमानं पुरुषर्षभ त्वमनन्यचित्तः सह सोदरीयैः}


\chapter{अध्यायः २८६}
\twolineshloka
{उशनोवाच}
{}


\twolineshloka
{नमस्तस्मै भगवते देवाय प्रभविष्णवे}
{यस्य पृथ्वी तलं तात साकाशं बाहुगोचरः}


\threelineshloka
{मूर्धा यस्य त्वनन्तं च स्थानं दानवसत्तम}
{तस्याहं ते प्रवक्ष्यामि विष्णोर्माहात्म्यमुत्तमम् ॥भीष्म उवाच}
{}


\twolineshloka
{तयोः संवदतोरेवमाजगाम महामुनिः}
{सनत्कुमारो धमार्त्मा संशयच्छेदनाय वै}


\twolineshloka
{स पूजितोऽसुरेन्द्रेण मुनिनोशनसा तथा}
{निषसादासने राजन्महार्हे मुनिपुङ्गवः}


\twolineshloka
{तमासीनं महाप्रज्ञमुशना वाक्यमब्रवीत्}
{ब्रूह्यस्मै दानवेन्दाय विष्णोर्माहात्म्यमुत्तमम्}


\twolineshloka
{सनत्कुमारस्तु वचः श्रुत्वा प्राह वचोऽर्थवत्}
{विष्णोर्माहात्म्यसंयुक्तं दानवेन्द्राय धीमते}


\twolineshloka
{शृणु सर्वमिदं दैत्य विष्णोर्माहात्म्यमुत्तमम्}
{विष्णौ जगत्स्थितं सर्वमिति विद्धि परंतप}


\threelineshloka
{अस्मिन्गच्छन्ति विलयमस्माच्च प्रभवन्त्युत}
{अवत्येष महाबाहुर्भूतग्रामं चराचरम्}
{एष चाक्षिपते काले काले च सृजते पुनः}


\twolineshloka
{नैष दानव ते शक्यस्तपसा नैव चेज्यया}
{संप्राप्तुमिन्द्रियाणां तु संयमेनैव शक्यते}


\twolineshloka
{बाह्ये चाभ्यन्तरे चैव कर्मणा मनसि स्थितः}
{निर्मलीकुरुते बुद्ध्या सोऽमुत्रानन्त्यमश्नुते}


\twolineshloka
{यथा हिरण्यकर्ता वै रूप्यमग्नौ विशोधयेत्}
{बहुशोऽतिप्रयत्नेन महताऽऽत्मकृतेन ह}


\twolineshloka
{तद्वज्जातिशतैर्जीवः शुद्ध्यतेऽल्पेन कर्मणा}
{यत्नेन महता चैयवाप्येकजातौ विशुद्ध्यते}


\twolineshloka
{लीलयाऽल्पं यथा गात्रात्प्रमृज्यादात्मनो रजः}
{बहुयत्नेन महता दोषनिर्हरणं तथा}


\twolineshloka
{यथा चाल्पेन माल्येन वासितं तिलसर्षपम्}
{न मुञ्चति स्वकं गन्धं तथा सूक्ष्मस्य दर्शनम्}


\twolineshloka
{तदेव बहुभिर्माल्यैर्वास्यमानं पुनः पुनः}
{विमुच्य तं स्वकं गन्धं माल्यगन्धेऽवतिष्ठते}


\twolineshloka
{एवं जातिशतैर्युक्तो गुणैरेव प्रसङ्गिषु}
{बुद्ध्या निवर्तते दोषो यत्नेनाभ्यासजेन ह}


\twolineshloka
{कर्मणा स्वेन रक्तानि विरक्तानि च दानव}
{यथा कर्मविशेषांश्च प्राप्नुवन्ति तथा शृणु}


\twolineshloka
{यथावत्संप्रवर्तन्ते यस्मिंस्तिष्ठति चानिशम्}
{तत्तेऽनुपूर्व्या व्याख्यास्ये तदिहैकमनाः शृणु}


\twolineshloka
{अनादिनिधनः श्रीमान्हरिर्नारायणः प्रभुः}
{देवः सृजति भूतानि स्थावराणि चराणि च}


\twolineshloka
{एष सर्वेषु भूतेषु क्षरश्चाक्षर एव च}
{एकादश विकारात्मा जगत्पिबति रश्मिभिः}


\twolineshloka
{पादौ तस्य महीं विद्धि मूर्धानं दिवमेव च}
{बाहवस्तु दिशो दैत्य श्रोत्रमाकाशमेव च}


\twolineshloka
{तस्य तेजोमयः सूर्यो मनश्चन्द्रमसि स्थितम्}
{बुद्धिर्ज्ञानगता नित्यं रसस्त्वप्सु प्रवर्तते}


\threelineshloka
{भ्रुवोरनन्तरास्तस्य ग्रहा दानवसत्तम}
{नक्षत्रचक्रं नेत्रं च आस्यमग्निं च दानव}
{तं विश्वभूतं विश्वादिं परमं विद्धि चेश्वरम्}


\twolineshloka
{रजस्तमश्च सत्वं च विद्धि नारायणात्मकम्}
{सोश्रमाणां मुखं तात कर्मणस्तत्फलं विदुः}


\twolineshloka
{अकर्मणः फलं चैव स एव परमोऽव्ययः}
{छन्दांसि यस्य रोमाणि ह्यक्षरं च सरस्वती}


\twolineshloka
{बह्वाश्रयो बहुमुखो धर्मो हृदि समाश्रितः}
{स ब्रह्मपरमो धर्मस्तपश्च सदसच्च सः}


\twolineshloka
{श्रोत्रशास्त्रग्रहोपेतः षोड््शर्त्विक््क्रतुश्च सः}
{पितामहश्च रुद्रश्च सोऽश्विनौ स पुरंदरः}


\threelineshloka
{मित्रोऽथ वरुणश्चैव यमोऽथ धनदस्तथा}
{ते पृथग्दर्शनास्तस्य संविदन्ति तथैकताम्}
{एकस्य विद्धि देवस्य सर्वं जगदिदं वशे}


\twolineshloka
{नानाभूतस्य दैत्येन्द्र तस्यैकत्वं वदन्त्यपि}
{जन्तुः पश्यति विज्ञानात्ततः सत्वं प्रकाशते}


\twolineshloka
{संहारविक्षेपसहस्रकोटीस्तिष्ठन्ति जीवाः प्रचरन्ति चान्ये}
{प्रजाविसर्गस्य च पारिमाण्यंवापीसहस्राणि बहूनि दैत्य}


\twolineshloka
{वाप्यः पुनर्योजनविस्तृतास्ताःक्रोशं च गम्भीरतयाऽवगाढाः}
{आयामतः पञ्चशताश्च सर्वाःप्रत्येकशो योजनतः प्रवृद्धाः}


\twolineshloka
{वाप्या जलं क्षिप्यति वालकोट्यात्वह्ना सकृच्चाप्यथ न द्वितीयम्}
{तासां क्षये विद्धि परं विसर्गंसंहारमेकं च तथा प्रजानाम्}


\twolineshloka
{ष़ड््जीववर्णाः परमं प्रमाणंकृष्णो धूम्रो नीलमथास्य मध्यम्}
{रक्तं पुनः सह्यतरं सुखं तुहारिद्रवर्णं सुसुखं च शुक्लम्}


\twolineshloka
{परं तु शुक्लं विमलं विशोकंगतक्लमं सिद्ध्यति दानवेन्द्र}
{गत्वा तु योनिप्रभवाणि दैत्यसहस्रशः सिद्धिमुपैति जीवः}


\twolineshloka
{गतिं च यां दर्शनमाह देवोगत्वा शुभं दर्शनमेव चापि}
{गतिः पुनर्वर्णकृता प्रजानांवर्णस्तथा कालकृतोऽसुरेन्द्र}


\twolineshloka
{शतं सहस्राणि चतुर्दशेहपरा गतिर्जीवगणस्य दैत्य}
{आरोहणं तत्कृतमेव विद्धिस्थानं तथा निःसरणं च तेषाम्}


\twolineshloka
{`योऽस्मादथ भ्रश्यति कालयोगात्कृष्णे वर्णे तिष्ठति सर्वकृष्टे}
{अतिप्रसक्तो निरयाच्च दैत्यततस्ततः संपरिवर्तते च ॥'}


\twolineshloka
{कृष्णस्य वर्णस्य गतिर्निकुष्टास मज्जते नरके पच्यमानः}
{स्थानं तथा दुर्गतिभिस्तु तस्यप्रजाविसर्गान्सुबहून्वदन्ति}


\twolineshloka
{शतं सहस्राणि ततश्चरित्वाप्राप्नोति वर्णं हरितं तु पश्चात्}
{स चैव तस्मिन्निवसत्यनिशोयुगक्षयं तमसा संवृतात्मा}


\twolineshloka
{स वै यदा सत्वगुणेन युक्तस्तमो व्यपोहन्घटते स्वबुद्ध्या}
{स लोहितं वर्णमुपैति नीलान्मनुष्यलोके परिवर्तते च}


\twolineshloka
{स तत्र संहारविसर्गमेकंस्वकर्मजैर्बन्धनैः क्लिश्यमानः}
{ततः स हारिद्रमुपैति वर्णंसंहारविक्षेपशते व्यतीते}


\twolineshloka
{हारिद्रवर्णस्तु प्रजाविसर्गात्सहस्रशस्तिष्ठति संचरन्वै}
{अविप्रमुक्तो निरये च दैत्यततः सहस्राणि दशापराणि}


\twolineshloka
{गतीः सहस्राणि च पञ्च तस्यचत्वारि संवर्तकृतानि चैव}
{विमुक्तमेनं निरयाच्च विद्धिसर्वेषु चान्येषु च संभवेषु}


% Check verse!
स देवलोके विहरत्यभीक्ष्णंततश्च्युतो मानुषतामुपैतिसंहारविक्षेपशतानि चाष्टौमर्त्येषु तिष्ठन्नमृतत्वमेवि
\twolineshloka
{सोऽस्मादय भ्रश्यति कालयोगात्कृष्णे तले तिष्ठति सर्वकृष्टे}
{यथा त्वयं सिध्यति जीवलोकस्तत्तेऽभिधास्याम्यसुरप्रवीर}


\threelineshloka
{दैवानि स व्यूहशतानि सप्तरक्तो हरिद्रोऽथ तथैव शुक्लः}
{संश्रित्य संधावति शुक्लमेतमष्टावरानर्च्यतमान्स लोकान्}
{}


\twolineshloka
{अष्टौ च षष्टिं च शतानि चैवमनोविरुद्धानि महाद्युतीनाम्}
{शुक्लस्य वर्णस्य परा गतिर्यात्रीण्येव रुद्धानि महानुभाव}


\twolineshloka
{संहाराविक्षेपमनिष्टमेकंचत्वारि चान्यानि वसत्यनीशः}
{षष्ठस्य वर्णस्य परा गतिर्यासिद्धावसिद्धस्य गतक्लमस्यर}


\threelineshloka
{सप्तोचरं तत्र वसत्यनीशःसंहारविक्षेपशतं सशेषः}
{संहारविक्षेपमनिष्टमेकंचत्वारि चान्यानि वसत्यनीशः}
{तस्मादुपावृत्य मनुष्यलोकेततो महान्मानुषतामुपैति}


\twolineshloka
{तस्मादुपावृत्य ततः क्रमेणसोग्रेण संतिष्ठति भूतसर्गम्}
{स सप्तकृत्वश्च परैति लोकान्संहारविक्षेपकृतप्रवासः}


\threelineshloka
{सप्तैव संहारमुपप्लवानिसंभाव्य संतिष्ठति जीवलोके}
{ततोऽव्ययं स्थानमनन्तमेतिदेवस्य विष्णोरथ ब्रह्मणथ}
{शेषस्य चैवाथ नरस्य चैवदेवस्य विष्णोः परमस्य चैव}


\twolineshloka
{संहारकाले परदिग्धकायाब्रह्माणमायान्ति सदा प्रजाहि}
{चेष्टात्मनो देवगणाश्च सर्वेये ब्रह्मलोके ह्यमराः स्म तेऽपि}


\twolineshloka
{प्रजानिसर्गे तु स शेषकालेस्थानानि स्वान्येव सरन्ति जीवाः}
{निःशेषतस्तत्पदं वान्ति चान्तेसर्वे देवा ये सदृशा मनुष्याः}


\twolineshloka
{ये तु च्युताः सिद्धलोकात्क्रमेणतेषां गतिं यान्ति यथाऽऽनुपूर्व्या}
{जीवाः परे तद्बलवेषरूपाःस्वकं विधिं यान्ति विषर्ययेण}


\twolineshloka
{स यावदेवास्ति सशेषभुक्तिःप्रजाश्च देव्यौ च तथैव शुक्ले}
{तावत्तदङ्गेषु विशुद्धभावःसंयम्य पञ्चेन्द्रियरूपमेतत्}


\fourlineindentedshloka
{शुद्धां गतिं तां परमां प्रयातिशुद्धेन नित्यं मनसा विचिन्वन्}
{ततोऽव्ययं स्थानमुपैति ब्रह्मदुष्प्रापमन्येन स शाश्वतं वै}
{इत्येतदाख्यातमहीनसत्वनारायणस्येह बलं मया ते ॥वृत्र उवाच}
{}


\twolineshloka
{एवं गते मे न विषादोस्ति कश्चित्सम्यक्च पश्यामि वचस्तथैतत्}
{श्रुत्वा तु ते वाचमदीनसत्वविकल्मषोस्म्यद्य तथा विपाष्मा}


\fourlineindentedshloka
{प्रवृत्तमेतद्भगवन्महर्षेमहाद्युतेश्चक्रमनन्तवीर्यम्}
{विष्णोरनन्तस्य सनातनं तत्स्थानं सर्गा यत्र सर्वं प्रवृत्ताः}
{स वै महात्मा पुरुषोत्तमो वैतस्मिज्जगत्सर्वमिदं प्रतिष्ठितम् ॥भीष्म उवाच}
{}


\threelineshloka
{एवमुक्त्वा स कौन्तेय वृत्रः प्राणानवासृजत्}
{योजयित्वा तथाऽऽत्मानं परं स्थानमवाप्तवान् ॥युधिष्ठिर उवाच}
{}


\threelineshloka
{अयं स भगवान्देवः पितामह जनार्दनः}
{सनत्कुमारो वृत्राय यत्तदाख्यातवान्पुरा ॥भीष्म उवाच}
{}


\twolineshloka
{मूलस्थायी स भगवान्स्वेनानन्तेन तेजसा}
{तत्स्थः सृजति तान्भावानात्मरूपान्महामनाः}


\threelineshloka
{तुरीयांशेन तस्येमं विद्धि केशवमच्युतम्}
{`तुरीयांशेन ब्रह्माणं तस्य विद्धि महात्मनः}
{'तुरीयार्धेन लोकांस्त्रीन्भावयत्येव बुद्धिमान्}


\threelineshloka
{अर्वाक्स्थितस्तु यः स्थायी कल्पान्ते परिवर्तते}
{स शेते भगवानप्सु योऽसावतिबलः प्रभुः}
{तान्विघाता प्रसन्नात्मा लोकांश्चरति शाश्वतान्}


\threelineshloka
{सर्वाण्यशून्यानि करोत्यनन्तःसनातनः संचरते च लोकान्}
{स चानिरुद्धः सृजते महात्मातत्स्थं जगत्सर्वमिदं विचित्रम् ॥युधिष्ठिर उवाच}
{}


\twolineshloka
{वृत्रेण परमार्थज्ञ दृष्टा मन्येत्मनो गतिः}
{सुखात्तस्मात्स सुखितो न शोचति पितामह}


\twolineshloka
{शुक्लः शुक्लाभिजातीयः साध्यो नावर्ततेऽनघ}
{तिर्यग्गतेश्च निर्मुक्तो निरयाच्च पितामह}


\twolineshloka
{हारिद्रवर्णे रक्ते वा वर्तमानस्तु पार्थिव}
{तिर्यगेवानुपश्येत कर्मभिस्तामसैर्वृतः}


\threelineshloka
{वयं तु भृशमापन्ना रक्ताः कष्टाः सुखेऽसुखे}
{कां गतिं प्रतिपत्स्यामो नीलां कृष्णाधमामथ ॥भीष्म उवाच}
{}


\twolineshloka
{शुद्धाभिजनसंपन्नाः पाण्डवाः संशितव्रताः}
{विहत्य देवलोकेषु पुनर्मानुषमेध्यथ}


\twolineshloka
{प्रजाविसर्गं च सुखेन लोकेप्रेत्यान्येदेहेषु सुखानि भुक्त्वा}
{सुखेन संयास्यथ सिद्धसङ्ख्यांमा वो भयं भवतु न वोऽस्तु पापम्}


\chapter{अध्यायः २८७}
\twolineshloka
{युधिष्ठिर उवाच}
{}


\twolineshloka
{अहो धर्मिष्ठता तस्य वृत्रस्यामिततेजसः}
{यस्य विज्ञानमतुलं विष्णोर्भक्तिश्च तादृशी}


\twolineshloka
{दुर्विज्ञेयमिदं तस्य विष्णोरमिततेजसः}
{कथं वा राजशार्दूल पदं तु ज्ञातवानसौ}


\twolineshloka
{भवता कथितं ह्येतच्छ्रद्दधे चाहमच्युत}
{भूयश्च मे समुत्पन्ना बुद्धिरव्यक्तदर्शना}


\twolineshloka
{कथं विनिहतो वृत्रः शक्रेण भरतर्षभ}
{धार्मिको विष्णुभक्तश्च तत्त्वज्ञश्च तदन्वये}


\twolineshloka
{एतन्मे संशयं ब्रूहि पृच्छते भरतर्षभ}
{वृत्रः स राजशार्दूल यथा शक्रेण निर्जितः}


\threelineshloka
{यथा चैवाभवद्युद्धं तच्चाचक्ष्व पितामह}
{विस्तरेण महाबाहो परं कौतूहलं हि मे ॥भीष्म उवाच}
{}


\twolineshloka
{रथेनेन्द्रः प्रयातो वै सार्धं सुरगणैः पुरा}
{ददर्शाथाग्रतो वृत्रं धिष्ठितं पर्वतोपमम्}


\twolineshloka
{योजनानां शतान्यूर्ध्वं पञ्चोन्छ्रितमरिंदम्}
{शतानि विस्तरेणाथ त्रीणि चाभ्यधिकानि वै}


\twolineshloka
{तत्प्रेक्ष्य तादृशं रूपं त्रैलोक्येनापि दुर्जयम्}
{वृत्रस्य देवाः संत्रस्ता न शान्तिमुपलेभिरे}


\twolineshloka
{शक्रस्य तु तदा राजन्नूरुस्तम्भो व्यजायत}
{भयाद्वृत्रस्य सहसा दृष्ट्वा तद्रूपमुत्तमम्}


\twolineshloka
{ततो नादः समभवद्वादित्राणां च निःस्वनः}
{देवासुराणां सर्वेषां तस्मिन्युद्धे ह्युपस्थिते}


\twolineshloka
{अथ वृत्रस्य कौरव्य दृष्ट्वा शक्रमवस्थितम्}
{न संभ्रमो न भीः काचिदास्था वा समजायत}


\twolineshloka
{ततः समभवद्युद्धं त्रैलोक्यस्य भयंकरम्}
{शक्रस्य च सुरेन्द्रस्य वृत्रस्य च महात्मनः}


\twolineshloka
{असिभिः पट्टसैः शूलैः शक्तितोमरम्रुद्गरैः}
{शिलाभिर्विविधाभिश्च कार्मुकैश्च महास्वनैः}


\twolineshloka
{अस्त्रैश्च विविर्धौर्दिव्यैः पावकोल्काभिरेव च}
{देवासुरैस्ततः सैन्यैः सर्वमासीत्समाकुलम्}


\twolineshloka
{पितामहपुरोगाश्च सर्वे देवगणास्तदा}
{ऋषयश्च महाभागास्तद्युद्धं द्रष्टुमागमन्}


\twolineshloka
{विमानाग्र्यैर्महाराज सिद्धाश्च भरतर्षभ}
{गन्धर्वाश्च विमानाग्रैरप्सरोभिः समागमन्}


\twolineshloka
{ततोऽन्तरिक्षमावृत्य वृत्रो धर्मभूतां वरः}
{अश्मवर्षेण देवेन्द्रं सर्वतः समवाकिरत्}


\twolineshloka
{ततो देवगणाः क्रुद्धाः सर्वतः शरवृष्टिभिः}
{अश्मवर्षमपोहन्त वृत्रप्रेरितमाहवे}


\twolineshloka
{वृत्रस्तु कुरुशार्दूल महामायो महाबलः}
{मोहयामास देवेन्द्रं मायायुद्धेन सर्वशः}


\threelineshloka
{तस्य वृत्रार्दितस्याथ मोह आसीच्छतक्रतोः}
{रथन्तरेण तं तत्र वसिष्ठः समबोधयत् ॥वसिष्ठ उवाच}
{}


\twolineshloka
{देवश्रेष्ठोऽस्ति देवेन्द्र दैत्यासुरनिबर्हण}
{त्रैलोक्यबलसंयुक्तः कस्माच्छक्र विषीदसि}


\twolineshloka
{एष ब्रह्मा च विष्णुश्च शिवश्चैव जगत्पतिः}
{सोमश्च भगवान्देवः सर्वे च परमर्षयः}


\threelineshloka
{`समुद्विग्नं समीक्ष्य त्वां स्वस्तीत्यूचुर्जयाय ते}
{'मा कार्षीः कश्मलं शक्र कश्चिदेवेतरो यथा}
{आर्यां युद्धे मतिं कृत्वा जहि शत्रून्सुराधिप}


\twolineshloka
{एष लोकगुरुस्त्र्यक्षः सर्वलोकनमस्कृतः}
{निरीक्षते त्वां भगवांस्त्यज मोहं सुराधिप}


\threelineshloka
{एते ब्रह्मर्षयश्चैव बृहस्पतिपुरोगमाः}
{स्तवेन शक्र दिव्येन स्तुवन्ति त्वां जयाय वै ॥भीष्म उवाच}
{}


\twolineshloka
{एवं संबोध्यमानस्य वसिष्ठेन महात्मना}
{अतीव वासवस्यासीद्बलमुत्तमतेजसः}


\twolineshloka
{ततो बुद्धिमुपागम्य भगवान्पाकशासनः}
{योगेन महता युक्तस्तां मायां व्यपकर्षत}


\twolineshloka
{ततोऽङ्गिरः सुतः श्रीमांस्ते चैव सुमहर्षयः}
{दृष्ट्वा वृत्रस्य विक्रान्तुमुपागम्य महेश्वरम्}


\twolineshloka
{ऊचुर्वृत्रविनाशार्थं लोकानां हितकाम्यया}
{ततो भगवतस्तेजो ज्वरो भूत्वा जगत्पतेः}


\twolineshloka
{समाविशत्तदा रौद्रं वृत्रं लोकपतिं तदा}
{विष्णुश्च भगवान्देवः सर्वलोकाभिपूजितः}


\threelineshloka
{ऐन्द्रं समाविशुद्वज्रं लोकसंरक्षणे रतः}
{ततो बृहस्पतिर्धीमानुपागम्य शतक्रतुम्}
{वसिष्ठश्च महातेजाः सर्वे च परमर्षयः}


\threelineshloka
{ते समासाद्य वरदं वासवं लोकपूजितम्}
{ऊचुरेकाग्रमनसो जहि वृत्रमिति प्रभो ॥महेश्वर उवाच}
{}


\twolineshloka
{एष वृत्रो महाञ्शक्रे बलेन महता वृतः}
{विश्वात्मा सर्वगश्चैव बहुमायश्च विश्रुतः}


\twolineshloka
{तदेनमसुरश्रेष्ठं त्रैलोक्येनापि दुर्जयम्}
{जहि त्वं योगमास्थाय मावमंस्थाः सुरेश्वर}


\twolineshloka
{अनेन हि तपस्तप्तं बलार्थममराधिप}
{षष्टिं वर्षसहस्राणि ब्रह्मा चास्मै वरं ददौ}


\twolineshloka
{महत्त्वं योगिनां चैव महामायत्वमेव च}
{महाबलत्वं च तथा तेजश्चाग्र्यं सुरेश्वर}


\threelineshloka
{एतत्त्वां मामकं तेजः समाविशति वासव}
{वृत्रमेवं त्ववध्यं तं वज्रेण जहि दानवम् ॥शक्र उवाच}
{}


\threelineshloka
{भगवंस्त्वत्प्रसादेन दितिजं सुदुरासदम्}
{वज्रेण निहनिष्यामि पश्यतस्ते सुरर्षभ ॥भीष्म उवाच}
{}


\twolineshloka
{आविश्यमाने दैत्ये तु ज्वरेणाथ महासुरे}
{देवतानामृषीणां च हर्षान्नादो महानभूत्}


\twolineshloka
{ततो दुन्दुभयश्चैव शङ्ख्याश्च सुमहास्वनाः}
{मुरजा डिण्डिभाश्चैव प्रावाद्यन्त सहस्रशः}


\twolineshloka
{असुराणां तु सर्वेषां स्मृतिलोपो महानभूत्}
{मायानाशश्च बलवान्क्षणेन समपद्यत}


\twolineshloka
{तमाविष्टमथो ज्ञात्वा ऋषयो देवतास्तथा}
{स्तुवन्तः शक्नमीशानं तथा प्राचोदयन्नपि}


\twolineshloka
{रथस्थस्य हि शक्रस्य युद्धकाले महात्मनः}
{ऋषिभिः स्तूयमानस्य रूपमासीन्सुदुर्दृशम्}


\chapter{अध्यायः २८८}
\twolineshloka
{भीष्म उवाच}
{}


\twolineshloka
{वृत्रस्य तु महाराज ज्वराविष्टस्य सर्वशः}
{अभवन्यानि लिङ्गानि शरीरे तानि मे शृणु}


\twolineshloka
{ज्वलितास्योऽभवद्धोरो वैवर्ण्यं चागमत्परम्}
{गात्रकम्पश्च सुमहाञ्श्वासश्चाप्यभवन्महान्}


\twolineshloka
{रोमहर्षश्च तीव्रोऽभून्निःश्वासश्च महान्नृप}
{शिवा चाशिवसंकाशा तस्य वक्रात्सुदारुणा}


\twolineshloka
{निष्पपात महाघोरा स्मृतिर्नष्टास्य भारत}
{उत्काश्च ज्वलितास्तस्य दीप्ताः पार्श्वे प्रपेदिरे}


\twolineshloka
{गृध्राः कङ्का बलाकाश्च वाचोऽमुञ्चन्सुदारुणाः}
{वृत्रस्योपरि संहृष्टाश्चक्रवत्परिबभ्रमुः}


\twolineshloka
{ततस्तं रथमास्थाय देवाप्यायित आहवे}
{वज्रोद्यतकरः शक्रस्तं दैत्यं प्रत्यवैक्षत}


\twolineshloka
{अमानुषमथो नादं स मुमोच महासुरः}
{व्यजृम्भच्चैव राजेन्द्र तीव्रज्वरसमन्वितः}


\twolineshloka
{अथास्य जृम्भतः शक्रस्ततो वज्रमवासृजत्}
{स वज्रः सुमहातेजा कालान्तकयमोपमः}


\twolineshloka
{क्षिप्रमेव महाकायं वृत्रं दैत्यमपातयत्}
{ततो नादः समभवत्पुनरेव समन्ततः}


\twolineshloka
{वृत्रं विनिहितं दृष्ट्वा देवानां भरतर्षभ}
{वृत्रं तु हत्वा मघवा दानवारिर्महायशाः}


\twolineshloka
{वज्रेण विष्णुयुक्तेन दिवमेव समाविशत्}
{अथ वृत्रस्य कौरव्य शरीरादभिनिःसृता}


\twolineshloka
{ब्रह्महत्या महाघोरा रौद्रा लोकभयावहा}
{करालदशना भीमा विकृता कृष्णपिङ्गला}


\twolineshloka
{प्रकीर्णमूर्धजा चैव घोरनेत्रा च भारत}
{कपालमालिनी चैव कृत्येव भरतर्षभ}


\twolineshloka
{रुधिरार्द्रा च धर्मज्ञ चीरवल्कलवासिनी}
{साऽभिनिष्क्रम्य राजेन्द्र तादृग्रृपा भयावहा}


\twolineshloka
{वज्रिणं मृगयामास तदा भरतसत्तम}
{कस्यचित्त्वथ कालस्य वृत्रहा कुरुनन्दन}


\twolineshloka
{स्वर्गायाभिमुखः प्रायाल्लोकानां हितकाम्यया}
{सा विनिःसरमाणं तु दृष्ट्वा शक्रं महौजसम्}


\twolineshloka
{कण्ठे जग्राह देवेन्द्रं सुलग्ना चाभवत्तदा}
{स हि तस्मिन्समुत्पन्ने ब्रह्मवध्याकृते भये}


\twolineshloka
{नलिन्या विसमध्यस्थ उवासाब्दगणान्बहून्}
{अनुसृत्य तु यत्नात्स तया वै ब्रह्महत्यया}


\twolineshloka
{तदा गृहीतः कौरव्य निस्तेजाः समपद्यत}
{तस्या व्यपोहने शक्रः परं यत्नं चकार ह}


\twolineshloka
{न चाशकत्तां देवेन्द्रो ब्रह्मवध्यां व्यपोहितुम्}
{गृहीत एव तु तया देवेन्द्रो भरतर्षभ}


\twolineshloka
{पितामहमुपागम्य शिरसा प्रत्यपूजयत्}
{ज्ञात्वा गृहीतं शक्रं स द्विजप्रवरवध्यया}


\twolineshloka
{ब्रह्मा स चिन्तयामास तदा भरतसत्तम}
{तामुवाच महाबाहो ब्रह्मवध्यां पितामहः}


\twolineshloka
{स्वरेण मधुरेणाथ सान्त्वयन्निव भारत}
{मुच्यतां त्रिदशेन्द्रोयं मत्प्रियं कुरु भामिनी}


\twolineshloka
{ब्रूहि किं ते करोम्यद्य कामं किं त्वमिहेच्छसि ॥ब्रह्महत्योवाच}
{}


\twolineshloka
{त्रिलोकपूजिते देवे प्रीते त्रैलोक्यकर्तरि}
{कृतमेव हि मन्यामि निवासं तु विधत्स्व मे}


\twolineshloka
{त्वया कृतेयं मर्यादा लोकसंरक्षणार्थिना}
{स्थापना वै सुमहती त्वया देव प्रवर्तिता}


\threelineshloka
{प्रीते तु त्वयि धर्मज्ञ सर्वलोकेश्वर प्रभो}
{शक्रादपगमिष्यामि निवासं संविधत्स्व मे ॥भीष्म उवाच}
{}


\twolineshloka
{तथेति तां प्राह तदा ब्रह्मवध्यां पितामहः}
{उपायतः स शक्रस्य ब्रह्मवध्यां व्यपोहितुम्}


\twolineshloka
{ततः स्वयंभुवा ध्यातस्तत्र वह्निर्महात्मना}
{ब्रह्माणमुपसंगम्य ततो वचनमब्रवीत्}


\threelineshloka
{प्राप्तोऽस्मि भगवन्देव त्वत्सकाशमनिन्दित्}
{यत्कर्तव्यं मया देव तद्भवान्वक्तुमर्हति ॥ब्रह्मोवाच}
{}


\threelineshloka
{बहुधा विभजिष्यामि ब्रह्मवध्यामिमामहम्}
{शक्रस्याद्य विमोक्षार्थं चतुर्भागं प्रतीच्छ वै ॥अग्निरुवाच}
{}


\threelineshloka
{मम मोक्षस्य कोऽन्तो वै ब्रह्मन्ध्यायस्व वै प्रभो}
{एतदिच्छामि विज्ञातुं तत्वतो लोकपूजित ॥ब्रह्मोवाच}
{}


\twolineshloka
{यस्त्वां ज्वलन्तमासाद्य स्वयं वै मानवः क्वचित्}
{बीजौषधिरसैर्वह्ने न यक्ष्यति तमोवृतः}


\twolineshloka
{तमेषा यास्यति क्षिप्रं तत्रैव च निवत्स्यति}
{ब्रह्मवध्या हव्यवाह व्येतु ते मानसो ज्वरः}


\twolineshloka
{इत्युक्तः प्रतिजग्राह तद्वचो हव्यकव्यभुक्}
{पितामहस्य भगवांस्तथा च तदभूत्प्रभो}


\twolineshloka
{ततो वृक्षौषधितृणं समाहूय पितामहः}
{इममर्थं महाराज वक्तुं समुपचक्रमे}


\twolineshloka
{`इयं पुत्रादनुप्राप्ता ब्रह्महत्या महाभया}
{पुरुहूतं चतुर्थांशमस्या यूयं प्रतीच्छत ॥'}


\twolineshloka
{ततो वृक्षौषधितृणं तथैवोक्तं यथातथम्}
{व्यथितं वह्निवद्राजन्ब्रह्माणमिदमब्रवीत्}


\twolineshloka
{अस्माकं ब्रह्मवध्यायाः कोऽन्तो लोकपितामह}
{स्वभावनिहतानस्मान्न पुनर्हन्तुमर्हसि}


\twolineshloka
{वयमग्निं तथा शीतं वर्षं च पवनेरितम्}
{सहामः सततं देव तथा च्छेदनभेदने}


\fourlineindentedshloka
{ब्रह्मवध्यामिमामद्य भवतः शासनाद्वयम्}
{ग्रहीष्यामस्त्रिलोकेश मोक्षं चिन्तयतां भवान्}
{ब्रह्मोवाच}
{}


\threelineshloka
{पर्वकाले तु संप्राप्ते यो वै छेदनभेदनम्}
{करिष्यति नरो मोहात्तमेषाऽनुगमिष्यति ॥भीष्म उवाच}
{}


\twolineshloka
{ततो वृक्षौषधितृणमेवमुक्तं महात्मना}
{ब्रह्माणमभिसंपूज्य जगामाशु यथागतम्}


\twolineshloka
{आहूयाप्सरसो देवस्ततो लोकपितामहः}
{वाचा मधुरया प्राह सान्त्वयन्निव भारत}


\threelineshloka
{इयमिन्द्रादनुप्राप्ता ब्रह्मवध्या वराङ्गनाः}
{चतुर्थमस्या भागांशं मयोक्ताः संप्रतीच्छत ॥अप्सरस ऊचुः}
{}


\threelineshloka
{ग्रहणे कृतबुद्धीनां देवेश तव शासनात्}
{मोक्षं समयतोऽस्माकं चिन्तयस्व पितामह ॥ब्रह्मोवाच}
{}


\threelineshloka
{रजस्वलासु नारीषु यो वै मैथुनमाचरेत्}
{तमेषा यास्यति क्षिप्रं व्येतु वो मानसो ज्वरः ॥भीष्म उवाच}
{}


\twolineshloka
{तथेति हृष्टमनस इत्युक्त्वाऽऽप्सरसां गणाः}
{स्वानि स्थानानि संप्राप्य रेमिरे भरतर्षभ}


\twolineshloka
{ततस्त्रिलोककृद्देवः पुनरेव महातपाः}
{अथः संचिन्तयामास ध्यातास्ताश्चाप्यथागमन्}


\twolineshloka
{तास्तु सर्वाः समागम्य ब्रह्माणममितौजसम्}
{इदमूचुर्वचो राजन्प्रणिपत्य पितामहम्}


\threelineshloka
{इमाः स्म देव संप्राप्तास्त्वत्सकाशमरिंदम्}
{शासनात्तव लोकेश समाज्ञापय नः प्रभो ॥ब्रह्मोवाच}
{}


\threelineshloka
{इयं वृत्रादनुप्राप्ता पुरुहूतं महाभया}
{ब्रह्मवध्या चतुर्थांशमस्या यूयं प्रतीच्छत ॥आप ऊचुः}
{}


\twolineshloka
{एवं भवतु लोकेश यथा वदसि नः प्रभो}
{मोक्षं समयतोऽस्माकं संचिन्तयितुमर्हसि}


\threelineshloka
{त्वं हि देवेश सर्वस्य जगतः परमा गतिः}
{कोऽन्यः प्रसाद्यो हि भवेद्यः कृच्छ्रान्नः समुद्धरेत् ॥ब्रह्मोवाच}
{}


\twolineshloka
{अल्पा इति मतिं कृत्वा यो नरो बुद्धिमोहितः}
{श्लेष्ममूत्रपुरीषाणि युष्मासु प्रतिमोक्ष्यति}


\twolineshloka
{तमियं यास्यति क्षिप्रं तत्रैव च निवत्स्यति}
{तथा वो भविता मोक्ष इति सत्यं ब्रवीमि वः}


\twolineshloka
{ततो विमुच्य देवेन्द्रं ब्रह्मवध्या युधिष्ठिर}
{यथा निसृष्टं तं वासमगमद्देवशासनात्}


\twolineshloka
{एवं शक्रेण संप्राप्ता ब्रह्मवध्या जनाधिप}
{पितामहमनुज्ञाप्य सोऽश्वमेधमकल्पयत्}


\twolineshloka
{श्रूयते च महाराज संप्राप्ता वासवेन वै}
{ब्रह्मवध्या ततः शुद्धिं हयमेधेन लब्धवान्}


\twolineshloka
{समवाप्य श्रियं देवो हत्वाऽरींश्च सहस्रशः}
{प्रहर्षमतुलं लेभे वासवः पृथिवीपते}


\twolineshloka
{वृत्रस्य रुधिराच्चैव बुद्बुदाः पार्थ जज्ञिरे}
{द्विजातिभिरभक्ष्यास्ते दीक्षितैश्च तपोधनैः}


\twolineshloka
{सर्वावस्थं त्वमप्येषां द्विजातीनां प्रियं कुरु}
{इमे हि भूतले देवाः प्रथिताः कुरुनन्दन}


\twolineshloka
{एवं शक्रेण कौरव्य बुद्धिसौक्ष्म्यान्महासुरः}
{उपायपूर्वं निहतो वृत्रो ह्यमिततेजसा}


\twolineshloka
{एवं त्वमपि कौन्तेय पृथिव्यामपराजितः}
{भविष्यसि यथा देवः शतक्रतुरमित्रहा}


\twolineshloka
{ये तु शक्रकथां दिव्यामिमां पर्वसुपर्वसु}
{विप्रमध्ये वदिष्यन्ति न ते प्राप्स्यन्ति किल्विषं}


\twolineshloka
{इत्येतद्वृत्रमाश्रित्य शक्रस्यात्यद्भुतं महत्}
{कथितं कर्म ते तात किं भूयः श्रोतुमिच्छसि}


\chapter{अध्यायः २८९}
\twolineshloka
{युधिष्ठिर उवाच}
{}


\twolineshloka
{पितामह महाप्राज्ञ सर्वशास्त्रविशारद}
{अस्मिन्वृत्रवधे तात विवक्षा मम जायते}


\twolineshloka
{ज्वरेण मोहितो वृत्रः कथितस्ते जनाधिप}
{निहतो वासवेनेह वज्रेणेति ममानघ}


\threelineshloka
{कथमेष महाप्राज्ञ ज्वरः प्रादुर्बभूव ह}
{ज्वरोत्पत्तिं निपुणतः श्रोतुमिच्छाम्यहं प्रभो ॥भीष्म उवाच}
{}


\twolineshloka
{शृणु राजञ्ज्वरस्येमं संभवं लोकविश्रुतम्}
{विस्तरं चास्य वक्ष्यामि यादृशश्चैव भारत}


\twolineshloka
{पुरा मेरोर्महाराज शृङ्गं त्रैलोक्यविश्रुतम्}
{ज्योतिष्कं नाम सावित्रं सर्वरत्नविभूषितम्}


\twolineshloka
{अप्रमेयमनाधृष्यं सर्वलोकेषु भारत}
{तत्र देवो गिरितटे हेमधातुविभूषिते}


\threelineshloka
{पर्यङ्क इव विभ्राजन्नुपविष्टो बभूव ह}
{शैलराजसुता चास्य नित्यं पार्श्वे स्थिता बभौ}
{तथा देवा महात्मानो वसवश्चामितौजसः}


\twolineshloka
{तथैव च महात्मानावश्विनौ भिषजां वरौ}
{तथा वैश्रवणो राजा गुह्यकैरभिसंवृतः}


\threelineshloka
{यक्षाणामीश्वरः श्रीमान्कैलासनिलयः प्रभुः}
{`शङ्खपद्मनिधिभ्यां च लक्ष्म्या परमया सह}
{'उपासन्त महात्मानमुशना च महाकविः}


\twolineshloka
{सनत्कुमारप्रमुखास्तथैव च महर्षयः}
{अङ्गिरः प्रमुखाश्चैव तथा देवर्षयोऽपरे}


\twolineshloka
{विश्वावसुश्च गन्धर्वस्तथा नारदपर्वतौ}
{अप्सरोगणसङ्घाश्च समाजग्मुरनेकशः}


\twolineshloka
{ववौ सुखः शिवो वायुर्नानागन्धवहः शुचिः}
{सर्वर्तुकुसुमोपेताः पुष्पवन्तो द्रुमास्तथा}


\twolineshloka
{तथा विद्याधराश्चैव सिद्धाश्चैव तपोधनाः}
{महादेवं पशुपतिं पर्युपासन्त भारत}


\twolineshloka
{भूतानि च महाराज नानारूपधराण्यथ}
{राक्षसाश्च महारौद्राः पिशाचाश्च महाबलाः}


\twolineshloka
{बहुरूपधरा हृष्टा नानाप्रहरणोद्यताः}
{देवस्यानुचरास्तत्र तस्थिरे चानलोपमाः}


\twolineshloka
{नन्दी च भगवांस्तत्र देवस्यानुमते स्थितः}
{प्रगृह्य ज्वलितं शूलं दीप्यमानं स्वतेजसा}


\twolineshloka
{गङ्गा च सरितां श्रेष्ठा सर्वतीर्थजलोद्भवा}
{पर्युपासत तं देव रूपिणी कुरुनन्दन}


\twolineshloka
{स एवं भगवांस्तत्र पूज्यमानः सुरर्षिभिः}
{देवैश्च सुमहातेजा महादेवो व्यतिष्ठत}


\twolineshloka
{कस्यचित्त्वथ कालस्य दक्षो नाम प्रजापतिः}
{पूर्वोक्तेन विधानेन यक्ष्यमाणोऽन्वपद्यत}


\twolineshloka
{ततस्तस्य मखं देवाः सर्वे शक्रपुरोगमाः}
{गमनाय समागम्य बुद्धिमापेदिरे तदा}


\twolineshloka
{ते विमानैर्महात्मानो ज्वलनार्कसमप्रभैः}
{देवस्यानुमतेऽगच्छन्गङ्गाद्वारमिति श्रुतिः}


\twolineshloka
{प्रस्थिता देवता दृष्ट्वा शैलराजसुता तदा}
{उवाच वचनं साध्वी देवं पशुपतिं पतिम्}


\threelineshloka
{भगवन्क्वनु यान्त्येते देवाः शक्रपुरोगमाः}
{ब्रूहि तत्त्वेन तत्त्वज्ञ संशयो मे महानयम् ॥महेश्वर उवाच}
{}


\threelineshloka
{दक्षो नाम महाभागे प्रजानां पतिरुत्तमः}
{हयमेधेन जयते तत्र यान्ति दिवौकसः ॥उमोवाच}
{}


\threelineshloka
{यज्ञमेतं महादेव किमर्थं नाधिगच्छति}
{केन वा प्रतिषेधेन गमनं ते न विद्यते ॥महेश्वर उवाच}
{}


\twolineshloka
{सुरैरेव महाभागे पूर्वमेतदनुष्ठितम्}
{यज्ञेषु सर्वेषु मम न भाग उपकल्पितः}


\threelineshloka
{पूर्वोपायोपपन्नेन मार्गेण वरवर्णिनि}
{न मे सुराः प्रयच्छन्ति भागं यज्ञस्य धर्मतः ॥उमोवाच}
{}


\twolineshloka
{भगवन्सर्वभूतेषु प्रभावाभ्यधिको गुणैः}
{अजय्यश्चाप्यधृष्यश्च तेजसा यशसा श्रिया}


\threelineshloka
{अनेन ते महाभाग प्रतिषेधेन भागतः}
{अतीव दुःखमुत्पन्नं वेपथुश्च ममानघ ॥भीष्म उवाच}
{}


\twolineshloka
{एवमुक्त्वा तु सा देवी देवं पशुपतिं पतिम्}
{तुष्णींभूताऽभवद्राजन्दह्यमानेन चेतसा}


\twolineshloka
{अथ देव्या मतं ज्ञात्वा हृद्गतं यच्चिकीर्षितम्}
{स समाज्ञापयामास तिष्ठ त्वमिति नन्दिनम्}


\twolineshloka
{ततो योगबलं कृत्वा सर्वयोगेश्वरेश्वरः}
{तं यज्ञं स महातेजा भीमैरनुचरैस्तदा}


\twolineshloka
{सहसा घातयामास देवदेवः पिनाकधृत्}
{केचिन्नादानमुञ्चन्त केचिद्धासांश्च चक्रिरे}


\twolineshloka
{रुधिरेणापरे राजंस्तत्राग्निं समवाकिरन्}
{केचिद्यूपान्समुत्पाट्य व्याक्षिपन्विकृताननाः}


\twolineshloka
{आस्यैरन्ये चाग्रसन्त तथैव परिचारकान्}
{ततः स यज्ञो नृपतेर्वध्यमानः समन्ततः}


\twolineshloka
{आस्थाय मृगरूपं वै स्वमेवाभ्यगमत्तदा}
{तं तु यज्ञं तथारूपं गच्छन्तमुपलभ्य सः}


\twolineshloka
{धनुरादाय बाणेन तदान्वसरत प्रभुः}
{ततस्तस्य सुरेशस्य क्रोधादमिततेजसः}


\twolineshloka
{ललाटात्प्रसृतो घोरः स्वेदबिन्दुर्बभूव ह}
{तस्मिन्यतितमात्रे च स्वेदबिन्दौ तदा भुवि}


\twolineshloka
{प्रादुर्बभूव सुमहानग्निः कालानलोपमः}
{तत्र चाजायत तदा पुरुषः पुरुषर्षभ}


\twolineshloka
{ह्रस्वोऽतिमात्रं रक्ताक्षो हरिश्मश्रुर्विभीषणः}
{ऊर्ध्वकेशोऽतिरोमाङ्गः श्येनोलूकस्तथैव च}


\twolineshloka
{करालकृष्णवर्णश्च रक्तवासास्तथैव च}
{तं यज्ञं सुमहासत्वोऽदहत्कक्षमिवानलः}


\twolineshloka
{व्यचरत्सर्वतो देवान्प्राद्रवत्स ऋषींस्तथा}
{देवाश्चाप्याद्रवन्सर्वे ततो भीता दिशो दश}


\twolineshloka
{तेन तस्मिन्विचरता पुरुषेण विशांपते}
{पृथिवी ह्यचलद्राजन्नतीव भरतर्षभ}


\threelineshloka
{हाहाभूतं जगत्सर्वमुपलक्ष्य तदा प्रभुः}
{पितामहो महादेवं दर्शयन्प्रत्यभाषत ॥ब्रह्मोवाच}
{}


\twolineshloka
{भवतोपि सुराः सर्वे भागं दास्यन्ति वै प्रभो}
{क्रियतां प्रतिसंहारः सर्वदेवेश्वर त्वया}


\twolineshloka
{इमा हि देवताः सर्वा ऋषयश्च परंतप}
{तव क्रोधान्महादेव न शान्तिमुपलेभिरे}


\twolineshloka
{यश्चैष पुरुषो जातः स्वेदात्ते विबुधोत्तम}
{ज्वरो नामैष धर्मज्ञ लोकेषु प्रचरिष्यति}


\twolineshloka
{एकीभूतस्य न त्वस्य धारणे तेजसः प्रभो}
{समर्था सकला पृथ्वी बहुधा सृज्यतामयम्}


\twolineshloka
{इत्युक्तो ब्रह्मणा देवो भागे चापि प्रकल्पिते}
{भगवन्तं तथेत्याह ब्रह्माणममितौजसम्}


\twolineshloka
{परां च प्रीतिमगमदुत्स्मयंश्च पिनाकधृत्}
{अवाप च तदा भागं यथोक्तं ब्रह्मणा भवः}


\twolineshloka
{ज्वरं च सर्वधर्मज्ञो बहुधा व्यसृजत्तदा}
{शान्त्यर्थं सर्वभूतानां शृणु तच्चापि पुत्रक}


\twolineshloka
{शीर्षाभितापो नागानां पर्वतानां शिलाजतु}
{अपां तु नीलिकां विद्धि निर्मोकं भुजगेषु च}


\twolineshloka
{खोरकः सौरभेयाणामूषरं पृथिवीतले}
{पशूनामपि धर्मज्ञ दृष्टिप्रत्यवरोधनम्}


\twolineshloka
{रन्ध्रागतमथाश्वानां शिखोद्भेदश्च बर्हिणाम्}
{नेत्ररोगः कोकिलानां ज्वरः प्रोक्तो महात्मना}


\twolineshloka
{अवीनां पित्तभेदश्च सर्वेषामिति नः श्रुतम्}
{शुकानामपि सर्वेषां हिक्किका प्रोच्यते ज्वरः}


\twolineshloka
{शार्दूलष्वथ धर्मज्ञ श्रमो ज्वर इहोच्यते}
{मानुषेषु तु धर्मज्ञ ज्वरो नामैष विश्रुतः}


\twolineshloka
{मरणे जन्मनि तथा मध्ये चाविशते नरम्}
{एतन्माहेश्वरं तेजो ज्वरो नाम सुदारुणः}


\twolineshloka
{नमस्यश्चैव मान्यश्च सर्वप्राणिभिरीश्वरः}
{अनेन हि समाविष्टो वृत्रो धर्मभूतां वरः}


\twolineshloka
{व्यजृम्भत ततः शक्रस्तस्मै वज्रमवासृजत्}
{प्रविश्य बज्रं वृत्रं च दारयामासं भारत}


\twolineshloka
{दारितश्च स वज्रेण महायोगी महासुरः}
{जगाम परमं स्थानं विष्णोरमिततेजसः}


\twolineshloka
{विष्णुभक्त्या हि तेनेदं जगद्व्याप्तमभूत्पुरा}
{तस्माच्च निहतो युद्धे विष्णोः स्थानमवाप्तवान्}


\twolineshloka
{इत्येष वृत्रमाश्रित्य ज्वरस्य महतो मया}
{विस्तरः कथितः पुत्र किमन्यत्प्रब्रवीमि ते}


\twolineshloka
{इमां ज्वरोत्पत्तिमदीनमानसःपठेत्सदा यः सुसमाहितो नरः}
{विमुक्तरोगः स सुखी मुदा युतोलभेत कामान्स यथा मनीषितान्}


\chapter{अध्यायः २९०}
\twolineshloka
{जनमेजय उवाच}
{}


\threelineshloka
{प्राचेतसस्य दक्षस्य कथं वैवस्वतेन्तरे}
{विनाशमगमद्ब्रह्मन्हयमेधः प्रजापतेः}
{`कथं स चाभवद्ब्रह्मन्हयमेव प्रजापतेः ॥'}


\fourlineindentedshloka
{देव्या मन्युकृतं मत्वा क्रुद्धः सर्वात्मकः प्रभुः}
{प्रसादात्तस्य दक्षेण स यज्ञः संधितः कथम्}
{एतद्वेदितुमिच्छेयं तन्मे ब्रूहि यथातथम् ॥वैशंपायन उवाच}
{}


\twolineshloka
{पुरा हिमवतः पृष्ठे दक्षो वै यज्ञमाहरत्}
{गङ्गाद्वारे शुभे देशे ऋषिसिद्धनिषेविते}


\twolineshloka
{गन्धर्वाप्सरसाकीर्णे नानाद्रुमलतावृते}
{ऋषिसङ्घैः परिवृतं दक्षं धर्मभृतां वरम्}


\twolineshloka
{पृथिव्यामन्तरिक्षे च ये च स्वर्लोकवासिनः}
{सर्वे प्राज्जलयो भूत्वा उपतस्थुः प्रजापतिम्}


\twolineshloka
{देवदानवगन्धर्वाः पिशाचोरगराक्षसाः}
{हाहा हूहूश्च गन्धर्वौ तुम्बुरुर्नारदस्तथा}


\twolineshloka
{विश्वावसुर्विश्वसेनो गन्धर्वाप्सरसस्तथा}
{आदित्या वसवो रद्राः साध्याः सह मरुद्गणैः}


\twolineshloka
{इन्द्रेण सहिताः सर्वे आगता यज्ञभागिनः}
{ऊष्मपाः सोमपाश्चैव धूमपा आज्यपास्तथा}


\twolineshloka
{ऋषयः पितरश्चैव आगता ब्रह्मणा सह}
{एते चान्ये च बहवो भूतग्रामाश्चतुर्विधाः}


\twolineshloka
{जरायुजाण्डजाश्चैव सहसा स्वेदजोद्भिजैः}
{आहूता मन्त्रिताः सर्वे देवाश्च सह पत्निभिः}


\twolineshloka
{विराजन्ते विमानस्था दीप्यमाना इवाग्नयः}
{तान्दृष्ट्वा मन्युनाऽऽविष्टो दधीचिर्वाक्यमब्रवीत्}


\twolineshloka
{नायं यज्ञो न वा धर्मो यत्र रुद्रो न इज्यते}
{वधबन्धं प्रपन्ना वै किंनु कालस्य पर्ययः}


\twolineshloka
{किंनु मोहान्न पश्यन्ति विनाशं पर्युपस्थितम्}
{उपस्थितं भयं घोरं न बुध्यन्ति यहाध्वरे}


\twolineshloka
{इत्युक्त्वा स महायोगी पश्यति ध्यानचक्षुषा}
{स पश्यति महादेवं देवीं च वरदां शुभाम्}


\twolineshloka
{नारदं च महात्मानं तस्या देव्याः समीपतः}
{संतोषं परमं लेभे इति निश्चित्य योगवित्}


\twolineshloka
{एकमन्त्रास्तु ते सर्वे येनेशो न निमन्त्रितः}
{तस्माद्देशादपक्रम्य दधीचिर्वाक्यमब्रवीत्}


\twolineshloka
{अपूज्यपूजनाच्चैव पूज्यानां चाप्यपूजनात्}
{नृघातकसमं पापं शश्वत्प्राप्नोति मानवः}


\twolineshloka
{अनृतं नोक्तपूर्वं मे न च वक्ष्ये कदाचन}
{देवतानामृषीणां च मध्ये सत्यं ब्रवीम्यहम्}


\threelineshloka
{आगतं पशुभर्तारं स्रष्टारं जगतः पतिम्}
{अध्वरे ह्यग्रभोक्तारं ह्यर्वेषां पश्यत प्रभुम् ॥दक्ष उवाच}
{}


\threelineshloka
{सन्ति नो बहवो रुद्राः शूलहस्ताः कपर्दिनः}
{एकादशस्थानगता नाहं वेद्मि महेश्वरम् ॥दधीचिरुवाच}
{}


\fourlineindentedshloka
{सर्वेषामेव मन्त्रोऽयं येनासौ न निमन्त्रितः}
{यथाऽहं शंकरादूर्ध्वं नान्यं पश्यामि दैवतम्}
{तथा दक्षस्य विपुलो यज्ञोऽयं नभविष्यति ॥दक्ष उवाच}
{}


\threelineshloka
{एतन्मखेशाय सुवर्णपात्रेहविः समस्तं विधिमन्त्रपूतम्}
{विष्णोर्नयाम्यप्रतिमस्य भागंप्रभुर्विभुश्चाहवनीय एषः ॥देव्युवाच}
{}


\threelineshloka
{किं नाम दानं नियमं तपो वाकुर्यामहं येन पतिर्ममाद्य}
{` लभेत भागं च तथैव सर्वंप्रभुर्विभुश्चाहवनीय एषः}
{'लभेत भागं भगवानचिन्त्योह्यर्धं तथा भागमथो तृतीयम्}


\twolineshloka
{एवं ब्रुवाणां भगवान्स्वपत्नींप्रहृष्टरूपः क्षुभितामुवाच}
{न वेत्सि मां देवि कृशोदराङ्गिकिं नाम युक्तं वचनं मखेशे}


\twolineshloka
{अहं विजानामि विशालनेत्रेध्यानेन हीना न विदन्त्यसन्तः}
{तवाद्य मोहेन च सेन्द्रदेवालोकास्त्रयः सर्वत एव मूढाः}


\threelineshloka
{मामध्वरे शंसितारः स्तुवन्तिरथन्तरं सामगाश्चोपगान्ति}
{मां ब्राह्मणा ब्रह्मविदो यजन्तेममाध्वर्यवः कल्पयन्ते च भागम् ॥देव्युवाच}
{}


\threelineshloka
{सुप्राकृतोऽपि पुरुषः सर्वः स्त्रीजनसंसदि}
{स्तौति गर्वायते चापि स्वमात्मानं न संशयः ॥श्रीभगवानुवाच}
{}


\twolineshloka
{नात्मानं स्तौमि देवेशि पश्य मे तनुमध्यमे}
{यं स्रक्ष्यामि वरारोहे यागार्थे वरवर्णिनि}


\twolineshloka
{इत्युक्त्वा भगवान्पत्नीमुमां प्राणैरपि प्रियाम्}
{सोऽसृजद्भगवान्वक्राद्भूतं घोरं प्रहर्षणम्}


\twolineshloka
{तमुवाचाक्षिप मखं दक्षस्येति महेश्वरः}
{ततो वक्राद्विमुक्तेन सिंहेनैकेन लीलया}


\twolineshloka
{देव्या मन्युव्यपोहार्थं हतो दक्षस्य वै क्रतुः}
{मन्युना च महाभीमा महाकाली महेश्वरी}


\twolineshloka
{आत्मनः कर्मसाक्षित्वे तेन सार्धं सहानुगा}
{देवस्यानुमतं मत्वा प्रणम्य शिरसा ततः}


\twolineshloka
{आत्मनः सदृशः शौर्याद्बलरूपसमन्वितः}
{स एव भगवान्क्रोधः प्रतिरूपसमन्वितः}


\twolineshloka
{अनन्तबलवीर्यश्च अनन्तबलपौरुषः}
{वीरभद्र इति ख्यातो देव्या मन्युप्रमार्जकः}


\twolineshloka
{सोऽसृजद्रोमकूषेभ्यो रौम्यान्नाम गणेश्वरान्}
{रुद्रतुल्या गणा रौद्रा रुद्रवीर्यपराक्रमाः}


\twolineshloka
{ते निपेतुस्ततस्तूर्णं दक्षयज्ञविहिंसया}
{भीमरूपा महाकायाः शतशोऽथ सहस्रशः}


\twolineshloka
{ततः किलकिलाशब्दैराकाशं पूरयन्ति च}
{तेन शब्देन महता त्रस्तास्तत्र दिवौकसः}


\twolineshloka
{पर्वताश्च व्यशीर्यन्त चकम्पे च वसुंधरा}
{मारुताश्चैव घूर्णन्ते चुक्षुभे वरुणालयः}


\twolineshloka
{अग्नयो नैव दीप्यन्ते नैव दीप्यति भास्करः}
{ग्रहा चैव प्रकाशन्ते नक्षत्राणि न चन्द्रमाः}


\twolineshloka
{ऋषयो न प्रकाशन्ते न देवा न च मानुषाः}
{एवं तु तिमिरीभूते निर्दहन्त्यपमानिताः}


\twolineshloka
{प्रहरन्त्यपरे घोरा यूपानुत्पाटयन्ति च}
{प्रमर्दन्ति तथा चान्ये विमर्दन्ति तथाऽपरे}


\twolineshloka
{आधावन्ति प्रधावन्ति वायुवेगा मनोजवाः}
{चूर्ण्यन्ते यज्ञपात्राणि दिव्यान्याभरणानि च}


\twolineshloka
{विशीर्यरमाणा दृश्यन्ते तारा इव नभस्तले}
{दिव्यान्नपानभक्ष्याणां राशयः पर्वतोपमाः}


\twolineshloka
{क्षीरनद्योऽथ दृश्यन्ते धृतपायसकर्दमाः}
{दधिमण्डेदका दिव्याः खण्डशर्करवालुकाः}


\twolineshloka
{षड्रसा निवहन्त्येता गुडकुल्या मनोरमाः}
{उच्चावचानि मांसानि भक्ष्याणि विविधानि च}


\twolineshloka
{पानकानि च दिव्यानि लेह्यचोष्याणि यानि च}
{भृञ्जते विविधैर्वक्रैर्विलुम्पन्त्याक्षिपन्ति च}


\twolineshloka
{रुद्रकोपान्महाकायाः कालाग्निसदृशोपमाः}
{क्षोभयन्सुरसैन्यानि भीक्षयन्तः समन्ततः}


\twolineshloka
{क्रीडन्ति विविधाकाराश्चिक्षिषुः सुरयोषितः}
{रुद्रक्रोधात्प्रयत्नेन सर्वदेवैः सुरक्षितम्}


\twolineshloka
{तं यज्ञमदहच्छीघ्नं रुद्रकर्मा समन्ततः}
{चकार भैरवं नादं सर्वभूतभयंकरम्}


\twolineshloka
{छित्त्वा शिरो वै यज्ञस्य ननाद च मुमोद च}
{ततो ब्रह्मादयो देवा दक्षश्चैव प्रजापतिः}


\threelineshloka
{ऊचुः प्राञ्जलयः सर्वे कथ्यतां को भवानिति}
{वीरभद्र उवाच}
{नाहं रुद्रो न वा देवी नैव भोक्तुमिहागतः}


\twolineshloka
{देव्या मन्युकृतं मत्वा क्रुद्धः सर्वात्मकः प्रभुः}
{द्रष्टुं वा नैव विप्रेन्द्रान्नैव कौतूहलेन वा}


\twolineshloka
{तव यज्ञविघातार्थं संप्राप्तं विद्धि मामिह}
{वीरभद्र इति ख्यातो रुद्रकोपाद्विनिःसृतः}


\twolineshloka
{भद्रकालीति विख्याता देव्याः कोपाद्विनिः सृता}
{प्रेषितौ देवदेवेन यज्ञान्तिकमिहागतौ}


\twolineshloka
{शरणं गच्छ विप्रेन्द्र देवदेवमुमापतिम्}
{वरं क्रोधोऽपि देवस्य वरदानं न चान्यतः}


\twolineshloka
{वीरभद्रवचः श्रुत्वा दक्षो धर्मभृतां वरः}
{तोषयामास स्तोत्रेण प्रणिपत्यं महेश्वरम्}


\twolineshloka
{प्रपद्ये देवमीशानं शाश्वतं ध्रुवमव्ययम्}
{महादेवं महात्मानं विश्वस्य जगतः पतिम्}


\twolineshloka
{दक्षप्रजापतेर्यज्ञेः द्रव्यैस्तैः सुसमाहितैः}
{आहूता देवताः सर्वा ऋषयश्च तपोधनाः}


\twolineshloka
{देवो नाहूयते तत्र विश्वकर्मा महेश्वरः}
{तत्र क्रुद्धा महादेवी गणांस्तत्र व्यसर्जयत्}


\twolineshloka
{प्रदीप्ते यज्ञवाटे तु विद्गुतेषु द्विजातिषु}
{तारागणमनुप्राप्ते रौद्रे दीप्ते महात्मनि}


\twolineshloka
{शूलनिर्भिन्नहृदयैः कूजद्भिः परिचारकैः}
{निखातोत्पाटितैर्यूरपविद्धैरितस्ततः}


\twolineshloka
{उत्पतद्भिः पतद्भिश्च गृध्रैरामिषगृद्धिभिः}
{पक्षवातविनिर्धूतैः शिवाशतनिनादितैः}


\twolineshloka
{यक्षगन्धर्वसङ्घैश्च पिशाचोरगराक्षमैः}
{प्राणापानौ संनिरुध्य वक्रस्थानेन यत्नतः}


\twolineshloka
{विचार्य सर्वतो दृष्टिं बहुदृष्टिरमित्रजित्}
{सहसा देवदेवेशो ह्यग्निकुण्डात्समुत्थितः}


\twolineshloka
{विभ्रत्सूर्यसहस्रस्य तेजः संवर्तकोपमः}
{स्मितं कृत्वाऽव्रवीद्वाक्यं ब्रूहि किं करवाणि ते}


\twolineshloka
{श्राविते च मखाध्याये देवानां गुरुणा ततः}
{तमुवाचाज्जलिं कृत्वा दक्षो देवं प्रजापतिः}


\twolineshloka
{भीतशङ्कितवित्रस्तः सवाष्पवदनेक्षणः}
{यदि प्रसन्नो भगवान्यदि चाहं भवत्प्रियः}


\twolineshloka
{यदि वाऽहभनुग्राह्यो यदि वा वरदो मम}
{यद्दग्धं भक्षितं पीतमशित्तं यच्च नाशितम्}


\threelineshloka
{चूर्णीकृतापविद्धं च यज्ञसंभारमीदृशम्}
{दीर्घकालेन महता प्रयत्नेन सुसंचितम्}
{तन्न मिथ्या भवेन्मह्यं वरमेतमहं वृणे}


\twolineshloka
{तथाऽस्त्वित्याह भगवान्भगनेत्रहरो हरः}
{धर्माध्यक्षो विरूपाक्षस्त्र्यक्षो देवः प्रजापतिः}


\threelineshloka
{जानुभ्यामवनीं गत्वा दक्षो लब्ध्वा भवाद्वरम्}
{नाम्नामष्टसहस्रेण स्तुतवान्वृषभध्वजम् ॥युधिष्ठिर उवाच}
{}


\threelineshloka
{यैर्नामघेयैः स्तुतवान्दक्षो देवं प्रजापतिः}
{वक्तुमर्हसि मे तात श्रोतुं श्रद्धा ममानघ ॥भीष्म उवाच}
{}


\twolineshloka
{श्रूयतां देवदेवस्य नामान्यद्भूतकर्मणः}
{गूढव्रतस्य गुह्यानि प्रकाशानि च भारत}


\twolineshloka
{नमस्ते देवदेवेश देवारिबलसूदन}
{देवेन्द्रबलविष्टम्भ देवदानवपूजित}


\twolineshloka
{सहस्राक्ष विरूपाक्ष त्र्यक्ष यक्षाधिपप्रिय}
{सर्वतः पाणिपादान्त सर्वतोक्षिशिरोमुखं}


\twolineshloka
{सर्वतः श्रुतिमंल्लोके सर्वमावृत्य तिष्ठसि}
{शङ्कुकर्ण महाकर्ण कुम्भकर्णार्णवालय}


\twolineshloka
{गजेन्द्रकर्ण गोकर्ण पाणिकर्ण नमोस्तु ते}
{शतोदर शतावर्त शतजिह्न नमोस्तु ते}


\twolineshloka
{गायन्ति त्वा गायत्रिणोऽर्चन्त्यर्कमर्किणः}
{ब्रह्माणं त्वा शतक्रतुमूर्ध्वं खमिव मेनिरे}


\twolineshloka
{मूर्तौ हि ते महामूर्ते समुद्राम्बरसन्निभ}
{सर्वा वै देवता ह्यस्मिन्गावो गोष्ठ इवासते}


\twolineshloka
{भवच्छरीरे पश्यामि सोममग्निं जलेश्वरम्}
{आदित्यमथ वै विष्णुं ब्रह्माणं च बृहस्पतिम्}


\twolineshloka
{भगवान्कारणं कार्यं क्रिया करणमेव च}
{असतश्च सतश्चैव तथैव प्रभवाप्ययौ}


\twolineshloka
{नमो भवाय शर्वाय रुद्राय वरदाय च}
{पशूनां पतये नित्यं नमोस्त्वन्धकघातिने}


\twolineshloka
{त्रिजटाय त्रिशीर्षाय त्रिशूलवरपाणिने}
{त्र्यम्बकाय त्रिनेत्राय त्रिपुरघ्नाय वै नमः}


\twolineshloka
{नमश्चण्डाय कृण्डाय अण्डायाण्डधराय च}
{दण्डिने समकर्णाय दण्डिमुण्डाय वै नमः}


\twolineshloka
{नमोर्ध्वदंष्ट्रकेशाय शुक्लायावतताय च}
{विलोहिताय धूम्राय नीलग्नीवाय वै नमः}


\twolineshloka
{नमोस्त्वप्रतिरूपाय विरूपाय शिवाय च}
{सूर्याय सूर्यमालाय सूर्यध्वजपताकिने}


\twolineshloka
{नमः प्रमथनाथाय वृषस्कन्धाय धन्विने}
{शत्रुंदमाय दण्डाय पर्णचीरपटाय च}


\twolineshloka
{नमो हिरण्यगर्भाय हिरण्यकवचाय च}
{हिरण्यकृतचूडाय हिरण्यपतये नमः}


\twolineshloka
{नमः स्तुताय स्तुत्याय स्तूयमानाय वै नमः}
{सर्वाय सर्वभक्षाय सर्वभूतान्तरात्मने}


\twolineshloka
{नमो होत्रेऽथ मन्त्राय शुक्लध्वजपताकिने}
{नमो नाभाय नाभ्याय नमः कटकटाय च}


\twolineshloka
{नमोस्तु कृशनासाय कृशाङ्गाय कृशाय च}
{संहृष्टाय विहृष्टाय नमः किलकिलाय च}


\twolineshloka
{नमोस्तु शयमानाय शयितायोत्थिताय च}
{स्थिताय धावमानाय मुण्डाय जटिलाय च}


\twolineshloka
{नमो नर्तनशीलाय मुखवादित्रवादिने}
{नाद्योपहारलुब्धाय गीतवादित्रशालिने}


\twolineshloka
{नमो ज्येष्ठाय श्रेष्ठाय वलप्रमथनाय च}
{कालनाथाय कल्याय क्षयायोपक्षयाय च}


\twolineshloka
{भीमदुन्दुभिहासाय भीमव्रतधराय च}
{उग्राय च नमो नित्यं नमोस्तु दशबाहवे}


\twolineshloka
{नमः कपालहस्ताय चितिभस्मप्रियाय च}
{विभीषणाय भीष्माय भीमव्रतधराय च}


\twolineshloka
{नमो विकृतवक्राय खङ्गजिह्वाय दंष्ट्रिणे}
{पक्वाममांसलुब्धाय तुम्बीवीणाप्रियाय च}


\twolineshloka
{नमो वृषाय वृष्याय गोवृषाय वृषाय च}
{कटंकटाय दण्डाय नमः पचपचाय च}


\twolineshloka
{नमः सर्ववरिष्ठाय वराय वरदाय च}
{वरमाल्यगन्धवस्त्राय वरातिवरदे नमः}


\twolineshloka
{नमो रक्तविरक्ताय भावनायाक्षमालिने}
{संभिन्नाय विभिन्नाय च्छायायातपनाय च}


\twolineshloka
{अघोरघोररूपाय घोरघोरतराय च}
{नमः शिवाय शान्ताय नमः शान्ततमाय च}


\twolineshloka
{एकपाद्वहुनेत्राय एकशीर्ष्णे नमोस्तु ते}
{रुद्राय क्षुद्रलुब्धाय संविभागप्रियाय च}


\twolineshloka
{पञ्चालाय सिताङ्गाय नमः शमशमाय च}
{नमश्चण्डिकघण्टाय घण्टायाघण्टघण्टिने}


\twolineshloka
{सहस्राध्मातघण्टाय घण्टामालाप्रियाय च}
{प्राणघण्टाय गन्धाय नमः कलकलाय च}


\twolineshloka
{हूंहूंहूंकारपाराय हूंहूंकारप्रियाय च}
{नमः शमशमे नित्यं गिरिवृक्षालयाया च}


\twolineshloka
{गर्भमांससृगालाय तारकाय तराय च}
{नमो यज्ञाय यजिने हुताय प्रहुताय च}


\twolineshloka
{यज्ञवाहाय दान्ताय तप्यायातपनाय च}
{नमस्तटाय तट्याय तटानां पतये नमः}


\twolineshloka
{अन्नदायान्नपतये नमस्त्वन्नभुजे तथा}
{नमः सहस्रशीर्षाय सहस्रचरणाय च}


\twolineshloka
{सहस्रोद्यतशूलाय सहस्रनयनाय च}
{नमो बालार्कवर्णाय बालरूपधराय च}


\twolineshloka
{बालानुचरगोप्ताय बालक्रीडनकाय च}
{नमोवृद्धाय लुब्धाय क्षुब्धाय क्षोभणाय च}


\twolineshloka
{तरङ्गाङ्कितकेशाय मुञ्जकेशाय वै नमः}
{नमः षट््कर्मतुष्टाय त्रिकर्मनिरताय च}


\twolineshloka
{वर्णाश्रमाणां विधिवत्पृथक्कर्मनिवर्तिने}
{नमो घुष्याय घोषाय नमः कलकलाय च}


\twolineshloka
{श्वेतपिङ्गलनेत्राय कृष्णरक्तेक्षणाय च}
{प्राणभग्नाय दण्डाय स्फोटनाय कृशाय च}


\twolineshloka
{धर्मकामार्थमोक्षाणां कथनीयकथाय च}
{साङ्ख्याय साङ्ख्यमुख्याय साङ्ख्ययोगप्रवर्तिने}


\twolineshloka
{नमो रथ्यविरथ्याय चतुष्पथरथाय च}
{कृष्णाजिनोत्तरीयाय व्यालयज्ञोपवीतिने}


\twolineshloka
{ईशानवज्रसंघातहरिकेश नमोस्तु ते}
{त्र्यम्बकाम्बिकनाथाय व्यक्ताव्यक्त नमोस्तु ते}


\twolineshloka
{काम कामद कामघ्न तृप्तातृप्तविचारिणे}
{सर्व सर्वद सर्वघ्न संन्ध्याराग नमोस्तु ते}


\threelineshloka
{`महाबल महाबाहो महासत्व महाद्युते}
{महामेघचलप्रख्य महाकाल नमोस्तु ते}
{स्थूलजीर्णाङ्गजटिले वत्कलाजिनधारिणे}


\twolineshloka
{दीप्तसूर्याग्निजटिने वत्कलाजिनवाससे}
{रसहस्रसूर्यप्रतिम तपोनित्य तमोस्तु ते}


\twolineshloka
{उन्मादनुशतावर्त गङ्गातोयार्द्रमूर्धज}
{चन्द्रवर्त युगावर्त मेघावर्त नमोस्तु ते}


\twolineshloka
{त्वमन्नमत्ता भोक्ता च अन्नदोऽन्नभुगेव च}
{अन्नस्रष्टा च पक्ता च पक्कभुक्पवनोऽनलः}


\twolineshloka
{जरायुजाण्डजाश्चैव स्वेदजाश्च तथोद्भिजाः}
{त्वमेव देवदेवेश भूतग्रामश्चतुर्विधः}


\twolineshloka
{चराचरस्य स्रष्टा त्वं प्रतिहर्ता तथैव च}
{त्वामाहुर्ब्रह्मविदुषो ब्रह्म ब्रह्मविदांवर}


\twolineshloka
{मनसः परमा योनिः खं वायुर्ज्योतिषां निधिः}
{ऋक्सामानि तथोङ्कारमाहुस्त्वां ब्रह्मवादिनः}


\twolineshloka
{हायिहायि हुवाहायि हावुहायि तथाऽसकृत्}
{गायन्ति त्वां सुरश्रेष्ठ सामगा ब्रह्मवादिनः}


\twolineshloka
{यजुर्मयो ऋङ्भयश्च त्वमाहुतिमयस्तथा}
{पठ्यसे स्तुतिभिश्चैव वेदोपनिषदां गणैः}


\twolineshloka
{ब्राह्मणाः क्षत्रिया वैश्याः शूद्रा वर्णावराश्च ये}
{त्वमेव मेघसङ्घाश्च विद्युत्स्तनितगर्जितः}


\twolineshloka
{संवत्सरस्त्वामुतवो मासो मासार्धमेव च}
{युगं निमेषाः काष्ठास्त्वं नक्षत्राणि ग्रहाः कलाः}


\twolineshloka
{वृक्षाणां ककुदोसि त्वं गिरीणां शिखराणि च}
{व्याघ्रो मृगाणां पततां तार्क्ष्योऽनन्तश्च भोगिनाम्}


\twolineshloka
{क्षीरादो ह्युदधीनां च यन्त्राणां धनुरेव च}
{वज्रः प्रहरणानां च व्रतानां सत्यमेव च}


\twolineshloka
{त्वमेव द्वेष इच्छा च रागो मोहः क्षमाक्षमे}
{व्यवसायो धृतिर्लोभः कामक्रोधौ जयाजयौ}


\twolineshloka
{त्वं गदी त्वं शरी चापी खट्वाङ्गी झर्झरी तथा}
{छेत्ता भेत्ता प्रहर्ता त्वं नेता मन्ता पिता मतः}


\twolineshloka
{दशलक्षणसंयुक्तो धर्मोऽर्थः काम एव च}
{गङ्गा समुद्राः सरितः पल्वलानि संरासि च}


\twolineshloka
{लता वल्ल्यस्तृणौषध्यः पशवो मृगपक्षिणः}
{द्रव्यकर्मसमारम्भः कालः पुष्पफलप्रदः}


\threelineshloka
{आदिश्चान्तश्च देवानां गायत्र्योङ्कार एव च}
{हरितो रोहितो नीलः कृष्णो रक्तस्तथाऽरुणः}
{कद्रुश्च कपिलश्चैव कपोतो मेचकस्तथा}


\twolineshloka
{अवर्णश्च सुवर्णश्च वर्णकारो घनोपमः}
{सुवर्णनामा च तथा सुवर्णप्रिय एव च}


\twolineshloka
{त्वमिन्द्रश्च यमश्चैव वरुणो धनदोऽनलः}
{उपप्लवश्चित्रभानुः स्वर्भानुर्भानुरेव च}


\twolineshloka
{होत्रं होता च होम्यं च हुतं चैव तथा प्रभुः}
{त्रिसौपर्णं तथा ब्रह्म यजुषां शतरुद्रियम्}


\twolineshloka
{पवित्रं च पवित्राणां मङ्गलानां च मङ्गलम्}
{गिरिको हिण्डुको वृक्षो जीवः पुद्गल एव च}


\twolineshloka
{प्राणः सत्त्वं रजश्चैव तमश्चाप्रमदस्तथा}
{प्राणोपानः समानश्च उदानो व्यान एव च}


\twolineshloka
{उन्मेषश्च निमेषश्च क्षुतं जृम्भितमेव च}
{लोहितान्तर्गता दृष्टिर्महावक्रो महोदरः}


\twolineshloka
{सूचीरोमा हरिश्मश्रुरूर्ध्वकेशश्चलाचलः}
{गीतवादित्रतत्त्वज्ञो गीतवादनकप्रियः}


\twolineshloka
{मत्स्यो जलचरो जाल्योऽकलः केलिकलः कलिः}
{अकालश्चातिकालश्च दुष्कालः काल एव च}


\twolineshloka
{मृत्युः क्षुरश्च कृत्यश्च पक्षोऽपक्षक्षयंकरः}
{मेघकालो महादंष्ट्रः संवर्तकबलाहकः}


\twolineshloka
{घण्टोऽघण्टो घटी घण्टी चरुचेली मिलीमिली}
{ब्रह्मकायिकमग्नीनां दण्डी मुण्डस्त्रिदण्डधृक्}


\twolineshloka
{चतुर्युगश्चतुर्वेदश्चातुर्होत्रप्रवर्तकः}
{चातुराश्रम्यवेता च चातुर्वर्ण्यकरश्च यः}


\twolineshloka
{सदा चाक्षप्रियो धूर्तो गणाध्यक्षो गणाधिपः}
{रक्तमाल्याम्बरघरो गिरिशो गिरिकप्रियः}


\twolineshloka
{शिल्पिकः शिल्पिनांश्रेष्ठः सर्वशिल्पप्रवर्तकः}
{भगनेत्राङ्कुशश्चण्डः पूष्णो दन्तविनाशनः}


\twolineshloka
{स्वाहास्वधावषट््कारो नमस्कारो नमो नमः}
{गूढव्रतो गुह्यतपास्तारकस्तारकामयः}


\twolineshloka
{धाता विधाता संधाता विधाता धारणो धरः}
{ब्रह्मा तपश्च सत्यं च ब्रह्मचर्यमथार्जवम्}


\twolineshloka
{भूतात्मा भूतकृद्भूतो भूतभव्यवोद्भवः}
{भूर्भुवः स्वरितश्चैव ध्रुवो दान्तो महेश्वरः}


\twolineshloka
{दीक्षितोऽदीक्षितः क्षान्तो दुर्दान्तोऽदान्तनाशनः}
{चन्द्रावर्तयुगावर्तः संवर्तः संप्रवर्तकः}


\twolineshloka
{कामो विन्दुरणुः स्थूलः कर्णिकारस्रजप्रियः}
{नन्दीमुखो भीममुखः सुमुखो दुर्मुखोऽमुखः}


\twolineshloka
{चतुर्मुखो बहुमुखो रणेष्वग्निमुखस्तथा}
{हिरण्यगर्भः शकुनिर्महोरगपतिर्विराट्}


\twolineshloka
{अधर्महा महापार्श्वश्चण्डधारो गणाधिपः}
{गोनर्दो गोप्रतारश्च गोवृषेश्वरवाहनः}


\twolineshloka
{त्रैलोक्यगोप्ता गोविन्दो गोमार्गोऽमार्ग एव च}
{श्रेष्ठः स्थिरश्च स्थाणुश्च निष्कम्पः कम्प एव च}


\twolineshloka
{दुर्वारणो दुर्विषहो दुःसहो दुरतिक्रमः}
{दुर्धर्पो दुष्प्रकम्पश्च दुर्विषो दुर्जयो जयः}


\twolineshloka
{शशः शशाङ्कः शमनः शीतोष्णक्षुज्जराधिकृत्}
{आधयो व्याधयश्चैव व्याधिहा व्याधिरेव च}


\twolineshloka
{मम यज्ञमृगव्याधो व्याधीनामागमो गमः}
{शिखण्डी पुण्डरीकाक्षः पुण्डरीकवनालयः}


\twolineshloka
{दण्डधारस्त्र्यम्बकश्च उग्रदण्डोऽण्डनाशनः}
{विषाग्निपाः सुरश्रेष्ठः सोमपास्त्वं मरुत्पतिः}


\threelineshloka
{अमृतपास्त्वं जगन्नाथ देवदेव गणेश्वरः}
{विषाग्निपा मृत्युपाश्च क्षीरपाः सोमपास्तथा}
{मधुश्च्युतानामग्रपास्त्वं त्वमेव तुषिताद्यपाः}


\twolineshloka
{हिरण्यरेताः पुरुषस्त्वमेवत्वं स्त्री पुमांस्त्वं च नपुंसकं च}
{बालो युवा स्थविरो जीर्णदंष्ट्रस्त्वंनागेन्द्र शक्रस्त्वं विश्वकृद्विश्वकर्ता}


\threelineshloka
{विश्वकृद्विश्वकृतां वरेण्यस्त्वं विश्वबाहोविश्वरूपस्तेजस्वी विश्वतोमुखः}
{चन्द्रादित्यौ चक्षुषी ते हृदयं च पितामहः}
{}


% Check verse!
महोदधिः सरस्वती वाग्बलमनलोऽनिलः अहोरात्रं निमेषोन्मेषकर्मा
\twolineshloka
{न ब्रह्मा न च गोविन्दः पौराणा ऋषयो न ते}
{माहात्म्यं वेदितुं शक्ता याथातथ्येन ते शिव}


\twolineshloka
{या मूर्तयः सुसूक्ष्मास्ते न मह्यं यान्ति दर्शनम्}
{त्राहि मां सततं रक्ष पिता पुत्रमिवौरसम्}


\twolineshloka
{रक्ष मां रक्षणीयोऽहं तवानघ नमोस्तु ते}
{भक्तानुकम्पी भगवान्भक्तश्चाहं सदा त्वयि}


\twolineshloka
{यः सहस्राण्यनेकापि पुंसामावृत्य दुर्दृशः}
{तिष्ठत्येकः समुद्रान्ते स मे गोप्ताऽस्तु नित्यशः}


\twolineshloka
{यं विनिद्रा जितश्वासाः सत्वस्थाः संयतेन्द्रियाः}
{ज्योतिः पश्यन्ति युञ्जानास्तस्मै योगात्मने नमः}


\twolineshloka
{जटिले दण्डिने नित्यं लम्बोदरशरीरिणे}
{कमण्डलुनिषङ्गाय तस्मै ब्रह्मात्मने नमः}


\twolineshloka
{यस्य केशेषु जीमूता नद्यः सर्वाङ्गसन्धिषु}
{कुक्षौ समुद्राश्चत्वारस्तस्मै तोयात्मने नमः}


\twolineshloka
{संभक्ष्य सर्वभूतानि युगान्ते पर्युपस्थिते}
{यः शेते जलमध्यस्थस्तं प्रपद्येऽम्बुशायिनम्}


\twolineshloka
{प्रविश्य वदनं राहोर्यः सोमं पिबते निशि}
{ग्रसत्यर्कं च स्वर्भानुर्भूत्वा मां सोऽभिरक्षतु}


\twolineshloka
{ये चानुपतिता गर्भा यथा भागानुपासते}
{नमस्तेभ्यः स्वधा स्वाहा प्राप्नुवन्तु मुदं तु ते}


\twolineshloka
{येऽङ्गुष्ठमात्राः पुरुषा देहस्थाः सर्वदेहिनाम्}
{रक्षन्तु ते हि मां नित्यं नित्यं चाप्याययन्तु माम्}


\twolineshloka
{येन रोदन्ति देहस्था देहिनो रोदयन्ति च}
{हर्षयन्ति न हृष्यन्ति नमस्तेभ्योऽस्तु नित्यशः}


\twolineshloka
{ये नदीषु समुद्रेषु पर्वतेषु गुहासु च}
{वृक्षमूलेषु गोष्ठेषु कान्तारे गहनेषु च}


\twolineshloka
{चतुष्पथेषु रथ्यासु चत्वरेषु तटेषु च}
{हस्त्यश्वरथशालासु जीर्णोद्यानालयेषु च}


\twolineshloka
{येषु पञ्चसु भूतेषु दिशासु विदिशासु च}
{चन्द्रार्कयोर्मध्यगता ये च चन्द्रार्करश्मिषु}


\twolineshloka
{रसातलगता ये च ये च तस्मै परं गताः}
{नमस्तेभ्यो नमस्तेभ्यो नमस्तेभ्योस्तु नित्यशः}


\twolineshloka
{येषां न विद्यते सङ्ख्या प्रमाणं रुपमेव च}
{असंख्येयगुणा रुद्रा नमस्तेभ्योस्तु नित्यशः}


\twolineshloka
{सर्वभूतकरो यस्मात्सर्वभूतपतिर्हरः}
{सर्वभूतान्तरात्मा च तेन त्वं न निमन्त्रितः}


\twolineshloka
{त्वमेव हीज्यसे यस्माद्यज्ञैर्विविधदक्षिणैः}
{त्वमेव कर्ता सर्वस्य तेन त्वं न निमन्त्रितः}


\twolineshloka
{अथवा मायया देव सूक्ष्मया तव मोहितः}
{एतस्मात्कारणाद्वाऽपि तेन त्वं न निमन्त्रितः}


\twolineshloka
{प्रसीद मम भद्रं ते भव भावगतस्य मे}
{त्वयि मे हृदयं देव त्वयि बुद्धिर्मनस्त्वयि}


\twolineshloka
{स्तुत्वैवं स महादेवं विरराम प्रजापतिः}
{भगवानपि सुप्रीतः पुनर्दक्षमभाषत}


\twolineshloka
{परितुष्टोऽस्मि ते दक्ष स्तवेनानेन सुव्रत}
{बहुनात्र किमुक्तेन मत्समीपे भविष्यसि}


\twolineshloka
{अश्वमेधसहस्रस्य वाजपेयशतस्य च}
{प्रजापते मत्प्रसादात्फलभागी भविष्यसि}


\twolineshloka
{अथैनमब्रवीद्वाक्यं लोकस्याधिपतिर्भवः}
{आश्वासनकरं वाक्यं वाक्यविद्वाक्यसंमितम्}


\twolineshloka
{दक्ष दक्ष न कर्तव्यो मन्युर्विघ्नमिमं प्रति}
{अहं यज्ञहरस्तुभ्यं दृष्टमेतत्पुरातनम्}


\twolineshloka
{भूयश्च ते वरं दद्मि तं त्वं गृह्णीष्व सुव्रत}
{प्रसन्नवदनो भूत्वा तदिहैकमनाः शृणु}


\twolineshloka
{वेदात्षडङ्गादुद्धृत्य साङ्ख्ययोगाच्च युक्तितः}
{तपः सुतप्तं विपुलं दुश्चरं देवदानवैः}


\twolineshloka
{अपूर्वं सर्वतोभद्रं सर्वतोमुखमव्ययम्}
{अब्दैर्दशाहसंयुक्तं गूढमप्राज्ञनिन्दितम्}


\twolineshloka
{वर्णाश्रमकृतैर्धर्मैर्विपरीतं क्वचित्समम्}
{गतान्तैरध्यवसितमत्याश्रममिदं व्रतम्}


\twolineshloka
{मया पाशुपतं दक्ष शुभमुत्पादितं पुरा}
{तस्य चीर्णस्य तत्सम्यक्फलं भवति पुष्कलम्}


\threelineshloka
{तच्चास्तु ते महाभाग त्यज्यतां मानसो ज्वरः}
{एवमुक्त्वा महादेवः सपत्नीकः सहानुगः}
{अदर्शनमनुप्राप्तो दक्षस्यामितविक्रमः}


\twolineshloka
{दक्षप्रोक्तं स्तवमिमं कीर्तयेद्यः शृणोति वा}
{नाशुभं प्राप्नुयात्किंचिद्दीर्घमायुरवाप्नुयात्}


\twolineshloka
{यथा सर्वेषु देवेषु वरिष्ठो भगवाञ्छिवः}
{तथा स्तवो वरिष्ठोऽयं स्तवानां ब्रह्मसंमितः}


\twolineshloka
{यशोराज्यसुखैश्वर्यकामार्थधनकाङ्क्षिभिः}
{श्रोतव्यो भक्तिमास्थाय विद्याकामैश्च यत्नतः}


\twolineshloka
{व्याधितो दुःखितो दीनश्चोरग्रस्तो भयार्दितः}
{राजकार्याभियुक्तो वा मुच्यते महतो भयात्}


\twolineshloka
{अनेनैव तु देहेन गणानां समतां व्रजेत्}
{तेजसा यशसा चैव युक्तो भवति निर्मलः}


\twolineshloka
{न राक्षसाः पिशाचा वा न भूता न विनायकाः}
{विघ्नं कुर्युर्गृहे तस्य यत्रायं पठ्यते स्तवः}


\twolineshloka
{शृणुयाच्चैव या नारी तद्भक्ता ब्रह्मचारिणी}
{पितृपक्षे मातृपक्षे पूज्या भवति देववत्}


\twolineshloka
{शृणुयाद्यः स्तवं कृत्स्नं कीर्तयेद्वा समाहितः}
{तस्य सर्वाणि कर्माणि सिद्धिं गच्छन्त्यभीक्ष्णशः}


\twolineshloka
{मनसा चिन्तितं यच्च यच्च वाचाऽनुकीर्तितम्}
{सर्वं संपद्यते तस्य स्तवस्यास्यानुकीर्तनाम्}


\twolineshloka
{देवस्य च गुहस्यापि देव्या नन्दीश्वरस्य च}
{बलिं सुविहितं कृत्वा दमेन नियमेन च}


\twolineshloka
{ततस्तु युक्तो गृह्णीयान्नामान्याशु यथाक्रमम्}
{ईप्सिताँल्लभते सोर्थान्भोगान्कामांश्च मानवः}


\twolineshloka
{मृतश्च स्वर्गमाप्नोति तिर्यक्षु च न जायते}
{इत्याह भगवान्व्यासः पराशरसुतः प्रभुः}


\chapter{अध्यायः २९१}
\twolineshloka
{* युधिष्ठिर उवाच}
{}


\threelineshloka
{अध्यात्मं नाम यदिदं पुरुषस्येह विद्यते}
{यदध्यात्मं यतश्चैव तन्मे ब्रूहि पितामह ॥भीष्म उवाच}
{}


\twolineshloka
{सर्वज्ञानं परं बुद्ध्या यन्मां त्वमनुपृच्छसि}
{तद्व्याख्यास्यामि ते तात तस्य व्याख्यामिमां शृणु}


\twolineshloka
{पृथिवी वायुराकाशमापो ज्योतिश्च पञ्चमम्}
{महाभूतानि भूतानां सर्वेषां प्रभवाप्ययौ}


\twolineshloka
{स तेषां गुणसंघातः शरीरं भरतर्षभ}
{सततं हि प्रलीयन्ते गुणास्ते प्रभवन्ति च}


\twolineshloka
{ततः सृष्टानि भूतानि तानि यान्ति पुनः पुनः}
{महाभूतानि भूतेभ्य ऊर्मयः सागरे यथा}


\twolineshloka
{प्रसारयित्वेहाङ्गानि कूर्मः संहरते यथा}
{तद्वद्भूतानि भूतानामल्पीयांसि स्थवीयसाम्}


\twolineshloka
{आकाशात्खलु यो घोषः संघातस्तु महीगुणः}
{वायोः प्राणो रसस्त्वद्भ्यो रूपं तेजस उच्यते}


\twolineshloka
{इत्येतन्मयमेवैतत्सर्वं स्थावरजङ्गमम्}
{प्रलये च तमभ्येति तस्मादुद्दिश्यते पुनः}


\twolineshloka
{महाभूतानि पञ्चैव सर्वभूतेषु भूतकृत्}
{विषयान्कल्पयामास यस्मिन्यदनुपश्यति}


\twolineshloka
{शब्दश्रोत्रे तथा खानि त्रयमाकाशयोनिजम्}
{रसः स्नेहश्च जिह्वा च अपामेते गुणाः स्मृताः}


\twolineshloka
{रूपं चक्षुर्विपाकश्च त्रिविधं ज्योतिरूच्यते}
{घ्रेयं घ्राणं शरीरं च एते भूमिगुणाः स्मृताः}


\twolineshloka
{प्राणः स्पर्शश्च चेष्टा च वायोरेते गुणाः स्मृताः}
{इति सर्वगुणा राजन्व्याख्याताः पाञ्चभौतिकाः}


\twolineshloka
{सत्त्वं रजस्तमः कालः कर्म बुद्धिश्च भारत}
{मनः षष्ठानि चैतेषु ईश्वरः समकल्पयत्}


\twolineshloka
{यदूर्ध्वपादतलयोरवाड्यूर्ध्नश्च पश्यसि}
{एतस्मिन्नेव कृत्स्नेयं वर्तते बुद्धिरन्तरे}


\twolineshloka
{इन्द्रियाणि नरे पञ्च षष्ठं तु मन उच्यते}
{सप्तमीं बुद्धिमेवाहुः क्षेत्रज्ञः पुनरष्टमः}


\twolineshloka
{इन्द्रियाणि च कर्ता च विचेतव्यानि भागशः}
{तमः सत्वं रजस्तैव तेऽपि भावास्तदाश्रयाः}


\twolineshloka
{चक्षुरालोचनायैव संशयं कुरुते मनः}
{बुद्धिरध्यवसानाय साक्षी क्षेत्रज्ञ उच्यते}


\threelineshloka
{तमः सत्वं रजश्चेति कालः कर्म च भारत}
{गुणैर्नेनीयते बुद्धिर्बुद्धिरेवेन्द्रियाणि च}
{मनः षष्ठानि सर्वाणि बुद्ध्यभावे कुतो गुणाः}


\twolineshloka
{येन पश्यति तच्चक्षुः शृण्वती श्रोत्रमुच्यते}
{जिघ्रती भवति घ्राणं रसती रसना रसान्}


\twolineshloka
{स्पर्शनं स्पर्शती स्पर्शान्बुद्धिर्विक्रियतेऽसकृत्}
{यदा प्रार्थयते किंचित्तदा भवति सा मनः}


\twolineshloka
{अधिष्ठानानि बुद्ध्या हि पृथगेतानि पञ्चधा}
{इन्द्रियाणीति तान्याहुस्तेषु दुष्टेषु दुष्यति}


\twolineshloka
{पुरुषे तिष्ठती बुद्धिस्त्रिषु भावेषु वर्तते}
{कदाचिल्लभते प्रीतिं कदाचिदपि शोचति}


\twolineshloka
{न सुखेन न दुःखेन कदाचिदपि वर्तते}
{सेयं भावात्मिका भावांस्त्रीनेतान्परिवर्तते}


\twolineshloka
{सरितां सागरो भर्ता यथा वेलामिवोर्मिमान्}
{इति भावगता बुद्धिर्भावे मनसि वर्तते}


\twolineshloka
{प्रवर्तमानं तु रजस्तद्भावेनानुर्तते}
{प्रहर्षः प्रीतिरानन्दः सुखं संशान्तचित्तता}


\twolineshloka
{कथंचिदुपपद्यन्ते पुरुषे सात्विका गुणाः}
{पिरदाहस्तथा शोकः संतापोऽपूर्तिरक्षमा}


\twolineshloka
{लिङ्गानि रजसस्तानि दृश्यन्ते हेत्वहेतुभिः}
{अविद्या रागमोहौ च प्रमादः स्तब्धता भयम्}


\twolineshloka
{असमृद्धिस्तथा दैन्यं प्रमोहः स्वप्नतन्द्रिता}
{कथंचिदुपवर्तन्ते विविधास्तामसा गुणाः}


\twolineshloka
{तत्र यत्प्रीतिसंयुक्तं काये मनसि वा भवेत्}
{वर्तते सात्विको भाव इत्युपेक्षेत तत्तथा}


\twolineshloka
{अथ यद्दुःखसंयुक्तमप्रीतिकरमात्मनः}
{प्रवृत्तं रज इत्येव तदसंरभ्य चिन्तयेत्}


\twolineshloka
{अथ यन्मोहसंयुक्तं काये मनसि वा भवेत्}
{अप्रतर्क्यमविज्ञेयं तमस्तदुपधारयेत्}


\twolineshloka
{इति बुद्धिगतीः सर्वा व्याख्याता यावतीरिह}
{एतद्बुद्ध्वा भवेद्बुद्धः किमन्यद्बुद्धलक्षणम्}


\twolineshloka
{सत्वक्षेत्रज्ञयोरेतदन्तरं विद्धि सूक्ष्मयोः}
{सृजतेऽत्र गुणानेक एको न सृजते गुणान्}


\twolineshloka
{पृथग्भूतौ प्रकृत्या तु संप्रयुक्तौ च सर्वदा}
{यथा मत्स्योऽद्भिरन्यः स्यात्संप्रयुक्तो भवेत्तथा}


\twolineshloka
{न गुणा विदुरात्मानं स गुणान्वेद सर्वतः}
{परिद्रष्टा गुणानां तु संस्रष्टा मन्यते यथा}


\twolineshloka
{आश्रयो नास्ति सत्वस्य गुणसर्गेण चेतना}
{सत्वमस्य सृजन्त्यन्ये गुणान्वेद कदाचन}


\twolineshloka
{सृजते हि गुणान्सत्वं क्षेत्रज्ञः परिपश्यति}
{संप्रयोगस्तयोरेष सत्वक्षेत्रज्ञयोर्ध्रुवः}


\twolineshloka
{इन्द्रियैस्तु प्रदीपार्थं क्रियते बुद्धिरन्तरा}
{निश्चक्षुर्भिरजानद्भिरिन्द्रियाणि प्रदीपवत्}


\twolineshloka
{एवं स्वभावमेवैतत्तद्बुद्ध्वा विहरन्नरः}
{अशोचन्नप्रहृष्यंश्च स वै विगतमत्सरः}


\twolineshloka
{स्वभावसिद्धमेवैतद्यदिमान्सृजते गुणान्}
{ऊर्णनाभिर्यथा सूत्रं विज्ञेयास्तन्तुवद्गुणाः}


\twolineshloka
{प्रध्वस्ता न निवर्तन्ते प्रवृत्तिर्नोपलभ्यते}
{एवमेके व्यवस्यन्ति निवृत्तिरिति चापरे}


\twolineshloka
{इतीदं हृदयग्रन्थिं बुद्धिचिन्तामयं दृढम्}
{विमुच्य सुखमासीत विशोकश्छिन्नसंशयः}


\twolineshloka
{ताम्येयुः प्रच्युताः पृथ्वीं मोहपूर्णां नदीं नराः}
{यथा गाधमविद्वांसो बुद्धियोगमयं तथा}


\twolineshloka
{नैव ताम्यन्ति विद्वांसः प्लवन्तः पारमम्भसः}
{अध्यात्मविदुषो धीरा ज्ञानं तु परमं प्लवः}


\twolineshloka
{न भवति विदुषां महद्भयंयदविदुषां सुमहद्भयं भवेत्}
{न हि गतिरधिकाऽस्ति कस्यचित्सकृदुपदर्शयतीह तुल्यताम्}


\twolineshloka
{यत्करोति बहुदोषमेकतस्तच्च दूषयति यत्पुरा कृतम्}
{नाप्रियं तदुभयं करोत्यसौयच्च दूषयति यत्करोति च}


\chapter{अध्यायः २९२}
\twolineshloka
{युधिष्ठिर उवाच}
{}


\threelineshloka
{शोकाद्दुःखाच्च मृत्योश्च त्रसन्ते प्राणिनः सदा}
{उभयं नो यथा न स्यात्तन्मे ब्रूहि पितामह ॥भीष्म उवाच}
{}


\threelineshloka
{अत्राप्युदाहरन्तीममितिहासं पुरातनम्}
{नारदस्य च संवादं समङ्गस्य च भारत ॥नारद उवाच}
{}


\twolineshloka
{उरसेव प्रणमसे बाहुभ्यां तरसीव च}
{संप्रहृष्टमना नित्यं विशोक इव लक्ष्यसे}


\threelineshloka
{उद्वेगं न हि ते किंचित्सुसूक्ष्ममपि लक्षये}
{नित्यतृप्त इव स्वस्थो बालवच्च विचेष्टसे ॥समङ्ग उवाच}
{}


\twolineshloka
{भूतं भव्यं भविष्यच्च सर्वभूतेषु नारद}
{तेषां तत्त्वानि जानामि ततो न विमना ह्यहम्}


\twolineshloka
{उपक्रमानहं वेद पुनरेव फलोदयान्}
{लोके फलानि चित्राणि ततो न विमना ह्यहम्}


\twolineshloka
{अनाथाश्चाप्रतिष्ठाश्च गतिमन्तश्च नारद}
{अन्धा जडाश्च जीवन्ति पश्यास्मानपि जीवतः}


\twolineshloka
{विहितेनैव जीवन्ति अरोगाङ्गा दिवौकसः}
{बलवन्तोऽबलाश्चैव तद्वदस्मान्सभाजय}


\twolineshloka
{सहस्रिणोऽपि जीवन्ति जीवन्ति शतिनस्तथा}
{शाकेन चान्ये जीवन्ति पश्यास्मानपि जीवतः}


\twolineshloka
{यदा न शोचेमहि किं नु नः स्याद्धर्मेण वा नारद कर्मणा वा}
{कृतान्तवश्यानि यदा सुखानिदुःखानि वा यन्न विधर्षयन्ति}


\twolineshloka
{यस्मै प्राज्ञाः कथयन्ते मनुष्याःप्रज्ञामूलं हीन्द्रियाणां प्रसादः}
{मुह्यन्ति शोचन्ति तथेन्द्रियाणिप्रज्ञालाभो नास्ति मूढेन्द्रियस्य}


\twolineshloka
{मूढस्य दर्पः स पुनर्मोह एवमूढस्य नायं न परोऽस्ति लोकः}
{न ह्येव दुःखानि सदा भवन्तिसुखस्य वा नित्यशो लाभ एव}


\twolineshloka
{भावात्मकं संपरिवर्तमानंन मादृशः संज्वरं जातु कुर्यात्}
{इष्टान्भोगान्नानुरुध्येत्सुखं वान चिन्तयेद्दुःखमभ्यागतं वा}


\twolineshloka
{समाहितो न स्पृहयेत्परेषांनानागतं चाभिनन्देच्च लाभम्}
{न चापि हृष्येद्विपुलेऽर्थलाभेतथाऽर्थनाशे च न वै विषीदेत्}


\twolineshloka
{न बान्धवा न च वित्तं न कौल्यंन च श्रुतं न च मन्त्रा न वीर्यम्}
{दुःखान्त्रातुं सर्व एवोत्सहन्तेपरत्र शीलेन तु यान्ति शान्तिम्}


\twolineshloka
{नास्ति बुद्धिरयुक्तस्य नायोगाद्विन्दते सुखम्}
{धृतिश्च दुःखत्यागश्चेत्युभयं तु सुखं नृप}


\twolineshloka
{प्रियं हि हर्षजननं हर्ष उत्सेकवर्धनः}
{उत्सेको नरकायैव तस्मात्तान्संत्यजाम्यहम्}


\twolineshloka
{एताञ्शोकभयोत्सेकान्मोहनान्सुखदुःखयोः}
{पश्यामि साक्षिवल्लोके देहस्यास्य विचेष्टनात्}


\twolineshloka
{अर्थकामौ परित्यज्य विशोको विगतज्वरः}
{तृष्णामोहौ तु संत्यज्य चरामि पृथिवीमिमाम्}


\twolineshloka
{न च मृत्योर्न चाधर्मान्न लोभान्न कुतश्चन}
{पीतामृतस्येवात्यन्तमिह वामुत्र वा भयम्}


\twolineshloka
{एतद्ब्रह्मन्विजानामि महत्कृत्वा तपोऽव्ययम्}
{तेन नारद संप्राप्तो न मां शोकः प्रबाधते}


\chapter{अध्यायः २९३}
\twolineshloka
{युधिष्ठिर उवाच}
{}


\threelineshloka
{अतत्त्वज्ञस्य शास्त्राणां संततं संशयात्मनः}
{अकृतव्यवसायस्य श्रेयो ब्रूहि पितामह ॥भीष्म उवाच}
{}


\twolineshloka
{गुरुपूजा च सततं वृद्धानां पर्युपासनम्}
{श्रवणं चैव विद्यानां कूटस्थं श्रेय उच्यते}


\twolineshloka
{अत्राप्युदाहरन्तीममितिहासं पुरातनम्}
{गालवस्य च संवादं देवर्षेर्नारदस्य च}


\twolineshloka
{वीतमोहक्लमं विप्रं ज्ञानतृप्तं जितेन्द्रियः}
{श्रेयस्कामो यतान्मानं नारदं गालवोऽब्रवीत्}


\twolineshloka
{यैः कैश्चित्संमतो लोके गुणैश्च पुरुषो नृषु}
{भवत्यनपगान्सर्वांस्तान्गुणाँल्लक्षयामहे}


\twolineshloka
{भवानेवंविधोऽस्माकं संशयं छेत्तुमर्हति}
{अमूढश्चिरमूढानां लोकतत्त्वमजानताम्}


\twolineshloka
{ज्ञाने ह्येवं प्रवृत्तिः स्यात्कार्याणामविशेषतः}
{यत्कार्यं न व्यवस्यामस्तद्भवान्वक्तुमर्हति}


\twolineshloka
{भगवन्नाश्रमाः सर्वे पृथगाचारदर्शिनः}
{इदं श्रेय इदं श्रेय इति सर्वे प्रबोधिताः}


\twolineshloka
{तांस्तु विप्रस्थितानदृष्ट्वा शास्त्रैः शास्त्राभिनन्दिनः}
{स्वशास्त्रैः पिरतुष्टाश्च श्रेयो नोपलभामहे}


\twolineshloka
{शास्त्रं यदि भवेदेकं श्रेयो व्यक्तं भवेत्तदा}
{शास्त्रैश्च बहुभिर्भूयः श्रेयो गुह्यं प्रवेशितम्}


\threelineshloka
{एतस्मात्कारणाच्छ्रेयो गहनं प्रतिभाति मे}
{ब्रवीतु भगवांस्तन्मे उपसन्नोस्म्यधीहि भो ॥नारद उवाच}
{}


\twolineshloka
{आश्रमास्तात चत्वारो यथा संकल्पिताः पृथक्}
{तान्सर्वाननुपश्य त्वं समाश्रित्यैव गालव}


\twolineshloka
{तेषांतेषां तथाहि त्वमाश्रमाणां ततस्ततः}
{नानारूपं गुणोद्देशं पश्य विप्रस्थितं पृथक्}


\twolineshloka
{न यान्ति चैव ते सम्यगभिप्रेतमसंशयम्}
{अन्येऽपश्यंस्तथा सम्यगाश्रमाणां परां गतिम्}


\twolineshloka
{यत्तु निःश्रेयसं सम्यक्तच्चैवासंशयात्मकम्}
{अनुग्रहं च मित्राणाममित्राणां च निग्रहम्}


\twolineshloka
{संग्रहं च त्रिवर्गस्य श्रेय आहुर्मनीषिणः}
{निवृत्तिः कर्मणः पापात्सततं पुण्यशीलता}


\twolineshloka
{सद्भिश्च समुदाचारः श्रेय एतदसंशयम्}
{मार्दवं सर्वभूतेषु व्यवहारेषु चार्जवम्}


\twolineshloka
{वाक्चैव मधुरा प्रोक्ता श्रेय एतदसंशयम्}
{देवताभ्यः पितृभ्यश्च संविभागोऽतिथिष्वपि}


\twolineshloka
{असंत्यागश्च भूत्यानां श्रेय एतदसंशयम्}
{सत्यस्य वचनं श्रेयः सत्यज्ञानं तु दुष्करम्}


\twolineshloka
{यद्भूतहितमत्यन्तमेतत्सत्यं ब्रवीम्यहम्}
{अहंकारस्य च त्यागः प्रमादस्य च निग्रहः}


\twolineshloka
{संतोषश्चैकचर्या च कूटस्थं श्रेय उच्यते}
{धर्मेण वेदाध्ययनं वेदाङ्गानां तथैव च}


\twolineshloka
{ज्ञानार्थानां च जिज्ञासा श्रेय एतदसंशयम्}
{शब्दरूपरसस्पर्शान्सह गन्धेन केवलान्}


% Check verse!
नात्यर्थमुपसेवेत श्रेयसोर्थी कथंचन
\twolineshloka
{नक्तंचर्यां दिवास्वप्नमालस्यं पैशुनं मदम्}
{अतियोगमयोगं च श्रेयसोर्थी परित्यजेत्}


\twolineshloka
{आत्मोत्कर्षं न मार्गेत परेषां परिनिन्दया}
{स्वगुणैरेव मार्गेति विप्रकर्षं पृथग्जनात्}


\twolineshloka
{निर्गुणास्त्वेव भूयिष्ठमात्मसंभाविता नराः}
{दोषैरन्यान्गुणवतः क्षिपन्त्यात्मगुणक्षयात्}


\twolineshloka
{अनूच्यमानास्तु पुनस्ते मन्यन्ते महाजनात्}
{गुणवत्तरमात्मानं स्वेन मानेन दर्पिताः}


\twolineshloka
{अब्रुवन्कस्यचिन्निन्दामात्मपूजामवर्णयन्}
{विपश्चिद्गुणसंपन्नः प्राप्नोत्येव महद्यशः}


\twolineshloka
{अब्रुवन्वाऽतिसुरभिर्गन्धः सुमनसां शुचिः}
{तथैवाव्याहरन्भाति विमलो भानुरम्बरे}


\twolineshloka
{एव मादीनि चान्यानि परित्यक्तानि मेधया}
{ज्वलन्ति यशसा लोके यानि न व्याहरन्ति च}


\twolineshloka
{न लोके दीप्यते मूर्खः केवलात्मप्रशंसया}
{अपि चापिहितः श्वभ्रे कृतविद्यः प्रकाशते}


\twolineshloka
{असदुच्चैरपि प्रोक्तः शब्दः समुपशाम्यति}
{दीप्यते त्वेव लोकेषु शनैरपि सुभाषितम्}


\twolineshloka
{मूढानामवलिप्तानामसारं भाषितं बहु}
{दर्शयत्यन्तरात्मानमग्निरूपमिवांशुमान्}


\twolineshloka
{एतस्मात्कारणात्प्रज्ञां मृगयन्ते पृथग्विधाम्}
{प्रज्ञालाभो हि भूतानामुत्तमः प्रतिभाति मे}


\twolineshloka
{नापृष्टः कस्यचिद्ब्रूयान्नाप्यन्यायेन पृच्छतः}
{जानन्नपि च मेधावी जडवत्समुपाविशेत्}


\twolineshloka
{ततो वासं परीक्षेत धर्मनित्येषु साधुषु}
{मनुष्येषु वदान्येषु स्वधर्मनिरतेषु च}


\twolineshloka
{चतुर्णां यत्र वर्णानां धर्मव्यतिकरो भवेत्}
{न तत्र वासं कुर्वीत श्रेयोर्थी वै कथंचन}


\twolineshloka
{निरारम्भोऽप्ययमिह यथालब्धोपजीवनः}
{पुण्यं पुण्येषु विमलं पापं पापेषु चाप्नुयात्}


\twolineshloka
{अपामग्नेस्तथेन्दोश्च स्पर्शं वेदयते यथा}
{तथा पश्यामहे स्पर्शमुभयोः पुण्यपापयोः}


\twolineshloka
{अपश्यन्तोऽन्यविषयं भुञ्जते विघसाशिनः}
{भुञ्जानाश्चान्यविषयान्विषयान्विद्धि कर्मणाम्}


\twolineshloka
{यत्रागमयमानानामसत्कारेण पृच्छताम्}
{प्रब्रूयाद्ब्रह्माणो धर्मं त्यजेत्तं देशमात्मवान्}


\twolineshloka
{शिष्योषाध्यायिका वृत्तिर्यत्र स्यात्सुसमाहिता}
{यथावच्छास्त्रसंपन्ना कस्तं देशं परित्यजेत्}


\twolineshloka
{आकाशस्था ध्रुवं यत्र दोषं ब्रूयुर्विपश्चिताम्}
{आत्मपूजाभिकामो वै को वसेत्तत्र पण्डितः}


\twolineshloka
{यत्र संलोडिता लुब्धैः प्रायशो धर्मसेतवः}
{प्रदीप्तमिव चेलान्तं कस्तं देशं न संत्यजेत्}


\twolineshloka
{यत्र धर्ममनाशङ्काश्चरेयुर्वीतमत्सराः}
{भवेत्तत्र वसेच्चैव पुण्यशीलेषु साधुषु}


\twolineshloka
{धर्ममर्थनिमित्तं च चरेयुर्यत्र मानवाः}
{न ताननुवसेज्जातु ते हि पापकृतो जनाः}


\twolineshloka
{कर्मणां यत्र पापेन वर्तन्ते जीवितेप्सवः}
{व्यवधावेत्ततस्तूर्णं ससर्पाच्छरणादिव}


\twolineshloka
{येन खट्वां समारूढः कर्मणाऽनुशयी भवेत्}
{आदितस्तन्न कर्तव्यमिच्छता भवमात्मनः}


\twolineshloka
{यत्र राजा च राज्ञश्च पुरुषाः प्रत्यनन्तराः}
{कुटुम्बिनामग्रभुजस्त्यजेत्तद्राष्ट्रमात्मवान्}


\twolineshloka
{श्रोत्रियास्त्वग्रभोक्तारो धर्मनित्याः सनातनाः}
{याजनाध्यापने युक्ता यत्र तद्राष्ट्रमावसेत्}


\twolineshloka
{स्वाहास्वधावषट््कारा यत्र सम्यगनुष्ठिताः}
{अजस्रं चैव वर्तन्ते वसेत्तत्राविचारयन्}


\twolineshloka
{अशुचीन्यत्र पश्येत ब्राह्मणान्वृत्तिकर्शितान्}
{त्यजेत्तद्राष्ट्रमासन्नमुपसृष्टमिवामिषम्}


\twolineshloka
{प्रीयमाणा नरा यत्र प्रयच्छेयुरयाचिताः}
{स्वस्थचित्तो वसेत्तत्र कृतकृत्य इवात्मवान्}


\twolineshloka
{दण्डो यत्राविनीतेषु सत्कारश्च कृतात्मसु}
{चरेत्तत्र वसेच्चैव पुण्यशीलेषु साधुषु}


\twolineshloka
{उपसृष्टेषु दान्तेषु दुराचारेषु साधुषु}
{अविनीतेषु लुब्धेषु सुमहद्दण्डधारणम्}


\twolineshloka
{यत्र राजा धर्मनित्यो राज्यं धर्मेण पालयेत्}
{अपास्य कामान्कामेशो वसेत्तत्राविचारयन्}


\twolineshloka
{यथाशीला हि राजानः सर्वान्विषयवासिनः}
{श्रेयसा योजयत्याशु श्रेयसि प्रत्युपस्थिते}


\twolineshloka
{पृच्छतस्ते मया तात श्रेय एतदुदाहृतम्}
{न हि शक्यं प्रधानेन श्रेयः सङ्ख्यातुमात्मनः}


\twolineshloka
{एवं प्रवर्तमानस्य वृत्तिं प्राणिहितात्मनः}
{तपसैवेह बहुलं श्रेयो व्यक्तं भविष्यति}


\chapter{अध्यायः २९४}
\twolineshloka
{युधिष्ठिर उवाच}
{}


\threelineshloka
{कथं नु युक्तः पृथिवीं चरेदस्मद्विधो नृपः}
{नित्यं कैश्च गुणैर्युक्तः सङ्गपाशाद्विमुच्यते ॥भीष्म उवाच}
{}


\threelineshloka
{अत्र ते वर्तयिष्येऽहमितिहासं पुरातनम्}
{अरिष्टनेमिना प्रोक्तं सगरायानुपृच्छते ॥सगर उवाच}
{}


\threelineshloka
{किं श्रेयः परमं ब्रह्मन्कृत्वेह सुखमश्नुते}
{कथं न शोचेन्न क्षुभ्येदेतदिच्छामि वेदितुम् ॥भीष्म उवाच}
{}


\twolineshloka
{एवमुक्तस्तदा तार्क्ष्यः सर्वशास्त्रविदां वरः}
{विबुध्य संपदं चाग्र्यां सद्वाक्यमिदमब्रवीत्}


\twolineshloka
{सुखं मोक्षसुखं लोके न च मूढोऽवगच्छति}
{प्रसक्तः पुत्रपशुषु धनधान्यसमाकुलः}


\twolineshloka
{सक्तबुद्धिरशान्तात्मा स न शक्यश्चिकित्सितुम्}
{स्नेहपाशसितो मूढो न स मोक्षाय कल्पते}


\twolineshloka
{स्नेहजानिह ते पाशान्वक्ष्यामि शृणु तान्मम}
{सकर्णकेन शिरसा शक्याश्छेत्तुं विजानता}


\twolineshloka
{संभाव्य पुत्रान्कालेन यौवनस्थान्निवेश्य च}
{समर्थाज्जीवने ज्ञात्वा मुक्तश्चर यथासुखम्}


\twolineshloka
{भार्यां पुत्रवतीं वृद्धां लालितां पुत्रवत्सलाम्}
{ज्ञात्वा प्रजहि कालेन परार्थमनुदृश्य च}


\twolineshloka
{सापत्यो निरपत्यो वा मुक्तश्चर यथासुखम्}
{इन्द्रियैरिन्द्रियार्थांस्त्वमनुभूय यथाविधि}


\twolineshloka
{कृतकौन्तूहलस्तेषु मुक्तश्चर यथासुखम्}
{उपपत्त्योपलब्धेषु लोकेषु च समो भव}


\twolineshloka
{एष तावत्समासेन तव संकीर्तितो मया}
{मोक्षार्थो विस्तरेणाथ भूयो वक्ष्यामि तच्छृणु}


\twolineshloka
{मुक्ता वीतभया लोके चरन्ति सुखिनो नराः}
{सक्तभावा विनश्यन्ति नरास्तत्र न संशयः}


\twolineshloka
{आहारसंचये सक्ता यथा कीटपिपीलिकाः}
{असक्ताः सुखिनो लोके सक्ताश्चैव विनाशिनः}


\twolineshloka
{स्वजने न च ते चिन्ता कर्तव्या मोक्षबुद्धिना}
{इमे मया विनाभूता भविष्यन्ति कथं त्विति}


\twolineshloka
{स्वयमुत्पद्यते जन्तुः स्वयमेव विवर्धते}
{सुखदुःखे तथा मृत्युं स्वयमेवाधिगच्छति}


\twolineshloka
{भोजनाच्छादने चैव मात्रा पित्रा च संग्रहम्}
{स्वकृते नाधिगच्छन्ति लोके नास्त्यकृतं पुरा}


\twolineshloka
{धात्रा विहितभक्ष्याणि सर्वभूतानि मेदिनीम्}
{लोके विपरिधावन्ति रक्षितानि स्वकर्मभिः}


\twolineshloka
{स्वयं मृत्पिण्डभूतस्य परतन्त्रस्य सर्वदा}
{को हेतुः स्वजनं द्वेष्टुं रक्षितं वाऽदृढात्मनः}


\twolineshloka
{स्वजनं हि यदा मृत्युर्हन्त्येव भुवि पश्यतः}
{कृतेऽपि यत्ने महति तत्र बोद्धव्यमात्मना}


\twolineshloka
{जीवन्तमपि चैवैनं भरणे रक्षणे तथा}
{असमाप्ते परित्यज्य पश्चादपि मरिष्यसि}


\twolineshloka
{यदा मृतं च स्वजनं न ज्ञास्यसि कथंचन}
{सुखितं दुःखितं वाऽपि ननु बोद्धव्यमात्मना}


\twolineshloka
{मृते वा त्वयि जीवे वा यदा भोक्ष्यति वै जनः}
{स्वकृतं ननु बुद्ध्वैवं कर्तव्यं हितमात्मनः}


\twolineshloka
{एवं विजानँल्लोकेऽस्मिन्कः कस्येत्यभिनिश्चितः}
{मोक्षे निवेशय मनो भूयश्चाप्युपधारय}


\twolineshloka
{क्षुत्पिपासादयो भावा जिता यस्येह देहिनः}
{क्रोधो लोभस्तथा मोहः सत्ववान्मुक्त एव सः}


\twolineshloka
{द्यूते पाने तथा स्त्रीषु मृगयायां च यो नरः}
{न प्रमाद्यति संमोहात्सततं मुक्त एव सः}


\twolineshloka
{दिवसेदिवसे नाम रात्रौरात्रौ पुमान्सदा}
{भोक्तव्यमिति यः स्विन्नो दोषबुद्धिः स उच्यते}


\twolineshloka
{आत्मभावं तथा स्त्रीषु सक्तमेव पुनः पुनः}
{यः पश्यति सदा युक्तो यथावन्मुक्त एव सः}


\twolineshloka
{संभवं च विनाशं च भूतानां चेष्टितं तथा}
{यस्तत्त्वतो विजानाति लोकेऽस्मिन्मुक्त एव सः}


\twolineshloka
{प्रस्थं वाहसहस्रेषु यात्रार्थं चैव कोटिषु}
{प्रासादे मञ्चकं स्थानं यः पश्यति स मुच्यते}


\twolineshloka
{मृत्युनाऽभ्याहतं लोकं व्याधिभिश्चोपपीडितम्}
{अवृत्तिकर्शितं चैव यः पश्यति स मुच्यते}


\twolineshloka
{यः पश्यति स संतुष्टो नपश्यंश्च विहन्यते}
{यश्चाप्यल्पेन संतुष्टो लोकेऽस्मिन्मुक्त एव सः}


\twolineshloka
{अग्नीषोमाविदं सर्वमिति यश्चानुपश्यति}
{न च संस्पृश्यते भावैरद्भुतैर्मुक्त एव सः}


\twolineshloka
{पर्यङ्कशय्या भूमिश्च सामने यस्य देहिनः}
{शाल्यन्नं च कदन्नं च यस्य स्यान्मुक्त एव सः}


\twolineshloka
{क्षौमं च कुशचीरं च कौशेयं वल्कलानि च}
{आविकं चर्म च समं यस्य स्यान्मुक्त एव सः}


\twolineshloka
{पञ्चभूतसमुद्भूतं लोकं यश्चानुपश्यति}
{तथाच वर्तते दृष्ट्वा लोकेऽस्मिन्मुक्त एव सः}


\twolineshloka
{सुखदुःखे समे यस्य लाभालाभौ जयाजयौ}
{इच्छाद्वेषौ भयोद्वेगौ सर्वथा मुक्त एव सः}


\twolineshloka
{रक्तमूत्रपुरीषाणां दोषाणां संचयांस्तथा}
{शरीरं दोषबहुलं दृष्ट्वा चैव विमुच्यते}


\twolineshloka
{वलीपलितसंयोगं कार्श्यं वैवर्ण्यमेव च}
{कुजभावं च जरयाः यः पश्यति स मुच्यतेत}


\twolineshloka
{पुंस्त्वोपघातं कालेन दर्शनोपरमं तथा}
{बाधिर्यं प्राणमन्दत्वं यः पश्यति स मुच्यते}


\twolineshloka
{गतानृषींस्तथा देवानसुरांश्च तथा गतान्}
{लोकादस्मात्परं लोकं यः पश्यति स मुच्यते}


\twolineshloka
{प्रभावैरन्वितास्तैस्तैः पार्थिवेन्द्राः सहस्रशः}
{ये गताः पृथिवीं त्यक्त्वा इति ज्ञात्वा विमुच्यते}


\twolineshloka
{अर्थांश्च दुर्लभाँल्लोके क्लेशांश्च सुलभांस्तथा}
{दुःखं चैव कुटुम्बार्थे यः पश्यति स मुच्यते}


\twolineshloka
{अपत्यानां च वैगुण्यं जनं विगुणमेव च}
{पश्यन्भूयिष्ठशो लोके को मोक्षं नाभिपूजयेत्}


\twolineshloka
{शास्त्राल्लोकाच्च यो बुद्धः सर्वं पश्यति मानवः}
{असारमिव मानुष्यं सर्वथा मुक्ता एव सः}


\twolineshloka
{एतच्छ्रुत्वा मम वचो भवांश्चरतु मुक्तवत्}
{गार्हस्थ्याद्यदि ते मोक्षे कृता बुद्धिरविक्लवा}


\twolineshloka
{तत्तस्य वचनं श्रुत्वा सम्यक्स पृथिवीपतिः}
{मोक्षजैश्च गुणैर्युक्तः पालयामास च प्रजाः}


\chapter{अध्यायः २९५}
\twolineshloka
{युधिष्ठिर उवाच}
{}


\twolineshloka
{तिष्ठते मे सदा तात कौतूहलमिदं हृदि}
{तदहं श्रोतुमिच्छामि त्वत्तः कुरुपितामह}


\twolineshloka
{कथं देवर्षिरुशना सदा काव्यो महामतिः}
{असुराणां प्रियकारः सुराणामप्रिये रतः}


\twolineshloka
{वर्धयामास तेजश्च किमर्थममितौजसाम्}
{नित्यं वैरनिबद्धाश्च दानवाः सुरसत्तमैः}


\twolineshloka
{कथं चाप्युशना प्राप शुक्रत्वममरद्युतिः}
{ऋद्धिं च स कथं प्राप्तः सर्वमेतद्ब्रवीहि मे}


\threelineshloka
{न याति च स तेजस्वी मध्येन नभसः कथम्}
{एतदिच्छामि विज्ञातुं निखिलेन पितामह ॥भीष्म उवाच}
{}


\twolineshloka
{शृणु राजन्नवहितः सर्वमेतद्यथातथम्}
{यथामति यथा चैतच्छ्रुतपूर्वं मयाऽनघ}


\twolineshloka
{एष भार्गवदायादो मुनिर्मान्यो दृढव्रतः}
{सुराणां विप्रियकरो निमित्ते कारणात्मके}


\twolineshloka
{इन्द्रोऽथ धनदो राजा यक्षरक्षोधिपः सदा}
{प्रभविष्णुश्च कोशस्य जगतश्च तथा प्रभुः}


\twolineshloka
{तस्यात्मानमथाविश्य योगसिद्धो महामुनिः}
{रुद्ध्वा धनपतिं देवं योगेन हृतवान्वसु}


\twolineshloka
{हृते धने ततः शर्म न लेभे धनदस्तथा}
{आपन्नमन्युः संविग्नः सोभ्यगात्सुरसत्तमम्}


\twolineshloka
{निवेदयामास तदा शिवायामिततेजसे}
{देवश्रेष्ठाय रुद्राय सौम्याय बहुरूपिणे}


\twolineshloka
{योगात्मकेनोशनसा रुद्ध्वा मम हृतं वसु}
{योगेनात्मगतं कृत्वा निःसृतश्च महातपाः}


\twolineshloka
{एतच्छ्रुत्वा ततः क्रुद्धो महायोगी महेश्वरः}
{संरक्तनयनो राजञ्शूलमादाय तस्थिवान्}


\twolineshloka
{क्वासौ क्वासाविति प्राह गृहीत्वा परमायुधम्}
{उशना दूरतस्तस्य ह्यभूज्ज्ञात्वा चिकीर्षितम्}


\twolineshloka
{स महायोगिनो बुद्ध्वा तं रोषं वै महात्मनः}
{गतिमागमनं वेत्ति स्थानं चैव ततः प्रभुः}


\twolineshloka
{संचिन्त्येग्रेण तपसा महात्मानं महेश्वरम्}
{उशना योगसिद्धात्मा शूलाग्रे प्रत्यदृश्यत}


\twolineshloka
{विज्ञातरूपः स तदा तपः सिद्धोऽथ धन्विना}
{ज्ञात्वा शूलं च देवेशः पाणिना समनामयत्}


\twolineshloka
{आनतेनाथ शूलेन पाणिनामिततेजसा}
{पिनाकमिति चोवाच शूलमुग्रायुधः प्रभुः}


\twolineshloka
{पाणिमध्यगतं दृष्ट्वा भार्गवं तमुमापतिः}
{आस्यं विवृत्य भ्रकुटिं पाणिं संप्राक्षिपच्छनैः}


\threelineshloka
{स तु प्रविष्ट उशना कोष्ठं माहेश्वरं प्रभुः}
{व्यचरच्चापि तत्रासौ महात्मा भृगुनन्दनः ॥युधिष्ठिर उवाच}
{}


\threelineshloka
{किमर्थं व्यचरद्राजन्नुशना तस्य धीमतः}
{जठरे देवदेवस्य किंचाकार्षीन्महाद्युतिः ॥भीष्म उवाच}
{}


\twolineshloka
{पुरा सोऽन्तर्जलगतः स्थाणुभूतो महाव्रतः}
{वर्षाणामभवद्राजन्प्रयुतान्यर्बुदानि च}


\twolineshloka
{उदतिष्ठत्तपस्तप्त्वा दुश्चरं च महाह्रदात्}
{ततो देवातिदेवस्तं ब्रह्मा वै समसर्पत}


\twolineshloka
{तपोवृद्धिमपृच्छच्च कुशलं चैवमव्ययः}
{तपः सुचीर्णमिति च प्रोवाच वृषभध्वजः}


\twolineshloka
{तत्संयोगेन वृद्धिं चाप्यपश्यत्स तु शंकरः}
{महामतिरचिन्त्यात्मा सत्यधर्मरतः सदा}


\twolineshloka
{स तेनाढ्यो महायोगी तपसा च धनेन च}
{व्यराजत महाराज त्रिषु लोकेषु वीर्यवान्}


\twolineshloka
{ततः पिनाकी योगात्मा ध्यानयोगं समाविशत्}
{उशना तु समुद्विग्नो निलिल्ये जठरे ततः}


\twolineshloka
{तुष्टाव च महायोगी देवं तत्रस्थ एव च}
{निःसारं काङ्क्षमाणः स तेन स्म प्रतिहन्यते}


\twolineshloka
{उशना तु तथोवाच जठरस्थो महामुनिः}
{प्रसादं मे कुरुष्वेति पुनः पुनररिंदम्}


\twolineshloka
{तमुवाच महादेवो गच्छ शिश्नेन मोक्षणम्}
{इति सर्वाणि स्रोतांसि रुद्ध्वा त्रिदशपुङ्गवः}


\twolineshloka
{अपश्यमानस्तद्द्वारं सर्वतः पिहितो मुनिः}
{पर्यक्रामद्दह्यमान इतश्चेतश्च तेजसा}


\twolineshloka
{स वै निष्क्रम्य शिश्नेन शुक्रत्वमभिपेदिवान्}
{कार्येण तेन नभसो नाध्यगच्छत मध्यतः}


\twolineshloka
{`तत एव च देवेषु अप्रविष्टो महामुनिः}
{पौरोहित्यं च दैत्यानां शक्रतेजोविवृद्धये ॥'}


\twolineshloka
{विनिष्क्रान्तं तु तं दृष्ट्वा ज्वलन्तमिव तेजसा}
{भवो रोषसमाविष्टः शूलोद्यतकरः स्थितः}


\threelineshloka
{अवारयत तं देवी क्रुद्धं पशुपतिं पतिम्}
{पुत्रत्वमगमद्देव्या वारिते शंकरे च सः ॥देव्युवाच}
{}


\twolineshloka
{हिंसनीयस्त्वया नैव मम पुत्रत्वमागतः}
{न हि देवोदरात्कश्चिन्निःसृतो नाशमर्हति}


\twolineshloka
{ततः प्रीतो भवो देव्याः प्रहसंश्चेदमब्रवीत्}
{गच्छत्वेष यथाकाममिति राजन्पुनः पुनः}


\twolineshloka
{ततः प्रणम्य वरदं देवं देवीमुमां तथा}
{उशना प्राप तद्धीमान्गतिमिष्टां महामुनिः}


\twolineshloka
{एतत्ते कथितं तात भार्गवस्य महात्मनः}
{चरितं भरतश्रेष्ठ यन्मां त्वं परिपृच्छसि}


\chapter{अध्यायः २९६}
\twolineshloka
{युधिष्ठिर उवाच}
{}


\twolineshloka
{अतः परं महाबाहो यच्छ्रेयस्तद्ब्रवीहि मे}
{न तृप्याम्यमृतस्येव वचसस्ते पितामह}


\threelineshloka
{किं कर्म पुरुषः कृत्वा शुभं पुरुषसत्तम}
{श्रेयः परमवाप्नोति प्रेत्य चेह च तद्वद ॥भीष्म उवाच}
{}


\twolineshloka
{अत्र ते वर्तयिष्यामि यथा पूर्वं महायशाः}
{पराशरं महात्मानं पप्रच्छ जनको नृपः}


\twolineshloka
{किं श्रेयः सर्वभूतानामस्मिँल्लोके परत्र च}
{यद्भवेत्प्रतिपत्त्व्यं तद्भवान्प्रब्रवीतु मे}


\threelineshloka
{ततः स तपसा युक्तः सर्वधर्मविधानवित्}
{नृपायानुग्रहमना मुनिर्वाक्यमथाब्रवीत् ॥पराशर उवाच}
{}


\twolineshloka
{धर्म एव कृतः श्रेयानिह लोके परत्र च}
{तस्माद्धि परमं नास्ति यथा प्राहुर्मनीषिणः}


\twolineshloka
{प्रतिपद्य नरो धर्मं स्वर्गलोके महीयते}
{धर्मात्मकः कर्मविधिर्देहिनां नृपसत्तम}


% Check verse!
तस्मिन्त्राश्रमिणः सन्तः स्वकर्माणीह कुर्वते
\twolineshloka
{चतुर्विधा हि लोकेऽस्मिन्यात्रा तात विधीयते}
{मर्त्या यत्रावतिष्ठन्ते सा च कामात्प्रवर्तते}


\twolineshloka
{सुकृतासुकृतं कर्म निषेव्य विविधैः क्रमैः}
{दशार्धप्रविभक्तानां भूतानां विविधा गतिः}


\twolineshloka
{सौवर्णं राजतं चापि यथा भाण्डं निषिच्यते}
{तथा निपिच्यते जन्तुः पूर्वकर्मवशानुगः}


\twolineshloka
{नाबीजाज्जायते किंचिन्नाकृत्वा सुखमेधते}
{सुकृतैर्विन्दते सौख्यं प्राप्य देहक्षयं नरः}


\twolineshloka
{दैवं तात न पश्यामि नास्ति दैवस्य साधनम्}
{स्वभावतो हि संसिद्धा देवगन्धर्वदानवाः}


\twolineshloka
{प्रेत्य यान्त्यकृतं कर्म न स्मरन्ति सदा जनाः}
{ते वै तस्य फलप्राप्तौ कर्म चापि चतुर्विधम्}


\twolineshloka
{लोकयात्राश्रयश्चैव शब्दो वेदाश्रयः कृतः}
{शान्त्यर्थं मनसस्तात नैतद्वृद्धानुशासनम्}


\twolineshloka
{चक्षुषा मनसा वाचा कर्मणा च चतुर्विधम्}
{कुरुते यादृशं कर्म तादृशं प्रतिपद्यते}


\twolineshloka
{निरन्तरं च मिश्रं च लभते कर्म पार्थिव}
{कल्याणं यदि वा पापं न तु नाशोऽस्य विद्यते}


\twolineshloka
{कदाचित्सुकृतं तात कूटस्थमिव तिष्ठति}
{मज्जमानस्य संसारे यावद्दुःखाद्विमुच्यते}


\twolineshloka
{ततो दुःखक्षयं कृत्वा सुकृतं कर्म सेवते}
{सुकृतक्षयाच्च दुष्कृतं तद्विद्धि मनुजाधिप}


\twolineshloka
{दमः क्षमा धृतिस्तेजः संतोषः सत्यवादिता}
{ह्रीरहिंसाऽव्यसनिता दाक्ष्यं चेति सुखावहाः}


\twolineshloka
{दुष्कृते सुकृते चापि न जन्तुर्नियतो भवेत्}
{नित्यं मनः समाधाने प्रयतेत विचक्षणः}


\twolineshloka
{नायं परस्य सुकृतं दुष्कृतं चापि सेवते}
{करोति यादृशं कर्म तादृशं प्रतिपद्यते}


\twolineshloka
{सुखदुःखे समाधाय पुमानन्येन गच्छति}
{अन्येनैव जनः सर्वः संगतो यश्च पार्थिवः}


\twolineshloka
{परेषां यदसूयेत न तत्कुर्यात्स्वयं नरः}
{यो ह्यसूयुस्तथा युक्तः सोऽवहासं नियच्छति}


\twolineshloka
{भीरू राजन्यो ब्राह्मणः सर्वभक्ष्योवैश्योऽनीहावान्हीनवर्णोऽलसश्च}
{विद्वांश्राशीलो वृत्तहीनः कुलीनःसत्याद्विभ्रष्टो ब्राह्मणस्त्री च तुष्टा}


\twolineshloka
{रागी युक्तः पचमानोऽऽत्महेतोर्मूर्खो वक्ता नृपहीनं च राष्ट्रम्}
{एते सर्वे शोच्यतां यान्ति राजन्यश्रायुक्तः स्नेहहीनः प्रजासु}


\chapter{अध्यायः २९७}
\twolineshloka
{पराशर उवाच}
{}


\twolineshloka
{मनोरथरथं प्राप्य इन्द्रियार्थहयं नरः}
{रश्मिभिर्ज्ञानसंभूतैर्यो गच्छति स बुद्धिमान्}


\twolineshloka
{सेवाश्रितेन मनसा वृत्तिहीनस्य शस्यते}
{द्विजातिहस्तान्निर्वृत्ता न तु तुल्यात्परस्परात्}


\twolineshloka
{आयुर्नसुलभं लब्ध्वा नावकर्षेद्विशांपते}
{उत्कर्षार्थं प्रयतते नरः पुण्येन कर्मणा}


\twolineshloka
{वर्णेभ्यो हि परिभ्रष्टो न वै संमानमर्हति}
{न तु यः सत्क्रियां प्राप्य राजसं कर्म सेवते}


\twolineshloka
{वर्णोत्कर्षमवाप्नोति नरः पुण्येन कर्मणा}
{दुर्लभं तमलब्धा हि हन्यात्पापेन कर्मणा}


\threelineshloka
{अज्ञानाद्धि कृतं पापं तपसैवाभिनिर्णुदेत्}
{पापं हि कर्म फलति पापमेव स्वयंकृतम्}
{तस्मात्पापं न सेवेत कर्म दुःखफलोदयम्}


\twolineshloka
{पापानुबन्धं यत्कर्म यद्यपि स्यान्महाफलम्}
{तन्न सेवेत मेधावी शुचिः कुशलिनं यथा}


\twolineshloka
{किंकष्टमनुपश्यामि फलं पापस्य कर्मणः}
{प्रत्यापन्नस्य हि ततो नात्मा तावद्विरोचते}


\twolineshloka
{प्रत्यापत्तिश्च यस्येह बालिशस्य न जायते}
{तस्यापि सुमहांस्तापः प्रस्थितस्योपजायते}


\twolineshloka
{विरक्तं शोध्यते वस्त्रं न तु कृष्णोपसंहितम्}
{प्रयत्नेन मनुष्येन्द्र पापमेवं निबोध मे}


\twolineshloka
{स्वयं कृत्वा तु यः पापं शुभमेवानुतिष्ठति}
{प्रायश्चित्तं नरः कर्तुमुभयं सोऽश्नुते पृथक्}


\twolineshloka
{अज्ञानात्तु कृतां हिंसामहिंसा व्यपकर्षति}
{ब्राह्मणाः शास्त्रनिर्देशादित्याहुर्ब्रह्मवादिनः}


\twolineshloka
{तथा कामकृतं नास्य विहिंसैवानुकर्षति}
{इत्याहुर्ब्रह्मशास्त्रज्ञा ब्राह्मणा ब्रह्मवादिनः}


\twolineshloka
{अहं तु तावत्पश्यामि कर्म यद्धर्तते कृतम्}
{गुणयुक्तं प्रकाशं वा पापेनानुपसंहितम्}


\twolineshloka
{यथा सूक्ष्माणि कर्माणि फलन्तीह यथातथम्}
{बुद्धियुक्तानि तानीह कृतानि मनसा सह}


\twolineshloka
{भवत्यल्पफलं कर्म सेवितं नित्यमुल्वणम्}
{अबुद्धिपूर्वं धर्मज्ञ कृतमुग्रेण कर्मणा}


\twolineshloka
{कृतानि यानि कर्माणि दैवतैर्मुनिभिस्तथा}
{न चरेत्तानि धर्मात्मा श्रुत्वा चापि न कुत्सयेत्}


\twolineshloka
{संचिन्त्य मनसा राजन्विदित्वा शक्तिमात्मनः}
{करोति यः शुभं कर्म स वै भद्राणि पश्यति}


\twolineshloka
{नवे कपाले सलिलं संन्यस्तं हीयते यथा}
{नवेतरे तथा भावं प्राप्नोति सुखभावितम्}


\twolineshloka
{सतोयेऽन्यत्तु यत्तोयं तस्मिन्नेव प्रसिच्यते}
{तद्धि वृद्धिमवाप्नोति सलिले सलिलं यथा}


\twolineshloka
{एवं कर्माणि यानीह बुद्धियुक्तानि पार्थिव}
{समानि चैव यानीह तानि पुण्यतमान्यपि}


\twolineshloka
{राज्ञा जेतव्याः शत्रवश्चोन्नताश्चसम्यक्कर्तव्यं पालनं च प्रजानाम्}
{अग्निश्चेयो बहुभिश्चापि यज्ञैरन्त्ये मध्ये वा वनमाश्रित्य स्थेयम्}


\twolineshloka
{दमान्वितः पुरुषो धर्मशीलोभूतानि चात्मानमिवानुपश्येत्}
{गरीयसः पूजयेदात्मशक्त्यासत्येन शीलेन सुखं नरेन्द्र}


\chapter{अध्यायः २९८}
\twolineshloka
{पराशर उवाच}
{}


\twolineshloka
{कः कस्य चोपकुरुते कश्च कस्मै प्रयच्छति}
{प्राणी करोत्ययं कर्म सर्वमात्मार्थमात्मना}


\twolineshloka
{गौरवेण परित्यक्तं निःस्नेहं परिवर्जयेत्}
{सोदर्यं भ्रातरमपि किमुतान्यं पृथग्जनम्}


\twolineshloka
{विशिष्टस्य विशिष्टाच्च तुल्यौ दानप्रतिग्रहौ}
{तयोः पुण्यतरं दानं तद्द्विजस्य प्रयच्छतः}


\twolineshloka
{न्यायागतं धनं वर्णैर्न्यायेनैव विवर्धितम्}
{संरक्ष्यं यत्नमास्थाय धर्मार्थमिति निश्चयः}


\twolineshloka
{न धर्मार्थी नृशंसेन कर्मणा धनमार्जयेत्}
{शक्तितः सर्वकार्याणि कुर्यान्नर्द्धिमनुस्मरन्}


\twolineshloka
{अपो हि प्रयताः शीतास्तापिता ज्वलनेन वा}
{शक्तितोऽतिथये दत्त्वा क्षुधार्तायाऽश्नुते फलम्}


\twolineshloka
{रन्तिदेवेन लोकेष्टा सिद्धिः प्राप्ता महात्मना}
{फलपत्रैरथो मूलैर्मुनीनचिंतवांश्च सः}


\twolineshloka
{तैरेव फलपत्रैश्च स माठरमतोषयत्}
{तस्माल्लेभे परं स्थानं शैब्योऽपि पृथिवीपतिः}


\twolineshloka
{देवतातिथिभृत्येभ्यः पितृभ्यश्चात्मनस्तथा}
{ऋणवाञ्जायते मर्त्यस्तस्मादनृणतां व्रजेत्}


\twolineshloka
{स्वाध्यायेन महर्षिभ्यो देवेभ्यो यज्ञकर्मणा}
{पितृभ्यः श्राद्धदानेन नृणामभ्यर्चनेन च}


\twolineshloka
{पाकशेपावहार्येण पालनेनात्मनोऽपि च}
{यथावद्भृत्यवर्गस्य चिकीर्षेत्कर्म आदितः}


\twolineshloka
{प्रयत्नेन च संसिद्धा धनैरपि विवर्जिताः}
{सम्यग्घृत्वा हुतवहं मुनयः सिद्धिमागताः}


\twolineshloka
{विश्वामित्रस्य पुत्रत्वमृचीकतनयोऽगमत्}
{ऋग्भिः स्तुत्वा महाबाहो देवान्वै यज्ञभागिनः}


\twolineshloka
{गतः शुक्रत्वमुशना देवदेवप्रसादनात्}
{देवीं स्तुत्वा तु गगने मोदते तेजसा वृतः}


\twolineshloka
{असितो देवलश्चैव तथा नारदपर्वतौ}
{कक्षीवाञ्जामदग्न्यश्च रामस्ताण्ड्यस्तथाऽऽत्मवान्}


\twolineshloka
{वसिष्ठो जमदग्निश्च विश्वामित्रोऽत्रिरेव च}
{भरद्वाजो हरिश्मश्रुः कुण्डधारः श्रुतश्रवाः}


\twolineshloka
{एते महर्षयः स्तुत्वा विष्णुमृग्भिः समाहिताः}
{लेभिरे तपसा सिद्धिं प्रसादात्तस्य धीमतः}


\twolineshloka
{अनर्हाश्चार्हतां प्राप्ताः सन्तः स्तुत्वा तमेव ह}
{न तु वृद्धिमिहान्विच्छेत्कर्म कृत्वा जुगुप्सितम्}


\twolineshloka
{येऽर्था धर्मेण ते सत्या येऽधर्मेण धिगस्तु तान्}
{धर्मं वै शाश्वतं लोके न जह्याद्धनकाङ्क्षया}


\twolineshloka
{आहिताग्निर्हि धर्मात्मा यः स पुण्यकृदुत्तमः}
{वेदा हि सर्वे राजेन्द्र स्थितास्त्रिष्वग्निषु प्रभो}


\twolineshloka
{च चाप्यग्न्याहितो विप्रः क्रिया यस्य न हीयते}
{श्रेयो ह्यनाहिताग्नित्वमग्निहोत्रं न निष्क्रियम्}


\twolineshloka
{अग्निरात्मा च माता च पिता जनयिता तथा}
{गुरुश्च नरशार्दूल परिचर्या यथातथम्}


\twolineshloka
{मानं त्यक्त्वा यो नरो वृद्धसेवीविद्वान्क्लीबः पश्यति प्रीतियोगात्}
{दाक्ष्येण हीनो धर्मयुक्तो न दान्तोलोकेऽस्मिन्वै पूज्यते सद्भिरार्यः}


\chapter{अध्यायः २९९}
\twolineshloka
{पराशर उवाच}
{}


\twolineshloka
{वृत्तिः सकाशाद्वर्णेभ्यस्त्रिभ्यो हीनस्य शोभना}
{प्रीत्योपनीता निर्दिष्टा धर्मिष्ठान्कुरुते सदा}


\twolineshloka
{वृत्तिश्चेन्नास्ति शूद्रस्य पितृपैतामही ध्रुवा}
{न वृत्तिं परतो मार्गेच्छुश्रूषां तु प्रयोजयेत्}


\twolineshloka
{सद्भिस्तु सह संसर्गः शोभते धर्मदर्शिभिः}
{नित्यं सर्वास्ववस्थासु नासद्भिरिति मे मतिः}


\twolineshloka
{यथोदयगिरौ द्रव्यं सन्निकर्षेण दीप्यते}
{तथा सत्सन्निकर्षेण हीनवर्णोऽपि दीयते}


\twolineshloka
{यादृशेन हि वर्णेन भाव्यते शुक्लमम्बरम्}
{तादृशं कुरुते रूपमेतदेवमवेहि मे}


\twolineshloka
{तस्माद्गुणेषु रज्येथा मा दोषेषु कदाचन}
{अनित्यमिह मर्त्यानां जीवितं हि चलाचलम्}


\twolineshloka
{सुखे वा यदि वा दुःखे वर्तमानो विचक्षणः}
{यश्चिनोति शुभान्येव स भद्राणीह पश्यति}


\twolineshloka
{धर्मादपेतं यत्कर्म यद्यपि स्यान्महाफलम्}
{न तत्सेवेत मेधावी न तद्धितमिहोच्यते}


\twolineshloka
{`धर्मेण सहितं यत्तु भवेदल्पफलोदयम्}
{तत्कार्यमविशङ्केन कर्मात्यन्तं सुखावहम् ॥'}


\twolineshloka
{यो हृत्वा गोसहस्राणि नृपो दद्यादरक्षिता}
{स शब्दमात्रफलभाग्राजा भवति तस्करः}


\twolineshloka
{स्वयंभूरसृजच्चाग्रे धातारं लोकसत्कृतम्}
{धाताऽसृजत्पुत्रमेकं लोकानां धारणे रतम्}


\twolineshloka
{तमर्चयित्वा वैश्यस्तु कुर्यादत्यर्थमृद्धिमत्}
{रक्षितव्यं तु राजन्यैरुपयोज्यं द्विजातिभिः}


\twolineshloka
{अजिह्नैरशठक्रोधैर्हव्यकव्यप्रयोक्तृभिः}
{शूर्दैर्निर्मार्जनं कार्यमेवं धर्मो न नश्यति}


\twolineshloka
{अप्रनष्टे ततो धर्मे भवन्ति सुखिताः प्रजाः}
{सुखेन तासां राजेन्द्र मोदन्ते दिवि देवताः}


\twolineshloka
{तस्माद्यो रक्षति नृपः स धर्मेणेति पूज्यते}
{अधीते चापि यो विप्रो वैश्यो यश्चार्जने रतः}


\twolineshloka
{यश्च शुश्रूषते शूद्रः सततं नियतेन्द्रियः}
{अतोऽन्यथा मनुष्येन्द्र स्वधर्मात्परिहीयते}


\twolineshloka
{प्राणसंतापनिर्दिष्टाः काकिण्योऽपि महाफलाः}
{न्यायेनोपार्जिता दत्ताः किमुतान्याः सहस्रशः}


\twolineshloka
{सत्कृत्य हि द्विजातिभ्यो यो ददाति नराधिपः}
{यादृशं तादृशं नित्यमश्नाति फलमूर्जितम्}


\twolineshloka
{अभिगम्य तु यद्दत्तं धर्म्यमाहुरभिष्टुतम्}
{याचितेन तु यद्दत्तं तदाहुर्मध्यमं फलम्}


\twolineshloka
{अवज्ञया दीयते यत्तथैवाश्रद्धयाऽपि वा}
{तदाहुरधमं दानं मुनयः सत्यवादिनः}


\twolineshloka
{अतिक्रामेन्मज्जमानो विविधेन नरः सदा}
{तथा प्रयत्नं कुर्वीत यथा मुच्येत संशयात्}


\twolineshloka
{दमेन शोभते विप्रः क्षत्रियो विजयेन तु}
{धनेन वैश्यः शृद्रस्तु नित्यं दाक्ष्येण शोभते}


\chapter{अध्यायः ३००}
\twolineshloka
{पराशर उवाच}
{}


\twolineshloka
{प्रतिग्रहार्जिता विप्रे क्षत्रिये युधि निर्जिताः}
{वैश्ये न्यायार्जिताश्चैव शूद्रे शुश्रूषयार्जिताः}


\twolineshloka
{स्वल्पाऽप्यर्थाः प्रशस्यन्ते धर्मस्यार्थे महाफलाः}
{नित्यं त्रयाणां वर्णानां शुश्रूषुः शूद्र उच्यते}


\twolineshloka
{क्षत्रधर्मा वैश्यधर्मा नावृत्तिः पतते द्विजः}
{शूद्रधर्मा यदा तु स्यात्तदा पतति वै द्विजः}


\twolineshloka
{वाणिज्यं पाशुपाल्यं च तथा शिल्पोपजिवनम्}
{सूद्रस्यापि विधीयन्ते यदा वृत्तिर्न जायते}


\twolineshloka
{रङ्गावतरणं चैव तथा रूपोपजीवनम्}
{मद्यमांसोपजीव्यं च विक्रयं लोहचर्मणोः}


\twolineshloka
{अपूर्विणा न कर्तव्यं कर्म लोके विगर्हितम्}
{कृतपूर्विणस्तु त्यजतो महान्धर्म इति श्रूतिः}


\twolineshloka
{संसिद्धः पुरुषो लोके यदाचरति पापकम्}
{मदेनाभिप्लुतमनास्तच्च न ग्राह्यमुच्यते}


\twolineshloka
{श्रूयन्ते हि पुराणेषु प्रजा धिग्दण्डशासनाः}
{दान्ता धर्मप्रधानाश्च न्यायधर्मानुवृत्तिकाः}


\twolineshloka
{धर्म एव सदा नॄणामिह राजन्प्रशस्यते}
{धर्मवृद्धा गुणानेव सेवन्ते हि नरा भुवि}


\twolineshloka
{तं धर्ममसुरास्तात नामृष्यन्त नराधिप}
{विवर्धमानाः क्रमशस्तत्र तेऽन्वाविशन्प्रजाः}


\twolineshloka
{तासां दर्पः समभवत्प्रजानां धर्मनाशनः}
{दर्पात्मनां ततः पश्चात्क्रोधस्तासामजायत}


\twolineshloka
{ततः क्रोधाभिभूतानां वृत्तं लज्जासमन्वितम्}
{ह्रीश्चैवाप्यनशद्राजंस्ततो मोहो व्यजायत}


\twolineshloka
{ततो मोहपरीतास्ता नापश्यन्त यथा पुरा}
{परस्परावमर्देन वर्धयन्त्यो यथासुखम्}


\twolineshloka
{ताः प्राप्य तु स धिग्दण्डो न कारणमतो भवत्}
{ततोऽभ्यगच्छन्देवांश्च ब्राह्मणांश्चावमन्य ह}


\twolineshloka
{एतस्मिन्नेव काले तु देवा देववरं शिवम्}
{अगच्छञ्शरणं धीरं बहुरूपं गुणाधिकम्}


\twolineshloka
{तेन स्म ते गगनगाः सपुराः पातिताः क्षितौ}
{त्रिधाऽप्येकेन बाणेन देवाप्यायिततेजसा}


\twolineshloka
{तेषामधिपतिस्त्वासीद्भीमो भीमपराक्रमः}
{देवतानां भयकरः स हतः शूलपाणिना}


\twolineshloka
{तस्मिन्हतेऽथ स्वं भावं प्रत्यपद्यन्त मानवाः}
{प्रावर्तन्त च वै वेदाः शास्त्राणि च यथा पुरा}


\twolineshloka
{ततोऽभिषिच्य राज्येन देवानां दिवि वासवम्}
{सप्तर्षयश्चान्वयुञ्जन्नराणां दण्डधारणे}


\twolineshloka
{सप्तर्षीणामथोर्ध्वं च विपृथुर्नाम पार्थिवः}
{राजानः क्षत्रियाश्चैव मण्डलेषु पृथक्पृथक्}


\twolineshloka
{महाकुलेषु ये जाता वृद्धाः पूर्वतराश्च ये}
{तेषामप्यासुरो भावो हृदयान्नापसर्पति}


\twolineshloka
{तस्मात्तेनैव भावेन सानुषङ्गेण पार्थिवाः}
{आसुराण्येव कर्माणि न्यसेवन्भीमविक्रमाः}


\twolineshloka
{प्रत्यतिष्ठंश्च तेष्वेव तान्येव स्थापयन्त्यपि}
{भजन्ते तानि चाद्यापि ये बालिशतरा नराः}


\twolineshloka
{तस्मादहं ब्रवीमि त्वां राजन्संचिन्त्य शास्त्रतः}
{संसिद्धावागमं कुर्यात्कर्म हिंसात्मकं त्यजेत्}


\twolineshloka
{न संकरेण द्रविणं प्रचिन्वीयाद्विचक्षणः}
{धर्मार्थं न्यायमुत्सृज्य न तत्कल्याणमुच्यते}


\twolineshloka
{स त्वमेवंविधो दान्तः क्षत्रियः प्रियबान्धवः}
{प्रजा भृत्यांश्च पुत्रांश्च स्वधर्मेणानुपालय}


\twolineshloka
{इष्टानिष्टसमायोगे वैरं सौहार्दमेव च}
{अथ जातिसहस्राणि बहूनि परिवर्तते}


\twolineshloka
{तस्माद्गुणेषु रज्येथा मा दोषेषु कथंचन}
{निर्गुणोऽपि हि दुर्बुद्धिरात्मनः सोतिरिच्यते}


\twolineshloka
{मानुषेषु महाराज धर्माधर्मौ प्रवर्ततः}
{न तथाऽन्येषु भूतेषु मनुष्यरहितेष्विह}


\twolineshloka
{धर्मशीलो नरो विद्वानीहकोऽनीहकोपि वा}
{आत्मभूतः सदा लोके चरेद्भूतान्यहिंसया}


\twolineshloka
{यदा व्यपेतहृल्लेखं मनो भवति तस्य वै}
{नानृतं चैव भवति तदा कल्याणमृच्छति}


\chapter{अध्यायः ३०१}
\twolineshloka
{पराशर उवाच}
{}


\twolineshloka
{एष धर्मविधिस्तात गृहस्थस्य प्रकीर्तितः}
{तपोविधिं तु वक्ष्यामि तन्मे निगदतः शृणु}


\twolineshloka
{प्रायेण च गृहस्थस्य ममत्वं नाम जायते}
{सङ्गागतं नरश्रेष्ठ भावै राजसतामसैः}


\twolineshloka
{गृहाण्याश्रित्य गावश्च क्षेत्राणि च धनानि च}
{दाराः पुत्राश्च भृत्याश्च भवन्तीह नरस्य वै}


\twolineshloka
{एवं तस्य प्रवृत्तस्य नित्यमेवानुपश्यतः}
{रागद्वेषौ विवर्धेते ह्यनित्यत्वमपश्यतः}


\twolineshloka
{रागद्वेषाभिभूतं च नरं द्रव्यवशानुगम्}
{मोहजस्तारतिर्नाम समुपैति नराधिप}


\twolineshloka
{कृतार्थ भोगिनं मत्वा सर्वो रतिपरायणः}
{लाभं प्रात्यसुखादन्यं रतितो नानुपश्यति}


\twolineshloka
{ततो लोभाभिभूतात्मा सङ्गाद्वधर्यते जनम्}
{दुष्टार्थं चैव तस्येह जनस्यार्थं चिकीर्षति}


\twolineshloka
{स जानत्रपि चाकार्यमर्थार्थं सेवते नरः}
{बालस्नेहपरीतात्मा तत्क्षयाच्चानुतप्यते}


\twolineshloka
{ततो दानेन संपन्नो रक्षन्नात्मपराजयम्}
{करोति येन भोनी स्यामिति तस्माद्विनश्यति}


\twolineshloka
{तद्यदि वुद्धियुक्तानां शाश्वतं ब्रह्मवादिनाम्}
{अभिच्छतां शुभं कर्म नराणां त्यजतां सुखम्}


\twolineshloka
{लोकायतननाशाच्च धननाशाच्च पार्थिव}
{आधिव्याधिप्रतापाच्च निर्वेदमुपगच्छति}


\twolineshloka
{निर्वेदादात्मसंबोधः संबोधादात्मदर्शनम्}
{शास्त्रार्थदर्शनाद्राजसंस्तप एवानुपश्यति}


\twolineshloka
{दुर्लभो हि मनुष्येन्द्र नरः प्रत्यवमर्शवान्}
{यो वै प्रियसुखे क्षीणे तपः कर्तुं व्यवस्यति}


\twolineshloka
{तपः सर्वगतं तात हीनस्यापि विधीयते}
{जितेन्द्रियस्य दान्तस्य स्वर्गमार्गप्रवर्तकम्}


\twolineshloka
{प्रजापतिः प्रजाः पूर्वमसृजत्तपसा विभुः}
{क्वचित्क्वचिद्ब्रतपरो व्रतान्यास्थाय पार्थिव}


\twolineshloka
{आदित्या वसवो रुद्रास्तथैवाग्न्यश्विमारुताः}
{विश्वेदेवास्तथा साध्याः पितरोऽथ मरुद्गणाः}


\twolineshloka
{यक्षराक्षसगन्धर्वाः सिद्धाश्चान्ये दिवौकसः}
{संसिद्धास्तपसा तात ये चान्ये स्वर्गवासिनः}


\twolineshloka
{ये चादौ ब्राह्मणाः सृष्टा ब्रह्मणा तपसा पुरा}
{ते भावयन्तः पृथिवीं विचरन्ति दिवं तथा}


\twolineshloka
{मर्त्यलोके च राजानो ये चान्ये गृहमेधिनः}
{महाकुलेषु दृश्यन्ते तत्सर्वं तपसः फलम्}


\twolineshloka
{कौशेयानि च वस्त्राणि शुभान्याभरणानि च}
{वाहनासनपानानि तत्सर्वं तपसः फलम्}


\twolineshloka
{मनोनुकूलाः प्रमदा रूपवत्यः सहस्रशः}
{वासः प्रासादपृष्ठे च तत्सर्वं तपसः फलम्}


\twolineshloka
{शयनानि च मुख्यानि भोज्यानि विविधानि च}
{अभिप्रेतानि सर्वाणि भवन्ति शुभकर्मिणाम्}


\twolineshloka
{नाप्राप्यं तपसः किंचिन्त्रैलोक्येऽपि परंतप}
{उपभोगपरित्यागः फलान्यकृतकर्मणाम्}


\twolineshloka
{सुखितो दुःखितो वाऽपि नरो लोभं परित्यजेत्}
{अवेक्ष्य मनसा शास्त्रं बुद्ध्या च नृपसत्तम}


\twolineshloka
{असंतोषोऽसुखथायेति लोभादिन्द्रियविभ्रमः}
{ततोऽस्य नश्यति प्रज्ञा विद्येवाभ्यासवर्जिता}


\twolineshloka
{नष्टप्रज्ञो यदा तु स्यात्तदा न्यायं न पश्यति}
{तस्मात्सुखक्षये प्राप्ते पुमानुग्रतपश्चरेत्}


\twolineshloka
{यदिष्टं तत्सुखं प्राहुर्द्वेष्यं दुःखमिहेष्यते}
{कृताकृतस्य तपसः फलं पश्यस्व यादृशम्}


\twolineshloka
{नित्यं भद्राणि पश्यन्ति विषयांश्चोपभुञ्जते}
{प्राकाश्यं चैवं गच्छन्ति कृत्वा निष्कल्मषं तपः}


\twolineshloka
{अप्रियाण्यवमानाश्च दुःखं बहुविधात्मकम्}
{फलार्थी सत्पथं त्यक्त्वा प्राप्नोति विषयात्मकम्}


\twolineshloka
{धर्मे तपसि दाने च विचिकित्साऽस्य जायते}
{स कृत्वा पापकान्येव निरयं प्रतिपद्यते}


\twolineshloka
{सुखे तु वर्तमानो वै दुःखे वाऽपि नरोत्तम्}
{सुवृत्ताद्यो न चलते शास्त्रचक्षुः स मानवः}


\twolineshloka
{इषुप्रपातमात्रं हि स्पर्शयोगे रतिः स्मृता}
{रसने दर्शने घ्राणे श्रवणे च विशांपते}


\twolineshloka
{ततोऽस्य जायते तीव्रा वेदना तत्क्षयात्पुनः}
{बुधा ये न प्रशंसन्ति मोक्षं सुखमनुत्तमम्}


\twolineshloka
{ततः फलार्थं सर्वस्य भवन्ति ज्यायसे गुणाः}
{धर्मवृद्ध्या च सततं कामार्थाभ्यां न हीयते}


\twolineshloka
{अप्रयत्नागताः सेव्या गृहस्थैर्विषयाः सदा}
{प्रयत्नेनोपगम्यश्च स्वधर्म इति मे मतिः}


\twolineshloka
{मानिनां कुलजातानां नित्यं शास्त्रार्थचक्षुषाम्}
{क्रियाधर्मविमुक्तानामशक्त्या संवृतात्मनाम्}


\twolineshloka
{क्रियमाणं यदा कर्म नाशं गच्छति मानुषम्}
{तेषां नान्यदृते लोके तपसः कर्म विद्यते}


\twolineshloka
{सर्वात्मनाऽनुकुर्वीत गृहस्थः कर्मनिश्चयम्}
{दाक्ष्येण हव्यकव्यार्थं स्वधर्मे विचरन्नृप}


\twolineshloka
{यथा नदीनदाः सर्वे सागरे यान्ति संस्थितिम्}
{एवमाश्रमिणः सर्वे गृहस्थे यान्ति संस्थितिम्}


\chapter{अध्यायः ३०२}
\twolineshloka
{जनक उवाच}
{}


\twolineshloka
{वर्णो विशेषवर्णानां महर्षे केन जायते}
{एतदिच्छाम्यहं ज्ञातुं तद्ब्रूहि वदतां वर}


\threelineshloka
{यदेतज्जायतेऽपत्यं स एवायमिति श्रुतिः}
{कथं ब्राह्मणतो जातो विशेषग्रहणं गतः ॥पराशर उवाच}
{}


\twolineshloka
{एवमेतन्महाराज येन जातः स एव सः}
{तपसस्त्वपकर्षेण जातिग्रहणतां गतः}


\twolineshloka
{सुक्षेत्राच्च सुबीजाच्च पुण्यो भवति संभवः}
{अतोऽन्यतरतो हीनादवरो नाम जायते}


\twolineshloka
{वक्राद्भुजाभ्यामूरुभ्यां पद्भ्यां चैवाथ जज्ञिरे}
{सृजतः प्रजापतेर्लोकानिति धर्मविदो विदुः}


\twolineshloka
{मुखजा ब्राह्मणास्तात बाहुजाः क्षत्रियाः स्मृताः}
{ऊरुजा धनिनो राजन्पादजाः परिचारकाः}


\twolineshloka
{चतुर्णामेव वर्णानामागमः पुरुषर्षभ}
{अतोन्ये त्वतिरिक्ता ये ते वै संकरजाः स्मृताः}


\twolineshloka
{क्षत्रियातिरथाम्बष्ठा उग्रा वैदेहकास्तथा}
{श्वपाकाः पुल्कसाः स्तेना निषादाः सूतमागधाः}


\threelineshloka
{अयोगाः कारणा व्रात्याश्चाण्डालाश्च नराधिप}
{एते चतुर्भ्यो वर्णेभ्यो जायन्ते वै परस्परात् ॥जनक उवाच}
{}


\twolineshloka
{ब्रह्मणैकेन जातानां नानात्वं गोत्रतः कथम्}
{बहूनीह हि लोके वै गोत्राणि मुनिसत्तम}


\threelineshloka
{यत्र तत्र कथं जाताः स्वयोनिं मुनयो गताः}
{शूद्रयोनौ समुत्पन्ना वियोनौ च तथा परे ॥पराशर उवाच}
{}


\twolineshloka
{राजन्नैतद्भवेद्ब्राह्ममपकृष्टेन जन्मना}
{महात्मनां समुत्पत्तिस्तपसा भावितात्मनाम्}


\twolineshloka
{उत्पाद्य पुत्रान्मुनयो नृपते यत्र तत्र ह}
{स्वेनैव तपसा तेषामृषित्वं विदधुः पुनः}


\twolineshloka
{पितामहश्च मे पूर्वमृश्यशृङ्गश्च काश्यपः}
{वेदस्ताण्ड्यः कृपश्चैव काक्षीवत्कमठादयः}


\twolineshloka
{यवक्रीतश्च नृपते द्रोणश्च वदतांवरः}
{आयुर्मतङ्गो दत्तश्च द्रुमदो मात्स्य एव च}


\twolineshloka
{एते स्वां प्रकृतिं प्राप्ता वैदेह तपसो बलात्}
{प्रतिष्ठिता वेदविदो दमेन तपसैव हि}


\twolineshloka
{मूलगोत्राणि चत्वारि समुत्पन्नानि पार्थिव}
{अङ्गिराः कश्यपश्चैव वसिष्ठो भृगुरेव च}


\threelineshloka
{कर्मतोऽन्यानि गोत्राणि समुत्पन्नानि पार्थिव}
{नामधेयानि तपसा तानि च ग्रहणं सताम् ॥जनक उवाच}
{}


\threelineshloka
{विशेषधर्मान्वर्णानां प्रब्रूहि भगवन्मम}
{ततः सामान्यधर्मांश्च सर्वत्र कुशलो ह्यसि ॥पराशर उवाच}
{}


\twolineshloka
{प्रतिग्रहो याजनं च तथैवाध्यापनं नृप}
{विशेषधर्मा विप्राणां रक्षा क्षत्रस्य शोभना}


\twolineshloka
{कृषिश्च पाशुपाल्यं च वाणिज्यं च विशामपि}
{द्विजानां परिचर्या च शूद्रकर्म नराधिप}


\twolineshloka
{विशेषधर्मा नृपते वर्णानां परिकीर्तिताः}
{धर्मान्साधारणांस्तात विस्तरेण शृणुष्व मे}


\twolineshloka
{आनृशंस्यमर्हिसा चाप्रमादः संविभागिता}
{श्राद्धकर्मातिथेयं च सत्यमक्रोध एव च}


\twolineshloka
{स्वेषु दारेषु संतोषः शौचं नित्याऽनसूयता}
{आत्मज्ञानं तितिक्षा च धर्माः साधारणा नृप}


\twolineshloka
{ब्राह्मणाः क्षत्रिया वैश्यास्रयो वर्णा द्विजातयः}
{अत्र तेषामधीकारो धर्मेषु द्विपदां वर}


\twolineshloka
{विकर्मावस्थिता वर्णाः पतन्ति नृपते त्रयः}
{उन्नमन्ति यथा सन्त आश्रित्येह स्वकर्मसु}


\twolineshloka
{न चापि शूद्रः पततीति निश्चयोन चापि संस्कारमिहार्हतीति वा}
{श्रुतिप्रयुक्तं न च धर्ममाप्नुतेन चास्य धर्मे प्रतिषेधनं कृतम्}


\twolineshloka
{वैदेहकं शूद्रमुदाहरन्तिद्विजा महाराज श्रुतोपपन्नाः}
{अहं हि पश्यामि नरेन्द्र देवंविश्वस्य विष्णुं जगतः प्रधानम्}


\twolineshloka
{सतां वृत्तमधिष्ठाय निहीना उद्दिधीर्षवः}
{मन्त्रवर्जं न दुष्यन्ति कुर्वाणाः पौष्टिकीः क्रियाः}


\threelineshloka
{यथायथा हि सद्वॄत्तमालम्बन्तीतरे जनाः}
{यथातथा सुखं प्राप्य प्रेत्य चेह च मोदते ॥जनक उवाच}
{}


\threelineshloka
{किं कर्म दूषयत्येनमथो जातिर्महामुने}
{संदेहो मे समुत्पन्नस्तन्मे व्याख्यातुमर्हसि ॥पराशर उवाच}
{}


\twolineshloka
{असंशयं महाराज उभयं दोषकारकम्}
{कर्म चैव हि जातिश्च विशेषं तु निशामय}


\twolineshloka
{जात्या च कर्मणा चैव दुष्टं कर्म न सेवते}
{जात्या दुष्टश्च यः पापं न करोति स पूरुषः}


\twolineshloka
{जात्या प्रधानं पुरुषं कुर्वाणं कर्म धिक्कृतम्कर्म तद्दूषयत्येनं तस्मात्कर्म न शोभनम् ॥जनक उवाच}
{}


\threelineshloka
{कानि कर्माणि धर्म्याणि लोकेऽस्मिन्द्विजसत्तम्}
{न हिंसन्तीह भूतानि क्रियमाणानि सर्वदा ॥पराशर उवाच}
{}


\twolineshloka
{शृणु मेऽत्र महाराज यन्मां त्वं परिपृच्छसि}
{यानि कर्माण्यहिंस्राणि नरं त्रायन्ति सर्वदा}


\twolineshloka
{संन्यस्याग्नीनुदासीनाः पश्यन्ति विगतज्वराः}
{नैःश्रेयसं कर्मपथं समारुह्य यथाक्रमम्}


\twolineshloka
{प्रश्रिता विनयोपेता दमनित्याः सुसंशिताः}
{पयान्ति स्थानमजरं सर्वकर्मविवर्जिताः}


\twolineshloka
{सर्वे वर्णा धर्मकार्याणि सम्यक्कृत्वा राजन्सत्यवाक्यानि चोक्त्वा}
{त्यक्त्वा धर्मं दारुणं जीवलोकेयान्ति स्वर्गं नात्र कार्यो विचारः}


\chapter{अध्यायः ३०३}
\twolineshloka
{पराशर उवाच}
{}


\twolineshloka
{पिता सखायो गुरवः स्त्रियश्चन निर्गुणानां प्रभवन्ति लोके}
{अनन्यभक्ताः प्रियवादिनश्चहिताश्च वश्याश्च तथैव राजन्}


\twolineshloka
{पिता परं दैवतं मानवानांमातुर्विशिष्टं पितरं वदन्ति}
{ज्ञानस्य लाभं परमं वदन्तिजितेन्द्रियार्थाः परमाप्नुवन्ति}


\twolineshloka
{रणाजिरे यत्र शराग्निसंस्तरेनृपात्मजो घातमवाप्य दह्यते}
{प्रयाति लोकानमरैः सुदुर्लभान्निषेवते स्वर्गफलं यथासुखम्}


\twolineshloka
{श्रान्तं भीतं भ्रष्टशस्त्रं रुदन्तंपराङ्भुखं पारिवर्हैश्च हीनम्}
{अनुद्यन्तं रोगिणं याचमानंन वै हिंस्याद्बालवृद्धौ च राजन्}


\twolineshloka
{पारिबर्हैः सुसंयुक्तमुद्यतं तुल्यतां गतम्}
{अतिक्रमेत्तं नृपतिः संग्रामे क्षत्रियात्मजम्}


\twolineshloka
{तुल्यादिह वधः श्रेयान्विशिष्टाच्चेति निश्चयः}
{निहीनात्कातराच्चैव कृपणाद्गर्हितो वधः}


\twolineshloka
{पापात्पापसमाचारान्निहीनाच्च नराधिप}
{पाप एष वधः प्रोक्तो नरकायेति निश्चयः}


\twolineshloka
{न कश्चित्राति वै राजन्दिष्टान्तवशमागतम्}
{सावशेषायुषं चापि कश्चिन्नैवापकर्षति}


\twolineshloka
{स्निग्धैश्च क्रियमाणानि कर्माणीह निवर्तयेत्}
{हिंसात्मकानि सर्वाणि नायुरिच्छेत्परायुषा}


\twolineshloka
{गृहस्थानां तु सर्वेषां विनाशमभिकाङ्क्षताम्}
{निधनं शोभनं तात पुलिनेषु क्रियावताम्}


\twolineshloka
{आयुषि क्षयमापन्ने पञ्चत्वमुपगच्छति}
{तथा ह्यकारणाद्भवति कारणैरुपपादितम्}


\twolineshloka
{तथा शरीरं भवति देहाद्येनोपपादितम्}
{अध्वानं गतकश्चायं प्राप्तश्चायं गृहाद्गृहम्}


\twolineshloka
{द्वितीयं कारणं तत्र नान्यत्किंचन विद्यते}
{तद्देहं देहिनां युक्तं मोक्षभूतेषु वर्तते}


\twolineshloka
{शिरास्नाय्वस्थिसंघातं बीभत्सामेध्यसंकुलम्}
{भूतानामिन्द्रियाणां च गुणानां च समागमम्}


\twolineshloka
{त्वगन्तं देहमित्याहुर्विद्वांसोऽध्यात्मचिन्तकाः}
{गुणैरपि परिक्षीणं शरीरं मर्त्यतां गतम्}


\twolineshloka
{शरीरिणा परित्यक्तं निश्चेष्टं गतचेतनम्}
{भूतैः प्रकृतिमापन्नैस्ततो भूमौ निमज्जति}


\threelineshloka
{भावितं कर्मयोगेन जायते तत्रतत्र ह}
{इदं शरीरं वैदेह म्रियते यत्रयत्र ह}
{तत्प्रपाते परो दृष्टो विसर्गः कर्मणस्तथा}


\twolineshloka
{न जायते तु नृपते कंचित्कालमयं पुनः}
{परिभ्रमति भूतात्मा द्यामिवाम्बुधरो महान्}


\twolineshloka
{स पुनर्जायते राजन्प्राप्येहायतनं नृपः}
{मनसः परमो ह्यात्मा इन्द्रियेभ्यः परं मनः}


\threelineshloka
{विविधानां च भूतानां जङ्गमाः परमा नृप}
{जङ्गमानामपि तथा द्विपदाः परमा मताः}
{द्विपदानामपि तथा द्विजा वै परमाः स्मृताः}


\twolineshloka
{द्विजानामपि राजेन्द्र प्रज्ञावन्तः परा मताः}
{प्राज्ञानामात्मसंबुद्धाः संबुद्धानाममानिनः}


\twolineshloka
{जातमन्वेति मरणं नृणामिति विनिश्चयः}
{अन्तवन्ति हि कर्माणि सेवन्ते गुणतः प्रजाः}


\twolineshloka
{आपन्ने तूत्तरां काष्ठां सूर्ये यो निधनं व्रजेत्}
{नक्षत्रे च मुहूर्ते च पुण्ये राजन्स पुण्यकृत्}


\twolineshloka
{अयोजयित्वा क्लेशेन जनं प्लाप्य च दुष्कृतम्}
{मृत्युनाऽऽत्मकृतेनेह कर्म कृत्वाऽऽत्मशक्तितः}


\twolineshloka
{विषमुद्बन्धनं दाहो दस्युहस्तात्तथ वधः}
{दंष्ट्रिभ्यश्च पशुभ्यश्च प्राकृतो वध उच्यते}


\twolineshloka
{न चैभिः पुण्यकर्माणो युज्यन्ते चाभिसंधिजैः}
{एवंविधैश्च बहुभिरपरैः प्राकृतैरपि}


\twolineshloka
{ऊर्ध्वं भित्त्वा प्रतिष्ठन्ते प्राणाः पुण्यवतां नृप}
{मध्यतो मध्यपुण्यानामधो दुष्कृतकर्मणाम्}


\twolineshloka
{एकः शत्रुर्न द्वितीयोस्ति शत्रुरज्ञानतुल्यः पुरुषस्य राजन्}
{येनावृतः कुरुते संप्रयुक्तोघोराणि कर्माणि सुदारुणानि}


\twolineshloka
{प्रबोधनार्थं श्रुतिधर्मयुक्तंवृद्धानुपास्य प्रभवेत यस्य}
{प्रयत्नसाध्यो हि स राजपुत्रप्रज्ञाशरेणोन्मथितः परैति}


\twolineshloka
{अधीत्य वेदं तपसा ब्रह्मचारीयज्ञाञ्शक्त्या सन्निसृज्येह पञ्च}
{वनं गच्छेत्पुरुषो धर्मकामःश्रेयः कृत्वा स्थापयित्वा स्ववंशम्}


\twolineshloka
{उपभोगैरपि त्यक्तं नात्मानं सादयेन्नरः}
{चण्डालत्वेऽपि मानुष्यं सर्वथा तात शोभनम्}


\twolineshloka
{इयं हि योनिः प्रथमा यां प्राप्य जगतीयते}
{आत्मा वै शक्यते त्रातुं कर्मभिः शुभलक्षणैः}


\twolineshloka
{कथं न विप्रणश्येम योनितोस्या इति प्रभो}
{कुर्वन्ति धर्मं मनुजाः श्रुतिप्रामाण्यदर्शनात्}


\twolineshloka
{यो दुर्लभतरं प्राप्य मानुष्यं द्विषते नरः}
{धर्मावमन्ता कामात्मा भवेत्स खलु वञ्च्यते}


\twolineshloka
{यस्तु प्रीतिपुराणेन चक्षुषा तात पश्यति}
{दीपोपमानि भूतानि यावदर्थान्न पश्यति}


\twolineshloka
{सान्त्वेनानुप्रदानेन प्रियवादेन चाप्युत}
{समदुःखसुखो भूत्वा स परत्र महीयते}


\twolineshloka
{दानं त्यागः शोभना मुर्तिमद्भ्योभूयः प्लाव्यं तपसा वै शरीरम्}
{सरस्वतीनैमिषपुष्करेषुये चाप्यन्ये पुण्यदेशाः पृथिव्याम्}


\twolineshloka
{गृहेषु येषामसवः पतन्तितेषामथो निर्हरणं प्रशस्तम्}
{यानेन वै प्रापणं च श्मशानेशुचौ देशे विधिना चैव दाहः}


\twolineshloka
{इष्टिः पुष्टिर्यजनं याजनं चदार्ग पुण्यानां कर्मणां च प्रयोगः}
{शक्त्या पित्र्यं यच्च किंचित्प्रशस्तंसर्वाण्यात्मार्थे मानवोऽयं करोति}


\twolineshloka
{`गृहस्थानां च सर्वेषां विनाशमभिकाङ्क्षताम्}
{निधनं शोभनं तात पुलिनेषु क्रियावताम् ॥'}


\threelineshloka
{धर्मशास्त्राणि वेदाश्च ष़डङ्गानि नराधिप}
{श्रेयसोर्थे विधीयन्ते नरस्याक्लिष्टकर्मणः ॥भीष्म उवाच}
{}


\twolineshloka
{एतद्वै सर्वमाख्यातं मुनिना सुमहात्मना}
{विदेहराजाय पुरा श्रयेसोर्थे नराधिप}


\chapter{अध्यायः ३०४}
\twolineshloka
{भीष्म उवाच}
{}


\threelineshloka
{पुनरेव तु पप्रच्छ जनको मिथिलाधिपः}
{पराशरं महात्मानं धर्मे परमनिश्चयम् ॥जनक उवाच}
{}


\threelineshloka
{किं श्रेयः का गतिर्ब्रह्मन्किं कृतं न विनश्यति}
{क्व गतो न निवर्तेत तन्मे ब्रूहि महामते ॥पराशर उवाच}
{}


\twolineshloka
{असङ्गः श्रेयसो मूलं ज्ञानं ज्ञानगतिः परा}
{चीर्णं तपो न प्रणश्येद्वापः क्षेत्रे न नश्यति}


\twolineshloka
{छित्त्वाऽधर्ममयं पाशं यदा धर्मेऽभिरज्यते}
{दत्त्वाऽभयकृतं दानं तदा सिद्धिमवाप्नुते}


\twolineshloka
{यो ददाति सहस्राणि गवामश्वशतानि च}
{अभयं सर्वभूतेभ्यः सदा तमभिवर्तते}


\twolineshloka
{वसन्विषयमध्येऽपि न वसत्येव बुद्धिमान्}
{संवसत्येव दुर्बुद्धिरसत्सु विषयेष्वपि}


\twolineshloka
{नाधर्मः श्लिष्यते प्राज्ञं पयः पुष्करपर्णवत्}
{अप्राज्ञमधिकं पापं श्लिष्यते जतुकाष्ठवत्}


\twolineshloka
{नाधर्मः कारणापेक्षी कर्तारमभिमुञ्चति}
{कर्ता खलु यथाकालं ततः समभिपद्यते}


\threelineshloka
{न भिद्यन्ते कृतात्मान आत्मप्रत्ययदर्शिनः}
{बुद्धिकर्मेन्द्रियाणां हि प्रमत्तो यो न बुध्यते}
{शुभाशुभे प्रसक्तात्मा प्राप्नोति सुमहद्भयम्}


\twolineshloka
{वीतरागो जितक्रोधः सम्यग्भवति यः सदा}
{विषये वर्तमानोऽपि न स पापेन युज्यते}


\twolineshloka
{मर्यादायां वर्तमानोऽपि नावसीदति}
{पुष्टस्रोत इवासक्तः स्फीतो भवति संचयः}


\twolineshloka
{यथा भानुगतं तेजो मणिः शुद्धः समाधिना}
{आदत्ते राजशार्दूल तथा योगः प्रवर्तते}


\twolineshloka
{यथा तिलानामिह पुण्यसंश्रयात्पृथक्पृथग्याति गुणोऽतिसाम्यताम्}
{तथा नराणां भुवि भावितात्मनांयथाश्रयं सत्वगुणः प्रवर्तते}


\twolineshloka
{जहाति दारान्विविधाश्च संपदःपदं च यानं विविधाश्च सत्क्रियाः}
{त्रिविष्टपे जातमतिर्यदा नरस्तदाऽस्य बुद्धिर्विषयेषु भिद्यते}


\twolineshloka
{प्रसक्तबुद्धिर्विषयेषु यो नरोन बुध्यते ह्यात्महितं कथंचन}
{स सर्वभावानुगतेन चेतसानृपाऽऽमिषेणेव झषो विकृष्यते}


\twolineshloka
{संघातवन्मर्त्यलोकः परस्परमपाश्रितः}
{कदलीगर्भनिःसारो नौरिवाप्सु निमज्जति}


\twolineshloka
{न धर्मकालः पुरुषस्य निश्चितोन चापि मृत्युः पुरुषं प्रतीक्षते}
{सदा हि धर्मस्य क्रियैव शोभनातदा नरो मृत्युमुखान्निवर्तते}


\twolineshloka
{यथाऽन्धः स्वगृहे युक्तो ह्यभ्यासादेव गच्छति}
{तथा युक्तेन मनसा प्राज्ञो गच्छति तां गतिम्}


\twolineshloka
{मरणं जन्मनि प्रोक्तं जन्म वै मरणाश्रितम्}
{अविद्वान्मोक्षधर्मेषु बद्धो भ्रमति चक्रवत्}


\threelineshloka
{बुद्धिमार्गप्रयातस्य सुखं त्विह परत्र च}
{विस्तराः क्लेशसंयुक्ताः संक्षेपास्तु सुखावहाः}
{परार्थं विस्तराः सर्वे त्यागमांत्महितं विदुः}


\twolineshloka
{यथा मृणालानुगतमाशु मुञ्चति कर्दमम्}
{तथाऽऽत्मा पुरुषस्येह मनसा परिमुच्यते}


\twolineshloka
{मनः प्रणयतेऽऽत्मानं स एनमभियुञ्जति}
{युक्तो यदा स भवति तदा तं पश्यते परम्}


\twolineshloka
{परार्थे वर्तमानस्तु स्वं कार्यं योऽभिमन्यते}
{इन्द्रियार्थेषु सक्तः स स्वकार्यात्परिहीयते}


\twolineshloka
{अधस्तिर्यग्गतिं चैव स्वर्गे चैव परां गतिम्}
{प्राप्नोति सुकृतैरात्मा प्राज्ञस्येहेतरस्य च}


\twolineshloka
{मृन्मये भाजने पक्वे यथा वै नश्यति द्रवः}
{तथा शरीरं तपसा तप्तं विषयमश्नुते}


\twolineshloka
{विषयानश्नुते यस्तु न स भोक्ष्यत्यसंशयम्}
{यस्तु भोगांस्त्यजेदात्मा स वै भोक्तुं व्यवस्यति}


\twolineshloka
{नीहारेण हि संवीतः शिश्नोदरपरायणः}
{जात्यन्ध इव पन्थानमावृतात्मा न बुध्यते}


\twolineshloka
{वणिग्यथा समुद्राद्वै यथार्थं लभते धनम्}
{तथा मर्त्यार्णवाज्जन्तोः कर्मविज्ञानतो गतिः}


\twolineshloka
{अहोरात्रमये लोके जरारूपेण संचरन्}
{मृत्युर्ग्रसति भूतानि पवनं पन्नगो यथा}


\twolineshloka
{स्वयं कृतानि कर्माणि जातो जन्तुः प्रपद्यते}
{नाकृतं लभते कश्चित्किंचिदत्र प्रियाप्रियम्}


\twolineshloka
{सयानं यान्तमासीनं प्रवृत्तं विषयेषु च}
{शुभाशुभानि कर्माणि प्रपद्यन्ते नरं सदा}


\twolineshloka
{न ह्यन्यत्तीरमासाद्य पुनस्तर्तुं व्यवस्यति}
{दुर्लभो दृश्यते ह्यस्य विनिपातो महार्णवे}


\twolineshloka
{यथा भावावसन्ना हि नौर्महाम्भसि तन्तुना}
{यथा मनोभियोगाद्वै शरीरं प्रचिकीर्षति}


\twolineshloka
{यथा समुद्रमभितः संश्रिताः सरितोऽपराः}
{तथाऽन्याप्रकृतिर्योगादभिसंश्रियते सदा}


\twolineshloka
{स्नेहपाशैर्बहुविधैरासक्तमनसो नराः}
{प्रकृतिस्था विषीदन्ति जले सैकतवेश्मवत्}


\twolineshloka
{शरीरगृहसंस्थस्य शौचतीर्थस्य देहिनः}
{बुद्धिमार्गप्रयातस्य सुखं त्विह परत्र च}


\twolineshloka
{विस्तराः क्लेशसंयुक्ताः संक्षेपास्तु सुखावहाः}
{परार्थं विस्तराः सर्वे त्यागमात्महितं विदुः}


\twolineshloka
{संकल्पजो मित्रवर्गो ज्ञातयः कारणात्मकाः}
{भार्या पुत्रश्च दासश्च स्वमर्थमनुयुञ्जते}


\twolineshloka
{न माता न पिता किंचित्कस्यचित्प्रतिपद्यते}
{दानपथ्यौदनो जन्तु स्वकर्मफलमश्नुते}


\twolineshloka
{माता पुत्रः पिता भ्राता भार्या मित्रजनस्तथा}
{अष्टापदपदस्थाने लाक्षामुद्रेव लक्ष्यते}


\twolineshloka
{सर्वाणि कर्माणि पुराकृतानिशुभाशुभान्यात्मनो यान्ति जन्तोः}
{उपस्थितं कर्मफलं विदित्वाबुद्धिं तथा चोदयतेऽन्तरात्मा}


\twolineshloka
{व्यवसायं समाश्रित्य सहायान्योऽधिगच्छति}
{न तस्य कश्चिदारम्भः कदाचिदवसीदति}


\twolineshloka
{अद्वैधमनसं युक्तं शूरं धीरं विपश्चितम्}
{न श्रीः संत्यजते नित्यमादित्यमिव रश्मयः}


\twolineshloka
{आस्तिक्यव्यवसायाभ्यामुपायान्वितया धिया}
{य आरभत्यनिन्द्यात्मा न सोऽर्थात्परिसीदति}


\twolineshloka
{सर्वःस्वानि शुभाशुभानि नियतं कर्माणि जन्तुःस्वयंगर्भात्संप्रतिपद्यते तदुभयं यत्तेन पूर्वं कृतम्}
{मृत्युश्चापरिहारवान्समगतिः कालेन विच्छेदिनादारोश्चूर्णमिवाश्मसारविहितं कर्मान्तिकं प्रापयेत्}


\threelineshloka
{स्वरूपतामात्मकृतं च विस्तरंकुलान्वयं द्रव्यसमृद्धिसंचयम्}
{नरो हि सर्वो भलते यथाकृतंशुभाशुभेनात्मकृतेन कर्मणा ॥भीष्म उवाच}
{}


\twolineshloka
{इत्युक्तो जनको राजन्याथातथ्यं मनीषिणा}
{श्रुत्वा धर्मविदां श्रेष्ठः परां मुदमवाप ह}


\chapter{अध्यायः ३०५}
\twolineshloka
{युधिष्ठिर उवाच}
{}


\threelineshloka
{सत्यं दमं क्षमां प्रज्ञां प्रशंसन्ति पितामह}
{विद्वांसो मनुजा लोके कथमेतन्मतं तव ॥भीष्म उवाच}
{}


\twolineshloka
{अत्र ते वर्तयिष्येऽहमितिहासं पुरातनम्}
{साध्यानामिह संवादं हंसस्य च युधिष्ठिर}


\threelineshloka
{हंसो भूत्वाऽथ सौवर्णस्त्वजो नित्यः प्रजापतिः}
{स वै पर्येति लोकांस्त्रीनथ साध्यानुपागमत् ॥साध्या ऊचुः}
{}


\twolineshloka
{शकुने वयं स्म देवा वै साध्यास्त्वामनुयुङ्क्ष्महे}
{पृच्छामस्त्वां मोक्षधर्मं भवांश्च किल मोक्षवित्}


\twolineshloka
{श्रुतोसि नः पण्डितो धीरवादीसाधुः शब्दश्चरते ते पतत्रिन्}
{किं मन्यसे श्रेष्ठतमं द्विज त्वंकस्मिन्मनस्ते रमते महात्मन्}


\threelineshloka
{तन्नः कार्यं पक्षिवर प्रशाधियत्कार्याणां मन्यसे श्रेष्ठमेकम्}
{यत्कृत्वा वै पुरुषः सर्वबन्धैर्विमुच्यते विहगेन्द्रेह शीघ्रम् ॥हंस उवाच}
{}


\twolineshloka
{इदं कार्यममृताशाः शृणुध्वंतपो दमः सत्यमात्माभिगुप्तिः}
{ग्रन्थीन्विमुच्य हृदयस्य सर्वान्प्रियाप्रिये स्वं वशमानयीत}


\twolineshloka
{नारुतुदः स्यान्न नृशंसवादीन हीनतः परमभ्याददीत}
{ययाऽस्य वाचा पर उद्विजेतन तां वदेदुशतिं पापलोक्याम्}


\twolineshloka
{वाक्सायका वदनान्निष्यतन्तियैराहतः शोचति रात्र्यहानि}
{परस्य नामर्मसु ते पतन्तितान्पण्डितो नावसृजेत्परेषु}


\twolineshloka
{परश्चेदेनमतिवादवाणैर्भृशं विध्येच्छम एवेह कार्यः}
{संरोष्यमाणः प्रतिहृष्यते यःस आदत्ते सुकृतं वै परस्य}


\twolineshloka
{क्षेपायमाणमभिषङ्गव्यलीकंनिगृह्णाति ज्वलितं यश्च मन्युम्}
{अदुष्टचेता मुदितोऽनसूयुःस आदत्ते सुकृतं वै परेषाम्}


\twolineshloka
{आक्रुश्यमानो न वदामि किंचित्क्षमाम्यहं ताड्यमानश्च नित्यम्}
{श्रेष्ठं ह्येतद्यत्क्षमामाहुरार्याःसत्यं तथैवार्जवमानृशंस्यम्}


\twolineshloka
{वेदस्योपनिषत्सत्यं सत्यस्योपनिषद्दमः}
{दमस्योपनिषन्मोक्ष एतत्सर्वानुशासनम्}


\twolineshloka
{वाचो वेगं मनसः क्रोधवेगंविधित्सावेगमुदरोपस्थवेगम्}
{एतान्वेगान्यो विषहेदुदीर्णांस्तं मन्येऽहं ब्राह्मणं वै मुनिं च}


\twolineshloka
{अक्रोधनः क्रुध्यतां वै विशिष्टस्तथा तितिक्षुरतितिक्षोर्विशिष्टः}
{अमानुषान्मानुषो वै विशिष्टस्तथाऽज्ञानाज्ज्ञानवान्वै विशिष्टः}


\twolineshloka
{आक्रुश्यमानो नाक्रोशेन्मन्युरेवं तितिक्षतः}
{आक्रोष्टारं निर्दहति सुकृतं चास्य विन्दति}


\twolineshloka
{यो नायुक्तः प्राह रूक्षं प्रियं वायो वा हतो न प्रतिहन्ति धैर्यात्}
{पापं च यो नेच्छति तस्य हन्तुस्तस्येह देवाः स्पृहयन्ति नित्यम्}


\twolineshloka
{पापीयसः क्षमेतैव श्रेयसः सदृशस्य च}
{विमानितो हतोक्रुष्ट एवं सिद्धिं गमिष्यति}


\twolineshloka
{सदाऽहमार्यान्निभृतोप्युपासेन मे विधित्सोत्सहते न रोषः}
{न चाप्यहं लिप्समानः परैमिन चैव किंचिद्विषमेण यामि}


\twolineshloka
{नाहं शप्तः प्रतिशपामि कंचिद्दमं द्वारं ह्यमृतस्येह वेद्मि}
{गुह्यं ब्रह्म तदिदं ब्रवीमिन मानुषाच्छ्रेष्ठतरं हि किंचित्}


\twolineshloka
{निर्मुच्यमानः पापेभ्यो घनेभ्य इव चन्द्रमाः}
{विरजाः कालमाकाङ्क्षन्धीरो धैर्येण सिध्यति}


\twolineshloka
{यः सर्वेषां भवति ह्यर्चनीयउत्सेचने स्तम्भ इवाभिजातः}
{यस्मै वाचं सुप्रसन्नां वदन्तिस वै देवान्गच्छति संयतात्मा}


\twolineshloka
{न तथा वक्तुमिच्छन्ति कल्याणान्पुरुषे गुणान्}
{यथैषां वक्तुमिच्छन्ति नैर्गुण्यमनुयुञ्जकाः}


\twolineshloka
{यस्य वाङ्भनसी गुप्ते सम्यक्प्रणिहिते सदा}
{वेदास्तपश्च त्यागश्च स इदं सर्वमाप्नुयात्}


\twolineshloka
{आक्रोशनविमानाभ्यां नाबुधान्बोधयेद्बुधः}
{तस्मान्न वर्धयेदन्यं न चात्मानं विहिंसयेंत्}


\twolineshloka
{अमृतस्येव संतृप्येदवमानस्य पण्डितः}
{सुखं ह्यवमतः शेते योऽवमन्ता स नश्यति}


\twolineshloka
{यत्क्रोधनो यजति यद्ददातियद्वा तपस्तप्यति यज्जुहोति}
{वैवस्वतस्तद्धरतेऽस्य सर्वंमोघः श्रमो भवति हि क्रोधनस्य}


\twolineshloka
{चत्वारि यस्य द्वाराणि सुगुप्तान्यमरोत्तमाः}
{उपस्थमुदरं हस्तौ वाक्चतुर्थी स धर्मवित्}


\twolineshloka
{सत्यं दमं ह्यार्जवमानृशंस्यंधृतिं तितिक्षामभिसेवमानः}
{स्वाध्यायनित्योऽस्पृहयन्यरेषामेकान्तशील्यूर्ध्वगतिर्भवेत्सः}


\twolineshloka
{सर्वान्देदाननुचरन्वत्सवच्चतुरः स्तनान्}
{न पावनतमं किंचित्सत्याद्गध्यगमं क्वचित्}


\twolineshloka
{आचक्षेऽहं मनुष्येभ्यो देवेभ्यः प्रतिसंचरन्}
{सत्यं स्वर्गस्य सोपानं पारावारस्य नौरिव}


\twolineshloka
{यादृशैः संविवदते यादृशांश्चोपसेवते}
{यादृगिच्छेच्च भवितुं तादृग्भवति पूरुषः}


\twolineshloka
{यदि सन्तं सेवति यद्यसन्तंतपस्विनं यदि वा स्तेनमेव}
{वासो यथा रागवशं प्रयातितथा स तेषां वशमभ्युपैति}


\twolineshloka
{सदा देवाः साधुभिः संवदन्तेन मानुषं विषयं यान्ति द्रष्टुम्}
{नेन्दुः समः स्यादसमो हि वायुरुच्चावचं विषयं यः स वेद}


\twolineshloka
{अदुष्टं वर्तमाने तु हृदयान्तरपूरुषे}
{तेनैव देवाः प्रीयन्ते सतां मार्गस्थितेन वै}


\twolineshloka
{विश्नोदरे ये निरताः सदैवस्तेना नरा वाक्यरुषाश्च नित्यम्}
{अपेतधर्मानिति तान्विदित्वादूराद्देवः संपरिवर्जयन्ति}


\twolineshloka
{न वै देवा हीनसत्वेन तोष्याःसर्वाशिना दुष्कृतकर्मणा वा}
{सत्यव्रता ये तु नराः कृतज्ञाधर्मे रतास्तैः सह संभजन्ते}


\threelineshloka
{अव्याहृतं व्याहृताच्छ्रेय आहुःसत्यं वदेद्व्याहृतं तद्द्वितीयम्}
{धर्म्यं वदेद्व्याहृतं तत्तृतीयंप्रियं वदेद्व्याहृतं तच्चतुर्थम् ॥साध्या ऊचुः}
{}


\threelineshloka
{केनायमावृतो लोकः केन वा न प्रकाशते}
{केन त्यजति मित्राणि केन स्वर्गं न गच्छति ॥हंस उवाच}
{}


\threelineshloka
{अज्ञानेनावृतो लोको मात्सर्यान्न प्रकाशते}
{लोभात्त्यजति मित्राणि सङ्गात्स्वर्गं न गच्छति ॥साध्या ऊचः}
{}


\threelineshloka
{कः स्विदेको रमते ब्राह्मणानांकः स्विदेको बहुभिर्जोषमास्ते}
{कः स्विदेको बलवान्दुर्बलोपिकः स्विदेषां कलहं नान्ववैति ॥हंस उवाच}
{}


\twolineshloka
{प्राज्ञ एको रमते ब्राह्मणानांप्राज्ञश्चैको बहुभिर्जोषमास्तेप्राज्ञ एको बलवान्दुर्बलोऽपिप्राज्ञ एषां कलहं नान्वबैति ॥साध्या ऊचुः}
{}


\threelineshloka
{किं ब्राह्मणानां देवत्वं किंच साधुत्वमुच्यते}
{असाधुत्वे च किं तेषां किमेषां मानुषं मतम् ॥हंस उवाच}
{}


\threelineshloka
{स्वाध्याय एषां देवत्वं व्रतं साधुत्वमुच्यते}
{असाधुत्वं परीवादो मृत्युर्मानुष्यमुच्यते ॥भीष्म उवाच}
{}


\twolineshloka
{` इत्युक्त्वा परमो देवो भगवान्नित्य अव्ययः}
{साध्यैर्देवगणैः सार्धं दिवमेवारुरोह सः}


\twolineshloka
{एतद्यशस्यमायुष्यं पुण्यं स्वर्गाय च ध्रुवम्}
{दर्शितं देवदेवेन परमेणाव्ययेन च ॥'}


\twolineshloka
{संवाद इत्ययं श्रेष्ठः साध्यानां परिकीर्तितः}
{क्षेत्रं वै कर्मणां योनिः सद्भावः सत्यमुच्यते}


\chapter{अध्यायः ३०६}
\twolineshloka
{युधिष्ठिर उवाच}
{}


\threelineshloka
{साङ्ख्ये योगे च मे तात विशेषं वक्तुमर्हसि}
{तव धर्मज्ञ सर्वं हि विदितं कुरुसत्तम ॥भीष्म उवाच}
{}


\twolineshloka
{साङ्ख्याः साङ्ख्यं प्रशंसन्ति योगा योगं द्विजातयः}
{वदन्ति कारणं श्रेष्ठं स्वपक्षोद्भावनाय वै}


\twolineshloka
{अनीश्वरः कथं मुच्येदित्येवं शत्रुसूदन}
{वदन्ति कारणश्रैष्ठ्यं योगाः सम्यङ्भनीषिणः}


\twolineshloka
{वदन्ति कारणं चेदं साङ्ख्याः सम्यग्द्विजातयः}
{विज्ञायेह गतीः सर्वा विरक्तो विषयेषु यः}


\twolineshloka
{ऊर्ध्वं च देहात्सुव्यक्तं विमुच्येदिति नान्यथा}
{एतदाहुर्महाप्राज्ञाः साङ्ख्यं वै मोक्षदर्शनम्}


\twolineshloka
{स्वपक्षे कारणं ग्राह्यं समर्थं वचनं हितम्}
{शिष्टानां हि मतं ग्राह्यं त्वद्विधैः शिष्टसंमतैः}


\twolineshloka
{प्रत्यक्षहेतवो योगाः साङ्ख्याः शास्त्रविनिश्चयाः}
{उभे चैते मते तत्त्वे मम तात युधिष्ठिर}


\twolineshloka
{उभे चैते मते ज्ञाने नृपते शिष्टसंमते}
{अनुष्ठिते यथाशास्त्रं नयेतां परमां गतिम्}


\fourlineindentedshloka
{तुल्यं शौचं तयोरेकं दया भूतेषु चानघ}
{व्रतानां धारणं तुल्यं दर्शनं न समं तयोः}
{`तयोस्तु दर्शनं सम्यक्सूक्ष्माभावे प्रसज्यते ॥' युधिष्ठिर उवाच}
{}


\threelineshloka
{यदि तुल्यं व्रतं शौचं दया चात्र फलं तथा}
{न तुल्यं दर्शनं कस्मात्तन्मे ब्रूहि पितामह ॥भीष्म उवाच}
{}


\twolineshloka
{रागं मोहं तथा स्नेहं कामं क्रोधं च केवलम्}
{योगाच्छित्त्वा ततो दोषान्पञ्चैतान्प्राप्नुवन्ति ते}


\twolineshloka
{यथा चानिमिषाः स्थूला जालं छित्त्वा पुनर्जलम्}
{प्राप्नुवन्ति तथा योगास्तत्पदं वीतकल्मषाः}


\twolineshloka
{तथैव वागुरां छित्त्वा बलवन्तो यथा मृगाः}
{प्राप्नुयुर्विमलं मार्गं विमुक्ताः सर्वबन्धनैः}


\twolineshloka
{लोभजानि तथा राजन्बन्धनानि बलान्विताः}
{छित्त्वा योगात्परं मार्गं गच्छन्ति विमलं शिवम्}


\twolineshloka
{अबलाश्च मृगा राजन्वागुरासु यथा परे}
{विनश्यन्ति न संदेहस्तद्वद्योगबलादृते}


\twolineshloka
{बलहीनाश्च कौन्तेय यथा जालं गता झषाः}
{अन्तं गच्छन्ति राजेन्द्र योगास्तद्वत्सुदुर्बलाः}


\twolineshloka
{यथा च शकुनाः सूक्ष्मं प्राप्य जालमरिंदम}
{तत्र सक्ता विपद्यन्ते मुच्यन्ते च बलान्विताः}


\twolineshloka
{कर्मजैर्बन्धनैर्बद्धास्तद्वद्योगाः परंतप}
{अबला वै विनश्यन्ति मुच्यन्ते च बलान्विताः}


\twolineshloka
{अल्पकश्च यथा राजन्वह्निः शाम्यति दुर्बलः}
{आक्रान्त इन्धनैः स्थूलैस्तद्वद्योगो बलः प्रभो}


\twolineshloka
{स एव च यदा राजन्वह्निर्जातबलः पुनः}
{समीरणयुतः क्षिप्रं दहेत्कृत्स्नां महीमपि}


\twolineshloka
{तद्वज्जातबलो योगी दीप्ततेजा महाबलः}
{अन्तकाल इवादित्यः कृत्स्नं संशोषयेज्जगत्}


\twolineshloka
{दुर्बलश्च यथा राजन्स्रोतसा हियते नरः}
{बलहीनस्तथा योगो विषयैर्ह्रियतेऽवशः}


\twolineshloka
{तदेव च महास्रोतो विष्टम्भयति वारणः}
{तद्वद्योगबलं लब्ध्वा व्यूहते विषयान्बहून्}


\twolineshloka
{विशन्ति चावशाः पार्थ योगाद्योगबलान्विताः}
{प्रजापतीनृषीन्देवान्महाभूतानि चेश्वराः}


\twolineshloka
{न यमो नान्तकः क्रुद्धो न नृत्युर्भीमविक्रमः}
{ईशते नृपते सर्वे योगस्यामिततेजसः}


\twolineshloka
{आत्मनां च सहस्राणि बहूनि भरतर्षभ}
{योगः कुर्याद्बलं प्राप्य तैश्च सर्वैर्महीं चरेत्}


\twolineshloka
{प्राप्नुयाद्विषयान्कश्चित्पुनश्चोग्रं तपश्चरेत्}
{संक्षिपेच्च पुनस्तात सूर्यस्तेतोगुणानिव}


\twolineshloka
{बलस्थस्य हि योगस्य बन्धनेशस्य पार्थिव}
{विमोक्षे प्रभविष्णुत्वमुपपन्नमसंशयम्}


\twolineshloka
{बलानि योगप्राप्तानि मयैतानि विशांपते}
{निदर्शनार्थं सूक्ष्माणि वक्ष्यामि च पुनस्तव}


\twolineshloka
{आत्मनश्च समाधाने धारणां प्रति वा विभो}
{निदर्शनानि सूक्ष्माणि शृणु मे भरतर्षभ}


\twolineshloka
{अप्रमत्तो यथा धन्वी लक्ष्यं हन्ति समाहितः}
{युक्तः सम्यक्तथा योगी मोक्षं प्राप्नोत्यसशयम्}


\twolineshloka
{स्नेहपूर्णे यथा पात्रे मन आधाय निश्चलम्}
{पुरुषो युक्त आरोहेत्सोपानं युक्तमानसः}


\twolineshloka
{युक्तस्तथाऽयमात्मानं योगः षार्थिव निश्चलम्}
{करोत्यमलमात्मानं भास्करोपमदर्शनम्}


\twolineshloka
{यथा च नावं कौन्तेय कर्णधारः समाहितः}
{महार्णवगतां शीघ्रं नयेत्पार्थिव पत्तनम्}


\twolineshloka
{तद्वदात्मसमाधानं युक्त्वा योगेन तत्ववित्}
{दुर्गमं स्थानमाप्नोति हित्वा देहमिमं नृप}


\twolineshloka
{सारथिश्च यथा युक्त्वा सदश्वान्सुसमाहितः}
{देशमिष्टं नयत्याशु धन्विनं पुरुषर्षभ}


\twolineshloka
{तथैव नृपते योगी धारणासु समाहितः}
{प्राप्नोत्याशु परं स्थानं लक्षं मुक्त इवाशुगः}


\twolineshloka
{आवेश्यात्मनि चात्मानं योगी तिष्ठति योचलः}
{पापं हन्ति पुनीतानां पदमाप्नोति सोऽजरम्}


\twolineshloka
{नाभ्यां कण्ठे च शीर्षे च हृदि वक्षसि पार्श्वयोः}
{दर्शने श्रवणे चापि घ्राणे चामितविक्रम}


\twolineshloka
{स्थानेष्वेतेषु यो योगी महाव्रतसमाहितः}
{आत्मना सूक्ष्ममात्मानं युङ्क्ते सम्यग्विशांपते}


\threelineshloka
{स शीघ्रमचलप्रख्यं कर्म दग्ध्या शुभाशुभम्}
{उत्तमं योगमास्थाय यदीच्छति विमुच्यते ॥युधिष्ठिर उवाच}
{}


\threelineshloka
{आहारान्कीदृशान्कृत्वा कानि जित्वा च भारत}
{योगी बलमवाप्नोति तद्भवान्वक्तुमर्हति ॥भीष्म उवाच}
{}


\twolineshloka
{कणानां भक्षणे युक्तः पिण्याकस्य च भारत}
{स्नेहानां वर्जने युक्तो योगी बलमवाप्नुयात्}


\twolineshloka
{भुञ्जानो यावकं रूक्षं दीर्घकालमरिंदम्}
{एकाहारो विशुद्धात्मा योगी बलमवाप्नुयात्}


\twolineshloka
{पक्षान्मासानृतूंश्चैतान्संवत्सरानहस्तथा}
{अपः पीत्वा पयोमिश्रा योगी बलमवाप्नुयात्}


\twolineshloka
{अखण्डमपि वा मांसं सततं मनुजेश्वर}
{उपोष्य सम्यक्शुद्धात्मा योगी बलमवाप्नुयात्}


\twolineshloka
{कामं जित्वा तथा क्रोधं शीतोष्णे वर्षमेव च}
{भयं शोकं तथा श्वासं पौरुषान्विषयांस्तथा}


\twolineshloka
{अरतिं दुर्जयां चैव घोरां तृष्णां च पार्थिव}
{स्पर्शं निद्रां तथा तन्द्रीं दुर्जयां नृपसत्तम}


\twolineshloka
{दीपयन्ति महात्मानः सूक्ष्ममात्मानमात्मना}
{वीतरागा महाप्रज्ञा ध्यानाध्ययनसंपदा}


\twolineshloka
{दुर्गस्त्वेप मतः पन्था ब्राह्मणानां विपश्चिताम्}
{न कश्चिद्व्रजति ह्यस्मिन्क्षेमेण भरतर्षभ}


\twolineshloka
{यथा कश्चिद्वनं घोरं बहुसर्पसरीसृपम्}
{श्वभ्रवत्तोयहीनं च दुर्गमं बहुकण्टकम्}


\twolineshloka
{अभक्षमटवीप्रायं दावदग्धमहीरुहम्}
{पन्थानं तस्कराकीर्णं क्षेमेणाभिपतेद्युवा}


\twolineshloka
{योगमार्गं तथाऽऽसाद्य यः कश्चिद्व्रजते द्विजः}
{क्षेमेणोपरमेन्मार्गाद्बहुदोषो हि स स्मृतः}


\twolineshloka
{सुस्थेयं क्षुरधारासु निशितासु महीपते}
{धारणासु तु योगस्य दुःस्थेयमकृतात्मभिः}


\twolineshloka
{विपन्ना धारणास्तात नयन्ति न शुभां गतिम्}
{नेतृहीना यथा नावः पुरुषानर्णवे नृप}


\twolineshloka
{यस्तु तिष्ठति कौन्तेय धारणासु यथाविधि}
{मरणं जन्म दुःखं च सुखं च स विमुञ्चति}


\twolineshloka
{नानाशास्त्रेषु निष्पन्नं योगेष्विदमुदाहृतम्}
{परं योगस्य यत्कृत्यं निश्चितं तद्द्विजातिषु}


\twolineshloka
{परं हि तद्ब्रह्ममयं महात्मन्ब्रह्माणमीशं वरदं च विष्णुम्}
{भवं च धर्मं च ष़डाननं चषट््ब्रह्मपुत्रांश्च महान््भावान्}


\twolineshloka
{तमश्च कष्टं सुमहद्रजश्चसत्वं विशुद्धं प्रकृतिं परां च}
{सिद्धिं च देवीं वरुणस्य पत्नींतेजश्च कृत्स्नं सुमहच्च धैर्यम्}


\twolineshloka
{ताराधिपं खे विमलं सतारंविश्वांश्च देवानुरगान्पितृंश्च}
{शैलांश्च कृत्स्नानुदधींश्च घोरान्नदीश्च सर्वाः सवनान्घनांश्च}


\twolineshloka
{नागान्नगान्यक्षगणान्दिशश्चगन्धर्वसंघान्पुरुषान्स्त्रियश्च}
{परात्परं प्राप्य महान्महात्माविशेत योगी नचिराद्विमुक्तः}


\twolineshloka
{कथा च येयं नृपते प्रसक्तादेवे महावीर्यतमौ शुभेयम्}
{योगी स सर्वानभिभूय मर्त्यान्नारायणात्मा कुरुते महात्मा}


\chapter{अध्यायः ३०७}
\twolineshloka
{युधिष्ठिर उवाच}
{}


\twolineshloka
{सम्पक्त्वयाऽयं नृपते वर्णितः शिष्टसंमतः}
{योगमार्गो यथान्यायं शिष्यायेह हितैषिणा}


\threelineshloka
{साङ्ख्ये त्विदानीं कार्त्स्न्येन विधिं प्रब्रूहि पृच्छते}
{त्रिषु लोकेषु यज्ज्ञानं सर्वं तद्विदितं हि ते ॥भीष्म उवाच}
{}


\twolineshloka
{शृणु मे त्वमिदं कृत्स्नं साङ्ख्यानां विदितात्मनाम्}
{विदितं यतिभिः सर्वैः कपिलादिभिरीश्वरैः}


\twolineshloka
{यस्मिन्नविभ्रमाः केचिद्दृश्यन्ते मनुजर्षभ}
{गुणाश्च यस्मिन्बहवो दोषहानिश्च केवला}


\twolineshloka
{ज्ञानेन परिसङ्ख्याय सदोषान्विषयान्नृप}
{मानुषान्दुर्जयान्कृत्स्नान्पैशाचान्विषयांस्तथा}


\twolineshloka
{राक्षसान्विषयाञ्ज्ञात्वा यक्षाणां विषयांस्तथा}
{विषयानौरगाञ्ज्ञात्वा गान्धर्वविषयांस्तथा}


\twolineshloka
{पितृणां विषयाञ्ज्ञात्वा तिर्यक्षु चरतां नृप}
{सुपर्णविषयाञ्ज्ञात्वा मरुतां विषयांस्तथा}


\twolineshloka
{राजर्षिविषयाञ्ज्ञात्वा ब्रह्मर्षिविषयांस्तथा}
{आसुरान्विषयाञ्ज्ञात्वा वैश्वदेवांस्तथैव च}


\twolineshloka
{देवर्षिविषयाञ्ज्ञात्वा योगानामपि चेश्वरान्}
{प्रजापतीनां विषयान्ब्रह्मणो विषयांस्तथा}


\twolineshloka
{आयुषश्च परं कालं लोके विज्ञाय तत्त्वतः}
{सुखस्य च परं तत्त्वं विज्ञाय वदतां वर}


\twolineshloka
{प्राप्ते काले च यद्दुःखं सततं विषयैषिणाम्}
{तिर्यक्षु पततां दुःखं पततां नरके च यत्}


\threelineshloka
{स्वर्गस्य च गुणान्कृत्स्नान्दोषान्सर्वांश्च भारत}
{`परिसंख्यानसंख्यानं सत्वं सांख्यगुणात्मकम्}
{'वेदवादे येऽपि दोषा गुणा ये चापि वैदिकाः}


\threelineshloka
{ज्ञानयोगे च ये दोषा गुणा योगे च ये नृप}
{साङ्ख्यज्ञाने च ये दोषास्तथैव च गुणा नृप}
{`इतरेषु च ये दोषा गुणास्तेषु च भारत ॥'}


\twolineshloka
{सत्वं दशगुणं ज्ञात्वा रजो नवगुणं तथा}
{तमश्चाष्टगुणं ज्ञात्वा वृद्धिं सप्तगुणां तथा}


\twolineshloka
{षङ्गुणं च मनो ज्ञात्वा नभः पञ्चगुणं तथा}
{बुद्धिं चतुर्गुणां ज्ञात्वा तमश्च त्रिगुणं तथा}


\twolineshloka
{द्विगुणं च रजो ज्ञात्वा सत्वमेकगुणं पुनः}
{सर्गं विज्ञाय तत्त्वेन प्रलये प्रेक्ष्य चात्मनः}


\twolineshloka
{ज्ञानविज्ञानसंपन्नाः कारणैर्भाविताः शुभाः}
{प्राप्नुवन्ति शुभं मोक्षं सूक्ष्मा इव नभः परम्}


\twolineshloka
{रूपेण दृष्टिं संयुक्तां घ्राणं गन्धगुणेन च}
{शब्दे सक्तं तथा श्रोत्रं जिह्वा रसगुणेषु च}


\twolineshloka
{त्वचं स्पर्शे तथा सक्तां वायुं नभसि चाश्रितम्}
{मोहं तमसि संयुक्तं लोभमर्थेषु संश्रितम्}


\twolineshloka
{विष्णौ क्रान्तं बलं शक्रे कोष्ठे सक्तं तथाऽनलम्}
{अप्सु देवीं समासक्तामपस्तेजसि संश्रिताः}


\twolineshloka
{तेजः सूक्ष्मे च संयुक्तं वायुं नभसि चाश्रितम्}
{नभो महति संयुक्तं महद्बुद्धौ च संश्रितम्}


\twolineshloka
{बुद्धिं तमसि संसक्तां तमो रजसि संश्रितम्}
{रजः सत्वे तथा सक्तं सत्वं सक्तं तथाऽऽत्मनि}


\twolineshloka
{सक्तमात्मानमीशे च देवे नारायणे तथा}
{देवं मोक्षे च संसक्तं मोक्षं सक्तं तु न क्वचित्}


\twolineshloka
{ज्ञात्वा सत्वयुतं देहं वृतं षोडशभिर्गुणैः}
{स्वभावं चेतनां चैव ज्ञात्वा देहसमाश्रिते}


\twolineshloka
{मध्यस्थमेकमात्मानं पापं यस्मिन्न विद्यते}
{द्वितीयं कर्म विज्ञाय नृपते विषयैषिणाम्}


\twolineshloka
{इन्द्रियाणीन्द्रियार्थांश्च सर्वानात्मनि संश्रितान्}
{दुर्लभत्वं च मोक्षस्य विज्ञाय श्रुतिपूर्वकम्}


\twolineshloka
{प्राणापानौ समानं च व्यानोदानौ च तत्त्वतः}
{आवहं चानिलं ज्ञात्वा प्रवहं चानिलं पुनः}


\twolineshloka
{सप्तवातांस्तथा शेषान्सप्तधा विहितान्पुनः}
{प्रजापतीनृषींश्चैव मार्गांश्चैव बहून्वरान्}


\twolineshloka
{सप्तर्षीश्च बहूञ्ज्ञात्वा राजर्षीश्च परंतप}
{सुरर्षीन्महतश्चान्यान्ब्रह्मर्षीन्सूर्यसन्निभान्}


\twolineshloka
{ऐश्वर्याच्च्यावितान्दृष्ट्वा कालेन महता नृप}
{महतां भूतसङ्घानां श्रुत्वा नाशं च पार्थिव}


\twolineshloka
{गतिं चाप्यशुभां ज्ञात्वा नृपते पापकर्मिणाम्}
{वैतरण्यां च यद्दुःखं पतितानां यमक्षये}


\twolineshloka
{योनीषु च विचित्रासु संसारानशुभांस्तथा}
{जठरे चाशुभे वासं शोणितोदकभाजने}


\twolineshloka
{श्लेष्ममूत्रपुरीषे च तीव्रगन्धसमन्विते}
{शुक्रशोणितसंघाते मज्जास्नायुपरिग्रहे}


\twolineshloka
{सिराशतसमाकीर्णे नवद्वारे पुरेऽशुचौ}
{विज्ञायाहितमात्मानं योगांश्च विविधान्नृप}


\twolineshloka
{तामसानां च जन्तूनां रमणीयावृतात्मनाम्}
{सात्विकानां च जन्तूनां कुत्सितं भरतर्षभ}


\twolineshloka
{गर्हितं महतामर्थे साङ्ख्यानां विदितात्मनाम्}
{उपप्लवांस्तथा घोराञ्शशिनस्तेजसस्तथा}


\twolineshloka
{ताराणां पतनं दृष्ट्वा नक्षत्राणां च पर्ययम्}
{द्वन्द्वानां विप्रयोगं च विज्ञाय कृपण नृप}


\twolineshloka
{अन्योन्यभक्षणं दृष्ट्वा भूतानामपि चाशुभम्}
{बाल्ये मोहं च विज्ञाय क्षयं देहस्य चाशुभम्}


\twolineshloka
{रागे मोहे च संप्राप्ते क्वचित्सत्वं समाश्रितम्}
{सहस्रेषु नरः कश्चिन्मोक्षबुद्धिं समाश्रितः}


\twolineshloka
{दुर्लभत्वं च मोक्षस्य विज्ञाय श्रुतिपूर्वकम्}
{बहुमानमलब्धेषु लब्धे मध्यस्थतां पुनः}


\twolineshloka
{विषयाणां च दौरात्म्यं विज्ञाय नृपते पुनः}
{गतासूनां च कौन्तेय देहान्दृष्ट्वा तथाऽशुभान्}


\twolineshloka
{वासं कुलेषु जन्तूनां दुःखं विज्ञाय भारत}
{ब्रह्मघ्नानां गतिं ज्ञात्वा पतितानां सुदारुणाम्}


\twolineshloka
{सुरापाने च सक्तानां ब्राह्मणानां दुरात्मनाम्}
{गुरूदारप्रसक्तानां गतिं विज्ञाय चाशुभाम्}


\twolineshloka
{जननीषु च वर्तन्ते येन सम्यग्युधिष्ठिर}
{सदेवकेषु लोकेषु येन वर्तन्ति मानवाः}


% Check verse!
तेन ज्ञानेन विज्ञाय गतिं चाशुभकर्मणाम्तिर्यग्योनिगतानां च विज्ञाय च गतिं पृथक्
\twolineshloka
{वेदवादांस्तथा चित्रानृतूनां पर्ययांस्तथा}
{क्षयं संवत्सराणां च मासानां च क्षयं तथा}


\twolineshloka
{पक्षक्षयं तथा दृष्ट्वा दिवसानां च संक्षयम्}
{क्षयं वृद्धिं च चन्द्रस्य दृष्ट्वा प्रत्यक्षतस्तथा}


\twolineshloka
{वृद्धिं दृष्ट्वा समुद्राणां क्षयं तेषां तथा पुनः}
{क्षयं धनानां दृष्ट्वा च पुनर्वृद्धिं तथैव च}


\twolineshloka
{संयोगानां क्षयं दृष्ट्वा युगानां च विशेषतः}
{क्षयं च दृष्ट्वा शैलानां क्षयं च सरितां तथा}


\twolineshloka
{वर्णानां च क्षयं दृष्ट्वा क्षयान्तं च पुनः पुनः}
{जरा मृत्युस्तथा जन्म दृष्ट्वा दुःखानि चैव ह}


\twolineshloka
{देहदोषांस्तथा ज्ञात्वा तेषां दुःखं च तत्त्वतः}
{देहविक्लवतां चैव सम्यग्विज्ञाय तत्त्वतः}


\fourlineindentedshloka
{आत्मदोषांश्च विज्ञाय सर्वानात्मनि संश्रितान्}
{स्वदेहादुत्थितान्गब्धांस्तथा विज्ञाय चाशुभान्}
{`मूत्रश्लेष्मपुरीषादीन्स्वेदजांश्च सुकुत्सितान् ॥' युधिष्ठिर उवाच}
{}


\threelineshloka
{कान्स्वगात्रोद्भवान्दोषान्पश्यस्यमितविक्रम}
{एतन्मे संशयं कृत्स्नं वक्तुमर्हसि तत्त्वतः ॥भीष्म उवाच}
{}


\twolineshloka
{पञ्च दोषान्प्रभो देहे प्रवदन्ति मनीषिणः}
{मार्गज्ञाः कापिलाः साङ्ख्याः शृणु तानरिसूदन}


% Check verse!
कामक्रोधौ भयं निद्रा पञ्चमः श्वास उच्यते
\twolineshloka
{एते दोषाः शरीरेषु दृश्यन्ते सर्वदेहिनाम्}
{छिन्दन्ति क्षमया क्रोधं कामं संकल्पवर्जनात्}


\twolineshloka
{सत्वसंसेवनान्निद्रामप्रमादाद्भयं तथा}
{छिन्दन्ति पञ्चमं श्वासमल्पाहारतया नृप}


\twolineshloka
{गुणान्गुणशतैर्ज्ञात्वा दोषान्दोषशतैरपि}
{हेतून्हेतुशतैश्चित्रैश्चित्रान्विज्ञाय तत्त्वतः}


\twolineshloka
{अपां फेनोपमं लोकं विष्णोर्मायाशतैश्वितम्}
{चित्रभित्तिप्रतीकाशं नलसारमनर्थकम्}


\twolineshloka
{तमः श्वभ्रनिभं दृष्ट्वा वर्षबुद्बुदसंनिभम्}
{क्लेशप्रायं सुखाद्धीनं नाशोत्तरमिहावशम्}


\twolineshloka
{रजस्तमसि संमग्नं पङ्के द्वीपमिवावशम्}
{साङ्ख्या राजन्महाप्राज्ञास्त्यक्त्वा देहं प्रजाकृतं}


\twolineshloka
{ज्ञानयोगेन साङ्ख्येन व्यापिना महता नृप}
{राजसानशुभान्गन्धांस्तांमसांश्च तथाविधान्}


\twolineshloka
{पुण्यांश्च सात्विकान्गन्धान्स्पर्शजान्देहसंश्रितान्}
{छित्त्वाऽऽशु ज्ञानशस्त्रेण तपो दण्डेन भारत}


\twolineshloka
{ततो दुःखोदधिं घोरं चिन्ताशोकमहाह्रदम्}
{व्याधिमृत्युमहाग्राहं महाभयमहोरगम्}


\twolineshloka
{तमःकूर्मं रजोमीनं प्रज्ञया संतरन्त्युत}
{स्नेहपङ्कं जरादुर्गं ज्ञानदीपमरिंदम्}


\twolineshloka
{कर्मागाधं सत्यतीरं स्थितव्रतमरिंदम्}
{हिंसाशीघ्रमहावेगं नानारससमाकरम्}


\twolineshloka
{नानाप्रीतिमहारत्नं दुःखज्वरसमीरणम्}
{शोकतृष्णामहावर्तं तीक्ष्णव्याधिमहागजम्}


\twolineshloka
{अस्थिसंघातसंघट्टं श्लेष्मफेनमरिंदम्}
{दानमुक्ताकरं घोरं शोणितह्रदविद्रुमम्}


\threelineshloka
{हसितोत्क्रुष्टनिर्घोषं नानाज्ञानसुदुस्तरम्}
{रोदनाश्रुमलक्षारं सङ्गत्यागपरायणम्}
{}


\twolineshloka
{पुत्रदारजलौकौघं मत्रिबान्धवपत्तनम्}
{अहिंसासत्यमर्यादं प्राणत्यागमहोर्मिणम्}


\twolineshloka
{वेदान्तगमनद्वीपं सर्वभूतदयोदकम्}
{मोक्षदुर्लाभविषयं व़डवामुखसागरम्}


\twolineshloka
{तरन्ति मुनयः सिद्धा ज्ञानयानेन भारत}
{तीर्त्वाऽतिदुस्तरं जन्म विशन्ति विमलं नभः}


\twolineshloka
{तत्र तान्सुकृतीन्साङ्ख्यान्सूर्यो वहति रश्मिभिः}
{पद्मतन्तुवदाविश्य प्रसह्य विषयान्नृप}


\twolineshloka
{तत्र तान्प्रवहो वायुः प्रतिगृह्णाति भारत}
{वीतरागान्यतीन्सिद्धान्वीर्ययुक्तांस्तपोधनान्}


\threelineshloka
{सूक्ष्मः शीतः सुगन्धी च सुखस्पर्शश्च भारत}
{सप्तानां मरुतां श्रेष्ठो लोकान्गच्छति यः शुभान्}
{स तान्वहति कौन्तेय नभसः परमां गतिम्}


\threelineshloka
{नभो वहति लोकेश रजसः परमां गतिम्}
{`तमो वहति लोकेश रजसः परमां गतिम्}
{'रजो वहति राजेन्द्र सत्वस्य परमां गतिम्}


\twolineshloka
{सत्वं वहति राजेन्द्र परं नारायणं प्रभुम्}
{प्रभुर्वहति शुद्धात्मा परमात्मानमात्मना}


\twolineshloka
{परमात्मानमासाद्य तद्भूता यतयोऽमलाः}
{अमृतत्वाय कल्पन्ते न निवर्तन्ति वा विभो}


\threelineshloka
{परमा सा गतिः पार्थ निर्द्वन्द्वानां महात्मनाम्}
{सत्यार्जवरतानां वै सर्वभूतदयावताम् ॥युधिष्ठिर उवाच}
{}


\twolineshloka
{स्थानमुत्तममासाद्य भगवन्तं स्थिरव्रताः}
{आजन्ममरणं वा ते स्मरन्त्युत न वाऽनघ}


\twolineshloka
{यदत्र तथ्यं तन्मे त्वं यथावद्वक्तुमर्हसि}
{त्वदृते पुरुषं नान्यं प्रष्टुमर्हामि कौरव}


\twolineshloka
{मोक्षे दोषो महानेष प्राप्य सिद्धिगतानृषीन्}
{यदि तत्रैव विज्ञाने वर्तन्ते यतयः परे}


\fourlineindentedshloka
{प्रवृत्तिलक्षणं धर्मं पश्यामि परमं नृप}
{मग्नस्य हि परे ज्ञाने किं न दुःखतरं भवेत्}
{भीष्म उवाच}
{}


\twolineshloka
{यथान्यायं त्वया तात प्रश्नः पृष्टः सुसंकटः}
{बुधानामपि संमोहः प्रश्नेऽस्मिन्भरतर्षभ}


\twolineshloka
{अत्रापि तत्त्वं परमं शृणु सम्यङ्भयेरितम्}
{बुद्धिश्च परमा यत्र कापिलानां महात्मनाम्}


\twolineshloka
{इन्द्रियाण्येव बुध्यन्ते स्वदेहे देहिनां नृप}
{कारणान्यात्मनस्तानि सूक्ष्मः पश्यति तैस्तु सः}


\twolineshloka
{आत्मना विप्रहीणानि काष्ठकुड्यसमानि तु}
{विनश्यन्ति न संदेहः फेना इव महार्णवे}


\twolineshloka
{इन्द्रियैः सह सुप्तस्य देहिनः शत्रुतापन}
{सूक्ष्मश्चरति सर्वत्र नभसीव समीरणः}


\twolineshloka
{स पश्यति यथान्यायं स्पर्शान्स्पृशति वा विभो}
{बुध्यमानो यथापूर्वमखिलेनेह भारत}


\twolineshloka
{इन्द्रियाणीह सर्वाणि स्वे स्वे स्थाने यथाविधि}
{अनीशत्वात्प्रलीयन्ते सर्पा हतविषा इव}


\twolineshloka
{इन्द्रियाणां तु सर्वेषां स्वस्थानेष्वेव सर्वशः}
{आक्रम्य गतयः सूक्ष्माश्चरत्यात्मा न संशयः}


\twolineshloka
{सत्वस्य च गुणान्कृत्स्नान्नजसश्च गुणान्पुनः}
{गुणांश्च तमसः सर्वान्गुणान्बुद्धेश्च भारत}


\twolineshloka
{गुणांश्च मनसश्चापि नभसश्च गुणांश्च सः}
{गुणान्वायोश्च धर्मात्मंस्तेजसां च गुणान्पुनः}


\twolineshloka
{अपां गुणांस्तथा पार्थ पार्थिंवांश्च गुणानपि}
{सर्वात्मना गुणैर्व्याप्तः क्षेत्रज्ञेषु युधिष्ठिर}


\twolineshloka
{आत्मा च याति क्षेत्रज्ञं कर्मणी च शुभाशुभे}
{शिष्या इव महात्मानमिन्द्रियाणि च तं प्रभो}


\twolineshloka
{प्रकृतिं चाप्यतिक्रम्य गच्छत्यात्मानमव्ययम्}
{परं नारायणं देवं निर्द्वन्द्वं प्रकृतेः परम्}


\twolineshloka
{विमुक्तः सर्वपापेभ्यः प्रविष्टस्तमनामयम्}
{परमात्मानमगुणं न निवर्तति भारत}


\twolineshloka
{शिष्टं तत्र मनस्तात इन्द्रियाणि च भारत}
{आगच्छन्ति यथाकालं गुरोः संदेशकारिणः}


\twolineshloka
{शक्यं चाल्पेन कालेन शान्तिं प्राप्तुं गुणार्थिना}
{एवमुक्तेन कौन्तेय युक्तज्ञानेन मोक्षिणा}


\twolineshloka
{साङ्ख्या राजन्महाप्राज्ञा गच्छन्ति परमां गतिम्}
{ज्ञानेनानेन कौन्तेय तुल्यं ज्ञानं न विद्यते}


\twolineshloka
{अत्र ते संशयो मा भूज्ज्ञानं सांख्यं परं मतम्}
{अक्षरं ध्रुवमव्यक्तं पूर्णं ब्रह्म सनातनम्}


\twolineshloka
{अनादिमध्यनिधनं निर्द्वन्द्वं कर्तृ शाश्वतम्}
{कूटस्थं चैव नित्यं च यद्वदन्ति शमात्मकाः}


\twolineshloka
{यतः सर्वाः प्रवर्तन्ते सर्गप्रलयविक्रियाः}
{यच्च शंसन्ति शास्त्रेषु वदन्ति परमर्षयः}


\twolineshloka
{सर्वे विप्राश्च देवाश्च तथा शमविदो जनाः}
{ब्रह्मण्यं परमं देवमनन्तं परमच्युतम्}


\twolineshloka
{प्रार्थयन्तश्च तं विप्रा वदन्ति गुणबुद्धयः}
{सम्यग्युक्तास्तथा योगाः साङ्ख्याश्चामितदर्शनाः}


\twolineshloka
{अमूर्तेस्तस्य कौन्तेय साङ्ख्यं मूर्तिरिति श्रुतिः}
{अभिज्ञानानि तस्याहुर्मतं हि भरतर्षभ}


\twolineshloka
{द्विविधानीह भूतानि पृथिव्यां पृथिवीपते}
{जङ्गमाजङ्गमाख्यानि जङ्गमं तु विशिष्यते}


\twolineshloka
{ज्ञानं महद्यद्धि महत्सु राजन्वेदेषु साङ्ख्येषु तथैव योगे}
{यच्चापि दृष्टं विविधं पुराणेसाङ्ख्यागतं तन्निखिलं नरेन्द्र}


\twolineshloka
{यच्चेतिहासेषु महत्सु दृष्टंयच्चार्थशास्त्रे नृप शिष्टजुष्टे}
{ज्ञानं च लोके यदिहास्ति किंचित्साङ्ख्यागतं तच्च महन्महात्मन्}


\twolineshloka
{शमश्च दृष्टः परमं बलं चज्ञानं च साङ्ख्यं च यथावदुक्तम्}
{तपांसि सूक्ष्माणि सुखानि चैवसाङ्ख्ये यथावद्विहितानि राजन्}


\twolineshloka
{विपर्यये तस्य हि पार्थ देवान्गच्छन्ति साङ्ख्याः सततं सुखेन}
{तांश्चानुसंचार्य ततः कृतार्थाःपतन्ति विप्रेषु यतेषु भूयः}


\twolineshloka
{हित्वा च देहं प्रविशन्ति मोक्षंदिवौकसो द्यामिव पार्थ साङ्ख्याः}
{अतोऽधिकं तेऽभिरता महार्थेसाङ्ख्ये द्विजाः पार्थिव शिष्टजुष्टे}


\twolineshloka
{तेषां न तिर्यग्गमनं हि दृष्टंनार्वाग्गतिः पापकृताधिवासः}
{च चाबुधानामपि ते द्विजातयोये ज्ञानमेतन्नृपतेऽनुरक्ताः}


% Check verse!
साङ्ख्यं विशालं परमं पुराणंमहार्णवं विमलमुदाहरन्तिकृत्स्नं च साङ्ख्यं नृपते महात्मानारायणो धारयतेऽप्रमेयम्
\twolineshloka
{एतन्मयोक्तं नरदेव तत्त्वंनारायणो विश्वमिदं पुराणम्}
{स सर्गकाले च करोति सर्गंसंहारकाले च तदत्ति भूयः}


% Check verse!
संहृत्य सर्वं निजदेहसंस्थंकृत्वाऽप्सु शेते जगदन्तरात्मा
\chapter{अध्यायः ३०८}
\twolineshloka
{युधिष्ठिर उवाच}
{}


\twolineshloka
{किं तदक्षरमित्युक्तं यस्मान्नावर्तते पुनः}
{किंच तत्क्षरमित्युक्तं यस्मादावर्तते पुनः}


\twolineshloka
{अक्षरधरयोर्व्यक्तिं पृच्छाम्यरिनिषूदन}
{उपलब्धुं महाबाहो तत्त्वेन कुरुनन्दन}


\twolineshloka
{त्वं हि ज्ञाननिधिर्विप्रैरुच्यसे वेदपारगैः}
{ऋषिभिश्च महाभागैर्यतिभिश्च महात्मभिः}


\twolineshloka
{शेषमत्यं दिनानां ते दक्षिणायनभास्करे}
{आवृत्ते भगवत्यर्के गन्तासि परमां गतिम्}


\twolineshloka
{त्वयि प्रतिगते श्रेयः कुतः श्रोष्यामहे वयम्}
{कुरुवंशप्रदीपस्त्वं ज्ञानदीपेन दीप्यसे}


\threelineshloka
{तदेतच्छ्रोतुमिच्छामि त्वत्तः कुरुकुलोद्वह}
{न तृष्यामीह राजेन्द्र शृण्वन्नमृतमीदृशम् ॥भीष्म उवाच}
{}


\twolineshloka
{अत्र ते वर्तयिष्यामि इतिहासं पुरातनम्}
{वसिष्ठस्य च संवादं करालजनकस्य च}


\twolineshloka
{वसिष्ठं श्रेष्ठमासीनमृषीणां भास्करद्युतिम्}
{पप्रच्छ जनको राजा ज्ञानं नैःश्रेयसं परम्}


\twolineshloka
{परमध्यात्मकुशलमध्यात्मगतिनिश्चयम्}
{मैत्रावरुणिमासीनमभिवाद्य कृताञ्जलिः}


\twolineshloka
{स्वक्षरं प्रश्रितं वाक्यं मधुरं चाप्यनुल्वणम्}
{पप्रच्छर्षिवरं राजा करालजनकः पुरा}


\twolineshloka
{भगवञ्श्रोतुमिच्छामि परं ब्रह्म सनातनम्}
{यस्मान्न पुनरावृत्तिमाप्नुवन्ति मनीषिणः}


\threelineshloka
{यच्च तत्क्षरमित्युक्तं यत्रेदं क्षरते जगत्}
{यच्चाक्षरमिति प्रोक्तं शिवं क्षेम्यमनामयम् ॥वसिष्ठ उवाच}
{}


\twolineshloka
{श्रूयतां पृथिवीपाल क्षरतीदं यथा जगत्}
{यन्न क्षरति पूर्वेण यावत्कालेन चाप्यथ}


\twolineshloka
{युगं द्वादशसाहस्रं कल्पं विद्धि चतुर्युगम्}
{दशकल्पशतावृत्तमहस्तद्ब्राह्ममुच्यते}


\twolineshloka
{रात्रिश्चैतावती राजन्यस्यान्ते प्रतिबुध्यते}
{सृजत्यनन्तकर्माणं महान्तं भूतमग्रजम्}


\twolineshloka
{मूर्तिमन्तममूर्तात्मा विश्वं शंभुः स्वयंभुवम्}
{अणिमा लघिमा प्राप्तिरीशानं ज्योतिरव्ययम्}


\twolineshloka
{सर्वतः पाणिपादं तत्सर्वतोक्षिशिरोमुखम्}
{सर्वतः श्रुतिमल्लोके सर्वमावृत्य तिष्ठति}


\twolineshloka
{हिरण्यगर्भो भगवानेष बुद्धिरिति स्मृतः}
{महानिति च योगेषु विरिञ्चिरिति चाप्यजः}


\twolineshloka
{साङ्ख्ये च पठ्यते शास्त्रे नामभिर्बहुधात्मकः}
{विचित्ररूपो विश्वात्मा एकाक्षर इति स्मृतः}


\twolineshloka
{वृतं नैकात्मकं येन कृतं त्रैलोक्यमात्मना}
{तथैव बहुरूपत्वाद्विश्वरूप इति स्मृतः}


\twolineshloka
{एष वै विक्रियापन्नः सृजत्यात्मानमात्मना}
{अहंकारं महातेजाः प्रजापतिरंकृतम्}


\twolineshloka
{अव्यक्ताद्व्यक्तमापन्नं विद्यासर्गं वदन्ति तम्}
{महान्तं चाप्यहंकारमविद्यासर्गमेव च}


\twolineshloka
{अपरश्च परश्चैव समुत्पन्नौ तथैकतः}
{विद्याविद्येति विख्याते श्रुतिशास्त्रार्थचिन्तकैः}


\twolineshloka
{भूतसर्गमहंकारात्तृतीयं विद्धि पार्थिव}
{अहंकारेषु सर्वेषु चतुर्थं विद्धि वैकृतम्}


\twolineshloka
{वायुर्ज्योतिरथाकाशमापोऽथ पृथिवी तथा}
{शब्दः स्पर्शश्च रूपं च रसो गन्धस्तथैव च}


\twolineshloka
{एवं युगपदुत्पन्नं दशवर्गमसंशयम्}
{पञ्चमं विद्धि राजेन्द्र भौतिकं सर्गमर्थवत्}


\twolineshloka
{श्रोत्रं त्वक्चक्षुषी जिह्वा घ्राणमेव च पञ्चमम्}
{वाक्च हस्तौ च पादौ च पायुर्मेढ्रं तथैव च}


\twolineshloka
{बुद्धीन्द्रियाणि चैतानि तथा कर्मेन्द्रियाणि च}
{संभूतानीह युगपन्मनसा सह पार्थिव}


\twolineshloka
{एषा तत्त्वचतुर्विशत्सर्वाकृतिषु वर्तते}
{यां ज्ञात्वा नाभिशोचन्ति ब्राह्मणास्तत्त्वदर्शिनः}


\twolineshloka
{एतद्देहं समाख्यानं त्रैलोक्ये सर्वदेहिषु}
{वेदितव्यं नरश्रेष्ठ सदेवनरदानवे}


\twolineshloka
{सयक्षभूतगन्धर्वे सकिन्नरमहोरगे}
{सचारणपिशाचे वै सदेवर्षिनिशाचरे}


\twolineshloka
{सदंशकीटमशक्रे सपूतिकृमिमूषिके}
{शुनि श्वपाके चैणेये सचाण्डाले सपुल्कसे}


\twolineshloka
{हस्त्यश्वखरशार्दूले सवृके गवि चैव ह}
{यच्च मूर्तिमयं किचित्सर्वत्रैतन्निदर्शनम्}


\twolineshloka
{जले भुवि तथाऽऽकाशे नान्यत्रेति विनिश्चयः}
{स्थानं देहवतामस्ति इत्येवमनुशुश्रुम}


\twolineshloka
{कृत्स्नमेतावतस्तात क्षरते व्यक्तसंज्ञितम्}
{अहन्यहनि भूतात्मा ततः क्षर इति स्मृतः}


\twolineshloka
{एतद्धि क्षरमित्युक्तं क्षरतीदं यथा जगत्}
{जगन्मोहात्मकं प्राहुरव्यक्तं व्यक्तसंज्ञकम्}


\twolineshloka
{महांश्चैवाग्रजो नित्यमेतत्क्षरनिदर्शनम्}
{कथितं ते महाराजन्यस्मान्नावर्तते पुनः}


\twolineshloka
{पञ्चर्विशतिमो विष्णुर्निस्तत्त्वस्तत्त्वसंज्ञितः}
{तत्त्वसंश्रयणादेतत्तत्वमाहुर्मनीषिणः}


\twolineshloka
{यन्मर्त्यमसृजद्व्यक्तं तत्तन्मूर्त्यधितिष्ठति}
{चतुर्विशतिमोऽव्यक्तो ह्यमूर्तः पञ्चविंशकः}


\twolineshloka
{स एव हृदि सर्वासु मूर्तिष्वात्मावतिष्ठते}
{चेतयंश्चेतनान्नित्यं सर्वमूर्तिरमूर्तिमान्}


\twolineshloka
{सर्वप्रत्ययधर्मिण्यां स सर्गः प्रत्ययात्मकः}
{गोचरे वर्तते नित्यं निर्गुणो गुणसंज्ञिते}


\twolineshloka
{एवमेष महानात्मा सर्गप्रलयकोविदः}
{विकुर्वाणः प्रकृतिमानभिमन्यत्यबुद्धिमान्}


\twolineshloka
{तमः सत्वरजोयुक्तस्तासु तास्विह योनिषु}
{लीयते प्रतिबुद्धत्वादबुद्धजनसेवनात्}


\twolineshloka
{सहवासनिवासात्मा नान्योऽहमिति मन्यते}
{योऽहं सोहमिति ह्युक्त्वा गुणानेवानुवर्तते}


\twolineshloka
{तमसा तामसान्भावान्विविधान्प्रतिपद्यते}
{रजसा राजसांश्चैव सात्विकान्सत्वसंश्रयात्}


\twolineshloka
{शुक्ललोहितकृष्णानि रूपाण्येतानि त्रीणि तु}
{सर्वाण्येतानि रूपाणि यानीह प्राकृतानि वै}


\twolineshloka
{तामसा निरयं यान्ति राजसा मानुपानथ}
{सात्विका देवलोकाय गच्छन्ति सुखभागिनः}


\twolineshloka
{निष्कैवल्येन पापेन तिर्यग्योनिमवाप्नुयात्}
{पुण्यपापेन मानुष्यं पुण्येनैकेन देवताः}


\twolineshloka
{एवमव्यक्तविषयं क्षरमाहुर्मनीषिणः}
{पञ्चविंशतिमो योऽयं ज्ञानादेव प्रवर्तते}


\chapter{अध्यायः ३०९}
\twolineshloka
{वसिष्ठ उवाच}
{}


\twolineshloka
{एवमप्रतिबुद्धत्वादबुद्धमनुवर्तनात्}
{देहाद्देहसहस्राणि तथा समभिपद्यते}


\twolineshloka
{तिर्यग्योनिसहस्रेषु कदाचिद्देवतास्वपि}
{उपपद्यति संयोगाद्गुणैः सह गुणक्षयात्}


\twolineshloka
{मानुषत्वाद्दिवं याति दिवो मानुष्यमेति च}
{मानुष्यान्निरयस्थानमनन्तं प्रतिपद्यते}


\twolineshloka
{कोशकारो यथाऽऽत्मानं कीटः समनुरुन्धति}
{सूत्रतन्तुगुणैर्नित्यं तथाऽयमगुणो गुणैः}


\twolineshloka
{द्वन्द्वमेति च निर्द्वन्द्वस्तासु तास्विह योनिषु}
{शीर्षरोगेऽक्षिरोगे च दन्तशूले गलग्रहे}


\twolineshloka
{जलोदरे तृषारोगे ज्वरगण्डे विषूचके}
{श्वित्रकुष्ठेऽग्निदग्धे च सिध्मापस्मारयोरपि}


\twolineshloka
{यानि चान्यानि द्वन्द्वानि प्राकृतानि शरीरिषु}
{उत्पद्यन्ते विचित्राणि तान्येषोऽप्यभिमन्यते}


% Check verse!
अभिमन्यत्यभीमानात्तथैव सुकृतान्यपि
\twolineshloka
{शुक्लवासाश्च दुर्वासाः शायी नित्यमधस्तथा}
{मण्डूकशायी च तथा वीरासनगतस्तथा}


\twolineshloka
{चीरधारणमाकाशे शयनं स्थानमेव च}
{इष्टकाप्रस्तरे चैव कण्टकप्रस्तरे तथा}


\twolineshloka
{भस्मप्रस्तरशायी च भूमिशय्याऽनुलेपनः}
{वीरस्थानाम्बुपङ्के च शयनं फलकेषु च}


\twolineshloka
{विविधासु च शय्यासु फलगृद्ध्यान्वितस्तथा}
{मुञ्जमेखलनग्नत्वं क्षौमकृष्णाजिनानि च}


\twolineshloka
{शणवालपरीधानो व्याघ्रचर्मपरिच्छदः}
{सिंहतचर्मपरीधानः पट्टवासास्तथैव च}


\twolineshloka
{फलकपरिधानश्च तथा कण्टकवस्रधृत्}
{कीटकार्पासवसनश्चीरवासास्तथैव च}


\twolineshloka
{वस्राणि चान्यानि बहून्यभिमन्यत्यबुद्धिमान्}
{भोजनानि विचित्राणि रत्नानि विविधानि च}


\twolineshloka
{एकवस्रान्तराशित्वमेककालिकभोजनम्}
{चतुर्थाष्टमकालश्च षष्ठकालिक एव च}


\twolineshloka
{ष़ड्रात्रभोजनश्चैव तथैवाष्टाहभोजनः}
{सप्तरात्रदशाहारो द्वादशाहिकभोजनः}


\twolineshloka
{मासोपवासी मूलाशी फलाहारस्तथैव च}
{वायुभक्षोऽम्बुपिण्याकदधिगोमयभोजनः}


\twolineshloka
{गोमूत्रभोजनश्चैव शाकपुष्पाद एव च}
{शेवालभोजनश्चैव तथाऽऽचामेन वर्तयन्}


\twolineshloka
{वर्तयञ्शीर्णपर्मैश्च प्रकीर्णफलभोजनः}
{विविधानि च कृच्छ्राणि सेवते सिद्धिकाङ्क्षया}


\twolineshloka
{चान्द्रायणानि विधिवल्लिङ्गानि विविधानि च}
{चातुराश्रम्यपन्थानमाश्रयत्यपथानपि}


\twolineshloka
{उपाश्रमानप्यपरान्पाषण्डान्विविधानपि}
{विविक्ताश्च शिलाच्छायास्तथा प्रस्रवणानि च}


\twolineshloka
{पुलिनानि विविक्तानि विविक्तानि वनानि च}
{देवस्थानानि पुण्यानि विविक्तानि सरांसि च}


\twolineshloka
{विविक्ताश्चापि शैलानां गुहा गृहनिभोपमाः}
{विविक्तानि च जप्यानि व्रतानि विविधानि च}


\twolineshloka
{नियमान्विविधांश्चापि विविधानि तपांसि च}
{यज्ञांश्च विविधाकारान्विधींश्च विविधांस्तथा}


\twolineshloka
{वणिक्पथं द्विजक्षत्रं वैश्यं शूद्रांस्तथैव च}
{दानं च विविधाकारं दीनान्धकृपणादिषु}


\twolineshloka
{अभिमन्यत्यसंबोधात्तथैव त्रिविधान्गुणान्}
{सत्वं रजस्तमश्चैव धर्मार्थौ काम एव च}


\twolineshloka
{प्रकृत्याऽऽत्मानमेवात्मा एवं प्रवि भजत्युत}
{स्वधाकारवषट््कारौ स्वाहाकारनमस्क्रियाः}


\twolineshloka
{याजनाध्यापनं दानं तथैवाहुः प्रतिग्रहम्}
{यजनाध्ययने चैव यच्चान्यदपि किंचन}


\twolineshloka
{जन्ममृत्युविवादे च तथा विशसनेऽपि च}
{शुभाशुभमयं सर्वमेतदाहुः क्रियाफलम्}


\twolineshloka
{प्रकृतिः कुरुते देवी भवं प्रलयमेव च}
{दिवसान्ते गुणानेतानभ्येत्यैकोऽवतिष्ठते}


\twolineshloka
{रश्मिजालमिवादित्यस्तत्तत्काले नियच्छति}
{एवमेषोऽसकृत्सर्वं क्रीडार्थमभिमन्यते}


\twolineshloka
{आत्मरूपगुणानेतान्विविधान्हृदयप्रियान्}
{एवमेव विकुर्वाणः सर्गप्रलयधर्मिणी}


\twolineshloka
{क्रियां क्रियापथे रक्तस्त्रिगुणांस्त्रिगुणाधिपः}
{क्रियां क्रियापथोपेतस्तथा तदभिमन्यते}


\twolineshloka
{प्रकृत्या सर्वमेवेदं जगदन्धीकृतं विभो}
{रजसा तमसा चैव व्याप्तं सर्वमनेकधा}


\twolineshloka
{एवं द्वन्द्वान्यथैतानि वर्तन्ते मयि नित्यशः}
{ममैवैतानि जायन्ते धावन्ते तानि मामिति}


\twolineshloka
{निस्तर्तव्यान्यथैतानि सर्वाणीति नराधिप}
{मन्यतेऽयं ह्यबुद्धित्वात्तथैव सुकृतान्यपि}


\twolineshloka
{भोक्तव्यानि मयैतानि देवलोकगतेन वै}
{इहैव चैनं भोक्ष्यामि शुभाशुभफलोदयम्}


\twolineshloka
{पुण्यमेव तु कर्तव्यं तत्कुत्वा सुसुखं मम}
{यावदन्तं च मे सौख्यं जात्यां जात्यां भविष्यति}


\twolineshloka
{भविष्यति च मे दुःखं कृतेनेहाप्यनन्तकम्}
{महद्दुःखं हि मानुष्यं निरये चापि मज्जनम्}


\twolineshloka
{निरयाच्चापि मानुष्यं कालेनैष्याम्यहं पुनः}
{मनुष्यत्वाच्च देवत्वं देवत्वात्पौरुषं पुनः}


\twolineshloka
{मनुष्यत्वाच्च निरयं पर्यायेणोपगच्छति}
{य एवं वेत्ति नित्यं वै निरात्मात्मगुणैर्वृतः}


\twolineshloka
{तेन देवमनुष्येषु निरये चोपपद्यते}
{ममत्वेनावृतो नित्यं तत्रैव परिवर्तते}


\twolineshloka
{सर्गकोटिसहस्राणि मरणान्तासु योनिषु}
{य एवं कुरुते कर्म शुभाशुभफलात्मकम्}


\threelineshloka
{स एवं फलमाप्नोति त्रिषु लोकेषु मूर्तिमान्}
{प्रकृतिः कुरुते कर्म शुभाशुभफलात्मकम्}
{प्रकृतिश्च तदश्नाति त्रिषु लोकेषु कामगा}


\twolineshloka
{तिर्यग्योनिमनुष्यत्वे देवलोके तथैव च}
{त्रीणि स्थानानि चैतानि जानीयात्प्राकृतानिह}


\twolineshloka
{अलिङ्गां प्रकृतिं त्वाहुर्लिङ्गैरनुमिमीमहे}
{तथैव पौरुषं लिङ्गमनुमानाद्धि गम्यते}


\twolineshloka
{स लिङ्गान्तरमासाद्य प्राकृतं लिङ्गमव्रणम्}
{व्रणद्वाराण्यधिष्ठाय कर्मणाऽऽत्मनि पश्यति}


\twolineshloka
{श्रोत्रादीनि तु सर्वाणि पञ्च कर्मेन्द्रियाण्यथ}
{वागादीनि प्रवर्तन्ते गुणेष्विह गुणैः सह}


\twolineshloka
{अहमेतानि वै सर्वं मय्येतानीन्द्रियाणि ह}
{निरिन्द्रियो हि मन्तेत व्रणवानस्मि निर्व्रणः}


\twolineshloka
{अलिङ्गो लिङ्गमात्मानमकालः कालमात्मनः}
{असत्वं सत्वमात्मानमतत्त्वं तत्त्वमात्मनः}


\twolineshloka
{अमृत्युर्मृत्युमात्मानमचरश्चरमात्मनः}
{अक्षेत्रः क्षेत्रमात्मानमसर्गः सर्गमात्मनः}


\twolineshloka
{अतपास्तप आत्मानमगतिर्गतिमात्मनः}
{अभवो भवमात्मानमभयो भयमात्मनः}


\twolineshloka
{`अकर्ता कर्तृ चात्मानमबीजो बीजमात्मनः}
{'अक्षरः क्षरमात्मानमबुद्धिस्त्वभिमन्यते}


\chapter{अध्यायः ३१०}
\twolineshloka
{वसिष्ठ उवाच}
{}


\twolineshloka
{एवमप्रतिबुद्धत्वादबुद्धजनसेवनात्}
{सर्गकोटिसहस्राणि मरणान्तानि गच्छति}


\twolineshloka
{धाम्ना धामसहस्राणि पतनान्तानि गच्छति}
{तिर्यग्योनौ मनुष्यत्वे देवलोके तथैव च}


\twolineshloka
{चन्द्रमा इव भूतानां पुनस्तत्र सहस्रशः}
{लीयतेऽप्रतिबुद्धत्वादेवमेष ह्यबुद्धिमान्}


\twolineshloka
{कला पञ्चदशी योनिस्तद्धाम इति मन्यते}
{नित्यमेतं विजानीहि सोमं वै षौडशीं कलाम्}


\twolineshloka
{कलया जायते जन्तुः पुनः पुनरबुद्धिमान्}
{धाम तस्योपयुञ्जन्ति भूय एवोपजायते}


\twolineshloka
{षोडशी तु कला सूक्ष्मा स सोम उपधार्यताम्}
{न तूपयुज्यते देवैर्देवानुपयुनक्ति सा}


\twolineshloka
{एतामक्षपयित्वा हि जायते नृपसत्तम}
{सा ह्यस्य प्रकृतिर्दृष्टा तत्क्षयान्मोक्ष उच्यते}


\twolineshloka
{तदेवं षोडशकलं देहमव्यक्तसंज्ञिकम्}
{ममायमिति मन्वानस्तत्रैव परिवर्तते}


\twolineshloka
{पञ्चविंशस्तथैवात्मा तस्यैवाप्रतिबोधनात्}
{विमलश्च विशुद्धश्च शुद्धामलनिषेवणात्}


\twolineshloka
{अशुद्ध एव शुद्धात्मा तादृग्भवति पार्थिव}
{अबुद्धसेवनाच्चापि बुद्धोऽप्यबुद्धतां व्रजेत्}


\threelineshloka
{तथैवाप्रतिबुद्धोऽपि विज्ञेयो नृपसत्तम}
{प्रकृतेस्त्रिगुणायास्तु सेवनात्प्राकृतो भवेत् ॥करालजनक उवाच}
{}


\twolineshloka
{अक्षरक्षरयोरेषु द्वयोः संबन्ध उच्यते}
{स्त्रीपुंसोश्चापि भगवन्संबन्धस्तद्वदुच्यते}


\twolineshloka
{ऋते तु पुरुषं नेह स्त्री गर्भं धारयत्युत}
{ऋते स्त्रियं न पुरुषो रूपं निर्वर्तयेत्तथा}


\twolineshloka
{अन्योन्यस्याभिसंबन्धादन्योन्यगुणसंश्रयात्}
{रूपं निर्वर्तयत्येतदेवं सर्वासु योनिषु}


\twolineshloka
{स्त्रीपुंसोरभिसंबन्धादन्योन्यगुणसंश्रयात्}
{ऋतौ निर्वर्त्यते रूपं तद्वक्ष्यामि निदर्शनम्}


\twolineshloka
{ये गुणाः पुरुषस्येह ये च मातृगुणास्तथा}
{अस्थि स्नायु च मज्जानं जानीमः पैतृकान्द्विज}


\twolineshloka
{त्वङ्भांसं शोणितं चेति मातृजान्यपि शुश्रुम}
{एवमेताद्द्विजश्रेष्ठ वेदे शास्त्रे च पठ्यते}


\twolineshloka
{प्रमाणं यच्च वेदोक्तं शास्त्रोक्तं यच्च पठ्यते}
{वेदशास्त्रप्रमाणानां प्रमाणं तत्सनातनम्}


\twolineshloka
{[अन्योन्यगुणसंरोधादन्योन्यगुणसंश्रयात्}
{]एवमेवाभिसंबद्धौ नित्यं प्रकृतिपूरुषौ}


\threelineshloka
{पश्यामि भगवंस्तस्मान्मोक्षधर्मो न विद्यते}
{अथवाऽनन्तरकृतं किंचिदेव निदर्शनम्}
{तन्ममाचक्ष्व तत्त्वेन प्रत्यक्षो ह्यसि सर्वथा}


\threelineshloka
{मोक्षकामा वयं चापि काङ्क्षामो यदनामयम्}
{अदेहमजरं नित्यमतीन्द्रियमनीश्वरम् ॥वसिष्ठ उवाच}
{}


\twolineshloka
{यदेतदुक्तं भवता देवशास्त्रनिदर्शनम्}
{एवमेतद्यथा चैतन्न गृह्णाति तथा भवान्}


\twolineshloka
{धार्यते हि त्वया ग्रन्थ उभयोर्वेदशास्त्रयोः}
{न च ग्रन्थस्य तत्त्वज्ञो यथावत्त्वं नरेश्वरः}


\twolineshloka
{यो हि वेदे च शास्त्रे च ग्रन्थधारणतत्परः}
{न च ग्रन्थार्थतत्त्वज्ञस्तस्य तद्वारणं वृथा}


\twolineshloka
{मारं स वहते तस्य ग्रन्थस्यार्थं न वेत्ति यः}
{यस्तु ग्रन्थार्थतत्त्वज्ञो नास्य ग्रन्थगुणो वृथा}


\twolineshloka
{ग्रन्थस्यार्थस्य पृष्टः संस्तादृशो वक्तुमर्हति}
{यथातत्त्वाभिगमनादर्थं तस्य स विन्दति}


\twolineshloka
{वस्तु संसत्सु कथयेद्ग्रन्थार्थस्थूलबुद्धिमान्}
{स कथं मन्दविज्ञानो ग्रन्थं वक्ष्यति निर्णयात्}


\twolineshloka
{निर्णयं चापि छिद्रात्मा न तं वक्ष्यति तत्त्वतः}
{सोपहास्यात्मतामेति यस्माच्चावाप्तवानपि}


\twolineshloka
{तस्मात्त्वं शृणु राजेन्द्र यथैतदनुदृश्यते}
{याथातथ्येन साङ्ख्येषु योगेषउ च महात्मसु}


\twolineshloka
{यदेव योगाः पश्यन्ति साङ्ख्यैस्तदवगम्यते}
{एकं साङ्ख्यं च योगं च यः पश्यति स बुद्धिमान्}


\twolineshloka
{त्वङ्भांसं रुघिरं मेदः पित्तं मज्जा च स्नायु च}
{एतदैन्द्रियकं तात तद्भवानिदमाह माम्}


\twolineshloka
{द्रव्याद्द्रव्यस्य निर्वृत्तिरिन्द्रियादिन्द्रियं तथा}
{देहाद्देहमवाप्नोति बीजाद्वीजं तथैव च}


\twolineshloka
{निरिन्द्रियस्याबीजस्य निर्द्रव्यस्याप्यदेहिनः}
{कथं गुणा भविष्यन्ति निर्गुणत्वान्महात्मनः}


\twolineshloka
{गुणा गुणेषु जायन्ते तत्रैव निविशन्ति च}
{एवं गुणाः प्रकृतितो जायन्ते निविशन्ति च}


\twolineshloka
{त्वङ्भांसं रुधिरं मेदः पित्तं मज्जाऽस्थि स्नायु च}
{अष्टौ तान्यथ शुक्रेण जानीहि प्राकृतानि वै}


\twolineshloka
{पुमांश्चैवापुमांश्चैव त्रैलिङ्ग्यं प्राकृतं स्मृतम्}
{न वा पुमान्पुमांश्चैव स लिङ्गीत्यभिधीयते}


\twolineshloka
{अलिङ्गात्प्रकृतिर्लिङ्गैरुपालभ्यति सात्मजैः}
{यथा पुष्पफलैर्नित्यमृतवो मूर्तयस्तथा}


\twolineshloka
{एवमप्यनुमानेन ह्यलिङ्गमुपलभ्यते}
{पञ्चविंशतिमस्तात लिङ्गेषु नियतात्मकः}


\twolineshloka
{अनादिनिधनोऽनन्तः सर्वदर्शी निरामयः}
{केवलं त्वभिमानित्वादगुणेष्वगुणा उच्यते}


\twolineshloka
{गुणा गुणवतः सन्ति निर्गुणस्य कुतो गुणाः}
{तस्मादेवं विजानन्ति ये जना गुणदर्शिनः}


\twolineshloka
{यदा त्वेष गुणनिव प्रकृतावनुमन्यते}
{तदा स गुणवानेव परमं नानुपश्यति}


\twolineshloka
{यत्तं बुद्धेः परं प्राहुः साङ्ख्ययोगाश्च सर्वशः}
{बुध्यमानं महाप्राज्ञमबुद्धपरिवर्जनात्}


\twolineshloka
{अप्रबुद्धमथाव्यक्तं गुणं प्राहुरनीश्वरम्}
{निर्गुणं चेश्वरं नित्यमधिष्ठातारमेव च}


\twolineshloka
{प्रकृतेश्च गुणानां च पञ्चविंशतिकं बुधाः}
{साङ्ख्ययोगे च कुशला बुध्यन्ते परमैषिणः}


\twolineshloka
{यदा प्रबुद्धास्त्वव्यक्तमवस्थाजन्मभीरवः}
{बुध्यमानं प्रबुद्धेन गमयन्ति समन्ततः}


\twolineshloka
{एतन्निदर्शनं सम्यगसम्यक्चार्थदर्शनम्}
{बुध्यमानाप्रबुद्धाना पृथग्पृथगरिंदम्}


\twolineshloka
{परस्परेणैतदुक्तं क्षराक्षरनिदर्शनम्}
{एकत्वमक्षरं प्राहुर्नानात्वं क्षरमुच्यते}


\twolineshloka
{पञ्चविंशतिनिष्ठोऽयं यदा सम्यक्प्रचक्षते}
{एकत्वं दर्शनं चास्य नानात्वं चाप्यदर्शनम्}


\twolineshloka
{तत्त्वनिस्तत्त्वयोरेतत्पृथग्नेव निदर्शनम्}
{पञ्चविंशतितत्वं तु तत्त्वमाहुर्मनीषिणः}


\twolineshloka
{निस्तत्त्वं पञ्चविंशस्य परमाहुर्निदर्शनम्}
{वर्गस्य वर्गमाचारं तत्त्वं तत्त्वात्सनातनम्}


\chapter{अध्यायः ३११}
\twolineshloka
{जनक उवाच}
{}


\twolineshloka
{मानात्वैकत्वमित्युक्तं त्वयैतदृषिसत्तम}
{पश्यामि वाभिसंदिग्धमेतयोर्वै निदर्शनम्}


\twolineshloka
{तथाऽबुद्धप्रबुद्धाभ्यां बुध्यमानस्य चानघ}
{स्थूलबुद्ध्या न पश्यामि तत्त्वमेतन्न संशयः}


\twolineshloka
{अक्षरक्षरयोर्युक्तं त्वया यदपि कारणम्}
{तदप्यस्थिरबुद्धित्वात्प्रनष्टमिव मेऽनघ}


\twolineshloka
{तदेत्तच्छ्रोतुमिच्छामि नानात्वैकत्वदर्शनम्}
{प्रबुद्धमप्रबुद्धं च बुध्यमानं च तत्त्वतः}


\threelineshloka
{विद्याविद्ये च भगवन्नक्षरं क्षरमेव च}
{साङ्ख्यं योगं च कार्त्स्न्येन पृथक्चैवापृथक्च ह ॥वसिष्ठ उवाच}
{}


\twolineshloka
{हन्त ते संप्रवक्ष्यामि यदेतदनुपृच्छसि}
{योगकृत्यं महाराज पृथगेव शृणुष्व मे}


\twolineshloka
{योगकृत्यं तु योगानां ध्यानमेव परं बलम्}
{तच्चापि द्विविधं ध्यानमाहुर्वेदविदो जनाः}


\twolineshloka
{एकाग्रता च मनसः प्राणायामस्तथैव च}
{प्राणायामस्तु सगुणो निर्गुणो मनसस्तथा}


\twolineshloka
{मूत्रोत्सर्गपुरीषे च भोजने च नराधिप}
{द्विकालं नाभियुज्जीत शेषं युञ्जीत तत्परः}


\twolineshloka
{इन्द्रियाणीन्द्रियार्थेभ्यो निवर्त्य मनसा मुनिः}
{दशद्वादशभिर्वापि चतुर्विशात्परं ततः}


\twolineshloka
{तं चोदनाभिर्मतिमानात्मानं चोदयेदथ}
{तिष्ठन्तमजरं यं तु यत्तदुक्तं मनीषिभिः}


\twolineshloka
{तैश्चात्मा सततं योज्य इत्येवमनुशुश्रुम्}
{द्रुतं ह्यहीनमनसो नान्यथेति विनिश्चयः}


% Check verse!
विमुक्तः सर्वसङ्गेभ्यो लध्वाहारो जितेन्द्रियः ॥पूर्वरात्रेऽपररात्रे च धारयेत मनोऽऽत्मनि
\twolineshloka
{स्थिरीकृत्येन्द्रियग्रामं मनसा मिथिलेश्वर}
{मनो बुद्ध्या स्थिरं कृत्वा पाषाण इव निश्चलः}


\twolineshloka
{स्थाणुवच्चाप्यकम्पः स्याद्दारुवच्चापि निश्चलः}
{बुधा विधिविधानज्ञास्तदा युक्तं प्रचक्षते}


\twolineshloka
{न शृणोति न चाघ्राति न रस्यति न पश्यति}
{न च स्पर्शं विजानाति न संकल्पयते मनः}


\twolineshloka
{न चाभिमन्यते किंचिन्न च बुध्यति काष्ठवत्}
{तदा प्रकृतिमापन्नं युक्तमाहुर्मनीषिणः}


\twolineshloka
{निर्वाते हि यथा दीप्यन्दीपस्तद्वत्प्रकाशते}
{निर्लिङ्गो विचलश्चोर्ध्वं न तिर्यग्गतिमाप्नुयात्}


\twolineshloka
{तदा तमनुपश्येत यस्मिन्दृष्टे तु कथ्यते}
{हृदयस्थोऽन्तरात्मेति ज्ञेयो ज्ञस्तात मद्विधैः}


\twolineshloka
{विधूम इव सप्तार्चिरादित्य इव रश्मिमान्}
{वैद्युतोऽग्निरिवाकाशे दृश्यतेऽऽत्मा तथाऽऽत्मनि}


\twolineshloka
{संपश्यन्ति महात्मानो धृतिमन्तो मनीषिणः}
{ब्राह्मणा ब्रह्मयोनिस्था ह्ययोनिममृतात्मकम्}


\twolineshloka
{तदेवाहुरणुभ्योऽणु तन्महद्भ्यो महत्तरम्}
{तत्तत्र सर्वभूतेषु ध्रुवं तिष्ठन्न दृश्यते}


\twolineshloka
{बुद्धिद्रव्येण दृश्येत मनोदीपेन लोककृत्}
{महतस्तमसस्तात पारे तिष्ठन्न तामसः}


\twolineshloka
{स च मानस इत्युक्तस्तत्वज्ञैर्वेदपारगैः}
{विमलो वितमस्कश्च निर्लिङ्गोऽलिङ्गसंज्ञकः}


\twolineshloka
{योगमेतत्तु योगानां मन्ये योगस्य लक्षणम्}
{एवं पश्यं प्रपश्यन्ति आत्मस्थमजरं परम्}


\twolineshloka
{योगदर्शनमेतावदुक्तं ते तत्वतो मया}
{साङ्ख्याज्ञानं प्रवक्ष्यामि परिसंख्यानदर्शनम्}


\twolineshloka
{अव्यक्तमाहुः प्रकृतिं परां प्रकृतिवादिनः}
{तस्मान्महत्समुत्पन्नं द्वितीयं राजसत्तम}


\twolineshloka
{अहंकारस्तु महतस्तृतीय इति नः श्रुतम्}
{पञ्चभूतान्यहंकारादाहुः साङ्ख्यनिदर्शिनः}


\twolineshloka
{एताः प्रकृतयश्चाष्टौ विकाराश्चापि षोडश}
{पञ्च चैव विशेषा वै तथा पञ्चेन्द्रियाणि च}


\twolineshloka
{एतावदेव तत्त्वानां साङ्ख्या आहुर्मनीषिणः}
{साङ्ख्ये विधिविधानज्ञा नित्यं साङ्ख्यपथे रताः}


\twolineshloka
{यस्माद्यदभिजायेत तत्तत्रैव प्रलीयते}
{लीयन्ते प्रतिलोमानि सृज्यन्ते चान्तरात्मना}


\twolineshloka
{अनुलोमेन जायन्ते लीयन्ते प्रतिलोमतः}
{गुणा गुणेषु सततं सागरस्योर्मयो यथा}


\twolineshloka
{सर्वप्रलय एतावान्प्रकृतेर्नृपसत्तम}
{एकत्वं प्रलये चास्य बहुत्वं च यदाऽसृजत्}


\twolineshloka
{एवमेव च राजेन्द्र विज्ञेयं ज्ञेयचिन्तकैः}
{अधिष्ठाता य इत्युक्तस्तस्याप्येतन्निदर्शनम्}


\twolineshloka
{एकत्वं च बहुत्वं च प्रकृतेरनुतत्त्ववान्}
{एकत्वं प्रलये चास्य बहुत्वं च प्रवर्तनात्}


\twolineshloka
{बहुधाऽऽत्मानमकरोत्प्रकृतिः प्रसावत्मिका}
{तच्च क्षेत्रं महानात्मा पञ्चविंशोऽधितिष्ठति}


\twolineshloka
{अधिष्ठातेति राजेन्द्र प्रोच्यते यतिसत्तमैः}
{अधिष्ठानादधिष्ठाता क्षेत्राणामिति नः श्रुतम्}


\twolineshloka
{क्षेत्रं जानाति चाव्यक्तं क्षेत्रज्ञ इति चोच्यते}
{अव्यक्तके पुरे शेते पुरुषश्चेति कथ्यते}


\twolineshloka
{अन्यदेव च क्षेत्रं स्यादन्यः क्षेत्रज्ञ उच्यते}
{क्षेत्रमव्यक्तमित्युक्तं ज्ञाता वै पञ्चविंशकः}


\twolineshloka
{अन्यदेव वचो ज्ञानं स्यादन्यज्ज्ञेयमुच्यते}
{ज्ञानमव्यक्तमित्युक्तं ज्ञेयो वै पञ्चविंशकः}


\twolineshloka
{अव्यक्तं क्षेत्रमित्युक्तं यथासत्वं तथेश्वरम्}
{अनीश्वरमतत्त्वं च तत्त्वं तत्पञ्चविंशकम्}


\twolineshloka
{साङ्ख्यदर्शनमेतावत्परिसङ्ख्यानदर्शनम्}
{साङ्ख्याः प्रकुर्वते चैव प्रकृतिं च प्रचक्षते}


\twolineshloka
{तत्त्वानि च चतुर्विशत्परिसंख्याय तत्त्वतः}
{साङ्ख्याः सह प्रकृत्या तु निस्तत्त्वः पञ्चविंशकः}


\twolineshloka
{पञ्चविंशो प्रबुद्धात्मा बुध्यमान इति स्मृतः}
{यदा तु बुध्यतेऽऽत्मानं तदा भवति केवलः}


\twolineshloka
{सम्यग्दर्शनमेतावद्भाषितं तव तत्त्वतः}
{एवमेतद्विजानन्तः साम्यतां प्रतियान्त्युत}


\twolineshloka
{सम्यङ्गिदर्शनं नाम प्रत्यक्षं प्रकृतेस्तथा}
{गुणतत्त्वाद्यथैतानि निर्गुणोऽन्यस्तथा भवेत्}


\twolineshloka
{न त्वेवं वर्तमानानामवृत्तिर्विद्यते पुनः}
{विद्यतेऽक्षरभावत्वे स परात्परमव्ययम्}


\twolineshloka
{पश्येरन्नेकमतयो न सम्पक्तेषु दर्शनम्}
{तेऽव्यक्तं प्रतिपद्यन्ते पुनः पुनररिंदम्}


\twolineshloka
{सर्वमेतद्विजानन्तो नासर्वस्य प्रबोधनात्}
{व्यक्तीभूता भविष्यन्ति व्यक्तस्य वशवर्तिनः}


\twolineshloka
{सर्वमवर्यक्तमित्युक्तमसर्वः पञ्चविंशकः}
{य एनमभिजानन्ति न भयं तेषु विद्यते}


\chapter{अध्यायः ३१२}
\twolineshloka
{वसिष्ठ उवाच}
{}


\twolineshloka
{साङ्ख्यदर्शनमेतावदुक्तं ते नृपसत्तम्}
{विद्याविद्ये त्विदानीं मे त्वं निबोधानुपूर्वशः}


\twolineshloka
{अविद्यामाहुरव्यक्तं सर्गप्रलयधर्मिणीम्}
{सर्गप्रलयनिर्मुक्तो विद्यो वै पञ्चविंशकः}


\threelineshloka
{`एकत्वं च बहुत्वं च प्रकृतेरनु तत्त्ववित्}
{'परस्परं तु विद्यां वै त्वं निबोधानुपूर्वशः}
{यथोक्तमृषिभिस्तात साङ्ख्यस्यास्य निदर्शनम्}


\twolineshloka
{कर्मेन्द्रियाणां सर्वेषां विद्या बुद्धीन्द्रियं स्मृतम्}
{बुद्धीन्द्रियाणां च तथा विशेषा इति नः श्रुतम्}


\twolineshloka
{विशेषाणां मनस्तेषां विद्यामाहुर्मनीषिणः}
{मनसः पञ्चभूतानि विद्या इत्यभिचक्षते}


\twolineshloka
{अहंकारस्तु भूतानां पञ्चानां नात्र संशयः}
{अहंकारस्य च तथा बुद्धिर्विद्या नरेश्वर}


\twolineshloka
{बुद्धेः प्रकृतिरव्यक्तं तत्त्वानां परमेश्वरम्}
{विद्या ज्ञेया नरश्रेष्ठ विधिश्च परमः स्मृतः}


\twolineshloka
{अव्यक्तस्य परं प्राहुर्विद्यां वै पञ्चविंशकम्}
{सर्वस्य सर्वमित्युक्तं ज्ञेयं ज्ञानस्य पार्थिव}


\twolineshloka
{ज्ञानमव्यक्तमित्युक्तं ज्ञेयो वै पञ्चविंशकः}
{तथैव ज्ञानमव्यक्तं विज्ञाता पञ्चविंशक}


\twolineshloka
{विद्याविद्यार्थितत्त्वेन मयोक्ता ते विशेषतः}
{अक्षरं च क्षरं चैव यदुक्तं तन्निबोध मे}


\twolineshloka
{उभावेतौ क्षरावुक्तावुभावेतौ क्षराक्षरौ}
{कारणं तु प्रवक्ष्यामि यथाख्यातो न जानतः}


\twolineshloka
{अनादिनिधनावेतावुभावेवेश्वरौ मतौ}
{तत्त्वसंज्ञावुभावेतौ प्रोच्यते ज्ञानचिन्तकैः}


\twolineshloka
{सर्गप्रलयधर्मत्वादव्यक्तं प्राहुरक्षरम्}
{तदेतद्गुणसर्गाय विकुर्वाणं पुनःपुनः}


\twolineshloka
{गुणानां महदादीनामुत्पद्यन्ते परम्पराः}
{अधिष्ठानं क्षेत्रमाहुरेतत्तत्पञ्चविंशकम्}


\twolineshloka
{यदा तु गुणजालं तदव्यक्तात्मनि संक्षिपेत्}
{तदा सह गुणैस्तैस्तु पञ्चविंशो विलीयते}


\twolineshloka
{गुणा गुणेषु लीयन्ते तदैका प्रकृतिर्भवेत्}
{क्षेत्रज्ञोऽपि यदा तात तत्क्षेत्रे संप्रलीयते}


\twolineshloka
{तदाऽक्षरत्वं प्रकृतिर्गच्छते गुणसंज्ञिता}
{निर्गुणत्वं च वैदेह गुणेष्वप्रतिवर्तनात्}


\twolineshloka
{एवमेव च क्षेत्रज्ञः क्षेत्रज्ञानपरिक्षयात्}
{प्रकृत्या निर्गुणस्त्वेष इत्येवमनुशुश्रुम}


\twolineshloka
{क्षरो भवत्येष यदा तदा गुणवती मिथः}
{प्रकृतिं त्विभजानाति निर्गुणत्वं तथाऽऽत्मनः}


\twolineshloka
{तदा विशुद्धो भवति प्रकृतेः परिवर्जनात्}
{अन्योऽहमन्येयमिति यदा बुध्यति बुद्धिमान्}


\twolineshloka
{तदैषा त्वन्यतामेति न च मिश्रत्वतां व्रजेत्}
{प्रकृत्या चैव राजेन्द्र मिश्रोऽनन्यश्च दृश्यते}


\twolineshloka
{यदा तु गुणजालं तत्प्राकृतं विजुगुप्सते}
{पश्यते चापरं पश्यं तदा पश्यन्न संस्वजेत्}


\threelineshloka
{किमहं कृतवानेवं योहं कालमिमं जनम्}
{`यदा मत्स्योदकं ज्ञानमनुवर्तितवांस्तदा}
{'मत्स्यो जालं ह्यविज्ञानादनुवर्तितवानिह}


\twolineshloka
{अहमेव हि संमोहादन्यमन्यं जनाज्जनम्}
{मत्स्यो यथोदकज्ञानादनुवर्तितवानहम्}


\twolineshloka
{मत्स्योऽन्यत्वं यथा ज्ञानादुदकान्नाभिमन्यते}
{आत्मानं तद्वदज्ञानादन्यत्वं चैव वेदयहम्}


\twolineshloka
{ममास्तु धिगबुद्धस्य योऽहमज्ञ इमं पुनः}
{अनुवर्तितवान्मोहादन्यमन्यं जनाज्जनम्}


\twolineshloka
{अयमत्र भेवद्बन्धुरनेन सह मे क्षमम्}
{साम्यमेकत्वतां यास्ये यादृशस्तादृशस्त्वहम्}


\twolineshloka
{तुल्यतामिह पश्यामि सदृशोऽहमनेन वै}
{अयं हि विमलोऽव्यक्तमहमीदृशकस्तथा}


\twolineshloka
{योऽहमज्ञानसंमोहादज्ञया संप्रवृत्तवान्}
{ससङ्गयाऽहं निःसङ्गः स्थितः कालमिमं त्वहम्}


\twolineshloka
{अनयाऽहं वशीभूतः कालमेतं न बुद्धवान्}
{उच्चमध्यमनीचानां तामहं कथमावसे}


\twolineshloka
{समानया न याचेह सहवासमहं कथम्}
{गच्छाम्यबुद्धभावत्वादेषेदानीं स्थिरो भवे}


\twolineshloka
{सहवासं न यास्यामि कालमेतद्धि वञ्चनात्}
{वञ्चितोस्म्यनया यद्धि निर्विकारो विकारया}


\twolineshloka
{न चायमपराधोऽस्या ह्यपराधो ह्ययं मम}
{योऽहमत्राभवं सक्तः पराङ्भुखमुपस्थितः}


\twolineshloka
{ततोस्मि बहुरूपासु स्थितो मूर्तिष्वमूर्तिमान्}
{अमूर्तश्चापि मूर्तात्मा ममत्वेन प्रधर्षितः}


\twolineshloka
{प्रकृतेरनयत्वेन तासु तास्विह योनिषु}
{निर्ममस्य ममत्वेन किं कृतं तासु तासु च}


\twolineshloka
{योनीषु वर्तमानेन नष्टसंज्ञेन चेतसा}
{न ममात्रानया कार्यमहंकारकृतात्मना}


\twolineshloka
{आत्मानं बहुधा कृत्वा येयं भूयो युनक्ति माम्}
{इदानीमेष बुद्धोस्मि निर्ममो निरहंकृतः}


\twolineshloka
{ममत्वमनया नित्यमहंकारकृतात्मकम्}
{अपेत्याहमिमां हित्वा संश्रयिष्ये निरामयम्}


\twolineshloka
{अनेन साम्यं यास्यामि नानयाऽहमचेतसा}
{क्षणं मम सहानेन नैकत्वमनया सह}


\twolineshloka
{एवं परमसंबोधात्पञ्चविंशोऽनुबुद्धवान्}
{अक्षरत्वं नियच्छेत त्यक्त्वा क्षरमनामयम्}


\twolineshloka
{अव्यक्तं व्यक्तकर्माणं सगुणं निर्गुणं तथा}
{निर्गुणं परमं दृष्ट्वा तादृग्भवति मैथिल}


\twolineshloka
{अक्षरक्षरयोरेतदुक्तं तत्वनिदर्शनम्}
{मयेह ज्ञानसंपन्नं यथाश्रूतिनिदर्शनात्}


\twolineshloka
{निःसंदिग्धं च सूक्ष्मं च विबुद्धं विमलं यथा}
{प्रवक्ष्यामि तुते भूयस्तन्निबोध यथाश्रुतम्}


\twolineshloka
{साङ्ख्ययोगौ मया प्रोक्तौ शास्त्रद्वयनिदर्शनात्}
{यदेव शास्त्रं साङ्ख्योक्तं योगदर्शनमेव तत्}


\twolineshloka
{प्रबोधनकरं ज्ञानं साङ्ख्यानामवनीपते}
{विस्पष्टं प्रोच्यते तत्र शिष्याणां हितकाम्यया}


\twolineshloka
{पृथक्चैवमिदं शास्त्रमित्याहुः कुशला जनाः}
{अस्मिंश्च शास्त्रे योगानां पुनर्दधि पुनः शरः}


\twolineshloka
{पञ्चविंशात्परं तत्त्वं न पश्यति नराधिप}
{साङ्ख्यानां तु परं तत्त्वं यथावदनुवर्णितम्}


\twolineshloka
{बुद्धमप्रतिबुद्धं च बुध्यमानं च तत्त्वतः}
{बुध्यमानं च बुद्धं च प्राहुर्योगनिदर्शनम्}


\chapter{अध्यायः ३१३}
\twolineshloka
{वसिष्ठ उवाच}
{}


\twolineshloka
{अप्रबुद्धमथाव्यक्तमिमं गुणविधिं शृणु}
{गुणान्धारयते ह्येषा सृजत्याक्षिपते तथा}


\twolineshloka
{अजस्रं त्विह क्रीडार्थं विकरोति जनाधिप}
{आत्मानं बहुधा कृत्वा तान्येव प्रविचक्षते}


\twolineshloka
{एतदेवं विकुर्वाणं बुध्यमानो न बुध्यते}
{अव्यक्तबोधनाच्चैव बुध्यमानं वदन्त्यपि}


\twolineshloka
{न त्वेव बुध्यतेऽव्यक्तं सगुणं वाऽथ निर्गुणम्}
{कदाचित्त्वेव खल्वेतदाहुरप्रतिबुद्धकम्}


\threelineshloka
{बुध्यते यदि वाऽव्यक्तमेतद्वै पञ्चविंशकम्}
{बुध्यमानो भवत्येष सङ्गात्मक इति श्रुतिः}
{अनेनाप्रतिबुद्धेति वदन्त्यव्यक्तमच्युतम्}


\twolineshloka
{अव्यक्तोबोधनाच्चापि बुध्यमानं वदन्त्युत}
{पञ्चविंशं महात्मानं न चासावपि बुध्यते}


\twolineshloka
{षङ्विशं विमलं बुद्धमप्रमेयं सनातनम्}
{स तु तं पञ्चविंशं च चतुर्विशं च बुध्यते}


\twolineshloka
{दृश्यादृश्यौ ह्यनुगतावुभावेव महाद्युती}
{अव्यक्तं तत्तु तद्ब्रह्म बुध्यते तात केवलम्}


\twolineshloka
{केवलं पञ्चविंशं च चतुर्विशं च पश्यति}
{बुध्यमानो यदात्मानमन्योऽहमिति मन्यते}


\twolineshloka
{तदा प्रकृतिमानेष भवत्यव्यक्तलोचनः}
{बुध्यते च परां बुद्धिं विमलाममलां यदा}


\twolineshloka
{षङ्विंशं राजशार्दूल तथा बुद्धत्वमाव्रजेत्}
{ततस्त्यजति सोऽव्यक्तं सर्गप्रलयधर्मि वै}


\twolineshloka
{निर्गुणः प्रकृतिं वेद गुणयुक्तामचेतनाम्}
{ततः केवलधर्माऽसौ भवत्यव्यक्तदर्शनात्}


\twolineshloka
{केवलेन समागम्य विमुक्तोऽऽत्मानमाप्नुयात्}
{एतं वै तत्त्वमित्याहुर्निस्तत्त्वमजरामरम्}


\twolineshloka
{तत्त्वसंश्रयणादेष तत्त्ववान्न च मानद}
{पञ्चविंशतितत्त्वानि प्रवदन्ति मनीषिणः}


\twolineshloka
{न चैष तत्त्ववांस्तात निस्तत्त्वस्त्वेष बुद्धिमान्}
{एष मुञ्चति तत्त्वं हि क्षिप्रं बुद्धत्वलक्षणम्}


\twolineshloka
{षाङ्विंशोऽहमिति प्राज्ञो गृह्यमाणोऽजरामरः}
{केवलेन बलेनैव समतां यात्यसंशयम्}


\twolineshloka
{षङ्विंशेन प्रबुद्धेन बुध्यमानो ह्यबुद्धिमान्}
{एवं नानात्वमित्युक्तं साङ्ख्यश्रुतिनिदर्शनात्}


\twolineshloka
{चेतनेन समेतस्य पञ्चविंशतिकस्य च}
{एकत्वं वै भवत्यस्य यदा बुद्ध्या न बुध्यते}


\twolineshloka
{बुध्यमानोप्रबुद्धेन समतां याति मैथिल}
{सङ्गधर्मा भवत्येष निःसङ्गात्मा नराधिप}


\twolineshloka
{निःसङ्गात्मानमासाद्य षङ्विंशकमजं विभुम्}
{विभुस्त्यजति चाव्यक्तं यदा त्वेताद्विबुध्यते}


\twolineshloka
{चतुर्विशं महाभाग षङ्विंशस्य प्रबोधनात्}
{एष ह्यप्रतिबुद्धश्च बुध्यमानश्च तेऽनघ}


\twolineshloka
{प्रोक्तो बुद्धश्च तत्त्वेन यथाश्रुतिनिदर्शनात्}
{नानात्वैकत्वमेतावद्द्रष्टव्यं शास्त्रदृष्टिभिः}


\twolineshloka
{मशकोदुम्बरे यद्वदन्यत्वं तद्वदेतयोः}
{मत्स्योदके यथा तद्वदन्यत्वमुपलभ्यते}


\twolineshloka
{एवमेवावगन्तव्यं नानात्वैकत्वमेतयोः}
{एतद्धि मोक्ष इत्युक्तमव्यक्तज्ञानसंज्ञितम्}


\twolineshloka
{पञ्चविंशतिकस्यास्य योऽयं देहेषु वर्तते}
{एष मोक्षयितव्येति प्राहुरव्यक्तगोचरात्}


\twolineshloka
{सोयमेवं विमुच्येत नान्यथेति विनिश्चयः}
{परेण परधर्मा च भवत्येष समेत्य वै}


\twolineshloka
{विशुद्धर्मा शुद्धेन बुद्धेन च स बुद्धिमान्}
{विमुक्तधर्मा मुक्तेन समेत्य पुरुषर्षभ}


\twolineshloka
{वियोगधर्मिणा चैव वियोगात्मा भवत्यथ}
{विमोक्षिणा विमोक्षश्च समेत्येह तथा भवेत्}


\twolineshloka
{शुद्धधर्मा शुचिश्चैव भवत्यमितदीप्तिमान्}
{विमलात्मा च भवति समेत्य विमलात्मना}


\twolineshloka
{केवलात्मा तथा चैव केवलेन समेत्य वै}
{स्वतन्त्रश्च स्वतन्त्रेण स्वतन्त्रत्वमवाप्नुते}


\twolineshloka
{एतावदेतत्कथितं मया तेतथ्यं महाराज यथार्थतत्त्वम्}
{अमत्सरत्वं प्रतिगृह्य चार्थंसनातनं ब्रह्म विशुद्धमाद्यम्}


\twolineshloka
{नावेदनिष्ठस्य जनस्य राजन्प्रदेयमेतत्परमं त्वया भवेत्}
{विवित्समानाय विबोधकारणंप्रबोधहेतोः प्रणतस्य शासनम्}


\twolineshloka
{न देयमेतच्च तथाऽनृतात्मनेशठाय क्लीबाय न जिह्नबुद्धये}
{न पण्डितज्ञानपरोपतापिनेदेयं त्वयेदं विनिबोध यादृशे}


\twolineshloka
{श्रद्धान्वितायाथ गुणान्वितायपरापवादाद्विरताय नित्यम्}
{विशुद्धयोगाय बुधाय चैवक्रियावते च क्षमिणे हिताय}


\twolineshloka
{विविक्तशीलाय विधिप्रियायविवादहीनाय बहुश्रुताम्}
{विजानते चैव दमक्षमावतेशक्ताय चैकात्मशमाय देहिनाम्}


\twolineshloka
{एतैर्गुणैर्हीनतमे न देयमेतत्परं ब्रह्म विशुद्धमाहुः}
{न श्रेयसा योक्ष्यति तादृशे कृतंधर्मप्रवक्तारमपात्रदानात्}


\twolineshloka
{पृथ्वीमिमां यद्यपि रत्नपूर्णांदद्यान्न देयं त्विदमव्रताय}
{जितेन्द्रियायैतदसंशयं तेभवेत्प्रदेयं परमं नरेन्द्र}


\twolineshloka
{कराल मा ते भयमस्तु किंचिदेतच्छ्रुतं ब्रह्म परं त्वयाऽद्य}
{यथावदुक्तं परमं पवित्रंविशोकमत्यन्तमनादिमध्यम्}


\twolineshloka
{अगाधजन्मामरणं च राजन्निरामयं वीतभयं शिवं च}
{समीक्ष्य मोहं त्यज बाऽद्य सर्वज्ञानस्य तत्त्वार्थमिदं विदित्वा}


\twolineshloka
{अवाप्तमेतद्धि मया सनातनाद्धिरण्यगर्भाद्यजतो नराधिप}
{प्रसाद्य यत्नेन तमुग्रतेजसंसनातनं ब्रह्म यथाऽद्य वै त्वया}


\threelineshloka
{पृष्टस्त्वया चास्मि यथा नरेन्द्रतथा मयेदं त्वयि चोक्तमद्य}
{तथाऽवाप्तं ब्रह्मणो मे नरेन्द्रमहाज्ञानं मोक्षविदां परायणम् ॥भीष्म उवाच}
{}


\twolineshloka
{एतदुक्तं परं ब्रह्म यस्मान्नावर्तते पुनः}
{पञ्चविंशो महाराज परमर्षिनिदर्शनात्}


\twolineshloka
{पुनरावृत्तिमाप्नोति परं ज्ञानमवाप्य च}
{नावबुध्यति तत्त्वेन बुध्यमानोऽजरामरम्}


\twolineshloka
{एतन्निः श्रेयसकरं ज्ञानं ते परमं मया}
{कथितं तत्त्वतस्तात श्रुत्वा देवर्षितो नृप}


\twolineshloka
{हिरण्यगर्भादृषिणा वसिष्ठेन महात्मना}
{वसिष्ठादृषिशार्दूलान्नारदोऽवाप्तवानिदम्}


\twolineshloka
{नारदाद्विदितं मह्यमेतद्ब्रह्म सनातनम्}
{मा शुचः कौरवेन्द्र त्वं श्रुत्वैतत्परमं पदम्}


\twolineshloka
{येन क्षराक्षरे वित्ते भयं तस्य न विद्यते}
{विद्यते तु भयं तस्य यो नैतद्वेत्ति पार्थिव}


\twolineshloka
{अविज्ञानाच्च मूढात्मा पुनः पुनरुपाद्रवत्}
{प्रेत्य जातिसहस्राणि मरणान्तान्युपाश्नुते}


\twolineshloka
{देवलोकं तथा तिर्यङ्भनुष्यमपि चाश्नुते}
{यदि शुध्यति कालेन तस्मादज्ञानसागरात्}


\threelineshloka
{`उत्तीर्णोऽस्मादगाधात्स परमाप्नोति शोभनम्}
{'अज्ञानसागरो घोरो ह्यव्यक्तोऽगाध उच्यते}
{अहन्यहनि मज्जन्ति यत्र भूतानि भारत}


\twolineshloka
{यस्मादगाधादव्यक्तादुत्तीर्णस्त्वं सनातनात्}
{तस्मात्त्वं विरजाश्चैव वितमस्कश्च पार्थिव}


\chapter{अध्यायः ३१४}
\twolineshloka
{भीष्म उवाच}
{}


\twolineshloka
{मृगयां विचरन्कश्चिद्विजने जनकात्मजः}
{वने ददर्श विप्रेन्द्रमृषिं वंशधरं भृगोः}


\twolineshloka
{तमासीनमुपासीनः प्रणम्य शिरसा मुनिम्}
{पश्चादनुमतस्तेन पप्रच्छ वसुमानिदम्}


\twolineshloka
{भगवन्किमिदं श्रेयः प्रेत्य चापीह वा भवेत्}
{पुरुषस्याध्रुवे देहे कामस्य वशवर्तिनः}


\threelineshloka
{सत्कृत्य परिपृष्टः सन्सुमहात्मा महातपाः}
{निजगाद ततस्तस्मै श्रेयस्करमिदं वचः ॥ऋषिरुवाच}
{}


\twolineshloka
{मनसः प्रतिकूलानि प्रेत्य चेह नचेच्छसि}
{भूतानां प्रतिकूलेभ्यो निवर्तस्य यतेन्द्रियः}


\twolineshloka
{धर्मः सदा हितः पुंसां धर्मश्चैवाश्रयः सताम्}
{धर्माल्लोकास्त्रयस्तात प्रवृत्ताः सचराचराः}


\twolineshloka
{स्वादुकामुक कामानां वैतृष्ण्यं किं न गच्छसि}
{मधु पश्यसि दुर्बुद्धे प्रपातं नानुपश्यसि}


\twolineshloka
{यथा ज्ञाने परिचयः कर्तव्यस्तत्फलार्थिना}
{तथा धर्मे परिचयः कर्तव्यस्तत्फलार्थिना}


\twolineshloka
{असता धर्मकामेन विशुद्धं कर्म दुष्करम्}
{सता तु धर्मकामेन सुकरं कर्म दुष्करम्}


\twolineshloka
{वने ग्राम्यसुखाचारो यथाग्राम्यस्तथैव सः}
{ग्रामे वनसुखाचारो यथा वनचरस्तथा}


\twolineshloka
{मनोवाक्कर्मगे धर्मे कुरु श्रद्धां समाहितः}
{निवृत्तौ वा प्रवृत्तौ वा संप्रधार्य गुणागुणान्}


\twolineshloka
{नित्यं च बहु दातव्यं साधुभ्यश्चानसूयता}
{प्रार्थितं ब्राह्मणेभ्यश्च सत्कृतं देशकालयोः}


\twolineshloka
{शुभेन विधिना लब्धमर्हाय प्रतिपादयेत्}
{क्रोधमुत्सृज्य दत्त्वाऽथ नानुतप्येन्न कीर्तयेत्}


\twolineshloka
{अनृशंसः शुचिर्दान्तः सत्यवागार्जवे स्थितः}
{योनिकर्मविशुद्धश्च पात्रं स्याद्वेदविद्द्विजः}


\twolineshloka
{संस्कृता चैकपत्नी च जात्या योनिरिहेष्यते}
{ऋग्यजुःसामगो विद्वान्षट््कर्मा पात्रमुच्यते}


\twolineshloka
{स एव धर्मः सोऽधर्मस्तं तं प्रति नरं भवेत्}
{पात्राकर्मविशेषेण देशकालाववेक्ष्य च}


\twolineshloka
{लीलयाऽल्पं यथा गात्रात्प्रमृज्यात्तु रजः पुमान्}
{बहुयत्नेन च महत्पापनिर्हरणं तथा}


\twolineshloka
{विरिक्तस्य यथा सम्यग्घृतं भवति भेषजम्}
{तथा निर्हृतदोषस्य प्रेत्य धर्मः सुखावहः}


\twolineshloka
{मानसं सर्वभूतेषु वर्तते वै शुभाशुभम्}
{अशुभेभ्यः समाक्षिप्य शुभेष्वेवावधारय}


\twolineshloka
{सर्वं सर्वेणा सर्वत्र क्रियमाणं च पूजय}
{स्वधर्मे यत्र रागस्ते कामं धर्मो विधीयताम्}


\twolineshloka
{अधृतात्मन्धृतौ तिष्ठ दुर्बद्धे बुद्धिमान्भव}
{अप्रशान्तः प्रशाम्य त्वमप्राज्ञः प्राज्ञवच्चर}


\twolineshloka
{तेजसा शक्यते प्राप्तुमुपायः सहचारिणा}
{इह च प्रेत्य च श्रेयस्तस्य मूलं धृतिः परा}


\twolineshloka
{राजर्षिरधृतिः स्वर्गात्पतितो हि महाभिषः}
{ययातिः क्षीणपुण्योपि धृत्या लोकानवाप्तवान्}


\threelineshloka
{तपस्विनां धर्मवतां विदुषां चोपसेवनात्}
{प्राप्स्यसे विपुलां बुद्धिं तथा श्रेयोऽभिपत्स्यसे ॥भीष्म उवाच}
{}


\twolineshloka
{स तु स्वभावसंपन्नस्तच्छ्रुत्वा मुनिभाषितम्}
{विनिवर्त्य मनः कामाद्धर्मे बुद्धिं चकार ह}


\chapter{अध्यायः ३१५}
\twolineshloka
{युधिष्ठिर उवाच}
{}


\twolineshloka
{धर्माधर्मविमुक्तं यद्विमुक्तं सर्वसंशयात्}
{जन्ममृत्युविमुक्तं च विमुक्तं पुण्यपापयोः}


\threelineshloka
{यच्छिवं नित्यमभयं नित्यमक्षरमव्ययम्}
{शुचि नित्यमनायासं तद्भवान्वक्तुमर्हति ॥भीष्म उवाच}
{}


\twolineshloka
{अत्र ते वर्तयिष्यामि इतिहासं पुरातनम्}
{याज्ञवल्क्यस्य संवादं जनकस्य च भारत}


\threelineshloka
{याज्ञवल्क्यमृषिश्रेष्ठं दैवरातिर्महायशाः}
{पप्रच्छ जनको राजा प्रश्नं प्रश्नविदांवरः ॥जनक उवाच}
{}


\twolineshloka
{कतीन्द्रियाणि विप्रर्षे कति प्रकृतयः स्मृताः}
{किमव्यक्तं परं ब्रह्म तस्माच्च परतस्तु किम्}


\twolineshloka
{प्रभवं चाप्ययं चैव कालसङ्ख्यां तथैव च}
{वक्तुमर्हसि विप्रेन्द्र त्वदनुग्रहकाङ्क्षिणः}


\threelineshloka
{अज्ञानात्परिपृच्छामि त्वं हि ज्ञानमयो निधिः}
{तदहं श्रोतुमिच्छामि सर्वमेतदसंशयम् ॥याज्ञवल्क्य उवाच}
{}


\twolineshloka
{श्रूयतामवनीपाल यदेतदनुपृच्छसि}
{योगानां परमं ज्ञानं साङ्ख्यानां च विशेषतः}


\twolineshloka
{त तवाविदितं किंचिन्मां तु जिज्ञासते भवान्}
{पृष्टेन चापि वक्तव्यमेष धर्मः सनातनः}


\twolineshloka
{अष्टौ प्रकृतयः प्रोक्ता विकाराश्चापि षोडश}
{आसां तु सप्त व्यक्तानि प्राहुरध्यात्मचिन्तकाः}


\twolineshloka
{अव्यक्तं च महांश्चैव तथाऽहंकार एव च}
{पृथिवी वायुराकाशमापो ज्योतिश्च पञ्चमम्}


\twolineshloka
{एताः प्रकृतयस्त्वष्टौ विकारानपि मे शृणु}
{श्रोत्रं त्वक्चैव चक्षुश्च जिह्वा घ्राणं च पञ्चमम्}


\twolineshloka
{शब्दः स्पर्शश्च रूपं च रसो गन्धस्तथैव च}
{वाक्च हस्तौ च पादौ च पायुर्मेढ्रं तथैव च}


\twolineshloka
{एते विशेषा राजेन्द्रा महाभूतेषु पञ्चसु}
{बुद्धीन्द्रियाण्यथैतानि सविशेषाणि मैथिल}


\twolineshloka
{मनः षोडशकं प्राहुरध्यात्मगतिचिन्तकाः}
{त्वं चैवान्ये च विद्वांसस्तत्त्वबुद्धिविशारदाः}


\twolineshloka
{अव्यक्ताच्च महानात्मा समुत्पद्यति पार्थिव}
{प्रथमं सर्गमित्येतदाहुः प्राधानिकं बुधाः}


\twolineshloka
{महतश्चाप्यहंकार उत्पद्यति नराधिप}
{द्वितीयं सर्गमित्याहुरेतद्बुद्ध्यात्मकं स्मृतम्}


\twolineshloka
{अहंकाराच्च संभूतं मनो भूतगुणात्मकम्}
{तृतीयः सर्ग इत्येष आहंकारिक उच्यते}


\twolineshloka
{मनसस्तु समुद्भूता महाभूता नराधिप}
{चतुर्थं सर्गमित्येतन्मानसं चिन्तनात्मकम्}


\twolineshloka
{शब्दः स्पर्शश्च रूपं च रसो गन्धस्तथैव च}
{पञ्चमं सर्गमित्याहुर्भौतिकं भूतचिन्तकाः}


\twolineshloka
{श्रोत्रं त्वक्चैव चक्षुश्च जिह्वा घ्राणं च पञ्चमम्}
{सर्गं तु षष्ठमित्याहुर्बहुचिन्तात्मकं स्मृतम्}


\twolineshloka
{अधः श्रोत्रेन्द्रियग्राम उत्पद्यदि नराधिप}
{सप्तमं सर्गमित्याहुरेतदैन्द्रियकं स्मृतम्}


\twolineshloka
{ऊर्ध्वं स्रोतस्तथा तिर्यगुत्पद्यति नराधिप}
{अष्टमं सर्गमित्याहुरेतदार्जवकं स्मृतम्}


\twolineshloka
{तिर्यक्स्रोतस्त्वधःस्रोत उत्पद्यति नराधिप}
{नवमं सर्गमित्याहुरेतदार्जवकं बुधाः}


\twolineshloka
{एते वै नव सर्गा हि तत्त्वानि च नराधिप}
{चतुर्विशतिरुक्तानि यथाश्रुतिनिदर्शनम्}


\twolineshloka
{अत ऊर्ध्वं महाराज गुणस्यैतस्य तत्त्वतः}
{महात्मभिरनुप्रोक्तां कालसङ्ख्यां निबोध मे}


\chapter{अध्यायः ३१६}
\twolineshloka
{याज्ञवल्क्य उवाच}
{}


\twolineshloka
{अव्यक्तस्य नरश्रेष्ठ कालसङ्ख्यां निबोध मे}
{पञ्चकल्पसहस्राणि द्विगुणान्यहरुच्यते}


\twolineshloka
{रात्रिरेतावती चास्य प्रतिबुद्धो नराधिप}
{सृजत्योषधिमेवाग्रे जीवनं सर्वदेहिनाम्}


\twolineshloka
{ततो ब्रह्माणमसृजद्धैरण्याण्डसमुद्भवम्}
{सा मूर्तिः सर्वभूतानामित्येवमनुशुश्रुम}


\twolineshloka
{संवत्सरमुषित्वाण्डे निष्क्रम्य च महामुनिः}
{संदधेऽर्धं महीं कृत्स्नां दिवमर्धं प्रजापतिः}


\twolineshloka
{द्यावापृथिव्योरिज्येष राजन्वेदेषु पठ्यते}
{तयोः शकलयोर्मध्यमाकाशमकरोत्प्रभुः}


\twolineshloka
{एतस्यापि च सङ्ख्यानं वेदवेदाङ्गपारगैः}
{दशकल्पसहस्राणि पादोनान्यहरुच्यते}


\twolineshloka
{रात्रिमेतावतीं चास्व प्राहुरध्यात्मचिन्तकाः}
{सृजत्यहंकारमृषिर्भूतं दिव्यात्मकं तथा}


\twolineshloka
{चतुरश्चापरान्पुत्रान्देहात्पूर्वं महानृषिः}
{ते वै पितृभ्यः पितरः श्रूयन्तें राजसत्तम}


\twolineshloka
{देवाः पितॄणां च सुता देवैर्लोकाः समावृताः}
{चराचरा नरश्रेष्ठ इत्येवमनुशुश्रुम}


\twolineshloka
{परमेष्ठी त्वहंकारोऽसृजद्भूतानि पञ्चधा}
{पृथिवी वायुराकाशमापो ज्योतिश्च पञ्चमम्}


\twolineshloka
{एतस्यापि निशामाहुस्तृतीयमथ कुर्वतः}
{पञ्च कल्पसहस्राणि तावदेवाहरुच्यते}


\twolineshloka
{शब्दः स्पर्शश्च रूपं च रसो गन्धश्च पञ्चमः}
{एते विशेषा राजेन्द्र महाभूतेषु पञ्चसु}


\twolineshloka
{यैराविष्टानि भूतानि अहन्यहनि पार्थिव}
{अन्योन्यं स्पृहयन्त्येते अन्योन्यस्याहिते रताः}


\twolineshloka
{अन्योन्यमतिवर्तन्ते अन्योन्यस्पर्धिनस्तथा}
{ते वध्यमाना ह्यन्योन्यं गुणैर्हारिभिरव्ययाः}


\twolineshloka
{इहैव परिवर्तन्ते तिर्यग्योनिप्रवेशिनः}
{त्रीणि कल्पसहस्राणि एतेषामहरुच्यते}


\twolineshloka
{रात्रिरेतावती चैव मनसश्च नराधिप}
{मनश्चरति राजेन्द्र चरितं सर्वमिन्द्रियैः}


\twolineshloka
{न चेन्द्रियाणि पश्यन्ति मन एवात्र पश्यति}
{चक्षुः पश्यति रूपाणि मनसा तु न चक्षुषा}


\twolineshloka
{मनसि व्याकुले चक्षुः पश्यन्नपि न पश्यति}
{अथेन्द्रियाणि सर्वाणि पश्यन्तीत्यभिचक्षते}


% Check verse!
मनस्युपरते राजन्निन्द्रियोपरमो भवेत्
\twolineshloka
{न चेन्द्रियव्युपरमे मनस्युपरमो भवेत्}
{एवं मनः प्रधानानि इन्द्रियाणि प्रभावयेत्}


\twolineshloka
{इन्द्रियाणां तु सर्वेषामीश्वरं मन उच्यते}
{एतद्विशन्ति भूतानि सर्वाणीह महायशः}


\chapter{अध्यायः ३१७}
\twolineshloka
{याज्ञवल्क्य उवाच}
{}


\twolineshloka
{तत्त्वानां सर्गसङ्ख्या च कालसङ्ख्या तथैव च}
{मया प्रोक्ताऽनुपूर्वेण संहारमपि मे शृणु}


\threelineshloka
{यता संहरते जन्तून्ससर्ज च पुनः पुनः}
{अनादिनिधनो ब्रह्मा नित्यश्चाक्षर एव च}
{}


\twolineshloka
{अहःक्षयमथो बुद्ध्वा निशि स्वप्नमनास्तथा}
{चोदयामास भगवानव्यक्तोऽहंकृतं नरम्}


\twolineshloka
{ततः शतसहस्रांशुरव्यक्तेनाभिचोदितः}
{कृत्वा द्वादशधाऽऽत्मानमादित्यो ज्वलदग्निवत्}


\twolineshloka
{चतुर्विधं प्रजाजातं निर्दहत्याशु तेजसा}
{जराय्वण्डस्वेदजातमुद्भिज्जं स नराधिप}


\twolineshloka
{एतदुन्मेषमात्रेण विनष्टं स्थाणुजङ्गमम्}
{कूर्मपृष्ठसमा भूमिर्भवत्यथ समन्ततः}


\twolineshloka
{जगद्दग्ध्वाऽमितबलः केवलां जगर्ती ततः}
{अम्भसा बलिना क्षिप्रमापूरयति सर्वशः}


\twolineshloka
{ततः कालाग्निमासाद्य तदम्भो याति संक्षयम्}
{विनष्टेऽम्भसि राजेन्द्र जाज्वलत्यनलो महान्}


\twolineshloka
{तमप्रमेयातिबलं ज्वलमानं विभावसुम्}
{ऊष्माणं सर्वभूतानां सप्ताचिंपमथाञ्जसा}


\twolineshloka
{भक्षयामास भगवान्वायुरष्टात्मको बली}
{विचरन्नमितप्राणस्तिर्यगूर्ध्वमधस्तथा}


\twolineshloka
{तमप्रतिबलं भीममाकाशं ग्रसते पुनः}
{आकाशमप्यभिनदन्मनो ग्रसति चारिकम्}


\twolineshloka
{मनो ग्रसति सर्वात्मा सोहंकारः प्रजापतिः}
{अहंकारो महानात्मा भूतभव्यभविष्यवित्}


\twolineshloka
{तमप्यनुपमात्मानं विश्वं शंभुः प्रजापतिः}
{अणिमा लघिमा प्राप्तिरीशानो ज्योतिरव्ययः}


\twolineshloka
{सर्वतः पाणिपादं तत्सर्वतोक्षिशिरोमुखम्}
{सर्वतः श्रुतिमल्लोके सर्वमावृत्य तिष्ठति}


\twolineshloka
{हृदयं सर्वभूतानां पर्वणाऽङ्गुष्ठमात्रकः}
{अणुग्रसत्यनन्तो हि महात्मा विश्वमीश्वरः}


\twolineshloka
{ततः समभवत्सर्वमक्षयाव्ययमव्रणम्}
{भूतभव्यभविष्याणां स्रष्टारमनघं तथा}


\twolineshloka
{एषोप्ययस्ते राजेन्द्र यथावत्समुदाहृतः}
{अध्यात्ममधिभूतं च श्रूयतां चाधिदैवतम्}


\chapter{अध्यायः ३१८}
\twolineshloka
{याज्ञवल्क्य उवाच}
{}


\twolineshloka
{पादावध्यात्ममित्याहुर्ब्राह्मणास्तत्त्वदर्शिनः}
{गन्तव्यमधिभूतं च विष्णुस्तत्राधिदैवतम्}


\twolineshloka
{पायुरध्यात्ममित्याहुर्यथातत्त्वार्थदर्शिनः}
{विसर्गमधिभूतं च मित्रस्तत्राधिदैवतम्}


\twolineshloka
{उपस्थोऽध्यात्ममित्याहुर्यथायोगप्रदर्शिनः}
{अधिभूतं तथाऽऽनन्दो दैवतं च प्रजापतिः}


\twolineshloka
{हस्तावध्यात्ममित्याहुर्यथासङ्ख्यानदर्शिनः}
{कर्तव्यमधिभूतं तु इन्द्रस्तत्राधिदैवतम्}


\twolineshloka
{वागध्यात्ममिति प्राहुर्यथाश्रुतिनिदर्शिनः}
{वक्तव्यमधिभूतं तु वह्निस्तत्राधिदैवतम्}


\twolineshloka
{चक्षुरध्यात्ममित्याहुर्यथाश्रुतिनिदर्शिनः}
{रूपमत्राधिभूतं तु सूर्यश्चाप्यधिदैवतम्}


\twolineshloka
{श्रोत्रमध्यात्ममित्याहुर्यथाश्रुतिनिदर्शिनः}
{शब्दस्तत्राधिभूतं तु दिशश्चात्राधिदैवतम्}


\twolineshloka
{जिह्वामध्यात्ममित्याहुर्यथाश्रुतिनिदर्शिनः}
{रस एवाधिभूतं तु आपस्तत्राधिदैवतम्}


\twolineshloka
{घ्राणमध्यात्ममित्याहुर्थथाश्रुतिनिदर्शिनः}
{गन्ध एवाधिभूतं तु पृथिवी चाधिदैवतम्}


\twolineshloka
{त्वगध्यात्ममिति प्राहुस्तत्त्वबुद्धिविशारदाः}
{स्पर्शमेवाधिभूतं तु पवनश्चाधिदैवतम्}


\twolineshloka
{मनोऽध्यात्ममिति प्राहुर्यथा शास्त्रविशारदाः}
{मन्तव्यमधिभूतं तु चन्द्रमाश्चाधिदैवतम्}


\twolineshloka
{आहंकारिकमध्यात्ममाहुस्तत्त्वनिदर्शिनः}
{अभिमानोऽधिभूतं तु बुद्धिश्चात्राधिदैवतम्}


\twolineshloka
{बुद्धिरध्यात्ममित्याहुर्यथावदभिदर्शिनः}
{बोद्धव्यमधिभूतं तु क्षेत्रज्ञश्चाधिदैवतम्}


\twolineshloka
{एषा ते व्यक्तितो राजन्विभूतिरनुदर्शिता}
{आदौ मध्ये तथाऽन्ते च यथा तत्त्वेन तत्त्ववित्}


\twolineshloka
{प्रकृतिर्गुणान्विकुरुते स्वच्छन्देनात्मकाम्यया}
{क्रीडार्थे तु महाराज शतशोऽथ सहस्रशः}


\twolineshloka
{यथा दीपसहस्राणि दीपान्मर्त्याः प्रकुर्वते}
{प्रकृतिस्तथा विकुरुते पुरुषस्य गुणान्बहून्}


\twolineshloka
{सत्वमानन्द उद्रेकः प्रीतिः प्राकाम्यमेव च}
{सुखं शुद्धत्वमारोग्यं संतोषः श्रद्दधानता}


\twolineshloka
{अकार्पण्यमसंरम्भः क्षमा धृतिरहिंसता}
{समता सत्यमानृण्यमार्जवं ह्रीरचापलम्}


\twolineshloka
{शौचमार्दवमाचारमलौल्यं हृद्यसंभ्रमः}
{इष्टानिष्टवियोगानां कृतानामविकत्थना}


\twolineshloka
{दानेन चात्मग्रहणमस्पृहत्वं परार्थता}
{सर्वभूतदया चैव सत्वस्यैते गुणाः स्मृताः}


\twolineshloka
{रजोगुणानां संघातो रूपमैश्वर्यविग्रहौ}
{अत्यागित्वमकारुण्यं सुखदुःखोपसेवनम्}


\twolineshloka
{परापवादेषु रतिर्विवादानां च सेवनम्}
{अहंकारस्त्वसत्कारश्चिन्ता वैरोपसेवनम्}


\twolineshloka
{परितापोऽभिहरणं ह्रीनाशोऽनार्जवं तथा}
{भेदः परुषता चैव कामक्रोधो मदस्तथा}


\twolineshloka
{दर्पो द्वेषोऽतिमानश्च एते प्रोक्ता रजोगुणाः}
{तामसानां तुं संघातान्प्रवक्ष्याम्युपधार्यताम्}


\twolineshloka
{मोहोऽप्रकाशस्तामिस्रमन्धतामिस्रसंज्ञितम्}
{मरणं चान्धतामिस्रं तामिस्रं क्रोध उज्यते}


\twolineshloka
{तमसो लक्षणानीह भक्षणाद्यभिरोचनम्}
{भोजनानामपर्याप्तिस्तथा पेयेष्वतृप्तता}


\twolineshloka
{गन्धवासो विहारेषु शयनेष्वासनेषु च}
{दिवास्वप्ने विवादे च प्रमादेषु च वै रतिः}


\twolineshloka
{नृत्यवादित्रगीतानामज्ञानाच्छ्रद्दधानता}
{द्वेषो धर्मविशेषाणामेते वै तामसा गुणाः}


\chapter{अध्यायः ३१९}
\twolineshloka
{याज्ञवल्क्य उवाच}
{}


\twolineshloka
{एते प्रधानस्य गुणास्त्रयः पुरुषसत्तम}
{कृत्स्नस्य चैव जगतस्तिष्ठन्त्यनपगाः सदा}


\twolineshloka
{अव्यक्तरूपो भगवाञ्शतधा च सहस्रधा}
{शतधा सहस्रधा चैव तथा शतसहस्रधा}


\twolineshloka
{कोटिशश्च करोत्येव प्रकृत्याऽऽत्मानमात्मना}
{सात्विकरणोत्तमं स्थानं राजसस्येह मध्यमम्}


\twolineshloka
{तामसस्वाधमं स्थानं प्राहुरध्यात्मचिन्तकाः}
{केवलेनेह पुण्येन भतिभूर्ध्वामवाप्नुयात्}


\twolineshloka
{पुण्यपापेन मानुष्यमधर्मेणाप्यघोगतिम्}
{द्वन्द्वनेषां त्रयाणां तु सन्निपातं च तत्वतः}


\twolineshloka
{सत्वस्य रजसश्चैव तमसश्च शृणुष्व मे}
{सत्वस्य तु रजो दृष्टं रजसश्च तमस्तथा}


\twolineshloka
{तमसश्च तथा सत्वं सत्वस्याव्यक्तमेव च}
{अध्यक्तः सत्वसंयुक्तो देवलोकमयाप्नुयात्}


\twolineshloka
{रवासत्वसमायुक्तो मानुषेषु प्रपद्यते}
{रसस्तमोभ्यां संयुक्तस्तिर्यग्योनिषु जायते}


\twolineshloka
{राजसैस्तामसैः सत्वैर्युक्तो मानुषमाप्नुयात्}
{पुण्यपापवियुक्तानां स्थानमाहुर्महात्मनाम्}


\threelineshloka
{शाश्वतं चाव्ययं चैवमक्षयं चामृतं च तत्}
{ज्ञानिनां संभवं श्रेष्ठं स्थानमव्रणमच्युतम्}
{अतीन्द्रियमबीजं च जन्ममृत्युतमोनुदम्}


\twolineshloka
{अव्यक्तस्थं परं यत्तत्पृष्टस्तेऽहं नराधिप}
{स एव प्रकृतिस्थो हि तत्स्थ इत्यभिधीयते}


\threelineshloka
{अचेतना चैव मता प्रकृतिश्चापि पार्थिव}
{एतेनाधिष्ठिता चैव सृजते संहरत्यपि ॥जनक उवाच}
{}


\twolineshloka
{अनादिनिधनावेतावुभावेव महामते}
{अमूर्तिमन्तावचलावप्रकम्प्यगुणागुणौ}


\twolineshloka
{अग्राह्यावृषिशार्दूल कथमेको ह्यचेतनः}
{चेतनावांस्तथा चैकः क्षेत्रज्ञ इति भाषितः}


\twolineshloka
{त्वं हि विप्रेन्द्र कार्त्स्न्येन र्मोक्षधर्ममुपाससे}
{साकल्पं मोक्षधर्मस्य श्रोतुमिच्छामि तत्त्वतः}


\twolineshloka
{निस्तत्वं केवलत्वं च विनाभावं तथैव च}
{दैवतानि च मे ब्रूहि देहं यान्याश्रितानि वै}


\twolineshloka
{तथैवोत्क्राणिणः स्थानं देहिनो वै विपद्यतः}
{कालेन यद्धि प्राप्नोति स्थानं तत्प्रब्रवीहि मे}


\threelineshloka
{साङ्ख्यज्ञानं च तत्त्वेन पृथग्योगं तथैव च}
{अरिष्टानि च तत्त्वानि वक्तुमर्हसि सत्तम}
{विदितं सर्वमेतत्ते पाणावामलकं यथा}


\chapter{अध्यायः ३२०}
\twolineshloka
{याज्ञवल्क्य उवाच}
{}


\twolineshloka
{न शक्यो निर्गुणस्तात गुणीकर्तुं विशांपते}
{गुणवांश्चाप्यगुणवान्यथातत्त्वं निबोध मे}


\twolineshloka
{गुणैर्हि गुणवानेव निर्गुणश्चागुणस्तथा}
{प्राहुरेवं महात्मानो मुनयस्तत्त्वदर्शिनः}


\twolineshloka
{गुणस्वभावस्त्वव्यक्तो गुणानेवाभिवर्तते}
{उपयुङ्क्ते च तानेव स चैवाज्ञः स्वभावतः}


\twolineshloka
{अव्यक्तस्तु न जानीते पुरुषोऽज्ञः स्वभावतः}
{न मत्तः परमोस्तीति नित्यमेवाभिमन्यते}


\twolineshloka
{अनेन कारणेनैतदव्यक्तं स्यादचेतनम्}
{नित्यत्वाच्चाक्षरत्वाच्च क्षरत्वान्न तदन्यथा}


\twolineshloka
{यदाऽज्ञानेन कुर्वीत गुणसर्गं पुनःपुनः}
{यदात्मानं न जानीते तदाऽऽत्मापि न मुच्यते}


\twolineshloka
{कर्तृत्वाच्चापि सर्गाणां सर्गधर्मा तथोच्यते}
{कर्तृत्वाच्चापि योगानां योगधर्मा तथोच्यते}


% Check verse!
कर्तृत्वात्प्रकृतीनां च तथा प्रकृतिधर्मिता
\twolineshloka
{कर्तृत्वाच्चापि वीजानां बीजधर्मा तथोच्यते}
{गुणानां प्रसवत्वाच्च प्रलयत्वात्तथैव च}


\twolineshloka
{`कर्तृत्वात्प्रलयानां तु तथा प्रलयधर्मि च}
{कर्तृत्वात्प्रभवाणां च तथा प्रभवधर्मि च}


\twolineshloka
{बीजत्वात्प्रकृतित्वाच्च प्रलयत्वात्तथैव च}
{'उपेक्षत्वादनन्यत्वादभिमानाच्च केवलम्}


\twolineshloka
{मन्यन्ते यतयः सिद्धा अध्यात्मज्ञा गतज्वराः}
{अनित्यं नित्यमव्यक्तं व्यक्तमेतद्धि शुश्रुम}


\twolineshloka
{अव्यक्तैकत्वमित्याहुर्नानात्वं पुरुषास्तथा}
{सर्वभूतदयावन्तः केवलं ज्ञानमास्थिताः}


\threelineshloka
{अन्यः स पुरुषोऽव्यक्तस्त्वध्रुवो ध्रुवसंज्ञकः}
{यथा मुञ्ज इषीकाणां तथैवैतद्धि जायते}
{`न चैव मुञ्जसंयोगादिषीका तत्र बुध्यते ॥'}


\twolineshloka
{अन्यच्च मशकं विद्यादन्यच्चोदुम्बरं तथा}
{च चोदुम्बरसंयोगैर्मशकस्तत्र लिप्यते}


\twolineshloka
{अन्य एव तथा मत्स्यस्तदन्यदुदुकं स्मृतम्}
{न चोदकस्य स्पर्शेन मत्स्यो लिप्यति सर्वशः}


\twolineshloka
{अन्यो ह्यग्निरुखाऽप्यन्या नित्यमेवमवेहि भोः}
{न चोपलिप्यते सोऽग्निरुखासंस्पर्शनेन वै}


\twolineshloka
{पुष्करं त्वन्यदेवात्र तथाऽन्यदुदकं स्मृतम्}
{न चोदकस्य स्पर्शेन लिप्यते तत्र पुष्करम्}


\twolineshloka
{एतेषां सहवासं च निवासं चैव नित्यशः}
{याथातथ्येन पश्यन्ति न नित्यं प्राकृता जनाः}


\twolineshloka
{ये त्वन्यथैव पश्यन्ति न सम्यक्तेषु दर्शनम्}
{ते व्यक्तं निरयं घोरं प्रविशन्ति पुनः पुनः}


\twolineshloka
{साङ्ख्यदर्शनमेतत्ते परिसङ्ख्यानमुत्तमम्}
{एवं हि परिसंख्याय साङ्ख्याः केवलतां गताः}


\twolineshloka
{ये त्वन्ये तत्त्वकुशलास्तेषामेतन्निदर्शनम्}
{अतः परं प्रवक्ष्यामि योगानामनुदर्शनम्}


\chapter{अध्यायः ३२१}
\twolineshloka
{याज्ञवल्क्य उवाच}
{}


\twolineshloka
{साङ्ख्यज्ञानं मया प्रोक्तं योगज्ञानं निबोध मे}
{यथाश्रुतं यथादृष्टं तत्त्वेन नृपसत्तम}


\twolineshloka
{नास्ति साङ्ख्यसमं ज्ञानं नास्ति योगसमं बलम्}
{तावुभावेकचर्यौ तावुभावनिधनौ स्मृतौ}


\twolineshloka
{पृथक्पृथक्प्रपश्यन्ति येऽप्यबुद्धिरता नराः}
{वयं तु राजन्पश्याम एकमेव तु निश्चयात्}


\twolineshloka
{यदेव योगाः पश्यन्ति तत्साङ्ख्यैरपि दृश्यते}
{एकं साङ्ख्यं च योगं च यः पश्यति स तत्त्ववित्}


\twolineshloka
{रुद्रप्रधानानपरान्विद्धि योगानरिंदम्}
{तेनैव चाथ देहेन विचरन्ति दिशो दश}


\twolineshloka
{यावद्धि प्रलयस्तात सूक्ष्मेणाष्टगुणेन ह}
{योगेन लोकान्विचरन्सुखं संन्यस्य चानघ}


\twolineshloka
{तावदेवाष्टगुणिनं योगप्राहुर्मनीषिणः}
{सूक्ष्ममष्टगुणं प्राहुर्नेतरं नृपसत्तम}


\twolineshloka
{द्विगुणं योगत्यं तु योगानां प्राहुरुत्तमम्}
{सगुणं निर्गुणं चैव यथाशास्त्रनिदर्शनम्}


\twolineshloka
{धारणं चैव मनसः प्राणायामश्च पार्थिव}
{एकाग्रता च मनसः प्राणायामस्तथैव च}


\threelineshloka
{प्राणायामो हि सगुणो निर्गुणं धारयेन्मनः}
{यद्यदृश्यति मुञ्चन्वै प्राणान्मैथिलसत्तम}
{वाताधिक्यं भवत्येव तस्मात्तं न समाचरेत्}


\twolineshloka
{निशायाः प्रथमे यामे चोदना द्वादश स्मृताः}
{मध्ये स्वप्नात्परे यामे द्वादशैव तु चोदनाः}


\twolineshloka
{तदेवमुपशान्तेन दान्तेनैकान्तशीलिना}
{आत्मारामेण बुद्धेन योक्तव्योऽऽत्मा न संशयः}


\twolineshloka
{पञ्चानामिन्द्रियाणां तु दोषानाक्षिप्य पञ्चधा}
{शब्दं रूपं तथा स्पर्शं रसं गन्धं तथैव च}


\twolineshloka
{प्रतिभामपवर्गं च प्रतिसंहृत्य मैथिल}
{इन्द्रियग्राममखिलं मनस्यभिनिवेश्य ह}


\twolineshloka
{मनस्तथैवाहंकारे प्रतिष्ठाप्य नराधिप}
{अहंकारं तथा बुद्धौ बुद्धिं च प्रकृतावपि}


\twolineshloka
{एवं हि परिसंख्याय ततो ध्यायन्ति केवलम्}
{विरजस्कमलं नित्यमनन्तं शुद्धमव्रणम्}


\twolineshloka
{तस्थुषं पुरुषं नित्यमभेद्यमजरामरम्}
{शाश्वतं चाव्ययं चैव ईशानं ब्रह्म चाख्यम्}


\twolineshloka
{युक्तस्य तु महाराज लक्षणान्युपधारम्}
{लक्षणं तु प्रसादस्य यथा तृप्तः सुखं स्वयेत्}


\twolineshloka
{निर्वाते तु यथा दीपो ज्वलेत्स्नेहस --धतः}
{निश्चलोर्ध्वशिखस्तद्वद्युक्तमाहुर्मनीषिण}


\twolineshloka
{पाषाण इव मेघोत्थैर्यथा बिन्दुभिराहतः}
{नालं चालयितुं शक्यस्तथा युक्तस्य लक्षणम्}


\twolineshloka
{शक्तदुन्दुभिनिर्घोषैर्विधिधैर्गीतवादितैः}
{क्रियमाणैर्न कम्पेत युक्तस्यैतन्निदर्शनम्}


\twolineshloka
{तैलपात्रं यथा पूर्णं कराभ्यां गृह्य पूरुषः}
{सोपानमारुहेद्भीतस्तर्ज्यमानोऽसिषणिभिः}


\twolineshloka
{संयतात्मा भयात्तेषां न पात्राद्बिन्दुमुत्सृजेत्}
{तथैवोत्तरमागम्य एकाग्रमनसस्तथा}


\twolineshloka
{स्थिरत्वादिन्द्रियाणां तु निश्चलस्तथैव च}
{एवं युक्तस्य तु मुनेर्लक्षणान्युपल----}


\twolineshloka
{स्वयुक्तः पश्यते ब्रह्म यत्तत्परम----यम्}
{महतस्तमसो मध्ये स्थितं ज्व नसा--भम्}


\twolineshloka
{एतेन केवलं याति त्यक्त्वा देहमसाक्षिकम्}
{कालेन महता राजञ्श्रुतिरेषा सनातनी}


\twolineshloka
{एतद्धि योगं योगानां किमन्यद्योगलक्षणम्}
{विज्ञाय तद्धि मन्यन्ते कृतकृत्या मनीषिणः}


\chapter{अध्यायः ३२२}
\twolineshloka
{याज्ञवल्क्य उवाच}
{}


\twolineshloka
{तथैवोत्क्रमतां स्थानं शृणुष्वावहितो नृप}
{पद्भ्यामुत्क्रममाणस्य वैष्णवं स्थानमुच्यते}


\twolineshloka
{जङ्घाभ्यां तु वसून्देवानाप्नुयादिति नः श्रुतम्}
{जानुभ्यां च महाभागान्साध्यान्देवानवाप्नुयात्}


\twolineshloka
{पायुनोत्क्रममाणस्तु मैत्रं स्थानमवाप्नुयात्}
{पृथिवीं जघनेनाथ ऊरुभ्यां च प्रजापतिम्}


\twolineshloka
{पार्श्वाभ्यां मरुतो देवान्नासाभ्यामिन्दुमेव च}
{बाहुभ्यामिन्द्रमित्याहुरुरसा रुद्रमेव च}


\twolineshloka
{ग्रीवया तु मुनिश्रेष्ठं नरमाप्नोत्यनुत्तमम्}
{विश्वेदेवान्मुखेनाथ दिशः श्रोत्रेण चाप्नुयात्}


\twolineshloka
{घ्राणेन गन्धवहनं नेत्राभ्यां सूर्यमेव च}
{भ्रूभ्यां चैवाश्विनौ देवौ ललाटेन पितृनथ}


\twolineshloka
{ब्रह्माणमाप्नोति विभुं मूर्ध्ना देवाग्रजं तथा}
{एतान्युत्क्रमणस्थानान्युक्तानि मिथिलेश्वर}


\twolineshloka
{अरिगनि प्रवक्ष्यामि विहितानि मनीषिभिः}
{संवत्सराद्धिमोक्षस्तु संभवेत शरीरिणः}


\twolineshloka
{योऽरुन्धतीं न पश्येत दृष्टपूर्वा कदाचन}
{तथैव ध्रुवमतित्याहुः पूर्णेन्दुं दीपमेव च}


\twolineshloka
{खण्डाभासं दक्षिणतस्तेऽपि संवत्सरायुषः}
{परचक्षुषि चात्मानं ये न पश्यन्ति पार्थिवः}


\twolineshloka
{आत्मच्छायाकृतीभूतं तेऽपि संवत्सरायुषः}
{अतिद्युतिरतिप्रज्ञा अप्रज्ञा चाद्युतिस्तथा}


\twolineshloka
{प्रकृतेर्विक्रियापत्तिः षण्मासान्मृत्युलक्षणम्}
{दैवतान्यवजानाति ब्राह्मणैश्च विरुध्यते}


\twolineshloka
{कृष्णश्यावच्छविच्छायः षण्मासान्मृत्युलक्षणम्}
{शीर्णनाभिं यथा चक्रं छिद्रं सोमं प्रपश्यति}


\threelineshloka
{तथैव च सहस्रांशुं सप्तरात्रेण मृत्युभाक्}
{शवगन्धमुपाघ्राति सप्तरात्रेण मृत्युभाक्}
{}


% Check verse!
कर्णनासावनमनं दन्तदृष्टिविरागितां ॥कर्णनासावनमनं दन्तदृष्टिविरागिता
\twolineshloka
{संज्ञालोपो निरूष्मत्वं सद्योमृत्युनिदर्शनम्}
{अकस्माच्च स्रवेद्यस्य वाममक्षि नराधिप}


\twolineshloka
{मूर्ध्रतश्चोत्पतेद्धूमः सद्योमृत्युनिदर्शनम्}
{एतावन्ति त्वरिष्टानि विदित्वा मानवोऽऽत्मवान्}


\twolineshloka
{निशि चाहनि चात्मानं योजयेत्परमात्मनि}
{प्रतीक्षमाणस्तत्कालं यः कालः प्रकृतो भवेत्}


\twolineshloka
{अथास्य नेष्टं मरणं स्थातुमिच्छेदिमां क्रियाम्}
{सर्वगन्धान्रसांश्चैव धारयीत समाहितः}


\threelineshloka
{`तथा मृत्युमुपादाय तत्परेणान्दरात्मना}
{'स साङ्ख्यधारणं चैव विदित्वा मनुजर्षभ}
{जयेच्च मृत्युं योगेन तत्परेणान्तरात्मना}


\twolineshloka
{गच्छेत्प्राप्याक्षयं कृत्स्नमजन्म शिवमव्ययम्}
{शाश्वतं स्थानमचलं दुष्प्रापमकृतात्मभिः}


\chapter{अध्यायः ३२३}
\twolineshloka
{याज्ञवल्क्य उवाच}
{}


\twolineshloka
{अव्यक्तस्थं परं यत्तत्पृष्टस्तेऽहं नराधिप}
{परं गुह्यमिमं प्रश्नं शृणुष्वावहितो नृप}


\twolineshloka
{यथार्षेणेह विधिना चरताऽवमतेन ह}
{मयाऽऽदित्यादवाप्तानि यजूंषि मिथिलाधिप}


\twolineshloka
{महता तपसा देवस्तपिष्णुः सेवितो मया}
{प्रीतेन चाहं विभुना सूर्येणोक्तस्तदाऽनघ}


\twolineshloka
{वरं वृणीष्व विप्रर्षे यदिष्टं ते सुदुर्लभम्}
{तं ते दास्यामि प्रीतात्मा मत्प्रसादो हि दुर्लभः}


\twolineshloka
{ततः प्रणम्य शिरसा मयोक्तस्तपतांवरः}
{यजूंषि नोपयुक्तानि क्षिप्रमिच्छामि वेदितुम्}


\twolineshloka
{ततो मां भगवानाह वितरिष्यामि ते द्विज}
{सरस्वतीह वाग्भूता शरीरं ते प्रवेक्ष्यति}


\twolineshloka
{ततो मामाह भगवानास्यं स्वं विवृतं कुरु}
{विवृतं च ततो मेऽऽस्यं प्रविष्टा च सरस्वती}


\twolineshloka
{ततो विदह्यमानोऽहं प्रविष्टोऽम्भस्तदाऽनघ}
{अविज्ञानादमर्षाच्च भास्करस्य महात्मनः}


\twolineshloka
{ततो विदह्यमानं मामुवाच भगवान्रविः}
{मुहूर्तं सह्यतां दाहस्ततः शीतीभविष्यति}


\twolineshloka
{शीतीभूतं च मां दृष्ट्वा भगवानाह भास्करः}
{प्रतिभास्यति ते वेदः सखिलः सोत्तरो द्विज}


\twolineshloka
{कृत्स्नं शतपथं चैव प्रणेष्यसि द्विजर्षभ}
{तस्यान्ते चाषुनर्भावे बुद्धिस्तव भविष्यति}


\twolineshloka
{प्राप्स्यसे च यदिष्टं तत्साङ्ख्ययोगेप्सितं पदम्}
{एतावदुक्त्वा भगवानस्तमेवाभ्यवर्तत}


\twolineshloka
{ततोऽनुव्याहृतं श्रुत्वा गते देवे विभावसौ}
{गृहमागत्य संहृष्टोऽचिन्तयं वै सरस्वतीम्}


\twolineshloka
{ततः प्रवृत्ताऽतिशुभा स्वरव्यञ्जनभूषिता}
{ओंकारमादितः कृत्वा मम देवी सरस्वती}


\twolineshloka
{ततोऽहमर्ध्यं विधिवत्सरस्वत्यै न्यवेदयम्}
{परं यत्नमवाप्यैव निषण्णस्तत्परायणः}


\twolineshloka
{ततः शतपथं कृत्स्नं सरहस्यं ससंग्रहम्}
{चक्रे सपरिशेषं च हर्षेण परमेण ह}


\twolineshloka
{कृत्वा चाध्ययनं तेषां शिष्याणां शतमुत्तमम्}
{विप्रियार्थं सशिष्यस्य मातुलस्य महात्मनः}


\twolineshloka
{ततः सशिष्येण मया सूर्येणेव गभस्तिभिः}
{व्यस्तो यज्ञो महाराज पितुस्तव महात्मनः}


\twolineshloka
{मिषतो देवलस्यापि ततोऽर्धं हृतवान्वसु}
{स्ववेददक्षिणायार्थे विमर्दे मातुलेन ह}


\twolineshloka
{सुमन्तुनाऽथ पैलेन तथा जैमिनिना च वै}
{पित्रा ते मुनिभिश्चैव ततोऽहमनुमानितः}


\twolineshloka
{दश पञ्च च प्राप्तानि यजूंष्यर्कान्मयाऽनघ}
{तथैव रोमहर्षेण पुराणमवधारितम्}


\twolineshloka
{बीजमेतत्पुरस्कृत्य देवीं चैव सरस्वतीम्}
{सूर्यस्य चानुभावेन प्रवृत्तोऽहं नराधिप}


\twolineshloka
{कर्तुं शतपथं चेदमपूर्वं च कृतं मया}
{यथाभिलपितं मार्गं तथा तच्चोपपादितम्}


\twolineshloka
{शिष्याणामखिलं कृत्स्नमनुज्ञातं ससंग्रहम्}
{सर्वे च शिष्याः शुचयो गताः परमहर्षिताः}


\twolineshloka
{शाखाः पञ्चदशेमास्तु विद्या भास्करदर्शिता}
{प्रतिष्ठाप्य यथाकामं वेद्यं तदनुचितयम्}


\twolineshloka
{किमत्र ब्रह्मण्यमृतं किंच वेद्यमनुत्तमम्}
{चिन्तयंस्तत्र चागत्य गन्धर्वो मामपृच्छत}


\twolineshloka
{विश्वावसुस्ततो राजन्वेदान्तज्ञानकोविदः}
{चतुर्विशांस्ततोऽपृच्छत्प्रश्नान्वेदस्य पार्थिव}


\twolineshloka
{पञ्चविंशतिमं प्रश्नं पप्रच्छान्वीक्षिकीं तदा}
{`तथैव पुरुषव्याघ्र मित्रं वरुणमेव च ॥'}


\twolineshloka
{ज्ञानं ज्ञेयं तथा ज्ञोऽज्ञः कस्तपा अतपास्तथा}
{सूर्यातिसूर्य इति च विद्याविद्ये तथैव च}


\twolineshloka
{वेद्यावेद्यं तथा राजन्नचलं चलमेव च}
{अव्ययं चाक्षरं क्षेम्यमेतत्प्रश्नमनुत्तमम्}


\twolineshloka
{अथोक्तश्च महाराज राजा गन्धर्वसत्तमः}
{पृष्टवानानुपूर्व्येण प्रश्नमर्थवदुत्तमम्}


\twolineshloka
{मुहूर्तमुष्यतां तावद्यावदेनं विचिन्तये}
{बाढमित्येव कृत्वा स तूर्ष्णीं गन्धर्व आस्थितः}


\twolineshloka
{ततोऽनुचिन्तयमहं भूयो देवीं सरस्वतीम्}
{मनसा स च मे प्रश्नो दध्नो धृतमिवोद्धृतः}


\twolineshloka
{तत्रोपनिषदं चैव परिशेषं च पार्थिव}
{मथ्नामि मनसा तात दृष्ट्वा चान्वीक्षिकीं पराम्}


\twolineshloka
{चतुर्थी राजशार्दूल विद्यैषा सांपरायिकी}
{उदीरिता मया तुभ्यं पञ्चविंशाऽधितिष्ठता}


\twolineshloka
{अथोक्तस्तु मया राजन्राजा विश्वावसुस्तदा}
{श्रूयतां यद्भवानस्मान्प्रश्नं संपृष्टवानिह}


\twolineshloka
{विश्वाविश्वेति यदिदं गन्धर्वेन्द्रानुपृच्छसि}
{विश्वाव्यक्तं परं विद्याद्भूतभव्यभयंकरम्}


\twolineshloka
{त्रिगुणं गुणकर्तृत्वाद्विश्वान्यो निष्कलस्तथा}
{विश्वाविश्वेति मिथुनमेवमेवानुदृश्यते}


\twolineshloka
{अव्यक्तं प्रकृतिः प्राहुः पुरुषेति च निर्गुणम्}
{तथैव मित्रं पुरुषं वरुणं प्रकृतिं तथा}


\twolineshloka
{ज्ञानं तु प्रकृतिं प्राहुर्ज्ञेयं पुरुषमेव च}
{अज्ञमव्यक्तमित्युक्तं ज्ञस्तु निष्कल उच्यते}


\twolineshloka
{कस्तपा अतपाः प्रोक्तः कोसौ पुरुष उच्यते}
{तपास्तु प्रकृतिं प्राहुरतपा निष्कलः स्मृतः}


\twolineshloka
{`सूर्यमव्यक्तमित्युक्तमतिसूर्यस्तु निष्कलः}
{अविद्या प्रोक्तमव्यक्तं विद्या पुरुष उच्यते ॥'}


\twolineshloka
{तथैवावेद्यमव्यक्तं वेद्यः पुरुष उच्यते}
{चलाचलमिति प्रोक्तं त्वया तदपि मे शृणु}


\twolineshloka
{चलां तु प्रकृतिं प्राहुः कारणं क्षेपसर्गयोः}
{अक्षेपसर्गयोः कर्ता निश्चलः पुरुषः स्मृतः}


% Check verse!
अज्ञावुभौ ध्रुवौ चैव अक्षयौ चाप्युभावपि
\threelineshloka
{अजौ नित्यावुभौ प्राहुरध्यात्मगतिनिश्चयाः}
{अक्षयत्वात्प्रजनने अजमत्राहुरव्ययम्}
{अक्षयं पुरुषं प्राहुः क्षयो ह्यस्य न विद्यते}


\twolineshloka
{गुणक्षयत्वात्प्रकृतिः कर्तृत्वादक्षयं बुधाः}
{एषा तेऽन्वीक्षिकी विद्या चतुर्थी सांपरायिकी}


\twolineshloka
{विद्योपेतं धनं कृत्वा कर्मणा नित्यकर्मणि}
{एकान्तदर्शना वेदाः सर्वे विश्वावसो स्मृताः}


\twolineshloka
{जायन्ते च म्रियन्ते च यस्मिन्नेते यतश्च्युताः}
{वेदार्थं ये न जानीते वेद्यं गन्धर्वसत्तम}


\twolineshloka
{साङ्गोपाङ्गानपि यदि पञ्च वेदानधीयते}
{वेदवेद्यं न जानीते वेदभारवहो हि सः}


\twolineshloka
{यो घृतार्थी खराक्षीरं मथेद्गन्धर्वसत्तम}
{विष्ठां तत्रानुपश्येत न मण़्डं न च वै घृतम्}


\twolineshloka
{तथा वेद्यमवेद्यं च वेदविद्यो न विन्दति}
{स केवलं मूढमतिर्वेदभारवहः स्मृतः}


\twolineshloka
{द्रष्टव्यौ नित्यमेवैतौ तत्परेणान्तरात्मना}
{यथाऽस्य जन्मनिधने न भवेतां पुनः पुनः}


\twolineshloka
{अजस्रं जन्मनिधनं चिन्तयित्वा त्रयीमिमाम्}
{परित्यज्य क्षयमिह अक्षयं धर्ममास्थितः}


\twolineshloka
{यदाऽनुपश्यतेऽत्यन्तमहन्यहनि काश्यप}
{तदा स केवलीभूतः षङ्विंशमनुपश्यति}


\twolineshloka
{अन्यश्च शाश्वतो व्यक्तस्तथाऽन्यः पञ्चविंशकः}
{तत्स्थं समनुपश्यन्ति तमेकमिति साधवः}


\threelineshloka
{तेनैतं नाभिनन्दन्ति पञ्चविंसकमच्युतम्}
{जन्ममृत्युभयाद्योगाः साख्याश्च परमैषिणः ॥विश्वावसुरुवाच}
{}


\twolineshloka
{पञ्चविंशं यदेतत्ते प्रोक्तं ब्राह्मणसत्तम}
{तदहं न तथा वेद्मि तद्भवान्वक्तुमर्हति}


\twolineshloka
{जैगीषव्यस्यासितस्य देवलस्य मया श्रुतम्}
{पराशरस्य विप्रर्षेर्वार्षगण्यस्य धीमतः}


\twolineshloka
{भिक्षोः पञ्चशिखस्यास्य कपिलस्य शुकस्य च}
{गौतमस्याष्टिंषेणस्य गर्गस्य च महात्मनः}


\twolineshloka
{नारदस्यासुरेश्चैव पुलस्त्यस्य च धीमतः}
{सनत्कुमारस्य ततः शुक्रस्य च महात्मनः}


\twolineshloka
{कश्यपस्य पितुश्चैव पूर्वमेव मया श्रुतम्}
{तदनन्तरं च रुद्रस्य विश्वरूपस्य धीमतः}


\twolineshloka
{दैवतेभ्यः पितृभ्यश्च दैतेयेभ्यस्ततस्ततः}
{प्राप्तमेतन्मया कृत्स्नं वेद्यं नित्यं वदन्त्युत}


\twolineshloka
{तस्मात्तद्वै भवद्बुद्ध्या श्रोतुमिच्छामि ब्राह्मण}
{भवान्प्रबर्हः शास्त्राणां प्रगल्भश्चातिबुद्धिमान्}


\twolineshloka
{न तवाविदितं किंचिद्भवाञ्श्रुतिनिधिः स्मृतः}
{कथ्यसे देवलोके च पितृलोके च ब्राह्मण}


\twolineshloka
{ब्रह्मलोकगताश्चैव कथयन्ति महर्षयः}
{पतिश्च तपतां शश्वदादित्यस्तव भाषिता}


\twolineshloka
{साङ्ख्यज्ञानं त्वया ब्रह्मन्नवाप्तं कृत्स्नमेव च}
{तथैव योगशास्त्रं च याज्ञवल्क्य विशेषतः}


\threelineshloka
{निःसंदिग्धं प्रबुद्धस्त्वं बुध्यमानश्चराचरम्}
{श्रोतुमिच्छामि तज्ज्ञानं घृतं मण्डमयं यथा ॥याज्ञवल्क्य उवाच}
{}


\twolineshloka
{कृत्स्नधारिणमेव त्वां मन्ये गन्धर्वसत्तम}
{जिज्ञासमे च मां राजंस्तन्निबोध यथाश्रुतम्}


\twolineshloka
{बुध्यमानो हि प्रकृतिं बुध्यते पञ्चविंशकः}
{न तु बुध्यति गन्धर्वप्रकृतिः पञ्चविंशकम्}


\twolineshloka
{अनेन प्रतिबोधेन प्रधानं प्रवदन्ति तत्}
{साङ्ख्ययोगार्थतत्त्वज्ञा यथ्नाश्रुतिनिदर्शनात्}


\twolineshloka
{पश्यंस्तथैव चापश्यन्पश्यत्यन्यः सदाऽनघ}
{षङ्विंशं पञ्चविंशं च चतुर्विशं च पश्यति}


\twolineshloka
{न तु पश्यति पश्यंस्तु यश्चैनमनुपश्यति}
{पञ्चविंशोऽभिमन्येत नान्योऽस्ति परतो मम}


\twolineshloka
{न चतुर्विशको ग्राह्यो मनुजैर्ज्ञानदर्शिभिः}
{मत्स्यो वोदकमन्वेति प्रवर्तेत प्रवर्तनात्}


\twolineshloka
{यथैव बुध्यते मत्स्यस्तथैषोऽप्यनुबुध्यते}
{स स्नेहात्सहवासाच्च साभिमानाच्च नित्यशः}


\twolineshloka
{स निमज्जति कालस्य यदैकत्वं न बुध्यते}
{उन्मज्जति हि कालस्य समत्वेनाभिसंवृतः}


\twolineshloka
{यदा तु मन्यतेऽन्योऽहमन्य एष इति द्वजिः}
{तदा स केवलीभूतः षङ्विंशमनुपश्यति}


\twolineshloka
{अन्यश्च राजन्परमस्तथाऽन्यः पञ्चविंशकः}
{तत्स्थत्वादनुपश्यन्ति एक एवेति साधवः}


\threelineshloka
{तेनैतन्नाभिनन्दन्ति पञ्चविंशकमच्युतम्}
{जन्ममृत्युभयाद्भीता योगाः साङ्ख्याश्च काश्यप}
{षङ्विंशमनुपश्यन्तः शुचयस्तत्परायणाः}


\twolineshloka
{यदा स केवलीभूतः षङ्विंशमनुपश्यति}
{तदा स सर्वविद्विद्वान्न पुनर्जन्म विन्दति}


\twolineshloka
{एवमप्रतिबुद्धश्च बुध्यमानश्च तेऽनघ}
{बुद्धिश्चोक्ता यथातत्त्वं मया श्रुतिनिदर्शनात्}


\threelineshloka
{पश्यापश्यं यो न पश्येत्क्षेम्यं तत्वं च काश्यप}
{केवलाकेवलं चान्यत्पञ्चविंशं परं च यत् ॥विश्वावसुरुवाच}
{}


\fourlineindentedshloka
{तथ्यं शुभं चैतदुक्तं त्वया विभो}
{सम्यक्क्षेम्यं दैवताद्यं यथावत्}
{स्वस्त्यक्षयं भवतश्चास्तु नित्यंबुद्ध्या सदा बुद्धियुक्तं नमस्ये ॥याज्ञवल्क्य उवाच}
{}


\twolineshloka
{एवमुक्त्वा संप्रयातो दिवं सविभ्राजन्वै श्रीमता दर्शनेन}
{दृष्टश्च तुष्ट्या परयाऽभिनन्द्यप्रदक्षिणं मम कृत्वा महात्मा}


\twolineshloka
{ब्रह्मादीनां खेचराणां क्षितौ चये चाधस्तात्संवसन्ते नरेन्द्र}
{तत्रैव तद्दर्शनं दर्शयन्वैसम्यक्क्षेम्यं ये पथं संश्रिता वै}


\twolineshloka
{साङ्ख्याः सर्वे साङ्ख्यधर्मे रताश्चतद्वद्योगा योगधर्मे रताश्च}
{ये चाप्यन्ये मोक्षकामा मनुष्यास्तेषामेतद्दर्शनं ज्ञानदृष्टम्}


\twolineshloka
{ज्ञानान्मोक्षो जायते राजसिंहनास्त्यज्ञानादेवमाहुर्नरेन्द्र}
{तस्माज्ज्ञानं तत्त्वतोऽन्तेषितव्यंयेनात्मानं मोक्षयेज्जन्ममृत्योः}


\twolineshloka
{प्राप्य ज्ञानं ब्राह्मणात्क्षत्रियाद्वावैश्याच्छ्रद्रादपि नीचादभीक्ष्णम्}
{श्रद्धातव्यं श्रद्दधानेन नित्यंन श्रद्धिनं जन्ममृत्यू विशेताम्}


\threelineshloka
{सर्वे वर्णा ब्राह्मणा ब्रह्मजाश्चसर्वे नित्यं व्याहरन्ते च ब्रह्म}
{` येनात्मानं मोक्षयेज्जन्ममृत्योस्तत्त्वं शास्त्रं ब्रह्मबुद्ध्या ब्रवीमि}
{'तत्त्वं शास्त्रं ब्रह्मबुद्ध्या ब्रवीमिसर्वं विश्वं ब्रह्म चैतत्समस्तम्}


\twolineshloka
{ब्रह्मास्यतो ब्राह्मणाः संप्रसूताबाहुभ्यां वै क्षत्रियाः संप्रसूताः}
{नाभ्यां वैश्याः पादतश्चापि शूद्राःसर्वे वर्णा नान्यथा वेदितव्याः}


\twolineshloka
{अज्ञानतः कर्मयोनिं भजन्तेतांतां राजंस्ते यथा यान्त्यभावम्}
{तथा वर्णा ज्ञानहीनाः पतन्तेघोरादज्ञानात्प्राकृतं योनिजालम्}


\twolineshloka
{तस्माज्ज्ञानं सर्वतो मार्गितव्यंसर्वत्रस्थं चैतदुक्तं मया ते}
{तत्स्थो ब्रह्मा तस्थिवांश्चापरो यस्तस्मै नित्यं मोक्षमाहुर्नरेन्द्र}


\threelineshloka
{यत्ते पृष्टं तन्मया चोपदिष्टंयाथातथ्यं तद्विशोको भजस्व}
{राजन्गच्छस्वैतदर्थस्य पारंसम्यक्प्रोक्तं स्वस्ति ते त्वस्तु नित्यम् ॥भीष्म उवाच}
{}


\twolineshloka
{स एवमनुशिष्टस्तु याज्ञवल्क्येन धीमता}
{प्रीतिमानभवद्राजा मिथिलाधिपतिस्तदा}


\twolineshloka
{गते मुनिवरे तस्मिन्कृते चापि प्रदक्षिणम्}
{दैवरातिर्नरपतिरासीनस्तत्र मोक्षवित्}


\twolineshloka
{गोकोटिं स्पर्शयामास हिरण्यस्य तथैव च}
{रत्नाञ्जलिमथैकैकं ब्राह्मणेभ्यो ददौ तदा}


\twolineshloka
{वेदहराज्यं च तदा प्रतिष्ठाप्य सुतस्य वै}
{यतिधर्ममुपास्यंश्चाप्यवसन्मिथिलाधिपः}


\twolineshloka
{साङ्ख्यज्ञानमधीयानो योगशास्त्रं च कृत्स्नशः}
{धर्माधर्मं च राजेन्द्र प्राकृतं परिगर्हयन्}


\twolineshloka
{अनन्त इति कृत्वा स नित्यं केवलमेव च}
{धर्माधर्मौ पुण्यपापे सत्यासत्ये तथैव च}


\twolineshloka
{जन्ममृत्यू च राजेन्द्र प्राकृतं तदचिन्तयत्}
{ब्रह्माव्यक्तस्य कर्मेदमिति नित्यं नराधिप}


\twolineshloka
{पश्यन्ति योगाः साङ्ख्याश्च स्वशास्त्रकृतलक्षणाः}
{इष्टानिष्टविमुक्तं हि तस्थौ ब्रह्म परात्परम्}


\twolineshloka
{नित्यं तदाहुर्विद्वांसः शुचि तस्माच्छुचिर्भव}
{दीयते यच्च लभते दत्तं यच्चानुमन्यते}


\threelineshloka
{`अव्यक्तेनेति तच्चिन्त्यमन्यथा मा विचन्तय}
{'ददाति च नरश्रेष्ठ प्रतिगृह्णाति यच्च ह}
{ददात्यव्यक्त इत्येतत्प्रतिगृह्णाति यच्च वै}


\twolineshloka
{आत्मा ह्येवात्मनो ह्येकः कोऽन्यस्तस्मात्परो भवेत्}
{एवं मन्यस्व सततमन्यथा मा विचिन्तय}


\twolineshloka
{यस्याव्यक्तं न विदितं सगुणं निर्गुणं पुनः}
{तेन तीर्थानि यज्ञाश्च सेवितव्या विपश्चिता}


\twolineshloka
{न स्वाध्यायैस्तपोभिर्वा यज्ञैर्वा कुरुनन्दन}
{लभतेऽव्यक्तिकं स्थानं ज्ञात्वाऽव्यक्तं महीयते}


\twolineshloka
{तथैव महतः स्थानमाहंकारिकमेव च}
{अहंकारात्परं चापि स्थानानि समवाप्नुयात्}


\twolineshloka
{ये त्वव्यक्तात्परं नित्यं जानते शास्त्रतत्पराः}
{जन्ममृत्युविमुक्तं च विमुक्तं सदसच्च यत्}


\twolineshloka
{एतन्मयाऽऽप्तं जनकात्पुरस्तात्तेनापि चाप्तं नृप याज्ञवल्क्यात्}
{ज्ञातं विशिष्टं न तथा हि यज्ञाज्ञानेन दुर्गं तरते न यज्ञैः}


\twolineshloka
{दुर्गं जन्म निधनं चापि राजन्न भौतिकं ज्ञानविदो वदन्ति}
{यज्ञैस्तपोभिर्नियमैर्व्रतैश्चदिवं समासाद्य पतन्ति भूमौ}


\twolineshloka
{तस्मादुपासस्व परं महच्छुचिशिवं विमोक्षं विमलं पवित्रम्}
{क्षेत्रं ज्ञात्वा पार्थिव ज्ञानयज्ञमुपास्य वै तत्त्वमृषिर्भविष्यसि}


\twolineshloka
{युदुपनिषदमुपाकरोत्तथाऽसौजनकनृपस्य पुरा हि याज्ञवल्क्यः}
{यदुपगणितशाश्वताव्ययं तच्छुभममृतत्वमशोकमर्च्छति}


\chapter{अध्यायः ३२४}
\twolineshloka
{युधिष्ठिर उवाच}
{}


\twolineshloka
{ऐश्वर्यं वा महत्प्राप्य धनं वा भरतर्षभ}
{दीर्घमायुरवाप्याथ कथं मृत्युमतिक्रमेत्}


\threelineshloka
{तपसा वा सुमहता कर्मणा वा श्रुतेन वा}
{रसायनप्रयोगैर्वा कैर्नाप्नोति जरान्तकौ ॥भीष्म उवाच}
{}


\twolineshloka
{अत्राप्युदाहरन्तीममितिहासं पुरातनम्}
{भिक्षोः पञ्चशिखस्येह संवादं जनकस्य च}


\twolineshloka
{वैदेहो जनको राजा महर्षि वेदवित्तमम्}
{यर्यपृच्छत्पञ्चशिखं छिन्नधर्मार्थसंशयम्}


\twolineshloka
{केन वृत्तेन भगवन्नतिक्रामेज्जरान्तकौ}
{तपसा वाऽथवा बुद्ध्या कर्मणा वा श्रुतेन वा}


\twolineshloka
{एवमुक्तः स वैदेहं प्रत्युवाचापरेक्षवित्}
{निवृत्तिर्नैतयोरस्ति नातिवृत्तिः कथंचन}


\twolineshloka
{न ह्यहानि निवर्तन्ते न मासा न पुनः क्षपाः}
{सोयं प्रपद्यतेऽध्वानं चिराय ध्रुवमध्रुवः}


\twolineshloka
{सर्वभूतसमुच्छेदः स्रोतसेवोह्यते सदा}
{ऊह्यमानं निमज्जन्तमप्लवे कालसागरे}


\twolineshloka
{जरामृत्युमहाग्राहे न कश्चिदतिवर्तते}
{नैवास्य कश्चिद्भवति नासौ भवति कस्यचित्}


\twolineshloka
{पथि संगतमेवेदं दारैरन्यैश्च बन्धुभिः}
{नायमत्यन्तसंवासो लब्धपूर्वो हि केनचित्}


\twolineshloka
{क्षिप्यन्ते तेनतेनैव निष्टनन्तः पुनः पुनः}
{कालेन जाता याता हि वायुनेवाभ्रसंचयाः}


\twolineshloka
{जरामृत्यू हि भूतानां खादितारौ वृकाविव}
{बलिनां दुर्बलानां च ह्रस्वानां महतामपि}


\twolineshloka
{एवंभूतेषु भूतेषु नित्यभूताध्रवेषु च}
{कथं हि हृष्येज्जातेषु मृतेषु च न संज्वरेत्}


\twolineshloka
{कुतोऽहमागतः कोऽस्मि क्व गमिष्यामि कस्य वा}
{कस्मिन्स्थितः क्व भविता कस्मात्किमनुशोचसि}


\twolineshloka
{द्रष्टा स्वर्गस्य न ह्यस्ति तथैव नरकस्य च}
{आगमास्त्वनतिक्रम्य दद्याच्चैव यजेत च}


\chapter{अध्यायः ३२५}
\twolineshloka
{युधिष्ठिर उवाच}
{}


\twolineshloka
{अपरित्यज्य गार्हस्थ्यं कुरुराजर्षिसत्तम}
{कः प्राप्तो भूपतिः सिद्धिं मोक्षतत्त्वं वदस्व मे}


\threelineshloka
{संन्यस्यते यथाऽत्माऽयं व्यक्तस्यात्मा यथा च यत्}
{परं मोक्षस्य यच्चापि तन्मे ब्रूहि पितामह ॥भीष्म उवाच}
{}


\twolineshloka
{अत्राप्युदाहरन्तीममितिहासं पुरातनम्}
{जनकस्य च संवादं सुलभायाश्च भारत}


\twolineshloka
{संन्यासफलिकः कश्चिद्बभूव नृपतिः पुरा}
{मैथिलो जनको नाम धर्मध्वज इति श्रुतः}


\twolineshloka
{स वेदे मोक्षशास्त्रे च स्वे च शास्त्रे कृतश्रमः}
{इन्द्रियाणि समाधाय शशास वसुधामिमाम्}


\twolineshloka
{तस्य वेदविदः प्राज्ञाः श्रुत्वा तां साधुवृत्तताम्}
{लोकेषु स्पृहयन्त्यन्ये पुरुषाः पुरुषेश्वरम्}


\twolineshloka
{अथ धर्मयुगे तस्मिन्योगधर्ममनुष्ठिता}
{महीमनुचचारैका सुलभा नाम भिक्षुकी}


\twolineshloka
{तया जगदिदं कृत्स्नमटन्त्या मिथिलेश्वरः}
{तत्रतत्र श्रुतो मोक्षे कथ्यमानस्त्रिदण्डिभिः}


\twolineshloka
{साऽतिसूक्ष्मां कथां श्रुत्वा तथ्यं नेति ससंशया}
{दर्शने जातसंकल्पा जनकस्य बभूव ह}


\twolineshloka
{तत्र सा विप्रहायाऽथ पूर्वरूपं हि योगतः}
{अबिभ्रदनवद्याङ्गी रूपमन्यदनुत्तमम्}


\twolineshloka
{चक्षुनिंमेषमात्रेण लध्वस्त्रगतिगामिनी}
{विदेहानां पुरीं सुभ्रूर्जगाम कमलेक्षणा}


\twolineshloka
{सा प्राप्य मिथिलां रम्यां समृद्धजनसंकुलाम्}
{भैक्ष्यचर्यापदेशेन ददर्श मिथिलेश्वरम्}


\twolineshloka
{राजा तस्याः परं दृष्ट्वा सौकुमार्यं पुनस्तदा}
{केयं कस्य कुतो वेति बभूवागतविस्मयः}


\twolineshloka
{ततोस्याः स्वागतं कृत्वा व्यादिश्य च वरासनम्}
{पूजितां पादशौचेन वरान्नेनाप्यतर्पयत्}


\twolineshloka
{अथ भुक्तवतीं प्रीतां राजा तां मन्त्रिभिर्वृतः}
{सर्वभाष्यविदां मध्ये चोदयामास भिक्षुकीम्}


\twolineshloka
{सुलभा त्वस्य धर्मेषु मुक्तो नेति ससंशया}
{सत्वं सत्वेन योगज्ञा प्रविष्टाऽभून्महीपते}


\twolineshloka
{नेत्राभ्यां नेत्रयोरस्य रश्मीन्संयोज्य रश्मिभिः}
{सा स्म तं चोदयिष्यन्ती योगबन्धैर्बबन्ध ह}


\twolineshloka
{जनकोप्युत्स्मयन्राजा भावमस्या विशेषयन्}
{प्रतिजग्राह भावेन भावमस्या नृपोत्तम}


\threelineshloka
{तदेकस्मिन्नधिष्ठाने संवादः श्रूतयामयम्}
{छत्रादिषु विमुक्तस्य मुक्तायाश्च त्रिदण्डकैः ॥जनक उवाच}
{}


\twolineshloka
{भगवत्याः क्व चर्येयं कृता क्व च गमिष्यसि}
{कस्य च त्वं कुतो वेति पप्रच्छैनां महिपतिः}


\twolineshloka
{श्रुते वयसि जातौ च सद्भावो नाधिगम्यते}
{एष्वर्थेषूत्तरं तस्मात्प्रवेद्यं मत्समागमे}


\twolineshloka
{छत्रादिषु विशेषेषु मुक्तं मां विद्धि तत्त्वतः}
{सत्वां संवेत्तुमिच्छामि मानार्हा हि मताऽसिमे}


\twolineshloka
{यस्माच्चैतन्मया प्राप्तं ज्ञानं वैशेषिकं पुरा}
{यस्य नान्यः प्रवक्ताऽस्ति मोक्षे तमपि मे शृणु}


\twolineshloka
{पराशरसगोत्रस्य वृद्धस्य सुमहात्मनः}
{भिक्षोः पञ्चशिखस्याहं शिष्यः परमसंमतः}


\twolineshloka
{साङ्ख्यज्ञाने च योगे च महीपालविधौ तथा}
{त्रिविधे मोक्षधर्मोस्मिन्गताध्वा छिन्नसंशयः}


\twolineshloka
{स यथा शास्त्रदृष्टेन मार्गेणेह परिभ्रमन्}
{वार्षिकांश्चतुरो मासान्पुरा मयि सुखोषितः}


\twolineshloka
{तेनाहं साङ्ख्यमुख्येन सुदृष्टार्थेन तत्त्वतः}
{श्रावितस्त्रिविधं मोक्षं न च राज्याद्धि चालितः}


\twolineshloka
{सोहं तामखिलां वृत्तिं त्रिविधां मोक्षसंहिताम्}
{मुक्तरागश्चराम्येकः पदे परमके स्थितः}


\twolineshloka
{वैराग्यं पुनरेतस्य मोक्षस्य परमो विधिः}
{ज्ञानादेव च वैराग्यं जायते येन मुच्यते}


\twolineshloka
{ज्ञानेन कुरुते यत्नं यत्नेन प्राप्यते महत्}
{महद्द्वन्द्वप्रमोक्षाय सा सिद्धिर्या वयोतिगा}


\twolineshloka
{सेयं परमिका सिद्धिः प्राप्ता निर्द्वन्द्वता मया}
{इहैव गतमोहेन चरता मुक्तसङ्गिना}


\twolineshloka
{यथा क्षेत्रं मृदुभूतमद्भिराप्लावितं तथा}
{जनयत्यङ्कुरं कर्म नृणां तद्वत्पुनर्भवम्}


\twolineshloka
{यथा चोत्तापितं बीजं कपाले यत्रतत्र वा}
{प्राप्याप्यङ्कुरहेतुत्वमबीजत्वान्न जायते}


\twolineshloka
{तद्वद्भगवताऽनेन शिखाप्रोक्तेन भिक्षुणा}
{ज्ञानं कृतमबीजं मे विषयेषु न जायते}


\twolineshloka
{नामिरज्यति कस्मिंश्चिन्नानर्थे न परिग्रहे}
{नाभिरज्यति चैतेषु व्यर्थत्वाद्रागरोषयोः}


\twolineshloka
{यश्च मे दक्षिणं बाहुं चन्दनेन समुक्षयेत्}
{सव्यं वाऽस्यापि यस्तक्षेत्समावेतावुभौ मम}


\twolineshloka
{सुखी सोऽहमवाप्तार्थः समलोष्टाश्मकाञ्चनः}
{मुक्तसङ्गः स्थितो राज्ये विशिष्टोऽन्यैस्त्रिदण्डिभिः}


\twolineshloka
{मोक्षे हि त्रिविधा निष्ठा दृष्टाऽन्यैर्मोक्षवित्तमैः}
{ज्ञानं लोकोत्तरं यच्च सर्वत्यागश्च कर्मणाम्}


\twolineshloka
{ज्ञाननिष्ठां वदन्त्येके मोक्षशास्त्रविदो जनाः}
{कर्मनिष्ठां तथैवान्ये यतयः सूक्ष्मदर्शिनः}


\twolineshloka
{प्रहायोभयमप्येव ज्ञानं कर्म च केवलम्}
{तृतीयेयं समाख्याता निष्ठा तेन महात्मना}


\twolineshloka
{यमे च नियमे चैव कामे द्वेषे परिग्रहे}
{माने दम्भे तथा स्नेहे सदृशास्ते कुटुम्बिभिः}


\twolineshloka
{त्रिदण्डादिषु यद्यस्ति मोक्षो ज्ञानेन केनचित्}
{छत्रादिषु कथं न स्यात्तुल्यहेतौ परिग्रहे}


\twolineshloka
{येनयेन हि यस्यार्थः कारणेनेह कर्मणि}
{तत्तदालम्बते सर्वद्रव्ये स्वार्थपरिग्रहे}


\twolineshloka
{दोषदर्शी तु गार्हस्थ्ये यो व्रजत्याश्रमान्तरे}
{उत्सृजन्परिगृह्णंश्च सोऽपि सङ्गान्न मुच्यते}


\twolineshloka
{आधिपत्ये तथा तुल्ये निग्रहानुग्रहात्मके}
{राजभिर्भिक्षुकास्तुल्या मुच्यन्ते केन हेतुना}


\twolineshloka
{अथ सत्याधिपत्येऽपि ज्ञानेनैवेह केवलम्}
{मुच्यन्ते किं न मुच्यन्ते पदे परमके स्थिताः}


\twolineshloka
{काषायधारणं मौण्ड्यं त्रिविष्टब्धं कमण्डलुम्}
{लिङ्गान्युत्पथभूतानि न मोक्षायेति मे मतिः}


\twolineshloka
{यदि सत्यपि लिङ्गेऽस्मिञ्ज्ञानमेवात्र कारणम्}
{निर्मोक्षायेह दुःखस्य लिङ्गमात्रं निरर्थकम्}


\twolineshloka
{अथवा दुःखशैथिल्यं वीक्ष्य लिङ्गे कृता मतिः}
{किं तदेवार्थसामान्यं छत्रादिषु न लक्ष्यते}


\twolineshloka
{आकिञ्चन्ये न मोक्षोस्ति कैञ्चन्ये नास्ति बन्धनम्}
{कैञ्चन्ये चेतरे चैव जन्तुर्ज्ञानेन मुच्यते}


\twolineshloka
{तस्माद्धर्मार्थकामेषु तथा राज्यपरिग्रहे}
{बन्धनायतनेष्वेषु विद्ध्यबन्धे पदे स्थितम्}


\twolineshloka
{राज्यैश्वर्यमयः पाशः स्नेहायतनबन्धनः}
{मोक्षाश्मनिशितेनेह च्छिन्नस्त्यागासिना मया}


\twolineshloka
{सोहमेवं गतो मुक्तो जातास्थस्त्वयि भिक्षुकि}
{अयथार्थं हि ते वर्णं वक्ष्यामि शृणु तन्मम}


\twolineshloka
{सौकुमार्यं तथा रूपं वपुरग्र्यं तथा वयः}
{तवैतानि समस्तानि नियमश्चेति संशयः}


\twolineshloka
{यच्चाप्यननुरूपं ते लिङ्गस्यास्य विचेष्टितम्}
{मुक्तोऽयं स्यान्न वेति स्याद्धर्षितो मत्परिग्रहः}


\twolineshloka
{न च कामसमायुक्ते युक्तेऽप्यस्ति त्रिदण्डके}
{न रक्ष्यते त्वया चेदं न मुक्तस्यास्ति गोपना}


\twolineshloka
{मत्पक्षसंश्रयाच्चायं शृणु यस्ते व्यतिक्रमः}
{आश्रयन्त्याः स्वभावेन मम पूर्वपरिग्रहम्}


\twolineshloka
{प्रवेशस्ते कृतः केन मम राष्ट्रे पुरेपि वा}
{कस्य वा सन्निकर्षात्त्वं प्रविष्टा हृदयं मम}


\twolineshloka
{वर्णप्रवरमुख्याऽसि ब्राह्मणी क्षत्रियस्त्वहम्}
{नावयोरेकयोगोऽस्ति मा कृथा वर्णसंकरम्}


\twolineshloka
{वर्तसे मोक्षधर्मेण त्वं गार्हस्थ्येऽहमाश्रमे}
{अयं चापि सुकष्टस्ते द्वितीयाश्रमसंकरः}


\twolineshloka
{सगोत्रां वाऽसगोत्रां वा न वेद त्वां न वेत्थ माम्}
{सगोत्रमाविशन्त्यास्ते तृतीयो गोत्रसंकरः}


\twolineshloka
{अथ जीवति ते भर्ता प्रोषितोप्यथवा क्वचित्}
{अगम्या परभार्येति चतुर्थो धर्मसंकरः}


\twolineshloka
{सा त्वमेतान्यकार्याणि कार्यापेक्षा व्यवस्यसि}
{अविज्ञानेन वा युक्ता मिथ्याज्ञानेन वा पुनः}


\twolineshloka
{अथवापि स्वतन्त्राऽसि स्वदोषेणेह केनचित्}
{यदि किंचिच्छ्रुतं तेऽस्ति सर्वं कृतमनर्थकम्}


\twolineshloka
{इदमन्यत्तृतीयं ये भावस्पर्शविघातकम्}
{दुष्टायाऽलक्ष्यते लिङ्गं विवृण्वत्या प्रकाशितम्}


\twolineshloka
{न मय्येवाभिसन्धिस्ते जयैषिण्या जये कृतः}
{येयं मत्परिषत्कृत्स्ना चेतुमिच्छसि तामपि}


\twolineshloka
{तथाऽर्हतस्ततश्च त्वं दृष्टिं स्वां प्रतिमुञ्चसि}
{मत्पक्षप्रतिघाताय स्वपक्षोद्भवनाय च}


\twolineshloka
{सा स्वेनामर्षजेन त्वमृद्धिमोहेन मोहिता}
{भूयः सृजसि योगांस्त्वं विषामृतमिवैकताम्}


\twolineshloka
{इच्छतोरत्र यो लाभः स्त्रीपुंसोरमृतोपमः}
{अलाभश्चापि रक्तस्य सोपि दोषो विषोषमः}


\twolineshloka
{मा त्याक्षीः साधु जानीष्व स्वशास्त्रमनुपालय}
{कृतेयं हि विजिज्ञासा मुक्तो नेति त्वया मम}


\threelineshloka
{एतत्सर्वं प्रतिच्छन्नं मयि नार्हसि गूहितम्}
{सा यदि त्वं स्वकार्येण यद्यन्यस्य महीपतेः}
{तत्त्वमत्र प्रतिच्छन्ना मयि नार्हसि गूहितम्}


\twolineshloka
{न राजानं मृषा गच्छेन्न द्विजातिं कथंचन}
{न स्त्रियं स्त्रीगुणोपेतां हन्युर्ह्येते मृषागताः}


\twolineshloka
{राज्ञां हि बलमैश्वर्यं ब्रह्म ब्रह्मविदां बलम्}
{रूपयौवनसौभाग्यं स्त्रीणां बलमनुत्तमम्}


\twolineshloka
{अत एतैर्बलैरेव बलिनः स्वार्थमिच्छता}
{आर्जवेनाभिगन्तव्या विनाशाय ह्यनार्जवम्}


\threelineshloka
{सा त्वं जातिं श्रुतं वृत्तं भावं प्रकृतिमात्मनः}
{कृत्यमागमने चैव वक्तुमर्हसि तत्त्वतः ॥भीष्म उवाच}
{}


\twolineshloka
{इत्येतैरसुखैर्वाक्यैरयुक्तैरसमञ्जसैः}
{प्रत्यादिष्टा नरेन्द्रेण सुलभा न व्यकम्पत}


\threelineshloka
{उक्तवाक्ये तु नृपतौ सुलभा चारुदर्शना}
{ततश्चारुतरं वाक्यं प्रचक्रामाथ भाषितुम् ॥सुलभोवाच}
{}


\twolineshloka
{नवभिर्नवभिश्चैव दोषैर्वाग्बुद्धिदूषणैः}
{अपेतमुपपन्नार्थमष्टादशगुणान्वितम्}


\twolineshloka
{सौक्ष्म्यं साङ्ख्यक्रमौ चोभौ निर्णयः सप्रयोजनः}
{पञ्चैतान्यर्थजातानि वाक्यमित्युच्यते नृप}


\twolineshloka
{एषामेकैकशोऽर्थानां सौक्ष्म्यादीनां स्वलक्षणम्}
{शृणु संसार्यमाणानां पदार्थपदवाक्यतः}


\twolineshloka
{ज्ञानं ज्ञेयेषु भिन्नेषु यदा भेदेन वर्तते}
{तत्रातिशयिनी बुद्धिस्तत्सौक्ष्म्यमिति वर्तते}


\twolineshloka
{दोषाणां च गुणानां च प्रमाणं प्रविभागतः}
{कंचिदर्थमभिप्रेत्य सा सङ्ख्येत्युपधार्यताम्}


\twolineshloka
{इदं पूर्वमिदं पश्चाद्वक्तव्यं यद्विवक्षितम्}
{क्रमयोगं तमप्याहुर्वाक्यं वाक्यविदो जनाः}


\twolineshloka
{धर्मकामार्थमोक्षेषु प्रतिज्ञाय विशेषतः}
{इदं तदिति वाक्यान्ते प्रोच्यते स विनिर्णयः}


\twolineshloka
{इच्छाद्वेषभवैर्दुःखैः प्रकर्षो यत्र जायते}
{तत्र या नृपते वृत्तिस्तत्प्रयोजनमिष्यते}


\twolineshloka
{तान्येतानि यथोक्तानि सौक्ष्म्यादीनि जनाधिप}
{एकार्थसमवेतानि वाक्यं मम निशामय}


\twolineshloka
{उपेतार्थमभिन्नार्थं न्यायवृत्तं न चाधिकम्}
{नाश्लक्ष्णं न च संदिग्धं वक्ष्यामि परमं ततः}


\twolineshloka
{न गुर्वक्षरसंयुक्तं पराङ्युखपदं न च}
{नानृतं न त्रिवर्गेण विरुद्धं नाप्यसंस्कृतम्}


\twolineshloka
{न न्यूनं नष्टशब्दं वा व्युत्क्रमाभिहितं न च}
{सदोषमभिकल्पेन निष्कारणमहेतुकम्}


\twolineshloka
{कामात्क्रोधाद्भयाल्लोभाद्दैन्याच्चानार्थकात्तथा}
{ह्रीतोऽनुक्रोशतो मानान्न वक्ष्यामि कथंचन}


\twolineshloka
{वक्ता श्रोता च वाक्यं च यदा त्वविकलं नृप}
{स ममेति विवक्षायां तदा सोर्थः प्रकाशते}


\twolineshloka
{वक्तव्ये तु यदा वक्ता श्रोतारमवमन्यते}
{स्वार्थमाह परार्थं तत्तदा वाक्यं न रोहति}


\twolineshloka
{अथ यः स्वार्थमुत्सृज्य परार्थं प्राह मानवः}
{विशङ्का जायते तस्मिन्वाक्यं तदपि दोषवत्}


\twolineshloka
{यस्तु वक्ता द्वयोरर्थमविरुद्धं प्रभाषते}
{श्रोतुश्चैवात्मनश्चैव स वक्ता नेतरो नृप}


\twolineshloka
{तदर्थवदिदं वाक्यमुपेतं वाक्यसंपदा}
{`अविक्षिप्तमना राजन्नेकाग्रः श्रोतुमर्हसि}


\twolineshloka
{काऽसि कस्य कुतो वेति त्वयाऽहमिति चोदिता}
{तत्रोत्तरमिदं वाक्यं राजन्नेकमनाः शृणु ॥'}


\twolineshloka
{यथा जतु च काष्ठं च पांसवश्चोदबिन्दवः}
{सुश्लिष्टानि तथा राजन्प्राणिनामिह संभवः}


\threelineshloka
{शब्दः स्पर्शो रसो रूपं गन्धः पञ्चेन्द्रियाणि च}
{पृथगात्मा दशात्मानं संश्लिष्टा जतुकाष्ठवत्}
{न चैषां चोदना काचिदस्तीत्येष विनिश्चयः}


\twolineshloka
{एकैकस्येह विज्ञानं नास्त्यात्मनि तथा परे}
{न वेद चक्षुश्चक्षुष्ट्वं श्रोत्रं नात्मनि वर्तते}


\twolineshloka
{तथैव व्यभिचारेण न वर्तन्ते परस्परम्}
{प्रश्लिष्टं च न जानन्ति यथाऽऽप इव पांवसः}


\twolineshloka
{वाह्यानन्यानपेक्षन्ते गुणांस्तानपि मे शृणु}
{रूपं चक्षुः प्रकाशश्च दर्शने हेतवस्त्रयः}


\twolineshloka
{यथैवात्र तथाऽन्येषु ज्ञानज्ञेयेषु हेतवः}
{ज्ञानज्ञेयांतरेतस्मिन्मनो नामापरो गुणः}


\threelineshloka
{विचारयति येनायं निश्चये साध्वसाधुनी}
{द्वादशस्त्वपरस्तत्र बुद्धिर्नाम गुणः स्मृतः}
{येंन संशयपूर्वेषु बोद्धव्येषु व्यवस्यति}


\twolineshloka
{अथ द्वादशके तस्मिन्सत्वं नामापरो गुणः}
{महासत्वोऽल्पसत्वो वा जन्तुयेनानुमीयते}


\twolineshloka
{क्षेत्रज्ञ इति चाप्यन्यो गुणस्तत चतुर्दशः}
{ममायमिति येनायं मन्यते न ममेति च}


\twolineshloka
{अथ पञ्चदशो राजन्गुणस्तत्रापरः स्मृतः}
{पृथक्कालसमूहस्य सामग्र्यं तदिहोच्यते}


\twolineshloka
{गुणस्त्वेकोऽपरस्तत्र संघात इति षोडशः}
{आकृतिर्व्यक्तिरित्येतौ गुणो यस्मिन्समाश्रितौ}


\twolineshloka
{सुखासुखे जन्ममृत्यू लाभालाभौ प्रियाप्रिये}
{इति चैकोनर्विशोऽयं द्वन्द्वयोग इति स्मृतः}


\twolineshloka
{ऊर्ध्वमेकोनर्विशत्या कालो नामापरो गुणः}
{इतीमं विद्धि विंशत्या भूतानां प्रभवाप्ययम्}


\twolineshloka
{विंशकश्चैव संघातो महाभूतानि पञ्च च}
{सदसद्भवायोगौ तु गुणावन्यौ प्रकाशकौ}


\twolineshloka
{इत्येवं विंशतिश्चैव गुणाः सप्त च ये स्मृताः}
{विधिः शुक्रं बलं चेति त्रय एते गुणाः परे}


\twolineshloka
{एवं विंशच्च दश च कलाः संख्यानतः स्मृताः}
{समग्रा यत्र वर्तन्ते तच्छरीरमिति स्मृतम्}


\twolineshloka
{अव्यक्तं प्रकृतिं त्वासां कलानां कश्चिदिच्छति}
{व्यक्तं चासां तथैवान्यः स्थूलदर्शी प्रपश्यति}


\twolineshloka
{अव्यक्तं यदि वा व्यक्तं द्वयं वाऽथ चतुष्टयम्}
{प्रकृतिं सर्वभूतानां पश्यन्त्यध्यात्मचिन्तकाः}


\threelineshloka
{सेयं प्रकृतिरव्यक्ता कलाभिर्व्यक्ततां गता}
{ततोऽहं त्वं च राजेन्द्र चे चाप्यन्ते शरीरिणः}
{}


\twolineshloka
{विन्दुन्यासादयोऽवस्थाः शुक्रशोणितसंभवाः}
{यासामेव निपातेन कललं नाम जायते}


\twolineshloka
{कललाद्बुद्बुदोत्पत्तिः पेशी वा बुद्बुदात्स्मृता}
{पेश्यास्त्वङ्गाभिनिर्वृत्तिर्नखरोमाणि चाङ्गतः}


\twolineshloka
{संपूर्णे नवमे मासि जन्तोर्जातस्य मैथिल}
{जायते नामरूपत्वं स्त्रीपुमान्वेति लिङ्गतः}


\twolineshloka
{जातमात्रं तु तद्रूपं दृष्ट्वा ताम्रनखाङ्गुलि}
{कौमारं रूपमापन्नं रूपवानुपलभ्यते}


\twolineshloka
{कौमाराद्यौवनं चापि स्थाविर्यं चापि यौवनात्}
{अनेन क्रमयोगेन पूर्वं पूर्वं न लभ्यते}


\twolineshloka
{कलानां पृथगर्थानां प्रतिभेदः क्षणे क्षणे}
{वर्तते सर्वभूतेषु सौक्ष्म्यात्तु न विभाव्यते}


\twolineshloka
{न चैषामप्ययो राजँल्लक्ष्यते प्रभवो न च}
{अवस्थायामवस्थायां दीपस्येवार्चिषो गतिः}


\twolineshloka
{तस्याप्येवंप्रक्तारस्य सदश्वस्येव धावतः}
{अजस्रं सर्वलोकस्य कः कुतो वा न वा कुतः}


\twolineshloka
{कस्येदं कस्य वा नेदं कुतो वेदं न वा कुतः}
{संबन्धः कोऽस्ति भूतानां स्वैरप्यवयवैरिह}


\twolineshloka
{यथाऽऽदित्यान्मणेश्चापि वीरुद्भ्यश्चैव पावकः}
{जायन्त्येवं समुदयात्कलानामिह जन्तवः}


\twolineshloka
{आत्मन्येवात्मनाऽऽत्मानं यथा त्वमनुपश्यसि}
{एवमेवात्मनाऽऽत्मानमन्यस्मिन्किं न पश्यसि}


\twolineshloka
{यद्यात्मनि परस्मिंश्च समतामध्यवस्यसि}
{अथ मां काऽसि कस्येति किमर्थमनुपृच्छसि}


\twolineshloka
{इदं मे स्यादिदं नेति द्वन्द्वैर्मुक्तस्य मैथिल}
{काऽसि कस्य कुतो वेति वचनैः किं प्रयोजनम्}


\twolineshloka
{रिपौ मित्रेऽथ मध्यस्थे विजये सन्धिविग्रहे}
{कृतवान्यो महीपालः किं तस्मिन्मुक्तलक्षणम्}


\twolineshloka
{त्रिवर्गं सप्तधा न्यस्तं यो न वेदेह कर्मसु}
{सङ्गवान्यस्त्रिवर्गे च किं तस्मिन्मुक्तलक्षणम्}


\twolineshloka
{प्रिये वाऽप्यप्रिये वाऽपि दुर्बले बलवत्यपि}
{यस्य नास्ति समं चक्षुः किं तस्मिन्मक्तलक्षणम्}


\twolineshloka
{तदयुक्तस्य ते मोक्षे योऽभिमानो भवेन्नृप}
{सुहृद्भिः सन्निवार्यस्तेऽविरक्तस्येव भेषजम्}


\twolineshloka
{तानि तान्यनुसंदृश्य सङ्गस्थानान्यरिंदम्}
{आत्मनाऽऽत्मनि संपश्य किमन्यन्मुक्तलक्षणम्}


\twolineshloka
{इमान्यन्यानि सूक्ष्माणि मोक्षमाश्रित्य केनचित्}
{चतुरङ्गप्रवृत्तानि सङ्गस्थानानि मे शृणु}


\twolineshloka
{य इमां पृथिवीं कृत्स्नामेकच्छत्रां प्रशास्ति ह}
{एकमेव स वै राजा पुरमध्यावसत्युत}


\twolineshloka
{तत्पुरे चैकमेवास्य गृहं यदधितिष्ठति}
{गृहे शयनमप्येकं निशायां यत्र लीयते}


\twolineshloka
{शय्यार्धं तस्य चाप्यत्र स्त्रीपूर्वमधितिष्ठति}
{तदनेन प्रसङ्गेन फलेनाल्पेन युज्यते}


\twolineshloka
{एवमेवोपभोगेषु भोजनाच्छादनेषु च}
{गुणेष्वपरिमेयेषु निग्रहानुग्रहं प्रति}


\twolineshloka
{परतन्त्रः सदा राजा स्वल्पे सोऽपि प्रसज्जते}
{सन्धिविग्रहयोगे च कुतो राज्ञः स्वतन्त्रता}


\twolineshloka
{स्त्रीषु क्रीडाविहारेषु नित्यमस्यास्वतन्त्रता}
{मन्त्रे चामात्यसहिते कुतस्तस्य स्वतन्त्रता}


\twolineshloka
{यदा ह्याज्ञापयत्यन्यांस्तत्रास्योक्ता स्वतन्त्रता}
{अवशः कार्यते तत्र तस्मिंस्तस्मिन्गुणे स्थितः}


\twolineshloka
{स्वप्नकामो न लभते स्वप्तुं कार्यार्थिभिर्जनैः}
{शयने चाप्यनुज्ञातः सुप्त उत्थाप्यतेऽवशः}


\twolineshloka
{स्नाह्यालभ पिब प्राश जुहुध्यगीन्यजेत्यपि}
{वदस्व शृणु चापीति विवशः कार्यते परैः}


\twolineshloka
{अभिगम्याभिगम्यैवं याचन्ते सततं नराः}
{न चाप्युत्सहते दातुं वित्तरक्षी महाजनान्}


\twolineshloka
{दाने कोशक्षयोऽप्यस्य वैरं चास्य प्रयच्छतः}
{क्षणेनास्योऽपवर्तन्ते दोषा वैराग्यकारकाः}


\twolineshloka
{प्राज्ञाञ्शूरांस्तथा वैद्यानेकस्थानपि शङ्कते}
{भयमप्यनये राज्ञो यैश्च नित्यमुपास्यते}


\twolineshloka
{तथा चैते प्रदुष्यन्ति राजन्ये कीर्तिता मया}
{तथैवास्य भयं तेभ्यो जायते पश्य यादृशम्}


\twolineshloka
{सर्वः स्वेस्वे गृहे राजा सर्वः स्वेस्वे गृहे गृही}
{निग्रहानुग्रहौ कुर्वंस्तुल्यो जनक राजभिः}


\twolineshloka
{पुत्रा दारास्तथैवात्मा कोशो मित्राणि संचयाः}
{परैः साधारणा ह्येते तैस्तैरेवास्य हेतुभिः}


\twolineshloka
{हतो देशः पुरं दग्धं प्रधानः कुञ्जरो मृतः}
{लोकसाधारणेष्वेषु मिथ्याज्ञानेन तप्यते}


\twolineshloka
{अमुक्तो मानसैर्दुःखैरिच्छाद्वेषप्रियोद्भवैः}
{शिरोरोगादिभी रोगैस्तथैव विनिपातिभिः}


\twolineshloka
{द्वन्द्वैस्तैस्तैरुपहतः सर्वतः परिशङ्कितः}
{बहुभिः प्रार्थितं राज्यमुपास्ते गणयन्निशाः}


\twolineshloka
{तदल्पसुखमत्यर्थं बहुदुःखमसारवत्}
{तृणाग्निज्वलनप्रख्यं फेनबुद्बुदसंनिभम्}


\twolineshloka
{को राज्यमभिपद्येत प्राप्य चोपशमं लभेत्}
{ममेदमिति यच्चेदं पुरं राष्ट्रं च मन्यसे}


\twolineshloka
{बलं कोशममात्यांश्च कस्यैतानि न वा नृप}
{मित्रामात्यपुरं राष्ट्रं दण्डः कोशो महीपतिः}


\threelineshloka
{`सप्ताङ्गश्चैष सङ्घातो राज्यमित्युच्यते नृप}
{'सप्ताङ्गस्यास्य राज्यस्य त्रिदण्डस्येव तिष्ठतः}
{अन्योन्यगुणयुक्तस्य कः केन गुणतोऽधिकः}


\twolineshloka
{तेषुतेषु हि कालेषु तत्तदङ्गं विशिष्यते}
{येन यत्सिध्यते कार्यं तत्प्राधान्याय कल्पते}


\twolineshloka
{सप्ताङ्गश्चैव संघातस्त्रयश्चान्ये नृपोत्तम}
{संभूय दशवर्गोऽयं भुङ््क्ते राज्यं हि राजवत्}


\twolineshloka
{यश्च राजा महोत्साहः क्षत्रधर्मे रतो भवेत्}
{स तुष्येद्दशभागेन ततस्त्वन्यो दशावरैः}


\twolineshloka
{नास्त्यसाधारणो राजा नास्ति राज्यमराजकम्}
{राज्येऽसति कुतो धर्मो धर्मेऽसति कुतः परम्}


\twolineshloka
{योप्यत्र परमो धर्मः पवित्रं राजराज्ययोः}
{पृथिवी दक्षिणा यस्य सोऽश्वमेधो न विद्यते}


\twolineshloka
{साऽहमेतानि कर्माणि राजदुःखानि मैथिल}
{समर्था शतशो वक्तुमथवाऽपि सहस्रशः}


\twolineshloka
{स्वदेहे नाभिषङ्गो मे कुतः परपरिग्रहे}
{न मामेवंविधां युक्तामीदृशं वक्तुमर्हसि}


\twolineshloka
{ननु नाम त्वया मोक्षः कृत्स्नः पञ्चशिखाच्छ्रुतः}
{सोपायः सोपनिषदः सोपसङ्गः सनिश्चयः}


\twolineshloka
{तस्य ते मुक्तसङ्गस्य पाशानाक्रम्य तिष्ठतः}
{छत्रादिषु विशेषेषु पुनः सङ्गः कथं नृप}


\twolineshloka
{श्रुतं ते न श्रुतं मन्ये मृषा वाऽपि श्रुतं श्रुतम्}
{अथवा श्रुतसंकाशं श्रुतमन्यच्छ्रुतं त्वया}


\twolineshloka
{अथापीमासु संज्ञासु लौकिकीषु प्रतिष्ठसे}
{अभिषङ्गावरोधाभ्यां बद्धस्त्वं प्राकृतो यथा}


\twolineshloka
{सत्वेनानुप्रवेशो हि योऽयं त्वयि कृतो मया}
{किं तवापकृतं तत्र यदि मुक्तोऽसि सर्वशः}


\twolineshloka
{नियमो ह्येषु धर्मेषु यतीनां शून्यवासिता}
{शून्यमावासयन्त्या च मया किं कस्य दूषितम्}


\twolineshloka
{न पाणिभ्यां न बाहुभ्यां पादोरुभ्यां न चानघ}
{न गात्राक्यवैरन्यैः स्पृशामि त्वां नराधिप}


\twolineshloka
{कुले महति जातेन ह्रीमता दीर्घदर्शिना}
{नैतत्सदसि वक्तव्यं सद्वाऽसद्वा मिथः कृतम्}


\twolineshloka
{ब्राह्मणा गुरवश्चेमे तथा मान्या गुरूत्तमाः}
{त्वं चाथ गुरुरप्येषामेवमन्योन्यगौरवम्}


\twolineshloka
{तदेवमनुसंदृश्य वाच्यावाच्यं परीक्षता}
{स्त्रीपुंसोः समवायोऽयं त्वया वाच्यो न संसदि}


\twolineshloka
{यथा पुष्करपर्णस्थं जलं तत्पर्णसंस्थितम्}
{तिष्ठत्यस्पृशती तद्वत्त्वपि वत्स्यामि मैथिल}


\twolineshloka
{यदि चाद्य स्पृशन्त्या मे स्पर्शं जानासि कंचन}
{ज्ञानं कृतमबीजं ते कथं तेनेह भिक्षुणा}


\twolineshloka
{स गार्हस्थ्याच्च्युतश्च त्वं मोक्षं चानाप्य दुर्विदम्}
{उभयोरन्तराले वै वर्तसे मोक्षवादिकः}


\twolineshloka
{न हि मुक्तस्य मुक्तेन ज्ञस्यैकत्वपृथक्त्वयोः}
{भावाभावसमायोगे जायते वर्णसंकरः}


\twolineshloka
{वर्णाश्रमाः पृथक्त्वेन दृष्टार्थस्यापृथक्त्वतः}
{नान्यदन्यदिति ज्ञात्वा नान्यदन्यत्र वर्तते}


\twolineshloka
{पापौ कुण्डं तथा कुण्डे पयः पयसि मक्षिका}
{आश्रिताश्रययोगेन पृथक्त्वेनाश्रिताः पुनः}


\twolineshloka
{न तु कुण्डे पयोभावः पयश्चापि न मक्षिका}
{स्वयमेवाश्रयन्त्येते भावा न तु पराश्रयम्}


\twolineshloka
{पृथक्त्वादाश्रमाणां च वर्णान्यत्वे तथैव च}
{परस्परपृथक्त्वाच्च कथं ते वर्णसंकरः}


\twolineshloka
{नास्मि पर्णोत्तमा जात्या न वैश्या नावरा तथा}
{तव राजन्सवर्णाऽस्मि शुद्धयोनिरविप्लुता}


\twolineshloka
{प्रधानो नाम राजर्षिर्व्यक्तं ते श्रोत्रमागतः}
{कुले तस्य समुत्पन्नां सुलभां नाम विद्धि माम्}


\twolineshloka
{द्रोणश्च शतशृङ्गश्च वक्रद्धारश्च पर्वतः}
{मम सत्रेषु पूर्वेषां चिता मघवता सह}


\twolineshloka
{साहं तस्मिन्कुले जाता भर्तर्यसति मद्विधे}
{विनीता मोक्षधर्मेषु चराम्येका मुनिव्रतम्}


\twolineshloka
{नास्मि सत्रप्रतिच्छन्ना न परस्वाभिमानिनी}
{न धर्मसंकरकरी स्वधर्मेऽस्मि दृढव्रता}


\twolineshloka
{नास्थिरा स्वप्रतिज्ञायां नासमीक्ष्य प्रवादिनी}
{नासमीक्ष्यागता चेह त्वत्सकाशं जनाधिप}


\twolineshloka
{मोक्षे ते भावितां बुद्धिं श्रुत्वाऽहं कुशलैषिणी}
{तव मोक्षस्य चाप्यस्य जिज्ञासार्थमिहागता}


\twolineshloka
{न वर्गस्था ब्रवीम्येतत्स्वपक्षपरपक्षयोः}
{मुक्तो विमुच्यते यश्च शान्तौ यश्च न शाम्यति}


\twolineshloka
{यथा शून्ये पुराऽगारे भिक्षुरेकां निशां वसेत्}
{तथाऽहं त्वच्छरीरेऽस्मिन्निमां वत्स्यामि शर्वरीम्}


\threelineshloka
{साऽहं मानप्रदानेन वागातिथ्येन चार्चिता}
{सुप्ता सुशरणं प्रीता श्वो गमिष्यामि मैथिल ॥भीष्म उवाच}
{}


\twolineshloka
{इत्येतानि स वाक्यानि हेतुमन्त्यर्थवन्ति च}
{श्रुत्वा नाधिजगौ राजा किंचिदन्यदतः परं}


\chapter{अध्यायः ३२६}
\twolineshloka
{* युधिष्ठिर उवाच}
{}


\threelineshloka
{अव्यक्तव्यक्ततत्वानां निश्चयं भरतर्षभ}
{वक्तुमर्हसि कौरव्य देवस्याजस्य या कृतिः ॥भीष्म उवाच}
{}


\threelineshloka
{अत्राप्युदाहरन्तीमं संवादं गुरुशिष्ययोः}
{कपिलस्यासुरेश्चैव सर्वदुःखविमोक्षणम् ॥असुरिरुवाच}
{}


\twolineshloka
{अव्यक्तव्यक्ततत्वानां निश्चयं बुद्धिनिश्चयम्}
{भगवन्नमितप्रज्ञ वक्तुमर्हसि मेऽर्थितः}


\twolineshloka
{किं व्यक्तं किमव्यक्तं किं व्यक्ताव्यक्तं किमिति तत्वानि}
{किमाद्यं मध्यमं च तत्वानांकिमध्यात्माधिभूतदैवतं च}


\twolineshloka
{किंनु सर्गाप्ययं कति सर्गाः किं भूतंकिं भविष्यं किं भव्यं च किं ज्ञानम्}
{को ज्ञाता किं बुद्धं किमप्रबुद्धंकिं बुध्यमानं कति पर्वाणि}


\twolineshloka
{कति स्रोतांसि कति कर्मयोनयःकिमेकत्वं नानात्वम्}
{किं सहवासं निवासंकिं विद्याविद्यमिति ॥'}


\chapter{अध्यायः ३२७}
\twolineshloka
{`कपिल उवाच}
{}


% Check verse!
यद्भवानाह किं व्यक्तिं किमव्यक्तमिति अत्र ब्रूमः
% Check verse!
अव्यक्तमग्राह्यमतर्क्यमपरिमेयमव्यक्तं व्यक्तमुपलक्ष्यतेयथर्तवो मूर्तयस्तेषु पुष्पफलैर्व्यक्तिरुपलक्ष्यतेतद्वद्व्यक्तगुणैरुपलक्ष्यते
% Check verse!
प्राग्गतं प्रत्यग्गतमूर्ध्वमधस्तिर्यक्चशतश्चानुग्राह्यत्वात्साऽकृतिः
\twolineshloka
{व्यक्त उत्तमो रजः सत्वं तत्प्रधानं तत्वमक्षमजरमित्येवमादीन्यव्यक्तनामानिभवन्ति}
{एवमाह}


\twolineshloka
{अव्यक्तं बीजधर्माणं महाग्राहमचेतनम्}
{तस्मादेकगुणो जज्ञे तद्व्यक्तं तत्वमीश्वरः}


\twolineshloka
{तदेतदव्यक्तम्}
{प्रस्नवा घारणादानस्वभावमापोधारणे प्रजनने दानेगुणानां प्रकृतिः सपराप्रमत्तं तदेतदस्मिन्कार्यकरणे}


\threelineshloka
{यदप्युक्तं किमव्यक्तमिति तत्र ब्रूमः}
{व्यक्तं नामाऽऽसुरेयत्पूर्वमव्यक्तादुत्पन्नतत्वमीश्वरमप्रतिबुद्धगुणस्यगेतत्पुरुषसंज्ञिकंमहदित्युक्तं बुद्धिरिति च}
{सत्ता स्मृतिर्धृतिर्मेधा व्यवसायःसमाधिप्राप्तिरित्येवमादीनि व्यक्तपर्याये नामानि वदन्त्येवमाह}


\twolineshloka
{मम व्यक्तादुपात्तासिद्धिरागता संयमश्च महद्यतः}
{परसर्गश्च दीप्त्यर्थमौत्सुक्यं च परं तथा}


% Check verse!
यदेषोर्ध्यस्रोताभिर्महत्वादप्रतिबुद्धत्वाच्चात्मनःयकरोत्यहंकारमव्यक्ताव्यक्ततरम्
% Check verse!
यदप्युक्तं किमव्यक्ततरमिति अत्र ब्रूमः
\twolineshloka
{व्यक्ताव्यक्ततरं नाम तृतीयं पुरुषसंज्ञकम्}
{तदेतदुभयोर्विरिञ्चवैरिञ्चयोरेकैक उत्पत्तिः ॥विरिञ्चोऽभिमानिन्यविवेक ईर्ष्या कामः क्रोधो लोभो मदो दर्पोममकारश्चैतान्यहंकारपर्यायनामानि भवन्त्येवमाह}


\twolineshloka
{अहं कर्तेत्यहंकर्ता ससृजे विश्वमीश्वरः}
{तृतीयमेनं पुरुषमभिमानगुणं विदुः}


\twolineshloka
{अहंकाराद्युगपदुन्मादयामास पञ्च महाभूतानिशब्दस्पर्शरूपरसगन्धलक्षणानि}
{तान्येव बुद्ध्यन्त इत्येवमाह}


\twolineshloka
{भूतसङ्घमहङ्काराद्यो विद्वानवबुध्यसे}
{अभिमानमतिक्रम्य महान्तं प्रतितिष्ठते}


\threelineshloka
{भूतेषु चाप्यहंकारमश्वरूपस्तथोच्यते}
{पुनर्विषयहेत्वर्थे स मनस्संज्ञकः स्मृतः ॥विखराद्वैखरं युगपदिन्द्रियैः सहोत्पादयति}
{श्रोत्रं घ्राणंचक्षुर्जिह्वा त्वगित्येतानि शब्दस्पर्शरूपरसगन्धानवबुध्यन्त इति पञ्चबुद्धीन्द्रियाणि वदन्त्येवमाहुराचार्याः ॥वाग्घस्तौ पादपायुरानन्दश्चेति पञ्चेन्द्रियाणिविशेषमादित्योश्वीनि नक्षत्राणीत्येतानीन्द्रियाणां पर्यायनामानिवदन्त्येवमाह}


\twolineshloka
{अहंकारात्तथा भूतान्युत्पाद्य महदात्मनोः}
{वैखरत्वं ततो राज्ञा वैखर्यो विषयात्मकः}


\twolineshloka
{विकारस्थमहंकारमवबुध्याथ मानवः}
{महदैश्वर्यमाप्नोति यावदाचन्द्रतारकम् ॥'}


\chapter{अध्यायः ३२८}
\twolineshloka
{कपिल उवाच}
{}


\twolineshloka
{यदप्युक्तं कति तत्वानि भवन्ति तत्वमेतानि}
{यमानुपूर्व्यशः प्रोक्तान्येवमाह}


\threelineshloka
{तत्वान्यथोक्तानि}
{तथाविद्यो निबुद्ध्यते}
{न स पापेन लिप्येत निर्मुक्तः सर्वसंकरात्}


% Check verse!
यदप्युक्तं इहाद्यं मध्यमं च तत्वानामित्यत्र ब्रूमः
% Check verse!
एवमाद्यं मध्यमं चोक्तं बुद्ध्यादीनि त्रयोविंशतितत्वानिविशेषपर्यवसानानि ज्ञातव्यानि भवन्तीत्येव मामकेनेत्यत्रोच्यते
\threelineshloka
{तदेव तद्यदादत्तब्राह्मणक्षत्रियवैश्यशूद्रचण्डालपुल्कसादिरेतानि ज्ञातव्यानिबुद्ध्यादीनि विशेषपर्यवसानानि मन्तव्यानि प्रत्येतव्याभ्युक्तानिएतदाद्यं मध्यमं च}
{एतस्मात्तत्वानामुत्पत्तिर्भवति अत्र प्रलीयन्ते}
{केचिदाहुराचार्याः}


% Check verse!
अहमित्येतदात्मकं सशीरसङ्घातं त्रिषु लोकेषुव्यक्तमव्यक्ताधिष्ठितमेतद्देवदत्तसंज्ञकम्
\twolineshloka
{योगदशमपुरुषदर्शनानां तु पञ्चविंशतितत्वानांप्रतिबुध्यमानयोर्व्यतिरिक्तं शुचिव्यभ्रमित्याहुराचार्याः}
{एवमाह}


\twolineshloka
{चतुर्विशतितत्वज्ञस्त्वव्यक्ते प्रतितिष्ठति}
{पञ्चविंशतितत्वज्ञोऽप्यव्यक्तमधितिष्ठति ॥'}


\chapter{अध्यायः ३२९}
\twolineshloka
{युधिष्ठिर उवाच}
{}


\threelineshloka
{कथं निर्वेदमापन्नः शुको वैयासकिः पुरा}
{एतदिच्छाम्यहं श्रोतुं परं कौतूहलं हि मे ॥भीष्म उवाच}
{}


\threelineshloka
{प्राकृतेनैव वृत्तेन चरन्तमकुतोभयम्}
{अध्याप्य कृत्स्नं स्वाध्यायमन्वशाद्वै पिता सुतम् ॥व्यास उवाच}
{}


\twolineshloka
{धर्मं पुत्र निपेवस्व सुतीक्ष्णौ च हिमातपौ}
{क्षुत्पिपासे च वायुं च जय नित्यं जितेन्द्रियः}


\twolineshloka
{सत्यमार्जवमक्रोधमनसूयां दमं तपः}
{अहिंसां चानृशंस्यं च विधिवत्परिपालय}


\twolineshloka
{सत्ये तिष्ठ रतो धर्मे हित्वा सर्वमनार्जवम्}
{देवतातिथिशेषेण यात्रां प्राणस्य संलिह}


\twolineshloka
{फेनमात्रोपमे देहे जीवे शकुनिवत्स्थिते}
{अनित्ये प्रियसंवासे कथं स्वपिषि पुत्रक}


\twolineshloka
{अप्रमत्तेषु जाग्रत्सु नित्ययुक्तेषु शत्रुषु}
{अन्तरं लिप्यमानेषु बालस्त्वं नावबुध्यसे}


\twolineshloka
{अहःसु गण्यमानेषु क्षीयमाणे तथाऽऽयुषि}
{जीविते लिख्यमाने च किमुत्थाय न धावसि}


\twolineshloka
{ऐहलौकिकनीहन्ते मांसशोणितवर्धनम्}
{पारलौकिककार्येषु प्रसुप्ता भृशनास्तिकाः}


\twolineshloka
{धर्माय येऽभ्यसूयन्ति बुद्धिमोहान्विता नराः}
{अपथा गच्छतां तेषामनुयाताऽपि पीड्यते}


\twolineshloka
{ये तु तुष्टाः श्रुतिपरा महात्मानो महाबलाः}
{धर्म्यं पन्थानमारूढास्तानुपास्स्व च पृच्छ च}


\twolineshloka
{उपधार्य मतं तेषां बुधानां धर्मदर्शिनाम्}
{नियच्छ परया बुद्ध्या चित्तमुत्पथगामि वै}


\twolineshloka
{आद्यकालिकया बुद्ध्या दूरेश्च इति निर्भयाः}
{सर्वभक्ष्या न पश्यन्ति कर्मभूमिमचेतसः}


\twolineshloka
{धर्मं निःश्रेणिमास्थाय किंचित्किंचित्समारुह}
{कोशकारवदात्मानं वेष्टयन्नावबुध्यसे}


\twolineshloka
{नास्तिकं भिन्नमर्यादं कूलपातमिव स्थितम्}
{वामतः कुरु विस्रब्धो नरं वेणुमिवोद्धृतम्}


\twolineshloka
{कामक्रोधग्राहवतीं पञ्चेन्द्रियजलां नदीम्}
{नावं धृतिमयीं कृत्वा जन्मदुर्गाणि संतर}


\twolineshloka
{मृत्युनाऽभ्याहते लोके जरया परिपीडिते}
{अमोघासु पतन्तीषु धर्मयानेन संतर}


\twolineshloka
{तिष्ठन्तं च शयानं च मृत्युरन्वेषते यदा}
{निर्वृत्तिं लभते कस्मादकस्मान्मृत्युनाऽशितः}


\twolineshloka
{संचिन्वानकमेवैनं कामानामवितृप्तकम्}
{वृकीवोरणमासाद्य मृत्युरादायं गच्छति}


\twolineshloka
{क्रमशः संचितशिखो धर्मबुद्धिमयो महान्}
{अन्धकारे प्रवेष्टव्ये दीपो यत्नेन धार्यताम्}


\twolineshloka
{संपतन्देहजालानि कदाचिदिह मानुषे}
{ब्राह्मण्यं लभते जन्तुस्तत्पुत्र परिपालय}


\twolineshloka
{ब्राह्मणस्य तु देहोऽयं न कामार्थाय जायते}
{इह क्लेशाय तपसे प्रेत्य त्वनुपमं सुखम्}


\twolineshloka
{ब्राह्मण्यं बहुभिरवाप्यते तपोभिस्तल्लब्ध्वा न रतिपरेण हेलितव्यम्}
{स्वाध्याये तपसि दमे च नित्ययुक्तःमोक्षार्थी कुशलपरः सदा यतस्व}


\twolineshloka
{अव्यक्तप्रकृतिरयं कलाशरीरःसूक्ष्मात्मा क्षणत्रुटिका निमेषरोमा}
{यानेतत्समबलशुक्लकृष्णनेत्रोमासाङ्गो द्रवति वयोहयो नराणाम्}


\twolineshloka
{तं दृष्ट्वा प्रसृतमजस्रमुग्रवेगंगच्छन्तं सततमिहान्ववेक्षमाणम्}
{यक्षुस्ते यदि न परप्रणेतृनेयंधर्मे ते रमतु मनः परं निशाम्य}


\twolineshloka
{येऽमी तु प्रचलितधर्मकामवृत्ताःक्रोशन्तः सततमनिष्टसंप्रयोगात्}
{क्लिश्यन्तः परिगतवेदनाशरीराबह्वीभिः सुभृशमधर्मवागुराभिः}


\twolineshloka
{राजा सदा धर्मपरः शुभाशुभस्य गोप्तासमीक्ष्य सुकृतिनां दधाति लोकान्}
{बहुविधमपि चरति प्रविशतिसुखमनुपगतं निरवद्यम्}


\twolineshloka
{श्वानो भीषणकाया अयोमुखानि वयांसिबलगृध्रकुररपक्षिणां च संघातम्}
{नरकदने रुधिरपा गुरुवचननुदमुपरतं विशन्त्यसन्तः}


\twolineshloka
{मर्यादा नियताः स्वयम्भुवा य इहेमाःप्रभिनत्ति दश गुणा मनोऽनुगत्वात्}
{निवसति भृशमसुखं पितृविषयविपिनमवगाह्य स पापः}


\twolineshloka
{यो लुब्ध सुभृशं प्रियानृतश्च मनुष्यःसततनिकृतिवञ्चनाभिरतिः स्यात्}
{उपनिधिभिरसुखकृत्स परमनिरयगोभृशमसुखमनुभवति दुष्कृतकर्मा}


\twolineshloka
{उष्णां वैतरणीं महानदीमवगाढोऽसिपत्रवनभिन्नगात्रः}
{परशुवनशयोनिपतितोवसति च महानिरये भृशार्तः}


\twolineshloka
{महापादनि कत्थसे न चाप्यवेक्षसे परम्}
{चिरस्य मृत्युकारिकामनागतां न बुध्यसे}


\twolineshloka
{प्रयस्यतां किमास्यते समुत्थितं महद्भयम्}
{अतिप्रमार्थि दारुणं सुखस्य संविधीयताम्}


\twolineshloka
{पुरा मृतः प्रणीयसे यमस्य राजशासनात्}
{त्वमन्तकाय दारुणैः प्रयत्नमार्जवे कुरु}


\twolineshloka
{पुरा समूलबान्धवं प्रभुर्हरत्यदुःखवित्}
{कियत्तवेह जीवितं यमे न चास्ति वारकः}


\twolineshloka
{पुरा विवाति मारुतो यमस्य यः पुरःसरः}
{पुरैक एव नीयसे कुरुष्व सांपरायिकम्}


\twolineshloka
{पुरा स एक एव ते प्रवाति मारुतोऽन्तकः}
{पुरा च विभ्रमन्ति ते दिशो महाभयागमे}


\twolineshloka
{श्रुतिश्च सन्निरुध्यते पुरा तवेह पुत्रक}
{समाकुलस्य गच्छतः समाधिमुत्तमं कुरु}


\twolineshloka
{शुभाशुभे पुरा कृते प्रमादकर्मविप्लुते}
{स्मरन्पुराऽनुतप्यसे निधत्स्व केवलं निधिम्}


\twolineshloka
{पुरा जरा कलेवरं विजर्झरीकरोति ते}
{बलाङ्गरूपहारिणी निधत्स्व केवलं निधिम्}


\twolineshloka
{पुरा शरीरमन्तको भिनत्ति रोगसायकैः}
{प्रसह्य जीवितक्षये तपो महत्समारभ}


\twolineshloka
{पुरा वृका भयंकरा मनुष्यदेहगोचराः}
{अभिद्रवन्ति सर्वतो यतस्व पुण्यशीलने}


\twolineshloka
{पुरान्धकारमेककोऽनुपश्यसि त्वरस्व वै}
{पुरा हिरण्मयान्नगान्निरीक्षसेऽद्रिमूर्धनि}


\twolineshloka
{पुरा कुसङ्गतानि ते सुहृन्मुखाश्च शत्रवः}
{विचालयन्ति दर्शनाद्धटस्व पुत्र यत्परम्}


\twolineshloka
{धनस्य यस्य राजतो भयं न चास्ति चोरतः}
{मृतं च यन्न मुञ्चति समार्जयस्व तद्धनम्}


\twolineshloka
{न तत्र संविभज्यते स्वकर्मभिः परस्परम्}
{यदेव यस्य यौतकं तदेव तत्र सोऽश्नृते}


\twolineshloka
{परत्र तेन जीव्यते तदेव पुत्र जीयताम्}
{धनं यदक्षरं ध्रुवं समार्जयस्व तत्स्वयम्}


\twolineshloka
{न यावदेव पच्यते महाजनस्य यावकम्}
{अपक्व एव यावके पुरा प्रलीयते त्वरम्}


\twolineshloka
{न मातृपुत्रबान्धवा न संस्तुतः प्रियो जनः}
{अनुव्रजन्ति संकटे व्रजन्तमेकपातिनम्}


\twolineshloka
{यदेव कर्म केवलं पुराकृतं शुभाशुभम्}
{तदेव पुत्र यौतकं भवत्यमुत्र गच्छतः}


\twolineshloka
{हिरण्यरत्नसंचयाः शुभाशुभेन संचिताः}
{न तस्य देहसंक्षये भवन्ति कार्यसाधकाः}


\twolineshloka
{न पुत्र शान्तिरस्ति ते कृताकृतस्य कर्मणः}
{न साक्षिकोऽऽत्मना समो नृणामिहास्ति कश्चन}


\twolineshloka
{मनुष्यदेहशून्यकं भवत्यमुत्र गच्छतः}
{प्रविश्य बुद्धिचक्षुषा प्रदृश्यते हि सर्वशः}


\twolineshloka
{इहाग्निसूर्यवायवः शरीरमाश्रितास्त्रयः}
{त एव तस्य साक्षिणो भवन्ति धर्मदर्शिनः}


\twolineshloka
{अहनिंशेषु सर्वतः स्पृशत्सु सर्वचारिषु}
{प्रकाशगूढवृत्तिषु स्वधर्ममेव पालय}


\twolineshloka
{अनेकपारिपान्थके विरूपरौद्रमक्षिके}
{स्वमेव कर्म रक्ष्यतां स्वकर्म तत्र गच्छति}


\twolineshloka
{न तत्र संविभज्यते स्वकर्मभिः परस्परम्}
{तथा कृतं स्वकर्मजं तदेव भुज्यते फलम्}


\twolineshloka
{यथाऽप्सरोगणाः फलं सुखं महर्षिभिः सह}
{तथाऽऽप्नुवन्ति कर्मजं विमानकामगामिनः}


\twolineshloka
{यथेह यत्कृतं शुभं विपाप्मभिः कृतात्मभिः}
{तदाप्नुवन्ति मानवास्तथा विशुद्धयोनयः}


\twolineshloka
{प्रजापतेः सलोकतां बृहस्पतेः शतक्रतोः}
{व्रजन्ति ते परां गतिं गृहस्थधर्मसेतुभिः}


\twolineshloka
{सहस्रशोऽप्यनेकशः प्रवक्तुमुत्सहामहे}
{अबुद्धिमोहनं पुनः प्रभुस्तु तेन पावकः}


\twolineshloka
{गता त्रिरष्टवर्षता ध्रुवोऽसि पञ्चविंशकः}
{कुरष्व धर्मसंचयं वयो हि तेऽतिवर्तते}


\twolineshloka
{पुरा करोति सोऽन्तकः प्रमादगोमुखां चमूम्}
{यथागृहीतमुत्थितस्त्वरस्व धर्मपालने}


\twolineshloka
{यथा त्वमेव पृष्ठतस्त्वमग्रतो गमिष्यसि}
{तथा गतिं गमिष्यतः किमात्मना परेण वा}


\twolineshloka
{यदेकपातिनां सतां भवत्यमुत्र गच्छताम्}
{भयेषु सांपरायिकं निधत्स्व केवलं निधिम्}


\twolineshloka
{सतूलमूलबान्धवं प्रभुर्हरत्यसङ्गवान्}
{न सन्ति यस्य वारकाः कुरष्व धर्मसंनिधिम्}


\twolineshloka
{इदं निदर्शनं मया तवेह पुत्र संमतम्}
{स्वदर्शनानुपानतः प्रवर्णितं कुरुष्व तत्}


\twolineshloka
{ददाति यः स्वकर्मणा धनानि यस्यकस्यचित्}
{अबुद्धिमोहजैर्गुणैः स एक एव युज्यते}


\twolineshloka
{शुभं समस्तमश्नुते प्रकुर्वतः शुभाः क्रियाः}
{तदेतदर्थदर्शनं कृतज्ञमर्थसंहितम्}


\twolineshloka
{निबन्धनी रज्जुरेषा या ग्रामे वसतो रतिः}
{छित्त्वैतां सुकृतो यान्ति नैनां छिदन्ति दुष्कृतः}


\twolineshloka
{किं ते धनेन किं बन्धुभिस्तेकिं ते पुत्रैः पुत्रक यो मरिष्यसि}
{आत्मानमन्विच्छ गुहां प्रविष्टंपितामहास्ते क्व गताश्च सर्वे}


\twolineshloka
{श्वः कार्यमद्ये कुर्वीत पूर्वाह्णे चापराह्णिकम्}
{न हि प्रतीक्षते मृत्युः कृतं वाऽस्य न वाऽकृतम्}


\twolineshloka
{अनुगम्य विनाशान्ते निवर्तन्ते ह बान्धवाः}
{अग्नौ प्रक्षिप्य पुरुषं ज्ञातयः सुहृदस्तथा}


\twolineshloka
{नास्तिकान्निरनुक्रोशान्नरान्पापमते स्थितान्}
{वामतः कुरु विस्रब्धं परं प्रेप्सुरतन्द्रितः}


\twolineshloka
{एवमभ्याहते लोके कालेनोपनिपीडिते}
{सुमहद्धैर्यमालम्ब्य धर्मं सर्वात्मना कुरु}


\twolineshloka
{अथेमं दर्शनोपायं सम्यग्यो वेत्ति मानवः}
{सम्यक् स्वधर्मं कृत्वेह परत्र सुखमश्नुते}


\twolineshloka
{न देहभेदे मरणं विजानतांन च प्रणाशः स्वनुपालिते पथि}
{धर्मं हि यो बर्धयते स पण़्डितोय एव धर्माच्च्यवते स दह्यते}


\twolineshloka
{प्रयुक्तयोः कर्मपथि स्वकर्मणोःफलं प्रयोक्ता लभते यथाविधि}
{निहीनकर्मा निरयं प्रपद्यतेत्रिविष्टपं गच्छति धर्मपारगः}


\twolineshloka
{सोपानभूतं स्वर्गस्य मानुष्यं प्राप्य दुर्लभम्}
{तथाऽऽत्मानं समादध्याद्धश्यते न पुनर्यथा}


\twolineshloka
{यस्य नोत्क्रामति मतिः स्वर्गमार्गानुसारिणी}
{तमाहुः पुण्यकर्माणमशोच्यं मित्रबान्धवैः}


\twolineshloka
{यस्य नोपहता बुद्धिर्निश्चये ह्यवलम्बते}
{स्वर्गे कृतावकाशस्य नास्ति तस्य महद्भयम्}


\twolineshloka
{तपोवनेषु ये जातास्तत्रैव निधनं गताः}
{तेषामल्पतरो धर्मः कामभोगानजानताम्}


\twolineshloka
{यस्तु भोगान्परित्यज्य शरीरेण तपश्चरेत्}
{न तेन किंचिन्न प्राप्तं तन्मे बहुमतं फलम्}


\twolineshloka
{मातापितृसहस्राणि पुत्रदारशतानि च}
{अनागतान्यतीतानि कस्य ते कस्य वा वयम्}


\twolineshloka
{अहमेको न मे कश्चिन्नाहमन्यस्य कस्यचित्}
{न तं पश्यामि यस्याहं तन्न पश्यामि यो मम}


\twolineshloka
{न तेषां भवता कार्यं न कार्यं तव तैरपि}
{स्वकृतैस्तानि जातानि भवांश्चैव गमिष्यति}


\twolineshloka
{इह लोके हि धनिना परोऽपि स्वजनायते}
{स्वजनस्तु दरिद्राणां जीवतामपि नश्यति}


\twolineshloka
{संचिनोत्यशुभं कर्म कलत्रापेक्षया नरः}
{ततः क्लेशमवाप्नोति परत्रेह तथैव च}


\twolineshloka
{पश्यति च्छिन्नभूतं हि जीवलोकं स्वकर्मणा}
{तत्कुरुष्व तथा पुत्र कृत्स्नं यत्समुदाहृतम्}


\twolineshloka
{तदेतत्संप्रदृश्यैव कर्म भूमिं प्रपश्यतः}
{शुभान्याचरितव्यानि परलोकमभीप्सता}


\twolineshloka
{मासर्तुसंज्ञापरिवर्तकेनसूर्याग्निना रात्रिदिवेन्धनेन}
{स्वकर्मनिष्ठाफलसाक्षिकेणभूतानि कालः पचति प्रसह्य}


\threelineshloka
{धनेन किं यन्न ददाति नाश्नुतेबलेन किं येन रिपुं न बाधते}
{श्रुतेन किं येन न धर्ममाचरेत्किमात्मना यो न जितेन्द्रियो वशी ॥भीष्म उवाच}
{}


\twolineshloka
{इदं द्वैपायनवचो हितमुक्तं निशम्य तु}
{शुको गतः परित्यज्य पितरं मोक्षदैशिकम्}


\chapter{अध्यायः ३३०}
\twolineshloka
{युधिष्ठिर उवाच}
{}


\threelineshloka
{यद्यस्ति दत्तमिष्टं वा तपस्तप्तं तथैव च}
{गुरूणां वाऽपि शुश्रूषा तन्मे ब्रूहि पितामह ॥भीष्म उवाच}
{}


\twolineshloka
{आत्मनाऽनर्थयुक्तेन पापे निविशते मनः}
{स कर्म कलुषं कृत्वा क्लेशे महति धीयते}


\twolineshloka
{दुर्भिक्षादेव दुर्भिक्षं क्लेशात्क्लेशं भयाद्भयम्}
{मृतेभ्यः प्रमृता यान्ति दरिद्राः पापकर्मिणः}


\twolineshloka
{उत्सवादुत्सवं यान्ति स्वर्गात्स्वर्गं सुखात्सुखम्}
{श्रद्दधानाश्च दान्ताश्च धनस्याः शुभकारिणः}


\twolineshloka
{व्यालकुञ्जरदुर्गेषु सर्पचोरभयेषु च}
{हस्तावापेन गच्छन्ति नास्तिकाः किमतः परम्}


\twolineshloka
{प्रियदेवातिथेयाश्च वदान्याः प्रियसाधवः}
{क्षेम्यमात्मवता मार्गमास्थिता हस्तदक्षिणाः}


\twolineshloka
{पुलाका इव धान्येषु पूत्यण्डा इव पक्षिषु}
{तद्विधास्ते मनुष्येषु येषां धर्मो न कारणम्}


\twolineshloka
{सुशीघ्रमपि धावन्तं विधानमनुधावति}
{शेते सह शयानेन येनयेन यथाकृतम्}


\twolineshloka
{पापं तिष्ठति तिष्ठन्तं धावन्तमनुधावति}
{करोति कुर्वतः कर्म च्छायेवानुविधीयते}


\twolineshloka
{येनयेन यथा यद्यत्पुरा कर्म सुनिश्चितम्}
{तत्तदेव नरो भुङ्क्ते नित्यं विहितमात्मना}


\twolineshloka
{समानकर्मनिक्षेपं विधानपरिरक्षणम्}
{भूतग्राममिमं कालः समन्तादपकर्षति}


\twolineshloka
{अचोद्यमानानि यथा पुष्पाणि च फलानि च}
{स्वं कालं नातिवर्तन्ते तथा कर्म पुराकृतम्}


\twolineshloka
{समानश्चावमानश्च लाभालाभौ जयाजयौ}
{प्रवृत्ता न निवर्तन्ते निधनान्ताः पदेपदे}


\twolineshloka
{आत्मना विहितं दुःखमात्मना विहितं सुखम्}
{गर्भशय्यामुपादाय भजते पूर्वदेहिकम्}


\twolineshloka
{बालो युवा वा वृद्धश्च यत्करोति शुभाशुभम्}
{तस्यांतस्यामवस्थायां भुङ्क्ते जन्मनिजन्मनि}


\twolineshloka
{यथा धेनुसहस्रेषु वत्सो विन्दति मातरम्}
{तथा पूर्वकृतं कर्म कर्तारमनुगच्छति}


\twolineshloka
{मलिनं हि यथा वस्त्रं पश्चाच्छुध्यति वारिणा}
{उपवासैः प्रतप्तानां दीर्घं सुखमनन्तकम्}


\twolineshloka
{दीर्घकालेन तपसा सेवितेन तपोवने}
{धर्मनिर्धूतपापानां संसिध्यन्ते मनोरथाः}


\twolineshloka
{शकुनानामिवाकाशे मत्स्यानामिव चोदके}
{पदं यथा न दृश्येत तथा पुण्यकृतां गतिः}


\twolineshloka
{अलमन्यैरुपालब्धैः कीर्तितैश्च व्यतिक्रमैः}
{पेशलं चानुरूपं च कर्तव्यं हितमात्मनः}


\chapter{अध्यायः ३३१}
\twolineshloka
{युधिष्ठिर उवाच}
{}


\twolineshloka
{कथं व्यासस्य धर्मात्मा शुको जज्ञे महातपाः}
{सिद्धिं च परमां प्राप्तस्तन्मे ब्रूहि पितामह}


\twolineshloka
{कस्यां चोत्पादयामास शुकं व्यासस्तपोधनः}
{न ह्यस्य जननीं विद्मो जन्म चाग्र्यं महात्मनः}


\twolineshloka
{कथं च बालस्य सतः सूक्ष्मज्ञाने रता मतिः}
{यथा नान्यस्य लोकेऽस्मिन्द्वितीयस्येह कस्यचित्}


\twolineshloka
{एतदिच्छाम्यहं श्रोतुं विस्तरेण महामते}
{न हि मे तृप्तिरस्तीह शृण्वतोऽमृतमुत्तमम्}


\threelineshloka
{माहात्म्यमात्मयोगं च विज्ञानं च शुकस्य ह}
{यथावदानुपूर्व्येण तन्मे ब्रूहि पितामह ॥भीष्म उवाच}
{}


\twolineshloka
{न हायनैर्न पलितैर्न वित्तैर्न च बन्धुभिः}
{ऋषयश्चक्रिरे धर्मं योऽनूचानः स नो महान्}


\twolineshloka
{तपोमूलमिदं सर्वं यन्मां पृच्छसि पाण्डव}
{तदिन्द्रियाणि संयम्य तपो भवति नान्यथा}


\twolineshloka
{इन्द्रियाणां प्रसङ्गेन दोषमृच्छत्यसंशयम्}
{संनियम्य तु तान्येव सिद्धिमाप्नोति मानवः}


\twolineshloka
{अश्वमेधसहस्रस्य वाजपेयशतस्य च}
{योगस्य कलया तात न तुल्यं विद्यते फलम्}


\twolineshloka
{अत्र ते वर्तयिष्यामि जन्मयोगफलं तथा}
{शुकस्याग्र्यां गतिं चैव दुर्विदामकृतात्मभिः}


\twolineshloka
{मेरुशृङ्गे किल पुरा कर्णिकारवनायुते}
{विजहार महादेवो भीमैर्भूतगणैर्वृतः}


\twolineshloka
{शैलराजसुता चैव देवी तत्राभवत्पुरा}
{तत्र दिव्यं तपस्तेषे कृष्णद्वैपायनः प्रभुः}


\twolineshloka
{योगेनात्मानमाविश्य योगधर्मपरायणः}
{धारयन्स तपस्तेपे पुत्रार्थं कुरुसत्तम}


\twolineshloka
{अग्नेर्भूमेरपां वायोरन्तरिक्षस्य वा विभो}
{वीर्येण संमितः पुत्रो मम भूयादिति स्म ह}


\twolineshloka
{संकल्पेनाथ मौनेन दुष्प्रापमकृतात्मभिः}
{वरयामास देवेशमास्थितस्तप उत्तमम्}


% Check verse!
अतिष्ठन्मारुताहारः शतं किल समाः प्रभुःआराधयन्महादेवं बहुरूपमुमापतिम्
\twolineshloka
{तत्र ब्रह्मर्षयश्चैव सर्वे देवर्षयस्तथा}
{लोकपालाश्च लोकेशं साध्याश्च वसुभिः सह}


\twolineshloka
{आदित्याश्चैव रुद्राश्च दिवाकरनिशाकरौ}
{मारुतो मरुतश्चैव सागराः सरितस्तथा}


\twolineshloka
{अश्विनौ देवगन्धर्वास्तथा नारदपर्वतौ}
{विश्वावसुश्च गन्धर्वः सिद्धाश्चाप्सरसां गणाः}


\twolineshloka
{तत्र रुद्रो महादेवः कर्णिकारमयीं शुभाम्}
{धारयाणः स्रजं भाति ज्योत्स्नामिव निशाकरः}


\twolineshloka
{तस्मिन्दिव्ये वने रम्ये देवदेवर्षिसंकुले}
{आस्थितः परमं योगमृषिः पुत्रार्थमच्युतः}


\twolineshloka
{न चास्य हीयते प्राणो न ग्लानिरुपजायते}
{त्रयाणामपि लोकानां तदद्भुतमिवाभवत्}


\twolineshloka
{जटाश्च तेजसा तस्य वैश्वानरशिखोपमाः}
{प्रज्वलन्त्यः स्म दृश्यन्ते युक्तस्यामिततेजसः}


\twolineshloka
{मार्कण्डेयो हि भगवानेतदाख्यातवान्मम}
{स देवचरितानीह कथयामास मे तदा}


\twolineshloka
{एता अद्यापि कृष्णस्य तपसा तेन दीपिताः}
{अग्निवर्णा जटास्तात प्रकाशन्ते महात्मनः}


\twolineshloka
{एवंविधेन तपसा तस्य भक्त्या च भारत}
{महेश्वरः प्रसन्नात्मा चकार मनसा मतिम्}


\threelineshloka
{`ततस्तस्य महादेवो दर्शयामास साम्बिकः}
{'उवाच चैवं भगवांख्यम्बकः प्रहसन्निव}
{एवंविधस्ते तनयो द्वैपायन भविष्यति}


\twolineshloka
{यथा ह्यग्निर्यथा वायुर्यथा भूमिर्यथा जलम्}
{यथाऽऽकारास्तथा शुद्धो भविता ते सुतो महान्}


\twolineshloka
{तद्भावभावी तद्बुद्धिस्तदाऽऽत्मा तदपाश्रयः}
{तेजसाऽऽवृत्य लोकांस्त्रीन्यशः प्राप्स्यति ते सुतः}


\chapter{अध्यायः ३३२}
\twolineshloka
{भीष्म उवाच}
{}


\twolineshloka
{स लब्ध्वा परमं देवाद्वरं सत्यवतीसुतः}
{अरणीं तु ततो गृह्य ममन्थाग्निचिकीर्षया}


\twolineshloka
{अथ रूपं परं राजन्बिभ्रतीं स्वेन तेजसा}
{घृताचीं नामाप्सरसमपश्यद्भगवानृषिः}


\twolineshloka
{ऋषिरप्सरसं दृष्ट्वा सहसा काममोहितः}
{अभवद्भगवान्व्यासो वने तस्मिन्युधिष्ठिर}


\twolineshloka
{सा च दृष्ट्वा तदा व्यासं कामसंविग्नमानसम्}
{शुकी भूत्वा महाराज घृताची समुपागमत्}


\twolineshloka
{स तामप्सरसं दृष्ट्वा रूपेणान्येन संवृताम्}
{शरीरजेनानुगतः सर्वगात्रातिगेन ह}


\threelineshloka
{स तु धैर्येण महता निगृह्णन्हृच्छयं मुनिः}
{न शशाक नियन्तुं तद्व्यासः प्रविसृतं मनः}
{भावित्वाच्चैव भावस्य घृताच्या वपुषा हृतः}


\twolineshloka
{यत्नान्नियच्छतस्तस्य मुनेरग्निचिकीर्षया}
{अरण्यामेव सहसा तस्य शुक्रमवापतत्}


\twolineshloka
{सोऽविशङ्केन मनसा तथैव द्विजसत्तमः}
{अरणीं ममन्थ ब्रह्मर्षिस्तस्यां जज्ञे शुको नृप}


\twolineshloka
{शुक्रे निर्मथ्यमाने स शुको जज्ञे महातपाः}
{परमर्षिर्महायोगी अरणीगर्भसंभवः}


\twolineshloka
{यथाऽध्वरे समिद्धोऽग्निर्भाति हव्यमुदावहन्}
{तथारूपः शुको जज्ञे प्रज्वलन्निव तेजसा}


\twolineshloka
{विभ्रत्पितुश्च कौरव्य रूपवर्णमनुत्तमम्}
{बभौ तदा भावितात्मा विधूमोऽग्निरिवज्वलन्}


\twolineshloka
{तं गङ्गा सरितां श्रेष्ठा मेरुपृष्ठे जनेश्वर}
{स्वरूपिणी तदाऽभ्येत्य स्नापयामास वारिणा}


\twolineshloka
{अन्तरिक्षाच्च कौरव्य दण्डः कृष्णाजिनं च ह}
{पपात भुवि राजेन्द्र शुकस्थार्थे महात्मनः}


\twolineshloka
{जेगीयन्ते स्म गन्धर्वा ननृतुश्चाप्सरोगणाः}
{देवदुन्दुभयश्चैव प्रावाद्यन्त सहस्रशः}


\twolineshloka
{विश्वासुश्च गन्धर्वस्तथा तुम्बुरुनारदौ}
{हाहा हूहूश्च गन्धर्वौ तुष्टुवुः शुकसंभवम्}


\twolineshloka
{तत्र शक्रपुरोगाश्च लोकपालाः समागताः}
{देवा देवर्षयश्चैव तथा ब्रह्मर्षयोऽपि च}


\twolineshloka
{दिव्यानि सर्वपुष्पाणि प्रववर्ष च मारुतः}
{जङ्गमं स्थावरं चैव प्रहृष्टमभवज्जगत्}


\twolineshloka
{तं महात्मा स्वयं प्रीत्या देव्या सह महाद्युतिः}
{जतामात्रं मुनेः पुत्रं विधिनोपानयत्तदा}


\twolineshloka
{तस्य देवेश्वरः शक्रो दिव्यमद्भुतदर्शनम्}
{ददौ कमण्डलुं प्रीत्या देववासांसि चाभिभो}


\twolineshloka
{हंसाश्च शतपत्राश्च सारसाश्च सहस्रशः}
{प्रदक्षिणमवर्तन्त शुकाश्चाषाश्च भारत}


\twolineshloka
{आरणेयस्ततो दिव्यं प्राप्य जन्म महाद्युतिः}
{तत्रैवोवास मेधावी ब्रह्मचारी समाहितः}


\twolineshloka
{उत्पन्नमात्रं तं वेदाः सरहस्याः ससंग्रहाः}
{उपतस्थुर्महाराज यथाऽस्य पितरं तथा}


\twolineshloka
{बृहस्पतिं च वव्रे स वेदवेदाङ्गभाष्यवित्}
{उपाध्यायं महाराज धर्ममेवानुचिन्तयन्}


\twolineshloka
{सोऽधीत्य निखिलान्वेदान्सरहस्यान्ससंग्रहान्}
{इतिहासं च कार्त्स्न्येन धर्मशास्त्राणि चाभिभो}


\twolineshloka
{गुरवे दक्षिणां दत्त्वा समावृत्तो महामुनिः}
{उग्रं तपः समारेभे ब्रह्मचारी समाहितः}


\twolineshloka
{देवतानामृषीणां च बाल्येऽपि स महातपाः}
{संमन्त्रणीयो मान्यश्च ज्ञानेन तपसा तथा}


\twolineshloka
{न त्वस्य रमते बुद्धिराश्रमेषु नराधिप}
{त्रिषु गार्हस्थ्यमूलेषु मोक्षधर्मानुदर्शिनः}


\chapter{अध्यायः ३३३}
\twolineshloka
{भीष्म उवाच}
{}


\twolineshloka
{स मोक्षमनुचिन्त्यैव शुकः पितरमभ्यगात्}
{प्राहाभिवाद्य च गुरुं श्रेयोर्थी विनयान्वितः}


\twolineshloka
{मोक्षधर्मेषु कुशलो भगवान्प्रब्रवीतु मे}
{यथा मे मनसः शान्तिः परमा संभवेत्प्रभो}


\twolineshloka
{श्रुत्वा पुत्रस्य तु वचः परमर्षिरुवाच तम्}
{अधीहि पुत्र मोक्षं वै धर्मांश्च विविधानपि}


\twolineshloka
{पितुर्नियोगाज्जग्राह शुको धर्मभृतां वरः}
{योगशास्त्रं च निखिलं कापिलं चैव भारत}


\twolineshloka
{स तं ब्राह्नया श्रियः युक्तं ब्रह्मतुल्यपराक्रमम्}
{मेने पुत्रं यदा व्यासो मोक्षधर्मविशारदम्}


\twolineshloka
{उवाच गच्छेति तदा जनकं मिथिलेश्वरम्}
{स ते वक्ष्यति मोक्षार्थं निखिलं मिथिलेश्वरः}


\twolineshloka
{पितुर्नियोगादगमन्मैथिलं जनकं नृपम्}
{प्रष्टुं धर्मस्य निष्ठां वै मोक्षस्य च परायणम्}


\twolineshloka
{उक्तश्च मानुषेण त्वं पथा गच्छेत्यविस्मितः}
{न प्रभावेण गन्तव्यमन्तरिक्षचरेण वै}


\twolineshloka
{आर्जवेनैव गन्तव्यं न सुखान्वेषिणा तथा}
{नान्वेष्टव्या विशेषास्तु विशेषा हि प्रसङ्गिनः}


\twolineshloka
{अहंकारो न कर्तव्यो याज्ये तस्मिन्नराधिपे}
{स्यातव्यं च वशे तस्य स ते छेत्स्यति संशयम्}


\twolineshloka
{स धर्मकुशलो राजा मोक्षशास्त्रविशारदः}
{याज्यो मम स यद्ब्रूयात्तत्कार्यमविशङ्कया}


\twolineshloka
{एवमुक्तः स धर्मात्मा जगाम मिथिलां मुनिः}
{पद्भ्यां शक्तोन्तरिक्षेण क्रान्तुं पृथ्वीं ससागराम्}


\twolineshloka
{स गिरींश्चाप्यतिक्रम्य नदीतीर्थसरांसि च}
{बहुव्यालमृगाकीर्णा ह्यटवीश्च वनानि च}


\twolineshloka
{मेहोर्हरेश्च द्वे वर्षे वर्षं हैमवतं ततः}
{क्रमेणैवं व्यतिक्रम्य भारतं वर्षमासदत्}


\twolineshloka
{स देशान्विविधान्पश्यंश्चीनहूणनिषेवितान्}
{आर्यावर्तमिमं देशमाजगाम महामुनिः}


\twolineshloka
{पितुर्वचनमाज्ञाय तमेवार्थं विचिन्तयन्}
{अध्वानं सोऽतिचक्राम खचरः खे पतविव}


\twolineshloka
{पत्तनानि च रम्याणि स्फीतानि नगराणि च}
{रत्नानि च विचित्राणि पश्यन्नपि न पश्यति}


\twolineshloka
{उद्यावानि च रम्याणि तथैवायतनानि च}
{पुण्यानि चैव तीर्थानि सोत्यक्रामदथाध्वगः}


\twolineshloka
{सोचिरेणैव कालेन विदेहानाससाद ह}
{रक्षितान्धर्मराजेन जनकेन महात्मना}


\twolineshloka
{तत्र ग्रामान्बहून्पश्यन्बह्वन्नरसभोजनान्}
{पल्लीघोषान्समृद्धांश्च बहुगोकुलसंकुलान्}


\twolineshloka
{स्फीतांश्च शालियवसर्हंससारससेवितान्}
{पद्मिनीभिश्च शतशः श्रीमतीभिरलकृतान्}


\twolineshloka
{स विदेहानतिक्रम्य समृद्धजनसेवितान्}
{मिथिलोपवनं रम्यमाससाद समृद्धिमत्}


\twolineshloka
{हस्त्यश्वरथसंकीर्णं नरनारीसमाकुलम्}
{पश्यन्नपश्यन्निव तत्समतिक्रामदच्युतः}


\twolineshloka
{मनसा तं बहन्भारं तमेवार्थं विचिन्तयन्}
{आत्मारामः प्रसन्नात्मा मिथिलामाससाद ह}


\twolineshloka
{तस्या द्वारं समासाद्य द्वारपालैर्निवारितः}
{स्थितो ध्यानपरो मुक्तो विदितः प्रविवेश ह}


\twolineshloka
{स राजमार्गमासाद्य समृद्धजनकसंकुलम्}
{पार्थिवक्षयमासाद्य निःशङ्कः प्रविवेश ह}


\twolineshloka
{तत्रापि द्वारपलास्तमुग्रवाचा न्यषेधयन्}
{तथैव च शुक्रस्तत्र निर्मन्युः समतिष्ठत}


\twolineshloka
{न चातपाध्वसंतप्तः क्षुत्पिपासाश्रमान्वितः}
{प्रताम्यति ग्लायति वा नापैति च तथाऽऽतपात्}


\twolineshloka
{तेषां तु द्वारपालानामेकः शोकसमन्वितः}
{मध्यंगतमिवादित्यं दृष्ट्वा शुकमवस्थितम्}


\twolineshloka
{पूजयित्वा यथान्यायमभिवाद्य कृताञ्जलिः}
{प्रावेशयत्ततः कक्ष्यां प्रथमां राजवेश्मनः}


\twolineshloka
{तत्रासीनः शुकस्तात मोक्षमेवान्वचिन्तयत्}
{छायायामातपे चैव समदर्शी समद्युतिः}


\twolineshloka
{तं मुहूर्तादिवागम्य राज्ञो मन्त्री कृताञ्जलिः}
{प्रावेशयत्ततः कक्ष्यां द्वितीयां राजवेश्मनः}


\twolineshloka
{तत्रान्तः पुरसंबद्धं महच्चैत्ररथोपमम्}
{सुविभक्तजलाक्रीडं रम्यं पुष्पितपादपम्}


\twolineshloka
{तं दर्शयित्वा स शुकं मन्त्री जनकमुत्तमम्}
{अर्हमासनमादिश्य निश्चक्रामः ततः पुनः}


\twolineshloka
{तं चारुवेषाः सुश्रोण्यस्तरुण्यः प्रियदर्शनाः}
{सूक्ष्मरक्ताम्बरधरास्तप्तकाञ्चनभूषणाः}


\twolineshloka
{संलापालापकुशला नृत्तगीतविशारदाः}
{स्मितपूर्वाभिभाषिण्यो रूपेणाप्सरसां समाः}


\twolineshloka
{भावोपचारकुशला भावज्ञाः सत्वकोविदाः}
{परं पञ्चाशतं नार्यो वारमुख्याः समाद्रवन्}


\twolineshloka
{पाद्यादीनि प्रतिग्राह्य पूजया परयाऽर्चयन्}
{कालोपपन्नेन तदा स्वाद्वन्नेनाभ्यतर्पयन्}


\twolineshloka
{तस्य भुक्तवतस्तात तदन्तः पुरकाननम्}
{सुरम्यं दर्शयामासुरेकैकश्येन भारत}


\twolineshloka
{क्रीडन्त्यश्च हसन्त्यश्च गायन्त्यश्चापि ताः शुभम्}
{उदारसत्वं सत्वज्ञाः स्त्रियः पर्यचरंस्तथा}


\twolineshloka
{आरणेयस्तु शुद्धात्मा निःसंदेहस्त्रिकर्मकृत्}
{वश्येन्द्रियो जितक्रोधो न हृष्यति न कुप्यति}


\twolineshloka
{तस्मै शय्यासनं दिव्यं वरार्हः रत्नभूषितम्}
{स्पर्ध्यास्तरणसंकीर्णं ददुस्ताः परमस्त्रियः}


\twolineshloka
{पादशौचं तु कृत्वैव शुक्रः संध्यामुपास्य च}
{निषसादासने पुण्ये तमेवार्थं विचिन्तयन्}


\threelineshloka
{पूर्वरात्रे तु तत्रासौ हुत्वा ध्यानपरायणः}
{मध्यरात्रे यथान्यायं निद्रामाहारयत्प्रभुः}
{}


\twolineshloka
{ततो मुहूर्तादुत्थाय कृत्वा शौचमनन्तरम्}
{स्त्रीभिः परिवृतो धीमान्ध्यानमेवान्वपद्यत}


\twolineshloka
{अनेन विधिना कर्ष्णिस्तदहः शेषमच्युतः}
{तां च रात्रिं नृपकुले वर्तयामास भारत}


\chapter{अध्यायः ३३४}
\twolineshloka
{भीष्म उवाच}
{}


\twolineshloka
{ततः स राजा जनको मन्त्रिभिः सह भारत}
{पुरोहितं पुरस्कृत्य सर्वाण्यन्तः पुराणि च}


\twolineshloka
{आसनं च पुरस्कृत्य रत्नानि विविधानि च}
{शिरसा चार्ध्यमादाय गुरुपुत्रं समभ्यगात्}


\twolineshloka
{स तदासनमादाय बहुरत्नविभूषितम्}
{स्पर्ध्यास्तरणसंस्तीर्णं सर्वतोभद्रमृद्धिमत्}


\twolineshloka
{पुरोधसा संगृहीतं हस्तेनालभ्य पार्थिवः}
{प्रददौ गुरुपुत्राय शुकाय परमार्चितम्}


\twolineshloka
{तत्रोपविष्टं तं कार्ष्णि शास्त्रतः प्रतिपूज्य च}
{पाद्यं निवेद्य प्रथममर्ध्यं गां च न्यवेदयत्}


\twolineshloka
{स च तां मन्त्रवत्पूजां प्रत्यगृह्णाद्यथाविधि}
{प्रतिगृह्य तु तां पूजां जनकाद्द्विजसत्तमः}


\twolineshloka
{गां चैव समनुज्ञाय राजानमनुमान्य च}
{पर्यपृच्छन्महातेजा राज्ञः कुशलमव्ययम्}


\twolineshloka
{अनामयं च राजेन्द्र शुकः सानुचरस्य ह}
{अनुज्ञातः स तेनाथ निषसाद सहानुगः}


\threelineshloka
{कुशलं चाव्ययं चैव पृष्ट्वा वैयासकिं नृपः}
{किमागमनमित्येवं पर्यपृच्छत पार्थिवः ॥शुक उवाच}
{}


\twolineshloka
{पित्राऽहमुक्तो भद्रं ते मोक्षधर्मार्थकोविदः}
{विदेहराजो याज्यो मे जनको नाम विश्रुतः}


\twolineshloka
{तत्र गच्छस्व वै तूर्णं यदि ते हृदि संशयः}
{प्रवृत्तौ वा निवृत्तौ वा स ते च्छेत्स्यति संशयं}


\twolineshloka
{सोहं पितुर्नियोगात्त्वामुपप्रष्टुमिहागतः}
{तन्मे धर्मभृतां श्रेष्ठ यथावद्वक्तुमर्हसि}


\threelineshloka
{किं कार्यं ब्राह्मणेनेह मोक्षार्थश्च किमात्मकः}
{कथं च मोक्षः प्राप्तव्यो ज्ञानेन तपसाऽथवा ॥जनक उवाच}
{}


\twolineshloka
{यत्कार्यं ब्राह्मणेनेह जन्मप्रभृति तच्छृणु}
{कृतोपनयनस्तात भवेद्वेदपरायणः}


\twolineshloka
{तपसा गुरुवृत्त्या च ब्रह्मचर्येण चाभिभो}
{देवतानामृषीणां चाप्यनृणो ह्यनसूयकः}


\twolineshloka
{वेदानधीत्य नियतो दक्षिणामपवर्ज्य च}
{अभ्यनुज्ञामथ प्राप्य समावर्तेत वै द्विजः}


\twolineshloka
{समावृत्तश्च गार्हस्थ्ये स्वदारनिरतो वसेत्}
{अनसूयुर्यथान्यायमाहिताग्निरनावृतः}


\twolineshloka
{उत्पाद्य पुत्रं पौत्रं तु वन्याश्रमपदे वसेत्}
{तानेवाग्नीन्यथाशास्त्रमर्चयन्नतिथिप्रियः}


\threelineshloka
{स वनेऽग्नीन्यथान्यायमात्मन्यारोप्य धर्मवित्}
{निर्द्वन्द्वो बीतरागात्मा ब्रह्माश्रमपदे वसेत् ॥शुक उवाच}
{}


\twolineshloka
{उत्पन्ने ज्ञानविज्ञाने प्रत्यक्षे हृदि शाश्वते}
{किमवश्यं निवस्तव्यमाश्रमेषु वनेषु वा}


\threelineshloka
{एतद्भवन्तं पृच्छामि तद्भवान्वक्तुमर्हति}
{यथा वेदार्थतत्त्वेन ब्रूहि मे त्वं जनाधिप ॥जनक उवाच}
{}


\twolineshloka
{न विना ज्ञानविज्ञाने मोक्षस्याधिगमो भवेत्}
{न विना गुरुसंबन्धं ज्ञानस्याधिगमः स्मृतः}


\twolineshloka
{गुरुः प्लावयिता तस्य ज्ञानं प्लव इहोच्यते}
{विज्ञाय कृतकृत्यस्तु तीर्णस्तदुभयं त्यजेत्}


\twolineshloka
{अनुच्छेदाय लोकानामनुच्छेदाय कर्मणाम्}
{पूर्वैराचरितो धर्मश्चातुराश्रम्यसंश्रितः}


\twolineshloka
{अनेन क्रमयोगेन बहुजातिषु कर्मणाम्}
{कृत्वा शुभाशुभं कर्म मोक्षो नामेह लभ्यते}


\twolineshloka
{भावितैः करणैश्चायं बहुसंसारयोनिषु}
{आसादयति शुद्धात्मा मोक्षं वै प्रथमाश्रमे}


\twolineshloka
{तमासाद्य तु मुक्तस्य दृष्टार्थस्य विपश्चितः}
{त्रिष्वाश्रमेषु कोऽन्वर्थो भवेत्परमभीप्सतः}


\twolineshloka
{राजसांस्तामसांश्चैव नित्यं दोषान्विवर्जयेत्}
{सात्विकं मार्गमास्थाय पश्येदात्मानमात्मना}


\twolineshloka
{सर्वभूतेषु चात्मानं सर्वभूतानि चात्मनि}
{संपश्यन्नोपलिप्येत जले वारिचरो यथा}


\twolineshloka
{पक्षिवत्प्रायणादूर्ध्वममुत्रानन्त्यमश्नुते}
{विहाय देहान्निर्मुक्तो निर्द्वन्द्वः प्रशमं गतः}


\twolineshloka
{अत्र गाथाः पुरा गीताः शृणु राज्ञा ययातिना}
{धार्यन्तो या द्विजैस्तात मोक्षशास्त्रविशारदैः}


\twolineshloka
{ज्योतिरात्मनि नान्यत्र सर्वजन्तुषु तत्समम्}
{स्वयं च शक्यते द्रष्टुं सुसमाहितचेतसा}


\twolineshloka
{न बिभेति परो यस्मान्न बिभेति पराच्च यः}
{यश्च नेच्छति न द्वेष्टि ब्रह्म संपद्यते तु सः}


\twolineshloka
{यदा भावं न कुरुते सर्वभूतेषु पापकम्}
{कर्मणा मनसा वाचा ब्रह्म संपद्यते तदा}


\twolineshloka
{संयोज्य मनसाऽऽत्मानमीर्ष्यामुत्सृज्य मोहनीम्}
{त्यक्त्वा कामं च मोहं च ततो ब्रह्मत्वमश्नुते}


\twolineshloka
{यदा श्राव्ये च दृश्ये च सर्वभूतेषु चाप्ययम्}
{समो भवति निर्द्वन्द्वो ब्रह्म संपद्यते तदा}


\twolineshloka
{यदा स्तुतिं च निन्दां च समत्वेनैव पश्यति}
{काञ्चनं चायसं चैव सुखं दुःखं तथैव च}


\twolineshloka
{शीतमुष्णं तथैवार्थमनर्थं प्रियमप्रियम्}
{जीवितं मरणं चैव ब्रह्म संपद्यते तदा}


\twolineshloka
{प्रसार्येह यथाङ्गानि कूर्मः संहरते पुनः}
{तथेन्द्रियाणि मनसा संयन्तव्यानि भिक्षुणा}


\twolineshloka
{तमः परिगतं वेश्म यथा दीपेन दृश्यते}
{तथा बुद्धिप्रदीपेन शक्य आत्मा निरीक्षितुम्}


\twolineshloka
{एतत्सर्वं च पश्यामि त्वयि बुद्धिमतां वर}
{यच्चान्यदपि नोक्तं मे तत्त्वतो वेद तद्भवान्}


\twolineshloka
{ब्रह्मर्षे विदितश्चासि विषयान्तमुपागतः}
{गुरोस्तव प्रसादेन तव चैवोपशिक्षया}


\twolineshloka
{तस्यैव च प्रसादेन प्रादुर्भूतं महात्मनः}
{ज्ञानं दिव्यं ममापीदं तेनासि विदितो मम}


\twolineshloka
{अधिकं तव विज्ञानमधिका च गतिस्तव}
{अधिकं तव चैश्वर्यं तच्च त्वं नावबुध्यसे}


\twolineshloka
{बाल्याद्वा संशयाद्वापि भयाद्वाऽप्यविमोक्षणात्}
{उत्पन्ने चापि विज्ञाने नाधिगच्छति तां गतिं}


\twolineshloka
{व्यवसायेन शुद्धेन मद्विधैश्छिन्नसंशयः}
{विमुच्य हृदयग्रन्थीनासादयति तां गतिम्}


\twolineshloka
{भवांश्चोत्पन्नविज्ञानः स्थिरबुद्धिरलोलुपः}
{व्यवसायादृते ब्रह्मन्नासादयति तत्परम्}


\twolineshloka
{नास्ति ते सुखदुःखेषु विशेषो नास्ति लोलुपः}
{नौत्सुक्यं नृत्यगीतेषु न राग उपजायते}


\twolineshloka
{न बन्धुष्वनुबन्धस्ते न भयेष्वस्ति ते भयम्}
{पश्यामि त्वां महाभाग तुल्यलोष्टाश्मकाञ्चनम्}


\twolineshloka
{अहं त्वामनुपश्यामि ये चाप्यन्ते मनीषिणः}
{आस्थितं परमं मार्गमक्षयं तमनामयम्}


\twolineshloka
{यत्फलं ब्राह्मणस्येह मोक्षार्थश्च यदात्मकः}
{तस्मिन्वै वर्तसे विप्र किमन्यत्परिपृच्छसि}


\chapter{अध्यायः ३३५}
\twolineshloka
{भीष्म उवाच}
{}


\twolineshloka
{एतच्छ्रुत्वा तु वचनं कृतात्मा कृतनिश्चयः}
{आत्मनाऽऽत्मानमास्थाय दृष्ट्वा चात्मानमात्मना}


\twolineshloka
{कृतकार्यः सुखी शान्तस्तूष्णीं प्रायादुदङ्भुखः}
{शैशिरं गिरिमुद्दिश्य सधर्मा मातरिश्वनः}


\twolineshloka
{एतस्मिन्नेव काले तु देवर्षिर्नारदस्तथा}
{हिमवन्तमियाद्दुष्टुं सिद्धचारणसेवितम्}


\twolineshloka
{तमप्सरोगाकीर्णं गीतस्वननिनादितम्}
{किन्नराणां सहस्रैश्च भृङ्गराजैस्तथैव च}


% Check verse!
मद्गुभिः खञ्जरीटैश्च विचित्रैर्जीवजीवकैः
\twolineshloka
{चित्रवर्णैर्मयूरैश्च केकाशतविराजितैः}
{राजहंससमूहैश्च हृष्टैः परभृतैस्तथा}


\twolineshloka
{पक्षिराजो गरुत्मांश्च यं नित्यमधितिष्ठति}
{चत्वारो लोकपालाश्च देवाः सर्षिगणास्तथा}


\twolineshloka
{तत्र नित्यं समायान्ति लोकस्य हितकाम्यया}
{विष्णुना यत्र पुत्रार्थे तपस्तप्तं महात्मना}


\twolineshloka
{तत्रैव च कुमारेण बाल्ये क्षिप्ता दिवौकसः}
{शक्तिर्न्यस्ता क्षितितले त्रैलोक्यमवमन्य वै}


\twolineshloka
{तत्रोवाच जगत्स्कन्दः क्षिपन्वाक्यमिदं तदा}
{योऽन्योस्ति मत्तोऽभ्यधिको विप्रा यस्याधिकं प्रियाः}


\twolineshloka
{यो ब्रह्मण्यो द्वितीयोऽस्ति त्रिषु लोकेषु वीर्यवान्}
{सोभ्युद्धरत्विमां शक्तिमथवा कम्पयत्विति}


\twolineshloka
{तच्छुत्वा व्यथिता लोकाः क इमामुद्धरेदिति}
{अथ देवगणं सर्वं संभ्रान्तेन्द्रियमानसम्}


\twolineshloka
{अपश्यद्भगवान्विष्णुः क्षिप्तं सासुरराक्षसम्}
{किंन्वत्र सुकृतं कार्यं भवेदिति विचिन्तयन्}


\twolineshloka
{अनामृष्य ततः क्षेपमवैक्षत च पाविकम्}
{संप्रगृह्य विशुद्धात्मा शक्तिं प्रज्वलितां तदा}


\twolineshloka
{कम्पयामास सव्येन पाणिना पुरुषोत्तमः}
{शक्त्यां तु कम्प्यमानायां विष्णुना बलिना तदा}


\twolineshloka
{मेदिनी कम्पिता सर्वा सशैलवनकानना}
{शक्तेनापि समुद्धर्तुं कम्पिता साऽभवत्तदा}


\twolineshloka
{रक्षिता स्कन्दराजस्य धर्षणा प्रभविष्णुना}
{तां कम्पयित्वा भगवान्प्रह्लादमिदमब्रवीत्}


\twolineshloka
{पश्य वीर्यं कुमारस्य नैतदन्यः करिष्यति}
{सोऽमृष्यमाणस्तद्वाक्यं समुद्धरणनिश्चितः}


\twolineshloka
{जग्राह तां तदा शक्तिं न चैनामभ्यकम्पयत्}
{नादं महान्तं मुक्त्वा स मूर्च्छितो गिरिमूर्घनि}


\twolineshloka
{विह्वलः प्रापतद्भूमौ हिरण्यकशिपोः सुतः}
{तत्रोत्तरां दिशं गत्वा शैलराजस्य पार्श्वतः}


\twolineshloka
{तपोऽतप्यत दुर्घर्षं तात नित्यं वृषध्वजः}
{पावकेन परिक्षिप्तं दीप्यता यस्य चाश्रमम्}


\twolineshloka
{आदित्यपर्वतं नाम दुर्घर्षमकृतात्मभिः}
{न तत्र शक्यते गन्तुं यक्षराक्षसदानवैः}


\twolineshloka
{दशयोजनविस्तारमग्निज्वालसमावृतम्}
{भगवान्पावकस्तत्र स्वयं तिष्ठति वीर्यवान्}


\twolineshloka
{सर्वान्विघ्नान्प्रशमयन्महादेवस्य धीमतः}
{दिव्यं वर्षसहस्रं हि पादेनैकेन तिष्ठतः}


\twolineshloka
{देवान्संतापयंस्तत्र महादेवो महाव्रतः}
{ऐन्द्रीं तु दिशमास्थाय शैलराजस्य धीमतः}


\twolineshloka
{विविक्ते पर्वततटे पाराशर्यो महातपाः}
{वेदानध्यापयामास व्यासः शिष्यान्महातपाः}


\twolineshloka
{सुमन्तुं च महाभागं वैशंपायनमेव च}
{जैमिनिं च महाप्राज्ञं पैलं चापि तपस्विनम्}


\threelineshloka
{एभिः शिष्यैः परिवृतो व्यास आस्ते महातपाः}
{तत्राश्रमपदं रम्यं ददर्श पितुरुत्तमम्}
{आरणेयो विशुद्धात्मा नभसीव दिवाकरः}


\twolineshloka
{अथ व्यासः परिक्षिप्तं ज्वलन्तमिव पावकम्}
{ददृशे सुतमायान्तं दिवाकरसमप्रभम्}


\twolineshloka
{असज्जमानं वृक्षेषु शैलेषु विषयेषु च}
{योगयुक्तं महात्मानं यथा बाणं गुणच्युतम्}


\twolineshloka
{सोऽभिगम्य पितुः पादावगृह्णादरणीसुतः}
{यथोपजोषं तैश्चापि समागच्छन्महामुनिः}


\twolineshloka
{ततो निवेदयामास पित्रे सर्वमशेषतः}
{शुको जनकराजेन संवादं प्रीतमानसः}


\twolineshloka
{एवमध्यापयञ्शिष्यान्व्यासः पुत्रं च वीर्यवान्}
{उवास हिमवत्पृष्ठे पाराशर्यो महामुनिः}


\twolineshloka
{ततः कदाचिच्छिष्यास्तं परिवार्यावतस्थिरे}
{वेदाध्ययनसंपन्नाः शान्तात्मानो जितेन्द्रियाः}


\threelineshloka
{वेदेषु निष्ठां संप्राप्य साङ्गेष्वपि तपस्विनः}
{अथोचुस्ते तदा व्यासं शिष्याः प्राञ्जलयो गुरुम् ॥शिष्या ऊचुः}
{}


\twolineshloka
{महता तेजसा युक्ता यशसा चापि वर्धिताः}
{एकं त्विदानीमिच्छामो गुरुणाऽनुग्रहं कृतम्}


\twolineshloka
{इति तेषां वचः श्रुत्वा ब्रह्मर्षिस्तानुवाद ह}
{उच्यतामिति तद्वत्सा यद्वः कार्यं प्रियं मया}


\twolineshloka
{एतद्वाक्यं गुरोः श्रुत्वा शिष्यास्ते हृष्टमानसाः}
{पुनः प्राञ्जलयो भूत्वा प्रणम्य शिरसा गुरुम्}


\twolineshloka
{ऊचुस्ते सहिता राजन्निदं वचनमुत्तमम्}
{यदि प्रीत उपाध्यायो धन्याः स्मो मुनिसत्तम}


\twolineshloka
{काङ्क्षामस्तु वयं सर्वे वरं दत्तं महर्षिणा}
{पष्ठः शिष्यो न ते ख्यातिं गच्छेदत्र प्रसीद नः}


\twolineshloka
{चत्वारस्ते वयं शिष्या गुरुपूत्रश्च पञ्चमः}
{इह वेदाः प्रतिष्ठेरन्नेष नः काङ्क्षितो वरः}


\twolineshloka
{शिष्याणां वचनं श्रुत्वा व्यासो वेदार्थतत्त्ववित्}
{पराशरात्मजो धीमान्परलोकार्थचिन्तकः}


\twolineshloka
{उवाच शिष्यान्धर्मात्मा धर्म्यं नैःश्रेयसं वचः}
{ब्राह्मणाय सदा देयं ब्रह्म शुश्रूषवे तथा}


\twolineshloka
{ब्रह्मलोके निवासं यो ध्रुवं समभिकाङ्क्षते}
{भवन्तो बहुलाः सन्तु वेदो विस्तार्यतामयम्}


\threelineshloka
{नाशिष्ये संप्रदातव्यो नाव्रते नाकृतात्मनि}
{एते शिष्यगुणाः सर्वे विज्ञातव्या यथार्थतः}
{नापरीक्षितचारित्रे विद्या देया कथंचन}


\twolineshloka
{यथा हि कनकं शुद्धं तापच्छेदनिकर्षणैः}
{परीक्षेत तथा शिष्यानीक्षेत्कुलगुणादिभिः}


\twolineshloka
{न नियोज्याश्च वः शिष्या अनियोगे महाभये}
{यथामति यथापाठं तथा विद्या फलिष्यति}


\twolineshloka
{सर्वस्तरतु दुर्गाणि सर्वो भद्राणि पश्यतु}
{श्रावयेच्चतुरो वर्णान्कृत्वा ब्राह्मणमग्रतः}


\twolineshloka
{वेदस्याध्ययनं हीदं तच्च कार्यं महत्स्मृतम्}
{स्तुत्यर्थमिह देवानां वेदाः सृष्टाः स्वयंभुवा}


\twolineshloka
{यो निर्वदेत संमोहाद्ब्राह्मणां वेदपारगम्}
{सोऽभिध्यानाद्ब्राह्मणस्य पराभूयादसंशयम्}


\twolineshloka
{यश्चाधर्मेण विब्रूयाद्यश्चाधर्मेण पृच्छति}
{तयोरन्यतरः प्रैति विद्वेषं चाधिगच्छति}


\twolineshloka
{एतद्वः सर्वमाख्यातं स्वाध्यायस्य विधिं प्रति}
{उपकुर्याच्च शिष्याणामेतच्च हृद्वि वो भवेत्}


\chapter{अध्यायः ३३६}
\twolineshloka
{भीष्म उवाच}
{}


\twolineshloka
{एतच्छ्रुत्वा गुरोर्वाक्यं व्यासशिष्या महौजसः}
{अन्योन्यं हृष्टमनसः परिषस्वजिरे तदा}


\twolineshloka
{उक्ताः स्मो यद्भगवता तदात्वायतिसंहितम्}
{तन्नो मनसि संरूढं करिष्यामस्तथा च तत्}


\twolineshloka
{अन्योन्यं संविभाष्यैवं सुप्रीतमनसः पुनः}
{विज्ञापयन्ति स्म गुरुं पुनर्वाक्यविशारदाः}


\twolineshloka
{शैलादस्मान्महीं गन्तुं काङ्क्षितं नो महामुने}
{वेदाननेकधा कर्तुं यदि ते रुचितं प्रभो}


\twolineshloka
{शिष्याणां वचनं श्रुत्वा पराशरसुतः प्रभुः}
{प्रत्युवाच ततो वाक्यं धर्मार्थसहितं हितम्}


\twolineshloka
{क्षितिं वा देवलोकं वा गम्यतां यदि रोचते}
{अप्रमादश्च वः कार्यो ब्रह्म हि प्रचुरच्छलम्}


\twolineshloka
{तेऽनुज्ञातास्ततः सर्वे गुरुणा सत्यवादिना}
{जग्मुः प्रदक्षिणं कृत्वा व्यासं मूर्ध्नाऽभिवाद्य च}


\twolineshloka
{अवतीर्य महीं तेऽथ चातुर्होत्रमकल्पयन्}
{संयाजयन्तो विप्रांश्च राजन्यांश्च विशस्तथा}


\twolineshloka
{पूज्यमाना द्विजैर्नित्यं मोदमाना गृहे रताः}
{याजनाध्यापनरताः श्रीमन्तो लोकविश्रुताः}


\twolineshloka
{अवतीर्णेषु शिष्येषु व्यासः पुत्रसहायवान्}
{तूष्णीं ध्यानपरो धीमानेकान्ते समुपाविशत्}


\twolineshloka
{`एतस्मिन्नेव काले तु देवर्षिर्नारदस्तथा}
{हिमवन्तमगं द्रष्टुं सिद्धचारणसेवितम् ॥'}


\twolineshloka
{तं ददर्शाश्रमपदे नारदः सुमहातपाः}
{अथैनमब्रवीत्काले मधुराक्षरया गिरा}


\twolineshloka
{भोभो महर्षे वासिष्ठ ब्रह्मघोषो न वर्तते}
{एको ध्यानपरस्तूष्णीं किमास्से चिन्तयन्निव}


\twolineshloka
{ब्रह्मघोषैर्विरहितः पर्वतोऽयं न शोभते}
{रजसा तमसा चैव सोमः सोपप्लवो यथा}


\twolineshloka
{न भ्राजते यथापूर्वं निषादानामिवालयः}
{देवर्षिगणजुष्टोऽपि वेदध्वनिविनाकृतः}


\twolineshloka
{ऋषयश्च हि देवाश्च गन्धर्वाश्च महौजसः}
{वियुक्ता ब्रह्मघोषेण न भ्राजन्ते यथा पुरा}


\twolineshloka
{नारदस्य वचः श्रुत्वा कृष्णद्वैपायनोऽब्रवीत्}
{महर्षे यत्त्वया प्रोक्तं वेदवादविचक्षण}


\twolineshloka
{एतन्मनोऽनुकूलं मे भवानर्हसि भाषितुम्}
{सर्वज्ञः सर्वदर्शी च सर्वत्र च कुतूहली}


\twolineshloka
{त्रिषु लोकेषु यद्वृत्तं सर्वं तव मते स्थितम्}
{तदाज्ञापय विप्रर्षे ब्रूहि किं करवाणि ते}


\threelineshloka
{यन्मया समनुष्ठेयं ब्रह्मर्षे तदुदाहर}
{वियुक्तस्येह शिष्यैर्मे नातिहृष्टमिदं मनः ॥नारद उवाच}
{}


\twolineshloka
{अनाम्नायमला वेदा ब्राह्मणस्याव्रतं मलम्}
{मलं पृथिव्या बाह्वीकाः स्त्रीणां कौतूहलं मलम्}


\threelineshloka
{अधीयतां भवान्वेदान्सार्घं पुत्रेण धीमता}
{विधुन्वन्ब्रह्मघोषेण रक्षोभयकृतं तमः ॥भीष्म उवाच}
{}


\twolineshloka
{नारदस्य वचः श्रुत्वा व्यासः परमधर्मवित्}
{तथेत्युक्त्वाऽथ संहृष्टो वेदाभ्यासे दृढव्रतः}


% Check verse!
`* उवाच च महाप्राज्ञं नारदं पुनरेव हि
\threelineshloka
{मलं पृथिव्या बाह्लीका इत्युक्तमधुना त्वया}
{कीदृशाश्चैव वाह्लीका ब्रूहि मे वदतां वर ॥नारद उवाच}
{}


\twolineshloka
{अस्यां पृथिव्यां चत्वारो देशाः पापजनैर्वृताः}
{युगन्धरस्तु प्रथमस्तथा भूतिलकः स्मृतः}


\twolineshloka
{अच्युतच्छल इत्युक्तस्तृतीयः पारकृत्तमः}
{चतुर्थस्तु महापापो बाह्लीक इति संज्ञितः}


\twolineshloka
{भृगोष्ट्रगर्दभक्षीरं पिबन्त्यस्य युगन्धरे}
{एवकर्णास्तु दृश्यन्ते जना वै ह्यच्युतस्थले}


\twolineshloka
{मेहन्ति च मलं पापा विसृजन्ति जलेषु वै}
{नित्यं भूतिलकेत्यन्नं तज्जलं च पिबन्ति च}


\threelineshloka
{हरिबाह्यास्तु बाहीका न स्मरन्ति हरिं क्वचित्}
{ऐहलौकिकमोक्षं ते मांसशोणितवर्धनाः}
{वृथा जाता भविष्यन्ति बाह्लीका इति विश्रुताः}


\twolineshloka
{पुष्कराहारनिरताः पिशाचा यदभाषते}
{मुसुण्ठीं परिगृह्योग्रां तच्छृणुष्व महामुने}


\twolineshloka
{ब्राह्मणीं बहुपुत्रां तां पुष्करे स्नातुमागताम्}
{युगन्धरे पयः पीत्वा ह्युचिता ह्यच्युतस्थले}


\twolineshloka
{तथा भूतिलके स्नात्वा बाह्लीकांश्च निरीक्ष्य वै}
{आगताऽसि तथा स्नातुं कथं स्वर्गं न गच्छसि}


\twolineshloka
{इत्युक्त्वा ब्राह्मणीभाण्डं पोथयित्वा मुसुण्ठिना}
{उवाच क्रोधताम्राक्षी पिशाची तीर्थपालिका}


\twolineshloka
{एतत्तु ते दिवावृत्तं रात्रौ वृत्तमथान्यथा}
{गच्छ बाह्लीकसंसर्गादशुचित्वं न संशयः}


\twolineshloka
{यद्द्विषन्ति महात्मानं न स्मरन्ति जनार्दनम्}
{न तेषां पुण्यतीर्थेषु गतिः संसर्गिणामपि}


\twolineshloka
{उद्युक्ता ब्राह्मणी भीता प्रतियाता सुतैः सह}
{स्वदेहस्था जजापैवं सपुत्रा ध्यानतत्परा}


\twolineshloka
{अनन्तस्य हरेः शुद्धं नाम वै द्वादशाक्षरम्}
{वत्सरत्रितये पूर्णे ब्राह्मणी पुनरागता}


\twolineshloka
{सपुत्रा पुष्करद्वारं पिशाच्याह तथागतम्}
{नमस्ते ब्राह्मणि शुभे पूताऽहं तव दर्शनात्}


\twolineshloka
{कुरु तीर्थाभिषेकं च सपुत्रा पापवर्जिता}
{हरेर्नाम्ना च मां साध्वी जलेन स्प्रष्टुमर्हसि}


\twolineshloka
{इत्युक्ता ब्राह्मणी हृष्टा पुत्रैः सह शुभव्रता}
{जलेन प्रोक्षयामास द्वादशाक्षरसंयुतम्}


\twolineshloka
{तत्क्षणादभवच्छुद्धा पिशाची दिव्यरूपिणी}
{अप्सरा ह्यभवद्दिव्या गता स्वर्लोकमुत्तभम्}


\twolineshloka
{ब्राह्मणी चैव कालेन वासुदेवपरायणा}
{सपुत्रा चागता स्थानमच्युतस्य शुभं परम्}


% Check verse!
एतत्ते कथितं विद्वन्मुने कालोऽयमागतः
\twolineshloka
{गमिष्येऽहं महाप्राज्ञ आगमिष्यामि वै पुनः}
{इत्युक्त्वा स जगामाथ नारदो वदतांवरः}


\threelineshloka
{द्वैपायनस्तु भगवांस्तच्छ्रुत्वा मुनिसत्तमात्}
{'शुकेन सह पुत्रेण वेदाभ्यासमथाकरोत्}
{स्वरेणोच्चैः सशैक्ष्येण लोकानापूरयन्निव}


\twolineshloka
{तयोरभ्यसतोरेव नानाधर्मप्रवादिनोः}
{वातोऽतिमात्रं प्रववौ समुद्रानिलवेजितः}


\twolineshloka
{ततोऽनध्याय इति तं व्यासः पुत्रमवारयत्}
{शुको वारितमात्रस्तु कौतूहलसमन्वितः}


\twolineshloka
{अपृच्छत्पितरं ब्रह्मन्कुतो वायुरभूदयम्}
{आख्यातुमर्हति भवान्वायोः सर्वं विचेष्टितम्}


\twolineshloka
{शुकस्यैतद्वचः श्रुत्वा व्यासः परमधर्मवित्}
{अनध्यायनिमित्तेऽस्मिन्निदं वचनमब्रवीत्}


\twolineshloka
{दिव्यं ते चक्षुरुत्पन्नं स्वस्थं ते निश्चलं मनः}
{तमसा रजसा चापि त्यक्तः सत्वे व्यवस्थितः}


\twolineshloka
{आदर्शे रस्वामिव च्छायां पश्यस्यात्मानमात्मना}
{न्यस्यात्मनि स्वयं चेतो बुद्ध्या समनुचिन्तय}


\twolineshloka
{देवयानपथो विष्णुः पितृयानपथो रविः}
{द्वावेतौ प्रेत्य पन्थानौ दिवं चाधश्च गच्छतः}


\twolineshloka
{पृथिव्यामन्तरिक्षे च यत्र संवान्ति वायवः}
{सप्तैते वायुमार्गा वै तान्निबोधानुपूर्वशः}


\twolineshloka
{तत्र देवगणाः साध्याः संबभूवुर्महाबलाः}
{तेषामप्यभवत्पुत्रः समानो नाम दुर्जयः}


\twolineshloka
{उदानस्तस्य पुत्रोऽभूद्व्यानस्तस्याभवत्सुतः}
{अपानश्च ततो ज्ञेयः प्राणश्चापि तताऽपरः}


\twolineshloka
{अनपत्योऽभवत्प्राणो दुर्घर्षः शत्रुतापनः}
{पृथक्कर्माणि तेषां तु प्रवक्ष्यामि यथातथम्}


\twolineshloka
{प्राणिनां सर्वतो वायुश्रेष्टां वर्तयते पृथक्}
{प्राणनाच्चैव भूतानां प्राण इत्यभिधीयते}


\twolineshloka
{प्रेरयत्यभ्रसंघातान्धूमजांश्चोष्मजांश्च यः}
{प्रथमः प्रथमे मार्गे आवहो नाम योऽनिलः}


\twolineshloka
{अम्बरे स्नेहमभ्रेभ्यस्तटिद्भ्यश्च महाद्युतिः}
{प्रवहो नाम संवाति द्वितीयश्च सतोयदः}


\twolineshloka
{उदयं ज्योतिषां शश्वत्सोमादीनां करोति यः}
{अन्तर्देहेषु चोदानं यं वदन्ति मनीषिणः}


\twolineshloka
{यश्चतुर्भ्यः समुद्रेभ्यो वायुर्धारयते जलम्}
{उद्धृत्याददते चापो जीमूतेभ्योऽम्बरेऽनिलः}


\twolineshloka
{योऽद्भिः संयोज्य जीमूतान्पर्जन्याय प्रयच्छति}
{उद्वहो नाम वर्षिष्ठस्तृतीयः स सदागतिः}


\twolineshloka
{समुह्यमाना बहुधा येन नीताः पृथग्घनाः}
{वर्षमोक्षकृतारम्भास्ते भवन्ति घनाघनाः}


\twolineshloka
{संहता येन चाविद्धा भवन्ति नदनान्तराः}
{रक्षणार्थाय संभूता मेघत्वमुपश्यान्ति च}


\twolineshloka
{योऽसौ वहति देवानां विमानानि विहायसा}
{चतुर्थः संवहो नाम वायुः स गिरिमर्दनः}


\twolineshloka
{येन वेगवता तूर्णं रूक्षेणारुजता रसान्}
{वायुना विहता मेघा न भवन्ति बलाहकाः}


\twolineshloka
{दारुणोत्पातसंचारो नभसः स्तनयित्नुमान्}
{पञ्चमः स महावेगो विवहो नाम मारुतः}


\twolineshloka
{यस्मिन्पारिप्लवा दिव्या भवन्त्यापो विहायसा}
{पुण्यं चाकाशगङ्गायास्तोयं विष्टभ्य तिष्ठति}


\twolineshloka
{दूरात्प्रतिहतो यस्मिन्नेकरश्मिर्दिवाकरः}
{यो निरंशुः सहस्रस्य येन भाति वसुंधरा}


\twolineshloka
{यस्मादाप्यायते सोमो योनिर्दिव्योऽमृतस्य यः}
{षष्ठः पिरवहो नाम स वायुर्जयतांवरः}


\twolineshloka
{सर्वप्राणभृतां प्राणान्योऽनुकाले निरस्यति}
{यस्य वर्त्मानुवर्तेते मृत्युवैवस्वतावुभौ}


\twolineshloka
{सम्यगन्वीक्षतां बुद्ध्या शान्तयाऽध्यात्मचिन्तकाः}
{ध्यानाभ्यासाभिरामाणां योऽमृतत्वाय कल्पते}


\twolineshloka
{यं समासाद्य वेगेन दिशामन्तं प्रपेदिरे}
{दक्षस्य दशपुत्राणां सहस्राणि प्रजापतेः}


\twolineshloka
{येन सृष्टः पराभूतो यात्येव न निवर्तते}
{परावहो नाम परो वायुः स दुरतिक्रमः}


\twolineshloka
{एवमेतेऽदितेः पुत्रा मारुताः परमाद्भुताः}
{अनारतं ते संवान्ति सर्वगाः सर्वधारिणः}


\twolineshloka
{एतत्तु महदाश्चर्यं यदयं पर्वतोत्तमः}
{कम्पितः सहसा तेन वायुनाऽतिप्रवायता}


\twolineshloka
{विष्णोर्निः श्वासवातोऽयं यदा वेगसमीरितः}
{सहसोदीर्यते तात जगत्प्रव्यथते तदा}


\twolineshloka
{तस्माद्ब्रह्मविदो ब्रह्म नाधीयन्तेऽतिवायति}
{वायोर्वायुभयं ह्युक्तं ब्रह्म तत्पीडितं भवेत्}


\twolineshloka
{एतावदुक्त्वा वचनं पराशरसुतः प्रभुः}
{उक्त्वा पुत्रमधीष्वेति व्योमगङ्गामगात्तदा}


\chapter{अध्यायः ३३७}
\twolineshloka
{भीष्म उवाच}
{}


\twolineshloka
{एतस्मिन्नन्तरे भूते नारदः पुनरागमत्}
{शुकं स्वाध्यायनिरतं वेदार्थान्प्रष्टुमीप्सवा}


\twolineshloka
{देवर्षि तु शुको दृष्ट्वा नारदं समुपस्यितम्}
{अर्ध्यपूर्वेण विधिवा वेदोक्तेनाभ्यपूजयत्}


\twolineshloka
{नारदोऽयाजवीत्प्रीतो ब्रूहि ब्रह्मविदां वर}
{केन त्वां श्रेयसा वत्स योजयामीति हृष्टवत्}


\threelineshloka
{नारदस्य वचः श्रुत्वा शुकः प्रोवाच भारत}
{अस्मिँल्लोके हितं यत्स्वात्तेन मां योक्तुमर्हसि ॥नारद उवाच}
{}


\twolineshloka
{तत्त्वं जिज्ञासतां पूर्वनृषीणां भावितात्मनाम्}
{सनत्कुमारो भगवानिदं वतनमब्रवीत्}


\twolineshloka
{नास्ति विद्यासगं चक्षुर्नास्ति सत्यसमं तपः}
{नास्ति रागसमं दुःखं नास्ति त्यागसमं सुखम्}


\twolineshloka
{निवृत्तिः कर्मणा पापात्सततं पुण्यशीलता}
{सद्वृत्तिः सद्वदाचारः श्रेय एवदनुत्तमम्}


\twolineshloka
{नानुवयसुखं प्राप्य यः सज्जति न मुच्यते}
{नालं स दुःखमोक्षाय संयोगो दुःखलक्षणम्}


\twolineshloka
{सक्तस्य बुद्धिश्चलति मोहजालविवर्धनी}
{मोहवालावृतो दुःखमिह चामुत्र सोऽश्नुते}


\twolineshloka
{सर्वोपागात्तु कामस्व क्रोधस्य च विनिग्रहः}
{कार्यः श्रेयोर्थिना तौ हि श्रेयोघातार्थमुद्यतौ}


\twolineshloka
{नित्यं क्रोधात्तपो रक्षेच्छ्रियं रक्षेच्च मत्सरात्}
{विद्यां मानावमानाभ्यामात्मानं तु प्रमादतः}


\twolineshloka
{आनृशंस्यं परो धर्मः क्षमा च परमं बलम्}
{आत्मज्ञानं परं ज्ञानं न सत्याद्विद्यते परम्}


\twolineshloka
{सत्यस्य वचनं श्रेयः सत्यादपि हितं वदेत्}
{यद्भूतहितमत्यन्तमेतत्सत्यं मतं मम}


\twolineshloka
{सर्वारम्भपरित्यागी निराशीर्निष्परिग्रहः}
{येन सर्वं परित्यक्तं स विद्वान्स च पण्डितः}


\twolineshloka
{इन्द्रियैरिन्द्रियार्थान्यश्चरत्यात्मवशैरिह}
{आसज्जमानः शान्तात्मा निर्विकारः समाहितः}


\twolineshloka
{अत्मभूतैरतद्भूतः सह चैव विनैव च}
{स विमुक्तः परं श्रेयो नचिरेणाधिगच्छति}


\twolineshloka
{अदर्शनमसंस्पर्शस्तथाऽसंभाषणं तदा}
{यस्य भूतैः सह मुने स श्रेयो विन्दते परम्}


\twolineshloka
{न हिंस्यात्सर्वभूतानि मैत्रायगणतश्चरेत्}
{नेदं जन्म समासाद्य वैरं कुर्वीत केनचित्}


\twolineshloka
{आकिञ्चन्यं सुसंतोषो निराशीस्त्वमचापलम्}
{एतदाहुः परं श्रेय आत्मज्ञस्य जितात्मनः}


\twolineshloka
{परिग्रहं परित्यज्य भव तात जितेन्द्रियः}
{अशोकं स्थानमातिष्ठ इह चामुत्र चाभयम्}


\twolineshloka
{निरामिषा न शोचन्ति त्यजेदामिषमात्मनः}
{परित्यज्यामिषं सौभ्य दुःखतापाद्विमोक्ष्यसे}


\twolineshloka
{तपोनित्येन दान्तेन मुनिना संयतात्मना}
{अजितं जेतुकामेन भाव्यं सङ्गेष्वसङ्गिना}


\twolineshloka
{गुणसङ्गेष्वनासक्त एकचर्यारतः सदा}
{ब्राह्मणो नचिरादेव सुखमायात्यनुत्तमम्}


\twolineshloka
{द्वन्द्वारामेषु भूतेषु य एको रमते मुनिः}
{विद्धि प्रज्ञानतृप्तं तं ज्ञानतृप्तो न शोचति}


\twolineshloka
{शुभैर्लभति देवत्वं व्यामिश्रेर्जन्म मानुषम्}
{अशुभैश्चाप्यधोजन्म कर्मभिर्लभतेऽवशः}


\twolineshloka
{तत्र मृत्युवशो दुःखैः सततं समभिद्रुतः}
{संसारे पच्यते जन्तुस्तत्कथं नावबुध्यसे}


\twolineshloka
{अहिते हितसंज्ञस्त्वमध्रुवे ध्रुवसंज्ञकः}
{अनर्थे चार्थसंज्ञस्त्वं किमर्थं नावबुध्यसे}


\twolineshloka
{संवेष्ट्यमानं बहुभिर्मोहात्तन्तुभिरात्मजैः}
{कोशकार इवात्मानं वेष्टयन्नावबुध्यसे}


\twolineshloka
{अलं परिग्रहेणेह दोषवान्हि परिग्रहः}
{कृमिर्हि कोशकारस्तु बध्यते स परिग्रहात्}


\twolineshloka
{पुत्रदारकुटुम्बेषु सक्ताः सीदन्ति जन्तवः}
{सरःपङ्कार्णवे मग्ना जीर्णा वनगजा इव}


\twolineshloka
{महाजालसमाकृष्टान्स्थले मत्स्यानिवोद्धृतान्}
{मोहजालसमाकृष्टान्पश्य जन्तून्सुदुःखितान्}


\twolineshloka
{कुटुम्बं पुत्रदारांश्च शरीरं संचयाश्च ये}
{पारक्यमध्रुवं सर्वं किं स्वं सुकृतदुष्कृतम्}


\twolineshloka
{यदा सर्वान्परित्यज्य गन्तव्यमवशेन ते}
{अनर्थे किं प्रसक्तस्त्वं समर्थं नानुतिष्ठसि}


\twolineshloka
{अविश्रान्तमनालम्बमपाथेयमदैशिकम्}
{तमः कान्तारमध्वानं कथमेको गमिष्यसि}


\twolineshloka
{न हि त्वां प्रस्थितं कश्चित्पृष्ठतोऽनुगमिष्यति}
{सुकृतं दुष्कृतं च त्वां यास्यन्तमनुयास्यतः}


\twolineshloka
{विद्या कर्म च शौचं च ज्ञानं च बहुविस्तरम्}
{अर्थार्थमनुसार्यन्ते सिद्धार्थस्य विमुच्यते}


\twolineshloka
{निबन्धनी रज्जुरेषा या ग्रामे वसतो रतिः}
{छित्त्वैतां सुकृतो यान्ति नैनां छिन्दन्ति दुष्कृतः}


\twolineshloka
{रूपकूलां मनःस्रोतां स्पर्शद्वीपां रसावहाम्}
{गन्धपङ्कां शब्दजलां स्वर्गमार्गदुरावहाम्}


\twolineshloka
{क्षमारित्रां सत्यमयीं धर्मस्थैर्यपदाङ्कुराम्}
{त्यागवाताध्वगां शीघ्रां नौतार्यां तां नदीं तरेत्}


\twolineshloka
{त्यज धर्ममधर्मं च उभे सत्यानृते त्यज}
{उभे सत्यानृते त्यक्त्वा येन त्यजसि तं त्यज}


\twolineshloka
{त्यज धर्ममसंकल्पादधर्मं चाप्यलिप्सया}
{उभे सत्यानृते बुद्ध्या बुद्धिं परमनिश्चयात्}


\twolineshloka
{अस्थिस्थूणं स्नायुयुतं मांसशोणितलेपनम्}
{चर्मावनद्धं दुर्गन्धिं पूर्णं मूत्रपुरीषयोः}


\twolineshloka
{जराशोकसमाविष्टं रोगायतनमातुरम्}
{रजस्वलमनित्यं च भूतावासमिमं त्यज}


\twolineshloka
{इदं विश्वं जगत्सर्वमजगच्चापि यद्भवेत्}
{महाभूतात्मकं सर्वं महद्यत्परमाणु च}


\threelineshloka
{`महाभूतानि खं वायुरग्निरापस्तथा मही}
{षष्ठं तु चेतना या तु आत्मा सप्तममुच्यते}
{'अष्टमं तु मनो ज्ञेयं बुद्धिस्तु नवमी स्मृता}


\twolineshloka
{इन्द्रियाणि च पञ्चैव तमः सत्वं रजस्तथा}
{इत्येष सप्तदशको राशिरव्यक्तसंज्ञकः}


\twolineshloka
{सर्वैरिहेन्द्रियार्थैश्च व्यक्ताव्यक्तैर्हि संहितः}
{चतुर्विशक इत्येष व्यक्ताव्यक्तमयो गणः}


\twolineshloka
{एतैः सर्वैः समायुक्तः पुमानित्यभिधीयते}
{त्रिवर्गं तु सुखं दुःखं जीवितं मरणं तथा}


\twolineshloka
{य इदं वेद तत्त्वेन स वेद प्रभवाप्ययौ}
{पारम्पर्येह बोद्धव्यं ज्ञानानां यच्च किंचन}


\twolineshloka
{इन्द्रियैर्गृह्यते यद्यत्तत्तद्व्यक्तमिति स्थितिः}
{अव्यक्तमिति विज्ञेयं लिङ्गग्राह्यमतीन्द्रियम्}


\twolineshloka
{इन्द्रियैर्नियतैर्देही धाराभिरिव तर्प्यते}
{लोके विततमात्मानं लोकांश्चात्मनि पश्यति}


\twolineshloka
{परावरदृशः शक्तिर्ज्ञानमूला न नश्यति}
{पश्यतः सर्वभूतानि सर्वावस्थासु सर्वदा}


\twolineshloka
{ब्रह्मभूतस्य संयोगो नाशुभेनोपपद्यते}
{ज्ञानेन विविधान्क्लेशानतिवृत्तस्य मोहजान्}


\twolineshloka
{लोके बुद्धिप्रकाशेन लोकमार्गो न रिष्यते}
{अनादिनिधनज्ञं तमात्मनि स्थितमव्ययम्}


\twolineshloka
{अकर्तारममूर्तं च भगवानाह तीर्थवित्}
{यो जन्तुः स्वकृतैस्तैस्तैः कर्मभिर्नित्यदुःखितः}


\twolineshloka
{स दुःखप्रतिघातार्थं हन्ति जन्तूननेकधा}
{ततः कर्म समादत्ते पुनरन्यन्नवं बहु}


\twolineshloka
{तप्यतेऽथ पुनस्तेन भुक्त्वा पथ्यमिवातुरः}
{अजस्रमेव मोहान्धो दुःखेषु सुखसंज्ञितः}


\twolineshloka
{बध्यते मथ्यते चैव कर्मभिर्मन्थवत्सदा}
{ततो निबद्धः स्वां योनिं कर्मणामुदयादिह}


\twolineshloka
{परिभ्रमति संसारं चक्रवद्बहुवेदनः}
{सत्वं निर्वृत्तबन्धस्तु निवृत्तश्चापि कर्मतः}


\threelineshloka
{सर्ववित्सर्वजित्सिद्धौ भव भावविवर्जितः}
{संयमेन नवं बन्धं निवर्त्य तपसो बलात्}
{संप्राप्ता बहवः सिद्धिमप्यबाधां सुखोदयाम्}


\chapter{अध्यायः ३३८}
\twolineshloka
{नारद उवाच}
{}


\twolineshloka
{अशोकं शोकनाशार्थं शास्त्रं शान्तिकरं शिवम्}
{निशम्य लभते बुद्धिं तां लब्ध्वा सुखमेधते}


\twolineshloka
{शोकस्थानसहस्राणि भयस्थानशतानि च}
{दिवसेदिवसे मूढमाविशन्ति न पण्डितम्}


\twolineshloka
{तस्मादनिष्टनाशार्थमितिहासं निबोध मे}
{तिष्ठते चेद्वशे बुद्धिर्लभते शोकनाशनम्}


\twolineshloka
{अनिष्टसंप्रयोगाच्च विप्रयोगात्प्रियस्य च}
{मनुष्या मानसैर्दुःखैर्युज्यन्ते स्वल्पबुद्ध्यः}


\twolineshloka
{द्रव्येषु समतीतेषु ये गुणास्तान्न चिन्तयेत्}
{न तानाद्रियमाणस्य स्नेहबन्धः प्रमुच्यते}


\twolineshloka
{दोषदर्शी भवेत्तत्र यत्र रागः प्रवर्तते}
{अनिष्ट्वद्धितं पश्येत्तथा क्षिप्रं विरज्यते}


\twolineshloka
{नार्थो न धर्मो न यशो योऽतीतमनुशोचति}
{अप्यभावेन युज्येत तच्चास्य न निवर्तते}


\twolineshloka
{गुणैर्भूतानि युज्यन्ते वियुज्यन्ते तथैव च}
{सर्वाणि नैतदेकस्य शोकस्थानं हि युज्यते}


\twolineshloka
{मृतं वा यदि वा नष्टं योऽतीतमनुशोचति}
{दुःखेन लभते दुःखं द्वावनर्थौ प्रपद्यते}


\twolineshloka
{नाश्रु कुर्वन्ति ये बुद्ध्या दृष्ट्वा लोकेषु संततिम्}
{सम्यक्प्रपश्यतः सर्वं नाश्रुकर्मोपपद्यते}


\twolineshloka
{दुःखोपघाते शारीरे मानसे चाप्युपस्थिते}
{यस्मिन्न शक्यते कर्तुं यत्नस्तन्नानुचिन्तयेत्}


\twolineshloka
{भैषज्यमेतद्दुःखस्य यदेतन्नानुचिन्तयेत्}
{चिन्त्यमानं हि न व्येति भूयश्चापि प्रवर्धते}


\twolineshloka
{प्रज्ञया मानसं दुःखं हन्याच्छारीरमौषधैः}
{एतद्विज्ञानमसामर्थ्यं न बालैः समतामियात्}


\twolineshloka
{अनित्यं यौवनं रूपं जीवितं द्रव्यसंचयः}
{आरोग्यं प्रियसंसर्गो गृध्येत्तत्र न पण्डितः}


\twolineshloka
{न जानपदिकं दुःखमेकः शोचितुमर्हति}
{अशोचन्प्रतिकुर्वीत यदि पश्येदुपक्रमम्}


\twolineshloka
{सुखाद्बहुतरं दुःखं जीविते नात्र संशयः}
{स्निग्धत्वं चेन्द्रियार्थेषु मोहान्मरणमप्रियम्}


\twolineshloka
{परित्यजति यो दुःखं सुखं वाऽप्युभयं नरः}
{अभ्येति ब्रह्म सोत्यन्तं न तं शोचन्ति पण्डिताः}


\twolineshloka
{त्यजन्ते दुःखमर्था हि पालनेन च ते सुखाः}
{दुःखेन चाधिगम्यन्ते नाशमेषां न चिन्तयेत्}


\twolineshloka
{अन्यामन्यां धनावस्थां प्राप्य वैशेषिकीं नराः}
{अतृप्ता यान्ति विध्वंसं संतोषं यान्ति पण्डिताः}


\twolineshloka
{सर्वे क्षयान्ता निचयाः पतनान्ताः समुच्छ्रयाः}
{संयोगा विप्रयोगान्ता मरणान्तं हि जीवितम्}


\twolineshloka
{अन्तो नास्ति पिपासायास्तुष्टिस्तु परमं सुखम्}
{तस्मात्संतोषमेवेह धनं पश्यन्ति पण्डिताः}


\twolineshloka
{निमेषमात्रमपि हि वयो गच्छन्न तिष्ठति}
{स्वशरीरेष्वनित्येषु नित्यं किमनुचिन्तयेत्}


\twolineshloka
{भूतेषु भावं संचिन्त्य ये बुद्ध्वा मनसः परम्}
{न शोचन्ति गताध्वानः पश्यन्तः परमां गतिं}


\twolineshloka
{संचिन्वानकमेवैनं कामानामवितृप्तकम्}
{व्याघ्रः पशुमिवासाद्य मृत्युरादाय गच्छति}


\twolineshloka
{तथाऽप्युपायं संपश्येद्दुःखस्य परिमोक्षणे}
{अशोचन्नारभेतैव युक्तश्चाव्यसनी भवेत्}


\twolineshloka
{शब्दे स्पर्शे च रूपे च गन्धेषु च रसेषु च}
{नोपभोगात्परं किंचिद्धनिनो वाऽधनस्य च}


\twolineshloka
{प्राक्संप्रयोगाद्भूतानां नास्ति दुःखं परायणम्}
{विप्रयोगात्तु सर्वस्य न शोचेत्प्रकृतिस्थितः}


\twolineshloka
{धृत्या शिश्नोदरं रक्षेत्पाणिपादं च चक्षुषा}
{चक्षुःश्रोत्रे च मनसा मनो वाचं च विद्यया}


\twolineshloka
{प्रणयं प्रतिसंहृत्य सस्निग्धेष्वितरेषु च}
{विचरेदसमुन्नद्धः स सुखी स च पण्डितः}


\twolineshloka
{अध्यात्मरतिरासीनो निरपेक्षो निरामिषः}
{आत्मनैव सहायेन यश्चरेत्स सुखी भवेत्}


\chapter{अध्यायः ३३९}
\twolineshloka
{नारद उवाच}
{}


\twolineshloka
{सुखदुःखविपर्यासो यदा समनुपद्यते}
{नैनं प्रज्ञा सुनीतं वा त्रायते नापि पौरुषम्}


\twolineshloka
{स्वभावाद्यत्नमातिष्ठेद्यत्नवान्नावसीदति}
{जरामरणरोगेभ्यः प्रियमात्मानमुद्धरेत्}


\twolineshloka
{रुजन्ति हि शरीराणि रोगाः शारीरमानसाः}
{सायका इव तीक्ष्णाग्राः प्रयुक्ता दृढधन्विभिः}


\twolineshloka
{व्यथितस्य विधित्साभिस्त्रस्यतो जीवितैषिणः}
{अवशस्य विनाशाय शरीरमपकृष्यते}


\twolineshloka
{स्रवन्ति न निवर्तन्ते स्रोतांसि सरितामिव}
{आयुरादाय मर्त्यानां रात्र्यहानि पुनः पुनः}


\twolineshloka
{व्यत्ययो ह्ययमत्यन्तं पक्षयोः शुक्लकृष्णयोः}
{जातान्मर्त्याञ्जरयति निमेषान्नावतिष्ठते}


\twolineshloka
{सुखदुःखानि भूतानामजरो जरयत्यसौ}
{आदित्यो ह्यस्तमभ्येति पुनः पुनरुदेति च}


\twolineshloka
{अदृष्टपूर्वानादाय भावानपरिशङ्कितान्}
{इष्टानिष्टान्मनुष्याणामस्तं गच्छन्ति रात्रयः}


\twolineshloka
{योयदिच्छेद्यथाकाममयत्नाच्च तदाप्नुयात्}
{यदि स्यान्न पराधीनं पुरुषस्य क्रियाफलम्}


\twolineshloka
{संयताश्च हि दक्षाश्च मतिमन्तश्च मानवाः}
{दृश्यन्ते निष्फलाः सन्तः प्रहीणाः सर्वकर्मभिः}


\twolineshloka
{अपरे बालिशाः सन्तो निर्गुणाः पुरुषाधमाः}
{अशुभैरपि संयुक्ता दृश्यन्ते सर्वकामिनः}


\twolineshloka
{भूतानामपरः कश्चिद्धिंसायां सततोत्थितः}
{वञ्चनायां च लोकस्य स सुखेष्वेव जीर्यते}


\twolineshloka
{अचेष्टमानमासीनं श्रीः कंचिदुपतिष्ठते}
{कश्चित्कर्मानुसृत्यान्यो नाप्राप्यमधिगच्छति}


\twolineshloka
{अपराधं समाचक्ष्व पुरुषस्य स्वभावतः}
{शुक्रमन्यत्र संभूतं पुनरन्यत्र गच्छति}


\twolineshloka
{तस्य योनौ प्रसक्तस्य गर्भो भवति वा न वा}
{आम्रपुष्पोपमा यस्य निर्वृत्तिरुपलभ्यते}


\twolineshloka
{केषांचित्पुत्रकामानामनुसन्तानमिच्छताम्}
{सिद्धौ प्रयतमानानां न चाण्डमुपजायते}


\twolineshloka
{गर्भाच्चोद्विजमानानां क्रुद्धादाशीविषादिव}
{आयुष्माञ्जायते पुत्रः कथं प्रेतः पितेव ह}


\twolineshloka
{देवानिष्ट्वा तपस्तप्त्वा कृपणैः पुत्रगृद्धिभिः}
{दश मासान्परिधृता जायन्ते कुलपांसनाः}


\twolineshloka
{अपरे धनधान्यानि भोगांश्च पितृसंचितान्}
{विपुलानभिजायन्ते लब्धास्तैरेव मङ्गलैः}


\twolineshloka
{अन्योन्यं समभिप्रेत्य मैथुनस्य समागमे}
{उपद्रव इवाविष्टो योनिं गर्भः प्रपद्यते}


\twolineshloka
{शीर्णं परशरीराणि च्छिन्नबीजं शरीरिणम्}
{प्राणिनं प्राणंसरोधे मांसश्लेष्मविचेष्टितम्}


\twolineshloka
{निर्दग्धं परदेहेऽपि परदेहं चलाचलम्}
{विनश्यन्तं विनाशान्ते भावि नावमिवाहितम्}


\twolineshloka
{सङ्गत्या जठरे न्यस्तं रेतोबिन्दुमचेतनम्}
{केन यत्नेन जीवन्तं गर्भं त्वमिह पश्यसि}


\twolineshloka
{अन्नपानानि जीर्यन्ते यत्र भक्षाश्च भक्षिताः}
{तस्मिन्नेवोदरे गर्भः किं नान्नमिव जीर्यते}


\twolineshloka
{गर्भे मूत्रपुरीषाणां स्वभावनियता गतिः}
{धारणे वा विसर्गे वा न कर्ता विद्यतेऽवशः}


\twolineshloka
{स्रवन्ति ह्युदराद्गर्भा जायमानास्तथा परे}
{आगमेन तथाऽन्येषां विनाश उपपद्यते}


\twolineshloka
{एतस्माद्योनिसंबन्धाद्यो जीवः परिमुच्यते}
{प्रजां च लभते कांचित्पुनर्द्वन्द्वेषु सज्जति}


\twolineshloka
{स तस्य सहजातस्य सप्तमीं नवमीं दशाम्}
{प्राप्नुवन्ति ततः पञ्च न भवन्ति शतायुषः}


% Check verse!
नाभ्युत्थाने मनुष्याणां योगाः स्युर्नात्र संशयः ॥व्याधिभिश्च विमथ्यन्ते व्याधैः क्षुद्रमृगा इव
\twolineshloka
{व्याधिभिर्भक्ष्यमाणानां त्यजतां विपुलं धनम्}
{वेदनां नापकर्षन्ति यतमानाश्चिकित्सकाः}


\twolineshloka
{ते चापि निपुणा वैद्याः कुशलाः संभृतौषधाः}
{व्याधिभिः परिकृष्यन्ते मृगा व्याधैरिवार्दिताः}


\twolineshloka
{ते पिबन्तः कषायांश्च सर्पीषि विविधानि च}
{दृश्यन्ते जरया भग्ना नगा नागैरिवोत्तमैः}


\twolineshloka
{के वा भुवि चिकित्सन्ते रोगार्तान्मृगपक्षिणः}
{श्वापदानि दरिद्रांश्च प्रायो नार्ता भवन्ति ते}


\twolineshloka
{पौरानपि दुराधर्षान्नृपतीनुग्रतेजसः}
{आक्रम्य खादन्ते रोगाः पशून्पशुपचा इव}


\twolineshloka
{इति लोकमनाक्रन्दं मोहशोकपरिप्लुतम्}
{स्रोतसा सहसा क्षिप्तं ह्रियमाणं बलीयसा}


\twolineshloka
{न धनेन न राज्येन नोग्रेण तपसा तथा}
{स्वभावमतिवर्तन्ते ये नियुक्ताः शरीरिणः}


\twolineshloka
{न म्रियेरन्न जीर्येरन्सर्वे स्युः सर्वकामिनः}
{नाप्रियं प्रतिपश्येयुरुत्थानस्य फले सति}


\twolineshloka
{उपर्युपरि लोकस्य सर्वो भवितुमीहते}
{यतते च यथाशक्ति न च तद्वर्तते तथा}


\twolineshloka
{ऐश्वर्यमदमत्तांश्च मत्तान्मद्यमदेन च}
{अप्रमत्ताश्च शूराश्च विक्रान्ताः पर्युपासते}


\twolineshloka
{क्लेशाः प्रतिनिवर्तन्ते केषांचिदसमीक्षिताः}
{स्वंस्वं न पुनरन्येषां न किंचिदधिगम्यते}


\twolineshloka
{महच्च फलवैपम्यं दृश्यते कर्मसिद्धिषु}
{वहन्ति शिविकामन्ये यान्त्यन्ये शिविकागताः}


\twolineshloka
{सर्वेषामृद्धिकामानामन्ये रथपुरःसराः}
{मनुजाश्च गतस्त्रीकाः शतशो विविधाः स्त्रियः}


\twolineshloka
{द्वन्द्वारामेषु भूतेषु गच्छन्त्येकैकशो नराः}
{इदमन्यत्परं पश्य माऽत्र मोहं करिष्यसि}


\twolineshloka
{त्यज धर्ममधर्मं च उभे सत्यानृते त्यज}
{उभे सत्यानृते त्यक्त्वा येन त्यजसि तं त्यज}


\twolineshloka
{एतत्ते परमं गुह्यमाख्यातमृषिसत्तम}
{येन देवाः परित्यज्य मर्त्यलोकं दिवं गताः}


\twolineshloka
{नारदस्य वचः श्रुत्वा शुकः परमबुद्धिमान्}
{संचिन्त्य मनसा धीरो निश्चयं नाध्यगच्छत}


\twolineshloka
{पुत्रदारैर्महान्क्लेशो विद्याम्नाये महाञ्छ्रमः}
{किंनु स्याच्छाश्वतं स्थानमल्पक्लेशं महोदयम्}


\twolineshloka
{ततो मुहूर्तं संचिन्त्य निश्चितां गतिमात्मनः}
{परावरज्ञो धर्मस्य परां नैःश्रेयसीं गतिम्}


\twolineshloka
{कथं त्वहमसंश्लिष्टो गच्छेयं गतिमुत्तमाम्}
{नावर्तेयं यथा भूयो योनिसंसारसागरे}


\twolineshloka
{परं भावं हि काङ्क्षामि यत्र नावर्तते पुनः}
{सर्वसङ्गान्परित्यज्य निश्चितो मनसा गतिम्}


\twolineshloka
{तत्र यास्यामि यत्रात्मा शर्म मेऽधिगमिष्यति}
{अक्षयश्चाव्ययश्चैव यत्र स्थास्यामि शाश्वतः}


\twolineshloka
{न तु योगमृते शक्त्या प्राप्नुयां परमां गतिम्}
{अनुबन्धो विमुक्तस्य कर्मभिर्नोपपद्यते}


\twolineshloka
{यस्माद्योगं समास्थाय त्यक्त्वा गृहकलेवरम्}
{वायुभूतः प्रवेक्ष्यामि तेजोराशिं दिवाकरम्}


\twolineshloka
{न ह्येप क्षयतां याति सोमः सुरगणैर्यथा}
{कम्पितः पतते भूमिं पुनश्चैवाधिरोहति}


\twolineshloka
{क्षीयते हि सदा सोमः पुनश्चैवाभिपूर्यते}
{नेच्छाम्येवं विदित्वैते ह्रासवृद्धी पुनः पुनः}


\twolineshloka
{रविस्तु संतापयते लोकान्रश्मिभिरुल्बणैः}
{सर्वतस्तेज आदत्ते नित्यमक्षयमण्डलः}


\twolineshloka
{अतो मे रोचते गन्तुमादित्यं दीप्ततेजसम्}
{अत्र वत्स्यामि दुर्धर्षो निःसङ्गेनान्तरात्मना}


\twolineshloka
{सूर्यस्यर सदने चाहं निक्षिप्येदं कलेवरम्}
{ऋषिभिः सह वत्स्यामि सौरं तेजोऽतिदुःसहं}


\twolineshloka
{आपृच्छामि नगान्नागान्गिरीनुर्वी दिशो दश}
{देवदानवगन्धर्वान्पिशाचोरगराक्षसान्}


\twolineshloka
{लोकेषु सर्वभूतानि प्रवेक्ष्यामि न संशयः}
{पश्यन्तु योगवीर्यं मे सर्वे देवाः सहर्षिभिः}


\twolineshloka
{अथानुज्ञाप्य तमृषिं नारदं लोकविश्रुतम्}
{तस्मादनुज्ञां संप्राप्य जगाम पितरं प्रति}


\twolineshloka
{सोऽभिवाद्य महात्मानं कृष्णद्वैपायनं मुनिम्}
{शुकः प्रदक्षिणं कृत्वा कृष्णमापृष्टवान्मुनिम्}


\twolineshloka
{श्रुत्वा चर्षिस्तद्वचनं शुकस्यप्रीतो महात्मा पुनराह चैनम्}
{भोभो पुत्र स्थीयतां तावदद्ययावच्चक्षुः प्रीणयामि त्वदर्थे}


\twolineshloka
{निरपेक्षः शुको भूत्वा निःस्नेहो मुक्तसंशयः}
{मोक्षमेवानुसंचिन्त्य गमनाय मनो दधे}


\twolineshloka
{पितरं स परित्यज्य जगाम मुनिसत्तमः}
{कैलासपृष्ठं विपुलं सिद्धसङ्घनिषेवितम्}


\chapter{अध्यायः ३४०}
\twolineshloka
{भीष्म उवाच}
{}


\twolineshloka
{गिरिशृङ्गं समारुह्य सुतो व्यासस्य भारत}
{समे देशे विविक्ते स निःशलाक उपाविशत्}


\twolineshloka
{धारयामास चात्मानं यथाशास्त्रं यथाविधि}
{पादप्रभृतिगात्रेषु क्रमेण क्रमयोगवित्}


\twolineshloka
{ततः स प्राङ्भुखो विद्वानादित्ये नाचिरोदिते}
{पाणिपादं समाधाय विनीतवदुपाविशत्}


\twolineshloka
{न तत्र पक्षिसंपातो न शब्दो नापि दर्शनम्}
{यत्र वैयासकिर्धीमान्योक्तुं समुपचक्रमे}


\twolineshloka
{स ददर्श तदाऽऽत्मानं सर्वसङ्गविनिःसृतम्}
{प्रजहास ततो हासं शुकः संप्रेक्ष्य तत्परम्}


\twolineshloka
{स पुनर्योगमास्थाय मोक्षमार्गोपलब्धये}
{महायोगेश्वरो भूत्वा सोऽत्यक्रामद्विहायसम्}


\threelineshloka
{ततः प्रदक्षिणं कृत्वा देवर्षि नारदं ततः}
{निवेदयामास च तं स्वं योगं परमर्षये ॥शुक उवाच}
{}


\twolineshloka
{दृष्टो मार्गः प्रवृत्तोस्मि स्वस्ति तेऽस्तु तपोधन}
{त्वत्प्रसादाद्गमिष्यामि गतिमिष्टां महाद्युते}


\twolineshloka
{नारदेनाभ्यनुज्ञातः शुको द्वैपायनात्मजः}
{अभिवाद्य पुनर्योगमास्थायाकाशमाविशत्}


\twolineshloka
{कैलासपृष्ठादुत्पत्य स पपात दिवं तदा}
{अन्तरिक्षचरः श्रीमान्व्यासपुत्रः सुनिश्चितः}


\twolineshloka
{तमुद्यन्तं द्विजश्रेष्ठं वैनतेयसमद्युतिम्}
{ददृशुः सर्वभूतानि मनोऽमारुतरंहसम्}


\twolineshloka
{व्यवसायेन लोकांस्त्रीन्सर्वान्सोऽथ विचिन्तयन्}
{आस्थितो दिव्यमध्वानं पावकार्कसमप्रभः}


\twolineshloka
{तमेकमनसं यान्तमव्यग्रमकुतोभयम्}
{ददृशुः सर्वभूतानि जङ्गमानीतराणि च}


\twolineshloka
{यथाशक्ति यथान्यायं पूजयांचक्रिरे तदा}
{पुष्पवर्षेश्च दिव्यैस्तमलंचक्रुर्दिवौकसः}


\twolineshloka
{तं दृष्ट्वा विस्मिताः सर्वे गन्धर्वाप्सरसां गणाः}
{ऋषयश्चैव संसिद्धाः परं विस्मयमागताः}


\twolineshloka
{अन्तरिक्षगतः कोऽयं तपसा सिद्धिमागतः}
{अधः कायोर्ध्ववक्रश्च नेत्रैः समतिवाह्यते}


\threelineshloka
{ततः परमधर्मात्मा त्रिषु लोकेषु विश्रुतः}
{भास्करं समुदीक्षन्स प्राङ्भुखो वाग्यतोऽगमत्}
{शब्देनाकाशमखिलं पूरयन्निव सर्वशः}


\threelineshloka
{तमापतन्तं सहसा दृष्ट्वा सर्वाप्सरोगणाः}
{संभ्रान्तमनसो राजन्नासन्परमविस्मिताः}
{पञ्चचूडाप्रभृतयो भृशमुत्फुल्ललोचनाः}


\twolineshloka
{दैवतं कतमं ह्येतदुत्तमां गतिमास्थितम्}
{सुनिश्चितमिहायाति विमुक्तमिव निःस्पृहम्}


\twolineshloka
{ततः समभिचक्राम मलयं नाम पर्वतम्}
{उर्वशी पूर्वचित्तिश्च यं नित्यमुपसेवतः}


\twolineshloka
{तस्य ब्रह्मर्षिपुत्रस्य विस्मयं ययतुः परम्}
{अहो बुद्धिसमाधानं वेदाभ्यासरते द्विजे}


\twolineshloka
{अचिरेणैव कालेन नभश्चरति चन्द्रवत्}
{पितृशुश्रूषया बुद्धिं संप्राप्तोऽयमनुत्तमाम्}


\twolineshloka
{पितृभक्तो दृढतपाः पितुः सुदयितः सुतः}
{अनन्यमनसा तेन कथं पित्रा विसर्जितः}


\twolineshloka
{उर्वश्या वचनं श्रुत्वा शुकः परमधर्मवित्}
{उदैक्षत दिशः सर्वा वचने गतमानसः}


\twolineshloka
{सोऽन्तरिक्षं महीं चैव सशैलवनकाननाम्}
{विलोकयामास तदा सरांसि सरितस्तथा}


\twolineshloka
{ततो द्वैपायनसुतं बहुमानात्समन्ततः}
{कृताञ्जलिपुटाः सर्वा निरीक्षन्ते स्म देवताः}


\twolineshloka
{अब्रवीत्तास्तदा वाक्यं शुकः परमधर्मवित्}
{पिता यद्यनुगच्छेन्मां क्रोशमान शुकेति वै}


\twolineshloka
{तस्य प्रतिवचो देयं सर्वैरेव समाहितैः}
{एतन्मे स्नेहनः सर्वे वचनं कर्तुमर्हथ}


\twolineshloka
{शुकस्य वचन श्रुत्वा दिशः सजलकाननाः}
{समुद्राः सरितः शैलाः प्रत्यूचुस्तं समन्ततः}


\twolineshloka
{यथा ज्ञापयसे विप्र बाढमेवं भविष्यति}
{ऋषेर्व्याहरतो वाक्यं प्रतिवक्ष्यामहे वयम्}


\chapter{अध्यायः ३४१}
\twolineshloka
{भीष्म उवाच}
{}


\twolineshloka
{इत्येवमुक्त्वा वचनं ब्रह्मर्षिः सुमहातपाः}
{प्रातिष्ठत शुकः सिद्धिं हित्वा दोषांश्चतुर्विधान्}


\twolineshloka
{तमो ह्यष्टगुणं हित्वा जहौ पञ्चविधं रजः}
{ततः सत्वं जहौ धीमांस्तदद्भुतमिवाभवत्}


\twolineshloka
{ततस्तस्मिन्पदे नित्ये निर्गुणे लिङ्गवर्जिते}
{ब्रह्मणि प्रत्यतिष्ठत्स विधूमोऽग्निरिव ज्वलन्}


\twolineshloka
{उत्कापाता दिशां दाहो भूमिकम्पस्तथैव च}
{प्रादुर्भूताः क्षणे तस्मिंस्तदद्भुतमिवाभवत्}


\twolineshloka
{द्रुमाः शाखाश्च मुमुचुः शिखराणि च पर्वताः}
{निर्घातशब्दैर्गुरुभिर्भूमिर्व्यादीर्यतेव ह}


\twolineshloka
{न बभासे सहस्रांशुर्न जज्वाल च पावकः}
{ह्रदाश्च सरितश्चैव चुक्षुभुः सागरास्तथा}


\twolineshloka
{ववर्ष वासवस्तोयं रसवच्च सुगन्धि च}
{ववौ समीरणश्चापि दिव्यगन्धवहः शुचिः}


\twolineshloka
{स शृङ्गेऽप्रतिमे दिव्ये हिमवन्मेरुसंनिभे}
{संश्लिष्टे स्वतेपीते द्वे रुक्मरूप्यमये शुभे}


\twolineshloka
{शतयोजनविस्तारे तिर्यगूर्ध्वं च भारत}
{उदीचीं दिशमास्थाय रुचिरे संददर्श ह}


% Check verse!
सोऽविशङ्केन मनसा तथैवाभ्यपतच्छुकः
\twolineshloka
{ततः पर्वतशृङ्गे द्वे सहसैव द्विधाकृते}
{अदृश्येतां महाराज तदद्भुतमिवाभवत्}


\twolineshloka
{ततः पर्वतशृङ्गाभ्यां सहसैव विनिःसृतः}
{न च प्रतिजघानास्य स गतिं पर्वतोत्तमः}


\twolineshloka
{ततो महानभूच्छब्दो दिवि सर्वदिवौकसाम्}
{गन्धर्वाणामृषीणां च ये च शैलनिवासिनः}


\twolineshloka
{दृष्ट्वा शुकमतिक्रान्तं पर्वतं च द्विधा कृतम्}
{साधुसाध्विति तत्रासीन्नादः सर्वत्र भारत}


\twolineshloka
{स पूज्यमानो देवैश्च गन्धर्वैर्ऋशिभिस्तथा}
{यक्षराक्षससङ्घैश्च विद्याधरगणैस्तथा}


\twolineshloka
{दिव्यैः पुष्पैः समाकीर्णमन्तरिक्षं समन्ततः}
{आसीत्किल महाराज शुकाभिपतने तदा}


\twolineshloka
{ततो मन्दाकिनीं रम्यामुपरिष्टादभिव्रजन्}
{शुको ददर्श धर्मात्मा पुष्पितद्रुमकाननाम्}


\twolineshloka
{तस्यां क्रीडन्त्यभिरताः स्नान्ति चैवाप्सरोगणाः}
{शून्याकारं निराकाराः शुकं दृष्ट्वा विवाससः}


\twolineshloka
{तं प्रक्रामन्तमाज्ञाय पिता स्नेहसमन्वितः}
{उत्तमां गतिमास्थाय पृष्ठतोऽनुससार ह}


\twolineshloka
{शुकस्तु मारुतादूर्ध्वं गतिं कृत्वान्तरिक्षगाम्}
{दर्शयित्वा प्रभावं स्वं सर्वभूतोऽभवत्तदा}


\twolineshloka
{महायोगगतिं त्वग्र्यां व्यासोत्थाय महातपाः}
{निमेषान्तरमात्रेण शुकाभिपतनं ययौ}


\twolineshloka
{स ददर्श द्विधा कृत्वा पर्वताग्रं शुकं गतम्}
{शशंसुर्ऋषयस्तत्र कर्म पुत्रस्य तत्तदा}


\twolineshloka
{ततः शुकेति दीर्घेण शब्देनाक्रन्दितस्तदा}
{स्वयं पित्रा स्वरेणोच्चैस्त्रील्लोँकाननुनाद्य वै}


\twolineshloka
{शुकः सर्वगतो भूत्वा सर्वात्मा सर्वतोमुखः}
{प्रत्यभाषत धर्मात्मा भोःशब्देनानुनादयन्}


\twolineshloka
{तत एकाक्षरं नादं भोरित्येव समीरयन्}
{प्रत्याहरञ्जगत्सर्वमुच्चैः स्थावरजङ्गमम्}


\twolineshloka
{ततःप्रभृति चाद्यापि शब्दानुच्चारितान्पृथक्}
{गिरिगह्वरपृष्ठेषु व्याहरन्ति शुकं प्रति}


\twolineshloka
{अन्तर्हितः प्रभावं तु दर्शयित्वा शुकस्तदा}
{गुणान्संत्यज्य शब्दादीन्पदमभ्यगमत्परम्}


\twolineshloka
{महिमानं तु तं दृष्ट्वा पुत्रस्यामिततेजसः}
{निषसाद गिरिप्रस्थे पुत्रमेवानुचिन्तयन्}


\twolineshloka
{ततो मन्दाकिनीतीरे क्रीडन्तोऽऽप्सरसां गणाः}
{आसाद्य तमृषिं सर्वाः संभ्रान्ता गतचेतसः}


\twolineshloka
{जले निलिल्यिरे काश्चित्काश्चिद्गुल्मान्प्रपेदिरे}
{वसनान्याददुः काश्चित्तं दृष्ट्वा मुनिसत्तमम्}


\twolineshloka
{तां मुक्ततां तु विज्ञाय मुनिः पुत्रस्य वै तदा}
{सक्ततामात्मनश्चैव प्रीतोऽभूद्बीडितश्च ह}


\twolineshloka
{तं देवगन्धर्ववृतो महर्षिगणपूजितः}
{पिनाकहस्तो भगवानभ्यागच्छत शंकरः}


\twolineshloka
{तमुवाच महादेवः सान्त्वपूर्वमिदं वचः}
{पुत्रशोकाभिसंतप्तं कृष्णद्वैपायनं तदा}


\twolineshloka
{अग्नेर्भूमेरपां वायोरन्तरिक्षस्य चैव ह}
{वीर्येण सदृशः पुत्रः पुरा मत्तस्त्वया वृतः}


\twolineshloka
{स तथालक्षणो जातस्तपसा तव संभृतः}
{मम चैव प्रसादेन ब्रह्मतेजोमयः शुचिः}


\twolineshloka
{स गतिं परमां प्राप्तो दुष्प्रापामजितेन्द्रियैः}
{दैवतैरपि विप्रर्षे तं त्वं किमनुशोचसि}


\twolineshloka
{यावत्स्थास्यन्ति गिरयो यावत्स्थास्यन्ति सागराः}
{तावत्तवाक्षया कीर्तिः सपुत्रस्य भविष्यति}


\twolineshloka
{छायां स्वपुत्रसदृशीं सर्वतोऽनपगां सदा}
{द्रक्ष्यसे त्वं च लोकेऽस्मिन्मत्प्रसादान्महामुने}


\twolineshloka
{सोऽनुगीतो भगवता स्वयं रुद्रेण भारत}
{छायां पश्यन्परावृत्तः स मुनिः परया मुदा}


\twolineshloka
{इति जन्म गतिश्चैव शुकस्य भरतर्षभ}
{विस्तरेण समाख्याता यन्मां त्वं परिपृच्छसि}


\twolineshloka
{एतदाचष्ट मे राजन्देवर्षिर्नारदः पुरा}
{व्यासश्चैव महायोगी संजल्पेषु पदेपदे}


\twolineshloka
{इतिहासमिमं पुण्यं मोक्षधर्मार्थसंहितम्}
{धारयेद्यः शमपरः स गच्छेत्परमां गतिम्}


\chapter{अध्यायः ३४२}
\twolineshloka
{युधिष्ठिर उवाच}
{}


\twolineshloka
{गृहस्थो ब्रह्मचारी वा वानप्रस्थोऽथ भिक्षुकः}
{य इच्छेत्सिद्धिमास्थातुं देवतां कां यजेत सः}


\twolineshloka
{कुतो ह्यस्य ध्रुवः स्वर्गः कुतो नैःश्रेयसं परम्}
{विधिना केन जुहुयाद्दैवं पित्र्यं तथैव च}


\twolineshloka
{मुक्तश्च कां गतिं गच्छेन्मोक्षश्चैव किमात्मकः}
{स्वर्गतश्चैव किं कुर्याद्येन न च्यवते दिवः}


\threelineshloka
{देवतानां च को देवः पितॄणां च पिता तथा}
{तस्मात्परतरं यच्च तन्मे ब्रूहि पितामह ॥भीष्म उवाच}
{}


\twolineshloka
{गूढं मां प्रश्नवित्प्रश्नं पृच्छसे त्वमिहानघ}
{न ह्येतत्तर्कया शक्यं वक्तुं वर्षशतैरपि}


\twolineshloka
{ऋते देवप्रसादाद्वा राजञ्ज्ञानागमेन वा}
{गहनं ह्येतदाख्यानं व्याख्यातव्यं तवारिहन्}


\twolineshloka
{अत्राप्युदाहरन्तीममितिहासं पुरातनम्}
{नारदस्य च संवादमृषेर्नारायणस्य च}


\twolineshloka
{नारायणो हि विश्वात्मा चतुर्मूर्तिः सनातनः}
{धर्मात्मजः संबभूव पितैवं मेऽभ्यभाषत}


\twolineshloka
{कृते युगे महाराज पुरा स्वायंभुवेऽन्तरे}
{नरो नारायणश्चैव हरिः कृष्णस्तथैव च}


\twolineshloka
{तेषां नारायणनरौ तपस्तेपतुरव्ययौ}
{बदर्याश्रममासाद्य शकटे कनकामये}


\twolineshloka
{अष्टचक्रं हि तद्यानं भूतयुक्तं मनोरमम्}
{तत्राद्यौ लोकनाथौ तौ कृशौ धमनिसंततौ}


\twolineshloka
{तपसा तेजसा चैव दुर्निरीक्ष्यौ सुरैरपि}
{यस्य प्रसादं कुर्वाते स देवौ द्रष्टुमर्हति}


\twolineshloka
{नूनं तयोरनुमते हृदि हृच्छपचोदितः}
{महामेरोगिंरेः शृङ्गात्प्रत्युतो गन्धमादनम्}


\twolineshloka
{नारदः सुमहद्भूतं सर्वलोकानचीचरत्}
{तं देशमगमद्राजन्वदर्याश्रममाशुगः}


\twolineshloka
{तयोराह्निकवेलायां तस्य कौतूहलं त्वभूत्}
{इदं तदास्पदं कृत्स्नं यस्मिँल्लोकाः प्रतिष्ठिताः}


\twolineshloka
{सदेवासुरगन्धर्वाः सकिन्नरमहोरगाः}
{एका मूर्तिरियं पूर्वं जाता भूयश्चतुर्विधा}


\twolineshloka
{धर्मस्य कुलसंताने धर्मादेभिर्विवर्धितः}
{अहो ह्यनुगृहीतोऽद्य धर्म एभिः सुरैरिह}


\twolineshloka
{नरनारायणाभ्यां च कृष्णेन हरिणा तथा}
{अत्र कृष्णो हरिश्चैव कस्मिंश्चित्कारणान्तरे}


\twolineshloka
{स्थितौ धर्मसुतावेतौ तथा तपसि धिष्ठितौ}
{एतौ हि परमं धाम काऽनयोराह्निकक्रिया}


\twolineshloka
{पितरौ सर्वभूतानां दैवतं च यशस्विनौ}
{कां देवतां तु यजतः पितॄन्वा कान्महामती}


\twolineshloka
{इति संचिन्त्य मनसा भक्त्या नारायणस्य तु}
{सहसा प्रादुरभवत्समीपे देवयोस्तदा}


\twolineshloka
{कृते दैवे च पित्र्ये च ततस्ताभ्यां निरीक्षितः}
{पूजितश्चैव विधिना यथाप्रोक्तेन शास्त्रतः}


\twolineshloka
{तद्दृष्ट्वा महदाश्चर्यमपूर्वं विधिविस्तरम्}
{उपोपविष्टः सुप्रीतो नारदो भगवानृषिः}


\twolineshloka
{नारायणं संनिरीक्ष्य प्रसन्नेनान्तरात्मना}
{नमस्कृत्य महादेवमिदं वचनमब्रवीत्}


\twolineshloka
{वेदेषु सपुराणेषु साङ्गोपाङ्गेषु गीयसे}
{त्वमजः शाश्वतो धाता माता मृतमनुत्तमम्}


\twolineshloka
{प्रतिष्ठितं भूतभव्यं त्वयि सर्वमिदं जगत्}
{चत्वारो ह्याश्रमा देव सर्वे गार्हस्थ्यमूलकाः}


\threelineshloka
{यजन्ते त्वामहरहर्नानामूर्तिसमास्थितम्}
{पिता माता च सर्वस्य देवतानां च शाश्वतम्}
{कं त्वद्य यजसे देवं पितरं कं न विद्महे}


\twolineshloka
{`कमर्चसि महाभाग तन्मे प्रब्रूहि पृच्छतः ॥' श्रीभगवानुवाच}
{}


\twolineshloka
{अवाच्यमेतद्वक्तव्यमात्मगुह्यं सनातनम्}
{तव भक्तिमतो ब्रह्मन्वक्ष्यामि तु यथातथम्}


\twolineshloka
{यत्तत्सूक्ष्ममविज्ञेयमव्यक्तमचलं ध्रुवम्}
{इन्द्रियैन्द्रियार्थैश्च सर्वभूतैश्च वर्जितम्}


\twolineshloka
{स ह्यन्तरात्मा भूतानां क्षेत्रज्ञश्चेति कथ्यते}
{त्रिगुणव्यतिरिक्तो वै पुरुषश्चेति कल्पितः}


\twolineshloka
{तस्मादव्यक्तमुत्पन्नं त्रिगुणं द्विजसत्तम}
{अव्यक्ताव्यक्तभावस्था या सा प्रकृतिरव्यया}


\twolineshloka
{तां योनिमावयोर्विद्धि योसौ सदसदात्मकः}
{आवाभ्यां पूज्यते यो हि दैवे पित्र्ये च कल्प्यते}


\twolineshloka
{नास्ति तस्मात्परोऽन्यो हि पिता देवोऽथवा द्विज}
{आत्मा हि नौ स विज्ञेयस्ततस्तं पूजयावहे}


\twolineshloka
{तेनैषा प्रथिता ब्रह्मन्मर्यादा लोकमाविनी}
{दैवं पित्र्यं च कर्तव्यमिति तस्यानुशासनम्}


\twolineshloka
{ब्रह्मा स्थाणुर्मनुर्दक्षो भृगुर्धर्मस्तथा यमः}
{मरीचिरङ्गिराश्चात्रिः पुलस्त्यः पुलहः क्रतुः}


\twolineshloka
{वसिष्ठः परमेष्ठी च विवस्वान्सोम एव च}
{कर्दमश्चापि यः प्रोक्तः क्रोधो विक्रीत एव च}


\twolineshloka
{*एकविंशतिरुत्पन्नास्ते प्रजापतयः स्मृताः}
{तस्य देवस्य मर्यादां पूजयन्तः सनातनीम्}


\twolineshloka
{दैवं पित्र्यं च सततं तस्य विज्ञाय तत्त्वतः}
{आत्मप्राप्तानि च ततो जानन्ति द्विजसत्तमाः}


\twolineshloka
{स्वर्गस्था अपि ये केचित्तान्नमस्यन्ति देहिनः}
{ते तत्प्रसादाद्गच्छन्ति तेनादिष्टफलां गतिम्}


\twolineshloka
{ये हीनाः सप्तदशभिर्गुणैः कर्मभिरेव च}
{कलाः पञ्चदश त्यक्त्वा ते मुक्ता इति निश्चयः}


\twolineshloka
{मुक्तानां तु गतिर्ब्रह्मन्क्षेत्रज्ञ इति कल्पिता}
{स हि सर्वगतिश्चैव निर्गुणश्चैव कथ्यते}


\twolineshloka
{दृश्यते ज्ञानयोगेन आवां च प्रसृतौ ततः}
{एवं ज्ञात्वा तमात्मानं पूजयावः सनातनम्}


\twolineshloka
{तं वेदाश्चाश्रमाश्चैव नानातनुसमाश्रितम्}
{भक्त्या संपूजयन्त्यद्य गतिं चैषां ददाति सः}


\twolineshloka
{ये तु तद्भाविता लोके ह्येकान्तित्वं समास्थिताः}
{एतदभ्यधिकं तेषां यत्ते तं प्रविशन्त्युत}


\twolineshloka
{इति गुह्यसमुद्देशस्तव नारद कीर्तितः}
{भक्त्या प्रेम्णा च विप्रर्षे अस्मद्भक्त्या च ते श्रुतः}


\chapter{अध्यायः ३४३}
\twolineshloka
{भीष्म उवाच}
{}


\threelineshloka
{स एवमुक्तो द्विपदां वरिष्ठोनारायणेनोत्तमपूरुषेण}
{जगाद वाक्यं द्विपदां वरिष्ठंनारायणं लोकहिताधिवासम् ॥नारद उवाच}
{}


\twolineshloka
{यदर्थमात्मप्रभवेह जन्मत्वयोत्तमं धर्मगृहे चतुर्धा}
{तत्साध्यतां लोकहितार्थमद्यगच्छामि द्रष्टुं प्रकृतिं तवाद्याम्}


\twolineshloka
{वेदाः स्वधीता मम लोकनाथतप्तं तपो नानृतमुक्तपूर्वम्}
{पूजां गुरूणां सततं करोमिपरस्य गुह्यं न तु भिन्नपूर्वम्}


\twolineshloka
{गुप्तानि चत्वारि यथागमं मेशत्रौ च मित्रे च समोस्मि नित्यम्}
{तं चादिदेवं सततं प्रपन्नएकान्तभावेन वृणोभ्यजस्रम्}


\twolineshloka
{एभिर्विशेषैः परिशृद्धसत्वःकस्मान्न पश्येयमनन्तमीशम्}
{तत्पारमेष्ठ्यस्य वचो निशम्यनारायणः शाश्वतधर्मगोप्ता}


\twolineshloka
{गच्छेति तं नारदमुक्तबान्ससंपूजयित्वा विधिवत्क्रियाभिः}
{ततो विसृष्टः परमेष्ठिपुत्रःसोऽभ्यर्चयित्वा तमृषिं पुराणम्}


\twolineshloka
{खमुत्पपातोत्तमयोगयुक्तस्ततोऽधिमेरौ सहसा निलिल्ये}
{तत्रावतस्थे च मुनिर्मुर्हुतमेकान्तमासाद्य गिरेः स शृङ्क्ते}


\twolineshloka
{आलोकयन्नुत्तरपश्चिमेनददर्श चाप्यद्भुतमुक्तरूपम्}
{क्षीरोदधेर्योत्तरतो हि द्वीपःश्वेतः स नाम्ना प्रथितो विशालः}


\twolineshloka
{मेरोः सहस्रैः स हि योजनानांद्वात्रिंशतोर्ध्वं कविभिर्निरुक्तः}
{अनिन्द्रियाश्चानशनाश्च तत्रनिष्पन्दहीनाः सुसुगन्धिनस्ते}


\twolineshloka
{श्वेताः पुमांसो गतसर्वपापाश्चक्षुर्मुषः पापकृतां नराणाम्}
{वज्रास्थिकायाः सममानोन्मानादिव्यावयवरूपाः शुभसारोपेताः}


\twolineshloka
{छत्राकृतिशीर्षा मेघौघनिनादाःसममुष्कचतुष्का राजीवच्छदपादाः}
{षष्ठ्या दन्तैर्युक्ताः शुक्लैरष्टाभिर्दंष्ट्राभिर्येजिह्वाभिर्ये विश्ववक्रंलेलिह्यन्ते सूर्यप्रख्यम्}


\threelineshloka
{देवं भक्त्या विश्वोत्पन्नंयस्मात्सर्वे लोकाः संप्रसूताः}
{सर्वगात्राश्च सूक्ष्माः सहाङ्गकावेदा धर्मा मुनयः शान्ता देवाःसर्वे तस्य निसर्ग इति ॥युधिष्ठिर उवाच}
{}


\twolineshloka
{अनिन्द्रिया निराहारा अनिष्पन्दाः सुगन्धिनः}
{कथं ते पुरुषा जाताः का तेषां गतिरुत्तमा}


\twolineshloka
{ये च मुक्ता भवन्तीह नरा भरतसत्तम}
{तेषां लक्षणमेतद्धि तच्छ्वेतद्द्वीपवासिनाम्}


\threelineshloka
{तस्मान्मे संशयं छिन्धि परं कौतूहलं हि मे}
{त्वं हि सर्वकथारामस्त्वां चैवोपाश्रिता वयम् ॥भीष्म उवाच}
{}


\twolineshloka
{विस्तीर्णैषा कथा राजञ्श्रुता मे पितृसन्निधौ}
{यैषा तव हि वक्तव्या कथासारो हि सा मता}


\twolineshloka
{`शन्तनोः कथयामास नारदो मुनिसत्तमः}
{राज्ञा पृष्टः पुरा प्राह तत्राहं श्रुतवान्पुरा ॥'}


\twolineshloka
{राजोपरिचरो नाम बभूवाधिपतिर्भुवः}
{आखण़्डलसखः ख्यातो भक्तो नारायणं हरिं}


\twolineshloka
{धार्मिको नित्यभक्तश्च पितुर्नित्यमतन्द्रितः}
{साम्राज्यं तेन संप्राप्तं नारायणवरात्पुरा}


\twolineshloka
{सात्वतं विधिमास्थाय प्राक्सूर्यमुखनिःसृतम्}
{पूजयामास देवेशं तच्छेषेण पितामहान्}


\twolineshloka
{पितृशेषेण विप्रांश्च संविभज्याश्रितांश्च सः}
{शेषान्नभुक्सत्यपरः सर्वभूतेष्वहिंसकः}


\twolineshloka
{सर्वभावेन भक्तः स देवदेवं जनार्दनम्}
{अनादिमध्यनिधनं लोककर्तारमव्ययम्}


\twolineshloka
{तस्य नारायणे भक्तिं वहतोऽमित्रकर्शिनः}
{एकशय्यासनं देवो दत्तवान्देवराट् स्वयम्}


\twolineshloka
{आत्मराज्यं धनं चैव कलत्रं वाहनं तथा}
{यत्तद्भागवतं सर्वमिति तत्प्रेषितं सदा}


\twolineshloka
{काम्यनैमित्तिका राजन्यज्ञियाः परमक्रियाः}
{सर्वाः सात्वतमास्थाय विधिं चक्रे समाहितः}


\twolineshloka
{पाञ्चरात्रविदो मुख्यास्तस्य गेहे महात्मनः}
{वरान्नं भगवत्प्रोक्तं भुञ्जते वाऽग्रभोजनम्}


\threelineshloka
{तस्य प्रशासतो राज्यं धर्मेणामित्रघातिनः}
{नानृता वाक्समभवन्मनो दुष्टं न चाभवत्}
{न च कायेन कृतवान्स पापं परमण्वपि}


\twolineshloka
{ये हि ते ऋषयः ख्याताः सप्त चित्रशिखण्डिनः}
{तैरेकमतिभिर्भूत्वा यत्प्रोक्तं शास्त्रमुत्तमम्}


\twolineshloka
{वेदैश्चतुर्भिः समितं कृतं मेरौ महागिरौ}
{आस्यैः सप्तभिरुद्गीर्णं लोकधर्ममनुत्तमम्}


\twolineshloka
{मरीचिरत्र्यङ्गिरसौ पुलस्त्यः पुलहः क्रतुः}
{वसिष्ठश्च महातेजास्ते हि चित्रशिखण्डिनः}


\twolineshloka
{सप्त प्रकृतयो ह्येतास्तथा स्वायंभुवोऽष्टमः}
{एताभिर्धार्यते लोकस्ताभ्याः शास्त्रं विनिःसृतं}


\twolineshloka
{एकाग्रमनसो दान्ता मुनयः संयमे रताः}
{भूतभव्यभविष्यज्ञाः सत्यधर्मपरायणाः}


\twolineshloka
{इदं श्रेय इदं ब्रह्म इदं हितमनुत्तमम्}
{लोकान्संचिन्त्य मनसा ततः शास्त्रं प्रचक्रिरे}


\twolineshloka
{तत्र धर्मार्थकामा हि मोक्षः पश्चाच्च कीर्तितः}
{मर्यादा विविधाश्चैव दिवि भूमौ च संस्थिताः}


\twolineshloka
{आराध्य तपसा देवं हरिं नारायणं प्रभुम्}
{दिव्यं वर्षसहस्रं वै सर्वे ते ऋषिभिः सह}


\twolineshloka
{नारायणानुशिष्टा हि तदा देवी सरस्वती}
{विवेश तानृषीन्सर्वाल्लोँकानां हितकाम्यया}


\twolineshloka
{ततः प्रवर्तिता सम्यक्तपोविद्भिर्द्विजातिभिः}
{शब्दे चार्थे च हेतौ च एषा प्रथमसर्गजा}


\twolineshloka
{आदावेव हि तच्छास्त्रमोंकारस्वरपूजितम्}
{ऋषिभिः श्रावितं तत्र यत्र कारुणिकोह्यसौ}


\twolineshloka
{ततः प्रसन्नो भगवाननिर्दिष्टशरीरगः}
{ऋषीनुवाच तान्सर्वानदृश्यः पुरुषोत्तमः}


\twolineshloka
{कृतं शतसहस्रं हि श्लोकानां हितमुत्तमम्}
{लोकतन्त्रस्य कृत्स्नस्य यस्माद्धर्मः प्रवर्तते}


\twolineshloka
{प्रवृत्तौ च निवृत्तौ च यस्मादेतद्भविष्यति}
{यजुर्ऋक्सामभिर्जुष्टमथर्वाङ्गिरसैस्तथा}


\threelineshloka
{यथाप्रमाणं हिं मया कृतो ब्रह्म प्रसादतः}
{रुद्रश्च क्रोधजो विप्रा यूयं प्रकृतयस्तथा}
{}


\twolineshloka
{सूर्याचन्द्रमसौ वायुर्भूमिरापोऽग्निरेव च}
{सर्वे च नक्षत्रगणा यच्च भूताभिशब्दितम्}


\twolineshloka
{अधिकारेषु वर्तन्ते यथास्वं ब्रह्मवादिनः}
{सर्वे प्रमाणं हि यथा तथा तच्छास्त्रमुत्तमम्}


\twolineshloka
{भविष्यति प्रमाणं वै एतन्मदनुशासनम्}
{तस्मात्प्रवक्ष्यते धर्मान्मनुः स्वायंभुवः स्वयम्}


\twolineshloka
{उशना बृहस्पतिश्चैव यदोत्पन्नौ भविष्यतः}
{तदा प्रवक्ष्यतः शास्त्रं युष्मन्मतिभिरुद्धृतम्}


\twolineshloka
{स्वायंभुवेषु धर्मेषु शास्त्रे चोशनसा कृते}
{बृहस्पतिमते चैव लोकेषु प्रतिचारिते}


\twolineshloka
{युष्मत्कृतमिदं शास्त्रं प्रजापालो वसुस्ततः}
{बृहस्पतिसकाशाद्वै प्राप्स्यते द्विजसत्तमाः}


\twolineshloka
{स हि मद्भावनिरतो मद्भक्तश्च भविष्यति}
{तेन शास्त्रेण लोकेषु क्रियाः सर्वाः करिष्यति}


\twolineshloka
{एतद्धि युष्मच्छास्त्राणां शास्त्रमुत्तमसंज्ञितम्}
{एतदर्थ्यं च धर्म्यं च रहस्यं चैतदुत्तमम्}


\twolineshloka
{अस्य प्रवर्तनाच्चैव प्रजावन्तो भविष्यथ}
{स च राजश्रिया युक्तो भविष्यति महान्वसुः}


\twolineshloka
{संस्थिते तु नृपे तस्मिञ्शास्त्रमेतत्सनातनम्}
{अन्तर्धास्यति तत्सर्वमेतद्वः कथितं मया}


\twolineshloka
{एतावदुक्त्वा वचनमदृश्यः पुरुषोत्तमः}
{विसृज्य तानृषीन्सर्वान्कामपि प्रसृतो दिशम्}


\twolineshloka
{ततस्ते लोकपितरः सर्वलोकार्थचिन्तकाः}
{प्रावर्तयन्त तच्छास्त्रं धर्मयोनिं सनातनम्}


\twolineshloka
{उत्पन्नेऽङ्गिरसे चैव युगे प्रथमकल्पिते}
{साङ्गोपनिषदं शास्त्रं स्थापयित्वा बृहस्पतौ}


% Check verse!
जग्मुर्यथेप्सितं सर्वलोकानां सर्वधर्मप्रवर्तकाः
\chapter{अध्यायः ३४४}
\twolineshloka
{भीष्म उवाच}
{}


\twolineshloka
{ततोऽतीते महाकल्पे उत्पन्नेऽङ्गिरसः सुते}
{बभूवुर्निर्वृता देवा जाते देवपुरोहिते}


\twolineshloka
{बृहद्ब्रह्म महच्चेति शब्दाः पर्यायवाचकाः}
{एभिः समन्वितो राजन्गुणैर्विद्वान्बृहस्पतिः}


\twolineshloka
{तस्य शिष्यो बभूवाग्र्यो राजोपरिचरो वसुः}
{अधीतवांस्तदा शास्त्रं सम्यक्चित्रशिखण्डिजं}


\twolineshloka
{स राजा भावितः पूर्वं दैवेन विधिना वसुः}
{पालयामास पृथिवीं दिवमाखण्डलो यथा}


\twolineshloka
{तस्य यज्ञो महानासीदश्वमेधो महात्मनः}
{बृहस्पतिरुपाध्यायस्तत्र होता बभूव ह}


\twolineshloka
{प्रजापतिसुताश्चात्र सदस्याश्चाभवंस्त्रयः}
{एकतश्च द्वितश्चैव त्रितश्चैव मर्हषयः}


\twolineshloka
{धनुषाख्योऽथ रैभ्यश्च अर्वावसुपरावसू}
{ऋषिर्मोधातिथिश्चैव ताण्ड्यश्चैव महानृषिः}


\twolineshloka
{ऋषिः शान्तिर्महाभागस्तथा वेदशिराश्च यः}
{ऋषिश्रेष्ठश्च कपिलः शालिहोत्रपिता स्मृतः}


\twolineshloka
{आद्यः कठस्तैत्तिरिश्च वैशंपायनपूर्वजः}
{कण्वोऽथ देवहोत्रश्च एते षोडश कीर्तिताः}


\twolineshloka
{संभूताः सर्वसंभारास्तस्मिन्राजन्महाक्रतौ}
{न तत्र पशुघातोऽभूत्स राजैवं स्थितोऽभवत्}


\twolineshloka
{अहिंस्रः शुचिरक्षुद्रो निराशीः कर्मसंस्तुतः}
{आरण्यकपदोद्भूता भागास्तत्रोपकल्पिताः}


\twolineshloka
{प्रीतस्ततोऽस्य भगवान्देवदेवः पुरातनः}
{साक्षात्तं दर्शयामास सोदृश्योऽन्येन केनचित्}


\twolineshloka
{स्वयं भागमुपाघ्राय पुरोडाशं गृहीतवान्}
{अदृश्येन हृतो भागो देवेन हरिमेधसा}


\twolineshloka
{बृहस्पतिस्ततः क्रुद्धः स्रुचमुद्यम्य वेगितः}
{आकाशं घ्नन्स्रुचः पातै रोषादश्रूण्यवर्तयत्}


\twolineshloka
{उवाच चोपरिचरं मया भागोऽयमुद्यतः}
{ग्राह्यः स्वयं हि देवेन मत्प्रत्यक्षं न संशयः}


\threelineshloka
{उद्यता यज्ञभागा हि साक्षात्प्राप्ताः सुरैरिह}
{किमर्थमिह न प्राप्तो दर्शनं मे हरिर्नृप ॥भीष्म उवाच}
{}


\twolineshloka
{ततः स तं समुद्भूतं भूमिपालो महान्वसुः}
{प्रसादयामास मुनिं सदस्यास्ते च सर्वशः}


\twolineshloka
{`हुतस्त्वया वदानीह पुरोडाशस्य यावती}
{गृहीता देवदेवेन मत्प्रत्यक्षं न संशयः}


\twolineshloka
{इत्येवमुक्ते वसुना सरोषश्चाब्रवीद्गुरुः}
{न यजेयमहं चात्र परिभूतस्त्वया नृप}


\threelineshloka
{त्वया पशुर्वारितश्च कृतः पिष्टमयः पशुः}
{त्वं देवं पश्यसे नित्यं न पश्येयमहं कथम् ॥वसुरुवाच}
{}


\threelineshloka
{पशुहिंसा वारिता च यजुर्वेदादिमन्त्रतः}
{अहं न वारये हिंसां द्रक्ष्याम्येकान्तिको हरिम्}
{तस्मात्कोपो न कर्तव्यो भवता गुरुणा मयि}


\twolineshloka
{वसुमेवं ब्रुवाणं तु क्रुद्ध एव बृहस्पतिः}
{उवाच ऋत्विजश्चैव किं नः कर्मेति वारयन्}


\twolineshloka
{अथैकतो द्वितश्चैव त्रितश्चैव महर्षयः}
{'ऊचुश्चैनमसंभ्रान्ता न रोषं कर्तुमर्हसि}


\twolineshloka
{`शृणु त्वं वचनं पुत्र अस्माभिः समुदाहृतम्}
{'नैष धर्मः कृतयुगे यत्त्वं रोषमिहाहिथाः}


\threelineshloka
{अरोषणो ह्यसौ देवो यस्य भागोऽयमुद्यतः}
{न शक्यः स त्वया द्रष्टुमस्माभिर्वा बृहस्पते}
{यस्य प्रसादं कुरुते स वै तं द्रष्टुमर्हति}


\twolineshloka
{वयं हि ब्रह्मणः पुत्रा मानसाः परिकीर्तिताः}
{गता निःश्रेयसार्थं हि कदाचिद्दिशमुत्तराम्}


\twolineshloka
{तप्त्वा वर्षसहस्राणि चत्वारि तप उत्तमम्}
{एकपादा स्थिताः सम्यक्काष्ठभूताः समाहिताः}


\twolineshloka
{मेरोरुत्तरभागे तु क्षीरोदस्यानुकूलतः}
{स देशो यत्र नस्तप्तं तपः परमदारुणम्}


\twolineshloka
{वरेण्यं वरदं तं वै देवदेवं सनातनम्}
{कथं पश्येमहि वयं देवं नारायणं त्विति}


\twolineshloka
{अथ व्रतस्यावभृथे वागुवाचाशरीरिणी}
{स्निग्धगम्भीरया वाचा प्रहर्षणकरी विभो}


\twolineshloka
{सुतप्तं वस्तपो विप्राः प्रसन्नेनान्तरात्मना}
{यूयं जिज्ञासवो भक्ताः कथं द्रक्ष्यथ तं विभुम्}


\twolineshloka
{क्षीरोदधेरुत्तरतः श्वेतद्वीपो महाप्रभः}
{तत्र नारायणपरा मानवाश्चन्द्रवर्चसः}


\twolineshloka
{एकान्तभावोपगतास्ते भक्ताः पुरुषोत्तमम्}
{ते सहस्रार्चिषं देवं प्रविशन्ति सनातनम्}


\threelineshloka
{अनिन्द्रिया निराहारा अनिष्पन्दाः सुगन्धिनः}
{एकान्तिनस्ते पुरुषाः श्वेतद्वीपनिवासिनः}
{12-344-34ca गच्छध्वंतत्र मुनयस्तत्रात्मा मे प्रकाशितः}


\twolineshloka
{अथ श्रुत्वा वयं सर्वे वाचं तामशरीरिणीम्}
{यथाख्यातेन मार्गेण तं देशं प्रविशेमहि}


\twolineshloka
{प्राप्य श्वेतं महाद्वीपं तच्चित्तास्तद्दिदृक्षवः}
{`सहसा हि गताः सर्वे तेजसा तस्य मोहिताः ॥'}


\twolineshloka
{ततोऽस्मद्दृष्टिविषयस्तदा प्रतिहतोऽभवत्}
{न च पश्याम पुरुषं तत्तेजोहतदर्शनाः}


\twolineshloka
{ततो नः प्रादुरभवद्विज्ञानं देवयोगजम्}
{न किलातप्ततपसा शक्यते द्रष्टमञ्जसा}


\twolineshloka
{ततः पुनर्वर्षशतं तप्त्वा तात्कालिकं महत्}
{व्रतावसाने च शुभान्नरान्ददृशिमो वमय्}


\twolineshloka
{श्वेतांश्चन्द्रप्रतीकाशान्सर्वलक्षणलक्षितान्}
{नित्याञ्जलिकृतान्ब्रह्म जपतः प्रागुदङ्भुखान्}


\twolineshloka
{मानसो नाम स जपो जप्यते तैर्महात्मभिः}
{तेनैकाग्रमनस्त्वेन प्रीतो भवति वै हरिः}


\twolineshloka
{याऽभवन्मुनिशार्दूल भाः सूर्यस्य युगक्षये}
{एकैकस्य प्रभा तादृक्साऽभवन्मानवस्य ह}


\twolineshloka
{तेजोनिवासः स द्वीप इति वै मेनिरे वयम्}
{न तत्राभ्यधिकः कश्चित्सर्वे ते समतेजसः}


\twolineshloka
{अथ सूर्यसहस्रस्य प्रभां युगपदुत्थिताम्}
{सहसा दृष्टवन्तः स्म पुनरेव बृहस्पते}


\twolineshloka
{सहिताश्चाभ्यधावन्त ततस्ते मानवा द्रुतम्}
{कृताञ्जलिपुष्टा हृष्टा नम इत्येव वादिनः}


\twolineshloka
{ततो हि वदतां तेषामश्रौष्म विपुलं ध्वनिम्}
{बलिः किलोपह्रियते तस्य देवस्य तैर्नरैः}


\twolineshloka
{वयं तु तेजसा तस्य सहसा हृतचेतसः}
{न किंचिदपि पश्यामो हतचक्षुर्बलेन्द्रियाः}


\twolineshloka
{एकस्तु शब्दो विततः श्रुतोऽस्माभिरुदीरितः}
{`आकाशं पूरयन्सर्वं शिक्षाक्षरसमन्वितः}


\threelineshloka
{जितं ते पुण्डरीकाक्ष नमस्ते विश्वभावन}
{नमस्तेऽस्तु हृषीकेश महापुरुष पूर्वज}
{इति शब्दः श्रुतोऽस्माभिः शिक्षाक्षरसमन्वितः}


\twolineshloka
{एतस्मिन्नन्तरे वायुः सर्वगन्धवहः शुचिः}
{दिव्यान्युवाह पुष्पाणि कर्मण्याश्चौषधीस्तथा}


\twolineshloka
{तैरिष्टः पञ्चकालज्ञैर्हरिरेकान्तिभिर्नरैः}
{भक्त्या परमया युक्तैर्मनोवाक्कर्मभिस्तदा}


\twolineshloka
{नूनं तत्रागतो देवो यथा तैर्वागुदीरिता}
{वयं त्वेनं न पश्यामो मोहितास्तस्य मायया}


\twolineshloka
{मारुते सन्निवृत्ते च बलौ च प्रतिपादिते}
{चिन्ताव्याकुलितात्मानो जाताः स्मोङ्गिसांवर}


\twolineshloka
{मानवानां सहस्रेषु तेषु वै शुद्धयोनिषु}
{अस्मान्न कश्चिन्मनसा चक्षुषा वाऽप्यपूजयत्}


\twolineshloka
{तेऽपि स्वस्था मुनिगणा एक भावमनुव्रताः}
{नास्मासु दधिरे भावं ब्रह्मभावमनुष्ठिताः}


\threelineshloka
{ततोऽस्मान्सुपरिश्रान्तांस्तपसा चातिकर्शितान्}
{उवाच स्वस्थं किमपि भूतं तत्राशरीरकम् ॥देव उवाच}
{}


\twolineshloka
{दृष्टा वः पुरुषाः श्वेताः सर्वेन्द्रियविवर्जिताः}
{दृष्टो भवति देवेश एभिर्दृष्टैर्द्विजोत्तमैः}


\twolineshloka
{गच्छध्वं मुनयः सर्वे यथागतमितोऽचिरात्}
{न स शक्यस्त्वभक्तेन द्रष्टुं देवः कथंचन}


\twolineshloka
{कामं कालेन महता एकान्तित्वमुपागतैः}
{शक्यो द्रष्टुं स भगवान्प्रभामण्डलदुर्दृशः}


\twolineshloka
{महत्कार्यं च कर्तव्यं युष्माभिर्द्विजसत्तमाः}
{इतः कृतयुगेऽतीते विपर्यासं गतेऽपि च}


\twolineshloka
{वैवस्वतेऽन्तरे विप्राः प्राप्ते त्रेतायुगे पुनः}
{सुराणां कार्यसिद्ध्यर्थं सहाया वै भविष्यथ}


\twolineshloka
{ततस्तदद्भुतं वाक्यं निशम्यैवामृतोपमम्}
{तस्य प्रसादात्प्राप्ताः स्मो देशमीप्सिंतमञ्जसा}


\twolineshloka
{एवं सुतपसा चैव हव्यकव्यस्तैथैव च}
{देवोऽस्माभिर्न दृष्टः स कथं त्वं द्रष्टुमर्हसि}


\twolineshloka
{नारायणो महद्भूतं विश्वसृग्घव्यकव्यभुक्}
{अनादिनिधनोऽव्यक्तो देवदानवपूजितः}


\threelineshloka
{एवमेकतवाक्येन द्वितत्रितमतेन च}
{अनुनीतः सदस्यैश्च बृहस्पतिरुदारधीः}
{समापयत्ततो यज्ञं दैवतं समपूजयत्}


\twolineshloka
{समाप्तयज्ञो राजाऽपि प्रजां पालितवान्वसुः}
{ब्रह्मशापाद्दिवो भ्रष्टः प्रविवेश महीं ततः}


\twolineshloka
{स राजा राजशार्दूल सत्यधर्मपरायणः}
{अन्तर्भूमिगतश्चैव सततं धर्मवत्सलः}


\twolineshloka
{नारायणपरो भूत्वा नारायणजपं जपन्}
{तस्यैव च प्रसादेन पुनरेवोत्थितस्तु सः}


\twolineshloka
{महीतलाद्गतः स्थानं ब्रह्मणः समनन्तरम्}
{परां गतिमनुप्राप्त इति नैष्ठिकमञ्जसा}


\chapter{अध्यायः ३४५}
\twolineshloka
{युधिष्ठिर उवाच}
{}


\threelineshloka
{यदा भक्तो भगवति आसीद्राजा महान्वसुः}
{किमर्थं स परिभ्रष्टो विवेश विवरं भुवः ॥भीष्म उवाच}
{}


\twolineshloka
{अत्राप्युदाहरन्तीममितिहासं पुरातनम्}
{ऋषीणां चैव संवादं त्रिदशानां च भारत}


\twolineshloka
{`इयं वै कर्मभूमिः स्यात्स्वर्गो भोगाय कल्पितः}
{तस्मादिन्द्रो महीं प्राप्य यजनाप तु दीक्षितः}


\twolineshloka
{सवनीयपशोः काल आगते तु बृहस्पतिः}
{पिष्टमानीयतामत्र पश्वर्थमिति भाषत}


\twolineshloka
{तच्छ्रुत्वा देवताः सर्वा इदमूचुर्द्विजोत्तमम्}
{बृहस्पतिं मांसगृद्धाः पृथक्पृथगिदं पुनः ॥'}


\threelineshloka
{अजेन यष्टव्यमिति प्राहुर्देवा द्विजोत्तमान्}
{स च छागोप्यजो ज्ञेयो नान्यः पशुरिति स्थितिः ॥ऋषय ऊचुः}
{}


\twolineshloka
{बीजैर्यज्ञेषु यष्टव्यमिति वै वैदिकी श्रुतिः}
{अजसंज्ञानि बीजानि च्छागं नो हन्तुमर्हथ}


\threelineshloka
{नैष धर्मः सतां देवा यत्र बध्येत वै पशुः}
{इदं कृतयुगं श्रेष्ठं कथं वध्येत वै पशुः ॥भीष्म उवाच}
{}


\twolineshloka
{तेषां संवदतामेवमृषीणां विबुधैः सह}
{मार्गागतो नृपश्रेष्ठस्तं देशं प्राप्तवान्वसुः}


\twolineshloka
{अन्तरिक्षचरः श्रीमान्सहस्रबलवाहनः}
{तं दृष्ट्वा सहसाऽऽयान्तं वसुं ते त्वन्तरिक्षगम्}


\threelineshloka
{ऊचुर्द्विजातयो देवानेष च्छेत्स्यति संशयम्}
{यज्वा दानपतिः श्रेष्ठः सर्वभूतहितप्रियः}
{कथंस्विदन्यथा ब्रूयादेष वाक्यं महान्वसुः}


\twolineshloka
{एवं ते संविदं कृत्वा विबुधा ऋषयस्तथा}
{अपृच्छन्सहिताऽभ्येत्य वसुं राजानमन्तिकात्}


\twolineshloka
{भो राजन्केन यष्टव्यमजेनाहोस्विदौषधैः}
{एतन्नः संशयं छिन्धि प्रमाणं नो भवान्मतः}


\threelineshloka
{स तान्कृताञ्जलिर्भूत्वा परिपप्रच्छ वै वसुः}
{कस्य वै को मतः पक्षो ब्रूत सत्यं द्विजोत्तमाः ॥ऋषय ऊचुः}
{}


\threelineshloka
{धान्यैर्यष्टव्यमित्येव पक्षोऽस्माकं नराधिप}
{देवानां तु पशुः पक्षो मतो राजन्वदस्व नः ॥भीष्म उवाच}
{}


\twolineshloka
{देवानां तु मतं ज्ञात्वा वसुना पक्षसंश्रयात्}
{छागेनाजेन यष्टव्यमेवमुक्तं वचस्तदा}


\twolineshloka
{कुपितास्ते ततः सर्वे मुनयः सूर्यवर्चसः}
{ऊचुर्वसुं विमानस्थं देवपक्षार्थवादिनम्}


\twolineshloka
{सुरपक्षो गृहीतस्ते यस्मात्तस्माद्दिवः पत}
{अद्यप्रभृति ते राजन्नाकाशे विहता गतिः}


\threelineshloka
{अस्माच्छापाभिघातेन महीं भित्त्वा प्रवेक्ष्यसि}
{` विरुद्धं वेदसूत्राणामुक्तं यदि भवेन्नृप}
{वयं विरुद्धवचना यदि तत्र पतामहे ॥'}


\twolineshloka
{ततस्तस्मिन्मुहूर्तेऽथ राजोपरिचरस्तदा}
{अधो वै संबभूवाशु भूमेर्विवरगो नृप}


% Check verse!
स्मृतिस्त्वेवं न विजहौ तदा नारायणाज्ञया
\twolineshloka
{देवास्तु सहिताः सर्वे वमोः शापविमोक्षणम्}
{चिन्तयामासुरव्यग्राः सुकृतं हि नृपस्य तत्}


\twolineshloka
{अनेनास्मत्कृते राज्ञा शापः प्राप्तो महात्मना}
{अस्य प्रतिप्रियं कार्यं सहितैर्नो दिवौकसः}


\twolineshloka
{इति बुद्ध्या व्यवस्याशु गत्वा निश्चयमीश्वराः}
{ऊचुः संहृष्टमनसो राजोपरिचरं तदा}


\twolineshloka
{ब्रह्मण्य देवभक्तस्त्वं सुरासुरगुरुर्हरिः}
{कामं स तव तुष्टात्मा कुर्याच्छापविभोक्षणम्}


\twolineshloka
{मानना तु द्विजातीनां कर्तव्या वै महात्मनाम्}
{अवश्यं तपसा तेषां फलितव्यं नृपोत्तम}


\twolineshloka
{यतस्त्वं सहसा भ्रष्ट आकाशान्मेदिनीतलम्}
{`विरुद्धं वेदसूत्राणां न वक्तव्यं हितार्थिना}


\threelineshloka
{अस्मत्पक्षनिमित्तेन व्यसनं प्राप्तमीदृशम्}
{'एकं त्वनुग्रहं तुभ्यं दद्मो वै नृप्रसत्तम}
{यावत्त्वं शापदोषेण कालमासिप्यसेऽनघ}


\twolineshloka
{भूमेर्विवरगो भूत्वा तावत्त्वं कालमाप्स्यसि}
{यज्ञेषु सुहुतां विप्रैर्वसोर्धारां समाहितैः}


\twolineshloka
{प्राप्स्यसेऽस्मदनुध्यानान्मा च त्वां ग्लानिराविशेत्}
{न क्षुत्पिपासे राजेन्द्र भूमेश्छिद्रे भविष्यतः}


\twolineshloka
{वसोर्धारामिपीतत्वात्तेजसाऽऽप्यायितेन च}
{स देवोऽस्मद्वरात्प्रीतो ब्रह्मलोकं हि नेष्यति}


\twolineshloka
{एवं दत्त्वा वरं राज्ञे सर्वे ते च दिवौकसः}
{ऋतुं समाप्य पिष्टेन मुनीनां वचनात्तदा ॥'}


\twolineshloka
{गताः धमवनं देवा ऋषगश्च तपोधनाः}
{`गृहीत्वा दक्षिणां सर्वे गतः स्वानाश्रमान्पुनः}


\twolineshloka
{वसुं विचिन्त्य शक्रश्च प्रविनेशामरावतीम्}
{वसुर्विवरगस्तत्र व्यलीकस्य फलं गुरोः ॥'}


\twolineshloka
{चक्रे वसुस्ततः पूजां विष्वक्सेनाय भारत}
{जप्यं जगौ च सततं नारायणमुखोद्गवम्}


\twolineshloka
{तत्रापि पञ्चभिर्यज्ञैः पञ्चकालानरिंदम्}
{अयजद्धरिं सुरपतिं भूमेर्विवरगोऽपि सन्}


\twolineshloka
{ततोऽस्य तुष्टो भगवान्भक्त्या नारायणो हरिः}
{अनन्यभक्तस्य सतस्तत्परस्य जितात्मनः}


\twolineshloka
{वरदो भगवान्विष्णुः समीपस्थं द्विजोत्तमम्}
{गरुत्मन्तं महावेगमावभाषेऽप्सितं तदा}


\twolineshloka
{द्विजोत्तम महाभाग पश्यतां वचनान्मम}
{सम्राड्राजा वसुर्नाम धर्मात्मा संशितव्रतः}


\twolineshloka
{ब्राह्मणानां प्रकोपेन प्रविष्टो वसुधातलम्}
{मानितास्ते तु विप्रेन्द्रास्त्वं तु गच्छ द्विजोत्तम्}


\twolineshloka
{भूमेर्विवरसंगुप्तं गरुडेह ममाज्ञया}
{अधश्चरं नृपश्रेष्ठं खेचरं कुरु माचिरम्}


\twolineshloka
{गरुत्मानथ विक्षिप्य पक्षौ मारुतवेगवान्}
{विवेश विवरं भूमेर्यत्रास्ते वाग्यतो वसुः}


\twolineshloka
{तत एनं समुत्क्षिप्य सहसा विनतासुतः}
{उत्पपात नभस्तूर्णं तत्र चैनममुञ्चत}


\twolineshloka
{अस्मिन्मुहुर्ते संजज्ञे राजोपरिचरः पुनः}
{सशरीरो गतश्चैव ब्रह्मलोकं नृपोत्तमः}


\twolineshloka
{एवं तेनापि कौन्तेय वाग्दोषाद्देवताज्ञया}
{प्राप्ता गतिरधस्तात्तु द्विजशापान्महात्मना}


\threelineshloka
{केवलं पुरुषस्तेन सेवितो हरिरीश्वरः}
{ततः शीघ्रं जहौ शापं ब्रह्मलोकमवाप च ॥भीष्म उवाच}
{}


\threelineshloka
{एतत्ते सर्वमाख्यातं संभूता मानवा यथा}
{नारदोऽपि यथा श्वेतं द्वीपं स गतवानृषिः}
{तत्ते सर्वं प्रवक्ष्यामि शृणुष्वैकमना नृप}


\chapter{अध्यायः ३४६}
\twolineshloka
{भीष्म उवाच}
{}


\twolineshloka
{प्राप्य श्वेतं महाद्वीपं नारदो भगवानृषिः}
{ददर्श तानेव नराञ्श्वेतांश्चन्द्रसमप्रभान्}


\twolineshloka
{पूजयामास शिरसा मनसा तैश्च पूजितः}
{दिदृक्षुर्जप्यपरमः सर्वकृच्छ्रगतः स्थितः}


\threelineshloka
{भूत्वैकाग्रमना विप्र ऊर्ध्वबाहुः समाहितः}
{स्तोत्रं जगौ स विश्वाय निर्गुणाय गुणात्मने ॥नारद उवाच}
{}


% Check verse!
ओँ नमस्ते देवदेवेश निष्क्रिय निर्गुण लोकसाक्षिन् क्षेत्रज्ञपुरुषोत्तम अनन्त पुरुष महापुरुष पुरुषोत्तम त्रिगुण प्रधान अमृतअमृताख्य अनन्ताख्य व्योम अमृतात्मन् सनातन सदसद्व्यक्ताव्यक्त ऋतधामआदिदेव वसुप्रद प्रजापते सुप्रजापते वनस्पते महाप्रजापते ऊर्जस्पतेवाचस्पते जगत्पते मनस्पते दिवस्पते मरुत्पते सलिलपते पृथिवीपते दिक्पतेपूर्वनिवास गुह्य ब्रह्मपुरोहित ब्रह्मकायिक राजिक महाराजिकचातुर्महाराजिक आभासुर महाभासुर सप्तमहाभाग सप्तमहास्वरयाम्य महायाम्यसंज्ञासंज्ञ *तुपित महातुषित प्रमर्दन परिनिर्मित अपरिनिर्मितवशवर्तिन् अपरिनिन्दित अपरिमित वशवर्तिन् अवशवर्तिन् यज्ञ महायज्ञअसंयज्ञ यज्ञसंभव यज्ञयोने यज्ञगर्भ यज्ञहृदय यज्ञस्तुत यज्ञभागहरपञ्चयज्ञ पञ्चकालकर्तृपते पाञ्चरात्रिक वैकुण्ठ अपराजित मानसिक नावमिकनामनामिक परस्वामिन् सुस्नातहंस परमहंस महाहंस परमज्ञेय हिरण्येशयवेदेशय देवेशय कुशेशयब्रह्मेशय पद्मेशय विश्वेश्वर विष्वक्सेन त्वंजगदन्वयस्त्वं जगत्प्रकृतिस्तवाग्निरास्यं वडवामुखोऽग्निस्त्वमाहुतिःसारथिस्त्वं वषट््कारस्त्वमोंकारस्त्वं तपस्त्वं मनस्त्वं चन्द्रमाःपूर्णाङ्गस्त्वं चक्षुराज्यम् त्वं सूर्यस्त्वं दिशांगजस्त्वं दिग्भानोविदिग्भानो हयशिरः प्रथमत्रिसौपर्णो वर्णधरः पञचाग्रे त्रिणाचिकेतषडङ्गनिधान प्राग्जोतिष ज्येष्ठसामग सामिकव्रतधराथर्वशिराः पञ्चमहाकल्पफेनपाचार्य बालखिल्य वैखानसा भग्नयोगा भग्नव्रता भग्नपरिसंख्यान युगादेयुगमध्य युगनिधनाखण्डल प्राचीनगर्भकौशिक पुरुष्टुत पुरुहूतविश्वकृद्विश्वजिद्विश्वरूपानन्तगतेऽनन्तभोगाऽनन्ताऽनादेऽमध्याऽव्यक्तमध्याऽव्यक्तनिधनव्रतावास समुद्राधिवास यशोवास तपोवास दमावास लक्ष्म्यावास विद्यावासकीर्त्यावास श्रीवास सर्वावास वासुदेव सर्वच्छन्दक हरिहय हरिमेधमहायज्ञभागहर वरप्रद सुखप्रद धनप्रद हरिमेध यम नियम महानियमकृच्छ्राऽतिकृच्छ्रा महाकुच्छ्र सर्वकृच्छ्र नियमधर निवृत्तभ्रमनिवृत्तिधर्मप्रवरगत प्रवचनगत पृश्निगर्भप्रवृत्तप्रवृत्तवेदक्रियाऽजसर्वगते सर्वदर्शिनं नग्राह्याऽक्षयाऽचल महाविभूतेमाहात्म्यशरीर पवित्र महापवित्र हिरण्यमय बृहदप्रतर्क्याऽविज्ञेयब्रह्माग्र्य प्रजासर्गकर प्रजानिधनकर महामायाधर विद्याधर योगधरचित्रशिखण्डिन् वरप्रद पुरोडाशभागहर गताध्वरच्छिन्नतृष्ण च्छिन्नसंशयसर्वतोवृत्त निवृत्तरूप ब्राह्मणरूप च्छिन्नसंशय सर्वतोवृत्तनिवृत्तरूप ब्राह्मणरूप ब्राह्मणप्रिय विश्वर्मूर्ते महामूर्ते बान्धवभक्तवत्सल ब्रह्मण्यदेव भक्तोऽहं त्वां दिदृक्षुरेकान्तदर्शनायनमोनमः
\chapter{अध्यायः ३४७}
\twolineshloka
{भीष्म उवाच}
{}


\threelineshloka
{एवं स्तुतः स भगवान् गुह्यैस्तथ्यैश्च नामभिः}
{`भगवान्विश्वसृक्सिंहः सर्वमूर्तिमयः प्रभुः}
{'दर्शयामास मुनये रूपं तत्परमं हरिः}


\twolineshloka
{किंचिच्चन्द्राद्विशुद्धात्मा किंचिच्चन्द्राद्विशेषवान्}
{कृशानुवर्णः किंचिच्च किंचिद्धिष्ण्याकृतिः प्रभुः}


\twolineshloka
{शुकपत्रनिभः किंचित्किंचित्स्फटिकसन्निभः}
{नीलाञ्जनचयप्रख्यो जातरूपप्रभः क्वचित्}


\twolineshloka
{प्रबालाङ्कुरवर्णश्च श्वेतवर्णस्तथा क्वचित्}
{क्वचित्सुवर्णवर्णाभो वैदूर्यसदृशः क्वचित्}


\twolineshloka
{नीलवैर्दूर्यसदृश इन्द्रनीलनिभः क्वचित्}
{मयूरग्रीववर्णाभो मुक्ताहारनिभः क्वचित्}


\twolineshloka
{एतान्बहुधान्वर्णान्रूपैर्बिभ्रत्सनातनः}
{सहस्रनयनः श्रीमाञ्छतशीर्षः सहस्रपात्}


\twolineshloka
{सहस्रोदरबाहुश्च अव्यक्त इति च क्वचित्}
{ओंकारमुद्गिरन्वक्रात्सावित्रीं च तदन्वयाम्}


\twolineshloka
{शेषेभ्यश्चैव वक्रेभ्यश्चतुर्वेदान्गिरन्बहून्}
{आरण्यकं जगौ देवो हरिर्नारायणो वशी}


\threelineshloka
{वेदिं कमण्डलुं दर्भान्मणिरूपांस्तथा कुशान्}
{अजिनं दण्डकाष्ठं च ज्वलितं च हुताशनम्}
{धारयामास देवेशो हस्तैर्यज्ञपतिस्तदा}


\twolineshloka
{तं प्रसन्नं प्रसन्नात्मा नारदो द्विजसत्तमः}
{वाग्यतः प्रणतो भूत्वा ववन्दे परमेश्वरम्}


\twolineshloka
{तमुवाच नतं मूर्ध्ना देवानामादिरव्ययः ॥श्रीभगवानुवाच}
{}


\twolineshloka
{एकतश्च द्वितश्चैव त्रितश्चैव महर्षयः}
{इमं देशमनुप्राप्ता मम दर्शनलालसाः}


\twolineshloka
{न च मां ते ददृशिरे न च द्रक्ष्यति कश्चन}
{ऋते ह्यैकान्तिकश्रेष्ठात्त्वं चैवैकान्तिकोत्तम}


\twolineshloka
{ममैतास्तनवः श्रेष्ठा जाता धर्मगृहे द्विज}
{तास्त्वं भजस्व सततं साधयस्व यथागतम्}


\threelineshloka
{वृणीष्व च वरं प्रिय मत्तस्त्वं यदिहेच्छसि}
{प्रसन्नोऽहं तवाद्येह विश्वमूर्तिरिहाव्ययः ॥नारद उवाच}
{}


\twolineshloka
{अद्य मे तपसो देव यमस्य नियमस्य च}
{सद्यः फलमवाप्तं वै दृष्टो यद्भगवान्मया}


\threelineshloka
{वर एष ममात्यन्तं दृष्टस्त्वं यत्सनातनः}
{भगवन्विश्वदृक् सिंहः सर्वमूर्तिर्महान्प्रभुः ॥भीष्म उवाच}
{}


\twolineshloka
{एवं संदर्शयित्वा तु नारदं परमेष्ठिजम्}
{उवाच वचनं भूयो गच्छ नारद माचिरम्}


\twolineshloka
{इमे ह्यनिन्द्रियाहारा मद्भक्ताश्चन्द्रवर्चसः}
{एकाग्राश्चिन्तयेयुर्मां नैषां विघ्नो भवेदिति}


\twolineshloka
{सिद्धा ह्येते महाभागाः पुरा ह्येकान्तिनोऽभवन्}
{तमोरजोभिर्निर्मुक्ता मां प्रवेक्ष्यन्त्यसंशयम्}


\twolineshloka
{न दृश्यश्चक्षुषा योऽसौ न स्पृश्यः स्पर्शनेन च}
{न घ्रेयश्चैव गन्धेन रसेन च विवर्जितः}


\twolineshloka
{सत्वं रजस्तमश्चैव न गुणास्तं भजन्ति वै}
{यश्च सर्वगतः साक्षी लोकस्यात्मेति कथ्यते}


\twolineshloka
{भूतग्रामशरीरेषु नश्यत्सु न विनश्यति}
{अजो नित्यः शाश्वतश्च निर्गुणो निष्कलस्तथा}


\twolineshloka
{द्विर्द्वादशेभ्यस्तत्त्वेभ्यः ख्यातो यः पञ्चविंशकः}
{पुरुषो निष्क्रियश्चैव ज्ञानदृश्यश्च कथ्यते}


\twolineshloka
{यं प्रविश्य भवन्तीह मुक्ता वै द्विजसत्तमाः}
{स वासुदेवो विज्ञेयः परमात्मा सनातनः}


\twolineshloka
{पश्य देवस्य माहात्म्यं महिमानं च नारद}
{शुभाशुभैः कर्मभिर्यो न लिप्यति कदाचन}


\twolineshloka
{सत्वं रजस्तमश्चेति गुणानेतान्प्रचक्षते}
{एते सर्वशरीरेषु तिष्ठन्ति विचरन्ति च}


\twolineshloka
{एतान्गुणांस्तु क्षेत्रज्ञो भुङ्क्ते नैभिः स भुज्यते}
{निर्गुणो गुणभुक्चैव गुणस्रष्टा गुणातिगः}


\twolineshloka
{जगत्प्रतिष्ठा देवर्षे पृथिव्यप्सु प्रलीयते}
{ज्योतिष्यापः प्रलीयन्ते ज्योतिर्वायौ प्रलीयते}


\twolineshloka
{खे वायुः प्रलयं याति मनस्याकाशमेव च}
{मनो हि परमं भूतं तदव्यक्ते प्रलीयते}


\twolineshloka
{अव्यक्तं पुरुषे ब्रह्मन्निष्क्रिये संप्रलीयते}
{नास्ति तस्मात्परतरः पुरुषाद्वै सनातनात्}


\threelineshloka
{नित्यं हि नास्ति जगति भूतं स्थावरजङ्गमम्}
{ऋते तमेकं पुरुषं वासुदेवं सनातनम्}
{सर्वभूतात्मभूतो हि वासुदेवो महाबलः}


\twolineshloka
{पृथिवी वायुराकाशमापो ज्योतिश्च पञ्चमम्}
{ते समेता महात्मानः शरीरमिति संज्ञितम्}


\twolineshloka
{तदाविशति यो ब्रह्मन्न दृश्यो लघुविक्रमः}
{उत्पन्न एव भवति शरीरं चेष्टयन्प्रभुः}


\twolineshloka
{न विना धातुसंघातं शरीरं भवति क्वचित्}
{न च जीवं विना ब्रह्मन्वायवश्चेष्टयन्त्युत}


\twolineshloka
{स जीवः परिसंख्यातः शेषः संकर्षणः प्रभुः}
{तस्मात्सनत्कुमारत्वं योऽलभत्स्वेन कर्मणा}


\twolineshloka
{यस्मिंश्च सर्वभूतानि प्रद्युम्नः परिपठ्यते}
{तस्मात्प्रसूतो यः कर्ता कारणं कार्यमेव च}


\twolineshloka
{तस्मत्सर्वं संभवति जगत्स्थावरजङ्गमम्}
{सोऽनिरुद्धः स ईशानो व्यक्तिः सा सर्वकर्मसु}


\twolineshloka
{यो वासुदेवो भगवान्क्षेत्रज्ञो निर्गुणात्मकः}
{ज्ञेयः स एव राजेन्द्र जीवः संकर्षणः प्रभुः}


\twolineshloka
{संकर्षणाच्च प्रद्युम्नो मनोभूतः स उच्यते}
{प्रद्युम्नाद्योऽनिरूद्धस्तु सोहंकारः स ईश्वरः}


\twolineshloka
{मत्तः सर्वं संभवति जगत्स्थावरजङ्गमम्}
{अक्षरं च क्षरं चैव सच्चासच्चैव नारद}


\twolineshloka
{मां प्रविश्य भवन्तीह मुक्ता भक्तास्तु ये मम}
{अहं हि पुरुषो ज्ञेयो निष्क्रियः पञ्चविंशकः}


\twolineshloka
{निर्गुणो निष्कलश्चैव निर्द्वन्द्वो निष्परिग्रहः}
{एतत्त्वया न विज्ञेयं रूपवानिति दृश्यते}


\twolineshloka
{इच्छन्मुहूर्तान्नश्येयमीशोऽहं जगतो गुरुः}
{माया ह्येषा मया सृष्टा यन्मां पश्यसि नारद}


\twolineshloka
{सर्वभूतगुणैर्युक्तं नैवं त्वं ज्ञातुमर्हसि}
{मयैतत्कथितं सम्यक्तव मूर्तिचतुष्टयम्}


\twolineshloka
{अहं हि जीवसंज्ञो वै मयि जीवः समाहितः}
{मैवं ते बुद्धिरत्राभूर्द्दृषो जीवो मयेति वै}


\twolineshloka
{अहं सर्वत्रगो ब्रह्मन्भूतग्रामान्तरात्मकः}
{भूतग्रामशरीरेषु नश्यत्सु न नशाम्यहम्}


\twolineshloka
{सिद्धा हि ते महाभागा नरा ह्येकान्तिनोऽभवन्}
{तमोरजोभ्यां निर्मुक्ताः प्रवेक्ष्यन्ति च मां मुने}


\threelineshloka
{`अहं कर्ता च कार्यं च कारणं चापि नराद}
{न दृश्यश्चक्षुषा देवः स्पृश्यो न स्पर्शनेन च}
{आघ्रेयो नैव गन्धेन रसेन च विसर्जितः}


\twolineshloka
{सत्वं रजस्तमश्चैव न गुणास्ते भवन्ति हि}
{स हि सर्वगतः साक्षी लोकस्यात्मेति कथ्यते ॥'}


\twolineshloka
{हिरण्यगर्भो लोकादिश्चतुर्वक्रोऽनिरुक्तगः}
{ब्रह्मा सनातनो देवो मम बह्वर्थचिन्तकः}


\twolineshloka
{ललाटाच्चैव मे रुद्रो देवः क्रोधाद्विनिःसृतः}
{पश्यैकादश मे रुद्रान्दक्षिणं पार्श्वमास्थितान्}


\twolineshloka
{द्वादशैव तथाऽऽदित्यान्वामपार्श्वे समास्थितान्}
{अग्रतश्चैव मे पश्य वसूनष्टौ सुरोत्तमान्}


\twolineshloka
{नासत्यं चैव दस्रं च भिषजौ पश्य पृष्ठतः}
{सर्वान्प्रजापतीन्पश्य पश्य सप्तऋर्षीस्तथा}


\twolineshloka
{वेदान्यज्ञांश्च शतशः पश्यामृतमथौषधीः}
{तपांसि नियमांश्चैव यमानपि पृथग्विधान्}


\twolineshloka
{तथाऽष्टगुणमैश्वर्यमेकस्थं पश्य मूर्तिमत्}
{श्रियं लक्ष्मीं च कीर्तिं च पृथिवीं च ककुद्मिनी}


\twolineshloka
{वेदानां मातरं पश्य मत्स्थां देवीं सरस्वतीम्}
{ध्रुवं च ज्योतिषां श्रेष्ठं पश्य नारद खेचरम्}


\twolineshloka
{अम्भोधरान्समुद्रांश्च सरांसि सरितस्तथा}
{मूर्तिमन्तः पितृगणांश्चतुरः पश्य सत्तम}


\twolineshloka
{त्रींश्चैवेमान्गुणान्पश्य मत्स्थान्मूर्तिविवर्जितान्}
{देवकार्यादपि मुने पितृकार्यं विशिष्यते}


\twolineshloka
{देवानां च पितृणां च पिता ह्येकोऽहमादितः}
{अहं हयशिरा भूत्वा समुद्रे पश्चिमोत्तरे}


\twolineshloka
{पिबामि सुहुतं हव्यं कव्यं च श्रद्धयाऽन्वितम्}
{मया सृष्टः पुरा ब्रह्मा मां यज्ञमयजत्स्वयम्}


\twolineshloka
{ततस्तस्मै वरान्प्रीतो दत्तवानस्म्यनुत्तमान्}
{मत्पुत्रत्वं च कल्पादौ लोकाध्यक्षत्वमेव च}


\twolineshloka
{अहंकारकृतं चैव नामपर्यायवाचकम्}
{त्वया कृतां च मर्यादां नातिक्रंस्यति कश्चन}


\twolineshloka
{त्वं चैव वरदो ब्रह्मन्वरेप्सूनां भविष्यसि}
{सुरासुरगणानां च ऋषीणां च तपोधन}


\twolineshloka
{पितृणां च महाभाग सततं संशितव्रत}
{विविधानां च भूतानां त्वमुपास्यो भविष्यसि}


\twolineshloka
{प्रादुर्भावगतश्चाहं सुरकार्येषु नित्यदा}
{अनुशास्यस्त्वया ब्रह्मन्नियोज्यश्च सुतो यथा}


\threelineshloka
{एतांश्चान्यांश्च रुचिरान्ब्रह्मणेऽमिततेजसे}
{`एवं रुद्राय मनवे इन्द्रायामिततेजसे}
{'अहं दत्त्वा वरान्प्रीतो निवृत्तिपरमोऽभवम्}


\twolineshloka
{निर्वाणं सर्वधर्माणां निवृत्तिः परमा स्मृता}
{तस्मान्निवृत्तिमापन्नश्चरेत्सर्वाङ्गनिर्वृतः}


\twolineshloka
{विद्यासहायवन्तं मामादित्यस्थं सनातनम्}
{कपिलं प्राहुराचार्याः साङ्ख्यनिश्चितनिश्चयाः}


\twolineshloka
{हिरण्यगर्भो भगवानेष च्छन्दसि संस्तुतः}
{सोहं योगगतिर्ब्रह्मन्योगशास्त्रेषु शब्दितः}


\twolineshloka
{एषोऽहं व्यक्तिमाश्रित्य तिष्ठामि दिवि शाश्वतः}
{ततो युगसहस्रान्ते संहरिष्ये जगत्पुनः}


\twolineshloka
{कृत्वाऽऽत्मस्थानि भूतानि स्थावराणि चराणि च}
{एकाकी विद्यया सार्धं विहरिष्ये जगत्पुनः}


\twolineshloka
{ततो भूयो जगत्सर्वं करिष्यामीह विद्यया}
{अस्मिन्मूर्तिश्चतुर्थी या साऽसृजच्छेषमव्ययम्}


\twolineshloka
{स हि संकर्षणः प्रोक्तः प्रद्युम्नः सोप्यजीजनत्}
{प्रद्युम्नादनिरुद्धोऽहं सर्गो मम पुनः पुनः}


\twolineshloka
{अनिरुद्धात्तथा ब्रह्मा तन्नाभिकमलोद्भवः}
{ब्रह्मणः सर्वभूतानि चराणि स्थावराणि च}


\threelineshloka
{एतां सृष्टिं विजानीहि कल्पादिषु पुनः पुनः}
{यथा सूर्यस्य गगनादुदयास्तमने इह}
{नष्टे पुनर्वलात्काल आनयत्यमितद्युते}


\chapter{अध्यायः ३४८}
\twolineshloka
{भीष्म उवाच}
{}


\fourlineindentedshloka
{नारदः परिप्रपच्छ भगवन्तं जनार्दनम्}
{एकार्णवे महाघोरे नष्टे स्थावरजङ्गमे}
{श्रीभगवानुवाच}
{}


\threelineshloka
{शृणु नारद तत्वेन प्रादुर्भावान्महामुने}
{मत्स्यः कूर्मो वराहश्च नरसिंहोऽथ वामनः}
{रामो रामश्च रामश्च बुद्धः कल्कीति ते दश}


\twolineshloka
{पूर्वं मीनो भविष्यामि स्थापयिष्याम्यहं प्रजाः}
{लोकान्वै धारयिष्यामि मज्जमानान्महार्णवे}


\twolineshloka
{द्वितीयः कूर्मरूपो मे हेमकूटनिभः स्मृतः}
{मन्दरं धारयिष्यामि अमृतार्थं द्विजोत्तम}


\twolineshloka
{मप्रां महार्णवे घोरे भाराक्रान्तां भुवं पुनः}
{ततो बलादहं विद्वन्सर्वभूतहिताय वै ॥'}


\threelineshloka
{सत्वैराक्रान्तसर्वाङ्गां नष्टां सागरमेखलाम्}
{आनयिष्यामि स्वं स्थानं वाराहं रूपमास्थितः}
{हिरण्याक्षं हनिष्यामि दैतेयं बलगर्वितम्}


\twolineshloka
{नारसिंहं वषुः कृत्वा हिरण्यकशिषुं पुनः}
{सुरकार्ये हनिष्यामि यज्ञघ्नं दितिनन्दनम्}


\threelineshloka
{विरोचनस्य बलवान्बलिः पुत्रो महासुरः}
{अवध्यः सर्वलोकानां सदेवासुररक्षसाम्}
{भविष्यति स शक्रं च स्वराज्याच्च्यावयिष्यति}


\twolineshloka
{त्रैलोक्येऽपहृते तेन विमुखे च शचीपतौ}
{अदित्यां द्वादशः पुत्रः संभविष्यामि कश्यपान्}


\twolineshloka
{`वटुर्गत्वा यज्ञसदः स्तूयमानो द्विजोत्तमैः}
{यज्ञस्तुतिं करिष्यामि श्रुत्वा प्रीतो भवेद्वलिः}


\twolineshloka
{किमिच्छसि वटो ब्रूहीत्युक्तो याचे महद्वरम्}
{दीयतां त्रिपदीमात्रमिति याचे महाऽऽसुरम्}


\twolineshloka
{स दद्यान्मयि संप्रीतः प्रतिषिद्धश्च मन्त्रिभिः}
{यावज्जलं हस्तगतं त्रिभिर्विक्रमणैर्युतम् ॥'}


\twolineshloka
{ततो राज्यं प्रदास्यामि शक्रायामिततेजसे}
{देवताः स्थापयिष्यामि स्वस्वस्थानेषु नारद}


\twolineshloka
{बलिं चैव करिष्यामि पातालतलवासिनम्}
{दानवं च बलिश्रेष्ठमबध्यं सर्वदैवतैः}


\twolineshloka
{त्रेतायुगे भविष्यामि रामो भृगुकुलोद्वहः}
{क्षत्रं चोत्सादयिष्यामि समृद्धबलवाहनम्}


\twolineshloka
{संधौ तु समनुप्राप्ते त्रेतायां द्वापरस्य च}
{रामो दाशरथिर्भूत्वा भविष्यामि जगत्पतिः}


\twolineshloka
{त्रितोपघाताद्वैरूप्यमेकतोऽथ द्वितस्तथा}
{प्राप्स्येते वानरत्वं हि प्रजापतिसुतावृषी}


\threelineshloka
{तयोर्ये त्वन्वये जाता भविष्यन्ति वनौकसः}
{महाबला महावीर्याः शक्रतुल्यपराक्रमाः}
{}


\twolineshloka
{ते सहाया भविष्यन्ति सुरकार्ये मम द्विज ॥ततो रक्षःपतिं घोरं पुलस्त्यकुलपांसनम्}
{हनिष्ये रावणं रौद्रं सगणं लोककण्टकम्}


% Check verse!
` विभीषणाय दास्यामि राज्यं तस्य यथाक्रमम्अयोध्यावासिनः सर्वान्नेष्येऽहं लोकमव्ययम् ॥'
\twolineshloka
{द्वापरस्य कलेश्चैव संधौ पार्यवसानिके}
{प्रादुर्भावः कंसहेतोर्मथुरायां भविष्यति}


\twolineshloka
{तत्राहं दानवान्हत्वा सुबहून्देवकण्टकान्}
{कुशस्थलीं करिष्यासि निवासं द्वारकां पुरीम्}


\twolineshloka
{वसानस्तत्र वै पुर्यामदितेर्विप्रियंकरम्}
{हनिष्ये नरकं भौमं मुरं पीठं च दानवम्}


\twolineshloka
{प्राग्ज्योतिषं पुरं रम्यं नानाधनसमन्वितम्}
{कुशस्थलीं नष्यिष्यामि हत्वा वै दानवोत्तमान्}


% Check verse!
`कृकलास भूतं च नृगं मोचयिष्ये च वै पुनः
\twolineshloka
{तत्र पौत्रनिमित्तेन गत्वा वै शोणितं पुरम्}
{वाणस्य च पुरं गत्वा करिष्ये कदनं महत् ॥'}


\twolineshloka
{शंकरं रमहासेनं बाणप्रियहिते रतम्}
{पराजेष्याम्यथोद्युक्तौ देवौ लोकनमस्कृतौ}


\twolineshloka
{ततः सुतं बलेर्जित्वा बाणं बाहुसहस्त्रिणम्}
{विनाशयिष्यामि ततः सर्वान्सौभनिवासिनः}


\twolineshloka
{यः कालयवनः ख्यातो गर्गतेजोभिसंवृतः}
{भविष्यति वधस्तस्य मत्त एव द्विजोत्तम}


\twolineshloka
{`कंसं केशिं तथाक्रूरमरिष्टं च महासुरम्}
{चाणूरं च महावीर्यं मुष्टिकं च महाबलम्}


\twolineshloka
{प्रलम्बं धेनुकं चैव अरिष्टं वृषरूपिणम्}
{कालीयं च वशे कृत्वा यमुनाया महाह्रदे}


\twolineshloka
{गोकुलेषु ततः पश्चाद्भवार्थे तु महागिरिम्}
{सप्तरात्रं धरिष्यामि वर्षमाणे तु वासवे}


\threelineshloka
{अपक्रान्ते ततो वर्षे गिरिमूर्ध्निं व्यवस्थितः}
{इन्द्रेण सह संवादं करिष्यामि तदा द्विज}
{लघ्वाच्छिद्य धनं सर्वं वासुदेवं च पौण्ड्रकम् ॥'}


% Check verse!
जरासन्धश्च बलवान्सर्वराजविरोधनःभविष्यत्यसुरः स्फीतो भूमिपालो गिरिव्रजे
\twolineshloka
{मम बुद्धिपरिस्पन्दाद्वधस्तस्य भविष्यति}
{शिशुपालं वधिष्यामि यज्ञे धर्मसुतस्य वै}


\twolineshloka
{`दुर्योधनापराधेन युधिष्ठिरगुणेन च}
{'समागतेषु बलिषु पृथिव्यां सर्वराजसु}


\twolineshloka
{वासविः सुसहायो वै मम त्वेको भविष्यति}
{युधिष्ठिरं स्थापयिष्ये स्वराज्ये भ्रातृभिः सह}


\twolineshloka
{एवं लोका वदिष्यन्ति नरनारायणावृषी}
{उद्युक्तौ दहतः क्षत्रं लोककार्यार्थमीश्वरौ}


% Check verse!
`शस्त्रैर्निपतिताः सर्वे नृपा यास्यन्ति वै दिवम् ॥'
\twolineshloka
{कृत्वा भारावतरणं वसुधाया यथेप्सितम्}
{सर्वसात्वतमुख्यानां द्वारकायाश्च सत्तम}


\twolineshloka
{करिष्ये प्रलयं घोरमात्मज्ञानाभिसंश्रयः}
{`द्वारकामात्मसात्कृत्वा समुद्रं गमयाम्यहम्}


\twolineshloka
{ततः कलियुगस्यादौ द्विजराजतरुं श्रितः}
{भीषया मागधेनैव धर्मराजगृहे वसन्}


\twolineshloka
{काषायवस्रसंवीतो मुण्डितः शुक्लदन्तवान्}
{शुद्धोदनसुतो बुद्धो मोहयिष्यामि मानवान्}


\twolineshloka
{शूद्राः सुद्धेषु भुज्यन्ते मयि बुद्धत्वमागते}
{भविष्यन्ति नराः सर्वे बुद्धाः काषायसंवृताः}


\twolineshloka
{अनध्याया भविष्यन्ति विप्रा यागविवर्जिताः}
{अग्निहोत्राणि सीदन्ति गुरुपूजा च नश्यति}


\twolineshloka
{न शृण्वन्ति पितुः पुत्रा न स्नुषा नैव भ्रातरः}
{न पौत्रा न कलत्रा वा वर्तन्तेऽप्यधमोत्तमाः}


\twolineshloka
{एवंभूतं जगत्सर्वं श्रुतिस्मृतिविवर्जितम्}
{भविष्यति कलौ पूर्णे ह्यशुद्धो धर्मसंकरः}


\twolineshloka
{तेषां सकाशाद्धर्मज्ञा देवब्रह्मविदो नराः}
{भविष्यन्ति ह्यशुद्धाश्च न्यायच्छलविभाषिणः}


\threelineshloka
{ये नष्टधर्मश्रोतारस्ते समाः पापनिश्चये}
{तस्मादेता न संभाष्या न स्पृश्या च हितार्थिभिः}
{उपवासत्रयं कुर्यात्तत्संसर्गविशुद्धये}


\twolineshloka
{ततः कलियुगस्यान्ते ब्राह्मणो हरिपिङ्गलः}
{कल्किर्विष्णुयशः पुत्रो याज्ञवल्क्यः पुरोहितः}


\threelineshloka
{तस्मिन्नाशे वनग्रामे तिष्ठेत्सोन्नासिमो हयः}
{सहया ब्राह्मणाः सर्वे तैरहं सहितः पुनः}
{म्लेच्छानुत्सादयिष्यामि पाषण़्डांश्चैव सर्वशः}


\twolineshloka
{पाषण्डश्च कलौ तत्र माययैव विनश्यते}
{पाषण़्डकांश्चैव हत्वा तत्रान्तं प्रलये ह्यहम्}


\twolineshloka
{ततः पश्चाद्भविष्यामि यज्ञेषु निरतः सदा}
{राज्यं प्रशासति पुनः कुन्तीपुत्र युधिष्ठिरे ॥'}


\twolineshloka
{कर्माण्यपरिमेयानि चतुर्मूर्तिधरो ह्यहम्}
{कृत्वा लोकान्गमिष्यामि स्वानहं ब्रह्मसत्कृतान्}


\threelineshloka
{हंसः कूर्मश्च मत्स्यश्च प्रादुर्भावा द्विजोत्तम}
{वराहो नरसिंहश्च वामनो राम एव च}
{रामो दाशरथिश्चैव सात्वतः कल्किरेव च}


\twolineshloka
{यदा वेदश्रुतिर्नष्टा मया प्रत्याहृता पुनः}
{सर्वदाः सश्रुतीकाश्च कृताः पूर्वं कृते युगे}


\twolineshloka
{अतिक्रान्ताः पुराणेषु श्रुतास्ते यदि वा क्वचित्}
{अतिक्रान्ताश्च बहवः प्रादुर्भावा ममोत्तमाः}


\twolineshloka
{लोककार्याणि कृत्वा च पुनः स्वां प्रकृतिं गताः}
{न ह्येतद्ब्रह्मणा प्राप्तमीदृशं मम दर्शनम्}


\fourlineindentedshloka
{यत्त्वया प्राप्तमद्येह एकान्तगतबुद्धिना}
{एतत्ते सर्वमाख्यातं ब्रह्मन्भक्तिमतो मया}
{पुराणं च भविष्यं च सरहस्यं च सत्तम ॥भीष्म उवाच}
{}


\twolineshloka
{एवं स भगवान्देवो विश्वमूर्तिधरोऽव्ययः}
{एतावदुक्त्वा वचनं तत्रैवान्तर्दधे पुनः}


\twolineshloka
{नारदोऽपि महातेजाः प्राप्यानुग्रहमीप्सितम्}
{नरनारायणौ द्रष्टुं बदर्याश्रममाद्रवत्}


\twolineshloka
{इदं महोपनिषदं चतुर्वेदसमन्वितम्}
{सांख्ययोगकृतं तेन पञ्चरात्रानुशब्दितम्}


\threelineshloka
{नारायंणमुखोदीतं नारदोऽश्रावयत्पुनः}
{ब्रह्मणः सदने तात यथादृष्टं यथाश्रुतम् ॥युधिष्ठिर उवाच}
{}


\twolineshloka
{एतदाश्चर्यभूतं हि माहात्म्यं तस्य धीमतः}
{किं वै ब्रह्मा न जानीते यतः शुश्राव नारदात्}


\threelineshloka
{पितामहोऽपि भगवांस्तस्माद्देवादनन्तरः}
{कथं स न विजानीयात्प्रभावममितौजसः ॥भीष्म उवाच}
{}


\twolineshloka
{महाकल्पसहस्राणि महाकल्पशतानि च}
{समतीतानि राजेन्द्र सर्गाश्च प्रलयाश्च ह}


\threelineshloka
{सर्गस्यादौ स्मृतो ब्रह्मा प्रजासर्गकरः प्रभुः}
{जानाति देवप्रवरं भूयश्चातोधिकं नृप}
{परमात्मानमीशानमात्मनः प्रभवं तथा}


\twolineshloka
{ये त्वन्ये ब्रह्मसदने सिद्धसङ्घाः समागताः}
{तेभ्यस्तच्छ्रावयामास पुराणं वेदसंमितम्}


\threelineshloka
{अष्टाविंशत्सहस्राणि ऋषीणां भावितात्मनाम्}
{आत्मानुगामिनां ब्रह्मा श्रावयामास तत्वतः}
{एवं पुरा प्राप्तमिदं भानुना मुनिभाषितम्}


\threelineshloka
{वर्षषष्टिसहस्राणि षष्टिवर्षशतानि च}
{सूर्यस्य तपतो लोकान्निर्मिता ये पुरःसराः}
{तेषामकथयत्सूर्यः सर्वेषां भावितात्मनाम्}


\twolineshloka
{सूर्यानुगामिभिस्तात ऋषिभिस्तैर्महात्मभिः}
{मेरौ समागता देवाः श्राविताश्चेदनुत्तमम्}


\twolineshloka
{देवानां तु सकाशाद्वै ततः श्रुत्वाऽसितो द्विजः}
{श्रावयामास राजेन्द्र पितॄन्वै मुनिसत्तमः}


\twolineshloka
{मम चापि पिता तात कथयामास शंतनुः}
{ततो मयापि श्रुत्वा च कीर्तितं तव भारत}


\twolineshloka
{सुरैर्वा मुनिभिर्वापि पुराणं यैरिदं श्रुतम्}
{सर्वे ते परमात्मानं पूजयन्ते समन्ततः}


\twolineshloka
{इदमाख्यानमार्षेयं पारम्पर्यागतं नृप}
{नावासुदेवभक्ताय त्वया देयं कथंचन}


\threelineshloka
{`आख्यानमुत्तमं चेदं श्रावयेद्यः सदा नृप}
{तदैव मनुजो भक्तः शुचिर्भूत्वा समाहितः}
{प्राप्नुयादचिराद्राजन्विष्णुलोकं च शाश्वतम् ॥'}


\twolineshloka
{मत्तोन्यानि च ते राजन्नुपाख्यानशतानि वै}
{यानि श्रुतानि सर्वाणि तेषां सारोयमुद्धृतः}


\twolineshloka
{सुरासुरैर्यथा राजन्निर्मथ्यामृतमुद्धृतम्}
{एवमेतत्पुरा विप्रैः कथामृतमिहोद्धृतम्}


\twolineshloka
{यश्चेदं पठते नित्यं यश्चेदं शृणुयान्नरः}
{एकान्तभावोपगत एकान्ते सुसमाहितः}


\twolineshloka
{प्राप्य श्वेतं महाद्वीपं भूत्वा चन्द्रप्रभो नरः}
{स सहस्रार्चिपं देवं प्रविशेन्नात्र संशयः}


\twolineshloka
{मुच्येदार्तस्तथा रोगाच्छ्रुत्वेमामादितः कथाम्}
{जिज्ञासुर्लभते कामान्भक्तो भक्तगतिं व्रजेत्}


\twolineshloka
{त्वयापि सततं राजत्रभ्यर्च्यः पुरुषोत्तमः}
{स हि माता पिता चैव कृत्स्नस्य जगतो गुरुः}


\threelineshloka
{ब्रह्मण्यदेवो भगवान्प्रीयतां ते सनातनः}
{युधिष्ठिर महाबाहो महाबुद्धिर्जनार्दनः ॥वैशंपायन उवाच}
{}


\twolineshloka
{श्रुत्वैतदाख्यानवरं धर्मराड््जनमेजय}
{भ्रातरश्चास्य ते सर्वे नारायणपराभवन्}


\twolineshloka
{जितं भगवता तेन पुरुषेणेति भारत}
{नित्यं जप्यपरा भूत्वा सारस्वतमुदीरयन्}


\twolineshloka
{यो ह्यस्माकं गुरुः श्रेष्ठः कृष्णद्वैपायनो मुनिः}
{जगौ परमकं जप्यं नारायणमुदीरयन्}


\threelineshloka
{गत्वान्तरिक्षात्सततं क्षीरोदममृताशयम्}
{पूजयित्वा च देवेशं पुनरायात्स्वगाश्रमम् ॥भीष्म उवाच}
{}


\threelineshloka
{एतत्ते सर्वमाख्यातं नारदोक्तं मयेरितम्}
{पारम्पर्यागतं ह्येतत्पित्रा मे कथितं पुरा ॥सौतिरुवाच}
{}


\twolineshloka
{एतत्ते सर्वमाख्यातं वैशंपायनकीर्तितम्}
{जनमेजयेन तच्छ्रुत्वा कृतं सम्यग्यथाविधि}


\threelineshloka
{यूयं हि तप्ततपसः सर्वे च चरितव्रताः}
{शौनकस्य महासत्रं प्राप्ताः सर्वे द्विजोत्तमाः}
{}


\twolineshloka
{यजध्वं सुहुतैर्यज्ञैः शाश्वतं परमेश्वरम्}
{पारम्पर्यागतं ह्येतत्पित्रा मे कथितं पुरा}


\chapter{अध्यायः ३४९}
\twolineshloka
{शौकन उवाच}
{}


\twolineshloka
{कथं स भगवान्देवो यज्ञेष्वग्रहरः प्रभुः}
{यज्ञधारी च सततं वेदवेदाङ्गवित्तथा}


\twolineshloka
{निवृत्तं चास्थितो धर्मं क्षेमी भागवतः प्रभुः}
{निवृत्तिधर्मान्विदधे स एव भगवान्प्रभुः}


\twolineshloka
{कथं प्रवृत्तिधर्मेषु भागार्हा देवताः कृताः}
{कथं निवृत्तिधर्माश्च कृता व्यावृत्तबुद्धयः}


\threelineshloka
{एतं नः संशयं सौते छिन्धि गुह्यं सनातनम्}
{त्वया नारायणकथाः श्रुता वै धर्मसंहिताः ॥सौतिरुवाच}
{}


\twolineshloka
{जनमेजयेन यत्पृष्टः शिष्यो व्यासस्य धीमतः}
{तत्तेऽहं कथयिष्यामि पौराणं शौनकोत्तम}


\twolineshloka
{श्रुत्वा माहात्म्यमेतस्य देहिनां परमात्मनः}
{जनमेजयो महाप्राज्ञो वैशंपायनमब्रवीत्}


\twolineshloka
{इमे सब्रह्यका लोकाः ससुरासुरमानवाः}
{क्रियास्वभ्युदयोक्तासु सक्ता दृश्यन्ति सर्वशः}


\threelineshloka
{मोक्षश्चोक्तस्त्वया ब्रह्मन्निर्वाणं परमं सुखम्}
{ये तु मुक्ता भवन्तीह पुण्यपापविवर्जिताः}
{ते सहस्रार्चिषं देवं प्रविशन्तीह शुश्रुम्}


\twolineshloka
{अयं हि दुरनुष्ठेयो मोक्षधर्मः सनातनः}
{यं हित्वा देवताः सर्वा हव्यकव्यभुजोऽभवन्}


\twolineshloka
{किंच ब्रह्मा च रुद्रश्च बलभित्प्रभुः}
{सूर्यस्ताराधिपो वायुरग्निर्वरुण एव च}


\twolineshloka
{आकाशं जगती चैव ये च शेषा दिवौकसः}
{प्रलयं न विजानन्ति आत्मनः परिनिर्मितम्}


\threelineshloka
{ततस्तेनास्थिता मार्गं ध्रुवमक्षरमव्ययम्}
{स्मृत्वा कालपरीमाणं प्रवृत्तिं ये समास्थिताः}
{दोषः कालपरीमाणो महानेष क्रियावताम्}


\twolineshloka
{एतन्मे संशयं विप्र हृदि शल्यमिवार्पितम्}
{छिन्धीतिहासकथनात्परं कौतूहलं हि मे}


\twolineshloka
{कथं भागहराः प्रोक्ता देवताः क्रतुषु द्विज}
{किमर्थं चाध्वरे ब्रह्मन्निज्यन्ते त्रिदिवौकसः}


\threelineshloka
{ये च भागं प्रगृह्णन्ति यज्ञेषु द्विजसत्तम}
{ते यजन्तो महायज्ञैः कस्य भागं ददन्ति वै ॥वैशंपायन उवाच}
{}


\threelineshloka
{अहो गूढतमः प्रश्नस्त्वया पृष्टो जनेश्वर}
{नातप्ततपसा ह्येष नावेदविदुषा तथा}
{नापुराणविदा चैव शक्यो व्याहर्तुमञ्जसा}


\twolineshloka
{हन्त ते कथयिष्यामि यन्मे पृष्टः पुरा गुरुः}
{कृष्णद्वैपायनो व्यासो वेदव्यासो महानृषिः}


\twolineshloka
{सुमन्तुर्जैमिनिश्चैव पैलश्च सुदृढव्रतः}
{अहं चतुर्थः शिष्यो वै पञ्चमश्च शुकः स्मृतः}


\twolineshloka
{एतान्समागतान्सर्वान्पञ्च शिष्यान्दमान्वितान्}
{शौचाचारसमायुक्ताञ्जितक्रोधाञ्जितेन्द्रियान्}


\twolineshloka
{वेदानध्यापयामास महाभारतपञ्चमान्}
{मेरौ गिरिवरे रम्ये सिद्धचारणसेविते}


\twolineshloka
{तेषामभ्यस्यतां वेदान्कदाचित्संशयोऽभवत्}
{एष वै यस्त्वया पृष्टस्तेन तेषां प्रकीर्तितः}


% Check verse!
ततः श्रुतो मया चापि तवाख्येयोऽद्य भारत
\twolineshloka
{शिष्याणां वचनं श्रुत्वा सर्वाज्ञानतमोनुदः}
{पराशरसुतः श्रीमान्व्यासो वाक्यमथाब्रवीत्}


\twolineshloka
{मया हि सुमहत्तप्तं तपः परमदारुणम्}
{भूतं भव्यं भविष्यं च जानीयामिति सत्तमाः}


\twolineshloka
{तस्य मे तप्ततपसो निगृहीतेन्द्रियस्य च}
{नारायणप्रसादेन क्षीरोदस्यानुकूलतः}


\twolineshloka
{त्रैकालिकमिदं ज्ञानं प्रादुर्भूतं यथेप्सितम्}
{तच्छृणुध्वं यथान्यायं वक्ष्ये संशयमुत्तमम्}


\twolineshloka
{यथा वृत्तं हि कल्पादौ दृष्टं मे ज्ञानचक्षुषा}
{परमात्मेति यं प्राहुः साङ्ख्ययोगविदो जनाः}


\twolineshloka
{महापुरुषसंज्ञां स लभते स्वेन कर्मणा}
{तस्मात्प्रसूतमव्यक्तं प्रधानं तं विदुर्बुधाः}


\twolineshloka
{अव्यक्ताद्व्यक्तमुत्पन्नं लोकसृष्ट्यर्थमीश्वरात्}
{अनिरुद्धो हि लोकेषु महानात्मेति कथ्यते}


\twolineshloka
{योसौ व्यक्तत्वमापन्नो निर्ममे च पितामहम्}
{योऽहंकार इति प्रोक्तः सर्वतेजोमयो हि सः}


\twolineshloka
{पृथिवी वायुराकाशमापो ज्योतिश्च पञ्चमम्}
{अहंकारप्रसूतानि महाभूतानि पञ्चधा}


\twolineshloka
{महाभूतानि सृष्ट्वैव तान्गुणान्निर्ममे पुनः}
{भूतेभ्यश्चैव निष्पन्ना मूर्तिमन्तश्च ताञ्शृणु}


\threelineshloka
{मरीचिरङ्गिराश्चात्रिः पुलस्त्यः पुलहः क्रतुः}
{वसिष्ठश्च महात्मा वै मनुः स्वायंभुवस्तथा}
{ज्ञेयाः प्रकृतयोऽष्टौ ता यासु लोकाः प्रतिष्ठिताः}


\twolineshloka
{वेदान्वेदाङ्गसंयुक्तान्यज्ञयज्ञाङ्गसंयुतान्}
{निर्ममे लोकसिद्ध्यर्थं ब्रह्मा लोकपितामहः}


% Check verse!
अष्टाभ्यः प्रकृतिभ्यश्च जातं विश्वमिदं जगत्
\twolineshloka
{रुद्रो रोषात्मको जातो दशान्यान्सोसृजस्त्वयम्}
{एकादशैते रुद्रास्तु विकाराः पुरुषाः स्मृताः}


\twolineshloka
{ते रुद्राः प्रकृरतिश्चैव सर्वे चैव सुरर्षयः}
{उत्पन्ना लोकसिद्ध्यर्थं ब्रह्माणं समुपस्थिताः}


\twolineshloka
{वयं सृष्टा हि भगवंस्त्वया च प्रभविष्णुना}
{येन यस्मिन्नधीकारे वर्तितव्यं पितामह}


\twolineshloka
{योसौ त्वयाऽभिनिर्दिष्टो ह्यधिकारोऽर्थचिन्तकः}
{परिपाल्यः कथं तेन साहंकारेण कर्तृणा}


\threelineshloka
{प्रदिशस्व बलं तस्य योऽधिकारार्थचिन्तकः}
{एवमुक्तो महादेवो देवांस्तानिदमब्रवीत् ॥ब्रह्मोवाच}
{}


\twolineshloka
{साध्वहं ज्ञापितो देवा युष्माभिर्भद्रमस्तु वः}
{ममाप्येषा समुत्पन्ना चिन्ता या भवतामिह}


\twolineshloka
{लोकतन्त्रस्य कृत्स्नस्य कथं कार्यः परिग्रहः}
{कथं बलक्षयो न स्याद्युष्माकं ह्यात्मनश्च वै}


\twolineshloka
{इतः सर्वेऽपि गच्छामः शरणं लोकसाक्षिणम्}
{महापुरुषमव्यक्तं स नो वक्ष्यति यद्धितम्}


\twolineshloka
{ततस्ते ब्रह्मणा सार्धमृषयो विबुधास्तथा}
{क्षीरादस्योत्तरं कूलं जग्मूर्लोकहितार्थिनः}


\twolineshloka
{ते तपः समुपातिष्ठन्ब्रह्मोक्तं वेदकल्पितम्}
{स महानियमो नाम तपश्चर्या सुदारुणा}


\twolineshloka
{ऊर्ध्वदृग्बाहवश्चैव एकाग्रमनसोऽभवन्}
{एकपादस्थिताः सर्वे काष्ठभूताः समाहिताः}


\threelineshloka
{दिव्यं वर्षसहस्रं ते तपस्तप्त्वा सुदारुणम्}
{शुश्रुवुर्मधुरां वाणीं वेदवेदाङ्गभूषिताम् ॥वागुवाच}
{}


\twolineshloka
{भोभोः सब्रह्यका देवा ऋषयश्च तपोधनाः}
{स्वागतेनार्च्य वः सर्वाञ्श्रावये वाक्यमुत्तमम्}


\twolineshloka
{विज्ञातं वो मया कार्यं तच्च लोकहितं महत्}
{प्रवृत्तियुक्तं कर्तव्यं युष्मत्प्राणोपबृंहणम्}


\twolineshloka
{सुतप्तं वस्तपो देवा ममाराधनकाम्यया}
{भोक्ष्यथास्य महासत्वास्तपसः फलमुत्तमम्}


\twolineshloka
{एष ब्रह्मा लोकगुरुः सर्वलोकपितामहः}
{यूयं च विबुधश्रेष्ठा मां यजध्वं समाहिताः}


\threelineshloka
{सर्वे भागान्कल्पयध्वं यज्ञेषु मम नित्यशः}
{तत्र श्रेयोऽभिधास्यामि यथाऽधीकारमीश्वराः ॥वैशंपायन उवाच}
{}


\twolineshloka
{श्रुत्वैतद्देवदेवस्य वाक्यं हृष्टतनूरुहाः}
{ततस्ते विबुधाः सर्वे ब्रह्मा ते च महर्षयः}


\twolineshloka
{वेददृष्टेन विधिना वैष्णवं क्रतुमाहरन्}
{तस्मिन्सत्रे सदा ब्रह्मा स्वयं भागमकल्पयत्}


\twolineshloka
{देवा देवर्षयश्चैव स्वंस्वं भागमकल्पयम्}
{ते कार्तयुगधर्माणो भागाः परमसत्कृताः}


\twolineshloka
{प्राहुरादित्यवर्णं तं पुरुषं तमसः परम्}
{बृहन्तं सर्वगं देवमीशानं वरदं प्रभुम्}


\twolineshloka
{ततोऽथ वरदौ देवस्तान्सर्वानमरान्स्थितान्}
{अशरीरो बभापेदं वाक्यं स्वस्थो महेश्वरः}


\threelineshloka
{येन यः कल्पितो भागः स तथा मामुपागतः}
{प्रीतोऽहं प्रदिशाम्यद्य फलमावृत्तिलक्षणम्}
{एतद्वो लक्षणं देवा मत्प्रसादसमुद्भवम्}


\twolineshloka
{यूयं यज्ञैरिज्यमानाः समाप्तवरदक्षिणैः}
{युगेयुगे भविष्यध्वं प्रवृत्तिफलभागिनः}


\twolineshloka
{यज्ञैर्ये चापि यक्ष्यन्ति सर्वलोकेषु वै सुराः}
{कल्पयिष्यन्ति वो भागांस्ते नरा वेदकल्पितान्}


\twolineshloka
{यो मे यथा कल्पितवान्भागमस्मिन्महाक्रतौ}
{स तथा यज्ञभागार्हो वेदसूत्रे मया कृतः}


\twolineshloka
{यूयं लोकान्भावयध्वं यज्ञभागफलोचिताः}
{सर्वार्थचिन्तका लोके मयाऽधीकारनिर्मिताः}


\twolineshloka
{याः क्रियाः प्रचरिष्यन्ति प्रवृत्तिफलसत्कृताः}
{ताभिराप्यायितबला लोकान्वै धारयिष्यथ}


\twolineshloka
{यूयं हि भाविता यज्ञैः सर्वयज्ञेषु मानवैः}
{मां ततो भावयिष्यध्वमेषा वो भावना मम}


\twolineshloka
{इत्यर्थं निर्मिता वेदा यज्ञाश्चौषधिभिः सह}
{एभिः सम्यक्प्रयुक्तैर्हि प्रीयन्ते देवताः क्षितौ}


\threelineshloka
{निर्माणमेतद्युष्माकं प्रवृत्तिगुणकल्पितम्}
{मया कृतं सुरश्रेष्ठा यवात्कल्पक्षयादिह}
{चिन्तयध्वं लोकहितं यथादीकारमीश्वराः}


\twolineshloka
{मरीचिरङ्गिराश्चात्रिः पुलस्त्यः पुलहः क्रतुः}
{वसिष्ठ इति सप्तैते मनसा निर्मिता हि ते}


\twolineshloka
{एते वेदविदो मुख्या वेदाचार्याश्च कल्पिताः}
{प्रवृत्तिधर्मिणश्चैव प्राजापत्ये च कल्पिताः}


\twolineshloka
{अयं क्रियावतां पन्था व्यक्तीभूतः सनातनः}
{अनिरुद्ध इति प्रोक्तो लोकसर्गकरः प्रभुः}


\twolineshloka
{सनः सनत्सुजातश्च सनकः समनन्दनः}
{सनत्कुमारः कपिलः सप्तमश्च सनातनः}


\twolineshloka
{सप्तैते मानसाः प्रोक्ता ऋषयो ब्रह्मणः सुताः}
{स्वयमागतविज्ञाना निवृत्तिं धर्ममास्थिताः}


\twolineshloka
{एते योगविदो मुख्याः साङ्ख्यशास्त्रविशारदाः}
{आचार्या धर्मशास्त्रेषु मोक्षधर्मप्रवर्तकाः}


\twolineshloka
{यतोऽहं प्रसृतः पूर्वमव्यक्तात्रिगुणो महान्}
{तस्मात्परतरो योसौ क्षेत्रज्ञ इति कल्पितः}


\twolineshloka
{सोहं क्रियावतां पन्थाः पुनरावृत्तिदुर्लभः}
{यो यथा निर्मितो जन्तुर्यस्मिन्यस्मिंश्च कर्मणि}


\twolineshloka
{प्रवृत्तौ वा निवृत्तौ वा तत्फलं सोश्नुतेऽवशः}
{एष लोकगुरुर्ब्रह्मा जगदादिकरः प्रभुः}


\twolineshloka
{एष माता पिता चैव युष्माकं च पितामहः}
{मयाऽनुशिष्टो भविता सर्वभूतवरप्रदः}


\twolineshloka
{अस्य चैवात्मजो रुद्रो ललाटाद्यः समुत्थितः}
{ब्रह्मानुशिष्टो भविता सर्वभूतधरः प्रभुः}


\twolineshloka
{गच्छध्वं स्वानधीकारांश्चिन्तयध्वं यथाविधि}
{प्रवर्तन्तां क्रियाः सर्वाः सर्वलोकेषु माचिरम्}


\twolineshloka
{प्रदिश्यन्तां च कर्माणि प्राणिनां गतयस्तथा}
{परिनिष्ठितकालानि आयूंषीह सुरोत्तमाः}


\twolineshloka
{इदं कृतयुगं नाम कालः श्रेष्ठः प्रवर्तितः}
{अहिंस्या यज्ञपशवो युगेऽस्मिन्न तदन्यथा}


\twolineshloka
{चतुष्पात्सकलो धर्मो भविष्यत्यत्र वै सुराः}
{ततस्त्रेतायुगं नाम त्रयी यत्र भविष्यति}


\twolineshloka
{प्रोक्षिता यत्र पशवो वधं प्राप्स्यन्ति वै मखे}
{यत्र पादश्चतुर्थो वै धर्मस्य न भविष्यति}


\twolineshloka
{ततो वै द्वापरं नाम मिश्रः कालो भविष्यति}
{द्विपादहीनो धर्मश्च युगे तस्मिन्भविष्यति}


\threelineshloka
{ततस्तिष्येऽथ संप्राप्ते युगे कलिपुरस्कृते}
{एकपादस्थितो धर्मो यत्र तत्र भविष्यति ॥देवा ऊचुः}
{}


\fourlineindentedshloka
{देवा देवर्षयश्चोचुस्तमेवंवादिनं गुरुम्}
{एकपादस्थिते धर्मे यत्र क्वचन गामिनि}
{कथं कर्तव्यमस्माभिर्भगवंस्तद्वदस्व नः ॥श्रीभगवानुवाच}
{}


\twolineshloka
{`गुरवो यत्र पूज्यन्ते साधुवृत्तसमन्विताः}
{वस्तव्यं तत्र युष्माभिर्यत्र धर्मो न हीयते ॥'}


\fourlineindentedshloka
{यत्र वेदाश्च यज्ञाश्च तपः सत्यं दमस्तथा}
{अहिंसा धर्मसंयुक्ताः प्रचरेयुः सुरोत्तमाः}
{स वो देशः सेवितव्यो मा वोऽधर्मः पदा स्पृशेत् ॥व्यास उवाच}
{}


\twolineshloka
{तेऽनुशिष्टा भगवता देवाः सपिगणास्तथा}
{नमस्कृत्वा भगवते जग्मुर्देशान्यथेप्सितान्}


\twolineshloka
{गतेषु त्रिदिवौकस्सु ब्रह्मैकः पर्यवस्थितः}
{दिदृक्षुर्भगवन्तं तमनिरुद्धतनौ स्थितम्}


\twolineshloka
{तं देवो दर्शयामास कृत्वा हयशिरो महत्}
{साङ्गानावर्तयन्वेदान्कमण्डलुत्रिदण्डधृक्}


\twolineshloka
{ततोऽश्वशिरसं दृष्ट्वातं देवममितौजसम्}
{लोककर्ता प्रभुर्ब्रह्मा लोकानां हितकाम्यया}


\threelineshloka
{मूर्ध्ना प्रणम्य वरदं तस्थौ प्राञ्जलिरग्रतः}
{स परिष्वज्य देवेन वचनं श्रावितस्तदा ॥भगवानुवाच}
{}


\threelineshloka
{लोककार्यगतीः सर्वास्त्वं चिन्तय यथाविधि}
{धाता त्वं सर्वभूतानां त्वं प्रभुर्जगतो गुरुः}
{त्वय्यावेशितभारोऽहं धृतिं प्राप्स्याम्यथाञ्जसा}


\threelineshloka
{यदा च सुरकार्यं ते अविषह्यं भविष्यति}
{प्रादुर्भावं गमिष्यामि तदात्मज्ञानदैशिकः ॥व्यास उवाच}
{}


\twolineshloka
{एवमुक्त्वा हयशिरास्तत्रैवान्तरधीयत}
{तेनानुशिष्टो ब्रह्मापि स्वं लोकमचिराद्गतः}


\twolineshloka
{एवमेष महाभागः पद्मनाभः सनातनः}
{यज्ञेष्वग्रहरः प्रोक्तो यज्ञधारी च नित्यदा}


\twolineshloka
{निवृत्तिं चास्थितो धर्मं गमिमक्षयधर्मिणाम्}
{प्रवृत्तिधर्मान्विदधे कृत्वा लोकस्य चित्रताम्}


\twolineshloka
{स आदिः स मध्यः स चान्तः प्रजानांस धाता स धेयं स कर्ता स कार्यम्}
{युगान्ते प्रसुप्तः सुसंक्षिप्य लोकान्युगादौ प्रबुद्धो जगद्ध्युत्ससर्ज}


\twolineshloka
{तस्मै नमध्वं देवाय निर्गुणाय महात्मने}
{अजाय विश्वरूपाय धाम्ने सर्वदिवौकसाम्}


\twolineshloka
{महाभूताधिपतये रुद्राणां पतये तथा}
{आदित्यपतये चैव वसूनां पतये तथा}


\twolineshloka
{अश्विभ्यां पतये चैव मरुतां पतये तथा}
{वेदयज्ञाधिपतये वेदाङ्गपतयेऽपि च}


\twolineshloka
{समुद्रावसिने नित्यं हरये मुञ्जकेशिने}
{शान्ताय सर्वभूतानां मोक्षधर्मानुभाषिणे}


\twolineshloka
{तपसां तेजसां चैव पतये यशसामपि}
{वचसां पतये नित्यं सरितां पतये तथा}


\threelineshloka
{कपर्दिने वराहाय एकशृङ्गाय धीमते}
{विवस्वतेऽश्वशिरसे चतुर्मूर्तिधृते सदा}
{सूक्ष्माय ज्ञानदृश्याय अजरायाक्षयाय च}


\twolineshloka
{एष देवः संचरति सर्वत्र गतिरव्ययः}
{[एष चैतत्परं ब्रह्म ज्ञेयो विज्ञानचक्षुषा ॥]}


\twolineshloka
{एवमेतत्पुरा दृष्टं मया वै ज्ञानचक्षुषा}
{कथितं तच्च वै सर्वं मया पृष्टेन तत्त्वतः}


\threelineshloka
{क्रियतां मद्वचः शिष्याः सेव्यतां हरिरीश्वरः}
{गीयतां वेदशब्दैश्च पूज्यतां च यथाविधि ॥वैशंपायन उवाच}
{}


\twolineshloka
{इत्युक्तास्तु वयं तेन वेदव्यासेन धीमता}
{सर्वे शिष्या सुतश्चास्य शुकः परमधर्मवित्}


\twolineshloka
{स चास्माकमुपाध्यायः सहास्माभिर्विशांपते}
{चतुर्वेदोद्गताभिस्तमृग्भिः समभितुष्टुवे}


\twolineshloka
{एतत्ते सर्वमाख्यातं यन्मां त्वं परिपृच्छसि}
{एवं मेऽकथयद्राजन्पुरा द्वैपायनो गुरुः}


\twolineshloka
{यश्चेदं शृणुयान्नित्यं यश्चैनं परिकीर्तयेत्}
{नमो भगवते कृत्वा समाहितमतिर्नरः}


\twolineshloka
{भवत्यरोगो मतिमान्बलरूपसमन्वितः}
{आतुरो मुच्यते रोगाद्बद्धो मुच्येत बन्धनात्}


\twolineshloka
{कामकामी लभेत्कामं दीर्घं चायुरवाप्नुयात्}
{ब्राह्मणः सर्ववेदी स्यात्क्षत्रियो विजयी भवेत्}


\twolineshloka
{वैश्यो विपुललाभः स्याच्छूद्रः सुखमवाप्नुयात्}
{अपुत्रो लभते पुत्रं कन्या चैवेप्सितं पतिम्}


\twolineshloka
{लग्नगर्भा विमुच्येत गर्भिणी जनयेत्सुतम्}
{बन्ध्या प्रसवमाप्नोति पुत्रपौत्रसमृद्धिमत्}


\twolineshloka
{क्षेमेण गच्छेदध्वानमिदं यः पठते पथि}
{यो यं कामं कामयते स तमाप्नोति च ध्रुवम्}


\twolineshloka
{इदं महर्षेर्वचनं विनिश्चितंमहात्मनः पुरुषवरस्य कीर्तितम्}
{समागमं चर्षिदिवौकसामिमंनिशम्य भक्ताः सुसुखं लभन्ते}


\chapter{अध्यायः ३५०}
\twolineshloka
{जनमेजय उवाच}
{}


\twolineshloka
{अस्तौषीद्वैदिकैर्व्यासः सशिष्यो मधुसूदनम्}
{नामभिर्विविधैरेषां निरुक्तं भगवन्मम}


\threelineshloka
{वक्तुमर्हसि शुश्रूषोः प्रजापतिपतेर्हरः}
{श्रुत्वा भवेयं यत्पूतः शरच्चन्द्र इवामलः ॥वैशंपायन उवाच}
{}


\twolineshloka
{शृणु राजन्यथाचष्ट फल्गुनस्य हरिः प्रभुः}
{प्रसन्नात्मात्मनो नाम्नां निरुक्तं गुणकर्मजम्}


\threelineshloka
{नामभिः कीर्तितैस्तस्य केशवस्य महात्मनः}
{पृष्टवान्केशवं राजन्फगुनः परवीरहा ॥अर्जुन उवाच}
{}


\twolineshloka
{भगवन्भूतभव्येश सर्वभूतसृगव्यय}
{लोकधाम जगन्नाथ लोकानामभयप्रद}


\twolineshloka
{यानि नामानि ते देव कीर्तितानि महर्षिभिः}
{वेदेषु सपुराणेषु यानि गुह्यानि कर्मभिः}


\threelineshloka
{तेषां निरुक्तं त्वत्तोऽहं श्रोतुमिच्छामि केशव}
{न ह्यन्यो वर्णयेन्नाम्नां निरुक्तं त्वामृते प्रभो ॥श्रीभगवानुवाच}
{}


\twolineshloka
{ऋग्वेदे सयजुर्वेदे तथैवाथर्वसामसु}
{पुराणे सोपनिषदे तथैव ज्योतिषेऽर्जुन}


\twolineshloka
{साङ्ख्ये च योगशास्त्रे च आयुर्वेदे तथैव च}
{बहूनि मम नामानि कीर्तितानि महर्षिभिः}


\twolineshloka
{गौणानि तत्र नामानि कर्मजानि च कानिचित्}
{निरुक्तं कर्मजानां त्वं शृणुष्व प्रयतोऽनघ}


\twolineshloka
{कथ्यमानं मया तात त्वं हि मेऽर्धं स्मृतः पुरा}
{नमोऽतियशमे तस्मै देवानां परमात्मने}


\twolineshloka
{नारायणाय विश्वाय निर्गुणाय गुणात्मने}
{यस्य प्रसादजो ब्रह्मा रुद्रस्य क्रोधसंभवः}


\twolineshloka
{योसौ योनिर्हि सर्वस्य स्थावरस्य चरस्य च}
{अष्टादशगुणं यत्तत्सत्वं सत्ववतांवर}


\twolineshloka
{प्रकृतिः सा परा मह्यं रोदसी लोकधारिणी}
{ऋता सत्याऽमरा जय्या लोकानामात्मसंज्ञिता}


\twolineshloka
{तस्मात्सर्वाः प्रवर्तन्ते सर्गप्रलयविक्रियाः}
{तपो यज्ञश्च यष्टा च पुराणः पुरुषो विराट्}


\twolineshloka
{अनिरुद्ध इति प्रोक्तो लोकानां प्रभवाप्ययः}
{ब्राह्मे रात्रिक्षये प्राप्ते तस्य ह्यमिततेजसः}


\twolineshloka
{प्रसादात्प्रादुरभवत्पद्ममर्कनिभं क्षणात्}
{तत्र ब्रह्मा समभवत्स तस्यैव प्रसादजः}


\twolineshloka
{अह्नः क्षये ललाटाच्च सुतो देवस्य वै तथा}
{क्रोधाविष्टस्य संजज्ञे रुद्रः संहारकारकः}


\twolineshloka
{एतौ द्वौ विबुधश्रेष्ठौ प्रसादक्रोधजावुभौ}
{तदादर्शितपन्थानौ सृष्टिसंहारकारकौ}


\twolineshloka
{निमित्तमात्रं तावत्र सर्वप्राणिवरप्रदौ}
{कपदीं जटिलो मुण्डः श्मशानगृहसेवकः}


\twolineshloka
{उग्रव्रतचरो रुद्रो योगी त्रिपुरदारणः}
{दक्षक्रतुहरश्चैव भगनेत्रहरस्तथा}


\twolineshloka
{नारायणात्मको ज्ञेयः पाण्डवेय युगेयुगे}
{तस्मिन्हि पूज्यमाने वै देवदेवे महेश्वरे}


\twolineshloka
{संपूजितो भवत्पार्थ देवो नारायणः प्रभुः}
{अहमात्मा हि लोकानां विश्वेषां पाण्डुनन्दन}


\twolineshloka
{तस्मादात्मानमेवाग्रे रुद्रं संपूजयाम्यहम्}
{यद्यहं नार्चयेयं वै ईशानं वरदं शिवम्}


\twolineshloka
{आत्मानं नार्चयेत्कश्चिदिति मे भावितात्मनः}
{मया प्रमाणं हि कृतं लोकः समनुवर्तते}


\twolineshloka
{प्रमाणानि हि पूज्यानि ततस्तं पूजयाम्यहम्}
{यस्तं वेत्ति स मां वेत्ति योऽनु तं स हि मामनु}


\twolineshloka
{रुद्रो नारायणश्चैव सत्वमेकं द्विधा कृतम्}
{लोके चरति कौन्तेय व्यक्तिस्थं सर्वकर्मसु}


\twolineshloka
{न हि मे केनचिद्देयो वरः पाण्डवनन्दन}
{इति संचिन्त्य मनसा पुराणं रुद्रमीश्वरम्}


\twolineshloka
{पुत्रार्थमाराधितवानहमात्मानमात्मना}
{न हि विष्णुः प्रणमति कस्मैचिद्विबुधाय च}


\twolineshloka
{ऋते आत्मानमेवेति ततो रुद्रं नमाम्यहम्}
{सब्रह्मकाः सरुद्राश्च सेन्द्रा देवाः सहर्षिभिः}


\twolineshloka
{अर्चयन्ति सुरश्रेष्ठं देवं नारायणं हरिम्}
{भविष्यतां वर्ततां च भूतानां चैव भारत}


\threelineshloka
{सर्वेषामग्रणीर्विष्णुः सेव्यः पूज्यश्च नित्यशः}
{नमस्व हव्यदं विष्णुं तथा शरणदं नमः}
{}


\twolineshloka
{वरदं नमस्व कौन्तेय हव्यकव्यभुजं नमः}
{चतुर्विधा मम जना भक्ता एव हि मे श्रुतम्}


\twolineshloka
{तेषामेकान्तिनः श्रेष्ठा ये चैवानन्यदेवताः}
{अहमेव गतिस्तेषां निराशीः कर्मकारिणाम्}


\twolineshloka
{ये च शिष्टास्त्रयो भक्ताः फलकामा हि ते मताः}
{सर्वे च्यवनधर्माणः प्रतिबुद्धस्तु श्रेष्ठभाक्}


\twolineshloka
{ब्रह्माणं शितिकण्ठं च याश्चान्या देवताः स्मृताः}
{प्रबुद्धचर्याः सेवन्तो मामेवैष्यन्ति यत्फलम्}


\twolineshloka
{भक्तं प्रति विशेषस्ते एष पार्थानुकीर्तितः}
{त्वं चैवाहं च कौन्तेय नरनारायणौ स्मृतौ}


\twolineshloka
{भारावतरणार्थं तु प्रविष्टौ मानुषीं तनुम्}
{नानीभ्यध्यात्मयोगांश्च योऽहं यस्माच्च भारत}


\twolineshloka
{निवृत्तिलक्षणो धर्मस्तथाऽऽभ्यदयिकोऽपि च}
{नराणामयनं ख्यातमहमेकः सनातनः}


\twolineshloka
{आपो नारा इति प्रोक्ता आपो वै नरसूनवः}
{अयनं मम ताः पूर्वमतो नारायणोस्म्यहम्}


\twolineshloka
{छादयामि जगद्विश्वं भूत्वा सूर्य इवांशुभिः}
{सर्वभूताधिवासश्च वासुदेवस्ततो ह्यहम्}


\threelineshloka
{गतिश्च सर्वभूतानां प्रजनश्चापि भारत}
{व्याप्ते म रोदसी पार्थ कान्तिश्चाभ्यधिका मम}
{}


\twolineshloka
{अधिभूतनिविष्टश्च तद्विश्वं चास्मि भारत}
{क्रमणाच्चाप्यहं पार्थ विष्णुरित्यभिसंज्ञितः}


\twolineshloka
{दमात्सिद्धिं परीप्सन्तो मां जनाः कामयन्ति ह}
{दिवं चोर्वी च मध्यं च तस्माद्दामोदरो ह्यहम्}


\twolineshloka
{पृश्निरित्युच्यते चान्नं वेदा आपोऽमृतं तथा}
{ममैतानि सदा गर्भः पृश्निगर्भस्ततो ह्यहम्}


\twolineshloka
{ऋषयः प्राहुरेवं मां त्रितं कूपनिपातितम्}
{पृश्निगर्भ त्रितं पाहीत्येकतद्वितपातितम्}


\twolineshloka
{ततः स ब्रह्मणः पुत्र आद्यो ह्यृषिवरस्त्रितः}
{उत्ततारोदपानाद्वै पृश्निगर्भानुकीर्तनात्}


\twolineshloka
{सूर्यस्य तपतो लोकानग्नेः सोमस्य चाप्युत}
{अंशवो यत्प्रकाशन्ते ममैते केशसंज्ञिताः}


\twolineshloka
{सर्वज्ञाः केशवं तस्मान्मामाहुर्द्विजसत्तमाः}
{स्वपत्न्यामाहितो गर्भ उचथ्येन महात्मना}


\twolineshloka
{उचथ्येऽन्तहिंते चैव कदाचिद्देवताज्ञया}
{बृहस्पतिरथाविन्दत्तां पत्नीं तस्य धीमतः}


\twolineshloka
{ततो वै तमृषिश्रेष्ठं मैथुनोपगतं तथा}
{उवाच गर्भः कौन्तेय पञ्चभूतगुणात्मकः}


\twolineshloka
{पूर्वागतोऽहं वरद नार्हस्यम्बां प्रबाधितुम्}
{एतद्बृहस्पतिः श्रुत्वा चुक्रोध च शशाप च}


\twolineshloka
{मैथुनायागतो यस्मात्त्वयाऽहं विनिवारितः}
{तस्मादन्धो यास्यसि त्वं मच्छापान्नात्र संशयः}


\twolineshloka
{स शापादृषिमुख्यस्य दीर्घं तम उपेयिवान्}
{स हि दीर्घतमा नाम नाम्ना ह्यासीदृषिः पुरा}


\twolineshloka
{वेदानवाप्य चतुरः साङ्गोपाङ्गान्सनातनान्}
{प्रयोजयामास तदा नाम गुह्यमिदं मम}


\twolineshloka
{आनुपूर्व्येण विधिना केशवेति पुनः पुनः}
{स चक्षुष्मान्समभवद्गौतमश्चाभवत्पुनः}


\twolineshloka
{एवं हि वरदं नाम केशवेति ममार्जुन}
{देवानामथ सर्वेषामृषीणां च महात्मनाम्}


\twolineshloka
{अग्निः सोमेन संयुक्त एकयोनिर्मुखं कृतम्}
{अग्नीषोममयं तस्माज्जगत्कृत्स्नं चराचरम्}


\twolineshloka
{अपि हि पुराणे भवति एकयोन्यावग्नीषोमौ देवाश्चाग्निमुखा इति}
{एकयोनित्वाच्च परस्परं हर्षयन्तो लोकान्धारयन्त इति}


\chapter{अध्यायः ३५१}
\twolineshloka
{अर्जुन उवाच}
{}


\threelineshloka
{अग्नीषोमौ कथं पूर्वमेकयोनी प्रकीर्तितौ}
{एष मे संशयो वीर तं छिन्धि मधुसूदन ॥श्रीभगवानुवाच}
{}


\twolineshloka
{हन्त ते वर्तयिष्यामि पुराणं पाण्डुनन्दन}
{आत्मतेजोद्भवं पार्थ शृणुष्वैकमना नृप}


\threelineshloka
{संप्रक्षालनकालेऽतिक्रान्ते चतुर्युगसहस्रान्ते}
{अव्यक्ते सर्वभूतप्रलये सर्वभूतस्थावरजङ्गमे}
{ज्योतिर्धरणिवायुरहिते अन्धे तमसि जलैकार्णवेलोके}


% Check verse!
ममायमित्यविदितभूतसंज्ञकेऽद्वितीये प्रतिष्ठिते
% Check verse!
न वै रात्र्यां न दिवसे न सति नासति न व्यक्ते नचाप्यव्यक्तेव्यवस्थिते
% Check verse!
एतस्यामवस्थायांनारायणगुणाश्रयादजरामरादतीन्द्रियादग्राह्यादसंभवात्सत्यादहिंस्याल्लवादिभिरद्वितीयादप्रवृत्तिविशेषादवैरादक्षयादमरादजरादमूर्तितःसर्वव्यापिनः सर्वकर्तुः शाश्वतात्तमसः परात्पुरुषः प्रादुर्भूतोस्यपुरुषस्य ब्रह्मयोनेर्ब्रह्मणः प्रादुर्भावे हरिरव्ययः
% Check verse!
निदर्शनमपि ह्यत्र भवति
\twolineshloka
{नासीदहो न रात्रिरासीन्न सदासीन्नासदासीत्तम एवपुरस्तादभवद्विश्वरूपम्}
{सा विश्वरूपस्य रजनी हिएवमस्यार्थोऽनुभाष्यते}


\fourlineindentedshloka
{तस्येदानीं तमः संभवस्य पुरुषस्य ब्रह्मयोनेर्ब्रह्मणःप्रादुर्भावे स पुरुषः प्रजाः सिसृक्षमाणो नेत्राभ्यामग्नीषोमौ ससर्ज}
{ततो भूतसर्गेषु सृष्टेषु प्रजाः क्रमवशाद्ब्रह्मक्षत्रमुपातिष्ठन्}
{यःसोमस्तद्ब्रह्म यद्ब्रह्म ते ब्राह्मणा योऽग्निस्तत्क्षत्रंक्षत्राद्ब्रह्मबलवत्तरम्}
{कस्मादिति परं भूतं नोत्पन्नपूर्वंदीप्यमानेऽग्नौ जुहोति यो ब्राह्मणमुखे जुहोतीति कृत्वा ब्रवीमिभूतसर्गः कृतो ब्रह्मणा भूतानि च प्रतिष्ठाप्य त्रैलोक्यं धार्यत इतिमन्त्रवादोपि हि भवति}


% Check verse!
त्वमग्ने यज्ञानां होता विश्वेषां हितो देवानां मानुषाणां च जगतइति
\twolineshloka
{निदर्शनं चात्र भावति विश्वेपामग्ने यज्ञानां त्वं होतेति}
{त्वंहितो देवैर्मनुष्यैर्जगत इति}


% Check verse!
अग्निर्हि यज्ञानां होता कर्ता स चाग्निर्ब्रह्म
\fourlineindentedshloka
{न ह्यृते मन्त्राणां हवनमस्ति न विना पुरुषं तपः संभवति}
{हविर्मन्त्राणां संपूजा विद्यते देवमानुपऋषीणामनेन त्वं होतेतिनियुक्तः}
{ये च मानुषहोत्राधिकारास्ते चक्रुर्ब्राह्मणस्य हि याजनंविधीयते न क्षत्रवैश्ययोर्द्विजात्योस्तस्माद्ब्राह्मणा ह्यग्निभूतायज्ञानुद्वहन्ति}
{यज्ञास्ते देवांस्तर्पयन्ति देवाः पृथिवीं भावयन्तिशतपथेऽपि हि ब्राह्मणमुखे भवति}


% Check verse!
अग्नौ समिद्धे स जुहोति यो विद्वान् ब्राह्मणमुखेनाहुतिंजुहोति
% Check verse!
एवमप्यग्निभूता ब्राह्मणा विद्वांसोऽग्निं भावयन्तिअग्निर्विष्णुः सर्वभूतान्यनुप्रविश्य प्राणान्धारयति
\threelineshloka
{अपिचात्र सनत्कुमारगीताः श्लोका भवन्ति}
{ब्रह्मा विश्वं सृजत्पूर्वं सर्वादिर्निरवस्करः}
{ब्रह्मघोषैर्दिवं तिष्ठन्त्यमरा ब्रह्मयोनयः}


\twolineshloka
{ब्राह्मणानामृतं वाक्यं कर्मश्रद्धातपांसि च}
{धारयन्ति महीं द्यां च शैत्याद्वाय्वमृतं तथा}


\twolineshloka
{नास्ति सत्यात्परो धर्मो नास्ति मातृसमो गुरुः}
{ब्राह्मणेभ्यः परं नास्ति प्रेत्य चेह च भूतये}


\twolineshloka
{नैषामुक्षा वहति नोत वाहान गर्गरो मथ्यति संप्रदाने}
{अपध्वस्ता दस्युभूता भवन्तियेषां राष्ट्रे ब्राह्मणा वृत्तिहीनाः}


\twolineshloka
{ते च पुराणेतिहासप्रामाण्यान्नारायणमुखोद्गताः}
{सर्वात्मानः सर्वकर्तारः सर्वभावाश्च ब्राह्मणाश्च}


% Check verse!
वाक्संयमकाले हि तस्य वरप्रदस्य देवदेवस्य ब्राह्मणाः प्रथमंप्रादुर्भूता ब्राह्मणेभ्यश्च शेषा वर्णाः प्रादुर्भूताः
% Check verse!
इत्थं च सुरासुरविशिष्टा ब्राह्मणा वेदमया ब्रह्मभूतेन पुरास्वयमेवोत्पादिताः सुरासुरमहर्षयो भूतविशेषाः स्थापिता निगृहीताश्चतेषां प्रभावः श्रूयताम्
\twolineshloka
{अहल्याधर्षणनिमित्तं हि गौतमाद्धरिश्मश्रुतामिन्द्रः प्राप्तः}
{गौतमनिमित्तं चेन्द्रो मुष्कवियोगं मेपवृषणत्वं चावाप}


% Check verse!
अश्विनोर्ग्रहप्रतिषेधोद्यतवज्रस्य पुरन्दरस्य च्यवनेन स्तम्भितौवाहू
% Check verse!
ऋतुवधप्राप्तमन्युना च दक्षेण भूयस्तपसा चात्मानं संयोज्यत्रिनेत्राकृतिरन्या ललाटे रुद्रस्योत्पादिता
\twolineshloka
{त्रिपुरवधार्थं दीक्षामुपगतस्य रुद्रस्य उशनसाजटाः शिरसउत्कृत्याग्नौ प्रयुक्तास्ततः प्रादुर्भूता भुजगास्तैरस्य भुजगैःपीड्यमानः कण्ठो नीलतामुपगतः}
{पूर्वे च मन्वन्तरे स्वायंभुवेनारायणहस्तबन्धग्रहणान्नीलकण्ठत्वमुपनीतः}


% Check verse!
अमृतोत्पादने पुनर्भक्षणतां वायुसमीकृतस्य विषस्योपगतश्चतद्भक्षणमिति तन्निमित्तमेव चन्द्रकला ब्रह्मणा निहिता

आङ्गिरसबृहस्पतेरुपस्पृशतो न प्रसादं गतवत्यः किलापः

अथबृहस्पतिरद्भ्यश्चुक्रोध यस्मान्ममोपस्पृशतः कलुपीभूता नचप्रसादमुपगतास्ततस्मादद्यप्रभृति झषमकरमत्स्यकच्छपजन्तुमण्डूकसंकीर्णाःकलुषीभवतेति

तदाप्रभृत्यापो यादोभिः संकीर्णाः कलुषीभवतेति

तदाप्रभृत्यापो यादोभिः संकीर्णाः संवृत्ताः
\twolineshloka
{विश्वरूपो हि वै त्वाष्ट्रः पुरीहितो देवानामासीत्}
{स्वस्त्रीयोसुराणां स प्रत्यक्षं देवेभ्योभागमदात्परोक्षमसुरेभ्यः}


% Check verse!
अथ हिरण्यकशिषुं पुरस्कृत्य विश्वरूपमातरं स्वसारमसुरा वरमयाचन्तहे स्वसरयं ते पुत्रस्त्वाष्ट्रो विश्वरूपस्त्रिशिरा देवानां पुरोहितःप्रत्यक्षं देवेभ्योभागमदात् परोक्षमस्माकं ततो देवा वर्धन्ते वयंक्षीयामस्तदेनं त्वं वारयितुमर्हसि तथा यथाऽस्मान्भजेदिति
\twolineshloka
{अथ विश्वरूपं नन्दनवनमुपगतं मातोवाच पुत्र किं परपक्षवर्धनस्त्वंमातुलपक्षं नाशयसि}
{नार्हस्येवं कर्तुमिति स विश्वरूपोमातुर्वाक्यमनतिक्रमणीयमिति मत्वा संपूज्य हिरण्यकशिपुमगात्}


% Check verse!
हैरण्यगर्भाच्च वसिष्ठाद्धिरण्यकशिषुः शापंप्राप्तवान्यस्मात्त्वयाऽन्यो वृतो होतातस्मादसमाप्तयज्ञस्त्वमपूर्वात्सत्वजाताद्वधं प्राप्स्यसीतितच्छापदानाद्धिरण्यकशिषुः प्राप्तवान्वधम्
% Check verse!
अथ विश्वरूपो मातृपक्षवर्धनोत्यर्थं तपस्व्यभवत्तस्यव्रतभङ्गार्थमिन्द्रो बह्नीः श्रीमत्योऽप्सरसो नियुयोज ताश्च दृष्ट्वामनः क्षुभितं तस्याभवत्तासु चाप्सरःसु नचिरादेव सक्तोऽभवत्सक्तं चैनंज्ञात्वा अप्सरस ऊचुर्गच्छामहे वयं यथागतमिति
\twolineshloka
{तास्त्वाष्ट्र उवचा}
{क्व गमिष्यथास्यतां तावन्मया सह श्रेयोभविष्यतीति तास्तमब्रुवन्वयं देवस्त्रियोऽप्सरस इन्द्रं देवं वरदं पुराप्रभविष्णुं वृणीमह इति}


% Check verse!
अथ ता विश्वरूपोऽब्रवीदद्यैव सेन्द्रा देवा नभविष्यन्तीति ततोमन्त्राञ्जजाप तैर्मन्त्रैरवर्धतत्रिशिरा एकेनास्येन सर्वलोकेषुयथावद्द्विजैः क्रियावद्भिर्यज्ञेषु सुहृतं सोमं पपौ एके(1)नान्नमेकेनसेन्द्रान्देवानथेन्द्रस्तं विवर्धमानं सोमपानाप्यायितसर्वगात्रंदृष्ट्वा चिन्तामापेदे सह देवैः
% Check verse!
ते देवाः सेन्द्रा ब्रह्माणमभिजग्मुस्त ऊचुर्विश्वरूपेणसर्वयज्ञेषु सुहुतः सोमः पीयते वयमभागाः संवृत्ता असुरपक्षो वर्धते वयंक्षीयामस्तदर्हसि नो विधातुं श्रेयोऽनन्तरमिति
% Check verse!
तान्ब्रह्मोवाच ऋषिर्भार्गवस्तपस्तप्यते दधीचः स याच्यतां वरं सयथा कलेवरं जह्यात् तस्यास्थिभिर्वज्रं क्रियतामिति
% Check verse!
ततो देवास्तत्रागच्छन्यत्र दधीचो भगवानृषिस्तपस्तेपे सेन्द्रादेवास्तमभिगम्योचुर्भगवंस्तपसा कुशलमविघ्नं चेति
% Check verse!
तान्दधीच उवाच स्वागतं भवतां उच्यतां किं क्रियतां यद्वक्ष्यथतत्करिष्यामि
% Check verse!
ते तमब्रुवञ्शरीपरित्यागं लोकहितार्थंभगवान्कर्तुमर्हतीति
% Check verse!
`एवमुक्तो दधीचस्तानब्रवीत्

सहस्रं वर्षाणामैन्द्रं पदमवाप्यतेमया यदि जह्याम्

तथेत्युक्त्वेन्द्रः स्वस्थानं दत्वा तपस्व्यभवत्

इन्द्रो दधीचोऽभवत्

तावत्पूर्वेण सेन्द्रा देवा आगमन्कालोऽयंदेहन्यासायेति

' अथ दधीचस्तथैवा विमनाः सुखदुःखसमो महायोगी आत्मनिपरमात्मानं समाधाय शरीरपरित्यागं चकार
\twolineshloka
{`श्रुतिरप्यत्र भवति इन्द्रो दधीचोस्थिभिकृतमिति' तस्यपरमात्मन्यपसृते तान्यस्थीति विधाता संगृह्य वज्रमकरोत्तेनवज्रेणाभेद्येनामधृष्येण ब्रह्मास्थिसंभूतेन विष्णुप्रविष्टेनेन्द्रोविश्वरूपं जघान शिरसां चास्य च्छेदनमकरोत्तक्ष्ण यज्ञपशोः शिरस्तेददानीत्युक्त्वा}
{तस्मादनन्तर विश्वरूपगात्रमथनसंभवंत्वाष्ट्रोत्पादितमेवारिं वृत्रमिन्द्रो जघान}


% Check verse!
(2)तस्यां द्वैधीभूतानां ब्रह्मवध्यायां भयादिन्द्रो देवराज्यंपर्यत्यजदप्सु संभवां च शीतलां मानससरोगतां नलिनीं प्रतिपेदे तत्रचैश्वर्ययोगादणुमात्रो भूत्वा विसग्रन्थिं प्रविवेश
% Check verse!
अथ ब्रह्मवध्याकृते प्रनष्टे त्रैलोक्यनाथे शचीपतौ जगदनीश्वरंबभूव देवान् रजस्तमश्चाविवेशमन्त्रा न प्रावर्तन्त महार्षीणां रक्षांसिप्रादुरभवन् ब्रह्म चोत्सादनं जगामानिन्द्राश्चाबलालोकाः सुप्रधृष्याबभूवुः
% Check verse!
अथ देवा ऋषयश्चायुषः पुत्रं नहुषं नाम देवराज्येऽभिषिपिचुर्नहुषःपञ्चभिः शतैर्ज्योतिषां ललाटे ज्वलद्भिः सर्वतेजोहरैस्त्रिविष्टपंपालयांबभूव
% Check verse!
अथ लोकाः प्रकृतिमापेदिरे स्वस्थाश्च हृष्टाश्च बभूवुः
% Check verse!
अथोवाच नहुषः सर्वं मां शक्रोपभोग्यमुपस्थितमृते शचीमिति सएवमुक्त्वा शचीसमीपमगमद्वृहस्पतिगृहे चासीनामुवाचनां सुभगेऽहमिन्द्रोदेवानां भजस्व मामिति तं शचीप्रत्युवाच प्रकृत्या त्वं धर्मवत्सलःसोमवंशोद्भवश्च नार्हसि परपत्नीधर्षणं कर्तुमिति
% Check verse!
तामथोवाच नहुष ऐन्द्रं पदमध्यास्यते मयाऽहमिन्द्रस्यराज्यरत्नहरो नात्राधर्मः कश्चित्त्वमिन्द्रोपभुक्तेति सा तमुवाचास्तिमम किंचिद्ब्रतमपर्यवसितं तस्यावभृथे त्वामुपगमिष्यामिकैश्चिदेवाहोभिरिति स शच्यैवमभिहितो जगाम
% Check verse!
अथ शची दुःखशोकार्ता भर्तृदर्शनलालसानहुषभयगृहीताबृहस्पतिमुपागच्छत्स च तामत्युद्विग्नां दृष्ट्वैव ध्यानं प्रविश्यभर्तृकार्यतत्परां ज्ञात्वा बृहस्पतिरुवाचानेनैव व्रतेन तपसा चान्वितादेवीं वरदामुपश्रुतिमाह्वय तदा सा ते इन्द्रं दर्शयिष्यतीति साऽथमहानियमस्थिता देवीं वरदामुपश्रुतिं मन्त्रैराह्वयत्सोपश्रुतिःशचीसमीपमगादुवाच चैनामियमस्तीति त्वयाऽऽहूतोपस्थिता किं ते प्रियंकरवाणीति तां भूर्ध्ना प्रणम्योवाच शची भगवत्यर्हसि मे भर्तारंदर्शयितुं त्वं सत्या माता सतां चेति सैनां मानसं सरोऽनयत्तत्रेन्द्रंविसग्रन्थिगतमदर्शयत्
\twolineshloka
{तामथ पत्नीं शचीं कृशां रलानां चेन्द्रो दृष्ट्वा चिन्तयांबभूवअहो मम दुःखमिदमुपगतं नष्टं हि मामियमन्विष्ययत्पत्न्यभ्यगमद्दुःस्वार्तेति तामिन्द्र उवाच (1)कथं वर्यसीति सातमुवाच नहु(2)पो मामाह्वयति पत्नीं कर्तुं कालश्चास्य मया कृत इतितामिन्द्र उवाच गच्छ नहुषस्त्वया वाच्योऽपूर्वेण मामृषियुक्तेन यानेनत्वमधिरूढ उद्वहस्वेति}
{इन्द्रस्य महान्ति वाहनानि सन्ति मनःप्रियाण्यधिरूढानि मया त्वमन्येनोपयातुमर्हतीति सैवमुक्ता हृष्टाजगामेन्द्रोपि विसग्रन्थिमेवाविवेश भूयः}


% Check verse!
अथेन्द्राणीमभ्यागतां दृष्ट्वा तामुवाच नहुषो `यन्मे त्वया कालःपरिकल्पितः' पूर्णः स काल इति तं शच्यब्रवीच्छक्रेण यथोक्तं समहर्षियुक्तं बाहनमधिरूढः शचीसमीपमुपागच्छत्
% Check verse!
अथ मैत्रावरुणिः कुम्भयोनिरगस्त्य ऋषिवरो महर्षीन्धिक््क्रियमाणांस्तान्नहुषेणापश्यत् तद्दुष्करमिति स्वयमपि गृहीतःपद्भ्यां चास्पृश्यत ततः स नहुषमब्रवीदकार्यप्रवृत्त पाप पतस्व महींसर्पो भव यावद्भूमिर्गिरयश्च तिष्ठेयुस्तावदिति समहर्षिवाक्यसमकालमेवतस्माद्यानादवापतत्
\twolineshloka
{अथानिन्द्रं पुनस्त्रैलोक्यमभवत् ततो देवा ऋषयश्च भगवन्तंविष्णुं शरणमिन्द्रार्थेऽभिजग्मुरूचुश्चैनं भगवन्निन्द्रंब्रह्महत्याभिभूतं त्रातुमर्हसीति ततः स वरदस्तानब्रवीदश्वमेधं यज्ञंवैष्णवं शक्रोऽभियजतां ततः स्वस्थानं प्राप्स्यतीति ततो देवाऋषयश्चेन्द्रं नापश्यन्यदा तदा शचीमूचुर्गच्छ सुभगे इन्द्रमानयस्वेतिसा तत्सर इन्द्रमाह्वयत्}
{इन्द्रश्च तस्मात्सरसः प्रत्युत्थाय गत्वासरस्वतीमभिजगाम बृहस्पतिश्चाश्वमेघं महाक्रतुं शक्रायाहारत् तत्रकृष्णसारङ्गं मेध्यमश्वमत्सृज्य पावनं तमेव कृत्वा इन्द्रं मरुत्पतिंबृहस्पतिः स्वं स्थानं प्रापयामास}


% Check verse!
ततः स देवराट् देवैर्ऋषिभिः स्तूयमानस्त्रिविष्टपस्थो निष्कल्मषोबभूव ह ब्रह्मवध्यां चतुर्षु स्थानेषु व्यभजत् वनितावृक्षगिर्यवनिषु

'वनितासु रजः

वृक्षेषु निर्यासः

गिरिषु शिम्बः पृथिव्यामूषरःतेऽस्पृश्याः

तस्माद्धविरलवणं पच्यते, एवमिन्द्रो ब्रह्मतेजःप्रभावोपवृंहितः शत्रुवधं कृत्वा स्वं स्थानं प्रापितः
\fourlineindentedshloka
{`नहुषस्य शापमोक्षार्थं देवैर्ऋषिभिश्च याच्यमानोऽगस्त्यःप्राह}
{यावत्स्वकुलजः श्रीमान्धर्मराड््भ्रातृभिर्युतः}
{भीमस्तस्यानुजस्तं त्वं ग्रहीता तु युधिष्ठरः}
{कथयित्वा स्वकान्प्रश्नांस्त्वां च तं च विमोक्ष्यति ॥'}


% Check verse!
आकाशगङ्गागतश्च पुरा भरद्वाजोमहर्षिरुपास्पृशत्रीन्क्रमान्क्रमता विष्णुनाऽभ्यासादित स भरद्वाजेनसलक्षणेन पाणिनोरसि ताडितः सलक्षणोरस्कः संवृत्तः
% Check verse!
भृगुणा महर्षिणा शप्तोऽग्निः सर्वभक्षत्वमुपानीतः
% Check verse!
अदितिर्वै देवानामन्नमपचदेतद्भुक्त्वा सुरा असुरान्हनिष्यन्तीतितत्र बुधो व्रतचर्यारामाप्तावागच्छददितिं चायाचद्भिक्षां देहीति तत्रदेवैः पूर्वमेतत्प्राश्यं नान्येनेत्यदितिर्भिक्षां नादादथभिक्षाप्रत्याख्यानरूषितेन बुधेन ब्रह्मभूतेनादितिः शप्ता अदिरेरुदरेभविष्यति व्यथा विवस्वतो द्वितीयजन्मन्यण्डसंज्ञितस्य अण्डंमातुरदित्या मारितं स मार्ताण्डो विवस्वानभवच्छ्राद्धदेवः
% Check verse!
दक्षस्य या दै दुहितरः षष्टिरासंस्तासां कश्यपारय त्रयोदशप्रादाद्दश धर्माय दश मनवे सप्तविँशतिमिन्दवे तासु तुल्यासुनक्षत्राख्यां गतासु सोमो रोहिण्यामभ्यधिकं प्रीतिमान भूत्ततस्ताःशिष्टाः पत्न्य इर्ष्याव्रत्यः पितुः समीपं गत्वेममर्थंशशंसुर्भगवन्नस्मासु तुल्यप्रभावासु सोमो रोहिणीं प्रत्यधिकं भजतीतिसोऽब्रवीद्यक्ष्मैनमावेक्ष्यत इति दक्षशापाच्च सोमं राजानं यक्ष्माविवेश स यक्ष्मणाऽऽविष्टो दक्षमगाद्दक्षश्चैनमब्रवीन्न समं वर्तस इतितत्रर्षयः सोममब्रुवन्क्षीयसे यक्ष्मणा पश्चिमायां दिशि समुद्रेहिरण्यसरस्तीर्थं तत्र गत्वा आत्मानमभिषेचयस्वेत्यथागच्छत्सोमस्तत्रहिरण्यसरस्तीर्थं गत्वा चात्मनः सेचनमकरोत् स्नात्वा चात्मानं पाप्मनोमोक्षयामास तत्र चाव भासितस्तीर्थे यदा सोमस्तदाप्रभृति च तीर्थंतत्प्रभासमिति नाम्ना ख्यातं बभूव
% Check verse!
तच्छापादद्यापि क्षीयते सोमोऽमावास्यान्तरस्थःपौर्णमासीमात्रेऽधिष्ठितो मेघलेखाप्रतिच्छन्नं बपुर्दर्शयति मेघसदृशंवर्णमगमत्तदस्य शशलक्ष्मविमलमभवत्
% Check verse!
स्थूलशिरा महर्षिर्मेरोः प्रागुत्तरे दिग्विभागे तपस्तेपेततस्तस्य तपस्तप्यमानस्य सर्वगन्धवहः शुचिर्वायुर्वायमानःशरीरमस्पृशत्स तपसा तापितशरीरः कृशो वायुनोपवीज्यमानो हृदयेपरितोषमगमत्तत्र किल तस्यानिलव्यजनकृतपरितोषस्य सद्यो वनस्पतयःपुष्पशोभां निदर्शितवन्त इति स एताञ्शशाप न सर्वकालं पुष्पफलवन्तोभविष्यथेति
% Check verse!
नारायणो लोकहितार्थं व़डवामुखो नाम पुरा महर्षिर्बभूव तस्य मेरौतपस्तप्यतः समुद्र आहूतो नागतस्तेनामर्षितेनात्मगात्रोष्मणाः समुद्रःस्तिमितजलः कृतः स्वेदप्रस्यन्दनसदृशश्चास्य लवणभावो जनितः
\twolineshloka
{उक्तश्चाप्यपेयो भविष्यस्तेतच्च ते तोयं बडवामुखसंज्ञितेनपेपीयमानं मधुरं भविष्यति तदेतदद्यापि बडवामुखसंज्ञितेनानुवर्तिना तोयंसमुद्रात्पीयते}
{`पुनरुमा दक्षकोपाद्धिमवतो गिरेर्दुहिता बभूव ॥'}


% Check verse!
हिमवतो गिरेर्दुहितरमुमां कन्यां रुद्रश्चकमे भृगुरपि चमहर्षिर्हिमवन्तमागत्याब्रवीत् कन्यामिमां मे देहीतितमब्रवीद्धिमवानभिलषितो वरो दुहितुर्हि रुद्र इतितमब्रवीद्भृगुर्यस्मात्त्वयाऽहं कन्यावरणकृतभावःप्रत्याख्यातस्तस्मान्न रत्नानां भवान्भाजनं भविष्यतीति
% Check verse!
अद्यप्रभृत्येतदवस्थितमृषिवचनं तदेवंविधं माहात्म्यंब्राह्मणानाम्
% Check verse!
क्षत्रमपि च ब्राह्मणप्रसादादेव शाश्वतीमव्ययां च पृथिवींपत्नीमभिगम्य बुभुजे
% Check verse!
तदेतद्ब्रह्मक्षत्रग्नीषोमीयं तेन जगद्धार्यरते
\chapter{अध्यायः ३५२}
\twolineshloka
{`भगवानुवाच}
{}


\threelineshloka
{नाम्नां निरुक्तं वक्ष्यामि शृणुष्वैकाग्रमानसः}
{सूर्याचन्द्रमसौ शश्वत्केशैर्मे अंशुसंज्ञितैः}
{'बोधयंस्तापयंश्चैव जगदुत्तिष्ठते पृथक्}


\threelineshloka
{बोधनात्तापनाच्चैव जगतो हर्षणं भवेत्}
{अग्नीषोमकृतैरेभिः कर्मभिः पाण्डुनन्दन}
{हृषीकेशोऽहमीशानो वरदो लोकभावनः}


\twolineshloka
{इलोपहूतं गेहेषु हरे भागं क्रतुष्वहम्}
{वर्णो मे हरितः श्रेष्ठस्तस्माद्धरिरहं स्मृतः}


\twolineshloka
{धामसारो हि लोकानामृतं चैव विचारितम्}
{ऋतधामा ततो विप्रैः सद्यश्चाहं प्रकीर्तितः}


\twolineshloka
{नष्टां च धरणीं पूर्वमविन्दं वै गुहागताम्}
{गोविन्द इति तेनाहं देवैर्वाग्भिरभिष्टुतः}


\twolineshloka
{शिपिविष्टेति चाख्यायां हीनरोमा च यो भवेत्}
{तेनाविष्ट तु यत्किंचिच्छिपिविष्टेति च स्मृतः}


\twolineshloka
{यास्को मामृपिरव्यग्रो नैकयज्ञेषु गीतवान्}
{शिपिविष्ट इति ह्यस्माद्गुह्यनामधरो ह्यहम्}


\twolineshloka
{स्तुत्वा मां शिपिविष्टेति यास्क ऋषिरुदारधीः}
{मत्प्रसादादधो नष्टं निरुक्तमभिजग्मिवान्}


\twolineshloka
{न हि जातो न जायेयं न जनिष्ये कदाचन}
{क्षेत्रज्ञः सर्वभूतानां तस्मादहमजं स्मृतः}


\twolineshloka
{नोक्तपूर्वं मया क्षुद्रमश्लीलं वा कदाचन}
{ऋता ब्रह्मसुता सा मे सत्यदेवी सरस्वती}


\twolineshloka
{सच्चासच्चैव कौन्तेय मया वेशितमात्मनि}
{पौष्करे ब्रह्मसदने सत्यं मासृषयो विदुः}


\twolineshloka
{सत्वान्न च्युतपूर्वोऽहं सत्यं वै विद्धि मत्कुतम्}
{जन्मनीहाभवेत्सत्वं पौर्विकं मे धनंजय}


\twolineshloka
{निराशीः कर्मसंयुक्तः सत्वतश्चाप्यकल्मषः}
{सात्वतज्ञानगदृष्टोऽहं सत्वतामिति सात्वतः}


\twolineshloka
{कृषाणि मेदिनीं पार्थ भूत्वा कार्ष्णायसो महान्}
{कृष्णो वर्णश्च मे यस्मात्तस्मात्कृषणोऽहमर्जुन}


\twolineshloka
{मया संश्लेषिता भूमिरद्भिर्व्योम च वायुना}
{वायुश्च तेजसा सार्धं वैकुण्ठत्वं ततो मम}


\twolineshloka
{निर्वाणं परमं ब्रह्म धर्मोऽसौ पर उच्यते}
{तस्मान्न च्युतपूर्वोऽहमच्युतस्तेन कर्मणा}


\twolineshloka
{पृथिवीनभसी चोभे विश्रुते विश्वतोमुखे}
{तयोः संधारणार्थं हि मामधोक्षजमञ्जसा}


\twolineshloka
{निरुक्तं वेदविदुपो वेदशब्दार्थचिन्तकाः}
{ते मां गायन्ति प्राग्वंशे अधोक्षज इति स्मृतः}


\twolineshloka
{शब्द एकमतैरेप व्याहृतः परमर्षिभिः}
{नान्यो ह्यधोक्षजो लोके ऋते नारायणं प्रभुम्}


\twolineshloka
{धृतं ममार्चिषो लोके जन्तूनां प्राणधारणम्}
{घृतार्चिरहमव्यग्रैर्वेदज्ञैः परिकीर्तितः}


\twolineshloka
{त्रयो हि धातवः ख्याताः कर्मजा इति ये स्मृताः}
{पित्तं श्लेष्मा च वायुश्च एष संघात उच्यते}


\twolineshloka
{एतैश्च धार्यते जन्तुरेतैः क्षीणैश्च क्षीयते}
{आयुर्वेदविदस्तस्मात्रिधातुं मां प्रचक्षते}


\twolineshloka
{वृषो हि भगवान्धर्मः ख्यातो लोकेषु भारत}
{नेघण्टुकपदाख्याने विद्धि मां वृषमुत्तमम्}


\twolineshloka
{कपिर्वराहः श्रेष्ठश्च धर्मश्च वृष उच्यते}
{त्समाद्वृषाकपिं प्राह कश्यपो मां प्रजापतिः}


\threelineshloka
{न चादिं न मध्यं तथा चैव नान्तंकदाचिद्विमन्ते द्विजा मे सुराश्च}
{अनाद्यो ह्यमध्यस्तथा चाप्यनन्तः}
{प्रगीतोऽहमीशो विभूर्लोकसाक्षी}


\twolineshloka
{शुचीनि श्रवणीयानि शृणोमीह धनंजय}
{न च पापानि गृह्णामि ततोऽहं वै शुचिश्रवाः}


\twolineshloka
{एकशृङ्गः पुरा भूत्वा वराहो नन्दिवर्धनः}
{इमां चोद्धृतवान्भूमिमेकशृङ्गस्ततो ह्यहम्}


\twolineshloka
{तथैवासं त्रिककुदो वाराहं रूपमास्थितः}
{त्रिककुत्तेन विख्यातः शरीरस्य तु मापनात्}


\twolineshloka
{निरिञ्च इति यत्प्रोक्तं कापिल ज्ञानचिन्तकैः}
{स प्रजापतिरेवाहं चेतनात्सर्वलोककृत्}


\twolineshloka
{विद्यासहायवन्तं मामादित्यस्थं सनातनम्}
{कपिलं प्राहुराचार्याः साङ्ख्या निश्चितनिश्चयाः}


\twolineshloka
{हिरण्यगर्भो द्युतिमान्य एष च्छन्दसि स्तुतः}
{योगैः संपूज्यते नित्यं स एवाहं विभुः स्मृतः}


\threelineshloka
{एकविंशतिसाहस्रं ऋग्वेदं मां प्रचक्षते}
{सहस्रशाखं यत्साम ये वै वेदविदो जनाः}
{गायन्त्यारण्यके विप्रा मद्भक्तास्ते हि दुर्लभाः}


\twolineshloka
{षट््पञ्चाशतमष्टौ च सप्तत्रिंशतमित्यत}
{यस्मिञ्शाखा यजुर्वेदे सोहमाध्वर्यवे स्मृतः}


\twolineshloka
{पञ्चकल्पमथर्वाणं कृत्याभिः परिबृंहितम्}
{कल्पयन्ति हि मां विप्रा अथर्वाणविदस्तथा}


\twolineshloka
{शाखाभेदाश्च ये केचिद्याश्च शाखासु गीतयः}
{स्वरवर्णसमुच्चाराः सर्वांस्तान्विद्धि मत्कृतान्}


\twolineshloka
{यत्तद्धयशिरः पार्थ समुदेति वरप्रदम्}
{सोहमेवोत्तरे भागे क्रमाक्षरविभागवित्}


\twolineshloka
{रामादेशितमार्गेण मत्प्रसादान्महात्मना}
{पाञ्चालेन क्रमः प्राप्तस्तस्माद्भूतात्सनातनात्}


\threelineshloka
{बाभ्रव्यगोत्रः स बभौ प्रथमं क्रमपारगः}
{नारायणाद्वरं लब्ध्वा प्राप्य योगमनुत्तमम्}
{क्रमं प्रणीय शिक्षां च प्रणयित्वा स गालवः}


\threelineshloka
{पुण्डरीकोऽथ राजा च ब्रह्मदत्तः प्रतापवान्}
{जातीमरणजं दुःखं स्मृत्वास्मृत्वा पुनः पुनः}
{सप्तजातिषु मुख्यत्वाद्योगाना संपदं गतः}


\twolineshloka
{पुराऽहमात्मजः पार्थ प्रथितः कारणान्तरे}
{धर्मस्य कुरुशार्दूल ततोऽहं धर्मजः स्मृतः}


\twolineshloka
{नरनारायणौ पूर्वं तपस्तेपतुरव्ययम्}
{धर्मयानं समारूढौ पर्वते गन्धमादने}


\twolineshloka
{तत्कालसमये चैव दक्षयज्ञो बभूव ह}
{न चैवाकल्पयद्भागं दक्षो रुद्रस्य भारत}


\twolineshloka
{ततो दधीचिवचनाद्दक्षयज्ञमपाहरत्}
{ससर्ज शूलं कोपेन प्रज्वलन्तं मुहुर्मुहुः}


\threelineshloka
{तच्छूलं भस्मसात्कृत्वा दक्षयज्ञं सविस्तरम्}
{आवयोः सहसाऽगच्छद्वदर्याश्रममन्तिकात्}
{वेगेन महता पार्थ पतन्नारायणोरसि}


\twolineshloka
{तत्तस्यतेजसाऽऽविष्टाः केशा नारायणस्य ह}
{बभूवुर्मुञ्जवर्णास्तु ततोऽहं मुञ्जकेशवान्}


\twolineshloka
{तच्च शूलं विनिर्धूतं हुंकारेण महात्मना}
{जगाम शंकरकरं नारायणसमाहतम्}


% Check verse!
अथ रुद्र उपाधावत्तावृषी तपसाऽन्वितौ
\twolineshloka
{तत एनं समुद्धूतं कण्ठे जग्राह पाणिना}
{नारायणः स विश्वात्मा तेनास्य शितिकण्ठता}


\twolineshloka
{अथ रुद्रविघातार्थमिषीकां नर उद्धरन्}
{मन्त्रैश्च संयुयोजाशु सोऽभवत्परशुर्महान्}


\twolineshloka
{क्षिप्तश्च सहसा तेन खण्डनं प्राप्तवांस्तदा}
{ततोऽहं खण्डपरशुः स्मृतः परशुखण्डनात्}


\twolineshloka
{`रुद्रस्य भागं प्रददुर्भागमुच्छेषणं पुनः}
{श्रुतिरप्यत्र भवति वेदैरुक्तस्तथा पुनः}


\twolineshloka
{उच्छेपणभागो वै रुद्रस्तस्योच्छेपणेन होतव्यमिति सर्वे गम्यरूपेणतदा ॥' अर्जुन उवाच}
{}


\threelineshloka
{अस्मिन्युद्धे तु वार्ष्णेय त्रैलोक्यशमने तदा}
{को जय प्राप्तवांस्तत्र शंसैतन्मे जनार्दन ॥श्रीभगवानुवाच}
{}


\twolineshloka
{तयोः संरब्धयोर्युद्धे रुद्रनारायणात्मनोः}
{उद्विग्राः सहसा कृत्स्नाः सर्वे लोकास्तदाऽभवन्}


\twolineshloka
{नागृह्णात्पाव्नकः शुभ्रं मूखेषु सुहुतं हविः}
{वेदा न प्रतिभान्ति स्म ऋषीणां भावितात्मनां}


\twolineshloka
{देवान्रजस्तमश्चैव समाविविशतुस्तदा}
{वसुधा संचकम्पे च नभश्च विपफाल ह}


\twolineshloka
{निष्प्रभाणि च तेजांसि ब्रह्मा चैवासनच्युतः}
{अगाच्छोपं समुद्रश्च हिमवांश्च व्यशीर्यत}


\threelineshloka
{तस्मिन्नेवं समुत्पन्ने निमित्ते पाण्डुनन्दन}
{ब्रह्मा वृतो देवगणार्ऋषिभिश्च महात्मभिः}
{आजगामाशु तं देशं यत्र युद्धमवर्तत}


\threelineshloka
{सोऽञ्जलिप्रग्रहो भूत्वा चतुर्वक्रो निरुक्तगः}
{उवाच वचनं रुद्रं लोकानामस्तु वै शिवम्}
{त्यजायुधानि विश्वेश जगतो हितकाम्यया}


\threelineshloka
{यदक्षरमथाव्यक्तमीशं लोकस्य भावनम्}
{कूटस्थं कर्तृनिर्द्वन्द्वमकर्तेति च यं विदुः}
{व्यक्तिभावगतस्यास्य एका मूर्तिरियं शुभा}


\twolineshloka
{नरो नारायणश्चैव जातौ धर्मकुलोद्वहौ}
{तपसा महता युक्तौ देवश्रेष्ठौ महाव्रतौ}


\twolineshloka
{अहं प्रसादजस्तस्य कुतश्चित्कारणान्तरे}
{त्वं चैव क्रोधजस्तात पूर्वसर्गे सनातनः}


\twolineshloka
{मया च सार्धं वरद विबुधैश्च महर्षिभिः}
{प्रसादयाशु लोकानां शान्तिर्भवतु माचिरम्}


\threelineshloka
{ब्रह्मणा त्वेवमुक्तस्तु रुद्रः क्रोधाग्निमुत्सृजन्}
{प्रसादयामास ततो देवं नारायणं प्रभुम्}
{शरणं च जगामाद्यं वरेण्यं वरदं हरिम्}


\twolineshloka
{ततोऽथ वरदो देवो जितक्रोधो जितेन्द्रियः}
{प्रीतिमानभवत्तत्र रुद्रेण सह संगतः}


\twolineshloka
{ऋषिभिर्ब्रह्मणा चैव विबुधैश्च सुपूजितः}
{उवाच देवमीशानमीशः स जगतो हरिः}


\twolineshloka
{यस्त्वां वेत्ति स मां वेत्ति यस्त्वामनु स मामनु}
{नावयोरन्तरं किंचिन्मा तेऽभूद्वुद्धिरन्यथा}


\threelineshloka
{अद्यप्रभृति श्रीवत्सः शूलाङ्को मे भवत्वयम्}
{मम पाण्यङ्कितश्चापि श्रीकण्ठस्त्वं भविष्यसि ॥श्रीभगवानुवाच}
{}


\threelineshloka
{एवंलक्षणमुत्पाद्य परस्परकृतं तदा}
{सख्यं चैवातुलं कृत्वा रुद्रेण सहितावृषी}
{तपस्तेपतुरव्यग्रौ विसृज्य त्रिदिवौकसः}


\threelineshloka
{एष ते कथितः पार्थ नारायणजयो मृधे}
{नामानि चैव गुह्यानि निरुक्तानि च भारत}
{ऋषिभिः कथितानीह यानि संकीर्तितानि ते}


\twolineshloka
{एवं बहुविधै रूपैश्चरामीह वसुंधराम्}
{ब्रह्मलोकं च कौन्येय गोलोकं च सनातनम्}


% Check verse!
मया त्वं रक्षितो युद्धे महान्तं प्राप्तवाञ्जयम्
\twolineshloka
{यस्तु ते सोग्रतो याति युद्धे संप्रत्युपस्थिते}
{तं विद्धि रुद्रं कौन्तेय देवदेवं कपर्दिनम्}


\twolineshloka
{कालः स एव विहितः क्रोधजेति मया तव}
{निहतांस्तेन वै पूर्वं हतवानसि यान्रिपून्}


\twolineshloka
{अप्रमेयप्रभावं तं देवदेवमुमापतिम्}
{नमस्व देवं प्रयतो विश्वेशं हरमक्षयम्}


\twolineshloka
{यश्च ते कथितः पूर्वं क्रोधजेति पुनः पुनः}
{तस्य प्रभाव एवाग्रे यच्छ्रुतं ते धनंजय}


\chapter{अध्यायः ३५३}
\twolineshloka
{शौनक उवाच}
{}


\twolineshloka
{सौते सुमहदाख्यानं भवता परिकीर्तितम्}
{यच्छ्रुत्वा मुनयः सर्वे विस्मयं परमं गताः}


\twolineshloka
{सर्वाश्रमाभिगमनं सर्वतीर्थावगाहनम्}
{न तथा फलदं सौते नारायणकथा यथा}


\twolineshloka
{पाविताङ्गाः स्म संवृत्ताः श्रुत्वेमामादितः कथाम्}
{नारायणाश्रयां पुण्यां सर्वपापप्रमोचनीम्}


\twolineshloka
{दुर्दर्शो भगवान्देवः सर्वलोकनमस्कृतः}
{सब्रह्मकैः सुरैः कृत्स्नैरन्यैश्चैव महर्षिभिः}


\twolineshloka
{दृष्टवान्नारदो यत्तु देवं नारायणं हरिम्}
{नूनमेनद्ध्यनुमतं तस्य देवस्य सूतज}


\fourlineindentedshloka
{यदृष्टवाञ्जगन्नाथमनिरुद्दतनौ स्थितम्}
{यत्प्राद्रवत्पुनर्भूयो नारदो देवसत्तमौ}
{नरनारायणौ द्रष्टुं कारणं तद्ब्रवीहि मे ॥सौतिरुवाच}
{}


\twolineshloka
{तस्मिन्यज्ञे वर्तमाने राज्ञः पारिक्षितस्य वै}
{कर्मान्तरेषु विधिवद्वर्तमानेषु शौनक}


\threelineshloka
{कृष्णद्वैपायनं व्यासमृषिं वेदनिधिं प्रभुम्}
{परिपप्रच्छ राजेन्द्रः पितामहपितामहम् ॥जनमेजय उवाच}
{}


\twolineshloka
{श्वेतद्वीपान्निवृत्तेन नारदेन सुरर्षिणा}
{ध्यायता भगवद्वाक्यं चेष्टितं किमतः परम्}


\twolineshloka
{बदर्याश्रममागम्य समागम्य च तावृषी}
{कियन्तं कालमवसत्कां कथां पृष्टवांश्च सः}


\twolineshloka
{इदं शतसहस्राद्धि भारताख्यानविस्तरात्}
{आमन्थ्य मतिमन्थेन ज्ञानोदधिमनुत्तमम्}


\threelineshloka
{नवनीतं यथा दध्नो मलयाच्चन्दनं यथा}
{अरण्यकं च वेदेभ्य ओषधीभ्योऽमृतं यथा}
{समुद्धृतमिदं ब्रह्मन्कथामृतमिदं तथा}


\twolineshloka
{तपोनिधे त्वयोक्तं हि नारायणकथाश्रयम्}
{स ईशो भगवान्देवः सर्वभूतात्मभावनः}


\twolineshloka
{अहो नारायणं तेजो दुर्दर्शं द्विजसत्तम}
{यत्राविशन्ति कल्पान्ते सर्वे ब्रह्मादयः सुराः}


\twolineshloka
{ऋषयश्च सगन्धर्वा यच्च किंचिच्चराचरम्}
{न ततोऽस्ति परं मन्ये पावतं दिवि चेह च}


\twolineshloka
{सर्वाश्रमाभिगमनं सर्वतीर्थावगाहनम्}
{न तथा फलदं चापि नारायणकथा यथा}


\twolineshloka
{सर्वथा पाविताः स्मेह श्रुत्वेमामादितः कथाम्}
{हरेर्विश्वेश्वरस्येह सर्वपापप्रणाशनीम्}


\twolineshloka
{न चित्रं कृतवांस्तत्र यदार्यो मे धनंजयः}
{वासुदेवसहायो यः प्राप्तवाञ्जयमुत्तमम्}


\twolineshloka
{न चास्य किंचिदप्राप्यं मन्ये लोकेष्वपि त्रिषु}
{त्रैलोक्यनाथो विष्णुः स यथाऽसीत्साह्यकृत्सखा}


\twolineshloka
{धन्याश्च सर्व एवासन्ब्रह्मंस्ते मम पूर्वजाः}
{हिताय श्रेयसे चैव येषामासीज्जनार्दनः}


\twolineshloka
{तपसाऽप्यथ दुर्दर्शो भगवाँल्लोकपूजितः}
{यं दृष्टवन्तस्ते साक्षाच्छ्रीवत्साङ्कविभूषणम्}


\twolineshloka
{तेभ्यो धन्यतरश्चैव नारदः परमेष्ठिजः}
{`दृष्टवान्यो हरिं देवं नारायणमजं विभुम् ॥'}


\twolineshloka
{न चाल्पतेजसमृषिं वेद्मि नारदमव्ययम्}
{श्वेतद्वीपं समासाद्य येन दृष्टः स्वयं हरिः}


\twolineshloka
{देवप्रसादानुगतं व्यक्तं तत्तस्य दर्शनम्}
{यद्दृष्टवांस्तदा देवमनिरूद्धतनौ स्थितम्}


\twolineshloka
{बदरीमाश्रमं यत्तु नारदः प्राद्रवत्पुनः}
{नरनारायणौ द्रष्टुं किं नु तत्कारणं मुने}


\threelineshloka
{श्वेतद्वीपान्निवृत्तश्च नारदः परमेष्ठिजः}
{बदरीमाश्रमं प्राप्य समागम्य च तावृषी}
{कियन्तं कालमवसत्प्रश्नान्कान्पृष्टवांश्च ह}


\fourlineindentedshloka
{श्वेतद्वीपादुपावृत्ते तस्मिन्वा सुमहात्मनि}
{किमब्रूतां महात्मानौ नरनारायणावृषी}
{तदेतन्मे यथातत्त्वं सर्वमाख्यातुमर्हसि ॥`सौतिरुवाच}
{}


\threelineshloka
{तस्य तद्वचनं श्रुत्वा कृष्णद्वैपायनस्तदा}
{शशास शिष्यमासीनं वैशंपायनमन्तिके}
{तदस्मै सर्वमाचक्ष्व यन्मत्तः श्रुतवानसि}


\threelineshloka
{गुरोर्वचनमाज्ञाय स तु विप्रर्षभस्तदा}
{आचचक्षे ततः सर्वमितिहासं पुरातनम् ॥'वैशंपायन उवाच}
{}


\twolineshloka
{नमो भगवते तस्मै व्यासायामिततेजसे}
{यस्य प्रसादाद्वक्ष्यामि नारायणकथामिमाम्}


\threelineshloka
{प्राप्य श्वेतं महाद्वीपं दृष्ट्वा च हरिमव्ययम्}
{निवृत्तो नारदो राजस्तरसा मेरुमागमत्}
{हृदयेनोद्वहन्भारं यदुक्तं परमात्मना}


\twolineshloka
{पश्चादस्याभवद्राजन्नात्मनः साध्वसं महत्}
{यद्गत्वा दूरमध्वानं क्षेमी पुनरिहागतः}


\twolineshloka
{मेरोः प्रचक्राम ततः पर्वतं गन्धमादनम्}
{निपपात च खात्तूर्णं विशालां बदरीमनु}


\twolineshloka
{ततः स ददृशे देवौ पुराणावृषिसत्तमौ}
{तपश्चरन्तौ सुमहदात्मनिष्ठौ महाव्रतौ}


\twolineshloka
{तेजसाऽभ्यधिकौ सूर्यात्सर्वलोकविरोचनात्}
{श्रीवत्सलक्षणौ पूज्यौ जटामण्डलधारिणौ}


\twolineshloka
{जालपादभुजौ तौ तु पादयोश्चक्रलक्षणौ}
{व्यूढोरस्कौ दीर्घभुजौ तथा मुष्कचतुष्किणौ}


\threelineshloka
{षष्टिदन्तावष्टदंष्ट्रौ मेघौघसदृशस्वनौ}
{स्वास्यौ पृथुललाटौ च सुभ्रूसुहनुनासिकौ}
{आतपत्रेण सदृशे शिरसी देवयोस्तयोः}


\twolineshloka
{एवं लक्षणसंपन्नौ महापुरुषसंज्ञितौ}
{तौ दृष्ट्वा नारदो हृष्टस्ताभ्यां च प्रतिपूजितः}


\twolineshloka
{स्वागतेनाभिभाष्याथ पृष्टश्चानामयं तथा}
{बभूवान्तर्गतमतिर्निरीक्ष्य पुरुषोत्तमौ}


\twolineshloka
{सदोगतास्तत्र ये वै सर्वभूतनमस्कृताः}
{श्वेतद्वीपे मया दृष्टास्तादृशावृषिसत्तमौ}


\twolineshloka
{इति संचिन्त्य मनसा कृत्वा चाभिप्रदक्षिणाम्}
{स चोपविविशे तत्र पीठे कुशमये शुभे}


\twolineshloka
{ततस्तौ तपसां वासौ यशसां तेजसामपि}
{ऋषी शमदमोपेतौ कृत्वा पौर्वाह्णिकं विधिम्}


\twolineshloka
{यश्चान्नारदमव्यग्रौ पाद्यार्ध्याभ्यामथार्चतः}
{पीठयोश्चोपविष्टौ तौ कृतातिथ्याह्निकौ नृपौ}


\twolineshloka
{तेषु तत्रोपविष्टेषु स देशोऽभिव्यराजत}
{भ्राज्याहुतिमहाञ्वालैर्यज्ञवाटो यथाऽग्निभिः}


\twolineshloka
{अथ नारायणस्तत्र नारदं वाक्यमब्रवीत्}
{सुखोपविष्टं विश्रान्तं कृतातिथ्यं सुखस्थितम्}


\threelineshloka
{अपीदानीं स भगवान्परमात्मा सनातनः}
{श्वेतद्वीपे त्वया दृष्ट आवयोः प्रकृतिः परा ॥नारद उवाच}
{}


\twolineshloka
{दृष्टो मे पुरुषः श्रीमान्विश्वरूपधरोऽव्ययः}
{सर्वे लोका हि तत्रस्थास्तथा देवाः सहर्षिभिः}


% Check verse!
अद्यापि चैनं पश्यामि युवां पश्यन्सनातनौ
\twolineshloka
{यैर्लक्षणैरुपेतः स हरिरव्यरक्तरूपधृत्}
{तैर्लक्षणैरुपेतौ हि व्यक्तरूपधरौ युवाम्}


\twolineshloka
{दृष्टौ युवां मया तत्र तस्य देवस्य पार्श्वतः}
{इहैव चागतोऽस्म्यद्य विसृष्टः परमात्मना}


\twolineshloka
{को हि नाम भवेत्तस्य तेजसा यशसा श्रिया}
{सदृशस्त्रिषु लोकेषु ऋते धर्मात्मजौ युवाम्}


\twolineshloka
{तेन मे कथितः कृत्स्नो धर्मः क्षेत्रज्ञसंज्ञितः}
{प्रादुर्भावाश्च कथिता भविष्या इह ये यथा}


\twolineshloka
{तत्र ये पुरुषाः श्वेताः पञ्चेन्द्रियविवर्जिताः}
{प्रतिबुद्धाश्च ते सर्वे भक्ताश्च पुरुषोत्तमम्}


\twolineshloka
{तेऽर्चयन्ति सदा देवं तैः सार्धं रमते च सः}
{प्रियभक्तो हि भगवान्परमात्मा द्विजप्रियः}


\twolineshloka
{रमते सोऽर्च्यमानो हि सदा भागवतप्रियः}
{विश्वभुक्सर्वगो देवो माधवो भक्तवत्सलः}


\twolineshloka
{स कर्ता कारणं चैव कार्यं चातिबलद्युतिः}
{हेतुश्चाज्ञाविधानं च तत्त्वं चैव महायशाः}


\twolineshloka
{तपसा योज्य सोत्मानं श्वेतद्वीपात्परं हि यत्}
{तेज इत्यभिविख्यातं स्वयं भासावभासितम्}


\twolineshloka
{शान्तिः सा त्रिषु लोकेषु विहिता भावितात्मना}
{एतया शुभया बुद्ध्या नैष्ठिकं व्रतमास्थितः}


\twolineshloka
{न तत्र सूर्यस्तपति न सोमोऽभिविराजते}
{न वायुर्वाति देवेशे तपश्चरति दुश्चरम्}


\twolineshloka
{वेदीमष्टनलोत्सेधां भूमावास्थाय विश्वकृत्}
{एकपादस्थितो देव ऊर्ध्वबाहुरुदङ्भुखः}


\twolineshloka
{साङ्गानावर्तयन्वेदांस्तपस्तेपे सुदुश्चरम्}
{यद्ब्रह्म ऋषयश्चैव स्वयं पशुपतिश्च यत्}


\twolineshloka
{शेषाश्च विबुधश्रेष्ठा दैत्यदानवराक्षसाः}
{नागाः सुपर्णा गन्धर्वाः सिद्धा राजर्पयश्च ते}


\twolineshloka
{हव्यं कव्यं च सततं विधियुक्तं प्रयुञ्जते}
{कृत्स्नं तु तस्य देवस्य चरणावुपतिष्ठतः}


\twolineshloka
{याः क्रियाः संप्रयुक्ताश्च एकान्तगतबुद्धिभिः}
{ताः सर्वाः शिरसा देवः प्रतिगृह्णाति वै स्वयं}


\twolineshloka
{न तस्यान्यः प्रियतरः प्रतिबुद्धैर्महात्मभिः}
{विद्यते त्रिषु लोकेषु ततोऽस्यैकान्तिकं गतः}


\twolineshloka
{इह चैवागतोऽस्म्यद्य विसृष्टः परमात्मना}
{एवं मे भगवान्देवः स्वयमाख्यातवान्हरिः}


% Check verse!
आसिष्ये तत्परो भूत्वा युवाभ्यां सह नित्यशः
\chapter{अध्यायः ३५४}
\twolineshloka
{नरनारायणावूचतुः}
{}


\twolineshloka
{धन्योस्यनुगृहीतोसि यत्ते दृष्टः स्वयंप्रभुः}
{न हि तं दृष्टवान्कश्चित्पद्मयोनिरपि स्वयम्}


\twolineshloka
{अव्यक्तयोनिर्भगवान्दुर्दर्शः पुरुषोत्तमः}
{नारदैतद्धि नौ सत्यं वचनं समुदाहृतम्}


\twolineshloka
{नास्य भक्तात्प्रियतरो लोके कश्चन विद्यते}
{ततः स्वयं दर्शितवान्स्वमात्मानं द्विजोत्तम}


\twolineshloka
{तपो हि तप्यतस्तस्य यत्स्थानं परमात्मनः}
{न तत्संप्राप्नुते कश्चिदृते ह्यावां द्वितोत्तम}


\twolineshloka
{या हि सूर्यसहस्रस्य समस्तस्य भवेद्द्युतिः}
{स्थानस्य सा भवेत्तस्य स्वयं तेन विराजता}


\twolineshloka
{तस्मादुत्तिष्ठते विप्र देवाद्विश्वभुवः पतेः}
{क्षमा क्षमावतां श्रेष्ठ यया भूमिस्तु युज्यते}


\twolineshloka
{तस्माच्चोत्तिष्ठते देवात्सर्वभूतहिताद्रसः}
{आपो हि तेन युज्यन्ते द्रवत्वं प्राप्नुवन्ति च}


\twolineshloka
{तस्मादेव समुद्भूतं तेजो रूपगुणात्मकम्}
{येन संयुज्यते सूर्यस्ततो लोके विराजते}


\twolineshloka
{तस्माद्देवात्समुद्भूतः स्पर्शस्तु पुरुषोत्तमात्}
{येन स्म युज्यते वायुस्ततो लोकान्विवात्यसौ}


\twolineshloka
{तस्माच्चोत्तिष्ठते शब्दः सर्वलोकेश्वरात्प्रभोः}
{आकाशं युज्यते येन ततस्तिष्ठत्यसंवृतम्}


\twolineshloka
{तस्माच्चोत्तिष्ठते देवात्सर्वभूतगतं मनः}
{चन्द्रमा येन संयुक्तः प्रकाशगुणधारणः}


\twolineshloka
{सद्भूतोत्पादकं नाम तत्स्थानं वेदसंज्ञितम्}
{विद्यासहायो यत्रास्ते भगवान्हव्यकव्यभुक्}


\twolineshloka
{ये हि निष्कल्मषा लोके पुण्यपापविवर्जिताः}
{तेषां वै क्षेममध्वानं गच्छतां द्विजसत्तम}


\twolineshloka
{सर्वलोके तमोहन्ता आदित्यो द्वारमुच्यते}
{` ज्वालामाली महातेजा येनेदं धार्यते जगत् ॥'}


\twolineshloka
{आदित्यदग्धसर्वाङ्गा अदृश्याः केनचित्क्वचित्}
{परमाणुभूता भूत्वा तु तं देवं प्रविशन्त्युत}


\twolineshloka
{तस्मादपि च निर्मुक्ता अनिरुद्धतनौ स्थिताः}
{मनोभूतास्ततो भूत्वा प्रद्युम्नं प्रविशन्त्युत}


\twolineshloka
{प्रद्युम्नाच्चापि निर्मुक्ता जीवं संकर्षणं ततः}
{विशन्ति विप्रप्रवराः साङ्ख्या भागवतैः सह}


\threelineshloka
{ततस्त्रैगुण्यहीनास्ते परमात्मानमञ्जसा}
{प्रविशन्ति द्विजश्रेष्ठाः क्षेत्रज्ञं निर्गुणात्मकम्}
{सर्वावासं वासुदेवं क्षेत्रज्ञं विद्धि तत्त्वतः}


\twolineshloka
{समाहितमनस्काश्च नियताः संयतेन्द्रियाः}
{एकान्तभावोपगता वासुदेवं विशन्ति ते}


\twolineshloka
{आवामपि च धर्मस्य गुहे जातौ द्विजोत्तम}
{रम्यां विशालामाश्रित्य तप उग्रं समास्थितौ}


\twolineshloka
{ये तु तस्यैव देवस्य प्रादुर्भावाः सुरप्रियाः}
{भविष्यन्ति त्रिलोकस्थास्तेषां स्वस्तीत्यथो द्विज}


\twolineshloka
{विधिना स्वेन युक्ताभ्यां यथापूर्वं द्विजोत्तम}
{आस्थिताभ्यां सर्वकृच्छ्रं व्रतं सम्यगनुत्तमम्}


\twolineshloka
{`स्वार्थेन विधिना युक्तः सर्वकृच्छ्रव्रते स्थितः}
{'आवाभ्यामपि दृष्टस्त्वं श्वेतद्वीपे तपोधन}


\twolineshloka
{समागतो भगवता संकल्पं कृतवांस्तथा}
{सर्वं हि नौ संविदितं त्रैलोक्ये सचराचरे}


\threelineshloka
{यद्भविष्यति वृत्तं वा वर्तते वा शुभाशुभम्}
{सर्वं स ते कथितवान्देवदेवो महामुने ॥वैशंपायन उवाच}
{}


\twolineshloka
{एतच्छ्रुत्वा तयोर्वाक्यं तपस्युग्रे च वर्ततोः}
{नारदः प्राञ्जलिर्भूत्वा नारायणपरायणः}


\twolineshloka
{जजाप विधिवन्मन्त्रान्नारायणगतान्बहून्}
{दिव्यं वर्षसहस्रं हि नरनारायणाश्रमे}


\twolineshloka
{अवसत्स महातेजा नारदो भगवानृषिः}
{तावेवाभ्यर्चयन्देवौ नरनारायणौ च तौ}


\chapter{अध्यायः ३५५}
\twolineshloka
{वैशंपायन उवाच}
{}


\twolineshloka
{कस्यचित्त्वथ कालस्य नारदः परमेष्ठिजः}
{दैवं कृत्वा यथान्यायं पित्र्यं चक्रे ततः परम्}


\twolineshloka
{ततस्तं वचनं प्राह ज्येष्ठो धर्मात्मजः प्रभुः}
{क इज्यते द्विजश्रेष्ठ दैवे पित्र्ये च कल्पिते}


\threelineshloka
{त्वया मतिमतां श्रेष्ठ तन्मे शंस यथातथम्}
{किमेतत्क्रियते कर्म फलं वाऽस्य किमिष्यते ॥नारद उवाच}
{}


\twolineshloka
{त्वयैतत्कथितं पूर्वं दैवं कर्तव्यमित्यपि}
{दैवतं च परो ज्ञेयः परमात्मा सनातनः}


\twolineshloka
{ततस्तद्भावितो नित्यं यजे वैकुण्ठमव्ययम्}
{तस्माच्च प्रसृतः पूर्वं ब्रह्मा लोकपितामहः}


\twolineshloka
{मम वै पितरं प्रीतः परमेष्ठ्यप्यजीजनत्}
{अहं संकल्पजस्तस्य पुत्रः प्रथमकल्पितः}


\twolineshloka
{यजामि वै पितॄन्साधो नारायणविधौ कृते}
{एवं स एव भगवान्पिता माता पितामहः}


\twolineshloka
{इज्यते पितृयज्ञेषु मया नित्यं जगत्पतिः}
{श्रुतिश्चाप्यपरा देवाः पुत्रान्हि पितरोऽयजन्}


\twolineshloka
{वेदश्रुतिः प्रनष्टा च पुनरध्यापिता सुतैः}
{ततस्ते मन्त्रदाः पुत्राः पितॄणामिति वैदिकम्}


\twolineshloka
{नूनं सुरैस्तद्विदितं युवयोर्भावितात्मनोः}
{पुत्राश्च पितरश्चैव परस्परमपूजयन्}


\threelineshloka
{त्रीन्पिण्डान्न्यस्य वै पित्र्यान्पूर्वं दत्त्वा कुशानिति}
{कथं तु पिण्डसंज्ञां ते पितरो लेभिरे पुरा ॥नरनारायणावूचतुः}
{}


\twolineshloka
{इमां हि धरणीं पूर्वं नष्टां सागरमेखलाम्}
{गोविन्द उज्जहाराशु वाराहं रूपमास्थितः}


\twolineshloka
{स्थापयित्वा तु धरणीं स्वे स्थाने पुरुषोत्तमः}
{जलकर्दमलिप्ताङ्गो लोककार्यार्थमुद्यतः}


\twolineshloka
{प्राप्ते चाह्निककाले तु मध्यदेशगते रवौ}
{दंष्ट्राविलग्नांस्त्रीन्पिण्डान्विधूय सहसा प्रभुः}


\twolineshloka
{स्थापयामास वै पृथ्व्यां कुशानास्तीर्य नारद}
{स तेष्वात्मानमुद्दिश्य पित्र्यं चक्रे यथाविधि}


\twolineshloka
{संकल्पयित्वा त्रीन्पिण्डान्स्वेनैव विधिना प्रभुः}
{आत्मगात्रोष्मसंभूतैः स्नेहगर्भैस्तिलैरपि}


\threelineshloka
{प्रोक्ष्यापसव्यं देवेशः प्राङ्भुखः कृतवान्स्वयम्}
{मर्यादास्थापनार्थं च ततो वचनमुक्तवान् ॥वृषाकपिरुवाच}
{}


\twolineshloka
{अहं हि पितरः स्रष्टुमुद्यतो लोककृत्स्वयम्}
{तस्य चिन्तयतः सद्यः पितृकार्यविधीन्परान्}


\twolineshloka
{दंष्ट्राभ्यां प्रविनिर्धूता ममैते दक्षिणां दिशम्}
{आश्रिता धरणीं पीड्य तस्मात्पितर एव ते}


\twolineshloka
{त्रयो मूर्तिविहीना वै पिण्डमूर्तिधरास्त्विमे}
{भवन्तु पितरो लोके मया सृष्टाः सनातनाः}


\threelineshloka
{पिता पितामहश्चैव तथैव प्रपितामहः}
{अहमेवात्र विज्ञेयस्त्रिषु पिण्डेषु संस्थितः}
{नास्ति मत्तोऽधिकः कश्चित्को वान्योर्च्यो मया स्वयं}


\twolineshloka
{अहमेव पिता लोके अहमेव पितामहः}
{पितामहपिता चैव अहमेवात्र कारणम्}


\threelineshloka
{इत्येतदुक्त्वा वचनं देवदेवो वृषाकपिः}
{वराहपर्वते विप्र दत्त्वा पिण्डान्सविस्तरान्}
{आत्मानं पूजयित्वैव तत्रैवादर्शनं गतः}


\twolineshloka
{एतदर्थं सुभमते पितरः पिण्डसंज्ञिताः}
{लभन्ते सततं पूजां वृषाकपिवचो यथा}


\twolineshloka
{ये यजन्ति पितॄन्देवान्गुरूंश्चैवातिर्थीस्तथा}
{गाश्चैव द्विजमुख्यांश्च पितरं मातरं तथा}


\twolineshloka
{कर्मणा मनसा वाचा विष्णुमेव यजन्ति ते}
{अन्तर्गतः स भगवान्सर्वसत्वशरीरगः}


\twolineshloka
{समः सर्वेषु भूतेषु ईश्वरः सुखदुःखयोः}
{महान्महात्मा सर्वात्मा नारायण इति श्रुतिः}


\chapter{अध्यायः ३५६}
\twolineshloka
{वैशंपायन उवाच}
{}


\twolineshloka
{श्रुत्वैतन्नारदो वाक्यं नरनारायणेरितम्}
{अत्यन्तं भक्तिमान्देवे एकान्तित्वमुपेयिवान्}


\threelineshloka
{उषित्वा वर्षसाहस्रं नरनारायणाश्रमे}
{श्रुत्वा भगवदाख्यानं दृष्ट्वा च हरिमव्ययम्}
{जगाम हिमवत्कुक्षावाश्रमं स्वं सुरार्चितम्}


\twolineshloka
{तावपि ख्यातयशसौ नरनारायणावृषी}
{तस्मिन्नेवाश्रमे रम्ये तेपतुस्तप सत्तमम्}


\twolineshloka
{त्वमप्यमितविक्रान्तः पाण्डवानां कुलोद्वहः}
{पावितात्माऽद्य संवृत्तः श्रुत्वेमामादितः कथाम्}


\twolineshloka
{नैव तस्यापरो लोको नायं पार्थिवसत्तम}
{कर्मणा मनसा वाचा यो द्विष्याद्विष्णुमव्ययम्}


\twolineshloka
{मज्जन्ति पितरस्तस्य नरके शाश्वतीः समाः}
{यो द्विष्याद्विबुधश्रेष्ठं देवं नारायणं हरिम्}


\twolineshloka
{कथं नाम भवेद्द्वेष्य आत्मा लोकस्य कस्यचित्}
{आत्मा हि पुरुषव्याघ्र ज्ञेयो विष्णुरिति श्रुतिः}


\threelineshloka
{य एष गुरुरस्माकमृषिर्गन्धवतीसुतः}
{तेनैतत्कथितं तात माहात्म्यं परमात्मनः}
{तस्माच्छ्रुतं मया चेदं कथितं च तवानघ}


\twolineshloka
{नारदेन तु संप्राप्तः सरहस्यः ससंग्रहः}
{एष धर्मो जगन्नाथात्साक्षान्नारायणान्नृप}


\twolineshloka
{एवमेष महान्धर्मः स ते पूर्वं नृपोत्तम}
{कथितो हरिगीतासु समासविधिकल्पितः}


\threelineshloka
{कृष्णद्वैपायनं व्यासं विद्धि नारायणं प्रभुम्}
{को ह्यन्यः पुण्डरीकाक्षान्महाभारतकृद्भवेत्}
{धर्मान्नानाविधांश्चैव को ब्रूयात्तमृते प्रभुम्}


\threelineshloka
{वर्ततां ते महायज्ञो यथासंकल्पितस्त्वया}
{संकल्पिताश्वमेधस्त्वं श्रुतधर्मा च तत्त्वतः ॥सौतिरुवाच}
{}


\twolineshloka
{एतत्तु महदाख्यानं श्रुत्वा पारीक्षितो नृपः}
{ततो यज्ञसमाप्त्यर्थं क्रियाः सर्वाः समारभत्}


\twolineshloka
{नारायणीयमाख्यानमेतत्ते कथितं मया}
{पृष्टेन शौनकाद्येह नैमिषारण्यवासिषु}


\twolineshloka
{नारदेन पुरा यद्वै गुरवे तु निवेदितम्}
{ऋषीणां पाण्डवानां च शृण्वतोः कृष्णभीष्मयोः}


\twolineshloka
{स हि परमर्षिर्जनभुवनपतिःपृथुधरणिधरः श्रुतिविनयपरः}
{शमनियमनिधिर्यमनियमपरो द्विजवरसहितस्तव च भवतु गतिर्हरिरमरहितः}


\threelineshloka
{असुरवधकरस्तपसांनिधिःसुमहतां यशसां च भाजनम्}
{एकान्तिनां शरणदोऽभयदो गतिदोगतिदोस्तु वः सुखभागकरः}
{मधुकैटभहा कृतधर्मविदां गतिदोभयदो मखभागहरोस्तु शरणं स ते}


\twolineshloka
{त्रिगुणो विगुणश्चतुरात्मधरःपूर्तेष्टयोश्च फलभागहरः}
{विदधातु नित्यमजितोऽतिचलोगतिरात्मवतां सुकृतिनामृषीणाम्}


\twolineshloka
{तं लोकसाक्षिणमजं पुरुषं पुराणंरविवर्णमीश्वरं गतिं बहुशः}
{प्रणमध्वमेकमतयो यतःसलिलोद्भवोपि तमृषिं प्रणतः}


\twolineshloka
{स हि लोकयोनिरसृतस्य पदंसूक्ष्मं परायणमचलं हि पदम्}
{तत्साङ्ख्ययोगिभिरुदाहृतं तंबुद्ध्या यतात्मभिरिदं सनातनम्}


\chapter{अध्यायः ३५७}
\twolineshloka
{शौनक उवाच}
{}


\twolineshloka
{श्रुतं भगवतस्तस्य माहात्म्यं परमात्मनः}
{जन्मधर्मगृहे चैव नरनारायणात्मकम्}


\twolineshloka
{महावराहसृष्टा च पिण्डोत्पत्तिः पुरातनी}
{प्रवृत्तौ च निवृत्तौ च यो यथा परिकल्पितः}


\twolineshloka
{तथा स नः श्रुतो ब्रह्मन्कथ्यमानस्त्वयाऽनघ}
{हव्यकव्यभुजो विष्णुरुदक्पूर्वे महोदधौ}


\twolineshloka
{यच्च तत्कथितं पूर्वं त्वया हयशिरो महत्}
{तच्च दृष्टं भगवता ब्रह्मणा परमेष्ठिना}


\twolineshloka
{किं तदुत्पादितं पूर्वं हरिणा लोकधारिणा}
{रूपं प्रभावं महतामपूर्वं धीमतांवर}


\twolineshloka
{दृष्ट्वा हि विवुधश्रेष्ठमपूर्वममितौजसम्}
{तदश्वशिरसं पुण्यं ब्रह्मा किमकरोन्मुने}


\fourlineindentedshloka
{एतन्नः संशयं ब्रह्मन्पुराणं ब्रह्मसंभवम्}
{कथयस्वोत्तममते महापुरुषसंश्रितम्}
{पाविताः स्म त्वया ब्रह्मन्पुण्याः कथय ताः कथाः ॥सौतिरुवाच}
{}


\twolineshloka
{कथयिष्यामि ते सर्वं पुराणं वेदसंमितम्}
{जगौ यद्भगवान्व्यासो राज्ञः पारिक्षितस्य वै}


\threelineshloka
{श्रुत्वाऽश्वशिरसो मूर्ति देवस्य हरिमेधसः}
{उत्पन्नसंशयो राजा एतदेवमचोदयत् ॥जनमेजय उवाच}
{}


\threelineshloka
{यत्तद्दर्शितवान्ब्राह्म देवं हयशिरोधरम्}
{किमर्थं तत्समभवद्वपुर्देवोपकल्पितम् ॥वैशंपायन उवाच}
{}


\twolineshloka
{यत्किंचिदिह लोके वै देहबद्धं विशांपते}
{सर्वं पञ्चभिराविष्टं भूतैरीश्वरबुद्धिजैः}


\twolineshloka
{ईश्वरो हि जगत्स्रष्टा प्रभुर्नारायणो विराट्}
{भूतान्तरात्मा वरदः सगुणो निर्गुणोपि च}


% Check verse!
भूतप्रलयमव्यक्तं शृणुष्व नृपसत्तम
\twolineshloka
{धरण्यामथ लीनायामप्सु चैकार्णवे पुरा}
{ज्योतिर्भूते जले चापि लीने ज्योतिषि चानिले}


\twolineshloka
{वायौ चाकाशसंलीने आकाशे च मनोनुगे}
{व्यक्ते मनसि संलीने व्यक्ते चाव्यक्ततां गते}


\twolineshloka
{अव्यक्ते पुरुषं याते पुंसि सर्वगतेऽपि च}
{तम एवाभवत्सर्वं न प्राज्ञायत किंचन}


\twolineshloka
{तमसो ब्रह्मसंभूतं तमोमूलमृतात्मकम्}
{तद्विश्वभावसंज्ञान्तं पौरुषीं तनुमाश्रितम्}


\twolineshloka
{सोऽनिरुद्ध इति प्रोक्तस्तत्प्रधानं प्रचक्षते}
{तदव्यक्तमिति ज्ञेयं त्रिगुणं नृपसत्तम}


\twolineshloka
{विद्यासहायवान्देवो विष्वक्सेनो हरिः प्रभुः}
{`आदिकर्ता स भूतानामप्रमेयो हरिः प्रभुः}


\twolineshloka
{अप्स्वेव शयनं चक्रे निद्रायोगमुपागतः}
{जगतश्चिन्तयन्सृष्टिं चित्रां बहुगुणोद्भवाम्}


\threelineshloka
{तस्य चिन्तयतः सृष्टिं महानात्मगुणः स्मृतः}
{अहंकारस्ततो जातो ब्रह्मा शुभचतुर्मुखः}
{हिरण्यगर्भो भगवान्सर्वलोकपितामहः}


\twolineshloka
{पद्मेऽनिरुद्धात्संभूतस्तदा पद्मनिभेक्षणः}
{सहस्रपत्रे द्युतिमानुपविष्टः सनातनः}


\twolineshloka
{ददृशेऽद्भुतसंकाशो लोकानाप्याययन्प्रभुः}
{सत्वस्थः परमेष्ठी स ततो भूतगणान्सृजन्}


\twolineshloka
{पूर्वमेव च पद्मस्य पत्रे सूर्यांशुसप्रभे}
{नारायणकृतौ बिन्दू अपामास्तां गुणोत्तरौ}


\twolineshloka
{तावपश्यत्स भगवाननादिनिधनोऽच्युतः}
{एकस्तत्राभवद्विन्दुर्मध्वाभो रुचिरप्रभः}


\twolineshloka
{स तामसो मधुर्जातस्तदा नारायणाज्ञया}
{कठिनस्त्वपरो विन्दुः कैटभो राजसस्तु सः}


\twolineshloka
{तावभ्यधातवां श्रेष्ठौ तमोरजगुणान्वितौ}
{बलवन्तौ गदाहस्तौ पद्मनालानुसारिणौ}


\twolineshloka
{ददृशातेऽरविन्दस्थं ब्रह्माणममितप्रभवम्}
{सृजन्तं प्रथमं वेदांश्चतुरश्चारुविग्रहान्}


\twolineshloka
{ततो विग्रहवन्तस्तान्वेदान्दृष्ट्वाऽसुरोत्तमौ}
{सहसा जगृहतुर्वेदान्ब्रह्मणः पश्यतस्तदा}


\twolineshloka
{अथ तौ दानवश्रेष्ठौ वेदान्गृह्य सनातनान्}
{रसां विविशतुस्तूर्णमुदक्पूर्वे महोदधौ}


\threelineshloka
{ततो हृतेषु देवेषु ब्रह्मा कश्मलमाविशत्}
{ततो वचनमीशानं प्राह वेदैर्विनाकृतः ॥ब्रह्मोवाच}
{}


\twolineshloka
{वेदा मे परमं चक्षुर्वेदा मे परमं बलम्}
{वेदा मे परमं धाम वेदा मे ब्रह्म चोत्तरम्}


\twolineshloka
{मम वेदा हृताः सर्वे दानवाभ्यां बलादितः}
{अन्धकारा हि मे लोका जाता वेदैर्विना कृताः}


\twolineshloka
{वेदानृते हि किं कुर्या लोकानां सृष्टिमुत्तमाम्}
{अहो बत महद्दुःखं वेदनाशनजं मम}


\twolineshloka
{प्राप्तं दुनोति हृदयं तीव्रं शोकपरायणम्}
{को हि शोकार्णवे मग्नं मामितोऽद्य समुद्धरेत्}


\twolineshloka
{वदांस्तांश्चानयेन्नष्टान्कस्य चाहं प्रियो भवे}
{इत्येवं भाषमाणस्य ब्रह्मणो नृपसत्तम}


\threelineshloka
{हरेः स्तोत्रार्थमुद्भूता बुद्धिर्बुद्धिमतां वर}
{ततो जगौ परं जप्यं साञ्जलिप्रग्रहः प्रभुः ॥ब्रह्मोवाच}
{}


\twolineshloka
{ॐ नमस्ते ब्रह्महृदय नमस्ते मम पूर्वज}
{लोकाद्यभुवनश्रेष्ठ साङ्ख्ययोगनिधे प्रभो}


\threelineshloka
{व्यक्ताव्यक्तकराचिन्त्य क्षेमं पन्थानमास्थितः}
{विश्वभुक्सर्वभूतानामन्तरात्मन्नयोनिज}
{अहं प्रसादजस्तुभ्यं लोकधाम स्वयंभुवः}


\twolineshloka
{त्वत्तो मे मानसं जन्म प्रथमं द्विजपूजितम्}
{चाक्षुषं वै द्वितीयं मे जन्म चासीत्पुरातनम्}


\twolineshloka
{त्वत्प्रसादात्तु मे जन्म तृतीयं वाचिकं महत्}
{त्वत्तः श्रवणजं चापि चतुर्थं जन्म मे विभो}


\twolineshloka
{नासत्यं चापि मे जन्म त्वत्तः पञ्चममुच्यते}
{अण्डजं चापि मे जन्म त्वत्तः षष्ठं विनिर्मितं}


\twolineshloka
{इदं च सप्तमं जन्म पद्मजन्मेति वै प्रभो}
{सर्गेसर्गे ह्यहं पुत्रस्तव त्रिगुणवर्जित}


\twolineshloka
{प्रथमः पुण्डरीकाक्षः प्रधानगुणकल्पितः}
{त्वमीश्वरः स्वभावश्च भूतानां त्वं प्रभावन}


\threelineshloka
{त्वया विनिर्मितोऽहं वै वेदचक्षुर्वयोतिग}
{ते मे वेदा हृताश्चक्षुरन्धो जातोस्मि जागृहि}
{ददस्व चक्षूंषि मम प्रियोऽहं ते प्रियोसि मे}


\twolineshloka
{एवं स्तुतः स भगवान्पुरुषः सर्वतोमुखः}
{जहौ निद्रामथ तदा वेदकार्यार्थमुह्यतः}


\threelineshloka
{ऐश्वर्येण प्रयोगेण द्वितीयां तनुमास्थितः}
{सुनासिकेन कायेन भूत्वा चन्द्रप्रभस्तदा}
{कृत्वा हयशिरः शुभ्रं वेदानामालयं प्रभुः}


\twolineshloka
{तस्य मूर्धा समभवद्द्यौः सनक्षत्रतारकाः}
{केशाश्चास्याभवन्दीर्घा रवेरंशुसमप्रभाः}


\twolineshloka
{कर्णावाकाशपाताले ललाटं भूतधारिणी}
{गङ्गासरस्वती पुण्ये भ्रुवावास्तां महाद्युती}


\twolineshloka
{चक्षुषी सोमसूर्यौं ते नासा संध्या पुनः स्मृता}
{ॐकारस्त्वथ संस्कारो विद्युज्जिह्वा च निर्मिता}


\threelineshloka
{दन्ताश्च पितरो राजन्सोमपा इति विश्रुताः}
{गोलोको ब्रह्मलोकश्च ओष्ठावास्तां महात्मनः}
{ग्रीवा चास्याभवद्राजन्कालरात्रिर्गुणोत्तरा}


\twolineshloka
{एतद्धयशिरः कृत्वा नानामूर्तिभिरावृतम्}
{अन्तर्दधौ स विश्वेशो विवेश च रसां प्रभुः}


\twolineshloka
{रसां पुनः प्रविष्टश्च योगं परममास्थितः}
{शैक्ष्यं स्वरं समास्थाय उद्गीतं प्रासृजत्स्वरम्}


\twolineshloka
{सस्वरः सानुनादी च सर्वशः स्निग्ध एव च}
{बभूवान्तर्जलगतः सर्वभूतगुणोदितः}


\twolineshloka
{ततस्तावसुरौ कृत्वा वेदान्समयबन्धनान्}
{रसातले विनिक्षिप्य यतः शब्दस्ततो द्रुतौ}


\twolineshloka
{एतस्मिन्नन्तरे राजन्देवो हयशिरोधरः}
{जग्राह वेदानखिलान्रसालगतान्हरिः}


% Check verse!
प्रादाच्च ब्रह्मणे भूयस्ततः स्वां प्रकृतिं गतः
\twolineshloka
{स्थापयित्वा हयशिरा उदक्पूर्वे महोदधौ}
{वेदानामालयश्चापि बभूवाश्वरिरास्ततः}


\twolineshloka
{अथ किंचिदपश्यन्तौ दानवौ मधुकैटभौ}
{यत्र देवा विनिक्षिप्तास्तत्स्थानं शून्यमेव च}


\twolineshloka
{तत उत्तममास्थाय वेगं बलवतां वरौ}
{पुनरुत्तस्थतुः शीघ्रं रसानामालयात्तदा}


\threelineshloka
{ददृशाते च पुरुषं तमेवादिकरं प्रभुम्}
{श्वेतं चन्द्रविशुद्धाभमनिरुद्धतनौ स्थितम्}
{टभूयोप्यमितविक्रान्तं निद्रायोगमुपागतम्}


\twolineshloka
{आत्मप्रमाणरचिते अपामुपरि कल्पिते}
{शयने नागभोगाढ्ये ज्वालामालासमावृते}


\twolineshloka
{निष्कल्मषेण सत्वेन संपन्नं रुचिरप्रभम्}
{तं दृष्ट्वा दानवेन्द्रौ तौ महाहासममुञ्चताम्}


\twolineshloka
{ऊचतुश्च समाविष्टौ रजसा तमसा च तौ}
{अयं स पुरुषः श्वेतः शेते निद्रामुपागतः}


\threelineshloka
{अनेन नूनं वेदानां कृतमाहरणं रसात्}
{कस्यैष कोनु खल्वेष किंच स्वपिति भोगवान्}
{इच्युच्चारितवाक्यौ तौ बोधयामासतुर्हरिम्}


\twolineshloka
{युद्धार्थिनौ हि विज्ञाय विबुद्धः पुरुषोत्तमः}
{निरीक्ष्य चासुरेन्द्रौ तौ ततो युद्धे मनोदधे}


% Check verse!
अथ युद्धं समभवत्तयोर्नारायणस्य वै
\twolineshloka
{रजस्तमोविष्टतनू तावुभौ मधुकैटभौ}
{ब्रह्मणोपचितिं कुर्वञ्जधान मधुसूदनः}


\twolineshloka
{ततस्तयोर्वधेनाशु वेदापहरणेन च}
{शोकापनयनं चक्रे ब्रह्मणः पुरुषोत्तमः}


\twolineshloka
{ततः परिवृतो ब्रह्मा हरिणा वेदसत्कृतः}
{निर्ममे स तदा लोकान्कृत्स्नान्स्थावरजङ्गमान्}


\twolineshloka
{दत्त्वा पितामहायाग्र्यां मतिं लोकविसर्गिकीम्}
{तत्रैवान्तर्दधे देवो यत एवागतो हरिः}


\twolineshloka
{तौ दानवौ हरिर्हत्वा कृत्वा हयशिरस्तनुम्}
{पुनः प्रवृत्तिधर्मार्थं तामेव विदधे तनुम्}


\twolineshloka
{एवमेष महाभागो बभूवाश्वशिरा हरिः}
{पौराणमेतत्प्रख्यातं रूपं वरदमैश्वरम्}


\twolineshloka
{यो ह्येतद्ब्राह्मणो नित्यं शृणुयाद्धारयीत वा}
{न तस्याध्ययनं नाशमुपगच्छेत्कदाचन}


\twolineshloka
{आराध्य तपसोग्रेण देवं हयशिरोधरम्}
{पाञ्चालेन क्रमः प्राप्तो रामेण पथि देशिते}


\twolineshloka
{एतद्धयशिरो राजन्नाख्यानं तव कीर्तितम्}
{पुराणं वेदसमितं यन्मां त्वं परिपृच्छसि}


\twolineshloka
{यांयामिच्छेत्तनुं देवः कर्तुं कार्यविधौ क्वचित्}
{तातां कुर्याद्विकुर्वाणः स्वयमात्मानमात्मना}


\twolineshloka
{एष वेदनिधिः श्रीमानेष वै तपसोनिधिः}
{एष योगश्च साङ्ख्यं च ब्रह्म चाग्र्यं हविर्विभुः}


\twolineshloka
{नारायणपरा वेदा याज्ञा नारायणात्मकाः}
{तपो नारायणपरं नारायणपरा गतिः}


\twolineshloka
{नारायणपरं सत्यमृतं नारायणात्मकम्}
{नारायणपरो धर्मः पुनरावृत्तिदुर्लभः}


\twolineshloka
{प्रवृत्तिलक्षणश्चैव धर्मो नारायणात्मकः}
{नारायणात्मको गन्धो भूमौ श्रेष्ठतमः स्मृतः}


\twolineshloka
{अपां चापि गुणा राजन्रसा नारायणात्मकाः}
{ज्योतिषां च परं रूपं स्मृतं नारायणात्मकम्}


\twolineshloka
{नारायणात्मकश्चापि स्पर्शो वायुगुणः स्मृतः}
{नारायणात्मकश्चैव शब्द आकाशसंभवः}


\twolineshloka
{मनश्चापि ततो भूतमव्यक्तगुणलक्षणम्}
{नारायणपरं कालो ज्योतिषामयनं च यत्}


\twolineshloka
{नारायणपरा कीर्तिः श्रीश्च लक्ष्णीश्च देवताः}
{नारायणपरं साङ्ख्यं योगो नारायणात्मकः}


\twolineshloka
{कारणं पुरुषो ह्येषां प्रधानं चापि कारणम्}
{स्वभावश्चैव कर्माणि दैवं येषां च कारणम्}


\twolineshloka
{अधिष्ठानं तथा कर्ता करणं च पृथग्विधम्}
{विविधा च तथा चेष्टा दैवं चैवात्र पञ्चमम्}


\twolineshloka
{पञ्चकारणसंख्यातो निष्ठा सर्वत्र वै हरिः}
{तत्त्वं विज्ञासमानानां हेतुभिः सर्वतोमुखैः}


\twolineshloka
{तत्त्वमेको महायोगी हनिर्नारायणः प्रभुः}
{ब्रह्मादीनां सलोकानामृषीणां च महात्मनाम्}


\twolineshloka
{साङ्ख्यानां योगिनां चापि यतीनामात्मवेदिनाम्}
{मनीषितं विजानाति केशवो न तु तस्य ते}


\twolineshloka
{ये केचित्सर्वलोकेषु दैवं पित्र्यं च कुर्वते}
{दानानि च प्रयच्छन्ति तप्यन्ते च तपो महत्}


\twolineshloka
{सर्वेषामाश्रयो विष्णुरैश्वरं विधिमास्थितः}
{सर्वभूतकृतावासो वासुदेवेति चोच्यते}


\twolineshloka
{अयं हि नित्यः परमो महर्षिर्महाविभूतिर्गुणवान्गुणाख्यः}
{गुणैश्च संयोगमुपैति शीघ्रंकालो यथर्तावृतुसंप्रयुक्तः}


\threelineshloka
{नैवास्य विन्दन्ति गतिं महात्मनोन चागतिं कश्चिदिहानुपश्यति}
{ज्ञानात्मकाः संयमिनो महर्षयः}
{पश्यन्ति नित्यं पुरुषं गुणाधिकम्}


\chapter{अध्यायः ३५८}
\twolineshloka
{जनमेजय उवाच}
{}


\twolineshloka
{अहो ह्येकान्तिनः सर्वान्प्रीणाति भगवान्हरिः}
{विधिप्रयुक्तां पूजां च गृह्णाति शिरसा स्वयम्}


\twolineshloka
{ये तु दग्धेन्धना लोके पुण्यपापविवर्जिताः}
{तेषां च या हि निर्दिष्टा पारम्पर्यागता गतिः}


\twolineshloka
{चतुर्थ्यां चैव ते गत्यां गच्छन्ति पुरुषोत्तमम्}
{एकान्तिनस्तु पुरुषा गच्छन्ति परमं पदम्}


\twolineshloka
{नूनमेकान्तधर्मोऽयं श्रेष्ठो नारायणप्रियः}
{अगत्वा गतयतिस्रो यद्गच्छत्यव्ययं हरिम्}


\twolineshloka
{सहोपनिषदान्वेदान्ये विप्राः सम्यगास्थिताः}
{पठन्ति विधिमास्थाय ये चापि यतिधर्मिणः}


\twolineshloka
{तेभ्यो विशिष्टां जानामि गतिमेकान्तिनां नृणाम्}
{केनैष धर्मः कथितो देवेन ऋषिणाऽपि वा}


\threelineshloka
{एकान्तिनां च का चर्या कदा चोत्पादिता विभो}
{एतन्मे संशयं छिन्धि परं कौतूहलं हि मे ॥वैशंपायन उवाच}
{}


\twolineshloka
{समुपोढेष्वनीकेषु कुरुपाण्डवयोर्मृधे}
{अर्जुने विमनस्के च गीता भगवता स्वयम्}


\twolineshloka
{आगतिश्च गतिश्चैव पूर्वं ते कथिता मया}
{गहनो ह्येष धर्मो वै दुर्विज्ञेयोऽकृतात्मभिः}


\twolineshloka
{संमितः सामवेदेन पुरैवादियुगे कृतः}
{धार्यते स्वयमीशेन राजन्नारायणेन ह}


\twolineshloka
{एतदर्थं महाराज पृष्टः पार्थेन नारदः}
{ऋषिमध्ये महाभागः शृण्वतोः कृष्णभीष्मयोः}


\twolineshloka
{गुरुणा च मयाऽप्येव कथितो नृपसत्तम}
{यथा तत्कथितं तत्र नारदेन तथा शृणु}


\twolineshloka
{यदाऽऽसीन्मानजं जन्म नारायणमुखोद्गतम्}
{ब्रह्मणः पृथिवीपाल तदा नारायणः स्वयम्}


\twolineshloka
{तेन धर्मेण कृतवान्दैवं पित्र्यं च भारत}
{फेनपा ऋषयश्चैव तं धर्मं प्रतिपेदिरे}


\twolineshloka
{वैखानसाः फेनपेभ्यो धर्मं तं प्रतिपदिरे}
{वैखानसेभ्यः सोपस्तु ततः सोऽन्तर्दधे पुनः}


\twolineshloka
{यदाऽऽसीच्चाक्षुषं जन्म द्वितीयं ब्रह्मणो नृप}
{यदा पितामहेनैव सोमाद्धर्मः परिश्रुतः}


\twolineshloka
{नारायणात्मको राजन्रुद्राय प्रददौ च तम्}
{ततो योगस्थितो रुद्रः पुरा कृतयुगे नृप}


\twolineshloka
{वालखिल्यानृषीन्सर्वान्धर्ममेनमपाठयत्}
{अन्तर्दधे ततो भूयस्तस्य देवस्य मायया}


\twolineshloka
{तृतीयं ब्रह्मणो जन्म यदासीद्वाचिकं महत्}
{तत्रैष धर्मः संभूतः स्वयं नारायणान्नृप}


\twolineshloka
{सुपर्णो नाम तमृषिः प्राप्तवान्पुरुषोत्तमात्}
{तपसा वै सुतप्तेन दमेन नियमेन च}


\twolineshloka
{त्रिः परिक्रान्तवानेतत्सुपर्णो धर्मसुत्तमम्}
{यस्मात्तस्माद्व्रतं ह्येतत्रिसौपर्णमिहोच्येत}


\twolineshloka
{ऋग्वेदपाठपठितं व्रतमेतद्धि दुश्चरम्}
{सुपर्णाच्चाप्यधिगतो धर्म एष सनातनः}


\twolineshloka
{वायुना द्विपदश्रिष्ठे प्रथितो जगदायुषा}
{वायोः सकाशात्प्राप्तश्च ऋषिभिर्विघसाशिभिः}


\twolineshloka
{तेभ्यो महोदधिश्चैव प्राप्तवान्धर्ममुत्तमम्}
{अन्तर्दधे ततो भूयो नारायणसमाहृतः}


\twolineshloka
{यदा भूयः श्रवणजा सृष्टिरासीन्महात्मनः}
{ब्रह्मणः पुरुषव्याघ्र तत्र कीर्तयतः शृणु}


\twolineshloka
{जगत्स्रष्टुमना देवो हरिर्नारायणः स्वयम्}
{चिन्तयामास पुरुषं जगत्सर्गकरं प्रभुम्}


\twolineshloka
{अथ चिन्तयतस्तस्य कर्णाभ्यां पुरुषः स्मृतः}
{प्रजासर्गकरो ब्रह्मा तमुवाच जगत्पतिः}


\twolineshloka
{सृज प्रजाः पुत्र सर्वा मुखतः पादतस्तथा}
{श्रेयस्तव विधास्यामि बलं तेजश्च सुव्रत}


\twolineshloka
{धर्मं च मत्तो गृह्णीष्व सात्वतं नाम नामतः}
{तेन सृष्टं कृतयुगं स्थापयस्व यथाविधि}


\twolineshloka
{ततो ब्रह्मा नमश्चक्रे देवाय हरिमेधसे}
{धर्मं चाग्र्यं स जग्राह सरहस्यं ससंग्रहम्}


\twolineshloka
{आरण्यकेन सहितं नारायणमुखोद्गतम्}
{उपदिश्य ततो धर्मं ब्रह्मणेऽमिततेजसे}


\twolineshloka
{तं कार्तयुगधर्माणं निराशीः कर्मसंज्ञितम्}
{जगाम तमसः पारं यत्राव्यक्तं व्यवस्थितम्}


\twolineshloka
{ततोऽथ वरदो देवो ब्रह्मा लोकपितामहः}
{असृजत्स ततो लोकान्कृत्स्नान्स्थावरजङ्गमान्}


\twolineshloka
{ततः प्रावर्तत तदा आदौ कृतयुगं शुभम्}
{ततो हि सात्वतो धर्मो व्याप्य लोकानवस्थितः}


\twolineshloka
{तेनैवाद्येन धर्मेण ब्रह्मा लोकविसर्गकृत्}
{पूजयामास देवेशं हरिं नारायणं प्रभुम्}


\twolineshloka
{धर्मप्रतिष्ठाहेतोश्च मनुं स्वारोचिषं ततः}
{अध्यापयामास तदा लोकानां हितकाम्यया}


\twolineshloka
{ततः स्वारोचिषः पुत्रं स्वयं शङ्खपदं नृप}
{अध्यापयत्पुराऽव्यग्रः सर्वलोकपतिर्विभुः}


\threelineshloka
{ततः शङ्खपदश्चापि पुत्रमात्मजमौरसम्}
{दिशापालं सुधर्माणमध्यापयत भारत}
{सोऽन्तर्दधे ततो भूयः प्राप्ते त्रेतायुगे पुनः}


\twolineshloka
{नासत्ये जन्मनि पुरा ब्रह्मणः पार्थिवोत्तम}
{धर्ममेतं स्वयं देवो हरिर्नारायणः प्रभुः}


\twolineshloka
{तज्जगादारविन्दाक्षो ब्रह्मणः पश्यतस्तदा}
{सनत्कुमारो भगवांस्ततः प्राधीतवान्नृप}


\twolineshloka
{सनत्कुमारादपि च वीरणो वै प्रजापतिः}
{कृतादौ कुरुशार्दूल धर्ममेतदधीतवान्}


\twolineshloka
{वीरणश्चाप्यधीत्यैनं रैभ्याय मुनये ददौ}
{रैभ्यः पुत्राय शुद्धाय सुव्रताय सुमेधसे}


\twolineshloka
{कुक्षिपालाय च ददौ विशालाय च धर्मिणे}
{ततोऽप्यन्तर्दधे भूयो नारायणमुखोद्गतः}


\twolineshloka
{अण्डजे जन्मनि पुनर्ब्रह्मणे हरियोनये}
{एष धर्मः समुद्भूतो नारायणमुखात्पुनः}


\twolineshloka
{गृहीतो ब्रह्मणा राजन्प्रयुक्तश्च यथाविधि}
{अध्यापिताश्च मुनयो नाम्ना बर्हिपदो नृप}


\twolineshloka
{बर्हिषद्भ्यश्च संप्राप्तः सामवेदान्तगं द्विजम्}
{ज्येष्ठं नामाभिविख्यातं ज्येष्ठसामव्रतो हरिः}


\twolineshloka
{ज्येष्ठाच्चाप्यनुसंक्रान्तो राजानमविकम्पनम्}
{अन्तर्दधे ततो राजन्नेष दर्मः प्रभो हरेः}


\twolineshloka
{यदिदं सप्तमं जन्म पद्मजं ब्रह्मणो नृप}
{तत्रैष धर्मः कथितः स्वयं नारायणेन ह}


\twolineshloka
{पितामहाय शुद्धाय युगादौ लोकधारिणे}
{पितामहश्च दक्षाय धर्ममेतं पुरा ददौ}


\twolineshloka
{ततो ज्येष्ठे तु दौहित्रे प्रादाद्दक्षो नृपोत्तम}
{आदित्ये सवितुर्ज्येष्ठे विवस्वाञ्जगृहे ततः}


\twolineshloka
{त्रेतायुगादौ च ततो विवस्वान्ममवे ददौ}
{मनुश्च लोकभूत्यर्थं सुतायेक्ष्वाकवे ददौ}


\twolineshloka
{इक्ष्वाकुणा च कथितो व्याप्य लोकानवस्थितः}
{गमिष्यति क्षयान्ते च पुनर्नारायणं नृप}


\twolineshloka
{यतीनां चापि यो धर्मः स ते पूर्वं नृपोत्तम}
{कथितो हरिगीतासु समासविधिकल्पितः}


\twolineshloka
{नारदेन सुसंप्राप्तः सरहस्यः ससंग्रहः}
{एष धर्मो जगन्नाथात्साक्षान्नारायणान्नृप}


\twolineshloka
{एवमेव महान्धर्मे आद्यो राजन्सनातनः}
{दुर्विज्ञेयो दुष्करश्च सात्वतैर्धार्यते सदा}


\twolineshloka
{धर्मज्ञानेन चैतेन सुप्रयुक्तेन कर्मणा}
{अहिंसाधर्मयुक्तेन प्रीयते हरिरीश्वरः}


\twolineshloka
{एकव्यूहविभागो वा क्वचिद्द्विव्यूहसंज्ञितः}
{त्रिव्यूहश्चापि संख्यातश्चतुर्व्यूहश्च दृश्यते}


\twolineshloka
{हरिरेव हि क्षेत्रज्ञो निर्ममो निष्कलस्तथा}
{जीवश्च सर्वभूतेषु पञ्चभूतगुणातिगः}


\twolineshloka
{मनश्च प्रथितं राजन्पञ्चन्द्रियसमीरणम्}
{एष लोकनिधिः श्रीमानेषु लोकविसर्गकृत्}


\twolineshloka
{अकर्ता चैव कर्ता च कार्यं कारणमेव च}
{यथेच्छति तथा राजन्क्रीडते पुरुषोऽव्ययः}


\twolineshloka
{एष एकान्तिधर्मस्ते कीर्तितो नृपसत्तम}
{मया गुरुप्रसादेन दुर्विज्ञेयोऽकृतात्मभिः}


\twolineshloka
{एकान्तिनो हि पुरुषा दुर्लभा बहवो नृप}
{यद्येकान्तिभिराकीर्णं जगत्स्यात्कुरुनन्दनः}


\twolineshloka
{अहिंसकैरात्मविद्भिः सर्वभूतहिते रतैः}
{भवेत्कृतयुगप्राप्तिराशीः कर्मविवर्जिता}


\twolineshloka
{एवं स भगवान्व्यासो गुरुर्मम विशांपते}
{कथयामास धर्मज्ञो धर्मराजे द्विजोत्तमः}


\twolineshloka
{ऋषीणां संनिधौ राजञ्शृण्वतोः कृष्णभीष्मयोः}
{तस्याप्यकथयत्पूर्वं नारदः सुमहातपाः}


\fourlineindentedshloka
{देवं परमकं ब्रह्म श्वेतं चन्द्राभमच्युतम्}
{यत्र चैकान्तिनो यान्ति नारायणपरायणाः}
{`तदेव परमं स्थानं मुक्तानां केवलं भवेत् ॥' जनमेजय उवाच}
{}


\threelineshloka
{एवं बहुविधं धर्मं प्रविबुद्धैर्निषेवितम्}
{न कुर्वन्ति कथं विप्रा अन्ये नानाव्रते स्थिताः ॥वैशंपायन उवाच}
{}


\twolineshloka
{तिस्रः प्रकृतयो राजन्देहबन्धेषु निर्मिताः}
{सात्विकी राजसी चैव तामसी चैव भारत}


\twolineshloka
{देहबन्धेषु पुरुषः श्रेष्ठः कुरुकुलोद्वह}
{सात्विकः पुरुषव्याघ्र भवेन्मोक्षाय निश्चितः}


\twolineshloka
{अत्रापि स विजानाति पुरुषं ब्रह्मवित्तमम्}
{नारायणां परं मोक्षे ततो वै सात्विकः स्मृतः}


\twolineshloka
{मनीषितं च प्राप्नोति चिन्तयन्पुरुषोत्तमम्}
{एकान्तभक्तः सततं नारायणपरायणः}


\twolineshloka
{मनीषिणो हि ये केचिद्यतयो मोक्षधर्मिणः}
{तेषां विच्छिन्नतृष्णानां योगक्षेमवहो हरिः}


\twolineshloka
{जायमानं हि पुरुषं यं पश्येन्मधुसूदनः}
{सात्विकस्तु स विज्ञेयो भवेन्मोक्षे च निश्चितः}


\twolineshloka
{साङ्ख्ययोगेन तुल्यो हि धर्म एकान्तिसेवितः}
{नारायणात्मके मोक्षे ततो यान्ति परां गतिं}


\twolineshloka
{नारायणेन दृष्टस्तचु प्रतिबुद्धो भवेत्पुमान्}
{एवमात्मेच्छया राजन्प्रतिबुद्धो न जायते}


\threelineshloka
{राजसी तामसी चैव व्यामिश्रे प्रकृती स्मृते}
{तदात्मकं हि पुरुषं जायमानं विशांपते}
{प्रवृत्तिलक्षणैर्युक्तं नावेक्षति हरिः स्वयम्}


\twolineshloka
{पश्यत्येनं जायमानं ब्रह्मा लोकपितामहः}
{रजसा तपसा चैव मानसं समभिप्लुतम्}


\threelineshloka
{कामं देवाश्च ऋषयः सत्वस्था नृपसत्तम}
{हीनाः सत्वेन सूक्ष्मेण ततो वैकारिकाः स्मृताः ॥जनमेजय उवाच}
{}


\threelineshloka
{कथं वैकारिको गच्छेत्पुरुषः पुरुषोत्तमम्}
{वद सर्वं यथादृष्टं प्रवृत्तिं च यथाक्रमम् ॥वैशंपायन उवाच}
{}


\twolineshloka
{सुसूक्ष्मं तत्त्वसंयुक्तं संयुक्तं त्रिभिरक्षरैः}
{पुरुषः पुरुषं गच्छेन्निष्क्रियं पञ्चविंशकम्}


\twolineshloka
{एवमेकं साङ्ख्ययोगं वेदारण्यकमेव च}
{परस्पराङ्गान्येतानि पाञ्चरात्रं च कथ्यते}


% Check verse!
एष एकान्तिनां धर्मो नारायणपरात्मकः
\twolineshloka
{यथा समुद्रात्प्रसृता जलौघास्तमेव राजन्पुनराविशन्ति}
{इमे तथा ज्ञानमहाजलौघानारायणं वै पुनराविशन्ति}


\twolineshloka
{एष ते कथितो धर्मः सात्वतो यदुबान्धव}
{कुरुष्वैनं यथान्यायं यदि शक्तोसि भारत}


\twolineshloka
{एवं हि स महाभागो नारदो गुरवे मम}
{श्वेतानां यतिनां चाह एकान्तगतिमख्याम्}


\twolineshloka
{व्यासश्चाकथयत्प्रीत्या धर्मपुत्राय धीमते}
{स एवायं मया तुभ्यमाख्यातः प्रसृतो गुरोः}


\twolineshloka
{इत्थं हि दुश्चरो धर्म एष पार्थिवसत्तम}
{यथैव त्वं तथैवान्ये न भजन्ति च मोहिताः}


\twolineshloka
{कृष्ण एव हि लोकानां भावनो मोहनस्तथा}
{संहारकारकश्चैव कारणं च विशांपते}


\chapter{अध्यायः ३५९}
\twolineshloka
{जनमेजय उवाच}
{}


\twolineshloka
{साख्यं योगः पाञ्चरात्रं वेदारण्यकमेव च}
{ज्ञानान्येतानि ब्रह्मर्षे लोकेषु प्रचरन्ति ह}


\twolineshloka
{किमेतान्येकनिष्ठानिं पृथङ्निष्ठानि वा मुने}
{प्रब्रूहि वै मया पृष्टः प्रवृत्तिं च यथाक्रमम्}


\threelineshloka
{`कथं वैकारिको गच्छेत्पुरुषः पुरुषोत्तमम्}
{वदस्व त्वं मया पृष्टः प्रवृत्तिं च यथाक्रमम् ॥'वैशंपायन उवाच}
{}


\twolineshloka
{जज्ञे बहुज्ञं परमत्युदारंयं द्वीपमध्ये सुतमात्मवन्तम्}
{पराशरात्सत्यवती महर्षितस्मै नमोऽज्ञानतमोनुदाय}


\twolineshloka
{पितामहाद्यं प्रवदन्ति षष्ठंमहर्षिमार्षेयविभूतियुक्तम्}
{नारायणस्यांशजमेकपुत्रंद्वैपायनं वेदमहानिधानम्}


\threelineshloka
{तमादिकालेषु महाविभूतिर्नारायणो ब्रह्म महानिधानम्}
{ससर्ज पुत्रार्थमुदारतेजाव्यासं महात्मानमजं पुराणम् ॥जनमेजाय उवाच}
{}


\twolineshloka
{त्वयैव कथितः पूर्वं संभवो द्विजसत्तम}
{वसिष्ठस्य सुतः शक्तिः शक्तिपुत्रः पराशरः}


\twolineshloka
{पराशरस्य दायादः कृष्णद्वैपायनो मुनिः}
{भूयो नारायणसुतं त्वमेवैनं प्रभाषसे}


\threelineshloka
{किमतः पूर्वकं जन्म व्यासस्यामिततेजसः}
{कथयस्वोत्तममते जन्म नारायणोद्भवम् ॥वैशंपायन उवाच}
{}


\twolineshloka
{वेदार्थवेत्तुव्यासस्य धर्मिष्ठस्य तपोनिधेः}
{गुरोर्मे ज्ञाननिष्ठस्य हिमवत्पाद आसतः}


\twolineshloka
{कृत्वा भारतमाख्यानं तपः श्रान्तस्य धीमतः}
{शुश्रूषां तत्परा राजन्कृतवन्तो वयं तदा}


\twolineshloka
{सुमन्तुर्जैमिनिश्चैव पैलश्च सुदृढव्रतः}
{अहं चतुर्थः शिष्यो वै शुको व्यासात्मजस्तथा}


\twolineshloka
{एभिः परिवृतो व्यासः शिष्यैः पञ्चभिरुत्तमैः}
{शुशुभे हिमवत्पादे भूतैर्भूतपतिर्यथा}


\twolineshloka
{वेदानावर्तयन्साङ्गान्भारतार्थांश्च सर्वशः}
{तमेकमनसं दान्तं युक्ता वयमुपास्महे}


\twolineshloka
{कथान्तरेऽथकस्मिंश्चित्पृष्टोऽस्माभिर्द्विजोत्तमः}
{वेदार्थान्भारतार्थांश्च जन्म नारायणात्तथा}


\twolineshloka
{स पूर्वमुक्त्वा वेदार्थान्भारतार्थांश्च तत्त्ववित्}
{नारायणादिदं जन्म व्याहर्तुमुपचक्रमे}


\twolineshloka
{शृणुध्वमाख्यानवरमिदमार्षेयमुत्तमम्}
{आदिकालोद्भवं विप्रास्तपसाऽधिगतं मया}


\twolineshloka
{प्राप्ते प्रजाविसर्गे वै सप्तमे पद्मसंभवे}
{नारायणो महायोगी शभाशुभविवर्जितः}


\twolineshloka
{ससृजे नाभितः पूर्वं ब्रह्माणममितप्रभः}
{ततः स प्रादुरभवदथैनं वाक्यमब्रवीत्}


\twolineshloka
{मम त्वं नाभितो जातः प्रजासर्गकरः प्रभुः}
{सृज प्रजास्त्वं विविधा ब्रह्मन्सजडपण्डिताः}


\twolineshloka
{स एवमुक्तो विमुखश्चिन्ताव्याकुलमानसः}
{प्रणम्य वरदं देवमुवाच हरिमीश्वरम्}


\twolineshloka
{का शक्तिर्मम देवेश प्रजाः स्रष्टुं नमोस्तु ते}
{अप्रज्ञावानहं देव विधत्स्व यदनन्तरम्}


\threelineshloka
{स एवमुक्तो भगवान्भूत्वाऽथान्तर्हितस्ततः}
{चिन्तयामास देवेशो बुद्धिं बुद्धिमतांवरः}
{स्वरूपिणी ततो बुद्धिरुपतस्थे हरिं प्रभुम्}


\twolineshloka
{योगेन चैनां निर्योगः स्वयं नियुयुजे तदा}
{स तामैश्वर्ययोगस्थां बुद्धिं गतिमतीं सतीम्}


\threelineshloka
{उवाच वचनं देवो बुद्धिं वै प्रभुरव्ययः}
{ब्रह्माणं प्रविशस्वेति लोकसृष्ट्यर्थसिद्धये}
{ततस्तमीश्वरादिष्टा बुद्धिः क्षिप्रं विवेश सा}


\twolineshloka
{अथैनं बुद्धिसंयुक्तं पुनः स ददृशे हरिः}
{भूयश्चैव वचः प्राह सृजेमा विविधाः प्रजाः}


\twolineshloka
{बाढमित्येव कृत्वाऽसौ यथाऽऽज्ञां शिरसा हरेः}
{एवमुक्त्वा स भगवांस्तत्रैवान्तरधायत}


\twolineshloka
{प्राप चैनं मुहूर्तेन स्वं स्थानं देवसंज्ञितम्}
{तां चैव प्रकृतिं प्राप्य एकीभावगतोऽभवत्}


\twolineshloka
{अथास्य बुद्धिरभवत्पुनरन्या तदा किल}
{सृष्टाः प्रजा इमाः सर्वा ब्रह्मणा परमेष्ठिना}


\twolineshloka
{दैत्यदानवगन्धर्वरक्षोगणसमाकुला}
{जाता हीयं वसुमती भाराक्रान्ता तपस्विनी}


\twolineshloka
{बहवो बलिनः पृथ्व्यां दैत्यदानवराक्षसाः}
{भविष्यन्ति तपोयुक्ता वरानप्राप्स्यन्ति चोत्तमान्}


\twolineshloka
{अवश्यमेव तैः सर्वैर्वरदानेन दर्पितैः}
{बाधितव्याः सुरगणा ऋषयश्च तपोधनाः}


\twolineshloka
{तत्र न्याय्यमिदं कर्तुं भारावतरणं मया}
{अथ नानासमुद्भूतैर्वसुधायां यथाक्रमम्}


\twolineshloka
{निग्रहेण च पापानां साधूनां प्रग्रहेण च}
{इदं तपस्विनी सत्या धारयिष्यति मेदिनी}


\twolineshloka
{मया ह्येषा हि ध्रियते पातालस्थेन भोगिना}
{तस्मात्पृथ्व्याः परित्राणं करिष्ये संभवं गतः}


\twolineshloka
{एवं स चिन्तयित्वा तु भगवान्मघधुसूदनः}
{रुपाण्यनेकान्यसृजत्प्रादुर्भावभवाय सः}


\twolineshloka
{वाराहं नारसिहं च वामनं मानुषं तथा}
{एभिर्मया निहन्तव्याः दुर्विनीताः सुरारयः}


\twolineshloka
{अथ भूयो जगत्स्रष्टा भोःशब्देनानुनादयन्}
{सरस्वतीमुच्चचार तत्र सारस्वतोऽभवत्}


\twolineshloka
{अपान्तरतमा नाम सुतो वाक्संभवः प्रभोः}
{भूतभव्यभविष्यज्ञः सत्यवादी दृढव्रतः}


\twolineshloka
{तमुवाच नतं मूर्ध्ना देवानामादिवरव्ययः}
{वेदाख्याने श्रुतिः कार्या त्वया मतिमतांवर}


\twolineshloka
{तस्मात्कुरु यथाज्ञप्तं ममैतद्वचनं मुने}
{तेन भिन्नास्तदा वेदा मनोः स्वायंभुवेन्तरे}


\twolineshloka
{ततस्तुतोष भगवान्हरिस्तेनास्य कर्मणा}
{तपसा च सुतप्तेन यमेन नियमेन च}


\twolineshloka
{मन्वन्तरेषु पुत्र त्वमेवं लोकप्रवर्तकः}
{भविष्यस्यचलो ब्रह्मन्नप्रधृष्यश्च नित्यशः}


\twolineshloka
{पुनस्तिष्ये च संप्राप्ते कुरवो नाम भारताः}
{भविष्यन्ति महात्मानो राजानः प्रथिता भुवि}


\twolineshloka
{तेषां त्वत्तः प्रसूतानां कुलभेदो भविष्यति}
{परस्परविनाशार्थं त्वामृते द्विजसत्तम}


\twolineshloka
{तत्राप्यनेकधा वेदान्भेत्स्यसे तपसाऽन्वितः}
{कृष्णे युगे च संप्राप्ते कृष्णवर्णो भविष्यसि}


\twolineshloka
{धर्माणां विविधानां च कर्ता ज्ञानकरस्तथा}
{भविष्यसि तपोयुक्तो न च रागाद्विमोक्ष्यसे}


\twolineshloka
{वीतरागश्च पुत्रस्ते परमात्मा भविष्यति}
{महेश्वरप्रसादेन नैतद्वचनमन्यथा}


\twolineshloka
{यं मानसं वै प्रवदन्ति विप्राःपितामहस्योत्तमबुद्धियुक्तम्}
{वसिष्ठमग्र्यं च तपोनिधानंयस्यातिसूर्यं व्यरिरिच्यते भाः}


\twolineshloka
{तस्यान्वपे चापि ततो महर्षिःपराशरो नाम महाप्रभावः}
{पिता स ते वेदनिधिर्वरिष्ठोमहातपा वै तपसो निवासः}


% Check verse!
कानीनगर्भः पितृकन्यकायांतस्मादृषेस्त्वं भविता च पुत्रः
\twolineshloka
{भूतभव्यभविष्याणां ज्ञानानां वेत्स्यसे गतिम्}
{ये ह्यतिक्रान्तकाः पूर्वं सहस्रसुगपर्ययाः}


\twolineshloka
{तांश्च सर्वान्मयोद्दिष्टान्द्रक्ष्यसे तपसाऽन्वितः}
{पुनर्द्रक्ष्यसि चानेकसहस्रयुगपर्ययान्}


\threelineshloka
{अनादिनिधनं लोके चक्रहस्तं च मां मुने}
{अनुध्यानान्मम मुने नैतद्वचनमन्यथा}
{भविष्यति महासत्व ख्यातिश्चाप्यतुला तव}


\threelineshloka
{* शनैश्चरः सूर्यपुत्रो भविष्यति मनुर्महान्}
{तस्मिन्मन्वन्तरे चैव मन्वादिगणपूर्वकः}
{त्वमेव भविता वत्स मत्प्रसादान्न संशयः}


\twolineshloka
{[यत्किंचिद्विद्यते लोके सर्वं तन्मद्विचेष्टितम्}
{अन्यो ह्यन्यं चिन्तयति स्वच्छन्दं विदधाम्यहम् ॥]}


\twolineshloka
{एवं सारस्वतमृषिमपांतरतमं तथा}
{युक्त्वा वचनमीशानः साधयस्वेत्यथाब्रवीत्}


\twolineshloka
{सोहं तस्य प्रसादेन देवस्य हरिमेधसः}
{अपांतरतमो नाम्ना ततो जातोऽऽज्ञया हरेः}


% Check verse!
पुनश्च जातो विख्यातो वसिष्ठकुलनन्दनः
\twolineshloka
{तदेतत्कथितं जन्म मया पूर्वकमात्ममः}
{नारायणप्रसादेन तदा नारायणांशजम्}


\twolineshloka
{मया हि सुमहत्तप्तं तपः परमदारुणम्}
{पुरा मतिमतां श्रेष्ठाः परमेण समाधिना}


\threelineshloka
{एतद्वः कथितं सर्वं यन्मां पृच्छत पुत्रकाः}
{पूर्वजन्म भविष्यं च भक्तानां स्नेहतो मया ॥वैशंपायन उवाच}
{}


\twolineshloka
{एष ते कथितः पूर्वः संभवोऽस्मद्गुरोर्नृप}
{व्यासस्याक्लिष्टमनसो यथा पृष्टः पुनः शृणु}


\twolineshloka
{साङ्ख्यं योगः पाञ्चरात्रं वेदाः पाशुपतं तथा}
{ज्ञानान्येतानि राजर्षे विद्धि नानामतानि वै}


\twolineshloka
{साङ्ख्यस्य वक्ता कपिलः परमर्षिः स उच्यते}
{हिरण्यगर्भो योगस्य वेत्ता नान्यः पुरातनः}


\twolineshloka
{अपांतपतमाश्चैव वेदाचार्यः स उच्यते}
{प्राचीनगर्भं तमृषिं प्रवदन्तीह केचन}


\twolineshloka
{उमापतिर्भूतपतिः श्रीकण्ठो ब्रह्मणः सुतः}
{उक्तवानिदमव्यग्रो ज्ञानं पाशुपतं शिवः}


\twolineshloka
{पाञ्चरात्रस्य कृत्स्नस्य वक्ता तु भगवान्स्वयम्}
{सर्वेषु च नृपश्रेष्ठ ज्ञानेष्वेतेषु दृश्यते}


\twolineshloka
{यथागमं यथाज्ञानं निष्ठा नारायणः प्रभुः}
{न चैनमेवं जानन्ति तमोभूता विशांपते}


\twolineshloka
{तमेव शास्त्रकर्तारं प्रवदन्ति मनीषिणः}
{निष्ठां नारायणमृषिं नान्योस्तीति च वादिनः}


\twolineshloka
{निःसंशयेषु सर्वेषु नित्यं वसति वै हरिः}
{ससंशयान्हेतुबलान्नाध्यावसति माधवः}


\twolineshloka
{पाञ्चरात्रविदो ये तु यथाक्रमपरा नृप}
{एकान्तभावोपगतास्ते हरिं प्रविशन्ति वै}


\twolineshloka
{साङ्ख्यं च योगं च सनातने द्वेवेदाश्च सर्वे निखिलेन राजन्}
{सर्वैः समस्तैर्ऋषिभिर्निरुक्तोनारायणो विश्वमिदं पुराणम्}


\twolineshloka
{शुभाशुभं कर्म समीरितं यत्प्रवर्तते सर्वलोकेषु किंचित्}
{तस्मादृपेस्तद्भवतीति विद्याद्दिव्यन्तरिक्षे भुवि चाप्सु चेति}


\chapter{अध्यायः ३६०}
\twolineshloka
{जनमेजय उवाच}
{}


\threelineshloka
{बहवः पुरुषा ब्रह्मन्नुताहो एक एव तु}
{को ह्यत्र पुरुषः श्रेष्ठः को वा योनिरिहोच्यते ॥वैशंपायन उवाच}
{}


\twolineshloka
{बहवः पुरुषा लोके साङ्ग्ययोगविचारणे}
{नैतदिच्छन्ति पुरुषमेकं कुरुकुलोद्वह}


\twolineshloka
{बहनां पुरुषाणां च यथैका योनिरुच्यते}
{तथा तं पुरुषं विश्वं व्याख्यास्यामि गुणाधिकम्}


\twolineshloka
{नमस्कृत्वा च गुरवे व्यासाय विदितात्मने}
{तपोयुक्ताय दान्ताय वन्द्याय परमपये}


\twolineshloka
{इदं पुरुषसूक्तं हि सर्ववेदेषु पार्थिव}
{ऋतं सत्यं च विख्यातमृपिसिंहेन चिन्तितम्}


\twolineshloka
{उत्सर्गेणापवादेन ऋषिभिः कपिलादिभिः}
{अध्यान्मचिन्तामाश्रित्य शास्त्राण्युक्तानि भारत}


\twolineshloka
{समासतेस्तु यद्व्यासः पुरुषैकत्वमुक्तवान्}
{तत्तेऽहं संप्रवक्ष्यामि प्रसादादमितौजसः}


\twolineshloka
{अत्राप्युदाहरन्तीममितिहासं पुरातनम्}
{ब्रह्मणा सह संवादं त्र्यम्बकस्य विशांपते}


\twolineshloka
{क्षीरोदस्य समुद्रस्य मध्ये हाटकसप्रभः}
{वैजयन्त इति ख्यातः पर्वतप्रवरो नृप}


\twolineshloka
{तत्राध्यात्मगतिं देव एकाकी प्रविचिन्तयन्}
{वैराजसदनान्नित्यं वैजयन्तं निपेवते}


\threelineshloka
{अथ तत्राऽऽसतस्तस्य चतुर्वक्रस्य धीमतः}
{ललाटप्रभवः पुत्रः शिव आगाद्यदृच्छया}
{आकाशेन महायोगी पुरा त्रिनयनः प्रभुः}


\twolineshloka
{ततः खान्निपपाताशु धरणीधरमूर्धनि}
{अग्रतश्चाभवत्प्रीतो ववन्दे चापि पादयोः}


\fourlineindentedshloka
{तं पादयोनिंपतितं दृष्ट्वा सव्येन पाणिना}
{अत्थापयामास तदा प्रभुरेकः प्रजापतिः}
{उवाच चैनं भगवांश्चिरस्यागतमात्मजम् ॥पितामह उवाच}
{}


\fourlineindentedshloka
{स्वागतं ते महाबाहो दिष्ट्या प्राप्तोसि मेऽन्तिकम्}
{कच्चित्ते कुशलं पुत्र स्वाध्यायतपसोः सदा}
{नित्यमुग्रतपास्त्वं हि ततः पृच्छामि ते पुनः ॥रुद्र उवाच}
{}


\twolineshloka
{त्वत्प्रसादेन भगवन्स्वाध्यायतपसोर्मम}
{कृशलं चाव्ययं चैव सर्वस्य जगतस्त्वथ}


\twolineshloka
{चिरदृष्टोमि भगवन्वैराजसदने मया}
{ततोऽहं पर्वतं प्राप्तस्त्विमं त्वत्पादसेवितम्}


\twolineshloka
{कौतूहलं चापि हि मे एकान्तगमनेन ते}
{नैतत्कारणमल्पं हि भविष्यति पितामह}


\twolineshloka
{किंनु तत्सदनं श्रेष्ठं क्षुत्पिपासाविवर्जितम्}
{सुरासुरैरध्युपितमृषिभिश्चामितप्रभैः}


\threelineshloka
{गन्धर्वैरेप्सरोभिश्च सततं संनिषेवितम्}
{उत्सृज्येमं गिरिवरमेकाकी प्राप्तवानसि ॥ब्रह्मोवाच}
{}


\threelineshloka
{वैजयन्तो गिरिवरः सततं सेव्यते मया}
{अत्रैकाग्रेण मनसा पुरुषश्चिन्त्यते विराट् ॥रुद्र उवाच}
{}


\twolineshloka
{बहवः पुरुषा ब्रह्मंस्त्वया सृष्टाः स्वयंभुव}
{सृज्यन्ते चापरे ब्रह्मन्स चैकः पुरुषो विराट्}


\threelineshloka
{को ह्यसौ चिन्त्यते ब्रह्मंस्त्वयैकः पुरुषोत्तमः}
{एतन्मे संशयं छिन्धि महत्कौतूहलं हि मे ॥ब्रह्मोवाच}
{}


\twolineshloka
{बहवः पुरुषाः पुत्र त्वया ये समुदाहृताः}
{एवमेतदतिक्रान्तं द्रष्टव्यं नैवमित्यपि}


\twolineshloka
{आधारं तु प्रवक्ष्यामि एकस्य पुरुषस्य ते}
{बहूनां पुरुषाणां स यथैका योनिरुच्यते}


\twolineshloka
{तथा तं पुरुषं विश्वं परमं सुमहत्तमम्}
{निर्गुणं निर्गुणा भूत्वा प्रविशन्ति सनातनम्}


\chapter{अध्यायः ३६१}
\twolineshloka
{ब्रह्मोवाच}
{}


\twolineshloka
{शृणु पुत्र यथाह्येष पुरुषः शाश्वतोऽव्ययः}
{अक्षयश्चाप्रमेयश्च सर्वगश्च निरुच्यते}


\twolineshloka
{न स शक्यस्त्वया द्रष्टुं मयाऽन्यैर्वाऽपि सत्तम}
{सगुणैर्निर्गुणैर्विश्वो ज्ञानदृश्यो ह्यसौ स्मृतः}


\twolineshloka
{अशरीरः शरीरेषु सर्वेषु निवसत्यसौ}
{वसन्नपि शरीरेषु न स लिप्यति कर्मभिः}


\twolineshloka
{ममान्तरात्मा तव च ये चान्ये देहसंज्ञिताः}
{सर्वेषां साक्षिभूतोऽसौ न ग्राह्यः केनचिक्वचित्}


\twolineshloka
{विश्वमूर्धा विश्वभुजो विश्वपादाक्षिनासिकः}
{एकश्चरति क्षेत्रेषु स्वैरचारी यथासुखम्}


\twolineshloka
{क्षेत्राणि हि शरीराणि बीजं चापि शुभाशुभम्}
{तानि वेत्ति स योगात्मा ततः क्षेत्रज्ञ उच्यते}


\twolineshloka
{नागतिर्न गतिस्तस्य ज्ञेया भूतेषु केनचित्}
{साङ्ख्येन विधिना चैव योगेन च यथाक्रमम्}


\twolineshloka
{चिन्तयामि गतिं चास्य न गतिं वेद्मि चोत्तराम्}
{यथाज्ञानं तु वक्ष्यामि पुरुषं तु सनातनम्}


\twolineshloka
{तस्यैकत्वं महत्त्वं च स चैकः पुरुषः स्मृतः}
{महापुरुषशब्दं स बिभर्त्येकः सनातनः}


\threelineshloka
{एको हुताशो बहुधा समिध्यतेएकः सूर्यस्तपसो योनिरेका}
{एको वायुर्बहुधा वाति लोकेमहोदधिश्चाम्भसां योनिरेकः}
{पुरुषश्चैको निर्गुणो विश्वरूपस्तं निर्गुणं पुरुषं चाविशन्ति}


\twolineshloka
{हित्वा गुणमयं सर्वं कर्मं हित्वा शुभाशुभम्}
{उभे सत्यानृते त्यक्त्वा एवं भवति निर्गुणः}


\twolineshloka
{अचिन्त्यं चापि तं ज्ञात्वा भावसूक्ष्मं चतुष्टयम्}
{विचरेद्योऽसमुन्नद्धः स गच्छेत्पुरुषं शुभम्}


\twolineshloka
{एकं हि परमात्मानं केचिदिच्छन्ति पण्डिताः}
{एकात्मानं तथाऽऽत्मानमपरेध्यात्मचिन्तकाः}


\twolineshloka
{तत्र यः परमात्मा हि स नित्यो निर्गुणः स्मृतः}
{स हि नारायणो ज्ञेयः सर्वात्मा पुरुषो हि सः}


\twolineshloka
{न लिप्यते फलैश्चापि पद्मपत्रमिवाम्भसा}
{कर्मात्मा त्वपरो योसौ मोक्षबन्धैः स युज्यते}


\twolineshloka
{ससप्तदशकेनापि राशिना युज्यते च सः}
{एवं बहुविधः प्रोक्तः पुरुषस्ते यथाक्रमम्}


\twolineshloka
{यत्तत्कृत्स्नं लोकतन्त्रस्य धामवेद्यं परं बोधनीयं च वेदैः}
{मन्ता मन्तव्यं प्राशिता प्राशनीयंघ्राता घ्रेयं स्पर्शिता स्पर्शनीयम्}


\twolineshloka
{द्रष्टा द्रष्टव्यं श्राविता श्रावणीयंज्ञाता ज्ञेयं सगुणं निर्गुणं च}
{यद्वै प्रोक्तं तात सम्यक्प्रधानंनित्यं चैतच्छाश्वतं चाव्ययं च}


\twolineshloka
{यद्वै सूते धातुराद्यं विधानंतद्वै विप्राः प्रवदन्तेऽनिरुद्धम्}
{यद्वै लोके वैदिकं कर्म साधुआशीर्युक्तं तद्धि तस्योपभोग्यम्}


\twolineshloka
{देवाः सर्वे मनुयः साधु दान्तास्तं प्राग्वंशे यज्ञभागं भजन्ते}
{अहं ब्रह्मा आद्य ईशः प्रजानांतस्माज्जातस्त्वं च मत्तः प्रसूतः}


% Check verse!
मत्तो जगज्जङ्गमं स्थावरं चसर्वे वेदाः सरहस्या हि पुत्र
\twolineshloka
{चतुर्विभक्तः पुरुषः स क्रीडति यथेच्छति}
{एवं स भगवान्देवः स्वेन ज्ञानेन बोधयत्}


\twolineshloka
{एतत्ते कथितं पुत्र यथावदनुपृच्छतः}
{साङ्ख्यज्ञाने तथा योगे यथावदनुवर्णितम्}


\chapter{अध्यायः ३६२}
\twolineshloka
{युधिष्ठिर उवाच}
{}


\threelineshloka
{धर्माः पितामहेनोक्ता मोक्षधर्माश्रिताः शुभाः}
{धर्ममाश्रमिणां श्रेष्ठं वक्तुमर्हति मे भवान् ॥भीष्म उवाच}
{}


\twolineshloka
{सर्वत्र विहितो धर्मः सत्यः सत्यफलोदयः}
{बहुद्वारस्य धर्मस्य नेहास्ति विफला क्रिया}


\twolineshloka
{यस्मिन्यस्मिंश्च विषये यो यो याति विनिश्चयम्}
{स तमेवाभिजानाति नान्यं भरतसत्तम}


\twolineshloka
{इमां च त्वं नरव्याघ्र श्रोतुमर्हसि मे कथाम्}
{पुरा शक्रस्य कथितां नारदेन महर्षिणा}


\twolineshloka
{महर्षिर्नारदो राजन्सिद्धस्त्रैलोक्यसंमतः}
{पर्येति क्रमशो लोकान्वायुरव्याहतो यथा}


\twolineshloka
{स कदाचिन्महेष्वास देवराजालयं गतः}
{सत्कृतश्च महेन्द्रेण प्रत्यासन्नगतोऽभवत्}


\twolineshloka
{तं कृतक्षणमासीनं पर्यपृच्छच्छत्तीपतिः}
{महर्षे किंचिदाश्चर्यमस्ति दृष्टं त्वयाऽनघ}


\twolineshloka
{दृष्टमेव हि विप्रर्षे त्रैलोक्यं सचराचरम्}
{जातकौतूहलो नित्यं सिद्धश्चरसि साक्षिवत्}


\twolineshloka
{न ह्यस्त्यविदितं लोके देवर्षे तव किंचन}
{श्रुतं वाऽप्यनुभूतं वा दृष्टं वा कथयस्व मे}


\twolineshloka
{तस्मै राजन्सुरेन्द्राय नारदो वदतांवरः}
{आसीनायोपपन्नाय प्रोक्तवान्विपुलां कथाम्}


\twolineshloka
{यथा येन च कल्पेन स तस्मै द्विजसत्तमः}
{कथां कथितवान्पृष्टस्तथा त्वमपि मे शृणु}


\chapter{अध्यायः ३६३}
\twolineshloka
{भीष्म उवाच}
{}


\twolineshloka
{आसीत्किल नरश्रेष्ठ महापद्मे पुरोत्तमे}
{गङ्गाया दक्षिणे तीरे कश्चिद्विप्रः समाहितः}


\twolineshloka
{सौम्यः सोमान्वये जातो जितात्मा गोत्रतो भृगुः}
{धर्मनित्यो जितक्रोधो नित्यतप्तो जितेन्द्रियः}


\twolineshloka
{तपःस्वाध्यायनिरतः सत्यः सज्जनसंमतः}
{न्यायप्राप्तेन वित्तेन स्वेन शीलेन चान्वितः}


\twolineshloka
{ज्ञाति संबन्धिविपुले पुत्रपौत्रप्रतिष्ठिते}
{कुले महति विख्याते विशिष्टां वृत्तिमास्थितः}


\twolineshloka
{स पुत्रान्बहुलाँल्लब्ध्वा विपुले कर्मणि स्थितः}
{कुलधर्माश्रितो राजन्धरर्मचर्यास्थितोऽभवत्}


\twolineshloka
{ततः स धर्मं वेदोक्तं तथा शास्त्राक्तमेव च}
{शिष्टाचीर्णं च धर्मं च त्रिविधं चिन्त्य चेतसा}


\twolineshloka
{किंनु मे स्याच्छुभं कृत्वा किं कृतं किं परायणम्}
{इत्येवं चिन्तयन्नित्यं न च याति विनिश्चयम्}


\twolineshloka
{तस्यैवं चिन्त्यमानस्य धर्मं परममास्थितः}
{कदाचिदतिथिः प्राप्तो ब्राह्मणः सुसमाहितः}


\twolineshloka
{स तस्मै सत्क्रियां चक्रे क्रियायुक्तेन हेतुना}
{विश्रान्तं सुसमासीनमिदं वचनमब्रवीत्}


\chapter{अध्यायः ३६४}
\twolineshloka
{ब्राह्मण उवाच}
{}


\twolineshloka
{समुत्पन्ने विधानेऽस्मिन्वाङ्भाधुर्येण तेऽनघ}
{मित्रत्वमभिसंपन्नः किंचिद्वक्ष्यामि तच्छृणु}


\twolineshloka
{गृहस्थधर्मं विप्रेन्द्र श्रुत्वा धर्मगतं त्वहम्}
{धर्मं परमकं कुर्यां को हि मार्गो भवेद्द्विज}


\twolineshloka
{अहमात्मानमास्थाय एक एवात्मनि स्थितम्}
{द्रष्टुमिच्छन्न पश्यामि बद्धः साधारणैर्गुणैः}


\twolineshloka
{यावदेतदतीतं मे वयः पुत्रफलाश्रितम्}
{तावदिच्छामि पाथेयमादातुं पारलौकिकम्}


\twolineshloka
{अस्मिन्हि लोकसंभारे परं पारमभीप्सतः}
{उत्पन्ना मे मतिरियं कुतो धर्ममयः प्लवः}


\twolineshloka
{संयुज्यमानानि निशाम्य लोकेनिर्यात्यमानानि च सात्विकानि}
{दृष्ट्वा तु धर्मध्वजकेतुमालांप्रकीर्यमाणामुपरि प्रजानाम्}


\twolineshloka
{न मे मनो रज्यति भोगरागैर्दृष्ट्वा गतिं प्रार्थयतः परत्र}
{तेनातिथे बुद्धिबलाश्रयेणधर्मेण धर्मे विनियुङ्क्ष्व मां त्वम्}


\twolineshloka
{सोऽतिथिर्वचनं तस्य श्रुत्वा धर्माभिभाषिणः}
{प्रोवाच वचनं श्लक्षणं प्राज्ञो मधुरया गिरा}


\twolineshloka
{अहमप्यत्र मुह्यामि ममाप्येष मनोरथः}
{न च संनिश्चयं चामि बहुद्वारे त्रिविष्टये}


\twolineshloka
{केचिन्मोक्षं प्रशंसन्ति केचिद्यज्ञफलं द्विजाः}
{वानप्रस्थाश्रयाः केचिद्गार्हस्थ्यं केचिदाश्रिताः}


\twolineshloka
{राजधर्माश्रयाः केचित्केचिदात्मफलाश्रयाः}
{गुरुधर्माश्रयाः केचित्केचिद्वाक्संयमाश्रयाः}


\twolineshloka
{मातरं पितरं केचिच्छुश्रूषन्तो दिवं गताः}
{अहिंसया परे स्वर्गं सत्येन च तथाऽपरे}


\twolineshloka
{आहवेऽभिमुखः केचिन्निहतास्त्रिदिवं गताः}
{केचिदुञ्छव्रतैः सिद्धाः स्वर्गमार्गं समाश्रिताः}


\twolineshloka
{केचिदध्ययने युक्ता वेदव्रतपराः शुभाः}
{बुद्धिमन्तो गताः स्वर्गं तुष्टात्मानो जितेन्द्रियाः}


\twolineshloka
{आर्जवेनापरे युक्ता निहतानार्जवैर्जनैः}
{ऋजवो नाकपृष्ठे वै शुद्धात्मानः प्रतिष्ठिताः}


\twolineshloka
{एवं बहुविधैर्लोकैर्धर्मद्वारैरनावृतैः}
{ममापि मतिराविद्धा मेघलेखेव वायुना}


\chapter{अध्यायः ३६५}
\twolineshloka
{अतिथिरुवाच}
{}


\twolineshloka
{उपदेशं तु ते विप्र करिष्येऽहं यथाक्रमम्}
{गुरुणा मे यथाख्यातमर्थतत्त्वं तु मे शृणु}


\twolineshloka
{यत्र पूर्वाभिसर्गे वै धर्मचक्रं प्रवर्तितम्}
{नैमिषे गोमतीतीरे तत्र नागह्रदो महान्}


\twolineshloka
{समग्रैस्रिदशैस्तत्र इष्टमासीद्द्विजर्षभ}
{यत्रेन्द्रातिक्रमं चक्रे माधाता राजसत्तमः}


\twolineshloka
{कृताधिवासो धर्मात्मा तत्र चक्षुःश्रवा महान्}
{पद्मनाभो महानागः पद्म इत्येव विश्रुतः}


\twolineshloka
{स वाचा कर्मणा चैव मनसा च द्विजर्षभः}
{प्रसादयति भूतानि त्रिविधे वर्त्मनि स्थितः}


\twolineshloka
{साम्ना भेदेन दानेन दणडेनेति चतुर्विधम्}
{पिपमस्थं समस्थं च चक्षुर्ध्यानेन रक्षति}


\twolineshloka
{तमतिक्रम्य विधिना प्रष्टुमर्हसि काङ्क्षित्तम्}
{स ते परमकं धर्मं न मिथ्या दर्शयिष्यरति}


\twolineshloka
{स हि सर्वातिथिर्नाणो बुद्धिशास्त्रविशारदः}
{गुणैरनुपमैर्युक्तः समस्तैराभिकामिकैः}


\twolineshloka
{प्रकृत्या नित्यसलिलो नित्यमध्ययने रतः}
{तपोदमाभ्यां संयुक्तो वृत्तेनानवरेण च}


\twolineshloka
{यज्वा दानपतिः क्षान्तो वृत्ते च परमे स्थितः}
{सत्यवागनसूयुश्च शीलवान्नियतेन्द्रियः}


\twolineshloka
{शेषान्नभोक्ता वचनानुकूलोहितार्जवोत्कृष्टकृताकृतज्ञः}
{अवैरकृद्भूतहिते नियुक्तोगङ्गाह्रदाम्भोभिजनोपपन्नः}


\chapter{अध्यायः ३६६}
\twolineshloka
{ब्राह्मण उवाच}
{}


\twolineshloka
{अतिभारोद्यतस्यैव भारावतरणं महत्}
{पराश्वासकरं वाक्यमिदं मे भवतः श्रुतम्}


\twolineshloka
{अध्यक्लान्तस्य शयनं स्थानक्लान्तस्य चासनम्}
{तृषितस्येव पानीयं क्षुधार्तस्येव भोजनम्}


\twolineshloka
{ईप्सितस्येव संप्राप्तिरर्थस्य समयेऽतिथे}
{एपितस्यात्मनः काले बृद्धस्यैव सुता यथा}


\twolineshloka
{मनसा चिन्तितस्येव प्रीतिस्निग्धस्य दर्शनम्}
{प्रह्लादयति मां वाक्यं भवता यदुदीरितम्}


\twolineshloka
{मनश्चक्षुरिवाकाशे पश्यामि विमृशामि च}
{प्रज्ञानवचनाद्योयमुपदेशो हि मे कृतः}


\twolineshloka
{वाढमेवं करिष्यामि यथा मे भाषते भवान्}
{इमां हि रजनीं साधो निवसस्व मया सह}


\threelineshloka
{प्रभाते यास्यति चवान्पर्याश्वस्तः सुखोपितः}
{असौ हि भगवान्सूर्यो मन्दरश्मिरवाङ्भुखः ॥भीष्म उवाच}
{}


\twolineshloka
{ततस्तेन कृतातिथ्यः सोऽनिथिः शत्रुसूदन}
{उवास किल तां रात्रिं सह तेन द्विजेन वै}


\twolineshloka
{तत्वं च धर्मसंयुक्तं तयोः कथयतोस्तदा}
{व्यतीता सा निशा कृत्स्ना सुखेन दिवसोपमा}


\twolineshloka
{ततः प्रभातसमये सोऽतिथिस्तेन पूजितः}
{ब्राह्मणेन यथाशक्त्या स्वकार्यमभिकाङ्क्षता}


\twolineshloka
{ततः स विप्रः कृतकर्मनिश्चयःकृताभ्यनुज्ञः स्वजनेन धर्मकृत्}
{यथोपदिष्टं भुजगेन्द्रसंश्रयंजगाम काले सुकृतैकनिश्चयः}


\chapter{अध्यायः ३६७}
\twolineshloka
{भीष्म उवाच}
{}


\twolineshloka
{स वनानि विचित्राणि तीर्थानि च सरांसि च}
{अभिगच्छन्क्रमेण स्म कंचिन्मुनिमुपस्थितः}


\twolineshloka
{तं स तेन यथोद्दिष्टं नागं विप्रेण ब्राह्मणः}
{पर्यपृच्छद्यथान्यायं श्रुत्वैव च जगाम सः}


\twolineshloka
{सोऽभिगम्य यथान्यायं नागायतनमर्थवित्}
{प्रोक्तवानहमस्मीति भोःशब्दालंकृतं वचः}


\threelineshloka
{तत्तस्य वचनं श्रुत्वा रूपिणी धर्मवत्सला}
{दर्शयामास तं विप्रं नागपत्नी पतिव्रता}
{}


\threelineshloka
{सा तस्मै विधिवत्पूजां चक्रे धर्मपरायणा}
{स्वागतेनागतं कृत्वा किं करोमीति चाब्रवीत् ॥ब्राह्मण उवाच}
{}


\twolineshloka
{विश्रान्तोऽभ्यर्चिंतश्चास्मि भवत्या श्लक्ष्णया गिरा}
{द्रष्टुमिच्छामि भवति देवं नागमनुत्तमम्}


\threelineshloka
{एतद्धि परमं कार्यमेतन्मे परमप्सितम्}
{अनेन चार्थेनास्म्यद्य संप्राप्तः पन्नगाश्रमम् ॥नागभार्योवाच}
{}


\twolineshloka
{आर्यः सूर्यरथं वोढुं गतोऽसौ मासचारिकः}
{सप्ताष्टभिर्दिनैर्विप्र दर्शयिष्यत्यसंशयम्}


\threelineshloka
{एतद्विदितमार्यस्य विवासकरणं तव}
{भर्तुर्भवतु किंचान्यत्क्रियतां तद्वदस्व मे ॥ब्राह्मण उवाच}
{}


\twolineshloka
{अनेन निश्चयेनाहं साध्वि संप्राप्तवानिह}
{प्रतीक्षन्नागमं देवि वत्स्याम्यस्मिन्प्रहावने}


\twolineshloka
{संप्राप्तस्यैव चाव्यग्रमावेद्योऽहमिहागतः}
{मयाभिगमनं प्राप्तो वाच्यश्च वचनं त्वया}


\twolineshloka
{अहमप्यत्र वत्स्यामि गोमत्याः पुलिने शुभे}
{कालं परिमिताहारो यथोक्तं परिपालयन्}


\twolineshloka
{ततः स विप्रस्तां नागीं समाधाय पुनःपुनः}
{वेदवित्पुलिनं नद्याः प्रययौ ब्राह्मणर्भषः}


\chapter{अध्यायः ३६८}
\twolineshloka
{भीष्म उवाच}
{}


\twolineshloka
{अथ तेन नरश्रेष्ठ ब्राह्मणेन तपस्विना}
{निराहारेण वसता दुःखितास्ते भुजङ्गमाः}


\twolineshloka
{सर्वे संभूय सहिता ह्यस्य नावस्य बान्धवाः}
{भ्रातरस्तनया भार्या ययुस्तं ब्राह्मणं प्रति}


\twolineshloka
{तेऽपश्यन्पुलिने तं वै विविक्ते नियतव्रतम्}
{समासीनं निराहारं द्विजं जप्यपरायणम्}


\twolineshloka
{ते सर्वे समभिक्रम्य विप्रमभ्यर्च्य चासकृत्}
{ऊचुर्वाक्यमसंदिग्धमातिथेयस्य बान्धवाः}


\twolineshloka
{षष्ठो हि दिवसस्तेऽद्य प्राप्तस्येह तपोधन}
{न चाभिभाषसे किंचिदाहारं धर्मवत्सल}


\twolineshloka
{आमानभिगतश्चासि वयं च त्वामुपस्थिताः}
{कार्यं चातिथ्यमस्माभिरीप्सितं तव ऋद्धिमत्}


\twolineshloka
{मूलं फलं वा पर्णं वा पयो वा द्विजसत्तम}
{आहारहेतोरन्नं वा भोक्तुमर्हसि ब्राह्मण}


\twolineshloka
{त्यक्ताहारेण भवता वने निवसता त्वया}
{बालवृद्धमिदं सर्वं पीड्यते धर्मसंकरात्}


\threelineshloka
{न हि नो भ्रूणहा कश्चित्पन्नगेष्विह विद्यते}
{पूर्वाशी वा कुले ह्यस्मिन्देवतातिथिबन्धुषु ॥ब्राह्मण उवाच}
{}


\twolineshloka
{उपदेशेन युष्माकमाहारोऽयं कृतो मया}
{द्विरूनं दशरात्रं वै नागस्यागमनं प्रति}


\twolineshloka
{यद्यष्टरात्रेऽतिक्रान्ते नागमिष्यति पन्नगः}
{तदाहारं करिष्यामि तन्निमित्तमिद व्रतम्}


\twolineshloka
{कर्तव्यो न च संतापो गम्यतां च यथागतम्}
{तन्निमित्तमिदं सर्वं नैतद्भेत्तुगिहार्हथ}


\twolineshloka
{ते तेन समनुज्ञाता ब्राह्मणेन भुजङ्गमाः}
{स्वमेव भवनं जग्मुरकृतार्था नरर्षभ}


\chapter{अध्यायः ३६९}
\twolineshloka
{भीष्म उवाच}
{}


\twolineshloka
{अथ काले बहुतिथे पूर्णे प्राप्तो भुजङ्गमः}
{दत्ताभ्यनुज्ञः स्वं वेश्म कृतकर्मा विवस्वता}


\twolineshloka
{तं भार्याऽप्युपचक्राम पादशौचादिभिर्गुणैः}
{उपपन्नां च तां साध्वीं पन्नगः पर्यपृच्छत}


\twolineshloka
{अथ त्वमसि कल्याणि देवतातिथिपूजने}
{पूर्वमुक्तेन विधिना युक्ता कर्मसु वर्तसे}


\threelineshloka
{न खल्वस्यकृतार्थेन स्त्रीबुद्ध्या मार्दवीकृता}
{मद्वियोगेन सुश्रोणि विमुक्ता धर्मसेतुना ॥नागभार्योवाच}
{}


\twolineshloka
{शिष्याणां गुरुशुश्रूषा विप्राणां वेदधारणम्}
{भृत्यानां स्वामिवचनं राज्ञो लोकानुपालनम्}


\twolineshloka
{सर्वभूतपरित्राणं क्षत्रधर्म इहोच्यते}
{वैश्यानां यज्ञसंवृत्तिरातिथेयसमन्विता}


\twolineshloka
{विप्रक्षत्रियवैश्यानां शुश्रूषा शूद्रकर्म तत्}
{गृहस्थधर्मो नागेन्द्र सर्वभूतहितैषिता}


\twolineshloka
{नियताहारता नित्यं व्रतचर्या यथाक्रमम्}
{धर्मो हि धर्मसंबन्धादिन्द्रियाणां विशेषतः}


\twolineshloka
{अहं कस्य कुतो वाऽपि कः को मे ह भवेदिति}
{प्रयोजनमतिर्नित्यमेवं मोक्षाश्रमे वसेत्}


\twolineshloka
{पतिव्रतात्वं भार्यायाः परमो धर्म उच्यते}
{तवोपदेशान्नागेन्द्र तच्च तत्त्वेन वेद्मि वै}


\twolineshloka
{साऽहं धर्मं विजानन्ती धर्ननित्ये त्वयि स्थिते}
{सत्पथं कथमृत्सृज्य यास्यामि विपथं पथः}


\twolineshloka
{देवतानां महाभाग धर्मचर्या न हीयते}
{अतिथीनां च सत्कारे नित्ययुक्ताऽस्म्यतन्द्रिता}


\twolineshloka
{सप्ताष्टदिवसास्त्वद्य विप्रस्येहागतस्य वै}
{तच्च कार्यं न मे ख्याति दर्शनं तव काङ्क्षति}


\twolineshloka
{गोमत्यास्त्वेष पुलिने त्वद्दर्शनसमुत्सुकः}
{आसीनो वर्तयन्ब्रह्म ब्राह्मणः संशितव्रतः}


\twolineshloka
{अहं त्वनेन नागेन्द्र सत्यपूर्वं समाहिता}
{प्रस्थाप्यो मत्सकाशं स संप्राप्तो भुजगोत्तमः}


\twolineshloka
{एतच्छ्रुत्वा महाप्राज्ञ तत्र गन्तुं त्वमर्हसि}
{दातुमर्हसि वा तस्य दर्शनं दर्शनश्रवः}


\chapter{अध्यायः ३७०}
\twolineshloka
{नाग उवाच}
{}


\twolineshloka
{अथ ब्राह्मणरूपेण कं तं समनुपश्यसि}
{मानुषं केवलं विप्रं देवं वाऽथ शुचिस्मिते}


\twolineshloka
{को हि मां मानुषः शक्तो द्रष्टुकामो यशस्विनि}
{संदर्शनरुचिर्वाक्यमाज्ञापूर्वं वदिष्यति}


\twolineshloka
{सुरासुरगणानां च देवर्षीनां च भामिनि}
{ननु नागा महावीर्याः सौरभेयास्तरस्विनः}


\threelineshloka
{वन्दनीयाश्च वरदा वयमप्यनुयायिनः}
{मनुष्याणां विशेषेण नावेक्ष्या इति मे मतिः ॥नारभार्योवाच}
{}


\twolineshloka
{आर्जवेन विजानामि नासौ देवोऽनिलाशन}
{एकं तस्मिन्विजानामि भक्तिमानतिरोपण}


\twolineshloka
{स हि कार्यान्तराकाङ्क्षी जलेप्सुः स्तोकको यथा}
{वर्षं वर्षप्रियः पक्षी दर्शनं तव काङ्क्षते}


\twolineshloka
{हित्वा त्वद्दर्शनं किंचिद्विघ्नं न प्रतिपालयेत्}
{तुल्योप्यभिजने जातो न कश्चित्पर्युपासते}


\twolineshloka
{तद्रोषं सहजं त्यक्त्वा त्वमेनं द्रष्टुमर्हसि}
{आशाच्छेदेन तस्याद्य नात्मानं दरधुमर्हसि}


\twolineshloka
{आशया ह्यभिपन्नानामकृत्वाऽश्रुप्रमार्जनम्}
{राजा वा राजपुत्रो वा भ्रूणहत्यैव युज्यते}


\twolineshloka
{मौने ज्ञानफलावाप्तिर्दानेन च यशो महत्}
{वाग्मित्वं सत्यवाक्येन परत्र च महीयते}


\twolineshloka
{भूप्रदानेन च गतिं लभत्याश्रमसंमिताम्}
{न्याय्यस्यार्धस्य संप्राप्तिं कृत्वा फलमुपाश्रुते}


\threelineshloka
{अभिप्रेतामसंश्लिष्टां कृत्वा चात्महितां क्रियाम्}
{न याति निरयं कश्चिदिति धर्मविदो विदुः ॥नाम उवाच}
{}


\twolineshloka
{अभिमानैर्न मानो मे जातिदोषेण वै महान्}
{रोषः संकल्पजः साध्वि दग्धो वागग्निना त्वया}


\twolineshloka
{न च रोषादहं साध्वि पश्येयमधिकं तमः}
{तस्य वक्तव्यतां याति विशेषेण भुजङ्गमाः}


\twolineshloka
{रोषस्य हि वशं गत्वा दशग्नीवः प्रतापवान्}
{तथा शक्रप्रतिस्पर्धी हतो रामेण संयुगे}


\twolineshloka
{अन्तःपुरगतं वत्सं श्रुत्वा रामेण निर्हृतम्}
{धर्मणारोषसंविग्नाः कार्तवीर्यसुता हताः}


\twolineshloka
{जामदग्न्येन रामेण सहस्रनयनोपमः}
{संयुगे निहतो रोषात्कार्तवीर्यो महाबलः}


\twolineshloka
{तदेष तपसां शत्रुः श्रेयसां विनिपातकः}
{निगृहीतो मया रोषः श्रुत्वैवं वचनं तव}


\twolineshloka
{आत्मानं च विशेषेण प्रशंसाम्यनपायिनि}
{यस्य मे त्वं विशालाक्षि भार्या गुणसमन्विता}


\twolineshloka
{एष तत्रैव गच्छामि यत्र तिष्ठत्यसौ द्विजः}
{सर्वथा चोक्तवान्वाक्यं स कृतार्थः प्रयास्यति}


\chapter{अध्यायः ३७१}
\twolineshloka
{भीष्म उवाच}
{}


\twolineshloka
{स पन्नगपतिस्तत्र प्रययौ ब्राह्मणं प्रति}
{तमेव मनसा ध्यायन्कार्यवत्तां विचारयन्}


\twolineshloka
{तमतिक्रम्य नागेन्द्रो मतिमान्स नरेश्वर}
{प्रोवाच मधुरं वाक्यं प्रकृत्या धर्मवत्सलः}


\twolineshloka
{भोभो क्षाम्याभिभाषए त्वां न रोषं कर्तुमर्हसि}
{इह त्वमभिसंप्राप्तः कस्यार्थे किं प्रयोजनम्}


\threelineshloka
{आभिमुख्यादभिक्रम्य स्नेहात्पृच्छामि ते द्विज}
{विविक्ते गोमतीतीरे कं वा त्वं पर्युपाससे ॥ब्राह्मण उवाच}
{}


\twolineshloka
{धर्मारण्यं हि मां विद्धि नागं द्रष्टुमिहागतम्}
{पद्मनाभं द्विजश्रेष्ठ तत्र मे कार्यमाहितम्}


\twolineshloka
{तस्य चाहमसान्निध्ये श्रुतवानस्मि तं गतम्}
{स्वजनात्तं प्रतीक्षामि पर्जन्यमिव कर्षकः}


\threelineshloka
{तस्य चाक्लेशकरणं स्वस्तिकारसमाहितम्}
{आवर्तयामि तद्ब्रह्म योगयुक्तो निरामयः ॥नाग उवाच}
{}


\twolineshloka
{अहो कल्याणवृत्तस्त्वं साधुः सज्जनवत्सलः}
{अवाच्यस्त्वं महाभाग परं स्नेहेन पश्यसि}


\twolineshloka
{अहं स नागो विप्रर्षे यथा मां विन्दते भवान्}
{आज्ञापय यथास्वैरं किं करोमि प्रियं तव}


\twolineshloka
{भवन्तं स्वजनादस्मि संप्राप्तं श्रुतवानहम्}
{अतस्त्वां स्वयमेवाहं द्रष्टुमभ्यागतो द्विज}


\twolineshloka
{संप्राप्तश्च भवानद्य कृतार्थः प्रतियास्यति}
{विस्रब्धो मां द्विजश्रेष्ठ विषये योक्तमर्हसि}


\threelineshloka
{वयं हि भवता सर्वे गुणक्रीता विशेषतः}
{यस्त्वमात्महितं त्यक्त्वा मामेवेहानुरुध्यसे ॥ब्राह्मण उवाच}
{}


\twolineshloka
{आगतोऽहं महाभाग तव दर्शनलालसः}
{कंचिदर्थमनर्यज्ञः प्रष्टुकामो भुजंगम}


\twolineshloka
{अहमात्मानमात्मस्थो मार्गगाणोऽऽत्मनो गतिम्}
{वासार्थिनं महाप्रज्ञं चलच्चित्तमुपास्मि ह}


\twolineshloka
{प्रकाशितस्त्वं स गुणैर्यशोगर्भगभस्तिभिः}
{शशाङ्ककरसंस्पर्शैर्हृद्यैरात्मप्रकाशितैः}


\twolineshloka
{तस्य मे प्रश्नमुत्पन्नं छिन्धि त्वमनिलाशन}
{पश्चात्कार्यं वदिष्यामि श्रोतुमर्हति तद्भवान्}


\chapter{अध्यायः ३७२}
\twolineshloka
{ब्राह्मण उवाच}
{}


\threelineshloka
{विवस्वतो गच्छति पर्ययेणवोडुं भवांस्तं यथमेकचक्रम्}
{आश्चर्यभूतं यदि तत्र किंचिद्दृष्टं त्वयाशंसितुमर्हसि त्वम् ॥नाग उवाच}
{}


\twolineshloka
{आश्चर्याणामनेकानां प्रतिष्ठा भगवान्रविः}
{यतो भूताः प्रवर्तन्ते सर्वे त्रैलोक्यसंमताः}


\twolineshloka
{यस्य रश्मिसहस्रेषु शाखास्विव विहंगमाः}
{वसन्त्याश्रित्य मुनयः संसिद्धा दैवतैः सह}


\twolineshloka
{यतो पायुर्विनिःसृत्य सूर्यरश्म्याश्रितो महान्}
{विजृम्भत्यम्बरे तत्र किमाश्चर्यमतः परम्}


\twolineshloka
{विभज्यं तं तु विप्रर्षे प्रजानां हितकाम्यया}
{तोयं सृजति वर्षासु किमाश्चर्यमतः परम्}


\twolineshloka
{यस्य मण्डलमध्यस्थो महात्मा परमत्विषा}
{12-372-6 दीप्तःसमीक्षतेऽलोकान्किमाश्चर्यमतः परम्}


\twolineshloka
{शुक्रो नामासितः पादो यश्च वारिधरोऽम्बरे}
{तोयं सृजति वर्षासु किमाश्चर्यमतः परम्}


\twolineshloka
{योऽष्टमासांस्तु शुचिना किरणेनोक्षितं पयः}
{प्रत्यादत्ते पुनः काले किमाश्चर्यमतः परम्}


\twolineshloka
{यस्य तेजोविशेषेषु स्वयमात्मा प्रतिष्ठितः}
{यतो बीजं मही चेयं धार्यते सचराचरम्}


\twolineshloka
{यत्र देवो महाबाहुः शाश्वतः पुरुषोत्तमः}
{अनादिनिधनो विप्र किमाश्चर्यमतः परम्}


\twolineshloka
{आश्चर्याणामिवाश्चर्यमिदमेकं तु मे शृणु}
{विमले यन्मया दृष्टमम्बरे सूर्यसंश्रयात्}


\twolineshloka
{पुरा मध्याह्नसमये लोकांस्तपति भास्करे}
{प्रत्यादित्यप्रतीकाशः सर्वतः समदृश्यत}


\twolineshloka
{स लोकांस्तेजसा सर्वान्स्वभासा निर्विभासयन्}
{आदित्याधिमुखोऽभ्येति गगनं पाटयन्निव}


\twolineshloka
{हुताहुतिरिव ज्योतिर्व्याप्य तेजोमरीचिभिः}
{अनिर्देश्येन रूपेण द्वितीय इव भास्करः}


\twolineshloka
{तस्याभिगमनप्राप्तौ हस्तौ दत्तौ विवस्वता}
{तेनापि दक्षिणो हस्तो दत्तः प्रत्यर्चितार्थिना}


\twolineshloka
{ततो भित्त्वैव गगन प्रविष्टो रश्मिमण्डलम्}
{एकीभूतं च तत्तेजः क्षणेनादित्यतां गतम्}


\twolineshloka
{तत्र नः संशयो जातस्तयोस्तेजः समागमे}
{अनयोः को भवेत्सूर्यो रथस्थो योऽयमागतः}


\twolineshloka
{ते वयं जातसंदेहाः पर्यपृच्छामहे रविम्}
{क एष दिवमाक्रम्य गतः सूर्य इवापरः}


\chapter{अध्यायः ३७३}
\twolineshloka
{सूर्य उवाच}
{}


\twolineshloka
{नैष देवोऽनिलसखो नासुरो न च यन्नगः}
{उञ्छवृत्तिव्रते सिद्धो मुनिरेष दिवं गतः}


\twolineshloka
{एष मूलफलाहारः शीर्णपर्णाशनस्तथा}
{अब्भक्षो वायुभक्षश्च आसीद्विप्रः समाहितः}


\twolineshloka
{भवश्चानेन विप्रेण संहिताभिरभिष्टुतः}
{स्वर्गद्वारे कृतोद्योगो येनासौ त्रिविदं गतः}


\twolineshloka
{असंगतिरनाकाङ्क्षी नित्यमुञ्छशिलाशनः}
{सर्वभूतहिते युक्त एष विप्रो भुजंगमाः}


\twolineshloka
{न हि देवा न गन्धर्वा नासुरा न च पन्नगाः}
{प्रभवन्तीह भूतानां प्राप्तानामुत्तमां गतिम्}


\threelineshloka
{एतदेवंविधं दृष्टमाश्चर्यं तत्र मे द्विज}
{संसिद्धो मानुषः कामं योसौ सिद्धगतिं गतः}
{सूर्येण सहितो ब्रह्मन्पृथिवीं परिवर्तते}


\chapter{अध्यायः ३७४}
\twolineshloka
{ब्राह्मण उवाच}
{}


\twolineshloka
{आश्चर्यं नात्र संदेहः सुप्रीतोस्मि भुजङ्गम}
{अन्वर्थोपगतैर्वाक्यैः पन्थानं चास्मि दर्शितः}


\threelineshloka
{स्वस्ति तेऽस्तु गमिष्यामि साधो भुजगसत्तम}
{स्मरणीयोस्मि भवता संप्रेषणनियोजनैः ॥नाग उवाच}
{}


\twolineshloka
{अनुक्त्वा हृद्गतं कार्यं क्वेदानीं प्रस्थितो भवान्}
{उच्यतां द्विज यत्कार्यं यदर्थं त्वमिहागतः}


\twolineshloka
{उक्तानुक्ते कृते कार्ये मामामन्त्र्य द्विजर्षभ}
{मया प्रत्यभ्यनुज्ञातस्ततो यास्यसि सुव्रत}


\twolineshloka
{न हि मां केवलं दृष्ट्वा त्यक्त्वा प्रणयवानिह}
{गन्तुमर्हसि विप्रर्षे वृक्षमूलगतो यथा}


\threelineshloka
{त्वयि चाहं द्विजश्रेष्ठ भवान्मयि न संशयः}
{लोकोऽयं भवतः सर्वः का चिन्ता मयि तेऽनघ ॥ब्राह्मण उवाच}
{}


\twolineshloka
{एवमेतन्महाप्राज्ञ विदितात्मन्भुजङ्गम}
{नातिरिक्तास्त्वया देवाः सर्वथैव यथातथम्}


\twolineshloka
{स एव त्वं स एवाहं योऽहं स तु भवानपि}
{अहं भवांश्च भूतानि सर्वे यत्र गताः सदा}


\twolineshloka
{आसीत्तु मे भोगिपते संशयः पुण्यरसंचये}
{सोहमुञ्छव्रतं साधो चरिष्याम्यर्थसाधनम्}


\twolineshloka
{एष मे निश्चयः साधो कृतं कारणमुत्तमम्}
{आमन्त्रयामि भद्रं ते कृतार्थोऽस्मि भुजङ्गम}


\chapter{अध्यायः ३७५}
\threelineshloka
{वैशंपायन उवाच}
{शरतल्पे महात्मानं शयानमपराजितम्}
{युधिष्ठिर उपागम्य प्रणिपत्येदमब्रवीत्}


