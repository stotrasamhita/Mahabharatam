\part{आदिपर्व}
\chapter{अध्यायः १}
% Check verse!
श्रीवेदव्यासाय नमः
\twolineshloka
{नारायणं नमस्कृत्य नरं चैव नरोत्तमम्}
{देवीं सरस्वतीं चैव(व्यासं) ततो जयमुदीरयेत्}


\twolineshloka
{`नारायणं सुरगुरुं जगदेकनाथं'भक्तप्रियं सकललोकनमस्कृतं च}
{त्रैगुण्यवर्जितमजं विभुमाद्यमीशंवन्दे भवघ्नममरासुरसिद्धवन्द्यम्'}


\twolineshloka
{`नमो धर्माय महते नमः कृष्णाय वेधसे}
{ब्राह्मणेभ्यो नमस्कृत्य धर्मान्वक्ष्यामि शाश्वतान्'}


% Check verse!
ॐ नमो भगवते वासुदेवाय

ॐ नमः पितामहाय

ॐ नमः प्रजापतिभ्यः

ॐ नमः कृष्णद्वैपायनाय

ॐ नमः सर्वविघ्नविनायकेभ्यः
% Check verse!
रोमहर्षणपुत्र उग्रश्रवाः सौतिः पौराणिकोनैमिशारण्ये शौनकस्य कुलपतेर्द्वादशवार्षिके सत्रे
\twolineshloka
{सुखासीनानभ्यगच्छद्ब्रह्मर्षीन्संशितव्रतान्}
{विनयावनतो भूत्वा कदाचित्सूतनन्दनः}


\twolineshloka
{तमाश्रममनुप्राप्य नैमिशारण्यवासिनः}
{`उवाच तानृषीन्सर्वान्धन्यो वोऽस्म्यद्यदर्शनात्}


\twolineshloka
{वेद वैयासिकीः सर्वाः कथा धर्मार्यैसंहिताः}
{वक्ष्यामि वो द्विजश्रेष्ठाः शृण्वन्त्वद्य तपोधनाः}


\twolineshloka
{तस्य तद्वचनं श्रुत्वा नैमिशारण्यवासिनः}
{चित्रा श्रोतुं कथास्तत्र परिव्रुस्तपस्विनः}


\twolineshloka
{अभिवाद्य मुनींस्तांस्तु सर्वानेव कुताञ्जलिः}
{अपृच्छत्स तपोवृद्धिं सद्भिश्चैवाभिपूजितः}


\twolineshloka
{अथ तेषूपविष्टेषु सर्वेष्वेव तपस्विषु}
{निर्दिष्टमासनं भेजे विनयाद्रौमहर्षणिः}


\twolineshloka
{सुखासीनं ततस्तं तु विश्रान्तमुपलक्ष्य च}
{अथापृच्छदृषिस्तत्र कश्चित्प्रस्तावयन्कथाः}


\twolineshloka
{कुत आगम्यते सौते क्वचायं विहृतस्त्वया}
{कालः कमलपत्राक्ष शंसैतत्पृच्छतो मम}


\twolineshloka
{एवं पृष्टोऽब्रवीत्सम्यग्यथावद्रौमहर्षणिः}
{वाक्यं वचनसंपन्नस्तेषां च चरिताश्रयम्}


\threelineshloka
{तस्मिन्सदसि विस्तीर्णे मुनीनां भावितात्मनाम्}
{सौतिरुवाच}
{जनमेजयस्य राजर्षेः सर्पसत्रे महात्मनः}


\twolineshloka
{समीपे पार्थिवेन्द्रस्य सम्यक्पारिक्षितस्य च}
{कृष्णद्वैपायनप्रोक्ताः सुपुण्या विविधाः कथाः}


\twolineshloka
{कथिताश्चापि विधिवद्या वैशंपायनेन वै}
{श्रुत्वाऽहं ता विचित्रार्था महाभारतसंश्रिताः}


\twolineshloka
{वहूनि संपरिक्रम्य तीर्थान्यायतनानि च}
{समन्तपञ्चकं नाम पुण्यं द्विजनिषेवितम्}


\twolineshloka
{गतवानस्मि तं देशं युद्धं यत्राभवत्पुरा}
{कुरूणां पाण्डवानां च सर्वेषां चहीक्षिताम्}


\twolineshloka
{दिदृक्षुंरागतस्तस्मात्समीपं भावतामिह}
{आयुष्मन्तः सर्व एव ब्रह्मभाता हि मे मताः ॥अस्मिन्यज्ञे महाभङ्गाः सूर्यपावकवर्चसः}


\twolineshloka
{कृताभिषेकाः शुचयः कृतजप्या हुताग्नयः}
{भवन्त आसते स्वस्था ब्रवीमि किमहं द्विजाः}


\threelineshloka
{पुराणसंहिताः पुण्याः कथा धर्मार्थसंश्रिताः}
{इतिवृत्तं नरेन्द्राणामृषीणां च महात्मनाम् ॥ऋषय ऊचुः}
{}


\twolineshloka
{द्वैपायनेन यत्प्रोक्तं पुराणं परमर्षिणा}
{सुरैर्ब्रह्मर्षिभिश्चैव श्रुत्वा यदभिपूजितम्}


\twolineshloka
{तस्याख्यानवरिष्ठस्य विचित्रपदपर्वणः}
{सूक्ष्मार्थन्याययुक्तस्य वेदार्थैर्भूषितस्य च}


\twolineshloka
{भारतस्येतिहासस्य पुण्यां ग्रन्थार्थसंयुताम्}
{संस्कारोपगतां ब्राह्मीं नानाशास्त्रोपबृंहिताम्}


\twolineshloka
{जनमेजयस्य यां राज्ञो वैशंपायन उक्तवान्}
{यथावत्स ऋषिः पृष्टः सत्रे द्वैपायनाज्ञया}


\threelineshloka
{वेदैश्चतुर्भिः सयुक्तां व्यासस्याद्भुतकर्मणः}
{संहितां श्रोतुमिच्छामः पुण्यां पापभयापहाम् ॥सौतिरुवाच}
{}


\twolineshloka
{आद्यं पुरुषमीशानं पुरुहूतं पुरुष्टुतम्}
{ऋतमेकाक्षरं ब्रह्म व्यक्ताव्यक्तं सनातनम्}


% Check verse!
असच्च सच्चैव च यद्विश्वं सदसतः परम्परावराणां स्रष्टारं पुराणं परमव्ययम्
\twolineshloka
{मङ्गल्यं मङ्गलं विष्णुं वरेण्यमनघं शुचिम्}
{नमस्कृत्य हृषीकेशं चराचरगुरुं हरिम्}


\twolineshloka
{महर्षेः पूजितस्येह सर्वलोकैर्महात्मनः}
{प्रवक्ष्यामि मतं पुण्यं व्यासस्याद्भुतकर्मणः}


\twolineshloka
{`नमो भगवते तस्मै व्यासायामिततेजसे}
{यस्य प्रसादाद्वक्ष्यामि नारायणकथामिमाम्}


\twolineshloka
{सर्वाश्रमाभिशमनं सर्वतीर्थावगाहनम्}
{न तथा फलद सूते नारायणकथा यथा}


\twolineshloka
{नास्ति नारायणसमं न भूतं न भविष्यति}
{एतेन सत्यवाक्येन सर्वार्थान्साधयाम्यहम्'}


\twolineshloka
{आचख्युः कवयः केचित्संप्रत्याचक्षते परे}
{आख्यास्यन्ति तथैवान्य इतिहासमिमं भुवि}


\twolineshloka
{इदं तु त्रिषु लोकेषु महज्ज्ञानं प्रतिष्ठितम्}
{विस्तरैश्च समासैश्च धार्यते यद्द्विजातिभिः}


\twolineshloka
{अलङ्कृतं शुभैः शब्दैः समयैर्दिव्यधनुषैः}
{छन्दोवृत्तैश्च विविधैरन्वितं विदुषांप्रियम्}


\twolineshloka
{तपसा ब्रह्मचर्येण व्यस्य वेदं सनातनम्}
{इतिहासमिमं चक्रे पुण्यं सत्यवतीसुतः}


\twolineshloka
{`पुण्ये हिमवतः पादे मेध्ये गिरिगुहालये}
{विशोध्य देहं धर्मात्मा दर्भसंस्तरमाश्रितः}


% Check verse!
शुचिः सनियमो व्यासः शान्तात्मातपसि स्थितःभारतस्येतिहासस्य धर्मेणान्वीक्ष्य तां गतिम्
% Check verse!
प्रविश्य योगं ज्ञानेन सोऽपश्यत्सर्वमन्ततः
\twolineshloka
{निष्प्रभेऽस्मिन्निरालोके सर्वतस्तमसा वृते}
{बृहदण्डमभूदेकं प्रजानां बीजमव्ययम्}


\twolineshloka
{युगस्यादिनिमित्तं तन्महद्दिव्यं प्रचक्षत}
{व्यस्मिंस्तच्छ्रूयते सत्यंज्योतिर्ब्रह्म सनातनम्}


\twolineshloka
{अद्भुतं चाप्यचिन्त्यं च सर्वत्र समतां मतम्}
{अव्यक्तं कारणं सूक्ष्मं यत्तत्सदसदात्मकम्}


\twolineshloka
{यस्मिन्पितामहो जज्ञे प्रभुरेकः प्रजापतिः}
{ब्रह्मा सुरगुरुः स्थाणुर्मनुः कः परमेष्ठ्यथ}


\twolineshloka
{प्राचेतसस्तथा दक्षो दक्षपुत्राश्च सप्तवै}
{ततः प्रजानां पतयः प्राभवन्नेकविंशतिः}


\twolineshloka
{पुरुषश्चाप्रमेयात्मा यं सर्वऋषयो विदु}
{विश्वेदेवास्तथाऽऽदित्या वसवोऽथाश्विनावपि}


\twolineshloka
{यक्षाः साध्याः पिशाचाश्च गुह्यकाः पितरस्तथा}
{ततः प्रसूता विद्वांसः शिष्टा ब्रह्मर्षिसत्तमाः}


\twolineshloka
{महर्षयश्च बहवः सर्वैः समुदिता गुणैः}
{आतो द्यौः पृथिवी वायुरन्तरिक्षं दिशस्तया}


\twolineshloka
{संवत्सरर्तवो मासाः पक्षाहोरात्रयः क्रमात्}
{यच्चान्यदपि तत्सर्वं संभूतं लोकसाक्षिकम्}


\twolineshloka
{यदिदं दृश्यते किंचिद्बूतं स्थावरजङ्गमम्}
{पुनःसंक्षिप्यते सर्वं जगत्प्राप्ते युगक्षये}


\twolineshloka
{यथर्तुष्वृतुलिङ्गानि नानारूपाणि पर्यये}
{दृश्यन्ते तानि तान्येव तथा भावा युगादिषु}


\twolineshloka
{एवमेतदनाद्यन्तं भूतसंघातकारकम्}
{अनादिनिधनं लोके चक्रं संपरिवर्तते}


\twolineshloka
{त्रयस्त्रिंशत्सहस्राणि त्रयस्त्रिंशच्छतानि च}
{त्रयस्त्रिंशच्च देवनां सृष्टिः संक्षेपलक्षणा}


\twolineshloka
{दिवः पुत्रो बृहद्भानुश्चक्षुरात्मा विभावसुः}
{सविता स ऋचीकोऽर्को भानुराशावहो रविः}


\twolineshloka
{पुत्रा विवस्वतः सर्वे मनुस्तेषां तथाऽवरः}
{देवभ्राट् तनयस्तस्य सुभ्राडिति ततः स्मृतः}


\twolineshloka
{सुभ्राजस्तु त्रयः पुत्राः प्रजावन्तो बहुश्रुताः}
{दशज्योतिः शतज्योतिः सहस्रज्योतिरेव च}


\twolineshloka
{दशपुत्रसहस्राणि दशज्योतेर्महात्मनः}
{ततो दशगुणाश्चान्ये शतज्योतेरिहात्मजाः}


\twolineshloka
{भूयस्ततो दशगुणाः सहस्रज्योतिषः सुताः}
{तेभ्योऽयं कुरुवंशश्च यदूनां भरतस्य च}


\twolineshloka
{ययातीक्ष्वाकृवंशश्च राजर्षीणां च सर्वशः}
{संभूता बहवो वंशा भूतसर्गाः सुविस्तराः}


\twolineshloka
{भूतस्थानानि सर्वाणि रहस्यं त्रिविधं च यत्}
{वेदा योगः सविज्ञानो धर्मोऽर्थः काम एव च}


\twolineshloka
{धर्मार्थकामयुक्तानि शास्त्राणि विविधानि च}
{लोकयात्राविधान च सर्व तद्दृष्टवानृषिः}


\twolineshloka
{`नीतिर्भरतवंशस्य विस्तारश्चैव सर्वशः}
{'इतिहासाः सहव्याख्या विविधाश्रुतयोऽपि च}


\twolineshloka
{इह सर्वमनुक्रान्तमुक्तं ग्रन्थस्य लक्षणम्}
{`संक्षेपेणेतिहासस्य ततो वक्ष्यति विस्तरम् ॥'}


\twolineshloka
{विस्तीर्यैतन्महज्ज्ञानमृषिः संक्षिप्य चाब्रवीत्}
{इष्टं हि विदुषां लोके समासव्यासधारणम्}


\twolineshloka
{मन्वादि भारतं केचिदास्तीकादि तथाऽपरे}
{तथोपरिचराद्यन्ये विप्राः सम्यगधीयिरे}


\twolineshloka
{विविधं संहिताज्ञानं दीपयन्ति मनीषिणः}
{व्याख्यातुं कुशलाः केचिद्ग्रन्थान्धारयितुं परे}


\twolineshloka
{तपसा ब्रह्मचर्येण व्यस्य वेदं सनातनम्}
{इतिहासमिमं चक्रे पुण्यं सत्यवतीत्सुतः}


\twolineshloka
{पराशरात्मजो विद्वान्ब्रह्मर्षिः संशितव्रतः}
{मातुर्नियोगाद्धर्मात्मा गाङ्गेयस्य च धीमतः}


\twolineshloka
{क्षेत्रे विचित्रवीर्यस्य कृष्णद्वैपायनः पुरा}
{त्रीनग्नीनिव कौरव्याञ्जनयामास वीर्यवान्}


\twolineshloka
{उत्पाद्य धृतराष्ट्रं च पाण्डुं विदुरमेव च}
{जगाम तपसे धीमान्पुनरेवाश्रमं प्रति}


\twolineshloka
{तेषु जातेषु वृद्धेषु गतेषु परमां गतिम्}
{अब्रवीद्भारतं लोके मानुषेऽस्मिन्महानृषिः}


\twolineshloka
{जनमेजयेन पृष्टः सन्ब्राह्मणैश्च सहस्रशः}
{शशास शिष्यमासीनं वैशंपायनमन्तिके}


\twolineshloka
{स सदस्यैः सहासीनं श्रावयामास भारतम्}
{कर्मान्तरेषु यज्ञस्य चोद्यमानः पुनः पुनः}


\twolineshloka
{विस्तारं कुरुवंशस्य गान्धार्या धर्मशीलताम्}
{क्षत्तुः प्रज्ञां धृतिं कुन्त्याः सम्यग्द्वैपायनोब्रवीत्}


\twolineshloka
{वासुदेवस्य माहात्म्यं पाण्डवानां च सत्यताम्}
{दुर्वृत्तं धार्तराष्ट्राणामुक्तवान्भगवानृषिः}


\twolineshloka
{इदं शतसहस्रं तु श्लोकानां पुण्यकर्मणाम्}
{उपाख्यानैः सह ज्ञेयं श्राव्यं भारतमुत्तमम्}


\twolineshloka
{चतुर्विंशतिसाहस्रीं चक्रे भारतसंहिताम्}
{उपाख्यानैर्विना तावद्भारतं प्रोच्यते बुधैः}


\twolineshloka
{ततोऽध्यर्धशतं भूयः संक्षेपं कृतवानृषिः}
{अनुक्रमणिकाध्यायं वृत्तान्तं सर्वपर्वणाम्}


\twolineshloka
{तस्याख्यानवरिष्ठस्य कृत्वा द्वैपायनः प्रभुः}
{कथमध्यापयानीह शिष्यानित्यन्वचिन्तयत्}


\twolineshloka
{तस्य तच्चिन्तितं ज्ञात्वा ऋषेर्द्वैपायनस्य च}
{तत्राजगाम भगवान्ब्रह्मा लोकगुरुः स्वयम्}


\twolineshloka
{प्रीत्यर्थं तस्य चैवर्षेर्लोकानां हितकाम्यया}
{तं दृष्ट्वा विस्मितो भूत्वा प्राञ्जलिः प्रणतः स्थितः}


% Check verse!
आसनं कल्पयामास सर्वैर्मुनिगणैर्वृतः
\twolineshloka
{हिरण्यमर्भमासीनं तस्मिंस्तु परमासने}
{परिवृत्यासनभ्याशे वासवेयः स्थितोऽभवत्}


\twolineshloka
{अनुज्ञातोऽथ कृष्णस्तु ब्रह्मणा परमेष्ठिना}
{निषसादासनाभ्याशे प्रीयमाणः शुचिस्मितः}


\twolineshloka
{उवाच स महातेजा ब्रह्माणं परमेष्ठिनम्}
{कृतं मयेदं भगवन्काव्यं परमपूजितम्}


\twolineshloka
{ब्रह्मन्वेदरहस्य च यच्चान्यत्स्थापितं मया}
{साङ्गोपनिषदां चैव वेदानां विस्तरक्रिया}


\twolineshloka
{इतिहासपुरापानामुन्मेषं निमिषं च यत्}
{भूतं भव्यं भविष्यच्च त्रिविधं कालसंज्ञितम्}


\twolineshloka
{जरामृत्युभयव्याधिभावाभावविनिश्चयः}
{विविधस्य च धर्मस्य ह्याश्रमाणां च लक्षणम्}


\twolineshloka
{चातुर्वर्ण्यविधानं च पुराणानां च कृत्स्नशः}
{तपसो ब्रह्मचर्यस्य पृथिव्याश्चन्द्रसूर्ययोः}


\twolineshloka
{ग्रहनक्षत्रताराणां प्रमाणं च युगैः सह}
{ऋचो यजूषि सामानि वेदाध्यात्मं तथैव च}


\twolineshloka
{न्यायशिक्षा चिकित्सा च दानं पाशुपतं तथा}
{इति नैकाश्रयं जन्म दिव्यमानुषसंज्ञितम्}


\twolineshloka
{तीर्थानां चैव पुण्यानां देशानां चैव कीर्तनम्}
{नदीनां पर्वतानां च वनानां सागरस्य च}


\twolineshloka
{पुराणां चैव दिव्यानां कल्पानां युद्धकौशलम्}
{वाक्यजातिविशेषाश्च लोकयात्राक्रमश्च यः}


\threelineshloka
{यच्चापि सर्वगं वस्तु तच्चैव प्रतिपादितम्}
{परं न लेखकः कश्चिदेतस्य भुवि विद्यते ॥ब्रह्मोवाच}
{}


\twolineshloka
{तपोविशिष्टदपि वै वसिष्ठान्मुनिपुंगवात्}
{मन्ये श्रेष्ठव्यं त्वां वै रहस्यज्ञानवेदनात्}


\twolineshloka
{जन्मप्रभृति सत्यां ते वेद्मि गां ब्रह्मवादिनीम्}
{त्वयाच काव्यमित्युक्तं तस्मात्काव्यं भविष्यति}


\twolineshloka
{अस्य काव्यस्य कवयो न समर्था विशेषणे}
{विशेषणे गृहस्थस्य शेषास्त्रय इवाश्रमाः}


\twolineshloka
{`जडान्धबधिरोन्मत्तं तमोभूतं जगद्भवेत्}
{यदि ज्ञानहुताशेन त्वया नोज्ज्वलियं भवेत्}


\twolineshloka
{तमसान्धस्य लोकस्य वेष्टितस्य स्वकर्मभिः}
{ज्ञानाञ्जनशलाकाभिर्बुद्धिनेत्रोत्सवः कृतः'}


\twolineshloka
{धर्मार्थकाममोक्षार्थैः समासव्यासकीर्तनैः}
{त्वया भारतसूर्येण नृणां विनिहतं तमः}


\twolineshloka
{पुराणपूर्णचन्द्रेण श्रुतिज्योत्स्नाप्रकाशिना}
{नृणां कुमुदसौम्यानां कृतं बुद्धिप्रसादनम्}


\twolineshloka
{इतिहासप्रदीपेन मोहावरणघातिना}
{लोकगर्भगृहं कृत्स्नं यथावत्संप्रकाशितम्}


\twolineshloka
{संग्रहाध्यायबीजो वै पौलोमास्तीकमूलवान्}
{संभवस्कन्धविस्तारः सभापर्वविटङ्कवान्}


\twolineshloka
{आरण्यपर्वरूपाढ्यो विराटोद्योगसारवान्}
{भीष्मपर्वमहाशाखो द्रोणपर्वपलाशवान्}


\twolineshloka
{कर्णपर्वसितैः पुष्पैः शल्यपर्वसुगन्धिभिः}
{स्त्रीपर्वैषीकविश्रामः शान्तिपर्वमहाफलः}


\twolineshloka
{अश्वमेधामृतसस्त्वाश्रमस्थानसंश्रयः}
{मौसलश्रुतिसंक्षेपः शिष्टद्विजनिषेवितः}


\twolineshloka
{सर्वेषां कविमुख्यानामुपजीव्यो भविष्यति}
{पर्जन्यइव भूतानामक्षयो भारद्रुमः}


\threelineshloka
{काव्यस्य लेखनार्थाय गणेशः स्मर्यतां मुने}
{सौतिरुवाच}
{}


\twolineshloka
{एवमाभाष्य तं ब्रह्मा जगाम स्वं निवेशनम्}
{भगवान्स जगत्स्रष्टा ऋषिदेवगणैः सह}


% Check verse!
ततः सस्मार हेरम्बं व्यासः सत्यवतीसुतः
\twolineshloka
{स्मृतमात्रो गणेशानो भक्तचिन्तितपूरकः}
{तत्राजगाम विघ्नेशो वेदव्यासो यतः स्थितः}


\twolineshloka
{पूजितश्चोपविष्टश्च व्यासेनोक्तस्तदानघ}
{लेखको भारतस्यास्य भव त्वं गणनायक ॥मयैव प्रोच्यमानस्य मनसा कल्पितस्य च}


\twolineshloka
{श्रुत्वैतत्प्राह विघ्नेशो यदि मे लेखनी क्षणम्}
{लिखतो नावतिष्ठेत तदा स्यां लेखको ह्यहम्}


\twolineshloka
{व्यासोऽप्युवाच तं देवमबुद्ध्वा मा लिख क्वचित्}
{ओमित्युक्त्वा गणेशोपि बभूव किल लेखकः}


\twolineshloka
{ग्रन्थग्रन्थिं तदा चक्रे मुनिर्गूढं कुतूहलात्}
{यस्मिन्प्रतिज्ञया प्राह मुनिर्द्वैपायनस्त्विदम्}


\twolineshloka
{अष्टौ श्लोकसहस्राणि अष्टौ श्लोकशतानि च}
{अहं वेद्मि शुको वेत्ति संजयो वेत्ति वा न वा}


\twolineshloka
{तच्छ्लोककूटमद्यापि ग्रथितं सुदृढं मुने}
{भेत्तुं न शक्यतेऽर्थस्यं गूढत्वात्प्रश्रितस्य च}


\twolineshloka
{सर्वज्ञोपि गणेशो यत्क्षणमास्ते विचारयन्}
{तावच्चकार व्यासोपि श्लोकानन्यान्बहूनपि}


\twolineshloka
{तस्य वृक्षस्य वक्ष्यामि शाखापुष्पफलोदयम्}
{स्वादुमेध्यरसोपेतमच्छेद्यममरैरपि}


\twolineshloka
{अनुक्रमणिकाध्यायं वृत्तान्तं सर्वपर्वणाम्}
{इदं द्वैपायनः पूर्वं पुत्रमध्यापयच्छुकम्}


\twolineshloka
{ततोऽन्येभ्योऽनुरूपेभ्यः शिष्येभ्यः प्रददौ प्रभुषष्टिं शतसहस्राणि चकारान्यां स संहिताम्}
{त्रिंशच्छतसहस्रं च देवलोके प्रतिष्ठितम्}


\twolineshloka
{पित्र्ये पञ्चदश प्रोक्तं रक्षोयक्षे चतुर्दश}
{एकं शतसहस्रं तु मानुषेषु प्रतिष्ठितम्}


\twolineshloka
{नारदोऽश्रावयद्देवानसितो देवलः पितृन्}
{गन्धर्वयक्षरक्षांसि श्रावयामास वै शुकः}


\twolineshloka
{`वैशंपायनविप्रर्षिः श्रावयामास पार्थिवम्}
{पारिक्षितं महात्मानं नाम्ना तु जनमेजयम्'}


\twolineshloka
{अस्मिंस्तु मानुषे लोके वैशंपायन उक्तवान्}
{शिष्यो व्यासस्य धर्मात्मा सर्ववेदविदां वरः}


% Check verse!
एकं शतसहस्रं तु मयोक्तं वै निबोधत
\threelineshloka
{दुर्योधनो मन्युमयो महाद्रुमःकर्णः स्कन्धः शकुनिस्तस्य शाखाः}
{दुश्शासनः पुष्पफले समृद्धेमूलं राजा धृतराष्ट्रोऽमनीषि}
{}


\twolineshloka
{युधिष्ठिरे धर्ममयो महाद्रुमःस्कन्धोऽर्जुनो भीमसेनोऽस्य शाखाः}
{माद्रीसुतौ पुष्पफले समृद्धेमूलं कृष्णो ब्रह्म च ब्राह्मणाश्च}


\twolineshloka
{पाण्डुर्जित्वा बहून्देशान्युधा विक्रमणेन च}
{अरण्ये मृगयाशीलो न्यवसत्सजनस्तथा}


\twolineshloka
{मृगव्यवायनिधनात्कृच्छ्रां प्राप स आपदम्}
{जन्मप्रभृति पार्थानां तत्राचारविधिक्रमः}


\twolineshloka
{मात्रोरभ्युपपत्तिश्च धर्मोपनिषदं प्रति}
{धर्मानिलेन्द्रांस्ताभिः साऽऽजुहाव सुतवाञ्छया}


\twolineshloka
{`ततो धर्मोपनिषदं भूत्वा भर्तुः प्रिया पृथा}
{धर्मानिलेन्द्रांस्ताभिः साऽऽजुहाव सुतवाञ्छया}


\threelineshloka
{तद्दत्तोपनिषन्माद्री चाश्विनावाजुहाव च}
{जाताः पार्थास्ततः सर्वे कुन्त्या माद्र्याश्च मन्त्रतः}
{'तापसैः सह संवृद्धा मातृभ्यां परिरक्षिताः}


\twolineshloka
{मेध्यारण्येषु पुण्येषु महतामाश्रमेषु च}
{`तेषु जातेषु सर्वेषु पाण्डवेषु महात्मसु}


\twolineshloka
{माद्र्या तु सह संगम्य ऋषिशापप्रभावतः}
{मृतः पाण्डुर्महापुण्ये शतशृङ्गे महागिरौ ॥'}


\twolineshloka
{ऋषिभिश्च समानीता धार्तराष्ट्रान्प्रति स्वयम्}
{शिशवश्चाभिरूपाश्च जटिला ब्रह्मचारिणः}


\twolineshloka
{पुत्राश्च भ्रातरश्चेमे शिष्याश्च सुहृदश्च वः}
{पाण्डवा एत इत्युक्त्वा मुनयोऽन्तर्हितास्ततः}


\twolineshloka
{तांस्तैर्निवेदितान्दृष्ट्वा पाण्डवान्कौरवास्तदा}
{शिष्टाश्च वर्णाः पौरा ये ते हर्षाच्चुक्रुशुर्भृशम्}


\twolineshloka
{आहुः केचिन्न तस्यैते तस्यैत इति चापरे}
{यदा चिरमृतः पाण्डुः कथं तस्येतदि चापरे}


\twolineshloka
{स्वागतं सर्वथा दिष्ट्या पाण्डोः पश्याम सन्ततिम्}
{उच्यतां स्वागतमिति वाचोऽश्रूयन्त सर्वशः}


\twolineshloka
{तस्मिन्नुपरते शब्दे दिशः सर्वा निनादयन्}
{अन्तर्हितानां भूतानां निःस्वनस्तुमुलीऽभवत्}


\twolineshloka
{पुष्पवृष्टिः शुभा गन्धाः शङ्खदुन्दुभिनिःस्वनाः}
{आसन्प्रवेशे पार्थानां तदद्भुतमिवाभवत्}


\twolineshloka
{तत्प्रीत्या चैव सर्वेषां पौराणां हर्षसंभवः}
{शब्द आसीन्महांस्तत्र दिवस्पृक्कीर्तिवर्धनः}


\twolineshloka
{तेऽधीत्य निखिलान्वेदाञ्शास्त्राणि विविधानि च}
{न्यवसन्पाण्डवास्तत्र पूजिता अकुतोभयाः}


\twolineshloka
{युधिष्ठिरस्य शौचेन प्रीताः प्रकृतयोऽभवन्}
{धृत्या च भीमसेनस्य विक्रमेणार्जुनस्य च}


\twolineshloka
{गुरुशुश्रूषया कुन्त्या यमयोर्विनयेन च}
{तुतोष लोकः सकलस्तेषां शौर्यगुणेन च}


\twolineshloka
{समवाये ततो राज्ञां कन्यां भर्तृस्वयंवराम्}
{प्राप्तवानर्जुनः कृष्णां कृत्वा कर्म सुदुष्करम्}


\twolineshloka
{ततः प्रभृति लोकेऽस्मिन्पूज्यः सर्वधनुष्मताम्}
{आदित्य इव दुष्प्रेक्ष्यः समरेष्वपि चाभवत्}


\twolineshloka
{स सर्वान्पार्थिवाञ्जित्वा सर्वांश्च महतो गणान्}
{आजहारार्जुनो राज्ञो राजसूयं महाक्रतुम्}


\twolineshloka
{अन्नवान्दक्षिणावांश्च सर्वैः समुदितो गुणैः}
{युधिष्ठिरेण संप्राप्तो राजसूयो महाक्रतुः}


\twolineshloka
{सुनयाद्वासुदेवस्य भीमार्जुनबलेन च}
{घातयित्वा जरासन्धं चैद्यं च बलगर्वितम्}


\twolineshloka
{दुर्योधनं समागच्छन्नर्हणानि ततस्ततः}
{मणिकाञ्चनरत्नानि गोहस्त्यश्वधनानि च}


\twolineshloka
{विचित्राणि च वासांसि प्रावारावरणानि च}
{कम्बलाजिनरत्नानि राङ्कवास्तरणानि च}


\twolineshloka
{समृद्धां तां तथा दृष्ट्वा पाण्डवानां तदा श्रियम्}
{ईर्ष्यासमुत्थः सुमहांस्तस्य मन्युरजायत}


\twolineshloka
{विमानप्रतिमां तत्र मयेन सुकृतां सभाम्}
{पाण्डवानामुपहृतां स दृष्ट्वा पर्यतप्यत}


\twolineshloka
{तत्रावहसितश्चासीत्प्रस्कन्दन्निव संभ्रमात्}
{प्रत्यक्षं वासुदेवस्य भीमेनानभिजातवत्}


\twolineshloka
{स भोगान्विविधान्भुञ्जन्रत्नानि विविधानि च}
{कथितो धृतराष्ट्रस्य विवर्णो हरिणः कृशः}


\twolineshloka
{अन्वजानात्ततो द्यूतं धृतराष्ट्रः सुतप्रियः}
{तच्छ्रुत्वा वासुदेवस्य कोपः समभवन्महान्}


\twolineshloka
{नातिप्रीतमनाश्चासीद्विवादांश्चान्वमोदत}
{द्यूतादीननयान्घोरान्विविधांश्चाप्युपैक्षत}


\twolineshloka
{निरस्य विदुरं भीष्मं द्रोणं शारद्वतं कृपम्}
{विग्रहे तुमुले तस्मिन्दहन्क्षत्रं परस्परम्}


\twolineshloka
{जयत्सु पाण्डुपुत्रेषु श्रुत्वा सुमहदप्रियम्}
{दुर्योधनमतं ज्ञात्वा कर्णस्य शकुनेस्तथा}


\twolineshloka
{धृतराष्ट्रश्चिरं ध्यात्वा संजयं वाक्यमब्रवीत्}
{शृणु संजय सर्वं मे नचासूयितुमर्हसि}


\twolineshloka
{श्रुतवानसि मेधावी बुद्धिमान्प्राज्ञसंमतः}
{न विग्रहे मम मतिर्न च प्रीये कुलक्षये}


\twolineshloka
{न मे विशेषः पुत्रेषु स्वेषु पाम्डुसुतेषु वा}
{वृद्धं मामभ्यसूयन्ति पुत्रा मन्युपरायणाः}


\twolineshloka
{अहं त्वचक्षुः कार्पण्यात्पुत्रप्रीत्या सहामि तत्}
{मुह्यन्तं चानुमुह्यामि दुर्योधनमचेतनम्}


\twolineshloka
{राजसूये श्रियं दृष्ट्वा पाण्डवस्य महौजसः}
{तच्चावहसनं प्राप्य सभारोहणदर्शने}


\twolineshloka
{अमर्षितः स्वयं जेतुमशक्तः पाण्डवान्रणे}
{निरुत्साहश्च संप्राप्तुं सुश्रियं क्षत्रियोऽपि सन्}


\twolineshloka
{गान्धारराजसहितश्छद्मद्यूतममन्त्रयत्}
{तत्र यद्यद्यथा ज्ञातं मयां संजय तच्छृणु}


\twolineshloka
{श्रुत्वा तु मम वाक्यानि बुद्धियुक्तानि तत्त्वतः}
{ततो ज्ञास्यसि मां सौते प्रज्ञाटचक्षुषमित्युत}


\twolineshloka
{यदाऽश्रौषं धनुरायम्य चित्रंविद्धं लक्ष्यं पातितं वै पृथिव्याम्}
{कृष्णां हृतां प्रेक्षतां सर्वराज्ञांतदा नाशंसे विजयाय संजय}


\twolineshloka
{यदाऽश्रौषं द्वारकायां सुभद्रांप्रसह्योढां माधवीमर्जुनेन}
{इन्द्रप्रस्थं वृष्णिवीरौ च यातौतदा नाशंसे विजयाय संजय}


\twolineshloka
{यदाऽश्रौषं देवराजं प्रवृष्टंशरैर्दिव्यैर्वारितं चार्जुनेन}
{अग्निं तथा तर्पितं खाण्डवे चतदा नाशंसे विजयाय संजय}


\twolineshloka
{यदाऽश्रौषं जातुषाद्वेश्मनस्ता-न्मुक्तान्पार्थान्पञ्च कुन्त्या समेतान्}
{युक्तं चैषां विदुरं स्वार्थसिद्ध्यैतदा नाशंसे विजयाय संजय}


\twolineshloka
{यदाऽश्रौषं द्रौपदीं रङ्गमध्येलक्ष्यं भित्त्वा निर्जितामर्जुनेन}
{शूरान्पञ्चालान्पाण्डवेयांश्च युक्तां-स्तदा नाशंसे विजयाय संजय}


\twolineshloka
{यदाऽश्रौषं मागधानां वरिष्ठंजरासन्धं क्ष्वमध्ये ज्वलन्तम्}
{दोर्भ्यां हतं भीमसेनेन गत्वातदा नाशंसे विजयाय संजय}


\twolineshloka
{यदाऽश्रौषं दिग्जये पाण्डुपुत्रै-र्वशीकृतान्भूमिपालान्प्रसह्य}
{महाक्रतुं राजसूयं कृतं चतदा नाशंसे विजयाय संजय}


\twolineshloka
{यदाऽश्रौषं द्रौपदीमश्रुकण्ठींसभां नीतां दुःखितामेकवस्त्राम्}
{रजस्वलां नाथवतीमनाथव-त्तदा नाशंसे विजयाय संजय}


\twolineshloka
{यदाऽश्रौषं वाससां तत्र राशिंसमाक्षिपत्कितवो मन्दबुद्धिः}
{दुःशासनो गतवान्नैवं चान्तंतदा नाशंसे विजयाय संजय}


\twolineshloka
{यदाऽश्रौषं हृतराज्यं युधिष्ठिरंपराजितं सौबलेनाक्षवत्याम्}
{अन्वागतं भ्रातृभिरप्रमेयै-स्तदा नाशंसे विजयाय संजय}


\twolineshloka
{यदाश्रौषं विविधास्तत्र चेष्टाधर्मात्मनां प्रस्थितानां वनाय}
{ज्येष्ठप्रीत्या क्लिश्यतां पाण्डवानांतदा नाशंसे विजयाय संजय}


\twolineshloka
{यदाऽश्रौषं स्नातकानां सहस्रै-रन्वागतं धर्मराजं वनस्थम्}
{भिक्षाभुजां ब्राह्मणानां महात्मनांतदा नाशंसे विजयाय संजय}


\twolineshloka
{`यदाऽश्रौषं वनवासेन पार्था-न्समागतान्महर्षिभिः पुराणैः}
{उपास्यमानान्सगणैर्जातसख्यां-स्तदा नाशंसे विजयाय संजर्य ॥'}


\twolineshloka
{यदाश्रौषं त्रिदिवस्थं धनंजयंशक्रात्साक्षाद्दिव्यमस्त्रं यथावत्}
{अधीयानं शंसितं सत्यसन्धंतदा नाशंसे विजयाय संजय}


\twolineshloka
{यदाऽश्रोषं कालकेयास्ततस्तेपौलोमानो वरदानाच्च दृप्ताः}
{देवैरजेया निर्जिताश्चार्जुनेनतदा नाशंसे विजयाय संजय}


\twolineshloka
{यदाऽश्रौषमसुराणां वधार्थेकिरीटिनं यान्तममित्रकर्शनम्}
{कृतार्थं चाप्यागतं शक्रलोका-त्तदा नाशंसे विजयाय संजय}


\twolineshloka
{`यदाऽश्रौषं तीर्थयात्राप्रवृत्तंपाण्डोः सुतं सहितं लोमशेन}
{बृहदश्वादक्षहृदयं च प्राप्तंतदा नाशंसे विजयाय संजय ॥'}


\twolineshloka
{यदाऽश्रौषं वैश्रवणेन सार्धंसमागतं भीमन्यांश्च पार्थान्}
{तस्मिन्देशे मानुषाणामगम्येतदा नाशंसि विजयाय संजया}


\twolineshloka
{यदाऽश्रौषं घोषयात्रागतानांबन्धं गन्धर्वैर्मोक्षणं चार्जुनेन}
{स्वेषां सुतानां कर्णबुद्धौ रतानांतदा नाशंसे विजयाय संजय}


\twolineshloka
{यदाऽश्रौषं यक्षरूपेण धर्मंसमागतं धर्मराजेन सूत}
{प्रश्नान्कांश्चिद्विब्रुवाणं च सम्यक्तदा नाशंसे विजयाय संजय}


\twolineshloka
{यदाऽश्रौषं न विदुर्मामकास्तान्प्रच्छन्नरूपान्वसतः पाण्डवेयान्}
{विराटराष्ट्रे सह कृष्णया चतदा नाशंसे विजयाय संजय}


\twolineshloka
{`यदाऽश्रौषं कीचकानां वरिष्ठंनिषूदितं भ्रातृशतेन सार्धम्}
{द्रौपद्यर्थे भीमसेनेन सङ्ख्येतदा नाशंसे विजयाय संजय ॥'}


\twolineshloka
{यदाऽश्रौषं मामकानां वरिष्ठा-न्धनंजयेनैकरथेन भग्नान्}
{विराटराष्ट्रे वसता महात्मनातदा नाशंसे विजयाय संजय}


\twolineshloka
{यदाऽश्रौषं सत्कृतं मत्स्यराज्ञासुतां दत्तामुत्तरामर्जुनाय}
{तां चार्जुनः प्रत्यगृह्णात्सुतार्थेतदा नाशंसे विजयाय संजय}


\twolineshloka
{यदाऽश्रौषं निर्जितस्याधनस्यप्रव्राजितस्य स्वजनात्प्रच्युतस्य}
{अक्षौहिणीः सप्त युधिष्ठिरस्यतदा नाशंसे विजयाय संजय ॥ 196}


\twolineshloka
{यदाऽश्रौषं माधवं वासुदेवंसर्वात्मना पाण्डवार्थे निविष्टम्}
{यस्येमां गां विक्रममेकमाहु-स्तदा नाशंसे विजयाय संजय}


\twolineshloka
{यदाऽश्रौषं नरनारायणौ तौकृष्णार्जुनौ वदतो नारदस्य}
{अहं द्रष्टा ब्रह्मलोके च सम्यक्तदा नाशंसे विजयाय संजय}


\twolineshloka
{यदाऽश्रौषं लोकहिताय कृष्णंशमार्थिनमुपयातं कुरूणाम्}
{शमं कुर्वाणमकृतार्थं च यातंतदा नाशंसे विजयाय संजय ॥ 199}


\twolineshloka
{यदाऽश्रौषं कर्णदुर्योधनाभ्यांबुद्धिं कृतां निग्रहे केशवस्य}
{तं चात्मानं बहुधा दर्शयानंतदा नाशंसे विजयाय संजय}


\twolineshloka
{यदाऽश्रौषं वासुदेवे प्रयातेरथस्यैकामग्रतस्तिष्ठमानाम्}
{आर्तां पृथां सान्त्वितां केशवेनतदा नाशंसे विजयाय संजय}


\twolineshloka
{यदाऽश्रौषं मन्त्रिणं वासुदेवंतथा भीष्मं शान्तनवं च तेषाम्}
{भारद्वाजं चाशिषोऽनुब्रुवाणंतदा नाशंसे विजयाय संजय}


\twolineshloka
{यदा कर्णो भीष्ममुवाच वाक्यंनाहं योत्स्ये युध्यमाने त्वयीति}
{हित्वा सेनामपचक्राम चापितदा नाशंसे विजयाय संजय}


\twolineshloka
{यदाऽश्रौषं वासुदेवार्जुनौ तौतथा धनुर्गाण्डिवमप्रमेयम्}
{त्रीण्युग्रवीर्याणि समागतानितदा नाशंसे विजयाय संजय}


\twolineshloka
{यदाऽश्रौषं कश्ललेनाभिपन्नेरथोपस्थे सीदमानेऽर्जुने वै}
{कृष्णं लोकान्दर्शयानं शरीरेतदा नाशंसे विजयाय संजय}


\twolineshloka
{यदाऽश्रौषं भीष्ममित्रकर्शनंनिघ्नन्तमाजावयुतं रथानाम्}
{नैषां कश्चिद्वध्यते ख्यातरूप-स्तदा नाशंसे विजयाय संजय}


\twolineshloka
{यदाऽश्रौषं चापगेयेन सङ्ख्येस्वयं मृत्युं विहितं धार्मिकेण}
{तच्चाकार्षुः पाण्डवेयाः प्रहृष्टा-स्तदा नाशंसे विजयाय संजय}


\twolineshloka
{यदाऽश्रौषं भीष्ममत्यन्तशूरंहतं पार्थेनाहवेष्वप्रधृष्यम्}
{शिखण्डिनं पुरतः स्थापयित्वातदा नाशंसे विजयाय संजय}


\twolineshloka
{यदाऽश्रौषं शरतल्पे शयानंवृद्धं वीरं सादितं चित्रपुङ्खैः}
{भीष्मं कृत्वा सोमकानल्पशेषां-स्तदा नाशंसे विजयाय संजय}


\twolineshloka
{यदाऽश्रौषं शान्तनवे शयानेपानीयार्थे चोदितेनार्जुनेन}
{भूमिं भित्त्वा तर्पितं तत्र भीष्मंतदा नाशंसे विजयाय संजय}


\twolineshloka
{यदाश्रौषं शुक्रसूर्यौ च युक्तौकौन्तेयानामनुलोमौ जयाय}
{नित्यं चास्माञ्श्वापदा भीषयन्तितदा नाशंसे विजयाय संजय}


\twolineshloka
{यदा द्रोणो विविधानस्त्रमार्गा-न्निदर्शयन्समरे चित्रयोधी}
{न पाण्डवाञ्श्रेष्ठतरान्निहन्तितदा नाशंसे विजयायं संजय}


\twolineshloka
{यदाऽश्रौषं चास्मदीयान्महारथा-न्व्यवस्थितानर्जुनस्यान्तकाय}
{संशप्तकान्निहतानर्जुनेनतदा नाशंसे विजयाय संजय}


\twolineshloka
{यदाऽश्रौषं व्यूहमभेद्यमन्यै-र्भारद्वाजेनात्तशस्त्रेण गुप्तम्}
{भित्त्वा सौभद्रं वीरमेकं प्रविष्टंतदा नाशंसे विजयाय संजय}


\twolineshloka
{यदाऽभिमन्युं परिवार्य बालंसर्वे हत्त्वा हृष्टरूपा बभूवुः}
{महारथाः पार्थमशक्नुवन्त-स्तदा नाशंसे विजयाय संजय}


\twolineshloka
{यदाऽश्रौषमभिमन्युं निहत्यहर्षान्मूढान्क्रोशतो धार्तराष्ट्रान्}
{क्रोधादुक्तं सैन्धवे चार्जुनेनतदा नाशंसे विजयाय संजय}


\twolineshloka
{यदाऽश्रौषं सैन्धवार्थे प्रतिज्ञांप्रतिज्ञातां तद्वधायार्जुनेन}
{सत्यां तीर्णां शत्रुमध्ये च तेनतदा नाशंसे विजयाय संजय}


\twolineshloka
{यदाऽश्रौषं श्रान्तहये धनंजयेमुक्त्वा हयान्पाययित्वोपवृत्तान्}
{पुनर्युक्त्वा वासुदेवं प्रयातंतदा नाशंसे विजयाय संजय}


\twolineshloka
{यदाऽश्रौषं वाहनेष्वक्षमेषुरथोपस्थे तिष्ठता पाण्डवेन}
{सर्वान्योधान्वारितानर्जुनेनतदा नाशंसे विजयाय संजय}


\twolineshloka
{यदाऽश्रौषं नागबलैः सुदुःसहंद्रोणानीकं युयुधानं प्रमथ्य}
{यातं वार्ष्णेयं यत्र तौ कृष्णपार्थौतदा नाशंसे विजयाय संजय}


\twolineshloka
{यदाऽश्रौषं कर्णमासाद्य मुक्तंवधाद्भीमं कुत्सयित्वा वचोभिः}
{धनुष्कोट्याऽऽतुद्य कर्णेन वीरंतदा नाशंसे विजयाय संजय}


\twolineshloka
{यदा द्रोणः कृतवर्मा कृपश्चकर्णो द्रौणिर्मद्रराजश्च शूरः}
{अमर्षयन्सैन्धवं वध्यमानंतदा नाशंसे विजयाय संजय}


\twolineshloka
{यदाऽश्रौषं देवराजेन दत्तांदिव्यां शक्तिं व्यंसितां माधवेन}
{घटोत्कचे राक्षसे घोररूपेतदा नाशंसे विजयाय संजय}


\twolineshloka
{यदाऽश्रौषं कर्णघटोत्कचाभ्यांयुद्धे मुक्तां सूतपुत्रेण शक्तिम्}
{यया वध्यः समरे सव्यसाचीतदा नाशंसे विजयाय संजय}


\twolineshloka
{यदाऽश्रौषं द्रोणमाचार्यमेकंधृष्टद्युम्नेनाभ्यतिक्रम्य धर्मम्}
{रथोपस्थे प्रायगतं विशस्तंतदा नाशंसे विजयाय संजय}


\twolineshloka
{यदाऽश्रौषं द्रौणिना द्वैरथस्थंमाद्रीसुतं नकुलं लोकमध्ये}
{समं युद्धे मण्डलेभ्यश्चरन्तंतदा नाशंसे विजयाय संजय}


\twolineshloka
{यदा द्रोणे निहते द्रोणपुत्रोनारायणं दिव्यमस्त्रं विकुर्वन्}
{नैषामन्तं गतवान्पाण्डवानांतदा नाशंसे विजयाय संजय}


\twolineshloka
{यदाऽश्रौषं भीमसेनेन पीतंरक्तं भ्रातुर्युधि दुःशासनस्य}
{निवारितं नान्यतमेन भीमंतदा नाशंसे विजयाय संजय}


\twolineshloka
{यदाऽश्रौषं कर्णमत्यन्तशूरंहतं पार्थेनाहवेष्वप्रधृष्यम्}
{तस्मिन्भ्रातृणां विग्रहे देवगुह्येतदा नाशंसे विजयाय संजय}


\twolineshloka
{यदाऽश्रौषं द्रोणपुत्रं च शूरंदुःशासनं कृतवर्माणमुग्रम्}
{युधिष्टिरं धर्मराजं जयन्तंतदा नाशंसे विजयाय संजय}


\twolineshloka
{यदाऽश्रौषं निहतं मद्रराजंरणे शूरं धर्मराजेन सूत}
{सदा सङ्ग्रामे स्प्रधते यस्तु कृष्णंतदा नाशंसे विजयाय संजय}


\twolineshloka
{यदाऽश्रौषं कलहद्यूतमूलंमायाबलं सौबलं पाण्डवेन}
{हतं सङ्ग्रामे सहदेवेन पापंतदा नाशंसे विजयाय संजय}


\twolineshloka
{यदाऽश्रौषं श्रान्तमेकं शयानंह्रदं गत्वा स्तम्भयित्वा तदम्भः}
{दुर्योधनं विरथं भग्नशक्तिंतदा नाशंसे विजयाय संजय}


\twolineshloka
{यदाऽश्रौषं पाण्डवांस्तिष्ठमानान्गत्वा ह्रदे वासुदेवेन सार्धम्}
{अमर्षणं धर्षयतः सुतं मेतदा नाशंसे विजयाय संजय}


\twolineshloka
{यदाऽश्रौषं विविधांश्चित्रमार्गान्गदायुद्धे मण्डलशश्चरन्तम्}
{मिथ्या हतं वासुदेवस्य बुद्ध्यातदा नाशंसे विजयाय संजय}


\twolineshloka
{यदाऽश्रौषं द्रोणपुत्रादिभिस्तै-र्हतान्पञ्चालान्द्रौपदेयांश्च सुप्तान्}
{कृतं बीभत्समयशस्यं च कर्मतदा नाशंसे विजयाय संजय}


\twolineshloka
{यदाऽश्रौषं भीमसेनानुयाते-नाश्वत्थाम्ना परमास्त्रं प्रयुक्तम्}
{क्रुद्धेनैषीकमवधीद्येन गर्भंतदा नाशंसे विजयाय संजय}


\twolineshloka
{यदाऽश्रौषं ब्रह्मशिरोऽर्जुनेनस्वस्तीत्युक्त्वाऽस्त्रमस्त्रेण शान्तम्}
{अश्वत्थाम्ना मणिरत्नं च दत्तंतदा नाशंसे विजयाय संजय}


\twolineshloka
{यदाऽश्रौषं द्रोणपुत्रेण गर्भेवैराट्या वै पात्यमाने महास्त्रैः}
{संजीवयामीति हरेः प्रतिज्ञांतदा नाशंसे विजयाय संजय}


\twolineshloka
{द्वैपायनः केशवो द्रोणपुत्रंपरस्पेरणाभिशापैः शशाप}
{बुद्ध्वा चाहं बुद्धिहीनोऽद्य सूतसंतप्ये वै पुत्रपौत्रैश्च हीनः}


\twolineshloka
{शोच्या गान्धारी पुत्रपौत्रैर्विहीनातथा वध्वा पितृभिर्भ्रातृभिश्च}
{कृतं कार्यं दुष्करं पाण्डवेयैःप्राप्तं राज्यमसपत्नं पुनस्तैः}


\twolineshloka
{कष्टं युद्धे दश शेषाः श्रुता मेत्रयोऽस्माकं पाण्डवानां च सप्त}
{द्व्यूना विंशतिराहताऽक्षौहिणीनांतस्मिन्सङ्ग्रामे भैरवे क्षत्रियाणाम्}


\threelineshloka
{तमस्त्वतीव विस्तीर्णं मोह आविशतीव माम्}
{संज्ञां नोपलभे सूत मनो विह्वलतीव मे ॥सौतिरुवाच}
{}


\threelineshloka
{इत्युक्त्वा धृतराष्ट्रोऽथ विलप्य बहु दुःखितः}
{मूर्च्छितः पुनराश्वस्तः संजयं वाक्यमब्रवीत् ॥धृतराष्ट्र उवाच}
{}


\threelineshloka
{संजयैवं गते प्राणांस्त्यक्तुमिच्छामि मा चिरम्}
{स्तोकं ह्यपि न पश्यामि फलं जीवितधारणे ॥सौतिरुवाच}
{}


\twolineshloka
{तं तथा वादिनं दीनं विलपन्तं महीपतिम्}
{निःश्वसन्तं यथा नागं मुह्यमानं पुनः पुनः}


\threelineshloka
{गावल्गणिरिदं धीमान्महार्थं वाक्यमब्रवीत्}
{संजय उवाच}
{श्रुतवानसि वै राजन्महोत्साहान्महाबलान्}


\twolineshloka
{द्वैपायनस्य वदतो नारदस्य च धीमतः}
{महत्सु राजवंशेषु गुणैः समुदितेषु च}


\twolineshloka
{जातान्दिव्यास्त्रविदुषः शक्रप्रतिमतेजसः}
{धर्मेण पृथिवीं जित्वा यज्ञैरिष्ट्वाप्तदक्षिणैः}


\twolineshloka
{अस्मिँल्लोके यशः प्राप्य ततः कालवशं गतान्}
{शैब्यं महारथं वीरं सृंजयं जयतां वरम्}


\twolineshloka
{सुहोत्रं रन्तिदेवं च काक्षीवन्तमतौशिजम्}
{बाह्लीकं दमनं चैद्यं शर्यातिमजितं नलम्}


\twolineshloka
{विश्वामित्रममित्रघ्नमम्बरीषं महाबलम्}
{मरुत्तं मनुमिक्ष्वाकुं गयं भरतमेव च}


\twolineshloka
{रामं दाशरथिं चैव शशबिन्दुं भगीरथम्}
{कृतवीर्यं महाभागं तथैव जनमेजयम्}


\twolineshloka
{ययातिं शुभकर्माणं देवैर्यो याजितः स्वयम्}
{`चैत्ययूपाङ्किता भूमिर्यस्येयं सवनाकरा}


\twolineshloka
{इति राज्ञां चतुर्विंशन्नारदेन सुरर्षिणा}
{पुत्रशोकाभितप्ताय पुरा श्वैत्याय कीर्तितम्}


\twolineshloka
{तेभ्यश्चान्ये गताः पूर्वं राजानो बलवत्तराः}
{महारथा महात्मानः सर्वैः समुदिता गुणैः}


\twolineshloka
{पूरुः कुरुर्यदुः शूरो विष्वगश्वो महाद्युतिः}
{अणुहो युवनाश्वश्च ककुत्स्थो विक्रमी रघुः}


\twolineshloka
{विजयो वीतिहोत्रोऽह्गो भवः श्वेतो बृहद्गुरुः}
{उशीनरः शतरथः कङ्को दुलिदुहो द्रुमः}


\twolineshloka
{दम्भोद्भवः परो वेनः सगरः संकृतिर्निमिः}
{अजेयः परशुः पुण्ड्रः शंभुर्देवावृधोऽनघः}


\twolineshloka
{देवाह्वयः सुप्रतिमः सुप्रतीको बृहद्रथः}
{महोत्साहो विनीतात्मा सुक्रतुर्नैषधो नलः}


\twolineshloka
{सत्यव्रतः शान्तभयः सुमित्रः सुबलः प्रभुः}
{जानुजङ्घोऽनरण्योऽर्कः प्रियभृत्यः शुचिव्रतः}


\twolineshloka
{बलबन्धुर्निरामर्दः केतुशृङ्गो बृहद्बलः}
{धृष्टकेतुर्बृहत्केतुर्दीप्तकेतुर्निरामयः}


\twolineshloka
{अविक्षिच्चपलो धूर्तः कृतबन्धुर्दृढेषुधिः}
{महापुराणसंभाव्यः प्रत्यङ्गः परहा श्रुतिः}


\twolineshloka
{एते चान्ये च राजानः शतशोऽथ सहस्रशः}
{श्रूयन्ते शतशश्चान्ये सङ्ख्याताश्चैव पद्मशः}


\twolineshloka
{हित्वा सुविपुलान्भोगान्बुद्धिमन्तो महाबलाः}
{राजानो निधनं प्राप्तास्तव पुत्रैर्महत्तरः}


\twolineshloka
{येषां दिव्यानि कर्माणि विक्रमस्त्याग एव च}
{माहात्म्यमपि चास्तिक्यं सत्यं शौचं दयाऽर्जवम्}


\twolineshloka
{विद्वद्भिः कथ्यते लोके पुराणैः कविसत्तमैः}
{सर्वर्द्धिगुणसंपन्नास्ते चापि निधनं गताः}


\twolineshloka
{तव पुत्रा दुरात्मानः प्रतप्ताश्चैव मन्युना}
{लुब्धा दुर्वृत्तभूयिष्ठा न ताञ्छोचितुमर्हसि}


\twolineshloka
{श्रुतवानसि मेधावी बुद्धिमान्प्राज्ञसंमतः}
{येषां शास्त्रानुगा बुद्धिर्न ते मुह्यन्ति भारत}


\twolineshloka
{निग्रहानुग्रहौ चापि विदितौ ते नराधिप}
{नात्यन्तमेवानुवृत्तिः कार्या ते पुत्ररक्षणे}


\twolineshloka
{भवितव्यं तथा तच्च नानुशोचितुमर्हसि}
{दैवं पुरुषकारेण को निवर्तितुमर्हति}


\twolineshloka
{विधातृविहितं मार्गं न कश्चिदतिवर्तते}
{कालमूलमिदं सर्वं भावाभावौ सुखासुखे}


\twolineshloka
{कालः सृजति भूतानि कालः संहरते प्रजाः}
{संहरन्तं प्रजाः कालं कालः शमयते पुनः}


\twolineshloka
{कालो विकुरुते भावान्सर्वांल्लोके शुभाशुभान्}
{कालः संक्षिपते सर्वाः प्रजा विसृजते पुनः}


\twolineshloka
{कालः सुप्तेषु जागर्ति कालो हि दुरतिक्रमः}
{कालः सर्वेषु भूतेषु चरत्यविधतः समः}


\threelineshloka
{अतीतानागता भावा ये च वर्तन्ति सांप्रतम्}
{तान्कालनिर्मितान्बुद्ध्वा न संज्ञां हातुमर्हसि ॥सौतिरुवाच}
{}


\twolineshloka
{इत्येवं पुत्रशोकार्तं धृतराष्ट्रं जनेश्वरम्}
{आश्वास्य स्वस्थमकरोत्सूतो गावल्गणिस्तदा}


\twolineshloka
{धृतराष्ट्रोऽपि तच्छ्रुत्वा धृतिमेव समाश्रयत्}
{दिष्ट्येदमागतमिति मत्त्वा स प्राज्ञसत्तमः}


\twolineshloka
{लोकानां च हितार्थाय कारुण्यान्मुनिसत्तमः}
{अत्रोपनिषदं पुण्यां कृष्णद्वैपायनोऽब्रवीत्}


\threelineshloka
{विद्वद्भिः कथ्यते लोके पुराणे कविसत्तमैः}
{भारताध्ययनं पुण्यमपि पादमधीयतः}
{श्रद्दधानस्य पूयन्ते सर्वपापान्यशेषतः}


\twolineshloka
{देवा देवर्षयो ह्यत्र तथा ब्रह्मर्षयोऽमलाः}
{कीर्त्यन्ते शुमकर्माणस्तथा यक्षा महोरगाः}


\twolineshloka
{भगवान्वासुदेवश्च कीर्त्यतेऽत्र सनातनः}
{स हि सत्यमृतं चैव पवित्रं पुण्यमेव च}


\twolineshloka
{शाश्वतं ब्रह्म परमं ध्रुवं ज्योतिः सनातनम्}
{यस्य दिव्यानि कर्माणि कथन्ति मनीषिणः}


\twolineshloka
{असत्सत्सदसच्चैव यस्माद्विश्वं प्रवर्तते}
{सन्ततिश्च प्रवृत्तिश्च जन्ममृत्युपुनर्भवाः}


\twolineshloka
{अध्यात्मं श्रूयतें यत्र पञ्चभूतगुणात्मकम्}
{अव्यक्तादि परं यच्च स एव परिगीयते}


\twolineshloka
{यं ध्यायन्ति सदा मुक्ता ध्यानयोगबलान्विताः}
{प्रतिबिम्बमिवादर्शे पश्यन्त्यात्मन्यवस्थितम्}


\twolineshloka
{श्रद्दधानः सदा युक्तः सदा धर्मपरायणः}
{आसेवन्निममध्यायं नरः पापात्प्रमुच्यते}


\twolineshloka
{अनुक्रमणिकाध्यायं भारतस्येममादितः}
{आस्तिकः सततं शृण्वन्न कृच्छ्रेष्ववसीदति}


\twolineshloka
{उभे सन्ध्ये जपन्किंचित्सद्यो मुच्येत किल्बिषात्}
{अनुक्रमण्या यावत्स्यादह्नारात्र्या च संचितम्}


\twolineshloka
{भारतस्य वपुर्ह्येतत्सत्यं चामृतमेव च}
{नवनीतं यथा दध्नो द्विपदां ब्राह्मणो यथा}


\twolineshloka
{आरण्यकं च वेदेभ्य ओषधिभ्योऽमृतं यथा}
{ह्रदानामुदधिः श्रेष्ठो गौर्वरिष्ठा चतुष्पदाम्}


\twolineshloka
{यथैतानीतिहासानां तथा भारतमुच्यते}
{यश्चैनं श्रावयेच्छ्राद्धे ब्राह्मणान्पादमन्ततः}


\twolineshloka
{अक्षय्यमन्नपानं वै पितृंस्तस्योपतिष्ठते}
{इतिहासपुराणाभ्यां वेदं समुपबृंहयेत्}


\twolineshloka
{बिभेत्यल्पश्रुताद्वेदो मामयं प्रतरिष्यति}
{कार्ष्णं वेदमिमं विद्वाञ्श्रावयित्वार्थमश्नुते}


\twolineshloka
{भ्रूणहत्यादिकं चापि पापं जह्यादसंशयम्}
{य इमं शुचिरध्यायं पठेत्पर्वणि पर्वणि}


\twolineshloka
{अधीतं भारतं तेन कृत्स्नं स्यादिति मे मतिः}
{यश्चैनं शृणुयान्नित्यमार्षं श्रद्धासमन्वितः}


\twolineshloka
{स दीर्घमायुः कीर्तिं च स्वर्गतिं चाप्नुयान्नरः}
{एकतश्चतुरो वेदा भारतं चैतदेकतः}


\twolineshloka
{पुरा किल सुरैः सर्वैः समेत्य तुलया धृतम्}
{चतुर्भ्यः सरहस्येभ्यो वेदेभ्यो ह्यधिकं यदा}


\twolineshloka
{तदाप्रभृति लोकेऽस्मिन्महाभारतमुच्यते}
{महत्त्वे च गुरुत्वे च ध्रियमाणं यतोऽधिकम्}


\twolineshloka
{महत्त्वाद्भारवत्त्वाच्च महाभारतमुच्यते}
{निरुक्तमस्य यो वेद सर्वपापैः प्रमुच्यते}


\twolineshloka
{तपो नकल्कोऽध्ययनं नकल्कःस्वाभाविको वेदविधिर्नकल्कः}
{प्रसह्य वित्ताहरणं नकल्क-स्तान्येव भावोपहतानि कल्कः}


\chapter{अध्यायः २}
\twolineshloka
{ऋषय ऊचुः}
{}


\threelineshloka
{समन्तपञ्चकमिति यदुक्तं सूतनन्दन}
{एतत्सर्वं यथातत्त्वं श्रोतुमिच्छामहे वयम् ॥सौतिरुवाच}
{}


\twolineshloka
{शृणुध्वं मम भो विप्रा ब्रुवतश्च कथाः शुभाः}
{समन्तपञ्चकाख्यं च श्रोतुमर्हथ सत्तमाः}


\twolineshloka
{त्रेताद्वापरयोः सन्धौ रामः शस्त्रभृतां वरः}
{असकृत्पार्थिवं क्षत्रं जघानामर्षचोदितः}


\twolineshloka
{स सर्वं क्षत्रमुत्साद्य स्ववीर्येणानलद्युतिः}
{समन्तपञ्चके पञ्च चकार रुधिरह्रदान्}


\twolineshloka
{स तेषु रुधिराम्भःसु ह्रदेषु क्रोधमूर्च्छितः}
{पितॄन्संतर्पयामास रुधिरेणेति नः श्रुतम्}


\twolineshloka
{अथर्चीकादयोऽभ्येत्य पितरो राममब्रुवन्}
{राम राम महाभाग प्रीताः स्म तव भार्गव}


\threelineshloka
{अनया पितृभक्त्या च विक्रमेण तव प्रभो}
{वरं वृणीष्व भद्रं ते यमिच्छसि महाद्युते ॥राम उवाच}
{}


\twolineshloka
{यदि मे पितरः प्रीता यद्यनुग्राह्यता मयि}
{यच्च रोषाभिभूतेन क्षत्रमुत्सादितं मया}


\twolineshloka
{अतश्च पापान्मुच्येऽहमेष मे प्रार्थितो वरः}
{ह्रदाश्च तीर्थभूता मे भवेयुर्भुवि विश्रुताः}


\twolineshloka
{एवं भविष्यतीत्येवं पितरस्तमथाब्रुवन्}
{तं क्षमस्वेति निषिषिधुस्ततः स विरराम ह}


\twolineshloka
{तेषां समीपे यो देशो ह्रदानां रुधिराम्भसाम्}
{समन्तपञ्चकमिति पुण्यं तत्परिकीर्तितम्}


\twolineshloka
{येन लिङ्गेन यो देशो युक्तः समुपलक्ष्यते}
{तेनैव नाम्ना तं देशं वाच्यमाहुर्मनीषिणः}


\twolineshloka
{अन्तरे चैव संप्राप्ते कलिद्वापरयोरभूत्}
{समन्तपञ्चके युद्धं कुरुपाण्डवसेनयोः}


\twolineshloka
{तस्मिन्परमधर्मिष्ठे देशे भूदोषवर्जिते}
{अष्टादश समाजग्मुरक्षौहिण्यो युयुत्सया}


\twolineshloka
{समेत्य तं द्विजास्ताश्च तत्रैव निधं गताः}
{एतन्नामाभिनिर्वृत्तं तस्य देशस्य वै द्विजाः}


\threelineshloka
{पुण्यश्च रमणीयश्च स देशो वः प्रकीर्तितः}
{तदेतत्कथितं सर्वं मया ब्राह्मणसत्तमाः ॥यथा देशः स विख्यातस्त्रिषु लोकेषु सुव्रताः ॥ऋषय ऊचुः}
{}


\twolineshloka
{अक्षौहिण्य इति प्रोक्तं यत्त्वया सूतनन्दन}
{एतदिच्छामहे श्रोतुं सर्वमेव यथातथम्}


\threelineshloka
{अक्षौहिण्याः परीमाणं नराश्वरथदन्तिनाम्}
{यथावच्चैव नो ब्रूहि सर्वं हि विदितं तव ॥सौतिरुवाच}
{}


\twolineshloka
{एको रथो गजश्चैको नराः पञ्च पदातयः}
{त्रयश्च तुरगास्तज्ज्ञैः पत्तिरित्यभिधीयते}


\twolineshloka
{पत्तिं तु त्रिगुणामेतामाहुः सेनामुखं बुधाः}
{त्रीणि सेनामुखान्येको गुल्म इत्यभिधीयते}


\twolineshloka
{त्रयो गुल्मा गणो नाम वाहिनी तु गणास्त्रयः}
{स्मृतास्तिस्रस्तु वाहिन्यः पृतनेति विचक्षणैः}


\twolineshloka
{चमूस्तु पृतनास्तिस्रस्तिस्रश्चम्वस्त्वनीकिनी}
{अनीकिनीं दशगुणां प्राहुरक्षौहिणीं बुधाः}


\twolineshloka
{अक्षौहिण्याः प्रसंख्याता रथानां द्विजसत्तमाः}
{संख्यागणिततत्त्वज्ञैः सहस्राण्येकविंशतिः}


\twolineshloka
{शतान्युपरि चैवाष्टौ तथा भूयश्च सप्ततिः}
{गजानां च परीमाणमेतदेव विनिर्दिशेत्}


\twolineshloka
{ज्ञेयं शतसहस्रं तु सहस्राणि नवैव तु}
{नराणामपि पञ्चाशच्छतानि त्रीणि चानघाः}


\twolineshloka
{पञ्च षष्टिसहस्राणि तथाश्वानां शतानि च}
{दशोत्तराणि षट् प्राहुर्यथावदिह संख्यया}


\twolineshloka
{एतामक्षौहिणीं प्राहुः संख्यातत्त्वदितो जनाः}
{यां वः कथितवानस्मि विस्तरेण तपोधनाः}


\twolineshloka
{एतया संख्यया ह्यासन्कुरुपाण्डवसेनयोः}
{अक्षौहिण्यो द्विजश्रेष्ठाः पिण्डिताष्टादशैव तु}


\twolineshloka
{समेतास्तत्र वै देशे तत्रैव निधं गताः}
{कौरवान्कारणं कृत्वा कालेनाद्भुतकर्मणा}


\twolineshloka
{अहानि युयुधे भीष्मो दशैव परमास्त्रवित्}
{अहानि पञ्च द्रोणस्तु ररक्ष कुरुवाहिनीम्}


\twolineshloka
{अहनी युयुधे द्वे तु कर्णः परबलार्दनः}
{शल्योऽर्धदिवसं चैव गदायुद्धमतः परम्}


\twolineshloka
{तस्यैव दिवसस्यान्ते द्रौणिहार्दिक्यगौतमाः}
{प्रसुप्तं निशि विश्वस्तं जघ्नुर्यौधिष्ठिरं बलम्}


\twolineshloka
{यत्तु शौनक सत्रे ते भारताख्यानमुत्तमम्}
{जनमेजयस्य तत्सत्रे व्यासशिष्येण धीमता}


\twolineshloka
{कथितं विस्तरार्थं च यशो वीर्यं महीक्षिताम्}
{पौष्यं तत्र च पौलोममास्तीकं चादितः स्मृतम्}


\twolineshloka
{विचित्रार्थपदाख्यानमनेकसमयान्वितम्}
{प्रतिपन्नं नरैः प्राज्ञैर्वैराग्यमिव मोक्षिभिः}


\twolineshloka
{आत्मेव वेदितव्येषु प्रियेष्विव हि जीवितम्}
{इतिहासः प्रधानार्थः श्रेष्ठः सर्वागमेष्वयम्}


\twolineshloka
{अनाश्रित्येदमाख्यानं कथा भुवि न विद्यते}
{आहारमनपाश्रित्य शरीरस्येव धारणम्}


\twolineshloka
{तदेतद्भारतं नाम कविभिस्तूपजीव्यते}
{उदयतेप्सुभिर्भृत्यैरभिजात इवेश्वरः}


\twolineshloka
{इतिहासोत्तमे यस्मिन्नर्पिता बुद्धिरुत्तमा}
{स्वरव्यञ्जनयोः कृत्स्ना लोकवेदाश्रयेव वाक्}


\twolineshloka
{तस्य प्रज्ञाभिपन्नस्य विचित्रपदपर्वणः}
{सूक्ष्मार्थन्याययुक्तस्य वेदार्थैर्भूषितस्य च}


\twolineshloka
{भारतस्येतिहासस्य श्रूयतां पर्वसंग्रहः}
{पर्वानुक्रमणी पूर्वं द्वितीयः पर्वसंग्रहः}


\twolineshloka
{पौष्यं पौलोममास्तीकमादिरंशावतारणम्}
{ततः संभवपर्वोक्तमद्भुतं रोमहर्षणम्}


\twolineshloka
{दाहो जतुगृहस्यात्र हैडिम्बं पर्व चोच्यते}
{ततो बकवधः पर्व पर्व चैत्ररथं ततः}


\twolineshloka
{ततः स्वयंवरो देव्याः पाञ्चाल्याः पर्व चोच्यते}
{क्षात्रधर्मेण निर्जित्य ततो वैवाहिकं स्मृतम्}


\twolineshloka
{विदुरागमनं पर्व राज्यलाभस्तथैव च}
{अर्जुनस्य वने वासः सुभद्राहरणं ततः}


\twolineshloka
{सुभद्राहरणादूर्ध्वं ज्ञेया हरणहारिका}
{ततः खाण्डवदाहाख्यं तत्रैव मयदर्शनम्}


\twolineshloka
{सभापर्व ततः प्रोक्तं मन्त्रपर्व ततः परम्}
{जरासन्धवधः पर्व पर्व दिग्विजयं तथा}


\twolineshloka
{पर्व दिग्विजयादूर्ध्वं राजसूयिकमुच्यते}
{ततश्चार्घाभिहरणं शिशुपालवधस्ततः}


\twolineshloka
{द्यूतपर्व ततः प्रोक्तमनुद्यूतमः परम्}
{तत आरण्यकं पर्व किर्मीरवध एवच}


\twolineshloka
{अर्जुनस्याभिगमनं पर्व ज्ञेयमतः परम्}
{ईश्वरार्जुनयोर्युद्धं पर्व कैरातसंज्ञितम्}


\twolineshloka
{इन्द्रलोकाभिगमनं पर्व ज्ञेयमतः परम्}
{नलोपाख्यानमपि च धार्मिकं करुणोदयम्}


\twolineshloka
{तीर्थयात्रा ततः पर्व कुरुराजस्य धीमतः}
{जटासुरवधः पर्व यक्षयुद्धमतः परम्}


\twolineshloka
{निवातकवचैर्युद्धं पर्व चाजगरं ततः}
{मार्कण्डेयसमास्या च पर्वानन्तरमुच्यते}


\threelineshloka
{संवादश्च ततः पर्व द्रौपदीसत्यभामयोः}
{घोषयात्रा ततः पर्व ततः प्रायोपवेशनेम्}
{मन्त्रस्य निश्चयं चैव मृगस्वप्नोद्भवं ततः}


\twolineshloka
{व्रीहिद्रौणिकमाख्यानमैन्द्रद्युम्नं तथैव च}
{द्रौपदीहरणं पर्व जयद्रथविमोक्षणम्}


\twolineshloka
{रामोपाख्यानमत्रैव पर्व ज्ञेयमतः परम्}
{पतिव्रताया माहात्म्यं सावित्र्याश्चैवमद्भुतम्}


\twolineshloka
{कुण्डलाहरणं पर्व ततः परमिहोच्यते}
{आरणेयं ततः पर्व वैराटं तदनन्तरम् ॥पाण्डवानां प्रवेशश्च समयस्य च पालनम्}


\twolineshloka
{कीचकानां वधः पर्व पर्व ग्रोग्रहणं ततः}
{अभिमन्योश्च वैराट्याः पर्व वैवाहिकं स्मृतम्}


\twolineshloka
{उद्योगपर्व विज्ञेयमत ऊर्ध्वं महाद्भुतम्}
{ततः संजययानाख्यं पर्व ज्ञेयमतः परम्}


\twolineshloka
{प्रजागरं तथा पर्व धृतराष्ट्रस्य चिन्तया}
{पर्व सानत्सुजातं वै गुह्यमध्यात्मदर्शनम्}


\twolineshloka
{यानसन्धिस्ततः पर्व भगवद्यानमेव च}
{मातलीयमुपाख्यानं चरितं गालवस्य च}


\twolineshloka
{सावित्रं वामदेव्यं च वैन्योपाख्यानमेव च}
{जामदग्न्यमुपाख्यानं पर्व षोडशराजकम्}


\twolineshloka
{सभाप्रवेशः कृष्णस्य विदुलापुत्रशासनम्}
{उद्योगः सैन्यनिर्याणं विश्वोपाख्यानमेव च}


\twolineshloka
{ज्ञेयं विवादपर्वात्र कर्णस्यापि महात्मनः}
{`मन्त्रस्य निश्चयं कृत्वा कार्यस्य समनन्तरम्}


\twolineshloka
{श्वेतस्य वासुदेवेन चित्रं बहुकथाश्रयम्}
{'निर्याणं च ततः पर्व कुरुपाण्डवसेनयोः}


\twolineshloka
{रथातिरथसंख्या च पर्वोक्तं तदनन्तरम्}
{उलूकदूतागमनं पर्वामर्षविवर्धनम्}


% Check verse!
अम्बोपाख्यानमत्रैव पर्व ज्ञेयमतः परम् ॥भीष्माभिषेचनं पर्व ततश्चाद्भुतमुच्यते
% Check verse!
जम्बूखम्डविनिर्माणं पर्वोक्तं तदनन्तरम् ॥भूमिपर्व ततः प्रोक्तं द्वीपविस्तारकीर्तनम्
\threelineshloka
{`दिव्यं चक्षुर्ददौ यत्र संजयाय महामुनिः}
{'पर्वोक्तं भगवद्गीता पर्व भीष्मवधस्ततः}
{द्रोणाभिषेचनं पर्व संशप्तकवधस्ततः}


\twolineshloka
{अभिमन्युवधः पर्व प्रतिज्ञा पर्व चोच्यते}
{जयद्रथवधः पर्व घटोत्कचवधस्ततः}


\twolineshloka
{ततो द्रोणवधः पर्व विज्ञेयं लोमहर्षणम्}
{मोक्षो नारायणास्त्रस्य पर्वानन्तरमुच्यते}


\twolineshloka
{कर्णपर्व ततो ज्ञेयं शल्यपर्व ततः परम्}
{ह्रदप्रवेशनं पर्व गदायुद्धमतः परम्}


\twolineshloka
{सारस्वतं ततः पर्व तीर्थवंशानुकीर्तनम्}
{अत ऊर्ध्वं सुबीभत्सं पर्व सौप्तिकमुच्यते}


\twolineshloka
{ऐषीकं पर्व चोद्दिष्टमत ऊर्ध्वं सुदारुणम्}
{जलप्रदानिकं पर्व स्त्रीविलापस्ततः परम्}


\twolineshloka
{श्राद्धपर्व ततो ज्ञेयं कुरूणामौर्ध्वदेहिकम्}
{चार्वाकनिग्रहः पर्व रक्षसो ब्रह्मरूपिणः}


\twolineshloka
{आभिषेचनिकं पर्व धर्मराजसर्य धीमतः}
{प्रविभागो गृहाणां च पर्वोक्तं तदनन्तरम्}


\twolineshloka
{शान्तिपर्व ततो यत्र राजधर्मानुशासनम्}
{आपद्धर्मश्च पर्वोक्तं मोक्षधर्मस्ततः परम्}


\twolineshloka
{शुकप्रश्नाभिगमनं ब्रह्मप्रश्नानुशासनम्}
{प्रादुर्भावश्च दुर्वासःसंवादश्चैव मायया}


\twolineshloka
{ततः पर्व परिज्ञेयमानुशासनिकं परम्}
{स्वर्गारोहणिकं चैव ततो भीष्मस्य धीमतः}


\twolineshloka
{ततोऽश्वमेधिकं पर्व सर्वपापप्रणाशनम्}
{अनुगीता ततः पर्व ज्ञेयमध्यात्मवाचकम्}


\twolineshloka
{पर्व चाश्रमवासाख्यं पुत्रदर्शनमेव च}
{नारदागमनं पर्व ततः परमिहोच्यते}


\twolineshloka
{मौसलं पर्व चोद्दिष्टं ततो घोरं सुदारुणम्}
{महाप्रस्थानिकं पर्व स्वर्गारोहणिकं ततः}


\twolineshloka
{हरिवंशस्ततः पर्व पुराणं खिलसंज्ञितम्}
{विष्णुपर्व शिशोश्चर्या विष्णोः कंसवधस्तथा}


\twolineshloka
{भविष्यं पर्व चाप्युक्तं खिलेष्वेवाद्भुतं महत्}
{एतत्पर्वशतं पूर्णं व्यासेनोक्तं महात्मना}


\twolineshloka
{यथावत्सूतपुत्रेण रौमहर्षणिना ततः}
{उक्तानि नैमिशारण्ये पर्वाण्यष्टादशैव तु}


\twolineshloka
{समासो भारतस्यायमत्रोक्तः पर्वसंग्रहः}
{पौष्यं पौलोममास्तीकमादिरंशावतारणम्}


\twolineshloka
{संभवो जतुवेश्माख्यं हिडिम्बबकयोर्वधः}
{तथा चैत्ररथं देव्याः पाञ्चाल्याश्च स्वयंवरः}


\twolineshloka
{क्षात्रधर्मेण निर्जित्य ततो वैवाहिकं स्मृतम्}
{विदुरागमनं चैव राज्यलाभस्तथैव च}


\twolineshloka
{वनवासोऽर्जुनस्यापि सुभद्राहरणं ततः}
{हरणाहरणं चैव दहनं खाण्डवस्य च}


\twolineshloka
{मयस्य दर्शनं चैव आदिपर्वणि कथ्यते}
{पौष्ये पर्वणि माहात्म्यमुत्तङ्कस्योपवर्णितम्}


\twolineshloka
{पौलोमे भृगुवंशस्य विस्तारः परिकीर्तितः}
{आस्तीके सर्वनागानां गरुडस्य च संभवः}


\twolineshloka
{क्षीरोदमथं चैव जन्मोच्चैःश्रवसस्तथा}
{यजतः सर्पसत्रेण राज्ञः पारिक्षितस्य च}


\twolineshloka
{कथेयमभिनिर्वृत्ता भारतानां महात्मनाम्}
{विविधाः संभवा राज्ञामुक्ताः संभवपर्वणि}


\twolineshloka
{अन्येषां चैव शूराणामृषेर्द्वैपायनस्य च}
{अंशावतरणं चात्र देवानां परिकीर्तितम्}


\twolineshloka
{दैत्यानां दानवानां च यक्षाणां च महौजसाम्}
{नागानामथ सर्पाणां गन्धर्वाणां पतत्त्रिणाम्}


\twolineshloka
{अन्येषां चैव भूतानां विविधानां समुद्भवः}
{महर्षेराश्रमपदे कण्वस्य च तपस्विनः}


\twolineshloka
{शकुन्तलायां दुष्यन्ताद्भरतश्चापि जज्ञिवान्}
{यस्य लोकेषु नाम्नेदं प्रथितं भारतं कुलम्}


\twolineshloka
{वसूनां पुनरुत्पत्तिर्भागीरथ्यां महात्मनाम्}
{शन्तनोर्वेश्मनि पुनस्तेषां चारोहणं दिवि}


\twolineshloka
{तेजोंशानां च संपातो भीष्मस्याप्यत्र संभवः}
{राज्यान्निवर्तनं तस्य ब्रह्मचर्यव्रते स्थितिः}


\twolineshloka
{प्रतिज्ञापालनं चैव रक्षा चित्राङ्गदस्य च}
{हते चित्राङ्गदे चैव रक्षा भ्रातुर्यवीयसः}


\twolineshloka
{विचित्रवीर्यस्य तथा राज्ये संप्रतिपादनम्}
{धर्मस्य नृषु संभूतिरणीमाण्डव्यशापजा}


\twolineshloka
{कृष्णद्वैपायनाच्चैव प्रसूतिर्वरदानजा}
{धृतराष्ट्रस्य पाण्डोश्च पाण्डवानां च संभवः}


\twolineshloka
{वारणावतयात्रा च मन्त्रो दुर्योधनस्य च}
{कूटस्य धार्तराष्ट्रेण प्रेषणं पाण्डवान्प्रति}


\twolineshloka
{हितोपदेशश्च पथि धर्मराजस्य धीमतः}
{विदुरेण कृतो यत्र हितार्थं म्लेच्छभाषया}


\twolineshloka
{विदुरस्य च वाक्येन सुरुङ्गोपक्रमक्रिया}
{निषाद्याः पञ्चपुत्रायाः सुप्ताया जतुवेश्मनि}


\twolineshloka
{पुरोचनस्य चात्रैव दहनं संप्रकीर्तितम्}
{पाण्डवानां वने घोरे हिडिम्बायाश्च दर्शनम्}


\twolineshloka
{तत्रैव च हिडिम्बस्य वधो भीमान्महाबलात्}
{घटोत्कचस्य चोत्पत्तिंरत्रैव परिकीर्तिता}


\twolineshloka
{महर्षेर्दर्शनं चैव व्यासस्यामिततेजसः}
{तदाज्ञयैकचक्रायां ब्राह्मणस्य निवेशने}


\twolineshloka
{अज्ञातचर्यया वासो यत्र तेषां प्रकीर्तितः}
{बकस्य निधनं चैव नागराणां च विस्मयः}


\twolineshloka
{संभवश्चैव कृष्णाया धृष्टद्युम्नस्य चैव ह}
{ब्राह्मणात्समुपश्रुत्य व्यासवाक्यप्रचोदिताः}


\twolineshloka
{द्रौपदीं प्रार्थयन्तस्ते स्वयंवरदिदृक्षया}
{पञ्चालानभितो जग्मुर्यत्र कौतूहलान्विताः}


\twolineshloka
{अङ्गारपर्णं निर्जित्य गङ्गाकूलेऽर्जुनस्तदा}
{सख्यं कृत्वा ततस्तेन तस्मादेव च शुश्रुवे}


\twolineshloka
{तापत्यमथ वासिष्ठमौर्वं चाख्यानमुत्तमम्}
{भ्रातृभिः सहितः सर्वैः पञ्चालानभितो ययौ}


\twolineshloka
{पाञ्चालनगरे चापि लक्ष्यं भित्त्वा धनञ्जयः}
{द्रौपदीं लब्धवानत्र मध्ये सर्वमहीक्षिताम्}


\twolineshloka
{भीमसेनार्जुनौ यत्र संरब्धान्पृथिवीपतीन्}
{शल्यकर्णौ च तरसा जितवन्तौ महामृधे}


\twolineshloka
{दृष्ट्वा तयोश्च तद्वीर्यमप्रमेयममानुषम्}
{शङ्कमानौ पाण्डवांस्तान् रामकृष्णौ महामती}


\twolineshloka
{जग्मतुस्तैः समागन्तुं शालां भार्गववेश्मनि}
{पञ्चानामेकपत्नीत्वे विमर्शो द्रुपदस्य च}


\twolineshloka
{पञ्चेन्द्राणामुपाख्यानमत्रैवाद्भुतमुच्यते}
{द्रौपद्या देवविहीतो विवाहश्चाप्यमानुषः}


\twolineshloka
{क्षत्तुश्च धार्तराष्ट्रेण प्रेषणं पाण्डवान्प्रति}
{विदुरस्य च संप्राप्तिर्दर्शनं केशवस्य च}


\twolineshloka
{खाण्डवप्रस्थवासश्च तथा राज्यार्धसर्जनम्}
{नारदस्याज्ञया चैव द्रौपद्याः समयक्रिया}


\twolineshloka
{सुन्दोपसुन्दयोस्तद्वदाख्यानं परिकीर्तितम्}
{अनन्तरं च द्रौपद्या सहासीनं युधिष्ठिरम्}


\twolineshloka
{अनु प्रविश्य विप्रार्थे फाल्गुनो गृह्य चायुधम्}
{मोक्षयित्वा गृहं गत्वा विप्रार्थं कृतनिश्चयः}


\twolineshloka
{समयं पालयन्वीरो वनं यत्र जगाम ह}
{पार्थस्य वनवासे च उलूप्या पथि सङ्गमः}


\twolineshloka
{पुण्यतीर्थानुसंयानं बभ्रुवाहनजन्म च}
{तत्रैव मोक्षयामास पञ्च सोऽप्सरसः शुभाः}


\twolineshloka
{शापाद्ग्राहत्वमापन्ना ब्राह्मणस्य तपस्विनः}
{प्रभासतीर्थे पार्थेन कृष्णस्य च समागमः}


\twolineshloka
{द्वारकायां सुभद्रा च कामयानेन कामिनी}
{वासुदेवस्यानुमते प्राप्ता चैव किरीटिना}


\twolineshloka
{गृहीत्वा हरणं प्राप्ते कृष्णे देवकिनन्दने}
{अभिमन्योः सुभद्रायां जन्म चोत्तमतेजसः}


\twolineshloka
{द्रौपद्यास्तनयानां च संभवोऽनुप्रकीर्तितः}
{विहारार्थं च गतयोः कृष्णयोर्यमुनामनु}


\twolineshloka
{संप्राप्तिश्चक्रधनुषोः खाण्डवस्य च दाहनम्}
{मयस्य मोक्षो ज्वलनाद्भुजङ्गस्य च मोक्षणम्}


\twolineshloka
{महर्षेर्मन्दपालस्य शार्ङ्ग्या तनयसंभवः}
{इत्येतदादिपर्वोक्तं प्रथमं बहु विस्तरम्}


\twolineshloka
{अध्यायानां शते द्वे तु संख्याते परमर्षिणा}
{सप्तविंशतिरध्याया व्यासेनोत्तमतेजसा}


\twolineshloka
{अष्टौ श्लोकसहस्राणि अष्टौ श्लोकशतानि च}
{श्लोकाश्च चतुराशीतिर्मुनिनोक्ता महात्मना}


\threelineshloka
{द्वितीयं तु सभापर्व बहुवृत्तान्तमुच्यते}
{सभाक्रिया पाण्डवानां किङ्कराणां च दर्शनम्}
{}


\twolineshloka
{लोकपालसभाख्यानं नारदाद्देवदर्शिनः}
{राजसूयस्य चारम्भो जरासन्धवधस्तथा}


\twolineshloka
{गिरिव्रजे निरुद्धानां राज्ञां कृष्णेन मोक्षणम्}
{तथा दिग्विजयोऽत्रैव पाम्डवानां प्रकीर्तितः}


\twolineshloka
{राज्ञामागमनं चैव सार्हणानां महक्रतौ}
{राजसूयेऽर्घसंवादे शिशुपालवधस्तथा}


\twolineshloka
{यज्ञे विभूतिं तां दृष्ट्वा दुःखामर्षान्वितस्य च}
{दुर्योधनस्यावहासो भीमेन च सभातले}


\twolineshloka
{यत्रास्य मन्युरुद्भूतो येन द्यूतमकारयत्}
{यत्र धर्मसुतं द्यूते शकुनिः कितवोऽजयत्}


\twolineshloka
{यत्र द्यूतार्णवे मग्नां द्रौपदीं नौरिवार्णवात्}
{धृतराष्ट्रो महाप्राज्ञः स्नुषां परमदुःखिताम्}


\twolineshloka
{तारयामास तांस्तीर्णाञ्ज्ञात्वा दुर्योधनो नृपः}
{पुनरेव ततो द्यूते समाह्वयत पाण्डवान्}


\twolineshloka
{जित्वा स वनवासाय प्रेषयामास तांस्ततः}
{एतत्सर्वं सभापर्व समाख्यातं महात्मना}


\twolineshloka
{अध्यायाः सप्ततिर्ज्ञेयास्तथा चाष्टौ प्रसंख्यया}
{श्लोकानां द्वे सहस्रे तु पञ्च श्लोकशतानि च}


\twolineshloka
{श्लोकाश्चैकादश ज्ञेयाः पर्वण्यस्मिन्द्विजोत्तमाः}
{अतः परं तृतीयं तु ज्ञेयमारण्यकं महत्}


\twolineshloka
{वनवासं प्रयातेषु पाण्डवेषु महात्मसु}
{पौरानुगमनं चैव धर्मपुत्रस्य धीमतः}


\twolineshloka
{अन्नौषधीनां च कृते पाण्डवेन महात्मना}
{द्विजानां भरणार्थं च कृतमाराधनं रवेः}


\twolineshloka
{धौम्योपदेशात्तिग्मांशुप्रसादादन्नसंभः}
{हितं च ब्रुवतः क्षत्तुः परित्यागोऽम्बिकासुतात्}


\twolineshloka
{त्यक्तस्य पाण्डुपुत्राणां समीपगमनं तथा}
{पुनरागमनं चैव धृतराष्ट्रस्य शासनात्}


\twolineshloka
{कर्णप्रोत्साहनाच्चैव धार्तराष्ट्रस्य दुर्मतेः}
{वनस्थान्पाण्डवान्हन्तुं मन्त्रो दुर्योधनस्यच}


\twolineshloka
{तं दुष्टभावं विज्ञाय व्यासस्यागमनं द्रुतम्}
{निर्याणप्रतिषेधश्च सुरभ्याख्यानमेव च}


\twolineshloka
{मैत्रेयागमनं चात्र राज्ञश्चैवानुशासनम्}
{शापोत्सर्गश्च तेनैव राज्ञो दुर्योधनस्य च}


\twolineshloka
{किर्मीरस्य वधश्चात्र भीमसेनेन संयुगे}
{वृष्णीनामागमश्चात्र पाञ्चालानां च सर्वशः}


\twolineshloka
{श्रुत्वा शकुनिना द्यूते निकृत्या निर्जितांश्च तान्}
{क्रुद्धस्यानुप्रशमनं हरेश्चैव किरीटिना}


\twolineshloka
{परिदेवनं च पाञ्चाल्या वासुदेवस्य सन्निधौ}
{आश्वासनं च कृष्णेन दुःखार्तायाः प्रकीर्तितम्}


\twolineshloka
{तथा सौभवधाख्यानमत्रैवोक्तं महर्षिणा}
{सुभद्रायाः सपुत्रायाः कृष्णेन द्वारकां पुरीम्}


\twolineshloka
{नयनं द्रौपदेयानां धृष्टद्युम्नेन चैव ह}
{प्रवेशः पाण्डवेयानां रम्ये द्वैतवने ततः}


\twolineshloka
{धर्मराजस्य चात्रैव संवादः कृष्णया सह}
{संवादश्च तथा राज्ञा भीमस्यापि प्रकीर्तितः}


\twolineshloka
{समीपं पाण्डुपुत्राणां व्यासस्यागमनं तथा}
{प्रतिश्रुत्याथ विद्याया दानं राज्ञो महर्षिणा}


\twolineshloka
{गमनं काम्यके चापि व्यासे प्रतिगते ततः}
{अस्त्रहेतोर्विवासश्च पार्थस्यामिततेजसः}


\twolineshloka
{महादेवेन युद्धं च किरातवपुषा सह}
{दर्शनं लोकपालानामस्त्रप्राप्तिस्तथैव च}


\twolineshloka
{महेन्द्रलोकगमनमस्त्रार्थे च किरीटिनः}
{यत्र चिन्ता समुत्पन्ना धृतराष्ट्रस्य भूयसी}


\twolineshloka
{दर्शनं बृहदश्वस्य महर्षेर्भावितात्मनः}
{युधिष्ठिरस्य चार्तस्य व्यसने परिदेवनम्}


\twolineshloka
{नलोपाख्यानमत्रैव धर्मिष्ठं करुणोदयम्}
{दमयन्त्याः स्थितिर्यत्र नलस्य चरितं तथा}


\twolineshloka
{तथाक्षहृदयप्राप्तिस्तस्मादेव महर्षितः}
{लोमशस्यागमस्तत्र स्वर्गात्पाण्डुसुतान्प्रति}


\twolineshloka
{वनवासगतानां च पाण्डवानां महात्मनाम्}
{स्वर्गे प्रवृत्तिराख्याता लोमशेनार्जुनस्य वै}


\twolineshloka
{संदेशादर्जुनस्यात्र तीर्थाभिगमनक्रिया}
{तीर्थानां च फलप्राप्तिः पुण्यत्वं चापि कीर्तितम्}


\twolineshloka
{पुलस्त्यतीर्थयात्रा च नारदेन महर्षिणा}
{तीर्थयात्रा च तत्रैव पाण्डवानां महात्मनां}


% Check verse!
तथा यज्ञविभूतिश्च गयस्यात्र प्रकीर्तिता
\twolineshloka
{आगस्त्यमपि चाख्यानं यत्र वातापिभक्षणम्}
{लोपामुद्राभिपमनमपत्यार्थमृषेस्तथा}


\twolineshloka
{ऋश्यशृङ्गस्य चरितं कौमारब्रह्मचारिणः}
{जामदग्न्यस्य रामस्य चरितं भूरितेजसः}


\twolineshloka
{कार्तवीर्यवधो यत्र हैहयानां च वर्ण्यते}
{प्रभासतीर्थे पाण्डूनां वृष्णिभिश्च समागमः}


\twolineshloka
{सौकन्यमपि चाख्यानं च्यवनो यत्र भार्गवः}
{शर्यातियज्ञे नासत्यौ कृतवान्सोमपीथिनौ}


\twolineshloka
{ताभ्यां च यत्र स मुनिर्यौवनं प्रतिपादितः}
{मान्धातुश्चाप्युपाख्यानं राज्ञोऽत्रैवप्रकीर्तितं}


\twolineshloka
{जन्तूपाख्यानमत्रैव यत्र पुत्रेण सोमकः}
{पुत्रार्थमयजद्राजा लेभे पुत्रशतं च सः}


\twolineshloka
{ततः श्येनकपोतीयमुपाख्यानमनुत्तमम्}
{इन्द्राग्नी यत्र धर्मश्चाप्यजिज्ञासञ्शिबिं नृपम्}


\twolineshloka
{अष्टावक्रीयमत्रैव विवादो यत्र बन्दिना}
{अष्टावक्रस्य विप्रर्षेर्जनकस्याध्वरेऽभवत्}


\twolineshloka
{नैयायिकानां मुख्येन वरुणस्यात्मजेन च}
{पराजितो यत्र बन्दी विवादेन महात्मना}


\threelineshloka
{विजित्य सागरं प्राप्तं पितरं लब्धवानृषिः}
{यवक्रीतस्य चाख्यानं रैभ्यस्य च महात्मनः}
{गन्धमादनयात्रा च वासो नारायणाश्रमे}


\twolineshloka
{नियुक्तो भीमसेनश्च द्रौपद्या गन्धमादने}
{व्रजन्पथि महाबाहुर्दृष्टवान्पवनात्मजम्}


\twolineshloka
{कदलीषण्डमध्यस्थं हनूमन्तं महाबलम्}
{यत्र मन्दारपुष्पार्थे नलिनीं तामधर्षयत्}


\twolineshloka
{यत्रास्य युद्धमभवत्सुमहद्राक्षसैः सह}
{यक्षैश्चैव महावीर्यैर्मणिमत्प्रमुखैस्तथा}


\twolineshloka
{जटासुरस्य च वधो राक्षसस्य वृकोदरात्}
{वृषपर्वणो राजर्षेस्ततोऽभिगमनं स्मृतम्}


\twolineshloka
{आर्ष्टिषेणाश्रमे चैषां गमनं वास एव च}
{प्रोत्साहनं च पाञ्चाल्या भीमस्यात्र महात्मनः}


\twolineshloka
{कैलासारोहणं प्रोक्तं यत्र यक्षैर्बलोत्कटैः}
{युद्धमासीन्महाघोरं मणिमत्प्रमुखैः सह}


\twolineshloka
{समागमश्च पाण्डूनां यत्र वैश्रवणेन च}
{समागमश्चार्जुनस्य तत्रैव भ्रातृभिः सह}


\twolineshloka
{अवाप्य दिव्यान्यस्त्राणि गुर्वर्थं सव्यसाचिना}
{निवातकवचैर्युद्धं हिरण्यपुरवासिभिः}


\twolineshloka
{निवातकवचैर्घोरैर्दानवैः सुरशत्रुभिः}
{पौलोमैः कालकेयैश्च यत्र युद्धं किरीटिनः}


\twolineshloka
{वधश्चैषां समाख्यातो राज्ञस्तेनैव धीमता}
{अस्त्रसंदर्शनारम्भो धर्मराजस्य सन्निधौ}


\twolineshloka
{पार्थस्य प्रतिषेधश्छ नारदेन सुरर्षिणा}
{अवरोहणं पुनश्चैव पाण्डूनां गन्धमादनात्}


\twolineshloka
{भीमस्य ग्रहणं चात्र पर्वताभोगवर्ष्मणा}
{भुजगेन्द्रेण बलिना तस्मिन्सुगहने वने}


\twolineshloka
{अमोक्षयद्यत्र चैनं प्रश्नानुक्त्वा युधिष्ठिरः}
{काम्यकागमनं चैव पुनस्तेषां महात्मनाम्}


\twolineshloka
{तत्रस्थांश्च पुनर्द्रष्टुं पाण्डवान्परुषर्षभान्}
{वासुदेवस्यागमनमत्रैव परिकीर्तितम्}


\twolineshloka
{मार्कण्डेयसमास्यायामुपाख्यानानि सर्वशः}
{पृथोर्वैन्यस्य यत्रोक्तमाख्यानं परमर्षिणा}


\twolineshloka
{संवादश्च सरस्वत्यास्तार्क्ष्यर्षेः सुमहात्मनः}
{मत्स्योपाख्यानमत्रैव प्रोच्यते तदनन्तरम्}


\twolineshloka
{मार्कण्डेयसमास्या च पुराणं परिकीर्त्यते}
{ऐन्द्रद्युम्नामुपाख्यानं तथैवाङ्गिरसं स्मृतम्}


\twolineshloka
{पतिव्रतायाश्चाख्यानं तथैवाङ्गिरसं स्मृतम्}
{द्रौपद्याः कीर्तितश्चात्र संवादः सत्यभामया}


\twolineshloka
{पुनर्द्वैतवनं चैव पाण्डवाः समुपागताः}
{घोषयात्रा च गन्धर्वैर्यत्र बद्धः सुयोधनः}


\twolineshloka
{ह्रियमाणस्तु मन्दात्मा मोक्षितोऽसौ किरीटिना}
{धर्मराजस्य चात्रैव मृगस्वप्ननिदर्शनात्}


\twolineshloka
{काम्यके काननश्रेष्ठे पुनर्गमनमुच्यते}
{व्रीहिद्रौणिकमाख्यानमत्रैव बहुविस्तरम्}


\twolineshloka
{दुर्वाससोऽप्युपाख्यानमत्रैव परिकीर्तितम्}
{जयद्रथेनापहारो द्रौपद्याश्चाश्रमान्तरात्}


\twolineshloka
{यत्रैनमन्वयाद्भीमो वायुवेगसमो जवे}
{चक्रे चैनं पञ्चशिखं यत्र भीमो महाबलः}


\twolineshloka
{रामायणमुपाख्यानमत्रैव बहुविस्तरम्}
{यत्र रामेण विक्रम्य निहतो रावणो युधि}


\twolineshloka
{सावित्र्याश्चाप्युपाख्यानमत्रैव परिकीर्तितम्}
{कर्णस्य परिमोक्षोऽत्र कुण्डलाभ्यां पुरंदरात्}


\twolineshloka
{यत्रास्य शक्तिं तुष्टोऽसावदादेकवधाय च}
{आरणेयमुपाख्यानं यत्र धर्मोऽन्वशात्सुतम्}


\twolineshloka
{जग्मुर्लब्धवरा यत्र पाण्डवाः पश्चिमां दिशम्}
{एतदारण्यकं पर्व तृतीयं परिकीर्तितम्}


\twolineshloka
{अत्राध्यायशते द्वे तु संख्यया परिकीर्तिते}
{एकोनसप्ततिश्चैव तथाऽध्यायाः प्रकीर्तिताः}


\twolineshloka
{एकादश सहस्राणि श्लोकानां षट् शतानि च}
{चतुःषष्टिस्तथा श्लोकाः पर्वण्यस्मिन्प्रकीर्तिताः}


\twolineshloka
{अतः परं निबोधेदं वैराटं पर्व विस्तरम्}
{विराटनगरे गत्वा श्मशाने विपुलां शमीम्}


\twolineshloka
{दृष्ट्वा संनिदधुस्तत्र पाण्डवा ह्यायुधान्युत}
{यत्र प्रविश्य नगरं छद्मना न्यवसंस्तु ते}


\twolineshloka
{पाञ्चालीं प्रार्थयानस्य कामोपहतचेतसः}
{दुष्टात्मनो वधो यत्र कीचकस्य वृकोदरात्}


\twolineshloka
{पाण्डवान्वेषणार्थं च राज्ञो दुर्योधनस्य च}
{चाराः प्रस्थापिताश्चात्र निपुणाः सर्वतोदिशं}


\twolineshloka
{न च प्रवृत्तिस्तैर्लब्धा पाण्डवानां महात्मनाम्}
{गोग्रहश्च विराटस्य त्रिगर्तैः प्रथमं कृतः}


\twolineshloka
{यत्रास्य युद्धं सुमहत्तैरासील्लोमहर्षणम्}
{ह्रियमाणश्च यत्रासौ भीमसेनेन मोक्षितः}


\twolineshloka
{गोधनं च विराटस्य मोक्षितं यत्र पाण्डवैः}
{अनन्तरं च कुरुभिस्तस्य गोग्रहणं कृतम्}


\twolineshloka
{समस्ता यत्र पार्थेन निर्जिताः कुरवो युधि}
{प्रत्याहृतं गोधनं च विक्रमेण किरीटिना}


\twolineshloka
{विराटेनोत्तरा दत्ता स्नुषा यत्र किरीटिनः}
{अभिमन्युं समुद्दिश्य सौभद्रमरिघातिनम्}


\twolineshloka
{चतुर्थमेतद्विपुलं वैराटं पर्व वर्णितम्}
{अत्रापि परिसंख्याता अध्यायाः परमर्षिणा}


\twolineshloka
{सप्तषष्टिरथो पूर्णाः श्लोकानामपि मे शृणु}
{श्लोकानां द्वे सहस्रे तु श्लोकाः पञ्चाशदेव तु}


\twolineshloka
{उक्तानि वेदविदुषा पर्वण्यस्मिन्महर्षिणा}
{उद्योगपर्व विज्ञेयं पञ्चमं शृण्वतः परम्}


\twolineshloka
{उपप्लाव्ये निविष्टेषु पाण्डवेषु जिगीषया}
{दुर्योधनोऽर्जुनश्चैव वासुदेवमुपस्थितौ}


\twolineshloka
{साहाय्यमस्मिन्समरे भवान्नौ कर्तुमर्हति}
{इत्युक्ते वचने कृष्णो यत्रोवाच महामतिः}


\twolineshloka
{अयुध्यमानमात्मानं मन्त्रिणं पुरुषर्षभौ}
{अक्षौहिणीं वा सैन्यस्य कस्य किं वा ददाम्यहम्}


\twolineshloka
{वव्रे दुर्योधनः सैन्यं मन्दात्मा यत्र दुर्मतिः}
{अयुध्यभानं सचिवं वव्रे कृष्मं धनंजयः}


\twolineshloka
{मद्रराजं व राजानमायान्तं पाण्डवान्प्रति}
{उपहारैर्वञ्चायत्वा वर्त्मन्येव सुयोधनः}


\twolineshloka
{वरदं तं वरं वव्रे साहाय्यं क्रियतां मम}
{शल्यस्तस्मै प्रतिश्रुत्य जगामोद्दिश्य पाण्डवान्}


\twolineshloka
{शान्तिपूर्वं चाकथयद्यत्रेन्द्रविजयं नृपः}
{पुरोहितप्रेषणं च पाण्डवैः कौरवान्प्रति}


\twolineshloka
{वैचित्रवीर्यस्य वचः समादाय पुरोधसः}
{तथेन्द्रविजयं चापि यानं चैव पुरोधसः}


\twolineshloka
{संजयं प्रेषयामास शमार्थी पाण्डवान्प्रति}
{यत्र दूतं महाराजो धृतराष्ट्रः प्रतापवान्}


\twolineshloka
{श्रुत्वा च पाण्डवान्यत्र वासुदेवपुरोगमान्}
{प्रजागरः संप्रजज्ञे धृतराष्ट्रस्य चिन्तया}


\twolineshloka
{विदुरो यत्र वाक्यानि विचित्राणि हितानि च}
{श्रावयामास राजानं धृतराष्ट्रं मनीषिणम्}


\twolineshloka
{तथा सनत्सुजातेन यत्राध्यात्ममनुत्तमम्}
{मनस्तापान्वितो राजा श्रावितः शोकलालसः}


\twolineshloka
{प्रभाते राजसमितौ संजयो यत्र वा विभो}
{ऐकात्म्यं वासुदेवस्य प्रोक्तवानर्जुनस्य च}


\twolineshloka
{यत्र कृष्णो दयापन्नः सन्धिमिच्छन्महामतिः}
{स्वयमागाच्छणं कर्तुं नगरं नागसाह्वयम्}


\twolineshloka
{प्रत्याख्यानं च कृष्णस्य राज्ञा दुर्योधनेन वै}
{शमार्थे याचमानस्य पक्षयोरुभयोर्हितम्}


\twolineshloka
{दम्भोद्भवस्य चाख्यानमत्रैव परिकीर्तितम्}
{वरान्वेषणमत्रैव मातलेश्च महात्मनः}


\twolineshloka
{महर्षेश्चापि चरितं कथितं गालवस्य वै}
{विदुलायाश्च पुत्रस्य प्रोक्तं चाप्यनुशासनम्}


\twolineshloka
{कर्णदुर्योधनादीनां दुष्टं विज्ञाय मन्त्रितम्}
{योगेश्वरत्पं कृष्णेन यत्र राज्ञां प्रदर्शितम्}


\twolineshloka
{रथमारोप्य कृष्णेन यत्र कर्णोऽनुमन्त्रितः}
{उपायपूर्वं शौटीर्यात्प्रत्याख्यातश्च तेन सः}


\twolineshloka
{आगम्य हास्तिनपुरादुपप्लाव्यमरिंदमः}
{पाण्डवानां यथावृत्तं सर्वमाख्यातवान्हरिः}


\twolineshloka
{ते तस्य वचनं श्रुत्वा मन्त्रयित्वा च यद्धितम्}
{साङ्ग्रामिकं ततः सर्वं सञ्जं चक्रुः परंतपाः}


\twolineshloka
{ततो युद्धाय निर्याता नराश्वरथदन्तिनः}
{नगराद्धास्तिनपुराद्वलसंख्यानमेवच}


\twolineshloka
{यत्र राज्ञा ह्युलूकस्य प्रेषणं पाम्डवान्प्रति}
{श्वोभाविनि महायुद्धे दौत्येन कृतवान्प्रभुः}


\twolineshloka
{रथातिरथसंख्यानमम्बोपाख्यानमेव च}
{एतत्सुबहुवृत्तान्तं पञ्चमं पर्व भारते}


\twolineshloka
{उद्योगपर्व निर्दिष्टं सन्धिविग्रहमिश्रितम्}
{अध्यायानां शतं प्रोक्तं षडशीतिर्महर्षिणा}


\twolineshloka
{श्लोकानां षट् सहस्राणि तावन्त्येव शतानि च}
{श्लोकाश्च नवतिः प्रोक्तास्तथैवाष्टौ महात्मना}


\twolineshloka
{व्यासेनोदारमतिना पर्वण्यस्मिंस्तपोधनाः}
{अतः परं विचित्रार्थं भीष्मपर्व प्रचक्षते}


\twolineshloka
{जम्बूखण्डविनिर्माणं यत्रोक्तं संजयेन ह}
{यत्र यौधिष्ठिरं सैन्यं विषादमगमत्परम्}


\twolineshloka
{यत्र युद्धमभूद्धोरं दसाहानि सुदारुणम्}
{कश्मलं यत्र पार्थस्य वासुदेवो महामतिः}


\twolineshloka
{मोहजं नाशयामास हेतुभिर्मोक्षदर्शिभिः}
{समीक्ष्यादोक्षजः क्षिप्रं युधिष्ठिरहिते रतः}


\twolineshloka
{रथादाप्लुत्य वेगेन स्वयं कृष्ण उदारधीः}
{प्रतोदपाणिराधावद्भीष्मं हन्तुं व्यपेतभीः}


\twolineshloka
{वाक्यप्रतोदाभिहतो यत्र कृष्णेन पाण्डवः}
{गाण्डीवधन्वा समरे सर्वशस्त्रभृतां वरः}


\twolineshloka
{शिखण्डिनं पुरस्कृत्य यत्र पार्थो महाधनुः}
{विनिघ्नन्निशितैर्बाणै रथाद्भीष्ममपातयत्}


\twolineshloka
{शरतल्पगतश्चैव भीष्मो यत्र बभूव ह}
{षष्ठमेतत्समाख्यातं भारते पर्व विस्तृतम्}


\twolineshloka
{अध्यायानां शतं प्रोक्तं तथा सप्तदशापरे}
{पञ्च श्लोकसहस्राणि संख्ययाष्टौ शतानि च}


\twolineshloka
{श्लोकाश्च चतुराशीतिरस्मिन्पर्वणि कीर्तिताः}
{व्यासेन वेदविदुषा संख्याता भीष्मपर्वणि}


\twolineshloka
{द्रोणपर्व ततश्चित्रं बहुवृत्तान्तमुच्यते}
{सैनापत्येऽभिषिक्तोऽथ यत्राचार्यः प्रतापवान्}


\twolineshloka
{दुर्योधनस्य प्रीत्यर्थं प्रतिजज्ञे महास्त्रवित्}
{ग्रहणं धर्मराजस्य पाण्डुपुत्रस्य धीमतः}


\twolineshloka
{यत्र संशप्तकाः पार्थमपनिन्यू रणाजिरात्}
{भगदत्तो महाराजो यत्र शक्रसमो युधि}


\twolineshloka
{सुप्रतीकेन नागेन स हि शान्तः किरीटिना}
{यत्राभिमन्युं बहवो जघ्नुरेकं महारथाः}


\twolineshloka
{जयद्रथमुखा बालं शूरमप्राप्तयौवनम्}
{हतेऽभिमन्यौ क्रुद्धेन यत्र पार्थेन संयुगे}


\twolineshloka
{अक्षौहिणीः सप्त हत्वा हतो राजा जयद्रथः}
{यत्र भीमो महाबाहुः सात्यकिश्च महारथः}


\twolineshloka
{अन्वेषणार्थं पार्थस्य युधिष्ठिरनृपाज्ञया}
{प्रविष्टौ भारतीं सेनामप्रधृष्यां सुरैरपि}


\twolineshloka
{संशप्तकावशेषं च कृतं निःशेषमाहवे}
{संशप्तकानां वीराणां कोट्यो नव महात्मनाम्}


\twolineshloka
{किरीटिनाभिनिष्क्रम्य प्रापिता यमसादनम्}
{धृतराष्ट्रस्य पुत्राश्च तथा पाषाणयोधिनः}


\twolineshloka
{नारायणाश्च गोपालाः समरे चित्रयोधिनः}
{अलम्बुषः श्रुतायुश्च जलसन्धश्च वीर्यवान्}


\twolineshloka
{सौमदत्तिर्विराटश्च द्रुपदश्च महारथः}
{घटोत्कचादयश्चान्ये निहता द्रोणपर्वणि}


\twolineshloka
{अश्वत्थामापि चात्रैव द्रोणे युधि निपातिते}
{अस्त्रं प्रादुश्चकारोग्रं नारायणममर्षितः}


\twolineshloka
{आग्नेयं कीर्त्यते यत्र रुद्रमाहात्म्यमुत्तमम्}
{व्यासस्य चाप्यागमनं माहात्म्यं कृष्णपार्थयोः}


\twolineshloka
{सप्तमं भारते पर्व महदेतदुदाहृतम्}
{यत्र ते पृथिवीपालाः प्रायशो निधनं गताः}


\twolineshloka
{द्रोणपर्वणि ये शऊरा निर्दिष्टाः पुरुषर्षभाः}
{अत्राध्यायशतं प्रोक्तं तथाध्यायाश्च सप्ततिः}


\twolineshloka
{अष्टौ श्लोकसहस्राणि तथा नव शतानि च}
{श्लोका नव तथैवात्र संख्यातास्तत्त्वदर्शिना}


\twolineshloka
{पाराशर्येण मुनिनां संचिन्त्य द्रोणपर्वणि}
{अतः परं कर्णपर्व प्रोच्यते परमाद्भुतम्}


\twolineshloka
{सारथ्ये विनियोगश्च मद्रराजस्य धीमतः}
{आख्यातं यत्र पौरामं त्रिपुरस्य निपातनम्}


\twolineshloka
{प्रयाणे परुषश्चात्र संवादः कर्णशल्ययोः}
{हंसकाकीयमाख्यानं तत्रैवाक्षेपसंहितम्}


\twolineshloka
{वधः पाण्ड्यस्य च तथा अश्वत्थाम्ना महात्मना}
{दण्डसेनस्य च ततो दण्डस्य च वधस्तथा}


\twolineshloka
{द्वैरथे यत्र कर्णेन धर्मराजो युधिष्टिरः}
{संशयं गमितो युद्धे मिषतां सर्वधन्विनाम्}


\twolineshloka
{अन्योन्यं प्रति च क्रोधो युधिष्ठिरकिरीटिनोः}
{यत्रैवानुनयः प्रोक्तो माधवेनार्जुनस्य हि}


\twolineshloka
{प्रतिज्ञापूर्वकं चापि वक्षो दुःशासनस्य च}
{भित्त्वा वृकोदरो रक्तं पीतवान्यत्र संयुगे}


\twolineshloka
{द्वैरथे यत्र पार्थेन हतः कर्णो महारथः}
{अष्टमं पर्व निर्दिष्टमेतद्भारतचिन्तकैः}


\twolineshloka
{एकोनसप्ततिः प्रोक्ता अध्यायाः कर्णपर्वणि}
{चत्वार्येव सहस्राणि नव श्लोकशतानि च}


\twolineshloka
{चतुःषष्टिस्तथा श्लोकाः पर्वण्यस्मिन्प्रकीर्तिताः}
{अतः परं विचित्रार्थं शल्यपर्व प्रकीर्तितम्}


\twolineshloka
{हतप्रवीरे सैन्ये तु नेता मद्रेश्वरोऽभवत्}
{यत्र कौमारमाख्यानमभिषेकस्य कर्म च}


\twolineshloka
{वृत्तानि चाथ युद्धानि कीर्त्यन्ते यत्र भागशः}
{विनाशः कुरुमुख्यानां शल्यपर्वणि कीर्त्यते}


\twolineshloka
{शल्यस्य निधनं चात्र धर्मराजान्महात्मनः}
{शकुनेश्च वधोऽत्रैव सहदेवेन संयुगे}


\twolineshloka
{सैन्ये च हतभूयिष्ठे किंचिच्छिष्टे सुयोधनः}
{ह्रदं प्रविश्य यत्रासौ संस्तभ्यापोव्यवस्थितः}


\twolineshloka
{प्रवृत्तिस्तत्र चाख्याता यत्र भीमस्य लुब्धकैः}
{क्षेपयुक्तैर्वचोभिश्च धर्मराजस्य धीमतः}


\twolineshloka
{ह्रदात्समुत्थितो यत्र धार्तराष्ट्रोऽत्यमर्षणः}
{भीमेन गदया युद्धं यत्रासौ कृतवान्सह}


\twolineshloka
{समवाये च युद्धस्य रामस्यागमनं स्मृतम्}
{सरस्वत्याश्च तीर्थानां पुण्यता परिकीर्तिता}


\twolineshloka
{गदायुद्धं च तुमुलमत्रैव परिकीर्तितम्}
{दुर्योधनस्य राज्ञोऽथ यत्र भीमेन संयुगे}


\twolineshloka
{ऊरू भग्नौ प्रसह्याजौ गदया भीमवेगया}
{नवमं पर्व निर्दिष्टमेतदद्भुतमर्थवत्}


\twolineshloka
{एकोनपष्टिरध्यायाः पर्वण्यत्र प्रकीर्तिताः}
{संख्याता बहुवृत्तान्ताः श्लोकसंख्याऽत्र कथ्यते}


\twolineshloka
{त्रीणि श्लोकसहस्राणि द्वे शते विंशतिस्तथा}
{मुनिना संप्रणीतानि कौरवाणां यशोभृता}


\twolineshloka
{अतः परं प्रवक्ष्यामि सौप्तिकं पर्व दारुणम्}
{भग्नोरुं यत्र राजानं दुर्योधनममर्षणम्}


\twolineshloka
{अपयातेषु पार्थेषु त्रयस्तेऽभ्याययू रथाः}
{कृतवर्मा कृपो द्रौणिः सायाह्ने रुधिरोक्षितम्}


\twolineshloka
{समेत्य ददृशुर्भूमौ पतितं रणमूर्धनि}
{प्रतिजज्ञे दृढक्रोधो द्रौणिर्यत्र महारथः}


\twolineshloka
{अहत्वा सर्वपाञ्चालान्धृष्टद्युम्नपुरोगमान्}
{पाण्डवांश्च सहामात्यान्न विमोक्ष्यामि दंशनं}


\twolineshloka
{यत्रैवमुक्त्वा राजानमपक्रम्य त्रयो रथाः}
{सूर्यास्तमनवेलायामासेदुस्ते महद्वनम्}


\twolineshloka
{न्यग्रोधस्याथ महतो यत्राधस्ताद्व्यवस्थिताः}
{ततः काकान्बहून्रात्रौ दृष्ट्वोलूकेन हिंसितान्}


\twolineshloka
{द्रौणिः क्रोधसमाविष्टः पितुर्वधमनुस्मरन्}
{पाञ्चालानां प्रसुप्तानां वधं प्रति मनो दधे}


\twolineshloka
{गत्वा च शिबिरद्वारि दुर्दर्शं तत्र राक्षसम्}
{घोररूपमपश्यत्स दिवामावृत्य धिष्ठिरम्}


\twolineshloka
{तेन व्याघातमस्त्राणां क्रियमाणमवेक्ष्य च}
{द्रौणिर्यत्र विरूपाक्षं रुद्रमाराध्य सत्वरः}


\twolineshloka
{प्रसुप्तान्निशि विश्वस्तान्धृष्टद्युम्नपुरोगमान्}
{पाञ्चालान्सपरीवारान्द्रौपदेयांश्च सर्वशः}


\twolineshloka
{कृतवर्मणा च सहितः कृपेण च निजघ्निवान्}
{यत्रामुच्यन्त ते पार्थाः पञ्च कृष्णबलाश्रयात्}


\twolineshloka
{सात्यकिश्च महेष्वासः शेषाश्च निधनं गताः}
{पाञ्चालानां प्रसुप्तानां यत्र द्रोणसुताद्वधः}


\twolineshloka
{धृष्टद्युम्नस्य सूतेन पाण्डवेषु निवेदितः}
{द्रौपदी पुत्रशोकार्ता पितृभ्रातृवधार्दिता}


\twolineshloka
{कृतानशनसंकल्पा यत्र भर्तृनुपाविशत्}
{द्रौपदीवचनाद्यत्र भीमो भीमपराक्रमः}


\twolineshloka
{प्रियं तस्याश्चिकीर्षन्वै गदामादाय वीर्यवान्}
{अन्वधावत्सुसंक्रुद्धो भारद्वाजं गुरोः सुतम्}


\twolineshloka
{भीमसेनभयाद्यत्र दैवेनाभिप्रचोदितः}
{अपाण्डवायेति रुषा द्रौणिरस्त्रमवासडदत्}


\twolineshloka
{मैवमित्यब्रवीत्कृष्णः शमयंस्तस्य तद्वचः}
{यत्रास्त्रमस्त्रेण च तच्छमयामास फाल्गुनः}


\twolineshloka
{द्रौणेश्च द्रोहबुद्धित्वं वीक्ष्य पापात्मनस्तदा}
{द्रौणिद्वैपायनादीनां शापाश्चान्योन्यकारिताः}


\twolineshloka
{मणिं तथा समादाय द्रोणपुत्रान्महारथात्}
{पाण्डवाः प्रददुर्हृष्टा द्रौपद्यै जितकाशिनः}


\twolineshloka
{एतद्वै दशमं पर्व सौप्तिकं समुदाहृतम्}
{अष्टादशास्मिन्नद्यायाः पर्वम्युक्ता महात्मना}


\twolineshloka
{श्लोकानां कथितान्यत्र शतान्यष्टौ प्रसंख्यया}
{श्लोकाश्च सप्ततिः प्रोक्ता मुनिना ब्रह्मवादिना}


\twolineshloka
{सौप्तिकैषीकसंबन्धे पर्वण्युत्तमतेजसी}
{अत ऊर्ध्वमिदं प्राहुः स्त्रीपर्व करुणोदयम्}


\twolineshloka
{पुत्रशोकाभिसंतप्तः प्रज्ञाचक्षुर्नराधिपः}
{कृष्णोपनीतां यत्रासावायसीं प्रतिमां दृढां}


\twolineshloka
{भीमसेनद्रोहबुद्धिर्धृतराष्ट्रो बभञ्जह}
{तथा शोकाभितप्तस्य धृतराष्ट्रस्य धीमतः}


\twolineshloka
{संसारदहनं बुद्ध्या हेतुभिर्मोक्षदर्शनैः}
{विदुरेण च यत्रास्य राज्ञ आश्वासनं कृतम्}


\twolineshloka
{धृतराष्ट्रस्य चात्रैव कौरवायोधनं तथा}
{सान्तःपुरस्य गमनं शोकार्तस्य प्रकीर्तितम्}


\twolineshloka
{विलापो वीरपत्नीनां यत्रातिकरुणः स्मृतः}
{क्रोधावेशः प्रमोहश्च गान्धारीधृतराष्ट्रयोः}


\twolineshloka
{यत्र तान्क्षत्रियाः शूरान्सङ्ग्रामेष्वनिवर्तिनः}
{पुत्रान्भ्रातृन्पितॄंश्चैव ददृशुर्निहतान्रणे}


\twolineshloka
{पुत्रपौत्रवधार्तायास्तथात्रैव प्रकीर्तिता}
{गान्धार्याश्चापि कृष्णेन क्रोधोपशमनक्रिया}


\twolineshloka
{यत्र राजा महाप्राज्ञः सर्वधर्मभृतां वरः}
{राज्ञांतानि शरीराणि दाहयामास शास्त्रतः}


\twolineshloka
{तोयकर्मणि चारब्धे राज्ञामुदकदानिके}
{गूढोत्पन्नस्य चाख्यानं कर्णस्य पृथयात्मनः}


\twolineshloka
{सुतस्यैतदिह प्रोक्तं व्यासेन परमर्षिणा}
{एतदेकादशं पर्व शोकवैक्लव्यकारणम्}


\twolineshloka
{प्रणीतं सज्जनमनोवैक्लव्याश्रुप्रवर्तकम्}
{सप्तविंशतिरध्यायाः पर्वण्यस्मिन्प्रकीर्तिताः}


\twolineshloka
{श्लोकसप्तशती चापि पञ्चसप्ततिसंयुता}
{संख्यया भारताख्यानमुक्तं व्यासेन धीमता}


\twolineshloka
{अतः परं शान्तिपर्व द्वादशं बुद्धिवर्धनम्}
{यत्र निर्वेदमापन्नो धर्मराजो युधिष्ठिरः}


\twolineshloka
{घातयित्वा पितॄन्भ्रातॄन्पुत्रान्संबन्धिमातुलान्}
{शान्तिपर्वणि धर्माश्च व्याख्याताःशारतल्पिकाः}


\twolineshloka
{राजभिर्वेदितव्यास्ते सम्यग्ज्ञानबुभुत्सुभिः}
{आपद्धर्माश्च तत्रैव कालहेतुप्रदर्शिनः}


\twolineshloka
{यान्बुद्ध्वा पुरुषः सम्यक्सर्वज्ञत्वमवाप्नुयात्}
{मोक्षधर्माश्च कथिता विचित्रा बहुविस्तराः}


\twolineshloka
{द्वादशं पर्व निर्दिष्टमेतत्प्राज्ञजनप्रियम्}
{अत्र पर्वणि विज्ञेयमध्यायानां शतत्रयम्}


\twolineshloka
{विंशच्चैव तथाध्याया नव चैव तपोधाः}
{चतुर्दशसहस्राणि तथा सप्तशतानि च}


\twolineshloka
{सप्तश्लोकास्तथैवात्र पञ्चविंशतिसंख्यया}
{अत ऊर्ध्वं च विज्ञेयमनुशासनमुत्तमम्}


\twolineshloka
{यत्र प्रकृतिमापन्नः श्रुत्वा धर्मविनिश्चयम्}
{भीष्माद्भागीरथीपुत्रात्कुरुराजो युधिष्ठिरः}


\twolineshloka
{व्यवहारोऽत्र कार्त्स्न्येन धर्मार्थीयः प्रकीर्तितः}
{विविधानां च दानानां फलयोगाः प्रकीर्तिताः}


\twolineshloka
{तथा पात्रविशेषाश्च दानानां च परो विधिः}
{आचारविधियोगश्च सत्यस्य च परा गतिः}


\twolineshloka
{महाभाग्यं गवां चैव ब्राह्मणानां तथैव च}
{रहस्यं चैव धर्माणां देशकालोपसंहितम्}


\twolineshloka
{एतत्सुबहुवृत्तान्तमुत्तमं चानुशासनम्}
{भीष्मस्यात्रैव संप्राप्तिः स्वर्गस्य परिकीर्तिता}


\twolineshloka
{एतत्त्रयोदशं पर्व धर्मनिश्चयकारकम्}
{अध्यायानां शतं त्वत्र षट्चत्वारिंशदेव तु}


\twolineshloka
{श्लोकानां तु सहस्राणि प्रोक्तान्यष्टौ प्रसंख्यया}
{ततोऽश्वमेधिकं नाम पर्व प्रोक्तं चतुर्दशम्}


\twolineshloka
{तत्संवर्तमरुत्तीयं यत्राख्यानमनुत्तमम्}
{सुवर्णकोशसंप्राप्तिर्जन्म चोक्तं परीक्षितः}


\twolineshloka
{दग्धस्यास्त्राग्निना पूर्वं कृष्णात्संजीवनं पुनः}
{चर्यायां हयमुत्सृष्टं पाण्डवस्यानुगच्छतः}


\twolineshloka
{तत्र तत्र च युद्धानि राजपुत्रैरमर्षणैः}
{चित्राङ्गदायाः पुत्रेण स्वपुत्रेण धनंजयः}


\threelineshloka
{सङ्ग्रामे बभ्रुवाहेन संशयं चात्र जग्मिवान्}
{सुदर्शनं तथाऽऽख्यानं वैष्णवं धर्ममेव च}
{अश्वमेधे महायज्ञे नकुलाख्यानमेव च}


\twolineshloka
{इत्याश्वमेधिकं पर्व प्रोक्तमेतन्महाद्भुतम्}
{अध्यायानां शतं चैव त्रयोऽध्यायाश्च कीर्तिताः}


\twolineshloka
{त्रीणि श्लोकसहस्राणि तावन्त्येव शतानि च}
{विंशतिश्च तथा श्लोकाः संख्यातास्तत्त्वदर्शिना}


\twolineshloka
{ततस्त्वाश्रमवासाख्यं पर्व पञ्चदशं स्मृतम्}
{यत्र राज्यं समुत्सृज्य गान्धार्या सहितो नृपः}


\twolineshloka
{धृतराष्ट्रोश्रमपदं विदुरश्च जगाम ह}
{यं दृष्ट्वा प्रस्थितं साध्वी पृथाप्यनुययौ तदा}


\twolineshloka
{पुत्रराज्यं परित्यज्य गुरुशुश्रूषणे रता}
{यत्र राजा हतान्पुत्रान्पौत्रानन्यांश्च पार्थिवान्}


\twolineshloka
{लाकान्तरगतान्वीरानपश्यत्पुनरागतान्}
{ऋषेः प्रसादात्कृष्णस्य दृष्ट्वाश्चर्यमनुत्तमम्}


\twolineshloka
{त्यक्त्वा शोकं सदारश्च सिद्धिं परमिकां गतः}
{यत्र धर्मं समाश्रित्य विदुरः सुगतिं गतः}


\twolineshloka
{संजयश्च सहामात्यो विद्वान्गावल्गणिर्वशी}
{ददर्श नारदं यत्र धर्मराजो युधिष्ठिरः}


\twolineshloka
{नारदाच्चैव शुश्राव वृष्णीनां कदनं महत्}
{एतदाश्रमवासाख्यं पर्वोक्तं महदद्भुतम्}


\twolineshloka
{द्विचत्वारिंशदध्यायाः पर्वैतदभिसङ्ख्यया}
{सहस्रमेकं श्लोकानां पञ्चश्लोकशतानि च}


\twolineshloka
{षडेव च तथा श्लोकाः संख्यातास्तत्त्वदर्शिना}
{अतः परं निबोधेदं मौसलं पर्व दारुणम्}


\twolineshloka
{यत्र ते पुरुषव्याघ्राः शस्त्रस्पर्शहता युधि}
{ब्रह्मदण्डविनिष्पिष्टाः समीपे लवणाम्भसः}


\twolineshloka
{आपाने पानकलिता दैवेनाभिप्रचोदिताः}
{एरकारूपिभिर्वज्रैर्निजघ्नुरितरेतरम्}


\twolineshloka
{यत्र सर्वक्षयं कृत्वा तावुभौ रामकेशवौ}
{नातिचक्रामतुः कालं प्राप्तं सर्वहरं महत्}


\twolineshloka
{यत्रार्जुनो द्वारवतीमेत्य वृष्णिविनाकृताम्}
{दृष्ट्वा विपादमगमत्परां चार्तिं नरर्षभः}


\twolineshloka
{स संस्कृत्य नरश्रेष्ठं मातुलं शौरिमात्मनः}
{ददर्श यदुवीराणामापाने वैशसं महत्}


\twolineshloka
{शरीरं वासुदेवस्य रामस्य च महात्मनः}
{संस्कारं लम्भयामास वृष्णीनां च प्रधानतः}


\twolineshloka
{सवृद्धबालमादाय द्वारवत्यास्ततो जनम्}
{ददर्शापदि कष्टायां गाण्डीवस्य पराभवम्}


\twolineshloka
{सर्वेषां चैव दिव्यानामस्त्राणामप्रसन्नताम्}
{नाशं वृष्णिकलत्राणां प्रभावानामनित्यताम्}


\twolineshloka
{दृष्ट्वा निर्वेदमापन्नो व्यासवाक्यप्रचोदितः}
{धर्मराजं समासाद्य संन्यासं समरोचयत्}


\twolineshloka
{इत्येतन्मौसलं पर्व षोडशं परिकीर्तितम्}
{अध्यायाष्टौ समाख्याताः श्लोकानां च शतत्रयम्}


\twolineshloka
{श्लोकानां विंशतिश्चव संख्याता तत्त्वदर्शिना}
{महाप्रस्थानिकं तस्मादूर्ध्वं सप्तदशं स्मृतम्}


\twolineshloka
{यत्र राज्यं परित्यज्य पाण्डवाः पुरुषर्षभाः}
{द्रौपद्या सहिता देव्या महाप्रस्थानमास्थिताः}


\twolineshloka
{यत्र तेऽग्निं ददृशिरे लौहित्यं प्राप्य सागरम्}
{यत्राग्निना चोदितश्च पार्थस्तस्मै महात्मने}


\twolineshloka
{ददौ संपूज्य तद्दिव्यं गाण्डीवं धनुरुत्तमम्}
{यत्र भ्रातृन्निपतितान्द्रौपदीं च युधिष्ठिरः}


\twolineshloka
{दृष्ट्वा हित्वा जगामैव सर्वाननवलोकयन्}
{एतत्सप्तदशं पर्व महाप्रस्थानिकं स्मृतम्}


\twolineshloka
{यत्राध्यायास्त्रयः प्रोक्ताः श्लोकानां च शतत्रयम्}
{विंशतिश्च तथा श्लोकाः संख्यातास्तत्त्वदर्शिना}


\twolineshloka
{स्वर्गपर्व ततो ज्ञेयं दिव्यं यत्तदमानुषम्}
{प्राप्तं दैवरथं स्वर्गान्नेष्टवान्यत्र धर्मराट्}


\twolineshloka
{आरोदुं सुमहाप्राज्ञ आनृशंस्याच्छुना विना}
{तामस्याविचलां ज्ञात्वा स्थितिं धर्मे महात्मनः}


\twolineshloka
{श्वरूपं यत्र तत्त्यक्त्वा धर्मेणासौ समन्वितः}
{स्वर्गं प्राप्तःसच तथा यातनाविपुला भृशम्}


\twolineshloka
{देवदूतेन नरकं यत्र व्याजेन दर्शितम्}
{शुश्राव यत्र धर्मात्मा भ्रातॄणां करुणागिरः}


\twolineshloka
{निदेशे वर्तमानानां देशे तत्रैव वर्तताम्}
{अनुदर्शितश्च धर्मेण देवराज्ञा च पाण्डवः}


\twolineshloka
{आप्लुत्याकाशगङ्गायां देहं त्यक्त्वा स मानुषम्}
{स्वधर्मनिर्जितं स्थानं स्वर्गे प्राप्य स धर्मराट्}


\twolineshloka
{मुमुदे पूजितः सर्वैः सेन्द्रैः सुरगणैः सह}
{एतदष्टादशं पर्व प्रोक्तं व्यासेन धीमता}


\twolineshloka
{अध्यायाः पञ्च संख्याताः पर्वम्यस्मिन्महात्मना}
{श्लोकानां द्वे शते चैव प्रसंख्याते तपोधाः}


\twolineshloka
{नव श्लोकास्तथैवान्ये संख्याताः परमर्षिणा}
{अष्टादशैवमेतानि पर्वाण्येतान्यशेषतः}


\twolineshloka
{खिलेषु हरिवंशश्च भविष्यं च प्रकीर्तितम्}
{दश श्लोकसहस्राणि विंशच्छ्लोकशतानि च}


\twolineshloka
{खिलेषु हरिवंशे च संख्यातानि महर्षिणा}
{एतत्सर्वं समाख्यातं भारते पर्वसंग्रहः}


\twolineshloka
{अष्टादश समाजग्मुरक्षौहिण्यो ययुत्सया}
{तन्महादारुणं युद्धमहान्यष्टादशाभवत्}


\twolineshloka
{यो विद्याच्चतुरो वेदान्साङ्गोपनिषदो द्विजः}
{न चाख्यानमिदं विद्यान्नैव स स्याद्विचक्षणः}


\twolineshloka
{अर्थशास्त्रमिदं प्रोक्तं धर्मशास्त्रमिदं महत्}
{कामशास्त्रमिदं प्रोक्तं व्यासेनामितबुद्धिना}


\twolineshloka
{श्रुत्वा त्विदमुपाख्यानं श्राव्यमन्यन्न रोचते}
{पुंस्कोकिलगिरं श्रुत्वा रूक्षा ध्वाङ्क्षस्य वागिव}


\twolineshloka
{इतिहासोत्तमादस्माञ्जायन्ते कविबुद्धयः}
{पञ्चभ्य इव् भूतेभ्यो लोकसंविधयस्त्रयः}


\twolineshloka
{अस्याख्यानस्य विषये पुराणं वर्तते द्विजाः}
{अन्तरिक्षस्य विषये प्रजा इव चतुर्विधाः}


\twolineshloka
{क्रियागुणानां सर्वेषामिदमाख्यानमाश्रयः}
{इन्द्रियाणां समस्तानां चित्रा इव मनः क्रियाः}


\twolineshloka
{अनाश्रित्यैतदाख्यानं कथा भुवि न विद्यते}
{आहारमनपाश्रित्य शरीरस्येव धारणम्}


\twolineshloka
{इदं कविवरैः सर्वैराख्यानमुपजीव्यते}
{उदयप्रेप्सुभिर्भृत्यैरभिजात इवेश्वरः}


\twolineshloka
{अस्य काव्यस्य कवयो न समर्था विशेषणे}
{साधोरिव गृहस्थस्य शेषास्त्रय इवाश्रमाः}


\twolineshloka
{धर्मे मतिर्भवतु वः सततोत्थितानांस ह्येक एव परलोकगतस्य बन्धुः}
{अर्थाः स्त्रियश्च निपुणैरपि सेव्यमानानैवाप्तभावमुपयान्ति न च स्थिरत्वम्}


\twolineshloka
{द्वैपायनौष्ठपुटनिःसृतमप्रमेयंपुण्यं पवित्रमथ पापहरं शिवं च}
{यो भारतं समधिगच्छति वाच्यमानंकिं तस्य पुष्करजलैरभिषेचनेन}


\twolineshloka
{यदह्ना कुरुते पाप ब्राह्मणस्त्विन्द्रियैश्चरन्}
{महाभारतमाख्याय सन्ध्यां मुच्यति पश्चिमाम्}


\twolineshloka
{यद्रात्रौ कुरुते पापं कर्मणा मनसा गिरा}
{महाभारतमाख्याय पूर्वां सन्ध्यां प्रमुच्यते}


\twolineshloka
{यो गोशतं कनकशृङ्गमयं ददातिविप्राय वेदविदुषे च बहुश्रुताय}
{पुण्यां च भारतकथां शृणुयाच्च नित्यंतुल्यं फलं भवति तस्य च तस्य चैव}


\twolineshloka
{आख्यानं तदिदमनुत्तमं महार्थंविज्ञेयं महदिह पर्वसंग्रहेण}
{श्रुत्वादौ भवति नृणां सुखावगाहंविस्तीर्णं लवणजलं यथा प्लवेन}


\chapter{अध्यायः ३}
\twolineshloka
{सौतिरुवाच}
{}


\threelineshloka
{जनमेजयः पारिक्षितः सह भ्रातृभिः कुरुक्षेत्रेदीर्घसत्रमुपास्ते}
{तस्य भ्रातरस्त्रयः श्रुतसेन उग्रसेनो भीमसेन इति}
{तेषु तत्सत्रमुपासीनेष्वभ्यागच्छत्सारमेयः}


% Check verse!
जनमेजयस्य भ्रातृभिरभिहतो रोरूयमाणो मातुःसमीपमुपागच्छत्
\twolineshloka
{तं माता रोरूयमाणमुवाच}
{किं रोदिषि केनास्यभिहत इति}


% Check verse!
स एवमुक्तो मातरं प्रत्युवाच जनमेजयस्यभ्रातृभिरभिहतोऽस्मीति
% Check verse!
तं माता प्रत्युवाच व्यक्तं त्वया तत्रापराद्धं येनास्यभिहतइति
% Check verse!
स तां पुनरुवाच नापराध्यामि किंचिन्नावेक्षे हवींषि नावलिहइति
% Check verse!
तच्छ्रुत्वा तस्य माता सरमा पुत्रदुःखार्तातत्सत्रमुपागच्छद्यत्र स जनमेजयः सहभ्रातृभिर्दीर्घसत्रमुपास्ते
% Check verse!
स तया क्रुद्धया तत्रोक्तोऽयं मे पुत्रो न किंचिदपराध्यतिनावेक्षते हवींषि नावलेढि किमर्थमभिहत इति
% Check verse!
न किंचिदुक्तवन्तस्ते सा तानुवाच यस्मादयमभिहतोऽनपकारीतस्माददृष्टं त्वां भयमागमिष्यतीति
% Check verse!
जनमेजय एवमुक्तो देवशुन्या सरमया भृशं संभ्रान्तोविषण्णश्चासीत्
% Check verse!
स तस्मिन्सत्रे समाप्ते हास्तिनपुरं प्रत्येत्यपुरोहितमनुरूपमन्विच्छमानः परं यत्नमकरोद्यो मे पापकृत्यांशमयेदिति
% Check verse!
स कदाचिन्मृगयां गतः पारिक्षितो जनमेजयः कस्मिंश्चित्स्वविषयआश्रममपश्यत्
\twolineshloka
{तत्र कश्चिदृपिरासाञ्चक्रे श्रुतश्रवा नाम}
{तस्य तपस्यभिरतः पुत्र आस्ते सोमश्रवा नाम}


% Check verse!
तस्य तं पुत्रमभिगम्य जनमेजयः पारिक्षितः पौरोहित्यायवव्रे
% Check verse!
स नमस्कृत्य तमृषिमुवाच भगवन्नयं तव पुत्रो ममपुरोहितोऽस्त्विति
% Check verse!
स एवमुक्तः प्रत्युवाच जनमेजयं भो जनमेजय पुत्रोऽयं मम सर्प्यांजातो महातपस्वी स्वाध्यायसंपन्नो मत्तपोवीर्यसंभृतो मच्छुक्रंपीतवत्यास्तस्याः कुक्षौ जातः
% Check verse!
समर्थोऽयं भवतः सर्वाः पापकृत्याः शमयितुमन्तरेणमहादेवकृत्याम्
% Check verse!
अस्य त्वेकमुपांशुव्रतं यदेनं कश्चिद्ब्राह्मणःकंचिदर्थमभियाचेत्तं तस्मै दद्यादयं यद्येतदुत्सहसे ततोनयस्वैनमिति
% Check verse!
तेनैवमुक्तो जनमेजयस्तं प्रत्युवाच भगवंस्तत्तथाभविष्यतीति
\threelineshloka
{स तं पुरोहितमुपादायोपावृत्तो भ्रातृनुवाच मयाऽयं वृत उपाध्यायोयदयं ब्रूयात्तत्कार्यमविचारयद्भिर्भवद्भिरिति}
{तेनैवमुक्ता भ्रातरस्तस्य तथा चक्रुः}
{स तथा भ्रातॄन्संदिश्य तक्षशिलां प्रत्यभिप्रतस्थे तं च देशं वशेस्थापयामास}


% Check verse!
एतस्मिन्नन्तरे कश्चिदृषिर्धौम्यो नामापोदस्तस्य शिष्यास्त्रयोबभूवुः
% Check verse!
उपमन्युरारुणिर्बैदश्चेति स एकं शिष्यंमारुणिं पाञ्चाल्यंप्रेषयामास गच्छ केदारखण्डं बधानेति
\twolineshloka
{स उपाध्यायेन संदिष्ट आरुणिः पाञ्चाल्यस्तत्र गत्वा तत्केदारखण्डं बद्धुंनाशकत्}
{स क्लिश्यमानोऽपश्यदुपायं भवत्वेवं करिष्यामीति}


% Check verse!
स तत्र संविवेश केदारखण्डे शयाने व तथा तस्मिंस्तदुदकंतस्थौ
% Check verse!
ततः कदाचिदुपाध्याय आपोदो धौम्यः शिष्यावपृच्छत् क्व आरुणिःपाञ्चाल्यो गत इति
\twolineshloka
{तौ तं प्रत्यूचतुर्भगवंस्त्वयैव प्रेषितो गच्छ केदारखण्डं बधानेति}
{स एवमुक्तस्तौ शिष्यौ प्रत्युवाच तस्मात्तत्र सर्वे गच्छामो यत्र स गतइति}


\twolineshloka
{स तत्र गत्वा तस्याह्वानाय शब्दं चकारः}
{भो आरुणे पाञ्चाल्य क्वासि वत्सैहीति}


% Check verse!
स तच्छ्रुत्वा आरुणिरुपाध्यायवाक्यंतस्मात्केदारखण्डात्सहसोत्थायतमुपाध्यायमुपतस्थे
% Check verse!
प्रोवाच चैनमयमस्म्यत्र केदारखण्डे निःसरमाणमुदकमवारणीयंसंरोद्धुं संविष्टो भगवच्छब्दं श्रुत्वैव सहसा विदार्य केदारखण्डंभवन्तमुपस्थितः
% Check verse!
तदभिवादये भगवन्तमाज्ञापयतु भवान्कमर्थं करवाणीति
% Check verse!
स एवमुक्त उपाध्यायः प्रत्युवाच यस्माद्भवान्केदारखण्डंविदार्योत्थितस्तस्मादुद्दालक एवनाम्नाभवान्भविष्यतीत्युपाध्यायेनानुगृहीतः
\twolineshloka
{यस्माच्च त्वया मद्वचनमनुष्ठितं तस्माच्छ्रेयोऽवाप्स्यसि}
{सर्वे च ते वेदाः प्रतिभास्यन्ति सर्वाणि च धर्मशास्त्राणीति}


% Check verse!
स एवमुक्त उपाध्यायेनेष्टं देशं जगाम
\twolineshloka
{अथापरः शिष्यस्तस्यैवापोदस्य धौम्यस्योपमन्युर्नाम}
{तं चोपाध्यायः प्रेषयामास वत्सोपमन्यो गा रक्षस्वेति}


% Check verse!
स उपाध्यायवचनादरक्षद्गाः स चाहनि गा रक्षित्वा दिवसक्षयेगुरुगृहमागम्योपाध्यायस्याग्रतः स्थित्वा नमश्चक्रे
% Check verse!
तमुपाध्यायः पीवानमपश्यदुवाच चैनं वत्सोपमन्यो केन वृत्तिंकल्पयसि पीवानसि दृढमिति
% Check verse!
स उपाध्यायं प्रत्युवाच भो भैक्ष्येण वृत्तिं कल्पयामीतितमुपाध्यायः प्रत्युवाच
\twolineshloka
{मय्यनिवेद्य बैक्ष्यं नीपयोक्तव्यमिति}
{स तथेत्युक्तो भैक्ष्यं चरित्वोणध्यायन्यवेदयत्}


\twolineshloka
{स तस्मादुपाध्यायः सर्वमेव भैक्ष्यमगृह्णात्}
{स तथेत्युक्तः पुनररक्षद्गा अहनि रक्षित्वा निशामुखे गुरुकुलमागम्यगुरोरग्रतःस्थित्वा नमश्चक्रे}


\twolineshloka
{तमुपाध्यायस्तथापि पीवानमेव दृष्ट्वोवाच}
{वत्सोपमन्यो सर्वमशेषतस्ते भैक्ष्यं गृह्णामि केनेदानीं वृत्तिंकल्पयसीति}


\twolineshloka
{स एवमुक्त उपाध्यायं प्रत्युवाच}
{भगवते निवेद्य पूर्वमपरं चरामि तेन वृत्तिं कल्पयामीति तमुपाध्यायःप्रत्युवाच}


% Check verse!
नैषा न्याय्या गुरुवृत्तिरन्येषामपि भैक्ष्योपजीविनांवृत्त्युपरोधं करोषि इत्येवं वर्तमानो लुब्धोऽसीति
% Check verse!
स तथेत्युक्त्वा गा अरक्षद्रक्षित्वाचपुनरुपाध्यायगृहमागम्योपाध्यायस्याग्रतः स्थित्वा नमश्चक्रे
\twolineshloka
{तमुपाध्यायस्तथापि पीवानमेव दृष्ट्वा पुनरुवाच}
{वत्सोपमन्यो अहं ते सर्वं भैक्ष्यं गृह्णामि न चान्यच्चरसि पीवानसि भृशं केनवृत्तिं कल्पयसीति}


\threelineshloka
{स एवमुक्तस्तमुपाध्यायं प्रत्युवाच}
{भो एतासां गवां पयसा वृत्तिं कल्पयामीति}
{तमुवाचोपाध्यायो नैतन्न्याय्यं पय उपयोक्तुं भवतो मयानाभ्यनुज्ञातमिति}


% Check verse!
स तथेति प्रतिज्ञाय गा रक्षित्वा पुनरुपाध्यायगृहमेत्यगुरोरग्रतः स्थित्वा नमश्चक्रे
\twolineshloka
{तमुपाध्यायः पीवानमेव दृष्ट्वोवाच}
{वत्सोपमन्यो भैक्ष्यं नाश्नासि न चान्यच्चरसि पयो न पिबसि पीवानसि भृशंकेनेदानीं वृत्तिं कल्पयसीति}


\twolineshloka
{स एवमुक्त उपाध्यायं प्रत्युवाच}
{भोः फेनं पिबापि यमिमे वत्सा मातॄणां स्तनात्पिबन्त उद्गिरन्ति}


\fourlineindentedshloka
{तमुपाध्यायः प्रत्युवाच}
{एते त्वदनुकम्पया गुणवन्तो वत्साः प्रभूततरंफेनमुद्गिरन्ति}
{तदेषामपि वत्सानां वृत्त्युपरोधं करोष्येवं वर्तमानः}
{फेनमपि भवान्न पातुमर्हतीति स तथेति प्रतिश्रुत्य निराहारःपुनररक्षद्गाः}


% Check verse!
तथा प्रतिषिद्धो भैक्ष्यं नाश्नाति नचान्यच्चरति पयो न पिबतिफेनं नोपयुह्क्ते स कदाचिदरण्ये क्षुधार्तोऽर्कपत्राण्यभक्षयत्
\twolineshloka
{स तैरर्कपत्रैर्भक्षितैःक्षारतिक्तकटुरूक्षैस्तीक्ष्णविपाकैश्चक्षुष्युपहतोऽन्धो बभूव}
{ततः सोऽन्धोऽपि चङ्क्रम्यमाणः कूपेऽपतत्}


% Check verse!
अथ तस्मिन्ननागच्छति सूर्ये चास्ताचलावलम्बिनि उपाध्यायःशिष्यानवोचत्
\twolineshloka
{मयोपमन्युः सर्वतः प्रतिषिद्धः स नियतं कुपितस्ततो नागच्छतिचिरगतस्त्विति}
{ततोऽन्वेष्य इत्येवमुक्त्वा शिष्यैः सार्धमरण्यं गत्वा तस्याह्वानाय शब्दंचकार भो उपमन्यो क्वासि वत्सैहीति}


\twolineshloka
{स उपाध्यायस्य आह्वानवचनं श्रुत्वा प्रत्युवाचोच्चैरयमस्मिन्कूपेपतितोऽहमिति}
{तमुपाध्यायः प्रत्युवाच कथं त्वमस्मिन्कूपे पतित इति}


\twolineshloka
{स उपाध्यायं प्रत्युवाच अर्कपत्राणिभक्षयित्वान्धीभूतोस्म्यतश्चङ्क्रम्यमाणः कूपे पतित इति}
{तमुपाध्यायः प्रत्युवाच}


\twolineshloka
{अश्विनौ स्तुहि तौ देवभिषजौ त्वां चक्षुष्मन्तं कर्ताराविति}
{स एवमुक्त उपाध्यायेनोपमन्युः स्तोतुमुपचक्रमे देवावश्विनौवाग्भिर्ऋग्भिः}


\twolineshloka
{प्रपूर्वगौ पूर्वजौ चित्रभानूगिरा वां शंसामि तपसा ह्यनन्तौ}
{दिव्यौ सुपर्णौ विरजौ विमाना-वधिक्षिपन्तौ भुवनानि विश्वा}


\twolineshloka
{हिरण्मयौ शकुनी सांपरायौनासत्यदस्रौ सुनसौ वैजयन्तौ}
{शुक्लं वयन्तौ तरसा सुवेमा-वधिव्ययन्तावसितं विवस्वतः}


\twolineshloka
{ग्रस्तां सुपर्णस्य बलेन वर्तिका-ममुञ्चतामश्विनौ सौभगाय}
{तावत्सुवृत्तावनमं तमाय या-वसत्तमा गा अरुणा उदावहत्}


\twolineshloka
{षष्टिश्च गावस्त्रिशताश्च धेनवएकं वत्सं सुवते तं दुहन्ति}
{नानागोष्ठा विहिता एकदोहना-स्तावश्विनौ दुहतो घर्ममुक्थ्यम्}


\twolineshloka
{एकां नाभिं सप्त सथा अराः श्रिताःप्रधिष्वन्या विंशतिरर्पिता अराः}
{अनेमि चक्रं परिवर्ततेऽजरंमायाऽश्विनौ समनक्ति चर्षणी}


\twolineshloka
{एकं चक्रं वर्तते द्वादशारंषण्णाभिमेकाक्षममृतस्य धारणम्}
{यस्मिन्देवा अधि विश्वे विषक्ता-स्तावश्विनौ मुञ्चतो मा विषीदतम्}


\twolineshloka
{अश्विनाविन्दुममृतं वृत्तभूयौतिरोधत्तामश्विनौ दासपत्नी}
{हित्वा गिरिमश्विनौ गामुदाचरन्तौतद्वृष्टिमह्नात्प्रस्थितौ बलस्य}


\twolineshloka
{युवां दिशो जनयथो दशाग्रेसमानं मूर्ध्नि रथयानं वियन्ति}
{तासां यातमृषयोऽनुप्रयान्तिदेवा मनुष्याः क्षितिमाचरन्ति}


\twolineshloka
{युवां वर्णान्विकुरुथो विश्वरूपां-स्तेऽधिक्षियन्ते भुवनानि विश्वा}
{ते भानवोऽप्यनुसृताश्चरन्तिदेवा मनुष्याः क्षितिमाचरन्ति}


\twolineshloka
{तौ नासत्यावश्विनौ वां महेऽहंस्रजं च यां बिभृथः पुष्करस्य}
{तौ नासत्वावमृतावृतावृधा-वृते देवास्तत्प्रपदे न सूते}


\twolineshloka
{मुखेन गर्भं लभतां युवानौगतासुरेतत्प्रपदेन सूते}
{सद्यो जातो मातरमत्ति गर्भ-स्तावश्विनौ मुञ्चथौ जीवसे गाम्}


\threelineshloka
{स्तोतुं न शक्नोमि गुणैर्भवन्तौचक्षुर्विहीनः पथि संप्रमोहः}
{दुर्गेऽहमस्मिन्पतितोऽस्मि कूपेयुवां शरण्यौ शरणं प्रपद्ये ॥सौतिरुवाच}
{}


\twolineshloka
{एवमृग्भिश्चान्यैरस्तुवत्}
{इत्येवं तेनाभिष्टुतावश्विनावाजग्मतुराहतुश्चैं प्रीतौ स्व एषतेऽपूपोशानैनमिति}


% Check verse!
स एवमुक्तः प्रत्युवाच नानृतमूचतुर्भगवन्तौनत्वहमेतमपूपमुपयोक्तुमुत्सहे गुरवेऽनिवेद्येति
\twolineshloka
{ततस्तमश्विनावूचतुः}
{आवाभ्यां पुरस्ताद्भवत उपाध्यायेनैवमेवाभिष्टुताभ्यामपूपोदत्त उपयुक्तः सतेनानिवेद्य गुरवे त्वमपि तथैव कुरुष्व यथा कृतमुपाध्यायेनेति}


% Check verse!
स एवमुक्तः प्रत्युवाच एतत्प्रत्यनुनये भवन्तावश्विनौनोत्सहेऽहमनिवेद्य गुरवेऽपूपमुपयोक्तुमिति
\twolineshloka
{तमश्विनावाहतुः प्रीतौ स्वस्तवानया गुरुभक्त्या}
{उपाध्यायस्य ते कार्ष्णायसा दन्ता भवतोऽपि हिरण्मया भविष्यन्तिचक्षुष्मांश्च भविष्यसि श्रेयश्चावाप्स्यसीति}


% Check verse!
स एवमुक्तोऽश्विभ्यांलब्धचक्षुरुपाध्यायसकाशमागम्याभ्यवादयत्
% Check verse!
आचचक्षे च स चास्य प्रीतिमान्बभूव
% Check verse!
आह चैनं यथाऽश्विनावाहतुस्तथा त्वं श्रेयोऽवाप्स्यसीति
\twolineshloka
{सर्वे च ते वेदाःप्रतिभास्यन्ति सर्वाणि च धर्मशास्त्राणीति}
{एषा तस्यापि परीक्षोपमन्न्योः}


% Check verse!
अथापरः शिष्यस्तस्यैवापोदस्य धौम्यस्य बैदो नाम तमुपाध्यायःसमादिदेश वत्स बैद इहास्यतां तावन्मम गृहे कंचित्कालं शुश्रूषुणा चभवितव्यं श्रेयस्ते भविष्यतीति
\twolineshloka
{स तथेत्युक्त्वा गुरुकुले दीर्घकालं गुरुशुश्रूषणपरोऽवसत्}
{गौरिव नित्यं गुरुणा धूर्षु नियोज्यमानः शीतोष्णक्षुत्तृष्णादुःखसहःसर्वत्राप्रतिकूलस्तस्य महतात्कालेन गुरुः परितोषं जगाम}


\twolineshloka
{तत्परितोषाच्च श्रेयः सर्वज्ञतां चावाप}
{एषा तस्यापि परीक्षा बैदस्य}


\threelineshloka
{स उपाध्यायेनानुज्ञातः समावृत्तस्तस्माद्गुरुकुलवासाद्गृहाश्रमंप्रत्यपद्यत}
{तस्यापि स्वगृहे वसतस्त्रयः शिष्या बभूवुः स शिष्यान्नकिंचिदुवाच कर्म वा क्रियतां गुरुशुश्रूषा वेति}
{दुःखाभिज्ञो हि गुरुकुलवासस्य शिष्यान्परिक्लेशेन योजयितुंनेयेष}


% Check verse!
अथ कस्मिंश्चित्काले बैदं ब्राह्मणं जनमेजयः पौष्यश्चक्षत्रियावुपेत्योपाध्यायं वरयाञ्चक्रतुः
% Check verse!
स कदाचिद्याज्यकार्येणाभिप्रस्थित उत्तङ्कनामानं शिष्यंनियोजयामास
% Check verse!
भोयत्किंचिदस्मद्गृहे परिहीयते तदिच्छाम्यहमपरिहीयमानं भवताक्रियमाणमिति स एवं प्रतिसंदिश्योत्तङ्कं बैदः प्रवासं जगाम
\twolineshloka
{अथोत्तङ्कः शुश्रूषुर्गुरुनियोगमनुतिष्ठमानो गुरुकुले वसति स्म}
{स तत्र वसमान उपाध्यायस्त्रीभिः सहिताभिराहूयोक्तः}


% Check verse!
उपाध्यायानी ते ऋतुमती उपाध्यायश्च प्रोषितोऽस्यायथाऽयमृतुर्वन्ध्यो न भवति तथा क्रियतामेषा विषीदतीति
\threelineshloka
{एवमुक्तस्ताः स्त्रियः प्रत्युवाच}
{न मया स्त्रीणां वचनादिदमकार्यं करणीयम्}
{न ह्यहमुपाध्यायेन संदिष्टोऽकार्यमपि त्वया कार्यमिति}


\twolineshloka
{तस्य पुनरुपाध्यायः कालान्तरेण गृहमाजगाम तस्मात्प्रवासात्}
{स तु तद्वृत्तं तस्याशेषमुपलभ्य प्रीतिमानभूत्}


\twolineshloka
{उवाच चैनं वत्सोत्तङ्कं किं ते प्रियं करवाणीति}
{धर्मतो हि शुश्रूषितोऽस्मि भवता तेन प्रीतिः परस्परेण नौ संवृद्धा तदनुजानेभवन्तं सर्वानेव कामानवाप्स्यसि गम्यतामिति}


% Check verse!
स एवमुक्तः प्रत्युवाच किं ते प्रियं करवाणीति एवंह्याहुः
\twolineshloka
{यश्चाधर्मेण वै ब्रूयाद्यश्चाधर्मेण पृच्छति}
{तयोरन्यतरः प्रैति विद्वेषं चाधिगच्छति}


\twolineshloka
{सोहमनुज्ञातो भवता इच्छामीष्टं गुर्वर्थमुपहर्तुमिति}
{तेनैवमुक्त उपाध्यायः प्रत्युवाच वत्सोत्तङ्क उष्यतां तावदिति}


% Check verse!
स कदाचित्तमुपाध्यायमाहोत्तङ्क आज्ञापयतु भवान्किं तेप्रियमुपाहरामि गुर्वर्थमिति
% Check verse!
तमुपाध्यायः प्रत्युवाच वत्सोत्तङ्क बहुशो मां चोदयसिगुर्वर्थमुपाहरामीति तद्गच्छैनां प्रविश्योपाध्यायानीं पृच्छकिमुपाहरामीति एषा यद्ब्रवीति तदुपाहरस्वेति
% Check verse!
स एवमुक्तउपाध्यायेनोपाध्यायानीमपृच्छद्भवत्युपाध्यायेनास्म्यनुज्ञातो गृहंगन्तुमिच्छामीष्टं ते गुर्वर्थमुपहृत्यानृणो गन्तुं तदाज्ञापयतु भवतीकिमुपाहरामि गुर्वर्थमिति
% Check verse!
सैवमुक्तोपाध्यायानी तमुत्तङ्कं प्रत्युवाच गच्छ पौष्यं प्रतिराजानं कुण्डले भिक्षितुं तस्य क्षत्रियया पिनद्धे
\twolineshloka
{आनयस्वेतश्चतुर्थेऽहनि पुण्यकर्म भविता ताभ्यामाबद्धाभ्यां शोभमानाब्राह्मणान्परिवेष्टुमिच्छामि}
{तत्संपादयस्व एवं हि कुर्वतः श्रेयो भविताऽन्यथा कुतः श्रेय इति}


% Check verse!
स एवमुक्तस्तयोपाध्यायान्या प्रातिष्ठतोत्तङ्कः स पथिगच्छन्नपश्यदृषभमतिप्रमाणं तमधिरूढं च पुरुषमतिप्रमाणमेव स पुरुषउत्तङ्कमभ्यभाषत
% Check verse!
भोउत्तङ्कैतत्पुरीषमस्य ऋषभस्य भक्षयस्वेति स एवमुक्तोनैच्छत्
% Check verse!
तमाह पुरुषो भूयो भक्षयस्वोत्तङ्क मा विचारयोपाध्यायेनापि तेभक्षितं पूर्वमिति
% Check verse!
स एवमुक्तो बाढमित्युक्त्वा तदा तद्वृपभस्य मूत्रं पुरीषं चभक्षयित्वोत्तङ्कः संभ्रमाढुत्थित एवापोऽनुस्पृश्य प्रतस्थे
\twolineshloka
{यत्र स क्षत्रियः पौष्यस्तमुपेत्यासीनमपश्यदुत्तङ्कः}
{स उत्तङ्कस्तमुपेत्याशीर्भिरभिनन्द्योवाच}


\twolineshloka
{अर्थी भवन्तमुपागतोऽस्मीति स एनमभिवाद्योवाच}
{भगवन्पौष्यः खल्वहं किं करवाणीति}


\twolineshloka
{स तमुवाच गुर्वर्थं कुण्डलयोरर्थेनाभ्यागतोऽस्मि}
{ये वै ते क्षत्रिया पिनद्धे कुण्डले ते भवान्दातुमर्हतीति}


\twolineshloka
{तं प्रत्युवाच पौष्यः प्रविश्यान्तःपुरं क्षत्रिया याच्यतामिति}
{स तेनैवमुक्तः प्रविश्यान्तःपुरं क्षत्रियां नापश्यत्}


% Check verse!
स पौष्यं पुनरुवाच न युक्तं भवताऽहमनृतेनोपचरितुं न हितेऽन्तःपुरे क्षत्रिया सन्निहिता नैनां पश्यामि
\twolineshloka
{स एवमुक्तः पौष्यः क्षणमात्रं विमृश्योत्तङ्कं प्रत्युवाच}
{नियतं भवानुच्छिष्टः स्मर तावन्न हि सा क्षत्रिया उच्छिष्टेनाशुचिना शक्याद्रष्टुं पतिव्रतात्वात्सैषा नाशुचेर्दर्शनमुपैतीति}


\twolineshloka
{अथैवमुक्त उत्तङ्कः स्मृत्वोवाचास्ति खलु मया तु भक्षितंनोपस्पृष्टमागच्छतेति}
{तं पौष्यः प्रत्युवाच एष ते व्यतिक्रमो नोत्थितेनोपस्पृष्टं भवतिशीघ्रमागच्छतेति}


% Check verse!
अथोत्तङ्कस्तं तथेत्युक्त्वा प्राङ्मुख उपावेश्यसुप्रक्षालितपाणिपादवदनोनिःशब्दाभिरफेनाभिरनुष्णाभिर्हृद्गताभिरद्भिस्त्रिः पीत्वा द्विःपरिमृज्य खान्यद्भिरुपस्पृश्य चान्तःपुरं प्रविवेश
% Check verse!
ततस्तां क्षत्रियामपश्यत्सा च दृष्ट्वैवोत्तङ्कंप्रत्युत्थायाभिवाद्योवाच स्वागतं ते भगवन्नाज्ञापय किंकरवाणीति
\twolineshloka
{स तामुवाचैते कुण्डले गुर्वर्थं मे भिक्षिते दातुमर्हसीति}
{सा प्रीता तेन तस्य सद्भावेन पात्रमयमनतिक्रमणीयश्चेति मत्वा ते कुण्डलेअवमुच्यास्मै प्रायच्छदाह चैनमेते कुण्डले तक्षको नागराजः सुभृशंप्रार्थयत्यप्रमत्तो नेतुमर्हसीति}


\twolineshloka
{स एवमुक्तस्तां क्षत्रियां प्रत्युवाच भवती सुनिर्वृता भवतु}
{न मां शक्तस्तक्षको नागराजो धर्षयितुमिति}


\twolineshloka
{स एवमुक्त्वा तां क्षत्रियामामन्त्र्य पौष्यसकाशमागच्छत्}
{आह चैनं भोः पौष्य प्रीतोऽस्मीति तमुत्तङ्कं पौष्यः प्रत्युवाच}


% Check verse!
भगवंश्चिरेण पात्रमासाद्यते भवाश्च गुणवानतिथिस्तदिच्छे श्राद्धंकर्तुं क्रियतां क्षण इति
% Check verse!
तमुत्तङ्कः प्रत्युवाच कृतक्षण एवास्मि शीघ्रमिच्छामियथोपपन्नमन्नमुपस्कृतं भवतेति स तथेत्युक्त्वा यथोपपन्नेनान्नेनैनंभोजयामास
\twolineshloka
{अथोत्तङ्कः सकेशं शीतमन्नं दृष्ट्वा अशुच्येतदिति मत्वा तं पौष्यमुवाच}
{यस्मान्मे अशुच्यन्नं ददासि तस्मादन्धो भविष्यसीति}


\twolineshloka
{तं पौष्यः प्रत्युवाच}
{यस्मात्त्वमदुष्टमन्नंदूषयसि तस्मादनपत्यो भविष्यसीति तमुत्तङ्कःप्रत्युवाच}


\twolineshloka
{न युक्तं भवताऽन्नमशुचि दत्त्वा प्रतिशापं दातुं तस्मादन्नमेवप्रत्यक्षीकुरु}
{ततः पौष्यस्तदन्नमशुचि दृष्ट्वा तस्याशुचिभावमपरोक्षयामास}


% Check verse!
अथ तदन्नं मुक्तकेश्या स्त्रियोपहृतमनुष्णं सकेशं चाशुच्येतदितिमत्वा तमृषिमुत्तङ्कं प्रसादयामास
\threelineshloka
{भघवन्नेतदज्ञानादन्नं सकेशमुपाहृतं शीतं च}
{तत्क्षामये भवन्तं न भवेयमन्ध इति}
{तमुत्तङ्कः प्रत्युवाच}


\twolineshloka
{न मृषा ब्रवीमि भूत्वा त्वमन्धो नचिरादनन्धो भविष्यसीति}
{ममापि शापो भवता दत्तो न भवेदिति}


\twolineshloka
{तं पौष्यः प्रत्युवाच न चाहं शक्तः शापं प्रत्यादातुं न हि मेमन्युरद्याप्युपशमं गच्छति किं चैतद्भवता न ज्ञायते}
{यथा}


\twolineshloka
{नवनीतं हृदयं ब्राह्मणस्यवाचि क्षुरो निहितस्तीक्ष्णधारः}
{तदुभयमेतद्विपरीतं क्षत्रियस्यवाङ्गवनीतं हृदयं तीक्ष्णधारम् ॥ इति}


\twolineshloka
{तदेवंगते न शक्तोऽहं तीक्ष्णहृदयत्वात्तं शापमन्यथाकर्तुं गम्यतामिति}
{तमुत्तङ्कः प्रत्युवाच}


\threelineshloka
{भवताऽहमन्नस्याशुचिभावमालक्ष्य प्रत्यनुनीतः}
{प्राक् च तेऽभिहितं यस्माददुष्टमन्नं दूषयसि तस्मादनपत्योभविष्यसीति}
{दुष्टे चान्ने नैष मम शापो भविष्यतीति}


\twolineshloka
{साधयामस्तावदित्युक्त्वा प्रातिष्ठतोत्तङ्कस्ते कुण्डले गृहीत्वा}
{सोऽपश्यदथ पथि नग्नं क्षपणकमागच्छन्तं मुहुर्मुहुर्दृश्यमानमदृश्यमानंच}


\twolineshloka
{अथोत्तङ्कस्ते कुण्डले संन्यस्य भूमावुदकार्थं प्रचक्रमे}
{एतस्मिन्नन्तरे स क्षपणकस्त्वरमाण उपसृत्य ते कुण्डले गृहीत्वाप्राद्रवत्}


% Check verse!
तमुत्तङ्कोऽभिसृत्य कृतोदककार्यः शुचिः प्रयतो नमो देवेभ्योगुरुभ्यश्च कृत्वा महता जवेन तमन्वयात्
\twolineshloka
{तस्य तक्षको दृढमासन्नः सतं जग्राह}
{गृहीतमात्रः स तद्रूपं विहाय तक्षकस्वरूपं कृत्वा सहसा धरण्यां विवृतंमहाबिलं प्रविवेश}


\twolineshloka
{प्रविश्य च नागलोकं स्वभवनमगच्छत्}
{अथोत्तङ्कस्तस्याः क्षत्रियाया वचः स्मृत्वा तं तक्षकमन्वगच्छत्}


\threelineshloka
{स तद्बिलं दण्डकाष्ठेन चखान न चाशकत्}
{तं क्लिश्यमानमिन्द्रोऽपश्यत्स वज्रं प्रेषयामास}
{गच्छास्य ब्राह्मणस्य साहाय्यं कुरुष्वेति}


% Check verse!
अथ वज्रं दण्डकाष्ठमनुप्रविश्य तद्बिलमदारयत्
% Check verse!
तमुत्तङ्कोऽनुविवेश तेनैव बिलेन प्रविश्य च तंनागलोकमपर्यन्तमनेकविधप्रासादहर्म्यवलभीनिर्यूहशतसंकुलमुच्चावचक्रीडाश्चर्यस्थानावकीर्णमपश्यत्
\threelineshloka
{स तत्र नागांस्तानस्तुवदेभिः श्लोकैः}
{य ऐरावतराजानः सर्पाः समितिशोभाः}
{क्षरन्त इव जीमूताः सविद्युत्पवनेरिताः}


\twolineshloka
{सुरूपा बहुरूपाश्च तथा कल्माषकुण्डलाः}
{आदित्यवन्नाकपृष्ठे रेजुरैरावतोद्भवाः}


\twolineshloka
{बहूनि नागवेश्मानि गङ्गायास्तीर उत्तरे}
{तत्रस्थानपि संस्तौमि महतः पन्नगानहम्}


\twolineshloka
{इच्छेत्कोऽर्कांशुसेनायां चर्तुमैरावतं विना}
{शतान्यशीतिरष्टौ च सहस्राणि च विंशतिः}


\twolineshloka
{सर्पाणां प्रग्रहा यान्ति धृतराष्ट्रो यदैजति}
{ये चैनमुपसर्पन्ति ये च दूरपथं गताः}


\twolineshloka
{अहमैरावतज्येष्ठभ्रातृभ्योऽकरवं नमः}
{यस्य वासः कुरुक्षेत्रे खाण्डवे चाभवत्पुरा}


\twolineshloka
{तं नागराजमस्तौषं कुण्डलार्थाय तक्षकम्}
{तक्षकश्चाश्वसेनश्च नित्यं सहचरावुभौ}


\twolineshloka
{कुरुक्षेत्रं च वसतां नदीमिक्षुमतीमनु}
{जघन्यजस्तक्षकस्य श्रुतसेनेति यः सुतः}


\threelineshloka
{अवसद्यो महद्द्युम्नि प्रार्थयन्नागमुख्यताम्}
{करवाणि सदा चाहं नमस्तस्मै महात्मने ॥सौतिरुवाच}
{}


\twolineshloka
{एवं स्तुत्वा स विप्रर्षिरुत्तङ्को भुजगोत्तमान्}
{नैव ते कुण्डले लेभे ततश्चिन्तामुपागमत्}


\twolineshloka
{एवं स्तुवन्नपि नागान्यदा ते कुण्डले नालभत्तदाऽपश्यत्स्त्रियौ तन्त्रेअधिरोप्य सुवेमे पटं वयन्त्यौ}
{तस्मिंस्तन्त्रे कृष्णाः सिताश्च तन्तवश्चक्रं चापश्यद्द्वादशारं षड्भिःकुमारैः परिवर्त्यमानं पुरुषं चापश्यदश्वं च दर्शनीयम्}


% Check verse!
स तान्सर्वांस्तुष्टाव एभिर्मन्त्रवादश्लोकैः
\twolineshloka
{त्रीण्यर्पितान्यत्र शतानि मध्येषष्टिश्च नित्यं चरति ध्रुवेऽस्मिन्}
{चक्रे चतुर्विंशतिपर्वयोगेषड्वै कुमाराः परिवर्तयन्ति}


\twolineshloka
{तन्त्रं चेदं विश्वरूपे युवत्यौवयतस्तन्तून्सततं वर्तयन्त्यौ}
{कृष्णान्सितांश्चैव विवर्तयन्त्यौभूतान्यजस्रं भुवनानि चैव}


\twolineshloka
{वज्रस्य भर्ता भुवनस्य गोप्तावृत्रस्य हन्ता नमुचेर्निहन्ता}
{कृष्णे वसानो वसने महात्मासत्यानृते यो विविनक्ति लोके}


\twolineshloka
{यो वाजिनं गर्भमपां पुराणंवैश्वानरं वाहनमभ्युपैति}
{नमोऽस्तु तस्मै जगदीश्वरायलोकत्रयेशाय पुरंदराय}


\twolineshloka
{ततः स एनं पुरुषः प्राह प्रीतोऽस्मि तेऽहसनेन स्तोत्रेण किं ते प्रियंकरवाणीति}
{स तमुवाच नागा मे वशमीयुरिति}


% Check verse!
स चैनं पुरुषः पुनरुवाच एतमश्वमपाने धमस्वेति
% Check verse!
ततोऽश्वस्यापानमधमत्ततोऽश्वाद्धम्यमानात्सर्वस्रोतोभ्यःपावकार्चिषः सधूमा निष्पेतुः
% Check verse!
ताभिर्नागलोक उपधूपितेऽथसंभ्रान्तस्तक्षकोऽग्नेस्तेजोभयाद्विषण्णः कुण्डले गृहीत्वा सहसाभवनान्निष्क्रम्योत्तङ्कमुवाच
\twolineshloka
{इमे कुण्डले गृह्णातु भवानिति}
{स ते प्रतिजग्राहोत्तङ्कः प्रतिगृह्य च कुण्डलेऽचिन्तयत्}


% Check verse!
अद्य तत्पुण्यकमुपाध्यायान्या दूरं चाहमभ्यागतः स कथंसंभावयेयमिति
\twolineshloka
{तत एनं चिन्तयानमेव स पुरुष उवाच}
{उत्तङ्क एनमेवाश्वमधिरोह त्वां क्षणेनैवोपाध्यायकुलं प्रापयिष्यतीति}


\twolineshloka
{स तथेन्युक्त्वा तमश्वमधिरुह्य प्रत्याजगामोपाध्यायकुलं}
{उपाध्यायानी च स्नाता केशानावापयन्त्युपविष्टोत्तङ्को नागच्छतीति शापायास्यमनो दधे}


% Check verse!
अथैतस्मिन्नन्तरे स उत्तङ्कः प्रविश्य उपाध्यायकुलंउपाध्यायानीमभ्यवादयत्ते चास्यै कुण्डले प्रायच्छत्सा चैनंप्रत्युवाच
% Check verse!
उत्तङ्क देशे कालेऽभ्यागतः स्वागतं ते वत्स `इदानीं यद्यनागतोसिकोपितया मया शप्तो भविष्यसि' श्रेयस्तवोपस्थितंसिद्धिमाप्नुहीति
\twolineshloka
{अथोत्तङ्क उपाध्यायमभ्यवादयत्}
{तमुपाध्यायः प्रत्युवाच वत्सोत्तङ्क स्वागतं ते किं चिरं कृतमिति}


\twolineshloka
{तमुत्तङ्क उपाध्यायं प्रत्युवाच}
{भोस्तक्षकेण मे नागराजेन विघ्नः कृतोऽस्मिन्कर्मणि तेनास्मि नागलोकंगतः}


\twolineshloka
{तत्र च मया दृष्टे स्त्रियौ तन्त्रेऽधिरोप्य पटं वयन्त्यौ तस्मिंश्च कृष्णाःसिताश्च तन्तवः}
{किं तत्}


\threelineshloka
{तत्र च मया चक्रं दृष्टं द्वादशारं षट्चैनं कुमाराः परिवर्तयन्तितदपि किं}
{पुरुषश्चापि मया दृष्टः स चापि कः}
{अश्वश्चातिप्रमाणो दृष्टः स चापि कः}


% Check verse!
पथि गच्छता च मया ऋषभो दृष्टस्तं च पुरुषोऽधिरूढस्तेनास्मिसोपचारमुक्त उत्तङ्कास्य ऋषभस्य पुरीषं भक्षय उपाध्यायेनापि तेभक्षितमिति
\threelineshloka
{ततस्तस्य वचनान्मया तदृषभस्य पुरीषमुपयुक्तं स चापि कः}
{तदेतद्भवतोपदिष्टमिच्छेयं श्रोतुं किं तदिति}
{स तेनैवमुक्त उपाध्यायः प्रत्युवाच}


\twolineshloka
{ये ते स्त्रियौ धाता विधाता च ये च ते कृष्णाः सितास्तन्तवस्तेरात्र्यहनी}
{यदपि तच्चक्रं द्वादशारं षट्कुमाराः परिवर्तयन्ति तेपि षड्ऋतवः द्वादशाराद्वादश मासाः संवत्सरश्चक्रम्}


% Check verse!
यः पुरुषःस पर्जन्यः योऽश्वः सोऽग्निः य ऋषभस्त्वया पथि गच्छतादृष्टः स ऐरावतो नागराट्
% Check verse!
यश्चैनमधिरूढः पुरुषः स चेन्द्रः यदपि ते भक्षितं तस्य ऋषभस्यपुरीषं तदमृतं तेन खल्वसि तस्मिन्नागभवने न व्यापन्नस्त्वम्
\twolineshloka
{स हि भगवानिन्द्रो मम सखा त्वदनुक्रोशादिममनुग्रहं कृतवान्}
{तस्मात्कुण्डले गृहीत्वा पुनरागतोऽसि}


\twolineshloka
{तत्सौम्य गम्यतामनुजाने भवन्तं श्रेयोऽवाप्स्यसीति}
{स उपाध्यायेनानुज्ञातो भगवानुत्तङ्कः क्रुद्धस्तक्षकं प्रतिचिकीर्षमाणोहास्तिनपुरं प्रतस्थे}


\twolineshloka
{स हास्तिनपुरं प्राप्य नचिराद्विप्रसत्तमः}
{समागच्छत राजानमुत्तङ्को जनमेजयम्}


\twolineshloka
{पुरा तक्षशिलासंस्थं निवृत्तमपराजितम्}
{सम्यग्विजयिनं दृष्ट्वा समन्तान्मन्त्रिभिर्वृतम्}


\threelineshloka
{तस्मै जयाशिषः पूर्वं यथान्यायं प्रयुज्य सः}
{उवाचैनं वचः काले शब्दसंपन्नया गिरा ॥उत्तङ्क उवाच}
{}


\threelineshloka
{अन्यस्मिन्करणीये तु कार्ये पार्थिवसत्तम}
{अर्चयित्वा यथान्यायं प्रत्युवाच द्विजोत्तमम् ॥सौतिरुवाच}
{}


\threelineshloka
{एवमुक्तस्तु विप्रेण स राजा जनमेजयः}
{अर्चयित्वा यथान्यायं प्रत्युवाच द्विजोत्तमम् ॥जनमेजय उवाच}
{}


\threelineshloka
{आसां प्रजानां परिपालनेनस्वं क्षत्रधर्मं परिपालयामि}
{प्रव्रूहि मे किं करणीयमद्ययेनासि कार्येण समागतस्त्वम् ॥सौतिरुवाच}
{}


\threelineshloka
{स एवमुक्तस्तु नृपोत्तमेनद्विजोत्तमः पुण्यकृतां वरिष्ठः}
{उवाच राजानमदीनसत्वंस्वमेव कार्यं नृपते कुरुष्व ॥उत्तङ्क उवाच}
{}


\twolineshloka
{तक्षकेण महीन्द्रेन्द्र येन ते हिंसितः पिता}
{तस्मै प्रतिकुरुष्व त्वं पन्नगाय दुरात्मने}


\twolineshloka
{कार्यकालं हि मन्येऽहं विधिदृष्टस्य कर्मणः}
{तद्गच्छापचितिं राजन्पितुस्तस्य महात्मनः}


\twolineshloka
{तेन ह्यनपराधी स दष्टो दुष्टान्तरात्मना}
{पञ्चत्वमगमद्राजा वज्राहत इव द्रुमः}


\twolineshloka
{बलदर्पसमुत्सिक्तस्तक्षकः पन्नगाधमः}
{अकार्यं कृतवान्पापो योऽदशत्पितरं तव}


\twolineshloka
{राजर्षिवंशगोप्तारममरप्रतिमं नृपम्}
{यियासुं काश्यपं चैव न्यवर्तयत पापकृत्}


\twolineshloka
{होतुमर्हसि तं पापं ज्वलिते हव्यवाहने}
{सर्पसत्रे महाराज त्वरितं तद्विधीयताम्}


\twolineshloka
{एवं पितुश्चापचितिं कृतवांस्त्वं भविष्यसि}
{मम प्रियं च सुमहत्कृतं राजन् भविष्यति}


\threelineshloka
{कर्मणः पृथिवीपाल मम येन दुरात्मना}
{विघ्नः कृतो महाराज गुर्वर्थं चरतोऽनघ ॥सौतिरुवाच}
{}


\twolineshloka
{एतच्छ्रुत्वा तु नृपतिस्तक्षकाय चुकोप ह}
{उत्तङ्कवाक्यहविषा दीप्तोऽग्निर्हविषा यथा}


\twolineshloka
{अपृच्छत्स तदा राजा मन्त्रिणः स्वान्सुदुःखितः}
{उत्तङ्कस्यैव सांनिध्ये पितुः स्वर्गगतिं प्रति}


\twolineshloka
{तदैव हि स राजेन्द्रो दुःखशोकाप्लुतोऽभवत्}
{यदैव वृत्तं पितरमुत्तङ्कादशृणोत्तदा}


\chapter{अध्यायः ४}
% Check verse!
रोमहर्षणपुत्र उग्रश्रवाः सौतिः पौराणिको नैमिशारण्ये शौनकस्यकुलपतेर्द्वादशवार्षिके सत्रे ऋषीनभ्यागतानुपतस्थे
\fourlineindentedshloka
{पौराणिकः पुराणे कृतश्रमः स कृताञ्जलिस्तानुवाच}
{`मयोत्तङ्कस्य चरितमशेषमुक्तं जनमेजयस्य सार्पसत्रेनिमित्तान्तरमिदमपि}
{'किं भवन्तः श्रोतुमिच्छन्ति किमहं ब्रवाणीति ॥तमृषय ऊचुः}
{}


% Check verse!
परं रौमहर्षणे प्रवक्ष्यामस्त्वां नः प्रतिवक्ष्यसि वचःशुश्रूषतां कथायोगं नः कथायोगे
\twolineshloka
{तत्र भगवान् कुलपतिस्तु शौनकोऽग्निशरणमध्यास्ते}
{`दीर्घसत्रत्वात्सर्वाः कथाः श्रोतुं कालोस्ति ॥'}


\twolineshloka
{यौऽसौ दिव्याः कथा वेद देवतासुरसंश्रिताः}
{मनुष्योरगगन्धर्वकथा वेद च सर्वशः}


\twolineshloka
{स चाप्यस्मिन्मशे सौते विद्वान्कुलपतिर्दिवजः}
{दक्षो धृतव्रतो धीमाञ्शास्त्रे चारण्यके गुरुः}


\twolineshloka
{सत्यवादी शमपरस्तपस्वी नियतव्रतः}
{सर्वेषामेव नो मान्यः स तावत्प्रतिपाल्यताम्}


\threelineshloka
{तस्मिन्नध्यासति गुरावासनं परमार्चितम्}
{ततो वक्ष्यसि यत्त्वां स प्रक्ष्यति द्विजसत्तमः ॥सौतिरुवाच}
{}


\twolineshloka
{एवमस्तु गुरौ तस्मिन्नुपविष्टे महात्मनि}
{तेन पृष्टः कथाः पुण्या वक्ष्यामि विविधाश्रयाः}


\twolineshloka
{सोऽथ विप्रर्षभः सर्वं कृत्वा कार्यं यथाविधि}
{देवान्वाग्भिः पितॄनद्भिस्तर्पयित्वाऽऽजगाम ह}


\twolineshloka
{यत्र ब्रह्मर्षयः सिद्धाः सुखासीना धृतव्रताः}
{यज्ञायतनमाश्रित्य सूतपुत्रपुरस्पराः}


\twolineshloka
{ऋत्विक्ष्वथ सदस्येषु स वै गृहपतिस्तदा}
{उपविष्टेषूपविष्टः शौनकोऽथाब्रवीदिदम्}


\chapter{अध्यायः ५}
\twolineshloka
{शौनक उवाच}
{}


\threelineshloka
{पुराणमखिलं तात पिता तेऽधीतवान्पुरा}
{`भारताध्ययनं सर्वं कृष्णद्वैपायनात्तदा}
{'कच्चित्त्वमपि तत्सर्वमधीषे रौमहर्षणे}


\twolineshloka
{पुराणे हि कथा दिव्या आदिवंशाश्च धीमताम्}
{कथ्यन्ते ये पुराऽस्माभिः श्रुतपूर्वाः पितुस्तव}


\threelineshloka
{तत्र वंशमहं पूर्वं श्रोतुमिच्छामि भार्गवम्}
{कथयस्व कथामेतां कल्याः स्मः श्रवणे तव ॥सौतिरुवाच}
{}


\twolineshloka
{यदधीतं पुरा सम्यग्द्विजश्रेष्ठैर्महात्मभिः}
{वैशंपायनविप्राग्र्यैस्तैश्चापि कथितं यथा}


\twolineshloka
{यदधीतं च पित्रा मे सम्यक्कैव ततो मया}
{तावच्छृणुष्व यो देवैः सेन्द्रैः सर्षिमरुद्गणैः}


\twolineshloka
{पूजितः प्रवरो वंशो भार्गवो भृगुनन्दन}
{इमं वंशमहं पूर्वं भार्गवं ते महामुने}


\twolineshloka
{निगदामि यथायुक्तं पुराणाश्रयसंयुतम्}
{भृगुर्महर्षिर्भगवान्ब्रह्मणा वै स्वयंभुवा}


\twolineshloka
{वरुणस्य क्रतौ जातः पावकादिति नः श्रुतम्}
{भृगोः सुदयितः पुत्रश्च्यवनो नाम भार्गवः}


\twolineshloka
{च्यवनस्य च दायादः प्रमतिर्नाम धार्मिकः}
{प्रमतेरप्यभूत्पुत्रो घृताच्यां रुरुरित्युत}


\twolineshloka
{रुरोरपि सुतो जज्ञे शुनको वेदपारगः}
{प्रमद्वरायां धर्मात्मा तव पूर्वपितामहः}


\threelineshloka
{तपस्वी च यशस्वी च श्रुतवान्ब्रह्मवित्तमः}
{धार्मिकः सत्यवादी च नियतो नियताशनः ॥शौनक उवाच}
{}


\threelineshloka
{सूतपुत्र यथा तस्य भार्गवस्य महात्मनः}
{च्यवनत्वं परिख्यातं तन्ममाचक्ष्व पृच्छतः ॥सौतिरुवाच}
{}


\twolineshloka
{भृगोः सुदयिता भार्या पुलोमेत्यभिविश्रुता}
{तस्यां समभवद्गर्भो भृगुवीर्यसमुद्भवः}


\twolineshloka
{तस्मिन्गर्भेऽथ संभूते पुलोमायां भृगूद्वह}
{समये समशीलिन्यां धर्मपत्न्यां यशस्विनः}


\twolineshloka
{अभिषेकाय निष्क्रान्ते भृगौ धर्मभृतां वरे}
{आश्रमं तस्य रक्षोऽथ पुलोमाऽभ्याजगाम ह}


\twolineshloka
{तं प्रविश्याश्रमं दृष्ट्वा भृगोर्भार्यामनिन्दिताम्}
{हृच्छयेन समाविष्टो विचेताः समपद्यत}


\twolineshloka
{अभ्यागतं तु तद्रक्षः पुलोमा चारुदर्शना}
{न्यमन्त्रयत वन्येन फलमूलादिना तदा}


\twolineshloka
{तां तु रक्षस्तदा ब्रह्मन्हृच्छयेनाभिपीडितम्}
{दृष्ट्वा हृष्टमभूद्राजञ्जिहीर्षुस्तामनिन्दिताम्}


\twolineshloka
{जातमित्यब्रवीत्कार्यं जिहीर्षुर्मुदितः शुभाम्}
{सा हि पूर्वं वृता तेन पुलोम्ना तु शुचिस्मिता}


\twolineshloka
{तां तु प्रादात्पिता पश्चाद्भृगवे शास्त्रवत्तदा}
{तस्य तत्किल्बिषं नित्यं हृदि वर्तति भार्गव}


\twolineshloka
{इदमन्तरमित्येवं हर्तुं चक्रे मनस्तदा}
{अथाग्निशरणेऽपश्यज्ज्वलन्तं जातवेदसम्}


\twolineshloka
{तमपृच्छत्ततो रक्षः पावकं ज्वलितं तदा}
{शंस मे कस्य भार्येयमग्ने पृच्छे ऋतेन वै}


\twolineshloka
{मुखं त्वमसि देवानां वद पावक पृच्छते}
{मया हीयं वृता पूर्वं भार्यार्थे वरवर्णिनी}


\twolineshloka
{पश्चादिमां पिता प्रादाद्भृगवेऽनृतकारकः}
{सेयं यदि वरारोहा भृगोर्भार्या रहोगता}


\twolineshloka
{तथा सत्यं समाख्याहि जिहीर्षाम्याश्रमादिमाम्}
{स मन्युस्तत्र हृदयं प्रदहन्निव तिष्ठति}


\threelineshloka
{मत्पूर्वभार्यां यदिमां भृगुराप सुमध्यमाम्}
{`असंमतमिदं मेऽद्य हरिष्याम्याश्रमादिमाम्' ॥सौतिरुवाच}
{}


\twolineshloka
{एवं रक्षस्तमामन्त्र्य ज्वलितं जातवेदसम्}
{शङ्कमानं भृगोर्भार्यां पुनःपुनरपृच्छत}


\twolineshloka
{त्वमग्ने सर्वभूतानामन्तश्चरसि नित्यदा}
{साक्षिवत्पुण्यपापेषु सत्यं ब्रूहि कवे वचः}


\twolineshloka
{मत्पूर्वभार्याऽपहृता भृगुणाऽनृतकारिणा}
{सेयं यदि तथा मे त्वं सत्यमाख्यातुमर्हसि}


\threelineshloka
{श्रुत्वा त्वत्तो भृगोर्भार्यां हरिष्याम्याश्रमादिमाम्}
{जातवेदः पश्यतस्ते वद सत्यां गिरं मम ॥सौतिरुवाच}
{}


\twolineshloka
{तस्यैतद्वचनं श्रुत्वा सप्तार्चिर्दुःखितोऽभवत्}
{`सत्यं वदामि यदि मे शापः स्याद्ब्रह्मवित्तमात्}


\twolineshloka
{असत्यं चेदहं ब्रूयां पतिष्ये नरकान्ध्रुवम्}
{'भीतोऽनृताच्च शापाच्च भृगोरित्यब्रवीच्छनैः}


\twolineshloka
{त्वया वृता पुलोमेयं पूर्वं दानवनन्दन}
{किं त्वियं विधिना पूर्वं मन्त्रवन्न वृता त्वया}


\twolineshloka
{पित्रा तु भृगवे दत्ता पुलोमेयं यशस्विनी}
{ददाति न पिता तुभ्यं वरलोभान्महायशाः}


\twolineshloka
{अथेमां वेददृष्टेन कर्मणा विधिपूर्वकम्}
{भार्यामृषिर्भृगुः प्राप मां पुरस्कृत्य दानव}


\twolineshloka
{सेयमित्यवगच्छामि नानृतं वक्तुमुत्सहे}
{नानृतं हि सदा लोके पूज्यते दानवोत्तम}


\chapter{अध्यायः ६}
\twolineshloka
{सौतिरुवाच}
{}


\twolineshloka
{अग्नेरथ वचः श्रुत्वा तद्रक्षः प्रजहार ताम्}
{ब्रह्मन्वराहरूपेण मनोमारुतरंहसा}


\twolineshloka
{ततः स गर्भो निवसन्कुक्षौ भृगुकुलोद्वह}
{रोषान्मातुश्च्युतः कुक्षेश्च्यवनस्तेन सोऽभवत्}


\twolineshloka
{तं दृष्ट्वा मातुरुदराच्च्युतमादित्यवर्चसम्}
{तद्रक्षो भस्मसाद्भूतं पपात परिमुच्य ताम्}


\twolineshloka
{सा तमादाय सुश्रोणी ससार भृगुनन्दनम्}
{च्यवनं भार्गवं पुत्रं पुलोमा दुःखमूर्च्छिता}


\twolineshloka
{तां ददर्श स्वयं ब्रह्मा सर्वलोकपितामहः}
{रुदतीं बाष्पपूर्णाक्षीं भृगोर्भार्यामनिन्दिताम्}


\twolineshloka
{सान्त्वयामास भगवान्वधूं ब्रह्मा पितामहः}
{अश्रुबिन्दूद्भवा तस्याः प्रावर्तत महानदी}


\twolineshloka
{आवर्तन्ती सृतिं तस्या भृगोः पत्न्यास्तपस्विनः}
{तस्या मार्गं सृतवतीं दृष्ट्वा तु सरितं तदा}


\twolineshloka
{नाम तस्यास्तदा नद्याश्चक्रे लोकपितामहः}
{वधूसरेति भगवांश्च्यवनस्याश्रमं प्रति}


\fourlineindentedshloka
{स एवं च्यवनो जज्ञे भृगोः पुत्रः प्रतापवान्}
{तं ददर्श पिता तत्र च्यवनं तां च भामिनीम्}
{स पुलोमां ततो भार्यां पप्रच्छ कुपितो भृगुः ॥भृगुरुवाच}
{}


\twolineshloka
{केनासि रक्षसे तस्मै कथिता त्वं जिहीर्षवे}
{न हि त्वा वेद तद्रक्षो मद्भार्यां चारुहासिनीम्}


\threelineshloka
{तत्त्वमाख्याहि तं ह्यद्य शप्तुमिच्छाम्यहं रुषा}
{बिभेति को न शापान्मे कस्य चायं व्यतिक्रमः ॥पुलोमोवाच}
{}


\twolineshloka
{अग्निना भगवंस्तस्मै रक्षसेऽहं निवेदिता}
{ततो मामनयद्रक्षः क्रोशन्तीं कुररीमिव}


\threelineshloka
{साऽहं तव सुतस्यास्य तेजसा परिमोक्षिता}
{भस्मीभूतं च तद्रक्षो मामुत्सृज्य पपात वै ॥सौतिरुवाच}
{}


\twolineshloka
{इति श्रुत्वा पुलोमाया भृगुः परममन्युमान्}
{शशापाग्निमतिक्रुद्धः सर्वभक्षो भविष्यसि}


\chapter{अध्यायः ७}
\twolineshloka
{सौतिरुवाच}
{}


\twolineshloka
{शप्तस्तु भृगुणा वह्निः क्रुद्धो वाक्यमथाब्रवीत्}
{किमिदं साहसं ब्रह्मन्कृतवानसि मां प्रति}


\twolineshloka
{धर्मे प्रयतमानस्य सत्यं च वदतः समम्}
{पृष्टो यदब्रवं सत्यं व्यभिचारोऽत्र को मम}


\twolineshloka
{पृष्टो हिसाक्षीयः साक्ष्यं जानानोऽप्यन्यथा वदेत्}
{स पूर्वानात्मनः सप्त कुले हन्यात्तथाऽपरान्}


\twolineshloka
{यश्च कार्यार्थतत्त्वज्ञो जानानोऽपि न भाषते}
{सोऽपि तेनैव पापेन लिप्यते नात्र संशयः}


\twolineshloka
{शक्तोऽहमपि शप्तुं त्वां मान्यास्तु ब्राह्मणा मम}
{जानतोऽपि च ते ब्रह्मन्कथयिष्ये निबोध तत्}


\twolineshloka
{योगेन बहुधाऽऽत्मानं कृत्वा तिष्ठामि मूर्तिषु}
{अग्निहोत्रेषु सत्रेषु क्रियासु च मखेषु च}


\twolineshloka
{वेदोक्तेन विधानेन मयि यद्धूयते हविः}
{देवताः पितरश्चैव तेन तृप्ता भवन्ति वै}


\twolineshloka
{आपो देवगणाः सर्वे आपः पितृगणास्तथा}
{दर्शश्च पौर्णमासश्च देवानां पितृभिः सह}


\twolineshloka
{देवताः पितरस्तस्मात्पितरश्चापि देवताः}
{एकीभूताश्च दृश्यन्ते पृथक्त्वेन च पर्वसु}


\twolineshloka
{देवताः पितरश्चैव भुञ्जते मयि यद्भुतम्}
{देवतानां पितॄणां च मुखमेतदहं स्मृतम्}


\twolineshloka
{अमावास्यां हि पितरः पौर्णमास्यां हि देवताः}
{मन्मुखेनैव हूयन्ते भुञ्जते च हुतं हविः}


\threelineshloka
{सर्वभक्षः कथं त्वेषां भविष्यामि मुखं त्वहम्}
{सौतिरुवाच}
{चिन्तयित्वा ततो वह्निश्चके संहारमात्मनः}


\twolineshloka
{द्विजानामग्निहोत्रेषु यज्ञसत्रक्रियासु च}
{निरोंकारवषट्काराः स्वधास्वाहाविवर्जिताः}


\twolineshloka
{विनाऽग्निना प्रजाः सर्वास्तत आसन्सुदुःखिताः}
{अथर्षयः समुद्विग्ना देवान् गत्वाब्रुवन्वचः}


\twolineshloka
{अग्निनाशात्क्रियाभ्रांशाद्भ्रान्ता लोकास्त्रयोऽनघाः}
{विधध्वमत्र यत्कार्यं न स्यात्कालात्ययो यथा}


\twolineshloka
{अथर्षयश्च देवाश्च ब्रह्माणमुपगम्य तु}
{अग्नेरावेदयञ्शापं क्रियासंहारमेव च}


\twolineshloka
{भृगुणा वै महाभाग शप्तोऽग्निः कारणान्तरे}
{कथं देवमुखो भूत्वा यज्ञभागाग्रभुक् तथा}


\threelineshloka
{हुतभुक्सर्वलोकेषु सर्वभक्षत्वमेष्यति}
{सौतिरुवाच}
{श्रुत्वा तु तद्वचस्तेषामग्निमाहूय विश्वकृत्}


\twolineshloka
{उवाच वचनं श्लक्ष्णं भूतभावनमव्ययम्}
{लोकानामिह सर्वेषां त्वं कर्ता चान्त एव च}


\twolineshloka
{त्वं धारयसि लोकांस्त्रीन्क्रियाणां च प्रवर्तकः}
{स तथा कुरु लोकेश नोच्छिद्येरन्यथा क्रियाः}


\twolineshloka
{कस्मादेवं विमूढस्त्वमीश्वरः सन् हुताशेन}
{त्वं पवित्रं सदा लोके सर्वभूतगतिश्च ह}


\twolineshloka
{न त्वं सर्वशरीरेण सर्वभक्षत्वमेष्यसि}
{अपाने ह्यर्चिषो यास्ते सर्वं भक्ष्यन्ति ताः शिखिन्}


\twolineshloka
{क्रव्यादा च तनुर्या ते सा सर्वं भक्षयिष्यति}
{यथा सूर्यांशुभिः स्पृष्टं सर्वं शुचि विभाव्यते}


\twolineshloka
{तथा त्वदर्चिर्निर्दग्धं सर्वं शुचि भविष्यति}
{त्वमग्ने परमं तेजः स्वप्रभावाद्विनिर्गतम्}


\threelineshloka
{स्वतेजसैव तं शापं कुरु सत्यमृषेर्विभो}
{देवानां चात्मनो भागं गृहाण त्वं मुखे हुतम् ॥सौतिरुवाच}
{}


\twolineshloka
{एवमस्त्विति तं वह्निः प्रत्युवाच पितामहम्}
{जगाम शासनं कर्तुं देवस्य परमेष्ठिनः}


\twolineshloka
{देवर्षयश्च मुदितास्ततो जग्मुर्यथागतम्}
{ऋषयश्च यथा पूर्वं क्रियाः सर्वाः प्रचक्रिरे}


\twolineshloka
{दिवि देवा मुमुदिरे भूतसङ्घाश्च लौकिकाः}
{अग्निश्च परमां प्रीतिमवाप हतकल्मषः}


\threelineshloka
{एवं स भगवाञ्छापं लेभेऽग्निर्भृगुतः पुरा}
{एवमेष पुरा वृत्त हतिहासोऽग्निशापजः}
{पुलोम्नश्च विनाशोऽयं च्यवनस्य च संभवः}


\chapter{अध्यायः ८}
\twolineshloka
{सौतिरुवाच}
{}


\twolineshloka
{स चापि च्यवनो ब्रह्मन्भार्गवोऽजनयत्सुतम्}
{सुकन्यायां महात्मानं प्रमतिं दीप्ततेजसम्}


\twolineshloka
{प्रमतिस्तु रुरुं नाम घृताच्यां समजीजनत्}
{रुरुः प्रमद्वरायां तु शुनकं समजीजनम्}


\twolineshloka
{शुनकस्तु महासत्वः सर्वभार्गवनन्दनः}
{जातस्तपसि तीव्रे च स्थितः स्थिरयशास्ततः}


\twolineshloka
{तस्य ब्रह्मन्रुरोः सर्वं चरितं भूरितेजसः}
{विस्तरेण प्रवक्ष्यामि तच्छृणु त्वमशेषतः}


\twolineshloka
{ऋषिरासीन्महान्पूर्वं तपोविद्यासमन्वितः}
{स्थूलकेश इति ख्यातः सर्वभूतहिते रतः}


\twolineshloka
{एतस्मिन्नेव काले तु मेनकायां प्रजज्ञिवान्}
{गन्धर्वराजो विप्रर्षे विश्वावसुरिति स्मृतः}


\twolineshloka
{अप्सरा मेनका तस्य तं गर्भं भृगुनन्दन}
{उत्ससर्ज यथाकालं स्थूलकेशाश्रमं प्रति}


\twolineshloka
{उत्सृज्य चैव तं गर्भं नद्यास्तीरे जगाम सा}
{अप्सरा मेनका ब्रह्मन्निर्दया निरपत्रपा}


\twolineshloka
{कन्याममरगर्भाभां ज्वलन्तीमिव च श्रिया}
{तां ददर्श समुत्सृष्टां नदीतीरे महानृषिः}


\twolineshloka
{स्थूलकेशः स तेजस्वी विजने बन्धुवर्जिताम्}
{स तां दृष्ट्वा तदा कन्यां स्थूलकेशो महाद्विजः}


\twolineshloka
{जग्राह च मुनिश्रेष्ठः कृपाविष्टः पुपोष च}
{ववृधे सा वरारोहा तस्याश्रमपदे शुभे}


\twolineshloka
{जातकाद्याः क्रियाश्चास्या विधिपूर्वं यथाक्रमम्}
{स्थूलकेशो महाभागश्चकार सुमहानृषिः}


\twolineshloka
{प्रमदाभ्यो वरा सा तु सत्त्वरूपगुणान्विता}
{ततः प्रमद्वरेत्यस्या नाम चक्रे महानृषिः}


\twolineshloka
{तामाश्रमपदे तस्य रुरुर्दृष्ट्वा प्रमद्वराम्}
{बभूव किल धर्मात्मा मदनोपहतस्तदा}


\twolineshloka
{पितरं सखिभिः सोऽथ श्रावयामास भार्गवम्}
{प्रमतिश्चाभ्ययाचत्तां स्थूलकेशं यशस्विनम्}


\twolineshloka
{ततः प्रादात्पिता कन्यां रुरवे तां प्रमद्वराम्}
{विवाहं स्थापयित्वाग्रे नक्षत्रे भगदैवते}


\twolineshloka
{ततः कतिपयाहस्य विवाहे समुपस्थिते}
{सखीभिः क्रीडती सार्धं सा कन्यावरवर्णिनी}


\twolineshloka
{नापश्यत्संप्रसुप्तं वै भुजंगं तिर्यगायतम्}
{पदा चैनं समाक्रामन्मुमूर्षुः कालचोदिता}


\twolineshloka
{स तस्याः संप्रमत्तायाश्चोदितः कालधर्मणा}
{विषोपलिप्तान्दशनान्भृशमङ्गे न्यपातयत्}


\twolineshloka
{सा दष्टा तेन सर्पेण पपात सहसा भुवि}
{विवर्णा विगतश्रीका भ्रष्टाभरणचेतना}


\twolineshloka
{निरानन्दकरी तेषां बन्धूनां मुक्तमूर्धजा}
{व्यसुरप्रेक्षणीया सा प्रेक्षणीयतमाऽभवत्}


\twolineshloka
{प्रसुप्तेवाभवच्चापि भुवि सर्पविषार्दिता}
{भूयो मनोहरतरा बभूव तनुमध्यमा}


\twolineshloka
{ददर्श तां पिता चैव ये चैवान्ये तपस्विनः}
{विचेष्टमानां पतितां भूतले पद्मवर्चसम्}


\twolineshloka
{ततः सर्वे द्विजतराः समाजग्मुः कृपान्विताः}
{स्वस्त्यात्रेयो महाजानुः कुशिकः शङ्खमेखलः}


\twolineshloka
{उद्दालकः कठश्चैव श्वेतश्चैव महायशाः}
{भरद्वाजः कौणकृत्स्य आर्ष्टिषेणोऽथ गौतमः}


\twolineshloka
{प्रमतिः सह पुत्रेण तथान्ये वनवासिनः}
{तां ते कन्यां व्यसुं दृष्ट्वा भुजंगस्य विषार्दिताम्}


\twolineshloka
{रुरुदुः कृपयाऽविष्टा रुरुस्त्वार्तो बहिर्ययौ}
{ते च सर्वे द्विजश्रेष्ठास्तत्रैवोपाविशंस्तदा}


\chapter{अध्यायः ९}
\twolineshloka
{सौतिरुवाच}
{}


\twolineshloka
{तेषु तत्रोपविष्टेषु ब्राह्मणेषु महात्मसु}
{रुरुश्चुक्रोश गहनं वनं गत्वाऽतिदुःखितः}


\twolineshloka
{शोकेनाभिहतः सोऽथ विलपन्करुणं बहु}
{अब्रवीद्वचनं शोचन्प्रियां स्मृत्वा प्रमद्वराम्}


\twolineshloka
{शेते सा भुवि तन्वङ्गी मम शोकविवर्धिनी}
{`प्राणानपहरन्तीव पूर्णचन्द्रनिभानना}


\twolineshloka
{यदि पीनायतश्रोणी पद्मपत्रनिभेक्षणा}
{मुमूर्षुरपि मे प्राणानादायाशु गमिष्यति}


\twolineshloka
{पितृमातृसखीनां च लुप्तपिण्डस्य तस्य मे}
{'बान्धवानां च सर्वेषां किं नु दुःखमतःपरम्}


\twolineshloka
{यदि दत्तं तपस्तप्तं गुरवो वा मया यदि}
{सम्यगाराधितास्तेन संजीवतु मम प्रिया}


\twolineshloka
{यथा च जन्मप्रभृति यतात्माऽहं धृतव्रतः}
{प्रमद्वरा तथाद्यैषा समुत्तिष्ठतु भामिनी}


\twolineshloka
{[एवं लालप्यतस्तस्य भार्यार्थे दुःखितस्य च}
{देवदूतस्तदाऽभ्येत्य वाक्यमाह रुरुं वने ॥]}


\twolineshloka
{`कृष्णे विष्णौ हृषीकेशे लोकेशेऽसुरविद्विषि}
{यदि मे निश्चला भक्तिर्मम जीवतु सा प्रिया}


\twolineshloka
{विलप्यमाने तु रुरौ सर्वे देवाः कृपान्विताः}
{दूतं प्रस्थापयामासुः संदिश्यास्य हितं वचः}


\twolineshloka
{स दूतस्त्वरितोऽभ्येत्य देवानां प्रियकृच्छुचिः}
{उवाच देववचनं रुरुमाभाष्य दुःखितम्}


\twolineshloka
{देवैः सर्वैरहं ब्रह्मन्प्रेषितोऽस्मि तवान्तिकम्}
{त्वद्धितं त्वद्धितैरुक्तं शृणु वाक्यं द्विजोत्तम ॥'}


\twolineshloka
{अभिधत्से ह यद्वाचा रुरो दुःखान्न तन्मृषा}
{न तु मर्त्यस्य धर्मात्मन्नायुरस्ति गतायुषः}


\twolineshloka
{गतायुरेषा कृपणा गन्धर्वाप्सरसोः सुता}
{तस्माच्छोके मनस्तात मा कृथास्त्वं कथंचन}


\threelineshloka
{उपायश्चात्र विहितः पूर्वं देवैर्महात्मभिः}
{तं यदीच्छसि कर्तुं त्वं प्राप्स्यसीह प्रमद्वराम् ॥रुरुरुवाच}
{}


\threelineshloka
{क उपायः कृतो देवैर्बूहि तत्त्वेन खेचर}
{करिष्येऽहं तथा श्रुत्वा त्रातुमर्हति मां भवान् ॥देवेदूत उवाच}
{}


\threelineshloka
{आयुषोऽर्धं प्रयच्छ त्वं कन्यायै भृगुनन्दन}
{एवमुत्थास्यति रुरो तव भार्या प्रयद्वरा ॥रुरुरुवाच}
{}


\threelineshloka
{आयुषोऽर्धं प्रयच्छामि कन्यायै खेचरोत्तम}
{शृङ्गाररूपाभरणा समुत्तिष्ठतु मे प्रिया ॥सौतिरुवाच}
{}


\twolineshloka
{ततो गन्धर्वराजश्च देवदूतश्च सत्तमौ}
{धर्मराजमुपेत्येदं वचनं प्रत्यभाषताम्}


\threelineshloka
{धर्मराजायुषोऽर्धेन रुरोर्भार्या प्रमद्वरा}
{समुत्तिष्ठतु कल्याणी मृतैवं यदि मन्यसे ॥धर्मराज उवाच}
{}


\threelineshloka
{प्रमद्वरा रुरोर्भार्या देवदूत यदीच्छसि}
{उत्तिष्ठत्वायुषोऽर्धेन रुरोरेव समन्विता ॥सौतिरुवाच}
{}


\twolineshloka
{एवमुक्ते ततः कन्या सोदतिष्ठत्प्रमद्वरा}
{रुरोस्तस्यायुषोऽर्धेन सुप्तेव वरवर्णिनी}


\twolineshloka
{एतद्दृष्टं भविष्ये हि रुरोरुत्तमतेजसः}
{आयुषोऽतिप्रवृद्धस्य भार्यार्थेऽर्धमलुप्यत}


\twolineshloka
{तत इष्टेऽहनि तयोः पितरौ चक्रतुर्मुदा}
{विवाहं तौ च रेमाते परस्परहितैषिणौ}


\twolineshloka
{स लब्ध्वा दुर्लभां भार्यां पद्मकिञ्जल्कसुप्रभाम्}
{व्रतं चक्रे विनाशाय जिह्मगानां धृतव्रतः}


\twolineshloka
{स दृष्ट्वा जिह्मगान्सर्वांस्तीव्रकोपसमन्वितः}
{अभिहन्ति यथासत्त्वं गृह्य प्रहरणं सदा}


\twolineshloka
{स कदाचिद्वनं विप्रो रुरुरभ्यागमन्महत्}
{शयानं तत्र चापश्यड्डुण्डुभं वयसान्वितम्}


\twolineshloka
{तत उद्यम्य दम्डं स कालदण्डोपमं तदा}
{जिघांसुः कुपितो विप्रस्तमुवाचाथ डुण्डुभः}


\twolineshloka
{नापराध्यामि ते किंचिदहमद्य तपोधन}
{संरम्भाच्च किमर्थं मामभिहंसि रुषान्वितः}


\chapter{अध्यायः १०}
\twolineshloka
{रुरुरुवाच}
{}


\twolineshloka
{मम प्राणसमा भार्या दष्टासीद्भुजगेन ह}
{तत्र मे समयो घोर आत्मनोरग वै कृतः}


\threelineshloka
{भुजंगं वै सदा हन्यां यं यं पश्येयमित्युत}
{ततोऽहं त्वां जिघांसामि जीवितेनाद्य मोक्ष्यसे ॥डुण्डुभ उवाच}
{}


\twolineshloka
{अन्ये ते भुजगा ब्रह्मन्ये दश्तीह मानवान्}
{डुण्डुभानहिगन्धेन न त्वं हिंसितुमर्हसि}


\threelineshloka
{एकानर्थान्पृथग्धर्मानेकदुःखान्पृथक्सुखान्}
{डुण्डुभान्धर्मविद्भूत्वा न त्वं हिंसितुमर्हसि ॥सौतिरुवाच}
{}


\twolineshloka
{इति श्रुत्वा वचस्तस्य डुण्डुभस्य रुरुस्तदा}
{नावधीद्भयसंविग्नमृषिं मत्त्वाऽथ डुण्डुभम्}


\threelineshloka
{उवाच चैनं भगवान्रुरुः संशमयन्निव}
{केन त्वं भुजग ब्रूहि कोऽसीमां विक्रियां गतः ॥डुण्डुभ उवाच}
{}


\threelineshloka
{अहं पुरा रुरो नाम्ना ऋषिरासं सहस्रपात्}
{सोऽहं शापेन विप्रस्य भुजगत्वमुपागतः ॥रुरुरुवाच}
{}


\twolineshloka
{किमर्थं शप्तवान्कुद्धो द्विजस्त्वां भुजगोत्तम}
{कियन्तं चैव कालं ते वपुरेतद्भविष्यसि}


\chapter{अध्यायः ११}
\twolineshloka
{डुण्डुभ उवाच}
{}


\twolineshloka
{सखा बभूव मे पूर्वं खगमो नाम वै द्विजः}
{भृशं संशितवाक्तात तपोबलसमन्वितः}


\twolineshloka
{स मया क्रीडता बाल्ये कृत्वा तार्णं भुजंगमम्}
{अग्निहोत्रे प्रसक्तस्तु भीषितः प्रमुमोह वै}


\twolineshloka
{लब्ध्वा स च पुनः संज्ञां मामुवाच तपोधनः}
{नर्दहन्निव कोपेन सत्यवाक्संशितव्रतः}


\twolineshloka
{यथावीर्यस्त्वया सर्पः कृतोऽयं मद्बिभीषया}
{तथावीर्यो भुजंगस्त्वं मम शापाद्भविष्यसि}


\twolineshloka
{तस्याहं तपसो वीर्यं जानन्नासं तपोधन}
{भृशमुद्विग्नहृदयस्तमवोचमहं तदा}


\twolineshloka
{प्रणतः संभ्रमाच्चैव प्राञ्जलिः पुरतः स्थितः}
{सखेति हसतेदं ते नर्मार्थं वै कृतं मया}


\twolineshloka
{क्षन्तुमर्हसि मे ब्रह्मञ्शापोऽयं विनिवर्त्यताम्}
{सोऽथ मामब्रवीद्दृष्ट्वा भृशमुद्विग्नचेतसम्}


\twolineshloka
{मुहुरुष्णं विनिःश्वस्य सुसंभ्रान्तस्तपोधनः}
{नानृतं वै मया प्रोक्तं भवितेदं कथंचन}


\twolineshloka
{यत्तु वक्ष्यामि ते वाक्यं शृणु तन्मे तपोधन}
{श्रुत्वा च हृदि ते वाक्यमिदमस्तु सदाऽनघ}


\threelineshloka
{उत्पत्स्यति रुरुर्नाम प्रमतेरात्मजः शुचिः}
{तं दृष्ट्वा शापमोक्षस्ते भविता नचिरादिव}
{`एवमुक्तस्तु तेनाहमुरगत्वमवाप्तवान् ॥'}


\threelineshloka
{स त्वं रुरुरिति ख्यातः प्रमतेरात्मजोऽपि च}
{स्वरूपं प्रतिपद्याहमद्य वक्ष्यामि ते हितम् ॥सौतिरुवाच}
{}


\twolineshloka
{स डौण्डुभं परित्यज्य रूपं विप्रर्षभस्तदा}
{स्वरूपं भास्वरं भूयः प्रतिपेदे महायशाः}


\twolineshloka
{इदं चोवाच वचनं रुरुमप्रतिमौजसम्}
{अहिंसा परमो धर्मः सर्वप्राणभृतां वर}


\twolineshloka
{तस्मात्प्राणभृतः सर्वान्न हिंस्याद्ब्राह्मणः क्वचित्}
{ब्राह्मणः सौम्य एवेह भवतीति परा श्रुतिः}


\twolineshloka
{वेदवेदाङ्गविन्नाम सर्वभूताभयप्रदः}
{अहिंसा सत्यवचनं क्षमा चेति विनिश्चितम्}


\twolineshloka
{ब्राह्मणस्य परो धर्मो वेदानां धारणापि च}
{क्षत्रियस्य हि यो धर्मः स नेहेष्येत वै तव}


\twolineshloka
{दण्डधारणमुग्रत्वं प्रजानां परिपालनम्}
{तदिदं क्षत्रियस्यासीत्कर्म वै शृणु मे रुरो}


\twolineshloka
{जनमेजयस्य यज्ञेऽस्मिन्सर्पाणां हिंसनं पुरा}
{परित्राणं च भीतानां सर्पाणां ब्राह्मणादपि}


\twolineshloka
{तपोवीर्यबलोपेताद्वेदवेदाङ्गपारगात्}
{आस्तीकाद्द्विजमुख्याद्वै सर्पसत्रे द्विजोत्तम}


\chapter{अध्यायः १२}
\twolineshloka
{रुरुरुवाच}
{}


\twolineshloka
{कथं हिंसितवान्सर्पान्स राजा जनमेजयः}
{सर्पा वा हिंसितास्तत्र किमर्थं द्विजसत्तम}


\threelineshloka
{किमर्थं मोक्षिताश्चैव पन्नगास्तेन धीमता}
{आस्तीकेन द्विजश्रेष्ठ श्रोतुमिच्छाम्यशेषतः ॥ऋषिरुवाच}
{}


\threelineshloka
{श्रोष्यसि त्वं रुरो सर्वमास्तीकचरितं महत्}
{ब्राह्मणानां कथयतां त्वरावान्गमने ह्यहम् ॥सौतिरुवाच}
{}


\twolineshloka
{`इत्युक्त्वान्तर्हिते योगात्तस्मिन्नृषिवरे प्रभौ}
{संभ्रमाविष्टहृदयो रुरुर्मेने तदद्भुतम् ॥'}


\twolineshloka
{बलं परममास्थाय पर्यधावत्समन्ततः}
{तमृषिं नष्टमन्विच्छन्संश्रान्तो न्यपतद्भुवि}


\twolineshloka
{स मोहे परमं गत्वा नष्टसंज्ञ इवाभवत्}
{तदृषेर्वचनं तथ्यं चिन्तयानः पुनःपुनः}


\twolineshloka
{लब्धसंज्ञो रुरुश्चायात्तदाचख्यौ पितुस्तदा}
{`पित्रे तु सर्वमाख्याय डुण्डुभस्य वचोऽर्थवत्}


\twolineshloka
{अपृच्छत्पितरं भूयः सोस्तीकस्य वचस्तदा}
{आख्यापयत्तदाऽऽख्यानं डुण्डुभेनाथ कीर्तितम्}


\twolineshloka
{तत्कीर्त्यमानं भगवञ्श्रोतुमिच्छामि तत्त्वतः}
{'पिता चास्य तदाख्यानं पृष्टः सर्वं न्यवेदयत्}


\chapter{अध्यायः १३}
\twolineshloka
{शौनक उवाच}
{}


\twolineshloka
{किमर्थं राजशार्दूलः स राजा जनमेजयः}
{सर्पसत्रेण सर्पाणां गतोऽन्तं तद्वदस्व मे}


\twolineshloka
{निखिलेन यथातत्त्वं सौते सर्वमशेषतः}
{आस्तीकश्च द्विजश्रेष्ठः किमर्थं जपतां वरः}


\twolineshloka
{मोक्षयामास भुजगान्प्रदीप्ताद्वसुरेतसः}
{कस्य पुत्रः स राजासीत्सर्पसत्रं य आहरत्}


\threelineshloka
{स च द्विजातिप्रवरः कस्य पुत्रोऽभिधत्स्व मे}
{सौतिरुवाच}
{महदाक्यानमास्तीकं यथैतत्प्रोच्यते द्विज}


\threelineshloka
{सर्वमेतदशेषेण शृणु मे वदतां वर}
{शौनक उवाच}
{श्रोतुमिच्छाम्यशेषेण कथामेतां मनोरमाम्}


\threelineshloka
{आस्तीकस्य पुराणर्षेर्ब्राह्मणस्य यशस्विनः}
{सौतिरुवाच}
{इतिहासमिमं विप्राः पुराणं परिचक्षते}


\twolineshloka
{कृष्णद्वैपायनप्रोक्तं नैमिषारण्यवासिषु}
{पूर्वं प्रचोदितः सूतः पिता मे लोमहर्षणः}


\twolineshloka
{शिष्यो व्यासस्य मेधावी ब्राह्मणेष्विदमुक्तवान्}
{तस्मादहमुपश्रुत्य प्रवक्ष्यामि यथातथम्}


\twolineshloka
{इदमास्तीकमाख्यानं तुभ्यं शौनक पृच्छते}
{कथयिष्याम्यशेषेण सर्वपापप्रणाशनम्}


\twolineshloka
{आस्तीकस्य पिता ह्यासीत्प्रजापतिसमः प्रभुः}
{ब्रह्मचारी यताहारस्तपस्युग्रे रतः सदा}


\twolineshloka
{जरत्कारुरिति ख्यात ऊर्ध्वरेता महातपाः}
{यायावराणां प्रवरो धर्मज्ञः संशितव्रतः}


\twolineshloka
{स कदाचिन्महाभागस्तपोबलसमन्वितः}
{चचार पृथिवीं सर्वां यत्रसायंगृहो मुनिः}


\twolineshloka
{तीर्थेषु च समाप्लावं कुर्वन्नटति सर्वशः}
{चरन्दीक्षां महातेजा दुश्चरामकृतात्मभिः}


\twolineshloka
{वायुभक्षो निराहारः शुष्यन्ननिमिषो मुनिः}
{इतस्ततः परिचरन्दीप्तपावकसप्रभः}


\twolineshloka
{अटमानः कदाचित्स्वान्स ददर्श पितामहान्}
{लम्बमानान्महागर्ते पादैरूर्ध्वैरवाङ्मुखान्}


\twolineshloka
{तानब्रवीत्स दृष्ट्वै जरत्कारुः पितामहान्}
{के भवन्तोऽवलम्बन्ते गर्ते ह्यस्मिन्नधोमुखाः}


\threelineshloka
{वीरणस्तम्भके लग्नाः सर्वतः परिभक्षिते}
{मूषकेन निगूढेन गर्तेऽस्मिन्नित्यवासिना ॥पितर ऊचुः}
{}


\twolineshloka
{यायावरा नाम वयमृषयः संशितव्रताः}
{संतानप्रक्षयाद्ब्रह्मन्नधो गच्छाम मेदिनीम्}


\twolineshloka
{अस्माकं संततिस्त्वेको जरत्कारुरिति स्मृतः}
{मन्दभाग्योऽल्पभाग्यानां तप एकं समास्थितः}


\twolineshloka
{न स पुत्राञ्जनयितुं दारान्मूढश्चिकीर्षति}
{तेन लम्बामहे गर्ते संतानस्य क्षयादिह}


\twolineshloka
{अनाथास्तेन नाथेन यथा दुष्कृतिनस्तथा}
{`येषां तु संततिर्नास्ति मर्त्यलोके सुखावहा}


\twolineshloka
{न ते लभन्ते वसतिं स्वर्गे पुण्यकृतोऽपि हि}
{'कस्त्वं बन्धुरिवास्माकमनुशोचसि सत्तम}


\threelineshloka
{ज्ञातुमिच्छामहे ब्रह्मन्को भवानिह नः स्थितः}
{किमर्थं चैव नः शोच्याननुशोचसि सत्तम ॥जरत्कारुरुवाच}
{}


\threelineshloka
{मम पूर्वे भवन्तो वै पितरः सपितामहाः}
{ब्रूत किं करवाण्यद्य जरत्कारुरहं स्वयम् ॥पितर ऊचुः}
{}


\twolineshloka
{यतस्व यत्नवांस्तात संतानाय कुलस्य नः}
{आत्मनोऽर्थेऽस्मदर्थे च धर्म इत्येव वा विभो}


\twolineshloka
{न हि धर्मफलैस्तात न तपोऽभिः सुसंचितैः}
{तां गतिं प्राप्नुवन्तीह पुत्रिणो यां व्रजन्ति वै}


\threelineshloka
{तद्दारग्रहणे यत्नं संतत्यां च मनः कुरु}
{पुत्रकास्मन्नियोगात्त्वमेतन्नः परमं हितम् ॥जरत्कारुरुवाच}
{}


\twolineshloka
{न दारान्वै करिष्येऽहं न धनं जीवितार्थतः}
{भवतां तु हितार्थाय करिष्ये दारसंग्रहम्}


\twolineshloka
{समयेन च कर्ताऽहमनेन विधिपूर्वकम्}
{तथा यद्युपलप्स्यामि करिष्ये नान्यथा ह्यहम्}


\twolineshloka
{सनाम्नी या भवित्री मे दित्सिता चैव बन्धुभिः}
{भैक्ष्यवत्तामहं कन्यामुपयंस्ये विधानतः}


\twolineshloka
{दरिद्राय हि मे भार्यां को दास्यति विशेषतः}
{प्रतिग्रहीष्ये भिक्षां तु यदि कश्चित्प्रदास्यति}


\twolineshloka
{एवं दारक्रियाहेतोः प्रयतिष्ये पितामहाः}
{अनेन विधिना शश्वन्न करिष्येऽहमन्यथा}


\twolineshloka
{तत्र चोत्पत्स्यते जन्तुर्भवतां तारणाय वै}
{शाश्वतं स्थानमासाद्य मोदन्तां पितरो मम}


\chapter{अध्यायः १४}
\twolineshloka
{सौतिरुवाच}
{}


\twolineshloka
{ततो निवेशाय तदा स विप्रः संशितव्रतः}
{महीं चचार दारार्थी न च दारानविन्दत}


\twolineshloka
{स कदाचिद्वनं गत्वा विप्रः पितृवचः स्मरन्}
{चुक्रोश कन्याभिक्षार्थी तिस्रो वाचः शनैरिव}


\twolineshloka
{तं वासुकिः प्रत्यगृह्णादुद्यम्य भगिनीं तदा}
{न स तां प्रतिजग्राह न सनाम्नीति चिन्तयन्}


\twolineshloka
{सनाम्नीं चोद्यतां भार्यां गृह्णीयामिति तस्य हि}
{मनो निविष्टमभवज्जरत्कारोर्महात्मनः}


\threelineshloka
{तमुवाच महाप्राज्ञो जरत्कारुर्महातपाः}
{किंनाम्नी भगिनीयं ते ब्रूहि सत्यं भुजंगम ॥वासुकिरुवाच}
{}


\fourlineindentedshloka
{जरत्कारो जरत्कारुः स्वसेयमनुजा मम}
{प्रतिगृह्णीष्व भार्यार्थे मया दत्तां सुमध्यमाम्}
{त्वदर्थं रक्षिता पूर्वं प्रतीच्छेमां द्विजोत्तम ॥सौतिरुवाच}
{}


\twolineshloka
{एवमुक्त्वा ततः प्रादाद्भार्यार्थे वरवर्णिनीम्}
{स च तां प्रतिजग्राह विधिदृष्टेन कर्मणा}


\chapter{अध्यायः १५}
\twolineshloka
{सौतिरुवाच}
{}


\twolineshloka
{मात्रा हि भुजगाः शप्ताः पूर्वं ब्रह्मविदां वर}
{जनमेजयस्य वो यज्ञे धक्ष्यत्यनिलसारथिः}


\twolineshloka
{तस्य शापस्य शान्त्यर्थं प्रददौ पन्नगोत्तमः}
{स्वसारमृषये तस्मै सुव्रताय महात्मने}


\twolineshloka
{स च तां प्रतिजग्राह विधिदृष्टेन कर्मणा}
{आस्तीको नाम पुत्रश्च तस्यां जज्ञे महामनाः}


\twolineshloka
{तपस्वी च महात्मा च वेदवेदाङ्गपारगः}
{समः सर्वस्य लोकस्य पितृमातृभयापहः}


\twolineshloka
{अथ दीर्घस्य कालस्य पाण्डवेयो नराधिपः}
{आजहार महायज्ञं सर्पसत्रमिति श्रुतिः}


\twolineshloka
{तस्मिन्प्रवृत्ते सत्रे तु सर्पाणामन्तकाय वै}
{मोचयामास ताञ्शापादास्तीकः सुमहातपाः}


\twolineshloka
{भ्रातॄंश्च मातुलांश्चैव तथैवान्यान्स पन्नगान्}
{पितॄंश्च तारयामास संतत्या तपसा तथा}


\twolineshloka
{व्रतैश्च विविधैर्ब्रह्मन्स्वाध्यायैश्चानृणोऽभवत्}
{देवांश्च तर्पयामास यज्ञैर्विविधदक्षिणैः}


\twolineshloka
{ऋषींश्च ब्रह्मचर्येम सन्तत्या च पितामहान्}
{अपहृत्य गुरं भारं पितॄणां संशितव्रतः}


\twolineshloka
{जरत्कारुर्गतः स्वर्गं सहितः स्वैः पितामहैः}
{आस्तीकं च सुतं प्राप्य धर्मं चानुत्तमं मुनिः}


\threelineshloka
{जरत्कारुः सुमहता कालेन स्वर्गमेयिवान्}
{एतदाख्यानमास्तीकं यथावत्कथितं मया}
{प्रब्रूहि भृगुशार्दूल किमन्यत्कथयामि ते}


\chapter{अध्यायः १६}
\twolineshloka
{शौनक उवाच}
{}


\twolineshloka
{सौते त्वं कथयस्वेमां विस्तरेण कथां पुनः}
{आस्तीकस्य कवेःसाधोः शुश्रूषा परमा हिनः}


\twolineshloka
{मधुरं कथ्यते सौम्य श्लक्ष्णाक्षरपदं त्वया}
{प्रीयामहे भृशं तात पितेवेदं प्रभाषसे}


\threelineshloka
{अस्मच्छुश्रूषणे नित्यं पिता हि निरतस्तव}
{आचष्टैतद्यथाऽऽक्यानं पिता तेत्वं तथा वद ॥सौतिरुवाच}
{}


\twolineshloka
{आयुष्मन्निदमाख्यानमास्तीकं कथयामि ते}
{यथाश्रुतं कथयतः सकाशाद्वै पितुर्मया}


\twolineshloka
{पुरा देवयुगे ब्रह्मन्प्रजापतिसुते शुभे}
{आस्तां भगिन्यौ रूपेण समुपेतेऽद्भुतेऽनघ}


\twolineshloka
{ते भार्ये कश्यपस्यास्तां कद्रूश्च विनता च ह}
{प्रादात्ताभ्यां वरं प्रीतः प्रजापतिसमः पतिः}


\twolineshloka
{कश्यपो धर्मपत्नीभ्यां मुदा परमया युतः}
{वरातिसर्गं श्रुत्वैवं कश्यपादुत्तमं च ते}


\twolineshloka
{हर्षादप्रतिमां प्रीतिं प्रापतुः स्म वरस्त्रियौ}
{वव्रे कद्रूः सुतान्नागान्सहस्रं तुल्यवर्चसः}


\twolineshloka
{द्वौ पुत्रौ विनता वव्रे कद्रूपुत्राधिकौ बले}
{तेजसा वपुषा चैव विक्रमेणाधिकौ च तौ}


\twolineshloka
{तस्यै भर्ता वरं प्रादादीदृसौ ते भविष्यतः}
{एवमस्त्विति तं चाह कश्यपं विनता तदा}


\twolineshloka
{यथावत्प्रार्थितं लब्ध्वा वरं तुष्टाभवत्तदा}
{कृतकृत्या तु विनता लब्ध्वा वीर्याधिकौ सुतौ}


\twolineshloka
{कद्रूश्च लब्ध्वा पुत्राणां सहस्रं तुल्यवर्चसाम्}
{धार्यौ प्रयत्नतो गर्भावित्युक्त्वा स महातपाः}


\threelineshloka
{ते भार्ये वरसंतुष्टे कश्यपो वनमाविशत्}
{सौतुरिवाच}
{कालेन महता कद्रूरण्डानां दशतीर्दश}


\twolineshloka
{जनयामास विप्रेन्द्र द्वे चाण्डे विनता तदा}
{तयोरण्डानि निदधुः प्रहृष्टाः परिचारिकाः}


\twolineshloka
{सोपस्वेदेषु भाण्डेषु पञ्चवर्षशतानि च}
{ततः पञ्चशते काले कद्रूपुत्रा विनिःसृताः}


\twolineshloka
{अण्डाभ्यां विनतायास्तु मिथुनं न व्यदृश्यत}
{ततः पुत्रार्थिनी देवी व्रीडिता च तपस्विनी}


\twolineshloka
{अण्डं बिभेद विनता तत्र पुत्रमपश्यत}
{पूर्वार्धकायसंपन्नमितरेणाप्रकाशता}


\twolineshloka
{स पुत्रः क्रोधसंरब्धः शशापैनामिति श्रुतिः}
{योऽहमेवं कृतो मातस्त्वया लोभपरीतया}


\twolineshloka
{शरीरेणासमग्रेण तस्माद्दासी भविष्यसि}
{पञ्च वर्षशतान्यस्या यया विस्पर्धसे सह}


\twolineshloka
{एष च त्वां सुतो मातर्दासीत्वान्मोचयिष्यति}
{यद्येनमपि मातस्त्वं मामिवाण्डविभेदनात्}


\twolineshloka
{न करिष्यस्यनङ्गं वा व्यङ्गं वापि तपस्विनम्}
{प्रतिपालयितव्यस्ते जन्मकालोऽस्य धीरया}


\twolineshloka
{विशिष्टं बलमीप्सन्त्या पञ्चवर्षशतात्परः}
{एवं शप्त्वा ततः पुत्रो विनतामन्तरिक्षगः}


\twolineshloka
{अरुणोऽदृश्यत ब्रह्मन्प्रभातसमये तदा}
{`उद्यन्नथ सहस्रांशुर्दृष्ट्वा तमरुणं प्रभुः}


\twolineshloka
{स्वतेजसा प्रज्वलन्तमात्मनः समतेजसम्}
{सारथ्ये कल्पयामास प्रीयमाणस्तमोनुदः}


\twolineshloka
{सोऽपि तं रथमारुह्य भानोरमिततेजसः}
{सर्वलोकप्रदीपस्य ह्यमरोऽप्यरुणोऽभवत् ॥'}


\twolineshloka
{गरुडोऽपि यथाकालं जज्ञे पन्नगभोजनः}
{स जातमात्रो विनतां परित्यज्य खमाविशत्}


\twolineshloka
{आदास्यन्नात्मनो भोज्यमन्नं विहितमस्य यत्}
{विधात्रा भृगुशार्दूल क्षुधितः पतगेश्वरः}


\chapter{अध्यायः १७}
\twolineshloka
{सौतिरुवाच}
{}


\twolineshloka
{एतस्मिन्नेव काले तु भगिन्यौ ते तपोधन}
{अपश्यतां समायान्तमुच्चैः श्रवसमन्तिकात्}


\twolineshloka
{यं तु देवगणाः सर्वे हृष्टरूपमपूजयन्}
{मथ्यमानेऽमृते जातमश्वरत्नमनुत्तमम्}


\threelineshloka
{अमोघबलमश्वानामुत्तमं जविनां वरम्}
{श्रीमन्तमजरं दिव्यं सर्वलक्षणपूजितम् ॥शौनक उवाच}
{}


\threelineshloka
{कथं तदमृतं देवैर्मथितं क्व च शंस मे}
{`कारणं चात्र मथने संजातममृतात्परम् ॥'यत्र जज्ञे महावीर्यः सोऽश्वराजो महाद्युतिः ॥सौतिरुवाच}
{}


\twolineshloka
{ज्वलन्तमचलं मेरुं तेजोराशिमनुत्तमम्}
{आक्षिपन्तं प्रभां भानोः स्वशृङ्गैः काञ्चनोज्ज्वलैः}


\twolineshloka
{कनकाभरणं चित्रं देवगन्धर्वसेवितम्}
{अप्रमेयमनाधृष्यमधर्मबहुलैर्जनैः}


\twolineshloka
{व्यालैरावारितं घोरैर्दिव्यौषधिविदीपितम्}
{नाकमावृत्य तिष्ठन्तमुच्छ्रयेण महागिरिम्}


\twolineshloka
{अगम्यं मनसाप्यन्यैर्नदीवृक्षसमन्वितम्}
{नानापतगसङ्घैश्च नादितं सुमनोहरैः}


\twolineshloka
{तस्य शृङ्गमुपारुह्य बहुरत्नाचितं शुभम्}
{अनन्तकल्पमद्वन्द्वं सुराः सर्वे महौजसः}


\twolineshloka
{ते मन्त्रयितुमारब्धास्तत्रासीना दिवौकसः}
{अमृताय समागम्य तपोनियमसंयुताः}


\twolineshloka
{तत्र नारायणो देवो ब्रह्माणमिदमब्रवीत्}
{चिन्तयत्सु सुरेष्वेवं मन्त्रयत्सु च सर्वशः}


\twolineshloka
{देवैरसुरसङ्घैश्च मथ्यतां कलशोदधिः}
{भविष्यत्यमृतं तत्र मथ्यमाने महोदधौ}


\twolineshloka
{सर्वौषधीः समावाप्य सर्वरत्नानि चैव ह}
{मन्थध्वयुदधिं देवा वेत्स्यध्वममृतं ततः}


\chapter{अध्यायः १८}
\twolineshloka
{सौतिरुवाच}
{}


\twolineshloka
{ततोऽभ्रशिखराकारैर्गिरिशृङ्गैरलंकृतम्}
{मन्दरं पर्वतवरं लताजालसमाकुलम्}


\twolineshloka
{नानाविहंगसंघुष्टं नानादंष्ट्रिसमाकुलम्}
{किंनरैरप्सरोभिश्च देवैरपि च सेवितम्}


\twolineshloka
{एकादश सहस्राणि योजनानां समुच्छ्रितम्}
{अधोभूमेः सहस्रेषु तावत्स्वेव प्रतिष्ठितम्}


\twolineshloka
{तमुद्धर्तुमशक्ता वै सर्वे देवगणास्तदा}
{विष्णुमासीनमभ्येत्य ब्रह्माणं चेदमब्रुवन्}


\threelineshloka
{भवन्तावत्र कुरुतां बुद्धिं नैःश्रेयसीं पराम्}
{मन्दरोद्धरणे यत्नः क्रियतां च हिताय नः ॥सौतिरुवाच}
{}


\twolineshloka
{तथेति चाब्रवीद्विष्णुर्ब्रह्मणा सह भार्गव}
{अचोदयदमेयात्मा फणीन्द्रं पद्मलोचनः}


\twolineshloka
{ततोऽनन्तः समुत्थाय ब्रह्मणा परिचोदितः}
{नारायणेन चाप्युक्तस्तस्मिन्कर्मणि वीर्यवान्}


\twolineshloka
{अथ पर्वतराजानं तमनन्तो महाबलः}
{उज्जहार बलाद्ब्रह्मन्सवनं सवनौकसम्}


\twolineshloka
{ततस्तेन सुराः सार्धं समुद्रमुपतस्थिरे}
{तमूचुरमृतस्यार्थे निर्मथिष्यामहे जलम्}


\twolineshloka
{अपांपतिरथोवाच ममाप्यंशो भवेत्ततः}
{सोढाऽस्मि विपुलं मर्दं मन्दरभ्रमणादिति}


\twolineshloka
{ऊचुश्च कूर्मराजानमकूपारे सुरासुराः}
{अधिष्ठानं गिरेरस्य भवान्भवितुमर्हति}


\twolineshloka
{कूर्मेण तु तथेत्युक्त्वा पृष्ठमस्य समर्पितम्}
{तं शैलं तस्य पृष्ठस्थं वज्रेणेन्द्रोऽभ्यपीडयत्}


\threelineshloka
{मन्थानं मन्दरं कृत्वा तथा योक्त्रं च वासुकिम्}
{देवा मथितुमारब्धाः समुद्रं निधिमम्भसाम्}
{अमृतार्थे पुरा ब्रह्मस्तथैवासुरदानवाः}


% Check verse!
एकमन्तमुपाश्लिष्टा नागराज्ञो महासुराः
\twolineshloka
{विबुधाः सहिताः सर्वे यतः पुच्छं ततः स्थिताः}
{अनन्तो भगवान्देवो यतो नारायणः स्थितः}


\twolineshloka
{`वासुकेरग्रमाश्लिष्टा नागराज्ञो महासुराः}
{'शिर उत्क्षिप्य नागस्य पुनः पुनरवाक्षिपन्}


\twolineshloka
{वासुकेरथ नागस्य सहसा क्षिप्यतोऽसुरैः}
{सधूमाः सार्चिषो वाता निष्पेतुरसकृन्मुखात्}


\twolineshloka
{`वासुकेर्मथ्यमानस्य निःसृतेन विषेण च}
{अभवन्मिश्रितं तोयं तदा भार्गवनन्दन}


\twolineshloka
{मथनान्मन्दरेणाथ देवदानवबाहुभिः}
{विषं तीक्ष्णं समुद्भूतं हालाहलमिति श्रुतम्}


\twolineshloka
{देवाश्च दानवाश्चैव दग्धाश्चैव विषेण ह}
{अपाक्रामंस्ततो भीता विषादमगमंस्तदा}


\twolineshloka
{ब्रह्माणमब्रुवन्देवाः समेत्य मुनिपुंगवैः}
{मथ्यमानेऽमृते जातं विषं कालानलप्रभम्}


\twolineshloka
{तेनैव तापिता लोकास्तस्य प्रतिकुरुष्वह}
{एवमुक्तस्तदा ब्रह्मा दध्यौ लोकेश्वरं हरम्}


\twolineshloka
{त्र्यक्षं त्रिशूलिनं रुद्रे देवदेवमुमापतिम्}
{तदाऽथ चिन्तितो देवस्तज्ज्ञात्वा द्रुतमाययौ}


\twolineshloka
{तस्याथ देवस्तत्सर्वमाचचक्षे प्रजापतिः}
{तच्छ्रुत्वा दवेदेवेशो लोकस्यास्य हितेप्सया}


\twolineshloka
{अपिबत्तद्विषं रुद्रः कालानलसमप्रभम्}
{कण्ठे स्थापितवान्देवो लोकानां हितकाम्यया}


\twolineshloka
{यस्मात्तु नीलता कण्ठे नीलकण्ठस्ततः स्मृतः}
{पीतमात्रे विषे तत्र रुद्रेणामिततेजसा}


\twolineshloka
{देवाः प्रीताः पुनर्जग्मुश्चक्रुर्वै कर्म तत्तथा}
{मथ्यमानेऽमृतस्यार्थे भूयो वै देवदानवैः}


\twolineshloka
{वासुकेरथ नागस्य सहसा क्षिप्यतोऽसुरैः}
{सधूमाः सार्चिषो वाता निष्पेतुरसकृन्मुखात् ॥'}


\twolineshloka
{ते धूमसङ्घाः संभूता मेघसङ्घाः सविद्युतः}
{अभ्यवर्षन्सुरगणाञ्श्रमसंतापकर्शितान्}


\twolineshloka
{तस्माच्च गिरिकूटाग्रात्प्रच्युताः पुष्पवृष्टयः}
{सुरासुरगणान्सर्वान्समन्तात्समवाकिरन्}


\twolineshloka
{बभूवात्र महान्नादो महामेघरवोपमः}
{उदधेर्मथ्यमानस्य मन्दरेण सुरासुरैः}


\twolineshloka
{तत्र नानाजलचरा विनिष्पिष्टा महाद्रिणा}
{विलयं समुपाजग्मुः शतशो लवणाम्भसि}


\twolineshloka
{वारुणानि च भूतानि विविधानि महीधरः}
{पातालतलवासीनि विलयं समुपानयत्}


\twolineshloka
{तस्मिंश्च भ्राम्यमाणेऽद्रौ संघृष्यन्तः परस्परम्}
{न्यपतन्पतगोपेताः पर्वताग्रान्महाद्रुमाः}


\twolineshloka
{तेषां संघर्षजश्चाग्निरर्चिर्भिः प्रज्वलन्मुहुः}
{विद्युद्भिरिव नीलाभ्रमावृणोन्मन्दरं गिरिम्}


\twolineshloka
{ददाह कुञ्जरांस्तत्र सिंहांश्चैव विनिर्गतान्}
{विगतासूनि सर्वाणि सत्त्वानि विविधानि च}


\twolineshloka
{तमग्निममरश्रेष्ठः प्रदहन्तमितस्ततः}
{वारिणा मेघजेनेन्द्रः शमयामास सर्वशः}


\twolineshloka
{ततो नानाविधास्तत्र सुस्रुवुः सागराम्भसि}
{महाद्रुमाणां निर्यासा बहवश्चौषधीरसाः}


\twolineshloka
{तेषाममृतवीर्याणां रसानां पयसैव च}
{अमरत्वं सुरा जग्मुः काञ्चनस्य च निःस्रवात्}


\twolineshloka
{ततस्तस्य समुद्रस्य तञ्जातमुदकं पयः}
{रसोत्तमैर्विमिश्रं च ततः क्षीरादभूद्धृतम्}


\twolineshloka
{ततो ब्रह्माणमासीनं देवा वरदमब्रुवन्}
{श्रान्ताः स्म सुभृशं ब्रह्मन्नोद्भवत्यमृतं च तत्}


\twolineshloka
{ऋते नारायणं देवं सर्वेऽन्ये देवदानवाः}
{चिरारब्धमिदं चापि सागरस्यापि मन्थनम्}


\threelineshloka
{`ग्लानिरस्मान्समाविष्टा न चात्रामृतमत्थितम्}
{सौतिरुवाच}
{देवानां वचनं श्रुत्वा ब्रह्मा लोकपितामहः'}


\threelineshloka
{ततो नारायणं देवं वचनं चेदमब्रवीत्}
{विधत्स्वैषां बलं विष्णो भवानत्र परायणम् ॥विष्णुरुवाच}
{}


\threelineshloka
{बलं ददामि सर्वेषां कर्मैतद्ये समास्थिताः}
{क्षोभ्यतां कलशः सर्वैमन्दरः परिवर्त्यताम् ॥सौतिरुवाच}
{}


\twolineshloka
{नारायणवचः श्रुत्वा बलिनस्ते महोदधेः}
{तत्पयः सहिता भूयश्चक्रिरे भृशमाकुलम्}


\threelineshloka
{`तत्र पूर्वं विषं जातं तद्ब्रह्मवचनाच्छिवः}
{प्राग्रसल्लोकरक्षार्थं ततो ज्येष्ठा समुत्थिता}
{कृष्णरूपधरा देवी सर्वाभरणभूषिता ॥'}


\twolineshloka
{ततः शतसहस्रांशुर्मथ्यमानात्तु सागरात्}
{प्रसन्नात्मा समुत्पन्नः सोमः शीतांशुरुज्ज्वलः}


\twolineshloka
{श्रीरनन्तरमुत्पन्ना घृतात्पाण्डुरवासिनी}
{सुरा देवी समुत्पन्ना तुरगः पाण्डुरस्तथा}


\twolineshloka
{कौस्तुभस्तु मणिर्दिव्य उत्पन्नो घृतसंभवः}
{मरीचिविकचः श्रीमान्नारायणउरोगतः}


\threelineshloka
{श्रीः सुरा चैव सोमश्च तुरगश्च मनोजवः}
{`पारिजातश्च तत्रैव सुरभिश्च महामुने}
{जज्ञाते तौ तदा ब्रह्मन्सर्वकामफलप्रदौ}


\threelineshloka
{ततो जज्ञे महाकायश्चतुर्दन्तो महागजः}
{कपिला कामवृक्षश्च कौस्तुभश्चाप्सरोगणः}
{'यतो देवास्ततो जग्मुरादित्यपथमाश्रिताः}


\twolineshloka
{धन्वन्तरिस्ततो देवो वपुष्मानुदतिष्ठत}
{श्वेतं कमण्डलुं बिभ्रदमृतं यत्र तिष्ठति}


\twolineshloka
{एतदत्यद्भुतं दृष्ट्वा दानवानां समुत्थितः}
{अमृतार्थे महान्नादो ममेदमिति जल्पताम्}


\twolineshloka
{ततो नारायणो मायां मोहिनीं समुपाश्रितः}
{स्त्रीरूपमद्भुतं कृत्वा दानवानभिसंश्रितः}


\twolineshloka
{ततस्तदमृतं तस्यै ददुस्ते मूढचेतसः}
{स्त्रियै दानवदैतेयाः सर्वे तद्गतमानसाः}


\twolineshloka
{`सा तु नारायणी माया धारयन्ती कमण्डलुम्}
{आस्यमानेषु दैत्येषु पङ्क्त्या च प्रति दानवैः}


% Check verse!
देवानपाययद्देवी न दैत्यांस्ते च चुक्रुशुः
\chapter{अध्यायः १९}
\twolineshloka
{सौतिरुवाच}
{}


\twolineshloka
{अथावरणमुख्यानि नानाप्रहरणानि च}
{प्रगृह्याभ्यद्रवन्देवान्सहिता दैत्यदानवाः}


\twolineshloka
{ततस्तदमृतं देवो विष्णुरादाय वीर्यवान्}
{जहार दानवेन्द्रेभ्यो नरेण सहितः प्रभुः}


\twolineshloka
{ततो देवगणाः सर्वे पपुस्तदमृतं तदा}
{विष्णोः सकाशात्संप्राप्य संभ्रमे तुमुले सति}


\threelineshloka
{`पाययत्यमृतं देवान्हरौ बाहुबलान्नरः}
{निरोधयति चापेन दूरीकृत्य धनुर्धरान्}
{ये येऽमृतं पिबन्ति स्म ते ते युद्ध्यन्ति दानवैः;'}


\twolineshloka
{ततः पिबत्सु तत्कालं देवेष्वमृतमीप्सितम्}
{राहुर्विबुधरूपेण दानवः प्रापिबत्तदा}


\twolineshloka
{तस्य कण्ठमनुप्राप्ते दानवस्यामृते तदा}
{आख्यातं चन्द्रसूर्याभ्यां सुराणां हितकाम्यया}


\twolineshloka
{ततो भगवता तस्य शिरश्छिन्नमलंकृतम्}
{चक्रायुधेन चक्रेण पिबतोऽमृतमोजसा}


\twolineshloka
{तच्छैलशृह्गप्रङ्गिमं दानवस्य शिरो महत्}
{`चक्रेणोत्कृत्तमपतच्चालयद्वसुधातलम् ॥'}


\twolineshloka
{चक्रच्छिन्नं खमुत्पत्य ननादातिभयंकरम्}
{तत्कबन्धं पपातास्य विस्फुरद्धरणीतले}


\twolineshloka
{`त्रयोदश सहस्राणि चतुरश्रं समन्ततः}
{सपर्वतवनद्वीपां दैत्यस्याकम्पयन्महीम्}


\twolineshloka
{ततो वैरविनिर्बन्धः कृतो राहुमुखेन वै}
{शाश्वतश्चन्द्रसूर्याभ्यां ग्रसत्यद्यापि चैव तौ}


\twolineshloka
{विहाय भगवांश्चापि स्त्रीरूपमतुलं हरिः}
{नानाप्रहरणैर्भीमैर्दानवान्तमकम्पयत्}


\twolineshloka
{ततः प्रवृत्तः संग्रामः समीपे लवणाम्भसः}
{सुराणामसुराणां च सर्वघोरतरो महान्}


\twolineshloka
{प्रासाश्च विपुलास्तीक्ष्णा न्यपतन्त सहस्रशः}
{तोमराश्च सुतीक्ष्णाग्राः शस्त्राणि विविधानि च}


\twolineshloka
{ततोऽसुराश्चक्रभिन्ना वमन्तो रुधिरं बहु}
{असिशक्तिगदारुग्णा निपेतुर्धरणीतले}


\twolineshloka
{छिन्नानि पट्टिशैश्चैव शिरांसि युधि दारुणैः}
{तप्तकाञ्चनमालीनि निपेतुरनिशं तदा}


\twolineshloka
{रुधिरेणानुलिप्ताङ्गा निहताश्च महासुराः}
{अद्रीणामिव कूटानि धातुरक्तानि शेरते}


\twolineshloka
{आहाकारः समभवत्तत्र तत्र सहस्रशः}
{अन्योन्यंछिन्दतां शस्त्रैरादित्ये लोहितायति}


\twolineshloka
{परिघैरायसैस्तीक्ष्णैः सन्निकर्षे च मुष्टिभिः}
{निघ्नतां समरेऽन्योन्यं शब्दो दिवमिवास्पृशत्}


\twolineshloka
{छिन्धिभिन्धि प्रधाव त्वं पातयाभिसरेति च}
{व्यश्रूयन्त महाघोराः शब्दास्तत्र समन्ततः}


\twolineshloka
{एवं सुतुमुले युद्धे वर्तमाने महाभये}
{नरनारायणौ देवौ समाजग्मतुराहवम्}


\twolineshloka
{तत्र दिव्यं धनुर्दृष्ट्वा नरस्य भगवानपि}
{चिन्तयामास तच्चक्रं विष्णुर्दानवसूदनम्}


\twolineshloka
{ततोऽम्बराच्चिन्तितमात्रमागतंमहाप्रभं चक्रममित्रतापनम्}
{विभावसोस्तुल्यमकुण्ठमण्डलंसुदर्शनं संयति भीमदर्शनम्}


\twolineshloka
{तदागतं ज्वलितहुताशनप्रभंभयंकरं करिकरबाहुरच्युतः}
{मुमोच वै प्रबलवदुग्रवेगवा-न्महाप्रभं परनगरावदारणम्}


\twolineshloka
{तदन्तकज्वलनसमानवर्चसंपुनःपुनर्न्यपतत वेगवत्तदा}
{विदारयद्दितिदनुजान्सहस्रशःकरेरितं पुरुषवरेण संयुगे}


\twolineshloka
{दहत्क्वचिज्ज्वलन इवावलेलिह-त्प्रसह्य तानसुरगणान्न्यकृन्तत}
{प्रवेरितं वियति मुहुः क्षितौ तथापपौ रणे रुधिरमथो पिशाचवत्}


\twolineshloka
{तथाऽसुरा गिरिभिरदीनचेतसोमुहुर्मुहुः सुरगणमार्दयंस्तदा}
{महाबला विगलितमेघवर्चसःसहस्रशो गगनमभिप्रपद्यह}


\twolineshloka
{अथाम्बराद्भयजननाः प्रपेदिरेसपादपा बहुविधमेघरूपिणः}
{महाद्रयः परिगलिताग्रसानवःपरस्परं द्रुतमभिहत्य सस्वनाः}


\twolineshloka
{ततो मही प्रविचलिता सकाननामहाद्रिपाताभिहता समन्ततः}
{परस्परं भृशमभिगर्जतां मुहूरणाजिरे भृशमभिसंप्रवर्तिते}


\twolineshloka
{नरस्ततो वरकनकाग्रभूषणै-र्महेषुभिर्गगनपथं समावृणोत्}
{विदारयन्गिरिशिखराणि पत्रिभि-र्महाभयेऽसुरगणविग्रहे तदा}


\twolineshloka
{ततो महीं लवणजलं च सागरंमहासुराः प्रविविशुरर्दिताः सुरैः}
{वियद्गतं ज्वलितहुताशनप्रभंसुदर्शनं परिकुपितं निशाम्य ते}


\twolineshloka
{ततः सुरैर्विजयमवाप्य मन्दरःस्वमेव देशं गमितः सुपूजितः}
{विनाद्य खं दिवमपि चैव सर्वश-स्ततो गताः सलिलधरा यथागतम्}


\twolineshloka
{ततोऽमृतं सुनिहितमेव चक्रिरेसुराः पुरां मुदमभिगम्य पुष्कलाम्}
{ददो च तं निधिममृतस्य रक्षितुंकिरीटिने बलभिदथामरैः सह}


\chapter{अध्यायः २०}
\twolineshloka
{सौतिरुवाच}
{}


\twolineshloka
{एतत्ते कथितं सर्वममृतं मथितं यथा}
{यत्र सोऽश्वः समुत्पन्नः श्रीमानतुलविक्रमः}


\threelineshloka
{तं निशाम्य तदा कद्रूर्विनतामिदमब्रवीत्}
{उच्चैःश्रवा हि किंवर्णो भद्रे प्रब्रूहि मा चिरम् ॥विनतोवाच}
{}


\threelineshloka
{श्वेत एवाश्वराजोऽयं किं वा त्वं मन्यसे शुभे}
{ब्रूहि वर्णं त्वमप्यस्य ततोऽत्र विपणावहे ॥कद्रूरुवाच}
{}


\threelineshloka
{कृष्णवालमहं मन्ये हयमेनं शुचिस्मिते}
{एहि सार्धं मया दीव्य दासीभावाय भामिनि ॥सौतिरुवाच}
{}


\twolineshloka
{एवं ते समयं कृत्वा दासीभावाय वै मिथः}
{जग्मतुः स्वगृहानेव श्वो द्रक्ष्याव इति स्म ह}


\twolineshloka
{ततः पुत्रसहस्रं तु कद्रूर्जिह्यं चिकीर्षती}
{आज्ञापयामास तदा वाला भूत्वाऽञ्जनप्रभाः}


\twolineshloka
{आविशध्वं हयं क्षिप्रं दासी न स्यामहं यथा}
{नावपद्यन्त ये वाक्यं ताञ्शशाप भुजंगमान्}


\twolineshloka
{सर्पसत्रे वर्तमाने पावको वः प्रधक्ष्यति}
{जनमेजयस्य राजर्षेः पाण्डवेयस्य धीमतः}


\twolineshloka
{शापमेनं तु शुश्राव स्वयमेव पितामहः}
{अतिक्रूरं समुत्सृष्टं कद्र्वा दैवादतीव हि}


\twolineshloka
{सार्धं देवगणैः सर्वैर्वाचं तामन्वमोदत}
{बहुत्वं प्रेक्ष्य सर्पाणां प्रजानां हितकाम्यया}


\twolineshloka
{तिग्मवीर्यविषा ह्येते दन्दशूका महाबलाः}
{तेषां तीक्ष्णविषत्वाद्धि प्रजानां च हिताय च}


\twolineshloka
{युक्तं मात्रा कृतं तेषां परपीडोपसर्पिणाम्}
{अन्येषामपि सत्त्वानां नित्यं दोषपरास्तु ये}


\twolineshloka
{तेषां प्राणान्तिको दण्डो दैवेन विनिपात्यते}
{एवं संभाष्य देवस्तु पूज्य कद्रूं च तां तदा}


\twolineshloka
{आहूय कश्यपं देव इदं वचनमब्रवीत्}
{यदेते दन्दशूकाश्च सर्पा जातास्त्वयानघ}


\twolineshloka
{विषोल्बणा महाभोगा मात्रा शप्ताः परंतप}
{तत्र मन्युस्त्वया तात न कर्तव्यः कथंचन}


\threelineshloka
{दृष्टं पुरातनं ह्येतद्यज्ञे सर्पविनाशनम्}
{इत्युक्त्वा सृष्टिकृद्देवस्तं प्रसाद्य प्रजापतिम्}
{प्रादाद्विषहरीं विद्यां कश्यपाय महात्मने}


\twolineshloka
{`एवं शप्तेषु नागेषु कद्र्वातु द्विजसत्तम}
{अद्विग्नः शापतस्तस्याः कद्रूं कर्कोटकोऽब्रवीत्}


\twolineshloka
{मातरं परमप्रीतस्तथा भुजगसत्तमः}
{आविश्य वाजिनं मुख्यं बालो भूत्वाञ्जनप्रभः}


\twolineshloka
{दर्शयिष्यामि तत्राहमात्मानं काममाश्वस}
{एवमस्त्विति सा पुत्रं प्रत्युवाच यशस्विनी'}


\chapter{अध्यायः २१}
\twolineshloka
{सौतिरुवाच}
{}


\twolineshloka
{ततो रजन्यां व्युष्टायां प्रभातेऽभ्युदिते रवौ}
{कद्रूश्च विनता चैव भगिन्यौ ते तपोधन}


\threelineshloka
{अमर्षिते सुसंरब्धे दास्ये कृतपणे तदा}
{`स्मगरस्य परं पारं वेलावनविभूषितम्}
{'जग्मतुस्तुरगं द्रष्टुमुच्चैःश्रवसमन्तिकात्}


\twolineshloka
{ददृशातेऽथ ते तत्र समुद्रं निधिमम्भसाम्}
{महान्तमुदकागाधं क्षोभ्यमाणं महास्वनम्}


\twolineshloka
{तिमिंगिलझषाकीर्णं मकरैरावृतं तथा}
{सत्वैश्च बहुसाहस्रैर्नानारूपैः समावृतम्}


\twolineshloka
{भीषणैर्विकृतैरन्यैर्घोरैर्जलचरैस्तथा}
{उग्रैर्नित्यमनाधृष्यं कूर्मग्राहसमाकुलम्}


\twolineshloka
{आकरं सर्वरत्नानामालयं वरुणस्य च}
{नागानामालयं रम्यमुत्तमं सरितां पतिम्}


\twolineshloka
{पातालज्वलनावासमसुराणां च बान्धवम्}
{भयंकरं च सत्त्वानां पयसां निधिमर्णवम्}


\twolineshloka
{शुभं दिव्यममर्त्यानाममृतस्याकरं परम्}
{अप्रमेयमचिन्त्यं च सुपुण्यजलमद्भुतम्}


\twolineshloka
{घोरं जलचरारावरौद्रं भैरवनिःस्वनम्}
{गम्भीरावर्तकलिलं सर्वभूतभयंकरम्}


\twolineshloka
{वेलादोलानिलचलं क्षोभोद्वेगसमुच्छ्रितम्}
{वीचीहस्तैः प्रचलितैर्नृत्यन्तमिव सर्वतः}


\twolineshloka
{चन्द्रवृद्धिक्षयवशादुद्वृत्तोर्मिसमाकुलम्}
{पाञ्चजन्यस्य जननं रत्नाकरमनुत्तमम्}


\twolineshloka
{गां विन्दता भगवता गोविन्देनामितौजसा}
{वराहरूपिणा चान्तर्विक्षोभितजलाविलम्}


\twolineshloka
{ब्रह्मर्षिणा व्रतवता वर्षाणां शतमत्रिणा}
{अनासादितगाधं च पातालतलमव्ययम्}


\twolineshloka
{अध्यात्मयोगनिद्रां च पद्मनाभस्य सेवतः}
{युगादिकालशयनं विष्णोरमिततेजसः}


\twolineshloka
{वज्रपातनसंत्रस्तमैनाकस्याभयप्रदम्}
{डिम्बाहवार्दितानां च असुराणां परायणम्}


\twolineshloka
{बडवामुखदीप्ताग्नेस्तोयहव्यप्रदं शिवम्}
{अगाधपारं विस्तीर्णमप्रमेयं सरित्पतिम्}


\threelineshloka
{महानदीभिर्बह्वीभिः स्पर्धयेव सहस्रशः}
{अभिसार्यमाणमनिशं ददृशाते महार्णवम्}
{आपूर्यमाणमत्यर्थं नृत्यमानमिवोर्मिभिः}


\twolineshloka
{गम्भीरं तिमिमकरोग्रसंकुलं तंगर्जन्तं जलचररावरौद्रनादैः}
{विस्तीर्णं ददृशतुरम्बरप्रकाशंतेऽगाधं निधिमुरुमम्भसामनन्तम्}


\chapter{अध्यायः २२}
\twolineshloka
{सौतिरुवाच}
{}


\twolineshloka
{नागाश्च संविदं कृत्वा कर्तव्यमिति तद्वचः}
{निःस्नेहा वै दहेन्माता असंप्राप्तमनोरथा}


\twolineshloka
{प्रसन्ना मोक्षयेदस्मांस्तस्माच्छापाच्च भामिनी}
{कृष्णं पुच्छं करिष्यामस्तुरगस्य न संशयः}


\twolineshloka
{`इति निश्चित्य ते तस्य कृष्णा वाला इव स्थिताः}
{'एतस्मिन्नन्तरे ते तु सपत्न्यौ पणिते तदा}


\twolineshloka
{ततस्ते पणितं कृत्वा भगिन्यौ द्विजसत्तम}
{जग्मतुः परया प्रीत्या परं पारं महोदधेः}


\twolineshloka
{कद्रूश्च विनता चैव दाक्षायण्यौ विहायसा}
{आलोकयन्त्यावक्षोभ्यं समुद्रं निधिमम्भसाम्}


\twolineshloka
{वायुनाऽतीव सहसा क्षोभ्यमाणं महास्वनम्}
{तिमिंगिलसमाकीर्णं मकरैरावृतं तथा}


\twolineshloka
{संयुतं बहुसाहस्रैः सत्वैर्नानाविधैरपि}
{घोरर्घोरमनाधृष्यं गम्भीरमतिभैरवम्}


\twolineshloka
{आकरं सर्वरत्नानामालयं वरुणस्य च}
{नागानामालयं चापि सुरम्यं सरितां पतिम्}


\twolineshloka
{पातालज्वलनावासमसुराणां तथालयम्}
{भयंकराणां सत्त्वानां पयसो निधिमव्ययम्}


\twolineshloka
{शुभ्रं दिव्यममर्त्यानाममृतस्याकरं परम्}
{अप्रमेयमचिन्त्यं च सुपुम्यजलसंमितम्}


\twolineshloka
{महानदीभिर्बह्वीभिस्तत्र तत्र सहस्रशः}
{आपूर्यमाणमत्यर्थं नृत्यन्तमिव चोर्मिभिः}


\twolineshloka
{इत्येवं तरलतरोर्मिसंकुलं तेगम्भीरं विकसितमम्बरप्रकाशम्}
{पातालज्वलनशिखाविदीपिताङ्गंगर्जन्तं द्रुतमभिजग्मतुस्ततस्ते}


\chapter{अध्यायः २३}
\twolineshloka
{सौतिरुवाच}
{}


\twolineshloka
{तं समुद्रमतिक्रम्य कद्रूर्विनतया सह}
{न्यपतत्तुरगाभ्याशे न चिरादिव शीघ्रगा}


\twolineshloka
{ततस्ते तं हयश्रेष्ठं ददृशाते महाजवम्}
{शशाङ्ककिरणप्रख्यं कालवालमुभे तदा}


\twolineshloka
{निशाम्य च बहून्वालान्कृष्णान्पुच्छसमाश्रितान्}
{विषण्णरूपां विनतां कद्रूर्दास्ये न्ययोजयत्}


\twolineshloka
{ततः सा विनता तस्मिन्पणितेन पराजिता}
{अभवद्दुःखसंतप्ता दासीभावं समास्थिता}


\twolineshloka
{एतस्मिन्नन्तरे चापि गरुडः काल आगते}
{विना मात्रा महातेजा विदार्याण्डमजायत}


\twolineshloka
{महासत्त्वबलोपेतः सर्वा विद्योतयन्दिशः}
{कामरूपः कामगमः कामवीर्यो विहंगमः}


\twolineshloka
{अग्निराशिरिवोद्भासन्समिद्धोऽतिभयंकरः}
{विद्युद्विस्पष्टपिङ्गाक्षो युगान्ताग्निसमप्रभः}


\twolineshloka
{प्रवृद्धः सहसा पक्षी महाकायो नभोगतः}
{घोरो घोरस्वनो रौद्रो वह्निरौर्व इवापरः}


\twolineshloka
{तं दृष्ट्वा शरणं जग्मुर्देवाः सर्वे विभावसुम्}
{प्रमिपत्याब्रुवंश्चैनमासीनं विश्वरूपिणम्}


\threelineshloka
{अग्ने मा त्वं प्रवर्धिष्ठाः कच्चिन्नो न दिधक्षसि}
{असौ हि राशिः सुमहान्समिद्धस्तव सर्पति ॥अग्निरुवाच}
{}


\twolineshloka
{नैतदेवं यथा यूयं मन्यध्वमसुरार्दनाः}
{गरुडो बलवानेष मम तुल्यश्च तेजसा}


\twolineshloka
{जातः परमतेजस्वी विनतानन्दवर्धनः}
{तेजोराशिमिमं दृष्ट्वा युष्मान्मोहः समाविशत्}


\twolineshloka
{नागक्षयकरश्चै काश्यपेयो महाबलः}
{देवानां च हिते युक्तस्त्वहितो दैत्यरक्षसाम्}


\threelineshloka
{न भीः कार्या कथं चात्र पश्यध्वं सहिता मया}
{सौतिरुवाच}
{एवमुक्तास्तदा गत्वा गरुडं वाग्भिरस्तुवन्}


\threelineshloka
{ते दूरादभ्युपेत्यैनं देवाः सर्षिगणास्तदा}
{देवा ऊचुः}
{त्वमृषिस्त्वं महाभागस्त्वं देवः पतगेश्वरः}


\twolineshloka
{त्वं प्रभुस्तपनः सूर्यः परमेष्ठी प्रजापतिः}
{त्वमिन्द्रस्त्वं हयमुखस्त्वं शर्वस्त्वं जगत्पतिः}


\twolineshloka
{त्वं मुखं पद्मजी विप्रस्त्वमग्निः पवनस्तथा}
{त्वं हि धाता विधाता च त्वं विष्णुः सुरसत्तमः}


\twolineshloka
{त्वं महानभिभूः शश्वदमृतं त्वं महद्यशः}
{त्वं प्रभास्त्वमभिप्रेतं त्वं नस्त्राणमनुत्तमम्}


\threelineshloka
{`त्वं गतिः सततं त्वत्तः कथं नः प्राप्नुयाद्भयम्}
{'बलोर्मिमान्साधुरदीनसत्त्वःसमृद्धिमान्दुर्विषहस्त्वमेव}
{त्वत्तः सृतं सर्वमहीनकीर्तेह्यनागतं चोपगतं च सर्वम्}


\twolineshloka
{त्वमुत्तमः सर्वमिदं चराचरंगभस्तिभिर्भानुरिवावभाससे}
{समाक्षिपन्भानुमतः प्रभां मुहु-स्त्वमन्तकः सर्वमिदं ध्रुवाध्रुवम्}


\twolineshloka
{दिवाकरः परिकुपितो यथा दहे-त्प्रजास्तथा दहसि हुताशनप्रभ}
{भयंकरः प्रलय इवाग्निरुत्थितोविनाशयन्युगपरिवर्तनान्तकृत्}


\twolineshloka
{खगेश्वरं शरणमुपागता वयंमहौजसं ज्वलनसमानवर्चसम्}
{तडित्प्रभं वितिमिरमभ्रगोचरंमहाबलं गरुडमुपेत्य खेचरम्}


\twolineshloka
{परावरं वरदमजय्यविक्रमंतवौजस सर्वमिदं प्रतापितम्}
{जगत्प्रभो तप्तसुवर्णवर्चसात्वं पाहि सर्वांश्च सुरान्महात्मनः}


\twolineshloka
{भयान्विता नभसि विमानगामिनोविमानिता विपथगतिं प्रयान्ति ते}
{ऋषेः सुतस्त्वमसि दयावतः प्रभोमहात्मनः खगवर कश्यपस्य ह}


\twolineshloka
{स मा क्रुधः कुरु जगतो दयां परांत्वमीश्वरः प्रशममुपैहि पाहि नः}
{महाशनिस्फुरितसमस्वनेन तेदिशोऽम्बरं त्रिदिवमियं च मेदिनी}


\fourlineindentedshloka
{चलन्ति नः खग हृदयानि चानिशंनिगृह्यतां वपुरिदमग्निसन्निभम्}
{तव द्युतिं कुपितकृतान्तसन्निभांनिशाम्य नश्चलति मनोऽव्यवस्थितम्}
{प्रसीद नः पतगते प्रयाचतांशिवश्च नो भव भगवन्सुखावहः ॥सौतिरुवाच}
{}


\twolineshloka
{एवं स्तुतः सुपर्णस्तु देवैः सर्षिगणैस्तदा}
{तेजसः प्रतिसंहारमात्मनः स चकार ह}


\chapter{अध्यायः २४}
\twolineshloka
{सौतिरुवाच}
{}


\threelineshloka
{स श्रुत्वाऽथात्मनो देहं सुपर्णः प्रेक्ष्य च स्वयम्}
{शरीरप्रतिसंहारमात्मनः संप्रचक्रमे ॥सुपर्ण उवाच}
{}


\threelineshloka
{न मे सर्वाणि भूतानि विभियुर्देहदर्शनात्}
{भीमरूपात्समुद्विग्रास्तस्मात्तेजस्तु संहरे ॥सौतिरुवाच}
{}


\twolineshloka
{ततः कामगमः पक्षी कामवीर्यो विहंगमः}
{अरुणं चात्मनः पृष्ठमारोप्य स पितुर्गृहात्}


\twolineshloka
{मातुरन्तिकमागच्छत्परं तीरं महोदधेः}
{तत्रारुणश्च निक्षिप्तो पुरोदेशे महाद्युतेः}


\threelineshloka
{सूर्यस्तेजोभिरत्युग्रैर्लोकान्दग्धुमना यदा}
{रुरुरुवाच}
{किमर्थं भगवान्सूर्यो लोकान्दग्धुमनास्तदा}


\threelineshloka
{किमस्तापहृतं देवैर्येनमं मन्युराविशत्}
{प्रमतिरुवाच}
{चन्द्रार्काभ्यां यदा राहुराख्यातो ह्यमृतं पिबन्}


\twolineshloka
{वैरानुबन्धं कृतवांश्चन्द्रादित्यौ तदाऽनघ}
{बाध्यमानं ग्रहेणाथ ह्यादित्यं मन्युराविशत्}


\twolineshloka
{सुरार्थाय समुत्पन्नो रोषो राहोस्तु मां प्रति}
{बह्वनर्थकरं पापमेकोऽहं समवाप्नुयाम्}


\twolineshloka
{सहाय एव कार्येषु न च कृच्छ्रेषु दृश्यते}
{पश्यन्ति ग्रस्यमानं मां सहन्ते वै दिवौकसः}


\twolineshloka
{तस्माल्लोकविनाशार्थं ह्यवतिष्ठे न संशयः}
{एवं कृतमतिः सूर्यो ह्यस्तमभ्यगमद्गिरिम्}


\twolineshloka
{तस्माल्लोकविनाशाय संतापयत भास्करः}
{ततो देवानुपागम्य प्रोचुरेवं महर्षयः}


\twolineshloka
{आद्यार्धरात्रसमये सर्वलोकभयावहः}
{उत्पत्स्यते महान्दाहस्त्रैलोक्यस्य विनाशनः}


\twolineshloka
{ततो देवाः सर्षिगणा उपगम्य पितामहम्}
{अब्रुवन्किमिवेहाद्य महद्दाहकृतं भयम्}


\threelineshloka
{न तावद्दृश्यते सूर्यः क्षयोऽयं प्रतिभाति च}
{उदिते भगवन्भानौ कथनेतद्भविष्यति ॥पितामह उवाच}
{}


\twolineshloka
{एष लोकविनाशाय रविरुद्यन्तुमुद्यतः}
{दृश्यन्नेव हि लोकान्स भस्मराशीकरिष्यति}


\twolineshloka
{तस्य प्रतिविधानं च विहितं पूर्वमेव हि}
{कश्यपस्य सुतो धीमानरुणेत्यभिविश्रुतः}


\twolineshloka
{महाकायो महातेजाः स स्थास्यति पुरो रवेः}
{करिष्यति च सारथ्यं तेजश्चास्य हरिष्यति}


\threelineshloka
{लोकानां स्वस्ति चैवं स्यादृषीणां च दिवौकसाम्}
{प्रमतिरुवाच}
{ततः पितामहाज्ञातः सर्वं चक्रे तदाऽरुणः}


\twolineshloka
{उदितश्चैव सविता ह्यरुणेन समावृतः}
{एतत्ते सर्वमाख्यातं यत्सूर्यं मन्युराविशत्}


\twolineshloka
{अरुणश्च यथैवास्य सारथ्यमकरोत्प्रभुः}
{भूय एवापरं प्रश्नं शृणु पूर्वमुदाहृतम्}


\chapter{अध्यायः २५}
\twolineshloka
{सौतिरुवाच}
{}


\twolineshloka
{ततः कामगमः पक्षी महावीर्यो महाबलः}
{मातुरन्तिकमागच्छत्परं पारं महोदधेः}


\twolineshloka
{यत्र सा विनता तस्मिन्पणितेन पराजिता}
{अतीव दुःखसंतप्ता दासीभावमुपागता}


\twolineshloka
{ततः कदाचिद्विनतां प्रणतां पुत्रसन्निधौ}
{काले चाहूय वचनं कद्रूरिदमभाषत}


\threelineshloka
{नागानामालयं भद्रे सुरम्यं चारुदर्शनम्}
{समुद्रकुक्षावेकान्ते तत्र मां विनते नय ॥सौतिरुवाच}
{}


\twolineshloka
{ततः सुपर्णमाता तामवहत्सर्पमातरम्}
{पन्नगान्गरुडश्चापि मातुर्वचनचोदितः}


\twolineshloka
{स सूर्यमभितो याति वैनतेयो विहंगमः}
{सूर्यरश्मिप्रतप्ताश्च मूर्च्छिताः पन्नगाऽभवन्}


\twolineshloka
{तदवस्थान्सुतान्दृष्ट्वा कद्रूः शक्रमथास्तुवत्}
{नमस्ते सर्वदेवेश नमस्ते बलसूदन}


\twolineshloka
{नमुचिघ्न नमस्तेऽस्तु सहस्राक्ष शचीपते}
{सर्पाणां सूर्यतप्तानां वारिणा त्वं प्लवो भव}


\twolineshloka
{त्वमेव परमं त्राणमस्माकममरोत्तम}
{ईशो असि पवः स्रष्टुं त्वमनल्पं पुरंदर}


\twolineshloka
{त्वमेव मेघस्त्वं वायुस्त्वमग्निर्विद्युतोऽम्बरे}
{त्वमभ्रगणविक्षेप्ता त्वामेवाहुर्महाघनम्}


\twolineshloka
{त्वं वज्रमतुलं घोरं घोषवांस्त्वं बलाहकः}
{स्रष्टा त्वमेव लोकानां संहर्ता चापराजितः}


\twolineshloka
{त्वं ज्योतिः सर्वभूतानां त्वमादित्यो विभावसुः}
{त्वं महद्भूतमाश्चर्यं त्वं राजा त्वं सुरोत्तमः}


\twolineshloka
{त्वं विष्णुस्त्वं सहस्राक्षस्त्वं देवस्त्वं परायणम्}
{त्वं सर्वममृतं देव त्वं सोमः परमार्चितः}


\threelineshloka
{त्वं मुहूर्तस्तिथिस्त्वं च त्वं लवस्त्वं पुनः क्षणः}
{शुक्लस्त्वं बहुलस्त्वं च कला काष्ठा त्रुटिस्तथा}
{संवत्सरर्तवो मासा रजन्यश्च दिनानि च}


\twolineshloka
{त्वमुत्तमा सगिरिवना वसुंधरासभास्करं वितिमिरमम्बरं तथा}
{मदोदधिः सतिमितिमिङ्गिलस्तथामहोर्मिमान्बहुमकरो झषाकुलः}


\twolineshloka
{महायशास्त्वमिति सदाऽभिपूज्यसेमनीषिभिर्मुदितमना महर्षिभिः}
{अभिष्टुतः पिबसि च सोममध्वरेकषट्कृतान्यपि च हवींषि भूतये}


\twolineshloka
{त्वं विप्रैः सततमिहेज्यसे फलार्थंवेदाङ्गेष्वतुलबलौघ गीयसे च}
{त्वद्धेतोर्यजनपरायणा द्विजेन्द्रावेदाङ्गान्यभिगमयन्ति सर्वयत्नैः}


\chapter{अध्यायः २६}
\twolineshloka
{सौतिरुवाच}
{}


\twolineshloka
{एवं स्तुतस्तदा कद्र्वा भगवान्हरिवाहनः}
{नीलजीमूतसंघातैः सर्वमम्बरमावृणोत्}


\twolineshloka
{मेघानाज्ञापयामास वर्षध्वममृतं शुभम्}
{ते मेघा मुमुचुस्तोयं प्रभूतं विद्युदुज्ज्वलाः}


\twolineshloka
{परस्परमिवात्यर्थं गर्जन्तः सततं दिवि}
{संवर्तितमिवाकाशं जलदैः सुमहाद्भुतैः}


\twolineshloka
{सृजद्भिरतुलं तोयमजस्रं सुमहारवैः}
{संप्रनृत्तमिवाकाशं धारोर्मिभिरनेकशः}


\twolineshloka
{मेघस्तनितनिर्घोषौर्विद्युत्पवनकम्पितैः}
{तैर्मेघैः सततासारं वर्षद्भिरनिशं तदा}


\twolineshloka
{नष्टचन्द्रार्ककिरणमम्बरं समपद्यत}
{नागानामुत्तमो हर्षस्तथा वर्षति वासवे}


\threelineshloka
{आपूर्यत मही चापि सलिलेन समन्ततः}
{रसातलमनुप्राप्तं शीतलं विमलं जलम्}
{}


\twolineshloka
{तदा भूरभवच्छन्ना जलोर्मिभिरनेकशः}
{रामणीयकमागच्छन्मात्रा सह भुजंगमाः}


\chapter{अध्यायः २७}
\twolineshloka
{सौतिरुवाच}
{}


\twolineshloka
{संप्रहृष्टास्ततो नागा जलधाराप्लुतास्तदा}
{सुपर्णेनोह्यमानास्ते जग्मुस्तं द्वीपमाशु वै}


\twolineshloka
{तं द्वीपं मकरावासं विहितं विश्वकर्मणा}
{तत्र ते लवणं घोरं ददृशुः पूर्वमागताः}


\twolineshloka
{सुपर्णसहिताः सर्पाः काननं च मनोरमम्}
{सागराम्बुपरिक्षिप्तं पक्षिसङ्घनिजादितम्}


\twolineshloka
{विचित्रफलपुष्पाभिर्वनराजिभिरावृतम्}
{भवनैरावृतं रम्यैस्तथा पद्माकरैरपि}


\twolineshloka
{प्रसन्नसलिलैश्चापि ह्वदैर्दिव्यैर्विभूषितम्}
{दिव्यगन्धवहैः पुण्यैर्मारुतैरुपवीजितम्}


\twolineshloka
{उत्पतद्भिरिवाकाशं वृक्षैर्मलयजैरपि}
{शोभितं पुष्पवर्षाणि मुञ्चद्भिर्मारुतोद्धतैः}


\twolineshloka
{वायुविक्षिप्तकुसुमैस्तथाऽन्यैरपि पादपैः}
{किरद्भिरिव तत्रस्थान्नागान्पुष्पाम्बुवृष्टिभिः}


\twolineshloka
{मनःसंहर्षजं दिव्यं गन्धर्वाप्सरसां प्रियम्}
{मत्तभ्रमस्संघुष्टं मनोज्ञाकृतिदर्शनम्}


\twolineshloka
{रमणीयं शिवं पुण्यं सर्वैर्जनमनोहरैः}
{नानापक्षिरुतं रम्यं कद्रूपुत्रप्रहर्षणम्}


\twolineshloka
{तत्ते वनं समासाद्य विजह्रुः पन्नगास्तदा}
{अब्रुवंश्च महावीर्यं सुपर्णं पतगेश्वरम्}


\threelineshloka
{वहास्मानपरं द्वीपं सुरम्यं विमलोदकम्}
{त्वं हि देशान्बहून्रम्यान्व्रजन्पश्यसि खेचर ॥सौतिरुवाच}
{}


\threelineshloka
{स विचिन्त्याब्रवीत्पक्षी मातरं विनतो तदा}
{किं कारणं मया मातः कर्तव्यं सर्पभाषितम् ॥विनतोवाच}
{}


\threelineshloka
{दासीभूतास्मि दुर्योगात्सपत्न्याः पतगोत्तम}
{पणं वितथमास्थाय सर्पैरुपधिना कृतम् ॥सौतिरुवाच}
{}


\twolineshloka
{तस्मिंस्तु कथिते मात्रा कारणे गगनेतरः}
{उवाच वचनं सर्पांस्तेन दुःखेन दुःखितः}


\threelineshloka
{किमाहृत्य विदित्वा वा किं वा कृत्वेह पौरुषम्}
{दास्याद्वो विप्रमुच्येयं तथ्यं वदत लेलिहाः ॥सौतिरुवाच}
{}


\twolineshloka
{श्रुत्वा समब्रुवन्सर्पा आहरामृतमोजसा}
{ततो दास्याद्विप्रमोक्षो भविता तव खेचर}


\chapter{अध्यायः २८}
\twolineshloka
{सौतिरुवाच}
{}


\threelineshloka
{इत्युक्तो गरुडः सर्पैस्ततो मातरमब्रवीत्}
{गच्छाम्यमृतमाहर्तुं भक्ष्यमिच्छामि वेदितुम् ॥विनतोवाच}
{}


\twolineshloka
{समुद्रकुक्षावेकान्ते निषादालयमुत्तमम्}
{`भवनानि निषादानां तत्र सन्ति द्विजोत्तम}


\twolineshloka
{पापिनां नष्टलोकानां निर्घृणानां दुरात्मनाम्'}
{निषादानां सहस्राणि तान्भुक्त्वाऽमृतमानय}


\twolineshloka
{न च ते ब्राह्मणं हन्तुं कार्या बुद्धिः कथंचन}
{अवध्यः सर्वभूतानां ब्राह्मणो ह्यनलोपमः}


\twolineshloka
{अग्निरर्को विषं शस्त्रं विप्रो भवति कोपितः}
{गुरुर्हि सर्वभूतानां ब्राह्मणः परिकीर्तितः}


\twolineshloka
{एवमादिस्वरूपैस्तु सतां वै ब्राह्मणो मतः}
{स ते तात न हन्तव्यः संक्रुद्धेनापि सर्वथा}


\twolineshloka
{ब्राह्मणानामभिद्रोहो न कर्तव्यः कथंचन}
{न ह्येवमग्निर्नादित्यो भस्म कुर्यात्तथानघ}


\twolineshloka
{यथा कुर्यादभिक्रुद्धो ब्राह्मणः संशितव्रतः}
{तदेतैर्विविधैर्लिङ्गैस्त्वं विद्यास्तं द्विजोत्तमम्}


\threelineshloka
{भूतानामग्रभूर्विप्रो वर्णश्रेष्ठः पिता गुरुः}
{गरुड उवाच}
{किंरूपो ब्राह्मणो मातः किंशीलः किंपराक्रमः}


\twolineshloka
{किंस्विदग्निनिभो भाति किंस्वित्सौम्यप्रदर्शनः}
{यथाहमभिजानीयां ब्राह्मणं लक्षणैः शुभैः}


\threelineshloka
{तन्मे कारणतो मातः पृच्छतो वक्तुमर्हसि}
{विनतोवाच}
{यस्ते कण्ठमनुप्राप्तो निगीर्णं बडिशं यथा}


\twolineshloka
{दहेदङ्गारवत्पुत्रं तं विद्याद्ब्राह्मणर्षभम्}
{विप्रस्त्वया न हन्तव्यः संक्रुद्धेनापि सर्वदा}


\twolineshloka
{प्रोवाच चैन विनता पुत्रहार्दादिदं वचः}
{जठरे न च जीर्येद्यस्तं जानीहि द्विजोत्तमम्}


\twolineshloka
{पुनः प्रोवाच विनता पुत्रहार्दादिदं वचः}
{जानन्त्यप्यतुलं वीर्यमाशीर्वादपरायणा}


\threelineshloka
{प्रीता परमदुःखार्ता नागैर्विप्रकृता सती}
{विनतोवाच}
{पक्षौ ते मारुतः पातु चन्द्रसूर्यौ च पृष्ठतः}


\threelineshloka
{शिरश्च पातु वह्निस्ते वसवः सर्वतस्तनुम्}
{`विष्णुः सर्वगतः सर्वमह्गानि तव चैव ह}
{'अहं च ते सदा पुत्र शान्तिस्वस्तिपरायणा}


\threelineshloka
{इहासीना भविष्यामि स्वस्तिकारे रता सदा}
{अरिष्टं व्रज पन्थानं पुत्र कार्यार्थसिद्धये ॥सौतिरुवाच}
{}


\twolineshloka
{ततः स मातुर्वचनं निशम्यवितत्य पक्षौ नभ उत्पपात}
{ततो निषादान्बलवानुपागतोबुभुक्षितः काल इवान्तकोऽपरः}


\twolineshloka
{स तान्निषादानुपसंहरंस्तदारजः समुद्धूय नभःस्पृशं महत्}
{समुद्रकुक्षौ च विशोषयन्पयःसमीपजान्भूधरजान्विचालयन्}


\twolineshloka
{ततः स चक्रे महदाननं तदानिषादमार्गं प्रतिरुध्य पक्षिराट्}
{ततो निषादास्त्वरिताः प्रवव्रजु-र्यतो मुखं तस्य भुजंगभोजिनः}


\threelineshloka
{तदाननं विवृतमतिप्रमाणव-त्समभ्ययुर्गगनमिवार्दिताः खगाः}
{सहस्रशः पवनजोविमोहितायथा}
{ञनिलप्रचलितपादपे वने}


\twolineshloka
{ततः खगो वदनममित्रतापनःसमाहरत्परिचपलो मत्बलाः}
{निषूदयन्बहुविधमत्स्यजीविनोबभुक्षितो गगनचरेश्वरस्तदा}


\chapter{अध्यायः २९}
\twolineshloka
{सौतिरुवाच}
{}


\twolineshloka
{तस्य कण्ठमनुप्राप्तो ब्राह्मणः सह भार्यया}
{दहन्दीप्त इवाङ्गारस्तमुवाचान्तरिक्षगः}


\threelineshloka
{द्विजोत्तम विनिर्गच्छ तूर्णमास्यादपावृतात्}
{न हि मे ब्राह्मणो वध्यः पापेष्वपि रतः सदा ॥सौतिरुवाच}
{}


\threelineshloka
{ब्रुवाणमेवं गरुडं ब्राह्मणः प्रत्यभाषत}
{निषादी मम भार्येयं निर्गच्छतु मया सह ॥गरुड उवाच}
{}


\threelineshloka
{एतामपि निषादीं त्वं परिगृह्याशु निष्पत}
{तूर्णं संभावयात्मानमजीर्णं मम तेजसा ॥सौतिरुवाच}
{}


\twolineshloka
{ततः स विप्रो निष्क्रान्तो निषादीसहितस्तदा}
{वर्धयित्वा च गरुडमिष्टं देशं जगाम ह}


\twolineshloka
{सहभार्ये विनिष्क्रान्ते तस्मिन्विप्रे स पक्षिराट्}
{वितत्य पक्षावाकाशमुत्पपात मनोजवः}


\threelineshloka
{ततोऽपश्यत्स्वपितरं पृष्टश्चाख्यातवान्पितुः}
{यथान्यायममेयात्मा तं चोवाच महानृषिः ॥कश्यप उवाच}
{}


\threelineshloka
{कच्चिद्वः कुशलं नित्यं भोजने बहुलं सुत}
{कच्चिच्च मानुषे लोके तवान्नं विद्यते बहु ॥`क्व गन्तास्यतिवेगेन मम त्वं वक्तुमर्हसि ॥' गरुड उवाच}
{}


\twolineshloka
{माता मे कुशला शश्वत्तथा भ्राता तथा ह्यहम्}
{न हि मे कुशलं तात भोजने बहुले सदा}


\twolineshloka
{अहं हि सर्पैः प्रहितः सोममाहर्तुमुत्तमम्}
{महातुर्दास्यविमोक्षार्थमाहरिष्ये तमद्य वै}


\twolineshloka
{मात्रा चात्र समादिष्टो निषादान्भक्षयेति ह}
{न च मे तृप्तिरभवद्भक्षयित्वा सहस्रशः}


\twolineshloka
{तस्माद्भक्ष्यं त्वमपरं भगवन्प्रदिशस्व मे}
{यद्भुक्त्वाऽमृतमाहर्तुं समर्तः स्यामहं प्रभो}


\threelineshloka
{क्षुत्पिपासाविघातार्थं भक्ष्यमाख्यातु मे भवान्}
{कश्यप उवाच}
{इदं सरो महापुण्यं देवलोकेऽपि विश्रुतम्}


\twolineshloka
{यत्र कूर्माग्रजं हस्ती सदा कर्षत्यवाङ्मुखः}
{तयोर्जन्मान्तरे वैरं संप्रवक्ष्याम्यसोषतः}


\twolineshloka
{तन्मे तत्त्वं निबोधस्य यत्प्रमाणौ च तावुभौ}
{`शृणु त्वं वत्स भद्रं ते कथां वैराग्यवर्धिनीम्}


\twolineshloka
{पित्रोरर्थविभागे वै समुत्पन्नां पुराण्डज}
{'आसीद्विभावसुर्नाम महर्षिः कोपनो भृशम्}


\twolineshloka
{भ्राता तस्यानुजश्चासीत्सुप्रतीको महातपाः}
{स नेच्छति धनं भ्रात्रा सहैकस्थं महामुनिः}


\twolineshloka
{विभागं कीर्तयत्येव सुप्रतीको हि नित्यशः}
{अथाब्रवीच्च तं भ्राता सुप्रतीकं विभावसुः}


\threelineshloka
{`विभागे बहवो दोषा भविष्यन्ति महातपः}
{'विभागं बहवो मोहात्कर्तुमिच्छन्ति नित्यशः}
{ततो विभक्तास्त्वन्योन्यं नाद्रियन्तेऽर्थमोहिताः}


\twolineshloka
{ततः स्वार्थपरान्मूढान्पृथग्भूतान्स्वकैर्धनैः}
{विदित्वा भेदयन्त्येतानमित्रा मित्ररूपिणः}


\twolineshloka
{विदित्वा चापरे भिन्नानन्तरेषु पतन्त्यथ}
{भिन्नानामतुलो नाशः क्षिप्रमेव प्रवर्तते}


\twolineshloka
{तस्माद्विभागं भ्रातॄणां न प्रशंसन्ति साधवः}
{`एवमुक्तः सुप्रतीको भागं कीर्तयतेऽनिशम्}


\twolineshloka
{एवं निर्बध्यमानस्तु शशापैनं विभावसुः}
{'गुरुशास्त्रेऽनिबद्धानामन्योन्येनाभिशङ्किनाम्}


\threelineshloka
{नियन्तु न हि शक्यस्त्वं भेदतो धनमिच्छसि}
{यस्मात्तस्मात्सुप्रतीक हस्तित्वं समवाप्स्यसि ॥कश्यप उवाच}
{}


\twolineshloka
{शप्तस्त्वेवं सुप्रतीको विभावसुमथाब्रवीत्}
{त्वमप्यन्तर्जलचरः कच्छपः संभविष्यसि}


\twolineshloka
{एवमन्योन्यशापात्तौ सुप्रतीकविभावसू}
{गजकच्छपतां प्राप्तावर्थार्थं मूढचेतसौ}


\twolineshloka
{रोषदोषानुषङ्गेण तिर्यग्योनिगतावपि}
{परस्परद्वेषरतौ प्रमाणबलदर्पितौ}


\twolineshloka
{सरस्यस्मिन्महाकायौ पूर्ववैरानुसारिणौ}
{तयोरन्यतरः श्रीमान्समुपैति महागजः}


\twolineshloka
{यस्य बृंहितशब्देन कूर्मोऽप्यन्तर्जलेशयः}
{उत्थितोऽसौ महाकायः कृत्स्नं विक्षोभयन्सरः}


\twolineshloka
{तं दृष्ट्वाऽऽवेष्टितकरः पतत्येष गजो जलम्}
{दन्तहस्ताग्रलाङ्गूलपादवेगेन वीर्यवान्}


\twolineshloka
{विक्षोभयंस्ततो नागः सरो बहुझषाकुलम्}
{कूर्मोऽप्यभ्युद्यतशिरा युद्धायाभ्येति वीर्यवान्}


\twolineshloka
{षडुच्छ्रितो योजनानि गजस्तद्द्विगुणायतः}
{कूर्मस्त्रियोजनोत्सेधो दशयोजनमण्डलः}


\twolineshloka
{तावुभौ युद्धसंमत्तौ परस्परवधैषिणौ}
{उपयुज्याशु कर्मेदं साधये हितमात्मनः}


\threelineshloka
{महाभ्रघनसंकाशं तं भुक्त्वामृतमानय}
{महागिरिसमप्रख्यं घोररूपं च हस्तिनम् ॥सौतिरुवाच}
{}


\twolineshloka
{इत्युक्त्वा गरुडं सोऽथ माङ्गल्यमकरोत्तदा}
{युध्यतः सह देवैस्ते युद्धे भवतु मङ्गलम्}


\twolineshloka
{पूर्णकुम्भो द्विजा गावो यच्चान्यत्किंचिदुत्तमम्}
{शुभं स्वस्त्ययनं चापि भविष्यति तवाण्डजा}


\twolineshloka
{युध्यमानस्य सङ्ग्रामे देवैः सार्धं महाबल}
{ऋचो यजूंषि सामानि पवित्राणि हवींषि च}


\threelineshloka
{रहस्यानि च सर्वाणि सर्वे वेदाश्च ते बलम्}
{`वर्धयिष्यन्ति समरे भविष्यति खगोत्तम}
{'इत्युक्तो गरुडः पित्रा गतस्तं ह्वदमन्तिकात्}


\twolineshloka
{अपश्यन्निर्मलजलं नानापक्षिसमाकुलम्}
{स तत्स्मृत्वा पितुर्वाक्यं भीमवेगोऽन्तरिक्षगः}


\twolineshloka
{नखेन गजमेकेन कूर्ममेकेन चाक्षिपत्}
{सधुत्पपात चाकाशं तत उच्चैर्विहङ्गमः}


\twolineshloka
{सोऽलम्बं तीर्थणासाद्य देववृक्षानुपागमत्}
{ते भीताः समकम्पन्त तस्य पक्षानिलाहताः}


\twolineshloka
{न नो भञ्ज्यादिति तदा दिव्याः कनकशाखिनः}
{प्रचलाङ्गान्स तान्दृष्ट्वा मनोरथफलद्रुमान्}


\threelineshloka
{अन्यानतुलरूपाङ्गानुपचक्राम खेचरः}
{काञ्चनै राजतैश्चैव फलैर्वैदूर्यशाखिनः}
{सागराम्बुपरिक्षिप्तान्भ्राजमानान्महाद्रुमान्}


\twolineshloka
{`तेषां मध्ये महानासीत्पादपः सुमनोहरः}
{सहस्रयोजनोत्सेधो बहुशाखासमन्वितः}


\twolineshloka
{खगानामालयो दिव्यो नाम्ना रौहिणपादपः}
{यस्य छायां समाश्रित्य सद्यो भवति निर्वृतः;}


\threelineshloka
{तमुवाच खगश्रेष्ठं तत्र रौहिणपादपः}
{अतिप्रवृद्धः समुहानापतन्तं मनोजवम् ॥रौहिण उवाच}
{}


\threelineshloka
{यैषा मम महाशाखा शतयोजनमायता}
{एतामास्थाय शाखां त्वं खादेमौ गजकच्छपौ ॥सौतिरुवाच}
{}


\twolineshloka
{ततो द्रुमं पतगसहस्रसेवितंमहीधरप्रतिमवपुः प्रकम्पयन्}
{खगोत्तमो द्रुतमभिपत्य वेगवा-न्बभञ्ज तामविरलपत्रसंचयाम्}


\chapter{अध्यायः ३०}
\twolineshloka
{सौतिरुवाच}
{}


\twolineshloka
{स्पष्टमात्रा तु पद्भ्यां सा गरुडेन बलीयसा}
{अभज्यत तरोः शाखा भग्नां चैकामधारयत्}


\twolineshloka
{तां भङ्क्त्वा स महाशाखां स्मयमानो विलोकयन्}
{अथात्रं लम्बतोऽपश्यद्वालखिल्यानधोमुखान्}


\twolineshloka
{ऋषयो ह्यत्र लम्बन्ते न हन्यामिति तानृषीन्}
{तपोरतांल्लम्बमानान्ब्रह्मर्षीनभिवीक्ष्य सः}


\twolineshloka
{हन्यादेतान्संपतन्ती शाखेत्यथ विचिन्त्य सः}
{नखैर्दृढतरं वीरः संगृह्य गजकच्छपौ}


\twolineshloka
{स तद्विनाशसंत्रासादभिपत्य स्वगाधिपः}
{शाखामास्येन जग्राह तेषामेवान्ववेक्षया}


\twolineshloka
{अतिदैवं तु तत्तस्य कर्म दृष्ट्वा महर्षयः}
{विस्मयोत्कम्पहृदया नाम चक्रुर्महाखगे}


\twolineshloka
{गुरुं भारं समासाद्योड्डीन एष विहङ्गमः}
{गरुडस्तु खगश्रेष्ठस्तस्मात्पन्नगभोजनः}


\twolineshloka
{ततः शनैः पर्यपतत्पक्षैः शैलान्प्रकम्पयन्}
{एवं सोऽभ्यपतद्देशान्बहून्सगजकच्छपः}


\twolineshloka
{दयार्थं वालखिल्यानां न च स्थानमविन्दत}
{स गत्वा पर्वतश्रेष्ठं गन्धमादनमञ्जसा}


\twolineshloka
{ददर्श कश्यपं तत्र पितरं तपसि स्थितम्}
{ददर्श तं पिता चापि दिव्यरूपं विहङ्गमम्}


\twolineshloka
{तेजोवीर्यबलोपेतं मनोमारुतरंहसम्}
{शैलशृङ्गप्रतीकाशं ब्रह्मदण्डमिवोद्यतम्}


\twolineshloka
{अचिन्त्यमनभिध्येयं सर्वभूतभयंकरम्}
{महावीर्यधरं रौद्रं साक्षादग्निमिवोद्यतम्}


\twolineshloka
{अप्रधृष्यमजेयं च देवदानवराक्षसैः}
{भेत्तारं गिरिशृङ्गाणां समुद्रजलशोषणम्}


\fourlineindentedshloka
{लोकसंलोडनं घोरं कृतान्तसमदर्शनम्}
{तमागतमभिप्रेक्ष्य भगवान्कश्यपस्तदा}
{विदित्वा चास्यं संकल्पमिदं वचनमब्रवीत् ॥कश्यप उवाच}
{}


\threelineshloka
{पुत्र मा साहसं कार्षीर्मा सद्यो लप्स्यसे व्यथाम्}
{मा त्वां दहेयुः संक्रुद्धा वालखिल्या मरीचिपाः ॥सौतिरुवाच}
{}


\threelineshloka
{ततः प्रसादयामास कश्यपः पुत्रकारणात्}
{वालखिल्यान्महाभागांस्तपसा हतकल्मषान् ॥कश्यप उवाच}
{}


\threelineshloka
{प्रजाहितार्थमारम्भो गरुडस्य तपोधनाः}
{चिकीर्षति महत्कर्म तदनुज्ञातुमर्हथ ॥सौतिरुवाच}
{}


\twolineshloka
{एवमुक्ता भगवता मुनयस्ते समभ्ययुः}
{मुक्त्वा शाखां गिरिं पुण्यं हिमवन्त तपोऽर्थिनः}


\twolineshloka
{ततस्तेष्वपयातेषु पितरं विनतासुतः}
{शाखाव्याक्षिप्तवदनः पर्यपृच्छत कश्यपम्}


\threelineshloka
{भगवन्क्व विमुञ्चामि तरोः शाखामिमामहम्}
{वर्जितं मानुषैर्देशमाख्यातु भगवान्मम ॥सौतिरुवाच}
{}


\twolineshloka
{ततो निःपुरुषं शैलं हिमसंरुद्धकन्दरम्}
{अगम्यं मनसाप्यन्यैस्तस्याचख्यौ स कश्यपः}


\twolineshloka
{तं पर्वतं महाकुक्षिमुद्दिश्य स महाखगः}
{जवेनाभ्यपतत्तार्क्ष्यः सशाखागजकच्छपः}


\twolineshloka
{न तां वध्री परिणहेच्छतचर्मा महातनुम्}
{शाखिनो महतीं शाखां यां प्रगृह्य ययौ खगः}


\twolineshloka
{स ततः शतसाहस्रं योजनान्तरमागतः}
{कालेन नातिमहता गरुडः पतगेश्वरः}


\twolineshloka
{स तं गत्वा क्षणेनैव पर्वतं वचनात्पितुः}
{अमुञ्चन्महतीं शाखां सस्वनं तत्र खेचरः}


\twolineshloka
{पक्षानिलहतश्चास्य प्राकम्पत स शैलराट्}
{मुमोच पुष्पवर्षं च समागलितपादप}


\twolineshloka
{शृङ्गाणि च व्यशीर्यन्त गिरेस्तस्य समन्ततः}
{मणिकाञ्चनचित्राणि शोभयन्ति महागिरिम्}


\twolineshloka
{शाखिनो बहवश्चापि शाखयाऽभिहतास्तया}
{काञ्चनैः कुसुमैर्भान्ति विद्युत्वन्त इवाम्बुदाः}


\twolineshloka
{ते हेमविकचा भूमौ युताः पर्वतधातुभिः}
{व्यराजञ्छाखिनस्तत्र सूर्यांशुप्रतिरञ्जिताः}


\twolineshloka
{ततस्तस्य गिरेः शृङ्गमास्थाय स खगोत्तमः}
{भक्षयामास गरुडस्तावुभौ गजकच्छपौ}


\twolineshloka
{तावुभौ भक्षयित्वा तु स तार्क्ष्यः कूर्मकुञ्जरौ}
{ततः पर्वतकूटाग्रादुत्पपात महाजवः}


\twolineshloka
{प्रावर्तन्ताथ देवानामुत्पाता भयशंसिनः}
{इन्द्रस्य वज्रं दयितं प्रजज्वाल भयात्ततः}


\twolineshloka
{सधूमा न्यपतत्सार्चिर्दिवोल्का नभसश्च्युता}
{तथा वसूनां रुद्राणामादित्यानां च सर्वशः}


\twolineshloka
{साध्यानां मरुतां चैव ये चान्ये देवतागणाः}
{स्वं स्वं प्रहरणं तेषां परस्परमुपाद्रवत्}


\twolineshloka
{अभूतपूर्वं संग्रामे तदा देवासुरेऽपि च}
{ववुर्वाताः सनिर्घाताः पेतुरुल्काः सहस्रशः}


\twolineshloka
{निरभ्रमेव चाकाशं प्रजगर्ज महास्वनम्}
{देवानामपि यो देवः सोऽप्यवर्षत शोणितम्}


\twolineshloka
{मम्लुर्माल्यानि देवानां नेशुस्तेजांसि चैव हि}
{उत्पातमेघा रौद्राश्च ववृषुः शोणितं बहु}


\threelineshloka
{रजांसि मुकुटान्येषामुत्थितानि व्यधर्षयन्}
{ततस्त्राससमुद्विग्नः सह देवैः शतक्रतुः}
{उत्पातान्दारुणान्पश्यन्नित्युवाच बृहस्पतिम्}


\threelineshloka
{किमर्थं भगवन्घोरा उत्पाताः सहसोत्थिताः}
{न च शत्रुं प्रपश्यामि युधि यो नः प्रधर्षयेत् ॥बृहस्पतिरुवाच}
{}


\twolineshloka
{तवापराधाद्देवेन्द्र प्रमादाच्च शतक्रतो}
{तपसा वालखिल्यानां महर्षीणां महात्मनाम्}


\twolineshloka
{कश्यपस्य मुनेः पुत्रो विनतायाश्च खेचरः}
{हर्तुं सोममभिप्राप्तो बलवान्कामरूपधृक्}


\threelineshloka
{समर्थो बलिनां श्रेष्ठो हर्तुं सोमं विहंगमः}
{सर्वं संभावयाम्यस्मिन्नसाध्यमपि साधयेत् ॥सौतिरुवाच}
{}


\twolineshloka
{श्रुत्वैतद्वचनं शक्रः प्रोवाचामृतरक्षिणः}
{महावीर्यबलः पक्षी हर्तुं सोममिहोद्यतः}


\twolineshloka
{युष्मान्संबोधयाम्येष `गृहीत्वावरणायुधान्}
{परिवार्यामृतं सर्वे यूयं मद्वचनादिह}


\threelineshloka
{रक्षध्वं विबुधा वीरा' यथा न स हरेद्बलात्}
{अतुलं हि बलं तस्य बृहस्पतिरुवाच ह ॥सौतिरुवाच}
{}


\twolineshloka
{तच्छ्रुत्वा विबुधा वाक्यं विस्मिता यत्नमास्थिताः}
{परिवार्यामृतं तस्थूर्वज्री चेन्द्रः प्रतापवान्}


\twolineshloka
{धारयन्तो विचित्राणि काञ्चनानि मनस्विनः}
{कवचानि महार्हाणि वैदूर्यविकृतानि च}


\twolineshloka
{चर्माण्यपि च गात्रेषु भानुमन्ति दृढानि च}
{विविधानि च शस्त्राणि घोररूपाण्यनेकशः}


\twolineshloka
{शिततीक्ष्णाग्रधाराणि समुद्यम्य सुरोत्तमः}
{सविस्फुलिङ्गज्वालानि सधूमानि च सर्वशः}


\threelineshloka
{चक्राणि परघांश्चैव त्रिशूलानि परश्वधान्}
{शक्तीश्च विविधास्तीक्ष्णाः करवालांश्च निर्मलान्}
{स्वदेहरूपाण्यादाय गदाश्चोग्रप्रदर्शनाः}


\twolineshloka
{तैः शस्त्रैर्भानुमद्भिस्ते दिव्याभरणभूषिताः}
{भानुमन्तः सुरगणास्तस्थुर्विगतकल्मषाः}


\twolineshloka
{अनुपमबलवीर्यतेजसोधृतमनसः परिरक्षणेऽमृतस्य}
{असुरपुरविदारणाः सुराज्वलनसमिद्धवपुःप्रकाशिनः}


\twolineshloka
{इति समरवरं सुराः स्थितास्तेपरिघसहस्रशतैः समाकुलम्}
{विगलितमिव चाम्बरान्तरंतपनमरीचिविकाशितं बभासे}


\chapter{अध्यायः ३१}
\twolineshloka
{शौनक उवाच}
{}


\twolineshloka
{कोऽपराधो महेन्द्रस्य कः प्रमादश्च सूतज}
{तपसा वालखिल्यानां संभूतो गरुडः कथम्}


\twolineshloka
{कश्यपस्य द्विजातेश्च कथं वै पक्षिराट् सुतः}
{अधृष्टः सर्वभूतानामवध्यस्चाभवत्कथम्}


\threelineshloka
{कथं च कामचारी स कामवीर्यश्च खेचरः}
{एतदिच्छाम्यहं श्रोतुं पुराणे यदि पठ्यते ॥सौतिरुवाच}
{}


\twolineshloka
{विषयोऽयं पुराणस्य यन्मां त्वं परिपृच्छसि}
{शृणु मे वदतः सर्वमेतत्संक्षेपतों द्विज}


\twolineshloka
{यजतः पुत्रकामस्य कश्यपस्य प्रजापतेः}
{साहाय्यमृषयो देवा गन्धर्वाश्च ददुः किल}


\twolineshloka
{तत्रेध्मानयने शक्रो नियुक्तः कश्यपेन ह}
{मुनयो वालखिल्याश्च ये चान्ये देवतागणाः}


\twolineshloka
{शक्रस्तु वीर्यसदृशमिध्यभारं गिरिप्रभम्}
{समुद्यम्यानयामास नातिकृच्छ्रादिव प्रभुः}


\twolineshloka
{अथापश्यदृषीन्ह्रस्वानङ्गुष्ठोदरवर्ष्मणः}
{पलाशवर्तिकामेकां वहतः संहतान्पथि}


\twolineshloka
{प्रलीनान्स्वेष्विवाङ्गेषु निराहारांस्तपोधनान्}
{क्लिश्यमानान्मन्दबलान्गोष्पदे संप्लुतोदके}


\twolineshloka
{तान्सर्वान्विस्मयाविष्टो वीर्योन्मत्तः पुरन्दरः}
{अपहास्याभ्यगाच्छीघ्रं लम्बयित्वाऽवमन्य च}


\twolineshloka
{तेऽथ रोषसमाविष्टाः सुभृशं जातमन्यवः}
{आरेभिरे महत्कर्म तदा शक्रभयंकरम्}


\twolineshloka
{जुहुवुस्ते सुतपसो विधिवज्जातवेदसम्}
{मन्त्रैरुच्चावचैर्विप्रा येन कामेन तच्छृणु}


\twolineshloka
{कामवीर्यः कामगमो देवराजभयप्रदः}
{इन्द्रोऽन्यः सर्वदेवानां भवेदिति यतव्रताः}


\twolineshloka
{इन्द्राच्छतगुणः शौर्ये वीर्ये चैव मनोजवः}
{तपसो नः फलेनाद्य दारुणः संभवित्विति}


\twolineshloka
{तद्बुद्ध्वा भृशसंतप्तो देवराजः शतक्रतुः}
{जगाम शरणं तत्र कश्यपं संशितव्रतम्}


\threelineshloka
{तच्छ्रुत्वा देवराजस्य कश्यपोऽथ प्रजापतिः}
{वालखिल्यानुपागम्य कर्मसिद्धिमपृच्छत ॥`कश्यप उवाच}
{}


\threelineshloka
{केन कामेन चारब्धं भवद्भिर्होमकर्म च}
{याथातथ्येन मे ब्रूत श्रोतुं कौतूहलं हि मे ॥वालखिल्या ऊचुः}
{}


\twolineshloka
{अवज्ञाताः सुरेन्द्रेण मूढेनाकृतबुद्धिना}
{ऐश्वर्यमदमत्तेन सदाचारान्निरस्यता}


\twolineshloka
{तद्विघातार्थमारम्भो विधिवत्तस्य कश्यप ॥सौतिरुवाच}
{'}


\twolineshloka
{एवमस्त्विति तं चापि प्रत्यूचुः सत्यवादिनः}
{तान्कश्यप उवाचेदं सान्त्वपूर्वं प्रजापतिः}


\twolineshloka
{अयमिन्द्रस्त्रिभुवने नियोगाद्ब्रह्मणः कृतः}
{इन्द्रार्थे च भवन्तोऽपि यत्नवन्तस्तपोधनाः}


\twolineshloka
{न मिथ्या ब्रह्मणो वाक्यं कर्तुमर्हथ सत्तमाः}
{भवतां हि न मिथ्याऽयं संकल्पो वै चिकीर्षितः}


\threelineshloka
{भवत्वेष पतत्रीणामिन्द्रोऽतिबलसत्त्ववान्}
{प्रसादः क्रियतामस्य देवराजस्य याचतः ॥सौतिरुवाच}
{}


\threelineshloka
{एवमुक्ताः कश्यपेन वालखिल्यास्तपोधनाः}
{प्रत्यूचुरभिसंपूज्य मुनिश्रेष्ठं प्रजापतिम् ॥वालखिल्या ऊचुः}
{}


\twolineshloka
{इन्द्रार्थोऽयं समारम्भः सर्वेषां नः प्रजापते}
{अपत्यार्थं समारम्भो भवतश्चायमीप्सितः}


\threelineshloka
{तदिदं सफलं कर्म त्वयैव प्रतिगृह्यताम्}
{तथा चैवं विधत्स्वात्र यथा श्रेयोऽनुपश्यसि ॥सौतिरुवाच}
{}


\twolineshloka
{एतस्मिन्नेव काले तु देवी दाक्षायणी शुभा}
{विनता नाम कल्याणी पुत्रकामा यशस्विनी}


\twolineshloka
{तपस्तप्त्वा व्रतपरा स्नाता पुंसवने शुचिः}
{उपचक्राम भर्तारं तामुवाचाथ कश्यपः}


\twolineshloka
{आरम्भः सफलो देवि भविता यस्त्वयेप्सितः}
{जनयिष्यसि पुत्रौ द्वौ वीरौ त्रिभुवनेश्वरौ}


\twolineshloka
{तपसा वालखिल्यानां मम संकल्पतस्तथा}
{भविष्यतो महाभागौ पुत्रौ त्रैलोक्यपूजितौ}


\twolineshloka
{उवाच चैनां भगवान्कश्यपः पुनरेव ह}
{धार्यतामप्रमादेन गर्भोऽयं सुमहोदयः}


\threelineshloka
{एकः सर्वपतत्रीणामिन्द्रत्वं कारयिष्यति}
{लोकसंभावितो वीरः कामरूपो विहङ्गमः ॥सौतिरुवाच}
{}


\twolineshloka
{शतक्रतुमथोवाच प्रीयमाणः प्रजापतिः}
{त्वत्सहायौ महावीर्यौ भ्रातरौ ते भविष्यतः}


\twolineshloka
{नैताभ्यां भविता दोषः सकाशात्ते पुरंदर}
{व्येतु ते शक्र संतापस्त्वमेवेन्द्रो भविष्यसि}


\threelineshloka
{न चाप्येवं त्वया भूयः क्षेप्तव्या ब्रह्मवादिनः}
{न चावमान्या दर्पात्ते वाग्वज्रा भृशकोपनाः ॥सौतिरुवाच}
{}


\twolineshloka
{एवमुक्तो जगामेन्द्रो निर्विशङ्कस्त्रिविष्टपम्}
{विनता चापि सिद्धार्था बभूव मुदिता तथा}


\twolineshloka
{जनयामास पुत्रौ द्वावरुणं गरुडं तथा}
{विकलाङ्गोऽरुणस्तत्र भास्करस्य पुरःसरः}


\twolineshloka
{पतत्त्रीणां च गरुडमिन्द्रत्वेनाभ्यषिञ्चत}
{तस्यैतत्कर्म सुमहच्छ्रूयतां भृगुनन्दन}


\chapter{अध्यायः ३२}
\twolineshloka
{सौतिरुवाच}
{}


\twolineshloka
{`ततस्तस्माद्गिरिवरात्समुदीर्णमहाबलः}
{'गरुडः पक्षिराट् तूर्णं संप्राप्तो विबुधान्प्रति}


\twolineshloka
{तं दृष्ट्वातिबलं चैव प्राकम्पन्त सुरास्ततः}
{परस्परं च प्रत्यघ्नन्सर्वप्रहरणान्युत}


\twolineshloka
{तत्र चासीदमेयात्मा विद्युदग्निसमप्रभः}
{भौमनः सुमहावीर्यः सोमस्य परिरक्षिता}


\twolineshloka
{स तेन पतगेन्द्रेण पक्षतुण्डनखैः क्षतः}
{मुहूर्तमतुलं युद्धं कृत्वा विनिहतो युधि}


\twolineshloka
{रजश्चोद्धूय सुमहत्पक्षवातेन खेचरः}
{कृत्वा लोकान्निरालोकांस्तेन देवानवाकिरत्}


\twolineshloka
{तेनावकीर्णा रजसा देवा मोहमुपागमन्}
{न चैवं ददृशुश्छन्ना रजसाऽमृतरक्षिणः}


\twolineshloka
{एवं संलोडयामास गरुडस्त्रिदिवालयम्}
{पक्षतुण्डप्रहारैस्तु देवान्स विददार ह}


\threelineshloka
{ततो देवः सहस्राक्षस्तूर्णं वायुमचोदयत्}
{विक्षिपेमां रजोवृष्टिं तवेदं कर्म मारुत ॥सौतिरुवाच}
{}


\twolineshloka
{अथ वायुरपोवाह तद्रजस्तरसा बली}
{ततो वितिमिरे जाते देवाः शकुनिमार्दयन्}


\twolineshloka
{ननादोच्चैः स बलवान्महामेघ इवाम्बरे}
{वध्यमानः सुरगणैः सर्वभूतानि भीषयन्}


\twolineshloka
{उत्पपात महावीर्यः पक्षिराट् परवीरहा}
{समुत्पत्यान्तरिक्षस्थं देवानामुपरि स्थितम्}


\twolineshloka
{वर्मिमो विबुधाः सर्वे नानाशस्त्रैरवाकिरन्}
{पट्टिशैः परिधैः शूलैर्गदाभिश्च सवासवाः}


\twolineshloka
{क्षुरप्रैर्ज्वलितैश्चापि चक्रैरादित्यरूपिभिः}
{नानाशस्त्रविसर्गैस्तैर्वध्यमानः समन्ततः}


\threelineshloka
{कुर्वन्सुतुमुलं युद्धं पक्षिराण्ण व्यकम्पत}
{निर्दहन्निव चाकाशे वैनतेयः प्रतापवान्}
{पक्षाभ्यामुरसा चैव समन्ताद्व्यक्षिपत्सुरान्}


\twolineshloka
{ते विक्षिप्तास्ततो देवा दुद्रुवुर्गरुडार्दिताः}
{नखतुण्डक्षताश्चैव सुस्रुवुः शोणितं बहु}


\twolineshloka
{साध्याः प्राचीं सगन्धर्वा वसवो दक्षिणां दिशम्}
{प्रजग्मुः सहिता रुद्राः पतगेन्द्रप्रधर्षिताः}


\twolineshloka
{दिशं प्रतीचीमादित्या नासत्यावुत्तरां दिशम्}
{मुहुर्मुहुः प्रेक्षमाणा युध्यमानं महौजसः}


\twolineshloka
{अश्वक्रन्देन वीरेण रेणुकेन च पक्षिराट्}
{क्रथनेन च शूरेण तपनेन च खेचरः}


\twolineshloka
{उलूकश्वसनाभ्यां च निमेषेण च पक्षिराट्}
{प्ररुजेन च संग्रामं चकार पुलिनेन च}


\twolineshloka
{तान्पक्षनखतुण्डाग्रैरभिनद्विनतासुतः}
{युगान्तकाले संक्रुद्धः पिनाकीव परंतप}


\twolineshloka
{महाबला महोत्साहास्तेन ते बहुधा क्षताः}
{रेजुरभ्रघनप्रख्या रुधिरौघप्रवर्षिणः}


\twolineshloka
{तान्कृत्वा पतगश्रेष्ठः सर्वानुत्क्रान्तजीवितान्}
{अतिक्रान्तोऽमृतस्यार्थे सर्वतोऽग्निमपश्यत}


\twolineshloka
{आवृण्वानं महाज्वालमर्चिर्भिः सर्वतोऽम्बरम्}
{दहन्तमिव तीक्ष्णांशुं चण्डवायुसमीरितम्}


\threelineshloka
{`नभः स्पृशन्तं ज्वालाभिः सर्वभूतभयंकरम्}
{'ततो नवत्या नवतीर्मुखानांकृत्वा महात्मा गरुडस्तरस्वी}
{नदीः समापीय मुखैस्ततस्तैःसुशीघ्रमागम्य पुनर्जवेन}


\twolineshloka
{ज्वलन्तमग्निं तममित्रतापनःसमास्तरत्पत्ररथो नदीभिः}
{ततः प्रचक्रे वपुरन्यदल्पंप्रवेष्टुकामोऽग्निमभिप्रशाम्य}


\chapter{अध्यायः ३३}
\twolineshloka
{सौतिरुवाच}
{}


\twolineshloka
{जाम्बूनदमयो भूत्वा मरीचिनिकरोज्ज्वलः}
{प्रविवेश बलात्पक्षी वारिवेग इवार्णवम्}


\twolineshloka
{स चक्रं क्षुरपर्यन्तमपश्यदमृतान्तिके}
{परिभ्रमन्तमनिशं तीक्ष्णधारमयस्मयम्}


\twolineshloka
{ज्वलनार्कप्रभं घोरं छेदनं सोमहारिणाम्}
{घोररूपं तदत्यर्थं यन्त्रं देवैः सुनिर्मितम्}


\twolineshloka
{तस्यान्तरं स दृष्ट्वै पर्यवर्तत खेचरः}
{अरान्तरेणाभ्यपतत्संक्षिप्याङ्गं क्षणेन ह}


\twolineshloka
{अधश्चक्रस्य चैवात्र दीप्तानलसमद्व्युती}
{विद्युज्जिह्वौ महावीर्यौ दीप्तास्यौ दीप्तलोचनौ}


\twolineshloka
{चक्षुर्विषौ महाघोरौ नित्यं क्रुद्धौ तरस्विनौ}
{अमृतस्यैव रक्षार्थं ददर्श भुजगोत्तमौ}


\twolineshloka
{सदा संरब्धनयनौ सदा चानिमिषेक्षणौ}
{तयोरेकोऽपि यं पश्येत्स तूर्णं भस्मसाद्भवेत्}


\twolineshloka
{`तौ दृष्ट्वा सहसा खेदं जगाम विनतात्मजः}
{कथमेतौ महावीर्यौ जेतव्यौ हरिभोजिनौ}


\threelineshloka
{इति संचिन्त्य गरुडस्तयोस्तूर्णं निराकरः}
{'तयोश्चक्षूंषि रजसा सुपर्णः सहसाऽऽवृणोत्}
{ताभ्यामदृष्टरूपोऽसौ सर्वतः समताडयत्}


\twolineshloka
{तयोरङ्गे समाक्रम्य वैनतेयोऽन्तरिक्षगः}
{आच्छिनत्तरसा मध्ये सोममभ्यद्रवत्ततः}


\twolineshloka
{समुत्पाट्यामृतं तत्र वैनतेयस्ततो बली}
{उत्पपात जवेनैव यन्त्रमुन्मथ्य वीर्यवान्}


\twolineshloka
{अपीत्वैवामृतं पक्षी परिगृह्याशु निःसृतः}
{आगच्छदपरिश्रान्त आवार्यार्कप्रभां ततः}


\twolineshloka
{विष्णुना च तदाकाशे वैनतेयः समेयिवान्}
{तस्य नारायणस्तुष्टस्तेनालौल्येन कर्मणा}


\twolineshloka
{तमुवाचाव्ययो देवो वरदोऽस्मीति खेचरम्}
{स वव्रे तव तिष्ठेयमुपरीत्यन्तरिक्षगः}


\threelineshloka
{उवाच चैनं भूयोऽपि नारायणमिदं वचः}
{अजरश्चामरश्च स्याममृतेन विनाऽप्यहम् ॥सौतिरुवाच}
{}


\twolineshloka
{एवमस्त्विति तं विष्णुरुवाच विनतासुतम्}
{प्रतिगृह्य वनौ तौ च गरुडो विष्णुमब्रवीत्}


\twolineshloka
{भवतेपि वरं दद्यां वृणोतु भगवानपि}
{तं वव्रे वाहनं विष्णुर्नरुत्मन्तं महाबलम्}


\twolineshloka
{ध्वजं च चक्रे भगवानुपरि स्थास्यसीति तम्}
{एवमस्त्विति तं देवमुक्त्वा नारायणं खगः}


\twolineshloka
{वव्राज तरसा वेगाद्वायुं स्पर्धन्महाजवः}
{तं व्रजन्तं खगश्रेष्ठं वज्रेणेन्द्रोऽभ्यताडयत्}


\twolineshloka
{हरन्तममृतं रोषाद्गरुडं पक्षिणां वरम्}
{तमुवाचेन्द्रमाक्रन्दे गरुडः पततां वरः}


\twolineshloka
{प्रहसञ्श्लक्ष्णया वाचा तथा वज्रसमाहतः}
{ऋषेर्मानं करिष्यामि वज्रं यस्यास्थिसंभवम्}


\twolineshloka
{वज्रस्य च करिष्यामि तवैव च शतक्रतो}
{एतत्पत्रं त्यजाम्येकं यस्यान्तं नोपलप्स्यसे}


\twolineshloka
{न च वज्रनिपातेन रुजा मेऽस्तीह काचन}
{एवमुक्त्वा ततः पुत्रमुत्ससर्ज स पक्षिराट्}


\twolineshloka
{तदुत्सृष्टमभिप्रेक्ष्य तस्य पर्णमनुत्तमम्}
{हृष्टानि सर्धभूतानि नाम चक्रुर्गरुत्मतः}


\threelineshloka
{सुरूपं पत्रमालक्ष्य सुपर्णोऽयं भवत्विति}
{तद्दृष्ट्वा महदाश्चर्यं सहस्राक्षः पुरंदरः}
{खगो महदिदं भूतमिति मत्वाऽभ्यभाषत}


\twolineshloka
{बलं विज्ञातुमिच्छामि यत्ते परमनुत्तमम्}
{सख्यं चानन्तमिच्छामि त्वया सह खगोत्तम}


\chapter{अध्यायः ३४}
\twolineshloka
{सौतिरुवाच}
{}


\fourlineindentedshloka
{`इत्येवमुक्तो गरुडः प्रत्युवाच शचीपतिम्'}
{गरुड उवाच}
{सख्यं मेऽस्तु त्वया देव यथेच्छसि पुरंदर}
{बलं तु मम जानीहि महच्चासह्यमेव च}


\twolineshloka
{कामं नैतत्प्रशंसन्ति सन्तः स्वबलसंस्तवम्}
{`अनिमित्तं सुरश्रेष्ठ सद्यः प्राप्नोति गर्हणाम्}


\twolineshloka
{गुणसंकीर्तनं चापि पृष्टेनान्येन गोपते}
{वक्तव्यं न तु वक्तव्यं स्वयमेव शतक्रतो ॥'}


\twolineshloka
{सखेति कृत्वा तु सखे पृष्टो वक्ष्याम्यहं त्वया}
{न ह्यात्मस्तवसंयुक्तं वक्तव्यमनिमित्ततः}


\twolineshloka
{सपर्वतवनामुर्वीं ससागरजलामिमाम्}
{वहे पक्षेण वै शक्र त्वामप्यत्रावलम्बिनम्}


\threelineshloka
{सर्वान्संपिण्डितान्वापि लोकान्सस्थाणुजङ्गमान्}
{वहेयमपरिश्रान्तो विद्धीदं मे महद्बलम् ॥सौतिरुवाच}
{}


\twolineshloka
{इत्युक्तवचनं वीरं किरीटी श्रीमतां वरः}
{आह शौनक देवेन्द्रः सर्वलोकहितः प्रभुः}


\twolineshloka
{एवमेव यथात्थ त्वं सर्वं संभाव्यते त्वयि}
{संगृह्यतामिदानीं मे सख्यमत्यन्तमुत्तमम्}


\threelineshloka
{न कार्यं यदि सोमेन मम सोमः प्रदीयताम्}
{अस्मांस्ते हि प्रबाधेयुर्येभ्यो दद्याद्भवानिमम् ॥गरुड उवाच}
{}


\twolineshloka
{किंचित्कारणमुद्दिश्य सोमोऽयं नीयते मया}
{न दास्यामि समापातुं सोमं कस्मैचिदप्यहम्}


\threelineshloka
{यत्रेमं तु सहस्राक्ष निक्षिपेयमहं स्वयम्}
{त्वमादाय ततस्तृर्णं हरेथास्त्रिदिवेश्वर ॥शक्र उवाच}
{}


\threelineshloka
{वाक्येनानेन तुष्टोऽहं यत्त्वयोक्तमिहाण्डज}
{यमिच्छसि वरं मत्तस्तं गृहाण खगोत्तम ॥सौतिरुवाच}
{}


\threelineshloka
{इत्युक्तः प्रत्युवाचेदं कद्रूपुत्राननुस्मरन्}
{भवेयुर्भुजगाः शक्र मम भक्ष्या महाबलाः ॥गरुड उवाच}
{}


\threelineshloka
{ईशोऽहमपि सर्वस्य करिष्यामि तु तेऽर्थिताम्}
{भवेयुर्भुजगाः शक्र मम भक्ष्या महाबलाः ॥सौतिरुवाच}
{}


\twolineshloka
{तथेत्युक्त्वाऽन्वगच्छत्तं ततो दानवसूदनः}
{देवदेवं महात्मानं योगिनामीश्वरं हरिम्}


\twolineshloka
{स चान्वमोदत्तं चार्थं यथोक्तं गरुडेन वै}
{इदं भूयो वचः प्राह भगवांस्त्रिदशेश्वरः}


\twolineshloka
{हरिष्यामि विनिक्षिप्तं सोममित्यनुभाष्य तम्}
{आजगाम ततस्तूर्णं सुपर्णी मातुरन्तिकम्}


\twolineshloka
{`विनयावनतो भूत्वा वचनं चेदमब्रवीत्}
{इदमानीतममृतं देवानां भवनान्मया}


\threelineshloka
{प्रशाधि किमितो मातः करिष्यामि शुभव्रते}
{विनतोवाच}
{परितुष्टाऽहमेतेन कर्मणा तव पुत्रक}


% Check verse!
अजरश्चाभरश्चैव देवानां सुप्रियो भव

सौतिरुवाच

'अथ सर्पानुवाचेदं सर्वान्परमहृष्टवत्

गरुड उवाच

इदमानीतममृतं निक्षेप्स्यामि कुशेषु वः
\twolineshloka
{स्नाता मङ्गलसंयुक्तास्ततः प्राश्नीत पन्नगाः}
{भवद्भिरिदमासीनैर्यदुक्तं तद्वचस्तदा}


\threelineshloka
{अदासी चैव मातेयमद्यप्रभृति चास्तु मे}
{यथोक्तं भवतामेतद्वचो मे प्रतिपादितम् ॥सौतिरुवाच}
{}


\twolineshloka
{ततः स्नातुं गताः सर्पाः प्रत्युक्त्वा तं तथेत्युत}
{शक्रोऽप्यमृतमाक्षिप्य जगाम त्रिदिवं पुनः}


\twolineshloka
{अथागतास्तमुद्देशं सर्पाः सोमार्थिनस्तदा}
{स्नाताश्च कुतजप्याश्च प्रहृष्टाः कृतमङ्गलाः}


\twolineshloka
{`परस्परकृतद्वेषाः सोमप्राशनकर्मणि}
{अहं पूर्वमहं पूर्वमित्युक्त्वा ते समाद्रवन् ॥'}


\twolineshloka
{यत्रैतदमृतं चापि स्थापितं कुशसंस्तरे}
{तद्विज्ञाय हृतं सर्पाः प्रतिमायाकृतं च तत्}


\twolineshloka
{सोमस्थानमिदं चेति दर्भांस्ते लिलिहुस्तदा}
{ततो द्विधा कृता जिह्वाः सर्पाणां तेन कर्मणा}


\threelineshloka
{अभवंश्चामृतस्पर्शाद्दर्भास्तेऽथ पवित्रिणः}
{`नागाश्च वञ्चिता भूत्वा विसृज्य विनतां ततः}
{विषादमगमंस्तीव्रं गरुडस्य बलात्प्रभो ॥'}


\twolineshloka
{एवं तदमृतं तेन हृतमाहृतमेव च}
{द्विजिह्वाश्च कृताः सर्पा गरुडेन महात्मना}


\twolineshloka
{ततः सुपर्णः परमप्रहर्षवा-न्विहृत्य मात्रा सह तत्र कानने}
{भुजंगभक्षः परमार्चितः खगै-रहीनकीर्तिर्विनतामनन्दयत्}


\twolineshloka
{इमां कथां यः शृणुयान्नरः सदापठेत वा द्विजगणमुख्यसंसदि}
{असंशयं त्रिदिवमियात्स पुण्यभा-ङ्महात्मनः पतगपतेः प्रकीर्तनात्}


\chapter{अध्यायः ३५}
\twolineshloka
{शौनक उवाच}
{}


\twolineshloka
{भुजङ्गमानां शापस्य मात्रा चैव सुतेन च}
{विनतायास्त्वया प्रोक्तं कारणं सूतनन्दन}


\twolineshloka
{वरप्रदानं भर्त्रा च कद्रूविनतयोस्तथा}
{नामनी चैव ते प्रोक्ते पक्षिणोर्वैनतेययोः}


\threelineshloka
{पन्नगानां तु नामानि न कीतर्यसि सूतज}
{प्राधान्येनापि नामानि श्रोतुमिच्छामहे वयम् ॥सौतिरुवाच}
{}


\twolineshloka
{बहुत्वान्नामधेयानि पन्नगानां तपोधन}
{न कीर्तयिष्ये सर्वेषां प्राधान्येन तु मे शृणु}


\twolineshloka
{शेषः प्रथमतो जातो वासुकिस्तदनन्तरम्}
{ऐरावतस्तक्षकश्च कर्कोटकधनञ्जयौ}


\twolineshloka
{कालियो मणिनागश्च नागश्चापूरणस्तथा}
{नागस्तथा पिञ्जरक एलापत्रोऽथ वामनः}


\twolineshloka
{नीलानीलौ तथा नागौ कल्माषशबलौ तथा}
{आर्यकश्चोग्रकश्चैव नागः कलशपोतकः}


\twolineshloka
{सुमनाख्यो दधिमुखस्तथा विमलपिण्डकः}
{आप्तः कोटरकश्चैव शङ्खो वालिशिखस्तथा}


\twolineshloka
{निष्टानको हेमगुहो नहुषः पिङ्गलस्तथा}
{बाह्यकर्णो हस्तिपदस्तथा मुद्गरपिण्डकः}


\twolineshloka
{कम्बलाश्वतरौ चापि नागः कालीयकस्तथा}
{वृत्तसंवर्तकौ नागौ द्वौ च पद्माविति श्रुतौ}


\twolineshloka
{नागः शङ्खमुखश्चैव तथा कूष्माण्डकोऽपरः}
{क्षेमकश्च तथा नागो नागः पिण्डारकस्तथा}


\twolineshloka
{करवीरः पुष्पदंष्ट्रो बिल्वको बिल्वपाण्डुरः}
{मूषकादः शङ्खशिराः पूर्णभद्रो हरिद्रकः}


\twolineshloka
{अपराजितो ज्योतिकश्च पन्नगः श्रीवहस्तथा}
{कौरव्यो धृतराष्ट्रश्च शङ्खपिण्डश्च वीर्यवान्}


\twolineshloka
{विरजाश्च सुबाहुश्च शालिपिण्डश्च वीर्यवान्}
{हस्तिपिण्डः पिठरकः सुमुखः कौणपाशनः}


\twolineshloka
{कुठऱः कुञ्जरश्चैव तथा नागः प्रभाकरः}
{कुमुदः कुमुदाक्षश्च तित्तिरिर्हलिकस्तथा}


\twolineshloka
{कर्दमश्च महानागो नागश्च बहुमूलकः}
{कर्कराकर्करौ नागौ कुण्डोदरमहोदरौ}


\twolineshloka
{एते प्राधान्यतो नागाः कीर्तिता द्विजसत्तम}
{बहुत्वान्नामधेयानामितरे नानुकीर्तिताः}


\twolineshloka
{एतेषां प्रसवो यश्च प्रसवस्य च संततिः}
{असङ्ख्येयेति मत्त्वा तान्न ब्रवीमि तपोधन}


\twolineshloka
{बहूनीह सहस्राणि प्रयुतान्यर्बुदानि च}
{अशक्यान्येव सङ्ख्यातुं पन्नगानां तपोधन}


\chapter{अध्यायः ३६}
\twolineshloka
{शौन उवाच}
{}


\threelineshloka
{आख्याता भुजगास्तात वीर्यवन्तो दुरासदाः}
{शापं तं तेऽभिविज्ञाय कृतवन्तः किमुत्तरम् ॥सौतिरुवाच}
{}


\twolineshloka
{तेषां तु भगवाञ्छेषः कद्रूं त्यक्त्वा महायशाः}
{उग्रं तपः समातस्थे वायुभक्षो यतव्रतः}


\twolineshloka
{गन्धमादनमासाद्य बदर्यां च तपोरतः}
{गोकर्णे पुष्करारण्ये तथा हिमवतस्तटे}


\twolineshloka
{तेषु तेषु च पुण्येषु तीर्थेष्वायतनेषु च}
{एकान्तशीलो नियतः सततं विजितेन्द्रियः}


\twolineshloka
{तप्यमानं तपो घोरं तं ददर्श पितामहः}
{संशुष्कमांसत्वक्स्नायुं जटाचीरधरं मुनिम्}


\twolineshloka
{तमब्रवीत्सत्यधृतिं तप्यमानं पितामहः}
{किमिदं कुरुषे शेष प्रजानां स्वस्ति वै कुरु}


\threelineshloka
{त्वं हि तीव्रेण तपसा प्रजास्तापयसेऽनघ}
{ब्रूहि कामं च मे शेष यस्ते हृदि व्यवस्थितः ॥शेष उवाच}
{}


\twolineshloka
{सोदर्या मम सर्वे हि भ्रातरो मन्दचेतसः}
{सह तैर्नोत्सहे वस्तुं तद्भवाननुमन्यताम्}


\twolineshloka
{अभ्यसूयन्ति सततं परस्परममित्रवत्}
{ततोऽहं तप आतिष्ठे नैतन्पश्येयमित्युत}


\twolineshloka
{न मर्षयन्ति ससुतां सततं विनतां च ते}
{अस्माकं चापरो भ्राता वैनतेयोऽन्तरिक्षगः}


\twolineshloka
{तं च द्विषन्ति सततं स चापि बलवत्तरः}
{वरप्रदानात्स पितुः कश्यपस्य महात्मनः}


\twolineshloka
{सोऽहं तपः समास्थाय मोक्ष्यामीदं कलेवरम्}
{कथं मे प्रेत्यभावेऽपि न तैः स्यात्सह सङ्गमः}


\twolineshloka
{तमेवं वादिनं शेषं पितामह उवाच ह}
{जानामि शेष सर्वेषां भ्रातॄणां ते विचेष्टितम्}


\twolineshloka
{मातुश्चाप्यपराधाद्वै भ्रातॄणां ते महद्भयम्}
{कृतोऽत्र परिहारश्च पूर्वमेव भुजङ्गम}


\twolineshloka
{भ्रातॄणां तव सर्वेषां न शोकं कर्तुमर्हसि}
{वृणीष्व च वरं मत्तः शेष यत्तेऽभिकाङ्क्षितम्}


\fourlineindentedshloka
{दास्यामि हि वरं तेऽद्य प्रीतिर्मे परमा त्वयि}
{दिष्ट्या बुद्धिश्च ते धर्मे निविष्टा पन्नगोत्तम}
{भूयो भूयश्च ते बुद्धिर्धर्मे भवतु सुस्थिरा ॥शेष उवाच}
{}


\threelineshloka
{एष एव वरो देव काङ्क्षितो मे पितामह}
{धर्मे मे रमतां बुद्धिः शमे तपसि चेश्वर ॥ब्रह्मोवाच}
{}


\twolineshloka
{प्रीतोऽस्म्यनेन ते शेष दमेन च शमेन च}
{त्वया त्विदं वचः कार्यं मन्नियोगात्प्रजाहितम्}


\twolineshloka
{इमां महीं शैलवनोपपन्नांससागरग्रामविहारपत्तनाम्त्वं शेष सम्यक् चलितां यथाव-त्संगृह्य तिष्ठस्व यथाऽचला स्यात् ॥शेष उवाच}
{}


\threelineshloka
{यथाऽऽह देवो वरदः प्रजापति-र्महीपतिर्भूतपतिर्जगत्पतिः}
{तथा महीं धारयिताऽस्मि निश्चलांप्रयच्छतां मे विवरं प्रजापते ॥ब्रह्मोवाच}
{}


\threelineshloka
{अधो महीं गच्छ भुजङ्गमोत्तमस्वयं तवैषा विवरं प्रदास्यति}
{इमां धरां धारयता त्वया हि मेमहत्प्रियं शेष कृतं भविष्यति ॥सौतिरुवाच}
{}


\threelineshloka
{तथैव कृत्वा विवरं प्रविश्य सप्रभुर्भुवो भुजगवराग्रजः स्थितः}
{बिभर्ति देवीं शिरसा महीमिमांसमुद्रनेमिं परिगृह्य सर्वतः ॥ब्रह्मोवाच}
{}


\threelineshloka
{शेषोऽसि नागोत्तम धर्मदेवोमहीमिमां धारयसे यदेकः}
{अनन्तभोगैः परिगृह्य सर्वांयथाहमेवं बलभिद्यथा वा ॥सौतिरुवाच}
{}


\twolineshloka
{अधो भूमौ वसत्येवं नागोऽनन्तः प्रतापवान्}
{धास्यन्वसुधामेकः शासनाद्ब्रह्मणो विभोः}


\twolineshloka
{सुपर्णं च सहायं वै भगवानमरोत्तमः}
{प्रादादनन्ताय तदा वैनतेयं पितामहः}


\twolineshloka
{`अनन्तेऽभिप्रयाते तु वासुकिः स महाबलः}
{अभ्यषिच्यत नागैस्तु दैवतैरिव वासवः ॥'}


\chapter{अध्यायः ३७}
\twolineshloka
{सौतिरुवाच}
{}


\twolineshloka
{मातुः सकाशात्तं शापं श्रुत्वा वै पन्नगोत्तमः}
{वासुकिश्चिन्तयामास शापोऽयं न भवेत्कथम्}


\threelineshloka
{ततः स मन्त्रयामास भ्रातृभिः सह सर्वशः}
{ऐरावतप्रभृतिभिः सर्वैर्धर्मपरायणैः ॥वासुकिरुवाच}
{}


\twolineshloka
{अयं शापो यथोद्दिष्टो विदितं वस्तथाऽनघाः}
{तस्य शापस्य मोक्षार्थं मन्त्रयित्वा यतामहे}


\twolineshloka
{सर्वेषामेव शापानां प्रतिघातो हि विद्यते}
{न तु मात्राऽभिशप्तानां मोक्षः क्वचन विद्यते}


\twolineshloka
{अव्ययस्याप्रमेयस्य सत्यस्य च तथाग्रतः}
{शप्ता इत्येव मे श्रुत्वा जायते हृदि वेपथुः}


\threelineshloka
{नूनं सर्वविनाशोऽयमस्माकं समुपागतः}
{`शापः सृष्टो महाघोरो मात्रा खल्वविनीतया}
{न ह्येतां सोऽव्ययो देवः शपत्नीं प्रत्यषेधयत्}


\twolineshloka
{तस्मात्संमन्त्रयामोऽद्य भुजङ्गानामनामयम्}
{यथा भवेद्धि सर्वेषां मा नः कालोऽत्यगादयम्}


\twolineshloka
{सर्व एव हि नस्तावद्बुद्धिमन्तो विचक्षणाः}
{अपि मन्त्रयमाणा हि हेतुं पश्याम मोक्षणे}


\fourlineindentedshloka
{यथा नष्टं पुरा देवा गूढमग्निं गुहागतम्}
{यथा स यज्ञो न भवेद्यथा वाऽपि पराभवः}
{जनमेजयस्य सर्पाणां विनाशकरणाय वै ॥सौतिरुवाच}
{}


\twolineshloka
{तथेत्युक्त्वा ततः सर्वे काद्रवेयाः समागताः}
{समयं चकिरे तत्र मन्त्रबुद्धिविशारदाः}


\twolineshloka
{एके तत्राब्रुवन्नागा वयं भूत्वा द्विजर्षभाः}
{जनमेजयं तु भिक्षामो यज्ञस्ते न भवेदिति}


\twolineshloka
{अपरे त्वब्रुवन्नागास्तत्र पण्डितमानिनः}
{मन्त्रिणोऽस्य वयं सर्वे भविष्यामः सुसंमताः}


\twolineshloka
{स नः प्रक्ष्यति सर्वेषु कार्येष्वर्थविनिश्चयम्}
{तत्र बुद्धिं प्रदास्यामो यथा यज्ञो निवर्त्स्यति}


\twolineshloka
{स नो बहुमतान्राजा बुद्ध्या बुद्धिमतां वरः}
{यज्ञार्थं प्रक्ष्यति व्यक्तं नेति वक्ष्यामहे वयम्}


\twolineshloka
{दर्शयन्तो बहून्दोषान्प्रेत्य चेह च दारुणान्}
{हेतुभिः कारणैश्चैव यथा यज्ञो भवेन्न सः}


\twolineshloka
{अथवा य उपाध्यायः क्रतोस्तस्य भविष्यति}
{सर्पसत्रविधानज्ञो राजकार्यहिते रतः}


\twolineshloka
{तं गत्वा दशतां कश्चिद्भुजङ्गः स मरिष्यति}
{तस्मिन्मृते यज्ञकारे क्रतुः स न भविष्यति}


\twolineshloka
{ये चान्ये सर्पसत्रज्ञा भविष्यन्त्यस्य चर्त्विजः}
{तांश्च सर्वान्दशिष्यामः कृतमेवं भविष्यति}


\twolineshloka
{अपरे त्वब्रुवन्नागा धर्मात्मानो दयालवः}
{अबुद्धिरेषा भवतां ब्रह्महत्या न शोभनम्}


\twolineshloka
{सम्यक्सद्धर्ममूला वै व्यसने शान्तिरुत्तमा}
{अधर्मोत्तरता नाम कृत्स्नं व्यापादयेज्जगत्}


\twolineshloka
{अपरे त्वब्रुवन्नागाः समिद्धं जातवेदसम्}
{वर्षैर्निर्वापयिष्यामो मेघा भूत्वा सविद्युतः}


\twolineshloka
{स्रुग्भाण्डं निशि गत्वा च अपरे भुजगोत्तमाः}
{प्रमत्तानां हरन्त्वाशु विघ्न एवं भविष्यति}


\twolineshloka
{यज्ञे वा भुजगास्तस्मिञ्शतशोऽथ सहस्रशः}
{जनान्दशन्तु वै सर्वे नैवं त्रासो भविष्यति}


\twolineshloka
{अथवा संस्कृतं भोज्यं दूषयन्तु भुजङ्गमाः}
{स्वेन मूत्रपुरीषेण सर्वभोज्यविनाशिना}


\twolineshloka
{अपरे त्वब्रुवंस्तत्र ऋत्विजोऽस्य भवामहे}
{यज्ञविघ्नं करिष्यामो दक्षिणा दीयतामिति}


\twolineshloka
{वश्यतां च गतोऽसौ नः करिष्यति यथेप्सितम्}
{अपरे त्वब्रुवंस्तत्र जले प्रक्रीडितं नृपम्}


\twolineshloka
{गृहमानीय बध्नीमः क्रतुरेवं भवेन्न सः}
{अपरे त्वब्रुवंस्तत्र नागाः पण्डितमानिनः}


\twolineshloka
{दशामस्तं प्रगृह्याशु कृतपेवं भविष्यति}
{छिन्नं मूलमनर्थानां मृते तस्मिन्भविष्यति}


\twolineshloka
{एषा नो नैष्ठिकी बुद्धिः सर्वेषामीक्षणश्रवः}
{अथ यन्मन्यसे राजन्द्रुतं तत्संविधीयताम्}


\twolineshloka
{इत्युक्त्वा समुदैक्षन्त वासुकिं पन्नगोत्तमम्}
{वासुकिश्चापि संचिन्त्य तानुवाच भुजङ्गमान्}


\twolineshloka
{नैषा वो नैष्ठिकी बुद्धिर्मता कर्तुं भुजङ्गमाः}
{सर्वेषामेव मे बुद्धिः पन्नगानां न रोचते}


\twolineshloka
{किं तत्र संविधातव्यं भवतां स्याद्धितं तु यत्}
{श्रेयः प्रसादनं मन्ये कश्यपश्य महात्मनः}


\twolineshloka
{ज्ञातिवर्गस्य सौहार्दादात्मनश्च भुजङ्गमाः}
{न च जानाति मे बुद्धिः किंचित्कर्तुं वचो हिवः}


\twolineshloka
{मया हीदं विधातव्यं भवतां यद्धितं भवेत्}
{अनेनाहं भृशं तप्ये गुणदोषौ मदाश्रयौ}


\chapter{अध्यायः ३८}
\twolineshloka
{सौतिरुवाच}
{}


\twolineshloka
{सर्पाणां तु वचः श्रुत्वा सर्वेषामिति चेति च}
{वासुकेश्च वचः श्रुत्वा एलापत्रोऽब्रवीदिदम्}


\twolineshloka
{`प्रागेव दर्शिता बुद्धिर्मयैषा भुजगोत्तमाः}
{हेयेति यदि वो बुद्धिस्तवापि च तथा प्रभो}


\twolineshloka
{अस्तु कामं मभाद्यापि बुद्धिः स्मरणमागता}
{तां शृणुध्वं प्रवक्ष्यामि याथातथ्येन पन्नगाः ॥'}


\twolineshloka
{न स यज्ञो न भविता न स राजा तथाविधः}
{जनमेजयः पाण्डवेयो यतोऽस्माकं महद्भयम्}


\twolineshloka
{दैवेनोपहतो राजन्यो भवेदिह पूरुषः}
{स दैवमेवाश्रयेत नान्यत्तत्र परायणम्}


\twolineshloka
{तदिदं चैवमस्माकं भयं पन्नगसत्तमाः}
{दैवमेवाश्रयामोऽत्र शृणुध्वं च वचो मम}


\threelineshloka
{अहं शापे समुत्सृष्टे समश्रौषं वचस्तदा}
{मातुरुत्सङ्गमारूढो भयात्पन्नगसत्तमाः}
{देवानां पन्नगश्रेष्ठास्तीक्ष्णास्तीक्ष्ण इति प्रभो}


\threelineshloka
{`शापदुःखाग्नितप्तानां पन्नगानामनामयम्}
{कृपया परयाऽऽविष्टाः प्रार्थयन्तो दिवौकसः ॥'देवा ऊचुः}
{}


\twolineshloka
{का हि लब्ध्वा प्रियान्पुत्राञ्शपेदेवं पितामह}
{ऋते कद्रूं तीक्ष्णरूपां देवदेव तवाग्रतः}


\threelineshloka
{तथेति च वचस्तस्यास्त्वयाप्युक्तं पितामह}
{इच्छाम एतद्विज्ञातुं कारणं यन्न वारिता ॥ब्रह्मोवाच}
{}


\twolineshloka
{बहवः पन्नगास्तीक्ष्णा घोररूपा विषोल्बणाः}
{प्रजानां हितकामोऽहं न च वारितवांस्तदा}


\twolineshloka
{ये दन्दशूकाः क्षुद्राश्च पापाचारा विषोल्बणाः}
{तेषां विनाशो भविता न तु ये धर्मचारिणः}


\twolineshloka
{यन्निमित्तं च भविता मोक्षस्तेषां महाभयात्}
{पन्नगानां निबोधध्वं तस्मिन्काले समागते}


\twolineshloka
{यायावरकुले धीमान्भविष्यति महानृषिः}
{जरत्कारुरिति ख्यातस्तपस्वी नियतेन्द्रियः}


\fourlineindentedshloka
{तस्य पुत्रो जरत्कारोर्भविष्यति तपोधनः}
{आस्तीको नाम यज्ञं स प्रतिषेत्स्यति तं तदा}
{तत्र मोक्ष्यन्ति भुजगा ये भविष्यन्ति धार्मिकाः ॥देवा ऊचुः}
{}


\threelineshloka
{स मुनिप्रवरो ब्रह्मञ्जरत्कारुर्महातपाः}
{कस्यां पुत्रं महात्मानं जनयिष्यति वीर्यवान् ॥ब्रह्मोवाच}
{}


\twolineshloka
{`वासुकेर्भगिनी कन्या समुत्पन्ना सुशोभना}
{तस्मै दास्यति तां कन्यां वासुकिर्भुजगोत्तमः}


\threelineshloka
{तस्यां जनयिता पुत्रं वेदवेदाङ्गपारगम्}
{'सनामायां सनामा स कन्यायां द्विजसत्तमः ॥एलापत्र उवाच}
{}


\twolineshloka
{एवमस्त्विति तं देवाः पितामहमथाब्रुवन्}
{उत्क्वैवं वचनं देवान्विरिञ्चिस्त्रिदिवं ययौ}


\twolineshloka
{सोऽहमेवं प्रपश्यामि वासुके भगिनी तव}
{जरत्कारुरिति ख्याता तां तस्मै प्रतिपादय}


\twolineshloka
{भैक्षवद्भिक्षमाणाय नागानां भयशान्तये}
{ऋषये सुव्रतायैनामेष मोक्षः श्रुतो मया}


\chapter{अध्यायः ३९}
\twolineshloka
{सौतिरुवाच}
{}


\twolineshloka
{एलापत्रवचः श्रुत्वा ते नागा द्विजसत्तम}
{सर्वे प्रहृष्टमनसः साधुसाध्वित्यपूजयन्}


\twolineshloka
{ततःप्रभृति तां कन्यां वासुकिः पर्यरक्षत}
{जरत्कारुं स्वसारं वै परं हर्षमवाप च}


\twolineshloka
{ततो नातिमहान्कालः समतीत इवाभवत्}
{अथ देवासुराः `सर्वे ममन्थुर्वरुणालयम्}


\twolineshloka
{तत्र नेत्रमभून्नागो वासुकिर्बलिनां वरः}
{समाप्यैव च तत्कर्म पितामहमुपागमन्}


\twolineshloka
{देवा वासुकिना सार्धं पितामहमथाव्रुवन्}
{भगवञ्शापभीतोऽयं वासुकिस्तप्यते भृशम्}


\twolineshloka
{अस्यैतन्मानसं शल्यं समुद्धर्तुं त्वमर्हसि}
{जनन्याः शापजं देव ज्ञातीनां हितमिच्छतः}


\threelineshloka
{हितो ह्ययं सदास्मकं प्रियकारी च नागराट्}
{प्रसादं कुरु देवेश शमयास्य मनोज्वरम् ॥ब्रह्मोवाच}
{}


\twolineshloka
{मयैव तद्वितीर्णं वै वचनं मनसाऽमराः}
{एलापत्रेण नागेन यदस्याभिहितं पुरा}


\twolineshloka
{तत्करोत्वेष नागेन्द्रः प्राप्तकालं वचः स्वयम्}
{विनशिष्यन्ति ये पापा न तु ये धर्मचारिणः}


\twolineshloka
{उत्पन्नः स जरत्कारुस्तपस्युग्रे रतो द्विजः}
{तस्यैष भगिनीं काले जरत्कारुं प्रयच्छतु}


\threelineshloka
{एलापत्रेण यत्प्रोक्तं वचनं भुजगेन ह}
{पन्नगानां हितं देवास्तत्तथा न तदन्यथा ॥सौतिरुवाच}
{}


\twolineshloka
{एतच्छ्रुत्वा तु नागेन्द्रः पितामहवचस्तदा}
{संदिश्य पन्नगान्सर्वान्वासुकिः शापमोहितः}


\twolineshloka
{स्वसारमुद्यम्य तदा जरत्कारुमृषिं प्रति}
{सर्पान्बहूञ्जरत्कारौ नित्ययुक्तान्समादधत्}


\twolineshloka
{जरत्कारुर्यदा भार्यामिच्छेद्वरयितुं प्रभुः}
{शीघ्रमेत्य तदाऽऽख्येयं तन्नः श्रेयो भविष्यति}


\chapter{अध्यायः ४०}
\twolineshloka
{शौनक उवाच}
{}


\twolineshloka
{जरत्कारुरिति ख्यातो यस्त्वया सूतनन्दन}
{इच्छामि तदहं श्रोतुं ऋषेस्तस्य महात्मनः}


\threelineshloka
{किं कारणं जरत्कारोर्नामैतत्प्रथितं भुवि}
{जरत्कारुनिरुक्तिं त्वं यथावद्वक्तुमर्हसि ॥सौतिरुवाच}
{}


\twolineshloka
{जरेति क्षयमाहुर्वै दारुणं कारुसंज्ञितम्}
{शरीरं कारु तस्यासीत्तत्स धीमाञ्शनैःशनैः}


\twolineshloka
{क्षपयामास तीव्रेण तपसेत्यत उच्यते}
{जरत्कारुरिति ब्रह्मन्वासुकेर्भगिनी तथा}


\threelineshloka
{एवमुक्तस्तु धर्मात्मा शौनकः प्राहसत्तदा}
{उग्रश्रवसमामन्त्र्य उपपन्नमिति ब्रुवन् ॥शौनक उवाच}
{}


\fourlineindentedshloka
{उक्तं नाम यथा पूर्वं सर्वं तच्छ्रुतवानहम्}
{यथा तु जातो ह्यास्तीक एतदिच्छामि वेदितुम्}
{तच्छ्रुत्वा वचनं तस्य सौतिः प्रोवाच शास्त्रतः ॥सौतिरुवाच}
{}


\twolineshloka
{संदिश्य पन्नगान्सर्वान्वासुकिः सुसमाहितः}
{स्वसारमुद्यम्य तदा जरत्कारुमृषिं प्रति}


\twolineshloka
{अथ कालस्य महतः स मुनिः संशितव्रतः}
{तपस्यभिरतो धीमान्स दारान्नाभ्यकाङ्क्षत}


\twolineshloka
{स तूर्ध्वरेतास्तपसि प्रसक्तःस्वाध्यायवान्वीतभयः कृतात्मा}
{चचार सर्वां पृथिवीं महात्मान चापि दारान्मनसाध्यकाङ्क्षत्}


\twolineshloka
{ततोऽपरस्मिन्संप्राप्ते काले कस्मिंश्चिदेव तु}
{परिक्षिन्नाम राजासीद्ब्रह्मन्कौरववंशजः}


\twolineshloka
{यथा पाण्डुर्महाबाहुर्धनुर्धरवरो युधि}
{बभूव मृगयाशीलः पुरास्य प्रपितामहः}


\threelineshloka
{`तथा विख्यातवाँल्लोके परीक्षिदभिमन्युजः}
{'मृगान्विध्यन्वराहांश्च तरक्षून्महिषांस्तथा}
{अन्यांश्च विविधान्वन्यांश्चचार पृथिवीपतिः}


\twolineshloka
{स कदाचिन्मृगं विद्ध्वा बाणेनानतपर्वणा}
{पृष्ठतो धनुरादाय ससार गहने वने}


\twolineshloka
{यथैव भगवान्रुद्रो विद्ध्वा यज्ञमृगं दिवि}
{अन्वगच्छद्धनुष्पाणिः पर्यन्वेष्टुमितस्ततः}


\twolineshloka
{न हि तेन मृगो विद्धो जीवन्गच्छति वै वने}
{पूर्वरूपं तु तत्तूर्णं तस्यासीत्स्वर्गतिं प्रति}


\twolineshloka
{परिक्षितो नरेन्द्रस्य विद्धो यन्नष्टवान्मृगः}
{दूरं चापहृतस्तेन मृगेण स महीपतिः}


\twolineshloka
{परिश्रान्तः पिपासार्त आससाद मुनिं वने}
{गवां प्रचारेष्वासीनं वत्सानां मुखनिःसृतम्}


\twolineshloka
{भूयिष्ठमुपयुञ्जानं फेनमापिबतां पयः}
{तमभिद्रुत्य वेगेन स राजा संशितव्रतम्}


\twolineshloka
{अपृच्छद्धनुरुद्यम्य तं मुनिं क्षुच्छ्रमान्वितः}
{भोभो ब्रह्मन्नहं राजा परीक्षिदभिमन्युजः}


\twolineshloka
{मया विद्धो मृगो नष्टः कच्चित्तं दृष्टवानसि}
{स मुनिस्तं तु नोवाच किंचिन्मौनव्रते स्थितः}


\twolineshloka
{तस्य स्कन्धे मृतं सर्पं क्रुद्धो राजा समासजत्}
{समुत्क्षिप्य धनुष्कोट्या स चैनं समुपैक्षत}


\threelineshloka
{न स किंचिदुवाचैनं शुभं वा यदि वाऽशुभम्}
{स राजा क्रोधमुत्सृज्य व्यथितस्तं तथागतम्}
{दृष्ट्वा जगाम नगरमृषिस्त्वासीत्तथैव सः}


\twolineshloka
{न हि तं राजशार्दूलं क्षमाशीलो महामुनिः}
{स्वधर्मनिरतं भूपं समाक्षिप्तोऽप्यधर्षयत्}


\twolineshloka
{न हि तं राजशार्दूलस्तथा धर्मपरायणम्}
{जानाति भरतश्रेष्ठस्तत एनमधर्षयत्}


\twolineshloka
{तरुणस्तस्य पुत्रोऽभूत्तिग्मतेजा महातपाः}
{शृङ्गी नाम महाक्रोधो दुष्पसादोमहाव्रतः}


\twolineshloka
{स देवं परमासीनं सर्वभूतहिते रतम्}
{ब्रह्माणमुपतस्थे वै काले काले सुंसयतः}


\twolineshloka
{सतेन समनुज्ञातो ब्रह्मणा गृहमेयिवान्}
{सख्योक्तः क्रीडमानेन स तत्र हसता किल}


\fourlineindentedshloka
{संरम्भात्कोपनोऽतीव विषकल्पो मुनेः सुतः}
{उद्दिश्य पितरं तस्य यच्छ्रुत्वा रोषमाहरत्}
{ऋषिपुत्रेण नर्मार्थे कृशेन द्विजसत्तम ॥कृश उवाच}
{}


\twolineshloka
{तेजस्विनस्तव पिता तथैव च तपस्विनः}
{शवं स्कन्धेन वहति मा शृङ्गिन्गर्वितो भव}


\twolineshloka
{व्याहरत्स्वृषिपुत्रेषु मा स्म किंचिद्वचो वद}
{अस्मद्विधेषु सिद्धेषु ब्रह्मवित्सु तपस्विषु}


\twolineshloka
{क्व ते पुरुषमानित्वं क्व ते वाचस्तथाविधाः}
{दर्पजाः पितरं द्रष्टा यस्त्वं शवधरं तथा}


\twolineshloka
{पित्रा च तव तत्कर्म नानुरूपमिवात्मनः}
{कृतं मुनिजनश्रेष्ठ येनाहं भृशदुःखितः}


\chapter{अध्यायः ४१}
\twolineshloka
{सौतिरुवाच}
{}


\twolineshloka
{एवमुक्तः स तेजस्वी शृङ्गी कोपसमन्वितः}
{मृतधारं गुरुं श्रुत्वा पर्यतप्यत मन्युना}


\twolineshloka
{स त कृशमक्षिप्रेक्ष्य सूनृतां वाचमुत्सृजन्}
{अपृच्छत्तं कथं तातः `सर्वभूतहिते रतः}


\fourlineindentedshloka
{अनन्यचेताः सततं विष्णुं दवेमतोषयत्}
{वन्यान्नभोजी सततं मुनिर्मौनव्रते स्थितः}
{एवंभूतः स तेजस्वी' स मेऽद्य मृतधारकः ॥कृश उवाच}
{}


\threelineshloka
{राज्ञा परिक्षिता तात मृगयां परिधावता}
{अवसक्तः पितुस्तेऽद्य मृतः स्कन्धे भुजङ्गमः ॥शृङ्ग्युवाच}
{}


\threelineshloka
{किं मे पित्रा कृतं तस्य राज्ञोऽनिष्टं दुरात्मनः}
{ब्रूहि तत्कृश तत्त्वेन पश्य मे तपसो बलम् ॥कृश उवाच}
{}


\twolineshloka
{स राजा मृगयां यातः परिक्षिदभिमन्युजः}
{ससार मृगमेकाकी विद्ध्वा बाणेन शीघ्रगम्}


\twolineshloka
{न चापश्यन्मृगं राजा चरंस्तस्मिन्महावने}
{पितरं ते स दृष्ट्वैव पप्रच्छानभिभाषिणम्}


\twolineshloka
{तं स्थाणुभूतं तिष्ठन्तं क्षुत्पिपासाश्रमातुरः}
{पुनःपुनर्मृगं नष्टं प्रपच्छ पितरं तव}


\twolineshloka
{स च मौनव्रतोपेतो नैव तं प्रत्यभाषत}
{तस्य राजा धनुष्कोट्या सर्पं स्कन्धे समासजत्}


\threelineshloka
{शृङ्गिंस्तव पिता सोऽपि तथैवास्ते यतव्रतः}
{सोऽपि राजा स्वनगरं प्रस्थितो गजसाह्वयम् ॥सौतिरुवाच}
{}


\twolineshloka
{श्रुत्वैवमृषिपुत्रस्तु शवं स्कन्धे प्रतिष्ठितम्}
{कोपसंरक्तनयनः प्रज्वलन्निव मन्युना}


\threelineshloka
{आविष्टः स हि कोपेन शशाप नृपतिं तदा}
{वार्युपस्पृश्य तेजस्वी क्रोधवेगबलात्कृतः ॥शृङ्ग्युवाच}
{}


\twolineshloka
{योऽसौ वृद्धस्य तातस्य तथा कृच्छ्रगतस्य ह}
{स्कन्धे मृतं समास्राक्षीत्पन्नगं राजकिल्विषी}


\twolineshloka
{तं पापमतिसंक्रुद्धस्तक्षकः पन्नगेश्वरः}
{आशीविषस्तिग्मतेजा मद्वाक्यबलचोदितः}


\threelineshloka
{सप्तरात्रादितो नेता यमस्य सदनं प्रति}
{द्विजानामवमन्तारं कुरूणामयशस्करम् ॥सौतिरुवाच}
{}


\twolineshloka
{इति शप्त्वातिसंक्रुद्धः शृङ्गी पितरमभ्यगात्}
{आसीनं ग्रोव्रजे तस्मिन्वहन्तं शवपन्नगम्}


\twolineshloka
{स तमलक्ष्य पितरं शृङ्गी स्कन्धगतेन वै}
{शवेन भुजगेनासीद्भूयः क्रोधसमाकुलः}


\twolineshloka
{दुःखाच्चाश्रूणि मुमुचे पितरं चेदमब्रवीत्}
{श्रुत्वेमां धर्षणां तात तव तेन दुरात्मना}


\threelineshloka
{राज्ञा परिक्षिता कोपादशपं तमहं नृपम्}
{यथार्हति स एवोग्रं शापं कुरुकुलाधमः}
{सप्तमेऽहनि तं पापं तक्षकः पन्नगोत्तमः}


\fourlineindentedshloka
{वैवस्वतस्य सदनं नेता परमदारुणम्}
{सौतिरुवाच}
{तमब्रवीत्पिता ब्रह्मंस्तथा कोपसमन्वितम् ॥शमीक उवाच}
{}


\twolineshloka
{न मे प्रियं कृतं तात नैष धर्मस्तपस्विनाम्}
{वयं तस्य नरेन्द्रस्य विषये निवसामहे}


\twolineshloka
{न्यायतो रक्षितास्तेन तस्य पापं न रोचये}
{सर्वथा वर्तमानस्य राज्ञो ह्यस्मद्विधैः सदा}


\twolineshloka
{क्षन्तव्यं पुत्र धर्मो हि हतो हन्ति न संशयः}
{यदि राजा न संरक्षेत्पीडा नः परमा भवेत्}


\twolineshloka
{न शक्नुयाम चरितुं धर्मं पुत्र यथासुखम्}
{रक्षमाणा वयं तात राजभिर्धर्मदृष्टिभिः}


\twolineshloka
{चरामो विपुलं धर्मं तेषां भागोऽस्ति धर्मतः}
{सर्वथा वर्तमानस्य राज्ञः क्षन्तव्यमेव हि}


\twolineshloka
{परिक्षित्तु विशेषेण यथाऽस्य प्रपितामहः}
{रक्षत्यस्मांस्तथा राज्ञा रक्षितव्याः प्रजा विभो}


\twolineshloka
{अराजके जनपदे दोषा जायन्ति वै सदा}
{उद्वृत्तं सततं लोकं राजा दण्डेन शास्ति वै}


\twolineshloka
{दण्डात्प्रतिभयं भूयः शान्तिरुत्पद्यते तदा}
{नोद्विग्नश्चरते धर्मं नोद्विग्नश्चरते क्रियाम्}


\twolineshloka
{राज्ञा प्रतिष्ठितो धर्मो धर्मात्स्वर्गः प्रतिष्ठितः}
{राज्ञो यज्ञक्रियाः सर्वा यज्ञाद्देवाः प्रतिष्ठिताः}


\twolineshloka
{देवाद्वृष्टिः प्रवर्तेत वृष्टेरोषधयः स्मृताः}
{ओषधिभ्यो मनुष्याणां धारयन्सततं हितम्}


\twolineshloka
{मनुष्याणां च यो धाता राजा राज्यकरः पुनः}
{दशश्रोत्रियसमो राजा इत्येवं मनुरब्रवीत्}


\twolineshloka
{तेनेह क्षुधितेनैत्य श्रान्तेन मृगलिप्सुना}
{अजानता कृतं मन्ये व्रतमेतदिदं मम}


\twolineshloka
{कस्मादिदं त्वया बाल्यात्सहसा दुष्कृतं कृतम्}
{न ह्यर्हति नृपः शापमस्मत्तः पुत्र सर्वथा}


\chapter{अध्यायः ४२}
\twolineshloka
{शृङ्ग्युवाच}
{}


\twolineshloka
{यद्येतत्साहसं तात यदि वा दुष्कृतं कृतम्}
{प्रियं वाप्यप्रियं वा ते वागुक्ता न मृषा भवेत्}


\threelineshloka
{नैवान्यथेदं भविता पितरेष ब्रवीमि ते}
{नाहं मृषा ब्रवीम्येवं स्वैरेष्वपि कुतः शपन् ॥शमीक उवाच}
{}


\twolineshloka
{जानाम्युग्रप्रभावं त्वां तात सत्यगिरं तथा}
{नानृतं चोक्तपूर्वं ते नैतन्मिथ्या भविष्यति}


\twolineshloka
{पित्रा पुत्रो वयस्थोऽपि सततं वाच्य एव तु}
{यथा स्याद्गुणसंयुक्तः प्राप्नुयाच्च महद्यशः}


\twolineshloka
{किं पुनर्बाल एव त्वं तपसा भावितः सदा}
{वर्धते चेत्प्रभवतां कोपोऽतीव महात्मनाम्}


\threelineshloka
{`उत्सीदेयुरिमे लोकाः क्षणा चास्य प्रतिक्रिया}
{'सोऽहं पश्यामि वक्तव्यं त्वयि धर्मभृतां वर}
{पुत्रत्वं बालतां चैव तवावेक्ष्य च साहसम्}


\twolineshloka
{स त्वं शमपरो भूत्वा वन्यमाहारमाचरन्}
{चर क्रोधमिमं हित्वा नैवं धर्मं प्रहास्यसि}


\twolineshloka
{क्रोधो हि धर्मं हरति यतीनां दुःखसंचितम्}
{ततो धर्मविहीनानां गतिरिष्टा न विद्यते}


\twolineshloka
{शम एव यतीनां हि क्षमिणां सिद्धिकारकः}
{क्षमावतामयं लोकः परश्चैव क्षमावताम्}


\twolineshloka
{तस्माच्चरेथाः सततं क्षमाशीलो जितेन्द्रियः}
{क्षमया प्राप्स्यसे लोकान्ब्रह्मणः समनन्तरान्}


\twolineshloka
{मया तु शममास्थाय यच्छक्यं कर्तुमद्य वै}
{तत्करिष्याम्यहं तात प्रेपयिष्ये नृपाय वै}


\threelineshloka
{मम पुत्रेण शप्तोऽसि बालेनाकृतबुद्धिना}
{ममेमां धर्षणां त्वत्तः प्रेक्ष्य राजन्नमर्षिणा ॥सौतिरुवाच}
{}


\twolineshloka
{एवमादिश्य शिष्यं स प्रेषयामास सुव्रतः}
{परिक्षिते नृपतये दयापन्नो महातपाः}


\twolineshloka
{संदिश्य कुशलप्रश्नं कार्यवृत्तान्तमेव च}
{शिष्यं गौरमुखं नाम शीलवन्तं समाहितम्}


\twolineshloka
{सोऽभिगम्य ततः शीघ्रं नरेन्द्रं कुरुवर्धनम्}
{विवेश भवनं राज्ञः पूर्वं द्वास्थैर्निवेदितः}


\twolineshloka
{पूजितस्तु नरेन्द्रेण द्विजो गौरमुखस्तदा}
{आचख्यौ च परिश्रान्तो राज्ञः सर्वमशेषतः}


\twolineshloka
{शमीकवचनं घोरं यथोक्तं मन्त्रिसन्निधौ}
{शमीको नाम राजेन्द्र वर्तते विषये तव}


\twolineshloka
{ऋषिः परमधर्मात्मा दान्तः शान्तो महातपाः}
{तस्य त्वया नरव्याघ्र सर्पः प्राणैर्वियोजितः}


\twolineshloka
{अवसक्तो धनुष्कोट्या स्कन्धे मौनान्वितस्य च}
{क्षान्तवांस्तव तत्कर्म पुत्रस्तस्य न चक्षमे}


\twolineshloka
{तेन शप्तोऽसि राजेन्द्र पितुरज्ञातमद्य वै}
{तक्षकः सप्तरात्रेण मृत्युस्तव भविष्यति}


\twolineshloka
{तत्र रक्षां कुरुष्वेति पुनः पुनरथाब्रवीत्}
{तदन्यथा न शक्यं च कर्तुं केनचिदप्युत}


\threelineshloka
{न हि शक्नोति संयन्तुं पुत्रं कोपसमन्वितम्}
{ततोऽहं प्रेषितस्तेन तव राजन्हितार्थिना ॥सातिरुवाच}
{}


\twolineshloka
{इति श्रुत्वा वचो घोरं स राजा कुरुनन्दनः}
{पर्यतप्यत तत्पापं कृत्वा राजा महातपाः}


\twolineshloka
{तं च मौनव्रतं श्रुत्वा वने मुनिवरं तदा}
{भूय एवाभवद्राजा शोकसंतप्तमानसः}


\twolineshloka
{अनुक्रोशात्मतां तस्य शमीकस्यावधार्य च}
{पर्यतप्यत भूयोपि कृत्वा तत्किल्बिषं मुनेः}


\twolineshloka
{न हि मृत्युं तथा राजा श्रुत्वा वै सोऽन्वतप्यत}
{अशोचदमरप्रख्यो यथा कृत्वेह कर्म तत्}


\twolineshloka
{ततस्तं प्रेषयामास राजा गौरमुखं तदा}
{भूयः प्रसादं भगवान्करोत्विह ममेति वै}


\twolineshloka
{`श्रुत्वा तु वचनं राज्ञो मुनिर्गौरमुखस्तदा}
{समनुज्ञाप्य वेगेन प्रजगामाश्रमं गुरोः ॥'}


\twolineshloka
{तस्मिंश्च गतमात्रेऽथ राजा गौरमुखे तदा}
{मन्त्रिभिर्मन्त्रयामास सह संविग्नमानसः}


\twolineshloka
{संमन्त्र्य मन्त्रिभिश्चैव स तथा मन्त्रतत्त्ववित्}
{प्रासादं कारयामास एकस्तम्भं सुरक्षितम्}


\twolineshloka
{रक्षां च विदधे तत्र भिषजश्चौषधानि च}
{ब्राह्मणान्मन्त्रसिद्धांश्च सर्वतो वै न्ययोजयत्}


\twolineshloka
{राजकार्याणि तत्रस्थः सर्वाण्येवाकरोच्च सः}
{मन्त्रिभिः सह धर्मज्ञः समन्तात्परिरक्षितः}


\twolineshloka
{न चैनं कश्चिदारूढं लभते राजसत्तमम्}
{वातोऽपि निश्चरंस्तत्र प्रवेशे विनिवार्यते}


\twolineshloka
{प्राप्ते च दिवसे तस्मिन्सप्तमे द्विजसत्तमः}
{काश्यपोऽभ्यागमद्विद्वांस्तं राजानं चिकित्सितुम्}


\twolineshloka
{श्रुतं हि तेन तदभूद्यथा तं राजसत्तमम्}
{तक्षकः पन्नगश्रेष्ठो नेष्यते यमसादनम्}


\twolineshloka
{तं दष्टं पन्नगेन्द्रेण करिष्येऽहमपज्वरम्}
{तत्र मेऽर्थश्च धर्मश्च भवितेति विचिन्तयन्}


\twolineshloka
{तं ददर्श स नागेन्द्रस्तक्षकः काश्यपं पथि}
{गच्छन्तमेकमनसं द्विजो भूत्वा वयोऽतिगः}


\threelineshloka
{तमब्रवीत्पन्नगेन्द्रः काश्यपं मुनिपुंगवम्}
{क्व भवांस्त्वरितो याति किंच कार्यं चिकीर्षति ॥काश्यप उवाच}
{}


\twolineshloka
{नृपं कुरुकुलोत्पन्नं परिक्षितमरिंदमम्}
{तक्षकः पन्नगश्रेष्ठस्तेजसाऽध्य प्रधक्ष्यति}


\threelineshloka
{तं दष्टं पन्नगेन्द्रेण तेनाग्निसमतेजसम्}
{पाण्डवानां कुलकरं राजानममितौजसम्}
{1-42-4oc गच्छामित्वरितंसौम्य सद्यः कर्तुमपज्वरम्}


\threelineshloka
{`विज्ञातविषविद्योऽहं ब्राह्मणो लोकपूजितः}
{अस्मद्गुरुकटाक्षेण कल्योऽहं विषनाशने ॥'तक्षक उवाच}
{}


\threelineshloka
{अहं स तक्षको ब्रह्मंस्तं धक्ष्यामि महीपतिम्}
{निवर्तस्व न शक्तस्त्वं मया दष्टं चिकित्सितुम् ॥काश्यप उवाच}
{}


\twolineshloka
{अहं तं नृपतिं गत्वा त्वया दष्टमपज्वरम्}
{करिष्यामीति मे बुद्धिर्विद्याबलसमन्विता}


\chapter{अध्यायः ४३}
\twolineshloka
{तक्षक उवाच}
{}


\twolineshloka
{यदि दष्टं मयेह त्वं शक्तः किंचिच्चिकित्सितुम्}
{ततो वृक्षं मया दष्टमिमं जीवय काश्यप}


\threelineshloka
{परं मन्त्रबलं यत्ते तद्दर्शय यतस्व च}
{न्यग्रोधमेनं धक्ष्यामि पश्यतस्ते द्विजोत्तम ॥काश्यप उवाच}
{}


\fourlineindentedshloka
{दश नागेन्द्र वृक्षं त्वं यद्येतदभिमन्यसे}
{अहमेनं त्वया दष्टं जीवयिष्ये भुजंगम}
{`पश्य मन्त्रबलं मेऽद्य न्यग्रोधं दश पन्नग ॥' सौतिरुवाच}
{}


\twolineshloka
{एवमुक्तः स नागेन्द्रः काश्यपेन महात्मना}
{अदशद्वृक्षमभ्येत्य न्यग्रोधं पन्नगोत्तमः}


\twolineshloka
{स वृक्षस्तेन दष्टस्तु पन्नगेन महात्मना}
{आशीविषविषोपेतः प्रजज्वाल समन्ततः}


\threelineshloka
{तं दग्ध्वा स नगं नागः काश्यपं पुनरब्रवीत्}
{कुरु यत्नं द्विजश्रेष्ठ जीवयैव वनस्पतिम् ॥सौतिरुवाच}
{}


\twolineshloka
{भस्मीभूतं ततो वृक्षं पन्नगेन्द्रस्य तेजसा}
{भस्म सर्वं समाहृत्य काश्यपो वाक्यमब्रवीत्}


\twolineshloka
{विद्याबलं पन्नगेन्द्र पश्य मेऽद्य वनस्पतौ}
{अहं संजीवयाम्येनं पश्यतस्ते भुजंगम}


\twolineshloka
{ततः स भगवान्विद्वान्काश्यपो द्विजसत्तमः}
{भस्मराशीकृतं वृक्षं विद्यया समजीवयत्}


\twolineshloka
{अङ्कुरं कृतवांस्तत्र ततः पर्णद्वयान्वितम्}
{पलाशिनं शाखिनं च तथा विटपिनं पुनः}


\twolineshloka
{तं दृष्ट्वा जीवितं वृक्षं काश्यपेन महात्मना}
{उवाच तक्षको ब्रह्मन्नैतदत्यद्भुतं त्वयि}


\twolineshloka
{द्विजेन्द्र यद्विषं हन्या मम वा मद्विधस्य वा}
{कं त्वमर्थमभिप्रेप्सुर्यासि तत्र तपोधन}


\twolineshloka
{यत्तेऽभिलषितं प्राप्तं फलं तस्मान्नृपोत्तमात्}
{अहमेव प्रदास्यामि तत्ते यद्यपि दुर्लभम्}


\twolineshloka
{विप्रशापाभिभूते च क्षीणायुषि नराधिपे}
{घटमानस्य ते विप्र सिद्धिः संशयिता भवेत्}


\threelineshloka
{ततो यशः प्रदीप्तं ते त्रिषु लोकेषु विश्रुतम्}
{निरंशुरिव घर्मांशुरन्तर्धानमितो व्रजेत् ॥काश्यप उवाच}
{}


\threelineshloka
{धनार्थी याम्यहं तत्र तन्मे देहि भुजंगम}
{ततोऽहं विनिवर्तिष्ये स्वापतेयं प्रगृह्य वै ॥तक्षक उवाच}
{}


\threelineshloka
{यावद्धनं प्रार्थयसे तस्माद्राज्ञस्ततोऽधिकम्}
{अहमेव प्रदास्यामि निवर्तस्व द्विजोत्तम ॥सौतिरुवाच}
{}


\twolineshloka
{तक्षकस्य वचः श्रुत्वा काश्यपो द्विजसत्तमः}
{प्रदध्यौ सुमहातेजाराजानं प्रति बुद्धिमान्}


\twolineshloka
{दिव्यज्ञानः स तेजस्वी ज्ञात्वा तं नृपतिं तदा}
{क्षीणायुषं पाण्डवेयमपावर्तत काश्यपः}


\twolineshloka
{लब्ध्वा वित्तं मुनिवरस्तक्षकाद्यावदीप्सितम्}
{निवृत्ते काश्यपे तस्मिन्समयेन महात्मनि}


\twolineshloka
{जगाम तक्षकस्तूर्णं नगरं नागसाह्वयम्}
{अथ शुश्राव गच्छन्स तक्षको जगतीपतिम्}


\threelineshloka
{मन्त्रैर्विषहरैर्दिव्यै रक्ष्यमाणं प्रयत्नतः}
{सौतिरुवाच}
{स चिन्तयामास तदा मायायोगेन पार्थिवः}


\twolineshloka
{मया वञ्चयितव्योऽसौ क उपायो भवेदिति}
{ततस्तापसरूपेण प्राहिणोत्स भुजंगमान्}


\threelineshloka
{फलपत्रोदकं गृह्य राज्ञे नागोऽथ तक्षकः}
{तक्षक उवाच}
{गच्छध्वं यूयमव्यग्रा राजानं कार्यवत्तया}


\threelineshloka
{फलपत्रोदकं नागाः प्रतिग्राहयितुं नृपम्}
{सौतिरुवाच}
{ते तक्षकसमादिष्टास्तथा चक्रुर्भुजङ्गमाः}


\twolineshloka
{उपनिन्युस्तथा राज्ञे दर्भानम्भः फलानि च}
{तच्च सर्वं स राजेन्द्रः प्रतिजग्राह वीर्यवान्}


\twolineshloka
{कृत्वा तेषां च कार्याणि गम्यतामित्युवाच तान्}
{गतेषु तेषु नागेषु तापसच्छद्मरूपिषु}


\twolineshloka
{अमात्यान्सुहृदश्चैव प्रोवाच स नराधिपः}
{भक्षयन्तु भवन्तो वै स्वादूनीमानि सर्वशः}


\twolineshloka
{तापसैरुपनीतानि फलानि सहिता मया}
{ततो राजा ससचिवः फलान्यादातुमैच्छत}


\twolineshloka
{विधिना संप्रयुक्तो वै ऋषिवाक्येन तेन तु}
{यस्मिन्नेव फले नागस्तमेवाभक्षयत्स्वयम्}


\twolineshloka
{ततो भक्षयतस्तस्य फलात्कृमिरभूदणुः}
{ह्रस्वकः कृष्णनयनस्ताम्रवर्णोऽथ शौनक}


\twolineshloka
{स तं गृह्य नृपश्रेष्ठः सचिवानिदमब्रवीत्}
{अस्तमभ्येति सविता विषादद्य न मे भयम्}


\twolineshloka
{सत्यवागस्तु स मुनिः कृमिर्मां दशतामयम्}
{तक्षको नाम भूत्वा वै तथा परिहृतं भवेत्}


\twolineshloka
{ते चैनमन्ववर्तन्त मन्त्रिणः कालचोदिताः}
{एवमुक्त्वा स राजेन्द्रो ग्रीवायां सन्निवेश्यह}


\twolineshloka
{कृमिकं प्राहसत्तूर्णं मुमूर्षुर्नष्टचेतनः}
{प्रहसन्नेव भोगेन तक्षकेणाभिवेष्टितः}


\threelineshloka
{तस्मात्फलाद्विनिष्क्रम्य यत्तद्राज्ञे निवेदितम्}
{वेष्टयित्वा च भोगेन विनद्य च महास्वनम्}
{अदशत्पृथिवीपालं तक्षकः पन्नगेश्वरः}


\chapter{अध्यायः ४४}
\twolineshloka
{सौतिरुवाच}
{}


\twolineshloka
{तं तथा मन्त्रिणो दृष्ट्वा भोगेन परिवेष्टितम्}
{विषण्णवदनाः सर्वे रुरुदुर्भृशदुःखिताः}


\twolineshloka
{तं तु नागं ततो दृष्ट्वा मन्त्रिणस्ते प्रदुद्रुवुः}
{अपश्यन्त तथा यान्तमाकाशे नागमद्भुतम्}


\twolineshloka
{सीमन्तमिव कुर्वाणं नभसः पद्मवर्चसम्}
{तक्षकं पन्नगश्रेष्ठं भृशं शोकपरायणाः}


\twolineshloka
{ततस्तु ते तद्गृहमग्निना वृतंप्रदीप्यमानं विषजेन भोगिनः}
{भयात्परित्यज्य दिशः प्रपेदिरेपपात तच्चाशनिताडितं यथा}


\twolineshloka
{ततो नृपे तक्षकतेजसाहतेप्रयुज्य सर्वाः परलोकसत्क्रियाः}
{शुचिर्दिजो राजपुरोहितस्तदातथैव ते तस्य नृपस्य मन्त्रिणः}


\twolineshloka
{नृपं शिशुं तस्य सुतं प्रचक्रिरेसमेत्य सर्वे पुरवासिनो जनाः}
{नृपं यमाहुस्तममित्रघातिनंकुरुप्रवीरं जनमेजयं जनाः}


\twolineshloka
{स बाल एवार्यमतिर्नृपोत्तमःसहैव तैर्मन्त्रिपुरोहितैस्तदा}
{शशास राज्यं कुरुपुंगवाग्रजोयथाऽस्य वीरः प्रपितामहस्तथा}


\twolineshloka
{ततस्तु राजानममित्रतापनंसमीक्ष्य ते तस्य नृपस्य मन्त्रिणः}
{सुवर्णवर्माणमुपेत्य काशिपवपुष्टमार्थं वरयांप्रचक्रमुः}


\twolineshloka
{ततः स राजा प्रददौ वपुष्टमांकुरुप्रवीराय परीक्ष्य धर्मतः}
{स चापि तां प्राप्य मुदा युतोऽभव-न्न चान्यनारीषु मनो दधे क्वचित्}


\twolineshloka
{सरःसु फुल्लेषु वनेषु चैवप्रसन्नचेता विजहार वीर्यवान्}
{तथा स राजन्यवरो विजह्रिवान्यथोर्वशीं प्राप्य पुरा पुरूरवाः}


\twolineshloka
{वपुष्टमा चापि वरं पतिव्रताप्रतीतरूपा समवाप्य भूपतिम्}
{भावेन रामा रमयांबभूव तंविहारकालेष्ववरोधसुन्दरी}


\chapter{अध्यायः ४५}
\twolineshloka
{सौतिरुवाच}
{}


\twolineshloka
{एतस्मिन्नेव काले तु जरत्कारुर्महातपाः}
{चचार पृथिवीं कृत्स्नां यत्रसायंगृहो मुनिः}


\twolineshloka
{चरन्दीक्षां महातेजा दुश्चरामकृतात्मभिः}
{तीर्थेष्वाप्लवनं कुर्वन्पुण्येषु विचचार ह}


\twolineshloka
{वायुभक्षो निराहारः शुष्यन्नहरहर्मुनिः}
{सं ददर्श पितॄन्गर्ते लम्बमानानधोमुखान्}


\twolineshloka
{एकतन्त्ववशिष्टं वै वीरणस्तम्बमाश्रितान्}
{तं तन्तुं च शनैराखुमाददानं बिलेशयम्}


\twolineshloka
{निराहारान्कृशान्दीनान्गर्ते स्वत्राणमिच्छतः}
{उपसृत्य स तान्दीनान्दीनरूपोऽभ्यभाषत}


\twolineshloka
{के भवन्तोऽवलम्बन्ते वीरमस्तम्बमाश्रिताः}
{दुर्बलं खादितैर्मूलैराखुना बिलवासिना}


\twolineshloka
{वीरणस्तम्बके मूलं यदप्येकमिह स्थितम्}
{तद्भप्ययं शनैराखुरादत्ते दशनैः शितैः}


\twolineshloka
{छेत्स्यतेऽल्पावशिष्टत्वादेतदप्यचिरादिव}
{ततस्तु पतितारोऽत्र गर्ते व्यक्तमधोमुखाः}


\twolineshloka
{अत्र मे दुःखमुत्पन्नं दृष्ट्वा युष्मानधोमुखान्}
{कृच्छ्रामापदमापन्नान्प्रियं किं करवाणि वः}


\twolineshloka
{तपसोऽस्य चतुर्थेन तृतीयेनाथ वा पुनः}
{अर्धेन वापि निस्तर्तुमापदं ब्रूत मा चिरम्}


\threelineshloka
{अथवापि समग्रेण तरन्तु तपसा मम}
{भवन्तः सर्व एवेह काममेवं विधीयताम् ॥पितर ऊचुः}
{}


\twolineshloka
{कुतो भवान्ब्रह्मचारी यो नस्त्रातुमिहेच्छसि}
{न तु विप्राग्र्य तपसा शक्यमेतद्व्यपोहितुम्}


\twolineshloka
{अस्ति नस्तात तपसः फलं प्रवदतां वर}
{संतानप्रक्षयाद्ब्रह्मन्पतामो निरयेऽशुचौ}


\twolineshloka
{सन्तानं हि परो धर्मं एवमाह पितामहः}
{लम्बतामिह नस्तात न ज्ञानं प्रतिभाति वै}


\twolineshloka
{येन त्वां नाभिजानीमो लोके विख्यातपौरुषम्}
{वृद्धो भवान्महाभागोयोनः शोच्यान्सुदुःखितान}


\twolineshloka
{शोचते चैव कारुण्याच्छृणु ये वै वयं द्विज}
{यायावरा नाम वयमृषयः संशितव्रताः}


\twolineshloka
{लोकात्पुम्यादिह भ्रष्टाः सन्तानप्रक्षयान्मुने}
{प्रणष्टं नस्तपस्तीव्रं न हि नस्तन्तुरस्ति वै}


\twolineshloka
{अस्तित्वेकोऽद्य नस्तन्तुः सोऽपि नास्ति यथा तथा}
{मन्दभाग्योऽल्पभाग्यानां तप एकं समास्थितः}


\twolineshloka
{जरत्कारुरिति ख्यातो वेदवेदाङ्गपारगः}
{नियतात्मा महात्मा च सुव्रतः सुमहातपाः}


\twolineshloka
{तेन स्म तपसो लोभात्कृच्छ्रमापादिता वयम्}
{न तस्य भार्या पुत्रो वा बान्धवो वाऽस्ति कश्चन}


\twolineshloka
{तस्माल्लम्बामहे गर्ते नष्टसंज्ञा ह्यनाथवत्}
{स वक्तव्यस्त्वया दृष्टो ह्यस्माकं नाथवत्तया}


\twolineshloka
{पितरस्तेऽवलम्बन्ते गर्ते दीना अधोमुखाः}
{साधु दारान्कुरुष्वेति प्रजायस्वेति चाभि भोः}


\twolineshloka
{कुलतन्तुर्हि नः शिष्टः स एकैकस्तपोधन}
{यं तु पश्यसि नो ब्रह्मन्वीरणस्तम्बमाश्रयम्}


\twolineshloka
{एषोऽस्माकं कुलस्तम्ब आस्ते स्वकुलवर्धनः}
{यानि पश्यसि वै ब्रह्मन्मूलानीहास्य वीरुधः}


\twolineshloka
{एते नस्तन्तवस्तात कालेन परिभक्षिताः}
{यत्त्वेतत्पश्यसि ब्रह्मन्मूलमस्यार्धभक्षितम्}


\twolineshloka
{यत्र लम्बामहे गर्ते सोऽप्येकस्तप आस्थितः}
{यमाखुं पश्यसि ब्रह्मन्काल एष महाबलः}


\twolineshloka
{स तं तपोरतं मन्दं शनैः क्षपयते तुदन्}
{जरत्कारुं तपोलुब्धं मन्दात्मानमचेतसम्}


\twolineshloka
{न हि नस्तत्तपस्तस्य तारयिष्यति सत्तम}
{छिन्नमूलान्परिभ्रष्टान्कालोपहतचेतसः}


\twolineshloka
{अधः प्रविष्टान्पश्यास्मान्यथा दुष्कृतिनस्तथा}
{अस्मासु पतितेष्वत्र सह सर्वैः सबान्धवैः}


\twolineshloka
{छिन्नः कालेन सोऽप्यत्र गन्ता वै नरकं ततः}
{तपो वाऽप्यथ चा यज्ञो यच्चान्यत्पावनं महत्}


\twolineshloka
{तत्सर्वं न समं तात संतत्येति सतां मतम्}
{स तात दृष्ट्वा ब्रूयास्तं जरत्कारुं तपोधन}


\twolineshloka
{यथा दृष्टमिदं चात्र त्वयाऽऽख्येयमशेषतः}
{यथा दारान्प्रकुर्यात्स पुत्रानुत्पादयेद्यथा}


\twolineshloka
{वा ब्रह्मंस्त्वया वाच्यः सोऽस्माकं नाथवत्तया}
{बान्धवानां हितस्येह यथा चात्मकुलं तथा}


\twolineshloka
{कस्त्वं बन्धुमिवास्माकमनुशोचसि सत्तम}
{श्रोतुमिच्छाम सर्वेषां को भवानिह तिष्ठति}


\chapter{अध्यायः ४६}
\twolineshloka
{सौतिरुवाच}
{}


\threelineshloka
{एतच्छ्रुत्वा जरत्कारुर्भृशं शोकपरायणः}
{उवाच तान्पितॄन्दुःखाद्बाष्पसंदिग्धया गिरा ॥जरत्कारुरुवाच}
{}


\twolineshloka
{मम पूर्वे भवन्तो वै पितरः सपितामहाः}
{तद्ब्रूत यन्मया कार्यं भवतां प्रियकाम्यया}


\threelineshloka
{अहमेव जरत्कारुः किल्बिषी भवतां सुतः}
{ते दण्डं धारयत मे दुष्कृतेरकृतात्मनः ॥पितर ऊचुः}
{}


\threelineshloka
{पुत्र दिष्ट्याऽसि संप्राप्त इमं देशं यदृच्छया}
{किमर्थं च त्वया ब्रह्मन्न कृतो दारसङ्ग्रहः ॥जरत्कारुरुवाच}
{}


\twolineshloka
{ममायं पितरो नित्यं हृद्यर्थः परिवर्तते}
{ऊर्ध्वरेताः शरीरं वै प्रापयेयममुत्र वै}


\twolineshloka
{न दारान्वै करिष्येऽहमिति मे भावितं मनः}
{एवं दृष्ट्वा तु भवतः शकुन्तानिव लम्बतः}


\twolineshloka
{मया निवर्तिता बुद्धिर्ब्रह्मचर्यात्पितामहाः}
{करिष्ये वः प्रियं कामं निवेक्ष्येऽहमसंशयम्}


\twolineshloka
{सनाम्नीं यद्यहं कन्यामुपलप्स्ये कदाचन}
{भविष्यति च या काचिद्भैक्ष्यवत्स्वयमुद्यता}


\threelineshloka
{प्रतिग्रहीता तामस्मि न भरेयं च यामहम्}
{एवंविधमहं कुर्यां निवेशं प्राप्नुयां यदि}
{अन्यथा न करिष्येऽहं सत्यमेतत्पितामहाः}


\threelineshloka
{तत्र चोत्पत्स्यते जन्तुर्भवतां तारणाय वै}
{शाश्वताश्चाव्ययाश्चैव तिष्ठन्तु पितरो मम ॥सौतिरुवाच}
{}


\twolineshloka
{एवमुक्त्वा तु स पितॄंश्चचार पृथिवी मुनिः}
{न च स्म लभते भार्यां वृद्धोऽयमिति शानक}


\twolineshloka
{यदा निर्वेदमापन्नः पितृभिश्चोदिस्तथा}
{तदाऽरण्यं स गत्वोच्चैश्चुक्रोश भृशदुःखितः}


\twolineshloka
{सत्वरण्यगतः प्राज्ञः पितॄणा हितकाम्यया}
{उवाच कन्यां याचामि तिस्रो वाचः शनैरिमाः}


\twolineshloka
{यानि भूतानि सन्तीह स्थावराणि चराणि च}
{अन्तर्हितानि वा यानि तानि शृण्वन्तु मे वचः}


\twolineshloka
{उग्रे तपसि वर्तन्तं पितरश्चोदयन्ति माम्}
{निविशस्वेति दुःखार्ताः सन्तानस्य चिकीर्षया}


\twolineshloka
{निवेशायाखिलां भूमिं कन्याभैक्ष्यं चरामि भोः}
{दरिद्रो दुःखशीलश्च पितृभिः सन्नियोजितः}


\twolineshloka
{यस्य कन्याऽस्ति भूतस्य ये मयेह प्रकीर्तिताः}
{ते मे कन्यां प्रयच्छन्तु चरतः सर्वतोदिशम्}


\twolineshloka
{मम कन्या सनाम्नी या भैक्ष्यवच्चोदिता भवेत्}
{भरेयं चैव यां नाहं तां मे कन्यां प्रयच्छत}


\twolineshloka
{ततस्ते पन्नगा ये वै जरत्कारौ समाहिताः}
{तामादाय प्रवृत्तिं ते वासुकेः प्रत्यवेदयन्}


\twolineshloka
{तेषां श्रुत्वा स नागेन्द्रस्तां कन्यां समलङ्कृताम्}
{प्रगृह्यारण्यमगमत्समीपं तस्य पन्नगः}


\twolineshloka
{तत्र तां भैक्ष्यवत्कन्यां प्रादात्तस्मै महात्मने}
{नागेन्द्रो वासुकिर्ब्रह्मन्न स तां प्रत्यगृह्णत}


\twolineshloka
{असनामेति वै मत्वा भरणे चाविचारिते}
{मोक्षभावे स्थितश्चापि मन्दीभूतः परिग्रहे}


\twolineshloka
{ततो नाम स कन्यायाः पप्रच्छ भृगुनन्दन}
{वासुकिं भरणं चास्या न कुर्यामित्युवाच ह}


\chapter{अध्यायः ४७}
\twolineshloka
{सौतिरुवाच}
{}


\twolineshloka
{वासुकिस्त्वब्रवीद्वाक्यं जरत्कारुमृषिं तदा}
{सनाम्नी तव कन्येयं स्वसा मे तपसान्विता}


\fourlineindentedshloka
{भरिष्यामि च ते भार्यां प्रतीच्छेमां द्विजोत्तम}
{रक्षणं च करिष्येऽस्याः सर्वशक्त्या तपोधन}
{त्वदर्थं रक्ष्यते चैषा मया मुनिवरोत्तम ॥जरत्कारुरुवाच}
{}


\threelineshloka
{न भरिष्येऽहमेतां वै एष मे समयः कृतः}
{अप्रियं च न कर्तव्यं कृते चैनां त्यजाम्यहम् ॥सौतिरुवाच}
{}


\twolineshloka
{प्रतिश्रुते तु नागेन भरिष्ये भगिनीमिति}
{जरत्कारुस्तदा वेश्म भुजगस्य जगाम ह}


\twolineshloka
{तत्र मन्त्रविदां श्रेष्ठस्तपोवृद्धो महाव्रतः}
{जग्राह पाणिं धर्मात्मा विधिमन्त्रपुरस्कृतम्}


\twolineshloka
{ततो वासगृहं रम्यं पन्नगेन्द्रस्य संमतम्}
{जगाम भार्याम दाय स्तूयमानो महर्षिभिः}


\twolineshloka
{शयनं तत्र संक्लृप्तं स्पर्ध्यास्तरणसंवृतम्}
{तत्र भार्यासहायो वै जरत्कारुरुवास ह}


\twolineshloka
{स तत्र समयं चक्रे भार्यया सह सत्तमः}
{विप्रियं मे न कर्तव्यं न च वाच्यं कदाचन}


\twolineshloka
{त्यजेयं विप्रिये च त्वां कृते वासं च ते गृहे}
{एतद्गृहाण वचनं मया यत्समुदीरितम्}


\twolineshloka
{ततः परमसंविग्ना स्वसा नागपतेस्तदा}
{अतिदुःखान्विता वाक्यं तमुवाचैवमस्त्विति}


\twolineshloka
{तथैव सा च भर्तारं दुःखशीलमुपचारत्}
{उपायैः श्वेतकाकीयैः प्रियकामा यशस्विनी}


\twolineshloka
{ऋतुकाले ततः स्नाता कदाचिद्वासुकेः स्वसा}
{भर्तारं वै यथान्यायमुपतस्थे महामुनिम्}


\twolineshloka
{तत्र तस्याः समभवद्गर्भो ज्वलनसन्निभः}
{अतीव तेजसा युक्तो वैश्वानरसमद्युतिः}


\twolineshloka
{शुक्लपक्षे यथा सोमो व्यवर्धत तथैव सः}
{ततः कतिपयाहस्तु जरत्कारुर्महायशाः}


\twolineshloka
{उत्सङ्गेऽस्याः शिरः कृत्वा सुष्वाप परिखिन्नवत्}
{तस्मिंश्च सुप्ते विप्रेन्द्रे सवितास्तमियाद्गिरिम्}


\twolineshloka
{अह्नः परिक्षये ब्रह्मंस्ततः साऽचिन्तयत्तदा}
{वासुकेर्भगिनी भीता धर्मलोपान्मनस्विनी}


\twolineshloka
{किं नु मे सुकृतं भूयाद्भर्तुरुत्थापनं न वा}
{दुःखशीलो हि धर्मात्मा कथं नास्यापराध्नुयाम्}


\twolineshloka
{कोपो वा धर्मशीलस्य धर्मलोपोऽथवा पुनः}
{धर्मलोपो गरीयान्वै स्यादित्यत्राकरोन्मतिम्}


\twolineshloka
{उत्थापयिष्ये यद्येनं ध्रुवं कोपं करिष्यति}
{धर्मलोपो भवेदस्य सन्ध्यातिक्रमणे ध्रुवम्}


\twolineshloka
{इति निश्चित्य मनसा जरत्कारुर्भुजंगमा}
{तमृषिं दीप्ततपसं शयानमनलोपमम्}


\twolineshloka
{उवाचेदं वचः श्लक्ष्णं ततो मधुरभाषिणी}
{उत्तिष्ठ त्वं महाभाग सूर्योऽस्तमुपगच्छति}


\twolineshloka
{सन्ध्यामुपास्स्व भगवन्नपः स्पृष्ट्वा यतव्रतः}
{प्रादुष्कृताग्निहोत्रोऽयं मुहूर्तो रम्यदारुणः}


\twolineshloka
{सन्ध्या प्रवर्तते चेयं पश्चिमायां दिशि प्रभो}
{एवमुक्तः स भगवाञ्जरत्कारुर्महातपाः}


\twolineshloka
{भार्यां प्रस्फुरमाणौष्ठ इदं वचनमब्रवीत्}
{अवमानः प्रयुक्तोऽयं त्वया मम भुजंगमे}


\twolineshloka
{समीपे ते न वत्स्यामि गमिष्यामि यथागतम्}
{शक्तिरस्ति न वामोरु मयि सुप्ते विभावसोः}


\twolineshloka
{अस्तं गन्तुं यथाकालमिति मे हृदि वर्तते}
{न चाप्यवमतस्येह वासो रोचेत कस्यचित्}


\twolineshloka
{किं पुनर्धर्मशीलस्य मम वा मद्विधस्य वा}
{एवमुक्ता जरत्कारुर्भर्त्रा हृदयकम्पनम्}


\twolineshloka
{अब्रवीद्भगिनी तत्र वासुकेः सन्निवेशने}
{नावमानात्कृतवती तवाहं विप्रे बोधनम्}


\twolineshloka
{धर्मलोपो न ते विप्र स्यादित्येतन्मया कृतम्}
{उवाच भार्यामित्युक्तो जरत्कारुर्महातपाः}


\twolineshloka
{ऋषिः कोपसमाविष्टस्त्यक्तुकामो भुजंगमाम्}
{न मे वागनृतं प्राह गमिष्येऽहं भुजंगमे}


\twolineshloka
{समयो ह्येष मे पूर्वं त्वया सह मिथः कृतः}
{सुखमस्म्युषितो भद्रे ब्रूयास्त्वं भ्रातरं शुभे}


\twolineshloka
{इतो मयि गते भीरु गतः स भगवानिति}
{त्वं चापि मयि निष्क्रान्ते न शोकं कर्तुमर्हसि}


\twolineshloka
{इत्युक्ता साऽनवद्याङ्गी प्रत्युवाच मुनिं तदा}
{जरत्कारुं जरत्कारुश्चिन्ताशोकपरायणा}


\twolineshloka
{बाष्पगद्गदया वाचा मुखेन परिशुष्यता}
{कृताञ्जलिर्वरारोहा पर्यश्रुनयना ततः}


\twolineshloka
{धैर्यमालम्ब्य वामोरूर्हृदयेन प्रवेपता}
{न मामर्हसि धर्मज्ञ परित्यक्तुमनागसम्}


\twolineshloka
{धर्मे स्थितां स्थितो धर्मे सदा प्रियहिते रताम्}
{प्रदाने कारणं यच्च मम तुभ्यं द्विजोत्तम}


\twolineshloka
{तदलब्धवतीं मन्दां किं मां वक्ष्यति वासुकिः}
{मातृशापाभिभूतानां ज्ञातीनां मम सत्तम}


\twolineshloka
{अपत्यमीप्सितं त्वत्तस्तच्च तावन्न दृश्यते}
{त्वत्तो ह्यपत्यलाभेन ज्ञातीनां मे शिवं भवेत्}


\twolineshloka
{संप्रयोगो भवेन्नायां मम मोघस्त्वया द्विज}
{ज्ञातीनां हितमिच्छन्ती भगवंस्त्वां प्रसादये}


\twolineshloka
{इममव्यक्तरूपं मे गर्भमाधाय सत्तम}
{कथं त्यक्त्वा महात्मा सन्गन्तुमिच्छस्यनागसम्}


\twolineshloka
{एवमुक्तस्तु स मुनिर्भार्यां वचनमब्रवीत्}
{यद्युक्तमनुरूपं च जरत्कारुं तपोधनः}


\twolineshloka
{अस्त्ययं सुभगे गर्भस्तव वैश्वानरोपमः}
{ऋषिः परमधर्मात्मा वेदवेदाङ्गपारगः}


\twolineshloka
{एवमुक्त्वा स धर्मात्मा जरत्कारुर्महानृषिः}
{उग्राय तपसे भूयो जगाम कृतनिश्चयः}


\chapter{अध्यायः ४८}
\twolineshloka
{सौतिरुवाच}
{}


\twolineshloka
{गतमात्रं तु भर्तारं जरत्कारुरवेदयत्}
{भ्रातुः सकाशमागत्य यथातथ्यं तपोधन}


\threelineshloka
{ततः स भुजगश्रेष्ठः श्रुत्वा सुमहदप्रियम्}
{उवाच भगिनीं दीनां तदा दीनतरः स्वयम् ॥वासुकिरुवाच}
{}


\twolineshloka
{जानासि भद्रे यत्कार्यं प्रदाने कारणं च यत्}
{पन्नगानां हितार्थाय पुत्रस्ते स्यात्ततो यदि}


\twolineshloka
{स सर्पसत्रात्किल नो मोक्षयिष्यति वीर्यवान्}
{एवं पितामहः पूर्वमुक्तवांस्तु सुरैः सह}


\twolineshloka
{अप्यस्ति गर्भः सुभगे तस्मात्ते मुनिसत्तमात्}
{न चेच्छाम्यफलं तस्य दारकर्म मनीषिणः}


\twolineshloka
{कामं च मम न न्याय्यं प्रष्टुं त्वां कार्यमीदृशम्}
{किंतु कार्यगरीयस्त्वात्ततस्त्वाऽहमचूचुदम्}


\twolineshloka
{दुर्वार्यतां विदित्वा च भर्तुस्तेऽतितपस्विनः}
{नैनमन्वागमिष्यामि कदाचिद्धि शपेत्स माम्}


\twolineshloka
{आचक्ष्व भद्रे भर्तुः स्वं सर्वमेव विचेष्टितम्}
{उद्धरस्व च शल्यं मे घोरं हृदि चिरस्थितम्}


\threelineshloka
{जरत्कारुस्ततो वाक्यमित्युक्ता प्रत्यभाषत}
{आश्वासयन्ती सन्तप्तं वासुकिं पन्नगेश्वरम् ॥जरत्कारुरुवाच}
{}


\twolineshloka
{पृष्टो मयाऽपत्यहेतोः स महात्मा महातपाः}
{अस्तीत्युत्तरमुद्दिश्य ममेदं गतवांश्च सः}


\twolineshloka
{स्वैरेष्वपि न तेनाहं स्मरामि वितथं वचः}
{उक्तपूर्वं कुतो राजन्सांपराये स वक्ष्यति}


\twolineshloka
{न संतापस्त्वया कार्यः कार्यं प्रति भुजंगमे}
{उत्पत्स्यति च ते पुत्रो ज्वलनार्कसमप्रभः}


\threelineshloka
{इत्युक्त्वा स हि मां भ्रातर्गतो भर्ता तपोधनः}
{तस्माद्व्येतु परं दुःखं तवेदं मनसि स्थितम् ॥सौतिरुवाच}
{}


\twolineshloka
{एतच्छ्रुत्वा स नागेन्द्रो वासुकिः परया मुदा}
{एवमस्त्विति तद्वाक्यं भगिन्याः प्रत्यगृह्णत}


\twolineshloka
{सान्त्वमानार्थदानैश्च पूजया चारुरूपया}
{सोदर्यां पूजयामास स्वसारं पन्नगोत्तमः}


\twolineshloka
{ततः प्रववृधे गर्भो महातेजा महाप्रभः}
{यथा मोमो द्विजश्रेष्ठ शुक्लपक्षोदितो दिवि}


\twolineshloka
{अथ काले तु सा ब्रह्मन्प्रजज्ञे भुजगस्वसा}
{कुमारं देवगर्भाभं पितृमातृभयापहम्}


\twolineshloka
{ववृधे स तु तत्रैव नागराजनिवेशने}
{वेदांश्चाधिजगे साङ्गान्भार्गवच्यवनात्मजात्}


\twolineshloka
{चीर्णव्रतो बाल एव बुद्धिसत्त्वगुणान्वितः}
{नाम चास्याभवत्ख्यातं लोकेष्वास्तीक इत्युत}


\twolineshloka
{अस्तीत्युक्त्वा गतो यस्मात्पिता गर्भस्थमेव तम्}
{वनं तस्मादिदं तस्य नामास्तीकेति विश्रुतम्}


\twolineshloka
{स बाल एव तत्रस्थश्चरन्नमितबुद्धिमान्}
{गृहे पन्नगराजस्य प्रयत्नात्परिरक्षितः}


\twolineshloka
{भगवानिव देवेशः शूलपाणिर्हिरण्मयः}
{विवर्धमानः सर्वांस्तान्पन्नगानभ्यहर्षयत्}


\chapter{अध्यायः ४९}
\twolineshloka
{शौनक उवाच}
{}


\threelineshloka
{यदपृच्छत्तदा राजा मन्त्रिणो जनमेजयः}
{पितुः स्वर्गगतिं तन्मे विस्तरेण पुनर्वद ॥सौतिरुवाच}
{}


\threelineshloka
{शृणु ब्रह्मन्यथाऽपृच्छन्मन्त्रिणो नृपतिस्तदा}
{यथा चाख्यातवन्तस्ते निधनं तत्परिक्षितः ॥जनमेजय उवाच}
{}


\twolineshloka
{जानन्ति स्म भवन्तस्तद्यथावृत्तं पितुर्मम}
{आसीद्यथा स निधनं गतः काले महायशाः}


\threelineshloka
{श्रुत्वा भवत्सकाशाद्धि पितुर्वृत्तमशेषतः}
{कल्याणं प्रतिपत्स्यामि विपरीतं न जातुचित् ॥सौतिरुवाच}
{}


\threelineshloka
{मन्त्रिणोऽथाब्रुवन्वाक्यं पृष्टास्तेन महात्मना}
{सर्वे धर्मविदः प्राज्ञा राजानं जनमेजयम् ॥मन्त्रिण ऊचुः}
{}


\twolineshloka
{शृणु पार्थिव यद्ब्रूषे पितुस्तव महात्मनः}
{चरितं पार्थिवेन्द्रस्य यथा निष्ठां गतश्च सः}


\twolineshloka
{धर्मात्मा च महात्मा च प्रजापालः पिता तव}
{आसीदिहायथा वृत्तः स महात्मा शृणुष्व तत्}


\twolineshloka
{चातुर्वर्ण्यं स्वधर्मस्थं स कृत्वा पर्यरक्षत}
{धर्मतो धर्मविद्राजा धर्मो विग्रहवानिव}


\twolineshloka
{ररक्ष पृथिवीं देवीं श्रीमानतुलविक्रमः}
{द्वेष्टारस्तस्य नैवासन्स च द्वेष्टि न कंचन}


\twolineshloka
{समः सर्वेषु भूतेषु प्रजापतिरिवाभवत्}
{ब्राह्मणाः क्षत्रिया वैश्याः शूद्राश्चैव स्वकर्मसु}


\twolineshloka
{स्थितः सुमनसो राजंस्तेन राज्ञा स्वधिष्ठिताः}
{विधवानाथविकलान्कृपणांश्च बभार सः}


\twolineshloka
{सुदर्शः सर्वभूतानामासीत्सोम इवापरः}
{तुष्टपुष्टजनः श्रीमान्सत्यवाग्दृढविक्रमः}


\twolineshloka
{धनुर्वेदे तु शिष्योऽभून्नृपः शारद्वतस्य सः}
{गोविन्दस्य प्रियश्चासीत्पिता ते जनमेजय}


\twolineshloka
{लोकस्य चैव सर्वस्य प्रिय आसीन्महायशाः}
{परिक्षीणेषु कुरुषु सोत्तरायामजीजनत्}


\twolineshloka
{परिक्षिदभवत्तेन सौभद्रस्यात्मजो बली}
{राजधर्मार्थकुशलो युक्तः सर्वगुणैर्वृतः}


\twolineshloka
{जितेन्द्रियश्चात्मवांश्च मेधावी धर्मसेविता}
{षड्वर्गजिन्महाबुद्धिर्नीतिशास्त्रविदुत्तमः}


\twolineshloka
{प्रजा इमास्तव पिता षष्टिवर्षाण्यपालयत्}
{ततो दिष्टान्तमापन्नः सर्वेषां दुःखमावहन्}


\fourlineindentedshloka
{ततस्त्वं पुरुषश्रेष्ठ धर्मेण प्रतिपेदिवान्}
{इदं वर्षसहस्राणि राज्यं कुरुकुलागतम्}
{बाल एवाभिषिक्तस्त्वं सर्वभूतानुपालकः ॥जनमेजय उवाच}
{}


\twolineshloka
{नास्मिन्कुले जातु बभूव राजायो न प्रजानां प्रियकृत्प्रियश्च}
{विशेषतः प्रेक्ष्य पितामहानांवृत्तं महद्वृत्तपरायणानाम्}


\threelineshloka
{कथं निधनमापन्नः पिता मम तथाविधः}
{आचक्षध्वं यथावन्मे श्रोतुमिच्छामि तत्त्वतः ॥सौतिरुवाच}
{}


\threelineshloka
{एवं संचोदिता राज्ञा मन्त्रिणस्ते नराधिपम्}
{ऊचुः सर्वे यथावृत्तं राज्ञः प्रियहितैषिणः ॥मन्त्रिण ऊचुः}
{}


\twolineshloka
{स राजा पृथिवीपालः सर्वशस्त्रभृतां वरः}
{बभूव मृगयाशीलस्तव राजन्पिता सदा}


\twolineshloka
{यथा पाण्डुर्महाबाहुर्धनुर्धरवरो युधि}
{अस्मास्वासज्य सर्वाणि राजकार्याण्यशेषतः}


\twolineshloka
{स कदाचिद्वनगतो मृगं विव्याध पत्रिणा}
{विद्ध्वा चान्वसरत्तूर्णं तं मृगं गहने वने}


\twolineshloka
{पदातिर्बद्धनिस्त्रिंशस्ततायुधकलापवान्}
{न चाससाद गहने मृगं नष्टं पिता तव}


\twolineshloka
{परिश्रान्तो वयस्थश्च षष्टिवर्षो जरान्वितः}
{क्षुधितः स महारण्ये ददर्श मुनिसत्तमम्}


\twolineshloka
{स तं पप्रच्छ राजेन्द्रो मुनिं मौनव्रते स्थितम्}
{न च किंचिदुवाचेदं पृष्टोऽपि समुनिस्तदा}


\twolineshloka
{ततो राजा क्षुच्छ्रमार्तस्तं मुनिं स्थाणुवत्स्थितम्}
{मौनव्रतधरं शान्तं सद्यो मन्युवशं गतः}


\twolineshloka
{न बुबोध च तं राजा मौनव्रतधरं मुनिम्}
{स तं क्रोधसमाविष्टो धर्षयामास ते पिता}


\twolineshloka
{मृतं सर्पं धनुष्कोट्या समुत्क्षिप्य धरातलात्}
{तस्य शुद्धात्मनः प्रादात्स्कन्धे भरतसत्तम}


\twolineshloka
{न चोवाच स मेधावी तमथो साध्वसाधु वा}
{तस्थौ तथैव चाक्रुद्धः सर्पं स्कन्धेन धारयन्}


\chapter{अध्यायः ५०}
\twolineshloka
{मन्त्रिण ऊचुः}
{}


\twolineshloka
{ततः स राजा राजेन्द्र स्कन्धे तस्य भुजंगमम्}
{मुनेः क्षुत्क्षाम आसज्य स्वपुरं प्रययौ पुनः}


\twolineshloka
{ऋषेस्तस्य तु पुत्रोऽभूद्गति जातो महायशाः}
{शृङ्गी नाम महातेजास्तिग्मवीर्योऽतिकोपनः}


\twolineshloka
{ब्रह्माणं समुपागम्य मुनिः पूजां चकार ह}
{सोऽनुज्ञातस्ततस्तत्र शृङ्गी शुश्राव तं तदा}


\twolineshloka
{सख्युः सकाशात्पितरं पित्रा ते धर्षितं पुरा}
{मृतं सर्पं समासक्तं स्थाणुभूतस्य तस्य तम्}


\twolineshloka
{वहन्तं राजशार्दूल स्कन्धेनानपकारिणम्}
{तपस्विनमतीवाथ तं मुनिप्रवरं नृप}


\twolineshloka
{जितेन्द्रियं विशुद्धं च स्थितं कर्मण्यथाद्भुतम्}
{तपसा द्योतितात्मानं स्वेष्वङ्गेषु यतं तदा}


\threelineshloka
{शुभाचारं शुभकथं सुस्थितं तमलोलुपम्}
{अक्षुद्रमनसूयं च वृद्धं मौनव्रते स्थितम्}
{शरण्यं सर्वभूतानां पित्रा विनिकृतं तव}


\twolineshloka
{शशापाथ महातेजाः पितरं ते रुषान्वितः}
{ऋषेः पुत्रो महातेजा बालोऽपि स्थविरद्युतिः}


\twolineshloka
{स क्षिप्रमुदकं स्पृष्ट्वा रोषादिदमुवाच ह}
{पितरं तेऽभिसन्धाय तेजसा प्रज्वलन्निव}


\twolineshloka
{अनागसि गुरौ यो मे मृतं सर्पवासृजत्}
{तं नागस्तक्षकः क्रुद्धस्तेजसा प्रदहिष्यति}


\twolineshloka
{आशीविषस्तिग्मतेजा मद्वाक्यबलचोदितः}
{सप्तरात्रादितः पापं पश्य मे तपसो बलम्}


\twolineshloka
{इत्युक्त्वा प्रययौ तत्र पिता यत्राऽस्य सोऽभवत्}
{दृष्ट्वा च पितरं तस्मै तं शापं प्रत्यवेदयत्}


\twolineshloka
{स चापि मुनिशार्दूलः प्रेरयामास ते पितुः}
{शिष्यं गौरमुखं नाम शीलवन्तं गुणान्वितम्}


\twolineshloka
{आचख्यौं सत्त्व विश्रान्तो राज्ञः सर्वमशेषतः}
{शप्तोऽसि मम पुत्रेण यत्तो भव महीपते}


\twolineshloka
{तक्षकस्त्वां महाराज तेजसाऽसौ दहिष्यति}
{श्रुत्वा च तद्वचो घोरं पिता ते जनमेजय}


\twolineshloka
{यत्तोऽभवत्परित्रस्तस्तक्षकात्पन्नगोत्तमात्}
{ततस्तस्मिंस्तु दिवसे सप्तमे समुपस्थिते}


\twolineshloka
{राज्ञः समीपं ब्रह्मर्षिः काश्यपो गन्तुमैच्छत}
{तं ददर्शाथ नागेन्द्रस्तक्षकः काश्यपं तदा}


\threelineshloka
{तमब्रवीत्पन्नगेन्द्रः काश्यपं त्वरितं द्विजम्}
{क्व भवांस्त्वरितो याति किं च कार्यं चिकीर्षति ॥काश्यप उवाच}
{}


\twolineshloka
{यत्र राजा कुरुश्रेष्ठः परिक्षिन्नाम वै द्विज}
{तक्षकेण भुजंगेन धक्ष्यते किल सोऽद्य वै}


\threelineshloka
{गच्छाम्यहं तं त्वरितः सद्यः कर्तुमपज्वरम्}
{मयाऽभिपन्नं तं चापि न सर्पो धर्षयिष्यति ॥तक्षक उवाच}
{}


\twolineshloka
{किमर्थं तं मया दष्टं संजीवयितुमिच्छसि}
{अहं त तक्षको ब्रह्मन्पश्य मे वीर्यमद्भुतम्}


\threelineshloka
{न शक्तस्त्वं मया दष्टं तं संजीवयितुं नृपम्}
{मन्त्रिण ऊचुः}
{इत्युक्त्वा तक्षकस्तत्र सोऽदशद्वै वनस्पतिम्}


\twolineshloka
{स दष्टमात्रो नागेन भस्मीभूतोऽभवन्नगः}
{काश्यपश्च ततो राजन्नजीवयत तं नगम्}


\twolineshloka
{ततस्तं लोभयामास कामं ब्रूहीति तक्षकः}
{स एवमुक्तस्तं प्राह काश्यपस्तक्षकं पुनः}


\twolineshloka
{धनलिप्सुरहं तत्र यामीत्युक्तश्च तेन सः}
{तमुवाच महात्मानं तक्षकः श्लक्ष्णया गिरा}


\twolineshloka
{यावद्धनं प्रार्थयसे राज्ञस्तस्मात्ततोऽधिकम्}
{गृहाण मत्त एव त्वं सन्निवर्तस्व चानघ}


\twolineshloka
{स एवमुक्तो नागेन काश्यपो द्विपदां वरः}
{लब्ध्वा वित्तं निववृते तक्षकाद्यावदीप्सितम्}


\twolineshloka
{तस्मिन्प्रतिगते विप्रे छद्मनोपेत्य तक्षकः}
{तं नृपं नृपतिश्रेष्ठं पितरं धार्मिकं तव}


\twolineshloka
{प्रासादस्थं यत्तमपि दग्धवान्विषवह्निना}
{ततस्त्वं पुरुषव्याघ्र विजयायाभिषेचितः}


\twolineshloka
{एतद्दृष्टं श्रुतं चापि यथावन्नृपसत्तम}
{अस्माभिर्निखिलं सर्वं कथितं तेऽतिदारुणम्}


\threelineshloka
{श्रुत्वा चैतं नरश्रेष्ठ पार्थिवस्य पराभवम्}
{अस्य चर्षेरुदङ्कस्य विधत्स्व यदनन्तरम् ॥सौतिरुवाच}
{}


\threelineshloka
{एतस्मिन्नेव काले तु स राजा जनमेजयः}
{उवाच मन्त्रिणः सर्वानिदं वाक्यमरिदमः ॥जनमेजय उवाच}
{}


\twolineshloka
{अथ तत्कथितं केन यद्वृत्तं तद्वनस्पतौ}
{आश्चर्यभूतं लोकस्य भस्मराशीकृतं तदा}


\twolineshloka
{यद्वृक्षं जीवयामास काश्यपस्तक्षकेण वै}
{नूनं मन्त्रैर्हतविषो न प्रणश्येत काश्यपात्}


\twolineshloka
{चिन्तयामास पापात्मा मनसा पन्नगाधमः}
{दष्टं यदि मया विप्रः पार्थिवं जीवयिष्यति}


\twolineshloka
{तक्षकः संहतविषो लोके यास्यति हास्यताम्}
{विचिन्त्यैवं कृता तेन ध्रुवं तुष्टिर्द्विजस्य वै}


\twolineshloka
{भविष्यति ह्युपायेन यस्य दास्यामि यातनाम्}
{एकं तु श्रोतुमिच्छामि तद्वृत्तं निर्जने वने}


\fourlineindentedshloka
{संवादं पन्नगेन्द्रस्य काश्यपस्य च कस्तदा}
{श्रुतवान्दृष्टवांश्चापि भवत्सु कथमागतम्}
{श्रुत्वा तस्य विधास्येऽहं पन्नगान्तकरीं मतिम् ॥मन्त्रिण ऊचुः}
{}


\twolineshloka
{शृणु राजन्यथास्माकं येन तत्कथितं पुरा}
{समागतं द्विजेन्द्रस्य पन्नगेन्द्रस्य चाध्वनि}


\twolineshloka
{तस्मिन्वृक्षे नरः कश्चिदिन्धनार्थाय पार्थिव}
{विचिन्वन्पूर्वमारूढः शुष्कशाखावनस्पतौ}


\twolineshloka
{न बुध्येतामुभौ तौ च नगस्थं पन्नगद्विजौ}
{सह तेनैव वृक्षेण भस्मीभूतोऽभवन्नृप}


\twolineshloka
{द्विजप्रभावाद्राजेन्द्र व्यजीवत्स वनस्पतिः}
{तेनागम्य द्विजश्रेष्ठ पुंसाऽस्मासु निवेदितम्}


\fourlineindentedshloka
{यथा वृत्तं तु तत्सर्वं तक्षकस्य द्विजस्य च}
{एतत्ते कथितं राजन्यथादृष्टं श्रुतं च यत्}
{श्रुत्वा च नृपशार्दूल विधत्स्व यदनन्तरम् ॥सौतिरुवाच}
{}


\twolineshloka
{मन्त्रिणां तु वचः श्रुत्वा स राजा जनमेजयः}
{पर्यतप्यत दुःखार्तः प्रत्यपिंषत्करं करे}


\twolineshloka
{निःश्वासमुष्णमसकृद्दीर्घं राजीवलोचनः}
{मुमोचाश्रूणि च तदा नेत्राभ्यां प्ररुदन्नृपः}


\twolineshloka
{उवाच च महीपालो दुःखशोकसमन्वितः}
{दुर्धरं बाष्पमुत्सृज्य स्पृष्ट्वा चापो यथाविधि}


\threelineshloka
{मुहूर्तमिव च ध्यात्वा निश्चित्य मनसा नृपः}
{अमर्षी मन्त्रिणः सर्वानिदं वचनमब्रवीत् ॥जनमेजय उवाच}
{}


\threelineshloka
{श्रुत्वैतद्भवतां वाक्यं पितुर्मे स्वर्गतिं प्रति}
{निश्चितेयं मम मतिर्या च तां मे निबोधत}
{अनन्तरं च मन्येऽहं तक्षकाय दुरात्मने}


\twolineshloka
{प्रतिकर्तव्यमित्येवं येन मे हिंसितः पिता}
{शृङ्गिणं हेतुमात्रं यः कृत्वा दग्ध्वा च पार्थिवम्}


\twolineshloka
{इयं दुरात्मता तस्य काश्यपं यो न्यवर्तयत्}
{यद्यागच्छेत्स वै विप्रो ननु जीवेत्पिता मम}


\twolineshloka
{परिहीयेत किं तस्य यदि जीवेत्स पार्थिवः}
{काश्यपस्य प्रसादेन मन्त्रिणां विनयेन च}


\twolineshloka
{स तु वारितवान्मोहात्काश्यपं द्विजसत्तमम्}
{संजिजीवयिषुं प्राप्तं राजानमपराजितम्}


\twolineshloka
{महानतिक्रमो ह्येष तक्षकस्य दुरात्मनः}
{द्विजस्य योऽददद्द्रव्यं मा नृपं जीवयेदिति}


\twolineshloka
{उत्तङ्कस्य प्रियं कर्तुमात्मनश्च महत्प्रियम्}
{भवतां चैव सर्वेषां गच्छाम्यपचितिं पितुः}


\chapter{अध्यायः ५१}
\twolineshloka
{सौतिरुवाच}
{}


\twolineshloka
{एवमुक्त्वा ततः श्रीमान्मन्त्रिभिश्चानुमीदितः}
{आरुरोह प्रतिज्ञां स सर्पसत्राय पार्थिवः}


\twolineshloka
{ब्रह्मन्भरतशार्दूलो राजा पारिक्षितस्तदा}
{पुरोहितमथाहूय ऋत्विजो वसुधाधिपः}


\twolineshloka
{अब्रवीद्वाक्यसंपन्नः कार्यसंपत्करं वचः}
{यो मे हिंसितवांस्तातं तक्षकः स दुरात्मवान्}


\twolineshloka
{प्रतिकुर्यां यथा तस्य तद्भवन्तो ब्रुवन्तु मे}
{अपि तत्कर्म विदितं भवतां येन पन्नगम्}


\threelineshloka
{तक्षकं संप्रदीप्तेऽग्नौ प्रक्षिपेयं सबान्धवम्}
{यथा तेन पिता मह्यं पूर्वं दग्धो विषाग्निना ॥तथाऽहमपि तं पापं दग्धुमिच्छामि पन्नगम् ॥ऋत्विज ऊचुः}
{}


\twolineshloka
{अस्ति राजन्महात्सत्रं त्वदर्थं देवनिर्मितम्}
{सर्वसत्रमिति ख्यातं पुराणे परिपठ्यते}


\twolineshloka
{आहर्ता तस्य सत्रस्य त्वन्नान्योऽस्ति नराधिप}
{इति पौराणिकाः प्राहुरस्माकं चास्ति स क्रतुः}


\twolineshloka
{एवमुक्तः स राजर्षिर्मेने दग्धं हि तक्षकम्}
{हुताशनमुखे दीप्ते प्रविष्टमिति सत्तम}


\threelineshloka
{ततोऽब्रवीन्मन्त्रविदस्तान्राजा ब्राह्मणांस्तदा}
{आहरिष्यामि तत्सत्रं संभाराः संभ्रियन्तु मे ॥सौतिरुवाच}
{}


\twolineshloka
{ततस्त ऋत्विजस्तस्य शास्त्रतो द्विजसत्तम}
{तं देशं मापयामासुर्यज्ञायतनकारणात्}


\twolineshloka
{यथावद्वेदविद्वांसः सर्वे बुद्धेः परंगताः}
{ऋद्ध्या परमया युक्तमिष्टं द्विजगणैर्युतम्}


\twolineshloka
{प्रभूतधनधान्याढ्यमृत्विग्भिः सुनिषेवितम्}
{निर्माय चापि विधिवद्यज्ञायतनमीप्सितम्}


\twolineshloka
{राजानं दीक्षयामासुः सर्पसत्राप्तये तदा}
{इदं चासीत्तत्र पूर्वं सर्पसत्रे भविष्यति}


\twolineshloka
{निमित्तं महदुत्पन्नं यज्ञविघ्नकरं तदा}
{यज्ञस्यायतने तस्मिन्क्रियमाणे वचोऽब्रवीत्}


\twolineshloka
{स्थपतिर्बुद्धिसंपन्नो वास्तुविद्याविशारदः}
{इत्यब्रवीत्सूत्रधारः सूतः पौराणिकस्तदा}


\twolineshloka
{यस्मिन्देशे च काले च मापनेयं प्रवर्तिता}
{ब्राह्मणं कारणं कृत्वा नायं संस्थास्यते क्रतुः}


\twolineshloka
{एतच्छ्रुत्वा तु राजासौ प्राग्दीक्षाकालमब्रवीत्}
{क्षत्तारं न हि मे कश्चिदज्ञातः प्रविशेदिति}


\chapter{अध्यायः ५२}
\twolineshloka
{सौतिरुवाच}
{}


\twolineshloka
{ततः कर्म प्रववृते सर्पसत्रविधानतः}
{पर्यक्रामंश्च विदिवत्स्वे स्वे कर्मणि याजकाः}


\twolineshloka
{प्रावृत्य कृष्णवासांसि धूम्रसंरक्तलोचनाः}
{जुहुवुर्मन्त्रवच्चैव समिद्धं जातवेदसम्}


\twolineshloka
{कम्पयन्तश्च सर्वेषामुरगाणां मनांसि च}
{सर्पानाजुहुवुस्तत्र सर्वानग्निमुखे तदा}


\twolineshloka
{ततः सर्पाः समापेतुः प्रदीप्ते हव्यवाहने}
{विचेष्टमानाः कृपणमाह्वयन्तः परस्परम्}


\twolineshloka
{विस्फुरन्तः श्वसन्तश्च वेष्टयन्तः परस्परम्}
{पुच्छैः शिरोभिश्च भृशं चित्रभानुं प्रपेदिरे}


\twolineshloka
{श्वेताः कृष्णाश्च नीलाश्च स्थविराः शिशवस्तथा}
{नदन्तो विविधान्नादान्पेतुर्दीप्ते विभावसौ}


\twolineshloka
{क्रोशयोजनमात्रा हि गोकर्णस्य प्रमाणतः}
{पतन्त्यजस्रं वेगेन वह्नावग्निमतां वर}


\twolineshloka
{एवं शतसहस्राणि प्रयुतान्यर्बुदानि च}
{अवशानि विनष्टानि पन्नगानां तु तत्र वै}


\twolineshloka
{तुरगा इव तत्रान्ये हस्तिहस्ता इवापरे}
{मत्ता इव च मातङ्गा महाकाया महाबलाः}


\threelineshloka
{उच्चावचाश्च बहवो नानावर्णा विषोल्बणाः}
{घोराश्च परिघप्रख्या दन्दशूका महाबलाः}
{प्रपेतुरग्नावुरगा मातृवाग्दण्डपीडिताः}


\chapter{अध्यायः ५३}
\twolineshloka
{शौनक उवाच}
{}


\twolineshloka
{सर्पसत्रे तदा राज्ञः पाण्डवेयस्य धीमतः}
{जनमेजयस्य के त्वासन्नृत्विजः परमर्षयः}


\twolineshloka
{के सदस्या बभूवुश्च सर्पसत्रे सुदारुणे}
{विषादजननेऽत्यर्थं पन्नगानां महाभय}


\threelineshloka
{सर्वं विस्तरशस्तात भवाञ्छंसितुमर्हति}
{सर्पसत्रविधानज्ञविज्ञेयाः के च सूतज ॥सौतिरुवाच}
{}


\twolineshloka
{हन्त ते कथयिष्यामि नामानीह मनीषिणाम्}
{ये ऋत्विजः सदस्याश्च तस्यासन्नृपतेस्तदा}


\twolineshloka
{तत्र होता बभूवाथ ब्राह्मणश्चण्डभार्गवः}
{च्यवनस्यान्वये ख्यातो जातो वेदविदां वरः}


\twolineshloka
{उद्गाता ब्राह्मणो वृद्धो विद्वान्कौत्सौऽथ जैमिनिः}
{ब्र्हमाऽभवच्छार्ङ्गरवोऽथाध्वर्युश्चापि पिङ्गलः}


\twolineshloka
{सदस्यश्चाभवद्व्यासः पुत्रशिष्यसहायवान्}
{उद्दालकः प्रमतकः श्वेतकेतुश्च पिङ्गलः}


\twolineshloka
{असितो देवलश्चैव नारदः पर्वतस्तथा}
{आत्रेयः कुम्डजठरौ द्विजः कालघटस्तथा}


\twolineshloka
{वात्स्यः श्रुतश्रवा वृद्धो जपस्वाध्यायशीलवान्}
{कोहलो देवशर्मा च मौद्गल्यः समसौरभः}


\twolineshloka
{एते चान्ये च बहवो ब्राह्मणा वेदपारगाः}
{सदस्याश्चाभवंस्तत्र सत्रे पारिक्षितस्य ह}


\twolineshloka
{जुह्वत्स्वृत्विक्ष्वथ तदा सर्पसत्रे महाक्रतौ}
{अहयः प्रापतंस्तत्र घोराः प्राणिभयावहाः}


\twolineshloka
{वसामेदोवहाः कुल्या नागानां संप्रवर्तिताः}
{ववौ गन्धश्च तुमुलो दह्यतामनिशं तदा}


\twolineshloka
{पततां चैव नागानां धिष्ठितानां तथाम्बरे}
{अश्रूयतानिशं शब्दः पच्यतां चाग्निना भृशम्}


\twolineshloka
{तक्षकस्तु स नागेन्द्रः पुरंदरनिवेशनम्}
{गतः श्रुत्वैव राजानं दीक्षितं जनमेजयम्}


\twolineshloka
{ततः सर्वं यथावृत्तमाख्याय भुजगोत्तमः}
{अगच्छच्छरणं भीत आगस्कृत्वा पुरंदरम्}


\twolineshloka
{तमिन्द्रः प्राह सुप्रीतो न तवास्तीह तक्षक}
{भयं नागेन्द्र तस्माद्वै सर्पसत्रात्कदाचन}


\threelineshloka
{प्रसादितो मया पूर्वं तवार्थाय पितामहः}
{तस्मात्तव भयं नास्ति व्येतु तेनसो ज्वरः ॥सौतिरुवाच}
{}


\twolineshloka
{एवमाश्वासितस्तेन ततः स भुजगोत्तमः}
{उवास भवने तस्मिञ्शक्रस्य मुदितः सुखी}


\twolineshloka
{अजस्रं निपतत्स्वग्नौ नागेषु भृशदुःखितः}
{अल्पशेषपरीवारो वासुकिः पर्यतप्यत}


\twolineshloka
{कश्मलं चाविशद्धोरं वासुकिं पन्नगोत्तमम्}
{स घूर्णमानहृदयो भगिनीमिदमब्रवीत्}


\twolineshloka
{दह्यन्तेऽङ्गानि मे भद्रे न दिशः प्रतिभान्ति माम्}
{सीदामीव च संमोहाद्धूर्णतीव च मे मनः}


\twolineshloka
{दृष्टिर्भ्राम्यति मेऽतीव हृदयं दीर्यतीव च}
{पतिष्याम्यवशोऽद्याहं तस्मिन्दीप्ते विभावसौ}


\twolineshloka
{पारिक्षितस्य यज्ञोऽसौ वर्ततेऽस्मज्जिघांसया}
{व्यक्तं मयाऽभिगन्तव्यं प्रेतराजनिवेशनम्}


\twolineshloka
{अयं स कालः संप्राप्तो यदर्थमसि मे स्वसः}
{जरत्करौ(पुरा)मयादत्तात्रायस्वास्मान्सबान्धवान्}


\twolineshloka
{आस्तीकः किल यज्ञं तं वर्तन्तं भुजगोत्तमे}
{प्रतिषेत्स्यति मां पूर्वं स्वयमाह पितामहः}


\twolineshloka
{तद्वत्से ब्रूहि वत्सं स्वं कुमारं वृद्धसंमतम्}
{ममाद्य त्वं सभृत्यस्य मोक्षार्थं वेदवित्तमम्}


\chapter{अध्यायः ५४}
\twolineshloka
{सौतिरुवाच}
{}


\twolineshloka
{तत आहूय पुत्रं स्वं जरत्कारुर्भुजंगमा}
{वासुकेर्नागराजस्य वचनादिदमब्रवीत्}


\threelineshloka
{अहं तव पितुः पुत्र भ्रात्रा दत्ता निमित्ततः}
{कालः स चायं संप्राप्तस्तत्कुरुष्व यथातथम् ॥आस्तीक उवाच}
{}


\threelineshloka
{किंनिमित्तं मम पितुर्दत्ता त्वं मातुलेन मे}
{तन्ममाचक्ष्व तत्त्वेन श्रुत्वा कर्ताऽस्मि तत्तथा ॥सौतिरुवाच}
{}


\threelineshloka
{तत आचष्ट सा तस्मै बान्धवानां हितैषिणी}
{भगिनी नागराजस्य जरत्कारुरविक्लबा ॥जरत्कारुरुवाच}
{}


\twolineshloka
{पन्नगानामशेषाणां माता कद्रूरिति श्रुता}
{तया शप्ता रुषितया सुता यस्मान्निबोध तत्}


\twolineshloka
{उच्चैः श्रवाः सोऽश्वराजो यन्मिथ्या न कृतो मम}
{विनतार्थाय पणिते दासभावाय पुत्रकाः}


\twolineshloka
{जनमेजयस्य वो यज्ञे धक्ष्यत्यनिलसारथिः}
{तत्र पञ्चत्वमापन्नाः प्रेतलोकं गमिष्यथ}


\twolineshloka
{तां च शप्तवतीं देवः साक्षाल्लोकपितामहः}
{एवमस्त्विति तद्वाक्यं प्रोवाचानु मुमोद च}


\twolineshloka
{वासुकिश्चापि तच्छ्रुत्वा पितामहवचस्तदा}
{अमृते मथिते तात देवाञ्छरणमीयिवान्}


\twolineshloka
{सिद्धार्थाश्च सुराः सर्वे प्राप्यामृतमनुत्तमम्}
{भ्रातरं मे पुरस्कृत्य पितामहमुपागमन्}


\threelineshloka
{ते तं प्रसादयामासुः सुराः सर्वेऽब्जसंभवम्}
{राज्ञा वासुकिना सार्धं शापोऽसौन भवेदिति ॥देवा ऊचुः}
{}


\threelineshloka
{वासुकिर्नागराजोऽयं दुःखितो ज्ञातिकारणात्}
{अभिशापः स मातुस्तु भगवन्न भवेत्कथम् ॥ब्रह्मोवाच}
{}


\twolineshloka
{जरत्कारुर्जरत्कारुं यां भार्यां समवाप्स्यति}
{तत्र जातो द्विजः शापान्मोक्षयिष्यति पन्नगान्}


\twolineshloka
{एतच्छ्रुत्वा तु वचनं वासुकिः पन्नगोत्तमः}
{प्रादान्माममरप्रख्य तव पित्रे महात्मने}


\twolineshloka
{प्रागेवानागते काले तस्मात्त्व मय्यजायथाः}
{अयं स कालः संप्राप्तो भयान्नस्त्रातुमर्हसि}


\fourlineindentedshloka
{भ्रातरं चापि मे तस्मात्त्रातुमर्हसि पावकात्}
{न मोघं तु कृतं तत्स्याद्यदहं तव धीमते}
{पित्रे दत्ता विमोक्षार्थं कथं वा पुत्र मन्यसे ॥सौतिरुवाच}
{}


\twolineshloka
{एवमुक्तस्तथेत्युक्त्वा सास्तीको मातरं तदा}
{अब्रवीद्दुःखसंतप्तं वासुकिं जीवयन्निव}


\twolineshloka
{अहं त्वां मोक्षयिष्यामि वासुके पन्नगोत्तम}
{तस्माच्छापान्महासत्त्व सत्यमेतद्ब्रवीमि ते}


\twolineshloka
{भव स्वस्थमना नाग न हि ते विद्यते भयम्}
{प्रयतिष्ये तथा राजन्यथा श्रेयो भविष्यति}


\twolineshloka
{न मे वागनृतं प्राह स्वैरेष्वपि कुतोऽन्यथा}
{तं वै नृपवरं गत्वा दीक्षितं जनमेजयम्}


\twolineshloka
{वाग्भिर्मङ्गलयुक्ताभिस्तोषयिष्येऽद्य मातुल}
{यथा स यज्ञो नृपतेर्निवत्रिष्यति सत्तम}


\threelineshloka
{स संभावय नागेन्द्र मयि सर्वं महामते}
{न ते मयि मनो जातु मिथ्या भवितुमर्हति ॥वासुकिरुवाच}
{}


\threelineshloka
{आस्तीक परिघूर्णामि हृदयं मे विदीर्यते}
{दिशो न प्रतिजानामि ब्रह्मदण्डनिपीडितः ॥आस्तीक उवाच}
{}


\twolineshloka
{न सन्तापस्त्वया कार्यः कथंचित्पन्नगोत्तम}
{प्रदीप्ताग्नेः समुत्पन्नं नाशयिष्यामि ते भयम्}


\threelineshloka
{ब्रह्मदण्डं महाघोरं कालाग्निसमतेजसम्}
{नाशयिष्यामि माऽत्र त्वं भयंकार्षीः कथंचन ॥सौतिरुवाच}
{}


\twolineshloka
{ततः स वासुकेर्घोरमपनीय मनोज्वरम्}
{आधाय चात्मनोऽङ्गेषु जगाम त्वरितो भृशम्}


\twolineshloka
{जनमेजयस्य तं यज्ञं सर्वैः समुदितं गुणैः}
{मोक्षाय भुजगेन्द्राणामास्तीको द्विजसत्तमः}


\twolineshloka
{स गत्वाऽपश्यदास्तीको यज्ञायतनमुत्तमम्}
{वृतं सदस्यैर्बहुभिः सूर्यवह्निसमप्रभैः}


\twolineshloka
{स तत्र वारितो द्वास्थैः प्रविशन्द्विजसत्तमः}
{अभितुष्टाव तं यज्ञं प्रवेशार्थी परंतपः}


\twolineshloka
{स प्राप्य यज्ञायतनं वरिष्ठंद्विजोत्तमः पुण्यकृतां वरिष्ठः}
{तुष्टाव राजानमनन्तकीर्ति-मृत्विक्सदस्यांश्च तथैव चाग्निम्}


\chapter{अध्यायः ५५}
\twolineshloka
{आस्तीक उवाच}
{}


\twolineshloka
{सोमस्य यज्ञो वरुणस्य यज्ञःप्रजापतेर्यज्ञ आसीत्प्रयागे}
{तथा यज्ञोऽयं तव भारताग्र्यपारिक्षित स्वस्ति नोऽस्तु प्रियेभ्यः}


\twolineshloka
{शक्रस्य यज्ञः शतसङ्ख्य उक्त-स्तथापरं तुल्यसङ्ख्यं शतं वै}
{तथा यज्ञोऽयं तव भारताग्र्यपारिक्षित स्वस्ति नोऽस्तु प्रियेभ्यः}


\twolineshloka
{यमस्य यज्ञो हरिमेधसश्चयथा यज्ञो रन्तिदेवस्य राज्ञः}
{तथा यज्ञोऽयं तव भारताग्र्यपारिक्षित स्वस्ति नोऽस्तु प्रियेभ्यः}


\twolineshloka
{गयस्य यज्ञः शशबिन्दोश्च राज्ञोयज्ञसल्तथा वैश्रवणस्य राज्ञः}
{तथा यज्ञोऽयं तव भारताग्र्यपारिक्षित स्वस्ति नोऽस्तु प्रियेभ्यः}


\twolineshloka
{नृगस्य यज्ञस्त्वजमीढस्य चासी-द्यथा यज्ञो दाशरथेश्च राज्ञः}
{तथा यज्ञोऽयं तव भारताग्र्यपारिक्षित स्वस्ति नोऽस्तु प्रियेभ्यः}


\twolineshloka
{यज्ञः श्रुतो दिवि देवस्य सूनो-र्युधिष्ठिरस्याजमीढस्य राज्ञः}
{तथा यज्ञोऽयं तव भारताग्र्यपारिक्षित स्वस्ति नोऽस्तु प्रियेभ्यः}


\twolineshloka
{कृष्णस्य यज्ञः सत्यवत्याः सुतस्यस्वयं च कर्म प्रचकार यत्र}
{तथा यज्ञोऽयं तव भारताग्र्यपारिक्षित स्वस्ति नोऽस्तु प्रियेभ्यः}


\twolineshloka
{इमे च ते सूर्यसमानवर्चसःसमासते वृत्रहणः क्रतुं यथा}
{नैषां ज्ञातुं विद्यते ज्ञानमद्यदत्तं येभ्यो न प्रणश्येत्कदाचित्}


\twolineshloka
{ऋत्विक्समो नास्ति लोकेषु चैवद्वैपायनेनेति विनिश्चितं मे}
{एतस्य शिष्या हि क्षितिं संचरन्तिसर्वर्त्विजः कर्मसु स्वेषु दक्षाः}


\twolineshloka
{विभावसुश्चित्रभानुर्महात्माहिरण्यरेता हुतभुक्कृष्णवर्त्मा}
{प्रदक्षिमावर्तशिखः प्रदीप्तोहव्यं तवेदं हुतभुग्वष्टि देवः}


\twolineshloka
{नैह त्वदन्यो विद्यते जीवलोकेसमो नृपः पालयिता प्रजानाम्}
{धृत्या च ते प्रतीमनाः सदाहंत्वं वा वरुणो धर्मराजो यमो वा}


\twolineshloka
{शक्रः साक्षाद्वज्रपाणिर्यथेहत्राता लोकेऽस्मिंस्त्वं तथेह प्रजानाम्}
{मतस्त्वं नः पुरुषेन्द्रेह लोकेन च त्वदन्यो भूपतिरस्ति जज्ञे}


\twolineshloka
{खट्वाङ्गनाभगदिलीपकल्पययातिमान्धातृसमप्रभाव}
{आदित्यतेजःप्रतिमानतेजाभीष्मो यथा राजसि सुव्रतस्त्वम्}


\twolineshloka
{वाल्मीकिवत्ते निभृतं स्ववीर्यंवसिष्ठवत्ते नियतश्च कोपः}
{प्रभुत्वमिन्द्रत्वसमं मतं मेद्युतिश्च नारायणवद्विभाति}


\twolineshloka
{यमो यथा धर्मविनिश्चयज्ञःकृष्णो यथा सर्वगुणोपपन्नः}
{श्रियां निवासोऽसि यथा वसूनांनिधानभूतोऽसि तथा क्रतूनाम्}


\threelineshloka
{दम्भोद्भवेनासि समो बलेनरामो यथा शस्त्रविदस्त्रविच्च}
{और्वत्रिताभ्यामसि तुल्यतेजादुष्प्रेक्षणीयोऽसि भगीरथेन ॥सौतिरुवाच}
{}


\twolineshloka
{एवं स्तुताः सर्व एव प्रसन्नाराजा सदस्या ऋत्विजो हव्यवाहः}
{तेषां दृष्ट्वा भावितानीङ्गितानिप्रोवाच राजा जनमेजयोऽथ}


\chapter{अध्यायः ५६}
\twolineshloka
{जनमेजय उवाच}
{}


\threelineshloka
{बालोऽप्ययं स्थविर इवावभाषतेनायं बालः स्थविरोऽयं मतो मे}
{इच्छाम्यहं वरमस्मै प्रदातुंतन्मे विप्राः संविदध्वं यथावत् ॥सदस्या ऊचुः}
{}


\fourlineindentedshloka
{बालोऽपि विप्रो मान्य एवेह राज्ञा`यश्चाविद्वान्यश्च विद्वान्यथावत्}
{प्रसादयैनं त्वरितो नरेन्द्रद्विजातिवर्यं सकलार्थसिद्धये}
{'सर्वान्कामांस्त्वत्त एवार्हतेऽद्ययथा च नस्तक्षक एति शीघ्रम् ॥सौतिरुवाच}
{}


\threelineshloka
{व्याहर्तुकामे वरदे नृपे द्विजंवरं वृणीष्वेति ततोऽब्युवाच}
{होता वाक्यं नातिहृष्टान्तरात्माकर्मण्यस्मिंस्तक्षको नैति तावत् ॥जनमेजय उवाच}
{}


\threelineshloka
{यथा चेदं कर्म समाप्यते मेयथा च वै तक्षक एति शीघ्रम्}
{तथा भवन्तः प्रयतन्तु सर्वेपरं शक्त्या स हि मे विद्विषाणः ॥ऋत्विज ऊचुः}
{}


\twolineshloka
{यथा शस्त्राणि नः प्राहुर्यथा शंसति पावकः}
{इन्द्रस्य भवने राजंस्तक्षको भयपीडितः}


\twolineshloka
{यथा सूतो लोहिताक्षो महात्मापौराणिको वेदितवान्पुरस्तात्}
{स राजानं प्राह पृष्टस्तदानींयथाहुर्विप्रास्तद्वदेतन्नृदेव}


\twolineshloka
{पुराणमागम्य ततो ब्रवीम्यहंदत्तं तस्मै वरमिन्द्रेण राजन्}
{वसेह त्वं मत्सकाशे सुगुप्तोन पावकस्त्वां प्रदहिष्यतीति}


\threelineshloka
{एतच्छ्रुत्वा दीक्षितस्तप्यमानआस्ते होतारं चोदयन्कर्म काले}
{`इन्द्रेण सार्धं तक्षकं पातयध्वंविभावसौ न विमुच्येत नागः}
{'होता च यत्तोऽस्याजुहावाथ मन्त्रै-रथो महेन्द्रः स्वयमाजगाम}


\twolineshloka
{`आयातु चेन्द्रोऽपि सतक्षकः पते-द्विभावसौ नागराजेन तूर्णम्}
{जम्भस्य हन्तेति जुहाव होतातदा जगामाहिदत्ताभयः प्रभुः ॥'}


\twolineshloka
{विमानमारुह्य महानुभावःसर्वैर्देवैः परिसंस्तूयमानः}
{बलाहकैश्चाप्यनुगम्यमानोविद्याधरैरप्सरसां गणैश्च`नागस्य नाशो मम चैव नाशोभविष्यतीत्येव विचिन्तयानः ॥'}


\threelineshloka
{तस्योत्तरीये निहितः स नागोभयोद्विग्नः शर्म नैवाभ्यगच्छत्}
{ततो राजा मन्त्रविदोऽब्रवीत्पुनःक्रुद्धो वाक्यं तक्षकस्यान्तमिच्छन् ॥जनमेजय उवाच}
{}


\threelineshloka
{इन्द्रस्य भवने विप्रा यदि नागः स तक्षकः}
{तमिन्द्रेणैव सहितं पातयध्यं विभावसौ ॥सौतिरुवाच}
{}


\twolineshloka
{जनमेजयेन राज्ञा तु नोदितस्तक्षकं प्रति}
{होता जुहाव तत्रस्थं तक्षकं पन्नगं तथा}


\twolineshloka
{हूयमाने तथा चैव तक्षकः सपुरन्दरः}
{आकाशे ददृशे तत्र क्षणेन व्यथितस्तदा}


\twolineshloka
{पुरन्दरस्तु तं यज्ञं दृष्ट्वोरुभयमाविशत्}
{हित्वा तु तक्षकं त्रस्तः स्वमेव भवनं ययौ}


\fourlineindentedshloka
{इन्द्रे गते तु राजेन्द्र तक्षको भयमोहितः}
{मन्त्रशक्त्या पावकार्चिस्समीपमवशो गतः}
{`तं दृष्ट्वा ऋत्विजस्तत्र वचनं चेदमब्रुवन्' ॥ऋत्विज ऊचुः}
{}


\twolineshloka
{अयमायाति तूर्णं स तक्षकस्ते वशं नृप}
{श्रूयतेऽस्य महान्नादो नदतो भैरवं रवम्}


\twolineshloka
{नूनं मुक्तो वज्रभृता स नागोभ्रष्टो नाकान्मन्त्रवित्रस्तकायः}
{घूर्णन्नाकाशे नष्टसंज्ञोऽभ्युपैतितीव्रान्निश्वासान्निश्वसन्पन्नगेन्द्रः}


\threelineshloka
{वर्तते तव राजेन्द्र कर्मैतद्विधिवत्प्रभो}
{अस्मै तु द्विजमुख्याय वरं त्वं दातुमर्हसि ॥जनमेजय उवाच}
{}


\threelineshloka
{बालाभिरूपस्य तवाप्रमेयवरं प्रयच्छामि यथानुरूपम्}
{वृणीष्व यत्तेऽभिमतं हृदि स्थितंतत्ते प्रदास्याम्यपि चेददेयम् ॥सौतिरुवाच}
{}


\threelineshloka
{पतिष्यमाणे नागेन्द्रे तक्षके जातवेदसि}
{इदमन्तरमित्येवं तदास्तीकोऽभ्यचोदयत् ॥आस्तीक उवाच}
{}


\twolineshloka
{वरं ददासि चेन्मह्यं वृणोमि जनमेजय}
{सत्रं ते विरमत्वेतन्न पतेयुरिहोरगाः}


\twolineshloka
{एवमुक्तस्तदा तेन ब्रह्मन्पारिक्षितस्तु सः}
{नातिहृष्टमनाश्चेदमास्तीकं वाक्यमब्रवीत्}


\threelineshloka
{सुवर्णं रजतं गाश्च यच्चान्यन्मन्यसे विभो}
{तत्ते दद्यां वरं विप्र न निवर्तेत्क्रतुर्मम ॥आस्तीक उवाच}
{}


\threelineshloka
{सुवर्णं रजतं गाश्च न त्वां राजन्वृणोम्यहम्}
{सत्रं ते विरमत्वेतत्स्वस्ति मातृकुलस्य नः ॥सौतिरुवाच}
{}


\twolineshloka
{आस्तीकेनैवमुक्तस्तु राजा पारिक्षितस्तदा}
{पुनःपुनरुवाचेदमास्तीकं वदतां वरः}


\twolineshloka
{अन्यं वरय भद्रं ते वरं द्विज्वरोत्तम}
{अयाचत न चाप्यन्यं वरं स भृगुनन्दन}


\twolineshloka
{ततो वेदविदस्तात सदस्याः सर्व एव तम्}
{राजानमूचुः सहिता लभतां ब्राह्मणो वरम्}


\chapter{अध्यायः ५७}
\twolineshloka
{शौनक उवाच}
{}


\threelineshloka
{ये सर्पाः सर्पसत्रेऽस्मिन्पतिता हव्यवाहने}
{तेषां नामानि सर्वेषां श्रोतुमिच्छामि सूतज ॥सौतिरुवाच}
{}


\twolineshloka
{सहस्राणि बहून्यस्मिन्प्रयुतान्यर्बुदानि च}
{न शक्यं परिसंख्यातुं बहुत्वाद्द्विजसत्तम}


\twolineshloka
{यथास्मृति तु नामानि पन्नगानां निबोध मे}
{उच्यमानानि मुख्यानां हुतानां जातवेदसि}


\twolineshloka
{वासुकेः कुलजातांस्तु प्राधान्येन निबोध मे}
{नीलरक्तान्सितान्घोरान्महाकायान्विषोल्बणान्}


\twolineshloka
{अवशान्मातृवाग्दण्डपीडितान्कृपणान्हूतान्}
{कोटिशो मानसः पूर्णः शलः पालो हलीमकः}


\twolineshloka
{पिच्छलः कौणपश्चक्रः कालवेगः प्रकालनः}
{हिरण्यबाहुः शरणः कक्षकः कालदन्तकः}


\threelineshloka
{एते वासुकिजा नागाः प्रविष्टा हव्यवाहने}
{अन्ये च बहवो विप्र तथा वै कुलसंभवाः}
{प्रदीप्ताग्नौ हुताःसर्वे घोररूपा महाबलाः}


\twolineshloka
{तक्षकस्य कुले जातान्प्रवक्ष्यामि निबोध तान्}
{पुच्छाण्डको मण्डलकः पिण्डसेक्ता रभेणकः}


\twolineshloka
{उच्छिखः शरभो भङ्गो बिल्वतेजा विरोहणः}
{शिली शलकरो मूकः सुकुमारः प्रवेपनः}


\twolineshloka
{मुद्गरः शिशुरोमा च सुरोमा च महाहनुः}
{एते तक्षकजा नागाः प्रविष्टा हव्यवाहनम्}


\twolineshloka
{पारावतः पारियात्रः पाण्डरो हरिणः कृशः}
{विहङ्गः शरभो मोदः प्रमोदः संहतापनः}


\twolineshloka
{ऐरावतकुलादेते प्रविष्टा हव्यवाहनम्}
{कौरव्यकुलजान्नागाञ्शृणु मे त्वं द्विजोत्तम}


\twolineshloka
{एरकः कुण्डलो वेणी वेणीस्कन्धः कुमारकः}
{बाहुकः शृङ्गबेरश्च धूर्तकः प्रातरातकौ}


\twolineshloka
{कौरव्यकुलजास्त्वेते प्रविष्टा हव्यवाहनम्}
{धृतराष्ट्रकुले जाताञ्शृणु नागान्यथातथम्}


\twolineshloka
{कीर्त्यमानान्मया ब्रह्मन्वातवेगान्विषोल्बणान्}
{शङ्कुकर्णः पिठरकः कुठारमुखसेचकौ}


\twolineshloka
{पूर्णाङ्गदः पूर्णमुखः प्रहासः शकुनिर्दरिः}
{अमाहठः कामठकः सुषेणो मानसोऽव्ययः}


\threelineshloka
{`अष्टावक्रः कोमलकः श्वसनो मौनवेपगः}
{'भैरवो मुण्डवेदाङ्गः पिशङ्गश्चोदपारकः}
{ऋषभो वेगवान्नागः पिण्डारकमहाहनू}


\twolineshloka
{रक्ताङ्गः सर्वसारङ्गः समृद्धपटवासकौ}
{वराहको वीरणकः सुचित्रश्चित्रवेगिकः}


\twolineshloka
{पराशरस्तरुणको मणिः स्कन्धस्तथाऽऽरुणिः}
{इति नागा मया ब्रह्मन्कीर्तिताः कीर्तिवर्धनाः}


\twolineshloka
{प्राधान्येन बहुत्वात्तु न सर्वे परिकीर्तिताः}
{एतेषां प्रसवो यश्च प्रसवस्य च सन्ततिः}


\threelineshloka
{न शक्यं परिसंख्यातुं ये दीप्तं पावकं गताः}
{`द्विशीर्षाः पञ्चशीर्षाश्च सप्तशीर्षास्तथाऽपरे}
{दशशीर्षाः शतशीर्षास्तथान्ये बहुशीर्षकाः'}


\twolineshloka
{कालानलविषा घोरा हुताः शतसहस्रशः}
{महाकाया महावेगाः शैलशृङ्गसमुच्छ्रयाः}


\twolineshloka
{योजनायामविस्तारा द्वियोजनसमायताः}
{कामरूपाः कामबला दीप्तानलविषोल्बणाः}


% Check verse!
दग्धास्तत्र महासत्रे ब्रह्मदण्डनिपाडिताः
\chapter{अध्यायः ५८}
\twolineshloka
{सौतिरुवाच}
{}


\twolineshloka
{इदमत्यद्भुतं चान्यदास्तीकस्यानुशुश्रुम}
{तथा वरैश्छन्द्यमाने राज्ञा पारिक्षितेन हि}


\twolineshloka
{इन्द्रहस्ताच्च्युतो नागः ख एव यदतिष्ठत}
{ततश्चिन्तापरो राजा बभूव जनमेजयः}


\threelineshloka
{हूयमाने भृशं दीप्ते विधिवद्वसुरेतसि}
{न स्म स प्रापतद्वह्नौ तक्षको भयपीडितः ॥शौनक उवाच}
{}


\threelineshloka
{किं सूत तेषां विप्राणां मन्त्रग्रामो मनीषिणाम्}
{न प्रत्यभात्तदाऽग्नौ यत्स पपात न तक्षकः ॥सौतिरुवाच}
{}


\twolineshloka
{तमिन्द्रहस्ताद्वित्रस्तं विसंज्ञं पन्नगोत्तमम्}
{आस्तीकस्तिष्ठ तिष्ठेति वाचस्तिस्रोऽभ्युदैरयत्}


\twolineshloka
{वितस्थे सोऽन्तरिक्षे च हृदयेन विदूयता}
{यथा तिष्ठति वै कश्चित्खं च गां चान्तरा नरः}


\twolineshloka
{ततो राजाब्रवीद्वाक्यं सदस्यैश्चोदितो भृशम्}
{काममेतद्भवत्वेवं यथास्तीकस्य भाषितम्}


\twolineshloka
{समाप्यतामिदं कर्म पन्नगाः सन्त्वनामयाः}
{प्रीयतामयमास्तीकः सत्यं सूतवचोऽस्तु तत्}


\twolineshloka
{ततो हलहलाशब्दः प्रीतिजः समजायत}
{आस्तीकस्य वरे दत्ते तथैवोपरराम च}


\twolineshloka
{स यज्ञः पाण्डवेयस्य राज्ञः पारिक्षितस्य ह}
{प्रीतिमांश्चाभवद्राजा भारतो जनमेजयः}


\twolineshloka
{ऋत्विग्भ्यः ससदस्येभ्यो ये तत्रासन्समागताः}
{तेभ्यश्च प्रददौ वित्तं शतशोऽथ सहस्रशः}


\twolineshloka
{लोहिताक्षाय सूताय तथा स्थपतये विभुः}
{येनोक्तं तस्य तत्राग्रे सर्पसत्रनिवर्तने}


\twolineshloka
{निमित्तं ब्राह्मण इति तस्मै वित्तं ददौ बहु}
{दत्त्वा द्रव्यं यथान्यायं भोजनाच्छादनान्वितम्}


\twolineshloka
{प्रीतस्तस्मै नरपतिरप्रमेयपराक्रमः}
{ततश्चकारावभृथं विधिदृष्टेन कर्मणा}


\twolineshloka
{आस्तीकं प्रेषयामास गृहानेव सुसंस्कृतम्}
{राजा प्रीतमनाः प्रीतं कृतकृत्यं मनीषिणम्}


\twolineshloka
{पुनरागमनं कार्यमिति चैनं वचोऽब्रवीत्}
{भविष्यसि सदस्यो मे वाजिमेधे महाक्रतौ}


\twolineshloka
{तथेत्युक्त्वा प्रदुद्राव तदास्तीको मुदा युतः}
{कृत्वा स्वकार्यमतुलं तोषयित्वा च पार्थिवम्}


\threelineshloka
{स गत्वा परमप्रीतो मातुलं मातरं च ताम्}
{अभिगम्योपसंगृह्य तथा वृत्तं न्यवेदयत् ॥सौतिरुवाच}
{}


\twolineshloka
{एतच्छ्रुत्वा प्रीयमाणाः समेताये तत्रासन्पन्नगा वीतमोहाः}
{आस्तीके वै प्रीतिमन्तो बभूवु-रूचुश्चैनं वरमिष्टं वृमीष्व}


\threelineshloka
{भूयोभूयः सर्वशस्तेऽब्रुवंस्तंकिं ते प्रियं करवामाद्य विद्वन्}
{प्रीता वयं मोक्षिताश्चैव सर्वेकामं किं ते करवामाद्य वत्स ॥आस्तीक उवाच}
{}


\twolineshloka
{सायं प्रातर्ये प्रसन्नात्मरूपालोके विप्रा मानवा ये परेऽपि}
{धर्माख्यानं ये पठेयुर्ममेदंतेषां युष्मन्नैव किंचिद्भयं स्यात्}


\twolineshloka
{तैश्चाप्युक्तो भागिनेयः प्रसन्नै-रेतत्सत्यं काममेवं वरं ते}
{प्रीत्या युक्ताः कामितं सर्वशस्तेकर्तारः स्म प्रवणा भागिनेय}


\twolineshloka
{असितं चार्तिमन्तं च सुनीथं चापि यः स्मरेत्}
{दिवा वा यदि वा रात्रौ नास्य सर्पभयं भवेत्}


\threelineshloka
{यो जरत्कारुणा जातो जरत्कारौ महावशाः}
{आस्तीकः सर्पसत्रे वः पन्नगान्योऽभ्यरक्षत}
{तं स्मरन्तं महाभागा न मां हिंसितुमर्हथ}


\twolineshloka
{सर्पापसर्प भद्रं ते गच्छ सर्प महाविष}
{जनेमेजयस्य यज्ञान्ते आस्तीकवचनं स्मर}


\threelineshloka
{आस्तीकस्य वचः श्रुत्वा यः सर्पो न निवर्तते}
{शतधा भिद्यते मूर्धा शिंशवृक्षफलं यथा ॥सौतिरुवाच}
{}


\twolineshloka
{स एवमुक्तस्तु तदा द्विजेन्द्रःसमागतैस्तैर्भुजगेन्द्रमुख्यैः}
{संप्राप्य प्रीतिं विपुलां महात्माततो मनो गमनायाथ दध्रे}


\twolineshloka
{`इत्येवं नागराजोऽथ नागानां मध्यगस्तदा}
{उक्त्वा सहैव तैः सर्पैः स्वमेव भवनं ययौ ॥'}


\twolineshloka
{मोक्षयित्वा तु भुजगान्सर्पसत्राद्द्विजोत्तमः}
{जगाम काले धर्मात्मा दिष्टान्तं पुत्रपौत्रवान्}


\threelineshloka
{इत्याख्यानं मयास्तीकं यथावत्तव कीर्तितम्}
{यत्कीर्तयित्वा सर्पेभ्यो न भयं विद्यते क्वचित् ॥सौतिरुवाच}
{}


\twolineshloka
{यथा कथितवान्ब्रह्मन्प्रमतिः पूर्वजस्तव}
{पुत्राय रुरवे प्रीतः पृच्छते भार्गवोत्तम}


\twolineshloka
{यद्वाक्यं श्रुतवांश्चाहं तथा च कथितं मया}
{आस्तीकस्य कवेर्विप्र श्रीमच्चरितमादितः}


\twolineshloka
{यन्मां त्वं पृष्टवान्ब्रह्मञ्श्रुत्वा डुण्डुभभाषितम्}
{व्येतु ते सुमहद्ब्रह्मन्कौतूहलमरिन्दम}


\twolineshloka
{श्रुत्वा धर्मिष्ठमाख्यानमास्तीकं पुण्यवर्धनम्}
{`सर्वपापविनिर्मुक्तो दीर्घमायुरवाप्नुयात् ॥'}


\chapter{अध्यायः ५९}
\twolineshloka
{शौनक उवाच}
{}


\twolineshloka
{भृगुवंशात्प्रभृत्येव त्वया मे कीर्तितं महत्}
{आख्यानमखिलं तात सौते प्रीतोऽस्मितेन ते}


\twolineshloka
{प्रक्ष्यामि चैव भूयस्त्वां यथावत्सूतनन्दन}
{याः कथा व्याससंपन्नास्ताश्च भूयो विचक्ष्व मे}


\twolineshloka
{तस्मिन्परमदुष्पारे सर्पसत्रे महात्मनाम्}
{कर्मान्तरेषु यज्ञस्य सदस्यानां तथाऽध्वरे}


\threelineshloka
{या बभूवुः कथाश्चित्रा येष्वर्थेषु यथातथम्}
{त्वत्त इच्छामहे श्रोतुं सौते त्वं वै प्रचक्ष्व नः ॥सौतिरुवाच}
{}


\threelineshloka
{कर्मान्तरेष्वकथयन्द्विजा वेदाश्रयाः कथाः}
{व्यासस्त्वकथयच्चित्रमाख्यानं भारतं महत् ॥शौनक उवाच}
{}


\twolineshloka
{महाभारतमाख्यानं पाण्डवानां यशस्करम्}
{जनमेजयेन पृष्टः सन्कृष्णद्वैपायनस्तदा}


\twolineshloka
{श्रावयामास विधिवत्तदा कर्मान्तरे तु सः}
{तामहं विधिवत्पुण्यां श्रोतुमिच्छामि वै कथाम्}


\threelineshloka
{मनःसागरसंभूतां महर्षेर्भावितात्मनः}
{कथयस्व सतां श्रेष्ठ सर्वरत्नमयीमिमाम् ॥सौतिरुवाच}
{}


\twolineshloka
{हन्त ते कथयिष्यामि महदाख्यानमुत्तमम्}
{कृष्णद्वैपायनमतं महाभारतमादितः}


\twolineshloka
{शृणु सर्वमशेषेण कथ्यमानं मया द्विज}
{शंसितुं तन्महान्हर्षो ममापीह प्रवर्तते}


\chapter{अध्यायः ६०}
\twolineshloka
{सौतिरुवाच}
{}


\twolineshloka
{श्रुत्वा तु सर्पसत्राय दीक्षितं जनमेजयम्}
{अभ्यगच्छदृषिर्विद्वान्कृष्णद्वैपायनस्तदा}


\twolineshloka
{जनयामास यं काली शक्तेः पुत्रात्पराशरात्}
{कन्यैव यमुनाद्वीपे पाण्डवानां पितामहम्}


\twolineshloka
{जातमात्रश्च यः सद्य इष्ट्या देहमवीवृधत्}
{वेदांश्चाधिजगे साङ्गन्सेतिहासान्महायशाः}


\twolineshloka
{यं नाति तपसा कश्चिन्न वेदाध्ययनेन च}
{न व्रतैर्नोपवासैश्च न प्रसूत्या न मन्युना}


\twolineshloka
{विव्यासैकं चतुर्धा यो वेदं वेदविदां वरः}
{परावरज्ञो ब्रह्मर्षिः कविः सत्यव्रतः शुचिः}


\twolineshloka
{यः पाण्डुं धृतराष्ट्रं च विदुरं चाप्यजीजनत्}
{शन्तनोः संततिं तन्वन्पुण्यकीर्तिर्महायशाः}


\twolineshloka
{जनमेजयस्य राजर्षेः स महात्मा सदस्तथा}
{विवेश सहितः शिष्यैर्वेदवेदाङ्गपारगैः}


\twolineshloka
{तत्र राजानमासीनं ददर्श जनमेजयम्}
{वृतं सदस्यैर्बहुभिर्देवैरिव पुरन्दरम्}


\twolineshloka
{तथा मूर्धाभिषिक्तैश्च नानाजनपदेश्वरैः}
{ऋत्विग्भिर्ब्रह्मकल्पैश्च कुशलैर्यज्ञसंस्तरे}


\twolineshloka
{जनमेजयस्तु राजर्षिर्दृष्ट्वा तमृषिमागतम्}
{सगणोऽभ्युद्ययौ तूर्णं प्रीत्या भरतसत्तमः}


\twolineshloka
{काञ्चनं विष्टरं तस्मै सदस्यानुमतः प्रभुः}
{आसनं कल्पयामास यथा शक्रो बृहस्पतेः}


\twolineshloka
{तत्रोपविष्टं वरदं देवर्षिगणपूजितम्}
{पूजयामास राजेन्द्रः शास्त्रदृष्टेन कर्मणा}


\twolineshloka
{पाद्यमाचमनीयं च अर्घ्यं गां च विधानतः}
{पितामहाय कृष्णाय तदर्हाय न्यवेदयत्}


\twolineshloka
{प्रतिगृह्य तु तां पूजां पाम्डवाज्जनमेजयात्}
{गां चैव समनुज्ञाय व्यासः प्रीतोऽभवत्तदा}


\twolineshloka
{तथा च पूजयित्वा तं प्रणयात्प्रतितामहम्}
{उपोपविश्य प्रीतात्मा पर्यपृच्छदनामयम्}


\twolineshloka
{भगवानापि तं दृष्ट्वा कुशलं प्रतिवेद्य च}
{सदस्यैः पूजितः सर्वैः सदस्यान्प्रत्यपूजयत्}


\threelineshloka
{ततस्तु सहितः सर्वैः सदस्यैर्जनमेजयः}
{इदं पश्चाद्द्विजश्रेष्ठं पर्यपृच्छत्कृताञ्जलिः ॥जनमेजय उवाच}
{}


\twolineshloka
{कुरूणां पाण्डवानां च भवान्प्रत्यक्षदर्शिवान्}
{तेषां चरितमिच्छामि कथ्यमानं त्वया द्विज}


\twolineshloka
{कथं समभवद्भेदस्तेषामक्लिष्टकर्मणाम्}
{तच्च युद्धं कथं वृत्तं भूतान्तकरणं महत्}


\fourlineindentedshloka
{पितामहानां सर्वेषां दैवेनाविष्टचेतसाम्}
{कार्त्स्न्येनैतन्ममाचक्ष्व यथा वृत्तं द्विजोत्तम}
{`इच्छामि तत्त्वतः श्रोतुं भगवन्कुशलो ह्यसि' ॥सौतिरुवाच}
{}


\threelineshloka
{तस्य तद्वचनं श्रुत्वा कृष्णद्वैपायनस्तदा}
{शशास शिष्यमासीनं वैशंपायनमन्तिके ॥व्यास उवाच}
{}


\twolineshloka
{कुरूणां पाण्डवानां च यथा भेदोऽभवत्पुरा}
{तदस्मै सर्वमाचक्ष्व यन्मत्तः श्रुतवानसि}


\twolineshloka
{गुरोर्वचनमाज्ञाय स तु विप्रर्षभस्तदा}
{आचचक्षे ततः सर्वमितिहासं पुरातनम्}


\twolineshloka
{राज्ञे तस्मै सदस्येभ्यः पार्थिवेभ्यश्च सर्वशः}
{भेदं सर्वविनाशं च कुरुपाण्डवयोस्तदा}


\chapter{अध्यायः ६१}
\twolineshloka
{वैशंपायन उवाच}
{}


\twolineshloka
{`शृणु राजन्यथा वीरा भ्रातरः पञ्च पाण्डवाः}
{विरोधमन्वगच्छन्त धार्तराष्ट्रैर्दुरात्मभिः ॥'}


\twolineshloka
{गुरवे प्राङ्नमस्कृत्य मनोबुद्धिसमाधिभिः}
{संपूज्य च द्विजान्सर्वांस्तथान्यान्विदुषो जनान्}


\twolineshloka
{महर्षेर्विश्रुतस्येह सर्वलोकेषु धीमतः}
{प्रवक्ष्यामि मतं कृत्स्नं व्यासस्यामिततेजसः}


\twolineshloka
{श्रोतुं पात्रं च राजंस्त्वं प्राप्येमां भारतीं कथाम्}
{गुरोर्वक्त्रपरिस्पन्दो मनः प्रोत्साहतीव मे}


\twolineshloka
{शृणु राजन्यथा भेदः कुरुपाण्डवयोरभूत्}
{राज्यार्थे द्यूतसंभूतो वनवासस्तथैव च}


\twolineshloka
{यथा च युद्धमभवत्पृथिवीक्षयकारकम्}
{तत्तेऽहं कथयिष्यामि पृच्छते भरतर्षभ}


\twolineshloka
{मृते पितरि ते वीरा वनादेत्य स्वमन्दिरम्}
{नचिरादेव विद्वांसो वेदे धनुषि चाभवन्}


\twolineshloka
{तांस्तथा सत्ववीर्यौजःसंपन्नान्पौरसंमतान्}
{नामृष्यन्कुरवो दृष्ट्वा पाण्डवाञ्श्रीयशोभृतः}


\twolineshloka
{ततो दुर्योधनः क्रूरः कर्णश्च सहसौबलः}
{तेषां निग्रहनिर्वासान्विविधांस्ते समारभन्}


\twolineshloka
{ततो दुर्योधनः क्रूरः कुलिङ्गस्य मते स्थितः}
{पाम्डवान्विविधोपायै राज्यहेतोरपीडयत्}


\twolineshloka
{ददावथ विषं पापो भीमाय धृतराष्ट्रजः}
{जरयामास तद्वीरः सहान्नेन वृकोदरः}


\twolineshloka
{प्रमाणकोट्यां संसुप्तं पुनर्बद्ध्वा वृकोदरम्}
{तोयेषु भीमं गङ्गायाः प्रक्षिप्य पुरमाव्रजत्}


\twolineshloka
{यदा विबुद्धः कौन्तेयस्तदा संछिद्य बन्धनम्}
{उदतिष्ठन्महाबाहुर्भीमसेनो गतव्यथः}


\twolineshloka
{आशीविषैः कृष्णसर्पैः सुप्तं चैनमदंशयत्}
{सर्वेष्वेवाङ्गदेशेषु न ममार स शत्रुहा}


\twolineshloka
{तेषां तु विप्रकारेषु तेषु तेषु महामतिः}
{मोक्षणे प्रतिकारे च विदुरोऽवहितोऽभवत्}


\twolineshloka
{स्वर्गस्थो जीवलोकस्य यथा शक्रः सुखावहः}
{पाण्डवानां तथा नित्यं विदुरोऽपि सुखावहः}


\twolineshloka
{यदा तु विविधोपायैः संवृतैर्विवृतैरपि}
{नाशकद्विनिहन्तुं तान्दैवभाव्यर्थरक्षितान्}


\twolineshloka
{ततः संमन्त्र्य सचिवैर्वृषदुःशासनादिभिः}
{धृतराष्ट्रमनुज्ञाप्य जातुषं गृहमादिशत्}


\threelineshloka
{`तत्र तान्वासयामास पाण्डवानमितौजसः}
{'सुतप्रियैषी तान्राजा पाण्डवानम्बिकासुतः}
{ततो विवासयामास राज्यभोगबुभुक्षया}


\twolineshloka
{ते प्रातिष्ठन्त सहिता नगरान्नागसाह्वयात्}
{प्रस्थाने चाभवन्मन्त्री क्षत्ता तेषां महात्मनाम्}


\twolineshloka
{तेन मुक्ता जतुगृहान्निशीथे प्राद्रवन्वनम्}
{ततः संप्राप्य कौन्तेया नगरं वारणावतम्}


\twolineshloka
{न्यवसन्त महात्मानो मात्रा सह परन्तपाः}
{धृतराष्ट्रेण चाज्ञप्ता उषिता जातुषे गृहे}


\twolineshloka
{पुरोचनाद्रक्षमाणाः संवत्सरमतन्द्रिताः}
{सुरुङ्गां कारयित्वा तु विदुरेण प्रचोदिताः}


\twolineshloka
{आदीप्य जातुषं वेश्म दग्ध्वा चैव पुरोचनम्}
{प्राद्रवन्भयसंविग्ना मात्रा सह पन्तपाः}


\twolineshloka
{ददृशुर्दारुमं रक्षो हिडिम्बं वननिर्झरे}
{हत्वा च तं राक्षसेन्द्रं भीताः समवबोधनात्}


\twolineshloka
{निशि संप्राद्रवन्पार्था धार्तराष्ट्रभयार्दिताः}
{प्राप्ता हिडिम्बा भीमेन यत्र जातो घटोत्कचः}


\twolineshloka
{एकचक्रां ततो हत्वा पाण्डवाः संशितव्रताः}
{वेदाध्ययनसंपन्नास्तेऽभवन्ब्रह्मचारिणः}


\twolineshloka
{ते तत्र नियताः कालं कंचिदूषुर्नरर्षभाः}
{मात्रा सहैकचक्रायां ब्राह्मणस्य निवेशने}


\twolineshloka
{तत्राससाद क्षुधितं पुरुषादं वृकोदरः}
{भीमसेनो महाबाहुर्बकं नाम महाबलम्}


\twolineshloka
{तं चापि पुरुषव्याघ्रो बाहुवीर्येण पाण्डवः}
{निहत्य तरसा वीरो नागरान्पर्यसान्त्वयत्}


\twolineshloka
{ततस्ते शुश्रुवुः कृष्णां पञ्चालेषु स्वयंवराम्}
{श्रुत्वा चैवाभ्यगच्छ्त गत्वा चैवालभन्त ताम्}


\twolineshloka
{ते तत्र द्रौपदीं लब्ध्वा परिसंवत्सरोषिताः}
{विदिता हास्तिनपुरं प्रत्याजग्मुररिन्दमाः}


\twolineshloka
{ते उक्ता धृतराष्ट्रेण राज्ञा शान्तनवेन च}
{भ्रातृभिर्विग्रहस्तात कथं वो न भवेदिति}


\twolineshloka
{अस्माभिः खाण्डवप्रस्थे युष्मद्वासोऽनुचिन्तितः}
{तस्माज्जनपदोपेतं सुविभक्तमहापथम्}


\twolineshloka
{वासाय स्वाण्डवप्रस्थं व्रजध्वं गतमत्सराः}
{तयोस्ते वचनाज्जग्मुः सह सर्वैः सुहृज्जनैः}


\twolineshloka
{नगरं खाण्डवप्रस्थं रत्नान्यादाय सर्वशः}
{तत्र ते न्यवसन्पार्थाः संवत्सरगणांन्बहून्}


\twolineshloka
{वशे शस्त्रप्रतापेन कुर्वन्तोऽन्यान्महीभृतः}
{एवं धर्मप्रधानास्ते सत्यव्रतपरायणाः}


\twolineshloka
{अप्रमत्तोत्थिताः क्षान्ताः प्रतपन्तोऽहितान्बहून्}
{अजयद्भीमसेनस्तु दिशं प्राचीं महायशाः}


\twolineshloka
{उदीचीमर्जुनो वीरः प्रतीचीं नकुलस्तथा}
{दक्षिणां सहदेवस्तु विजिग्ये परवीरहा}


\twolineshloka
{एवं चक्रुरिमां सर्वे वशे कृत्स्नां वसुन्धराम्}
{पञ्चभिः सूर्यसङ्काशैः सूर्येण च विराजता}


\twolineshloka
{षट्सूर्येवाभवत्पृथ्वी पाण्डवैः सत्यविक्रमैः}
{ततो निमित्ते कस्मिंश्चिद्धर्मराजो युधिष्ठिरः}


\twolineshloka
{वनं प्रस्थापयामास तेजस्वी सत्यविक्रमः}
{प्राणेभ्योऽपि प्रियतरं भ्रातरं सव्यसाचिनम्}


\twolineshloka
{अर्जुनं पुरुषव्याघ्रं स्थिरात्मानं गुणैर्युतम्}
{स वै संवत्सरं पूर्णं मासं चैकं वने वसन्}


\twolineshloka
{`तीर्थयात्रां च कृतवान्नागकन्यामवाप च}
{पाण्ड्यस्य तनयां लब्ध्वा तत्र ताभ्यांसहोषितः'}


\twolineshloka
{ततोऽगच्छद्धृषीकेशं द्वारवत्यां कदाचन}
{लब्धवांस्तत्र बीभत्सुर्भार्यां राजीवलोचनाम्}


\twolineshloka
{अनुजां वासुदेवस्य सुभद्रां भद्रभाषिणीम्}
{सा शचीव महेन्द्रेण श्रीः कृष्णेनेव सङ्गता}


\twolineshloka
{सुभद्रा युयुजे प्रीत्या पाण्डवेनार्जुनेन ह}
{अतर्पयच्च कौन्तेयः खाण्डवे हव्यवाहनम्}


\twolineshloka
{बीभत्सुर्वासुदेवेन सहितो नृपस्तम}
{नातिभारो हि पार्थस्य केशवेन सहाभवत्}


\twolineshloka
{व्यवसायसहायस्य विष्णोः शत्रुवधेष्विव}
{पार्थायाग्निर्ददौ चापि गाण्डीवं धनुरुत्तमम्}


\twolineshloka
{इषुधी चाक्षयैर्बाणै रथं च कपिलक्षणम्}
{मोक्षयामास बीभत्सुर्मयं यत्र महासुरम्}


\twolineshloka
{स चकार सभां दिव्यां सर्वरत्नसमाचिताम्}
{तस्यां दुर्योधनो मन्दो लोभं चक्रे सुदुर्मतिः}


\twolineshloka
{ततोऽक्षैर्वञ्चयित्वा च सौबलेन युधिष्ठिरम्}
{वनं प्रस्थापयामास सप्तवर्षाणि पञ्च च}


\twolineshloka
{अज्ञातमेकं राष्ट्रे च ततो वर्षं त्रयोदशम्}
{ततश्चतुर्दशे वर्षे याचमानाः स्वकं वसु}


\twolineshloka
{नालभन्त महाराज ततो युद्धमवर्तत}
{ततस्ते क्षत्रमुत्साद्य हत्वा दुर्योधनं नृपम्}


\twolineshloka
{राज्यं विहतभूयिष्ठं प्रत्यपद्यन्त पाण्डवाः}
{`इष्ट्वा क्रतूंश्च विविधानश्वमेधादिकान्बहून्}


\twolineshloka
{धृतराष्ट्रे गते स्वर्गं विदुरे पञ्चतां गते}
{गमयित्वा क्रियां स्वर्ग्यां राज्ञाममिततेजसाम्}


\twolineshloka
{स्वं धाम याते वार्ष्णेये कृष्णदारान्प्ररक्ष्य च}
{महाप्रस्थानिकं कृत्वा गताः स्वर्गमनुत्तमम्'}


\twolineshloka
{एवमेतत्पुरावृत्तं तेषामक्लिष्टकर्मणाम्}
{भेदो राज्यविनाशश्च जयश्च जयतांवर}


\chapter{अध्यायः ६२}
\twolineshloka
{जनमेजय उवाच}
{}


\twolineshloka
{कथितं वै समासेन त्वया सर्वं द्विजोत्तम}
{महाभारतमाख्यानं कुरूणां चरितं महत्}


\twolineshloka
{कथां त्वनघ चित्रार्थां कथयस्व तपोधन}
{विस्तरश्रवणे जातं कौतूहलमतीव मे}


\twolineshloka
{स भवान्विस्तरेणेमां पुनराख्यातुमर्हति}
{न हि तृप्यामि पूर्वेषां शृण्वानश्चरितं महत्}


\twolineshloka
{न तत्कारणमल्पं वै धर्मज्ञा यत्र पाण्डवाः}
{अवध्यान्सर्वशो जघ्नुः प्रशस्यन्ते च मानवैः}


\twolineshloka
{किमर्थं ते नरव्याघ्राः शक्ताः सन्तो ह्यनागसः}
{प्रयुज्यमानान्संक्लेशान्क्षान्तवन्तो दुरात्मनाम्}


\twolineshloka
{कथं नागायुतप्राणो बाहुशाली वृकोदरः}
{परिक्लिश्यन्नपि क्रोधं धृतवान्वै द्विजोत्तम}


\twolineshloka
{कथं सा द्रौपदी कृष्णा क्लिश्यमाना दुरात्मभिः}
{शक्ता सती धार्तराष्ट्रान्नादहत्क्रोधचक्षुषा}


\twolineshloka
{कथं व्यसनिनं द्यूते पार्थौ माद्रीसुतौ तदा}
{अन्वयुस्ते नरव्याघ्रा बाध्यमाना दुरात्मभिः}


\twolineshloka
{कथं धर्मभृतां श्रेष्ठः सुतो धर्मस्य धर्मवित्}
{अनर्हः परमं क्लेशं सोढवान्स युधिष्ठिरः}


\twolineshloka
{कथं च बहुलाः सेनाः पाण्डवः कृष्णसारथिः}
{अस्यन्नेकोऽनयत्सर्वाः पितृलोकं धनंजयः}


\threelineshloka
{एतदाचक्ष्व मे सर्वं यथावृत्तं तपोधन}
{यद्यच्च कृतवन्तस्ते तत्रतत्र महारथाः ॥वैशंपायन उवाच}
{}


\twolineshloka
{क्षणं कुरु महाराज विपुलोऽयमनुक्रमः}
{पुण्याख्यानस्य वक्तव्यः कृष्णद्वैपायनेरितः}


\twolineshloka
{महर्षेः सर्वलोकेषु पूजितस्य महात्मनः}
{प्रवक्ष्यामि मतं कृत्स्नं व्यासस्यामिततेजसः}


\twolineshloka
{इदं शतसहस्रं हि श्लोकानां पुण्यकर्मणाम्}
{सत्यवत्यात्मजेनेह व्याख्यातममितौजसा}


\twolineshloka
{`उपाख्यानैः सह ज्ञेयं श्राव्यं भारतमुत्तमम्}
{संक्षेपेण तु वक्ष्यामि सर्वमेतन्नराधिप}


\threelineshloka
{अध्यायानां सहस्रे द्वे पर्वणां शतमेव च}
{श्लोकानां तु सहस्राणि नवतिश्च दशैव च}
{ततोऽष्टादशभिः पर्वैः संगृहीतं महर्षिणा'}


\twolineshloka
{य इदं श्रावयेद्विद्वान्ये चेदं शृणुयुर्नराः}
{ते ब्रह्मणः स्थानमेत्य प्राप्नुयुर्देवतुल्यताम्}


\twolineshloka
{इदं हि वेदैः समितं पवित्रमपि चोत्तमम्}
{श्राव्याणामुत्तमं चेदं पुराणमृषिसंस्तुतम्}


\twolineshloka
{अस्मिन्नर्थश्च कामश्च निखिलेनोपदेक्ष्यते}
{इतिहासे महापुण्ये बुद्धिश्च परिनैष्ठिकी}


\twolineshloka
{अक्षुद्रान्दानशीलांश्च सत्यशीलाननास्तिकान्}
{कार्ष्णं वेदमिमं विद्वाञ्छ्रावयित्वाऽर्थमश्नुते}


\twolineshloka
{भ्रूणहत्याकृतं चापि पापं जह्यादसंशयम्}
{इतिहासमिमं श्रुत्वा पुरुषोऽपि सुदारुणः}


\twolineshloka
{मुच्यते सर्वपापेभ्यो राहुणा चन्द्रमा यथा}
{जयो नामेतिहासोऽयं श्रोतव्यो विजिगीषुणा}


\twolineshloka
{महीं विजयते राजा शत्रूंश्चापि पराजयेत्}
{इदं पुंसवनं श्रेष्ठमिदं स्वस्त्ययनं महत्}


\twolineshloka
{महिषीयुवराजाभ्यां श्रोतव्यं बहुशस्तथा}
{वीरं जनयते पुत्रं कन्यां वा राज्यभागिनीम्}


\twolineshloka
{धर्मशास्त्रमिदं पुण्यमर्थशास्त्रमिदं परम्}
{मोक्षशास्त्रमिदं प्रोक्तं व्यासेनामितबुद्धिना}


\threelineshloka
{`धर्मे चार्थे च कामे च मोक्षे च भरतर्षभ}
{यदिहास्ति तदन्यत्र यन्नेहास्ति न कुत्रचित्}
{इदं हि ब्राह्मणैर्लोके आख्यातं ब्राह्मणेष्विह'}


\twolineshloka
{संप्रत्याचक्षते चेदं तथा श्रोष्यन्ति चापरे}
{पुत्राः शुश्रूषवः सन्ति प्रेष्याश्च प्रियकारिणः}


\twolineshloka
{भरतानां महज्जन्म शृण्वतामनसूयताम्}
{नास्ति व्याधिभयं तेषां परलोकभयं कुतः}


\twolineshloka
{शरीरेण कृतं पापं वाचा च मनसैव च}
{सर्वं संत्यजति क्षिप्रं य इदं शृणुयान्नरः}


\twolineshloka
{धन्यं यशस्यमायुष्यं पुण्यं स्वर्ग्यं तथैव च}
{कृष्णद्वैपायनेनेदं कृतं पुण्यचिकीर्षुणा}


\twolineshloka
{कीर्तिं प्रथयता लोके पाण्डवानां महात्मनाम्}
{अन्येषां क्षत्रियाणां च भूरिद्रविणतेजसाम्}


\twolineshloka
{सर्वविद्यावदातानां लोके प्रथितकर्मणाम्}
{य इदं मानवो लोके पुण्यार्थे ब्राह्मणाञ्छुचीन्}


\twolineshloka
{श्रावयेत महापुण्यं तस्य धर्मः सनातनः}
{कुरूणां प्रथितं वंशं कीर्तयन्सततं शुचिः}


\twolineshloka
{वंशमाप्नोति विपुलं लोके पूज्यतमो भवेत्}
{योऽधीते भारतं पुम्यं ब्राह्मणो नियतव्रतः}


\twolineshloka
{चतुरो वार्षिकान्मासान्सर्वपापैः प्रमुच्यते}
{विज्ञेयः स च वेदानां पारगो भारतं पठन्}


\twolineshloka
{देवा राजर्षयो ह्यत्र पुण्या ब्रह्मर्षयस्तथा}
{कीर्त्यन्ते धूतपाप्मानः कीर्त्यते केशवस्तथा}


\twolineshloka
{भगवांश्चापि देवेशो यत्र देवी च कीर्त्यते}
{अनेकजननो यत्र कार्तिकेयस्य संभवः}


\twolineshloka
{ब्राह्मणानां गवां चैव माहात्म्यं यत्र कीर्त्यते}
{सर्वश्रुतिसमूहोऽयं श्रोतव्यो धर्मबुद्धिभिः}


\twolineshloka
{य इदं श्रावयेद्विद्वान्ब्राह्मणानिह पर्वसु}
{धूतपाप्मा जितस्वर्गो ब्रह्म गच्छति शाश्वतम्}


\twolineshloka
{श्रावयेद्ब्राह्मणाञ्श्राद्धे यश्चेमं पादमन्ततः}
{अक्षय्यं तस्य तच्छ्राद्धमुपावर्तेत्पितॄनिह}


\twolineshloka
{अह्ना यदेनः क्रियते इन्द्रियैर्मनसाऽपि वा}
{ज्ञानादज्ञानतो वापि प्रकरोति नरश्च यत्}


\twolineshloka
{तन्महाभारताख्यानं श्रुत्वैव प्रविलीयते}
{भरतानां महज्जन्म महाभारतमुच्यते}


\twolineshloka
{निरुक्तमस्य यो वेद सर्वपापैः प्रमुच्यते}
{भरतानां महज्जन्म महाभारतमुच्यते}


\twolineshloka
{महतो ह्येनसो मर्त्यान्मोचयेदनुकीर्तितः}
{त्रिभिर्वर्षैर्महाभागः कृष्णद्वैपायनोऽब्रवीत्}


\twolineshloka
{नित्योत्थितः शुचिः शक्तो महाभारतमादितः}
{तपोनियममास्थाय कृतमेतन्महर्षिणा}


\twolineshloka
{तस्मान्नियमसंयुक्तैः श्रोतव्यं ब्राह्मणैरिदम्}
{कृष्णप्रोक्तामिमां पुण्यां भारतीमुत्तमां कथाम्}


\twolineshloka
{श्रावयिष्यन्ति ये विप्रा ये च श्रोष्यन्ति मानवाः}
{सर्वथा वर्तमाना वै न ते शोच्याः कृताकृतैः}


\twolineshloka
{नरेण धर्मकामेन सर्वः श्रोतव्य इत्यपि}
{निखिलेनेतिहासोऽयं ततः सिद्धिमवाप्नुयात्}


\twolineshloka
{न तां स्वर्गगतिं प्राप्य तुष्टिं प्राप्नोति मानवः}
{यां श्रुत्वैवं महापुण्यमितिहासमुपाश्नुते}


\twolineshloka
{शृण्वञ्श्राद्धः पुण्यशीलः श्रावयंश्चेदमद्भुतम्}
{नरः फलमवाप्नोति राजसूयाश्वमेधयोः}


\twolineshloka
{यथा समुद्रो भगवान्यथा मेरुर्महागिरिः}
{उभौ ख्यातौ रत्ननिधी तथा भारतमुच्यते}


\twolineshloka
{इदं हि वेदैः समितं पवित्रमषि चोत्तमम्}
{श्राव्यं श्रुतिसुखं चैव पावनं शीलवर्धनम्}


\twolineshloka
{य इदं भारतं राजन्वाचकाय प्रयच्छति}
{तेन सर्वा मही दत्ता भवेत्सागरमेखला}


\twolineshloka
{पारिक्षित कथां दिव्यां पुण्याय विजयाय च}
{कथ्यमानां मया कृत्स्नां शृणु हर्षकरीमिमाम्}


\twolineshloka
{त्रिभिर्वर्षैः सदोत्थायी कृष्णद्वैपायनो मुनिः}
{महाभारतमाख्यानं कृतवानिदमद्भुतम्}


% Check verse!
शृणु कीर्तयतस्तन्म इतिहासं पुरातनम्
\chapter{अध्यायः ६३}
\twolineshloka
{वैशंपायन उवाच}
{}


\twolineshloka
{पूरोर्वंशमहं धन्यं राज्ञाममिततेजसाम्}
{प्रवक्ष्यामि पितॄणां ते तेषां नामानि मे शृमु}


\twolineshloka
{अव्यक्तप्रभवो ब्रह्मा शाश्वतो नित्य अव्ययः}
{तस्मान्मरीचिः संजज्ञे दक्षश्चैव प्रजापतिः}


\twolineshloka
{अङ्गुष्ठाद्दक्षमसृजच्चक्षुर्भ्यां च मरीचिनम्}
{मरीचेः कश्यपः पुत्रो दक्षस्य दुहिताऽऽदितिः}


\twolineshloka
{अदित्यां कश्यपाद्विवस्थान्}
{विवस्वतो मनुर्मनोरिला}


\fourlineindentedshloka
{इलायाः पुरूरवाः}
{पुरूरवस आयुः}
{आयुषो नहुषः}
{नहुषस्य ययातिः}


\twolineshloka
{ययातेर्द्वे भार्ये बभूवतुः}
{उशनसो दुहिता देवयानी वृषपर्वणश्च दुहिता शर्मिष्ठा नाम}


\threelineshloka
{तत्रानुवंशो भवति}
{यदुं च तुर्वसुं चैव देवयानी व्यजायत}
{द्रुह्यं चानुं च पूरुं च शर्मिष्ठा वार्षपर्वणी}


\fourlineindentedshloka
{तत्र यदोर्यादवाः}
{पूरोः पौरवाः}
{पूरोर्भार्या कौसल्या बभूव}
{तस्यामस्य जज्ञे जनमेजयः}


\twolineshloka
{स त्रीन्हयमेधानाजहार}
{विश्वजिता चेष्ट्वा वनं प्रविवेश}


\twolineshloka
{जनमेजयस्तु सुनन्दां नामोपयेमे मागधीं}
{तस्यामस्य जज्ञे प्राचीन्वान्}


\twolineshloka
{यः प्राचीं दिशं जिगाय}
{यावत्सूर्यादयात् तत्तस्य प्राचीनत्वम्}


\twolineshloka
{प्राचीन्वांस्तु खल्वाश्मकीमुपयेमे यादवीम्}
{तस्यामस्य जज्ञे शस्यातिः}


% Check verse!
शय्यातिस्तु त्रिशङ्कोर्दुहितरं वराङ्गीं नामोपयेमे तस्यामस्यजज्ञेऽहंयातिः
\twolineshloka
{अहंयातिस्तु खलु कृतवीर्यदुहितरं भानुमतीं नामोपयेमे}
{तस्यामस्य जज्ञे सार्वभौमः}


\twolineshloka
{सार्वभौमस्तु खलु जित्वाऽऽजहार कैकयीं सुन्दरां नाम तामुपयेमे}
{तस्यामस्य जज्ञे जयत्सेनः}


\twolineshloka
{जयत्सेनस्तु खलु वैदर्भीमुपयेमे सुश्रवां नाम}
{तस्यामस्य जज्ञेऽपराचीनः}


\twolineshloka
{अपराचीनस्तु खलु वैदर्भीमपरामुपयेमे मर्यादां नाम}
{तस्यामस्य जज्ञेऽरिहः}


\twolineshloka
{अरिहः खल्वाङ्गीमुपेयेमे}
{तस्यामस्य जज्ञे महाभौमः}


\twolineshloka
{महाभौमस्तु खलु प्रसेनजिद्दुहितरमुपयेमे सुयज्ञां नाम}
{तस्यामस्य जज्ञे अयुतानायी}


% Check verse!
यः पुरुषमेधे पुरुषाणामयुतमानयत्तत्तस्यायुतानायित्वम्
\twolineshloka
{अयुतानायी तु खलु पृथुश्रवसो दुहितरमुपयेमे भासां नाम}
{तस्यामस्य जज्ञेऽक्रोधनः}


\twolineshloka
{अक्रोधनस्तु खलु कालिङ्गीं कण्डूं नामोपयेमे}
{तस्यामस्य जज्ञे देवातिथिः}


% Check verse!
देवातिथिस्तु खलु वैदर्भीमुपयेमे मर्यादां नाम तस्यामस्य जज्ञेऋचः
% Check verse!
ऋचस्तु खलु वामदेव्यामङ्गराजकन्यायामृक्षं पुत्रमजीजनत्
\twolineshloka
{ऋक्षस्तु खलु तक्षकदुहितरं ज्वलन्तीं नामोपयेमे}
{तस्यामन्त्यनारमुत्पादयामास}


\twolineshloka
{अन्त्यनारस्तु खलु सरस्वत्यां द्वादशवार्षिकं सत्रमाजहार}
{तमुदवसाने सरस्वत्यभिगम्य भर्तारं वरयामास}


\twolineshloka
{तस्यां पुत्रं जनयामास त्रस्नुं नाम}
{अत्रानुवंशो भवति}


\twolineshloka
{त्रस्नुं सरस्वती पुत्रमन्त्यनारादजीजनत्}
{इलिलं जनयामास कालिन्द्यांत्रस्नुरात्मजम्}


% Check verse!
इलिलस्तु रथन्तर्यां दुष्यन्तादीन्पञ्च पुत्रानजीजनत्
% Check verse!
दुष्यन्तस्तु लाक्षीं नाम भागीरथीमुपयेमे तस्यामस्य जज्ञेजनमेजयः
\twolineshloka
{सएव दुष्यन्तो विश्वामित्रदुहितरं शकुन्तलां नामोपयेमे}
{तस्यामस्य जज्ञे भरतः}


\threelineshloka
{तत्रेमौ श्लोकौ भवतः}
{माता भस्त्रा पितुः पुत्रो यस्माज्जातः स एव सः}
{भरस्व पुत्रं दौष्यन्तिं सत्यमाह शकुन्तला}


\twolineshloka
{रेतोधाः पुत्र उन्नयति नरदेव यमक्षयात्}
{त्वं चास्य धाता गर्भस्य सत्यमाह शकुन्तला}


\threelineshloka
{ततोऽस्य भरतत्वम्}
{भरतस्तु खलु काशेयीं सार्वसेनीमुपयेमे सुनन्दां नाम}
{तस्यामस्य जज्ञे भुमन्युः}


\twolineshloka
{भुमन्युस्तु खलु दाशार्हीमुपयेमे सुवर्णां नाम}
{तस्यामस्य जज्ञे सुहोत्रः}


\threelineshloka
{सुहोत्रस्तु खल्वैक्ष्वाकीमुपयेमे जयन्तीं नाम}
{तस्यामस्य जज्ञे हस्ती}
{य इदं पुरं निर्मापयामास}


\twolineshloka
{तस्माद्धास्तिनपुरत्वम्}
{हस्ती खलु त्रैगर्तीमुपयेमे यशोदां नाम तस्यामस्य जज्ञे विकुञ्जतः}


\twolineshloka
{विकुञ्जनस्तु खलु दाशार्हीमुपयेमे सुन्दरां नाम}
{तस्यामस्य जज्ञेऽजमीढः}


\twolineshloka
{अजमीढस्य तु चतुर्विंशतिपुत्रशतं बभूव}
{कैकय्यां नागायां गान्धार्यां विमलायामृक्षायामिति}


\twolineshloka
{पृथग्वंशकर्तारो नृपतयः}
{तत्र अजमीढादृक्षायां संवरणो जज्ञे स वंशकरः}


\twolineshloka
{सवरणस्तु वैवस्वतीं तपतीं नामोपयेमे}
{तस्यामस्य जज्ञे कुरुः}


\twolineshloka
{कुरुस्तु खलु दाशार्हीमुपयेमे शुभाङ्गीं नाम}
{तस्यामस्य जज्ञे विदूरथः}


\twolineshloka
{विदूरथस्तु खलु मागधीमुपयेमे संप्रियां नाम}
{तस्यामस्य जज्ञेऽनश्वान्}


\twolineshloka
{अनश्वांस्तु खलु मागधीमुपयेमेऽमृतां नाम}
{तस्यामस्य जज्ञे परिक्षित्}


\twolineshloka
{परिक्षित्खलु बाहुकामुपयेमे सुवेषां नाम}
{तस्यामस्य जज्ञे भीमसेनः}


\twolineshloka
{भीमसेनस्तु खलु कैकयीमुपयेमे सुकुमारीं नाम}
{तस्यामस्य जज्ञे परिश्रवाः}


\threelineshloka
{यमाहुः प्रतीप इति}
{प्रतीपस्तु खलु शैब्यामुपयेमे सुनन्दीं नाम}
{तस्यां त्रीन्पुत्रानुत्पादयामास देवापिं शन्तनुं बाह्लीकंचेति}


\twolineshloka
{देवापिस्तु खलु बाल एवारण्यं प्रविवेश}
{शन्तनुस्तु महीपालोऽभवत्}


\threelineshloka
{तत्र श्लोको भवति}
{यं यं कराभ्यां स्पृशति जीर्णं स सुखमश्नुते}
{पुनर्युवा च भवति तस्मात्तं शन्तनुं विदुः}


\threelineshloka
{तदस्य शन्तनुत्वं}
{शन्तनुस्तु खलु गङ्गां भागीरथीमुपयेमे तस्यामस्य जज्ञेदेवव्रतः}
{यमाहुर्भीष्म इति}


\twolineshloka
{भीष्मस्तु खलु पितुः प्रियचिकीर्षया सत्यवतीमानयामास मातरं}
{यामाहुः कालीति}


\twolineshloka
{तस्यां पूर्वं पुराशरात्कन्यागर्भो द्वैपायनः}
{तस्यामेव शन्तनोर्द्वौ पुत्रौ बभूवतुः चित्राङ्गदो विचित्रवीर्यश्च}


\twolineshloka
{चित्राङ्गदस्तु प्राप्तराज्य एव गन्धर्वेण निहृतः}
{ततो विचित्रवीर्यो राजा बभूव}


\twolineshloka
{विचित्रवीर्यस्तु खलुकाशिराजस्य सुते अम्बिकाम्बालिके उदवहत्}
{विचित्रवीर्योऽनुत्पन्नापत्य एव विदेहत्वं प्राप्तः}


% Check verse!
ततः सत्यवती चिन्तयामास कथं नु खलु शन्तनोः पिण्डविच्छेदो नस्यादिति
\twolineshloka
{साथ द्वैपायनं चिन्तयामास सोऽग्रतः स्थितः किं करवाणीति}
{तं सत्यवत्युवाच भ्राता तेऽनपत्य एव स्वर्गतःतस्यार्थेऽपत्यमुत्पादयेति}


% Check verse!
स परमित्युवाच स तत्र त्रीन्पुत्रानुत्पादयामास धृतराष्ट्रंपाण्डुं विदुरं चेति
% Check verse!
धृतराष्ट्रात्पुत्रशतं बभूव गान्धार्यां वरदानाद्द्वैपायनस्यतेषां च धार्तराष्ट्राणां चत्वारः प्रधानाः दुर्योधनो दुश्शासनोविकर्णश्चित्रसेनश्चेति
\twolineshloka
{पाण्डोस्तु कुन्ती माद्रीति स्त्रीरत्ने बभूवतुः}
{स मृगयां चरन्मैथुनगतमृषिं मृगचारिणं बाणेन जघान}


\twolineshloka
{स बाणविद्ध उवाच पाण्डुम्}
{अत्र श्लोको भवति}


\twolineshloka
{योऽकृतार्थं हि मां ग्रूर बाणेनाघ्ना मृगव्रतम्}
{त्वामप्येतादृशो भावः क्षिप्रमेवागमिष्यति}


\twolineshloka
{इति मृगव्रतचारिणा ऋषिणा शप्तः}
{स विषण्णरूपः पाण्डुस्तं शापं परिहरन्नोपसर्पति भार्ये}


\threelineshloka
{कदाचित्स आह}
{स्वचापल्यादिदं प्राप्तवानहम्}
{पुराणेषु पठ्यमानं शृणोमि नानपत्यस्य लोकाः सन्तीति}


% Check verse!
सा त्वं मदर्थे पुत्रानुत्पादयेति कुन्तीमुवाच
\threelineshloka
{सा कुन्ती पुत्रानुत्पादयामास धर्माद्युधिष्ठिरंमारुताद्भीमसेनमिन्द्रादर्जुनमिति}
{स हृष्टरूपः पाण्डुरुवाच}
{इयं ते सपत्नी भवति माद्र्यनपत्या व्रीडिता साध्वीअस्यामपत्यमुत्पाद्यतामिति}


\twolineshloka
{सा कुन्ती तस्यै माद्र्यै तथेति व्रतमादिदेश}
{ततस्तस्यां नकुलसहदेवौ यमावश्विभ्यां जज्ञाते}


\twolineshloka
{माद्रीं तु खलु स्वलङ्कृतां दृष्ट्वा पाण्डुर्भावं चक्रे}
{स तां प्राप्यैव विदेहत्वं प्राप्तः}


\twolineshloka
{ततस्तेन सह चितामन्वारुरोह माद्री}
{कुन्तीं चोवाच यमयोरार्ययाऽप्रमत्तया भवितव्यमिति}


% Check verse!
ततः पञ्चपाण्डवान्सह कुन्त्या हास्तिनपुरं नयन्ति स्मतपस्विनः
% Check verse!
तत्र भीष्माय धृतराष्ट्रविदुरयोः पाण्डोः स्वर्गगमनं याथातथ्यंनिवेदयन्तिस्म तपस्विनः
% Check verse!
पाण्डवान्सह कुन्त्या जतुगृहे दाहयितुकामोधृतराष्ट्रात्मजोऽभूत्
\twolineshloka
{तांश्च विदुरो मोक्षयामास}
{ततो भीमो हिडिम्बं हत्वा पुत्रमुत्पादयामास हिडिम्बायां घटोत्कचं नाम}


\twolineshloka
{ततश्चैकचक्रां जग्मुः कुशलिनः}
{ततः पाञ्चालविषयं गत्वा स्वयंवरे द्रौपदीं लब्ध्वाऽर्धराज्यंप्राप्येन्द्रप्रस्थनिवासिनस्तस्यांपुत्रानुत्पादयामासुर्द्रौपद्याम्}


% Check verse!
प्रतिविन्ध्यां युधिष्ठिरः

सुतसोमं वृकोदरः

श्रुतकीर्तिमर्जुनः

शतानीकं नकुलः

श्रुतसेनं सहदेव इति
\twolineshloka
{शैव्यस्य कन्यां देवकीं नामोपयेमे युधिष्ठिरः}
{तस्यां पुत्रं जनयामास यौधेयं नाम}


\twolineshloka
{भीमसेनस्तु वाराणस्यां काशिराजकन्यां जलन्धरां नामोपयेमे स्वयंवरस्थां}
{तस्यामस्य जज्ञे शर्वत्रातः}


\twolineshloka
{अर्जुनस्तु खलु द्वारवतीं गत्वा भगवतो वासुदेवस्य भगिनीं सुभद्रांनामोदवहद्भार्यां}
{तस्यामभिमन्युं नाम पुत्रं जनयामास}


\twolineshloka
{नकुलस्तु खलु चैद्यां रेणुमतीं नामोदवहत्}
{तस्यां पुत्रं जनयामास निरमित्रं नाम}


\twolineshloka
{सहदेवस्तु खलु माद्रीमेव स्वयंवरे विजयां नामोदवहद्भार्याम्}
{तस्यां पुत्रं जनयामास सुहोत्रं नाम}


\twolineshloka
{भीमसेनश्च पूर्वमेव हिडिम्बायां राक्षस्यां पुत्रमुत्पादयामास घटोत्कचंनाम}
{अर्जुनस्तु नागकन्यायामुलूप्यामिरावन्तं नाम पुत्रं जनयामास}


\twolineshloka
{ततो मणलूरुपतिकन्यायां चित्राङ्गदायामर्जुनः पुत्रमुत्पादयामास बभ्रुवाहनंनाम}
{एते त्रयोदश पुत्राः पाण्डवानाम्}


\twolineshloka
{विराटस्य दुहितरमुत्तरां नामाभिमन्युरुपेयेमे}
{तस्यामस्य परासुर्गर्भोऽजायत}


\twolineshloka
{तमुत्सङ्गे प्रतिजग्राह पृथा नियोगात्पुरुषोत्तमस्य}
{षाण्मासिकं गर्भमहं जीवयामि पादस्पर्शादिति वासुदेव उवाच}


\threelineshloka
{अहं जीवयामि कुमारमनन्तवीर्यं जात एवायमजायत}
{अभिमन्योः सत्येन चेयं पृथिवी धारयत्विति वासुदेवस्यपादस्पर्शात्सजीवोऽजायत}
{नाम तस्याकरोत्सुभद्रा}


\twolineshloka
{परिक्षीणे कुले जात उत्तरायां परंतपः}
{परिक्षिदभवत्तस्मात्सौभद्रात्तु यशस्विनः}


\twolineshloka
{परीक्षित्तु खलु भद्रवतीं नामोपयेमे}
{तस्यां तत्र भवाञ्जनमेजयः}


% Check verse!
जनमेजयात्तु भवतः खलु वपुष्टमायां पुत्रौ द्वौ शतानीकःशङ्कुकर्णश्च
\twolineshloka
{शतानीकस्तु खलु वैदेहीमुपयेमे}
{तस्यामस्य जज्ञे पुत्रोऽश्वमेधदत्तः}


\twolineshloka
{इत्येष पूरोर्वंशस्तु पाण्डवानां च कीर्तितः}
{पूरोर्वंशमिमं श्रुत्वा सर्वपापैः प्रमुच्यते}


\chapter{अध्यायः ६४}
\twolineshloka
{वैशंपायन उवाच}
{}


\twolineshloka
{राजोपरिचरो नाम धर्मनित्यो महीपतिः}
{बभूव मृगयाशीलः शश्वत्स्वाध्यायवाञ्छुचिः}


\twolineshloka
{स चेदिविषयं रम्यं वसुः पौरवनन्दनः}
{इन्द्रोपदेशाज्जग्राह रमणीयं महीपतिः}


\twolineshloka
{तमाश्रमे न्यस्तशस्त्रं निवसन्तं तपोनिधिम्}
{देवाः शक्रपुरोगा वै राजानमुपतस्थिरे}


\threelineshloka
{इन्द्रत्वमर्हो राजायं तपसेत्यनुचिन्त्य वै}
{तं सान्त्वेन नृपं साक्षात्तपसः संन्यवर्तयन् ॥देवा ऊचुः}
{}


\threelineshloka
{न संकीर्येत धर्मोऽयं पृथिव्यां पृथिवीपते}
{त्वया हि धर्मो विधृतः कृत्स्नं धारयते जगत् ॥इन्द्र उवाच}
{}


\threelineshloka
{`देवानहं पालयिता पालय त्वं हि मानुषान्}
{'लोके धर्मं पालय त्वं नित्ययुक्तः समाहितः}
{धर्मयुक्तस्ततो लोकान्पुण्यान्प्राप्स्यसि शाश्वतान्}


\twolineshloka
{दिविष्ठस्य भुविष्ठस्त्वं सखाभूतो मम प्रियः}
{ऊधः पृथिव्या यो देशस्तमावस नराधिप}


\twolineshloka
{पशव्यश्चैव पुण्यश्च प्रभूतधनधान्यवान्}
{स्वारक्ष्यश्चैव सौम्यश्च भोग्यैर्भूमिगुणैर्युतः}


\twolineshloka
{अर्थवानेष देशो हि धनरत्नादिभिर्युतः}
{वसुपूर्णा च वसुधा वस चेदिषु चेदिप}


\twolineshloka
{धर्मशीला जनपदाः सुसंतोषाश्च साधवः}
{न च मिथ्या प्रलापोऽत्र स्वैरेष्वपि कुतोऽन्यथा}


\twolineshloka
{न च पित्रा विभज्यन्ते पुत्रा गुरुहिते रताः}
{युञ्जते धुरि नो गाश्च कृशान्संधुक्षयन्ति च}


\twolineshloka
{सर्वे वर्णाः स्वधर्मस्थाः सदा चेदिषु मानद}
{न तेऽस्त्यविदितं किंचित्त्रिषु लोकेषु यद्भवेत्}


\twolineshloka
{दैवोपभोग्यं दिव्यं त्वामाकाशे स्फाटिकं महत्}
{आकाशगं त्वां मद्दत्तं विमानमुपपत्स्यते}


\twolineshloka
{त्वमेकः सर्वमर्त्येषु विमानवरमास्थितः}
{चरिष्यस्युपरिस्थो हि देवो विग्रहवानिव}


\twolineshloka
{ददामि ते वैजयन्तीं मालामम्लानपङ्कजाम्}
{धारयिष्यति सङ्ग्रामे या त्वां शस्त्रैरविक्षतम्}


\twolineshloka
{लक्षणं चैतदेवेह भविता ते नराधिप}
{इन्द्रमालेति विख्यातं धन्यमप्रतिमं महत्}


\twolineshloka
{यष्टिं च वैणवीं तस्मै ददौ वृत्रनिषूदनः}
{इष्टप्रदानमुद्दिश्य शिष्टानां प्रतिपालिनीम्}


\twolineshloka
{`एवं संसान्त्व्य नृपतिं तपसः संन्यवर्तयत्}
{प्रययौ दैवतैः सार्धं कृत्वा कार्यं दिवौकसाम्}


\twolineshloka
{ततस्तु राजा चेदीनामिन्द्राभरणभूषितः}
{इन्द्रदत्तं विमानं तदास्थाय प्रययौ पुरीम् ॥'}


\twolineshloka
{तस्याः शक्रस्य पूजार्थं भूमौ भूमिपतिस्तदा}
{प्रवेशं कारयामास सर्वोत्सववरं तदा}


\twolineshloka
{`मार्गशीर्षे महाराज पूर्वपक्षे महामखम्}
{ततःप्रभृति चाद्यापि यष्टेः क्षितिपसत्तमैः ॥'}


\twolineshloka
{प्रवेशः क्रियते राजन्यथा तेन प्रवर्तितः}
{अपरेद्युस्ततस्तस्याः क्रियतेऽत्युच्छ्रयो नृपैः}


\twolineshloka
{अलङ्कृताया पिटकैर्गन्धमाल्यैश्च भूषणैः}
{`माल्यदामपरिक्षिप्तां द्वात्रिंशत्किष्कुसंमिताम्}


\twolineshloka
{उद्धृत्य पिटके चापि द्वादशारत्निकोच्छ्रये}
{महारजनवासांसि परिक्षिप्य ध्वजोत्तमम्}


\twolineshloka
{वासोभिरन्नपानैश्च पूजितैर्ब्राह्मणर्षभैः}
{पुण्याहं वाचयित्वाथ ध्वज उच्छ्रियते तदा}


\twolineshloka
{शङ्खभेरीमृदङ्गैश्च संनादः क्रियते तदा'}
{भगवान्पूज्यते चात्र यष्टिरूपेण वासवः}


\twolineshloka
{स्वयमेव गृहीतेन वसोः प्रीत्या महात्मनः}
{`माणिभद्रादयो यक्षाः पूज्यन्ते दैवतैः सह}


\twolineshloka
{नानाविधानि दानानि दत्त्वार्थिभ्यः सुहृज्जनैः}
{अलङ्कृत्वा माल्यदामैर्वस्त्रैर्नानाविधैस्तथा}


\twolineshloka
{दृतिभिः सजलैः सर्वैः क्रीडित्वा नृपशासनात्}
{सभाजयित्वा राजानं कृत्वा नर्माश्रयाः कथाः}


\twolineshloka
{रमन्ते नागराः सर्वे तथा जानपदैः सह}
{सूताश्च मागधाश्चैव रमन्ते नटनर्तकाः}


\twolineshloka
{प्रीत्या तु नृपशार्दूल सर्वे चक्रुर्महोत्सवम्}
{सान्तःपुरः सहामात्यः सर्वाभरणभूषितः}


\twolineshloka
{महारजनवासांसि वसित्वा चेदिराट् तदा}
{जातिहिङ्गुलकेनाक्तः सदारो मुमुदे तदा}


% Check verse!
एवं जानपदाः सर्वे चक्रुरिन्द्रमहं वसुः ॥'यथा चेदिपतिः प्रीतश्चकारेन्द्रमहं वसुः ॥'
\twolineshloka
{एतां पूजां महेन्द्रस्तु दृष्ट्वा वसुकृतां शुभाम्}
{`हरिभिर्वाजिभिर्युक्तमन्तरिक्षगतं रथम्}


\twolineshloka
{आस्थाय सह शच्या च वृतो ह्यप्सरसां गणैः}
{'वसुना राजमुख्येन समागम्याब्रवीद्वचः}


\twolineshloka
{ये पूजयिष्यन्ति च मुदा यथा चेदिपतिर्नृपः}
{कारयिष्यन्ति च मुदा यथा चेदिपतिर्नृपः}


\twolineshloka
{तेषां श्रीर्विजयश्चैव सराष्ट्राणां भविष्यति}
{तथा स्फीतो जनपदो मुदितश्च भविष्यति}


\threelineshloka
{`निरीतिकानि सस्यानि भवन्ति बहुधा नृप}
{राक्षसाश्च पिशाचाश्च न लुम्पन्ते कथंचन ॥वैशंपायन उवाच}
{'}


\threelineshloka
{एवं महात्मना तेन महेन्द्रेण नराधिप}
{वसुः प्रीत्या मघवता महाराजोऽभिसत्कृतः}
{एवं कृत्वा महेन्द्रस्तु जगाम स्वं निवेशनम्}


\threelineshloka
{उत्सवं कारयिष्यन्ति सदा शक्रस्य ये नराः}
{भूमिरत्नादिभिर्दानैस्तथा पूज्या भवन्ति ते}
{वरदानमहायज्ञैस्तथा शक्रोत्सवेन च}


\twolineshloka
{संपूजितो मघवता वसुश्चेदीश्वरो नृपः}
{पालयामास धर्मेण चेदिस्थः पृथिवीमिमाम्}


\twolineshloka
{इन्द्रपीत्या चेदिपतिश्चकारेन्द्रमहं वसुः}
{पुत्राश्चास्य महावीर्याः पञ्चासन्नमितौजसः}


\twolineshloka
{नानाराज्येषु च सुतान्स सम्राडभ्यषेचयत्}
{महारथो मागधानां विश्रुतो यो बृहद्रथः}


\twolineshloka
{प्रत्यग्रहः कुशाम्बश्च यमाहुर्मणिवाहनम्}
{मत्सिल्लश्च यदुश्चैव राजन्यश्चापराजितः}


\twolineshloka
{एते तस्य सुता राजन्राजर्षेर्भूरितेजसः}
{न्यवेशयन्नामभिः स्वैस्ते देशांश्च पुराणि च}


\twolineshloka
{वासवाः पञ्च राजानः पृथग्वंशाश्च शास्वताः}
{वसन्तमिन्द्रप्रासादे आकाशे स्फाटिके च तम्}


\twolineshloka
{उपतस्थुर्महात्मानं गन्धर्वाप्सरसो नृपम्}
{राजोपरिचरेत्येवं नाम तस्याथ विश्रुतम्}


\twolineshloka
{पुरोपवाहिनीं तस्य नदीं शुक्तमतीं गिरिः}
{अरौत्सीच्चेतनायुक्तः कामात्कोलाहलः किल}


\twolineshloka
{गिरिं कोलाहलं तं तु पदा वसुरताडयत्}
{निश्चक्राम ततस्तेन प्रहारविवरेण सा}


\twolineshloka
{तस्यां नद्यां स जनयन्मिथुनं पर्वतः स्वयम्}
{तस्माद्विमोक्षणात्प्रीता नदी राज्ञे न्यवेदयत्}


\twolineshloka
{`महिषी भविता कन्या पुमान्सेनापतिर्भवेत्}
{शुक्तिमत्या वचःश्रुत्वा दृष्ट्वा तौ राजसत्तमः'}


\twolineshloka
{यः पुमानभवत्तत्र तं स राजर्षिसत्तमः}
{वसुर्वसुप्रदश्चक्रे सेनापतिमरिन्दमः}


\twolineshloka
{चकार पत्नीं कन्यां तु तथा तां गिरिकां नृपः}
{वसोः पत्नी तु गिरिका कामकालं न्यवेदयत्}


\twolineshloka
{ऋतुकालमनुप्राप्ता स्नाता पुंसवने शुचिः}
{तदहः पितरश्चैनपूचुर्जहि मृगानिति}


\twolineshloka
{तं राजसत्तमं प्रीतास्तदा मतिमतां वरः}
{स पितॄणां नियोगेन तामतिक्रम्य पार्थिवः}


\twolineshloka
{चकार मृगयां कामी गिरिकामेव संस्मरन्}
{अतीव रूपसंपन्नां साक्षाच्छ्रियमिवापराम्}


\twolineshloka
{अशोकैश्चम्पकैश्चूतैरनेकैरतिमुक्तकैः}
{पुन्नागैः कर्णिकारैश्च बकुलैर्दिव्यपाटलैः}


\twolineshloka
{पनसैर्नारिकेलैश्च चन्दनैश्चार्जुनैस्तथा}
{एतै रम्यैर्महावृक्षैः पुण्यैः स्वादुफलैर्युतम्}


\twolineshloka
{कोकिलाकुलसन्नादं मत्तभ्रमरनादितम्}
{वसन्तकाले तत्पश्यन्वनं चैत्ररथोपमम्}


\twolineshloka
{मन्मथाभिपरीतात्मा नापश्यद्गिरिकां तदा}
{अपश्यन्कामसंतप्तश्चरमाणो यदृच्छया}


\twolineshloka
{पुष्पसंछन्नशाखाग्रं पल्लवैरुपशोभितम्}
{अशोकस्तबकैश्छन्नं रमणीयमपश्यत}


\twolineshloka
{अधस्यात्तस्य छायायां सुखासीनो नराधिपः}
{मधुगन्धैश्च संयुक्तं पुष्पगन्धमनोहरम्}


\threelineshloka
{वायुना प्रेर्यमाणस्तु धूम्राय मुदमन्वगात्}
{`भार्यां चिन्तयमानस्य मन्मथाग्निरवर्धत}
{'तस्य रेतः प्रचस्कन्द चरतो गहने वने}


\twolineshloka
{स्कन्नमात्रं च तद्रेतो वृक्षपत्रेण भूमिपः}
{प्रतिजग्राह मिथ्या मे न पतेद्रेत इत्युत}


\twolineshloka
{`अङ्गुलीयेन शुक्लस्य रक्षां च विदधे नृपः}
{अशोकस्तबकै रक्तैः पल्लवैश्चाप्यबन्धयत् ॥'}


\twolineshloka
{इदं मिथ्या परिस्कन्नं रेतो मे न भवेदिति}
{ऋतुश्च तस्याः पत्न्या मे न मोघः स्यादिति प्रभुः}


\twolineshloka
{संचिन्त्यैवं तदा राजा विचार्य च पुनःपुनः}
{अमोघत्वं च विज्ञाय रेतसो राजसत्तमः}


\twolineshloka
{शुक्रप्रस्थापने कालं महिष्या प्रसमीक्ष्य वै}
{अभिमन्त्र्याथ तच्छुक्रमारात्तिष्ठन्तमाशुगम्}


\twolineshloka
{सूक्ष्मधर्मार्थतत्त्वज्ञो गत्वा श्येनं ततोऽब्रवीत्}
{मत्प्रियार्थमिदं सौम्य शुक्रं मम गृहं नय}


\threelineshloka
{गिरिकायाः प्रयच्छाशु तस्या ह्यार्तवमद्य वै}
{वैशंपायन उवाच}
{गृहीत्वा तत्तदा श्येनस्तूर्णमुत्पत्य वेगवान्}


\twolineshloka
{जवं परममास्थाय प्रदुद्राव विहंगमः}
{तमपश्यदथायान्तं श्येनं श्येनस्तथाऽपरः}


\twolineshloka
{अभ्यद्रवच्च तं सद्यो दृष्ट्वैवामिषशङ्कया}
{तुण्डयुद्धमथाकाशे तावुभौ संप्रचक्रतुः}


\twolineshloka
{युद्ध्यतोरपतद्रेतस्तच्चापि यमुनाम्भसि}
{तत्राद्रिकेति विख्याता ब्रह्मशापाद्वराप्सरा}


\twolineshloka
{मीनभावमनुप्राप्ता बभूव यमुनाचरी}
{श्येनपादपरिभ्रष्टं तद्वीर्यमथ वासवम्}


\twolineshloka
{जग्राह तरसोपेत्य साऽद्रिका मत्स्यरूपिणी}
{कदाचिदपि मत्सीं तां बबन्धुर्मत्स्यजीविनः}


% Check verse!
मासे च दशमे प्राप्ते तदा भरतसत्तम ॥उज्जह्रुरुदरात्तस्याः स्त्रीं पुमांसं च मानुषौ
\twolineshloka
{आश्चर्यभूतं तद्गत्वा राज्ञेऽथ प्रत्यवेदयन्}
{काये मत्स्या इमौ राजन्संभूतौ मानुषाविति}


\twolineshloka
{तयोः पुमांसं जग्राह राजोपरिचरस्तदा}
{स मत्स्यो नाम राजासीद्धार्मिकः सत्यसङ्गरः}


\twolineshloka
{साऽप्सरा मुक्तशापा च क्षणेन समपद्यत}
{या पुरोक्ता भगवता तिर्यग्योनिगता शुभा}


\twolineshloka
{मानुषौ जनयित्वा त्वं शापमोक्षमवाप्स्यसि}
{ततः साजनयित्वा तौ विशस्ता मत्स्यघातिभिः}


\twolineshloka
{संत्यज्य मत्स्यरूपं सा दिव्यं रूपमवाप्य च}
{सिद्धर्षिचारणपथं जगामाथ वराप्सराः}


\twolineshloka
{सा कन्या दुहिता तस्या मत्स्या मत्स्यसगन्धिनी}
{राज्ञा दत्ता च दाशाय कन्येयं ते भवत्विति}


\twolineshloka
{रूपसत्वसमायुक्ता सर्वैः समुदिता गुणैः}
{सा तु सत्यवती नाम मत्स्यघात्यभिसंश्रयात्}


\twolineshloka
{आसीत्सा मत्स्यगन्धैव कंचित्कालं शुचिस्मिता}
{शुश्रूषार्थं पितुर्नावं वाहयन्तीं जले च ताम्}


\twolineshloka
{तीर्थयात्रां परिक्रामन्नपश्यद्वै पराशरः}
{अतीव रूपसंपन्नां सिद्धानामपि काङ्क्षिताम्}


\twolineshloka
{दृष्ट्वैव स च तां धीमांश्चकमे चारुहासिनीम्}
{दिव्यां तां वासवीं कन्यां रम्भोरूं मुनिपुङ्गवः}


\threelineshloka
{संभवं चिन्तयित्वा तां ज्ञात्वा प्रोवाच शक्तिजः}
{क्व कर्णधारो नौर्येन नीयते ब्रूहि भामिनि ॥मत्स्यगन्धोवाच}
{}


\threelineshloka
{अनपत्यस्य दाशस्य सुता तत्प्रियकाम्यया}
{सहस्रजनसंपन्ना नौर्मया वाह्यते द्विज ॥पराशर उवाच}
{}


\twolineshloka
{शोभनं वासवि शुभे किं चिरायसि वाह्यताम्}
{कलशं भविता भद्रे सहस्रार्धेन संमितम्}


\threelineshloka
{अहं शेषो भिविष्यामि नीयतामचिरेण वै}
{वैशंपायन उवाच}
{मत्स्यगन्धा तथेत्युक्त्वा नावं वाहयतां जले}


\twolineshloka
{वीक्षमाणं मुनिं दृष्ट्वा प्रोवाचेदं वचस्तदा}
{मत्स्यगन्धेति मामाहुर्दाशराजसुतां जनाः}


\threelineshloka
{जन्म शोकाभितप्तायाः कथं ज्ञास्यसि कथ्यताम्}
{पराशर उवाच}
{दिव्यज्ञानेन दृष्टं हि दृष्टमात्रेण ते वपुः}


\twolineshloka
{प्रणयग्रहणार्थाय वक्ष्येव वासवि तच्छृणु}
{बर्हिषद इति ख्याताः पितरः सोमपास्तुते}


\twolineshloka
{तेषां त्वं मानसी कन्या अच्छोदा नाम विश्रुता}
{अच्छोदं नाम तद्दिव्यं सरो यस्मात्समुत्थितम्}


\twolineshloka
{त्वया न दृष्टपूर्वास्तु पितरस्ते कदाचन}
{संभूता मनसा तेषां पितॄन्स्वान्नाभिजानती}


\twolineshloka
{सा त्वन्यं पितरं वव्रे स्वानतिक्रम्य तान्पितॄन्}
{नाम्ना वसुरिति ख्यातं मनुपुत्रं दिवि स्थितम्}


\twolineshloka
{अद्रिकाऽप्सरसा युक्तं विमाने दिवि विष्ठितम्}
{सा तेन व्यभिचारेण मनसा कामचारिणी}


\twolineshloka
{पितरं प्रार्थयित्वाऽन्यं योगाद्भष्टा पपात सा}
{अपश्यत्पतमाना सा विमानत्रयमन्तिकात्}


\twolineshloka
{त्रसरेणुप्रमाणांस्तांस्तत्रापश्यत्स्वकान्पितॄन्}
{सुसूक्ष्मानपरिव्यक्तानङ्गैरङ्गेष्विवाहितान्}


\twolineshloka
{तातेति तानुवाचार्ता पतन्ती सा ह्यधोमुखी}
{तैरुक्ता सा तु माभैषीस्तेन सा संस्थिता दिवि}


\twolineshloka
{ततः प्रसादयामास स्वान्पितॄन्दीनया गिरा}
{तामूचुः पितरः कन्यां भ्रैष्टश्वर्यां व्यतिक्रमात्}


\twolineshloka
{भ्रष्टैश्वर्या स्वदोषेण पतसि त्वं शुचिस्मिते}
{यैरारभन्ते कर्माणि शरीरैरिह देवताः}


\twolineshloka
{तैरेव तत्कर्मफलं प्राप्नुवन्ति स्म देवताः}
{मनुष्यास्त्वन्यदेहेन शुभाशुभमिति स्थितिः}


\twolineshloka
{सद्यः फलन्ति कर्माणि देवत्वे प्रेत्य मानुषे}
{तस्मात्त्वं पतसे पुत्रि प्रेत्यत्वं प्राप्स्यसे फलम्}


\twolineshloka
{पितृहीना तु कन्या त्वं वसोर्हि त्वं सुता मता}
{मत्स्ययोनौ समुत्पन्ना सुताराज्ञो भविष्यसि}


\twolineshloka
{अद्रिका मत्स्यरूपाऽभूद्गङ्गायमनुसङ्गमे}
{पराशरस्य दायादं त्वं पुत्रं जनयिष्यसि}


\twolineshloka
{यो वेदमेकं ब्रह्मर्षिश्चतुर्धा विबजिष्यति}
{महाभिषक्सुतस्यैव शन्तनोः कीर्तिवर्धनम्}


\twolineshloka
{ज्येष्ठं चित्राङ्गदं वीरं चित्रवीरं च विश्रुतम्}
{एतान्सूत्वा सुपुत्रांस्त्वं पुनरेव गमिष्यसि}


\twolineshloka
{व्यतिक्रमात्पितॄणां च प्राप्स्यसे जन्म कुत्सितम्}
{अस्यैव राज्ञस्त्वं कन्या ह्यद्रिकायां भविष्यसि}


\twolineshloka
{अष्टाविंशे भवित्री त्वं द्वापरे मत्स्ययोनिजा}
{एवमुक्ता पुरा तैस्त्वं जाता सत्यवती शुभा}


\twolineshloka
{अद्रिकेत्यभिविख्याता ब्रह्मशापाद्वराप्सरा}
{मीनभावमनुप्राप्ता त्वं जनित्वा गता दिवम्}


\twolineshloka
{तस्यां जातासि सा कन्या राज्ञो वीर्येण चैवहि}
{तस्माद्वासवि भद्रं ते याचे वंशकरं सुतम्}


\twolineshloka
{संगमं मम कल्याणि कुरुष्वेत्यभ्यभाषत ॥वैशंपायन उवाच}
{}


\twolineshloka
{विस्मयाविष्टसर्वाङ्गी जातिस्मरणतां गता}
{साब्रवीत्पश्य भगवन्परपारे स्थितानृषीन्}


\twolineshloka
{आवयोर्दृष्टयोरेभिः कथं नु स्यात्समागमः}
{एवं तयोक्तो भगवान्नीहारमसृजत्प्रभुः}


\twolineshloka
{येन देशः स सर्वस्तु तमोभूत इवाभवत्}
{दृष्ट्वा सृष्टं तु नीहारं ततस्तं परमर्षिणा}


\threelineshloka
{विस्मिता साभवत्कन्या व्रीडिता च तपस्विनी}
{सत्यवत्युवाच}
{विद्धि मां भगवन्कन्यां सदा पितृवशानुगाम्}


\twolineshloka
{त्वत्संयोगाच्च दुष्येत कन्याभावो ममाऽनघ}
{कन्यात्वे दूषिते वापि कथं शक्ष्ये द्विजोत्तम}


\threelineshloka
{गृह गन्तुमुषे चाहं धीमन्न स्थातुमुत्सहे}
{एतत्संचिन्त्य भगवन्विधत्स्व यदनन्तरम् ॥वैशंपायन उवाच}
{}


\twolineshloka
{एवमुक्तवतीं तीं तु प्रीतिमानुषिसत्तमः}
{उवाच मत्प्रियं कृत्वा कन्यैव त्वं भविष्यति}


\twolineshloka
{वृणीष्व च वरं भीरुं यं त्वमिच्छसि भामिनि}
{वृथा हि न प्रसादो मे भूतपूर्वः शुचिस्मिते}


\twolineshloka
{एवमुक्ता वरं वव्रे गात्रसौगन्ध्यमुत्तमम्}
{सचास्यै भगवान्प्रादान्मनः काङ्क्षितं प्रभुः}


\twolineshloka
{ततो लब्धवरा प्रीता स्त्रीभावगुणभूषिता}
{जगाम सह संसर्गमृषिणाऽद्भुतकर्मणा}


\twolineshloka
{तेन गन्धवतीत्येवं नामास्याः प्रथितं भुवि}
{तस्यास्तु योजनाद्गन्धमाजिघ्रन्त नरा भुवि}


\twolineshloka
{तस्या योजनगन्धेति ततो नामापरं स्मृतम्}
{इति सत्यवती हृष्टा लब्ध्वा वरमनुत्तमम्}


\twolineshloka
{पराशरेण संयुक्ता सद्यो गर्भं सुषाव सा}
{जज्ञे च यमुनाद्वीपे पाराशर्यः स वीर्यवान्}


\twolineshloka
{स मातरमनुज्ञाप्य तपस्येव मनो दधे}
{स्मृतोऽहं दर्शयिष्यामि कृत्येष्विति च सोऽब्रवीत्}


\twolineshloka
{एवं द्वैपायनो जज्ञे सत्यवत्यां पराशरात्}
{न्यस्तोद्वीपे यद्बालस्तस्माद्द्वैपायनःस्मृतः}


\twolineshloka
{पादापसारिणं धर्मं स तु विद्वान्युगे युगे}
{आयुः शक्तिं च मर्त्यानां युगावस्थामवेक्ष्यच}


\twolineshloka
{ब्रह्मणो ब्राह्मणानां च तथानुग्रहकाङ्क्षया}
{विव्यास वेदान्यस्मत्स तस्माद्व्यास इति स्मृतः}


\twolineshloka
{वेदानध्यापयामास महाभारतपञ्चमान्}
{सुमन्तुं जैमिनिं पैलं शुकं चैव स्वमात्मजम्}


% Check verse!
प्रभुर्वरिष्ठो वरदो वैशंपायनमेव च

संहितास्तैः पृथक्त्वेन भारतस्य प्रकाशिताः ॥ 1-64a-1a [ततोरम्ये वनोद्देशे दिव्यास्तरणसंयुते

1-64a-1b वीरासनमुपास्थाय योगीध्यानपरोऽभवत् ॥ 1-64a-2a श्वेतपट्टगृहे रम्ये पर्यङ्के सोत्तरच्छदे

1-64a-2b तूष्णींभूतां तदा कन्यां ज्वलन्तीं योगतेजसा ॥ 1-64a-3aदृष्ट्वा तां तु समाधाय विचार्य च पुनः पुनः

1-64a-3b स चिन्तयामासमुनिः किं कृतं सुकृतं भवेत् ॥ 1-64a-4a शिष्टानां तु समाचारःशिष्टाचार इति स्मृतः

1-64a-4b श्रुतिस्मृतिविदो विप्रा धर्मज्ञाज्ञानिनः स्मृताः ॥ 1-64a-5a धर्मज्ञैर्विहितो धर्मः श्रौतः स्मार्तोद्विधा द्विजैः

1-64a-5b दानाग्निहोत्रमिज्या च श्रौतस्यैतद्धिलक्षणम् ॥ 1-64a-6a स्मार्तो वर्णाश्रमाचारो यमैश्च नियमैर्युतः

1-64a-6b धर्मे तु धारणे धातुः सहत्वे चापि पठ्यते ॥ 1-64a-7aतत्रेष्टफलभाग्धर्म आचार्यैरुपदिश्यते

1-64a-7b अनिष्टफलभाक्रेतितैरधर्मोऽपि भाष्यते ॥ 1-64a-8a तस्मादिष्टफलार्थाय धर्ममेवसमाश्रयेत्

1-64a-8b ब्राह्मो दैवस्तथैवार्षः प्राजापत्यश्चधार्मिकः ॥ 1-64a-9a विवाहा ब्राह्मणानां तु गान्धर्वो नैव धार्मिकः

1-64a-9b त्रिवर्णेतरजातीनां गान्धर्वासुरराक्षसाः ॥ 1-64a-10a पैशाचोनैव कर्तव्यः पैशाचश्चाष्टमोऽधमः

1-64a-10b सामर्षां व्यङ्गिकांकन्यां मातुश्च कुलजां तथा ॥ 1-64a-11a वृद्धां प्रव्राजितां वन्ध्यांपतितां च रजस्वलाम्

1-64a-11b अपस्मारकुले जातां पिङ्गलांकुष्ठिनींव्रणीम् ॥ 1-64a-12a न चास्नातां स्त्रियं गच्छेदिति धर्मानुशासनम्

1-64a-12b पिता पितामहो भ्राता माता मातुल एव च ॥ 1-64a-13aउपाध्यायर्त्विजश्चैव कन्यादाने प्रभूत्तमाः

1-64a-13b एतैर्दत्तांनिषेवेत नादत्तामाददीत च ॥ 1-64a-14a इत्येव ऋषयः प्राहुर्विवाहेधर्मवित्तमाः

1-64a-14b अस्या नास्ति पिता भ्राता माता मातुल एव च ॥ 1-64a-15a गान्धर्वेण विवाहेन न स्पृशामि यदृच्छया

1-64a-15bक्रियाहीनं तु गान्धर्वं न कर्तव्यमनापदि ॥ 1-64a-16a यदस्यां जायतेपुत्रो वेदव्यासो भवेदृषिः

1-64a-16b क्रियाहीनः कथं विप्रोभवेदृषिरुदारधीः ॥ 1-64a-17x वैशंपायन उवाच

1-64a-17a एवं चिन्तयतोभावं महर्षेर्भावितात्मनः

1-64a-17b ज्ञात्वा चैवाभ्यवर्तन्त पितरोबर्हिषस्तदा ॥ 1-64a-18a तस्मिन्क्षणे ब्रह्मपुत्रो वसिष्ठोऽपिसमेयिवान्

1-64a-18b पूर्वं स्वागतमित्युक्त्वा वसिष्ठः प्रत्यभाषत ॥ 1-64a-19x पितृगणा ऊचुः

1-64a-19a अस्माकं मानसीं कन्यामस्मच्छापेनवासवीम्

1-64a-19b यदिचच्छशि पुत्रार्थं कन्यां गृह्मीष्वमा चिरम् ॥ 1-64a-20a पितॄणां वचनं श्रुत्वा वसिष्ठः प्रत्यभाषत

1-64a-20bमहर्षीणां वचः सत्यं पुराणेपि मया श्रुतम् ॥ 1-64a-21a पराशरोब्रह्मचारी प्रजार्थी मम वंशधृत्

1-64a-21b एवं संभाषमाणे तु वसिष्ठेपितृभिः सह ॥ 1-64a-22a ऋषयोऽभ्यागमंस्तत्र नैमिशारण्यवासिनः

1-64a-22b विवाहं द्रष्टुमिच्छन्तः शक्तिपुत्रस्य धीमतः ॥ 1-64a-23aअरुन्धती महाभागा अदृश्यन्त्या सहैव सा

1-64a-23b विश्वकर्मकृतांदिव्यां पर्णशालां प्रविश्य सा ॥ 1-64a-24a वैवाहिकांस्तुसंभारान्संकल्प्य च यथाक्रमम्

1-64a-24b अरुन्धती सत्यवतीं वधूंसंगृह्य पाणिना ॥ 1-64a-25a भद्रासने प्रतिष्ठाप्य इन्द्राणींसमकल्पयत्

1-64a-25b आपूर्यमाणपक्षे तु वैशाखे सोमदैवते ॥ 1-64a-26aशुभग्रहे त्रयोदश्यां मुहूर्ते मैत्र आगते

1-64a-26b विवाहकालइत्युक्त्वा वसिष्ठो मुनिभिः सह ॥ 1-64a-27a यमुनाद्वीपमासाद्यशिष्यैश्च मुनिपङ्क्तिभिः

1-64a-27b स्थण्डिलं चतुरश्रं चगोमयेनोपलिप्य च ॥ 1-64a-28a अक्षतैः फलपुष्पैश्चस्वस्तिकैराम्रपल्लवैः

1-64a-28b जलपूर्णघटैश्चैव सर्वतः परिशोभितम् ॥ 1-64a-29a तस्य मध्ये प्रतिष्ठाप्य बृस्यां मुनिवरं तदा

1-64a-29bसिद्धार्थयवकल्कैश्च स्नातं सर्वौषधैरपि ॥ 1-64a-30a कृत्वार्जुनानिवस्त्राणि परिधाप्य महामुनिम्

1-64a-30b वाचयित्वा तुपुण्याहमक्षतैस्तु समर्चितः ॥ 1-64a-31a गन्धानुलिप्तः स्रग्वी चसप्रतोदो वधूगृहे

1-64a-31b अपदातिस्ततो गत्वा वधूज्ञातिभिरर्चितः ॥ 1-64a-32a स्नातामहतसंवीतां गन्धलिप्तां स्रगुज्ज्वलाम्

1-64a-32bवधूं मङ्गलसंयुक्तामिषुहस्तां समीक्ष्य च ॥ 1-64a-33a उवाच वचनं कालेकालज्ञः सर्वधर्मवित्

1-64a-33b प्रतिग्रहो दातृवशः श्रुतमेवं मयापुरा ॥ 1-64a-34a यथा वक्ष्यन्ति पितरस्तत्करिष्यामहे वयम्

1-64a-34xवैशंपायन उवाच

1-64a-34b तद्धर्मिष्ठं यशस्यं च वचनं सत्यवादिनः ॥ 1-64a-35a श्रुत्वा तु पितरः सर्वे निःसङ्गा निष्परिग्रहाः

1-64a-35bवसुं परमधर्मिष्ठमानीयेदं वचोऽब्रुवन् ॥ 1-64a-36a मत्स्ययोनौसमुत्पन्ना तव पुत्री विशेषतः

1-64a-36b पराशराय मुनये दातुमर्हसिधर्मतः ॥ 1-64a-37x वसुरुवाच

1-64a-37a सत्यं मम सुता सा हि दाशराजेनवर्धिता

1-64a-37b अहं प्रभुः प्रदाने तु प्रजापालः प्रजार्थिनाम् ॥ 1-64a-38x पितर ऊचुः

1-64a-38a निराशिषो वयं सर्वे निःसङ्गानिष्परिग्रहाः

1-64a-38b कन्यादानेन संबन्धो दक्षिणाबन्ध उच्यते ॥ 1-64a-39a कर्मभूमिस्तु मानुष्यं भोगभूमिस्त्रिविष्टपम्

1-64a-39b इहपुण्यकृतो यान्ति स्वर्गलोकं न संशयः ॥ 1-64a-40a इह लोके दुष्कृतिनोनरकं यान्ति निर्घृणाः

1-64a-40b दक्षिणाबन्ध इत्युक्ते उभेसुकृतदुष्कृते ॥ 1-64a-41a दक्षिणाबन्धसंयुक्ता योगिनः प्रपतन्ति ते

1-64a-41b तस्मान्नो मानसीं कन्यां योगाद्भ्रष्टां विशापते ॥ 1-64a-42aसुतात्वं तव संप्राप्तां सतीं भिक्षां ददस्व वै

1-64a-42b इत्युक्त्वापितरः सर्वे क्षणादन्तर्हितास्तदा ॥ 1-64a-43x वैशंपायन उवाच

1-64a-43a याज्ञवल्क्यं समाहूय विवाहाचार्यमित्युत

1-64a-43b वसुंचापि समाहूय वसिष्ठो मुनिभिः सह ॥ 1-64a-44a विवाहं कारयामासविधिदृष्टेन कर्मणा

1-64a-44x वसुरुवाच

1-64a-44b पराशर महाप्राज्ञतव दास्याम्यहं सुताम् ॥ 1-64a-45a प्रतीच्छ चैनां भद्रं ते पाणिंगृह्णीष्व पाणिना

1-64a-45x वैशंपायन उवाच

1-64a-45b वसोस्तु वचनंश्रुत्वा याज्ञवल्क्यमते स्थितः ॥ 1-64a-46a कृतकौतुकमङ्गल्यः पाणिनापाणिमस्पृशत्

1-64a-46b प्रभूताज्येन हविषा हुत्वामन्त्रैर्हुताशनम् ॥ 1-64a-47a त्रिरग्निं तु परिक्रम्य समभ्यर्च्यहुताशनम्

1-64a-47b महर्षीन्याज्ञवल्क्यादीन्दक्षिणाभिः प्रतर्प्यच ॥ 1-64a-48a लब्धानुज्ञोऽभिवाद्याशु प्रदक्षिणमथाकरोत्

1-64a-48b पराशरेकृतोद्वाहे देवाः सर्पिगणास्तदा ॥ 1-64a-49a हृष्टा जग्मुः क्षणादेववेदव्यासो भवत्विति

1-64a-49b एवं सत्यवती हृष्टा पूजां लब्ध्वायथेष्टतः ॥ 1-64a-50a पराशरेण संयुक्ता सद्यो गर्भं सुषाव सा

1-64a-50b जज्ञे च यमुनाद्वीपे पाराशर्यः स वीर्यवान् ॥ 1-64a-51aजातमात्रः स ववृधे सप्तवर्षोऽभवत्तदा

1-64a-51b स्नात्वाभिवाद्य पितरंतस्थौ व्यासः समाहितः ॥ 1-64a-52a स्वस्तीति वचनं चोक्त्वा ददौकलशमुत्तमम्

1-64a-52b गृहीत्वा कलशं प्राप्ते तस्थौ व्यासः समाहितः ॥ 1-64a-53a ततो दाशभयात्पत्नी स्नात्वा कन्या बभूव सा

1-64a-53bअभिवाद्य मुनेः पादौ पुत्रं जग्राह पाणिना ॥ 1-64a-54a स्पृष्टमात्रेतु निर्भर्त्स्य मातरं वाक्यमब्रवीत्

1-64a-54b मम पित्रा तुसंस्पर्शान्मातस्त्वमभवः शुचिः ॥ 1-64a-55a अद्य दाशसुता कन्या नस्पृशेर्मामनिन्दिते

1-64a-55x वैशंपायन उवाच

1-64a-55b व्यासस्यवचनं श्रुत्वा बाष्पपूर्णमुखी तदा ॥ 1-64a-56a मनुष्यभावात्सायोषित्पपात मुनिपादयोः

1-64a-56b महाप्रसादो भगवान्पुत्रं प्रोवाचधर्मवित् ॥ 1-64a-57a मा त्वमेवंविधं कार्षीर्नैतद्धर्म्यं मतं हि नः

1-64a-57b दूष्यौ न मातापितरौ तथा पूर्वोपकारिणौ ॥ 1-64a-58aधारणाद्दुःखसहनात्तयोर्माता गरीयसी

1-64a-58b बीजक्षेत्रसमायोगे सस्यंजायेत लौकिकम् ॥ 1-64a-59a जायते च सुतस्तद्वत्पुरुषस्त्रीसमागमे

1-64a-59b मृगीणां पक्षिणां चैव अप्सराणां तथैव च ॥ 1-64a-60aशूद्रयोन्यां च जायन्ते मुनयो वेदपारगाः

1-64a-60b ऋष्यशृङ्गोमृगीपुत्रः कण्वो बर्हिसुतस्तथा ॥ 1-64a-61a अगस्त्यश्च वसिष्ठश्चउर्वश्यां जनितावुभौ

1-64a-61b सोमश्रवास्तु सर्प्यां तुअश्विनावश्विसंभवौ ॥ 1-64a-62a स्कन्दः स्कन्नेन शुक्लेन जातः शरवणेपुरा

1-64a-62b एवमेव च देवानामृषीमां चैव संभवः ॥ 1-64a-63aलोकवेदप्रवृत्तिर्हि न मीमांस्या बुधैः सदा

1-64a-63b वेदव्यास इतिप्रोक्तः पुराणे च स्वयंभुवा ॥ 1-64a-64a धर्मनेता महर्षीणांमनुष्याणां त्वमेव च

1-64a-64b तस्मात्पुत्र न दूष्येत वासवीयोगचारिणी ॥ 1-64a-65a मत्प्रीत्यर्थं महाप्राज्ञ सस्नेहं वक्तुमर्हसि

1-64a-65b प्रजाहितार्थं संभूतो विष्णोर्भागो महानृषिः ॥ 1-64a-66aतस्मात्स्वमातरं स्नेहात्प्रबवीहि तपोधन

1-64a-66x वैशंपायन उवाच

1-64a-66b गुरोर्वचनमाज्ञाय व्यासः प्रीतोऽभवत्तदा ॥ 1-64a-67aचिन्तयित्वा लोकवृत्तं मातुरङ्कमथाविशत्

1-64a-67b पुत्रस्पर्शात्तुलोकेषु नान्यत्सुखमतीव हि ॥ 1-64a-68a व्यासं कमलपत्राक्षंपरिष्वज्याश्र्ववर्तयत्

1-64a-68b स्तन्यासारैः क्लिद्यमानापुत्रमाघ्राय मूर्धनि ॥ 1-64a-69x वासव्युवाच

1-64a-69aपुत्रलाभात्परं लोके नास्तीह प्रसवार्थिनाम्

1-64a-69b दुर्लभं चेतिमन्येऽहं मया प्राप्तं महत्तपः ॥ 1-64a-70a महता तपसा तात महायोगबलेनच

1-64a-70b मया त्वं हि महाप्राज्ञ लब्धोऽमृतमिवामरैः ॥ 1-64a-71aतस्मात्त्वं मामृषेः पुत्र त्यक्तुं नार्हसि सांप्रतम्

1-64a-71xवैशंपायन उवाच

1-64a-71b एवमुक्तस्ततः स्नेहाद्व्यासो मातरमब्रवीत् ॥ 1-64a-72a त्वया स्पृष्टः परिष्वक्तो मूर्ध्नि चाघ्रायितो मुहुः

1-64a-72b एतावन्मात्रया प्रीतो भविष्येषु नृपात्मजे ॥ 1-64a-73aस्मृतोऽहं दर्शयिष्यामि कृत्येष्विति च सोऽब्रवीत्

1-64a-73b समातरमनुज्ञाप्य तपस्येव मनो दधे ॥ 1-64a-74a ततः कन्यामनुज्ञाय पुनःकन्या भवत्विति

1-64a-74b पराशरोऽपि भगवान्पुत्रेण सहितो ययौ ॥ 1-64a-75a गत्वाश्रमपदं पुम्यमदृश्यन्त्या पराशरः

1-64a-75bजातकर्मादिसंस्कारं कारयामास धर्मतः ॥ 1-64a-76a कृतोपनयनो व्यासोयाज्ञवल्क्येन भारत

1-64a-76b वेदानधिजगौ साङ्गानोङ्कारेणत्रिमात्रया ॥ 1-64a-77a गुरवे दक्षिणां दत्त्वा तपः कर्तुं प्रचक्रमे

1-64a-77b एवं द्वैपायनो जज्ञे सत्यवत्यां पराशरात् ॥ 1-64a-78a द्वीपन्यस्तः स यद्वालस्तस्माद्द्वैपायनोऽभवत्

1-64a-78b पादापसारिणं धर्मंविद्वान्स तु युगे युगे ॥ 1-64a-79a आयुः शक्तिं च मर्त्यानांयुगाद्युगमवेक्ष्य च

1-64a-79b ब्रह्मर्षिर्ब्राह्मणानां चतथाऽनुग्रहकाङ्क्षया ॥ 1-64a-80a विव्यास वेदान्यस्माच्च वेदव्यास इतिस्मृतः

1-64a-80b ततः स महर्षिर्विद्वाञ्शिष्यानाहूय धर्मतः ॥ 1-64a-81a सुमन्तुं जैमिनिं पैलं शुकं चैव स्वमात्मजम्

1-64a-81bप्रभुर्वरिष्ठो वरदो वैशंपायनमेव च ॥ 1-64a-82a वेदानध्यापयामासमहाभारतपञ्चमान्

1-64a-82b संहितास्तैः पृथक्त्वेन भारतस्यप्रकीर्तिता ॥ 1-64a-83a ततः सत्यवती हृष्टा जगाम स्वं निवेशनम्

1-64a-83b तस्यास्तु योजनाद्गन्धमाजिघ्रन्ति नरा भुवि ॥ 1-64a-84aदाशराजस्तु तद्गन्धमाजिघ्रन्पीतिमावहत्

1-64a-84x दाशराज उवाच

1-64a-84b त्वामाहुर्मत्स्यगन्धेति कथं बाले सुगन्धता ॥ 1-64a-85aअपास्य मत्स्यगन्धत्वं केन दत्ता सुगन्धता 1-64a-85x सत्यवत्युवाच

1-64a-85b शक्ते- पुत्रो महाप्राज्ञः पराशर इति श्रुतः ॥ 1-64a-86aनावं वाहयमानाया मम दृष्ट्वां सुशिक्षितम्

1-64a-86b उपास्यमत्स्यगन्धत्वं योजनाद्गन्धतां ददौ ॥ 1-64a-87a ऋषेः प्रसादं दृष्ट्वातु जनाः प्रीतिमुपागमन्

1-64a-87b एवं लब्धो मया गन्धो न रोषंकर्तुमर्हसि ॥ 1-64a-88a दाशराजस्तु तद्वाक्यं प्रशशंस ननन्द च

1-64a-88b एतत्पवित्रं पुण्यं च व्याससमवमुत्तमम्

1-64a-88cइतिहासमिमं श्रुत्वा प्रजावन्तो भवन्ति च
\twolineshloka
{तथा भीष्मः शान्तनवो गङ्गायाममितद्युतिः}
{वसुवीर्यात्समभवन्महावीर्यो महायशाः}


\twolineshloka
{वैदार्थविच्च भगवानृषिर्विप्रो महायशाः}
{शूले प्रोतः पुराणर्षिरचोरश्चोरशङ्कया}


\twolineshloka
{अणीमाण्डव्य इत्येवं विख्यातः स महायशाः}
{स धर्ममाहूय पुरा महर्षिरिदमुक्तवान्}


\twolineshloka
{इषीकया मया बाल्याद्विद्धा ह्येका शकुन्तिका}
{तत्किल्बिषं स्मरे धर्म नान्यत्पापमहं स्मरे}


\twolineshloka
{तन्मे सहस्रममितं कस्मान्नेहाजयत्तपः}
{गरीयान्ब्राह्मणवधः सर्वभूतवधाद्यतः}


\threelineshloka
{तस्मात्त्वं किल्बिषादस्माच्छूद्रयोनौ जनिष्यसि}
{वैशंपायन उवाच}
{तेन शापेन धर्मोऽपि शूद्रयोनावजायत}


\twolineshloka
{विद्वान्विदुररूपेण धार्मिकः किल्बिषात्ततः}
{संजयो मुनिकल्पस्तु जज्ञे सूतो गवल्गणात्}


\twolineshloka
{सूर्याच्च कुन्तिकन्यायां जज्ञे कर्णो महाबलः}
{सहजं कवचं बिभ्रत्कुण्डलोद्योतिताननः}


\twolineshloka
{अनुग्रहार्थं लोकानां विष्णुर्लोकनमस्कृतः}
{वसुदेवात्तु देवक्यां प्रादुर्भूतो महायशाः}


\twolineshloka
{अनादिनिधनो देवः स कर्ता जगतः प्रभुः}
{अव्यक्तमक्षरं ब्रह्म प्रधानं त्रिगुणात्मकम्}


\twolineshloka
{आत्मानमव्ययं चैव प्रकृतिं प्रभवं प्रभुम्}
{पुरुषं विश्वकर्माणं सत्वयोगं ध्रुवाक्षरम्}


\twolineshloka
{अनन्तमचलं देवं हंसं नारायणं प्रभुम्}
{धातारमजमव्यक्तं यमाहुः परमव्ययम्}


\twolineshloka
{कैवल्यं निर्गुणं विश्वमनादिमजमव्ययम्}
{पुरुषः स विभुः कर्ता सर्वभूतपितामहः}


\twolineshloka
{धर्मसंस्थापनार्थाय प्रजज्ञेऽन्धकवृष्णिषु}
{अस्त्रज्ञौ तु महावीर्यौ सर्वशास्त्रविशारदौ}


\twolineshloka
{सात्यकिः कृतवर्मा च नारायणमनुव्रतौ}
{सत्यकाद्धृदिकाच्चैव जज्ञातेऽस्त्रविशारदौ}


\twolineshloka
{भरद्वाजस्य च स्कन्नं द्रोण्यां शुक्रमवर्धत}
{सहर्षेरुग्रतपसस्तस्माद्द्रोणो व्यजायत}


\twolineshloka
{गौतमान्मिथुनं जज्ञे शरस्तम्बाच्छरद्वतः}
{अश्वत्थाम्नश्च जननी कृपश्चैव महाबलः}


\twolineshloka
{अश्वत्थामा ततो जज्ञे द्रोणादेव महाबलः}
{तथैव धृष्टद्युम्नोऽपि साक्षादग्निसमद्युतिः}


\twolineshloka
{वैताने कर्मणि तते पावकात्समजायत}
{वीरो द्रोणविनाशाय धनुरादाय वीर्यवान्}


\twolineshloka
{तथैव वेद्यां कृष्णापि जज्ञे तेजस्विनी शुभा}
{विभ्राजमाना वपुषा बिभ्रती रूपमुत्तमम्}


\twolineshloka
{प्रह्रादशिष्यो नग्नजित्सुबलश्चाभवत्ततः}
{तस्य प्रजा धर्महन्त्री जज्ञे देवप्रकोपनात्}


\twolineshloka
{गान्धारराजपुत्रोऽभूच्छकुनिः सौबलस्तथा}
{दुर्योधनस्य जननी जज्ञातेऽर्थविशारदौ}


\twolineshloka
{कृष्णद्वैपायनाज्जज्ञे धृतराष्ट्रो जनेश्वरः}
{क्षेत्रे विचित्रवीर्यस्य पाण्डुश्चैव महाबलः}


\twolineshloka
{धर्मार्थकुशलो धीमान्मेधावी धूतकल्मषः}
{विदुरः शूद्रयोनौ तु जज्ञे द्वैपायनादपि}


\twolineshloka
{पाण्डोस्तु जज्ञिरे पञ्च पुत्रा देवसमाः पृथक्}
{द्वयोः स्त्रियोर्गुणज्येष्ठस्तेषामासीद्युधिष्ठिरः}


\twolineshloka
{धर्माद्युधिष्ठिरो जज्ञे मारुताच्च वृकोदरः}
{इन्द्राद्धनंजयः श्रीमान्सर्वशस्त्रभृतां वरः}


\twolineshloka
{जज्ञाते रूपसंपन्नावश्विभ्यां च यमावपि}
{नकुलः सहदेवश्च गुरुशुश्रूषणे रतौ}


\twolineshloka
{तथा पुत्रशतं जज्ञे धृतराष्ट्रस्य धीमतः}
{दुर्योधनप्रभृतयो युयुत्सुः करणस्तथा}


\twolineshloka
{ततो दुःशासनश्चैव दुःसहश्चापि भारत}
{दुर्मर्षणो विकर्णश्च चित्रसेनो विविंशतिः}


\twolineshloka
{जयः सत्यव्रतश्चैव पुरुमित्रश्च भारत}
{वैश्यापुत्रो युयुत्सुश्च एकादश महारथाः}


\twolineshloka
{अभिमन्युः सुभद्रायामर्जुनादभ्यजायत}
{स्वस्रीयो वासुदेवस्य पौत्रः पाण्डोर्महात्मनः}


\twolineshloka
{पाण्डवेभ्यो हि पाञ्चाल्यां द्रौपद्यां पञ्च जज्ञिरे}
{कुमारा रूपसंपन्नाः सर्वशास्त्रविशारदाः}


\twolineshloka
{प्रतिविन्ध्यो युधिष्ठिरात्सुतसोमो वृकोदरात्}
{अर्जुनाच्छ्रुतकीर्तिस्तु शतानीकस्तु नाकुलिः}


\twolineshloka
{तथैव सहदेवाच्च श्रुतसेनः प्रतापवान्}
{हिडिम्बायां च भीमेन वने जज्ञे घटोत्कचः}


\twolineshloka
{शिखण्डी द्रुपदाज्जज्ञे कन्या पुत्रत्वभागता}
{यां यक्षः पुरुषं चक्रे स्थूमः प्रियचिकीर्षया}


\twolineshloka
{कुरूणां विग्रहे तस्मिन्समागच्छन्बहून्यथ}
{राज्ञां शतसहस्राणि योत्स्यमानानि संयुगे}


\threelineshloka
{तेषामपरिमेयानां नामधेयानि सर्वशः}
{न शक्यानि समाख्यातुं वर्षाणामयुतैरपि}
{एते तु कीर्तिता मुख्या यैराख्यानमिदं ततम्}


\chapter{अध्यायः ६५}
\twolineshloka
{जनमेजय उवाच}
{}


\twolineshloka
{य एते कीर्तिता ब्रह्मन्ये चान्ये नानुकीर्तिताः}
{सम्यक्ताञ्श्रोतुमिच्छामिराज्ञश्चान्यान्सहस्रशः ॥ 1}


\threelineshloka
{यदर्थमिह संभूता देवकल्पा महारथाः}
{भुवि तन्मे महाभाग सम्यगाख्यातुमर्हसि ॥वैशम्पायन उवाच}
{}


\twolineshloka
{रहस्यं खल्विदं राजन्देवानामिति नः श्रुतम्}
{तत्तु ते कथयिष्यामि नमस्कृत्वा स्वयंभुवे}


\twolineshloka
{त्रिःसप्तकृत्वः पृथिवीं कृत्वा निःक्षित्रयां पुरा}
{जामदग्न्यस्तपस्तेपे महेन्द्रे पर्वतोत्तमे}


\twolineshloka
{तदा निःक्षत्रिये लोके भार्गवेण कृते सति}
{ब्राह्मणान्क्षत्रिया राजन्सुतार्थिन्योऽभिचक्रमुः}


\twolineshloka
{ताभिः सह समापेतुर्ब्राह्मणाः संशितव्रताः}
{ऋतावृतौ नरव्याघ्र न कामान्नानृतौ तथा}


\twolineshloka
{तेभ्यश्च तेभिरे गर्भं क्षत्रियास्ताः सहस्रशः}
{ततः सुषुविरे राजन्क्षत्रियान्वीर्यवत्तरान्}


\twolineshloka
{कुमारांश्च कुमारीश्च पुनः क्षत्राभिवृद्धये}
{एवं तद्ब्राह्मणैः क्षत्रं क्षत्रियासु तपस्विभिः}


\twolineshloka
{जातं वृद्धं च धर्मेण सुदीर्गेणायुषान्वितम्}
{चत्वारोऽपि ततो वर्णा बभूवुर्ब्राह्मणोत्तराः}


\twolineshloka
{अभ्यगच्छन्नृतौ नारीं न कामान्नानृतौ तथा}
{तथैवान्यानि भूतानि तिर्यग्योनिगतान्यपि}


\twolineshloka
{ऋतौ दारांश्च गच्छन्ति तत्तथा भरतर्षभ}
{ततोऽवर्धन्त धर्मेण सहस्रशतजीविनः}


\twolineshloka
{ताः प्रजाः पृथिवीपाल धर्मव्रतपरायणाः}
{आधिभिर्व्याधिभिश्चैव विमुक्ताः सर्वशो नराः}


\twolineshloka
{अथेमां सागरोपान्तां गां गजेन्द्रगताखिलाम्}
{अध्यतिष्ठत्पुनः क्षत्रं सशैलवनपत्तनाम्}


\twolineshloka
{प्रशासति पुनः क्षत्रे धर्मेणेमां वसुन्धराम्}
{ब्राह्मणाद्यास्ततो वर्णा लेभिरे मुदमुत्तमाम्}


\twolineshloka
{कामक्रोधोद्भवान्दोषान्निरस्य च नराधिपाः}
{धर्मेण दण्डं दण्डेषु प्रणयन्तोऽन्वपालयन्}


\twolineshloka
{तथा धर्मपरे क्षत्रे सहस्राक्षः शतक्रतुः}
{स्वादु देशे च काले च ववर्षाप्याययन्प्रजाः}


\twolineshloka
{न बाल एव म्रियते तदा कश्चिज्जनाधिप}
{न च स्त्रियं प्रजानाति कश्चिदप्राप्तयौवनाम्}


\twolineshloka
{एवमायुष्मतीभिस्तु प्रजाभिर्भरतर्षभ}
{इयं सागरपर्यन्ता ससापूर्यत मेदिनी}


\twolineshloka
{ईजिरे च महायज्ञैः क्षत्रिया बहुदक्षिणैः}
{साङ्गोपनिषदान्वेदान्विप्राश्चाधीयते तदा}


\twolineshloka
{न च विक्रीणते ब्र्हम ब्राह्मणाश्च तदा नृप}
{न च शूद्रसमभ्याशे वेदानुच्चारयन्त्युत}


\twolineshloka
{कारयन्तः कृषिं गोभिस्तथा वैश्याः क्षिताविह}
{युञ्जते धुरि नो गाश्च कृशाङ्गांश्चाप्यजीवयन्}


\twolineshloka
{फेनपांश्च तथा वत्सान्न दुहन्ति स्म मानवाः}
{न कूटमानैर्वणिजः पण्यं विक्रीणते तदा}


\twolineshloka
{कर्माणि च नरव्याघ्र धर्मोपेतानि मानवाः}
{धर्ममेवानुपश्यन्तश्चक्रुर्धर्मपरायणाः}


\twolineshloka
{स्वकर्मनिरताश्चासन्सर्वे वर्णा नराधिप}
{एवं तदा नरव्याघ्र धर्मो न ह्रसते क्वचित्}


\twolineshloka
{काले गावः प्रसूयन्ते नार्यश्च भरतर्षभ}
{भवन्त्यृतुषु वृक्षाणां पुष्पाणि च फलानि च}


\twolineshloka
{एवं कृतयुगे सम्यग्वर्तमाने तदा नृप}
{आपूर्यत मही कृत्स्ना प्राणिभिर्बहुभिर्भृशम्}


\twolineshloka
{एवं समुदिते लोके मानुषे भरतर्षभ}
{असुरा जज्ञिरे क्षेत्रे राज्ञां तु मनुजेश्वर}


\twolineshloka
{आदित्यैर्हि तदा दैत्या बहुशो निर्जिता युधि}
{ऐश्वर्याद्धंशिताः स्वर्गात्संबभूवुः क्षिताविह}


\twolineshloka
{इह देवत्वमिच्छन्तो मानुषेषु तपस्विनः}
{जज्ञिरे भुवि भूतेषु तेषु तेष्वसुरा विभो}


\twolineshloka
{गोष्वश्वेषु च राजेन्द्र खरोष्ट्रमहिषेषु च}
{क्रव्यात्सु चैव भूतेषु गजेषु च मृगेषु च}


\twolineshloka
{जातैरिह महीपाल जायमानैश्च तैर्मही}
{न शशाकात्मनात्मानमियं धारयितुं धरा}


\twolineshloka
{अथ जाता महीपालाः केचिद्बहुमदान्विताः}
{दितेः पुत्रा दनोश्चैव तदा लोकादिह च्युताः}


\twolineshloka
{वीर्यवन्तोऽवलिप्तास्ते नानारूपधरा महीम्}
{इमां सागरपर्यन्तां परीयुररिमर्दनाः}


\twolineshloka
{ब्राह्मणान्क्षत्रियान्वैश्याञ्शूद्रांश्चैवाप्यपीडयन्}
{अन्यानि चैव सत्वानि पीडयामासुरोजसा}


\twolineshloka
{त्रासयन्तोऽभिनिघ्नन्तः सर्वभूतगणांश्च ते}
{विचेरुः सर्वशो राजन्महीं शतसहस्रशः}


\twolineshloka
{आश्रमस्थान्महर्षींश्च धर्षयन्तस्ततस्ततः}
{अब्रह्मण्या वीर्यमदा मत्ता मदबलेन च}


\twolineshloka
{एवं वीर्यबलोत्सिक्तैर्भूरियं तैर्महासुरैः}
{पीड्यमाना मही राजन्ब्रह्माणमुपचक्रमे}


\twolineshloka
{न ह्यमी भूतसत्वौघाः पन्नगाः सनगां महीम्}
{तदा धारयितुं शेकुराक्रान्तां दानवैर्बलात्}


\twolineshloka
{ततो मही महीपाल भारार्ता भयपीडिता}
{जगाम शरणं देवं सर्वभूतपितामहम्}


\twolineshloka
{सा संवृतं महाभागैर्देवद्विजमहर्षिभिः}
{ददर्श देवं ब्रह्माणं लोककर्तारमव्ययम्}


\twolineshloka
{गन्धर्वैरप्सरोभिश्च बन्दिकर्मसु निष्ठितैः}
{वन्द्यमानं मुदोपतैर्ववन्दे चैनमेत्य सा}


\twolineshloka
{अथ विज्ञापयामास भूमिस्तं शरणार्थिनी}
{सन्निधौ लोकपालानां सर्वेषामेव भारत}


\twolineshloka
{तत्प्रधानात्मनस्तस्य भूमेः कृत्यं स्वयंभुवः}
{पूर्वमेवाभवद्राजन्विदितं परमेष्ठिनः}


\twolineshloka
{स्रष्टा हि जगतः कस्मान्न संबुध्येत भारत}
{ससुरासुरलोकानामशेषेण मनोगतम्}


\threelineshloka
{तामुवाच महाराज भूमिं भूमिपतिः प्रभुः}
{प्रभवः सर्वभूतानामीशः शंभुः प्रजापतिः ॥ब्रह्मोवाच}
{}


\twolineshloka
{यदर्थमभिसंप्राप्ता मत्सकाशं वसुन्धरे}
{तदर्थं सन्नियोक्ष्यामि सर्वानेव दिवौकसः}


\fourlineindentedshloka
{`उत्तिष्ठ गच्छ वसुधे स्वस्थानमिति साऽगमत्}
{' वैशंपायन उवाच}
{इत्युक्त्वा स महीं देवो ब्रह्मा राजन्विसृज्य च}
{आदिदेश तदा सर्वान्विबुधान्भूतकृत्स्वयम्}


\twolineshloka
{अस्या भूमेर्निरसितुं भारं भागैः पृथक्पृथक्}
{अस्यामेव प्रसूयध्वं तिरोधायेति चाब्रवीत्}


\twolineshloka
{तथैव च समानीय गन्धर्वाप्सरसां गणान्}
{उवाच भगवान्सर्वानिदं वचनमर्थवत्}


% Check verse!
ब्रह्मोवाच

स्वैः स्वैरंशैः प्रसूयध्वं यथेष्टं मानेषेषु च

वैशम्पायन उवाच

अथ शक्रादयः सर्वे श्रुत्वा सुरगुरोर्वचः

तथ्यमर्थ्यं च पथ्यं च तस्य ते जगृहुस्तदा
\twolineshloka
{अथ ते सर्वशोंशैः स्वैर्गन्तुं भूमिं कृतक्षणाः}
{नारायणममित्रघ्नं वैकुण्ठमुपचक्रमुः}


\twolineshloka
{यः स चक्रगदापाणिः पीतवासाः शितिप्रभः}
{पद्मनाभः सुरारिघ्नः पृथुचार्वञ्चितेक्षणः}


\twolineshloka
{प्रजापतिपतिर्देवः सुरनाथो महाबलः}
{श्रीवत्साङ्को हृषीकेशः सर्वदैवतपूजितः}


\twolineshloka
{तं भुवः शोधनायेन्द्र उवाच पुरुषोत्तमम्}
{अंशेनावतरेत्येवं तथेत्याह च तं हरिः}


\chapter{अध्यायः ६६}
\twolineshloka
{वैशंपायन उवाच}
{}


\twolineshloka
{अथ नारायणेनेन्द्रश्चकार सह संविदम्}
{अवतर्तुं महीं स्वर्गादंशतः महितः सुरैः}


\twolineshloka
{आदिश्य च स्वयं शक्रः सर्वानेव दिवौकसः}
{निर्जगाम पुनस्तस्मात्क्षयान्नारायणस्य ह}


\twolineshloka
{तेऽमरारिविनाशाय सर्वलोकहिताय च}
{अवतेरुः क्रमेणैव महीं स्वर्गाद्दिवौकसः}


\twolineshloka
{ततो ब्रह्मर्षिवंशेषु पार्थिवर्षिकुलेषु च}
{जज्ञिरे राजशार्दूल यथाकामं दिवौकसः}


\twolineshloka
{दानवान्राक्षसांश्चैव गन्धर्वान्पन्नगांस्तथा}
{पुरुषादानि चान्यानि जघ्नुः सत्वान्यनेकशः}


\threelineshloka
{दानवा राक्षसाश्चैव गन्धर्वाः पन्नगास्तथा}
{न तान्बलस्थान्बाल्येऽपि जघ्नुर्भरतसत्तम ॥जनमेजय उवाच}
{}


\twolineshloka
{देवदानवसङ्घानां गन्धर्वाप्सरसां तथा}
{मानवानां च सर्वेषां तथा वै यक्षरक्षसाम्}


\threelineshloka
{श्रोतुमिच्छामि तत्त्वेन संभवं कृत्स्नमादितः}
{प्राणिनां चैव सर्वेषां संभवं वक्तुमर्हसि ॥वैशम्पायन उवाच}
{}


\twolineshloka
{हन्त ते कथयिष्यामि नमस्कृत्य स्वयंभुवे}
{सुरादीनामहं सम्यग्लोकानां प्रभवाप्ययम्}


\twolineshloka
{ब्रह्मणो मानसाः पुत्रा विदिताः षण्महर्षयः}
{मरीचिरत्र्यह्गिरसौ पुलस्त्यः पुलहः क्रतुः}


\twolineshloka
{मरीचेः कश्यपः पुत्रः कश्यपात्तु इमाः प्रजाः}
{प्रजज्ञिरे महाभागा दक्षकन्यास्त्रयोदश}


\twolineshloka
{अदितिर्दितिर्दनुः काला दनायुः सिंहिका तथा}
{क्रोधा प्राधा च विश्वा च विनता कपिला मुनिः}


\twolineshloka
{कद्रूश्च मनुजव्याघ्र दक्षकन्यैव भारत}
{एतासां वीर्यसंपन्नं पुत्रपौत्रमनन्तकम्}


\twolineshloka
{अदित्यां द्वादशादित्याः संभूता भुवनेश्वराः}
{ये राजन्नामतस्तांस्ते कीर्तयिष्यामि भारत}


\twolineshloka
{धाता मित्रोऽर्यमा शक्रो वरुणस्त्वंश एव च}
{भगो विवस्वान्पूषा च सविता दशमस्तथा}


\twolineshloka
{एकादशस्तथा त्वष्टा द्वादशो विष्णुरुच्यते}
{जघन्यजस्तु सर्वेषामादित्यानां गुणाधिकः}


\twolineshloka
{एक एव दितेः पुत्रो हिरण्यकशिपुः स्मृतः}
{नाम्ना ख्यातास्तु तस्येमे पञ्च पुत्रा महात्मनः}


\twolineshloka
{प्रह्लादः पूर्वजस्तेषां संह्लादस्तदनन्तरम्}
{अनुह्लादस्तृतीयोऽभूत्तस्माच्च शिबिबाष्कलौ}


\twolineshloka
{प्रह्लादस्य त्रयः पुत्राः ख्याताः सर्वत्र भारत}
{विरोचनश्च कुम्भश्च निकुम्भश्चेति भारत}


\twolineshloka
{विरोचनस्य पुत्रोऽभूद्बलिरेकः प्रतापवान्}
{बलेश्च प्रथितः पुत्रो बाणो नाम महासुरः}


\twolineshloka
{रुद्रस्यानुचरः श्रीमान्महाकालेति यं विदुः}
{चत्वारिंशद्दनोः पुत्राः ख्याताः सर्वत्र भारत}


\twolineshloka
{तेषां प्रथमजो राजा विप्रचित्तिर्महायशाः}
{शम्बरो नमुचिश्चैव पुलोमा चेति विश्रुतः}


\twolineshloka
{असिलोमा च केशी च दुर्जयश्चैव दानवः}
{अयःशिरा अश्वशिरा अश्वशह्कुश्च वीर्यवान्}


\twolineshloka
{तथा गगनमूर्धा च वेगवान्केतुमांश्च सः}
{स्वर्भानुरश्वोऽश्वपतिर्वृषपर्वाऽजकस्तथा}


\twolineshloka
{अश्वग्रीवश्च सूक्ष्मश्च तुहुण्डश्च महाबलः}
{इषुपादेकचक्रश्च विरूपाक्षहराहरौ}


\threelineshloka
{निचन्द्रश्च निकुम्भश्च कुपटः कपटस्तथा}
{शरभः शलभश्चैव सूर्याचन्द्रमसौ तथा}
{एते ख्याता दनोर्वंशे दानवाः परिकीर्तिताः}


\twolineshloka
{अन्यौ तु खलु देवानां सूर्याचन्द्रमसौ स्मृतौ}
{अन्यौ दानवमुख्यानां सूर्याचन्द्रमसौ तथा}


\twolineshloka
{इमे च वंशाः प्रथिताः सत्ववन्तो महाबलाः}
{दनुपुत्रा महाराज दश दानववंशजाः}


\twolineshloka
{एकाक्षो मृतपो वीरः प्रलम्बनरकावपि}
{वातापिः शत्रुतपनः शठश्चैव महासुरः}


\twolineshloka
{गविष्ठश्च वनायुश्च दीर्घजिह्वश्च दानवः}
{असङ्ख्येयाः स्मृतास्तेषां पुत्राः पौत्राश्च भारत}


\twolineshloka
{सिंहिका सुषुवे पुत्रं राहुं चन्द्रार्कमर्दनम्}
{सुचन्द्रं चन्द्रहर्तारं तथा चन्द्रप्रमर्दनम्}


\twolineshloka
{क्रूरस्वभावं क्रूरायाः पुत्रपौत्रमनन्तकम्}
{गणः क्रोधवशो नाम क्रूरकर्माऽरिमर्दनः}


\twolineshloka
{दनायुषः पुनः पुत्राश्चत्वारोऽसुरपुंगवाः}
{विक्षरो बलवीरौ च वृत्रश्चैव महासुरः}


\twolineshloka
{कालायाः प्रथिताः पुत्राः कालकल्पाः प्रहारिणः}
{प्रविख्याता महावीर्या दानवेषु परन्तपाः}


\twolineshloka
{विनाशनश्च क्रोधश्च क्रोधहन्ता तथैव च}
{क्रोधशत्रुस्तथैवान्ये कालकेया इति श्रुताः}


\twolineshloka
{असुराणामुपाध्यायः शक्रस्त्वषिसुतोऽभवत्}
{ख्याताश्चोशनसः पुत्राश्चत्वारोऽसुरयाजकाः}


\twolineshloka
{त्वष्टा धरस्तथात्रिश्च द्वावन्यौ रौद्रकर्मिणौ}
{तेजसा सूर्यसंकाशा ब्रह्मलोकपरायणाः}


\twolineshloka
{इत्येष वंशप्रभवः कथितस्ते तरस्विनाम्}
{असुराणां सुराणां च पुराणे संश्रुतो मया}


% Check verse!
एतेषां यदपत्यं तु न शक्यं तदशेषतः ॥प्रसंख्यातुं महीपाल गुणभूतमनन्तकम्
\twolineshloka
{तार्क्ष्यश्चारिष्टनेमिश्च तथैव गरुडारुणौ}
{आरुणिर्वारुणिश्चैव वैनतेयाः प्रकीर्तिताः}


\twolineshloka
{शेषोऽनन्तो वासुकिश्च तक्षकश्च भुजङ्गमः}
{कूर्मश्च कुलिकश्चैव काद्रवेयाः प्रकीर्तिताः}


\twolineshloka
{भीमसेनोग्रसेनौ च सुपर्णो वरुणस्तथा}
{गोपतिर्धृतराष्ट्रश्च सूर्यवर्चाश्च सप्तमः}


\twolineshloka
{सत्यवागर्कपर्णश्च प्रयुतश्चापि विश्रुतः}
{भीमश्चित्ररथश्चैव विख्यातः सर्वविद्वशी}


\twolineshloka
{तथा शालिशिरा राजन्पर्जन्यश्च चतुर्दशः}
{कलिः पञ्चदशस्तेषां नारदश्चैव षोडशः ॥इत्येते देवगन्धर्वा मौनेयाः परिकीर्तिताः}


\twolineshloka
{अथ प्रभूतान्यन्यानि कीर्तयिष्यामि भारत}
{अनवद्यां मनुं वंशामसुरां मार्गणप्रियाम्}


\twolineshloka
{अरूपां सुभगां भासीमिति प्राधा व्यजायत}
{सिद्धः पूर्णश्च बर्हिश्च पूर्णायुश्च महायशाः}


\twolineshloka
{ब्रह्मचारी रतिगुणः सुपर्णश्चैव सप्तमः}
{विश्वावसुश्च भानुश्च सुचन्द्रो दशमस्तथा}


\twolineshloka
{इत्येते देवगन्धर्वाः प्राधेयाः परिकीर्तिताः}
{इमं त्वप्सरसां वंशं विदितं पुण्यलक्षणम्}


\twolineshloka
{अरिष्टाऽसूत सुभगा देवी देवर्षितः पुरा}
{अलम्बुषा मिश्रकेशी विद्युत्पर्णा तिलोत्तमा}


\twolineshloka
{अरुणा रक्षिता चैव रम्बा तद्वन्मनोरमा}
{केशिनी च सुबाहुश्च सुरता सुरजा तथा}


\twolineshloka
{सुप्रिया चातिबाहुश्च विख्यातौ च हाहा हूहूः}
{तुम्बुरुश्चेति चत्वारः स्मृता गन्धर्वसत्तमाः}


\twolineshloka
{अमृतं ब्राह्मणा गावो गन्धर्वाप्सरसस्तथा}
{अपत्यं कपिलायास्तु पुराणे परिकीर्तितम्}


\twolineshloka
{इति ते सर्वभूतानां संभवः कथितो मया}
{यथावत्संपरिख्यातो गन्धर्वाप्सरसां तथा}


\twolineshloka
{भुजंगानां सुपर्णानां रुद्राणां मरुतां तथा}
{गवां च ब्राह्मणानां च श्रीमतां पुण्यकर्मणाम्}


\twolineshloka
{आयुष्यश्चैव पुण्यश्च धन्यः श्रुतिसुखावहः}
{श्रोतव्यश्चैव सततं श्राव्यश्चैवानसूयता}


\twolineshloka
{इमं तु वंशं नियमेन यः पठे-न्महात्मनां ब्राह्मणदेवसन्निधौ}
{अपत्यलाभं लभते स पुष्कलंश्रियं यशः प्रेत्य च शोभनां गतिम्}


\chapter{अध्यायः ६७}
\twolineshloka
{वैशंपायन उवाच}
{}


\twolineshloka
{ब्रह्मणो मानसाः पुत्रा विदिताः षण्महर्षयः}
{एकादश सुताः स्थाणोः ख्याताः परमतेजसः}


\twolineshloka
{मृगव्याधश्च सर्पश्च निर्ऋतिश्च महायशाः}
{अजैकपादहिर्बिध्न्यः पिनाकी च परन्तपः}


\twolineshloka
{दहनोऽथेश्वरश्चैव कपाली च महाद्युतिः}
{स्थाणुर्भगश्च भगवान् रुद्रा एकादश स्मृताः}


\twolineshloka
{मरीचिरङ्गिरा अत्रिः पुलस्त्यः पुलहः क्रतुः}
{षडेते ब्रह्मणः पुत्रा वीर्यवन्तो महर्षयः}


\twolineshloka
{त्रयस्त्वङ्गिरसः पुत्रा लोके सर्वत्र विश्रुताः}
{बृहस्पतिरुतथ्यश्च संवर्तश्च धृतव्रताः}


\twolineshloka
{अत्रेस्तु बहवः पुत्राः श्रूयन्ते मनुजाधिप}
{सर्वे वेदविदः सिद्धाः शान्तात्मानो महर्षयः}


\twolineshloka
{राक्षसाश्च पुलस्त्यस्य वानराः किन्नरास्तथा}
{यक्षाश्च मनुजव्याघ्र पुत्रास्तस्य च धीमतः}


\twolineshloka
{पुलहस्य सुता राजञ्शरभाश्च प्रकीर्तिताः}
{सिंहाः किपुरुषा व्याघ्रा ऋक्षा ईहामृगास्तथा}


\twolineshloka
{क्रतोः क्रतुसमाः पुत्राः पतङ्गसहचारिणः}
{विश्रुतास्त्रिषु लोकेषु सत्यव्रतपरायणाः}


\twolineshloka
{दक्षस्त्वजायताङ्गुष्ठाद्दक्षिणाद्भगवानृषिः}
{ब्रह्मणः पृथिवीपाल शान्तात्मा सुमहातपाः}


\twolineshloka
{वामादजायताङ्गुष्ठाद्भार्या तस्य महात्मनः}
{तस्यां पञ्चाशतं कन्याः स एवाजनयन्मुनिः}


\twolineshloka
{ताः सर्वास्त्वनवद्याङ्ग्यः कन्याः कमललोचनाः}
{पुत्रिकाः स्थापयामास नष्टपुत्रः प्रजापतिः}


\twolineshloka
{ददौ स दश धर्माय सप्तविंशतिमिन्दवे}
{दिव्येन विधिना राजन्कश्यपाय त्रयोदश}


\twolineshloka
{नामतो धर्मपत्न्यस्ताः कीर्त्यमाना निबोध मे}
{कीर्तिर्लक्ष्मीर्धृतिर्मेधा पुष्टिः श्रद्धा क्रिया तथा}


\twolineshloka
{बुद्धिर्लज्जा मतिश्चैव पत्न्यो धर्मस्य ता दश}
{द्वाराण्येतानि धर्मस्य विहितानि स्वयंभुवा}


\twolineshloka
{सप्तविंशतिः सोमस्य पत्न्यो लोकस्य विश्रुताः}
{कालस्य नयने युक्ताः सोमपत्न्याः शुचिव्रताः}


\threelineshloka
{सर्वा नक्षत्रयोगिन्यो लोकयात्राविधानतः}
{पैतामहो मुनिर्देवस्तस्य पुत्रः प्रजापतिः}
{तस्याष्टौ वसवः पुत्रास्तेषां वक्ष्यामि विस्तरम्}


\twolineshloka
{धरो ध्रुवश्च सोमश्च अहश्चैवानिलोऽनलः}
{प्रत्यूषश्च प्रभासश्च वसवोऽष्टौ प्रकीर्तिताः}


\twolineshloka
{धूम्रायास्तु धरः पुत्रो ब्रह्मविद्यो ध्रुवस्तथा}
{चन्द्रमास्तु मनस्विन्याः श्वासायाः श्वसनस्तथा}


\twolineshloka
{रतायाश्चाप्यहः पुत्रः शाण्डिल्याश्च हुताशनः}
{प्रत्यूषश्च प्रभासश्च प्रभातायाः सुतौ स्मृतौ}


\threelineshloka
{धरस्य पुत्रो द्रविणो हुतहव्यवहस्तथा}
{`आपस्य पुत्रो वैतण्ड्यः श्रमः शान्तोमुनिस्तथा'}
{ध्रुवस्य पुत्रो भगवान्कालो लोकप्रकालनः}


\twolineshloka
{सोमस्य तु सुतो वर्चा वर्चस्वी येन जायते}
{मनोहरायाः शिशिरः प्राणोऽथ रमणस्तथा}


\twolineshloka
{अह्नः सुतस्तथा ज्योतिः शमः शान्तस्तथा मुनिः}
{अग्नेः पुत्रः कुमारस्तु श्रीमाञ्छरवणालयः}


\twolineshloka
{तस्य शाखो विशाखश्च नैगमेयश्च पृष्ठजः}
{कृत्तिकाभ्युपपत्तेश्च कार्तिकेय इति स्मृतः}


\twolineshloka
{अनिलस्य शिवा भार्या तस्याः पुत्रो मनोजवः}
{अविज्ञातगतिश्चैव द्वौ पुत्रावनिलस्य तु}


\threelineshloka
{प्रत्यूषस्य विदुः पुत्रमृषिं नाम्नाऽथ देवलम्}
{द्वौ पुत्रौ देवलस्यापि क्षमावन्तौ मनीषिणौ}
{बृहस्पतेस्तु भगिनी वरस्त्री ब्रह्मवादिनी}


\twolineshloka
{योगसिद्धा जगत्कृत्स्नमसक्ता विचचार ह}
{प्रभासस्य तु भार्या सा वसूनामष्टमस्य ह}


\twolineshloka
{विश्वकर्मा महाभागो जज्ञे शिल्पप्रजापतिः}
{कर्ता शिल्पसहस्राणां त्रिदशानां च वर्धकिः}


\twolineshloka
{भूषणानां च सर्वेषां कर्ता शिल्पवतां वरः}
{यो दिव्यानि विमानानि त्रिदशानां चकारह}


\twolineshloka
{मनुष्याश्चोपजीवन्ति यस्य शिल्पं महात्मनः}
{पूजयन्ति च यं नित्यं विश्वकर्माणमव्ययम्}


\twolineshloka
{स्तनं तु दक्षिणं भित्त्वा ब्रह्मणो नरविग्रहः}
{निःसृतो भगवान्धर्मः सर्वलोकसुखावहः}


\twolineshloka
{त्रयस्तस्य वराः पुत्राः सर्वभूतमनोहराः}
{शमः कामश्च हर्षश्च तेजसा लोकधारिणः}


\twolineshloka
{कामस्य तु रतिर्भार्या शमस्य प्राप्तिरङ्गना}
{नन्दा तु भार्या हर्षस्य यासु लोकाः प्रतिष्ठिताः}


\twolineshloka
{मरीचेः कश्यपः पुत्रः कश्यपस्य सुरासुराः}
{जज्ञिरे नृपशार्दूल लोकानां प्रभवस्तु सः}


\twolineshloka
{त्वाष्ट्री तु सवितुर्भार्या वडवारूपधारिणी}
{असूयत महाभागा सान्तरिक्षेऽस्विनावुभौ}


\twolineshloka
{द्वादशैवादितेः पुत्राः शक्रमुख्या नराधिप}
{तेषामवरजो विष्णुर्यत्र लोकाः प्रतिष्ठिताः}


\twolineshloka
{त्रयस्त्रिंशत यत्येते देवास्तेषामहं तव}
{अन्वयं संप्रवक्ष्यामि पक्षैश्च कुलतो गणान्}


\twolineshloka
{रुद्राणामपरः पक्षः साध्यानां मरुतां तथा}
{वसूनां भार्गवं विद्याद्विश्वेदेवांस्तथैव च}


\twolineshloka
{वैनतेयस्तु गरुडो बलवानरुणस्तथा}
{बृहस्पतिश्च भगवानादित्येष्वेव गण्यते}


\twolineshloka
{अश्विनौ गुह्यकान्विद्धि सर्वौषध्यस्तथा पशून्}
{एते देवगणा राजन्कीर्तितास्तेऽनुपूर्वशः}


\twolineshloka
{यान्कीर्तयित्वा मनुजः सर्वपापैः प्रमुच्यते}
{ब्रह्मणो हृदयं भित्त्वा निःसृतो भगवान्भृगुः}


\threelineshloka
{भृगोः पुत्रः कविर्विद्वाञ्छुक्रः कविसुतो ग्रहः}
{त्रैलोक्यप्राणयात्रार्थं वर्षावर्षे भयाभये}
{स्वयंभुवा नियुक्तः सन्भुवनं परिधावति}


\twolineshloka
{योगाचार्यो महाबुद्धिर्दैत्यानामभवद्गुरुः}
{सुराणां चापि मेधावी ब्रह्मचारी यतव्रतः}


\twolineshloka
{तस्मिन्नियुक्ते विधिना योगक्षेमाय भार्गवे}
{अन्यमुत्पादयामास पुत्रं भृगुरनिन्दितम्}


\twolineshloka
{च्यवनं दीप्ततपसं धर्मात्मानं यशस्विनम्}
{यः स रोषाच्च्युतो गर्भान्मातुर्मोक्षाय भारत}


\twolineshloka
{आरुषी तु मनोः कन्या तस्य पत्नी मनीषिणः}
{और्वस्तस्यां समभवदूरुं भित्त्वा महायशाः}


\twolineshloka
{महातेजा महावीर्यो बाल एव गुणैर्युतः}
{ऋचीकस्तस्य पुत्रस्तु जमदग्निस्ततोऽभवत्}


\threelineshloka
{जमदग्नेस्तु चत्वार आसन्पुत्रा महात्मनः}
{रामस्तेषां जघन्योऽभूदजघन्यैर्गुणैर्युतः}
{सर्वशस्त्रेषु कुशलः क्षत्रियान्तकरो वशी}


\twolineshloka
{और्वस्यासीत्पुत्रशतं जमदग्निपुरोगमम्}
{तेषां पुत्रसहस्राणि बभूवुर्भुवि विस्तरः}


\twolineshloka
{द्वौ पुत्रो ब्रह्मणस्त्वन्यौ ययोस्तिष्ठति लक्षणम्}
{लोके धाता विधाता च यौ स्थितौ मनुना सह}


\twolineshloka
{तयोरेव स्वसा देवी लक्ष्मीः पद्मगृहा शुभा}
{तस्यास्तु मानसाः पुत्रास्तुरगा व्योमचारिणः}


\twolineshloka
{वरुणस्य भार्या या ज्येष्ठा शुक्राद्देवी व्यजायत}
{तस्याः पुत्रं बलं विद्धि सुरां च सुरनन्दिनीम्}


\twolineshloka
{प्रजानामन्नकामानामन्योन्यपरिभक्षणात्}
{अधर्मस्तत्र संजातः सर्वभूतविनाशकः}


\twolineshloka
{तस्यापि निर्ऋतिर्भार्या नैर्ऋता येन राक्षसाः}
{घोरास्तस्यास्त्रयः पुत्राः पापकर्मरताः सदा}


\twolineshloka
{भयो महाभयश्चैव मृत्युर्भूतान्तकस्तथा}
{न तस्य भार्या पुत्रो वा कश्चिदस्त्यन्तको हि सः}


\twolineshloka
{काकीं श्येनीं तथा भासीं धृतराष्ट्रीं तथा शुकीम्}
{ताम्रा तु सुषुवे देवी पञ्चैता लोकविश्रुताः}


\twolineshloka
{उलूकान्सुषुवे काकी श्येनी श्येनान्व्यजायत}
{भासी भासानजनयद्गृध्रांश्चैव जनाधिप}


\twolineshloka
{धृतराष्ट्री तु हंसांश्च कलहंसांश्च सर्वशः}
{चक्रवाकांश्च भद्रा तु जनयामास सैव तु}


\twolineshloka
{शुकी च जनयामास शुकानेव यशस्विनी}
{कल्याणगुणसंपन्ना सर्वलक्षणपूजिता}


\twolineshloka
{नव क्रोधवशा नारीः प्रजज्ञे क्रोधसंभवाः}
{मृगी च मृगमन्दा च हरी भद्रमना अपि}


\twolineshloka
{मातङ्गी त्वथ शार्दूली श्वेता सुरभिरेव च}
{सर्वलक्षणसंपन्ना सुरसा चैव भामिनी}


\twolineshloka
{अपत्यं तु मृगाः सर्वे मृग्या नरवरोत्तम}
{ऋक्षाश्च मृगमन्दायाः सृमराश्च परंतप}


\twolineshloka
{ततस्त्वैरावतं नागं जज्ञे भद्रमनाः सुतम्}
{ऐरावतः सुतस्तस्या देवनागो महागजः}


\twolineshloka
{हर्याश्च हरयोऽपत्यं वानराश्च तरस्विनः}
{गोलाङ्गूलांश्च भद्रं ते हर्याः पुत्रान्प्रचक्षते}


\twolineshloka
{प्रजज्ञे त्वथ शार्दूली सिंहान्व्याग्राननेकशः}
{द्वीपिनश्च महासत्वान्सर्वानेव न सशंयः}


\twolineshloka
{मातङ्ग्यपि च मातङ्गानपत्यानि नराधिप}
{दिशां गजं तु श्वेताख्यं श्वेताऽजनयदाशुगम्}


\twolineshloka
{तथा दुहितरौ राजन्सुरभिर्वै व्यजायत}
{रोहिणी चैव भद्रं ते गन्धर्वी तु यशस्विनी}


\twolineshloka
{विमलामपि भद्रं ते अनलामपि भारत}
{रोहिण्यां जज्ञिरे गावो गन्धर्व्यां वाजिनः सुताः}


\twolineshloka
{`इरायाः कन्यका जातास्तिस्रः कमललोचनाः}
{वनस्पतीनां वृक्षाणां वीरुधां चैव मातरः}


\twolineshloka
{लतारुहे च द्वे प्रोक्ते वीरुधां चैव ताः स्मृताः}
{गृह्णन्ति ये विना पुष्पं फलानि तरवः पृथक्}


\twolineshloka
{लतासुतास्ते विज्ञेयास्तानेवाहुर्वनस्पतीन्}
{पुष्पैः फलग्रहान्वृक्षान्रुहायाः प्रसवं विदुः}


\twolineshloka
{लतागुल्मानि वृक्षाश्च त्वक्सारतृणजन्तवः}
{वीरुधो याः प्रजास्तस्यास्तत्र वंशः समाप्यते}


% Check verse!
सप्तपिम्डफलान्वृक्षाननलापि व्यजायत
\twolineshloka
{अनलायाः शुकी पुत्री कंकस्तु सुरसासुतः}
{अरुणस्य भार्या श्येनी तु वीर्यवन्तौ महाबलौ}


\twolineshloka
{संपातिं जनयामास वीर्यवन्तं जटायुषम्}
{सुरसाऽजनयन्नागान्कद्रूः पुत्रांस्तु पन्नगान्}


\threelineshloka
{द्वौ पुत्रौ विनतायास्तु विख्यातौ गरुडारुणौ}
{इत्येष सर्वभूतानां महतां मनुजाधिप}
{प्रभवः कीर्तितः सम्यङ्मया मतिमतां वर}


\twolineshloka
{यं श्रुत्वा पुरुषः सम्यङ्मुक्तो भवति पाप्मनः}
{सर्वज्ञतां च लभते रतिमग्र्यां च विन्दति}


\chapter{अध्यायः ६८}
\twolineshloka
{जनमेजय उवाच}
{}


\twolineshloka
{देवानां दानवानां च गन्धर्वोरगरक्षसाम्}
{सिंहव्याघ्रमृगाणां च पन्नगानां पतत्त्रिणाम्}


\fourlineindentedshloka
{अन्येषां चैव भूतानां संभवं भगवन्नहम्}
{श्रोतुमिच्छामि तत्त्वेन मानुषेषु महात्मनाम्}
{जन्म कर्म च भूतानामेतेषामनुपूर्वशः ॥वैशंपायन उवाच}
{}


\twolineshloka
{मानुषेषु मनुष्येन्द्र संभूता ये दिवौकसः}
{प्रथमं दानवाश्चैव तांस्ते वक्ष्यामि सर्वशः}


\twolineshloka
{विप्रचित्तिरिति ख्यातो य आसीद्दानवर्षभः}
{जरासन्ध इति ख्यातः स आसीन्मनुजर्षभः}


\twolineshloka
{दितेः पुत्रस्तु यो राजन्हिरण्यकशिपुः स्मृतः}
{स जज्ञे मानुषे लोके शिशुपालो नरर्षभः}


\twolineshloka
{संह्लाद इति विख्यातः प्रह्लादस्यानुजस्तु यः}
{स शल्य इति विख्यातो जज्ञे वाहीकपुङ्गवः}


\twolineshloka
{अनुह्लादस्तु तेजस्वी योऽभूत्ख्यातो जघन्यजः}
{धृष्टकेतुरिति ख्यातः स बभूव नरेश्वरः}


\twolineshloka
{यस्तु राजञ्शिबिर्नाम दैतेयः परिकीर्तितः}
{द्रुम इत्यभिविख्यातः स आसीद्भुवि पार्थिवः}


\twolineshloka
{बाष्कलो नाम यस्तेषामासीदसुरसत्तमः}
{भगदत्त इति ख्यातः सं जज्ञे पुरुषर्षभः}


\twolineshloka
{अयःशिरा अश्वशिरा अयःशङ्कुश्च वीर्यवान्}
{तथा गगनमूर्धा च वेगवांश्चात्र पञ्चमः}


\threelineshloka
{पञ्चैते जज्ञिरे राजन्वीर्यवन्तो महासुराः}
{केकयेषु महात्मानः पार्थिवर्षभसत्तमाः}
{केतुमानिति विख्यातो यस्ततोऽन्यःप्रतापवान्}


\twolineshloka
{अमितौजा इति ख्यातः सोग्रकर्मा नराधिपः}
{स्वर्भानुरिति विख्यातः श्रीमान्यस्तु महासुरः}


\twolineshloka
{उग्रसेन इति ख्यात उग्रकर्मा नराधिपः}
{यस्त्वश्व इति विख्यातः श्रीमानासीन्महासुरः}


\twolineshloka
{अशोको नाम राजाऽभून्महावीर्योऽपराजितः}
{तस्मादवरजो यस्तु राजन्नश्वपतिः स्मृतः}


\twolineshloka
{दैतेयः सोऽभवद्राजा हार्दिक्यो मनुजर्षभः}
{वृषपर्वेति विख्यातः श्रीमान्यस्तु महासुरः}


\twolineshloka
{दीर्घप्रज्ञ इति ख्यातः पृथिव्यां सोऽभवन्नृपः}
{अजकस्त्ववरो राजन्य आसीद्वृषपर्वणः}


\twolineshloka
{स शाल्व इति विख्यातः पृथिव्यामभवन्नृपः}
{अश्वग्रीव इति ख्यातः सत्ववान्यो महासुरः}


\twolineshloka
{रोचमान इति ख्यातः पृथिव्यां कोऽभवन्नृपः}
{सूक्ष्मस्तु मतिमान्राजन्कीर्तिमान्यः प्रकीर्तितः}


\twolineshloka
{बृहद्रथ इति ख्यातः क्षितावासीत्स पार्थिवः}
{तुहुण्ड इति विख्यातो य आसीदसुरोत्तमः}


\twolineshloka
{सेनाबिन्दुरिति ख्यातः स बूभव नराधिपः}
{इषुमान्नाम यस्तेषामसुराणां बलाधिकः}


\twolineshloka
{नग्नजिन्नाम राजासीद्भुवि विख्यातविक्रमः}
{एकचक्र इति ख्यात आसीद्यस्तु महासुरः}


\twolineshloka
{प्रतिविन्घ्य इति ख्यातो बभूव प्रथितः क्षितौ}
{विरूपाक्षस्तु दैतेयश्चित्रयोधी महासुरः}


\twolineshloka
{चित्रधर्मेति विख्यातः क्षितावासीत्स पार्थिवः}
{हरस्त्वरिहरो वीर आसीद्यो दानवोत्तमः}


\twolineshloka
{सुबाहुरिति विख्यातः श्रीमानासीत्स पार्थिवः}
{अहरस्तु महातेजाः शत्रुपक्षक्षयंकरः}


\twolineshloka
{बाह्लिको नाम राजा स बभूव प्रथितः क्षितौ}
{निचन्द्रश्चन्द्रवक्त्रस्तु य आसीदसुरोत्तमः}


\twolineshloka
{मुञ्जकेश इति ख्यातः श्रीमानासीत्स पार्थिवः}
{निकुम्भस्त्वजितः संख्ये महामतिरजायत}


\twolineshloka
{भूमौ भूमिपतिश्रेष्ठो देवाधिप इति स्मृतः}
{शरभो नाम यस्तेषां दैतेयानां महासुरः}


\twolineshloka
{पौरवो नाम राजर्षिः स बभूव नरोत्तमः}
{कुपटस्तु महावीर्यः श्रीमान्राजन्महासुरः}


\twolineshloka
{सुपार्श्व इति विख्यातः क्षितौ जज्ञे महीपतिः}
{कपटस्तु राजन्राजर्षिः क्षितौ जज्ञे महासुरः}


\twolineshloka
{पार्वतेय इति ख्यातः काञ्चनाचलसन्निभः}
{द्वितीयः शलभस्तेषामसुराणां बभूव ह}


\twolineshloka
{प्रह्लादो नाम बाह्लीकः स बभूव नराधिपः}
{चन्द्रस्तु दितिजश्रेष्ठो लोके ताराधिपोपमः}


\twolineshloka
{चन्द्रवर्मेति विख्यातः काम्बोजानां नराधिपः}
{अर्क इत्यभिविख्यातो यस्तु दानवपुङ्गवः}


\twolineshloka
{ऋषिको नाम राजर्षिर्बभूव नृपसत्तमः}
{मृतपा इति विख्यातो य आसीदसुरोत्तमः}


\twolineshloka
{पश्चिमानूपकं विद्धि तं नृपं नृपसत्तम}
{गविष्ठस्तु महातेजा यः प्रख्यातो महासुरः}


\twolineshloka
{द्रुमसेन इति ख्यातः पृथिव्यां सोऽभवन्नृपः}
{मयूर इति विख्यातः श्रीमान्यस्तु महासुरः}


\twolineshloka
{स विश्व इति विख्यातो बभूव पृथिवीपतिः}
{सुपर्ण इति विख्यातस्तस्मादवरजस्तु यः}


\twolineshloka
{कालकीर्तिरिति ख्यातः पृथिव्यां सोऽभवन्नृपः}
{चन्द्रहन्तेति यस्तेषां कीर्तितः प्रवरोऽसुरः}


\twolineshloka
{शुनको नाम राजर्षिः स बभूव नराधिपः}
{विनाशनस्तु चन्द्रस्य य आख्यातो महासुरः}


\twolineshloka
{जानकिर्नाम विख्यातः सोऽभवन्मनुजाधिपः}
{दीर्घजिह्वस्तु कौरव्य य उक्तो दानवर्षभः}


\threelineshloka
{काशिराजः स विख्यातः पृथिव्यां पृथिवीपते}
{ग्रहं तु सुषुवे यं तु सिंहिकार्केन्दुमर्दनम्}
{स क्राथ इति विख्यातो बभूव मनुजाधिपः}


\twolineshloka
{दनायुषस्तु पुत्राणां चतुर्णां प्रवरोऽसुरः}
{विक्षरो नाम तेजस्वी वसुमित्रो नृपः स्मृतः}


\twolineshloka
{द्वितीयो विक्षराद्यस्तु नराधिप महासुरः}
{पाण्ड्यराष्ट्राधिप इति विख्यातः सोऽभवन्नृपः}


\twolineshloka
{बली वीर इति ख्यातो यस्त्वासीदसुरोत्तमः}
{पौण्ड्रमात्स्यक इत्येवं बभूव स नराधिपः}


\twolineshloka
{वृत्र इत्यभिविख्यातो यस्तु राजन्महासुरः}
{मणिमान्नाम राजर्षिः स बभूव नराधिपः}


\twolineshloka
{क्रोधहन्तेति यस्तस्य बभूवावरजोऽसुरः}
{दण्ड इत्यभिविख्यातः स आसीन्नृपतिः क्षितौ}


\twolineshloka
{क्रोधवर्धन इत्येवं यस्त्वन्यः परिकीर्तितः}
{दण्डधार इति ख्यातः सोऽभवन्मनुजर्षभः}


\twolineshloka
{कालेयानां तु ये पुत्रास्तेषामष्टौ नराधिपाः}
{जज्ञिरे राजशार्दूल शार्दूलसमविक्रमाः}


\twolineshloka
{मगधेषु जयत्सेनस्तेषामासीत्स पार्थिवः}
{अष्टानां प्रवरस्तेषां कालेयानां महासुरः}


\twolineshloka
{द्वितीयस्तु ततस्तेषां श्रीमान्हरिहयोपमः}
{अपराजित इत्येवं स बभूव नराधिपः}


\twolineshloka
{तृतीयस्तु महातेजा महामायो महासुरः}
{निषादाधिपतिर्जज्ञे भुवि भीमपराक्रमः}


\twolineshloka
{तेषामन्यतमो यस्तु चतुर्थः परिकीर्तितः}
{श्रेणिमानिति विख्यातः क्षितौ राजर्षिसत्तमः}


\twolineshloka
{पञ्चमस्त्वभवत्तेषां प्रवरो यो महासुरः}
{महौजा इति विख्यातो बभूवेह परन्दपः}


\twolineshloka
{षष्ठस्तु मतिमान्यो वै तेषामासीन्महासुरः}
{अभीरुरिति विख्यातः क्षितौ राजर्षिसत्तमः}


\twolineshloka
{समुद्रसेनस्तु नृपस्तेषामेवाभवद्गणात्}
{विश्रुतः सागरान्तायां क्षितौ धर्मार्थतत्त्ववित्}


\twolineshloka
{बृहन्नामाष्टमस्तेषां कालेयानां नराधिप}
{बभूव राजा धर्मात्मा सर्वभूतहिते रतः}


\twolineshloka
{कुक्षिस्तु राजन्विख्यातो दानवानां महाबलः}
{पार्वतीय इति ख्यातः काञ्चनाचलसन्निभः}


\twolineshloka
{क्रथनश्च महावीर्यः श्रीमान्राजा महासुरः}
{सूर्याक्ष इति विख्यातः क्षितौ जज्ञे महीपतिः}


\twolineshloka
{असुराणां तु यः सकूर्यः श्रीमांश्चैव महासुरः}
{दरदो नाम बाह्लीको वरः सर्वमहीक्षिताम्}


\twolineshloka
{गणः क्रोधवशो नाम यस्ते राजन्प्रकीर्तितः}
{ततः संजज्ञिरे वीराः क्षिताविह नराधिपाः}


\twolineshloka
{मद्रकः कर्णवेष्टश्च सिद्धार्थः कीटकस्तथा}
{सुवीरश्च सुबाहुश्च महावीरोऽथ बाह्लिकः}


\twolineshloka
{क्रथो विचित्रः सुरथः श्रीमान्नीलश्च भूमिपः}
{चीरवासाश्च कौरव्य भूमिपालश्च नामतः}


\twolineshloka
{दन्तवक्त्रश्च नामासीद्दुर्जयश्चैव दानवः}
{रुक्मी च नृपशार्दूलो राजा च जनमेजयः}


\twolineshloka
{आषाढो वायुवेगश्च भूरितेजास्तथैव च}
{एकलव्यः सुमित्रश्च वाटधानोऽथ गोमुखः}


\twolineshloka
{कारूषकाश्च राजानः क्षेमधूर्तिस्तथैव च}
{श्रुतायुरुद्वहश्चैव बृहत्सेनस्तथैव च}


\twolineshloka
{क्षेमोग्रतीर्थः कुहरः कलिङ्गेषु नराधिपः}
{मतिमांश्च मनुष्येन्द्र ईश्वरश्चेति विश्रुतः}


\twolineshloka
{गणात्क्रोधवशादेष राजपूगोऽभवत्क्षितौ}
{जातः पुरा महाभागो महाकीर्तिर्महाबलः}


\twolineshloka
{कालनेमिरिति ख्यातो दानवानां महाबलः}
{स कंस इति विख्यात उग्रसेनसुतो बली}


\twolineshloka
{यस्त्वासीद्देवको नाम देवराजसमद्युतिः}
{स गन्धर्वपतिर्मुख्यः क्षितौ जज्ञे नराधिपः}


\twolineshloka
{बृहस्पतेर्बृहत्कीर्तेर्देवर्षेर्विद्धि भारत}
{अशाद्द्रोणं समुत्पन्नं भारद्वाजमयोनिजम्}


\twolineshloka
{धन्विनां नृपशार्दूल यः सर्वास्त्रविदुत्तमः}
{महाकीर्तिर्महातेजाः स जज्ञे मनुजेश्वर}


\twolineshloka
{धनुर्वेदे च वेदे च यं तं वेदविदो विदुः}
{वरिष्ठं चित्रकर्माणं द्रोणं स्वकुलवर्धनम्}


\twolineshloka
{महादेवान्तकाभ्यां च कामात्क्रोधाच्च भारत}
{एकत्वमुपसंपद्य जज्ञे शूरः परन्तपः}


\twolineshloka
{अश्वत्थामा महावीर्यः शत्रुपक्षभयावहः}
{वीरः कमलपत्राक्षः क्षितावासीन्नराधिपः}


\twolineshloka
{जज्ञिरे वसवस्त्वष्टौ गङ्गायां शन्तनोः सुताः}
{वसिष्ठस्य च शापेन नियोगाद्वासवस्य च}


\twolineshloka
{तेषामवरजो भीष्मः कुरूणामभयङ्करः}
{मतिमान्वेदविद्वाग्मी शत्रुपक्षक्षयङ्करः}


\twolineshloka
{जामदग्न्येन रामेण सर्वास्त्रविदुषां वरः}
{योऽप्युध्यत महातेजा भार्गवेण महात्मना}


\twolineshloka
{यस्तु राजन्कृपो नाम ब्रह्मर्षिरभवत्क्षितौ}
{रुद्राणां तु गणाद्विद्धि संभूतमतिपौरुषम्}


\twolineshloka
{शकुनिर्नाम यस्त्वासीद्राजा लोके महारथः}
{द्वापरं विद्धि तं राजन्संभूतमरिमर्दनम्}


\twolineshloka
{सात्यकिः सत्यसन्धश्च योऽसौ वृष्णिकुलोद्वहः}
{पक्षात्स जज्ञे मरुतां देवानामरिमर्दनः}


\twolineshloka
{द्रुपदश्चैव राजर्षिस्तत एवाभवद्गणात्}
{मानुषे नृप लोकेऽस्मिन्सर्वशस्त्रभृतां वरः}


\twolineshloka
{ततश्च कृतवर्माणं विद्धि राजञ्जनाधिपम्}
{तमप्रतिमकर्माणं क्षत्रियर्षभसत्तमम्}


\twolineshloka
{मरुतां तु गणाद्विद्धि संजातमरिमर्दनम्}
{विराटं नाम राजानं परराष्ट्रप्रतापनम्}


\twolineshloka
{अरिष्टायास्तु यः पुत्रो हंस इत्यभिविश्रुतः}
{स गन्धर्वपतिर्जज्ञे कुरुवंशविवर्धनः}


\twolineshloka
{धृतराष्ट्र इति ख्यातः कृष्णद्वैपायनात्मजः}
{दीर्घबाहुर्महातेजाः प्रज्ञाचक्षुर्नराधिपः ॥मातुर्दोषादृषेः कोपादन्ध एव व्यजायत}


\threelineshloka
{`मरुतां तु गणाद्वीरः सर्वशस्त्रभृतां वरः}
{पाण्डुर्जज्ञे महाबाहुस्तव पूर्वपितामहः}
{'तस्यैवावरजो भ्राता महासत्वो महाबलः}


\twolineshloka
{धर्मात्तु सुमहाभागं पुत्रं पुत्रवतां वरम्}
{विदुरं विद्धि तं लोके जातं बुद्धिमतां वरम्}


\twolineshloka
{कलेरंशस्तु संजज्ञे भुवि दुर्योधनो नृपः}
{दुर्बद्धिर्दुर्मतिश्चैव कुरूणामयशस्करः}


\twolineshloka
{जगतो यस्तु सर्वस्य विद्विष्टः कलिपूरुषः}
{यः सर्वां घातयामास पृथिवीं पृथिवीपते}


\twolineshloka
{उद्दीपितं येन वैरं भूतान्तकरणं महत्}
{पौलस्त्या भ्रातरश्चास्य जज्ञिरे मनुजेष्विह}


\twolineshloka
{शतं दुःशासनादीनां सर्वेषां क्रूरकर्मणाम्}
{दुर्मुखो दुःसहश्चैव ये चान्ये नानुकीर्तिताः}


\threelineshloka
{दुर्योधनसहायास्ते पौलस्त्या भरतर्षभ}
{वैश्यापुत्रो युयुत्सुश्च धार्तराष्ट्रः शताधिकः ॥जनमेजय उवाच}
{}


\threelineshloka
{ज्येष्ठानुज्येष्ठतामेषां नामधेयानि वा विभो}
{धृतराष्ट्रस्य पुत्राणामानुपूर्व्येण कीर्तय ॥वैशंपायन उवाच}
{}


\twolineshloka
{दुर्योधनो युयुत्सुश्च राजन्दुःशासनस्तथा}
{दुःसहो दुःशलश्चैव दुर्मुखश्च तथापरः}


\twolineshloka
{विविंशतिर्विकर्णश्च जलसन्धः सुलोचनः}
{विन्दानुविन्दौ दुर्धर्षः सुबाहुर्दुष्प्रधर्षणः}


\twolineshloka
{दुर्मर्षणो दुर्मुखश्च दुष्कर्णः कर्ण एव च}
{चत्रोपचित्रौ चित्राक्षश्चारुचित्राङ्गदश्च ह}


\twolineshloka
{दुर्मदो दुष्प्रहर्षश्च विवित्सुर्विकटः समः}
{ऊर्णनाभः पद्मनाभस्तथा नन्दोपनन्दकौ}


\twolineshloka
{सेनापतिः सुषेणश्च कुण्डोदरमहोदरौ}
{चित्रबाहुश्चित्रवर्मा सुवर्मा दुर्विरोचनः}


\twolineshloka
{अयोबाहुर्महाबाहुश्चित्रचापसुकुण्डलौ}
{भीमवेगो भीमबलो बलाकी भीमविक्रमः}


\twolineshloka
{उग्रायुधो भीमशरः कनकायुर्दृढायुधः}
{दृढवर्मा दृढक्षत्रः सोमकीर्तिरनूदरः}


\twolineshloka
{जरासन्धो दृढसन्धः सत्यसन्धः सहस्रवाक्}
{उग्रश्रवा उग्रसेनः क्षेममूर्तिस्तथैव च}


% Check verse!
अपराजितः पण्डितको विशालाक्षो दुराधनः
\twolineshloka
{दृढहस्तः सुहस्तश्च वातवेगसुवर्चसौ}
{आदित्यकेतुर्बह्वाशी नागदत्तानुयायिनौ}


\twolineshloka
{कवाची निषङ्गी दण्डी दण्डधारो धनुर्ग्रहः}
{उग्रो भीमरथो वीरो वीरबाहुरलोलुपः}


\twolineshloka
{अभयो रौद्रकर्मा च तथा दृढरथश्च यः}
{अनाधृष्यः कुम्डभेदी विरावी दीर्घलोचनः}


\twolineshloka
{दीर्घबाहुर्महाबाहुर्व्यूढोरुः कनकाङ्गदः}
{कुण्डजश्चित्रकश्चैव दुःशला च शताधिका}


\twolineshloka
{वैश्यापुत्रो युयुत्सुश्च धार्तराष्ट्रः शताधिकः}
{एतदेकशतं राजन्कन्या चैका प्रकीर्तिता}


\twolineshloka
{नामधेयानुपूर्व्या च ज्येष्ठानुज्येष्ठतां विदुः}
{सर्वे त्वतिरथाः शूराः सर्वे युद्धविशारदाः}


\twolineshloka
{सर्वे वेदविदश्चैव राजञ्शास्त्रे च परागाः}
{सर्वे सङ्घ्रामविद्यासु विद्याभिजनशोभिनः}


\twolineshloka
{सर्वेषामनुरूपाश्च कृता दारा महीपते}
{दुःशलां समये राजसिन्धुराजाय कौरवः}


\twolineshloka
{जयद्रथाय प्रददौ सौबलानुमते तदा}
{धर्मस्यांशं तु राजानं विद्धि राजन्युधिष्ठिरम्}


\twolineshloka
{भीमसेनं तु वातस्य देवराजस्य चार्जुनम्}
{अश्विनोस्तु तथैवांशौ रूपेणाप्रतिमौ भुवि}


\twolineshloka
{नकुलः सहदेवश्च सर्वभूतमनोहरौ}
{स्युवर्चा इति ख्यातः सोमपुत्रः प्रतापवान्}


\twolineshloka
{सोऽभिमन्युर्बृहत्कीर्तिरर्जुनस्य सुतोऽभवत्}
{यस्यावतरणे राजन्सुरान्सोमोऽब्रवीदिदम्}


\twolineshloka
{नाहं दद्यां प्रियं पुत्रं मम प्राणैर्गरीयसम्}
{समयः क्रियतामेष न शक्यमतिवर्तितुम्}


\twolineshloka
{सुरकार्यं हि नः कार्यमसुराणां क्षितौ वधः}
{तत्र यास्यत्ययं वर्चा न च स्थास्यति वै चिरम्}


\twolineshloka
{ऐन्द्रिर्नरस्तु भविता यस्य नारायणः सखा}
{सोर्जुनेत्यभिविख्यातः पाण्डोः पुत्रः प्रतापवान्}


\twolineshloka
{तस्यायं भविता पुत्रो बालो भुवि महारथः}
{ततः षोडशवर्षाणि स्थास्यत्यमरसत्तमाः}


\twolineshloka
{अस्य षोडशवर्षस्य स सङ्ग्रामो भविष्यति}
{यत्रांशा वः करिष्यन्ति कर्म वीरनिषूदनम्}


\twolineshloka
{नरनारायणाभ्यां तु स सङ्ग्रामो विनाकृतः}
{चक्रव्यूहं समास्थाय योधयिष्यन्ति वःसुराः}


\twolineshloka
{विमुखाञ्छात्रवान्सर्वान्कारयिष्यति मे सुतः}
{बालः प्रविश्य च व्यूहमभेद्यं विचरिष्यति}


\twolineshloka
{महारथानां वीराणां कदनं च करिष्यति}
{सर्वेषामेव शत्रूणां चतुर्थांशं नयिष्यति}


\twolineshloka
{दिनार्धेन महाबाहुः प्रेतराजपुरं प्रति}
{ततो महारथैर्वीरैः समेत्य बहुशो रणे}


\twolineshloka
{दिनक्षये महाबाहुर्मया भूयः समेष्यति}
{एकं वंशकरं पुत्रं वीरं वै जनयिष्यति}


\threelineshloka
{प्रनष्टं भारतं वंशं स भूयो धारयिष्यति}
{वैशंपायन उवाच}
{एतत्सोमवचः श्रुत्वा तथास्त्विति दिवौकसः}


\twolineshloka
{प्रत्यूचुः सहिताः सर्वे ताराधिपमपूजयन्}
{एवं ते कथितं राजंस्तव जन्म पितुः पितुः}


\twolineshloka
{अग्नेर्भागं तु विद्धि त्वं धृष्टद्युम्नं महारथण्}
{शिखण्डिनमथो राजंस्त्रीपूर्वं विद्धि राक्षसम्}


\twolineshloka
{द्रौपदेयाश्च ये पञ्च बभूवुर्भरतर्षभ}
{विश्वान्देवगणान्विद्धि संजातान्भरतर्षभ}


\twolineshloka
{प्रतिविन्ध्यः सुतसोमः श्रुतकीर्तिस्तथापरः}
{नाकुलिस्तु शतानीकः श्रुतसेनश्च वीर्यवान्}


\threelineshloka
{शूरो नाम यदुश्रेष्ठो वसुदेवपिताऽभवत्}
{तस्य कन्या पृथा नाम रूपेणासदृशी भुवि}
{}


\twolineshloka
{पितुः स्वस्रीयपुत्राय सोऽनपत्याय वीर्यवान्}
{अग्रमग्रे प्रतिज्ञाय स्वस्यापत्यस्य वै तदा}


\twolineshloka
{अग्रजातेति तां कन्यां शूरोऽनुग्रहकाङ्क्षया}
{अददत्कुन्तिभोजाय स तां दुहितरं तदा}


\twolineshloka
{सा नियुक्ता पितुर्गेहे ब्राह्मणातिथिपूजने}
{उग्रं पर्यचरद्धोरं ब्राह्मणं संशितव्रतम्}


\twolineshloka
{निकूढनिश्चयं धर्मे यं तं दुर्वाससं विदुः}
{समुग्रं शंसितात्मानं सर्वयत्नैरतोषयत्}


\twolineshloka
{तुष्टोऽभिचारसंयुक्तमाचचक्षे यथाविधि}
{उवाच चैनां भगवान्प्रीतोऽस्मि सुभगे तव}


\twolineshloka
{यं यं देवं त्वमेतेन मन्त्रेणावाहयिष्यसि}
{तस्य तस्य प्रसादात्त्वं देवि पुत्राञ्जनिष्यसि}


\twolineshloka
{एवमुक्ता च सा बाला तदा कौतूहलान्विता}
{कन्या सती देवमर्कमाजुहाव यशस्विनी}


\twolineshloka
{प्रकाशकर्ता भगवांस्तस्यां गर्भं दधौ तदा}
{अजीजनत्सुतं चास्यां सर्वशस्त्रभृतांवरम्}


\twolineshloka
{सकुण्डलं सकवचं देवगर्भं श्रियान्वितम्}
{दिवाकरसमं दीप्त्या चारुसर्वाङ्गभूषितम्}


\twolineshloka
{निगूहमाना जातं वै बन्धुपक्षभयात्तदा}
{उत्ससर्ज जले कुन्ती तं कुमारं यशस्विनम्}


\twolineshloka
{तमुत्सृष्टं जले गर्भं राधाभर्ता महायशाः}
{राधायाः कल्पयामास पुत्रं सोऽधिरथस्तदा}


\twolineshloka
{चक्रतुर्नामधेयं च तस्य बालस्य तावुभौ}
{दंपती वसुषेणेति दिक्षु सर्वासु विश्रुतम्}


\twolineshloka
{संवर्धमानो बलवान्सर्वास्त्रेषूत्तमोऽभवत्}
{वेदाङ्गानि च सर्वाणि जजाप जपतां वरः}


\twolineshloka
{यस्मिन्काले जपन्नास्ते धीमान्सत्यपराक्रमः}
{नादेयं ब्राह्मणेष्वासीत्तस्मिन्काले महात्मनः}


\twolineshloka
{तमिन्द्रो ब्राह्मणो भूत्वा पुत्रार्थे भूतभावनः}
{ययाचे कुण्डले वीरं कवचं च सहाङ्गजम्}


% Check verse!
उत्कृत्य कर्णो ह्यददत्कवचं कुण्डले तथा ॥शक्तिं शक्रो ददौ तस्मै विस्मितश्चेदमब्रवीत्
\threelineshloka
{देवासुरमनुष्याणां गन्धर्वोरगरक्षसाम्}
{यस्मिन्क्षेप्स्यसि दुर्धर्ष स एको न भविष्यति ॥वैशंपायन उवाच}
{}


\twolineshloka
{पुरा नाम च तस्यासीद्वसुषेण इति क्षितौ}
{ततो वैकर्तनः कर्णः कर्मणा तेन सोऽभवत्}


\twolineshloka
{आमुक्तकवचो वीरो यस्तु जज्ञे महायशाः}
{स कर्ण इति विख्यातः पृथायाः प्रथमः सुतः}


\twolineshloka
{स तु सूतकुले वीरो ववृधे राजसत्तम}
{कर्णं नरवरश्रेष्ठं सर्वशस्त्रभृतां वरम्}


\twolineshloka
{दुर्योधनस्य सचिवं मित्रं शत्रुविनाशनम्}
{दिवाकरस्य तं विद्धि राजन्नंशमनुत्तमम्}


\twolineshloka
{यस्तु नारायणो नाम देवदेवः सनातनः}
{तस्यांशो मानुषेष्वासीद्वासुदेवः प्रतापवान्}


\twolineshloka
{शेषस्यांशश्च नागस्य बलदेवो महाबलः}
{सनत्कुमारं प्रद्युम्नं विद्धि राजन्महौजसम्}


\twolineshloka
{एवमन्ये मनुष्येन्द्रा बहवोंशा दिवौकसाम्}
{जज्ञिरे वसुदेवस्य कुले कुलविवर्धनाः}


\twolineshloka
{गणस्त्वप्सरसां यो वै मया राजन्प्रकीर्तितः}
{तस्य भागः क्षितौ जज्ञे नियोगाद्वासवस्य ह}


\twolineshloka
{तानि षोडशदेवीनां सहस्राणि नराधिप}
{बभूवुर्मानुषे लोके वासुदेवपरिग्रहः}


\twolineshloka
{श्रियस्तु भागः संजज्ञे रत्यर्थं पृथिवीतले}
{[भीष्मकस्य कुले साध्वी रुक्मिणी नाम नामतः}


\twolineshloka
{द्रौपदी त्वथ संजज्ञे शची भागादनिन्दिता}
{]द्रुपदस्य कुले जाता वेदिमध्यादनिन्दिता}


\twolineshloka
{नातिह्रस्वा न महती नीलोत्पलसुगन्धिनी}
{पद्मायताक्षी सुश्रोणी स्वसिताञ्चितमूर्धजा}


\twolineshloka
{सर्वलक्षणसंपन्ना वैदूर्यमणिसंनिभा}
{पञ्चानां पुरुषेन्द्राणां चित्तप्रमथनी रहः}


\twolineshloka
{सिद्धिर्धृतिश्च ये देव्यौ पञ्चानां मातरौ तु ते}
{कुन्ती माद्री च जज्ञाते मतिस्तु कुबलात्मजा}


\twolineshloka
{इति देवासुराणां ते गन्धर्वाप्सरसां तथा}
{अंशावतरणं राजन्राक्षसानां च कीर्तितम्}


\twolineshloka
{ये पृथिव्यां समुद्भूता राजानो युद्धदुर्मदाः}
{महात्मानो यदूनां च ये जाता विपुले कुले}


\twolineshloka
{ब्राह्मणाः क्षत्रिया वैश्या मया ते परिकीर्तिताः}
{धन्यं यशस्यं पुत्रीयमायुष्यं विजयावहम्}


\twolineshloka
{इदमंशावतरणं श्रोतव्यमनसूयता}
{अंशावतरणं श्रुत्वा देवगन्धर्वरक्षसाम्}


% Check verse!
प्रभवाप्ययवित्प्राज्ञो न कृच्छ्रेष्ववसीदति
\chapter{अध्यायः ६९}
\twolineshloka
{जनमेजय उवाच}
{}


\twolineshloka
{त्वत्तः श्रुतमिदं ब्रह्मन्देवदानवरक्षसाम्}
{अंशावतरणं सम्यग्गन्धर्वाप्सरसां तथा}


\threelineshloka
{इमं तु भूय इच्छामि कुरूणां वंशमादितः}
{कथ्यमानं त्वया विप्र विप्रर्षिगणसन्निधौ ॥वैशंपायन उवाच}
{}


\twolineshloka
{धर्मार्थकामसहितं राजर्षीणां प्रकीर्तितम्}
{पवित्रं कीर्त्यमानं मे निबोधेदं मनीषिणाम्}


\twolineshloka
{प्रजापतेस्तु दक्षस्य मनोर्वैवस्वतस्य च}
{भरतस्य कुरोः पूरोराजमीढस्य चानघ}


\twolineshloka
{यादवानामिमं वंशं कौरवाणां च सर्वशः}
{तथैव भरतानां च पुण्यं स्वस्त्ययनं महत्}


\twolineshloka
{धन्यं यशस्यमायुष्यं कीर्तयिष्यामि तेऽनघ}
{तेजोभिरुदिताः सर्वे महर्षिसमतेजसः}


\twolineshloka
{दश प्राचेतसः पुत्राः सन्तः पुण्यजनाः स्मृताः}
{मुखजेनाग्निना यैस्ते पूर्वं दग्धा महौजसः}


\twolineshloka
{तेभ्यः प्राचेतसो जज्ञे दक्षो दक्षादिमाः प्रजाः}
{संभूताः पुरुषव्याघ्र स हि लोकपितामहः}


\twolineshloka
{वीरिण्या सह सङ्गम्य दक्षः प्राचेतसो मुनिः}
{आत्मतुल्यानजनयत्सहस्रं संशितव्रतान्}


\twolineshloka
{सहस्रसङ्ख्यानसंभूतान्दक्षपुत्रांश्च नारदः}
{मोक्षमध्यापयामास साङ्ख्यज्ञानमनुत्तमम्}


\threelineshloka
{`नाशार्थं योजयामास दिगन्तज्ञानकर्मसु'}
{ततः पञ्चाशतं कन्या पुत्रिका अभिसन्दधे}
{प्रजापतिः प्रजा दक्षः सिसृक्षुर्जनमेजय}


\twolineshloka
{ददौ दश स धर्माय कश्यपाय त्रयोदश}
{कालस्य नयने युक्ताः सप्तविंशतिमिन्दवे}


\twolineshloka
{त्रयोदशानां पत्नीनां या तु दाक्षायणी वरा}
{मारीचः कश्यपस्त्वस्यामादित्यान्समजीजनत्}


\twolineshloka
{इन्द्रादीन्वीर्यसंपन्नान्विवस्वन्तमथापि च}
{विवस्वतः सुतो जज्ञे यमो वैवस्वतः प्रभुः}


\twolineshloka
{`मार्ताण्डस्य यमी चापि सुता राजन्नजायत'}
{मार्तण्डस्य मनुर्धीमानजायत सुतः प्रभुः}


\twolineshloka
{धर्मात्मा स मनुर्धीमान्यत्र वंशः प्रतिष्ठितः}
{मनोर्वंशो मानवानां ततोऽयं प्रथितोऽभवत्}


\twolineshloka
{ब्रह्मक्षत्रादयस्तस्मान्मनोर्जातास्तु मानवाः}
{ततोऽभवन्महाराज ब्रह्म क्षत्रेण सङ्गतम्}


\twolineshloka
{ब्राह्मणा मानवास्तेषां साङ्गं वेदमदारयन्}
{वेनं धृष्णुं नरिष्यन्तं नाभागेक्ष्वाकुमेव च}


\twolineshloka
{कारूषमथ शर्यातिं तथा चैवाष्टमीमिलाम्}
{पृष्टध्रं नवमं प्राहुः क्षत्रधर्मपरायणम्}


\twolineshloka
{नाभागारिष्टदशमान्मनोः पुत्रान्प्रचक्षते}
{पञ्चाशत्तु मनोः पुत्रास्तथैवान्येऽभवन्क्षितौ}


\twolineshloka
{अन्योन्यभेदात्ते सर्वे विनेशुरिति नः श्रुतम्}
{पुरूरवास्ततो विद्वानिलायां समपद्यत}


\twolineshloka
{सा वै तस्याभवन्माता पिता चैवेति नः श्रुतम्}
{त्रयोदश समुद्रस्य द्वीपानश्नन्पुरूरवाः}


\twolineshloka
{अमानुषैर्वृतः सत्वैर्मानुषः सन्महायशाः}
{विप्रैः स विग्रहं चक्रे वीर्योन्मत्तः पुरूरवाः}


\twolineshloka
{जहार च स विप्राणां रत्नान्युत्क्रोशतामपि}
{सनत्कुमारस्तं राजन्ब्रह्मलोकादुपेत्य ह}


\twolineshloka
{अनुदर्शं ततश्चक्रे प्रत्यगृह्णान्न चाप्यसौ}
{ततो महर्षिभिः क्रुद्धैः सद्यः शप्तो व्यनश्यत}


\twolineshloka
{लोभान्वितो बलमदान्नष्टसंज्ञो नराधिपः}
{स हि गन्धर्वलोकस्थानुर्वश्या सहितो विराट्}


\twolineshloka
{आनिनाय क्रियार्थेऽग्नीन्यथावद्विहितांस्त्रिधा}
{षट् सुता जज्ञिरे चैलादायुर्धीमानमावसुः}


\twolineshloka
{दृढायुश्च वनायुश्च शतायुश्चोर्वशीसुताः}
{नहुषं वृद्धशर्माणं रजिं गयमनेनसम्}


\twolineshloka
{स्वर्भानवी सुतानेतानायोः पुत्रान्प्रचक्षते}
{आयुषो नहुषः पुत्रो धीमान्सत्यपराक्रमः}


\twolineshloka
{राज्यं शशास सुमहद्धर्मेण पृथिवीपते}
{पितॄन्देवानृषीन्विप्रान्गन्धर्वोरगराक्षसान्}


\twolineshloka
{नहुषः पालयामास ब्रह्मक्षत्रमथो विशः}
{स हत्वा दस्युसंघातानृषीन्करमदापयत्}


\twolineshloka
{पशुवच्चैव तान्पृष्ठे वाहयामास वीर्यवान्}
{कारयामास चेन्द्रत्वमभिभूय दिवौकसः}


\threelineshloka
{तेजसा तपसा चैव विक्रमेणौजसा तथा}
{`विश्लिष्टो नहुषः शप्तः सद्यो ह्यजगरोऽभवत्'}
{यतिं ययातिं संयातिमायातिमयतिं ध्रुवम्}


\twolineshloka
{नहुषो जनयामास षट् सुतान्प्रियवादिनः}
{यतिस्तु योगमास्थाय ब्रह्मीभूतोऽभवन्मुनिः}


\twolineshloka
{ययातिर्नाहुषः सम्राडासीत्सत्यपराक्रमः}
{स पालयामास महीमीजे च बहुभिर्मखैः}


\twolineshloka
{अतिभक्त्या पितॄनर्चन्देवांश्च प्रयतः सदा}
{अन्वगृह्णात्प्रजाः सर्वा ययातिरपराजितः}


\twolineshloka
{तस्य पुत्रा महेष्वासाः सर्वैः समुदिता गुणैः}
{देवयान्यां महाराज शर्मिष्ठायां च जज्ञिरे}


\twolineshloka
{देवयान्यामजायेतां यदुस्तुर्वसुरेव च}
{द्रुह्युश्चानुश्च पूरुश्च शर्मिष्ठायां च जज्ञिरे}


\twolineshloka
{स शाश्वतीः समा राजन्प्रजा धर्मेण पालयन्}
{जरामार्च्छन्महाघोरां नाहुषो रूपनाशिनीम्}


\twolineshloka
{जराऽभिभूतः पुत्रान्स राजा वचनमब्रवीत्}
{यदुं पूरुं तुर्वसुं च द्रुह्युं चानुं च भारत}


\twolineshloka
{यौवनेन चरन्कामान्युवा युवतिभिः सह}
{बिहर्तुमहमिच्छामि साह्यं कुरुत पुत्रकाः}


\twolineshloka
{तं पुत्रो दैवयानेयः पूर्वजो वाक्यमब्रवीत्}
{किं कार्यं भवतः कार्यमस्माकं यौवनेन ते}


\twolineshloka
{ययातिरब्रवीत्तं वै जरा मे प्रतिगृह्यताम्}
{यौवनेन त्वदीयेन चरेयं विषयानहम्}


\twolineshloka
{यजतो दीर्घसत्रैर्मे शापाच्चोशनसो मुनेः}
{कामार्थः परिहीणोऽयं तप्येयं तेन पुत्रकाः}


\threelineshloka
{मामकेन शरीरेण राज्यमेकः प्रशास्तु वः}
{अहं तन्वाऽभिनवया युवा काममवाप्नुयाम् ॥वैशंपायन उवाच}
{}


\twolineshloka
{ते न तस्य प्रत्यगृह्णन्यदुप्रभृतयो जराम्}
{तमब्रवीत्ततः पूरुः कनीयान्सत्यविक्रमः}


\twolineshloka
{राजंश्चराभिनवया तन्वा यौवनगोचरः}
{अहं जरां समादाय राज्ये स्थास्यामि तेज्ञया}


\twolineshloka
{एवमुक्तः स राजर्षिस्तपोवीर्यसमाश्रयात्}
{संचारयामास जरां तदा पुत्रे महात्मनि}


\twolineshloka
{पौरवेणाथ वयसा राजा यौवनमास्थितः}
{यायातेनापि वयसा राज्यं पूरुरकारयत्}


\twolineshloka
{ततो वर्षसहस्राणि ययातिरपराजितः}
{स्थितः स नृपशार्दूलः शार्दूलसमविक्रमः}


\twolineshloka
{ययातिरपि पत्नीभ्यां दीर्घकालं विहृत्य च}
{विश्वाच्या सहितो रेमे पुनश्चैत्ररथे वने}


\twolineshloka
{नाध्यगच्छत्तदा तृप्तिं कामानां स महायशाः}
{अवेत्य मनसा राजन्निमां गाथां तदा जगौ}


\twolineshloka
{न जातु कामः कामानामुपभोगेन शाम्यति}
{इविषा कृष्णवर्त्मेव भूय एवाभिवर्धते}


\twolineshloka
{यत्पृथिव्यां व्रीहियवं हिरण्यं पशवः स्त्रियः}
{नालमेकस्य तत्सर्वमिति मत्वा शमं व्रजेत्}


\twolineshloka
{यदा न कुरुते पापं सर्वभूतेषु कर्हिचित्}
{कर्मणा मनसा वाचा ब्रह्म संपद्यते तदा}


\twolineshloka
{न बिभेति यदा चायं यदा चास्मान्न बिभ्यति}
{यदा नेच्छति न द्वेष्टि ब्रह्म संपद्यते तदा}


\twolineshloka
{इत्यवेक्ष्य महाप्राज्ञः कामानां फल्गुतां नृप}
{समाधाय मनो बुद्ध्या प्रत्यगृह्णाज्जरां सुतात्}


\threelineshloka
{`ततो वर्षसहस्रान्ते ययातिरपराजितः}
{'दत्वा च यौवनं राजा पूरुं राज्येऽभिषिच्य च}
{अतृप्त एव कामानां पूरुं पुत्रमुवाच ह}


\threelineshloka
{त्वया दायादवानस्मि त्वं मे वंशकरः सुतः}
{पौरवो वंश इति ते ख्यातिं लोके गमिष्यति ॥वैशंपायन उवाच}
{}


\twolineshloka
{ततः स नृपशार्दूल पूरुं राज्येऽभिषिच्य च}
{ततः सुचरितं कृत्वा भृगुतुङ्गे महातपाः}


\twolineshloka
{कालेन महता पश्चात्कालधर्ममुपेयिवान्}
{कारयित्वा त्वनशनं सदारः स्वर्गमाप्तवान्}


\chapter{अध्यायः ७०}
\twolineshloka
{जनमेजय उवाच}
{}


\twolineshloka
{ययातिः पूर्वजोऽस्माकं दशमो यः प्रजापतेः}
{कथं स शुक्रतनयां लेभे परमदुर्लभाम्}


\threelineshloka
{एतदिच्छाम्यहं श्रोतुं विस्तरेण तपोधन}
{आनुपूर्व्या च मे शंस राज्ञो वंशकरान्पृथक् ॥वैशंपायन उवाच}
{}


\twolineshloka
{ययातिरासीन्नृपतिर्देवराजसमद्युतिः}
{तं शुक्रवृषपर्वाणौ वव्राते वै यथा पुरा}


\twolineshloka
{तत्तेऽहं संप्रवक्ष्यामि पृच्छते जनमेजय}
{देवयान्याश्च संयोगं ययातेर्नाहुषस्य च}


\twolineshloka
{सुराणामसुराणां च समजायत वै मिथः}
{ऐश्वर्यं प्रति संघर्षस्त्रैलोक्ये सचराचरे}


\twolineshloka
{जिगीषया ततो देवा वव्रिरेऽङ्गिरसं मुनिम्}
{पौरोहित्येन याज्यार्थे काव्यं तूशनसं परे}


\twolineshloka
{ब्राह्मणौ तावुभौ नित्यमन्योन्यस्पर्धिनौ भृशम्}
{तत्र देवा निजघ्नुर्यान्दानवान्युधि संगतान्}


\twolineshloka
{तान्पुनर्जीवयामास काव्यो विद्याबलाश्रयात्}
{ततस्ते पुनरुत्थाय योधयांचक्रिरे सुरान्}


\twolineshloka
{असुरास्तु निजघ्नुर्यान्सुरान्समरमूर्धनि}
{न तान्सञ्जीवयामास बृहस्पतिरुदारधीः}


\twolineshloka
{न हि वेद स तां विद्यां यां काव्योवेत्ति वीर्यवान्}
{सञ्जीविनीं ततो देवा विषादमगमन्परम्}


\twolineshloka
{ते तु देवा भयोद्विग्नाः काव्यादुशनसस्तदा}
{ऊचुः कचमुपागम्य ज्येष्ठं पुत्रं बृहस्पतेः}


\twolineshloka
{भजमानान्भजस्वास्मान्कुरु नः साह्यमुत्तमम्}
{या सा विद्या निवसति ब्राह्मणेऽमिततेजसि}


\twolineshloka
{शुक्रे तामाहर क्षिप्रं भागभाङ्गो भविष्यसि}
{वृषपर्वसमीपे हि शक्यो द्रष्टुं त्वया द्विजः}


\twolineshloka
{रक्षते दानवांस्तत्र न स रक्षत्यदानवान्}
{तमाराधयितुं शक्तो भवान्पूर्ववयाः कविम्}


\twolineshloka
{देवयानीं च दयितां सुतां तस्य महात्मनः}
{त्वमाराधयितुं शक्तो नान्यः कश्चन विद्यते}


\threelineshloka
{शीलदाक्षिण्यमाधुर्यैराचारेण दमेन च}
{देवयान्यां हि तुष्टायां विद्यां तां प्राप्स्यसि ध्रुवम् ॥वैशंपायन उवाच}
{}


\twolineshloka
{तथेत्युक्त्वा ततः प्रायाद्बृहस्पतिसुतः कचः}
{तदाऽभिपूजितो देवैः समीपे वृषपर्वणः}


\twolineshloka
{स गत्वा त्वरितो राजन्देवैः संप्रेषितः कचः}
{असुरेन्द्रपुरे शुक्रं दृष्ट्वा वाक्यमुवाच ह}


\twolineshloka
{ऋषेरङ्गिरसः पौत्रं पुत्रं साक्षाद्बृहस्पतेः}
{नाम्ना कच इति ख्यातं शिष्यं गृह्णात् मां भवान्}


\threelineshloka
{ब्रह्मचर्यं चरिष्यामि त्वय्यहं परमं गुरौ}
{अनुमन्यस्व मां ब्रह्मन्सहस्रं परिवत्सरान् ॥शुक्र उवाच}
{}


\threelineshloka
{कच सुस्वागतं तेऽस्तु प्रतिगृह्णामि ते वचः}
{अर्चयिष्येऽहमर्च्यं त्वामर्चितोऽस्तु बृहस्पतिः ॥वैशंपायन उवाच}
{}


\twolineshloka
{कचस्तु तं तथेत्युक्त्वा प्रतिजग्राह तद्व्रतम्}
{आदिष्टं कविपुत्रेण शुक्रेणोशनसा स्वयम्}


\twolineshloka
{व्रतस्य प्राप्तकालं स यथोक्तं प्रत्यगृह्णत}
{आराधयन्नुपाध्यायं देवयानीं च भारत}


\twolineshloka
{नित्यमाराधयिष्यंस्तां युवा यौवनगां मुनिः}
{गायन्नृत्यन्वादयंश्च देवयानीमतोषयत्}


\twolineshloka
{स शीलयन्देवयानीं कन्यां संप्राप्तयौवनाम्}
{पुष्पैः फलैः प्रेषणैश्च तोषयामास भारत}


\twolineshloka
{देवयान्यपि तं विप्रं नियमव्रतधारिणम्}
{गायन्ती च ललन्ती च रहः पर्यचरत्तथा}


\twolineshloka
{`गायन्तं चैव शुल्कं च दातारं प्रियवादिनम्}
{नार्यो नरं कामयन्ते रूपिणं स्रग्विणं तथा ॥'}


\twolineshloka
{पञ्चवर्षशतान्येवं कचस्य चरतो व्रतम्}
{तत्रातीयुरथो बुद्ध्वा दानवास्तं ततः कचम्}


\twolineshloka
{गा रक्षन्तं वने दृष्ट्वा रहस्येकममर्षिताः}
{जघ्नुर्बृहस्पतेर्द्वेषाद्विद्यारक्षार्थमेव च}


\twolineshloka
{हत्वा शालावृकेभ्यश्च प्रायच्छँल्लवशः कृतम्}
{ततो गावो निवृत्तास्ता अगोपाः स्वं निवेशनम्}


\threelineshloka
{सा दृष्ट्वा रहिता गाश्च कचेनाभ्यागता वनात्}
{उवाच वचनं काले देवयान्यथ भारत ॥देवयान्युवाच}
{}


\twolineshloka
{आहुतं चाग्निहोत्रं ते सूर्यश्चास्तं गतः प्रभो}
{अगोपाश्चागता गावः कचस्तात न दृश्यते}


\twolineshloka
{व्यक्तं हतो मृतो वापि कचस्तात भविष्यति}
{तं विना न च जीवेयमिति सत्यं ब्रवीमि ते}


\fourlineindentedshloka
{शुक्र उवाच}
{अयमेहीति संशब्द्य मृतं संजीवयाम्यहम्}
{वैशंपायन उवाच}
{ततः संजीविनीं विद्यां प्रयुज्य कचमाह्वयत्}


\twolineshloka
{भित्त्वा भित्त्वा शरीराणि वृकाणां स विनिर्गतः}
{आहूतः प्रादुरभवत्कचो हृष्टोऽथ विद्यया}


\threelineshloka
{कस्माच्चिरायितोऽसीति पृष्टस्तामाह भार्गवीम्}
{कच उवाच}
{समिधश्च कुशादीनि काष्ठभारं च भामिनि}


\twolineshloka
{गृहीत्वा श्रमभारार्तो वटवृक्षं समाश्रितः}
{गावश्च सहिताः सर्वा वृक्षच्छायामुपाश्रिताः}


\twolineshloka
{असुरास्तत्र मां दृष्ट्वा कस्त्वमित्यभ्यचोदयन्}
{बृहस्पतिसुतश्चाहं कच इत्यभिविश्रुतः}


\twolineshloka
{इत्युक्तमात्रे मां हत्वा पेषीकृत्वा तु दानवाः}
{दत्त्वा शालावृकेभ्यस्तु सुखं जग्मुः स्वमालयं}


\twolineshloka
{आहूतो विद्यया भद्रे भार्गवेण महात्मना}
{त्वत्समीपमिहायातः कथंचित्प्राप्तजीवितः}


\twolineshloka
{हतोऽहमिति चाचख्यौ पृष्टो ब्राह्मणकन्यया}
{स पुनर्देवयान्योक्तः पुष्पाण्याहर मे द्विज}


\twolineshloka
{वनं ययौ कचो विप्रो ददृशुर्दानवाश्च तम्}
{पुनस्तं पेषयित्वा तु समुद्राम्भस्यमिश्रयन्}


\threelineshloka
{चिरं गतं पुनः कन्या पित्रे तं संन्यवेदयत्}
{विप्रेण पुनराहूतो विद्यया गुरुदेहजः}
{पुनरावृत्य तद्वृत्तं न्यवेदयत तद्यथा}


\twolineshloka
{ततस्तृतीयं हत्वा तं दग्ध्वा कृत्वा च चूर्णशः}
{प्रायच्छन्ब्राह्मणायैव सुरायामसुरास्तथा}


\twolineshloka
{`अपिबत्सुरया सार्धं कचभस्म भृगूद्वहः}
{सा सायन्तनवेलायामगोपा गाः समागताः}


\twolineshloka
{देवयानी शङ्कमाना दृष्ट्वा पितरमब्रवीत्}
{'पुष्पाहारः प्रेषणकृत्कचस्तात न दृश्यते}


\threelineshloka
{व्यक्तं हतो मृतो वापि कचस्तात भविष्यति}
{तं विना न च जीवेयं कचं सत्यं ब्रवीमि ते ॥`वैशंपायन उवाच}
{}


\threelineshloka
{श्रुत्वा पुत्रीवचः काव्यो मन्त्रेणाहूतवान्कचम्}
{ज्ञात्वा बहिष्ठमज्ञात्वा स्वकुक्षिस्थं कचं नृप' ॥शुक्र उवाच}
{}


\twolineshloka
{बृहस्पतेः सुतः पुत्रि कचः प्रेतगतिं गतः}
{विद्यया जीवितोऽप्येवं हन्यते करवाम किम्}


\twolineshloka
{मैवं शुचो मा रुद देवयानिन त्वादृशी मर्त्यमनुप्रशोचते}
{यस्यास्तव ब्रह्म च ब्राह्मणाश्चसेन्द्रा देवा वसवोऽथाश्विनौ च}


\threelineshloka
{सुरद्विषश्चैव जगच्च सर्व-मुपस्थाने सन्नमन्ति प्रभावात्}
{अशक्योऽसौ जीवयितुं द्विजातिःसंजीवितो वध्यते चैव भूयः ॥देवयान्युवाच}
{}


\twolineshloka
{यस्याङ्गिरा वृद्धतमः पितामहोबृहस्पतिश्चापि पिता तपोनिधिः}
{ऋषेः पुत्रं तमथो वापि पौत्रंकथं न शोचेयमहं न रुद्याम्}


\threelineshloka
{स ब्रह्मचारी च तपोधनश्चसदोत्थितः कर्मसु चैव दक्षः}
{कचस्य मार्गं प्रतिपत्स्ये न भोक्ष्येप्रियो हि मे तात कचोऽभिरूपः ॥शुक्र उवाच}
{}


\fourlineindentedshloka
{असंशयं मामसुरा द्विषन्तिये शिष्यं मेऽनागसं सूदयन्ति}
{अब्राह्मणं कर्तुमिच्छ्ति रौद्रा-स्ते मां यथा व्यभिचरन्ति नित्यम्}
{अप्यस्य पापस्य भवेदिहान्तःकं ब्रह्महत्या न दहेदपीन्द्रम् ॥`वैशंपायन उवाच}
{}


\twolineshloka
{संचोदितो देवयान्या महर्षिः पुनराह्वयत्}
{संरम्भेणैव काव्यो हि बृहस्पतिसुतं कचम्}


\threelineshloka
{कचोऽपि राजन्स महानुभावोविद्याबलाल्लब्धमतिर्महात्मा}
{'गुरोर्हि भीतो विद्यया चोपहूतः}
{शनैर्वाक्यं जठरे व्याजहार}


\threelineshloka
{`प्रसीद भगवन्मह्यं कचोऽहमभिवादये}
{यथा बहुमतः पुत्रस्तथा मन्यतु मां भवान् ॥'वैशंपायन उवाच}
{}


\threelineshloka
{तमब्रवीत्केन पथोपनीत-स्त्वं चोदरे तिष्ठसि ब्रूहि विप्र}
{अस्मिन्मुहूर्ते ह्यसुरान्विनाश्यगच्छामि देवानहमद्य विप्र ॥कच उवाच}
{}


\twolineshloka
{तव प्रसादान्न जहाति मां स्मृतिःस्मरामि सर्वं यच्च यथा च वृत्तम्}
{नत्वेवं स्यात्तपसः संक्षयो मेततः क्लेशं घोरमिमं सहामि}


\threelineshloka
{असुरैः सुरायां भवतोऽस्मि दत्तोहत्वा दग्ध्वा चूर्णयित्वा च काव्य}
{ब्राह्मीं मायां चासुरीं विप्र मायांत्वयि स्थिते कथमेवातिवर्तेत् ॥शुक्र उवाच}
{}


\threelineshloka
{किं ते प्रियं करवाण्यद्य वत्सेवधेन मे जीवितं स्यात्कचस्य}
{नान्यत्र कुक्षेर्मम भेदनेनदृश्येत्कचो मद्गतो देवयानि ॥देवयान्युवाच}
{}


\threelineshloka
{द्वौ मां शोकावग्निकल्पौ दहेतांकचस्य नाशस्तव चैवोपघातः}
{कचस्य नाशे मम शर्म नास्तितवोपघाते जीवितुं नास्मि शक्ता ॥शुक्र उवाच}
{}


\twolineshloka
{संसिद्धरूपोऽसि बृहस्पतेः सुतयत्त्वां भक्तं भजते देवयानी}
{विद्यामिमां प्राप्नुहि जीवनीं त्वंन चेदिन्द्रः कचरूपी त्वमद्य}


\twolineshloka
{न निवर्तेत्पुनर्जीवन्कश्चिदन्यो ममोदरात्}
{ब्राह्मणं वर्जयित्वैकं तस्माद्विद्यामवाप्नुहि}


\threelineshloka
{पुत्रो भूत्वा भावय भावितो मा-मस्मद्देहादुपनिष्क्रम्य तात}
{समीक्षेथा धर्मवतीमवेक्षांगुरोः सकाशात्प्राप्य विद्यां सविद्यः ॥वैशंपायन उवाच}
{}


\twolineshloka
{गुरोः सकाशात्समवाप्य विद्यांभित्त्वा कुक्षिं निर्विचक्राम विप्रः}
{कचोऽभिरूपस्तत्क्षणाद्ब्राह्मणस्यशुक्लात्यये पौर्णमास्यामिवेन्दुः}


\twolineshloka
{दृष्ट्वा च तं पतितं ब्रह्मराशि-मुत्थापयामास मृतं कचोऽपि}
{विद्यां सिद्धां तामवाप्याभिवाद्यततः कचस्तं गुरुमित्युवाच}


\twolineshloka
{यः श्रोत्रयोरमृतं सन्निषिञ्चे-द्विद्यामविद्यस्य यथा त्वमार्यः}
{तं मन्येऽहं पितरं मातरं चतस्मै न द्रुह्येत्कृतमस्य जानन्}


\threelineshloka
{ऋतस्य दातारमनुत्तमस्यनिधिं निधीनामपि लब्धविद्याः}
{ये नाद्रियन्ते गुरुमर्चनीयंपापाँल्लोकांस्ते व्रजन्त्यप्रतिष्ठाः ॥वैशंपायन उवाच}
{}


\twolineshloka
{सुरापानाद्वञ्चनां प्राप्य विद्वा-न्संज्ञानाशं चैव महातिघोरम्}
{दृष्ट्वा कचं चापि तथाभिरूपंपीतं तदा सुरया मोहितेन}


\twolineshloka
{समन्युरुत्थाय महानुभाव-स्तदोशना विप्रहितं चिकीर्षुः}
{सुरापानं प्रति संजातमन्युःकाव्यः स्वयं वाक्यमिदं जगाद}


\twolineshloka
{यो ब्राह्मणोऽद्यप्रभृतीह कश्चि-न्मोहात्सुरां पास्यति मन्दबुद्धिः}
{अपेतधर्मा ब्र्हमहा चैव स स्या-दस्मिंल्लोके गर्हितः स्यात्परे च}


\threelineshloka
{मया चैतां विप्रधर्मोक्तिसीमांमर्यादां वै स्थापितां सर्वलोके}
{सन्तो विप्राः शुश्रुवांसो गुरूणांदेवा लोकाश्चोपशृण्वन्तु सर्वे ॥वैशंपायन उवाच}
{}


\twolineshloka
{इतीदमुक्त्वा स महानुभाव-स्तपोनिधीनां निधिरप्रमेयः}
{तान्दानवान्दैवविमूढबुद्धी-निदं समाहूय वचोऽभ्युवाच}


\twolineshloka
{आचक्षे वो दानवा बालिशाः स्थसिद्धः कचो वत्स्यति मत्सकाशे}
{सञ्जीविनीं प्राप्य विद्यां महात्मातुल्यप्रभावो ब्राह्मणो ब्रह्मभूतः}


\threelineshloka
{`योऽकार्षीद्दुष्करं कर्म देवानां कारणात्कचः}
{न तत्किर्तिर्जरां गच्छेद्याज्ञीयश्च भविष्यति ॥वैशंपायन उवाच}
{'}


\twolineshloka
{एतावदुक्त्वा वचनं विरराम स भार्गवः}
{दानवा विस्मयाविष्टाः प्रययुः स्वं निवेशनम्}


\twolineshloka
{गुरोरुष्य सकाशे तु दश वर्षशतानि सः}
{अनुज्ञातः कचो गन्तुमियेष त्रिदशालयम्}


\chapter{अध्यायः ७१}
\twolineshloka
{वैशंपायन उवाच}
{}


\twolineshloka
{समावृतव्रतं तं तु विसृष्टं गुरुणा कचम्}
{प्रस्थितं त्रिदशावासं देवयान्यब्रवीदिदम्}


\twolineshloka
{ऋषेरङ्गिरसः पौत्र वृत्तेनाभिजनेन च}
{भ्राजसे विद्यया चैव तपसा च दमेन च}


\twolineshloka
{ऋषिर्यथाङ्गिरा मान्यः पितुर्मम महायशाः}
{तथा प्रान्यश्च पूज्यश्च मम भूयो बृहस्पतिः}


\twolineshloka
{एवं ज्ञात्वा विजानीहि यद्ब्रवीमि तपोधन}
{व्रतस्थे नियमोपेते यथा वर्ताम्यहं त्वयि}


\threelineshloka
{स समावृतविद्यो मां भक्तां भजितुमर्हसि}
{गृहाण पाणिं विधिवन्मम मन्त्रपुरस्कृतम् ॥कच उवाच}
{}


\twolineshloka
{पूज्यो मान्यश्च भगवान्यथा तव पिता मम}
{तथा त्वमनवद्याङ्गि पूजनीयतरा मम}


\twolineshloka
{प्राणेभ्योऽपि प्रियतरा भार्गवस्य महात्मनः}
{त्वं भत्रे धर्मतः पूज्या गुरुपुत्री सदा मम}


\threelineshloka
{यथा मम गुरुर्नित्यं मान्यः शुक्रः पिता तव}
{देवयानि तथैव त्वं नैवं मां वक्तुमर्हसि ॥देवयान्युवाच}
{}


\twolineshloka
{गुरुपुत्रस्य पुत्रो वै न त्वं पुत्रश्च मे पितुः}
{तस्मात्पूज्यश्च मान्यश्च ममापि त्वं द्विजोत्तम}


\twolineshloka
{असुरैर्हन्यमाने च कच त्वयि पुनःपुनः}
{तदाप्रभृति या प्रीतिस्तां त्वमद्य स्मरस्व मे}


\threelineshloka
{सौहार्दे चानुरागे च वेत्थ मे भक्तिमुत्तमाम्}
{न मामर्हसि धर्मज्ञ त्यक्तुं भक्तामनागसम् ॥कच उवाच}
{}


\twolineshloka
{अनियोज्ये नियोक्तुं मां देवयानि न चार्हसि}
{प्रसीद सुभ्रु त्वं मह्यं गुरोर्गुरुतरा शुभे}


\twolineshloka
{यत्रोषितं विशालाक्षि त्वया चन्द्रनिभानने}
{तत्राहमुषितो भद्रे कुक्षौ काव्यस्य भामिनि}


\twolineshloka
{भगिनी धर्मतो मे त्वं मैवं वोचः सुमध्यमे}
{सुखमस्म्युषितो भद्रे न मन्युर्विद्यते मम}


\fourlineindentedshloka
{आपृच्छे त्वां गमिष्यामि शिवमाशंस मे पथि}
{अविरोधेन धर्मस्य स्मर्तव्योऽस्मि कथान्तरे}
{अप्रमत्तोत्थिता नित्यमाराधय गुरुं मम ॥देवयान्युवाच}
{}


\threelineshloka
{यदि मां धर्मकामार्थे प्रत्याख्यास्यसि याचितः}
{ततः कच न ते विद्या सिद्धिमेषा गमिष्यति ॥कच उवाच}
{}


\twolineshloka
{गुरुपुत्रीति कृत्वाऽहं प्रत्याचक्षे न दोषतः}
{गुरुणा चाननुज्ञातः काममेवं शपस्व माम्}


\twolineshloka
{आर्षं धर्मं ब्रुवाणोऽहं देवयानि यथा त्वया}
{शप्तो ह्यनर्हः शापस्य कामतोऽद्य न धर्मतः}


\twolineshloka
{तस्माद्भवत्या यः कामो न तथा स भविष्यति}
{ऋषिपुत्रो न ते कश्चिज्जातु पाणिं ग्रहीष्यति}


\threelineshloka
{फलिष्यति न ते विद्या यत्त्वं मामात्थ तत्तथा}
{अध्यापयिष्यामि तु यं तस्य विद्या फलिष्यति ॥वैशंपायन उवाच}
{}


\twolineshloka
{एवमुक्त्वा द्विजश्रेष्ठो देवयानीं कचस्तदा}
{त्रिदशेशालयं शीघ्रं जगाम द्विजसत्तमः}


\threelineshloka
{तमागतमभिप्रेक्ष्य देवा इन्द्रपुरोगमाः}
{बृहस्पतिं सभाज्येदं कचं वचनमब्रुवन् ॥देवा ऊचुः}
{}


\twolineshloka
{यत्त्वयास्मद्धितं कर्म कृतं वै परमाद्भुतम्}
{न ते यशः प्रणशिता भागभाक्व भविष्यसि}


\chapter{अध्यायः ७२}
\twolineshloka
{वैशंपायन उवाच}
{}


\twolineshloka
{कृतविद्ये कचे प्राप्ते हृष्टरूपा दिवौकसः}
{कचादधीत्य तां विद्यां कृतार्था भरतर्षभ}


\threelineshloka
{सर्व एव समागम्य शतक्रतुमथाब्रुवन्}
{कालस्ते विक्रमस्याद्य जहि शत्रून्पुरन्दर ॥वैशंपायन उवाच}
{}


\twolineshloka
{एवमुक्तस्तु सहितैस्त्रिदशैर्मघवांस्तदा}
{तथेत्युक्त्वा प्रचक्राम सोऽपश्यत वने स्त्रियः}


\twolineshloka
{क्रीडन्तीनां तु कन्यानां वने चैत्ररथोपमे}
{वायुभूतः स वस्त्राणि सर्वाण्येव व्यमिश्रयत्}


\twolineshloka
{ततो जलात्समुत्तीर्य कन्यास्ताः सहितास्तदा}
{वस्त्राणि जगृहुस्तानि यथाऽऽसन्नान्यनेकशः}


\twolineshloka
{तत्र वासो देवयानयाः शर्मिष्ठा जगृहे तदा}
{व्यतिमिश्रमजानन्ती दुहिता वृषपर्वणः}


\threelineshloka
{ततस्तयोर्मिथस्तत्र विरोधः समजायत}
{देवयान्याश्च राजेन्द्र शर्मिष्ठायाश्च तत्कृते ॥देवयान्युवाच}
{}


\threelineshloka
{कस्माद्गृह्णासि मे वस्त्रं शिष्या भूत्वा ममासुरि}
{समुदाचारहीनाया न ते साधु भविष्यति ॥सर्मिष्ठोवाच}
{}


\twolineshloka
{आसीनं च शयानं च पिता ते पितरं मम}
{स्तौतिबन्दीव चाभीक्ष्णं नीचैः स्थित्वा विनीतवत्}


\twolineshloka
{याचतस्त्वं हि दुहिता स्तुवतः प्रतिगृह्णतः}
{सुताहं स्तूयमानस्य ददतोऽप्रतिगृह्णतः}


\twolineshloka
{आदुन्वस्व विदुन्वस्व द्रुह्य कुप्यस्व याचकि}
{अनायुधा सायुधाया रिक्ता क्षुभ्यसि भिक्षुकि}


\threelineshloka
{लप्स्यसे प्रतियोद्धारं न हि त्वां गणयाम्यहम्}
{`प्रतिकूलं वदसि चेदितःप्रभृति याचकि ॥'वैशंपायन उवाच}
{}


\threelineshloka
{समुच्छ्रयं देवयानीं गतां सक्तां च वाससि}
{शर्मिष्ठा प्राक्षिपत्कूपे ततः स्वपुरमागमत्}
{हतेयमिति विज्ञाय शर्मिष्ठा पापनिश्चया}


\threelineshloka
{अनवेक्ष्य ययौ वेश्म क्रोधवेगपरायणा}
{`प्रविश्य स्वगृहं स्वस्था धर्ममासुरमास्थिता}
{'अथ तं देशमभ्यागाद्ययातिर्नहुषात्मजः}


\twolineshloka
{श्रान्तयुग्यः श्रान्तहयो मृगलिप्सुः पिपासितः}
{स नाहुषः प्रेक्षमाण उदपानं गतोदकम्}


\twolineshloka
{ददर्श राजा तां तत्र कन्यामग्निशिखामिव}
{तामपृच्छत्स दृष्ट्वैव कन्याममरवर्णिनीम्}


\twolineshloka
{सान्त्वयित्वा नृपश्रेष्ठः साम्ना परमवल्गुना}
{का त्वं ताम्रनखी श्यामा सुमृष्टमणिकुण्डला}


\twolineshloka
{दीर्घं ध्यायसि चात्यर्थं कस्माच्छ्वसिषि चातुरा}
{कथं च पतिताऽस्यस्मिन्कूपे वीरुत्तृणावृते}


\threelineshloka
{दुहिता चैव कस्य त्वं वद सत्यं सुमध्यमे}
{देवयान्युवाच}
{योऽसौ देवैर्हतान्दैत्यानुत्थापयति विद्यया}


\twolineshloka
{तस्य शुक्रस्य कन्याहं स मां नूनं न बुध्यते}
{एष मे दक्षिणो राजन्पाणिस्ताम्रनखाङ्गुलिः}


\twolineshloka
{समुद्धर गृहीत्वा मां कुलीनस्त्वं हि मे मतः}
{जानामि हि त्वां संशान्तं वीर्यवन्तं यशस्विनम्}


\threelineshloka
{तस्मान्मां पतितामस्मात्कूपादुद्धर्तुमर्हसि}
{वैशंपायन उवाच}
{तामथो ब्राह्मणीं कन्यां विज्ञायनहुषात्मजः}


\threelineshloka
{गृहीत्वा दक्षिणे पाणावुज्जहार ततोऽवटात्}
{उद्धृत्य चैनां तरसा तस्मात्कूपान्नराधिपः ॥`ययातिरुवाच}
{}


\twolineshloka
{गच्छ भद्रे यथाकामं न भयं विद्यते तव}
{इत्युच्यमाना नृपतिं देवयानीदमुत्तरम्}


\twolineshloka
{उवाच मामुपादाय गच्छ शीघ्रं प्रियो हि मे}
{गृहीताहं त्वया पाणौ तस्माद्भर्ता भविष्यसि}


\twolineshloka
{इत्येवमुक्तो नृपतिराह क्षत्रकुलोद्भवः}
{त्वं भद्रे ब्राह्मणी तस्मान्मया नार्हसि सङ्गमम्}


\threelineshloka
{सर्वलोकगुरुः काव्यस्त्वं तस्य दुहिता शुभे}
{तस्मादपि भयं मेऽद्य ततः कल्याणि नार्हसि ॥देवयान्युवाच}
{}


\threelineshloka
{यदि मद्वचनान्नाद्य मां नेच्छसि नराधिप}
{त्वामेव वरये पित्रा तस्माल्लप्स्यसि गच्छ हि ॥वैशंपायन उवाच}
{}


\twolineshloka
{आमन्त्रयित्वा सुश्रोणीं ययातिः स्वपुरं ययौ}
{गते तु नाहुषे तस्मिन्देवयान्यप्यनिन्दिता}


\twolineshloka
{क्वचिद्गत्वा च रुदती वृक्षमाश्रित्य धिष्ठिता}
{ततश्चिरायमाणायां दुहितर्यथ भार्गवः}


\twolineshloka
{संस्मृत्योवाच धात्रीं तां दुहितुः स्नेहविक्लवः}
{धात्रि त्वमानय क्षिप्रं देवयानीं समुध्यमाम्}


\twolineshloka
{इत्युक्तमात्रे सा धात्री त्वरिताऽऽनयितुं गता}
{यत्रयत्र सशीभिः सा गता पदममार्गत}


\twolineshloka
{सा ददर्श तथा दीनां श्रमार्तां रुदतीं स्थिताम्}
{वृत्तान्तं किमिदं भद्रे शीघ्रं वद पिताह्वयत्}


\threelineshloka
{एवमुक्ताह धात्रीं तां शर्मिष्ठावृजिनं कृतम्}
{उवाच शोकसंतप्ता घूर्णिकामागतां पुरः' ॥देवयान्युवाच}
{}


% Check verse!
त्वरितं घूर्णिके गच्छ शीघ्रमाचक्ष्व मे पितुः
\threelineshloka
{नेदानीं संप्रवेक्ष्यामि नगरं वृषपर्वणः}
{वैशंपायन उवाच}
{सा तत्र त्वरितं गत्वा घूर्णिकाऽसुरमन्दिरम्}


\twolineshloka
{दृष्ट्वा काव्यमुवाचेदं संभ्रमाविष्टचेतना}
{आचचक्षे महाप्राज्ञं देवयानीं वने हताम्}


\twolineshloka
{शर्मिष्ठया महाभाग दुहित्रा वृषपर्वणः}
{श्रुत्वा दुहितरं काव्यस्तत्र शर्मिष्ठया हताम्}


\twolineshloka
{त्वरया निर्ययौ दुःखान्मार्गमाणः सुतां वने}
{दृष्ट्वा दुहितरं काव्यो देवयानीं ततो वने}


\twolineshloka
{बाहुभ्यां संपरिष्वज्य दुःखितो वाक्यमब्रवीत्}
{आत्मदोषैर्नियच्छन्ति सर्वे दुःखसुखे जनाः}


\threelineshloka
{मन्ये दुश्चरितं तेऽस्ति यस्येयं निष्कृतिः कृता}
{देवयान्युवाच}
{निष्कृतिर्मेऽस्तु वा मास्तु शृणुष्वावहितो मम}


% Check verse!
शर्मिष्ठया यदुक्ताऽस्मि दुहित्रा वृषपर्वणः ॥सत्यं किलैतत्सा प्राह दैत्यानामसि गायनः
\twolineshloka
{एवं हि मे कथयति शर्मिष्ठा वार्षपर्वणी}
{वचनं तीक्ष्णपरुषं क्रोधरक्तेक्षणा भृशम्}


\twolineshloka
{स्तुवतो दुहिता नित्यं याचतः प्रतिगृह्णतः}
{अहं तु स्तूयमानस्य ददतोऽप्रतिगृह्णतः}


\twolineshloka
{इदं मामाह शर्मिष्ठा दुहिता वृषपर्वणः}
{क्रोधसंरक्तनयना दर्पपूर्णा पुनः पुनः}


\twolineshloka
{यद्यह स्तुवतस्तात दुहिता प्रतिगृह्णतः}
{प्रसादयिष्ये शर्मिष्ठामित्युक्ता तु सखी मया}


\threelineshloka
{`उक्ताप्येवं भृशं मां सा निगृह्य विजने वने}
{कूपे प्रक्षेपयामास प्रक्षिप्य गृहमागमत् ॥'शुक्र उवाच}
{}


\twolineshloka
{स्तुवतो दुहिता न त्वं याचतः प्रतिगृह्णतः}
{अस्तोतुः स्तूयमानस्य दुहिता देवयान्यसि}


\twolineshloka
{वृषपर्वैव तद्वेद शक्रो राजा च नाहुषः}
{अचिन्त्यं ब्रह्म निर्द्वन्द्वमैश्वरं हि बलं मम}


\twolineshloka
{`जानामि जीविनीं विद्यां लोकेस्मिञ्शाश्वतीं ध्रुवम्}
{मृतः संजीवते जन्तुर्यया कमललोचने}


\twolineshloka
{कत्थनं स्वगुणानां च कृत्वा तप्यति सज्जनः}
{ततो वक्तुमशक्तोऽस्मित्वं मे जानासि यद्बलम्}


\twolineshloka
{तसमादुत्तिष्ठ गच्छामः स्वगृहं कुलनन्दिनि}
{क्षमां कृत्वा विशालाक्षि क्षमासारा हि साधवः'}


\twolineshloka
{यच्च किंचित्सर्वगतं भूमौ वा यदि वा दिवि}
{तस्याहमीश्वरो नित्यं तुष्टेनोक्तः स्वयंभुवा}


\threelineshloka
{अहं जलं विमुञ्चामि प्रजानां हितकाम्यया}
{पुष्णाम्यौषधयः सर्वा इति सत्यं ब्रवीमि ते ॥वैशंपायन उवाच}
{}


\twolineshloka
{एवं विषादमापन्नां मन्युना संप्रपीडिताम्}
{वचनैर्मधुरैः श्लक्ष्णैः सान्त्वयामास तां पिता}


\chapter{अध्यायः ७३}
\twolineshloka
{शुक्र उवाच}
{}


\twolineshloka
{यः परेषां नरो नित्यमतिवादांस्तितिक्षते}
{देवयानि विजानीहि तेन सर्वमिदं जितम्}


\twolineshloka
{यः समुत्पतितं क्रोधं निगृह्णाति इयं यथा}
{स यन्तेत्युच्यते सद्भिर्न यो रश्मिषु लम्बते}


\twolineshloka
{यः समुत्पतितं क्रोधमक्रोधेन निरस्यति}
{देवयानि विजानीहि तेन सर्वमिदं जितम्}


\twolineshloka
{यः समुत्पतितं क्रोधं क्षमयेह निरस्यति}
{यथोरगस्त्वचं जीर्णां स वै पुरुष उच्यते}


\twolineshloka
{यः संधारयते मन्युं योऽतिवादांस्तितिक्षते}
{यश्च तप्तो न तपति दृढं सोऽर्थस्य भाजनम्}


\twolineshloka
{यो यजेदपरिश्रान्तो मासिमासि शतं समाः}
{न क्रुद्ध्येद्यश्च सर्वस्य तयोरक्रोधनोऽधिकः}


\twolineshloka
{`तस्मादक्रोधनः श्रेष्ठः कामक्रोधौ विगर्हितौ}
{क्रुद्धस्य निष्फलान्येव दानयज्ञतपांसि च}


\twolineshloka
{तस्मादक्रोधने यज्ञतपोदानफलं महत्}
{भवेदसंशयं भद्रे नेतरस्मिन्कदाचन}


\twolineshloka
{न यतिर्न तपस्वी च न यज्वा न च धर्मभाक्}
{क्रोधस्य यो वशं गच्छेत्तस्य लोकद्वयं न च}


\twolineshloka
{पुत्रो भृत्यः सुहृद्भ्राता भार्या धर्मश्च सत्यता}
{तस्यैतान्यपयास्यन्ति क्रोधशीलस्य निश्चितम्}


\threelineshloka
{यत्कुमाराः कुमार्यश्च वैरं कुर्युरचेतसः}
{न तत्प्राज्ञोऽनुकुर्वीत न विदुस्ते बलाबलम् ॥देवयान्युवाच}
{}


\twolineshloka
{वेदाहं तात बालाऽपि धर्माणां यदिहान्तरम्}
{अक्रोधे चातिवादे च वेद चापि बलाबलम्}


\twolineshloka
{`स्ववृत्तिमननुष्ठाय धर्ममुत्सृज्य तत्त्वतः}
{'शिष्यस्याशिष्यवृत्तेस्तु न क्षन्तव्यं बुभूषता}


\twolineshloka
{`प्रेष्यः शिष्यः स्ववृत्तिं हि विसृज्य विफलं गतः}
{'तस्मात्संकीर्णवृत्तेषु वासो मम न रोचते}


\twolineshloka
{पुमांसो ये हि निन्दन्ति वृत्तेनाभिजनेन च}
{न तेषु निवसेत्प्राज्ञः श्रेयोऽर्थी पापबुद्धिषु}


\twolineshloka
{ये त्वेनमभिजानन्ति वृत्तेनाभिजनेन वा}
{तेषु साधुषु वस्तव्यं स वासः श्रेष्ठ उच्यते}


\twolineshloka
{`सुयन्त्रितपरा नित्यं विहीनाश्च धनैर्वराः}
{दुर्वृत्ताः पापकर्माणश्चण्डाला धनिनोपि च}


\twolineshloka
{नैव जात्या हि चण्डालाः स्वकर्मविहितैर्विना}
{धनाभिजनविद्यासु सक्ताश्चण्डालधर्मिणः}


\twolineshloka
{अकारणाश्च द्वेष्यन्ति परिवादं वदन्ति ते}
{साधोस्तत्र न वासोस्ति पापिभिः पापतां व्रजेत्}


\twolineshloka
{सुकृते दुष्कृते वापि यत्र सज्जति यो नरः}
{ध्रुवं रतिर्भवेत्तस्य तस्माद्द्वेषं न रोचयेत् ॥'}


\twolineshloka
{वाग्दुरुक्तं महाघोरं दुहितुर्वृषपर्वणः}
{मम मथ्नाति हृदयमग्निकाम इवारणिम्}


\twolineshloka
{न ह्यतो दुष्करतरं मन्ये लोकेष्वपि त्रिषु}
{यः सपत्नश्रियं दीप्तां हीनश्रीः पर्युपासते}


\twolineshloka
{मरणं शोभनं तस्य इति विद्वज्जना विदुः}
{`अवमानमवाप्नोति शनैर्नीचसमागमात्}


\twolineshloka
{अतिवादा वक्त्रतो निःसरन्तियैराहतः शोचति रात्र्यहानि}
{परस्य वै मर्मसु ते पतन्तितस्माद्धीरो नैव मुच्येत्परेषु}


\twolineshloka
{निरोहेदायुधैश्छिन्नं संरोहेद्दग्धमाग्निना}
{वाक्क्षतं च न संरोहेदाशरीरं शरीरिणाम्}


\twolineshloka
{संरोहित शरैर्विद्धं नवं परशुना हतम्}
{वाचा दुरुक्तं बीभत्सं न संरोहेत वाक्क्षतम्'}


\chapter{अध्यायः ७४}
\twolineshloka
{वैशंपायन उवाच}
{}


\twolineshloka
{ततः काव्यो भृगुश्रेष्ठः समन्युरुपगम्य ह}
{वृषपर्वाणमासीनमित्युवाचाविचारयन्}


\twolineshloka
{नाधर्मश्चरितो राजन्सद्यः फलति गौरिव}
{शनैरावर्त्यमानो हि कर्तुर्मूलानि कृन्तति}


\twolineshloka
{पुत्रेषु वा नप्तृषु वा न चेदात्मनि पश्यति}
{फलत्येव ध्रुवं पायं गुरुभुक्तमिवोदरे}


\threelineshloka
{`अधीयानं हितं राजन्क्षमावन्तं जितेन्द्रियम्}
{'यदघातयथा विप्रं कचमाङ्गिरसं तदा}
{अपापशीलं धर्मज्ञं शुश्रूषुं मद्गृहे रतम्}


\twolineshloka
{`शर्मिष्ठया देवयानी क्रूरमुक्ता बहु प्रभो}
{विप्रकृत्य च संरम्भात्कूपे क्षिप्ता मनस्विनी}


\twolineshloka
{सा न कल्पेत वासाय तयाहं रहितः कथम्}
{वसेयमिह तस्मात्ते त्यजामि विषयं नृप ॥'}


\threelineshloka
{वधादनर्हतस्तस्य वधाच्च दुहितुर्मम}
{वषपर्वन्निबोधेयं त्यक्ष्यामि त्वां सबान्धवम्}
{स्थातुं त्वद्विषये राजन्न शक्ष्यामि त्वया सह}


\fourlineindentedshloka
{`मा शोच वृषपर्वंस्त्वं मा क्रुध्यस्व विशांपते}
{स्थातुं ते विषये राजन्न शक्ष्यामि तया विना}
{अस्या गतिर्गतिर्मह्यं प्रियमस्याः प्रियं मम ॥वृषपर्वोवाच}
{}


\threelineshloka
{यदि ब्रह्मन्घातयामि यदि वा क्रोशयाम्यहम्}
{शर्मिष्ठया देवयानीं तेन गच्छाम्यसद्गतिम् ॥शुक्र उवाच}
{'}


\threelineshloka
{अहो मामभिजानासि दैत्य मिथ्याप्रलापिनम्}
{यथेममात्मनो दोषं न नियच्छस्पुपेक्षसे ॥वृषपर्वोवाच}
{}


\twolineshloka
{नाधर्मं न मृषावादं त्वयि जानामि भार्गव}
{त्वयि धर्मश्च सत्यं च तत्प्रसीदतु नो भवान्}


\twolineshloka
{यद्यस्मानपहाय त्वमितो गच्छसि भार्गव}
{समुद्रं संप्रवेक्ष्यामि पूर्वं मद्बान्धवैः सह}


\fourlineindentedshloka
{पातालमथवा चाग्निं नान्यदस्ति परायणम्}
{यद्येव देवान्गच्छेस्त्वं मां च त्यक्त्वा ग्रहाधिप}
{सर्वत्यागं ततः कृत्वा प्रविशामि हुताशनम्' ॥शुक्र उवाच}
{}


\twolineshloka
{समुद्रं प्रविशध्वं वा दिशो वा द्रवतासुराः}
{दुहितुर्नाप्रियं सोहुं शक्तोऽहं दयिता हि मे}


\threelineshloka
{प्रसाद्यतां देवयानी जीवितं यत्र मे स्थितम्}
{योगक्षेमकरस्तेऽहमिन्द्रस्येव बृहस्पतिः ॥वृषपर्वोवाच}
{}


\threelineshloka
{यत्किंचिदसुरेन्द्राणां विद्यते वसु भार्गव}
{भुवि हस्तिगवाश्वं च तस्य त्वं मम चेश्चरः ॥शुक्र उवाच}
{}


\threelineshloka
{यत्किंचिदस्ति द्रविणं दैत्येन्द्राणां महासुर}
{तस्येश्वरोस्मि यद्येषा देवयानी प्रसाद्यताम् ॥वैशंपायन उवाच}
{}


\threelineshloka
{एवमुक्तस्तथेत्याह वृषपर्वा महाकविम्}
{देवयान्यन्तिकं गत्वा तमर्थं प्राह भार्गवः ॥देवयान्युवाच}
{}


\threelineshloka
{यदि त्वमीश्वरस्तात राज्ञो वित्तस्य भार्गव}
{नाभिजानामि तत्तेऽहं राजा तु वदतु स्वयम् ॥`वैशंपायन उवाच}
{}


\threelineshloka
{शुक्रस्य वचनं श्रुत्वा वृषपर्वा सबान्धवः}
{देवयानि प्रसीदेति पपात भुवि पादयोः ॥वृषपर्वोवाच}
{}


\fourlineindentedshloka
{स्तुत्यो वन्द्यश्च सततं मया तातश्च ते शुभे}
{'यं काममभिकामाऽसि देवयानि शुचिस्मिते}
{तत्तेऽहं संप्रदास्यामि यदि वापि हि दुर्लभम् ॥देवयान्युवाच}
{}


\threelineshloka
{दासीं कन्यासहस्रेण शर्मिष्ठामभिकामये}
{अनु मां तत्र गच्छेत्सा यत्र दद्याच्च मे पिता ॥वृषपर्वोवाच}
{}


\twolineshloka
{उत्तिष्ठ त्वं गच्छ धात्रि शर्मिष्ठां शीघ्रमानय}
{यं च कामयते कामं देवयानी करोतु तम्}


\threelineshloka
{`त्यजेदेकं कुलस्यार्थे ग्रामार्थे च कुलं त्यजेत्}
{ग्रामं जनपदस्यार्थे आत्मार्थे पृथिवीं त्यजेत्' ॥वैशंपायन उवाच}
{}


\twolineshloka
{ततो धात्री तत्र गत्वा शर्मिष्ठां वाक्यमब्रवीत्}
{उत्तिष्ठ भद्रे शर्मिष्ठे ज्ञातीनां सुखमावह}


\threelineshloka
{त्यजति ब्राह्मणः शिष्यान्देवयान्या प्रचोदितः}
{सायं कामयते कां स कार्योऽद्य त्वयाऽनघे ॥शर्मिष्ठोवाच}
{}


\fourlineindentedshloka
{यं सा कामयते कां करवाण्यहमद्य तम्}
{यद्येवमाह्वयेच्छुक्रो देवयानीकृते हि माम्}
{मद्दोषान्मागमच्छुक्रो देवयानी च मत्कृते ॥वैशंपायन उवाच}
{}


\threelineshloka
{ततः कन्यासहस्रेण वृता शिबिकया तदा}
{पितुर्नियोगात्त्वरिता निश्चक्राम पुरोत्तमात् ॥शर्मिष्ठोवाच}
{}


\threelineshloka
{अहं दासीसहस्रेण दासी ते परिचारिका}
{अनु त्वां तत्र यास्यामि यत्र दास्यति ते पिता ॥देवयान्युवाच}
{}


\threelineshloka
{स्तुवतो दुहिताऽहं ते याचतः प्रतिगृह्णतः}
{स्तूयमानस्य दुहिता कथं दासी भविष्यसि ॥शर्मिष्ठोवाच}
{}


\threelineshloka
{येनकेनचिदार्तानां ज्ञातीनां सुखमावहेत्}
{अतस्त्वामनुयास्यामि तत्र दास्यति ते पिता ॥वैशंपायन उवाच}
{}


\threelineshloka
{प्रतिश्रुते दासभावे दुहित्रा वृषपर्वणः}
{देवयानी नृपश्रेष्ठ पितरं वाक्यमब्रवीत् ॥देवयान्युवाच}
{}


\threelineshloka
{प्रविशामि पुरं तात तुष्टाऽस्मि द्विजसत्तम}
{अमोघं तव विज्ञानमस्ति विद्याबलं च ते ॥वैशंपायन उवाच}
{}


\twolineshloka
{एवमुक्तो दुहित्रा स द्विजश्रेष्ठो महायशाः}
{प्रविवेश पुरं हृष्टः पूजितः सर्वदानैवः}


\chapter{अध्यायः ७५}
\twolineshloka
{वैशंपायन उवाच}
{}


\twolineshloka
{अथ दीर्घस्य कालस्य देवयानी नृपोत्तम}
{वनं तदेव निर्याता क्रीडार्थं वरवर्णिनी}


\twolineshloka
{तेन दासीसहस्रेण सार्धं शर्मिष्ठया तदा}
{तमेव देशं संप्राप्ता यथाकामं चचार सा}


\twolineshloka
{ताभिः सखीभिः सहिता सर्वाभिर्मुदिता भृशम्}
{क्रीडन्त्योऽभिरताः सर्वाः पिबन्त्यो मधुमाधवीं}


\twolineshloka
{खादन्त्यो विविधान्भक्ष्यान्विदशन्त्यः फलानि च}
{पुनश्च नाडुषो राजा मृगलिप्सुर्यदृच्छया}


\twolineshloka
{तमेव देशं संप्राप्तो जलार्थी श्रमकर्शितः}
{ददर्श देवयानीं स शर्मिष्ठां ताश्च योषितः}


\threelineshloka
{पिबन्तीर्ललमानाश्च दिव्याभरणभूषिताः}
{आसनप्रवरे दिव्ये सर्वरत्नविभूषिते}
{उपविष्टां च ददृशे देवयानीं शुचिस्मिताम्}


\twolineshloka
{रूपेणाप्रतिमां तासां स्त्रीणां मध्ये वराङ्गनम्}
{`आसनाच्च ततः किंचिद्विहीनां हेमभीषिताम्}


\twolineshloka
{असुरेन्द्रसुतां चापि निषण्णां चारुहासिनीम्}
{ददर्श पादौ विप्रायाः संवहन्तीमनिन्दिताम्}


\threelineshloka
{गायन्त्योऽथ प्रनृत्यन्त्यो वादयन्त्योऽथ भारत}
{दृष्ट्वा ययातिमतुलं लज्जयाऽवनताः स्थिताः ॥'ययातिरुवाच}
{}


\fourlineindentedshloka
{द्वाभ्यां कन्यासहस्राभ्यां द्वे कन्ये परिवारिते}
{गोत्रे च नामनी चैव द्वयोः पृच्छाम्यहं शुभे}
{देवयान्युवाच}
{}


\twolineshloka
{आख्यास्याम्यहमादत्स्व वचनं मे नराधिप}
{शुक्रो नामासुरगुरुः सुतां जानीहि तस्य माम्}


\threelineshloka
{इयं च मे सखी दासी यत्राहं तत्र गामिनी}
{दुहिता दानवेन्द्रस्य शर्मिष्ठा वृषपर्वणः ॥ययातिरुवाच}
{}


\twolineshloka
{कथं तु ते सखी दासी कन्येयं वरवर्णिनी}
{असुरेन्द्रसुता सुभ्रूः परं कौतूहलं हि मे}


\twolineshloka
{`नैव देवी न गन्धर्वी न यक्षी न च किन्नरी}
{नैवंरूपा मया नारी दृष्टपूर्वा महीतले}


\twolineshloka
{श्रीरिवायतपद्माक्षी सर्वलक्षणशोभना}
{असुरेन्द्रसुता कन्या सर्वालङ्कारभूषिता}


\twolineshloka
{दैवेनोपहता सुभ्रूरुताहो तपसापि वा}
{अन्यथैषाऽनवद्याङ्गी दासी नेह भविष्यति}


\threelineshloka
{अस्या रूपेण ते रूपं न किंचित्सदृशं भवेत्}
{पुरा दुश्चरितेनेयं तव दासी भवत्यहो ॥'देवयान्युवाच}
{}


\twolineshloka
{सर्व एव नरश्रेष्ठ विधानमनुवर्तते}
{विधानविहितं मत्वा मा विचित्राः कथाःकृथाः}


\threelineshloka
{राजवद्रूपवेषौ ते ब्राह्मीं वाचं बिभर्षि च}
{को नाम त्वं कुतश्चासि कस्य पुत्रश्च शस मे ॥ययातिरुवाच}
{}


\threelineshloka
{ब्रहमचर्येण वेदो मे कृत्स्रः श्रुतिपथं गतः}
{राजाहं राजपुत्रश्च ययातिरिति विश्रुतः ॥देवयान्युवाच}
{}


\threelineshloka
{केनास्यर्थेन नृपते इमं देशमुपागतः}
{जिघृक्षुर्वारिजं किंचिदथवा मृगलिप्सया ॥ययातिरुवाच}
{}


\threelineshloka
{मृगलिप्सुरहं भद्रे पानीयार्थमुपागतः}
{बहुधाऽप्यनुयुक्तोऽस्मि तदनुज्ञातुमर्हसि ॥देवयान्युवाच}
{}


\threelineshloka
{द्वाभ्यां कन्यासहस्राभ्यां दास्या शर्मिष्ठया सह}
{त्वदधीनाऽस्मि भद्रं ते सखा भर्ता च मे भव ॥`वैशंपायन उवाच}
{}


\threelineshloka
{असुरेन्द्रसुतामीक्ष्य तस्यां सक्तेन चेतसा}
{शर्मिष्ठा महिषी मह्यमिति मत्वा वचोऽब्रवीत्' ॥ययातिरुवाच}
{}


\twolineshloka
{विद्ध्यौशनसि भद्रं ते न त्वामर्होऽस्मि भामिनि}
{अविवाह्या हि राजानो देवयानि पितुस्तव}


\threelineshloka
{`परभार्या स्वसा श्रेष्ठा सगोत्रा पतिता स्नुषा}
{अवरा भिक्षुकाऽस्वस्था अगम्याः कीर्तिता बुधैः ॥देवयान्युवाच}
{}


\fourlineindentedshloka
{संसृष्टं ब्रह्मणा क्षत्रं क्षत्रेण ब्रह्म संहितम्}
{`अन्यत्वमस्ति न तयोरेकान्ततरमास्थिते}
{'ऋषिश्चाप्यृषिपुत्रश्च नाहुषाङ्ग वहस्व माम् ॥ययातिरुवाच}
{}


\threelineshloka
{एकदेहोद्भवा वर्णाश्चत्वारोऽपि वराङ्गने}
{पृथग्धर्माः पृथक्शौचास्तेषां तु ब्राह्मणो वरः ॥देवयान्युवाच}
{}


\twolineshloka
{पाणि धर्मो नाहुषाऽयं न पुंभिः सेवितः पुरा}
{तं मे त्वमग्रहीरग्रे वृणोमि त्वामहं ततः}


\threelineshloka
{कथं नु मे मनस्विन्याः पाणिमन्यः पुमान्स्पृशेत्}
{गृहीतमृषिपुत्रेण स्वयं वाप्यृषिणा त्वया ॥ययातिरुवाच}
{}


\threelineshloka
{क्रुद्धादाशीविषात्सर्पाज्ज्वलनात्सर्वतोमुखात्}
{दुराधर्षतरो विप्रो ज्ञेयः पुंसा विजानता ॥देवयान्युवाच}
{}


\threelineshloka
{कथमाशीविषात्सर्पाज्ज्वलनात्सर्वतोमुखात्}
{दुराधर्षतरो विप्र इत्यात्थ पुरुषर्षभ ॥ययातिरुवाच}
{}


\twolineshloka
{एकमाशीविषो हन्ति शस्त्रेणैकश्च वध्यते}
{हन्ति विप्रः सराष्ट्राणि पुराण्यपि हि कोपितः}


\threelineshloka
{दुराधर्षतरो विप्रस्तस्माद्भीरु मतो मम}
{अतोऽदत्तां च पित्रा त्वां भद्रे न विवहाम्यहम् ॥देवयान्युवाच}
{}


\twolineshloka
{दत्तां वहस्व तन्मा त्वं पित्रा राजन्वृतो मया}
{आयचतो भयं नास्ति दत्तां च प्रतिगृह्णतः}


\fourlineindentedshloka
{`तिष्ठ राजन्मुहूर्तं च प्रेषयिष्याम्यहं पितुः}
{गच्छ त्वं धात्रिके शीघ्रं ब्रह्मकल्पमिहानय}
{स्वयंवरे वृतं शीघ्रं निवेदय च नाहुषम्' ॥वैशंपायन उवाच}
{}


\twolineshloka
{त्वरितं देवयान्याथ संदिष्टं पितुरात्मनः}
{सर्वं निवेदयामास धात्री तस्मै यथातथम्}


\fourlineindentedshloka
{श्रुत्वैव च स राजानं दर्शयामास भार्गवः}
{दृष्ट्वैव चागतं शुक्रं ययातिः पृथिवीपतिः}
{ववन्दे ब्राह्मणं काव्यं प्राञ्जलिः प्रणतः स्थितः ॥देवयान्युवाच}
{}


\fourlineindentedshloka
{राजायं नाहुषस्तात दुर्गमे पाणिमग्रहीत्}
{नान्यपूर्वगृहीतं मे तेनाहमभया कृता}
{नमस्ते देहि मामस्मै लोके नान्यं पतिं वृणे ॥शुक्र उवाच}
{}


\twolineshloka
{अन्यो धर्मः प्रियस्त्वन्यो वृतस्ते नाहुषः पतिः}
{कचशापात्त्वया पूर्वं नान्यद्भवितुमर्हति}


\fourlineindentedshloka
{वृतोऽनया पतिर्वीर सुतया त्वं ममेष्टया}
{स्वयं ग्रहे महान्दोषो ब्राह्मण्या वर्णसंकरात्}
{गृहाणेमां मया दत्तां महिषीं नहुषात्मज ॥ययातिरुवाच}
{}


\threelineshloka
{अधर्मो न स्पृशेदेष महान्मामिह भार्गव}
{वर्णसंकरजो ब्रह्मन्निति त्वां प्रवृणोम्यहम् ॥शुक्र उवाच}
{}


\twolineshloka
{अधर्मात्त्वां विमुञ्चामि शृणु त्वं वरमीप्सितम्}
{अस्मिन्विवाहे मा म्लासीरहं पापं नुदामि ते}


\twolineshloka
{वहस्व भार्यां धर्मेण देवयानीं सुमध्यमाम्}
{अनया सह संप्रीतिमतुलां समवाप्नुहि}


\twolineshloka
{इयं चापि कुमारी ते शर्मिष्ठा वार्षपर्वणी}
{संपूज्या सततं राजन्मा चैनां शयने ह्वयेः}


\threelineshloka
{रहस्येनां समाहूय न वदेर्न च संस्पृशेः}
{वहस्व भार्यां भद्रं ते यथा काममवाप्स्यसि ॥वैशंपायन उवाच}
{}


\twolineshloka
{एवमुक्तो ययातिस्तु शुक्रं कृत्वा प्रदक्षिणम्}
{शास्त्रोक्तविधिना राजा विवाहमकरोच्छुभम्}


\twolineshloka
{लब्ध्वा शुक्रान्महद्वित्तं देवयानीं तदोत्तमाम्}
{द्विसहस्रेण कन्यानां तथा शर्मिष्ठया सह}


\twolineshloka
{संपूजितश्च शुक्रेण दैत्यैश्च नृपसत्तमः}
{जगाम स्वपुरं हृष्टोऽनुज्ञातोऽथ महात्मना}


\chapter{अध्यायः ७६}
\twolineshloka
{वैशंपायन उवाच}
{}


\twolineshloka
{ययातिः स्वपुरं प्राप्य महेन्द्रपुरसंनिभम्}
{प्रविश्यान्तःपुरं तत्र देवयानीं न्यवेशयत्}


\twolineshloka
{देवयान्याश्चानुमते सुतां तां वृषपर्वणः}
{अशोकवनिकाभ्याशे गृहं कृत्वा न्यवेशयत्}


\twolineshloka
{वृतां दासीसहस्रेण शर्मिष्ठां वार्षपर्वणीम्}
{वासोभरन्नपानैश्च संविभज्य सुसत्कृताम्}


\twolineshloka
{देवयान्या तु सहितः स नृपो नहुषात्मजः}
{`प्रीत्या परमया युक्तो मुमुदे शाश्वतीः समाः}


\twolineshloka
{अशोकवनिकामध्ये देवयानी समागता}
{शर्मिष्ठया सा क्रीडित्वा रमणीये मनोरमे}


\twolineshloka
{तत्रैव तां तु निर्दिश्य राज्ञा सह ययौ गृहम्}
{एवमेव सह प्रीत्या बहु कालं मुमोद च ॥'}


% Check verse!
विजहार बहूनब्दान्देववन्मुदितः सुखी
\twolineshloka
{ऋतुकाले तु संप्राप्ते देवयानी वराङ्गना}
{लेभे गर्भं प्रथमतः कुमारं च व्यजायत}


\twolineshloka
{गते वर्षसहस्रे तु शर्मिष्ठा वार्षपर्वणी}
{ददर्श यौवनं प्राप्ता ऋतुं सा चान्वचिन्तयत्}


\twolineshloka
{`शुद्धा स्नाता तु शर्मिष्ठा सर्वालङ्कारशोभिता}
{अशोकशाखामालम्ब्य सुपुष्पस्तबकैर्वृताम्}


\twolineshloka
{आदर्शे मुखमुद्वीक्ष्य भर्तुर्दर्शनलालसा}
{शोकमोहसमाविष्टा वचनं चेदमब्रवीत्}


\threelineshloka
{अशोक शोकापनुद शोकोपहतचेतसाम्}
{त्वन्नामानं कुरुष्वाद्य प्रियसंदर्शनेन माम्}
{एवमुक्तवती सा तु शर्मिष्ठा पुनरब्रवीत् ॥'}


\twolineshloka
{ऋतुकालश्च संप्राप्तो न च मेऽस्ति वृतः पतिः}
{किं प्राप्तं किं नु कर्तव्यं किं वा कृत्वा सुखं भवेत्}


\twolineshloka
{देवयानी प्रजाताऽसौ वृथाऽहं प्राप्तयौवना}
{यथा तया वृतो भर्ता तथैवाहं वृणोमि तम्}


\twolineshloka
{राज्ञा पुत्रफलं देयमिति मे निश्चिता मतिः}
{अपीदानीं स धर्मात्मा ईयान्मे दर्शनं रहः}


\fourlineindentedshloka
{`केशैर्बध्या तु राजानं याचेऽहं सदृशं पतिम्}
{स्पृहेदिदं देवयानी पुत्रमीक्ष्य पुनःपुनः}
{क्रीडन्नन्तःपुरे तस्याः क्वचित्क्षणमवाप्य च ॥वैशंपायन उवाच}
{'}


\twolineshloka
{अथ निष्क्रम्य राजाऽसौ तस्मिन्काले यदृच्छया}
{अशोकवनिकाभ्याशे शर्मिष्ठां प्राप तिष्ठतीम्}


\threelineshloka
{तमेकं रहिते दृष्ट्वा शर्मिष्ठा चारुहासिनी}
{प्रत्युद्गम्याञ्जलिं कृत्वा राजानं वाक्यमब्रवीत् ॥शर्मिष्ठोवाच}
{}


\twolineshloka
{सोमस्येन्द्रस्य विष्णोर्वा यमस्य वरुणस्य वा}
{तव वा नाहुष गृहे कः स्त्रियं द्रष्टुमर्हति}


\threelineshloka
{रूपाभिजनशीलैर्हि त्वं राजन्वेत्थ मां सदा}
{सा त्वां याचे प्रसाद्याहमृतुं देहि नराधिप ॥ययातिरुवाच}
{}


\twolineshloka
{वेद्मि त्वां शीलसंपन्नां दैत्यकन्यामनिन्दिताम्}
{रूपं च ते न पश्यामि सूच्यग्रमपि निन्दितम्}


\threelineshloka
{`तदाप्रभृति दृष्ट्वा त्वां स्मराम्यनिशमुत्तमे'}
{अब्रवीदुशना काव्यो देवयानीं यदाऽवहम्}
{नेयमाह्वयितव्या ते शयने वार्षपर्वणी}


\twolineshloka
{`देवयान्याः प्रियं कृत्वा शर्मिष्ठामपि पोषय ॥' शर्मिष्ठोवाच}
{}


\twolineshloka
{न नर्मयुक्तमनृतं हिनस्तिन स्त्रीषु राजन्न विवाहकाले}
{प्राणात्यये सर्वधनापहारेपञ्चानृतान्याहुरपातकानि}


\twolineshloka
{पृष्टं तु साक्ष्ये प्रवदन्तमन्यथावदन्ति मिथ्या पतितं नरेन्द्र}
{एकार्थतायां तु समाहितायांमिथ्या वदन्तं ह्यनृतं हिनस्ति}


\threelineshloka
{`अनृतं नानृतं स्त्रीषु परिहासविवाहयोः}
{आत्मप्राणार्थघाते च तदेवोत्तमतां व्रजेत् ॥'ययातिरुवाच}
{}


\threelineshloka
{राजा प्रमाणं भूतानां स नश्येत मृषा वदन्}
{अर्थकृच्छ्रमपि प्राप्य न मिथ्या कर्तुमुत्सहे ॥शर्मिष्ठोवाच}
{}


\twolineshloka
{समावेतौ मतो राजन्पतिः सख्याश्च यः पतिः}
{समं विवाहमित्याहुः सख्या मेऽसि वृतः पतिः}


\twolineshloka
{`सह दत्तास्मि काव्येन देवयान्या मनीषिणा}
{पूज्या पोषयितव्येति न मृषा कर्तुमर्हसि ॥'}


\twolineshloka
{सुवर्णमणिमुक्तानि वस्त्राण्याभरणानि च}
{याचितॄणां ददासि त्वं गोभूम्यादीनि यानि च}


\twolineshloka
{बहिःस्थं दानमित्युक्तं न शरीराश्रितं नृप}
{दुष्करं पुत्रदानं च आत्मदानं च दुष्करम्}


\twolineshloka
{शरीरदानात्तत्सर्वं दत्तं भवति मारिष}
{यस्य यस्य यथा कामस्तस्य तस्य ददाम्यहम्}


\fourlineindentedshloka
{इत्युक्त्वा नगरे राजंस्त्रिकालं घोषितं त्वया}
{त्वयोक्तमनृतं राजन्वृथा घोषितमेव वा}
{तत्सत्यं कुरु राजेन्द्र यथा वैश्रवणस्तथा ॥ययातिरुवाच}
{}


\twolineshloka
{दातव्यं याचमानेभ्य इति मे व्रतमाहितम्}
{त्वं च याचसि मां कामं ब्रूहि किं करवाणि ते}


\threelineshloka
{`धनं वा यदि वा किंचिद्राज्यं वाऽपि शुचिस्मिते}
{' शर्मिष्ठोवाच}
{अधर्मात्पाहि मां राजन्धर्मं च प्रतिपादय}


\twolineshloka
{`नान्यं वृणे पुत्रकामा पुत्रात्परतरं न च}
{'त्वत्तोऽपत्यवती लोके चरेयं धर्ममुत्तमम्}


\twolineshloka
{त्रय एवाधना राजन्भार्या दासस्तथा सुतः}
{यत्ते समधिगच्छन्ति यस्यैते तस्य तद्धनम्}


\threelineshloka
{`पुत्रार्थं भर्तृपोषार्थं स्त्रियः सृष्टाः स्वयंभुवा}
{अपतिर्वापि या कन्या अनपत्या च या भवेत्}
{तासां जन्म वृथा लोके गतिस्तासां न विद्यते ॥'}


\threelineshloka
{देवयान्या भुजिष्याऽस्मि वश्या च तव भार्गवी}
{सा चाहं च त्वया राजन्भजनीये भजस्व माम् ॥वैशंपायन उवाच}
{}


\twolineshloka
{एवमुक्तस्तु राजा स तथ्यमित्यभिजज्ञिवान्}
{पूजयामास शर्मिष्ठां धर्मं च प्रत्यपादयत्}


\twolineshloka
{स समागम्य शर्मिष्ठां यथा काममवाप्य च}
{अन्योन्यं चाभिसंपूज्य जग्मतुस्तौ यथागतम्}


\twolineshloka
{तस्मिन्समागमे सुभ्रूः शर्मिष्ठा चारुहासिनी}
{लेभे गर्भं प्रथमतस्तस्मान्नृपतिसत्तमात्}


\twolineshloka
{प्रयज्ञे च ततः काले राजन्राजीवलोचना}
{कुमारं देवगर्भाभं राजीवनिभलोचनम्}


\chapter{अध्यायः ७७}
\twolineshloka
{वैशंपायन उवाच}
{}


\twolineshloka
{`तस्मिन्नक्षत्रसंयोगे शुक्ले पुण्यर्क्षगेन्दुना}
{स राजा मुमुदे सम्राट् तया शर्मिष्ठया सह}


\twolineshloka
{प्रजानां श्रीरिवाभ्याशे शर्मिष्ठा ह्यभवद्वधूः}
{पन्नगीवोग्ररूपा वै देवयानी ममाप्यभूत्}


\threelineshloka
{पर्जन्य इव सस्यानां देवानाममृतं यथा}
{तद्वन्ममापि संभूता शर्मिष्ठा वार्षपर्वणी}
{इत्येवं मनसा ज्ञात्वा देवयानीमवर्जयत् ॥'}


\twolineshloka
{श्रुत्वा कुमारं जातं तु देवयानी शुचिस्मिता}
{चिन्तयामास दुःखार्ता शर्मिष्ठां प्रति भारत}


\fourlineindentedshloka
{अभिगम्य च शर्मिष्ठां देवयान्यब्रवीदिदम्}
{देवयान्वुवाच}
{किमिदं वृजिनं सुभ्रु कृतं वै कामलुब्धया ॥शर्मिष्ठोवाच}
{}


\twolineshloka
{ऋषिरभ्यागतः कश्चिद्धर्मात्मा वेदपारगः}
{स मया वरदः कामं याचितो धर्मसंहितम्}


\fourlineindentedshloka
{`अपत्यार्थे स तु मया वृतो वै चारुहासिनि'}
{नाहमन्यायतः काममाचरामि शुचिस्मिते}
{तस्मादृषेर्ममापत्यमिति सत्यं ब्रवीमि ते ॥देवयान्युवाच}
{}


\threelineshloka
{शोभनं भीरु यद्येवमथ स ज्ञायते द्विजः}
{गोत्रनामाभिजनतो वेत्तुमिच्छामि तं द्विजम् ॥शर्मिष्ठोवाच}
{}


\threelineshloka
{तपसा तेजसा चैव दीप्यमानं यथा रविम्}
{तं दृष्ट्वा मम संप्रष्टुं शक्तिर्नासीच्छुचिस्मिते ॥देवयान्युवाच}
{}


\threelineshloka
{यद्येतदेवं शर्मिष्ठे न मन्युर्विद्यते मम}
{अपत्यं यदि ते लब्धं ज्येष्ठाच्छ्रेष्ठाच्च वै द्विजात् ॥वैशंपायन उवाच}
{}


\twolineshloka
{अन्योन्यमेवमुक्त्वा तु संप्रहस्य च ते मिथः}
{जगाम भार्गवी वेश्म तथ्यमित्यवजग्मुषी}


\twolineshloka
{ययातिर्देवयान्यां तु पुत्रावजनयन्नृपः}
{यदुं च तुर्वसुं चैव शक्रविष्णू इवापरौ}


\twolineshloka
{`तस्मिन्काले तु राजर्षिर्ययातिः पृथिवीपतिः}
{माध्वीकरससंयुक्तां मदिरां मदवर्धनीम्}


\twolineshloka
{पाययामास शुक्रस्य तनयां रक्तपिङ्गलाम्}
{पीत्वा पीत्वा च मदिरां देवयानी मुमोह सा}


\twolineshloka
{रुदती गायमाना च नृत्यन्ती च मुहुर्मुहुः}
{बहु प्रलपती देवी राजानमिदमब्रवीत्}


\twolineshloka
{राजवद्रूपवेषौ ते किमर्थं त्वमिहागतः}
{केन कार्येण संप्राप्तो निर्जनं गहनं वनम्}


\twolineshloka
{द्विजश्रेष्ठ नृपश्रेष्ठो ययातिश्चोग्रदर्शनः}
{तस्मादितः पलायस्व हितमिच्छसि चेद्द्विज}


\twolineshloka
{इत्येवं प्रलपन्तीं तां देवयानीं तु नाहुषः}
{भर्त्सयामास वचनैरनर्हां पापवर्धनीम्}


\twolineshloka
{ततो वर्षवरान्मूकान्व्यङ्गान्वृद्धांश्च पङ्गुकान्}
{रक्षणे देवयान्याः स पोषणे च शशास तान्}


\twolineshloka
{ततस्तु नाहुषो राजा शर्मिष्ठां प्राप्य बुद्धिमान्}
{रेमे च सुचिरं कालं तया शर्मिष्ठया सह ॥'}


\twolineshloka
{तस्मादेव तु राजर्षेः शर्मिष्ठा वार्षपर्वणी}
{द्रुह्युं चानुं च पूरुं च त्रीन्कुमारानजीजनत्}


\twolineshloka
{ततः काले तु कस्मिंश्चिद्देवयानी शुचिस्मिता}
{ययातिसहिता राजञ्जगाम रहितं वनम्}


\threelineshloka
{ददर्श च तदा तत्र कुमारान्देवरूपिणः}
{क्रीडमानान्सुविश्रब्धान्विस्मिता चेदमब्रवीत् ॥देवयान्युवाच}
{}


\threelineshloka
{`कस्यैते दारका राजन्देवपुत्रोपमाः शुभाः}
{वर्चसा रूपतश्चैव सदृशा मे मतास्तव ॥वैशंपायन उवाच}
{}


% Check verse!
एवं पृष्ट्वा तु राजानं कुमारान्पर्यपृच्छत
\threelineshloka
{तस्मिन्काले तु तच्छ्रुत्वा धात्री तेषां वचोऽब्रवीत्}
{किं न ब्रूत कुमारा वः पितरं वै द्विजर्षभम् ॥कुमारा ऊचुः}
{}


\threelineshloka
{ऋषिश्च ब्राह्मणश्चैव द्विजातिश्चैव नः पिता}
{शर्मिष्ठा नानृतं ब्रूते देवयानि क्षमस्व नः ॥'देवयान्युवाच}
{}


\twolineshloka
{किंनामधेयगोत्रो वः पुत्रका ब्राह्मणः पिता}
{प्रब्रूत तत्त्वतः क्षिप्रं कश्चासौ क्व च वर्तते}


\twolineshloka
{प्रब्रूत मे यथा तथ्यं श्रोतुमिच्छामि तं ह्यहम्}
{एवमुक्ताः कुमारस्ते देवयान्या सुमध्यया}


\threelineshloka
{तेऽदर्शयन्प्रदेशिन्या तमेव नृपसत्तमम्}
{शर्मिष्ठां मातरं चैव तथाऽऽचख्युश्च दारकाः ॥वैशंपायन उवाच}
{}


\twolineshloka
{इत्युक्त्वा सहितास्ते तु राजानमुपचक्रमुः}
{नाभ्यनन्दत तान्राजा देवयान्यास्तदान्तिके}


\twolineshloka
{रुदन्तस्तेऽथ शर्मिष्ठामभ्ययुर्बालकास्ततः}
{`अविब्रुवन्ती किंचिच्च राजानं चारुलोचना}


\twolineshloka
{नातिदूराच्च राजानं सा चातिष्ठदवाङ्मुखी}
{श्रुत्वा तेषां तु बालानां सव्रीड इव पार्थिवः}


\twolineshloka
{प्रतिवक्तुमशक्तोऽभूत्तूष्णींभूतोऽभवन्नृपः}
{दृष्ट्वा तु तेषां बालानां प्रणयं पार्थिवं प्रति}


\twolineshloka
{बुद्ध्वा तु तत्त्वतो देवी शर्मिष्ठापिदमब्रवीत्}
{अभ्यागच्छति मां कश्चिदृषिरित्येवमब्रवीः}


\twolineshloka
{ययातिमेवं राजानं त्वं गोपायसि भामिनि}
{पूर्वमेव मया प्रोक्तं त्वया तु वृजिनं कृतम्}


\threelineshloka
{मदधीना सती कस्मादकार्षीर्विप्रियं मम}
{तमेवाऽऽसुरधर्मं त्वमास्थिता न बिभेषि मे ॥'शर्मिष्ठोवाच}
{}


\twolineshloka
{यदुक्तमृषिरित्येव तत्सत्यं चारुहासिनि}
{न्यायतो धर्मतश्चैव चरन्ती न बिभेमि ते}


\twolineshloka
{यदा त्वया वृतो भर्ता वृत एव तदा मया}
{सखीभर्ता हि धर्मेण भर्ता भवति शोभने}


\twolineshloka
{पूज्यासि मम मान्या च ज्येष्ठा च ब्राह्मणी ह्यसि}
{त्वत्तोपि मे पूज्यतमो राजर्षिः किं न वेत्थ तत्}


\threelineshloka
{`त्वत्पित्रा मम गुरुणा सह दत्ते उभे शुभे}
{ततो भर्ता च पूज्यश्च पोष्यां पोषयतीह माम् ॥वैशंपायन उवाच}
{}


\twolineshloka
{श्रुत्वा तस्यास्ततो वाक्यं देवयान्यब्रवीदिदम्}
{रमस्वेह यथाकामं देव्या शर्मिष्ठया सह}


\twolineshloka
{राजन्नाद्येह वत्स्यामि विप्रियं मे कृतं त्वया}
{इति जज्वाल कोपेन देवयानी ततो भृशम्}


\twolineshloka
{निर्दहन्तीव सव्रीडां शर्मिष्ठां समुदीक्ष्य च}
{अपविध्य च सर्वाणि भूषणान्यसितेक्षणा ॥'}


\twolineshloka
{सहसोत्पतितां श्यामां दृष्ट्वा तां साश्रुलोचनाम्}
{तूर्णं सकाशं काव्यस्य प्रस्थितां व्यथितस्तदा}


\twolineshloka
{अनुवव्राज संभ्रान्तः पृष्ठतः सान्त्वयन्नृपः}
{न्यवर्तत नचैव स्म क्रोधसंरक्तलोचना}


\twolineshloka
{अविब्रुवन्ती किंचित्सा राजानं साश्रुलोचना}
{अचिरादेव संप्राप्ता काव्यस्योशनसोऽन्तिकम्}


\threelineshloka
{सा तु दृष्ट्वै पितरमभिवाद्याग्रतः स्थिता}
{अनन्तरं यायातिस्तु पूजयामास भार्गवम् ॥देवयान्युवाच}
{}


\twolineshloka
{अधर्मेण जितो धर्मः प्रवृत्तमधरोत्तरम्}
{शर्मिष्ठयाऽतिवृत्ताऽस्मि दुहित्रा वृषपर्वणः}


\twolineshloka
{त्रयोऽस्यां जनिताः पुत्रा राज्ञाऽनेन ययातिना}
{दुर्भगाया मम द्वौ तु पुत्रौ तात ब्रवीमि ते}


\threelineshloka
{धर्मज्ञ इति विख्यात एष राजा भृगूद्वह}
{अतिक्रान्तश्च मर्यादां काव्यैतत्कथयामि ते ॥शुक्र उवाच}
{}


\threelineshloka
{धर्मज्ञः सन्महाराज योऽधर्ममकृथाः प्रियम्}
{तस्माज्जरा त्वामचिराद्धर्षयिष्यति दुर्जया ॥ययातिरुवाच}
{}


\twolineshloka
{ऋतुं वै याचमानाया भगवन्नान्यचेतसा}
{दुहितुर्दानवेन्द्रस्य धर्म्यमेतत्कृतं मया}


\twolineshloka
{ऋतुं वै याचमानाया न ददाति पुमानृतुम्}
{भ्रूणहेत्युच्यते ब्रह्मन् स इह ब्रह्मवादिभिः}


\twolineshloka
{अभिकामां स्त्रियं यश्च गम्यां रहसि याचितः}
{नोपैति स च धर्मेषु भ्रूणहेत्युच्यते बुधैः}


\twolineshloka
{`यद्यद्वृणोति मां कश्चित्तत्तद्देयमिति व्रतम्}
{त्वया च सापि दत्ता मे नान्यं नाथमिहेच्छति'}


\fourlineindentedshloka
{इत्येतानि समीक्ष्याहं कारणानि भृगूद्वह}
{अधर्मभयसंविग्नः शर्मिष्ठामुपजग्मिवान्}
{`मत्वैतन्मे धर्म इति कृतं ब्रह्मन्क्षमस्व माम् ॥' शुक्र उवाच}
{}


\threelineshloka
{नन्वहं प्रत्यवेक्ष्यस्ते मदधीनोऽसि पार्थिव}
{मिथ्याचारस्य धर्मेषु चौर्यं भवति नाहुष ॥वैशंपायन उवाच}
{}


\threelineshloka
{क्रुद्धेनोशनसा शप्तो ययातिर्नाहुषस्तदा}
{पूर्वं वयः परित्यज्य जरां सद्योऽन्वपद्यत ॥ययातिरुवाच}
{}


\threelineshloka
{अतृप्तो यौवनस्याहं देवयान्यां भृगूद्वह}
{प्रसादं कुरु मे ब्रह्मञ्जरेयं न विशेच्च माम् ॥शुक्र उवाच}
{}


\threelineshloka
{नाहं मृषा ब्रवीम्येतज्जरां प्राप्तोऽसि भूमिप}
{जरां त्वेतां त्वमन्यस्मिन्संक्रामय यदीच्छसि ॥ययातिरुवाच}
{}


\threelineshloka
{राज्यभाक्स भवेद्ब्रह्मन्पुण्यभाक्कीर्तिभाक्तथा}
{यो मे दद्याद्वयः पुत्रस्तद्भवाननुमन्यताम् ॥शुक्र उवाच}
{}


\twolineshloka
{संक्रामयिष्यसि जरां येथेष्टं नहुषात्मज}
{मामनुध्याय भावेन न च पापमवाप्स्यसि}


\twolineshloka
{वयो दास्यति ते पुत्रो यः स राजा भविष्यति}
{आयुष्मान्कीर्तिमांश्चैव बह्वपत्यस्तथैव च}


\chapter{अध्यायः ७८}
\twolineshloka
{वैशंपायन उवाच}
{}


\threelineshloka
{जरां प्राप्य ययातिस्तु स्वपुरं प्राप्य चैव हि}
{पुत्रं ज्येष्ठं वरिष्ठं च यदुमित्यब्रवीद्वचः ॥ययातिरुवाच}
{}


\twolineshloka
{जरावली च मां तात पलितानि च पर्यगुः}
{काव्यस्योशनसः शापान्न च तृप्तोऽस्मि यौवने}


\twolineshloka
{त्वं यदो प्रतिपद्यस्व पाप्मानं जरया सह}
{यौवनेन त्वदीयेन चरेयं विषयानहम्}


\threelineshloka
{पूर्णे वर्षसहस्रे तु पुनस्ते यौवनं त्वहम्}
{दत्त्वा स्वं प्रतिपत्स्यामि पाप्मानं जरया सह ॥यदुरुवाच}
{}


\twolineshloka
{जरायां बहवो दोषाः पानभोजनकारिताः}
{तस्माज्जरां न ते राजन्ग्रहीष्य इति मे मतिः}


\twolineshloka
{सितश्मश्रुर्निरानन्दो जरया शिथिलीकृतः}
{वलीसङ्गतगात्रस्तु दुर्दर्शो दुर्बलः कृशः}


\twolineshloka
{अशक्तः कार्यकरणे परिभूतः स यौवतैः}
{सहोपजीविभिश्चैव तां जरां नाभिकामये}


\threelineshloka
{सन्ति ते बहवः पुत्रा मत्तः प्रियतरा नृप}
{जरां ग्रहीतुं धर्मज्ञ तस्मादन्यं वृणीष्व वै ॥ययातिरुवाच}
{}


\twolineshloka
{यत्त्वं मे हृदयाज्जातो वयः स्वं न प्रयच्छसि}
{तस्मादराज्यभाक्तात प्रजा तव भविष्यति}


\threelineshloka
{`प्रत्याख्यातस्तु राजा स तुर्वसुं प्रत्युवाच ह}
{'तुर्वसो प्रतिपद्यस्व पाप्मानं जरया सह}
{यौवनेन चरेयं वै विषयांस्तव पुत्रक}


\threelineshloka
{पूर्णे वर्षसहस्रे तु पुनर्दास्यामि यौवनम्}
{स्वं चैव प्रतिपत्स्यामि पाप्मानं जरया सह ॥तुर्वसुरुवाच}
{}


\threelineshloka
{न कामये जरां तात कामभोगप्रणाशिनीम्}
{बलरूपान्तकरणीं बुद्धिप्राणप्रणाशिनीम् ॥ययातिरुवाच}
{}


\twolineshloka
{यत्त्वं मे हृदयाज्जातो वयः स्वं न प्रयच्छसि}
{तस्मात्प्रजा समुच्छेदं तुर्वसो तव यास्यति}


\twolineshloka
{संकीर्णाचारधर्मेषु प्रतिलोमचरेषु च}
{पिशिताशिषु चान्त्येषु मूढ राजा भविष्यसि}


\threelineshloka
{गुरुदारप्रसक्तेषु तिर्यग्योनिगतेषु च}
{पशुधर्मेषु पापेषु म्लेच्छेषु त्वं भविष्यसि ॥वैशंपायन उवाच}
{}


\threelineshloka
{एवं स तुर्वसुं शप्त्वा ययातिः सुतमात्मनः}
{शर्मिष्ठायाः सुतं द्रुह्युमिदं वचनमब्रवीत् ॥ययातिरुवाच}
{}


\twolineshloka
{द्रुह्यो त्वं प्रतिपद्यस्व वर्णरूपविनाशिनीम्}
{जरां वर्षसहस्रं मे यौवनं स्वं ददस्व च}


\threelineshloka
{पूर्णे वर्षसहस्रे तु पुनर्दास्यामि यौवनम्}
{स्वं चादास्यामि भूयोऽहं पाप्मानं जरया सह ॥द्रुह्युरुवाच}
{}


\threelineshloka
{न गजं न रथं नाश्वं जीर्णो भुङ्क्ते न च स्त्रियम्}
{वाग्भङ्गश्चास्य भवति तां जरां नाभिकामये ॥ययातिरुवाच}
{}


\twolineshloka
{यत्त्वं मे हृदयाज्जातो वयः स्वं न प्रयच्छसि}
{तस्माद्द्रुह्यो प्रियः कामो न ते संपत्स्यते क्वचित्}


\twolineshloka
{यत्राश्वरथमुख्यानामश्वानां स्याद्गतं न च}
{हस्तिनां पीठकानां च गर्दभानां तथैव च}


\fourlineindentedshloka
{बस्तानां च गवां चैव शिबिकायास्तथैव च}
{उडुपप्लवसंतारो यत्र नित्यं भविष्यति}
{ययातिरुवाच}
{}


\threelineshloka
{अनो त्वं प्रतिपद्यस्व पाप्मानं जरया सह}
{एकं वर्षसहस्रं तु चरेयं यौवनेन ते ॥अनुरुवाच}
{}


\threelineshloka
{जीर्णः शिशुवदादत्ते कालेऽन्नमशुचिर्यथा}
{न जुहोति च कालेऽग्निं तां जरां नाभिकामये ॥ययातिरुवाच}
{}


\twolineshloka
{यत्त्वं मे हृदयाज्जातो वयः स्वं न प्रयच्छसि}
{जरादोषस्त्वया प्रोक्तस्तस्मात्त्वं प्रतिलप्स्यसे}


\threelineshloka
{प्रजाश्च यौवनप्राप्ता विनशिष्यन्त्यनो तव}
{अग्निप्रस्कन्दनपरस्त्वं चाप्येवं भविष्यसि ॥वैशंपायन उवाच}
{}


\threelineshloka
{प्रत्याख्यातश्चतुर्भिश्च शप्त्वा तान्यदुपूर्वकान्}
{पूरोः सकाशमगमन्मत्त्वा पूरुमलङ्घनम् ॥ययातिरुवाच}
{}


\twolineshloka
{पूरो त्वं मे प्रियः पुत्रस्त्वं वरीयान्भविष्यसि}
{जरा वली च मांतात पलितानि च पर्यगुः}


\threelineshloka
{काव्यस्योशनसः शापान्न च तृप्तोऽस्मि यौवने}
{पूरो त्वं प्रतिपद्यस्व पाप्मानं जरया सह}
{कंचित्कालं चरेयं वै विषयान्वयसातव}


\threelineshloka
{पूर्णे वर्षसहस्रे तु पुनर्दास्यामि यौवनम्}
{स्वं चैव प्रतिपत्स्यामि पाप्मानं जरया सह ॥वैशंपायन उवाच}
{}


\twolineshloka
{एवमुक्तः प्रत्युवाच पूरुः पितरमज्जसा}
{यदात्थ मां महाराज तत्करिष्यामि ते वचः}


\twolineshloka
{`गुरोर्वै वचनं पुण्यं स्वर्ग्यमायुष्करं नृणाम्}
{गुरुप्रसादात्त्रैलोक्यमन्वशासच्छतक्रतुः}


\twolineshloka
{गुरोरनुमतं प्राप्य सर्वान्कामानमाप्नुयात्}
{यावदिच्छसि तावच्च धारयिष्यामि ते जराम्'}


\twolineshloka
{प्रतिपत्स्यामि ते राजन्पाप्मानं जरया सह}
{गृहाण यौवनं मत्तश्चर कामान्यथेप्सितान्}


\threelineshloka
{जरयाहं प्रतिच्छन्नो वयोरूपधरस्तव}
{यौवनं भवते दत्त्वा चरिष्यामि यथात्थमाम् ॥ययातिरुवाच}
{}


\threelineshloka
{पूरो प्रीतोऽस्मि ते वत्स प्रीतश्चेदं ददामि ते}
{सर्वकामसमृद्धा ते प्रजा राज्ये भविष्यति ॥वैशंपायन उवाच}
{}


\twolineshloka
{एवमुक्त्वा ययातिस्तु स्मृत्वा काव्यं महातपाः}
{संक्रामयामास जरां तदा पूरौ महात्मनि}


\chapter{अध्यायः ७९}
\twolineshloka
{वैशंपायन उवाच}
{}


\threelineshloka
{पौरवेणाथ वयसा ययातिर्नहुषात्मजः}
{`रूपयौवनसंपन्नः कुमारः समपद्यत}
{'प्रीतियुक्तो नृपश्रेष्ठश्चरा विषयान्प्रियान्}


\twolineshloka
{यथाकामं यथोत्साहं यथाकालं यथासुखम्}
{धर्माविरुद्धं राजेन्द्रो यथा भवति सोऽन्वभूत्}


\twolineshloka
{देवानतर्पयद्यज्ञैः श्राद्धैस्तद्वित्पितॄनपि}
{दीनाननुग्रहैरिष्टैः कामैश्च द्विजसत्तमान्}


\twolineshloka
{अतिथीनन्नपानैश्च विशश्च परिपालनैः}
{आनृशंस्येन शूद्रांश्च दस्यून्सन्निग्रहेण च}


\twolineshloka
{धर्मेण च प्रजाः सर्वा यथावदनुरञ्जयन्}
{ययातिः पालयामास साक्षादिन्द्र इवापरः}


\twolineshloka
{स राजा सिंहविक्रान्तो युवा विषयगोचरः}
{अविरोधेन धर्मस्य चचार सुखमुत्तमम्}


\twolineshloka
{स संप्राप्य शुभान्कामांस्तृप्तः खिन्नश्च पार्थिवः}
{कालं वर्षसहस्रान्तं सस्मार मनुजाधिपः}


\twolineshloka
{परिसंख्याय कालज्ञः कलाः काष्ठाश्च वीर्यवान्}
{यौवनं प्राप्य राजर्षिः सहस्रपरिवत्सरान्}


\twolineshloka
{विश्वाच्या सहितो रेमे व्यभ्राजन्नन्दने वने}
{अलकायां स कालं तु मेरुशृङ्गे तथोत्तरे}


\twolineshloka
{यदा स पश्यते कालं धर्मात्मा तं महीपतिः}
{पूर्णं मत्वा ततः कालं पूरुं पुत्रमुवाच ह}


\twolineshloka
{यथाकामं यथोत्साहं यथाकालमरिन्दम}
{सेविता विषयाः पुत्र यौवनेन मया तव}


\twolineshloka
{न जातु कामः कामानामुपभोगेन शाम्यति}
{हविषा कृष्णवर्त्मेव भूय एवाभिवर्धते}


\twolineshloka
{यत्पृथिव्यां व्रीहियवं हिरण्यं पशवः स्त्रियः}
{एकस्यापि न पर्याप्तं तस्मान्नृष्णां परित्यजेत्}


\twolineshloka
{या दुस्त्यजा दुर्मतिभिर्या न जीर्यति जीर्यतः}
{योऽसौ प्राणान्तिको रोगस्तांतृष्णां त्यजतः सुखम्}


\twolineshloka
{पूर्णं वर्षसहस्रं मे विषयासक्तचेतसः}
{तथाप्यनुदिनं तृष्णा ममैतेष्वभिजायते}


\twolineshloka
{तस्मादेनामहं त्यक्त्वा ब्रह्मण्याधाय मानसम्}
{निर्द्वन्द्वो निर्ममो भूत्वा चरिष्यामि मृगैः सह}


\fourlineindentedshloka
{पूरो प्रीतोऽस्मि भद्रं ते गृहाणेदं स्वयौवनम्}
{राज्यं चेदं गृहाण त्वं `यावदिच्छसि यौवनम्}
{तावद्दीर्घायुषा भुङ्ख' त्वं हि मे प्रियकृत्सुतः ॥वैशंपायन उवाच}
{'}


\twolineshloka
{प्रतिपेदे जरां राजा ययातिर्नाहुषस्तदा}
{यौवनं प्रतिपेदे च पूरुः स्वं पुनरात्मवान्}


\twolineshloka
{अभिषेक्तुकामं नृपतिं पूरुं पुत्रं कनीयसम्}
{ब्राह्मणप्रमुखा वर्णा इदं वचनमब्रुवन्}


\twolineshloka
{कथं शुक्रस्य नप्तारं देवयान्याः सुतं प्रभो}
{ज्येष्ठं यदुमतिक्रम्य राज्यं पूरोः प्रयच्छसि}


\twolineshloka
{यदुर्ज्येष्ठस्तव सुतो जातस्तमनु तुर्वसुः}
{शर्मिष्ठायाः सुतो द्रुह्युस्ततोऽनुः पूरुरेव च}


\threelineshloka
{कथं ज्येष्ठानतिक्रम्य कनीयान्राज्यमर्हति}
{एतत्संबोधयामस्त्वां धर्मं त्वं प्रतिपालय ॥ययातिरुवाच}
{}


\twolineshloka
{ब्राह्मणप्रमुखा वर्णाः सर्वे शृण्वन्तु मे वचः}
{ज्येष्ठं प्रति यथा राज्यं न देयं मे कथंचन}


\twolineshloka
{मम ज्येष्ठेन यदुना नियोगो नानुपालितः}
{प्रतिकूलः पितुर्यश्च न स पुत्रः सतां मतः}


\twolineshloka
{मातापित्रोर्वचनकृद्धितः पथ्यश्च यः सुतः}
{स पुत्रः पुत्रवद्यश्च वर्तते पितृमातृषु}


\twolineshloka
{`पुदिति नरकस्याख्या दुःखं च नरकं विदुः}
{पुतस्त्राणात्ततः पुत्त्रमिहेच्छन्ति परत्र च}


\twolineshloka
{आत्मनः सदृशः पुत्रः पितृदेवर्षिपूजने}
{यो बहूनां गुणकरः स पुत्रो ज्येष्ठ उच्यते}


\twolineshloka
{मूकोऽन्धो बधिरः श्वित्री स्वधर्मं नानुतिष्ठति}
{चोरः किल्बिषिकः पुत्रो ज्येष्ठो न ज्येष्ठ उच्यते}


\threelineshloka
{ज्येष्ठांशहारी गुणकृदिह लोके परत्र च}
{श्रेयान्पुत्रो गुणोपेतः स पुत्रो नेतरो वृथा}
{वदन्ति धर्मं धर्मज्ञाः पितॄणां पुत्रकारणात्}


\twolineshloka
{वेदोक्तं संभवं मह्यमनेन हृदयोद्भवम्}
{तस्य जातमिदं कृत्स्नमात्मा पुत्र इति श्रुतिः'}


\twolineshloka
{यदुनाऽहमवज्ञातस्तथा तुर्वसुनापि च}
{द्रुह्युना चानुना चैव मय्यवज्ञ कृता भृशम्}


\twolineshloka
{पूरुणा तु कृतं वाक्यं मानितं च विशेषतः}
{कनीयान्मम दायादो धृता येन जरा मम}


\twolineshloka
{मम कामः स च कृतः पूरुणा मित्ररूपिणा}
{शुक्रेण च वरोदत्तः काव्येनोशनसा स्वयम्}


\twolineshloka
{पुत्रो यस्त्वाऽनुवर्तेत स राजा पृथिवीपतिः}
{`यो वानुवर्ती पुत्राणां स पुत्रो दायभाग्भवेत्'}


\threelineshloka
{भवतोऽनुनयाम्येवं पूरू राज्येऽभिषिच्यताम्}
{प्रकृतय ऊचुः}
{यः पुत्रो गुणसंपन्नो मातापित्रोर्हितः सदा}


\twolineshloka
{सर्वमर्हति कल्याणं कनीयानपि सत्तमः}
{`वेद धमार्थशास्त्रेषु मुनिभिः कथितं पुरा'}


\threelineshloka
{अर्हः पूरुरिदं राज्यं यः सुतः प्रियकृत्तव}
{वरदानेन शुक्रस्य न शक्यं वक्तुमुत्तरम् ॥वैशंपायन उवाच}
{}


\twolineshloka
{पौरजानपदैस्तुष्टैरित्युक्तो नाहुषस्तदा}
{अभ्यषिञ्चत्ततः पूरुं राज्ये स्वे सुतमात्मनः}


\twolineshloka
{`यदुं च तुर्वसुं चोभौ द्रुह्युं चैव सहानुजम्}
{अन्तेषु स विनिक्षिप्य नाहुषः स्वात्मजान्सुतान्'}


\twolineshloka
{दत्त्वा च पूरवे राज्यं वनवासाय दीक्षितः}
{पुरात्स निर्ययौ राजा ब्राह्मणैस्तापसैः सह}


\twolineshloka
{`देवयान्या च सहितः शर्मिष्ठया च भारत}
{अकरोत्स वने राजा सभार्यस्तप उत्तमम्'}


\twolineshloka
{यदोस्तु यादवा जातास्तुर्वसोर्यवनाः स्मृताः}
{द्रुह्योः सुतास्तु वै भोजा अनोस्तु म्लेच्छजातयः}


\twolineshloka
{पूरोस्तु पौरवो वंशो यत्र जातोऽसि पार्थिव}
{इदं वर्षसहस्राणि राज्यं कारयितुं वशी}


\chapter{अध्यायः ८०}
\twolineshloka
{वैशंपायन उवाच}
{}


\twolineshloka
{एवं स नाहुषो राजा ययातिः पुत्रमीप्सितम्}
{राज्येऽभिषिच्य मुदितो वानप्रस्थोऽभवन्मुनिः}


\twolineshloka
{उषित्वा च वने वासं ब्राह्मणैः संशितव्रतः}
{फलमूलाशनो दान्तस्ततः स्वर्गमितो गतः}


\twolineshloka
{स गतः स्वर्निवासं तं निवसन्मुदितः सुखी}
{कालेन नातिमहता पुनः शक्रेण पातितः}


\fourlineindentedshloka
{`साधुभिः संगतिं लब्ध्वा पुनः स्वर्गमुपेयिवान्}
{जनमेजय उवाच}
{स्वर्गतश्च पुनर्ब्रह्मन्निवसन्देववेश्मनि}
{कालेन नातिमहता कथं शक्रेण पातितः'}


\twolineshloka
{निपतन्प्रच्युतः स्वर्गादप्राप्तो मेदिनीतलम्}
{स्थित आसीदन्तरिक्षे स तदेति श्रुतं मया}


\twolineshloka
{तत एव पुनश्चापि गतः स्वर्गमिति श्रुतम्}
{राज्ञा वसुमता सार्धमष्टकेन च वीर्यवान्}


\twolineshloka
{प्रतर्दनेन शिविना समेत्य किल संसदि}
{कर्मणा केन स दिवं पुनः प्राप्तो महीपतिः}


\twolineshloka
{सर्वमेतदशेषेण श्रोतुमिच्छामि तत्त्वतः}
{कथ्यमानं त्वया विप्र विप्रर्षिगणसंनिधौ}


\twolineshloka
{देवराजसमो ह्यासीद्ययातिः पृथिवीपतिः}
{वर्धनः कुरुवंशस्य विभावसुसमद्युतिः}


\threelineshloka
{तस्य विस्तीर्णयशसः सत्यकीर्तेर्महात्मनः}
{चरितं श्रोतुमिच्छामि दिवि चेह च सर्वशः ॥वैशंपायन उवाच}
{}


\twolineshloka
{हन्त ते कथयिष्यामि ययातेरुत्तरां कथाम्}
{दिवि चेह च पुण्यार्थां सर्वपापप्रणाशिनीम्}


\twolineshloka
{ययातिर्नाहुषो राजा पूरुं पुत्रं कनीयसम्}
{राज्येऽभिषिच्य मुदितः प्रावव्राज वनं तदा}


\twolineshloka
{अन्त्युषे स विनिक्षिप्य पुत्रान्यदुपुरोगमान्}
{फलमूलाशनो राजा वने संन्यवसच्चिरम्}


\twolineshloka
{शंसितात्मा जितक्रोधस्तर्पयन्पितृदेवताः}
{अग्नींश्च विधिवज्जुह्वन्वानप्रस्थविधानतः}


\twolineshloka
{अथितीन्पूजयामास वन्येन हविषा विभुः}
{शिलोञ्छवृत्तिमास्थाय शेषान्नकृतभोजनः}


\twolineshloka
{पूर्णं वर्षसहस्रं च एवंवृत्तिरभून्नृपः}
{अब्भक्षः शरदस्त्रिंशदासीन्नियतवाङ्मनाः}


\twolineshloka
{ततश्च वायुभक्षोऽभूत्संवत्सरमतन्द्रितः}
{तथा पञ्चाग्निमध्ये च तपस्तेपे स वत्सरम्}


\twolineshloka
{एकपादः स्तितश्चासीत्षण्मासाननिलाशनः}
{पुण्यकीर्तिस्ततः स्वर्गे जगामावृत्य रोदसी}


\chapter{अध्यायः ८१}
\twolineshloka
{वैशंपायन उवाच}
{}


\twolineshloka
{स्वर्गतः स तु राजेन्द्रो निवसन्देववेश्मनि}
{पूजितस्त्रिदशैः साध्यैर्मरुद्भिर्वसुभिस्तथा}


\twolineshloka
{देवलोकं ब्रह्मलोकं संचरन्पुण्यकृद्वशी}
{अवसत्पृथिवीपालो दीर्घकालमिति श्रुतिः}


\threelineshloka
{स कदाचिन्नृपश्रेष्ठो ययातिः शक्रमागमत्}
{कथान्ते तत्र शक्रेण स पृष्टः पृथिवीपतिः ॥शक्र उवाच}
{}


\threelineshloka
{यदा स पूरुस्तव रूपेण राज-ञ्जरां गृहीत्वा प्रचचार भूमौ}
{तदा च राज्यं संप्रदायैव तस्मैत्वया किमुक्तः कथयेह सत्यम् ॥ययातिरुवाच}
{}


\twolineshloka
{गङ्गायमुनयोर्मध्ये कृत्स्नोयं विषयस्तव}
{मध्ये पृथिव्यास्त्वं राजा भ्रातरोऽन्त्याधिपास्तव}


\twolineshloka
{`न च कुर्यान्नरो दैन्यं शाठ्यं क्रोधं तथैव च}
{जैहयं च मत्सरं वैरं सर्वत्रेदं न कारयेत्}


\twolineshloka
{मातरं पितरं ज्येष्ठं विद्वांसं च तपोधनम्}
{क्षमावन्तं च राजेन्द्र नावमन्येत बुद्धिमान्}


\twolineshloka
{ळशक्तस्तु क्षमते नित्यमशक्तः क्रुध्यते नरः}
{दुर्जनः सुजनं द्वेष्टि दुर्बलो बलवत्तरम्}


\twolineshloka
{रूपवन्तमरूपी च धनवन्तं च निर्धनः}
{अकर्मी कर्मिणं द्वेष्टि धार्मिकं च नधार्मिकः}


\twolineshloka
{निर्गुणो गुणवन्तं च पुत्रैतत्कलिलक्षणम्}
{विपरीतं च राजेन्द्र एतेषु कृतलक्षणम्}


\twolineshloka
{ब्राह्मणो वाथ वा राजा वैश्यो वा शूद्र एव वा}
{प्रशस्तेषु प्रसक्ताश्चेत्प्रशस्यन्ते यशस्विनः}


\twolineshloka
{तस्मात्प्रशस्ते राजेन्द्र नरः सक्तमना भवेत्}
{अलोकज्ञा ह्यप्रशस्ता भ्रातरस्ते ह्यबुद्धयः}


\fourlineindentedshloka
{अन्त्याधिपतयः सर्वे ह्यभवन्गुरुशासनात्}
{इन्द्र उवाच}
{त्वं हि धर्मविदो राजन्कत्थसे धर्मसुत्तमम्}
{कथयस्व पुनर्मेऽद्य लोकवृत्तान्तमुत्तमम् ॥ययातिरुवाच'}


\twolineshloka
{अक्रोधनः क्रोधनेभ्यो विशिष्ट-स्तथा तितिक्षुरतितिक्षोर्विशिष्टः}
{अमानुषेभ्यो मानुषाश्च प्रधानाविद्वांस्तथैवाविदुषः प्रधानः}


\twolineshloka
{आक्रुश्यमानो नाकोशेन्मन्युरेव तितिक्षतः}
{आक्रोष्टारं निर्दहति सुकृतं चास्य विन्दति}


\twolineshloka
{नारुन्तुदः स्यान्न नृशंसवादीन हीनतः परमभ्याददीत}
{ययाऽस्य वाचा पर उद्विजेतन तां वदेद्रुशतीं पापलोक्याम्}


\twolineshloka
{अरुन्तुदं पुरुषं तीक्ष्णवाचंवाक्कण्टकैर्वितुदन्तं मनुष्यन्}
{विद्यादलक्ष्मीकतमं जनानांमुखे निबद्धां निर्ऋतिं वहन्तम्}


\twolineshloka
{सद्भिः पुरस्तादभिपूजितः स्या-त्सद्भिस्तथा पृष्ठतो रक्षितः स्यात्}
{सदाऽसतामतिवादांस्तितिक्षे-त्सतां वृत्तं चाददीतार्यवृत्तः}


\twolineshloka
{वाक्सायका वदनान्निष्पतन्तियैराहतः शोचति रात्र्यहानि}
{परस्य ये मर्मसु संपतन्तितान्पण्डितो नावसृजेत्परेषु}


\twolineshloka
{नहीदृशं संवननं त्रिषु लोकेषु विद्यते}
{दया मैत्री च भूतेषु दानं च मधुरा च वाक्}


\twolineshloka
{तस्मात्सान्त्वं सदा वाच्यं न वाच्यं परुषं क्वचित्}
{पूज्यान्संपूजयेद्दद्यान्न च याचेत्कदाचन}


\chapter{अध्यायः ८२}
\twolineshloka
{इन्द्र उवाच}
{}


\threelineshloka
{सर्वाणि कर्माणि समाप्य राजन्गृहं परित्यज्य वनं गतोऽसि}
{तत्त्वां पृच्छामि नहुषस्य पुत्रकेनासि तुल्यस्तपसा ययाते ॥ययातिरुवाच}
{}


\threelineshloka
{नाहं देवमनुष्येषु गन्धर्वेषु महर्षिषु}
{आत्मनस्तपसा तुल्यं कंचित्पश्यामि वासव ॥इन्द्र उवाच}
{}


\threelineshloka
{यदाऽवमंस्थाः सदृशः श्रेयसश्चअल्पीयसश्चाविदितप्रभावः}
{तस्माल्लोकास्त्वन्तवन्तस्तवे मेक्षीणे पुण्ये पतिताऽस्यद्य राजन् ॥ययातिरुवाच}
{}


\threelineshloka
{सुरर्षिगन्धर्वनरावमाना-त्क्षयं गता मे यदि शक्रलोकाः}
{इच्छाम्यहं सुरलोकाद्विहीनःसतां मध्ये पतितुं देवराज ॥इन्द्र उवाच}
{}


\threelineshloka
{सतां सकाशे पतिताऽसि राजं-श्च्युतः प्रतिष्ठां यत्र लब्धासि भूयः}
{एतद्विदित्वा च पुनर्ययातेत्वं माऽवमंस्थाः सदृशः श्रेयसश्च ॥वैशंपायन उवाच}
{}


\threelineshloka
{ततः प्रहायामरराजजुष्टा-न्पुण्याँल्लोकान्पतमानं ययातिम्}
{संप्रेक्ष्य राजर्षिवरोऽष्टकस्त-मुवाच सद्धर्मविधानगोप्ता ॥अष्टक उवाच}
{}


\twolineshloka
{कस्त्वं युवा वासवतुल्यरूपःस्वतेजसा दीप्यमानो यथाऽग्निः}
{पतस्युदीर्णाम्बुधरान्धकारा-त्खात्खेचराणां प्रवरो यथाऽर्कः}


\twolineshloka
{दृष्ट्वा च त्वां सूर्यपथात्पतन्तंवैश्वानरार्कद्युतिमप्रमेयम्}
{किं नु स्विदेतत्पततीति सर्वेवितर्कयन्तः परिमोहिताः स्मः}


\twolineshloka
{दृष्ट्वा च त्वां धिष्ठितं देवमार्गेशक्रार्कविष्णुप्रतिमप्रभावम्}
{अभ्युद्गतास्त्वां वयमद्य सर्वेतत्त्वं प्रपाते तव जिज्ञासमानाः}


\twolineshloka
{न चापि त्वां धृष्णुमः प्रष्टुमग्रेन च त्वमस्मान्पृच्छसि ये वयं स्मः}
{तत्त्वां पृच्छामि स्पृहणीयरूपकस्य त्वं वा किंनिमित्तं त्वमागाः}


\twolineshloka
{भयं तु ते व्येतु विषादमोहौत्यजाशु चैवेन्द्रसमप्रभाव}
{त्वां वर्तमानं हि सतां सकाशेनालं प्रसोढुं बलहाऽपि शक्रः}


\twolineshloka
{सन्तः प्रतिष्ठा हि सुखच्युतानांसतां सदैवामरराजकल्प}
{ते सङ्गताः स्थावरजङ्गमेशाःप्रतिष्ठितस्त्वं सदृशेषु सत्सु}


\twolineshloka
{प्रभुरग्निः प्रतपने भूमिरावपने प्रभुः}
{प्रभुः सूर्यः प्रकाशित्वे सतां चाभ्यागतः प्रभुः}


\chapter{अध्यायः ८३}
\twolineshloka
{ययातिरुवाच}
{}


\twolineshloka
{अहं ययातिर्नहुषस्य पुत्रःपूरोः पिता सर्वभूतावमानात्}
{प्रभ्रंशितः सुरसिद्धर्षिलोका-त्परिच्युतः प्रपताम्यल्पपुण्यः}


\threelineshloka
{अहं हि पूर्वो वयसा भवद्भ्य-स्तेनाभिवादं भवतां न प्रयुञ्जे}
{यो विद्यया तपसा जन्मना वावृद्धः स पूज्यो भवति द्विजानाम् ॥अष्टक उवाच}
{}


\threelineshloka
{अवादीस्त्वं वयसा यः प्रवृद्धःस वै राजन्नाभ्यधिकः कथ्यते च}
{यो विद्यया तपसा संप्रवृद्धःस एव पूज्यो भवति द्विजानाम् ॥ययातिरुवाच}
{}


\twolineshloka
{प्रतिकूलं कर्मणां पापमाहु-स्तद्वर्ततेऽप्रवणे पापलोक्यम्}
{सन्तोऽसतां नानुवर्तन्ति चैत-द्यथा चैषामनुकूलास्तथाऽऽसन्}


\twolineshloka
{अभूद्धनं मे विपुलं गतं त-द्विचेष्टमानो नाधिगन्ता तदस्मि}
{एवं प्रधार्यात्महिते निविष्टोयो वर्तते स विजानाति जीवः}


\twolineshloka
{महाधनो यो यजते सुयज्ञै-र्यः सर्वविद्यासु विनीतबुद्धिः}
{वेदानधीत्य तपसा योज्य देहंदिवं स यायात्पुरुषो वीतमोहः}


\threelineshloka
{न जातु हृष्येन्महता धनेनवेदानधीयीतानहंकृतः स्यात्}
{नानाभावा बहवो जीवलोकेदैवाधीना नष्टचेष्टाधिकाराः}
{तत्तत्प्राप्य न विहन्येत धीरोदिष्टं बलीय इति मत्वाऽऽत्मबुद्ध्या}


\twolineshloka
{सुखं हि जन्तुर्यदि वाऽपि दुःखंदैवाधीनं विन्दते नात्मशक्त्या}
{तस्माद्दिष्टं बलवन्मन्यमानोन संज्वरेन्नापि हृष्येत्कथंचित्}


\twolineshloka
{दुःखैर्न तप्येन्न सुखैः प्रहृष्ये-त्समेन वर्तेत सदैव धीरः}
{दिष्टं बलीय इति मन्यमानोन संज्वरेन्नापि हृष्येत्कथंचित्}


\twolineshloka
{भये न मुह्याम्यष्टकाहं कदाचि-त्सन्तापो मे मानसो नास्ति कश्चित्}
{धाता यथा मां विदधीत लोकेध्रुवं तथाऽहं भवितेति मत्वा}


\twolineshloka
{संस्वेदजा अण्डजाश्चोद्भिदश्चसरीसृपाः कृमयोऽथाप्सु मत्स्याः}
{तथाश्मनस्तृणकाष्ठं च सर्वेदिष्टक्षये स्वां प्रकृतिं भजन्ति}


\threelineshloka
{अनित्यतां सुखदुःस्वस्य बुद्ध्वाकस्मात्संतापमष्टकाहं भजेयम्}
{किं कुर्यां वै किं च कृत्वा न तप्येतस्मात्सन्तापं वर्जयाम्यप्रमत्तः ॥वैशंपायन उवाच}
{}


\threelineshloka
{एवं व्रुवाणं नृपतिं ययाति-मथाष्टकः पुनरेवान्वपृच्छत्}
{मातामहं सर्वगुणोपपन्नंतत्रस्थितं स्वर्गलोके यथावत् ॥अष्टक उवाच}
{}


\threelineshloka
{ये ये लोकाः पार्थिवेन्द्रप्रधाना-स्त्वया भुक्ता यं च कालं यथावत्}
{तान्मे राजन्ब्रूहि सर्वान्यथाव-त्क्षेत्रज्ञवद्भाषसे त्वं हि धर्मान् ॥ययातिरुवाच}
{}


\twolineshloka
{राजाऽहमासमिह सार्वभौम-स्ततो लोकान्महतश्चाजयं वै}
{तत्रावसं वर्षसहस्रमात्रंततो लोकं परमस्म्यभ्युपेतः}


\twolineshloka
{ततः पुरीं पुरुहूतस्य रम्यांसहस्रद्वारां शतयोजनायताम्}
{अध्यावसं वर्षसहस्रमात्रंततो लोकं परमस्म्यभ्युपेतः}


\twolineshloka
{ततो दिव्यमजरं प्राप्य लोकंप्रजापतेर्लोकपतेर्दुरापम्}
{तत्रावसं वर्षसहस्रमात्रंततो लोकं परमस्म्यभ्युपेतः}


\twolineshloka
{स देवदेवस्य निवेशने चविहृत्य लोकानवसं यथेष्टम्}
{संपूज्यमानस्त्रिदशैः समस्तै-स्तुल्यप्रभावद्युतिरीश्वराणाम्}


\twolineshloka
{तथाऽऽवसं नन्दने कामरूपीसंवत्सराणामयुतं शतानाम्}
{सहाप्सरोभिर्विहरन्पुण्यगन्धा-न्पश्यन्नगान्पुष्पितांश्चारुरूपान्}


\twolineshloka
{तत्र स्थितं मां देव सुखेषु सक्तंकालेऽतीते महति ततोऽतिमात्रम्}
{दूतो देवानामब्रवीदुग्ररूपोध्वंसेत्युच्चैस्त्रिः प्लुतेन स्वरेण}


\twolineshloka
{एतावन्मे विदितं राजसिंहततो भ्रष्टोऽहं नन्दनात्क्षीणपुण्यः}
{वाचोऽश्रौषं चान्तरिक्षे सुराणांसानुक्रोंशाः शोचतां मां नरेन्द्र}


\twolineshloka
{अहो कष्टं क्षीणपुण्यो ययातिःपतत्यसौ पुण्यकृत्पुण्यकीर्तिः}
{तानब्रुवं पतमानस्ततोऽहंसतां मध्ये निपतेयं कथं नु}


\twolineshloka
{तैराख्याता भवतां यज्ञभूमिःसमीक्ष्य चेमां त्वरितमुपागतोऽस्मि}
{हविर्गन्धं देशिकं यज्ञभूमे-र्धूमापाङ्गं प्रतिगृह्य प्रतीतः}


\chapter{अध्यायः ८४}
\twolineshloka
{अष्टक उवाच}
{}


\threelineshloka
{यदाऽवसो नन्दने कामरूपीसंवत्सराणामयुतं शतानाम्}
{किं कारणं कार्तयुगप्रधानहित्वा च त्वं वसुधामन्वपद्यः ॥ययातिरुवाच}
{}


\threelineshloka
{ज्ञातिः सुहृत्स्वजनो वा यथेहक्षीणे वित्ते त्यज्यते मानवैर्हि}
{तथा तत्र क्षीणपुण्यं मनुष्यंत्यजन्ति सद्यः सेश्वरा देवसङ्घाः ॥अष्टक उवाच}
{}


\threelineshloka
{तस्मिन्कथं क्षीणपुण्या भवन्तिसंमुह्यते मेऽत्र मनोऽतिमात्रम्}
{किं वा विशिष्टाः कस्य धामोपयान्तितद्वै ब्रूहि क्षेत्रवित्त्वं मतो मे ॥ययातिरुवाच}
{}


\twolineshloka
{इमं भौमं नरकं ते पतन्तिललाप्यमाना नरदेव सर्वे}
{ते कङ्कगोमायुबलाशनार्थेक्षीणे पुण्ये बहुधा प्रव्रजन्ति}


\threelineshloka
{तस्मादेतद्वर्जनीयं नरेन्द्रदुष्टं लोके गर्हणीयं च कर्म}
{आख्यातं ते पार्थिव सर्वमेवभूयश्चेदानीं वद किं ते वदामि ॥अष्टक उवाच}
{}


\threelineshloka
{यदा तु तान्वितुदन्ते वयांसितथा गृध्राः शितिकण्ठाः पतङ्गाः}
{कथं भवन्ति कथमाभवन्तिन भौममन्यं नरकं शृणोमि ॥ययातिरुवाच}
{}


\twolineshloka
{ऊर्ध्वं देहात्कर्मणो जृम्भमाणा-द्व्यक्तं पृथिव्यामनुसंचरन्ति}
{इमं भौमं नरकं ते पतन्तिनावेक्षन्ते वर्षपूगाननेकान्}


\threelineshloka
{षष्टिं सहस्राणि पतन्ति व्योम्नितथा अशीतिं परिवत्सराणि}
{तान्वै तुदन्ति पततः प्रपातंभीमा भौमा राक्षसास्तीक्ष्णदंष्ट्राः ॥अष्टक उवाच}
{}


\threelineshloka
{यदेनसस्ते पततस्तुदन्तिभीमा भौमा राक्षसास्तीक्ष्णदंष्ट्राः}
{कथं भवन्ति कथमाभवन्तिकथंभूता गर्भभूता भवन्ति ॥ययातिरुवाच}
{}


\twolineshloka
{अस्रं रेतः पुष्पफलानुपृक्त-मन्वेति तद्वै पुरुषेण सृष्टम्}
{स वै तस्या रज आपद्यते वैस गर्भभूतः समुपैति तत्र}


\threelineshloka
{वनस्पतीनोषधीश्चाविशन्तिआपो वायुं पृथिवीं चान्तरिक्षम्}
{चतुष्पदं द्विपदं चाति सर्व-मेवंभूता गर्भभूता भवन्ति ॥अष्टक उवाच}
{}


\twolineshloka
{अन्यद्वपुर्विदधातीह गर्भ-मुताहोस्वित्स्वेन कायेन याति}
{आपद्यमानो नरयोनिमेता-माचक्ष्व मे संशयात्प्रब्रवीमि}


\threelineshloka
{शरीरदेहातिसमुच्छ्रयं चचक्षुःश्रोत्रे लभते केन संज्ञाम्}
{एतत्तत्त्वं सर्वमाचक्ष्व पृष्टःक्षेत्रज्ञं त्वां तात मन्याम सर्वे ॥ययातिरुवाच}
{}


\twolineshloka
{वायुः समुत्कर्षति गर्भयोनि-मृतौ रेतः पुष्पफलानुपृक्तम्}
{स तत्र तन्मात्रकृताधिकारःक्रमेण संवर्धयतीह गर्भम्}


\twolineshloka
{स जायमानो विगृहीतमात्रःसंज्ञामधिष्ठाय ततो मनुष्यः}
{स श्रोत्राभ्यां वेदयतीह शब्दंस वै रूपं पश्यति चक्षुषा च}


\threelineshloka
{घ्राणेन गन्धं जिह्वयाऽथो रसं चत्वचा स्पर्शं मनसा वेदभावम्}
{इत्यष्टकेहोपहितं हि विद्धिमहात्मनः प्राणभृतः शरीरे ॥अष्टक उवाच}
{}


\threelineshloka
{यः संस्थितः पुरुषो दह्यते वानिखन्यते वापि निकृष्यते वा}
{अभावभूतः स विनाशमेत्यकेनात्मानं चेतयते परस्तात् ॥ययातिरुवाच}
{}


\twolineshloka
{हित्वा सोऽसून्सुप्तवन्निष्टनित्वापुरोधाय सुकृतं दुष्कृतं वा}
{अन्यां योनिं पवनाग्रानुसारीहित्वा देहं भजते राजसिंह}


\twolineshloka
{पुण्यां योनिं पुण्यकृतो व्रजन्तिपापां योनिं पापकृतो व्रजन्ति}
{कीटाः पतङ्गाश्च भवन्ति पापान मे विवक्षास्ति महानुभाव}


\threelineshloka
{चतुष्पदा द्विपदाः षट्पदाश्चतथाभूता गर्भभूता भवन्ति}
{आख्यातमेतन्निखिलेन सर्वंभूयस्तु किं पृच्छसि राजसिंह ॥अष्टक उवाच}
{}


\threelineshloka
{किंस्वित्कृत्वा लभते तात लोका-न्मर्त्यः श्रेष्ठांस्तपसा विद्यया च}
{तन्मे पृष्टः शंस सर्वं यथाव-च्छुभाँल्लोकान्येन गच्छेत्क्रमेण ॥ययातिरुवाच}
{}


\threelineshloka
{तपश्च दानं च शमो दमश्चह्रीरार्जवं सर्वभूतानुकम्पा}
{स्वर्गस्य लोकस्य वदन्ति सन्तोद्वाराणि सप्तैव महान्ति पुंसाम्}
{नश्यन्ति मानेन तमोऽभिभूताःपुंसः सदैवेति वदन्ति सन्तः}


\twolineshloka
{अधीयानः पण्डितंमन्यमानोयो विद्यया हन्ति यशः परेषाम्}
{तस्यान्तवन्तश्च भवन्ति लोकान चास्य तद्ब्रह्म फलं ददाति}


\twolineshloka
{चत्वारि कर्माण्यभयङ्कराणिभयं प्रयच्छन्त्ययथाकृतानि}
{मानाग्निहोत्रमुत मानमौनंमानेनाधीतमुत मानयज्ञः}


\twolineshloka
{न मानमान्यो मुदमाददीतन सन्तापं प्राप्नुयाच्चावमानात्}
{सन्तः सतः पूजयन्तीह लोकेनासाधवः साधुबुद्धिं लभन्ते}


\twolineshloka
{इति दद्यामिति यज इत्यदीय इति व्रतम्}
{इत्येतानि भयान्याहुस्तानि वर्ज्यानि सर्वशः}


\twolineshloka
{ये चाश्रयं वेदयन्ते पुराणंमनीषिणो मानसमार्गरुद्धम्}
{तन्निःश्रेयस्तेन संयोगमेत्यपरां शान्तिं प्रत्युः प्रेत्य चेह}


\chapter{अध्यायः ८५}
\twolineshloka
{अष्टक उवाच}
{}


\threelineshloka
{चरन्गृहस्थः कथमेति धर्मा-न्कथं भिक्षुः कथमाचार्यकर्मा}
{वानप्रस्थः सत्पथे सन्निविष्टोबहून्यस्मिन्संप्रति वेदयन्ति ॥ययातिरुवाच}
{}


\twolineshloka
{आहूताध्यायी गुरुकर्मस्वचोद्यःपूर्वोत्थायी चरमं चोपशायी}
{मृदुर्दान्तो धृतिमानप्रमत्तःस्वाध्याशीलः सिध्यति ब्रह्मचारी}


\twolineshloka
{धर्मागतं प्राप्य धनं यजेतदद्यात्सदैवातिथीन्भोजयेच्च}
{अनाददानश्च परैरदत्तंसैषा गृहस्थोपनिषत्पुराणी}


\twolineshloka
{स्ववीर्यजीवी वृजिनान्निवृत्तोदाता परेभ्यो न परोपतापी}
{तादृङ्मुनिः सिद्धिमुपैति मुख्यांवसन्नरण्ये नियताहारचेष्टः}


\twolineshloka
{अशिल्पजीवी गुणवांश्चैव नित्यंजितेन्द्रियः सर्वतो विप्रयुक्तः}
{अनोकशायी लघुरल्पप्रसार-श्चरन्देशानेकचरः स भिक्षुः}


\twolineshloka
{रात्र्या यया वाऽभिजिताश्च लोकाभवन्ति कामाभिजिताः सुखाश्च}
{तामेव रात्रिं प्रयतेत विद्वा-नरण्यसंस्थो भवितुं यतात्मा}


\threelineshloka
{दशैव पूर्वान्दश चापरांश्चज्ञातीनथात्मानमथैकविंशम्}
{अरण्यवासी सुकृते दधातिविमुच्यारण्ये स्वशरीरधातून् ॥अष्टक उवाच}
{}


\threelineshloka
{कतिस्विदेव मुनयः कति मौनानि चाप्युत}
{भवन्तीति तदाचक्ष्व श्रोतुमिच्छामहे वयम् ॥ययातिरुवाच}
{}


\threelineshloka
{अरण्ये वसतो यस्य ग्रामो भवति पृष्ठतः}
{ग्रामे वा वसतोऽरण्यं स मुनिः स्याज्जनाधिप ॥अष्टक उवाच}
{}


\threelineshloka
{कथंस्विद्वसतोऽरण्ये ग्रामो भवति पृष्ठतः}
{ग्रामे वा वसतोऽरण्यं कथं भवति पृष्ठतः ॥ययातिरुवाच}
{}


\twolineshloka
{न ग्राम्यमुपयुञ्जीत य आरण्यो मुनिर्भवेत्}
{तथास्य वसतोऽरण्ये ग्रामो भवति पृष्ठतः}


\twolineshloka
{अनग्निरनिकेतश्चाप्यगोत्रचरणो मुनिः}
{कौपीनाच्छादनं यावत्तावदिच्छेच्च चीवरम्}


\twolineshloka
{यावत्प्राणाभिसन्धानं तावदिच्छेच्च भोजनम्}
{तथाऽस्य वसतो ग्रामेऽरण्यं भवति पृष्ठतः}


\twolineshloka
{यस्तु कामान्परित्यज्य त्यक्तकर्मा जितेन्द्रियः}
{आतिष्ठेच्च मुनिर्मौनं स लोके सिद्धिमाप्नुयात्}


\twolineshloka
{धौतदन्तं कृत्तनखं सदा स्नातमलङ्कृतम्}
{असितं सितकर्माणं कस्तमर्हति नार्चितुम्}


\twolineshloka
{तपसा कर्शितः क्षामः क्षीणमांसास्थिशोणितः}
{स च लोकमिमं जित्वा लोकं विजयते परम्}


\twolineshloka
{यदा भवति निर्द्वन्द्वो मुनिर्मौनं समास्थितः}
{अथ लोकमिमं जित्वा लोकं विजयते परम्}


\twolineshloka
{आस्येन तु यदाऽहारं गोवन्मृगयते मुनिः}
{अथास्य लोकः सर्वोऽयं सोऽमृतत्वाय कल्पते}


\fourlineindentedshloka
{सामान्यधर्मः सर्वेषां क्रोधो लोभो द्रुहाऽक्षमा}
{विहाय मत्सरं शाठ्यं दर्पं दम्भं च पैशुनम्}
{क्रोधं लोभं ममत्वं च यस्य नास्ति स धर्मवित् ॥अष्टक उवाच}
{}


\threelineshloka
{नित्याशनो ब्रह्मचारी गृहस्थो वनगो मुनिः}
{नाधर्ममशनात्प्राप्येत्कथं ब्रूहीह पृच्छते ॥ययातिरुवाच}
{}


\twolineshloka
{अष्टौ ग्रासा मुनेः प्रोक्ताः षोडशारण्यवासिनः}
{द्वात्रिंशत्तु गृहस्थस्य अमितं ब्रह्मचारिणः}


% Check verse!
इत्येवं कारणैर्ज्ञेयमष्टकैतच्छुभाशुभम्
\chapter{अध्यायः ८६}
\twolineshloka
{अष्टक उवाच}
{}


\threelineshloka
{कतरस्त्वनयोः पूर्वं देवानामेति साम्यताम्}
{उभयोर्धावतो राजन्सूर्याचन्द्रमसोरिव ॥ययातिरुवाच}
{}


\twolineshloka
{अनिकेतो गृहस्थेषु कामवृत्तेषु संयतः}
{ग्राम एव वसन्भिक्षुस्तयोः पूर्वतरं गतः}


\twolineshloka
{अप्राप्य दीर्घमायुस्तु यः प्राप्तो विकृतिं चरेत्}
{तप्यते यदि तत्कृत्वा चरेत्सोऽन्यत्तपस्ततः}


\twolineshloka
{पापानां कर्मणां नित्यं बिभीयाद्यस्तु मानवः}
{सुखमप्याचरन्नित्यं सोऽत्यन्तं सुखमेधते}


\threelineshloka
{यद्वै नृशंसं तदसत्यमाहु-र्यः सेवते धर्ममनर्थबुद्धिः}
{अस्वोऽप्यनीशश्च तथैव राजं-स्तदार्जवं स समाधिस्तदार्यम् ॥अष्टक उवाच}
{}


\threelineshloka
{केनासि हूतः प्रहितोऽसि राज-न्युवा स्रग्वी दर्शनीयः सुवर्चाः}
{कुतऋ आयातः कतरस्यां दिशि त्व-मुताहोस्वित्पार्थिवं स्थानमस्ति ॥ययातिरुवाच}
{}


\threelineshloka
{इमं भौमं नरकं क्षीणपुण्यःप्रवेष्टुमुर्वीं गगनाद्विप्रहीणः}
{`विद्वांश्चैवं मतिमानार्यबुद्धि-र्ममाभवत्कर्मलोक्यं च सर्वम्'}
{उक्त्वाऽहं वः प्रपतिष्याम्यनन्तरंत्वरन्ति मां लोकपा ब्राह्मणा ये}


\threelineshloka
{सतां सकाशे तु वृतः प्रपात-स्ते सङ्गता गुणवन्तश्च सर्वे}
{शक्राच्च लब्धो हि वरो मयैषपतिष्यता भूमितलं नरेन्द्र ॥अष्टक उवाच}
{}


\threelineshloka
{पृच्छामि त्वां मा प्रपत प्रपातंयदि लोकाः पार्थिव सन्ति मेऽत्र}
{यद्यन्तरिक्षे यदि वा दिवि स्थिताःक्षेत्रज्ञं त्वां तस्य धर्मस्य मन्ये ॥ययातिरुवाच}
{}


\threelineshloka
{यावत्पृथिव्यां विहितं गवाश्वंसहारण्यैः पशुभिः पार्वतैश्च}
{तावल्लोका दिवि ते संस्थिता वैतथा विजानीहि नरेन्द्रसिंह ॥अष्टक उवाच}
{}


\threelineshloka
{तांस्ते ददामि मा प्रपत प्रपातंये मे लोका दिवि राजेन्द्र सन्ति}
{यद्यन्तरिक्षे यदि वा दिवि श्रिता-स्तानाक्रम क्षिप्रमपेतमोहः ॥ययातिरुवाच}
{}


\twolineshloka
{नास्मद्विधोऽब्राह्मणो ब्रह्मविच्चप्रतिग्रहे वर्तते राजमुख्य}
{यथा प्रदेयं सततं द्विजेभ्य-स्तथाऽददं पूर्वमहं नरेन्द्र}


\threelineshloka
{नाब्राह्मणः कृपणो जातु जीवे-द्या चाप्यस्याऽब्राह्मणी वीरपत्नी}
{सोऽहं नैवाकृतपूर्वं चरेयंविधित्समानः किमु तत्र साधुः ॥प्रतर्दन उवाच}
{}


\threelineshloka
{पृच्छामि त्वां स्पृहणीयरूपप्रतर्दनोऽहं यदि मे सन्ति लोकाः}
{यद्यन्तरिक्षे यदि वा दिवि श्रिताःक्षेत्रज्ञं त्वां तस्य धर्मस्य मन्ये ॥ययातिरुवाच}
{}


\threelineshloka
{सन्ति लोका बहवस्ते नरेन्द्रअप्येकैकः सप्तसप्ताप्यहानि}
{मधुच्युतो घृतपृक्ता विशोका-स्ते नान्तवन्तः प्रतिपालयन्ति ॥प्रतर्दन उवाच}
{}


\threelineshloka
{तांस्ते ददानि मा प्रपत प्रपातंये मे लोकास्तव ते वै भवन्तु}
{यद्यन्तरिक्षे यदि वा दिवि श्रिता-स्तानाक्रम क्षिप्रमपेतमोहः ॥ययातिरुवाच}
{}


\twolineshloka
{न तुल्यतेजाः सुकृतं कामयेतयोगक्षेमं पार्थिव पार्थिवः सन्}
{दैवादेशादापदं प्राप्य विद्वां-श्चरेन्नृशंसं न हि जातु राजा}


\twolineshloka
{धर्म्यं मार्गं यतमानो यशस्यंकुर्यान्नृपो धर्ममवेक्षमाणः}
{न मद्विधो धर्मबुद्धिः प्रजान-न्कुर्यादेवं कृपणं मां यथाऽत्थ}


\twolineshloka
{कुर्यादपूर्वं न कृतं यदन्यै-र्विधित्समानः किमु तत्र साधु}
{`धर्माधर्मौ सुविनिश्चित्य सम्य-क्कार्याकार्येष्वप्रमत्तश्चरेद्यः}


\twolineshloka
{स वै धीमान्सत्यसन्धः कृतात्माराजा भवेल्लोकपालो महिम्ना}
{यदा भवेत्संशयो धर्मकार्येकामार्थौ वा यत्र विन्दन्ति सम्यक्}


\twolineshloka
{कार्यं तत्र प्रथमं धर्मकार्यंयन्नो विरुध्यादर्थकामौ स धर्मःवैशंपायन उवाच}
{' ब्रुवाणमेवं नृपतिं ययातिंनृपोत्तमो वसुमानब्रवीत्तम्}


\chapter{अध्यायः ८७}
\twolineshloka
{वसुमानुवाच}
{}


\threelineshloka
{पृच्छामि त्वां वसुमानौषदश्वि-र्यद्यस्ति लोको दिवि मे नरेन्द्र}
{यद्यन्तरिक्षे प्रथितो महात्मन्क्षेत्रज्ञं त्वां तस्य धर्मस्य मन्ये ॥ययातिरुवाच}
{}


\threelineshloka
{यदन्तरिक्षं पृथिवी दिशश्चयत्तेजसा तपते भानुमांश्च}
{लोकास्तावन्तो दिवि संस्थिता वैतेनान्तवन्तः प्रतिपालयन्ति ॥वसुमानुवाच}
{}


\threelineshloka
{तांस्ते ददानि मा प्रपत प्रपातंये मे लोकास्तव ते वै भवन्तु}
{क्रीणीष्वैतांस्तृणकेनापि राज-न्प्रतिग्रहस्ते यदि धीमन्प्रदुष्टः ॥ययातिरुवाच}
{}


\threelineshloka
{न मिथ्याऽहं विक्रयं वै स्मरामिवृथा गृहीतं शिशुकाच्छङ्कमानः}
{कुर्यां न चैवाकृतपूर्वमन्यै-र्विधित्समानः किमु तत्र साधुः ॥वसुमानुवाच}
{}


\threelineshloka
{तांस्त्वं लोकान्प्रतिपद्यस्व राज-न्मया दत्तान्यदि नेष्टः क्रयस्ते}
{अहं न तान्वै प्रतिगन्ता नरेन्द्रसर्वे लोकास्तव ते वै भवन्तु ॥शिबिरुवाच}
{}


\threelineshloka
{पृच्छामि त्वां शिबिरौशीनरोऽहंममापि लोका यदि सन्तीह तात}
{यद्यन्तरिक्षे यदि वा दिवि श्रिताःक्षेत्रज्ञं त्वां तस्य धर्मस्य मन्ये ॥ययातिरुवाच}
{}


\threelineshloka
{यत्त्वं वाचा हृदयेनापि साधू-न्परीप्समानान्नावमंस्था नरेन्द्र}
{तेनानन्ता दिवि लोकाः श्रितास्तेविद्युद्रूपाः स्वनवन्तो महान्तः ॥शिबिरुवाच}
{}


\threelineshloka
{तांस्त्वं लोकान्प्रतिपद्यस्व राज-न्मया दत्तान्यदि नेष्टः क्रयस्ते}
{न चाहं तान्प्रतिपत्स्ये ह दत्त्वायत्र गत्वा नानुशोचन्ति धीराः ॥ययातिरुवाच}
{}


\threelineshloka
{यथा त्वमिन्द्रप्रतिमप्रभाव-स्ते चाप्यनन्ता नरदेव लोकाः}
{तथाऽद्य लोके न रमेऽन्यदत्तेतस्माच्छिबे नाभिनन्दामि दायम् ॥अष्टक उवाच}
{}


\threelineshloka
{न चेदेकैकशो राजँल्लोकान्नः प्रतिनन्दसि}
{सर्वे प्रदाय भवते गन्तारो नरकं वयम् ॥ययातिरुवाच}
{}


\threelineshloka
{यदर्होऽहं तद्यतध्वं सन्तः सत्याभिनन्दिनः}
{अहं तन्नाभिजानामि यत्कृतं न मया पुरा ॥अष्टक उवाच}
{}


\threelineshloka
{कस्यैते प्रतिदृश्यन्ते रथाः पञ्च हिरण्मयाः}
{यानारुह्य नरो लोकानभिवाञ्छति शाश्वतान् ॥ययातिरुवाच}
{}


\threelineshloka
{युष्मानेते वहिष्यन्ति रथाः पञ्च हिरण्मयाः}
{उच्चैः सन्तः प्रकाशन्ते ज्वलन्तोऽग्निशिखा इव ॥`वैशंपायन उवाच}
{}


\twolineshloka
{अश्वमेधे महायज्ञे स्वयंभुविहिते पुरा}
{हयस्य यानि चाङ्गानि संनिकृत्य यथाक्रमम्}


\twolineshloka
{होताऽध्वर्युरथोद्गाता ब्रह्मणा सह भारत}
{अग्नौ प्रास्यन्ति विधिवत्समस्ताः षोडशर्त्विजः}


\twolineshloka
{धूमगन्धं च पापिष्ठा ये जिघ्रन्ति नरा भुवि}
{विमुक्तपापाः पूतास्ते तत्क्षणेनाभवन्नराः}


\twolineshloka
{एतस्मिन्नन्तरे चैव माधवी सा तपोधना}
{मृगचर्मपरीताङ्गी परिधाय मृगत्वचम्}


\twolineshloka
{मृगैः परिचरन्ती सा मृगाहारविचेष्टिता}
{यज्ञवाटं मृगगणैः प्रविश्य भृशविस्मिता}


\twolineshloka
{आघ्रायन्ती धूमगन्धं मृगैरेव चचार सा}
{यज्ञवाटमटन्ती सा पुत्रांस्तानपराजितान्}


\twolineshloka
{पश्यन्ती यज्ञमाहात्म्यं मुदं लेभे च माधवी}
{असंस्पृशन्तं वसुधां ययातिं नाहुषं यदा}


\twolineshloka
{दिविष्ठं प्राप्तमाज्ञाय ववन्दे पितरं तदा}
{तदा वसुमनापृच्छन्मातरं वै तपस्विनीम्}


\threelineshloka
{भवत्या यत्कृतमिदं वन्दनं पादयोरिह}
{कोयं देवोपमो राजा याऽभिवन्दसि मे वद ॥माधव्युवाच}
{}


\twolineshloka
{शृणुध्वं सहिताः पुत्रा नाहुषोयं पिता मम}
{ययातिर्मम पुत्राणां मातामह इति स्मृतः}


\threelineshloka
{पूरुं मे भ्रातरं राज्ये समावेश्य दिवं गतः}
{केन वा कारणेनैवमिह प्राप्तो महायशाः ॥वैशंपायन उवाच}
{}


\twolineshloka
{तस्यास्तद्वचनं श्रुत्वा स्वर्गाद्भ्रष्टेति चाब्रवीत्}
{सा पुत्रस्य वचः श्रुत्वा संभ्रमाविष्टचेतना}


\twolineshloka
{माधवी पितरं प्राह दौहित्रपरिवारितम्}
{तपसा निर्जिताँल्लोकान्प्रतिगृह्णीष्व मामकान्}


\fourlineindentedshloka
{पुत्राणामिव पौत्राणां धर्मादधिगतं धनम्}
{स्वार्थणेव वदन्तीह ऋषयो धर्मपाठकाः}
{तस्माद्दानेन तपसा चास्माकं दिवमाव्रज ॥ययातिरुवाच}
{}


\twolineshloka
{यदि धर्मफलं ह्येतच्छोभनं भविता तव}
{दुहित्रा चैव दौहित्रैस्तारितोऽहं महात्मभिः}


\twolineshloka
{तस्मात्पवित्रं दौहित्रमद्यप्रभृति पैतृके}
{त्रीणि श्राद्धे पवित्राणि दौहित्रः कुतपस्तिलाः}


\twolineshloka
{त्रीणि चात्र प्रशंसन्ति शौचमक्रोधमत्वराम्}
{भोक्तारः परिवेष्टारः श्रावितारः पवित्रकाः}


\twolineshloka
{दिवसस्याष्टमे भागे मन्दीभवति भास्करे}
{स कालः कुतपो नाम पितॄणां दत्तमक्षयम्}


\twolineshloka
{तिलाः पिशाचाद्रक्षन्ति दर्भा रक्षन्ति राक्षसात्}
{रक्षन्ति श्रोत्रियाः पङ्क्तिं यतिभिर्भुक्तमक्षयम्}


\threelineshloka
{लब्ध्वा पात्रं तु विद्वांसं श्रोत्रियं सुव्रतं शुचिम्}
{स कालः कालतो दत्तं नान्यथा काल इष्यते ॥वैशंपायन उवाच}
{}


\threelineshloka
{एवमुक्त्वा ययातिस्तु पुनः प्रोवाच बुद्धिमान्}
{सर्वे ह्यवभृथस्नातास्त्वरध्वं कार्यगौरवात् ॥'अष्टक उवाच}
{}


\threelineshloka
{आतिष्ठ स्वरथं राजन्विक्रमस्व विहायसम्}
{वयमप्यनुयास्यामो यदा कालो भविष्यति ॥ययातिरुवाच}
{}


\threelineshloka
{सर्वैरिदानीं गन्तव्यं सह स्वर्गजितो वयम्}
{एष नो विरजाः पन्था दृश्यते देवसद्मनः ॥वैशंपायन उवाच}
{}


\threelineshloka
{`अष्टकश्च शिबिश्चैव काशेयश्च प्रतर्दनः}
{ऐक्ष्वाकवो वसुमनाश्चत्वारो भूमिपास्तदा}
{सर्वे ह्यवभृथस्नाताः स्वर्गताः साधवः सह ॥'}


\threelineshloka
{तेऽधिरुह्य रथान्सर्वे प्रयाता नृपस्तमाः}
{आक्रमन्तो दिवं भाभिर्धर्मेणावृत्य रोदसी ॥अष्टक उवाच}
{}


\threelineshloka
{अहं मन्ये पूर्वमेकोऽस्मि गन्तासखा चेन्द्रः सर्वथा मे महात्मा}
{कस्मादेवं शिबिरौशीनरोऽय-मेकोऽत्यगात्सर्ववेगेन वाहान् ॥ययातिरुवाच}
{}


\twolineshloka
{अददद्याचमानाय यावद्वित्तमविन्दत}
{उशीनरस्य पुत्रोऽयं तस्माच्छ्रेष्ठोहि वः शिबिः}


\twolineshloka
{दानं तपः संत्यमथाऽपि धर्मोह्रीः श्रीः क्षमा सौम्यमथो विधित्सा}
{राजन्नेतान्यप्रमेयाणि राज्ञःशिबेः स्थितान्यप्रतिमस्य बुद्ध्या}


\threelineshloka
{एवं वृत्तो ह्रीनिषेवश्च यस्मा-त्तस्माच्छिबिरत्यगाद्वै रथेन}
{वैशंपायन उवाच}
{अथाष्टकः पुनरेवान्वपृच्छ-न्मातामहं कौतुकेनेन्द्रकल्पम्}


\threelineshloka
{पृच्छामि त्वां नृपते ब्रूहि सत्यंकुतश्च कश्चासि सुतश्च कस्य}
{कृतं त्वया यद्धि न तस्य कर्तालोके त्वदन्यः क्षत्रियो ब्राह्मणो वा ॥ययातिरुवाच}
{}


\twolineshloka
{ययातिरस्मि नहुषस्य पुत्रःपूरोः पिता सार्वबौमस्त्विहासम्}
{गुह्यं चार्थं मामकेभ्यो ब्रवीमिमातामहोऽहं भवतां प्रकाशम्}


\twolineshloka
{सर्वामिमां पृथिवीं निर्जिगायदत्त्वा प्रतस्थे विपिनं ब्राह्मणेभ्यः}
{मेध्यानश्वानेकशतान्सुरूपां-स्तदा देवाः पुण्यभाजो भवन्ति}


\twolineshloka
{अदामहं पृथिवीं ब्राह्मणेभ्यःपूर्णामिमामखिलां वाहनेन}
{गोभिः सुवर्णेन धनैश्च मुख्यै-स्तदाऽददं गाः शथमर्बुदानि}


\twolineshloka
{सत्येन मे द्यौश्च वसुन्धरा चतथैवाग्निज्वर्लते मानुषेषु}
{न मे वृथा व्याहृतमेव वाक्यंसत्यं हि सन्तः प्रतिपूजयन्ति}


\twolineshloka
{यदष्टक प्रब्रवीमीह सत्यंप्रतर्दनं चौषदश्विं तथैव}
{सर्वे च लोका मुनयश्च देवाःसत्येन पूज्या इति मे मनोगतम्}


\threelineshloka
{यो नः स्वर्गजितः सर्वान्यथावृत्तं निवेदयेत्}
{अनसूयुर्द्विजाग्र्येभ्यः स लभेन्नः सलोकताम् ॥वैशंपायन उवाच}
{}


\twolineshloka
{एवं राजा स महात्मा ह्यतीवस्वैर्दौहित्रैस्तारितोऽमित्रसाह}
{त्यक्त्वा महीं परमोदारकर्मास्वर्गं गतः कर्मभिर्व्याप्य पृथ्वीम्}


\chapter{अध्यायः ८८}
\twolineshloka
{जनमेजय उवाच}
{}


\twolineshloka
{पुत्रं ययातेः प्रबूहि पूरुं धर्मभृतां वरम्}
{आनुपूर्व्येण ये चान्ये पूरोर्वंशविवर्धनाः}


\threelineshloka
{विस्तरेण पुनर्ब्रूहि दौष्यन्तेर्जनमेजयात्}
{संबभूव यथा राजा भरतो द्विजसत्तम ॥वैशंपायन उवाच}
{}


\twolineshloka
{पूरुर्नृपतिशार्दूलो यथैवास्य पिता नृप}
{धर्मनित्यः स्थितो राज्ये शक्रतुल्यपराक्रमः}


\twolineshloka
{प्रवीरेश्वररौद्राश्वास्त्रयः पुत्रा महारथाः}
{पूरोः पौष्ट्यामजायन्त प्रवीरो वंशकृत्ततः}


\twolineshloka
{मनस्युरभवत्तस्माच्छूरसेनीसुतः प्रभुः}
{पृथिव्याश्चतुरन्ताया गोप्ता राजीवलोचनः}


\twolineshloka
{शक्तः संहननो वाग्मी सौवीरीतनयास्त्रयः}
{मनस्योरभवन्पुत्राः शूराः सर्वे महारथाः}


\twolineshloka
{अन्वग्भानुप्रभृतयो मिश्रकेश्यां मनस्विनः}
{रौद्राश्वस्य महेष्वासा दशाप्सरसि सूनवः}


\twolineshloka
{यज्वानो जज्ञिरे शूराः प्रजावन्तो बहुश्रुताः}
{सर्वे सर्वास्त्रविद्वासः सर्वे धर्मपरायणाः}


\twolineshloka
{ऋचेयुरथ कक्षेयुः कृकणेयुश्च वीर्यवान्}
{स्थण्डिलेयुर्वनेयुश्च जलेयुश्च महायशाः}


\twolineshloka
{तेजेयुर्बलावान्धीमान्सत्येयुश्चन्द्रविक्रमः}
{धर्मेयुः सन्नतेयुश्च दशमो देवविक्रमः}


\twolineshloka
{अनाधृष्टिरभूत्तेषां विद्वान्भुवि तथैकराट्}
{ऋचेयुरथ विक्रान्तो देवानामिव वासवः}


\twolineshloka
{अनाधृष्टिसुतस्त्वासीद्राजसूयाश्वमेधकृत्}
{मतिनार इति ख्यातो राजा परमधार्मिकः}


\twolineshloka
{मतिनारसुता राजंश्चत्वारोऽमितविक्रमाः}
{तंसुर्महानतिरथो द्रुह्युश्चाप्रतिमद्युतिः}


\twolineshloka
{तेषां तंसुर्महावीर्यः पौरवं वंशमुद्वहन्}
{आजहार यशो दीप्तं जिगाय च वसुंधराम्}


\twolineshloka
{ईलिनं तु सुतं तंसुर्जनयामास वीर्यवान्}
{सोऽपि कृत्स्नामिमां भूमिं विजिग्ये जयतां वरः}


\twolineshloka
{रथन्तर्यां सुतान्पञ्च पञ्चभूतोपमांस्ततः}
{ईलिनो जनयामास दुष्यन्तप्रभृतीन्नृपान्}


\twolineshloka
{दुष्यन्तं शूरभीमौ च प्रवसुं वसुमेव च}
{तेषां श्रेष्ठोऽभवद्राजा दुष्यन्तो दुर्जयो युधि}


\twolineshloka
{दुष्यन्ताल्लक्षणायां तु जज्ञे वै जनमेजयः}
{शकुन्तलायां भरतो दौष्यन्तिरभवत्सुतः}


% Check verse!
तस्माद्भरतवंशस्य विप्रतस्थे महद्यशः
\chapter{अध्यायः ८९}
\twolineshloka
{जनमेजय उवाच}
{}


\threelineshloka
{भगवन्विस्तरेणेह भरतस्य महात्मनः}
{जन्म कर्म च सुश्रूषोस्तन्मे शंसितुमर्हसि ॥वैशंपायन उवाच}
{}


\twolineshloka
{पौरवाणां वंशकरो दुष्यन्तो नाम वीर्यवान्}
{पृथिव्याश्चतुरन्ताया गोप्ता भरतसत्तम}


\twolineshloka
{चतुर्भागं भुवः कृत्स्नं यो भुङ्क्ते मनुजेश्वरः}
{समुद्रावरणांश्चापि देशान्स समितिंजयः}


\twolineshloka
{आम्लेच्छावधिकान्सर्वान्स भुङ्क्ते रिपुमर्दनः}
{रत्नाकरसमुद्रान्तांश्चातुर्वर्ण्यजनावृतान्}


\twolineshloka
{न वर्णसङ्करकरो न कृष्याकरकृज्जनः}
{न पापकृत्कश्चिदासीत्तस्मिन्राजनि शासति}


\twolineshloka
{धर्मे रतिं सेवमाना धर्मार्थावभिपेदिरे}
{तदा नरा नरव्याघ्र तस्मिञ्जनपदेश्वरे}


\twolineshloka
{नासीच्चोरभयं तात न क्षुधाभयमण्वपि}
{नासीद्व्याधिभयं चापि तस्मिञ्जनपदेश्वरे}


\twolineshloka
{स्वधर्मै रेमिरे वर्णा दैवे कर्मणि निःस्पृहाः}
{तमाश्रित्य महीपालमासंश्चैवाकुतोभयाः}


\twolineshloka
{कालवर्षी च पर्जन्यः सस्यानि रसवन्ति च}
{सर्वरत्नसमृद्धा च मही पशुमती तथा}


\twolineshloka
{स्वकर्मनिरता विप्रा नानृतं तेषु विद्यते}
{स चाद्भुतमहावीर्यो वज्रसंहननो युवा}


\twolineshloka
{उद्यम्य मन्दरं दोर्भ्यां वहेत्सवनकाननम्}
{चतुष्पथगदायुद्धे सर्वप्रहरणेषु च}


\twolineshloka
{नागपृष्ठेऽश्वपृष्ठे च बभूव परिनिष्ठतः}
{बले विष्णुसमश्चासीत्तेजसा भास्करोपमः}


\twolineshloka
{अक्षोभ्यत्वेऽर्णवसमः सहिष्णुत्वे धरासमः}
{संमतः स महीपालः प्रसन्नपुरराष्ट्रवान्}


% Check verse!
भूयो धर्मपरैर्भावैर्मुदितं जनमादिशत्
\chapter{अध्यायः ९०}
\twolineshloka
{जनमेजय उवाच}
{}


\twolineshloka
{संभवं भरतस्याहं चरितं च महामतेः}
{शकुन्तलायाश्चोत्पत्तिं श्रोतुमिच्छामि तत्त्वतः}


\twolineshloka
{दुष्यन्तेन च वीरेण यथा प्राप्ता शकुन्तला}
{तं वै पुरुषसिंहस्य भगवन्विस्तरं त्वहम्}


\threelineshloka
{श्रोतुमिच्छामि तत्त्वज्ञ सर्वं मतिमतां वर}
{वैशंपायन उवाच}
{स कदाचिन्महाबाहुः प्रभूतबलवाहनः}


\twolineshloka
{वनं जगाम गहनं हयनागशतैर्वृतः}
{बलेन चतुरङ्गेण वृतः परमवल्गुना}


\twolineshloka
{खड्गशक्तिधरैर्वीरैर्गदामुसलपाणिभिः}
{प्रासतोमरहस्तैश्च ययौ योधशतैर्वृतः}


\twolineshloka
{सिंहनादैश्च योधानां शङ्खदुनदुभिनिःस्वनैः}
{रथनेमिस्वनैश्चैव सनागवरबृंहितैः}


\twolineshloka
{नानायुधधरैश्चापि नानावेषधरैस्तथा}
{ह्रेषितस्वनमिश्रैश्च क्ष्वेडितास्फोटितस्वनैः}


\twolineshloka
{आसीत्किलकिलाशब्दस्तस्मिन्गच्छति पार्थिवे}
{प्रासादवरशृङ्गस्थाः परया नृपशोभया}


\twolineshloka
{ददृशुस्तं स्त्रियस्तत्र शूरमात्मयशस्करम्}
{शक्रोपमममित्रघ्नं परवारणवारणम्}


\twolineshloka
{पश्यन्तः स्त्रीगणास्तत्र वज्रपाणिं स्म मेनिरे}
{अयं स पुरुषव्याघ्रो रणे वसुपराक्रमः}


\twolineshloka
{यस्य बाहुबलं प्राप्य न भवन्त्यसुहृद्गणाः}
{इति वाचो ब्रुवन्त्यस्ताः स्त्रियः प्रेम्णा नराधिपं}


\twolineshloka
{तुष्टुवुः पुष्पवृष्टीश्च ससृजुस्तस्य मूर्धनि}
{तत्रतत्र च विप्रेन्द्रैः स्तूयमानः समन्ततः}


\twolineshloka
{निर्ययौ परमप्रीत्या वनं मृगजिघांसया}
{तं देवराजप्रतिमं मत्तवारणधूर्गतम्}


\twolineshloka
{द्विजक्षत्रियविट्शूद्रा निर्यान्तमनुजग्मिरे}
{ददृशुर्वर्धमानास्ते आशीर्भिश्च जयेन च}


\twolineshloka
{सुदूरमनुजग्मुस्तं पौरजानपदास्तथा}
{न्यवर्तन्त ततः पश्चादनुज्ञाता नृपेण ह}


\twolineshloka
{सुपर्णप्रतिमेनाथ रथेन वसुधाधिपः}
{महीमापूरयामास घोषेण त्रिदिवं तथा}


\twolineshloka
{स गच्छन्ददृशे धीमान्नन्दनप्रतिमं वनम्}
{बिल्वार्कखदिराकीर्णं कपित्थधवसंकुलम्}


\twolineshloka
{विषमं पर्वतस्रस्तै रश्मभिश्च समावृतम्}
{निर्जलं निर्मनुष्यं च बहुयोजनमायतम्}


\twolineshloka
{मृगसिंहैर्वृतं घोरैरन्यैश्चापि वनेचरैः}
{तद्वनं मनुजव्याघ्रः सभृत्यबलवाहनः}


\twolineshloka
{लोडयामास दुष्यन्तः सूदयन्विविधान्मृगान्}
{बाणगोचरसंप्राप्तांस्तत्र व्याघ्रगणान्बहून्}


\twolineshloka
{पातयामास दुष्यन्तो निर्बिभेद च सायकैः}
{दूरस्थान्सायकैः कांश्चिदभिनत्स नराधिपः}


\twolineshloka
{अभ्याशमागतांश्चान्यान्खड्गेन निरकृन्तत}
{कांश्चिदेणान्समाजघ्ने शक्त्या शक्तिमतां वरः}


\twolineshloka
{गदामण्डलतत्त्वज्ञश्चचारामितविक्रमः}
{तोमरैरसिभिश्चापि गदामुसलकम्पनैः}


\twolineshloka
{चचार स विनिघ्नन्वै वन्यांस्तत्र मृगद्विजान्}
{राज्ञा चाद्भुतवीर्येण योधैश्च समरप्रियैः}


\twolineshloka
{लोड्यमानं महारण्यं तत्यजुः स्म मृगाधिपाः}
{तत्र विद्रुतयूथानि हतयूथपतीनि च}


\twolineshloka
{मृगयूथान्यथौत्सुक्याच्छब्दं चक्रुस्ततस्ततः}
{शुष्काश्चापि नदीर्गत्वा जलनैराश्यकर्शिताः}


\twolineshloka
{व्यायामक्लान्तहृदयाः पतन्ति स्म विचेतसः}
{क्षुत्पिपासापरीताश्च श्रान्ताश्च पतिता भुवि}


\twolineshloka
{केचित्तत्र नरव्याघ्रैरभक्ष्यन्त बुभुक्षितैः}
{केचिदग्निमथोत्पाद्य संसाध्य च वनेचराः}


\twolineshloka
{भक्षयन्ति स्म मांसानि प्रकुट्य विधिवत्तदा}
{तत्र केचिद्गजा मत्ता बलिनः शस्त्रविक्षताः}


\twolineshloka
{संकोच्याग्रकरान्भीताः प्राद्रवन्ति स्म वेगिताः}
{शकृन्मूत्रं सृजन्तश्च क्षरन्तः शोणितं बहु}


\threelineshloka
{वन्या गजवरास्तत्र ममृदुर्मनुजान्बहून्}
{तद्वनं बलमेघेन शरधारेण संवृतम्}
{व्यरोचत मृगाकीर्णं राज्ञा हतमृगाधिपम्}


\chapter{अध्यायः ९१}
\twolineshloka
{वैशंपायन उवाच}
{}


\twolineshloka
{ततो मृगसहस्राणि हत्वा सबलवाहनः}
{तत्र मेघघनप्रख्यं सिद्धचारणसेवितम्}


\twolineshloka
{वनमालोकयामास नगराद्योजनद्वये}
{मृगाननुचरन्वन्याञ्श्रमेण परिपीडितः}


\twolineshloka
{मृगाननुचरन्राजा वेगेनाश्वानचोदयत्}
{राजा मृगप्रसङ्गेन वनमन्यद्विवेश ह}


\twolineshloka
{एक एवोत्तमबलः क्षुत्पिपासाश्रमान्वितः}
{स वनस्यान्तमासाद्य महच्छून्यं समासदत्}


\twolineshloka
{तच्चाप्यतीत्य नृपतिरुत्तमाश्रमसंयुतम्}
{मनःप्रह्लादजननं दृष्टिकान्तमतीव च}


\twolineshloka
{सीतमारुतसंयुक्तं जगामान्यन्महद्वनम्}
{पुष्पितैः पादपैः कीर्णमतीव सुखशाद्वलम्}


\twolineshloka
{विपुलं मधुरारावैर्नादितं विहगैस्तथा}
{पुंस्कोकिलनिनादैश्च झिल्लीकगणनादितम्}


\twolineshloka
{प्रवृद्धविटपैर्वृक्षैः सुखच्छायैः समावृतम्}
{षट्पदाघूर्णिततलं लक्ष्म्या परमया युतम्}


\twolineshloka
{नापुष्पः पादपः कश्चिन्नाफलो नापि कण्टकी}
{षट्पदैर्नाप्यपाकीर्णस्तस्मिन्वै काननेऽभवत्}


\twolineshloka
{विगहैर्नादितं पुष्पैरलङ्कृतमतीव च}
{सर्वर्तुकुसुमैर्वृक्षैः सुखच्छायैः समावृतम्}


\twolineshloka
{मनोरमं सहेष्वासो विवेश वनमुत्तमम्}
{मारुता कलितास्तत्र द्रुमाः कुसुमशाखिनः}


\twolineshloka
{पुष्पवृष्टिं विचित्रां तु व्यसजंस्ते पुनः पुनः}
{दिवस्पृशोऽथ संघुष्टाः पक्षिभिर्मधुरस्वनैः}


\twolineshloka
{विरेजुः पादपास्तत्र विचित्रकुसुमाम्बराः}
{तेषां तत्र प्रवालेषु पुष्पभारावनामिषु}


\twolineshloka
{रुवन्ति रावान्मधुरान्षट्पदा मधुलिप्सवः}
{तत्र प्रदेशांश्च बहून्कुसुमोत्करमण्डितान्}


\twolineshloka
{लतागृहपरिक्षिप्तान्मनसः प्रीतिवर्धनान्}
{संपश्यन्सुमहातेजा बभूव मुदितस्तदा}


\twolineshloka
{परस्पराश्लिष्टशाखैः पादपैः कुसुमान्वितैः}
{अशोभत वनं तत्तु महेन्द्रध्वजसन्निभैः}


\twolineshloka
{सिद्धचारणसङ्घैश्च गन्धर्वाप्सरसां गणैः}
{सेवितं वनमत्यर्थं मत्तवानरकिन्नरैः}


\twolineshloka
{सुखः शीतः सुगन्धी च पुष्परेणुवहोऽनिलः}
{परिक्रामन्वने वृक्षानुपैतीव रिरंसया}


\twolineshloka
{एवंगुणसमायुक्तं ददर्श स वनं नृपः}
{नदीकच्छोद्भं कान्तमुच्छ्रितध्वजसन्निभम्}


\twolineshloka
{प्रेक्षमाणो वनं तत्तु सुप्रहृष्टविहङ्गमम्}
{आश्रमप्रवरं रम्यं ददर्श च मनोरमम्}


\twolineshloka
{नानावृक्षसमाकीर्णं संप्रज्वलितपावकम्}
{तं तदाऽप्रतिमं श्रीमानाश्रमं प्रत्यपूजयत्}


\twolineshloka
{यतिभिर्वालखिल्यैश्च वृतं मुनिगणान्वितम्}
{अग्न्यगारैश्च बहुभिः पुष्पसंस्तरसंस्तृतम्}


\twolineshloka
{महाकच्छैर्बृहद्भिश्च विभ्राजितमतीव च}
{मालिनीमभितो राजन्नदीं पुण्यां सुखोदकाम्}


\twolineshloka
{नैकपक्षिगणाकीर्णां तपोवनमनोरमाम्}
{तत्रव्यालमृगान्सैम्यान्पश्यन्प्रीतिमवाप सः}


\twolineshloka
{तं चाप्रतिरथः श्रीमानाश्रमं प्रत्यपद्यत}
{देवलोकप्रतीकाशं सर्वतः सुमनोहरम्}


\twolineshloka
{नदीं चाश्रमसंश्लिष्टां पुण्यतोयां ददर्श सः}
{सर्वप्राणभृतां तत्र जननीमिव धिष्ठिताम्}


\twolineshloka
{सचक्रवाकपुलिनां पुष्पफेनप्रवाहिनीम्}
{सकिन्नरगणावासां वारनर्क्षनिषेविताम्}


\twolineshloka
{पुण्यस्वाध्यायसंघुष्टा पुलिनैरुपशोभिताम्}
{मत्तवारणशार्दूलभुजगेन्द्रनिषेविताम्}


\twolineshloka
{तस्यास्तीरे भगवतः काश्यपस्य महात्मनः}
{आश्रमप्रवरं रम्यं महर्षिगणसेवितम्}


\twolineshloka
{नदीमाश्रमसंबद्धां दृष्ट्वाश्रमपदं तथा}
{चकाराभिप्रवेशाय मतिं स नृपतिस्तदा}


\twolineshloka
{अलङ्कृतं द्वीपवत्या मालिन्या रम्यतीरया}
{नरनारायणस्थानं गङ्गयेवोपशोभितम्}


\twolineshloka
{मत्तबर्हिणसंघुष्टं प्रविवेश महद्वनम्}
{तत्स चैत्ररथप्रख्यं समुपेत्य नरर्षभः}


\twolineshloka
{अतीव गुणसंपन्नमनिर्देश्यं च वर्चसा}
{महर्षिं काश्यपं द्रष्टुमथ कण्वं तपोधनम्}


\twolineshloka
{ध्वजिनीमश्वसंबाधां पदातिगजसङ्कुलाम्}
{अवस्थाप्य वनद्वारि सेनामिदमुवाच सः}


\twolineshloka
{मुनिं विरजसं द्रष्टुं गमिष्यामि तपोधनम्}
{काश्यपं स्थीयतामत्र यावदागमनं मम}


% Check verse!
तद्वनं नन्दनप्रख्यमासाद्य मनुजेश्वरः ॥क्षुत्पिपासे जहौ राजा मुदं चावाप पुष्कलाम्
\twolineshloka
{सामात्यो राजलिङ्गानि सोपनीय नराधिपः}
{पुरोहितसहायश्च जगामाश्रममुत्तमम्}


\threelineshloka
{दिदृक्षुस्तत्र तमृषिं तपोराशिमथाव्ययम्}
{ब्रह्मलोकप्रतीकाशमाश्रमं सोऽभिवीक्ष्य ह}
{षट्पदोद्गीतसंघुष्टं नानाद्विजगणायुतम्}


\threelineshloka
{विस्मयोत्फुल्लनयनो राजा प्रीतो बभूवह}
{ऋचो बह्वृचमुख्यैश्च प्रेर्यमाणाः पदक्रमैः}
{शुश्राव मनुजव्याघ्रो विततेष्विह कर्मसु}


\twolineshloka
{यज्ञविद्याङ्गविद्भिश्च यजुर्विद्भिश्च शोभितम्}
{मधुरैः सामगीतैश्च ऋषिभिर्नियतव्रतैः}


\twolineshloka
{भारुण्डसामगीताभिरथर्वशिरसोद्गतैः}
{यतात्मभिः सुनियतैः शुशुभे स तदाश्रमः}


\twolineshloka
{अथर्ववेदप्रवराः पूगयज्ञियसामगाः}
{संहितामीरयन्ति स्म पदक्रमयुतां तु ते}


\twolineshloka
{शब्दसंस्कारसंयुक्तर्ब्रुवद्भिश्चापरैर्द्विजैः}
{नादितः स बभौ श्रीमान्ब्रह्मलोक इवापरः}


\twolineshloka
{यज्ञसंस्तरविद्भिश्च क्रमशिक्षाविशारदैः}
{न्यायतत्त्वात्मविज्ञानसंपन्नैर्वेदपारगैः}


\twolineshloka
{नानावाक्यसमाहारसमवायविशारदैः}
{विशेषकार्यविद्भिश्च मोक्षधर्मपरायणैः}


\twolineshloka
{स्तापनाक्षेपसिद्धान्तपरमार्थज्ञतां गतैः}
{शब्दच्छन्दोनिरुक्तज्ञैः कालज्ञानविशारदैः}


\twolineshloka
{द्रव्यकर्मगुणज्ञैश्च कार्यकारणवेदिभिः}
{पक्षिवानररुतज्ञैश्च व्यासग्रन्थसमाश्रितैः}


\twolineshloka
{नानाशास्त्रेषु मुख्यैश्च शुश्राव स्वनमीरितम्}
{लोकायतिकमुख्यैश्च समन्तादनुनादितम्}


\twolineshloka
{तत्रतत्र च विप्रेन्द्रान्नियतान्संशितव्रतान्}
{जपहोमपरान्विप्रान्ददर्श परवीरहा}


\twolineshloka
{आसनानि विचित्राणि रुचिराणि महीपतिः}
{प्रयत्नोपहितानि स्म दृष्ट्वा विस्मयमागमत्}


\twolineshloka
{देवतायतनानां च प्रेक्ष्य पूजां कृतां द्विजैः}
{ब्रह्मलोकस्थमात्मानं मेने स नृपसत्तमः}


\twolineshloka
{स काश्यपतपोगुप्तमाश्रमप्रवरं शुभम्}
{नातृप्यत्प्रेक्षमाणो वै तपोवनगुणैर्युतम्}


\twolineshloka
{स काश्यपस्यायतनं महाव्रतै-र्वृतं समान्तादृषिभिस्तपोधनैः}
{विवेश सामात्यपुरोहितोऽरिहाविविक्तमत्यर्थमनोहरं शुभम्}


\chapter{अध्यायः ९२}
\twolineshloka
{वैशंपायन उवाच}
{}


\twolineshloka
{ततो गच्छन्महाबाहुरेकोऽमात्यान्विसृज्य तान्}
{नापश्यच्चाश्रमे तस्मिंस्तमृषिं संशितव्रतम्}


\twolineshloka
{सोऽपश्यमानस्तमृषिं शून्यं दृष्ट्वा तथाऽऽश्रमम्}
{उवाच क इहेत्युच्चैर्वनं सन्नादयन्निव}


\twolineshloka
{श्रुत्वाऽथ तस्य तं शब्दं कन्या श्रीरिव रूपिणी}
{निश्चक्रामाश्रमात्तस्मात्तापसीवेषधारिणी}


\twolineshloka
{सा तं दृष्ट्वैव राजानं दुष्यन्तमसितेक्षणा}
{`सुप्रीताऽभ्यागतं तं तु पूज्यं प्राप्तमथेश्वरम्}


\twolineshloka
{रूपयौवनसंपन्ना शीलाचारवती शुभा}
{सा तमायतपद्माक्षं व्यूढोरस्कं महाभुजम्}


\twolineshloka
{सिंहस्कन्धं दीर्घबाहुं सर्वलक्षणपूजितम्}
{स्पृष्टं मधुरया वाचा साऽब्रवीज्जनमेजया ॥'}


\twolineshloka
{स्वागतं त इति क्षिप्रमुवाच प्रतिपूज्य च}
{आसनेनार्चयित्वा च पाद्येनार्घ्येण चैव हि}


\twolineshloka
{पप्रच्छानामयं राजन्कुशलं च नराधिपम्}
{यथावदर्चयित्वाऽथ पृष्ट्वा चानामयं तदा}


\twolineshloka
{उवाच स्मयमानेव किं कार्यं क्रियतामिति}
{`आश्रमस्याभिगमने किं त्वं कार्यं चिकीर्षसि}


\twolineshloka
{कस्त्वमद्येह संप्राप्तो महर्षेराश्रमं शुभम्}
{'तामब्रवीत्ततो राजा कन्यां मधुरभाषिणीम्}


\twolineshloka
{दृष्ट्वा सर्वानवद्याङ्गीं यथावत्प्रतिपूजितः}
{`राजर्षेस्तस्य पुत्रोऽहमिलिनस्य महात्मनः}


\twolineshloka
{दुष्यन्त इति मे नाम सत्यं पुष्करलोचने}
{'आगतोऽहं महाभागमृषिं कण्वमुपासितुम्}


% Check verse!
क्व गतो भगवान्भद्रे गन्ममाचक्ष्व शोभने

शकुन्तलोवाच

गतः पिता मे भगवान्फलान्याहर्तुमाश्रमात्

मुहूर्तं संप्रतीक्षस्व द्रष्टास्येनमुपागतम् ॥वैशंपायन उवाच


\twolineshloka
{अपश्यमानस्तमृषिं तथा चोक्तस्तया च सः}
{तां दृष्ट्वा च वरारोहां श्रीमतीं चारुहासिनीम्}


\twolineshloka
{विभ्राजमानां वपुषा तपसा च दमेन च}
{रूपयौवनसंपन्नामित्युवाच महीपतिः}


\twolineshloka
{का त्वं कस्यासि सुश्रोणि किमर्थं चागता वनम्}
{एवंरूपगुणोपेता कुतस्त्वमसि शोभने}


\twolineshloka
{दर्शनादेव हि शुभे त्वया मेऽपहृतं मनः}
{इच्छामि त्वामहं ज्ञातुं तन्ममाचक्ष्व शोभने}


\twolineshloka
{`स्थितोस्म्यमितसौभाग्ये विवक्षुश्चास्मि किंचन}
{शृणु मे नागनासोरु वचनं मत्तकाशिनि}


\twolineshloka
{राजर्षेरन्वये जातः पूरोरस्मि विशेषतः}
{वृण्वे त्वामद्य सुश्रोणि दुष्यन्तो वरवर्णिनि}


\twolineshloka
{न मेऽन्यत्र क्षत्रियाया मनो जातु प्रवर्तते}
{ऋषिपुत्रीषु चान्यासु नावरासु परासु च}


\twolineshloka
{तस्मात्प्रणिहितात्मानं विद्दि मां कलभाषिणि}
{यस्यां मे त्वयि भावोऽस्ति क्षत्रिया ह्यसि का वदा}


\threelineshloka
{न हि मे भीरु विप्रायां मनः प्रसहते गतिम्}
{भजे त्वामायतापाङ्गे भक्तं भजितुमर्हसि}
{भुङ्क्ष राज्यं विशालाक्षि बुद्धिं मात्वन्यथा कृथाः'}


\twolineshloka
{एवमुक्ता तु सा कन्या तेन राज्ञा तमाश्रमे}
{उवाच हसती वाक्यमिदं सुमधुराक्षरम्}


\twolineshloka
{कण्वस्याहं भगवतो दुष्यन्त दुहिता मता}
{तपस्विनो धृतिमतो धर्मज्ञस्य महात्मनः}


\threelineshloka
{`अस्वतन्त्रास्मि राजेन्द्र काश्यपो मे गुरुः पिता}
{तमेव प्रार्थय स्वार्थं नायुक्तं कर्तुमर्हसि ॥'दुष्यन्त उवाच}
{}


\twolineshloka
{ऊर्ध्वरेता महाभागे भगवाँल्लोकपूजितः}
{चलेद्धि वृत्ताद्धर्मोपि न चलेत्संशितव्रतः}


\threelineshloka
{कथं त्वं तस्य दुहिता संभूता वरवर्णिनी}
{संशयो मे महानत्र तन्मे छेत्तुमिहार्हसि ॥शकुन्तलोवाच}
{}


\twolineshloka
{यथाऽयमागमो मह्यं यथा चेदमभूत्पुरा}
{`अन्यथा सन्तमात्मानमन्यथा सत्सु भाषते}


\twolineshloka
{स पापेनावृतो मूर्खस्तेन आत्मापहारकः}
{'शृणु राजन्यथातत्त्वं यथाऽस्मि दुहिता मुनेः}


\twolineshloka
{ऋषिः कश्चिदिहागम्य मम जन्माभ्यचोदयत्}
{`ऊर्ध्वरेता यथासि त्वं कुतस्त्वेयं शकुन्तला}


\threelineshloka
{पुत्री त्वत्तः कथं जाता तत्त्वं मे ब्रूहि काश्यप}
{'तस्मै प्रोवाच भगवान्यथा तच्छृणु पार्थिवा ॥कण्व उवाच}
{}


\twolineshloka
{तप्यमानः किल पुरा विश्वामित्रो महत्तपः}
{सुभृशं तापयामास शक्रं सुरगणेश्वरम्}


\twolineshloka
{तपसा दीप्तवीर्योऽयं स्थानान्मां च्यावयेदिति}
{भीतः पुरन्दरस्तस्मान्मेनकामिदमब्रवीत्}


\twolineshloka
{गुणैरप्सरसां दिव्यैर्मेनके त्वं विशिष्यसे}
{श्रेयो मे कुरु कल्याणि यत्त्वां वक्ष्यामि तच्छृणु}


\twolineshloka
{असावादित्यशङ्काशो विश्वामित्रो महातपाः}
{तप्यमानस्तपो घोरं मम कम्पयते मनः}


\twolineshloka
{मेनके तव भारोऽयं विश्वामित्रः सुमध्यमे}
{शंसितात्मा सुदुर्धर्ष उग्रे तपसि वर्तते}


\twolineshloka
{स मां न च्यावयेत्स्थानात्तं वै गत्वा प्रलोभय}
{चर तस्य तपोविघ्नं कुरु मे प्रियमुत्तमम्}


\threelineshloka
{रूपयौवनमाधुर्यचेष्टितस्मितभाषणैः}
{लोभयित्वा वरारोहे तपसस्तं निवर्तय ॥मेनकोवाच}
{}


\twolineshloka
{महातेजाः स भगवांस्तथैव च महातपाः}
{कोपनश्च तथा ह्येनं जानाति भगवानपि}


\twolineshloka
{तेजस्तपसश्चैव कोपस्य च महात्मनः}
{त्वमप्युद्विजसे यस्य नोद्विजेयमहं कथम्}


\twolineshloka
{महाभागं वसिष्ठं यः पुत्रैरिष्टैर्व्ययोजयत्}
{क्षत्रजातश्च यः पूर्वमभवद्ब्राह्मणो बलात्}


\twolineshloka
{शौचार्थं यो नदीं चक्रे दुर्गमां बहुभिर्जलैः}
{यां तां पुण्यतमां लोके कौशिकीति विदुर्जनाः}


\twolineshloka
{बभार यत्रास्य पुरा काले दुर्गे महात्मनः}
{दारान्मतङ्गो धर्मात्मा राजर्षिर्व्याधतां गतः}


\twolineshloka
{अतीतकाले दुर्भिक्षे अभ्येत्य पुनराक्षमम्}
{मुनिः पारेति नद्या वै नाम चक्रे तदा प्रभुः}


\twolineshloka
{मतङ्गं याजयाञ्चक्रे यत्र प्रीतमनाः स्वयम्}
{त्वं च सोमं भयाद्यस्य गतः पातुं सुरेश्वर}


\threelineshloka
{चकारान्यं च लोकं वै क्रुद्धो नक्षत्रसंपदा}
{प्रतिश्रवणपूर्वाणि नक्षत्राणि चकार यः}
{गुरुशापहतस्यापि त्रिशङ्कोः शरणं ददौ}


\twolineshloka
{ब्रह्मर्षिशापं राजर्षिः कथं मोक्ष्यति कौशिकः}
{अवमत्य तदा देवैर्यज्ञाङ्गं तद्विनाशितम्}


\twolineshloka
{अन्यानि च महातेजा यज्ञाङ्गान्यसृजत्प्रभुः}
{निनाय च तदा स्वर्गं त्रिशङ्कुं स महातपाः}


\twolineshloka
{एतानि यस्य कर्माणि तस्याहं भृशमुद्विजे}
{यथाऽसौ न दहेत्क्रुद्धस्तथाऽऽज्ञापय मां विभो}


\twolineshloka
{तेजसा निर्दहेल्लोकान्कम्पयेद्धरणीं पदा}
{संक्षिपेच्च महामेरुं तूर्णमावर्तयेद्दिशः}


\twolineshloka
{तादृशं तपसा युक्तं प्रदीप्तमिव पावकम्}
{कथमस्मद्विधा नारी जितेन्द्रियमभिस्पृशेत्}


\twolineshloka
{हुताशनमुखं दीप्तं सूर्यचन्द्राक्षितारकम्}
{कालजिह्वं सुरश्रेष्ठ कथमस्मद्विधा स्पृशेत्}


\twolineshloka
{यमश्च सोमश्च महर्षयश्चसाध्या विश्वे वालस्विल्याश्च सर्वे}
{एतेऽपि यस्योद्विजन्ते प्रभावा-त्तस्मात्कस्मान्मादृशी नोद्विजेत}


\twolineshloka
{त्वयैवमुक्ता च कथं समीप-मृषेर्न गच्छेयमहं सुरेन्द्र}
{रक्षां च मे चिन्तय देवराजयथा त्वदर्थं रक्षिताऽहं चरेयम्}


\twolineshloka
{कामं तु मे मारुतस्तत्र वासःप्रक्रीडिताया विवृणोतु देव}
{भवेच्च मे मन्मथस्तत्र कार्येसहायभूतस्तु तव प्रसादात्}


\twolineshloka
{वनाच्च वायुः सुरभिः प्रवाया-त्तस्मिन्काले तमृषिं लोभयन्त्याः}
{तथेत्युक्त्वा विहिते चैव तस्मिं-स्ततो ययौ साऽऽश्रमं कौशिकस्य}


\chapter{अध्यायः ९३}
\twolineshloka
{कण्व उवाच}
{}


\twolineshloka
{एवमुक्तस्तया शक्रः संदिदेश सदागतिम्}
{प्रातिष्ठत तदा काले मेनका वायुना सह}


\twolineshloka
{अथापश्यद्वरारोहा तपसा दग्धकिल्बिषम्}
{मिश्वामित्रं तप्यमानं मेनका भीरुराश्रमे}


\twolineshloka
{अभिवाद्य ततः सा तं प्राक्रीडदृषिसन्निधौ}
{अपोवाह च वासोऽस्या मारुतः शशिसंनिभम्}


\twolineshloka
{सागच्छत्त्वरिता भूमिं वासस्तदभिलिप्सती}
{कुत्सयन्तीव सव्रीडं मारुतं वरवर्णिनी}


\twolineshloka
{पश्यतस्तस्य राजर्षेरप्यग्निसमतेजसः}
{विश्वामित्रस्ततस्तां तु विषमस्थामनिन्दिताम्}


\twolineshloka
{गृद्धां वाससि संभ्रान्तां मेनकां मुनिसत्तमः}
{अनिर्देश्यवयोरूपामपश्यद्विवृतां तदा}


\twolineshloka
{तस्या रूपगुणान्दृष्ट्वा स तु विप्रर्षभस्तदा}
{चकार भावं संसर्गे तया कामवशं गतः}


\twolineshloka
{न्यमन्त्रयत चाप्येनां सा चाप्यैच्छदनिन्दिता}
{तौ तत्र सुचिरं कालमुभौ व्यवहरतां तदा}


\twolineshloka
{रममाणौ यथाकामं यतैकदिवसं तथा}
{`एवं वर्षसहस्राणामतीतं नान्वचिन्तयत्}


\twolineshloka
{कामक्रोधावजितवान्मुनिर्नित्यं समाहितः}
{चिरार्जितस्य तपसः क्षयं स कृतवान्मुनिः}


\twolineshloka
{तपसः संक्षयादेव मुनिर्मोहं समाविशत्}
{मोहाभिभूतः क्रोधात्मा ग्रसन्मूलफलान्मुनिः}


\twolineshloka
{पादैर्जलरवं कृत्वा अन्तर्द्वीपे कुटीं गतः}
{मेनका गन्तुकामा तु शुश्राव जलनिस्वनम्}


\twolineshloka
{तपसा दीप्तवीर्योऽसावाकाशादेति याति च}
{अद्य संज्ञां विजानामि ययाऽद्य तपसः क्षयः}


\twolineshloka
{गन्तुं न युक्तमित्युक्त्वा ऋतुस्नाताथ मेनका}
{कामरागाभिभूतस्य मुनेः पार्स्वं जगाम ह ॥'}


\twolineshloka
{जनयामास स मुनिर्मेनकायां शकुन्तलाम्}
{प्रस्थे हिमवतो रम्ये मालिनीमभितो नदीम्}


\twolineshloka
{`देवगर्भोपमां बालां सर्वाभरणभूषिताम्}
{शयानां शयने रम्ये मेनका वाक्यमब्रवीत्}


\twolineshloka
{महर्षेरुग्रतपसस्तेजस्त्वमसि भामिनि}
{तस्मात्स्वर्गं गमिष्यामि देवकार्यार्थमागता ॥'}


\twolineshloka
{जातमुत्सृज्य तं गर्भं मेनका मालिनीमनु}
{कृतकार्या ततस्तूर्णमगच्छच्छक्रसंसदम्}


\threelineshloka
{तं वने विजने गर्भं सिंहव्याघ्रसमाकुले}
{दृष्ट्वा शयानं शकुनाः समन्तात्पर्यवारयन्}
{नेमां हिंस्युर्वने बालां क्रव्यादा मांसगृद्धिनः}


\twolineshloka
{पर्यरक्षन्त तां तत्र शकुन्ता मेनकात्मजाम्}
{उपस्प्रष्टुं गतश्चाहमपश्यं शयितामिमाम्}


\threelineshloka
{निर्जने विपिने रम्ये शकुन्तैः परिवारिताम्}
{`मां दृष्ट्वैवाभ्यपद्यन्त पादयोः पतिता द्विजाः}
{अब्रुञ्शकुनाः सर्वे कलं मधुरभाषिणः}


\twolineshloka
{विश्वामित्रसुतां ब्रह्मन्न्यासभूतां भरस्व वै}
{कामक्रोधावजितवान्सखा ते कौशिकीं गतः}


\twolineshloka
{तस्मात्पोषय तत्पुत्रीं दयावानिति तेऽब्रुवन्}
{सर्वभूतरुतज्ञोऽहं दयावान्सर्वजन्तुषु ॥'}


% Check verse!
आनयित्वा ततश्चैनां दुहितृत्वे न्यवेशयम्
\twolineshloka
{शरीरकृत्प्राणदाता यस्य चान्नानि भुञ्जते}
{क्रमेणैते त्रयोऽप्युक्ताः पितरो धर्मशासने}


\twolineshloka
{निर्जने तु वने यस्माच्छकुन्तैः परिवारिता}
{शकुन्तलेति नामास्याः कृतं चापि ततो मया}


\threelineshloka
{एवं दुहितरं विद्धि मम विप्र शकुन्तलाम्}
{शकुन्तला च पितरं मन्यते मामनिन्दिता ॥शकुन्तलोवाच}
{}


\twolineshloka
{एतदाचष्ट पृष्टः सन्मम जन्म महर्षये}
{सुतां कण्वस्य मामेवं विद्धि त्वं मनुजाधिप}


\twolineshloka
{कण्वं हि पितरं मन्ये पितरं स्वमजानती}
{इति ते कथितं राजन्यथावृत्तं श्रुतं मया}


\chapter{अध्यायः ९४}
\twolineshloka
{दुष्यन्त उवाच}
{}


\twolineshloka
{सुव्यक्तं राजपुत्री त्वं यथा कल्याणि भाषसे}
{भार्या मे भव सुश्रोणि ब्रूहि किं करवाणि ते}


\twolineshloka
{सुवर्णमालां वासांसि कुण्डले परिहाटके}
{नानापत्तनजे शुभ्रे मणिरत्ने च शोभने}


\twolineshloka
{आहरामि तवाद्याहं निष्कादीन्यजिनानि च}
{सर्वं राज्यं तवाद्यास्तु भार्या मे भव शोभने}


\threelineshloka
{गान्धर्वेण च मां भीरु विवाहेनैहि सुन्दरि}
{विवाहानां हि रम्भोरु गान्धर्वः श्रेष्ठ उच्यते ॥शकुन्तलोवाच}
{}


\twolineshloka
{फलाहारो गतो राजन्पिता मे इत आश्रमात्}
{मुहूर्तं संप्रतीक्षस्व स मां तुभ्यं प्रदास्यति}


\twolineshloka
{`पिता हि मे प्रभुर्नित्यं दैवतं परमं मम}
{यस्मै मां दास्यति पिता स मे भर्ता भविष्यति}


\twolineshloka
{पिता रक्षति कौमारे भर्ता रक्षति यौवने}
{पुत्रस्तु स्थाविरे भावे न स्त्री स्वातन्त्र्यमर्हति}


\twolineshloka
{समन्यमाना राजेन्द्र पितरं मे तपस्विनम्}
{अधर्मेण हि धर्मिष्ठ कथं वरमुपास्महे}


\fourlineindentedshloka
{दुष्यन्त उवाच}
{मामैवं वद कल्याणि तपोराशिं दमात्मकम्}
{शकुन्तलोवाच}
{मन्युप्रहरणा विप्रा न विप्राः शस्त्रपाणयः}


\twolineshloka
{मन्युना घ्नन्ति ते शत्रून्वज्रेणेन्द्र इवासुरान्}
{अग्निर्दहति तेजोभिः सूर्यो दहति रश्मिभिः}


\threelineshloka
{राजा दहति दण्डेन ब्राह्मणो मन्युना दहेत्}
{क्रोधिता मन्युना घ्नन्ति वज्रपाणिरिवासुरान् ॥दुष्यन्त उवाच}
{}


\threelineshloka
{जाने भद्रे महर्षिं तं तस्य मन्युर्न विद्यते}
{'इच्छामि त्वां वरारोहे भजमानामनिन्दिते}
{त्वदर्थं मां स्थितं विद्धि त्वद्गतं हि मनो मम}


\twolineshloka
{आत्मनो बन्धुरात्मैव गतिरात्मैव चात्मनः}
{आत्मनैवात्मनो दानं कर्तुमर्हसि धर्मतः}


\twolineshloka
{अष्टावेव समासेन विवाहा धर्मतः स्मृताः}
{ब्राह्मो दैवस्तथैवार्षः प्राजापत्यस्तथाऽऽसुरः}


\twolineshloka
{गान्धर्वो राक्षसश्चैव पैशाचश्चाष्टमः स्मृतः}
{तेषां धर्म्यान्यथापूर्वं मनुः स्वायंभुवोऽब्रवीत्}


\twolineshloka
{प्रशस्तांश्चतुरः पूर्वान्ब्राह्मणस्योपधारय}
{षडानुपूर्व्या क्षत्रस्य विद्धि धर्म्याननिन्दिते}


\twolineshloka
{राज्ञां तु राक्षसोऽप्युक्तो विट्शूद्रेष्वासुरः स्मृतः}
{पञ्चानां तु त्रयो धर्म्या अधर्म्यौ द्वौ स्मृताविह}


\twolineshloka
{पैशाच आसुरश्चैव न कर्तव्यौ कदाचन}
{अनेन विधिना कार्यो धर्मस्यैषा गतिः स्मृता}


\twolineshloka
{गान्धर्वराक्षसौ क्षत्रे धर्म्यौ तौ मा विशङ्किथाः}
{पृथग्वा यदि वा मिश्रौ कर्तव्यौ नात्र संशयः}


\threelineshloka
{सा त्वं मम सकामस्य सकामा वरवर्णिनी}
{गान्धर्वेण विवाहेन भार्या भवितुमर्हसि ॥शकुन्तलोवाच}
{}


\twolineshloka
{यदि धर्मपथस्त्वेव यदि चात्मा प्रभुर्मम}
{प्रदाने पौरवश्रेष्ठ शृणु मे समयं प्रभो}


\twolineshloka
{सत्यं मे प्रतिजानीहि यथा वक्ष्याम्यहं रहः}
{मयि जायेत यः पुत्रः स भवेत्त्वदनन्तरः}


\threelineshloka
{युवराजो महाराज सत्यमेतद्ब्रवीमि ते}
{यद्येतदेवं दुष्यन्त अस्तु मे सङ्गमस्त्वया ॥वैशंपायन उवाच}
{}


\threelineshloka
{`तस्यास्तु सर्वं संश्रुत्य यथोक्तं स विशांपतिः}
{दुष्यन्तः पुनरेवाह यद्यदिच्छसि तद्वद ॥शकुन्तलोवाच}
{}


\twolineshloka
{ख्यातो लोकप्रवादोयं विवाह इति शास्त्रतः}
{वैवाहिकीं क्रियां सन्तः प्रशंसन्ति प्रजाहिताम्}


\twolineshloka
{लोकप्रवादशान्त्यर्थं विवाहं विधिना कुरु}
{सन्त्यत्र यज्ञपात्राणि दर्भाः सुमनसोऽक्षताः}


\twolineshloka
{यथा युक्तो विवाहः स्यात्तथा युक्ता प्रजा भवेत्}
{तस्मादाज्यं हविर्लाजाः सिकता ब्राह्मणास्तव}


\threelineshloka
{वैवाहिकानि चान्यानि समस्तानीह पार्थिव}
{दुरुक्तमपि राजेन्द्र क्षन्तव्यं धर्मकारणात् ॥वैशंपायन उवाच}
{'}


\twolineshloka
{एवमस्त्विति तां राजा प्रत्युवाचाविचारयन्}
{अपि च त्वां हि नेष्यामि नगरं स्वं शुचिस्मिते}


\twolineshloka
{यथा त्वमर्हा सुश्रोणि मन्यसे तद्ब्रवीमि ते}
{एवमुक्त्वा स राजर्षिस्तामनिन्दितगामिनीम्}


\twolineshloka
{`पुरोहितं समाहूय वचनं युक्तमब्रवीत्}
{राजपुत्र्या यदुक्तं वै न वृथा कर्तुमुत्सहे}


\twolineshloka
{क्रियाहीनो हि न भवेन्मम पुत्रो महाद्युतिः}
{तथा कुरुष्व शास्त्रोक्तं विवाहं मा चिरंकुरु}


\twolineshloka
{एवमुक्तो नृपतिना द्विजः परमयन्त्रितः}
{शोभनं राजराजेति विधिना कृतवान्द्विजः}


\twolineshloka
{शासनाद्विप्रमुख्यस्य कृतकौतुकमङ्गलः}
{'जग्राह विधिवत्पाणावुवास च तया सह}


\twolineshloka
{विश्वास्य चैनां स प्रायादब्रवीच्च पुनः पुनः}
{प्रेषयिष्ये तवार्थाय वाहिनीं चतुरङ्गिणीम्}


\twolineshloka
{`त्रैविद्यवृद्धैः सहितां नानाराजजनैः सह}
{शिबिकासहस्रैः सहिता वयमायान्ति बान्धवाः}


\twolineshloka
{मूकाश्चैव किराताश्च कुब्जा वामनकैः सह}
{सहिताः कञ्चुकिवरैर्वाहिनी सूतमागधैः}


\twolineshloka
{शङ्खदुन्दुभिनिर्घोषैर्वनं च समुपैष्यति}
{तथा त्वामानयिष्यामि नगरं स्वं शुचिस्मिते}


\threelineshloka
{अन्यथा त्वां न नेष्यामि स्वनिवेशमसत्कृताम्}
{सर्वमङ्गलसत्कारैः सुभ्रु सत्यं करोमि ते ॥वैशंपायन उवाच}
{}


\twolineshloka
{एवमुक्त्वा स राजर्षिस्तामनिन्दितगामिनीम्}
{परिष्वज्य च बाहुभ्यां स्मितपूर्वमुदैक्षत}


\twolineshloka
{प्रदक्षिणीकृतां देवीं पुनस्तां परिषस्वजे}
{शकुन्तला सा सुमुखी पपात नृपपादयोः}


\twolineshloka
{तां देवीं पुनरुत्थाप्य मा शुचेति पुनः पुनः}
{शपेयं सुकृतेनैव प्रापयिष्ये नृपात्मजे ॥'}


\twolineshloka
{इति तस्याः प्रतिश्रुत्य स नृपो जनमेजय}
{मनसा चिन्तयन्प्रायात्काश्यपं प्रति पार्थिव}


\threelineshloka
{भगवांस्तपसा युक्तः श्रुत्वा किं नु करिष्यति}
{तं न प्रसाद्यागतोऽहं प्रसीदेति द्विजोत्तमम्}
{एवं संचिन्तयन्नेव प्रविवेश स्वकं पुरम्}


\twolineshloka
{ततो मुहूर्ते याते तु कण्वोऽप्याश्रममागमत्}
{शकुन्तला च पितरं ह्रिया नोपजगाम तम्}


\twolineshloka
{`शङ्कितेव च विप्रर्षिमुपचक्राम सा शनैः}
{ततोऽस्य भारं जग्राह आसनं चाप्यकल्पयत्}


\twolineshloka
{प्राक्षालयच्च सा पादौ काश्यपस्य महात्मनः}
{न चैनं लज्जयाऽशक्नोदक्षिभ्यामभिवीक्षितुम्}


\twolineshloka
{शकुन्तला च सव्रीडा तमृषिं नाभ्यभाषत}
{तस्मात्स्वधर्मात्स्खलिता भीता सा भरतर्षभ}


\threelineshloka
{अभवद्दोषदर्शित्वाद्ब्रह्मचारिण्ययन्त्रिता}
{स तदा व्रीडितां दृष्ट्वा ऋषिस्तां प्रत्यभाषत ॥कण्व उवाच}
{}


\threelineshloka
{सव्रीडैव च दीर्घायुः पुरेव भविता न च}
{वृत्तं कथय रम्भोरु मा त्रासं च प्रकल्पय ॥वैशंपायन उवाच}
{}


\twolineshloka
{ततः प्रक्षाल्य पादौ सा विश्रान्तं पुनरब्रवीत्}
{निधाय कामं तस्यर्षेः कन्दानि च फलानि च}


\twolineshloka
{ततः संवाह्य पादौ सा विश्रान्तं वेदिमध्यमा}
{शकुन्तला पौरवाणां दुष्यन्तं जग्मुषी पतिम्}


\threelineshloka
{ततः कृच्छ्रादतिशुभा सव्रीडा श्रमती तदा}
{सगद्गदमुवाचेदं काश्यपं सा शुचिस्मिता ॥शकुन्तलोवाच}
{}


\twolineshloka
{राजा ताताजगामेह दुष्यन्त इलिलात्मजः}
{मया पतिर्वृतो योऽसौ दैवयोगादिहागतः}


\twolineshloka
{तस्य तात प्रसीद त्वं भर्ता मे सुमहायशाः}
{अतः सर्वं तु यद्वृत्तं दिव्यज्ञानेन पश्यसि}


\threelineshloka
{अभयं क्षत्रियकुले प्रसादं कर्तुमर्हसि}
{वैशंपायन उवाच}
{चक्षुषा स तु दिव्येन सर्वं विज्ञाय काश्यपः}


\threelineshloka
{ततो धर्मिष्ठतां मत्वा धर्मे चास्खलितं मनः}
{उवाच भगवान्प्रीतस्तद्वृत्तं सुमहातपाः ॥कण्व उवाच}
{}


\twolineshloka
{एवमेतन्मया ज्ञातं दृष्टं दिव्येन चक्षुषा}
{त्वयाऽद्य राजान्वयया मामनादृत्य यत्कृतम्'}


\twolineshloka
{पुंसा सह समायोगो न स धर्मोपघातकः}
{न भयं विद्यते भद्रे मा शुचः सुकृतं कृतम्}


\twolineshloka
{क्षत्रियस्य तु गान्धर्वो विवाहः श्रेष्ठ उच्यते}
{सकामायाः सकामेन निमन्त्रः श्रेष्ठ उच्यते}


\twolineshloka
{`किं पुनर्विधिवत्कृत्वा सुप्रजस्त्वमवाप्स्यसि}
{'धर्मात्मा च महात्मा च दुष्यन्तः पुरुषोत्तमः}


\twolineshloka
{अभ्यागच्छत्पतिर्यस्त्वां भजमानां शकुन्तले}
{महात्मा जनिता लोके पुत्रस्तव महायशाः}


\twolineshloka
{स च सर्वां समुद्रान्तां कृत्स्नां भोक्ष्यति मेदिनीम्}
{परं चाभिप्रयातस्य चक्रं तस्य महात्मनः}


\twolineshloka
{भविष्यत्यप्रतिहतं सततं चक्रवर्तिनः}
{प्रसन्न एव तस्याहं त्वकृते वरवर्णिनि}


\fourlineindentedshloka
{ऋतवो बहवस्ते वै गता व्यर्थाः शुचिस्मिते}
{सार्थकं सांप्रतं ह्येतन्न च पाप्मास्ति तेऽनघे}
{गृहाण च वरं मत्तस्तत्कृते यदभीप्सितम् ॥शकुन्तलोवाच}
{}


\fourlineindentedshloka
{मया पतिर्वृतो योऽसौ दुष्यन्तः पुरुषोत्तमः}
{मम चैव पतिर्दृष्टो देवतानां समक्षतः}
{तस्मै ससचिवाय त्वं प्रसादं कर्तुमर्हसि ॥वैशंपायन उवाच}
{}


\twolineshloka
{इत्येवमुक्त्वा मनसा प्रणिधाय मनस्विनी}
{ततो धर्मिष्ठतां वव्रे राज्ये चास्खलनं तथा}


\twolineshloka
{शकुन्तलां पौरवाणां दुष्यन्तहितकाम्यया}
{`एवमस्त्विति तां प्राह कण्वो धर्मभृतां वरः}


\twolineshloka
{पस्पर्श चापि पाणिभ्यां सुतां श्रीमिवरूपिणीम् ॥कण्व उवाच}
{}


\threelineshloka
{अद्यप्रभृति देवी त्वं दुष्यन्तस्य महात्मनः}
{पतिव्रतानां या वृत्तिस्तां वृत्तिमनुपालय ॥वैशंपायन उवाच}
{}


\twolineshloka
{इत्येवमुक्त्वा धर्मात्मा तां विशुद्ध्यर्थमस्पृशत्}
{स्पृष्टमात्रे शरीरे तु परं हर्षमवाप सा ॥'}


\chapter{अध्यायः ९५}
\twolineshloka
{वैशंपायन उवाच}
{}


\twolineshloka
{प्रतिज्ञाय च दुष्यन्ते प्रतियाते दिने दिने}
{`गर्भश्च ववृधे तस्यां राजपुत्र्यां महात्मनः ॥'}


\twolineshloka
{शकुन्तला चिन्तयन्ती राजानं कार्यगौरवात्}
{दिवारात्रमनिद्रैव स्नानभोजनवर्जिता}


\twolineshloka
{राजप्रेषणिका विप्राश्चतुरङ्गबलान्विताः}
{अद्य श्वो वा परश्वो वा समायान्तीति निस्चिता}


\twolineshloka
{दिनान्पक्षानृतून्मासानयनानि च सर्वशः}
{गण्यमानानि वर्षाणि व्यतीयुस्त्रीणि भारत}


\threelineshloka
{त्रिषु वर्षेषु पूर्णेषु ऋषेर्वचनगौरवात्}
{ऋषिपत्न्यः सुबहुशो हेतुमद्वाक्यमब्रुवन् ॥ऋषिपत्न्य ऊचुः}
{}


\twolineshloka
{शृणु भद्रे लोकवृत्तं श्रुत्वा यद्रोचते तव}
{तत्कुरुष्व हितं देवि नावमान्यं गुरोर्वचः}


\twolineshloka
{देवानां दैवतं विष्णुर्विप्राणामग्निर्ब्रह्म च}
{नारीणां दैवतं भर्ता लोकानां ब्राह्मणो गुरुः}


\threelineshloka
{सूतिकाले प्रसूष्वेति भगवांस्ते पिताऽब्रवीत्}
{करिष्यामीति कर्तव्यं तदा ते सुकृतं भवेत् ॥वैशंपायन उवाच}
{}


\twolineshloka
{पत्नीनां वचनं श्रुत्वा साधु साध्वित्यचिन्तयत्}
{'गर्भं सुषाव वामोरूः कुमारममितौजसम्}


\twolineshloka
{त्रिषु वर्षेषु पूर्णेषु प्राजायत शकुन्तला}
{रूपौदार्यगुणोपेतं दौष्यन्तिं जनमेजय}


\twolineshloka
{`जाते' तस्मिन्नन्तरिक्षात्पुष्पवृष्टिः पपात ह}
{देवदुन्दुभयो नेदुर्ननृतुश्चाप्सरोगणाः}


\twolineshloka
{गायद्भिर्मधुरं तत्र देवैः शक्रोऽभ्युवाच ह}
{शकुन्तले तव सुतश्चक्रवर्ती भविष्यति}


\twolineshloka
{बलं तेजश्च रूपं च न समं भुवि केनचित्}
{आहर्ता वाजिमेधस्य शतसङ्ख्यस्य पौरवः}


\twolineshloka
{अनेकारपि साहस्रै राजसूयादिभिर्मखैः}
{स्वार्थं ब्राह्मणसात्कृत्वा दक्षिणाममितां ददत्}


\twolineshloka
{देवतानां वचः श्रुत्वा कण्वाश्रमनिवासिनः}
{सभाजयन्तः कण्वस्य सुतां सर्वे महर्षयः}


\threelineshloka
{शकुन्तला च तच्छ्रुत्वा परं हर्षमवाप सा}
{द्विजानाहूय मुनिभिः सत्कृत्य च महायशाः}
{'}


\twolineshloka
{जातकर्मादिसंस्कारं कण्वः पुण्यवतां वरः}
{तस्याथ कारयामास वर्धमानस्य चासकृत्}


\twolineshloka
{यथाविधि यथान्यायं क्रियाः सर्वास्त्वकारयत्}
{दन्तैः शुक्लैः शिखरिभिःसिंहसंहननोऽभवत्}


\twolineshloka
{चक्राङ्कितकरः श्रीमान्स्वयं विष्णुरिवापरः}
{`चतुष्किष्कुर्महातेजा महामूर्धा महाबलः ॥'}


\twolineshloka
{कुमारो देवगर्भाभः स तत्राशु व्यवर्धत}
{`ऋषेर्भयात्तु दुष्यन्तः स्मरन्नैवाह्वयत्तदा}


\twolineshloka
{गते काले तु महति न सस्मार तपोधनाम्}
{'षड्वर्षेषु ततो बालः कण्वाश्रमपदं प्रति}


\twolineshloka
{व्याघ्रान्सिंहान्वराहांश्च वृकांश्च महिषांस्तथा}
{`ऋक्षांश्चाभ्यहनद्व्यालान्पद्भ्यामाश्रमपीडकान्}


\twolineshloka
{बलाद्भुजाभ्यां संगृह्य बलवान्संनियम्य च}
{'बद्ध्वा वृक्षेषु दौष्यन्तिराश्रमस्य समन्ततः}


\twolineshloka
{आरुरोह द्रुमांश्चैव क्रीडन्स्म परिधावति}
{`वनं च लोडयामास सिंहव्याघ्रगणैर्वृतम्}


\twolineshloka
{ततश्च राक्षसान्सर्वान्पिशाचांश्च रिपून्रणे}
{मुष्टियुद्धेन तान्हत्वा ऋषीनाराधयत्तदा}


\twolineshloka
{कश्चिद्दितिसुतस्तं तु हन्तुकामो महाबलः}
{वध्यमानांस्तु दैतेयानमर्षी तं समभ्ययात्}


\twolineshloka
{तमागतं प्रहस्यैव बाहुभ्यां परिगृह्य च}
{दृढं चाबध्य बाहुभ्यां पीडयामास तं तदा}


\twolineshloka
{मर्दितो न शशाकास्मान्मोचितुं बलवत्तया}
{प्राक्रोशद्भैरवं तत्र द्वारेभ्यो निःसृतं त्वसृक्}


\twolineshloka
{तेन शब्देन वित्रस्ता मृगाः सिंहादयो गणाः}
{सुस्रुवुश्च शकृन्मूत्रमाश्रमस्थाश्च सुस्रुवुः}


\twolineshloka
{निरसुं जानुभिः कृत्वा विससर्ज च सोऽपतत्}
{तद्दृष्ट्वा विस्मयं जग्मुः कुमारस्य विचेष्टितम्}


\twolineshloka
{नित्यकालं वध्यमाना दैतेया राक्षसैः सह}
{कुमारस्य भयादेव नैव जग्मुस्तदाश्रमम् ॥'}


\twolineshloka
{ततोऽस्य नाम चक्रुस्ते कण्वाश्रमनिवासिनः}
{कण्वेन सहिताः सर्वे दृष्ट्वा कर्मातिमानुषम्}


\twolineshloka
{अस्त्वयं सर्वदमनः सर्वं हि दमयत्यसौ}
{स सर्वदमनो नाम कुमारः समपद्यत}


\twolineshloka
{विक्रमेणौजसा चैव बलेन च समन्वितः}
{`अप्रेषयति दुष्यन्ते महिष्यास्तनयस्य च}


\twolineshloka
{पाण्डुभावपरीताङ्गीं चिन्तया समभिप्लुताम्}
{लम्बालकां कृशां दीनां तथा मलिनवाससम्}


\twolineshloka
{`शकुन्तलां च संप्रेक्ष्य प्रदध्यौ स मुनिस्तदा}
{शास्त्राणि सर्ववेदाश्च द्वादशाब्दस्य चाभवन्'}


\chapter{अध्यायः ९६}
\twolineshloka
{वैशंपायन उवाच}
{}


\twolineshloka
{तं कुमारमृषिर्दृष्ट्वा कर्म चास्यातिमानुषम्}
{समयो यौवराज्याय इत्यनुध्याय स द्विजः}


\threelineshloka
{`शकुन्तलां समाहूय कण्वो वचनमब्रवीत्}
{कण्व उवाच}
{शृणु भद्रे मम सुते मम वाक्यं शुचिस्मिते}


\twolineshloka
{पतिव्रतानां नारीणां विशिष्टमिति चोच्यते}
{पतिशुश्रूषणं पूर्वं मनोवाक्कायचेष्टितैः}


\twolineshloka
{अनुज्ञाता मया पूर्वं पूजयैतद्व्रतं तव}
{एतेनैव च वृत्तेन पुण्याँल्लोकानवाप्य च}


\twolineshloka
{तस्यान्ते मानुषे लोके विशिष्टां तप्स्यसे श्रियम्}
{तस्माद्भद्रेऽद्य यातव्यं समीपं पौरवस्य ह}


\twolineshloka
{स्वयं नायाति मत्वा ते गतं कालं शुचिस्मिते}
{गत्वाऽऽराधय राजानं दुष्यन्तं हितकाम्यया}


\twolineshloka
{दौष्यन्तिं यौवराज्यस्थं दृष्ट्वा प्रीतिमवाप्स्यसि}
{देवतानां गुरूणां च क्षत्रियाणां च भामिनि}


\twolineshloka
{भर्तॄणां च विशेषेम हितं संगमनं भवेत्}
{तस्मात्पुत्रि कुमारेण गन्तव्यं मत्प्रियेप्सया}


\twolineshloka
{प्रतिवाक्यं न दद्यास्त्वं शपिता मम पादयोः ॥वैशंपायन उवाच}
{}


\twolineshloka
{एवमुक्त्वा सुतां तत्र पौत्रं कण्वोऽभ्यभाषत}
{परिष्वज्य च बाहुभ्यां मूर्ध्न्युपाघ्राय पौरवम्}


\twolineshloka
{सोमवंशोद्भवो राजा दुष्यन्त इति विश्रुतः}
{तस्याग्रमहिषी चैषा तव माता शुचिव्रता}


\twolineshloka
{गन्तुकामा भर्तृपार्श्वं त्वया सह सुमध्यमा}
{गत्वाऽभिवाद्य राजनं यौवराज्यमवाप्स्यसि}


\twolineshloka
{स पिता तव राजेन्द्रस्तस्य त्वं वशगो भव}
{पितृपैतामहं राज्यमातिष्ठस्व स्वभावतः}


\twolineshloka
{तस्मिन्काले स्वराज्यस्थो मामनुस्मर पौरव ॥वैशंपायन उवाच}
{}


\twolineshloka
{अभिवाद्य मुनेः पादौ पौरवो वाक्यमब्रवीत्}
{त्वं पिता मम विप्रर्षे त्वं माता त्वं गतिश्च मे}


\twolineshloka
{न चान्यं पितरं मन्ये त्वामृते तु महातपः}
{तव शुश्रूषणं पुण्यमिह लोके परत्र च}


\twolineshloka
{शकुन्तला भर्तृकामा स्वयं यातु यथेष्टतः}
{अहं सुश्रूषणपरः पादमूले वसामि वः}


\twolineshloka
{क्रीडां व्यालमृगैः सार्धं करिष्ये न पुरा यथा}
{त्वच्छासनपरो नित्यं स्वाध्यायं च करोम्यहम्}


\twolineshloka
{एवमुक्त्वा तु संश्लिष्य पादौ कण्वस्य तिष्ठतः}
{तस्य तद्वचनं श्रुत्वा प्ररुरोद शकुन्तला}


\twolineshloka
{स्नेहात्पितुश्च पुत्रस्य हर्षशोकसमन्विता}
{निशाम्य रुदतीमार्तां दौष्यन्तिर्वाक्यमब्रवीत्}


\threelineshloka
{श्रुत्वा भगवतो वाक्यं किं रोदिषि शकुन्तले}
{गन्तव्यं काल्य उत्थाय भर्तृप्रीतिस्ववास्ति चेत् ॥शकुन्तलोवाच}
{}


\twolineshloka
{एकस्तु कुरुते पापं फलं भुङ्क्ते महाजनः}
{मया निवारितो नित्यं न करोषि वचो मम}


\twolineshloka
{निःसृतान्कुञ्जरान्नित्यं बाहुभ्यां संप्रमथ्य वै}
{वनं च लोडयन्नित्यं सिंहव्याघ्रगणैर्वृतम्}


\twolineshloka
{एवंविधानि चान्यानि कृत्वा वै पुरुनन्दन}
{रुषितो भगवांस्तात तस्मादावां विवासितौ}


\threelineshloka
{नाहं गच्छामि दुष्यन्तं नास्मि पुत्र हितैषिणी}
{पादमूले वसिष्यामि महर्षेर्भावितात्मनः ॥वैशंपायन उवाच}
{}


\fourlineindentedshloka
{एवमुक्त्वा तु रुदती पपात मुनिपादयोः}
{एवं विलपतीं कण्वश्चानुनीय च हेतुभिः}
{पुनः प्रोवाच भगवानानृशंस्याद्धितं वचः ॥कण्व उवाच}
{}


\twolineshloka
{शकुन्तले शृणुष्वेदं हितं पथ्यं च भामिनि}
{पतिव्रताभावगुणान्हित्वा साध्यं न किंचन}


\twolineshloka
{प्रतिव्रतानां देवा वै तुष्टाः सर्वरप्रदाः}
{प्रसादं च करिष्यन्ति आपदो मोक्षयन्ति च}


\threelineshloka
{पतिप्रसादात्पुण्यं च प्राप्नुवन्ति न चाशुभम्}
{तस्माद्गत्वा तु राजानमाराधय शुचिस्मिते ॥वैशंपायन उवाच}
{}


\twolineshloka
{शकुन्तलां तथोक्त्वा वै शाकुन्तलमथाब्रवीत्}
{दौहित्रो मम पौत्रस्त्वमिलिलस्य महात्मनः}


\twolineshloka
{शृणुष्व वचनं सत्यं प्रब्रवीमि तवानघ}
{मनसा भर्तृकामा वै वाग्भिरुक्त्वा पृथग्विधम्}


\twolineshloka
{गन्तुं नेच्छति कल्याणी तस्मात्तात वहस्व वै}
{शक्तस्त्वं प्रतिगन्तुं च मुनिभिः सह पौरव ॥'}


\twolineshloka
{इत्युक्त्वा सर्वदमनं कण्वः शिष्यानथाब्रवीत्}
{शकुन्तलामिमां शीग्रं सपुत्रामाश्रमादितः}


\twolineshloka
{भर्तुः प्रापयताभ्याशं सर्वलक्षणपूजिताम्}
{नारीणां चिरवासो हि बान्धवेषु न रोचते}


\twolineshloka
{कीर्तिचारित्रधर्म्नस्तस्मान्नयत मा चिरम् ॥`वैशंपायन उवाच}
{}


\twolineshloka
{धर्माभिपूजितं पुत्रं काश्यपेन निशाम्य तु}
{काश्यपात्प्राप्य चानुज्ञां मुमुदे च शकुन्तला}


\threelineshloka
{कण्वस्य वचनं श्रुत्वा प्रतिगच्छेति चासकृत्}
{तथेत्युक्त्वा तु कण्वं च मातरं पौरवोऽब्रवीत्}
{किं चिरायसि मातस्त्वं गमिष्यामो नृपालयम्}


\twolineshloka
{एवमुक्त्वा तु तां देवीं दुष्यन्तस्य महात्मनः}
{अभिवाद्य मुनेः पादौ गन्तुमैच्छत्स पौरवः}


\twolineshloka
{शकुन्तला च पितरमभिवाद्य कृताञ्जलिः}
{प्रदक्षिणीकृत्य तदा पितरं वाक्यमब्रवीत्}


\twolineshloka
{अज्ञानान्मे पिता चेति दुरुक्तं वापि चानृतम्}
{अकार्यं वाप्यनिष्टं वा क्षन्तुमर्हति तद्भवान्}


\twolineshloka
{एवमुक्तो नतशिरा मुनिर्नोवाच किंचन}
{मनुष्यभावात्कण्वोऽपि मुनिरश्रूण्यवर्तयत्}


\twolineshloka
{अब्भक्षान्वायुभक्षांश्च शीर्णपर्णाशनान्मुनीन्}
{फलमूलाशिनो दान्तान्कृशान्धमनिसंततान्}


\twolineshloka
{व्रतिनो जटिलान्मुण्डान्वल्कलाजिनसंवृतान्}
{समाहूय मुनिः कण्वः कारुण्यादिदमब्रवीत्}


\twolineshloka
{मया तु लालिता नित्यं मम पुत्री यशस्विनी}
{वने जाता विवृद्धा च न च जानाति किंचन}


\twolineshloka
{आश्रमात्तु पथा सर्वैर्नीयतां क्षत्रियालयम्}
{द्वितीययोजने विप्राः प्रतिष्ठानं प्रतिष्ठितम्}


\twolineshloka
{प्रतिष्ठाने पुरे राजा शाकुन्तलपितामहः}
{अध्युवास चिरं कालमुर्वश्या सहितः पुरा}


\twolineshloka
{अनूपजाङ्गलयुतं धनधान्यसमाकुलम्}
{प्रतिष्ठितं पुरवरं गङ्गायामुनसङ्गमे}


\threelineshloka
{तत्र सङ्गममासाद्य स्नात्वा हुतहुताशनाः}
{शाकमूलफलाहारा निवर्तध्वं तपोधनाः}
{अन्यथा तु भवेद्विप्रा अध्वनो गमने श्रमः ॥'}


\twolineshloka
{तथेत्युक्त्वा च ते सर्वे प्रातिष्ठन्त महौजसः}
{`शकुन्तलां पुरस्कृत्य दुष्यन्तस्य पुरं प्रति}


\twolineshloka
{गृहीत्वा चामरप्रख्यं पुत्रं कमललोचनम्}
{आजग्मुश्च पुरं रम्यं दुष्यन्ताध्युषितं वनात्}


\twolineshloka
{शकुन्तलां समादाय मुनयो धर्मवत्सलाः}
{ते वनानि नदीः शैलान्गिरिप्रस्रवणानि च}


\twolineshloka
{कन्दराणि नितम्बांश्च राष्ट्राणि नगराणि च}
{आश्रमाणि च पुण्यानि गत्वा चैव गतश्रमाः}


\twolineshloka
{शनैर्मध्याह्नवेलायां प्रतिष्ठानं समाययुः}
{तां पुरीं पुरुहूतेन ऐलस्यार्थे विनिर्मिताम्}


\twolineshloka
{परिघाट्टालकैर्मुख्यैरुपकल्पशतैरपि}
{शतघ्नीचक्रयन्त्रैश्च गुप्तामन्यैर्दुरासदाम्}


\twolineshloka
{हर्म्यप्रसादसंबाधां नानापण्यविभूषिताम्}
{मण्टपैः ससभै रम्यैः प्रपाभिश्च समावृताम्}


\twolineshloka
{राजमार्गेण महता सुविभक्तेन शोभिताम्}
{कैलासशिखराकारैर्गोपुरैः समलङ्कृताम्}


\twolineshloka
{द्वारतोरणनिर्यूहैर्मङ्गलैरुपशोभिताम्}
{उद्यानाम्रवणोपेतां महतीं सालमेखलाम्}


\twolineshloka
{सर्वपुष्करिणीभिश्च उद्यानैश्च समावृताम्}
{वर्णाश्रमैः स्वधर्मस्थैर्नित्योत्सवसमाहितैः}


\twolineshloka
{धनधान्यसमृद्धैश्च संतुष्टै रत्नपूजितैः}
{कृतयज्ञैश्च विद्वद्भिरग्निहोत्रपरैः सदा}


\twolineshloka
{वर्जिता कार्यकरणैर्दानशीलैर्दयापरैः}
{अधर्मभीरुभिः सर्वैः स्वर्गलोकजिगीषुभिः}


\twolineshloka
{एवंविधजनोपेतमिन्द्रलोकमिवापरम्}
{तस्मिन्नगरमध्ये तु राजवेश्म प्रतिष्ठितम्}


\twolineshloka
{इन्द्रसद्मप्रतीकाशं संपूर्णं वित्तसंचयैः}
{तस्य मध्ये सभा दिव्या नानारत्नविभूषिता}


\twolineshloka
{तस्यां सभायां राजर्षिः सर्वालङ्कारभूषितः}
{ब्राह्मणैः क्षत्रियैश्चापि मन्त्रिभिश्चापि संवृतः}


\twolineshloka
{संस्तूयमानो राजेन्द्रः सूतमागधबन्दिभिः}
{कार्यार्थिषु तदाऽभ्येत्य कृत्वा कार्यं गतेषु सः}


\twolineshloka
{सुखासीनोऽभवद्राजा तस्मिन्काले महर्षयः}
{शकुन्तानां स्वनं श्रुत्वा निमित्तज्ञास्त्वलक्षयन्}


\fourlineindentedshloka
{शकुन्तले निमित्तानि शोभनानि भवन्ति नः}
{कार्यसिद्धिं वदन्त्येते ध्रुवं राज्ञी भविष्यसि}
{अस्मिंस्तु दिवसे पुत्रो युवराजो भविष्यति ॥वैशंपायन उवाच}
{}


\twolineshloka
{वर्धमानपुरद्वारं तूर्यघोषनिनादितम्}
{शकुन्तलां पुरस्कृत्य विविशुस्ते महर्षयः}


\twolineshloka
{प्रविशन्तं नृपसुतं प्रशशंसुश्च वीक्षकाः}
{वर्धमानपुरद्वारं प्रविशन्नेव पौरवः}


% Check verse!
इन्द्रलोकस्थमात्मानं मेने हर्षसमन्वितः
\threelineshloka
{ततो वै नागराः सर्वे समाहूय परस्परम्}
{द्रष्टुकामा नृपसुतं समपद्यन्त भारत ॥नागरा ऊचुः}
{}


\twolineshloka
{देवतेव जनस्याग्रे भ्राजते श्रीरिवागता}
{जयन्तेनेव पौलोमी इन्द्रलोकादिहागता}


\twolineshloka
{इति ब्रुवन्तस्ते सर्वे महर्षीनिदमब्रुवन्}
{अभिवादयन्तः सहिता महर्षीन्देववर्चसः}


\threelineshloka
{अद्य नः सफलं जन्म कृतार्थाश्च ततो वयम्}
{एवं ये स्म प्रपश्यामो महर्षीन्सूर्यवर्चसः ॥वैशंपायन उवाच}
{}


\twolineshloka
{इत्युक्त्वा सहिताः केचिदन्वगच्छन्त पौरवम्}
{हैमवत्याः सुतमिव कुमारं पुष्करेक्षणम्}


\twolineshloka
{ये केचिदब्रुवन्मूढाः शाकुन्तलदिदृक्षवः}
{कृष्णाजिनेन संछन्नाननिच्छन्तो ह्यवेक्षितुम्}


\twolineshloka
{पिशाचा इव दृश्यन्ते नागराणां विरूपिणः}
{विना सन्ध्यां पिशाचास्ते प्रविशन्ति पुरोत्तमम्}


\twolineshloka
{क्षुत्पिपासार्दितान्दीनान्वल्कलाजिनवाससः}
{त्वगस्थिभूतान्निर्मांसान्धमनीसन्ततानपि}


\twolineshloka
{पिङ्गलाक्षान्पिङ्गजटान्दीर्घदन्तान्निरूदरान्}
{विशीर्षकानूर्ध्वहस्तान्दृष्ट्वा हास्यन्ति नागराः}


\twolineshloka
{एवमुक्तवतां तेषां गिरं श्रुत्वा महर्षयः}
{अन्योन्यं ते समाहूय इदं वचनमब्रुवन्}


\twolineshloka
{उक्तं भगवता वाक्यं न कृतं सत्यवादिना}
{पुरप्रवेशनं नात्र कर्तव्यमिति शासनम्}


\twolineshloka
{किं कारणं प्रवेक्ष्यामो नगरं दुर्जनैर्वृतम्}
{त्यक्तसङ्गस्य च मुनेर्नगरे किं प्रयोजनम्}


\twolineshloka
{गमिष्यामो वनं तस्माद्गङ्गायामुनसङ्गमम्}
{एवमुक्त्वा मुनिगणाः प्रतिजग्मुर्यथागतम्}


\chapter{अध्यायः ९७}
\twolineshloka
{वैशंपायन उवाच}
{}


\twolineshloka
{गतान्मुनिगणान्दृष्ट्वा पुत्रं संगृह्य पाणिना}
{मातापितृभ्यां रहिता यथा शोचन्ति दारकाः}


\twolineshloka
{तथा शोकपरीताङ्गी धृतिमालम्ब्य दुःखिता}
{गतेषु तेषु विप्रेषु राजमार्गेण भामिनी}


\twolineshloka
{पुत्रेणैव सहायेन सा जगाम शनैः शनैः}
{अदृष्टपूर्वान्पश्यन्वै राजमार्गेण पौरवः}


\twolineshloka
{हर्म्यप्रसादचैत्यांश्च सभा दिव्या विचित्रिताः}
{कौतूहलसमाविष्टो दृष्ट्वा विस्मयमागतः}


\twolineshloka
{सर्वे ब्रुवन्ति तां दृष्ट्वा पद्महीनामिव श्रियम्}
{गत्या च संहीसदृशीं कोकिलेन स्वरे समाम्}


\twolineshloka
{मुखेन चन्द्रसदृशीं श्रिया पद्मालयासमाम्}
{स्मितेन कुन्दसदृशीं पद्मगर्भसमत्वचम्}


\twolineshloka
{पद्मपत्रविशालाक्षीं तप्तजाम्बूनदप्रभाम्}
{करान्तमितमध्यैषा सुकेशी संहतस्तनी}


\twolineshloka
{जघनं सुविशालं वै ऊरू करिकरोपमौ}
{रक्ततुङ्गतलौ पादौ धरण्यां सुप्रतिष्ठितौ}


\twolineshloka
{एवं रूपसमायुक्ता स्वर्गलोकादिवागता}
{इति स्म सर्वेऽमन्यन्त दुष्यन्तनगरे जनाः}


\twolineshloka
{पुनः पुनरवोचंस्ते शाकुन्तलगुणानपि}
{सिंहेक्षणः सिंहदंष्ट्रः सिंहस्कन्धो महाभुजः}


\twolineshloka
{सिंहोरस्कः सिंहबलः सिंहविक्रान्तगाम्ययम्}
{पृथ्वंसः पृथुवक्षाश्च छत्राकारशिरा महान्}


\twolineshloka
{पाणिपादतले रक्तो रक्तास्यो दुन्दुभिस्वनः}
{राजलक्षणयुक्तश्च राजश्रीश्चास्य लक्ष्यते}


\twolineshloka
{आकारेण च रूपेण शरीरेणापि तेजसा}
{दुष्यन्तेन समो ह्येष कस्य पुत्रो भविष्यति}


\twolineshloka
{एवं ब्रुवन्तस्ते सर्वे प्रशशंसुः सहस्रशः}
{युक्तिवादानवोचन्त सर्वाः प्राणभृतः स्त्रियः}


\twolineshloka
{बान्धवा इव सस्नेहा अनुजग्मुः शकुन्तलाम्}
{पौराणां तद्वचः श्रुत्वा तूष्णींभूता शकुन्तला}


\twolineshloka
{वेश्मद्वारं समासाद्य विह्वला सा नृपात्मजा}
{चिन्तयामास सहसा कार्यगौरवकारणात्}


\twolineshloka
{लज्जया च परीताङ्गी राजन्राजसमक्षतः}
{अघृणा किं नु वक्ष्यामि दुष्यन्तं मम कारणात्}


\twolineshloka
{एवमुक्त्वा तु कृपणा चिन्तयन्ती शकुन्तला}
{'अभिसृत्य च राजानं वेदिता सा प्रवेशिता}


\twolineshloka
{सह तेन कुमारेण तरुणादित्यवर्चसा}
{`सिंहासनस्थं राजानं महेन्द्रसदृशद्युतिम्}


\twolineshloka
{शकुन्तला नतशिराः परं हर्षमवाप्य च}
{'पूजयित्वा यथान्यायमब्रवीत्तं शकुन्तला}


\twolineshloka
{`अभिवादय राजानं पितरं ते दृढव्रतम्}
{एवमुक्त्वा सुतं तत्र लज्जानतमुखी स्थिता}


\twolineshloka
{स्तम्भमालिङ्ग्य राजानं प्रसीदस्वेत्युवाच सा}
{शाकुन्तलोपि राजानमभिवाद्य कृताञ्जलिः}


\twolineshloka
{हर्षेणोत्फुल्लनयनो राजानं चान्ववैक्षत}
{दुष्यन्तो धर्मबुद्ध्या तु चिन्तयन्नेव सोब्रवीत्}


\threelineshloka
{किमागमनकार्यं ते ब्रूहि त्वं वरवर्णिनि}
{करिष्यामि न संदेहः सपुत्राया विशेषतः ॥शकुन्तलोवाच}
{}


\twolineshloka
{प्रसीदस्व महाराज वक्ष्यामि पुरुषोत्तम}
{एष पुत्रो हि ते राजन्मय्युत्पन्नः परंतप}


\twolineshloka
{तस्मात्पुत्रस्त्वया राजन्यौवराज्येऽभिषिच्यताम्}
{यथोक्तमाश्रमे तस्मिन्वर्तस्व पुरुषोत्तम}


\threelineshloka
{मया समागमे पूर्वं कृतः स समयस्त्वया}
{तत्त्वं स्मर महाबाहो कण्वाश्रमपदं प्रति ॥वैशंपायन उवाच}
{}


\twolineshloka
{तस्योपभोगसक्तस्य स्त्रीषु चान्यासु भारत}
{शकुन्तला सपुत्रा च मनस्यन्तरधीयत}


\twolineshloka
{स धारयन्मनस्येनां सपुत्रां सस्मितां तदा}
{तदोपगृह्य मनसा चिरं सुखमवाप सः}


\twolineshloka
{सोऽथ श्रुत्वापि तद्वाक्यं तस्या राजा स्मरन्नपि}
{अब्रवीन्न स्मरामीति त्वया भद्रे समागमम्}


\twolineshloka
{मैथुनं च वृथा नाहं गच्छेयमिति मे मतिः}
{नाभिजानामि कल्याणि त्वया सह समागमम्'}


\twolineshloka
{धर्मार्थकामसंबन्धं न स्मरामि त्वया सह}
{गच्छ वा तिष्ठ वा कामं यद्वापीच्छसि तत्कुरु}


\chapter{अध्यायः ९८}
\twolineshloka
{एवमुक्ता वरारोहा व्रीडितेव मनस्विनी}
{विसंज्ञेव च दुःखेन तस्थौ स्थूणेव निश्चला}


\twolineshloka
{संरम्भामर्षताम्राक्षी स्फुरमाणोष्ठसंपुटा}
{कटाक्षैर्निर्दहन्तीव तिर्यग्राजानमैक्षत}


\twolineshloka
{आकारं गूहमाना च मन्युना च समीरितम्}
{तपसा संभृतं तेजो धारयामास वै तदा}


\twolineshloka
{सा मुहूर्तमिव ध्यात्वा दुःखामर्षसमन्विता}
{भर्तारमभिसंप्रेक्ष्य यथान्यायं वचोऽब्रवीत्}


\twolineshloka
{जानन्नपि महाराज कस्मादेवं प्रभाषसे}
{न जानामीति निःशङ्कं यथान्यः प्राकृतस्तथा}


\twolineshloka
{तस्य ते हृदयं वेद सत्यस्यैवानृतस्य च}
{साक्षिणं बत कल्याणमात्मानमवमन्यसे}


\twolineshloka
{योऽन्यथा सन्तमात्मानन्यथा प्रतिपद्यते}
{किं तेन न कृतं पापं चोरेणात्मापहारिणा}


\twolineshloka
{एकोऽहमस्मीति च मन्यसे त्वंन हृच्छयं वेत्सि मुनिं पुराणम्}
{यो वेदिता कर्मणः पापकस्यतस्यान्तिके त्वं वृजिनं करोषि}


\twolineshloka
{`धर्म एव हि साधूनां सर्वेषां हितकारणम्}
{नित्यं मिथ्याविहीनानां न च दुःखावहो भवेत्'}


\twolineshloka
{मन्यते पापकं कृत्वा न कश्चिद्वेत्ति मामिति}
{विदन्ति चैनं देवाश्च यश्चैवान्तरपूरुषः}


\twolineshloka
{आदित्यचन्द्रावनिलोऽनलश्चद्यौर्भूमिरापो हृदयं यमश्च}
{अहश्च रात्रिश्च उभे च सन्ध्येधर्मश्च जानाति नरस्य वृत्तम्}


\twolineshloka
{यमो वैवस्वतस्तस्य निर्यातयति दुष्कृतम्}
{हृदि स्थितः कर्मसाक्षी क्षेत्रज्ञो यस्य तुष्यति}


\twolineshloka
{न तुष्यति च यस्यैष पुरुषस्य दुरात्मनः}
{तं यमः पापकर्माणं निर्भर्त्सयति दुष्कृतम्}


\twolineshloka
{योऽवमत्यात्मनात्मानमन्यथा प्रतिपद्यते}
{न तस्य देवाः श्रेयांसो यस्यात्मापि न कारणम्}


\twolineshloka
{स्वयं प्राप्तेति मामेवं मावमंस्था पतिव्रताम्}
{अर्चार्हां नार्चयसि मां स्वयं भार्यामुपस्थिताम्}


\twolineshloka
{किमर्थं मां प्राकृतवदुपप्रेक्षसि संसदि}
{नखल्वहमिदं शून्ये रौमि किं न शृणोषि मे}


\twolineshloka
{यदि मे याचमानाया वचनं न करिष्यसि}
{दुष्यन्त शतधा त्वद्य मूर्धा ते विफलिष्यति}


\twolineshloka
{जायां पतिः संप्रविश्य यदस्यां जायते पुनः}
{जायायास्तद्धि जायात्वं पौराणाः कवयो विदुः}


\twolineshloka
{यदागमवतः पुंसस्तदपत्यं प्रजायते}
{तत्तारयति संतत्या पूर्वप्रेतान्पितामहान्}


\twolineshloka
{पुन्नाम्नो नरकाद्यस्मात्पितरं त्रायते सुतः}
{तस्मात्पुत्र इति प्रोक्तः पूर्वमेव स्वयंभुवा}


\twolineshloka
{`पुत्रेण लोकाञ्जयन्ति पौत्रेणानन्त्यमश्नुते}
{अथ पौत्रस्य पुत्रेण मोदन्ते प्रपितामहाः ॥'}


\twolineshloka
{सा भार्या या गृहे दक्षा सा भार्या या प्रजावती}
{सा भार्या या पतिप्राणा सा भार्या या पतिव्रता}


\twolineshloka
{अर्धं भार्या मनुष्यस्य भार्या श्रेष्ठतमः सखा}
{भार्या मूलं त्रिवर्गस्य यः सभार्यः स बन्धुमान्}


\twolineshloka
{भार्यावन्तः क्रियावन्तः सभार्या गृहमेधिनः}
{भार्यावन्तः प्रमोदन्ते भार्यावन्तः श्रियावृताः}


\twolineshloka
{सखायः प्रविविक्तेषु भवन्त्येताः प्रियंवदाः}
{पितरो धर्मकार्येषु भवन्त्यार्तस्य मातरः}


\threelineshloka
{कान्तारेष्वपि विश्रामो जनस्याध्वनि कस्य वै}
{यः सदारः स विश्वास्यस्तस्माद्दाराः परा गतिः}
{}


\twolineshloka
{संसरन्तमभिप्रेतं विषमेष्वेकपातिनम्}
{भार्यैवान्वेति भर्तारं सततं या पतिव्रता}


\twolineshloka
{प्रथमं संस्थिता भार्या पतिं प्रेत्य प्रतीक्षते}
{पूर्वप्रेतं तु भर्तारं पश्चात्साप्यनुगच्छति}


\twolineshloka
{एतस्मात्कारणाद्राजन्पाणिग्रहणमिष्यते}
{यदाप्नोति पतिर्भार्यामिह लोके परत्र च}


\twolineshloka
{`पोषणार्थं शरीरस्य पाथेयं स्वर्गतस्य वै}
{'आत्माऽऽत्मनैव जनितः पुत्र इत्युच्यते बुधैः}


\twolineshloka
{तस्माद्भार्यां पतिः पश्येन्मातृवत्पुत्रमातरम्}
{`अन्तरात्मैव सर्वस्य पुत्रो नामोच्यते सदा}


\twolineshloka
{गती रूपं च चेष्टा च आवर्ता लक्षणानि च}
{पितॄणां यानि दृश्यन्ते पुत्राणां सन्ति तानि च}


\twolineshloka
{तेषां शीलगुणाचारास्तत्संपर्काच्छुभाशुभात्}
{'भार्यायां जनितं पुत्रमादर्शे स्वमिवाननम्}


\twolineshloka
{जनिता मोदते प्रेक्ष्य स्वर्गं प्राप्येव पुण्यकृत्}
{`पतिव्रतारूपधराः परबीजस्य संग्रहात्}


\twolineshloka
{कुलं विनाश्य भर्तॄणां नरकं यान्ति दारुणम्}
{परेण जनिताः पुत्राः स्वभार्यायां यथेष्टतः}


\twolineshloka
{मम पुत्रा इति मतास्ते पुत्रा अपि शत्रवः}
{द्विषन्ति प्रतिकुर्वन्ति न ते वचनंकारिणः}


\twolineshloka
{द्वेष्टि तांश्च पिता चापि स्वबीजे न तथा नृप}
{न द्वेष्टि पितरं पुत्रो जनितारमथापि वा}


\twolineshloka
{न द्वेष्टि जनिता पुत्रं तस्मादात्मा सुतो भवेत्}
{'दह्यमाना मनोदुःखैर्व्याधिभिस्तुमुलैर्जनः}


\twolineshloka
{ह्लादन्ते स्वेषु दारेषु घर्मार्ताः सलिलेष्विव}
{`विप्रवासकृशा दीना नरा मलिनवाससः}


\twolineshloka
{तेऽपि स्वदारांस्तुष्यन्ति दरिद्रा धनलाभवत्}
{'अप्रियोक्तोपि दाराणां न ब्रूयादप्रियं बुधः}


\twolineshloka
{रतिं प्रीतिं च धर्मं च तदायत्तमवेक्ष्य च}
{`आत्मनोऽर्धमिति श्रौतं सा रक्षति धनं प्रजाः}


\twolineshloka
{शरीरं लोकयात्रां वै धर्मं स्वर्गमृषीन्पितॄन्}
{'आत्मनो जन्मनः क्षेत्रं पुण्या रामाः सनातनाः}


\twolineshloka
{ऋषीणामपि का शक्तिः स्रष्टुं रामामृते प्रजा-}
{`देवानामपि का शक्तिः कर्तुं संभवमात्मनः}


\twolineshloka
{पण्डितस्यापि लोकेषु स्त्रीषु सृष्टिः प्रतिष्ठिता}
{ऋषिभ्यो ह्यृषयः केचिच्चण्डालीष्वपि जज्ञिरे'}


\twolineshloka
{परिसृत्य यथा सूनुर्धरणीरेणुकुण्ठितः}
{पितुरालिङ्गतेऽङ्गानि किमस्त्यभ्यधिकं ततः}


\twolineshloka
{स त्वं सूनुमनुप्राप्तं साभिलाषं मनस्विनम्}
{प्रेक्षमाणं कटाक्षेण किमर्थमवमन्यसे}


\twolineshloka
{अण्डानि बिभ्रति स्वानि न त्यजन्ति पिपीलिकाः}
{किं पुनस्त्वं न मन्येथाः सर्वथा पुत्रमीदृशम्}


\twolineshloka
{न भरेथाः कथं नु त्वं मयि जातं स्वमात्मजम्}
{`ममाण्डानीति वर्ध्ते कोकिलाण्डानि वायसाः}


\twolineshloka
{किं पुनस्त्वं न मन्येथाः सर्वज्ञः पुत्रमीदृशम्}
{मलयाच्चन्दनं जातमतिशीतं वदन्ति वै}


\twolineshloka
{शिशोरालिङ्गनं तस्माच्चन्दनादधिकं भवेत्}
{'न वाससां न रामाणां नापां स्पर्शस्तथाविधः}


\twolineshloka
{शिशुनालिङ्ग्यमानस्य स्पर्शः सूनोर्यथा सुखः}
{पुत्रस्पर्शात्प्रियतरः स्पर्शो लोके न विद्यते}


\twolineshloka
{स्पृशतु त्वां समालिङ्ग्य पुत्रोऽयं प्रियदर्शनः}
{ब्राह्मणो द्विपदां श्रेष्ठो गौर्वरिष्ठा चतुष्पदाम्}


% Check verse!
मुरुर्गरीयसां श्रेष्ठः पुत्रः स्पर्शवतां वरः ॥त्रिषु वर्षेषु पूर्णेषु प्रजातोऽयमरिन्दमः
\twolineshloka
{`अद्यायं मन्नियोगात्तु तवाह्वानं प्रतीक्षते}
{कुमारो राजशार्दूल तव शोकप्रणाशनः ॥'}


\twolineshloka
{आहर्ता वाजिमेधस्य शतसङ्ख्यस्य पौरवः}
{`राजसूयादिकानन्यान्क्रतूनमितदक्षिणान्}


\twolineshloka
{इति गौरन्तरिक्षे मां सूतके ह्यवदत्पुरा}
{हन्त स्वमङ्कमारोप्य स्नेहाद्ग्रामान्तरं गताः}


\twolineshloka
{मूर्ध्नि पुत्रानुपाघ्राय प्रतिनन्दन्ति मानवाः}
{वेदेष्वपि वदन्तीमं मन्त्रग्रामं द्विजातयः}


\twolineshloka
{जातकर्मणि पुत्राणां तवापि विदितं ध्रुवम्}
{अङ्गादङ्गात्संभवसि हृदयादधिजायसे}


\twolineshloka
{आत्मा वै पुत्रनामासि स जीव शरदः शतम्}
{उपजिघ्रन्ति पितरो मन्त्रेणानेन मूर्धनि}


\twolineshloka
{पोषणं त्वदधीनं मे सन्तानमपि चाक्षयम्}
{तस्मात्त्वं जीव मे पुत्र स सुखी शरदां शतम्}


\twolineshloka
{एको भूत्वा द्विधा भूत इति वादः प्रवर्तते}
{त्वदङ्गेभ्यः प्रसूतोऽयं पुरुषात्पुरुषः परः}


\twolineshloka
{सरसीवामलेऽऽत्मानं द्वितीयं पश्य ते सुतम्}
{`सरसीवामले सोमं प्रेक्षात्मानं त्वमात्मनि'}


\twolineshloka
{यथाचाहवनीयोऽग्निर्वर्हपत्यात्प्रणीयते}
{एवं त्वत्तः प्रणीतोऽयं त्वमेकः सन्द्विधा कृतः}


\twolineshloka
{मृगापकृष्टेन हि वै मृगयां परिधावता}
{अहमासादिता राजन्कुमारी पितुराश्रमे}


\twolineshloka
{उर्वशी पूर्वचित्तिश्च सहजन्या च मेनका}
{विश्वाची च घृताची च षडेवाप्सरसां वराः}


\twolineshloka
{तासां वै मेनका नाम ब्रह्मयोनिर्वराप्सराः}
{दिवः संप्राप्य जगतीं विश्वामित्रादजीजनत्}


\twolineshloka
{`श्रीमानृषिर्धर्मपरो वैश्वानर इवापरः}
{ब्रह्मयोनिः कुशो नाम विश्वामित्रपितामहः}


\twolineshloka
{कुशस्य पुत्रो बलवान्कुशनाभश्च धार्मिकः}
{गाधिस्तस्य सुतो राजन्विश्वामित्रस्तु गाधिजः}


\twolineshloka
{एवंविधो मम पिता मेनका जननी वरा}
{'सा मां हिमवतः पृष्ठे सुषुवे मेनकाऽप्सराः}


\twolineshloka
{परित्यज्य च मां याता परात्मजमिवासती}
{`पक्षिणः पुम्यवन्तस्ते सहिता धर्मतस्तदा}


\twolineshloka
{पक्षैस्तैरभिगुप्ता च तस्मादस्मि शकुन्तला}
{ततोऽहमृषिणा दृष्टा काश्यपेन महात्मना}


\twolineshloka
{जलार्थमग्निहोत्रस्य गतं दृष्ट्वा तु पक्षिणः}
{न्यासभूतामिव मुनेः प्रददुर्मां दयावतः}


\twolineshloka
{कण्वस्त्वालोक्य मां प्रीतो हसन्तीति हविर्भुजः}
{स माऽरणिमिवादाय स्वमाश्रममुपागमत्}


\twolineshloka
{सा वै संभाविता राजन्ननुक्रोशान्महर्षिणा}
{तेनैव स्वसुतेवाहं राजन्वै वरवर्णिनी}


\twolineshloka
{विश्वामित्रसुता चाहं वर्धिता मुनिना नृप}
{यौवने वर्तमानां च दृष्टवानसि मां नृप}


\twolineshloka
{आश्रमे पर्णशालायां कुमारीं विजने तदा}
{धात्रा प्रचोदितां शून्ये पित्रा विरहितां मिथः}


\twolineshloka
{वाग्भिस्त्वं सूनृताभिर्मामपत्यार्थमचूचुदः}
{अकार्षीस्त्वाश्रमे वासं धर्मकामार्थनिश्चितम्}


\twolineshloka
{गान्धर्वेण विवाहेन विधिना पाणिमग्रहीः}
{साऽहं कुलं च शीलं च सत्यवादित्वमात्मनः}


\twolineshloka
{स्वधर्मं च पुरस्कृत्य त्वामद्य शरणं गता}
{तस्मान्नर्हसि संश्रुत्य तथेति वितथं वचः}


\twolineshloka
{स्वधर्मं पृष्ठतः कृत्वा परित्यक्तुमुपस्थिताम्}
{त्वन्नाथां लोकनाथस्त्वं नार्हसि त्वमनागसम्'}


\twolineshloka
{किं नु कर्माशुभं पूर्वं कृतवत्यस्मि पार्थिव}
{यदहं बान्धवैस्त्यक्ता बाल्ये संप्रति वै त्वया}


\threelineshloka
{कामं त्वया परित्यक्ता गमिष्याम्यहमाश्रमम्}
{इमं बालं तु संत्युक्तं नार्हस्यात्मजमात्मना ॥दुष्यन्त उवाच}
{}


\twolineshloka
{न पुत्रमभिजानामि त्वयि जातं शकुन्तले}
{असत्वचना नार्यः कस्ते श्रद्धास्यते वचः}


\twolineshloka
{`अश्रद्धेयमिदं वाक्यं कथयन्ती न लज्जसे}
{विशेषतो मत्सकाशे दुष्टतापसि गम्यताम्'}


\twolineshloka
{क्व महर्षिस्तपस्युग्रः क्वाप्सराः सा च मेनका}
{क्व च त्वमेवं कृपणा तापसीवेषधारिणी}


\twolineshloka
{अतिकायश्च पुत्रस्ते बालोऽतिबलवानयम्}
{कथमल्पेन कालेन सालस्कन्ध इवोद्गतः}


\twolineshloka
{सुनिकृष्टा च योनिस्ते पुंश्चली प्रतिभासि मे}
{यदृच्छया कामरागाज्जाता मेनकया ह्यसि}


\twolineshloka
{सर्वमेव परोक्षं मे यत्त्वं वदसि तापसि}
{`सर्वा वामाः स्त्रियो लोके सर्वाः कामपरायणाः}


\twolineshloka
{सर्वाः स्त्रियः परवशाः सर्वाः क्रोधसमाकुलाः}
{असत्योक्ताः स्त्रियः सर्वा न कण्वं वक्तुमर्हसि'}


\twolineshloka
{मेनका निरनुक्रोशा वर्धकी जननी तव}
{यया हिमवतः पादे निर्माल्यवदुपेक्षिता}


\twolineshloka
{स चापि निरनुक्रोशः क्षत्रयोनिः पिता तव}
{विश्वामित्रो ब्राह्मणत्वे लुब्धः कामपरायणः}


\twolineshloka
{सुषाव सुरनारी मां विश्वामित्राद्यथेष्टतः}
{अहो जानामि ते जन्म कुत्सितं कुलटे जनैः}


\twolineshloka
{मेनकाऽप्सरसां श्रेष्ठा महर्षिश्चापि ते पिता}
{तयोरपत्यं कस्मात्त्वं पुंश्चलीवाभिभाषसे}


\twolineshloka
{जातिश्चापि निकृष्टो ते कुलीनेति विजल्पसे}
{जनयित्वा त्वमुत्सृष्टा कोकिलेन परैर्भृता}


\twolineshloka
{अरिष्टैरिव दुर्बद्धिः कण्वो वर्धयिता पिता}
{अश्रद्धेयमिदं वाक्यं यत्त्वं जल्पसि तापसि}


\twolineshloka
{ब्रुवन्ती राजसान्निध्ये गम्यतां यत्र चेच्छसि}
{`सुवर्णमणिमुक्तानि वस्त्राण्याभरणानि च}


\twolineshloka
{यदिहेच्छसि भोगार्थं तापसि प्रतिगृह्यताम्}
{नाहं त्वां द्रष्टुमिच्छामि यथेष्टं गम्यतामितः'}


\chapter{अध्यायः ९९}
\twolineshloka
{शकुन्तलोवाच}
{}


\twolineshloka
{राजन्सर्षपमात्राणि परच्छिद्राणि पश्यसि}
{आत्मनो बिल्वमात्राणि पश्यन्नपि न पश्यसि}


\twolineshloka
{मेनका त्रिदशेष्वेव त्रिदशाश्चानु मेनकाम्}
{ममैवोद्रिच्यते जन्म दुष्यन्त तव जन्मतः}


\twolineshloka
{क्षितौ चरसि राजंस्त्वमन्तरिक्षे चराम्यहम्}
{आवयोरन्तरं पश्य मेरुसर्षपयोरिव}


\twolineshloka
{महेन्द्रस्य कुबेरस्य यमस्य वरुणस्य च}
{भवनान्यनुसंयामि प्रभावं पश्य मे नृप}


\twolineshloka
{`पुरा नरवरः पुत्र उर्वश्यां जनितस्तदा}
{आयुर्नाम महाराज तव पूर्वपितामहः}


\twolineshloka
{महर्षयश्च बहवः क्षत्रियाश्च परन्तपाः}
{अप्सरःसु ऋषीणां च मातृदोषो न विद्यते ॥'}


\twolineshloka
{सत्यश्चापि प्रवादोऽयं प्रवक्ष्यामि च ते नृप}
{निदर्शनार्थं न द्वेषाच्छ्रुत्वा तत्क्षन्तुमर्हसि}


\twolineshloka
{विरूपो यावदादर्शे नात्मनो वीक्षते मुखम्}
{मन्यते तावदात्मानमन्येभ्यो रूपवत्तरम्}


\twolineshloka
{यदा तु रूपमादर्शे विरूपं सोऽभिवीक्षते}
{तदा ह्रीमांस्तु जानीयादन्तरं नेतरं जनम्}


\twolineshloka
{अतीव रूपसंपन्नो न कंचिदवमन्यते}
{अतीव जल्पन्दुर्वाचो भवतीह विहेतुकः}


\twolineshloka
{`पांसुपातेन हृष्यन्ति कुञ्जरा मदशालिनः}
{तथा परिवदन्नन्यान्हृष्टो भवति दुर्मतिः}


\twolineshloka
{सत्यधर्मच्युतात्पुंसः क्रुद्धादाशीविषादिव}
{सुनास्तिकोप्युद्विजते जनः किं पुनरास्तिकः}


\twolineshloka
{स्वयमुत्पाद्य पुत्रं वै सदृशं योऽवमन्यते}
{तस्य देवाः श्रियं घ्नन्ति तत्रैनं कलिराविशेत्}


\twolineshloka
{अभव्येऽप्यनृतेऽशुद्धे नास्तिके पापकर्मणि}
{दुराचारे कलिर्ह्याशु न कलिर्धर्मचारिषु ॥'}


\twolineshloka
{मूर्खो हि जल्पतां पुंसां श्रुत्वा वाचः शुभाशुभाः}
{अशुभं वाक्यमादत्ते पुरीषमिव सूकरः}


\twolineshloka
{प्राज्ञस्तु जल्पतां पुंसां श्रुत्वा वाचः शुभाशुभाः}
{गुणवद्वाक्यमादत्ते हंसः क्षीरमिवाम्भसि}


\twolineshloka
{`आत्मनो दुष्टभावत्वं जानन्नीचोऽप्रसन्नधीः}
{परेषामपि जानाति स्वधर्मसदृशान्गुणान्}


\twolineshloka
{दह्यमानास्तु तीव्रेण नीचाः परयशोग्निना}
{अशक्तास्तद्गतिं गन्तुं ततो निन्दां प्रकुर्वते ॥'}


\twolineshloka
{अन्यान्परिवदन्साधुर्यथा हि परितप्यते}
{तथा परिवदन्नन्यान्हृष्टो भवति दुर्जनः}


\twolineshloka
{`अपवादरता मूर्खा भवन्ति हि विशेषतः}
{नापवादरताः सन्तो भवन्ति स्म विशेषतः ॥'}


\twolineshloka
{अभिवाद्य यथा वृद्धान्साधुर्गच्छति निर्वृतिम्}
{एवं सज्जनमाक्रुश्य मूर्खो भवति निर्वृतः}


\twolineshloka
{सुखं जीवन्त्यदोषज्ञा मूर्खा दोषानुदर्शिनः}
{यथा वाच्याः परैः सन्तः परानाहुस्तथाविधान्}


\twolineshloka
{अतो हास्यतरं लोके किंचिदन्यन्न विद्यते}
{यदि दुर्जन इत्याहुः सज्जनं दुर्जनाः स्वयम्}


% Check verse!
`दारुणाल्लोकसंक्लेशाद्दुःखमाप्नोत्यसंशयम् ॥'कुलवंशप्रतिष्ठां हि पितरः पुत्रमब्रुवन्
\twolineshloka
{उत्तमं सर्वधर्माणां तस्मात्पुत्रं तु न त्यजेत्}
{स्वपत्नीप्रभवाँल्लब्धान्कृतान्समयवर्धितान्}


\twolineshloka
{क्रीतान्कन्यासु चोत्पन्नान्पुत्रान्वै मनुरब्रवीत्}
{`ते च षड्वन्धुदायादाः षडदायादबान्धवाः}


\twolineshloka
{धर्मकृत्यवहा नॄणां मनसः प्रीतिवर्धनाः}
{त्रायन्ते नरकाज्जाताः पुत्रा धर्मप्लवाः पितॄन्}


\twolineshloka
{स त्वं नृपतिशार्दूल न पुत्रं त्यक्तुमर्हसि}
{तस्मात्पुत्रं च सत्यं च पालयस्व महीपते}


\threelineshloka
{उभयं पालयस्वैतन्नानृतं वक्तुमर्हसि}
{'आत्मानं सत्यधर्मौ च पालयेथा महीपते}
{नरेन्द्रसिंह कपटं न हि वोढुं त्वमर्हसि}


\twolineshloka
{वरं कूपशताद्वापी वरं वापीशतात्क्रतुः}
{वरं क्रतुशतात्पुत्रः सत्यं पुत्रशताद्वरम्}


\twolineshloka
{अश्वमेधसहस्रं च सत्यं च तुलया धृतम्}
{अश्वमेधसहस्राद्धि सत्यमेव विशिष्यते}


\twolineshloka
{सर्ववेदाधिगमनं सर्वतीर्थावगाहनम्}
{सत्यस्यैव च राजेन्द्र कलां नार्हति षोडशीम्}


\twolineshloka
{नास्ति सत्यसमो धर्मो न सत्याद्विद्यते परम्}
{न हि तीव्रतरं पापमनृतादिह विद्यते}


\twolineshloka
{राजन्सत्यं परो धर्मः सत्याच्च समयः परः}
{मात्याक्षीः समयं राजन्सत्यं सङ्गतमस्तु ते}


\twolineshloka
{`यः पापो न विजानीयात्कर्म कृत्वा नराधिप}
{न हि तादृक्परं पापमनृतादिह विद्यते}


\twolineshloka
{यस्य ते हृदयं वेद सत्यस्यैवानृतस्य च}
{कल्याणावेक्षणं तस्मात्कर्तुमर्हसि धर्मतः}


\twolineshloka
{यो न कामान्न च क्रोधान्न द्रोहादतिवर्तते}
{अमित्रं वापि मित्रं वा स एवोत्तमपूरुषः ॥'}


\twolineshloka
{अनृतश्चेत्प्रसङ्गस्ते श्रद्दधासि न चेत्स्वयम्}
{आश्रमं गन्तुमिच्छामि त्वादृशो नास्ति सङ्गतं}


\twolineshloka
{`पुत्रत्वे शङ्कमानस्य त्वं बुद्ध्या निश्चयं कुरु}
{गतिः स्वरः स्मृतिः सत्वं शीलं विद्या च विक्रमः}


\twolineshloka
{धृष्णुप्रकृतिभावौ च आवर्ता रोमराजयः}
{समा यस्य यदा स्युस्ते तस्य पुत्रो न संशयः}


\twolineshloka
{सादृश्येनोद्धऋतं बिम्बं तव देहाद्विशांपते}
{तातेति भाषमाणं वै मा स्म राजन्वृथा कृथाः}


\threelineshloka
{ऋते च गर्दभीक्षीरात्पयः पास्यति मे सुतः}
{'ऋतेपि त्वां च दुष्यन्त शैलराजावतंसिकाम्}
{चतुरन्तामिमामुर्वीं पुत्रो मे पालयिष्यति}


\twolineshloka
{`शकुन्तले तव सुतश्चक्रवर्ती भविष्यति}
{एवमुक्तं महेन्द्रेण भविष्यति न चान्यथा}


\twolineshloka
{साक्षित्वे बहवो ह्युक्ता देवदूतादयो मया}
{न ब्रुवन्ति तथा सत्यमुताहो नानृतं किल}


\twolineshloka
{असाक्षिणी मन्दबाग्या गमिष्यामि यथागतम् ॥' वैशंपायन उवाच}
{}


\twolineshloka
{एतावदुक्त्वा वचनं प्रातिष्ठत शकुन्तला}
{`तस्याः क्रोधसमुत्थोग्निः सधूमो मूर्ध्न्यदृश्यत}


\twolineshloka
{संनियम्यात्मनोऽङ्गेषु ततः क्रोधाग्निमात्मजम्}
{प्रस्थितैवानवद्याङ्गी सह पुत्रेण वै वनम्'}


\chapter{अध्यायः १००}
\twolineshloka
{वैशंपायन उवाच}
{}


\twolineshloka
{अथान्तरिक्षे दुष्यन्तं वागुवाचाशरीरिणी}
{ऋत्विक्पुरोहिताचार्यैर्मन्त्रिभिश्चाभिसंवृतम्}


\twolineshloka
{माता भस्त्रा पितुः पुत्रो यस्माज्जातः स एव सः}
{भरस्व पुत्रं दौष्यन्तिं सत्यमाह शकुन्तला}


\twolineshloka
{`सर्वेभ्यो ह्यङ्गमङ्गेभ्यः साक्षादुत्पद्यते सुतः}
{आत्मा चैव सुतो नाम तेनैव तव पौरव}


\twolineshloka
{आहितं ह्यात्मनाऽऽत्मानं परिरक्ष इमं सुतम्}
{अनन्यां त्वं प्रतीक्षस्व मावमंस्थाः शकुन्तलाम्}


\twolineshloka
{स्त्रियः पवित्रमतुलमेतद्दुष्यन्त धर्मतः}
{मासि मासि रजो ह्यासां दुरितान्यपकर्षति}


\threelineshloka
{ततः सर्वाणि भूतानि व्याजह्रस्तं समन्ततः}
{देवा ऊचुः}
{आहितस्त्वत्तनोरेष मावमंस्थाः शकुन्तलाम्'}


\twolineshloka
{रेतोधाः पुत्र उन्नयति नरदेव यमक्षयात्}
{त्वं चास्य धाता गर्भस्य सत्यमाह शकुन्तला}


\twolineshloka
{`पतिर्जायां प्रविशति स तस्यां जायते पुनः}
{अन्योन्यप्रकृतिर्ह्येषा मावमंस्थाः शकुन्तलाम् ॥'}


\twolineshloka
{जाया जनयते पुत्रमात्मनोऽङ्गाद्द्विधा कृतम्}
{तस्माद्भरस्व दुष्यन्त पुत्रं शाकुन्तलं नृप}


\twolineshloka
{सुभूतिरेषा न त्याज्या जीवितं जीवयात्मजम्}
{शाकुन्तलं महात्मानं दुष्यन्त भर पौरवम्}


\twolineshloka
{भर्तव्योऽयं त्वया यस्मादस्माकं वचनादपि}
{तस्माद्भवत्वयं नाम्ना भरतो नाम ते सुतः}


\threelineshloka
{`भरताद्भारती कीर्तिर्येनेदं भारतं कुलम्}
{अपरे ये च पूर्वे च भारता इति तेऽभवन् ॥वैशंपायन उवाच}
{}


\twolineshloka
{एवमुक्त्वा ततो देवा ऋषयश्च तपोधनाः}
{पतिव्रतेति संहृष्टाः पुष्पवृष्टिं ववर्षिरे ॥'}


\twolineshloka
{तच्छ्रुत्वा पौरवो वाक्यं व्याहृतं वे दिवौकसाम्}
{`सिंहासनात्समुत्थाय प्रणम्य च दिवौकसः ॥'}


\twolineshloka
{पुरोहितममात्यांश्च संप्रहृष्टोऽब्रवीदिदम्}
{शृण्वन्त्वेतद्भवन्तोऽपि देवदूतस्य भाषितम्}


\twolineshloka
{`शृण्वन्तु देवतानां च महर्षीणां च भाषितम्'}
{अहमप्येवमेवैनं जानामि सुतमात्मजम्}


\threelineshloka
{यद्यहं वचनादस्या गृह्णीयामिममात्मजम्}
{भवेद्धि शङ्का लोकस्य नैवं शुद्धो भवेदयम् ॥वैशंपायन उवाच}
{}


\twolineshloka
{तां विशोध्य तदा राजा देवैः सह महर्षिभिः}
{हृष्टः प्रमुदितश्चापि प्रतिजग्राह तं सुतम्}


\threelineshloka
{ततस्तस्य तदा राजा पितृकर्माणि सर्वशः}
{कारयामास मुदितः प्रीतिमानात्मजस्य ह}
{मूर्ध्नि चैनं समाघ्राय सस्नेहं परिषस्वजे}


\twolineshloka
{सभाज्यमानो विप्रैश्च स्तूयमानश्च बन्दिभिः}
{मुदं स परमां लेभे पुत्रस्पर्शनजां नृपः}


\twolineshloka
{स्वां चैव भार्यां धर्मज्ञः पूजयामास धर्मतः}
{अब्रवीच्चैव तां राजा सान्त्वपूर्वमिदं वचः}


\twolineshloka
{लोकस्यायं परोक्षस्तु संबन्धो नौ पुराऽभवत्}
{कृतो लोकसमक्षोऽद्य संबन्धो वै पुनः कृतः}


% Check verse!
तस्मादेतन्मया तस्य तन्निमित्तं प्रभाषितम्
\twolineshloka
{शङ्केत वाऽयं लोकोऽथ स्त्रीभावान्मयि संगतम्}
{तस्मादेतन्मया चापि तच्छुद्ध्यर्थं विचारितम्}


\twolineshloka
{`ब्राह्मणाः क्षत्रिया वैश्याः शूद्राश्चैव पृथग्विधाः}
{त्वां देवि वूजयिष्यन्ति निर्विशङ्कां पतिव्रताम्'}


% Check verse!
पुत्रश्चायं वृतो राज्ये त्वमग्रमहिषी भव
\twolineshloka
{यच्च कोपनयात्यर्थं त्वयोक्तोऽस्म्यप्रियं प्रिये}
{प्रणयिन्या विशालाक्षितत्क्षान्तं ते मया शुभे}


\threelineshloka
{`अनृतं वाप्यनिष्टं वा दुरुक्तं वातिदुष्कृतम्}
{त्वयाप्येवं विशालाक्षि क्षन्तव्यं मम दुर्वचः}
{क्षान्त्या पतिकृते नार्यः पातिव्रत्यं व्रजन्ति ताः}


\twolineshloka
{एवमुक्त्वा तु राजर्षिस्तामनिन्दितगामिनीम्}
{अन्तःपुरं प्रवेश्याथ दुष्यन्तो महिषीं प्रियाम्}


\twolineshloka
{वासोभिरन्नपानैश्च पूजयित्वा तु भारत}
{`स मातरमुपस्थाय रथन्तर्यामभाषत}


\twolineshloka
{मम पुत्रो वने जातस्तव शोकप्रणशनः}
{ऋणादद्य विमुक्तोऽहं तव पौत्रेण शोभने}


\twolineshloka
{विश्वामित्रसुता चेयं कण्वेन च विवर्धिता}
{स्नुषा तव महाभागे प्रसीदस्व शकुन्तलाम्}


\twolineshloka
{पुत्रस्य वचनं श्रुत्वा पौत्रं सा परिषस्वजे}
{पादयोः पतितां तत्र रथन्तर्या शकुन्तलाम्}


\twolineshloka
{परिष्वज्य च बाहुभ्यां हर्षादश्रुण्यवर्तयत्}
{उवाच वचनं सत्यं लक्षये लक्षणानि च}


\twolineshloka
{तव पुत्रो विशालाक्षि चक्रवर्ती भविष्यति}
{तव भर्ता विशालाक्षि त्रैलोक्यविजयी भवेत्}


\twolineshloka
{दिव्यान्भोगाननुप्राप्ता भव त्वं वरवर्णिनि}
{एवमुक्ता रथ्तर्या परं हर्षमवाप सा}


\twolineshloka
{शकुन्तलां तदा राजा शास्त्रोक्तेनैव कर्मणा}
{ततोऽग्रमहिषीं कृत्वा सर्वाभरणभूषिताम्}


\twolineshloka
{ब्राह्मणेभ्यो धनं दत्त्वा सैनिकानां च भूपतिः}
{दौष्यन्तिं च ततो राजा पुत्रं शाकुन्तलं तदा'}


\twolineshloka
{भरतं नामतः कृत्वा यौवराज्येऽभ्यषेचयत्}
{`भरते भारमावेश्य कृतकृत्योऽभवन्नृपः}


\twolineshloka
{ततो वर्षशतं पूर्णं राज्यं कृत्वा नराधिपः}
{गत्वा वनानि दुष्यन्तः स्वर्गलोकमुपेयिवान् ॥'}


\chapter{अध्यायः १०१}
\twolineshloka
{`वैशंपायन उवाच}
{}


\twolineshloka
{दुष्यन्ताद्भरतो राज्यं यथान्यायमवाप सः}
{'तस्य तत्प्रथितं कर्म प्रावर्तत महात्मनः}


\twolineshloka
{भास्वरं दिव्यमजितं लोकसंनादनं महत्}
{स विजित्य महीपालांश्चकार वशवर्तिनः}


\twolineshloka
{चचार च सतां धर्मं प्राप्य चानुत्तमं यशः}
{स राजा चक्रवर्त्यासीत्सार्वभौमः प्रतापवान्}


\twolineshloka
{ईजे च बहुभिर्यज्ञैर्यथा शक्रो मरुत्पतिः}
{याजयामास तं कण्वो दक्षवद्भूरिदक्षिणम्}


\twolineshloka
{श्रीमद्गोविततं नाम वाजिमेधमवाप सः}
{यस्मिन्सहस्रं पद्मानां कण्वाय भरतो ददौ}


\twolineshloka
{सोऽश्वमेधशतैरीजे यमुनामनु तीरगः}
{त्रिशतैश्च सरस्वत्यां गङ्गामनु चतुःशतैः}


\twolineshloka
{दौष्यन्तिर्भरतो यज्ञैरीजे शाकुन्तलो नृपः}
{तस्माद्भरतवंशस्य विप्रतस्थे महद्यशः}


\twolineshloka
{भरतस्य वरस्त्रीषु पुत्राः संजज्ञिरे पृथक्}
{नाभ्यनन्दत्तदा राजा नानुरूपा ममेति तान्}


\twolineshloka
{ततस्तान्मातरः क्रुद्धाः पुत्रान्निन्युर्यमक्षयम्}
{ततस्तस्य नरेन्द्रस्य वितथं पुत्रजन्म तत्}


\twolineshloka
{ततो महद्भिः क्रतुभिरीजानो भरतस्तदा}
{लेभे पुत्रं भरद्पाजाद्भुमन्युं नाम भारत}


\twolineshloka
{ततः पुत्रिणमात्मानं ज्ञात्वा पौरवनन्दनः}
{भुमन्युं भरतश्रेष्ठ यौवराज्येऽभ्यषेचयत्}


\twolineshloka
{ततो दिविरथो नाम भुमन्योरभवत्सुतः}
{सुहोत्रश्च सुहोता च सुहविः सुयजुस्तथा}


\twolineshloka
{पुष्करिण्यामृचीकश्च भुमन्योरभवन्सुताः}
{तेषां ज्येष्ठः सुहोत्रस्तु राज्यमाप महीक्षिताम्}


\twolineshloka
{राजसूयाश्वमेधाद्यैः सोऽयजद्बहुभिः सवैः}
{सुहोत्रः पृथिवीं कृत्स्नां बुभुजे सागराम्बराम्}


\twolineshloka
{पूर्णां हस्तिगवाश्वैश्च बहुरत्नसमाकुलाम्}
{ममज्जेव मही तस्य भूरिभारावपीडिता}


\twolineshloka
{हस्त्यश्वरथसंपूर्णा मनुष्यकलिला भृशम्}
{सुहोत्रे राजनि तदा धर्मतः शासति प्रजाः}


\twolineshloka
{चैत्ययूपाङ्किता चासीद्भूमिः शतसहस्रशः}
{प्रवृद्धजनसस्या च सर्वदैव व्यरोचत}


\twolineshloka
{ऐक्ष्वाकी जनयामास सुहोत्रात्पृथिवीपतेः}
{अजमीढं सुमीढं च पुरुमीढं च भारत}


\twolineshloka
{अजमीढो वरस्तेषां तस्मिन्वंशः प्रतिष्ठितः}
{षट् पुत्रान्सोप्यजनयत्तिसृषु स्त्रीषु भारत}


\twolineshloka
{ऋक्षं धूमिन्यथो नीली दुष्यन्तपरमेष्ठइनौ}
{केशिन्यजनयज्जह्नुं सुतौ च जनरूषिणौ}


\threelineshloka
{विदुः संवरणं वीरमृक्षाद्राथन्तरीसुतम्}
{तथेमे सर्वपञ्चाला दुष्यन्तपरमेष्ठिनोः}
{अन्वयाः कुशिका राजञ्जन्होरमिततेजसः}


\twolineshloka
{जनरूषणयोर्ज्येष्ठमृक्षमाहुर्जनाधिपम्}
{ऋक्षात्संवरणो जज्ञे राजन्वंशकरस्तव}


\twolineshloka
{आर्क्षे संवरणे राजन्प्रशासति वसुन्धराम्}
{संक्षयः सुमहानासीत्प्रजानामिति नः श्रुतम्}


\twolineshloka
{व्यशीर्यत ततो राष्ट्रं क्षयैर्नानाविधैस्तदा}
{क्षुन्मृत्युभ्यामनावृष्ट्या व्याधिभिश्च समाहतम्}


\twolineshloka
{अभ्यघ्नन्भारतांश्चैव सपत्नानां बलानि च}
{चालयन्वसुधां चेमां बलेन चतुरङ्गिणा}


\twolineshloka
{अभ्ययात्तं च पाञ्चल्यो विजित्य तरसा महीम्}
{अक्षौहिणीभिर्दशभिः स एनं समरेऽजयत्}


\twolineshloka
{ततः सदारः सामात्यः सपुत्रः ससुहृज्जनः}
{राजा संवरणस्तस्मात्पलायत महाभयात्}


\threelineshloka
{`ते प्रतीचीं पराभूताः प्रपन्ना भारता दिशम्'}
{सिन्धोर्नदस्य महतो निकुञ्जे न्यवसंस्तदा}
{नदीविपयपर्यन्ते पर्वतस्य समीपतः}


\twolineshloka
{तत्रावसन्बहून्कालान्भारता दुर्गमाश्रिताः}
{तेषां निवसतां तत्र सहस्रं परिवत्सरान्}


\twolineshloka
{अथाभ्यगच्छद्भरतान्वसिष्ठो भगवानृषिः}
{तमागतं प्रयत्नेन प्रत्युद्गम्याभिवाद्य च}


\twolineshloka
{अर्घ्यमभ्याहरंस्तस्मै ते सर्वे भारतास्तदा}
{निवेद्य सर्वमृषये सत्कारेण सुवर्चसे}


\twolineshloka
{तमासने चोपविष्टं राजा वव्रे स्वयं तदा}
{पुरोहितो भवान्नोऽस्तु राज्याय प्रयतेमहि}


\twolineshloka
{ओमित्येवं वसिष्ठोऽपि भारतान्प्रत्यपद्यत}
{अथाभ्यषिञ्चत्साम्राज्ये सर्वक्षत्रस्य पौरवम्}


\twolineshloka
{विषाणभूतं सर्वस्यां पृथिव्यामिति नः श्रुतम्}
{भरताध्युषितं पूर्वं सोऽध्यतिष्ठत्पुरोतोतमम्}


\twolineshloka
{पुनर्बलिभृतश्चैव चक्रे सर्वमहीक्षितः}
{ततः स पृथिवीं प्राप्य पुनरीजे महाबलः}


\twolineshloka
{आजमीढो महायज्ञैर्बहुभिर्भूरिदक्षिणैः}
{ततः संवरणात्सौरी तपती सुषुवे कुरुम्}


\threelineshloka
{राजत्वे तं प्रजाः सर्वा धर्मज्ञ इति वव्रिरे}
{`महिम्ना तस्य कुरवो लेभिरे प्रत्ययं भृशम्}
{'तस्य नाम्नाऽभिविख्यातं पृथिव्यां कुरुजाङ्गलं}


\twolineshloka
{कुरुक्षेत्रं स तपसा पुण्यं चक्रे महातपाः}
{अश्ववन्तमभिष्यन्तं तथा चैत्ररथं मुनिम्}


\twolineshloka
{जनमेजयं च विख्यातं पुत्रांश्चास्यानुशुश्रुम}
{पञ्चैतान्वाहिनी पुत्रान्व्यजायत मनस्विनी}


\twolineshloka
{अविक्षितः परिक्षित्तु शबलाश्वस्तु वीर्यवान्}
{आदिराजो विराजश्च शाल्मलिश्च महाबलः}


\threelineshloka
{उच्चैःश्रवा भङ्गकारो जितारिश्चाष्टमः स्मृतः}
{एतेषामन्ववाये तु ख्यातास्ते कर्मजैर्गुणैः}
{जनमेजयादयः सप्त तथैवान्ये महारथाः}


\twolineshloka
{परीक्षितोऽभवन्पुत्राः सर्वे धर्मार्थकोविदाः}
{कक्षसेनोग्रसेनौ तु चित्रसेनश्च वीर्यवान्}


\twolineshloka
{इन्द्रसेनः सुषेणश्च भीमसेनश्च नामतः}
{जनमेजयस्य तनया भुवि ख्याता महाबलाः}


\twolineshloka
{धृतराष्ट्रः प्रथमजः पाण्डुर्बाह्लीक एव च}
{निषधश्च महातेजास्तथा जाम्बूनदो बली}


\twolineshloka
{कुण्डोदरः पदातिश्च वसातिश्चाष्टमः स्मृतः}
{सर्वे धर्मार्थकुशलाः सर्वभूतहिते रताः}


\twolineshloka
{धृतराष्ट्रोऽथ राजासीत्तस्य पुत्रोऽथ कुण्डिकः}
{हस्ती वितर्कः क्राथश्च कुण्डिनश्चापि पञ्चमः}


\twolineshloka
{हविश्रवास्तथेन्द्राभो भुमन्युश्चापराजितः}
{धृतराष्ट्रसुतानां तु त्रीनेतान्प्रथितान्भुवि}


\twolineshloka
{प्रतीपं धर्मनेत्रं च सुनेत्रं चापि भारत}
{प्रतीपः प्रथितस्तेषां बभूवाप्रतिमो भुवि}


\twolineshloka
{प्रतीपस्य त्रयः पुत्रा जज्ञिरे भरतर्षभ}
{देवापिः शन्तनुश्चैव बाह्लीकश्च महारथः}


\twolineshloka
{देवापिस्तु प्रवव्राज तेषां धर्महितेप्सया}
{शन्तनुश्च महीं लेभे बाह्लीकश्च महारथः}


\twolineshloka
{भरतस्यान्वयास्त्वेते देवकल्पा महारथाः}
{बभूवुर्ब्रह्मकल्पाश्च बहवो राजसत्तमाः}


\twolineshloka
{एवंविधा महाभागा देवरूपाः प्रहारिणः}
{अन्ववाये महाराज ऐलवंशविवर्धनाः}


\twolineshloka
{`गङ्गातीरं समागम्य दीक्षितो जनमेजय}
{अश्वमेधसहस्राणि वाजपेयशतानि च}


\twolineshloka
{पुनरीजे महायज्ञैः समाप्तवरदक्षिणैः}
{अग्निष्टोमातिरात्राणामुक्थानां सोमवत्पुनः}


\twolineshloka
{वाजपेयेष्टिसत्राणां सहस्रैश्च सुसंभृतैः}
{दृष्ट्वा शाकुन्तलो राजा तर्पयित्वा द्विजान्धनैः}


\twolineshloka
{पुनः सहस्रं पद्मानां कण्वाय भरतो ददौ}
{जम्बूनदस्य शुद्धस्य कनकस्य महायशाः}


\twolineshloka
{यस्य यूपाः शतव्यामाः परिणाहेऽथ काञ्चनाः}
{सहस्रव्याममुद्वृद्धाः सेन्द्रैर्देवैः समुच्छ्रिताः}


\twolineshloka
{स्वलङ्कृता भ्राजमानाः सर्वरत्नैर्मनोरमैः}
{हिरण्यं द्विरदानश्वान्महिषोष्ट्रगजाविकम्}


\twolineshloka
{दासीदासं धनं धान्यं गाः सुशीलाः सवत्सकाः}
{भूमिं यूपसहस्राङ्कां कण्वाय बहुदक्षिणाम्}


\twolineshloka
{बहूनां ब्रह्मकल्पानां धनं दत्त्वा क्रतून्बहून्}
{ग्रामान्गृहाणि शुभ्राणि कोटिशोथाददत्तदा}


\twolineshloka
{भरताद्भारती कीर्तिर्येनेदं भारतं कुलम्}
{'भरतस्यान्वये जाता देवकल्पा महारथाः}


\twolineshloka
{बहवो ब्रह्मकल्पाश्च बभूवुः क्षत्रसत्तमाः}
{तेषामपरिमेयानि नामधेयानि सन्त्युत}


\twolineshloka
{तेषां कुले यथा मुख्यान्कीर्तयिष्यामि भारत}
{महाभागान्देवकल्पान्सत्यार्जवपरायणान्}


\chapter{अध्यायः १०२}
\twolineshloka
{वैशंपायन उवाच}
{}


\twolineshloka
{इक्ष्वाकुवंशप्रभो राजासीत्पृथिवीपतिः}
{महाभिषगिति ख्यातः सत्यवाक्सत्यविक्रमः}


\twolineshloka
{सोऽश्वमेधसहस्रेण राजसूयशतेन च}
{तोषयामास देवेशं स्वर्गं लेभे ततः प्रभुः}


\twolineshloka
{ततः कदाचिद्ब्रह्माणमुपासांचक्रिरे सुराः}
{तत्र राजर्षयो ह्यासन्स च राजा महाभिषक्}


\twolineshloka
{अथ गङ्गा सरिच्छ्रेष्ठा समुपायात्पितामहम्}
{तस्या वासः समुद्धूतं मारुतेन शशिप्रभम्}


\twolineshloka
{ततोऽभवन्सुरगणाः सहसाऽवाङ्मुखास्तदा}
{महाभिषक्तु राजर्षिरशङ्को दृष्टवान्नदीम्}


\twolineshloka
{सोपध्यातो भगवता ब्रह्मणा तु महाभिषक्}
{उक्तश्च जातो मर्त्येषु पुनर्लोकानवाप्स्यसि}


\twolineshloka
{यया हृतमनाश्चासि गङ्गया त्वं हि दुर्मते}
{सा ते वै मानुषे लोके विप्रियाण्याचरिष्यति}


\threelineshloka
{यदा ते भविता मन्युस्तदा शापाद्विमोक्ष्यते}
{वैशंपायन उवाच}
{स चिन्तयित्वा नृपतिर्नृपानन्यांस्तपोधनान्}


\twolineshloka
{प्रतीपं रोचयामास पितरं भूरितेजसम्}
{सा महाभिषजं दृष्ट्वा नदी दैर्याच्च्युतं नृपम्}


\twolineshloka
{तमेव मनसा ध्यायन्त्युपावृत्ता सरिद्वरा}
{सा तु विध्वस्तवपुषः कश्मलाभिहतान्नृप}


\twolineshloka
{ददर्श पथि गच्छन्ती वसून्देवान्दिवौकसः}
{तथारूपांश्च तान्दृष्ट्वा प्रपच्छ सरितां वरा}


\twolineshloka
{किमिदं नष्टरूपाः स्थ कच्चित्क्षेमं दिवौकसाम्}
{तामूचुर्वसवो देवाः शप्ताः स्मो वै महानदि}


\twolineshloka
{अल्पेऽपराधे संरम्भाद्वसिष्ठेन महात्मना}
{विमूढा हि वयं सर्वे प्रच्छन्नमृषिसत्तमम्}


\twolineshloka
{सन्ध्यां वसिष्ठमासीनं तमत्यभिसृताः पुरा}
{तेन कोपाद्वयं शप्ता योनौ संभवतेति ह}


\twolineshloka
{न तच्छक्यं निवर्तयितुं यदुक्तं ब्रह्मवादिना}
{त्वमस्मान्मानुषी भूत्वा सूष्व पुत्रान्वसून्भुवि}


\twolineshloka
{न मानुषीणां जठरं प्रविशेम वयं शुभे}
{इत्युक्ता तैश्च वसुभिस्तथेत्युक्त्वाऽब्रवीदिदम्}


% Check verse!
गङ्गोवाच

मर्त्येषु पुरुषश्रेष्ठः को वः कर्ता भविष्यति

वसव ऊचुः

प्रतीपस्य सुतो राजा शान्तनुर्लोकविश्रुतः

भविता मानुषे लोके स नः कर्ता भविष्यति ॥गङ्गोवाच


\threelineshloka
{ममाप्येवं मतं देवा यथा मां वदथानघाः}
{प्रियं तस्य करिष्यामि युष्माकं चेतदीप्सितम् ॥वसव ऊचुः}
{}


\twolineshloka
{जातान्कुमारान्स्वानप्सु प्रक्षेप्तुं वै त्वमर्हसि}
{यथा नचिरकालं नो निष्कृतिः स्यात्त्रिलोकगे}


\threelineshloka
{जिघृक्षवो वयं सर्वे सुरभिं मन्दबुद्धयः}
{शप्ता ब्रह्मर्षिणा तेन तांस्त्वं मोचय चाशु नः ॥गङ्गोवाच}
{}


\threelineshloka
{एवमेतत्करिष्यामि पुत्रस्तस्य विधीयताम्}
{नास्य मोघः सङ्गमः स्यात्पुत्रहेतोर्मया सह ॥वसव ऊचुः}
{}


\twolineshloka
{तुरीयार्धं प्रदास्यामो वीर्यस्यैकैकशो वयम्}
{तेन वीर्येण पुत्रस्ते भविता तस्य चेप्सितः}


\twolineshloka
{न संपत्स्यति मर्त्येषु पुनस्तस्य तु सन्ततिः}
{तस्मादपुत्रः पुत्रस्ते भविष्यति स वीर्यवान्}


\twolineshloka
{एवं ते समयं कृत्वा गङ्गया वसवः सह}
{जग्मुः संहृष्टमनसो यथासंकल्पमञ्जसा}


\chapter{अध्यायः १०३}
\twolineshloka
{वैशंपायन उवाच}
{}


\twolineshloka
{ततः प्रतीपो राजाऽऽसीत्सर्वभूतहितः सदा}
{निषसाद समा बह्वीर्गङ्गाद्वारगतो जपन्}


\twolineshloka
{तस्य रूपगुणोपेता गङ्गा स्त्रीरूपधारिणी}
{उत्तीर्य सलिलात्तस्माल्लोभनीयतमाकृतिः}


\twolineshloka
{अधीयानस्य राजर्षेर्दिव्यरूपा मनस्विनी}
{दक्षिणं शालसङ्काशमूरुं भेजे शुभानना}


\fourlineindentedshloka
{प्रतीपस्तु महीपालस्तामुवाच यशस्विनीम्}
{`वाक्यं वाक्यविदां श्रेष्ठो धर्मनिश्चयतत्त्ववित्}
{'करोमि किं ते कल्याणि प्रियं यत्तेऽभिकाङ्क्षितम् ॥स्त्र्युवाच}
{}


\threelineshloka
{त्वामहं कामये राजन्भजमानां भजस्व माम्}
{त्यागः कामवतीनां हि स्त्रीणां सद्भिर्विगर्हितः ॥प्रतीप उवाच}
{}


\twolineshloka
{नाहं परस्त्रियं कामाद्गच्छेदं वरवर्णिनि}
{न चासवर्णां कल्याणि धर्म्यमेतद्धि मे व्रतम्}


\threelineshloka
{`यः स्वदारान्परित्यज्य पारक्यां सेवते स्त्रियम्}
{निरयान्नैव मुच्यते यावदाभूतसंप्लवम् ॥'स्त्र्युवाच}
{}


\threelineshloka
{नाश्रेयस्यस्मि नागम्या न वक्तव्या च कर्हिचित्}
{भजन्तीं भज मां राजन्दिव्यां कन्यां वरस्त्रियम् ॥प्रतीप उवाच}
{}


\twolineshloka
{त्वया निवृत्तमेतत्तु यन्मां चोदयसि प्रियम्}
{अन्यथा प्रतिपन्नं मां नाशयेद्धर्मविप्लवः}


\twolineshloka
{प्राप्य दक्षिणमूरुं मे त्वमाश्लिष्टा वराङ्गने}
{अपत्यानां स्नुषाणां च भीरु विद्ध्येतदासनम्}


\twolineshloka
{सव्योरुः कामिनीभोग्यस्त्वया स च विवर्जितः}
{तस्मादहं नाचरिष्ये त्वयि कामं वराङ्गने}


\threelineshloka
{स्नुषा मे भव सुश्रोणि पुत्रार्थं त्वां वृणोम्यहम्}
{स्नुषापक्षं हि वामोरु त्वमागम्य समाश्रिता ॥स्त्र्युवाच}
{}


\twolineshloka
{एवमप्यस्तु धर्मज्ञ संयुज्येयं सुतेन ते}
{त्वद्भक्त्या तु भजिष्यामि प्रख्यातं भारतं कुलम्}


\twolineshloka
{पृथिव्यां पार्थिवा ये च तेषां यूयं परायणम्}
{गुणा न हि मया शक्या वक्तुं वर्षशतैरपि}


\twolineshloka
{कुलस्य ये वः प्रथितास्तत्साधुत्वमथोत्तमम्}
{समयेनेह धर्मज्ञ आचरेयं च यद्विभो}


\twolineshloka
{तत्सर्वमेव पुत्रस्ते न मीमांसेत कर्हिचित्}
{एवं वसन्ती पुत्रे ते वर्धयिष्याम्यहं रतिम्}


\fourlineindentedshloka
{पुत्रैः पुण्यैः प्रियैश्चैव स्वर्गं प्राप्स्यति ते सुतः}
{वैशंपायन उवाच}
{तथेत्युक्त्वा तु सा राजंस्तत्रैवान्तरधीयत}
{`अदृश्या राजसिंहस्य पश्यतः साऽभवत्तदा ॥'}


\twolineshloka
{पुत्रजन्म प्रतीक्षन्वै स राजा तदधारयत्}
{एतस्मिन्नेव काले तु प्रतीपः क्षत्रियर्षभः}


\twolineshloka
{तपस्तेपे सुतस्यार्थे सभार्यः कुरुनन्दन}
{`प्रतीपस्य तु भार्यायां गर्भः श्रीमानवर्धत}


\twolineshloka
{श्रिया परमया युक्तः शरच्छुक्ले यथा शशी}
{ततस्तु दशमे मासि प्राजायत रविप्रभम्}


\twolineshloka
{कुमारं देवगर्भाभं प्रतीपमहिषी तदा}
{'तयोः समभवत्पुत्रो वृद्धयोः स महाभिषक्}


\twolineshloka
{शान्तस्य जज्ञे सन्तानस्तस्मादासीत्स शान्तनुः}
{`तस्य जातस्य कृत्यानि प्रतीपोऽकारयत्प्रभुः}


\twolineshloka
{जातकर्मादि विप्रेण वेदोक्तैः कर्मभिस्तदा}
{नामकर्म च विप्रास्तु चक्रुः परमसत्कृतम्}


\twolineshloka
{शान्तनोरवनीपाल वेदोक्तैः कर्मभिस्तदा}
{ततः संवर्धितो राजा शान्तनुर्लोकधार्मिकः}


\twolineshloka
{स तु लेभे परां निष्ठां प्राप्य धर्मभृतां वरः}
{धनुर्वेदे च वेदे च गतिं स परमा गतः}


\twolineshloka
{यौवनं चापि संप्राप्तः कुमारो वदतां वरः}
{'संस्मरंश्चाक्षयाँल्लोकान्विजातान्स्वेन कर्मणा}


\twolineshloka
{पुण्यकर्मकृदेवासीच्छान्तनुः कुरुसत्तमः}
{प्रतीपः शान्तनुं पुत्रं यौवनस्थं ततोऽन्वशात्}


\twolineshloka
{पुरा स्त्री मां समभ्यागाच्छान्तनो भूतये तव}
{त्वामाव्रजेद्यदि रहः सा पुत्र वरवर्णिनी}


\twolineshloka
{कामयानाऽभिरूपाढ्या दिव्यस्त्री पुत्रकाम्यया}
{सा त्वया नानुयोक्तव्या कासि कस्यासि चाङ्गने}


\twolineshloka
{यच्च कुर्यान्न तत्कर्म सा प्रष्टव्या त्वयाऽनघ}
{सन्नियोगाद्भजन्तीं तां भजेथा इत्युवाच तम्}


\twolineshloka
{एवं संदिश्य तनयं प्रतीपः शान्तनुं तदा}
{स्वे च राज्येऽभिषिच्यैनं वनं राजा विवेश ह}


\twolineshloka
{स राजा शान्तनुर्धीमान्देवराजसमद्युतिः}
{`बभूव सर्वलोकस्य सत्यवागिति संमतः}


\twolineshloka
{पीनस्कन्धो महाबाहुर्मत्तवारणविक्रमः}
{अन्वितः परिपूर्णार्थैः सर्वैर्नृपतिलक्षणैः}


\twolineshloka
{अमात्यलक्षणोपेतः क्षत्रधर्मविशेषवित्}
{वशे चक्रे महीमेको विजित्य वसुधाधिपान्}


\twolineshloka
{वेदानागमयत्कृत्स्नान्राजधर्मांश्च सर्वशः}
{ईजे च बहुभिः सत्रैः क्रतुभिर्भूरिदक्षिणैः}


\twolineshloka
{तर्पयामास विप्रांश्च वेदाध्ययनकोविदान्}
{रत्नैरुच्चावचैर्गोभिर्ग्रामैरश्वैर्धनैरपि}


\twolineshloka
{वयोरूपेण संपन्नः पौरुषेण बलेन च}
{ऐश्वर्येण प्रतापेन विक्रमेण धनेन च}


\twolineshloka
{वर्तमानश्च सत्येन सर्वधर्मविशारदः}
{तं महीपं महीपाला राजराजमकुर्वत}


\twolineshloka
{वीतशोकभयाबाधाः सुखस्वप्नप्रबोधाः}
{श्रिया भरतशार्दूल समपद्यन्त भूमिपाः}


\twolineshloka
{नियमैः सर्ववर्णानां ब्रह्मोत्तरमवर्तत}
{ब्राह्मणाभिमुखं क्षत्रं क्षत्रियाभिमुखा विशः}


\twolineshloka
{ब्रह्मक्षत्रानुकूलांश्च शूद्राः पर्यचरन्विशः}
{एवं पशुवराहाणां तथैव मृगपक्षिणाम्}


\twolineshloka
{शान्तनावथ राज्यस्थे नावर्तत वृथा वधः}
{असुखानामनाथानां तिर्यग्योनिषु वर्तताम्}


\twolineshloka
{स एव राजा सर्वेषां भूतानामभवत्पिता}
{स हस्तिनाम्नि धर्मात्मा विहरन्कुरुनन्दनः}


\twolineshloka
{तेजसा सूर्यकल्पोऽभूद्वायुना च समो बले}
{अन्तकप्रतिमः कोपे क्षमया पृथिवीसमः}


\twolineshloka
{बभूव राजा सुमतिः प्रजानां सत्यविक्रमः}
{स वनेषु च रम्येषु शैलप्रस्रवणेषु च ॥'}


\twolineshloka
{चचार मृगयाशीलः शान्तनुर्वनगोचरः}
{स मृगान्महिषांश्चैव विनिघ्नन्राजसत्तमः}


\twolineshloka
{गङ्गामनु चचारैकः सिद्धचारणसेविताम्}
{स कदाचिन्महाराज ददर्श परमां स्त्रियम्}


\twolineshloka
{जाज्वल्यमानां वपुषा साक्षाच्छ्रियमिवापराम्}
{सर्वानवद्यां सुदतीं दिव्याभरणभूषिताम्}


\twolineshloka
{सूक्ष्माम्बरधरामेकां पद्मोदरसमप्रभाम्}
{`स्नातगात्रां धौतवस्त्रां गङ्गातीररुहे वने}


\twolineshloka
{प्रकीर्णकेशीं पाणिभ्यां संस्पृशन्तीं शिरोरुहान्}
{रूपेण वयसा कान्त्या शरीरावयवैस्तथा}


\twolineshloka
{हावभावविलासैश्च लोचनाञ्चलविक्रियैः}
{श्रोणीभारेण मध्येन स्तनाभ्यामुरसा दृशा}


\twolineshloka
{कवरीभरेण पादाभ्यामिङ्गितेन स्मितेन च}
{कोकिलालापसंल्लापैर्न्यक्कुर्वन्तीं त्रिलोकगाम्}


\twolineshloka
{वाणीं च गिरिजां लक्ष्मीं योषितोन्याः सुराङ्गनाः}
{सा च शान्तनुमब्यागादलक्ष्मीमपकर्षती ॥'}


\twolineshloka
{तां दृष्ट्वा हृष्टरोमाऽभूद्विस्मितो रूपसंपदा}
{पिबन्निव च नेत्राभ्यां नातृप्यत नराधिपः}


\twolineshloka
{सा च दृष्ट्वैव राजानं विचरन्तं महाद्युतिम्}
{स्नेहादागतसौहार्दा नातृप्यत विलासिनी}


\twolineshloka
{`गङ्गा कामेन राजानं प्रेक्षमाणा विलासिनी}
{चञ्चूर्यताग्रतस्तस्य किन्नरीवाप्सरोपमा}


\twolineshloka
{दृष्ट्वा प्रहृष्टरूपोऽभूद्दर्शनादेव शान्तनुः}
{रूपेणातीत्य तिष्ठन्तीं सर्वा राजन्ययोषितः ॥'}


\twolineshloka
{तामुवाच ततो राजा सान्त्वयञ्श्लक्ष्णया गिरा}
{देवी वा दानवी वा त्वं गन्धर्वी चाथवाऽप्सराः}


\twolineshloka
{यक्षी वा पन्नगी वाऽपि मानुषी वा सुमध्यमे}
{`याऽसि काऽसि सुरप्रख्ये महिषी मे भवानघे}


\twolineshloka
{त्वां गता हि मम प्रामा वसु यन्मेऽस्ति किंचन}
{'याचे त्वां सुरगर्भाभे भार्या मे भव शोभने}


\chapter{अध्यायः १०४}
\twolineshloka
{वैशंपायन उवाच}
{}


\twolineshloka
{एतच्छ्रुत्वा वचो राज्ञः सस्मितं मृदु वल्गु च}
{यशस्विनी च साऽगच्छच्छान्तनोर्भूतये तदा}


\twolineshloka
{सा तु दृष्ट्वा नृपश्रेष्ठं चरन्तं तीरमाश्रितम्}
{वसूनां समयं स्मृत्वाऽथाभ्यगच्छदनिन्दिता}


\twolineshloka
{प्रजार्थिनी राजपुत्रं शान्तनुं पृथिवीपतिम्}
{प्रतीपवचनं चापि संस्मृत्यैव स्वयं नृपम्}


\threelineshloka
{कालोऽयमिति मत्वा सा वसूनां शापचोदिता}
{उवाच चैव राज्ञः सा ह्लादयन्ती मनो गिरा ॥गङ्गोवाच}
{}


\twolineshloka
{भविष्यामि महीपाल महिषी ते वशानुगा}
{न तु त्वं वा द्वितीयो वा ज्ञातुमिच्छेत्कथंचन}


\twolineshloka
{यत्तु कुर्यामहं राजञ्शुभं वा यदि वाऽशुभम्}
{न तद्वारयितव्याऽस्मि न वक्तव्या तथाऽप्रियम्}


\twolineshloka
{एवं हि वर्तमानेऽहं त्वयि वत्स्यामि पार्थिव}
{वारिता विप्रियं चोक्ता त्यजेयं त्वामसंशयम्}


\threelineshloka
{एष मे समयो राजन्भज मां त्वं यथेप्सितम्}
{अनुनीताऽस्मि ते पित्रा भर्ता मे त्वं भव प्रभो ॥वैशंपायन उवाच}
{}


\twolineshloka
{तथेति सा यदा तूक्ता तदा भरतसत्तम}
{प्रहर्षमतुलं लेभे प्राप्य तं पार्थिवोत्तमम्}


\twolineshloka
{प्रतिज्ञाय तु तत्तस्यास्तथेति मनुजाधिपः}
{रथमारोप्य तां देवीं जगाम स तया सह}


\twolineshloka
{सा च शान्तनुमभ्यागात्साक्षाल्लक्ष्मीरिवापरा}
{आसाद्य शान्तनुस्तां च बुभुजे कामतो वशी}


\twolineshloka
{न प्रष्टव्येति मन्वानो न स तां किंचिदूचिवान्}
{स तस्याः शीलवृत्तेन रूपौदार्यगुणेन च}


\twolineshloka
{उपचारेण च रहस्तुतोष जगतीपतिः}
{स राजा परमप्रीतः परमस्त्रीप्रलालितः}


\twolineshloka
{दिव्यरूपा हि सा देवी गङ्गा त्रिपथगामिनी}
{मानुषं विग्रहं कृत्वा श्रीमन्तं वरवर्णिनी}


\twolineshloka
{भाग्योपनतकामस्य भार्या चोपनताऽभवत्}
{शन्तनोर्नृपसिंहस्य देवराजसमद्युतेः}


\twolineshloka
{संभोगस्नेहचातुर्यैर्हावलास्यैर्मनोहरैः}
{राजानं रमयामास यथा रज्येत स प्रभुः}


% Check verse!
स राजा रतिसक्तोऽभूदुत्तमस्त्रीगुणैर्हृतः
\chapter{अध्यायः १०५}
\twolineshloka
{वैशंपायन उवाच}
{}


\twolineshloka
{संवत्सरानृतून्मासान्बुबुधे न बहून्गतान्}
{रममाणस्तया सार्धं यथाकामं नरेश्वरः}


\twolineshloka
{`दिविष्ठान्मानुषांश्चैव भोगान्भुङ्क्ते स वै नृपः}
{'आसाद्य शान्तनुः श्रीमान्मुमुदे योषितां वराम्}


\twolineshloka
{ऋतुकाले तु सा देवी दिव्यं गर्भमधारयत्}
{अष्टावजनयत्पुत्रांस्तस्मादमरसन्निभान्}


\twolineshloka
{जातं जातं च सा पुत्रं क्षिपत्यम्यसि भारत}
{सूतके कण्ठमाक्रम्य तान्निनाय यमक्षयम्}


\twolineshloka
{प्रीणाम्यहं त्वामित्युक्त्वा गङ्गास्रोतस्यमज्जयत्}
{तस्य तन्न प्रियं राज्ञः शान्तनोरभवत्तदा}


\twolineshloka
{न च तां किंचनोवाच त्यागाद्भीतो महीपतिः}
{`अमीमांस्या कर्मयोनिरागमश्चेति शान्तनुः}


\twolineshloka
{स्मरन्पितृवचश्चैव नापृच्छत्पुत्रकिल्बिषम्}
{जाताञ्जातांश्च वै हन्ति सा स्त्री सप्त वरान्सुतान्}


\twolineshloka
{शान्तनुर्धर्मभङ्गाच्च नापृच्छत्तां कथंचन}
{अष्टमं तु जिघांसन्त्यां चुक्षुभे शान्तनोर्धृतिः ॥'}


\twolineshloka
{अथैनामष्टमे पुत्रे जाते प्रहसतीमिव}
{उवाच राजा दुःखार्तः परीप्सन्पुत्रमात्मनः}


\twolineshloka
{`आलभन्तीं तदा दृष्ट्वा तां स कौरवनन्दनः}
{अब्रवीद्भरतश्रेष्ठो वाक्यं परमदुःखितः ॥'}


\threelineshloka
{मावधीः कस्य काऽसीति किं हिनत्सि सुतानिति}
{पुत्रघ्नि सुमहत्पापं संप्राप्तं ते सुगर्हितम् ॥गङ्गोवाच}
{}


\twolineshloka
{पुत्रकाम न ते हन्मि पुत्रं पुत्रवतां वर}
{जीर्णस्तु मम वासोऽयं यथा स समयः कृतः}


\twolineshloka
{अहं गङ्गा जह्नुसुता महर्षिगणसेविता}
{देवकार्यार्थसिद्ध्यर्थमुषिताऽहं त्वया सह}


\twolineshloka
{इमेऽष्टौ वसवो देवा महाभागा महौजसः}
{वसिष्ठशापदोषेण मानुषत्वमुपागताः}


\twolineshloka
{तेषां जनयिता नान्यस्त्वदृते भुवि विद्यते}
{मद्विधा मानुषी धात्री लोके नास्तीह काचन}


\twolineshloka
{तस्मात्तज्जननीहेतोर्मानुषत्वमुपागता}
{जनयित्वा वसूनष्टौ जिता लोकास्त्वयाऽक्षयाः}


\twolineshloka
{देवानां समयस्त्वेष वसूनां संश्रुतो मया}
{जातं जातं मोक्षयिष्ये जन्मतो मानुषादिति}


\twolineshloka
{तत्ते शापाद्विनिर्मुक्ता आपवस्य महात्मनः}
{स्वस्ति तेस्तु गमिष्यामि पुत्रं पाहि महाव्रतम्}


\twolineshloka
{`अयं तव सुतस्तेषां वीर्येण कुलनन्दनः}
{संभूतोतिजनं कर्म करिष्यति न संशयः ॥'}


\twolineshloka
{एष पर्यायवासो मे वसूनां सन्निधौ कृतः}
{मत्प्रसूतिं विजानीहि गङ्गादत्तमिमं सुतम्}


\chapter{अध्यायः १०६}
\twolineshloka
{शान्तनुरुवाच}
{}


\threelineshloka
{आपवो नाम कोन्वेष वसूनां किं च दुष्कृतम्}
{`शशाप यस्मात्कल्याणि स वसूंश्चारुदर्शने}
{'यस्याभिशापात्ते सर्वे मानुषीं योनिमागताः}


\twolineshloka
{अनेन च कुमारेण त्वया दत्तेन किं कृतम्}
{यस्य चैव कृतेनायं मानुषेषु निवत्स्यति}


\threelineshloka
{ईशा वै सर्वलोकस्य वसवस्ते च वै कथम्}
{मानुषेषूदपद्यन्त तन्ममाचक्ष्व जाह्नवि ॥वैशंपायन उवाच}
{}


\threelineshloka
{एवमुक्ता तदा गङ्गा राजानमिदमब्रवीत्}
{भर्तारं जाह्नवी देवी शान्तनुं पुरुषर्षभ ॥गङ्गोवाच}
{}


\twolineshloka
{यं लेभे वरुणः पुत्रं पुरा भरतसत्तम}
{वसिष्ठनामा स मुनिः ख्यात आपव इत्युत}


\twolineshloka
{तस्याश्रमपदं पुण्यं मृगपक्षिसमन्वितम्}
{मेरोः पार्श्वे नगेन्द्रस्य सर्वर्तुकुसुमावृतम्}


\twolineshloka
{स वारुणिस्तपस्तेपे तस्मिन्भरतसत्तम}
{वने पुण्यकृतां श्रेष्ठः स्वादुमूलफलोदके}


\twolineshloka
{दक्षस्य दुहिता या तु सुरभीत्यभिशब्दिता}
{गां प्रजाता तु सा देवी कश्यपाद्भरतर्षभ}


\twolineshloka
{अनुग्रहार्थं जगतः सर्वकामदुहां वराम्}
{तां लेभे गां तु धर्मात्मा होमधेनुं स वारुणिः}


\twolineshloka
{सा तस्मिंस्तापसारण्ये वसन्ती मुनिसेविते}
{चचार पुण्ये रम्ये च गौरपेतभया तदा}


\twolineshloka
{अथ तद्वनमाजग्मुः कदाचिद्भरतर्षभ}
{पृथ्वाद्या वसवः सर्वे देवा देवर्षिसेवितम्}


\twolineshloka
{ते सदारा वनं तच्च व्यचरन्त समन्ततः}
{रेमिरे रमणीयेषु पर्वतेषु वनेषु च}


\twolineshloka
{तत्रैकस्याथ भार्या तु वसोर्वासवविक्रम}
{संचरन्ती वने तस्मिन्गां ददर्श सुमध्यमा}


\twolineshloka
{नन्दिनीं नाम राजेन्द्र सर्वकामधुगुत्तमाम्}
{सा विस्मयसमाविष्टा शीलद्रविणसंपदा}


\twolineshloka
{द्यवे वै दर्शयामास तां गां गोवृषभेक्षण}
{आपीनां च सुदोग्ध्रीं च सुवालधिखुरां शुभां}


\twolineshloka
{उपपन्नां गुणैः सर्वैः शीलेनानुत्तमेन च}
{एवंगुणसमायुक्तां वसवे वसुनन्दिनी}


\twolineshloka
{दर्शयामास राजेन्द्र पुरा पौरवनन्दन}
{द्यौस्तदा तां तु दृष्ट्वैव गां गजेन्द्रेन्द्रविक्रम}


\twolineshloka
{उवाच राजंस्तां देवीं तस्या रूपगुणान्वदन्}
{एषा गौरुत्तमा देवी वारुणेरसितेक्षणा}


\twolineshloka
{ऋषेस्तस्य वरारोहे यस्येदं वनमुत्तमम्}
{अस्याः क्षीरं पिबेन्मर्त्यः स्वादु यो वै सुमध्यमे}


\threelineshloka
{दशवर्षसहस्राणि स जीवेत्स्थिरयौवनः}
{वैशंपायन उवाच}
{एतच्छ्रुत्वा तु सा देवी नृपोत्तम सुमध्यमा}


\twolineshloka
{तमुवाचानवद्याङ्गी भर्तारं दीप्ततेजसम्}
{अस्ति मे मानुषे लोके नरदेवात्मजा सखी}


\twolineshloka
{नाम्नाजितवती नाम रूपयौवनशालिनी}
{उशीनरस्य राजर्षेः सत्यसन्धस्य धीमतः}


\twolineshloka
{दुहिता प्रथिता लोके मानुषे रूपसंपदा}
{तस्या हेतोर्महाभाग सवत्सां गां ममेप्सिताम्}


\twolineshloka
{आनयस्वामरश्रेष्ठ त्वरितं पुण्यवर्धन}
{यावदस्याः पयः पीत्वा सा सखी मम मानद}


\twolineshloka
{मानुषेषु भवत्वेका जरारोगविवर्जिता}
{एतन्मम महाभाग कर्तुमर्हस्यनिन्दित}


\twolineshloka
{प्रियात्प्रियतरं ह्यस्मान्नास्ति मेऽन्यत्कथंचन}
{एतच्छ्रुत्वा वचस्तस्या देव्याः प्रियचिकीर्षया}


\twolineshloka
{पृथ्वाद्यैर्भ्रातृभिः सार्धं द्यौस्तदा तां जहार गाम्}
{तया कमलपत्राक्ष्या नियुक्तो द्यौस्तदा नृप}


\twolineshloka
{ऋषेस्तस्य तपस्वीव्रं न शशाक निरीक्षितुम्}
{हृता गौः सा तदा तेन प्रपातस्तु न तर्कितः}


\twolineshloka
{अथाश्रमपदं प्राप्तः फलान्यादाय वारुणिः}
{न चापश्यत्स गां तत्र सवत्सां काननोत्तमे}


\twolineshloka
{ततः स मृगयामास वने तस्मिंस्तपोधनः}
{नाध्यागमच्च मृगयंस्तां गां मुनिरुदारधीः}


\twolineshloka
{ज्ञात्वा तथाऽपनीतां तां वसुभिर्दिव्यदर्शनः}
{ययौ क्रोधवशं सद्यः शशाप च वसूंस्तदा}


\twolineshloka
{यस्मान्मे वसवो जह्रुर्गां वै दोग्ध्रीं सुवालधिम्}
{तस्मात्सर्वे जनिष्यन्ति मानुषेषु न संशयः}


\twolineshloka
{एवं शशाप भगवान्वसूंस्तान्भरतर्षभ}
{वशं क्रोधस्य संप्राप्त आपवो मुनिसत्तमः}


\twolineshloka
{शप्त्वा च तान्महाभागस्तपस्येव मनो दधे}
{एवं स शप्तवान्राजन्वसूनष्टौ तपोधनः}


\twolineshloka
{महाप्रभावो ब्रह्मर्षिर्देवान्क्रोधसमन्वितः}
{`एवं शप्तास्ततस्तेन मुनिना यामुनेन वै}


\twolineshloka
{अष्टौ समस्ता वंसवो दिवो दोषेण सत्तम}
{'अथाश्रमपदं प्राप्तास्ते वै भूयो महात्मनः}


\twolineshloka
{शप्ताः स्म इति जानन्त ऋषिं तमुपचक्रमुः}
{प्रसादयन्तस्तमृषिं वसवः पार्थिवर्षभ}


\twolineshloka
{लेभिरे न च तस्मात्ते प्रसादमृषिसत्तमात्}
{आपवात्पुरुषव्याघ्र सर्वधर्मविशारदात्}


\twolineshloka
{उवाच च स धर्मात्मा शप्ता यूयं धरादयः}
{अनुसंवत्सरात्सर्वे शापमोक्षमवाप्स्यथ}


\twolineshloka
{अयं तु यत्कृते यूयं मया शप्ताः स वत्स्यति}
{द्यौस्तदा मानुषे लोके दीर्घकालं स्वकर्मणः}


\twolineshloka
{नानृतं तच्चिकीर्षामि क्रुद्धो युष्मान्यदब्रुवम्}
{न प्रजास्यति चाप्येष मानुषेषु महामनाः}


\twolineshloka
{भविष्यति च धर्मात्मा सर्वशास्त्रविशारदः}
{पितुः प्रियहिते युक्तः स्त्रीभोगान्वर्जयिष्यति}


\twolineshloka
{एवमुक्त्वा वसून्सर्वान्सजगाम महानृषिः}
{ततो मामुपजग्मुस्ते समेता वसवस्तदा}


\twolineshloka
{अयाचन्त च मां राजन्वरं तच्च मया कृतम्}
{जाताञ्जातान्प्रक्षिपास्मान्स्वयं गङ्गे त्वमम्भसि}


\twolineshloka
{एवं तेषामहं सम्यक् शप्तानां राजसत्तम}
{मोक्षार्थं मानुषाल्लोकाद्यथावत्कृतवत्यहम्}


\twolineshloka
{अयं शापादृषेस्तस्य एक एव नृपोत्तम}
{द्यौ राजन्मानुषे लोके चिरं वत्स्यति भारत}


\threelineshloka
{अयं कुमारः पुत्रस्ते विवृद्धः पुनरेष्यति}
{अहं च ते भविष्यामि आह्वानोपगता नृप ॥वैशंपायन उवाच}
{}


\twolineshloka
{एतदाख्याय सा देवी तत्रैवान्तरधीयत}
{आदाय च कुमारं तं जगामाथ यथेप्सितम्}


\twolineshloka
{स तु देवव्रतो नाम गाङ्गेय इति चाभवत्}
{द्युनामा शान्तनोः पुत्रः शान्तनोरधिको गुणैः}


\twolineshloka
{शान्तनुश्चापि शोकार्तो जगाम स्वपुरं ततः}
{तस्याहं कीर्तयिष्यामि शान्तनोरधिकान्गुणान्}


\twolineshloka
{महाभाग्यं च नृपतेर्भारतस्य महात्मनः}
{यस्येतिहासो द्युतिमान्महाभारतमुच्यते}


\chapter{अध्यायः १०७}
\twolineshloka
{वैशंपायन उवाच}
{}


\twolineshloka
{स राजा शान्तनुर्धीमान्देवराजर्षिसत्कृतः}
{धर्मात्मा सर्वलोकेषु सत्यवागिति विश्रुतः}


\threelineshloka
{शान्तनोः कीर्तयिष्यामि सर्वानेव गुणानहम्}
{दमो दानं क्षमा बुद्धिर्ह्रीर्धृतिस्तेज उत्तमम्}
{नित्यान्यासन्महासत्वे शान्तनौ पुरुषर्षभे}


\twolineshloka
{एवं स गुणसंपन्नो धर्मार्थकुशलो नृपः}
{आसीद्भरतवंशस्य गोप्ता सर्वजनस्य च}


\twolineshloka
{कम्बुग्रीवः पृथुव्यंसो मत्तवारणविक्रमः}
{अन्वितः परिपूर्णार्थैः सर्वैर्नृपतिलक्षणैः}


\twolineshloka
{तस्य कीर्तिमतो वृत्तमवेक्ष्य सततं नराः}
{धर्म एव परः कामादर्थाच्चेति व्यवस्थितः}


\twolineshloka
{एवमासीन्महासत्वः शान्तनुर्भरतर्षभ}
{न चास्य सदृशः कश्चिद्धर्मतः पार्थिवोऽभवत्}


\twolineshloka
{वर्तमानं हि धर्मेषु सर्वधर्मभृतां वरम्}
{तं महीपा महीपालं राजराज्येऽभ्यषेचयन्}


\twolineshloka
{वीतशोकभयाबाधाः सुखस्वप्ननिबोधनाः}
{पतिं भारतगोप्तारं समपद्यन्त भूमिपाः}


\twolineshloka
{तेन कीर्तिमता शिष्टाः शक्रप्रतिमतेजसा}
{यज्ञदानक्रियाशीलाः समपद्यन्त भूमिपाः}


\twolineshloka
{शान्तनुप्रमुखैर्गुप्ते लोके नृपतिभिस्तदा}
{नियमात्सर्ववर्णानां धर्मोत्तरमवर्तत}


\twolineshloka
{ब्रह्म पर्यचरत्क्षत्रं विशः क्षत्रमनुव्रताः}
{ब्रह्मक्षत्रानुरक्ताश्च शूद्राः पर्यचरन्विशः}


\twolineshloka
{स हास्तिनपुरे रम्ये कुरूणां पुटभेदने}
{वसन्सागरपर्यन्तामन्वशासद्वसुन्धराम्}


\twolineshloka
{स देवराजसदृशो धर्मज्ञः सत्यवागृजुः}
{दानधर्मतपोयोगाच्छ्रिया परमया युतः}


\twolineshloka
{अरागद्वेषसंयुक्तः सोमवत्प्रियदर्शनः ॥तेजसा सूर्यकल्पोऽभूद्वायुवेगसमो जवे}
{अन्तकप्रतिमः कोपे क्षमया पृथिवीसमः}


\twolineshloka
{वधः पशुवराहाणां तथैव मृगपक्षिणाम्}
{शान्तनौ पृथिवीपाले नावर्तत तथा नृप}


\twolineshloka
{ब्रह्मधर्मोत्तरे राज्ये शान्तनुर्विनयात्मवान्}
{समं शशास भूतानि कामरागविवर्जितः}


\threelineshloka
{`चकोरनेत्रस्ताम्रास्यः सिंहर्षभगतिर्युवा}
{गुणैरनुपमैर्युक्तः समस्तैराभिगामिकैः}
{गम्भीरः सत्वसंपन्नः पूर्णचन्द्रनिभाननः ॥'}


\twolineshloka
{देवर्षिपितृयज्ञार्थमारभ्यन्त तदा क्रियाः}
{न चाधर्मेण केषांचित्प्राणिनामभवद्वधः}


\twolineshloka
{असुखानामनाथानां तिर्यग्योनिषु वर्तताम्}
{स एव राजा सर्वेषां भूतानामभवत्पिता}


\twolineshloka
{तस्मिन्कुरुपतिश्रेष्ठे राजराजेश्वरे सति}
{श्रिता वागभवत्सत्यं दानधर्माश्रितं मनः}


\threelineshloka
{`यज्ञार्थं पशवः सृष्टाः संतानार्थं च मैथुनम्}
{'स समाः षोडशाष्टौ च चतस्रोऽष्टौ तथाऽपराः}
{रतिमप्राप्नुवन्स्त्रीषु बभूव वनगोचरः}


\twolineshloka
{तथारूपस्तथाचारस्तथावृत्तस्तथाश्रुतः}
{गाङ्गेयस्तस्य पुत्रोऽभून्नाम्ना देवव्रतो वसुः}


\twolineshloka
{सर्वास्त्रेषु स निष्णातः पार्थिवेष्वितरेषु च}
{महाबलो महासत्वो महावीर्यो महारथः}


\twolineshloka
{स कदाचिन्मृगं विद्ध्वा गङ्गामनुसरन्नदीम्}
{भागीरथीमल्पजलां शान्तनुर्दृष्टवान्नृपः}


\twolineshloka
{तां दृष्ट्वा चिन्तयामास शान्तनुः पुरुषर्षभः}
{स्यन्दते किं न्वियं नाद्य सरिच्छ्रेष्ठा यथा पुरा}


\twolineshloka
{ततो निमित्तमन्विच्छन्ददर्श स महामनाः}
{कुमारं रूपसंपन्नं बृहन्तं चारुदर्शनम्}


\twolineshloka
{दिव्यमस्त्रं विकुर्वाणं यथा देवं पुरन्दरम्}
{कृत्स्नां गङ्गां समावृत्य शरैस्तीक्ष्णैरवस्थितम्}


\twolineshloka
{तां शरैराचितां दृष्ट्वा नदीं गङ्गां तदन्तिके}
{अभवद्विस्मितो राजा दृष्ट्वा कर्मातिमानुषम्}


\twolineshloka
{जातमात्रं पुरा दृष्टं तं पुत्रं शान्तनुस्तदा}
{नोपलेभे स्मृतिं धीमानभिज्ञातुं तमात्मजम्}


\twolineshloka
{स तु तं पितरं दृष्ट्वा मोहयामास मायया}
{संमोह्य तु ततः क्षिप्रं तत्रैवान्तरधीयत}


\twolineshloka
{तदद्बुतं ततो दृष्ट्वा तत्र राजा स शान्तनुः}
{सङ्कमानः सुतं गङ्गामब्रवीद्दर्शयेति ह}


\twolineshloka
{दर्शयामास तं गङ्गा बिभ्रती रूपमुत्तमम्}
{गृहीत्वा दक्षिणे पाणौ तं कुमारमलङ्कृतम्}


\threelineshloka
{अलङ्कृतामाभरणैर्विरजोम्बरधारिणीम्}
{दृष्टपूर्वामपि स तां नाभ्यजानात्स शान्तनुः ॥गङ्गोवाच}
{}


\twolineshloka
{यं पुत्रमष्टमं राजंस्त्वं पुरा मय्यविन्दथाः}
{स चायं पुरुषव्याघ्र सर्वास्त्रविदनुत्तमः}


\twolineshloka
{गृहाणेमं महाराज मया संवर्धितं सुतम्}
{आदाय पुरुषव्याघ्र नयस्वैनं गृहं विभो}


\twolineshloka
{वेदानधिजगे साङ्गान्वसिष्ठादेष वीर्यवान्}
{कृतास्त्रः परमेष्वासो देवराजसमो युधि}


\twolineshloka
{सुराणां संमतो नित्यमसुराणां च भारत}
{उशा वेद यच्छास्त्रमयं तद्वेद सर्वशः}


\twolineshloka
{तथैवाङ्गिरसः पुत्रः सुरसुरनमस्कृतः}
{यद्वेद शास्त्रं तच्चापि कृत्स्नमस्मिन्प्रतिष्ठितम्}


\twolineshloka
{तव पुत्रे महाबाहौ साङ्गोपाङ्गं महात्मनि}
{ऋषिः परैरनाधृष्यो जामदग्न्यः प्रतापवान्}


\twolineshloka
{यदस्त्रं वेद राभश्च तदेतस्मिन्प्रतिष्ठितम्}
{महेष्वासमिमं राजन्राजधर्मार्थकोविदम्}


\fourlineindentedshloka
{मया दत्तं निजं पुत्रं वीरं वीर गृहं नय}
{वैशंपायन उवाच}
{`इत्युक्त्वा सा महाभागा तत्रैवान्तरधीयत}
{'तयैवं समनुज्ञातः पुत्रमादाय शान्तनुः}


\twolineshloka
{भ्राजमानं यथाऽदित्यमाययौ स्वपुरं प्रति}
{पौरवस्तु पुरीं गत्वा पुरन्दरपुरोपमाम्}


\twolineshloka
{सर्वकामसमृद्धार्थं मेने सोत्मानमात्मना}
{पौरवेषु ततः पुत्रं राज्यार्थमभयप्रदम्}


\twolineshloka
{गुणवन्तं महात्मानं यौवराज्येऽभ्यषेचयत्}
{पौरवाञ्शान्तनोः पुत्रः पितरं च महायशाः}


\twolineshloka
{राष्ट्रं च रञ्जयामास वृत्तेन भरतर्षभ}
{स तथा सह पुत्रेण रममाणो महीपतिः}


\twolineshloka
{वर्तयामास वर्षाणि चत्वार्यमितविक्रमः}
{स कदाचिद्वनं यातो यमुनामभितो नदीम्}


\twolineshloka
{महीपतिरनिर्देश्यमाजिघ्रद्गन्धमुत्तमम्}
{तस्य प्रभवमन्विच्छन्विचचार समन्ततः}


\twolineshloka
{स ददर्श तदा कन्यां दाशानां देवरूपिणीम्}
{तामपृच्छत्स दृष्ट्वैव कन्यामसितलोचनाम्}


\twolineshloka
{कस्य त्वमसि का चासि किं च भीरु चिकीर्षसि}
{साऽब्रवीद्दाशकन्याऽस्मि धर्मार्थं वाहये तरिम्}


\twolineshloka
{पितुर्नियोगाद्भद्रं ते दाशराज्ञो महात्मनः}
{रूपमाधुर्यगन्धैस्तां संयुक्तां देवरूपिणीम्}


\twolineshloka
{समीक्ष्य राजा दाशेयीं कामयामास शान्तनुः}
{स गत्वा पितरं तस्या वरयामास तां तदा}


\twolineshloka
{पर्यपृच्छत्ततस्तस्याः पितरं सोत्मकारणात्}
{स च तं प्रत्युवाचेदं दाशराजो महीपतिम्}


\twolineshloka
{जातमात्रैव मे देया वराय वरवर्णिनी}
{हृदि कामस्तु मे कश्चित्तं निबोध जनेश्वर}


\twolineshloka
{यदीमां धर्मपत्नीं त्वं मत्तः प्रार्थयसेऽनघ}
{सत्यवागसि सत्येन समयं कुरु मे ततः}


\threelineshloka
{समयेन प्रदद्यां ते कन्यामहमिमां नृप}
{न हि मे त्वत्समः कश्चिद्वरो जातु भविष्यति ॥शान्तनुरुवाच}
{}


\threelineshloka
{श्रुत्वा तव वरं दाश व्यवस्येयमहं तव}
{दातव्यं चेत्प्रदास्यामि न त्वदेयं कथंचन ॥दाश उवाच}
{}


\threelineshloka
{अस्यां जायेत यः पुत्रः स राजा पृथिवीपते}
{त्वदूर्ध्वमभिषेक्तव्यो नान्यः कश्चन पार्थिव ॥वैशंपायन उवाच}
{}


\twolineshloka
{नाकामयत तं दातुं वरं दाशाय शान्तनुः}
{शरीरजेन तीव्रेण दह्यमानोऽपि भारत}


\twolineshloka
{स चिन्तयन्नेव तदा दाशकन्यां महीपतिः}
{प्रत्ययाद्धास्तिनपुरं कामोपहतचेतनः}


\twolineshloka
{ततः कदाचिच्छोचन्तं शान्तनुं ध्यानमास्थितम्}
{पुत्रो देवव्रतोऽभ्येत्य पितरं वाक्यमब्रवीत्}


\twolineshloka
{सर्वतो भवतः क्षेमं विधेयाः सर्वपार्थिवाः}
{तत्किमर्थमिहाभीक्ष्णं परिशोचसि दुःखितः}


\twolineshloka
{ध्यायन्निव च मां राजन्नाभिभाषसि किंचन}
{न चाश्वेन विनिर्यासि विवर्णो हरिणः कृशः}


\threelineshloka
{व्याधिमिच्छामि ते ज्ञातुं प्रतिकुर्यां हि तत्र वै}
{`वैशंपायन उवाच}
{स तं काममवाच्यं वै दाशकन्यां प्रतीदृशम्}


\twolineshloka
{विवर्तुं नाशकत्तस्मै पिता पुत्रस्य शान्तनुः}
{'एवमुक्तः स पुत्रेण शान्तनुः प्रत्यभाषत}


\twolineshloka
{असंशयं ध्यानपरो यथा वत्स तथा शृणु}
{अपत्यं नस्त्वमेवैकः कुले महति भारत}


\twolineshloka
{शस्त्रनित्यश्च सततं पौरुषे पर्यवस्थितः}
{अनित्यतां च लोकानामनुशोचामि पुत्रक}


\twolineshloka
{कथंचित्तव गाङ्गेय विपत्तौ नास्ति नः कुलम्}
{असंशयं त्वमेवैकः शतादपि वरः सुतः}


\twolineshloka
{न चाप्यहं वृथा भूयो दारान्कर्तुमिहोत्सहे}
{संतानस्याविनाशाय कामये भद्रमस्तु ते}


\threelineshloka
{अनपत्यतैकपुत्रत्वमित्याहुर्धर्मवादिनः}
{`चक्षुरेकं च पुत्रश्च अस्ति नास्ति च भारत}
{चक्षुर्नाशे तनोर्नाशः पुत्रनाशे कुलक्षयः ॥'}


\twolineshloka
{अग्निहोत्रं त्रयी विद्या यज्ञाश्च सहदक्षिणाः}
{सर्वाण्येतान्यपत्यस्य कलां नार्हन्ति षोडशीम्}


\twolineshloka
{एवमेतन्मनुष्येषु तच्च सर्वं प्रजास्विति}
{यदपत्यं महाप्राज्ञ तत्र मे नास्ति संशयः}


\twolineshloka
{`अपत्येनानृणो लोके पितॄणां नास्ति संशयः}
{'एषा त्रयी पुराणानां देवतानां च शाश्वती}


% Check verse!
`अपत्यं कर्म विद्या च त्रीणि ज्योतींषि भारत ॥'
\twolineshloka
{त्वं च शूरः सदाऽमर्षी शस्त्रनित्यश्च भारत}
{नान्यत्र युद्धात्तस्मात्ते निधनं विद्यते क्वचित्}


\threelineshloka
{सोऽस्मि संशयमापन्नस्त्वयि शान्ते कथं भवेत्}
{इति ते कारणं तात दुःखस्योक्तमशेषतः ॥वैशंपायन उवाच}
{}


\twolineshloka
{ततस्तत्कारणं राज्ञो ज्ञात्वा सर्वमशेषतः}
{देवव्रतो महाबुद्धिः प्रज्ञया चान्वचिन्तयत्}


\twolineshloka
{अपत्यफलसंयुक्तमेतच्छ्रुत्वा पितुर्वचः}
{सूतं भूयोऽपि संतप्त आह्वयामास वै पितुः}


\twolineshloka
{सूतस्तु कुरुमुख्यस्य उपयातस्तदाज्ञया}
{तमुवाच महाप्राज्ञो भीष्मो वै सारथिं पितुः}


\twolineshloka
{त्वं सारथे पितुर्मह्यं सखासि रथधूर्गतः}
{अपि जानासि यदि वै कस्यां भावो नृपस्य तु}


\threelineshloka
{तदाचक्ष्व भवान्पृष्टः करिष्ये न तदन्यथा}
{सूत उवाच}
{दाशकन्या कुरुश्रेष्ठ तत्र भावः पितुर्गतः}


\twolineshloka
{वृतः स नरदेवेन तदा वचनमब्रवीत्}
{योऽस्यां पुमान्भवेज्जातः स राजा त्वदनन्तरम्}


\twolineshloka
{नाकामयत तं दातुं पिता तव वरं तदा}
{स चापि निश्चयस्तस्य न च दद्यां ततोऽन्यथा}


\fourlineindentedshloka
{एतत्ते कथितं वीर कुरुष्व यदनन्तरम्}
{वैशंपायन उवाच}
{ततः स पितुराज्ञाय मतं सम्यगवेक्ष्य च}
{ज्ञात्वा च मानसं पुत्रः प्रययौ यमुनां प्रति}


\twolineshloka
{क्षत्रियैः सह धर्मात्मा पुराणैर्धर्मचारिभिः}
{उच्चैश्श्रवसमागम्य कन्यां वव्रे पितुः स्वयम्'}


\twolineshloka
{तं दाशः प्रतिजग्राह विधिवत्प्रतिपूज्य च}
{अब्रवीच्चैनमासीनं राजसंसदि भारत}


\twolineshloka
{`राज्यशुल्का प्रदातव्या कन्येयं याचतां वर}
{अपत्यं यद्भवेदस्याः स राजाऽस्तु पितुः परम् ॥'}


\twolineshloka
{त्वमेवात्र महाबाहो शान्तनोर्वंशवर्धनः}
{पुत्रः शस्त्रभृतां श्रेष्ठः किं नु वक्ष्यामि ते वचः}


\threelineshloka
{`कुमारिकायाः शुल्कार्थं किंचिद्वक्ष्यामि भारत}
{'कोहि संबन्धखं श्लाघ्यमीप्सितं यौनमीदृशम्}
{अतिक्रामन्न तप्येत साक्षादपि शतक्रतुः}


\twolineshloka
{अपत्यं चैतदार्यस्य यो युष्माकं समो गुणैः}
{यस्य शुक्रात्सत्यवती संभूता वरवर्णिनी}


\twolineshloka
{तेन मे बहुशस्तात पिता ते परिकीर्तितः}
{अर्हः सत्यवतीं वोढुं धर्मज्ञः स नराधिपः}


\twolineshloka
{`इयं सत्यवती देवी पितरं तेऽब्रवीत्तथा}
{अर्थितश्चापिराजर्षिः प्रत्याख्यातः पुरा मया'}


\twolineshloka
{कन्यापितृत्वात्किंचित्तु वक्ष्यामि त्वां नराधिप}
{बलवत्सपत्नतामत्र दोषं पश्यामि केवलम्}


\twolineshloka
{`भूयांसं त्वयि पश्यामि तद्दोषमपराजित}
{'यस्य हि त्वं सपत्नः स्या गन्धर्वस्यासुरस्य वा}


\twolineshloka
{न स जातु चिरं जीवेत्त्वयि क्रुद्धे परन्तप}
{एतावानत्र दोषो हि नान्यः कश्चन पार्थिव}


\twolineshloka
{एतज्जानीहि भद्रं ते दानादाने परन्तप ॥वैशंपायन उवाच}
{}


\twolineshloka
{एवमुक्तस्तु गाङ्गेयस्तद्युक्तं प्रत्यभाषत}
{शृण्वतां भूमिपालानां पितुरर्थाय भारत}


\twolineshloka
{`इदं वचनमाधत्स्व नास्ति वक्तास्य मत्समः}
{अन्यो जातो न जनिता न च कश्चन संप्रति'}


\twolineshloka
{एवमेतत्करिष्यामि यथा त्वमनुभाषसे}
{योऽस्यां जनिष्यते पुत्रः स नो राजा भविष्यति}


\twolineshloka
{इत्युक्तः पुनरेव स्म तं दाशः प्रत्यभाषत}
{चिकीर्षुर्दुष्करं कर्म राज्यार्थे भरतर्षभ}


\twolineshloka
{त्वमेव नाथः संप्राप्तः शान्तनोरमितद्युते}
{कन्यायाश्चैव धर्मात्मन्प्रभुर्दानाय चेश्वरः}


\twolineshloka
{इदं तु वचनं सौम्य कार्यं चैव निबोध मे}
{कौमारिकाणां शीलेन वक्ष्याम्यहमरिन्दम}


\twolineshloka
{यत्त्वया सत्यवत्यर्थे सत्यधर्मपरायण}
{राजमध्ये प्रतिज्ञातमनुरूपं तवैव तत्}


\threelineshloka
{नान्यथा तन्महाबाहो संशयोऽत्र न कश्चन}
{तवापत्यं भवेद्यत्तु तत्र नः संशयो महान् ॥वैशंपायन उवाच}
{}


\threelineshloka
{तस्यैतन्मतमाज्ञाय सत्यधर्मपरायणः}
{प्रत्यजानात्तदा राजन्पितुः प्रियचिकीर्षया ॥गाङ्गेय उवाच}
{}


\twolineshloka
{`उच्चैश्श्रवः समाधत्स्व प्रतिज्ञां जनसंसदि}
{ऋषयो वाथ वा देवा भूतान्यन्तर्हितानि च}


\twolineshloka
{यानि यानीह शृण्वन्तु नास्ति वक्तास्य मत्समः}
{'दाशराज निबोधेदं वचनं मे नृपोत्तम}


\twolineshloka
{शृण्वतां भूमिपालानां यद्ब्रवीमि पितुः कृते}
{राज्यं तावत्पूर्वमेव मया त्यक्तं नराधिपाः}


\twolineshloka
{अपत्यहेतोरपि च करिष्येऽद्य विनिश्चयम्}
{अद्यप्रभृति मे दाश ब्रह्मचर्यं भविष्यति}


\twolineshloka
{अपुत्रस्यापि मे लोका भविष्यन्त्यक्षया दिवि}
{`न हि जन्मप्रभृत्युक्तं मया किंचिदिहानृतम्}


\twolineshloka
{यावत्प्राणा ध्रियन्ते वै मम देहं समाश्रिताः}
{तावन्न जनयिष्यामि पित्रे कन्यां प्रयच्छ मे}


\threelineshloka
{परित्यजाम्यहं राज्यं मैथुनं चापि सर्वशः}
{ऊर्ध्वरेता भविष्यामि दाश सत्यं ब्रवीमि ते ॥'वैशंपायन उवाच}
{}


\twolineshloka
{तस्य तद्वचनं श्रुत्वा संप्रहृष्टतनूरुहः}
{ददानीत्येव तं दाशो धर्मात्मा प्रत्यभाषत}


\twolineshloka
{ततोन्तरिक्षेऽप्सरसो देवाः सर्षिगणास्तदा}
{`तद्दृष्टा दुष्करं कर्म प्रशशंसुश्च पार्थिवाः ॥'}


\twolineshloka
{अभ्यवर्षन्त कुसुमैर्भीष्मोऽयमिति चाब्रुवन्}
{ततः स पितुरर्थाय तामुवाच यशस्विनीम्}


\twolineshloka
{अधिरोह रथं मातर्गच्छावः स्वगृहानिति}
{एवमुक्त्वा तु भीष्मस्तां रथमारोप्य भामिनीं}


\twolineshloka
{आगम्य हास्तिनपुरं शान्तनोः संन्यवेदयत्}
{तस्य तद्दुष्करं कर्म प्रशशंसुर्नाराधिपाः}


\twolineshloka
{समेताश्च पृथक्चैव भीष्मोयमिति चाब्रुवन्}
{तच्छ्रुत्वा दुष्करं कर्म कृतं भीष्मेण शान्तनुः}


\twolineshloka
{बभूव दुःखितो राजा चिररात्राय भारत}
{स तेन कर्मणा सूनोः प्रीतस्तस्मै वरं ददौ ॥'}


\twolineshloka
{स्वच्छन्दमरणं तुष्टो ददौ तस्मै महात्मने}
{न ते मृत्युः प्रभविता यावज्जीवितुमिच्छसि}


% Check verse!
त्वत्तो ह्यनुज्ञां संप्राप्य मृत्युः प्रभविताऽनघ
\chapter{अध्यायः १०८}
\twolineshloka
{वैशंपायन उवाच}
{}


\twolineshloka
{`चेदिराजसुतां ज्ञात्वा दाशराजेन वर्धिताम्}
{विवाहं कारयामास शास्त्रदृष्टेन कर्मणा ॥'}


\twolineshloka
{ततो विवाहे निर्वृत्ते स राजा शान्तनुर्नृपः}
{तां कन्यां रूपसंपन्नां स्वगृहे संन्यवेशयत्}


\twolineshloka
{ततः शान्तनवो धीमान्सत्यवत्यामजायत}
{वीरश्चित्राङ्गदो नाम वीर्यवान्पुरुषेश्वरः}


\twolineshloka
{अथापरं महेष्वासं सत्यवत्यां सुतं प्रभुः}
{विचित्रवीर्यं राजानं जनयामास वीर्यवान्}


\twolineshloka
{अप्राप्तवति तस्मिंस्तु यौवनं पुरुषर्षभे}
{स राजा शान्तनुर्धीमान्कालधर्ममुपेयिवान्}


\twolineshloka
{स्वर्गते शान्तनौ भीष्मश्चित्राङ्गदमरिन्दनम्}
{स्थापयामास वै राज्ये सत्यवत्या मते स्थितः}


\twolineshloka
{स तु चित्राङ्गदः शौर्यात्सर्वांश्चिक्षेप पार्थिवान्}
{मनुष्यं न हि मेन स कंचित्सदृशमात्मनः}


\threelineshloka
{तं क्षिपन्तं सुरांश्चैव मनुष्यानसुरांस्तथा}
{गन्धर्वराजो बलवांस्तुल्यनामाऽभ्ययात्तदा ॥गन्धर्व उवाच}
{}


\twolineshloka
{`त्वं वै सदृशनामासि युद्धं देहि नृपात्मज}
{नाम वाऽन्यत्प्रगृह्णीष्व यदि युद्धं न दास्यसि}


\twolineshloka
{त्वयाहं युद्धमिच्छामि त्वत्सकाशं तु नामतः}
{आगतोस्मि वृथाऽऽभाष्य न गच्छेन्नाम ते मम}


\twolineshloka
{इत्युक्त्वा गर्जमानौ तौ हिरण्वत्यास्तटं गतौ'}
{तेनास्य सुमहद्युद्धं कुरुक्षेत्रे बभूव ह}


\twolineshloka
{तयोर्बलवतोस्तत्र गन्धर्वकुरुमुख्ययोः}
{नद्यास्तीरे हिरण्वत्याः समास्तिस्रोऽभवद्रणः}


\twolineshloka
{तस्मिन्विमर्दे तुमुले शस्त्रवर्षसमाकुले}
{मायाधिकोऽवधीद्वीरं गन्धर्वः कुरुसत्तमम्}


\twolineshloka
{स हत्वा तु नरश्रेष्ठं चित्राङ्गदमरिन्दमम्}
{अन्ताय कृत्वा गन्धर्वो दिवमाचक्रमे ततः}


\twolineshloka
{तस्मिन्पुरुषशार्दूले निहते भूरितेजसि}
{भीष्मः शान्तनवो राजा प्रेतकार्याण्यकारयत्}


\twolineshloka
{विचित्रवीर्यं च तदा बालमप्राप्तयौवनम्}
{कुरुराज्ये महाबाहुरभ्यषिञ्चदनन्तरम्}


\twolineshloka
{विचित्रवीर्यः स तदा भीष्मस्य वचने स्थितः}
{अन्वशासन्महाराज पितृपैतामहं पदम्}


\twolineshloka
{स धर्मशास्त्रकुशलं भीष्मं शान्तनवं नृपः}
{पूजयामास धर्मेण स चैनं प्रत्यपालयत्}


\chapter{अध्यायः १०९}
\twolineshloka
{वैशंपायन उवाच}
{}


\twolineshloka
{हते चित्राङ्गदे भीष्मो बाले भ्रातरि कौरव}
{पालयामास तद्राज्यं सत्यवत्या मते स्थितः}


\threelineshloka
{`तथा विचित्रवीर्यं तु वर्तमानं सुखेऽतुले}
{'संप्राप्तयौवनं दृष्ट्वा भ्रातरं धीमतां वरः}
{भीष्मो विचित्रवीर्यस्य विवाहायाकरोन्मतिम्}


\twolineshloka
{अथ काशिपतेर्भीष्मः कन्यास्तिस्रोऽप्सरोपमाः}
{शुश्राव सहिता राजन्वृण्वाना वै स्वयंवरम्}


\twolineshloka
{ततः स रथिनां श्रेष्ठो रथेनैकेन शत्रुजित्}
{जगामानुमते मातुः पुरीं वाराणसीं प्रभुः}


\twolineshloka
{तत्र राज्ञः समुदितान्सर्वतः समुपागतान्}
{ददर्श कन्यास्ताश्वै भीष्मः शान्तनुनन्दनः}


\twolineshloka
{`तासां कामेन संमत्ताः सहिताः काशिकोसलाः}
{वङ्गाः पुण्ड्राः कलिङ्गाश्च ते जग्मुस्तां पुरींप्रति ॥'}


\twolineshloka
{कीर्त्यमानेषु राज्ञां तु तदा नामसु सर्वशः}
{एकाकिनं तदा भीष्मं वृद्धं शान्तनुनन्दनम्}


\twolineshloka
{सोद्वेगा इव तं दृष्ट्वा कन्याः परमशोभनाः}
{अपाक्रामन्त ताः सर्वा वृद्ध इत्येव चिन्तया}


\twolineshloka
{वृद्धः परमधर्मात्मा वलीपलितधारणः}
{किकारणमिहायातो निर्लज्जो भरतर्षभः}


\twolineshloka
{मिथ्याप्रतिज्ञो लोकेषु किं वदिष्यति भारत}
{ब्रह्मचारीति भीष्मो हि वृथैव प्रथितो भुवि}


\threelineshloka
{इत्येवं प्रबुवन्तस्ते हसन्ति स्म नृपाधमाः}
{वैशंपायन उवाच}
{क्षत्रियाणां वचः श्रुत्वा भीष्मश्चुक्रोध भारत}


\twolineshloka
{भीष्मस्तदा स्वयं कन्या वरयामास ताः प्रभुः}
{उवाच च महीपालान्राजञ्जलदनिःस्वनः}


\twolineshloka
{रथमारोप्य ताः कन्या भीष्मः प्रहरतां वरः}
{आहूय दानं कन्यानां गुणवद्भ्यः स्मृतं बुधैः}


\twolineshloka
{अलङ्कृत्य यथाशक्ति प्रदाय च धनान्यपि}
{प्रयच्छन्त्यपरे कन्यां मिथुनेन गवामपि}


\twolineshloka
{वित्तेन कथितेनान्ये बलेनान्येऽनुमान्य च}
{प्रमत्तामुपयन्त्यन्ये स्वयमन्ये च विन्दते}


\twolineshloka
{आर्षं विधिं पुरस्कृत्य दारान्विन्दन्ति चापरे}
{अष्टमं तमथो वित्त विवाहं कविभिर्वृतम्}


\twolineshloka
{स्वयंवरं तु राजन्याः प्रशंसन्त्युपयान्ति च}
{प्रमथ्य तु हृतामाहुर्ज्यायसीं धर्मवादिनः}


\twolineshloka
{ता इमाः पृथिवीपाला जिहीर्षामि बलादितः}
{ते यतध्वं परं शक्त्या विजयायेतराय वा}


\threelineshloka
{स्थितोऽहं पृथिवीपाला युद्धाय कृतनिश्चयः}
{वैशंपायन उवाच}
{एवमुक्त्वा महीपालान्काशिराजं च वीर्यवान्}


\twolineshloka
{सर्वाः कन्याः स कौरव्यो रथमारोप्य च स्वकम्}
{आमन्त्र्य च स तान्प्रायाच्छीघ्रं कन्याः प्रगृह्य ताः}


\twolineshloka
{ततस्ते पार्थिवाः सर्वे समुत्पेतुरमर्षिताः}
{संस्पृशन्तः स्वकान्बाहून्दशन्तो दशनच्छदान्}


\twolineshloka
{तेषामाभरणान्याशु त्वरितानां विमुञ्चताम्}
{आमुञ्चतां च वर्माणि संभ्रमः सुमहानभूत्}


\twolineshloka
{ताराणामिव संपातो बभूव जनमेजय}
{भूषणानां च सर्वेषां कवचानां च सर्वशः}


\twolineshloka
{सवर्मभिर्भूषणैश्च प्रकीर्यद्बिरितस्ततः}
{सक्रोधामर्षजिह्मभ्रूकषायीकृतलोचनाः}


\twolineshloka
{सूतोपक्लृप्तान् रुचिरान्सदश्वैरुपकल्पितान्}
{रथानास्थाय ते वीराः सर्वप्रहरणान्विताः}


\threelineshloka
{प्रयान्तमथ कौरव्यमनुसस्रुरुदायुधाः}
{ततः समभवद्युद्धं तेषां तस्य च भारत}
{एकस्य च बहूनां च तुमुलं रोमहर्षणम्}


\twolineshloka
{ते त्विषून्दशसाहस्रांस्तस्मिन्युगपदाक्षिपन्}
{अप्राप्तांश्चैव तानाशु भीष्मः सर्वांस्तथाऽन्तरा}


\twolineshloka
{अच्छिनच्छरवर्षेण महता लोमवाहिना}
{ततस्ते पार्थिवाः सर्वे सर्वतः परिवार्य तम्}


\twolineshloka
{ववृषुः शरवर्षेण वर्षेणेवाद्रिमम्बुदाः}
{स तं बाणमयं वर्षं शरैरावार्य सर्वतः}


\twolineshloka
{ततः सर्वान्महीपालान्पर्यविध्यत्त्रिभिस्त्रिभिः}
{एकैकस्तु ततो भीष्मं राजन्विव्याध पञ्चभिः}


\twolineshloka
{स च तान्प्रतिविव्याध द्वाभ्यां द्वाभ्यां पराक्रमन्}
{तद्युद्धमासीत्तुमुलं घोरं देवासुरोपमम्}


\twolineshloka
{पश्यतां लोकवीराणां शरशक्तिसमाकुलम्}
{स धनूंषि ध्वजाग्राणि वर्माणि च शिरांसि}


\twolineshloka
{चिच्छेद समरे भीष्मः शतशोथ सहस्रशः}
{तस्यातिपुरुषं कर्म लाघवं रथचारिणः}


\twolineshloka
{रक्षणं चात्मनः संख्ये शत्रवोऽप्यभ्यपूजयन्}
{`अक्षतः क्षपयित्वान्यानसङ्ख्येयपराक्रमः}


\twolineshloka
{आनिनाय स काश्यस्य सुताः सागरगासुतः}
{'तान्विनिर्जित्य तु रणे सर्वशस्त्रभृतां वरः}


\twolineshloka
{कन्याभिः सहितः प्रायाद्भारतो भारतान्प्रति}
{ततस्तं पृष्ठतो राजञ्शाल्वराजो महारथः}


\twolineshloka
{अभ्यगच्छदमेयात्मा भीष्मं शान्तनवं रणे}
{वारणं जघने भिन्दन्दन्ताभ्यामपरो यथा}


\twolineshloka
{वासितामनुसंप्राप्तो यूथपो बलिनां वरः}
{स्त्रीकामस्तिष्ठतिष्ठेति भीष्ममाह स पार्थिवः}


\twolineshloka
{साल्वराजो महाबाहुरमर्षेण प्रचोदितः}
{ततः स पुरुषव्याघ्रो भीष्मः परबलार्दनः}


\twolineshloka
{तद्वाक्याकुलितः क्रोधाद्विधूमोग्निरिव ज्वलन्}
{विततेषुधनुष्पाणिर्विकुञ्चितललाटभृत्}


\twolineshloka
{क्षत्रधर्मं समास्थाय व्यपेतभयसंभ्रमः}
{निवर्तयामास रथं साल्वं प्रति महारथः}


\twolineshloka
{निवर्तमानं तं दृष्ट्वा राजानः सर्व एव ते}
{प्रेक्षकाः समपद्यन्त भीष्मसाल्वसमागमे}


\twolineshloka
{तौ वृषाविव नर्दन्तौ बलिनौ वासितान्तरे}
{अन्योन्यमभिवर्तेतां बलविक्रमशालिनौ}


\twolineshloka
{ततो भीष्मं शान्तनवं शरैः शतसहस्रशः}
{साल्वराजो नरश्रेष्ठः समवाकिरदाशुगैः}


\twolineshloka
{पूर्वमभ्यर्दितं दृष्ट्वा भीष्मं साल्वेन ते नृपाः}
{विस्मिताः समपद्यन्त साधुसाध्विति चाब्रुवन्}


\twolineshloka
{लाघवं तस्य ते दृष्ट्वा समरे सर्वपार्थिवाः}
{अपूजयन्त संहृष्टा वाग्भिः साल्वं नराधिपम्}


\twolineshloka
{क्षत्रियाणां ततो वाचः श्रुत्वा परपुञ्जयः}
{क्रुद्धः शान्तनवो भीष्मस्तिष्ठतिष्ठेत्यभाषत}


\twolineshloka
{सारथिं चाब्रवीत्क्रुद्धो याहि यत्रैष पार्थिवः}
{यावदेनं निहन्म्यद्य भुजङ्गमिव पक्षिराट्}


\twolineshloka
{ततोऽस्त्रं वारुणं सम्यग्योजयामास कौरवः}
{तेनाश्वांश्चतुरोऽमृद्गात्साल्वराजस्य भूपते}


\twolineshloka
{अस्त्रैरस्त्राणि संवार्य साल्वराजस्य कौरवः}
{भीष्मो नृपतिशार्दूल न्यवधीत्तस्य सारथिम्}


\twolineshloka
{अस्त्रेण चास्याथैन्द्रेण न्यवधीत्तुरगोत्तमान्}
{कन्याहेतोर्नरश्रेष्ठ भीष्मः शान्तनवस्तदा}


\twolineshloka
{जित्वा विसर्जयामास जीवन्तं नृपसत्तमम्}
{ततः साल्वः स्वनगरं प्रययौ भरतर्षभ}


\twolineshloka
{स्वराज्यमन्वशाच्चैव धर्मेण नृपतिस्तदा}
{राजानो ये च तत्रासन्स्वयंवरदिदृक्षवः}


\twolineshloka
{स्वान्येव तेऽपि राष्ट्राणि जग्मुः परपुरञ्जयाः}
{एवं विजित्य ताः कन्या भीष्मः प्रहरतां वरः}


\twolineshloka
{प्रययौ हास्तिनपुरं यत्र राजा स कौरवः}
{विचित्रवीर्यो धर्मात्मा प्रशास्ति वसुधामिमाम्}


\twolineshloka
{यथा पितास्य कौरव्यः शान्तनुर्नृपसत्तमः}
{सोऽचिरेणैव कालेन अत्यक्रामन्नराधिप}


\twolineshloka
{वनानि सरितश्चैव शैलांश्च विनिधान्द्रुमान्}
{अक्षतः क्षपयित्वाऽरीन्सङ्ख्येऽसङ्ख्येयविक्रमः}


\twolineshloka
{आनयामास काश्यस्य सुताः सागरगासुतः}
{स्नुषा इव स धर्मात्मा भगिनीरिव चानुजाः}


\twolineshloka
{यथा दुहितश्चैव परिगृह्य ययौ कुरून्}
{आनिन्ये स महाबाहुर्भ्रातुः प्रियचिकीर्षया}


\twolineshloka
{ताः सर्वगुणसंपन्ना भ्राता भ्रात्रे यवीयसे}
{भीष्मो विचित्रवीर्याय प्रददौ विक्रमाहृताः}


\twolineshloka
{एवं धर्मेण धर्मज्ञः कृत्वा कर्मातिमानुषम्}
{भ्रातुर्विचित्रवीर्यस्य विवाहायोपचक्रमे}


\threelineshloka
{सत्यवत्या सह मिथः कृत्वा निश्चयमात्मवान्}
{विवाहं कारयिष्यन्तं भीष्मं काशिपतेः सुता}
{ज्येष्ठा तासामिदं वाक्यमब्रवीद्धसती तदा}


\twolineshloka
{मया सौभपतिः पूर्वं मनसा हि वृतः पतिः}
{तेन चास्मि वृता पूर्वमेष कामश्च मे पितुः}


\twolineshloka
{मया वरयितव्योऽभूत्साल्वस्तस्मिन्स्वयंवरे}
{एतद्विज्ञाय धर्मज्ञ धर्मतत्त्वं समाचर}


\twolineshloka
{एवमुक्तस्तया भीष्मः कन्यया विप्रसंसदि}
{चिन्तामभ्यगमद्वीरो युक्तां तस्यैव कर्मणः}


\twolineshloka
{`अन्यसक्ता त्वियं कन्या ज्येष्ठा त्वम्बा मया जिता}
{वाचा दत्ता मनोदत्ता कृतमङ्गलवाचना}


\twolineshloka
{निर्दिष्टा तु परस्यैव सा त्याज्या परचिन्तिनी}
{इत्युक्त्वा चानुमान्यैव भ्रातरं स्ववशानुगम् ॥'}


\twolineshloka
{विनिश्चित्य स धर्मज्ञो ब्राह्मणैर्वेदपारगैः}
{अनुजज्ञे तदा ज्येष्ठामम्बां काशिपतेः सुताम्}


\twolineshloka
{अम्बिकाम्बालिके भार्ये प्रादाद्भ्रात्रे यवीयसे}
{भीष्मो विचित्रवीर्याय विधिदृष्टेन कर्मणा}


\twolineshloka
{तयोः पाणी गृहीत्वा तु रूपयौवनदर्पितः}
{विचित्रवीर्यो धर्मात्मा नाम्बामैच्छत्कथंचन}


% Check verse!
`अम्बामन्यस्य कीर्त्यन्तीमब्रवीच्चारुदर्शनाम्

विचित्रवीर्य उवाच

पापस्य फलमेवैष कामोऽसाधुर्निरर्थकः

परतन्त्रोपभोगो मामार्य नाऽऽयोक्तुमर्हसि ॥भीष्म उवाच


\twolineshloka
{प्रातिष्ठच्छान्तनोर्वंशस्तात यस्य त्वमन्वयः}
{अकामवृत्तो धर्मात्मन्साधु मन्ये मतं तव}


\twolineshloka
{इत्युक्त्वाम्बां समालोक्य विधिवद्वाक्यमब्रवीत्}
{विसृष्टा ह्यसि गच्छ त्वं यथाकाममनिन्दिते}


\twolineshloka
{नानियोज्ये समर्थोऽहं नियोक्तुं भ्रातरं प्रियम्}
{अन्यबावगतां चापि को नारीं वासयेद्गृहे}


\twolineshloka
{अतस्त्वां न नियोक्ष्यामि अन्यकामासि गम्यताम्}
{अहमप्यूर्ध्वरेता वै निवृत्तो दारकर्मणि}


\threelineshloka
{न संबन्धस्तदावाभ्यां भविता वै कथंचन}
{वैशंपायन उवाच}
{इत्युक्ता सा गता तत्र सखीभिः परिवारिता}


\twolineshloka
{निर्दिष्टा हि शनै राजन्साल्वराजपुरं प्रति}
{अथाम्बा साल्वंमागम्य साऽब्रवीत्प्रतिपूज्य तं}


\twolineshloka
{पुरा निर्दिष्टभावा त्वामागतास्मि वरानन}
{देवव्रतं समुत्सृज्य सानुजं भरतर्षभम्}


\twolineshloka
{प्रतिगृह्णीष्व भद्रं ते विधिवन्मां समुद्यताम् ॥वैशंपायन उवाच}
{}


\twolineshloka
{तयैवमुक्तः साल्वोपि प्रहसन्निदमब्रवीत्}
{निर्जिताऽसीह भीष्मेण मां विनिर्जित्य राजसु}


\threelineshloka
{अन्येन निर्जितां भद्रे विसृष्टां तेन चालयात्}
{न गृह्णामि वरारोहे तत्र चैव तु गम्यताम् ॥वैशंपायन उवाच}
{}


\twolineshloka
{इत्युक्ता सा समागम्य कुरुराज्यमनुत्तमम्}
{अम्बाब्रवीत्ततो भीष्मं त्वयाऽहं सहसा हृता}


\twolineshloka
{क्षत्रधर्ममवेक्षस्व त्वं भर्ता मम धर्मतः}
{यां यः स्वयंवरे कन्यां निर्जयेच्छौर्यसंपदा}


\twolineshloka
{राज्ञः सर्वान्विनिर्जित्य स तामुद्वाहयेद्ध्रुवम्}
{अतस्त्वमेव भर्ता मे त्वयाऽहं निर्जिता यतः}


\twolineshloka
{तस्माद्वहस्व मां भीष्म निर्जितां संसदि त्वया}
{ऊर्ध्वरेता ह्यहमिति प्रत्युवाच पुनःपुनः}


\twolineshloka
{भीष्मं सा चाब्रवीदम्बा यथाजैषीस्तथा कुरु}
{एवमन्वगमद्भीष्मं षट्समाः पुष्करेक्षणा}


\twolineshloka
{ऊर्ध्वरेतास्त्वहं भद्रे विवाहविमुखोऽभवम्}
{तमेव साल्वं गच्छ त्वं यः पुरा मनसा वृतः}


\twolineshloka
{अन्यसक्तं किमर्थं त्वमात्मानमवदः पुरा}
{अन्यसक्तां वधूं कन्यां वासयेत्स्वगृहे न हि}


\twolineshloka
{नाहमुद्वाहयिष्ये त्वां मम भ्रात्रे यवीयसे}
{विचित्रवीर्याय शुभे यथेष्टं गम्यतामिति}


\twolineshloka
{भूयः साल्वं समभ्येत्य राजन्गृह्णीष्व मामिति}
{नाहं गृह्णाम्यन्यजितामिति साल्वनिराकृता}


\twolineshloka
{ऊर्ध्वरेतास्त्वहमिति भीष्मेण च निराकृता}
{अम्बा भीष्मं पुनः साल्वं भीष्मं साल्वं पुनः पुनः}


\twolineshloka
{गमनागमनेनैवमनैषीत्षट् समा नृप}
{अश्रुभिर्भूमिमुक्षन्ती शोचन्ती सा मनस्विनी}


\twolineshloka
{पीनोन्नतकुचद्वन्द्वा विशालजघनेक्षणा}
{श्रोणीभरालसगमा राकाचन्द्रनिभानना}


\threelineshloka
{वर्षत्कादम्बिनीमूर्ध्नि स्फुरन्ती चञ्चलेव सा}
{सा ततो द्वादश समा बाहुदामभितो नदीम्}
{पार्श्वे हिमवतो रम्ये तपो घोरं समाददे}


\twolineshloka
{संक्षिप्तकरणा तत्र तप आस्थाय सुव्रता}
{पादाङ्गुष्ठेन साऽतिष्ठदकम्पन्त ततः सुराः}


\twolineshloka
{तस्यास्तत्तु तपो दृष्ट्वा सुराणां क्षोभकारकम्}
{विस्मितश्चैव हृष्टश्च तस्यानुग्रहबुद्धिमान्}


\twolineshloka
{अनन्तसेनो भगवान्कुमारो वरदः प्रभुः}
{मानयन्राजपुत्रीं तां ददौ तस्यै शुभां स्रजम्}


\twolineshloka
{एषा पुष्करिणी दिव्या यथावत्समुपस्थिता}
{अम्बे त्वच्छोकशमनी माला भुवि भविष्यति}


\twolineshloka
{एतां चैव मया दत्तां मालां यो धारयिष्यति}
{सोऽस्य भीष्मस्य निधने कारणं वै भविष्यति}


\chapter{अध्यायः ११०}
\twolineshloka
{अम्बोवाच}
{}


\twolineshloka
{अन्यपूर्वेति मां साल्वो नाभिनन्दति बालिशः}
{साहं धर्माच्च कामाच्च विहीना शोकधारिणी}


\twolineshloka
{अपतिः क्षत्रियान्सर्वानाक्रन्दामि समन्ततः}
{इयं वः क्षत्रिया माला या भीष्मं निहनिष्यति}


\twolineshloka
{अहं च भार्या तस्य स्यां यो भीष्मं घातयिष्यति}
{तस्याश्चङ्क्रम्यमाणायाः समाः पञ्च गताः पराः}


\twolineshloka
{नाभवच्छरणं कश्चित्क्षत्रियो भीष्मजाद्भयात्}
{अगच्छत्सोमकं साऽम्बा पाञ्चालेषु यशस्विनम्}


\twolineshloka
{सत्यसन्धं महेष्वासं सत्यधर्मपरायणम्}
{सा सभाद्वारमागम्य पाञ्चालैरभिरक्षितम्}


\twolineshloka
{पाञ्चालराजमाक्रन्दत्प्रगृह्य सुभुजा भुजौ}
{भीष्मेण हन्यमानां मां मज्जन्तीमिव च ह्रदे}


\twolineshloka
{यज्ञसेनाभिधावेह पाणिमालम्ब्य चोद्धर}
{तेन मे सर्वधर्माश्च रतिभोगाश्च केवलाः}


\twolineshloka
{उभौ च लोकौ कीर्तिश्च समूलौ सफलौ हृतौ}
{***न्त्येवं न विन्दामि राजन्यं शरणं क्वचित्}


\twolineshloka
{किं नु निःक्षत्रियो लोको यत्रानाथोऽवसीदति}
{समागम्य तु राजानो मयोक्ता राजसत्तमाः}


\twolineshloka
{शृण्वन्तु सर्वे राजानो मयोक्तं राजसत्तमाः}
{इक्ष्वाकूणां तु ये वृद्धाः पाञ्चालानां च ये वराः}


\threelineshloka
{त्वत्प्रसादाद्विवाहेऽस्मिन्मा धर्मो मा पराजयेत्}
{प्रसीद यज्ञसेनेह गतिर्मे भव सोमक ॥यज्ञसेन उवाच}
{}


\twolineshloka
{जानामि त्वां बोधयामि राजपुत्रि विशेषतः}
{यथाशक्ति यथाधर्मं बलं संधारयाम्यहम्}


\twolineshloka
{अन्यस्मात्पार्थिवाद्यत्ते भयं स्यात्पार्थिवात्मजे}
{तस्यापनयने हेतुं संविधातुमहं प्रभुः}


\twolineshloka
{नहि शान्तनवस्याहं महास्त्रस्य प्रहारिणः}
{ईश्वरः क्षत्रियाणां हि बलं धर्मोऽनुवर्तते}


\twolineshloka
{सा साधु व्रज कल्याणि न मां भीष्मो दहेद्बलात्}
{न प्रत्यगृह्णंस्ते सर्वे किमित्येव न वेद्म्यहम्}


\threelineshloka
{न हि भीष्मादहं धर्मं शक्तो दातुं कथंचन}
{वैशंपायन उवाच}
{इत्युक्ता स्रजमासज्य द्वारि राज्ञो व्यपाद्रवत्}


\twolineshloka
{व्युदस्तां सर्वलोकेषु तपसा संशितव्रताम्}
{तामन्वगच्छद्द्रुपदः सान्त्वं जल्पन्पुनः पुनः}


\twolineshloka
{स्रजं गृहाण कल्याणि न नो वैरं प्रसञ्जय ॥अम्बोवाच}
{}


\twolineshloka
{एवमेव त्वया कार्यमिति स्म प्रतिकाङ्क्षते}
{न तु तस्यान्यथा भावो दैवमेतदमानुषम्}


\threelineshloka
{यश्चैनां स्रजमादाय स्वयं वै प्रतिमोक्षते}
{स भीष्मं समरे हन्ता मम धर्मप्रणाशनम् ॥वैशंपायन उवाच}
{}


\twolineshloka
{तां स्रजं द्रुपदो राजा कंचित्कालं ररक्ष सः}
{ततो विस्रम्भमास्थाय तूष्णीमेतामुपैक्षत}


\twolineshloka
{तां शिखण्डिन्यबध्नात्तु बाला पितुरवज्ञया}
{तां पिता त्वत्यजच्छीघ्रं त्रस्तो भीष्मस्य किल्बिषात्}


\twolineshloka
{इषीकं ब्राह्मणं भीता साभ्यगच्छत्तपस्विनम्}
{गङ्गाद्वारि तपस्यन्तं तुष्टिहेतोस्तपस्विनी}


\twolineshloka
{उपचाराभितुष्टस्तामब्रवीदृषिसत्तमः}
{गङ्गाद्वारे विभजनं भविता नचिरादिव}


\twolineshloka
{तत्र गन्धर्वराजानं तुम्बुरुं प्रियदर्शनम्}
{आराधयितुमीहस्व सम्यक्परिचरस्व तम्}


\twolineshloka
{अहमप्यत्र साचिव्यं कर्तास्मि तव शोभने}
{तं तदाचर भद्रं ते स ते श्रेयो विधास्यति}


\twolineshloka
{ततो विभजनं तत्र गन्धर्वाणामवर्तत}
{तत्र द्वाववशिष्येतां गन्धर्वावमितौ जसौ}


\twolineshloka
{तयोरेकः समीक्ष्यैनां स्त्रीबुभूषुरुवाच ह}
{इदं गृह्णीष्व पुंलिङ्गं वृणे स्त्रीलिङ्गमेव ते}


\twolineshloka
{नियमं चक्रतुस्तत्र स्त्री पुमांश्चैव तावुभौ}
{ततः पुमान्समभवच्छिखण्डी परवीरहा}


\twolineshloka
{स्त्री भूत्वा ह्यपचक्राम स गन्धर्वो मुदान्वितः}
{लब्ध्वा तु महतीं प्रीतिं याज्ञसेनिर्महायशाः}


\twolineshloka
{ततो बुद्बुदकं गत्वा पुनरस्त्राणि सोऽकरोत्}
{तत्र चास्त्राणि दिव्यानि कृत्वा स सुमहाद्युतिः}


\twolineshloka
{स्वदेशमभिसंपदे पाञ्चालं कुरुनन्दन}
{सोऽभिवाद्य पितुः पादौ महेष्वासः कृताञ्जलिः}


% Check verse!
उवाच भवता भीष्मान्न भेतव्यं कथंचन
\chapter{अध्यायः १११}
\twolineshloka
{वैशंपायन उवाच}
{}


\twolineshloka
{अम्बायां निर्गतायां तु भीष्मः शान्तनवस्तदा}
{न्यायेन कारयामास राज्ञो वैवाहिकीं क्रियाम्}


\threelineshloka
{अम्बिकाम्बालिके चैव परिणीयाग्निसंनिधौ}
{`तयोः पाणी गृहीत्वा तु कौरव्यो रूपदर्पितः}
{'विचित्रवीर्यो धर्मात्मा कामात्मा समपद्यत}


\twolineshloka
{ते चापि बृहतीश्यामे नीलकुञ्चितमूर्धजे}
{रक्ततुङ्गनखोपेते पीनश्रोणिपयोधऱे}


\twolineshloka
{आत्मनः प्रतिरूपोऽसौ लब्धः पतिरिति स्थिते}
{विचित्रवीर्यं कल्याण्यौ पूजयामासतुः शुभे}


\threelineshloka
{`अन्योन्यं प्रति सक्ते च एकभावे इव स्थिते}
{'स चाश्विरूपसदृशो देवतुल्यपराक्रमः}
{सर्वासामेव नारीणां चित्तप्रमथनो रहः}


\twolineshloka
{ताभ्यां सह समाः सप्त विहरन्पृथिवीपतिः}
{विचित्रवीर्यस्तरुणो यक्ष्मणा समगृह्यत}


\twolineshloka
{सुहृदां यतमानानामाप्तैः सह चिकित्सकैः}
{जगामास्तमिवादित्यः कौरव्यो यमसादनम्}


\twolineshloka
{धर्मात्मा स तु गाङ्गेयः चिन्ताशोकपरायणः}
{प्रेतकार्याणि सर्वाणि तस्य सम्यगकारयत्}


\twolineshloka
{राज्ञो विचित्रवीर्यस्य सत्यवत्या मते स्थितः}
{ऋत्विग्बिः सहितो भीष्मः सर्वैश्च कुरुपुङ्गवैः}


\chapter{अध्यायः ११२}
\twolineshloka
{वैशंपायन उवाच}
{}


\twolineshloka
{ततः सत्यवती दीना कृपणा पुत्रगृद्धिनी}
{पुत्रस्य कृत्वा कार्याणि स्नुषाभ्यां सह भारत}


\threelineshloka
{समाश्वास्य स्नुषे ते च भर्तृशोकनिपीडिते}
{धर्मं च पितृवंशं च मातृवंशं च भामिनी}
{प्रसमीक्ष्य महाभागा गाङ्गेयं वाक्यमब्रवीत्}


\threelineshloka
{`दुःखार्दिता तु सा देवी मज्जन्ती शोकसागरे}
{शन्तनोर्धर्मनित्यस्य कौरव्यस्य यशस्विनः}
{'त्वयि पिण्डश्च कीर्तिश्च संतानश्च प्रतिष्ठितः}


\threelineshloka
{`भ्राता विचित्रवीर्यस्ते भूतानामन्तमेयिवान्}
{'यथाकर्म शुभं कृत्वा स्वर्गोपगमनं ध्रुवम्}
{यथा चायुर्ध्रुवं सत्ये त्वयि धर्मस्तथा ध्रुवः}


\twolineshloka
{वेत्थ धर्मांश्च धर्मज्ञ समासेनेतरेण च}
{विविधास्त्वं श्रुतीर्वेत्थ वेदाङ्गानि च सर्वशः}


\twolineshloka
{व्यवस्थानं च ते धर्मे कुलाचारं च लक्षये}
{प्रतिपत्तिं च कृच्छ्रेषु शुक्राङ्गिरसयोरिव}


\twolineshloka
{तस्मात्सुभृशमाश्वस्य त्वयि धर्मभृतां वर}
{कार्ये त्वां विनियोक्ष्यामि तच्छ्रुत्वा कर्तुमर्हसि}


\twolineshloka
{मम पुत्रस्तव भ्राता वीर्यवान्सुप्रियश्च ते}
{बाल एव गतः स्वर्गमपुत्रः पुरुषर्षभ}


\twolineshloka
{इमे महिष्यौ भ्रातुस्ते काशिराजसुते शुभे}
{रूपयौवनसंपन्ने पुत्रकामे च भारत}


\twolineshloka
{तयोरुत्पादयापत्यं सन्तानाय कुलस्य नः}
{मन्नियोगान्महाबाहो धर्मं कर्तुमिहार्हसि}


\threelineshloka
{राज्ये चैवाभिषिच्यस्व भारताननुशाधि च}
{दारांश्च कुरु धर्मेण मा निमज्जीः पितामहान् ॥वैशंपायन उवाच}
{}


\twolineshloka
{तथोच्यमानो मात्रा स सुहृद्भिश्च परन्तपः}
{इत्युवाचाथ धर्मात्मा धर्म्यमेवोत्तरं वचः}


\twolineshloka
{असंशयं परो धर्मस्त्वया मातरुदाहृतः}
{त्वमपत्यं प्रति च मे प्रतिज्ञां वेत्थ वै परां}


\twolineshloka
{जानासि च यथावृत्तं शुल्कहेतोस्त्वदन्तरे}
{स सत्यवति सत्यं ते प्रतिजानाम्यहं पुनः}


\twolineshloka
{परित्यजेयं त्रैलोक्यं राज्यं देवेषु वा पुनः}
{यद्वाऽप्यधिकमेताभ्यां न तु सत्यं कथंचन}


\twolineshloka
{त्यजेच्च पृथिवी गन्धमापश्च रसमात्मनः}
{ज्योतिस्तथा त्यजेद्रूपं वायुः स्पर्शगुणं त्यजेत्}


\twolineshloka
{प्रभां समुत्सृजेदर्को धूमकेतुस्तथोष्मताम्}
{त्यजेच्छब्दं तथाऽऽकाशं सोमः शीतांशुतां त्यजेत्}


\twolineshloka
{विक्रमं वृत्रहा जह्याद्धर्मं जह्याच्च धर्मराट्}
{न त्वहं सत्यमुत्स्रष्टुं व्यवसेयं कथंचन}


\twolineshloka
{`तन्न जात्वन्यथा कुर्यां लोकानामपि संक्षये}
{अमरत्वस्य वा हेतोस्त्रैलोक्यस्य धनस्य च ॥'}


\twolineshloka
{एवमुक्ता तु पुत्रेण भूरिद्रविणतेजसा}
{माता सत्यवती भीष्ममुवाच तदनन्तरम्}


\twolineshloka
{जानामि ते स्थितिं सत्ये परां सत्यपराक्रम}
{इच्छन्सृजेथास्त्रींल्लोकानन्यांस्त्वं स्वेन तेजसा}


\twolineshloka
{जानामि चैवं सत्यं तन्मदर्थे यच्च भाषितम्}
{आपद्धर्मं त्वमावेक्ष्य वह पैतामहीं धुरम्}


\twolineshloka
{यथा ते कुलतन्तुश्च धर्मश्च न पराभवेत्}
{सुहृदश्च प्रहृष्येरंस्तथा कुरु परन्तप}


\threelineshloka
{`आत्मनश्च हितं तात प्रियं च मम भारत}
{'लालप्यमानां तामेवं कृपणां पुत्रगृद्धिनीम्}
{धर्मादपेतं ब्रुवतीं भीष्मो भूयोऽब्रवीदिदम्}


\twolineshloka
{राज्ञि धर्मानवेक्षस्व मा नः सर्वान्व्यनीनशः}
{सत्याच्च्युतिः क्षत्रियस्य न धर्मेषु प्रशस्यते}


\twolineshloka
{शान्तनोरपि संतानं यथा स्यादक्षयं भुवि}
{तत्ते धर्मं प्रवक्ष्यामि क्षात्रं राज्ञि सनातनम्}


\twolineshloka
{श्रुत्वा तं प्रतिपद्यस्व प्राज्ञैः सह पुरोहितैः}
{आपद्धर्मार्थकुशलैर्लोकतन्त्रमवेक्ष्य च}


\chapter{अध्यायः ११३}
\twolineshloka
{भीष्म उवाच}
{}


\twolineshloka
{जामदग्न्येन रामेण पितुर्वधममृष्यता}
{राजा परशुना पूर्वं हैहयाधिपतिर्हतः}


\twolineshloka
{शतानि दशबाहूनां निकृत्तान्यर्जुनस्य वै}
{लोकस्याचरितो धर्मस्तेनाति किल दुश्चरः}


\twolineshloka
{पुनश्च धनुरादाय महास्त्राणि प्रमुञ्चता}
{निर्दग्धं क्षत्रमसकृद्रथेन जयता महीम्}


\twolineshloka
{एवमुच्चावचैरस्त्रैर्भार्गवेण महात्मना}
{त्रिःसप्तकृत्वः पृथिवी कृता निःक्षत्रिया पुरा}


\twolineshloka
{एवं निःक्षत्रिये लोके कृते तेन महर्षिणा}
{ततः संभूय सर्वाभिः क्षत्रियाभिः समन्ततः}


\twolineshloka
{उत्पादितान्यपत्यानि ब्राह्मणैर्वेदपारगैः}
{पाणिग्राहस्य तनय इति वेदेषु निश्चितम्}


\twolineshloka
{धर्मं मनसि संस्थाप्य ब्राह्मणांस्ताः समभ्ययुः}
{लोकेऽप्याचरितो दृष्टः क्षत्रियाणां पुनर्भवः}


\twolineshloka
{ततः पुनः समुदितं क्षत्रं समभवत्तदा}
{इमं चैवात्र वक्ष्येऽहमितिहासं पुरातनम्}


\twolineshloka
{अथोचथ्य इति ख्यात आसीद्धीमानृषिः पुरा}
{ममता नाम तस्यासीद्भार्या परमसंमता}


\twolineshloka
{उचथ्यस्य यवीयांस्तु पुरोधास्त्रिदिवौकसाम्}
{बृहस्पतिर्बृहत्तेजा ममतामन्वपद्यत}


\twolineshloka
{उवाच ममता तं तु देवरं वदतां वरम्}
{अन्तर्वत्नी त्वहं भ्रात्रा ज्येष्ठेनारम्यतामिति}


\twolineshloka
{अयं च मे महाभाग कुक्षावेव बृहस्पते}
{औचथ्यो देवमत्रापि षडङ्गं प्रत्यधीयत}


\threelineshloka
{अमोघरेतास्त्वं चापि द्वयोर्नास्त्यत्र संभवः}
{तस्मादेवं गते त्वद्य उपारमितुमर्हसि ॥वैशंपायन उवाच}
{}


\twolineshloka
{एवमुक्तस्तदा सम्यक् बृहस्पतिरुदारधीः}
{कामात्मानं तदात्मानं न शशाक नियच्छितुम्}


\twolineshloka
{स बभूव ततः कामी तया सार्धमकामया}
{उत्सृजन्तं तु तं रेतः स गर्भस्थोऽभ्यभाषत}


\twolineshloka
{भोस्तात मा गमः कामं द्वयोर्नास्तीह संभवः}
{अल्पावकाशो भगवन्पूर्वं चाहमिहागतः}


\twolineshloka
{अमोघरेताश्च भवान्न पीडां कर्तुमर्हति}
{अश्रुत्वैव तु तद्वाक्यं गर्भस्थस्य बृहस्पतिः}


\twolineshloka
{जगाम मैथुनायैव ममतां चारुलोचनाम्}
{शुक्रोत्सर्गं ततो बुद्ध्वा तस्या गर्भगतो मुनिः}


\twolineshloka
{पद्भ्यामारोधयन्मार्गं शुक्रस्य च बृहस्पतेः}
{स्थानमप्राप्तमथ तच्छुक्रं प्रतिहतं तदा}


\twolineshloka
{पपात सहसा भूमौ ततः क्रुद्धो बृहस्पतिः}
{तं दृष्ट्वा पतितं शुक्रं शशाप स रुषान्वितः}


\twolineshloka
{उचथ्यपुत्रं गर्भस्थं निर्भर्त्स्य भगवानृषिः}
{यन्मां त्वमीदृशे काले सर्वभूतेप्सिते सति}


\twolineshloka
{एवमात्थ वचस्तस्मात्तमो दीर्घं प्रवेक्ष्यसि}
{स वै दीर्घतमा नाम शापादृषिरजायत}


\twolineshloka
{बृहस्पतेर्बृहत्कीर्तेर्बृहस्पतिरिवौजसा}
{जात्यन्धो वेदवित्प्राज्ञः पत्नीं लेभे स विद्यया}


\twolineshloka
{तरुणीं रुपसंपन्नां प्रद्वेषीं नाम ब्राह्मणीम्}
{स पुत्राञ्जनयामास गौतमादीन्महायशाः}


\twolineshloka
{ऋषेरुचथ्यस्य तदा सन्तानकुलवृद्धये}
{धर्मात्मा च महात्मा च वेदवेदाङ्गपारगः}


\twolineshloka
{गोधर्मं सौरभेयाच्च सोऽधीत्य निखिलं मुनिः}
{प्रावर्तत तदा कर्तुं श्रद्धावांस्तमशङ्कया}


\twolineshloka
{ततो वितथमर्यादं तं दृष्ट्वा मुनिसत्तमाः}
{क्रुद्धा मोहाभिभूतास्ते सर्वे तत्राश्रमौकसः}


\twolineshloka
{अहोऽयं भिन्नमर्यादो नाश्रमे वस्तुमर्हति}
{तस्मादेनं वयं सर्वे पापात्मानं त्यजामहे}


\twolineshloka
{इत्यन्योन्यं समाभाष्य ते दीर्घतमसं मुनिम्}
{पुत्रलाभाच्च सा पत्नी न तुतोष पतिं तदा}


\threelineshloka
{प्रद्विषन्तीं पतिर्भार्यां किं मां द्वेक्षीति चाब्रवीत्}
{प्रद्वेष्युवाच}
{भार्याया भरणाद्भर्ता पालनाच्च पतिः स्मृतः}


\threelineshloka
{अहं त्वद्भरणाशक्ता जात्यन्धं ससुतं तदा}
{नित्यकालं श्रमेणार्ता न भरेयं महातपः ॥भीष्म उवाच}
{}


\twolineshloka
{तस्मास्तद्वचनं श्रुत्वा ऋषिः कोपसमन्वितः}
{प्रत्युवाच ततः पत्नीं प्रद्वेषीं ससुतां तदा}


\threelineshloka
{नीयतां क्षत्रियकुले धनार्थश्च भविष्यति}
{प्रद्वेष्युवाच}
{त्वया दत्तं धनं विप्र नेच्छेयं दुःखकारणम्}


\threelineshloka
{यथेष्टं कुरु विप्रेन्द्र न भेरयं पुरा यथा}
{दीर्घतमा उवाच}
{अद्यप्रभृति मर्यादा मया लोके प्रतिष्ठिता}


\twolineshloka
{एक एव पतिर्नार्या यावज्जीवं परायणम्}
{मृते जीवति वा तस्मिन्नापरं प्राप्नुयान्नरम्}


\twolineshloka
{अभिगम्य परं नारी पतिष्यति न संशयः}
{अपतीनां तु नारीणामद्यप्रभृति पातकम्}


\twolineshloka
{यद्यस्ति चेद्धनं सर्वं वृथाभोगा भवन्तु ताः}
{अकीर्तिः परिवादाश्च नित्यं तासां भवन्तु वै}


\twolineshloka
{इति तद्वचनं श्रुत्वा ब्राह्मणी भृशकोपिता}
{गङ्गायां नीयतामेष पुत्रा इत्येवमब्रवीत्}


\twolineshloka
{लोभमोहाभिभूतास्ते पुत्रास्तं गौतमादयः}
{वद्ध्वोडुपे परिक्षिप्य गङ्गायां समवासृजन्}


\twolineshloka
{कस्मादन्धश्च वृद्धश्च भर्तव्योऽयमिति स्म ते}
{चिन्तयित्वा ततः क्रूराः प्रतिजग्मुरथो गृहान्}


\twolineshloka
{सोऽनुस्रोतस्तदा विप्रः प्लवमानो यदृच्छया}
{जगाम सुबहून्देशानन्धस्तेनोडुपेन ह}


\twolineshloka
{तं तु राजा बलिर्नाम सर्वधर्मविदां वरः}
{अपश्यन्मज्जनगतः स्रोतसाऽभ्याशमागतम्}


\twolineshloka
{जग्राह चैनं धर्मात्मा बलिः सत्यपराक्रमः}
{ज्ञात्वा चैवं स वव्रेऽथ पुत्रार्थे भरतर्षभ}


\threelineshloka
{`तं पूजयित्वा राजर्षिर्विश्रान्तं मुनिमब्रवीत्}
{'सन्तानार्थं महाभाग भार्यासु मम मानद}
{पुत्रान्धर्मार्थकुशलानुत्पादयितुमर्हसि}


\threelineshloka
{भीष्म उवाच}
{एवमुक्तः स तेजस्वी तं तथेत्युक्तवानृषिः}
{तस्मैस राजा स्वां भार्यां सुदेष्णां प्राहिणोत्तदा}


\twolineshloka
{अन्धं वृद्धं च तं मत्वा न सा देवी जगाम ह}
{स्वां तु धात्रेयिकां तस्मै वृद्धाय प्राहिणोत्तदा}


\twolineshloka
{तस्यां काक्षीवदादीन्स शूद्रयोनावृषिस्तदा}
{जनयामास धर्मात्मा पुत्रानेकादशैव तु}


\twolineshloka
{काक्षीवदादीन्पुत्रांस्तान्दृष्ट्वा सर्वानधीयतः}
{उवाच तमृषिं राजा ममेम इति भारत}


\twolineshloka
{नेत्युवाच महर्षिस्तं ममेम इति चाब्रवीत्}
{शूद्रयोनौ मया हीमे जाताः काक्षीवदादयः}


\threelineshloka
{अन्धं वृद्धं च मां दृष्ट्वा सुदेष्णा महिषी तव}
{अवमन्य ददौ मूढा शूद्रां धात्रेयिकां मम ॥भीष्म उवाच}
{}


\twolineshloka
{ततः प्रसादयामास पुनस्तमृषिसत्तमम्}
{बलिः सुदेष्णां स्वां भार्यां तस्मै स प्राहिणोत्पुनः}


\twolineshloka
{तां स दीर्घतमाङ्गेषु स्पृष्ट्वा देवीमथाब्रवीत्}
{भविष्यन्ति कुमारास्ते तेजसाऽऽदित्यवर्चसः}


\twolineshloka
{अङ्गो वङ्गः कलिङ्गश्च पुण्ड्रः सुह्मश्च ते सुताः}
{तेषां देशाः समाख्याताः स्वनामकथिता भुवि}


\twolineshloka
{अङ्गस्याङ्गोऽभवद्देशो वङ्गो वङ्गस्य च स्मृतः}
{कलिङ्गविषयश्चैव कलिङ्गस्य च स स्मृतः}


\twolineshloka
{पुण्ड्रस्य पुण्ड्राः प्रख्याताः सुह्माः सुह्मस्य च स्मृताः}
{एवं बलेः पुरा वंशः प्रख्यातो वै महर्षिजः}


\threelineshloka
{एवमन्ये महेष्वासा ब्राह्मणैः क्षत्रिया भुवि}
{जाताः परमधर्मज्ञा वीर्यवन्तो महाबलाः}
{एतच्छ्रुत्वा त्वमप्यत्र मातः कुरु यथेप्सितम्}


\chapter{अध्यायः ११४}
\twolineshloka
{भीष्म उवाच}
{}


\twolineshloka
{पुनर्भरतवंशस्य हेतुं सन्तानवृद्धये}
{वक्ष्यामि नियतं मातस्तन्मे निगदतः शृणु}


\threelineshloka
{ब्राह्मणो गुणवान्कश्चिद्धनेनोपनिमन्त्र्यताम्}
{विचित्रवीर्यक्षेत्रेषु यः समुत्पादयेत्प्रजाः ॥`वैशंपायन उवाच}
{}


\twolineshloka
{भीष्मस्य तु वचः श्रुत्वा धर्महेत्वर्थसंहितम्}
{माता सत्यवती भीष्मं पुनरेवाभ्यभाषत}


\twolineshloka
{औचथ्यमधिकृत्येदमङ्गं च यदुदाहृतम्}
{पौराणी श्रुतिरित्येषा प्राप्तकालमिदं कुरु}


\twolineshloka
{त्वं हि पुत्र कुलस्यास्य ज्येष्ठः श्रेष्ठश्च भारत}
{यथा च ते पितुर्वाक्यं मम कार्यं तथाऽनघ}


\twolineshloka
{मम पुत्रस्तव भ्राता यवीयान्सुप्रियश्च ते}
{बाल एव गतः स्वर्गं भारतो भरतर्षभ}


\twolineshloka
{इमे महिष्यौ तस्येह काशिराजसुते उभे}
{रूपयौवनसंपन्ने पुत्रकामे च भारत}


\threelineshloka
{धर्म्यमेतत्परं ज्ञात्वा सन्तानाय कुलस्य च}
{आभ्यां मम नियोगात्तु धर्मं चरितुमर्हसि ॥भीष्म उवाच}
{}


\twolineshloka
{असंशयं परो धर्मस्त्वयाः मातः प्रकीर्तितः}
{त्वमप्येतां प्रतिज्ञां तु वेत्थ या मयि वर्तते}


\threelineshloka
{अमरत्वस्य वा हेतोस्त्रैलोक्यसदनस्य वा}
{उत्सृजेयमहं प्राणान्न तु सत्यं कथंचन ॥सत्यवत्युवाच}
{}


\twolineshloka
{जानामि त्वयि धर्मज्ञ सत्यं सत्यपराक्रम}
{इच्छंस्त्वमिह लोकांस्त्रीन्सृजेरन्यानरिन्दम}


\threelineshloka
{यथा तु नः कुलं चैव धर्मश्च न पराभवेत्}
{सुहृदश्च प्रहृष्टाः स्युस्तथा त्वं कर्तुमर्हसि ॥भीष्म उवाच}
{}


\twolineshloka
{त्वमेव कुलवृद्धासि गौरवं तु परं त्वयि}
{सोपायं कुलसन्ताने वक्तुमर्हसि नः परम्}


\twolineshloka
{स्त्रियो हि परमं गुह्यं धारयन्ति सदा कुले}
{पुरुषांश्चैव मायाभिर्बह्वीभिरुपगृह्णते}


\threelineshloka
{सा सत्यवति संपश्य धर्मं सत्यपरायणे}
{यथा न जह्यां सत्यं च न सीदेच्च कुलं हि नः ॥'वैशंपायन उवाच}
{}


\twolineshloka
{ततः सत्यवती भीष्मं वाचा संसज्जमानया}
{विहसन्तीव सव्रीडमिदं वचनमब्रवीत्}


\twolineshloka
{सत्यमेतन्महाबाहो यथा वदसि भारत}
{विश्वासात्ते प्रवक्ष्यामि सन्तानाय कुलस्य नः}


\twolineshloka
{न ते शक्यमनाख्यातुमापद्धर्मं तथाविधम्}
{त्वमेव नः कुले धर्मस्त्वं सत्यं त्वं परा गतिः}


\threelineshloka
{`यत्त्वं वक्ष्यसि तत्कार्यमस्माभिरिति मे मतिः}
{'तस्मान्निशम्य सत्यं मे कुरुष्व यदनन्तरम्}
{`शृणु भीष्म वचो मह्यं धर्मार्थसहितं हितम्}


\twolineshloka
{न च विस्रम्भकथितं भवान्सूचितुमर्हति}
{यस्तु राजा वसुर्नाम श्रुतस्ते भरतर्षभ}


\twolineshloka
{तस्य शुक्लादहं मत्स्या धृता कुक्षौ पुरा किल}
{मातरं मे जलाद्धृत्वा दाशः परमधर्मवित्}


\twolineshloka
{मां तु स्वगृहमानीय दुहितृत्वेऽभ्यकल्पयत्}
{धर्मयुक्तः स धर्मेण पिता चासीत्ततो मम ॥'}


\twolineshloka
{धर्मयुक्तस्य धर्मार्थं पितुरासीत्तरी मम}
{सा कदाचिदहं तत्र गता प्रथमयौवनम्}


\twolineshloka
{अथ धर्मविदां श्रेष्ठः परमर्षिः पराशरः}
{आजगाम तरीं धीमांस्तरिष्यन्यमुनां नदीम्}


\threelineshloka
{स तार्यमाणो यमुनां मामुपेत्याब्रवीत्तदा}
{सान्त्वपूर्वं मुनिश्रेष्ठः कामार्तो मधुरं वचः}
{उक्त्वा जन्म कुलं मह्यं नासि दाशसुतेति च}


\twolineshloka
{तमहं शापभीता च पितुर्भीता च भारत}
{वरैरसुलभैरुक्ता न प्रत्याख्यातुमुत्सहे}


\twolineshloka
{`प्रेक्ष्य तांस्तु महाभागान्पारावारे ऋषीन्स्थितान्}
{यमुनातीरविन्यस्तान्प्रदीप्तानिव पावकान्}


\twolineshloka
{पुरस्तादरुणश्चैव तरुणः संप्रकाशते}
{येनैषा ताम्रवस्त्रेव द्यौः कृता प्रविजृम्भिता}


\twolineshloka
{उक्तमात्रो मया तत्र नीहारमसृजत्प्रभुः}
{पराशरः सत्यधृतिर्द्वीपे च यमुनाम्भसि ॥'}


\twolineshloka
{अभिभूय स मां बालां तेजसा वशमानयत्}
{तमसा लोकमावृत्य नौगतामेव भारत}


\twolineshloka
{मत्स्यगन्धो महानासीत्पुरा मम जुगुप्सितः}
{तमपास्य शुभं गन्धमिमं प्रादात्स मे मुनिः}


\twolineshloka
{ततो मामाह स मुनिर्गर्भमुत्सृज्य मामकम्}
{द्वोपेऽस्या एव सरितः कन्यैव त्वं भविष्यसि}


\twolineshloka
{कन्यात्वं च ददौ प्रीतः पुनर्विद्वांस्तपोधनः}
{तस्य वीर्यमहं दृष्ट्वा तथा युक्तं महात्मनः}


\threelineshloka
{विस्मिता व्यथिता चैव प्रादामात्मानमेव च}
{ततस्तदा महात्मा स कन्यायां मयि भारत}
{प्रहृष्टोऽजनयत्पुत्रं द्वीप एव पराशरः ॥'}


\twolineshloka
{पाराशर्यो महायोगी स बभूव महानृषिः}
{कन्यापुत्रो मम पुरा द्वैपायन इति श्रुतः}


\twolineshloka
{यो व्यस्य वेदांश्चतुरस्तपसा भगवानृषिः}
{लोके व्यासत्वमापेदे कार्ष्ण्यात्कृष्णत्वमेव च}


\twolineshloka
{सत्यवादी शमपरस्तपस्वी दग्धकिल्बिषः}
{सद्योत्पन्नः स तु महान्सह पित्रा ततो गतः}


\twolineshloka
{स नियुक्तो मया व्यक्तं त्वया चाप्रतिमद्युतिः}
{भ्रातुः क्षेत्रेषु कल्याणमपत्यं जनयिष्यति}


\twolineshloka
{स हि मामुक्तवांस्तत्र स्मरेः कृच्छ्रेषु मामिति}
{तं स्मरिष्ये महाबाहो यदि भीष्म त्वमिच्छसि}


\twolineshloka
{तव ह्यनुमते भीष्म नियतं स महातपाः}
{विचित्रवीर्यक्षेत्रेषु पुत्रानुत्पादयिष्यति}


\fourlineindentedshloka
{वैशंपायन उवाच}
{महर्षेः कीर्तने तस्य भीष्मः प्राञ्जलिरब्रवीत्}
{`देशकालौ च जानासि क्रियतामर्थसिद्धये}
{'धर्ममर्थं च कामं च त्रीनेतान्योनुपश्यति}


\twolineshloka
{अर्थमर्थानुबन्धं च धर्मं धर्मानुबन्धनम्}
{कामं कामानुबन्धं च विपरीतान्पृथक्पृथक्}


\twolineshloka
{यो विचिन्त्य धिया धीरो व्यवस्यति स बुद्धिमान्}
{तदिदं धर्मयुक्तं च हितं चैव कुलस्य नः}


\threelineshloka
{उक्तं भवत्या यच्छ्रेयस्तन्मह्यं रोचते भृशम्}
{वैशंपायन उवाच}
{ततस्तस्मिन्प्रतिज्ञाते भीष्मेण कुरुनन्दन}


\twolineshloka
{कृष्णद्वैपायनं काली चिन्तयामास वै मुनिम्}
{स वेदान्विब्रुवन्धीमान्मातुर्विज्ञाय चिन्तितम्}


\twolineshloka
{प्रादुर्बभूवाविदितः क्षणेन कुरुनन्दन}
{तस्मै पूजां ततः कृत्वा सुताय विधिपूर्वकम्}


\twolineshloka
{परिष्वज्य च बाहुभ्यां प्रस्रवैरभ्यषिञ्चत}
{मुमोच बाष्पं दाशेयी पुत्रं दृष्ट्वा चिरस्य तु}


\twolineshloka
{तामद्भिः परिषिच्यार्तां महर्षिरभिवाद्य च}
{मातरं पूर्वजः पुत्रो व्यासो वचनमब्रवीत्}


\twolineshloka
{भवत्या यदभिप्रेतं तदहं कर्तुमागतः}
{शाधि मां धर्मतत्त्वज्ञे करवाणि प्रियं तव}


\twolineshloka
{तस्मै पूजां ततोऽकार्षीत्पुरोधाः परमर्षये}
{स च तां प्रतिजग्राह विधिमन्मन्त्रपूर्वकम्}


\twolineshloka
{पूजितो मन्त्रपूर्वं तु विधिवत्प्रीतिमाप सः}
{तमासनगतं माता पृष्ट्वा कुशलमव्ययम्}


\twolineshloka
{सत्यवत्यथ वीक्ष्यैनमुवाचेदमनन्तरम्}
{मातापित्रोः प्रजायन्ते पुत्राः साधारणाः कवे}


\twolineshloka
{तेषां पिता यथा स्वीमी तथा माता न संशयः}
{विधानविहितः स त्वं यथा मे प्रथमः सुतः}


\twolineshloka
{विचित्रवीर्यो ब्रह्मर्षे तथा मेऽवरजः सुतः}
{यथैव पितृतो भीष्मस्तथा त्वमपि मातृतः}


\twolineshloka
{भ्राता विचित्रवीर्यस्य यथा वा पुत्र मन्यसे}
{अयं शान्तनवः सत्यं पालयन्सत्यविक्रमः}


\twolineshloka
{बुद्धिं न कुरुतेऽपत्ये तथा राज्याऽनुशासने}
{स त्वं व्यपेक्षया भ्रातुः सन्तानाय कुलस्य च}


\twolineshloka
{भीष्मस्य चास्य वचनान्नियोगाच्च ममानघ}
{अनुक्रोशाच्च भूतानां सर्वेषां रक्षणाय च}


\twolineshloka
{आनृशंस्याच्च यद्ब्रूयां तच्छ्रुत्वा कर्तुमर्हसि}
{यवीयसस्व भ्रातुर्भार्ये सुरसुतोपमे}


\twolineshloka
{रूपयौवनसंपन्ने पुत्रकामे च धर्मतः}
{तयोरुत्पादयापत्यं समर्थो ह्यसि पुत्रक}


\threelineshloka
{अनुरूपं कुलस्यास्य संतत्याः प्रसवस्य च}
{व्यास उवाच}
{वेत्थ धर्मं सत्यवति परं चापरमेव च}


\twolineshloka
{तथा तव महाप्राज्ञे धर्मे प्रणिहिता मतिः}
{तस्मादहं त्वन्नियोगाद्धर्ममुद्दिश्य कारणम्}


\twolineshloka
{ईप्सितं ते करिष्यामि दृष्टं ह्येतत्सनातनम्}
{भ्रातुः पुत्रान्प्रदास्यामि मित्रावरुणयोः समान्}


\twolineshloka
{व्रतं चरेतां ते देव्यौ निर्दिष्टमिह यन्मया}
{संवत्सरं यथान्यायं ततः शुद्धे भविष्यतः}


\threelineshloka
{नहि मामव्रतोपेता उपेयात्काचिदङ्गना}
{सत्यवत्युवाच}
{सद्यो यथा प्रपद्येते देव्यौ गर्भं तथा कुरु}


\twolineshloka
{अराजकेषु राष्ट्रेषु प्रजाऽनाथा विनश्यति}
{नश्यन्ति च क्रियाः सर्वा नास्ति वृष्टिर्न देवता}


\threelineshloka
{कथं चाराजकं राष्ट्रं शक्यं धारयितुं प्रभो}
{तस्माद्गर्भं समाधत्स्व भीष्मः संवर्धयिष्यति ॥व्यास उवाच}
{}


\twolineshloka
{यदि पुत्रः प्रदातव्यो मया भ्रातुरकालिकः}
{विरूपतां मे सहतां तयोरेतत्परं व्रतम्}


\twolineshloka
{यदि मे सहते गन्धं रूपं वेषं तथा वपुः}
{अद्यैव गर्भं कौसल्या विशिष्टं प्रतिपद्यताम्}


\threelineshloka
{`तस्यापि च शतं पुत्रा भवितारो न संशयः}
{गोप्तारः कुरुवंशस्य भवत्याः शोकनाशनाः ॥'वैशंपायन उवाच}
{}


\twolineshloka
{एवमुक्त्वा महातेजा व्यासः सत्यवतीं तदा}
{शयने सा च कौसल्या शुचिवस्त्रा ह्यलङ्कृता}


\twolineshloka
{समागमनमाकाङ्क्षेदिति सोऽन्तर्हितो मुनिः}
{ततोऽभिगम्य सा देवी स्नुषां रहसि संगताम्}


\twolineshloka
{धर्म्यमर्थसमायुक्तमुवाच वचनं हितम्}
{कौसल्ये धर्मतन्त्रं त्वां यद्ब्रवीमि निबोध तत्}


\twolineshloka
{भरतानां समुच्छेदो व्यक्तं मद्भाग्यसंक्षयात्}
{व्यथितां मां च संप्रेक्ष्य पितृवंशं च पीडितम्}


\twolineshloka
{भीष्मो बुद्धिमदान्मह्यं कुलस्यास्य विवृद्धये}
{सा च बुद्धिस्त्वय्यधीना पुत्रि प्रापय मां तथा}


\twolineshloka
{नष्टं च भारतं वंशं पुनरेव समुद्धऱ}
{पुत्रं जनय सुश्रोणि देवराजसमप्रभम्}


\twolineshloka
{स हि राज्यधुरं गुर्वीमुद्वक्ष्यति कुलस्य नः}
{`एवमुक्त्वा तु सा देवी स्नुषां सत्यवती तदा ॥'}


\twolineshloka
{सा धर्मतोऽनुनीयैनां कथंचिद्धर्मचारिणीम्}
{भोजयामास विप्रांश्च देवर्षीनतिथींस्तथा}


\chapter{अध्यायः ११५}
\twolineshloka
{वैशंपायन उवाच}
{}


\twolineshloka
{ततः सत्यवती काले वधूं स्नातामृतौ तदा}
{संवेशयन्ती शयने शनैर्वचनमब्रवीत्}


\twolineshloka
{कौसल्ये देवरस्तेऽस्ति सोऽद्य त्वाऽनुप्रवेक्ष्यति}
{अप्रमत्ता प्रतीक्षैनं निशीथे ह्यागमिष्यति}


\twolineshloka
{श्वश्र्वास्तद्वचनं श्रुत्वा शयाना शयने शुभे}
{साऽचिन्तयत्तदा भीष्ममन्यांश्च कुरुपुङ्गवान्}


\twolineshloka
{`ततः सुप्तजनप्रायेऽर्धरात्रे भगवानृषिः}
{दीप्यमानेषु दीपेषु शरणं प्रविवेश ह}


\twolineshloka
{ततोऽम्बिकायां प्रथमं नियुक्तः सत्यवागृषिः}
{जगाम तस्याः शयनं विपुले तपसि स्थितः}


\twolineshloka
{तं समीक्ष्य तु कौसल्या दुष्प्रेक्षमतथोचिता}
{विरूप इति वित्रस्ता संकुच्यासीन्निमीलिता}


\twolineshloka
{विरूपो हि जटी चापि दुर्वर्णः परुषः कृशः}
{सुगन्धेतरगन्धश्च सर्वथा दुष्प्रधर्षणः ॥'}


\twolineshloka
{तस्य कृष्णस्य कपिलां जटां दीप्ते च लोचने}
{बब्रूणि चैव श्मश्रूमि दृष्ट्वा देवी न्यमीलयत्}


\twolineshloka
{संभूव तया सार्धं मातुः प्रियचिकीर्षया}
{भयात्काशिसुता तं तु नाशक्नोदभिवीक्षितुम्}


\twolineshloka
{ततो निष्क्रान्तमागम्य माता पुत्रमुवाच ह}
{अप्यस्यां गुणवान्पुत्र राजपुत्रो भविष्यति}


\twolineshloka
{निशम्य तद्वचो मातुर्व्यासः सत्यवतीसुतः}
{`प्रोवाचातीन्द्रियज्ञानो विधिना संप्रचोदितः ॥'}


\twolineshloka
{नागायुतसमप्राणो विद्वान्राजर्षिसत्तमः}
{महाभागो महावीर्यो महाबुद्धिर्भविष्यति}


\twolineshloka
{तस्य चापि शतं पुत्रा भविष्यन्ति महात्मनः}
{किंतु मातुः स वैगुण्यादन्ध एव भविष्यति}


\twolineshloka
{तस्य तद्वचनं श्रुत्वा माता पुत्रमथाऽब्रवीत्}
{नान्धः कुरूणां नृपतिरनुरूपस्तपोधन}


\twolineshloka
{ज्ञातिवंशस्य गोप्तारं पितॄणां वंशवर्धनम्}
{`अपरस्यामपि पुनर्मम शोकविनाशनम्}


\twolineshloka
{तस्मादवरजं पुत्रं जनयान्यं नराधिपम्}
{भ्रातुर्भार्याऽवरा चेयं रूपयौवनशालिनी}


\twolineshloka
{अस्यामुत्पादयाऽपत्यं मन्नियोगाद्गुणाधिकम्}
{'द्वितीयं कुरुवंशस्य राजानं दातुमर्हसि}


\twolineshloka
{स तथेति प्रतिज्ञाय निश्चक्राम महायशाः}
{साऽपि कालेन कौसल्या सुषुवेऽन्धं तमात्मजम्}


\twolineshloka
{पुनरेव तु सा देवी परिभाष्य स्नुषां ततः}
{ऋषिमावाहयत्सत्या यथापूर्वमरिन्दम}


\twolineshloka
{`अम्बालिकां समाहूय तस्यां सत्यवती सुतम्}
{भूयो नियोजयामास सन्तानाय कुलस्य वै}


\twolineshloka
{विषण्णाम्बालिका साध्वी निषण्णा शयनोत्तमे}
{कोन्वेष्यतीति ध्यायन्ती नियतां संप्रतीक्षते'}


\twolineshloka
{ततस्तेनैव विधिना महर्षिस्तामपद्यत}
{अम्बालिकामथाऽभ्यागादृषिं दृष्ट्वा च साऽपि तम्}


\twolineshloka
{विवर्णा पाण्डुसंकाशा समपद्यत भारत}
{तां भीतां पाण्डुसंकाशां विषण्णां प्रेक्ष्य भारत}


\twolineshloka
{व्यासः सत्यवतीपुत्र इदं वचनमब्रवीत्}
{यस्मात्पाण्डुत्वमापन्ना विरूपं प्रेक्ष्य मामिह}


\twolineshloka
{तस्मादेष सुतस्ते वै पाण्डुरेव भविष्यति}
{नाम चास्यैतदेवेह भविष्यति शुभानने}


\twolineshloka
{इत्युक्त्वा स निराक्रामद्भगवानृषिसत्तमः}
{ततो निष्क्रान्तमालोक्य सत्या पुत्रमथाऽब्रवीत्}


\twolineshloka
{`कुमारो ब्रूहि मे पुत्र अप्यत्र भविता शुभः}
{'शशंस स पुनर्मात्रे तस्य बालस्य पाण्डुताम्}


\twolineshloka
{`भविष्यति सुविक्रान्तः कुमारो दिक्षु विश्रुतः}
{पाण्डुत्वं वर्णतस्तस्य मातृदोषाद्भविष्यति}


\twolineshloka
{तस्य पुत्रा महेष्वासा भविष्यन्तीह पञ्च वै}
{इत्युक्त्वा मातरं तत्र सोऽभिवाद्य जगाम ह ॥'}


\twolineshloka
{तं माता पुनरेवान्यमेकं पुत्रमयाचत}
{तथेति च महर्षिस्तां मातरं प्रत्यभाषत}


\twolineshloka
{ततः कुमारं सा देवी प्राप्तकालमजीजनत्}
{पाण्डुलक्षणसंपन्नं दीप्यमानमिव श्रिया}


\twolineshloka
{यस्य पुत्रा महेष्वासा जज्ञिरे पञ्च पाण्डवाः}
{`तयोर्जन्मक्रियाः सर्वा यथावदनुपूर्वशः}


\twolineshloka
{कारयामास वै भीष्मो ब्राह्मणैर्वेदपारगैः}
{अन्धं दृष्ट्वाऽम्बिकापुत्रं जातं सत्यवती सुतम्}


\twolineshloka
{कौसल्यार्थे समाहूय पुत्रमन्यमयाचत}
{अन्धोयमन्यमिच्छामि कौसल्यातनयं शुभम्}


\twolineshloka
{एवमुक्तो महर्षिस्तां मातरं प्रत्यभाषत}
{नियता यदि कौसल्या भविष्यति पुनःशुभा}


\twolineshloka
{भविष्यति कुमारोऽस्या धर्मशास्त्रार्थतत्ववित्}
{तां समाधाय वै भूयः स्नुषां सत्यवती पुनः ॥'}


\twolineshloka
{ऋतुकाले ततो ज्येष्ठां वधूं तस्मै न्ययोजयत्}
{सा तु रूपं च गन्धं च महर्षेः प्रविचिन्त्य तं}


\twolineshloka
{नाकरोद्वचनं देव्या भयात्सुरसुतोपमा}
{ततःस्वैर्भूषणैर्दासीं भूषयित्वाऽप्सरोपमाम्}


\twolineshloka
{प्रेषयामास कृष्णाय ततः काशिपतेः सुता}
{सा तं त्वृषिमनुप्राप्तं प्रत्युद्गम्याभिवाद्य च}


\twolineshloka
{संविवेशाभ्यनुज्ञाता सत्कृत्योपचचार ह}
{`वाग्भावोपप्रदानेन गात्रसंस्पर्शनेन च ॥'}


\twolineshloka
{कामोपभोगेन रहस्तस्यां तुष्टिमगादृषिः}
{तया सहोषितो राजन्महर्षिः संशितव्रतः}


\threelineshloka
{उत्तिष्ठन्नब्रवीदेनामभुजिष्या भविष्यसि}
{अयं च ते शुभे गर्भः श्रेयानुदरमागतः}
{धर्मात्मा भविता लोके सर्वबुद्धिमतां वरः}


\twolineshloka
{स जज्ञे विदुरो नाम कृष्णद्वैपायनात्मजः}
{धृतराष्ट्रस्य वै भ्राता पाण्डोश्चैव महात्मनः}


\twolineshloka
{धर्मो विदुररूपेण शापात्तस्य महात्मनः}
{माण्डव्यस्यार्थतत्त्वज्ञः कामक्रोधविवर्जितः}


\twolineshloka
{कृष्णद्वैपायनोऽप्येतत्सत्यवत्यै न्यवेदयत्}
{प्रलम्भमात्मनश्चैव शूद्रायाः पुत्रजन्म च}


\twolineshloka
{स धर्मस्यानृणो भूत्वा पुनर्मात्रा समेत्य च}
{तस्यै गर्भं समावेद्य तत्रैवान्तरधीयत}


\twolineshloka
{एते विचित्रवीर्यस्य क्षेत्रे द्वैपायनादपि}
{जज्ञिरे देवगर्भाभाः कुरुवंशविवर्धनाः}


\chapter{अध्यायः ११६}
\twolineshloka
{जनमेजय उवाच}
{}


\threelineshloka
{किं कृतं कर्म धर्मेण येन शापमुपेयिवान्}
{कस्य शापाच्च ब्रह्मर्षेः शूद्रयोनावजायत ॥वैशंपायन उवाच}
{}


\twolineshloka
{बभूव ब्राह्मणः कश्चिन्माण्डव्य इति विश्रुतः}
{धृतिमान्सर्वधर्मज्ञः सत्ये तपसि च स्थितः}


\threelineshloka
{`स तीर्थयात्रां विचरन्नाजगाम यदृच्छया}
{संनिकृष्टानि तीर्थानि ग्रामाणां यानि कानि च}
{तत्राश्रमपदं कृत्वा वसति स्म महामुनिः ॥'}


\twolineshloka
{स आश्रमपदद्वारि वृक्षमूले महातपाः}
{ऊर्ध्वबाहुर्महायोगी तस्थौ मौनव्रतान्वितः}


\twolineshloka
{तस्य कालेन महता तस्मिंस्तपसि वर्ततः}
{तमाश्रममनुप्राप्ता दस्यवो लोप्त्रहारिणः}


\twolineshloka
{अनुसार्यमाणा बहुभी रक्षिभिर्भरतर्षभ}
{`तामेव वसतिं जग्मुस्ते ग्रामाल्लोप्त्रहारिणः}


\twolineshloka
{यस्मिन्नावसथे शेते स मुनिः संशितव्रतः}
{'ते तस्यावसथे लोप्त्रं दस्यवः कुरुसत्तम}


\twolineshloka
{निधाय च भयाल्लीनास्तत्रैवानागते बले}
{तेषु लीनेष्वथो शीघ्रं ततस्तद्रक्षिणां बलम्}


\twolineshloka
{आजगाम ततोऽपश्यंस्तमृषिं तस्करानुगाः}
{तमपृच्छंस्ततो राजंस्तथावृत्तं तपोधनम्}


\twolineshloka
{कतरेण पथा याता दस्यवो द्विजसत्तम}
{तेन गच्छामहे ब्रह्मन्यथा शीघ्रतरं वयम्}


\twolineshloka
{तथा तु रक्षिमां तेषां ब्रुवतां स तपोधनः}
{न किंचिद्वचनं राजन्नब्रवीत्साध्वसाधु वा}


\twolineshloka
{ततस्ते राजपुरुषा विचिन्वानास्तमाश्रमम्}
{ददृशुस्तत्र लीनांस्तांश्चोरांस्तद्द्रव्यमेव च}


\twolineshloka
{ततः शङ्का समभवद्रक्षिणां तं मुनिं प्रति}
{संयम्यैनं ततो राज्ञे दस्यूंश्चैव न्यवेदयन्}


\twolineshloka
{तं राजा सह तैश्चोरैरन्वशाद्वध्यतामिति}
{स रक्षिभिस्तैरज्ञातः शूले प्रोतो महातपाः}


\twolineshloka
{ततस्ते शूलमारोप्य तं मुनिं रक्षिणस्तदा}
{प्रतिजग्मुर्महीपालं धनान्यादाय तान्यथ}


\twolineshloka
{शूलस्थः स तु धर्मात्मा कालेन महता ततः}
{निराहारोऽपि विप्रर्षिर्मरणं नाभ्यपद्यत}


\twolineshloka
{धारयामास च प्राणानृषींश्च समुपानयत्}
{शूलाग्रे तप्यमानेन तपस्तेन महात्मना}


\threelineshloka
{सन्तापं परमं जग्मुर्मुनयस्तपसाऽन्विताः}
{ते रात्रौ शकुना भूत्वा सन्निपत्त्य तु भारत}
{दर्शन्तो यथाशक्ति तमपृच्छन्द्विजोत्तमम्}


\twolineshloka
{श्रोतुमिच्छामहे ब्रह्मन्किं पापं कृतवानसि}
{येनेह समनुप्राप्तं शूले दुःखभयं महत्}


\chapter{अध्यायः ११७}
\twolineshloka
{वैशंपायन उवाच}
{}


\twolineshloka
{ततः स मुनिशार्दूलस्तानुवाच तपोधनान्}
{दोषतः कं गमिष्यामि न हि मेऽन्योपराध्यति}


\twolineshloka
{तं दृष्ट्वा रक्षिणस्तत्र तथा बहुतिथेऽहनि}
{न्यवेदयंस्तथा राज्ञे यथावृत्तं नराधिप}


\threelineshloka
{राजा च तमृषिं श्रुत्वा निष्क्रम्य सह मन्त्रिभिः}
{प्रसादयामास तदा शूलस्थमृषिसत्तमम् ॥प्रजोवाच}
{}


\threelineshloka
{यन्मयाऽपकृतं मोहादज्ञानादृषिसत्तम}
{प्रसादये त्वां तत्राहं न मे त्वं क्रोद्धुमर्हसि ॥वैशंपायन उवाच}
{}


\twolineshloka
{एवमुक्तस्ततो राज्ञा प्रसादमकरोन्मुनिः}
{कृतप्रसादं राजा तं ततः समवतारयत्}


\twolineshloka
{अवतार्य च शूलाग्रात्तच्छूलं निश्चकर्ष ह}
{अशक्नुवंश्च निष्क्रष्टुं शूलं मूले स चिच्छिदे}


\threelineshloka
{स तथान्तर्गतेनैव शूलेन व्यचरन्मुनिः}
{कण्ठपार्श्वान्तरस्थेन शङ्कुना मुनिराचत्}
{पुष्पभाजनधारी स्यादिति चिन्तापरोऽभवत् ॥'}


% Check verse!
स चातितपसा लोकान्विजिग्ये दुर्लभान्परैः
\twolineshloka
{अणीमाण्डव्य इति च ततो लोकेषु गीयते}
{स गत्वा सदनं विप्रो धर्मस्य परमार्थवित्}


\twolineshloka
{आसनस्थं ततो धर्मं दृष्ट्वोपालभत प्रभुम्}
{किं नु तद्दुष्कृतं कर्म मया कृतमजानता}


\threelineshloka
{यस्येयं फलनिर्वृत्तिरीदृश्यासादिता मया}
{शीघ्रमाचक्ष्व मे तत्त्वं पश्य मे तपसो बलम् ॥धर्म उवाच}
{}


\twolineshloka
{पतङ्गिकानां पुच्छेषु त्वयेषीका प्रवेशिता}
{कर्मणस्तस्य ते प्राप्तं फलमेतत्तपोधन}


\threelineshloka
{स्वल्पमेव यथा दत्तं दानं बहुगुणं भवेत्}
{अधर्म एवं विप्रर्षे बहुदुःखफलप्रदः ॥अणीमाण्डव्य उवाच}
{}


\threelineshloka
{कस्मिन्काले मया तत्तु कृतं ब्रूहि यथातथम्}
{तेनोक्तं धर्मराजेन बालभावे त्वया कृतम् ॥अणीमाण्डव्य उवाच}
{}


\twolineshloka
{बालो हि द्वादशाद्वर्षाज्जन्मतो यत्करिष्यति}
{न भविष्यत्यधर्मोऽत्र न प्रज्ञास्यति वै दिशः}


\twolineshloka
{अल्पेऽपराधेऽपि महान्मम दण्डस्त्वया धृतः}
{गरीयान्ब्राह्मणवधः सर्वभूतवधादपि}


\twolineshloka
{शूद्रयोनावतो धर्म मानुषः संभविष्यसि}
{मर्यादां स्थापयाम्यद्य लोके कर्मफलोदयाम्}


\threelineshloka
{आ चतुर्दशकाद्वर्षान्न भविष्यति पातकम्}
{परतः कुर्वतामेव दोष एव भविष्यति ॥वैशंपायन उवाच}
{}


\twolineshloka
{एतेन त्वपराधेन शापात्तस्य महात्मनः}
{धर्मो विदुररूपेण शूद्रयोनावजायत}


\twolineshloka
{धर्मे चार्थे च कुशलो लोभक्रोधविवर्जितः}
{दीर्घदर्शी शमपरः कुरूणां च हिते रतः}


\chapter{अध्यायः ११८}
\twolineshloka
{वैशंपायन उवाच}
{}


\threelineshloka
{`धृतराष्ट्रे च पाण्डौ च विदुरे च महात्मनि}
{'एषु त्रिषु कुमारेषु जातेषु कुरुजाङ्गलम्}
{कुरवोऽथ कुरुक्षेत्रं त्रयमेतदवर्धत}


\twolineshloka
{ऊर्ध्वसस्याऽभवद्भूमिः सस्यानि फलवन्ति च}
{यथर्तुवर्षी पर्जन्यो बहुपुष्पफला द्रुमाः}


\twolineshloka
{वाहनानि प्रहृष्टानि मुदिता मृगपक्षिणः}
{गन्धवन्ति च माल्यानि रसवन्ति फलानि च}


\twolineshloka
{वणिग्भिश्चान्वकीर्यन्त नगराण्यथ शिल्पिभिः}
{शूराश्च कृतविद्याश्च सन्तश्च सुखिनोऽभवन्}


\twolineshloka
{नाभवन्दस्यवः केचिन्नाधर्मरुचयो जनाः}
{प्रदेशेष्वपि राष्ट्राणां कृतं युगमवर्तत}


\twolineshloka
{धर्मक्रिया यज्ञशीलाः सत्यव्रतपरायणाः}
{अन्योन्यप्रीतिसंयुक्ता व्यवर्धन्त प्रजास्तदा}


\twolineshloka
{मानक्रोधविहीनाश्च नरा लोभविवर्जिताः}
{अन्योन्यमभ्यनन्दन्त धर्मोत्तरमवर्तत}


\twolineshloka
{तन्महोदधिवत्पूर्णं नगरं वै व्यरोचत}
{द्वारतोरणनिर्यूहैर्युक्तमभ्रचयोपमैः}


\threelineshloka
{प्रसादशतसंबाधं महेन्द्रपुरसन्निभम्}
{नदीषु वनखण्डेषु वापीपल्वलसानुषु}
{काननेषु च रम्येषु विजह्रुर्मुदिता जनाः}


\twolineshloka
{उत्तरैः कुरुभिः सार्धं दक्षिणाः कुरवस्तथा}
{विस्पर्धमाना व्यचरंस्तथा देवर्षिचारणैः}


\twolineshloka
{नाभवत्कृपणः कश्चिन्नाभवन्विधवाः स्त्रियः}
{तस्मिञ्जनपदे रम्ये कुरुभिर्बहुलीकृते}


\twolineshloka
{कूपारामसभावाप्यो ब्राह्मणावसथास्तथा}
{बभूवुः सर्वर्द्धियुतास्तस्मिन्राष्ट्रे सदोत्सवाः}


\twolineshloka
{भीष्मेण धर्मतो राजन्सर्वतः परिरक्षिते}
{बभूव रमणीयश्च चैत्ययूपशताङ्कितः}


\twolineshloka
{स देशः परराष्ट्राणि विमृज्याभिप्रवर्धितः}
{भीष्मेण विहितं राष्ट्रे धर्मचक्रमवर्तत}


\twolineshloka
{क्रियमाणेषु कृत्येषु कुमाराणां महात्मनाम्}
{पौरजानपदाः सर्वे बभूवुः परमोत्सुकाः}


\twolineshloka
{गृहेषु कुरुमुख्यानां पौराणां च नराधिप}
{दीयतां भुज्यतां चेति वाचोऽश्रूयन्त सर्वशः}


\twolineshloka
{धृतराष्ट्रश्च पाण्डुश्च विदुरश्च महामतिः}
{जन्मप्रभृति भीष्मेण पुत्रवत्परिपालिताः}


\twolineshloka
{संस्कारैः संस्कृतास्ते तु व्रताध्ययनसंयुताः}
{श्रमव्यायामकुशलाः समपद्यन्त यौवनम्}


\twolineshloka
{धनुर्वेदे च वेदे च गदायुद्धेऽसिचर्मणि}
{तथैव गजशिक्षायां नीतिशास्त्रेषु पारगाः}


\twolineshloka
{इतिहासपुराणेषु नानाशिक्षासु बोधिताः}
{वेदवेदाङ्गतत्त्वज्ञाः सर्वत्र कृतनिश्चयाः}


\twolineshloka
{`वैदिकाध्ययने युक्तो नीतिशास्त्रेषु पारगः}
{भीष्मेण राजा कौरव्यो धृतराष्ट्रोऽभिषेचितः}


\twolineshloka
{धनुर्वेदेऽश्वपृष्ठे च गदायुद्धेऽसिचर्मणि}
{तथैव गजशिक्षायामस्त्रेषु विविधेषु च}


\twolineshloka
{अर्थधर्मप्रधानासु विद्यासु विविधासु च}
{गतः पारं यदा पाण्डुस्तदा सेनापतिः कृतः ॥'}


\twolineshloka
{पाण्डुर्धनुषि विक्रान्तो नरेष्वभ्यदिकोऽभवत्}
{अन्येभ्यो बलवानासीद्धृतराष्ट्रो महीपतिः}


\twolineshloka
{अमात्यो मनुजेन्द्रस्य बाल एव यशस्विनः}
{भीष्मेण सर्वधर्माणां प्रणेता विदुरः कृतः}


\twolineshloka
{`सर्वशास्त्रार्थतत्त्वज्ञो बुद्धिमेधापटुर्युवा}
{भावेनागमयुक्तेन सर्वं वेदयते जगत् ॥'}


\twolineshloka
{त्रिषु लोकेषु न त्वासीत्कश्चिद्विदुरसंमितः}
{धर्मनित्यस्तथा राजन्धर्मं च परमं गतः}


\twolineshloka
{प्रनष्टं शान्तनोर्वंशं समीक्ष्य पुनरुद्धृतम्}
{ततो निर्वचनं लोके सर्वराष्ट्रेष्ववर्तत}


\twolineshloka
{वीरसूनां काशिसुते देशानां कुरुजाङ्गलम्}
{सर्वध्रमविदां भीष्मः पुराणां गजसाह्वयम्}


\twolineshloka
{धृतराष्ट्रस्त्वचक्षुष्ट्वाद्रज्यं न प्रत्यपद्यत}
{पारसवत्वाद्विदुरो राजा पाण्डुर्बभूव ह}


\twolineshloka
{`अथ शुश्राव विप्रेभ्यः कुन्तिभोजमहीपतेः}
{रूपयौवनसंपन्नां सुतां सागरगासुतः}


\twolineshloka
{सुबलस्य च कल्याणीं गान्धाराधिपतेः सुताम्}
{सुतां च मद्रराजस्य रूपेणाप्रतिमां भुवि ॥'}


\twolineshloka
{कदाचिदथ गाङ्गेयः सर्वनीतिमतां वरः}
{विदुरं धर्मतत्त्वज्ञं वाक्यमाह यथोचितम्}


\chapter{अध्यायः ११९}
\twolineshloka
{भीष्म उवाच}
{}


\twolineshloka
{गुणैः समुदितं सम्यगिदं नः प्रथितं कुलम्}
{अत्यन्यान्पृथिवीपालान्पृथिव्यामधिराज्यभाक्}


\twolineshloka
{रक्षितं राजभिः पूर्वं धर्मविद्भिर्महात्मभिः}
{नोत्सादमगमच्चेदं कदाचिदिह नः कुलम्}


\twolineshloka
{मया च सत्यवत्या च कृष्णेन च महात्मना}
{समवस्थापित भूयो युष्मासु कुलतन्तुषु}


\twolineshloka
{तच्चैतद्वर्धते भूयः कुलं सागरवद्यथा}
{तथा मया विधातव्यं त्वया चैव न संशयः}


\twolineshloka
{श्रूयते यादवी कन्या स्वनुरूपा कुलस्य नः}
{सुबलस्यात्मजा चैव तथा मद्रेश्वरस्य च}


\twolineshloka
{कुलीना रूपवत्यश्च ताः कन्याः पुत्र सर्वशः}
{उचिताश्चैव संबन्धे तेऽस्माकं क्षत्रियर्षभाः}


\threelineshloka
{मन्ये वरयितव्यास्ता इत्यहं धीमतां वर}
{सन्तानार्थं कुलस्यास्य यद्वा विदुर मन्यसे ॥विदुर उवाच}
{}


\threelineshloka
{भवान्पिता भावन्माता भवान्नः परमो गुरुः}
{तस्मात्स्वयं कुलस्यास्य विचार्य कुरु यद्धितम् ॥वैशंपायन उवाच}
{}


\twolineshloka
{अथ शुश्राव विप्रेभ्यो गान्धारीं सुबलात्मजाम्}
{आराध्य वरदं देवं भगनेत्रहरं हरम्}


\twolineshloka
{गान्धारी किल पुत्राणां शतं लेभे वरं शुभा}
{इति शुश्राव तत्त्वेन भीष्मः कुरुपितामहः}


\twolineshloka
{ततो गान्धारराजस्य प्रेषयामास भारत}
{अचक्षुरिति तत्रासीत्सुबलस्य विचारणा}


\twolineshloka
{कुलं ख्यातिं च वृत्तं च बुद्ध्या तु प्रसमीक्ष्य सः}
{ददै तां धृतराष्ट्राय गान्धारीं धर्मचारिणीम्}


\twolineshloka
{गान्धारी त्वथ शुश्राव धृतराष्ट्रमचक्षुषम्}
{आत्मानं दिप्सितं चास्मै पित्रा मात्रा च भारत}


\twolineshloka
{ततः सा पटमादाय कृत्वा बहुगुणं तदा}
{बबन्ध नेत्रे स्वे राजन्पतिव्रतपरायणा}


\twolineshloka
{नाभ्यसूयां पतिमहमित्येवं कृतनिश्चया}
{ततो गान्धारराजस्य पुत्रः शकुनिरभ्ययात्}


\threelineshloka
{स्वसारं परया लक्ष्म्या युक्तामादाय कौरवान्}
{तां तदा धृतराष्ट्राय ददौ परमसत्कृताम्}
{भीष्मस्यानुमते चैव विवाहं समकारयत्}


\twolineshloka
{दत्त्वा स भगिनीं वीरो यथार्हं च परिच्छदम्}
{पुनरायात्स्वनगरं भीष्मेण प्रतिपूजितः}


\twolineshloka
{गान्धार्यपि वरारोहा शीलाचारविचिषेटितैः}
{तुष्टिं कुरूणां सर्वेषां जनयामास भारत}


\twolineshloka
{`गान्धारी सा पतिं दृष्ट्वा प्रज्ञाचक्षुषमीश्वरम्}
{अतिचाराद्भृशं भीता भर्तुः सा समचिन्तयत्}


\twolineshloka
{सा दृष्टिविनिवृत्त्या हि भर्तुश्च समतां ययौ}
{नहि सूक्ष्मेप्यतीचारे भर्तुः सा ववृते तदा}


\twolineshloka
{वृत्तेनाराध्य तान्सर्वान्गुरून्पतिपरायणा}
{वाचापि पुरुषानन्यान्सुव्रता नान्वकीर्तयत्}


\twolineshloka
{तस्याः सहोदरीः कन्याः पुनरेव ददौ दश}
{गान्धारराजः सुबलो भीष्मेण च वृतस्तदा}


\twolineshloka
{सत्यव्रतां सत्यसेनां सुदेष्णां चापि संहिताम्}
{तेजश्श्र्वां सुश्रवां च तथैव निकृतिं शुभाम्}


\twolineshloka
{शंभ्वठां च दशार्णां च गान्धारीर्दश विश्रुताः}
{एकाह्ना प्रतिजग्राह धृतराष्ट्रो जनेश्वरः}


\twolineshloka
{ततः शान्तनवो भीष्मो धानुष्कस्तास्ततस्ततः}
{अददाद्धृतराष्ट्रस्य राजपुत्रीः परश्शतम् ॥'}


\chapter{अध्यायः १२०}
\twolineshloka
{वैशंपायन उवाच}
{}


\twolineshloka
{शूरो नाम यदुश्रेष्ठो वसुदेवपिताऽभवत्}
{तस्य कन्या पृथा नाम रूपेणाप्रतिमा भुवि}


\twolineshloka
{पितृष्वस्रीयाय स तामनपत्याय भारत}
{अग्र्यमग्रे प्रतिज्ञाय स्वस्यापत्यं स सत्यवाक्}


\twolineshloka
{अग्रजामथ तां कन्यां शूरोऽनुग्रहकाङ्क्षिणे}
{प्रददौ कुन्तिभोजाय सखा सख्ये महात्मने}


\twolineshloka
{नियुक्ता सा पितुर्गेहे ब्राह्मणातिथिपूजने}
{उग्रं पर्यचरत्तत्र ब्राह्मणं संशितब्रतम्}


\twolineshloka
{निगूढनिश्चयं धर्मे यं तं दुर्वाससं विदुः}
{तमुग्रं संशितात्मानं सर्वयत्नैरतोषयत्}


\twolineshloka
{`दध्याज्यकादिभिर्नित्यं व्यञ्जनैः प्रत्यहं शुभा}
{सहस्रसङ्ख्यैर्योगीन्द्रमुपचारदनुत्तमा}


\threelineshloka
{दुर्वासा वत्सरस्यान्ते ददौ मन्त्रमनुत्तमम्'}
{यशस्विन्यै पृथायै तदापद्धर्मान्ववेक्षया}
{अभिचाराभिसंयुक्तमब्रवीच्चैव तां मुनिः}


\twolineshloka
{यं यं देवं त्वमेतेन मन्त्रेणावाहयिष्यसि}
{तस्य तस्य प्रभावेण तव पुत्रो भविष्यति}


\twolineshloka
{तथोक्ता सा तु विप्रेण कुन्ती कौतूहलान्विता}
{कन्या सती देवमर्कमाजुहाव यशस्विनी}


\twolineshloka
{ततो घनान्तरं कृत्वा स्वमार्गं तपनस्तदा}
{उपतस्थे स तां कन्यां पृथां पृथुललोचनाम्}


\twolineshloka
{सा ददर्श तमायान्तं भास्करं लोकभावनम्}
{विस्मिता चानवद्याङ्गी दृष्ट्वा तन्महदद्भुतम्}


\fourlineindentedshloka
{`साब्रवीद्भगवन्कस्त्वमाविर्भूतो ममाग्रतः}
{आदित्य उवाच}
{आहूतोपस्थितं भद्रे ऋषिमन्त्रेण चोदितम्}
{विद्धि मां पुत्रलाभाय देवमर्कं शुचिस्मिते}


\twolineshloka
{पुत्रस्ते निर्मितः सुभ्रु शृणु महादृक्छुभानने}
{आदित्ये कुण्डले बिभ्रत्कवचं चैव मामकम्}


\twolineshloka
{शस्त्रास्त्राणामभेद्यश्च भविष्यति शुचिस्मिते}
{नास्य किंचिददेयं च ब्राह्मणेभ्यो भविष्यति}


\threelineshloka
{चोद्यमानो मया चापि न क्षमं चिन्तयिष्यति}
{दास्यत्येव हि विप्रेभ्यो मानी चैव भविष्यति ॥वैशंपायन उवाच}
{}


\twolineshloka
{एवमुक्ता ततः कुन्ती गोपतिं प्रत्युवाच ह}
{कन्या पितृसा चाहं पुरुषार्थो न चैव मे ॥'}


\twolineshloka
{कश्चिन्मे ब्राह्मणः प्रादाद्वरं विद्यां च शत्रुहन्}
{तद्विजिज्ञासयाऽऽह्वानं कृतवत्यस्मि ते विभो}


\threelineshloka
{एतस्मिन्नपराधे त्वां शिरसाऽहं प्रसादये}
{योषितो हि सदा रक्ष्यास्त्वपराधेऽपि नित्यशः ॥सूर्य उवाच}
{}


\twolineshloka
{वेदाहं सर्वमेवैतद्यद्दुर्वासा वरं ददौ}
{संत्यज्य भयमेवेह क्रियतां सङ्गमो मम}


\twolineshloka
{अमोघं दर्शनं मह्यमाहूतश्चास्मि ते शुभे}
{वृथाऽऽह्वानेऽपि ते भीरु दोषः स्यान्नात्र संशयः}


\twolineshloka
{`यद्येवं मन्यसे भीरु किमाह्वयसि भास्करम्}
{यदि मामवजानासि ऋषिः स न भविष्यति}


\threelineshloka
{मन्त्रदानेन यस्मात्त्वमवलेपेन दर्पिता}
{कुलं च तेऽद्य धक्ष्यामि क्रोधदीप्तेन चक्षुषा' ॥वैशंपायन उवाच}
{}


\twolineshloka
{एवमुक्ता बहुविधं सान्त्वपूर्वं विवस्वता}
{सा तु नैच्छद्वरारोहा कन्याहमिति भारत}


\twolineshloka
{बन्धुपक्षभयाद्भीता लज्जया च यशस्विनी}
{तामर्कः पुनरेवेदब्रवीद्भरतर्षभ}


\twolineshloka
{मत्प्रसादान्न ते राज्ञि भविता दोष इत्युत ॥`कुन्त्युवाच}
{}


\threelineshloka
{प्रसीद भगवन्मह्यमवलेपो हि नास्ति मे}
{त्वयैव परिहार्यं स्यात्कन्याभावस्य दूषणम् ॥आदित्य उवाच}
{}


\threelineshloka
{व्यपयातु भयं तेऽद्य कुमारं प्रसमीक्षसे}
{मया त्वं चाप्यनुज्ञाता पुनः कन्या भविष्यसि ॥वैशंपायन उवाच}
{}


\twolineshloka
{एवमुक्ता ततः कुन्ती संप्रहृष्टतनूरुहा}
{सङ्गताऽभूत्तदा सुभ्रूरादित्येन महात्मना}


\twolineshloka
{प्रकाशकर्मा तपनः कन्यागर्भं ददौ पुनः}
{'तत्र वीरः समभवत्सर्वशस्त्रभृतां वरः}


\twolineshloka
{आमुक्तकवचः श्रीमान्देवगर्भः श्रियान्वितः}
{सहजं कवचं बिभ्रत्कुण्डलोद्द्योतिताननः}


\twolineshloka
{अजायत सुतः कर्णः सर्वलोकेषु विश्रुतः}
{प्रादाच्च तस्यै कन्यात्वं पुनः स परमद्युतिः}


\twolineshloka
{दत्त्वा च तपतां श्रेष्ठो दिवमाचक्रमे ततः}
{दृष्ट्वा कुमारं जातं सा वार्ष्णेयी दीनमानसा}


\twolineshloka
{एकाग्रं चिन्तयामास किं कृत्वा सुकृतं भवेत्}
{गूहमानापचारं सा बन्धुपक्षभयात्तदा}


\twolineshloka
{मञ्जूषां रत्नसंपूर्णां कृत्वा बालसमाश्रिताम्}
{उत्ससर्ज कुमारं तं जले कुन्ती महाबलम्}


\twolineshloka
{तमुत्सृष्टं जले गर्भं राधाभर्ता महायशाः}
{पुत्रत्वे कल्पयामास सभार्यः सूतनन्दनः}


\twolineshloka
{नामधेयं च चक्राते तस्य बालस्य तावुभौ}
{वसुना सह जातोऽयं वसुषेणो भवत्विति}


\twolineshloka
{स वर्धमानो बलवान्सर्वास्त्रेषूद्यतोऽभवत्}
{आपृष्ठतापादादित्यमुपातिष्ठत वीर्यवान्}


\twolineshloka
{तस्मिन्काले तु जपतस्तस्य वीरस्य धीमतः}
{नादेयं ब्राह्मणेष्वासीत्किंचिद्वसु महीतले}


\twolineshloka
{`ततः काले तु कस्मिंश्चित्स्वप्नान्ते कर्णमब्रवीत्}
{आदित्यो ब्राह्मणो भूत्वा शृणु वीर वचो मम}


\twolineshloka
{प्रभातायां रजन्यां त्वामागमिष्यति वासवः}
{न तस्य भिक्षा दातव्या विप्ररूपी भविष्यति}


\threelineshloka
{निश्चयोऽस्यापहर्तुं ते कवचं कुण्डले तथा}
{अतस्त्वां बोधयाम्येष स्मर्तासि वचनं मम ॥कर्ण उवाच}
{}


\twolineshloka
{शक्रो मां विप्ररूपेण यदि वै याचते द्विज}
{कथं तस्मै न दास्यामि यथा चास्म्यवबोधितः}


\threelineshloka
{विप्राः पूज्यास्तु देवानां सततं प्रियमिच्छताम्}
{तं देवदेवं जानन्वै न शक्तोऽस्म्यवमन्त्रणे ॥सूर्य उवाच}
{}


\threelineshloka
{यद्येवं शृणु मे वीर वरं ते सोऽपि दास्यति}
{शक्तिं त्वमपि याचेथाः सर्वशत्रुविबाधिनीम् ॥वैशंपायन उवाच}
{}


\twolineshloka
{एवमुक्त्वा द्विजः स्वप्ने तत्रैवान्तरधीयत}
{कर्णः प्रबुद्धस्तं स्वप्नं चिन्तयानोऽभवत्तदा}


\twolineshloka
{तमिन्द्रो ब्राह्मणो भूत्वा पुत्रार्थं भूतभावनः}
{कुण्डले प्रार्थयामास कवचं च महाद्युतिः}


\twolineshloka
{उत्कृत्याविमनाः स्वाङ्गात्कवचं रुधिरस्रवम्}
{कर्णौ पार्श्वे च द्वे छित्त्वा प्रायच्छत्स कृताञ्जलिः}


\twolineshloka
{प्रतिगृह्य तु देवेशस्तुष्टस्तेनास्य कर्मणा}
{अहो साहसमित्याह मनसा वासवो हसन्}


\twolineshloka
{देवदानवयक्षाणां गन्धर्वोरगरक्षसाम्}
{न तं पश्यामि यो ह्येतत्कर्म कर्ता भविष्यति}


\fourlineindentedshloka
{प्रीतोऽस्मि कर्मणा तेन वरं ब्रूहि यदिच्छसि}
{कर्ण उवाच}
{इच्छामि भगवद्दत्तां शक्तिं शत्रुनिबर्हणीम् ॥वैशंपायन उवाच}
{}


\twolineshloka
{शक्तिं तस्मै ददौ शक्रो विस्मयाद्वाक्यमब्रवीत्}
{देवदानवयक्षाणां गन्धर्वोरगरक्षसाम्}


\threelineshloka
{यस्मै क्षेप्स्यसि रुष्टः सन्सोऽनया न भविष्यति}
{हत्वैकं समरे शत्रुं ततो मामागमिष्यति}
{इत्युक्त्वान्तर्दधे शक्रो वरं दत्त्वा तु तस्य वै}


\twolineshloka
{प्राङ्नाम तस्य कथितं वसुषेण इति क्षितौ}
{कर्णो वैकर्तनश्चैव कर्मणा तेन सोऽभवत्}


\chapter{अध्यायः १२१}
\twolineshloka
{वैशंपायन उवाच}
{}


\twolineshloka
{सत्वरूपगुणोपेता धर्मारामा महाव्रता}
{दुहिता कुन्तिभोजस्य पृथा पृथुललोचना}


\twolineshloka
{तां तु तेजस्विनीं कन्यां रूपयौवनशालिनीम्}
{व्यावृण्वन्पार्थिवाः केचिदतीव स्त्रीगुणैर्युताम्}


\twolineshloka
{ततः सा कुन्तिभोजेन राज्ञाहूय नराधिपान्}
{पित्रा स्वयंवरे दत्ता दुहिता राजसत्तम}


\twolineshloka
{ततः सा रङ्गमध्यस्थं तेषां राज्ञां मनस्विनी}
{ददर्श राजशार्दूलं पाण्डुं भरतसत्तमम्}


\twolineshloka
{सिंहदर्पं महोरस्कं वृषभाक्षं महाबलम्}
{आदित्यमिव सर्वेषां राज्ञां प्रच्छाद्य वै प्रभाः}


\twolineshloka
{तिष्ठन्तं राजसमितौ पुरन्दरमिवापरम्}
{तं दृष्ट्वा साऽनवद्याङ्गी कुन्तिभोजसुता शुभा}


\twolineshloka
{पाण्डुं नरवरं रङ्गे हृदयेनाकुलाऽभवत्}
{ततः कामपरीताङ्गी सकृत्प्रचलमानसा}


\twolineshloka
{व्रीडमान स्रजं कुन्ती राज्ञः स्कन्धे समासजत्}
{तं निशम्य वृतं पाण्डुं कुन्त्या सर्वे नराधिपाः}


\twolineshloka
{यथागतं समाजग्मुर्गजैरश्वै रथैस्तथा}
{ततस्तस्याः पिता राजन्विवाहमकरोत्प्रभुः}


\twolineshloka
{स तया कुन्तिभोजस्य दुहित्रा कुरुनन्दनः}
{युयुजेऽमितसौभाग्यः पौलोम्या मघवानिव}


\threelineshloka
{कुन्त्याः पाण्डोश्च राजेन्द्र कुन्तिभोजो महीपतिः}
{कृत्वोद्वाहं तदा तं तु नानावसुभिरर्चितम्}
{स्वपुरं प्रेषयामास स राजा कुरुसत्तम}


\twolineshloka
{ततो बलेन महता नानाध्वजपताकिना}
{स्तूयमानः स चाशीर्भिर्ब्राह्मणैश्च महर्षिभिः}


\twolineshloka
{संप्राप्य नगरं राजा पाण्डुः कौरवनन्दनः}
{न्यवेशत तां भार्यां कुन्तीं स्वभवने प्रभुः}


\chapter{अध्यायः १२२}
\twolineshloka
{वैशंपायन उवाच}
{}


\twolineshloka
{ततः शान्तनवो भीष्मो राज्ञः पाण्डोर्यशस्विनः}
{विवाहस्यापरस्यार्थे चकार मतिमान्मतिम्}


\twolineshloka
{सोऽमात्यैः स्थविरैः सार्धं ब्राह्मणैश्च महर्षिभिः}
{बलेन चतुरङ्गेण ययौ मद्रपतेः पुरम्}


\twolineshloka
{तमागतमभिश्रुत्य भीष्मं वाहीकपुङ्गवः}
{प्रत्युद्गम्यार्चयित्वा च पुरं प्रावेशयन्नृपः}


\twolineshloka
{दत्त्वा तस्यासनं शुभ्रं पाद्यमर्घ्यं तथैव च}
{मधुपर्कं च मद्रेशः पप्रच्छागमनेऽर्थिताम्}


\twolineshloka
{तं भीष्मः प्रत्युवाचेदं मद्रराजं कुरूद्वहः}
{आगतं मां विजानीहि कन्यार्थिनमरिन्दम}


\twolineshloka
{श्रूयते भवतः साध्वी स्वसा माद्री यशस्विनी}
{तामहं वरयिष्यामि पाण्डोरर्थे यशस्विनीम्}


\twolineshloka
{युक्तरूपो हि संबन्धे त्वं नो राजन्वयं तव}
{एतत्संचिन्त्य मद्रेश गृहाणास्मान्यथाविधि}


\twolineshloka
{तमेवंवादिनं भीष्मं प्रत्यभाषत मद्रपः}
{न हि मेऽन्यो वरस्त्वत्तः श्रेयानिति मतिर्मम}


\twolineshloka
{पूर्वैः प्रवर्तितं किंचित्कुलेऽस्मिन्नृपसत्तमैः}
{साधु वा यदि वाऽसाधु तन्नातिक्रान्तुमुत्सहे}


\twolineshloka
{व्यक्तं तद्भवतश्चापि विदितं नात्र संशयः}
{न च युक्तं तथा वक्तुं भवान्देहीति सत्तम}


\twolineshloka
{कुलधर्मः स नो वीर प्रमाणं परमं च तत्}
{तेन त्वां न ब्रवीम्येतदसंदिग्धं वचोऽरिहन्}


\twolineshloka
{तं भीष्मः प्रत्युवाचेदं मद्रराजं जनाधिपः}
{धर्म एष परो राजन्स्वयमुक्तः स्वयंभुवा}


\twolineshloka
{नात्र कश्चन दोषोऽस्ति पूर्वैर्विधिरयं कृतः}
{विदितेयं च ते शल्य मर्यादा साधुसंमता}


\twolineshloka
{इत्युक्त्वा स महातेजाः शातकुम्भं कृताकृतम्}
{रत्नानि च विचित्राणि शल्यायादात्सहस्रशः}


\twolineshloka
{गजानश्वान्रथांश्चैव वासांस्याभरणानि च}
{मणिमुक्ताप्रवालं च गाङ्गेयो व्यसृजच्छुभम्}


\twolineshloka
{तत्प्रगृह्य धनं सर्वं शल्यः संप्रतीमानसः}
{ददौ तां समलङ्कृत्य स्वसारं कौरवर्षभे}


\twolineshloka
{स तां माद्रीमुपादाय भीष्मः सागरगासुतः}
{आजगाम पुरीं धीमान्प्रविष्टो गजसाह्वयम्}


\threelineshloka
{तत इष्टेऽहनि प्राप्ते मुहूर्ते साधुसंमते}
{`विवाहं कारायामास भीष्मः पाण्डोर्महात्मनः}
{'जग्राह विधिवत्पाणिं माद्र्याः पाण्डुर्नराधिपः}


\twolineshloka
{ततो विवाहे निर्वृत्ते स राजा कुरुनन्दनः}
{स्थापयामास तां भार्यां शुभे वेश्मनि भामिनीं}


\twolineshloka
{स ताभ्यां व्यचरत्सार्धं भार्याभ्यां राजसत्तमः}
{कुन्त्या माद्र्या च राजेन्द्रो यथाकामं यथासुखम्}


\twolineshloka
{ततः स कौरवो राजा विहृत्य त्रिदशा निशाः}
{जिगीषया महीं पाण्डुर्निरक्रामत्पुरात्प्रभो}


\threelineshloka
{स भीष्मप्रमुखान्वृद्धानभिवाद्य प्रणम्य च}
{धृतराष्ट्रं च कौरव्यं तथान्यान्कुरुसत्तमान्}
{आमन्त्र्य प्रययौ राजा तैश्चैवाप्यनुमोजितः}


\twolineshloka
{मङ्गलाचारयुक्ताभिराशीर्भिरभिनन्दितः}
{गजवाजिरथौघेन बलेन महताऽगमत्}


\twolineshloka
{स राजा देवगर्भाभो विजिगीषुर्वसुन्धराम्}
{हृष्टपुष्टबलैः प्रायात्पाण्डुः शत्रूननेकशः}


\twolineshloka
{पूर्वमागस्कृतो गत्वा दशार्णाः समरे जिताः}
{पाण्डुना नरसिंहेन कौरवाणां यशोभृता}


\twolineshloka
{ततः सेनामुपादाय पाण्डुर्नानाविधध्वजाम्}
{प्रभूतहस्त्यश्वयुतां पदातिरथसंकुलाम्}


\twolineshloka
{आगस्कारी महीपानां बहूनां बलदर्पितः}
{गोप्ता मगधराष्ट्रस्य दीर्घो राजगृहे हतः}


\twolineshloka
{ततः कोशं समादाय वाहनानि च भूरिशः}
{पाण्डुना मिथिलां गत्वा विदेहाः समरे जिताः}


\twolineshloka
{तथा काशिषु सुह्मेषु पुण्ड्रेषु च नरर्षभ}
{स्वबाहुबलवीर्येण कुरूणामकरोद्यशः}


\twolineshloka
{तं शरौघमहाज्वालं शस्त्रार्चिषमरिन्दमम्}
{पाण्डुपावकमासाद्य व्यवह्यन्त नराधिपाः}


\twolineshloka
{ते ससेनाः ससेनेन विध्वंसितबला नृपाः}
{पाण्डुना वशगाः कृत्वा कुरुकर्मसु योजिताः}


\twolineshloka
{तेन ते निर्जिताः सर्वे पृथिव्यां सर्वपार्थिवाः}
{तमेकं मेनिरे शूरं देवेष्विव पुरन्दरम्}


\twolineshloka
{तं कृताञ्जलयः सर्वे प्रणता वसुधाधिपाः}
{उपाजग्मुर्धनं गृह्य रत्नानि विविधानि च}


\twolineshloka
{मणिमुक्ताप्रवालं च सुवर्णं रजतं बहु}
{गोरत्नान्यश्वरत्नानि रथरत्नानि कुञ्जरान्}


\threelineshloka
{खरोष्ट्रमहिषांश्चैव यच्च किंचिदजाविकम्}
{कम्लाजिनरत्नानि राङ्कवास्तरणानि च}
{तत्सर्वं प्रतिजग्राह राजा नागपुराधिपः}


\twolineshloka
{तदादाय ययौ पाण्डुः पुनर्मदितवाहनः}
{हर्षयिष्यन्स्वराष्ट्राणि पुरं च गजसाह्वयम्}


\twolineshloka
{शान्तनो राजसिंहस्य भरतस्य च धीमतः}
{प्रनष्टः कीर्तिजः शब्दः पाण्डुना पुनराहृतः}


\twolineshloka
{ये पुरा कुरुराष्ट्राणि जह्रुः कुरुधनानि च}
{ते नागपुरसिंहेन पाण्डुना करदीकृताः}


\twolineshloka
{इत्यभाषन्त राजानो राजामात्याश्च सङ्गताः}
{प्रतीतमनसो हृष्टाः पौरजानपदैः सह}


\twolineshloka
{प्रत्युद्ययुश्च तं प्राप्तं सर्वे भीष्मपुरोगमाः}
{ते नदूरमिवाध्वानं गत्वा नागपुरालयाः}


\twolineshloka
{आवृतं ददृशुर्हृष्टा लोकं बहुविधैर्धनैः}
{नानायानसमानीतै रत्नैरुच्चावचैस्तदा}


\twolineshloka
{हस्त्यश्वरथरत्नैश्च गोभिरुष्ट्रैस्तथाऽऽविभिः}
{नान्तं ददृशुरासाद्य भीष्मेण सह कौरवाः}


\twolineshloka
{सोऽभिवाद्य पितुः पादौ कौसल्यानन्दवर्धनः}
{यथाऽर्हं मानयामास पौरजानपदानपि}


\twolineshloka
{प्रमृद्य परराष्ट्राणि कृतार्थं पुनरागतम्}
{पुत्रमाश्लिष्य भीष्मस्तु हर्षादश्रूण्यवर्तयत्}


\twolineshloka
{स तूर्यशतशङ्खानां भेरीणां च महास्वनैः}
{हर्षयन्सर्वशः पौरान्विवेश गजसाह्वयम्}


\chapter{अध्यायः १२३}
\twolineshloka
{वैशंपायन उवाच}
{}


\twolineshloka
{धृतराष्ट्राभ्यनुज्ञातः स्वबाहुविजितं धनम्}
{भीष्माय सत्यवत्यै च मात्रे चोपजहार सः}


\twolineshloka
{विदुराय च वै पाण्डुः प्रेषयामास तद्धनम्}
{सुहृदश्चापि धर्मात्मा धनेन समतर्पयत्}


\twolineshloka
{ततः सत्यवतीं भीष्मं कौसल्यां च यशस्विनीम्}
{शुभैः पाण्डुर्जितैरर्थैस्तोषयामास भारत}


\twolineshloka
{ननन्द माता कौसल्या तमप्रतिमतेजसम्}
{जयन्तमिव पौलोमी परिष्वज्य नर्षभम्}


\twolineshloka
{तस्य वीरस्य विक्रान्तैः सहस्रशतदक्षिणैः}
{अश्वमेधशतैरीजे धृतराष्ट्रो महामखैः}


\twolineshloka
{संप्रयुक्तस्तु कुन्त्या च माद्र्या च भरतर्षभ}
{जिततन्द्रिस्तदा पाण्डुर्बभूव वनगोचरः}


\twolineshloka
{हित्वा प्रासादनिलयं शुभानि शयनानि च}
{अरण्यनित्यः सततं बभूव मृगयापरः}


\twolineshloka
{स चरन्दक्षिणं पार्श्वं रम्यं हिमवतो गिरः}
{उवास गिरिपृष्ठेषु महाशालवनेषु च}


\twolineshloka
{रराज कुन्त्या माद्र्या च पाण्डुः सह वने चरन्}
{करेण्वोरिव मध्यस्थः श्रीमान्पौरन्दरो गजः}


\threelineshloka
{भारतं सह भार्याभ्यां खङ्गबाणधनुर्धरम्}
{विचित्रकवचं वीरं परमास्त्रविदं नृपम्}
{देवोऽयमित्यमन्यन्त चरन्तं वनवासिनः}


\twolineshloka
{तस्य कामांश्च भोगांश्च नरा नित्यमतन्द्रिताः}
{उपजह्रुर्वनान्तेषु धृतराष्ट्रेण चोदिताः}


\twolineshloka
{तदासाद्य महारण्यं मृगव्यालनिषेवितम्}
{तत्र मैथुनकालस्थं ददर्श मृगयूथपम्}


\twolineshloka
{ततस्तं च मृगीं चैव रुक्मपुङ्खैः सुपत्रिभिः}
{निर्बिभेद शरैस्तीक्ष्णैः पाण्डुः पञ्चभिराशुगैः}


\twolineshloka
{स च राजन्महातेजा ऋषिपुत्रस्तपोधनः}
{भार्यया सह तेजस्वी मृगरूपेण सङ्गतः}


\threelineshloka
{स संयुक्तस्तया मृग्या मानुषीं वाचमीरयन्}
{क्षणेन पतितो भूमौ विललापातुरो मृगः ॥मृग उवाच}
{}


\twolineshloka
{काममन्युवशं प्राप्ता बुद्ध्यन्तरगता अपि}
{वर्जयन्ति नृशंसानि पापेष्वपि रता नराः}


\twolineshloka
{न विधिं ग्रसते प्रज्ञा प्रज्ञां तु ग्रसते विधिः}
{विधिपर्यागतानर्थान्प्रज्ञावान्प्रतिपद्यते}


\threelineshloka
{शस्वद्धर्मात्मनां मुख्ये कुले जातस्य भारत}
{कामलोभाभिभूतस्य कथं ते चलिता मतिः ॥पाण्डुरुवाच}
{}


\twolineshloka
{शत्रूणां या वधे वृत्तिः सा मृगाणां वधे स्मृता}
{राज्ञां मृग न मां मोहात्त्वं गर्हयितुमर्हसि}


\twolineshloka
{अच्छद्मनाऽमायया च मृगाणां वध इष्यते}
{स एव धर्मो राज्ञां तु तद्धि त्वं किं नु गर्हसे}


\twolineshloka
{अगस्त्यः सत्रमासीनश्चकार मृगयामृषिः}
{आरण्यान्सर्वदैवत्यान्मृगान्प्रोक्ष्य महावने}


\twolineshloka
{प्रमाणदृष्टधर्मेण कथमस्मान्विगर्हसे}
{अगस्त्यस्याभिचारेण युष्माकं विहितो वधः}


\twolineshloka
{न रिपून्वै समुद्दिश्य विमुञ्चन्ति नराः शरान्}
{रन्ध्र एषां विशेषेण वधकालः प्रशस्यते}


\threelineshloka
{प्रमत्तमप्रमत्तं वा विवृतं घ्नन्ति चौजसा}
{उपायैर्विविधैस्तीक्ष्णैः कस्मान्मृग विगर्हसे ॥मृग उवाच}
{}


\twolineshloka
{नाहं घ्न्तं मृगन्राजन्विगर्हे चात्मकारणात्}
{मैथुनं तु प्रतीक्ष्यं मे त्वयेहाद्यानृशंस्यतः}


\twolineshloka
{सर्वभूतहिते काले सर्वभूतेप्सिते तथा}
{को हि विद्वान्मृगं हन्याच्चरन्तं मैथुनं वने}


\twolineshloka
{अस्यां मृग्यां च राजेन्द्र हर्षान्मैथुनमाचरम्}
{पुरुषार्थफलं कर्तुं तत्त्वया विफलीकृतम्}


\twolineshloka
{पौरवाणां महाराज तेषामक्लिष्टकर्मणाम्}
{वंशे जातस्य कौरव्य नानुरूपमिदं तव}


\twolineshloka
{नृशंसं कर्म सुमहत्सर्वलोकविगर्हितम्}
{अस्वर्ग्यमयशस्यं चाप्यधर्मिष्ठं च भारत}


\twolineshloka
{स्त्रीभोगानां विशेषज्ञः शास्त्रधर्मार्थतत्त्ववित्}
{नार्हस्त्वं सुरसङ्काश कर्तुमस्वर्ग्यमीदृशम्}


\twolineshloka
{त्वया नृशंसकर्तारः पापाचाराश्च मानवाः}
{निग्राह्याः पार्थिवश्रेष्ठ त्रिवर्गपरिवर्जिताः}


\twolineshloka
{किं कृतं ते नरश्रेष्ठ मामिहानागसं घ्नतः}
{मुनिं मूलफलाहारं मृगवेषधरं नृप}


\twolineshloka
{वसमानमरण्येषु नित्यं शमपरायणम्}
{त्वयाऽहं हिंसितो यस्मात्तस्मात्त्वामप्यनागसः}


\twolineshloka
{द्वयोर्नृशंसकर्तारमवशं काममोहितम्}
{जीवितान्तकरो भावो मैथुने समुपैष्यति}


\twolineshloka
{अहं हि किंदमो नाम तपसा भावितो मुनिः}
{व्यपत्रपन्मनुष्याणां मृग्यां मैथुनमाचरम्}


\twolineshloka
{मृगो भत्वा मृगैः सार्धं चरामि गहने वने}
{न तु ते ब्रह्महत्येयं भविष्यत्यविजानतः}


\twolineshloka
{मृगरूपधरं हत्वा मामेवं काममोहितम्}
{अस्य तु त्वं फलं मूढ प्राप्स्यसीदृशमेव हि}


\twolineshloka
{प्रियया सह संवासं प्राप्य कामविमोहितः}
{त्वमप्यस्यामवस्थायां प्रेतलोकं गमिष्यसि}


\threelineshloka
{अन्तकाले हि संवासं यया गन्तासि कान्तया}
{प्रेतराजपुरं प्राप्तं सर्वभूतदुरत्ययम्}
{भक्त्या मतिमतां श्रेष्ठ सैव त्वाऽनुगमिष्यति}


\threelineshloka
{वर्तमानः सुखे दुःखं यथाऽहं प्रापितस्त्वया}
{तथा त्वां च सुखं प्राप्तं दुःखमभ्यागमिष्यति ॥वैशंपायन उवाच}
{}


\twolineshloka
{एवमुक्त्वा सुदुःखार्तो जीवितात्स व्यमुच्यत}
{मृगः पाण्डुश्च दुःखार्तः क्षणेन समपद्यत}


\chapter{अध्यायः १२४}
\twolineshloka
{वैशंपायन उवाच}
{}


\threelineshloka
{तं व्यतीतमुपक्रम्य राजा स्वमिव बान्धवम्}
{सभार्यः शोकदुःखार्तः पर्यदेवयदातुरः ॥पाण्डुरुवाच}
{}


\twolineshloka
{सतामपि कुले जाताः कर्मणा बत दुर्गतिम्}
{प्राप्नुवन्त्यकृतात्मानः कामजालविमोहिताः}


\twolineshloka
{शश्वद्धर्मात्मना जातो बाल एव पिता मम}
{जीवितान्तमनुप्राप्तः कामात्मैवेति नः श्रुतम्}


\twolineshloka
{तस्य कामात्मनः क्षेत्रे राज्ञः संयतवागृषिः}
{कृष्णद्वैपायनः साक्षाद्भगवान्मामजीजनत्}


\twolineshloka
{तस्याद्य व्यसने बुद्धिः संजातेयं ममाधमा}
{त्यक्तस्य देवैरनयान्मृगयां परिधावतः}


\twolineshloka
{मोक्षमेव व्यवस्यामि बन्धो हि व्यसनं महत्}
{सुवृत्तिमनुवर्तिष्ये तामहं पितुरव्ययाम्}


\twolineshloka
{अतीव तपसात्मानं योजयिष्याम्यसंशयम्}
{तस्मादेकाहमेकाहमेकैकस्मिन्वनस्पतौ}


\twolineshloka
{चरन्भैक्षं मुनिर्मुण्डश्चरिष्याम्याश्रमानिमान्}
{पांसुना समवच्छन्नः शून्यागारकृतालयः}


\twolineshloka
{वृक्षमूलनिकेतो वा त्यक्तसर्वप्रियाप्रियः}
{न शोचन्न प्रहृष्यंश्च तुल्यनिन्दात्मसंस्तुतिः}


\twolineshloka
{निराशीर्निर्नमस्कारो निर्द्वन्द्वो निष्परिग्रहः}
{न चाप्यवहसन्कंचिन्न कुर्वन्भ्रुकुटीं क्वचित्}


\twolineshloka
{प्रसन्नवदनो नित्यं सर्वभूतहिते रतः}
{जङ्गमाजङ्गमं सर्वमविहिंसंश्चतुर्विधम्}


\twolineshloka
{स्वासु प्रजास्विव सदा समः प्राणभृतः प्रति}
{एककालं चरन्भैक्षं कुलानि दश पञ्च च}


\twolineshloka
{असंभवे वा भैक्षस्य चरन्ननशनान्यपि}
{अल्पमल्पं च भुञ्जानः पूर्वालाभे न जातुचित्}


\twolineshloka
{अन्यान्यपि चरँल्लोभादलाभे सप्त पूरयन्}
{अलाभे यदि वा लाभे समदर्शी महातपाः}


\twolineshloka
{वास्यैकं तक्षतो बाहुं चन्दनेनैकमुक्षतः}
{नाकल्याणं न कल्याणं चिन्तयन्नुभयोस्तयोः}


\twolineshloka
{न जिजीविषुवत्किंचिन्न मुमूर्षुवदाचरन्}
{जीवितं मरणं चैव नाभिन्दन्न च द्विषन्}


\twolineshloka
{याः काश्चिज्जीवता शक्याः कर्तुमभ्युदयक्रियाः}
{ताः सर्वाः समतिक्रम्य निमेषादिव्यवस्थितः}


\twolineshloka
{तासु चाप्यनवस्थासु त्यक्तसर्वेन्द्रियक्रियः}
{संपरित्यक्तधर्मार्थः सुनिर्णिक्तात्मकल्मषः}


\twolineshloka
{निर्मुक्तः सर्वपापेभ्यो व्यतीतः सर्ववागुराः}
{न वशे कस्यचित्तिष्ठन्सधर्मा मातरिश्वनः}


\twolineshloka
{एतया सततं वृत्त्या चरन्नेवंप्रकारया}
{देहं संस्थापयिष्यामि निर्भयं मार्गमास्थितः}


\twolineshloka
{नाहं सुकृपणे मार्गे स्ववीर्यक्षयशोचिते}
{स्वधर्मात्सततापेते चरेयं वीर्यवर्जितः}


\threelineshloka
{सत्कृतोऽसत्कृतो वाऽपि योऽन्यां कृपणचक्षुषा}
{उपैति वृत्तिं कामात्मा स शुनां वर्तते पथि ॥वैशंपायन उवाच}
{}


\twolineshloka
{एवमुक्त्वा सुदुःखार्तो निःश्वासपरमो नृपः}
{अवेक्षमाणः कुन्तीं च मान्द्रीं च समभाषत}


\twolineshloka
{कौसल्या विदुरः क्षत्ता राजा च सह बन्धुभिः}
{आर्या सत्यवती भीष्मस्ते च राजपुरोहिताः}


\threelineshloka
{ब्राह्मणाश्च महात्मानः सोमपाः संशितव्रताः}
{पौरवृद्धाश्च ये तत्र निवसन्त्यस्मदाश्रयाः}
{प्रसाद्य सर्वे वक्तव्याः पाण्डुः प्रव्राजितो वने}


\twolineshloka
{निशम्य वचनं भर्तुस्त्यागधर्मकृतात्मनः}
{तत्समं वचनं कुन्ती माद्री च समभाषताम्}


\threelineshloka
{अन्येऽपि ह्याश्रमाः सन्ति ये शक्या भरतर्षभ}
{`आवाभ्यां सह वस्तुं वै धर्ममाश्रित्य चिन्त्यताम्}
{'आवाभ्यां धर्मपत्नीभ्यां सह तप्तुं तपो महत्}


\twolineshloka
{शरीरस्यापि मोक्षाय धर्मं प्राप्य महाफलम्}
{त्वमेव भविता भर्ता स्वर्गस्यापि न संशयः}


\twolineshloka
{प्रणिधायेन्द्रियग्रामं भर्तृलोकपरायणे}
{त्यक्त्वा कामसुखे ह्यावां तप्स्यवो विपुलं तपः}


\threelineshloka
{यदि चावां महाप्राज्ञ त्यक्ष्यसि त्वं विशांपते}
{अद्यैवावां प्रहास्यावो जीवितं नात्र संशयः ॥पाण्डुरुवाच}
{}


\twolineshloka
{यदि व्यवसितं ह्येतद्युवयोर्धर्मसंहितम्}
{स्ववृत्तिमनुवर्तिष्ये तामहं पितुरव्ययाम्}


\twolineshloka
{त्यक्त्वा ग्राम्यसुखाहारं तप्यमानो महत्तपः}
{वल्कली फलमूलाशी चरिष्यामि महावने}


\twolineshloka
{अग्नौ जुह्वन्नुभौ कालावुभौ कालावुपस्पृशन्}
{कृशः परिमिताहारश्चीरचर्मजटाधरः}


\twolineshloka
{शीतवातातपसहः क्षुत्पिपासानवेक्षकः}
{तपसा दुश्चरेणेदं शरीरमुपशोषयन्}


\twolineshloka
{एकान्तशीली विमृशन्पक्वापक्वेन वर्तयन्}
{पितृन्देवांश्च वन्येन वाग्भिरद्भिश्च तर्पयन्}


\twolineshloka
{वानप्रस्थजनस्यापि दर्शनं कुलवासिनः}
{नाप्रियाण्याचरिष्यामि किं पुनर्ग्रामवासिनाम्}


\threelineshloka
{एवमारण्यशास्त्राणामुग्रमुग्रतरं विधिम्}
{काङ्क्षमाणोऽहमास्थास्ये देहस्यास्याऽऽसमापनात् ॥वैशंपायन उवाच}
{}


\twolineshloka
{इत्येवमुक्त्वा भार्ये ते राजा कौरवनन्दनः}
{ततश्चूडामणिं निष्कमङ्गदे कुण्डलानि च}


\twolineshloka
{वासांसि च महार्हाणि स्त्रीणामाभरणानि च}
{`वाहनानि च मुख्यानि शस्त्राणि कवचानि च}


\twolineshloka
{हेमभाण्डानि दिव्यानि पर्यङ्कास्तरणानि च}
{मणिमुक्ताप्रवालानि रत्नानि विविधानि च ॥'}


\threelineshloka
{प्रदाय सर्वं विप्रेभ्यः पाण्डुर्भृत्यानभाषत}
{गत्वा नागपुरं वाच्यं पाण्डुः प्रव्राजितो वने}
{अर्थं कामं सुखं चैव रतिं च परमात्मिकाम्}


\twolineshloka
{प्रतस्थे सर्वमुत्सृज्य सभार्यः कुरुनन्दनः}
{ततस्तस्यानुयातारस्ते चैव परिचारकाः}


\twolineshloka
{श्रुत्वा भरतसिंहस्य विधिधाः करुणा गिरः}
{भममार्तस्वरं कृत्वा हाहेति परिचुक्रुशुः}


\twolineshloka
{उष्णमश्रु विमुञ्चन्तस्तं विहाय महीपतिम्}
{ययुर्नागपुरं तूर्णं सर्वमादाय तद्धनम्}


\twolineshloka
{ते गत्वा नगरं राज्ञो यथावृत्तं महात्मनः}
{कथयांचक्रिरे राज्ञस्तद्धनं विविधं ददुः}


\twolineshloka
{श्रुत्वा तेभ्यस्ततः सर्वं यथावृत्तं महावने}
{धृतराष्ट्रो नरश्रेष्ठः पाण्डुमेवान्वशोचत}


\twolineshloka
{न शय्यासनभोगेषु रतिं विन्दति कर्हिचित्}
{भ्रातृशोकसमाविष्टस्तमेवार्थं विचिन्तयन्}


\twolineshloka
{राजपुत्रस्तु कौरव्य पाण्डुर्मूलफलाशनः}
{जगाम सह पत्नीभ्यां ततो नागशतं गिरिम्}


\twolineshloka
{स चैत्ररथमासाद्य कालकूटमतीत्य च}
{हिमवन्तमतिक्रम्य प्रययौ गन्धमादनम्}


\twolineshloka
{रक्ष्यमाणो महाभूतैः सिद्धैश्च परमर्षिभिः}
{उवास स महाराज समेषु विषमेषु च}


\twolineshloka
{इन्द्रद्युम्नसरः प्राप्य हंसकूटमतीत्य च}
{शतशृङ्गे महाराज तापसः समतप्यत}


\chapter{अध्यायः १२५}
\twolineshloka
{वैशंपायन उवाच}
{}


\twolineshloka
{तत्रापि तपसि श्रेष्ठे वर्तमानः स वीर्यवान्}
{सिद्धचारणसङ्घानां बभूव प्रियदर्शनः}


\twolineshloka
{सुश्रूषुरनहंवादी संयतात्मा जितेन्द्रियः}
{स्वर्गं गन्तुं पराक्रान्तः स्वेन वीर्येण भारत}


\twolineshloka
{केषांचिदभवद्भ्राता केषांचिदभवत्सखा}
{ऋषयस्त्वपरे चैनं पुत्रवत्पर्यपालयन्}


\twolineshloka
{स तु कालेन महता प्राप्य निष्कल्मषं तपः}
{ब्रह्मर्षिसदृशः पाण्डुर्बभूव भरतर्षभ}


\twolineshloka
{अमावास्यां तु सहिता ऋषयः संशितव्रताः}
{ब्रह्माणं द्रष्टुकामास्ते संप्रतस्थुर्महर्षयः}


\threelineshloka
{संप्रयातानृषीन्दृष्ट्वा पाण्डुर्वचनमब्रवीत्}
{भवन्तः क्व गमिष्यन्ति ब्रूत मे वदतां वराः ॥ऋषय ऊचुः}
{}


\fourlineindentedshloka
{समावायो महानद्य ब्रह्मलोके महात्मनाम्}
{देवानां च ऋषीणां च पितॄणां च महात्मनाम्}
{वयं तत्र गमिष्यामो द्रष्टुकामाः स्वयंभुवम् ॥वैशंपायन उवाच}
{}


\twolineshloka
{पाण्डुरुत्थाय सहसा गन्तुकामो महर्षिभिः}
{स्वर्गपारं तितीर्षुः स शतशृङ्गादुदङ्मुखः}


\twolineshloka
{प्रतस्थे सह पत्नीभ्यामब्रुवंस्तं च तापसाः}
{उपर्युपरि गच्छन्तः शैलराजमुदङ्मुखाः}


\twolineshloka
{दृष्टवन्तो गिरौ रम्ये दुर्गान्देशान्बहून्वयम्}
{विमानशतसंबाधां गीतस्वरनिनादिताम्}


\twolineshloka
{आक्रीडभूमिं देवानां गन्धर्वाप्सरसां तथा}
{उद्यानानि कुबेरस्य समानि विषमाणि च}


\twolineshloka
{महानदीनितम्बांश्च गहनान्गिरिगह्वरान्}
{सन्ति नित्यहिमा देशा निर्वृक्षमृगपक्षिणः}


\twolineshloka
{सन्ति क्वचिन्महादर्यो दुर्गाः काश्चिद्दुरासदाः}
{नातिक्रामेत पक्षी यान्कुत एवेतरे मृगाः}


\twolineshloka
{वायुरेको हि यात्यत्र सिद्धाश्च परमर्षयः}
{गच्छन्त्यौ शैलराजेऽस्मिन्राजपुत्र्यौ कथं न्विमे}


\threelineshloka
{न सीदेतामदुःखार्हे मा गमो भरतर्षभ}
{पाण्डुरुवाच}
{अप्रजस्य महाभागा न द्वारं परिचक्षते}


\twolineshloka
{स्वर्गे तेनाभितप्तोऽहमप्रजस्तु ब्रवीमि वः}
{सोऽहमुग्रेण तपसा सभार्यस्त्यक्तजीवितः}


\threelineshloka
{अनपत्योऽपि विन्देयं स्वर्गमुग्रेण कर्मणा}
{ऋषय ऊचुः}
{अस्ति वै तव धर्मात्मन्विद्म देवोपम शुभम्}


\twolineshloka
{अपत्यमनघं राजन्वयं दिव्येन चक्षुषा}
{दैवोद्दिष्टं नरव्याघ्र कर्मणेहोपपादय}


\twolineshloka
{अक्लिष्टं फलमव्यग्रो विन्दते बुद्धिमान्नरः}
{तस्मिन्दृष्टे फले राजन्प्रयत्नं कर्तुमर्हसि}


\threelineshloka
{अपत्यं गुणसंपन्नं लब्धा प्रीतिकरं ह्यसि}
{वैशंपायन उवाच}
{तच्छ्रुत्वा तापसवचः पाण्डुस्चिन्तापरोऽभवत्}


% Check verse!
आत्मनो मृगशापेन जानन्नुपहतां क्रियाम्
\chapter{अध्यायः १२६}
\twolineshloka
{वैशंपायन उवाच}
{}


\twolineshloka
{अथ पारसवीं कन्यां देवकस्य महीपतेः}
{रूपयौवनसंपन्नां स सुश्रावापगासुतः}


\twolineshloka
{ततस्तु वरयित्वा तामानीय भरतर्षभः}
{विवाहं कारयामास विदुरस्य महामतेः}


\twolineshloka
{तस्यां चोत्पादयामास विदुरः कुरुनन्दन}
{पुत्रान्विनयसंपन्नानात्मनः सदृशान्गुणैः}


\twolineshloka
{ततः पुत्रशतं जज्ञे गान्धार्या जनमेजय}
{धृतराष्ट्रस्य वैश्यायामेकश्चापि शतात्परः}


\threelineshloka
{पाण्डोः कृन्त्यां च माद्र्यां च पुत्राः पञ्च महारथाः}
{देवेभ्यः समपद्यन्त सन्तानाय कुलस्य वै ॥जनमेजय उवाच}
{}


\twolineshloka
{कथं पुत्रशतं जज्ञे गान्धार्यां द्विजसत्तम}
{कियता चैव कालेन तेषामायुश्च किं परम्}


\twolineshloka
{कथं चैकः स वैश्यायां धृतराष्ट्रसुतोऽभवत्}
{कथं च सदृशीं भार्यां गान्धारीं धर्मचारिणीम्}


\twolineshloka
{आनुकूल्ये वर्तमानां धृतराष्ट्रोऽत्यवर्तत}
{कथं च शप्तस्य सतः पाण्डोस्तेन महात्मना}


\twolineshloka
{समुत्पन्ना दैवतेभ्यः पुत्राः पञ्च महारथाः}
{एतद्विद्वन्यथान्यायं विस्तरेण तपोधन}


\threelineshloka
{कथयस्व न मे तृप्तिः कथ्यमानेषु बन्धुषु}
{वैशंपायन उवाच}
{ऋषिं बुभुक्षितं श्रान्तं द्वैपायनमुपस्थितम्}


\twolineshloka
{तोषयामास गान्धारी व्यासस्तस्यै वरं ददौ}
{सा वव्रे सदृशं भर्तुः पुत्राणां शतमात्मनः}


% Check verse!
ततः कालेन सा गर्भमगृह्णाज्ज्ञानचक्षुषः
\twolineshloka
{गान्धार्यामाहिते गर्भे पाण्डुरम्बालिकासुतः}
{अगच्छत्परमं दुःखमपत्यार्थमरिन्दम}


\twolineshloka
{गर्भिण्यामथ गान्धार्यां पाण्डुः परमदुःखितः}
{मृगाभिशापादात्मानं शोचन्नुपरतक्रियः}


\threelineshloka
{स गत्वा तपसा सिद्धिं विश्वामित्रो यथा भुवि}
{देहान्यासे कृतमना इदं वचनमब्रवीत् ॥पाण्डुरुवाच}
{}


\twolineshloka
{चतुर्भिर्ऋणवानित्थं जायते मनुजो भुवि}
{पितृदेवमनुष्याणामृषीणामथ भामिनि}


\twolineshloka
{एतेभ्यस्तु यथाकालं यो न मुच्येत धर्मवित्}
{न तस्य लोकाः सन्तीति तता लोकविदो विदुः}


\twolineshloka
{यज्ञेन देवान्प्रीणाति स्वाध्यायात्तपसा ऋषीन्}
{पुत्रैः श्राद्धैरपि पितॄनानृशंस्येन मानवान्}


\twolineshloka
{ऋषिदेवमनुष्याणामृणान्मुक्तोऽस्मि धर्मतः}
{पितॄणां तु न मुक्तोऽस्मि तच्च तेभ्यो विशिष्यते}


\twolineshloka
{देहनाशे भवेन्नाशः पितॄणामेष निश्चयः}
{इतरेषां त्रयाणां तु नाशे ह्यात्मा विनश्यति}


\twolineshloka
{इह तस्मात्प्रजालाभे प्रयतन्ते द्विजोत्तमाः}
{यथैवाहं पितुः क्षेत्रे सृष्टस्तेन महात्मना}


\threelineshloka
{तथैवास्मिन्मम क्षेत्रे कथं सृज्येत वै प्रजा}
{वैशंपायन उवाच}
{स समानीय कुन्तीं च माद्रीं च भरतर्षभः}


\twolineshloka
{आचष्ट पुत्रलाभस्य व्युष्टिं सर्वक्रियाधिकाम्}
{उत्तमादवराः पुंसः काङ्क्षन्तो पुत्रमापदि}


\threelineshloka
{अपत्यं धर्मफलदं श्रेष्ठादिच्छन्ति साधवः}
{अनुनीय तु ते सम्यङ्महाब्राह्मणसंसदि}
{ब्राह्मणं गुणवन्तं हि चिन्तयामास धर्मवित्}


\twolineshloka
{सोऽब्रवीद्विजने कुन्तीं धर्मपत्नीं यशस्विनीम्}
{अपत्योत्पादने यत्नमापदि त्वं समर्थय}


\twolineshloka
{अपत्यं नाम लोकेषु प्रतिष्ठा धर्मसंहिता}
{इति कुन्ति विदुर्धीराः शाश्वतं धर्मवादिनः}


\twolineshloka
{इष्टं दत्तं तपस्तप्तं नियमश्च स्वनुष्ठितः}
{सर्वमेवानपत्यस्य न पावनमिहोच्यते}


\twolineshloka
{सोऽहमेवं विदित्वैतत्प्रपश्यामि शुचिस्मिते}
{अनपत्यः शुभाँल्लोकान्नप्राप्स्यामीति चिन्तयन्}


\threelineshloka
{`अनपत्यो हि मरणं कामये नैव जीवितम्}
{'मृगाभिशापं जानासि विजने मम केवलम्}
{नृशंसकर्मणा कृत्स्नं यथा ह्युपहतं तथा}


\twolineshloka
{इमे वै बन्धुदायादाः षट् पुत्रा धर्मदर्शने}
{षडेवाबन्धुदायादाः पुत्रास्ताञ्छृणु मे पृथे}


\twolineshloka
{स्वयंजातः प्रणीतश्च परिक्रीतश्च यः सुतः}
{पौनर्भवश्च कानीनः स्वैरिण्यां यश्च जायते}


\twolineshloka
{दत्तः क्रीतः कृत्रिमश्च उपगच्छेत्स्वयं च यः}
{सहोढो ज्ञातिरेताश्च हीनयोनिधृतश्च यः}


\twolineshloka
{पूर्वपूर्वतमाभावं मत्त्वा लिप्सेत वै सुतम्}
{उत्तमाद्देवरात्पुंसः काङ्क्षन्ते पुत्रमापदि}


\twolineshloka
{अपत्यं धर्मफलदं श्रेष्ठं विन्दन्ति मानवाः}
{आत्मशुक्रादपि पृथे मनुः स्वायंभुवोऽब्रवीत्}


% Check verse!
तस्मात्प्रहेष्याम्यद्य त्वां हीनः प्रजननात्स्वयम्
\twolineshloka
{सदृशाच्छ्रेयसो वा त्वं विद्ध्यपत्यं यशस्विनि}
{शृणु कुन्ति कथामेतां शारदण्डायिनीं प्रति}


\twolineshloka
{`या हि ते भगिनी साध्वी श्रुतसेना यशस्विनी}
{अवाह तां तु कैकेयः शारदाण्डायनिर्महान् ॥'}


\twolineshloka
{सा वीरपत्नी गुरुणा नियुक्ता पुत्रजन्मनि}
{पुष्पेण प्रयता स्नाता निशि कुन्ति चतुष्पथे}


\twolineshloka
{वरयित्वा द्विजं सिद्धं हुत्वा पुंसवनेऽनलम्}
{कर्मण्यवसिते तस्मिन्सा तेनैव सहावसत्}


\threelineshloka
{तत्र त्रीञ्जनयामास दुर्जयादीन्महारथान्}
{तथा त्वमपि कल्याणि ब्राह्मणात्तपसाधिकात्}
{मन्नियोगाद्यत क्षिप्रमपत्योत्पादनं प्रति}


\chapter{अध्यायः १२७}
\twolineshloka
{वैशंपायन उवाच}
{}


\threelineshloka
{एवमुक्ता महाराज कुन्ती पाण्डुमभाषत}
{कुरूणामृषभं वीरं तदा भूमिपतिं पतिम् ॥कुन्त्युवाच}
{}


\twolineshloka
{न मामर्हसि धर्मज्ञ वक्तुमेवं कथंचन}
{धर्मपत्नीमभिरतां त्वयि राजीवलोचने}


\twolineshloka
{त्वमेव तु महाबाहो मय्यपत्यानि भारत}
{वीर वीर्योपपन्नानि धर्मतो जनयिष्यसि}


\twolineshloka
{स्वर्गं मनुजशार्दूल गच्छेयं सहिता त्वया}
{अपत्याय च मां गच्छ त्वमेव कुरुनन्दन}


\twolineshloka
{न ह्यहं मनसाप्यन्यं गच्छेयं त्वदृते नरम्}
{त्वत्तः प्रति विशिष्टश्च कोऽन्योऽस्ति भुवि मानवः}


\twolineshloka
{इमां च तावद्धर्मात्मन्पौराणीं शृणु मे कथाम्}
{परिश्रुतां विशालाक्ष कीर्तयिष्यामि यामहम्}


\twolineshloka
{व्युषइताश्व इति ख्यातो बभूव किल पार्थिवः}
{पुरा परमधर्मिष्ठः पूरोर्वंशविवर्धनः}


\twolineshloka
{तस्मिंश्च यजमाने वै धर्मात्मनि महाभुजे}
{उपागमंस्ततो देवाः सेन्द्रा देवर्षिभिः सह}


\twolineshloka
{अमाद्यदिन्द्रः सोमेन दक्षिणाभिर्द्विजातयः}
{व्युषिताश्वस्य राजर्षेस्ततो यज्ञे महात्मनः}


\twolineshloka
{देवा ब्रह्मर्षयश्चैव चक्रुः कर्म स्वयं तदा}
{व्युषिताश्वस्ततो राजन्नति मर्त्यान्व्यरोचत}


\twolineshloka
{सर्वभूतान्प्रति यथा तपनः शिशिरात्यये}
{स विजित्य गृहीत्वा च नृपतीन्राजसत्तमः}


\twolineshloka
{प्राच्यानुदिच्यान्पाश्चात्यान्दाक्षिणात्यानकालयत्}
{अश्वमेधे महायज्ञे व्युषिताश्वः प्रतापवान्}


\twolineshloka
{बभूव स हि राजेन्द्रो दशनागबलान्वितः}
{अप्यत्र गाथां गायन्ति ये पुराणविदो जनाः}


\twolineshloka
{व्युषिताश्वे यशोवृद्धे मनुष्येन्द्रे कुरूत्तम}
{व्युषिताश्वः समुद्रान्तां विजित्येमां वसुन्धराम्}


\twolineshloka
{अपालयत्सर्ववर्णान्पिता पुत्रानिवौरसान्}
{यजमानो महायज्ञैर्ब्राह्मणेभ्यो धनं ददौ}


\twolineshloka
{अनन्तरत्नान्यादाय स जहार महाक्रतून्}
{सुषाव च बहून्सोमान्सोमसंस्थास्ततान च}


\twolineshloka
{आसीत्काक्षीवती चास्य भार्या परमसंमता}
{भद्रा नाम मनुष्येन्द्र रूपेणासदृशी भुवि}


\twolineshloka
{कामयामासतुस्तौ च परस्परमिति श्रुतम्}
{स तस्यां कामसंपन्नो यक्ष्मणा समपद्यत}


\twolineshloka
{तेनाचिरेण कालेन जगामास्तमिवांशुमान्}
{तस्मिन्प्रेते मनुष्येन्द्रे भार्याऽस्य भृशदुःखिता}


\threelineshloka
{अपुत्रा पुरुषव्याघ्र विललापेति नः श्रुतम्}
{भद्रा परमदुःखार्ता तन्निबोध जनाधिप ॥भद्रोवाच}
{}


\twolineshloka
{नारी परमधर्मज्ञ सर्वा भर्तृविनाकृता}
{पतिं विना जीवति या न सा जीवति दुःखिता}


% Check verse!
पतिं विना मृतं श्रेयो नार्याः क्षत्रियपुंगव ॥त्वद्गतिं गन्तुमिच्छामि प्रसीदस्वनयस्वमाम्
\twolineshloka
{त्वया हीना क्षणमपि नाहं जीवितुमुत्सहे}
{प्रसादं कुरु मे राजन्नितस्तूर्णं नयस्व माम्}


\twolineshloka
{पृष्ठतोऽनुगमिष्यामि समेषु विषमेषु च}
{त्वामहं नरशार्दूल गच्छन्तमनिवर्तितुम्}


\twolineshloka
{छायेवानुगता राजन्सततं वशवर्तिनी}
{भविष्यामि नरव्याघ्र नित्यं प्रियहिते रता}


\twolineshloka
{अद्यप्रभृति मां राजन्कष्टा हृदयशोषणाः}
{आधयोऽभिभविष्यन्ति त्वामृते पुष्करेक्षण}


\twolineshloka
{अभाग्यया मया नूनं वियुक्ताः सहचारिणः}
{तेन मे विप्रयोगोऽयमुपपन्नस्त्वया सह}


\twolineshloka
{विप्रयुक्ता तु या पत्या मुहूर्तमपि जीवति}
{दुःखं जीवति सा पापा नरकस्थेव पार्थिव}


\twolineshloka
{संयुक्ता विप्रयुक्ताश्च पूर्वदेहे कृता मया}
{तदिदं कर्मभिः पापैः पूर्वदेहेषु संचितम्}


\threelineshloka
{दुःखं मामनुसंप्राप्तं राजंस्त्वद्विप्रयोगजम्}
{अद्यप्रभृत्यहं राजन्कुशसंस्तरशायिनी}
{भविष्याम्यसुखाविष्टा त्वद्दर्शनपरायणा}


\threelineshloka
{दर्शयस्व नरव्याघ्र शाधि मामसुखान्विताम्}
{कृपणां चाथ करुणं विलपत्नीं नरेश्वर ॥कन्त्युवाच}
{}


\twolineshloka
{एवं बहुविधं तस्यां विलपन्त्यां पुनःपुनः}
{तं शवं संपरिष्वज्य वाक्किलाऽन्तर्हिताऽब्रवीत्}


\twolineshloka
{उत्तिष्ठ भद्रे गच्छ त्वं ददानीह वरं तव}
{जनयिष्याम्यपत्यानि त्वय्यहं चारुहासिनि}


\twolineshloka
{आत्मकीये वरारोहे शयनीये चतुर्दशीम्}
{अष्टमीं वा ऋतुस्नाता संविशेथा मया सह}


\twolineshloka
{एवमुक्ता तु सा देवी तथा चक्रे पतिव्रता}
{यथोक्तमेव तद्वाक्यं भद्रा पुत्रार्थिनी तदा}


\twolineshloka
{सा तेन सुषुवे देवी शवेन भरतर्षभ}
{त्रीञ्शाल्वांश्चतुरो मद्रान्सुतान्भरतसत्तम}


\twolineshloka
{तथा त्वमपि मय्येवं मनसा भरतर्षभ}
{शक्तो जनयितुं पुत्रांस्तपोयोगबलान्वितः}


\chapter{अध्यायः १२८}
\twolineshloka
{वैशंपायन उवाच}
{}


\threelineshloka
{एवमुक्तस्तया राजा तां देवीं पुनरब्रवीत्}
{धर्मविद्धर्मसंयुक्तमिदं वचनमुत्तमम् ॥पाण्डुरुवाच}
{}


\twolineshloka
{एवमेतत्पुरा कुन्ति व्युषिताश्वश्चकार ह}
{यथा त्वयोक्तं कल्याणि स ह्यासीदमरोपमः}


\twolineshloka
{अथ त्विदं प्रवक्ष्यामि धर्मतत्त्वं निबोध मे}
{पुराणमृषिभिर्दृष्टं धर्मविद्भिर्महात्मभिः}


\twolineshloka
{अनावृताः किल पुरा स्त्रिय आसन्वरानने}
{कामचारविहारिण्यः स्वतन्त्राश्चारुहासिनि}


\twolineshloka
{तासां व्युच्चरमाणानां कौमारात्सुभगे पतीन्}
{नाधर्मोऽभूद्वरारोहे स हि धर्मः पुराऽभवत्}


\twolineshloka
{तं चैव धर्मं पौराणं तिर्यग्योनिगताः प्रजाः}
{अद्याप्यनुविधीयन्ते कामक्रोधविवर्जिताः}


\threelineshloka
{प्रमाणदृष्टो धर्मोऽयं पूज्यते च महर्षिभिः}
{उत्तरेषु च रम्भोरु कुरुष्वद्यापि पूज्यते}
{स्त्रीणामनुग्रहकरः स हि धर्मः सनातनः}


\twolineshloka
{`नाग्निस्तृप्यति काष्ठानां नापगानां महोदधिः}
{नान्तकः सर्वभूतानां न पुंसां वामलोचनाः}


\twolineshloka
{एवं तृष्णा तु नारीणां पुरुषं पुरुषं प्रति}
{अगम्यागमनं स्त्रीणां नास्ति नित्यं शुचिस्मिते}


\twolineshloka
{पुत्रं वा किल पौत्रं वा कासांचिद्धातरं तथा}
{रहसीह नरं दृष्ट्वा योनिरुत्क्लिद्यते तदा}


\twolineshloka
{एतत्स्वाभाविकं स्त्रीणां न निमित्तकृतं शुभे}
{'अस्मिंस्तु लोके नचिरान्मर्यादेयं शुचिस्मिते}


\twolineshloka
{स्थापिता येन यस्माच्च तन्मे विस्तरतः शृणु}
{बभूवोद्दालको नाम महर्षिरिति नः श्रुतम्}


\twolineshloka
{श्वेतकेतुरिति ख्यातः पुत्रस्तस्याभवन्मुनिः}
{मर्यादेयं कृता तेन धर्म्या वै श्वेतकेतुना}


% Check verse!
कोपात्कमलपत्राक्षि यदर्थं तन्निबोध मे
\twolineshloka
{`श्वेतकेतोः पिता देवि तप उग्रं समास्थितः}
{ग्रीष्मे पञ्चतपा भूत्वा वर्षास्वाकाशगोऽभवत्}


\twolineshloka
{शिशइरे सलिलस्थायी सह पत्न्या महातपाः}
{उद्दालकं तपस्यन्तं नियमेन समाहितम्}


\twolineshloka
{तस्य पुत्रः श्वेतकेतुः परिचर्यां चकार ह}
{अभ्यागच्छद्द्विजः कश्चिद्वलीपलितसंततः}


\twolineshloka
{तं दृष्ट्वैव मुनिः प्रीतः पूजयामास शास्त्रतः}
{स्वागतेन च पाद्येन मृदुवाक्यैश्च भारत}


\twolineshloka
{शाकमूलफलाद्यैश्च वन्यैरन्यैरपूजयत्}
{क्षुत्पिपासाश्रमेंणार्तः पूजितश्च महर्षिणा}


\twolineshloka
{विश्रान्तो मुनिमासाद्य पर्यपृच्छद्द्विजस्तदा}
{उद्दालक महर्षे त्वं सत्यं मे ब्रूहि माऽनृतम्}


\threelineshloka
{ऋषिपुत्रः कुमारोऽयं दर्शनीयो विशेषतः}
{तव पुत्रमिमं मन्ये कृतकृत्योऽसि तद्वद ॥उद्दालक उवाच}
{}


\twolineshloka
{मम पत्नी महाप्राज्ञ कुशिकस्य सुता मता}
{मामेवानुगता पत्नी मम नित्यमनुव्रता}


\twolineshloka
{अरुन्धतीव पत्नीनां तपसा कर्शितस्तनी}
{अस्यां जातः श्वेतकेतुर्मम पुत्रो महातपाः}


\threelineshloka
{वेदवेदाङ्गविद्विप्र मच्छासनपरायणः}
{लोकज्ञः सर्वलोकेषु विश्रुतः सत्यवाग्घृणी ॥ब्राह्मण उवाच}
{}


\twolineshloka
{अपुत्री भार्यया चार्थी वृद्धोऽहं मन्दचाक्षुषः}
{पित्र्यादृणादनिर्मुक्तः पूर्वमेवाकृतस्त्रियः}


\threelineshloka
{प्रजारणिस्तु पत्नी ते कुलशीलसमन्विता}
{सदृशी मम गोत्रेण वहाम्येनां क्षमस्व मे ॥पाण्डुरुवाच}
{}


\twolineshloka
{इत्युक्त्वा मृगशावाक्षीं चीरकृष्णाजिनाम्बराम्}
{यष्ट्याधारः स्रस्तगात्रो मन्दचक्षुरबुद्धिमान्}


\twolineshloka
{स्वव्यापाराक्षमां श्रेष्ठमचित्तामात्मनि द्विजः}
{'श्वेतकेतोः किल पुरा समक्षं मातरं पितुः}


\twolineshloka
{जग्राह ब्राह्मणः पापौ गच्छाव इति चाब्रवीत्}
{ऋषिपुत्रस्तदा कोपं चकारामर्षितस्तदा}


\twolineshloka
{मातरं तां तथा दृष्ट्वा नीयमानां बलादिव}
{`तपसा दीप्तवीर्यो हि श्वेतकेतुर्न चक्षमे}


\twolineshloka
{संगृह्य मातरं हस्ते श्वेतकेतुरभाषत}
{दुर्ब्राह्मण विमुञ्च त्वं मातरं मे पतिव्रताम्}


\twolineshloka
{स्वयं पिता मे ब्रह्मर्षिः क्षमावान्ब्रह्मवित्तमः}
{शापानुग्रहयोः शक्तः तूष्णींभूतो महाव्रतः}


\twolineshloka
{तस्य पत्नी दमोपेता मम माता विशेषतः}
{पतिव्रतां तपोवृद्धां साध्वाचारैरलङ्कृताम्}


% Check verse!
अप्रमादेन ते ब्रह्मन्मातृभूतां विमुञ्च वै
\threelineshloka
{एवमुक्त्वा तु याचन्तं विमुञ्चेति मुहुर्मुहुः}
{प्रत्यवोचद्द्विजो राजन्नप्रगल्भमिदं वचः ॥ब्राह्मण उवाच}
{}


\twolineshloka
{अपत्यार्थी श्वेतकेतो वृद्धोऽहं मन्दचाक्षुषः}
{पिता ते ऋणनिर्मुक्तस्त्वया पुत्रेण काश्यप}


\twolineshloka
{ऋणादहमनिर्मुक्तो वृद्धोऽहं विगतस्पृहः}
{मम को दास्यति सुतां कन्यां संप्राप्तयौवनाम्}


\twolineshloka
{प्रजारणिमिमां पत्नीं विमुञ्च त्वं महातपः}
{एकया प्रजया प्रीतो मातरं ते ददाम्यहम्}


\twolineshloka
{एवमुक्तः श्वेतकेतुर्लज्जया क्रोधमेयिवान्}
{'क्रुद्धं तं तु पिता दृष्ट्वा श्वेतकेतुमुवाच ह}


\twolineshloka
{मा तात कोपं कार्षीस्त्वमेष धर्मः सनातनः}
{अनावृता हि सर्वेषां वर्णानामङ्गना भुवि}


\twolineshloka
{यथा गावः स्थिताः पुत्र स्वेस्वे वर्णे तथा प्रजाः}
{`तथैव च कुटुम्बेषु न प्रमाद्यन्ति कर्हिचित्}


\twolineshloka
{ऋतुकाले तु संप्राप्ते भर्तारं न जहुस्तदा}
{'ऋषिपुत्रोऽथ तं धर्मं श्वेतकेतुर्न चक्षमे}


\twolineshloka
{चकार चैव मर्यादामिमां स्त्रीपुंसयोर्भुवि}
{मानुषेषु महाभागे न त्वेवान्येषु जन्तुषु}


\twolineshloka
{तदाप्रभृति मर्यादा स्थितेयमिति नः श्रुतम्}
{व्युच्चरन्त्याः पतिं नार्या अद्यप्रभृति पातकम्}


\twolineshloka
{भ्रूणहत्यासमं घोरं भविष्यत्यसुखावहम्}
{`अद्याप्यनुविधीयन्ते कामक्रोधविवर्जिताः}


\twolineshloka
{उत्तरेषु महाभागे कुरुष्वेवं यशस्विनि}
{पुराणदृष्टो धर्मोऽयं पूज्यते च महर्षिभिः ॥'}


\twolineshloka
{भार्यां तथा व्युच्चरतः कौमारब्रह्मचारिणीम्}
{पतिव्रतामेतदेव भविता पातकं भुवि}


\threelineshloka
{नियुक्ता पतिना भार्या यद्यपत्यस्य कारणात्}
{न कुर्यात्तत्तथा भीरु सैनः सुमहदाप्नुयात्}
{इति तेन पुरा भीरु मर्यादा स्थापिता बलात्}


\twolineshloka
{उद्दालकस्य पुत्रेण धर्म्या वै श्वेतकेतुना}
{सौदासेन च रम्भोरु नियुक्ता पुत्रजन्मनि}


\twolineshloka
{मदयन्ती जगामर्षिं वसिष्ठमिति नः श्रुतम्}
{तस्माल्लेभे च सा पुत्रमश्मकं नाम भामिनी}


\twolineshloka
{एवं कृतवती सापि भर्तुः प्रियचिकीर्षया}
{अस्माकमपि ते जन्म विदितं कमलेक्षणे}


\twolineshloka
{कृष्णद्वैपायनाद्भीरु कुरूणां वंशवृद्धये}
{अत एतानि सर्वाणि कारणानि समीक्ष्य वै}


\twolineshloka
{ममैतद्वचनं धर्म्यं कर्तुमर्हस्यनिन्दिते}
{ऋतावृतौ राजपुत्रि स्त्रिया भर्ता पतिव्रते}


\twolineshloka
{नातिवर्तव्य इत्येवं धर्मं धर्मविदो विदुः}
{शेषेष्वन्येषु कालेषु स्वातन्त्र्यं स्त्री किलार्हति}


\twolineshloka
{धर्ममेवं जनाः सन्तः पुराणं परिचक्षते}
{भर्ता भार्यां राजपुत्रि धर्म्यं वाऽधर्म्यमेव वा}


\twolineshloka
{यद्ब्रूयात्तत्तथा कार्यमिति वेदविदो विदुः}
{विशेषतः पुत्रगृद्धी हीनः प्रजननात्स्वयम्}


\twolineshloka
{यथाऽहमनवद्याङ्गि पुत्रदर्शनलालसः}
{अयं रक्ताङ्गुलिनखः पद्मपत्रनिभः शुभे}


\twolineshloka
{प्रसादनार्थं सुश्रोणि शिरस्यभ्युद्यतोऽञ्जलिः}
{मन्नियोगात्सुकेशान्ते द्विजातेस्तपसाऽधिकात्}


\threelineshloka
{पुत्रान्गुणसमायुक्तानुत्पादयितुमर्हसि}
{त्वत्कृतेऽहं पृथुश्रोणि गच्छेयं पुत्रिणां गतिम् ॥वैशंपायन उवाच}
{}


\twolineshloka
{एवमुक्ता ततः कुन्ती पाण्डुं परपुरञ्जयम्}
{प्रत्युवाच वरारोहा भर्तुः प्रियहिते रता}


\threelineshloka
{`अधर्मः सुमहानेषु स्त्रीणां भरतसत्तम}
{यत्प्रसादयते भर्ता प्रसाद्यः क्षत्रियर्षभ}
{शृणु चेदं महाबाहो मम प्रीतिकरं चः ॥'}


\twolineshloka
{पितृवेश्मन्यहं बाला नियुक्ताऽतिथिपूजने}
{उग्रं पर्यचरं तत्र ब्राह्मणं संशितव्रतम्}


\twolineshloka
{निगूढनिश्चयं धर्मे यं तं दुर्वाससं विदुः}
{तमहं संशितात्मानं सर्वयत्नैरतोषयम्}


\twolineshloka
{स मेऽभिचारसंयुक्तमाचष्ट भगवान्वरम्}
{मन्त्रं त्विमं च मे प्रादादब्रवीच्चैव मामिदम्}


\twolineshloka
{यं यं देवं त्वमेतेन मन्त्रेणावाहयिष्यसि}
{अकामो वा सकामो वा वशं ते समुपैष्यति}


\twolineshloka
{तस्य तस्य प्रसादात्ते राज्ञि पुत्रो भविष्यति}
{इत्युक्ताऽहं तदा तेन पितृवेश्मनि भारत}


\twolineshloka
{ब्राह्मणस्य वचस्तथ्यं तस्य कालोऽयमागतः}
{अनुज्ञाता त्वया देवमाह्वयेयमहं नृप}


\threelineshloka
{`यां मे विद्यां महाराज अददात्स महायशाः}
{तयाऽऽहूतः सुरः पुत्रं प्रदास्यति सुरोपमम्}
{अनपत्यकृतं यस्ते शोकं वीर विनेष्यति ॥'}


\chapter{अध्यायः १२९}
\twolineshloka
{`कुन्त्युवाच}
{}


\twolineshloka
{अपत्यकाम एवं स्यान्ममापत्यं भवेदिति}
{विप्रं वा गुणसंपन्नं सर्वभूतहिते रतम्}


\twolineshloka
{अनुजानीहि भद्रं ते दैवतं हि पतिः स्त्रियः}
{यं त्वं वक्ष्यसि धर्मज्ञ देवं ब्राह्मणमेव च}


\twolineshloka
{यथोद्दिष्टं त्वया वीर तत्कर्तास्मि महाभुज}
{देवात्पुत्रफलं सद्यो विप्रात्कालान्तरे भवेत्}


\threelineshloka
{आवाहयामि कं देवं कदा वा भरतर्षभ}
{त्वत्त आज्ञां प्रतीक्षन्तीं विद्ध्यस्मिन्कर्मणीप्सिते ॥पाण्डुरुवाच}
{}


\twolineshloka
{धन्योऽञस्म्यनुगृहीतोऽस्मि त्वं नो धात्री कुलस्य हि}
{नमो महर्षये तस्मै येन दत्तो वरस्तव}


\twolineshloka
{न चाधर्मेण धर्मज्ञे शक्याः पालयितुं प्रजाः}
{तस्मात्त्वं पुत्रलाभाय सन्तानाय ममैव च}


% Check verse!
प्रवरं सर्वदेवानां धर्ममावाहयाबले

वैशंपायन उवाच

पाण्डुना समनुज्ञाता भारतेन यशस्विना

मतिं चक्रे महाराज धर्मस्यावाहने तदा ॥' पाण्डुरुवाच


\twolineshloka
{अद्यैव त्वं वरारोहे प्रयतस्व यथाविधि}
{धार्मिकश्च कुरूणां हि भविष्यति न संशयः}


\twolineshloka
{दत्तस्य तस्य धर्मेण नाधर्मे रंस्यते मनः}
{धर्मादिकं हि धर्मज्ञे धर्मान्तं धर्ममध्यमम्}


\twolineshloka
{अपत्यमिष्टं लोकेषु यशःकीर्तिविवर्धनम्}
{तस्माद्धर्मं पुरस्कृत्य नियता त्वं शुचिस्मिते}


\twolineshloka
{आकाराचारसंपन्ना भजस्वाराधय स्वयम् ॥वैशंपायन उवाच}
{}


\twolineshloka
{सा तथोक्ता तथेत्युक्त्वा तेन भर्त्रा वराङ्गना}
{अभिवाद्याभ्यनुज्ञाता प्रदक्षिणमथाकरोत्}


\twolineshloka
{संवत्सरोषिते गर्भे गान्धार्या जनमेजय}
{आजुबहाव ततो धर्मं कुन्ती गर्भार्थमच्युतम्}


\twolineshloka
{सा बलिं त्वरिता देवी धर्मायोपजहार ह}
{जजाप विधिवज्जप्यं दत्तं दुर्वाससा पुरा}


\twolineshloka
{`जानन्ती धर्ममग्र्यं वै धर्मं वशमुपानयत्}
{आहूतो नियमात्कुन्त्या सर्वभूतनमस्कृतः ॥'}


\twolineshloka
{आजगाम ततो देवीं धर्मो मन्त्रबलात्ततः}
{विमाने सूर्यसङ्काशे कुन्ती यत्र जपस्थिता}


\threelineshloka
{`ददृशे भगवान्धर्मः सन्तानार्थाय पाण्डवे}
{'विहस्य तां ततो ब्रूयाः कुन्ति किं ते ददाम्यहम्}
{सा तं विहस्यमानापि पुत्रं देह्यब्रवीदिदम्}


\twolineshloka
{`तस्मिन्बहुमृगेऽरण्ये शतशृङ्गे नगोत्तमे}
{पाण्डोरर्थे महाभागा कुन्ती धर्ममुपागमत्}


\twolineshloka
{ऋतुकाले शुचिः स्नाता शुक्लवस्त्रा यशस्विनी}
{शय्यां जग्राह सुश्रोणी सह धर्मेण सुव्रता ॥'}


\twolineshloka
{धर्मेण सह संगम्य योगमूर्तिधरेण सा}
{लेभे पुत्रं महाबाहुं सर्वप्राणभृतां वरम्}


\twolineshloka
{ऐन्द्रे चन्द्रमसा युक्ते मुहूर्तेऽभिजितेऽष्टमे}
{दिवा मध्यगते सूर्ये तिथौ पूर्णे हि पूजिते}


\twolineshloka
{समृद्धयसशं कुन्ती सुषाव प्रवरं सुतम्}
{जातमात्रे सुते तस्मिन्वागुवाचाशरीरिणी}


\twolineshloka
{एष धर्मभृतां श्रेष्ठो भविष्यति नरोत्तमः}
{विक्रान्तः सत्यवाक्चैव राजा पृथ्व्यां भविष्यति}


\threelineshloka
{युधिष्ठिर इति ख्यातः पाण्डोः प्रथमजः सुतः}
{भविता प्रथितो राजा त्रिषु लोकेषु विश्रुतः}
{यशसा तेजसा चैव वृत्तेन च समन्वितः}


\twolineshloka
{संवत्सरे द्वितीये तु गान्धार्या उदरं महत्}
{न च प्राजायत तदा ततस्तां दुःखमाविशत्}


\twolineshloka
{श्रुत्वा कुन्तीसुतं जातं बालार्कसमतेजसम्}
{उदस्यात्मनः स्थैर्यमुपालभ्य च सौबली}


\twolineshloka
{कौरवस्यापरिज्ञातं यत्नेन महता स्वयम्}
{उदरं घातयामास गान्धारी शोकमूर्छिता}


\twolineshloka
{ततो जज्ञे मांसपेशी लोहाष्ठीलेव संहता}
{द्विवर्षसंभृता कुक्षौ तामुत्स्रष्टुं प्रचक्रमे}


\twolineshloka
{अथ द्वैपायनो ज्ञात्वा त्वरितः समुपागमत्}
{तां स मांसमयीं पेशीं ददर्श जपतां वरः}


\threelineshloka
{ततोऽवदत्सौबलेयीं किमिदं ते चिकीर्षितम्}
{सा चात्मनो मतं सर्वं शशंस परमर्षये ॥गान्धार्युवाच}
{}


\twolineshloka
{ज्येष्ठं कुन्तीसुतं जातं श्रुत्वा रविसमप्रभम्}
{दुःखेन परमेणेदमुदरं घातितं मया}


\threelineshloka
{शतं च किल पुत्राणां वितीर्णं मे त्वया पुरा}
{इयं च मे मांसपेशी जाता पुत्रशताय वै ॥व्यास उवाच}
{}


\twolineshloka
{एवमेतत्सौबलेयि नैतज्जात्वन्यथा भवेत्}
{वितथं नोक्तपूर्वं मे स्वैरेष्वपि कुतोऽन्यथा}


\twolineshloka
{घृतपूर्णं कुण्डशतं क्षिप्रमेव विधीयताम्}
{सुगुप्तेषु च देशेषु रक्षा चैव विधीयताम्}


\twolineshloka
{शीताभिरद्भिरष्ठीलामिमां च परिषिञ्चय ॥वैशंपायन उवाच}
{}


\twolineshloka
{सा सिच्यमाना ह्यष्ठीला ह्यभवच्छतधा तदा}
{अङ्गुष्ठपर्वभात्राणां गर्भाणां तत्क्षणं तथा}


\twolineshloka
{एकाधिकशतं पूर्णं यथायोगं विशांपते}
{ततः कुण्डशतं तत्र आनाय्य तु महानृषिः}


\twolineshloka
{मांसपेश्यास्तदा राजन्क्रमशः कालपर्ययात्}
{ततस्तांस्तेषु कुण्डेषु गर्भान्सर्वान्समादधत्}


\twolineshloka
{स्वनुगुप्तेषु देशेषु रक्षां चैषां व्यधापयत्}
{शशास चैव कृष्णो वै गर्भाणां रक्षणं तथा}


\twolineshloka
{उवाच चैनां भगवान्कालेनैतावता पुनः}
{स्फुटमानेषु कुण्डेषु जाताञ्जानीहि शोभने}


\twolineshloka
{उद्धाटनीयान्येतानि कुण्डानीति च सौबलीम्}
{इत्युक्त्वा भगवान्व्यासस्तथा प्रतिविधाय च}


\twolineshloka
{जगाम तपसे धीमान्हिमवन्तं शिलोच्चयम्}
{अह्नोत्तराः कुमारस्ते कुण्डेभ्यस्तु समुत्थिताः}


\twolineshloka
{तेनैवैषां क्रमेणासीज्ज्योष्ठानुज्येष्ठता तदा}
{जन्मतश्च प्रमाणेन ज्येष्ठः कुन्तीसुतोऽभवत्}


\twolineshloka
{धार्मिकं च सुतं दृष्ट्वा पाण्डुः कुन्तीमथाऽब्रवीत्}
{प्राहुः क्षत्रं बलज्येष्ठं बलज्येष्ठं सुतं वृणु}


\twolineshloka
{ततः कुन्तीमभिक्रम्य शशासातीव भारत}
{वायुमावाहयस्वेति स देवो बलवत्तरः}


\twolineshloka
{अश्वमेधः क्रतुश्रेष्ठो ज्योतिःश्रेष्ठो दिवाकरः}
{ब्राह्मणो द्विपदां श्रेष्ठो देवश्रेष्ठश्च मारुतः}


\twolineshloka
{मारुतं मरुतां श्रेष्ठं सर्वप्राणिभिरीडितम्}
{आवाहय त्वं नियमात्पुत्रार्थं वरवर्णिनि}


\threelineshloka
{स नो यं दास्यति सुतं स प्राणबलवान्नृषु}
{भविष्यति वरारोहे बलज्येष्ठा हि भूमिपाः ॥वैशंपायन उवाच}
{}


\twolineshloka
{तथोक्तवति सा काले वायुमेवाजुहाव ह}
{द्वितीयेनोपहारेण तेनोक्तविधिना पुनः}


\twolineshloka
{तैरेव नियमैः स्थित्वा मन्त्रग्राममुदैरयत्}
{आजगाम ततो वायुः किं करोमीति चाब्रवीत्}


\twolineshloka
{लज्जान्विता ततः कुन्ती पुत्रमैच्छन्महाबलम्}
{तथास्त्विति च तां वायुः समालभ्य दिवं गतः}


\twolineshloka
{तस्यां जज्ञे महावीर्यो भीमो भीमपराक्रमः}
{तमप्यतिबलं जातं वागुवाचाशरीरिणी}


\twolineshloka
{सर्वेषां बलिनां श्रेष्ठो जातोऽयमिति भारत}
{जातमात्रे कुमारे तु सर्वलोकस्य पार्थिवाः}


\twolineshloka
{मूत्रं प्रसुस्रुवुः सर्वे व्यथां चापि प्रपेदिरे}
{वाहनानि व्यशीर्यन्त व्यमुञ्चन्नश्रुबिन्दवः}


\twolineshloka
{यथाऽनिलः समुद्भूतः समर्थः कम्पने भुवः}
{तथा ह्युपचिताङ्गो वै भीमो भीमपराक्रमः}


\twolineshloka
{इदं चाद्भुतमत्रासीज्जातमात्रे वृकोदरे}
{यदऱ्कात्पतितो मातुः शिलां गात्रैरचूर्णयत्}


\twolineshloka
{कुन्ती तु सह पुत्रेण याता सुरुचिरं सरः}
{स्नात्वा च सुतमादाय दशमेऽहनि यादवी}


\twolineshloka
{दैवतान्यर्चयिष्यन्ती निर्जगामाश्रमात्पृथा}
{शैलाभ्याशेन गच्छन्त्यास्तदा भरतसत्तम}


\twolineshloka
{निश्चक्राम महाव्याघ्रो जिघांसुर्गिरिगह्वरात्}
{तमापतन्तं शार्दूलं विकृष्य धनुरुत्तमम्}


\twolineshloka
{निर्बिभेद शरैः पाण्डुस्त्रिभिस्तिरदशविक्रमः}
{नादेन महता तां तु पूरयन्तं गिरेर्गुहाम्}


\twolineshloka
{दृष्ट्वा शैलमुपारोढुमैच्छत्कुन्ती भयात्तदा}
{त्रासात्तस्याः सुतस्त्वङ्कात्पपात भरतर्षभ}


\twolineshloka
{पर्वतस्योपरिस्थायामधस्तादपतच्छिशुः}
{स शिलां चूर्णयामास वज्रवद्वज्रिचोदितः}


\twolineshloka
{पुत्रस्नेहात्ततः पाण्डुरभ्यधावद्गिरेस्तटम्}
{पतता तेन शतधा शिला गात्रैर्विचूर्णिता}


\twolineshloka
{शिलां च चूर्णितां दृष्ट्वा परं विस्मयमागमत्}
{स तु जन्मनि भीमस्य विनदन्तं विनादितम्}


\twolineshloka
{ददर्श गिरिशृङ्गस्थं व्याघ्रं व्याघ्रपराक्रमः}
{दारसंरक्षणार्थाय पुत्रसंरक्षणाय च}


\twolineshloka
{सदा बाणधनुष्पाणिरभवत्कुरुनन्दनः}
{मघे चन्द्रमसा युक्ते सिंहे चाभ्युदिते गुरौ}


\twolineshloka
{दिवा मध्यगते सूर्ये तिथौ पुण्ये त्रयोदशे}
{पित्र्ये मुहूर्ते सा कुन्ती सुषुवे भीममच्युतम्}


\twolineshloka
{यस्मिन्नहनि भमस्तु जज्ञे भीमपराक्रमः}
{तामेव रात्रिं पूर्वां तु जज्ञे दुर्योधनो नृपः}


\twolineshloka
{स जातमात्र एवाथ धृतराष्ट्रसुतो नृप}
{रासभारावसदृशं रुराव च ननाद च}


\twolineshloka
{तं खराः प्रत्यभाषन्त गृध्रगोमायुवायसाः}
{क्रव्यादाः प्राणदन्घोराः शिवाश्चाशिवनिस्वनाः}


\twolineshloka
{वाताश्च प्रववुश्चापि दिग्दाहश्चाभवत्तदा}
{ततस्तु भीतवद्राजा धृतराष्ट्रोऽब्रवीदिदम्}


\twolineshloka
{समानीय बहून्विप्रान्भीष्मं विदुरमेव च}
{अन्यांश्च सुहृदो राजन्कुरून्सर्वांस्तथैव च}


\twolineshloka
{युधिष्ठिरो राजपुत्रो ज्येष्ठो नः कुलवर्धनः}
{प्राप्तः स्वगुणतो राज्यं न तस्मिन्वाच्यमस्तिनः}


\twolineshloka
{अयं त्वनन्तरस्तस्मादपि राजा भविष्यति}
{एतद्विब्रूत मे तथ्यं यदत्र भविता ध्रुवम्}


\twolineshloka
{`अस्मिञ्जाते निमित्तानि शंसन्ती हाशिवं महत्}
{अतो ब्रवीमि विदुर द्रुतं मां भयमाविशत् ॥'}


\twolineshloka
{वाक्यस्यैतस्य निधेन दिक्षु सर्वासु भारत}
{क्रव्यादाः प्राणदन्घोराः शिवाश्चाशिवनिस्वनाः}


\twolineshloka
{लक्षयित्वा निमित्तानि तानि घोराणि सर्वशः}
{तेऽब्रुवन्ब्राह्मणा राजन्विदुरश्च महामतिः}


\twolineshloka
{यथेमानि निमित्तानि घोराणि मनुजाधिप}
{उत्थितानि सुते जाते ज्येष्ठे ते पुरुषर्षभ}


\twolineshloka
{व्यक्तं कुलान्तकरणो भवितैष सुतस्तव}
{तस्य शान्तिः परित्यागे गुप्तावपनयो महान्}


\twolineshloka
{`एष दुर्योधनो राजा मधुपिङ्गललोचनः}
{न केवलं कुलस्यान्तं क्षत्रियान्तं करिष्यति ॥'}


\twolineshloka
{शतमेकोनमप्यस्तु पुत्राणां ते महीपते}
{त्यजैनमेकं शान्तिं चेत्कुलस्येच्छसि भारत}


\twolineshloka
{एकेन कुरु वै क्षेमं कुलस्य जगतस्तथा}
{त्यजेदेकं कुलस्यार्थे ग्रामस्यार्थे कुलं त्यजेत्}


\twolineshloka
{ग्रामं जनपदस्यार्थे आत्मार्थे पृथिवीं त्यजेत्}
{स तथा विदुरेणोक्तस्तैश्च सर्वैर्द्विजोत्तमैः}


\twolineshloka
{न चकार तथा राजा पुत्रस्नेहसमन्वितः}
{ततः पुत्रशतं पूर्ण धृतराष्ट्रस्य पार्थिव}


\twolineshloka
{अह्नांशतेन संजज्ञे कन्या चैका शताधिका}
{गान्धार्यां क्लिश्यमानायामुदरेण विवर्धता}


\twolineshloka
{`वैश्या सा त्वम्बिकापुत्रं कन्या परिचचार ह}
{तया समभवद्राजा धृतराष्ट्रो यदृच्छया ॥'}


\threelineshloka
{तस्मिन्संवत्सरे राजन्धृतराष्ट्रान्महायशाः}
{जज्ञे धीमांस्ततस्तस्यां युयुत्सुः करमो नृप}
{एवं पुत्रशतं जज्ञे धृतराष्ट्रस्य धीमतः}


\twolineshloka
{महारथानां वीराणां कन्या चैका शताधिका}
{युयुत्सुश्च महातेजा वैश्यापुत्रः प्रतापवान्}


\chapter{अध्यायः १३०}
\twolineshloka
{जनमेजय उवाच}
{}


\twolineshloka
{धृतराष्ट्रस्य पुत्राणामादितः कथितं त्वया}
{ऋषेः प्रसादात्तु शतं न च कन्या प्रकीर्तिता}


\twolineshloka
{वैश्यापुत्रो युयुत्सुश्च कन्या चैका शताधिका}
{गान्धारराजदुहिता शतपुत्रेति चानघ}


\twolineshloka
{उक्ता महर्षिणा तेन व्यासेनामिततेजसा}
{कथं त्विदानीं भगवन्कन्यां त्वं तु ब्रवीषि मे}


\twolineshloka
{यदि भागशतं पेशी कृता तेन महर्षिणा}
{न प्रजास्यति चेद्भूयः सौबलेयी कथंचन}


\threelineshloka
{कथं तु संभवस्तस्या दुःशलाया वदस्व मे}
{यथावदिह विप्रर्षे परं मेऽत्र कुतूहलम् ॥वैशंपायन उवाच}
{}


\twolineshloka
{साध्वयं प्रश्न उद्दिष्टः पाण्डवेय ब्रवीमि ते}
{तां मांसपेशीं भगवान्स्वयमेव महातपाः}


\twolineshloka
{शीताभिरद्भिरासिच्य भागं भागमकल्पयत्}
{यो यथा कल्पितो भागस्तंत धात्र्या तथा नृप}


\twolineshloka
{घृतपूर्णेषु कुण्डेषु एकैकं प्राक्षिपत्तदा}
{एतस्मिन्नन्तरे साध्वी गान्धारी सुदृढव्रता}


\threelineshloka
{दुहितुः स्नेहसंयोगमनुध्याय वराङ्गना}
{`नाब्रवीत्तमृषिं किंचिद्गौरवाच्च यशस्विनी}
{'मनसा चिन्तयद्देवी एतत्पुत्रशतं मम}


\twolineshloka
{भविष्यति न संदेहो न ब्रवीत्यन्यथा मुनिः}
{ममेयं परमा तुष्टिर्दुहिता मे भवेद्यदि}


\twolineshloka
{एका शताधिका बाला भविष्यति कनीयसी}
{ततो दौहित्रजाल्लोकादबाह्योऽसौ पतिर्मम}


\twolineshloka
{अधिका किल नारीणां प्रीतिर्जामातृजा भवेत्}
{यदि नाम ममापि स्याद्दुहितैका शताधिका}


\twolineshloka
{कृतकृत्या भवेयं वै पुत्रदौहित्रसंवृता}
{यदि सत्यं तपस्तप्तं दत्तं वाऽप्यथवा हुतम्}


\twolineshloka
{गुरवस्तोषिता वापि तथाऽस्तु दुहिता मम}
{एतस्मिन्नेव काले तु कृष्णद्वैपायनः स्वयम्}


\twolineshloka
{व्यभजत्स तदा पेशीं भगवानृषिसत्तमः}
{`गण्यमानेषु कुण्डेषु शते पूर्णे महात्मना}


\threelineshloka
{अभवच्चापरं खण्डं वामहस्ते तदा किल}
{'गणयित्वा शतं पूर्णमंशानामाह सौबलीम् ॥व्यास उवाच}
{}


\twolineshloka
{पूर्णं पुत्रशतं त्वेतन्न मिथ्या वागुदाहृता}
{दैवयोगाच्च भागैकः परिशिष्टः शतात्परः}


\threelineshloka
{एषा ते सुभगा कन्या भविष्यति यतेप्सिता}
{वैशंपायन उवाच}
{ततोऽन्यं घृतकुम्भं च समानाय्य महातपाः}


\twolineshloka
{तं चापि प्राक्षिपत्तत्र कन्याभागं तपोधनः}
{`संभूता चैव कालेन सर्वेषां च यवीयसी}


\twolineshloka
{ऐतत्ते कथितं राजन्दुःशलाजन्म भारत}
{ब्रूहि राजेन्द्र किं भूयो वर्तयिष्यामि तेऽनघ}


\chapter{अध्यायः १३१}
\twolineshloka
{जनमेजय उवाच}
{}


\threelineshloka
{ज्येष्ठाऽनुज्येष्ठतां तेषां नामानि च पृथक्पृथक्}
{धृतराष्ट्रस्य पुत्राणामानुपूर्व्यात्प्रकीर्तय ॥वैशंपायन उवाच}
{}


\twolineshloka
{दुर्योधनो युयुत्सुश्च राजन्दुःशासनस्तथा}
{दुःसहो दुःशलश्चैव जलसन्धः समः सहः}


\twolineshloka
{विन्दानुविन्दौ दुर्धर्षः सुबाहुर्दुष्प्रधर्षणः}
{दुर्मर्षणो दुर्मुखश्च दुष्कर्णः कर्ण एव च}


\twolineshloka
{विविंशतिर्विकर्णश्च शलः सत्वः सुलोचनः}
{चित्रोपचित्रौ चित्राक्षश्चारुचित्रः शरासनः}


\twolineshloka
{दुर्मदो दुर्विगाहश्च विवित्सुर्विकटाननः}
{ऊर्णनाभः सुनाभश्च तथा नन्दोपनन्दकौ}


\twolineshloka
{चित्रबाणश्चित्रवर्मा सुवर्मा दुर्विमोचनः}
{अयोबाहुर्महाबाहुश्चित्राङ्गश्चित्रकुण्डलः}


\twolineshloka
{भीमवेगो भीमबलो बलाकी बलवर्धनः}
{उग्रायुधः सुषेणश्च कुण्डधारो महोदरः}


\twolineshloka
{चित्रायुधो निषङ्गी च पाशी वृन्दारकस्तथा}
{दृढवर्मा दृढक्षत्रः सोमकीर्तिरनूदरः}


\twolineshloka
{दृढसन्धो जरासन्धः सत्यसन्धः सदः सुवाक्}
{उग्रश्रवा उग्रसेनः सेनानीर्दुष्पराजयः}


\twolineshloka
{अपराजितः कुण्डशायी विशालाक्षो दुराधरः}
{दृढहस्तः सुहस्तश्च वातवेगसुवर्चसौ}


\twolineshloka
{आदित्यकेतुर्बह्वाशी नागदत्तोऽग्रयाय्यपि}
{कवची क्रथनः कुण्डी कुण्डधारो धनुर्धरः}


\twolineshloka
{उग्रभीमरथौ वीरौ वीरबाहुरलोलुपः}
{अभयो रौद्रकर्मा च तथा दृढरथाश्रयः}


\twolineshloka
{अनाधृष्यः कुण्डभेदी विरावी चित्रकुण्डलः}
{प्रमथश्च प्रमाथी च दीर्घरोमश्च वीर्यवान्}


\twolineshloka
{दीर्घबाहुर्महाबाहुर्व्यूढोराः कनकध्वजः}
{कुण्डाशी विराजाश्चैव दुःशला च शताधिका}


\twolineshloka
{इति पुत्रशतं राजन्कन्या चैव शताधिका}
{नामधेयानुपूर्व्येण विद्धि जन्मक्रमं नृप}


\twolineshloka
{सर्वे त्वतिरथाः शूराः सर्वे युद्धविशारदाः}
{सर्वे वेदविदश्चैव सर्वे सर्वास्त्रकोविदाः}


\twolineshloka
{सर्वेषामनुरूपाश्च कृता दारा महीपते}
{धृतराष्ट्रेण समये परीक्ष्य विविवन्नृप}


\twolineshloka
{दुःशलां चापि समये धृतराष्ट्रो नराधिपः}
{जयद्रथाय प्रददौ विधिना भरतर्षभ}


\twolineshloka
{`इति पुत्रशतं राजन्युयुत्सुश्च शताधिकः}
{कन्यका दुःशला चैव यथावत्कीर्तितं मया'}


\chapter{अध्यायः १३२}
\twolineshloka
{वैशंपायन उवाच}
{}


\twolineshloka
{जाते बलवतां श्रेष्ठे पाण्डुश्चिन्तापरोऽभवत्}
{कथमन्यो मम सुतो लोके श्रेष्ठो भवेदिति}


\twolineshloka
{दैवे पुरुषकारे च लोकोऽयं संप्रतिष्ठितः}
{तत्र दैवं तु विधिना कालयुक्तेन लभ्यते}


\twolineshloka
{इन्द्रो हि राजा देवानां प्रधान इति नः श्रुतम्}
{अप्रमेयबलोत्साहो वीर्यवानमितद्युतिः}


\twolineshloka
{तं तोषयित्वा तपसा पुत्रं लप्स्ये महाबलम्}
{यं दास्यति स मे पुत्रं स वीरयान्भविष्यति}


\twolineshloka
{अमानुषान्मानुषांश्च संग्रामे स हनिष्यति}
{कर्मणा मनसा वाचा तस्मात्तप्स्ये महत्तपः}


\twolineshloka
{ततः पाण्डुर्महाराजो मन्त्रयित्वा महर्षिभिः}
{दिदेश कुन्त्याः कौरव्यो व्रतं सांवत्सरं शुभम्}


\twolineshloka
{आत्मना च महाबाहुरेकपादस्थितोऽभवत्}
{उग्रं स तप आस्थाय परमेण समाधिना}


\twolineshloka
{आरिराधयिषुर्देवं त्रिदशानां तमीश्वरम्}
{सूर्येण सह धर्मात्मा पर्यतप्यत भारत}


\threelineshloka
{तं तु कालेन महता वासवः प्रत्यपद्यत}
{शक्र उवाच}
{पुत्रं तव प्रदास्यामि त्रिषु लोकेषु विश्रुतम्}


\twolineshloka
{ब्राह्मणानां गवां चैव सुहृदां चार्थसाधकम्}
{दुर्हृदां शोकजननं सर्वबान्धवनन्दनम्}


\twolineshloka
{सुतं तेऽग्र्यं प्रदास्यामि सर्वामित्रविनाशनम्}
{इत्युक्तः कारैवो राजा वासवेन महात्मना}


\twolineshloka
{उवाच कुन्तीं धर्मात्मा देवराजवचः स्मरन्}
{उदर्कस्तव कल्याणि तुष्टो देवगणेश्वरः}


\twolineshloka
{दातुमिच्छति ते पुत्रं यथा संकल्पितं त्वया}
{अतिमानुषकर्माणं यशस्विनमरिन्दमम्}


\twolineshloka
{नीतिमन्तं महात्मानमादित्यसमतेजसम्}
{दुराधर्षं क्रियावन्तमतीवाद्भुतदर्शनम्}


\threelineshloka
{पुत्रं जनय सुश्रोणि धाम क्षत्रियतेजसाम्}
{लब्धः प्रसादो देवेन्द्रात्तमाह्वय शुचिस्मिते ॥वैशंपायन उवाच}
{}


\twolineshloka
{एवमुक्ता ततः शक्रमाजुहाव यशस्विनी}
{अथाजगाम देवेन्द्रो जनयामास चार्जुनम्}


\twolineshloka
{`उत्तराभ्यां तु पूर्वाभ्यां फल्गुनीभ्यां ततो दिवा}
{जातस्तु फाल्गुने मासि तेनासौ फल्गुनःस्मृतः'}


\threelineshloka
{जातमात्रे कुमारे तु `सर्वभूतप्रहर्षिणी}
{सूतके वर्तमानां तां' वागुवाचाशरीरिणी}
{महागम्भीरनिर्घोषा नभो नादयती तदा}


\twolineshloka
{शृण्वतां सर्वभूतानां तेषां चाश्रमवासिनाम्}
{कुन्तीमाभाष्य विस्पष्टमुवाचेदं शुचिस्मिताम्}


\twolineshloka
{कार्तवीर्यसमः कुन्ति शिवतुल्यपराक्रमः}
{एष शक्र इवाजय्यो यशस्ते प्रथयिष्यति}


\twolineshloka
{अदित्या विष्णुना प्रीतिर्यथाऽभूदभिवर्धिता}
{तथा विष्णुसमः प्रीतिं वर्धयिष्यति तेऽर्जुनः}


\twolineshloka
{एष मद्रान्वशे कृत्वा कुरूंश्च सह सोमकैः}
{चेदिकाशिकरूषांश्च कुरुलक्ष्मीं वहिष्यति}


\twolineshloka
{एतस्य भुजवीर्येण खाण्डवे हव्यवाहनः}
{मेदसा सर्वभूतानां तृप्तिं यास्यति वै पराम्}


\twolineshloka
{ग्रामणीश्च महीपालानेष जित्वा महाबलः}
{भ्रातृभिः सहितो वीरस्त्रीन्मेधानाहरिष्यति}


\twolineshloka
{जामदग्न्यसमः कुन्ति विष्णुतुल्यपराक्रमः}
{एष वीर्यवतां श्रेष्ठो भविष्यति महायशाः}


\twolineshloka
{एष युद्धे महादेवं तोषयिष्यति शङ्करम्}
{अस्त्रं पाशुपतं नाम तस्मात्तुष्टादवाप्स्यति}


\twolineshloka
{निवातकवचा नाम दैत्या विबुधविद्विषः}
{शक्राज्ञया महाबाहुस्तान्वधिष्यति ते सुतः}


\twolineshloka
{तथा दिव्यानि चास्त्राणि निखिलेनाहरिष्यति}
{विप्रनष्टां श्रियं चायमाहर्ता पुरुषर्षभः}


\twolineshloka
{एतामत्यद्भुतां वाचं कुन्ती शुश्राव सूतके}
{वाचमुच्चरितामुच्चैस्तां निशम्य तपस्विनाम्}


\twolineshloka
{बभूव पमो हर्षः शतशृङ्गनिवासिनाम्}
{तथा देवमहर्षीणां सेन्द्राणां च दिवौकसाम्}


\twolineshloka
{आकाशे दुन्दुभीनां च बभूव तुमुलः स्वनः}
{उदतिष्ठन्महाघोरः पुष्पवृष्टिभिरावृतः}


\threelineshloka
{समवेत्य च देवानां गणाः पार्थमपूजयन्}
{काद्रवेया वैनतेया गन्धर्वाप्सरसस्तथा}
{प्रजानां पतयः सर्वे सप्त चैव महर्षयः}


\twolineshloka
{भरद्वाजः खस्यपो गौतमश्चविश्वामित्रो जमदग्निर्वसिष्ठः}
{यश्चोदितो भास्करेऽभूत्प्रनष्टेसोऽप्यत्रात्रिर्भगवानाजगाम}


\twolineshloka
{मरीचिरङ्गिराश्चैव पुलस्त्यः पुलहः क्रतुः}
{दक्षः प्रजापतिश्चैव गन्धर्वाप्सरसस्तथा}


\twolineshloka
{दिव्यमाल्याम्बरधराः सर्वालङ्कारभूषिताः}
{उपगायन्ति बीभत्सुं नृत्यन्तेऽप्सरसां गणाः}


\twolineshloka
{तथा महर्षयश्चापि जेपुस्तत्र समन्ततः}
{गन्धर्वैः सहितः श्रीमान्प्रागायत च तुम्बुरुः}


\twolineshloka
{भीमसेनोग्रसेनौ च ऊर्णायुरनघस्तथा}
{गोपतिर्धृतराष्ट्रश्च सूर्यवर्चास्तथाष्टमः}


\twolineshloka
{युगपस्तृणपः कार्ष्णिर्नन्दिश्चित्ररथस्तथा}
{त्रयोदशः शालिशिराः पर्जन्यश्च चतुर्दशः}


\twolineshloka
{कलिः पञ्चदशश्चैव नारदश्चात्र षोडशः}
{ऋत्वा बृहत्त्वा बृहकः करालश्च महामनाः}


\twolineshloka
{ब्रह्मचारी बहुगुणः सुवर्णश्चेति विश्रुतः}
{विश्वावसुर्भुमन्युश्च सुचन्द्रश्च शरुस्तथा}


\twolineshloka
{गीतमाधुर्यसंपन्नौ विख्यातौ च हहाहुहू}
{इत्येते देवगन्धर्वा जग्मुस्तत्र नराधिप}


\twolineshloka
{तथैवाप्सरसो हृष्टाः सर्वालङ्कारभूषिताः}
{ननृतुर्वै महाभागा जगुश्चायतलोचनाः}


\twolineshloka
{अनूचानाऽनवद्या च गुणमुख्या गुणावरा}
{अद्रिका च तथा सोमा मिश्रकेशी त्वलम्बुषा}


\twolineshloka
{मरीचिः शुचिका चैव विद्युत्पर्णा तिलोत्तमा}
{अम्बिका लक्षणा क्षेमा देवी रम्भा मनोरमा}


\twolineshloka
{असिता च सुबाहुश्च सुप्रिया च वपुस्तथा}
{पुण्डरीका सुगन्धा च सुरसा च प्रमाथिनी}


\twolineshloka
{काम्या शारद्वती चैव ननृतुस्तत्र संघशः}
{मेनका सहजन्या च कर्णिका पुञ्जिकस्थला}


\twolineshloka
{ऋतुस्थला घृताची च विश्वाची पूर्वचित्त्यपि}
{उम्लोचेति च विख्याता प्रम्लोचेति च ता दश}


\twolineshloka
{उर्वश्येकादशी तासां जगुश्चायतलोचनाः}
{धाताऽर्यमा च मित्रश्च वरुणोंऽशो भगस्तथा}


\twolineshloka
{इन्द्रो विवस्वान्पूषा च पर्जन्यो दशमः स्मृतः}
{ततस्त्वष्टा ततो विष्णुरजघन्यो जघन्यजः}


% Check verse!
इत्येते द्वादशादित्या ज्वलन्तः सूर्यवर्चसः
\twolineshloka
{मृगव्याधश्च सर्पश्च निर्ऋतिश्च महायशाः}
{अजैकपादहिर्बुध्न्यः पिनाकी च परंतप}


\twolineshloka
{दहनोऽथेश्वरश्चैव कपाली च विशांपते}
{स्थाणुर्भगश्च भगवान्रुद्रास्तत्रावतस्थिरे}


\twolineshloka
{अश्विनौ वसवश्चाष्टौ मरुतश्च महाबलाः}
{विश्वेदेवास्तथा साध्यास्तत्रासन्परितः स्थिताः}


\twolineshloka
{कर्कोटकोऽथ सर्पश्च वासुकिश्च भुजङ्गमः}
{कच्छपश्चाथ कुण्डश्च तक्षकश्च महोरगः}


\twolineshloka
{आययुस्तपसा युक्ता महाक्रोधा महाबलाः}
{एते चान्ये च बहवस्तत्र नागा व्यवस्थिताः}


\twolineshloka
{तार्क्ष्यश्चारिष्टनेमिश्च गरुडश्चासितध्वजः}
{अरुणश्चारुणिश्चैव वैनतेया व्यवस्थिताः}


\twolineshloka
{तांश्च देवगणान्सर्वांस्तपःसिद्धा महर्षयः}
{विमानगिर्यग्रगतान्ददृशुर्नेतरे जनाः}


\twolineshloka
{तद्दृष्ट्वा महदाश्चर्यं विस्मिता मुनिसत्तमाः}
{अधिकां स्म ततो वृत्तिमवर्तन्पाण्डवं प्रति}


\twolineshloka
{पाण्डुः प्रीतेन मनसा देवतादीनपूजयत्}
{पाण्डुना पूजिता देवाः प्रत्यूचुर्नरसत्तमम्}


\twolineshloka
{प्रादुर्बूतो ह्ययं धर्मो देवतानां प्रसादतः}
{मातरिश्वा ह्ययं भीमो बलवानरिमर्दनः}


\twolineshloka
{साक्षादिन्द्रः स्वयं जातः प्रसादाच्च शतक्रतोः}
{पितृत्वाद्देवतानां हि नास्ति पुण्यतरस्त्वया}


\twolineshloka
{पितॄणामृणनिर्मुक्तः स्वर्गं प्राप्स्यसि पुण्यभाक्}
{इत्युक्त्वा देवताः सर्वा विप्रजग्मुर्यथागतम्}


\twolineshloka
{पाण्डुस्तु पुनरेवैनां पुत्रलोभान्महायशाः}
{प्रादिशद्दर्शनीयार्थी कुन्ती त्वेनमथाब्रवीत्}


\twolineshloka
{नातश्चतुर्थं प्रसवमापस्त्वपि वदन्त्युत}
{अतःपरं स्वैरिणी स्याद्बन्धकी पञ्चमे भवेत्}


\twolineshloka
{स त्वं विद्वन्धर्ममिममधिगम्य कथं नु माम्}
{अपत्यार्थं समुत्क्रम्य प्रमादादिव भाषसे}


\chapter{अध्यायः १३३}
\twolineshloka
{वैशंपायन उवाच}
{}


\twolineshloka
{कुन्तीपुत्रेषु जातेषु धृतराष्ट्रात्मजेषु च}
{मद्रराजसुता पाण्डुं रहो वचनमब्रवीत्}


\twolineshloka
{न मेऽस्ति त्वयि सन्तापो विगुणेऽपि परन्तप}
{नावरत्वे वरार्हायाः स्थित्वा चानघ नित्यदा}


\twolineshloka
{गान्धार्याश्चैव नृपते जातं पुत्रशतं तथा}
{श्रुत्वा न मे तथा दुःखमभवत्कुरुनन्दन}


\twolineshloka
{इदं तु मे महद्दुःखं तुल्यतायामपुत्रता}
{दिष्ट्या त्विदानीं भर्तुर्मे कुन्त्यामप्यस्ति सन्ततिः}


\twolineshloka
{यदि त्वपत्यसन्तानं कुन्तिराजसुता मयि}
{कुर्यादनुग्रहो मे स्यात्तव चापि हितं भवेत्}


\threelineshloka
{संरम्भो हि सपत्नीत्वाद्वक्तुं कुन्तिसुतां प्रति}
{यदि तु त्वं प्रसन्नो मे स्वयमेनां प्रचोदय ॥पाण्डुरुवाच}
{}


\twolineshloka
{ममाप्येष सदा माद्रि हृद्यर्थः परिवर्तते}
{न तु त्वां प्रसहे वक्तुमिष्टानिष्टविवक्षया}


\threelineshloka
{तव त्विदं मतं मत्वा प्रयतिष्याम्यतः परम्}
{मन्ये ध्रुवं मयोक्ता सा वचनं प्रतिपत्स्यते ॥वैशंपायन उवाच}
{}


\threelineshloka
{ततः कुन्तीं पुनः पाण्डुर्विविक्त इदमब्रवीत्}
{`अनुगृह्णीष्व कल्याणि मद्रराजसुतामपि}
{'कुलस्य मम सन्तानं लोकस्य च कुरु प्रियम्}


\twolineshloka
{मम चापिण्डनाशाय पूर्वेषां च महात्मनाम्}
{मत्प्रियार्थं च कल्याणि कुरु कल्याणमुत्तमम्}


\twolineshloka
{यशसोऽर्थाय चैव त्वं कुरु कर्म सुदुष्करम्}
{प्राप्याधिपत्यमिन्द्रेण यज्ञैरिष्टं यशोऽर्थिना}


\twolineshloka
{तथा मन्त्रविदो विप्रास्तपस्तप्त्वा सुदुष्करम्}
{गुरूनभ्युपगच्छन्ति यशसोऽर्थाय भामिनि}


\twolineshloka
{तथा राजर्षयः सर्वे ब्राह्मणाश्च तपोधनाः}
{चक्रुरुच्चावचं कर्म यशसोऽर्थाय दुष्करम्}


\threelineshloka
{सा त्वं माद्रीं प्लवेनैव तारयैनामनिन्दिते}
{अपत्यसंविधानेन परां कीर्तिमवाप्नुहि ॥`कुन्त्युवाच}
{}


\threelineshloka
{धर्मं वै धर्मशास्त्रोक्तं यथा वदसि तत्तथा}
{तस्मादनुग्रहं तस्याः करोमि कुरुनन्दन ॥'वैशंपायन उवाच}
{}


\twolineshloka
{एवमुक्ताऽब्रवीन्मार्द्रीं सकृच्चिन्तय दैवतम्}
{तस्मात्ते भविताऽपत्यमनुरूपमसंशयम्}


\twolineshloka
{`ततो मन्त्रे कृते तस्मिन्विधिदृष्टेन कर्मणा}
{ततो राजसुता स्नाता शयने संविवेश ह ॥'}


\threelineshloka
{ततो माद्री विचार्यैका जगाम मनसाऽश्विनौ}
{तावागम्य सुतौ तस्यां जनयामासतुर्यमौ}
{नकुलं सहदेवं च रूपेणाप्रतिमौ भुवि}


\twolineshloka
{तथैव तावपि यमौ वागुवाचाशरीरिणी}
{`धर्मतो भक्तितश्चैव शीलतो विनयैस्तथा}


\twolineshloka
{सत्वरूपगुणोपेतौ भवतोऽत्यश्विनाविति}
{मासते तेजसाऽत्यर्थं रूपद्रविणसंपदा}


\twolineshloka
{नामानि चक्रिरे तेषां शतशृङ्गनिवासिनः}
{भक्त्या च कर्मणा चैव तथाऽऽशीर्भिर्विशांपते}


\twolineshloka
{ज्येष्ठं युधिष्ठिरेत्येवं भीमसेनेति मध्यमम्}
{अर्जुनेति तृतीयं च कुन्तीपुत्रानकल्पयन्}


\twolineshloka
{पूर्वजं नकुलेत्येवं सहदेवेति चापरम्}
{माद्रीपुत्रावकथयंस्ते विप्राः प्रीतमानसाः}


\twolineshloka
{अनुसंवत्सरं जाता अपि ते कुरुसत्तमाः}
{पाण्डुपुत्रा व्यराजन्त पञ्चसंवत्सरा इव}


\twolineshloka
{महासत्त्वा महावीर्या महाबलपराक्रमाः}
{पाण्डुर्दृष्ट्वा सुतांस्तांस्तु देवरूपान्महौजसः}


\twolineshloka
{मुदं परमिकां लेभे ननन्द च नराधिपः}
{ऋषीणामपि सर्वेषां शतशृङ्गनिवासिनाम्}


\twolineshloka
{प्रिया बभूवुस्तासां च तथैव मुनियोषिताम्}
{कुन्तीमथ पुनः पाण्डुर्माद्र्यर्थे समचोदयत्}


\twolineshloka
{तमुवाच पृथा राजन् रहस्युक्ता तदा सती}
{उक्ता सक्वद्द्वन्द्वमेषा लेभे तेनास्मि वञ्चिता}


\twolineshloka
{बिभेम्यस्याः परिभवात्कुस्त्रीणां गतिरिदृशी}
{नाज्ञासिषमहं मूढा द्वन्द्वाह्वाने फलद्वयम्}


\twolineshloka
{तस्मान्नाहं नियोक्तव्या त्वयैषोऽस्तु वरो मम}
{एवं पाण्डोः सुताः पञ्च देवदत्ता महाबलाः}


\twolineshloka
{संभूताः कीर्तिमन्तश्च कुरुवंशविवर्धनाः}
{शुभलक्षणसंपन्नाः सोमवत्प्रियदर्शनाः}


\twolineshloka
{सिंहदर्पा महेष्वासाः सिंहविक्रान्तगामिनः}
{सिंहग्रीवा मनुष्येन्द्रा ववृधुर्देवविक्रमाः}


\twolineshloka
{विवर्धमानास्ते तत्र पुण्ये हैमवते गिरौ}
{विस्मयं जनयामासुर्महर्षीणां समेयुषाम्}


\threelineshloka
{`जातमात्रानुपादाय शतशृङ्गनिवासिनः}
{पाण्डोः पुत्रानमन्यन्त तापसाः स्वानिवात्मजान् ॥वैशंपायन उवाच}
{}


% Check verse!
ततस्तु वृष्णयः सर्वे वसुदेवपुरोगमाः
\twolineshloka
{पाण्डुः शापभयाद्भीतः शतशृङ्गमुपेयिवान्}
{तत्रैव मुनिभिः सार्धं तापसोऽभूत्तपस्विभिः}


\twolineshloka
{शाकमूलफलाहारस्तपस्वी नियतेन्द्रियः}
{योगध्यानपरो राजा बभूवेति च वादकाः}


\twolineshloka
{प्रबुवन्ति स्म बहवस्तच्छ्रुत्वा शोककर्शिताः}
{पाण्डोः प्रीतिसमायुक्ताः कदा श्रोष्याम संकथाः}


\twolineshloka
{इत्येवं कथयन्तस्ते वृष्णयः सह बान्धवैः}
{पाण्डोः पुत्रागमं श्रुत्वा सर्वे हर्षसमन्विताः}


\twolineshloka
{सभाजयन्तस्तेऽन्योन्यं वसुदेवं वचोऽब्रुवन्}
{न भवेरन्क्रियाहीनाः पाण्डुपुत्रा महाबलाः}


\twolineshloka
{पाण्डोः प्रियहितान्वेषी प्रेषय त्वं पुरोहितम्}
{वसुदेवस्तथेत्युक्त्वा विससर्ज पुरोहितम्}


\twolineshloka
{युक्तानि च कुमाराणां पारबर्हाण्यनेकशः}
{कुन्तीं माद्रीं च संदिश्य दासीदासपरिच्छदम्}


\twolineshloka
{गावो हिरण्यं रौप्यं च प्रेषयामास भारत}
{तानि सर्वाणि संगृह्य प्रययौ स पुरोहितः}


\twolineshloka
{तमागतं द्विजश्रेष्ठं काश्यपं वै पुरोहितम्}
{पूजयामास विधिवत्पाण्डुः परपुरञ्जयः}


\twolineshloka
{पृथा माद्री च संहृष्टे वसुदेवं प्रशंसताम्}
{ततः पाण्डुः क्रियाः सर्वाः पाण्डवानामकारयत्}


\twolineshloka
{गर्भाधानादिकृत्यानि चौलोपनयनानि च}
{काश्यपः कृतवान्सर्वमुपाकर्म च भारत}


\twolineshloka
{चौलोपनयनादूर्ध्वमृषभाक्षा यशस्विनः}
{वैदिकाध्ययने सर्वे समपद्यन्त पारगाः}


\twolineshloka
{शर्यातेः प्रथमः पुत्रः शुक्रो नाम परन्तपः}
{येन सागरपर्यन्ता धुषा निर्जिता मही}


\twolineshloka
{अश्वमेधशतैरिष्ट्वा स महात्मा महामखैः}
{आराध्य देवताः सर्वाः पितॄनपि महामतिः}


\twolineshloka
{शतशृह्गे तपस्तेपे शाकमूलफलाशनः}
{तेनोपकरणश्रेष्ठैः शिक्षया चोपबृंहिताः}


\twolineshloka
{तत्प्रसादाद्धनुर्वेदे समपद्यन्त पारगाः}
{गदायां पारगो भीमस्तोमरेषु युधिष्ठिरः}


\twolineshloka
{असिचर्मणि निष्णातौ यमौ सत्त्ववतां वरौ}
{धनुर्वेदे गतः पारं सव्यसाची परन्तपः}


\twolineshloka
{शुक्रेण समनुज्ञातो मत्समोऽयमिति प्रभो}
{अनुज्ञाय ततो राजा शक्तिं खङ्गं ततः शरान्}


\twolineshloka
{धनुश्च दमतां श्रेष्ठस्तालमात्रं महाप्रभम्}
{विपाठक्षुरनाराचान्गृध्रपक्षैरलङ्कृतान्}


\twolineshloka
{ददौ पार्थाय संहृष्टो महोरगसमप्रभान्}
{अवाप्य सर्वशस्त्राणि मुदितो वासवात्मजः}


% Check verse!
मेने सर्वान्महीपालानपर्याप्तान्स्वतेजसः
\twolineshloka
{एकवर्षान्तरास्त्वेवं परस्परमरिन्दमाः}
{अन्ववर्तन्त पार्थाश्च माद्रीपुत्रौ तथैव च ॥'}


\twolineshloka
{ते च पञ्च शतं चैव कुरुवंशविवर्धनाः}
{सर्वे ववृधिरेऽल्पेन कालेनाप्स्विव पङ्कजाः}


\chapter{अध्यायः १३४}
\twolineshloka
{`जनमेजय उवाच}
{}


\threelineshloka
{कस्मिन्वयसि संप्राप्ताः पाण्डवा गजसाह्वयम्}
{समपद्यन्त देवेभ्यस्तेषामायुश्च किं परम् ॥वैशंपायन उवाच}
{}


\twolineshloka
{पाण्डवानामिहायुष्यं शृणु कौरवनन्दन}
{जगाम हास्तिनपुरं षोडशाब्दो युधिष्ठिरः}


\twolineshloka
{भीमसेनः पञ्चदशो बीभत्सुर्वै चतुर्दशः}
{त्रयोदशाब्दौ च यमौ जग्मतुर्नागसाह्वयम्}


\twolineshloka
{तत्र त्रयोदशाब्दानि धार्तराष्ट्रैः सहोषिताः}
{षण्मासाञ्जातुषगृहान्मुक्ता जातो घटोत्कचः}


\twolineshloka
{षण्मासानेकचक्रायां वर्षं पाञ्चालके गृहे}
{धार्तराष्ट्रैः सहोषित्वा पञ्च वर्षाणि भारत}


\twolineshloka
{इन्द्रप्रस्थे वसन्तस्ते त्रीणि वर्षाणि विंशतिम्}
{द्वादशाब्दानथैकं च बभूवुर्द्यूतनिर्जिताः}


\twolineshloka
{भुक्त्वा षट्त्रिंशतं राजन्सागरान्तां वसुन्धराम्}
{मासैः षड्भिर्महात्मानः सर्वे कृष्णपरायणाः}


\twolineshloka
{राज्ये परीक्षितं स्थाप्य दिष्टां गतिमवाप्नुवन्}
{एवं युधिष्ठिरस्यासीदायुरष्टोत्तरं शतम्}


\twolineshloka
{अर्जुनात्केशवो ज्येष्ठस्त्रिभिर्मासैर्महाद्युतिः}
{कृष्णात्संकर्षणो ज्येष्ठस्त्रिभिर्मासैर्महाबलः}


\twolineshloka
{पाण्डुः पञ्चमहातेजास्तान्पश्यन्पर्वते सुतान्}
{रेमे स काश्यपयुतः पत्नीभ्यां सुभृशं तदा}


\twolineshloka
{सुपुष्पितवने काले प्रवृत्ते मधुमाधवे}
{पूर्णे चतुर्दशे वर्षे फल्गुनस्य च धीमतः}


\twolineshloka
{यस्मिन्नृक्षे समुत्पन्नः पार्थस्तस्य च धीमतः}
{तस्मिन्नुत्तरफल्गुन्यां प्रवृत्ते स्वस्तिवाचने}


\threelineshloka
{रक्षणे विस्मृता कुन्ती व्यग्रा ब्राह्मणभोजने}
{पुरोहितेन सहितान्ब्राह्मणान्पर्यवेषयत् ॥'वैशंपायन उवाच}
{}


\twolineshloka
{दर्शनीयांस्ततः पुत्रान्पाण्डुः पञ्च महावने}
{तान्पश्यन्पर्वते रम्ये स्वबाहुबलमाश्रितः}


\twolineshloka
{सुपुष्पितवने काले कदाचिन्मधुमाधवे}
{भूतसंमोहने राजा सभार्यो व्यचरद्वनम्}


\twolineshloka
{पलाशैस्तिलकैश्चूतैश्चम्पकैः पारिभद्रकैः}
{अन्यैश्च बहुभिर्वृक्षैः फलपुष्पसमृद्धिभिः}


\twolineshloka
{जलस्थानैश्च विविधैः पद्मिनीभिश्च शोभितम्}
{पाण्डोर्वनं तत्संप्रेक्ष्य प्रजज्ञे हृदि मन्मथः}


\twolineshloka
{प्रहृष्टमनसं तत्र विचरन्तं यथाऽमरम्}
{तं माद्र्यनुजगामैका वसनं बिभ्रती शुभम्}


\twolineshloka
{समीक्षमाणः स तु तां वयःस्थां तनुवाससम्}
{तस्य कामः प्रवृते गहनेऽग्निरिवोद्गतः}


\twolineshloka
{रहस्येकां तु तां दृष्ट्वा राजा राजीवलोचनाम्}
{न शशाक नियन्तुं तं कामं कामवशीकृतः}


\twolineshloka
{`अथ सोऽष्टादशे वर्षे ऋतौ माद्रमलङ्कृताम्}
{आजुहाव ततः पाण्डुः परीतात्मा यशस्विनीम् ॥'}


\twolineshloka
{तत एनां बलाद्राजा निजग्राह रहोगताम्}
{वार्यमाणस्तया देव्या विस्फुरन्त्या यथाबलम्}


\twolineshloka
{स तु कामपरीतात्मा तं शापं नान्वबुध्यत}
{माद्रीं मैथुनधर्मेण सोऽन्वगच्छद्बलादिव}


\twolineshloka
{जीवितान्ताय कौरव्य मन्मथस्य वशं गतः}
{शापजं भयमुत्सृज्य विधिना संप्रचोदितः}


\twolineshloka
{तस्य कामात्मनो बुद्धिः साक्षात्कालेन मोहिता}
{संप्रमथ्येन्द्रियग्रामं प्रनष्टा सह चेतसा}


\twolineshloka
{स तया सह संगम्य भार्यया कुरुनन्दनः}
{पाण्डुः परमधर्मात्मा युयुजे कालधर्मणा}


\twolineshloka
{ततो माद्री समालिङ्ग्य राजानं गतचेतसम्}
{मुमोच दुःखजं शब्दं पुनः पुनरतीव हि}


\twolineshloka
{सह पुत्रैस्ततः कुन्ती माद्रीपुत्रौ च पाण्डवौ}
{आजग्मुः सहितास्तत्र यत्र राजा तथागतः}


\twolineshloka
{ततो माद्र्यब्रवीद्राजन्नार्ता कुन्तीमिदं वचः}
{एकैव त्वमिहागच्छ तिष्ठन्त्वत्रैव दारकाः}


\threelineshloka
{तच्छ्रुत्वा वचनं तस्यास्तत्रैवाधाय दरकान्}
{हता}
{हमिति विक्रुश्च सहसैवाजगाम सा}


\twolineshloka
{दृष्ट्वा पाण्डुं च माद्रीं च शयानौ धरणीतले}
{कुन्ती शोकपरीताङ्गी विललाप सुदुःखिता}


\twolineshloka
{रक्ष्यमाणो मया नित्यं वीरः सततमात्मवान्}
{कथं त्वामत्यतिक्रान्तः शापं जानन्वनौकसः}


\twolineshloka
{ननु नाम त्वया माद्रि रक्षितव्यो नराधिपः}
{सा कथं लोभितवती विजने त्वं नराधिपम्}


\twolineshloka
{कथं दीनस्य सततं त्वामासाद्य रहोगताम्}
{तं विचिन्तयतः शापं प्रहर्षः समजायत}


\threelineshloka
{धन्या त्वमसि बाह्लीकि मत्तो भाग्यतरा तथा}
{दृष्टवत्यसि यद्वक्त्रं प्रहृष्टस्य महीपतेः ॥माद्र्युवाच}
{}


\threelineshloka
{विलपन्त्या मया देवि वार्यमाणेन चासकृत्}
{आत्मा न वारितोऽनेन सत्यं दिष्टं चिकीर्षुणा ॥`वैशंपायान उवाच}
{}


\twolineshloka
{तस्यास्तद्वचनं श्रुत्वा कुन्ती शोकाग्निदीपिता}
{पपात सहसा भूमौ छिन्नमूल इव द्रुमः}


\twolineshloka
{निश्चेष्टा पतिता भूमौ मोहेन न चचाल सा}
{तस्मिन्क्षणे कृतस्नानमहताम्बरसंवृतम्}


\twolineshloka
{अलङ्कारकृतं पाण्डुं शयानं शयने शुभे}
{कुन्तीमुत्थाप्य माद्री तु मोहेनाविष्टचेतनाम्}


\twolineshloka
{आर्ये एहीति तां कुन्तीं दर्शयामास कौरव}
{पादयोः पतिता कुन्ती पुनरुत्थाय भूमिपम्}


\twolineshloka
{रक्तचन्दनदिग्धांङ्गं महारजनवाससम्}
{सस्मितेन च वक्त्रेण वदन्तमिव भारतम्}


\twolineshloka
{परिरभ्य ततो मोहाद्विललापाकुलेन्द्रिया}
{माद्री चापि समालिङ्ग्य राजानं विललाप सा}


\twolineshloka
{तं तथा शायिनं पुत्रा ऋषयः सह चारणैः}
{अभ्येत्य सहिताः सर्वे शोकादश्रूण्यवर्तयन्}


\twolineshloka
{अस्तं गतमिवादित्यं संशुष्कमिव सागरम्}
{दृष्ट्वा पाण्डुं नरव्याघ्रं शोचन्ति स्म महर्षयः}


\threelineshloka
{समानशोका ऋषयः पाण्डवाश्च बभूविरे}
{ते समाश्वासिते विप्रैर्विलेपतुरनिन्दिते ॥कुन्त्युवाच}
{}


\twolineshloka
{हा राजन्कस्य नो हित्वा गच्छसि त्रिदशालयम्}
{हा राजन्मम मन्दायाः कथं माद्रीं समेत्य वै}


\twolineshloka
{निधनं प्राप्तवान्राजन्मद्भाग्यपरिसंक्षयात्}
{युधिष्ठिरं भीमसेनमर्जुनं च यमावुभौ}


\twolineshloka
{कस्य हित्वा प्रियान्पुत्रान्प्रयातोऽसि विशांपते}
{नूनं त्वां त्रिदशा देवाः प्रतिनन्दन्ति भारत}


\twolineshloka
{यतो हि तप उग्रं वै चरितं ब्रह्मसंसदि}
{आवाभ्यां सहितो राजन्गमिष्यसि दिवं शुभम्}


\twolineshloka
{आजमीढाजमीढानां कर्मणा चरतां गतिम्}
{ननु नाम सहावाभ्यां गमिष्यामीति यत्त्वया}


\fourlineindentedshloka
{प्रतिज्ञाता कुरुश्रेष्ठ यदाऽस्मि वनमागता}
{आवाभ्यां चैव सहितो गमिष्यसि विशांपते}
{मुहूर्तं क्षम्यतां राजन्द्रक्ष्येऽहं च मुखं तव ॥वैशंपायन उवाच}
{}


\twolineshloka
{विलपित्वा भृशं चैव निःसंज्ञे पतिते भुवि}
{यथा हते मृगे मृग्यौ लुब्धैर्वनगते तथा}


\twolineshloka
{युधिष्ठिरमुखाः सर्वे पाण्डवा वेदपरागाः}
{तेऽभ्यागत्य पितुर्मूले निःसंज्ञाः पतिता भुवि}


\twolineshloka
{पाण्डोः पादौ परिष्वज्य विलपन्ति स्म पाण्डवाः}
{हा विनष्टाः स्म तातेति हा अनाथा भवामहे}


\twolineshloka
{त्वद्विहीना महाप्राज्ञ कथं जीवाम बालकाः}
{लोकनाथस्य पुत्राः स्मो न सनाथा भवामहे}


\twolineshloka
{क्षणेनैव महाराज अहो लोकस्य चित्रता}
{नास्मद्विधा राजपुत्रा अधन्याः सन्ति भारत}


\twolineshloka
{त्वद्विनाशाच्च राजेन्द्र राज्यप्रस्खलनात्तदा}
{पाण्डवाश्च वयं सर्वे प्राप्ताः स्म व्यसनं महत्}


\threelineshloka
{किं करिष्यामहे राजन्कर्तव्यं च प्रसीदताम्}
{भीमसेन उवाच}
{हित्वा राज्यं च भोगांश्च शतशृङ्गनिवासिना}


\twolineshloka
{त्वया लब्धाः स्म राजेन्द्र महता तपसा वयम्}
{हित्वा मानं वनं गत्वा स्वयमाहृत्य भक्षणम्}


\twolineshloka
{शाकमूलफलैर्वन्यैर्भरणं वै त्वया कृतम्}
{पुत्रानुत्पाद्य पितरो यमिच्छ्ति महातम्नः}


\twolineshloka
{त्रिवर्गफलमिच्छन्तस्तस्य कालोऽयमागतः}
{अभुक्त्वैव फलं राजन्गन्तुं नार्हसि भारत}


\twolineshloka
{इत्येवमुक्त्वा पितरं भीमोऽपि विललाप ॥अर्जुन उवाच}
{}


\twolineshloka
{प्रनष्टं भारतं वंशं पाण्डुना पुनरुद्धृतम्}
{तस्मिंस्तदा वनगते नष्टं राज्यमराजकम्}


\twolineshloka
{पुनर्निःसारितं क्षत्रं पाण्डुपुत्रैश्च पञ्चभिः}
{एतच्छ्रुत्वाऽनुमोदित्वा गन्तुमर्हसि शङ्कर}


\threelineshloka
{इत्येवमुक्त्वा पितरं विललाप धनञ्जयः}
{यमावूचतुः}
{दुःसहं च तपः कृत्वा लब्ध्वा नो भरतर्षभ}


\twolineshloka
{पुत्रलाभस्य महतः शुश्रूषादिफलं त्वया}
{न चावाप्तं किंचिदेव पुरा दशरथो यथा}


\twolineshloka
{एवमुक्त्वा यमौ चापि विलेपतुरथातुरौ ॥' कुन्त्युवाच}
{}


\twolineshloka
{अहं ज्येष्ठा धर्मपत्नी ज्येष्ठं धर्मफलं मम}
{अवश्यं भाविनो भावान्मा मां माद्रि निवर्तय}


\twolineshloka
{अन्विष्यामीह भर्तारमहं प्रेतवशं गतम्}
{उत्तिष्ठ त्वं विसृज्यैनमिमान्रक्षस्व दारकान्}


\threelineshloka
{`अवाप्य पुत्रांल्लब्धार्थान्वीरपत्नीत्वमर्थये}
{वैशंपायन उवाच}
{मद्रराजसुता कुन्तीमिदं वचनमब्रवीत् ॥'}


\twolineshloka
{अहमेवानुयास्यामि भर्तारमपलापिनम्}
{न हि तृप्ताऽस्मि कामानां ज्येष्ठा मामनुमन्यताम्}


\twolineshloka
{मां चाभिगम्य क्षीणोऽयं कामाद्भरतसत्तमः}
{समुच्छिद्यामि तत्कामं कथं नु यमसादने}


\twolineshloka
{`मम हतोर्हि राजाऽयं दिवं राजर्पिसत्तमः}
{न चैव तादृशी बुद्धिर्बान्धवाश्च न तादृशाः}


\twolineshloka
{न चोत्सहे धारयितुं प्राणान्भर्त्रा विना कृता}
{तस्मात्तमनुयास्यामि यान्तं वैवस्वतक्षयम्}


\twolineshloka
{वर्तेयं न समां वृत्तिं जात्वहं न सुतेषु ते}
{तथाहि वर्तमानां मामधर्मः संस्पृशेन्महान्}


\twolineshloka
{तस्मान्मे सुतयोर्देवि वर्तितव्यं स्वपुत्रवत्}
{अन्वेष्यामि च भर्तारं व्रजन्तं यमसादनम् ॥'}


\twolineshloka
{मां हि कामयमानोऽयं राजा प्रेतवशं गतः}
{राज्ञः शरीरेण सह मामपीदं कलेवरम्}


\fourlineindentedshloka
{दग्धव्यं सुप्रतिच्छन्नं त्वेतदार्ये प्रियं कुरु}
{दारकेष्वप्रमत्ता त्वं भवेश्चाभिहिता मया}
{अतोऽहं न प्रपश्यामि संदेष्टव्यं हितं तव ॥`वैशंपायन उवाच}
{}


\twolineshloka
{ऋषयस्तान्समाश्वास्य पाण्डवान्सत्यविक्रमान्}
{ऊचुः कुन्तीं च माद्रीं च समाश्वास्य तपस्विनः}


\twolineshloka
{सुभगे बालपुत्रा तु न मर्तव्यं तथंचन}
{पाण्डवांश्चापि नेष्यामः कुरुराष्ट्रं परन्तपान्}


\twolineshloka
{अधर्मेष्वर्थजातेषु धृतराष्ट्रश्च लोभवान्}
{स कदाचिन्न वर्तेत पाण्डवेषु यथाविधि}


\twolineshloka
{कुन्त्याश्च वृष्णयो नाथाः कुन्तिभोजस्तथैव च}
{माद्र्याश्च बलिनांश्रेष्ठः शल्यो भ्राता महारथः}


\twolineshloka
{भर्त्रा तु मरणं सार्धं फलवन्नात्र संशयः}
{युवाभ्यां दुष्करं चैतद्वदन्ति द्विजपुङ्गवाः}


\twolineshloka
{मृते भर्तरि साध्वी स्त्री ब्रह्मचर्यव्रते स्थिता}
{यमैश्च नियमैः पूता मनोवाक्कायजैः शुभा}


\twolineshloka
{भर्तारं चिन्तयन्ती सा भर्तारं निस्तरेच्छुभा}
{तारितश्चापि भर्ता स्यादात्मा पुत्रस्तथैव च}


\twolineshloka
{तस्माञ्जीवितमेवैतद्युवयोर्विद्म शोभनम् ॥कुन्त्युवाच}
{}


\twolineshloka
{यथा पाण्डोस्तु निर्देशस्तथा विप्रगणस्य च}
{आज्ञा शिरसि निक्षिप्ता करिष्यामि च तत्तथा}


\threelineshloka
{यदाद्दुर्भगवन्तोऽपि तन्मन्ये शोभनं परम्}
{भर्तुश्च मम पुत्राणामात्मनश्च न संशयः ॥माद्र्युवाच}
{}


\twolineshloka
{कुन्ती समर्था पुत्राणां योगक्षेमस्य धारणे}
{अस्या हि न समा बुद्ध्या यद्यपि स्यादरुन्धती}


\twolineshloka
{कुन्त्याश्च वृष्णयो नाथाः कुन्तिभोजस्तथैव च}
{नाहं त्वमिव पुत्राणां समर्था धारणे तथा}


\twolineshloka
{साऽहं भर्तारमन्विष्ये संतृप्ता नापि भोगतः}
{भर्तृलोकस्य तु ज्येष्ठा देवी मामनुमन्यताम्}


\threelineshloka
{धर्मज्ञस्य कृतज्ञस्य सत्यसन्धस्य धीमतः}
{पादौ परिचरिष्यामि तथार्याऽद्यानुमन्यताम् ॥वैशंपायन उवाच}
{}


\twolineshloka
{एवमुक्त्वा तदा राजन्मद्रराजसुता शुभा}
{ददौ कुन्त्यै यमौ माद्री शिरसाऽभिप्रणम्य च}


\twolineshloka
{अभिवाद्य महर्षीन्सा परिष्वज्य च पाण्डवान्}
{मूर्ध्न्युपाघ्राय बहुशः पार्थानात्मसुतौ तदा}


\twolineshloka
{हस्ते युधिष्ठिरं गृह्य माद्री वाक्यमभाषत}
{कुन्ती माता अहं धात्री युष्माकं तु पिता मृतः}


\twolineshloka
{युधिष्ठिरः पिता ज्येष्ठश्चतुर्णां धर्मतः सदा}
{वृद्धाद्युपासनासक्ताः सत्यधर्मपरायणाः}


\threelineshloka
{तादृशा न विनश्यन्ति नैव यान्ति पराभवम्}
{तस्मात्सर्वे कुरुध्वं वै गुरुवृत्तिमतन्द्रिताः ॥वैशंपायन उवाच}
{}


\threelineshloka
{ऋषीणां च पृथायाश्च नमस्कृत्य पुनःपुनः}
{आयासकृपणा माद्री प्रत्युवाच पृथां तदा ॥माद्र्युवाच}
{}


\twolineshloka
{ऋषीणां संनिधावेषां यथा वागभ्युदीरिता}
{दिदृक्षमाणायाः स्वर्गं न ममैषा वृथा भवेत्}


\twolineshloka
{धन्या त्वमसि वार्ष्णेयि नास्ति स्त्री सदृशी त्वया}
{वीर्यं तेजश्च योगं च माहात्म्यं च यशस्विनाम्}


\twolineshloka
{कुन्ति द्रक्ष्यसि पुत्राणां पञ्चानाममितौजसाम्}
{आर्या चाप्यभिवाद्या च मम पूज्या च सर्वतः}


\twolineshloka
{ज्येष्ठा वरिष्ठा त्वं देवि भूषिता स्वगुणैः शुभैः}
{अभ्यनुज्ञातुमिच्छामि त्वया यावनन्दिनि}


\threelineshloka
{धर्मं स्वर्गं च कीर्तिं च त्वत्कृतेऽहमवाप्नुयाम्}
{यथा तथा विधत्स्वेह मा च कार्षीर्विचारणां ॥वैशंपायन उवाच}
{}


\twolineshloka
{बाष्पसंदिग्धया वाचा कुन्त्युवाच यशस्विनी}
{अनुज्ञाताऽसि कल्याणि त्रिदिवे सङ्गमोऽस्तु ते}


\threelineshloka
{भर्त्रा सह विशालिक्षि क्षिप्रमद्यैव भामिनि}
{संगतास्वर्गलोके त्वं रमेथाः शाश्वतीः समाः ॥वैशंपायन उवाच}
{}


\twolineshloka
{ततः पुरोहितः स्नात्वा प्रेतकर्मणि पारगः}
{हिरण्यशकलानाज्यं तिलं दधि च तण्डुलान्}


\twolineshloka
{उदकुम्भांश्च परशुं समानीय तपस्विभिः}
{अश्वमेधाग्निमाहृत्य यथान्यायं समन्ततः}


\twolineshloka
{काश्यपः कारयामास पाण्डोः प्रेतस्य तां क्रियाम्}
{पुरोहितोक्तविधिना पाण्डोः पुत्रो युधिष्ठिरः}


\twolineshloka
{तेनाग्निनाऽदहत्पाण्डुं कृत्वा चापि क्रियास्तदा}
{रुदञ्छोकाभिसंतप्तः पपात भुवि पाण्डवः}


\twolineshloka
{ऋषीन्पुत्रान्पृथां चैव विसृज्य च नृपात्मज}
{'नमस्कृत्य चिताग्निस्थं धर्मपत्नी नरर्षभम्}


% Check verse!
मद्रराजसुता तूर्णमन्वारोहद्यशस्विनी
\twolineshloka
{`अहताम्बरसंवीतो भ्रातृभिः सहितोऽनघः}
{उदकं कृतवांस्तत्र पुरोहितमते स्थितः}


\twolineshloka
{अर्हतस्तस्य कृत्यानि शतशृङ्गनिवासिनः}
{तापसा विधिवच्चक्रुश्चारणा ऋषिभिः सह}


\chapter{अध्यायः १३५}
\twolineshloka
{वैशंपायन उवाच}
{}


\threelineshloka
{पाण्डोरुपरमं दृष्ट्वा देवकल्पा महर्षयः}
{ततो मन्त्रविदः सर्वे मन्त्रयांचक्रिरे मिथः ॥तापसा ऊचुः}
{}


\twolineshloka
{हित्वा राज्यं च राष्ट्रं च स महात्मा महायशाः}
{अस्मिंस्थाने तपस्तप्त्वा तापसाञ्शरणं गतः}


\twolineshloka
{स जातमात्रान्पुत्रांश्च दारांश्च भवतामिह}
{प्रादायोपनिधिं राजा पाण्डुः स्वर्गमितो गतः}


\threelineshloka
{तस्येमानात्मजान्देहं भार्यां च सुमहात्मनः}
{स्वराष्ट्रं गृह्य गच्छामो धर्म एष हि नः स्मृतः ॥वैशंपायन उवाच}
{}


\twolineshloka
{ते परस्परमामन्त्र्य देवकल्पा महर्षयः}
{पाण्डोः पुत्रान्पुरस्कृत्य नगरं नागसाह्वयम्}


\twolineshloka
{उदारमनसः सिद्धा गमने चक्रिरे मनः}
{भीष्माय पण्डवान्दातुं धृतराष्ट्राय चैव हि}


\twolineshloka
{तस्मिन्नेव क्षणे सर्वे तानादाय प्रतस्थिरे}
{पाण्डोर्दारांश्च पुत्रांश्च शरीरे ते च तापसाः}


\twolineshloka
{सुखिनी सा पुरा भूत्वा सततं पुत्रवत्सला}
{प्रपन्ना दीर्घमध्वानं संक्षिप्तं तदमन्यत}


\twolineshloka
{सा त्वदीर्घेण कालेन संप्राप्ता कुरुजाङ्गलम्}
{वर्धमानपुरद्वारमाससाद यशस्विनी}


\twolineshloka
{द्वारिणं तापसा ऊचू राजानं च प्रकाशय}
{ते तु गत्वा क्षणेनैव सभायां विनिवेदिताः}


\twolineshloka
{तं चारणसहस्राणां मुनीनामागमं तदा}
{श्रुत्वा नागपुरे नॄणां विस्मयः समपद्यत}


\twolineshloka
{मुहूर्तोदित आदित्ये सर्वे बालपुरस्कृताः}
{सदारास्तापसान्द्रष्टुं निर्ययुः पुरवासिनः}


\twolineshloka
{स्त्रीसङ्घाः क्षत्रसङ्घाश्च यानसङ्घसमास्थिताः}
{ब्राह्मणैः सह निर्जग्मुर्ब्राह्मणानां च योषितः}


\twolineshloka
{तथा विट्शूद्रसङ्घानां महान्यतिकरोऽभवत्}
{न कश्चिदकरोदीर्ष्यामभवन्धर्मबुद्धयः}


\threelineshloka
{तथा भीष्मः शान्तनवः सोमदत्तो}
{ञथ बाह्लिकः}
{प्रज्ञाचक्षुश्च राजर्षिः क्षत्ता च विदुरः स्वयम्}


\twolineshloka
{सा च सत्यवती देवी कौसल्या च यशस्विनी}
{राजदारैः परिवृता गान्धारी चापि निर्ययौ}


\twolineshloka
{धृतराष्ट्रस्य दायादा दुर्योधनपुरोगमाः}
{भूषिता भूषणैश्चित्रैः शतसङ्ख्या विनिर्ययुः}


\twolineshloka
{तान्महर्षिगणान्दृष्ट्वा शिरोभिरभिवाद्य च}
{उपोपविविशुः सर्वे कौरव्याः सपुरोहिताः}


\twolineshloka
{तथैव शिरसा भूमावभिवाद्य प्रणम्य च}
{उपोपविविशुः सर्वे पौरजानपदा अपि}


\twolineshloka
{तमकूजमभिज्ञाय जनौघं सर्वशस्तदा}
{पूजयित्वा यथान्यायं पाद्येनार्घ्येण च प्रभो}


\threelineshloka
{भीष्मो राज्यं च राष्ट्रं च महर्षिभ्यो न्यवेदयत्}
{तेषामथो वृद्धतमः प्रत्युत्थाय जटाजिनी}
{ऋषीणां मतमाज्ञाय महर्षिरिदमब्रवीत्}


\twolineshloka
{यः स कौरव्यदायादः पाण्डुर्नाम नराधिपः}
{कामभोगान्परित्यज्य शतशृङ्गमितो गतः}


\twolineshloka
{`राजा भोगान्परित्यज्य तपस्वी संबभूव ह}
{स यथोक्तं तपस्तेपे पत्रमूलफलाशनः}


\twolineshloka
{पत्नीभ्यां सह धर्मात्मा संचित्कालमतन्द्रितः}
{तेन वृत्तसमाचारैस्तपसा च तपस्विनः}


\twolineshloka
{तोषितास्तापसास्तत्र शतशृङ्गनिवासिनः}
{स्वर्गलोकं गन्तुकामं तापसाः संनिवार्य तम्}


\twolineshloka
{उद्यन्तं सह पत्नीभ्यां विप्रा वचनमब्रुवन्}
{अनपत्यस्य राजेन्द्र पुण्या लोका न सन्ति ते}


\twolineshloka
{तस्माद्धर्मं च वायुं च महेन्द्रं च तथाऽश्विनौ}
{आराधयस्व राजेन्द्र पत्नीभ्यां सह देवताः}


\twolineshloka
{प्रीताः पुत्रान्प्रदास्यन्ति ऋणमुक्तो भविष्यसि}
{तपसा दिव्यचक्षुष्ट्वात्पश्यामस्ते तथा सुतान्}


\twolineshloka
{अस्माकं वचनं श्रुत्वा देवानाराधयत्तदा}
{'ब्रह्मचर्यव्रतस्थस्य तस्य दिव्येन हेतुना}


\twolineshloka
{साक्षाद्धर्मादयं पुत्रस्तत्र जातो युधिष्ठिरः}
{तथैनं बलिनां श्रेष्ठं तस्य राज्ञो महात्मनः}


\twolineshloka
{मातरिश्वा ददौ पुत्रं भीमं नाम महाबलम्}
{पुरन्दरादयं जज्ञे कुन्त्यां सत्यपराक्रमः}


\twolineshloka
{`अस्मिञ्जाते महेष्वासे पृथामिन्द्रस्तदाऽब्रवीत्}
{मत्प्रसादादयं जातः कुन्ति सत्यपराक्रमः}


\twolineshloka
{अजेयानपि जेताऽरीन्देवतादीन्न संशयः}
{'यस्य कीर्तिर्महेष्वासान्सर्वानभिभविष्यति}


\twolineshloka
{`युधिष्ठिरो राजसूयं भ्रातृवीर्यादवाप्स्यति}
{एष जेता मनुष्यांश्च सर्वान्गन्धर्वराक्षसान्}


\twolineshloka
{एष दुर्योधनादीनां कौरवाणां च जेष्यति}
{वीरस्यैतस्य विक्रान्तैर्धर्मपुत्रो युधिष्ठिरः}


\twolineshloka
{यक्ष्यते राजसूयाद्यैर्धर्म एव परः सदा}
{'यौ तु माद्री महेष्वासावसूत पुरुषोत्तमौ}


\twolineshloka
{अश्विभ्यां पुरुषव्याघ्राविमौ तावपि तिष्ठतः}
{`नकुलः सहदेवश्च तावप्यमिततेजसौ ॥'}


\twolineshloka
{चरता धर्मनित्येन वनवासं यशस्विना}
{एष पैतामहो वंशः पाण्डुना पुनरुद्धृतः}


\twolineshloka
{पुत्राणां जन्म वृद्धिं च वैदिकाध्ययनानि च}
{पश्यन्तः सततं पाण्डोः परां प्रीतिमवाप्स्यथ}


\twolineshloka
{वर्तमानः सतां वृत्ते पुत्रलाभमवाप च}
{पितृलोकं गतः पाण्डुरितः सप्तदशेऽहनि}


\twolineshloka
{तं चितागतमाज्ञाय वैश्वानरमुखे हुतम्}
{प्रविष्टा पावकं माद्री हित्वा जीवितमात्मनः}


\twolineshloka
{सा गता सह तेनैव पतिलोकमनुव्रता}
{तस्यास्तस्य च यत्कार्यं क्रियतां तदनन्तरम् ॥'}


\twolineshloka
{पृथां च शरणं प्राप्तां पाण्डवांश्च यशस्विनः}
{यथावदनुमन्यन्तां धर्मो ह्येष सनातनः}


\twolineshloka
{इमे तयोः शरीरे द्वे पुत्राश्चेमे तयोर्वराः}
{क्रियाभिरनुगृह्यन्तां सह मात्रा परंतपाः}


\threelineshloka
{प्रेतकार्ये निवृत्ते तु पितृमेधं महायशाः}
{लभतां सर्वधर्मज्ञः पाण्डुः कुरुकुलोद्वहः ॥वैशंपायन उवाच}
{}


\twolineshloka
{एवमुक्त्वा कुरून्सर्वान्कुरूणामेव पश्यताम्}
{क्षणेनान्तर्हिताः सर्वे तापसा गुह्यकैः सह}


\twolineshloka
{गन्धर्वनगराकारं तथैवान्तर्हितं पुनः}
{ऋषिसिद्धगणं दृष्ट्वा विस्मयं ते परं ययुः}


\chapter{अध्यायः १३६}
\twolineshloka
{धृतराष्ट्र उवाच}
{}


\twolineshloka
{पाण्डोर्विदुर सर्वाणि प्रेतकार्याणि कारय}
{राजवद्राजसिंहस्य माद्र्याश्चैव विशेषतः}


\twolineshloka
{पशून्वासांसि रत्नानि धनानि विविधानि च}
{पाण्डोः प्रयच्छ माद्र्याश्च येभ्यो यावच्च वाञ्छितम्}


\twolineshloka
{यथा च कुन्ती सत्कारं कुर्यान्माद्र्यास्तथा कुरु}
{यथा न वायुर्नादित्यः पश्येतां तां सुसंवृताम्}


\threelineshloka
{न शोच्यः पाण्डुरनघः प्रशस्यः स नराधिपः}
{यस्य पञ्च सुता वीरा जाताः सुरसुतोपमाः ॥वैशंपायन उवाच}
{}


\twolineshloka
{विदुरस्तं तथेत्युक्त्वा भीष्मेण सह भारत}
{पाण्डुं संस्कारयामास देशे परमपूजिते}


\twolineshloka
{ततस्तु नगरात्तूर्णमाज्यगन्धपुरस्कृताः}
{निर्हृताः पावका दीप्ताः पाण्डो राजन्पुरोहितैः}


\twolineshloka
{अथैनामार्तवैः पुष्पैर्गन्धैश्च विविधैर्वरैः}
{शिबिकां तामलङ्कृत्य वाससाऽऽच्छाद्य सर्वशः}


\twolineshloka
{तां तथा शोभितां माल्यैर्वासोभिश्च महाधनैः}
{अमात्या ज्ञातयश्चैनं सुहृदश्चोपतस्थिरे}


\twolineshloka
{नृसिंहं नरयुक्तेन परमालङ्कृतेन तम्}
{अवहन् यानमुख्येन सह माद्र्या सुसंवृतम्}


\twolineshloka
{पाण्डुरेणातपत्रेण चामरव्यजनेन च}
{सर्ववादित्रनादैश्च समलञ्चक्रिरे ततः}


\twolineshloka
{रत्नानि चाप्युपादाय बहूनि शतशो नराः}
{प्रददुः काङ्क्षमाणेभ्यः पाण्डोस्तस्यौर्ध्वदेहिके}


\twolineshloka
{अथ च्छत्राणि शुभ्राणि चामराणि बृहन्ति च}
{आजह्रुः कौरवस्यार्थे वासांसि रुचिराणि च}


\twolineshloka
{याजकैः शुक्लवासोभिर्हूयमाना हुताशनाः}
{अगच्छन्नग्रतस्तस्य दीप्यमानाः स्वलङ्कृताः}


\twolineshloka
{ब्राह्मणाः क्षत्रिया वैश्याः शूद्राश्चैव सहस्रशः}
{रुदन्तः शोकसंतप्ता अनुजग्मुर्नराधिपम्}


\twolineshloka
{अयमस्मानपाहाय दुःखे चाधाय शाश्वते}
{कृत्वा चास्माननाथांश्च क्व यास्यति नराधिपः}


\twolineshloka
{क्रोशन्तः पाण्डवाः सर्वे भीष्मो विदुर एव च}
{`बाह्लीकः सोमदत्तश्च तथा भूरिश्रवा नृपः}


\twolineshloka
{अन्योन्यं वै समाश्लिष्य अनुजग्मुः सहस्रशः}
{'रमणीये वनोद्देशे गङ्गातीरे समे शुभे}


\twolineshloka
{न्यासयामासुरथं तां शिबिकां सत्यवादिनः}
{सभार्यस्य नृसिंहस्य पाण्डोरक्लिष्टकर्मणः}


\twolineshloka
{ततस्तस्य शरीरं तु सर्वगन्धाधिवासितम्}
{शुचिकालीयकादिग्धं दिव्यचन्दनरूषितम्}


\twolineshloka
{पर्यषिञ्चञ्जलेनाशु शातकुम्भमयैर्घटैः}
{चन्दनेन च शुक्लेन सर्वतः समलेपयन्}


\twolineshloka
{कालागुरुविमिश्रेण तथा तुङ्गरसेन च}
{अथैनं देशजैः शुक्लैर्वासोभिः समयोजयन्}


\twolineshloka
{संछन्नः स तु वासोभिर्जीवन्निव नराधिपः}
{शुशुभे स नव्याघ्रो महार्हशयनोचितः}


\twolineshloka
{`हयमेधाग्निना सर्वे याजकाः सपुरोहिताः}
{वेदोक्तेन विधानेन क्रियांचक्रुः समन्त्रकम् ॥'}


\twolineshloka
{याजकैरभ्यनुज्ञाते प्रेतकर्मण्यनिष्ठिते}
{घृतावसिक्तं राजानं सह माद्र्या स्वलङ्कृतम्}


\twolineshloka
{तुङ्गपद्मकमिश्रेण चन्दनेन सुगन्धिना}
{`सरलं देवदारुं च गुग्गुलं लाक्षया सह}


\twolineshloka
{हरिचन्दनकाष्ठैश्च हरिबेरैरुशीरकैः}
{'अन्यैश्च विविधैर्गन्धैर्विधिना समदाहयन्}


\twolineshloka
{ततस्तयोः शरीरे द्वे दृष्ट्वा मोहवशं गता}
{हाहा पुत्रेति कौसल्या पपात सहसा भुवि}


\twolineshloka
{तां प्रेक्ष्य पतितामार्तां पौरजानपदो जनः}
{रुरोद दुःखसंतप्तो राजभक्त्या कृपाऽन्वितः}


\twolineshloka
{कुन्त्याश्चैवार्तनादेन सर्वाणि च विचुक्रुशुः}
{मानुषैः सह भूतानि तिर्यग्योनिगतान्यपि}


\twolineshloka
{तथा भीष्मः शान्तनवो विदुरश्च महामतिः}
{सर्वशः कौरवाश्चैव प्राणदन्भृशदुःखिताः}


\twolineshloka
{ततो भीष्मोऽथ विदुरो राजा च सह पाण्डवैः}
{उदकं चक्रिरे तस्य सर्वाश्च कुरुयोषितः}


\twolineshloka
{चुक्रुशुः पाण्डवाः सर्वे भीष्मः शान्तनवस्तथा}
{विदुरो ज्ञातयश्चैव चक्रुश्चाप्युदकक्रियाः}


\twolineshloka
{कृतोदकांस्तानादाय पाण्डवाञ्छोककर्शितान्}
{सर्वाः प्रकृतयो राजञ्शोचमाना न्यवारयन्}


\twolineshloka
{यथैव पाण्डवा भूमौ सुषुपुः सह बान्धवैः}
{तथैव नागरा राजञ्शिश्यिरे ब्राह्मणादयः}


\twolineshloka
{तद्गतानन्दमस्वस्थमाकुमारमहृष्टवत्}
{बभूव पाण्डवैः सार्धं नगरं द्वादश क्षपाः}


\chapter{अध्यायः १३७}
\twolineshloka
{वैशंपायन उवाच}
{}


\twolineshloka
{ततः क्षत्ता च भीष्मश्च व्यासो राजा च बन्धुभिः}
{ददुः श्राद्धं तदा पाण्डोः स्वधामृतमयं तदा}


\threelineshloka
{`पुरोहितसहायास्ते यथान्यायमकुर्वत}
{'कुरूंश्च विप्रमुख्यांश्च भोजयित्वा सहस्रशः}
{रत्नौघान्द्विजमुख्येभ्यो दत्त्वा ग्रामवरांस्तथा}


\twolineshloka
{कृतशौचांस्ततस्तांस्तु पाण्डवान्भरतर्षभान्}
{आदाय विविशुः सर्वे पुरं वारणसाह्वयम्}


\twolineshloka
{सततं स्मानुशोचन्तस्तमेव भरतर्षभम्}
{पौरजानपदाः सर्वे मृतं स्वमिव बान्धवम्}


\twolineshloka
{श्राद्धावसाने तु तदा दृष्ट्वा तं दुःखितं जनम्}
{संमूढां दुःखशोकार्तां व्यासो मातरमब्रवीत्}


\twolineshloka
{अतिक्रान्तसुखाः कालाः पर्युपस्थितदारुणाः}
{श्वःश्वः पापिष्ठदिवसाः पृथिवी गतयौवना}


\twolineshloka
{बहुमायासमाकीर्णो नानादोषसमाकुलः}
{लुप्तधर्मक्रियाचारो घोरः कालो भविष्यति}


\twolineshloka
{कुरूणामनयाच्चापि पृथिवी न भविष्यति}
{गच्छ त्वं योगमास्थाय युक्ता वस तपोवने}


\twolineshloka
{माद्राक्षीस्त्वं कुलस्यास्य घोरं संक्षयमात्मनः}
{तथेति समनुज्ञाय सा प्रविश्याब्रवीत्स्नुषाम्}


\twolineshloka
{अम्बिक तव पौत्रस्य दुर्नयात्किल भारताः}
{सानुबन्धा विनङ्क्ष्यन्ति पौराश्चैवेति नः श्रुतम्}


\twolineshloka
{तत्कौसल्यामिमामार्तां पुत्रशोकाभिपीडिताम्}
{वनमादाय भद्रं ते गच्छावो यदि मन्यसे}


\twolineshloka
{तथेत्युक्ता त्वम्बिकया भीष्ममामन्त्र्य सुव्रता}
{वनं ययौ सत्यवती स्नुषाभ्यां सह भारत}


\twolineshloka
{ताः सुघोरं तपस्तप्त्वा देव्यो भरतसत्तम ॥देहं त्यक्त्वा महाराज गतिमिष्टां ययुस्तदा ॥वैशंपायन उवाच}
{}


\twolineshloka
{अथाप्तवन्तो वेदोक्तान्संस्कारान्पाण्डवास्तदा}
{संव्यवर्धन्त भोगांस्ते भुञ्जानाः पितृवेश्मनि}


\twolineshloka
{धार्तराष्ट्रैश्च सहिताः क्रीडन्तो मुदिताः सुखम्}
{बालक्रीडासु सर्वासु विशिष्टास्तेजसाऽभवन्}


\twolineshloka
{जवे लक्ष्याभिहरणे भोज्ये पांसुविकर्षणे}
{धार्तराष्ट्रान्भीमसेनः सर्वान्स परिमर्दति}


\twolineshloka
{हर्षात्प्रक्रीडमानांस्तान् गृह्य राजन्निलीयते}
{शिरःसु विनिगृह्यैतान्योजयामास पाण्डवैः}


\twolineshloka
{शतमेकोत्तरं तेषां कुमाराणां महौजसाम्}
{एक एव निगृह्णाति नातिकृच्छ्राद्वृकोदरः}


\twolineshloka
{कचेषु च निगृह्यैनान्विनिहत्य बलाद्बली}
{चकर्ष क्रोशतो भूमौ घृष्टजानुशिरोंसकान्}


\twolineshloka
{दश बालाञ्जले क्रीडन्भुजाभ्यां परिगृह्य सः}
{आस्ते स्म सलिले मग्नो मृतकल्पान्विमुञ्चति}


\twolineshloka
{फलानि वृक्षमारुह्य विचिन्वन्ति च ये तदा}
{तदा पादप्रहारेण भीमः कम्पयते द्रुमान्}


\twolineshloka
{प्रहारवेगाभिहता द्रुमा व्याघूर्णितास्ततः}
{सफलाः प्रपतन्ति स्म द्रुमात्स्रस्ताः कुमारकाः}


\twolineshloka
{`केचिद्भग्नशिरोरस्काः केचिद्भग्नकटीमुखाः}
{निपेतुर्भ्रातरः सर्वे भीमसेनभुजार्दिताः ॥'}


\twolineshloka
{न ते नियुद्धे न जवे न योग्यासु कदाचन}
{कुमारा उत्तरं चक्रुः स्पर्धमाना वृकोदरम्}


\twolineshloka
{एवं स धार्तराष्ट्रांश्च स्पर्धमानो वृकोदरः}
{अप्रियेऽतिष्ठदत्यन्तं बाल्यान्न द्रोहचेतसा}


\twolineshloka
{ततो बलमतिख्यातं धार्तराष्ट्रः प्रतापवान्}
{भीमसेनस्य तज्ज्ञात्वा दुष्टभावमदर्शयत्}


\twolineshloka
{तस्य धर्मादपेतस्य पापानि परिपश्यतः}
{मोहादैश्वर्यलोभाच्च पापा मतिरजायत}


\twolineshloka
{अयं बलवतां श्रेष्ठः कुन्तीपुत्रो वृकोदरः}
{मध्यमः कुन्तिपुत्राणां निकृत्या सन्निगृह्यतां}


\twolineshloka
{प्राणवान्विक्रमी चैव शौर्येण महताऽन्वितः}
{स्पर्धते चापि सहितानस्मानेको वृकोदरः}


\twolineshloka
{तं तु सुप्तं पुरोद्याने गङ्गायां प्रक्षिपामहे}
{अथ तस्मादवरजं श्रेष्ठं चैव युधिष्ठिरम्}


\threelineshloka
{प्रसह्य बन्धने बद्ध्वा प्रशासिष्ये वसुन्धराम्}
{एवं स निश्चयं पापः कृत्वा दुर्योधनस्तदा}
{नित्यमेवान्तरप्रेक्षी भीमस्यासीन्महात्मनः}


\twolineshloka
{ततो जलविहारार्थं कारयामास भारत}
{चैलकम्बलवेश्मानि विचित्राणि महान्ति च}


\twolineshloka
{सर्वकामैः सुपूर्णानि पताकोच्छ्रायवन्ति च}
{तत्र संजनयामास नानागाराण्यनेकशः}


\twolineshloka
{उदकक्रीडनं नाम कारयामास भारत}
{प्रमाणकोट्यां तं देशं स्थलं किंचिदुपेत्यह}


\twolineshloka
{`क्रीडावसाने ते सर्वे शुचिवस्त्राः स्वलङ्कृताः}
{सर्वकामसमृद्धं तदन्नं बुभुजिरे शनैः}


\twolineshloka
{दिवसान्ते परिश्रान्ता विहृत्य च कुरूद्वहाः}
{विहारावसथेष्वेव वीरा वासमरोचयन्}


\twolineshloka
{खिन्नस्तु बलवान्भीमो व्यायामाभ्यधिकस्तदा}
{वाहयित्वा कुमारांस्ताञ्जलक्रीडागतान्विभुः}


\twolineshloka
{प्रमाणकोट्यां वासार्थं सुष्वापारुह्य तत्स्थलम्}
{शीतं वासं समासाद्य शान्तो मदविमोहितः}


\twolineshloka
{निश्चेष्टः पाण्डवो राजन्सुष्वाप मृतवत्क्षितौ}
{ततो बद्ध्वा लतापाशैर्भीमं दुर्योधनः शनैः}


\twolineshloka
{प्रमाणकोट्यां संसुप्तं गङ्गायां प्राक्षिपज्जले}
{ततः प्रबुद्धः कौन्तेयः सर्वान्संछिद्य बन्धनान्}


\twolineshloka
{उदतिष्ठद्बलाद्भूयो भीमः प्रहरतां वरः}
{स विमुक्तो महातेजा नाज्ञासीत्तेन तत्कृतम्}


\threelineshloka
{पुनर्निद्रावशं प्राप्तस्तत्रैव प्रास्वपद्बली}
{अर्धरात्र्यां व्यतीतायामुत्तस्थुः कुरुपाण्जवाः}
{दुर्योधनस्तु कौन्तेयं दृष्ट्वा निर्वेदमभ्यगात्}


\twolineshloka
{सुप्तं चापि पुनः सर्पैस्तीक्ष्णदंष्ट्रैर्महाविषैः}
{कुपितैर्दंशयामास सर्वेष्वेवाङ्गसन्धिषु}


\twolineshloka
{दंष्ट्राश्च दंष्ट्रिणां मर्मस्वपि तेन निपातिताः}
{त्वचं न चास्य बिभिदुः सारत्वात्पृथुपक्षसः}


\twolineshloka
{प्रबुद्धो भीसेनस्तान्सर्वान्सर्पानपोथयत्}
{सारथिं चास्य दयितमपहस्तेन जघ्निवान्}


\twolineshloka
{तथान्यदिवसे राजन्हन्तुकामोऽत्यमर्षणः}
{वलनेन सहामन्त्र्य सौबलस्य मते स्थितः}


\twolineshloka
{भोजने भीमसेनस्य ततः प्राक्षेपयद्विषम्}
{कालकूटं विषं तीक्ष्णं संभृतं रोमहर्षणम्}


\twolineshloka
{तच्चापि भुक्त्वाऽजरदॉयदविकारो वृकोदरः}
{विकारं नाभ्यजनयत्सुतीक्ष्णमपि तद्विषम्}


\twolineshloka
{भीमसंहननो भीमस्त्समादजरयद्विषम्}
{ततोऽन्यदिवसे राजन्हन्तुकामो वृकोदरम्}


\twolineshloka
{सौबलेन सहायेन धार्तराष्ट्रोऽभ्यचिन्तयत्}
{चिन्तयन्नालभन्निद्रां दिवारात्रमतन्द्रितः}


\twolineshloka
{एवं दुर्योधनः कर्णः शकुनिश्चापि सौबलः}
{अनेकैरप्युपायैस्ताञ्जिघांसन्ति स्म पाण्डवान्}


\threelineshloka
{वैश्या पुत्रस्तदाचष्ट पार्थानां हितकाम्यया}
{पाण्डवा ह्यपि तत्सर्वं प्रत्यजानन्नरिन्दमाः}
{उद्भावनमकुर्वन्तो विदुरस्य मते स्थिताः ॥'}


\chapter{अध्यायः १३८}
\twolineshloka
{`वैशंपायन उवाच}
{}


\twolineshloka
{ततस्ते मन्त्रयामासुर्दुर्योधनपुरोगमाः}
{प्राणवान्विक्रमी चापि शौर्ये च महति स्थितः}


\twolineshloka
{स्पर्धते चापि सततमस्मानेव वृकोदरः}
{तं तु सुप्तं पुरोद्याने जले शूले क्षिपामहे}


\twolineshloka
{ततो जलविहारार्थं कारयामास भारत}
{प्रमाणकोट्यामुद्देशे स्थलं किंचिदुपेत्य ह ॥'}


\twolineshloka
{भक्ष्यं भोज्यं च पेयं च चोष्यं लेह्यमथापि च}
{उपपादितं नरैस्तत्र कुशलैः सूदकर्मणि}


\twolineshloka
{न्यवेदयंस्तत्पुरुषा धार्तराष्ट्राय वै तदा}
{ततो दुर्योधनस्तत्र पाण्डवानाह दुर्मतिः}


\twolineshloka
{गङ्गां चैवानुयास्याम उद्यानवनशोभिताम्}
{सहिता भ्रातरः सर्वे जलक्रीडामवाप्नुमः}


\twolineshloka
{एवमस्त्विति तं चापि प्रत्युवाच युधिष्ठिरः}
{ते रथैर्नगराकारैर्देशजैश्च गजोत्तमैः}


\twolineshloka
{निर्ययुर्नगराच्छूराः कौरवाः पाण्डवैः सह}
{उद्यानवनमासाद्य विसृज्य च महाजनम्}


\twolineshloka
{विशन्ति स्म तदा वीराः सिंहा इव गिरेर्गुहाम्}
{उद्यानमभिपश्यन्तो भ्रातरः सर्व एव ते}


\twolineshloka
{उपस्थानगृहैः शुभ्रैर्वलभीभिश्च शोभितम्}
{गवाक्षकैस्तथा जालैर्यन्त्रैः सांचारिकैरपि}


\twolineshloka
{संमार्जितं सौधकारैश्चित्रकारैश्च चित्रितम्}
{दीर्घिकाभिश्च पूर्णाभिस्तथा पुष्करिणीषु च}


\twolineshloka
{जलं तच्छुशुभे च्छन्नं फुल्लैर्जलरुहैस्तथा}
{उपच्छन्ना वसुमती तथा पुष्पैर्यथर्तुकैः}


\twolineshloka
{तत्रोपविष्टास्ते सर्वे पाण्डवाः कौरवाश्च ह}
{उपच्छन्नान्बहून्कामांस्ते भुञ्जन्ति ततस्ततः}


\twolineshloka
{अथोद्यानवरे तस्मिंस्तथा क्रीडागताश्चते}
{परस्परस्य वक्त्रेषु ददुर्भक्ष्यांस्ततस्ततः}


\twolineshloka
{ततो दुर्योधनः पापस्तद्भक्ष्ये कालकूटकम्}
{विषं प्रक्षेपयामास भीमसेनजिघांसया}


\twolineshloka
{स्वयमुत्थाय चैवाथ हृदयेन क्षुरोपमः}
{स वाचाऽमृतकल्पश्च भ्रातृवच्च सुहृद्यथा}


\twolineshloka
{स्वयं प्रक्षिपते भक्ष्यं बहु भीमस्य पाकृत्}
{प्रभक्षितं च भीमेन तं वै दोषमजानता}


\twolineshloka
{ततो दुर्योधनस्तत्र हृदयेन हसन्निव}
{कृतकृत्यमीवात्मानं मन्यते पुरुषाधमः}


\twolineshloka
{ततस्ते सहिताः सर्वे जलक्रीडामकुर्वत}
{पाण्डवा धार्तराष्ट्राश्च तदा मुदितमानसाः}


\twolineshloka
{विहारावसथेष्वेव वीरा वासमरोचयन्}
{भीमस्तु बलवान्भुक्त्वा व्यायामाभ्यधिकं जले}


\twolineshloka
{वाहयित्वा कुमारांस्ताञ्जलक्रीडागतांस्तदा}
{प्रमाणकोट्यां वासार्थी सुष्वापावाप्य तत्स्थलम्}


\twolineshloka
{शीतं वातं समासाद्य श्रान्तो मदविमोहितः}
{विषेण च परीताङ्गो निश्चेष्टः पाण्डुनन्दनः}


\twolineshloka
{ततो बद्ध्वा लतापाशैर्भीमं दुर्योधनः स्वयम्}
{`शूलान्यप्सु निखायाशु प्रादेशाभ्यन्तराणि च}


\twolineshloka
{लतापाशैर्दृढं बद्धं स्थलाज्जलमपातयत्}
{सशेषत्वान्न संप्राप्तो जले शूलिनि पाण्डवः}


\threelineshloka
{पपात यत्र तत्रास्य शूलं नासीद्यदृच्छया}
{'स निःसंज्ञो जलस्यान्तमवाग्वै पाण्डवोऽविशत्}
{आक्रामन्नागभवने तदा नागकुमारकान्}


\twolineshloka
{ततः समेत्य बहुभिस्तदा नागैर्महाविषैः}
{अदश्यत भृशं भीमो महादंष्ट्रैर्विपोल्बणैः}


\twolineshloka
{ततोऽस्य दश्यमानस्य तद्विषं कालकूटकम्}
{हतं सर्पविषेणैव स्थावरं जङ्गमेन तु}


\twolineshloka
{दंष्ट्राश्च दंष्ट्रिणां तेषां मर्मस्वपि निपातिताः}
{त्वचं नैवास्य बिभिदुः सारत्वात्पृथुवक्षसः}


\twolineshloka
{ततः प्रबुद्धः कौन्तेयः सर्वं संछिद्य बन्धनम्}
{पोथयामास तान्सर्पान्केचिद्भीताः प्रदुद्रुवुः}


\twolineshloka
{हतावशेषा भीमेन सर्वे वासुकिमभ्ययुः}
{ऊचुश्च सर्पराजानं वासुकिं वासवोपमम्}


\twolineshloka
{अयं नरो वै नागरेन्द्र ह्यप्सु बद्ध्वा प्रवेशितः}
{यथा च नो मतिर्व्रीर विषपीतो भविष्यति}


\twolineshloka
{निश्चेष्टोऽस्माननुप्राप्तः स च दष्टोऽन्वबुध्यत}
{ससंज्ञश्चापि संवृत्तश्छित्त्वा बन्धनमाशु नः}


\twolineshloka
{पोथयन्तं महाबाहुं त्वं वै तं ज्ञातुमर्हसि}
{ततो वासुकिरभ्येत्य नागैरनुगतस्तदा}


\twolineshloka
{पश्यति स्म महाबाहुं भीमं भीमपराक्रमम्}
{आर्यकेण च दृष्टः स पृथाया आर्यकेम च}


\twolineshloka
{तदा दौहित्रदौहित्रः परिष्वक्तः सुपीडितम्}
{सुप्रीतश्चाभवत्तस्य वासुकिः स महायशाः}


\twolineshloka
{अब्रवीत्तं च नागेन्द्रः किमस्य क्रियतां प्रियम्}
{धनौघो रत्ननिचयो वसु चास्य प्रदीयताम्}


\twolineshloka
{एवमुक्तस्तदा नागो वासुकिं प्रत्यभाषत}
{यदि नागेन्द्र तुष्टोऽसि किमस्य धनसंचयैः}


\twolineshloka
{रसं पिबेत्कुमारोऽयं त्वयि प्रीते महाबलः}
{बलं नागसहस्रस्य यस्मिन्कुण्डे प्रतिष्ठितम्}


\twolineshloka
{यावत्पिबति बालोऽयं तावदस्मै प्रदीयताम्}
{एवमस्त्विति तं नागं वासुकिः प्रत्यभाषत}


\twolineshloka
{ततो भीमस्तदा नागैः कृतस्वस्त्ययनः शुचिः}
{प्राङ्मुखश्चोपविष्टश्च रसं पिबति पाण्डवः}


\twolineshloka
{एकोच्छ्वासात्ततः कुण्डं पिबति स्म महाबलः}
{एवमष्टौ स कुण्डानि ह्यपिबत्पाण्डुनन्दनः}


\twolineshloka
{ततस्तु शयने दिव्ये नागदत्ते महाभुजः}
{अशेत भीमसेनस्तु यथासुखमरिंदमः}


\chapter{अध्यायः १३९}
\twolineshloka
{वैशंपायन उवाच}
{}


\twolineshloka
{`दुर्योधनस्तु पापात्मा भीममाशीविषहदे}
{प्रक्षिप्य कृतकृत्यं स्वमात्मानं मन्यते तदा}


\twolineshloka
{प्रभातायां रजन्यां च प्रविवेश पुरं ततः}
{ब्रुवाणो भीमसेनस्तु यातो ह्यग्रत एव नः ॥'}


\twolineshloka
{युधिष्ठिरस्तु धर्मात्मा ह्यविदन्पापमात्मनि}
{स्वेनानुमानेन परं साधुं समनुपश्यति}


\twolineshloka
{सोऽभ्युपेत्य तदा पार्थो मातरं भ्रातृवत्सलः}
{अभिवाद्याब्रवीत्कुन्तीमम्ब भीम इहागतः}


\twolineshloka
{क्व गतो भविता मातर्नेह पश्यामि तं शुभे}
{उद्यानानि वनं चैव विचितानि समन्ततः}


\twolineshloka
{तदर्थं न च तं वीरं दृष्टवन्तो वृकोदरम्}
{मन्यमानास्ततः सर्वे यातो नः पूर्वमेव सः}


\twolineshloka
{आगताः स्म महाभागे व्याकुलेनान्तरात्मना}
{इहागम्य क्व नु गतस्त्वया वा प्रेषितः क्व नु}


\twolineshloka
{कथयस्व महाबाहुं भीमसेनं यशस्विनि}
{नहि मे शुध्यते भावस्तं वीरं प्रति शोभने}


\twolineshloka
{यतः प्रसुप्तं मन्येऽहं भीम् नेति हतस्तु सः}
{इत्युक्ता च ततः कुन्ती धर्मराजेन धीमता}


\twolineshloka
{हाहेति कृत्वा संभ्रान्ता प्रत्युवाच युधिष्ठिरम्}
{न पुत्र भीमं पश्यामि न मामभ्येत्यसाविति}


\twolineshloka
{शीघ्रमन्वेषणे यत्नं कुरु तस्यानुजैः सह}
{द्रुतं गत्वा पुरोद्यानं विचिन्वन्तिस्म पाण्डवाः}


\twolineshloka
{भीमभीमेति ते वाचा पाण्डवाः समुदैरयन्}
{विचिन्वन्तोऽथ ते सर्वे न स्म पश्यन्ति भ्रातरम्}


\twolineshloka
{आगताः स्वगृहं भूय इदमूचुः पृथां तदा}
{न दृश्यते महाबाहुरम्ब भीमो वने चितः}


\threelineshloka
{विचितानि च सर्वाणि ह्युद्यानानि नदीस्तथा}
{वैशंपायन उवाच}
{ततो विदुरमानाय्य कुन्ती सा स्वं निवेशनम्}


% Check verse!
उवाच बलवान्क्षत्तर्भीमसेनो न दृश्यते
\twolineshloka
{उद्यानान्निर्गताः सर्वे भ्रातरो भ्रातृभिः सह}
{तत्रैकस्तु महाबाहुर्भीमो नाभ्येति मामिह}


\twolineshloka
{न च प्रीणयते चक्षुः सदा दुर्योधनस्य सः}
{क्रूरोऽसौ दुर्मतिः क्षुद्रो राज्यलुब्धोऽनपत्रपः}


\threelineshloka
{निहन्यादपि तं वीरं जातमन्युः सुयोधनः}
{तेन मे व्याकुलं चित्तं हृदयं दह्यतीव च ॥विदुर उवाच}
{}


\twolineshloka
{मैवं वदस्व कल्याणि शेषसंरक्षणं कुरु}
{प्रत्यादिष्टो हि दुष्टात्मा शेषेऽपि प्रहरेत्तव}


\threelineshloka
{दीर्घायुषस्तव सुता यथोवाच महामुनिः}
{आगमिष्यति ते पुत्रः प्रीतिं चोत्पादयिष्यति ॥वैशंपायन उवाच}
{}


\twolineshloka
{एवमुक्त्वा ययौ विद्वान्विदुरः स्वं निवेशनम्}
{कुन्ती चिन्तापरा भूत्वा सहासीना सुतैर्गृहे}


\twolineshloka
{ततोऽष्टमे तु दिवसे प्रत्यबुध्यत पाण्डवः}
{तस्मिंस्तदा रसे जीर्णे सोऽप्रमेयबलो बली}


\twolineshloka
{तं दृष्ट्वा प्रतिबुध्यन्तं पाण्डवं ते भुजङ्गमाः}
{सान्त्वयामासुरव्यग्रा वचनं चेदमब्रुवन्}


\twolineshloka
{यत्ते पीतो महाबाहो रसोऽयं वीर्यसंभृतः}
{तस्मान्नागायुतबलो रणेऽधृष्यो भविष्यसि}


\twolineshloka
{गच्छाद्य त्वं च स्वगृहं स्नातो दिव्यैरिमैर्जलैः}
{भ्रातरस्तेऽनुतप्यन्ति त्वां विना कुरुपुङ्गव}


\twolineshloka
{ततः स्नातो महाबाहुः शुचिशुक्लाम्बरस्रजः}
{ततो नागस्य भवने कृतकौतुकमङ्गलः}


\twolineshloka
{ओषधीभिर्विषघ्नीभिः सुरभीभिर्विशेषतः}
{भुक्तवान्परमान्नं च नागैर्दत्तं महाबलः}


\twolineshloka
{पूजितो भुजगैर्वीर आशीर्भिश्चाभिनन्दितः}
{दिव्याभरणसंछन्नो नागानामन्त्र्य पाण्डवाः}


\twolineshloka
{उदतिष्ठत्प्रहृष्टात्मा नागलोकादरिन्दमः}
{उत्क्षिप्य च तदा नागैर्जलाज्जलरुहेक्षणः}


\twolineshloka
{तस्मिन्नेव वनोद्देशे स्थापितः कुरुनन्दनः}
{ते चान्तर्दधिरे नागाः पाण्डवस्यैव पश्यतः}


\twolineshloka
{तत उत्थाय कौन्तेयो भीमसेनो महाबलः}
{आजगाम महाबाहुर्मातुरन्तिकमञ्जसा}


\twolineshloka
{ततोऽभिवाद्य जननीं ज्येष्ठं भ्रातरमेव च}
{कनीयसः समाघ्राय शिरस्स्वरिविमर्दनः}


\twolineshloka
{तैश्चापि संपरिष्वक्तः सह मात्रा नरर्षभैः}
{अन्योन्यगतसौहार्दाद्दिष्ट्या दिष्ट्येति चाब्रुवन्}


\twolineshloka
{ततस्तत्सर्वमाचष्ट दुर्योधनविचेष्टितम्}
{भ्रातॄणां भीमसेनश्च महाबलपराक्रमः}


\twolineshloka
{नागलोके च यद्वृत्तं गुणदोषमशेषतः}
{तच्च सर्वमशेषेण कथयामास पाण्डवः}


\twolineshloka
{ततो युधिष्ठिरो राजा भीममाह वचोऽर्थवत्}
{तूष्णीं भव न ते जल्प्यमिदं कार्यं कथंचन}


\twolineshloka
{इतःप्रभृति कौन्तेयं रक्षतान्योन्यमादृताः}
{एवमुक्त्वा महाबाहुर्धर्मराजो युधिष्ठिरः}


\twolineshloka
{भ्रातृभिः सहितः सर्वैरप्रमत्तोऽभवत्तदा}
{यदा त्ववगतास्ते वै पाण्डवास्तस्य कर्म तत्}


\twolineshloka
{नत्वेव बहुलं चक्रुः प्रयता मन्त्ररक्षणे}
{धर्मात्मा विदुरस्तेषां प्रददौ मतिमान्मतिम्}


\twolineshloka
{दुर्योधनोऽपि तं दृष्ट्वा पाण्डवं पुनरागतम्}
{निश्वसंश्चिन्तयंश्चैवमहन्यहनि तप्यते}


\twolineshloka
{कुमारान्क्रीडमानांस्तान्दृष्ट्वा राजातिदुर्मदान्}
{गुरुं शिक्षार्थमन्विष्य गौतमं तान्न्यवेदयत्}


\threelineshloka
{शरस्तम्बे समुद्भूतं वेदशास्त्रार्थपारगम्}
{`राज्ञा निवेदितास्तस्मै ते च सर्वे च निष्ठिताः}
{'अधिजग्मुश्च कुरवो धनुर्वेदं कृपात्तु ते}


\chapter{अध्यायः १४०}
\twolineshloka
{जनमेजय उवाच}
{}


\threelineshloka
{कृपस्यापि मम ब्रह्मन्संभवं वक्तुमर्हसि}
{शरस्तम्बात्कथं जज्ञे कथं वाऽस्त्राण्यवाप्तवान् ॥वैशंपायन उवाच}
{}


\twolineshloka
{महर्षेर्गौतमस्यासीच्छरद्वान्नाम गौतमः}
{पुत्रः किल महाराज जातः सहशरैर्विभो}


\twolineshloka
{न तस्य वेदाध्ययने तथा बुद्धिरजायत}
{यथास्य बुद्धिरभवद्धनुर्वेदे परन्तप}


\twolineshloka
{अधिजग्मुर्यथा वेदास्तपसा ब्रह्मचारिणः}
{तथा स तपसोपेतः सर्वाण्यस्त्राण्यवाप ह}


\twolineshloka
{धनुर्वेदपरत्वाच्च तपसा विपुलेन च}
{भृशं सन्तापयामास देवराजं स गौतमः}


\twolineshloka
{ततो जालवतीं नाम देवकन्यां सुरेश्वरः}
{प्राहिणोत्तपसो विघ्नं कुरु तस्येति कौरव}


\twolineshloka
{सा हि गत्वाऽऽश्रमं तस्य रमणीयं शरद्वतः}
{धनुर्बाणधरं बाला लोभयामास गौतमम्}


\twolineshloka
{तामेकवसनां दृष्ट्वा गौतमोऽप्सरसं वने}
{लोकेऽप्रतिमसंस्थानां प्रोत्फुल्लनयनोऽभवत्}


\twolineshloka
{धनुश्च हि शरास्तस्य कराभ्यामपतन्भुवि}
{वेपथुश्चास्य सहसा शरीरे समजायत}


\twolineshloka
{स तु ज्ञानगरीयस्त्वात्तपसश्च समर्थनात्}
{अवतस्थे महाप्राज्ञो धैर्येण परमेण ह}


\twolineshloka
{यस्तस्य सहसा राजन्विकारः समदृश्यत}
{तेन सुस्राव रेतोऽस्य स च तन्नान्वबुध्यत}


\twolineshloka
{धनुश्च सशरं त्यक्त्वा तथा कृष्णाजिनानि च}
{स विहायाश्रमं तं च तां चैवाप्सरसं मुनिः}


\twolineshloka
{जगाम रेतस्तत्तस्य शरस्तम्बे पपात च}
{शरस्तम्बे च पतितं द्विधा तदभवन्नृप}


\twolineshloka
{तस्याथ मिथुं जज्ञे गौतमस्य शरद्वतः}
{`महर्षेर्गौतमस्यास्य ह्याश्रमस्य समीपतः ॥'मृगयां चरतो राज्ञः शान्तनोस्तु यदृच्छया}


\twolineshloka
{कश्चित्सेनाचरोऽरण्ये मिथुनं तदपश्यत}
{धनुश्च सशरं दृष्ट्वा तथा कृष्णाजिनानि च}


\twolineshloka
{ज्ञात्वा द्विजस्य चापत्ये धनुर्वेदान्तगस्य ह}
{स राज्ञे दर्शयामास मिथुनं सशरं धनुः}


\twolineshloka
{स तदादाय मिथुनं राजा च कृपयान्वितः}
{आजगाम गृहानेव मम पुत्राविति ब्रुवन्}


\twolineshloka
{ततः संवर्धयामास संस्कारैश्चाप्ययोजयत्}
{प्रातिपेयो नरश्रेष्ठो मिथुनं गौतमस्य तत्}


\threelineshloka
{कृपया यन्मया बालाविमौ संवर्धिताविति}
{तस्मात्तयोर्नाम चक्रे तदेव स महीपतिः}
{`तस्मात्कृप इति ख्यातः कृपी कन्या च साऽभवत् ॥'}


\twolineshloka
{पितापि गौतमस्तत्र तपसा ताववन्दित}
{आगत्य तस्मै गोत्रादि सर्वमाख्यातवांस्तदा}


\twolineshloka
{`कृपोऽपि च तदा राजन्धनुर्वेदपरोऽभवत्}
{'चतुर्विधं धनुर्वेदं शास्त्राणि विविधानि च}


\twolineshloka
{निशिलेनास्य तत्सर्वं गुह्यमाख्यातवान्पिता}
{सोऽचिरेणैव कालेन परमाचार्यतां गतः}


\twolineshloka
{कृपमाहूय गाङ्गेयस्तव शिष्या इति ब्रुवन्}
{पौत्रान्परिसमाधाय कृपमाराधयत्तदा}


\twolineshloka
{ततोऽधिजग्मुः सर्वे ते धनुर्वेदं महारथाः}
{धृतराष्ट्रात्मजाश्चैव पाण्डवाः सह यादवैः}


\fourlineindentedshloka
{वृष्णयश्च नृपाश्चान्ये नानादेशसमागताः}
{`कृपमाचार्यमासाद्य परमास्त्रज्ञतां गतः}
{'वैशंपायन उवाच}
{विशेषार्थी ततो भीष्मः पौत्राणां विनयेप्सया}


\twolineshloka
{इष्वस्त्रज्ञान्पर्यपृच्छदाचार्यान्वीर्यसंमतान्}
{नाल्पधीर्नामहाभागस्तथा नानस्त्रकोविदः}


\twolineshloka
{नादेवसत्वो विनयेत्कुरूनस्त्रे महावलान्}
{इति संचिन्त्य गाङ्गेयस्तदा भरतसत्तमः}


\twolineshloka
{द्रोणाय वेदविदुषे भारद्वाजाय धमते}
{पाण्डवान्कौरवांश्चैव ददौ शिष्यान्नरर्षभ}


\twolineshloka
{शास्त्रतः पूजितश्चैव सम्यक्तेन महात्मना}
{स भीष्मेण महाभागस्तुष्टोऽस्त्रविदुषां वरः}


\twolineshloka
{प्रतिजग्राह तान्सर्वाञ्शिष्यत्वेन महायशाः}
{शिक्षयामास च द्रोणो धनुर्वेदमशेषतः}


\threelineshloka
{तेऽचिरेणैव कालेन सर्वशस्त्रविशारदाः}
{बभूवुः कौरवा राजन्पाण्डवाश्चामितौजसः ॥जनमेजय उवाच}
{}


\twolineshloka
{कथं समभवद्द्रोणः कथं चास्त्राण्यवाप्तवान्}
{कथं चागात्कुरून्ब्रह्मन्कस्य पुत्रः स वीर्यवान्}


\threelineshloka
{कथं चास्य सुतो जातः सोश्वत्थामाऽस्त्रवित्तमः}
{एतदिच्छाम्यहं श्रोतुं विस्तरेण प्रकीर्तय ॥वैशंपायन उवाच}
{}


\twolineshloka
{गङ्गाद्वारं प्रति महान्बभूव भगवानृषिः}
{भरद्वाज इति ख्यातः सततं संशितव्रतः}


\twolineshloka
{सोऽभिषेक्तुं गतो गङ्गां पूर्वमेवागतां सतीम्}
{महर्षिभिर्भरद्वाजो हविर्धाने चरन्पुरा}


\twolineshloka
{ददर्शाप्सरसं साक्षाद्धृताचीमाप्लुतामृषिः}
{रूपयौवनसंपन्नां मददृप्तां मदालसाम्}


\twolineshloka
{तस्या वायुर्नदीतीर वसनं पर्यवर्तत}
{व्यपकृष्टाम्बरां दृष्ट्वा तामृषिश्चकमे ततः}


\twolineshloka
{तत्र संसक्तमनसो भरद्वाजस्य धीमतः}
{हृष्टस्य रेतश्चस्कन्द तदृषिर्द्रोण आदधे}


\twolineshloka
{ततः समभवद्द्रोणः कलशे तस्य धीमतः}
{अध्यगीष्ट स वेदांश्च वेदाङ्गानि च सर्वशः}


\twolineshloka
{अग्नेरस्त्रमुपादाय यदृषिर्वेद काश्यपः}
{अध्यगच्छद्भरद्वाजस्तदस्त्रं देवकार्यतः}


\twolineshloka
{अग्निवेश्यं महाभागं भरद्वाजः प्रतापवान्}
{प्रत्यपादयदाग्नेयमस्त्रमस्त्रविदां वरः}


\threelineshloka
{`कनिष्ठजातं स मुनिर्भ्राता भ्रातरमन्तिके}
{अग्निवेश्यस्तथा द्रोणं तदा भरतसत्तम}
{'भारद्वाजं तदाग्नेयं महास्त्रं प्रत्यपादयत्}


\twolineshloka
{भरद्वाजसखा चासीत्पृषतो नाम पार्थिवः}
{तस्यापि द्रुपदो नाम तथा समभवत्सुतः}


\twolineshloka
{स नित्यमाश्रमं गत्वा द्रोणेन सह पार्थिवः}
{चिक्रीडाध्ययनं चैव चकार क्षत्रियर्षभः}


\twolineshloka
{ततो व्यतीते पृषते स राजा द्रुपदोऽभवत्}
{पञ्चालेषु महाबाहुरुत्तरेषु नरेश्वरः}


\twolineshloka
{भरद्वाजोऽपि भगवानारुरोह दिवं तदा}
{तत्रैव च वसन्द्रोणस्तपस्तेपे महातपाः}


\twolineshloka
{वेदवेदाङ्गविद्वान्स तपसा दग्धकिल्बिषः}
{ततः पितृनियुक्तात्मा पुत्रलोभान्महायशाः}


\twolineshloka
{शारद्वतीं ततो भार्यां कृपीं द्रोणोऽन्वविन्दत}
{अग्निहोत्रे च धर्मे च दमे च सतत रताम्}


\twolineshloka
{अलभद्गौतमी पुत्रमश्वत्थामानमेव च}
{स जातमात्रो व्यनदद्यथैवोच्चैःश्रवा हयः}


\twolineshloka
{तच्छ्रुत्वान्तर्हितं भूतमन्तरिक्षस्थमब्रवीत्}
{अश्वस्येवास्य यत्स्थाम नदतः प्रदिशो गतम्}


\twolineshloka
{अश्वत्थामैव बालोऽयं तस्मान्नाम्ना भविष्यति}
{सुतेन तेन सुप्रीतो भारद्वाजस्ततोऽभवत्}


\twolineshloka
{तत्रैव च वसन्धीमान्धनुर्वेदपरोऽभवत्}
{स शुश्राव महात्मानं जामदग्न्यं परंतपम्}


\twolineshloka
{सर्वज्ञानविदं विप्रं सर्वशस्त्रभृतां वरम्}
{ब्राह्मणेभ्यस्तदा राजन्दित्सन्तं वसु सर्वशः}


\twolineshloka
{स रामस्य धनुर्वेदं दिव्यान्यस्त्राणि चैव ह}
{श्रउत्वा तेषु मनश्चक्रे नीतिशास्त्रे तथैव च}


\twolineshloka
{ततः स व्रतिभिः शिष्यैस्तपोयुक्तैर्महातपाः}
{वृतः प्रायान्महावाहुर्महेन्द्रं पर्वतोत्तमम्}


\twolineshloka
{ततो महेन्द्रमासाद्य भारद्वाजो महातपाः}
{क्षत्रघ्नं तममित्रघ्नमपश्यद्भृगुनन्दनम्}


\twolineshloka
{ततो द्रोणो वृतः शिष्यैरुपगम्य भृगूद्वहम्}
{आचख्यावात्मनो नाम जन्म चाङ्गिरसः कुले}


\twolineshloka
{निवेद्य शिरसा भूमौ पादौ चैवाभ्यवादयत्}
{ततस्तं सर्वमुत्सृज्य वनं जिगमिषुं तदा}


\twolineshloka
{जामदग्न्यं महात्मानं भारद्वाजोऽब्रवीदिदम्}
{भरद्वाजात्समुत्पन्नं तथा त्वं मामयोनिजम्}


\twolineshloka
{आगतं वित्तकामं मां विद्धि द्रोणं द्विजोत्तम}
{तमब्रवीन्महात्मा स सर्वक्षत्रियमर्दनः}


\twolineshloka
{स्वागतं ते द्विजश्रेष्ठ यदिच्छसि वदस्व मे}
{एवमुक्तस्तु रामेण भारद्वाजोऽब्रवीद्वचः}


\threelineshloka
{रामं प्रहरतां श्रेष्ठं दित्सन्तं विविधं वसु}
{अहं धनमनन्तं हि प्रार्थये विपुलव्रत ॥राम उवाच}
{}


\twolineshloka
{हिरण्यं मम यच्चान्यद्वसु किंचिदिह स्थितम्}
{ब्राह्मणेभ्यो मया दत्तं सर्वमेतत्तपोधन}


\twolineshloka
{तथैवेयं धरा देवी सागरान्ता सपत्तना}
{कश्यपाय मया दत्ता कृत्स्ना नगरमालिनी}


\twolineshloka
{शरीरमात्रमेवाद्य ममेदमवशेषितम्}
{अस्त्राणि च महार्हाणि शस्त्राणि विविधानि च}


\threelineshloka
{अस्त्राणि वा शरीरं वा ब्रह्मञ्शस्त्राणि वा पुनः}
{वृणीष्व किं प्रयच्छामि तुभ्यं द्रोण वदाशु तत् ॥द्रोण उवाच}
{}


\twolineshloka
{अस्त्राणि मे समग्राणि ससंहाराणि भार्गव}
{स प्रयोगरहस्यानि दातुमर्हस्यशेषतः}


\threelineshloka
{`एतद्वसु वसूनां हि सर्वेषां विप्रसत्तम}
{'तथेत्युक्त्वा ततस्तस्मै प्रादादस्त्राणि भार्गवः}
{सरहस्यव्रतं चैव धनुर्वेदमशेषतः}


\twolineshloka
{प्रतिगृह्य तु तत्सर्वं कृतास्त्रे द्विजसत्तमः}
{प्रियं सखायं सुप्रीतो जगाम द्रुपदं प्रति}


\chapter{अध्यायः १४१}
\twolineshloka
{वैशंपायन उवाच}
{}


\twolineshloka
{ततो द्रुपदमासाद्य भारद्वाजः प्रतापवान्}
{अब्रवीत्पार्थिवं राजन्सखायं विद्धि मामिह}


\twolineshloka
{इत्येवमुक्तः सख्या स प्रीतिर्पूर्वं जनेश्वरः}
{भारद्वाजेन पाञ्चाल्यो नामृष्यत वचोऽस्य तत्}


\threelineshloka
{स क्रोधामर्षजिह्मभ्रूः कषायीकृतलोचनः}
{ऐश्वर्यमदसंपन्नो द्रोणं राजाऽब्रवीदिदम् ॥द्रुपद उवाच}
{}


\twolineshloka
{अकृतेयं तव प्रज्ञा ब्रह्मन्नातिसमञ्जसा}
{यन्मां व्रवीषि प्रसभं सखा तेऽहमिति द्विज}


\twolineshloka
{न हि राज्ञामुदीर्णानामेवंभूतैर्नरैः क्वचित्}
{सख्यं भवति मन्दात्मञ्श्रिया हीनैर्धनच्युतैः}


\twolineshloka
{सौहृदान्यपि जीर्यन्ते कालेन परिजीर्यतः}
{सौहृदं मे त्वया ह्यासीत्पूर्वं सामर्थ्यबन्धनम्}


\twolineshloka
{न सख्यमजरं लोके हृदि तिष्ठति कस्यचित्}
{कामश्चैतन्नाशयति क्रोधो वैनं रहत्युत}


\twolineshloka
{मैवं जीर्णमुपास्स्व त्वं सख्यं भवदुपाधिकृत्}
{आसीत्सख्यं द्विजश्रेष्ठ त्वया मेऽर्थनिबन्धनम्}


\twolineshloka
{न दरिद्रो वसुमतो नाविद्वान्विदुषः सखा}
{न शूरस्य सखा क्लीबः सखिपूर्वं किमिष्यते}


\twolineshloka
{ययोरेव समं वित्तं ययोरेव समं श्रुतम्}
{तयोर्विवाहः सख्यं च न तु पुष्टविपुष्टयोः}


\threelineshloka
{नाश्रोत्रियः श्रोत्रियस्य नारथी रथिनः सखा}
{नाराजा पार्थिवस्यापि सखिपूर्वं किमिष्यते ॥वैशंपायन उवाच}
{}


\twolineshloka
{द्रुपदेनैवमुक्तस्तु भारद्वाजः प्रतापवान्}
{मुहूर्तं चिन्तयित्वा तु मन्युनाऽभिपरिप्लुतः}


\twolineshloka
{स विनिश्चित्य मनसा पाञ्चाल्यं प्रतिबुद्धिमान्}
{`शिष्यैः परिवृतः श्रीमान्पुत्रेण सहितस्तथा ॥'}


\twolineshloka
{जगाम कुरुमुख्यानां नागरं नागसाह्वयम्}
{तां प्रतिज्ञां प्रतिज्ञाय यां कर्ता नचिरादिव}


\twolineshloka
{स नागपुरमागम्य गौतमस्य निवेशने}
{भारद्वाजोऽवसत्तत्र प्रच्छन्नं द्विजसत्तमः}


\twolineshloka
{ततोऽस्य तनुजः पार्थान्कृपस्यानन्तरं प्रभुः}
{अस्त्राणि शिक्षयामास नाबुध्यन्त च तं जनाः}


\twolineshloka
{एवं स तत्र गूढात्मा कंचित्कालमुवास ह}
{कुमारास्त्वथ निष्क्रम्य समेता गजसाह्वयात्}


\twolineshloka
{क्रीडन्तो वीटया तत्र वीराः पर्यचरन्मुदा}
{`तेषां संक्रीडमानानामुदपानेऽङ्गुलीयकम्}


\twolineshloka
{पपात धर्मपुत्रस्य वीटा तत्रैव चापतत्}
{गर्तान्बुना प्रतिच्छन्नं तारारूपमिवाम्बरे}


\threelineshloka
{दृष्ट्वा चैते कुमाराश्च तं यत्नात्पर्यवारयन्}
{'ततस्ते यत्नमातिष्ठन्वीटामुद्धर्तुमादृताः}
{न च ते प्रत्ययद्यन्त कर्म वीटोपलब्धये}


\twolineshloka
{ततोऽन्योन्यमवैक्षन्त व्रीडयावनताननाः}
{तस्या योगमविदन्तो भृशं चोत्कण्ठिताभवन्}


\twolineshloka
{तेऽपश्यन्ब्राह्मणं श्याममापन्नं पलितं कृशम्}
{कृत्यवन्तमदूरस्थमग्निहोत्रपुरस्कृतम्}


\twolineshloka
{ते तं दृष्ट्वा महातमानमुपगम्य कुमारकाः}
{भग्नोत्साहक्रियात्मानो ब्राह्मणं पर्यवारयन्}


\twolineshloka
{अथ द्रोणः कुमारांस्तान्दृष्ट्वा कत्यवशस्तदा}
{प्रहस्य मन्दं पैशल्यादभ्यभाषत वीर्यवान्}


\twolineshloka
{अहो वो धिग्बलं क्षात्रं धिगेतां वः कृतास्त्रताम्}
{भरतस्यान्वये जाता ये वीटां नाधिगच्छत}


\twolineshloka
{वीटां च मुद्रिकां चैव ह्यहमेतदपि द्वयम्}
{उद्धरेयमिषीकाभिर्भोजनं मे प्रदीयताम्}


\twolineshloka
{ततोऽब्रवीत्तदा द्रोणं कुन्तीपुत्रो युधिष्ठिरः}
{कृपस्यानुमते ब्रह्मन्भिक्षामाप्नुहि शाश्वतीम्}


\threelineshloka
{एवमुक्तः प्रत्युवाच प्रहस्य भरतानिदम्}
{द्रोण उवाच}
{एषा मुष्टिरिषीकाणां मयाऽस्त्रेणाभिमन्त्रिता}


\twolineshloka
{अस्या वीर्यं निरीक्षध्वं यदन्येषु न विद्यते}
{वेत्स्यामीषिकया वीटां तामिषीकां तथाऽन्यया}


\threelineshloka
{तामन्यया समायोगे वीटाया ग्रहणं मम}
{वैशंपायन उवाच}
{ततो यथोक्तं द्रोणेन तत्सर्वं कृतमञ्जसा}


\twolineshloka
{तदवेक्ष्य कुमारास्ते विस्मयोत्फुल्ललोचनाः}
{आश्चर्यमिदमत्यन्तमिति मत्वा वचोऽब्रुवन्}


\twolineshloka
{मुद्रिकामणि विप्रर्षे शीध्रमेतां समुद्धरथ वैशंपायन उवाच}
{ततः शरं समादाय धनुश्चापि महायशाः}


\twolineshloka
{शरेण विद्ध्वा मुद्रां तामूर्ध्वमावाहयत्प्रभुः}
{स शरं समुपादाय कूपादङ्गुलिवेष्टनम्}


\twolineshloka
{ददौ ततः कुमाराणां विस्मितानामविस्मितः}
{मुद्रिकामुद्धृतां दृष्ट्वा तमाहुस्ते कुमारकाः}


\threelineshloka
{अभिवन्दामहे ब्रह्मन्नैतदन्येषु विद्यते}
{कोऽसि कस्यासि जानीमो वयं किं करवामहे ॥वैशंपायन उवाच}
{}


\twolineshloka
{एवमुक्तस्ततो द्रोणः प्रत्युवाच कुमारकान्}
{आचक्षध्वं च भीष्माय रूपेण च गुणैश्च माम्}


\threelineshloka
{स एव सुमहातेजाः सांप्रतं प्रतिपत्स्यते}
{वैशंपायन उवाच}
{तथेत्युक्त्वा च गत्वा च भीष्ममूचुः कुमारकाः}


\twolineshloka
{ब्राह्मणस्य वचः कृत्स्नं तच्च कर्म तथाविधम्}
{भीष्मः श्रुत्वा कुमाराणां द्रोणं तं प्रत्यजानत}


\twolineshloka
{युक्तरूपः स हि गुरुरित्येवमनुचिन्त्य च}
{अथैनमानीय तदा स्वयमेव सुसत्कृतम्}


\threelineshloka
{परिपप्रच्छ निपुणं भीष्मः शस्त्रभृतां वरः}
{हेतुमागमने तच्च द्रोणः सर्वं न्यवेदयत् ॥द्रोण उवाच}
{}


\twolineshloka
{महर्षेरग्निवेश्यस्य सकाशमहमच्युत}
{अस्त्रार्थमगमं पूर्वं धनुर्वेदजिघृक्षया}


\twolineshloka
{ब्रह्मचारी विनीतात्मा जटिलो बहुलाः समाः}
{अवसं सुचिरं तत्र गुरुशुश्रूषणे रतः}


\twolineshloka
{पाञ्चाल्यो राजपुत्रश्च यज्ञसेनो महाबलः}
{इष्वस्त्रहेतोर्न्यवसत्तस्मिन्नेव गुरौ प्रभुः}


\twolineshloka
{स मे तत्र सखा चासीदुपकारी प्रियश्च मे}
{तेनाहं सह संगम्य वर्तयन्सुचिरं प्रभो}


\twolineshloka
{बाल्यात्प्रभृति कौरव्य सहाध्ययनमेव च}
{स मे सखा सदा तत्र प्रियवादी प्रियंकरः}


\twolineshloka
{अब्रवीदिति मां भीष्म वचनं प्रीतिवर्धनम्}
{अहं प्रियतमः पुत्रः पितुर्द्रोण महात्मनः}


\twolineshloka
{अभिषेक्ष्यति मां राज्ये स पाञ्चाल्यो यदा तदा}
{तद्भोज्यं भवता राज्यमर्धं सत्येन ते शपे}


\twolineshloka
{मम भोगाश्च वित्तं च त्वदधीनं सुखानि च}
{एवमुक्त्वाऽथ वव्राज कृतास्त्रः पूजितो मया}


\twolineshloka
{तच्च वाक्यमहं नित्यं मनसा धारयंस्तदा}
{सोऽहं पितृनियोगेन पुत्रलोभाद्यशस्विनीम्}


\twolineshloka
{शारद्वतीं महाप्रज्ञामुपयेमे महाव्रताम्}
{अग्निहोत्रे च सत्रे च दमे च सततं रताम्}


\twolineshloka
{लेभे च गौतमी पुत्रमश्वत्थामानमौरसम्}
{भीमविक्रमकर्माणमादित्यसमतेजसम्}


\threelineshloka
{पुत्रेण तेन प्रीतोऽहं भरद्वाजो मया यथा}
{गोक्षीरं पिबतो दृष्ट्वा धनिनस्तत्र पुत्रकान्}
{अश्वत्थामारुदद्बालस्तन्मे संदेहयद्दिशः}


\twolineshloka
{न स्नातकोऽवसीदेत वर्तमानः स्वकर्मसु}
{इति संचिन्त्य मनसा तं देशं बहुशो भ्रमन्}


\twolineshloka
{विशुद्धणिच्छन्गाङ्गेय धर्मोपेतं प्रतिग्रहम्}
{अन्तादन्तं परिक्रम्य नाध्यगच्छं पयस्विनीम्}


\twolineshloka
{यवपिष्टोदकेनैनं लोभयेयं कुमारकम्}
{पीत्वा पिष्टरसं बालः क्षीरं पीतं मयाऽपि च}


\twolineshloka
{ननर्तोत्थाय कौरव्य हृष्टो बाल्याद्विमोहितः}
{तं दृष्ट्वा नृत्यमानं तु बालैः परिवृतं सुतम्}


\twolineshloka
{हास्यतामुपसंप्राप्तं कश्मलं तत्र मेऽभवत्}
{द्रोणं धिगस्त्वधनिनं यो धनं नाधिगच्छति}


\twolineshloka
{पिष्टोदकं सुतो यस्य पीत्वा क्षीरस्य तृष्णया}
{नृत्यतिस्म मुदाविष्टः क्षीरं पीतं मयाऽप्युत}


\twolineshloka
{इति संभाषतां वाचं श्रुत्वा मे बुद्धिरच्यवत्}
{आत्मानं चात्मना गर्हन्मनसेदं व्यचिन्तयम्}


\twolineshloka
{अपि चाहं पुरा विप्रैर्वर्जितो गर्हितो वसे}
{परोपसेवां पापिष्ठां न च कुर्यां धनेप्सया}


\twolineshloka
{इति मत्वा प्रियं पुत्रं भीष्मादाय ततो ह्यहम्}
{पूर्वस्नेहानुरागित्वात्सदारः सौमकिं गतः}


\twolineshloka
{अभिषिक्तं तु श्रुत्वैव कृतार्थोऽस्मीति चिन्तयन्}
{प्रियं सखायं सुप्रीतो राज्यस्थं समुपागमम्}


\twolineshloka
{संस्मरन्सङ्गमं चैव वचनं चैव तस्य तत्}
{ततो द्रुपदमागम्य सखइपूर्वमहं प्रभो}


\twolineshloka
{अब्रुवं पुरुषव्याघ्र सखायं विद्धि मामिति}
{उपस्थितस्तु द्रुपदं सखिवच्चास्मि सङ्गतः}


\twolineshloka
{स मां निराकारमिव प्रहसन्निदमब्रवीत्}
{अकृतेयं तव प्रज्ञा ब्रह्मन्नातिसमञ्जसा}


\twolineshloka
{यदात्थ मां त्वं प्रसभं सखा तेऽहमिति द्विज}
{संगतनीह जीर्यन्ति कालेन परिजीर्यतः}


\twolineshloka
{सौहृदं मे त्वया ह्यासीत्पूर्वं सामर्थ्यबन्धनम्}
{नाश्रोत्रियः श्रोत्रियस्य नारथी रथिनः सखा}


\twolineshloka
{साम्याद्धि सख्यं भवति वैषम्यान्नोपपद्यते}
{न सख्यमजरं लोके विद्यते जातु कस्यचित्}


\twolineshloka
{कामो वैनं विहरति क्रोधो वैनं रहत्युत}
{मैवं जीर्णमुपास्स्व त्वं सख्यं भवदुपाधिकृत्}


\twolineshloka
{आसीत्सख्यं द्विजश्रेष्ठ त्वया मेऽर्थनिबन्धनम्}
{नह्यनाढ्यः सखाऽऽढ्यस्य नाविद्वान्विदुषः सखा}


\twolineshloka
{न शूरस्य सखा क्लीबः सखिपूर्वं किमिष्यते}
{न हि राज्ञामुदीर्णानामेवं भूतैर्नरैः क्वचित्}


\twolineshloka
{सख्यं भवति मन्दात्मञ्छ्रिया हीनैर्धनच्युतैः}
{नाश्रोत्रियः श्रोत्रियस्य नारथी रथिनः सखा}


\twolineshloka
{नाराजा पार्थिवस्यापि सखिपूर्वं किमिष्यते}
{अहं त्वया न जानामि राज्यार्थे संविदं कृताम्}


\twolineshloka
{एकरात्रं तु ते ब्रह्मन्कामं दास्यामि भोजनम्}
{एवमुक्तस्त्वहं तेन सदारः प्रस्थितस्तदा}


\twolineshloka
{तां प्रतिज्ञां प्रतिज्ञाय यां कर्ताऽस्म्यचिरादिव}
{द्रुपदेनैवमुक्तोऽहं मन्युनाऽभिपरिप्लुतः}


\twolineshloka
{अभ्यागच्छं कुरून्भीष्म शिष्यैरर्थी गुणान्वितैः}
{ततोऽहं भवतः कामं संवर्धयितुमागतः}


\fourlineindentedshloka
{इदं नागपुरं रम्यं ब्रूहि किं करवाणि ते}
{वैशंपायन उवाच}
{एवमुक्तस्तदा भीष्मो भारद्वाजमभाषत ॥भीष्म उवाच}
{}


\twolineshloka
{अपज्यं क्रियतां चापं साध्वस्त्रं प्रतिपादय}
{भुङ्क्ष भोगान्भृशं प्रीतः पूज्यमानः कुरुक्षये}


\twolineshloka
{कुरूणामस्ति यद्वित्तं राज्यं चेदं सराष्ट्रकम्}
{त्वमेव परमो राजा सर्वे वाक्यकरास्तव}


\twolineshloka
{यच्च ते प्रार्थितं ब्रह्मन्कृतं तदिति चिन्त्यताम्}
{दिष्ट्या प्राप्तोऽसि विप्रर्षे महान्मेऽनुग्रहः कृतः}


\chapter{अध्यायः १४२}
\twolineshloka
{`वैशंपायन उवाच}
{}


\twolineshloka
{प्रतिजग्राह तं भीष्मो गुरुं पाण्डुसुतैः सह}
{पौत्रानादाय तान्सर्वान्वसूनि विविधानि च}


\twolineshloka
{शिष्य इति ददौ राजन्द्रोणाय विधिपूर्वकम्}
{तदा द्रोणोऽब्रवीद्वाक्यं भीष्मं बुद्धिमतां वरम्}


\twolineshloka
{कृपस्तिष्ठति चाचार्यः शस्त्रज्ञः प्राज्ञसंमतः}
{मयि तिष्ठति चेद्विप्रो वैमनस्यं गमिष्यति}


\twolineshloka
{युष्मान्किंचिच्च याचित्वा धनं संगृह्य हर्षितः}
{स्वमाश्रमपदं राजन्गमिष्यामि यथागतम्}


\twolineshloka
{एवमुक्ते तु विप्रेन्द्रं भीष्मः प्रहरतां वरः}
{अब्रवीद्द्रोणमाचार्यमुख्यं शस्त्रविदां वरम्}


\fourlineindentedshloka
{कृपस्तिष्ठतु पूज्यश्च भर्तव्यश्च मया सदा}
{त्वं गुरुर्भव पौत्राणामाचार्यस्त्वं मतो मम}
{प्रतिगृह्णीष्व पुत्रांस्त्वमस्त्रज्ञान्कुरु वै सदा ॥'वैशंपायन उवाच}
{}


\twolineshloka
{ततः संपूजितो द्रोणो भीष्मेण द्विपदां वरः}
{विशश्राम महातेजाः पूजितः कुरुवेश्मनि}


\twolineshloka
{विश्रान्तेऽथ गुरौ तस्मिन्पौत्रानादाय कौरवान्}
{शिष्यत्वेन ददौ भीष्मो वसूनि विविधानि च}


\twolineshloka
{गृहं च सुपरिच्छन्नं धनधान्यसमाकुलम्}
{भारद्वाजाय सुप्रीतः प्रत्यपादयत प्रभुः}


\twolineshloka
{स ताञ्शिष्यान्महेष्वासः प्रतिजग्राह कौरवान्}
{पाण्डवान्धार्तराष्ट्रांश्च द्रोणो मुदितमानसः}


\threelineshloka
{प्रतिगृह्य च तान्सर्वान्द्रोणो वचनमब्रवीत्}
{रहस्येकः प्रतीतात्मा कृतोपसदनांस्तथा ॥द्रोण उवाच}
{}


\threelineshloka
{कार्यं मे काङ्क्षितं किंचिद्धृदि संपरिवर्तते}
{कृतास्त्रैस्तत्प्रदेयं मे तदेतद्वदतानघाः ॥वैशंपायन उवाच}
{}


\twolineshloka
{तच्छ्रुत्वा कौरवेयास्ते तूष्णीमासन्विशांपते}
{अर्जुनस्तु ततः सर्वं प्रतिजज्ञे परन्तप}


\twolineshloka
{ततोऽर्जुनं तदा मूर्ध्नि समाघ्राय पुनः पुनः}
{प्रीतिपूर्वं परिष्वज्य प्ररुरोद मुदा तदा}


\twolineshloka
{`अश्वत्थामानमाहूय द्रोणो वचनमब्रवीत्}
{सखायं विद्धि ते पार्थं मया दत्तः प्रगृह्यताम्}


\twolineshloka
{साधुसाध्विति तं पार्थः परिष्वज्येदमब्रवीत्}
{अद्यप्रभृति विप्रेन्द्र परवानस्मि धर्मतः}


\twolineshloka
{शिष्योऽहं त्वत्प्रसादेन जीवामि द्विजसत्तम}
{इत्युक्त्वा तु तदा पार्थः पादौ जग्राह पाण्डवः'}


\twolineshloka
{ततो द्रोणः पाण्डुपुत्रानस्त्राणि विविधानि च}
{ग्राहयामास दिव्यानि मानुषाणि च वीर्यवान्}


\twolineshloka
{राजपुत्रास्तथा चान्ये समेत्य भरतर्षभ}
{अभिजग्मुस्ततो द्रोणमस्त्रार्थे द्विजसत्तमम्}


\twolineshloka
{वृष्णयश्चान्धकाश्चैव नानादेश्याश्च पार्थिवाः}
{सूतपुत्रश्च राधेयो गुरुं द्रोणमियात्तदा}


\twolineshloka
{स्पर्धमानस्तु पार्थेन सूतपुत्रोऽत्यमर्षणः}
{दुर्योधनं समाश्रित्य सोऽवमन्यत पाण्डवान्}


\twolineshloka
{अभ्ययात्स ततो द्रोणं धनुर्वेदचिकीर्षया}
{शिक्षाभुजवलोद्योगैस्तेषु सर्वेषु पाण्डवः}


\twolineshloka
{अस्त्रविद्यानुरागाच्च विशिष्टोऽभवदर्जुनः}
{तुल्येष्वस्त्रप्रयोगेषु लाघवे सौष्टवेषु च}


\twolineshloka
{सर्वेषामेव शिष्याणां बभूवाभ्यधिकोऽर्जुनः}
{ऐन्द्रिमप्रतिमं द्रोण उपदेशेष्वमन्यत}


\twolineshloka
{एवं सर्वकुमाराणामिष्वस्त्रं प्रत्यपादयत्}
{कमण्डलुं च सर्वेषां प्रायच्छच्चिरकारणात्}


\twolineshloka
{पुत्राय च ददौ कुम्भमविलम्बनकारणात्}
{यावत्ते नोपगच्छ्ति तावदस्मै परां क्रियाम्}


\twolineshloka
{द्रोण आचष्ट पुत्राय कर्म तज्जिष्णुरौहत}
{ततः स वारुणास्त्रेण पूरयित्वा कमण्डलुम्}


\twolineshloka
{सममाचार्यपुत्रेण गुरुमभ्येति फाल्गुनः}
{आचार्यपुत्रात्तस्मात्तु विशेषोपचयेऽपृथक्}


\twolineshloka
{न व्यहीयत मेधावी पार्थोऽप्यस्त्रविदां वरः}
{अर्जुनः परमं यत्नमातिष्ठद्गुरुपूजने}


\twolineshloka
{अस्त्रे च परमं योगं प्रियो द्रोणस्य चाभवत्}
{तं दृष्ट्वा नित्यमुद्युक्तमिष्वस्त्रं प्रति फाल्गुनम्}


\threelineshloka
{आहूय वचनं द्रोणो रहः सूदमभाषत}
{अन्धकारेऽर्जुनायान्नं न देयं ते कदाचन}
{न चाख्येयमिदं चापि मद्वाक्यं विजयेत्वया}


\twolineshloka
{ततः कदाचिद्भुञ्जाने प्रववौ वायुरर्जुने}
{तेन तत्र प्रदीपः स दीप्यमानो विलोपितः}


\twolineshloka
{भुक्त एव तु कौन्तेयो नास्यादन्यत्र वर्तते}
{हस्तस्तेजस्विनस्तस्य अनुग्रहणकारणात्}


\twolineshloka
{तदभ्यासकृतं मत्वा रात्रावपि स पाण्डवः}
{योग्यां चक्रे महाबाहुर्धनुषा पाण्डुनन्दनः}


\threelineshloka
{तस्य ज्यातलनिर्घोषं द्रोणः शुश्राव भारत}
{उपेत्य चैनमुत्थाय परिष्वज्येदमब्रवीत् ॥द्रोण उवाच}
{}


\threelineshloka
{प्रयतिष्ये तथा कर्तुं यथा नान्यो धनुर्धरः}
{त्वत्समो भविता लोके सत्यमेतद्ब्रवीमि ते ॥वैशंपायन उवाच}
{}


\twolineshloka
{ततो द्रोणोऽर्जुनं भूयो हयेषु च गजेषु च}
{रथेषु भूमावपि च रणशिक्षामशिक्षयत्}


\twolineshloka
{गदायुद्धेऽसिचर्यायां तोमरप्रासशक्तिषु}
{द्रोणः संकीर्णयुद्धे च शिक्षयामास कौरवान्}


\twolineshloka
{तस्य तत्कौशलं श्रुत्वा धनुर्वेदजिघृक्षवः}
{सजानो राजपुत्राश्च समाजग्मुः सहस्रशः}


\threelineshloka
{`तान्सर्वाञ्शिक्षयामास द्रोणः शस्त्रभृतां वरः}
{'ततो निषादराजस्य हिरण्यधनुषः सुतः}
{एकलव्यो महाराज द्रोणमभ्याजगाम ह}


\threelineshloka
{न स तं प्रतिजग्राह नैषादिरिति चिन्तयन्}
{शिष्यं धनुषि धर्मज्ञस्तेषामेवान्ववेक्षया ॥`द्रोण उवाच}
{}


\threelineshloka
{शिष्योऽसि मम नैषादे प्रयोगे बलत्तरः}
{निवर्तस्व गृहानेव अनुज्ञातोऽसि नित्यशः ॥वैशंपायन उवाच}
{'}


\twolineshloka
{स तु द्रोणस्य शिरसा पादौ गृह्य परन्तपः}
{अरण्यमनुसंप्राप्य कृत्वा द्रोणं महीमयम्}


\twolineshloka
{तस्मिन्नाचार्यवृत्तिं च परमामास्थितस्तदा}
{इष्वस्त्रे योगमातस्थे परं नियममास्थितः}


\twolineshloka
{परया श्रद्धयोपेतो योगेन परमेण च}
{विमोक्षादानसन्धाने लघुत्वं परमाप सः}


\threelineshloka
{`लाघवं चास्त्रयोगं च नचिरात्प्रत्यपद्यत}
{'अथ द्रोणाभ्यनुज्ञाताः कदाचित्कुरुपाण्डवाः}
{रथैर्विनिर्ययुः सर्वे मृगयामरिमर्दनाः}


\twolineshloka
{तत्रोपकरणं गृह्य नरः कश्चिद्यदृच्छया}
{राजन्ननुजगामैकः श्वानमादाय पाण्डवान्}


\twolineshloka
{तेषां विचरतां तत्र तत्तत्कर्मचिकीर्षया}
{श्वाचरन्स पथा क्रीडन्नैषादिं प्रति जग्मिवान्}


\twolineshloka
{स कृष्णमलदिग्धाङ्गं कृष्णाजिनजटाघरम्}
{नैषादिं श्वा समालक्ष्य भषंस्तस्थौ तदन्तिके}


\twolineshloka
{तदा तस्याथ भषतः शुनः सप्त शरान्मुखे}
{लाघवं दर्शन्नस्त्रे मुमोच युगपद्यथा}


\twolineshloka
{स तु श्वा शरपूर्णास्यः पाण्डवानाजगाम ह}
{तं दृष्ट्वा पाण्डवा वीराः परं विस्मयमागताः}


\twolineshloka
{लाघवं शब्धवेधित्वं दृष्ट्वा तत्परमं तदा}
{प्रेक्ष्य तं व्रीडिताश्चासन्प्रशशंसुश्च सर्वशः}


\twolineshloka
{तं ततोऽन्वेषमाणास्ते वने वननिवासिनम्}
{ददृशुः पाण्डवा राजन्नस्यन्तमनिशं शरान्}


\threelineshloka
{न चैनमभ्यजानंस्ते तदा विकृतदर्शनम्}
{तथैनं परिपप्रच्छ्रुः को भवान्कस्य वेत्युत ॥एकलव्य उवाच}
{}


\threelineshloka
{निषादाधिपतेर्वीरा हिरण्यधनुषः सुतम्}
{द्रोणशिष्यं च मां वित्त धुर्वेदकृतश्रमम् ॥वैशंपायन उवाच}
{}


\twolineshloka
{ते तमाज्ञाय तत्त्वेन पुनरागम्य पाण्डवाः}
{यथा वृत्तं वने सर्वं द्रोणायाचख्युरद्भुतम्}


\twolineshloka
{कौन्तेयस्त्वर्जुनो राजन्नेकलव्यमनुस्मरन्}
{रहो द्रोणं समासाद्य प्रणयादिदमब्रवीत्}


\twolineshloka
{नन्वहं परिरभ्यैकः प्रीतिपूर्वमिदं वचः}
{भवतोक्तो न मे शिष्यस्त्वद्विशिष्टो भविष्यति}


\fourlineindentedshloka
{अथ कस्मान्मद्विशिष्टो लोकादपि च वीर्यवान्}
{अन्योऽस्ति भवतः शिष्यो निषादाधिपतेः सुतः}
{वैशंपायन उवाच}
{}


\twolineshloka
{मुहूर्तमिव तं द्रोणश्चिन्तयित्वा विनिश्चयम्}
{सव्यसाचिनमादाय नैषादिं प्रति जग्मिवान्}


\twolineshloka
{ददर्श मलदिग्धाङ्गं जटिलं चीरवाससम्}
{एकलव्यं धनुष्पाणिमस्यन्तमनिशं शरान्}


\twolineshloka
{एकलव्यस्तु तं दृष्ट्वा द्रोणमायान्तमन्तिकात्}
{अभिगम्योपसंगृह्य जगाम शिरसा महीम्}


\twolineshloka
{पूजयित्वा ततो द्रोणं विधिवत्स निषादजः}
{निवेद्य शिष्यमात्मानं तस्थौ प्राञ्जलिरग्रतः}


\twolineshloka
{ततो द्रोणोऽब्रवीद्राजन्नेकलव्यमिदं वचः}
{यदि शिष्योऽसि मे वीर वेतनं दीयतां मम}


\twolineshloka
{एकलव्यस्तु तच्छ्रुत्वा प्रीयमाणोऽब्रवीदिदम्}
{किं प्रयच्छामि भगवन्नाज्ञापयतु मां गुरुः}


\threelineshloka
{न हि किंचिददेयं मे गुरवे ब्रह्मवित्तम}
{वैशंपायन उवाच}
{तमब्रवीत्त्वयाङ्गुष्ठो दक्षिणो दीयतामिति}


\twolineshloka
{एकलव्यस्तु तच्छ्रुत्वा वचो द्रोणस्य दारुणम्}
{प्रतिज्ञामात्मनो रक्षन्सत्ये च नियतः सदा}


\twolineshloka
{तथैव हृष्टवदनस्तथैवादीनमानसः}
{छित्त्वाऽविचार्य तं प्रादाद्द्रोणायाङ्गुष्ठमात्मनः}


\twolineshloka
{ततः शरं तु नैषादिरङ्गुलीभिर्व्यकर्षत}
{न तथा च स शीघ्रोऽभूद्यथा पूर्वं नराधिप}


\twolineshloka
{ततोऽर्जुनः प्रीतमना बभूव विगतज्वरः}
{द्रोणश्च सत्यवागासीन्नान्योऽभिभविताऽर्जुनं}


\twolineshloka
{द्रोणस्य तु तदा शिष्यौ गदायोग्यौ बभूवतुः}
{दुर्योधनश्च भीमश्च सदा संरब्धणानसौ}


\twolineshloka
{अश्वत्थामा रहस्येषु सर्वेष्वभ्यधिकोऽभवत्}
{तथाऽतिपुरुषानन्यान्त्सारुकौ यमजावुभौ}


\twolineshloka
{युधिष्ठिरो रथश्रेष्ठः सर्वत्र तु धनञ्जयः}
{प्रथितः सागरान्तायां रथयूथपयूथपः}


\twolineshloka
{बुद्धियोगबलोत्साहः सर्वास्त्रेषु च निष्ठितः}
{अस्त्रे गुर्वनुरागे च विशिष्टोऽभवदर्जुनः}


\twolineshloka
{तुल्येष्वस्त्रोपदेशेषु सौष्ठवेन च वीर्यवान्}
{एकः सर्वकुमाराणां बभूवातिरथोऽर्जुनः}


\twolineshloka
{प्राणाधिकं भीमसेनं कृतविद्यं धनञ्जयम्}
{धार्तराष्ट्रा दुरात्मानो नामृष्यन्त परस्परम्}


\twolineshloka
{तांस्तु सर्वान्समानीय सर्वविद्यास्त्रशिक्षितान्}
{द्रोणः प्रहरणज्ञाने जिज्ञासुः पुरुषर्षभः}


\threelineshloka
{कृत्रिमं भासमारोप्य वृक्षाग्रे शिल्पिभिः कृतम्}
{अविज्ञातं कुमाराणां लक्ष्यभूतमुपादिशत् ॥द्रोण उवाच}
{}


\twolineshloka
{शीघ्रं भन्तः सर्वेऽपि धनूंष्यादाय सर्वशः}
{भासमेतं समुद्दिश्य तिष्ठध्वं सन्धितेषतः}


\threelineshloka
{मद्वाक्यसमकालं तु शिरोऽस्य विनिपात्यताम्}
{एकैकशो नियोक्ष्यामि तथा कुरुत पुत्रकाः ॥वैशंपायन उवाच}
{}


\twolineshloka
{ततो युधिष्ठिरं पूर्वमुवाचाङ्गिरसां वरः}
{संधत्स्व बामं दुर्धर्ष मद्वाक्यान्ते विमुञ्चतम्}


\twolineshloka
{ततो युधिष्ठिरः पूर्वं धनुर्गृह्य परन्तपः}
{तस्थौ भासं समुद्दिश्य गुरुवाक्यप्रचोदितः}


\twolineshloka
{ततो विततधन्वानं द्रोणस्तं कुरुनन्दनम्}
{स मुहूर्तादुवाचेदं वचनं भरतर्षभ}


\twolineshloka
{पश्यसि त्वं द्रुमाग्रस्थं भासं नरवरात्मज}
{पश्यामीत्येवमाचार्यं प्रत्युवाच युधिष्ठिरः}


\twolineshloka
{स मुहूर्तादिव पुनर्द्रोणस्तं प्रत्यभाषत}
{अथ वृक्षमिमं मां वा भ्रातॄन्वाऽपि प्रपश्यसि}


\twolineshloka
{तमुवाच स कौन्तेयः पश्याम्येनं नवस्पतिम्}
{भन्तं च तथा भ्रातॄन्भासं चेति पुनःपुनः}


\twolineshloka
{तमुवाचापसर्पेति द्रोणोऽप्रीतमना इव}
{नैतच्छक्यं त्वया वेद्धुं लक्ष्यमित्येव कुत्सयन्}


\twolineshloka
{ततो दुर्योधनादींस्तान्धार्तराष्ट्रान्महायशाः}
{तेनैव क्रमयोगेन जिज्ञासुः पर्यपृच्छत}


\twolineshloka
{अन्यांश्च शिष्यान्भीमादीन्राज्ञश्चैवान्यदेशजान्}
{यदा च सर्वे तत्सर्वं पश्याम इति कुत्सिताः}


\chapter{अध्यायः १४३}
\twolineshloka
{वैशंपायन उवाच}
{}


\twolineshloka
{ततो धनञ्जयं द्रोणः स्मयमानोऽभ्यभाषत}
{त्वयेदानीं प्रहर्तव्यमेतल्लक्ष्यं विलोक्यताम्}


\twolineshloka
{मद्वाक्यसमकालं ते मोक्तव्योऽत्र भवेच्छरः}
{वितत्य कार्मुकं पुत्र तिष्ठ तावन्मुहूर्तकम्}


\twolineshloka
{एवमुक्तः सव्यसाची मण्डलीकृतकार्मुकः}
{तस्थौ भासं समुद्दिश्य गुरुवाक्यप्रचोदितः}


\twolineshloka
{मुहूर्तादिव तं द्रोणस्तथैव समभाषत}
{पश्यस्येनं स्थितं भासं द्रुमं मामपि चार्जुन}


\twolineshloka
{पश्याम्येकं भासमिति द्रोणं पार्थोऽभ्यभाषत}
{न तु वृक्षं भवन्तं वा पश्यामीति च भारत}


\twolineshloka
{ततः प्रीतमना द्रोणो मुहूर्तादिव तं पुनः}
{प्रत्यभाषत दुर्धर्षः पाण्डवानां महारथम्}


\twolineshloka
{भासं पश्यसि यद्येनं तथा ब्रूहि पुनर्वचः}
{शिरः पश्यामि भासस्य न गात्रमिति सोऽब्रवीत्}


\twolineshloka
{अर्जुनेनैवमुक्तस्तु द्रोणो हृष्टतनूरुहः}
{मुञ्चस्वेत्यब्रवीत्पार्थं स मुमोचाविचारयन्}


\twolineshloka
{ततस्तस्य गस्थस्य क्षुरेण निशितेन च}
{शिर उत्कृत्य तरसा पातयामास पाण्डवः}


\twolineshloka
{तस्मिन्कर्मणि संसिद्धे पर्यष्वजत पाण्डवम्}
{मेने च द्रुपदं सङ्ख्ये सानुबन्धं पराजितम्}


\twolineshloka
{कस्य चित्त्वथ कालस्य सशिष्योऽङ्गिरसां वरः}
{जगाम गङ्गामभितो मज्जितुं भरतर्षभ}


\twolineshloka
{अवगाढमथो द्रोणं सलिले सलिलेचरः}
{ग्राहो जग्राह बलवाञ्जङ्घान्ते कालचोदितः}


\twolineshloka
{स समर्थोऽपि मोक्षाय शिष्यान्सर्वानचोदयत्}
{ग्राहं हत्वा तु मोक्ष्यध्वं मामिति त्वरयन्निव}


\twolineshloka
{तद्वाक्यसमकालं तु बीभत्सुर्निशितैः शरैः}
{अवार्यैः पञ्चभिर्ग्राहं मग्नमम्भस्यताडयत्}


\twolineshloka
{इतरे त्वथ संमूढास्तत्रपत्र प्रपेदिरे}
{तं तु दृष्ट्वा क्रियोपेतं द्रोणोऽमन्यत पाण्डवम्}


\twolineshloka
{विशिष्टं सर्वशिष्येभ्यः प्रीतिमांश्चाभवत्तदा}
{स पार्थबाणैर्बहुधा खण्डशः परिकल्पितः}


\twolineshloka
{ग्राहः पञ्चत्वमापेदे जङ्घां त्यक्त्वा महात्मनः}
{`सर्वक्रियाभ्यनुज्ञानात्तथा शिष्यान्समानयत्}


\twolineshloka
{दुर्योधनं चित्रसेनं दुःशासनविविंशती}
{अर्जुनं च समानीय ह्यश्वत्थामानमेव च}


\twolineshloka
{शिशुकं मृण्मयं कृत्वा द्रोणो गङ्गाजले ततः}
{शिष्याणां पश्यतां चैव क्षिपति स्म महाभुजः}


\twolineshloka
{चक्षुषी वाससा चैव बद्ध्वा प्रादाच्छरासनम्}
{शिशुकं विद्ध्यतेमं वै जलस्थं बद्धचक्षुषः}


\twolineshloka
{तत्क्षणेनैव बीभत्सुरावापैर्दशभिर्वशी}
{पञ्चकैरनुविव्याध मग्नं शिशुकमम्भसि}


\threelineshloka
{ताः स दृष्ट्वा क्रियाः सर्वा द्रोणोऽमन्यत पाण्डवम्}
{विशिष्टं सर्वशिष्येभ्यः प्रीतिमांश्चाभवत्तदा}
{'तथाब्रवीन्महात्मानं भारद्वाजो महारथम्}


\twolineshloka
{गृहाणेदं महाबाहो विशिष्टमतिदुर्धरम्}
{अस्त्रं ब्रह्मशिरो नाम सप्रयोगनिवर्तनम्}


\twolineshloka
{न च ते मानुषेष्वेतत्प्रयोक्तव्यं कथंचन}
{जगद्विनिर्दहेदेतदल्पतेजसि पातितम्}


\twolineshloka
{असामान्यमिदं तात लोकेष्वस्त्रं निगद्यते}
{तद्धारयेथाः प्रयतः शृणु चेदं वचो मम}


\twolineshloka
{बाधेतामानुषः शत्रुर्यदि त्वां वीर कश्चन}
{तद्वधाय प्रयुञ्जीथास्तदस्त्रमिदमाहवे}


\threelineshloka
{तथेति संप्रतिश्रुत्य बीभत्सुः स कृताञ्जलिः}
{जग्राह परमास्त्रं तदाह चैनं पुनर्गुरुः}
{भविता त्वत्समो नान्यः पुमाँल्लोके धनुर्धरः}


\chapter{अध्यायः १४४}
\twolineshloka
{वैशंपायन उवाच}
{}


\twolineshloka
{कृतास्त्रान्धार्तराष्ट्रांश्च पाण्डुपुत्रांश्च भारत}
{दृष्ट्वा द्रोणोऽब्रवीद्राजन्धृतराष्ट्रं जनेश्वरम्}


\twolineshloka
{कृपस्य सोमदत्तस्य वाह्लीकस्य च धीमतः}
{गाङ्गेयस्य च सान्निध्ये व्यासस्य विदुरस्य च}


\twolineshloka
{राजन्संप्राप्तविद्यास्ते कुमाराः कुरुसत्तम}
{ते दर्शयेयुः स्वां शिक्षां राजन्ननुमते तव}


\threelineshloka
{ततोऽब्रवीन्महाराजः प्रहृष्टेनान्तरात्मना}
{धृतराष्ट्र उवाच}
{भारद्वाज महत्कर्म कृतं ते द्विजसत्तम}


\twolineshloka
{यदानुमन्यसे कालं यस्मिन्देशे यथायथा}
{तथातथा विधानाय स्वयमाज्ञापयस्व माम्}


\twolineshloka
{स्पृहयाम्यद्य निर्वेदान्पुरुषाणां सचक्षुषाम्}
{अस्त्रहेतोः पराक्रान्तान्ये मे द्रक्ष्यन्ति पुत्रकान्}


\twolineshloka
{क्षत्तर्यद्गुरुराचार्यो ब्रवीति कुरु तत्तथा}
{न हीदृशं प्रियं मन्ये भविता धर्मवत्सल}


\twolineshloka
{ततो राजानमामन्त्र्य विदुरानुमतोपि हि}
{भारद्वाजो महाप्राज्ञो मापयामास मेदिनीम्}


\twolineshloka
{समामवृक्षां निर्गुलमामुदक्प्रवणसंस्थिताम्}
{तस्यां भूमौ बलिं चक्रे तिथौ नक्षत्रपूजिते}


\twolineshloka
{अवघुष्टं पुरं चापि तदर्थं भरतर्षभ}
{रङ्गभूमौ सुविपुलं शास्त्रदृष्टं यथाविधि}


\twolineshloka
{प्रेक्षागारं सुविहितं चक्रस्ते तस्य शिल्पिनः}
{रक्षां सर्वायुधोपेतां स्त्रीणां चैव नरर्षभ}


\twolineshloka
{मञ्चांश्च कारयामासुर्यत्र जानपदा जनाः}
{विपुलानुच्छ्रयोपेताञ्शिबिकाश्च महाधनाः}


\threelineshloka
{तस्मिंस्ततोऽहनि प्राप्ते राजा ससचिवस्तदा}
{`सान्तःपुरः सहामात्यो व्यासस्यानुमते तदा}
{'भीष्मं प्रमुखतः कृत्वा कृपं चाचार्यसत्तमम्}


\twolineshloka
{`बाह्लीकं सोमदत्तं च भूरिश्रवसमेव च}
{कुरूनन्यांश्च सचिवानादाय नगराद्बहिः}


% Check verse!
रङ्गभूमिं समासाद्य ब्राह्मणैः सहितो नृपः ॥'
\twolineshloka
{मुक्ताजालपरिक्षिप्तं वैदूर्यममिशोभितम्}
{शातकुम्भमयं दिव्यं प्रेक्षागारमुपागमत्}


\twolineshloka
{गान्धारी च महाभागा कुन्ती च जयतां वर}
{स्त्रियश्च राज्ञः सर्वास्ताः सप्रेष्याः सपरिच्छदाः}


\twolineshloka
{हर्षादारुरुहुर्मञ्चान्मेरुं देवस्त्रियो यथा}
{ब्राह्मणक्षत्रियाद्यं च चातुर्वर्ण्यं पुराद्द्रुतम्}


\twolineshloka
{दर्शनेप्सुः समभ्यागात्कुमाराणां कृतास्त्रताम्}
{क्षणेनैकस्थतां तत्र दर्शनेप्सुर्जगाम ह}


\twolineshloka
{प्रवादितैश्च वादित्रैर्जनकौतूहलेन च}
{महार्णव इव क्षुब्धः समाजः सोऽभवत्तदा}


\twolineshloka
{ततः शुक्लाम्बरधरः शुक्लयज्ञोपवीतवान्}
{शुक्लकेशः सितश्मश्रुः शुक्लाल्यानुलेपनः}


\twolineshloka
{रङ्गमध्यं तदाचार्यः सपुत्रः प्रविवेश ह}
{नभो जलधरैर्हीनं साङ्गारक इवांशुमान्}


\twolineshloka
{व्यासस्यानुमते चक्रे बलिं बलवतां वरः}
{ब्राह्मणांस्तु सुमन्त्रज्ञान्कारयामास मङ्गलम्}


\twolineshloka
{`सुवर्णमणिरत्नानि वस्त्राणि विविधानि च}
{प्रददौ दक्षिणां राजा द्रोणाय च कृपाय च ॥'}


\twolineshloka
{सुखपुण्याहघोषस्य पुण्यस्य समनन्तरम्}
{विविशुर्विविधं गृह्य शस्त्रोपकरणं नराः}


\twolineshloka
{ततो बद्धाङ्गुलित्राणा बद्धकक्ष्या महारथाः}
{बद्धथूणाः सधनुषो विविशुर्भरतर्षभाः}


\twolineshloka
{`रङ्गमध्ये स्थितं द्रोणमभिवाद्य नरर्षभाः}
{चक्रुः पूजां यथान्यायं द्रोणस्य च कृपस्य च}


\twolineshloka
{आशीर्भिश्च प्रयुक्ताभिः सर्वे संहृष्टमानसाः}
{अभिवाद्य पुनः शस्त्रान्बलिपुष्पैः समर्चितान्}


\twolineshloka
{रक्तचन्दनसंमिश्रैः स्वयमर्चन्ति कौरवाः}
{रक्तचन्दनदिग्धाश्च रक्तमाल्यानुधारिणः}


\twolineshloka
{सर्वे रक्तपताकाश्च सर्वे रक्तान्तलोचनाः}
{द्रोणेन समनुज्ञाता गृह्य शस्त्रं परन्तपाः}


\twolineshloka
{धनूंषि पूर्व संगृह्य तप्तकाञ्चनभूषिताः}
{सज्यानि विविधाकाराः शरैः सन्धाय कौरवा}


% Check verse!
ज्याघोषं तलघोषं च कृत्वा भूतान्यमोहयन् ॥'
\twolineshloka
{अनुज्येष्ठं च ते तत्र युधिष्ठिरपुनरोगमाः}
{चक्रुरस्त्रं महावीर्याः कुमाराः परमाद्भुतम्}


\twolineshloka
{`केषांचित्तत्र माल्येषु शरा निपतिता नृप}
{केषांचित्पुष्पमुकुटे निपतन्ति स्म सायकाः}


\twolineshloka
{केचिल्लक्ष्याणि विविधैर्बाणैराहितलक्षणैः}
{बिभिदुर्लाघवोत्सृष्टैर्गुरूणि च लघूनि च ॥'}


\twolineshloka
{केचिच्छराक्षेपभयाच्छिरांस्यवननामिरे}
{मनुजा धृष्टमपरे वीक्षाञ्चक्रुः सुविस्मिताः}


\twolineshloka
{ते स्म लक्ष्याणि बिभिदुर्बाणैर्नामाङ्कशोभितैः}
{विविधैर्लाघवोत्सृष्टैरुह्यन्तो वाजिभिर्द्रुतम्}


\twolineshloka
{तत्कुमारबलं तत्र गृहीतशरकार्मुकम्}
{गन्धर्वनगराकारं प्रेक्ष्य ते विस्मिताभवन्}


\twolineshloka
{सहसा चुक्रुशुश्चान्ये नराः शतसहस्रशः}
{विस्मयोत्फुल्लनयनाः साधुसाध्विति भारत}


\twolineshloka
{कृत्वा धनुषि ते मार्गान्रथचर्यासु चासकृत्}
{गजपृष्ठेऽश्वपृष्ठे च नियुद्धे च महाबलाः}


\twolineshloka
{गृहीतखड्गचर्माणस्ततो भूयः प्रहारिणः}
{त्सरुमार्गान्यथोद्दिष्टांश्चेरुः सर्वासु भूमिषु}


\twolineshloka
{लाघवं सौष्ठवं शोभां स्थिरत्वं दृढमुष्टिताम्}
{ददृशुस्तत्र सर्वेषां प्रयोगं खड्गचर्मणोः}


\twolineshloka
{अथ तौ नित्यसंहृष्टौ सुयोधनवृकोदरौ}
{अवतीर्णौ गदाहस्तावेकशृङ्गाविवाचलौ}


\twolineshloka
{बद्धकक्ष्यौ महाबाहू पौरुषे पर्यवस्थितौ}
{बृंहन्तौ वासिताहेतोः समदाविव कुञ्जरौ}


\twolineshloka
{तौ प्रदक्षिणसव्यानि मण्डलानि महाबलौ}
{चेरतुर्मण्डलगतौ समदाविव कुञ्जरौ}


\twolineshloka
{विदुरो धृतराष्ट्राय गान्धार्याः पाण्डवारणिः}
{न्यवेदयेतां तत्सर्वं कुमाराणां विचेष्टितम्}


\chapter{अध्यायः १४५}
\twolineshloka
{वैशंपायन उवाच}
{}


\twolineshloka
{कुरुराजे हि रङ्गस्थे भीमे च बलिनां वरे}
{पक्षपातकृतस्नेहः स द्विधेवाभवज्जनः}


\twolineshloka
{जय हे कुरुराजेति जय हे भीम इत्युत}
{पुरुषाणां सुविपुलाः प्रणादाः सहसोत्थिताः}


\twolineshloka
{ततः क्षुब्धार्णवनिभं रङ्गमालोक्य बुद्धिमान्}
{भारद्वाजः प्रियं पुत्रमश्वत्थामानमब्रवीत्}


\threelineshloka
{वारयैतौ महावीर्यौ कृतयोग्यावुभावपि}
{मा भूद्रङ्गप्रकोपोऽयं भीमदुर्योधनोद्भवः ॥वैशंपायन उवाच}
{}


\threelineshloka
{`तत उत्थाय वेगेन अश्वत्थामा न्यवारयत्}
{गुरोराज्ञा भीम इति गान्धारे गुरुशासनम्}
{अलं शिक्षाकृतं वेगमलं साहसमित्युत ॥'}


\twolineshloka
{ततस्तावुद्यतगतौ गुरुपुत्रेण वारितौ}
{युगान्तानिलसंक्षुब्धौ महावेलाविवार्णवौ}


\twolineshloka
{ततो रङ्गाङ्गणगतो द्रोणो वचनमब्रवीत्}
{निवार्य वादित्रगणं महामेघनिभस्वनम्}


\twolineshloka
{यो मे पुत्रात्प्रियतरः सर्वशस्त्रविशारदः}
{ऐन्द्रिरिन्द्रानुजसमः स पार्थो दृश्यतामिति}


\twolineshloka
{आचार्यवचनेनाथ कृतस्वस्त्ययनो युवा}
{बद्धगोधाङ्गुलित्राणः पूर्णतूणः सकार्मुकः}


\twolineshloka
{काञ्चनं कवचं बिभ्रत्प्रत्यदृश्य फाल्गुनः}
{सार्कः सेन्द्रायुधतडित्ससन्ध्य इव तोयदः}


\threelineshloka
{ततः सर्वस्य रङ्गस्य समुत्पिञ्जलकोऽभवत्}
{प्रावाद्यन्त च वाद्यानि सशङ्खानि समन्ततः ॥प्रेक्षका ऊचुः}
{}


\twolineshloka
{एष कुन्तीसुतः श्रीमानेष मध्यमपाण्डवः}
{एष पुत्रो महेन्द्रस्य कुरूणामेष रक्षिता}


\threelineshloka
{एषोऽस्त्रविदुषां श्रेष्ठ एष धर्मभृतां वरः}
{एष शीलवतां चापि शीलज्ञाननिधिः परः ॥वैशंपायन उवाच}
{}


\twolineshloka
{इत्येवं तुमुला वाचः शुश्रुवुः प्रेक्षकेरिताः}
{कुन्त्याः प्रस्रवसंयुक्तैरस्रैः क्लिन्नमुरोऽभवत्}


\twolineshloka
{तेन शब्देन महता पूर्णश्रुतिरथाब्रवीत्}
{धृतराष्ट्रो नरश्रेष्ठो विदुरं हृष्टमानसः}


\threelineshloka
{क्षत्तः क्षुब्धार्णवनिभः किमेष सुमहास्वनः}
{सहसैवोत्थितो रङ्गे भिन्दन्निव नभस्तलम् ॥विदुर उवाच}
{}


\threelineshloka
{एष पार्थो महाराज फाल्गुनः पाण्डुनन्दनः}
{अवतीर्णः सकवचस्तत्रैव सुमिहास्वनः ॥धृतराष्ट्र उवाच}
{}


\threelineshloka
{धन्योऽस्म्यनुगृहीतोऽस्मि रक्षितोऽस्मि महामते}
{पृथारणिसमुद्भूतैस्त्रिभिः पाण्डववह्निभिः ॥वैशंपायन उवाच}
{}


\twolineshloka
{तस्मिन्प्रमुदिते रङ्गे कथंचित्प्रत्युपस्थिते}
{दर्शयामास बीभत्सुराचार्यायास्त्रलाघवम्}


\twolineshloka
{आग्नेयेनासृजद्वह्निं वारुणेनासृजत्पयः}
{वायव्येनासृजद्वह्निं पार्जन्येनासृजद्धनान्}


\twolineshloka
{भौमेन प्रासृजद्भूमिं पार्वतेनासृजद्गिरीन्}
{अन्तर्धानेन चास्त्रेण पुनरन्तर्हितोऽभवत्}


\twolineshloka
{क्षणात्प्रांशुः क्षणाद्ध्रस्वः क्षणाच्च रथधूर्गतः}
{क्षणेन रथमध्यस्थः क्षणेनावतरन्महीम्}


\twolineshloka
{सुकुमारं च सूक्ष्मं च गुरु चापि गुरुप्रियः}
{सौष्ठवेनाभिसंयुक्तः सोऽविध्यद्विविधैः शरैः}


\twolineshloka
{भ्रमतश्च वराहस्य लोहस्य प्रमुखे समम्}
{पञ्चबाणानसंक्तान्संमुमोचैकबाणवत्}


\twolineshloka
{गव्ये विषाणकोशे च चले रज्ज्ववलम्बिनि}
{निचखान महावीर्यः सायकानेकविंशतिम्}


\twolineshloka
{इत्येवमादि सुमहत्खड्गे धनुषि चानघ}
{गदायां शस्त्रकुशलो मण्डलानि ह्यदर्शयत्}


\twolineshloka
{ततः समाप्तभूयिष्ठे तस्मिन्कर्मणि भारत}
{मन्दीभूते समाजे च वादित्रस्य च निःस्वने}


\twolineshloka
{द्वारदेशात्समुद्भूतो माहात्म्यबलसूचकः}
{वज्रनिष्पेषसदृशः शुश्रुवे भुजनिःस्वनः}


\twolineshloka
{दीर्यन्ते किं नु गिरयः किंस्विद्भूमिर्विदीर्यते}
{किंस्विदापूर्यते व्योम जलधाराघनैर्घनैः}


\twolineshloka
{रङ्गस्यैवं मतिरभूत्क्षणेन वसुधाधिप}
{द्वारं चाभिमुखाः सर्वे बभूवुः प्रेक्षकास्तदा}


\twolineshloka
{पञ्चभिर्भ्रातृभिः पार्थैर्द्रोणः परिवृतो वभौ}
{पञ्चतारेण संयुक्तः सावित्रेणेव चन्द्रमाः}


\twolineshloka
{अश्वत्थाम्ना च सहितं भ्रातॄणां शतमूर्जितम्}
{दुर्योधनममित्रघ्नमुत्थितं पर्यवारयत्}


\twolineshloka
{स तैस्तदा भ्रातृभिरुद्यतायुधै-र्गदाग्रपाणिः समवस्थितैर्वृतः}
{बभौ यथा दानवसंक्षये पुरापुनन्दरो देवगणैः समावृतः}


\chapter{अध्यायः १४६}
\twolineshloka
{वैशंपायन उवाच}
{}


\threelineshloka
{`एतस्मिन्नेव काले तु तस्मिञ्जनसमागमे}
{'दत्तेऽवकाशे पुरुषैर्विस्मयोत्फुल्ललोचनः}
{विवेश रङ्गं विस्तीर्णं कर्णः परपुरञ्जयः}


\twolineshloka
{सहजं कवचं बिभ्रत्कुण्डलोद्द्योतिताननः}
{स धनुर्बद्धनिस्त्रिंशः पादचारीव पर्वतः}


\twolineshloka
{कन्यागर्भः पृथुयशाः पृथायाः पृथुलोचनः}
{तीक्ष्णांशोर्भास्करस्यांशः कर्णोऽरिगणसूदनः}


\twolineshloka
{सिंहर्षभगजेन्द्राणां बलवीर्यपराक्रमः}
{दीप्तिकान्तिद्युतिगुणैः सूर्येन्दुज्वलनोपमः}


\twolineshloka
{प्रांशुः कनकतालाभः सिंहसंहननो युवा}
{असङ्ख्येयगुणः श्रीमान्भास्करस्यात्मसंभवः}


\twolineshloka
{स निरीक्ष्य महाबाहुः सर्वतो रङ्गमण्डलम्}
{प्रणामं द्रोणकृपयोर्नात्यादृतमिवाकरोत्}


\twolineshloka
{स समाजजनः सर्वो निश्चलः स्थिरलोचनः}
{कोऽयमित्यागतक्षोभः कौतूहलपरोऽभवत्}


\twolineshloka
{सोऽब्रवीन्मेघगम्भीरस्वरेण वदतां वरः}
{भ्राता भ्रातरमज्ञातं सावित्रः पाकशासनिम्}


\twolineshloka
{पार्थ यत्ते कृतं कर्म विशेषवदहं ततः}
{करिष्ये पश्यतां नॄणां माऽऽत्मना विस्मयं गमः}


\twolineshloka
{असमाप्ते ततस्तस्य वचने वदतां वर}
{यन्त्रोत्क्षिप्त इवोत्तस्थौ क्षिप्रं वै सर्वतो जनः}


\twolineshloka
{प्रीतिश्च मनुजव्याघ्र दुर्योधनमुपाविशत्}
{ह्रीश्च क्रोधश्च बीभत्सुं क्षणेनान्वाविवेश ह}


\twolineshloka
{ततो द्रोणाभ्यनुज्ञातः कर्णः प्रियरणः सदा}
{यत्कृतं तत्र पार्थेन तच्चकार महाबलः}


\twolineshloka
{अथ दुर्योधनस्तत्र भ्रातृभिः सह भारत}
{कर्णं परिष्वज्य मुदा ततो वचनमब्रवीत्}


\threelineshloka
{स्वागतं ते महाबाहो दिष्ट्या प्राप्तोऽसि मानद}
{अहं च कुरुराज्यं च यथेष्टमुपभुज्यताम् ॥कर्ण उवाच}
{}


\threelineshloka
{कृतं सर्वमहं मन्ये सखित्वं च त्वया वृणे}
{द्वन्द्वयुद्धं च पार्थेन कर्तुमिच्छाम्यहं प्रभो ॥`वैशंपायन उवाच}
{}


\twolineshloka
{एवमुक्तस्तु कर्णेन राजन्दुर्योधनस्तदा}
{कर्णं दीर्घाञ्चितभुजं परिष्वज्येदमब्रवीत् ॥'}


\threelineshloka
{भुङ्क्ष्व भोगान्मया सार्धं बन्धूनां प्रियकृद्भव}
{दुर्हृदां कुरु सर्वेषां मूर्ध्नि पादमरिन्दम ॥वैशंपायन उवाच}
{}


\threelineshloka
{ततः क्षिप्तमिवात्मानं मत्वा पार्थोऽभ्यभाषत}
{कर्णं भ्रातृसमूहस्य मध्येऽचलमिव स्थितम् ॥अर्जुन उवाच}
{}


\threelineshloka
{अनाहूतोपसृष्टानामनाहूतोपजल्पिनाम्}
{ये लोकास्तान्हतः कर्ण मया त्वं प्रतिपत्स्यसे ॥कर्ण उवाच}
{}


\twolineshloka
{रङ्गोऽयं सर्वसामान्यः किमत्र तव फाल्गुन}
{वीर्यश्रेष्ठाश्च राजानो बलं धर्मोऽनुवर्तते}


\threelineshloka
{किं क्षेपैर्दुर्बलायासैः शरैः कथय भारत}
{गुरोः समक्षं यावत्ते हराम्यद्य शिरः शरैः ॥वैशंपायन उवाच}
{}


\twolineshloka
{ततो द्रोणाभ्यनुज्ञातः पार्तः परपुरञ्जयः}
{भ्रातृभिस्त्वरयाश्लिष्टो रणायोपजगाम तम्}


\twolineshloka
{ततो दुर्योधनेनापि स भ्रात्रा समरोद्यतः}
{परिष्वक्तः स्थितः कर्णः प्रगृह्य सशरं धनुः}


\twolineshloka
{ततः सविद्युत्स्तनितैः सेन्द्रायुधपुरोगमैः}
{आवृतं गगनं मेघैर्बलाकापङ्क्तिहासिभिः}


\twolineshloka
{ततः स्नेहाद्धरिहयं दृष्ट्वा रङ्गावलोकिनम्}
{भास्करोऽप्यनयन्नाशं समीपोपगतान्घनान्}


\twolineshloka
{मेघच्छायोपगूढस्तु ततोऽदृश्यत फाल्गुनः}
{सूर्यातपपरिक्षिप्तः कर्णोऽपि समदृश्यत}


\twolineshloka
{धार्तराष्ट्रा यतः कर्णस्तस्मिन्देशे व्यवस्थिताः}
{भारद्वाजः कृपो भीष्मो यतः पार्थस्ततोऽभवन्}


\twolineshloka
{द्विधा रङ्गः समभवत्स्त्रीणां द्वैधमजायत}
{कुन्तिभोजसुता मोहं विज्ञातार्था जगाम ह}


\twolineshloka
{तां तथा मोहमापन्नां विदुरः सर्वधर्मवित्}
{कुन्तीमाश्वासयामास प्रेष्याभिश्चन्दनोदकैः}


\twolineshloka
{ततः प्रत्यागतप्राणा तावुभौ परिदंशितौ}
{पुत्रौ दृष्ट्वा सुसंभ्रान्ता नान्वपद्यत किंचन}


\twolineshloka
{तावुद्यतमहाचापौ कृपः शारद्वतोऽब्रवीत्}
{द्वन्द्वयुद्धसमाचारे कुशलः सर्वधर्मवित्}


\twolineshloka
{अयं पृथायास्तनयः कनीयान्पाण्डुनन्दनः}
{कौरवो भवता सार्धं द्वन्द्वयुद्धं करिष्यति}


\twolineshloka
{त्वमप्येवं महाबाहो मातरं पितरं कुलम्}
{कथयस्व नरेन्द्राणां येषां त्वं कुलभूषणम्}


\threelineshloka
{ततो विदित्वा पार्थस्त्वां प्रतियोत्स्यति वा न वा}
{वृथाकुलसमाचारैर्न युध्यन्ते नृपात्मजाः ॥वैशंपायन उवाच}
{}


\threelineshloka
{एवमुक्तस्य कर्णस्य व्रीडावनतमाननम्}
{बभौ वर्षाम्बुविक्लिन्नं पद्ममागलितं यथा ॥दुर्योधन उवाच}
{}


\twolineshloka
{आचार्य त्रिविधा योनी राज्ञां शास्त्रविनिश्चये}
{सत्कुलीनश्च शूरश्च यश्च सेनां प्रकर्षति}


\twolineshloka
{`अद्भ्योऽग्निर्ब्रह्मतः क्षत्रमश्मनो लोहमुत्थितम्}
{तेषां सर्वत्रगं तेजः स्वासु योनिषु शाम्यति ॥'}


\threelineshloka
{यद्ययं फाल्गुनो युद्धे नाराज्ञा योद्धुमिच्छति}
{तस्मादेषोऽङ्गविषये मया राज्येऽभिषिच्यते ॥वैशंपायन उवाच}
{}


\twolineshloka
{`ततो राजानमामन्त्र्य गाङ्गेयं च पितामहम्}
{अभिषेकस्य संभारान्समानीय द्विजातिभिः}


\twolineshloka
{गोसहस्रायुतं दत्त्वा युक्तानां पुण्यकर्मणाम्}
{अर्होऽयमङ्गराज्यस्य इति वाच्य द्विजातिभिः'}


\twolineshloka
{ततस्तस्मिन्क्षणे कर्णः सलाजकुसुमैर्घटैः}
{काञ्चनैः काञ्चने पीठे मन्त्रविद्भिर्महारथः}


\twolineshloka
{अभिषिक्तोऽङ्गराजे स श्रिया युक्तो महाबलः}
{`स मौलिहारकेयूरः सहस्ताभरणाङ्गदः}


\twolineshloka
{राजलिङ्गैस्तथाऽन्यैश्च भूषितो भूषणैः शुभैः}
{'सच्छत्रवालव्यजनो जयशब्दोत्तरेण च}


\twolineshloka
{उवाच कौरवं राजन्वचनं स वृषस्तदा}
{अस्य राज्यप्रदानस्य सदृशं किं ददानि ते}


\twolineshloka
{प्रब्रूहि राजशार्दूल कर्ता ह्यस्मि तथा नृप}
{अत्यन्तं सख्यमिच्छामीत्याह तं स सुयोधनः}


\twolineshloka
{एवमुक्तस्ततः कर्णस्तथेति प्रत्युवाच तम्}
{हर्षाच्चोभौ समाश्लिष्य परां मुदमवापतुः}


\chapter{अध्यायः १४७}
\twolineshloka
{वैशंपायन उवाच}
{}


\twolineshloka
{ततः स्रस्तोत्तरपटः सप्रस्वेदः सवेपथुः}
{विवेशाधिरथो रङ्गं यष्टिप्राणो ह्वयन्निव}


\twolineshloka
{तमालोक्य धनुस्त्यक्त्वा पितृगौरवयन्त्रितः}
{कर्णोऽभिषेकार्द्रशिराः शिरसा समवन्दत}


\twolineshloka
{ततः पादाववच्छाद्य पटान्तेन ससंभ्रमः}
{पुत्रेति परिपूर्णार्थमब्रवीद्रथसारथिम्}


\twolineshloka
{परिष्वज्य च तस्याथ मूर्धानं स्नेहविक्लवः}
{अङ्गराज्याभिषेकार्द्रमश्रुभिः सिषिचे पुनः}


\twolineshloka
{तं दृष्ट्वा सूतपुत्रोऽयमिति संचिन्त्य पाण्डवः}
{भीमसेनस्तदा वाक्यमब्रवीत्प्रहसन्निव}


\twolineshloka
{न त्वमर्हसि पार्थेन सूतपुत्र रणे वधम्}
{कुलस्य सदृशस्तूर्णं प्रतोदो गृह्यतां त्वया}


\threelineshloka
{अङ्गराज्यं च नार्हस्त्वमुपभोक्तुं नराधम}
{श्वा हुताशसमीपस्थं पुरोडाशमिवाध्वरे ॥वैशंपायन उवाच}
{}


\twolineshloka
{एवमुक्तस्ततः कर्णः किंचित्प्रस्फुरिताधरः}
{गगनस्थं विनिःश्वस्य दिवाकरमुदैक्षत}


\twolineshloka
{ततो दुर्योधनः कोपादुत्पपात महाबलः}
{भ्रातृपद्मवनात्तस्मान्मदोत्कट इव द्विपः}


\twolineshloka
{सोऽब्रवीद्भीमकर्माणं भीमसेनमवस्थितम्}
{वृकोदर न युक्तं ते वचनं वक्तुमीदृशम्}


\twolineshloka
{क्षत्रियाणां बलं ज्यष्ठं योक्तव्यं क्षत्रबन्धुना}
{शूराणां च नदीनां च प्रभवो दुर्विभावनः}


\twolineshloka
{सलिलादुत्थितो वह्निर्येन व्याप्तं चराचरम्}
{दधीचस्यास्थितो वज्रं कृतं दानवसूदनम्}


\twolineshloka
{आग्नेयः कृत्तिकापुत्रो रौद्रो गाङ्गेय इत्यपि}
{श्रूयते भगवान्देवः सर्वगुह्यमयो गुहः}


\twolineshloka
{क्षत्रियेभ्यश्च ये जाता ब्राह्मणास्ते च ते श्रुताः}
{विश्वामित्रप्रभृतयः प्राप्ता ब्रह्मत्वमव्ययम्}


\twolineshloka
{आचार्यः कलशाज्जातो द्रोणः शस्त्रभृतां वरः}
{गौतमस्यान्ववाये च शरस्तम्बाच्च गौतमः}


\threelineshloka
{भवतां च यथा जन्म तदप्यागमितं मया}
{सकुण्डलं सकवचं सर्वलक्षणलक्षितम्}
{कथमादित्यसदृशं मृगी व्याघ्रं जनिष्यति}


\twolineshloka
{पृथिवीराज्यमर्होऽयं नाङ्गराज्यं नरेश्वरः}
{अनेन बाहुवीर्येण मया चाज्ञानुवर्तिना}


\twolineshloka
{यस्य वा मनुजस्येदं न क्षान्तं मद्विचेष्टितम्}
{रथमारुह्य पद्भ्यां स विनामयतु कार्मुकम्}


\twolineshloka
{ततः सर्वस्य रङ्गस्य हाहाकारो महानभूत्}
{साधुवादानुसंबद्धः सूर्यश्चास्तमुपागमत्}


\twolineshloka
{ततो दुर्योधनः कर्णमालम्ब्याग्रकरे नृपः}
{दीपिकाभिः कृतालोकस्तस्माद्रङ्गाद्विनिर्ययौ}


\twolineshloka
{पाण्डवाश्च सहद्रोणाः सकृपाश्च विशांपते}
{भीष्मेण सहिताः सर्वे ययुः स्वं स्वं निवेशनम्}


\twolineshloka
{अर्जुनेति जनः कश्चित्कश्चित्कर्णेति भारत}
{कश्चिद्दुर्योधनेत्येवं ब्रुवन्तः प्रस्थितास्तदा}


\twolineshloka
{कुन्त्याश्च प्रत्यभिज्ञाय दिव्यलक्षणसूचितम्}
{पुत्रमङ्गेश्वरं स्नेहाच्छन्ना प्रीतिरजायत}


\twolineshloka
{दुर्योधनस्यापि तदा कर्णमासाद्य पार्थिव}
{भयमर्जुनसंजातं क्षिप्रमन्तरधीयत}


\twolineshloka
{स चापि वीरः कृतशस्त्रनिश्रमःपरेण साम्नाऽभ्यवदत्सुयोधनम्}
{युधिष्ठिरस्याप्यभवत्तदा मति-र्न कर्णतुल्योऽस्ति धनुर्धरः क्षितौ}


\chapter{अध्यायः १४८}
\twolineshloka
{वैशंपायन उवाच}
{}


\twolineshloka
{पाण्डवान्धार्तराष्ट्रांश्च कृतास्त्रान्प्रसमीक्ष्य सः}
{गुर्वर्थं दक्षिणां काले प्राप्तेऽमन्यत वै गुरुः}


\twolineshloka
{`अस्त्रशिक्षामनुज्ञातान्रङ्गद्वारमुपागतान्}
{भारद्वाजस्ततस्तांस्तु सर्वानेवाभ्यभाषत}


\threelineshloka
{इच्छामि दत्तां सहितां मह्यं परमदक्षिणाम्}
{एवमुक्तास्ततः सर्वे शिष्या द्रोणमथाब्रुवन्}
{भगवन्किं प्रयच्छाम आज्ञापयतु नो गुरुः ॥'}


\twolineshloka
{ततः शिष्यान्समाहूय आचार्योऽर्थमचोदयत्}
{द्रोणः सर्वानशेषेण दक्षिणार्थं महीपते}


\twolineshloka
{पञ्चालराजं द्रुपदं गृहित्वा रणमूर्धनि}
{पर्यानयत भद्रं वः सा स्यात्परमदक्षिणा}


\twolineshloka
{तथेत्युक्त्वा तु ते सर्वे रथैस्तूर्णं प्रहारिणः}
{आचार्यधनदानार्थं द्रोणेन सहिता ययुः}


\twolineshloka
{ततोऽभिजग्मुः पञ्चालान्निघ्नन्तस्ते नरर्षभाः}
{ममृदुस्तस्य नगरं द्रुपदस्य महौजसः}


\twolineshloka
{दुर्योधनश्च कर्णश्च युयुत्सुश्च महाबलः}
{दुःशासनो विकर्णश्च जलसन्धः सुलोचनः}


\twolineshloka
{एते चान्ये च बहवः कुमारा बहुविक्रमाः}
{अहं पूर्वमहं पूर्वमित्येवं क्षत्रियर्षभाः}


\twolineshloka
{ततो वरराथारूढाः कुमाराः सादिभिः सह}
{प्रविश्य नगरं सर्वे राजमार्गमुपाययुः}


\twolineshloka
{तस्मिन्काले तु पाञ्चालः श्रुत्वा दृष्ट्वा महद्बलम्}
{भ्रातृभिः सहितो राजंस्त्वरया निर्ययौ गृहात्}


\twolineshloka
{ततस्तु कृतसन्नाहा यज्ञसेनसहोदराः}
{शरवर्षाणि मुञ्चन्तः प्रणेदुः सर्व एव ते}


\twolineshloka
{ततो रथेन शुभ्रेण समासाद्य तु कौरवान्}
{यज्ञसेनः शरान्घोरान्ववर्ष युधि दुर्जयः}


\twolineshloka
{पूर्वमेव तु संमन्त्र्य पार्थो द्रोणमथाऽब्रवीत्}
{दर्पोद्रेकात्कुमाराणामाचार्यं द्विजसत्तमम्}


\twolineshloka
{एषां पराक्रमस्यान्ते वयं कुर्याम साहसम्}
{एतैरशक्यः पाञ्चालो ग्रहीतुं रणमूर्धनि}


\twolineshloka
{एवमुक्त्वा तु कौन्तेयो भ्रातृभिः सहितोऽनघः}
{अर्धक्रोशे तु नगरादतिष्ठद्बहिरेव सः}


\twolineshloka
{द्रुपदः कौरवान्दृष्ट्वा प्राधावत समन्ततः}
{शरजालेन महता मोहयन्कौरवीं चमूम्}


\twolineshloka
{तमुद्यतं रथेनैकमाशुकारिणमाहवे}
{अनेकमिव सन्त्रासान्मेनिरे तत्र कौरवाः}


\twolineshloka
{द्रुपदस्य शरा घोरा विचेरुः सर्वतोदिशम्}
{ततः शङ्खाश्च भेर्यश्च मृदङ्गाश्च सहस्रशः}


\twolineshloka
{प्रावाद्यन्त महाराज पञ्चालानां निवेशने}
{सिंहनादश्च संजज्ञे पञ्चालानां महात्मनाम्}


\twolineshloka
{धनुर्ज्यातलशब्दश्च संस्पृश्य गगनं महान्}
{दुर्योधनो विकर्णश्च सुबाहुर्दीर्घलोचनः}


\twolineshloka
{दुःशाशनश्च संक्रुद्धः शरवर्षैरवाकिरन्}
{सोऽतिविद्धो महेष्वासः पार्षतो युधि दुर्जयः}


\twolineshloka
{व्यधमत्तान्यनीकानि तत्क्षणादेव भारत}
{दुर्योधनं विकर्णं च कर्णं चापि महाबलम्}


\twolineshloka
{नानानृपसुतान्वीरान्सैन्यानि विविधानि च}
{अलातचक्रवत्सर्वं चरन्बाणैरतर्पयत्}


\twolineshloka
{ततस्तु नागराः सर्वे मुसलैर्यष्टिभिस्तदा}
{अभ्यवर्षन्त कौरव्यान्वर्षमाणा घा इव}


\twolineshloka
{सबालवृद्धाः काम्पिल्याः कौरवानभ्ययुस्तदा}
{श्रुत्वा सुतुमुलं युद्धं कौरवानेव भारत}


\twolineshloka
{द्रवन्तिस्म नदन्तिस्म क्रोशन्तः पाण्डवान्प्रति}
{पाडवास्तु स्वनं श्रुत्वा आर्तानां रोमहर्षणम्}


\twolineshloka
{अभिवाद्य ततो द्रोणं रथानारुरुहुस्तदा}
{युधिष्ठिरं निवार्याशु मा युध्यस्वेति पाण्डवम्}


\twolineshloka
{माद्रेयौ चक्ररक्षौ तु फाल्गुनश्च तदाऽकरोत्}
{सेनाग्रगो भीमसेनस्तदाभूद्गदया सह}


\twolineshloka
{तदा शत्रुस्वनं श्रुत्वा भ्रातृभिः सहितोऽनघः}
{आयाज्जवेन कौन्तेयो रथेनानादयन्दिशः}


\twolineshloka
{पञ्चालानां ततः सेनामुद्धूतार्णवनिःस्वनाम्}
{भीमसेनो महाबाहुर्दण्डपाणिरिवान्तकः}


\twolineshloka
{प्रविवेश महासेनां मकरः सागरं यथा}
{`चतुरङ्गबलाकीर्णे ततस्तस्मिन्रणोत्सवे ॥'}


% Check verse!
स्वयमभ्यद्रवद्भीमो नागानीकं गदाधरः
\twolineshloka
{स युद्धकुशलः पार्थो बाहुवीर्येण चातुलः}
{अहनत्कुञ्जरानीकं गदया कालरूपधृक्}


\twolineshloka
{ते गजा गिरिसङ्काशाः क्षरन्तो रुधिरं बहु}
{भीमसेनस्य गदया भिन्नमस्तकपिण्डकाः}


\twolineshloka
{पतन्ति द्विरदा भूमौ वज्रघातादिवाचलाः}
{गजानश्वान्रथांश्चैव पातयामास पाण्डवः}


\twolineshloka
{पदातींश्च रथांश्चैव न्यवधीदर्जुनाग्रजः}
{गोपाल इव दण्डेन यथा पशुगणान्वने}


\twolineshloka
{चालयन्रथनागांश्च संचचाल वृकोदरः}
{भारद्वाजप्रियं कर्तुमुद्यतः फाल्गुनस्तदा}


\twolineshloka
{पार्षतं शरजालेन क्षिपन्नागात्स पाण्डवः}
{हयौघांश्च रथौघांश्च गजौघांश्च समन्ततः}


\twolineshloka
{पातयन्समरे राजन्युगान्ताग्रिरिव ज्वलन्}
{ततस्ते हन्यमाना वै पञ्चालाः सृञ्जयास्तथा}


\twolineshloka
{शरैर्नानाविधैस्तूर्णं पार्थं संछाद्य सर्वशः}
{सिंहनादं मुखैः कृत्वा समयुध्वन्त पाण्डवम्}


\twolineshloka
{तद्युद्धमभवद्धोरं समुहाद्भुतदर्शनम्}
{सिंहनादस्वनं श्रुत्वा नामृष्यत्पाकशासनिः}


\twolineshloka
{ततः किरीटी सहसा पञ्चालान्समरेऽद्रवत्}
{छादयन्निषुजालेन महता मोहयन्निव}


\twolineshloka
{शीघ्रमभ्यस्यतो बाणान्संदधानस्य चानिशम्}
{नान्तरं ददृशे किंचित्कौन्तेयस्य यशस्विनः}


\twolineshloka
{`न दिशो नान्तरिक्षं च तदा नैव च मेदिनी}
{अदृश्यत महाराज तत्र किंचिन्न सङ्गरे}


\twolineshloka
{पाञ्चालानां कुरूणां च साधुसाध्विति निस्वनः}
{तत्र तूर्यनिनादश्च शङ्खानां च महास्वनः ॥'}


\twolineshloka
{सिंहनादश्च संजज्ञे साधुशब्देन मिश्रितः}
{ततः पाञ्चालराजस्तु तथा सत्यजिता सह}


\twolineshloka
{त्वरमाणोऽभिदुद्राव महेन्द्रं शम्बरो यथा}
{महता शरवर्षेण पार्थः पाञ्चालमावृणोत्}


\twolineshloka
{ततो हलहलाशब्द आसीत्पाञ्चालके बले}
{जिवृक्षति महासिंहे गजानामिव यूथपम्}


\twolineshloka
{दृष्ट्वा पार्थं तदायान्तं सत्यजित्सत्यविक्रमः}
{पाञ्चालं वै परिप्रेप्सुर्धनञ्जयमदुद्रुवत्}


\twolineshloka
{ततस्त्वर्जुनपाञ्चालौ युद्धाय समुपागतौ}
{व्यक्षोभयेतां तौ सैन्यमिन्द्रवैरोचनाविव}


\twolineshloka
{ततः सत्यजितं पार्थो दशभिर्मर्मभेदिभिः}
{विव्याध बवलद्गाढं तदद्भुतमिवाभवत्}


\twolineshloka
{ततः शरशतैः पार्थं पाञ्चालः शीघ्रमार्दयत्}
{पार्थस्तु शरवर्षेण च्छाद्यमानो महारथः}


\twolineshloka
{वेगं चक्रे महावेगो धनुर्ज्यामवमृज्य च}
{ततः सत्यजितश्चापं छित्वा राजानमभ्ययात्}


\twolineshloka
{अथान्यद्धनुरादाय सत्यजिद्वेगवत्तरम्}
{साश्वं ससूतं सरथं पार्थं विव्याध सत्वरः}


\twolineshloka
{स तं न ममृषे पार्थः पाञ्चालेनार्दितो युधि}
{ततस्तस्य विनाशार्थं सत्वरं व्यसृजच्छरान्}


\twolineshloka
{हयान्ध्वजं धनुर्मुष्टिमुभौ तौ पार्ष्णिसारथी}
{स तथा भिद्यमानेषु कार्मुकेषु पुनः पुनः}


\twolineshloka
{हयेषु विनिकृत्तेषु विमुखोऽभवदाहवे}
{स सत्यजितमालेक्य तथा विमुखमाहवे}


\twolineshloka
{वेगेन महता राजन्नभ्यधावत पार्षतम्}
{तदा चक्रे महद्युद्धमर्जुनो जयतां वरः}


\twolineshloka
{तस्य पार्थो धनुश्छित्त्वा ध्वजं चोर्व्यामपातयत्}
{पञ्चभिस्तस्य विव्याध हयान्सूतं च सायकैः}


\twolineshloka
{तत उत्सृज्य तच्चापमाददानः शरावरम्}
{खड्गमुद्धृत्य कौन्तेयः सिंहनादमथाकरोत्}


\twolineshloka
{पाञ्चालस्य रथस्येषामाप्लुत्य सहसाऽपतत्}
{पाञ्चालरथमास्थाय अवित्रस्तो धनञ्जयः}


\twolineshloka
{विक्षोभ्याम्भोनिधिंतार्क्ष्यस्तंनागमिव सोऽग्रहीत्}
{ततस्तु सर्वपाञ्चाला विद्रवन्ति दिशो दश}


\twolineshloka
{दर्शयन्सर्वसैन्यानां स बाह्वोर्बलमात्मनः}
{सिंहनादस्वनं कृत्वा निर्जगाम धनञ्जयः}


\threelineshloka
{आयान्तमर्जुनं दृष्ट्वा कुमाराः सहितास्तदा}
{ममृदुस्तस्य नगरं द्रुपदस्य महात्मनः ॥अर्जुन उवाच}
{}


\threelineshloka
{संबन्धी कुरुवीराणां द्रुपदो राजसत्तमः}
{मा वधीस्तद्बलं भीम गुरुदानं प्रदीयताम् ॥वैशंपायन उवाच}
{}


\twolineshloka
{भीमसेनस्तदा राजन्नर्जुनेन निवारितः}
{अतृप्तो युद्धधर्मेषु न्यवर्तत महाबलः}


\twolineshloka
{ते यज्ञसेनं द्रुपदं गृहीत्वा रणमूर्धनि}
{उपाजग्मुः सहामात्यं द्रोणाय भरतर्षभ}


\twolineshloka
{भग्नदर्पं हृतधनं तं तथा वशमागतम्}
{स वैरं मनसा ध्यात्वा द्रोणो द्रुपदमब्रवीत्}


\twolineshloka
{विमृज्य तरसा राष्ट्रं पुरं ते मृदितं मया}
{प्राप्य जीवन्रिपुवशं सखिपूर्वं किमिष्यते}


\twolineshloka
{एवमुक्त्वा प्रहस्यैनं किंचित्स पुनरब्रवीत्}
{मा भैः प्राणभयाद्वीर क्षमिणो ब्राह्मणा वयम्}


\twolineshloka
{आश्रमे क्रीडितं यत्तु त्वया बाल्ये मया सह}
{तेन संवर्धितः स्नेहः प्रीतिश्च क्षत्रियर्षभ}


\twolineshloka
{प्रार्थयेयं त्वया सख्यं पुनरेव जनाधिप}
{वरं ददामि ते राजन्राज्यस्यार्धमवाप्नुहि}


\twolineshloka
{अराजा किल नो राज्ञः सखा भवितुमर्हति}
{अतः प्रयतितं राज्ये यज्ञसेन मया तव}


\threelineshloka
{राजासि दक्षिणे कूले भागीरथ्याहमुत्तरे}
{सखायं मां विजानीहि पाञ्चाल यदि मन्यसे ॥द्रुपद उवाच}
{}


\threelineshloka
{अनाश्चर्यमिदं ब्रह्मन्विक्रान्तेषु महात्मसु}
{प्रीये त्वयाऽहं त्वत्तश्च प्रीतिमिच्छामि शाश्वतीम् ॥वैशंपायन उवाच}
{}


\twolineshloka
{एवमुक्तः स तं द्रोणो मोक्षयामास भारत}
{सत्कृत्य चैनं प्रीतात्मा राज्यार्धं प्रत्यपादयत्}


\twolineshloka
{माकन्दीमथ गङ्गायास्तीरे जनपदायुताम्}
{सोऽध्यावसद्दीनमनाः काम्पिल्यं च पुरोत्तमम्}


\twolineshloka
{दक्षिणांश्चापि पञ्चालान्यावच्चर्मण्वती नदी}
{द्रोणेन चैवं द्रुपदः परिभूयाथ पालितः}


\twolineshloka
{क्षात्रेण च बलेनास्य नापश्यत्स पराजयम्}
{हीनं विदित्वा चात्मानं ब्राह्मेण स बलेनतु}


\twolineshloka
{पुत्रजन्म परीप्सन्वै पृथिवीमन्वसंचरत्}
{अहिच्छत्रं च विषयं द्रोणः समभिपद्यत}


\twolineshloka
{एवं राजन्नहिच्छत्रा पुरीजनपदायुता}
{युधि निर्जित्य पार्थेन द्रोणाय प्रतिपादिता}


\chapter{अध्यायः १४९}
\twolineshloka
{`वैशंपायन उवाच}
{}


\twolineshloka
{द्रोणेन वैरं द्रुपदो न सुष्वाप स्मरंस्तदा}
{क्षात्रेण वै बलेनास्य नाऽशशंसे पराजयम्}


\twolineshloka
{हीनं विदित्वा चात्मानं ब्राह्मेणापि बलेन च}
{द्रुपदोऽमर्षणाद्राजा कर्मसिद्धान्द्विजोत्तमान्}


\twolineshloka
{अन्विच्छन्परिचक्राम ब्राह्मणावसथान्बहून्}
{नास्ति श्रेष्ठं ममापत्यं धिग्बन्धूनिति च ब्रुवन्}


\twolineshloka
{निश्वासपरमो ह्यासीद्द्रोणं प्रतिचिकीर्षया}
{न सन्ति मम मित्राणि लोकेऽस्मिन्नास्ति वीर्यवान्}


\twolineshloka
{पुत्रजन्म परीप्सन्वै पृथिवीमन्वयादिमाम्}
{प्रभावशिक्षाविनयाद्द्रोणस्यास्त्रबलेन च}


\twolineshloka
{कर्तुं प्रयतमानो वै न शशाक पराजयम्}
{अभितः सोऽथ कल्माषीं गङ्गातीरे परिभ्रमन्}


\twolineshloka
{ब्राह्मणावसथं पुण्यमाससाद महीपतिः}
{तत्र नास्नातकः कश्चिन्न चासीदव्रतो द्विजः}


\twolineshloka
{तथैव तौ महाभागौ सोऽपश्यच्छंसितव्रतौ}
{याजोपयाजौ ब्रह्मर्षी भ्रातरौ पृषतात्मजः}


\twolineshloka
{संहिताध्ययने युक्तौ गोत्रतश्चापि काश्यपौ}
{अरण्ये युक्तरूपौ तौ ब्राह्मणावृषिसत्तमौ}


\twolineshloka
{स उपामन्त्रयामास सर्वकामैरतन्द्रितः}
{बुद्ध्वा तयोर्बलं बुद्धिं कनीयांसमुपह्वरे}


\twolineshloka
{प्रपेदे छन्दयन्कामैरुपयाजं धृतव्रतम्}
{गुरुशुश्रूषणे युक्तः प्रियकृत्सर्वकामदम्}


\twolineshloka
{पाद्येनासनदानेन तथाऽर्घ्येण फलैश्च तम्}
{अर्हयित्वा यथान्यायमुपयाजोऽब्रवीत्ततः}


\threelineshloka
{येन कार्यविशेषेण त्वमस्मानभिकाङ्क्षसे}
{कृतश्चायं समुद्योगस्तद्ब्रवीतु भवानिति ॥वैशंपायन उवाच}
{}


\twolineshloka
{स बुद्ध्वा प्रीतिसंयुक्तमृषीणामुत्तमं तदा}
{उवाच छन्दयन्कामैर्द्रुपदः स तपस्विनम्}


\twolineshloka
{येन मे कर्मणा ब्रह्मन्पुत्रः स्याद्द्रोणमृत्येव}
{उपयाज चरस्वैतत्प्रदास्यामि धनं तव}


\fourlineindentedshloka
{उपयाज उवाच}
{नाहं फलार्थी द्रुपद योऽर्थी स्यात्तत्र गम्यताम्}
{वैशंपायन उवाच}
{प्रत्याख्यातस्तु तेनैवं स वै सज्जनसंनिधौ}


\twolineshloka
{आराधयिष्यन्द्रुपदः स तं पर्यचरत्तदा}
{ततः संवत्सरस्यान्ते द्रुपदं द्विजसत्तमः}


\twolineshloka
{उपयाजोऽब्रवीद्वाक्यं काले मधुरया गिरा}
{ज्येष्ठो भ्राता न मेऽत्याक्षीद्विचरन्विजने वने}


\twolineshloka
{अपरिज्ञातशौचायां भूमौ निपतितं फलम्}
{तदपश्यमहं भ्रातुरसांप्रतमनुव्रजन्}


\twolineshloka
{विमर्शं हि फलादाने नायं कुर्यात्कथंचन}
{नापश्यत्फलं दृष्ट्वा दोषांस्तस्याऽऽनुबन्धिकान्}


\twolineshloka
{विविनक्ति न शौचार्थी सोऽन्यत्रापि कथं भवेत्}
{संहिताध्ययनस्यान्ते पञ्चयज्ञान्निरूप्य च}


\twolineshloka
{भैक्षमुञ्छेन सहितं भुञ्जानस्तु तदा तदा}
{कीर्तयत्येव राजर्षे भोजनस्य रसं पुनः}


\twolineshloka
{संहिताध्ययनं कुर्वन्वने गुरुकुले वसन्}
{भैक्षमुच्छिष्टमन्येषां भुङ्क्ते स्म सततं तथा}


\twolineshloka
{कीर्तयन्गुणमन्नानामथ प्रीतो मुहुर्मुहुः}
{एवं फलार्थिनस्त्समान्मन्येऽहं तर्कचक्षुषा}


\twolineshloka
{तं वै गच्छेह नृपते त्वां स संयाजयिष्यति ॥वैशंपायन उवाच}
{}


\twolineshloka
{उपयाजवचः श्रुत्वा याजस्याश्रममभ्यगात्}
{जुगुप्समानो नृपतिर्मनसेदं विचिन्तयन्}


\twolineshloka
{भृशं संपूज्य पूजार्हमृषिं याजमुवाच ह}
{गोशतानि ददान्यष्टौ याज याजय मां विभो}


\twolineshloka
{द्रोणवैरान्तरे तप्तं विषण्णं शरणागतम्}
{ब्रह्मबन्धुप्रणिहितं न क्षत्रं क्षत्रियो जयेत्}


\twolineshloka
{तस्माद्द्रोणभयार्तं मां भवांस्त्रातुमिहार्हति}
{येन मे कर्मणा ब्रह्मन्पुत्रः स्याद्द्रोणमृत्यवे}


\twolineshloka
{अर्जुनस्यापि वै भार्या भवेद्या वरवर्णिनी}
{स हि ब्रह्मविदां श्रेष्ठो ब्राह्मे क्षात्रेऽप्यनुत्तमः}


\twolineshloka
{ततो द्रोणस्तु माऽजैषीत्सखिविग्रहकारणात्}
{क्षत्रियो नास्ति तुल्योऽस्य पृथिव्यां कश्चिदग्रणीः}


\twolineshloka
{भारताचार्यमुख्यस्य भारद्वाजस्य धीमतः}
{द्रोणस्य शरजालानि रिपुदेहहराणि च}


\twolineshloka
{षडरत्नि धनुश्चास्य खड्गमप्रतिम तथा}
{स हि ब्राह्मणवेषेण क्षात्रं वेगमसंशयम्}


\twolineshloka
{प्रतिहन्ति महेष्वासो भारद्वाजो महामनाः}
{कार्तवीर्यसमो ह्येष खट्वाङ्गप्रतिमो रणे}


% Check verse!
क्षत्रोच्छेदाय विहितो जामदग्न्य इवास्थितः
\twolineshloka
{सहितं क्षत्रवेगेन ब्रह्मवेगेन सांप्रतम्}
{उपपन्नं हि मन्येऽहं भारद्वाजं यशस्विनम्}


\twolineshloka
{नेषवस्तमपाकुर्युर्न च प्रासा न चासयः}
{ब्राह्मं तस्य महातेजो मन्त्राहुतिहुतं यथा}


\twolineshloka
{तस्य ह्यस्त्रबलं घोरमप्रसह्यं परैर्भुवि}
{शत्रून्समेत्य जयति क्षत्रं ब्रह्मपुरस्कृतम्}


\twolineshloka
{ब्रह्मक्षत्रे च सहिते ब्रह्मतेजो विशिष्यते}
{सोऽहं क्षत्रबलाद्दीनो ब्रह्मतेजः प्रपेदिवान्}


\twolineshloka
{द्रोणाद्विशिष्टमासाद्य भवन्तं ब्रह्मवित्तमम्}
{द्रोणान्तकमहं पुत्रं लभेयं युधि दुर्जयम्}


\threelineshloka
{द्रोणमृत्युर्यथा मेऽद्य पुत्रो जायेत वीर्यवान्}
{तत्कर्म कुरु मे याज निर्वपाम्यर्बुद्धं गवाम् ॥वैशंपायन उवाच}
{}


\twolineshloka
{तथेत्युक्त्वा तुं तं याजो यज्ञार्थमुपकल्पयन्}
{गुर्वर्थ इति चाकाममुपयाजमचोदयत्}


\twolineshloka
{द्रुपदं च महाराजमिदं वचनमब्रवीत्}
{मा भैस्त्वं संप्रदास्यामि कर्मणा भवतः सुतम्}


\threelineshloka
{क्षिप्रमुत्तिष्ठ चाव्यग्रः संभारानुपकल्पय}
{वैशंपायन उवाच}
{एवमुक्त्वा प्रतिज्ञाय कर्म चास्याददे मुनिः}


\twolineshloka
{ब्राह्मणो द्विपदां श्रेष्ठो यथाविधि कथाक्रमम्}
{याजो द्रोणविनाशाय याजयामास तं नृपम्}


\twolineshloka
{गुर्वर्थेऽयोजयत्कर्म याजस्यापि समीपतः}
{ततस्तस्य नरेन्द्रस्य उपयाजो महातपाः}


\fourlineindentedshloka
{आचव्यौ कर्म वैतानं तथा पुत्रफलाय वै}
{इह पुत्रो महावीर्यो महातेजा महाबलः}
{इष्यते यद्विधो राजन्भविता स तथाविधः ॥वैशंपायन उवाच}
{}


\twolineshloka
{भारद्वाजस्य हन्तारं सोऽभिसन्धाय पार्थिवः}
{आजहेऽथ तदा राजन्द्रुपदः कर्म सिद्धये}


\twolineshloka
{ब्राह्मणो द्विपदां श्रेष्ठो जुहाव च यथाविधि}
{कौसवी नाम तस्यासीद्या वै तां पुत्रगृद्धिनः}


\twolineshloka
{सौत्रामणिं तथा पत्नीं ततः कालेऽभ्ययात्तदा}
{याजस्तु सवनस्यान्ते देवीमाह्वापयत्तदा}


\threelineshloka
{प्रेहि मां राज्ञि पृषति मिथुनं त्वामुपस्थितम्}
{कुमारश्च कुमारी च पितृवंशविवृद्धये ॥पृषत्युवाच}
{}


\threelineshloka
{नालिप्तं वै मम मुखं पुण्यान्गन्धान्बिभर्मि च}
{न पत्नी तेऽस्मि सूत्यर्थे तिष्ठ याज मम प्रिये ॥याज उवाच}
{}


\threelineshloka
{याजेन श्रपितं हव्यमुपयाजेन मन्त्रितम्}
{कथं कामं न संदध्यात्पृषति प्रेहि तिष्ठ वा ॥वैशंपायन उवाच}
{}


\twolineshloka
{एवमुक्त्वा तु याजेन हुते हविषि संस्कृते}
{उत्तस्थौ पावकात्तस्मात्कुमारो देवसंनिभः}


\twolineshloka
{ज्वालावर्णो घोररूपः किरीटी वर्म धारयन्}
{वीरः सखङ्गः सशरो धनुष्मान्स नदन्मुहुः}


\twolineshloka
{सोऽभ्यरोहद्रथवरं तेन च प्रययौ तदा}
{जातमात्रे कुमारे च वाक्किलान्तर्हिताब्रवीत्}


\twolineshloka
{एष शिष्यश्च मृत्युश्च भारद्वाजस्य जायते}
{भयापहो राजपुत्रः पाञ्चालानां यशस्करः}


\twolineshloka
{राज्ञः शोकापहो जात एष द्रोणवधाय हि}
{इत्यवोचन्महद्भूतमदृश्यं खेचरं तदा}


\twolineshloka
{ततः प्रणेदुः पाञ्चालाः प्रहृष्टाः साधुसाध्विति}
{द्वितीयायां च होत्रायां हुते हविषि मन्त्रिते}


\twolineshloka
{कुमारी चापि पाञ्चाली वेदिमध्यात्समुत्थिता}
{प्रत्याख्याते पृषत्या च याजके भरतर्षभ}


\twolineshloka
{पुनः कुमारी पाञ्चाली सुभगा वेदिमध्यगा}
{अन्तर्वेद्यां समुद्भूता कन्या सा सुमनोहरा}


\twolineshloka
{श्यामा पद्मपलाशाक्षी नीलकुञ्चितमूर्धजा}
{मानुषं विग्रहं कृत्वा साक्षाच्छ्रीरिव वर्णिनी}


\twolineshloka
{ताम्रतुङ्गनखी सुभ्रूश्चारुपीनपयोधरा}
{नीलोत्पलसमो गन्धो यस्याः क्रोशात्प्रधावति}


\twolineshloka
{या बिभर्ति परं रूपं यस्या नास्त्युपमा भुवि}
{देवदानवयक्षाणामीप्सिता वरवर्णिनी}


\twolineshloka
{तां चापि जातां सुश्रोणीं वागुवाचाशरीरिणी}
{सर्वयोषिद्वरा कृष्णा क्षयं क्षत्रस्य नेष्यति}


\twolineshloka
{सुरकार्यमियं काले करिष्यति सुमध्यमा}
{अस्या हेतोः क्षत्रियाणां महदुत्पत्स्यते भयम्}


\twolineshloka
{तच्छ्रुत्वा सर्वपाञ्चालाः प्रणेदुः सिंहसङ्घवत्}
{न चैनान्हर्षसंपन्नानियं सेहे वसुन्धरा}


\twolineshloka
{तथा तु मिथुनं जज्ञे द्रुपदस्य महात्मनः}
{कुमारश्च कुमारी च मनोज्ञौ तौ नरर्षभौ}


\twolineshloka
{श्रिया परमया युक्तौ क्षात्रेण वपुषा तथा}
{तौ दृष्ट्वा पृषती याजं प्रपेदे सा सुतार्थिनी}


\twolineshloka
{न वै मदन्यां जननीं जानीयातामिमाविति}
{तथेत्युवाच तां याजो राज्ञः प्रियचिकीर्षया}


\twolineshloka
{तयोस्तु नामनी चक्रुर्द्विजाः संपूर्णमानसाः}
{धृष्टत्वादप्रधृष्यत्वात् द्युम्नाद्युत्संभवादपि}


\twolineshloka
{धृष्टद्युम्नः कुमारोऽयं द्रुपद्सय भवत्विति}
{कृष्णेत्येवाभवत्कन्या कृष्णा भूत्सा हि वर्णतः}


\twolineshloka
{तथा तन्मिथुनं जज्ञे द्रुपदस्य महामखे}
{वैदिकाध्ययने पारं धृष्टद्युम्नो गतस्तदा}


\twolineshloka
{धृष्टद्युम्नं तु पाञ्चाल्यमानीय स्वं निवेशनम्}
{उपाकरोदस्त्रहेतोर्भारद्वाजः प्रतापवान्}


\twolineshloka
{अमोक्षणीयं दैवं हि भावि मत्वा महामतिः}
{तथा तत्कृतवान्द्रोण आत्मकीर्त्यनुरक्षणात्}


% Check verse!
सर्वास्त्राणि स तु क्षिप्रमाप्तवान्परया धिया
\chapter{अध्यायः १५०}
\twolineshloka
{जनमेजय उवाच}
{}


\twolineshloka
{द्रुपदस्यापि विप्रर्षे श्रोतुमिच्छामि संभवम्}
{कथं चापि समुत्पन्नः कथमस्त्राण्यवाप्तवान् ॥'}


\threelineshloka
{एतदिच्छामि भगवंस्त्वत्तः श्रोतुं द्विजोत्तम}
{कौतूहलं जन्मसु मे कीर्त्यमानेष्वनेकशः ॥वैशंपायन उवाच}
{}


\twolineshloka
{राजा बभूव पाञ्चालः पुत्रार्थी पुत्रकारणात्}
{वनं गतो महाराजस्तपस्तेपे सुदारुणम्}


\twolineshloka
{आराधयन्प्रयत्नेन महर्षीन्संशितव्रतान्}
{तस्य संतप्यमानस्य वने मृगगणायुते}


\twolineshloka
{कालस्तु सुमहान्राजन्नत्ययात्सुतकारणात्}
{स तु राजा महातेजास्तपस्तीव्रं समाददे}


\twolineshloka
{कंचित्कालं वायुभक्षो निराहारस्तथैव च}
{तथैव तु महाबाहोर्वर्तमानस्य भारत}


\twolineshloka
{कालस्तस्य महाराज यातो वै नृपसत्तम}
{ततो नातिचिरात्काले वसन्ते कामदीपने}


\twolineshloka
{फुल्लाशोकवने चैव प्राणिनां सुमनोहरे}
{नद्यास्तीरं ततो गत्वा गङ्गायाः पद्मलोचनः}


\twolineshloka
{नियमस्थश्च राजासीत्तदा भरतसत्तम}
{ततो नातिचिरात्काले वनं तन्मनुजेश्वर}


\twolineshloka
{संप्राप्ता ह्यप्सरा राजन्मेनकेत्यभिविश्रुता}
{पुष्पद्रुमान्सज्जमाना राज्ञो दर्शनमागमत्}


\twolineshloka
{न ददर्श तु सा राजंस्तत्र स्थानगतं नृपम्}
{दृष्ट्वा चाप्सरसं तां तु शुक्रं राज्ञोऽपतद्भुवि}


\twolineshloka
{ततः स राजा राजेन्द्र लज्जया नृपतिः स्वयम्}
{पद्भ्यामाक्रमतायुष्मंस्ततस्तु द्रुपदोऽभवत्}


\twolineshloka
{ततस्तु तपसा तस्य राजर्षेर्भावितात्मनः}
{पुत्रः समभवच्छीघ्रं पदोस्तस्य क्रमेण तु}


\twolineshloka
{तेनास्य ऋषयः सर्वे समागम्य तपोधनाः}
{नाम चुक्रुर्हि विद्वांसो द्रुपदोऽस्त्विति भारत}


\twolineshloka
{स तस्यैवाश्रमे राजन्भरद्वाजस्य भारत}
{ववृधे सुमुखं तत्र कामैः सर्वैर्नृपोत्तम}


\twolineshloka
{पाञ्चालोऽपि हि राजेन्द्र स्वराज्यं गतवान्प्रभुः}
{भरद्वाजस्य विद्यार्थं सुतं दत्वा महात्मनः}


\twolineshloka
{स कुमारस्ततो राजन्द्रोणेन सहितो वने}
{वेदांश्चाधिजगे साङ्गान्धनुर्वेदांश्च भारत}


\twolineshloka
{परया स मुदा युक्तो विचचार वने सुखम्}
{तस्यैवं वर्तमानस्य वने वनचरैः सह}


\twolineshloka
{कालेनातिचिराद्राजन्पिता स्वर्गमुपेयिवान्}
{स समागम्य पाञ्चालैः पाञ्चालेष्वभिषेचितः}


\twolineshloka
{प्राप्तश्च राज्यं राजेन्द्र सुहृदां प्रीतिवर्धनः}
{राज्यं ररक्ष धर्मेण यथा चेन्द्रस्त्रिविष्टपम्}


\twolineshloka
{एतन्मया ते राजेन्द्र यथावत्परिकीर्तितम्}
{द्रुपदस्य च राजर्षेर्धृष्टद्युम्नस्य जन्म च}


\chapter{अध्यायः १५१}
\twolineshloka
{वैशंपायन उवाच}
{}


\twolineshloka
{`धृतराष्ट्रस्तु राजन्द्रे यदा कौरवनन्दनम्}
{युधिष्ठिरं विजानन्वै समर्थं राज्यरक्षणे}


\twolineshloka
{यौवराज्याभिषेकार्थममन्त्रयत मन्त्रिभिः}
{ते तु बुद्ध्वान्वतप्यन्त धृतराष्ट्रात्मजास्तदा}


\twolineshloka
{ततः संवत्सरस्यान्ते यौवराज्याय पार्थिव}
{स्थापितो धृतराष्ट्रेण पाण्डुपुत्रो युधिष्ठिरः}


\twolineshloka
{ततोऽदीर्घेण कालेन कुन्तीपुत्रो युधिष्ठिरः}
{पितुरन्तर्दधे कीर्तिं शीलवृत्तसमाधिभिः}


\twolineshloka
{असियुद्धे गदायुद्धे रथयुद्धे च पाण्डवः}
{संकर्षणादशिक्षद्वै शश्वच्छिक्षां वृकोदरः}


\twolineshloka
{समाप्तशिक्षो भीमस्तु द्युमत्सेनसमो बले}
{पराक्रमेण संपन्नो भ्रातॄणामचरद्वशे}


\twolineshloka
{प्रगाढदृढमुष्टित्वे लाघवे वेधने तथा}
{क्षुरनाराचभल्लानां विपाठानां च तत्त्ववित्}


\twolineshloka
{ऋजुवक्रविशालानां प्रयोक्ता फाल्गुनोऽभवत्}
{लाघवे सौष्ठवे चैव नान्यः कश्चन विद्यते}


\twolineshloka
{बीभत्सुसदृशो लोक इति द्रोणो व्यवस्थितः}
{ततोऽब्रवीद्गुडाकेशं द्रोणः कौरवसंसदि}


\twolineshloka
{अगस्त्यस्य धनुर्वेदे शिष्यो मम गुरुः पुरा}
{अग्निवेश्य इति ख्यातस्तस्य शिष्योऽस्मि भारत}


\twolineshloka
{तीर्थात्तीर्थं गमयितुमहमेतत्समुद्यतः}
{तपसा यन्मया प्राप्तममोघमशनिप्रभम्}


\twolineshloka
{अस्त्रं ब्रह्मशिरो नाम यद्दहेत्पृथिवीमपि}
{ददता गुरुणा चोक्तं न मनुष्येष्विदंत्वया}


\twolineshloka
{भारद्वाज विमोक्तव्यमल्पवीर्येषु संयुगे}
{यद्यदन्तर्हितं भूतं किंचिद्युद्ध्येत्त्वया सह}


\twolineshloka
{महातेजस्त्वमेतेन हन्याः शस्त्रेण संयुगे}
{त्वया प्राप्तमिदं वीर दिव्यं नान्योऽर्हति त्विदम्}


\twolineshloka
{समयस्तु त्वया रक्ष्यो मुनिसृष्टो विशांपते}
{आचार्यदक्षिणां देहि ज्ञातिग्रामस्य पश्यतः}


\twolineshloka
{ददानीति प्रतिज्ञाते फाल्गुनेनाब्रवीद्गुरुः}
{युद्धेऽहं प्रतियोद्धव्यो युध्यमानस्त्वयाऽनघ}


\twolineshloka
{तथेति च प्रतिज्ञाय द्रोणाय कुरुपुङ्गवः}
{उपसंगृह्य चरणावुपतस्थे विनीतवत्}


\twolineshloka
{द्रोणो जगाद वचनं समालिङ्ग्य तु फाल्गुनम्}
{यन्मयोक्तं पुरा पार्थ तव लोके नरं क्वचित्}


\twolineshloka
{सदृशं कारये नैव सर्वप्रहरणे युधि}
{तत्कृतं च मया सम्यक्तव तुल्यो न वर्तते}


\twolineshloka
{देवा युधि न शक्तास्त्वां योद्धुं दैत्या न दानवाः}
{नाहं त्वत्तो विशिष्टोऽस्मि किं पुनर्मानवा रणे}


\twolineshloka
{एकस्तवाधिको लोके यो हि वृष्णिकुलोद्भवः}
{कृष्णः कमलपत्राक्षः कंसकालियसूदनः}


\twolineshloka
{स जेता सर्वलोकानां सर्वप्रहरणायुधः}
{नैतावता ते पार्थाहं भवाम्यनृतवागिह}


\twolineshloka
{तदधीनं जगत्सर्वं तत्प्रलीनं तदुद्भवम्}
{तत्पदं न विजानन्ति ब्रह्मेशानादयोऽपि वा}


\twolineshloka
{तन्नाभिप्रभवो ब्रह्मा सर्वभूतानि निर्ममे}
{स एव कर्ता भोक्ता च संहर्ता च जगन्मयम्}


\twolineshloka
{स एव भूतं भव्यं च भवच्च पुरुषः परः}
{नित्यः सर्वगतः स्थाणुरचलोऽयं सनातनः}


\twolineshloka
{प्रादुर्भवति योगात्मा पालनार्थं स लीलया}
{तत्तुल्यो हि न जायेत न जातो न जनिष्यते}


\twolineshloka
{स हि मातुलजोऽभूत्ते चराचरगुरुः पिता}
{को हि तं जेतुमीहेत जानन्नात्महितं नरः}


\twolineshloka
{श्यालश्च ते सखा चासौ तस्य त्वं प्राणवल्लभः}
{स्नेहमभ्यधिकं तस्य तव सख्यमवस्थितम्}


\twolineshloka
{न तेन भवतो युद्धं भविता नर्मतोऽपि वा}
{अपिचार्थे तव पुरा शक्रेण किल चोदितः}


\twolineshloka
{गोकुले वर्धमानस्तु नन्दगोपस्य कारणात्}
{ममांशः पाण्डवो लोके पृथिव्यां पुरुषोत्तमः}


\twolineshloka
{कौन्तेयावरजः श्रीमानर्जुनो नाम वीर्यवान्}
{भुवो भारापहरणे साहाय्यं ते करिष्यति}


\twolineshloka
{तदर्थमभयं देहि पाहि चास्मत्कृते प्रभो}
{इत्युक्तः पुण्डरीकाक्षस्तदा शक्रेण फल्गुन}


\twolineshloka
{तमुवाच ततः श्रीमाञ्शङ्खचक्रगदाधरः}
{जानामि पाण्डवे वंशे जातं पार्थं पितृष्वसुः}


\twolineshloka
{पुत्रं परमधर्मिष्ठं सर्वशस्त्रभृतां वरम्}
{पालयामि त्वदंशं तं सर्वलोकमहाभुजम्}


\twolineshloka
{आवयोः सख्यसदृशं न च लोके भविष्यति}
{यस्तद्भक्तः समद्भक्तो यस्तं द्वेष्टि स मामपि}


\twolineshloka
{यन्मे वित्तं तु तत्तस्य तं विनाहं न जीवये}
{इति पार्थ पुरा शक्रमाह सर्वेश्वरो हरिः}


\twolineshloka
{तस्मात्तवापि सदृशस्तं विनाभ्यधिकः पुमान्}
{न चेह भविता लोके तमेव शरणं व्रज}


\twolineshloka
{शरण्यः सर्वभूतानां देवदेवो जनार्दनः ॥' वैशंपायन उवाच}
{}


\twolineshloka
{तथेति च प्रतिज्ञाय द्रोणाय कुरुपुङ्गवः}
{उपसंगृह्य चरणौ युधिष्ठिरवशोऽभवत्}


\twolineshloka
{स्वभावादगमच्छब्दो महीं सागरमेखलाम्}
{अर्जुनस्य समो लोके नास्ति कश्चिद्धनुर्धरः}


\twolineshloka
{गदायुद्धेऽसियुद्धे च रथयुद्धे च पाण्डवः}
{पारगश्च धनुर्युद्धे बभूवाथ धनञ्जयः}


\threelineshloka
{नीतिमान्सकलां नीतिं विबुधाधिपतेस्तदा}
{`अस्त्रे शस्त्रे च शास्त्रे च रथनागाश्वकर्मणि}
{'अवाप्य सहदेवोऽपि भ्रातॄणां ववृते वशे}


\twolineshloka
{द्रोणेनैवं विनीतश्च भ्रातॄणां नकुलः प्रियः}
{चित्रयोधी समाख्यातो बभूवातिरथोदितः}


\twolineshloka
{त्रिवर्षकृतयज्ञस्तु गन्धर्वाणामुपप्लवे}
{अर्जुनप्रमुखैः पार्थैः सौवीरः समरे हतः}


\twolineshloka
{न शशाक वशे कर्तुं यं पाण्डुरपि वीर्यवान्}
{सोऽर्जुनेन वशं नीतो राजासीद्यवनाधिपः}


\twolineshloka
{अतीव बलसंपन्नः सदा मानि कुरून्प्रति}
{विपुलो नाम सौवीरः शस्तः पार्थेन धीमता}


\twolineshloka
{दत्तामित्र इति ख्यातं सङ्ग्रामे कृतनिश्चयम्}
{सुमित्रं नाम सौवीरमर्जुनोऽदमयच्छरैः}


\twolineshloka
{भीमसेनसहायश्च रथानामयुतं च सः}
{अर्जुनः समरे प्राच्यान्सर्वानेकरथोऽजयत्}


\twolineshloka
{तथैवैकरथो गत्वा दक्षिणामजयद्दिशम्}
{धनौघं प्रापयामास कुरुराष्ट्रं धनञ्जयः}


\twolineshloka
{`यतः पञ्चदशे वर्षे सर्वमेतच्चकार सः}
{तं दृष्ट्वा धार्तराष्ट्राणां ततो भयमजायत}


\twolineshloka
{यः सर्वान्धृतराष्ट्रस्य पुत्रान्विप्रचकार ह}
{भीमसेनो महाबाहुर्बलाद्बलवतां वरः}


\twolineshloka
{अदुष्टभावं तं दोषैर्जगृहुर्दोषबुद्धयः}
{धार्तराष्ट्रास्तथा सर्वे भयाद्भीमस्य कर्मणा}


\twolineshloka
{तं दृष्ट्वा कर्मभिः पार्थान्सर्वानागतलक्षणान्}
{बलाद्बहुगुणांस्तेभ्यो बिभियुर्दोषबुद्धयः ॥'}


\twolineshloka
{एवं सर्वे महात्मानः पाण्डवा मनुजोत्तमाः}
{परराष्ट्राणि निर्जित्य स्वराष्ट्रं ववृधुः पुरा}


\threelineshloka
{ततो बलमतिख्यातं विज्ञाय दृढधन्विनाम्}
{दूषितः सहसा भावो धृतराष्ट्रस्य पाण्डुषु}
{स चिन्तापरमो राजा न निद्रामलभन्निशि}


\chapter{अध्यायः १५२}
\twolineshloka
{वैशंपायन उवाच}
{}


\twolineshloka
{प्राणाधिकं भीमसेनं कृतविद्यं धनञ्जयम्}
{दुर्योधनो लक्षयित्वा पर्यतप्यत दुर्मनाः}


\twolineshloka
{ततो वैकर्तनः कर्णः शकुनिश्चापि सौबलः}
{अनेकारभ्युपायैस्ते जिघांसन्ति स्म पाण्डवान्}


\twolineshloka
{पाण्डवा अपि तत्सर्वं प्रतिचक्रुर्यथाबलम्}
{उद्भावनमकुर्वन्तो विदुरस्य मते स्थिताः}


\twolineshloka
{गुणैः समुदितान्दृष्ट्वा पौराः पाण्डुसुतांस्तदा}
{कथयाञ्चक्रिरे तेषां गुणान्संसत्सु भारत}


\twolineshloka
{राज्यप्राप्तिं च संप्राप्तं ज्येष्ठं पाण्डुसुतं तदा}
{कथयन्ति स्म संभूय चत्वरेषु सभासु च}


\twolineshloka
{प्रज्ञाश्चक्षुरचक्षुष्ट्वाद्धृतराष्ट्रो जनेश्वरः}
{राज्यं न प्राप्तवान्पूर्वं स कथं नृपतिर्भवेत्}


\twolineshloka
{तथा शान्तनवो भीष्मः सत्यसन्धो महाव्रतः}
{प्रत्याख्याय पुरा राज्यं न स जातु ग्रहीष्यति}


\twolineshloka
{ते वयं पाण्डवज्येष्ठं तरुणं वृद्धशीलिनम्}
{अभिषिञ्चाम साध्वद्य सत्यकारुण्यवेदिनम्}


\threelineshloka
{स हि भीष्मं शान्तनवं धृतराष्ट्रं च धर्मवित्}
{सपुत्रं विविधैर्भोगैर्योजयिष्यति पूजयन् ॥वैशंपायन उवाच}
{}


\twolineshloka
{तेषां दुर्योधनः श्रुत्वा तानि वाक्यानि जल्पताम्}
{युधिष्ठिरानुरक्तानां पर्यतप्यत दुर्मतिः}


\twolineshloka
{स तप्यमानो दुष्टात्मा तेषां वाचो न चक्षमे}
{ईर्ष्यया चापि संतप्तो धृतराष्ट्रमुपागमत्}


\twolineshloka
{ततो विरहितं दृष्ट्वा पितरं प्रतिपूज्य सः}
{पौरानुरागसंतप्तः पश्चादिदमभाषत}


\twolineshloka
{श्रुता मे जल्पतां तात पौराणामशिवा गिरः}
{त्वामनादृत्य भीष्मं च पतिमिच्छन्ति पाण्डवम्}


\twolineshloka
{मतमेतच्च भीष्मस्य न स राज्यं बुभुक्षति}
{अस्माकं तु परां पीडां चिकीर्षन्ति पुरे जनाः}


\twolineshloka
{पितृतः प्राप्तवान्राज्यं पाण्डुरात्मगुणैः पुरा}
{त्वमन्धगुणसंयोगात्प्राप्तं राज्यं न लब्धवान्}


\twolineshloka
{स एष पाण्डोर्दायाद्यं यदि प्राप्नोति पाण्डवः}
{तस्य पुत्रो ध्रुवं प्राप्तस्तस्य तस्यापि चापरः}


\twolineshloka
{ते वयं राजवंशेन हीनाः सह सुतैरपि}
{अवज्ञाता भविष्यामो लोकस्य जगतीपते}


\twolineshloka
{सततं निरयं प्राप्ताः परपिण्डोपजीविनः}
{न भवेम यथा राजंस्तथा नीतिर्विधीयताम्}


\threelineshloka
{यदि त्वं हि पुरा राजन्निदं राज्यमवाप्तवान्}
{ध्रुवं प्राप्स्याम च वयं राज्यमप्यवशे जने ॥`वैशंपायन उवाच}
{}


\twolineshloka
{धृतराष्ट्रस्तु पुत्रस्य श्रुत्वा वचनमीदृशम्}
{मुहूर्तमिव संचिन्त्य दुर्योधनमथाब्रवीत्}


\twolineshloka
{धर्मनित्यस्तथा पाण्डुः सुप्रीतो मयि कौरवः}
{सर्वेषु ज्ञातिषु तथा मदीयेषु विशेषतः}


\twolineshloka
{नात्र किंचन जानाति भोजनादि चिकीर्षितम्}
{निवेदयति तत्सर्वं मयि धर्मभृतां वरः}


\twolineshloka
{तस्य पुत्रो यथा पाण्डुस्तथा धर्मपरः सदा}
{गुणावाँल्लोकविख्यातो नगरे च प्रतिष्ठितः}


\twolineshloka
{स कथं शक्यतेऽस्माभिरपाक्रष्टुं नरर्षभः}
{राज्यमेष हि नः प्राप्तः ससहयो विशेषतः}


\twolineshloka
{भृता हि पाण्डुनाऽमात्या बलं च सततं मतम्}
{धृताः पुत्राश्च पौत्राश्च तेषामपि विशेषतः}


\twolineshloka
{ते तथा संस्तुतास्तात विषये पाण्डुना नराः}
{कथं युधिष्ठिरस्यार्थे न नो हन्युः सबान्धवान्}


\twolineshloka
{नैते विषयमिच्छेयुर्धर्मत्यागे विशेषतः}
{ते वयं कौरवेन्द्राणामेतेषां च महात्मनाम्}


\twolineshloka
{कथं न वाच्यतां तात गच्छेम जगतस्तथा ॥दुर्योधन उवाच}
{}


\twolineshloka
{मध्यस्थः सततं भीष्मो द्रोणपुत्रो मयि स्थितः}
{यतः पुतस्ततो द्रोणो भविता नात्र संशयः}


\twolineshloka
{कृपः शारद्वतश्चैव यत एव वयं ततः}
{बागिनेयं ततो द्रौणिं न त्यक्ष्यति कथंचन}


\twolineshloka
{क्षत्ता तु बन्धुरस्माकं प्रच्छन्नस्तु ततः परैः}
{न चैकः स समर्थोऽस्मान्पाण्डवार्थे प्रबाधितुम्}


\twolineshloka
{सुविस्रब्धान्पाण्डुसुतान्सह मात्रा विवासय}
{वारणावतमद्यैव नात्र दोषो भविष्यति}


\twolineshloka
{विनिद्राकरणं घोरं हृदि शल्यमिवार्पितम्}
{शोपकपावकमुद्धूतं कर्मणानेन नाशय ॥'}


\chapter{अध्यायः १५३}
\twolineshloka
{वैशंपायन उवाच}
{}


\twolineshloka
{श्रुत्वा पाण्डुसुतान्वीरान्बलोद्रिक्तान्महौजसः}
{धृतराष्ट्रो महीपालश्चिन्तामगमदातुरः}


\twolineshloka
{तत आहूय मन्त्रज्ञं राजशास्त्रार्थवित्तमम्}
{कणिकं मन्त्रिणां श्रेष्ठं धृतराष्ट्रऽब्रवीद्वचः}


\fourlineindentedshloka
{उत्सक्ताः पाण्डवा नित्यं तेभ्योऽसूये द्विजोत्तम}
{तत्र मे निश्चिततमं सन्धिविग्रहकारणम्}
{कणिक त्वं ममाचक्ष्व करिष्ये वचनं तव ॥वैशंपायन उवाच}
{}


\twolineshloka
{`दुर्योधनोऽथ शकुनिः कर्णदुःशासनावपि}
{कणिकं ह्युपसंगृह्य मन्त्रिणं सौबलस्य च}


\twolineshloka
{पप्रच्छुर्भरतश्रेष्ठ पाण्डवान्प्रति नैकधा}
{प्रबुद्धाः पाण्डवा नित्यं सर्वे तेभ्यस्त्रसामहे}


\threelineshloka
{अनूनं सर्वपक्षाणां यद्भवेत्क्षेमकारकम्}
{भारद्वाज तदाचक्ष्व करिष्यामः कथं वयम् ॥वैशंपायन उवाच}
{'}


\threelineshloka
{स प्रसन्नमनास्तेन परिपृष्टो द्विजोत्तमः}
{उवाच वचनं तीक्ष्णं राजशास्त्रार्थदर्शनम् ॥कणिक उवाच}
{}


\twolineshloka
{शृणु राजन्निदं तत्र प्रोच्यमानं मयानघ}
{न मेऽभ्यसूया कर्तव्या श्रुत्वैतत्कुरुसत्तम}


\twolineshloka
{नित्यमुद्यतदण्डः स्यान्नित्यं विवृतपौरुषः}
{अच्छिद्रश्छिद्रदर्शी स्यात्परेषां विवरानुगः}


\twolineshloka
{नित्यमुद्यतदण्डाद्धि भऋशमुद्विजते जनः}
{तस्मात्सर्वाणि कार्याणि दण्डेनैव विधारयेत्}


\twolineshloka
{नास्य च्छिद्रं परः पश्येच्छिद्रेण परमन्वियात्}
{गूहेत्कूर्म इवाङ्गानि रक्षेद्विवरमात्मनः}


\twolineshloka
{`नित्यं च ब्राह्मणाः पूज्या नृपेण हितमिच्छता}
{सृष्टो नृपो हि विप्रामां पालने दुष्टनिग्रहे}


\twolineshloka
{उभाब्यां वर्धते धर्मो धर्मवृद्ध्या जितावुभौ}
{लोकश्चायं परश्चैव ततो धर्मं समाचरेत्}


\twolineshloka
{कृतापराधं पुरुषं दृष्ट्वा यः क्षमते नृपः}
{तेनावमानमाप्नोति पापं चेह परत्र च}


\twolineshloka
{यो विभूतिमवाप्योच्चै राज्ञो विकुरुतेऽधमः}
{तमानयित्वा हत्वा च दद्याद्धीनाय तद्धनम्}


\twolineshloka
{नो चेद्धुरि नियुक्ता ये स्थास्यन्ति वशमात्मनः}
{राजा नियुञ्ज्यात्पुरुषानाप्तान्धर्मार्थकोविदान्}


\twolineshloka
{ये नियुक्तास्तथा केचिद्राष्ट्रं वा यदि वा पुरम्}
{ग्रामं जनपदं वापि बाधेयुर्यदि वा न वा}


\twolineshloka
{परीक्षणार्थं विसृजेदानतांश्छन्नरूपिणः}
{परीक्ष्य पापकं जह्याद्धनमादाय सर्वशः ॥'}


\twolineshloka
{नासम्यक्कृत्यकारी स्यादुपक्रम्य कदाचन}
{कण्टको ह्यपि दुश्छिन्न आस्रावं जनयेच्चिरम्}


\twolineshloka
{वधमेव प्रशंसन्ति शत्रूणामपकारिणाम्}
{सुविदीर्णं सुविक्रान्तं सुयुद्धं सुपलायितम्}


\twolineshloka
{आपद्यापदि काले कुर्वीत न विचारयेत्}
{नावज्ञेयो रिपुस्तात दुर्बलोऽपि कथं चन}


\twolineshloka
{अल्पोऽप्यग्निर्वनं कृत्स्नं दहत्याश्रयसंश्रयात्}
{अन्धः स्यादन्धवेलायां बाधिर्यमपि चाश्रयेत्}


\twolineshloka
{कुर्यात्तृणमयं चापं शयीत मृगशायिकाम्}
{सान्त्वादिभिरुपायैस्तु हन्याच्छत्रुं वशे स्थितं}


\twolineshloka
{दया न तस्मिन्कर्तव्या शरणागत इत्युत}
{निरुद्विग्नो हि भवति न हताज्जायते भयम्}


\twolineshloka
{हन्यादमित्रं दानेन तथा पूर्वापकारिणम्}
{हन्यात्त्रीन्पञ्च सप्तेति परपक्षस्य सर्वशः}


\twolineshloka
{मूलमेवादितश्छिन्द्यात्परपक्षस्य नित्यशः}
{ततः सहायांस्तत्पक्षान्सर्वांश्च तदनन्तरम्}


\twolineshloka
{छिन्नमूले ह्यधिष्ठाने सर्वे तज्जीविनो हताः}
{कथं नु शाखास्तिष्ठेरंश्छिन्नमूले वनस्पतौ}


\twolineshloka
{एकाग्रः स्यादविवृतो नित्यं विवरदर्शकः}
{राजन्नित्यं सपत्नेषु नित्योद्विग्नः समाचरेत्}


\twolineshloka
{अग्न्याधानेन यज्ञेन काषायेण जटाजिनैः}
{लोकान्विश्वासयित्वैव ततो लुम्पेद्यथा वृकः}


\twolineshloka
{अङ्कुशं शोचमित्याहुरर्थानामुपधारणे}
{आनाम्य फलितां शाखां पक्वं पक्वं प्रशातयेत्}


\twolineshloka
{फलार्थोऽयं समारम्भो लोके पुंसां विपश्चिताम्}
{वहेदमित्रं स्कन्धेन यावत्कालस्य पर्ययः}


\twolineshloka
{ततः प्रत्यागते काले भिन्द्याद्धृटमिवाश्मनि}
{अमित्रो न विमोक्तव्यः कृपणं बह्वपि ब्रुवन्}


\twolineshloka
{कृपा न तस्मिन्कर्तव्या हन्यादेवापकारिणम्}
{हन्यादमित्रं सान्त्वेन तथा दानेन वा पुनः}


\threelineshloka
{तथैव भेददण्डाभ्यां सर्वोपायैः प्रशातयेत्}
{धृतराष्ट्र उवाच}
{कथं सान्त्वेन दानेन भेदैर्दण्डेन वा पुनः}


\threelineshloka
{अमित्रः शक्यते हन्तुं तन्मे ब्रूहि यथातथम्}
{कणिक उवाच}
{शृणु राजन्यथा वृत्तं वने निवसतः पुरा}


\twolineshloka
{जम्बुकस्य महाराज नीतिशास्त्रार्थदर्शिनः}
{अथ कश्चित्कृतप्रज्ञः शृगालः स्वार्थपण्डितः}


\twolineshloka
{सखिभिर्न्यवसत्सार्धं व्याघ्राखुवृकबभ्रुभिः}
{तेऽपश्यन्विपिने तस्मिन्बलिनं मृगयूथपम्}


\threelineshloka
{अशक्ता ग्रहणे तस्य ततो मन्त्रममन्त्रयन्}
{जम्बुक उवाच}
{असकृद्यतितो ह्येष हन्तुं व्याघ्र वने त्वया}


\twolineshloka
{युवा वै जवसंपन्नो बुद्धिशाली न शक्यते}
{मूषिकोऽस्य शयानस्य चरणौ भक्षयत्वयम्}


\twolineshloka
{अथैनं भक्षितैः पादैर्व्याघ्रो गृह्णातु वै ततः}
{ततो वै भक्षयिष्यामः सर्वे मुदितमानसाः}


\twolineshloka
{जम्बुकस्य तु तद्वाक्यं तथा चक्रः समाहिताः}
{मूषिकाभक्षितैः पादैर्मृगं व्याघ्रोऽवधीत्तदा}


\twolineshloka
{दृष्ट्वैवाचेष्टमानं तु भूमौ मृगकलेवरम्}
{स्नात्वाऽऽगच्छत भद्रं वोरक्षामीत्याह जम्बुकः}


\twolineshloka
{शृगालवचनात्तेऽपि गताः सर्वे नदीं ततः}
{स चिन्तापरमो भूत्वा तस्थौ तत्रैव जम्बुकः}


\threelineshloka
{अथाजगाम पूर्वं तु स्नात्वा व्याघ्रो महाबलः}
{ददर्श जम्बुकं चैव चिन्ताकुलितमानसम् ॥व्याघ्र उवाच}
{}


\threelineshloka
{किं शोचसि महाप्राज्ञ त्वं नो बुद्धिमतां वरः}
{अशित्वा पिशितान्यद्य विहरिष्यामहे वयम् ॥जम्बुक उवाच}
{}


\twolineshloka
{शृणु मे त्वं महाबाहो यद्वाक्यं मूषिकोऽब्रवीत्}
{धिग्बलं मृगराजस्य मयाद्यायं मृगो हतः}


\threelineshloka
{मद्बाहुबलमाश्रित्य तृप्तिमद्य गमिष्यति}
{तस्यैवं गर्जितं श्रुत्वा ततो भक्ष्यं न रोचये ॥व्याघ्र उवाच}
{}


\twolineshloka
{ब्रवीति यदि स ह्येवं काले ह्यस्मि प्रबोधितः}
{स्वबाहुबलमाश्रित्य हनिष्येऽहं वनेचरान्}


\twolineshloka
{खादिष्ये तत्र मांसानि इत्युक्त्वा प्रस्थितोवनम्}
{एतस्मिन्नेव काले तु मूषिकोऽप्याजगाम ह}


\twolineshloka
{तमागतमभिप्रेक्ष्य शृगालोऽप्यब्रवीद्वचः}
{शृणु मीषिक भद्रं ते नकुलो यदिहाब्रवीत्}


\twolineshloka
{मृगमांसं न खादेयं गरमेतन्न रोचते}
{मूषिकं भक्षयिष्यामि तद्भवाननुमन्यताम्}


\twolineshloka
{तच्छ्रुत्वा मूषिको वाक्यं संत्रस्तः प्रगतो बिलम्}
{ततः स्नात्वा स वै तत्र आजगाम वृको नपृ}


\twolineshloka
{तमागतमिदं वाक्यमब्रवीज्जम्बुकस्तदा}
{मृगराजो हि संक्रुद्धो न ते साधु भविष्यति}


\twolineshloka
{सकलत्रस्त्विहायाति कुरुष्व यदनन्तरम्}
{एवं संचोदितस्तेन जम्बुकेन तदा वृकः}


\twolineshloka
{ततोऽवलुम्पनं कृत्वा प्रयातः पिशिताशनः}
{एतस्मिन्नेव काले तु नकुलोऽप्याजगाम ह}


\twolineshloka
{तमुवाच महाराज नकुलं जम्बुको वने}
{स्वबाहुबलमाश्रित्य निर्जितास्तेऽन्यतो गताः}


\threelineshloka
{मम दत्वा नियुद्धं त्वं भुङ्क्ष्व मांसं यथेप्सितम्}
{नकुल उवाच}
{मृगराजो वृकश्चैव बुद्धिमानपि मूषिकः}


\threelineshloka
{निर्जिता यत्त्वया वीरास्तस्माद्वीरतरो भवान्}
{न त्वयाप्युत्सहे योद्धुमित्युक्त्वा सोऽप्युपागमत् ॥कणिक उवाच}
{}


\threelineshloka
{एवं तेषु प्रयातेषु जम्बुको हृष्टमानसः}
{खादति स्म तदा मांसमेकः सन्मन्त्रनिश्चयात्}
{एवं समाचरन्नित्यं सुखमेधेत भूपतिः}


\twolineshloka
{भयेन भेदयेद्भीरुं शूरमञ्जलिकर्मणा}
{लुब्धमर्थप्रदानेन समं न्यूनं तथौजसा}


% Check verse!
एवं ते कथितं राजन् शृणु चाप्यपरं तथा
\twolineshloka
{पुत्रः सखा वा भ्राता वा पिता वा यदि वा गुरुः}
{रिपुस्यानेषु वर्तन्तो हन्तव्या भूतिमिच्छता}


\threelineshloka
{शपथेनाप्यरिं हन्यादर्थदानेन वा पुनः}
{विषेण मायया वापि नोपेक्षेते कथंचन}
{उभौ चेत्संशयोपेतौ श्रद्धवांस्तत्र वर्धते}


\twolineshloka
{गुरोरप्यवलिप्तस्य कार्याकार्यमजानतः}
{उत्पथं प्रतिपन्नस्य न्याय्यं भवति शासनम्}


\threelineshloka
{क्रुद्धोऽप्यक्रुद्धरूपः स्यात्स्मितपूर्वाभिभाषिता}
{`न चैनं क्रोधसंदीप्तं विद्यात्कश्चित्कथंचन}
{'न चाप्यन्यमपध्वंसेत्कदाचित्कोपसंयुतः}


\twolineshloka
{प्रहरिष्यन्प्रियं ब्रूयात्प्रहरन्नपि भारत}
{प्रहृत्य च प्रियं ब्रूयाच्छोचन्निव रुदन्निव}


\twolineshloka
{आश्वासयेच्चापि परं सान्त्वधर्मार्थवृत्तिभिः}
{अथ तं प्रहरेत्काले तथा विचलितं पथि}


\twolineshloka
{अपि घोरापराधस्य धर्ममाश्रित्य तिष्ठतः}
{स हि प्रच्छाद्यते दोषः शैलो मेघैरिवासितैः}


\twolineshloka
{यः स्यादनुप्राप्तवधस्तस्यागारं प्रदीपयेत्}
{अधनान्नास्तिकांश्चोरान्विषकर्मसु योजयेत्}


\twolineshloka
{प्रत्युत्थानासनाद्येन संप्रदानेन केनचित्}
{अतिविश्रब्धघाती स्यात्तीक्ष्णंदष्ट्रो निमग्नकः}


\twolineshloka
{अशङ्कितेभ्यः शङ्केत शङ्कितेभ्यश्च सर्वशः}
{अशङ्क्याद्भयमुत्पन्नमपि मूलं निकृन्तति}


\twolineshloka
{न विश्वसेदविश्वस्ते विश्वस्ते नातिविश्वसेत्}
{विश्वासाद्भयमुत्पन्नं मूलान्यपि निकृन्तति}


\twolineshloka
{चारः सुविहितः कार्य आत्मनश्च परस्य वा}
{पाषण्डांस्तापसादींश्च परराष्ट्रेषु योजयेत्}


\twolineshloka
{उद्यानेषु विहारेषु देवतायतनेषु च}
{पानागारेषु रथ्यासु सर्वतीर्थेषु चाप्यथ}


\twolineshloka
{चत्वरेषु च कूपेषु पर्वतेषु वनेषु च}
{समवायेषु सर्वेषु सरित्सु च विचारयेत्}


\twolineshloka
{वाचा भृशं विनीतः स्याद्धृदयेन तथा क्षुरः}
{स्मितपूर्वाभिभाषी स्यात्सृष्टो रौद्रस्य कर्मणः}


\twolineshloka
{अञ्जलिः शपथः सान्त्वं शिरसा पादवन्दनम्}
{आशाकरणमित्येवं कर्तव्यं भूतिमिच्छता}


\twolineshloka
{सुपुष्पितः स्यादफलः फलवान्स्याद्दुरारुहः}
{आमः स्यात्पक्वसङ्काशो नच जीर्येत कर्हिचित्}


\twolineshloka
{त्रिवर्गे त्रिविधा पीडा ह्यनुबन्धास्तथैव च}
{अनुबन्धाः शुभा ज्ञेयाः पीडास्तु परिवर्जयेत्}


\twolineshloka
{धर्मं विचरतः पीडा सापि द्वाभ्यां नियच्छति}
{अर्थं चाप्यर्थलुब्धस्य कामं चातिप्रवर्तिनः}


\twolineshloka
{अगर्वितात्मा युक्तश्च सान्त्वयुक्तोऽनसूयिता}
{अवेक्षितार्थः शुद्धात्मा मन्त्रयीत द्विजैः सहा}


\twolineshloka
{कर्मणा येन केनैव मृदुना दारुणेन च}
{उद्धरेद्दीनमात्मानं समर्थो धर्ममाचरेत्}


\twolineshloka
{न संशयमनारुह्य नरो भद्राणि पश्यति}
{संशयं पुनरारुह्य यदि जीवति पश्यति}


\twolineshloka
{यस्य बुद्धिः परिभवेत्तमतीतेन सान्त्वयेत्}
{अनागतेन दुर्बुद्धिं प्रत्युत्पन्नेन पण्डितम्}


\twolineshloka
{योऽरिणा सह सन्धाय शयीत कृतकृत्यवत्}
{स वृक्षाग्रे यथा सुप्तः पतितः प्रतिबुध्यते}


\twolineshloka
{मन्त्रसंवरणे यत्नः सदा कार्योऽनसूयता}
{आकारमभिरक्षेत चारेणाप्यनुपालितः}


\twolineshloka
{`न रात्रौ मन्त्रयेद्विद्वान्न च कैश्चिदुपासितः}
{प्रासादाग्रे शिलाग्रे वा विशाले विजनेपि वा}


\twolineshloka
{समन्तात्तत्र पश्यद्भिः सहाप्तैरेव मन्त्रयेत्}
{नैव संवेशयेत्तत्र मन्त्रवेश्मनि शारिकाम्}


\twolineshloka
{शुकान्वा शारिका वापि बालमूर्खजडानपि}
{प्रविष्टानपि निर्वास्य मन्त्रयेद्धार्मिकैर्द्विजैः}


\twolineshloka
{नीतिज्ञैर्न्यायशास्त्रज्ञैरितिहासे सुनिष्ठितैः}
{रक्षां मन्त्रस्य निश्छिद्रां मन्त्रान्ते निश्चयेत्स्वयम्}


\twolineshloka
{वीरोपवर्णितात्तस्माद्धर्मार्थाभ्यामथात्मना}
{एकेन वाथ विप्रेण ज्ञातबुद्धिर्विनिश्चयेत्}


\twolineshloka
{तृतीयेन न चान्येन व्रजेन्निश्चयमात्मवान्}
{षट्कर्णश्छिद्यते मन्त्र इति नीतिषु पठ्यते}


\threelineshloka
{निःसृतो नाशयेन्मन्त्रो हस्तप्राप्तामपि श्रियम्}
{स्वमतं च परेषां च विचार्य च पुनःपुनः}
{गुणवद्वाक्यमादद्यान्नैव तृप्येद्विचक्षणः ॥'}


\twolineshloka
{नाच्छित्वा परमर्माणि नाकृत्वा कर्म दारुणम्}
{नाहत्वा मत्स्यघातीव प्राप्नोति महतीं श्रियम्}


\twolineshloka
{कर्शितं व्याधितं क्लिन्नमपानीयमघासकम्}
{परिविश्वस्तमन्दं च प्रहर्तव्यमरेर्बलम्}


\twolineshloka
{नार्थिकोऽर्थिनमभ्येति कृतार्थे नास्ति सङ्गतम्}
{तस्मात्सर्वाणि साध्यानि सावशेषाणि कारयेत्}


\twolineshloka
{संग्रहे विग्रहे चैव यत्नः कार्योऽनसूयता}
{उत्साहश्चापि यत्नेन कर्तव्यो भूतिमिच्छता}


\twolineshloka
{नास्य कृत्यानि बुध्येरन्मित्राणि रिपवस्तथा}
{आरब्धान्येव पश्येरन्सुपर्यवसितान्यपि}


\twolineshloka
{भीतवत्संविधातव्यं यावद्भयमनागतम्}
{आगतं तु भयं दृष्ट्वा प्रहर्तव्यमभीतवत्}


\twolineshloka
{दैवेनोपहतं शत्रुमनुगृह्णाति यो नरः}
{स मृत्युमुपगृह्णाति गर्भमश्वतरी यथा}


\twolineshloka
{अनागतं हि बुध्येत यच्च कार्यं पुरः स्थितम्}
{न तु बुद्धिक्षयात्किंचिदतिक्रामेत्प्रयोजनम्}


\threelineshloka
{उत्साहश्चापि यत्नेन कर्तव्यो भूतिमिच्छता}
{विभज्य देशकालौ च दैवं धर्मादयस्त्रयः}
{नैःश्रेयसौ तु तौ ज्ञेयौ देशकालाविति स्थितिः}


\twolineshloka
{तालवत्कुरुते मूलं बालः शत्रुरुपेक्षितः}
{गहनेऽग्निरिवोत्सृष्टः क्षिप्रं संजायते महान्}


\twolineshloka
{अग्निस्तोकमिवात्मानं सन्धुक्षयति यो नरः}
{स वर्धमानो ग्रसते महान्तमपि संचयम्}


% Check verse!
`आदावेव ददानीति प्रियं ब्रूयान्निरर्थकम् ॥'
\twolineshloka
{आशां कालवतीं कुर्यात्कालं विघ्नेन योजयेत्}
{विघ्नं निमित्ततो ब्रूयान्निमित्तं वाऽपि हेतुतः}


\twolineshloka
{क्षुरो भूत्वा हरेत्प्राणान्निशितः कालसाधनः}
{प्रतिच्छन्नो लोमहारी द्विषतां परिकर्तनः}


\twolineshloka
{पाण्डवेषु यथान्यायमन्येषु च कुरूद्वह}
{वर्तमानो न मज्जेस्त्वं तथा कृत्यं समाचर}


\twolineshloka
{सर्वकल्याणसंपन्नो विशिष्ट इति निश्चयः}
{तस्मात्त्वं पाण्डुपुत्रेभ्यो रक्षात्मानं नराधिप}


\threelineshloka
{भ्रातृव्या बलवन्तस्ते पाण्डुपुत्रा नराधिप}
{पश्चात्तापो यथा न स्यात्तथा नीतिर्विधीयताम् ॥वैशंपायन उवाच}
{}


\twolineshloka
{एवमुक्त्वा संप्रतस्थे कणिकः स्वगृहं ततः}
{धृतराष्ट्रोऽपि कौरव्यः शोकार्तः समपद्यत}


\chapter{अध्यायः १५४}
\twolineshloka
{वैशंपायन उवाच}
{}


\threelineshloka
{कणिकस्य मतं श्रुत्वा कार्त्स्न्येन भरतर्षभ}
{दुर्योधनश्च कर्णश्च शकुनिश्चापि सौबलः}
{दुशासनचतुर्थास्ते मन्त्रयामासुरेकदा}


\twolineshloka
{ते कौरव्यमनुज्ञाप्य धृतराष्ट्रं नराधिपम्}
{दहने तु सपुत्रायाः कुन्त्या बुद्धिमकारयन्}


\twolineshloka
{तेषामिङ्गितभावज्ञो विदुरस्तत्त्वदर्शिवान्}
{आकारेण च तं मन्त्रं बुबुधे दुष्टचेतसाम्}


\twolineshloka
{ततो विदितवेद्यात्मा पाण्डवानां हिते रतः}
{पलायने मतिं चक्रे कुन्त्याः पुत्रैः सहानघः}


\twolineshloka
{ततो वातसहां नावं यन्त्रयुक्तां पताकिनीम्}
{ऊर्मिक्षमां दृढां कृत्वा कुन्तीमिदमुवाच ह}


\twolineshloka
{एष जातः कुलस्यास्य कीर्तिवंशप्रणाशनः}
{धृतराष्ट्रः परीतात्मा धर्मं त्यजति शाश्वतम्}


\twolineshloka
{इयं वारिपथे युक्ता तरङ्गपवनक्षमा}
{नौर्यया मृत्युपाशात्त्वं सपुत्रा मोक्ष्यसे शुभे}


\twolineshloka
{तच्छ्रुत्वा व्यथिता कुन्ती पुत्रैः सह यशस्विनी}
{नावमारुह्य गङ्गायां प्रययौ भरतर्षभ}


\twolineshloka
{ततो विदुरवाक्येन नावं विक्षिप्य पाण्डवाः}
{धनं चादाय तैर्दत्तमरिष्टं प्राविशन्वनम्}


\twolineshloka
{निषादी पञ्चपुत्रा तु जातेषु तत्र वेश्मनि}
{कारणाभ्यागता दग्धा सह पुत्रैरनागसा}


\twolineshloka
{स च म्लेच्छाधमः पापो दग्धस्तत्र पुरोचनः}
{वञ्चिताश्च दुरात्मानो धार्तराष्ट्राः सहानुगाः}


\twolineshloka
{अविज्ञाता महात्मनो जनानामक्षतास्तथा}
{जनन्या सह कौन्तेया मुक्ता विदुरमन्त्रिताः}


\twolineshloka
{ततस्तस्मिन्पुरे लोका नगरे वारणावते}
{दृष्ट्वा जतुगृहं दग्धमन्वशोचन्त दुःखिताः}


\twolineshloka
{प्रेषयामासू राजानं यथावृत्तं निवेदितुम्}
{संवृतस्ते महान्कामः पाण्डवान्दग्धवानसि}


\twolineshloka
{सकामो भव कौरव्य भुङ्क्ष्व राज्यं सपुत्रकः}
{तच्छ्रुत्वा धृतराष्ट्रस्तु सहपुत्रेण शोचयन्}


\threelineshloka
{प्रेतकार्याणि च तथा चकार सह बान्धवैः}
{पाण्डवानां तथा क्षत्ता भीष्मश्च कुरुसत्तमः ॥जनमेजय उवाच}
{}


\twolineshloka
{पुनर्विस्तरशः श्रोतुमिच्छामि द्विजसत्तम}
{दाहं जतुगृहस्यैव पाण्डवानां च मोक्षणम्}


\threelineshloka
{सुनृशंसमिदं कर्म तेषां क्रूरोपसंहितम्}
{कीर्तयस्व यथावृत्तं परं कौतूहलं मम ॥वैशंपायन उवाच}
{}


\threelineshloka
{शृणु विस्तरशो राजन्वदतो मे परन्तप}
{दाहं जतुगृहस्यैतत्पाण्डवानां च मोक्षणम् ॥वैशंपायन उवाच}
{}


\threelineshloka
{ततो दुर्योधनो राजा धृतराष्ट्रमभाषत}
{पाण्डवेभ्यो भयं नः स्यात्तान्विवासयतां भवान्}
{निपुणेनाभ्युपायेन नगरं वारणावतम्}


\twolineshloka
{धृतराष्ट्रस्तु पुत्रेण श्रुत्वा वचनमीरितम्}
{मुहूर्तमिव संचिन्त्य दुर्योधनमथाब्रवीत्}


\twolineshloka
{धर्मनित्यः सदा पाण्डुस्तथा धर्मपरायणः}
{सर्वेषु ज्ञातिषु तथा मयि त्वासीद्विशेषतः}


\twolineshloka
{नासौ किंचिद्विजानाति भोजनादिचिकीर्षितम्}
{निवेदयति नित्यं हि मम राज्यं धृतव्रतः}


\twolineshloka
{तस्य पुत्रो यथा पाण्डुस्तथा धर्मपरायणः}
{गुणवान्लोकविख्यातः पौरवाणां सुसंमतः}


\twolineshloka
{स कथं शक्यतेऽस्माभिरपाकर्तुं बलादितः}
{पितृपैतामाहाद्राज्यात्ससहायो विशेषतः}


\twolineshloka
{भृता हि पाण्डुनाऽमात्या बलं च सततं भृतम्}
{भृताः पुत्राश्च पौत्राश्च तेषामपि विशेषतः}


\threelineshloka
{ते पुरा सत्कृतास्तात पाण्डुना नागरा जनाः}
{कथं युधिष्ठिरस्यार्थे न नो हन्युः सबान्धवान् ॥दुर्योधन उवाच}
{}


\twolineshloka
{एवमेतन्मया तात भावितं दोषमात्मनि}
{दृष्ट्वा प्रकृतयः सर्वा अर्थमानेन पूजिताः}


\twolineshloka
{ध्रुवमस्मत्सहायास्ते भविष्यन्ति प्रधानतः}
{अर्थवर्गः सहामात्यो मत्संस्थोऽद्य महीपते}


\twolineshloka
{स भवान्पाण्डवानाशु विवासयितुमर्हति}
{मृदुनैवाभ्युपायेन नगरं वारणावतम्}


\threelineshloka
{यदा प्रतिष्ठितं राज्यं मयि राजन्भविष्यति}
{तदा कुन्ती सहापत्या पुनरेष्यति भारत ॥धृतराष्ट्र उवाच}
{}


\twolineshloka
{दुर्योधन ममाप्येतद्धृदि संपरिवर्तते}
{अभिप्रायस्य पापत्वान्नैवं तु विवृणोम्यहम्}


\twolineshloka
{न च भीष्मो न च द्रोणो न च क्षत्ता न गौतमः}
{विवास्यमानान्कौन्तेयाननुमंस्यन्ति कर्हिचित्}


\twolineshloka
{समा हि कौरवेयाणां वयं ते चैव पुत्रक}
{नैते विषममिच्छेयुर्धर्मयुक्ता मनस्विनः}


\threelineshloka
{ते वयं कौरवेयाणामेतेषां च महात्मनाम्}
{कथं न वध्यतां तात गच्छाम जगतस्तथा ॥दुर्योधन उवाच}
{}


\twolineshloka
{मध्यस्थः सततं भीष्मो द्रोणपुत्रो मयि स्थितः}
{यतः पुत्रस्ततो द्रोणो भविता नात्र संशयः}


\twolineshloka
{कृपः शारद्वतश्चैव यत एतौ ततो भवेत्}
{द्रोणं च भागिनेयं च न स त्यक्ष्यति कर्हिचित्}


\twolineshloka
{क्षत्ताऽर्थबद्धस्त्वस्माकं प्रच्छन्नं संयतः परैः}
{न चैकः स समर्थोऽस्मान्पाम्डवार्थेऽधिबाधितुम्}


\twolineshloka
{सुविस्रब्धः पाण्डुपुत्रान्सह मात्रा प्रवासय}
{वारणावतमद्यैव यथा यान्ति यथा कुरु}


\twolineshloka
{विनिद्रकरणं घोरं हृदि शल्यमिवार्पितम्}
{शोकपावकमुद्भूतं कर्मणैतेन नाशय}


\chapter{अध्यायः १५५}
\twolineshloka
{वैशंपायन उवाच}
{}


\threelineshloka
{ततो दुर्योधनो राजा सर्वास्तु प्रकृतीः शनैः}
{अर्थमानप्रदानाभ्यां संजहार सहानुजः}
{`युयुत्सुमपनीयैकं धार्तराष्ट्रं सहोदरम् ॥'}


\twolineshloka
{धृतराष्ट्रप्रयुक्तास्तु केचित्कुशलमन्त्रिणः}
{कथयाञ्चक्रिरे रम्यं नगरं वारणावतम्}


\twolineshloka
{अयं समाजः सुमहान्रमणीयतमो भुवि}
{उपस्थितः पशुपतेर्नगरे वारणावते}


\twolineshloka
{सर्वरत्नसमाकीर्णे पुण्यदेशे मनोरमे}
{इत्येवं धृतराष्ट्रस्य वचनाच्चक्रिरे कथाः}


\twolineshloka
{कथ्यमाने तथा रम्ये नगरे वारणावते}
{गमने पाण्डुपुत्राणां जज्ञे तत्र मतिर्नृप}


\twolineshloka
{यदा त्वमन्यत नृपो जातकौतूहला इति}
{उवाचैतानेत्य तदा पाण्डवानम्बिकासुतः}


\twolineshloka
{`अधीतानि च शास्त्राणि युष्माभिरिह कृत्स्नशः}
{अस्त्राणि च तथा द्रोणाद्गौतमाच्च शरद्वतः}


\threelineshloka
{कृतकृत्या भवन्तस्तु सर्वविद्याविशारदाः}
{सोऽहमेवं गते ताताश्चिन्तयामि समन्ततः}
{रक्षणे व्यवहारे च राज्यस्य सततं हिते ॥'}


\twolineshloka
{ममैते पुरुषा नित्यं कथयन्ति पुनःपुनः}
{रमणीयतमं लोके नगरं वारणावतम्}


\twolineshloka
{ते ताता यदि मन्यध्वमुत्सवं वारणावते}
{सगणाः सान्वयाश्चैव विहरध्वं यथाऽमराः}


\twolineshloka
{ब्राह्मणेभ्यश्च रत्नानि गायनेभ्यश्च सर्वशः}
{प्रयच्छध्वं यथाकामं देवा इव सुवर्चसः}


\twolineshloka
{कंचित्कालं विहृत्यैवमनुभूय परां मुदम्}
{इदं वै हास्तिनपुरं सुखिनः पुनरेष्यथ}


\threelineshloka
{`निवसध्वं च तत्रैव संरक्षणपरायणाः}
{वैलक्षण्यं न वै तत्र भविष्यति परंतपाः ॥'वैशंपायन उवाच}
{}


\twolineshloka
{धृतराष्ट्रस्य तं काममनुबुध्य युधिष्ठिरः}
{आत्मनश्चासहायत्वं तथेति प्रत्युवाच तम्}


\twolineshloka
{ततो भीष्मं शान्तनवं विदुरं च महामतिम्}
{द्रोणं च बाह्लिकं चैव सोमदत्तं च कौरवम्}


\twolineshloka
{कृपमाचार्यपुत्रं च भूरिश्रवसमेव च}
{मान्यानन्यानमात्यांश्च ब्राह्मणांश्च तपोधनान्}


\threelineshloka
{पुरोहितांश्च पौत्रांश्च गान्धारीं च यशस्विनीम्}
{`सर्वा मातॄरुपस्पृष्ट्वा विदुरस्य च योषितः}
{'युधिष्ठिरः शनैर्दीन उवाचेदं वचस्तदा}


\twolineshloka
{रमणीये जनाकीर्णे नगरे वारणावते}
{सगणास्तत्र यास्यामो धृतराष्ट्रस्य शासनात्}


\threelineshloka
{प्रसन्नमनसः सर्वे पुण्या वाचो विमुञ्चत}
{आशीर्भिर्बृहितानस्मान्न पापं प्रसहिष्यते ॥वैशंपायन उवाच}
{}


\twolineshloka
{एवमुक्तास्तु ते सर्वे पाण्डुपुत्रेण कौरवाः}
{प्रसन्नवदना भूत्वा तेऽन्ववर्तन्त पाण्डवान्}


\twolineshloka
{स्वस्त्यस्तु वः पथि सदा भूतेभ्यश्चैव सर्वशः}
{मा च वोस्त्वशुभं किंचित्सर्वशः पाण्डुनन्दनाः}


\twolineshloka
{ततः कृतस्वस्त्ययना राज्यलाभाय पार्थिवाः}
{कृत्वा सर्वाणि कार्याणि प्रययुर्वारणावतम्}


\chapter{अध्यायः १५६}
\twolineshloka
{वैशंपायन उवाच}
{}


\twolineshloka
{धृतराष्ट्रप्रयुक्तेषु पाण्डुपुत्रेषु भारत}
{दुर्योधनः परं हर्षमगच्छत्स दुरात्मवान्}


\threelineshloka
{`ततः सुबलपुत्रश्च कर्णदुर्योधनावपि}
{दहने सह पुत्रायाः कुन्त्या मतिमकुर्वत}
{मन्त्रयित्वा स तैः सार्धं दुरात्मा धृतराष्ट्रजः ॥'}


\twolineshloka
{स पुरोचनमेकान्तमानीय भरतर्षभ}
{गृहीत्वा दक्षिणे पाणौ सचिवं वाक्यमब्रवीत्}


\twolineshloka
{ममेयं वसुसंपूर्णा पुरोचन वसुन्धरा}
{यथैव भविता तात तथा त्वं द्रष्टुमर्हसि}


\twolineshloka
{न हि मे कश्चिदन्योऽस्ति विश्वासिकतरस्त्वया}
{सहायो येन सन्धाय मन्त्रयेयं यथा त्वया}


\twolineshloka
{संरक्ष तात मन्त्रं च सपत्नांश्च ममोद्धर}
{निपुणेनाभ्युपायेन यद्ब्रवीमि तथा कुरु}


\twolineshloka
{पाण्डवा धृतराष्ट्रेण प्रेषिता वारणावतम्}
{उत्सवे विहरिष्यन्ति धृतराष्ट्रस्य शासनात्}


\twolineshloka
{स त्वं रासभयुक्तेन स्यन्दनेनाशुगामिना}
{वारणावतमद्यैव यथा यासि तया कुरु}


\twolineshloka
{तत्र गत्वा चतुःशालं गृहं परमसंवृतम्}
{नगरोपान्तमाश्रित्य कारयेथा महाधनम्}


\twolineshloka
{शणसर्जरसादीनि यानि द्रव्याणि कानिचित्}
{आग्नेयान्युत सन्तीह तानि तत्र प्रदापय}


\threelineshloka
{`बल्वजेन च संमिश्रं मधूच्छिष्टेन चैव हि}
{'सर्पिस्तैलवसाभिश्च लाक्षया चाप्यनल्पया}
{मृत्तिकां मिश्रयित्वा त्वं लेपं कुड्येषु दापय}


\twolineshloka
{शणं तैलं घृतं चैव जतु दारूणि चैव हि}
{तस्मिन्वेश्मनि सर्वाणि निक्षिपेथाः समन्ततः}


\twolineshloka
{`लाक्षाशममधूच्छिष्टवेष्टितानि मृदापि च}
{लेपयित्वा गुरूण्याशु दारुयन्त्राणि दापय ॥'}


\twolineshloka
{यथा च तन्न पश्येरन्परीक्षन्तोऽपि पाण्डवाः}
{आग्नेयमिति तत्कार्यमपि चान्येऽपि मानवाः}


\twolineshloka
{वेश्मन्येवं कृते तत्र कृत्वा तान्परमार्चितान्}
{वासयेथाः पाण्डवेयान्कुन्तीं च ससुहृज्जनाम्}


\twolineshloka
{आसनानि च दिव्यानि यानानि शयनानि न}
{निघातव्यानि पाण्डूनां यथा तुष्येच्च मे पिता}


\threelineshloka
{यथा च तन्न जानन्ति नगरे वारणावते}
{`यथा रमेरन्विस्रब्धा नगरे वारणावते}
{'तथा सर्वं विधातव्यं यावत्कालस्य पर्ययः}


\twolineshloka
{ज्ञात्वा च तान्सुविश्वस्ताञ्शयानानकुतोभयान्}
{अग्निस्त्वया ततो देयो द्वारतस्तस्य वेश्मनः}


\twolineshloka
{दग्धानेवं स्वके गेहे दाहिताः पाण्डवा इति}
{न गर्हयेयुरस्मान्वै पाण्डवार्थाय कर्हिचित्}


\twolineshloka
{स तथेति प्रतिज्ञाय कौरवाय पुरोचनः}
{प्रायाद्रासभयुक्तेन स्यन्दनेनाशुगामिना}


\twolineshloka
{स गत्वा त्वरितं राजन्दुर्योधनमते स्थितः}
{यथोक्तं राजपुत्रेण सर्वं चक्रे पुरोचनः}


% Check verse!
`तेषां तु पाण्डवेयानां गृहं रौद्रमकल्पयत् ॥'
\chapter{अध्यायः १५७}
\twolineshloka
{वैशंपायन उवाच}
{}


\twolineshloka
{पाण्डवास्तु रथान्युक्तान्सदश्वैरनिलोपमैः}
{आरोहमाणा भीष्मस्य पादौ जगृहुरार्तवत्}


\twolineshloka
{राज्ञश्च धृतराष्ट्रस्य द्रोणस्य च महात्मनः}
{अन्येषां चैव वृद्धानां कृपस्य विदुरस्य च}


\twolineshloka
{एवं सर्वान्कुरून्वृद्धानभिवाद्य यतव्रताः}
{समालिङ्ग्य समानान्वै बालैश्चाप्यभिवादिताः}


\twolineshloka
{सर्वा मातॄस्तथाऽऽपृच्छ्य कृत्वा चैव प्रदक्षिणम्}
{प्रकृतीरपि सर्वाश्च प्रययुर्वारणावतम्}


\twolineshloka
{विदुरश्च महाप्राज्ञस्तथाऽन्ये कुरुपुङ्गवाः}
{पौराश्च पुरुषव्याघ्रानन्वीयुः शोककर्शिताः}


\twolineshloka
{तत्र केचिद्ब्रुवन्ति स्म ब्राह्मणा निर्भयास्तदा}
{दीनान्दृष्ट्वा पाण्डुसुतानतीव भृशदुःखिताः}


\twolineshloka
{विषमं पश्यते राजा सर्वथा स सुमन्दधीः}
{कौरव्यो धृतराष्ट्रस्तु न च धर्मं प्रपश्यति}


\twolineshloka
{न हि पापमपापात्मा रोचयिष्यति पाण्डवः}
{भीमो वा बलिनां श्रेष्ठः कौन्तेयो वा धनञ्जयः}


\twolineshloka
{कुत एव महात्मानौ माद्रीपुत्रौ करिष्यतः}
{तान्राज्यं पितृतः प्राप्तान्धृतराष्ट्रो न मृष्यति}


\twolineshloka
{अधर्म्यमिदमत्यन्तं कथं भीष्मोऽनुमन्यते}
{विवास्यमानानस्थाने नगरे योऽभिमन्यते}


\twolineshloka
{पितेव हि नृपोऽस्माकमभूच्छान्तनवः पुरा}
{विचित्रवीर्यो राजर्षिः पाण्डुश्च कुरुनन्दनः}


\twolineshloka
{स तस्मिन्पुरुषव्याघ्रे देवभावं गते सति}
{राजपुत्रानिमान्बालान्धृतराष्ट्रो न मृष्यति}


\threelineshloka
{वयमेतदनिच्छन्तः सर्व एव पुरोत्तमात्}
{गृहान्विहाय गच्छामो यत्र गन्ता युधिष्ठिरः ॥वैशंपायन उवाच}
{}


\twolineshloka
{तांस्तथा वादिनः पौरान्दुःकितान्दुःखकर्शितः}
{उवाच मनसा ध्यात्वा धर्मराजो युधिष्ठिरः}


\twolineshloka
{पिता मान्यो गुरुः श्रेष्ठो यदाह पृथिवीपतिः}
{अशङ्कमानैस्तत्कार्यमस्माभिरिति नो व्रतम्}


\twolineshloka
{भवन्तः सुहृदोऽस्माकं यात कृत्वा प्रदक्षिणम्}
{प्रतिनन्द्य तथाऽऽशीर्भिर्निवर्तध्वं यथागृहम्}


\twolineshloka
{यदा तु कार्यमस्माकं भवद्भिरुपपत्स्यते}
{तदा करिष्यथास्माकं प्रियाणि च हितानि च}


\twolineshloka
{एवमुक्तास्तदा पौराः कृत्वा चापि प्रदक्षिणम्}
{आशीर्भिश्चाभिनन्द्यैताञ्जग्मुर्नगरमेव हि}


\twolineshloka
{पौरेषु विनिवृत्तेषु विदुरः सर्वधर्मवित्}
{बोधयन्पाण्डवश्रेष्ठमिदंवचनमब्रवीत्}


\threelineshloka
{प्राज्ञः प्राज्ञं प्रलापज्ञः प्रलापज्ञमिदं वचः}
{यो जानाति परप्रज्ञां नीतिशास्त्रानुसारिणीम्}
{विज्ञायेह तथा कुर्यादापदं निस्तरेद्यथा}


\twolineshloka
{अलोहं निशितं शस्त्रं शरीपरिकर्तनम्}
{यो वेत्ति न तु तं घ्नन्ति प्रतिघातविदं द्विषः}


\twolineshloka
{कक्षघ्नः शिशिरघ्नश्च महाकक्षे बिलौकसः}
{न दहेदिति चात्मानं यो रक्षति स जीवति}


\twolineshloka
{नाचक्षुर्वेत्ति पन्थानं नाचक्षुर्विन्दते दिशः}
{नाधृतिर्भूतिमाप्नोति बुध्यस्वैवं प्रबोधितः}


\twolineshloka
{अनाप्तैर्दत्तमादत्ते नरः शस्त्रमलोहजम्}
{श्वाविच्छरणमासाद्य प्रमुच्येत हुताशनात्}


\twolineshloka
{चरन्मार्गान्विजानाति नक्षत्रैर्विन्दते दिशः}
{आत्मना चात्मनः पञ्च पीडयन्नानुपीड्यते}


\twolineshloka
{एवमुक्तः प्रत्युवाच धर्मराजो युधिष्ठिरः}
{विदुरं विदुषां श्रेष्ठं ज्ञातमित्येव पाण्डवः}


\twolineshloka
{अनुशिक्ष्यानुगम्यैतान्कृत्वा चैव प्रदक्षिणम्}
{पाण्डवानभ्यनुज्ञाय विदुरः प्रययौ गृहान्}


\twolineshloka
{निवृत्ते विदुरे चापि भीष्मे पौरजने तथा}
{अजातशत्रुमासाद्य कुन्ती वचनमब्रवीत्}


\twolineshloka
{क्षत्ता यदब्रवीद्वाक्यं जनमध्येऽब्रुवन्निव}
{त्वया च स तथेत्युक्तो जानीमो न च तद्वयम्}


\threelineshloka
{यदीदं शक्यमस्माभिर्ज्ञातुं नैव च दोषवत्}
{श्रोतुमिच्छामि तत्सर्वं संवादं तव तस्य च ॥युधिष्ठिर उवाच}
{}


\threelineshloka
{विषादग्नेश्च बोद्धव्यमिति मां विदुरोऽब्रवीत्}
{पन्थानो वेदितव्याश्च नक्षत्रैश्च तथा दिशः}
{`कुड्याश्चविदिताःकार्याःस्याच्छुद्धिरितिचाब्रवीत्}


\threelineshloka
{जितेन्द्रियश्च वसुधां प्राप्स्यतीति च मेऽब्रवीत्}
{विज्ञातमिति तत्सर्वं प्रत्युक्तो विदुरो मया ॥वैशंपायन उवाच}
{}


\twolineshloka
{अष्टमेऽहनि रोहिण्यां प्रयाताः फाल्गुनस्य ते}
{वारणावतमासाद्य ददृशुर्नागरं जनम्}


\chapter{अध्यायः १५८}
\twolineshloka
{वैशंपायन उवाच}
{}


\twolineshloka
{ततः सर्वाः प्रकृतयो नगराद्वारणावतात्}
{सर्वमङ्गलसंयुक्ता यथाशास्त्रमतन्द्रिताः}


\twolineshloka
{श्रुत्वाऽगतान्पाण्डुपुत्रान्नानायानैः सहस्रशः}
{अभिजग्मुर्नरश्रेष्ठाञ्श्रुत्वैव परया मुदा}


\twolineshloka
{ते समासाद्य कौन्तेयान्वारणावतका जनाः}
{कृत्वा जयाशिषः सर्वे परिवार्योपतस्थिरे}


\twolineshloka
{तैर्वृतः पुरुषव्याघ्रो धर्मराजो युधिष्ठिरः}
{विबभौ देवसङ्काशो वज्रपाणिरिवामरैः}


\twolineshloka
{सत्कृताश्चैव पौरैस्ते पौरान्सत्कृत्य चानघ}
{अलङ्कृतं जनाकीर्णं विविशुर्वारणावतम्}


\twolineshloka
{ते प्रविश्य पुरीं वीरास्तूर्णं जग्मुरथो गृहान्}
{ब्राह्मणानां महीपाल रतानां स्वेषु कर्मसु}


\twolineshloka
{नगराधिकृतानां च गृहाणि रथिनां वराः}
{उपतस्थुर्नरश्रेष्ठा वैश्यशूद्रगृहाण्यपि}


\twolineshloka
{अर्चिताश्च नरैः पौरैः पाण्डवा भरतर्षभ}
{जग्मुरावसथं पश्चात्पुरोचनपुरःसराः}


\twolineshloka
{तेभ्यो भक्ष्याणि पानानि शयनानि शुभानि च}
{आसनानि च मुख्यानि प्रददौ स पुरोचनः}


\twolineshloka
{तत्र ते सत्कृतास्तेन सुमहार्हपरिच्छदाः}
{उपास्यमानाः पुरुषैरूषुः पुरनिवासिभिः}


\twolineshloka
{दशरात्रोषितानां तु तत्र तेषां पुरोचनः}
{निवेदयामास गृहं शिवाख्यमशिवं तदा}


\twolineshloka
{तत्र ते पुरुषव्याघ्रा विविशुः सपरिच्छदाः}
{पुरोचनस्य वचनात्कैलासमिव गुह्यकाः}


\twolineshloka
{तच्चागारमभिप्रेक्ष्य सर्वधर्मभृतां वरः}
{उवाचाग्नेयमित्येवं भीमसेनं युधिष्ठिरः}


\twolineshloka
{जिघ्राणोऽस्य वसागन्धं सर्पिर्जतुविमिश्रितम्}
{कृतं हि व्यक्तमाग्नेयमिदं वेश्म परन्तप}


\twolineshloka
{शणसर्जरसं व्यक्तमानीय गृहकर्मणि}
{मुञ्जबल्वजवंशादिद्रव्यं सर्वं घृतोक्षितम्}


\twolineshloka
{`तृणबल्वजकार्पासवंशदारुकटान्यपि}
{आग्नेयान्यत्र क्षिप्तानि परितो वेश्मनस्तथा ॥'}


\twolineshloka
{शिल्पिभिः सुकृतं ह्याप्तैर्विनीतैर्वेश्मकर्मणि}
{विश्वस्तं मामयं पापो दग्धुकामः पुरोचनः}


\twolineshloka
{तथा हि वर्तते मन्दः सुयोधनवशे स्थितः}
{इमां तु तां महाबुद्धिर्विदुरो दृष्टवांस्तथा}


\twolineshloka
{आपदं तेन मां पार्थ स संबोधितवान्पुरा}
{ते वयं बोधितास्तेन नित्यमस्मद्धितैषिणा}


\threelineshloka
{पित्रा कनीयसा स्नेहाद्बुद्धिमन्तो शिवं गृहम्}
{अनार्यैः सुकृतं गूढैर्दुर्योधनवशानुगैः ॥भीमसेन उवाच}
{}


\twolineshloka
{यदीदं गृहामाग्नेयं विहितं मन्यते भवान्}
{तत्रैव साधु गच्छामो यत्र पूर्वोषिता वयम्}


\twolineshloka
{`दर्शयित्वा पृथग्गन्तुं न कार्यं प्रतिभाति मे}
{अशुभं वा शुभं वा स्यात्तैर्वसाम सहैव तु}


\twolineshloka
{अद्यप्रभृति चास्मासु गतेषु भयविह्वलः}
{रूढमूलो भवेद्राज्ये धार्तराष्ट्रो जनेश्वरः}


\twolineshloka
{तदीयं तु भवेद्राज्यं तदीयाश्च जना इमे}
{तस्मात्सहैव वत्स्यामो गलन्यस्तपदा वयम्}


\twolineshloka
{अस्माकं कालमासाद्य राज्यमाच्छिद्य शत्रुतः}
{अर्थं पैतृकमस्माकं सुखं भोक्ष्याम शाश्वतम्}


\twolineshloka
{धृतराष्ट्रवचोऽस्माभिः किमर्थमनुपाल्यते}
{तेभ्यो भित्त्वाऽन्यथागन्तुं दौर्बल्यं ते कुतो नृप}


\twolineshloka
{आपत्सु रक्षिताऽस्माकं विदुरोऽस्ति महामतिः}
{मध्यस्थ एव गाङ्गेयो राज्यभोगपराङ्मुखः}


\twolineshloka
{बाह्लीकप्रमुखा वृद्धा मध्यस्था एव सर्वदा}
{अस्मदीयो भवेद्द्रोणः फल्गुनप्रेमसंयुतः}


\twolineshloka
{तस्मात्सहैव वस्तव्यं न गन्तव्यं कथं नृप}
{अथवास्मासु ते कुर्युः किमशक्ताः पराक्रमैः}


\twolineshloka
{क्षुद्राः कपटिनो धूर्ता जाग्रत्सु मनुजेश्वर}
{किं न कुर्युः पुरा मह्यं किं न दत्तं महाविषम्}


\twolineshloka
{आशीविषैर्महाघोरैः सर्पैस्तैः किं न दंशितः}
{प्रमाणकोट्यां संगृह्य निद्रापरवशे मयि}


\twolineshloka
{सर्पैर्दृष्टिविषैर्गोरैर्गङ्गायां शूलसन्ततौ}
{किं तैर्न पातितो भूप तदा किं मृतवानहम्}


\twolineshloka
{आपत्सु तासु घोरासु दुष्प्रयुक्तासु पापिभिः}
{अस्मानरक्षद्यो देवो जगद्यस्य वशे स्थितम्}


\twolineshloka
{चराचरात्मकं सोऽद्य यातः कुत्र नृपोत्तम}
{यावत्सोढव्यमस्माभिस्तावत्सोढास्मि यत्नतः}


\twolineshloka
{यदा न शक्ष्यतेऽस्माभिस्तदा पश्याम नो हितम्}
{किं द्रष्टव्यमिहास्माभिर्विगृह्य तरसा बलात्}


\twolineshloka
{सान्त्ववादेन दानेन भेदेनापि यतामहे}
{अर्धराज्यस्य संप्राप्त्यै ततो दण्डः प्रशस्यते}


\threelineshloka
{तस्मात्सहैव वस्तव्यं तन्मनोर्पितशल्यवत्}
{दर्शयित्वा पृथक् क्वापि न गन्तव्यं सुभीतवत् ॥'युधिष्ठिर उवाच}
{}


\twolineshloka
{इह यत्तैर्निराकारैर्वस्तव्यमिति रोचये}
{अप्रमत्तैर्विचिन्वद्भिर्गतिमिष्टां ध्रुवामितः}


\twolineshloka
{यदि विन्देत चाकारमस्माकं स पुरोचनः}
{क्षिप्रकारी ततो भूत्वा प्रसह्यापि दहेत्ततः}


\twolineshloka
{नायं बिभेत्युपक्रोशादधर्माद्वा पुरोचनः}
{तथा हि वर्तते मन्दः सुयोधनवशे स्थितः}


\twolineshloka
{अपि चायं प्रदग्धेषु भीष्मोऽस्मासु पितामहः}
{कोपं कुर्यात्किमर्थं वा कौरवान्कोपयीत सः}


\twolineshloka
{अथवापीह दग्धेषु भीष्मोऽस्माकं पितामहः}
{धर्म इत्येव कुप्येरन्ये चान्ये कुरुपुङ्गवाः}


\twolineshloka
{`उपपन्नं तु दग्धेषु कुलवंशानुकीर्तिताः}
{कुप्येरन्यदि धर्मज्ञास्तथान्ये कुरुपुङ्गवाः ॥'}


\twolineshloka
{वयं तु यदि दाहस्य बिभ्यतः प्रद्रवेमहि}
{स्पशैर्नो घातयेत्सर्वान्राज्यलुब्धः सुयोधनः}


\twolineshloka
{अपदस्थान्पदे तिष्ठन्नपक्षान्पक्षसंस्थितः}
{हीनकोशान्महाकोशः प्रयोगैर्घातयेद्ध्रुवम्}


\twolineshloka
{तदस्माभिरिमं पापं तं च पापं सुयोधनम्}
{वञ्चयद्भिर्निवस्तव्यं छन्नं वीर क्वचित्क्वचित्}


\twolineshloka
{ते वयं मृगयाशीलाश्चराम वसुधामिमाम्}
{तथा नो विदिता मार्गा भविष्यन्ति पलायतां}


\twolineshloka
{भौमं च बिलमद्यैव करवाम सुसंवृतम्}
{गूढोद्गतान्न नस्तत्र हुताशः संप्रधक्ष्यति}


\twolineshloka
{द्रवतोऽत्र यथा चास्मान्न बुध्येत पुरोचनः}
{पौरो वापि जनः कश्चित्तथा कार्यमतन्द्रितैः}


\chapter{अध्यायः १५९}
\twolineshloka
{वैशंपायन उवाच}
{}


\twolineshloka
{विदुरस्य सुहृत्कश्चित्खनकः कुशलः क्वचित्}
{विविक्ते पाण्डवान्राजन्निदं वचनमब्रवीत्}


\twolineshloka
{प्रहितो विदुरेणास्मि खनकः कुशलो ह्यहम्}
{पाण्डवानां प्रियं कार्यमिति किं करवाणि वः}


\twolineshloka
{प्रच्छन्नं विदुरेणोक्तं प्रियं यन्म्लेच्छभाषया}
{त्वया च तत्तथेत्युक्तमेतद्विश्वासकारणम्}


\twolineshloka
{कृष्णपक्षे चतुर्दश्यां रात्रावस्यां पुरोचनः}
{भवनस्य तव द्वारि प्रदास्यति हुताशनम्}


\threelineshloka
{मात्रा सह प्रदग्धव्याः पाण्डवाः पुरुषर्षभाः}
{इति व्यवसितं तस्य धार्तराष्ट्रस्य दुर्मतेः ॥वैशंपायन उवाच}
{}


\twolineshloka
{उवाच तं सत्यधृतिः कुन्तीपुत्रो युधिष्ठिरः}
{अभिजानामि सौम्य त्वां सुहृदं विदुरस्य वै}


\twolineshloka
{शुचिमाप्तं प्रियं चैव सदा च दृढभक्तिकम्}
{न विद्यते कवेः किंचिदविज्ञातं प्रयोजनम्}


\twolineshloka
{यथा तस्य तथा नस्त्वं निर्विशेषा वयं त्वयि}
{भवतश्च यथा तस्य पालयास्मान्यथा कविः}


\twolineshloka
{इदं शरणमाग्नेयं मदर्थमिति मे मतिः}
{पुरोचनेन विहितं धार्तराष्ट्रस्य शासनात्}


\twolineshloka
{स पापः कोशवांश्चैव ससहायश्च दुर्मतिः}
{अस्मानपि च पापात्मा नित्यकालं प्रबाधते}


\twolineshloka
{स भवान्भोक्षयत्वस्मान्यत्नेनास्माद्धुताशनात्}
{अस्मास्विह हि दग्धेषु सकामः स्यात्सुयोधनः}


\twolineshloka
{समृद्धमायुधागारमिदं तस्य दुरात्मनः}
{वप्रान्तं निष्प्रतीकारमाश्रित्येदं कृतं महत्}


\twolineshloka
{इदं तदशुभं नूनं तस्य कर्म चिकीर्षितम्}
{प्रागेव विदुरो वेद तेनास्मानन्वबोधयत्}


\threelineshloka
{सेयमापदनुप्राप्ता क्षत्ता यां दृष्टवान्पुरा}
{पुरोचनस्याविदितानस्मांस्त्वं प्रतिमोचय ॥वैशंपायन उवाच}
{}


\twolineshloka
{स तथेति प्रतिश्रुत्य खनको यत्नमास्थितः}
{परिखामुत्किरन्नाम चकार च महाबिलम्}


\twolineshloka
{चक्रे च वेश्मनस्तस्य मध्ये नातिमहद्बिलम्}
{कपाटयुक्तमज्ञातं समं भूम्याश्च भारत}


\threelineshloka
{पुरोचनभयादेव व्यदधात्संवृतं मुखम्}
{स तस्य तु गृहद्वारि वसत्यशुभधीः सदा}
{तत्र ते सायुधाः सर्वे वसन्ति स्म क्षपां नृप}


\twolineshloka
{दिवा चरन्ति मृगयां पाण्डवेया वनाद्वनम्}
{विश्वस्तवदविश्वस्ता वञ्चयन्तः पुरोचनम्}


% Check verse!
अतुष्टास्तुष्टवद्राजन्नूषुः परमविस्मिताः
\twolineshloka
{न चैनानन्वबुध्यन्त नरा नगरवासिनः}
{अन्यत्र विदुरामात्यात्तस्मात्खनकसत्तमात्}


\chapter{अध्यायः १६०}
\twolineshloka
{वैशंपायन उवाच}
{}


\twolineshloka
{तांस्तु दृष्ट्वा सुमनसः परिसंवत्सरोषितान्}
{विश्वस्तानिव संलक्ष्य हर्षं चक्रे पुरोचनः}


\twolineshloka
{`स तु संचिन्तयामास प्रहृष्टेनान्तरात्मना}
{प्राप्तकालमिदं मन्ये पाण्डवानां विनाशने}


\twolineshloka
{तमस्यान्तर्गतं भावं विज्ञाय कुरुपुङ्गवः}
{चिन्तयामास मतिमान्धर्मपुत्रो युधिष्ठिरः ॥'}


\twolineshloka
{पुरोचने तथा हृष्टे कौन्तेयोऽथ युधिष्ठिरः}
{भीमसेनार्जुनौ चोभौ यमौ प्रोवाच धर्मवित्}


\twolineshloka
{अस्मानयं सुविश्वस्तान्वेत्ति पापः पुरोचनः}
{वञ्चितोऽयं नृशंसात्मा कालं मन्ये पलायने}


\threelineshloka
{आयुधागारमादीप्य दग्ध्वा चैव पुरोचनम्}
{षट्प्राणिनो निधायेह द्रवामोऽनभिलक्षिताः ॥वैशंपायन उवाच}
{}


\twolineshloka
{अथ दानापदेशेन कुन्ती ब्राह्मणभोजनम्}
{चक्रे निशि महाराज आजग्मुस्तत्र योषितः}


\twolineshloka
{ता विहृत्य यथाकामं भुक्त्वा पीत्वा च भारत}
{जग्मुर्निशिं गृहानेव समनुज्ञाप्य माधवीम्}


\twolineshloka
{`पुरोचनप्रणिहिता पृथां या सेवते सदा}
{निषादी दुष्टहृदया नित्यमन्तरचारिणी}


\twolineshloka
{निषादी पञ्चपुत्रा सा तस्मिन्भोज्ये यदृच्छया}
{पुराभ्यासकृतस्नेहा सखी कुन्त्याः सुतैः सह}


\threelineshloka
{आनीय मधुमूलानि फलानि विविधानि च}
{अन्नार्थिनी समभ्यागात्सपुत्रा कालचोदिता}
{सुपापा पञ्चपुत्रा च सा पृथायाः सखी मता ॥'}


\twolineshloka
{सा पीत्वा मदिरां मत्ता सपुत्रा मदविह्वला}
{सह सर्वैः सुतै राजंस्तस्मिन्नेव निवेशने}


\twolineshloka
{सुष्वाप विगतज्ञाना मृतकल्पा नराधिप}
{अथ प्रवाते तुमुले निशि सुप्ते जने तदा}


\twolineshloka
{तदुपादीपयद्भीमः शेते यत्र पुरोचनः}
{ततो जतुगृहद्वारं दीपयामास पाण्डवः}


\twolineshloka
{समन्ततो ददौ पश्चादग्निं तत्र निवेशने}
{`पूर्वमेव गृहं शोध्य भीमसेनो महामतिः}


\twolineshloka
{पाण्डवैः सहितां कुन्तीं प्रावेशयत तद्बिलम्}
{दत्त्वाग्निं सहसा भीमो गृहे तत्परितः सुधीः}


\twolineshloka
{गृहस्थं द्रविणं गृह्य निर्जगाम बिलेन सः}
{'ज्ञात्वा तु तद्गृहं सर्वमादीप्तं पाण्डुनन्दनाः}


\twolineshloka
{सुरङ्गां विविशुस्तूर्णं मात्रा सार्धमरिन्दमाः}
{ततः प्रतापः सुमहाञ्छब्दश्चैव विभावसोः}


\twolineshloka
{प्रादुरासीत्तदा तेन बुबुधे स जनव्रजः}
{तदवेक्ष्य गृहं दीप्तमाहुः पौराः कृशाननाः}


\twolineshloka
{दुर्योधनप्रयुक्तेन पापेनाकृतबुद्धिना}
{गृहमात्मविनाशाय कारितं दाहितं च तत्}


\twolineshloka
{अहो धिग्धृतराष्ट्रस्य बुद्धिर्नातिसमञ्जसा}
{यः शुचीन्पाण्डुदायादान्दाहयामास शत्रुवत्}


\threelineshloka
{दिष्ट्या त्विदानीं पापात्मादग्ध्वा दग्धः पुरोचनः}
{अनागसः सुविश्वस्तान्यो ददाह नरोत्तमान् ॥वैशंपायन उवाच}
{}


\twolineshloka
{एवं ते विलपन्ति स्म वारणावतका जनाः}
{परिवार्य गृहं तच्च तस्थू रात्रौ समन्ततः}


\twolineshloka
{पाण्डवाश्चापि ते सर्वे सह मात्रा सुदुःखिताः}
{बिलेन तेन निर्गत्य जग्मुर्द्रुतमलक्षिताः}


\twolineshloka
{तेन निद्रोपरोधेन साध्वसेन च पाण्डवाः}
{न शेकुः सहसा गन्तुं सह मात्रा परन्तपाः}


\twolineshloka
{भीमसेनस्तु राजेन्द्र भीमवेगपराक्रमः}
{जगाम भ्रातॄनादाय सर्वान्मातरमेव च}


\twolineshloka
{स्कन्धमारोप्य जननीं यमावङ्केन वीर्यवान्}
{पार्थौ गृहीत्वा पाणिभ्यां भ्रातरौ सुमहाबलः}


\twolineshloka
{उरसा पादपान्भञ्जन्महीं पद्भ्यां विदारयन्}
{स जगामाशु तेजस्वी वातरंहा वृकोदरः}


\chapter{अध्यायः १६१}
\twolineshloka
{वैशंपायन उवाच}
{}


\twolineshloka
{एतस्मिन्नेव काले तु यथासंप्रत्ययं कविः}
{विदुरः प्रेषयामास तद्वनं पुरुषं शुचिम्}


\twolineshloka
{`आत्मनः पाण्डवानां च विश्वास्यं ज्ञातपूर्वकम्}
{गङ्गासंतरणार्थाय ज्ञाताभिज्ञानवाचिकम् ॥'}


\twolineshloka
{स गत्वा तु यथोद्देशं पाण्डवान्ददृशे वने}
{जनन्या सह कौरव्याननयज्जाह्नवीतटम्}


\twolineshloka
{विदितं तन्महाबुद्धेर्विदुरस्य महात्मनः}
{ततस्त्रस्यापि चारेण चेष्टितं पापचेतसः}


\twolineshloka
{ततः प्रवासितो विद्वान्विदुरेण नरस्तदा}
{पार्थानां दर्शयामास मनोमारुतगामिनीम्}


\twolineshloka
{सर्ववातसहां नावं यन्त्रयुक्तां पताकिनीम्}
{शिवे भागीरथीतीरे नरैर्विस्रम्भिभिः कृताम्}


\twolineshloka
{ततः पुनरथोवाच ज्ञापकं पूर्वचोदितम्}
{युधिष्ठिर निबोधेयं संज्ञार्थं वचनं कवेः}


\threelineshloka
{कक्षघ्नः शिशिरघ्नश्च महाकक्षे बिलौकसः}
{न हन्तीत्येवमात्मानं यो रक्षति स जीवति}
{`बोद्धव्यमिति यत्प्राह विदुरस्तदिदं तथा ॥'}


\threelineshloka
{तेन मां प्रेषितं विदिधि विश्वस्तं संज्ञयाऽनया}
{भूयश्चैवाह मां क्षत्ता विदुरः सर्वतोऽर्थवित्}
{`अधिक्षिपन्धार्तराष्ट्रं सभ्रातृकमुदारधीः ॥'}


\threelineshloka
{कर्णं दुर्योधनं चैव भ्रातृभिः सहितं रणे}
{शकुनिं चैव कौन्तेय विजेताऽसि न संशयः ॥`वैशंपायन उवाच}
{}


\twolineshloka
{पाण्डवाश्चापि गत्वाथ गङ्गायास्तीरमुत्तमम्}
{निषादाधिपतिं वीरं दाशं परमधार्मिकम्}


\twolineshloka
{दीपिकाभिः कृतालोकं मार्गमाणं च पाण्डवान्}
{ददृशुः पाण्डवेयास्ते नाविकं त्वरयाऽन्वितम्}


\threelineshloka
{निषादस्तत्र कौन्तेयानभिज्ञानं न्यवेदयत्}
{विदुरस्य महाबुद्धेर्म्लेच्छभाषादि यत्तदा ॥नाविक उवाच}
{}


\twolineshloka
{विदुरेणास्मि सन्दिष्टो दत्त्वा बहु धनं महत्}
{गङ्गातीरे निविष्टस्त्वं पाण्डवांस्तारयेति ह}


\twolineshloka
{सोऽहं चतुर्दशीमद्य गङ्गाया अविदूरतः}
{चारेरन्वेषयाम्यस्मिन्वने मृगगणान्विते}


\twolineshloka
{प्रभवन्तोऽथ भद्रं वो नावमारुह्य गम्यताम्}
{युक्तारित्रपताकां च निश्छिद्रां मन्दिरोपमाम् ॥'}


\threelineshloka
{इयं वारिपथे युक्ता नौरप्सु सुखगामिनी}
{मोचयिष्यति वः सर्वानस्माद्देशान्न संशयः ॥वैशंपायन उवाच}
{}


\twolineshloka
{अथ तान्व्यथितान्दृष्ट्वा सह मात्रा नरोत्तमान्}
{नावमारोप्य गङ्गायां प्रस्थितानब्रवीत्पुनः}


\twolineshloka
{विदुरो मूर्ध्न्युपाघ्राय परिष्वज्य वचो मुहुः}
{अरिष्टं गच्छताव्यग्राः पन्थानमिति चाब्रवीत्}


\twolineshloka
{इत्युक्त्वा स तु तान्वीरान्पुमान्विदुरचोदितः}
{तारयामास राजेन्द्र गङ्गां नावा नरर्षभान्}


\twolineshloka
{तारयित्वा ततो गङ्गां पारं प्राप्तांश्च सर्वशः}
{जयाशिषः प्रयुज्याथ यथागतमगाद्धि सः}


\twolineshloka
{पाण्डवाश्च महात्मानः प्रतिसन्दिश्य वै कवेः}
{गङ्गामुत्तीर्य वेगेन जग्मुर्गूढमलक्षिताः}


\twolineshloka
{`ततस्ते तत्र तीर्त्वा तु गङ्गामुत्तुङ्गवीचिकाम्}
{जवेन प्रययुर्वीरा दक्षिणां दिशमास्थिताः}


\twolineshloka
{विज्ञाय निशि पन्थानं नक्षत्रैर्दक्षिणामुखाः}
{वनाद्वनान्तरं राजन्गहनं प्रतिपेदिरे}


\twolineshloka
{श्रान्तास्ततः पिपासार्ताः क्षुधिता भयकातराः}
{पुनरूचुर्महावीर्यं भीमसेनमिदं वचः}


\threelineshloka
{इतः कष्टतरं किं नु यद्वयं गहने वने}
{दिशश्च न प्रजानीमो गन्तुं चैतेन शक्नुमः}
{तं च पापं न जानीमो दग्धो वाथ पुरोचनः}


\twolineshloka
{कथं नु विप्रमुच्येम भयादस्मादलक्षिताः}
{शीघ्रमस्मानुपादाय तथैव व्रज भारत}


\threelineshloka
{त्वं हि नो बलवानेको यथा सततगस्तथा}
{इत्युक्तो धर्मराजेन भीमसेनो महाबलः}
{आदाय कुन्तीं भ्रातॄंश्च जगामाशु स पावनिः'}


\chapter{अध्यायः १६२}
\twolineshloka
{वैशंपायन उवाच}
{}


\twolineshloka
{अथ रात्र्यां व्यतीतायामशेषो नागरो जनः}
{तत्राजगाम त्वरितो दिदृक्षुः पाण्डुनन्दनान्}


\twolineshloka
{निर्वापयन्तो ज्वलनं ते जना ददृशुस्ततः}
{जातुषं तद्गृहं दग्धममात्यं च पुरोचनम्}


\twolineshloka
{नूनं दुर्योधनेनेदं विहितं पापकर्मणा}
{पाण्डवानां विनाशायेत्येवं ते चुक्रुशुर्जनाः}


\twolineshloka
{विदिते धृतराष्ट्रस्य धार्तराष्ट्रो न संशयः}
{दग्धवान्पाण्डुदायादान्न ह्येतत्प्रतिषिद्धवान्}


\twolineshloka
{नूनं शान्तनवोऽपीह न धर्मनुवर्तते}
{द्रोणश्च विदुरश्चैव कृपश्चान्ये च कौरवाः}


\twolineshloka
{`नावेक्षन्ते ह तं धर्मं धर्मात्मानोऽप्यहो विधेः}
{श्रुतवन्तोऽपि विद्वांसो धनवद्वशगा अहो}


\twolineshloka
{साधूननाथान्धर्मिष्ठात्सत्यव्रतपरायणान्}
{नावेक्षन्ते महान्तोऽपि दैवं तेषां परायणम्}


\twolineshloka
{ते वयं धृतराष्ट्राय प्रेषयामो दुरात्मने}
{संवृत्तस्ते परः कामः पाण्डवान्दग्धवानसि}


\twolineshloka
{ततो व्यपोहमानास्ते पाण्डवार्थे हुताशनम्}
{निषादीं ददृशुर्दग्धां पञ्चपुत्रामनागसम्}


\twolineshloka
{इतः पश्यत कुन्तीयं दग्धा शेते तपस्विनी}
{पुत्रैः सहैव वार्ष्णेयी हन्तेत्याहुः स्म नागराः}


\twolineshloka
{खनकेन तु तेनैव वेश्म शोधयता बिलम्}
{पांसुभिः पिहितं तच्च पुरुषैस्तैर्न लक्षितम्}


\twolineshloka
{ततस्ते प्रेषयामासुर्धृतराष्ट्राय नागराः}
{पाण्डवानग्निना दग्धानमात्यं च पुरोचनम्}


\twolineshloka
{श्रुत्वा तु धृतराष्ट्रस्तद्राजा सुमहदप्रियम्}
{विनाशं पाण्डुपुत्राणां विललाप सुदुःखितः}


\threelineshloka
{अन्तर्हृष्टमनाश्चासौ बहिर्दुःखसमन्वितः}
{अन्तःशीतो बहिश्चोष्णो ग्रीष्मेऽगाधह्वदोयथा ॥धृतराष्ट्र उवाच}
{}


\twolineshloka
{अद्य पाण्डुर्मृतो राजा मम भ्राता महायशाः}
{तेषु वीरेषु दग्धेषु मात्रा सह विशेषतः}


\twolineshloka
{गच्छन्तु पुरुषाः शीघ्रं नगरं वारणावतम्}
{सत्कारयन्तु तान्वीरान्कुन्तीं राजसुतां च ताम्}


\twolineshloka
{ये च तत्र मृतास्तेषां सुहृदः सन्ति तानपि}
{कारयन्तु च कुल्यानि शुभ्राणि च बृहन्ति च}


% Check verse!
मम दग्धा महात्मानः कुलवंशविवर्धनाः
\threelineshloka
{एवं गते मया शक्यं यद्यत्कारयितुं हितम्}
{पाण्डवानां च कुन्त्याश्च तत्सर्वं क्रियतां धनैः ॥`वैशंपायन उवाच}
{}


\twolineshloka
{समेताश्च ततः सर्वे भीष्मेण सह कौरवाः}
{धृतराष्ट्रः सपुत्रश्च गङ्गामभिमुखा ययुः}


\twolineshloka
{एकवस्त्रा निरानन्दा निराभरणवेष्टनः}
{उदकं कर्तुकामा वै पाण्डवानां महात्मनाम् ॥'}


\twolineshloka
{एवं गत्वा ततश्चक्रे ज्ञातिभिः परिवारितः}
{उदकं पाण्डुपुत्राणां धृतराष्ट्रोऽम्बिकासुतः}


\twolineshloka
{रुरुदुः सहिताः सर्वे भृशं शोकपरायणाः}
{हा युधिष्ठिर कौरव्य हा भीम इति चापरे}


\twolineshloka
{हा फल्गुनेति चाप्यन्ये हा यमाविति चापरे}
{कुन्तीमार्ताश्च शोचन्त उदकं चक्रिरे जनाः}


\twolineshloka
{अन्ये पौरजनाश्चैवमन्वशोचन्त पाण्डवान्}
{विदुरस्त्वल्पशश्चक्रे शोकं वेद परं हि सः}


\twolineshloka
{विदुरो धृतराष्ट्रस्य जानन्सर्वं मनोगतम्}
{तेनायं विधिना सृष्टः कुटिलः कपटाशयः}


\twolineshloka
{इत्येवं चिन्तयन्राजन्विदुरो विदुषां वरः}
{लोकानां दर्शयन्दुःखं दुःखितैः सह बान्धवैः}


\twolineshloka
{मनसाऽचिन्तयत्पार्थान्कियद्दूरं गता इति}
{सहिताः पाण्डवाः पुत्रा इति चिन्तापरोऽभवत्}


\threelineshloka
{ततः प्रव्यथितो भीष्मः पाण्डुराजसुतान्मृतान्}
{सह मात्रेति तच्छ्रुत्वा विललाप रुरोद च ॥भीष्म उवाच}
{}


\twolineshloka
{हा युधिष्ठिर हा भीम हा धनञ्जय हा यमौ}
{हा पृथे सह पुत्रैस्त्वमेकरात्रेण स्वर्गता}


\twolineshloka
{मात्रा सह कुमारास्ते सर्वे तत्रैव संस्थिताः}
{न हि तौ नोत्सहेयातां भीमसेनधनञ्जयौ}


\twolineshloka
{तरसा वेगितात्मानौ निर्भेत्तुमपि मन्दरम्}
{परासुत्वं न पश्यामि पृथायाः सह पाण्डवैः}


\twolineshloka
{सर्वथा विकृतं तत्तु यदि ते निधनं गताः}
{धर्मराजः स निर्दिष्टो ननु विप्रैर्युधिष्ठिरः}


\twolineshloka
{पृथिव्यां च रथिश्रेष्ठो भविता स धनञ्जयः}
{सत्यव्रतो धर्मदत्तः सत्यवाक्छुभलक्षणः}


\twolineshloka
{कथं कालवशं प्राप्तः पाण्डवेयो युधिष्ठिरः}
{आत्मानमुपमां कृत्वा परेषां वर्तते तु यः}


\twolineshloka
{मात्रा सहैव कौरव्यः कथं कालवशं गतः}
{पालितः सुचिरं कालं फलकाले यथा द्रुमः}


\twolineshloka
{भग्नः स्याद्वायुवेगेन तथा राजा युधिष्ठिरः}
{यौवराज्येऽभिषिक्तेन पितुर्येनाहृतं यशः}


\twolineshloka
{आत्मनश्च पितुश्चैव सत्यधर्मप्रवृत्तिभिः}
{यच्च सा वनवासेन तन्माता दुःखभागिनी}


\twolineshloka
{कालेन सह संमग्नो धिक्कृतान्तमनर्थकम्}
{यच्च सा वनवासेन तन्माता दुःखभागिनी}


\twolineshloka
{पुत्रगृध्नुतया कुन्ती न भर्तारं मृतात्वनु}
{अल्पकालं कुले जाता भर्तुः प्रीतिमवाप या}


\twolineshloka
{दग्धाऽद्य सह पुत्रैः सा असंपूर्णमनोरा}
{मृतो भीम इति श्रुत्वा मनो न श्रद्दधाति मे}


\twolineshloka
{एतच्च चिन्तयानस्य व्यथितं बहुधा मनः}
{अवधूय च मे देहं हृदयेन विदीर्यते}


\twolineshloka
{पीनस्कन्धश्चारुबाहुर्मेरुकूटसमो युवा}
{मृतो भीम इति श्रुत्वा मनो न श्रद्दधाति मे}


\twolineshloka
{अतित्यागी च योधी च क्षिप्रहस्तो दृढायुधः}
{प्रपत्तिमाँल्लब्धलक्षो रथयानविशारदः}


\twolineshloka
{दूरपाती त्वसंभ्रान्तो महावीर्यो महास्त्रवान्}
{अदीनात्मा नरश्रेष्ठः श्रेष्ठः सर्वधनुष्मताम्}


\twolineshloka
{येन प्राच्याश्च सौवीरा दाक्षिणात्याश्च निर्जिताः}
{ख्यापितं येन शूरेण त्रिषु लोकेषु पौरुषम्}


\twolineshloka
{यस्मिञ्जाते विशोकाऽभूत्कुन्ती पाण्डुश्च वीर्यवान्}
{पुरन्दरसमो जिष्णुः कथं कालवशं गतः}


\twolineshloka
{कथं तावृषभष्कन्धौ सिंहविक्रान्तगामिनौ}
{मर्त्यधर्ममनुप्राप्तौ यमावरिनिवर्हणौ}


\twolineshloka
{वत्सा गताः क्व मां वृद्धं विहाय भृशमातुरम्}
{हा स्नुषे मम वार्ष्णेयि निधाय हृदि मे शुचम्}


\twolineshloka
{वारणावतयात्रायां के स्युर्वै शकुनाः पथि}
{एवमल्पायुषो लोके भविष्यन्ति पृथासुताः}


\twolineshloka
{संशप्ता इति कैर्यूयं वत्सान्दर्शय मे पृथे}
{ममैव नाथा मन्नाथा मम नेत्राणि पाण्डवाः}


\twolineshloka
{हा पाण्डवा मे हे वत्सा हा सिंहशिशवो मम}
{मातङ्गा हा ममोत्तुङ्गा हा ममानन्दवर्धनाः}


\twolineshloka
{मम हीनस्य युष्माभिः सर्वलोकास्तमोवृताः}
{कदा द्रष्टाऽस्मि कौन्तेयांस्तरुणादित्यवर्चसः}


\twolineshloka
{अदृष्ट्वा वो महाबाहून्पुत्रवन्मम नन्दनाः}
{क्व गतिर्मे क्व गच्छामि कुतो द्रक्ष्यामि मे शिशून्}


\threelineshloka
{हा युधिष्ठिर हा भीम हा हा फल्गुन हा यमौ}
{मा गच्छत निवर्तध्वं मयि कोपं विमुञ्चत ॥वैशंपायन उवाच}
{}


\twolineshloka
{श्रुत्वा तत्क्रन्दितं तस्य तिलोदं च प्रसिञ्चतः}
{देशं कालं समाज्ञाय विदुरः प्रत्यभाषत}


\twolineshloka
{मा शोचीस्त्वं नरव्याघ्र जहि शोकं महाधृते}
{न तेषां विद्यते मृत्युः प्राप्तकालं कृतं मया}


\twolineshloka
{एतच्च तेभ्य उदकं विप्रसिञ्च न भारत}
{क्षत्तेदमब्रवीद्भीष्मं कौरवाणामशृण्वताम्}


\threelineshloka
{क्षत्तारमुपसंगम्य बाष्पोत्पीडकलस्वरः}
{मन्दंमन्दमुवाचेदं विदुरं संगमे नृप ॥भीष्म उवाच}
{}


\twolineshloka
{कथं ते तात जीवन्ति पाण्डोः पुत्रा महाबलाः}
{कथमस्मत्कृते पक्षः पाण्डोर्न हि निपातितः}


\threelineshloka
{कथं मत्प्रमुखाः सर्वे प्रमुक्ता महतो भयात्}
{जननी गरुडेनेव कुरवस्ते समुद्धृताः ॥वैशंपायन उवाच}
{}


\threelineshloka
{एवमुक्तस्तु कौरव्य कौरवाणामशृण्वताम्}
{आचचक्षे स धर्मात्मा भीष्मायाद्भुतकर्मणे ॥विदुर उवाच}
{}


\twolineshloka
{धृतराष्ट्रस्य शकुने राज्ञो दुर्योधनस्य च}
{विनाशे पाण्डुपुत्राणां कृतो मतिविनिश्चयः}


\twolineshloka
{तत्राहमपि च ज्ञात्वा तस्य पापस्य निश्चयम्}
{तं जिघांसुरहं चापि तेषामनुमते स्थितः}


\twolineshloka
{ततो जतुगृहं गत्वा दहनेऽस्मिन्नियोजिते}
{पृथायाश्च सपुत्राया धार्तराष्ट्रस्य शासनात्}


\twolineshloka
{ततः खनकमाहूय सुरङ्गं वै बिलं तदा}
{सुगूढं कारयित्वा ते कुन्त्या पाण्डुसुतास्तदा}


\twolineshloka
{निष्क्रामिता मया पूर्वं मा स्म शोके मनः कृथाः}
{ततस्तु नावमारोप्य सहपुत्रां पृथामहम्}


\twolineshloka
{दत्त्वाऽभयं सपुत्रायै कुन्त्यै गृहमदाहयम्}
{तस्मात्ते मा स्म भूद्दुःखं मुक्ताः पापात्तु पाण्डवाः}


\twolineshloka
{निर्गताः पाण्डवा राजन्मात्रा सह परन्तपाः}
{अग्निहादान्महाघोरान्मया तस्मादुपायतः}


\twolineshloka
{मा स्म शोकमिमं कार्षीर्जीवन्त्येव च पाण्डवाः}
{प्रच्छन्ना विचिरिष्यन्ति यावत्कालस्य पर्ययः}


\twolineshloka
{तस्मिन्युधिष्ठिरं काले द्रक्ष्यन्ति भुवि मानवाः}
{विमलं कृष्णपक्षान्ते जगच्चन्द्रमिवोदितम्}


\threelineshloka
{न तस्य नाशं पश्यामि यस्य भ्राता धनञ्जयः}
{भीमसेनश्च दुर्धर्षौ माद्रीपुत्रौ च तौ यमौ ॥वैशंपायन उवाच}
{}


\threelineshloka
{ततः संहृष्टसर्वाङ्गो भीष्मो विदुरमब्रवीत्}
{दिष्ट्यादिष्ट्येति संहृष्टः पूजयानो महामतिम् ॥भीष्म उवाच}
{}


\threelineshloka
{युक्तं चैवानुरूपं च कृतं तात शुभं त्वया}
{वयं विमोक्षिता दुःखात्पाण्डुपक्षो न नाशितः ॥वैशंपायन उवाच}
{}


\twolineshloka
{एवमुक्त्वा विवेशाथ पुरं जनशताकुलम्}
{कुरुभिः सहितो राजन्नागरैश्च पितामहः}


\twolineshloka
{अथाम्बिकेयः सामात्यः सकर्णः सहसौबलः}
{सात्मजः पार्थनाशस्य स्मरंस्तथ्यं जर्ष च}


\twolineshloka
{भीष्मश्च राजन्दुर्धर्षो विदुरश्च महामतिः}
{जहृषाते स्मरन्तौ तौ जातुषाग्नेर्विमोचनम्}


\twolineshloka
{सत्यशीलगुणाचारै रागैर्जानपदोद्भवैः}
{द्रोणादयश्च धर्मैस्तु तेषां तान्मोचितान्विदुः}


\twolineshloka
{शौर्यलावण्यमाहात्म्यै रूपैः प्राणबलैरपि}
{स्वस्थान्पार्थानमन्यन्त पौरजानपदास्तथा}


\twolineshloka
{अन्ये जनाः प्राकृताश्च स्त्रियश्च बहुलास्तदा}
{शङ्कमाना वदन्ति स्म दग्धा जीवन्ति वा न वा}


\chapter{अध्यायः १६३}
\twolineshloka
{वैशंपायन उवाच}
{}


\twolineshloka
{तेन विक्रममाणेन ऊरुवेगसमीरितम्}
{वनं सवृक्षविटपं व्याघूर्णितमिवाभवत्}


\twolineshloka
{जङ्गावातो ववौ चास्य शुचिशुक्रागमे यथा}
{आवर्जितालतावृक्षं मार्गं चक्रे महाबलः}


\twolineshloka
{स मृद्गन्पुष्पितांश्चैव फलितांश्च वनस्पतीन्}
{अवरुज्य ययौ गुल्मान्पथस्तस्य समीपजान्}


\twolineshloka
{स रोषित इव क्रुद्धो वने भञ्जन्महाद्रुमान्}
{त्रिप्रस्रुतमदः शुष्मी षष्टिवर्षो मतङ्गराट्}


\twolineshloka
{गच्छतस्तस्य वेगेन तार्क्ष्यमारुतरंहसः}
{भीमस्य पाण्डुपुत्राणां मूर्च्छेव समजायत}


\twolineshloka
{असकृच्चापि सतीर्य दूरपारं भुजप्लवैः}
{पथि प्रच्छन्नमासेदुर्धार्तराष्ट्रभयात्तदा}


\twolineshloka
{कृच्छ्रेण मातरं चैव सुकुमारीं यशस्विनीम्}
{अवहत्स तु पृष्ठेन रोधःसु विषमेषु च}


\twolineshloka
{अगमच्च वनोद्देशमल्पमूलफलोदकम्}
{क्रूरपक्षिमृगं घोरं सायाह्ने भरतर्षभ}


\twolineshloka
{घोरा समभवत्सन्ध्या दारुणा मृगपक्षिणः}
{अप्रकाशा दिशः सर्वा वातैरासन्ननार्तवैः}


\twolineshloka
{शीर्णपर्णफलै राजन्बहुगुल्मक्षुपद्रुमैः}
{भग्नावभुग्नभूयिष्ठैर्नानाद्रुमसमाकुलैः}


\twolineshloka
{ते श्रमेण च कौरव्यास्तृष्णया च प्रपीडिताः}
{नाशक्नुवंस्तदा गन्तुं निद्रया च प्रवृद्धया}


\threelineshloka
{न्यविशन्ति हि ते सर्वे निरास्वादे महावने}
{`रात्र्यामेव गतास्तूर्णं चतुर्विंशतियोजनम्}
{'ततस्तृषा परिक्लन्ता कुन्ती वचनमब्रवीत्}


\twolineshloka
{माता सती पाण्डवानां पञ्चानां मध्यतः स्थिता}
{तृष्णया हि परीताऽहमनाथेव महावने}


\twolineshloka
{`इतः परं तु शक्ताहं गन्तुं च न पदात्पदम्}
{शयिष्ये वृक्षमूलेऽत्र धार्तराष्ट्रा हरन्तु माम्}


\twolineshloka
{शृणु भीम वचो मह्यं तव बाहुबलात्पुरः}
{स्थातुं न शक्ताः कौरव्याः किं बिभेषि पृथासुत}


\threelineshloka
{अन्यो रथो न मेऽस्तीह भीमसेनादृते भुवि}
{धार्तराष्ट्राद्वृथा भीरुर्न मां स्वप्तुमिहेच्छसि ॥वैशंपायन उवाच}
{}


\twolineshloka
{भीमपृष्ठस्थिता चेत्थं दूयमानेन चेतसा}
{निश्यध्वनि रुदन्ती सा निद्रावश्मुपागता ॥'}


\twolineshloka
{तच्छ्रुत्वा भीमसेनस्य मातृस्नेहात्प्रजल्पितम्}
{कारुण्येन मनस्तप्तं गमनायोपचक्रमे}


\twolineshloka
{ततो भीमो वनं घोरं प्रविश्य विजनं महत्}
{न्यग्रोधं विपुलच्छायं रमणीयं ददर्श ह}


\twolineshloka
{तत्र निक्षिप्य तान्सर्वानुवाच भरतर्षभः}
{पानीयं मृगयामीह तावद्विश्रम्यतामिह}


\twolineshloka
{एते रुवन्ति मधुरं सारसा जलचारिणः}
{ध्रुवमत्र जलस्थानं महच्चेति मतिर्मम}


\twolineshloka
{अनुज्ञातः स गच्छेति भ्रात्रा ज्येष्ठेन भारत}
{जगाम तत्र यत्र स्म सारसा जलचारिणः}


\twolineshloka
{`अपश्यत्पद्मिनीखण्डमण्डितं स सरोवरम्}
{'स तत्र पीत्वा पानीयं स्नात्वा च भरतर्षभ}


\twolineshloka
{तेषामर्थे च जग्राह भ्रातॄणां भ्रातृवत्सलः}
{उत्तरीयेण पानीयमानयामास भारत}


\threelineshloka
{`पङ्कजानामनेकैश्च पत्रैर्बध्वा पृथक्पृथक्}
{'गव्यूतिमात्रादागत्य त्वरितो मातरं प्रति}
{शोकदुःखपरीतात्मा निशश्वासोरगो यथा}


\twolineshloka
{स सुप्तां मातरं `भ्रातॄन्निद्राविद्रावितौजसः}
{महारौद्रे वने घोरे वृक्षमूले सुशीतले}


\twolineshloka
{विक्षिप्तकरपादांश्च दीर्घोच्छ्वासान्महाबलान्}
{ऊर्ध्ववक्त्रान्महाकायान्पञ्चेन्द्रानिव भूपते}


\twolineshloka
{अज्ञातवृक्षनित्यस्थप्रेतराक्षससाध्वसान्}
{दृष्ट्वा वै भृशशोकार्तो बिललापानिलात्मजः ॥'}


% Check verse!
भृशं शोकपरीत्मा विललाप वृकोदरः
\twolineshloka
{अतः कष्टतरं किं नु द्रष्टव्यं हि भविष्यति}
{यत्पश्यामि महीसुप्तान्भ्रातॄनद्य सुमन्दभाक्}


\twolineshloka
{शयनेषु परार्ध्येषु ये पुरा वारणावते}
{नाधिजग्मुस्तदा निद्रां तेऽद्य सुप्ता महीतले}


\twolineshloka
{स्वसारं वसुदेवस्य शत्रुसङ्घावमर्दिनः}
{कुन्तिराजसुतां कुन्तीं सर्वलक्षणपूजिताम्}


\twolineshloka
{स्नुषां विचित्रवीर्यस्य भार्यां पाण्डोर्महात्मनः}
{तथैव चास्मज्जननीं पुण्डरीकोदरप्रभाम्}


\twolineshloka
{सुकुमारतरामेनां महार्हशयनोचिताम्}
{शयानां पश्यताऽद्येह पृथिव्यामतथोचिताम्}


\twolineshloka
{धर्मादिन्द्राच्च वाताच्च सुपुवे या सुतानिमान्}
{सेयं भूमौ परिश्रान्ता शेषे प्रासादशायिनी}


\twolineshloka
{किं नु दुःखतरं शक्यं मया द्रष्टुमतः परम्}
{योऽहमद्य नरव्याघ्रान्सुप्तान्पश्यामि भूतले}


\twolineshloka
{त्रिषु लोकेषु यो राज्यं धर्मनित्योऽर्हते नृपः}
{सोऽयं भूमौ परिश्रान्तः शेते प्राकृतवत्कथम्}


\twolineshloka
{अयं नीलाम्बुदश्यामो नरेष्वप्रतिमोऽर्जुनः}
{शेते प्राकृतवद्भूमौ ततो दुःखतरं नु किम्}


\twolineshloka
{अश्विनाविव देवानां याविमौ रूपसंपदा}
{तौ प्राकृतवदद्येमौ प्रसुप्तौ धरणीतले}


\twolineshloka
{ज्ञातयो यस्य नै स्युर्विषमाः कुलपांसनाः}
{स जीवेत सुखं लोके ग्रामद्रुम इवैकजः}


\twolineshloka
{एको वृक्षो हि यो ग्रामे भवेत्पर्णफलान्वितः}
{चैत्यो भवति निर्ज्ञातिरध्वनीनैश्च पूजितः}


\twolineshloka
{येषां च बहवः शूरा ज्ञातयो धर्ममाश्रिताः}
{ते जीवन्ति सुखं लोके भवन्ति च निरामयाः}


\twolineshloka
{बलवन्तः समृद्धार्था मित्रबान्धवनन्दनाः}
{जीवन्त्यन्योन्यमाश्रित्य द्रुमाः काननजा इव}


\twolineshloka
{वयं तु धृतराष्ट्रेण दुष्पुत्रेण दुरात्मना}
{`राज्यलुब्धेन मूर्खेण दुर्मन्त्रिसहितेन वै}


\twolineshloka
{दुष्टेनाधर्मशीलेन स्वार्थनिष्ठैकबुद्धिना}
{'विवासिता न दग्धाश्च क्षत्तुर्बुद्धिपराक्रमात्}


\twolineshloka
{तस्मान्मुक्ता वयं दाहादिमं वृक्षमुपाश्रिताः}
{कां दिशं प्रतिपत्स्यामः प्राप्ताः क्लेशमनुत्तमम्}


\twolineshloka
{सकामो भव दुर्बुद्धे धार्तराष्ट्राल्पदर्शन}
{नूनं देवाः प्रसन्नास्ते नानुज्ञां मे युधिष्ठिरः}


\twolineshloka
{प्रयच्छति वधे तुभ्यं तेन जीवसि दुर्मते}
{नन्वद्य ससुतामात्यं सकर्णानुजसौबलम्}


\twolineshloka
{गत्वा क्रोधसमाविष्टः प्रेषयिष्ये यमक्षयम्}
{किं नु शक्यं मया कर्तुं यत्तेन क्रुध्यते नृपः}


\twolineshloka
{धर्मात्मा पाण्डवश्रेष्ठः पापाचार युधिष्ठिरः}
{एवमुक्त्वा महाबाहुः क्रोधसन्दीप्तमानसः}


\twolineshloka
{करं करेण निष्पिष्य निःश्वसन्दीनमानसः}
{पुनर्दीनमना भूत्वा शान्तार्चिरिव पावकः}


\twolineshloka
{भ्रातॄन्महीतले सुप्तानवैक्षत वृकोदरः}
{विश्वस्तानिव संविष्टान्पृथग्जनसमानिव}


\twolineshloka
{नातिदूरेण नगरं वनादस्माद्धि लक्षये}
{जागर्तव्ये स्वपन्तीमे हन्त जागर्म्यहंस्वयम्}


\twolineshloka
{प्राश्यन्तीमे जलं पश्चात्प्रतिबुद्धा जितक्लमाः}
{इति भीमो व्यवस्यैव जजागार स्वयं तदा}


\chapter{अध्यायः १६४}
\twolineshloka
{वैशंपायन उवाच}
{}


\twolineshloka
{तत्र तेषु शयानेषु हिडिम्बो नाम राक्षसः}
{अविदूरे वनात्तस्माच्छालवृक्षं समाश्रितः}


\twolineshloka
{क्रूरो मानुषमांसादो महावीर्यपराक्रमः}
{प्रावृड्जलधरश्यामः पिङ्गाक्षो दारुणाकृतिः}


\twolineshloka
{दष्ट्राकरालवदनः करालो भीमदर्शनः}
{लम्बस्फिग्लम्बजठरो रक्तश्मश्रुशिरोरुहः}


\twolineshloka
{महावृक्षगलस्कन्धः शङ्कर्णो विभीषणः}
{यदृच्छया तानपश्यत्पाण्डुपुत्रान्महारथान्}


\twolineshloka
{विरूपरूपः पिङ्गाक्षः करालो घोरदर्शनः}
{पिशितेप्सुः क्षुधार्तश्च जिघ्रन्गन्धं यदृच्छया}


\twolineshloka
{ऊर्ध्वाङ्गुलिः स कण्डूयन्धुन्वन्रूक्षाञ्शिरोरुहान्}
{जृम्भमाणो महावक्त्रः पुनःपुनरवेक्ष्य च}


\twolineshloka
{हृष्टो मानुषमांसस्य महाकायो महाबलः}
{आघ्राय मानुषं गन्धं भगिनीमिदमब्रवीत्}


\twolineshloka
{उपपन्नं चिरस्याद्य भक्ष्यं मम मनःप्रियम्}
{जिघ्रतः प्रस्रुता स्नेहाज्जिह्वा पर्येति मे मुखात्}


\twolineshloka
{अष्टौ दंष्ट्राः सुतीक्ष्णाग्राश्चिरस्यापातदुःसहाः}
{देहेषु मज्जयिष्यामि स्निग्धेषु पिशितेषु च}


\twolineshloka
{आक्रम्य मानुषं कण्ठमाच्छिद्य धमनीमपि}
{उष्णं नवं प्रपास्यामि फेनलिं रुधिरं बहु}


\twolineshloka
{गच्छ जानीहि के त्वेते शेरते वनमाश्रिताः}
{मानुषो बलवान्गन्धो घ्राणं तर्पयतीव मे}


\twolineshloka
{हत्वैतान्मानुषान्सर्वानानयस्व ममान्तिकम्}
{अस्मद्विषयसुप्तेभ्यो नैतेभ्यो भयमस्ति ते}


\twolineshloka
{एषामुत्कृत्य मांसानि मानुषाणां यथेष्टतः}
{भक्षयिष्याव सहितौ कुरु पूर्णं वचो मम}


\threelineshloka
{भक्षयित्वा च मांसानि मानुषाणां प्रकामतः}
{नृत्याव सहितावावां दत्ततालावनेकशः ॥वैशंपायन उवाच}
{}


\twolineshloka
{एवमुक्ता हिडिम्बा तु हिडिम्बेन महावने}
{भ्रातुर्वचनमाज्ञाय त्वरमाणेव राक्षसी}


\twolineshloka
{`आप्लुत्याप्लुत्य च तरूनगच्छत्पाण्डवान्प्रति}
{'जगाम तत्र यत्र स्म शेरते पाण्डवा वने}


\twolineshloka
{ददर्श तत्र सा गत्वा पाण्डवान्पृथया सह}
{शयानान्भीमसेनं च जाग्रतं त्वपराजितम्}


\twolineshloka
{`उपास्यमानान्भीमेन रूपयौवनशालिनः}
{सुकुमारांश्च पार्थान्सा व्यायामेन च कर्शितान्}


\twolineshloka
{दुःखेन संप्रयुक्तांश्च सहज्येष्ठान्प्रमाथिनः}
{रौद्री सती राजपुत्रं दर्शनीयप्रदर्शनम् ॥'}


\twolineshloka
{दृष्ट्वैव भीमसेनं सा सालस्कन्धमिवोद्यतम्}
{राक्षसी कामयामास रूपेणाप्रतिमं भुवि}


\twolineshloka
{`अन्तर्गतेन मनसा चिन्तयामास राक्षसी'}
{अयं श्यामो महाबाहुः सिंहस्कन्धो महाद्युतिः}


\twolineshloka
{कम्बुग्रीवः पुष्कराक्षो भर्ता युक्तो भवेन्मम}
{नाहं भ्रातृवचो जातु कुर्यां क्रूरमसांप्रतम्}


\twolineshloka
{पतिस्नेहोऽतिबलवान्न तथा भ्रातृसौहृदम्}
{मुहूर्तमिव तृप्तिश्च भवेद्धातुर्ममैव च}


\twolineshloka
{हतैरेतैरहत्वा तु मोदिष्ये शाश्वतीः समाः}
{`निश्चित्येत्थं हिडिम्बा सा भीमं दृष्ट्वा महाभुजं}


\twolineshloka
{उत्सृज्य राक्षसं रूपं मानुषं रूपमास्थिता}
{'सा कामरूपिणी रूपं कृत्वा मदनमोहिता}


\twolineshloka
{उपतस्थे महात्मानं भीमसेनमनिन्दिता}
{`इङ्गिताकारकुशला सोपासर्पच्छनैः शनैः}


\twolineshloka
{विनम्यमानेव लता दिव्याभरणभूषिता}
{शनैः शनैश्च तां भीमः समीपमुपसर्पतीम्}


\twolineshloka
{हर्षमाणां तदा पश्यत्तन्वीं पीनपयोधराम्}
{चन्द्राननां पद्मनेत्रां नीलकुञ्चितमूर्धजाम्}


\twolineshloka
{कृष्णां सुपाण्डुरैर्दन्तैर्बिम्बोष्ठीं चारुदर्शनाम्}
{दृष्ट्वा तां रूपसंपन्नां भीमो विस्मयमागतः}


\twolineshloka
{उपचारगुणैर्युक्तां ललितैर्हाससंमितैः}
{समीपमुपसंप्राप्य भीमं साथ वरानता}


\twolineshloka
{वचो वचनवेलायां भीमं प्रोवाच भामिनी}
{'लज्जया नम्यमानेव सर्वाभरणभूषिता}


\twolineshloka
{स्मितपूर्वमिदं वाक्यं भीमसेनमथाब्रवीत्}
{कुतस्त्वमसि संप्राप्तः कश्चासि पुरुषर्षभ}


\twolineshloka
{क इमे शेरते चेह पुरुषा देवरूपिणः}
{केयं वै बृहती श्यामा सुकुमारी तवानघ}


\twolineshloka
{शेते वनमिदं प्राप्य विश्वस्ता स्वगृहे यथा}
{नेदं जानीथ गहनं वनं राक्षससेवितम्}


% Check verse!
वसति ह्यत्र पापात्मा हिडिम्बो नाम राक्षसः
\twolineshloka
{तेनाहं प्रेषिता भ्रात्रा दुष्टभावेन रक्षसा}
{बिभक्षयिषता मांसं युष्माकममरोपमाः}


\twolineshloka
{साऽहं त्वमभिसंप्रेक्ष्य देवगर्भसमप्रभम्}
{नान्यं भर्तारमिच्छामि सत्यमेतद्ब्रवीमि ते}


\twolineshloka
{एतद्विज्ञाय धर्मज्ञ युक्तं मयि समाचर}
{कामोपहतचित्तां हि भजमानां भजस्व माम्}


\twolineshloka
{त्रास्यामि त्वां महाबाहो राक्षसात्पुरुषादकात्}
{वत्स्यावो गिरिदुर्गेषु भर्ता भव ममानघ}


\twolineshloka
{`इच्छामि वीर भद्रं ते मा मे प्राणान्विहासिषः}
{त्वया ह्यहं परित्यक्ता न जीवेयमरिन्दम ॥'}


\threelineshloka
{अन्तरिक्षचरी ह्यस्मि कामतो विचरामि च}
{अतुलामाप्नुहि प्रीतिं तत्र तत्र मया सह ॥भीमसेन उवाच}
{}


\twolineshloka
{`एष ज्येष्ठो मम भ्राता मान्यः परमको गुरुः}
{अनिविष्टो हि तन्नाहं परिविद्यां कथंचन ॥'}


\twolineshloka
{मातरं भ्रातरं ज्येष्ठं कनिष्ठानपरानपि}
{परित्यजेत कोन्वद्य प्रभवन्निह राक्षसि}


\threelineshloka
{को हि सुप्तानिमान्भ्रातॄन्दत्त्वा राक्षसभोजनम्}
{मातरं च नरो गच्छेत्कामार्त इव मद्विधः ॥राक्षस्युवाच}
{}


\threelineshloka
{`एकं त्वां मोक्षयिष्यामि सह मात्रा परन्तप}
{सोदरानुत्सृजैनांस्त्वमारोह जघनं मम ॥भीम उवाच}
{}


\threelineshloka
{नाहं जीवितुमाशंसे भ्रातॄनुत्सृज्य राक्षसि}
{यथाश्रद्धं व्रजैका हि विप्रियं मे प्रभाषसे ॥राक्षस्युवाच}
{'}


\threelineshloka
{यत्ते प्रियं तत्करिष्ये सर्वानेतान्प्रबोधय}
{मोक्षयिष्याम्यहं कामं राक्षसात्पुरुषादकात् ॥भीमसेन उवाच}
{}


\twolineshloka
{सुखसुप्तान्वने भ्रातॄन्मातरं चैव राक्षसि}
{न भयाद्बोधयिष्यामि भ्रातुस्तव दुरात्मनः}


\twolineshloka
{न हि मे राक्षसा भीरु सोढुं शक्ताः पराक्रमम्}
{न मनुष्या न गन्धर्वा न यक्षाश्चारुलोचने}


\twolineshloka
{गच्छ वा तिष्ठ वा भद्रे यद्वा पीच्छसि तत्कुरु}
{तं वा प्रेषय तन्वङ्गि भ्रातरं पुरुषादकम्}


\chapter{अध्यायः १६५}
\twolineshloka
{वैशंपायन उवाच}
{}


\twolineshloka
{तां विदित्वा चिरगतां हिडिम्बो राक्षसेश्वरः}
{अवतीर्य द्रुमात्तस्मादाजगामाशु पाण्डवान्}


\twolineshloka
{लोहिताक्षो महाबाहुरूर्ध्वकेशो महाननः}
{मेघसङ्घातवर्ष्मा च तीक्ष्णदंष्ट्रो भयानकः}


\twolineshloka
{तलं तलेन संहत्य बाहू विक्षिप्य चासकृत्}
{उद्वृत्तनेत्रः संक्रुद्धो दन्तान्दन्तेषु निष्कुषन्}


\threelineshloka
{कोऽद्य मे भोक्तुकामस्य विघ्नं चरति दुर्मतिः}
{न बिभेति हिडिम्बी च प्रेषिता किमनागता ॥वैशंपायन उवाच}
{}


\twolineshloka
{तमापतन्तं दृष्ट्वै तथा विकृतदर्शनम्}
{हिडिम्बोवाच वित्रस्ता भीमसेनमिदं वचः}


\twolineshloka
{आपतत्येष दुष्टात्मा संक्रुद्धः पुरुषादकः}
{साऽहं त्वां भ्रातृभिः सार्धं यद्ब्रवीमि तथा कुरु}


\twolineshloka
{अहं कामगमा वीर रक्षोबलसमन्विता}
{आरुहेमां मम श्रोणिं नेष्यामि त्वां विहायसा}


\threelineshloka
{प्रबोधयैतान्संसुप्तान्मातरं च परन्तप}
{सर्वानेव गमिष्याभि गृहीत्वा वो विहायसा ॥भीम उवाच}
{}


\twolineshloka
{मा भैस्त्वं पृथुसुश्रोणि नैष कश्चिन्मयि स्थिते}
{अहमेनं हनिष्यामि पश्यन्त्यास्ते सुमध्यमे}


\twolineshloka
{नायं प्रतिबलो भीरु राक्षसापसदो मम}
{सोढुं युधि परिस्पन्दमथवा सर्वराक्षसाः}


\twolineshloka
{पश्य बाहू सुवृत्तौ मे हस्तिहस्तनिभाविमौ}
{ऊरू परिघसङ्काशौ संहतं चाप्युरो महत्}


\threelineshloka
{विक्रमं मे यथेन्द्रस्य साऽद्य द्रक्ष्यसि शोभने}
{माऽवमंस्थाः पृथुश्रोणि मत्वा मामिह मानुषम् ॥हिडिम्बोवाच}
{}


\threelineshloka
{नावमन्ये नरव्याघ्र त्वामहं देवरूपिणम्}
{दृष्टप्रभावस्तु मया मानुषेष्वेव राक्षसः ॥वैशंपायन उवाच}
{}


\twolineshloka
{तथा संजल्पतस्तस्य भीमसेनस्य भारत}
{वाचः शुश्राव ताः क्रुद्धो राक्षसः पुरुषादकः}


\twolineshloka
{अवेक्षमाणस्तस्याश्च हिडिम्बो मानुषं वपुः}
{स्रग्दामपूरितशिखां समग्रेन्दुनिभाननाम्}


\twolineshloka
{सुभ्रूनासाक्षिकेशान्तां सुकुमारनखत्वचम्}
{सर्वाभरणसंयुक्तां सुसूक्ष्माम्बरधारिणीम्}


\twolineshloka
{तां तथा मानुषं रूपं बिभ्रतीं सुमनोहरम्}
{पुंस्कामां शङ्कमानश्च चुक्रोध पुरुषादकः}


\twolineshloka
{संक्रुद्धो राक्षसस्तस्या भगिन्याः कुरुसत्तम}
{उत्फाल्य विपुले नेत्रे ततस्तामिदमब्रवीत्}


\twolineshloka
{को हि मे भोक्तुकामस्य विघ्नं चरति दुर्मतिः}
{न बिभेषि हिडिम्बे किं मत्कोपाद्विप्रमोहिता}


\twolineshloka
{धिक्त्वामसति पुंस्कामे मम विप्रियकारिणि}
{पूर्वेषां राक्षसेन्द्राणां सर्वेषामयशस्करि}


\threelineshloka
{यानिमानाश्रिताऽकार्षीर्विप्रियं समुहन्मम}
{एष तानद्य वै सर्वान्हनिष्यामि त्वया सह ॥वैशंपायन उवाच}
{}


\twolineshloka
{एवमुक्त्वा हिडिम्बां स हिडिम्बो लोहितेक्षणः}
{वधायाभिपपातैनान्दन्तैर्दन्तानुपस्पृशन्}


\twolineshloka
{गर्जन्तमेवं विजने भीमसेनोऽभिवीक्ष्य तम्}
{रक्षन्प्रबोधं भ्रातॄणां मातुश्च परवीरहा}


\threelineshloka
{तमापतान्तं संप्रेक्ष्य भीमः प्रहरतां वरः}
{भर्त्सयामास तेजस्वी तिष्ठतिष्ठेति चाब्रवीत् ॥वैशंपायन उवाच}
{}


\twolineshloka
{भीमसेनस्तु तं दृष्ट्वा राक्षसं प्रहसन्निव}
{भगिनीं प्रति सङ्क्रुद्धमिदं वचनमब्रवीत्}


\twolineshloka
{किं ते हिडिम्ब एतैर्वा सुखसुप्तैः प्रबोधितैः}
{मामासादय दुर्बुद्धे तरसा त्वं नराशन}


\twolineshloka
{मय्येव प्रहरैहि त्वं न स्त्रियं हन्तुमर्हसि}
{विशेषतोऽनपकृते परेणापकृते सति}


\twolineshloka
{न हीयं स्ववशा बाला कामयत्यद्य मामिह}
{चोदितैषा ह्यनङ्गेन शरीरान्तरचारिणा}


\twolineshloka
{भगिनी तव दुर्वृत्त रक्षसां वै यशोहर}
{त्वन्नियोगेन चैवेयं रूपं मम समीक्ष्य च}


\twolineshloka
{कामयत्यद्य मां भीरुस्तव नैषापराध्यति}
{अनङ्गेन कृते दोषे नेमां गर्हितुमर्हसि}


\twolineshloka
{मयि तिष्ठति दुष्टात्मन्न स्त्रियं हन्तुमर्हसि}
{संगच्छस्व मया सार्धमेकेनैको नराशन}


\threelineshloka
{अहमेको गमिष्यामि त्वामद्य यमसादनम्}
{अद्य मद्बलनिष्पिष्टं शिरो राक्षस दीर्यताम्}
{कुञ्जरस्येव पादेन विनिष्पिष्टं बलीयसाः}


\twolineshloka
{अद्य गात्राणि ते कङ्काः श्येना गोमायवस्तथा}
{कर्षन्तु भुवि संहृष्टा निहतस्य मया मृधे}


\twolineshloka
{क्षणेनाद्य करिष्येऽहमिदं वनमराक्षसम्}
{पुरा यद्दूषितं नित्यं त्वया भक्षयता नरान्}


\twolineshloka
{अद्य त्वां भगिनी रक्षः कृष्यमाणं मयाऽसकृत्}
{द्रक्ष्यत्यद्रिप्रतीकाशं सिंहेनेव महाद्विपम्}


\threelineshloka
{निराबाधास्त्वयि हते मया राक्षसपांसन}
{वनमेतच्चरिष्यन्ति पुरुषा वनचारिणः ॥हिडिम्ब उवाच}
{}


\twolineshloka
{गर्जितेन वृथा किं ते कत्थितेन च मानुष}
{कृत्वैतत्कर्मणा सर्वं कत्थेया मा चिरं कृथाः}


\twolineshloka
{बलिनं मन्यसे यच्चाप्यात्मानं सपराक्रमम्}
{ज्ञास्यस्यद्य समागम्य मयात्मानं बलाधिकम्}


\twolineshloka
{न तावदेतान्हिंसिष्ये स्वपन्त्वेते यथासुखम्}
{एष त्वामेव दुर्बुद्धे निहन्म्यद्याप्रियंवदम्}


\threelineshloka
{पीत्वा तवासृग्गात्रेभ्यस्ततः पश्चादिमानपि}
{हनिष्यामि ततः पश्चादिमां विप्रियकारिणीम् ॥वैशंपायन उवाच}
{}


\twolineshloka
{एवमुक्त्वा ततो बाहुं प्रगृह्य पुरुषादकः}
{अभ्यद्रवत संक्रुद्धो भीमसेनमरिन्दमम्}


\twolineshloka
{तस्याभिद्रवतस्तूर्णं भीमो भीमपराक्रमः}
{वेगेन प्रहितं बाहुं निजग्राह हसन्निव}


\twolineshloka
{निगृह्य तं बलाद्भीमो विस्फुरन्तं चकर्ष ह}
{तस्माद्देशाद्धनूंष्यष्टौ सिंहः क्षुद्रमृगं यथा}


\twolineshloka
{ततः स राक्षसः क्रुद्धः पाण्डवेन बलार्दितः}
{भीमसेनं समालिङ्ग्य व्यनदद्भैरवं रवम्}


\twolineshloka
{पुनर्भीमो बलादेनं विचकर्ष महाबलः}
{मा शब्दः सुखसुप्तानां भ्रातॄणां मे भवेदिति}


\twolineshloka
{`हस्ते गृहीत्वा तद्रक्षो दूरमन्यत्र नीतवान्}
{पृच्छे गृहीत्वा तुण्डेन गरुडः पन्नगं यथा ॥'}


\twolineshloka
{अन्योन्यं तौ समासाद्य विचकर्षतुरोजसा}
{हिडिम्बो भीमसेनश्च विक्रमं चक्रतुः परम्}


\twolineshloka
{बभञ्जतुस्तदा वृक्षाँल्लताश्चाकर्षतुस्तदा}
{मत्ताविव चं संरब्धौ वारणौ षष्टिहायनौ}


\twolineshloka
{`पादपानुद्धरन्तौ तावूरुवेगेन वेगितौ}
{स्फोटयन्तौ लताजालान्यूरुभ्यां गृह्य सर्वशः}


\twolineshloka
{वित्रासयन्तौ तौ शब्दैः सर्वतो मृगपक्षिणः}
{बलेन बलिनौ मत्तावन्योन्यवधकाङ्क्षिणौ}


\twolineshloka
{भीमराक्षसयोर्युद्धं तदाऽवर्तत दारुणम्}
{पुरा देवासुरे युद्धे वृत्रवासवयोरिव}


\twolineshloka
{भङूक्त्वा वृक्षान्महाशाखांस्ताडयामासतुः क्रुधा}
{सालतालतमालाम्रवटार्जुनविभीतकान्}


\twolineshloka
{न्यग्रोधप्लक्षखर्जूरपनसानश्मकण्टकान्}
{एतानन्यान्महावृक्षानुत्खाय तरसाऽखिलान्}


\twolineshloka
{उत्क्षिप्यान्योन्यरोषेण ताडयामासतू रणे}
{यदाऽभवद्वनं सर्वं निर्वृक्षं वृक्षसङ्कुलम्}


\twolineshloka
{तदा शिलाश्च कुञ्जांश्च वृक्षान्कण्टकिनस्तथा}
{ततस्तौ गिरिशृङ्गाणि पर्वतांश्चाभ्रलेलिहान्}


\twolineshloka
{शैलांश्च गण्डपाषाणानुत्खायादाय वैरिणौ}
{चिक्षेपतुरुपर्याजावन्योन्यं विजयेषिणौ}


\twolineshloka
{तद्वनं परितः पञ्चयोजनं निर्महीरुहम्}
{निर्लतागुल्मपाषाणं निर्मृगं चक्रतुर्भृशम्}


\twolineshloka
{तयोर्युद्धेन राजेन्द्र तद्वनं भीमरक्षसोः}
{मुहूर्तेनाभवत्कूमर्पृष्ठवच्छ्लक्ष्णमव्ययम्}


\twolineshloka
{ऊरुबाहुपरिक्लेशात्कर्षन्तावितरेतरम्}
{उत्कर्षन्तौ विकर्षन्तौ प्रकर्षन्तौ परस्परम्}


\twolineshloka
{तौ स्वनेन विना राजन्गर्जन्तौ च परस्परम्}
{पाषाणसंघट्टनिभैः प्रहारैरभिजघ्नतुः}


\threelineshloka
{अन्योन्यं च समालिङ्ग्य विकर्षन्तौ परस्परम्}
{बाहुयुद्धमभूद्धोरं बलिवासवयोरिव}
{युद्धसंरम्भनिर्गच्छत्फूत्काररवनिस्वनम् ॥'}


\twolineshloka
{तयोः शब्देन महता विबुद्धास्ते नरर्षभाः}
{सह मात्रा च ददृशुर्हिडिम्बामग्रतःस्थिताम्}


\chapter{अध्यायः १६६}
\twolineshloka
{वैशंपायन उवाच}
{}


\twolineshloka
{प्रबुद्धास्ते हिडिम्बाया रूपं दृष्ट्वातिमानुषम्}
{विस्मिताः पुरुषव्याघ्रा बभूवुः पृथया सह}


\twolineshloka
{ततः कुन्ती समीक्ष्यैनां विस्मिता रूपसंपदा}
{उवाच मधुरं वाक्यं सान्त्वपूर्वमिदं शैनः}


\twolineshloka
{कस्य त्वं सुरगर्भाभे कावाऽसि वरवर्णिनि}
{केन कार्येण संप्राप्ता कुतश्चागमनं तव}


\threelineshloka
{यदि वाऽस्य वनस्य त्वं देवता यदि वाऽप्सराः}
{आचक्ष्व मम तत्सर्वं किमर्थं चेह तिष्ठसि ॥हिडिम्बोवाच}
{}


\twolineshloka
{यदेतत्पश्यसि वनं नीलमेघनिं महत्}
{निवासो राक्षसस्यैष हिडिम्बस्य ममैव च}


\twolineshloka
{तस्य मां राक्षसेन्द्रस्य भगिनीं विद्दि भामिनि}
{भ्रात्रा संप्रेषितामार्ये त्वां सपुत्रां जिघांसता}


\twolineshloka
{क्रूरबुद्धेरहं तस्य वचनादागता त्विह}
{अद्राक्षं नवहेमाभं तव पुत्रं महाबलम्}


\twolineshloka
{ततोऽहं सर्वभूतानां भावे विचरता शुभे}
{चोदिता तव पुत्रार्थं मन्मथेन वशानुगा}


\twolineshloka
{ततो वृतो मया भर्ता तव पुत्रो महाबलः}
{अपनेतुं च यतितो न चैव शकितो मया}


\twolineshloka
{चिरायमाणां मां ज्ञात्वा ततः स पुरुषादकः}
{स्वयमेवागतो हन्तुमिमान्सर्वांस्तवात्मजान्}


\twolineshloka
{स तेन मम कान्तेन तव पुत्रेण धीमता}
{बलादितो विनिष्पिष्य व्यपनीतो महात्मना}


\threelineshloka
{विकर्षन्तौ महावेगौ गर्जमानौ परस्परम्}
{पश्यैवं युधि विक्रान्तावेतौ च नरराक्षसौ ॥वैशंपायन उवाच}
{}


\twolineshloka
{तस्याः श्रुत्वैव वचनमुत्पपात युधिष्ठिरः}
{अर्जुनो नकुलश्चैव सहदेवश्च वीर्यवान्}


\twolineshloka
{तौ ते ददृशुरासक्तौ विकर्षन्तौ परस्परम्}
{काङ्क्षमाणौ जयं चैव सिंहाविव बलोत्कटौ}


\twolineshloka
{अथान्योन्यं समाश्लिष्य विकर्षन्तौ पुनःपुनः}
{दावाग्निधूमसदृशं चक्रतुः पार्थिवं रजः}


\twolineshloka
{वसुधारेणुसंवीतौ वसुधाधरसन्निभौ}
{बभ्राजतुर्यथा शैलौ नीहारेणाभिसंवृतौ}


\twolineshloka
{राक्षसेन तदा भीमं क्लिश्यमानं निरीक्ष्य च}
{उवाचेदं वचः पार्थः प्रहसञ्छनकैरिव}


\twolineshloka
{भीम माभैर्महाबाहो न त्वां बुध्यामहे वयम्}
{समेतं भीमरूपेण रक्षसा श्रमकर्शिताः}


\threelineshloka
{साहाय्येऽस्मि स्थितः पार्थ पातयिष्यामि राक्षसम्}
{नकुलः सहदेवश्च मातरं गोपयिष्यतः ॥भीम उवाच}
{}


\twolineshloka
{उदासीनो निरीक्षस्व न कार्यः संभ्रमस्त्वया}
{न जात्वयं पुनर्जीवेन्मद्बाह्वन्तरमागतः}


\twolineshloka
{`भुजयोरन्तरं प्राप्तो भीमसेनस्य राक्षसः}
{अमृत्वा पार्थवीर्येण मृतो मा भूदिति ध्वनिः}


\threelineshloka
{अयमस्मांस्तु नो हन्याज्जातु पार्थ राक्षसः}
{जीवन्तं न प्रमोक्ष्यामि मा भैषीर्भरतर्षभ ॥'अर्जुन उवाच}
{}


\twolineshloka
{`पूर्वरात्रे प्रयुक्तोऽसि भीम क्रूरेण रक्षसा}
{क्षपा व्युष्टा न चेदानीं समाप्तोसीन्महारणः ॥'}


\twolineshloka
{किमेनन चिरं भीम जीवता पापरक्षसा}
{गन्तव्ये न चिरं स्थातुमिह शक्यमरिन्दम}


\twolineshloka
{पुरा संरज्यते प्राची पुरा सन्ध्या प्रवर्तते}
{रौद्रे मुहूर्ते रक्षांसि प्रबलानि भवन्त्युत}


\twolineshloka
{त्वरस्व भीम मा क्रीड जहि रक्षो विभीषणम्}
{पुरा विकुरुते मायां भुजयोः सारमर्पय}


\twolineshloka
{`माहात्म्यमात्मनो वेत्थ नराणां हितकाम्यया}
{रक्षो जहि यथा शक्रः पुरा वृत्रं महाबलम्}


\twolineshloka
{अथवा मन्यसे भारं त्वमिमं राक्षसं युधि}
{आतिष्ठे तव साहाय्यं शीघ्रमेव तु हन्यताम्}


\threelineshloka
{अथवा त्वहमेवैनं हनिष्यामि वृकोदर}
{कृतकर्मा परिश्रान्तः साधु तावदुपारम ॥'वैशंपायन उवाच}
{}


\twolineshloka
{अर्जुनेनैवमुक्तस्तु भीमो रोषाज्ज्वलन्निव}
{बलमाहारयामास यद्वायोर्जगतः क्षये}


\twolineshloka
{ततस्तस्याम्बुदाभस्य भीमो रोषात्तु रक्षसः}
{अत्क्षिप्याभ्रामयद्देहं तूर्णं शतगुणं तदा}


\threelineshloka
{`इति चोवाच संक्रुद्धो भ्रामयन्राक्षसीं तनुम्}
{भीमसेनो महाबाहुरभिगर्जन्मुहुर्मुहुः ॥'भीम उवाच}
{}


\twolineshloka
{नरमांसैर्वृथा पुष्टो वृथा वृद्धो वृथामतिः}
{वृथामरणमर्हस्त्वं वृथाद्य न भविष्यसि}


\threelineshloka
{क्षेममद्य करिष्यामि यथा वनमकण्टकम्}
{न पुनर्मानुषान्हत्वा भक्षयिष्यसि राक्षस ॥वैशंपायन उवाच}
{}


\twolineshloka
{इत्युक्त्वा भीमसेनस्तं निष्पिष्य धरणीतले}
{बाहुभ्यामवपीड्याशु पशुमारममारयत्}


\twolineshloka
{स मार्यमाणो भीमेन ननाद विपुलं स्वनम्}
{पूरयंस्तद्वनं सर्वं जलार्द्रे इव दुन्दुभिः}


\twolineshloka
{बाहुभ्यां योक्त्रयित्वा तं बलवान्पाण्डुनन्दनः}
{`समुद्धाम्य शिरश्चास्य सग्रीवं तदपाहरत्}


\twolineshloka
{ततो भित्त्वा शिरश्चास्य सग्रीवं तदुदाक्षिपत्}
{तस्य निष्कर्णनयनं निर्जिह्वं रुधिरोक्षितम्}


\twolineshloka
{प्राविद्धं भीमसेनेन शिरो विदशनं बभौ}
{प्रसारितभुजोद्धृष्टो भिन्नमांसत्वगन्तरः}


\twolineshloka
{कबन्धभूतस्तत्रासीद्दनुर्वज्रहतो तथा}
{हिडिम्बं निहतं दृष्ट्वा संहृष्टास्ते तरस्विनः}


\twolineshloka
{हिडिम्बा सा च संप्रेक्ष्य निहतं राक्षसं रणे}
{अदृश्याश्चैव ये स्वस्स्थाः समेताः सर्षिचारणाः}


\twolineshloka
{पूजयन्ति स्म तं हृष्टाः साधुसाध्विति पाण्डवम्}
{भ्रातरश्चापि संहृष्टा युधिष्ठिरपुरोगमाः}


\threelineshloka
{अपूजयन्नरव्याघ्रं भीमसेनमरिन्दमम्}
{'अभिपूज्य महात्मानं भीमं भीमपराक्रमम्}
{पुनरेवार्जुनो वाक्यमुवाचेदं वृकोदरम्}


\twolineshloka
{अदूरे नगरं मन्ये वनादस्मादहं विभो}
{शीघ्रं गच्छाम भद्रं ते न नो विद्यात्सुयोधनः}


\twolineshloka
{ततः सर्वे तथेत्युक्त्वा मात्रा सह महारथाः}
{प्रययुः पुरुषव्याघ्रा हिडिम्बा चैव राक्षसी}


\chapter{अध्यायः १६७}
\twolineshloka
{वैशंपायन उवाच}
{}


\twolineshloka
{सा तानेवापतत्तूर्णं भगिनी तस्य रक्षसः}
{अब्रुवाणा हिडिम्बा तु राक्षसी पाण्डवान्प्रति}


\twolineshloka
{अभिवाद्य ततः कुन्तीं धर्मराजं च पाण्डवम्}
{अभिपूज्य ततः सर्वान्भीमसेनमभाषत}


\twolineshloka
{अहं ते दर्शादेव मन्मथस्य वशं गता}
{क्रूरं भ्रातृवचो हित्वा सा त्वामेवानिरुन्धती}


\threelineshloka
{राक्षसे रौद्रसङ्काशे तवापश्यं विचेष्टितम्}
{अहं शुश्रूषुरिच्छेयं तव गात्रं निषेवितुम् ॥'भीमसेन उवाच}
{}


\threelineshloka
{स्मरन्ति वैरं रक्षांसि मायामाश्रित्य मोहिनीम्}
{हिडिम्बे व्रज पन्थानं त्वमिमं भ्रातृसेवितम् ॥युधिष्ठिर उवाच}
{}


\twolineshloka
{क्रुद्धोऽपि पुरुषव्याघ्र भीम मा स्म स्त्रियं वधीः}
{शरीरगुप्त्यभ्यधिकं धर्मं गोपाय पाण्डव}


\threelineshloka
{वधाभिप्रायमायान्तमवधीस्त्वं महाबलम्}
{रक्षसस्तस्य भगिनी किं नः क्रुद्धा करिष्यति ॥वैशंपायन उवाच}
{}


\twolineshloka
{हिडिम्बा तु ततः कुन्तीमभिवाद्य कृताञ्जलिः}
{युधिष्ठिरं तु कौन्तेयमिदं वचनमब्रवीत्}


\twolineshloka
{आर्ये जानासि यद्दुःखमिह स्त्रीणामनङ्गजम्}
{तदिदं मामनुप्राप्तं भीमसेनकृते शुभे}


\twolineshloka
{सोढं तत्परमं दुःखं मया कालप्रतीक्षया}
{सोऽयमभ्यागतः कालो भविता मे सुखोदयः}


\twolineshloka
{मया ह्युत्सृज्य सुहृदः स्वधर्मं स्वजनं तथा}
{वृतोऽयं पुरुषव्याघ्रस्तव पुत्रः पतिः शुभे}


\twolineshloka
{वीरेणाऽहं तथाऽनेन त्वया चापि यशस्विनी}
{प्रत्याख्याता न जीवामि सत्यमेतद्ब्रवीमि ते}


\twolineshloka
{यदर्हसि कृपां कर्तुं मयि त्वं वरवर्णिनि}
{मत्वा मूढेति तन्मां त्वं भक्ता वाऽनुगतेति वा}


\threelineshloka
{भर्त्राऽनेन महाभागे संयोजय सुतेन ह}
{समुपादाय गच्छेयं यथेष्टं देवरूपिणम्}
{पुनश्चैवानयिष्यामि विस्रम्भं कुरु मे शुभे}


\twolineshloka
{`अहं हि समये लप्स्ये प्राग्भ्रातुरपर्वजनात्}
{ततः सोऽभ्यपतद्रात्रौ भीमसेनजिघांसया}


\twolineshloka
{यथायथा विक्रमते यथारिमधितिष्ठति}
{तथातथा समासाद्य पाण्डवं काममोहिता}


\twolineshloka
{न यातुधान्यहं त्वार्ये न चास्मि रजनीचरी}
{ईशा रक्षस्स्वसा ह्यस्मि राज्ञि सालकटङ्कटी}


\twolineshloka
{पुत्रेण तव संयुक्ता युवतिर्देववर्णिनी}
{सर्वान्वोऽहमुपस्थास्ये पुरस्कृत्य वृकोदरम्}


\twolineshloka
{अप्रमत्ता प्रमत्तेषु शुश्रूषुरसकृत्त्वहम्}
{'वृजिने तारयिष्यामि दासीवच्च नरर्षभाः}


\twolineshloka
{पृष्ठेन वो वहिष्यामि विमानं सुकृतानिव}
{यूयं प्रसादं कुरुत भीमसेनो भजेत माम्}


\twolineshloka
{`एवं ब्रुवन्ती ह तथा प्रत्याख्याता क्रियां प्रति}
{भूम्यां दुष्कृतिनो लोकान्गमिष्येऽहं न संशयः}


\twolineshloka
{अहं हि मनसा ध्यात्वा सर्वं वेत्स्यामि सर्वदा}
{'आपन्निस्तरणे प्राणान्धारयिष्ये न केनचित्}


\twolineshloka
{सर्वमावृत्य कर्तव्यं धर्मं समनुपश्यता}
{आपत्सु यो धारयति स वै धर्मविदुत्तमः}


\twolineshloka
{व्यसनं ह्येव धर्मस्य धर्मिणामापदुच्यते}
{पुम्यात्प्राणान्धारयति पुण्यं वै प्राणधारणम्}


\twolineshloka
{येन केनाचरेद्धर्मं तस्मिन्गर्हा न विद्यते}
{`महतोऽत्र स्त्रियं कामाद्वाधितां त्राहि मामपि}


\twolineshloka
{धर्मार्थकाममोक्षेषु दयां कुर्वन्ति साधवः}
{तत्तु धर्ममिति प्राहुर्मुनयो धर्मवत्सलाः}


\twolineshloka
{दिव्यज्ञानेन जानामि व्यतीतानागतानहम्}
{तस्माद्वक्ष्यामि वः श्रेय आसन्नं सर उत्तमम्}


\twolineshloka
{अद्यासाद्य सरः स्नात्वा विश्रम्य च वनस्पतौ}
{श्वः प्रभाते महद्भूतं प्रादुर्भूतं जगत्पतिम्}


\twolineshloka
{व्यासं कमलपत्राक्षं दृष्ट्वा शोकं विहास्यथ}
{धार्तराष्ट्राद्विवासं च दहनं वारणावते}


\twolineshloka
{त्राणं च विदुरात्तुभ्यं विदितं ज्ञानचक्षुषा}
{आवासे शालिहोत्रस्य स वो वासं विधास्यति}


\twolineshloka
{वर्षवातातपसहो ह्ययं पुण्यो वनस्पतिः}
{पीतमात्रे तु पानीये क्षुत्पिपासे विनश्यतः}


\twolineshloka
{तपसा शालिहोत्रेण सरो वृक्षश्च निर्मितः}
{कादम्बाः सारसा हंसाः कुरर्यः कुररैः सह}


\threelineshloka
{रुवन्ति मधुरं गीतं गान्धर्वस्वनमिश्रितम्}
{वैशंपायन उवाच}
{तस्यास्तद्वचनं श्रुत्वा कुन्ती वचनमब्रवीत्}


\threelineshloka
{युधिष्ठिरं महाप्राज्ञं सर्वधर्मविशारदम्}
{कुन्त्युवाच}
{त्वं हि धर्मभृतां श्रेष्ठो मयोक्तं शृणु भारत}


\twolineshloka
{राक्षस्येषा हि वाक्येन धर्मं वदति साधु वै}
{भावेन दुष्टा भीमं वै किं करिष्यति राक्षसी}


\threelineshloka
{भजतां पाण्डवं वीरमपत्यार्थं यदीच्छसि}
{' युधिष्ठिर उवाच}
{एवमेतद्यथाऽऽत्थ त्वं हिडिम्बे नात्र संशयः}


\twolineshloka
{स्थातव्यं तु त्वया धर्मे यथा ब्रूयां सुमध्यमे}
{`नित्यं कृताह्निका स्नाता कृतशौचा सुरूपिणी ॥'}


\twolineshloka
{स्नातं कृताह्निकं भद्रे कृतकौतुकमङ्गलम्}
{भीमसेनं भजेथास्त्वमुदिते वै दिवाकरे}


\twolineshloka
{अहस्सु विहरानेन यथाकामं मनोजवा}
{अयं त्वानयितव्यस्ते भीमसेनः सदा निशि}


\twolineshloka
{प्राक्सन्ध्यातो विमोक्तव्यो रक्षितव्यश्च नित्यशः}
{एवं रमस्व भीमेन यावद्गर्भस्य वेदनम्}


\twolineshloka
{एष ते समयो भद्रे शुश्रूषा चाप्रमत्तया}
{नित्यानुकूलया भूत्वा कर्तव्यं शोभनं त्वया}


\chapter{अध्यायः १६८}
\twolineshloka
{वैशंपायन उवाच}
{}


\twolineshloka
{युधिष्ठिरवचः श्रुत्वा कुन्तीमङ्गेऽधिरोप्य सा}
{भीमार्जुनान्तरगता यमाभ्यां च पुरस्कृता}


\twolineshloka
{तिर्यग्युधिष्ठिरे याति हिडिम्बा भीमगामिनी}
{शालिहोत्रसरो रम्यमाससाद जलार्थिनी}


\twolineshloka
{वनस्पतितलं गत्वा परिमृज्य गृहं यथा}
{पाण्डवानां च वासं सा कृत्वा पर्णमयं तथा}


\twolineshloka
{आत्मनश्च तथा कुन्त्या एकोद्देशे चकार सा}
{पाण्डवास्तु ततः स्नात्वा शुद्धाः सन्ध्यामुपास्य च}


\twolineshloka
{तृषिताः क्षुत्पिपासार्ता जलमात्रेण वर्तयन्}
{शालिहोत्रस्ततो ज्ञात्वा क्षुधार्तान्पाण्डवांस्तदा}


\twolineshloka
{मनसा चिन्तयामास पानीयं भोजनं महत्}
{ततस्ते पाण्डवाः सर्वे विश्रान्ताः पृथया सह}


\twolineshloka
{यथा जतुगृहे वृत्तं राक्षसेन कृतं च यत्}
{कृत्वा कथा बहुविधाः कथान्ते पाण्डुनन्दनम्}


\twolineshloka
{कुन्ती राजसुता वाक्यं भीमसेनमथाब्रवीत्}
{यथा पाण्डुस्तथा मान्यस्तव ज्येष्ठो युधिष्ठिरः}


\twolineshloka
{अहं धर्मविदाऽनेन मान्या गुरुतरा तव}
{तस्मात्पाण्डुहितार्थं मे युवराज हितं कुरु}


\twolineshloka
{निकृता धार्तराष्ट्रेण पापेनाकृतबुद्धिना}
{दुष्कृतस्य प्रतीकारं न पश्यामि वृकोदर}


\twolineshloka
{तस्मात्कतिपयाहेन योगक्षेमं भविष्यति}
{क्षेमं दुर्गमिमं वासं वत्स्याम हि यथा वयम्}


\twolineshloka
{इदमन्यन्महदुःखं धर्मकृच्छ्रं वृकोदर}
{दृष्ट्वैव त्वां महाप्राज्ञ अनङ्गाभिप्रचोदिता}


\twolineshloka
{युधिष्ठिरं च मां चैव वरयामास धर्मतः}
{धर्मार्थं देहि पुत्रं त्वं स नः श्रेयः करिष्यति}


\threelineshloka
{प्रतिवाक्यं तु नेच्छामि आवयोर्वचनं कुरु}
{वैशंपायन उवाच}
{तथेति तत्प्रतिज्ञाय भीमसेनोऽब्रवीदिदम्}


\twolineshloka
{शासनं ते करिष्यामि वेदशासनमित्यपि}
{समक्षं भ्रातृमध्ये तु तां चोवाच स राक्षसीम्}


\twolineshloka
{शृणु राक्षसि सत्येन समयं ते वदाम्यहम्}
{'यावत्कालेन भवति पुत्रस्योत्पादनं शुभे}


\twolineshloka
{तावत्कालं चरिष्यामि त्वया सह सुमध्यमे}
{`विशेषतो मत्सकाशे मा प्रकाशय नीचताम्}


\threelineshloka
{उत्तमस्त्रीगुणोपेता भजेथा वरवर्णिनि}
{वैशंपायन उवाच}
{सा तथेति प्रतिज्ञाय हिडिम्बा राक्षसी तथा}


\twolineshloka
{गताऽहनि निवेशेषु भोज्यं राजार्हमानयत्}
{सा कदाचिद्विहारार्थं हिडिम्बा कामरूपिणी}


\twolineshloka
{भीमसेनमुपादाय ऊर्ध्वमाचक्रमं तदा}
{शैलशृङ्गेषु रम्येषु देवतायतनेषु च}


\twolineshloka
{मृगपक्षिविघृष्टेषु रमणीयेषु सर्वेषु}
{कृत्वा सा परमं रूपं सर्वाभरणभूषिता}


\twolineshloka
{संजल्पन्ती सुमधुरं रमयामास पाण्डवम्}
{तथैव वनदुर्गेषु पर्वतद्रुमसानुषु}


\twolineshloka
{सरस्तु रमणीयेषु पद्मोत्पलवनेषु च}
{नदीद्वीपप्रदेशेषु वैडूर्यसिकतेषु च}


\twolineshloka
{देवारण्येषु पुण्येषु तथा पर्वतसानुषु}
{सुतीर्थवनतोयासु तथा गिरिनदीषु च}


\twolineshloka
{सागरस्य प्रदेशेषु भणिहेमयुतेषु च}
{गुह्यकानां निवासेषु कुलपर्वतसानुषु}


\twolineshloka
{सर्वर्तुफलवृक्षेषु मानसेषु वनेषु च}
{बिभ्रती परमं रूपं रमयामास पाण्डवम्}


\twolineshloka
{`यथा सुमोदते स्वर्गे सुकृत्यप्सरसा सह}
{सुतरां परमप्रीतस्तथा रेमे महाद्युतिः}


\twolineshloka
{शुभं हि जघनं तस्याः सवर्णमणिमेखलम्}
{न ततर्प तदा मृद्गन्भीमसेनो मुहुर्मुहुः ॥'}


\twolineshloka
{रमयन्ती ततो भीमं तत्रतत्र मनोजवा}
{सा रेमे तेन संहर्षात्तृप्यन्ती च मुहुर्मुहुः}


\twolineshloka
{अहस्सु रमयन्ती सा निशाकालेषु पाण्डवम्}
{आनीय वै स्वके गेहे दर्शयामास मातरम्}


\twolineshloka
{भ्रातृभिः सहितो नित्यं स्वपते पाण्डवस्तथा}
{कुन्त्याः परिचरन्ती सा तस्याः पार्श्वेवसन्निशां}


\twolineshloka
{कामांश्च मुखवासादीनानयिष्यति भोजनम्}
{तस्यां रात्र्यां व्यतीतायामाजगाम महाव्रतः}


\fourlineindentedshloka
{पाराशर्यो महाप्राज्ञो दिव्यदर्शी महातपाः}
{तेऽभिवाद्य महात्मानं कृष्णद्वैपायनं शुभम्}
{तस्थुः प्राञ्जलयः सर्वे सस्नुषा चैव माधवी ॥श्रीव्यास उवाच}
{}


\twolineshloka
{मयेदं मनसा पूर्वं विदितं भरतर्षभाः}
{यथा स्थितैरधर्मेण धार्तराष्ट्रौर्विवासिताः}


\twolineshloka
{तद्विदित्वाऽस्मि संप्राप्तश्चिकीर्ष्वै परं हितम्}
{न विषादो हि वः कार्यः सर्वमेतत्सुखाय वः}


\twolineshloka
{सुहृद्वियोजनं कर्म पुराकृतमरिन्दमाः}
{तस्य सिद्धिरियं प्राप्ता मा शोचत परन्तपाः}


\twolineshloka
{समाप्ते दुष्कृते चैव यूयं ते वै न संशयः}
{स्वराष्ट्रे विहरिष्यन्तो भविष्यथ सबान्धवाः}


\twolineshloka
{दीनतो बालतश्चैव स्नेहं कुर्वन्ति बान्धवाः}
{तस्मादभ्यधिकः स्नेहो युष्मासु मम संप्रति}


\twolineshloka
{स्नेहपूर्वं चिकीर्षामि हितं यत्तन्निबोधत}
{वसतेह प्रतिच्छन्ना ममागमनकाङ्क्षिणः}


\twolineshloka
{एतद्वै शालिहोत्रस्य तपसा निर्मितं सरः}
{रमणीयमिदं तोयं क्षुत्पिपासाश्रमापहम्}


\twolineshloka
{कार्यार्थिनस्तु षण्मासान्विहरध्वं यथासुखम् ॥वैशंपायन उवाच}
{}


\twolineshloka
{एवं स तान्समाश्वास्य व्यासः पार्थानरिन्दमान्}
{स्नेहाच्च संपरिष्वज्य कुन्तीमाश्वासयत्प्रभुः}


\twolineshloka
{स्नुषे मा रोद मा रोदेत्येवं व्यासोऽब्रवीद्वचः}
{जीवपुत्रे सुतस्तेऽयं धर्मनित्यो युधिष्ठिरः}


\twolineshloka
{पृथिव्यां पार्थइवान्सर्वान्प्रशासिष्यति धर्मराट्}
{धर्मेण जित्वा पृथिवीमखिलां धर्मकृद्वशी}


\twolineshloka
{स्थापयित्वा वशे सर्वां सपर्वतवनां शुभाम्}
{भीमसेनार्जुनबलाद्भोक्ष्यत्ययमसंशयम्}


\twolineshloka
{पुत्रास्तव च माद्र्याश्च पञ्चैते लोकविश्रुताः}
{स्वराष्ट्रे विहरिष्यन्ति सुखं सुमनसस्तदा}


\twolineshloka
{यक्ष्यन्ति च नरव्याघ्रा विजित्य पृथिवीमिमाम्}
{राजसूयाश्वमेधाद्यैः क्रतुभिर्भूरिदक्षिणैः}


\twolineshloka
{अनुगृह्य सुहृद्वर्गं धनेन च सुखेन च}
{पितृपैतामहं राज्यमहारिष्यन्ति ते सुताः}


\twolineshloka
{स्नुषा कमलपत्राक्षी नाम्ना कमलपालिका}
{वशवर्तिनी तु भीमस्य पुत्रमेषा जनिष्यति}


\twolineshloka
{तेन पुत्रेण कृच्छ्रेषु भविष्यथ च तारिताः}
{इह मासं प्रतीक्षध्वमागमिष्याम्यहं पुनः}


\twolineshloka
{देशकालौ विदित्वैवं यास्यध्वं परमां मुदम् ॥वैशंपायन उवाच}
{}


\twolineshloka
{स तैः प्राञ्जलिभिः सर्वैस्तथेत्युक्तो जनाधिप}
{जगाम भगवान्व्यासो यथागतमृषिः प्रभुः}


\chapter{अध्यायः १६९}
\twolineshloka
{वैशंपायन उवाच}
{}


\twolineshloka
{गते भगवति व्यासे पाण्डवा विगतज्वराः}
{ऊषुस्तत्र च षण्मासान्वटवृक्षे यथासुखम्}


\twolineshloka
{शाकमूलफलाहारास्तपः कुर्वन्ति पाण्डवाः}
{अनुज्ञाता महाराज ततः कमलपालिका}


\twolineshloka
{रमयन्ती सदा भीमं तत्रतत्र मनोजवा}
{दिव्याभरणवस्त्रा हि दिव्यस्रगनुलेपना}


\twolineshloka
{एवं भ्रातॄन्सप्त मासान्हिडिम्बाऽवासयद्वने}
{पाण्डवान्भीमसेनार्थे राक्षसी कामरूपिणी}


\twolineshloka
{सुखं स विहरन्भीमस्तत्कालं पर्यणामयत्}
{ततोऽलभत सा गर्भं राक्षसी कामरूपिणी}


\twolineshloka
{अतृप्ता भीमसेनेन सप्तमासोपसंगता}
{'प्रजज्ञे राक्षसी पुत्रं भीमसेनान्महाबलात्}


\twolineshloka
{विरूपाक्षं महावक्त्रं शङ्कुकर्णं विभीषणम्}
{भीमरूपं सुताम्राक्षं तीक्ष्णदंष्ट्रं महारथम्}


\twolineshloka
{महेष्वासं महावीर्यं महासत्वं महाजवम्}
{महाकायं महाकालं महाग्रीवं महाभुजम्}


\twolineshloka
{अमानुषं मानुषजं भीमवेगमरिन्दमम्}
{पिशाचकानतीत्यान्यान्बभूवाति स मानुषान्}


\twolineshloka
{बालोऽपि विक्रमं प्राप्तो मानुषेषु विशांपते}
{सर्वास्त्रेषु वरो वीरः प्रकाममभवद्बली}


\twolineshloka
{सद्यो हि गर्भं राक्षस्यो लभन्ते प्रसवन्ति च}
{कामरूपधराश्चैव भवन्ति बहुरूपिकाः}


\twolineshloka
{प्रणम्य विकचः पादावगृह्णात्स पितुस्तदा}
{मातुश्च परमेष्वासस्तौ च नामास्य चक्रतुः}


\twolineshloka
{घटोहास्योत्कच इति माता तं प्रत्यभाषत}
{अभवत्तेन नामास्य घटोत्कच इति स्म ह}


\twolineshloka
{अनुरक्तश्च तानासीत्पाण्डवान्स घटोत्कचः}
{तेषां च दयितो नित्यमात्मनित्यो बभूव ह}


\twolineshloka
{घटोत्कचो महाकायः पाण्डवान्पृथया सह}
{अभिवाद्य यथान्यायमब्रवीच्च प्रभाष्य तान्}


\twolineshloka
{किं करोम्यहमार्याणां निःशङ्कं वदतानघाः}
{तं ब्रुवन्तं भैमसेनिं कुन्ती वचनमब्रवीत्}


\threelineshloka
{त्वं कुरूणां कुले जातः साक्षाद्भीमसमो ह्यसि}
{ज्येष्ठः पुत्रोसि पञ्चानां साहाय्यं कुरु पुत्रक ॥वैशंपायन उवाच}
{}


\threelineshloka
{पृथयाप्येवमुक्तस्तु प्रणम्यैव वचोऽब्रवीत्}
{यथा हि रावणो लोके इन्द्रजिच्च महाबलः}
{वर्ष्मवीर्यसमो लोके विशिष्टश्चाभवं नृषु}


\twolineshloka
{कृत्यकाल उपस्थास्ये पितॄनिति घटोत्कचः}
{आमन्त्र्य रक्षसां श्रेष्ठः प्रतस्थे चोत्तरां दिशम्}


\threelineshloka
{स हि सृष्टो भगवता शक्तिहेतोर्महात्मना}
{वर्णस्याप्रतिवीर्यस्य प्रतियोद्धा महारथः ॥`भीम उवाच}
{}


\threelineshloka
{सह वासो मया जीर्णस्त्वया कमलपालिके}
{पुनर्द्रक्ष्यसि राज्यस्थानित्यभाषत तां तदा ॥हिडिम्बोवाच}
{}


\twolineshloka
{पदा मां संस्मरेः कान्त रिरंसू रहसि प्रभो}
{तदा तव वशं भूय आगन्तास्म्याशु भारत}


\twolineshloka
{इत्युक्त्वा सा जगामाशु भावमासज्य पाण्डवे}
{हिडिम्बा समयं स्मृत्वा स्वां गतिं प्रत्यपद्यत}


\chapter{अध्यायः १७०}
\twolineshloka
{`वैशंपायन उवाच}
{}


\twolineshloka
{ततस्ते पाण्डवाः सर्वे शालिहोत्राश्रमे तदा}
{पूजितास्तेन वन्येन तमामन्त्र्य महामुनिम्}


\twolineshloka
{जटाः कृत्वाऽऽत्मनः सर्वे वल्कलाजिनवाससः}
{कुन्त्या सह महात्मानो बिभ्रतस्तापसं वपुः}


\twolineshloka
{ब्राह्मं वेदमधीयाना वेदाङ्गानि च सर्वशः}
{नीतिशास्त्रं च धर्मज्ञा न्यायज्ञानं च पाण्डवाः}


\threelineshloka
{शालिहोत्रप्रसादेन लब्ध्वा प्रीतिमवाप्य च}
{'ते वनेन वनं गत्वा घ्नन्तो मृगगणान्बहून्}
{अपक्रम्य ययू राजंस्त्वरमाणा महारथाः}


\twolineshloka
{मत्स्यांस्त्रिगर्तान्पाञ्चालान्कीचकानन्तरेण च}
{रमणीयान्वनोद्देशान्प्रेक्षमाणाः सरांसि च}


\twolineshloka
{क्वचिद्वहन्तो जननीं त्वरमाणा महारथाः}
{`क्वचिच्छ्रान्ताश्च कान्तारे क्वचित्तिष्ठन्ति हर्षिताः'}


\twolineshloka
{क्वचिच्छन्देन गच्छन्तस्ते जग्मुः प्रसभं पुनः}
{पथि द्वैपायन सर्वे ददृशुः स्वपितामहम्}


\threelineshloka
{तेऽभिवाद्य महात्मानं कृष्णद्वैपायनं तदा}
{तस्थुः प्राञ्जलयः सर्वे सह मात्रा परन्तपाः ॥व्यास उवाच}
{}


\twolineshloka
{तदाश्रमान्निर्गमनं मया ज्ञातं नरर्षभाः}
{घटोत्कचस्य चोत्पत्तिं ज्ञात्वा प्रीतिरवर्धत}


\threelineshloka
{इदं नगरमभ्याशे रमणीयं निरामयम्}
{वसतेह प्रतिच्छन्ना ममागमनकाङ्क्षिणः ॥वैशंपायन उवाच}
{}


\threelineshloka
{एवं स तान्समाश्वास्य व्यासः पार्तानरिन्दमान्}
{एकचक्रामभिगतां कुन्तीमाश्वासयत्प्रभुः ॥श्रीव्यास उवाच}
{}


\twolineshloka
{कुर्यान्न केवलं धर्मं दुष्कृतं च तथान नरः}
{सुकृतं दुष्कृतं लोके न कर्ता नास्ति कश्चन}


\twolineshloka
{अवश्यं लभते कर्ता फलं वै पुण्यपापयोः}
{दुष्कृतस्य फलेनैव प्राप्तं व्यसनमुत्तमम्}


\threelineshloka
{तस्मान्माधवि मानार्हे मा च शोके मनः कृथाः}
{वैशंपायन उवाच}
{एवमुक्त्वा निवेश्यैनान्ब्राह्मणस्य निवेशने}


% Check verse!
जगाम भगवान्व्यासो यताकाममृषिः प्रभुः
\chapter{अध्यायः १७१}
\twolineshloka
{जनमेजय उवाच}
{}


\threelineshloka
{एकचक्रां गतास्ते तु कुन्तीपुत्रा महारथाः}
{अत ऊर्ध्वं द्विजश्रेष्ठ किमकुर्वत पाण्डवाः ॥वैशंपायन उवाच}
{}


\twolineshloka
{एकचक्रां गतास्ते तु कन्तीपुत्रा महारथाः}
{ऊषुर्नातिचिरं कालं ब्राह्मणस्य निवेशने}


\twolineshloka
{रमणीयानि पश्यन्तो वनानि विविधानि च}
{पार्थिवानपि चोद्देशान्सरितश्च सरांसि च}


\twolineshloka
{चेरुर्भैश्रं तदा ते तु सर्व एव विशांपते}
{`युधिष्ठिरं च कुन्तीं च चिन्तयन्त उपासते}


\threelineshloka
{भैक्षं चरन्तस्तु तदा जटिला ब्रह्मचारिणः}
{बभूवुर्नागराणां च गुणैः संप्रियदर्शनाः ॥नागरा ऊचुः}
{}


\twolineshloka
{दर्शनीया द्विजाः शुभ्रा देवगर्भोपमाः शुभाः}
{भैक्षानर्हाश्च राज्यार्हाः सुकुमारास्तपस्विनः}


\twolineshloka
{नैते यथार्थतो विप्राः सुकुमारास्तपस्विनः}
{चरन्ति भूमौ प्रच्छन्नाः कस्माच्चित्कारणादिह}


\twolineshloka
{सर्वलक्षणसंपन्ना भैक्षं नार्हन्ति नित्यशः}
{कार्यार्थिनश्चरन्तीति तर्कयन्त इति ब्रुवन्}


\twolineshloka
{बन्धूनामागमान्नित्यमुपचारैस्तु नागराः}
{भाजनानि च पूर्णानि भक्ष्यभोज्यैरकारयन्}


\twolineshloka
{मौनव्रतेन संयुक्ता भैश्रं गृह्णन्ति पाण्डवाः}
{माता चिरगतान्ज्ञात्वा शोचन्ती पाण्डवान्प्रति}


\twolineshloka
{दुःखाश्रुपूर्णनयना लिखन्त्यास्ते महीतलम्}
{भिक्षित्वा द्विजगेहेषु चिन्तयन्तश्च मातरम्}


\twolineshloka
{त्वरमाणा निवर्तन्ते मातृगौरवयन्त्रिताः}
{मात्रे निवेदयन्ति स्म कुन्त्यै भैक्षं दिवानिशम्}


\twolineshloka
{सर्वं संपूर्णभैक्षान्नं मातृदत्तं पृथक्पृथक्}
{विभज्याभुञ्जतेष्टं ते यथाभागं पृथक्पृथक्}


\twolineshloka
{अर्धं स्म भुञ्जते पञ्च सह मात्रा परन्तपाः}
{अर्धं सर्वस्य भैक्षस्य भीमो भुङ्क्ते महाबलः}


\twolineshloka
{स नाशितश्च भवति कल्याणान्नभुजिः पुरा}
{स वैवर्ण्यं च कार्श्यं च जगामातृप्तिकारितम्}


\twolineshloka
{आज्यबिन्दुर्यथा वह्नौ महति ज्वलिते भवेत्}
{तथार्धभागं भीमस्य भिक्षान्नस्य नरोत्तम}


\twolineshloka
{तथैव वसतां तत्र तेषां राजन्महात्मनाम्}
{अतिचक्राम सुमहान्कालोऽथ भरतर्षभ}


\twolineshloka
{भीमोऽपि क्रीडयित्वाथ मिथो ब्राह्मणबन्धुषु}
{कुम्भकारेण संबन्धाल्लेभे पात्रं महत्तरम्}


\twolineshloka
{कुम्भकारोऽददात्पात्रं महत्कृत्वातिमात्रकम्}
{प्रहसन्भीमसेनाय विस्मितस्तस्य कर्मणा}


\twolineshloka
{तस्याद्भुतं कर्म कुर्वन्मृद्भारं महदाददे}
{मृद्भारैः शतसाहस्रैः कुम्भकारमतोषयत्}


\twolineshloka
{चक्रे चक्रे च मृद्भाण्डान्सततं भैक्षमाचरन्}
{तदादायागतं दृष्ट्वा हसन्ति प्रहसन्ति च}


\threelineshloka
{भक्ष्यभोज्यानि विविधान्यादाय प्रक्षिपन्ति च}
{एवमेव सदा भुक्त्वा मात्रे वदति वै रहः}
{न चाशितोऽस्मि भवति कल्याणान्नभृतः पुरा ॥'}


\twolineshloka
{ततः कदाचिद्भैक्षाय गतास्ते पुरषर्षभाः}
{संगत्य भीमसेनस्तु तत्रास्ते पृथया सह}


\twolineshloka
{अथार्तिजं महाशब्दं ब्राह्मणस्य निवेशने}
{भृशमुत्पतितं घोरं कुन्ती शुश्राव भारत}


\twolineshloka
{रोरुयमाणांस्तान्दृष्ट्वा परिदेवयतश्च सा}
{कारुण्यात्साधुभावाच्च कुन्ती राजन्न चक्षमे}


\twolineshloka
{मथ्यमानेव दुःखेन हृदयेन पृथा तदा}
{उवाच भीमं कल्याणी कृपान्वितमिदं वचः}


\twolineshloka
{वसामः सुसुखं पुत्र ब्राह्मणस्य निवेशने}
{अज्ञाता धार्तराष्ट्रस्य सत्कृतां वीतमन्यवः}


% Check verse!
सा चिन्तये सदा पुत्र ब्राह्मणस्यास्य किं न्वहम् ॥कदा प्रियं करिष्यामि यत्कुर्युरुषिताः सुखम्
\twolineshloka
{एतावान्पुरुषस्तात कृतं यस्मिन्न नश्यति}
{यावच्च कुर्यादन्योऽस्य कुर्यादभ्यधिकं ततः}


\threelineshloka
{तदिदं ब्राह्मणस्यास्य दुःखमापतितं ध्रुवम्}
{न तत्र यदि साहाय्यं कुर्याम सुकृतं भवेत् ॥भीमसेन उवाच}
{}


\threelineshloka
{ज्ञायतामस्य यद्दुःखं यतश्चैव समुत्थितम्}
{विदित्वा व्यवसिष्यामि यद्यपि स्यात्सुदुष्करम् ॥वैशंपायन उवाच}
{}


\twolineshloka
{एवं तौ कथयन्तौ च भूयः सुश्रुवतुः स्वनम्}
{आर्तिजं तस्य विप्रस्य सभार्यस्य विशांपते}


\twolineshloka
{अन्तःपुरं ततस्तस्य ब्राह्मणस्य महात्मनः}
{विवेश त्वरिता कुन्ती बद्धवत्सेव सौरभी}


\threelineshloka
{ततस्तं ब्राह्मणं तत्र भार्यया च सुतेन च}
{दुहित्रा चैव सहितं ददर्श विकृताननम् ॥ब्राह्मण उवाच}
{}


\twolineshloka
{धिगिदं जीवितं लोके गतसारमनर्थकम्}
{दुःखमूलं पराधीनं भृशमप्रियभागि च}


\twolineshloka
{जीविते परमं दुःखं जीविते परमो ज्वरः}
{जीविते वर्तमानस्य द्वन्द्वानामागमो ध्रुवः}


\twolineshloka
{आत्मा ह्येको हि धर्मार्थौ कामं चैव निषेवते}
{एतैश्च विप्रयोगोऽपि दुःखं परमनन्तकम्}


\twolineshloka
{आहुः केचित्परं मोक्षं स च नास्ति कथंचन}
{अर्थप्राप्तौ तु नरकः कृत्स्न एवोपपद्यते}


\twolineshloka
{अर्थेप्सुता परं दुःखमर्थप्राप्तौ ततोऽधिकम्}
{जातस्नेहस्य चार्थेषु विप्रयोगे महत्तरम्}


\twolineshloka
{`यावन्तो यस्य संयोगा द्रव्यैरिष्टैर्भवन्त्युत}
{तावन्तोऽस्य निख्यन्ते हृदये शोकशङ्कवः}


\twolineshloka
{तदिदं जीवितं प्राप्य अल्पकालं महाभयम्}
{त्यागो हि न मया प्राप्तो भार्यया सहितेन च ॥'}


\twolineshloka
{न हि योगं प्रपश्यामि येन मुच्येयमापदः}
{पुत्रदारेण वा सार्धं प्राद्रवेयमनामयम्}


\twolineshloka
{यतितं वै मया पूर्वं वेत्थ ब्राह्मणि तत्तथा}
{क्षेमं यतस्ततो गन्तुं त्वया तु मम न श्रुतम्}


\twolineshloka
{इह जाता विवृद्धाऽस्मि पिता माता ममेति वै}
{उक्तवत्यसि दुर्मेधे याच्यमाना मयाऽसकृत्}


\twolineshloka
{स्वर्गतोऽपि पिता वृद्धस्तथा माता चिरं तव}
{बन्धवा भूतपूर्वाश्च तत्र वासे तु का रतिः}


\twolineshloka
{`न भोजनं विरुद्धं स्यान्न स्त्रीदेशो निबन्धनः}
{सुदूरमपि तं देशं व्रजेद्गरुडहंसवत् ॥'}


\twolineshloka
{सोऽयं ते बन्धुकामाया अशृण्वन्त्या वचो मम}
{बन्धुप्रणाशः संप्राप्तो भृशं दुःखकरो मम}


\twolineshloka
{अथवा मद्विनाशो यं न हि शक्ष्यामि कंचन}
{परित्यक्तुमहं बन्धुं स्वयं जीवन्नृशंसवत्}


\twolineshloka
{सहधर्मचरीं दान्तां नित्यं मातृसमां मम}
{सखायं विहितां देवैर्नित्यं परमिकां गतिम्}


\twolineshloka
{पित्रा मात्रा च विहितां सदा गार्हस्थ्यभागिनीम्}
{वरयित्वा यथान्यायं मन्त्रवत्परिणीय च}


\twolineshloka
{कुलीनां शीलसंपन्नामपत्यजननीमपि}
{त्वामहं जीवितस्यार्थे साध्वीमनपकारिणीम्}


\twolineshloka
{परित्यक्तुं न शक्ष्यामि भार्यां नित्यमनुव्रताम्}
{कुत एव परित्यक्तुं सुतां शक्ष्याम्यहं स्वयम्}


\twolineshloka
{बालामप्राप्तवयसमजातव्यञ्जनाकृतिम्}
{भर्तुरर्थाय निक्षिप्तां न्यासं धात्रा महात्मना}


\twolineshloka
{यया दौहित्रजाँल्लोकानाशंसे पितृभिः सह}
{स्वयमुत्पाद्य तां बालां कथमुत्स्रष्टुमुत्सहे}


\twolineshloka
{मन्यन्ते केचिदधिकं स्नेहं पुत्रे पितुर्नराः}
{कन्यायां केचिदपरे मम तुल्यावुभौ स्मृतौ}


\twolineshloka
{यस्यां लोकाः प्रसूतिश्च स्थिता नित्यमथो सुखम्}
{अपापां तामहं बालां कथमुत्स्रष्टुमुत्सहे}


\twolineshloka
{`कुत एव परित्यक्तुं सुतं शक्ष्याम्यहं स्वयम्}
{प्रार्थयेयं परां प्रीतिं यस्मिन्स्वर्गफलानि च}


\twolineshloka
{यस्य जातस्य पितरो मुखं दृष्ट्वा दिवं गताः}
{अहं मुक्तः पितृऋणाद्यस्य जातस्य तेजसा}


\twolineshloka
{दयितं मे कथं बालमहं त्यक्तुमिहोत्सहे}
{तमहं ज्येष्ठपुत्रं मे कुलनिर्हारकं विभुम्}


\twolineshloka
{मम पिण्डोदकनिधिं कथं त्यक्ष्यामि पुत्रकम्}
{त्यागोऽयं मम संप्राप्तो ममन्वा मे सुतस्य वा}


\twolineshloka
{तव वा तव पुत्र्या वा अत्र वासस्य तत्फलम्}
{न शृणोषि वचो मह्यं तत्फलं भुङ्क्ष्व भामिनि}


\twolineshloka
{अथवाहं न शक्ष्यामि स्वयं मर्तुं सुतं मम}
{एकं त्यक्तुं न शक्नोति भवतीं च सुतामपि}


\twolineshloka
{अथ मद्रक्षणार्थं वा न हि शक्ष्यामि कंचन}
{परित्यक्तुमहं बन्धुं स्वयं जीवन्नृशंसवत्}


\twolineshloka
{आत्मानमपि चोत्सृज्य गते प्रेतवशं मयि}
{'त्यक्ता ह्येते मया व्यक्तं नेह शक्ष्यन्ति जीवितुम्}


\twolineshloka
{एषां चान्यतमत्यागो नृशंसो गर्हितो बुधैः}
{आत्मत्यागे कृते चेमे मरिष्यन्ति मया विना}


\threelineshloka
{स कृच्छ्रामहमापन्नो न शक्तस्तर्तुमापदम्}
{अहो धिक्कां गतिं त्वद्य गमिष्यामि सबान्धवः}
{सर्वैः सह मृतं श्रेयो न च मे जीवितुं क्षमम्}


\chapter{अध्यायः १७२}
\twolineshloka
{ब्राह्मण्युवाच}
{}


\twolineshloka
{न सन्तापस्त्वया कार्यः प्राकृतेनेव कर्हिचित्}
{न हि सन्तापकालोऽयं वैद्यस्य तव विद्यते}


\twolineshloka
{अवश्यं निधनं सर्वैर्गन्तव्यमिह मानवैः}
{अवश्यभाविन्यर्थे वै संतापो नेह विद्यते}


\twolineshloka
{भार्या पुत्रोऽथ दुहिता सर्वमात्मार्थमिष्यते}
{व्यथां जहि सुबुद्ध्या त्वं स्वयं यास्यामि तत्र च}


\twolineshloka
{एतद्धि परमं नार्याः कार्यं लोके सनातनम्}
{प्राणानपि परित्यज्य यद्भर्तुर्हितमाचरेत्}


\twolineshloka
{तच्च तत्र कृतं कर्म तवापीदं सुखावहम्}
{भवत्यमुत्र चाक्षय्यं लोकेऽस्मिंश्च यशस्करम्}


\twolineshloka
{एष चैव गुरुर्धर्मो यं प्रवक्ष्याम्यहं तव}
{अर्थश्च तव धर्मश्च भूयानत्र प्रदृश्यते}


\twolineshloka
{यदर्थमिष्यते भार्या प्राप्तः सोऽर्थस्त्वया मयि}
{कन्या चैका कुमारश्च कृताहमनृणा त्वया}


\twolineshloka
{समर्थः पोषणे चासि सुतयो रक्षणे तथा}
{न त्वहं सुतयोः शक्ता तथा रक्षणपोषणे}


\twolineshloka
{मम हि त्वद्विहीनायाः सर्वप्राणधनेश्वर}
{कथं स्यातां सुतौ बालौ भरेयं च कथं त्वहम्}


\twolineshloka
{कथं हि विधवाऽनाथा बालपुत्रा विना त्वया}
{मिथुनं जीवयिष्यामि स्थिता साधुगते पथि}


\twolineshloka
{अहं कृतावलेपैश्च प्रार्थ्यमानामिमां सुताम्}
{अयुक्तैस्तव संबन्धे कथं शक्ष्यामि रक्षितुम्}


\twolineshloka
{उत्सृष्टमामिषं भूमौ प्रार्थयन्ति यथा खगाः}
{प्रार्थयन्ति जनाः सर्वे पतिहीनां तथा स्त्रियम्}


\twolineshloka
{साऽहं विचाल्यमाना वै प्रार्थ्यमाना दुरात्मभिः}
{स्थातुं पथि न शक्ष्यामि सज्जनेष्टे द्विजोत्तम}


\twolineshloka
{`स्त्रीजन्म गर्हितं नाथ लोके दुष्टजनाकुले}
{मातापित्रोर्वशे कन्या प्रौढा भर्तृवशे तथा}


% Check verse!
अभावे चानयोः पुत्रे खतन्त्रा स्त्री विगर्हिता
\twolineshloka
{अनाथत्वं स्त्रियो द्वारं दुष्टानां विवृतं हि तत्}
{वस्त्रखण्डं घृताक्तं हि यथा संकृष्यते श्वभिः ॥'}


\twolineshloka
{कथं तव कुलस्यैकमिमं बालमनागसम्}
{पितृपैतामहे मार्गे नियोक्तुमहमुत्सहे}


\twolineshloka
{कथं शक्ष्यामि बालेऽस्मिन्गुणानाधातुमीप्सितान्}
{अनाथे सर्वतो लुप्ते यथा त्वं धर्मदर्शिवान्}


\twolineshloka
{इमामपि च ते बालामनाथां परिभूय माम्}
{अनर्हाः प्रार्थयिष्यन्ति शूद्रा वेदश्रुतिं यथा}


\twolineshloka
{तां चेदहं न दित्सेयं सद्गुणैरुपबृंहिताम्}
{प्रमथ्यैनां हरेयुस्ते हविर्ध्वाङ्क्षा इवाध्वरात्}


\twolineshloka
{संप्रेक्षमाणा पुत्रीं ते नानुरूपमिवात्मनः}
{अनर्हवशमापन्नामिमां चापि सुतां तव}


\twolineshloka
{अवज्ञाता च लोकेषु तथान्मानमजानती}
{अवलिप्तैरैर्ब्रह्मन्मरिष्यामि न संशयः}


\twolineshloka
{तौ च हीनौ मया बालौ त्वया चैव तथात्मजौ}
{विनश्येतां न सन्देहो मत्स्याविव जलक्षये}


\twolineshloka
{त्रितयं सर्वथाप्येवं विनशिष्यत्यसंशयम्}
{त्वया विहीनं तस्मात्त्वं मां परित्यक्तुमर्हसि}


\twolineshloka
{व्युष्टिरेषा परा स्त्रीणां पूर्वं भर्तुः परा गतिः}
{ननु ब्रह्मन्सपुत्राणामिति धर्मविदो विदुः}


\twolineshloka
{`अनिष्टमिह पुत्राणां विषये परिवर्तितुम्}
{हरिद्राञ्जनपुष्पादिसौमङ्गल्ययुता सती}


\twolineshloka
{मरणं याति या भर्तुस्तद्दत्तजलपायिनी}
{भर्तृपादार्पितमनाः सा याति गिरिजापदम्}


\twolineshloka
{गिराजायाः सखी भूत्वा मोदते नगकन्यया}
{मितं ददाति हि पिता मितं माता मितं सुतः}


\twolineshloka
{अमितस्य हि दातारं का पतिं नाभिनन्दति}
{आश्रमाश्चाग्निसंस्कारा जपहोमव्रतानि च}


\twolineshloka
{स्त्रीणां नैते विधातव्या विना पतिमनिन्दितम्}
{क्षमा शौचमनाहारमेतावद्विहितं स्त्रियाः ॥'}


\twolineshloka
{परित्यक्तः सुतश्चायं दुहितेयं तथा मया}
{बान्धवाश्च परित्यक्तास्त्वदर्थं जीवितं च मे}


\twolineshloka
{यज्ञैस्तपोभिर्नियमैर्दानैश्च विविधैस्तथा}
{विशिष्यते स्त्रिया भर्तुर्नित्यं प्रियहिते स्थितिः}


\twolineshloka
{तदिदं यच्चिकीर्षामि धर्मं परमसंमतम्}
{इष्टं चैव हितं चैव तव चैव कुलस्य च}


\twolineshloka
{इष्टानि चाप्यपत्यानि द्रव्याणि सुहृदः प्रियाः}
{आपद्धर्मप्रमोक्षाय भार्या चापि सतां मतम्}


\twolineshloka
{आपदर्थे धनं रक्षेद्दारान्रक्षेद्धनैरपि}
{आत्मानं सततं रक्षेद्दारैरपि धनैरपि}


\twolineshloka
{दृष्टादृष्टफलार्थं हि भार्या पुत्रो धनं गृहम्}
{सर्वमेतद्विधातव्यं बुधानामेष निश्चयः}


\twolineshloka
{एकतो वा कुलं कृत्स्नमात्मा वा कुलवर्धनः}
{`उभयोः कोधिको विद्वन्नात्मा चैवाधिकः कुलात्}


\twolineshloka
{आत्मनो विद्यमानत्वाद्भुवनानि चतुर्दश}
{विद्यन्ते द्विजशार्दूल आतमा रक्ष्यस्ततस्त्वया}


\twolineshloka
{आत्मन्यविद्यमाने चेदस्य नास्तीह किंचन}
{एतज्जगदिदं सर्वमात्मना न समं किल ॥'}


\twolineshloka
{स कुरुष्व मया कार्यं तारयात्मानमात्मना}
{अनुजानीही मामार्य सुतौ मे परिपालय}


\twolineshloka
{अवध्याः स्त्रिय इत्याहुर्धर्मज्ञा धर्मनिश्चये}
{धर्मज्ञान्राक्षसानाहुर्न हन्यात्स च मामपि}


\twolineshloka
{निःसंशयं वधः पुंसां स्त्रीणां संशयितो वधः}
{अतो मामेव धर्मज्ञ प्रस्थापयितुमर्हसि}


\threelineshloka
{भुक्तं प्रियाण्यवाप्तानि धर्मश्च चरितो मया}
{`त्वच्छुश्रूषणसंभूता कीर्तिश्चाप्यतुला मम}
{'त्वत्प्रसूतिः प्रिया प्राप्ता न मां तप्स्यत्यजीवितं}


\twolineshloka
{जातपुत्रा च वृद्धा च प्रियकामा च ते सदा}
{समीक्ष्यैतदहं सर्वं व्यवसायं करोम्यतः}


\twolineshloka
{उत्सृज्यापि हि मामार्य प्राप्स्यस्यन्यामपि स्त्रियम्}
{ततः प्रतिष्ठितो धर्मो भविष्यति पुनस्तव}


\twolineshloka
{न चाप्यधर्मः कल्याण बहुपत्नीकता नृणाम्}
{स्त्रीणामधर्मः सुमहान्भर्तुः पूर्वस्य लङ्घने}


\threelineshloka
{एतत्सर्वं समीक्ष्य त्वमात्मत्यागं च गर्हितम्}
{आत्मानं तारयाद्याशु कुलं चेमौ च दारकौ ॥वैशंपायन उवाच}
{}


\twolineshloka
{एवमुक्तस्तया भर्ता तां समालिङ्ग्य भारत}
{मुमोच बाष्पं शनकैः सभार्यो भृशदुःखितः}


\twolineshloka
{`मैवं वद त्वं कल्याणि तिष्ठ चेह सुमध्यमे}
{न तु भार्यां त्यजेत्प्राज्ञः पुत्रान्वापि कदाचन}


\threelineshloka
{विशेषतः स्त्रियं रक्षेत्पुरुषो बुद्धिमानिह}
{त्यक्त्वा तु पुरुषो जीवेन्न हातव्यानिमान्सदा}
{न वेत्ति कामं धर्मं च अर्थं मोक्षं च तत्त्वतः ॥'}


\chapter{अध्यायः १७३}
\twolineshloka
{वैशंपायन उवाच}
{}


\twolineshloka
{तयोर्दुःखितयोर्वाक्यमतिमात्रं निशम्य तु}
{ततो दुःखपरीताङ्गी कन्या तावभ्यभाषत}


\twolineshloka
{किमेवं भृशदुःखार्तौ रोरूयेतामनाथवत्}
{ममापि श्रूयतां वाक्यं श्रुत्वा च क्रियतां क्षमम्}


\twolineshloka
{धर्मतोऽहं परित्याज्या युवयोर्नात्र संशयः}
{त्यक्तव्यां मां परित्यज्य त्राहि सर्वं मयैकया}


\twolineshloka
{इत्यर्थमिष्यतेऽपत्यं तारयिष्यति मामिति}
{अस्मिन्नुपस्थिते काले तरध्वं प्लववन्मया}


\twolineshloka
{इह वा तारयेद्दुर्गादुत वा प्रेत्य भारत}
{सर्वथा तारयेत्पुत्रः पुत्र इत्युच्यते बुधैः}


\twolineshloka
{आकाङ्क्षन्ते च दौहित्रान्मयि नित्यं पितामहाः}
{तत्स्वयं वै परित्रास्ये रक्षन्ती जीवितं पितुः}


\twolineshloka
{भ्राता च मम बालोऽयं गते लोकममुं त्वयि}
{अचिरेणैव कालेन विनश्येत न संशयः}


\twolineshloka
{तातेपि हि गते स्वर्गं विनष्टे च ममानुजे}
{पिण्डः पितॄणां व्युच्छिद्येत्तत्तेषां विप्रियं भवेत्}


\twolineshloka
{पित्रा त्यक्ता तथा मात्रा भ्रात्रा चाहमसंशयम्}
{दुःखाद्दुःखतरं प्राप्य म्रियेयमतथोचिताम्}


\twolineshloka
{त्वयि त्वरोगे निर्मुक्तो माता भ्राता च मे शिशुः}
{सन्तानश्चैव पिण्डश्च प्रतिष्ठास्यत्यसंशयम्}


\twolineshloka
{आत्मा पुत्रः सखी भार्या कृच्छ्रं तु दुहिता किल}
{स कृच्छ्रान्मोचयात्मानं मां च धर्मे नियोजया}


\twolineshloka
{अनाथा कृपणा बाला यत्र क्वचन गामिनी}
{भविष्यामि त्वया तात विहीना कृपणा सदा}


\twolineshloka
{अथवाहं करिष्यामि कुलस्यास्य विमोचनम्}
{फलसंस्था भविष्यामि कृत्वा कर्म सुदुष्करम्}


\twolineshloka
{अथवा यास्यसे तत्र त्यक्त्वा मां द्विजसत्तम}
{पीडिताऽहं भविष्यामि तदवेक्षस्व मामपि}


\twolineshloka
{तदस्मदर्थं धर्मार्थं प्रसवार्थं च सत्तम}
{आत्मानं परिरक्षस्व त्यक्तव्यां मां च संत्यज}


\twolineshloka
{अवश्यकरणीये च मा त्वां कालोत्यगादयम्}
{किं त्वतः परमं दुःखं यद्वयं स्वर्गते त्वयि}


\threelineshloka
{याचमानाः परादन्नं परिधावेमहि श्ववत्}
{त्वयि त्वरोगे निर्मुक्ते क्लेशादस्मात्सबान्धवे}
{अमृतेव सती लोके भविष्यामि सुखान्विता}


\twolineshloka
{इतः प्रदाने देवाश्च पितरश्चेति नः श्रुतम्}
{त्वया दत्तेन तोयेन भविष्यति हिताय वै}


\twolineshloka
{`इत्येतदुभयं तात निशाम्य तव यद्धितम्}
{तद्व्यवस्य तथाम्बाया हितं स्वस्य सुतस्य च}


\threelineshloka
{मातापित्रोश्च पुत्रास्तु भवितारो गुणान्विताः}
{न तु पुत्रस्य पितरो पुनर्जातु भविष्यतः ॥'वैशंपायन उवाच}
{}


\twolineshloka
{एवं बहुविधं तस्या निशम्य परिदेवितम्}
{पिता माता च सा चैव कन्या प्ररुरुदुस्त्रयः}


\twolineshloka
{ततः प्ररुदितान्सर्वान्निशम्याथ सुतस्तदा}
{उत्फुल्लनयनो बालः कलमव्यक्तमब्रवीत्}


\twolineshloka
{मा पिता रुद मा मातर्मा स्वसस्त्विति चाब्रवीत्}
{प्रहसन्निव सर्वांस्तानेकैकमनुसर्पति}


\threelineshloka
{ततः स तृणमादाय प्रहृष्टः पुनरब्रवीत्}
{अनेनाहं हनिष्यामि राक्षसं पुरुषादकम् ॥वैशंपायन उवाच}
{}


\twolineshloka
{तथापि तेषां दुःखेन परीतानां निशम्य तत्}
{बालस्य वाक्यमव्यक्तं हर्षः समभवन्महान्}


\twolineshloka
{अयं काल इति ज्ञात्वा कुन्ती समुपसृत्य तान्}
{गतासूनमृतेनेव जीवयन्तीदमब्रवीत्}


\chapter{अध्यायः १७४}
\twolineshloka
{कुन्त्युवाच}
{}


\threelineshloka
{कुतोमूलमिदं दुःखं ज्ञातुमिच्छामि तत्त्वतः}
{विदित्वाप्यपकर्षेयं शक्यं चेदपकर्षितुम् ॥ब्राह्मण उवाच}
{}


\twolineshloka
{उपपन्नं सतामेतद्यद्ब्रवीपि तपोधने}
{न तु दुःखमिदं शक्यं मानुषेण व्यपोहितुम्}


\twolineshloka
{`तथापि तत्त्वमाख्यास्ये दुःखस्यैतस्य संभवम्}
{शक्यं वा यदि वाऽशक्यं शृणु भद्रे यथातथम्}


\threelineshloka
{समीपे नगरस्यास्य वको वसति राक्षसः}
{इतो गव्यूतिमात्रे}
{ञस्ति यमुनागह्वरे गुहा}


\twolineshloka
{तस्यां घोरः स वसति जिघांसुः पुरुषादकः}
{बकाभिधानो दुष्टात्मा राक्षसानां कुलाधमः}


\twolineshloka
{ईशो जनपदस्यास्य पुरस्य च महाबलः}
{पुष्टो मानुषमांसेन दुर्बुद्धिः पुरुषादकः}


\twolineshloka
{प्रबलः कामरूपी च राक्षसस्तु महाबलः}
{तेनोपसृष्टा नगरी वर्षमद्य त्रयोदशम्}


\twolineshloka
{तत्कृते परचक्राच्च भूतेभ्यश्च न नो भयम्}
{पुरुषादेन रौद्रेण भक्ष्यमाणा दुरात्मना}


\twolineshloka
{अनाथा नगरी नाथं त्रातारं नाधिगच्छति}
{गुहायां च वसंस्तत्र बाधते सततं जनम्}


\twolineshloka
{स्त्रियो बालांश्च वृद्धांश्च यूनश्चापि दुरात्मवान्}
{अत्र मन्त्रैश्च होमैश्च भोजनैश्च स राक्षसः}


\twolineshloka
{ईडितो द्विजमुख्यैश्च पूजितश्च दुरात्मवान्}
{यदा च सकलानेवं प्रसूदयति राक्षसः}


\twolineshloka
{अथैनं ब्राह्मणाः सर्वे समये समयोजयन्}
{मा स्म कामाद्वधी रक्षो दास्यामस्ते सदा वयं}


\twolineshloka
{पर्यायेण यथाकाममिह मांसोदनं प्रभो}
{अन्नं मांससमायुक्तं तिलचूर्णसमन्वितम्}


\twolineshloka
{सर्पिषा च समायुक्तं व्यञ्जनैश्च समन्वितम्}
{सूपांस्त्रीन्सतिलान्पिण्डाँल्लाजापूपसुरासवान्}


\twolineshloka
{शृताशृतान्पानकुम्भान्स्थूलमांसं शृताशृतम्}
{वनमाहिषवाराहभाल्लूकं च शृताशृतम्}


\twolineshloka
{सर्पिःकुम्भांश्च विविधान्दधिकुम्भांस्तथा बहून्}
{सद्यःसिद्धसमायुक्तं तिलचूर्णैः समाकुलम्}


\twolineshloka
{कुलाच्च पुरुषं चैकं बलीवर्दौ च कालकौ}
{प्राप्स्यसि त्वमसंक्रुद्धो रक्षोभागं प्रकल्पितम्}


\twolineshloka
{तिष्ठेह समयेऽस्माकमित्ययाचन्त तं द्विजाः}
{बाढमित्येव तद्रक्षस्तद्वचः प्रत्यगृह्णत}


\twolineshloka
{परचक्राटवीकेभ्यो रक्षणं स करोति च}
{तस्मिन्भागे सुनिर्दिष्टे स्थितः स समये बली}


\twolineshloka
{एकैकं चैव पुरुषं संप्रयच्छन्ति वेतनम्}
{स वारो बहुभिर्वर्षैर्भवत्यसुकरो नरैः}


\twolineshloka
{तद्विमोक्षाय ये केचिद्यतन्ते पुरुषाः क्वचित्}
{सपुत्रदारांस्तान्हत्वा तद्रक्षो भक्षयत्युत ॥'}


\threelineshloka
{वेत्रकीयगृहे राजा नायं नयमिहास्थितः}
{उपायं तं न कुरुते यत्नादपि स मन्दधीः}
{अनामयं जनस्यास्य येन स्यादद्य शाश्वतम्}


\twolineshloka
{एतदर्हा वचं नूनं वसामो दुर्बलस्य ये}
{विषये नित्यवास्तव्याः कुराजानमुपाश्रिताः}


\twolineshloka
{ब्राह्मणाः कस्य वक्तव्याः कस्य वाच्छन्दचारिणः}
{गुणैरेते हि वत्स्यन्ति कामगाः पक्षिणो यथा}


\threelineshloka
{राजानं प्रथमं विन्देत्ततो भार्यां ततो धनम्}
{`राजन्यसति लोकेऽस्मिन्कुतो भार्या कुतो धनम्}
{वयस्य संचयेनास्य ज्ञातीन्पुत्रांश्च तारयेत्}


\twolineshloka
{विपीरतं मया चेदं त्रयं सर्वमुपार्जितम्}
{तदिमामापदं प्राप्य भृशं तप्यामहे वयम्}


\twolineshloka
{सोऽयमस्माननुप्राप्तो वारः कुलविनाशनः}
{भोजनं पुरुषश्चैकः प्रदेयं वेतनं मया}


\twolineshloka
{न च मे विद्यते वित्तं संक्रेतुं पुरुषं क्वचित्}
{सुहृज्जनं प्रदातुं च न शक्ष्यामि कदाचन}


\twolineshloka
{गतिं चान्यां न पश्यामि तस्मान्मोक्षाय रक्षसः}
{सोऽहं दुःखार्णवे मग्नो महत्यसुकरे भृशम्}


\twolineshloka
{सहैवैतैर्गमिष्यामि बान्धवैरद्य राक्षसम्}
{ततो नः सहितान्क्षुद्रः सर्वानेवोपभोक्ष्यति}


% Check verse!
`दुःखमूलमिदं भद्रे मयोक्तं प्रश्नतोऽनघे ॥'
\chapter{अध्यायः १७५}
\twolineshloka
{कुन्त्युवाच}
{}


\twolineshloka
{न विषादस्त्वया कार्यो भयादस्मात्कथंचन}
{उपायः परिदृष्टोऽत्र तस्मान्मोक्षाय रक्षसः}


\threelineshloka
{`नैव स्वयं सपुत्रस्य गमनं तत्र रोचये}
{'एकस्तव सुतो बालः कन्या चैका तपस्विनी}
{न चैतयोस्तथा पत्न्या गमनं तव रोचये}


\threelineshloka
{मम पञ्च सुता ब्रह्मंस्तेषामेको गमिष्यति}
{त्वदर्थं बलिमादाय तस्य पापस्य रक्षसः ॥ब्राह्मण उवाच}
{}


\twolineshloka
{नाहमेतत्करिष्यामि जीवितार्थी कथंचन}
{ब्राह्मणस्यातिथेश्चैव स्वार्थे प्राणान्वियोजयन्}


\twolineshloka
{न त्वेतदकुलीनासु नाधर्मिष्ठासु विद्यते}
{यद्ब्राह्ममार्थं विसृजेदात्मानमपि चात्मजम्}


\twolineshloka
{आत्मनस्तु वधः श्रेयो बोद्धव्यमिति रोचते}
{ब्रहम्वध्याऽऽत्मवध्या वा श्रेयानात्मवधो मम}


\twolineshloka
{ब्रह्मवध्या परं पापं निष्कृतिर्नात्र विद्यते}
{अबुद्धिपूर्वं कृत्वापि प्रत्यवायो हि विद्यते}


\twolineshloka
{न त्वहं वधमाकाङ्क्षे स्वयमेवात्मनः शुभे}
{परैः कृते वधे पापं न किंचिन्मयि विद्यते}


\twolineshloka
{अभिसंधौ कृते तस्मिन्ब्राह्मणस्य वधे मया}
{निष्कृतिं न प्रपश्यामि नृशंसं क्षुद्रमेव च}


\twolineshloka
{आगतस्य गृहं त्यागस्तथैव शरणार्थिनः}
{याचमानस्य च वधो नृशंसो गर्हितो बुधैः}


\twolineshloka
{कुर्यान्न निन्दितं कर्म न नृशंसं कथंचन}
{इति पूर्वे महात्मान आपद्धर्मविदो विदुः}


\threelineshloka
{श्रेयांस्तु सहदारस्य विनाशोऽद्य मम स्वयम्}
{ब्राह्मणस्य वधं नाहमनुमंस्ये कदाचन ॥कुन्त्युवाच}
{}


\twolineshloka
{ममाप्येषा मतिर्ब्रह्मन्विप्रा रक्ष्या इति स्थिरा}
{न चाप्यनिष्टः पुत्रो मे यदि पुत्रशतं भवेत्}


\twolineshloka
{न चासौ राक्षसः शक्तो मम पुत्रविनाशने}
{वीर्यमन्मन्त्रसिद्धश्च तेजस्वी च सुतो मम}


\twolineshloka
{राक्षसाय च तत्सर्वं प्रापयिष्यति भोजनम्}
{मोक्षयिष्यति चात्मानमिति मे निश्चिता मतिः}


\twolineshloka
{समागताश्च वीरेण दृष्टपूर्वाश्च राक्षसाः}
{बलवन्तो महाकाया निहताश्चाप्यनेकशः}


\twolineshloka
{न त्विदं केषुचिद्ब्रह्मान्व्याहर्तव्यं कथंचन}
{विद्यार्थिनो हि मे पुत्रान्विप्रकुर्युः कुतूहलात्}


\threelineshloka
{गुरुणा चाननुज्ञातो ग्राहयेद्यः सुतो मम}
{न स कुर्यात्तथा कार्यं विद्ययेति सतां मतम् ॥वैशंपायन उवाच}
{}


\twolineshloka
{एवमुक्तस्तु पृथया स विप्रो भार्यया सह}
{हृष्टः संपूजयामास तद्वाक्यममृतोपमम्}


\twolineshloka
{ततः कुन्ती च विप्रश्च सहितावनिलात्मजम्}
{तमब्रूतां कुरुष्वेति स तथेत्यब्रवीच्च तौ}


\chapter{अध्यायः १७६}
\twolineshloka
{वैशंपायन उवाच}
{}


\twolineshloka
{करिष्य इति भीमेन प्रतिज्ञातेऽथ भारत}
{आजग्मुस्ते ततः सर्वे भैक्षमादाय पाण्डवाः}


\twolineshloka
{`भीमसेनं ततो दृष्ट्वा आपूर्णवदनं तथा}
{बुबोध धर्मराजस्तु हृषितं भीममच्युतम्}


\twolineshloka
{तोषस्य कारणं यत्तु मनसाऽचिन्तयद्गुरुः}
{स समीक्ष्य तदा राजन्योद्धुकामं युधिष्ठिरः ॥'}


\twolineshloka
{आकारेणैव तं ज्ञात्वा पाण्डुपुत्रो युधिष्ठिरः}
{रहः समुपविश्यैकस्ततः पप्रच्छ मातरम्}


\threelineshloka
{किं चिकीर्षत्ययं कर्म भीमो भमपराक्रमः}
{भवत्यनुमते कच्चित्स्वयं वा कर्तुमिच्छति ॥कुन्त्युवाच}
{}


\twolineshloka
{ममैव वचनादेष करिष्यति परन्तपः}
{ब्राह्मणार्थे महत्कृत्यं मोक्षाय नगरस्य च}


\threelineshloka
{`बकाय कल्पितं पुत्र महान्तं बलिमुत्तमम्}
{भीमो भुनक्तु सुस्पष्टमप्येकाहं तपःसुतः ॥'युधिष्ठिर उवाच}
{}


\twolineshloka
{किमिदं साहसं तीक्ष्णं भवत्या दुष्करं कृतम्}
{परित्यागं हि पुत्रस्य न प्रशंसन्ति साधवः}


\twolineshloka
{कथं परसुतस्यार्थे स्वसुतं त्यक्तुमिच्छसि}
{लोकवेदविरुद्धं हि पुत्रत्यागात्कृतं त्वया}


\twolineshloka
{यस्य बाहू समाश्रित्य सुखं सर्वे शयामहे}
{राज्यं चापहृतं क्षुद्रैराजिहीर्षामहे पुनः}


\twolineshloka
{यस्य दुर्योधनो वीर्यं चिन्तयन्नमितौजसः}
{न शेते रजनीः सर्वा दुःखाच्छकुनिना सह}


\twolineshloka
{यस्य वीरस्य वीर्येण मुक्ता जतुगृहाद्वयम्}
{अन्येभ्यश्चैव पापेभ्यो निहतश्च पुरोचनः}


\twolineshloka
{यस्य वीर्यं समाश्रित्य वसुपूर्णां वसुन्धराम्}
{इमां मन्यामहे प्राप्तां निहत्य धृतराष्ट्रजान्}


\threelineshloka
{तस्य व्यवसितस्त्यागो बुद्धिमास्थाय कां त्वया}
{कच्चिन्नु दुःखैर्बुद्धिस्ते विलुप्ता गतचेतसः ॥कुन्त्युवाच}
{}


\twolineshloka
{युधिष्ठिर न संतापस्त्वया कार्यो वृकोदरे}
{न चायं बुद्धिदौर्बल्याद्व्यवसायः कृतो मया}


\threelineshloka
{`न च शोकेन बुद्धिः सा विलुप्ता गतचेकसः}
{'इह विप्रस्य भवने वयं पुत्र सुखोषिताः}
{अज्ञाता धार्तराष्ट्राणां सत्कृता वीतमन्यवः}


\twolineshloka
{तस्य प्रतिक्रिया पार्थ मयेयं प्रसमीक्षिता}
{एतावानेव पुरुषः कृतं यस्मिन्न नश्यति}


\twolineshloka
{यावच्च कुर्यादन्योऽस्य कुर्याद्बहुगुमं ततः}
{`ब्राह्मणार्थे महान्धर्मो जानामीत्थं वृकोदरे ॥'}


\twolineshloka
{दृष्ट्वा भीमस्य विक्रान्तं तदा जतुगृहे महत्}
{हिडिम्बस्य वधाच्चैवं विश्वासो मे वृकोदरे}


\twolineshloka
{बाह्वोर्बलं हि भीमस्य नागायुतसमं महत्}
{येन यूयं गजप्रख्या निर्व्यूढा वारणावतात्}


\twolineshloka
{वृकोदरेण सदृशो बलेनान्यो न विद्यते}
{यो व्यतीयाद्युधि श्रेष्ठमपि वज्रधरं स्वयम्}


\twolineshloka
{जातमात्रः पुरा चैव ममाङ्कात्पतितो गिरौ}
{शरीरगौरवादस्य शिला गात्रैर्विचूर्णिता}


\twolineshloka
{तदहं प्रज्ञया ज्ञात्वा बलं भीमस्य पाण्डव}
{प्रतिकार्ये च विप्रस्य ततः कृतवती मतिम्}


\twolineshloka
{नेदं लोभान्न चाज्ञानान्न च मोहाद्विनिश्चितम्}
{बुद्धिपूर्वं तु धर्मस्य व्यवसायः कृतो मया}


\twolineshloka
{अर्थौ द्वावपि निष्पन्नौ युधिष्ठिर भविष्यतः}
{प्रतीकारश्च वासस्य धर्मश्च चरितो महान्}


\twolineshloka
{यो ब्राह्मणस्य साहाय्यं कुर्यादर्थेषु कर्हिचित्}
{क्षत्रियः स शुभाँल्लोकानाप्नुयादिति मे मतिः}


\twolineshloka
{क्षत्रियस्यैव कुर्वाणः क्षत्रियो वधमोक्षणम्}
{विपुलां कीर्तिमाप्नोति लोकेऽस्मिंश्च परत्र च}


\twolineshloka
{वैश्यस्यार्थे च साहाय्यं कुर्वाणः क्षत्रियो भुवि}
{स सर्वेष्वपि लोकेषु प्रजा रञ्जयते ध्रुवम्}


\twolineshloka
{शूद्रं तु मोचयेद्राजा शरणार्थिनमागतम्}
{प्राप्नोतीह कुले जन्म सद्द्रव्ये राजपूजिते}


\twolineshloka
{एवं मां भगवान्व्यासः पुरा पौरवनन्दन}
{प्रोवाचासुकरप्रज्ञस्तस्मादेवं चिकीर्षितम्}


\chapter{अध्यायः १७७}
\twolineshloka
{युधिष्ठिर उवाच}
{}


\twolineshloka
{उपपन्नमिदं मातस्त्वया यद्बुद्धिपूर्वकम्}
{आर्तस्य ब्राह्मणस्यैतदनुक्रोशादिदं कृतम्}


\twolineshloka
{ध्रुवमेष्यति भीमोऽयं निहत्य पुरुषादकम्}
{सर्वथा ब्राह्मणस्यार्थे यदनुक्रोशवत्यसि}


\threelineshloka
{यथा त्विदं न विन्देयुर्नरा नगरवासिनः}
{तथाऽयं ब्राह्मणो वाच्यः परिग्राह्यश्च यत्नतः ॥वैशंपायन उवाच}
{}


\twolineshloka
{`युधिष्ठिरेण संमन्त्र्य ब्राह्मणार्थमरिन्दम}
{कुन्ती प्रविश्य तान्सर्वान्मन्त्रयामास भारत}


\twolineshloka
{अथ रात्र्यां व्यतीतायां भीमसेनो महाबलः}
{ब्राह्मणं समुपागम्य वचश्चेदमुवाच ह}


\twolineshloka
{आपदस्त्वां विमोक्ष्येऽहं सपुत्रं ब्राह्मण प्रियम्}
{मा भैषी राक्षसात्तस्मान्मां ददातु बलिं भवान्}


\twolineshloka
{इद मामशितं कर्तुं प्रयतस्व सकृद्गृहे}
{आथात्मानं प्रादास्यामि तस्मै घोराय रक्षसे}


\threelineshloka
{त्वरध्वं किं विलम्बध्वे मा चिरं कुरुतानघाः}
{व्यवस्येयं मनः प्राणैर्युष्मान्रक्षितुमद्य वै ॥वैशंपायन उवाच}
{}


\twolineshloka
{एवमुक्तः स भीमेन ब्राह्मणो भरतर्षभ}
{सुहृदां तत्समाख्याय ददावन्नं सुसंस्कृतम्}


\twolineshloka
{पिशितोदनमाजह्रुरथास्मै पुरवासिनः}
{सघृतं सोपदंशं च सूपैर्नानाविधैः सह}


\twolineshloka
{तदाऽशित्वा भीमसेनो मांसानि विविधानि च}
{मोदकानि च मुख्यानि चित्रोदनचयान्बहून्}


\twolineshloka
{ततोऽपिबद्दधिघटान्सुबहून्द्रोणसंमितान्}
{तस्य भुक्तवतः पौरा यथावत्समुपार्जितान्}


\twolineshloka
{उपजह्रुर्भृतं भागं समृद्धमनसस्तदा}
{ततो रात्र्यां व्यतीतायां सव्यञ्जनदधिप्लुतम्}


\twolineshloka
{समारुह्यान्नसंपूर्णं शकटं स वृकोदरः}
{प्रययौ तूर्यनिर्घोषैः पौरैश्च परिवारितः}


\twolineshloka
{आत्मानमेषोऽन्नभूतो राक्षसाय प्रदास्यति}
{तरुणोऽप्रतिरूपश्च दृढ औदरिको युवा}


\twolineshloka
{वाग्भिरेवंप्रकाराभिः स्तूयमानो वृकोदरः}
{चुचोद स बलीवर्दौ युक्तौ सर्वाङ्गकालकौ}


\twolineshloka
{वादित्राणां प्रवादेन ततस्तं पुरुषादकम्}
{अभ्यगच्छत्सुसंहृष्टः सर्वत्र मनुजैर्वृतः}


\twolineshloka
{संप्राप्य स च तं देशमेकाकी समुपाययौ}
{पुरुषादभयाद्भीतस्तत्रैवासीज्जनव्रजः}


\twolineshloka
{स गत्वा दूरमध्वानं दक्षिणामभितो दिशम्}
{यतोपदिष्टमुद्देशे ददर्श विपुलं द्रुमम्}


\twolineshloka
{केशमज्जास्थिमेदोभिर्बाहूरुचरणैरपि}
{आर्द्रैः शुष्कैश्च संकीर्णमभितोऽथ वनस्पतिम्}


\twolineshloka
{गृध्रकङ्कबलच्छन्नं गोमायुगणसेवितम्}
{उग्रगन्धमचक्षुष्यं श्मशानमिव दारुणम्}


\twolineshloka
{तं प्रविश्य महावृक्षं चिन्तयामास वीर्यवान्}
{यावन्न पश्यते रक्षो बकाभिख्यं बलोत्तरम्}


\twolineshloka
{आचितं विविधैर्भोज्यैरन्नैरुच्चावचैरिदम्}
{शकटं सूपसंपूर्णं यावद्द्रक्ष्यति राक्षसः}


\twolineshloka
{तावदेवेह भोक्ष्येऽहं दुर्लभं हि पुनर्भवेत्}
{विप्रकीर्येत सर्वं हि प्रयुद्धे मयि रक्षसा}


\twolineshloka
{अभोज्यं हि शवस्पर्शे निगृहीते बके भवेत्}
{स त्वेवं भीमकर्मा तु भीमसेनोऽभिलक्ष्य च}


\twolineshloka
{उपविश्य विविक्तेऽन्नं भुङ्क्ते स्म परमं परः}
{तं ततः सर्वतोऽपश्यन्द्रुमानारुह्य नागराः}


\twolineshloka
{नारक्षो बलिमश्नीयादेवं बहु च मानवाः}
{भुङ्क्ते ब्राह्मणरूपेण बकोऽयमिति चाब्रुवन्}


\threelineshloka
{स तं हसति तेजस्वी तदन्नमुपभुज्य च}
{'आसाद्य तु वनं तस्य रक्षसः पाण्डवो बली}
{आजुहाव ततो नाम्ना तदन्नमुपपादयन्}


\twolineshloka
{ततः स राक्षसः क्रुद्धो भीमस्य वचनात्तदा}
{आजगाम सुसंक्रुद्धो यत्र भीमो व्यवस्थितः}


\twolineshloka
{महाकायो महावेगो दारयन्निव मेदिनीम्}
{लोहिताक्षः करालश्च लोहितश्मश्रुमूर्धजः}


\twolineshloka
{आकर्णाद्भिन्नवक्त्रश्च शङ्कुकर्णो विभीषणः}
{त्रिशिखां भ्रुकुटिं कृत्वा संदश्य दशनच्छदम्}


\twolineshloka
{भुञ्जानमन्नं तं दृष्ट्वा भीमसेनं स राक्षसः}
{विवृत्य नयने क्रुद्ध इदं वचनमब्रवीत्}


\twolineshloka
{कोऽयमन्नमिदं भुङ्क्ते मदर्थमुपकल्पितम्}
{पश्यतो मम दुर्बुद्धिर्यियासुर्यमसादनम्}


\twolineshloka
{भीमसेनस्ततः श्रुत्वा प्रहसन्निव भारत}
{राक्षसं तमनादृत्य भुङ्क्त एव पराङ्मुखः}


\twolineshloka
{रवं स भैरवं कृत्वा समुद्यम्य करावुभौ}
{अभ्यद्रवद्भीमसेनं जिङांसुः पुरुषादकः}


\twolineshloka
{तथापि परिभूयैनं प्रेक्षमाणो वृकोदरः}
{राक्षसं भुङ्क्त एवान्नं पाण्डवः परवीरहा}


\twolineshloka
{अमर्षेण तु संपूर्णः कुन्तीपुत्रं वृकोदरम्}
{जघान पृष्ठे पाणिभ्यामुभाभ्यां पृष्ठतः स्थितः}


\twolineshloka
{तथा बलवता भीमः पाणिभ्यां भृशमाहतः}
{नैवावलोकयामास राक्षसं भुङ्क्त एव सः}


\twolineshloka
{ततः स भूयः संक्रुद्धो वृक्षमादाय राक्षसः}
{ताडयिष्यंस्तदा भीमं पुनरभ्यद्रवद्बली}


\twolineshloka
{क्षिप्तं क्रुद्धेन तं वृक्षं प्रतिजग्राह वीर्यवान्}
{सव्येन पाणिना भीमो दक्षिणेनाप्यभुङ्क्त ह}


\twolineshloka
{`शकटान्नं ततो भुक्त्वा रक्षसः पाणिना सह}
{गृह्णन्नेव तदा वृक्षं निःशेषं पर्वतोपमम्}


\twolineshloka
{भीमसेनो हसन्नेव भुक्त्वा त्यक्त्वा च राक्षसम्}
{पीत्वा दधिघटान्पूर्णान्घृतकुम्भाञ्शतं शतम्}


\twolineshloka
{वार्युपस्पृश्य संहृष्टस्तस्थौ गिरिरिवापरः}
{भ्रामयन्तं महावृक्षमायान्तं भीमदर्शनम्}


\twolineshloka
{दृष्ट्वोत्थायाहवे वीरः सिंहनादं व्यनादयत्}
{भुजवेगं तथाऽऽस्फोटं क्ष्वेलितं च महास्वनम्}


\threelineshloka
{कृत्वाऽऽह्वयत संक्रुद्धो भीमसेनोऽथ राक्षसम्}
{उवाचाशनिशब्देन ध्वनिना भीषयन्निव ॥भीम उवाच}
{}


\twolineshloka
{बहुकालं सुपुष्टं ते शरीरं राक्षसाधम}
{द्विपच्चतुष्पन्मांसैश्च बहुभिश्चौदनैरपि}


\twolineshloka
{मद्बाहुबलमाश्रित्य न त्वं भूयस्त्वशिष्यसि}
{अद्य मद्बाहुनिष्पिष्टो गमिष्यसि यमालयम्}


\twolineshloka
{अद्यप्रभृति स्वप्स्यन्ति वेत्रकीयनिवासिनः}
{निरुद्विग्नाः पुरस्यास्य कण्टके सूद्धृते मया}


\threelineshloka
{अद्य युद्धे शरीरं ते कङ्कगोमायुवायसाः}
{मया हतस्य खादन्तु विकर्षन्तु च भूतले ॥वैशंपायन उवाच}
{}


\twolineshloka
{एवमुक्त्वा सुसंक्रुद्धः पार्थो बकजिघांसया}
{उपाधावद्बकश्चापि पार्थं पार्थिवसत्तम}


\twolineshloka
{महाकायो महावेगो दारयन्निव मेदिनीम्}
{पिशङ्गरूपः पिङ्गाक्षो भीमसेनमभिद्रवत्}


\twolineshloka
{त्रिशिखां भ्रुकुटीं कृत्वा संदश्य दशनच्छदम्}
{भृशं स भूयः संक्रुद्धो वृक्षमादाय राक्षसः}


\twolineshloka
{ताडयिष्यंस्तदा भीमं तरसाऽभ्यद्रवद्बली}
{क्रुद्धेनाभिहतं वृक्षं प्रतिजग्राह लीलया}


\twolineshloka
{सव्येन पाणिना भीमः प्रहसन्निव भारत}
{ततः स पुनरुद्यम्य वृक्षान्बहुविधान्बली}


\twolineshloka
{प्राहिणोद्भीमसेनाय बकोऽपि बलवान्रणे}
{सर्वानपोहयद्वृक्षान्स्वस्य हस्तस्य शाखया}


% Check verse!
तद्वृक्षयुद्धमभवद्वृक्षषण्डविनाशनम् ॥महत्सुघोरं राजेन्द्र बकपाण्डवयोस्तदा
\twolineshloka
{नाम विश्राव्य स बकः समभिद्रुत्य पाण्डवम्}
{समयुध्यत तीव्रेण वेगेन पुरुषादकः}


\twolineshloka
{तयोर्वेगेन महता पृथिवी समकम्पत}
{पादपांश्च महामात्रांश्चूर्णयामासतुः क्षणात्}


\twolineshloka
{द्रुतमागत्य पाणिभ्यां गृहीत्वा चैनमाक्षिपत्}
{आक्षिप्तो भीमसेनश्च पुनरेवोत्थितो हसन्}


\twolineshloka
{आलिङ्ग्यापि बकं भीमो न्यहनद्वसुधातले}
{भीमो विसर्जयित्वैनं समाश्वसिहि राक्षस}


\twolineshloka
{इत्युक्त्वा पुनरास्फोट्य उत्तिष्ठेति च सोऽब्रवीत्}
{समुत्पत्य ततः क्रुद्धो रूपं कृत्वा महत्तरम्}


\twolineshloka
{विरूपः सहसा तस्थौ2 तर्जयित्वा वृकोदरम्}
{अहसद्भीमसेनोऽपि राक्षसं भीमदर्शनम्}


\twolineshloka
{असौ गृहीत्वा पाणिभ्यां पृष्ठतश्च व्यवस्थितः}
{जानुभ्यां पीडयित्वाथ पातयामास भूतले}


\twolineshloka
{पुनः क्रुद्धो विसृज्यैनं राक्षसं क्रोधजीवितम्}
{स्वां कटीमीषदुन्नम्य बाहू तस्य परामृशत्}


\twolineshloka
{तस्य बाहू समादाय त्वरमाणो वृकोदरः}
{उत्क्षिप्य चावधूयैनं पातयन्बलवान्भुवि}


\twolineshloka
{तं तु वामेन पादेन क्रुद्धो भीमपराक्रमः}
{उरस्येनं समाजघ्ने भीमस्तु पतितं भुवि}


\twolineshloka
{व्यात्ताननेन घोरेण लम्बजिह्वेन रक्षसा}
{तेनाभिद्रुत्य भीमेन भीमो मूर्ध्नि समाहतः}


\twolineshloka
{एवं निहन्यमानः स राक्षसेन बलीयसा}
{रोषेण महताऽऽविष्टो भीमो भीमपराक्रमः}


\twolineshloka
{गृहीत्वा मध्यमुत्क्षिप्य बली जग्राह राक्षसम्}
{तावन्योन्यं पीडयन्तौ पुरुषादवृकोदरौ}


% Check verse!
मत्ताविव महानागावन्योन्यं विचकर्षतुः
\twolineshloka
{बाहुविक्षेपशब्दैश्च भीमराक्षसयोस्तदा}
{वेत्रकीयपुरी सर्वा वित्रस्ता समपद्यत ॥'}


\twolineshloka
{तयोर्वेगेन महता तत्र भूमिरकम्पत}
{पादपान्वीरुधश्चैव चूर्णयामासतू रयात्}


\twolineshloka
{समागतौ च तौ वीरावन्योन्यवधकाङ्क्षिणौ}
{गिरिभिर्गिरिशृह्गैश्च पाषाणैः पर्वतच्युतैः}


\twolineshloka
{अन्योन्यं ताडयन्तौ तौ चूर्णयामासतुस्तदा}
{आयामविस्तराभ्यां च परितो योजनद्वयम्}


\twolineshloka
{निर्महीरुहपाषाणतृणकुञ्जलतावलिम्}
{चक्रतुर्युद्धदुर्मत्तौ कूर्मपृष्ठोपमां महीम्}


\twolineshloka
{मुहूर्तमेवं संयुध्य समं रक्षःकुरूद्वहौ}
{ततो रक्षोविनाशाय मतिं कृत्वा कुरूत्तमः}


\twolineshloka
{दन्तान्कटकटीकृत्य दष्ट्वा च दशनच्छदम्}
{नेत्रे संवृत्य विकटं तिर्यक्प्रैक्षत राक्षसम्}


\twolineshloka
{अथ तं लोलयित्वा तु भीमसेनो महाबलः}
{अगृह्णात्परिरभ्यैनं बाहुभ्यां परिरभ्य च}


\twolineshloka
{जानुभ्यां पार्श्वयोः कुक्षौ पृष्ठे वक्षसि जघ्निवान्}
{भग्नोरुबाहुहृच्चैव विस्रंसद्देहबन्धनः}


\twolineshloka
{प्रस्वेददीर्घनिश्वासो निर्यञ्जीवाक्षितारकः}
{अजाण्डास्फोटनं कुर्वन्नाक्रोशंश्च श्वसञ्छनैः}


\twolineshloka
{भूमौ निपत्य विलुठन्दण्डाहत इवोरगः}
{विस्फुरन्तं महाकायं परितो विचकर्ष ह}


\twolineshloka
{विकृष्यमाणो वेगेन पाण्डवेन बलीयसा}
{समयुज्यत तीव्रेण श्रमेण पुरुषादकः ॥'}


\twolineshloka
{हीयमानबलं रक्षः समीक्ष्य पुरुषर्षभः}
{निष्पिष्य भूमौ जानुभ्यां समाजघ्ने वृकोदरः}


\threelineshloka
{ततो}
{ञस्य जानुना पृष्ठमवपीड्य बलादिव}
{बाहुना परिजग्राह दक्षिणेन शिरोधराम्}


\twolineshloka
{सव्येन च कटीदेशे गृह्य वाससि पाण्डवः}
{जानुन्यारोप्य तत्पृष्ठं महाशब्दं बभञ्ज ह}


\twolineshloka
{ततोऽस्य रुधिरं वक्त्रात्प्रादुरासीद्विशांपते}
{भज्यमानस्य भीमेन तस्य घोरस्य रक्षसः}


\chapter{अध्यायः १७८}
\twolineshloka
{वैशंपायन उवाच}
{}


\twolineshloka
{ततः स भग्नपार्श्वाङ्गो नदित्वा भैरवं रवम्}
{शैलराजप्रतीकाशो गतासुरभवद्बकः}


\twolineshloka
{तेन शब्देन वित्रस्तो जनस्तस्याथ रक्षसः}
{निष्पपात गृहाद्राजन्सहैव परिचारिभिः}


\threelineshloka
{`बकानुजस्तदा राजन्भीमं शरणमेयिवान्}
{ततस्तु निहतं दृष्ट्वा राक्षसेन्द्रं महाबलम्}
{राक्षसाः परमत्रस्ता भीमं शऱणमाययुः ॥'}


\twolineshloka
{तान्भीतान्विगतज्ञानान्भीमः प्रहरतां वरः}
{सांत्वयामास बलवान्समये च न्यवेशयत्}


\twolineshloka
{न हिंस्या मानुषा भूयो युष्माभिरिति कर्हिचित्}
{हिंसतां हि वधः शीघ्रमेवमेव भवेदिति}


\twolineshloka
{तस्य तद्वचनं श्रुत्वा तानि रक्षांसि भारत}
{एवमस्त्विति तं प्राहुर्जगृहुः समयं च तम्}


\threelineshloka
{`सगणस्तु बकभ्राता प्राणमत्पाण्डवं तदा}
{'ततः प्रभृति रक्षांसि तत्र सौम्यानि भारत}
{नगरे प्रत्यदृश्यन्त नरैर्नगरवासिभिः}


\threelineshloka
{ततो भीमस्तमादाय गतासुं पुरुषादकम्}
{`निष्कर्णनेत्रं निर्जिह्वं निःसंज्ञं कण्ठपीडनात्}
{कुर्वन्बहुविधां चेष्टां पुरद्वारमकर्षत}


\twolineshloka
{द्वारदेशे विनिक्षिप्य पुरमागात्स मारुतिः}
{स एव राक्षसो नूनं पुनरायाति नः पुरीम्}


\twolineshloka
{सबालवृद्धाः पुरुषा इति भीताः प्रदुद्रुवुः}
{ततो भीमो बकं हत्वा गत्वा ब्राह्मणवेश्म तत्}


\twolineshloka
{बलीवर्दौ च शकटं ब्राह्मणाय न्यवेदयत्}
{तूष्णीमन्तर्गृहं गच्छेत्यभिधाय द्विजोत्तमम्}


\twolineshloka
{मातृभ्रातृसमक्षं च गत्वा शयनमेव च}
{आचचक्षेऽथ तत्सर्वं रात्रौ युद्धमभूद्यथा ॥'}


\twolineshloka
{ततो नरा विनिष्क्रान्ता नगरात्कल्यमेव तु}
{ददृशुर्निहतं भूमौ राक्षसं रुधिरोक्षितम्}


\twolineshloka
{तमद्रिकूटसदृशं विनिकीर्णं भयानकम्}
{दृष्ट्वा संहृष्टरोमाणो बभूवुस्तत्र नागराः}


\twolineshloka
{एकचक्रां ततो गत्वा प्रवृत्तिं प्रददुः पुरे}
{ततः सहस्रशो राजन्नरा नगरवासिनः}


\threelineshloka
{तत्राजग्मुर्बकं द्रष्टुं सस्त्रीवृद्धकुमारकाः}
{ततस्ते विस्मिताः सर्वे कर्म दृष्ट्वाऽतिमानुषम्}
{दैवतान्यर्चयाञ्चक्रुः प्रार्थितानि पुरा भयात्}


\twolineshloka
{ततः प्रगणयामासुः कस्य वारोऽद्य भोजने}
{ज्ञात्वा चागम्य तं विप्रं पप्रच्छुः सर्व एव ते}


\threelineshloka
{एवं पृष्टः स बहुशो रक्षमाणश्च पाण्डवान्}
{उवाच नागरान्सर्वानिदं विप्रर्षभस्तदा ॥ब्राह्मण उवाच}
{}


\twolineshloka
{आज्ञापितं मामशने रुदन्तं सह बन्धुभिः}
{ददर्श ब्राह्मणः कश्चिन्मन्त्रसिद्धो महामनाः}


\twolineshloka
{परिपृच्छ्य स मां पूर्वं परिक्लेशं पुरस्य च}
{अब्रवीद्ब्राह्मणश्रेष्ठो विश्वास्य प्रहसन्निव}


\twolineshloka
{प्रापयिष्याम्यहं तस्मा अन्नमेतद्दुरात्मने}
{मन्निमित्तं भयं चापि न कार्यमिति चाब्रवीत्}


\twolineshloka
{स तदन्नमुपादाय गतो बकवनं प्रति}
{तेन नूनं भवेदेतत्कर्म लोकहितं कृतम्}


\twolineshloka
{ततस्ते ब्राह्मणाः सर्वे क्षत्रियाश्च सुविस्मिताः}
{वैश्याः शूद्राश्च मुदिताश्चक्रुर्ब्रह्ममहं तदा}


\twolineshloka
{ततो जानपदाः सर्वे आजग्मुर्नगरं प्रति}
{तमद्भुततमं द्रष्टुं पार्थास्तत्रैव चावसन्}


\twolineshloka
{वेत्रकीयगृहे सर्वे परिवार्य वृकोदरम्}
{विस्मयादभ्यगच्छन्त भीमं भीमपराक्रमम्}


\twolineshloka
{न वै न संभवेत्सर्वं ब्राह्मणेषु महात्मसु}
{इति सत्कृत्य तं पौराः परिवव्रुः समन्ततः}


\twolineshloka
{अयं त्राता हि खेदानां पितेव परमार्थतः}
{अस्य शुश्रूषवः पादौ परिचर्य उपास्महे}


\twolineshloka
{पशुमद्दधिमनच्चास्य वारं भक्तमुपाहरन्}
{तस्मिन्हते ते पुरुषा भीताः समनुबोधनाः}


\twolineshloka
{ततः संप्राद्रवन्पार्थाः सह मात्रा परन्तपाः}
{आगच्छन्नेकचक्रां ते गाण्डवाः संशितव्रताः}


\threelineshloka
{वैदिकाध्ययने युक्ता जटिला ब्रह्मचारिणः}
{अवसंस्ते च तत्रापि ब्राह्मणस्य निवेशने}
{मात्रर सहैकचक्रायां दीर्घकालं सहोषिताः}


\chapter{अध्यायः १७९}
\twolineshloka
{जनमेजय उवाच}
{}


\threelineshloka
{ते तथा पुरुषव्याघ्रा निहत्य बकराक्षसम्}
{अत ऊर्ध्वं ततो ब्रह्मन्किमकुर्वत पाण्डवाः ॥वैशंपायन उवाच}
{}


\twolineshloka
{तत्रैव न्यवसन्राजन्निहत्य बकराक्षसम्}
{अधीयानाः परं ब्रह्म ब्राह्मणस्य निवेशने}


\twolineshloka
{ततः कतिपयाहस्य ब्राह्मणः संशितव्रतः}
{प्रतिश्रयार्थी तद्वेश्म ब्राह्मणस्याजगाम ह}


\twolineshloka
{स म्यक् पूजयित्वा तं विप्रं विप्रर्षभस्तदा}
{ददौ प्रतिश्रयं तस्मै सदा सर्वातिथिव्रतः}


\twolineshloka
{ततस्ते पाण्डवाः सर्वे सह कुन्त्या नरर्षभाः}
{उपासाञ्चक्रिरे विप्रं कथयन्तं कथाः शुभाः}


\twolineshloka
{कथयामास देशांश्च तीर्थानि सरितस्तथा}
{राज्ञश्च विविधाश्चर्यान्देशांश्चैव पुराणि च}


\twolineshloka
{स तत्राकथयद्विप्रः कथान्ते जनमेजय}
{पञ्चालेष्वद्भुताकारं याज्ञसेन्याः स्वयंवरम्}


\twolineshloka
{धृष्टद्युम्नस्य चोत्पत्तिमुत्पत्तिं च शिखण्डिनः}
{अयोनिजत्वं कृष्णाया द्रुपदस्य महामखे}


\threelineshloka
{तदद्भुततमं श्रुत्वा लोके तस्य महात्मनः}
{विस्तरेणैव पप्रच्छुः कथां ते पुरुषर्षभाः ॥पाण्डवा ऊचुः}
{}


\twolineshloka
{कथं द्रुपदपुत्रस्य धृष्टद्युम्नस्य पावकात्}
{वेदीमध्याच्च कृष्णायाः संभवः कथमद्भुतः}


\fourlineindentedshloka
{कथं द्रोणान्महेष्वासात्सर्वाण्यस्त्राण्यशिक्षत}
{`धृष्टद्युम्नो महेष्वासः कथं द्रोणस्य मृत्युदः}
{'कथं विप्र सखायौ तौ भिन्नौ कस्य कृतेन वा ॥वैशंपायन उवाच}
{}


\twolineshloka
{एवं तैश्चोदितो राजन्स विप्रः पुरुषर्षभैः}
{कथयामास तत्सर्वं द्रौपदीसंभवं तदा}


\chapter{अध्यायः १८०}
\twolineshloka
{ब्राह्मण उवाच}
{}


\twolineshloka
{गङ्गाद्वारं प्रति महान्बभूवर्षिर्महातपाः}
{भरद्वाजो महाप्राज्ञः सततं संशितव्रतः}


\twolineshloka
{सोऽभिषेक्तुं गतो गङ्गां पूर्वमेवागतां सतीम्}
{ददर्शाप्सरसं तत्र घृताचीमाप्लुतामृषिः}


\twolineshloka
{तस्या वायुर्नदीतीरे वसनं व्यहरत्तदा}
{अपकृष्टाम्बरां दृष्ट्वा तामृषिश्चकमे तदा}


\twolineshloka
{तस्यां संसक्तमनसः कौमारब्रह्मचारिणः}
{चिरस्य रेतश्चस्कन्द तदृषिर्द्रोण आदधे}


\twolineshloka
{ततः समभवद्द्रोणः कुमारस्तस्य धीमतः}
{अध्यगीष्ट स वेदांश्च वेदाङ्गानि च सर्वशः}


\twolineshloka
{भरद्वाजस्य तु सखा पृषतो नाम पार्थिवः}
{तस्यापि द्रुपदो नाम तदा समभवत्सुतः}


\twolineshloka
{स नित्यमाश्रमं गत्वा द्रोणेन सह पार्षतः}
{चिक्रीडाध्ययनं चैव चकार क्षत्रियर्षभः}


\twolineshloka
{ततस्तु पृषतेऽतीते स राजा द्रुपदोऽभवत्}
{द्रोणोऽपि रामं शुश्राव दित्सन्तं वसु सर्वशः}


\threelineshloka
{वनं तु प्रस्थितं रामं भरद्वाजसुतोऽब्रवीत्}
{आगतं वित्तकामं मां विद्धि द्रोणं द्विजोत्तम ॥राम उवाच}
{}


\threelineshloka
{शरीरमात्रमेवाद्य मया समवशेषितम्}
{अस्त्राणि वा शरीरं वा ब्रह्मन्नेकतमं वृणु ॥द्रोण उवाच}
{}


\threelineshloka
{अस्त्राणि चैव सर्वाणि तेषां संहारमेव च}
{प्रयोगं चैव सर्वेषां दातुमर्हति मे भवान् ॥ब्राह्मण उवाच}
{}


\twolineshloka
{तथेत्युक्त्वा ततस्तस्मै प्रददौ भृगुनन्दनः}
{प्रतिगृह्य तदा द्रोणः कृतकृत्योऽभवत्तदा}


\twolineshloka
{संप्रहृष्टमना द्रोणो रामात्परमसम्मतम्}
{ब्रह्मास्त्रं समनुज्ञाप्य नरेष्वभ्यधिकोऽभवत्}


\threelineshloka
{ततो द्रुपदमासाद्य भारद्वाजः प्रतापवान्}
{अब्रवीत्पुरुषव्याघ्रः सखायं विद्धि मामिति ॥द्रुपद उवाच}
{}


\threelineshloka
{नाश्रोत्रियः श्रोत्रियस्य नारथी रथिनः सखा}
{नाराजा पार्थिवस्यापि सखिपूर्वं किमिष्यते ॥ब्राह्मण उवाच}
{}


\twolineshloka
{स विनिश्चित्य मनसा पाञ्चाल्यं प्रति बुद्धिमान्}
{जगाम कुरुमुख्यानां नगरं नागसाह्वयम्}


\twolineshloka
{तस्मै पौत्रान्समादाय वसूनि विविधानि च}
{प्राप्ताय प्रददौ भीष्मः शिष्यान्द्रोणाय धीमते}


\twolineshloka
{द्रोणः शिष्यांस्ततः पार्थानिदं वचनमब्रवीत्}
{समानीय तु ताञ्शिष्यान्द्रुपदस्यासुखाय वै}


\threelineshloka
{आचार्यवेतनं किंचिद्धृदि यद्वर्तते मम}
{कृतास्त्रैस्तत्प्रदेयं स्यात्तदृतं वदतानघाः}
{सोऽर्जुनप्रमुखैरुक्तस्तथाऽस्त्विति गुरुस्तदा}


\twolineshloka
{यदा च पाण्डवाः सर्वे कृतास्त्राः कृतनिश्चयाः}
{ततो द्रोणोऽब्रवीद्भूयो वेतनार्थमिदं वचः}


\twolineshloka
{पार्षतो द्रुपदो नाम छत्रवत्यां नरेश्वरः}
{तस्मादाकृष्य तद्राज्यं मम शीघ्रं प्रदीयताम्}


\twolineshloka
{`धार्तराष्ट्राश्च ते भीताः पाञ्चालान्पाण्डवादयः}
{धार्तराष्ट्रैश्च सहिताः पुनर्द्रोणेन चोदिताः}


\twolineshloka
{यज्ञसेनेन संगम्य कर्णदुर्योधनादयः}
{निर्जिताः संन्यवर्तन्त तथा ते क्षत्रियर्षभाः ॥'}


\twolineshloka
{ततः पाण्डुसुताः पञ्च निर्जित्य द्रुपदं युधि}
{द्रोणाय दर्शयामासुर्बद्ध्वा ससचिवं तदा}


\twolineshloka
{`महेन्द्र इव दुर्धर्षो महेन्द्र इव दानवम्}
{महेन्द्रपुत्रः पाञ्चालं जितवानर्जुनस्तदा}


\fourlineindentedshloka
{तद्दृष्ट्वा तु महावीर्यं फल्गुनस्य महौजसः}
{व्यस्मयन्त जनाः सर्वे यज्ञसेनस्य बान्धवाः}
{द्रोण उवाच}
{}


\twolineshloka
{प्रार्थयामि त्वया सख्यं पुनरेव नराधिप}
{अराजा किल नो राज्ञः सखा भवितुमर्हति}


\threelineshloka
{अतः प्रयतितं राज्ये यज्ञसेन त्वया सह}
{राजाऽसि दक्षिणे कूले भागीरथ्याहमुत्तरे ॥ब्राह्मण उवाच}
{}


\twolineshloka
{एवमुक्तो हि पाञ्चाल्यो भारद्वाजेन धीमता}
{उवाचास्त्रविदां श्रेष्ठं द्रोणं ब्राह्मणसत्तमम्}


\twolineshloka
{एवं भवतु भद्रं ते भारद्वाज महामते}
{सख्यं तदेव भवतु शश्वद्यदभिमन्यसे}


\twolineshloka
{एवमन्योन्यमुक्त्वा तौ कृत्वा सख्यमनुत्तमम्}
{जग्मतुर्द्रोणपाञ्चाल्यौ यथागतमरिन्दमौ}


\twolineshloka
{असत्कारः स तु महान्मुहूर्तमपि तस्य तु}
{नापैति हृदयाद्राज्ञो दुर्मनाः स कृशोऽभवत्}


\chapter{अध्यायः १८१}
\twolineshloka
{ब्राह्मण उवाच}
{}


\twolineshloka
{अमर्षी द्रुपदो राजा कर्मसिद्धान्द्विजर्षभान्}
{अन्विच्छन्परिचक्राम ब्राह्मणावसथान्बहून्}


\threelineshloka
{पुत्रजन्म परीप्सन्वै शोकोपहतचेतनः}
{`द्रोणेन वैरं द्रुपदो न सुष्वाप स्मरन्सदा}
{'नास्ति श्रेष्ठमपत्यं म इति नित्यमचिन्तयत्}


\twolineshloka
{जातान्पुत्रान्स निर्वेदाद्धिग्बन्धूनिति चाब्रवीत्}
{निःश्वासपरमश्चासीद्द्रोणं प्रतिचिकीर्षया}


\twolineshloka
{प्रभावं विनयं शिक्षां द्रोणस्य चरितानि च}
{क्षात्रेण च बलेनास्य चिन्तयन्नाध्यगच्छत}


\twolineshloka
{प्रतिकर्तुं नृपश्रेष्ठो यतमानोऽपि भारत}
{अभितः सोऽथ कल्माषीं गङ्गाकूले परिभ्रमन्}


\twolineshloka
{ब्राह्मणावसथं पुम्यमाससाद महीपतिः}
{तत्र नास्नातकः कश्चिन्न चासीदव्रती द्विजः}


\twolineshloka
{अधीयानौ महाभागौ सोऽपश्यत्संशितव्रतौ}
{याजोपयाजौ ब्रह्मर्षी शाम्यन्तौ परमेष्ठिनौ}


\twolineshloka
{संहिताध्ययने युक्तौ गोत्रतश्चापि काश्यपौ}
{तारणेयौ युक्तरूपौ ब्राह्मणावृषिसत्तमौ}


\twolineshloka
{स तावामन्त्रयामास सर्वकामैरतन्द्रितः}
{बुद्ध्वा बलं तयोस्तत्र कनीयांसमुपह्वरे}


\twolineshloka
{प्रपेदे च्छन्दयन्कामैरुपयाजं धृतव्रतम्}
{पादशुश्रूषणे युक्तः प्रियवाक्सर्वकामदः}


\twolineshloka
{अर्चयित्वा यथान्यायमुपयाजमुवाच सः}
{येन मे कर्मणा ब्रह्मन्पुत्रः स्याद्द्रोणमृत्यवे}


\twolineshloka
{`अर्जुनस्य भवेद्भार्या भवेद्या वरवर्णिनी}
{'उपयाज कृते तस्मिन् गवां दाताऽस्मि तेऽर्बुदं}


\twolineshloka
{यद्वा तेऽन्यद्द्विजश्रेष्ठ मनसः सुप्रियं भवेत्}
{सर्वं तत्ते प्रदाताऽहं न हि मेऽत्रास्ति संशयः}


\twolineshloka
{इत्युक्तो नाहमित्येवं तमृषिः प्रत्यभाषत}
{आराधयिष्यन्द्रुपदः स तं पर्यचरत्पुनः}


\twolineshloka
{ततः संवत्सरस्यान्ते द्रुपदं स द्विजोत्तमः}
{उपयाजोऽब्रवीत्काले राजन्मधुरया गिरा}


\twolineshloka
{ज्येष्ठो भ्राता ममागृह्माद्विचरन् गहने वने}
{अपरिज्ञातशौचायां भूमौ निपतितं फलम्}


\twolineshloka
{तदपश्यमहं भ्रातुरसाम्प्रतमनुव्रजन्}
{विमर्शं संकरादाने नायं कुर्यात्कदाचन}


\twolineshloka
{दृष्ट्वा फलस्य नापश्यद्दोषान्पापानुबन्धकान्}
{विविनक्ति न शौचं यः सोऽन्यत्रापि कथं भवेत्}


\twolineshloka
{संहिताध्ययनं कुर्वन्वसन्गुरुकुले च यः}
{भैक्षमुत्सृष्टमन्येषां भुङ्क्ते स्म च यदा तदा}


\twolineshloka
{कीर्तयन्गुणमन्नानामघृणी च पुनः पुनः}
{तं वै फलार्थिनं मन्ये भ्रातरं तर्कचक्षुषा}


\twolineshloka
{तं वै गच्छस्व नृपते स त्वां संयाजयिष्यति}
{जुगुप्समानो नृपतिर्मनसेदं विचिन्तयन्}


\twolineshloka
{उपयाजवचः श्रुत्वा याजस्याश्रममभ्यगात्}
{अभिसम्पूज्य पूजार्हमथ याजमुवाच ह}


\twolineshloka
{अयुतानि ददान्यष्टौ गवां याजय मां विभो}
{द्रोणवैराभिसन्तप्तं प्रह्लादयितुमर्हसि}


\twolineshloka
{स हि ब्रह्मविदां श्रेष्ठो ब्रह्मास्त्रे चाप्यनुत्तमः}
{तस्माद्द्रोणः पराजैष्ट मां वै स सखिविग्रहे}


\twolineshloka
{क्षत्रियो नास्ति तस्यास्यां पृथिव्यां कश्चिदग्रणीः}
{कौरवाचायर्मुख्यस्य भारद्वाजस्य धीमतः}


\twolineshloka
{द्रोणस्य शरजालानि प्राणिदेहहराणि च}
{षडरत्नि धनुश्चास्य दृश्यते परमं महत्}


\twolineshloka
{स हि ब्राह्मणवेषेण क्षात्रं वेगमशंसयम्}
{प्रतिहन्ति महेष्वासो भारद्वाजो महामनाः}


\twolineshloka
{क्षत्रोच्छेदाय विहितो जामदग्न्य इवास्थितः}
{तस्य ह्यस्त्रबलं घोरमप्रधृष्यं नरैर्भुवि}


\twolineshloka
{ब्राह्मं सन्धारयंस्तेजो हुताहुतिरिवानलः}
{समेत्य स दहत्याजौ क्षात्रधर्मपुरःसरः}


\twolineshloka
{ब्रह्मक्षत्रे च विहिते ब्राह्मं तेजो विशिष्यते}
{सोऽहं क्षात्राद्बलाद्धीनो ब्राह्मं तेजः प्रपेदिवान्}


\twolineshloka
{द्रोणाद्विशिष्टमासाद्य भवन्तं ब्रह्मवित्तमम्}
{द्रोणान्तकमहं पुत्रं लभेयं युधि दुर्जयम्}


\twolineshloka
{तत्कर्म कुरु मे मे याज वितराम्यर्बुदं गवाम्}
{तथेत्युक्त्वा तु तं याजो याज्यार्थमुपकल्पयत्}


\twolineshloka
{गुर्वर्थ इति चाकाममुपयाजमचोदयत्}
{याजो द्रोणविनाशाय प्रतिजज्ञे तथा च सः}


\twolineshloka
{ततस्तस्य नरेन्द्रस्य उपयाजो महातपाः}
{आचख्यौ कर्म वैतानं तदा पुत्रफलाय वै}


\twolineshloka
{स च पुत्रो महावीर्यो महातेजा महाबलः}
{इष्यते यद्विधो राजन्भविता ते तथाविधः}


\twolineshloka
{भारद्वाजस्य हन्तारं सोऽभिसन्धाय भूपतिः}
{आजह्वे तत्तथा सर्वं द्रुपदः कर्मसिद्धये}


\threelineshloka
{याजस्तु हवनस्यान्ते देवीमाज्ञापयत्तदा}
{प्रेहि मां राज्ञि पृषति मिथुनं त्वामुपस्थितम् ॥राज्ञ्युवाच}
{}


\threelineshloka
{अवलिप्तं मुखं ब्रह्मन्दिव्यान्गन्धान्बिभर्मि च}
{सूतार्थे नोपलब्धाऽस्मि तिष्ठ याज मम प्रिये ॥याज उवाच}
{}


\threelineshloka
{याजेन श्रपितं हव्यमुपयाजाभिमन्त्रितम्}
{कथं कामं न सन्दध्यात्सा त्वं विप्रेहि तिष्ठ वा ॥ब्राह्मण उवाच}
{}


\twolineshloka
{एवमुक्त्वा तु याजेन हुते हविषि संस्कृते}
{उत्तस्थौ पावकात्तस्मात्कुमारो देवसन्निभिः}


\twolineshloka
{ज्वालावर्णो घोररूपः किरीटी वर्म चोत्तमम्}
{बिभ्रत्सखङ्गः सशरो धनुष्मान्विनदन्मुहुः}


\twolineshloka
{सोऽध्यारोदद्रथवरं तेन च प्रययौ तदा}
{ततः प्रणेदुः पञ्चालाः प्रहृष्टाः साधुसाध्विति}


\twolineshloka
{हर्षाविष्टांस्ततश्चैतान्नेयं सेहे वसुन्धरा}
{भयापहो राजपुत्रः पञ्चालानां यशस्करः}


\twolineshloka
{राज्ञः शोकापहो जात एष द्रोणवधाय वै}
{इत्युवाच महद्भूतमदृश्यं खेचरं तदा}


\twolineshloka
{कुमारी चापि पाञ्चाली वेदीमध्यात्समुत्थिता}
{सुभगा दर्शनीयाङ्गी स्वसितायतलोचना}


\twolineshloka
{श्यामा पद्मपलाशाक्षी नीलकुञ्चितमूर्धजा}
{ताम्रतुङ्गनखी सुभ्रूश्चारुपीनपयोधरा}


\twolineshloka
{मानुषं विग्रहं कृत्वा साक्षादमरवर्णिनी}
{नीलोत्पलसमो गन्धो यस्याः क्रोशात्प्रधावति}


\twolineshloka
{या बिभर्ति परं रूपं यस्या नास्त्युपमा भुवि}
{देवदानवयक्षाणामीप्सितां देवरूपिणीम्}


\twolineshloka
{`सदृशी पाण्डुपुत्रस्य अर्जुनस्येति भारत}
{ऊचुः प्रहृष्टमनसो राजभक्तिपुरस्कृताः ॥'}


\twolineshloka
{तां चापि जातां सुश्रोणीं वागुवाचाशरीरिणी}
{सर्वयोषिद्वरा कृष्णा निनीषुः क्षत्रियान्क्षयम्}


\twolineshloka
{सुरकार्यमियं काले करिष्यति सुमध्यमा}
{अस्या हेतोः कौरवाणां महदुत्पत्स्यते भयम्}


\twolineshloka
{तच्छ्रुत्वा सर्वपञ्चालाः प्रणेदुः सिंहसङ्घवत्}
{न चैतान्हर्षसम्पूर्णानियं सेहे वसुन्धरा}


\threelineshloka
{`पाञ्चालराजस्तां दृष्ट्वा हर्षादश्रूण्यवर्तयत्}
{परिष्वज्य च तां कृष्णां स्नुषा पाण्डोरिति ब्रुवन्}
{अङ्कमारोप्य पाञ्चालीं राजा हर्षमवाप सः ॥'}


\twolineshloka
{तौ दृष्ट्वा पार्षती याजं प्रपेदे वै सुतार्थिनी}
{न वै मदन्यां जननीं जानीयातामिमाविति}


\twolineshloka
{तथेत्युवाच तां याजो राज्ञः प्रियचिकीर्षया}
{तयोश्च नामनी चक्रुर्द्विजाः सम्पूर्णमानसाः}


\twolineshloka
{धृष्टत्वादत्यमर्षित्वाद्द्युम्नाद्युत्संभवादपि}
{धृष्टद्युम्नः कुमारोऽयं द्रुपदस्य भवत्विति}


\threelineshloka
{कृष्णेत्येवाब्रुवकन्कृष्णां कृष्णा}
{ञभूत्सा हि वर्णतः}
{तथा तन्मिथुनं जज्ञे द्रुपदस्य महामखे}


% Check verse!
`वैदिकाध्ययने पारं धृष्टद्युम्नो गतः परम् ॥'
\twolineshloka
{धृष्टद्युम्नं तु पाञ्चाल्यमानीय स्वं निवेशनम्}
{उपाकरोदस्त्रहेतोर्भारद्वाजः प्रतापवान्}


\twolineshloka
{अमोक्षणीयं दैवं हि भावि मत्वा महामतिः}
{तथा तत्कृतवान्द्रोण आत्मकीर्त्यनुरक्षणात्}


\chapter{अध्यायः १८२}
\twolineshloka
{`ब्राह्मण उवाच}
{}


\twolineshloka
{श्रुत्वा जतुगृहे वृत्तं ब्राह्मणाः संशितव्रताः}
{पाञ्चालराजं द्रुपदमिदं वचनमब्रुवन्}


\twolineshloka
{धार्तराष्ट्राः सहामात्या मन्त्रयित्वा परस्परम्}
{पाण्डवानां विनाशाय मतिं चक्रुः सुदुष्कराम्}


\twolineshloka
{दुर्योधनेन प्रहितः पुरोचन इति श्रुतः}
{वारणावतमासाद्य कृत्वा जतुगृहं महत्}


\twolineshloka
{तस्मिन्गृहे सुविस्रब्धान्पाण्डवान्पृथया सह}
{अर्धरात्रे महाराज दग्धवानतिदुर्मतिः}


\twolineshloka
{तेनाग्निना स्वयं चापि दग्धः क्षुद्रो नृशंसवत्}
{एतच्छ्रुत्वा सुसंहृष्टो धृतराष्ट्रः सबान्धवः}


\twolineshloka
{अल्पशोकः प्रहृष्टात्मा शशास विदुरं तदा}
{पाण्डवानां महाप्राज्ञ कुरु पिण्डोदकक्रियाम्}


\twolineshloka
{अहो विधिवशादेव गतास्ते यमसादनम्}
{इत्युक्त्वा प्रारदत्तत्र धृतराष्ट्रः सबान्धवः}


\twolineshloka
{श्रुत्वा भीष्मेण विदुरः कृतवानौर्ध्वदेहिकम्}
{पाण्डवानां विनाशाय कृतं कर्म दुरात्मना}


\threelineshloka
{एतत्कार्यस्य कर्ता तु न दृष्टो न श्रुतः पुरा}
{एतद्वृत्तं महाभाग पाण्डवान्प्रति नः श्रुतम् ॥ब्राह्मण उवाच}
{}


\twolineshloka
{श्रुत्वा तु वचनं तेषां यज्ञसेनो महामतिः}
{यथा तज्जनकः शोचेदौरसस्य विनाशे}


\twolineshloka
{तथाऽतप्यत वै राजा पाण्डवानां विनाशने}
{समाहूय प्रकृतयः सहिताः सर्वनागरैः}


\threelineshloka
{कारुण्यादेव पाञ्चालः प्रोवाचेदं वचस्तदा}
{द्रुपद उवाच}
{अहो रूपमहो धैर्यमहो वीर्यमहो बलम्}


\twolineshloka
{चिन्तयामि दिवारात्रमर्जुनं प्रति बान्धवाः}
{भ्रातृभिः सहितो मात्रा सोऽदह्यत हुताशने}


\twolineshloka
{किमाश्चर्यमितो लोके कालो हि दुरतिक्रमः}
{मिथ्याप्रतिज्ञो लोकेषु किं करिष्यामि सांप्रतं}


\twolineshloka
{अन्तर्गतेन दुःखेन दह्यमानो दिवानिशम्}
{याजोपयाजौ सत्कृत्य याचितौ तौ मयाऽनघौ}


\twolineshloka
{भारद्वाजस्य हन्तारं देवीं चाप्यर्जुनस्य वै}
{लोकस्तद्वेद यच्चापि तथा याजेन मे श्रुतम्}


\twolineshloka
{याजेन पुत्रकामीयं हुत्वा चोत्पादिताविमौ}
{धृष्टद्युम्नश्च कृष्णा च मम तुष्टिकरावुभौ}


\threelineshloka
{किं करिष्यामि ते नष्टाः पाण्डवाः पृथया सह}
{ब्राह्मण उवाच}
{इत्येवमुक्त्वा पाञ्चालः शुशोच परमातुरः}


\twolineshloka
{दृष्ट्वा शोचन्तमत्यर्थं पाञ्चालमिदमब्रवीत्}
{पुरोधाः सत्वसंपन्नः सम्यग्विद्याविशेषवित्}


\twolineshloka
{वृद्धानुशासने सक्ताः पाण्डवा धर्मचारिणः}
{तादृशा न विनश्यन्ति नैव यान्ति पराभवम्}


\twolineshloka
{मया दृष्टमिदं सत्यं शृणु त्वं मनुजाधिप}
{ब्राह्मणैः कथितं सत्यं वेदेषु च मया श्रुतम्}


\twolineshloka
{बृहस्पतिमतेनाथ पौलोम्या च पुरा श्रुतम्}
{नष्ट हन्द्रो बिसग्रन्थ्यामुपश्रुत्या हि दर्शितः}


\twolineshloka
{उपश्रुतिर्महाराज पाण्डवार्थे मया श्रुता}
{यत्रकुत्रापि जीवन्ति पाण्डवास्ते न संशयः}


\twolineshloka
{मया दृष्टानि लिङ्गानि इहैवैष्यन्ति पाण्डवाः}
{यन्निमित्तमिहायान्ति तच्छृणुष्व नराधिप}


\twolineshloka
{स्वयंवरः क्षत्रियाणां कन्यादाने प्रदर्शितः}
{स्वयंवरस्तु नगरे घुष्यतां राजसत्तम}


\twolineshloka
{यत्र वा निवसन्तस्ते पाण्डवाः पृथया सह}
{दूरस्था वा समीपस्था स्वर्गस्था वाऽपि पाण्डवाः}


\threelineshloka
{श्रुत्वा स्वयंवरं राजन्समेष्यन्ति न संशयः}
{तस्मात्स्वयंवरो राजन्घुष्यतां मा चिरं कृथाः ॥ब्राह्मण उवाच}
{}


\twolineshloka
{श्रुत्वा पुरोहितेनोक्तं पाञ्चालः प्रीतिमांस्तदा}
{घोषयामास नगरे द्रौपद्यास्तु स्वयंवरम्}


\twolineshloka
{पुष्यमासे तु रोहिण्यां शुक्लपक्षे शुभे तिथौ}
{दिवसैः पञ्चसप्तत्या भविष्यति न संशयः}


\twolineshloka
{देवगन्धर्वयक्षाश्च ऋषयश्च तपोधनाः}
{स्वयंवरं द्रष्टुकामा गच्छन्त्येव न संशयः}


\twolineshloka
{तव पुत्रा महात्मानो दर्शनीयो विशेषतः}
{यदृच्छया सा पाञ्चाली गच्छेद्वान्यतमं पतिम्}


\twolineshloka
{को हि जानाति लोकेषु प्रजापतिमतं शुभण्}
{तस्मात्सपुत्रा गच्छेथा यदि ब्राह्मणि रोचते}


\twolineshloka
{नित्यकालं सुभिक्षास्ते पाञ्चालास्तु तपोधने}
{यज्ञसेनस्तु राजा स ब्रह्मण्यः सत्यसङ्गरः}


\twolineshloka
{ब्रह्मण्या नागराः सर्वे ब्राह्मणाश्चातिथिप्रियाः}
{नित्यकालं प्रदास्यन्ति आमन्त्रणमयाचितम्}


\twolineshloka
{अहं च तत्र गच्छामि ममैभिः सह शिष्यकैः}
{एकसार्थाः प्रयाताः स्मो ब्राह्मण्या यदि रोचते ॥एतावदुक्त्वा वचनं ब्राह्मणो विरराम ह}


\chapter{अध्यायः १८३}
\twolineshloka
{वैशंपायन उवाच}
{}


\twolineshloka
{एतच्छ्रुत्वा ततः सर्वे पाण्डवा भरतर्षभ}
{मनसा द्रौपदीं जग्मुरनङ्गशरपीडिताः ॥'}


\twolineshloka
{ततस्तां रजनीं राजञ्छल्यविद्धा इवाभवन्}
{सर्वे चास्वस्थमनसो बभूवुस्ते महाबलाः}


\twolineshloka
{ततः कुन्ती सुतान्दृष्ट्वा सर्वांस्तद्गतचेतसः}
{युधिष्ठिरमुवाचेदं वचनं सत्यवादिनी}


\twolineshloka
{चिररात्रोषिताः स्मेह ब्राह्मणस्य निवेशने}
{रममाणाः पुरे रम्ये लब्धभैक्षा महात्मनः}


\twolineshloka
{यानीह रमणीयानि वनान्युपवनानि च}
{सर्वाणि तानि दृष्टानि पुनःपुनररिन्दम}


\twolineshloka
{पुनर्दृष्टानि तानीह प्रीणयन्ति न नस्तथा}
{भैक्षं च न तथा वीर लभ्यते कुरुनन्दन}


\twolineshloka
{ते वयं साधु पञ्चालान्गच्छाम यदि मन्यसे}
{अपूर्वदर्शनं वीर रमणीयं भविष्यति}


\twolineshloka
{सुभिक्षाश्चैव पञ्चालाः श्रूयन्ते शत्रुकर्शन}
{यज्ञसेनश्च राजाऽसौ ब्रह्मण्य इति सुश्रुम}


\threelineshloka
{एकत्र चिरवासश्च क्षमो न च मतो मम}
{ते तत्र साधु गच्छामो यदि त्वं पुत्र मन्यसे ॥युधिष्ठिर उवाच}
{}


\threelineshloka
{भवत्या यन्मतं कार्यं तदस्माकं परं हितम्}
{अनुजांस्तु न जानामि गच्छेयुर्नेति वा पुनः ॥वैशंपायन उवाच}
{}


\twolineshloka
{ततः कुन्ती भीमसेनमर्जुनं यमजौ तथा}
{उवाच गमनं ते च तथेत्येवाब्रुवंस्तदा}


\twolineshloka
{तत आमन्त्र्य तं विप्रं कुन्ती राजसुतैः सह}
{प्रतस्थे नगरीं रम्यां द्रुपदस्य महात्मनः}


\chapter{अध्यायः १८४}
\twolineshloka
{वैशंपायन उवाच}
{}


\twolineshloka
{वसत्सु तेषु प्रच्छन्नं पाण्डवेषु महात्मसु}
{आजगामाथ तान्द्रष्टुं व्यासः सत्यवतीसुतः}


\twolineshloka
{तमागतमभिप्रेक्ष्य प्रत्युद्गम्य परन्तपाः}
{प्रणिपत्याभिवाद्यैनं तस्थुः प्राञ्जलयस्तदा}


\twolineshloka
{समनुज्ञाप्य तान्सर्वानासीनान्मुनिरब्रवीत्}
{प्रच्छन्नं पूजितः पार्थैः प्रीतिपूर्वमिदं वचः}


\twolineshloka
{अपि धर्मेण वर्तध्वं शास्त्रेण च परन्तपाः}
{अपि विप्रेषु पूजा वः पूजार्हेषु न हीयते}


\twolineshloka
{अथ धर्मार्थवद्वाक्यमुक्त्वा स भगवानृषिः}
{विचित्राश्च कथास्तास्ताः पुनरेवेदमब्रवीत्}


\twolineshloka
{आसीत्तपोवने काचिदृषेः कन्या महात्मनः}
{विलग्नमध्या सुश्रोणी सुभ्रूः सर्वगुणान्विता}


\twolineshloka
{कर्मभिः स्वकृतैः सा तु दुर्भगा समपद्यत}
{नाध्यगच्छत्पतिं सा तु कन्या रूपवती सती}


\twolineshloka
{तपस्तप्तुमथारेभे पत्यर्थमसुखा ततः}
{तोषयामास तपसा सा किलोग्रेण शङ्करम्}


\twolineshloka
{तस्याः स भगवांस्तुष्टस्तामुवाच यशस्विनीम्}
{वरं वरय भद्रं ते वरदोऽस्मीति शङ्करः}


\twolineshloka
{अथेश्वरमुवाचेदमात्मनः सा वचो हितम्}
{पतिं सर्वगुणोपेतमिच्छामीति पुनःपुनः}


\twolineshloka
{तामथ प्रत्युवाचेदमीशानो वदतां वरः}
{पञ्च ते पतयो भद्रे भविष्यन्तीति भारताः}


\twolineshloka
{एवमुक्ता ततः कन्या देवं वरदमब्रवीत्}
{एकमिच्छाम्यहं देव त्वत्प्रसादात्पतिं प्रभो}


\twolineshloka
{पुनरेवाब्रवीद्देव इदं वचनमुत्तमम्}
{पञ्चकृत्वस्त्वया ह्युक्तः पतिं देहीत्यहं पुनः}


\threelineshloka
{देहमन्यं गतायास्ते यथोक्तं तद्भविष्यति}
{व्यास उवाच}
{द्रुपदस्य कुले जज्ञे सा कन्या देवरूपिणी}


\threelineshloka
{निर्दिष्टा भवतां पत्नी कृष्णा पार्षत्यनिन्दिता}
{पाञ्चालनगरे तस्मान्निवसध्वं महाबलाः}
{सुखिनस्तामनुप्राप्य भविष्यथ न संशयः}


\twolineshloka
{एवमुक्त्वा महाभागः पाण्डवान्स पितामहः}
{पार्थानामन्त्र्य कुन्तीं च प्रातिष्ठत महातपाः}


\chapter{अध्यायः १८५}
\twolineshloka
{वैशंपायन उवाच}
{}


\twolineshloka
{गते भगवति व्यासे पाण्डवा हृष्टमानसाः}
{`ते प्रयाता नरव्याघ्रा मात्रा सह परन्तपाः}


\twolineshloka
{ब्राह्मणान्गच्छतो पश्यन्पाञ्चालान्सगणान्पथि}
{अथ ते ब्राह्मणा ऊचुः पाण्डवान्ब्रह्मचारिणः}


\threelineshloka
{क्व भवन्तो गमिष्यन्ति कुतो वाऽऽगच्छथेति ह}
{युधिष्ठिर उवाच}
{प्रयातानेकचक्रायाः सोदर्यान्देवदर्शिनः}


\twolineshloka
{भवन्तो नोऽभिजानन्तु सहितान्ब्रह्मचारिणः}
{गच्छतो नस्तु पाञ्चालान्द्रुपदस्य पुरं प्रति}


\twolineshloka
{इच्छामो भवतो ज्ञातुं परं कौतूहलं हि नः ॥ब्राह्मणा ऊचुः}
{}


\twolineshloka
{एते सार्धं प्रयाताः स्मो वयमप्यत्र गामिनः}
{तत्राप्यद्भुतसङ्काश उत्सवो भविता महान्}


\twolineshloka
{ततस्तु यज्ञसेनस्य द्रुपदस्य महात्मनः}
{यासावयोनिजा कन्या स्थास्यते सा स्वयंवरे}


\twolineshloka
{दर्शनीयाऽनवद्याङ्गी सुकुमारी यशस्विनी}
{धृष्टद्युम्नस्य भगिनी द्रोणशत्रोः प्रतापिनः}


\twolineshloka
{जातो यः पावकाच्छूरः सशरः सशरासनः}
{सुसमिद्धान्महाभागः सोमकानां महारथः}


\twolineshloka
{यस्मिन्संजायमाने हि वागुवाचाशरीरिणी}
{एष मृत्युश्च शिष्यश्च भारद्वाजस्य जायते}


\twolineshloka
{स्वसा तस्य तु वेद्याश्च जाता तस्मिन्महामखे}
{स्त्रीरत्नमसितापाङ्गी श्यामा नीलोत्पलद्युतिः}


\twolineshloka
{तां यज्ञसेनस्य सुतां द्रौपदीं परमां स्त्रियम्}
{गच्छामस्तत्र वै द्रष्टुं तं चैवास्याः स्वयंवरम्}


\twolineshloka
{राजानो राजपुत्राश्च यज्वानो भूरिदक्षिणाः}
{स्वाध्यायवन्तः शुचयो महात्मानो धृतव्रताः}


\twolineshloka
{तरुणा दर्शनीयाश्च बलवन्तो दुरासदाः}
{महारथाः कृतास्त्राश्च समेष्यन्तीह भूमिपाः}


\twolineshloka
{ते तत्र विविधं दानं विजयार्थं नरेश्वराः}
{प्रदास्यन्ति धनं गाश्च भक्ष्यभोज्यानि सर्वशः}


\twolineshloka
{प्रतिलप्स्यामहे सर्वं दृष्ट्वा कृष्णां स्वयंवरे}
{यं च सा क्षत्रियं रङ्गे कुमारी वरयिष्यति}


\twolineshloka
{तदा वैतालिकाश्चैव नर्तकाः सूतमागधाः}
{निबोधकाश्च देशेभ्यः समेष्यन्ति महाबलाः}


\twolineshloka
{एतत्कौतूहलं तत्र दृष्ट्वा वै प्रतिगृह्य च}
{सहास्माभिर्महात्मानो मात्रा सह निवत्स्यथ}


\twolineshloka
{दर्शनीयांश्च वः सर्वानेकरूपानवस्थितान्}
{समीक्ष्य कृष्मा वरयेत्संगत्यान्यतमं पतिम्}


\threelineshloka
{अयमेकश्च वो भ्राता दर्शनीयो महाभुजः}
{नियुध्यमानो विजयेत्संगत्य द्रविणं महत् ॥युधिष्ठिर उवाच}
{}


\twolineshloka
{परमं भो गमिष्यामो द्रष्टुं तत्र स्वयंवरम्}
{द्रौपदीं यज्ञसेनस्य कन्यां तस्यास्तथोत्सवम् ॥'}


\chapter{अध्यायः १८६}
\twolineshloka
{वैशंपायन उवाच}
{}


\twolineshloka
{ते प्रतस्थुः पुरस्कृत्य मातरं पुरुषर्षभाः}
{समैरुदङ्मुकैर्मार्गैर्यथोद्दिष्टं च भारत}


\twolineshloka
{अहोरात्रेणाभ्यगच्छन्पाञ्चालनगरं प्रति}
{अभ्याजग्मुर्लोकनदीं गङ्गां भागीरथीं प्रति}


\threelineshloka
{चन्द्रास्तमयवेलायामर्धरात्रसमागमे}
{वारि चैवानुमज्जन्तस्तीर्थं सोमाश्रयायणम्}
{}


\twolineshloka
{आसेदुः पुरुषव्याघ्रा गङ्गायां पाण्डुनन्दनाः ॥उल्मुकं तु समुद्यम्य तेषामग्रे धनञ्जयः}
{}


\twolineshloka
{प्रकाशार्थं ययौ तत्र रक्षार्थं च महारथः ॥तत्र गङ्गाजले रम्ये विविक्ते क्रीडयन् स्त्रियः}
{}


\twolineshloka
{शब्दं तेषां स शुश्राव नदीं समुपसर्पताम्}
{तेन शब्देन चाविष्टश्चुक्रोध बलवद्बली}


\twolineshloka
{स दृष्ट्वा पाण्डवांस्तत्र सह मात्रा परन्तपान्}
{विष्फारयन्धनुर्घोरमिदं वचनमब्रवीत्}


\twolineshloka
{सन्ध्या संरज्यते घोरा पूर्वरात्रागमेषु या}
{अशीतिभिर्लवैर्हीनं तन्मुहूर्तं प्रचक्षते}


\twolineshloka
{विहितं कामचाराणां यक्षगन्धर्वरक्षसाम्}
{शेषमन्यन्मनुष्याणां कर्मचारेषु वै स्मृतम्}


\twolineshloka
{लोभात्प्रचारं चरतस्तासु वेलासु वै नरान्}
{उपक्रान्ता निगृह्णीमो राक्षसैः सह बालिशान्}


\twolineshloka
{अतो रात्रौ प्राप्नुवतो जलं ब्रह्मविदो जनाः}
{गर्हयन्ति नरान्सर्वान्बलस्थान्नृपतीनपि}


\twolineshloka
{आराच्च तिष्ठतास्माकं समीपं नोपसर्पत}
{कस्मान्मां नाभिजानीत प्राप्तं भागीरथीजलम्}


\twolineshloka
{अङ्गारपर्णं गन्धर्वं वित्त मां स्वबलाश्रयम्}
{अहं हि मानी चेर्ष्युश्च कुबेरस्य प्रियः सखा}


\twolineshloka
{अङ्गारपर्णमित्येवं ख्यातं चेदं वनं मम}
{अनुगङ्गं चरन्कामांश्चित्रं यत्र रमाम्यहम्}


\threelineshloka
{न कौणपाः शृङ्गिणो वा न देवा न च मानुषाः}
{कुबेरस्य यथोष्णीषं किं मां समुपसर्पथ ॥अर्जुन उवाच}
{}


\twolineshloka
{समुद्रे हिमवत्पार्श्वे नद्यामस्यां च दुर्मते}
{रात्रावहनि सन्ध्यायां कस्य क्लृप्तः परिग्रहः}


\twolineshloka
{भुक्तो वाऽप्यथ वाऽभुक्तो रात्रावहनि खेचर}
{न कालनियमो ह्यस्ति गङ्गां प्राप्य सरिद्वरां}


\twolineshloka
{वयं च शक्तिसम्पन्ना अकाले त्वामधृष्णुम}
{अशक्ता हि रणे क्रूर युष्मानर्चन्ति मानवाः}


\twolineshloka
{पुरा हिमवतश्चैषा हेमशृङ्गाद्विनिःसृता}
{गङ्गा गत्वा समुद्राम्भः सप्तधा समपद्यत}


\twolineshloka
{गङ्गां च यमुनां चैव प्लक्षजातां सरस्वतीम्}
{रथस्थां सरयूं चैव गोमतीं गण्डकीं तथा}


\twolineshloka
{अपर्युषितपापास्ते नदीः सप्त पिबन्ति ये}
{इयं भूत्वा चैकवप्रा शुचिराकाशगा पुनः}


\threelineshloka
{देवेषु गङ्गा गन्धर्व प्राप्नोत्यलकनन्दताम्}
{तथा पितॄन्वैतरणी दुस्तरा पापकर्मभिः}
{गङ्गा भवति वै प्राप्य कृष्णद्वैपायनोऽब्रवीत्}


\twolineshloka
{असम्बाधा देवनदी स्वर्गसंपादनी शुभा}
{कथमिच्छसि तां रोद्धुं नैष धर्मः सनातनः}


\threelineshloka
{अनिवार्यमसम्बाधं तव वाचा कथं वयम्}
{न स्पृशेम यथाकामं पुण्यं भागीरथीजलम् ॥वैशंपायन उवाच}
{}


\twolineshloka
{अङ्गारपर्णस्तच्छ्रुत्वा क्रुद्ध आनाम्य कार्मुकम्}
{मुमोच बाणान्निशितानहीनाशीविषानिव}


\threelineshloka
{उल्मुकं भ्रामयंस्तूर्णं पाण्डवश्चर्म चोत्तरम्}
{व्यपोहत शरांस्तस्य सर्वानेव धनञ्जयः ॥अर्जुन उवाच}
{}


\twolineshloka
{बिभीषिका वै गन्धर्व नास्त्रज्ञेषु प्रयुज्यते}
{अस्त्रज्ञेषु प्रयुक्तेयं फेनवत्प्रविलीयते}


\twolineshloka
{मानुषानति गन्धर्वान्सर्वान् गन्धर्व लक्षये}
{तस्मादस्त्रेण दिव्येन योत्स्येऽहं न तु मायया}


\twolineshloka
{पुराऽस्त्रमिमाग्नेयं प्रादात्किल बृहस्पतिः}
{भरद्वाजाय गन्धर्व गुरुर्मान्यः शतक्रतोः}


\threelineshloka
{भरद्वजादग्निवेश्य अग्निवेश्याद्गुरुर्मम}
{साध्विदं मह्यमददद्द्रोणो ब्राह्मणसत्तमः ॥वैशंपायन उवाच}
{}


\twolineshloka
{इत्युक्त्वा पाण्डवः क्रुद्धो गन्धर्वाय मुमोच ह}
{प्रदीप्तमस्त्रमाग्नेयं ददाहास्य रथं तु तत्}


\twolineshloka
{विरथं विप्लुतं तं तु स गन्धऱ्वं महाबलः}
{अस्त्रतेजःप्रमूढं च प्रपतन्तमवाङ्मुखम्}


\twolineshloka
{शिरोरुहेषु जग्राह माल्यवत्सु धनञ्जयः}
{भ्रातॄन्प्रति चकर्षाथ सोऽस्त्रपातादचेतसम्}


\threelineshloka
{युधिष्ठिरं तस्य भार्या प्रपेदे शरणार्थिनी}
{नाम्ना कुम्भीनसी नाम पतित्राणमभीप्सती ॥गन्धर्व्युवाच}
{}


\threelineshloka
{त्रायस्व मां महाभाग पतिं चेमं विमुञ्च मे}
{गन्धर्वी शरणं प्राप्ता नाम्ना कुम्भीनसी प्रभो ॥युधिष्ठिर उवाच}
{}


\threelineshloka
{युद्धे जितं यशोहीनं स्त्रीनाथमपराक्रमम्}
{को निहन्याद्रिपुं तात मुञ्चेमं रिपुसूदन ॥अर्जुन उवाच}
{}


\threelineshloka
{जीवितं प्रतिपद्यस्व गच्छ गन्धर्व मा शुचः}
{प्रदिशत्यभयं तेऽद्य कुरुराजो युधिष्ठिरः ॥गन्धर्व उवाच}
{}


\twolineshloka
{जितोऽहं पूर्वकं नाम मुञ्चाम्यङ्गारपर्णताम्}
{यशोहीनं न च श्लाघ्यं स्वं नाम जनसंसदि}


\twolineshloka
{साध्विमं लब्धवाँल्लाभं योऽहं दिव्यास्त्रधारिणम्}
{गान्धर्व्या माययेच्छामि संयोजयितुमर्जुनम्}


\twolineshloka
{अस्त्राग्निना विचित्रोऽयं दग्धो मे रथ उत्तमः}
{सोऽहं चित्ररथो भूत्वा नाम्ना दग्धरथोऽभवं}


\twolineshloka
{संभृता चैव विद्येयं तपसेह मया पुरा}
{निवेदयिष्ये तामद्य प्राणदाय महात्मने}


\twolineshloka
{संस्तम्भयित्वा तरसा जितं शरणमागतम्}
{यो रिपुं योजयेत्प्राणैः कल्याणं किं न सोऽर्हति}


\twolineshloka
{चाक्षुषी नाम विद्येयं यां सोमाय ददौ मनुः}
{ददौ स विश्वावसवे मम विश्वावसुर्ददौ}


\twolineshloka
{सेयं कापुरुषं प्राप्ता गुरुदत्ता प्रणश्यति}
{आगमोऽस्या मया प्रोक्तो वीर्यं प्रतिनिबोध मे}


\twolineshloka
{यच्चक्षुषा द्रष्टुमिच्छेत्रिषु लोकेषु किंचन}
{तत्पश्येद्यादृशं चेच्छेत्तादृशं द्रष्टुमर्हति}


\twolineshloka
{एकपादेन षण्मासान्स्थितो विद्यां लभेदिमाम्}
{अनुनेष्याम्यहं विद्यां स्वयं तुभ्यं व्रते कृते}


\twolineshloka
{विद्यया ह्यनया राजन्वयं नृभ्यो विशेषिताः}
{अविशिष्टाश्च देवानामनुभावप्रदर्शिनः}


\twolineshloka
{गन्धर्वजानामश्वानामहं पुरुषसत्तम}
{भ्रातृभ्यस्तव तुभ्यं च पृथग्दाता शतं शतं}


\twolineshloka
{देवगन्धर्ववाहास्ते दिव्यवर्णा मनोजवाः}
{क्षीणाक्षीणा भवन्त्येते न हीयन्ते च रंहसः}


\twolineshloka
{पुरा कृतं महेन्द्रस्य वज्रं वृत्रनिबर्हणम्}
{दशधा शतधा चैव तच्छीर्णं वृत्रमूर्धनि}


\twolineshloka
{ततो भागीकृतो देवैर्वज्रभाग उपास्यते}
{लोके यशोधनं किंचित्सैव वज्रतनुः स्मृता}


\twolineshloka
{वज्रपाणिर्ब्राह्मणः स्यात्क्षत्रं वज्ररथं स्मृतम्}
{वैश्या वै दानवज्राश्च कर्मवज्रा यवीयसः}


\twolineshloka
{क्षत्रवज्रस्य भागेन अवध्या वाजिनः स्मृताः}
{रथाङ्गं वडबा सूते शूराश्चाश्वेषु ये मताः}


\threelineshloka
{कामवर्णाः कामजवाः कामतः समुपस्थिताः}
{इति गन्धर्वजाः कामं पूरयिष्यन्ति मे हयाः ॥अर्जुन उवाच}
{}


\threelineshloka
{यदि प्रीतेन मे दत्तं संशये जीवितस्य वा}
{विद्याधं श्रुतं वाऽपि न तद्गन्धर्व रोचये ॥गन्धर्व उवाच}
{}


\twolineshloka
{संयोगो वै प्रीतिकरो महत्सु प्रतिदृश्यते}
{जीवितस्य प्रदानेन प्रीतो विद्यां ददामि ते}


\threelineshloka
{त्वत्तोऽप्यहं ग्रहीष्यामि अस्त्रमाग्नेयमुत्तमम्}
{तथैव योग्यं बीभत्सो चिराय मरतर्षभ ॥अर्जुन उवाच}
{}


\twolineshloka
{त्वत्तोऽस्त्रेण वृणोम्यश्वान्संयोगः शास्वतोऽस्तुनौ}
{सखे तद्ब्रूहि गन्धर्व युष्मभ्यो यद्भयं भवेत्}


\threelineshloka
{कारणं ब्रूहि गन्धर्व किं तद्येन स्म धर्षिताः}
{यान्तो वेदविदः सर्वे सन्तो रात्रावरिन्दमाः ॥गन्धर्व उवाच}
{}


\twolineshloka
{अनग्नयोऽनाहुतयो न च विप्रपुरस्कृताः}
{यूयं ततो धर्षिताः स्थ मया वै पाण्डुनन्दनाः}


\twolineshloka
{`यक्षराक्षसगन्धर्वपिशाचपतगोरगाः}
{धर्षन्ति नरव्याघ्र न ब्राह्मणपुरस्कृतान्}


\twolineshloka
{जानतापि मया तस्मात्तेजश्चाभिजनं च वः}
{इयमग्निमतां श्रेष्ठ धर्षिता वै पुरागतिः}


\twolineshloka
{को हि वस्त्रिषु लोकेषु न वेद भरतर्षभ}
{स्वैर्गुणैर्विस्तृतं श्रीमद्यशोऽग्र्यं भूरिवर्चसाम्'}


\twolineshloka
{यक्षराक्षसगन्धर्वाः पिशाचोरगदानवाः}
{विस्तरं कुरुवंशस्य धीमन्तः कथयन्ति ते}


\twolineshloka
{नारदप्रभृतीनां तु देवर्षीणां मया श्रुतम्}
{गुणान्कथयतां वीर पूर्वेषां तव धीमताम्}


\twolineshloka
{स्वयं चापि मया दृष्टश्चरता सागराम्बराम्}
{इमां वसुमतीं कृत्स्नां प्रभावः सुकुलस्य ते}


\twolineshloka
{वेदे धनुषि चाचार्यमभिजानामि तेऽर्जुन}
{विश्रुतं त्रिषु लोकेषु भारद्वाजं यशस्विनम्}


\twolineshloka
{`सर्ववेदविदां श्रेष्ठं सर्वशस्त्रभृतां वरम्}
{द्रोणमिष्वस्त्रकुशलं धनुष्यह्गिरसां वरम् ॥'}


\threelineshloka
{धर्मं वायुं च शक्रं च विजानाम्यश्विनौ तथा}
{पाण्डुं च कुरुशार्दूल षडेतान्कुरुवर्धनान्}
{पितॄनेतानहं पार्थ देवमानुषसत्तमान्}


\twolineshloka
{दिव्यात्मानो महात्मानः सर्वशस्त्रभृतां वराः}
{भवन्तो भ्रातरः शूराः सर्वे सुचरितव्रताः}


\twolineshloka
{उत्तमां च मनोबुद्धिं भवतां भावितात्मनाम्}
{जानन्नपि च वः पार्थ कृतवानिह धर्षणाम्}


\twolineshloka
{स्त्रीसकाशे च कौरव्य न पुमान्क्षन्तुमर्हति}
{धर्षणामात्मनः पश्यन्ब्राहुद्रविणमाश्रितः}


\twolineshloka
{नक्तं च बलमस्माकं भूय एवाभिवर्धते}
{यतस्ततो मां कौन्तेय सदारं मन्युराविशत्}


\twolineshloka
{सोऽहं त्वयेह विजितः सङ्ख्ये तापत्यवर्धन}
{येन तेनेह विधिना कीर्त्यमानं निबोध मे}


\threelineshloka
{ब्रह्मचर्यं परो धर्मः स चापि नियतस्त्वयि}
{यस्मात्तस्मादहं पार्थ रणे}
{ञस्मि विजितस्त्वया}


\twolineshloka
{यस्तु स्यात्क्षत्रियः कश्चित्कामवृत्तः परन्तप}
{नक्तं च युधि युध्येत न स जीवेत्कथंचन}


\twolineshloka
{यस्तु स्यात्कामवृत्तोऽपि पार्थ ब्रह्मपुरस्कृतः}
{जयेन्नक्तञ्चरान्सर्वान्स पुरोहितधूर्गतः}


\twolineshloka
{तस्मात्तापत्य यत्किंचिन्नृणां श्रेय इहेप्सितम्}
{तस्मिन्कर्मणि योक्तव्या दान्तात्मानः पुरोहिताः}


\twolineshloka
{वेदे षडङ्गे निरताः शुचयः सत्यवादिनः}
{धर्मात्यागः कृतात्मानः स्युर्नृपाणां पुरोहिताः}


\twolineshloka
{जयश्च नियतो राज्ञः स्वर्गश्च तदनन्तरम्}
{यस्य स्याद्धर्मविद्वाग्मी पुरोधाः शीलवाञ्शुचिः}


\twolineshloka
{लाभं लब्धुमलब्धं वा लब्धं वा परिरक्षितुम्}
{पुरोहितं प्रकुर्वीत राजा गुणसमन्वितम्}


\twolineshloka
{पुरोहितमते तिष्ठेद्य इच्छेद्भूतिमात्मनः}
{प्राप्तुं वसुमतीं सर्वां सर्वशः सागराम्बराम्}


\twolineshloka
{न हि केवलशौर्येण तापत्याभिजनेन च}
{जयेदब्राह्मणः कश्चिद्भूमिं भूमिपतिः क्वचित्}


\twolineshloka
{तस्मादेवं विजानीहि कुरूणां वंशवर्धन}
{ब्राह्मणप्रमुखं राज्यं शक्यं पालयितुं चिरम्}


\chapter{अध्यायः १८७}
\twolineshloka
{अर्जुन उवाच}
{}


\twolineshloka
{तापत्य इति यद्वाक्यमुक्तवानसि मामिह}
{तदहं ज्ञातुमिच्छामि तापत्यार्थं विनिश्चितम्}


\threelineshloka
{तपती नाम का चैषा तापत्या यत्कृते वयम्}
{कौन्तेया हि वयं साधो तत्त्वमिच्छामि वेदितुम् ॥वैशंपायन उवाच}
{}


\twolineshloka
{एवमुक्तः स गन्धर्वः कुन्तीपुत्रं धनञ्जयम्}
{विश्रुतं त्रिषु लोकेषु श्रावयामास वै कथाम्}


\twolineshloka
{हन्त ते कथयिष्यामि कथामेतां मनोरमाम्}
{यथावदखिलां पार्थ सर्वबुद्धिमतां वर}


\twolineshloka
{उक्तवानस्मि येन त्वां तापत्य इति यद्वचः}
{तत्तेऽहं कथयिष्यामि शृणुष्वैकमना भव}


\twolineshloka
{य एष दिवि धिष्ण्येन नाकं व्याप्नोति तेजसा}
{एतस्य तपती नाम बभूव सदृशी सुता}


\twolineshloka
{विवस्वतो वै देवस्य सावित्र्यवरजा विभो}
{विश्रुता त्रिषु लोकेषु तपती तपसा युता}


\twolineshloka
{न देवी नासुरी चैव न यक्षी न च राक्षसी}
{नाप्सरा न च गन्धर्वी तथा रूपेण काचन}


\twolineshloka
{सुविभक्तानवद्याङ्गी स्वसितायतलोचना}
{स्वाचारा चैव साध्वी च सुवेषा चैव भामिनी}


\twolineshloka
{त तस्याः सदृशं कंचित्त्रिषु लोकेषु भारत}
{भर्तारं सविता मेने रूपशीलगुणश्रुतैः}


\threelineshloka
{संप्राप्तयौवनां पश्यन्देयां दुहितरं तु ताम्}
{`द्व्यष्टवर्षां तु तां श्यामां सविता रूपशालिनीम्}
{'नोपलेभे ततः शान्तिं संप्रदानं विचिन्तयन्}


\twolineshloka
{अथर्क्षपुत्रः क्रान्तेय कुरूणामृषभो बली}
{सूर्यमाराधयामास नृपः संवरणस्तदा}


\twolineshloka
{अर्ध्यमाल्योपहाराद्यैर्गन्धैश्च नियतः शुचिः}
{नियमैरुपवासैश्च तपोभिर्विविधैरपि}


\twolineshloka
{सुश्रूषुरनहंवादी शुचिः पौरवनन्दन}
{अंशुमन्तं समुद्यन्तं पूजयामास भक्तिमान्}


\twolineshloka
{ततः कृतज्ञं धर्मज्ञं रूपेणासदृशं भुवि}
{तपत्याः सदृशं मेने सूर्यः संवरणं पतिम्}


\twolineshloka
{दातुमैच्छत्ततः कन्यां तस्मै संवरणाय ताम्}
{नृपोत्तमाय कौरव्य विश्रुताभिजनाय च}


\twolineshloka
{यथा हि दिवि दीप्तांशुः प्रभासयति तेजसा}
{तथा भुवि महिपालो दीप्त्या संवरणोऽभवत्}


\twolineshloka
{यथाऽर्चयन्ति चादित्यमुद्यन्तं ब्रह्मवादिनः}
{तथा संवरणं पार्थ ब्राह्मणावरजाः प्रजाः}


\twolineshloka
{स सोममति कान्तत्वादादित्यमति तेजसा}
{बभूव नृपतिः श्रीमान्सुहृदां दुर्हृदामपि}


\twolineshloka
{एवंगुणस्य नृपतेस्तथावृत्तस्य कौरव}
{तस्मै दातुं मनश्चक्रे तपतीं तपनः स्वयम्}


\twolineshloka
{स कदाचिदथो राजा श्रीमानमितविक्रमः}
{चचार मृगयां पार्थ पर्वतोपवने किल}


\twolineshloka
{चरतो मृगयां तस्य क्षुत्पिपासासमन्वितः}
{ममार राज्ञः कौन्तेय गिरावप्रतिमो हयः}


\twolineshloka
{स मृताश्वश्चरन्पार्थ पद्भ्यामेव गिरौ नृपः}
{ददर्शासदृशीं लोके कन्यामायतलोचनाम्}


\twolineshloka
{स एव एकामासाद्य कन्यां परबलार्दनः}
{तस्थौ नृपतिशार्दूलः पश्यन्नविचलेक्षणः}


\twolineshloka
{स हि तां तर्कयामास रूपतो नृपतिः श्रियम्}
{पुनः संतर्कयामास रवेर्भ्रष्टामिव प्रभाम्}


\twolineshloka
{वपुषा वर्चसा चैव शिखामिव विभावसोः}
{प्रसन्नत्वेन कान्त्या च चन्द्ररेखामिवामलाम्}


\twolineshloka
{गिरिपृष्ठे तु सा यस्मिन्स्थिता स्वसितलोचना}
{विभ्राजमाना शुशुभे प्रतिमेव हिरण्मयी}


\twolineshloka
{तस्या रूपेण स गिरिर्वेषेण च विशेषतः}
{ससवृक्षक्षुपलतो हिरण्मय इवाभवत्}


\twolineshloka
{अवमेने च तां दृष्ट्वा सर्वलोकेषु योषितः}
{अवाप्तं चात्मनो मेने स राजा चक्षुषः फलं}


\twolineshloka
{जन्मप्रभृति यत्किचिंद्दृष्टवान्स महीपतिः}
{रूपं न सदृशं तस्यास्तर्कयामास किंचन}


\twolineshloka
{तया बद्धमनश्चक्षुः पाशैर्गुणमयैस्तदा}
{न चचाल ततो देशाद्बुबुधे न च किंचन}


\twolineshloka
{अस्या नूनं विशालाक्ष्याः सदेवासुरमानुषम्}
{लोकं निर्मथ्य धात्रेदं रूपमाविष्कृतं कृतम्}


\twolineshloka
{एवं संतर्कयामास रूपद्रविणसंपदा}
{कन्यामसदृशीं लोके नृपः संवरणस्तदा}


\twolineshloka
{तां च दृष्ट्वैव कल्याणीं कल्याणाभिजनो नृपः}
{जगाम मनसा चिन्तां कामबाणेन पीडितः}


\twolineshloka
{दह्यमानः स तीव्रेण नृपतिर्मन्मथाग्निना}
{अप्रगल्भां प्रगल्भस्तां तदोवाच मनोहराम्}


\twolineshloka
{काऽसि कस्यासि रम्भोरु किमर्थं चेह तिष्ठसि}
{कथं च निर्जनेऽरण्ये चरस्येका शुचिस्मिते}


\twolineshloka
{त्वं हि सर्वानवद्याङ्गी सर्वाभरणभूषिता}
{विभूषणमिवैतेषां भूषणानामभीप्सितम्}


\twolineshloka
{न देवीं नासुरीं चैव न यक्षीं न च राक्षसीम्}
{न च भोगवतीं मन्ये न गन्धवीं न मानुषीम्}


\twolineshloka
{या हि दृष्टा मया काश्चिच्छ्रुता वाऽपि वराङ्गनाः}
{न तासां सदृशीं मन्ये त्वामहं मत्तकाशिनि}


\twolineshloka
{दृष्ट्वैव चारुवदने चन्द्रात्कान्ततरं तव}
{वदनं पद्मपत्राक्षं मां मथ्नातीव मन्मथः}


\twolineshloka
{एवं तां स महीपालो बभाषे न तु सा तदा}
{कामार्तं निर्जनेऽरण्ये प्रत्यबाषथ किंचन}


\twolineshloka
{ततो लालप्यमानस्य पार्थिवस्यायतेक्षणा}
{सौदामिनीव चाभ्रेषु तत्रैवान्तरधीयत}


\twolineshloka
{तामन्वेष्टुं स नृपतिः परिचक्राम सर्वतः}
{वनं वनजपत्राक्षीं भ्रमन्नुन्मत्तवत्तदा}


\twolineshloka
{अपश्यमानः स तु तां बहु तत्र विलप्य च}
{निश्चेष्टः पार्थिवश्रेष्ठो मुहूर्तं स व्यतिष्ठत}


\chapter{अध्यायः १८८}
\twolineshloka
{गन्धर्व उवाच}
{}


\twolineshloka
{अथ तस्यामदृश्यायां नृपतिः काममोहितः}
{पातनः शत्रुसङ्घानां पपात धरणीतले}


\twolineshloka
{तस्मिन्निपतिते भूमावथ सा चारुहासिनी}
{पुनः पीनायतश्रोणी दर्शयामास तं नृपम्}


\twolineshloka
{अथाबभाषे कल्याणी वाचा मधुरया नृपम्}
{तं कुरूणां कुलकरं कामाभिहतचेतसम्}


\twolineshloka
{उवाच मधुरं वाक्यं तपती हसतीव सा}
{उत्तिष्ठोत्तिष्ठ भद्रं ते न त्वमर्हस्यरिन्दम}


\twolineshloka
{मोहं नृपतिशार्दूल गन्तुमाविष्कृतः क्षितौ}
{एवमुक्तोऽथ नृपतिर्वाचा मधुरया तदा}


\twolineshloka
{ददर्श विपुलश्रोणीं तामेवाभिमुखे स्थिताम्}
{अथ तामसितापाङ्गीमाबभाषे स पार्थिवः}


\twolineshloka
{मन्मथाग्निपरीतात्मा सन्दिग्धाक्षरया गिरा}
{साधु त्वमसितापाङ्गि कामार्तं मत्तकाशिनि}


\twolineshloka
{भजस्व भजमानं मां प्राणा हि प्रजहन्ति माम्}
{त्वदर्थं हि विशालाक्षि मामयं निशितैः शरैः}


\twolineshloka
{कामः कमलगर्भाभे प्रतिविध्यन्न शाम्यति}
{दष्टमेवमनाक्रन्दे भद्रे काममहाहिना}


\twolineshloka
{सा त्वं पीनायतश्रोणी मामाप्नुहि वरानने}
{त्वदधीना हि मे प्राणाः किन्नरोद्गीतभाषिणि}


\twolineshloka
{चारुसर्वानवद्याङ्गि पद्मेन्दुप्रतिमानने}
{न ह्यहं त्वदृते भीरु शक्ष्यामि खलु जीवितुम्}


\twolineshloka
{कामः कमलपत्राक्षि प्रतिविध्यति मामयम्}
{तस्मात्कुरु विशालाक्षि मय्यनुक्रोशमङ्गने}


\twolineshloka
{भक्तं मामसितापाङ्गि न परित्यक्तुमर्हसि}
{त्वं हि मां प्रीतियोगेन त्रातुमर्हसि भामिनि}


\twolineshloka
{त्वद्दर्शनकृतस्नेहं मनश्चलति मे भृशम्}
{न त्वां दृष्ट्वा पुनश्चान्यां द्रष्टुं कल्याणि रोचते}


\twolineshloka
{प्रसीद वशगोऽहं ते भक्तं मां भज भामिनि}
{दृष्ट्वैव त्वां वरारोहे मन्मथो भृशमङ्गने}


\twolineshloka
{अन्तर्गतं विशालाक्षि विध्यति स्म पतत्त्रिभिः}
{मन्मथाग्निसमुद्भूतं दाहं कमललोचने}


\twolineshloka
{प्रीतिसंयोगयुक्ताभिरद्भिः प्रह्लादयस्व मे}
{पुष्पायुधं दुराधर्षं प्रचण्डशरकार्मुकम्}


\twolineshloka
{त्वद्दर्शनसमुद्भूतं विध्यन्तं दुःसहैः शरैः}
{उपशामय कल्याणि आत्मदानेन भामिनि}


\threelineshloka
{गान्धर्वेण विवाहेन मामुपैहि वराङ्गने}
{विवाहानां हि रम्भोरु गान्धर्वः श्रेष्ठ उच्यते ॥तपत्युवाच}
{}


\twolineshloka
{नाहमीशाऽऽत्मनो राजन्कन्या पितृमती ह्यहम्}
{मयि चेदस्ति ते प्रीतिर्याचस्व पितरं मम}


\twolineshloka
{यथा हि ते मया प्राणाः संभृताश्च नरेश्वर}
{दर्शनादेव भूयस्त्वं तथा प्राणान्ममाहरः}


\twolineshloka
{न चाहमीशा देहस्य तस्मान्नृपतिसत्तम}
{समीपं नोपगच्छामि न स्वतन्त्रा हि योषितः}


\twolineshloka
{का हि सर्वेषु लोकेषु विश्रुताभिजनं नृपम्}
{कन्या नाभिलषेन्नाथं भार्तारं भक्तवत्सलम्}


\twolineshloka
{तस्मादेवं गते काले याचस्व पितरं मम}
{आदित्यं प्रणिपातेन तपसा नियमेन च}


\twolineshloka
{स चेत्कामयते दातुं तव मामरिसूदन}
{भविष्याम्यद्य ते राजन्सततं वशवर्तिनी}


\twolineshloka
{अहं हि तपती नाम सावित्र्यवरजा सुता}
{अस्य लोकप्रदीपस्य सवितुः क्षत्रियर्षभ}


\chapter{अध्यायः १८९}
\twolineshloka
{गन्धर्व उवाच}
{}


\twolineshloka
{एवमुक्त्वा ततस्तूर्णं जगामोर्ध्वमनिन्दिता}
{`तपतीतपतीत्येव विललापातुरो नृपः}


\threelineshloka
{प्रास्स्वलच्चासकृद्राजा पुनरुत्थाय धावति}
{धावमानस्तु तपतीमदृष्ट्वैव महीपतिः}
{'स तु राजा पुनर्भूमौ तत्रैव निपपात ह}


\twolineshloka
{अन्वेषमाणः सबलस्तं राजानं नृपोत्तमम्}
{अमात्यः सानुयात्रश्च तं ददर्श महावने}


\twolineshloka
{क्षितौ निपतितं काले शक्रध्वजमिवोच्छ्रितम्}
{त हि दृष्ट्वा महेष्वासं निरस्तं पतितं भुवि}


\twolineshloka
{बभूव सोऽस्य सचिवः संप्रदीप्त इवाग्निना}
{त्वरया चोपसङ्गम्य स्नेहादागतसंभ्रमः}


\twolineshloka
{तं समुत्थापयामास नृपतिं काममोहितम्}
{भूतलाद्भूमिपालेशं पितेव पतितं सुतम्}


\twolineshloka
{प्रज्ञया वयसा चैव वृद्धः कीर्त्या नयेन च}
{अमात्यस्तं समुत्थाप्य बभूव विगतज्वरः}


\twolineshloka
{उवाच चैनं कल्याण्या वाचा मधुरयोत्थिम्}
{मा भैर्मनुजशार्दूल भद्रमस्तु तवानघ}


\twolineshloka
{क्षुत्पिपासापरिश्रान्तं तर्कयामास वै नृपम्}
{पतितं पातनं सङ्ख्ये शात्रवाणां महीतले}


\twolineshloka
{वारिणा च सुशीतेन शिरस्तस्याभ्यषेचयत्}
{अस्पृशन्मुकुटं राज्ञः पुण्डरीकसुगन्धिना}


\twolineshloka
{ततः प्रत्यागतप्राणस्तद्बलं बलवान्नृपः}
{सर्वं विसर्जयामास तमेकं सचिवं विना}


\twolineshloka
{ततस्तस्याज्ञया राज्ञो विप्रतस्थे महद्बलम्}
{स तु राजा गिरिप्रस्थे तस्मिन्पुनरुपाविशत्}


\twolineshloka
{ततस्तस्मिन् गिरिवरे शुचिर्भूत्वा कृताञ्जलिः}
{आरिराधयिषुः सूर्यं तस्थावूर्ध्वमुखः क्षितौ}


\twolineshloka
{जगाम मनसा चैव वसिष्ठमृषिसत्तमम्}
{पुरोहितममित्रघ्नं तदा संवरणो नृपः}


\twolineshloka
{नक्तन्दिनमथैकत्र स्थिते तस्मिञ्जनाधिपे}
{अथाजगाम विप्रर्षिस्तदा द्वादशमेऽहनि}


\twolineshloka
{स विदित्वैव नृपतिं तपत्या हृतमानसम्}
{दिव्येन विधिना ज्ञात्वा भावितात्मा महानृपिः}


\twolineshloka
{तथा तु नियतात्मानं तं नृपं मुनिसत्तमः}
{आबभाषे स धर्मात्मा तस्यैवार्थचिकीर्षया}


\twolineshloka
{स तस्य मनुजेन्द्रस्य पश्यतो भगवानृषिः}
{ऊर्ध्वमाचक्रमे द्रष्टुं भास्करं भास्करद्युतिः}


\threelineshloka
{`योजनानां तु नियुतं क्षणाद्गत्वा तपोधनः}
{'सहस्रांशुं ततो विप्रः कृताञ्जलिरुपस्थितः}
{वसिष्ठोहमिति प्रीत्या स चात्मानं न्यवेदयत्}


\twolineshloka
{तमुवाच महातेजा विवस्वान्मुनिसत्तमम्}
{महर्षे स्वागतं तेऽस्तु कथयस्व यथेप्सितम्}


\twolineshloka
{यदिच्छसि महाभाग मत्तः प्रवदतां वर}
{तत्ते दद्यामभिप्रेतं यद्यपि स्यात्सुदुर्लभम्}


\twolineshloka
{एवमुक्तः स तेनर्षिर्वसिष्ठः संस्तुवन्गिरा}
{प्रणिपत्य विवस्वन्तं भानुमन्तमथाब्रवीत्}


\threelineshloka
{`योजनानां चतुष्षष्टिं निमेषात्त्रिशतं तथा}
{अश्वैर्गच्छति नित्यं यस्तत्पार्श्वस्थोऽब्रवीदिदम् ॥वसिष्ठ उवाच}
{}


\twolineshloka
{अजाय लोकत्रयपावनायभूतात्मने गोपतये वृषाय}
{सूर्याय सर्गप्रलयालयायनमो महाकारुणिकोत्तमाय}


\twolineshloka
{विवस्वते ज्ञानभृतेऽन्तरात्मनेजगत्प्रदीपाय जगद्धितैषिणे}
{स्वयंभुवे दीप्तसहस्रचक्षुषेसुरोत्तमायामिततेजसे नमः}


\threelineshloka
{नमः सवित्रे जगदेकचक्षुषेजगत्प्रसूतिस्थितिनाशहेतवे}
{त्रयीमयाय त्रिगुणात्मधारिणेविरिञ्चनारायणशङ्करात्मने ॥सूर्य उवाच}
{}


\threelineshloka
{संस्तुतो वरदः सोऽहं वरं वरय सुव्रत}
{स्तुतिस्त्वयोक्ता भक्तानां जप्येयं वग्दोस्म्यहम्' ॥वसिष्ठ उवाच}
{}


\twolineshloka
{यैषा ते तपती नाम सावित्र्यवरजा सुता}
{तां त्वां संवरणस्यार्थे वरयामि विभावसो}


\threelineshloka
{स हि राजा बृहत्कीर्तिर्धर्मार्थविदुदारधीः}
{युक्तः संवरणो भर्ता दुहितुस्ते विहङ्गम ॥गन्धर्व उवाच}
{}


\twolineshloka
{इत्युक्तः स तदा तेन ददानीत्येव निश्चितः}
{प्रत्यभाषत तं विप्रं प्रतीनन्द्य दिवाकरः}


\twolineshloka
{वरः संवरणो राज्ञां त्वमृषीणां वरो मुने}
{तपती योषितां श्रेष्ठा किमन्यत्रापवर्जनात्}


\twolineshloka
{ततः सर्वानवद्याङ्गीं तपतीं तपनः स्वयम्}
{ददौ संवरणस्यार्थे वसिष्ठाय महात्मने}


\twolineshloka
{प्रतिजग्राह तां कन्यां महर्षिस्तपतीं तदा}
{वसिष्ठोऽथ विसृष्टस्तु पुनरेवाजगाम ह}


\twolineshloka
{यत्र विख्यातकीर्तिः स कुरूणामृषभोऽभवत्}
{स राजा मन्मथाविष्टस्तद्गतेनान्तरात्मना}


\twolineshloka
{दृष्ट्वा च देवकन्यां तां तपतीं चारुहासिनीम्}
{वसिष्ठेन सहायान्तीं संहृष्टोऽभ्यधिकं बभौ}


\twolineshloka
{रुरुचे साऽधिकं सुभ्रूरापतन्ती नभस्तलात्}
{सौदामनीव विभ्रष्टा द्योतयन्ती दिशस्त्विषा}


\twolineshloka
{कृच्छ्राद्द्वादशरात्रे तु तस्य राज्ञः समाहिते}
{आजगाम विशुद्धात्मा वसिष्ठो भगवानृषिः}


\twolineshloka
{तपसाऽऽराध्य वरदं देवं गोपतिमीश्वरम्}
{लेभे संवरणो भार्यां वसिष्ठस्यैव तेजसा}


\twolineshloka
{ततस्तस्मिन्गिरिश्रेष्ठे देवगन्धर्वसेविते}
{जग्राह विधिवत्पाणिं तपत्याः स नरर्षभः}


\twolineshloka
{वसिष्ठेनाभ्यनुज्ञातस्तस्मिन्नेव धराधरे}
{सोऽकामयत राजर्षिर्विहर्तुं सह भार्यया}


\twolineshloka
{ततः पुरे च राष्ट्रे च वनेषूपवनेषु च}
{आदिदेश महीपालस्तमेव सचिवं तदा}


\twolineshloka
{नृपतिं त्वभ्यनुज्ञाप्य वसिष्ठोऽथापचक्रमे}
{सोऽथ राजा गिरौ तस्मिन्विजहारामरो यथा}


\twolineshloka
{ततो द्वादशवर्षाणि काननेषु वनेषु च}
{रेमे तस्मिन्गिरौ राजा तयैव सह भार्यया}


\twolineshloka
{तस्य राज्ञः पुरे तस्मिन्समा द्वादश सत्तम}
{न ववर्ष सहस्राक्षो राष्ट्रे चैवास्य भारत}


\twolineshloka
{ततस्तस्यामनावृष्ट्यां प्रवृत्तायामरिन्दम}
{प्रजाः क्षयमुपाजग्मुः सर्वाः सस्थाणुजङ्गमाः}


\twolineshloka
{तस्मिंस्तथाविधे काले वर्तमाने सुदारुणे}
{नावश्यायः पपातोर्व्यां ततः सस्यानि नाऽरुहन्}


\twolineshloka
{ततो विभ्रान्तमनसो जनाः क्षुद्भपीडिताः}
{गृहाणि संपरित्यज्य बभ्रमुः प्रदिशो दिशः}


\twolineshloka
{ततस्तस्मिन्पुरे राष्ट्रे त्यक्तदारपरिग्रहाः}
{परस्परममर्यादाः क्षुधार्ता जघ्निरे जनाः}


\twolineshloka
{तत्क्षुधार्तैर्निरानन्दैः शवभूतैस्तथा नरैः}
{अभवत्प्रेतराजस्य पुरं प्रेतैरिवावृतम्}


\twolineshloka
{ततस्तत्तादृशं दृष्ट्वा स एव भगवानृषिः}
{अभ्याद्रवत धर्मात्मा वसिष्ठो मुनिसत्तमः}


\threelineshloka
{तं च पार्थिवशार्दूलमानयामास तत्पुरम्}
{तपत्या सहितं राजन्वर्षे द्वादशमे गते}
{ततः प्रवृष्टस्तत्रासीद्यथापूर्वं सुरारिहा}


\twolineshloka
{तस्मिन्नृपतिशार्दूले प्रविष्टे नगरं पुनः}
{प्रववर्ष सहस्राक्षः सस्यानि जनयन्प्रभुः}


\twolineshloka
{ततः सराष्ट्रं मुमुदे तत्पुरं परया मुदा}
{तेन पार्थिवमुख्येन भावितं भावितात्मना}


\threelineshloka
{ततो द्वादश वर्षाणि पुनरीजे नराधिपः}
{तपत्या सहितः पत्न्या यथा शच्या मरुत्पतिः ॥गन्धर्व उवाच}
{}


\twolineshloka
{एवमासीन्महाभागा तपती नाम पौर्विकी}
{तव वैवस्वती पार्थ तापत्यस्त्वं यया मतः}


\twolineshloka
{तस्यां संजनयामास कुरुं संवरणो नृपः}
{तपत्यां तपतां श्रेष्ठ तापत्यस्त्वं ततोऽर्जुन}


\chapter{अध्यायः १९०}
\twolineshloka
{वैशंपायन उवाच}
{}


\twolineshloka
{स गन्धर्ववचः श्रुत्वा तत्तदा भरतर्षभ}
{अर्जुनः परया भक्त्या पूर्णचन्द्र इवाबभौ}


\twolineshloka
{उवाच च महेष्वासो गन्धर्वं कुरुसत्तमः}
{जातकौतूहलोऽतीव वसिष्ठस्य तपोबलात्}


\twolineshloka
{वसिष्ठ इति तस्यैतदृषेर्नाम त्वयेरितम्}
{एतदिच्छाम्यहं श्रोतुं यथावत्तद्वदस्व मे}


\threelineshloka
{य एष गन्धर्वपते पूर्वेषां नः पुरोहितः}
{आसीदेतन्ममाचक्ष्व क एष भगवानृषिः ॥गन्धर्व उवाच}
{}


\twolineshloka
{ब्रह्मणो मानसः पुत्रो वसिष्ठोऽरुन्धतीपतिः}
{तपसा निर्जितौ शश्वदजेयावमरैरपि}


\twolineshloka
{कामक्रोधावुभौ यस्य चरणौ समुवाहतुः}
{इन्द्रियाणां वशकरो वशिष्ठ इति चोच्यते}


\twolineshloka
{`यथा कामश्च क्रोधश्च निर्जितावजितौ नरैः}
{जितारयो जिता लोकाः पन्थानश्च जिता दिशः ॥'}


\twolineshloka
{यस्तु नोच्छेदनं चक्रे कुशिकानामुदारधीः}
{विश्वामित्रापराधेन धारयन्मन्युमुत्तमम्}


\twolineshloka
{पुत्रव्यसनसंतप्तः शक्तिमानप्यशक्तवत्}
{विश्वामित्रविनाशाय न चक्रे कर्म दारुणम्}


\twolineshloka
{मृतांश्च पुनराहर्तुं शक्तः पुत्रान्यमक्षयात्}
{कृतान्तं नातिचक्राम वेलामिव महोदधिः}


\twolineshloka
{यं प्राप्य विजितात्मानं महात्मानं नराधिपाः}
{इक्ष्वाकवो महीपाला लेभिरे पृथिवीमिमाम्}


\twolineshloka
{पुरोहितमिमं प्राप्य वसिष्ठमृषिसत्तमम्}
{ईजिरे क्रतुभिश्चैव नृपास्ते कुरुनन्दन}


\twolineshloka
{स हि तान्याजयामास सर्वान्नृपतिसत्तमान्}
{ब्रह्मर्षिः पाण्डवश्रेष्ठ बृहस्पतिरिवामरान्}


\twolineshloka
{तस्माद्धर्मप्रधानात्मा वेदधर्मविदीप्सितः}
{ब्राह्मणो गुणवान्कश्चित्पुरोधाः प्रतिदृश्यताम्}


\twolineshloka
{क्षत्रियेणाभिजातेन पृथिवीं जेतुमिच्छता}
{पूर्वं पुरोहितः कार्यः पार्थ राज्याभिवृद्धये}


\threelineshloka
{महीं जिगीषता राज्ञा ब्रह्म कार्यं पुरःस्कृतम्}
{तस्मात्पुरोहितः कश्चिद्गुणवान्विजितेन्द्रियः}
{विद्वान्भवतु वो विप्रो धर्मकामार्थतत्त्ववित्}


\chapter{अध्यायः १९१}
\twolineshloka
{अर्जुन उवाच}
{}


\threelineshloka
{किंनिमित्तमभूद्वैरं विश्वामित्रवसिष्ठयोः}
{वसतोराश्रमे दिव्ये शंस नः सर्वमेव तत् ॥गन्धर्व उवाच}
{}


\twolineshloka
{इदं वासिष्ठमाख्यानं पुराणं परिचक्षते}
{पार्थ सर्वेषु लोकेषु यथावत्तन्निबोध मे}


\twolineshloka
{कान्यकुब्जे महानासीत्पार्थिवो भरतर्षभ}
{गाधीति विश्रुतो लोके कुशिकस्यात्मसंभवः}


\twolineshloka
{तस्य धर्मात्मनः पुत्रः समृद्धबलवाहनः}
{विश्वामित्र इति ख्यतो बभूव रिपुमर्दनः}


\twolineshloka
{स चचार सहामात्यो मृगयां गहने वने}
{मृगान्विध्यन्वराहांश्च रम्येषु मरुधन्वसु}


\twolineshloka
{व्यायामकर्शितः सोऽथ मृगलिप्सुः पिपासितः}
{आजगाम नरश्रेष्ठ वसिष्ठस्याश्रमं प्रति}


\twolineshloka
{तमागतमभिप्रेक्ष्य वसिष्ठः श्रेष्ठभागृषिः}
{विश्वामित्रं नरश्रेष्ठं प्रतिजग्राह पूजया}


\twolineshloka
{पाद्यार्घ्याचमनीचैस्तं स्वागतेन च भारत}
{तथैव परिजग्राह वन्येन हविषा तदा}


\twolineshloka
{तस्याथ कामधुग्धेनुर्वसिष्ठस्य महात्मनः}
{उक्ता कामान्प्रयच्छेति सा कामान्दुदुहे ततः}


\twolineshloka
{`बाष्पाढ्यस्योदनस्यैव राशयः पर्वतोपमाः}
{निष्ठानानि च सूपांश्च दधिकुल्यास्तथैव च}


\twolineshloka
{कूपांश्च घृतसंपूर्णान्गौड्यान्नानि सहस्रशः}
{इक्षून्मधूनि लाजांश्च मैरेयांश्च वरासवान् ॥'}


\twolineshloka
{ग्राम्यारण्याश्चौषधीश्च दुदुहे पय एव च}
{षड्रसं चामृतनिभं रसायनमनुत्तमम्}


\twolineshloka
{भोजनीयानि पेयानि भक्ष्याणि विविधानि च}
{लेह्यान्यमृतकल्पानि चोष्याणि च तथार्जुना}


\twolineshloka
{रत्नानि च महार्हाणि वासांसि विविधानि च}
{तैः कामैः सर्वसंपूर्णैः पूजितश्च महिपतिः}


\twolineshloka
{सामात्यः सबलश्चैव तुतोष स भृशं तदा}
{षडुन्नतां सुपार्श्वोरुं पृथुपञ्चसमावृताम्}


\twolineshloka
{मण्डूकनेत्रां स्वाकारां पीनोधसमनिन्दिताम्}
{सुवालघिं शङ्कुकर्णां चारुशृङ्गां मनोरमाम्}


\twolineshloka
{पुष्टायतशिरोग्रीवां विस्मितः सोऽभिवीक्ष्यताम्}
{अभिनन्द्य स तां राजा नन्दिनीं गाधिनन्दनः}


\twolineshloka
{अब्रवीच्च भृशं तुष्टः स राजा तमृषिं तदा}
{अर्बुदेन गवां ब्रह्मन्मम राज्येन वा पुनः}


\threelineshloka
{नन्दिनीं संप्रयच्छस्व भुङ्क्ष्व राज्यं महामुने}
{वसिष्ठ उवाच}
{देवतातिथिपित्रर्थं याज्यार्थं च पयस्विनी}


\fourlineindentedshloka
{अदेया नन्दिनीयं वै राज्येनापि तवानघ}
{विश्वामित्र उवाच}
{`रत्नं हि भगवन्नेतद्रत्नहारी च पार्थिवः}
{'क्षत्रियोऽहं भवान्विप्रस्तपःस्वाध्यायसाधनः}


\twolineshloka
{ब्राह्णेषु कुतो वीर्यं प्रशान्तेषु धृतात्मसु}
{अर्बुदेन गवां यस्त्वं न ददासि ममेप्सितम्}


\threelineshloka
{स्वधर्मं न प्रहास्यामि नेष्यामि च बलेन गाम्}
{वसिष्ठ उवाच}
{बलस्थश्चासि राजा च बाहुवीर्यश्च क्षत्रियः}


\threelineshloka
{यथेच्छसि तथा क्षिप्रं कुरु मा त्वं विचारय}
{गन्धर्व उवाच}
{एवमुक्तस्तथा पार्थ विश्वामित्रो बलादिव}


\threelineshloka
{हंसचन्द्रप्रतीकाशां नन्दिनीं तां जहार गाम्}
{`सा तदा ह्रियमाणा च विश्वामित्रबलैर्बलात्}
{'कशादण्डप्रणुदिता काल्यमाना इतस्ततः}


\twolineshloka
{हंभायमाना कल्याणी वसिष्ठस्याथ नन्दिनी}
{आगम्याभिमुखी पार्थ तस्थौ भगवदुन्मुखी}


\threelineshloka
{भृशं च ताड्यमाना वै न जगामाश्रमात्ततः}
{वसिष्ठ उवाच}
{शृणोमि ते रवं भद्रे विनदन्त्याः पुनः पुनः}


\fourlineindentedshloka
{ह्रियसे त्वं बलाद्भद्रे विश्वामित्रेण नन्दिनि}
{किं कर्तव्यं मया तत्र क्षमावान्ब्राह्मणो ह्यहम्}
{गन्धर्व उवाच}
{}


\threelineshloka
{सा भयान्नन्दिनी तेषां बलानां भरतर्षभ}
{विश्वामित्रभयोद्विग्ना वसिष्ठं समुपागमत् ॥गौरुवाच}
{}


\threelineshloka
{कशाग्रदण्डाभिहतां क्रोशन्तीं मामनाथवत्}
{विश्वामित्रबलैर्घोरैर्भगवन् किमुपेक्षसे ॥गन्धर्व उवाच}
{}


\threelineshloka
{एवं तस्यां तदा पार्थ धर्षितायां महामुनिः}
{न चुक्षुभे तदा धैर्यान्न चचाल धृतव्रतः ॥वसिष्ठ उवाच}
{}


\threelineshloka
{क्षत्रियाणां बलं तेजो ब्राह्मणानां क्षमा बलम्}
{क्षमा मां भजते यस्माद्गम्यतां यदि रोचते ॥नन्दिन्युवाच}
{}


\threelineshloka
{किं नु त्यक्ताऽस्मि भगवन्यदेवं त्वं प्रभाषसे}
{अत्यक्ताऽहं त्वया ब्रह्मन्नेतुं शक्या न वै बलात् ॥वसिष्ठ उवाच}
{}


\threelineshloka
{न त्वां त्यजामि कल्याणि स्थीयतां यदि शक्यते}
{दृढेन दाम्ना बद्ध्वैष वत्सस्ते हियते बलात् ॥गन्धर्व उवाच}
{}


\twolineshloka
{स्थीयतामिति तच्छ्रुत्वा वसिष्ठस्य पयस्विनी}
{ऊर्ध्वाञ्चितशिरोग्रीवा प्रबभौ रौद्रदर्शना}


\twolineshloka
{क्रोधरक्तेक्षणा सा गौर्हंभारवघनस्वना}
{विश्वामित्रस्य तत्सैन्यं व्यद्रावयत सर्वशः}


\twolineshloka
{कशाग्रदण्डाभिहता काल्यमाना ततस्ततः}
{क्रोधरक्तेक्षणा क्रोधं भूय एव समाददे}


\twolineshloka
{आदित्य इव मध्याह्ने क्रोधदीप्तवपुर्बभौ}
{अङ्गारवर्षं मुञ्चन्ती मुहुर्वालधितो महत्}


\twolineshloka
{असृजत्पह्लवान्पुच्छात्प्रस्रवाद्द्राविडाञ्छकान्}
{योनिदेशाच्च यवानाञ्शकृतः शबरान्बहून्}


\twolineshloka
{मूत्रतश्चासृजत्कांश्चिच्छबरांश्चैव पार्श्वतः}
{पौण्ड्रान्किरातान्यवनान्सिंहलान्बर्बरान्खसान्}


\twolineshloka
{चिबुकांश्च पुलिन्दांश्च चीनान्हूणान्सकेरलान्}
{ससर्ज फेनतः सा गौर्म्लेच्छान्बहुविधानपि}


\twolineshloka
{तैर्विसृष्टैर्महासैन्यैर्नानाम्लेच्छगणैस्तदा}
{नानावरणसंछन्नैर्नानायुधधरैस्तथा}


\twolineshloka
{अवाकीर्यत संरब्धैर्विश्वामित्रस्य पश्यतः}
{एकैकश्च तदा योधः पञ्चभिः सप्तभिर्वृतः}


\twolineshloka
{अस्त्रवर्षेण महता वध्यमानं बलं तदा}
{प्रभग्नं सर्वतस्त्रस्तं विश्वामित्रस्य पश्यतः}


\twolineshloka
{`तस्य तच्चतुरङ्गं वै बलं परमदुःसहम्}
{प्रभग्नं सर्वतो घोरं पयस्विन्या विनिर्जितम् ॥'}


\twolineshloka
{न च प्राणैर्वियुज्यन्ते केचित्तत्रास्य सैनिकाः}
{विश्वामित्रस्य संक्रुद्धैर्वासिष्ठैर्भरतर्षभ}


\twolineshloka
{सा गौस्तत्सकलं सैन्यं कालयामास दूरतः}
{विश्वामित्रस्य तत्सैन्यं काल्यमानं त्रियोजनम्}


\twolineshloka
{क्रोशमानं भयोद्विग्नं त्रातारं नाध्यगच्छत}
{`विश्वामित्रस्ततो दृष्ट्वा क्रोधाविष्टः स रोदसी}


\twolineshloka
{ववर्ष शरवर्षाणि वसिष्ठे मुनिसत्तमे}
{घोररूपांश्च नाराचान्क्षुरान्भल्लान्महामुनिः}


\twolineshloka
{विश्वामित्रप्रयुक्तांस्तान्वैणवेन व्यमोचयत्}
{वसिष्ठस्य तदा दृष्ट्वा कर्मकौशलमाहवे}


\twolineshloka
{विश्वामित्रोऽपि कोपेन भूयः शत्रुनिपातनः}
{दिव्यास्त्रवर्षं तस्मै स प्राहिणोन्मुनये रुषा}


\twolineshloka
{आग्नेयं वारुणं चैन्द्रं याम्यं वायव्यमेव च}
{विससर्ज महाभागे वसिष्ठे ब्रह्मणः सुते}


\twolineshloka
{अस्त्राणि सर्वतो ज्वालां विसृजन्ति प्रपेदिरे}
{युगान्तसमये घोराः पतङ्गस्येव रश्मयः}


\twolineshloka
{वसिष्ठोऽपि महातेजा ब्रह्मशक्तिप्रयुक्तया}
{यष्ट्या निवारयामास सर्वाण्यस्त्राणि स स्मयन्}


\twolineshloka
{ततस्ते भस्मसाद्भूताः पतन्ति स्म महीतले}
{अपोह्य दिव्यान्यस्त्राणि वसिष्ठो वाक्यमब्रवीत्}


\threelineshloka
{निर्जितोऽसि महाराज दुरात्मन्गाधिनन्दन}
{यदि तेऽस्ति परं शौर्यं तद्दर्शय मयि स्थिते ॥गन्धर्व उवाच}
{}


\twolineshloka
{विश्वामित्रस्तथा चोक्तो वसिष्ठेन नराधिपः}
{नोवाच किंचिद्व्रीडाढ्यो विद्रावितमहाबलः'}


\threelineshloka
{दृष्ट्वा तन्महदाश्चर्यं ब्रह्मतेजोभवं तदा}
{विश्वामित्रः क्षत्रभावान्निर्विण्णो वाक्यमब्रवीत्}
{धिग्बलं क्षत्रियबलं ब्रह्मतेजोबलं बलम्}


\threelineshloka
{बलाबले विनिश्चित्य तप एव परं बलम्}
{गन्धर्व उवाच}
{स राज्यं स्फीतमुत्सृज्य तां च दीप्तां नृपश्रियम्}


\twolineshloka
{भोगांश्च पृष्ठतः कृत्वा तपस्येव मनो दधे}
{स गत्वा तपसा सिद्धिं लोकान्विष्टभ्य तेजसा}


\twolineshloka
{तताप सर्वान्दीप्तौजा ब्राह्मणत्वमवाप्तवान्}
{अपिबच्च ततः सोममिन्द्रेण सह कौशिकः}


% Check verse!
`एवंवीर्यस्तु राजर्षिर्विप्रर्षिः संबभूव ह'
\chapter{अध्यायः १९२}
\twolineshloka
{`अर्जुन उवाच}
{}


\twolineshloka
{ऋष्योस्तु यत्कृते वैरं विश्वामित्रवसिष्ठयोः}
{बभूव गन्धर्वपते शंस तत्सर्वमेव मे}


\twolineshloka
{माहात्म्यं च वसिष्ठस्य ब्राह्मण्यं ब्रह्मतेजसः}
{विश्वामित्रस्य च तथा क्षत्रस्य च महात्मनः}


\twolineshloka
{न शृण्वानस्त्वहं तृप्तिमुपगच्छामि खेचर}
{आख्याहि गन्धर्वपते शंस तत्सर्वमेव मे}


\twolineshloka
{माहात्म्यं च वसिष्ठस्य विश्वामित्रस्य भाषते ॥गन्धर्व उवाच}
{}


\twolineshloka
{इदं वासिष्ठमाख्यानं पुराणं पुण्यमुत्तमम्}
{पार्थ सर्वेषु लोकेषु विश्रुतं तन्निबोध मे ॥'}


\twolineshloka
{कल्माषपाद इत्येवं लोके राजा बभूव ह}
{इक्ष्वाकुवंशजः पार्थ तेजसाऽसदृशो भुवि}


\twolineshloka
{स कदाचिद्वनं राजा मृगयां निर्ययौ पुरात्}
{मृगान्विध्यन्वराहांश्च चचार रिपुमर्दनः}


\twolineshloka
{तस्मिन्वने महाघोरे खङ्गांश्च बहुशोऽहनत्}
{हत्वा च सुचिरं श्रान्तो राजा निववृते ततः}


\twolineshloka
{अकामयत्तं याज्यार्थे विश्वामित्रः प्रतापवान्}
{स तु राजा महात्मानं वासिष्ठमृषिसत्तमम्}


\twolineshloka
{तृषार्तश्च क्षुधार्तश्च एकायनगतः पथि}
{अपश्यदजितः सङ्ख्ये मुनिं प्रतिमुखागतम्}


\twolineshloka
{शक्तिं नाम महाभागं वसिष्ठकुलवर्धनम्}
{ज्येष्ठं पुत्रं पुत्रशताद्वसिष्ठस्य महात्मनः}


\twolineshloka
{अपगच्छ पथोऽस्माकमित्येवं पार्थिवोऽब्रवीत्}
{तथा ऋषिरुवाचैनं सान्त्वयञ्श्लक्ष्णया गिरा}


\twolineshloka
{मम पन्था महाराज धर्म एष सनातनः}
{`वृद्धभीरुनृपस्नातस्त्रीरोगिवरचक्रिणाम्}


\twolineshloka
{पन्था देयो नृपैस्तेषामन्यैस्तैस्तस्य भूपतेः}
{'राज्ञा सर्वेषा धर्मेषु देयः पन्था द्विजातये}


\twolineshloka
{एवं परस्परं तौ तु पथोऽर्थं वाक्यमूचतुः}
{अपसर्पापसर्पेति वागुत्तरमकुर्वताम्}


\twolineshloka
{ऋषिस्तु नापचक्राम तस्मिन्धर्मपथे स्थितः}
{`अपि राजा मुनेर्मार्गात्क्रोधान्नापजगाम ह ॥'}


\twolineshloka
{अमुञ्चन्तं तु पन्थानं तमृषिं नृपसत्तमः}
{जगाम कशया मोहात्तदा राक्षसन्मुनिम्}


\twolineshloka
{कशाप्रहाराभिहतस्ततः स मुनिसत्तमः}
{तं शशाप नृपश्रेष्ठं वासिष्ठः क्रोधमूर्च्छितः}


\twolineshloka
{हंसि राक्षसवद्यस्माद्राजापशद तापसम्}
{तस्मात्त्वमद्यप्रभृति पुरुषादो भविष्यसि}


\twolineshloka
{मनुष्यपिशिते सक्तश्चरिष्यसि महीमिमाम्}
{गच्छ राजाधमेत्युक्तः शक्तिना वीर्यशक्तिना}


\twolineshloka
{ततो याज्यनिमित्तं तु विश्वामित्रवसिष्ठयोः}
{वैरमासीत्तदा तं तु विश्वामित्रोऽन्वपद्यत}


\twolineshloka
{तयोर्विवदतोरेवं समीपमुपचक्रमे}
{ऋषिरुग्रतपाः पार्थ विश्वामित्रः प्रतापवान्}


\twolineshloka
{ततः स बुबुधे पश्चात्तमृषिं नृपसत्तमः}
{ऋषेः पुत्रं वसिष्ठस्य वसिष्ठमिव तेजसा}


\twolineshloka
{अन्तर्धाय ततोऽत्मानं विश्वामित्रोऽपि भारत}
{तावुभावतिचक्राम चिकीर्षन्नात्मनः प्रियम्}


\twolineshloka
{स तु शप्तस्तदा तेन शक्तिना वै नृपोत्तमः}
{जगाम शरणं शक्तिं प्रसादयितुमर्हयन्}


\twolineshloka
{तस्य भावं विदित्वा स नृपतेः कुरुसत्तम}
{विश्वामित्रस्ततो रक्ष आदिदेश नृपं प्रति}


\twolineshloka
{शापात्तस्य तु विप्रर्षेर्विश्वामित्रस्य चाज्ञया}
{राक्षसः किङ्करो नाम विवेश नृपतिं तदा}


\twolineshloka
{रक्षसा तं गृहीतं तु विदित्वा मुनिसत्तमः}
{विश्वामित्रोऽप्यपाक्रामत्तस्माद्देशादरिन्दम}


\twolineshloka
{ततः स नृपतिर्विद्वान्रक्षन्नात्मानमात्मना}
{बलवत्पीड्यमानोऽपि रक्षसान्तर्गतेन ह}


\twolineshloka
{ददर्शाथ द्विजः कश्चिद्राजानं प्रस्थितं वनम्}
{अयाचत क्षुधापन्नः समांसं भोजनं तदा}


\twolineshloka
{तमुवाचाथ राजर्षिर्द्विजं मित्रसहस्तदा}
{आस्स्व ब्रह्मंस्त्वमत्रैव मुहूर्तं प्रतिपालयन्}


\twolineshloka
{निवृत्तः प्रतिदास्यामि भोजनं ते यथेप्सितम्}
{इत्युक्त्वा प्रययौ राजा तस्थौ च द्विजसत्तमः}


\twolineshloka
{ततो राजा परिक्रम्य यथाकामं यथासुखम्}
{निवृत्तोऽन्तःपुरं पार्थ प्रविवेश महामनाः}


\twolineshloka
{`अन्तर्गतस्तदा राजा श्रुत्वा ब्राह्मणभाषितम्}
{सोऽन्तःपुरं प्रविश्याथ न सस्मार नराधिपः ॥'}


\twolineshloka
{ततोऽर्धरात्र उत्थाय सूदमानाय्य सत्वरम्}
{उवाच राजा संस्मृत्य ब्राह्मणस्य प्रतिश्रुतम्}


\threelineshloka
{गच्छामुष्मिन्वनोद्देशे ब्राह्मणो मां प्रतीक्षते}
{अन्नार्थी तं त्वमन्नेन समांसेनोपपादय ॥गन्धर्व उवाच}
{}


\twolineshloka
{एवमुक्तस्ततः सूदः सोऽनासाद्यामिषं क्वचित्}
{निवेदयामास तदा तस्मै राज्ञे व्यथान्वितः}


\twolineshloka
{राजा तु रक्षसाविष्टः सूदमाह गतव्यथः}
{अप्येनं नरमांसेन भोजयेति पुनः पुनः}


\twolineshloka
{तथेत्युक्त्वा ततः सूदः संस्थानं वध्यघातिनाम्}
{गत्वाऽऽजहार त्वरितो नरमांसमपेतभीः}


\twolineshloka
{एतत्संस्कृत्य विधिवदन्नोपहितमाशु वै}
{तस्मै प्रादाद्ब्राह्मणाय क्षुधिताय तपस्विने}


\threelineshloka
{स सिद्धचक्षुषा दृष्ट्वा तदन्नं द्विजसत्तमः}
{अभोज्यमिदमित्याह क्रोधपर्याकुलेक्षणः ॥ब्राह्मण उवाच}
{}


% Check verse!
यस्मादभोज्यमन्नं मे ददाति स नृपाधमः
\threelineshloka
{सक्तो मानुषमांसेषु यथोक्तः शक्तिना पुरा}
{उद्वेजनीयो भूतानां चरिष्यति महीमिमाम् ॥गन्धर्व उवाच}
{}


\twolineshloka
{द्विरनुव्याहृते राज्ञः स शापो बलवानभूत्}
{रक्षोबलसमाविष्टो विसंज्ञश्चाभवन्नृपः}


\twolineshloka
{ततः स नृपतिश्रेष्ठो रक्षसापहृतेन्द्रियः}
{उवाच शख्तिं तं दृष्ट्वा न चिरादिव भारत}


\twolineshloka
{यस्मादसदृशः शापः प्रयुक्तोऽयं मयि त्वया}
{तस्मात्त्वत्तः प्रवर्तिष्ये खादितुं पुरुषानहम्}


\twolineshloka
{एवमुक्त्वा ततः सद्यस्तं प्राणैर्विप्रयोज्य च}
{शक्तिं तं भक्षयामास व्याघ्रः पशुमिवेप्सितम्}


\twolineshloka
{शक्तिनं तु मृतं दृष्ट्वा विश्वामित्रः पुनःपुनः}
{वसिष्ठस्यैव पुत्रेषु तद्रक्षः संदिदेश ह}


\twolineshloka
{स ताञ्शक्त्यवरान्पुत्रान्वसिष्ठस्य महात्मनः}
{भक्षयामास संक्रुद्धः सिंहः क्षुद्रमृगानिव}


\twolineshloka
{वसिष्ठो घातिताञ्श्रुत्वा विश्वामित्रेण तान्सुतान्}
{धारयामास तं शोकं महाद्रिरिव मेदिनीम्}


\twolineshloka
{चक्रे चात्मविनाशाय बुद्धिं स मुनिसत्तमः}
{न त्वेव कौशिकोच्छेदं मेने मतिमतां वरः}


\twolineshloka
{स मेरुकूटादात्मानं मुमोच भगवानृषिः}
{गिरेस्तस्य शिलायां तु तूलराशाविवापतत्}


\twolineshloka
{न ममार च पातेन स यदा तेन पाण्डव}
{तदाग्निमिद्धं भगवान्संविवेश महावने}


\twolineshloka
{तं तदा सुसमिद्धोऽपि न ददाह हुताशनः}
{दीप्यमानोऽप्यमित्रघ्न शीतोऽग्निरभवत्ततः}


\twolineshloka
{स समुद्रमभिप्रेक्ष्य शोकाविष्टो महामुनिः}
{बद्ध्वा कण्ठे शिलां गुर्वीं निपपात तदाम्भसि}


\threelineshloka
{स समुद्रोर्मिवेगेन स्थले न्यस्तो महामुनिः}
{न ममार यदा विप्रः कथंचित्संशितव्रतः}
{जगाम स ततः खिन्नः पुनरेवाश्रमं प्रति}


\chapter{अध्यायः १९३}
\twolineshloka
{गन्धर्व उवाच}
{}


\twolineshloka
{ततो दृष्ट्वाश्रमपदं रहितं तैः सुतैर्मुनिः}
{निर्जगाम सुदुःखार्तः पुनरप्याश्रमात्ततः}


\twolineshloka
{सोऽपश्यत्सरितं पूर्णां प्रावृट्काले नवाम्भसा}
{वृक्षान्बहुविधान्पार्थ हरन्तीं तीरजान्बहून्}


\twolineshloka
{अथ चिन्तां समापेदे पुनः कौरवनन्दन}
{अम्भस्यस्या निमज्जेयमिति दुःखसमन्वितः}


\twolineshloka
{ततः पाशैस्तदात्मानं गाढं बद्ध्वा महामुनिः}
{तस्या जले महानद्या निममज्ज सुदुःखितः}


\twolineshloka
{अथ च्छित्त्वा नदी पाशांस्तस्यारिबलसूदन}
{स्थलस्थं तमृषिं कृत्वा विपाशं समवासृजत्}


\twolineshloka
{उत्ततार ततः पाशैर्विमुक्तः स महानृषिः}
{विपाशेति च नामास्या नद्याश्चक्रे महानृषिः}


\threelineshloka
{`सा विपाशेति विख्याता नदी लोकेषु भारत}
{ऋषेस्तस्य नरव्याघ्र वचनात्सत्यवादिनः}
{उत्तीर्य च तदा राजन्दुःखितो भगवानृषिः ॥'}


\twolineshloka
{शोके बुद्धिं तदा चक्रे न चैकत्र व्यतिष्ठत}
{सोऽगच्छत्पर्वतांश्चैव सरितश्च सरांसि च}


\twolineshloka
{दृष्ट्वा स पुनरेवर्षिर्नदीं हैमवतीं तदा}
{चण्डग्राहवतीं भीमां तस्याः स्रोतस्यपातयत्}


\twolineshloka
{सा तमग्निसं विप्रमनुचिन्त्य सरिद्वरा}
{शतधा विद्रुता तस्माच्छतद्रुरिति विश्रुता}


\twolineshloka
{ततः स्थलगतं दृष्ट्वा तत्राप्यात्मानमात्मना}
{मर्तुं न शक्यमित्युक्त्वा पुवरेवाश्रमं ययौ}


\twolineshloka
{स गत्वा विविधाञ्शैलान्देशान्बहुविधांस्तथा}
{अदृशन्त्याख्यया वध्वाथाश्रमेनुसृतोऽभवत्}


\twolineshloka
{अथ शुश्राव संगत्या वेदाध्ययननिःस्वनम्}
{पृष्ठतः परिपूर्णार्थं षड्मिरङ्गैरलङ्कृतम्}


\twolineshloka
{अनुव्रजति कोन्वेष मामित्येवाथ सोऽब्रवीत्}
{अदृश्यन्त्येवमुक्ता वै तं स्नुषा प्रत्यभाषत}


\threelineshloka
{शक्तोभार्या महाभाग तपोयुक्ता तपस्विनम्}
{अहमेकाकिनी चापि त्वया गच्छामि नापरः ॥वसिष्ठ उवाच}
{}


\threelineshloka
{पुत्रि कस्यैष साङ्गस्य वेदस्याध्ययनस्वनः}
{पुरा साङ्गस्य वेदस्य शक्तेरिव मया श्रुतः ॥अदृश्यन्त्युवाच}
{}


\threelineshloka
{अयं कुक्षौ समुत्पन्नः शक्तेर्गर्भः सुतस्य ते}
{समा द्वादश तस्येह वेदानभ्यस्यतो मुने ॥गन्धर्व उवाच}
{}


\twolineshloka
{एवमुक्तस्तया हृष्टो वसिष्ठः श्रेष्ठभागृषिः}
{अस्ति सन्तानमित्युक्त्वा मृत्योः पार्थ न्यवर्तत}


\twolineshloka
{ततः प्रतिनिवृत्तः स तया वध्वा सहानघ}
{कल्माषपादमासीनं ददर्श विजने वने}


\twolineshloka
{स तु दृष्ट्वैव तं राजा क्रुद्ध उत्थाय भारत}
{आविष्टो रक्षसोग्रेण इयेषात्तुं तदा मुनिम्}


\twolineshloka
{अदृश्यन्ती तु तं दृष्ट्वा क्रूरकर्माणमग्रतः}
{भयसंविग्नया वाचा वसिष्ठमिदमब्रवीत्}


\twolineshloka
{असौ मृत्युरिवोग्रेण दण्डेन भगवन्नितः}
{प्रगृहीतेन काष्ठेन राक्षसोऽभ्येति दारुणः}


\twolineshloka
{तं निवारयितुं शक्तो नान्योऽस्ति भुवि कश्चन}
{स्वदृतेऽद्य महाभाग सर्ववेदविदां वर}


\threelineshloka
{पाहि मां भगवन्पापादस्माद्दारुणदर्शनात्}
{राक्षसोऽयमिहात्तुं वै नूनमावां समीहते ॥वसिष्ठ उवाच}
{}


\twolineshloka
{माभैः पुत्रि न भेतव्यं राक्षसात्तु कथंचन}
{नैतद्रक्षो भयं यस्मात्पश्यसि त्वमुपस्थितम्}


\threelineshloka
{राजा कल्माषपादोऽयं वीर्यवान्प्रथितो भुवि}
{स एषोऽस्मिन्वनोद्देशे निवसत्यतिभीषणः ॥गन्धर्व उवाच}
{}


\twolineshloka
{तमापतन्तं संप्रेक्ष्य वसिष्ठो भगवानृषिः}
{वारयामास तेजस्वी हुङ्कारेणैव भारत}


\twolineshloka
{मन्त्रपूतेन च पुनः स तमभ्युक्ष्य वारिणा}
{मोक्षयामास वै शापात्तस्माद्योगान्नराधिपम्}


\twolineshloka
{स हि द्वादश वर्षाणि वासिष्ठस्यैव तेजसा}
{ग्रस्त आसीद्ग्रहेणेव पर्वकाले दिवाकरः}


\twolineshloka
{रक्षसा विप्रमुक्तोऽथ स नृपस्तद्वनं महत्}
{तेजसा रञ्जयामास न्ध्याभ्रमिव भास्करः}


\twolineshloka
{प्रतिलभ्य ततः संज्ञामभिवाद्य कृताञ्जलिः}
{उवाच नृपतिः काले वसिष्ठमृषिसत्तमम्}


\threelineshloka
{सौदासोऽहं महाभाग याज्यस्ते मुनिसत्तम}
{अस्मिन्काले यदिष्टं ते ब्रूहि किं करवाणि ते ॥वसिष्ठ उवाच}
{}


\threelineshloka
{वृत्तमेतद्यथाकालं गच्छ राज्यं प्रशाधि वै}
{ब्राह्मणं तु मनुष्येन्द्र माऽवमंस्थाः कदाचन ॥राजोवाच}
{}


\twolineshloka
{नावमंस्ये महाभाग कदाचिद्ब्राह्मणर्षभान्}
{त्वन्निदेशे स्थितः सम्यक् पूजयिष्याम्यहं द्विजान्}


\twolineshloka
{इक्ष्वाकूणां च येनाहमनृणः स्यां द्विजोत्तम}
{तत्त्वत्तः प्राप्तुमिच्छामि सर्ववेदविदां वर}


\threelineshloka
{अपत्यायेप्सिताय त्वं महिषीं गन्तुमर्हसि}
{शीलरूपगुणोपेतामिक्ष्वाकुकुलवृद्धये ॥गन्धर्व उवाच}
{}


\twolineshloka
{ददानीत्येव तं तत्र राजानं प्रत्युवाच ह}
{वसिष्ठः परमेष्वासं सत्यसन्धो द्विजोत्तमः}


\twolineshloka
{ततः प्रतिययौ काले वसिष्ठः सह तेन वै}
{ख्यातां पुरीमिमां लोकेष्वयोध्यां मनुजेश्वर}


\twolineshloka
{तं प्रजाः प्रतिमोदन्त्यः सर्वाः प्रत्युद्गतास्तदा}
{विपाप्मानं महात्मानं दिवौकस इवेश्वरम्}


\twolineshloka
{सुचिराय मनुष्येन्द्रो नगरीं पुण्यलक्षणाम्}
{विवेश सहितस्तेन वसिष्ठेन महर्षिणा}


\twolineshloka
{ददृशुस्तं महीपालमयोध्यावासिनो जनाः}
{पुरोहितेन सहितं दिवाकरमिवोदितम्}


\twolineshloka
{स च तां पूरयामास लक्ष्म्या लक्ष्मीवतां वरः}
{अयोध्यां व्योम शीतांशुः शरत्काल इवोदितः}


\twolineshloka
{संसक्तिमृष्टपन्थानं पताकाध्वजशोभितम्}
{मनः प्रह्लादयामास तस्य तत्पुरमुत्तमम्}


\twolineshloka
{तुष्टपुष्टजनाकीर्णा सा पुरी कुरुनन्दन}
{अशोभत तदा तेन शक्रेणेवामरावती}


\twolineshloka
{ततः प्रविष्टे राजर्षौ तस्मिंस्तत्पुरमुत्तमम्}
{राज्ञस्तस्याज्ञया देवी वसिष्ठमुपचक्रमे}


\twolineshloka
{ऋतावथ महर्षिस्तु संबभूव तया सह}
{देव्या दिव्येन विधिना वसिष्ठः श्रेष्ठभागृषिः}


\twolineshloka
{ततस्तस्यां समुत्पन्ने गर्भे स मुनिसत्तमः}
{राज्ञाभिवादितस्तेन जगाम मुनिराश्रमम्}


\twolineshloka
{दीर्घकालेन सा गर्भं सुषुवे न तु तं यदा}
{तदा देव्यश्मना कुक्षिं निर्बिभेद यशस्विनी}


\twolineshloka
{ततो द्वादशमे वर्षे स जज्ञे पुरषर्षभः}
{अश्मको नाम राजर्षिः पौदन्यं यो न्यवेशयत्}


\chapter{अध्यायः १९४}
\twolineshloka
{गन्धर्व उवाच}
{}


\twolineshloka
{आश्रमस्था ततः पुत्रमदृश्यन्ती व्यजायत}
{शक्तेः कुलकरं राजन् द्वितीयमिव शक्तिनम्}


\twolineshloka
{जातकर्मादिकास्तस्य क्रियाः स मुनिसत्तमः}
{पौत्रस्य भरतश्रेष्ठ चकार भगवान्स्वयम्}


\twolineshloka
{परासुः स यतस्तेन वसिष्ठः स्थापितो मुनिः}
{गर्भस्थेन ततो लोके पराशर इति स्मृतः}


\twolineshloka
{अमन्यत स धर्मात्मा वसिष्ठं पितरं मुनिः}
{जन्मप्रभृति तस्मिंस्तु पितरीवान्ववर्तत}


\twolineshloka
{स तात इति विप्रर्षिं वसिष्ठं प्रत्यभाषत}
{मातुः समक्षं कौन्तेय अदृश्यन्त्याः परन्तप}


\twolineshloka
{तातेति परिपूर्णार्थं तस्य तन्मधुरं वचः}
{अदृश्यन्त्यश्रुपूर्णाक्षी शृण्वती तमुवाच ह}


\twolineshloka
{मा तात ताततातेति ब्रूह्येनं पितरं पितुः}
{रक्षसा भक्षितस्तात तव तातो वनान्तरे}


\twolineshloka
{मन्यसे यं तु तातेति नैष तातस्तवानघ}
{आर्य एष पिता तस्य पितुस्तव यशस्विनः}


\twolineshloka
{स एवमुक्तो दुःखार्तः सत्यवागृषिसत्तमः}
{सर्वलोकविनाशाय मतिं चक्रे महामनाः}


\twolineshloka
{तं तथा निश्चितात्मानं स महात्मा महातपाः}
{ऋषिर्ब्रह्मविदां श्रष्ठो मैत्रावरुणिरन्त्यधीः}


\threelineshloka
{वसिष्ठो वारयामास हेतुना येन तच्छृणु}
{वसिष्ठ उवाच}
{कृतवीर्य इति ख्यातो बभूव पृथिवीपतिः}


\twolineshloka
{याज्यो वेदविदां लोके भृगूणां पार्थिवर्षभः}
{स तानग्रभुजस्तात धान्येन च धनेन च}


\twolineshloka
{सोमान्ते तर्पयामास विपुलेन विशांपतिः}
{तस्मिन्नृपतिशार्दूले स्वर्यातेऽथ कथंचन}


\twolineshloka
{बभूव तत्कुलेयानां द्रव्यकार्यमुपस्थितम्}
{भृगूणां तु धनं ज्ञात्वा राजानः सर्व एव ते}


\twolineshloka
{याचिष्णवोऽभिजग्मुस्तांस्ततो भार्गवसत्तमान्}
{भूमौ तु निददुः केचिद्भृगवो धनमक्षयम्}


\twolineshloka
{ददुः केचिद्द्विजातिभ्यो ज्ञात्वा क्षत्रियतो भयम्}
{भृहवस्तु ददुः केचित्तेषां वित्तं यथेप्सितम्}


\twolineshloka
{क्षत्रियाणां तदा तात कारणान्तरदर्शनात्}
{ततो महीतलं तात क्षत्रियेण यदृच्छया}


\twolineshloka
{खनताऽधिगतं वित्तं केनच्चिद्धृगुवेश्मनि}
{तद्वित्तं ददृशुः सर्वे समेताः क्षत्रियर्षभाः}


\twolineshloka
{अवमन्य ततः क्रोधाद्भृगूंस्ताञ्छरणगतान्}
{निजघ्नुः परमेष्वासाः सर्वांस्तान्निशितैः शरैः}


\twolineshloka
{आगर्भादवकृन्तन्तश्चेरुः सर्वां वसुन्धराम्}
{तत उच्छिद्यमानेषु भृगुष्वेवं भयात्तदा}


\twolineshloka
{भृगुपत्न्यो गिरिं दुर्गं हिमवन्तं प्रपेदिरे}
{तासामन्यतमा गर्भं भयाद्दध्रे महौजसम्}


\twolineshloka
{ऊरुणैकेन वाभोरूर्भर्तुः कुलविवृद्धये}
{तं गर्भमुपलभ्याशु ब्राह्मण्येका भयार्दिता}


\twolineshloka
{गत्वा वै कथयामास क्षत्रियाणामुपह्वरे}
{ततस्ते क्षत्रिया जग्मुस्तं गर्भं हन्तुमुद्यताः}


\twolineshloka
{ददृशुर्ब्राह्मणीं तेऽथ दीप्यमानां स्वतेजसा}
{अथ गर्भः स भित्त्वोरुं ब्राह्मण्या निर्जगाम ह}


\twolineshloka
{मुष्णन्दृष्टीः क्षत्रियाणां मध्याह्न इव भास्करः}
{ततश्चक्षुर्विहीनास्ते गिरिदुर्गेषु बभ्रमुः}


\twolineshloka
{ततस्ते मोघसङ्कल्पा भयार्ताः क्षत्रियाः पुनः}
{ब्राह्मणीं शरमं जग्मुर्दृष्ट्यर्थं तामनिन्दिताम्}


\twolineshloka
{ऊचुश्चैनां महाभागां क्षत्रियास्ते विचेतसः}
{ज्योतिःप्रहीणा दुःखार्ताः शान्तार्चिष इवाग्नयः}


\twolineshloka
{भगवत्याः प्रसादेन गच्छेत्क्षत्रमनामयम्}
{उपारम्य च गच्छेम सहिताः पापकर्मणः}


\twolineshloka
{सपुत्रा त्वं प्रसादं नः कर्तुमर्हसि शोभने}
{पुनर्दृष्टिप्रदानेन राज्ञः संत्रातुमर्हसि}


\chapter{अध्यायः १९५}
\twolineshloka
{ब्राह्मण्युवाच}
{}


\twolineshloka
{नाहं गृह्णामि वस्ताता दृष्टीर्नास्मि रुषान्विता}
{अयं तु भार्गवो नूनमूरुजः कुपितोऽद्य वः}


\twolineshloka
{तेन चक्षूंषि वस्ताता व्यक्तं कोपान्महात्मना}
{स्मरता निहतान्बन्धूनादत्तानि न संशयः}


\twolineshloka
{गर्भानपि यदा यूयं भृगूणां घ्नत पुत्रकाः}
{तदायमूरुणा गर्भो मया वर्षशतं धृतः}


\twolineshloka
{षडङ्गश्चाखिलो वेद इमं गर्भस्थमेव ह}
{विवेश भृगुवंशस्य भूयः प्रियचिकीर्षया}


\twolineshloka
{सोऽयं पितृवधाद्व्यक्तं क्रोधाद्वो हन्तुमिच्छति}
{तेजसा तस्य दिव्येन चक्षूंषि मुषितानि वः}


\threelineshloka
{तमेव यूयं याचध्वमौर्वं मम सुतोत्तमम्}
{अयं वः प्रणिपातेन तुष्टो दृष्टीः प्रदास्यति ॥वसिष्ठ उवाच}
{}


\twolineshloka
{एवमुक्तास्ततः सर्वे राजानस्ते तमूरुजम्}
{ऊचुः प्रसीदेति तदा प्रसादं च चकार सः}


\twolineshloka
{अनेनैव च विख्यातो नाम्ना लोकेषु सत्तमः}
{स और्व इति विप्रर्षिरूरुं भित्त्वा व्यजायत}


\twolineshloka
{चक्षूंषि प्रतिलभ्याथ प्रतिजग्मुस्ततो नृपाः}
{भार्गवस्तु मुनिर्मेने सर्वलोकपराभवम्}


\twolineshloka
{स चक्रे तात लोकानां विनाशाय मतिं तदा}
{सर्वेषामेव कार्त्स्न्येन मनः प्रवणमात्मनः}


\twolineshloka
{इच्छन्नपचितिं कर्तुं भृगूणां भृगुनन्दनः}
{सर्वलोकविनाशाय तपसा सहतैधितः}


\twolineshloka
{तापयामास ताँल्लोकान्सदेवासुरमानुषान्}
{तपसोग्रेण महता नन्दयिष्यन्पितामहान्}


\twolineshloka
{ततस्तं पितरस्तात विज्ञाय कुलनन्दनम्}
{पितृलोकादुपागम्य सर्व ऊचुरिदं वचः}


\twolineshloka
{और्व दृष्टः प्रभावस्ते तपसोग्रस्य पुत्रक}
{प्रसादं कुरु लोकानां नियच्छ क्रोधमात्मनः}


\twolineshloka
{नानीशैर्हि तदा तात भृगुभिर्भावितात्मभिः}
{वधो ह्युपेक्षितः सर्वैः क्षत्रियाणां विहिंसताम्}


\twolineshloka
{आयुषा विप्रकृष्टेन यदा नः खेद आविशत्}
{तदाऽस्माभिर्वधस्तात क्षत्रियैरीप्सितः स्वयम्}


\twolineshloka
{निखातं यच्च वै वित्तं केनचिद्गृगुवेश्मनि}
{वैरायैव तदा न्यस्तं क्षत्रियान्कोपयिष्णुभिः}


\twolineshloka
{किं हि वित्तेन नः कार्यं स्वर्गेप्सूनां द्विजोत्तम}
{यदस्माकं धनाध्यक्षः प्रभूतं धनमाहरत्}


\twolineshloka
{यदा तु मृत्युरादातुं न नः शक्नोति सर्वशः}
{तदाऽस्माभिरयं दृष्ट उपायस्तात संमतः}


\twolineshloka
{आत्महा च पुमांस्तात न लोकाँल्लभते शुभान्}
{ततोऽस्माभिः समीक्ष्यैवं नात्मनात्मा निपातितः}


\twolineshloka
{न चैतन्नः प्रियं तात यदिदं कर्तुमिच्छसि}
{नियच्छेदं मनः पापात्सर्वलोकपराभवात्}


\twolineshloka
{मा वधीः क्षत्रियांस्तात न लोकान्सप्त पुत्रक}
{दूषयन्तं तपस्तेजः क्रोधमुत्पतितं जहि}


\chapter{अध्यायः १९६}
\twolineshloka
{और्व उवाच}
{}


\twolineshloka
{उक्तवानस्मि यां क्रोधात्प्रतिज्ञां पितरस्तदा}
{सर्वलोकविनाशाय न सा मे वितथा भवेत्}


\twolineshloka
{वृथारोषप्रतिज्ञो वै नाहं जीवितुमुत्सहे}
{अनिस्तीर्णो हि मां रोषो दहेदग्निरिवारणिम्}


\twolineshloka
{यो हि कारणतः क्रोधं संजातं क्षन्तुमर्हति}
{नालं स मनुजः सम्यक् त्रिवर्गं परिरक्षितुम्}


\twolineshloka
{अशिष्टानां नियन्ता हि शिष्टानां परिरक्षिता}
{स्थाने रोषः प्रयुक्तः स्यान्नृपैः सर्वजिगीषुभिः}


\twolineshloka
{अश्रौषमहमूरुस्थो गर्भशय्यागतस्तदा}
{आरावं मातृवर्गस्य भृगूणां क्षत्रियैर्वधे}


\twolineshloka
{सामरैर्हि यदा लोके भृगूणां क्षत्रियाधमैः}
{आगर्भोत्सादनं क्षान्तं तदा मां मन्युराविशत्}


\twolineshloka
{प्रकीर्णकेशाः किल मे मातरः पितरस्तथा}
{भयात्सर्वेषु लोकेषु नाधिजग्मुः परायणम्}


\twolineshloka
{तान्भृगूणां यदा दारान्कश्चिन्नाभ्युपपद्यत}
{माता तदा दधारेयमूरुणैकेन मां शुभा}


\twolineshloka
{प्रतिषेद्धा हि पापस्य यदा लोकेषु विद्यते}
{तदा सर्वेषु लोकेषु पापकृन्नोपपद्यते}


\twolineshloka
{यदा तु प्रतिषेद्धारं पापो न लभते क्वचित्}
{तिष्ठन्ति बहवो लोकास्तदा पापेषु कर्मसु}


\twolineshloka
{जानन्नपि च यः पापं शक्तिमान्न नियच्छति}
{ईशः सन्सोऽपि तेनैव कर्मणा संप्रयुज्यते}


\twolineshloka
{राजभिश्चेश्वरैश्चैव यदि वै पितरो मम}
{शक्तैर्न शकितास्त्रातुमिष्टं मत्वेह जीवितम्}


\twolineshloka
{अत एषामहं क्रुद्धो लोकानामीश्वरो ह्यहम्}
{भवतां च वचो नालमहं समभिवर्तितुम्}


\twolineshloka
{ममापि चेद्भवेदेवमीश्वरस्य सतो महत्}
{उपेक्षमाणस्य पुनर्लोकानां किल्बिषाद्भयम्}


\twolineshloka
{यश्चायं मन्युजो मेऽग्निर्लोकानादातुमिच्छति}
{दहेदेष च मामेव निगृहीतः स्वतेजसा}


\threelineshloka
{भवतां च विजानामि सर्वलोकहितेप्सुताम्}
{तस्माद्विधद्ध्वं यच्छ्रेयो लोकानां मम चेश्वराः ॥पितर ऊचुः}
{}


\twolineshloka
{य एष मन्युजस्तेऽग्निर्लोकानादातुमिच्छति}
{अप्सु तं मुञ्च भद्रं ते लोका ह्यप्सु प्रतिष्ठिताः}


\twolineshloka
{आपोमयाः सर्वरसाः सर्वमापोमयं जगत्}
{तस्मादप्सु विमुञ्चेमं क्रोधाग्निं द्विजसत्तम}


\twolineshloka
{अयं तिष्ठतु ते विप्र यदीच्छसि महोदधौ}
{मन्युजोऽग्निर्दहन्नापो लोका ह्यापोमयाः स्मृताः}


\threelineshloka
{एवं प्रतिज्ञा सत्येयं तवानघ भविष्यति}
{न चैवं सामरा लोका गमिष्यन्ति पराभवम् ॥वसिष्ठ उवाच}
{}


\twolineshloka
{ततस्तं क्रोधजं तात और्वोऽग्निं वरुणालये}
{उत्ससर्ज स चैवाप उपयुङ्क्ते महोदधौ}


\twolineshloka
{महद्धयशिरो भूत्वा यत्तद्वेदविदो विदुः}
{तमग्निमुद्हिरद्वक्त्रात्पिबत्यापो महोदधौ}


\twolineshloka
{तस्मात्त्वमपि भद्रं ते न लोकान्हन्तुमर्हसि}
{पराशरं पराँल्लोकाञ्जानञ्ज्ञानवतां वर}


\chapter{अध्यायः १९७}
\twolineshloka
{गन्धर्व उवाच}
{}


\twolineshloka
{एवमुक्तः स विप्रर्षिर्वसिष्ठेन महात्मना}
{न्ययच्छदात्मनः क्रोधं सर्वलोकपराभवात्}


\twolineshloka
{ईजे च स महातेजाः सर्ववेदविदां वरः}
{ऋषी राक्षससत्रेण शाक्तेयोऽथ पराशरः}


\twolineshloka
{ततो वृद्धांश्च बालांश्च राक्षसान्स महामुनिः}
{ददाह वितते यज्ञे शक्तेर्वधमनुस्मरन्}


\twolineshloka
{न हि तं वारयामास वसिष्ठो रक्षसां वधात्}
{द्वितीयामस्य मां भाङ्क्षं प्रतिज्ञामिति निश्चयात्}


\twolineshloka
{त्रयाणां पावकानां च सत्रे तस्मिन्महामुनिः}
{आसीत्पुरस्ताद्दीप्तानां चतुर्थ इव पावकः}


\twolineshloka
{तेन यज्ञेन शुभ्रेण हूयमानेन शक्तितः}
{तद्विदीपितमाकाशं सूर्येणेव घनात्यये}


\twolineshloka
{तं वसिष्ठादयः सर्वे मुनयस्तत्र मेनिरे}
{तेजसा दीप्यमानं वै द्वितीयमिव भास्करम्}


\twolineshloka
{ततः परमदुष्प्रापमन्यैर्ऋषिरुधारधीः}
{समापिपयिषुः सत्रं तमत्रिः समुपागमत्}


\twolineshloka
{तथा पुलस्त्यः पुलहः क्रतुश्चैव महाक्रतुः}
{तत्राजग्मुरमित्रघ्न रक्षसां जीवितेप्सया}


\twolineshloka
{पुलस्त्यस्तु वधात्तेषां रक्षसां भरतर्षभ}
{उवाचेदं वचः पार्थ पराशरमरिन्दमम्}


\twolineshloka
{कच्चित्तातापविघ्नं ते कच्चिन्नन्दसि पुत्रक}
{अजानतामदोषाणां सर्वेषां रक्षसां वधात्}


\twolineshloka
{प्रजोच्छेदमिमं मह्यं न हि कर्तु त्वमर्हसि}
{नैष तात द्विजातीनां धर्मो दृष्टस्तपस्विनाम्}


\twolineshloka
{शम एव परो धर्मस्तमाचर पराशर}
{अधर्मिष्ठं वरिष्ठः सन्कुरुषे त्वं पराशर}


\twolineshloka
{शक्तिं चापि हि धर्मज्ञं नातिक्रान्तुमिहार्हसि}
{प्रजायाश्च ममोच्छेदं न चैवं कर्तुमर्हसि}


\twolineshloka
{शापाद्धि शक्तेर्वासिष्ठ तदा तदुपपादितम्}
{आत्मजेन स दोषेण शक्तिर्नीत इतो दिवम्}


\threelineshloka
{न हि तं राक्षसः कश्चिच्छक्तो भक्षयितुं मुने}
{`वासिष्ठो भक्षितश्चासीत्कौशिकोत्सृष्टरक्षसा}
{शापं न कुर्वन्ति तदा न च त्राणपरायणाः}


\twolineshloka
{क्षमावन्तोऽदहन्देहं देहमन्यद्भवत्विति}
{'आत्मनैवात्मनस्तेन दृष्टो मृत्युस्तदाऽभवत्}


\twolineshloka
{निमित्तभूतस्तत्रासीद्विश्वामित्रः पराशर}
{राजा कल्माषपादश्च दिवमारुह्य मोदते}


\twolineshloka
{ये च शक्त्यवराः पुत्रा वसिष्ठस्य महामुने}
{ते च सर्वे मुदा युक्ता मोदन्ते सहिताः सुरैः}


\twolineshloka
{सर्वमेतद्वसिष्ठस्य विदितं वै महामुने}
{रक्षसां च समुच्छेद एष तात तपस्विनाम्}


\threelineshloka
{निमित्तभूतस्त्वं चात्र क्रतौ वासिष्ठनन्दन}
{तत्सत्रं मुञ्च भद्रं ते समाप्तमिदमस्तु ते ॥गन्धर्व उवाच}
{}


\twolineshloka
{एवमुक्तः पुलस्त्येन वसिष्ठेन च धीमता}
{तदा समापयामास सत्रं शाक्तो महामुनिः}


\twolineshloka
{सर्वराक्षससत्राय संभृतं पावकं तदा}
{उत्तरे हिमवत्पार्श्वे उत्ससर्ज महावने}


\twolineshloka
{स तत्राद्यापि रक्षांसि वृक्षानश्मन एव च}
{भक्षयन्दृश्यते वह्निः सदा पर्वणि पर्वणि}


\chapter{अध्यायः १९८}
\twolineshloka
{`गन्धर्व उवाच}
{}


\twolineshloka
{पुनश्चैव महातेजा विश्वामित्रजिघांसया}
{अग्निं संभृतवान्घोरं शाक्तेयः सुमहातपाः}


\threelineshloka
{वासिष्ठसंभृतश्चाग्निर्विश्वामित्रहितैषिणा}
{तेजसा वह्नितुल्येन ग्रस्तः स्कन्देन धीमता ॥'अर्जुन उवाच}
{}


\twolineshloka
{राज्ञा कल्माषपादेन गुरौ ब्रह्मविदां वरे}
{कारणं किं पुरस्कृत्य भार्या वै सन्नियोजिता}


\twolineshloka
{जानता वै परं धर्मं वसिष्ठेन महात्मना}
{अगम्यागमनं कस्मात्कृतं तेन महर्षिणा}


\threelineshloka
{अधर्मिष्ठं वसिष्ठेन कृतं चापि पुरा सखे}
{एतन्मे संशयं सर्वं छेत्तुमर्हसि पृच्छतः ॥गन्धर्व उवाच}
{}


\twolineshloka
{धनञ्जय निबोधेयं यन्मां त्वं परिपृच्छसि}
{वसिष्ठं प्रति दुर्धर्ष तथा मित्रसहं नृपम्}


\twolineshloka
{कथितं ते मया सर्वं यथा शप्तः स पार्थिवः}
{शक्तिना भरतश्रेष्ठ वासिष्ठेन महात्मना}


\twolineshloka
{स तु शापवशं प्राप्तः क्रोधपर्याकुलेक्षणः}
{निर्जगाम पुराद्राजा सहदारः परन्तपः}


\twolineshloka
{अरण्यं निर्जनं गत्वा सदारः परिचक्रमे}
{नानामृगगणाकीर्णं नानासत्वसमाकुलम्}


\twolineshloka
{नानागुल्मलताच्छन्नं नानाद्रुमसमावृतम्}
{अरण्यं घोरसन्नादं शापग्रस्तः परिभ्रमन्}


\twolineshloka
{स कदाचित्क्षुधाविष्टो मृगयन्भक्ष्यमात्मनः}
{ददर्श सुपरिक्लिष्टः कस्मिंश्चिन्निर्जने वने}


\twolineshloka
{ब्राह्मणं ब्राह्मणीं चैव मिथुनायोपसंगतौ}
{तौ तं वीक्ष्य सुवित्रस्तावकृतार्थौ प्रधावितौ}


\twolineshloka
{तयोः प्रद्रवतोर्विप्रं जग्राह नृपतिर्बलात्}
{दृष्ट्वा गृहीतं भर्तारमथ ब्राह्मण्यभाषत}


\twolineshloka
{शृणु राजन्मम वचो यत्त्वां वक्ष्यामि सुव्रत}
{आदित्यवंशप्रभवस्त्वं हि लोके परिश्रुतः}


\twolineshloka
{अप्रमत्तः स्थि धर्मे गुरुशुश्रूषणे रतः}
{शापोपहत दुर्धर्ष न पापं कर्तुमर्हसि}


\twolineshloka
{ऋतुकाले तु संप्राप्ते भर्तृव्यसनकर्शिता}
{अकृतार्था ह्यहं भर्त्रा प्रसवार्थं समागता}


\twolineshloka
{प्रसीद नृपतिश्रेष्ठ भर्ताऽयं मे विसृज्यताम्}
{एवं विक्रोशमानायास्तस्यास्तु न नृशंसवत्}


\twolineshloka
{भर्तारं भक्षयामास व्याघ्रो मृगमिवेप्सितम्}
{तस्याः क्रोधाभिभूताया यान्यश्रूण्यपतन्भुवि}


\twolineshloka
{सोऽग्निः समभवद्दीप्तस्तं च देशं व्यदीपयत्}
{ततः सा शोकसंतप्ता भर्तृव्यसनकर्शिता}


\twolineshloka
{कल्माषपादं राजर्षिमशपद्ब्राह्मणी रुषा}
{यस्मान्ममाकृतार्थायास्त्वया क्षुद्र नृशंसवत्}


\twolineshloka
{प्रेक्षन्त्या भक्षितो मेऽद्य प्रियो भर्ता महायशाः}
{तस्मात्त्वमपि दुर्बुद्धे मच्छापपरिविक्षतः}


\threelineshloka
{पत्नीमृतावनुप्राप्य सद्यस्त्यक्ष्यसि जीवितम्}
{`तेन प्रसाद्यमाना सा प्रसादमकरोत्तदा}
{'यस्य चर्षेर्वसिष्ठस्य त्वया पुत्रा विनाशिताः}


\twolineshloka
{तेन संगम्य ते भार्या तनयं जनयिष्यति}
{सते वंशकरः पुत्रो भविष्यति नृपाधम}


\twolineshloka
{एवं शप्त्वा तु राजानं सा तमाङ्गिरसी शुभा}
{तस्यैव सन्निधौ दीप्तं प्रविवेश हुताशनम्}


\twolineshloka
{वसिष्ठश्च महाभागः सर्वमेतदवैक्षत}
{ज्ञानयोगेन महता तपसा च परन्तप}


\twolineshloka
{मुक्तशापश्च राजर्षिः कालेन महता ततः}
{ऋतुकालेऽभिपतितो मदयन्त्या निवारितः}


\twolineshloka
{न हि सस्मार स नृपस्तं शापं काममोहितः}
{देव्याः सोऽथ वचः श्रुत्वा संभ्रान्तो नृपसत्तमः}


\threelineshloka
{तं शापमनुसंस्मृत्य पर्यतप्यद्भृशं तदा}
{एतस्मात्कारणाद्राजा वसिष्ठं सन्ययोजयत्}
{स्वदारेषु नरश्रेष्ठ शापदोषसमन्वितः}


\chapter{अध्यायः १९९}
\twolineshloka
{अर्जुन उवाच}
{}


\threelineshloka
{अस्माकमनुरूपो वै यः स्याद्गन्धर्व वेदवित्}
{पुरोहितस्तमाचक्ष्व सर्वं हि विदितं तव ॥गन्धर्व उवाच}
{}


\threelineshloka
{यवीयान्देवलस्यैष वने भ्राता तपस्यति}
{धौम्य उत्कोचके तीर्थे तं वृणुध्वं यीच्छथ ॥वैशंपायन उवाच}
{}


\twolineshloka
{ततोऽर्जुनोऽस्त्रमाग्नेयं प्रददौ तद्यथाविधि}
{गन्धर्वाय `स च प्रीतो वचनं चेदमब्रवीत्}


\twolineshloka
{मयि सन्ति हयश्रेष्ठास्तव दास्यामि वै सखे}
{उपकारकृतं मित्रं प्रतिकारेण योजये}


\twolineshloka
{गृह्णीष्व चाक्षुषीं विद्यामिमां भरतसत्तम}
{एवमुक्तोऽर्जुनः' प्रीतो वचनं चेदमब्रवीत्}


\twolineshloka
{त्वय्येव तावत्तिष्ठन्तु हया गन्धर्वसत्तम}
{कार्यकाले ग्रहीष्यामः स्वस्ति तेऽस्त्विति चाब्रवीत्}


\twolineshloka
{तेऽन्योन्यमभिसंपूज्य गन्धर्वः पाण्डवाश्च ह}
{रम्याद्भागीरथीतीराद्यथाकामं प्रतस्थिरे}


\twolineshloka
{तत उत्कोचकं तीर्थं गत्वा धौम्याश्रमं तु ते}
{तं वव्रुः पाण्डवा धौम्यं पौरोहित्याय भारत}


\twolineshloka
{तान्धौम्यः प्रतिजग्राह सर्ववेदविदां वरः}
{वन्येन फलमूलेन पौरोहित्येन चैव ह}


\twolineshloka
{ते समाशंसिरे लब्धां श्रियं राज्यं च पाण्डवाः}
{ब्राह्मणं तं पुरस्कृत्य पाञ्चालीं च स्वयंवरे}


\twolineshloka
{पुरोहितेन तेनाथ गुरुणा संगतास्तदा}
{नाथवन्तमिवात्मानं मेनिरे भरतर्षभाः}


\twolineshloka
{स हि वेदार्थतत्त्वज्ञस्तेषां गुरुरुदारधीः}
{वेदविच्चैव वाग्मी च धौम्यः श्रीमान्द्विजोत्तमः}


\twolineshloka
{तेजसा चैव बुद्ध्या च रूपेण यशसा श्रिया}
{मन्त्रैश्च विविधैर्धौम्यस्तुल्य आसीद्बृहस्पतेः}


\twolineshloka
{स चापि विप्रस्तान्मेने स्वभावाभ्यधिकान्भुवि}
{तेन धर्मविदा पार्था योज्या सर्वविदा वृताः}


\twolineshloka
{मेनिरे सहिता वीराः प्राप्तं राज्यं च पाण्डवाः}
{बुद्धिवीर्यबलोत्साहैर्युक्ता देवा इवापरे}


\twolineshloka
{कृतस्वस्त्ययनास्तेन ततस्ते मनुजाधिपाः}
{मेनिरे सहिता गन्तुं पाञ्चाल्यास्तं स्वयंवरम्}


\chapter{अध्यायः २००}
\twolineshloka
{वैशंपायन उवाच}
{}


\threelineshloka
{ततस्ते नरशार्दूला भ्रातरः पञ्च पाण्डवाः}
{तं ब्राह्मणं पुरस्कृत्य पाञ्चाल्याश्च स्वयंवरम्}
{प्रययुर्द्रौपदीं द्रष्टुं तं च देशं महोत्सवम्}


\twolineshloka
{ततस्ते तं महात्मानं शुद्धात्मानमकल्मषम्}
{ददृशुः पाण्डवा वीराः पथि द्वैपायनं तदा}


\twolineshloka
{तस्मै यथावत्सत्कारं कृत्वा तेन च सत्कृताः}
{कथान्ते चाभ्यनुज्ञाताः प्रययुर्द्रुपदक्षयम्}


\twolineshloka
{पश्यन्तो रमणीयानि वनानि च सरांसि च}
{तत्रतत्र वसन्तश्च शनैर्जग्मुर्महराथाः}


\twolineshloka
{स्वाध्यायवन्तः शुचयो मधुराः प्रियवादिनः}
{आनुपूर्व्येण संप्राप्ताः पाञ्चालान्पाण्डुनन्दनाः}


\twolineshloka
{ते तु दृष्ट्वा पुरं तच्च स्कन्धावारं च पाण्डवाः}
{कुम्भकारस्य शालायां निवासं चक्रिरे तदा}


\twolineshloka
{तत्र भैक्षं समाजह्रुर्ब्राह्मणीं वृत्तिमाश्रिताः}
{तान्संप्राप्तांस्तथा वीराञ्जज्ञिरे न नराः क्वचित्}


\twolineshloka
{`यज्ञसेनस्तु पाञ्चालो भीष्मद्रोमकृतागसम्}
{ज्ञात्वाऽऽत्मानं तदारेभे त्राणायात्मक्रियां क्षमां}


\twolineshloka
{अवाप्य धृष्टद्युम्नं हि न स द्रोणमचिन्तयत्}
{स तु वैरप्रसङ्गाच्च भीष्माद्भयमचिन्तयत्}


\twolineshloka
{कन्यादानात्तु शरणं सोऽमन्यत महीपतिः}
{'यज्ञसेनस्य कामस्तु पाण्डवाय किरीटिने}


\twolineshloka
{दास्यामि कृष्मामिति वै न चैनं विवृणोति सः}
{`जामातृबलसंयोगं मेने हि बलवत्तरम् ॥'}


\twolineshloka
{सोऽन्वेषमाणः कौन्तेयान्पाञ्चालो जनमेजय}
{दृढं धनुरथानम्यं कारयामास भारत}


\twolineshloka
{`वैयाघ्रपद्यस्योग्रं वै सृञ्जयस्य महीपतिः}
{तद्धनुः किन्धुरं नाम देवदत्तमुपानयत्}


\twolineshloka
{आयसी तस्य च ज्याऽऽसीत्प्रतिबद्धा महाबला}
{न तु ज्यां प्रसहेदन्यस्तद्धनुःप्रवरं महत्}


\twolineshloka
{शङ्करेण वरं दत्तं प्रीतेन च महात्मना}
{तन्निष्फलं स्यान्न तु मे इति प्रामाण्यमागतः}


\twolineshloka
{मया कर्तव्यमधुना दुष्करं लक्ष्यवेधनम्}
{इति निश्चित्य मनसा कारितं लक्ष्यमुत्तमम् ॥'}


\threelineshloka
{यन्त्रं वैहायसं चापि कारयामास कृत्रिमम्}
{तेन यन्त्रेण सहितं राजँल्लक्ष्यं च काञ्चनम् ॥द्रुपद उवाच}
{}


\threelineshloka
{इदं सज्यं धनुः कृत्वा सज्जैरेभिश्च सायकैः}
{अतीत्य लक्ष्य यो वेद्धा स लब्धा मत्सुतामिति ॥वैशंपायन उवाच}
{}


\twolineshloka
{इति स द्रुपदो राजा स्वयंवरमघोषयत्}
{तच्छ्रुत्वा पार्थिवाः सर्वे समीयुस्तत्र भारत}


\twolineshloka
{ऋषयश्च महात्मानः स्वयंवरदिदृक्षवः}
{दुर्योधनपुरोगाश्च सकर्णाः कुरवो नृप}


\twolineshloka
{`यादवा वासुदेवेन सार्धमन्धकवृष्णयः}
{'ततोऽर्चिता राजगुणा द्रुपदेन महात्मना}


\twolineshloka
{उपोपविष्टा मञ्चेषु द्रष्टुकामाः स्वयंवरम्}
{`ब्राह्मणाश्च महाभागा देशेभ्यः समुपागमन्}


\twolineshloka
{ब्राह्मणैरेव सहिताः पाण्डवाः समुपाविशन्}
{त्रयस्त्रिंशत्सुराः सर्वे विमानैर्व्योम्न्यवस्थिताः}


\twolineshloka
{ततः पौरजनाः सर्वे सागरोद्धूतनिःस्वनाः}
{शिंशुमारशिरः प्राप्य न्यविशंस्ते स्म पार्थिवाः}


\twolineshloka
{प्रागुत्तरेण नगराद्भूमिभागे समे शुभे}
{समाजवाटः शुशुभे भवनैः सर्वतो वृतः}


\twolineshloka
{प्राकारपरिखोपेतो द्वारतोरणमण्डितः}
{वितानेन विचित्रेण सर्वतः समलङ्कृतः}


\twolineshloka
{तूर्यौघशतसङ्कीर्णः परार्ध्यागुरुधूपितः}
{चन्दनोदकसिक्तश्च माल्यदामोपशोभितः}


\twolineshloka
{कैलासशिखरप्रख्यैर्नभस्तलविलेखिभिः}
{सर्वतः संवृतः शुभ्रैः प्रासादैः सुकृतोच्छ्रयै}


\twolineshloka
{सुवर्णजालसंवीतैर्मणिकुट्टिमभूषणैः}
{सुखारोहणसोपानैर्महासनपरिच्छदैः}


\twolineshloka
{स्रग्दामसमवच्छन्नैरगुरूत्तमवासितैः}
{हंसांशुवर्णैर्बहुभिरायोजनसुगन्धिभिः}


\twolineshloka
{असंबाधशतद्वारैः शयनासनशोभितैः}
{बहुधातुपिनद्धाङ्गैर्हिमवच्छिखरैरिव}


\twolineshloka
{तत्र नानाप्रकारेषु विमानेषु स्वलङ्कृताः}
{स्पर्धमानास्तदाऽन्योन्यं निषेदुः सर्वपार्थिवाः}


\twolineshloka
{तत्रोपविष्टान्ददृशुर्महासत्वान्पृथग्जनाः}
{राजसिंहान्महाभागान्कृष्णागुरुविभूषितान्}


\twolineshloka
{महाप्रसादान्ब्राह्मण्यान्स्वराष्ट्रपरिरक्षिणः}
{प्रियान्सर्वस्य लोकस्य सुकृतैः कर्मभिः शुभैः}


\twolineshloka
{मञ्चेषु च परार्द्ध्येषु पौरजानपदा जनाः}
{कृष्णादर्शनसिद्ध्यर्थं सर्वतः समुपाविशन्}


\twolineshloka
{ब्राह्मणैस्ते च सहिताः पाण्डवाः समुपाविशन्}
{ऋद्धिं पञ्चालराजस्य पश्यन्तस्तामनुत्तमाम्}


\twolineshloka
{ततः समाजो ववृधे स राजन्दिवसान्बहून्}
{रत्नप्रदानबहुलः शोभितो नटनर्तकैः}


\threelineshloka
{वर्तमाने समाजे तु रमणीयेऽह्नि षोडशे}
{`मैत्रे मुहूर्ते तस्याश्च राजदाराः पुराविदः}
{पुत्रवत्यः सुवसनाः प्रतिकर्मोपचक्रमुः}


\twolineshloka
{वैडूर्यमयपीठे तु निविष्टां द्रौपदीं तदा}
{सतूर्यं स्नापयाञ्चक्रुः स्वर्णकुम्भस्थितैर्जलैः}


\twolineshloka
{तां निवृत्ताभिषेकां च दुकूलद्वयधारिणीम्}
{निन्युर्मणिस्तम्भवतीं वेदिं वै सुपरिष्कृताम्}


\twolineshloka
{निवेश्य प्राङ्मुखीं हृष्टां विस्मिताक्ष्यः प्रसाधिकाः}
{केनालङ्करणेनेमामित्यन्योन्यं व्यलोकयन्}


\twolineshloka
{धूपोष्मणा च केशानामार्द्रभावं व्यपोहयन्}
{बबन्धुरस्या धम्मिल्लं माल्यैः सुरभिगन्धिभिः}


\twolineshloka
{दूर्वामधूकरचितं माल्यं तस्या ददुः करे}
{चक्रुश्च कृष्णागरुणा पत्रसङ्गं कुचद्वये}


\twolineshloka
{रेजे सा चक्रवाकाङ्का स्वर्णदीर्घा सरिद्वरा}
{अलकैः कुटिलैस्तस्या मुखं विकसितं बभौ}


\twolineshloka
{आसक्तभृङ्गं कुसुमं शशिम्बिम्बं जिगाय तत्}
{कालाञ्जनं नयनयोराचारार्थं समादधुः}


\twolineshloka
{भूषणं रत्नखचितैरलंचक्रुर्यथोचितम्}
{माता च तस्याः पृषती हरितालमनश्शिलाम्}


\twolineshloka
{अङ्गुलीभ्यामुपादाय तिलकं विदधे मुखे}
{अलङ्कृतां वधूं दृष्ट्वा योषितो मुदमाययुः}


\twolineshloka
{माता न मुमुदे तस्याः पतिः कीदृग्भविष्यति}
{सौविदल्लाः समागम्य द्रुपदस्याज्ञया ततः}


\twolineshloka
{एनामारोपयामासुः करिणीं कुचभूषिताम्}
{ततोऽवाद्यन्त वाद्यानि मङ्गलानि दिवि स्पृशन्}


\twolineshloka
{विलासिनीजनाश्चापि प्रवरं करिणीशतम्}
{माङ्गल्यगीतं गायन्त्यः पार्स्वयोरुभयोर्ययुः}


\twolineshloka
{जनापसरणे व्यग्राः प्रतिहार्यः पुरा ययुः}
{कोलाहलो महानासीत्तस्मिन्पुरवरे तदा}


\twolineshloka
{धृष्टद्युम्नो ययावग्रे हयमारुह्य भारत}
{द्रुपदो रङ्गदेशे तु बलेन महता युतः}


\twolineshloka
{तस्थौ व्यूह्य महानीकं पालितं दृढधन्विभिः}
{'आप्लुताङ्गीं सुवसनां सर्वाभरणभूषिताम्}


\twolineshloka
{मालां च समुपादाय काञ्चनीं समलङ्कृताम्}
{`आगतां ददृशुः सर्वे रङ्गभूमिमलङ्कृताम् ॥'}


\threelineshloka
{अवतीर्णा ततो रङ्गं द्रौपदी भरतर्षभ}
{`तस्थौ प्रमुदितान्सर्वान्नृपतीन्रङ्गमण्डले}
{प्रेक्षन्ती व्रीडितापाङ्गी द्रष्टृणां सुमनोहरा ॥'}


\twolineshloka
{पुरोहितः सोमकानां मन्त्रविद्ब्राह्मणः शुचिः}
{परिस्तीर्य जुहावाग्निमाज्येन विधिवत्तदा}


\twolineshloka
{संतर्पयित्वा ज्वलनं ब्राह्मणान्स्वस्ति वाच्य च}
{वारयामास सर्वाणि वादित्राणि समन्ततः}


\twolineshloka
{निःशब्दे तु कृते तस्मिन्धृष्टद्युम्नो विशांपते}
{कृष्णामादाय विधिवन्मेघदुन्दुभिनिःस्वनः}


\twolineshloka
{रङ्गमध्यं गतस्तत्र मेघगम्भीरया गिरा}
{वाक्यमुच्चैर्जगादेदं श्लक्ष्णमर्थवदुत्तमम्}


\twolineshloka
{इदं धनुर्लक्ष्यमिमे च बाणाःशृण्वन्तु मे भूपतयः समेताः}
{छिद्रेण यन्त्रस्य मसर्पयध्वंशरैः शितैर्व्योमचरैर्दशार्धैः}


\twolineshloka
{एतन्महत्कर्म करोति यो वैकुलेन रूपेण बलेन युक्तः}
{तस्याद्य भार्या भगिनी ममेयंकृष्णा भवित्री न मृषा ब्रवीमि}


\twolineshloka
{तानेवमुक्त्वा द्रुपदस्य पुत्रःपश्चादिदं तां भगिनीमुवाच}
{नाम्ना च गोत्रेण च कर्मणा चसंकीर्तयन्भूमिपतीन्समेतान्}


\chapter{अध्यायः २०१}
\twolineshloka
{धृष्टद्युम्न उवाच}
{}


\twolineshloka
{दुर्योधनो दुर्विषहो दुर्मुखो दुष्प्रधर्षणः}
{विविंशतिर्विकर्णश्च सहो दुःशासनस्तथा}


\twolineshloka
{युयुत्सुर्वायुवेगश्च भीमवेगरवस्तथा}
{उग्रायुधो बलाकी च करकायुर्विरोचनः}


\twolineshloka
{कुण्डकश्चित्रसेनश्च सुवर्चाः कनकध्वजः}
{नन्दको बाहुशाली च तुहुण्डो विकटस्तथा}


\twolineshloka
{एते चान्ये च बहवो धार्तराष्ट्रा महाबलाः}
{कर्णेन सहिता वीरास्त्वदर्थं समुपागताः}


\twolineshloka
{असङ्ख्याता महात्मानः पार्थिवाः क्षत्रियर्षभाः}
{शकुनिः सौबलश्चैव वृषकोऽथ बृहद्बलः}


\twolineshloka
{एते गान्धारराजस्य सुताः सर्वे समागताः}
{अश्वत्थामा च भोजश्च सर्वशस्त्रभृतां वरौ}


\twolineshloka
{समवेतौ महात्मानौ त्वदर्थे समलङ्कृतौ}
{बृहन्तो मणिमांश्चैव दण्डधारश्च पार्थिवः}


\twolineshloka
{सहदेवजयत्सेनौ मेघसन्धिश्च पार्थिवः}
{विराटः सह पुत्राभ्यां शङ्खेनैवोत्तरेण च}


\twolineshloka
{वार्धक्षेमिः सुशर्मा च सेनाबिन्दुश्च पार्थिवः}
{सुकेतुः सह पुत्रेण सुनाम्ना च सुवर्चसा}


\twolineshloka
{सुचित्रः सुकुमारश्च वृकः सत्यधृतिस्तथा}
{सूर्यध्वजो रोचमानो नीलश्चित्रायुधस्तथा}


\twolineshloka
{अंशुमांश्चेकितानश्च श्रेणिमांश्च महाबलः}
{समुद्रसेनपुत्रश्च चन्द्रसेनः प्रतापवान्}


\twolineshloka
{जलसन्धः पितापुत्रौ विदण्डो दण्ड एव च}
{पौण्ड्रको वासुदेवश्च भगदत्तश्च वीर्यवान्}


\twolineshloka
{कलिङ्गस्ताम्रलिप्तश्च पत्तनाधिपतिस्तथा}
{मद्रराजस्तथा शल्यः सहपुत्रो महारथः}


\twolineshloka
{रुक्माङ्गदेन वीरेण तथा रुक्मरथेन च}
{कौरव्यः सोमदत्तश्च पुत्रश्चास्य महारथः}


\twolineshloka
{समवेतास्त्रयः शूरा भूरिर्भूरिश्रवाः शलः}
{सुदक्षिणश्च काम्भोजो दृढधन्वा च पौरवः}


\twolineshloka
{बृहद्बलः सुषेणश्च शिबिरौशीनस्तथा}
{पटच्चरनिहन्ता च कारूषाधिपतिस्तथा}


\twolineshloka
{सङ्कर्षणो वासुदेवो रौक्मिणेयश्च वीर्यवान्}
{साम्बश्च चारुदेष्णश्च प्राद्युम्निः सगदस्तथा}


\twolineshloka
{अक्रूरः सात्यकिश्चैव उद्धवश्च महामतिः}
{कृतवर्मा च हार्दिक्यः पृथुर्विपृथुरेव च}


\twolineshloka
{विदूरथश्च कङ्कश्च शङ्कुश्च सगवेषणः}
{आशावहोऽनिरुद्धश्च समीकः सारिमेजयः}


\twolineshloka
{वीरो वातपतिश्चैव झिल्लीपिण्डारकस्तथा}
{उशीनरश्च विक्रान्तो वृष्णयस्ते प्रकीर्तिताः}


\twolineshloka
{भगीरथो बृहत्क्षत्रः सैन्धवश्च जयद्रथः}
{बृहद्रथो बाह्लिकश्च श्रउतायुश्च महारथः}


\twolineshloka
{उलूकः कैतवो राजा चित्राङ्गदशुभाङ्गदौ}
{वत्सराजश्च मतिमान्कोसलाधिपतिस्तथा}


\twolineshloka
{शिशुपालख्च विक्रान्तो जरासन्धस्तथैव च}
{एते चान्ये च बहवो नानाजनपदेश्वराः}


\threelineshloka
{त्वदर्थमागता भद्रे क्षत्रियाः प्रथिता भुवि}
{एते भेत्स्यन्ति विक्रान्तास्त्वदर्थे लक्ष्यमुत्तमम्}
{विध्यते य इदं लक्ष्यं वरयेथाः शुभेऽद्य तम्}


\chapter{अध्यायः २०२}
\twolineshloka
{वैशंपायन उवाच}
{}


\twolineshloka
{तेऽलङ्कृताः कुण्डलिनो युवानःपरस्परं स्पर्धमाना नरेन्द्राः}
{अस्त्रं बलं चात्मनि मन्यमानाःसर्वें समुत्पेतुरुदायुधास्ते}


\twolineshloka
{रूपेण वीर्येण कुले चैवशीलेन वित्तेन च यौवनेन}
{समिद्धदर्पा मदवेगभिन्नामत्ता यथा हैमवता गजेन्द्राः}


\twolineshloka
{परस्परं स्पर्धया प्रेक्षमाणाःसङ्कल्पजेनाभिपरिप्लुताङ्गाः}
{कृष्णा ममैवेत्यभिभाषमाणानृपाः समुत्पेतुरथासनेभ्यः}


\twolineshloka
{ते क्षत्रिया रङ्गगताः समेताजिगीषमाणा द्रुपदात्मजां ताम्}
{चकाशिरे पर्वतराजकन्या-मुमां यथा देवगणाः समेताः}


\twolineshloka
{कन्दर्पबाणाभिनिपीडिताङ्गांकृष्णागतैस्ते हृदयैर्नरेन्द्राः}
{रङ्गावतीर्णा द्रुपदात्मजार्थंद्वेषं प्रचक्रुः सुहृदोऽपि तत्र}


\twolineshloka
{अथाययुर्देवगणा विमानैरुद्रादित्या वसवोऽथाश्विनौ च}
{साध्याश्च सर्वे मरुतस्तथैवयमं पुरस्कृत्य धनेश्वरं च}


\twolineshloka
{दैत्याः सुपर्णाश्च महोरगाश्चदेवर्षयो गुह्यकाश्चारणाश्च}
{विश्वावसुर्नारदपर्वतौ चगन्धर्वमुख्याः सहसाऽप्सरोभिः}


\twolineshloka
{हलायुधस्तत्र जनार्दनश्चवृष्ण्यन्धकाश्चैव यताप्रधानम्}
{प्रेक्षां स्म चक्रुर्यदुपुङ्गवास्तेस्थिताश्च कृष्णस्य मते महान्तः}


\twolineshloka
{दृष्ट्वा तु तान्मत्तगजेन्द्ररूपा-न्पञ्चाभिपद्मानिव वारणेन्द्रान्}
{भस्मावृताङ्गानिव हव्यवाहान्कृष्णः प्रदध्यौ यदुवीरमुख्यः}


\twolineshloka
{शशंस रामाय युधिष्ठिरं सभीमं सजिष्णुं च यमौ च वीरौ}
{शनैःशनैस्तान्प्रसमीक्ष्य रामोजनार्दनं प्रीतमना ददर्श ह}


\twolineshloka
{अन्ये तु वीरा नृपपुत्रपौत्राःकृष्णागतैर्नत्रमनःस्वभावैः}
{व्यायच्छमाना ददृशुर्न तान्वैसन्दष्टदन्तच्छदताम्रनेत्राः}


\twolineshloka
{तथैव पार्थाः पृथुबाहवस्तेवीरौ यमौ चैव महानुभावौ}
{तां द्रौपदीं प्रेक्ष्य तदा स्म सर्वेकन्दर्पबाणाभिहता बभूवुः}


\twolineshloka
{देवर्षिगन्धर्वसमाकुलं त-त्सुपर्णनागासुरसिद्धजुष्टम्}
{दिव्येन गन्धेन समाकुलं चदिव्यैश्च पुष्पैरवकीर्यमाणम्}


\twolineshloka
{महास्वनैर्दुन्दुभिनादितैश्चबभूव तत्सङ्कुलमन्तरिक्षम्}
{विमानसंबाधमभूत्समन्ता-त्सवेणुवीणापणवानुनादम्}


\threelineshloka
{`समाजवाटोपरि संस्थितानांमेघैः समन्तादिव गर्जमानैः}
{'ततस्तु ते राजगणाः क्रमेणकृष्णानिमित्तं कृतविक्रमाश्च}
{सकर्णदुर्योधनशाल्वशल्य-द्रौणायनिक्राथसुनीथवक्राः}


\twolineshloka
{कलिङ्गवङ्गाधिपपाण्ड्यपौण्ड्राविदेहराजो यवनाधिपश्च}
{अन्ये च नानानृपपुत्रपौत्राराष्ट्राधिपाः पङ्कजपत्रनेत्राः}


\twolineshloka
{किरीटहाराङ्गदचक्रवालै-र्विभूषिताङ्गाः पृथुबाहवस्ते}
{अनुक्रमं विक्रमसत्वयुक्ताबलेन वीर्येण च नर्दमानाः}


\twolineshloka
{तत्कार्मुकं संहननोपपन्नंसज्यं न शेकुर्मनसाऽपि कर्तुम्}
{ते विक्रमन्तः स्फुरता दृढेनविक्षिप्यमाणा धनुषा नरेन्द्राः}


\twolineshloka
{विचेष्टमाना धरणीतलस्थायथाबलं शैक्ष्यगुणक्रमाश्च}
{गतौजसः स्नस्तकिरीटहाराविनिःश्वसन्तः शमयांबभूवुः}


\twolineshloka
{हहाकृतं तद्धनुषा दृढेनविस्रस्तहाराङ्गदचक्रवालम्}
{कृष्णानिमित्तं विनिवृत्तकामंराज्ञां तदा मण्डलमार्तमासीत्}


\twolineshloka
{`एवं तेषु निवृत्तेषु क्षत्रियेषु समन्ततः}
{चेदीनामधिपो वीरो बलवानन्तकोपमः}


\twolineshloka
{दमघोषात्मजो धीमाञ्शिशुपालो महाद्युतिः}
{धनुषोऽभ्याशमागम्य तस्थौ राज्ञां समक्षतः}


\twolineshloka
{तदप्यारोप्यमाणं तु माषमात्रेऽभ्यताडयत्}
{धनुषा पीड्यमानस्तु जानुभ्यामगमन्महीम्}


\twolineshloka
{तत उत्थाय राजा स स्वराष्ट्राण्यभिजग्मिवान्}
{ततो राजा जरासन्धो महावीर्यो महाबलः}


\twolineshloka
{कम्बुग्रीवः पृथुव्यंसो मत्तवारणविक्रमः}
{मत्तवारणताम्राक्षो मत्तवारणवेगवान्}


\twolineshloka
{धनुषोऽभ्याशमागत्य तस्थौ गिरिरिवाचलः}
{धनुरारोप्यमाणं तु सर्षमात्रेऽभ्यताडयत्}


\twolineshloka
{ततः शल्यो महावीर्यो मद्रराजो महाबलः}
{धनुरारोप्यमाणं तु मुद्गमात्रेऽभ्यताडयत्}


\twolineshloka
{तदैवागात्स्वयं राज्यं पश्चादनवलोकयन्}
{इदं धनुर्वरं कोऽद्य सज्यं कुर्वीत पार्थिवः}


\twolineshloka
{इति निश्चित्य मनसा भूय एव स्थितस्तदा}
{ततो दुर्योधनो राजा धार्तराष्ट्रः परन्तपः}


\twolineshloka
{मानी दृढास्त्रसंपन्नः सर्वैश्च नृपलक्षणैः}
{उत्थितः सहसा तत्र भ्रातृमध्ये महाबलः}


\twolineshloka
{विलोक्य द्रौपदीं हृष्टो धनुषोऽभ्याशमागमत्}
{स बभौ धनुरादाय शक्रश्चापधरो यथा}


\twolineshloka
{धनुरारोपयामास तिलमात्रेऽभ्यताडयत्}
{आरोप्यमाणं तद्राजा धनुषा बलिना तदा}


\twolineshloka
{उत्तानशय्यमपतदङ्गुल्यन्तरताडितः}
{स ययौ ताडितस्तेन व्रीडन्निव नराधिपः}


\twolineshloka
{ततो वैकर्तनः कर्णो वृषा वै सूतनन्दनः}
{धनुरभ्याशमागम्य तोलयामास तद्धनुः}


\twolineshloka
{तं चाप्यारोप्यमाणं तद्रोममात्रेऽभ्यताडयत्}
{त्रैलोक्यविजयी कर्णः सत्वे त्रैलोक्यविश्रुतः}


\twolineshloka
{धनुषा सोऽपि निर्धूत इति सर्वे भयाकुलाः}
{एवं कर्णे विनिर्धूते धनुषा च नृपोत्तमाः}


\twolineshloka
{चक्षुर्भिरपि नापश्यन्विनम्रमुखपङ्कजाः}
{दृष्ट्वा कर्णं विनिर्धूतं लोके वीरा नृपोत्तमाः}


% Check verse!
निराशा धनुरुद्धारे द्रौपदीसंगमेऽपि च
\twolineshloka
{तस्मिंस्तु संभ्रान्तजने समाजेनिक्षिप्तवादेषु जनाधिपेषु}
{कुन्तीसुतो जिष्णुरियेष कर्तुंसज्यं धनुस्तत्सशरं च वीरः}


\twolineshloka
{ततो वरिष्ठः सुरदानवाना-मुदरधीर्वृष्णिकुलप्रवीरः}
{जहर्ष रामेण स पीड्य हस्तंहस्तंगतां पाण्डुसुतस्य मत्वा}


\twolineshloka
{न जज्ञिरेऽन्ये नृपविप्रमुख्याःसंछन्नरूपानथ पाण्डुपुत्रान्}
{विना हि लोके च यदुप्रवीरौधौम्यं हि धर्मं सह सोदरांश्च}


\chapter{अध्यायः २०३}
\twolineshloka
{वैशंपायन उवाच}
{}


\twolineshloka
{यदा निवृत्ता राजानो धनुषः सज्यकर्मणः}
{अथोदतिष्ठद्विप्राणां मध्याज्जिष्णुरुदारधीः}


\twolineshloka
{उदक्रोशन्विप्रमुख्या विधुन्वन्तोऽजिनानि च}
{दृष्ट्वा संप्रस्थितं पार्थमिन्द्रकेतुसमप्रभम्}


\twolineshloka
{केचिदासन्विमनसः केचिदासन्मुदान्विताः}
{आहुः परस्परं केचिन्निपुणा बुद्धिजीविनः}


\twolineshloka
{यत्कर्णशल्यप्रमुखैः क्षत्रियैर्लोकविश्रुतैः}
{नानतं बलवद्भिर्हि धनुर्वेदपरायणैः}


\twolineshloka
{तत्कथं त्वकृतास्त्रेण प्राणतो दुर्बलीयसा}
{वटुमात्रेण शक्यं हि सज्यं कर्तुं धनुर्द्विजाः}


\twolineshloka
{अवहास्या भविष्यन्ति ब्राह्मणाः सर्वराजसु}
{कर्मण्यस्मिन्नसंसिद्धे चापलादपरीक्षिते}


\twolineshloka
{यद्येष दर्पाद्धर्षाद्वाप्यथ ब्राह्मणचापलात्}
{प्रस्थितो धनुरायन्तुं वार्यतां साधु मा गमत्}


\twolineshloka
{नावहास्या भविष्यामो न च लाघवमास्थिताः}
{न च विद्विष्टतां लोके गमिष्यामो महीक्षिताम्}


\twolineshloka
{केचिदाहुर्युवा श्रीमान्नागराजकरोपमः}
{पीनस्कन्धोरुबाहुश्च धैर्येण हिमवानिव}


\twolineshloka
{सिंहखेलगतिः श्रीमान्मत्तनागेन्द्रविक्रमः}
{संभाव्यमस्मिन्कर्मेदमुत्साहाच्चानुमीयते}


\twolineshloka
{शक्तिरस्य महोत्साहा न ह्यशक्तः स्वयं व्रजेत्}
{न च तद्विद्यते किंचित्कर्म लोकेषु यद्भवेत्}


\twolineshloka
{ब्राह्मणानामसाध्यं च नृषु संस्थानचारिषु}
{अब्भक्षा वायुभक्षाश्च फलाहारा दृढव्रताः}


\twolineshloka
{दुर्बला अपि विप्रा हि बलीयांसः स्वेतजसा}
{ब्राह्मणो नावमन्तव्यः सदसद्वा समाचरन्}


\twolineshloka
{सुखं दुःखं महद्ध्रस्वं कर्म यत्समुपागतम्}
{जामदग्न्येन रामेण निर्जिताः क्षत्रिया युधि}


\twolineshloka
{पीतः समुद्रोऽगस्त्येन अगाधो ब्रह्मतेजसा}
{तस्माद्ब्रुवन्तु सर्वेऽत्र वटुरेष धनुर्महान्}


\threelineshloka
{आरोपयतु शीघ्रं वै तथेत्यूचुर्द्विजर्षभाः}
{वैशंपायन उवाच}
{एवं तेषां विलपतां विप्राणां विविधा गिरः}


\twolineshloka
{अर्जुनो धनुषोऽभ्याशे तस्थौ गिरिरिवाचलः}
{`अर्जुनः पाण्डवश्रेष्ठो धृष्टद्युम्नमथाब्रवीत्}


\threelineshloka
{एतद्धनुर्ब्राह्मणानां सज्यं कर्तुमलं तु किम्}
{वैशंपायन उवाच}
{तस्य तद्वचनं श्रुत्वा धृष्टद्युम्नोऽब्रवीद्वचः}


\twolineshloka
{ब्राह्मणो वाथ राजन्यो वैश्यो वा शूद्र एव वा}
{एतेषां यो धनुःश्रेष्ठं सज्यं कुर्याद्द्विजोत्तम}


\twolineshloka
{तस्मै प्रदेया भगिनी सत्यमुक्तं मया वचः ॥वैशंपायन उवाच}
{}


\twolineshloka
{ततः पश्चान्महातेजाः पाण्डवो रणदुर्जयः}
{'स तद्धनुः परिक्रम्य प्रदक्षिणमथाकरोत्}


\twolineshloka
{प्रणम्य शिरसा देवमीनं वरदं प्रभुम्}
{कृष्णं च मनसा कृत्वा जगृहे चार्जुनो धनुः}


\twolineshloka
{यत्पार्थिवै रुक्मसुनीथवक्रैराधेयदुर्योधनशल्यसाल्वैः}
{तदा धनुर्वेदपरैर्नृसिंहैःकृतं न सज्यं महतोऽपि यत्नात्}


\twolineshloka
{तदर्जुनो वीर्यवतां सदर्प-स्तदैन्द्रिरिन्द्रावरजप्रभावः}
{सज्यं च चक्रे निमिषान्तरेणशरांश्च जग्राह दशार्दसङ्ख्यान्}


\twolineshloka
{विव्याध लक्ष्यं निपपात तच्चछिद्रेण भूमौ सहसातिविद्धम्}
{ततोऽन्तरिक्षे च बभूव नादःसमाजमध्ये च महान्निनादः}


% Check verse!
पुष्पाणि दिव्यानि ववर्ष देवःपार्थस्य मूर्ध्नि द्विषतां निहन्तुः
\threelineshloka
{चेलानि विव्यधुस्तत्र ब्राह्मणाश्च सहस्रशः}
{विलक्षितास्ततश्चक्रुर्हाहाकारांश्च सर्वशः}
{न्यपतंश्चात्र नभसः समन्तात्पुष्पवृष्टयः}


\twolineshloka
{शताङ्गानि च तूर्याणि वादकाः समवादयन्}
{सूतमागधसङ्घाश्चाप्यस्तुवंस्तत्र सुस्वराः}


\twolineshloka
{तं दृष्ट्वा द्रुपदः प्रीतो बभूव रिपुसूदनः}
{सह सैन्यैश्च पार्थस्य साहाय्यार्थमियेष सः}


\twolineshloka
{तस्मिंस्तु शब्दे महति प्रवृद्धेयुधिष्ठिरो धर्मभृतां वरिष्ठः}
{आवासमेवोपजगाम शीघ्रंसार्धं यमाभ्यां पुरुषोत्तमाभ्याम्}


\twolineshloka
{विद्धं तु लक्ष्यं प्रसमीक्ष्य कृष्णापार्थं च शक्रप्रतिमं निरीक्ष्य}
{`स्वभ्यस्तरूपापि नवेव नित्यंविनापि हासं हसतीव कन्या}


\twolineshloka
{मदादृतेऽपि स्खलतीव भावै-र्वाचा विना व्याहरतीव दृष्ट्या}
{आदाय शुक्लं वरमाल्यदामजगाम कुन्तीसुतमुत्स्मयन्ती}


\twolineshloka
{गत्वा च पश्चात्प्रसमीक्ष्य कृष्णापार्थस्य वक्षस्यविशङ्कमाना}
{क्षिप्त्वा स्रजं पार्थिववीरमध्येवराय वव्रे द्विजसङ्घमध्ये}


\twolineshloka
{शचीव देवेन्द्रमथाग्निदेवंस्वाहेव लक्ष्मीश्च यथा मुकुन्दम्}
{उषेव सूर्यं मदनं रतीवमहेश्वरं पर्वतराजपुत्री ॥'}


\twolineshloka
{स तामुपादाय विजित्य रङ्गेद्विजातिभिस्तैरभिपूज्यमानः}
{रङ्गान्निरक्रामदचिन्त्यकर्मापत्न्या तया चाप्यनुगम्यमानः}


\chapter{अध्यायः २०४}
\twolineshloka
{वैशंपायन उवाच}
{}


\twolineshloka
{तस्मै दित्सति कन्यां तु ब्राह्मणाय तदा नृपे}
{कोप आसीन्महीपानामालोक्यान्योन्यमन्तिकात्}


\threelineshloka
{`ऊचुः सर्वे समागम्य परस्परहितैषिणः}
{वयं सर्वे समाहूता द्रुपदेन दुरात्मना}
{संहत्य चाभ्यगच्छाम स्वयंवरदिदृक्षया ॥'}


\twolineshloka
{अस्मानयमतिक्रम्य तृणीकृत्य च संगतान्}
{दातुमिच्छति विप्राय द्रौपदीं योषितां वराम्}


\twolineshloka
{अवरोप्येह वृक्षस्तु फलकाले निपात्यते}
{निहन्मैनं दुरात्मानं योयमस्मान्न मन्यते}


\twolineshloka
{न ह्यर्हत्येष संमानं नापि वृद्धक्रमं गुणैः}
{हन्मैनं सह पुत्रेण दुराचारं नृपद्विषम्}


\twolineshloka
{अयं हि सर्वानाहूय सत्कृत्य च नराधिपान्}
{गुणवद्भोजयित्वान्नं ततः पश्चान्न मन्यते}


\twolineshloka
{अस्मिन्राजसमवाये देवानामिव सन्नये}
{किमयं सदृशं कंचिन्नृपतिं नैव दृष्टवान्}


\twolineshloka
{न च विप्रेष्वधीकारो विद्यते वरणं प्रति}
{स्वयंवरः क्षत्रियाणामितीयं प्रथिता श्रुतिः}


\twolineshloka
{अथवा यदि कन्येयं न च कंचिद्बुभूषति}
{अग्नावेनां परिक्षिप्य याम राष्ट्राणि पार्थिवाः}


\twolineshloka
{ब्राह्मणो यदि चापल्याल्लोभाद्वा कृतवानिदम्}
{विप्रियं पार्थिवेन्द्राणां नैष वध्यः कथंचन}


\twolineshloka
{ब्राह्मणार्थं हि नो राज्यं जीवितं हि वसूनि च}
{पुत्रपौत्रं च यच्चान्यदस्माकं विद्यते धनम्}


\twolineshloka
{अवमानभयाच्चैव स्वधर्मस्य च रक्षणात्}
{स्वयंवराणामन्येषां मा भूदेवंविधा गतिः}


\twolineshloka
{इत्युक्त्वा राजशार्दूला रुष्टाः परिघबाहवः}
{द्रुपदं तु जिघांसन्तः सायुधाः समुपाद्रवन्}


\twolineshloka
{तान्गृहीतशरावापान्क्रुद्धानापततो बहून्}
{द्रुपदो वीक्ष्य संग्रासाद्ब्राह्मणाञ्छरणं गतः}


\twolineshloka
{`न भयान्नापि कार्पण्यान्न प्राणपरिरक्षणात्}
{जगाम द्रुपदो विप्राञ्शमार्थी प्रत्यपद्यत ॥'}


\twolineshloka
{वेगेनापततस्तांस्तु प्रभिन्नानिव वारणान्}
{पाण्डुपुत्रौ महेष्वासौ प्रतियातावरिन्दमौ}


\twolineshloka
{ततः समुत्पेतुरुदायुधास्तेमहीक्षितो बद्धगोधाङ्गुलित्राः}
{जिघांसमानाः कुरुराजपुत्रा-वमर्षयन्तोऽर्जुनभीमसेनौ}


\twolineshloka
{ततस्तु भीमोऽद्भुतभीमकर्मामहाबलो वज्रसमानसारः}
{उत्पाट्य दोर्भ्यां द्रुममेकवीरोनिष्पत्रयामास यथा गजेन्द्रः}


\twolineshloka
{तं वृक्षमादाय रिपुप्रमाथीदण्डीव दण्डं पितृराज उग्रम्}
{तस्थौ समीपे पुरुषर्षभस्यपार्थस्य पार्थः पृथुदीर्घबाहुः}


\twolineshloka
{तत्प्रेक्ष्य कर्मातिमनुष्यबुद्धि-र्जिष्णुः स हि भ्रातुरचिन्त्यकर्मा}
{विसिष्मिये चापि भयं विहायतस्थौ धनुर्गृह्य महेन्द्रकर्मा}


\twolineshloka
{तत्प्रेक्ष्य कर्मातिमनुष्यबुद्धि-र्जिष्णोः सहभ्रातुरचिन्त्यकर्मा}
{दामोदरो भ्रातरमुग्रवीर्यंहलायुधं वाक्यमिदं बभाषे}


\twolineshloka
{य एष सिंहर्षभखेलगामीमदद्धनुः कर्षति तालमात्रम्}
{एषोऽर्जुनो नात्र विचार्यमस्तियद्यस्मि संकर्षण वासुदेवः}


\twolineshloka
{यस्त्वेष वृक्षं तरसाऽवभज्यराज्ञां निकारे सहसा प्रवृत्तः}
{वृकोदरान्नान्य इहैतदद्यकर्तुं समर्थः समरे पृथिव्याम्}


\twolineshloka
{योऽसौ पुरस्तात्कमलायताक्षोमहातनुः सिंहगतिर्विनीतः}
{गौरः प्रलम्बोज्ज्वलचारुघोणोविनिःसृतः सोऽप्युत धर्मपुत्रः}


\threelineshloka
{यौ तौ कुमाराविव कार्तिकेयौद्वावाश्विनेयाविति मे वितर्कः}
{मुक्ता हि तस्माज्जतुवेश्मदाहा-न्मया श्रुताः पाण्डुसुताः पृथा च ॥वैशंपायन उवाच}
{}


\twolineshloka
{तमब्रवीन्निर्जलतोयदाभोहलायुधोऽनन्तरजं प्रतीतः}
{प्रीतोऽस्मि दृष्ट्वा हि पितृष्वसारंपृथां विमुक्तां सह कौरवाग्र्यैः}


\chapter{अध्यायः २०५}
\twolineshloka
{वैशंपायन उवाच}
{}


\twolineshloka
{अजिनानि विधुन्वन्तः करकाश्च द्विजर्षभाः}
{ऊचुस्ते भीर्न कर्तव्या वयं योत्स्यामहे परान्}


\twolineshloka
{तानेवं वदतो विप्रानर्जुनः प्रहसन्निव}
{उवाच प्रेक्षका भूत्वा यूयं तिष्ठत पार्श्वतः}


\twolineshloka
{अहमेनानजिह्माग्रैः शतशो विकिरञ्छरैः}
{वारयिष्यामि संक्रुद्धान्मन्त्रैराशीविषानिव}


\twolineshloka
{इति तद्धनुरानम्य शुल्कावाप्तं महाबलः}
{भ्रात्रा भीमेन सहितस्तस्थौ गिरिरिवाचलः}


\twolineshloka
{ततः कर्णमुखान्दृष्ट्वा क्षत्रियान्युद्धदुर्मदान्}
{संपेततुरभीतौ तौ गजौ प्रतिगजानिव}


\twolineshloka
{ऊचुश्च वाचः परुषास्ते राजानो युयुत्सवः}
{आहवे हि द्विजस्यापि वधो दृष्टो युयुत्सतः}


\twolineshloka
{इत्येवमुक्त्वा राजानः सहसा दुद्रुवुर्द्विजान्}
{ततः कर्णो महातेजा जिष्णुं प्रति ययौ रणे}


\twolineshloka
{युद्धार्थी वासिताहेतोर्गजः प्रतिगजं यथा}
{भीमसेनं ययौ शल्यो मद्राणामीश्वरो बली}


\twolineshloka
{दुर्योधनादयः सर्वे ब्राह्मणैः सह संगताः}
{मृदुपूर्वमयत्नेन प्रत्ययुध्यंस्तदाऽऽहवे}


\twolineshloka
{ततोऽर्जुनः प्रत्यविध्यदापतन्तं शितैः शरैः}
{कर्णं वैकर्तनं श्रीमान्विकृष्य बलवद्धनुः}


\twolineshloka
{तेषां शराणां वेगेन शितानां तिग्मतेजसाम्}
{विमुह्यमानो राधेयो यत्नात्तमनुधावति}


\twolineshloka
{तावुभावप्यनिर्देश्यौ लाघवाज्जयतां वरौ}
{अयुध्येतां सुसंरब्धावन्योन्यविजिगीषिणौ}


\twolineshloka
{कृते प्रतिकृतं पश्य पश्य बाहुबलं च मे}
{इति शूरार्थवचनैरभाषेतां परस्परम्}


\twolineshloka
{ततोऽर्जुनस्य भुजयोर्वीर्यमप्रतिमं भुवि}
{ज्ञात्वा वैकर्तनः कर्णः संरब्धः समयोधयत्}


\threelineshloka
{अर्जुनेन प्रयुक्तांस्तान्बाणान्वेगवतस्तदा}
{प्रतिहत्य ननादोच्चैः सैन्यानि तदपूजयन् ॥कर्ण उवाच}
{}


\twolineshloka
{तुष्यामि ते विप्रमुख्य भुजवीयर्स्य संयुगे}
{अविषादस्य चैवास्य शस्त्रास्त्रविजयस्य च}


\twolineshloka
{किं त्वं साक्षाद्धनुर्वेदो रामो वा विप्रसत्तम}
{अथ साक्षाद्धरिहयः साक्षाद्वा विष्णुरच्युतः}


\twolineshloka
{आत्मप्रच्छादनार्थं वै बाहुवीर्यमुपाश्रितः}
{विप्ररूपं विधायेदं मन्ये मां प्रतियुध्यसे}


\twolineshloka
{न हि मामाहवे क्रुद्धमन्यः साक्षाच्छचीपतेः}
{पुमान्योधयितुं शक्तः पाण्डवाद्वा किरीटिनः}


\twolineshloka
{`दग्धा जतुगृहे सर्वे पाण्डवाः सार्जुनास्तदा}
{विनार्जुनं वा समरे मां निहन्तुमशक्नुवन् ॥'}


\twolineshloka
{तमेवंवादिनं तत्र फाल्गुनः प्रत्यभाषत}
{नास्मि कर्ण धनुर्वेदो नास्मि रामः प्रतापवान्}


\twolineshloka
{ब्राह्मणोऽस्मि युधां श्रेष्ठः सर्वशस्त्रभृतां वरः}
{ब्राह्मे पौरन्दरे चास्त्रे निष्ठितोगुरुशासनात्}


\threelineshloka
{स्थितोऽस्म्यद्य रणे जेतुं त्वां वै वीर स्थिरो भव}
{`निर्जितोऽस्मीति वा ब्रूहि ततो व्रज यथासुखम् ॥वैशंपायन उवाच}
{}


\twolineshloka
{एवमुक्त्वाऽथ कर्णस्य धनुश्चिच्छेद पाण्डवः}
{ततोऽन्यद्धनुरादाय संयोद्धुं सन्दधे शरम्}


\twolineshloka
{दृष्ट्वा तच्चापि कौन्तेयश्छित्वा तद्धनुराशुगैः}
{तथा वैकर्तनं कर्णं बिभेद समरेऽर्जुनः}


\twolineshloka
{ततः कर्णस्तु राधेयः छिन्नछन्वा महाबलः}
{शरैरतीव विद्धाङ्गः पलायनमथाकरोत्}


\twolineshloka
{पुनरायान्मुहूर्तेन गृहीत्वा सशरं धनुः}
{ववर्ष शरवर्षाणि पार्थं वैकर्तनस्तथा}


\twolineshloka
{तानि वै शरजालानि कौन्तेयोऽभ्यहनच्छरैः}
{ज्ञात्वा सर्वाञ्शरान्घोरान्कर्णोऽदावद्द्रुतं बहिः ॥'}


\twolineshloka
{ब्राह्मं तेजस्तदाऽजय्यं मन्यमानो महारथः}
{अपरस्मिन्वनोद्देशे वीरौ शल्यवृकोदरौ}


\twolineshloka
{बलिनौ युद्धसंपन्नौ विद्यया च बलेन च}
{अन्योन्यमाह्वयन्तौ तु मत्ताविव महागजौ}


\twolineshloka
{मुष्टिभिर्जानुभिश्चैव निघ्नन्तावितरेतरम्}
{प्रकर्षणाकर्षणयोरभ्याकर्षविकर्षणैः}


\twolineshloka
{आचकर्षतुरन्योन्यं मुष्टिभिश्चापि जघ्नतुः}
{ततश्चटचटाशब्दः सुघोरो ह्यभवत्तयोः}


\twolineshloka
{पाषाणसंपातनिभैः प्रहारैरभिजघ्नतुः}
{मुहूर्तं तौ तदाऽन्योन्यं समरे पर्यकर्षताम्}


\twolineshloka
{ततो भीमः समुत्क्षिप्य बाहुभ्यां शल्यमाहवे}
{अपातयत्कुरुश्रेष्ठो ब्राह्मणा जहसुस्तदा}


\twolineshloka
{तत्राश्चर्यं भीमसेनश्चकार पुरुषर्षभः}
{यच्छल्यं पातितं भूमौ नावधीद्बलिनं बली}


\twolineshloka
{पातिते भीमसेनेन शल्ये कर्णे च शङ्किते}
{`विस्मयः परमो जज्ञे सर्वेषां पश्यतां नृणाम्}


\twolineshloka
{ततो राजसमूहस्य पश्यतो वृक्षमारुजत्}
{ततस्तु भीमं संज्ञाभिर्वारयामास धर्मराट्}


\twolineshloka
{आकारज्ञस्तथा भ्रातुः पाण्डवोऽपि न्यवर्तत}
{धर्मराजश्च कौरव्य दुर्योधनममर्षणम्}


\twolineshloka
{अयोधयत्सभामध्ये पश्यतां वै महीक्षिताम्}
{ततो दुर्योधनस्तं तु ह्यवज्ञाय युधिष्ठिरम्}


\twolineshloka
{नायोधयत्तदा तेन बलवान्वै सुयोधनः}
{एतस्मिन्नन्तरेऽविध्यद्बाणेनानतपर्वणा}


\twolineshloka
{दुर्योधनममित्रघ्नं धर्मराजो युधिष्ठिरः}
{ततो दुर्योधनः क्रुद्धो दण्डाहत इवोरगः}


\twolineshloka
{प्रत्ययुध्यत राजानं यत्नं परममास्थितः}
{छित्त्वा राजा धनुः सज्यं धार्तराष्ट्रस्य संयुगे}


\twolineshloka
{अभ्यवर्षच्छरौघैस्तं स हित्वा प्राद्रवद्रणम्}
{दुःशासनस्तु संक्रुद्धः सहदेवेन पार्थिव}


\threelineshloka
{युद्ध्वा च सुचिरं कालं सहदेवेन निर्जितः}
{उत्सृज्य च धनुः सङ्ख्ये जानुभ्यामवनीं गतः}
{उत्थाय सोऽभिदुद्राव सोसिं जग्राह चर्म च}


\twolineshloka
{विकर्णचित्रसेनाभ्यां निगृहीतश्च कौरवः}
{मा युद्धमिति कौरव्य ब्राह्मणेनाबलेन वै}


\twolineshloka
{दुःसहो नकुलश्चोभौ युद्धं कर्तुं समुद्यतै}
{तौ दृष्ट्वा कौरवा युद्धं वाक्यमूचुर्महाबलौ}


\twolineshloka
{निवर्तन्तां भवन्तो वै कुतो विप्रेषु क्रूरता}
{दुर्बला ब्राह्मणाश्चेमे भवन्तो वै महाबलाः}


\twolineshloka
{द्वावत्र ब्राह्मणौ क्रूरौ वाय्विन्द्रसदृशौ बले}
{ये वा के वा नमस्तेभ्यो गच्छामः स्वपुरं वयम्}


\twolineshloka
{एवं संभाषमाणास्ते न्यवर्तन्ताथ कौरवाः}
{जहृषुर्ब्राह्मणास्तत्र समेतास्तत्र सङ्घशः}


\twolineshloka
{बहुशस्ते ततस्तत्र क्षत्रिया रणमूर्धनि}
{प्रेक्षमाणास्तथाऽतिष्ठन्ब्राह्मणांश्च समन्ततः}


\twolineshloka
{ब्राह्मणाश्च जयं प्राप्ताः कन्यामादाय निर्ययुः}
{विजिते भीमसेनेन शल्ये कर्णे च निर्जिते}


\twolineshloka
{दुर्योधने चापयाते तथा दुःशासने रणात्}
{'शङ्किताः सर्वराजानः परिवव्रुर्वृकोदरम्}


\twolineshloka
{ऊचुश्च सहितास्तत्र साध्विमौ ब्राह्मणर्षभौ}
{विज्ञायेतां क्वजन्मानौ क्वनिवासौ तथैव च}


\twolineshloka
{को हि राधासुतं कर्णं शक्तो योधयितुं रणे}
{अन्यत्र रामाद्द्रोणाद्वा पाण्डवाद्वा किरीटिनः}


\twolineshloka
{कृष्णाद्वा देवकीपुत्रात्कृपाद्वापि शरद्वतः}
{को वा दुर्योधनं शक्तः प्रतियोथयितुं रणे}


\twolineshloka
{तथैव मद्राधिपतिं शल्यं बलवतांवरम्}
{बलदेवादृते वीरात्पाण्डवाद्वा वृकोदरात्}


\twolineshloka
{वीराद्दुर्योधनाद्वाऽन्यः शक्तः पातयितुं रणे}
{क्रियतामवहारोऽस्माद्युद्धाद्ब्राह्मणसंवृतात्}


\threelineshloka
{ब्राह्मणा हि सदा रक्ष्याः सापराधाऽपिनित्यदा}
{अथैनानुपलभ्येह पुनर्योत्स्याम हृष्टवत् ॥वैशंपायन उवाच}
{}


\twolineshloka
{तांस्तथावादिनः सर्वान्प्रसमीक्ष्य क्षितीश्वरान्}
{अथान्यान्पुरुषांश्चापि कृत्वा तत्कर्म संयुगे}


\twolineshloka
{तत्कर्म भीमस्य समीक्ष्य कृष्णःकुन्तीसुतौ तौ परिशङ्कमानः}
{निवारयामास महीपतींस्ता-न्धर्मेण लब्धेत्यनुनीय सर्वान्}


\twolineshloka
{एवं ते विनिवृत्तास्तु युद्धाद्युद्धविशारदाः}
{यथावासं ययुः सर्वे विस्मिता राजसत्तमाः}


\twolineshloka
{वृत्तो ब्रह्मोत्तरो रङ्गः पाञ्चाली ब्राह्मणैर्वृता}
{इति ब्रुवन्तः प्रययुर्ये तत्रासन्समागताः}


\twolineshloka
{ब्राह्मणैस्तु प्रतिच्छन्नौ रौरवाजिनवासिभिः}
{कृच्छ्रेण जग्मतुस्तौ तु भीमसेनधनञ्जयौ}


\twolineshloka
{विमुक्तौ जनसंबाधाच्छत्रुभिः परिविक्षतौ}
{कृष्णयानुगतौ तत्र नृवीरौ तौ विरेजतुः}


\twolineshloka
{पौर्णमास्यां घनैर्मुक्तौ चन्द्रसूर्याविवोदितौ}
{तेषां माता बहुविधं विनाशं पर्यचिन्तयत्}


\twolineshloka
{अनागच्छत्सु पुत्रेषु भैक्षकाले च लिङ्घिते}
{धार्तराष्ट्रैर्हताश्च स्युर्विज्ञाय कुरुपुङ्गवाः}


\twolineshloka
{मायान्वितैर्वा रक्षोभिः सुघोरैर्दृढवैरिभिः}
{विपरीतं मतं जातं व्यासस्यापि महात्मनः}


\twolineshloka
{इत्येवं चिन्तयामासं सुतस्नेहावृता पृथा}
{ततः सुप्तजनप्राये दुर्दिने मेघसंप्लुते}


\twolineshloka
{महत्यथापराह्णे तु घनैः सूर्य इवावृतः}
{ब्राह्मणैः प्राविशत्तत्र जिष्णुर्भार्गववेश्म तत्}


\chapter{अध्यायः २०६}
\twolineshloka
{वैशंपायन उवाच}
{}


\twolineshloka
{गत्वा तु तां भार्गवकर्मशालांपार्थौ पृथां प्राप्य महानुभावौ}
{तां याज्ञसेनीं परमप्रतीतौभिक्षेत्यथावेदयतां नराग्र्यौ}


\twolineshloka
{कुटीगता सा त्वनवेक्ष्य पुत्रौप्रोवाच भुङ्क्तेति समेत्य सर्वे}
{पश्चाच्च कुन्ती प्रसमीक्ष्य कृष्णांकष्टं मया भाषितमित्युवाच}


\threelineshloka
{साऽधर्मभीता परिचिन्तयन्तीतां याज्ञसेनीं परमप्रतीताम्}
{पाणौ गृहीत्वोपजगाम कुन्तीयुधिष्ठिरं वाक्यमुवाच चेदम् ॥कुन्त्युवाच}
{}


\twolineshloka
{इयं तु कन्या द्रुपदस्य राज्ञ-स्तवानुजाभ्यां मयि संनिसृष्टा}
{यथोचितं पुत्र मयाऽपि चोक्तंसमेत्य भुङ्क्तेति नृप प्रमादात्}


\threelineshloka
{मया कथं नानृतमुक्तमद्यभवेत्कुरूणामृषभ ब्रवीहि}
{पञ्चालराजस्य सुतामधर्मोन चोपवर्तेत न विभ्रमेच्च ॥वैशंपायन उवाच}
{}


\twolineshloka
{स एवमुक्तो मतिमान्नृवीरोमात्रा मुहूर्तं तु विचिन्त्य राजा}
{कुन्तीं समाश्वास्य कुरुप्रवीरोधनञ्जयं वाक्यमिदं बभाषे}


\threelineshloka
{त्वया जिता फाल्गुन याज्ञसेनीत्वयैव शोभिष्यति राजपुत्री}
{प्रज्वाल्यतामग्निरमित्रसाहगृहाण पाणिं विधिवत्त्वमस्याः ॥अर्जुन उवाच}
{}


\twolineshloka
{मा मां नरेन्द्र त्वमधर्मभाजंकृथा न धर्मोऽयमशिष्टदृष्टः}
{भवान्निवेश्यः प्रथमं ततोऽयंभीमो महाबाहुरचिन्त्यकर्मा}


\twolineshloka
{अहं ततो नकुलोऽनन्तरं मेपश्चादयं सहदेवस्तरस्वी}
{वृकोदरोऽहं च यमौ च राज-न्नियं च कन्या भवतो नियोज्याः}


\threelineshloka
{एवं गते यत्करणीयमत्रधर्म्यं यशस्यं कुरु तद्विचिन्त्य}
{पाञ्चालराजस्य हितं च यत्स्या-त्प्रशाधि सर्वे स्म वशे स्थितास्ते ॥वैशंपायन उवाच}
{}


\twolineshloka
{जिष्णोर्वचनमाज्ञाय भक्तिस्नेहसमन्वितम्}
{दृष्टिं निवेशयामासुः पाञ्चाल्यां पाण्डुनन्दनाः}


\twolineshloka
{दृष्ट्वा ते तत्र पश्यन्तीं सर्वे कृष्णां यशस्विनीम्}
{संप्रेक्ष्यान्योन्यमासीना हृदयैस्तामधारयन्}


\twolineshloka
{तेषां तु द्रौपदीं दृष्ट्वा सर्वेषाममितौजसाम्}
{संप्रमथ्येन्द्रियग्रामं प्रादुरासीन्मनोभवः}


\twolineshloka
{काम्यं हि रूपं पाञ्चाल्या विधात्रा विहितं स्वयम्}
{बभूवाधिकमन्याभ्यः सर्वभूतमनोहरम्}


\twolineshloka
{तेषामाकारभावज्ञः कुन्तीपुत्रो युधिष्ठिरः}
{द्वैपायनवचः कृत्स्नं सस्मार मनुजर्षभः}


\threelineshloka
{अब्रवीत्सहितान्भ्रातॄन्मिथो भेदभयान्नृपः}
{सर्वेषां द्रौपदी भार्या भविष्यति हि नः शुभा ॥`जनमेजय उवाच}
{}


\threelineshloka
{सताऽपि शक्तेन च केशवेनसज्यं धनुस्तन्न कृतं किमर्थम्}
{विद्धं च लक्ष्यं न च कस्य हेतो-राचक्ष्व तन्मे द्विपदां वरिष्ठ ॥वैशंपायन उवाच}
{}


\twolineshloka
{शक्तेन कृष्णेन च कार्मुकं त-न्नारोपितं ज्ञातुकामेन पार्थान्}
{परिश्रवादेव बभूव लोकेजीवन्ति पार्था इति निश्चयोऽस्य}


\twolineshloka
{अन्यानशक्तान्नृपतीन्समीक्ष्यस्वयंवरे कार्मुकेणोत्तमेन}
{धनञ्जयस्तद्धनुरेकवीरःसज्यं करोतीत्यभिवीक्ष्य कृष्णः}


\twolineshloka
{इति स्वयं वासुदेवो विचिन्त्यपार्थान्विवित्सन्विविधैरुपायैः}
{न तद्धनुः सज्यमियेप कर्तुंबभूवुरस्येष्टतमा हि पार्थाः ॥'}


\twolineshloka
{भ्रातुर्वचस्तत्प्रसमीक्ष्य सर्वेज्येष्ठस्य पाण्डोस्तनयास्तदानीम्}
{तमेवार्थं ध्यायमाना मनोभिःसर्वे च ते तस्थुरदीनसत्वाः}


\twolineshloka
{वृष्णिप्रवीरस्तु कुरुप्रवीरा-नाशंसमानः सहरौहिणेयः}
{जगाम तां भार्गवकर्मशालांयत्रासते ते पुरुषप्रवीराः}


\twolineshloka
{तत्रोपविष्टं पृथुदीर्घबाहुंददर्श कृष्णः सहरौहिणेयः}
{अजातशत्रुं परिवार्य तांश्चा-प्युपोपविष्टाञ्ज्वलनप्रकाशान्}


\twolineshloka
{ततोऽब्रवीद्वासुदेवोऽभिगम्यकुन्तीसुतं धर्मभृतां वरिष्ठम्}
{कृष्णोऽहमस्मीति निपीड्य पादौयुधिष्ठिरस्याजमीढस्य राज्ञः}


\twolineshloka
{तथैव तस्याप्यनु रौहिणेय-स्तौ चापि हृष्टाः कुरवोऽभ्यनन्दन}
{पितृष्वसुश्चापि यदुप्रवीरा-वगृह्णतां भारतमुख्य पादौ}


\twolineshloka
{अजातशत्रुश्च कुरुप्रवीरःपप्रच्छ कृष्णं कुशलं विलोक्य}
{कथं वयं वासुदेव त्वयेहगूढा वसन्तो विदिताश्च सर्वे}


\twolineshloka
{तमब्रवीद्वासुदेवः प्रहस्यगूढोऽप्यग्निर्ज्ञायत एव राजन्}
{तं विक्रमं पाण्डवेयानतीत्यकोऽन्यः कर्ता विद्यते मानुषेषु}


\twolineshloka
{दिष्ट्या सर्वे पावकाद्विप्रमुक्तायूयं घोरात्पाण्डवाः शत्रुसाहाः}
{दिष्ट्या पापो धृतराष्ट्रस्य पुत्रःसहामात्यो न सकामोऽभविष्यत्}


\threelineshloka
{भद्रं वोऽस्तु निहितं यद्गुहायांविवर्धध्वं ज्वलना इवैधमानाः}
{मा वो विद्युः पार्थिवाः केचिदेवयास्यावहे शिबिरायैव तावत्}
{सोऽनुज्ञातः पाण्डवेनाव्ययश्रीःप्रायाच्छीघ्रं बलदेवेन सार्धम्}


\chapter{अध्यायः २०७}
\twolineshloka
{वैशंपायन उवाच}
{}


\twolineshloka
{धृष्टद्युमनस्तु पाञ्चाल्यः पृष्ठतः कुरुनन्दनौ}
{अन्वगच्छत्तदा यान्तौ भार्गवस्य निवेशने}


\twolineshloka
{सोज्ञायमानः पुरुषानवधाय समन्ततः}
{स्वयमारान्निलीनोऽभूद्भार्गवस्य निवेशने}


\twolineshloka
{सायं च भीमस्तु रिपुप्रमाथीजिष्णुर्यमौ चापि महानुभावौ}
{भैक्षं चरित्वा तु युधिष्ठिरायनिवेदयांचक्रुरदीनसत्वाः}


\twolineshloka
{ततस्तु कुन्ती द्रुपदात्मजां ता-मुवाच काले वचनं वदान्या}
{ततोऽग्रमादाय कुरुष्व भद्रेबलिं च विप्राय च देहि भिक्षाम्}


\twolineshloka
{ये चान्नमिच्छन्ति ददस्व तेभ्यःपरिश्रिता ये परितो मनुष्याः}
{ततश्च शेषं प्रविभज्य शीघ्र-मर्धं चतुर्णां मम चात्मनश्च}


\twolineshloka
{अर्धं तु भीमाय च देहि भद्रेय एष नागर्षभतुल्यरूपः}
{गौरो युवा संहननोपपन्नएषो हि वीरो बहुभुक् सदैव}


\twolineshloka
{सा हृष्टरूपैव तु राजपुत्रीतस्या वचः साध्वविशङ्कमाना}
{यथावदुक्तं प्रचकार साध्वीते चापि सर्वे बुभुजुस्तदन्नम्}


\twolineshloka
{कुशैस्तु भूमौ शयनं चकारमाद्रीपुत्रः सहदेवस्तपस्वी}
{अथात्मकीयान्यजिनानि सर्वेसंस्तीर्य वीराः सुषुपुर्धरण्याम्}


\twolineshloka
{अगस्त्यकान्तामभितो दिशं तुशिरांसि तेषां कुरुसत्तमानाम्}
{कुन्ती पुरस्तात्तु बभूव तेषांपादान्तरे चाथ बभूव कृष्णा}


\twolineshloka
{अशेत भूमौ सह पाण्डुपुत्रैःपादोपधानीव कृता कुशेषु}
{न तत्र दुःखं मनसापि तस्यान चावमेने कुरुपुङ्गवांस्तान्}


\twolineshloka
{ते तत्र शूराः कथयांबभूवुःकथा विचित्राः पृतनाधिकाराः}
{अस्त्राणि दिव्यानि रथांश्च नागान्खड्गान्गदाश्चापि परश्वधांश्च}


\twolineshloka
{तेषां कथास्ताः परिकीर्त्यमानाःपाञ्चालराजस्य सुतस्तदानीम्}
{सुश्राव कृष्णां च तदा निषण्णांते चापि सर्वे ददृशुर्मनुष्याः}


\twolineshloka
{धृष्टद्युम्नो राजपुत्रस्तु सर्वंवृत्तं तेषां कथितं चैव रात्रौ}
{सर्वं राज्ञे द्रुपदायाखिलेननिवेदयिष्यंस्त्वरितो जगाम}


\twolineshloka
{पाञ्चालराजस्तु विषण्णरूप-स्तान्पाण्डवानप्रतिविन्दमानः}
{धृष्टद्युम्नं पर्यपृच्छन्महात्माक्व सा गता केन नीता च कृष्णा}


\twolineshloka
{कच्चिन्न शूद्रेण न हीनजेनवैश्येन वा करदेनोपपन्ना}
{कच्चित्पदं मूर्ध्नि न पङ्कदिग्धंकच्चिन्न माला पतिता श्मशाने}


\twolineshloka
{कच्चित्स वर्णप्रवरो मनुष्यउद्रिक्तवर्णोऽप्युत एव कच्चित्}
{कच्चिन्न वामो मम मूर्ध्नि पादःकृष्णाभिमर्शेन कृतोऽद्य पुत्र}


\twolineshloka
{कच्चिन्न तप्स्ये परमप्रतीतःसंयुज्य पार्थेन नरर्षभेण}
{वदस्व तत्त्वेन महानुभावकोऽसौ विजेता दुहितुर्ममाद्य}


\twolineshloka
{विचित्रवीर्यस्य सुतस्य कच्चि-त्कुरुप्रवीरस्य ध्रियन्ति पुत्राः}
{कच्चित्तु पार्थेन यवीयसाऽध्यधनुर्गृहीतं निहतं च लक्ष्यम्}


\chapter{अध्यायः २०८}
\twolineshloka
{वैशंपायन उवाच}
{}


\threelineshloka
{ततस्तथोक्तः परिहृष्टरूपःपित्रे शशंसाथ स राजपुत्रः}
{धृष्टद्युम्नः सोमकानां प्रबर्होवृत्तं यथा येन हृता च कृष्णा ॥धृष्टद्युम्न उवाच}
{}


\twolineshloka
{योऽसौ युवा व्यायतलोहिताक्षःकृष्णाजिनी देवसमानरूपः}
{यः कार्मुकाग्र्यं कृतवानधिज्यंलक्षं च यः पातितवान्पृथिव्याम्}


\twolineshloka
{असज्जमानश्च ततस्तरस्वीवृतो द्विजाग्र्यैरभिपूज्यमानः}
{चक्राम वज्रीव दितेः सुतेषुसर्वैश्च देवैर्ऋषिभिश्च जुष्टः}


\twolineshloka
{कृष्मा प्रगृह्याजिनमन्वयात्तंनागं यथा नागवधूः प्रहृष्टा}
{`श्यामो युवा वारणमत्तगामीकृत्वा महत्कर्म सुदुष्करं तत्}


\twolineshloka
{यः सूतपुत्रेण चकार युद्धंशङ्केऽर्जुनं तं त्रिदशेशवीर्यम्}
{'अमृष्यमाणेषु नराधिपेषुक्रुद्धेषु वै तत्र समापतत्सु}


\twolineshloka
{ततोऽपरः पार्थिवसङ्घमध्येप्रवृद्धमारुज्य महीप्ररोहम्}
{प्राकालयत्तेन स पार्थिवौघान्भीमोऽन्तकः प्राणभृतो यथैव}


\threelineshloka
{तौ पार्थिवानां मिषतां नरेन्द्रकृष्णामुपादाय गतौ नराग्र्यौ}
{`विक्षोभ्य विद्राव्य च पार्तिवांस्ता-न्स्वतेजसा दुष्प्रतिवीक्ष्यरूपौ}
{'विभ्राजमानाविव चन्द्रसूर्यौबाह्यां पुराद्भार्गवकर्मशालाम्}


\twolineshloka
{तत्रोपविष्टार्चिरिवानलस्यतेषां जनित्रीति मम प्रतर्कः}
{तथाविधैरेव नरप्रवीरै-रुपोपविष्टैस्त्रिबिरग्निकल्पैः}


\twolineshloka
{तस्यास्ततस्तावभिवाद्य पादा-वुक्त्वा च कृष्णामभिवादयेति}
{स्थितौ च तत्रैव निवेद्य कृष्णांभिक्षाप्रचाराय गता नराग्र्याः}


\twolineshloka
{तेषां तु भैक्षं प्रतिगृह्य कृष्णादत्वा बलिं ब्राह्मणसाच्च कृत्वा}
{तां चैव वृद्धां परिवेष्य तांश्चनरप्रवीरान्स्वयमप्यभुङ्क्त}


\twolineshloka
{सुप्तास्तु ते पार्थिव सर्व एवकृष्णा च तेषां चरणोपधाने}
{आसीत्पृथिव्यां शयनं च तेषांदर्भाजिनाग्रास्तरणोपपन्नम्}


\twolineshloka
{ते नर्दमाना इव कालमेघाःकथा विचित्राः कथयांबभूवुः}
{न वैश्यशूद्रौपयिकीः कथास्तान च द्विजानां कथयन्ति वीराः}


\twolineshloka
{निःसंशयं क्षत्रियपुंगवास्तेयथा हि युद्धं कथयन्ति राजन्}
{आशा हि नो व्यक्तमियं समृद्धामुक्तान्हि पार्थाञ्शृणुमोऽग्निदाहात्}


\threelineshloka
{यथा हि लक्ष्यं निहतं धनुश्चसज्यं कृतं तेन तथा प्रसह्य}
{यथा हि भाषन्ति परस्परं तेछन्ना ध्रुवं ते प्रचरन्ति पार्थाः ॥ 1-208015x वैशंपायन उवाच}
{}


\twolineshloka
{ततः स राजा द्रुपदः प्रहृष्टःपुरोहितं प्रेषायामास तेषाम्}
{विद्याम युष्मानिति भाषमाणोमहात्मानः पाण्डुसुताः स्थ कच्चित्}


\twolineshloka
{गृहीतवाक्यो नृपतेः पुरोधागत्वा प्रशंसामभिधाय तेषाम्}
{वाक्यं समग्रं नृपतेर्यथाव-दुवाच चानुक्रमविक्रमेण}


\twolineshloka
{विज्ञातुमिच्छत्यवनीश्वरो वःपाञ्चालराजो वरदो वरार्हाः}
{लक्ष्यस्य वेद्धारमिमं हि दृष्ट्वाहर्षस्य नान्तं प्रतिपद्यते सः}


\twolineshloka
{आख्यात च ज्ञातिकुलानुपूर्वीपदं शिरःसु द्विषतां कुरुध्वम्}
{प्रह्लादयध्वं हृदयं ममेदंपाञ्चालराजस्य च सानुगस्य}


\twolineshloka
{पाण्डुर्हि राजा द्रुपदस्य राज्ञःप्रियः सखा चात्मसमो बभूव}
{तस्यैष कामो दुहिता ममेयंस्नुषा यदि स्यादिह कौरवस्य}


\twolineshloka
{अयं हि कामो द्रुपदस्य राज्ञोहृदि स्थितो नित्यमनिन्दिताङ्गाः}
{यदर्जुनो वै पृथुदीर्घबाहु-र्धर्मेण विन्देत सुतां ममैताम्}


\twolineshloka
{कृतं हि तत्स्यात्सुकृतं ममेदंयशश्च पुण्यं च हितं तदेतत्}
{अथोक्तवाक्यं हि पुरोहितं स्थितंततो विनीतं समुदीक्ष्य राजा}


\threelineshloka
{समीपतो भीममिदं शशासप्रदीयतां पाद्यमर्ध्यं तथाऽस्मै}
{मान्यः पुरोधा द्रुपदस्य राज्ञ-स्तस्मै प्रयोज्याऽभ्यधिका हि पूजा ॥वैशंपायन उवाच}
{}


\twolineshloka
{भीमस्ततस्तत्कृतवान्नरेन्द्रतां चैव पूजां प्रतिगृह्य हर्षात्}
{सुखोपविष्टं तु पुरोहितं तदायुधिष्ठिरो ब्राह्मणमित्युवाच}


\twolineshloka
{पाञ्चालराजेन सुता निसृष्टास्वधर्मदृष्टेन यथा न कामात्}
{प्रदिष्टशुंल्का द्रुपदेन राज्ञासा तेन वीरेण तथाऽनुवृत्ता}


\twolineshloka
{न तत्र वर्णेषु कृता विवक्षान चापि शीले न कुले न गोत्रे}
{कृतेन सज्येन हि कार्मुकेणविद्धेन लक्ष्येण हि सा विसृष्टा}


\twolineshloka
{सेयं तथाऽनेन महात्मनेहकृष्णा जिता पार्थिवसङ्घमध्ये}
{नैवं गते सौमकिरद्य राजासन्तापमर्हत्यसुखाय कर्तुम्}


\twolineshloka
{कामश्च योऽसौ द्रुपदस्य राज्ञःस चापि संपत्स्यति पार्थिवस्य}
{संप्राप्यरूपां हि नरेन्द्रकन्या-मिमामहं ब्राह्मण साधु मन्ये}


\twolineshloka
{न तद्धनुर्मन्दबलेन शक्यंमौर्व्या समायोजयितुं तथाहि}
{न चाकृतास्त्रेण न हीनजेनलक्ष्यं तथा पातयितुं हि शक्यम्}


\twolineshloka
{तस्मान्न तापं दुहितुर्निमित्तंपाञ्चालराजोऽर्हति कर्तुमद्य}
{न चापि तत्पातनमन्यथेहकर्तुं हि शक्यं भुवि मानवेन}


\twolineshloka
{एवं ब्रुवत्येव युधिष्ठिरे तुपाञ्चालराजस्य समीपतोऽन्यः}
{तत्राजगामाशु नरो द्वितीयोनिवेदयिष्यन्निह सिद्धमन्नम्}


\chapter{अध्यायः २०९}
\twolineshloka
{दूत उवाच}
{}


\twolineshloka
{जन्यार्थमन्नं द्रुपदेन राज्ञाविवाहहेतोरुपसंस्कृतं च}
{तदाप्नुवध्वं कृतसर्वकार्याःकृष्णा च तत्रैतु चिरं न कार्यम्}


\threelineshloka
{इमे रथाः काञ्चनपद्मचित्राःसदश्वयुक्ता वसुधाधिपार्हाः}
{एतान्समारुह्य परैत सर्वेपाञ्चालराजस्य निवेशनं तत् ॥वैशंपायन उवाच}
{}


\twolineshloka
{ततः प्रयाताः कुरुपुंगवास्तेपुरोहितं तं परियाप्य सर्वे}
{आस्थाय यानानि महान्ति तानिकुन्ती च कृष्णा च सहैकयाने}


\twolineshloka
{`स्त्रीभिः सुगन्धाम्बरमाल्यदानै-र्विभूषिता आभरणैर्विचित्रैः}
{माङ्गल्यगीतध्वनिवाद्यघोषै-र्मनोहरैः पुण्यकृतां वरिष्ठैः}


% Check verse!
संगीयमानाः प्रययुः प्रहृष्टादीपैर्ज्वलद्भिः सहिताश्च विप्रैः
\twolineshloka
{स वै तथोक्तस्तु युधिष्ठिरेणपाञ्चालराजस्य पुरोहितोऽग्र्यः}
{सर्वं यथोक्तं कुरुनन्दनेननिवेदयामास नृपाय गत्वा ॥'}


\twolineshloka
{श्रुत्वा तु वाक्यानि पुरोहितस्ययान्युक्तवान्भारत धर्मराजः}
{जिज्ञासयैवाथ कुरूत्तमानांद्रव्याण्यनेकान्युपसंजहार}


\twolineshloka
{फलानि माल्यानि च संस्कृतानिवर्माणि चर्माणि तथाऽऽसनानि}
{गाश्चैव राजन्नथ चैव रज्जू-र्बीजानि चान्यानि कृषीनिमित्तम्}


\twolineshloka
{अन्येषु शिल्पेषु च यान्यपि स्युःसर्वाणि कृत्यान्यखिलेन तत्र}
{क्रीडानिमित्तान्यपि यानि तत्रसर्वाणि तत्रोपजहार राजा}


\twolineshloka
{वर्माणि चर्माणि च भानुमन्तिखड्गा महान्तोऽश्वरथाश्च चित्राः}
{धनूंषि चाग्र्याणि शराश्च चित्राःशक्त्यृष्टयः काञ्चनभूषणाश्च}


\twolineshloka
{प्रासा भुशुण्ड्यश्च परश्वधाश्चसांग्रामिकं चैव तथैव सर्वम्}
{शय्यासनान्युत्तमवस्तुवन्तितथैव वासो विविधं च तत्र}


\twolineshloka
{कुन्ती तु कृष्णां परिगृह्य साध्वी-मन्तःपुरं द्रुपदस्याविवेश}
{स्त्रियश्च तां कौरवराजपत्नींप्रत्यर्चयामासुरदीनसत्वाः}


\twolineshloka
{तान्सिंहविक्रान्तगतीन्निरीक्ष्यमहर्षभाक्षानजिनोत्तरीयान्}
{गूढोत्तरांसान्भुजगेन्द्रभोग-प्रलम्बबाहून्पुरुषप्रवीरान्}


\twolineshloka
{राजा च राज्ञः सचिवाश्च सर्वेपुत्राश्च राज्ञः सुहृदस्तथैव}
{प्रेष्याश्च सर्वे निखिलेन राज-न्हर्षं समापेतुरतीव तत्र}


\twolineshloka
{ते तत्र वीराः परमासनेषुसपादपीठेष्वविशङ्कमानाः}
{यथानुपूर्व्याद्विविशुर्नराग्र्या-स्तथा महार्हेषु न विस्मयन्तः}


\twolineshloka
{उच्चावचं पार्थिवभोजनीयंपात्रीषु जाम्बूनदराजतीषु}
{दासाश्च दास्यश्च सुमृष्टवेषाःसंभोजकाश्चाप्युपजह्रुरन्नम्}


\twolineshloka
{ते तत्र भुक्त्वा पुरुषप्रवीरायथात्मकामं सुभृशं प्रतीताः}
{उत्क्रम्य सर्वाणि वसूनि राज-न्सांग्रामिकं ते विविशुर्नृवीराः}


\twolineshloka
{तल्लक्षयित्वा द्रुपदस्य पुत्राराजा च सर्वैः सह मन्त्रिमुख्यैः}
{समर्थयामासुरुपेत्य हृष्टाःकुन्तीसुतान्पार्थिवराजपुत्रान्}


\chapter{अध्यायः २१०}
\twolineshloka
{वैशंपायन उवाच}
{}


\twolineshloka
{तत आहूय पाञ्चाल्यो राजपुत्रं युधिष्ठिरम्}
{परिग्रहेण ब्राह्मेण परिगृह्य महाद्युतिः}


\twolineshloka
{पर्यपृच्छददीनात्मा कुन्तीपुत्रं सुवर्चसम्}
{कथं जानीम भवतः क्षत्रियान्ब्राह्मणानुत}


\twolineshloka
{वैश्यान्वा गुणसंपन्नानथ वा शूद्रयोनिजान्}
{मायामास्थाय वा सिद्धांश्चरतः सर्वतोदिशम्}


\twolineshloka
{कृष्णाहेतोरनुप्राप्ता देवाः संदर्शनार्थिनः}
{ब्रवीतु नो भवान्सत्यं सन्देहो ह्यत्र नो महान्}


\twolineshloka
{अपि नः संशयस्यान्ते मनः संतुष्टिमावहेत्}
{अपि नो भागधेयानि शुभानि स्युः परन्तप}


\twolineshloka
{इच्छया ब्रूहि तत्सत्यं सत्यं राजसु शोभते}
{इष्टापूर्तेन च तथा वक्तव्यमनृतं न तु}


\threelineshloka
{श्रुत्वा ह्यमरसङ्काश तव वाक्यमरिंदम}
{ध्रुवं विवाहकरणमास्थास्यामि विधानतः ॥युधिष्ठिर उवाच}
{}


\twolineshloka
{मा राजन्विमना भूस्त्वं पाञ्चाल्य प्रीतिरस्तु ते}
{ईप्सितस्ते ध्रुवः कामः संवृत्तोऽयमसंशयम्}


\twolineshloka
{वयं हि क्षत्रिया राजन्पाण्डोः पुत्रा महात्मनः}
{ज्येष्ठं मां विद्धि कौन्तेयं भीमसेनार्जुनाविमौ}


\twolineshloka
{आभ्यां तव सुता राजन्निर्जिता राजसंसदि}
{यमौ च तत्र कुन्ती च यत्र कृष्मा व्यवस्थिता}


\twolineshloka
{व्येतु ते मानसं दुःखं क्षत्रियाः स्मो नरर्षभ}
{पद्मिनीव सुतेयं ते ह्रदादन्यह्रदं गता}


\threelineshloka
{इति तथ्यं महाराज सर्वमेतद्ब्रवीमि ते}
{भवान्हि गुरुरस्माकं परमं च परायणम् ॥वैशंपायन उवाच}
{}


\twolineshloka
{ततः स द्रुपदो राजा हर्षव्याकुललोचनः}
{प्रतिवक्तुं मुदा युक्तो नाशकत्तं युधिष्टिरम्}


\twolineshloka
{यत्नेन तु स तं हर्षं सन्निगृह्य परंतपः}
{अनुरूपं तदा वाचा प्रत्युवाच युधिष्ठिरम्}


\twolineshloka
{पप्रच्छ चैनं धर्मात्मा यथा ते प्रद्रुताः पुरात्}
{स तस्मै सर्वमाचख्यावानुपूर्व्येण पाण्डवः}


\twolineshloka
{तच्छ्रुत्वा द्रुपदो राजा कुन्तीपुत्रस्य भाषितम्}
{विगर्हयामास तदा धृतराष्ट्रं नरेश्वरम्}


\twolineshloka
{आश्वासयामास च तं कुन्तीपुत्रं युधिष्ठिरम्}
{प्रतिजज्ञे च राज्याय द्रुपदो वदतां वरः}


\twolineshloka
{ततः कुन्ती च कृष्णा च भीमसेनार्जुनावपि}
{यमौ च राज्ञा संदिष्टं विविशुर्भवनं महत्}


\twolineshloka
{तत्र ते न्यवसन्राजन्यज्ञसेनेन पूजिताः}
{प्रत्याश्वस्तस्ततो राजा सह पुत्रैरुवाच तम्}


\threelineshloka
{गृह्णातु विधिवत्पाणिमद्यायं कुरुनन्दनः}
{पुण्येऽहनि महाबाहुरर्जुनः कुरुतां क्षणम् ॥वैशंपायन उवाच}
{}


\twolineshloka
{तमब्रवीत्ततो राजा धर्मात्मा च युधिष्ठिरः}
{`ममापि दारसंबन्धः कार्यस्तावद्विशांपते}


% Check verse!
तस्मात्पूर्वं मया कार्यं तद्भवाननुमन्यताम्

' द्रुपद उवाच

भवान्वा विधिवत्पाणिं गृह्णातु दुहितुर्मम

यस्य वा मन्यसे वीर तस्य कृष्णामुपादिश ॥युधिष्ठिर उवाच


\twolineshloka
{सर्वेषां महिषी राजन्द्रौपदी नो भविष्यति}
{एवं प्रव्याहृतं पूर्वं मम मात्रा विशांपते}


\twolineshloka
{अहं चाप्यनिविष्टो वै भीमसेनश्च पाण्डवः}
{पार्थेन विजिता चैषा रत्नभूता सुता तव}


\twolineshloka
{एष नः समयो राजँल्लब्धस्य सह भोजनम्}
{न च तं हातुमिच्छामः समयं राजसत्तम}


\fourlineindentedshloka
{`अक्रमेण निवेशे च धर्मलोपो महान्भवेत्}
{'सर्वेषां धर्मतः कृष्णा महिषी नो भविष्यति}
{आनुपूर्व्येण सर्वेषां गृह्णातु ज्वलने करान् ॥द्रुपद उवाच}
{}


\twolineshloka
{एकस्य बह्व्यो विहिता महिष्यः कुरुनन्दन}
{नैकस्या बहवः पुंसः श्रूयन्ते पतयः क्वचित्}


\fourlineindentedshloka
{`सोऽयं न लोके वेदे वा जातु धर्मः प्रशस्ते}
{'लोकवेदविरुद्धं त्वं नाधर्मं धर्मविच्छुचिः}
{कर्तुमर्हसि कौन्तेय कस्मात्ते बुद्धिरीदृशी ॥युधिष्ठिर उवाच}
{}


\twolineshloka
{सूक्ष्मो धर्मो महाराज नास्य विद्मो वयं गतिम्}
{पूर्वेषामानुपूर्व्येण यातं वर्त्माऽनुयामहे}


\twolineshloka
{न मे वागनृतं प्राह नाधर्मे धीयते मतिः}
{एवं चैव वदत्यम्बा मम चैतन्मनोगतम्}


\fourlineindentedshloka
{`आश्रमे रुद्रनिर्दिष्टाद्व्यासादेतन्मया श्रुतम्}
{'एष धर्मो ध्रुवो राजंश्चरैनमविचारयन्}
{मा च शङ्का तत्र ते स्यात्कथंचिदपि पार्थिव ॥द्रुपद उवाच}
{}


\threelineshloka
{त्वं च कुन्ती च कौन्तय धृष्टद्युम्नश्च मे सुतः}
{कथयन्त्विति कर्तव्यं श्वः काल्ये करवामहे ॥वैशंपायन उवाच}
{}


\twolineshloka
{ते समेत्य ततः सर्वे कथयन्ति स्म भारत}
{अथ द्वैपायनो राजन्नभ्यागच्छद्यदृच्छया}


\chapter{अध्यायः २११}
\twolineshloka
{वैशंपायन उवाच}
{}


\twolineshloka
{ततस्ते पाण्डवाः सर्वे पाञ्चाल्यश्च महायशाः}
{प्रत्युथाय महात्मानं कृष्णं सर्वेऽभ्यवादयन्}


\twolineshloka
{प्रतिनन्द्य स तां पूजां पृष्ट्वा कुशलमन्ततः}
{आसने काञ्चने शुद्धे निषसाद महामनाः}


\twolineshloka
{अनुज्ञातास्तु ते सर्वे कृष्णेनामिततेजसा}
{आसनेषु महार्हेषु निषेदुर्द्विपदां वराः}


\twolineshloka
{ततो मुहूर्तान्मधुरां वाणीमुच्चार्य पार्षतः}
{पप्रच्छ तं महात्मानं द्रौपद्यर्थं विशांपते}


\threelineshloka
{कथमेका बहूनां स्यान्न च स्याद्धर्मसंकरः}
{एतन्मे भगवान्सर्वं प्रब्रवीतु यथातथम् ॥व्यास उवाच}
{}


\threelineshloka
{अस्मिन्धर्मे विप्रलब्धे लोकवेदविरोधके}
{यस्य यस्य मतं यद्यच्छ्रोतुमिच्छामि तस्य तत् ॥द्रुपद उवाच}
{}


\twolineshloka
{अधर्मोऽयं मम मतो विरुद्धो लोकवेदयोः}
{न ह्येका विद्यते पत्नी बहूनां द्विजसत्तम}


\twolineshloka
{न चाप्याचरितः पूर्वैरयं धर्मो महात्मभिः}
{न चाप्यधर्मो विद्वद्भिश्चरितव्यः कथंचन}


\threelineshloka
{ततोऽहं न करोम्येनं व्यवसायं क्रियां प्रति}
{धर्मः सदैव संदिग्धः प्रतिभाति हि मे त्वयम् ॥धृष्टद्युम्न उवाच}
{}


\twolineshloka
{यवीयसः कथं भार्यां ज्येष्ठो भ्राता द्विजर्षभ}
{ब्रह्मन्समभिवर्तेत सद्वृत्तः संस्तपोधन}


\twolineshloka
{न तु धर्मस्य सूक्ष्मत्वाद्गतिं विद्मः कथंचन}
{अधर्मो धर्म इति वा व्यवसायो न शक्यते}


\threelineshloka
{कर्तुमस्मद्विधैर्ब्रह्मंस्ततोऽयं न व्यवस्यते}
{पञ्चानां महिषी कृष्णा भवत्विति कथंचन ॥युधिष्ठिर उवाच}
{}


\twolineshloka
{न मे वागनृतं प्राह नाधर्मे धीयते मतिः}
{वर्तते हि मनो मेऽत्र नैषोऽधर्मः कथंचन}


\twolineshloka
{श्रूयते हि पुराणेऽपि जटिला नाम गौतमी}
{ऋषीनध्यासितवती सप्त धर्मभृतां वरा}


\twolineshloka
{तथैव मुनिजा वार्क्षी तपोभिर्भावितात्मनः}
{संगताभूद्दश भ्रातॄनकेन्म्नः प्रचेतसः}


\twolineshloka
{गुरोर्हि वचनं प्राहुर्धर्म्यं धर्मज्ञसत्तम}
{गुरूणां चैव सर्वेषां माता परमको गुरुः}


\threelineshloka
{सा चाप्युक्तवती वाचं भैक्षवद्भुज्यतामिति}
{तस्मादेतदहं मन्ये परं धर्मं द्विजोत्तम ॥कुन्त्युवाच}
{}


\fourlineindentedshloka
{एतमेतद्यथा प्राह धर्मचारी युधिष्ठिरः}
{भुज्यतां भ्रातृभिः सार्धमित्यर्जुनमचोदयम्}
{अनृतान्मे भयं तीव्रं मुच्येऽहमनृतात्कथम् ॥व्यास उवाच}
{}


\twolineshloka
{अनृतान्मोक्ष्यसे भद्रे धर्मश्चैव सनातनः}
{ननु वक्ष्यामि सर्वेषां पाञ्चाल शृमु मे स्वयम्}


\threelineshloka
{यथाऽयं विहितो धर्मो यतश्चायं सनातनः}
{यथा च प्राह कौन्तेयस्तथा धर्मो न संशयः ॥वैशंपायन उवाच}
{}


\twolineshloka
{तत उत्थाय भगवान्व्यासो द्वैपायनः प्रभुः}
{करे गृहीत्वा राजानं राजवेश्म समाविशत्}


\twolineshloka
{पाण्डवाश्चापि कुन्ती च धृष्टद्युम्नश्च पार्षतः}
{विविशुस्तेऽपि तत्रैव प्रतीक्षन्ते स्म तावुभौ}


\twolineshloka
{ततो द्वैपायनस्त्समै नरेन्द्राय महात्मने}
{आचख्यौ तद्यथा धर्मो बहूनामेकपत्निता}


\twolineshloka
{`यथा च ते ददुश्चैव राजपुत्र्याः पुरा वरम्}
{धर्माद्यास्तपसा तुष्टाः पञ्चपत्नीत्वमीश्वराः ॥'}


\chapter{अध्यायः २१२}
\twolineshloka
{व्यास उवाच}
{}


\twolineshloka
{मा भूद्राजंस्तव तापो मनस्थःपञ्चानां भार्या दुहिता ममेति}
{मातुरेषा प्रार्थिता स्यात्तदानींपञ्चानां भार्या दुहिता ममेति}


\twolineshloka
{याजोपयाजौ धर्मरतौ तपोभ्यांतौ चक्रतुः पञ्चपतित्वमस्याः}
{तत्पञ्चभिः पाण्डुसुतैरवाप्ताभार्या कृष्णा मोदतां वै कुलं ते}


\twolineshloka
{लोके नान्यो विद्यते त्वद्विशिष्टः सर्वारीणामप्रधृष्योऽसि राजन्}
{भूयस्त्विदं शृणु मे त्वं विशोकोयथाऽऽगतं पञ्चपत्नीत्वमस्याः}


\twolineshloka
{एषा नालायनी पूर्वं मौद्गल्यं स्थविरं पतिम्}
{आराधयामास तदा कुष्ठिनं तमनिन्दिता}


\twolineshloka
{त्वगस्थिभूतं कटुकं लोलमीर्ष्युं सुकोपनम्}
{सुगन्धेतरगन्धाढ्यं वलीपलितमूर्धजम्}


\twolineshloka
{स्थविरं विकृताकारं शीर्यमाणनखत्वचम्}
{उच्छिष्टमुपभुञ्जाना पर्युपास्ते महामुनिम्}


\twolineshloka
{ततः कदाचिदङ्गुष्ठो भुञ्जानस्य व्यशीर्यत}
{अन्नादुद्धृत्य तच्चान्नमुपभुङ्क्तेऽविशङ्किता}


\threelineshloka
{तेन तस्याः प्रसन्नेन कामव्याहारिणा तदा}
{वरं वृणीष्वेत्यसकृदुक्ता वव्रे वरं तदा ॥मौद्गल्य उवाच}
{}


\twolineshloka
{नाहं वृद्धो न कटुको नेर्व्यावान्नैव कोपनः}
{न च दुर्गन्धवदनो न कृशो न च लोलुपः}


\threelineshloka
{कथं त्वां रमयामीह कथं त्वां वासयाम्यहम्}
{वद कल्याणि भद्रं ते यथा त्वं मनसेच्छसि ॥व्यास उवाच}
{}


\threelineshloka
{सा तमक्लिष्टकर्माणं वरदं सर्वकामदम्}
{भर्तारमनवद्याङ्गी प्रसन्नं प्रत्युवाच ह ॥नालायन्युवाच}
{}


\twolineshloka
{पञ्चधा प्रविभक्तात्मा भगवांल्लोकविश्रुतः}
{रमय त्वमचिन्त्यात्मन्पुनश्चैकत्वमागतः}


\twolineshloka
{तां तथेत्यब्रवीद्धीमान्महर्षिर्वै महातपाः}
{स पञ्चधा तु भूत्वा तां रमयामास सर्वतः}


\twolineshloka
{नालायनीं सुकेशान्तां मौद्गल्यश्चारुहासिनीम्}
{आश्रमेष्वधिकं चापि पूज्यमानो महर्षिभिः}


\twolineshloka
{स चचार यथाकामं कामरूपवपुः पुनः}
{यदा ययौ दिवं चापि तत्र देवर्षिभिः सह}


\twolineshloka
{चचार सोऽमृताहारः सुरलोके चचार ह}
{पूज्यमानस्तथा शच्या शक्रस्य भवनेष्वपि}


\twolineshloka
{महेन्द्रसेनया सार्धं पर्यधावद्रिरंसया}
{सूर्यस्य च रथं दिव्यमारुह्य भगवान्प्रभुः}


\twolineshloka
{पर्युपेत्य पुनर्मेरु मेरौ वासमरोचयत्}
{आकाशगङ्गामाप्लुत्य तया सह तपोधनः}


\twolineshloka
{रश्मिजालेषु चन्द्रस्य उवाच च यथाऽनिलः}
{गिरिरूपधरो योगी स महर्षिस्तदा पुनः}


\twolineshloka
{तत्प्रभावेन सा तस्य मध्ये जज्ञे महानदी}
{यदा पुष्पाकुलः सालः संजज्ञे भगवानृषिः}


\twolineshloka
{लतात्वमनुसंपेदे तमेवाभ्यनुवेष्टती}
{पुपोष च वपुर्यस्य तस्य तस्यानुगं पुनः}


\twolineshloka
{सा पुपोष समं भर्त्रा स्कन्धेनापि चचार ह}
{ततस्तस्य च तस्याश्च तुल्या प्रीतिरवर्धत}


\twolineshloka
{तथा सा भगवांस्तस्याः प्रसादादृषिसत्तमः}
{विरराम च सा चैव दैवयोगेन भामिनी}


\twolineshloka
{स च तां तपसा देवीं रमयामास योगतः}
{एकपत्नी तथा भूत्वा सदैवाग्रे यशस्विनी}


\twolineshloka
{अरुन्धतीव सीतेव बभूवातिपतिव्रता}
{दमयन्त्याश्च मातुः स विशेषमधिकं ययौ}


\twolineshloka
{एतत्तथ्यं महाराज मा ते भूद्बुद्धिरन्यथा}
{सा वै नालायनी जज्ञे दैवयोगेन केनचित्}


\twolineshloka
{राजंस्तवात्मजा कृष्णा वेद्यां तेजस्विनी शुभा}
{तस्मिंस्तस्या मनः सक्तं न चचाल कदाचन}


% Check verse!
तथा प्रणिहितो ह्यात्मा तस्यास्तस्मिन्द्विजोत्तमे
\chapter{अध्यायः २१३}
\twolineshloka
{द्रुपद उवाच}
{}


\threelineshloka
{ब्रूहि तत्कारणं येन ब्रह्मन्नालायनी शुभा}
{जाता ममाध्वरे कृष्मा सर्ववेदविदां वर ॥व्यास उवाच}
{}


\twolineshloka
{शृणु राजन्यथा ह्यस्या दत्तो रुद्रेण वै वरः}
{यदर्थं चैव संभूता तव गेहे यशस्विनी}


\twolineshloka
{हन्त त कथयिष्यामि कृष्मायाः पौर्वदेहिकम्}
{इन्द्रसेनेति विख्याता पुरा नालायनी शुभा}


\twolineshloka
{मौद्गल्यस्य पतिमासाद्य चचार विगतज्वरा}
{मौद्गल्यस्य महर्षेश्च रममाणस्य वै तया}


\twolineshloka
{संवत्सरगणा राजन्व्यतीयुः क्षणवत्तदा}
{ततः कदाचिद्धर्मात्मा तृप्तः कामैर्व्यरज्यत}


\fourlineindentedshloka
{अन्विच्छन्परमं धर्मं ब्रह्मयोगपरोऽभवत्}
{उत्ससर्ज स तां विप्रः सा तदा चापतद्भुवि}
{मौद्गल्यो राजशार्दूल तपोभिर्भावितः सदा ॥नालायन्युवाच}
{}


\threelineshloka
{प्रसीद भगवन्मह्यं न मामुत्स्रष्टुमर्हसि}
{अवितृप्तास्मि ब्रह्मर्षे कामानां कामसेवनात् ॥मौद्गल्य उवाच}
{}


\twolineshloka
{यस्मात्त्वं मामनिःशङ्का ह्यवक्तव्यं भाषसे}
{आचरन्ती तपोविघ्नं तस्माच्छृणु वचो मम}


\twolineshloka
{भविष्यसि नृलोके त्वं राजपुत्री यशस्वि}
{पाञ्चालराजस्य सुता द्रुपदस्य महात्मनः}


\threelineshloka
{भवितारस्तत्र तव पतयः पञ्च विप्लुताः}
{तैः सार्धं मधुराकारैश्चिरं रतिमवाप्स्यसि ॥वैशंपायन उवाच}
{}


\twolineshloka
{सैवं शप्ता तु विमना वनं प्रागाद्यशस्विनी}
{भोगैरतृप्ता देवेशं तपसाऽऽराधयत्तदा}


\twolineshloka
{निराशीर्मारुताहारा निराहारा तथैव च}
{अनुवर्तमाना त्वादित्यं तथा पञ्चतपाभवत्}


\twolineshloka
{तीव्रेण तपसा तस्यास्तुष्टः पशुपतिः स्वयम्}
{वचं प्रादात्तदा रुद्रः सर्वलोकेश्वरः प्रभुः}


\twolineshloka
{भविष्यति परं जन्म भविष्यति वराङ्गना}
{भविष्यन्ति परं भद्रे पतयः पञ्च विश्रुताः}


\threelineshloka
{महेन्द्रवपुषः सर्वे महेन्द्रसमविक्रमाः}
{तत्रस्था च महत्कर्म सुराणां त्वं करिष्यसि ॥स्त्र्युवाच}
{}


\threelineshloka
{एकः खलु मया भर्ता वृतः पञ्च त्विमे कथम्}
{एको भवति नैकस्या बहवस्तद्ब्रवीहि मे ॥महेश्वर उवाच}
{}


\threelineshloka
{पञ्चकृत्वस्त्वया चोक्तः पतिं देहीत्यहं पुनः}
{पञ्चते पतयो भद्रे भविष्यन्ति सुखावहाः ॥स्त्र्युवाच}
{}


\twolineshloka
{धर्म एकः पतिः स्त्रीणां पूर्वमे प्रकल्पितः}
{बहुपत्नीकता पुंसो धर्मश्च बहुभिः कृतः}


\twolineshloka
{स्त्रीधर्मः पूर्वमेवायं निर्मितो मुनिभिः सदा}
{सहधर्मचरी भर्तुरेका एकस्य चोच्यते}


\twolineshloka
{एको हि भर्ता नारीणां कौमार इति लौकिकः}
{आपत्सु च नियोगेन भर्तुर्नार्याः परः स्मृतः}


\twolineshloka
{गच्छेत न तृतीयं तु तस्या निष्कृतिरुच्यते}
{चतुर्थे पतिता नारी पञ्चमे वर्धकी भवेत्}


\threelineshloka
{एवं गते धर्मपथे न वृणे बहुपुंस्कताम्}
{अलोकाचरितात्त्समात्कथं मुच्येय सङ्करात् ॥महेश्वर उवाच}
{}


\threelineshloka
{अनावृताः पुरा नार्यो ह्यासञ्शुध्यन्ति चार्तवे}
{सकृदुक्तं त्वया नैतन्नाधर्मस्ते भविष्यति ॥स्त्र्युवाच}
{}


\twolineshloka
{यदि मे पतयः पञ्च रतिमिच्छामि तैर्मिथः}
{कौमारं च भवेत्सर्वैः संगमे संगमे च मे}


\threelineshloka
{पतिशुश्रूषया चैव सिद्धिः प्राप्ता मया पुरा}
{भोगेच्छा च मया प्राप्ता स च भोगश्च मे भवेत् ॥रुद्र उवाच}
{}


\twolineshloka
{रतिश्च भद्रे सिद्धिश्च न भजेते परस्परम्}
{भोगानाप्स्यसि सिद्धिं च योगेन च महत्त्वताम्}


\twolineshloka
{अन्यदेहान्तरे चैव रूपभाग्यगुणान्विता}
{पञ्चभिः प्राप्य कौमारं महाभागा भविष्यसि}


\twolineshloka
{गच्छ गङ्गाजलस्था च नरं पश्यसि यं शुभे}
{तमानय ममाभ्याशं सुरराजं शुचिस्मिते}


\twolineshloka
{इत्युक्ता विश्वरूपेण रुद्रं कृत्वा प्रदक्षिणम्}
{जगाम गङ्गामुद्दिश्य पुण्यां त्रिपथगां नदीम्}


\chapter{अध्यायः २१४}
\twolineshloka
{व्यास उवाच}
{}


\twolineshloka
{पुरा वै नैमिशारण्ये देवाः सत्रमुपासते}
{तत्र वैवस्वतो राजञ्शामित्रमकरोत्तदा}


\twolineshloka
{ततो यमो दीक्षितस्तत्र राज-न्नामारयत्कंचिदपि प्रजानाम्}
{ततः प्रजास्ता बहुला बभूवुःकालातिपातान्मरणप्रहीणाः}


\twolineshloka
{सोमश्च शक्रो वरुणः कुबेरःसाध्या रुद्रा वसवोऽथाश्विनौ च}
{प्रजापतिर्भुवनस्य प्रणेतासमाजग्मुस्तत्र देवास्तथाऽन्ये}


\threelineshloka
{ततोऽब्रुवँल्लोकगुरुं समेताभयात्तीव्रान्मानुषाणां विवृद्ध्या}
{तस्माद्भयादुद्विजन्तः सुखेप्सवःप्रयाम सर्वे शरणं भवन्तम् ॥पितामह उवाच}
{}


\threelineshloka
{किं वो भयं मानुषेभ्यो यूयं सर्वे यदाऽमराः}
{मा वो मर्त्यसकाशाद्वै भयं भवितुमर्हति ॥देवा ऊचुः}
{}


\threelineshloka
{मर्त्या अमर्त्याः संवृत्ता न विशेषोऽस्ति कश्चन}
{अविशेषादुद्विजन्तो विशेषार्थमिहागताः ॥श्रीभगवानुवाच}
{}


\twolineshloka
{वैवस्वतो व्यापृतः सत्रहेतो-स्तेन त्विमे न म्रियन्ते मनुष्याः}
{तस्मिन्नेकाग्रे कृतसर्वकार्येतत एषां भवितैवान्तकालः}


\threelineshloka
{वैवस्वतस्यैव तनुर्विभक्तावीर्येण युष्माकमुत प्रवृद्धा}
{सैषामन्तो भविता ह्यन्तकालेन तत्र वीर्यं भविता नरेषु ॥व्यास उवाच}
{}


\twolineshloka
{ततस्तु ते पूर्वजदेववाक्यंश्रुत्वा जग्मुर्यत्र देवा यजन्ते}
{समासीनास्ते समेता महाबलाभागीरथ्यां ददृशुः पुण्डरीकम्}


\twolineshloka
{दृष्ट्वा च तद्विस्मितास्ते बभूवु-स्तेषामिन्द्रस्तत्र शूरो जगाम}
{सोऽपश्यद्योषामथ पावकप्रभांयत्र देवी गङ्गा सततं प्रसूता}


\twolineshloka
{सा तत्र योषा रुदती जलार्थिनीगङ्गां देवीं व्यवगाह्य व्यतिष्ठत्}
{तस्याश्रुबिन्दुः पतितो जले य-स्तत्पद्ममासीदथ तत्र काञ्चनम्}


\threelineshloka
{तदद्भुतं प्रेक्ष्य वज्री तदानी-मपृच्छत्तां योषितमन्तिकाद्वै}
{का त्वं भद्रे रोदिषि कस्य हेतो-र्वाक्यं तथ्यं कामयेऽहं ब्रवीहि ॥स्त्र्युवाच}
{}


\threelineshloka
{त्वं वेत्स्यसे मामिह याऽस्मि शक्रयदर्थं चाहं रोदिमि मन्दभाग्या}
{आगच्छ राजन्पुरतो गमिष्येद्रष्टाऽसि तद्रोदिमि यत्कृतेऽहम् ॥व्यास उवाच}
{}


\twolineshloka
{तां गच्छन्तीमन्वगच्छत्तदानींसोऽपश्यदारात्तरुणं दर्शनीयम्}
{सिद्धासनस्थं युवतीसहायंक्रीडन्तमक्षैर्गिरिराजमूर्ध्नि}


\twolineshloka
{तमब्रवीद्देवराजो ममेदंत्वं विद्धि विद्वन्भुवनं वशे स्थितम्}
{ईशोऽहस्मीति समन्युरब्रवी-द्दृष्ट्वा तमक्षैः सुभृशं प्रमत्तम्}


\twolineshloka
{क्रुद्धं च शक्रं प्रसमीक्ष्य देवोजहाय शक्रं च शैरुदैक्षत}
{संस्तम्भितोऽभूदथ देवराज-स्तेनेक्षितः स्थाणुरिवावतस्थे}


\twolineshloka
{यदा तु पर्याप्तमिहास्य क्रीडयातदा देवीं रुदतीं तामुवाच}
{आनीयतामेष यतोऽहमारा-न्नैनं दर्पः पुनरप्याविशेत}


\twolineshloka
{ततः शक्रः स्पृष्टमात्रस्तया तुस्रस्तैरङ्गैः पतितोऽभूद्धरण्याम्}
{तमब्रवीद्भगवानुग्रतेजामैवं पुनः शक्र कृथाः कथंचित्}


\twolineshloka
{निवर्तयैनं च महाद्रिराजंबलं च वीर्यं च तवाप्रमेयम्}
{छिद्रस्य चैवाविश मध्यमस्ययत्रासते त्वद्विधाः सूर्यभासः}


\twolineshloka
{स तद्विवृत्य विवरं महागिरे-स्तुल्यद्युतींश्चतुरोऽन्यान्ददर्श}
{स तानभिप्रेक्ष्य बभूव दुःखितःकच्चिन्नाहं भविता वै यथेमे}


\twolineshloka
{ततो देवो गिरिशो वज्रपाणिंविवृत्य नेत्रे कुपितोऽभ्युवाच}
{दरीमेतां प्रविश त्वं शतक्रतोयन्मां बाल्यादवमंस्थाः पुरस्तात्}


\twolineshloka
{उक्तस्त्वेवं विभुना देवराजःप्रावेपताऽऽर्तो भृशमेवाभिषङ्गात्}
{स्रस्तैरङ्गैरनिलेनेव नुन्न-मश्वत्थपत्रं गिरिराजमूर्ध्नि}


\twolineshloka
{स प्राञ्जलिर्वै वृषवाहनेनप्रवेपमानः सहसैवमुक्तः}
{उवाच देवं बहुरूपमुग्र-मद्याशेषस्य भुवनस्य त्वं भवाद्यः}


\twolineshloka
{तमब्रवीदुग्रवर्चाः प्रहस्यनैवंशीलाः शेषमिहाप्नुवन्ति}
{एतेऽप्येवं भवितारः पुरस्ता-त्तस्मादेतां दरीमाविश्य शेष्य}


\twolineshloka
{एषा भार्या भविता वो न संशयोयोनिं सर्वे मानुषीमाविशध्वम्}
{तत्र यूयं कर्म कृत्वाऽविषह्यंबहूनन्यान्निधनं प्रापयित्वा}


\fourlineindentedshloka
{`अस्त्रैर्दिव्यैर्मानुषान्योधयित्वाशूरान्सर्वानाहवे संविजित्य}
{'आगन्तारः पुनरेवेन्द्रवलोकंस्वकर्मणा पूर्वचितं महार्हम्}
{सर्वं मया भाषितमेतदेवंकर्तव्यमन्यद्विविधार्थयुक्तम् ॥पूर्वेन्द्रा ऊचुः}
{}


\threelineshloka
{गमिष्यामो मानुषं देवलोका-द्दुराधरो विहितो यत्र मोक्षः}
{देवास्त्वस्मानादधीरञ्जनन्यांधर्मो वायुर्मघवानश्विनौ च ॥व्यास उवाच}
{}


\twolineshloka
{एतच्छ्रुत्वा वज्रपाणिर्वचस्तुदेवश्रेष्ठं पुनरेवेदमाह}
{वीर्येणाहं पुरुषं कार्यहेतो-र्दद्यामेषां पञ्चमं मत्प्रसूतम्}


\twolineshloka
{विश्वभुग्भूतधामा च शिबिरिन्द्रः प्रतापवान्}
{शान्तिश्चतुर्थस्तेषां वै तेजस्वी पञ्चमः स्मृतः}


\twolineshloka
{तेषां कामं भगवानुग्रधन्वाप्रादादिष्टं सन्निसर्गाद्यथोक्तम्}
{तां चाप्येषां योषितं लोककान्तांश्रियं भार्यां व्यदधान्मानुषेषु}


\twolineshloka
{तैरेव सार्धं तु ततः स देवोजगाम नारायणमप्रमेयम्}
{अनन्तमव्यक्तमजं पुराणंसनातनं विश्वमनन्तरूपम्}


\threelineshloka
{स चापि तद्व्यदधात्सर्वमेवततः सर्वे संबभूवुर्धरण्यमाम्}
{`नरं तु देवं विबुधप्रधान-मिन्द्राज्जिष्णुं पञ्चमं कल्पयित्वा}
{'स चापि केशौ हरिरुद्बबर्हशुक्लमेकमपरं चापि कृष्णम्}


\threelineshloka
{तौ चापि केशौ विशतां यदूनांकुले स्त्रियौ देवकीं रोहिणीं च}
{तयोरेको बलदेवो बभूवयोऽसौ श्वेतस्य देवस्य केशः}
{कृष्णो द्वितीयः केशवः संबभूवकेशो योऽसौ वर्णतः कृष्ण उक्तः}


\twolineshloka
{ये ते पूर्वं शक्ररूपा निबद्धा-स्तस्यां दर्यां पर्वतस्योत्तरस्य}
{इहैव ते पाण्डवा वीर्यवन्तःशक्रस्यांशः पाण्डवः सव्यसाची}


\twolineshloka
{एवमेते पाण्डवाः संबभूवु-र्ये ते राजन्पूर्वमिन्द्रा बभूवुः}
{लक्ष्मीश्चैषां पूर्वमेवोपदिष्टाभार्या यैषा द्रौपदी दिव्यरूपा}


\twolineshloka
{कथं हि स्त्री कर्मणा ते महीतला-त्समुत्तिष्ठेदन्यतो दैवयोगात्}
{यस्या रूपं सोमसूर्यप्रकाशंगन्धश्चास्याः कोशमात्रात्प्रवाति}


\threelineshloka
{इदं चान्यत्प्रीतिपूर्वं नरेन्द्रददानि ते वरमत्यद्भुतं च}
{दिव्यं चक्षुः पश्य कुन्तीसुतांस्त्वंपुण्यैर्दिव्यैः पूर्वदेहैरुपेतान् ॥वैशंपायन उवाच}
{}


\twolineshloka
{ततो व्यासः परमोदारकर्माशुचिर्विप्रस्तपसा तस्य राज्ञः}
{चक्षुर्दिव्यं प्रददौ तांश्च सर्वान्राजाऽपश्यत्पूर्वदेहैर्यथावत्}


\twolineshloka
{ततो दिव्यान्हेमकिरीटमालिनःशक्रप्रख्यान्पावकादित्यवर्णान्}
{बद्धापीडांश्चारुरूपांश्च यूनोव्यूढोरस्कांस्तालमात्रान्ददर्श}


\twolineshloka
{दिव्यैर्वस्त्रैरजोभिः सुगन्धै-र्माल्यैश्चाग्र्यैः शोभमानानतीव}
{साक्षात्त्र्यक्षान्वा वसूंश्चापि रुद्रा-नादित्यान्वा सर्वगुणोपपन्नान्}


\twolineshloka
{तान्पूर्वेन्द्रानभिवीक्ष्याभिरूपा-त्र्शक्रात्मजं चेन्द्ररूपं निशम्य}
{प्रीतो राजा द्रुपदो विस्मितश्चदिव्यां मायां तामवेक्ष्याप्रमेयाम्}


\twolineshloka
{तां चैवाग्र्यां स्त्रियमतिरूपयुक्तांदिव्यां साक्षात्सोमवह्निप्रकाशाम्}
{योग्यां तेषां रूपतेजोयशोभिःपत्नी मत्वा हृष्टवान्पार्थिवेन्द्रः}


\threelineshloka
{स तद्दृष्ट्वा महदाश्चर्यरूपंजग्राह पादौ सत्यवत्याः सुतस्य}
{नैतच्चित्रं परमर्षे त्वयीतिप्रसन्नचेताः स उवाच चैनम् ॥`व्यास उवाच}
{}


\twolineshloka
{इदं चापि पुरावृत्तं तन्निबोध च भूमिम}
{कीर्त्यमानं नृपर्षीणां पूर्वेषां दारकर्मणि}


\twolineshloka
{नितन्तुर्नाम राजर्षिर्बभूव भुवि विश्रुतः}
{तस्य पुत्रा महेष्वासा बभूवुः पञ्च भूमिताः}


\twolineshloka
{साल्वेयः शूरसेनश्च श्रुतसेनश्च वीर्यवान्}
{तिन्दुसारोऽतिसारश्च क्षत्रियाः क्रतुयाजिनः}


\twolineshloka
{नातिचक्रमुरन्योन्यमन्योन्यस्य प्रियंवदाः}
{अनीर्ष्यवो धर्मविदः सौम्याश्चैव प्रियकराः}


\twolineshloka
{एतान्नैतन्तवान्पञ्च शिबिपुत्री स्वयंवरे}
{अवाप स्वपतीन्वीरान्भौमाश्वी मनुजाधिपान्}


\twolineshloka
{वीणेव मधुरारावा गान्धर्वस्वरमूर्च्छिता}
{उत्तमा सर्वनारीणां भौमाश्वी ह्यभवत्तदा}


\twolineshloka
{यस्या नैतन्तवाः पञ्च पतयः क्षत्रियर्षभाः}
{बभूवुः पृथिवीपालाः सर्वैः समुदिता गुणैः}


\twolineshloka
{तेषामेकाभवद्भार्या राज्ञामौशीनरी शुभा}
{भौमाश्वी नाम भद्रं ते तथारूपगुणान्विता}


\twolineshloka
{पञ्चभ्यः पञ्चधा पञ्च दायादान्सा व्यजायत}
{तेभ्यो नैतन्तवेभ्यस्तु राजशार्दूल वै तदा}


\twolineshloka
{पृथगाख्याऽभवत्तेषां भ्रातॄणां पञ्चधा भुवि}
{यथावत्कीर्त्यमानांस्ताञ्छृणु मे राजसत्तम}


\twolineshloka
{साल्वेयाः शूरसेनाश्च श्रुतसेनाश्च पार्थिवाः}
{तिन्दुसारातिसाराश्च वंशा एषां नृपोत्तम}


\fourlineindentedshloka
{एवमेकाऽभवद्भार्या भौमाश्वी भुवि विश्रुता}
{तथैव द्रुपदैषा ते सुता वै देवरूपिणी}
{पञ्चानां विहिता पत्नी कृष्णा पार्षत्यनिन्दिता' ॥व्यास उवाच}
{}


\twolineshloka
{आसीत्तपोवंने काचिदृषेः कन्या महात्मनः}
{नाध्यगच्छत्पतिं सा तु कन्या रूपवती सती}


\twolineshloka
{तोषयामास तपसा सा किलोग्रेण शङ्करम्}
{तामुवाचेश्वरः प्रीतो वृणु काममिति स्वयम्}


\twolineshloka
{सैवमुक्ताऽब्रवीत्कन्या देवं वरदमीश्वरम्}
{पतिं सर्वगुणोपेतमिच्छामीति पुनःपुनः}


\twolineshloka
{ददौ तस्यै स देवेशस्तं वरं प्रीतमानसः}
{पञ्च ते पतयो भद्रे भविष्यन्तीति शङ्करः}


\twolineshloka
{सा प्रसादयती देवमिदं भूयोऽभ्यभाषत}
{एकं पतिं गुणोपेतं त्वत्तोऽर्हामीति शङ्कर}


\twolineshloka
{तां देवदेवः प्रीतात्मा पुनः प्राह शुभं वचः}
{पञ्चकृत्वस्त्वयोक्तोऽहं पतिं देहीति वै पुनः}


\twolineshloka
{तत्तथा भविता भद्रे वचस्तद्भद्रमस्तु ते}
{देहमन्यं गतायास्ते सर्वमेतद्भविष्यति}


\twolineshloka
{द्रुपदैषा हि सा जज्ञे सुता वै देवरूपिणी}
{पञ्चानां विहिता पत्नी कृष्णा पार्षत्यनिन्दिता}


\twolineshloka
{`सैव नालायनी भूत्वा रूपेणाप्रतिमा भुवि}
{मौद्गल्यं पतिमास्थाय शिवाद्वरमभीप्सती}


\twolineshloka
{एतद्देवरहस्यं ते श्रावितं राजसत्तम}
{नाख्यातव्यं कस्यचिद्वै देवगुह्यमिदं यतः ॥'}


\twolineshloka
{स्वर्गश्रीः पाण्डवार्थं तु समुत्पन्ना महामखे}
{सेह तप्त्वा तपो घोरं दुहितृत्वं तवागता}


\twolineshloka
{सैषा देवी रुचिरा देवजुष्टापञ्चानामेका स्वकृतेनेह कर्मणा}
{सृष्टा स्वयं देवपत्नी स्वयंभुवाश्रुत्वा राजन्द्रुपदेष्टं कुरुष्व}


\chapter{अध्यायः २१५}
\twolineshloka
{द्रुपद उवाच}
{}


\twolineshloka
{अश्रुत्वैवं वचनं ते महर्षेमया पूर्वं यतितं संविधातुम्}
{न वै शक्यं विहितस्यापयानंतदेवेदमुपपन्नं विधानम्}


\twolineshloka
{दिष्टस्य ग्रन्थिरनिवर्तनीयःस्वकर्मणा विहितं तेन किंचित्}
{कृतं निमित्तमिह नैकहेतो-स्तदेवेदमुपन्नं विधानम्}


\twolineshloka
{यथैव कृष्णोक्तवती पुरस्ता-न्नैकान्पतीन्मे भगवान्ददातु}
{स चाप्येवं वरमित्यब्रवीत्तांदेवो हि वेत्ता परमं यदत्र}


\threelineshloka
{यदि चैवं विहितः शङ्करेणधर्मोऽधर्मो वा नात्र ममापराधः}
{गृह्णन्त्विमे विधितत्पाणिमस्यायथोपजोषं विहितैषां हि कृष्णा ॥व्यास उवाच}
{}


\twolineshloka
{नायं विधिर्मानुषाणां विवाहेदेवा ह्येते द्रौपदी चापि लक्ष्मीः}
{प्राक्कर्मणः सुकृतात्पाण्डवानांपञ्चानां भार्या देवदेवप्रसादात्}


\threelineshloka
{तेषामेवायं विहितः स्याद्विवाहोयथा ह्येष द्रौपदीपाण्डवानाम्}
{अन्येषां नृणां योषितां चन धर्मः स्यान्मानवोक्तो नरेन्द्र ॥वैशंपायन उवाच}
{}


\threelineshloka
{तत आजग्मतुस्तत्र तौ व्यासद्रुपदावुभौ}
{कुन्ती सपुत्रा यत्रास्ते धृष्टद्युम्नश्च पार्षतः}
{ततो द्वैपायनः कृष्णो युधिष्ठिरमथागमत् ॥'}


\twolineshloka
{ततोऽब्रवीद्भगवान्धर्मराज-मद्यैव पुण्येऽहनि पाण्डवेय}
{पुण्ये पुष्ये योगमुपैति चन्द्रमाःपाणिं कृष्णायास्त्वं गृहाणाद्य पूर्वं}


% Check verse!
`एवमुक्त्वा धर्मराजं भीमादीनप्यभाषत
\threelineshloka
{क्रमेण पुरुषव्याघ्राः पाणिं गृह्णन्तु पाणिभिः}
{एवमेव मया सर्वं दृष्टमेतत्पुराऽनघाः ॥वैशंपायन उवाच}
{}


\threelineshloka
{ततो राजा यज्ञसेनः सपुत्रोजन्यार्थणुक्तं बहु तत्तदग्र्यम्}
{`समर्थयामास महानुभावोहृष्टः सपुत्रः सहबन्धुवर्गः}
{'समानयामास सुतां च कृष्णा-माप्लाव्य रत्नैर्बहुभिर्विभूष्य}


\twolineshloka
{ततस्तु सर्वे सुहृदो नृपस्यसमाजग्मुः सहिता मन्त्रिणश्च}
{द्रष्टुं विवाहं परमप्रतीताद्विजाश्च पौराश्च यथाप्रधानाः}


\twolineshloka
{ततोऽस्य वेश्माग्र्यजनोपशोभितंविस्तीर्णपद्मोत्पलभूषिताजिरम्}
{बलौघरत्नौघविचित्रमाबभौनभो यथा निर्मलतारकान्वितम्}


\twolineshloka
{ततस्तु ते कौरवराजपुत्राविभूषिताः कुण्डलिनो युवानः}
{महार्हवस्त्राम्बरचन्दनोक्षिताःकृताभिषेकाः कृतमङ्गलक्रियाः}


\twolineshloka
{पुरोहितेनाग्निसमानवर्चसासहैव धौम्येन यताविधि प्रभो}
{क्रमेण सर्वे विविशुस्ततः सदोमहर्षभा गोष्ठमिवाभिनन्दिनः}


\twolineshloka
{ततः समाधाय स वेदपरागोजुहाव मन्त्रैर्ज्वलितं हुताशनम्}
{युधिष्ठिरं चाप्युपनीय मन्त्रवि-न्नियोजयामास सहैव कृष्णया}


\twolineshloka
{प्रदक्षिणं तौ प्रगृहीतपाणीसमानयामास स वेदपरागः}
{`विप्रांश्च संतर्प्य युधिष्ठिरो धनै-र्गोभिश्च रत्नैर्विविधैश्च पूर्वम्}


\twolineshloka
{तदा स राजा द्रुपदस्य पुत्रिका-पाणिं प्रजग्राह हुताशनाग्रतः}
{धौम्येन मन्त्रैर्विधिवद्भुतेऽग्नौसहाग्निकल्पैर्ऋषिभिः समेत्य}


\twolineshloka
{ततोऽन्तरिक्षात्कुसुमानि पेतु-र्ववौ च वायुः सुमनोज्ञगन्धः}
{ततोऽभ्यनुज्ञाप्य समाजशोभितंयुधिष्ठिरं राजपुरोहितस्तदा}


\twolineshloka
{विप्रांश्च सर्वान्सुहृदश्च राज्ञःसमेत्य राजानमदीनसत्वम्}
{जगाद भूयोऽपि महानुभावोवचोऽर्थयुक्तं मनुजेश्वरं तम्}


\twolineshloka
{गृह्णन्त्वथान्ये नरदेवकन्या-पाणिं यथावन्नरदेवपुत्राः}
{तमभ्यनन्दद्द्रुपदस्तथा द्विजंतथा कुरुष्वेति तमादिदेश}


\twolineshloka
{पुरोहितस्यानुमतेन राज्ञ-स्ते राजपुत्रा मुदिता बभूवुः}
{क्रमेण चान्ये च नराधिपात्मजावरस्त्रियास्ते जगृहुः करं तदा ॥'}


% Check verse!
अहन्यहन्युत्तमरूपधारिणोमहारथाः कौरववंशवर्धनाः
\twolineshloka
{इदं च तत्राद्भुतरूपमुत्तमंजगाद देवर्षिरतीतमानुषम्}
{महानुभावा किल सा सुमध्यमाबभूव कन्यैव गते गतेऽहनि}


\twolineshloka
{`पतिश्वशुरता ज्येष्ठे पतिदेवरताऽनुजे}
{मध्यमेषु च पाञ्चाल्यास्त्रितयं त्रितयं त्रिषु ॥'}


\twolineshloka
{कृते विवाहे द्रुपदो धनं ददौमहारथेभ्यो बहुरूपमुत्तमम्}
{शतं रथानां वरहेममालिनांचतुर्युजां हेमखलीनमालिनाम्}


\twolineshloka
{शतं गजानामपि पद्मिनां तथाशतं गिरिणामिव हेमशृङ्गिणाम्}
{तथैव दासीशतमग्र्ययौवनंमहार्हवेषाभरणाम्बरस्रजम्}


\twolineshloka
{पृथक्पृथग्दिव्यदृशां पुनर्ददौतदा धनं सौमकिरग्निसाक्षिकम्}
{तथैव वस्त्राणि विभूषणानिप्रभावयुक्तानि महानुभावः}


\twolineshloka
{कृते विवाहे च ततस्तु पाण्डवाःप्रभूतरत्नामुपलभ्य तां श्रियम्}
{विजह्रुरिन्द्रप्रतिमा महाबलाःपुरे तु पाञ्चालनृपस्य तस्य ह}


\chapter{अध्यायः २१६}
\twolineshloka
{वैशंपायन उवाच}
{}


\twolineshloka
{पाण्डवैः सह संयोगं गतस्य द्रुपदस्य ह}
{न बभूव भयं किंचिद्देवेभ्योऽपि कथंचन}


\twolineshloka
{कुन्तीमासाद्य ता नार्यो द्रुपदस्य महात्मनः}
{नाम संकीर्तयन्त्योऽस्या जग्मुः पादौ स्वमूर्धभिः}


\twolineshloka
{कृष्णा च क्षौमसंवीता कृतकौतुकमङ्गला}
{कृताभिवादना श्वश्र्वास्तस्थौ प्रह्वा कृताञ्जलिः}


\twolineshloka
{रूपलक्षणसंपन्नां शीलाचारसमन्विताम्}
{द्रौपदीमवदत्प्रेम्णा पृथाशीर्वचनं स्नुषाम्}


\twolineshloka
{यथेन्द्राणी हरिहये स्वाहा चैव विभावसौ}
{रोहिणी च यथा सोमे दमयन्ती यथा नले}


\twolineshloka
{यथा वैश्रवणे भद्रा वसिष्ठे चाप्यरुन्धती}
{यथा नारायणे लक्ष्मीस्तथा त्वं भव भर्तृषु}


\twolineshloka
{जीवसूर्वीरसूर्भद्रे बहुसौख्यसमन्विता}
{सुभगा भोगसंपन्ना यज्ञपत्नी पतिव्रता}


\twolineshloka
{अतिथीनागतान्साधून्वृद्दान्बालांस्तथा गुरून्}
{पूजयन्त्या यथान्यायं शश्वद्गच्छन्तु ते समाः}


\twolineshloka
{कुरुजाङ्गलमुख्येषु राष्ट्रेषु नगरेषु च}
{अनु त्वमभिषिच्यस्व नृपतिं धर्मवत्सला}


\twolineshloka
{पतिभिर्निर्जितामुर्वीं विक्रमेण महाबलैः}
{कुरु ब्राह्मणसात्सर्वामश्वमेधे महाक्रतौ}


\twolineshloka
{पृथिव्यां यानि रत्नानि गुणवन्ति गुणान्विते}
{तान्याप्नुहि त्वं कल्याणि सुखिनी शरदां शतं}


\threelineshloka
{यथा च त्वाऽभिनन्दामि वध्वद्य क्षौमसंवृताम्}
{तथा भूयोऽभिनन्दिष्ये जातपुत्रां गुणान्वितां ॥वैशंपायन उवाच}
{}


\twolineshloka
{ततस्तु कृतदारेभ्यः पाण्डुभ्यः प्राहिणोद्धरिः}
{वैदूर्यमणिचित्राणि हैमान्याभरणानि च}


\twolineshloka
{वासांसि च महार्हाणि नानादेश्यानि माधवः}
{कम्बलाजिनरत्नानि स्पर्शवन्ति शुभानि च}


\twolineshloka
{शयनासनयानानि विविधानि महान्ति च}
{वैदूर्यवज्रचित्राणि शतशो भाजनानि च}


\twolineshloka
{रूपयौवनदाक्षिण्यैरुपेताश्च स्वलङ्कृताः}
{प्रेष्याः संप्रददौ कृष्णो नानादेश्याः स्वलङ्कृताः}


\twolineshloka
{गजान्विनीतान्भद्रांश्च सदश्वांश्च स्वलङ्कृतान्}
{रथांश्च दान्तान्सौवर्णैः शुभ्रैः पट्टैरलङ्कृतान्}


\twolineshloka
{कोटिशश्च सुवर्णं च तेषामकृतकं तथा}
{वीथीकृतममेयात्मा प्राहिणोन्मधुसूदनः}


\twolineshloka
{तत्सर्वं प्रतिजग्राह धर्मराजो युधिष्ठिरः}
{मुदा परमया युक्तो गोविन्दप्रियकाम्यया}


\chapter{अध्यायः २१७}
\twolineshloka
{वैशंपायन उवाच}
{}


\twolineshloka
{ततो राज्ञां चैरराप्तैः प्रवृत्तिरुपनीयत}
{पाण्डवैरुपसंपन्ना द्रौपदी पतिभिः शुभा}


\twolineshloka
{येन तद्धनुरादाय लक्ष्यं विद्धं महात्मना}
{सोऽर्जुनो जयतां श्रेष्ठो महाबाणधनुर्धरः}


\twolineshloka
{यः शल्यं मद्रराजं वै प्रोत्क्षिप्यापातयद्बली}
{त्रासयामास संक्रुद्धो वृक्षेण पुरुषान्रणे}


\twolineshloka
{न चास्य संभ्रमः कश्चिदासीत्तत्र महात्मनः}
{स भीमो भीमसंस्पर्शः शत्रुसेनाङ्गपातनः}


\twolineshloka
{`योऽसावत्यक्रमीद्युध्यन्युद्धे दुर्योधनं तथा}
{स राजा पाण्डवश्रेष्ठः पुण्यभाग्बुद्धिवर्धनः}


\twolineshloka
{दुर्योधनस्यावरजैर्यौ युध्येतां प्रतीपवत्}
{तौ यमौ वृत्तसंपन्नौ संपन्नबलविक्रमौ ॥'}


\twolineshloka
{ब्रह्मरूपधराञ्श्रुत्वा प्रशान्तान्पाण्डुनन्दनान्}
{कौन्तेयान्मनुजेन्द्राणां विस्मयः समजायत}


\twolineshloka
{`पौरा हि सर्वे राजन्याः समपद्यन्त विस्मिताः}
{'सपुत्रा हि पुरा कुन्ती दग्धा जतुगृहे श्रुता}


\twolineshloka
{`सर्वभूमिपतीनां च राष्ट्राणां च यशस्विनी}
{'पुनर्जातानिव च तांस्तेऽमन्यन्त नराधिपाः}


\twolineshloka
{धिगकुर्वंस्तदा भीष्मं धृतराष्ट्रं च कौरवम्}
{कर्मणाऽतिनृशंसेन पुरोचनकृतेन वै}


\twolineshloka
{धार्मिकान्वृत्तसंपन्नान्मातुः प्रियहिते रतान्}
{यदा तानीदृशान्पार्थानुत्सादयितुमिच्छति}


\threelineshloka
{ततः स्वयंवरे वृत्ते धार्तराष्ट्राश्च भारत}
{मन्त्रयन्ति ततः सर्वे कर्णसौबलदूषिताः ॥शकुनिरुवाच}
{}


\twolineshloka
{कश्चिच्छत्रुः कर्शनीयः पीडनीयस्तथाऽपरः}
{उत्सादनीयाः कौन्तेयाः सर्वक्षत्रस्य मे मताः}


\twolineshloka
{एवं पराजिताः सर्वे यदि यूयं गमिष्यथ}
{अकृत्वा संविदं कांचिन्मनस्तप्स्यत्यसंशयम्}


\twolineshloka
{अयं देशश्च कालश्च पाण्डवाहरणाय नः}
{न चेदेवं करिष्यध्वं लोके हास्या भविष्यथ}


\twolineshloka
{यमेते संश्रिता वस्तुं कामयन्ते च भूमिपम्}
{सोऽल्पवीर्यबलो राजा द्रुपदो वै मतो मम}


\twolineshloka
{यावदेतान्न जानन्ति जीवतो वृष्णिपुङ्गवाः}
{चैद्यश्च पुरुषव्याघ्रः शिशुपालः प्रतापवान्}


\twolineshloka
{एकीभावं गतो राज्ञा द्रुपदेन महात्मना}
{दुराधर्षतरा राजन्भविष्यन्ति न संशयः}


\twolineshloka
{यावच्चञ्चलतां सर्वे प्राप्नुवन्ति नराधिपाः}
{तावदेव व्यवस्यामः पाण्डवानां वधं प्रति}


\twolineshloka
{मुक्ता जतुगृहाद्भीमादाशीविषमुखादिव}
{पुनस्ते यदि मुच्यन्ते महन्नो भयमागतम्}


\twolineshloka
{तेषामिहोपयातानामेषां तु चिरवासिनाम्}
{अन्तरे दुष्करं स्थातुं गजयोर्महतोरिव}


\twolineshloka
{हनध्वं प्रगृहीतानि बलानि बलिनां वराः}
{यावन्नः कुरुसेनायां पतन्ति पतगा इव}


\threelineshloka
{तावत्सर्वाभिसारेण पुरमेतद्विहन्यताम्}
{एतन्मम मतं चैव प्राप्तकालं नरर्षभ ॥वैशंपायन उवाच}
{}


\twolineshloka
{शकुनेर्वचनं श्रुत्वा भाषमाणस्य दुर्मतेः}
{सोमदत्तिरिदं वाक्यं जगाद परमं ततः}


\twolineshloka
{प्रकृतीः सप्त वै ज्ञात्वा आत्मनश्च परस्य च}
{तथा देशं च कालं च षड्विधान्स नयोद्गुणान्}


\twolineshloka
{स्थानं वृद्धिं क्षयं चैव भूमिं मित्राणि विक्रमम्}
{प्रसमीक्ष्याभियुञ्जीत परं व्यसनपीडितम्}


\twolineshloka
{ततोऽहं पाण्डवान्मन्ये मित्रकोशसमन्वितान्}
{बलस्थान्विक्रमस्थांश्च स्वकृतैः प्रकृतिप्रियान्}


\twolineshloka
{वपुषा हि तु भूतानां नेत्राणि हृदयानि च}
{श्रोत्रं मधुरया वाचा रमयत्यर्जुनो नृणाम्}


\twolineshloka
{न तु केवलदैवेन प्रजा भावेन भेजिरे}
{यद्बभूव मनःकान्तं कर्मणा स चकार तत्}


\twolineshloka
{न ह्ययुक्तं न चासक्तं नानृतं न च विप्रियम्}
{भाषितं चारुभाषस्य जज्ञे पार्थस्य भारती}


\twolineshloka
{तानेवंगुणसंपन्नान्संपन्नान्राजलक्षणैः}
{न तान्पश्यामि ये शक्ताः समुच्छेत्तुं यथा बलात्}


\twolineshloka
{प्रभावशक्तिर्विपुला मन्त्रशक्तिश्च पुष्कला}
{तथैवोत्साहशक्तिश्च पार्थेष्वप्यधितिष्ठति}


\twolineshloka
{मौलमित्रबलानां च कालज्ञो वै युधिष्टिरः}
{साम्ना दानेन भेदेन दण्डेनेति युधिष्ठिरः}


\twolineshloka
{अमित्रांश्च ततो जेतुनं न रोषेणेति मे मतिः}
{परिक्रीय धनैः शत्रुं मित्राणि च बलानि च}


\twolineshloka
{मूलं च सुकृतं कृत्वा भुङ्क्ते भूमिं च पाण्डवः}
{अशक्यान्पाण्डवान्मन्ये देवैरपि सवासवैः}


\twolineshloka
{येषामर्थे सदा युक्तौ कृष्णसंकर्षणावुभौ}
{श्रेयश्च यदि मन्यद्वं मन्मतं यदि वा मतम्}


\twolineshloka
{संविदं पाण्डवैः सर्वैः कृत्वा याम यथागतम्}
{गोपुराट्टालकैरुच्चैरुपतल्पशतैरपि}


\twolineshloka
{गुप्तं पुरवरश्रेष्ठमेतदद्भिश्च संवृतम्}
{तृणधान्येन्धरसैस्तथा यन्त्रायुधौषधैः}


\twolineshloka
{युक्तं बहुकवाटैश्च द्रव्यागारसुवेदिकैः}
{भीमोच्छ्रितमहाचक्रं बृहदट्टालसंवृतम्}


\twolineshloka
{दृढप्राकारनिर्यूहं शतघ्नीशतसंकुलम्}
{ऐष्टको दारवो वप्रो मानुषश्चेति यः स्मृतः}


\twolineshloka
{प्राकारकर्तृभिर्वीरैर्नृगर्भस्तत्र पूजितः}
{तदेतन्नरगर्भेण पाण्डरेण विराजते}


\twolineshloka
{सालेनानेकतालेन सर्वतः संवृतं पुरम्}
{अनुरक्ताः प्रकृतयो द्रुपदस्य महात्मनः}


\twolineshloka
{दानमानार्जिताः सर्वे बाह्याभ्यन्तरगाश्च ये}
{प्रतिरुद्धानिमाञ्ज्ञात्वा राजभिर्भीमविक्रमैः}


\twolineshloka
{उपयास्यन्ति दाशार्हाः समुदग्रोच्छ्रितायुधाः}
{तस्मात्संन्धिं वयं कृत्वा धार्तराष्ट्रस्य पाण्डवैः}


\threelineshloka
{स्वराष्ट्रमेव गच्छामो यद्याप्तं वचनं मम}
{एतन्मम मतं सर्वैः क्रियतां यदि रोचते}
{एतद्धि सुकृतं मन्ये क्षेमं चापि महीभिताम्}


\chapter{अध्यायः २१८}
\twolineshloka
{वैशंपायन उवाच}
{}


\twolineshloka
{सौमदत्तेर्वचः श्रुत्वा कर्णो वैकर्तनो वृषा}
{उवाच वचनं काले कालज्ञः सर्वकर्मणाम्}


\twolineshloka
{नीतिपूर्वमिदं सर्वमुक्तं वचनमर्थवत्}
{वचनं नाभ्यसूयामि श्रूयतां यद्वचस्त्विति}


\twolineshloka
{द्वैधीभावो न गन्तव्यः सर्वकर्मसु मानवैः}
{द्विधाभूतेन मनसा अन्यत्कर्म न सिध्यति}


\twolineshloka
{संप्रयाणासनाभ्यां तु कर्शनेन तथैव च}
{नैतच्छक्यं पुरं हर्तुमाक्रन्दश्चाप्यशोभनः}


\twolineshloka
{अवमर्दनकालोऽत्र मतश्चिन्तयतो मम}
{यावन्नो वृष्णयः पार्ष्णिं न गृह्णन्तिरणप्रियाः}


\twolineshloka
{भवन्तश्च तथा हृष्टाः स्वबाहुबलशालिनः}
{प्राकारमवमृद्रन्तु परिघाः पूरयन्त्वपि}


\twolineshloka
{प्रस्रावयन्तु सलिलं क्रियतां विषमं समम्}
{तृणकाष्ठेन महता खातमस्य प्रपूर्यताम्}


\twolineshloka
{घुष्यतां राजमार्गेषु परेषां यो हनिष्यति}
{नागमश्वं पदातिं वा दानमानं स लप्स्यति}


\twolineshloka
{नागे दशसहस्राणि पञ्च चाश्वपदातिषु}
{रथे वै द्विगुणं नागाद्वसु दास्यन्ति पार्थिवाः}


\twolineshloka
{यश्च कामसुखे सक्तो बालश्च स्थविरश्च यः}
{अयुद्धमनसो ये च ते तु तिष्ठन्तु भीरवः}


\twolineshloka
{प्रदरश्च न दातव्यो न गन्तव्यमचोदितैः}
{यशो रक्षत भद्रं वो जेष्यामो वै रिपून्वयम्}


\threelineshloka
{अनुलोमाश्च नो वाताः सततं मृगपक्षिणः}
{अग्नयश्च विराजन्ते शस्त्राणि कवचानि च ॥वैशंपायन उवाच}
{}


\twolineshloka
{ततः कर्णवचः श्रुत्वा धार्तराष्ट्रप्रियैषिणः}
{निर्ययुः पृथिवीपालाश्चालयन्तः परान्रणे}


\twolineshloka
{न हि तेषां मनःसक्तिरिन्द्रियार्थेषु सर्वशः}
{यथा परिरपुघ्नानां प्रसभं युद्ध एव च}


\twolineshloka
{वैकर्तनपुरोव्रातः सैन्धवोर्मिमहास्वनः}
{दुःशासनमहामत्स्यो दुर्योधनमहाग्रहः}


\twolineshloka
{स राजसागरो भीमो भीमघोषप्रदर्शनः}
{अभिदुद्राव वेगेन पुरं तदपसव्यतः}


\twolineshloka
{तदनीकमनाधृष्यं शस्त्राग्निव्यालदीपितम्}
{समुत्कम्पितमाज्ञाय चुक्रुशुर्द्रुपदात्मजाः}


\twolineshloka
{ते मेघसमनिर्घोषैर्बलिनः स्यन्दनोत्तमैः}
{निर्ययुर्नगरद्वारात्त्रासयन्तः परान्र}


\twolineshloka
{धृष्टद्युम्नः शिखण्डी च सुमित्रः प्रियदर्शनः}
{चित्रकेतुः सुकेतुश्च ध्वजकेतुश्च वीर्यवान्}


\twolineshloka
{पुत्रा द्रुपदराजस्य बलवन्तो जयैषिणः}
{द्रुपदस्य महावीर्यः पाण्डरोष्णीषकेतनः}


\twolineshloka
{पाण्डरव्यजनच्छत्रः पाण्डरध्वजवाहनः}
{स पुत्रगणमध्यस्थः शुशुभे राजसत्तमः}


\twolineshloka
{चन्द्रमा ज्योतिषां मध्ये पौर्णमास्यामिवोदितः}
{अथोद्धूतपताकाग्रमजिह्मगतिमव्ययम्}


\twolineshloka
{द्रुपदानीकमायान्तं कुरुसैन्यमभिद्रवत्}
{तयोरुभयतो जज्ञे तेषां तु तुमुलः स्वनः}


\twolineshloka
{बलयोः संप्रसरतोः सरितां स्रोतसोरिव}
{प्रकीर्णरथनागाश्वैस्तान्यनीकानि सर्वशः}


\twolineshloka
{ज्योतींषईव प्रकीर्णानि सर्वतः प्रचकाशिरे}
{उत्कृष्टभेरीनिनदे संप्रवृत्ते महारवे}


\twolineshloka
{अमर्षिता महात्मानः पाण्डवा निर्ययुस्ततः}
{रथांश्च मेघनिर्घोषान्युक्तान्परमवाजिभिः}


\twolineshloka
{धून्वन्तो ध्वजिनः शुभ्रानास्थाय भरतर्षभाः}
{ततः पाण्डुसुतान्दृष्ट्वा रथस्थानात्तकार्मुकान्}


\twolineshloka
{नृपाणामभवत्कम्पो वेपथुर्हृदयेषु च}
{निर्यातेष्वथ पार्थेषु द्रोपदं तद्बलं रणे}


\twolineshloka
{आविशत्परमो हर्षः प्रमोदश्च जयं प्रति}
{सुमुहूर्तं व्यतिकरः सैन्यानामभवद्भृशम्}


\twolineshloka
{ततो द्वन्द्वमयुध्यन्त मृत्युं कृत्वा पुरस्कृतम्}
{जघ्नतुः समरे तस्मिन्सुमित्रप्रियदर्शनौ}


\twolineshloka
{जयद्रथश्च कर्णश्च पश्यतः सव्यसाचिनः}
{अर्जुनः प्रेक्ष्य निहतौ सौमित्रप्रियदर्शनौ}


\twolineshloka
{जयद्रथसुतं तत्र जघान पितुरन्तिके}
{वृषसेनादवरजं सुदामानं धनंजयः}


\twolineshloka
{कर्णपुत्रं महेष्वासं रथनीडादपातयत्}
{तौ सुतौ निहतौ दृष्ट्वा राजसिंहौ तरस्विनौ}


\twolineshloka
{नामृष्येतां महाबाहू प्रहारमिव सद्गजौ}
{तौ जग्मतुरसंभ्रान्तौ फल्गुनस्य रथंप्रति}


\twolineshloka
{प्रतिमुक्ततलत्राणौ शपमानौ परस्परम्}
{सन्निपातस्तयोरासीदतिघोरो महामृधे}


\twolineshloka
{वृत्रशम्बरयोः सङ्क्ये वज्रिणेव महारणे}
{त्रीनश्वाञ्जघ्नतुस्तस्य फल्गुनस्य नर्षभौ}


\twolineshloka
{ततः किलिकिलाशब्दः कुरूणामभवत्तदा}
{तान्हयान्निहतान्दृष्ट्वा भीमसेनः प्रतापवान्}


\twolineshloka
{निमेषान्तरमात्रेण रथमश्वैरयोजयत्}
{उपयातं रथं दृष्ट्वा दुर्योधनपुरःसरौ}


\twolineshloka
{सौबलः सौमदत्तिश्च समेयातां परन्तपौ}
{तैः पञ्चभिरदीनात्मा भीमसेनो महाबलः}


\twolineshloka
{अयुध्यत तदा वीरैरिन्द्रियार्थैरिवेश्वरः}
{तैर्निरुद्धो न संत्रासं जगाम समितिंजयः}


\twolineshloka
{पञ्चभिर्द्विरदैर्मत्तैर्निरुद्ध इव केसरी}
{तस्यैते युगपत्पञ्च पञ्चभिर्निशितैः शरैः}


\twolineshloka
{सारथिं वाजिनश्चैव निन्युर्वैवस्वतक्षयम्}
{हताश्वात्स्यन्दनश्रेष्ठादवरुह्य महारथः}


\twolineshloka
{चचार विविधान्मार्गानसिमुद्यम्य पाण्डवः}
{अश्वस्कन्धेषु चक्रेषु युगेष्वीषासु चैव हि}


\twolineshloka
{व्यचरत्पातयञ्शत्रून्सुपर्ण इव भोगिनः}
{विधनुष्कं विकवचं विरथं च समीक्ष्य तम्}


\twolineshloka
{अभिपेतुर्नव्याघ्रा अर्जुनप्रमुखा रथाः}
{धृष्टद्युम्नः शिखण्डी च यमौ च युधि दुर्जयौ}


\twolineshloka
{तस्मिन्महारथे युद्धे प्रवृत्ते शरवृष्टिभिः}
{रथध्वजपताकाश्च सवर्मन्तरधीयत}


\twolineshloka
{तत्प्रवृत्तं चिरं कालं युद्धं सममिवाभवत्}
{रथेन तान्महाबाहुरर्जुनो व्यधमत्पुनः}


\twolineshloka
{तमापतन्तं दृष्ट्वेव महाबाहुर्धनुर्धरः}
{कर्णोऽस्त्रविदुषां श्रेष्ठो वारयामास सायकैः}


\twolineshloka
{स तेनाभिहतः पार्थो वासविर्वज्रसन्निभान्}
{त्रीञ्शरान्संदधे क्रुद्धो वधात्क्रुद्धस्य पाण्डवः}


\twolineshloka
{तैः शरैराहतं कर्णं ध्वजयष्टिमुपाश्रितम्}
{अपोवाह रथाच्चाशु सूतः परपुरंजयम्}


\twolineshloka
{ततः पराजिते कर्णे धार्तराष्ट्रान्महाभयम्}
{विवेश समुदग्रांश्च पाण्डवान्प्रसमीक्ष्य तु}


\twolineshloka
{तत्प्रकम्पितमत्यर्थं तद्दृष्ट्वा सौबलो बलम्}
{गिरा मधुरया चापि समाश्वासयतासकृत्}


\twolineshloka
{धार्तराष्ट्रैस्ततः सर्वैर्दुर्योधनपुरःसरैः}
{धृतं तत्पुनरेवासीद्बलं पार्थप्रपीडितम्}


\twolineshloka
{ततो दुर्योधनं दृष्ट्वा भीमो भीमपराक्रमः}
{अक्रुध्यत्स महाबाहुरगारं जातुषं स्मरन्}


\twolineshloka
{ततः संग्रामशिरसि ददर्श विपुलद्रुमम्}
{आयामभूतं तिष्ठन्तं स्कन्धपञ्चाशदुन्नतम्}


\twolineshloka
{महास्कन्धं महोत्सेधं शक्रध्वजमिवोच्छ्रितम्}
{तमुत्पाठ्य च पाणिभ्यामुद्यम्य चरणावपि}


\twolineshloka
{अभिपेदे परान्सङ्ख्ये वज्रपाणिरिवासुरान्}
{भीमसेनभयार्तानि फल्गुनाभिहतानि च}


\twolineshloka
{न शेकुस्तान्यनीकानि धार्तराष्ट्राण्युदीक्षितुम्}
{तानि संभ्रान्तयोधानि श्रान्तवाजिगजानि च}


\twolineshloka
{दिशः प्राकालयद्भीमो दिवीवाभ्राणि मारुतः}
{तान्निवृत्तान्निरानन्दान्नरवारणवाजिनः}


\twolineshloka
{नानुसस्रुर्न चाजघ्नुर्नोचुः किंचिच्च दारुणम्}
{स्वमेव शिबिरं जग्मुः क्षत्रियाः शरविक्षताः}


\twolineshloka
{परेऽप्यभिययुर्हृष्टाः पुरं पौरसुखावहाः}
{मुहूर्तमभवद्युद्धं तेषां वै पाण्डवैः सह}


\twolineshloka
{यावत्तद्युद्धमभवन्महद्देवासुरोपमम्}
{तावदेवाभवच्छान्तं निवृत्ता वै महारथाः}


\twolineshloka
{सुव्रतं चक्रिरे सर्वे सुवृतामब्रुवन्वधूम्}
{कृतार्थं द्रुपदं चोचुर्धृष्टद्युम्नं च पार्षतम्}


\twolineshloka
{शकुनिः सिन्धुराजश्च कर्णदुर्योधनावपि}
{तेषां तदाभवद्दुःखं हृदि वाचा तु नाब्रुवन्}


\twolineshloka
{ततः प्रयाता राजानः सर्व एव यथागतम्}
{धार्तराष्ट्रा हि ते सर्वे गता नागपुरं तदा}


\twolineshloka
{प्रागेव पूर्निरोधात्तु पाण्डवैरश्वसादिनः}
{प्रेषिता गच्छतारिष्टानस्मानाख्यात शौरये}


\twolineshloka
{तेऽचिरेणैव कालेन संप्राप्ता यादवीं पुरीम्}
{ऊचुः संकर्षणोपेन्द्रौ वचनं वचनक्षमौ}


\twolineshloka
{कुशलं पाण्डवाः सर्वानाहुः स्मान्धकवृष्णयः}
{आत्मनश्चाहतानाहुर्विमुक्ताञ्जातुषाद्गृहात्}


\twolineshloka
{समाजे द्रौपदीं लब्धामाहू राजीवलोचनाम्}
{आत्मनः सदृशीं सर्वैः शीलवृत्तसमाधिभिः}


\twolineshloka
{तच्छ्रुत्वा वचनं कृष्णस्तानुवाचोत्तरं वचः}
{सर्वमेतदहं जाने वधात्तस्य तु रक्षसः}


\twolineshloka
{तत उद्योजयामास माधवश्चतुरङ्गिणीम्}
{सेनामुपानयत्तूर्णं पाञ्चालनगरीं प्रति}


\twolineshloka
{ततः संकर्षणश्चैव केशवश्च महाबलः}
{यादवैः सह सर्वैश्च पाण्डवानभिजग्मतुः}


\twolineshloka
{पितृष्वसारं संपूज्य नत्वा चैव तु यादवीम्}
{द्रौपदीं भूषणैः शुभ्रैर्भूषयित्वा यथाविधि}


\twolineshloka
{पाण्डवान्हर्षयित्वा तु पूजयामासतुश्च तान्}
{न्यायतः पूजितौ राज्ञा द्रुपदेन महात्मना}


\twolineshloka
{यादवाः पूजिताः सर्वे पाण्डवैश्च महात्मभिः}
{रेमिरे पाण्डवैः सार्धं ते पाञ्चालपुरे तदा}


\chapter{अध्यायः २१९}
\twolineshloka
{वैशंपायन उवाच}
{}


\twolineshloka
{वृत्ते स्वयंवरे चैव राजानः सर्व एव ते}
{यथागतं विप्रजग्मुर्विदित्वा पाण्डवान्वृतान्}


\twolineshloka
{अथ दुर्योधनो राजा विमना भ्रातृभिः सह}
{अश्वत्थाम्ना मातुलेन कर्णेन च कृपेण च}


\twolineshloka
{विनिवृत्तो वृतं दृष्ट्वा द्रौपद्या श्वेतवाहनम्}
{तं तु दुःशासनोऽव्रीडो मन्दंमन्दमिवाब्रवीत्}


\twolineshloka
{यद्यसौ ब्राह्मणो न स्याद्विन्देत द्रौपदीं न सः}
{न हि तं तत्त्वतो राजन्वेद कश्चिद्धनंजयम्}


\twolineshloka
{दैवं च परमं मन्ये पौरुषं चाप्यनर्थकम्}
{धिगस्तु पौरुषं मन्त्रं यद्धरन्तीह पाण्डवाः}


\twolineshloka
{`बध्वा चक्षूंषि नः पार्था राज्ञां च द्रुपदात्मजाम्}
{उद्वाह्य राज्ञां तैर्न्यस्तो वामः पादः पृथासुतैः}


\twolineshloka
{विमुक्ताः कथमेतेन जतुवेश्मविर्भुजः}
{अस्माकं पौरुषं सत्वं बुद्धिश्चापि गता ततः}


\twolineshloka
{वयं हता मातुलाद्य विश्वस्य च पुरोचनम्}
{अदग्ध्वा पाण्डवानेतान्स्वयं दग्धो हुताशने}


\threelineshloka
{मत्तो मातुल मन्येऽहं पाण्डवा बुद्धिमत्तराः}
{तेषां नास्ति भयं मृत्योर्मुक्तानां जतुवेश्मनः ॥वैशंपावयन उवाच}
{}


\twolineshloka
{एवं संभाषमाणास्ते निन्दन्तश्च पुरोचनम्}
{पञ्चपुत्रां किरातीं च विदुरं च महामतिम् ॥'}


% Check verse!
विविशुर्हास्तिनपुरं दीना विगतचेतसः
\twolineshloka
{त्रस्ता विगतसंकल्पा दृष्ट्वा पार्थान्महौजसः}
{मुक्तान्हव्यभुजश्चैव संयुक्तान्द्रुपदेन च}


\twolineshloka
{धृष्टद्युम्नं तु संचिन्त्य तथैव च शिखण्डिनम्}
{द्रुपदस्यात्मजांश्चान्यान्सर्वयुद्धविशारदान्}


\twolineshloka
{विदुरस्त्वथ ताञ्श्रुत्वा द्रौपद्या पाण्डवान्वृतान्}
{व्रीडितान्धार्तराष्ट्रांश्च भग्नदर्पानुपागतान्}


\twolineshloka
{ततः प्रीतमनाः क्षत्ता धृतराष्ट्रं विशांपते}
{उवाच दिष्ट्या कुरवो वर्ध्त इति विस्मितः}


\twolineshloka
{वैचित्रवीर्यस्तु नृपो निशम्य विदुरस्य तत्}
{अब्रवीत्परमप्रीतो दिष्ट्या दिष्ट्येति भारत}


\twolineshloka
{मन्यते स वृतं पुत्रं ज्येष्ठं द्रुपदकन्यया}
{दुर्योधनमविज्ञानात्प्रज्ञाचक्षुर्नरेश्वरः}


\twolineshloka
{अथ त्वाज्ञापयामास द्रौपद्या भूषणं बहु}
{आनीयतां वै कृष्णेति पुत्रं दुर्योधनं तदा}


\twolineshloka
{`अथ स्म पश्चाद्विदुर आचख्यावम्बिकात्मजम्}
{कौरव्या इति सामान्यान्न मन्येथास्तवात्मजान्}


\twolineshloka
{वर्धन्त इति मद्वाक्याद्वर्धिताः पाण्डुनन्दनाः}
{कृष्णया संवृताः पार्था विमुक्ता राजसङ्गरात्}


\twolineshloka
{दिष्ट्या कुशलिनो राजन्पूजिता द्रुपदेन च ॥वैशंपायन उवाच}
{}


\threelineshloka
{एतच्छ्रुत्वा तु वचनं विदुरस्य नराधिपः}
{आकारच्छादनार्थाय दिष्ट्यादिष्ट्येति चाब्रवीत् ॥धृतराष्ट्र उवाच}
{}


\twolineshloka
{एवं विदुर भद्रं ते यदि जीवन्ति पाण्डवाः}
{न ममौ मे तनौ प्रीतिस्त्वद्वाक्यामृतसंभवा}


\twolineshloka
{साध्वाचारतया तेषां संबन्धो द्रुपदेन च}
{बभूव परमश्लाघ्यो दिष्ट्यादिष्ट्येति चाब्रवीत्}


\twolineshloka
{अन्ववाये वसोर्जातः प्रवरे मात्स्यके कुले}
{वृत्तविद्यातपोवृद्धः पार्थिवानां च संमतः}


\twolineshloka
{पुत्राश्चास्य तथा पौत्राः सर्वे सुचरितव्रताः}
{तेषां संबन्धिनश्चान्ये बहवः सुमहाबलाः}


\twolineshloka
{यथैव पाण्डोः पुत्रास्ते ततोऽप्यभ्यधिका मम}
{सेयमभ्यधिकान्येभ्यो वृत्तिर्विदुर मे मता}


\twolineshloka
{या प्रीतिः पाण्डुपुत्रेषु न साऽन्यत्र ममाभिभो}
{नित्योऽयं चिन्तितः क्षत्तः सत्यं सत्येन शपे}


\twolineshloka
{यत्ते कुशलिनो वीराः पाण्डुपुत्रा महारथाः}
{मित्रवन्तोऽभवन्पुत्रा दुर्योधनमुखास्तथा}


\twolineshloka
{मया श्रुतं यदा वह्नेर्दग्धाः पाण्डुसुता इति}
{तदाऽदह्यं दिवारात्रं न भोक्ष्ये न स्वपामि च}


\threelineshloka
{असहायाश्चं मे पुत्रा लूनपक्षा इव द्विजाः}
{तत्त्वतः शृणु मे क्षत्तः सुसहायाः सुता मम}
{अद्य वै स्थिरसाम्राज्यमाचन्द्रार्कं ममाभवत् ॥'}


\threelineshloka
{को हि द्रुपदमासाद्य मित्रं क्षत्तः सबान्धवम्}
{न बुभूषेद्भवेनार्थी गतश्रीरपि पार्थिवः ॥वैशंपायन उवाच}
{}


\threelineshloka
{तं तथा भाषमाणं तु विदुरः प्रत्यभाषत}
{नित्यं भवतु ते बुद्धिरेषा राजञ्छतं समाः}
{इत्युक्त्वा प्रययौ राजन्विदुरः स्वं निवेशनम्}


\twolineshloka
{ततो दुर्योधनश्चापि राधेयश्च विशांपते}
{धृतराष्ट्रमुपागम्य वचोऽब्रूतामिदं तदा}


\twolineshloka
{सन्निधौ विदुरस्य त्वां दोषं वक्तुं न शक्नुवः}
{विविक्तमिति वक्ष्यावः किं तवेदं चिकीर्षितम्}


\twolineshloka
{सपत्नवृद्धिं यत्तात मन्यसे वृद्धिमात्मनः}
{अभिष्टौषि च यत्क्षत्तुः समीपे द्विपदांवर}


\twolineshloka
{अन्यस्मिन्नृप कर्तव्ये त्वमन्यत्कुरुषेऽनघ}
{तेषां बलविघातो हि कर्तव्यस्तात नित्यशः}


\twolineshloka
{ते वयं प्राप्तकालस्य चिकीर्षां मन्त्रयामहे}
{यथा नो न ग्रसेयुस्ते सपुत्रबलबान्धवान्}


\chapter{अध्यायः २२०}
\twolineshloka
{`वैशंपायन उवाच}
{}


\threelineshloka
{दुर्योधनेनैवमुक्तः कर्णेन च विशांपते}
{पुत्रं च सूतपुत्रं च धृतराष्ट्रोऽब्रवीदिदम् ॥ 'धृतराष्ट्र उवाच}
{}


\twolineshloka
{अहमप्येवमेवैतच्चिकीर्षामि यथा युवाम्}
{विवेक्तुं नाहमिच्छामि त्वाकारं विदुरं प्रति}


\twolineshloka
{ततस्तेषां गुणानेव कीर्तयामि विशेषतः}
{नावबुध्येत विदुरो ममाभिप्रायमिङ्गितैः}


\threelineshloka
{यच्च त्वं मन्यसे प्राप्तं तद्ब्रवीहि सुयोधन}
{राधेय मन्यसे यच्च प्राप्तकालं वदाशु मे ॥दुर्योधन उवाच}
{}


\twolineshloka
{अद्य तान्कुशलैर्विप्रैः सुगुप्तैराप्तकारिभिः}
{कुन्तीपुत्रान्भेदयामो माद्रीपुत्रौ च पाण्डवौ}


\twolineshloka
{अथवा द्रुपदो राजा महद्भिर्वित्तसंचयैः}
{पुत्राश्चास्य प्रलोभ्यन्ताममात्याश्चैव सर्वशः}


\twolineshloka
{परित्यजेद्यथा राजा कुन्तीपुत्रं युधिष्ठिरम्}
{अथ तत्रैव वा तेषां निवासं रोचयन्तु ते}


\twolineshloka
{इहैषां दोषवद्वासं वर्णयन्तु पृथक्पृथक्}
{ते भिद्यमानास्तत्रैव मनः कुर्वन्तु पाण्डवाः}


\twolineshloka
{अथवा कुशळाः केचिदुपायनिपुणा नराः}
{इतरेतरतः पार्थान्भेदयन्त्वनुरागतः}


\twolineshloka
{व्युत्थापयन्तु वा कृष्णां बहुत्वात्सुकरं हि तत्}
{अथवा पाण्डवांस्तस्यां भेदयन्तु ततश्च ताम्}


\twolineshloka
{भीमसेनस्य वा राजन्नुपायकुशलैर्नरैः}
{मृत्युर्विधीयतां छन्नैः स हि तेषां बलाधिकः}


\twolineshloka
{तमाश्रित्य हि कौन्तेयः पुरा चास्मान्न मन्यते}
{सहि तीक्ष्णश्च शूरश्च तेषां चैव परायणम्}


\twolineshloka
{तस्मिंस्त्वभिहते राजन्हतोत्साहा हतौजसः}
{यतिष्यन्ते न राज्याय स हि तेषां व्यपाश्रयः}


\twolineshloka
{अजेयो ह्यर्जुनः सङ्ख्ये पृष्ठगोपे वृकोदरे}
{तमृते फाल्गुनो युद्धे राधेयस्य न पादभाक्}


\twolineshloka
{ते जानानास्तु दौर्बल्यं भीमसेनमृते महत्}
{अस्मान्बलवतो ज्ञात्वा न यतिष्यन्ति दुर्बलाः}


\twolineshloka
{इहागतेषु वा तेषु निदेशवशवर्तिषु}
{प्रवर्तिष्यामहे राजन्यथाशास्त्रं निबर्हणम्}


\twolineshloka
{`दर्पं वा वदतां तेषां केचिदत्र मनस्विनः}
{द्रुपदस्यात्मजा राजन्प्रभिद्यन्ते ततः परैः ॥'}


\twolineshloka
{अथवा दर्शनीयाभिः प्रमदाभिर्विलोभ्यताम्}
{एकैकस्तत्र कौन्तेयस्ततः कृष्णा विरज्यताम्}


\twolineshloka
{प्रेष्यतां चैव राधेयस्तेषामागमनाय वै}
{तैस्तैः प्रकारैः सन्नीय पात्यन्तामाप्तकारिभिः}


\twolineshloka
{एतेषामप्युपायानां यस्ते निर्दोषवान्मतः}
{तस्य यप्रोगमातिष्ठ पुरा कालोऽतिवर्तते}


\twolineshloka
{यावद्ध्यकृतविश्वासा द्रुपदे पार्थिवर्षभे}
{तावदेव हि ते शक्या न शक्यास्तु ततः परम्}


\twolineshloka
{एषा मम मतिस्तात निग्रहाय प्रवर्तते}
{साध्वी वा यदि वाऽसाध्वी किं वा राधेय मन्यसे}


\chapter{अध्यायः २२१}
\twolineshloka
{कर्ण उवाच}
{}


\twolineshloka
{दुर्योधन तव प्रज्ञा न सम्यगिति मे मतिः}
{न ह्युपायेन ते शक्याः पाण्डवाः कुरुवर्धन}


\twolineshloka
{पूर्वमेव हि ते सूक्ष्मैरुपायैर्यतितास्त्वया}
{निग्रहीतुं तदा वीर न चैव शकितास्त्वया}


\twolineshloka
{इहैव वर्तमानास्ते समीपे तव पार्थिव}
{अजातपक्षाः शिशवः शकिता नैव बाधितुम्}


\twolineshloka
{जातपक्षा विदेशस्था विवृद्धाः सर्वशोऽद्य ते}
{नोपायसाध्याः कौन्तेया ममैषा मतिरच्युता}


\twolineshloka
{न च ते व्यसनैर्योक्तुं शक्या दिष्टकृतेन च}
{शकिताश्चेप्सवश्चैव पितृपैतामहं पदम्}


\twolineshloka
{परस्परेण भेदश्च नाधातुं तेषु शक्यते}
{एकस्यां ये रताः पत्न्यां न भिद्यन्ते परस्परम्}


\twolineshloka
{न चापि कृष्णा शक्येत तेभ्यो भेदयितुं परैः}
{परिद्यूनान्वृतवती किमुताद्य मृजावतः}


\twolineshloka
{ईप्सितश्च गुणः स्त्रीणामेकस्या बहुभर्तृता}
{तं च प्राप्तवती कृष्णा न सा भेदयितुं क्षमा}


\twolineshloka
{आर्यव्रतश्च पाञ्चाल्यो न स राजा धनप्रियः}
{न संत्यक्ष्यति कौन्तेयान्राज्यदानैरपि ध्रुवम्}


\twolineshloka
{तथाऽस्म पुत्रो गुणवाननुरक्तश्च पाण्डवान्}
{तस्मान्नोपायसाध्यांस्तानहं मन्ये कथंचन}


\twolineshloka
{इदं त्वद्य क्षमं कर्तुमस्माकं पुरुषर्षभ}
{यावन्न कृतमूलास्ते पाण्डवेया विशांपते}


\threelineshloka
{तावत्प्रहरणीयास्ते तत्तुभ्यं तात रोचताम्}
{अस्मत्पक्षो महान्यावद्यावत्पाञ्चालको लघुः}
{तावत्प्रहरणं तेषां क्रियतां मा विचारय}


\twolineshloka
{वाहनानि प्रभूतानि मित्राणि च कुलानि च}
{यावन्न तेषां गान्धारे तावद्विक्रम पार्थिव}


\twolineshloka
{यावच्च राजा पाञ्चाल्यो नोद्यमे कुरुते मनः}
{सह पुत्रैर्महावीर्यैस्तावद्विक्रम पार्थिव}


\twolineshloka
{यावन्नायाति वार्ष्णेयः कर्षन्यादववाहिनीम्}
{राज्यार्थे पाण्डवेयानां पाञ्चाल्यसदनं प्रति}


\twolineshloka
{वसूनि विविधान्भोगान्राज्यमेव च केवलम्}
{नात्याज्यमस्ति कृष्णस्य पाण्डवार्थे कथंचन}


\twolineshloka
{विक्रमेण मही प्राप्ता भरतेन महात्मना}
{विक्रमेण च लोकांस्त्रीञ्जितवान्पाकशासनः}


\twolineshloka
{विक्रमं च प्रशंसन्ति क्षत्रियस्य विशांपते}
{स्वको हि धर्मः शूराणां विक्रमः पार्थिवर्षभ}


\twolineshloka
{ते बलेन वयं राजन्महता चतुरङ्गिणा}
{प्रमथ्य द्रुपदं शीघ्रमानयामेह पाण्डवान्}


\twolineshloka
{न हि साम्ना न दानेन न भदेन च पाण्डवाः}
{शक्याः साधयितुं तस्माद्विक्रमेणैव ताञ्जहि}


\threelineshloka
{तान्विक्रमेण जित्वेमामखिलां भुङ्क्ष्व मेदिनीम्}
{अतो नान्यं प्रपश्यामि कार्योपायं जनाधिप ॥वैशंपायन उवाच}
{}


\twolineshloka
{श्रुत्वा तु राधेयवचो धृतराष्ट्रः प्रतापवान्}
{अभिपूज्य ततः पश्चादिदं वचनमब्रवीत्}


\twolineshloka
{उपपन्नं महाप्राज्ञे कृतास्त्रे सूतनन्दने}
{त्वयि विक्रमसंपन्नमिदं वचनमीदृशम्}


\twolineshloka
{भूय एव तु भीष्मश्च द्रोणो विदुर एव च}
{युवां च कुरुतं बुद्धिं भवेद्या नः सुखोदया}


\twolineshloka
{तत आनाय्य तान्सर्वान्मन्त्रिणः सुमहायशाः}
{धृतराष्ट्रो महाराज मन्त्रयामास वै तदा}


\chapter{अध्यायः २२२}
\twolineshloka
{भीष्म उवाच}
{}


\twolineshloka
{न रोचते विग्रहो मे पाण्डुपुत्रैः कथंचन}
{यथैव धृतराष्ट्रो मे तथा पाण्डुरसंशयम्}


\twolineshloka
{गान्धार्याश्च यथा पुत्रास्तथा कुन्तीसुता मम}
{यथा च मम ते रक्ष्या धृतराष्ट्र तथा तव}


\twolineshloka
{यथा च मम राज्ञश्च तथा दुर्योधनस्य ते}
{तथा कुरूणां सर्वेषामन्येषामपि पार्थिव}


\twolineshloka
{एवं गते विग्रहं तैर्न रोचेसन्धाय वीरैर्दीयतामर्धभूमिः}
{तेषामपीदं प्रपितामहानांराज्यं पितुश्चैव कुरूत्तमानाम्}


\twolineshloka
{दुर्योधन यथा र्जायं त्वमिदं तात पश्यसि}
{मम पैतृकमित्येवं तेऽपि पश्यन्ति पाण्डवाः}


\twolineshloka
{यदि राज्यं न ते प्राप्ताः पाण्डवेया यशस्विनः}
{कुत एव तवापीदं भारतस्यापि कस्यचित्}


\twolineshloka
{अधर्मेण च राज्यं त्वं प्राप्तवान्भरतर्षभ}
{तेऽपि राज्यमनुप्राप्ताः पूर्वमेवेति मे मतिः}


\twolineshloka
{मधुरेणैव राज्यस्य तेषामर्धं प्रदीयताम्}
{एतद्धि पुरुषव्याघ्र हितं सर्वजनस्य च}


\twolineshloka
{अतोऽन्यथा चेत्क्रियते न हितं नो भविष्यति}
{तवाप्यकीर्तिः सकला भविष्यति न संशयः}


\twolineshloka
{कीर्तिरक्षणमातिष्ठ कीर्तिर्हि परमं बलम्}
{नष्टकीर्तेर्मनुष्यस्य जीवितं ह्यफळं स्मृतम्}


\twolineshloka
{यावत्कीर्तिर्मनुष्यस्य न प्रणश्यति कौरव}
{तावज्जीवति गान्धरे नष्टकीर्तिस्तु नश्यति}


\twolineshloka
{तमिमं समुपातिष्ठ धर्मं कुरुकुलोचितम्}
{अनुरूपं महाबाहो पूर्वेषामात्मनः कुरु}


\twolineshloka
{दिष्ट्या ध्रियन्ते पार्था हि दिष्ट्या जीवति सा पृथा}
{दिष्ट्या पुरोचनः पापो न सकामोऽत्ययं गतः}


\twolineshloka
{यदाप्रभृति दग्धास्ते कुन्तिभोजसुतासुताः}
{तदाप्रभृति गान्धारे न शक्नोम्यभिवीक्षितुम्}


\threelineshloka
{लोके प्राणभृतां किंचिच्छ्रुत्वा कुन्तीं तथागताम्}
{न चापि दोषेण तथा लोको मन्येत्पुरोचनम्}
{यथा त्वां पुरुषव्याघ्र लोको दोषेण गच्छति}


\twolineshloka
{तदिदं जीवितं तेषां तव किल्बिषनाशनम्}
{संमन्तव्यं महाराज पाण्डवानां च दर्शनम्}


\twolineshloka
{न चापि तेषां वीराणां जीवतां कुरुनन्दन}
{पित्र्योंशः शक्य आदातुमपि वज्रभृता स्वयं}


\twolineshloka
{ते सर्वेऽवस्थिता धर्मे सर्वे चैवैकचेतसः}
{अधर्मेण निरस्ताश्च तुल्ये राज्ये विशेषतः}


\twolineshloka
{यदि धर्मस्त्वया कार्यो यदि कार्यं प्रियं च मे}
{क्षेमं च यदि कर्तव्यं तेषामर्धं प्रदीयताम्}


\chapter{अध्यायः २२३}
\twolineshloka
{द्रोण उवाच}
{}


\twolineshloka
{मन्त्राय समुपानीतैर्धृतराष्ट्र हितैर्नृप}
{धर्म्यमर्थ्यं यशस्यं च वाच्यमित्यनुशुश्रुम}


\twolineshloka
{ममाप्येषा मतिस्तात या भीष्मस्य महात्मनः}
{संविभज्यास्तु कौन्तेया धर्म एष सनातनः}


\twolineshloka
{प्रेष्यतां द्रुपदायाशु नऱः कश्चित्प्रियंवदः}
{बहुलं रत्नमादाय तेषामर्थाय भारत}


\twolineshloka
{मिथः कृत्यं च तस्मै स आदाय वसु गच्छतु}
{वृद्धिं च परमां ब्रूयात्तत्संयोगोद्भवां तथा}


\twolineshloka
{संप्रीयमाणं त्वां ब्रूयाद्राजन्दुर्योधनं तथा}
{असकृद्द्रुपदे चैव धृष्टद्युम्ने च भारत}


\twolineshloka
{उचितत्वं प्रियत्वं च योगस्यापि च वर्णयेत्}
{पुनःपुनश्च कौन्येयान्माद्रीपुत्रौ च सान्त्वयन्}


\twolineshloka
{हिरण्मयानि शुभ्राणि बहून्याभरणानि च}
{वचनात्तव राजेन्द्र द्रौपद्याः संप्रयच्छतु}


\twolineshloka
{तथा द्रुपदपुत्राणां सर्वेषां भरतर्षभ}
{पाण्डवानां च सर्वेषां कुन्त्या युक्तानि यानि च}


\threelineshloka
{`दत्त्वा तानि महार्हाणि पाण्डवान्संप्रहर्षय}
{'एवं सान्त्वसमायुक्तं द्रुपदं पाण्डवैः सह}
{उक्त्वा सोऽनन्तरं ब्रूयात्तेषामागमनं प्रति}


\twolineshloka
{अनुज्ञातेषु वीरेषु बलं गच्छतु शोभनम्}
{दुःशासनो विकर्णश्चाप्यानेतुं पाण्डवानिह}


\twolineshloka
{ततस्ते पाण्डवाः श्रेष्ठाः पूज्यमानाः सदा त्वया}
{प्रकृतीनामनुमते पदे स्थास्यन्ति पैतृके}


\threelineshloka
{एतत्तव महाराज तेषु पुत्रेषु चैव हि}
{वृत्तमौपयिकं मन्ये भीष्मेण सह भारत ॥कर्ण उवाच}
{}


\twolineshloka
{योजितावर्थमानाभ्यां सर्वकार्येष्वनन्तरौ}
{न मन्त्रयेतां त्वच्छ्रेयः किमद्भुततरं ततः}


\twolineshloka
{दुष्टेन मनसा यो वै प्रच्छन्नेनान्तरात्मना}
{ब्रूयान्नि)श्रेयसं नाम कथं कुर्यात्सतां मतम्}


\twolineshloka
{न मित्राण्यर्थकृच्छ्रेषु श्रेयसे चेतराय वा}
{विधिपूर्वं हि सर्वस्य दुःखं वा यदि वा सुखम्}


\twolineshloka
{कृतप्रज्ञोऽकृतप्रज्ञो बालो वृद्धश्च मानवः}
{ससहायोऽसहायश्च सर्वं सर्वत्र विन्दति}


\twolineshloka
{श्रूयते हि पुरा कश्चिदम्बुवीच इतीश्वरः}
{आसीद्राजगृहे राजा मागधानां महीक्षिताम्}


\twolineshloka
{स हीनः करणैः सर्वैरुच्छ्वासपरमो नृपः}
{अमात्यसंस्थः सर्वेषु कार्येष्वेवाभवत्तदा}


\twolineshloka
{तस्यामात्यो महाकर्णिर्बभूवैकेश्वरस्तदा}
{स लब्धबलमात्मानं मन्यमानोऽवमन्यते}


\twolineshloka
{स राज्ञ उपभोग्यानि स्त्रियो रत्नधनानि च}
{आददे सर्वशो मूढ ऐश्वर्यं च स्वयं तदा}


\twolineshloka
{तदादाय च लुब्धस्य लोभाल्लोभोऽभ्यवर्धत}
{तथाहि सर्वमादाय राज्यमस्य जिहीर्षति}


\twolineshloka
{हीनस्य करणैः सर्वैरच्छ्वासपरमस्य च}
{यतमानोऽपि तद्राज्यं न शशाकेति नः श्रुतं}


\twolineshloka
{किमन्यद्विहिता नूनं तस्य सा पुरुषेन्द्रता}
{यदि ते विहितं राज्यं भविष्यति विशांपते}


\twolineshloka
{मिषतः सर्वलोकस्य स्थास्यते त्वयि तद्धुवम्}
{अतोऽन्यथा चेद्विहितं यतमानो न लप्स्यसे}


\threelineshloka
{एवं विद्वन्नुपादत्स्व मन्त्रिणां साध्वसाधुताम्}
{दुष्टानां चैव बोद्धव्यमदुष्टानां च भाषितम् ॥द्रोण उवाच}
{}


\twolineshloka
{विद्म ते भावदोषेण यदर्थमिदमुच्यते}
{दुष्ट पाण्डवहेतोस्त्वं दोषमाख्यापयस्युत}


\twolineshloka
{हितं तु परमं कर्ण ब्रवीमि कुलवर्धनम्}
{अथ त्वं मन्यसे दुष्टं ब्हूहि यत्परमं हितम्}


\twolineshloka
{अतोऽन्यथा चेत्क्रियते यद्ब्रवीमि परं हितम्}
{कुरवो वै विनङ्क्ष्यन्ति नचिरेणैव मे मतिः}


\chapter{अध्यायः २२४}
\twolineshloka
{विदुर उवाच}
{}


\twolineshloka
{राजन्निःसंशयं श्रेयो वाच्यस्त्वमसि बान्धवैः}
{न त्वशुश्रूषमाणे वै वाक्यं संप्रति तिष्ठति}


\twolineshloka
{प्रियं हितं च तद्वाक्यमुक्तवान्कुरुसत्तमः}
{भीष्मः शान्तनवो राजन्प्रतिगृह्णासि तन्न च}


\twolineshloka
{तथा द्रोणेन बहुधा भाषितं हितमुत्तमम्}
{तच्च राधासुतः कर्णो मन्यते न हितं तव}


\twolineshloka
{चिन्तयंश्च न पश्यामि राजंस्तव सुहृत्तमम्}
{आभ्यां पुरुषसिंहाभ्यां यो वा स्यात्प्रज्ञयाधिकः}


\twolineshloka
{इमौ हि वृद्धौ वयसा प्रज्ञया च श्रुतेन च}
{समौ च त्वयि राजेन्त्र तथा पाण्डुसुतेषु च}


\twolineshloka
{धर्मे चानवरौ राजन्सत्यतायां च भारत}
{रामाद्दाशरथेश्चैव गयाच्चैव न संशयः}


\twolineshloka
{न चोक्तवन्तावश्रेयः पुरस्तादपि किंचन}
{न चाप्यपकृतं किंचिदनयोर्लक्ष्यते त्वयि}


\twolineshloka
{तावुभौ पुरुषव्याघ्रावनागसि नृपे त्वयि}
{न मन्त्रयेतां त्वच्छ्रेयः कथं सत्यपराक्रमौ}


\twolineshloka
{प्रज्ञावन्तौ नरश्रेष्ठावस्मिँल्लोके नराधिप}
{त्वन्निमित्तमतो नेमौ किंचिज्जिह्मं वदिष्यतः}


\twolineshloka
{इति मे नैष्ठिकी बुद्धिर्वर्तते कुरुनन्दन}
{न चार्थहेतोर्धर्मज्ञौ वक्ष्यतः पक्षसंश्रितम्}


\twolineshloka
{एतद्धि परमं श्रेयो मन्येऽहं तव भारत}
{दुर्योधनप्रभृतयः पुत्रा राजन्यथा तव}


\twolineshloka
{तथैव पाण्डवेयास्ते पुत्रा राजन्न संशयः}
{तेषु चेदहितं किंचिन्मन्त्रयेयुरतद्विदः}


\threelineshloka
{मन्त्रिणस्ते न च श्रेयः प्रपश्यन्ति विशेषतः}
{अथ ते हृदये राजन्विशेषः स्वेषु वर्तते}
{अन्तरस्थं विवृण्वानाः श्रेयः कुर्युर्न ते ध्रुवम्}


\twolineshloka
{एतदर्थमिमौ राजन्महात्मानौ महाद्युती}
{नोचतुर्विवृतं किंचिन्न ह्येष तव निश्चयः}


\twolineshloka
{यच्चाप्यशक्यतां तेषामाहतुः पुरुषर्षभौ}
{तत्तथा पुरुषव्याघ्र तव तद्भद्रमस्तु ते}


\twolineshloka
{कथं हि पाण्डवः श्रीमान्सव्यसाची धनञ्जयः}
{शक्यो विजेतुं संग्रामे राजन्मघवतापि हि}


\twolineshloka
{`भीमसेनो महाबाहुर्नागायुतबलो महान्}
{राक्षसानां भयकरो बाहुशाली महाबलः}


\twolineshloka
{हिडिम्बो निहतो येन बाहुयुद्धेन भारत}
{यो रावणसमो युद्धे तथा च बकराक्षसः}


\twolineshloka
{स युध्यमानो राजेन्द्र भीमो भीमपराक्रमः}
{'कथं स्म युधि शक्येत विजेतुममरैरपि}


\twolineshloka
{तथैव कृतिनौ युद्धे यमौ यमसुताविव}
{कथं विजेतुं शक्यौ तौ रणे जीवितुमिच्छता}


\twolineshloka
{यस्मिन्धृतिरनुक्रोशः क्षमा सत्यं पराक्रमः}
{नित्यानि पाण्डवे ज्येष्ठे स जीयेत रणे कथम्}


\twolineshloka
{येषां पक्षधरो रामो येषां मन्त्री जनार्दनः}
{किं नु तैरजितं सङ्ख्ये येषां पक्षे च सात्यकिः}


\twolineshloka
{द्रुपदः श्वशुरो येषां येषां स्यालाश्च पार्षताः}
{धृष्टद्युम्नमुखा वीरा भ्रातरो द्रुपदात्मजाः}


\threelineshloka
{`चैद्यश्च येषां भ्राता च शिशुपालो महारथः}
{'सोऽशक्यतां च विज्ञाय तेषामग्रे च भारत}
{दायाद्यतां च धर्मेण सम्यक्तेषु समाचर}


\twolineshloka
{इदं निर्दिष्टमयशः पुरोचनकृतं महत्}
{तेषामनुग्रहेणाद्य राजन्प्रक्षालयात्मनः}


\twolineshloka
{तेषामनुग्रहश्चायं सर्वेषां चैव नः कुले}
{जीवितं च परं श्रेयः क्षत्रस्य च विवर्धनम्}


\twolineshloka
{द्रुपदोऽपि महान्राजा कृतवैरश्च नः पुरा}
{तस्य संग्रहणं राजन्स्वपक्षस्य विवर्धनम्}


\twolineshloka
{बलवन्तश्च दाशार्हा बहवश्च विशांपते}
{यतः कृष्णस्ततः सर्वे यतः कृष्णस्ततो जयः}


\twolineshloka
{यच्च साम्नैव शक्येत कार्यं साधयितुं नृप}
{को दैवशप्तस्तत्कार्यं विग्रहेण समाचरेत्}


\twolineshloka
{श्रुत्वा च जीवतः पार्थान्पौरजानपदा जनाः}
{बलवद्दर्शने हृष्टास्तेषां राजन्प्रियं कुरु}


\twolineshloka
{दुर्योधनश्च कर्णश्च शकुनिश्चापि सौबलः}
{अधर्मयुक्ता दुष्प्रज्ञा बाला मैषां वचः कृथाः}


\twolineshloka
{उक्तमेतत्पुरा राजन्मया गुणवतस्तव}
{दुर्योधनापराधेन प्रजेयं वै विनङ्क्ष्यति}


\chapter{अध्यायः २२५}
\twolineshloka
{धृतराष्ट्र उवाच}
{}


\threelineshloka
{भीष्मः शान्तनवो विद्वान्द्रोणश्च भगवानृषिः}
{`हितं च परमं सत्यमब्रूतां वाक्यमुत्तमम्}
{'हितं च परमं वाक्यं त्वं च सत्यं ब्रवीषि माम्}


\twolineshloka
{यथैव पाण्डोस्ते वीराः कुन्तीपुत्रा महारथाः}
{तथैव धर्मतः सर्वे मम पुत्रा न संशयः}


\twolineshloka
{यथैव मम पुत्राणामिदं राज्यं विधीयते}
{तथैव पाण्डुपुत्राणामिदं राज्यं न संशयः}


\twolineshloka
{क्षत्तरानय गच्छैतान्सह मात्रा सुसत्कृतान्}
{तया च देवरूपिण्या कृष्णया सह भारत}


\twolineshloka
{दिष्ट्या जीवन्ति ते पार्था दिष्ट्या जीवति सा पृथा}
{दिष्ट्या द्रुपदकन्यां च लब्धवन्तो महारथाः}


\twolineshloka
{दिष्ट्या वर्धामहे सर्वे दिष्ट्या शान्तः पुरोचनः}
{दिष्ट्या मम परं दुःखमपनीतं महाद्युते}


\fourlineindentedshloka
{`त्वमेव गत्वा विदुर तानिहानय मा चिरम्}
{वैशंपायन उवाच}
{एवमुक्तस्ततः क्षत्ता रथमारुह्य शीघ्रगम्}
{आगात्कतिपयाहोभिः पाञ्चालान्राजधर्मवित्'}


\twolineshloka
{ततो जगाम विदुरो धृतराष्ट्रस्य शासनात्}
{सकाशं यज्ञसेनस्य पाण्डवानां च भारत}


\twolineshloka
{समुपादाय रत्नानि वसूनि विविधानि च}
{द्रौपद्याः पाण्डवानां च यज्ञसेनस्य चैव ह}


\twolineshloka
{तत्र गत्वा स धर्मज्ञः सर्वशास्त्रविशारदः}
{द्रुपदं न्यायतो राजन्संयुक्तमुपतस्थिवान्}


\twolineshloka
{स चापि प्रतिजग्राह धर्मेण विदुरं ततः}
{चक्रतुश्च यथान्यायं कुशलप्रश्नसंविदम्}


\twolineshloka
{ददर्श पाण्डवांस्तत्र वासुदेवं च भारत}
{स्नेहात्परिष्वज्य स तान्पप्रच्छानामयं ततः}


\twolineshloka
{तैश्चाप्यमितबुद्दिः स पूजितो हि यथाक्रमम्}
{वचनाद्धृतराष्ट्रस्य स्नेहयुक्तं पुनःपुनः}


\twolineshloka
{पप्रच्छानामयं राजंस्ततस्तान्पाण्डुनन्दनान्}
{प्रददौ चापि रत्नानि विविधानि वसूनि च}


\twolineshloka
{पाण्डवानां च कुन्त्याश्च द्रौपद्याश्च विशांपते}
{द्रुपदस्य च पुत्राणां यथा दत्तानि कौरवैः}


\threelineshloka
{प्रोवाच चामितमतिः प्रश्रितं विनयान्वितः}
{द्रुपदं पाण्डुपुत्राणां सन्निधौ केशवस्य च ॥विदुर उवाच}
{}


\twolineshloka
{राजञ्छृणु सहामात्यः सपुत्रश्च वचो मम}
{धृतराष्ट्रः सपुत्रस्त्वां सहामात्यः सबान्धवः}


\twolineshloka
{अब्रवीत्कुशलं राजन्प्रीयमाणः पुनःपुनः}
{प्रीतिमांस्ते दृढं चापि संबन्धेन नराधिप}


\twolineshloka
{तथा भीष्मः शान्तनवः कौरवैः सह सर्वशः}
{कुशलं त्वां महाप्राज्ञः सर्वतः परिपृच्छति}


\twolineshloka
{भारद्वाजो महाप्राज्ञो द्रोणः प्रियसखस्तव}
{समाश्लेषमुपेत्य त्वां कुशलं परिपृच्छति}


\twolineshloka
{धृतराष्ट्रश्च पाञ्चाल्य त्वया संबन्धमेयिवान्}
{कृतार्थं मन्यतेत्मानं तथा सर्वेऽपि कौरवाः}


\twolineshloka
{न तथा राज्यसंप्राप्तिस्तेषां प्रीतिकरी मता}
{यथा संबन्धकं प्राप्य यज्ञसेन त्वया सह}


\twolineshloka
{एतद्विदित्वा तु भवान्प्रस्थापयतु पाण्डवान्}
{द्रष्टुं हि पाण्डुपुत्रांश्च त्वरन्ति कुरवो भृशम्}


\twolineshloka
{विप्रोषिता दीर्घकालमेते चापि नरर्षभाः}
{उत्सुका नगरं द्रष्टुं भविष्यन्ति तथा पृथा}


\twolineshloka
{कृष्णामपि च पाञ्चालीं सर्वाः कुरुवरस्त्रियः}
{द्रष्टुकामाः प्रतीक्ष्ते पुरं च विषयाश्च नः}


\twolineshloka
{स भवान्पाण्डुपुत्राणामाज्ञापयतु मा चिरम्}
{गमनं सहदाराणामेतदत्र मतं मम}


\threelineshloka
{निसृष्टेषु त्वया राजन्पाण्डवेषु महात्मसु}
{ततोऽहं प्रेषयिष्यामि धृतराष्ट्रस्य शीघ्रगान्}
{आगमिष्यन्ति कौन्तेयाः कुन्ती च सह कृष्णया}


\chapter{अध्यायः २२६}
\twolineshloka
{द्रुपद उवाच}
{}


\twolineshloka
{एवमेतन्महाप्राज्ञ यथात्थ विदुराद्य माम्}
{ममापि परमो हर्षः संबन्धेऽस्मिन्कृते प्रभो}


\twolineshloka
{गमनं चापि युक्तं स्याद्दृढमेषां महात्मनाम्}
{न तु तावन्मया युक्तमेतद्वक्तुं स्वयं गिरा}


\twolineshloka
{यदा तु मन्यते वीरः कुन्तीपुत्रो युधिष्ठिरः}
{भीमसेनार्जुनौ चैव यमौ च पुरुषर्षभौ}


\threelineshloka
{रामकृष्णौ च धर्मज्ञौ तदा गच्छन्तु पाण्डवाः}
{एतौ हि पुरुषव्याघ्रावेषां प्रियहिते रतौ ॥युधिष्ठिर उवाच}
{}


\threelineshloka
{परवन्तो वयं राजंस्त्वयि सर्वे सहानुगाः}
{यथा वक्ष्यसि नः प्रीत्या तत्करिष्यामहे वयम् ॥वैशंपायन उवाच}
{}


\threelineshloka
{ततोऽब्रवीद्वासुदेवो गमनं रोचते मम}
{यथा वा मन्यते राजा द्रुपदः सर्वधर्मवित् ॥द्रुपद उवाच}
{}


\twolineshloka
{यथैव मन्यते वीरो दाशार्हः पुरुषोत्तमः}
{प्राप्तकालं महाबाहुः सा बुद्धिर्निश्चिता मम}


\twolineshloka
{यथैव हि महाभागाः कौन्तेया मम सांप्रतम्}
{तथैव वासुदेवस्य पाण्डुपुत्रा न संशयः}


\threelineshloka
{न तद्ध्यायति कौन्तेयः पाण्डुपुत्रो युधिष्ठिरः}
{यथैषां पुरुषव्याघ्रः श्रेयो ध्यायति केशवः ॥वैशंपायन उवाच}
{}


\twolineshloka
{`पृथायास्तु ततो वेश्म प्रविवेश महामतिः}
{पादौ स्पृष्ट्वा पृथायास्तु शिरसा च महीं गतः}


\threelineshloka
{दृष्ट्वा तु देवरं कुन्ती शुशोच च मुहुर्मुहुः}
{कुन्त्युवाच}
{वैचित्रवीर्य ते पुत्राः कथंचिज्जीवितास्त्वया}


\twolineshloka
{त्वत्प्रसादाज्जतुगृहे मृताः प्रत्यागतास्तथा}
{कूर्मी चिन्तयते पुत्रान्यत्र वा तत्र संमता}


\twolineshloka
{चिन्तया वर्धिताः पुत्रा यथा कुशलिनस्तथा}
{तव पुत्रास्तु जीवन्ति त्वद्भक्त्या भरतर्षभ}


\twolineshloka
{यथा परभृतः पुत्रानरिष्टा वर्धयेत्सदा}
{तथैव पुत्रास्तु मम त्वया तात सुरक्षिताः}


\threelineshloka
{क्लेशास्तु बहवः प्राप्तास्तथा प्राणान्तिका मया}
{अतः परं न जानामि कर्तव्यं ज्ञातुमर्हसि ॥विदुर उवाच}
{}


\twolineshloka
{न विनश्यन्ति लोकेषु तव पुत्रा महाबलाः}
{अचिरेणैव कालेन स्वराज्यस्था भवन्ति ते}


\threelineshloka
{बान्धवैः सहिताः सर्वे न शोकं कुरु माधवि}
{वैशंपायन उवाच}
{'ततस्ते समनुज्ञाता द्रुपदेन महात्मना}


\twolineshloka
{पाण्डवाश्चैव कृष्णश्च विदुरश्च महामतिः}
{आदाय द्रौपदीं कृष्णां कुन्तीं चैव यशस्विनीम्}


\twolineshloka
{सविहारं सुखं जग्मुर्नगरं नागसाह्वयम्}
{`सुवर्णकक्ष्याग्रैवेयान्सुवर्णाङ्कुशभूषितान्}


\twolineshloka
{जाम्बूनदपरिष्कारान्प्रभिन्नकरटामुखान्}
{अधिष्ठितान्महामात्रैः सर्वशस्त्रसमन्वितान्}


\twolineshloka
{सहस्रं प्रददौ राजा गजानां वरवर्मिणाम्}
{रथानां च सहस्रं वै सुवर्णमणिचित्रितम्}


\twolineshloka
{चतुर्युजां भानुमच्च पञ्चानां प्रददौ तदा}
{सुवर्णपरिबर्हाणां वरचामरमालिनाम्}


\threelineshloka
{जात्यश्वानां च पञ्चाशत्सहस्रं प्रददौ नृपः}
{दासीनामयुतं राजा प्रददौ वरभूषणम्}
{ततः सहस्रं दासानां प्रददौ वरधन्वनाम्}


\twolineshloka
{हैमानि शय्यासनबाजनानिद्रव्याणि चान्यानि च गोधनानि}
{पृथक्पृथक्वैव ददौ स कोटिंपाञ्चालराजः परमप्रहृष्टः}


% Check verse!
शिबिकानां शतं पूर्णं वाहान्पञ्चशतं नरान्
\twolineshloka
{एवमेतानि पाञ्चालो जन्यार्थे प्रददौ धनम्}
{हरणं चापि पाञ्चाल्या ज्ञातिदेयं च सोमकः}


\twolineshloka
{धृष्टद्युम्नो ययौ तत्र भगिनीं गृह्य भारत}
{नानद्यमानो बहुशस्तूर्यघोषैः सहस्रशः ॥'}


\twolineshloka
{श्रुत्वा चोपस्थितान्वीरान्धृतराष्ट्रोऽम्बिकासुतः}
{प्रतिग्रहाय पाण्डूनां प्रेषयामास कौरवान्}


\twolineshloka
{विकर्णं च महेष्वासं चित्रसेनं च भारत}
{द्रोणं च परमेष्वासं गौतमं कृपमेव च}


\twolineshloka
{तैस्तैः परिवृताः शूरैः शोभमाना महारथाः}
{नगरं हास्तिनपुरं शनैः प्रविविशुस्तदा}


\twolineshloka
{`पाण्डवानागताञ्छ्रुत्वा नागरास्तु कुतूहलात्}
{मण्डयाञ्चक्रिरे तत्र नगरं नागसाह्वयम्}


\twolineshloka
{मुक्तपुष्पावकीर्णं तु जलसिक्तं तु सर्वतः}
{धूपितं दिव्यधूपेन मङ्गलैश्चाभिसंवृतम्}


\twolineshloka
{पताकोच्छ्रितमाल्यं च पुरमप्रतिमं बभौ}
{शङ्खभेरीनिनादैश्च नानावादित्रनिस्वनैः ॥'}


\twolineshloka
{कौतूहलेन नगरं पूर्यमाणमिवाभवत्}
{यत्र ते पुरुषव्याघ्राः शोकदुःखसमन्विताः}


\twolineshloka
{`निर्गताश्च पुरात्पूर्वं धृतराष्ट्रप्रबाधिताः}
{पुनर्निवृत्ता दिष्ट्या वै सह मात्रा परन्तपाः}


\twolineshloka
{इत्येवमीरिता वाचो जनैः प्रियचिकीर्षुभिः}
{'तत उच्चावचा वाचः प्रियाः सर्वत्र भारत}


\twolineshloka
{उदीरितास्तदाऽशृण्वन्पाण्डवा हृदयंगमाः}
{पौरा ऊचुःअयं स पुरुषव्याघ्रः पुनरायाति धर्मवित्}


\twolineshloka
{यो नः स्वानिव दायादान्धर्मेण परिरक्षति}
{अद्य पाण्डुर्महाराजो वनादिव मनःप्रियम्}


\twolineshloka
{आगतश्चैवमस्माकं चिकीर्षन्नात्र संशयः}
{किं न्वद्य सुकृतं कर्म सर्वेषां नः प्रियं परम्}


\threelineshloka
{यन्नः कुन्तीसुता वीरा भर्तारः पुनरागताः}
{यदि दत्तं यदि हुतं यदि वाप्यस्ति नस्तपः}
{तेन तिष्ठन्तु नगरे पाण्डवाः शरदां शतम्}


\twolineshloka
{ततस्ते धृतराष्ट्रस्य भीष्मस्य च महात्मनः}
{अन्येषां च तदर्हाणां चक्रुः पादाभिवन्दनम्}


\twolineshloka
{पृष्टास्तु कुशलप्रश्नं सर्वेण नगरेण ते}
{समाविशन्त वेश्मानि धृतराष्ट्रस्य शासनात्}


\chapter{अध्यायः २२७}
\twolineshloka
{`वैशंपायन उवाच}
{}


\twolineshloka
{दुर्योधनस्य महिषी काशिराजसुता तदा}
{धृतराष्ट्रस्य पुत्राणां वधूभिः सहिता तदा}


\twolineshloka
{पाञ्चालीं प्रतिजग्राह साध्वीं श्रियमिवापराम्}
{पूजयामास पूजार्हां शचीदेवीमिवागताम्}


\twolineshloka
{ववन्दे तत्र गान्धारीं कृष्णया सह माधवी}
{आशिषश्च प्रयुक्त्वा तु पाञ्चालीं परिषस्वजे}


\twolineshloka
{परिष्वज्यैव गान्धारी कृष्णां कमललोचनाम्}
{पुत्राणां मम पाञ्चाली मृत्युरेवेत्यमन्यत}


\twolineshloka
{संचिन्त्य विदुरं प्राह युक्तितः सुबलात्मजा}
{कुन्तीं राजसुतां क्षत्तः सवधूं सपरिच्छदाम्}


\twolineshloka
{पाण्डोर्निवेशनं शीघ्रं नीयतां यदि रोचते}
{करणेन मुहूर्तेन नक्षत्रेण शुभे तिथौ}


\twolineshloka
{यथा सुखं तथा कुन्ती रंस्यते स्वगृहे सुतैः}
{तथेत्येव तदा क्षत्ता कारयामास तत्तथा}


\twolineshloka
{पूजयामासुरत्यर्थं बान्धवाः पाण्डवांस्तदा}
{नागराः श्रेणिमुख्याश्च पूजयन्ति स्म पाण्डवान्}


\twolineshloka
{भीष्मो द्रोणः कृपः कर्णो बाह्लीकः ससुतस्तदा}
{शासनाद्धृतराष्ट्रस्य अकुर्वन्नतिथिक्रियाम्}


\twolineshloka
{एवं विहरतां तेषां पाण्डवानां महात्मनाम्}
{नेता सर्वस्य कार्यस्य विदुरो राजशासनात् ॥'}


\threelineshloka
{विश्रान्तास्ते महात्मानः कंचित्कालं सकेशवाः}
{आहूता धृतराष्ट्रेण राज्ञा शान्तनवेन च ॥धृतराष्ट्र उवाच}
{}


\twolineshloka
{भ्रातृभिः सह कौन्तेय निबोधेदं वचो मम}
{`पाण्डुना वर्धितं राज्यं पाण्डुना पालितं जगत्}


\twolineshloka
{शासनान्मम कौन्तेय मम भ्राता महाबलः}
{कृतवान्दुष्करं कर्म नित्यमेव विशांपते}


\twolineshloka
{तस्मात्त्वमपि कौन्तेय शासनं कुरु मा चिरम्}
{मम पुत्रा दुरात्मानः सर्वेऽहंकारसंयुताः}


\twolineshloka
{शासनं न करिष्यन्ति मम नित्यं युधिष्ठिर}
{स्वकार्यनिरतैर्नित्यमवलिप्तैर्दुरात्मभिः ॥'}


\twolineshloka
{पुनर्वै विग्रहो मा भूत्खाण्डवप्रस्थमाविश}
{न हि वो वसतस्तत्र कश्चिच्छक्तः प्रबाधितुम्}


\twolineshloka
{संरक्ष्यमाणान्पार्थेन त्रिदशानिव वज्रिणा}
{अर्धराज्यं तु संप्राप्य खाण्डवप्रस्थणाविश}


\twolineshloka
{`केशवो यदि मन्यते तत्कर्तव्यमसंशयम् ॥' वैशंपायन उवाच}
{}


\threelineshloka
{प्रतिगृह्य तु तद्वाक्यं नृपं सर्वे प्रणम्य च}
{`वासुदेवेन संमन्त्र्य पाण्डवाः समुपाविशन् ॥धृतराष्ट्र उवाच}
{}


\twolineshloka
{अभिषेकस्य संभारान्क्षत्तरानय मा चिरम्}
{अभिषिक्तं करिष्यामि ह्यद्य वै कुरुनन्दनम्}


\twolineshloka
{ब्राह्मणा नैगमश्रेष्ठाः श्रेणीमुख्याश्च सर्वतः}
{आहूयन्तां प्रकृतयो बान्धवाश्च विशेषतः}


\twolineshloka
{पुण्याहं वाच्यतां तात गोसहस्रं प्रदीयताम्}
{ग्राममुख्याश्च विप्रेभ्यो दीयन्तां बहुदक्षिणाः}


\twolineshloka
{अङ्गदे मकुटं क्षत्तर्हस्ताभरणमानय}
{मुक्तावलीश्च हारं च निष्काणि कटकानि च}


\twolineshloka
{कटिबन्धश्च सूत्रं च तथोदरनिबन्धनम्}
{अष्टोत्तरसहस्रं तु ब्राह्मणाधिष्ठिता गजाः}


\twolineshloka
{जाह्नवीसलिलं शीघ्रमानीयन्तां पुरोहितैः}
{अभिषेकोदकक्लिन्नं सर्वाभरणभूषितम्}


\twolineshloka
{औपवाह्योपरिगतं दिव्यचारमरवीजितम्}
{सुवर्णमणिचित्रेण श्वेतच्छत्रेण शोभितम्}


\twolineshloka
{जयेति द्विजवाक्येनु स्तूयमानं नृपैस्तथा}
{दृष्ट्वा कुन्तीसुतं ज्येष्ठमाजमीढं युधिष्ठिरम्}


\twolineshloka
{प्रीताः प्रीतेन मनसा प्रशंसन्तु परे जनाः}
{पाण्डोः कृतोपकारस्य राज्यं दत्वा ममैव च}


\threelineshloka
{प्रतिक्रिया कृतमिदं भविष्यति न संशयः}
{भीष्मो द्रोणः कृपः क्षत्ता साधुसाध्वित्यथाब्रुवन् ॥श्रीवासुदेव उवाच}
{}


\twolineshloka
{युक्तमेतन्महाभाग कौरवाणां यशस्करम्}
{शीघ्रमद्यैव राजेन्द्र त्वयोक्तं कर्तुमर्हसि}


\twolineshloka
{इत्येवमुक्तो वार्ष्णेयस्त्वरयामास तत्तदा}
{तथोक्तं धृतराष्ट्रेण कारयामास केशवः}


\twolineshloka
{तस्मिन्क्षणे महाराज कृष्णद्वैपायनस्तदा}
{आगत्य कुरुभिः सर्वैः पूजितः ससुहृद्गणैः}


\twolineshloka
{मूर्धाभिषिक्तैः सहितो ब्राह्मणैर्वेदपारगैः}
{कारयामास विधिवत्केशवानुमते तदा}


\twolineshloka
{कृपो द्रोणश्च भीष्मश्च धौम्यश्च व्यासकेशवौ}
{बाह्लीकः सोमदत्तश्च चातुर्वेद्यपुरस्कृताः}


\twolineshloka
{अभिषेकं तदा चक्रुर्भद्रपीठे सुसंस्कृतम् ॥व्यास उवाच}
{}


\twolineshloka
{जित्वा तु पृथिवीं कृत्स्नां वशे कृत्वा नृपान्भवान्}
{राजसूयादिभिर्यज्ञैः क्रतुभिर्वरदक्षिणैः}


\twolineshloka
{स्नात्वा ह्यवभृथस्नानं मोदतां बान्धवैः सह}
{एवमुक्त्वा तु ते सर्वे आशीर्भिरभिपूजयन्}


\twolineshloka
{मूर्धाभिषिक्तः कौरव्यः सर्वाभरणभूषितः}
{जयेति संस्तुतो राजा प्रददौ धनमक्षयम्}


\twolineshloka
{सर्वमूर्धाभिषिक्तैश्च पूजितः कुरनन्दनः}
{औपवाह्यमथारुह्य श्वेतच्छत्रेण शोभितः}


\twolineshloka
{रराज राजाभिमतो महेन्द्र इव दैवतैः}
{ततः प्रदक्षिणीकृत्य नगरं नागसाह्वयम्}


\twolineshloka
{प्रविवेश तदा राजा नागरैः पूजितो गृहम्}
{मूर्धाभिषिक्तं कौन्तेयमभ्यगच्छन्त कौरवाः}


\twolineshloka
{गान्धारिपुत्राः शोचन्तः सर्वे ते सह बान्धवैः}
{ज्ञात्वा शोकं च पुत्राणां धृतराष्ट्रोऽब्रवीदिदं}


\twolineshloka
{समक्षं वासुदेवस्य कुरूणां च समक्षतः}
{अभिषेकस्त्वया प्राप्तो दुष्प्रापो ह्यकृतात्मभिः}


\twolineshloka
{गच्छ त्वमद्यैव नृप कृतकृत्योऽसि कौरव}
{आयुः पुरूरवा राजन्नहुषेण ययातिना}


\twolineshloka
{तत्रैव निवसन्ति स्म खाण्डवे तु नृपोत्तम}
{राजधानी तु सर्वेषां पौरवाणां महाभुज}


\twolineshloka
{विनाशितं मुनिगणैर्लोभाद्बुधसुतस्य वै}
{तस्मात्त्वं खाण्डवप्रस्थं पुरं राष्ट्रं च वर्धय}


\twolineshloka
{ब्राह्मणाः क्षत्रिया वैश्याः शूद्राश्च कृतलक्षणाः}
{त्वद्भक्त्या जन्तवश्चान्ये भजन्त्येव पुरं शुभम्}


\threelineshloka
{पुरं राष्ट्रं समृद्धं वै धनधान्यसमाकुलम्}
{तस्माद्गच्छस्व कौन्तेय भ्रातृभिः सहितोऽनघ ॥वैशंपायन उवाच}
{}


\twolineshloka
{प्रतिगृह्य तु तद्वाक्यं तस्मै सर्वे प्रणम्य च}
{रथैर्नागैर्हयैश्चापि सहितास्तु पदातिभिः}


\twolineshloka
{प्रतस्थिरे ततो घोषसंयुक्तैः स्यन्दनैर्वरैः}
{तान्दृष्ट्वा नागराः सर्वे भक्त्या चैव प्रतस्थिरे}


\twolineshloka
{गच्छतः पाण्डवैः सार्धं दृष्ट्वा नागपुरालयात्}
{पाण्डवैः सहिता गन्तुं नार्हतेति च नागरान्}


\twolineshloka
{घोषयामास नगरे धार्तराष्ट्रः ससौबलः}
{'ततस्ते पाण्डवास्तत्र गत्वा कृष्णपुरोगमाः}


\twolineshloka
{मण्डयाञ्चक्रिरे तद्वै पुरं स्वर्गादिव च्युतम्}
{`वासुदेवो जगन्नाथश्चिन्तयामास वासवम्}


\twolineshloka
{महेन्द्रश्चिन्तितो राजन्विश्वकर्माणमादिशत्}
{विश्वकर्मन्महाप्राज्ञ अद्यप्रभृति तत्पुरम्}


\twolineshloka
{इन्द्रप्रस्थमिति ख्यातं दिव्यं भूम्यां भविष्यति}
{महेन्द्रशासनाद्गत्वा विश्वकर्मा तु केशवम्}


\twolineshloka
{प्रणम्य प्रणिपातार्हं किं करोमीत्यभाषत}
{वासुदेवस्तु तच्छ्रुत्वा विश्वकर्माणमूचिवान्}


\threelineshloka
{कुरुष्व कुरुराजस्य महेन्द्रपुरसन्निभम्}
{इन्द्रेण कृतनामानमिन्द्रप्रस्थं महापुरम् ॥वैशंपायन उवाच}
{'}


\twolineshloka
{ततः पुण्ये शिवे देशे शान्तिं कृत्वा महारथाः}
{स्वस्तिवाच्य यथान्यायमिन्द्रप्रस्थं भवत्विति}


\twolineshloka
{तत्पुरं मापयामासुर्द्वैपायनपुरोगमाः}
{`ततः स विश्वकर्मा तु चकार पुरमुत्तमम् ॥'}


\twolineshloka
{सागरप्रतिरूपाभिः परिखाभिरलङ्कृतम्}
{प्राकारेण च संपन्नं दिवमावृत्य तिष्ठता}


\twolineshloka
{पाण्डुराभ्रप्रकाशेन हिमरश्मिनिभेन च}
{शुशुभे तत्पुरश्रेष्ठं नागैर्भोगवती यथा}


\twolineshloka
{द्विपक्षगरुडप्रख्यैर्द्वारैः सौधैश्च शोभितम्}
{गुप्तमभ्रचयप्रख्यैर्गोपुरैर्मन्दरोपमैः}


\twolineshloka
{विविधैरपि निर्विद्धैः शस्त्रोपेतैः सुसंवृतैः}
{शक्तिभिश्चावृतं तद्धि द्विजिह्वैरिव पन्नगैः}


\twolineshloka
{तल्पैश्चाभ्यासिकैर्युक्तं शुशुभे योधरक्षितम्}
{थीक्ष्णाङ्कुशशतघ्नीभिर्यन्त्रजालैश्च शोभितम्}


\twolineshloka
{आयसैश्च महाचक्रैः शुशुभे तत्पुरोत्तमम्}
{सुविभक्तमहारथ्यं देवताबाधवर्जितम्}


\twolineshloka
{विरोचमानं विविधैः पाण्डुरैर्भवनोत्तमैः}
{तत्त्रिविष्टपसंकाशमिन्द्रप्रस्थं व्यरोचत}


\twolineshloka
{मेघवृन्दमिवाकाशे विद्धं विद्युत्समावृतम्}
{तत्र रम्ये शिवे देशे कौरव्यस्य निवेशनम्}


\twolineshloka
{शुशुभे धनसंपूर्णं धनाध्यक्षक्षयोपमम्}
{तत्रागच्छन्द्विजा राजन्सर्ववेदविदां वराः}


\twolineshloka
{निवासं रोचयन्ति स्म सर्वभाषाविदस्तथा}
{वणिजश्चाययुस्तत्र नानादिग्भ्यो धनार्थिनः}


\twolineshloka
{सर्वशिल्पविदस्तत्र वासायाभ्यागमंस्तदा}
{उद्यानानि च रम्याणि नगरस्य समन्ततः}


\twolineshloka
{आम्रैराम्रातकैर्नीपैरशोकैश्चम्पकैस्तथा}
{पुन्नागैर्नागपुष्पैश्च लकुचैः पनसैस्तथा}


\twolineshloka
{शालतालतमालैश्च बकुलैश्च सकेतकैः}
{मनोहरैः सुपुष्पैश्च फलभारावनामितैः}


\twolineshloka
{प्राचीनामलकैर्लोध्रैरङ्कोलैश्च सुपिष्पितैः}
{जम्बूभिः पाटलाभिश्च कुब्जकैरतिमुक्तकैः}


\twolineshloka
{करवीरैः पारिजातैरन्यैश्च विविधैर्द्रुमैः}
{नित्यपुष्पफलोपेतैर्नानाद्विजगणायुतैः}


\twolineshloka
{मत्तबर्हिणसंघुष्टकोकिलैश्च सदामदैः}
{गृहैरादर्शविमलैर्विविधैश्च लतागृहैः}


\twolineshloka
{मनोहरैश्चित्रगृहैस्तथाऽजगतिप्रवतैः}
{वापीभिर्विविधाभिश्च पूर्णाभिः परमाम्भसा}


\twolineshloka
{सरोभिरतिरम्यैश्च पद्मोत्पलसुगन्धिभिः}
{हंसकारण्डवयुतैश्चक्रवाकोपशोभितैः}


\twolineshloka
{रम्याश्च विविधास्तत्र पुष्करिण्यो वनावृताः}
{तडागानि च रम्याणि बृहन्ति सुबहूनि च}


\twolineshloka
{`नदी च नन्दिनी नाम सा पुरीमुपगूहति}
{चातुर्वर्ण्यसमाकीर्णमन्यैः शिल्पिभिरावृतम्}


\twolineshloka
{सर्वदाभिसृतं सद्भिः कारितं विश्वकर्मणा}
{उपभोगसमृद्धैश्च सर्वद्रव्यसमावृतम्}


\twolineshloka
{नित्यमार्यजनोपेतं नरनारीगणैर्युतम्}
{वाजिवारणसंपूर्णं गोभिरुष्ट्रैः खरैरजैः}


\twolineshloka
{तत्त्रिविष्टपसङ्काशमिन्द्रप्रस्थं व्यरोचत}
{पुरीं सर्वगुणोपेतां निर्मितां विश्वकर्मणा}


\twolineshloka
{पौरवाणामधिपतिः कुन्तीपुत्रो युधिष्ठिरः}
{कृतमङ्गलसत्कारैर्ब्राह्मणैर्वेदपारगैः}


\twolineshloka
{द्वैपायनं पुरस्कृत्य धौम्यस्याभिमते स्थितः}
{भ्रातृभिः सहितो राजा राजमार्गमतीत्य च}


\twolineshloka
{औपवाह्यगतो राजा केशवेन सहाभिभूः}
{तोरणद्वारसुमुखं द्वात्रिंशद्द्वारसंयुतम्}


\twolineshloka
{वर्धमानपुरद्वारात्प्रविवेश महाद्युतिः}
{शङ्खदुन्दुभिनिर्घोषाः श्रूयन्ते बहवो भृशम्}


\twolineshloka
{जयेति ब्राह्मणगिरः श्रूयन्ते च सहस्रशः}
{संस्तूयमानो मुनिभिः सूतमागधबन्दिभिः}


\twolineshloka
{औपवाह्यगतो राजा राजमार्गमतीत्य च}
{कृतमङ्गलसत्कारं प्रविवेश गृहोत्तमम्}


\twolineshloka
{प्रविश्य भवनं राजा नागरैरभिसंवृतः}
{प्रहृष्टमुदितैरासीत्सत्कारैरभिपूजितः}


\twolineshloka
{पूजयामास विप्रेन्द्रान्केशेन महात्मना}
{ततस्तु राष्ट्रं नगरं नरनारीगणायुतम्}


% Check verse!
गोधनैश्च समाकीर्णं सस्यैर्वृद्धिं तदागमत् ॥'
\twolineshloka
{तेषां पुण्यजनोपेतं राष्ट्रमाविशतां महत्}
{पाण्डवानां महाराज शश्वत्प्रीतिरवर्धत}


\twolineshloka
{`सौबलेन च कर्णेन धार्तराष्ट्रैः कृपेण च}
{'तथा भीष्मेण राज्ञा च धर्मप्रणयिना सदा}


\twolineshloka
{पाण्डवाः समपद्यन्त खाण्डवप्रस्थवासिनः}
{पञ्चभिस्तैर्महेष्वासैरिन्द्रकल्पैः समावृतम्}


\twolineshloka
{शुशुभे तत्पुरश्रेष्ठं नागैर्भोगवती यथा}
{`ततस्तु विश्वकर्माणं पूजयित्वा विसृज्य च}


\twolineshloka
{द्वैपायनं च संपूज्य विसृज्य च नराधिपः}
{वार्ष्णेयमब्रवीद्राजा गन्तुकामं कृतक्षणम्}


\twolineshloka
{तव प्रसादाद्वार्ष्णेय राज्यं प्राप्तं मयाऽनघ}
{प्रसादादेव ते वीर शून्यं राष्ट्रं सुदुर्गमम्}


\twolineshloka
{तवैव तु प्रसादेन राज्यस्थाश्च भवामहे}
{गतिस्त्वमापत्कालेऽपि पाण्डवानां च माधव}


\threelineshloka
{ज्ञात्वा तु कृत्यं कर्तव्यं कारयस्व भवान्हि नः}
{यदिष्टमनुमन्तव्यं पाण्डवानां त्वयाऽनघ ॥श्रीवासुदेव उवाच}
{}


\twolineshloka
{त्वत्प्रभावान्महाराज्यं संप्राप्तं हि स्वधर्मतः}
{पितृपैतामहं राज्यं कथं न स्यात्तव प्रभो}


\twolineshloka
{धार्तराष्ट्रा दुराचाराः किं करिष्यन्ति पाण्डवान्}
{यथेष्टं पालय जगच्छश्वद्धर्मधुरं वह}


\twolineshloka
{पुनः पुनश्च संहर्षाद्ब्राह्मणान्भर पौरव}
{अद्यैव नारदः श्रीमानागमिष्यति सत्वरः}


\twolineshloka
{आदत्स्व तस्य वाक्यानि शासनं कुरु तस्य वै}
{एवमुक्त्वा ततः कुन्तीमभिवाद्य जनार्दनः}


\threelineshloka
{उवाच श्लक्ष्णया वाचा गमिष्यामि नमोस्तु ते}
{कुन्त्युवाच}
{जातुषं गृहमासाद्य मया प्राप्तं यदानघ}


\twolineshloka
{आर्येण समभिज्ञातं त्वया वै यदुपुङ्गव}
{त्वया नाथेन गोविन्द दुःखं तीर्णं महत्तरम्}


\twolineshloka
{त्वं हि नाथस्त्वनाथानां दरिद्राणां विशेषतः}
{सर्वदुःखानि शाम्यन्ति तव संदर्शनान्मम}


\twolineshloka
{स्मरस्वैनान्महाप्राज्ञ तेन जीवन्ति पाण्डवाः ॥वैशंपायन उवाच}
{}


\twolineshloka
{करिष्यामीति चामन्त्र्य अभिवाद्य पितृष्वसाम्}
{गमनाय मतिं चक्रे वासुदेवः सहानुगः ॥'}


\twolineshloka
{तान्निवेश्य ततो वीरः सह रामेण कौरवान्}
{ययौ द्वारवतीं राजन्पाण्डवानुमते तदा}


\chapter{अध्यायः २२८}
\twolineshloka
{जनमेजय उवाच}
{}


\twolineshloka
{एवं संप्राप्य राज्यं तदिन्द्रप्रस्थे तपोधन}
{अत ऊर्ध्वं नरव्याघ्राः किमकुर्वत पाण्डवाः}


\twolineshloka
{सर्व एव महात्मानः सर्वे मम पितामहाः}
{द्रौपदी धर्मपत्नी च कथं तानन्ववर्तत}


\twolineshloka
{कथमासुश्च कृष्णायामेकस्यां ते नरर्षभाः}
{वर्तमाना महाभागा नाभिद्यन्त परस्परम्}


\threelineshloka
{श्रोतुमिच्छाम्यहं तत्र विस्तरेण यथातथम्}
{तेषां चेष्टितमन्योन्यं युक्तानां कृष्णया सह ॥वैशंपायन उवाच}
{}


\twolineshloka
{धृतराष्ट्राभ्यनुज्ञाता इन्द्रप्रस्थं प्रविश्य तत्}
{रेमिरे पुरुषव्याघ्राः कृष्णया सह पाण्डवाः}


\twolineshloka
{प्राप्य राज्यं महातेजाः सत्यसन्धो युधिष्ठिरः}
{पालयामास धर्मेण पृथिवीं भ्रातृभिः सह}


\twolineshloka
{जितारयो महात्मानः सत्यधर्मपरायणाः}
{एवं पुरमिदं प्राप्य तत्रोषुः पाण्डुनन्दनाः}


\twolineshloka
{कुर्वाणाः पौरकार्याणि सर्वाणि भरतर्षभाः}
{आसांचक्रुर्महार्हेषु पार्थिवेष्वासनेषु च}


\twolineshloka
{तेषु तत्रोपविष्टेषु पाण्डवेषु महात्मसु}
{आययौ धर्मराजं तु द्रष्टुकामोऽथ नारदः}


\twolineshloka
{`पथा नक्षत्रजुष्टेन सुपर्णाचरितेन च}
{चन्द्रसूर्यप्रकाशेन सेवितेन महर्षिभिः}


\twolineshloka
{नभस्स्थलेन दिव्येन दुर्लभेनातपस्विनाम्}
{भूतार्चितो भूतधरां राष्ट्रमन्दिरभूषिताम्}


\twolineshloka
{अवेक्षमाणो द्युतिमानाजगाम महातपाः}
{सर्ववेदान्तगो विप्रः सर्ववेदाङ्गपारगः}


\twolineshloka
{परेण तपसा युक्तो ब्राह्मेण तपसा वृतः}
{नये नीतौ च निस्तो विश्रुतश्च महामुनिः}


\twolineshloka
{परात्परतरं प्राप्तो धर्मान्समभिजग्मिवान्}
{भावितात्मा गतरजाः शान्तो मृदुर्ऋजुर्दिवजः}


\twolineshloka
{धर्मेणाधिगतः सर्वैर्देवदानवमानुषैः}
{क्षीणकर्मसु पापेषु भूतेषु विविधेषु च}


\twolineshloka
{सर्वथा कृतमर्यादो वेदेषु विविधेषु च}
{शतशः सोमपा यज्ञे पुण्ये पुण्यकृदग्निचित्}


\twolineshloka
{ऋक्सामयजुषां वेत्ता न्यायदृग्धर्मकोविदः}
{ऋजुरारोहबान्वृद्धो भूयिष्ठपथिकोऽनघः}


\twolineshloka
{श्लक्ष्णया शिखयोपेतः संपन्नः परमत्विषा}
{अवदाते च सूक्ष्मे च दिव्ये च रचिते शुभे}


\twolineshloka
{महेन्द्रदत्ते महती बिभ्रत्परमवाससी}
{जाम्बूनदमये दिव्ये गण्डूपदमुखे नवे}


\twolineshloka
{अग्न्यर्कसदृशे दिव्ये धारयन्कुण्डले शुभे}
{राजतच्छत्रमुच्छ्रित्य चित्रं परमवर्चसम्}


\twolineshloka
{प्राप्य दुष्प्रापमन्येन ब्रह्मवर्चसमुत्तमम्}
{भवने भूमिपालस्य बृहस्पतिरिवाप्लुतः}


\twolineshloka
{संहितायां च सर्वेषां स्थितस्योपस्थितस्य च}
{द्विपदस्य च धर्मस्य क्रमधर्मस्य पारगः}


\twolineshloka
{गाधा सामानुसामज्ञः साम्नां परमवल्गुनाम्}
{आत्मनः सर्वमोक्षिभ्यः कृतिमान्कृत्यवित्सदा}


\twolineshloka
{यजुर्धर्मैर्बहुविधैर्मतो मतिमतां वरः}
{विदितार्थः समश्चैव च्छेत्ता निगमसंशयान्}


\twolineshloka
{अर्थनिर्वचने नित्यं संशयच्छिदसंशयः}
{प्रकृत्या धर्मकुशलो दाता धर्मविशारदः}


\twolineshloka
{लोपेनागमधर्मेण संक्रमेण च वृत्तिषु}
{एकशब्दांश्च नानार्थानेकार्थांश्च पृथक्कृतान्}


\twolineshloka
{पृथगर्थाभिधानांश्च प्रयोगानन्ववेक्षिता}
{प्रमाणभूतो लोकेषु सर्वाधिकरणेषु च}


\twolineshloka
{सर्ववर्णविकारेषु नित्यं कुशलपूजितः}
{स्वरेऽस्वरे च विविधे वृत्तेषु विविधेषु च}


\twolineshloka
{समस्थानेषु सर्वेषु समाम्नायेषु धातुषु}
{उद्देश्यानां समाख्याता सर्वमाख्यातमुद्दिशन्}


\twolineshloka
{अभिसन्धिषु तत्त्वज्ञः पदान्यङ्गान्यनुस्मरन्}
{कालधर्मेण निर्दिष्टं यथार्थं च विचारयन्}


\twolineshloka
{चिकीर्षितं च यो वेत्ता यथा लोकेन संवृतम्}
{विभाषितं च समयं भाषितं हृदयंगमम्}


\twolineshloka
{आत्मने च परस्मै च स्वरसंस्कारयोगवित्}
{एषां स्वराणां ज्ञाता च बोद्धा प्रवचनः स्वराट्}


\twolineshloka
{विज्ञाता चोक्तवाक्यानामेकतां बहुतां तथा}
{बोद्धा हि परमार्थांश्च विविधांश्च व्यतिक्रमान्}


\twolineshloka
{अभेदतश्च बहुशो बहुशश्चापि भेदतः}
{वक्ता विविधवाक्यानां नानादेशसमीक्षिता}


\twolineshloka
{पञ्चागमांश्च विविधानादेशांश्च समीक्षिता}
{नानार्थकुशलस्तत्र तद्धितेषु च कृत्स्नशः}


\twolineshloka
{परिभूषयिता वाचां वर्णतः स्वरतोऽर्थतः}
{प्रत्ययं च समाख्याता नियतं प्रतिधातुकम्}


\twolineshloka
{पञ्च चाक्षरजातानि स्वरसंज्ञानि यानि च}
{तमागतमृषिं दृष्ट्वा प्रत्युद्गम्याभिवाद्य च ॥'}


\twolineshloka
{आसनं रुचिरं तस्मै प्रददौ स युधिष्ठिरः}
{`कृष्णाजिनोत्तरे तस्मिन्नुपविष्टो महानृषिः ॥'}


\threelineshloka
{देवर्षेरुपविष्टस्य स्वयमर्ध्यं यथाविधि}
{प्रादाद्युधिष्ठिरो धीमान्राज्यं तस्मै न्यवेदयत्}
{प्रतिगृह्य तु तां पूजामृषिः प्रीतमनास्तदा}


\twolineshloka
{आशीर्भिर्वर्धयित्वा च तमुवाचास्यतामिति}
{निषसादाभ्यनुज्ञातस्ततो राजा युधिष्ठिरः}


\twolineshloka
{प्रेषयामास कृष्णायै भगवन्तमुपस्थितम्}
{श्रुत्वैतद्द्रौपदी चापि शुचिर्भूत्वा समाहिता}


\twolineshloka
{जगाम तत्र यत्रास्ते नारदः पाण्डवैः सह}
{तस्याभिवाद्य चरणौ देवर्षेर्धर्मचारिणी}


\twolineshloka
{कृताञ्जलिः सुसंवीता स्थिताऽथ द्रुपदात्मजा}
{तस्याश्चापि स धर्मात्मा सत्यवागृषिसत्तमः}


\twolineshloka
{आशिषो विविधाः प्रोच्य राजपुत्र्यास्तु नारदः}
{गम्यतामिति होवाच भगवांस्तामनिन्दिताम्}


\twolineshloka
{गतायामथ कृष्णायां युधिष्ठिरपुरोगमान्}
{विविक्ते पाण्डवान्सर्वानुवाच भगवानृषिः}


\twolineshloka
{पाञ्चाली भवतामेका धर्मपत्नी यशस्विनी}
{यथा वो नात्र भेदः स्यात्तथा नीतिर्विधीयतां}


\twolineshloka
{सुन्दोपसुन्दौ हि पुरा भ्रातरौ सहितावुभौ}
{आस्तामवध्यावन्येषां त्रिषु लोकेषु विश्रुतौ}


\twolineshloka
{एकराज्यावेकगृहावेकशय्यासनाशनौ}
{तिलोत्तमायास्तौ हेतोरन्योन्यमभिजघ्नतुः}


\threelineshloka
{रक्ष्यतां सौहृदं तस्मादन्योन्यप्रीतिभावकम्}
{यथा वो नात्र भेदः स्यात्तत्कुरुष्व युधिष्ठिर ॥युधिष्ठिर उवाच}
{}


\twolineshloka
{सुन्दोपसुन्दावसुरौ कस्य पुत्रौ महामुने}
{उत्पन्नश्च कथं भेदः कथं चान्योन्यमघ्नताम्}


\twolineshloka
{अप्सरा देवकन्या वै कस्य चैषा तिलोत्तमा}
{यस्याः कामेन संमत्तौ जघ्नतुस्तौ परस्परम्}


\twolineshloka
{एतत्सर्वं यथा वृत्तं विस्तरेण तपोधन}
{श्रोतुमिच्छामहे ब्रह्मन्परं कौतूहलं हि मे}


\chapter{अध्यायः २२९}
\twolineshloka
{नारद उवाच}
{}


\twolineshloka
{शणु मे विस्तरेणेममितिहासं पुरातनम्}
{भ्रातृभिः सहितः पार्थ यथा वृत्तं युधिष्ठिर}


\twolineshloka
{महासुरस्यान्ववाये हिरण्यकशिपोः पुरा}
{निकुम्भो नाम दैत्येन्द्रस्तेजस्वी बलवानभूत्}


\twolineshloka
{तस्य पुत्रौ महावीर्यौ जातौ भीमपराक्रमौ}
{सुन्दोपसुन्दौ दैत्येन्द्रौ दारुणौ क्रूरमानसौ}


\twolineshloka
{तावेकनिश्चयौ दैत्यावेककार्यार्थसंमतौ}
{निरन्तरमवर्तेतां समदुःखसुखावुभौ}


\twolineshloka
{विनाऽन्योन्यं न भुञ्जाते विनाऽन्योन्यं न जल्पतः}
{अन्योन्यस्य प्रियकरावन्योन्यस्य प्रियंवदौ}


\twolineshloka
{एकशीलसमाचारौ द्विधैवैकोऽभवत्कृतः}
{तौ विवृद्धौ महावीर्यौ कार्येष्वप्येकनिश्चयौ}


\twolineshloka
{त्रैलोक्यविजयार्थाय समाधायैकनिश्चयम्}
{दीक्षां कृत्वा गतौ विन्ध्यं तावुग्रं तेपतुस्तपः}


\twolineshloka
{तौ तु दीर्घेण कालेन तपोयुक्तौ बभूवतुः}
{क्षुत्पिपासापरिश्रान्तौ जटावल्कलधारिणौ}


\threelineshloka
{मलोपचितसर्वाङ्गौ वायुभक्षौ बभूवतुः}
{आत्ममांसानि जुह्वान्तौ पादाङ्गुष्ठाग्रधिष्ठितौ}
{ऊर्ध्वबाहू चानिमिषौ दीर्घकालं धृतव्रतौ}


\twolineshloka
{तयोस्तपःप्रभावेण दीर्घकालं प्रतापितः}
{धूमं प्रमुमुचे विन्ध्यस्तद्भुतमिवाभवत्}


\twolineshloka
{ततो देवा भयं जग्मुरुग्रं दृष्ट्वा तयोस्तपः}
{तपोविघातार्थमथो देवा विघ्नानि चक्रिरे}


\twolineshloka
{रत्नैः प्रलोभयामासुः स्त्रीभिश्चोभौ पुनःपुनः}
{न च तौ चक्रतुर्भङ्गं व्रतस्य सुमहाव्रतौ}


\twolineshloka
{अथ मायां पुनर्देवास्तयोश्चक्रुर्महात्मनोः}
{भगिन्यो मातरो भार्यास्तयोश्चात्मजनस्तथा}


\twolineshloka
{प्रपात्यमाना विस्रस्ताः शूलहस्तेन रक्षसा}
{भ्रष्टाभरणकेशान्ता भ्रष्टाभरणवाससः}


\twolineshloka
{अभिभाष्य ततः सर्वास्तौ त्राहीति विचुक्रुशुः}
{न च तौ चक्रतुर्भङ्गं व्रतस्य सुमहाव्रतौ}


\twolineshloka
{यदा क्षोभं नोपयाति नार्तिमन्यतरस्तयोः}
{ततः स्त्रियस्ता भूतं च सर्वमन्तरधीयत}


\twolineshloka
{ततः पितामहः साक्षादभिगम्य महासुरौ}
{वरेण च्छ्दयामास क्वलोकहितः प्रभुः}


\twolineshloka
{ततः सुन्दोपसुन्दौ तौ भ्रातरौ दृढविक्रमौ}
{दृष्ट्वा पितामहं देवं तस्थतुः प्राञ्जली तदा}


\twolineshloka
{ऊचतुश्च प्रभुं देवं ततस्तौ सहितौ तदा}
{आवयोस्तपसाऽनेन यदि प्रीतः पितामहः}


\threelineshloka
{मायाविदावस्त्रविदौ बलिनौ कामरूपिणौ}
{उभावप्यमरौ स्यावः प्रसन्नो यदि नौ प्रभुः ॥ब्रह्मोवाच}
{}


\twolineshloka
{ऋतेऽमरत्वं युवयोः सर्वमुक्तं भविष्यति}
{अन्यद्वृणीतं मृत्योश्च विधानममरैः सम्}


\twolineshloka
{प्रभविष्याव इति यन्महदभ्युद्यतं तपः}
{युवयोर्हेतुनानेन नामरत्वं विधीयते}


\twolineshloka
{त्रैलोक्यविजयार्थाय भवद्भ्यामास्थितं तपः}
{हेतुनाऽनेन दैत्येन्द्रौ न वां कामं करोम्यहम्}


\fourlineindentedshloka
{सुन्दोपसुन्दावूचतुः}
{त्रिषु लोकेषु यद्भूतं किंचित्स्थावरजङ्गमम्}
{सर्वस्मान्नौ भयं न स्यादृतेऽन्योन्यं पितामह ॥पितामह उवाच}
{}


\threelineshloka
{यत्प्रार्थितं यथोक्तं च काममेतद्ददानि वाम्}
{मृत्योर्विधानमेतच्च यथावद्वा भविष्यति ॥नारद उवाच}
{}


\twolineshloka
{ततः पितामहो दत्त्वा वरमेतत्तदा तयोः}
{निवर्त्य तपसस्तौ च ब्रह्मलोकं जगाम ह}


\twolineshloka
{लब्ध्वा वराणि दैत्येन्द्रावथ तौ भ्रातरावुभौ}
{अवध्यौ सर्वलोकस्य स्वमेव भवनं गतौ}


\twolineshloka
{तौ तु लब्धवरौ दृष्ट्वा कृतकामौ मनस्विनौ}
{सर्वः सुहृञ्जनस्ताभ्यां प्रहर्षमुपजग्मिवान्}


\twolineshloka
{ततस्तौ तु जटा भित्त्वा मौलिनौ संबभूवतुः}
{महार्हाभरणोपेतौ विरजोम्बरधारिणौ}


\twolineshloka
{अकालकौमुदीं चैव चक्रतुः सार्वकालिकीम्}
{नित्यः प्रमुदितः सर्वस्तयोश्चैव सुहृञ्जनः}


\twolineshloka
{भक्ष्यतां भुज्यतां नित्यं दीयतां रम्यतामिति}
{गीयेतां पीयतां चेति शभ्दश्चासीद्गृहे गृहे}


\twolineshloka
{तत्रतत्र महानादैरुत्कृष्टतलनादितैः}
{हृष्टं प्रमुदितं सर्वं दैत्यानामभवत्पुरम्}


\twolineshloka
{तैस्तैर्विहारैर्बहुभिर्दैत्यानां कामरूपिणाम्}
{समाः संक्रीडतां तेषामहरेकमिवाभवत्}


\chapter{अध्यायः २३०}
\twolineshloka
{नारद उवाच}
{}


\twolineshloka
{उत्सवे वृत्तमात्रे तु त्रैलोक्याकाङ्क्षिणावुभौ}
{मन्त्रयित्वा ततः सेनां तावज्ञापयतां तदा}


\twolineshloka
{सुहृद्भिरप्यनुज्ञातौ दैत्यैर्वृद्धैश्च मन्त्रिभिः}
{कृत्वा प्रास्थानिकं रात्रौ मघासु ययतुस्तदा}


\twolineshloka
{गदापिट्टशधारिण्या शूलमुद्गरहस्तया}
{प्रस्थितौ सह वर्मिण्या महत्या दैत्यसेनया}


\twolineshloka
{मङ्गलैः स्तुतिभिश्चापि विजयप्रतिसंहितैः}
{चारणैः स्तूयमानौ तौ जग्मतुः परया मुदा}


\twolineshloka
{तावन्तरिक्षमुत्प्लुत्य दैत्यौ कामगमावुभौ}
{देवानामेव भवनं जग्मतुर्युद्दुर्मदौ}


\twolineshloka
{तयोरागमनं ज्ञात्वा वरदानं च तत्प्रभोः}
{हित्वा त्रिविष्टपं जग्मुर्ब्रह्मलोकं ततः सुराः}


\twolineshloka
{ताविन्द्रलोकं निर्जित्य यक्षरक्षोगणांस्तदा}
{खेचराण्यपि भूतानि जघ्नतुस्तीव्रविक्रमौ}


\twolineshloka
{अन्तर्भूमिगतान्नागाञ्जित्वा तौ च महारथौ}
{समुद्रवासिनीः सर्वा म्लेच्छजातीर्विजिग्यतुः}


\twolineshloka
{ततः सर्वां महीं जेतुमारब्धावुग्रशासनौ}
{सैनिकांश्च समाहूय सुतीक्ष्णं वाक्यमूचतुः}


\twolineshloka
{राजर्षयो महायज्ञैर्हव्यकव्यैर्द्विजातयः}
{तेजो बलं च देवानां वर्धन्ति श्रियं तथा}


\twolineshloka
{तेषामेवं प्रवृत्तानां सर्वेषामसुरद्विषाम्}
{संभूय सर्वैरस्माभिः कार्यः सर्वात्मना वधः}


\twolineshloka
{एवं सर्वान्समादिश्य पूर्वतीरे महोदधेः}
{क्रूरां मतिं समास्थाय जग्मतुः सर्वतोमुखौ}


\twolineshloka
{यज्ञैर्यजन्ति ये केचिद्याजयन्ति च ये द्विजाः}
{तान्सर्वान्प्रसभं हत्वा बलिनौ जग्मतुस्ततः}


\twolineshloka
{आश्रमेष्वग्निहोत्राणि मुनीनां भावितात्मनाम्}
{गृहीत्वा प्रक्षिपन्त्यप्सु विश्रब्धं सैनिकास्तयोः}


\twolineshloka
{तपोधनैश्च ये क्रुद्धैः शापा उक्ता महात्मभिः}
{नाक्रामन्त तयोस्तेऽपि वरदाननिराकृताः}


\twolineshloka
{नाक्रामन्त यदा शापा बाणा मुक्ताः शिलास्विव}
{नियमान्संपरित्यज्य व्यद्रवन्त द्विजातयः}


\twolineshloka
{पृथिव्यां ये तपःसिद्धा दान्ताः शमपरायणाः}
{तयोर्भयाद्दुद्रुवुस्ते वैनतेयादिवोरगाः}


\twolineshloka
{मथितैराश्रमैर्भग्नैर्विकीर्णकलशस्रुवैः}
{शून्यमासीज्जगत्सर्वं कालेनेव हतं तदा}


\twolineshloka
{ततो राजन्नदृश्यद्भिर्ऋषिभिश्च महासुरौ}
{उभौ विनिश्चयं कृत्वा विकुर्वाते वधैषिणौ}


\twolineshloka
{प्रभिन्नकरटौ मत्तौ भूत्वा कुञ्जररूपिणौ}
{संलीनमपि दुर्गेषु निन्यतुर्यमसादनम्}


\twolineshloka
{सिंहौ भूत्वा पुनर्व्याघ्रौ पुनश्चान्तर्हितावुभौ}
{तैस्तैरुपायैस्तौ क्रूरावृषीन्दृष्ट्वा निजघ्नतुः}


\twolineshloka
{निवृत्तयज्ञस्वाध्याया प्रनष्टनृपतिद्विजा}
{उत्सन्नोत्सवयज्ञा च बभूव वसुधा तदा}


\twolineshloka
{हाहाभूता भयार्ता च निवृत्तविपणापणा}
{निवृत्तदेवकार्या च पुण्योद्वाहविवर्जिता}


\twolineshloka
{निवृत्तकृषिगोरक्षा विध्वस्तनगराश्रमा}
{अस्थिकङ्कालसंकीर्णा भूर्बभूवोग्रदर्शना}


\twolineshloka
{निवृत्तपितृकार्यं च निर्वषट्कारमङ्गलम्}
{जगत्प्रतिभयाकारं दुष्प्रेक्ष्यमभवत्तदा}


\twolineshloka
{चन्द्रादित्यौ ग्रहास्तारा नक्षत्राणि दिवौकसः}
{जग्मुर्विषादं तत्कर्म दृष्ट्वा सुन्दोपसुन्दयोः}


\twolineshloka
{एवं सर्वा दिशो दैत्यौ जित्वा क्रूरेण कर्मणा}
{निःसपत्नौ कुरुक्षेत्रे निवेशमभिचक्रतुः}


\chapter{अध्यायः २३१}
\twolineshloka
{नारद उवाच}
{}


\twolineshloka
{ततो देवर्षयः सर्वे सिद्धाश्च परमर्षयः}
{जग्मुस्तदा परमार्तिं दृष्ट्वा तत्कदनं महत्}


\twolineshloka
{तेऽभिजग्मुर्जितक्रोधा जितात्मानो जितेन्द्रियाः}
{पितामहस्य भनं जगतः कृपया तदा}


\twolineshloka
{ततो ददृशुरासीनं सह देवैः पितामहम्}
{सिद्धैर्ब्रह्मर्षिभिश्चैव समन्तात्परिवारितम्}


\twolineshloka
{तत्र देवो महादेवस्तत्राग्निर्वायुना सह}
{चन्द्रादित्यौ च शक्रश्च पारमेष्ठ्यास्तथर्षयः}


\twolineshloka
{वैखानसा वालखिल्या वानप्रस्था मरीचिपाः}
{अजाश्चैवाविमूढाश्च तेजोगर्भास्तपस्विनः}


\twolineshloka
{ऋषयः सर्व एवैते पितामहमुपागमन्}
{ततोऽभिगम्य ते दीनाः सर्व एव महर्षयः}


\twolineshloka
{सुन्दोपसुन्दयौः कर्म सर्वमेव शशंसिरे}
{यथा हृतं यथा चैव कृतं येन क्रमेण च}


\twolineshloka
{न्यवेदयंस्ततः सर्वमखिलेन पितामहे}
{ततो देवगणाः सर्वे ते चैव परमर्षयः}


\twolineshloka
{तमेवार्थं पुरस्कृत्य पितामहमचोदयन्}
{ततः पितामहः श्रुत्वा सर्वेषां तद्वचस्तदा}


\twolineshloka
{मुहूर्तमिव संचिन्त्य कर्तव्यस्य च निश्चयम्}
{तयोर्वधं समुद्दिश्य विश्वकर्माणमाह्वयत्}


\twolineshloka
{दृष्ट्वा च विश्वकर्माणं व्यादिदेश पितामहः}
{सृज्यतां प्रार्थनीयैका प्रमदेति महातपाः}


\twolineshloka
{पितामहं नमस्कृत्य तद्वाक्यमभिनन्द्य च}
{निर्ममे योषितं दिव्यां चिन्तयित्वा पुनःपुनः}


\twolineshloka
{त्रिषु लोकेषु यत्किंचिद्भूतं स्थावरजङ्गमम्}
{समानयद्दर्शनीयं तत्तदत्र स विश्ववित्}


\twolineshloka
{कोटिशश्चैव रत्नानि तस्या गात्रे न्यवेशत्}
{तां रत्नसङ्घातमयीमसृजद्देवरूपिणीम्}


\twolineshloka
{सा प्रयत्नेन महता निर्मिता विश्वकर्मणा}
{त्रिषु लोकेषु नारीणां रूपेणाप्रतिमाभवत्}


\twolineshloka
{न तस्याः सूक्ष्ममप्यस्ति यद्गात्रे रूपसंपदा}
{नियुक्ता यत्र वा दृष्टिर्न सज्जति निरीक्षताम्}


\twolineshloka
{सा विग्रहवतीव श्रीः कामरूपा वपुष्मती}
{`पितामहमुपातिष्ठत्किं करोमीति चाब्रवीत्}


\twolineshloka
{प्रीतो भूत्वा स दृष्ट्वैव प्रीत्या चास्यै वरं ददौ}
{कान्तत्वं सर्वभूतानां साश्रियानुत्तमं वपुः}


\twolineshloka
{सा तेन वरदानेन कर्तुश्च क्रियया तदा}
{'जहार सर्वभूतानां चक्षूंषि च मनांसि च}


\twolineshloka
{तिलंतिलं समानीय रत्नानां यद्विनिर्मिता}
{तिलोत्तमेति तत्तस्या नाम चक्रे पितामहः}


\threelineshloka
{ब्रह्माणं सा नमस्कृत्य प्राञ्जलिर्वाक्यमब्रवीत्}
{किं कार्यं मयि भूतेश येनास्म्यद्येह निर्मिता ॥पितामह उवाच}
{}


\twolineshloka
{गच्छ सुन्दोपसुन्दाभ्यामसुराभ्यां तिलोत्तमे}
{प्रार्थनीयेन रूपेण कुरु भद्रे प्रलोभनम्}


\threelineshloka
{त्वत्कृते दर्शादेव रूपसंपत्कृतेन वै}
{विरोधः स्याद्यथा ताभ्यामन्योन्येन तथा कुरु ॥नारद उवाच}
{}


\twolineshloka
{सा तथेति प्रतिज्ञाय नमस्कृत्य पितामहम्}
{चकार मण्डलं तत्र विबुधानां प्रदक्षिणम्}


\twolineshloka
{प्राङ्मुखो भगवानास्ते दक्षिणेन महेश्वरः}
{देवाश्चैवोत्तरेणासन्सर्वतस्त्वृषयोऽभवन्}


\twolineshloka
{कुर्वन्त्यां तु तदा तत्र मण्डलं तत्प्रदक्षिणम्}
{इन्द्रः स्थाणुश्च भगवान्धैर्येण तु परिच्युतौ}


\twolineshloka
{द्रष्टुकामस्य चात्यर्थं गतायां पार्श्वतस्तथा}
{अन्यदञ्चितपद्माक्षं दक्षिणं निःसृतं मुखम्}


\twolineshloka
{पृष्ठतः परिवर्तन्त्यां पश्चिमं निःसृतं मुखम्}
{गतायां चोत्तरं पार्श्वमुत्तरं निःसृतं मुखम्}


\twolineshloka
{महेन्द्रस्यापि नेत्राणां पृष्ठतः पार्श्वतोग्रतः}
{रक्तान्तानां विशालानां सहस्रं सर्वतोऽभवत्}


\twolineshloka
{एवं चतुर्मुखः स्थाणुर्महादेवोऽभवत्पुरा}
{तथा सहस्रनेत्रश्च बभूव बलसूदनः}


\twolineshloka
{तथा देवनिकायानां महर्षीणां च सर्वशः}
{मुखानि चाभ्यवर्तन्त येन याति तिलोत्तमा}


\twolineshloka
{तस्या गात्रे निपतिता दृष्टिस्तेषां महात्मनाम्}
{सर्वेषामेव भूयिष्ठमृते देवं पितामहम्}


\twolineshloka
{गच्छन्त्यां तु तया सर्वे देवाश्च परमर्षयः}
{कृतमित्येव तत्कार्यं मेनिरे रूपसंपदा}


\threelineshloka
{तिलोत्तमायां तस्यां तु गतायां लोकभावनः}
{`कृतं कार्यमिति श्रीमानब्रवीच्च पितामहः}
{'सर्वान्विसर्जयामास देवानृषिगणांश्च तान्}


\chapter{अध्यायः २३२}
\twolineshloka
{नारद उवाच}
{}


\twolineshloka
{जित्वा तु पृथिवीं दैत्यौ निःसपत्नौ गतव्यथौ}
{कृत्वा त्रैलोक्यमव्यग्रं कृतकृत्यौ बभूवतुः}


\twolineshloka
{देवगन्धर्वयक्षाणां नागपार्थिवरक्षसाम्}
{आदाय सर्वरत्नानि परां तुष्टिमुपागतौ}


% Check verse!
यदा न प्रतिषेद्धारस्तयोः सन्तीह केचन ॥निरुद्योगौ तदा भूत्वा विजह्रातेऽमराविव
\twolineshloka
{स्त्रीभिर्माल्यैश्च गन्धैश्च भक्ष्यभोज्यैः सुपुष्कलैः}
{पानैश्च विविधैर्हृद्यैः परां प्रीतिमवापतुः}


\twolineshloka
{अन्तःपुरवनोद्याने पर्वतेषु वनेषु च}
{यथेप्सितेषु देशेषु विजह्रातेऽमराविव}


\twolineshloka
{ततः कदाचिद्विन्ध्यस्य प्रस्थे समशिलातले}
{पुष्पिताग्रेषु सालेषु विहारमभिजग्मतुः}


\twolineshloka
{दिव्येषु सर्वकामेषु समानीतेषु तावुभौ}
{वरासनेषु संहृष्टौ सह स्त्रीभिर्निषीदतुः}


\twolineshloka
{ततो वादित्रनृत्ताभ्यामुपातिष्ठन्त तौ स्त्रियः}
{गीतैश्च स्तुतिसंयुक्तैः प्रीत्या समुपजग्मिरे}


\twolineshloka
{ततस्तिलोत्तमा तत्र वने पुष्पाणि चिन्वती}
{वेषं सा क्षिप्तमाधाय रक्तेनैकेन वाससा}


\twolineshloka
{नदीतीरेषु जातान्सा कर्णिकारान्प्रचिन्वती}
{शनैर्जगाम तं देशं यत्रास्तां तौ महासुरौ}


\twolineshloka
{तौ तु पीत्वा वरं पानं मदरक्तान्तलोचनौ}
{दृष्ट्वैव तां वरारोहां व्यथितौ संबभूवतुः}


\twolineshloka
{तावुत्थायासनं हित्वा जग्मतुर्यत्र सा स्थिता}
{उभौ च कामसंमत्तावुभौ प्रार्थयतश्च ताम्}


\twolineshloka
{दक्षिणे तां करे सुभ्रूं सुन्दो जग्राह पाणिना}
{उपसुन्दोपि जग्राह वामे पाणौ तिलोत्तमाम्}


\twolineshloka
{वरप्रदानमत्तौ तावौरसेन बलेन च}
{धनरत्नमदाभ्यां च सुरापानमदेन च}


\twolineshloka
{सर्वैरेतैर्मदैर्मत्तावन्योन्यं भ्रुकुटीकृतौ}
{मदकामसमाविष्टौ परस्परमथोचतुः}


\twolineshloka
{मम भार्या तव गुरुरिति सुन्दोऽभ्यभाषत}
{मम भार्या तव वधूरुपसुन्दोऽभ्यभाषत}


\twolineshloka
{नैषा तव ममैषेति ततस्तौ मन्युराविशत्}
{तस्या रूपेण संमत्तौ विगतस्नेहसौहृदौ}


\twolineshloka
{तस्या हेतोर्गदे भीमे संगृह्णीतामुभौ तदा}
{प्रगृह्य च गदे भीमे तस्यां तौ काममोहितौ}


\twolineshloka
{अहंपूर्वमहंपूर्वमित्यन्योन्यं निजघ्नतुः}
{तौ गदाभिहतौ भीमौ पेततुर्धरणीतले}


\twolineshloka
{रुधिरेणावसिक्ताङ्गौ द्वाविवार्कौ नभश्च्युतौ}
{ततस्ता विद्रुता नार्यः स च दैत्यगणस्तथा}


\twolineshloka
{पातालमगमत्सर्वो विषादभयकम्पितः}
{ततः पितामहस्तत्र सहदेवैर्महर्षिभिः}


\twolineshloka
{आजगाम विशुद्धात्मा पूजयंश्च तिलोत्तमाम्}
{वरेण च्छन्दयामास भगवान्प्रपितामहः}


\twolineshloka
{वरं दित्सुः स तत्रैनां प्रीतः प्राह पितामहः}
{आदित्यचरिताँल्लोकान्विचरिष्यसि भामिनि}


\twolineshloka
{तेजसा च सुदृष्टां त्वां न करिष्यति कश्चन}
{एवं तस्यै वरं दत्वा सर्वलोकपितामहः}


\twolineshloka
{इन्द्रे त्रैलोक्यमाधाय ब्रह्मलोकं गतः प्रभुः}
{एवं तौ सहितौ भऊत्वा सर्वार्थेष्वेकनिश्चयौ}


\twolineshloka
{तिलोत्तमार्थं संक्रुद्धावन्योन्यमभिजघ्नतुः}
{तस्माद्ब्रवीमि वः स्नेहात्सर्वाभरतसत्तमाः}


\threelineshloka
{यथा वो नात्र भेदः स्यात्सर्वेषां द्रौपदीकृते}
{तथा कुरुत भद्रं वो मम चेत्प्रियमिच्छथ ॥वैशंपायन उवाच}
{}


\threelineshloka
{एवमुक्ता महात्मानो नारदेन महर्षिणा}
{समयं चक्रिरे राजंस्तेऽन्योन्यवशमागताः}
{समक्षं तस्य देवर्षेर्नारदस्यामितौजसः}


\twolineshloka
{`एकैकस्य गृहे कृष्णा वसेद्वर्षमकल्मषा'द्रौपद्या नः सहासीनानन्योन्यं योऽभिदर्शयेत्}
{स नो द्वादश मासानि ब्रह्मचारी वने वसेत्}


\twolineshloka
{कृते तु समये तस्मिन्पाण्डवैर्धर्मचारिभिः}
{नारदोऽप्यगमत्प्रीत इष्टं देशं महामुनिः}


\twolineshloka
{एवं तैः समयः पूर्वं कृतो नारदचोदितैः}
{न चाभिद्यन्त ते सर्वे तदान्योन्येन भारत}


\twolineshloka
{`अभ्यनन्दन्त ते सर्वे तदान्योन्यं च पाण्डवाः}
{एतद्विस्तरशः सर्वमाख्यातं ते नराधिप}


% Check verse!
काले च तस्मिन्संपन्ने यथावज्जनमेजय ॥'
\chapter{अध्यायः २३३}
\twolineshloka
{वैशंपायन उवाच}
{}


\twolineshloka
{एवं ते समयं कृत्वा न्यवसंस्तत्र पाण्डवाः}
{वशे शस्त्रप्रतापेन कुर्वन्तोऽन्यान्महीक्षितः}


\twolineshloka
{तेषां मनुजसिंहानां पञ्चानाममितौजसाम्}
{बभूव कृष्णा सर्वेषां पार्थानां वशवर्तिनी}


\twolineshloka
{ते तया तैश्च सा वीरैः पतिभिः सह पञ्चभिः}
{बभूव परमप्रीता नागैरिव सरस्वती}


\twolineshloka
{वर्तमानेषु धर्मेण पाण्डवेषु महात्मसु}
{व्यवर्धन्कुरवः सर्वे हीनदोषाः सुखान्विताः}


\twolineshloka
{अथ दीर्घेण कालेन ब्राह्मणस्य विशांपते}
{कस्यचित्तस्करा जह्रुः केचिद्गा नृपसत्तम}


\twolineshloka
{ह्रियमाणे धने तस्मिन्ब्राह्मणः क्रोधमूर्च्छितः}
{आगम्य खाण्डवप्रस्थमुदक्रोशत्स पाण्डवान्}


\twolineshloka
{ह्रियते गोधनं क्षुद्रैर्नृशंसैरकृतात्मभिः}
{प्रसह्य चास्मद्विषयादभ्यधावत पाण्डवाः}


\twolineshloka
{ब्राह्मणस्य प्रशान्तस्य हविर्ध्वाङ्क्षैः प्रलुप्यते}
{शार्दूलस्य गुहां शून्यां नीचः क्रोष्टाभिमर्दति}


\twolineshloka
{अरक्षितारं राजानं बलिषद्भागहारिणम्}
{तमाहुः सर्वलोकस्य समग्रं पापचारिणम्}


\threelineshloka
{ब्राह्मणस्वे हृते चोरैर्धर्मार्थे च विलोपिते}
{रोरूयमाणे च मयि क्रियतां हस्तधारणा ॥वैशंपायन उवाच}
{}


\twolineshloka
{रोरूयमाणस्याभ्याशे भृशं विप्रस्य पाण्डवः}
{तानि वाक्यानि शुश्राव कुन्तीपुत्रो धनञ्जयः}


\twolineshloka
{श्रुत्वैव च महाबाहुर्मा भैरित्याह तं द्विजम्}
{आयुधानि च यत्रासन्पाण्डवानां महात्मनां}


\twolineshloka
{कृष्णया सह तत्रास्ते धर्मराजो युधिष्ठिरः}
{संप्रवेशाय चाशक्तो गमनाय च पाण्डवः}


\twolineshloka
{तस्य चार्तस्य तैर्वाक्यैश्चोद्यमानः पुनःपुनः}
{आक्रन्दे तत्र कौन्तेयश्चिन्तयामास दुःखितः}


\twolineshloka
{ह्रियमाणे धने तस्मिन्ब्राह्मणस्य तपस्विनः}
{अश्रुप्रमार्जनं तस्य कर्तव्यमिति निश्चयः}


\twolineshloka
{उपर्रेक्षणजोऽधर्मः सुमहान्स्यान्महीपतेः}
{यद्यस्य रुदतो द्वारि न करोम्यद्य रक्षणम्}


\twolineshloka
{अनास्तिक्यं च सर्वेषामस्माकमपि रक्षणे}
{प्रतितिष्ठेत लोकेऽस्मिन्नधर्मश्चैव नो भवेत्}


\twolineshloka
{अनापृच्छय तु राजानं गते मयि न संशयः}
{अजातशत्रोर्नृपतेर्मयि चैवानृतं भवेत्}


\twolineshloka
{अनुप्रवेशे राज्ञस्तु वनवासो भवेन्मम}
{सर्वमन्यत्परिहृतं धर्षणात्तु महीपतेः}


\twolineshloka
{अधर्मो वै महानस्तु वने वा मरणं मम}
{शरीरस्य विनाशेन धर्म एव विशिष्यते}


\twolineshloka
{एवं विनिश्चित्य ततः कुन्तीपुत्रो धनञ्जयः}
{अनुप्रविश्य राजानमापृच्छय च विशांपते}


\twolineshloka
{`मुखमाच्छाद्य निबिडमुत्तरीयेण वाससा}
{अग्रजं चार्जुनो गेहादभिवाद्याशु निःसृतः ॥'}


\twolineshloka
{धनुरादाय संहृष्टो ब्राह्मणं प्रत्यभाषत}
{ब्राह्मणा गम्यतां शीघ्रं यावत्परधनैर्षिणः}


\twolineshloka
{न दूरे ते गताः क्षुद्रास्तावद्गच्छावहे सह}
{यावन्निवर्तयाम्यद्य चोरहस्ताद्धनं तव}


\twolineshloka
{सोऽनुसृत्य महाबाहुर्धन्वी वर्मी रथी ध्वजी}
{शरैर्विध्वस्य तांश्चोरानवजित्य च तद्धनम्}


\twolineshloka
{ब्राह्मणं समुपाकृत्य यशः प्राप्य च पाण्डवः}
{ततस्तद्गेधनं पार्थो दत्त्वा तस्मै द्विजातये}


\twolineshloka
{आजगाम पुरं वीरः सव्यसाची धनञ्जयः}
{सोऽभिवाद्य गुरून्सर्वान्सर्वैश्चाप्यभिनन्दितः}


\twolineshloka
{धर्मराजमुवाचेदं व्रतमादिश मे प्रभो}
{समयः समतिक्रान्तो भवत्संदर्शने मया}


\twolineshloka
{वनवासं गमिष्यामि समयो ह्येष नः कृतः}
{इत्युक्तो धर्मरास्तु सहसा वाक्यमप्रियम्}


\twolineshloka
{कथमित्यब्रवीद्वाचा शोकार्तः सज्जमानया}
{युधिष्ठिरो गुडाकेशं भ्राता भ्रातरमच्युतम्}


\twolineshloka
{उवाच दीनो राजा च धनञ्जयमिदं वचः}
{प्रमाणमस्मि यदि ते मत्तः शृणु वचोऽनघ}


\twolineshloka
{अनुप्रवेशे यद्वीर कृतवांस्त्वं ममाप्रियम्}
{सर्वं तदनुजानामि व्यलीकं न च मे हृदि}


\twolineshloka
{गुरोरनुप्रवेशो हि नोपघातो यवीयसः}
{यवीयसोऽनुप्रवेशो ज्येष्ठस्य विधिलोपकः}


\threelineshloka
{निवर्तस्व महाबाहो कुरुष्व वचनं मम}
{न हि ते धर्मलोपोऽस्ति न च ते धर्पणा कृता ॥अर्जुन उवाच}
{}


\threelineshloka
{न व्याजेन चरेद्धर्ममिति मे भवतः श्रुतम्}
{न सत्याद्विचलिष्यामि सत्येनायुधमालभे ॥वैशंपायन उवाच}
{}


\twolineshloka
{सोऽभ्यनुज्ञाय राजानं वनचर्याय दीक्षितः}
{वने द्वादश मासांस्तु वासायानुजगाम ह}


\chapter{अध्यायः २३४}
\twolineshloka
{वैशंपायन उवाच}
{}


\twolineshloka
{तं प्रयान्तं महाबाहुं कौरवाणां यशस्करम्}
{अनुजग्मुर्महात्मानो ब्राह्मणा वेदपारगाः}


\twolineshloka
{वेदवेदाङ्गविद्वासस्तथैवाध्यात्मचिन्तकाः}
{भैक्षाश्च भगवद्भक्ताः सूताः पौराणिकाश्च ये}


\twolineshloka
{कथकाश्चापरे राजञ्श्रमणाश्च वनौकसः}
{दिव्याख्यानानि ये चापि पठन्ति मधुरं द्विजाः}


\twolineshloka
{एतैश्चान्यैश्च बहुभिः सहायैः पाण्डुनन्दनः}
{वृतः श्लक्ष्णकथैः प्रायान्मरुद्भिरिव वासवः}


\twolineshloka
{रमणीयानि चित्राणि वनानि च सरांसि च}
{सरितः सागरांश्चैव देशानपि च भारत}


\twolineshloka
{पुण्यान्यपि च तीर्थानि ददर्श भरतर्षभः}
{स गङ्गाद्वारमाश्रित्य निवेशमकरोत्प्रभुः}


\twolineshloka
{तत्र तस्याद्भुतं कर्म शृणु त्वं जनमेजय}
{कृतवान्यद्विशुद्धात्मा पाण्डूनां प्रवरो हि सः}


\twolineshloka
{निविष्टे तत्र कौन्तेये ब्राह्मणेषु च भारत}
{अग्निहोत्राणि विप्रास्ते प्रादुश्चक्रुरनेकशः}


\twolineshloka
{तेषु प्रबोध्यमानेषु ज्वलितेषु हुतेषु च}
{कृतपुष्पोपहारेषु तीरान्तरगतेषु च}


\twolineshloka
{कृताभिषेकैर्विद्वद्भिर्नियतैः सत्पथे स्थितैः}
{शुशुभेऽतीव तद्राजन्गङ्गाद्वारं महात्मभिः}


\twolineshloka
{तथा पर्याकुले तस्मिन्निवेशे पाण्डवर्षभः}
{अभिषेकाय कौन्तेयो गङ्गामवततार ह}


\twolineshloka
{तत्राभिषेकं कृत्वा स तर्पयित्वा पितामहान्}
{उत्तितीर्षुर्जलाद्राजन्नग्निकार्यचिकीर्षया}


\twolineshloka
{अपकृष्टो महाबाहुर्नागराजस्य कन्यया}
{अन्तर्जले महाराज उलूप्या कामयानया}


\twolineshloka
{ददर्श पाण्डवस्तत्र पावकं सुसमाहितः}
{कौरव्यस्याथ नागस्य भवने परमार्चिते}


\twolineshloka
{तत्राग्निकार्यं कृतवान्कुन्तीपुत्रो धनञ्जयः}
{अशङ्कमानेन हुतस्तेनातुष्यद्धुताशनः}


\twolineshloka
{अग्निकार्यं स कृत्वा तु नागराजसुतां तदा}
{प्रसहन्निव कौन्तेय इदं वचनमब्रवीत्}


\threelineshloka
{किमिदं साहसं भीरु कृतवत्यसि भामिनि}
{कश्चायं सुभगे देशः का च त्वं कस्य वात्मजा ॥उलूप्युवाच}
{}


\twolineshloka
{ऐरावतकुले जातः कौरव्यो नाम पन्नगः}
{तस्यास्मि दुहिता राजन्नुलूपी नाम पन्नगी}


\twolineshloka
{साऽहं त्वामभिषेकार्थमवतीर्णं समुद्गाम्}
{दृष्ट्वैव पुरुषव्याघ्र कन्दर्पेणाभिमूर्च्छिता}


\threelineshloka
{तां मामनङ्गग्लपितां त्वत्कृते कुरुनन्दन}
{अनन्यां नन्दयस्वाद्य प्रदानेनात्मनोऽनघ ॥अर्जुन उवाच}
{}


\twolineshloka
{ब्रह्मचर्यमिदं भद्रे मम द्वादशमासिकम्}
{धर्मराजेन चादिष्टं नाहमस्मि स्वयं वशः}


\twolineshloka
{तव चापि प्रियं कर्तुमिच्छामि जलचारिणि}
{अनृतं नोक्तपूर्वं च मया किंचन कर्हिचित्}


\threelineshloka
{कथं च नानृतं मे स्यात्तव चापि प्रियं भवेत्}
{न च पीड्येत मे धर्मस्तथा कुर्या भुजङ्गमे ॥उलूप्युवाच}
{}


\twolineshloka
{जानाम्यहं पाण्डवेय यथा चरसि मेदिनीम्}
{यथा च ते ब्रह्मचर्यमिदमादिष्टवान्गुरुः}


\twolineshloka
{परस्परं वर्तमानान्द्रुपदस्यात्मजां प्रति}
{यो नोऽनुप्रविशेन्मोहात्स वै द्वादशमासिकम्}


\twolineshloka
{वने चरेद्ब्रह्मचर्यमिति वः समयः कृतः}
{तदिदं दौपदीहेतोरन्योन्यस्य प्रवासनम्}


\twolineshloka
{कृतवांस्तत्र धर्मार्थमत्र धर्मो न दुष्यति}
{परित्राणं च कर्तव्यमार्तानां पृथुलोचन}


\twolineshloka
{कृत्वा मम परित्राणं तव धर्मो न लुप्यते}
{यदि वाप्यस्य धर्मस्य सूक्ष्मोऽपि स्याद्व्यतिक्रमः}


\twolineshloka
{स च ते धर्म एव स्याद्दत्वा प्राणान्ममार्जुन}
{भक्तां च भज मां पार्थ सतामेतन्मतं प्रभो}


\twolineshloka
{न करिष्यसि चेदेवं मृतां मामुपधारय}
{प्राणदानान्महाबाहो चर धर्ममनुत्तमम्}


\twolineshloka
{शरणं च प्रपन्नास्मि त्वामद्य पुरुषोत्तम}
{दीनाननाथान्कौन्तेय परिरक्षसि नित्यशः}


\fourlineindentedshloka
{साऽहं शऱणमभ्येमि रोरवीमि च दुःखिता}
{याचे त्वां चाभिकामाहं तस्मात्कुरु मम प्रियम्}
{स त्वमात्मप्रदानेन सकामां कर्तुमर्हसि ॥वैशंपायन उवाच}
{}


\twolineshloka
{एवमुक्तस्तु कौन्तेयः पन्नगेश्वरकन्यया}
{कृतवांस्तत्तथा सर्वं धर्ममुद्दिश्य कारणम्}


\twolineshloka
{स नागभवने रात्रिं तामुषित्वा प्रतापवान्}
{`पुत्रमुत्पादयामास स तस्यां सुमनोहरम्}


\twolineshloka
{इरावन्तं महाभागं महाबलपराक्रमम्}
{'उदितेऽभ्युत्थितः सूर्ये कौरव्यस्य निवेशनात्}


\twolineshloka
{आगतस्तु पुनस्तत्र गङ्गाद्वारं तया सह}
{परित्यज्य गता साध्वी उलूपी निजमन्दिरं}


\twolineshloka
{दत्त्वा वरमजेयत्वं जले सर्वत्र भारत}
{साध्या जलचराः सर्वे भविष्यन्ति न संशयः}


\chapter{अध्यायः २३५}
\twolineshloka
{वैशंपायन उवाच}
{}


\twolineshloka
{कथयित्वा च तत्सर्वं ब्राह्मणेभ्यः स भारत}
{प्रययौ हिमवत्पार्श्वं ततो वज्रधरात्मजः}


\twolineshloka
{अगस्त्यवटमासाद्य वसिष्ठस्य च पर्वतम्}
{भृगुतुङ्गे च कौन्तेयः कृतवाञ्शौचमात्मनः}


\twolineshloka
{प्रददौ गोसहस्राणि सुबहूनि च भारत}
{निवेशांश्च द्विजातिभ्यः सोऽददत्कुरुसत्तमः}


\twolineshloka
{हिरण्यबिन्दोस्तीर्थे च स्नात्वा पुरुषसत्तमः}
{दृष्टवान्पाण्डवश्रेष्ठः पुण्यान्यायतनानि च}


\twolineshloka
{अवतीर्य नरश्रेष्ठो ब्राह्मणैः सह भारत}
{प्राचीं दिशमभिप्रेप्सुर्जगाम भरतर्षभः}


\twolineshloka
{आनुपूर्व्येण तीर्थानि दृष्टवान्कुरुसत्तमः}
{नदीं चोत्पलिनीं रम्यामरण्यं नैमिषं प्रति}


\twolineshloka
{नन्दामपरनन्दां च कौशिकीं च यशस्विनीम्}
{महानदीं गयां चैव गङ्गामपि च भारत}


\twolineshloka
{एवं तीर्थानि सर्वाणि पश्यमानस्तथाश्रमान्}
{आत्मनः पावनं कुर्वन्ब्राह्मणेभ्यो ददौ च गाः}


\twolineshloka
{अङ्गवङ्गकलिङ्गेषु यानि तीर्थानि कानिचित्}
{जगाम तानि सर्वाणि पुण्यान्यायतनानि च}


\threelineshloka
{दृष्ट्वा च विधिवत्तानि धनं चापि ददौ ततः}
{कलिङ्गराष्ट्रद्वारेषु ब्राह्मणाः पाण्डवानुगाः}
{अभ्यनुज्ञाय कौन्तेयमुपावर्तन्त भारत}


\twolineshloka
{स तु तैरभ्यनुज्ञातः कुन्तीपुत्रो धनञ्जयः}
{सहायैरल्पकैः शूरः प्रययौ यत्र सागरः}


\twolineshloka
{स कलिङ्गानतिक्रम्य देशानायतनानि च}
{हर्म्याणि रमणीयानि प्रेक्षणाणो ययौ प्रभुः}


\twolineshloka
{महेन्द्रपर्वतं दृष्ट्वा तापसैरुपशोभितम्}
{`गोदावर्यां ततः स्नात्वा तामतीत्य महाबलः}


\twolineshloka
{कावेरीं तां समासाद्य सङ्गमे सागरस्य च}
{स्नात्वा संपूज्य देवांश्च पितॄंश्च मुनिभिः सह'}


% Check verse!
समुद्रतीरेण शनैर्मणलूरं जगाम ह
\twolineshloka
{तत्र सर्वाणि तीर्थानि पुण्यान्यायतनानि च}
{अभिगम्य महाबाहुरभ्यगच्छन्महीपतिम्}


\twolineshloka
{मणलूरेश्वरं राजन्धर्मज्ञं चित्रवाहनम्}
{तस्य चित्राङ्गदा नाम दुहिता चारुदर्शना}


\twolineshloka
{तां ददर्श पुरे तस्मिन्विचरन्तीं यदृच्छया}
{दृष्ट्वा च तां वरारोहां चकमे चैत्रवाहनीम्}


\twolineshloka
{अभिगम्य च राजानमवदत्स्वं प्रयोजनम्}
{देहि मे खल्विमां राजन्क्षत्रियाय महात्मने}


\twolineshloka
{तच्छ्रुत्वा त्वब्रवीद्राजा कस्य पुत्रोऽसि नाम किम्}
{उवाच तं पाण्डवोऽहं कुन्तीपुत्रो धनञ्जयः}


\twolineshloka
{तमुवाचाथ राजा स सान्त्वपूर्वमिदं वचः}
{राजा प्रभञ्जनो नाम कुलेऽस्मिन्संबभूव ह}


\twolineshloka
{अपुत्रः प्रसवेनार्थी तपस्तेपे स उत्तमम्}
{उग्रेण तपसा तेन देवदेवः पिनाकधृक्}


\twolineshloka
{ईश्वरस्तोषितः पार्थ देवदेवः उमापतिः}
{स तस्मै भघवान्प्रादादेकैकं प्रसवं कुले}


\twolineshloka
{एकैकः प्रसवस्तस्माद्भवत्यस्मिन्कुले सदा}
{तेषां कुमाराः सर्वेषां पूर्वेषां मम जज्ञिरे}


\twolineshloka
{एका च मम कन्येयं कुलस्योत्पादनी भृशम्}
{पुत्रो ममायमिति मे भावना पुरुषर्षभ}


\twolineshloka
{पुत्रिकाहेतुविधिना संज्ञिता भरतर्षभ}
{तस्मादेकः सुतो योऽस्यां जायते भारत त्वया}


\twolineshloka
{एतच्छुल्कं भवत्वस्याः कुलकृज्जायतामिह}
{एतेन समयेनेमां प्रतिगृह्णीष्व पाण्डव}


\threelineshloka
{स तथेति प्रतिज्ञाय तां कन्यां प्रतिगृह्य च}
{`मासे त्रयोदशे पार्थः कृत्वा वैवाहिकीं क्रियाम्}
{'उवास नगरे तस्मिन्मासांस्त्रीन्स तया सह}


\chapter{अध्यायः २३६}
\twolineshloka
{वैशंपायन उवाच}
{}


\twolineshloka
{ततः समुद्रे तीर्थानि दक्षिमे भरतर्षभः}
{अभ्यगच्छत्सुपुण्यानि सोभितानि तपस्विभिः}


\twolineshloka
{वर्जयन्ति स्म तीर्तानि तत्र पञ्च सम तापसाः}
{अवकीर्णानि यान्यासन्पुरस्तात्तु तपस्विभिः}


\twolineshloka
{अगस्त्यतीर्थं सौभद्रं पौलोमं च सुपावनम्}
{कारन्धमं प्रसन्नं च महमेधफलं च तत्}


\twolineshloka
{भारद्वाजस्य तीर्थं तु पापप्रशमनं महत्}
{एतानि पञ्च तीर्थानि ददर्श कुरुसत्तमः}


\twolineshloka
{विविक्तान्युपलक्ष्याथ तानि तीर्थानि पाण्डवः}
{दृष्ट्वा च वर्ज्यमानानि मुनिभिर्धर्मबुद्धिभिः}


\threelineshloka
{तपस्विनस्ततोऽपृच्छत्प्राञ्जलिः कुरुनन्दनः}
{तीर्थानीमानि वर्ज्यन्ते किमर्थं ब्रह्मवादिभिः ॥तापसा ऊचुः}
{}


\threelineshloka
{ग्राहाः पञ्च वसन्त्येषु हरन्ति च तपोधनान्}
{तत एतानि वर्ज्यन्ते तीर्थानि कुरुनन्दन ॥वैशंपायन उवाच}
{}


\twolineshloka
{तेषां श्रुत्वा महाबाहुर्वार्यमाणस्तपोधनैः}
{जगाम तानि तीर्थानि द्रष्टुं पुरुषसत्तमः}


\twolineshloka
{ततः सौभद्रमासाद्य महर्षेस्तीर्थमुत्तमम्}
{विगाह्य सहसा शूरः स्नानं चक्रे परन्तपः}


\twolineshloka
{अथ तं पुरुषव्याघ्रमन्तर्जलचरो महान्}
{जग्राह चरणे ग्राहः कुन्तीपुत्रं धनञ्जयम्}


\twolineshloka
{स तमादाय कौन्तेयो विस्फुरन्तं जलेचरम्}
{उदतिष्ठन्महाबाहुर्बलेन बलिनां वरः}


\twolineshloka
{उत्कृष्ट एव ग्राहस्तु सोऽर्जुनेन यशस्विना}
{बभूव नारी कल्याणी सर्वाभरणभूषइता}


\twolineshloka
{दीप्यमाना श्रिया राजन्दिव्यरूपा मनोरमा}
{तदद्भुतं महद्दृष्ट्वा कुन्तीपुत्रो धनञ्जयः}


\twolineshloka
{तां स्त्रियं परमप्रीत इदं वचनमब्रवीत्}
{का वै त्वमसि कल्याणिकुतो वाऽसि जलेचरी}


\threelineshloka
{किमर्थं च महत्पापमिदं कृतवती पुरा}
{वर्गोवाच}
{अप्सराऽस्मि महाबाहो देवारण्यविहारिणी}


\twolineshloka
{इष्टा धनपतेर्नित्यं वर्गा नाम महाबल}
{मम सख्यश्चतस्रोऽन्याः सर्वाः कामगमाः शुभाः}


\twolineshloka
{ताभिः सार्धं प्रयाताऽस्मि लोकपालनिवेशनम्}
{ततः पश्यामहे सर्वा ब्राह्मणं संशितव्रतम्}


\twolineshloka
{रूपवन्तमधीयानमेकमेकान्तचारिणम्}
{तस्यैव तपसा राजंस्तद्वयं तेजसा वृतम्}


\twolineshloka
{आदित्य इव तं देशं इत्वं सर्व व्यकाशयत्}
{तस्य दृष्ट्वा तपस्तादृग्रूपचाद्भुतमुत्तमम्}


\twolineshloka
{अवतीर्णाः स्म तं देशं तपोविघ्नचिकीर्षया}
{अहं च सौरभेयी च समीची बुद्बुदा लता}


\twolineshloka
{यौगपद्येन तं विप्रमभ्यगच्छाम भारत}
{गायन्त्योऽथहसन्त्यश्च लोभयित्वा च तं द्विजं}


\twolineshloka
{स च नास्मासु कृतवान्मनो वीर कथंचन}
{नाकम्पत महातेजाः स्थितस्तपसि निर्मले}


\twolineshloka
{सोऽशपत्कुपितोऽस्मांस्तु ब्राह्मणः क्षत्रियर्षभ}
{ग्राहभूता जले यूयं चरिष्यथ शतं समाः}


\chapter{अध्यायः २३७}
\twolineshloka
{वर्गोवाच}
{}


\twolineshloka
{ततो वयं प्रव्यथिताः सर्वा भारतसत्तम}
{अयाम शरणं विप्रं तं तपोधनमच्युतम्}


\twolineshloka
{रूपेण वयसा चैव कन्दर्पेण च दर्पिताः}
{अयुक्तं कृतवत्यः स्म क्षन्तुमर्हसि नो द्विज}


\twolineshloka
{एष एव वधोऽस्माकं स्वयं प्राप्तस्तपोधन}
{यद्वयं संशितात्मानं प्रलोब्धुं त्वामिहागताः}


\twolineshloka
{अवध्यास्तु स्त्रियः सृष्टा मन्यन्ते धर्मचारिणः}
{तस्माद्धर्मेण वर्ध त्वं नास्मन्हिंसितुमर्हसि}


\twolineshloka
{सर्वभूतेषु धर्मज्ञ मैत्रो ब्राह्मण उच्यते}
{सत्यो भवतु कल्याण एष वादो मनीषिणाम्}


\threelineshloka
{शरणं च प्रपन्नानां शिष्टाः कुर्वन्ति पालनम्}
{शरणं त्वां प्रपन्नाः स्म तस्मात्त्वं क्षन्तुमर्हसि ॥वैशंपायन उवाच}
{}


\threelineshloka
{एवमुक्तः स धर्मात्मा ब्राह्मणः शुभकर्मकृत्}
{प्रसादं कृतवान्वीर रविसोमसमप्रभः ॥ब्राह्मण उवाच}
{}


\twolineshloka
{शतं शतसहस्रं तु सर्वमक्षय्यवाचकम्}
{परिमाणं शतं त्वेतन्नेदमक्षय्यवाचकम्}


\twolineshloka
{यदा च वो ग्राहभूता गृह्णन्तीः पुरुषाञ्जले}
{उत्कर्षति जलात्तस्मात्स्थलं पुरुषसत्तमः}


\twolineshloka
{तदा यूयं पुनः सर्वाः स्वं रूपं प्रतिपत्स्यथ}
{अनृतं नोक्तपूर्वं मे हसतापि कदाचन}


\fourlineindentedshloka
{तानि सर्वाणि तीर्थानि ततः प्रभृति चैव ह}
{नारीतीर्थानि नाम्नेह ख्यातिं यास्यन्ति सर्वशः}
{पुण्यानि च भविष्यन्ति पावनानि मनीषिणां ॥वर्गोवाच}
{}


\twolineshloka
{ततोऽभिवाद्य तं विप्रं कृत्वा चापि प्रदक्षिणम्}
{अचिन्तयामोऽपसृत्य तस्माद्देशात्सुदुःखिताः}


\twolineshloka
{क्व नु नाम वयं सर्वाः कालेनाल्पेन तं नरम्}
{समागच्छेम यो नस्तद्रूपमापादयेत्पुनः}


\twolineshloka
{ता वयं चिन्तयित्वैव मुहूर्तादिव भारत}
{दृष्टवत्यो महाभागं देवर्षिमुत नारदम्}


\twolineshloka
{संप्रहृष्टाः स्म तं दृष्ट्वा देवर्षिममितद्युतिम्}
{अभिवाद्य च तं पार्थ स्थिताः स्म व्रीडिताननाः}


\twolineshloka
{स नोऽपृच्छद्दुःखमूलमुक्तवत्यो वयं च तम्}
{श्रउत्वा तत्र यथावृत्तमिदं वचनमब्रवीत्}


\twolineshloka
{दक्षिणे सागरानूपे पञ्च तीर्थानि सन्ति वै}
{पुण्यानि रमणीयानि तानि गच्छत मा चिरं}


\twolineshloka
{तत्राशु पुरुषव्याघ्रः पाण्डवेयो धनञ्जयः}
{मोक्षयिष्यति शुद्धात्मा दुःखादस्मान्न संशयः}


\threelineshloka
{`इत्युक्त्वा नारदः सर्वास्तत्रैवान्तरधीयत}
{'तस्य सर्वा वयं वीर श्रुत्वा वाक्यमितो गताः}
{तदिदं सत्यमेवाद्य मोक्षिताहं त्वयाऽनघ}


\threelineshloka
{एतास्तु मम ताः सख्यश्चतस्रोऽन्या जले श्रिताः}
{कुरु कर्म शुभं वीर एताः सर्वा विमोक्षय ॥वैशंपायन उवाच}
{}


\twolineshloka
{ततस्ताः पाण्डवश्रेष्ठः सर्वा एव विशांपते}
{`अवगाह्य च तत्तीर्थं गृहीतो ग्राहिभिस्तदा}


\twolineshloka
{ग्राहीभिश्चोत्तताराशु तरयामास तत्क्षणात्}
{सा चाप्सरा बभूवाशु सर्वाभरणभूषिता}


% Check verse!
एवं क्रमेण ताः सर्वा मोक्षयामास वीर्यवान् ॥'
\twolineshloka
{उत्थाय च जलात्तस्मात्प्रतिलभ्य वपुः स्वकम्}
{तास्तदाऽप्सरसो राजन्नदृश्यन्त यथा पुरा}


\twolineshloka
{तीर्थानि शोधयित्वा तु तथानुज्ञाय ताः प्रभुः}
{चित्राङ्गदां पुनर्द्रष्टुं मणलूरं पुनर्ययौ}


\twolineshloka
{तस्यामजनयत्पुत्रं राजानं बभ्रुवाहनम्}
{तं दृष्ट्वा पाण्डवो राजंश्चित्रवाहनमब्रवीत्}


\twolineshloka
{चित्राङ्गदायाः शुल्कं त्वं गृहाण बभ्रुवाहनम्}
{अनेन च भविष्यामि ऋणान्मुक्तो नराधिप}


\twolineshloka
{चित्राङ्गदां पुनर्वाक्यमब्रवीत्पाण्डुनन्दनः}
{इहैव भव भद्रं ते वर्धेथा बभ्रुवाहनम्}


\twolineshloka
{इन्द्रपस्थनिवासं मे त्वं तत्रागत्य रंस्यसि}
{कुन्ती युधिष्ठिरं भीमं भ्रातरौ मे कनीयसौ}


\twolineshloka
{आगत्य तत्र पश्येथा अन्यानपि च बान्धवान्}
{बान्धवैः सहिताः सर्वैर्नन्दसे त्वमनिन्दिते}


\twolineshloka
{धर्मे स्थितः सत्यधृतिः कौन्तेयोऽथ युधिष्ठिरः}
{जित्वा तु पृथिवीं सर्वां राजसूयं करिष्यति}


\twolineshloka
{तत्रागच्छन्ति राजानः पृथिव्यां नृपसंज्ञिताः}
{बहूनि रत्नान्यादाय आगमिष्यति ते पिता}


\twolineshloka
{एकसार्थं प्रयातासि चित्रवाहनसेवया}
{द्रक्ष्यामि राजसूये त्वां पुत्रं पालय मा शुचः}


\twolineshloka
{बभ्रुवाहननाम्ना तु मम प्राणो बहिश्चरः}
{तस्माद्भरस्व पुत्रं वै पुरुषं वंशवर्धनम्}


\twolineshloka
{चित्रावाहनदायादं धर्मात्पौरवनन्दनम्}
{पाण्डवानां प्रियं पुत्रं तस्मात्पालय सर्वदा}


\twolineshloka
{विप्रयोगेण संतापं मा कृथास्त्वमनिन्दिते}
{चित्राङ्गदामेवमुक्त्वा `सागरानूपमाश्रितः}


\twolineshloka
{स्थानं दूरं समाप्लुत्य दत्त्वा बहुधनं तदा}
{केरलान्समतिक्रम्य' गोकर्णमभितोऽगमत्}


\twolineshloka
{आद्यं पशुपतेः स्थानं दर्शनादेव मुक्तिदम्}
{यत्र पापोऽपि मनुजः प्राप्नोत्यभयदं पदम्}


\chapter{अध्यायः २३८}
\twolineshloka
{वैशंपायन उवाच}
{}


\twolineshloka
{सोऽपरान्तेषु तीर्थानि पुण्यान्यायतनानि च}
{सर्वाण्येवानुपूर्व्येण जगामामितविक्रमः}


\twolineshloka
{समुद्रे पश्चिमे यानि तीर्थान्यायतनानि च}
{तानि सर्वाणि गत्वा स प्रभासमुपजग्मिवान्}


\twolineshloka
{`चिन्तयामास रात्रौ तु गदेन कथितं पुरा}
{सुभद्रायाश्च माधुर्यरूपसंपद्गुणानि च}


\twolineshloka
{प्राप्तुमं तां चिन्तयामास कोऽत्रोपायो भवेदिति}
{वेषवैकृत्यमापन्नः परिव्राजकरूपधृत्}


\twolineshloka
{कुकुरान्धकवृष्णीनामज्ञातो वेषधारणात्}
{भ्रममाणश्चरन्भैक्षं परिव्राजकवेषवान्}


\twolineshloka
{येनकेनाप्युपायेन प्रविश्य च गृहं महत्}
{दृष्ट्वा सुभद्रां कृष्णस्य भगिनीमेकसुन्दरीम्}


\twolineshloka
{वासुदेवमतं ज्ञात्वा करिष्यामि हितं शुभम्}
{एवं विनिश्चयं कृत्वा दीक्षितो वै तदाऽभवत्}


\twolineshloka
{त्रिदण्डी मुण्डितः कुण्डी अक्षमालाङ्गुलीयकः}
{योगभारं वहन्पार्थो वटवृक्षस्य कोटरम्}


\twolineshloka
{प्रविशंश्चैव बीभत्सुर्वृष्टिं वर्षति वासवे}
{चिन्तयामास देवेशं केशवं क्लेशनाशनम्}


\twolineshloka
{केशवश्चिन्तितं ज्ञात्वा दिव्यज्ञानेन दृष्टवान्}
{शयानः शयने दिव्ये सत्यभामासहायवान्}


\twolineshloka
{केशवः सहसा राजञ्जहाय च ननन्द च}
{पुनः पुनः सत्यभामा चाब्रवीत्पुरुषोत्तमम्}


\twolineshloka
{भगवंश्चिन्तयाविष्टः शयने शयितः सुखम्}
{भवान्बहुप्रकारेण जहास च पुनः पुनः}


\threelineshloka
{श्रोतव्यं यदि वा कृष्ण प्रसादो यदि ते मयि}
{वक्तुमर्हसि देवेश तच्छ्रोतुं कामयाम्यहम् ॥श्रीभगवानुवाच}
{}


\twolineshloka
{पितृष्वसुर्यः पुत्रो मे भीमसेनानुजोऽर्जुनः}
{तीर्थयात्रां गतः पार्थः कारणात्समयात्तदा}


\twolineshloka
{तीर्थयात्रासमाप्तौ तु निवृत्तो निशि भारतः}
{सुभध्रां चिन्तयामास रूपेणाप्रतिमां भुवि}


\twolineshloka
{चिन्तयेन्नेव तां भद्रां यतिरूपधरोऽर्जुनः}
{यतिरूपप्रतिच्छन्नो द्वारकां प्राप्य माधवीम्}


\twolineshloka
{येनकेनाप्युपायेन दृष्ट्वा तु वरवर्णिनीम्}
{वासुदेवमतं ज्ञात्वा प्रयतिष्ये मनोरथम्}


\twolineshloka
{एवं व्यवसितः पार्थो यतिलिङ्गेन पाण्डवः}
{छायायां वटवृक्षस्य वृष्टिं वर्षति वासवे}


\twolineshloka
{योगभारं वहन्नेव मानसं दुःखमाप्तवान्}
{एतदर्थं विजानीहि हसन्तं मां मुदा प्रिये}


\twolineshloka
{भ्रातरं तव पश्येति सत्यभामां व्यसर्जयत्}
{तत उत्थाय शयनात्प्रस्थितो मधुसूदनः ॥'}


\twolineshloka
{प्रभासदेशं संप्राप्तं बीभत्सुमपराजितम्}
{तीर्थान्यनुचरन्तं तं शुश्राव मधुसूदनः}


\twolineshloka
{चाराणां चैव वचनादेकाकी स जनार्दनः}
{तत्राभ्यगच्छत्कौन्तेयं महात्मातं स माधवः}


% Check verse!
ददृशाते तदान्योन्यं प्रभासे कृष्णपाण्डवौ
\twolineshloka
{तावन्योन्यं समाश्लिष्य पृष्ट्वा च कुशलं वने}
{आस्तां प्रियसखायौ तौ नरनारायणावृषी}


\twolineshloka
{ततोऽर्जुनं वासुदेवस्तां चर्यां पर्यपृच्छत}
{किमर्थं पाण्डवैतानि तीर्थान्यनुचरस्युत}


\twolineshloka
{ततोऽर्जनो यथावृत्तं सर्वमाख्यातवांस्तदा}
{श्रुत्वोवाच च वार्ष्णेय एवमेतदिति प्रभुः}


\twolineshloka
{तौ विहृत्य यथाकामं प्रभासे कृष्णपाण्डवौ}
{महीधरं रैवतकं वासायैवाभिजग्मतुः}


\twolineshloka
{पूर्वमेव तु कृष्णस्य वचनात्तं महीधरम्}
{पुरुषा मण्डयाञ्चक्रुरुपजह्रश्च भोजनम्}


\twolineshloka
{प्रतिगृह्यार्जुनः सर्वमुपभुज्य च पाण्डवः}
{सहैव वासुदेवेन दृष्टवान्नटनर्तकान्}


\twolineshloka
{अभ्यनुज्ञाय तान्सर्वानर्चयित्वा च पाण्डवः}
{सत्कृतं शनं दिव्यमभ्यगच्छन्महामतिः}


\threelineshloka
{ततस्तत्र महाबाहुः शयानः शयने शुभे}
{तीर्थानां पल्वलानां च पर्वतानां च दर्शनम्}
{आपगानां वनानां च कथयामास सात्वते}


\twolineshloka
{एवं स कथयन्नेव निद्रया जनमेजय}
{कौन्तेयोऽपि हृतस्तस्मिञ्शयने स्वर्गसन्निभे}


\twolineshloka
{मधुरेणैव गीतेन वीणाशब्देन चैव ह}
{प्रबोध्यमानो बुबुधे स्तुतिभिर्मङ्गलैस्तता}


\twolineshloka
{स कृत्वाऽवश्यकार्याणि वार्ष्णेयेनाभिनन्दितः}
{`वार्ष्णेयं समनुज्ञाप्य तत्र वासमरोचयत्}


\threelineshloka
{तथेत्युक्त्वा वासुदेवो भोजनं वै शशास ह}
{यतिरूपधरं पार्थं विसृज्य सहसा हरिः}
{'रथेन काञ्चनाङ्गेन द्वारकामभिजग्मिवान्}


% Check verse!
अलङ्कृता द्वारका तु बभूव जनमेजय
\twolineshloka
{दिदृक्षन्तश्च गोविन्दं द्वारकावासिनो जनाः}
{नरेन्द्रमार्गमाजग्मुस्तूर्णं शतसहस्रशः}


\twolineshloka
{`क्षणार्धमपि वार्ष्णेया गोविन्दविरहाक्षमाः}
{कौतूहलसमाविष्टा भृशमुत्प्रेक्ष्य संस्थिताः ॥'}


\twolineshloka
{अवलोकेषु नारीणां सहस्राणि शतानि च}
{भोजवृष्ण्यन्धकानां च समवायो महानभूत्}


\twolineshloka
{स तथा सत्कृतः सर्वैर्भोजवृष्ण्यन्धकात्मजैः}
{अभिवाद्याभिवाद्यांश्च सर्वैश्च प्रतिनन्दितः}


\twolineshloka
{कुमारैः सर्वशो वीरः सत्कारेणाभिचोदितः}
{समानवयसः सर्वानाश्लिष्य स पुनःपुनः}


\twolineshloka
{कृष्णः स्वभवनं रम्यं प्रविवेश महाबलः}
{प्रभासादागतं देव्यः सर्वाः कृष्णमपूजयन्}


\chapter{अध्यायः २३९}
\twolineshloka
{वैशंपायन उवाच}
{}


\twolineshloka
{ततः कतिपयाहस्य तस्मिन्रैवतके गिरौ}
{वृष्ण्यन्धकानामभवदुत्सवो नृपसत्तम}


\twolineshloka
{तत्र दानं ददुर्वीरा ब्राह्मणेभ्यः सहस्रशः}
{भोजवृष्ण्यन्धकाश्चैव महे तस्य गिरेस्तदा}


\twolineshloka
{प्रसादै रत्नचित्रैश्च गिरेस्तस्य समन्ततः}
{स देशः शोभितो राजन्कल्पवृक्षैश्च सर्वशः}


\twolineshloka
{वादित्राणि च तत्रान्ये वादकाः समवादयन्}
{ननृतुर्नर्तकाश्चैव जगुर्गेयानि गायनाः}


\twolineshloka
{अलङ्कृताः कुमाराश्च वृष्णीनां सुमहौजसाम्}
{यानैर्हाटकचित्रैश्च चञ्चूर्यन्ते स्म सर्वशः}


\twolineshloka
{पौराश्च पादचारेण यानैरुच्चावचैस्तथा}
{सदाराः सानुयात्राश्च शतशोऽथ सहस्रशः}


\twolineshloka
{ततो हलधरः क्षीबो रेवतीसहितः प्रभुः}
{अनुगम्यमानो गन्धर्वैरचरत्रत्र भारत}


\twolineshloka
{तथैव राजा वृष्णीनामुग्रसेनः प्रतापवान्}
{अनुगीयमानो गन्धर्वैः स्त्रीसहस्रसहायवान्}


\twolineshloka
{रौक्मिणेयश्च साम्बश्च क्षीबौ समरदुर्मदौ}
{दिव्यमाल्याम्बरधरौ विजह्वातेऽमराविव}


\twolineshloka
{अक्रूरः सारणश्चैव गदो बभ्रुर्विदूरथः}
{निशठश्चारुदेष्णश्च पृथुर्विपृथुरेव च}


\twolineshloka
{सत्यकः सात्यकिश्चैव भङ्गकारमहारवौ}
{हार्दिक्य उद्धवश्चैव ये चान्ये नानुकीर्तिताः}


\twolineshloka
{एते परिवृताः स्त्रीभिर्गन्धर्वैश्च पृथक्पृथक्}
{तमुत्सवं रैवतके शोभयाञ्चक्रिरे तदा}


\twolineshloka
{`वासदेवो ययौ तत्र सह स्त्रीभिर्मुदान्वितः}
{दत्त्वा दानं द्विजातिभ्यः परिव्राजमपश्यत ॥'}


\twolineshloka
{चित्रकौतूहले तस्मिन्वर्तमाने महाद्भुते}
{वासुदेवश्च पार्थश्च सहितौ परिजग्मतुः}


\twolineshloka
{तत्र चङ्क्रममाणौ तौ वसुदेवसुतां शुभाम्}
{अलङ्कृतां सखीमध्ये भद्रां ददृशतुस्तदा}


\twolineshloka
{दृष्ट्वैव तामर्जुनस्य कन्दर्पः समजायत}
{तं तदैकाग्रमनसं कृष्णः पार्थमलक्षयत्}


\twolineshloka
{अब्रवीत्पुरुषव्याघ्रः प्रसहन्निव भारत}
{वनेचरस्य किमिदं कामेनालोड्यते मनः}


\fourlineindentedshloka
{ममैषा भगिनी पार्थ सारणस्य सहोदरी}
{सुभद्रा नाम भद्रं ते पितुर्मे दयिता सुता}
{यदि ते वर्तते बुद्धिर्वक्ष्यामि पितरं स्वयम् ॥अर्जुन उवाच}
{}


\twolineshloka
{दुहिता वसुदेवस्य वासुदेवस्य च स्वसा}
{रूपेण चैषा संपन्ना कमिवैषा न मोहयेत्}


\twolineshloka
{कृतमेव तु कल्याणं सर्वं मम भवेद्ध्रुवम्}
{यदि स्यान्मम वार्ष्णेयी महिषीयं स्वसा तव}


\threelineshloka
{प्राप्तौ तु क उपायः स्यात्तं व्रवीहि जनार्दन}
{आस्थास्यामि तदा सर्वं यदि शक्यं नरेण तत् ॥वासुदेव उवाच}
{}


\twolineshloka
{स्वयं वरः क्षत्रियाणां विवाहः पुरुषर्षभ}
{स च संशयितः पार्थ स्वभावस्यानिमित्ततः}


\twolineshloka
{प्रसह्य हरणं चापि क्षत्रियाणां प्रशस्यते}
{विवाहहेतुः शूराणामिति धर्मविदो विदुः}


\fourlineindentedshloka
{स त्वमर्जुन कल्याणीं प्रसह्य भगिनीं मम}
{`यतिरूपधरस्तं तु यथा कालविपाकता}
{'हर स्वयंवरे ह्यस्याः को वै वेद चिकीर्षितम् ॥वैशंपायन उवाच}
{}


\twolineshloka
{ततोऽर्जुनश्च कृष्णश्च विनिश्चित्येतिकृत्यताम्}
{शीघ्रगान्पुरुषानन्प्रेषयामासतुस्तदा}


\twolineshloka
{धर्मराजाय तत्सर्वमिन्द्रप्रस्थगताय वै}
{श्रुत्वैव च महाबाहुरनुजज्ञे समातृकः}


\twolineshloka
{`भीमसेनस्तु तच्छ्रुत्वा कृतकृत्यं स्म मन्यते}
{इत्येवं मनुजैरुक्तं कृष्णः श्रुत्वा महामतिः}


\twolineshloka
{अनुज्ञाप्य तदा पार्थं हृदि स्थाप्य चिकीर्षितम्}
{इत्येवं मनुजैः सार्धं द्वारकां समुपेयिवान्}


\chapter{अध्यायः २४०}
\twolineshloka
{वैशंपायन उवाच}
{}


\twolineshloka
{चैराः संचारिते तस्मिन्ननुज्ञाते युधिष्ठिरे}
{वासुदेवाभ्यनुज्ञातः कथयित्वेतिकृत्यताम्}


\twolineshloka
{कृष्णस्य मतमास्थाय प्रययौ भरतर्षभः}
{द्वारकाया उपवने तस्थौ वै कार्यसाधनः}


\twolineshloka
{निवृत्ते ह्युत्सवे तस्मिन्गिरौ रैवतके तदा}
{वृष्णयोऽप्यगमन्सर्वे पुरीं द्वारवतीमनु}


\twolineshloka
{चिन्तयानस्ततो भद्रामुपविष्टः शिलातले}
{रमणीये वनोद्देशे बहुपादपसंवृते}


\twolineshloka
{सालतालाश्वकर्णैश्च बकुलैरर्जुनैस्तथा}
{चम्पकाशोकपुन्नागैः केतकैः पाटलैस्तथा}


\twolineshloka
{कर्णिकारैरशोकैश्च अङ्कोलैरतिमुक्तकैः}
{एवमादिभिरन्यैश्च संवृते स शिलातले}


\twolineshloka
{पुनःपुनश्चिन्तयानः सुभद्रां भद्रभाषिणीम्}
{यदृच्छया चोपपन्नान्वृष्णिवीरान्ददर्श सः}


\twolineshloka
{बलदेवं च हार्दिक्यं साम्बं सारणमेव च}
{प्रद्युम्नं च गदं चैव चारुदेष्णं विदूरथम्}


\twolineshloka
{भानुं च हरितं चैव विपृथुं पृथुमेव च}
{तथान्यांश्च बहून्पश्यन्हृदि शोकमधारयत्}


\twolineshloka
{ततस्ते सहिताः सर्वे यतिं दृष्ट्वा समुत्सुकाः}
{वृष्णयो विनयोपेताः परिवार्योपतस्थिरे}


\twolineshloka
{ततोऽर्जुनः प्रीतमनाः स्वागतं व्याजहार सः}
{आस्यतामास्यतां सर्वै रमणीये शिलातले}


\twolineshloka
{इत्येवमुक्ता यतिना प्रीतास्ते यादवर्षभाः}
{उपोपविविशुः सर्वे ते स्वागतमिति ब्रुवन्}


\twolineshloka
{परितः सन्निविष्टेषु वृष्णिवीरेषु पाण्डवः}
{आकारं गूहमानस्तु कुशलप्रश्नमब्रवीत्}


\twolineshloka
{सर्वत्र कुशलं चोक्त्वा बलदेवोऽब्रवीदिदम्}
{प्रसादं कुरु मे विप्र कुतस्त्वं चागतो ह्यसि}


\threelineshloka
{त्वया दृष्टानि पुण्यानि वद त्वं वदतांवर}
{पर्वतांश्चैव तीर्थानि वनान्यायतनानि च ॥वैशंपायन उवाच}
{}


\twolineshloka
{तीर्थानां दर्शनं चैव पर्वतानां च भारत}
{आपगानां वनानां च कथयामास ताः कथाः}


\twolineshloka
{श्रुत्वा धर्मकथाः पुण्या वृष्णिवीरा मुदान्विताः}
{अपूजयंस्तदा भिक्षुं कथान्ते जनमेजय}


\twolineshloka
{ततस्तु यादवाः सर्वे मन्त्रयन्ति स्म भारत}
{अयं देशातिथिः श्रीमान्यतिलिङ्गधरो द्विजः}


\twolineshloka
{आवासं कमुपाश्रित्य वसेत निरुपद्रवः}
{इत्येवं प्रब्रुवन्तस्तु रौहिणेयं च यादवाः}


\twolineshloka
{ददृशुः कृष्णमायान्तं सर्वे यादवनन्दनम्}
{एहि केशव तातेति रौहिणेयो वचोऽब्रवीत्}


\fourlineindentedshloka
{यतिलिङ्गधरो विद्वान्देशातिथिरयं द्विजः}
{वर्षमासनिवासार्थमागतो नः पुरं प्रति}
{स्थाने यस्मिन्निवसतु तन्मे ब्रूहि जनार्दन ॥श्रीकृष्ण उवाच}
{}


\twolineshloka
{त्वयि स्थिते महाभाग परवानस्मि धर्मतः}
{स्वयं तु रुचिरे स्थाने वासयेर्यदुनन्दन}


\twolineshloka
{प्रीतः स तेन वाक्येन परिष्वज्य जनार्दनम्}
{बलदेवोऽब्रवीद्वाक्यं चिन्तयित्वा महाबलः}


\twolineshloka
{आरामे तु वसेद्धीमांश्चतुरो वर्षमासकान्}
{कन्यागृहे सुभद्राया भुक्त्वा भोजनमिच्छया}


\threelineshloka
{लतागृहेषु वसतामिति मे धीयते मतिः}
{लब्धानुज्ञास्त्वया तात मन्यन्ते सर्वयादवाः ॥श्रीकृष्ण उवाच}
{}


\twolineshloka
{बलवान्दर्शनीयश्च वाग्मी श्रीमान्बहुश्रुतः}
{कन्यापुरसमीपे तु न युक्तमिति मे मतिः}


\twolineshloka
{गुरुः शास्ता च नेता च शास्त्रज्ञो धर्मवित्तमः}
{त्वयोक्तं न विरुध्येहं करिष्यामि वचस्तव}


\twolineshloka
{शुभाशुभस्य विज्ञाता नान्योऽस्मि भुवि कश्चन ॥बलदेव उवाच}
{}


\twolineshloka
{अयं देशातिथिः श्रीमान्सर्वधर्मभृतां वरः}
{धृतिमान्विनयोपेतः सत्यबादी जितेन्द्रियः}


\twolineshloka
{यतिलिङ्गधरो ह्येष को विजानाति मानसम्}
{त्वमिमं पुण्डरीकाक्ष नीत्वा कन्यापुरं शुभम्}


\twolineshloka
{निवेदय सुभद्रायै मद्वाक्यपरिचोदितः}
{भक्ष्यैर्भोज्यैश्च पानैश्च अन्नैरिष्टैश्च पूजय}


\chapter{अध्यायः २४१}
\twolineshloka
{वैशंपायन उवाच}
{}


\twolineshloka
{स तथेति प्रतिज्ञाय सहितो यतिना हरिः}
{कृत्वा तु संविदं तेन प्रहृष्टः केशवोऽभवत्}


\threelineshloka
{पर्वते तौ विहृत्यैव यथेष्टं कृष्णपाण्डवौ}
{तां पुरीं प्रविवेशाथ गृह्य हस्तेन पाण्डवम्}
{प्रविश्य च गृहं रम्यं सर्वभोगसमन्वितम्}


\twolineshloka
{पार्थमावेदयामास रुक्मिणीसत्यभामयोः}
{हृषीकेशवचः श्रुत्वा ते उभे चोचतुर्भृशम्}


\twolineshloka
{मनोरथो महानेष हृदि नौ परिवर्तते}
{कदा द्रक्षाव बीभत्सुं पाण्डवं पुरमागतम्}


\twolineshloka
{इत्येवं हर्षमाणे ते वदन्त्यौ सुभृशं प्रियम्}
{रुग्मिणीसत्यभामे वै दृष्ट्वा प्रीतोऽभवद्यतिः}


\twolineshloka
{सर्वेषां हर्षमाणानां पार्थो हर्षमुपागमत्}
{प्राप्तमज्ञातरूपेण चागतं चार्जुनं हरिः}


\twolineshloka
{सत्कृत्य पूज्यमानं तु प्रीत्या चैव ह्यपूजयत्}
{स तं प्रियातिथिं श्रेष्ठं समीक्ष्य यतिमागतम्}


\twolineshloka
{सोदर्यां भगिनीं कृष्णः सुभद्रामिदमब्रवीत्}
{अयं देशातिथिर्भद्रे संशितव्रतवानृषिः}


\twolineshloka
{प्राप्नोतु सततं पूजां तव कन्यापुरे वसन्}
{आर्येण च परिज्ञातः पूजनीयो यतिस्त्वया}


\twolineshloka
{रागाद्भरस्व वार्ष्णेयि भक्ष्यैर्भोज्यैर्यतिं सदा}
{एष यद्यदृषिर्ब्रूयात्कार्यमेव न संशयः}


\twolineshloka
{सखीभिः सहिता भद्रे भवास्य वशवर्तिनी}
{पुरा हि यतयो भद्रे ये भैक्षार्थमनुव्रताः}


\fourlineindentedshloka
{ते बभूवुर्दशार्हाणां कन्यापुरनिवासिनः}
{तेभ्यो भोज्यानि भक्ष्याणि यथाकालमतन्द्रिताः}
{कन्यापुरगताः कन्याः प्रयच्छन्ति यशस्विनि ॥वैशंपायन उवाच}
{}


\twolineshloka
{सा तथेत्यब्रवीत्कृष्णं करिष्यामि यथाऽऽथ माम्}
{तोषयिष्यामि वृत्तेन कर्मणा च द्विजर्षभम्}


\twolineshloka
{एवमेतेन रूपेण कंचित्कालं धनञ्जयः}
{उवास भक्ष्यैर्भोज्यैश्च भद्रया परमार्चितः}


\twolineshloka
{तस्य सर्वगुणोपेतां वासुदेवसहोदरीम्}
{पश्यतः सततं भद्रां प्रादुरासीन्मनोभवः}


\twolineshloka
{गूहयन्निव चाकारमालोक्य वरवर्णिनीम्}
{दीर्घमुष्णं विनिश्वस्य पार्थः कामवशं गतः}


\twolineshloka
{स कृष्णां द्रौपदीं मेने न रूपे भद्रया समाम्}
{प्राप्तां भूमान्विन्द्रसेनां साक्षाद्वा वरुणात्मजाम्}


\twolineshloka
{अतीतकाले संप्राप्ते सर्वास्तापि सुरस्त्रियः}
{न समा भद्रया लोके इत्येवं मन्यतेऽर्जुनः}


\twolineshloka
{अतीतसमये काले सोदर्याणां धनञ्जयः}
{न सस्मार सुभद्रायां कामाङ्कुशनिवारितः}


\twolineshloka
{क्रीडारतिपरां भद्रां सखीगणसमावृताम्}
{प्रीयते स्मार्जुनः पश्यन्स्वाहामिव विभावसुः}


\twolineshloka
{पाण्डवस्य सुभद्रायाः सकाशे तु यशस्विनः}
{समुत्पत्तिः प्रभावश्च गदेन कथितः पुरा}


\twolineshloka
{श्रुत्वा चाशनिनिर्घोषं केशवेनापि धीमता}
{उपमामर्जुनं कृत्वा विस्तरः कथितः पुरा}


\twolineshloka
{क्रुद्धमानप्रलापश्च वृष्णीनामर्जुनं प्रति}
{पौरुषं चोपमां कृत्वा प्रावर्तत धनुष्मताम्}


\twolineshloka
{अन्योन्यकलहे चापि विवादे चापि वृष्णयः}
{अर्जुनोपि न मे तुल्यः कुतस्त्वमिति चाब्रुवन्}


\twolineshloka
{जातांश्च पुत्रान्गृह्णन्त आशिषो वृष्णयोऽब्रवन्}
{अर्जुनस्य समो वीर्ये भव तात धनुर्धरः}


\twolineshloka
{तस्मात्सुभद्रा चकमे पौरुषाद्भरतर्षभम्}
{सत्यसन्धस्य रूपेण चातुर्येण च मोहिता}


\twolineshloka
{चारणातिथिसंघानां गदस्य च निशम्य सा}
{अदृष्टे कृतभावाभूत्सुभद्रा भरतर्षभे}


\twolineshloka
{कीर्तयन्ददृशे यो यः कथंचित्कुरुजाङ्गलम्}
{तं तमेव तदा भद्रा बीभत्सुं स्म हि पृच्छति}


\twolineshloka
{अभीक्ष्णश्रवणादेवमभीक्ष्णपरिपृच्छनात्}
{प्रत्यक्ष इव भद्रायाः पाण्डवः प्रत्यपद्यत}


\twolineshloka
{भुजौ भुजगसङ्काशौ ज्याघातेन किणीकृतौ}
{पार्थोऽयमिति पश्यन्त्या निःशंसयमजायत}


\twolineshloka
{यथारूपं हि शुश्राव सुभद्रा भरतर्षभम्}
{तथारूपमवेक्ष्यैनं परां प्रीतिमवाप सा}


\twolineshloka
{सा कदाचिदुपासीनं पप्रच्छ कुरुनन्दनम्}
{कथं देशाश्च चरिता नानाजनपदाः कथम्}


\twolineshloka
{सरांसि सरितश्चैव वनानि च कथं यते}
{दिशः काश्च कथं प्राप्ताश्चरता भवता सदा}


\twolineshloka
{स तथोक्तस्तदा भद्रां बहुनर्मामृतं ब्रुवन्}
{उवाच परमप्रीतस्तथा बहुविधाः कथाः}


\twolineshloka
{निशण्य विविधं तस्य लोके चरितमात्मनः}
{तथा परिगतो भावः कन्यायाः समपद्यत}


\twolineshloka
{पर्वसन्धौ तु कस्मिंश्चित्सुभद्रा भरतर्षभम्}
{रहस्येकान्तमासाद्य हर्षमाणाऽभ्यभाषत}


\twolineshloka
{यतिना रचता देशान्खाण्डवप्रस्थवासिनी}
{कश्चिद्भगवता दृष्टा पृथाऽस्माकं पितृष्वसा}


\twolineshloka
{भ्रातृभिः प्रीयते सर्वैर्दृष्टः कच्चिद्युधिष्ठिरः}
{कच्चिद्धर्मपरो भीमो धर्मराजस्य धीमतः}


\twolineshloka
{निवृत्तसमयः कच्चिदपराधाद्धनञ्जयः}
{नियमे कामभोगानां वर्तमानः प्रिये रतः}


\twolineshloka
{क्व नु पार्थश्चरत्यद्य बहिः स वसतीर्वसन्}
{सुखोचितो ह्यदुःखार्हो दीर्घबाहुररिन्दमः}


\twolineshloka
{कच्चिच्छ्रुतो वा दृष्टो वा पार्थो भगवताऽर्जुनः}
{निशम्य वचनं तस्यास्तामुवाच हसन्निव}


\twolineshloka
{आर्या कुशलिनी कुन्ती सहपुत्रा सहस्नुषा}
{प्रीयते पश्यती पुत्रान्खाण्डवप्रस्थ आसते}


\twolineshloka
{अनुज्ञातश्च मात्रा च सोदरैश्च धनञ्जयः}
{द्वारकामावसत्येको यतिलिङ्गेन पाण्डवः}


\twolineshloka
{पश्यन्ती सततं कस्मान्नाभिजानासि माधवि}
{निशण्य वचनं तस्य वासुदेवसहोदरी}


\twolineshloka
{निश्वासबहुला तस्थौ क्षितिं विलिखती तदा}
{ततः परमसंहृष्टः सर्वशस्त्रभृतां वरः}


\twolineshloka
{अर्जुनोऽहमिति प्रीतस्तामुवाच धनञ्जयः}
{यथा तव गतो भावः श्रवणान्मयि भामिनि}


\twolineshloka
{त्वद्गतः सततं भावस्तथा तव गुणैर्मम}
{प्रशस्तेऽहनि धर्मेण भद्रे स्वयमहं वृतः}


\twolineshloka
{सत्यवानिव सावित्र्या भविष्यामि पतिस्तव ॥वैशंपायन उवाच}
{}


\twolineshloka
{एवमुक्त्वा ततः पार्थः प्रविवेश लतागृहम्}
{ततः सुभद्रा ललिता लज्जाभावसमन्विता}


\twolineshloka
{मुमोह शयने दिव्ये शयाना न तथोचिता}
{नाकरोद्यतिपूजां सा लज्जाभावमुपेयुषी}


\twolineshloka
{कन्यापुरे तु यद्वृत्तं ज्ञात्वा दिव्येन चक्षुषा}
{शशास रुक्मिणीं कृष्णो भोजनादि तदार्जुने}


\twolineshloka
{तदाप्रभृति तां भद्रां चिन्तयन्वै धनञ्जयः}
{आस्ते स्म स तदाऽऽरामे कामेन भृशपीडितः}


\twolineshloka
{सुभद्रा चापि न स्वस्था पार्थं प्रति बभूव सा}
{कृशा विवर्णवदना चिन्ताशोकपरायणा}


\twolineshloka
{निश्वासपरमा भद्रा मानसेन मनस्विनी}
{न शय्यासनभोगेषु रतिं विन्दति केनचित्}


\threelineshloka
{न नक्तं न दिवा शेते बभूवोन्मत्तदर्शना}
{एवं शोकपरां भद्रां देवी वाक्यमथाब्रवीत्}
{मा शोकं कुरु वार्ष्णेयि धृतिमालम्ब्य शोभने}


\twolineshloka
{रुक्मिण्येवं सुभद्रां तां कृष्णस्यानुमते तदा}
{रहोगत्य तदा श्वश्रूं देवकीं वाक्यमब्रवीत्}


\twolineshloka
{अर्जुनो यतिरूपेण ह्यागतः सुसमाहितः}
{कन्यापुरमथाविश्य पूजितो भद्रया मुदा}


\twolineshloka
{तं विदित्वा सुभद्रापि लज्जया परिमोहिता}
{दिवानिशं शयाना सा नाकरोद्भोजनादिकम्}


\twolineshloka
{एवमुक्ता तया देवी भद्रां शोकपरायणाम्}
{तत्समीपं समागत्य श्लक्ष्णं वाक्यमथाब्रवीत्}


\twolineshloka
{मा शोकं कुरु वार्ष्णेयि धृतिमालम्ब्य शोभने}
{राज्ञे निवेदयित्वापि वसुदेवाय धीमते}


\twolineshloka
{कृष्णायापि तथा भद्रे प्रहर्षं कारयामि ते}
{पश्चाज्जानामि ते वार्तां मा शोकं कुरु भामिनि}


\twolineshloka
{एवमुक्त्वा तु सा माता भद्रायाः प्रियकारिणी}
{निवेदयामास तदा भद्रामानकदुन्दुभेः}


\twolineshloka
{रहस्येकासना तत्र भद्राऽस्वस्थेति चाब्रवीत्}
{आरामे तु यतिः श्रीमानर्जुनश्चेति नः श्रुतम्}


\threelineshloka
{अक्रूराय च कृष्माय आहुकाय च सात्येकः}
{निवेद्यतां महाप्राज्ञ श्रोतव्यं यदि बान्धवैः ॥वैशंपायन उवाच}
{}


\twolineshloka
{वसुदेवस्तु तच्छ्रुत्वा अक्रूराहुकयोस्तथा}
{निवेदयित्वा कृष्णेन मन्त्रयामास तैस्तदा}


\twolineshloka
{इदं कार्यमिदं कृत्यमिदमेवेति निश्चितः}
{अक्रूरश्चोग्रसेनश्च सात्यकिश्च गदस्तथा}


\twolineshloka
{पृथुश्रवाश्च कृष्णश्च सहिताः शिनिना मुहुः}
{रुक्मिणी सत्यभामा च देवकी रोहिणी तथा}


\twolineshloka
{वसुदेवेन सहिताः पुरोहितमते स्थिताः}
{विवाहं मन्त्रयामासुर्द्वादशेऽहनि भारत}


\twolineshloka
{अज्ञातं रौहिणेयस्य उद्धवस्य च भारत}
{विवाहं तु सुभद्रायाः कर्तुकामो गदाग्रजः}


\twolineshloka
{महादेवस्य पूजार्थं महोत्सव इति ब्रुवन्}
{चतुस्त्रिंशदहोरात्रं सुभद्रार्तिप्रशान्तये}


\twolineshloka
{नगरे घोषयास हितार्थं सव्यसाचिनः}
{इतश्चतुर्थे त्वहनि अन्तर्द्वीपं तु गम्यताम्}


\twolineshloka
{सदारैः सानुयात्रैश्च सपुत्रैः सहबाधवैः}
{गन्तव्यं सर्ववर्मैश्च गन्तव्यं सर्वयादवैः}


\twolineshloka
{एवमुक्तास्तु ते सर्वे तथा चक्रुश्च सर्वशः}
{ततः सर्वदशार्हाणामन्तर्द्वीपे च भारत}


\twolineshloka
{चतुस्त्रिंशदहोरात्रं बभूव परमोत्सवः}
{कृष्णरामाहुकाक्रूरप्रद्युम्नशिनिसत्यकाः}


\twolineshloka
{समुद्रं प्रययुर्हृष्टाः कुकुरान्धकवृष्णयः}
{युक्तयन्त्रपताकाभिर्वृष्णयो ब्राह्मणैः सह}


\twolineshloka
{समुद्रं प्रययुर्नौभिः सर्वे पुरनिवासिनः}
{ततस्त्वरितमागत्य दाशार्हगणपूजितम्}


\twolineshloka
{सुभद्रा पुण्डरीकाक्षमब्रवीद्यतिशासनात्}
{कृत्यवान्द्वादशाहानि स्थाता स भगवानिह}


\twolineshloka
{तिष्ठतस्तस्य कः कुर्यादुपस्थानविधिं सदा}
{तमुवाच हृषीकेशः कस्त्वदन्यो विशेषतः}


\twolineshloka
{तमृषिं प्रत्युपस्थातुमितो नार्हति माधवि}
{त्वमेवास्मन्मतेनाद्य महर्षेर्वशवर्तिनी}


\twolineshloka
{कुरु सर्वाणि कार्याणि कीर्तिं धर्ममवेक्ष्य च}
{तस्य चातिथिमुख्यस्य सर्वेषां च तपस्विनाम्}


\twolineshloka
{संविधानपरा भद्रे भव त्वं वशवर्तिनी ॥वैशंपायन उवाच}
{}


\twolineshloka
{एवमादिश्य भिक्षां च भद्रां च मधुसूदनः}
{ययौ शङ्खप्रणादेन भेरीणां निस्वनेन च}


\twolineshloka
{ततस्तु द्वीपमासाद्य दानधर्मपरायणाः}
{उग्रसेनमुखाश्चान्ये विजहुः कुकुरान्धकाः}


\twolineshloka
{पटहानां प्रणादैश्च भेरीणां निस्वनेन च}
{सप्तयोजनविस्तार आयतो दशयोजनम्}


\twolineshloka
{बभूव स महाद्वीपः सपर्वतमहावनः}
{सेतुपुष्करिणीजालैराक्रीडः सर्वसात्वताम्}


\twolineshloka
{वापीपल्वलसङ्घैश्च काननैश्च मनोरमैः}
{वासुदेवस्य क्रीडार्थं योग्यः सर्वप्रहर्षतः}


\twolineshloka
{कुकुरान्धकवृष्णीनां तथा प्रियकरस्तदा}
{बभूव परमोपेतस्त्रिविष्टप इवापरः}


\twolineshloka
{चतुस्त्रिंशदहोरात्रं दानधर्मपरायणाः}
{उग्रसेनमुखाः सर्वे विजहुः कुकुरान्धकाः}


\twolineshloka
{विचित्रमाल्याभरणाश्चित्रगन्धानुलेपनाः}
{विहाराभिगताः सर्वे यादवा हर्षसंयुताः}


\threelineshloka
{सुनृत्तगीतवादित्रै रममाणास्तदाऽभवन्}
{प्रतियाते दशार्हाणामृषभे शार्ङ्गधन्वनि}
{सुभध्रोद्वाहनं पार्थः प्राप्तकालममन्यत}


\chapter{अध्यायः २४२}
\twolineshloka
{वैशंपायन उवाच}
{}


\twolineshloka
{कुकुरान्धकवृष्णीनामपयानं च पाण्डवः}
{विनिश्चित्य ततः पार्थः सुभद्रामिदमब्रवीत्}


\twolineshloka
{शृणु भद्रे यथाशास्त्रं हितार्थं मुनिभिः कृतम्}
{विवाहं बहुधा सत्सु वर्णानां धर्मसंयुतम्}


\twolineshloka
{कन्यायास्तु पिता भ्राता माता मातुल एव वा}
{पितुः पिता पितुर्भ्राता दाने तु प्रभुतां गतः}


\twolineshloka
{महोत्सवं पशुपतेर्द्रष्टुकामः पिता तव}
{अन्तर्द्वीपं गतो भद्रे पुत्रैः पौत्रैः सबान्धवैः}


\twolineshloka
{मम चैव विशालाक्षि विदेशस्था हि बान्धवाः}
{तस्मात्सुभद्रे गान्धर्वो विवाहः पञ्चमः स्मृतः}


\twolineshloka
{समागमे तु कन्यायाः क्रियाः प्रोक्ताश्चतुर्विधाः}
{तेषां प्रवृत्तिं साधूनां शृणु माधवि तद्यथा}


\twolineshloka
{वरमाहूय विधिना पित्रा दत्ता तथार्थिने}
{सा पत्नी तु परैरुक्ता सा वश्या तु पतिव्रता}


\twolineshloka
{भृत्यानां भरणार्थाय आत्मनः पोषणाय च}
{दाने गृहीता या नारी सा भार्येति स्मृता बुधैः}


\twolineshloka
{धर्मतो वरयित्वा तु आनीय स्वं निवेशनम्}
{न्यायेन दत्तातारुण्ये दाराः पितृकृताः स्मृताः}


\twolineshloka
{गान्धर्वेण विवाहेन रागात्पुत्रार्थकारणात्}
{आत्मनाऽनुगृहीता या वश्या सा तु प्रजावती}


\twolineshloka
{जनयेद्या तु भर्तारं जाया इत्येव नामतः}
{पत्नी भार्या च दाराश्च जाया चेति चतुर्विधाः}


\twolineshloka
{चतस्र एवाग्निसाक्ष्याः क्रियायुक्ताश्च धर्मतः}
{गान्धर्वस्तु क्रियाहीनो रागादेव प्रवर्तते}


\twolineshloka
{सकामायाः सकामेन निर्मन्त्रो रहसि स्मृतः}
{मयोक्तमक्रियं चापि कर्तव्यं माधवि त्वया}


\twolineshloka
{अयनं चैव मासश्च ऋतुः पक्षस्तथा तिथिः}
{करणं च मुहूर्तं च लग्नसंपत्तथैव च}


\twolineshloka
{विवाहस्य विशालाक्षि प्रशस्तं चोत्तरायणम्}
{वैशाखश्चैव मासानां पक्षाणां शुक्ल एव च}


\twolineshloka
{नक्षत्राणां तथा हस्तस्तृतीया च तिथिष्वपि}
{लग्नो हि मकरः श्रेष्ठः करणानां बवस्तथा}


\twolineshloka
{मैत्रो मुहूर्तो वैवाह्य आवयोः शुभकर्मणि}
{सर्वसंपदियं भद्रे अद्य रात्रौ भविष्यति}


\twolineshloka
{भगवानस्तमभ्येति आदित्यस्तपतां वरः}
{रात्रौ विवाहकालोऽयं भविष्यति न संशयः}


\twolineshloka
{नारायणोऽपि सर्वज्ञो नावबुध्येत विश्वकृत्}
{धर्मसङ्कटमापन्ने किं नु कृत्वा सुखं भवेत्}


\threelineshloka
{मनोभवेन कामेन मोहितं मां प्रलापिनम्}
{प्रतिवाक्यं च मे देवि किं न वक्ष्यसि माधवि ॥वैशंपायन उवाच}
{}


\twolineshloka
{अर्जुनस्य वचः श्रुत्वा चिन्तयन्ती जनार्दनम्}
{नोवाच किंचिद्वचनं बाष्पदूषितलोचना}


\twolineshloka
{रागोन्मादप्रलापी सन्नर्जुनो जयतां वरः}
{चिन्तयामास पितरं प्रविश्य च लतागृहम्}


\twolineshloka
{चिन्तयानं तु कौन्तेयं मत्वा शच्या शचीपतिः}
{सहितो नारदाद्यैश्च मुनिभिश्च महामनाः}


\twolineshloka
{गन्धर्वैरप्सरोभिश्च चारणैश्चापि गुह्यकैः}
{अरुन्धत्या वसिष्ठेन ह्याजगाम कुशस्थलीम्}


\twolineshloka
{चिन्तितं च सुभद्रायाश्चिन्तयित्वा जनार्दनः}
{निद्रयापहृतज्ञानं रौहिणेयं विना तदा}


\twolineshloka
{सहाक्रूरेण शिनिना सत्यकेन गदेन च}
{वसुदेवेन देवक्या आहूकेन च धीमता}


\twolineshloka
{आजगाम पुरीं रात्रौ द्वारकां स्वजनैर्वृतः}
{पूजयित्वा तु देवेशो नारदादीन्महायशाः}


\twolineshloka
{कुशलप्रश्नमुक्त्वा तु देवेन्द्रेणाभियाचितः}
{वैवाहिकीं क्रियां कृष्णः स तथेत्येवमुक्तवान्}


\threelineshloka
{आहुको वसुदेवश्च सहाक्रूरः ससात्यकिः}
{अभिप्रणम्य शिरसा पाकशासनमब्रुवन्}
{देवदेव नमस्तेस्तु लोकनाथ जगत्पते}


\threelineshloka
{वयं धन्याः स्म सहितैर्बान्धवैः सहिताः प्रभो}
{कृतप्रसादास्तु वयं तव वाक्येन विश्वजित् ॥वैशंपायन उवाच}
{}


\twolineshloka
{एवमुक्त्वा प्रसाद्यैनं पूजयित्वा प्रयत्नतः}
{महेन्द्रशासनात्सर्वे सहिता ऋषिभिस्तदा}


\twolineshloka
{विवाहं कारयामासुः शक्रपुत्रस्य शास्त्रतः}
{अरुन्धती शची देवी रुग्मिणी देवकी तथा}


\twolineshloka
{दिव्यस्त्रीभिश्च सहिताः सुभद्रायाः शुभाः क्रियाः}
{अर्जुनेऽपि तथा सर्वाः क्रिया भद्राः प्रयोजयन्}


\twolineshloka
{महर्षिः काश्यपो होता सदस्या नारदादयः}
{पुण्याशिषः प्रयोक्तारः सर्वे ते हि तदार्जुने}


\twolineshloka
{अभिषेकं तदा कृत्वा महेन्द्रः पाकशासनिम्}
{लोकपालैस्तु सहितः सर्वदेवैरभिष्टुतः}


\twolineshloka
{किरीटाङ्गदहाराद्यैर्हस्ताभरणकुण्डलैः}
{भूषयित्वा तदा पार्थं द्वितीयमिव वासवम्}


\twolineshloka
{पुत्रं परिष्वज्य तदा प्रीतिमाप पुरन्दरः}
{शछी देवी तदा भद्रामरुन्धत्यादिभिस्तथा}


\twolineshloka
{कारयामास वैवाह्यमङ्गलान्यादवस्त्रियः}
{सहाप्सरोभिर्मुदिता भूषयित्वा स्वभूषणैः}


\twolineshloka
{पौलोमीमिव मन्यन्ते सुभद्रां तत्र योषितः}
{ततो विवाहो ववृधे कृतः सर्वगुणान्वितः}


\twolineshloka
{तस्याः पाणिं गृहीत्वा तु मन्त्रैर्होमपुरस्कृतम्}
{सुभद्रया बभौ जिष्णुः शच्या इव शचीपतिः}


\twolineshloka
{सा जिष्णुमधिकं भेजे सुभद्रा चारुदर्शना}
{पार्थस्य सदृशी भद्रा रूपेण वयसा तथा}


\twolineshloka
{सुभद्रायाश्च पार्थोऽपि सदृशो रूपलक्षणैः}
{इत्यूचुश्च तदा देवाः प्रीताः सेन्द्रपुरोगमाः}


\twolineshloka
{एवं निवेश्य देवास्ते गन्धर्वैः साप्सरोगणैः}
{आमन्त्र्य यादवाः सर्वे विप्रजग्मुर्यथागतम्}


\twolineshloka
{यादवाः पार्थमामन्त्र्य अन्तर्द्वीपं गतास्तदा}
{वासुदेवस्तदा पार्थमुवाच यदुनन्दनः}


\twolineshloka
{द्वाविंशद्दिवसान्पार्थ इहोष्य भरतर्षभ}
{मामकं रथमारुह्य शैब्यसुग्रीवयोजितम्}


\fourlineindentedshloka
{सुभद्रया सुखं पार्थ खाण्डवप्रस्थमाविश}
{यादवैः सहितः पश्चादागमिष्यामि भारत}
{यतिवेषेण नियतो वस त्वं रुक्मिणीगृहे ॥वैशंपायन उवाच}
{}


\twolineshloka
{एवमुक्त्वा प्रचक्राम अन्तर्द्वीपं जनार्दनः}
{कृतोद्वाहस्ततः पार्थः कृतकार्योऽभवत्तदा}


\twolineshloka
{तस्यां चोपगतो भावः पार्थस्य सुमहात्मनः}
{तस्मिन्भावः सुभद्राया अन्योन्यं समवर्धत}


\twolineshloka
{स तथा युयुजे वीरो भद्रया भरतर्षभः}
{अभिनिष्पन्नया रामः सीतयेव समन्वितः}


\twolineshloka
{अपि जिष्णुर्विजज्ञे तां ह्रीं श्रियं सन्नतिक्रियाम्}
{देवतानां वरस्त्रीणां रूपेण सदृशीं सतीम्}


\twolineshloka
{स प्रकृत्या श्रिया दीप्त्या संदिदीपे तयाऽधिकम्}
{उद्यत्सहस्रदीप्तांशुः शरदीव दिवाकरः}


\twolineshloka
{सा तु तं मनुजव्याघ्रमनुरक्ता यशस्विनी}
{कन्यापुरगता भूत्वा तत्परा समपद्यत}


\chapter{अध्यायः २४३}
\twolineshloka
{वैशंपायन उवाच}
{}


\twolineshloka
{वृष्म्यन्धकपुरात्तस्मादपयातुं धनञ्जयः}
{विनिश्चित्य तया सार्धं सुभद्रामिदमब्रवीत्}


\twolineshloka
{द्विजानां गुणमुख्यानां यथार्हं प्रतिपादय}
{भोज्यैर्भक्ष्यैश्च कामैश्च स्वपुरीं प्रतियास्यताम्}


\twolineshloka
{आत्मनश्च समुद्दिश्य महाव्रतसमापनम्}
{गच्छ भद्रे स्वयं तूर्णं महाराजनिवेशनम्}


\twolineshloka
{तेजोबलजवोपेतैः शुक्लैर्हयवरोत्तमैः}
{वाजिभिः शैव्यसुग्रीवमेघपुष्पबलाहकैः}


\twolineshloka
{युक्तं रथवरं तूर्णमिहानय सुसत्कृतम्}
{व्रतार्थमिति भाषित्वा सखीभिः सुभगे सह}


\twolineshloka
{क्षिप्रमादाय पर्येहि सह सर्वायुधेन च}
{अनुकर्षान्तपताकाश्च तूणीरांश्च धनूंषि च}


\twolineshloka
{सर्वान्रथवरे स्थाप्य सोत्सेधाश्च महागदाः ॥वैशंपायन उवाच}
{}


\twolineshloka
{अर्जुनेनैवमुक्ता सा सुभद्रा भद्रभाषिणी}
{जगाम नृपतेर्वेश्म सखीभिः सहिता तदा}


\twolineshloka
{व्रतार्थमिति तत्रस्थान्रक्षिणो वाक्यमब्रवीत्}
{रथेनानेन यास्यामि महाव्रतसमापनम्}


\twolineshloka
{शैब्यसुग्रीवयुक्तेन सायुधेनैव शार्ङ्गिणः}
{रथेन रमणीयेन प्रयास्यामि व्रतार्थिनी}


\twolineshloka
{सुभध्रयैवमुक्ते तु जनाः प्राञ्जलयोऽभवन्}
{योजयित्वा रथवरं कल्याणैरभिभाष्य ताम्}


\twolineshloka
{यथोक्तं सर्वमारोप्य आयुधानि च भामिनी}
{क्षिप्रमादाय कल्याणी सुभद्राऽर्जुनमब्रवीत्}


\threelineshloka
{रथोऽयं रथिनां श्रेष्ठ आनीतस्तव शासनात्}
{स त्वं याहि यथाकामं कुरून्कौरवनन्दन ॥वैशंपायन उवाच}
{}


\twolineshloka
{निवेद्य तु रथं भर्तुः सुभद्रा भद्रसंमता}
{ब्राह्मणानां तदा हृष्टा ददौ सा विविधं वसु}


\twolineshloka
{स्नेहवन्ति च भोज्यानि प्रददावीप्सितानि च}
{यथाकामं यथाश्रद्धं वस्त्राणि विविधानि च}


\twolineshloka
{तर्पिता विविधैर्भोज्यैस्तान्यवाप्य वसूनि च}
{ब्राह्मणाः स्वगृहं जग्मुः प्रयुज्य परमाशिषः}


\twolineshloka
{सुभद्रया तु विज्ञप्तः पूर्वमेव धनञ्जयः}
{अभीशुग्रहणे पार्थ न मेऽस्ति सदृशो भुवि}


\twolineshloka
{तस्मात्सा पूर्वमारुह्य रश्मीञ्जग्राह माधवी}
{सोदरा वासुदेवस्य कृतस्वस्त्ययना हयान्}


\twolineshloka
{व्यत्ययित्वा तु तल्लिङ्गं यतिवेषं धनञ्जयः}
{आमुच्य कवचं वीरः समुच्छ्रितमहद्धनुः}


\twolineshloka
{आरुरोह रथश्रेष्ठं शुक्लवासा धनञ्जयः}
{महेन्द्रदत्तं मुकुटं तथैवाभरणानि च}


\twolineshloka
{अलङ्कृत्य तु कौन्तेयः प्रयातुमुपचक्रमे}
{ततः कन्यापुरे घोषस्तुमुलः समपद्यत}


\twolineshloka
{दृष्ट्वा नववरं पार्थं बाणखड्गधनुर्धरम्}
{अभीशुहस्तां सुश्रोणीमर्जुनेन रथे स्थिताम्}


\twolineshloka
{ऊचुः कन्यास्तदा यान्तीं वासुदेवसहोदराम्}
{सर्वकामसमृद्धा त्वं सुभद्रे भद्रभाषिणि}


\twolineshloka
{वासुदेवप्रियं लब्ध्वा भर्तारं वीरमर्जुनम्}
{सर्वसीमन्तिनीनां त्वां श्रेष्ठां कृष्णसहोदरीम्}


\twolineshloka
{मन्यामहे महाभागे सुभद्रे भद्रभाषिणि}
{यस्मात्सर्वमनुष्याणां श्रेष्ठो भर्ता तवार्जुनः}


\threelineshloka
{उपपन्नस्त्वया वीरः सर्वलोकमहारथः}
{हे प्रयाहि गृहान्भद्रे सुहृद्भिः संगमोऽस्तु ते ॥वैशंपायन उवाच}
{}


\twolineshloka
{एवमुक्ता प्रहृष्टाभिः सखीभिः प्रतिनन्दिता}
{भद्रा भद्रजवोपेतानश्वान्पुनरचोदयत्}


\twolineshloka
{पार्श्वे चामरहस्ता सा सखी तस्याङ्गनाऽभवत्}
{ततः कन्यापुरद्वारात्सघोषादभिनिःसृतम्}


\twolineshloka
{ददृशुस्तं रथश्रेष्ठं जना जीमूतनिस्वनम्}
{सुभद्रासङ्गृहीतस्य रथस्य महतः स्वनम्}


\twolineshloka
{मेघस्वनमिवाकाशे शुश्रुवुः पुरवासिनः}
{सुभद्रया तु संपन्ने तिष्ठन्रथवरेऽर्जुनः}


\twolineshloka
{प्रबभौ च तयोपेतः कैलास इव गङ्गया}
{पार्थः सुभद्रासहितो विरराज महारथः}


\twolineshloka
{विराजते यथा शक्रो राजञ्शच्या समन्वितः}
{सुभद्रां प्रेक्ष्य पार्थेन ह्रियमाणां यशस्विनीम्}


\twolineshloka
{चक्रुः किलकिलाशब्दानासाद्य बहवो जनाः}
{दाशार्हाणां कुलस्य श्रीः सुभद्रा मद्रभाषिणी}


\twolineshloka
{अभिकामा सकामेन पार्थेन सह गच्छति}
{अथापरे तु संक्रुद्धा गृह्णीत घ्नत माचिरम्}


\twolineshloka
{इति संवार्य शस्त्राणि ववर्षुरभितो दिशम्}
{इति संभाषमाणानां स नादः सुमहानभूत्}


\twolineshloka
{स तेन जनघोषेण वीरो गज इवार्दितः}
{ववर्ष शरवर्षाणि न तु कंचन रोषयत्}


\twolineshloka
{मुमोच निशितान्बाणान्दीप्यमानान्स्वतेजसा}
{प्रासादवरसङ्घेषु हर्म्येषु भवनेषु च}


\twolineshloka
{क्षोभयित्वा पुरश्रेष्ठं गरुत्मानिव सागरम्}
{प्रेक्षन्रैवकतद्वारं निर्ययौ भरतर्षभः}


\chapter{अध्यायः २४४}
\twolineshloka
{वैशंपायन उवाच}
{}


\twolineshloka
{शासनात्पुरुषेन्द्रस्य बलेन महता बली}
{गिरौ रैवतके नित्यं बभूव विपृथुश्रवाः}


\twolineshloka
{प्रवासे वासुदेवस्य तस्मिन्हलधरोपमः}
{संबभूव तदा गोप्ता पुरस्य पुरवर्धनः}


\twolineshloka
{प्राप्य पाण्डवनिर्याणं निर्ययौ विपृथुश्रवाः}
{निशम्य पुरनिर्घोषं स्वमनीकमचोदयत्}


\twolineshloka
{सोऽभिपत्य तदाध्वानं ददर्श पुरुषर्षभम्}
{निःसृतं द्वारकाद्वारादंशुमन्तमिवाम्बिरात्}


\twolineshloka
{सविद्युतमिवाम्भोदं प्रेक्षतां तं धनुर्धरम्}
{पार्थमानर्तयोधानां विस्मयः समपद्यत}


\twolineshloka
{उदीर्णरथनागाश्वमनीकमभिवीक्ष्य तत्}
{उवाच परमप्रीता सुभद्रा भद्रभाषिणी}


\twolineshloka
{संग्रहीतुमभिप्रायो दीर्घकालकृतो मम}
{युध्यमानस्य सङ्ग्रामे रथं तव नरर्षभ}


\twolineshloka
{ओजस्तेजोद्युतिबलैरन्वितस्य महात्मनः}
{पार्थ ते सारथित्वेन भविता शिक्षितास्म्यहम्}


\twolineshloka
{एवमुक्तः प्रियां प्रीतः प्रत्युवाच नरर्षभः}
{चोदयाश्वानसंसक्तान्विश्तु विपृथोर्बलम्}


\threelineshloka
{बहुभिर्युध्यमानस्य तावकान्विजिघांसतः}
{पश्य बाहुबलं भद्रे शरान्विक्षिपतो मम ॥वैशंपायन उवाच}
{}


\twolineshloka
{एवमुक्ता तदा भद्रा पार्थेन भरतर्षभ}
{चुचोद साश्वान्संहृष्टा ते ततो विविशुर्बलम्}


\twolineshloka
{तदाहतमहावाद्यं समुदग्रध्वजायुतम्}
{अनीकं विपृथोर्हृष्टं पार्थमेवान्ववर्तत}


\twolineshloka
{रथैर्बहुविधाकारैः सदश्वैश्च महाजवैः}
{किरन्तः शरवर्षाणि परिवव्रुर्धनञ्जयम्}


\twolineshloka
{तेषामस्त्राणि संवार्य दिव्यास्त्रेण महास्त्रवित्}
{आवृणोन्महदाकाशं शरैः परपुरञ्जयः}


\twolineshloka
{तेषां बाणान्महाबाहुर्मुकुटान्यङ्गदानि च}
{चिच्छेद निशितैर्बाणैः शरांश्चैव धनूंषि च}


\twolineshloka
{युगानीषान्वरूथानि यन्त्राणि विविधानि च}
{अजिघांसन्परान्पार्थश्चिच्छेद निशितैः शरैः}


\twolineshloka
{विधनुष्कान्विकवचान्विरथांश्च महारथान्}
{कृत्वा पार्थः प्रियां प्रीतः प्रेक्ष्यतामित्यदर्शयत्}


\twolineshloka
{सा दृष्ट्वा महदाश्चर्यं सुभद्रा पार्थमब्रवीत्}
{अवाप्तार्थाऽस्मि भद्रं ते याहि पार्थ यथासुखम्}


\twolineshloka
{स सक्तं पाण्डुपुत्रेण समीक्ष्य विपृथुर्बलम्}
{त्वरमाणोऽभिसंक्रम्य स्थीयतामित्यभाषत}


\twolineshloka
{ततः सेनापतेर्वाक्यं नात्यवर्तन्त यादवाः}
{सागरे मारुतोद्धूता वेलामिव महोर्मयः}


\twolineshloka
{ततो रथवरात्तूर्णमवरुह्य नरर्षभः}
{अभिगम्य नरव्याघ्रं प्रहृष्टः परिषस्वजे}


\twolineshloka
{सोऽब्रवीत्पार्थमासाद्य दीर्घकालमिदं तव}
{निवासमभिजानामि शङ्खचक्रगदाधरात्}


\twolineshloka
{न मेऽस्त्यविदितं किंचिद्यद्यदाचितं त्वया}
{सुभद्रार्थं प्रलोभेन प्रीतस्तव जनार्दनः}


\twolineshloka
{प्राप्तस्य यतिलिङ्गेन वासितस्य धनञ्जय}
{बन्धुमानसि रामेण महेन्द्रावरजेन च}


\twolineshloka
{मामेव च सदाकाङ्क्षी मन्त्रिणं मधुसूदनः}
{अन्तरेण सुभद्रां च त्वां च तात धनञ्जय}


\twolineshloka
{इमं रथवरं दिव्यं सर्वशस्त्रसमन्वितम्}
{इदमेवानुयात्रं च निर्दिश्य गदपूर्वजः}


\twolineshloka
{अन्तर्द्वीपं तदा वीर गतो वृष्णिसुखावहः}
{दीर्घकालावरुद्धं त्वां संप्राप्तं प्रियया सह}


\twolineshloka
{पश्यन्तु भ्रातरः सर्वे वज्रपाणिमिवामराः}
{आयाते तु दशार्हाणामृषभे शार्ङ्गधन्वनि}


\twolineshloka
{भद्रामनुगमिष्यन्ति रत्नानि च वसूनि च}
{अरिष्टं याहि पन्थानं सुखी भव धनञ्जय}


\twolineshloka
{नष्टशोकैर्विशोकस्य सुहृद्भिः संगमोऽस्तु ते ॥वैशंपायन उवाच}
{}


\twolineshloka
{ततो विपृथुमामन्त्र्य पार्थः प्रीतोऽभिवाद्य च}
{कृष्णस्य मतमास्थाय कृष्णस्य रथमास्थितः}


\twolineshloka
{पूर्वमेव तु पार्थाय कृष्णेन विनियोजितम्}
{सर्वरत्नसुसंपूर्णं सर्वभोगसमन्वितम्}


\twolineshloka
{रथेन काञ्चनाङ्गेन कल्पितेन यथाविधि}
{शैब्यसुग्रीवयुक्तेन किङ्किणीजालमालिना}


\twolineshloka
{सर्वशस्त्रोपपन्नेन जीमूतरवनादिना}
{ज्वलनार्चिःप्रकाशेन द्विषतां हर्षनाशिना}


\threelineshloka
{सन्नद्धः कवची खड्गी बद्धगोधाङ्गुलित्रवान्}
{युक्तः सेनानुयात्रेण रथणारोप्य माधवीम्}
{रथेनाकाशगेनैव पययौ *स्वपुरं प्रति}


\twolineshloka
{ह्रियमाणां तु तां दृष्ट्वा सुभद्रां सैनिका जनाः}
{विक्रोशन्तोऽद्रवन्सर्वे द्वारकामभितः पुरीम्}


\twolineshloka
{ते समासाद्य सहिताः सुधर्मामभितः सभाम्}
{सभापालस्य तत्सर्वमाचख्युः पार्थविक्रमम्}


\twolineshloka
{तेषां श्रुत्वा सभापालो भेरीं सान्नाहिकीं ततः}
{समाजघ्ने महाघोषां जाम्बूनदपरिष्कृताम्}


\threelineshloka
{क्षुब्धास्तेनाथ शब्देन भोजवृष्ण्यन्धकास्तदा}
{`अन्तर्द्वीपात्समुत्पेतुः सहसा सहितास्तदा}
{'अन्नपानमपास्याथ समापेतुः समन्ततः}


\twolineshloka
{तत्र जाम्बूनदाङ्गानि स्पर्ध्यास्तरणवन्ति च}
{मणिविद्रुमचित्राणि ज्वलिताग्निप्रभाणि च}


\twolineshloka
{भेजिरे पुरुषव्याघ्रा वृष्ण्यन्धकमहारथाः}
{सिंहासनानि शतशो धिष्ण्यानीव हुताशनाः}


\twolineshloka
{तेषां समुपविष्टानां देवानामिव सन्नये}
{आचख्यौ चेष्टितं जिष्णोः सभापालः सहानुगः}


\twolineshloka
{तच्छ्रुत्वा वृष्णिवीरास्ते मदसंरक्तलोचनाः}
{अमृष्यमाणाः पार्थस्य समुत्पेतुरहङ्कृताः}


\twolineshloka
{योजयध्वं रथानाशु प्रासानाहरतेति च}
{धनूंषि च महार्हाणि कवचानि बृहन्ति च}


\twolineshloka
{सूतानुच्चुक्रुशुः केचिद्रथान्योजयतेति च}
{स्वयं च तुरगान्केचिदयुञ्जन्हेमभूषितान्}


\twolineshloka
{रथेष्वानीयमानेषु कवचेषु ध्वजेषु च}
{अभिक्रन्दे नृवीराणां तदासीत्तुमुलं महत्}


\twolineshloka
{वनमाली ततः क्षीबः कैलासशिखरोपमः}
{नीलवासा मदोत्सिक्त इदं वचनमब्रवीत्}


\twolineshloka
{किमिदं कुरुथाप्रज्ञास्तूष्णींभूते जनार्दने}
{अस्य भावमविज्ञाय संक्रुद्धा मोघगर्जिताः}


\twolineshloka
{एष तावदभिप्रायमाख्यातु स्वं महामतिः}
{यदस्य रुचिरं कर्तुं तत्कुरुध्वमतन्द्रिताः}


\twolineshloka
{ततस्ते तद्वचः श्रुत्वा ग्राह्यरूपं हलायुधात्}
{तूष्णींभूतास्ततः सर्वे साधुसाध्विति चाब्रुवन्}


\twolineshloka
{समं वचो निशम्यैव बलदेवस्य धीमतः}
{पुनरेवसभामध्ये सर्वे ते समुपाविशन्}


\threelineshloka
{ततोऽब्रवीद्वासुदेवं वचो रामः परन्तपः}
{`त्रैलोक्यनाथ हे कृष्ण भूतभव्यभविष्यकृत्}
{'किमवागुपविष्टोऽसि प्रेक्षमाणो जनार्दन}


\twolineshloka
{सत्कृतस्त्वत्कृते पार्थः सर्वैरस्माभिरच्युत}
{न च सोऽर्हति तां पूजां दुर्बुद्धिः कुलपांसनः}


\twolineshloka
{को हि तत्रैव भुक्तावान्नं भाजनं भेत्तुमर्हति}
{मन्यमानः कुले जातमात्मानं पुरुषः क्वचित्}


\twolineshloka
{इच्छन्नेव हि संबन्धं कृतं पूर्वं च मानयन्}
{को हि नाम भवेनार्थी साहसेन समाचरेत्}


\twolineshloka
{सोऽवमन्य तथाऽस्माकमनादृत्य च केशवम्}
{प्रसह्य हृतवानद्य सुभद्रां मृत्युमात्मनः}


\twolineshloka
{कथं हि शिरसो मध्ये कृतं तेन पदं मम}
{मर्षयिष्यामि गोविन्द पादस्पर्शमिवोरगः}


\twolineshloka
{अद्य निष्कौरवामेकः करिष्यामि वसुन्धराम्}
{न हि मे मर्षणीयोऽयमर्जुनस्य व्यतिक्रमः}


\twolineshloka
{तं तथा गर्जमानं तु मेघदुन्दुभिनिःस्वनम्}
{अन्वपद्यन्त ते सर्वे भोजवृष्ण्यन्धकास्तदा}


\chapter{अध्यायः २४५}
\twolineshloka
{वैशंपायन उवाच}
{}


\twolineshloka
{उक्तवन्तो यथावीर्यमसकृत्सर्ववृष्णयः}
{ततोऽब्रवीद्वासुदेवो वाक्यं धर्मार्थसंयुतम्}


\threelineshloka
{`मयोक्तं न श्रुतं पूर्वं सहितैः सर्वयादवैः}
{अतिक्रान्तमतिक्रान्तं न निवर्तेत कर्हिचित्}
{शृणुध्वं सहिताः सर्वे मम वाक्यं सहेतुकम् ॥'}


\twolineshloka
{नावमानं कुलस्यास्य गुडाकेशः प्रयुक्तवान्}
{संमानोऽभ्यधिकस्तेन प्रयुक्तोऽयं न संशयः}


\twolineshloka
{अर्थलुब्धान्न वः पार्थो मन्यते सात्वतान्सदा}
{स्वयंवरमनाधृष्यं मन्यते चापि पाण्डवः}


\twolineshloka
{प्रदानमपि कन्यायाः पशुवत्को नु मन्यते}
{विक्रयं चाप्यपत्यस्य कः कुर्यात्पुरुषो भुवि}


\threelineshloka
{एतान्दोषांस्तु कौन्तेयो दृष्टवानिति मे मतिः}
{`क्षत्रियाणां तु वीर्येण प्रशस्तं हरणं बलात्}
{'अतः प्रसह्य हृतवान्कन्यां धर्मेण पाण्डवः}


\twolineshloka
{उचितश्चैव संबन्धः सुभद्रा च शयस्विनी}
{एष चापीदृशः पार्थः प्रसह्य हृतवानतः}


\twolineshloka
{भरतस्यान्वये जातं शान्तनोश्च यशस्विनः}
{कुन्तिभोजात्माजापुत्रं का बुभूषेत नार्जुनम्}


\twolineshloka
{न तं पश्यामि यः पार्थं विजयेत रणे बलात्}
{वर्जयित्वा विरूपाक्षं भगनेत्रहरं हरम्}


\twolineshloka
{अपि सर्वेषु लोकेषु सेन्द्ररुद्रेषु मारिष}
{स च नाम रथस्तादृङ्मदीयास्ते च वाजिनः}


\threelineshloka
{`मम शस्त्रं विशेषेण तूणौ चाक्षयसायकौ}
{'योद्धा पार्थश्च शीघ्रास्त्रः को नु तेन समो भवेत्}
{तमभिद्रुत्य सान्त्वेन परमेण धनञ्जयम्}


\twolineshloka
{निवर्तयत संहृष्टा ममैषा परमा मतिः}
{यदि निर्जित्य वः पार्थो बलाद्गच्छेत्स्वकं पुरं}


\threelineshloka
{प्रणश्येद्वो यशः सद्यो न तु सान्त्वे पराजयः}
{`पितृष्वसायाः पुत्रो मे संबन्धं नार्हति द्विषाम्}
{'तच्छ्रुत्वा वासुदेवस्य तथा कर्तुं जनाधिप}


\twolineshloka
{`उद्योगं कृतवन्तस्ते भेरीं सन्नाद्य यादवाः}
{अर्जुनस्तु तदा श्रुत्वा भेरीसन्नादनं महत्}


\twolineshloka
{कौन्तेयस्त्वरमाणस्तु सुभद्रामभ्यभाषत}
{आयान्ति वृष्णयः सर्वे ससुहृज्जनबान्धवाः}


\twolineshloka
{त्वदर्थं योद्धुकामास्ते मदरक्तान्तलोचनाः}
{प्रमत्तानशुचीन्मूढान्सुरामत्तान्नराधमान्}


\twolineshloka
{वमनं पानशीलांस्तान्करिष्यामि शरोत्तमैः}
{उताहो वा मदोन्मत्तान्नयिष्यामि यमक्षयम्}


\twolineshloka
{एवमुक्त्वा प्रियां पार्थो न्यवर्तत महाबलः}
{निवर्तमानं दृष्ट्वैव सुभद्रा त्रस्ततां गता}


\twolineshloka
{एवं मा वद पार्थेति पादयोः पतिता तदा}
{सुभद्रा तु कलिर्जाता वृष्णीनां निधाय च}


\fourlineindentedshloka
{एवं ब्रुवन्तः पौरास्ते ह्यपवादरताः प्रभो}
{मम शोकं वर्धयन्ति तस्मान्नाशं न चिन्तये}
{परिवादभयान्मुक्ता त्वत्प्रसादाद्भवाम्यहम् ॥वैशंपायन उवाच}
{}


\twolineshloka
{एवमुक्तस्ततः पार्थः प्रियया भद्रया तदा}
{गमनाय मतिं चक्रे पार्थः सत्यपराक्रमः}


\twolineshloka
{स्तितपूर्वं तदाऽऽभाष्य परिष्वज्य प्रियां तदा}
{उत्थाप्य च पुनः पार्थो याहि याहीति चाब्रवीत्}


\twolineshloka
{ततः सुभद्रा त्वरिता रश्मीन्संगृह्य पाणिना}
{चोदयामास जवनाञ्शीग्रमश्वान्कृतत्वरा}


\twolineshloka
{ततस्तु कृतसन्नाहा वृष्णिवीराः समाहिताः}
{प्रत्यानयार्थं पार्थस्य जवनैस्तुरगोत्तमैः}


\twolineshloka
{राजमार्गमनुप्राप्ता दृष्ट्वा पार्थस्य विक्रमम्}
{प्रासादपङ्क्तिस्तम्भेषु वेदिकासु ध्वजेषु च}


\twolineshloka
{अर्जुनस्य शरान्दृष्ट्वा विस्मयं परमं गताः}
{केशवस्य वचस्तथ्यं मन्यमानास्तु यादवाः}


\twolineshloka
{अतीत्य तं रैवतकं श्रुत्वा तु विपृथोर्वचः}
{अर्जुनेन कृतं श्रुत्वा गन्तुकामास्तु वृष्णयः}


\twolineshloka
{श्रुत्वा दीर्घं गतं पार्थं न्यवर्तन्त महारथाः}
{पुरोद्यानमतिक्रम्य विशालं च गिरिव्रजम्}


\twolineshloka
{सानुमुज्जयिनीं चैव वनान्युपवनानि च}
{पुण्येष्वानर्तराष्ट्रेषु वापीपद्मसरांसि च}


\twolineshloka
{प्राप्य धेनुमतीतीर्थमश्वरोधसरः प्रति}
{प्रेक्षावर्तं ततः शैलमम्बुदं च नगोत्तमम्}


\twolineshloka
{आराच्छृङ्गमथासाद्य तीर्णः करवतीं नदीम्}
{प्राप्य साल्वेयराष्ट्राणि निषधानप्यतीत्य च}


\twolineshloka
{देवापृथुपुरं पश्यन् सर्वतः सुसमाहितः}
{तमतीत्य महाबाहुर्देवारण्यमपश्यत}


\twolineshloka
{पूजयामासुरायान्तं देवारण्ये महर्षयः}
{स वनानि नदीः शैलान् गिरिप्रस्रवणानि च}


\twolineshloka
{अतीत्य च तदा पार्थः सुभद्रासारथिस्तदा}
{कौरवं विषयं प्राप्य विशोकः समपद्यत}


\twolineshloka
{सोदर्याणां महाबाहुः सिंहाशयमिवाशयम्}
{दूरादुपवनोपेतं समन्तात्सलिलावृतम्}


% Check verse!
भद्रया मुदितो जिष्णुर्ददर्श वृजिनं पुरम्
\chapter{अध्यायः २४६}
\twolineshloka
{वैशंपायन उवाच}
{}


\twolineshloka
{क्रोशणात्रे पुरस्यासीद्गोष्ठं पार्थस्य शोभनम्}
{तत्रापि यात्वा बीभत्सुर्निविष्टो यदुकन्यया}


\twolineshloka
{ततः सुभद्रां सत्कृत्य पार्थो वचनमब्रवीत्}
{गोपिकानां तु वेषेण गच्छ त्वं वृजिनं पुरम्}


\twolineshloka
{कामव्याहारिणी कृष्णा रोचतां ते वचो मम}
{दृष्ट्वा तु परुषं ब्रूयात्सह तत्र मयागताम्}


\twolineshloka
{अन्यवेषेण तु गतां दृष्ट्वा सा त्वां प्रियं वदेत्}
{यत्तु सा प्रथमं ब्रूयान्न तस्यास्ति निवर्तनम्}


\twolineshloka
{तस्मान्मानं च दर्पं च व्यपनीय स्वयं व्रज}
{तस्य तद्वचनं श्रुत्वा सुभद्रा प्रत्यभाषत}


\twolineshloka
{एवमेतत्करिष्यामि यथा त्वं पार्थ भाषसे}
{सुभद्रावचनं श्रुत्वा सुप्रीतः पाकशासनिः}


\twolineshloka
{गोपालान्स समानीय त्वरितो वाक्यमब्रवीत्}
{तरुम्यः सन्ति यावन्त्यस्ताः सर्वा व्रजयोषितः}


\twolineshloka
{आगच्छन्तु गमिष्यन्त्या भद्रया सह सङ्गताः}
{इन्द्रप्रस्थं पुरवरं कृष्णां द्रष्टुं यशस्विनीम्}


\twolineshloka
{एतच्छ्रुत्वा तु गोपालैरानीता व्रजयोषितः}
{ततस्ताभिः परिवृतां व्रजस्त्रीभिः समन्ततः ॥'}


\twolineshloka
{सुभद्रां त्वरमाणश्च रक्तकौशेयवासिनीम्}
{पार्थः प्रस्थापयामास कृत्वा गोपालिकावपुः}


\twolineshloka
{साऽधिकं तेन रूपेण शोभमाना यशस्विनी}
{`गोपालिकामध्यगता प्रययौ वृजिनं पुरम्}


\twolineshloka
{त्वरिता खाण्डवप्रस्थमाससाद विवेश च}
{'भवनं श्रेष्ठमासाद्य वीरपत्नी वराङ्गना}


\twolineshloka
{ववन्दे पृथुताम्राक्षी पृथां भद्रा पितृष्वसाम्}
{तां कुन्ती चारुसर्वाङ्गीमुपाजिघ्रत मूर्धनि}


\twolineshloka
{प्रीत्या परमया युक्ता आशीर्भिर्युञ्जताऽतुलाम्}
{ततोऽभिगम्य त्वरिता पूर्णेन्दुसदृशानना}


\twolineshloka
{ववन्दे द्रौपदीं भद्रा प्रेष्याऽहमिति चाब्रवीत्}
{प्रत्युत्थाय तदा कृष्णा स्वसारं माधवस्य च}


\twolineshloka
{परिष्वज्यावदत्प्रीत्या निःसपत्नोस्तु ते पतिः}
{`वीरसूर्भव भद्रे त्वं भव भर्तृप्रिया तथा}


\twolineshloka
{ओजसा निर्मिता बह्वीरुवाच परमाशिषः}
{'तथैव मुदिता भद्रा तामुवाच तथास्त्विति}


\twolineshloka
{`ततः सुभद्रां वार्ष्णेयी परिष्वज्य शुभाननाम्}
{अङ्के निवेश्य मुदिता वसुदेवं प्रशस्य च}


\twolineshloka
{ततः किलकिलाशब्दः क्षणेन समपद्यत}
{हर्षादानर्तयोधानामासाद्य वृजिनं पुरम्}


\twolineshloka
{देवपुत्रप्रकाशास्ते जाम्बूनदमयध्वजाः}
{पृष्ठतोऽनुययुः पार्थं पुरुहूतमिवामराः}


\twolineshloka
{गोभिरुष्ट्रैः सदश्वैश्च युक्तानि बहुला जनाः}
{ददृशुर्यानमुख्यानि दाशार्हपुरवासिनाम्}


\twolineshloka
{ततः पुरवरे यूनां पुंसां वाच उदीरिताः}
{अर्जुने प्रतियाति स्म अश्रूयन्त समन्ततः}


\threelineshloka
{प्रवासादागतं पार्थं दृष्ट्वा स्वमिव बान्धवम्}
{सोऽभिगम्य नरश्रेष्ठो दाशार्हशतसंवृतः}
{}


\twolineshloka
{पौरैः पुरवरं प्रीत्या परया चाभिनन्दितः}
{प्राप्य चान्तःपुरद्वारमवरुह्य नरर्षभः}


\twolineshloka
{ववन्दे धौम्यमासाद्य मातरं च धनञ्जयः}
{स्पृष्ट्वा च चरणौ राज्ञो भीमस्य च धनञ्जयः}


\twolineshloka
{यमाभ्यां वन्दितो हृष्टः सस्वजे तौ ननन्द च}
{ब्राह्मणप्रमुखान्सर्वान्भ्रातृभिः सह सङ्गतः}


\twolineshloka
{यथार्हं मानयामास पौरजानपदानपि}
{तत्रस्थान्यनुयातानि तीर्थान्यायतनानि च}


\twolineshloka
{निवेदयामास तदा राज्ञे सर्वं स्वनुष्ठितम्}
{भ्रातृभ्यश्चैव सर्वेभ्यः कथयामास भारत}


\twolineshloka
{श्रुत्वा सर्वं महाप्राज्ञो धर्मराजो युधिष्ठिरः}
{पुरस्तादेव तेषां तु पूजयामास चार्जुनम्}


\twolineshloka
{पाण्डवेन यथार्हं तु पूजार्हेण सुपूजितः}
{न्यविशच्चाभ्यनुज्ञातो राज्ञा तुष्टो यशस्विना}


\twolineshloka
{तामदीनां सुपूजार्हां सुभद्रां प्रीतिवर्धिनीम्}
{साक्षाच्छ्रियममन्यन्त पार्थाः कृष्णसहोदराम्}


\twolineshloka
{गुरूणां श्वशुराणां च देवराणां तथैव च}
{सुभद्रा स्वेन वृत्तेन बभूव परमप्रिया ॥'}


\twolineshloka
{ततस्ते हृष्टमनसः पाण्डवेया महारथाः}
{कुन्ती च परमप्रीता कृष्णा च सततं तथा}


\chapter{अध्यायः २४७}
\twolineshloka
{वैशंपायन उवाच}
{}


\twolineshloka
{अथ शुश्राव निर्वृत्ते वृष्णीनां परमोत्सवे}
{अर्जुनेन हृतां भद्रां शङ्खचक्रगदाधरः}


\twolineshloka
{पुरस्तादेव पौराणां संशयः समजायत}
{जानता वासुदेवेन वासितो भरतर्षभः}


\twolineshloka
{लोकस्य विदितं ह्यद्य पूर्वं विपृथुना यथा}
{सान्त्वयित्वाभ्यनुज्ञातो भद्रया सह सङ्गतः}


\twolineshloka
{दित्सता सोदरां तस्मै पतत्त्रिवरकेतुना}
{अर्हते पार्थिवेन्द्राय पार्थायायतलोचनाम्}


\twolineshloka
{सत्कृत्य पाण्डवश्रेष्ठं प्रेषयामास चार्जुनम्}
{भद्रया सह बीभत्सुः प्रापितो वृजिनं पुरम्}


\twolineshloka
{इति पौरजनाः सर्वे वदन्ति च समन्ततः}
{'श्रुत्वा तु पुण्डरीकाक्षः संप्राप्तं स्वपुरोत्तमम्}


\twolineshloka
{अर्जुनं पाण्डवश्रेष्ठमिन्द्रप्रस्थगतं तथा}
{`यियासुः खाण्डवप्रस्थममन्त्रयत केशवः}


\twolineshloka
{पूर्वं सत्कृत्य राजानमाहुकं मधुसूदनः}
{तथा विपृथुमक्रूरं संकर्षणविडूरथौ}


\twolineshloka
{मन्त्रयामास तैः सार्धं पुरस्तादभिमानितैः}
{संकर्षणेन संमन्त्र्य ह्यनुज्ञातो महामनाः}


\twolineshloka
{संप्रीतः प्रीयमाणेन वृष्णिराज्ञा जनार्दनः}
{अभिमन्त्र्याभ्यनुज्ञातो योजयामास वाहिनीम्}


\twolineshloka
{ततस्तु यानान्यासाद्य दाशार्हाणां यशस्विनाम्}
{सिंहनादः प्रहृष्टानां क्षणेन समपद्यत}


\twolineshloka
{योजयन्तः सदश्वांस्तु यानयुग्यं रथांस्तथा}
{गजांश्च परमप्रीतः समपद्यन्त वृष्णयः ॥'}


\twolineshloka
{वृष्ण्यन्धकमहामात्रैः सह वीरैर्महारथैः}
{भ्रातृभिश्च कुमारैश्च योधैश्च शतशो वृतः}


\twolineshloka
{सैन्येन महता शौरिरभिगुप्तः समन्ततः}
{तत्र दानपतिः श्रीमाञ्जगाम स महायशाः}


\twolineshloka
{अक्रूरो वृष्णिवीराणां सेनापतिररिन्दमः}
{अनाधृष्टिर्महातेजा उद्धवश्च महायशाः}


\twolineshloka
{साक्षाद्वृहस्पतेः शिष्यो महाबुद्धिर्महामनाः}
{सत्यकः सात्यकिश्चैव कृतवर्मा च सात्वतः}


\twolineshloka
{प्रद्युम्नश्चैव साम्बश्च निशङ्कुः शङ्कुरेव च}
{चारुदेष्णश्च विक्रान्तो झिल्ली विपृथुरेव च}


\twolineshloka
{सारणश्च महाबाहुर्गदश्च विदुषां वरः}
{एते चान्ये च बहवो वृष्णिभोजान्धकास्तथा}


\twolineshloka
{आजग्मः खाण्डवप्रस्थमादाय हरणं बहु}
{`उपहारं समादाय पृथुवृष्णिपुरोगमाः}


\threelineshloka
{प्रययुः सिंहनादेन सुभध्रामवलोककाः}
{ते त्वदीर्घेण कालेन कृष्णेन सह यादवाः}
{पुरमासाद्य पार्थानां परां प्रीतिमवाप्नुवन् ॥'}


\twolineshloka
{ततो युधिष्ठिरो सजा श्रुत्वा माघवमागतम्}
{प्रतिग्रहार्थं कृष्णस्य यमौ प्रास्थापयत्तदा}


\twolineshloka
{ताभ्यां प्रतिगृहीतं तु वृष्णिचक्रं महर्द्धिमत्}
{विवेश खाण्डवप्रस्थं पताकाध्वजशोभितम्}


\twolineshloka
{संमृष्टसिक्तपन्थानं पुष्पप्रकरशोभितम्}
{चन्दनस्य रसैः शीतैः पुम्यगन्धैर्निषेवितम्}


\twolineshloka
{दह्यताऽगुरुणा चैव देशे देशे सुगन्धिना}
{हृष्टपुष्टजनाकीर्णं वणिग्भिरुपशोभितम्}


\twolineshloka
{प्रतिपेदे महाबाहुः सह रामेण केशवः}
{वृष्ण्यन्धकैस्तथा भोजैः समेतः पुरुषोत्तमः}


\twolineshloka
{संपूज्यमानः पौरैश्च ब्राह्मणैश्च सहस्रशः}
{विवेश भवनं राज्ञः पुरन्दरगृहोपमम्}


\twolineshloka
{युधिष्ठिरस्तु रामेण समागच्छद्यथाविधि}
{मूर्ध्नि केशवमाघ्राय बाहुभ्यां परिषस्वजे}


\twolineshloka
{तं प्रीयमाणो गोविन्दो विनयेनाभिपूजयन्}
{भीमं च पुरुषव्याघ्रं विधिवत्प्रत्यपूजयत्}


\twolineshloka
{तांश्च वृष्ण्यन्धकश्रेष्ठान्कुन्तीपुत्रो युधिष्ठिरः}
{प्रतिजग्राह सत्कारैर्यथाविधि यथागतम्}


\twolineshloka
{गुरुवत्पूजयामास कांश्चित्कांश्चिद्वयस्यवत्}
{कांश्चिदभ्यवदत्प्रेम्णा कैश्चिदप्यभिवादितः}


\twolineshloka
{`ततः पृथा च पार्थाश्च मुदिताः कृष्णया सह}
{पुण्डरीकाक्षमासाद्य बभूवुर्मुदितेन्द्रियाः}


\twolineshloka
{हर्षादभिगतौ दृष्ट्वा संकर्षणजनार्दनौ}
{बन्धुमन्तं पृथा पार्थं युधिष्ठिरममन्यत}


\twolineshloka
{ततः सङ्कर्षणाक्रूरावप्रमेयावदीनवत्}
{भद्रवत्यै सुभद्रायै धनौघमुपजह्रतुः}


\twolineshloka
{प्रवालानि च हाराणि भूषणानि सहस्रशः}
{कुथास्तरपरिस्तोमान्व्याघ्राजिनपुरस्कृतान्}


\twolineshloka
{विविधैश्चैव रंत्नौगैर्दीप्तप्रभमजायत}
{शयनासनयानैश्च युधिष्ठिरनिवेशनम्}


\twolineshloka
{ततः प्रीतिकरो यूनां विवाहपरमोत्सवः}
{भद्रवत्यै सुभद्रायै सप्तरात्रमवर्तत ॥'}


\twolineshloka
{तेषां ददौ हृषीकेशो जन्यार्थे धनमुत्तमम्}
{हरणं वै सुभद्राया ज्ञातिदेयं महायशाः}


\twolineshloka
{रथानां काञ्चनाङ्गानां किङ्किणीजालमालिनाम्}
{चतुर्युजामुपेतानां सूतैः कुशलशिक्षितैः}


\twolineshloka
{सहस्रं प्रददौ कृष्मो गवामयुतमेव च}
{श्रीमान्माथुरदेश्यानां दोग्ध्रीणां पुण्यवर्चसाम्}


\twolineshloka
{बडवानां च शुद्धानां चन्द्रांशुसमवर्चसाम्}
{ददौ जनार्दनः प्रीत्या सहस्रं हेमभूषितम्}


\twolineshloka
{तथैवाश्वतरीणां च दान्तानां वातरंहसाम्}
{शतान्यञ्जनकेशीनां श्वेतानां पञ्चपञ्च च}


\twolineshloka
{स्नानपानोत्सवे चैव प्रयुक्तं वयसान्वितम्}
{स्त्रीणां सहस्रं गौरीणां सुवेषाणां सुवर्चसाम्}


\twolineshloka
{सुवर्णशतकण्ठीनामरोमाणां स्वलङ्कृताम्}
{परिचर्यासु दक्षाणां प्रददौ पुष्करेक्षणः}


\twolineshloka
{पृष्ठ्यानामपि चाश्वानां बाह्लिकानां जनार्दनः}
{ददौ शतसहस्राख्यं कन्याधनमनुत्तमम्}


\twolineshloka
{कृताकृतस्य मुख्यस्य कनकस्याग्निवर्चसः}
{मनुष्यभारान्दाशार्हो ददौ दश जनार्दनः}


\twolineshloka
{गजानां तु प्रभिन्नानां त्रिधा प्रस्रवतां मदम्}
{गिरिकूटनिकाशानां समरेष्वनिवर्तिनाम्}


\twolineshloka
{क्लृप्तानां पटुघण्टानां चारूणां हेममालिनाम्}
{हस्त्यारोहैरुपेतानां सहस्रं साहसप्रियः}


\twolineshloka
{रामः पाणिग्रहणिकं ददौ पार्थाय लाङ्गली}
{प्रीयमाणो हलधरः संबन्धं प्रति मानयन्}


\twolineshloka
{स महाधनरत्नौघो वस्त्रकम्बलफेनवान्}
{महागजमहाग्राहः पताकाशैवलाकुलः}


\twolineshloka
{पाण्डुसागरमाविद्धः प्रविवेश महाधनः}
{पूर्णमापूरयंस्तेषां द्विषच्छोकावहोऽभवत्}


\twolineshloka
{प्रतिजग्राह तत्सर्वं धर्मराजो युधिष्टिरः}
{पूजयामास तांश्चैव वृष्ण्यन्धकमहारथान्}


\twolineshloka
{ते समेता महात्मानः कुरुवृष्ण्यन्धकोत्तमाः}
{विजह्रुरमरावासे नराः सुकृतिनो यथा}


\twolineshloka
{तत्रतत्र महानादैरुत्कृष्टतलनादितैः}
{यथायोगं यथाप्रीति विजह्रुः कुरुवृष्णयः}


\twolineshloka
{एवमुत्तमवीर्यास्ते विहृत्य दिवसान्बहून्}
{पूजिताः कुरुभिर्जग्मुः पुनर्द्वारवतीं प्रति}


\twolineshloka
{रामं पुरुस्कृत्य ययुर्वृष्म्यन्धकमहारथाः}
{रत्नान्यादाय शुभ्राणि दत्तानि कुरुसत्तमैः}


\twolineshloka
{`रामः सुभद्रां संपूज्य परिष्वज्य स्वसां तदा}
{न्यासेति द्रौपदीमुक्त्वा परिधाय महाबलः}


\twolineshloka
{पितृष्वसायाश्चरणावभिवाद्य ययौ तदा}
{तस्मिन्काले पृथा प्रीता पूजयामास तं तथा}


\twolineshloka
{स वृष्णिवीरः पार्थैश्च पौरैश्च परमार्चितः}
{ययौ द्वारवतीं रामो वृष्णिभिः सह संयुतः}


\threelineshloka
{वासुदेवस्तु पार्थेन तत्रैव सह भारत}
{`चतुस्त्रिंशदहोरात्रं रममाणो महाबलः}
{'उवास नगरे रम्ये शक्रप्रस्थे महात्मना}


\twolineshloka
{व्यचरद्यमुनातीरे मृगयां स महायशाः}
{मृगान्विध्यन्वराहांश्च रेमे सार्धं किरीटिना}


\twolineshloka
{ततः सुभद्रा सौभद्रं केशवस्य प्रिया स्वसा}
{जयन्तमिव पौलोमी ख्यातिमन्तमजीजनत्}


\twolineshloka
{दीर्घबाहुं महोरस्कं वृषभाक्षमरिन्दमम्}
{सुभद्रा सुषुवे वीरमभिमन्युं नरर्षभम्}


\twolineshloka
{अभीश्च मन्युमांश्चैव ततस्तमरिमर्दनम्}
{अभिमन्युरिति प्राहुरार्जुनिं पुरुषर्षभम्}


\twolineshloka
{स सात्वत्यामतिरथः संबभूव घनञ्जयात्}
{मखे निर्मथनेनेव शमीगर्भाद्धुताशनः}


\twolineshloka
{यस्मिञ्जाते महातेजाः कुन्तीपुत्रो युधिष्ठिरः}
{अयुतं गा द्विजातिभ्यः प्रादान्निष्कांश्च भारत}


\twolineshloka
{दयितो वासुदेवस्य वाल्यात्प्रभृति चाभवत्}
{पितॄणां चैव सर्वेषां प्रजानामिव चन्द्रमाः}


\twolineshloka
{जन्मप्रभृति कृष्णश्च चक्रे तस्य क्रियाः शुभाः}
{स चापि ववृधे बालः शुक्लपक्षे यथा शशी}


\twolineshloka
{चतुष्पादं दशविधं धनुर्वेदमरिन्दमः}
{अर्जुनाद्वेद वेदज्ञः सकलं दिव्यमानुषम्}


\twolineshloka
{विज्ञानेष्वपि चास्त्राणां सौष्ठवे च महाबलः}
{क्रियास्वपि च सर्वासु विशेषानभ्यसिक्षयत्}


\twolineshloka
{आगमे च प्रयोगे च चक्रे तुल्यमिवात्मना}
{तुतोष पुत्रं सौभद्रं प्रेक्षणाणो धनञ्जयः}


\twolineshloka
{सर्वसंहननोपेतं सर्वलक्षणलक्षितम्}
{दुर्धर्षमृषभस्कन्धं व्यात्ताननमिवोरगम्}


\twolineshloka
{सिंहदर्पं महेष्वासं मत्तमातङ्गविक्रमम्}
{मेघदुन्दुभिनिर्घोषं पूर्णचन्द्रनिभाननम्}


\twolineshloka
{कृष्णस्य सदृशं शौर्ये वीर्ये रूपे तथाऽऽकृतौ}
{ददर्श पुत्रं बीभत्सुर्मघवानिव तं यथा}


\twolineshloka
{पाञ्चाल्यपि तु पञ्चभ्यः पतिभ्यः शुभलक्षणा}
{लेभे पञ्च सुतान्वीराञ्श्रेष्ठान्पञ्चाचलानिव}


\twolineshloka
{युधिष्ठिरात्प्रतिविन्ध्यं सुतसोमं वृकोदरात्}
{अर्जुनाच्छ्रुतकर्माणं शतानीकं च नाकुलिम्}


\twolineshloka
{सहदेवाच्छ्रुतसेनमेतान्पञ्च महारथान्}
{पाञ्चाली सुषुवे वीरानादित्यानदितिर्यथा}


\twolineshloka
{शास्त्रतः प्रतिविन्ध्यन्तमूचुर्विप्रायुधिष्ठिरम्}
{परप्रहरणज्ञाने प्रतिविन्ध्यो भत्वयम्}


\twolineshloka
{सुतेसोमसहस्रे तु सोमार्कसमतेजसम्}
{सुतसोमं महेष्वासं सुषुवे भीमसेनतः}


\twolineshloka
{श्रुतं कर्म महत्कृत्वा निवृत्तेन किरीटिना}
{जातः पुत्रस्तथेत्येवं श्रुतकर्मा ततोऽभवत्}


\twolineshloka
{शतानीकस्य राजर्षेः कौरव्यस्य महात्मनः}
{चक्रे पुत्रं सनामानं नकुलः कीर्तिवर्धनम्}


\twolineshloka
{ततस्त्वजीजनत्कृष्णा नक्षत्रे वह्निदैवते}
{सहदेवात्सुतं तस्माच्छ्रुतसेनेति तं विदुः}


\twolineshloka
{एकवर्षान्तरास्त्वेते द्रौपदेया यशस्विनः}
{अन्वजायन्त राजेन्द्र परस्परहितैषिणः}


\twolineshloka
{जातकर्माण्यानुपूर्व्याच्चूडोपनयनानि च}
{चकार विधिवद्धौम्यस्तेषां भरतसत्तम}


\twolineshloka
{कृत्वा च वेदाध्ययनं ततः सुचरितव्रताः}
{जगृहुः सर्वमिष्वस्त्रमर्जुनाद्दिव्यमानुषम्}


\twolineshloka
{दिव्यगर्भोपमैः पुत्रैर्व्यूढोरस्कैर्महारथैः}
{अन्विता राजशार्दूल पाण्डवा मुदमाप्नुवन्}


\chapter{अध्यायः २४८}
\twolineshloka
{वैशंपायन उवाच}
{}


\twolineshloka
{इन्द्रप्रस्थे वसन्तस्ते जघ्रुरन्यान्नराधिपान्}
{शासनाद्धृतराष्ट्रस्य राज्ञः शान्तनवस्य च}


\twolineshloka
{आश्रित्य धर्मराजानं सर्वलोकोऽवसत्सुखम्}
{पुण्यलक्षणकर्माणं स्वदेहमिव देहिनः}


\twolineshloka
{स समं धर्मकामार्थान्सिषेवे भरतर्षभ}
{त्रीनिवात्मसमान्बन्धून्नीतिमानिव मानयन्}


\twolineshloka
{तेषां समविभक्तानां क्षितौ देहवतामिव}
{बभौ धर्मार्थकामानां चतुर्थ इव पार्थिवः}


\twolineshloka
{अध्येतारं परं वेदान्प्रयोक्तारं महाध्वरे}
{रक्षितारं शुभाँल्लोकाँल्लोभिरे तं जनाधिपम्}


\twolineshloka
{अधिष्ठानवती लक्ष्मीः परायणवती मतिः}
{वर्धमानोऽखिलो धर्मस्तेनासीत्पृथिवीक्षिताम्}


\twolineshloka
{भ्रातृभिः सहितौ राजा चतुर्भिरधिकं बभौ}
{प्रयुज्यमानैर्विततो वेदैरिव महाध्वरः}


\twolineshloka
{तं तु धौम्यादयो विप्राः परिवार्योपतस्थिरे}
{बृहस्पतिसमा मुख्याः प्रजापतिमिवामराः}


\twolineshloka
{धर्मराजे ह्यतिप्रीत्या पूर्णचन्द्र इवामले}
{प्रजानां रेमिरे तुल्यं नेत्राणि हृदयानि च}


\twolineshloka
{न तु केवलदैवेन प्रजा भावेन रेमिरे}
{यद्बभूव मनःकान्तं कर्मणा स चकार तत्}


\twolineshloka
{न ह्ययुक्तं न चासत्यं नासह्यं न च वाऽप्रियम्}
{भाषितं चारुभाषस्य जज्ञे पार्थस्य धीमतः}


\twolineshloka
{स हि सर्वस्य लोकस्य हितमात्मन एव च}
{चिकीर्षन्सुमहातेजा रेमे भरतसत्तम}


\twolineshloka
{तथा तु मुदिताः सर्वे पाण्डवा विगतज्वराः}
{अवसन्पृथिवीपालाँस्तापयन्तः स्वतेजसा}


\twolineshloka
{ततः कतिपयाहस्य बीभत्सुः कृष्णमब्रवीत्}
{उष्णानि कृष्ण वर्तन्ते गच्छावो यमुनां प्रति}


\threelineshloka
{सुहृज्जनवृतौ तत्र विहृत्य मधुसूदन}
{सायाह्ने पुनरेष्यावो रोचतां ते जनार्दन ॥वासुदेव उवाच}
{}


\threelineshloka
{कुन्तीमातर्ममाप्येतद्रोचते यद्वयं जले}
{सुहृज्जनवृताः पार्थ विहरेम यथासुखम् ॥वैशंपायन उवाच}
{}


\twolineshloka
{आमन्त्र्य तौ धर्मराजमनुज्ञाप्य च भारत}
{जग्मतुः पार्थगोविन्दौ सुहृज्जनवृतौ ततः}


\twolineshloka
{`विहरन्खाण्डवप्रस्थे काननेषु च माधवः}
{पुष्पितोपवनां दिव्यां ददर्श यमुनां नदीम्}


\twolineshloka
{तस्यास्तीरे वनं दिव्यं सर्वर्तुसुमनोहरम्}
{आलयं सर्वभूतानां खाण्डवं खड्गचर्मभृत्}


\twolineshloka
{ददर्श कृत्स्नं तं देशं सहितः सव्यसाचिना}
{ऋक्षगोमायुशार्दूलवृककृष्णमृगान्वितम्}


\twolineshloka
{शाखामृगगणैर्जुष्टं गजद्वीपिनिषेवितम्}
{शकबर्हिणदात्यूहहंससारसनादितम्}


\twolineshloka
{नानामृगसहस्रैश्च पक्षिभिश्च समावृतम्}
{मानार्हं तच्च सर्वेषां देवदानवरक्षसाम्}


\twolineshloka
{आलयं पन्नगेन्द्रस्य तक्षकस्य महात्मनः}
{वेणुशाल्मलिबिल्वातिमुक्तजंब्वाम्रचम्पकैः}


\twolineshloka
{अङ्कोलपनसाश्वत्थतालजम्बीरवञ्जुलैः}
{एकपद्मकतालैश्च शतशश्चैव रौहिणैः}


\twolineshloka
{नानावृक्षैः समायुक्तं नानागुल्मसमावृतम्}
{वेत्रकीचकसंयुक्तमाशीविषनिषेवितम्}


\twolineshloka
{विगतार्कं महाभोगविततद्रुमसङ्कटम्}
{व्यालदंष्ट्रिगणाकीर्णं वर्जितं सर्वमानुषैः}


\twolineshloka
{रक्षसां भुजगेन्द्राणां पक्षिणां च महालयम्}
{खाण्डवं सुमहाप्राज्ञः सर्वलोकविभागवित्}


\twolineshloka
{दृष्टवान्सर्वलोकेश अर्जुनेन समन्वितः}
{पीताम्बरधरो देवस्तद्वनं बहुधा चरन्}


\twolineshloka
{सद्रुमस्य सयक्षस्य सभूतगणपक्षिणः}
{खाण्डवस्य विनाशं तं ददर्श मधुसूदनः ॥'}


\twolineshloka
{विहारदेशं संप्राप्य नानाद्रुममनुत्तमम्}
{गृहैरुच्चावचैर्युक्तं पुरन्दरपुरोपमम्}


\twolineshloka
{भक्ष्यैर्भोज्यैश्च पेयैश्च रसवद्बिर्महाधनैः}
{माल्यैश्च विविधैर्गन्धैस्तथा वार्ष्मेयपाण्डवौ}


\twolineshloka
{तदा विविशतुः पूर्णं रत्नैरुच्चावचैः शुभैः}
{यथोपजोषं सर्वश्च जनश्चिक्रीड भारत}


\twolineshloka
{स्त्रियश्च विपुलश्रोष्ण्यश्चारुपीनपयोधराः}
{मदस्खलितगामिन्यश्चिक्रीडुर्वामलोचनाः}


\twolineshloka
{वने काश्चिज्जले काश्चित्काश्चिद्वेश्मसु चाङ्गनाः}
{यथादेशं यथाप्रीति चिक्रीडुः पार्थकृष्णयोः}


\threelineshloka
{वासुदेवप्रिया नित्यं सत्यभामा च भामिनी}
{द्रौपदी च सुभद्रा च वासांस्याभरणानि च}
{प्रायच्छन्त महाराज स्त्रीणां ताः स्म मदोत्कटाः}


\twolineshloka
{काश्चित्प्रहृष्टा ननृतुश्चुक्रुशुश्च तथा पराः}
{जहसुश्च परा नार्यः पपुश्चान्या वरासवम्}


\twolineshloka
{रुरुधुश्चापरास्तत्र प्रजघ्नुश्च परस्परम्}
{मन्त्रयामासुरन्याश्च रहस्यानि परस्परम्}


\twolineshloka
{वेणुवीणामृदङ्गानां मनोज्ञानां च सर्वशः}
{शब्देन पूर्यते हर्म्यं तद्वनं सुमहर्द्धिमत्}


\twolineshloka
{तस्मिंस्तदा वर्तमानो कुरुदाशार्हनन्दनौ}
{समीपं जग्मतुः कंचिदुद्देशं सुमनोहरम्}


\twolineshloka
{तत्र गत्वा महात्मानौ कृष्णौ परपुरञ्जयौ}
{महार्हासनयो राजंस्ततस्तौ सन्निषीदतुः}


\twolineshloka
{तत्र पूर्वव्यतीतानि विक्रान्तानीतराणि च}
{बहूनि कथयित्वा तौ रेमाते पार्थमाधवौ}


\twolineshloka
{तत्रोपविष्टौ मुदितौ नाकपृष्ठेऽश्विनाविव}
{अभ्यागच्छत्तदा विप्रो वासुदेवधनञ्जयौ}


\twolineshloka
{बृहच्छालप्रतीकाशः प्रतप्तकनकप्रभः}
{हरिपिङ्गोज्ज्वलश्मश्रुः प्रमाणायामतः समः}


\twolineshloka
{तरुणादित्यसङ्काशश्चीरवासा जटाधरः}
{पद्मपत्राननः पिङ्गस्तेजसा प्रज्वलन्निव}


\threelineshloka
{जगाम तौ कृष्णपार्थौ दिधक्षुः खाण्डवं वनम्}
{उपसृष्टं तु तं कृष्णो भ्राजमानं द्विजोत्तमम्}
{अर्जनो वासुदेवश्च तूर्णमुत्पत्य तस्थतुः}


\chapter{अध्यायः २४९}
\twolineshloka
{वैशंपायन उवाच}
{}


\twolineshloka
{सोऽब्रवीदर्जुनं चैव वासुदेवं च सात्वतम्}
{लोकप्रवीरौ तिष्ठन्तौ काण्डवस्य समीपतः}


\twolineshloka
{ब्राह्मणो बहुभोक्ताऽस्मि बुञ्जेऽपरिमितं सदा}
{भिक्षे वार्ष्णेयपार्थौ वामेकां तृप्तिं प्रयच्छतम्}


\twolineshloka
{एवमुक्तो तमब्रूतां ततस्तौ कृष्णपाण्डवौ}
{केनान्नेन भवांस्तृप्येत्तस्यान्नस्य यतावहे}


\threelineshloka
{एवमुक्तः स भगवानब्रवीत्तावुभौ ततः}
{भाषमाणौ तदा वीरौ किमन्नं क्रियतामिति ॥ब्राह्मण उवाच}
{}


\twolineshloka
{नाहमन्नं बुभुक्षे वै पावकं मां निबोधतम्}
{यदन्नमनुरूपं मे तद्युवां संप्रयच्छतम्}


\twolineshloka
{इदमिन्द्रः सदा दावं खाण्डवं परिरक्षति}
{न च शक्नोम्यहं दग्धुं रक्ष्यमाणं महात्मना}


\twolineshloka
{वसत्यत्र सखा तस्य तक्षकः पन्नगः सदा}
{सगणस्तत्कृते दावं परिरक्षति वज्रभृत्}


\twolineshloka
{तत्र भूतान्यनेकानि रक्ष्यन्तेऽस्य प्रसङ्गतः}
{तं दिधक्षुर्न शक्नोमि दग्धुं शक्रस्य तेजसा}


\twolineshloka
{स मां प्रज्वलितं दृष्ट्वा मेघाम्भोभिः प्रवर्षति}
{ततो दग्धुं न शक्नोमि दिधक्षुर्दावमीप्सितम्}


\twolineshloka
{स युवाभ्यां सहायाभ्यामस्त्रविद्भ्यां समागतः}
{दहेयं खाण्डवं दावमेतदन्नं वृतं मया}


\threelineshloka
{युवां ह्युदकधारास्ता भूतानि च समन्ततः}
{उत्तमास्त्रविदौ सम्यक्सर्वतो वारयिष्यथः ॥जनमेजय उवाच}
{}


\twolineshloka
{किमर्थं भगवानग्निः खाण्डवं दग्धुमिच्छति}
{रक्ष्यमाणं महेन्द्रेण नानासत्वसमायुतम्}


\twolineshloka
{न ह्येतत्कारणं ब्रह्मन्नल्पं संप्रतिभाति मे}
{यद्ददाह सुसंक्रुद्धः खाण्डवं हव्यवाहनः}


\threelineshloka
{एतद्विस्तरशो ब्रह्मन्श्रोतुमिच्छामि तत्त्वतः}
{खाण्डवस्य पुरा दाहो यथा समभवन्मुने ॥वैशंपायन उवाच}
{}


\twolineshloka
{शृणु मे ब्रुवतो राजन्सर्वमेतद्यथातथम्}
{यन्निमित्तं ददाहाग्निः खाण्डवं पृथिवीपते}


\twolineshloka
{हन्त ते कथयिष्यामि पौराणीमृषिसंस्तुताम्}
{कथामिमां नरश्रेष्ठ खाण्डवस्य विनाशिनीम्}


\twolineshloka
{पौराणः श्रूयते राजन्राजा हरिहयोपमः}
{श्वेतकिर्नाम विख्यातो बलविक्रमसंयुतः}


\twolineshloka
{यज्वा दानपतिर्धीमान्यथा नान्योऽस्ति कश्चन}
{`जग्राह दीक्षां स नृपः तदा द्वादशवार्षिकीम्}


\twolineshloka
{तस्य सत्रे सदा तस्मिन्समागच्छन्महर्षयः}
{वेदवेदाङ्गविद्वांसो ब्राह्मणाश्च सहस्रशः ॥'}


% Check verse!
ईजे च स महायज्ञैः क्रतुभिश्चाप्तदक्षिणैः
\twolineshloka
{तस्य नान्याऽभवद्बुद्धिर्दिवसे दिवसे नृप}
{सत्रे क्रियासमारम्भे दानेषु विविधेषु च}


\twolineshloka
{ऋत्विग्भिः सहितो धीमानेवमीजे स भूमिपः}
{ततस्तु ऋत्विजश्चास्य धूमव्याकुललोचनाः}


\twolineshloka
{कालेन महता खिन्नास्तत्यजुस्ते नराधिपम्}
{ततः प्रसादयामास ऋत्विजस्तान्महीपतिः}


\twolineshloka
{चक्षुर्विकलतां प्राप्ता न प्रपेदुश्च ते क्रतुम्}
{ततस्तेषामनुमते तद्विप्रैस्तु नराधिपः}


\twolineshloka
{सत्रं समापयामास ऋत्विग्भिरपरैः सह}
{तस्यैवंवर्तमानस्य कदाचित्कालपर्यये}


\twolineshloka
{सत्रमहार्तुकामस्य संवत्सरशतं किल}
{ऋत्विजो नाभ्यपद्यन्त समाहर्तुं महात्मनः}


\twolineshloka
{स च राजाऽकरोद्यत्नं महान्तं ससुहृज्जनः}
{प्रणिपातेन सान्त्वेन दानेन च महायशाः}


\twolineshloka
{ऋत्विजोऽनुनयामास भूयो भूयस्त्वतन्द्रितः}
{ते चास्य तमभिप्रायं न चक्रुरमितौजसः}


\twolineshloka
{स चाश्रमस्थान्राजर्षिस्तानुवाच रुषान्वितः}
{यद्यहं पतितो विप्राः शुश्रूषायां न च स्थितः}


\twolineshloka
{आशु त्याज्योऽस्मि युष्माभिर्ब्राह्मणैश्च जुगुप्सितः}
{तन्नार्हथ क्रतुश्रद्धां व्याघातयितुमद्य ताम्}


\twolineshloka
{अस्थाने वा परित्यागं कर्तुं मे द्विजसत्तमाः}
{प्रपन्न एव वो विप्राः प्रसादं कर्तुमर्हथ}


\twolineshloka
{सान्त्वदानादिभिर्वाक्यैस्तत्त्वतः कार्यवत्तया}
{प्रसादयित्वा वक्ष्यामि यन्नः कार्यं द्विजोत्तमाः}


\twolineshloka
{अथवाऽहं परित्यक्तो भवद्भिर्द्वेषकारणात्}
{ऋत्विजोऽन्यान्गमिष्यामि याजनार्थं द्विजोत्तमाः}


\twolineshloka
{एतावदुक्त्वा वचनं विरराम स पार्थिवः}
{यदा न शेकू राजानं याजनार्थं परन्तप}


\twolineshloka
{ततस्ते याजकाः क्रुद्धास्तमूचुर्नृपसत्तमम्}
{तव कर्माम्यजस्रं वै वर्तन्ते पार्थिवोत्तम}


\twolineshloka
{ततो वयं परिश्रान्ताः सततं कर्मवाहिनः}
{श्रमादस्मात्परिश्रान्तान्स त्वं नस्त्यक्तुमर्हसि}


\twolineshloka
{बुद्धिमोहं समास्थाय त्वरासंभावितोऽनघ}
{गच्छ रुद्रसकाशं त्वं सहि त्वां याजयिष्यति}


\twolineshloka
{साधिक्षेपं वचः श्रुत्वा संक्रुद्धः श्वेतकिर्नृपः}
{कैलासं पर्वतं गत्वा तप उग्रं समास्थितः}


\twolineshloka
{आराधयन्महादेवं नियतः संशितव्रतः}
{उपवासपरो राजन्दीर्घकालमतिष्ठत}


\twolineshloka
{कदाचिद्द्वादशे काले कदाचिदपि षोडशे}
{आहारमकरोद्राजा मूलानि च फलानि च}


\twolineshloka
{ऊर्ध्वबाहुस्त्वनिमिषस्तिष्ठन्स्थाणुरिवाचलः}
{षण्मासानभवद्राजा श्वेतकिः सुसमाहितः}


\twolineshloka
{तं तथा नृपशार्दूलं तप्यमानं महत्तपः}
{शङ्करः परमप्रीत्या दर्शयामास भारत}


\twolineshloka
{उवाच चैनं भगवान्स्निग्धगम्भीरया गिरा}
{प्रीतोऽस्मि नरशार्दूल तपसा ते परन्तप}


\twolineshloka
{वरं वृणीष्व भद्रं ते यं त्वमिच्छसि पार्थिव}
{एतच्छ्रुत्वा तु वचनं रुद्रस्यामिततेजसः}


\twolineshloka
{प्रणिपत्य महात्मानं राजर्षिः प्रत्यभाषत}
{यदि मे भगवान्प्रीतः सर्वलोकनमस्कृतः}


\twolineshloka
{स्वयं मां देवदेवेश याजयस्व सुरेश्वर}
{एतच्छ्रुत्वा तु वचनं राज्ञा तेन प्रभाषितम्}


\twolineshloka
{उवाच भगवान्प्रीतः स्मितपूर्वमिदं वचः}
{नास्माकमेतद्विषये वर्तते याजनं प्रति}


\twolineshloka
{त्वया च सुमहत्तप्तं तपो राजन्वरार्थिना}
{याजयिष्यामि राजंस्त्वां समयेन परन्तप}


\twolineshloka
{समा द्वादश राजेन्द्र ब्रह्मचारी समाहितः}
{सततं त्वाज्यधाराभिर्यदि तर्पयसेऽनलम्}


\twolineshloka
{कामं प्रार्थयसे यं त्वं मत्तः प्राप्स्यसि तं नृप}
{एवमुक्तश्च रुद्रेण श्वेतकिर्मनुजाधिपः}


\twolineshloka
{तथा चकार तत्सर्वं यथोक्तं शूलपाणिना}
{पूर्णे तु द्वादशे वर्षे पुनरायान्महेश्वरः}


\twolineshloka
{दृष्टैव च स राजानं शङ्करो लोकभावनः}
{उवाच परमप्रीतः श्वेतकिं नृपसत्तमम्}


% Check verse!
तोषितोऽहं नृपश्रेष्ठ त्वयेहाद्येन कर्मणा ॥याजनं ब्राह्मणानां तु विधिदृष्टं परन्तप
\twolineshloka
{अतोऽहं त्वां स्वयं नाद्य याजयामि परन्तप}
{ममांशस्तु क्षितितले महाभागो द्विजोत्तमः}


\twolineshloka
{दुर्वासा इति विख्यातः स हि त्वां याजयिष्यति}
{मन्नियोगान्महातेजाः संभाराः संभ्रियन्तु ते}


\twolineshloka
{एतच्छ्रुत्वा तु वचनं रुद्रेण समुदाहृतम्}
{स्वपुरं पुनरागम्य संभारान्पुनरार्जयत्}


\threelineshloka
{ततः संभृतसंभारो भूयो रुद्रमुपागमत्}
{`उवाच च महिपालः प्राञ्जलिः प्रणतः स्थितः}
{'संभृता मम संभाराः सर्वोपकरणानि च}


\twolineshloka
{त्वत्प्रसादानमहादेव श्वो मे दीक्षा भवेदिति}
{एतच्छ्रुत्वा तु वचनं तस्य राज्ञो महात्मनः}


\twolineshloka
{दुर्वाससं समाहूय रुद्रो वचनमब्रवीत्}
{एष राजा महाभागः श्वेतकिर्द्विजसत्तम}


\twolineshloka
{एनं याजय विप्रेन्द्र मन्नियोगेन भूमिपम्}
{बाढमित्येव वचनं रुद्रं त्वृषिरुवाच ह}


\twolineshloka
{ततः सत्रं समभवत्तस्य राज्ञो महात्मनः}
{यथाविधि यथाकालं यथोक्तं बहुदक्षिणम्}


\twolineshloka
{तस्मिन्परिसमाप्ते तु राज्ञः सत्रे महात्मनः}
{दुर्वाससाऽभ्यनुज्ञाता विप्रतस्थुः स्म याजकाः}


\twolineshloka
{ये तत्र दीक्षिताः सर्वे सदस्याश्च महौजसः}
{सोऽपि राजन्महाभागः स्वपुरं प्राविशत्तदा}


\twolineshloka
{पूज्यमानो महाभागैर्ब्राह्मणैर्वेदपारगैः}
{बन्दिभिः स्तूयमानश्च नागरैश्चाभिनन्दितः}


\twolineshloka
{एवंवृत्तः स राजर्षिः श्वेतकिर्नृपसत्तमः}
{कालेन महता चापि ययौ स्वर्गमभिष्टुतः}


\twolineshloka
{ऋत्विग्भिः सहितः सर्वैः सदस्यैश्च समन्वितः}
{तस्य सत्रे पपौ वह्निर्हविद्वार्दशवत्सरान्}


\twolineshloka
{सततं चाज्यधाराभिरैकात्म्ये तत्र कर्मणि}
{हविषा च ततो वह्निः परां तृप्तिमगच्छत}


\twolineshloka
{न चैच्छत्पुनरादातुं हविरन्यस्य कस्यचित्}
{पाण्डुवर्णो विवर्णश्च न यतावत्प्रकाशते}


\twolineshloka
{ततो भघवतो वह्नेर्विकारः समजायत}
{तेजसा विप्रहीणश्च ग्लानिश्चैनं समाविशत्}


\twolineshloka
{स लक्षयित्वा चात्मानं तेजोहीनं हुताशनः}
{जगाम सदनं पुण्यं ब्रह्मणो लोकपूजितम्}


\twolineshloka
{तत्र ब्रह्माणमासीनमिदं वचनमब्रवीत्}
{भगवन्परमा प्रीतिः कृता श्वेतकिना मम}


\twolineshloka
{अरुचिश्चाभवत्तीव्रा तां न शक्नोम्यपोहितुम्}
{तेजसा विप्रहीणोऽस्मि बलेन च जगत्पते}


\twolineshloka
{इच्छेयं त्वत्प्रसादेन स्वात्मनः प्रकृतिं स्थिराम्}
{एतच्छ्रुत्वा हुतवहाद्भगवान्सर्वलोककृत्}


\twolineshloka
{हव्यवाहमिदं वाक्यमुवाच प्रहसन्निव}
{त्वया द्वादश वर्षाणि वसोर्धाराहुतं हविः}


\twolineshloka
{उपयुक्तं महाभाग तेन त्वां ग्लानिराविशत्}
{तेजसा विप्रहीणत्वात्सहसा हव्यवाहन}


\twolineshloka
{मागमस्त्वं व्यथां वह्ने प्रकृतिस्थो भविष्यसि}
{अरुचिं नाशयिष्येऽहं समयं प्रतिपद्य ते}


\twolineshloka
{पुरा देवनियोगेन यत्त्वया भस्मसात्कृतम्}
{आलयं देवशत्रूणां सुघोरं खाण्डवं वनम्}


\twolineshloka
{तत्र सर्वाणि सत्वानि निवसन्ति विभावसो}
{तेषां त्वं मेदसा तृप्तः प्रकृतिस्थो भविष्यसि}


\twolineshloka
{गच्छ शीघ्रं प्रदग्धुं त्वं ततो मोक्ष्यसि किल्विषात्}
{एतच्छुत्वा तु वचनं परमेष्ठिमुखाच्च्युतम्}


\threelineshloka
{उत्तमं जवमास्थाय प्रदुद्राव हुताशनः}
{आगम्य खाण्डवं दावमुत्तमं वीर्यमास्थितः}
{सहसा प्राज्वलच्चाग्निः क्रुद्धो वायुसमीरितः}


\twolineshloka
{प्रदीप्तं खाण्डवं दृष्ट्वा ये स्युस्तत्र निवासिनः}
{परमं यत्नेमातिष्ठन्पावकस्य प्रशान्तये}


\twolineshloka
{करैस्तु करिणः शीघ्रं जलमादाय सत्वराः}
{सिषिचुः पावकं क्रुद्धाः शतशोऽथ सहस्रशः}


\twolineshloka
{बहुशीर्षास्ततो नागाः शिरोभिर्जलसन्ततिम्}
{मुमुचुः पावकाभ्याशे सत्वराः क्रोधमूर्च्छिताः}


\twolineshloka
{तथैवान्यानि सत्वानि नानाप्रहरणोद्यमैः}
{विलयं पावकं शीघ्रमनयन्भरतर्षभ}


\twolineshloka
{अनेन तु प्रकारेण भूयोभूयश्च प्रज्वलन्}
{सप्तकृत्वः प्रशमितः खाण्डवे हव्यवाहनः}


\chapter{अध्यायः २५०}
\twolineshloka
{वैशंपायन उवाच}
{}


\twolineshloka
{स तु नैराश्यमापन्नः सदा ग्लानिसमन्वितः}
{पितामहमुपागच्छत्संक्रुद्धो हव्यवाहनः}


\twolineshloka
{तच्च सर्वं यथान्यायं ब्रह्मणे संन्यवेदयत्}
{उवाच चैवनं भगवान्मुहूर्तं स विचिन्त्य तु}


\twolineshloka
{उपायः परिदृष्टो मे यथा त्वं धक्ष्यसेऽनघ}
{कालं च कंचित्क्षमतां ततस्तद्धक्ष्यते भवान्}


\twolineshloka
{भविष्यतः सहायौ ते नरनारायणौ तदा}
{ताभ्यां त्वं सहितो दावं धक्ष्यसे हव्यवाहन}


\twolineshloka
{एवमस्त्विति तं वह्निर्ब्रह्माणं प्रत्यभाषत}
{संभूतौ तौ विदित्वा तु नरनारायणावृषी}


\twolineshloka
{कालस्य महतो राजंस्तस्य वाक्यं स्वयंभुवः}
{अनुस्मृत्य जगामाथ पुनरेव पितामहम्}


\twolineshloka
{अब्रवीच्च तदा ब्रह्मा यथा त्वं धक्ष्यसेऽनल}
{खाण्डवं दावमद्यैव मिषतोऽस्य शचीपतेः}


\twolineshloka
{नरनारायणौ यौ तौ पूर्वदेवौ विभावसो}
{संप्राप्तौ मानुषे लोके कार्यार्थं हि दिवौकसाम्}


\twolineshloka
{अर्जुनं वासुदेवं च यौ तौ लोकोऽभिमन्यते}
{तावेतौ सहितावेहि खाण्डवस्य समीपतः}


\twolineshloka
{तौ त्वं याचस्व साहाय्ये दाहार्थं खाण्डवस्य च}
{ततो धक्ष्यसि तं दावं रक्षितं त्रिदशैरपि}


\twolineshloka
{तौ तु सत्वानि सर्वाणि यत्नतो वारयिष्यतः}
{देवराजं च सहितौ तत्र मे नास्ति संशयः}


\twolineshloka
{एतच्छ्रुत्वा तु वचनं त्वरितो हव्यवाहनः}
{कृष्णपार्थावुपागम्य यमर्थं त्वभ्यभाषत}


\twolineshloka
{तं ते कथितवानस्मि पूर्वमेव नृपोत्तम}
{तच्छ्रुत्वा वचनं त्वग्नेर्बीभत्सुर्जातवेदसम्}


\threelineshloka
{अब्रवीन्नृपशार्दूल तत्कालसदृशं वचः}
{दिधक्षुं खाण्डवं दावमकामस्य शतक्रतोः ॥अर्जुन उवाच}
{}


\twolineshloka
{उत्तमास्त्राणि मे सन्ति दिव्यानि च बहूनि च}
{यैरहं शक्नुयां योद्धुमपि वज्रधरान्बहून्}


\twolineshloka
{धनुर्मे नास्ति भगवन्बाहुवीर्येण संमितम्}
{कुर्वतः समरे यत्नं वेगं यद्विषहेन्मम}


\twolineshloka
{शरैश्च मेऽर्थो बहुभिरक्षयैः क्षिप्रमस्यतः}
{न हि वोढुं रथः शक्तः शरान्मम यथेप्सितान्}


\twolineshloka
{अश्वांश्च दिव्यानिच्छेयं पाण्डुरान्वातरंहसः}
{रथं च मेघनिर्घोषं सूर्यप्रतिमतेजसम्}


\twolineshloka
{तथा कृष्णस्य वीर्येण नायुधं विद्यते समम्}
{येन नागान्पिशाचांश्च निहन्यान्माधवो रणे}


\twolineshloka
{उपायं कर्मसिद्धौ च भघवन्वक्तुमर्हसि}
{निवारयेयं येनेन्द्रं वर्षमाणं महावने}


\twolineshloka
{पौरुषेण तु यत्कार्यं तत्कर्ताऽहं स्म पावक}
{करणानि समर्थानि भगवन्दातुमर्हसि}


\chapter{अध्यायः २५१}
\twolineshloka
{वैशंपायन उवाच}
{}


\twolineshloka
{एवमुक्तः स भगवान्धूमकेतुर्हुताशनः}
{चिन्तयामास वरुणं लोकपालं दिदृक्षया}


\twolineshloka
{आदित्यमुदके देवं निवसन्तं जलेश्वरम्}
{स च तच्चिन्तितं ज्ञात्वा दर्शयामास पावकम्}


\twolineshloka
{तमब्रवीद्भमकेतुः प्रतिगृह्य जलेश्वरम्}
{चतुर्थं लोकपालानां देवदेवं सनातनम्}


\twolineshloka
{सोमेन राज्ञा यद्दत्तं धनुश्चैवेषुधी च ते}
{तत्प्रयच्छोभयं शीघ्रं रथं च कपिलक्षणम्}


\twolineshloka
{कार्यं च सुमहत्पार्थो गाण्डीवेन करिष्यति}
{चक्रेण वासुदेवश्च तन्ममाद्य प्रदीयताम्}


\twolineshloka
{ददानीत्येव वरुणः पावकं प्रत्यभाषत}
{तदद्भुतं महावीर्यं यशःकीर्तिविवर्धनम्}


\twolineshloka
{सर्वशस्त्रैरनाधृष्यं सर्वशस्त्रप्रमाथि च}
{सर्वायुधमहामात्रं परसैन्यप्रधर्षणम्}


\twolineshloka
{एकं शतसहस्रेण संमितं राष्ट्रवर्धनम्}
{चित्रमुच्चावचैर्वर्णैः शोभितं श्लक्ष्णमव्रणम्}


\twolineshloka
{देवदानवगन्धर्वैः पूजितं शाश्वतीः समाः}
{प्रादाच्चैव धनूरत्नमक्षय्यौ च महेषुधी}


\twolineshloka
{रथं च दिव्याश्वयुजं कपिप्रवरकेतनम्}
{उपेतं राजतैरश्वैर्गान्धर्वैर्हेममालिभिः}


\twolineshloka
{पाण्डुराभ्रप्रतीकाशैर्मनोवायुसमैर्जवे}
{सर्वोपकरणैर्युक्तमजय्यं देवदानवैः}


\twolineshloka
{भावुमन्तं महाघोषं सर्वरत्नमनोरमम्}
{ससर्ज यं सुतपसा भौमनो भुवनप्रभुः}


\twolineshloka
{प्रजापतिरनिर्देश्यं यस्य रूपं रवेरिव}
{यं स्म सोमः समारुह्य दानवानजयत्प्रभुः}


\twolineshloka
{नवमेघप्रतीकाशं ज्ललन्तमिव च श्रिया}
{आश्रितौ तं रथश्रेष्ठं शक्रायुधसमावुभौ}


\twolineshloka
{तापनीया सुरुचिरा ध्वजयष्टिरनुत्तमा}
{तस्यां तु वानरो दिव्यः सिंहशार्दूलकेतनः}


\threelineshloka
{`हनूमान्नाम तेजस्वी कामरूपी महाबलः}
{'दिधक्षन्निव तत्र स्म संस्थितो मूर्ध्न्यशोभत}
{ध्वजे भूतानि तत्रासन्विविधानि महान्ति च}


\twolineshloka
{नादेन रिपुसैन्यानां येषां संज्ञा प्रणश्यति}
{स तं नानापताकाभिः शोभितं रथसत्तमम्}


\twolineshloka
{प्रदक्षिणमुपावृत्य दैवतेभ्यः प्रणम्य च}
{सन्नद्धः कवची खड्गी बद्धगोधाङ्गुलित्रकः}


\twolineshloka
{आरुरोह तदा पार्थो विमानं सुकृती यथा}
{तच्च दिव्यं धनुः श्रेष्ठं ब्रह्मणा निर्मितं पुरा}


\twolineshloka
{गाण्डीवमुपसंगृह्य बभूव मुदितोऽर्जुनः}
{हुताशनं पुरस्कृत्य ततस्तदपि वीर्यवान्}


\twolineshloka
{जग्राह बलमास्थाय ज्यया च युयुजे धनुः}
{मौर्व्यां तु योज्यमानायां बलिना पाण्डवेन ह}


\twolineshloka
{येऽशृष्वन्कूजितं यत्र तेषां वै व्यथितं मनः}
{लब्ध्वा रथं धनुश्चैव तथाऽक्षय्ये महेषुधी}


\twolineshloka
{बभूव कल्यः कौन्तेयः प्रहृष्टः साह्यकर्मणि}
{वज्रनाभं ततश्चक्रं ददौ कृष्णाय पावकः}


\twolineshloka
{आग्नेयमस्त्रं दयितं स च कल्योऽभवत्तदा}
{अब्रवीत्पावकश्चैवमेतेन मधुसूदन}


\twolineshloka
{अमानुषानपि रणे जेष्यसि त्वमसंशयम्}
{अनेन तु मनुष्याणां देवानामपि चाहवे}


\twolineshloka
{रक्षःपिशाचदैत्यानां नागानां चाधिकस्तथा}
{भविष्यसि न सन्देहः प्रवरोऽपि निबर्हणे}


\threelineshloka
{क्षिप्तं क्षिप्तं रणे चैतत्त्वया माधव शत्रुषु}
{हत्वाऽप्रतिहतं सङ्ख्ये पाणिमेष्यति ते पुनः ॥वैशंपायन उवाच}
{}


\twolineshloka
{वरुणश्च ददौ तस्मै गदामशनिनिःस्वनाम्}
{दैत्यान्तकरणीं घोरां नाम्ना कौमोदकीं प्रभुः}


\twolineshloka
{ततः पावकमब्रूतां प्रहृष्टावर्जुनाच्युतौ}
{कृतास्त्रौ शस्त्रसंपन्नौ रथिनौ ध्वजिनावपि}


\threelineshloka
{कल्यौ स्वो भगवन्योद्धुमपि सर्वैः सुरासुरैः}
{किं पुनर्वज्रिणैकेन पन्नगार्थे युयुत्सता ॥अर्जुन उवाच}
{}


\threelineshloka
{चक्रपाणिर्हृषीकेशो विचरन्युधि वीर्यवान्}
{चक्रेण भस्मसात्सर्वं विसृष्टेन तु वीर्यवान्}
{त्रिषु लोकेषु तन्नास्ति यन्न कुर्याज्जनार्दनः}


\twolineshloka
{गाण्डीवं धनुरादाय तथाऽक्षय्ये महेषुधी}
{अहमप्युत्सहे लोकान्विजेतुं युधि पावक}


\twolineshloka
{सर्वतः परिवार्यैवं दावमेतं महाप्रभो}
{कामं संप्रज्वलाद्यैव कल्यौ स्वः साह्यकर्मणि}


\threelineshloka
{`यदि खाण्डवमेष्यति प्रमादा-त्सगणो वा परिरक्षितुं महेन्द्रः}
{शरताडितगात्रकुण्डलानांकदनं द्रक्ष्यति देववाहिनीनाम् ॥'वैशंपायन उवाच}
{}


\twolineshloka
{एवमुक्तः स भगवान्दाशार्हेणार्जुनेन च}
{तैजसं रपमास्थाय दावं दग्धुं प्रचक्रमे}


\twolineshloka
{सर्वतः परिवार्याथ सप्तार्चिर्ज्वलनस्तथा}
{ददाह खाण्डवं दावं युगान्तमिव दर्शयन्}


\twolineshloka
{प्रतिगृह्य समाविश्य तद्वनं भरतर्षभ}
{मेघस्तनितनिर्घोषः सर्वभूतान्यकम्पयत्}


\twolineshloka
{दह्यतस्तस्य च बभौ रूपं दावस्य भारत}
{मेरोरिव नगेन्द्रस्य कीर्णस्यांशुमतोंशुभिः}


\chapter{अध्यायः २५२}
\twolineshloka
{वैशंपायन उवाच}
{}


\twolineshloka
{तौ रथाभ्यां रथश्रेष्ठौ दावस्योभयतः स्थितौ}
{दिक्षु सर्वासु भूतानां चक्राते कदनं महत्}


\twolineshloka
{यत्र यत्र च दृश्यन्ते प्राणिनः खाण्डवालयाः}
{पलायन्तः प्रवीरौ तौ तत्रतत्राभ्यधावताम्}


\twolineshloka
{छिद्रं न स्म प्रपश्यन्ति रथयोराशुचारिणोः}
{आविद्धावेव दृश्येते रथिनौ तौ रथोत्तमौ}


\twolineshloka
{खाण्डवे दह्यमाने तु भूतान्यथ सहस्रशः}
{उत्पेतुर्भैरवान्नादान्विनदन्तः समन्ततः}


\twolineshloka
{दग्धैकदेशा बहवो निष्टप्ताश्च तथाऽपरे}
{स्फुटिताक्षा विशीर्णाश्च विप्लुताश्च तथाऽपरे}


\twolineshloka
{समालिङ्ग्य सुतानन्ये पितॄन्भ्रातॄनथाऽपरे}
{त्यक्तुं न शेकुः स्नेहेन तत्रैव निधनं गताः}


\twolineshloka
{सन्दष्टदशनाश्चान्ये समुत्पेतुरनेकशः}
{ततस्तेऽतीव घूर्णन्तः पुनरग्नौ प्रपेदिरे}


\twolineshloka
{दग्धपक्षाक्षिचरणा विचेष्टन्तो महीतले}
{तत्रतत्र स्म दृश्यन्ते विनश्यन्तः शरीरिणः}


\twolineshloka
{जलाशयेषु तप्तेषु क्वाथ्यमानेषु वह्निना}
{गतसत्वाः स्म दृश्यन्ते कूर्ममत्स्याः समन्ततः}


\twolineshloka
{शरीरैरपरे दीप्तैर्देहवन्त इवाग्नयः}
{अदृश्यन्त वने तत्र प्राणिनः प्राणिसंक्षये}


\twolineshloka
{कांश्चिदुत्पततः पार्थः शरैः संछिद्य खण्डशः}
{पातयामास विहगान्प्रदीप्ते वसुरेतसि}


\twolineshloka
{ते शराचितसर्वाङ्गा निनदन्तो महारवान्}
{ऊर्ध्वमुत्पत्य वेगेन निपेतुः खाण्डवे पुनः}


\twolineshloka
{शरैरभ्याहतानां च सङ्घशः स्म वनौकसाम्}
{विरावः शुश्रुवे घोरः समुद्रस्येव मथ्यतः}


\twolineshloka
{वह्नेश्चापि प्रदीप्तस्य खमुत्पेतुर्महार्चिषः}
{जनयामासुरुद्वेगं सुमहान्तं दिवौकसाम्}


\fourlineindentedshloka
{तेनार्चिषा सुसन्तप्ता देवाः सर्षिपुरोगमाः}
{ततो जग्मुर्महात्मानः सर्व एव दिवौकसः}
{शतक्रतुं सहस्राक्षं देवेशमसुरार्दनम् ॥देवा ऊचुः}
{}


\threelineshloka
{किं न्विमे मानवाः सर्वे दह्यन्ते चित्रभानुना}
{कच्चिन्न संक्षयः प्राप्तो लोकानाममरेश्वर ॥वैशंपायन उवाच}
{}


\twolineshloka
{तच्छ्रुत्वा वृत्रहा तेभ्यः स्वयमेवान्ववेक्ष्य च}
{खाण्डवस्य विमोक्षार्थं प्रययौ हरिवाहनः}


\twolineshloka
{महता रथबृन्देन नानारूपेण वासवः}
{आकाशं समवाकीर्य प्रववर्ष सुरेश्वरः}


\twolineshloka
{ततोऽक्षमात्रा व्यसृजन्धाराः शतसहस्रशः}
{चोदिता देवराजेन जलदाः खाण्डवं प्रति}


\twolineshloka
{असंप्राप्तास्तु ता धारास्तेजसा जातवेदसः}
{ख एव समुशुष्यन्त नकाश्चित्पावकं गताः}


\twolineshloka
{ततो नमुचिहा क्रुद्धो भृशमर्चिष्मतस्तदा}
{पुनरेव महामेघैरम्भांसि व्यसृजद्बहु}


\twolineshloka
{अर्चिर्धाराभिसंबद्धं धूमविद्युत्समाकुलम्}
{बभूव तद्वनं घोरं स्तनयित्नुसमाकुलम्}


\chapter{अध्यायः २५३}
\twolineshloka
{वैशंपायन उवाच}
{}


\twolineshloka
{तस्याथ वर्षतो वारि पाण्डवः प्रत्यवारयत्}
{शरवर्षेण बीभत्सुरुत्तमास्त्राणि दर्शयन्}


\twolineshloka
{खाण्डवं च वनं सर्वं पाण्डवो बहुभिः शरैः}
{प्राच्छादयदमेयात्मा नीहारेणेव चन्द्रमाः}


\twolineshloka
{न च स्म किंचिच्छक्नोति भूतं निश्चरितुं ततः}
{संछाद्यमाने खे बाणैरस्यता सव्यसाचिना}


\twolineshloka
{तक्षकस्तु न तत्रासीन्नागराजो महाबलः}
{दह्यमाने वने तस्मिन्कुरुक्षेत्रं गतो हि सः}


\twolineshloka
{अश्वसेनोऽभवत्तत्र तक्षकस्य सुतो बली}
{स यत्नमकरोत्तीव्रं मोक्षार्थं जातवेदसः}


\twolineshloka
{न शशाक स निर्गन्तुं निरुद्धोऽर्जुनपत्रिभिः}
{मोक्षयामास तं माता निगीर्य भुजगात्मजा}


\twolineshloka
{तस्य पूर्वं शिरो ग्रस्तं पुच्छमस्य निगीर्यते}
{निगीर्य सोर्ध्वमक्रामत्सुतं नागी मुमुक्षया}


\twolineshloka
{तस्याः शरेण तीक्ष्णेन पृथुधारेण पाण्डवः}
{शिरश्चिच्छेद गच्छन्त्यास्तामपश्यच्छचीपतिः}


\twolineshloka
{तं मुमोचयिषुर्वज्री वातवर्षेण पाण्डवम्}
{मोहयामास तत्कालमश्वसेनस्त्वमुच्यत}


\twolineshloka
{तां च मायां तदा दृष्टा घोरां नागेन वञ्चितः}
{द्विधा त्रिधा च खगतान्प्राणिनः पाण्डवोच्छिनत्}


\twolineshloka
{शशाप तं च संक्रुद्धो बीभत्सुर्जिह्मगामिनम्}
{पावको वासुदेवश्चाप्यप्रतिष्ठो भविष्यसि}


\twolineshloka
{ततो जिष्णुः सहस्राक्षं खं वितत्याशुगैः शरैः}
{योधयामास संक्रुद्धो वञ्चनां तामनुस्मरन्}


\twolineshloka
{देवराजोऽपि तं दृष्ट्वा संरब्धं समरेऽर्जुनम्}
{स्वमस्त्रमसृजत्तीव्रं छादयित्वाऽखिलं नभः}


\twolineshloka
{ततो वायुर्महाघोषः क्षोभयन्सर्वसागरान्}
{वियत्स्थो जनयन्मेघाञ्जलधारासमाकुलान्}


\twolineshloka
{ततोऽशनिमुचो घोरांस्तडित्स्तनितनिःस्वनान्}
{तद्विघातार्थमसृजदर्जुनोऽप्यस्त्रमुत्तमम्}


\twolineshloka
{वायव्यमभिमन्त्र्याथ प्रतिपत्तिविशारदः}
{तेनेन्द्राशनिमेघानां वीर्यौजस्तद्विनाशितम्}


\twolineshloka
{जलधाराश्च ताः शोषं जग्मुर्नेशुश्च विद्युतः}
{क्षणेन चाभवद्व्योम संप्रशान्तरजस्तमः}


\twolineshloka
{सुखशीतानिलवहं प्रकृतिस्थार्कमण्डलम्}
{निष्प्रतीकारहृष्टश्च हुतभुग्विविधाकृतिः}


\twolineshloka
{सिच्यमानो वसौघैस्तैः प्राणिनां देहनिःसृतैः}
{प्रजज्वालाथ सोऽर्चिष्मान्स्वनादैः पूरयञ्जगत्}


\twolineshloka
{कृष्णाभ्यां रक्षितं दृष्ट्वा तं च दावमहङ्कृताः}
{खमुत्पेतुर्महाराज सुपर्णाद्याः पतत्त्रिणः}


\twolineshloka
{गरुत्मान्वज्रसदृशैः पक्षतुण्डनखैस्तथा}
{प्रहर्तुकामो न्यपतदाकाशात्कृष्णपाण्डवौ}


\twolineshloka
{तथैवोरगसङ्घाताः पाण्डवस्य समीपतः}
{उत्सृजन्तो विषं घोरं निपेतुर्ज्वलिताननाः}


\twolineshloka
{तांश्चकर्त शरैः पार्थः सरोषाग्निसमुक्षितैः}
{विविशुश्चापि तं दीप्तं देहाभावाय पावकम्}


\twolineshloka
{ततः सुराः सगन्धर्वा यक्षराक्षसपन्नगाः}
{उत्पेतुर्नादमतुलमुत्सृजन्तो रणार्थिनः}


\twolineshloka
{अयःकणपचक्राश्मभुशुण्ड्युद्यतबाहवः}
{कृष्णपार्थौ जिघांसन्तः क्रोधसंमूर्च्छितौजसः}


\twolineshloka
{तेषामतिव्याहरतां शस्त्रवर्षं प्रमुञ्चताम्}
{प्रममाथोत्तमाङ्गानि बीभत्सुर्निशितैः शरैः}


\twolineshloka
{कृष्णश्च सुमहातेजाश्चक्रेणारिविनाशनः}
{दैत्यदानवसङ्घानां चकार कदनं महत्}


\twolineshloka
{अथापरे शरैर्विद्धाश्चक्रवेगेरितास्तथा}
{वेलामिव समासाद्य व्यतिष्ठन्नमितौजसः}


\twolineshloka
{ततः शक्रोऽतिसक्रुद्धस्त्रिदशानां महेश्वरः}
{पाण्डुरं गजमास्थाय तावुभौ समुपाद्रवत्}


\twolineshloka
{वेगेनाशनिमादाय वज्रमस्त्रं च सोऽसृजत्}
{हतावेताविति प्राह सुरानसुरसूदनः}


\twolineshloka
{ततः समुद्यतां दृष्ट्वा देवेन्द्रेण महाशनिम्}
{जगृहुः सर्वशस्त्राणि स्वानि स्वानि सुरास्तथा}


\twolineshloka
{कालदण्डं यमो राजन् गदां चैव धनेश्वरः}
{पाशांश्च तत्र वरुणो विचित्रां च तथाऽशनिम्}


\twolineshloka
{स्कन्दः शक्तिं समादाय तस्थौ मेरुरिवाचलः}
{ओषधीर्दीप्यमानाश्च जगृहातेऽस्विनावपि}


\twolineshloka
{जगृहे च धनुर्धाता मुसलं तु जयस्तथा}
{पर्वतं चापि जग्राह क्रूद्धस्त्वष्टा महाबलः}


\twolineshloka
{अंशस्तु शक्तिं जग्राह मृत्युर्देवः परश्वधम्}
{प्रगृह्य परिघं घोरं विचचारार्यमा अपि}


\twolineshloka
{मित्रश्च क्षुरपर्यन्तं चक्रमादाय तस्थिवान्}
{पूषा भगश्च संक्रुद्धः सविता च विशांपते}


\twolineshloka
{आत्तकार्मुकनिस्त्रिंशाः कृष्णापार्थौ प्रदुद्रुवुः}
{रुद्राश्च वसवश्चैव मरुतश्च महाबलाः}


\twolineshloka
{विश्वेदेवास्तथा साध्या दीप्यमानाः स्वतेजसा}
{एते चान्ये च बहवो देवास्तौ पुरुषोत्तमौ}


\twolineshloka
{कृष्णपार्थौ जिघांसन्तः प्रतीयुर्विविधायुधाः}
{तत्राद्भुतान्यदृश्यन्त निमित्तानि महाहवे}


\twolineshloka
{युगान्तसमरूपाणि भूतसंमोहनानि च}
{तथा दृष्ट्वा सुसंरब्धं शक्रं देवैः सहाच्युतौ}


\twolineshloka
{अभीतौ युधि दुर्धर्षौ तस्थतुः सज्जकार्मुकौ}
{आगच्छतस्ततो देवानुभौ युद्धविशारदौ}


\twolineshloka
{व्यताडयेतां संक्रुद्धौ शरैर्वज्रोपमैस्तदा}
{असकृद्भग्नसंकल्पाः सुराश्च बहुशः कृताः}


\twolineshloka
{भयाद्रणं परित्यज्य शख्रमेवाभिशिश्रियुः}
{दृष्ट्वा निवारितान्देवान्माधवेनार्जुनेन च}


\twolineshloka
{आश्चर्यमगमंस्तत्र मुनयो नभसि स्थिताः}
{शक्रश्चापि तयोर्वीर्यमुपलभ्यासकृद्रणे}


\twolineshloka
{बभूव परमप्रीतो भूयश्चैतावयोधयत्}
{ततोऽश्मवर्षं सुमहद्व्यसृजत्पाकशासनः}


\twolineshloka
{भूय एव तदा वीर्यं जिज्ञासुः सव्यसाचिनः}
{तच्छैरर्जुनो वर्षं प्रतिजघ्नेऽत्यमर्षितः}


\twolineshloka
{विफलं क्रियमाणं तत्समवेक्ष्य शतक्रतुः}
{भूयः संवर्धयामास तद्वर्षं पाकशासनः}


\twolineshloka
{सोश्मवर्षं महावेगैरिषुभिः पाकशासनिः}
{विलयं गमयामास हर्षयन्पितरं तथा}


\twolineshloka
{तत उत्पाट्य पाणिभ्यां मन्दराच्छिखरं महत्}
{सद्रुमं व्यसृजच्छक्रो जिघांसुः पाण्डुनन्दनम्}


\twolineshloka
{ततोऽर्जुनो वेगवद्भिर्ज्वलिताग्रैरजिह्मगैः}
{शरैर्विध्वंसयामास गिरेः शृङ्गं सहस्रधा}


\twolineshloka
{गिरेर्विशीर्यमाणस्य तस्य रूपं तदा बभौ}
{सार्कचन्द्रग्रहस्येव नभसः परिशीर्यतः}


\twolineshloka
{तेनाभिपतता दावं शैलेन महता भृशम्}
{शृङ्गेण निहतास्तत्र प्राणिनः खाण्डवालयाः}


\chapter{अध्यायः २५४}
\twolineshloka
{वैशंपायन उवाच}
{}


\twolineshloka
{तथा शैलनिपातेन भीषिताः खाण्डवालयाः}
{दानवा राक्षसा नागास्तरक्ष्वृक्षवनौकसः}


\twolineshloka
{द्विपाः प्रभिन्नाः शार्दूलाः सिंहाः केसरिणस्तथा}
{मृगाश्च महिषाश्चैव शतशः पक्षिणस्तथा}


\twolineshloka
{समुद्विग्ना विससृपुस्तथान्या भूतजातयः}
{तं दावं समुदैक्षन्त कृष्णौ चाभ्युद्यतायुधौ}


\twolineshloka
{उत्पातनादशब्देन त्रासिता इव चाभवन्}
{ते वनं प्रसमीक्ष्याथ दह्यमानमनेकधा}


\twolineshloka
{कृष्णमभ्युद्यतास्त्रं च नादं मुमुचुरुल्बणम्}
{तेन नादेन रौद्रेण नादेन च विभावसोः}


\twolineshloka
{ररास गगनं कृत्स्नमुत्पातजलदैरिव}
{ततः कृष्णो महाबाहुः स्वतेजोभास्वरं महत्}


\twolineshloka
{चक्रं व्यसृजदत्युग्रं तेषां नाशाय केशवः}
{तेनार्ता जातयः क्षुद्राः सदानवनिशाचराः}


\twolineshloka
{निकृत्ताः शतशः सर्वा निपेतुरनलं क्षणात्}
{तत्रादृश्यन्त ते दैत्याः कृष्णचक्रविदारिताः}


\twolineshloka
{वसारुधिरसंपृक्ताः सन्ध्यायामिव तोयदाः}
{पिशाचान्पक्षिणो नागान्पशूंश्चैव सहस्रशः}


\twolineshloka
{निघ्नंश्चरति वार्ष्णेयः कालवत्तत्र भारत}
{क्षिप्तं क्षिप्तं पुनश्चक्रं कृष्णस्यामित्रघातिनः}


\twolineshloka
{छित्त्वानेकानि सत्वानि पाणिमेति पुनः पुनः}
{तथा तु निघ्नतस्तस्य पिशाचोरगराक्षसान्}


\twolineshloka
{बभूव रूपमत्युग्रं सर्वभूतात्मनस्तदा}
{समेतानां च सर्वेषां दानवानां च सर्वशः}


\twolineshloka
{विजेता नाभवत्कश्चित्कृष्णपाण्डवयोर्मृधे}
{तयोर्बलात्परित्रातुं तं च दावं यदा सुराः}


\twolineshloka
{नाशक्नुवञ्शमयितुं तदाऽभूवन्पराङ्मुखाः}
{शतक्रतुस्तु संप्रेक्ष्य विमुखानमरांस्तथा}


\twolineshloka
{बभूव मुदितो राजन्प्रशंसन्केशवार्जुनौ}
{निवृत्तेष्वथ देवेषु वागुवाचाशरीरिणी}


\twolineshloka
{शतक्रतुं समाभाष्य महागम्भीरनिःस्वना}
{न ते सखा सन्निहितस्तक्षको भुजगोत्तमः}


\twolineshloka
{दाहकाले खाण्डवस्य कुरुक्षेत्रं गतो ह्यसौ}
{न च शक्यौ युधा जेतुं कथंचिदपि वासव}


\twolineshloka
{वासुदेवार्जुनावेतौ निबोध वचनान्मम}
{नरनारायणावेतौ पूर्वदेवौ दिवि श्रुतौ}


\twolineshloka
{भवानप्यभिजानाति यद्वीर्यौ यत्पराक्रमौ}
{नैतौ शक्यौ दुराधर्षौ विजेतुमजितौ युधि}


\twolineshloka
{अपि सर्वेषु लोकेषु पुराणावृषिसत्तमौ}
{पूजनीयतमावेतावपि सर्वैः सुरासुरैः}


\twolineshloka
{यक्षराक्षसगन्धर्वनरकिन्नरपन्नगैः}
{तस्मादितः सुरैः सार्धं गन्तुमर्हसि वासव}


\twolineshloka
{दिष्टं चाप्यनुपश्यैतत्खाण्डवस्य विनाशनम्}
{इति वाक्यमुपश्रुत्य तथ्यमित्यमरेश्वरः}


\twolineshloka
{क्रोधामर्षौ समुत्सृज्य संप्रतस्थे दिवं तदा}
{तं प्रस्थितं महात्मानं समवेक्ष्य दिवौकसः}


\twolineshloka
{सहिताः सेनया राजन्ननुजग्मुः पुरन्दरम्}
{देवराजं तदा यान्तं सह देवैरवेक्ष्य तु}


\twolineshloka
{वासुदेवार्जुनौ वीरौ सिंहनादं विनेदतुः}
{देवराजे गते राजन्प्रहृष्टौ केशवार्जुनौ}


\twolineshloka
{निर्विशङ्कं वनं वीरौ दाहयामासतुस्तदा}
{स मारुत इवाभ्राणि नाशयित्वाऽर्जुनः सुरान्}


\twolineshloka
{व्यधमच्छरसङ्घातैर्देहिनः खाण्डवालयान्}
{न च स्म किंचिच्छक्नोति भूतं निश्चरितुं ततः}


\twolineshloka
{संछिद्यमानमिषुभिरस्यता सव्यसाचिना}
{नाशक्नुवंश्च भूतानि महान्त्यपि रणेऽर्जुनम्}


\twolineshloka
{निरीक्षितुममोघास्त्रं योद्धुं चापि कुतो रणे}
{शतं चैकेन विव्याध शतेनैकं पतत्रिणाम्}


\twolineshloka
{व्यसवस्तेऽपतन्नग्नौ साक्षात्कालहता इव}
{न चालभन्त ते शर्म रोधःसु विषमेषु च}


\twolineshloka
{पितृदेवनिवासेषु सन्तापश्चाप्यजायत}
{भूतसङ्घाश्च बहवो दीनाश्चक्रुर्महास्वनम्}


\twolineshloka
{रुरुदुर्वारणाश्चैव तथा मृगतरक्षवः}
{तेन शब्देन वित्रेसुर्गङ्गोदधिचरा झषाः}


\twolineshloka
{विद्याधरगणाश्चैव ये च तत्र वनौकसः}
{न त्वर्जुनं महाबाहो नापि कृष्णं जनार्दनम्}


\twolineshloka
{निरीक्षितुं वै शक्नोति कश्चिद्योद्धुं कुतः पुनः}
{एकायनगता येऽपि निष्पेतुस्तत्र केचन}


\twolineshloka
{राक्षसा दानवा नागा जघ्ने चक्रेण तान्हरिः}
{ते तु भिन्नशिरोदेहाश्चक्रवेगाद्गतासवः}


\twolineshloka
{पेतुरन्ये महाकायाः प्रदीप्ते वसुरेतसि}
{समांसरुधिरौधैश्च वसाभिश्चापि तर्पितः}


\twolineshloka
{उपर्याकाशगो भूत्वा विधूमः समपद्यत}
{दीप्ताक्षो दीप्तजिह्वश्च संप्रदीप्तमहाननः}


\twolineshloka
{दीप्तोर्ध्वकेशः पिङ्गाक्षः पिबन्प्राणभृतां वसाम्}
{तां स कृष्णार्जुनकृतां सुधां प्राप्य हुताशनः}


\twolineshloka
{बभूव मुदितस्तृप्तः परां निर्वृतिमागतः}
{तथाऽसुरं मयं नाम तक्षकस्य निवेशनात्}


\twolineshloka
{विप्रद्रवन्तं सहसा ददर्श मधुसूदनः}
{तमग्निः प्रार्थयामास दिधक्षुर्वातसारथिः}


\twolineshloka
{शरीरवाञ्जटी भूत्वा नदन्निव बलाहकः}
{विज्ञाय दानवेन्द्राणां मयं वै शिल्पिनां वरम्}


\twolineshloka
{जिघांसुर्वासुदेवस्तं चक्रमुद्यम्य धिष्ठितः}
{स चक्रमुद्यतं दृष्ट्वा दिधक्षन्तं च पावकम्}


\twolineshloka
{अभिधावार्जुनेत्येवं मयस्त्राहीति चाब्रवीत्}
{तस्य भीतस्वनं श्रुत्वा मा भैरिति धनंजयः}


\twolineshloka
{प्रत्युवाच मयं पार्थो जीवयन्निव भारत}
{तं न भेतव्यमित्याह मयं पार्थो दयापरः}


\twolineshloka
{तं पार्थेनाभये दत्ते नमुचेर्भ्रातरं मयम्}
{न हन्तुमैच्छद्दाशार्हः पावको न ददाह च}


\twolineshloka
{तद्वनं पावको धीमान्दिनानि दश पञ्च च}
{ददाह कृष्णपार्थाभ्यां रक्षितः पाकशासनात्}


\twolineshloka
{तस्मिन्वने दह्यमाने षडग्निर्न ददाह च}
{अश्वसेनं मयं चैव चतुरः शार्ङ्गकांस्तथा}


\chapter{अध्यायः २५५}
\twolineshloka
{जनमेजय उवाच}
{}


\twolineshloka
{किमर्थं शार्ङ्गकानग्निर्न ददाह तथा गते}
{तस्मिन्वने दह्यमाने ब्रह्मन्नेतत्प्रचक्ष्व मे}


\twolineshloka
{अदाहे ह्यश्वसेनस्य दानवस्य मयस्य च}
{कारणं कीर्तितं ब्रह्मञ्शार्ङ्गकाणां न कीर्तितम्}


\threelineshloka
{तदेतदद्भुतं ब्रह्मञ्शार्ङ्गकाणामनामयम्}
{कीर्तयस्वाग्निसंमर्दे कथं ते न विनाशिताः ॥वैशंपायन उवाच}
{}


\twolineshloka
{यदर्थं शार्ङ्गकानग्निर्न ददाह तथा गते}
{तत्ते सर्वं प्रवक्ष्यामि यथा भूतमरिन्दम}


\twolineshloka
{धर्मज्ञानां मुख्यतमस्तपस्वी संशितव्रतः}
{आसीन्महर्षिः श्रुतवान्मन्दपाल इति श्रुतः}


\twolineshloka
{स मार्गमाश्रितो राजन्नृषीणामूर्ध्वरेतसाम्}
{स्वाध्यायवान्धर्मरतस्तपस्वी विजितेन्द्रियः}


\twolineshloka
{स गत्वा तपसः पारं देहमुत्सृज्य भारत}
{जगाम पितृलोकाय न लेभे तत्र तत्फलम्}


\threelineshloka
{स लोकानफलान्दृष्ट्वा तपसा निर्जितानपि}
{पप्रच्छ धर्मराजस्य समीपस्थान्दिवौकसः ॥मन्दपाल उवाच}
{}


\twolineshloka
{किमर्थमावृता लोका ममैते तपसाऽर्जिताः}
{किं मया न कृतं तत्र यस्यैतत्कर्मणः फलम्}


\threelineshloka
{तत्राहं तत्करिष्यामि यदर्थमिदमावृतम्}
{फलमेतस्य तपसः कथयध्वं दिवौकसः ॥देवा ऊचुः}
{}


\twolineshloka
{ऋणिनो मानवा ब्रह्मञ्जायन्ते येन तच्छृणु}
{क्रियाभिर्ब्रह्मचर्येण प्रजया च न संशयः}


\twolineshloka
{तदपाक्रियते सर्वं यज्ञेन तपसा सुतैः}
{तपस्वी यज्ञकृच्चासि न च ते विद्यते प्रजा}


\twolineshloka
{त इमे प्रसवस्यार्थे तव लोकाः समावृताः}
{प्रजायस्व ततो लोकानुपभोक्ष्यसि पुष्कलान्}


\threelineshloka
{पुन्नाम्नो नरकात्पुत्रस्त्रायते पितरं श्रुतिः}
{तस्मादपत्यसन्ताने यतस्व ब्रह्मसत्तम ॥वैशंपायन उवाच}
{}


\twolineshloka
{तच्छ्रुत्वा मन्दपालस्तु वचस्तेषां दिवौकसाम्}
{क्व नु शीघ्रमपत्यं स्याद्बहुलं चेत्यचिन्तयत्}


\twolineshloka
{स चिन्तयन्नभ्यगच्छत्सुबहुप्रसवान्खगान्}
{शार्ङ्गिकां शार्ङ्गको भूत्वा जरितां समुपेयिवान्}


\twolineshloka
{तस्यां पुत्रानजनयच्चतुरो ब्रह्मवादिनः}
{तानपास्य स तत्रैव जगाम लपितां प्रति}


\twolineshloka
{बालान्स तानण्डगतान्सह मात्रा मुनिर्वने}
{तस्मिन्गते महाभागे लपितां प्रति भारत}


\twolineshloka
{अपत्यस्नेहसंयुक्ता जरिता बह्वचिन्तयत्}
{तेन त्यक्तानसंत्याज्यानृषीनण्डगतान्वने}


\twolineshloka
{न जहौ पुत्रशोकार्ता जरिता खाण्डवे सुतान्}
{बभार चैतान्संजातान्स्ववृत्त्या स्नेहविक्लवा}


\twolineshloka
{ततोऽग्निं खाण्डवं दग्धुमायान्तं दृष्टवानृषिः}
{मन्दपालश्चरंस्तस्मिन्वने लपितया सह}


\twolineshloka
{तं संकल्पं विदित्वाग्नेर्ज्ञात्वा पुत्रांश्च बालकान्}
{सोऽभितुष्टाव विप्रर्षिब्रार्ह्मणो जातवेदसम्}


\threelineshloka
{पुत्रान्प्रतिवदन्भीतो लोकपालं महौजसम्}
{मन्दपाल उवाच}
{त्वमग्ने सर्वलोकानां मुखं त्वमसि हव्यवाट्}


\twolineshloka
{त्वमन्तः सर्वभूतानां गूढश्चरसि पावक}
{त्वामेकमाहुः कवयस्त्वामाहुस्त्रिविधं पुनः}


\twolineshloka
{त्वामष्टधा कल्पयित्वा यज्ञवाहमकल्पयन्}
{त्वया विश्वमिदं सृष्टं वदन्ति परमर्षयः}


\twolineshloka
{त्वदृते हि जगत्कृत्स्नं सद्यो नश्येद्धुताशन}
{तुभ्यं कृत्वा नमो विप्राः स्वकर्मविजितां गतिम्}


\twolineshloka
{गच्छन्ति सह पत्नीभिः सुतैरपि च शाश्वतीम्}
{त्वामग्ने जलदानाहुः खेविषक्तान्सविद्युतः}


\twolineshloka
{दहन्ति सर्वभूतानि त्वत्तो निष्क्रम्य हेतयः}
{जातवेदस्त्वयैवेदं विश्वं सृष्टं महाद्युते}


\twolineshloka
{तवैव कर्मविहितं भूतं सर्वं चराचरम्}
{त्वयापो विहिताः पूर्वं त्वयि सर्वमिदं जगत्}


\twolineshloka
{त्वयि हव्यं च कव्यं च यथावत्संप्रतिष्ठितम्}
{त्वमेव दहनो देव त्वं धाता त्वं बृहस्पतिः}


\threelineshloka
{त्वमश्विनौ यमौ मित्रः सोमस्त्वमसि चानिलः}
{वैशंपायन उवाच}
{एवं स्तुतस्तदा तेन मन्दपालेन पावकः}


\twolineshloka
{तुतोष तस्य नृपते मुनेरमिततेजसः}
{उवाच चैनं प्रीतात्मा किमिष्टं करवाणि ते}


\twolineshloka
{तमब्रवीन्मन्दपालः प्राञ्जलिर्हव्यवाहनम्}
{प्रदहन्खाण्डवं दावं मम पुत्रान्विसर्जय}


\twolineshloka
{तथेति तत्प्रतिश्रुत्य भगवान्हव्यवाहनः}
{खाण्डवे तेन काले न प्रजज्वाल दिदक्षया}


\chapter{अध्यायः २५६}
\twolineshloka
{वैशंपायन उवाच}
{}


\twolineshloka
{ततः प्रज्वलिते वह्नौ शार्ङ्गकास्ते सुदुःखिताः}
{व्यथिताः परमोद्विग्ना नाधिजग्मुः परायणम्}


\threelineshloka
{निशाम्य पुत्रकान्बालान्माता तेषां तपस्विनी}
{जरिता शोकदुःखार्ता विललाप सुदुःखिता ॥जरितोवाच}
{}


\twolineshloka
{अयमग्निर्दहन्कक्षमित आयाति भीषणः}
{जगत्संदीपयन्भीमो मम दुःखविवर्धनः}


\twolineshloka
{इमे च मां कर्षयन्ति शिशवो मन्दचेतसः}
{अबर्हाश्चरणैर्हीनाः पूर्वेषां नः परायणाः}


\twolineshloka
{त्रासयंश्चायमायाति लेलिहानो महीरुहान्}
{अजातपक्षाश्च सुता न शक्ताः सरणे मम}


\twolineshloka
{आदाय च न शक्नोमि पुत्रांस्तरितुमात्मना}
{न च त्यक्तुमहं शक्ता हृदयं दूयतीव मे}


\twolineshloka
{कं तु जह्यामहं पुत्रं कमादाय व्रजाम्यहम्}
{किंनु मे स्यात्कृतं कृत्वा मन्यध्वं पुत्रकाः कथम्}


\twolineshloka
{चिन्तयाना विमोक्षं वो नाधिगच्छामि किंचन}
{छादयिष्यामि वो गात्रैः करिष्ये मरणं सह}


\twolineshloka
{जरितारौ कुलं ह्येतज्ज्येष्ठत्वेन प्रतिष्ठितम्}
{सारिसृक्कः प्रजायेत पितॄणां कुलवर्धनः}


\twolineshloka
{स्तम्बमित्रस्तपः कुर्याद्द्रोणो ब्रह्मविदां वरः}
{इत्येवमुक्त्वा प्रययौ पिता वो निर्घृणः पुरा}


\fourlineindentedshloka
{कमुपादाय शक्येयं गन्तुं कष्टाऽऽपदुत्तमा}
{किं नु कृत्वा कृतं कार्यं भवेदिति च विह्वला}
{नापश्यत्स्वधिया मोक्षं स्वसुतानां तदानलात् ॥वैशंपायन उवाच}
{}


\twolineshloka
{एवं ब्रुवाणां शार्ङ्गास्ते प्रत्यूचुरथ मातरम्}
{स्नेहमुत्सृज्य मातस्त्वं पत यत्र न हव्यवाट्}


\twolineshloka
{अस्मास्विह विनष्टेषु भवितारः सुतास्तव}
{त्वयि मातर्विनष्टायां न नः स्यात्कुलसन्ततिः}


\twolineshloka
{अन्ववेक्ष्यैतदुभयं क्षेमं स्याद्यत्कुलस्य नः}
{तद्वै कर्तुं परः कालो मातरेष भवेत्तव}


\threelineshloka
{मा त्वं सर्वविनाशाय स्नेहं कार्षीः सुतेषु नः}
{न हीदं कर्म मोघं स्याल्लोककामस्य नः पितुः ॥जरितोवाच}
{}


\twolineshloka
{इदमाखोर्बिलं भूमौ वृक्षस्यास्य समीपतः}
{तदाविशध्वं त्वरिता वह्नेरत्र न वो भयम्}


\twolineshloka
{ततोऽहं पांसुना छिद्रमपिधास्यामि पुत्रकाः}
{एवं प्रतिकृतं मन्ये ज्वलतः कृष्णवर्त्मनः}


\threelineshloka
{तत एष्याम्यतीतेऽग्नौ विहन्तुं पांसुशंचयम्}
{रोचतामेष वो वादो मोक्षार्थं च हुताशनात् ॥शार्ङ्गका ऊचुः}
{}


\twolineshloka
{अबर्हान्मांसभूतान्नः क्रव्यादाखुर्विनाशयेत्}
{पश्यमाना भयमिदं प्रवेष्टुं नात्र शक्नुमः}


\twolineshloka
{कथमग्निर्न नो धक्ष्येत्कथमाखुर्न नाशयेत्}
{कथं न स्यात्पिता मोघः कथं माता ध्रियेत नः}


\twolineshloka
{बिल आखोर्विनाशः स्यादग्नेराकाशचारिणाम्}
{अन्ववेक्ष्यैतदुभयं श्रेयान्दाहो न भक्षणम्}


\twolineshloka
{गर्हितं मरणं नः स्यादाखुना भक्षिते बिले}
{शिष्टादिष्टः परित्यागः शरीरस्य हुताशनात्}


% Check verse!
`अग्निदाहे तु नियतं ब्रह्मलोके ध्रुवा गतिः ॥'
\chapter{अध्यायः २५७}
\twolineshloka
{जरितोवाच}
{}


\threelineshloka
{अस्माद्बिलान्निष्पतितमाखुं श्येनो जहार तम्}
{क्षुद्रं पद्भ्यां गृहीत्वा च यातो नात्र भयं हि वः ॥शार्ङ्गका ऊचुः}
{}


\twolineshloka
{न हृतं तं वयं विद्मः श्येनेनाखुं कथंचन}
{अन्येऽपि भितारोऽत्र तेभ्योऽपि भमेव नः}


\twolineshloka
{संशयो वह्निरागच्छेद्दृष्टं वायोर्निवर्तनम्}
{मृत्युर्नो बिलवासिभ्यो बिले स्यान्नात्र संशयः}


\threelineshloka
{निःसंशयात्संशयितो मृत्युर्मातर्विशिष्यते}
{चर खे त्वं यथान्यायं पुत्रानाप्स्यसि शोभनान् ॥जरितोवाच}
{}


\twolineshloka
{अहं वेगेन तं यान्तमद्राक्षं पततां वरम्}
{बिलादाखुं समादाय श्येनं पुत्रा महाबलम्}


\twolineshloka
{तं पतन्तं महावेगा त्वरिता पृष्ठतोऽन्वगाम्}
{आशिषोऽस्य प्रयुञ्जाना हरतो मूषिकं बिलात्}


\twolineshloka
{यो नो द्वेष्टारमादाय श्येनराज प्रधावसि}
{भव त्वं दिवमास्थाय निरमित्रो हिरण्मयः}


\twolineshloka
{स यदा भक्षितस्तेन श्येनेनाखुः पतत्रिणा}
{तदाहं तमनुज्ञाप्य प्रत्युपायां पुनर्गृहम्}


\threelineshloka
{प्रविशध्वं बिलं पुत्रा विश्रब्धा नास्ति वो भयम्}
{श्येनेन मम पश्यन्त्या हृत आखुर्महात्मना ॥शार्ङ्गका ऊचुः}
{}


\threelineshloka
{न विद्महे हृतं मातः श्येनैनाखुं कथंचन}
{अविज्ञाय न शक्यामः प्रवेष्टं विवरं भुवः ॥जरितोवाच}
{}


\threelineshloka
{अहं तमभिजानामि हृतं श्येनेन मूषिकम्}
{नास्ति वोऽत्र भयं पुत्राः क्रियतां वचनं मम ॥शार्ङ्गका ऊचुः}
{}


\twolineshloka
{न त्वं मिथ्योपचारेण मोक्षयेथा भयाद्धि नः}
{समाकुलेषु ज्ञानेषु न बुद्धिकृतमेव तत्}


\twolineshloka
{न चोपकृतमस्माभिर्न चास्मान्वेत्थ ये वयम्}
{पीड्यमाना बिभर्ष्यस्मान्का सती के वयं तव}


\twolineshloka
{तरुणी दर्शीयाऽसि समर्था भर्तुरेषणे}
{अनुगच्छ पतिं मातुः पुत्रानाप्स्यसि शोमनान्}


\threelineshloka
{वयमस्निं समाविश्य लोकानाप्स्याम शोभनान्}
{अथास्मान्न दहेदग्निरायास्त्वं पुनरेव नः ॥वैशंपायन उवाच}
{}


\twolineshloka
{एवमुक्ता ततः शार्ङ्गी पुत्रानुत्सृज्य खाण्डवे}
{जगाम त्वरिता देशं क्षेममग्नेरनामयम्}


\twolineshloka
{ततस्तीक्ष्णार्चिरभ्यागात्त्वरितो हव्यवाहनः}
{यत्र शार्ङ्गा वभूवुस्ते मन्दपालस्य पुत्रकाः}


\threelineshloka
{ततस्तं ज्वलितं दृष्ट्वा ज्वलनं ते विहङ्गमाः}
{`व्यथिताः करुणा वाचः श्रावयामासुरन्तिकात्}
{'जरितारिस्ततो वाक्यं श्रावयामास पावकम्}


\chapter{अध्यायः २५८}
\twolineshloka
{जरितारिरुवाच}
{}


\twolineshloka
{पुरतः कृच्छ्रकालस्य धीमाञ्जागर्ति पुरुषः}
{स कृच्छ्रकालं संप्राप्य व्यथां नैवैति कर्हिचित्}


\threelineshloka
{यस्तु कृच्छ्रमनुप्राप्तं विचेता नावबुध्यते}
{सकृच्छ्रकाले व्यथितो न श्रेयो विन्दते महत् ॥सारिसृक्व उवाच}
{}


\threelineshloka
{धीरस्त्वमसि मेधावी प्राणकृच्छ्रमिदं च नः}
{प्राज्ञः शूरो बहूनां हि भवत्येको न संशयः ॥स्तम्बमित्र उवाच}
{}


\threelineshloka
{ज्येष्ठस्तातो भवति वै ज्येष्ठो मुञ्चति कृच्छ्रतः}
{ज्येष्ठश्चेन्न प्रजानाति नीयान्किं करिष्यति ॥द्रोण उवाच}
{}


\threelineshloka
{हिरण्यरेतास्त्वरितो जलन्नायाति नः क्षयम्}
{सप्तजिह्वाननः क्रूरो लिहानो विसर्पति ॥वैशंपायन उवाच}
{}


\threelineshloka
{एवं संभाष्य तेऽन्योन्यं मन्दपालस्य पुत्रकाः}
{तुष्टुवुः प्रयता भूत्वा यथाऽग्निं शृणु पार्थिव ॥जरितारिरुवाच}
{}


\twolineshloka
{आत्माऽसि वायोर्ज्वलन शरीरमसि वीरुधाम्}
{योनिरापश्च ते शुक्रं योनिस्त्वमसि चाम्भसः}


\threelineshloka
{ऊर्ध्वं चाधश्च सर्पन्ति पृष्ठतः पार्श्वतस्तथा}
{अर्चिषस्ते महावीर्य रश्यमः सवितुर्यथा ॥सारिसृक्क उवाच}
{}


\twolineshloka
{माता प्रणष्टा पितरं न विद्मःपक्षा जाता नै नो धूमकेतो}
{न नस्त्राता विद्यते वै त्वदन्य-स्तस्मादस्मांस्त्राहि बालांस्त्वमग्ने}


\twolineshloka
{यदग्ने ते शिवं रूपं ये च ते सप्त हेतयः}
{तेन नः परिपाहि त्वमार्तान्नः शरणैषिणः}


\threelineshloka
{त्वमेवैकस्तपसे जप्तवेदोनान्यस्तप्ता विद्यते गोषु देव}
{ऋषीनस्मान्बालकान्पालयस्वपरेणास्मान्प्रेहि वै हव्यवाह ॥स्तम्बमित्र उवाच}
{}


\twolineshloka
{सर्वमग्ने त्वमेवैकस्त्वयि सर्वमिदं जगत्}
{त्वं धारयसि भूतानि भुवनं त्वं बिभर्षि च}


\twolineshloka
{त्वमग्निर्हव्यवाहस्त्वं त्वमेव परमं हविः}
{मनीषिणस्त्वां जानन्ति बहुधा चैकधापि च}


\threelineshloka
{सृष्ट्वा लोकांस्त्रीनिमान्हव्यवाहकाले प्राप्ते पचसि पुनः समिद्धः}
{त्वं सर्वस्य भुवनस्य प्रसूति-स्त्वमेवाग्ने भवसि पुनः प्रतिष्ठा ॥द्रोण उवाच}
{}


\twolineshloka
{त्वमन्नं प्राणिभिर्भुक्तमन्तर्भूतो जगत्पते}
{नित्यप्रवृद्धः पचसि त्वयि सर्वं प्रतिष्ठितम्}


\twolineshloka
{सूर्यो भूत्वा रश्मिभिर्जातवेदोभूमेरम्भो भूमिजातान्रसांश्च}
{विश्वानादाय पुनरुत्सृज्य कालेदृष्ट्वा वृष्ट्या भावयसीह शुक्र}


\twolineshloka
{त्वत्त एताः पुनः शुक्र वीरुधो हरितच्छदाः}
{जायन्ते पुष्करिण्यश्च सुभद्रश्च महोदधिः}


\twolineshloka
{इदं वै सद्म तिग्मांशो वरुणस्य परायणम्}
{शिवस्त्राता भवास्माकं माऽस्मानद्य विनाशय}


\threelineshloka
{पिङ्गाक्ष लोहितग्रीव कृष्णवर्त्मन्हुताशन}
{परेण प्रेहि मुञ्चास्मान्सागरस्य गृहानिव ॥वैशंपायन उवाच}
{}


\threelineshloka
{एवमुक्तो जातवेदा द्रोणेन ब्रह्मवादिना}
{द्रोणमाह प्रतीतात्मा मन्दपालप्रतिज्ञया ॥अग्निरुवाच}
{}


% Check verse!
ऋषिर्द्रोणस्त्वमसि वै ब्रह्मैतद्व्याहृतंईप्सितं ते करिष्यामि न च ते
\twolineshloka
{मन्दपालेन वै यूयं मम पूर्वं निवेदिताः}
{वर्जयेः पुत्रकान्मह्यं दहन्दावमिति स्म ह}


\fourlineindentedshloka
{तस्य तद्वचनं द्रोण त्वया यच्चेह भाषितम्}
{उभयं मे गरीयस्तु ब्रूहि किं करवाणि ते}
{भृशं प्रीतोऽस्मि भद्रं ते ब्रह्मंस्तोत्रेण सत्तम ॥द्रोण उवाच}
{}


\threelineshloka
{इमे मार्जारकाः शुक्र नित्यमुद्वेजयन्ति नः}
{एतान्कुरुष्व दग्धांस्त्वं हुताशन सबान्धवान् ॥वैशंपायन उवाच}
{}


\twolineshloka
{तथा तत्कृतवानग्निरभ्यनुज्ञाय शार्ङ्गकान्}
{ददाह खाण्डवं दावं समिद्धो जनमेजय}


\chapter{अध्यायः २५९}
\twolineshloka
{वैशंपायन उवाच}
{}


\twolineshloka
{मन्दपालोऽञपि कौरव्यं चिन्तयामास पुत्रकान्}
{उक्त्वाऽपि च स तिग्मांशुं नैव शर्माधिगच्छति}


\twolineshloka
{स तप्यमानः पुत्रार्थे लपितामिदमब्रवीत्}
{कथं नु शक्ताः शरणे लपिते मम पुत्रकाः}


\twolineshloka
{वर्धमाने हुतवहे वाते चाशु प्रवायति}
{असमर्था विमोक्षाय भविष्यन्ति ममात्मजाः}


\twolineshloka
{कथं त्वशक्ता त्राणाय माता तेषां तपस्विनी}
{भविष्यति हि शोकार्ता पुत्रत्राणमपश्यती}


\twolineshloka
{कथमुड्डीयनेऽशक्तान्पतने च ममात्मजान्}
{सन्तप्यमाना बहुधा वाशमाना प्रधावती}


\twolineshloka
{जरितारिः कथं पुत्रः सारिसृक्कः कथं च मे}
{स्तम्बमित्रः कथं द्रोणः कथं सा च तपस्विनी}


\twolineshloka
{लालप्यमानं तमृषइं मन्दपालं तथा वने}
{लपिता प्रत्युवाचेदं सासूयमिव भारत}


\twolineshloka
{न ते पुत्रेष्ववेक्षाऽस्ति यानृषीनुक्तवानसि}
{तेजस्विनो वीर्यवन्तो न तेषां ज्वलनाद्भयम्}


\twolineshloka
{त्वयाऽग्नौ ते परीताश्च स्वयं हि मम सन्निधौ}
{श्रुतं तथा चेति ज्वलनेन महात्मना}


\twolineshloka
{पलो न तां वाचमुक्त्वा मिथ्या करिष्यति}
{न्धुकृत्ये न तेन ते स्वस्थ मानसम्}


\twolineshloka
{तामेव तु ममामित्रां चिन्तयन्परितप्यसे}
{ध्रुवं मयि न ते स्नेहो यथा तपयं पुराऽभवत्}


\twolineshloka
{नहि पक्षवता न्याय्यं निः हेन सुहृज्जने}
{पीड्यमान उपद्रष्टुं शक्तेना मा कथंचन}


\threelineshloka
{गच्छ त्वं जरितामेव यदर्थं परितप्यसे}
{चरिष्याम्यहमप्येका यथा पुरुषाश्रिता ॥मन्दपाल उवाच}
{}


\twolineshloka
{नाहमेवं चरे लोके यथा त्वमभिमन्यसे}
{अपत्यहेतोर्विचरे तच्च कृच्छ्रगतं मम}


\twolineshloka
{भूतं हित्वा च भाव्यर्थे योऽवलम्बेत्स मन्दधीः}
{अवमन्येत तं लोको यथेच्छसि तथा कुरु}


\threelineshloka
{एष हि प्रज्वलन्नग्निर्लेलिहानी महीरुहान्}
{आविग्ने हृदि सन्तापं जनयत्यशिवं मम ॥वैशंपायन उवाच}
{}


\twolineshloka
{`भर्तुर्हि वाक्यं सा श्रुत्वा लपिता दुःखिताऽभवत्}
{सान्त्वयामास च पुनः पति पतिपरायणा ॥'}


\twolineshloka
{तस्माद्देशादतिक्रान्ते ज्वलने जरिता पुनः}
{जगाम पुत्रकानेन जरिता पुत्रगृद्धिनी}


\twolineshloka
{सा तान्कुशलिनः सर्वान्विमुक्ताञ्जातवेदसः}
{रोरूयमाणान्ददृशे वने पुत्रान्निरामयान्}


\twolineshloka
{अश्रूणि मुमुचे तेषां दर्शनात्सा पुनःपुनः}
{`न श्रद्धेयं ततस्तेषांर्शनं वै पुनःपुनः}


\twolineshloka
{इति मत्वाऽब्रवीद्वाकजरिता पुत्रगृद्धिनी}
{'एकाकशश्च पुत्रांस्तन्त्र्शमानान्वपद्यत}


% Check verse!
`जरिता तु परिष्वज्युत्रस्नेहाच्चुचुम्ब ह ॥'
\twolineshloka
{ततोऽभ्यगच्छत्सहसमन्दपालोऽपि भारत}
{अथ ते सर्व एवैनं भ्यनन्दंस्तदा सुताः}


\twolineshloka
{`गुरुत्वान्मन्दपालस्तपसश्च विशेषतः}
{अभिवादामहे सर्वे तपक्षाः प्रसादतः}


\threelineshloka
{एवमुक्तवतां तेषां तनन्द्य महातपाः}
{परिष्वज्य ततो मू उपाघ्राय च बलकान्}
{पुत्रान्स्वयं समाहूयतः प्रोवाच गौतमः ॥'}


\threelineshloka
{लालप्यमानमेकैकंरितां च पुनःपुनः}
{न चैवोचुस्तदा किंतमृषिं साध्वसाधु वा ॥मन्दपाल उवाच}
{}


\twolineshloka
{ज्येष्ठः सुतस्ते कत कतमस्तस्य चानुजः}
{मध्यमः कतमश्चैव यान्कतमश्च ते}


\fourlineindentedshloka
{एवं ब्रुवन्तं दुःखाकं मा न प्रतिभाषसे}
{कृतवानस्मि हव्यानैव शान्तिमितो लभे}
{`एवमुक्त्वा तु तां मन्दपालस्तदाऽस्पृशत् ॥' जरितोवाच}
{}


\twolineshloka
{किं नु ज्येष्ठेन ते किमनन्तरजेन ते}
{किं वा मध्यमजातेन किं कनिष्ठेन वा पुनः}


\threelineshloka
{यां त्वं मां सर्वतो हीनामुत्सृज्यासि गतः पुरा}
{तामेव लपितां गच्छ तरुणीं चारुहासिनीम् ॥मन्दपाल उवाच}
{}


\twolineshloka
{न स्त्रीणां विद्यते किंचिदमुत्र पुरुषान्तरात्}
{सापत्नकमृते लोके नान्यदर्थविनाशनम्}


\twolineshloka
{वैराग्निदीपनं चैव भृशुद्वेगकारि च}
{सुव्रता चापि कल्याणी सर्वभूतेषु विश्रुता}


\twolineshloka
{अरुन्धती महात्मानं वसिष्ठं पर्यशङ्कत}
{विशुद्धभावमत्यन्तं सदा प्रियहिते रतम्}


\threelineshloka
{सप्तर्षिमध्यगं वीरमवमेने च तं मुनिम्}
{अपध्यानेन सा तेन धूमारुणसमप्रभा}
{लक्ष्याऽलक्ष्या नाभिरूपा निमित्तमिव पश्यति}


\twolineshloka
{अपत्यहेतोः संप्राप्तं तथा त्वमपि मामिह}
{इष्टमेवं गते हि त्वं सा तथैवाद्य वर्तते}


\threelineshloka
{न हि भार्येति विश्वासः कार्यः पुंसा कथंचन}
{न हि कार्यमनुध्याति नारी पुत्रवती सती ॥वैशंपायन उवाच}
{}


\twolineshloka
{ततस्ते सर्व एवैनं पुत्राः सम्यगुपासते}
{स च तानात्मजान्सर्वानाश्वासयितुमुद्यतः}


\chapter{अध्यायः २६०}
\twolineshloka
{मन्दपाल उवाच}
{}


\twolineshloka
{युष्माकमपवर्गार्थं ती ज्वलनो मया}
{अग्निना च तथेत्येतिज्ञातं महात्मना}


\twolineshloka
{अग्नेर्वचनमाज्ञाय धर्मज्ञतां च वः}
{भवतां च परं वीर्यं नाहमिहागतः}


\threelineshloka
{न सन्तापो हि वर्त्थः पुत्रका हृदि मां प्रति}
{ऋषीन्वेद हुताशो ब्रह्म तद्विदितं च वः ॥वैशंपायन उवाच}
{}


\twolineshloka
{एवमाश्वासितान्पुत्रान्भार्यामादाय स द्विजः}
{मन्दपालस्ततो देशादन्यं देशं जगाम ह}


\twolineshloka
{भगवानापि तिग्मांशुः समिद्धः खाण्डवं ततः}
{ददाह सह कृष्णाभ्यां जनयञ्जगतो हितम्}


\twolineshloka
{वसामेदोवहाः कुल्यास्तत्र पीत्वा च पावकः}
{जगाम दर्शयामास चार्जुनम्}


\twolineshloka
{ततोऽञन्तरिक्षाद्भगवानवतीर्य पुरन्दरः}
{मरुद्गणैर्वृतः पार्थं केशवं चेदमब्रवीत्}


\twolineshloka
{कृतं युवाभ्यां कर्मेदममरैरपि दुष्करम्}
{वरं वृणीतं तुष्टोऽस्मि दुर्लभं पुरुषेष्विह}


\twolineshloka
{पार्थस्तु वरयामास शक्रादस्त्राणि सर्वशः}
{प्रदातुं तच्च शक्रस्तु कालं चक्रे महाद्युतिः}


\twolineshloka
{यदा प्रसन्नो भगवान्महादेवो भविष्यति}
{तदातुभ्यं प्रदास्यामि पाण्डवास्त्राणि सर्वशः}


\twolineshloka
{अहमेव च तं कालं वेत्स्यामि कुरुनन्दन}
{तपसा महता चापि दास्यामि भवतोऽप्यहम्}


\twolineshloka
{आग्नेयानि च सर्वाणि वायव्यानि च सर्वशः}
{मदीयानि च सर्वाणि ग्रहीष्यसि धनञ्जय}


\twolineshloka
{वासुदेवोऽपि जग्राह प्रीतिं पार्थेन शाश्वतीम्}
{ददौ सुरपतिश्चैव वरं कृष्णाय धीमते}


\twolineshloka
{एवं दत्त्वा वरं ताभ्यां सह देवैर्मरुत्पतिः}
{हुताशनमनुज्ञाप्य जगामत्रिदिवं प्रभुः}


\twolineshloka
{पावकश्च तदा दावं दग्ध्वसमृगपक्षिणम्}
{अहोभिरेकविंशद्भिर्विरराग्सुतर्पितः}


\twolineshloka
{जग्ध्वा मांसानि पीत्वा चदांसि रुधिराणि च}
{युक्तः परमया प्रीत्या तावुत्वाच्युतार्जुनौ}


\twolineshloka
{युवाभ्यां पुरुषाग्र्याभ्यां ततोऽस्मि यथासुखम्}
{अनुजानामि वां वीरौ चरतंत्र वाञ्छितम्}


\twolineshloka
{`गाण्डिवं च धनुर्दिव्यमक्षौ च महेषुधी}
{कपिध्वजो रथश्चायं तव द महामते}


\twolineshloka
{अनेन धनुषा चैव रथेनाने भारत}
{विजेष्यसि रणे शत्रून्सदेवामानुषान् ॥'}


\twolineshloka
{एवं तौ समनुज्ञातौ पाववेमहात्मना}
{अर्जुनो वासुदेवश्च दानवश्चयस्तथा}


\twolineshloka
{परिक्रम्य ततः सर्वे त्रयोऽभरतर्षभ}
{रमणीये नदीकूले सहितामुपाविशंन्}


