\part{पतिव्रतामाहात्म्यपर्व}
\newcommand{\onelineindentedshloka}[2]{
    {#1\hspace{\shlokaspaceskip}}\\
\onelineshloka{\hspace{\shlokaspaceskip}#2}}

\dnsub{श्रीवेदव्यासाय नमः}

\twolineshloka*
{नारायणं नमस्कृत्य नरं चैव नरोत्तमम्}
{देवीं सरस्वतीं व्यासं ततो जयमुदीरयेत्}

\chapter{चतुर्नवत्यधिकद्विशततमोऽध्यायः॥२९४॥}

\uvacha{युधिष्ठिर उवाच}

\twolineshloka
{नात्मानमनुशोचामि नेमान्भ्रातॄन्महामुने}
{हरणं चापि राज्यस्य यथेमां द्रुपदात्मजाम्}


\twolineshloka
{द्यूते दुरात्मभिः क्लिष्टाः कृष्णया तारिता वयम्}
{जयद्रथेन चपुनर्वनाच्चापि हृता बलात्}


\twolineshloka
{अस्ति सीमन्तिनी काचिद्दृष्टपूर्वाऽपिवा श्रुता}
{पतिव्रता महाभागा यथेयं द्रुपदात्मजा}

\uvacha{मार्कण्डेय उवाच}



\twolineshloka
{शृणु राजन्कुलस्त्रीणां महाभाग्यं युधिष्ठिर}
{सर्वमेतद्यथाप्राप्तं सावित्र्या राजकन्यया}


\twolineshloka
{आसीन्मद्रेषु धर्मात्मा राजा परमधार्मिकः}
{ब्रह्मण्यश्चमहात्मा च सत्यसन्धो जितेन्द्रियः}


\twolineshloka
{यज्वा दानपतिर्दक्षः पौरजानपदप्रियः}
{पार्थिवोऽश्वपतिर्नाम सर्वभूतहिते रतः}


\twolineshloka
{क्षमावाननपत्यश्च सत्यवाग्विजितेन्द्रियः}
{अतिक्रान्तेन वयसा सन्तापमुपजग्मिवान्}


\twolineshloka
{अपत्योत्पादनार्थं च तीव्रं नियममास्थितः}
{काले परिमिताहारो ब्रह्मचारी जितेन्द्रियः}


\twolineshloka
{हुत्वा शतसहस्रं स सावित्र्या राजसत्तम}
{षष्ठेषष्ठे तदाकाले बभूव मितमोजनः}


\twolineshloka
{एतेन नियमेनासीद्वर्षाण्यष्टादशैव तु}
{पूर्णे त्वष्टादशे वर्षे सावित्री तुष्टिमभ्यगात्}


\twolineshloka
{रूपिणी तु तदा राजन्दर्शयामास तं नृपम्}
{अग्निहोत्रात्समुत्थाय हर्षेण महताऽन्विता}


\twolineshloka
{उवाच चैनं वरदा वचनं पार्थिवं तदा}
{सा तमश्वपतिं राजन्सावित्री नियमे स्थितम्}


\twolineshloka
{ब्रह्मचर्येण शुद्धेन दमेन नियमेन च}
{सर्वात्मना च भक्त्या च तुष्टाऽस्मि तव पार्थिवाः}


\twolineshloka
{वरं वृणीष्वाश्वपते मद्रराज यदीप्सितम्}
{न प्रामादश्च धर्मेषु कर्तव्यस्ते कथञ्चन}

\uvacha{अश्वपतिरुवाच}



\twolineshloka
{अपत्यार्थः समारम्भः कृतो धर्मेप्सया मया}
{पुत्रा मे बहवो देवि भवेयुः कुलभावनाः}


\twolineshloka
{तुष्टाऽसि यदि मे देवि वरमेतं वृणोम्यहम्}
{सन्तानं परमो धर्म इत्याहुर्मां द्विजातयः}

\uvacha{सावित्र्युवाच}



\twolineshloka
{पूर्वमेव मया राजन्नभिप्रायमिमं तव}
{ज्ञात्वा पुत्रार्थमुक्तो वै भगवांस्ते पितामहः}


\twolineshloka
{प्रसादाच्चैव तस्मात्ते स्वयं विहितवत्यहम्}
{कन्या तेजस्विनी सौम्य क्षिप्रमेव भविष्यति}


\twolineshloka
{उत्तरं च न ते किञ्चिद्व्याहर्तव्यं कथञ्चन}
{पितामहनियोगेन तुष्टा ह्येतद्ब्रवीमि ते}


\twolineshloka
{स तथेति प्रतिज्ञाय सावित्र्या वचनं नृपः}
{प्रसादयामास पुनः क्षिप्रमेतद्भविष्यति}


\twolineshloka
{अन्तर्हितायां सावित्र्यां जगाम स्वपुरं नृपः}
{स्वराज्ये चावसद्बीरः प्रजा धर्मेण पालयन्}


\twolineshloka
{कस्मिंश्चित्तु गते काले स राजा नियतब्रतः}
{ज्येष्ठायां धर्मचारिण्यां महिष्यां गर्भमादधे}


\twolineshloka
{राजपुत्र्यास्तु गर्भः स मानव्या भरतर्षभ}
{व्यवर्धत तदा शुक्ले तारापतिरिवाम्बरे}


\twolineshloka
{प्राप्ते काले तु सुषुवे कन्यां राजीवलोचनाम्}
{क्रियाश्च तस्या मुदितश्चक्रे च नृपसत्तमः}

\twolineshloka
{सावित्र्या प्रीतया दत्ता सावित्र्या हुतया ह्यपि}
{सावित्रीत्येव नामास्याश्चक्रुर्विप्रास्तथा पिता}


\twolineshloka
{सा विग्रहवतीव श्रीव्यवर्धत नृपात्मजा}
{कालेन चापि सा कन्या यौवनस्ता बभूव ह}


\twolineshloka
{तां सुमध्यां पृथुश्रोणीं प्रतिमां काञ्चनीमिव}
{प्राप्तेयं देवकन्येति दृष्ट्वा सम्मेनिरे जनाः}


\twolineshloka
{तां तु पद्मपलाशाक्षीं ज्वलन्तीमिव तेजसा}
{न कश्चिद्वरयामास तेजसा प्रतिवारितः}


\twolineshloka
{अथोपोष्य शिरःस्नाता देवतामभिगम्य सा}
{हुत्वाग्निं विधिवद्विप्रान्वाचयामास पर्वणि}


\twolineshloka
{ततः सुमनसः शेषाः प्रतिगृह्य महात्मनः}
{पितुः समीपमगमद्देवी श्रीरिव रूपिणी}


\twolineshloka
{साऽभिवाद्य पितुः पादौ शेषाः पूर्वं निवेद्य च}
{कृताञ्जलिर्वरारोहा नृपतेः पार्श्वमास्थिता}


\twolineshloka
{यौवनस्थां तु तां दृष्ट्वा स्वां सुतां देवरूपिणीम्}
{अयाच्यमानां च वरैर्नृपतिर्दुःखितोऽभवत्}

\uvacha{राजोवाच}



\twolineshloka
{पुत्रि प्रदानकालस्ते न च कश्चिद्वृणोति माम्}
{स्वयमन्विच्छ भर्तारं गुणैः सदृशमात्मनः}


\twolineshloka
{प्रार्थितः पुरुषो यश्च स निवेद्यस्त्वया मम}
{विमृश्याहं प्रदास्यामि वरय त्वं यथेप्सितम्}


\twolineshloka
{श्रुतं हि धर्मशास्त्रेषु पठ्यमानं द्विजातिभिः}
{तथा त्वमपिकल्याणि गदतो मे वचः शृणु}


\twolineshloka
{अप्रदाता पिता वाच्यो वाच्यश्चानुपयन्पतिः}
{मृते पितरि पुत्रश्च वाच्यो मातुररक्षिता}


\twolineshloka
{इदं मे वचनं क्षुत्वा भर्तुरन्वेषणे न्वर}
{देवतानां यथा याच्यो न भवेयं तथा कुरु}


\twolineshloka
{एवमुक्त्वा दुहितरं तथा वृद्धांश्च मन्त्रिणः}
{व्यादिदेशानुयात्रं च गम्यतां चेत्यचोदयत्}


\twolineshloka
{साऽभिवाद्य पितुः पादौ व्रीडितेव मनस्विनी}
{पितुर्वचनमाज्ञाय निर्जगामाविचारितम्}


\twolineshloka
{सा हैमं रथमास्थाय स्थविरैः सचिवैर्वृता}
{तपोवनानिरम्याणि राजर्षीणां जगाम ह}


\twolineshloka
{मान्यानां तत्र वृद्धानां कृत्वा पादाभिवादनम्}
{वनानि क्रमशस्तात सर्वाण्येवाभ्यगच्छत}


\twolineshloka
{एवं तीर्थेषु सर्वेषु धनोत्सर्गं नृपात्मजा}
{कुर्वती द्विजमुख्यानां तं तं देशं जगाम ह}


॥इति श्रीमन्महाभारते अरण्यपर्वणि पतिव्रतामाहात्म्यपर्वणि चतुर्नवत्यधिकद्विशततमोऽध्यायः॥२९४॥


\chapter{पञ्चनवत्यधिकद्विशततमोऽध्यायः॥२९५॥}

\uvacha{मार्कण्डेय उवाच}

\twolineshloka
{अथ मद्राधिपो राजा नारदेन समागतः}
{उपविष्टः सभामध्ये कथायोगेन भारत}


\twolineshloka
{ततोऽभिगम्य तीर्थानि सर्वाण्येवाश्रमांस्तथा}
{आजगाम पितुर्वेश्म सावित्री सह मन्त्रिभिः}


\twolineshloka
{नारदेन सहासीनं सा दृष्ट्वा पितरं शुभा}
{उभयोरेव शिरसा चक्रे पादाभिवादनम्}

\uvacha{नारद उवाच}



\twolineshloka
{क्व गताऽभूत्सुतेयं ते कुतश्चैवागता नृप}
{किमर्थं युवतीं भद्र न चैनां सम्प्रयच्छसि}

\uvacha{अश्वपतिरुवाच}



\twolineshloka
{कार्येण खल्वनेनैव प्रेषिताद्यैव चागता}
{एतस्याः शृणु देवर्षे भर्तारं योऽनया वृतः}

\uvacha{मार्कण्डेय उवाच}



\twolineshloka
{सा ब्रूहि विस्तरेणेति पित्रा संयोदिता शुभा}
{तदैव तस्य वचनं प्रतिगृह्येदमब्रवीत्}


\twolineshloka
{आसीत्साल्वेषु धर्मात्मा क्षत्रियः पृथिवीपतिः}
{द्युमत्सेन इति ख्यातः पश्चाच्चान्धो बभूव ह}


\twolineshloka
{विनष्टचक्षुषस्तस्य बालपुत्रस्य धीमतः}
{सामीप्येन हृतं राज्यं छिद्रेऽस्मिन्पूर्ववैरिणा}


\twolineshloka
{स बालवत्सया सार्धं भार्यया प्रस्थितो वनम्}
{महारण्यं गतश्चापि तपस्तेपे महाव्रतः}


\twolineshloka
{तस्य पुत्रः पुरे जातः संवृद्धश्च तपोवने}
{सत्यवाननुरूपो मे भर्तेति मनसा वृतः}

\uvacha{नारद उवाच}



\twolineshloka
{अहो वत महत्पापं सावित्र्या नृपते कृतम्}
{अजानन्त्या यदनया गुणवान्सत्यवान्वृतः}


\twolineshloka
{सत्यं वदत्यस्य पिता सत्यं माता प्रभाषते}
{तथाऽस्य ब्राह्मणाश्चक्रुर्नामैतत्सत्यवानिति}


\twolineshloka
{बालस्याश्वाः प्रियाश्चास्य करोत्यश्वांश्च मृन्मयान्}
{चित्रेऽपि विलिखत्यश्वांश्चित्राश्व इति चोच्यते}

\uvacha{राजोवाच}



\twolineshloka
{अपीदानीं स तेजस्वी बुद्धिमान्वा नृपात्मजः}
{क्षमावानपि वा शृरः सत्यवान्पितृवत्सलः}

\uvacha{नारद उवाच}



\twolineshloka
{विवस्वानिव तेजस्वी बृहस्पतिसमो मतौ}
{महेन्द्र इव वीरश्च वसुधेव क्षमान्वितः}

\uvacha{अश्वपतिरुवाच}



\twolineshloka
{अपि राजात्मजो दाता ब्रह्मण्यश्चापि सत्यवान्}
{रूपवानप्युदारो वाऽप्यथवा प्रियदर्शनः}

\uvacha{नारद उवाच}



\twolineshloka
{साङ्कृते रन्तिदेवस्य स्वशक्त्या दानतः समः}
{ब्रह्मण्यः सत्यवादी च शिबिरौशीनरो यथा}


\twolineshloka
{ययातिरिव चोदारः सोमवत्प्रियदर्शनः}
{रूपेणान्यतमोऽश्विभ्यां द्युमत्सेनसुतो बली}


\threelineshloka
{स वदान्यः स तेजस्वी धीमांश्चैव क्षमान्वितः}
{स दान्तः स मृदुः शूरः स सत्यः संयतेन्द्रियः}
{सन्मैत्रः सोनसूयश्च स ह्रीमान्द्युतिमांश्च सः}


\twolineshloka
{नित्यशश्चार्जवं तस्मिन्धृतिस्तत्रैव च ध्रुवा}
{सङ्क्षेपतस्तपोवृद्धैः शीलवृद्धैश्च कथ्यते}

\uvacha{अश्वपतिरुवाच}



\twolineshloka
{गुणैरुपेतं सर्वैस्तं भगवन्प्रब्रवीषि मे}
{दोषानप्यस्य मे ब्रूहि यदि सन्तीह केचन}

\uvacha{नारद उवाच}



\twolineshloka
{एक एवास्य दोषो हि गुणानाक्रम्य तिष्ठति}
{स च दोषः प्रयत्नेन न शक्यमतिवर्तितुम्}


\twolineshloka
{एको दोषोऽस्ति नान्योऽस्य सोद्यप्रभृति सत्यवान्}
{संवत्सरेण क्षीणायुर्देहन्यासं करिष्यति}

\uvacha{राजोवाच}



\twolineshloka
{एहि सावित्रि गच्छस्व अन्यं वरय शोभने}
{तस्य दोषो महानेको गुणानाक्रम्य च स्थितः}


\twolineshloka
{यथा मे भगवानाह नारदो देवसत्कृतः}
{संवत्सरेण सोऽल्पायुर्देहन्यासं करिष्यति}

\uvacha{सावित्र्युवाच}



\twolineshloka
{सकृदंशो निपतति सकृत्कन्या प्रदीयते}
{सकृदाह ददानीति त्रीण्येतानि सकृत् सकृत्}


\twolineshloka
{दीर्घायुरथवाऽल्पायुः सगुणो निर्गुणोऽपि वा}
{सकृद्वृतो मया भर्ता न द्वितीयं वृणोम्यहम्}


\twolineshloka
{मनसा निश्चयं कृत्वाततो वाचाऽभिधीयते}
{क्रियते कर्मणा पश्चात्प्रमाणं मे मनस्ततः}

\uvacha{नारद उवाच}



\twolineshloka
{स्थिरा बुद्धिर्नरश्रेष्ठ सावित्र्या दुहितुस्तव}
{नैषा वारयितुं शक्या धर्मादस्मात्कथञ्चन}


\twolineshloka
{नान्यस्मिन्पुरुषे सन्ति ये सत्यवति वै गुणाः}
{प्रदानमेव तस्मान्मे रोचते दुहितुस्तव}

\uvacha{राजोवाच}



\twolineshloka
{अविचाल्यमेतदुक्तं तथ्यं च भवता वचः}
{करिष्याम्येतदेवं च गुरुर्हि भगवान्मम}

\uvacha{नारद उवाच}



\twolineshloka
{अविघ्नमस्तु सावित्र्याः प्रदाने दुहितुस्तव}
{साधयिष्याम्यहं तावत्सर्वेषां भद्रमस्तु वः}

\uvacha{मार्कण्डेय उवाच}



\twolineshloka
{एवमुक्त्वा स्वमुत्पत्य नारदस्त्रिदिवं गतः}
{राजाऽपि दुहितुः सज्जं वैवाहिकमकारयत्}


॥इति श्रीमन्महाभारते अरण्यपर्वणि पतिव्रतामाहात्म्यपर्वणि पञ्चनवत्यधिकद्विशततमोऽध्यायः॥२९५॥


\chapter{षण्णवत्यधिकद्विशततमोऽध्यायः॥२९६॥}

\uvacha{मार्कण्डेय उवाच}

\twolineshloka
{अथ कन्याप्रदाने स तमेवार्थं विचिन्तयन्}
{समानित्ये च तत्सर्वम्भाण्डं वैवाहिकं नृपः}


\twolineshloka
{ततो वृद्धान्द्विजान्सर्वानृत्विक्सभ्यपुरोहितान्}
{समाहूय दिने पुण्ये प्रययौ सह कन्यया}


\twolineshloka
{मेध्यारण्यं स गत्वा च द्युमत्सेनाश्रमं नृपः}
{पद्भ्यामेव द्विजैः सार्धं राजर्षिं तमुपागमत्}


\twolineshloka
{तत्रापश्यन्महाभागं सालवृक्षमुपाश्रितम्}
{कौश्यां बृस्यां समासीनं चक्षुर्हीनं नृपं तदा}


\twolineshloka
{स राजा तस्य राजर्षेः कृत्वापूजां यथाऽर्हतः}
{वाचा सुनियतो भूत्वा चकारात्मनिवेदनम्}


\twolineshloka
{तस्यार्घ्यमासनं चैव गां चावेद्य स धर्मवित्}
{किमागमनमित्येवं राजा राजानमब्रवीत्}


\twolineshloka
{तस्य सर्वमभिप्रायमितिकर्तव्यतां च ताम्}
{सत्यवन्तं समुद्दिश्य सर्वमेव न्यवेदयत्}


\twolineshloka
{सावित्री नाम राजर्षे कन्येयं मम शोभना}
{तां स्वधर्मेण धर्मज्ञ स्नुषार्थे त्वं गृहाण मे}

\uvacha{द्युमत्सेन उवाच}



\fourlineindentedshloka
{च्युताः स्म राज्याद्वनवासमाश्रिताश्-}
{चराम धर्मं नियतास्तपस्विनः}
{कथं त्वनर्हा वनवासमाश्रमे}
{सहिष्यति क्लेशमिमं सुता तव}

\uvacha{अश्वमतिरुवाच}



\fourlineindentedshloka
{सुखं च दुःखं च भवाभवात्मकं}
{यदा विजानाति सुताऽहमेव च}
{न मद्विधे युज्यते वाक्यमीदृशं}
{विनिश्चयेनाभिगतोऽस्मि ते नृप}


\twolineshloka
{आशां नार्हसि मे हन्तुं सौहृदात्प्रणतस्य च}
{अभितश्चागतं प्रेम्णा प्रत्याख्यातुं न माऽर्हसि}


\twolineshloka
{अनुरूपो हि युक्तश्च त्वं ममाहं तवापि च}
{स्नुषां प्रतीच्छ मे कन्यां भार्यां सत्यवतस्ततः}

\uvacha{द्युमत्सेन उवाच}



\twolineshloka
{पूर्वमेवाभिलवितः सम्बन्धो मे त्वया सह}
{भ्रष्टराज्यस्त्वहमिति तत एतद्विचारितम्}


\twolineshloka
{अभिप्रायस्त्वयं यो मे पूर्वमेवाभिकाङ्क्षितः}
{स निर्वर्ततु मेऽद्यैव काङ्क्षितो ह्यसि मेऽतिथिः}


\twolineshloka
{ततः सर्वान्समानाय्य द्विजानाश्रमवासिनः}
{यथाविधि समुद्वाहं कारयामासतुर्नृपौ}


\twolineshloka
{दत्त्वा सोऽश्वपतिः कन्यां यथार्हं सपरिच्छदम्}
{ययौ स्वमेव भवनं युक्तः परमया मुदा}


\twolineshloka
{सत्यवानपि तां भार्यां लब्ध्वा सर्वगुणान्विताम्}
{मुमुदे सा च तं लब्ध्वा भर्तारं मनसेप्सितम्}


\twolineshloka
{गते पितरि सर्वाणि सन्न्यस्याभरणानि सा}
{जगृहेवल्कलान्येव वस्त्रं काषायमेव च}


\twolineshloka
{परिचारैर्गुणैश्चैव प्रश्रयेण दमेन च}
{सर्वकामक्रियाभिश्च सर्वेषां तुष्टिमादधे}


\twolineshloka
{श्वश्रूं शरीरसत्कारैः सर्वैराच्छादनादिभिः}
{श्वशुरं देवसत्कारैर्वाचः संयमनेन च}


\twolineshloka
{तथैव प्रियवादेन नैषुणेन शमेन च}
{रहश्चैवोपचारेण भर्तारं पर्यतोषयत्}


\twolineshloka
{एवं तत्राश्रमे तेषां तदा निवसतां सताम्}
{कालस्तपस्यतां कश्चिदपाक्रामत भारत}


\twolineshloka
{सावित्र्या ग्लायमानायास्तिष्ठन्त्यास्तु दिवानिशम्}
{नारदेन यदुक्तं तद्वाक्यं मनसि वर्तते}


॥इति श्रीमन्महाभारते अरण्यपर्वणि पतिव्रतामाहात्म्यपर्वणि षण्णवत्यधिकद्विशततमोऽध्यायः॥२९६॥


\chapter{सप्तनवत्यधिकद्विशततमोऽध्यायः॥२९७॥}

\uvacha{मार्कण्डेय उवाच}

\twolineshloka
{ततः काले बहुतिथे व्यतिक्रान्ते कदाचन}
{प्राप्तः स कालो मर्तव्यं यत्र सत्यवता नृप}


\twolineshloka
{गणयन्त्याश्च सावित्र्या दिवसदिवसे गते}
{यद्वाक्यं नारदेनोक्तं वर्तते हृदि नित्यशः}


\twolineshloka
{चतुर्थेऽहनि मर्तव्यमिति सञ्चिन्त्य भामिनी}
{व्रतं त्रिरात्रमुद्दिश्य दिवारात्रं स्थिताऽभवत्}


\twolineshloka
{त्रयोदश्यां चोपवासं प्रतिपत्सु च पारणम्}
{आयुष्यं वर्धते भ र्तुर्व्रतेनानेन भारत}


\twolineshloka
{तं श्रुत्वा नियमं तस्या भृशं दुःखान्वितो नृपः}
{उत्थाय वाक्यं सावित्रीमब्रवीत्परिसान्त्वयन्}


\twolineshloka
{अतितीव्रोऽयमारम्भस्त्वयाऽऽरब्धो नृपात्मजे}
{तिसृणां वसतीनां हि स्तानं परमदुश्चरम्}

\uvacha{सावित्र्युवाच}



\twolineshloka
{न कार्यस्तात सन्तापः पारयिष्याम्यहं व्रतम्}
{व्यसंसायकृतं हीदं व्यवसायश्च कारणम्}

\uvacha{द्युमत्सेन उवाच}



\twolineshloka
{व्रतं भिन्धीति वक्तुं त्वां नास्मि शक्तः कथञ्चन}
{पारयस्वेति वचनं युक्तमस्मद्विधो वदेत्}

\uvacha{मार्कण्डेय उवाच}



\twolineshloka
{एवमुक्त्वा द्युमत्सेनो विरराम महामनाः}
{तिष्ठन्ती चैव सावित्री काण्ठभूतेव लक्ष्यते}


\twolineshloka
{श्वोभूते भर्तृमरणे सावित्र्या भरतर्षभ}
{दुःखान्वितायास्तिष्ठन्त्याः सा रात्रिर्व्यत्यवर्तत}


\twolineshloka
{अद्य तद्दिवसं चेति हुत्वा दीप्तं हुताशनम्}
{युगमात्रोदिते सूर्येकृत्वा पौर्वाङ्णिकीः क्रियाः}


\threelineshloka
{व्रतं समाप्यसावित्री स्ना त्वा शुद्धा यशस्विनी}
{ततः सर्वान्द्विजान्वृद्धाञ्श्वश्रूं श्वशुरमेव च}
{अभिवाद्यानुपूर्व्येण प्राञ्जलिर्नियता स्थिता}


\twolineshloka
{अवैधव्याशिषस्ते तु सावित्र्यर्थं हिताः शुभाः}
{ऊचुस्तपस्विनः सर्वे तपोवननिवासिनः}


\twolineshloka
{एवमस्त्विति सावित्री ध्यानयोगपरायणा}
{मनसा ता गिरः सर्वाः प्रत्यगृह्णात्तपस्विनी}


\twolineshloka
{तं कालं तं मुहूर्तं च प्रतीक्षन्ती नृपात्मजा}
{यथोक्तं नारदवचश्चिन्तयन्ती सुदुःखिता}


\twolineshloka
{ततस्तु श्वश्रूश्वशुरावूचतुस्तां नृपात्मजाम्}
{एकान्तमास्थितां वाक्यं प्रीत्या भरतसत्तम}


\twolineshloka
{व्रतं यथोपदिष्टं तु तथा तत्पारितं त्वया}
{आहारकालः सम्प्राप्तः क्रियतां यदनन्तरम्}

\uvacha{सावित्र्युवाच}



\twolineshloka
{अस्तं गते मयाऽऽदित्ये भोक्तव्यं कृतकामया}
{एष मे हृदि सङ्कल्पः समयश्च कृतो मया}

\uvacha{मार्कण्डेय उवाच}



\twolineshloka
{एवं सम्भाषमाणायाः सावित्र्या भोजनं प्रति}
{स्कन्धे परशुमादाय सत्यवान्प्रस्थितो वनम्}


\twolineshloka
{सावित्री त्वाह भर्तारं नैकस्त्वं गन्तुमर्हसि}
{सह त्वया गमिष्यामि न हित्वां हातुमुत्सहे}

\uvacha{सत्यवानुवाच}



\twolineshloka
{वनं न गतपूर्वं ते दुःख पन्थाश्च भामिनि}
{व्रतोपवासक्षामा च कथं पद्भ्यां गमिष्यसि}

\uvacha{सावित्र्युवाच}



\twolineshloka
{उपवासान्न मे ग्लानिर्नास्ति चापि परिश्रमः}
{गमने च कृतोत्साहां प्रतिषेद्धुं न माऽर्हसि}

\uvacha{सत्यवानुवाच}



\twolineshloka
{यदि ते गमनोत्साहः करिष्यामि तव प्रयम्}
{मम त्वामन्त्रय गुरून्न मां दोषः स्पृशेदयम्}

\uvacha{मार्कण्डेय उवाच}



\twolineshloka
{साऽभिवाद्याब्रवीच्छ्वश्रूं श्वशुरं च महाव्रता}
{अयं गच्छति मे भर्ता फलाहारो महावनम्}


\twolineshloka
{इच्छेयमभ्यनुज्ञाता आर्यया श्वशुरेण ह}
{अनेन सह निर्गन्तुं न मेऽद्य विरहः क्षमः}


\twolineshloka
{गुर्वग्निहोत्रार्तकृतेप्रस्थितश्च सुतस्तव}
{न निवार्यो निवार्यः स्यादन्यथा प्रस्थितो वनम्}


\twolineshloka
{संवत्सरः किञ्चिदूनो न निष्क्रान्ताऽहमाश्रमात्}
{वनं कुसुमितं द्रष्टुं परं कौतूहलं हि मे}

\uvacha{द्युमत्सेन उवाच}



\twolineshloka
{यदा प्रभृति सावित्री पित्रा दत्ता स्नुषा मम}
{नानयाऽभ्यर्थनायुक्तमुक्तपूर्वं स्मराम्यहम्}


\twolineshloka
{तदेषा लभतां कामं यथाभिलषितं वधूः}
{अप्रमादश्च कर्तव्यः पुत्रि सत्यवतः पथि}

\uvacha{मार्कण्डेय उवाच}



\twolineshloka
{उभाभ्यामभ्यनुज्ञाता सा जगाम यशस्विनी}
{सहभर्त्रा हसन्तीव हृदयेन विदूयता}


\twolineshloka
{सा वनानि विचित्राणि रमणीयानि सर्वशः}
{मयूरगणजुष्टानि ददर्श विपुलेक्षणा}


\twolineshloka
{नदीः पुण्यवहाश्चैव पुष्पितांश्च नगोत्तमान्}
{सत्यवानाह पश्येति सावित्रीं मधुरं वचः}


\twolineshloka
{निरीक्षमाणा भर्तारं सर्वावस्थमनिन्दिता}
{मृतमेव हि तं मेने काले मुनिवचः स्मरन्}


\twolineshloka
{अनुव्रजन्ती भर्तारं जगाम मृदुगामिनी}
{द्विधेव हृदयं कृत्वा तं च कालमवेक्षती}


॥इति श्रीमन्महाभारते अरण्यपर्वणि पतिव्रतामाहात्म्यपर्वणि सप्तनवत्यधिकद्विशततमोऽध्यायः॥२९७॥


\chapter{अष्टनवत्यधिकद्विशततमोऽध्यायः॥२९८॥}

\uvacha{मार्कण्डेय उवाच}

\twolineshloka
{अथ भार्यासहायः स फलान्यादाय वीर्यवान्}
{कठिनं पूरयामास ततः काण्ठान्यपाटयत्}


\twolineshloka
{तस्य पाटयतः काष्ठं स्वेदो वै समजायत}
{व्यायामेन च तेनास्य जज्ञे शिरसि वेदना}


\twolineshloka
{सोऽभिगम्य प्रियां भार्यामुवाच श्रमपीडितः}
{व्यायामेन ममानेन जाता शिरसि वेदना}


\twolineshloka
{अङ्गानि चैव सावित्रि हृदयं दूयतीव च}
{अस्वस्थमिव चात्मानं लक्षये मितभाषिणि}


\threelineshloka
{शूलैरिव शिरो विद्धमिदं संलक्षयाम्यहम्}
{भ्रमन्तीव दिशः सर्वा श्चक्रारूढं मनो मम}
{तत्स्वप्तुमिच्छे कल्याणि न स्तातुं शक्तिरस्ति मे}


\twolineshloka
{सा समासाद्य सावित्री भर्तारमुपगम्य च}
{उत्सङ्गेऽस्य शिर कृत्वा निषसाद महीतले}


\twolineshloka
{ततः सा नारदवचो विमृशन्ती तपस्विनी}
{तं मुहूर्तं क्षणं वेलां दिवसं च युयोज ह}


\threelineshloka
{हन्त प्राप्तः स कालोऽयमिति चिन्तापरा सती}
{मुहूर्तादेव चापश्यत् पुरुषं रक्तवाससम्}
{बद्धमौलिं वपुष्मन्तमादित्यसमतेजसम्}


\twolineshloka
{श्यामावदातं रक्ताक्षं पाशहस्तं भयावहम्}
{स्थितं सत्यवतः पार्श्वे निरीक्षन्तं तमेव च}


\twolineshloka
{तं दृष्ट्वा सहसोत्थाय भर्तुर्न्यस्य शनैः शिरः}
{कृताञ्जलिरुवाचाऽऽर्ता हृदयेन प्रवेपती}


\twolineshloka
{दैवतं त्वाभिजानामि वपुरेतद्ध्यमानुषम्}
{कामया ब्रूहि देवेश कस्त्वं किञ्च चिकीर्षसि}

\uvacha{यम उवाच}



\twolineshloka
{पतिव्रताऽसि सावित्रि तथैव च तपोन्विता}
{अतस्त्वामभिभाषामि विद्धि मां त्वं शुभे यमम्}


\twolineshloka
{अयं ते सत्यवान्भर्ता क्षीणायुः पार्थिवात्मजः}
{नेष्यामि तमहं बद्ध्वा विद्ध्येतन्मे चिकीर्षितम्}

\uvacha{सावित्र्युवाच}



\twolineshloka
{श्रूयते भगवन्दूतास्तवागच्छन्ति मानवान्}
{नेतुं किल भवान्कस्मादागतोऽसि स्वयं प्रभो}

\uvacha{मार्कण्डेय उवाच}



\twolineshloka
{इत्युक्तः पितृराजस्तां भगवान्स्वचिकीर्षितम्}
{यथावत्सर्वमाख्यातुं तत्प्रियार्थं प्रचक्रमे}


\twolineshloka
{अयं च धर्मसंयुक्तो रूपवान्गुणसागरः}
{नार्हो मत्पुरुषैर्नेतुमतोऽस्मि स्वयमागतः}


\twolineshloka
{ततः सत्यवतः कायात्पाशबद्धं वशङ्गतम्}
{अङ्गुष्ठमात्रं पुरुषं निश्चकर्ष यमो बलात्}


\twolineshloka
{ततः समुद्धृतप्राणं गतश्वासं हतप्रभम्}
{निर्विचेष्टं शरीरं तद्बभूवा प्रियदर्शनम्}


\twolineshloka
{यमस्तु तं ततो बद्ध्वा प्रयातो दक्षिणामुखः}
{सावित्री चैव दुःखार्ता यममेवान्वगच्छत}


\threelineshloka
{भर्तुः शरीररक्षां च विधाय हि तपस्विनी}
{भर्तारमनुगच्छन्ती  तथावस्थं सुमध्यमा}
{नियमव्रतसंसिद्धा महाभागा पतिव्रता}

\uvacha{यम उवाच}



\twolineshloka
{निवर्त गच्छ सावित्रि कुरुष्वास्यौर्ध्वदैहिकम्}
{कृतम्भर्तुस्त्वयाऽऽनृण्यं यावद्गम्यं गतं त्वया}

\uvacha{सावित्र्युवाच}



\twolineshloka
{यत्र मे नीयते भर्ता स्वयं वा यत्र गच्छति}
{मया च तत्र गन्तव्यमेष धर्मः सनातनः}


\twolineshloka
{तपसा गुरुभक्त्या च भर्तुः स्नेहाद्व्रतेन च}
{तव चैव प्रसादेन न मे प्रतिहता गतिः}


\twolineshloka
{प्राहुः साप्तपदं मैत्रं बुधास्तत्त्वार्थदर्शिनः}
{मित्रतां च पुरस्कृत्य किञ्चिद्वक्ष्यामि तच्छृणु}


\fourlineindentedshloka
{नानात्मवन्तस्तु वने चरन्ति}
{धर्मं च वासं च परिश्रमं च}
{विज्ञानतो धर्ममुदाहरन्ति}
{तस्मात्सन्तो धर्ममाहुः प्रधानम्}


\fourlineindentedshloka
{एकस्य धर्मेण सतां मतेन}
{सर्वे स्म तं मार्गमनुप्रपन्नाः}
{मा वै द्वितीयं मा तृतीयं च वाञ्छे}
{तस्मात्सन्तो धर्ममाहुः प्रधानम्}

\uvacha{यम उवाच}



\fourlineindentedshloka
{निवर्त तुष्टोऽस्मि तवानया गिरा}
{स्वराक्षरव्यञ्जनहेतुयुक्तया}
{वरं वृणीष्वेह विनाऽस्य जीवितं}
{ददानि ते सर्वमनिन्दिते वरम्}

\uvacha{सावित्र्युवाच}



\fourlineindentedshloka
{च्युतः स्वराज्याद्वनवासमाश्रितो}
{विनष्टचक्षुः श्वशुरो ममाश्रमे}
{स लब्धचक्षुर्बलवान्भवेन्नृपस्-}
{तव प्रसादाज्ज्वलनार्कसन्निभः}

\uvacha{यम उवाच}



\fourlineindentedshloka
{ददानि तेऽहं तमनिन्दिते वरं}
{यथा त्वयोक्तं भविता च तत्तथा}
{तवाध्वना ग्लानिमिवोपलक्षये}
{निवर्त गच्छस्व न ते श्रमो भवेत्}

\uvacha{सावित्र्युवाच}



\fourlineindentedshloka
{श्रमः कुतो भर्तृसमीपतो हि मे}
{यतो हि भर्ता मम सा गतिर्ध्रुवा}
{यतः पतिं नेष्यसि तत्र मे गतिः}
{सुरेश भूयश्च वचो निबोध मे}


\fourlineindentedshloka
{सतां सकृत्सङ्गतमीप्सितं परं}
{ततः परं मित्रमिति प्रचक्षते}
{न चाफलं सत्पुरुषेण सङ्गतं}
{ततः सतां सन्निवसेत्समागमे}

\uvacha{यम उवाच}



\fourlineindentedshloka
{मनोऽनुकूलं बुधबुद्धिवर्धनं}
{त्वया यदुक्तं वचनं हिताश्रयम्}
{विना पुनः सत्यवतोऽस्य जीवितं}
{वरं द्वितीयं वरयस्व भामिनि}

\uvacha{सावित्र्युवाच}



\fourlineindentedshloka
{हृतं पुरा मे श्वशुरस्य धीमतःस्}
{वमेव राज्यं लभतां स पार्थिवः}
{जह्यात्स्वधर्मान्न च मे गुरुर्यथा}
{द्वितीयमेतद्वरयामि ते वरम्}

\uvacha{यम उवाच}



\fourlineindentedshloka
{स्वमेवं राज्यं प्रतिपत्स्यतेऽचिरान्-}
{न च स्वधर्मात्परिहीयते नृपः}
{कृतेन कामेन मया नृपात्मजे}
{निवर्त गच्छस्व न ते श्रमो भवेत्}

\uvacha{सावित्र्युवाच}



\fourlineindentedshloka
{प्रजास्त्वयैता नियमेन संयता}
{नियम्य चैता नयसे निकामया}
{ततो यमत्वं तव देव विश्रुतं}
{निबोध चेमां गिरमीरितां मया}


\twolineshloka
{अद्रोहः सर्वभूतेषु कर्मणा मनसा गिरा}
{अनुग्रहश्च दानं च सतां धर्मः सनातनः}


\twolineshloka
{एवम्प्रायश्च लोकोऽयं मनुष्याः शक्तिपेशलाः}
{सन्तस्त्वेवाप्यमित्रेषु दयां प्राप्तेषु कुर्वते}

\uvacha{यम उवाच}



\fourlineindentedshloka
{पिपासितस्येव भवेद्यथा पयस्-}
{तथा त्वया वाक्यमिदं समीरितम्}
{विना पुनः सत्यवतोऽस्य जीवितं}
{वरं वृणीष्वेह शुभे यदिच्छसि}

\uvacha{सावित्र्युवाच}



\fourlineindentedshloka
{ममानपत्यः पृथिवीपतिः पिता}
{भवत्पितुः पुत्रशतं तथौरसम्}
{कुलस्य सन्तानकरं च यद्भवेत्}
{तृतीयमेतद्वरयामि ते वरम्}

\uvacha{यम उवाच}



\fourlineindentedshloka
{कुलस्य सन्तानकरं सुवर्चसं}
{शतं सुतानां पितुरस्तु ते शुभे}
{कृतेन कामेन नराधिपात्मजे}
{निवर्त दूरं हि पथस्त्वमागता}

\uvacha{सावित्र्युवाच}



\fourlineindentedshloka
{न दूरमेतन्मम भर्तृसन्निधौ}
{मनो हि मे दूरतरं प्रधावति}
{अथ व्रजन्नेव गिरं समुद्यतां}
{मयोच्यमानां शृणु भूय एव च}


\fourlineindentedshloka
{विवस्वतस्त्वं तनयः प्रतापवान्स्-}
{ततो हि वैवस्वत उच्यसे बुधैः}
{समेन धर्मेण चरन्ति ताः प्रजास्}
{ततस्तवेहेवर धर्मराजता}


\twolineshloka
{आत्मन्यपि न विश्वासस्तथा भवति सत्सु यः}
{तस्मात्सत्सु विशेषेण सर्वः प्रणयमिच्छति}


\twolineshloka
{सौहदात्सर्वभूतानां विश्वासो नाम जायते}
{तस्मात्सत्सु विशेषेण विश्वासं कुरुते जनः}

\uvacha{यम उवाच}


\fourlineindentedshloka
{उदाहृतं ते वचनं यदङ्गने}
{शुभे न तादृक् त्वदृते श्रुतं मया}
{अनेन तुष्टोऽस्मि विनाऽस्य जीवितं}
{वरं चतुर्थं वरयस्व गच्छ च}

\uvacha{सावित्र्युवाच}



\fourlineindentedshloka
{ममात्मजं सत्यवतस्तथौरसं}
{भवेदुभाभ्यामिह यत्कुलोद्वहम्}
{शतं सुतानां बलवीर्यशालिना-}
{मिदं चतुर्थं वरयामि ते वरम्}

\uvacha{यम उवाच}


\fourlineindentedshloka
{शतं सुतानां बलवीर्यशालिनां}
{भविष्यति प्रीतिकरं तवाबले}
{परिश्रमस्ते न भवेन्नृपात्मजे}
{निवर्त दूरं हि पथस्त्वमागता}

\uvacha{सावित्र्युवाच}



\fourlineindentedshloka
{सतां सदा शाश्वतधर्मवृत्तिः}
{सन्तो न सीदन्ति न च व्यथन्ति}
{सतां सद्भिर्नाफलः सङ्गमोऽस्ति}
{सद्भ्यो भयन्नानुवर्तन्ति सन्तः}


\fourlineindentedshloka
{सन्तो हि सत्येन नयन्ति सूर्यं}
{सन्तो भूमिं तपसा धारयन्ति}
{सन्तो गतिर्भूतभव्यस्य राजन्}
{सतां मध्ये नावसीदन्ति सन्तः}


\twolineshloka
{आर्यजुष्टमिदं वृत्तमिति विज्ञाय शाश्वतम्}
{सन्तः परार्थं कुर्वाणा नावेक्षन्ति प्रतिक्रियाः}


\fourlineindentedshloka
{न च प्रसादः सत्पुरुषेषु मोघो}
{न चाप्यर्थो नश्यति नापि मानः}
{यस्मादेतन्नियतं सत्सु नित्यं}
{तस्मात्सन्तो रक्षितारो भवन्ति}

\uvacha{यम उवाच}


\fourlineindentedshloka
{यथा यथा भाषसि धर्मसंहितं}
{मनोनुकूलं सुपदं महार्थवत्}
{तथा तथा मे त्वयि भक्तिरुत्तमा}
{वरं वृणीष्वाप्रतिमं पतिव्रते}

\uvacha{सावित्र्युवाच}



\fourlineindentedshloka
{न तेऽपवर्गः सुकृताद्विना कृतस्-}
{तथा यथाऽन्येषु वरेषु मानद}
{वरं वृणे जीवतु सत्यवानयं}
{यथा मृता ह्येवमहं पतिं विना}


\fourlineindentedshloka
{न कामये भर्तृविनाकृता सुखं}
{न कामये भर्तृविनाकृता दिवम्}
{न कामये भर्तृविनाकृता श्रियं}
{न भर्तृहीना व्यवसामि जीवितुम्}


\fourlineindentedshloka
{वरातिसर्गः शतपुत्रता मम}
{त्वयैव दत्तो ह्रियते च मे पतिः}
{वरं वृणे जीवतु सत्यवानयं}
{तवैव सत्यं वचनं भविष्यति}

\uvacha{मार्कण्डेय उवाच}



\twolineshloka
{तथेत्युक्त्वा तु तं पाशं मुक्त्वा वैवस्वतो यमः}
{धर्मराजः प्रहृष्टात्मा सावित्रीमिदमब्रवीत्}


\twolineshloka
{एष भद्रे मया मुक्तो भर्ता ते कुलनन्दिनि}
{तोषितोऽहं त्वया साध्वि वाक्यैर्धर्मार्तसंहितैः}


\twolineshloka
{अरोगस्तव नेयश्च सिद्धार्थः स भविष्यति}
{चतुर्वर्षशतायुश्च त्वया सार्धमवाप्स्यति}


\twolineshloka
{इष्ट्वा यज्ञैश्च धर्मेण ख्यातिं लोके गमिष्यति}
{त्वयि पुत्रशतं चैव सत्यवाञ्जनयिष्यति}


\twolineshloka
{ते चापि सर्वे राजानः क्षत्रियाः पुत्रपौत्रिणः}
{ख्यातास्त्वन्नामधेयाश्च भविष्यन्तीह शाश्वताः}


\threelineshloka
{पितुश्च ते पुत्रशतं भविता तव मातरि}
{मालव्यां मालवा नाम शाश्वताः पुत्रपौत्रिणः}
{भ्रातरस्ते भविष्यन्ति क्षत्रियास्त्रिदशोपमाः}


\twolineshloka
{एवं तस्यै वरं दत्त्वा धर्मराजः प्रतापवान्}
{निवर्तयित्वा सावित्रीं स्वमेव भवनं ययौ}


\twolineshloka
{सावित्र्यपि यमे याते भर्तारं प्रतिलभ्य च}
{जगाम तत्र यत्रास्या भर्तुः शावं कलेवरम्}


\twolineshloka
{सा भूमौ प्रेक्ष्यभर्तारमुपसृत्योपगृह्य च}
{उत्सङ्गे शिर आरोप्य भूमावुपविवेश ह}


\twolineshloka
{संज्ञां चस पुनर्लब्ध्वा सावित्रीमभ्यभाषत}
{प्रोष्यागत इव प्रेम्णा पुनःपुनरुदीक्ष्यवै}


\twolineshloka
{सुचिरं बत सुप्तोऽस्मि किमर्थं नावबोधितः}
{क्व चासौ पुरुषः श्यामो योसौ मां सञ्चकर्षह}

\uvacha{सावित्र्युवाच}



\twolineshloka
{सुचिरं त्वम्प्रसुप्तोऽसि ममाह्के पुरुषर्षभ}
{गतः स भगवान्देवः प्रजासंयमनो यमः}


\twolineshloka
{विश्रान्तोऽसि महाभाग विनिद्रश्च नृपात्मज}
{यदि शक्यं समुत्तिष्ठ विगाढां पश्य शर्वरीम्}

\uvacha{मार्कण्डेय उवाच}



\twolineshloka
{उपलभ्यततः संज्ञां सुखसुप्त इवोत्थितः}
{दिशः सर्वा वनान्तांश्च निरीक्ष्योवाच सत्यवान्}


\twolineshloka
{फलाहारोऽस्मि निष्क्रान्तस््वया सह सुमध्यमे}
{ततः पाटयतः काष्ठं शरिसो मे रुजाऽभवत्}


\twolineshloka
{शिरोभितापसन्तप्तः स्थातुं चिरमशक्नुवन्}
{तवोत्सङ्गे प्रसुप्तोऽस्मि इति सर्वं स्मरे शुभे}


\twolineshloka
{त्वयोपगूढस्य च मे निद्रयाऽपहृतं मनः}
{ततोऽपश्यं तमो घोरं पुरुषं च महौजसम्}


\twolineshloka
{तद्यदि त्वं विजानासि किं तद्ब्रूहि सुमध्यमे}
{स्वप्नो मे यदिवा दृष्टो यदि वा सत्यमेव तत्}


\twolineshloka
{तमुवाचाथ सावित्री रजनी व्यवगाहते}
{श्वस्ते सर्वंयथावृत्तमाख्यास्यामि नृपात्मज}


\twolineshloka
{उत्तिष्ठोत्तिष्ठ भद्रं ते पितरौ पश्य सुव्रत}
{विगाढा रजनी चेयं निवृत्तश्च दिवाकरः}


\twolineshloka
{नक्तञ्चराश्चरन्त्येते हृष्टाः क्रूराभिभाषिणः}
{श्रूयन्ते पर्णशब्दाश्च मृगाणां चरतां वने}


\twolineshloka
{एता घोरं शिवा नादान्दिशं दक्षिणपश्चिमाम्}
{आस्थाय विरुवन्त्युग्राः कम्पयन्त्यो मनो मम}

\uvacha{सत्यवानुवाच}



\twolineshloka
{वनं प्रतिभयाकारं घनेन तमसा वृतम्}
{न विज्ञास्यसि पन्थानं गन्तुं चैव न शक्ष्यसि}

\uvacha{सावित्र्युवाच}



\twolineshloka
{अस्मिन्नद्य वने दग्धे शुष्कवृक्षः स्थितो ज्वलन्}
{वायुना धम्यमानोऽत्र दृश्यतेऽग्निः क्वचित् क्वचित्}


\twolineshloka
{ततोऽग्निमानयित्वेह ज्वालयिप्यामि सर्वतः}
{काष्ठानीमानि सन्तीह जहि सन्तापमात्मनः}


\twolineshloka
{यदि नोत्सहसे गन्तुं सरुजं त्वां हि लक्षये}
{न च ज्ञास्यसि पन्थानं तमसा संवृते वने}


\twolineshloka
{श्वः प्रभाते वने दृश्ये यास्यावोऽनुमते तव}
{वसावेह क्षपामेकां रुचितं यदि तेऽनघ}

\uvacha{सत्यवानुवाच}



\twolineshloka
{शिरोरुजा निवृत्ता मे स्वस्थान्यङ्गानि लक्षये}
{मातापितृभ्यामिच्छामि संयोगं त्वत्प्रसादजम्}


\twolineshloka
{न कदाचिद्विकाले हि गतपूर्वोहमाश्रमात्}
{अनागतायां सन्ध्यायां माता मे प्ररुणद्धि माम्}


\twolineshloka
{दिवाऽपिमयि निष्क्रान्ते सन्तप्येते गुरू मम}
{विचिनोति हि मां तातः सहैवाश्रमवासिभिः}


\twolineshloka
{मात्रा पित्रा च सुभृशं दुःखिताभ्यामहं पुरा}
{उपालब्धश्च बहुशश्चिरेणागच्छसीति हि}


\twolineshloka
{कात्ववस्था तयोरद्य मदर्थमिति चिन्तये}
{तयोरदृश्ये मयि च महद्दुःखं भविष्यति}


\twolineshloka
{पुरा मामूचतुश्चैव रात्रावस्रायमाणकौ}
{भृशं सुदुःखितौ वृद्धौ बहुशः प्रीतिसंयुतौ}


\twolineshloka
{त्वया हीनौ न जीवाव मुहूर्तमपि पुत्रक}
{यावद्धरिष्यसे पुत्र तावन्नौ जीवितं ध्रुवम्}


\twolineshloka
{वृद्धयोरन्धयोर्दृष्टिस्त्वयि वंशः प्रतिष्ठितः}
{त्वयि पिण्डश्च कीर्तिश्च सन्तानश्चावयोरिति}


\twolineshloka
{माता वृद्धा पिता वृद्धस्तयोर्यष्टिरहं किल}
{तौ रात्रौ मामपश्यन्तौ कामवस्थां गमिष्यतः}


\twolineshloka
{निद्रायाश्चाभ्यसूयामि यस्या हेतोः पिता मम}
{माता च संशयं प्राप्ता मत्कृतेऽनपकारिणी}


\twolineshloka
{अहं च संशयं प्राप्तः कृच्छ्रामापदमास्थितः}
{मातापितृभ्यां हि विना नाहं जीवितुमुत्सहे}


\twolineshloka
{व्यक्तमाकुलया बुद्ध्या प्रज्ञाचक्षुः पिता मम}
{एकैकमस्यां वेलायां पृच्छत्याश्रमवासिनम्}


\twolineshloka
{नात्मानमनुशोचामि यथाऽहं पितरं शुभे}
{भर्तारं चाप्यनुगतां मातरं भृशदुःखिताम्}


\twolineshloka
{मत्कृते न हि तावद्य सन्तापं परमेष्यतः}
{जीवन्तावनुजीवामि भर्तव्यौ तौ मयेति ह}


\threelineshloka
{तयोः प्रियं मे कर्तव्यमिति जीवामि चाप्यहम्}
{परमं दैवतं तौ मे पूजनीयौ सदा मया}
{तयोस्तु मे सदाऽस्त्येवं  व्रतमेतत्पुरातनम्}

\uvacha{मार्कण्डेय उवाच}



\twolineshloka
{एवमुक्त्वा स धर्मात्मा गुरुभक्तो गुरुप्रियः}
{उच्छ्रित्य बाहू दुःखार्तः सुस्वरं प्ररुरोद ह}


\twolineshloka
{ततोऽब्रवीत्तथा दृष्ट्वाभर्तारं शोककर्शितम्}
{प्रमृज्याश्रूणि पाणिभ्यां सावित्री धर्मचारिणी}


\twolineshloka
{यदि मेऽस्ति तपस्तप्तं यदि दत्तं हुतं यदि}
{श्वश्रूश्वशुरभर्तॄणां मम पुण्याऽस्तु शर्वरी}


\twolineshloka
{न स्मराम्युक्तपूर्वं वै स्वैरेष्वप्यनृतां गिरम्}
{तेन सत्येन तावद्य ध्रियेतां श्वशुरौ मम}

\uvacha{सत्यवानुवाच}



\twolineshloka
{कामये दर्शनं पित्रोर्याहि सावित्रि माचिरम्}
{अपिनाम गुरू तौ हि  पश्येयं ध्रियमाणकौ}


\twolineshloka
{पुरा मातुः पितुर्वाऽपियदि पश्यामि विप्रियम्}
{न जीविष्ये वरारोहे सत्येनात्मानमालभे}


\twolineshloka
{यदि धर्मे च ते बुद्धिर्मां चेज्जीवन्तमिच्छसि}
{मम प्रियं वा कर्तव्यं गच्छावाश्रममन्तिकात्}

\uvacha{मार्कण्डेय उवाच}



\twolineshloka
{सावित्री तत उत्थाय केशान् संयम्य भामिनी}
{पतिमुत्थापयामास बाहुभ्यां परिगृह्य वै}


\twolineshloka
{उत्ताय सत्यवांश्चापि प्रमृज्याङ्गानि पाणिना}
{सर्वा दिशः समालोक्य कठिने दृष्टिमादधे}


\twolineshloka
{तमुवाचाथसावित्री श्वः फलानि हरिष्यसि}
{योगक्षेमार्थमेतं ते नेष्यामि परशुं त्वहम्}


\twolineshloka
{कृत्त्वा कठिनभारं सा वृक्षशाखावलम्बिनम्}
{गृहीत्वा परशुं भर्तुः सकाशे पुनरागमत्}


\twolineshloka
{वामे स्कन्धे तु वामोरूर्भर्तुर्बाहुं निवेश्य च}
{दक्षिणेन परिष्वज्य जगाम गजगामिनी}

\uvacha{सत्यवानुवाच}



\twolineshloka
{अभ्यासगमनाद्भीरु पन्थानो विदिता मम}
{वृक्षान्तरालोकितया ज्योत्स्नया चापि लक्षये}


\twolineshloka
{आगतौ स्वः पथा येन फलान्यवचितानि च}
{यथागतं शुभे गच्छ पन्थानं मा विचारय}


\twolineshloka
{पलाशखण्डे चैतस्मिन्पन्था व्यावर्तते द्विधा}
{तस्योत्तरेण यः पन्थास्तेन गच्छ त्वरस्व च}


\twolineshloka
{स्वस्थोऽस्मि बलवानस्मि दिदृक्षुः पितरावुभौ}
{ब्रुवन्नेव त्वरायुक्तः सम्प्रायादाश्रमं प्रति}


॥इति श्रीमन्महाभारते अरण्यपर्वणि पतिव्रतामाहात्म्यपर्वणि अष्टनवत्यधिकद्विशततमोऽध्यायः॥२९८॥


\chapter{एकोनत्रिशततमोऽध्यायः॥२९९॥}

\uvacha{मार्कण्डेय उवाच}

\twolineshloka
{एतस्मिन्नेव काले तु द्युमत्सेनो महाबलः}
{लब्धचक्षुः प्रसन्नायां दृष्ट्यां सर्वं ददर्श ह}


\twolineshloka
{स सर्वानाश्रमान्गत्वा शैब्यया सह भार्यया}
{पुत्रहेतोः परामार्तिं जगाम भरतर्षभ}


\twolineshloka
{तावाश्रमान्नदीश्चैववनानि च सरांसि च}
{तस्यां निशि विचिन्वन्तौ दम्पती परिजग्मतुः}


\twolineshloka
{श्रुत्वा शब्दं तु यं कञ्चिदुन्मुखौ सुतशङ्कया}
{सावित्रीसहितोऽभ्येति सत्यवानित्यभाषताम्}


\twolineshloka
{भिन्नैश्च परुषैः पादैः सव्रणैः शोणितोक्षितैः}
{कुशकण्टकविद्धाङ्गावुनमत्ताविव धावतः}


\twolineshloka
{ततोऽभिसृत्य तैर्विप्रैः सर्वैराश्रमवासिभिः}
{परिवार्य समाश्वास्य तावानीतौ स्वमाश्रमम्}


\twolineshloka
{तत्रभार्यासहायः स वृतो वृद्धैस्तपोधनैः}
{आश्वासितोपि चित्रार्थैः पूर्वराजकथाश्रयैः}


\threelineshloka
{ततस्तौ पुनराश्वस्तौ वृद्धौ पुत्रदिदृक्षया}
{बाल्यवृत्तानि पुत्रस्य सावित्र्या दर्शनानि च}
{शोकं जग्मतुरन्योन्यं स्मरन्तौ भृशदुःखितौ}


\twolineshloka
{हापुत्र हासाध्वि वधु क्वासिक्वासीत्यरोदताम्}
{ब्राह्मणः सत्यवाक्येषामुवाचेदं तयोर्वचः}

\uvacha{सुवर्चा उवाच}



\twolineshloka
{यथास्य भार्या सावित्री तपसा च दमेन च}
{आचारेण च संयुक्ता तथा जीवति सत्यवान्}

\uvacha{गौतम उवाच}



\twolineshloka
{वेदाः साङ्गा मयाऽधीतास्तपो मे सञ्चितं महत्}
{कौमारब्रह्मचर्यं च गुरवोऽग्निश्च तोषिताः}


\twolineshloka
{समाहितेन चीर्णानि सर्वाण्येव व्रतानि मे}
{वायुभक्षोपवासश्च कृतोमे विधिवत्सदा}


\twolineshloka
{अनेन तपसा वेद्मि सर्वं परचिकीर्षितम्}
{सत्यमेतन्निबोधध्वं ध्रियते सत्यवानिति}

\uvacha{शिष्य उवाच}



\twolineshloka
{उपाध्यायस्य मे वक्राद्यथा वाक्यं विनिःसृतम्}
{नैव जातु भवेन्मिथ्या तथा जीवति सत्यवान्}

\uvacha{ऋषय ऊचुः}



\twolineshloka
{यथाऽस्य भार्या सावित्री सर्वैरेव सुलक्षणैः}
{अवैधव्यकरैर्युक्ता तथा जीवति सत्यवान्}

\uvacha{भारद्वाज उवाच}



\twolineshloka
{यथाऽस्य भार्या सावित्री तपसा च दमेन च}
{आचारेण च संयुक्ता तथा जीवति सत्यवान्}

\uvacha{दाल्भ्या रउवाच}



\twolineshloka
{यथा दृष्टिः प्रवृत्ता ते सावित्र्याश्च यथा व्रतम्}
{गताऽऽहारमकृत्वैव तथा जीवति सत्यवान्}

\uvacha{आपस्तम्ब उवाच}



\twolineshloka
{यथा वदन्ति शान्तायां दिशि वै मृगपक्षिणः}
{पार्थिवीं चैववृद्धिं ते तथा जीवति सत्यवान्}

\uvacha{धौम्य उवाच}



\twolineshloka
{सर्वैर्गुणैरुपेतस्ते यथा पुत्रो जनप्रियः}
{दीर्घायुर्लक्षणोपेतस्तथा जीवति सत्यवान्}

\uvacha{मार्कण्डेय उवाच}



\twolineshloka
{एवमाश्वासितस्तैस्तु सत्यवाग्भिस्तपस्विभिः}
{तांस्तान्विगणयन्सर्वांस्ततः स्थिर इवाभवत्}


\twolineshloka
{ततो मुहूर्तात्सावित्री भर्त्रा सत्यवता सह}
{आजगामाऽऽश्रमं रात्रौ प्रहृष्टा प्रविवेश ह}


\twolineshloka
{दृष्ट्वा चोत्पतिताः सर्वे हर्षं जग्मुश्च ते द्विजाः}
{कण्ठं माता पिता चास्य  समालिङ्ग्याभ्यरोदताम्}

\uvacha{ब्राह्मणा ऊचुः}



\twolineshloka
{पुत्रेण सङ्गतं त्वां तु चक्षुष्मन्तं निरीक्ष्य च}
{सर्वे वयं वै पृच्छामो वृद्धिं वै पृथिवीपते}


\twolineshloka
{समागमेन पुत्रस्य सावित्र्या दर्शनेन च}
{चक्षुषश्चात्मनो लाभात्रिभिर्दिष्ठ्या विवर्धसे}


\twolineshloka
{सर्वैरस्माभिरुक्तं यत्तथा तन्नात्र संशयः}
{भूयोभूयः समृद्धिस्ते क्षिप्रमेव भविष्यति}

\uvacha{मार्कण्डेय उवाच}



\twolineshloka
{ततोऽग्निं तत्र सञ्ज्वाल्य द्विजास्ते सर्व एव हि}
{उपासाञ्चक्रिरे पार्थ द्युमत्सेनं महीपतिम्}


\twolineshloka
{शैव्या च सत्यवांश्चैव सावित्री चैकतः स्थिताः}
{सर्वैस्तैरभ्यनुज्ञाता विशोका समुपाविशन्}


\twolineshloka
{ततो राज्ञा सहासीनाः सर्वे ते वनवासिनः}
{जातकौतूहलाः पार्थ पप्रच्छुर्नृपतेः सुतम्}


\twolineshloka
{प्रागेव नागतं कस्मात्सभार्येण त्वया विभो}
{विरात्रे चागतं कस्मात्कोनु बन्धस्तवाभवत्}


\twolineshloka
{सन्तापितः पिता माता वयं चैव नृपात्मज}
{कस्मादिति न जानीमस्तत्सर्वं वक्तुमर्हिसि}

\uvacha{सत्यवानुवाच}



\twolineshloka
{पित्राऽहमभ्यनुज्ञातः सावित्रीसहितो गतः}
{अथ मेऽभूच्छिरोदुःखं वने काष्ठानि भिन्दतः}


\twolineshloka
{सुप्तश्चाहं वेदनया चिरमित्युपलक्षये}
{तावत्कालं न च मया सुप्तपूर्वं कदाचन}


\twolineshloka
{सर्वेषामेव भवतां सन्तापो मा भवेदिति}
{अतो विरात्रागमनं नान्यदस्तीह कारणम्}

\uvacha{गौतम उवाच}



\twolineshloka
{अकस्माच्चक्षुषः प्राप्तिर्द्युमत्सेनस्य ते पितुः}
{नास्य त्वं कारणं वेत्सि सावित्री वक्तुमर्हति}


\twolineshloka
{श्रोतुमिच्छामि सावित्रि त्वं हि वेत्थ परावरम्}
{त्वां हि जानामि सावित्रि सावित्रीमिव तेजसा}


\twolineshloka
{त्वमत्र हेतुं जानीषे तस्मात्सत्यं निरुच्यताम्}
{रहस्यं यदि ते नास्ति किञ्चिदत्र वदस्व नः}

\uvacha{सावित्र्युवाच}



\twolineshloka
{एवमेतद्यथा वेत्थ सङ्कल्पो नान्यथा हि वः}
{न हि किञ्चिद्रहस्यं मे श्रूयतां तथ्यमेव यत्}


\twolineshloka
{मृत्युर्मे पत्युराख्यातो नारदेन महात्मना}
{स चाद्य दिवसः प्राप्तस्ततो नैनं जहाम्यहम्}


\twolineshloka
{सुप्तं चैनं यमः साक्षादुपागच्छत्सकिङ्करः}
{स एनमनयद्बद्ध्वा दिशं पितृनिषेविताम्}


\twolineshloka
{अस्तौषं तमहं देवं सत्येन वचसा विभुम्}
{पञ्च वै तेन मे दत्ता वराः शृणुत तान्मम}


\twolineshloka
{चक्षुषी च स्वराज्यञ्च द्वौ वरौ श्वशुरस्य मे}
{लब्धं पितुः पुत्रशतं पुत्राणां चात्मनः शतम्}


\twolineshloka
{चतुर्वर्षशतायुर्मे भर्ता लब्धश्च सत्यवान्}
{भर्तुर्हि जीवितार्थं तु मया चीर्णं त्विदं व्रतम्}



\twolineshloka
{एतत्सर्वं मयाऽऽख्यातं कारणं विस्तरेण वः}
{यथावृत्तं सुखोदर्कमिदं दुःखं महन्मम}

\uvacha{ऋषय ऊचुः}



\fourlineindentedshloka
{निमज्जमानं व्यसनैरभिद्रुतं}
{कुलं नरन्द्रस्य तमोमये ह्रदे}
{त्वया सुशीलव्रतपुण्यया कुलं}
{समुद्धृतं साध्वि पुनः कुलीनया}

\uvacha{मार्कण्डेय उवाच}



\fourlineindentedshloka
{तथा प्रशस्य ह्यभिपूज्य चैव}
{वरस्त्रियं तामृषयः समागताः}
{नरेन्द्रमामन्त्र्य सपुत्रमञ्जसा}
{शिवेन जग्मुर्मुदिताः स्वमालयम्}


॥इति श्रीमन्महाभारते अरण्यपर्वणि पतिव्रतामाहात्म्यपर्वणि एकोनत्रिशततमोऽध्यायः॥२९९॥


\chapter{त्रिशततमोऽध्यायः॥३००॥}

\uvacha{मार्कण्डेय उवाच}

\twolineshloka
{तस्यां रात्र्यां व्यतीतायामुदिते सूर्यमण्डले}
{कृतपौर्वाह्णिकाः सर्वे समेयुस्ते तपोधनाः}


\twolineshloka
{तदेव सर्वं सावित्र्या महाभाग्यं महर्षयः}
{द्युमत्सेनाय नातृप्यन्कथयन्तः पुनः पुनः}


\twolineshloka
{ततः प्रकृतयः सर्वाः साल्वेभ्योऽभ्यागता नृपम्}
{आचख्युर्निहतं चैव स्वेनामात्येन तं द्विषम्}


\twolineshloka
{तं मन्त्रिणा हतं प्रोच्य ससहायं सबान्धवम्}
{न्यवेदयन्यथावृत्तं विद्रुतं च द्विषद्बलम्}


\twolineshloka
{ऐकमत्यं च सर्वस्य जनस्य स्वं नृपं प्रति}
{सचक्षुर्वाऽप्यचक्षुर्वा स नो राजा भवत्विति}


\twolineshloka
{अनेन निश्चयेनेह वयं प्रस्थापिता नृप}
{प्राप्तानीमानि यानानि चतुरङ्गं च ते बलम्}


\twolineshloka
{प्रयाहि राजन्भद्रं ते घुष्टस्ते नगरे जयः}
{अध्यास्स्व चिररात्राय पितृपैतामहं पदम्}

\uvacha{मार्कण्डेय उवाच}



\twolineshloka
{चक्षुष्मन्तं च तं दृष्ट्वा राजानं वपुषाऽन्वितम्}
{मूर्ध्ना निपतिताः सर्वेविस्मयोत्फुल्ललोचनाः}


\twolineshloka
{ततोऽभिवाद्य तान्वृद्धान्द्विजानाश्रमवासिनः}
{तैश्चाभिपूजितः सर्वैः प्रययौ नगरं प्रति}


\twolineshloka
{शैव्या च सह सावित्र्या स्वास्तीर्णेन सुवर्चसा}
{नरयुक्तेन यानेन प्रययौ सेनया वृता}


\twolineshloka
{ततोऽभिषिषिचुः प्रीत्या द्युमत्सेनं पुरोहिताः}
{पुत्रं चास्य महात्मानं यौवराज्येऽभ्यषेचयन्}


\twolineshloka
{ततः कालेन महता सावित्र्याः कीर्तिवर्धनम्}
{तद्वै पुत्रशतं जज्ञे शूराणामनिवर्तिनाम्}


\twolineshloka
{भ्रातृणां सोदराणां च तथैवास्याभवच्छतम्}
{मद्राधिपस्याश्वपतेर्मालव्यां सुमहाबलम्}


\twolineshloka
{एवमात्मा पिता माता श्वश्रूः श्वशुर एव च}
{भर्तुः कुलं च सावित्र्या सर्वं कृच्छ्रात्समुद्धृतं}


\twolineshloka
{तथैवैषा हि कल्याणी द्रौपदी शीलसम्मता}
{तारयिष्यति वः सर्वान्सावित्रीव कुलाङ्गना}

\uvacha{वैशम्पायन उवाच}



\twolineshloka
{एवं स पाण्डवस्तेन अनुनीतो महात्मना}
{विशोको विज्वरो राजन्काम्यके न्यवसत्तदा}


\twolineshloka
{यश्चेदं शृणुयाद्भक्त्या सावित्र्याख्यानमुत्तमम्}
{स सुखी सर्वसिद्धार्थो न दुःखं प्राप्नुयान्नरः}


॥इति श्रीमन्महाभारते अरण्यपर्वणि पतिव्रतामाहात्म्यपर्वणि त्रिशततमोऽध्यायः॥३००॥

पतिव्रतामाहात्म्यपर्व समाप्तम्॥१९॥ 