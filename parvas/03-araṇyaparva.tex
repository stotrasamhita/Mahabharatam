\part{अरण्यपर्व}
\chapter{अध्यायः १}
\twolineshloka
{श्रीवेदव्यासाय नमः}
{}


\threelineshloka
{नारायणं नमस्कृत्य नरं चैव नरोत्तमम्}
{देवीं सरस्वतीं व्यासं ततो जयमुदीरयेत् ॥जनमेजय उवाच एवं द्यूतजिताः पार्थाः कोपिताश्च दुरात्मभिः}
{धार्तराष्ट्रैः सहामात्यैर्निकृत्या द्विजसत्तम}


\twolineshloka
{श्राविताः परुषा वाचः सृजन्तीर्वैरमुत्तमम्}
{किमकुर्वत कौरव्या मम पूर्वपितामहाः}


\twolineshloka
{कथं चैश्वर्यभूयिष्ठाः सहसा दुःखमेयुषः}
{वने विजह्रिरे पार्थाः शक्रप्रतिमतेजसः}


\twolineshloka
{के चैतानन्ववर्तन्त प्राप्तान्व्यसनमुत्तमम्}
{किमाचाराः किमाहाराः क्वच वासो महात्मनाम्}


\twolineshloka
{कथं द्वादश वर्षाणि वने तेषां महात्मनाम्}
{व्यतीयुर्ब्राह्मणश्रेष्ठ शूराणामनिवर्तिनाम्}


\twolineshloka
{कथं च राजपुत्री सा प्रवरा सर्वयोषिताम्}
{पतिव्रता महाभागा सततं प्रियवादिनी}


\twolineshloka
{वनवासमदुःखार्हा दारुणं प्रत्यपद्यत}
{एतदाचक्ष्व मे सर्वं विस्तरेण तपोधन}


\threelineshloka
{श्रोतुमिच्छामि तत्सर्वं भूरिद्रविणतेजसाम्}
{कथ्यमानं त्वया विप्र परं कौतूहलं हि मे ॥वैशंपायन उवाच}
{}


\twolineshloka
{एवं द्यूतजिताः पार्थाः कोयिताश्च दुरात्मभिः}
{धार्तराष्ट्रैः सहामात्यैर्निर्ययुर्गजसाह्वयात्}


\twolineshloka
{वर्धमानपुरद्वारादभिनिष्क्रम्य ते तदा}
{उदङ्मुखाः शस्त्रभृतः प्रययुः सह कृष्णया}


\twolineshloka
{इन्द्रसेनादयश्चैतान्भृत्याः परिचतुर्दश}
{रथैरनुययुः शीघ्रैः स्त्रिय आदाय सर्वशः}


\twolineshloka
{`ततस्ते पुरुषव्याघ्रा रथानास्थाय भारत}
{ददृशुर्जाह्नवीतीरे प्रमाणाख्यं महावटम् ॥'}


\fourlineindentedshloka
{व्रजतस्तान्विदित्वा तु पौराः शोकाभिपीडिताः}
{गर्हयन्तोऽसकृद्भीष्मविदुरद्रोणगौतमान्}
{ऊचुर्वै समयं कृत्वा समागम्य परस्परम् ॥पौरा ऊचुः}
{}


\threelineshloka
{नेदमस्ति कुलं सर्वं न वयं नच नो गृहाः}
{यत्रदुर्योधनः पापः सौबलेयेन पालितः}
{कर्णदुःशासनाद्यैश्च राज्यमेतच्चिकीर्षति}


\twolineshloka
{न तत्कुलं न चाचारो न धर्मोऽर्थः कुतः सुखम्}
{यत्र पापसहायोऽयं पापो राज्यं चिकीर्षति}


\twolineshloka
{दुर्योधनो गुरुद्वेषी त्यक्तधर्मः प्रियानृतः}
{अर्तलुब्धोऽभिमानी च नीचः प्रकृतिनिर्घृणः}


\twolineshloka
{नेयमस्ति मही कृत्स्ना यत्र दुर्योधनो नृपः}
{साधु गच्छामहे सर्वे यत्र गच्छन्ति पाण्डवाः}


\threelineshloka
{सानुक्रोशा महात्मानो विजितेन्द्रियशत्रवः}
{हीमन्तः कीर्तिमन्तश्च धर्माचारपरायणाः ॥वैशंपायन उवाच}
{}


\twolineshloka
{एवमुक्त्वाऽनुजग्मुस्ते पाण्डवांस्तान्समेत्य च}
{ऊचुः प्राञ्जलयः सर्वे तान्कुन्तीमाद्रिनन्दनान्}


\twolineshloka
{क्वगमिष्यथ भद्रंवस्त्यक्त्वाऽस्मान्दुःखभागिनः}
{वयमप्यनुयास्यामो यत्र यूयं गमिष्यथ}


\twolineshloka
{अधर्मेण जिताञ्श्रुत्वा युष्मांस्त्यक्तघृणैः परैः}
{उद्विग्नाः स्मो भृशं सर्वे नास्मान्हातुमिहार्हथ}


\twolineshloka
{भक्तानुरक्तान्सुहृदः सदा प्रियहिते रतान्}
{कुराजाघिष्ठिते राज्ये न विनश्येम सर्वशः}


\twolineshloka
{श्रूयतां चाभिधास्यामो गुणदोषान्नरर्षभाः}
{शुभाशुभाधिवासेन संसर्गः कुरुते यथा}


\twolineshloka
{वस्त्रमापस्तिलान्भूमिं गन्धो वासयते यथा}
{पुष्पाणामधिवासेन तथा संसर्गजा गुणाः}


\twolineshloka
{मोहजालस्य योनिर्हि मूढैरेव समागमः}
{अहन्यहनि धर्मस्य योनिः साधुसमागमः}


\twolineshloka
{तस्मात्प्राज्ञैश्च वृद्धैश्च सुस्वभावैस्तपस्विभिः}
{सद्भिश्च सह संसर्गः कार्यः शमपरायणैः}


\twolineshloka
{येषां त्रीण्यवदातानि विद्या योनिश्च कर्म च}
{ते सेव्यास्तैः समास्याहि शास्त्रेभ्योपि गरीयसी}


\twolineshloka
{निरारम्भा ह्यपि वयं पुण्यशीलेषु साधुषु}
{पुण्यमेवाप्नुयामेह पापं पापोपसेवनात्}


\twolineshloka
{असतां दर्शनात्स्पर्शात्संजल्पाच्च सहासवात्}
{धर्माचाराः प्रहीयन्ते सिद्ध्यन्ति च न मानवाः}


\twolineshloka
{बुद्धिश्च हीयते पुंसां नीचैः सह समागमात्}
{मध्यमैर्मध्यतां याति श्रेष्ठैः श्रेष्ठत्वमाप्नुयुः}


\threelineshloka
{अनीचैर्नाप्यविषयैर्नाधर्मिष्ठैर्विशेषतः}
{ये गुणाः कीर्तिता लोके धर्मकामार्थसंभवाः}
{लोकाचारेषु संभूता वेदोक्ताः शिष्टसंमताः}


\threelineshloka
{ते युष्मासु समस्ताश्च व्यस्ताश्चैवेह सद्गुणाः}
{इच्छामो गुणवन्मध्ये वस्तुं श्रेयोभिकाङ्क्षिणः ॥युधिष्ठिर उवाच}
{}


\twolineshloka
{धन्या वयं यदस्माकं स्नेहकारुण्ययन्त्रिताः}
{असतोऽपि गुणानाहुर्ब्राह्मणप्रमुखाः प्रजाः}


\twolineshloka
{तदहं भ्रातृसहितः सर्वान्विज्ञापयामि वः}
{नान्यथा तद्धि कर्तव्यमस्मत्स्नेहानुकम्पया}


\twolineshloka
{भीष्मः पितामहो राजा विदुरो जननी च मे}
{सुहृज्जनश्च प्रायो मे नगरे नागसाह्वये}


\twolineshloka
{ते त्वस्मद्धितकामार्थं पालनीयाः प्रयत्नतः}
{युष्माभिः सहिताः सर्वे शोकसंतापविह्वलाः}


\twolineshloka
{निवर्ततागता दूरं ममागमनकाङ्क्षिणः}
{स्वजने न्यासभूते मे कार्या स्नेहान्विता मतिः}


\threelineshloka
{एतद्धि मम कार्याणां परमं हृदि संस्थितम्}
{कृता तेन तु तुष्टिर्मे सत्कारश्च भविष्यति ॥वैशंपायन उवाच}
{}


\twolineshloka
{तथानुमन्त्रितास्तेन धर्मराजेन ताः प्रजाः}
{चक्रुरार्तस्वरं घोरं हा राजन्निति दुःखिताः}


\twolineshloka
{गुणान्पार्थस्य संस्मृत्य दुःखार्ताः परमातुराः}
{अकामाः संन्यवर्तन्त समागम्याथ पाण्डवान्}


\twolineshloka
{निवृत्तेषु तुं पौरेषु रथानास्थाय पाण्डवाः}
{आजग्मुर्जाह्नवीतीरे प्रमाणाख्यं महावटम्}


\twolineshloka
{ते तं दिषसशेषेण वटं गत्वा तु पाण्डवाः}
{ऊषुस्तां रजनीं वीराः संस्पृश्य सलिलं शुचि}


\twolineshloka
{उदकेनैव तां रात्रिमृषुस्ते दुःखकर्शिताः}
{अनुजग्मुश्च तत्रैतान्स्नेहात्केचिद्द्विजातयः}


\twolineshloka
{साग्रयोऽनग्नयश्चैव सशिष्यगणबान्धवाः}
{स तैः परिवृतो राजा शुशुभे ब्रह्मवादिभिः}


\twolineshloka
{तेषां प्रादुष्कृताग्नीनां मुहूर्ते रम्यदारुणे}
{ब्रह्मघोषपुरस्कारः संजल्पः समजायत}


\twolineshloka
{राजानं तु कुरुश्रेष्ठं ते हंसमधुरस्वराः}
{आश्वासयन्तो विप्राग्र्याः क्षयां सर्वां व्यनोदयत्}


\twolineshloka
{`राजा तु भ्रातृभिः सार्धं तथा सर्वैः सुहृद्गणैः}
{अशेत तां निशांराजन्दुःखशोकसमाहितः ॥'}


\chapter{अध्यायः २}
\twolineshloka
{वैशंपायन उवाच}
{}


\threelineshloka
{प्रभातायां तु शर्वर्यां तेषामक्लिष्टकर्मणाम्}
{वनं यियासतां विप्रास्तस्थुर्भिक्षाभुजोऽग्रतः}
{तानुवाच ततो राजा कुन्तीपुत्रो युधिष्ठिरः}


\twolineshloka
{वयं हि हृतसर्वस्वा हृतराज्या हृतश्रियः}
{फलमूलामिषाहारा वनं यास्याम दुःखिताः}


\twolineshloka
{वनं च दोषबहुलं बहुव्यालसरीसृपम्}
{परिक्लेशश्चवो मन्ये ध्रुवं तत्र भविष्यति}


\threelineshloka
{ब्राह्मणानां परिक्लेशो दैवतान्यपि सादयेत्}
{किंपुनर्मामितो विप्रा निवर्तध्वं यथेष्टतः ॥ब्राह्मणा ऊचुः}
{}


\twolineshloka
{गतिर्या भवतां राजंस्तां वयं गन्तुमुद्यताः}
{नार्हथास्मान्परित्यक्तुं भक्तान्सद्धर्मदर्शिन}


\threelineshloka
{स्नेहकर्माणि भक्तेषु दैवतान्यपि कुर्वते}
{विशेषतो ब्राह्मणेषु सदाचारावलम्बिषु ॥युधिष्ठिर उवाच}
{}


\twolineshloka
{ममापि परमा भक्तिर्ब्राह्मणेषु सदा द्विजाः}
{सहायविपरिभ्रंशस्त्वयं सादयतीव माम्}


\twolineshloka
{आहरेयुर्हि ये सर्वे फलमूलमृगांस्तथा}
{त इमे शोकजैर्दुःखैर्भ्रातरो मे विमोहिताः}


\threelineshloka
{द्रौपद्या विप्रकर्षेण राज्यापहरणेन च}
{दुःखार्दितानिमान्क्लेशैर्नाहं योक्तुमिहोत्सहे ॥ब्राह्मणा ऊचुः}
{}


\twolineshloka
{अस्मत्पोषणजा चिन्ता मा भूत्ते हृदि पार्थिव}
{स्वयमाहृत्य वन्यानि त्वानुयास्यामहे वयम्}


\threelineshloka
{अनुध्यानेन जप्येन विधास्यामः शिव तव}
{कथाभिश्चानुकूलाभिः सह रंस्यामहे वने ॥युधिष्ठिर उवाच}
{}


\twolineshloka
{एवमेतन्न संदेहो रमेयं ब्राह्मणैः सह}
{न्यूनभावात्तु पश्यामि प्रत्यादेशमिवात्मनः}


\threelineshloka
{कथं द्रक्ष्यामि वः सर्वान्स्वयमाहृत्य भोजिनः}
{मद्भक्त्या क्लिश्यतोऽनर्हान्धिक्पापान्धृतराष्ट्रजान् ॥वैशंपायन उवाच}
{}


% Check verse!
इत्युक्त्वा नृपतिः शोचन्निषसाद महीतले
\twolineshloka
{तमध्यात्मरतो विद्वाञ्शौनको नाम वै द्विजः}
{योगे साङ्ख्ये च कुशलो राजानमिदमब्रवीत्}


\twolineshloka
{शोकस्थानसहस्राणि हर्षस्थानशतानि च}
{दिवसेदिवसे मूढमाविशन्ति न पण्डितम्}


\twolineshloka
{न हि ज्ञानविरुद्धेषु बहुदोषेषु कर्मसु}
{श्रेयोघातिषु सज्जन्ते बुद्धिमन्तो भवद्विधाः}


\twolineshloka
{अष्टाङ्गां बुद्धिमाहुर्यां सर्वाश्रेयोभिघातिनीम्}
{श्रुतिस्मृतिसमायुक्तां राजन्सा त्वय्यवस्थिता}


\twolineshloka
{`शुश्रूषा श्रवणं चैव ग्रहणं धारणं तथा}
{ऊहापोहोऽर्थविज्ञानं तत्वज्ञानं च धीगुणाः ॥'}


\twolineshloka
{अर्थकृच्छ्रेषु दुर्गेषु व्यापत्सु स्वजनेष्वपि}
{शारीरमानसैर्दुःखैर्न सीदन्ति भवद्विधाः}


\twolineshloka
{श्रूयतां चाभिधास्यामि जनकेन यथा पुरा}
{आत्मव्यवस्थानकरागीताः श्लोका महात्मना}


\twolineshloka
{मनोदेहसमुत्थाभ्यां दुःखाभ्यामर्दितं जगत्}
{तयोर्व्याससमासाभ्यांशमोपायमिमं शृणु}


\twolineshloka
{व्याधेरनिष्टसंस्पर्शाच्छ्रमादिष्टविवर्जनात्}
{दुःखं चतुर्भिः शारीरं कारणैः संप्रवर्तते}


\twolineshloka
{तदा तत्प्रतिकाराच्च सततं चाविचिन्तनात्}
{आधिव्याधिप्रशमनं क्रियायोगबलेन तु}


\twolineshloka
{मतिमन्तो व्यथोपेताः शमं प्रागेव कुर्वते}
{मानसस्य प्रियाख्यानैः संभोगोपनयैर्नृणाम्}


\twolineshloka
{मानसेन हि दुःखेन शरीरमुपतप्यते}
{अयःपिण्डेन तप्तेन कुम्भसंस्थमिवोदकम्}


\twolineshloka
{मानसं शमयेत्तस्माज्ज्ञानेनाग्निमिवाम्बुना}
{प्रशन्ते मानसे दुःखे शारीरमुपशाम्यति}


\twolineshloka
{मनसो दुःखमूलं तु स्नेह इत्युपलभ्यते}
{स्नेहात्तु सज्जते जन्तुर्दुःखयोगमुपैति च}


\twolineshloka
{स्नेहमूलानि दुःखानि स्नेहजानि भयानि च}
{शोकहर्षौ तथायासः सर्वं स्नेहात्प्रवर्तते}


\twolineshloka
{स्नेहात्कारुण्यरागौ च प्रजास्वीर्ष्यादयस्तथा}
{अश्रेयस्कावुभावेतौ पूर्वस्तत्र गुरुः स्मृतः}


\twolineshloka
{कोटराग्निर्यथाऽशेषं समूलं पादपं दहेत्}
{धर्मार्थिनं तथाऽल्पोपि रागदोषो विनाशयेत्}


\twolineshloka
{विप्रयोगे न तु त्यागी दोषदर्शी समागमे}
{विरागं भजते जन्तुर्निर्वैरो निष्परिग्रहः}


\twolineshloka
{तस्मात्स्नेहं स्वपक्षेभ्यो मित्रेभ्यो धनसंचयात्}
{स्वशरीरसमुत्थं च ज्ञानेन विनिवर्तयेत्}


\twolineshloka
{ज्ञानान्वितेषु युक्तेषु शास्त्रज्ञेषु कृतात्मसु}
{न तेषु सज्जते स्नेहः पद्मपत्रेष्विवोदकम्}


\twolineshloka
{रागाभिभूतः पुरुषः कामेन परिकृष्यते}
{इच्छा संजायते तस्य ततस्तृष्णा विवर्धते}


\twolineshloka
{तृष्णा हि सर्वपापिष्ठा नित्योद्देगकरी नृणाम्}
{अधर्मबहुला चैव घोरा पापानुबन्धिनी}


\twolineshloka
{या दुस्त्यजा दुर्मतिभिर्या न जीर्यति जीर्यतः}
{योसौ प्राणान्तिको रोगस्तां तृष्णांत्यजतः सुखम्}


\twolineshloka
{अनाद्यन्ता तु सा तृष्णा अन्तर्देहगता नृणाम्}
{विनाशयति भूतानि अयोनिज इवानलः}


\twolineshloka
{यथैधः स्वसमुत्थेन वह्निना नाशमृच्छति}
{तथाऽकृतात्मा लोभेन सहजेन विनश्यति}


\threelineshloka
{राजतः सलिलादग्नेश्चोरतः स्वजनादपि}
{`अर्थिभ्यः कालतस्तस्मान्नित्यमर्थवतां भयम्'}
{भयमर्थवतां नित्यं मृत्योः प्राणभृतामिव}


\twolineshloka
{यथा ह्याभिषमाकाशे पक्षिभिः श्वापदैर्भुवि}
{भक्ष्यते सलिले मत्स्यैस्तथा सर्वेण वित्तवान्}


\threelineshloka
{अर्थ एव हि केषांचिदनर्थं भजते नृणाम्}
{अर्थश्रेयसि चासक्तो न श्रेयो विन्दते नरः}
{तस्मादर्थागमाः सर्वे मनोमोहविवर्धनाः}


\twolineshloka
{कार्पण्यं दर्पमानौ च भयमुद्वेग एव च}
{अर्थजानि विदुः प्राज्ञा दुःखान्येतानि देहिनाम्}


\twolineshloka
{अर्थस्योपार्जने दुःखमार्जितानां च रक्षणे}
{नाशे दुःखं व्यये दुःखं घ्नन्ति चैवार्थकारणात्}


\twolineshloka
{अर्था दुःखं परित्यक्तुं पालिताश्चैव शत्रवः}
{दुःखेन चाधिगम्यन्ते तेषां नाशं न चिन्तयेत्}


\threelineshloka
{असन्तोषपरा मूढाः संतोषं यान्ति पण्डिताः}
{अन्तो नास्ति पिपासायाः संतोषः परमं सुखम्}
{तस्मात्संतोषमेवेह परं पश्यन्ति पण्डितः}


\twolineshloka
{अनित्यं यौवनं रूपं जीवितं रत्नसंचयः}
{ऐश्वर्यं प्रियसंवासो गृद्ध्येत्तत्र न पण्डितः}


\threelineshloka
{त्यजेत स च यांस्तस्मात्तज्जान्क्लेशान्सहेत च}
{न हि संचयवान्कश्चिद्दृश्यते निरुपद्रवः}
{अतश्च धार्मिकैः पुम्भिरनीहार्थः प्रशस्यते}


\twolineshloka
{धर्मार्थं यस्य वित्तेहा वरं तस्य नरीहता}
{प्रक्षालनाद्धि पङ्कस्य श्रेयो ह्यस्पर्शनं नृणाम्}


\threelineshloka
{युधिष्ठिरैवमर्थेषु न स्पृहां कर्तुमर्हसि}
{धर्मेण यदि ते कार्यं विमुक्तेच्छो भवार्थतः ॥ 3-2-51xयुधिष्ठिर उवाच}
{}


\twolineshloka
{नार्थोपभोगलिप्सार्थमियमर्थेप्सुता मम}
{भरणार्थं तु विप्राणां ब्रह्मन्काङ्क्षे न लोभतः}


\twolineshloka
{कथं ह्यस्मद्विधो ब्रह्मन्वर्तमानो गृहाश्रमे}
{भरणं पालनं चापि न कुर्यादनुयायिनाम्}


\twolineshloka
{संविभागो हि भूतानां सर्वेषामेव दृश्यते}
{तथैवापचमानेभ्यः प्रदेयं गृहमेधिना}


\twolineshloka
{तृणानि भूमिरुदकं वाक्चतुर्थी च सूनृता}
{सतामेतानि गेहेषु नोच्छिद्यन्ते कदाचन}


\twolineshloka
{देयमार्तस्य शयनं स्थितश्रान्तस्य चासनम्}
{तृषितस्य च पानीयं क्षुधितस्य च भोजनम्}


\threelineshloka
{चक्षुर्दद्यान्मनो दद्याद्वाचं दद्याच्च सूनृताम्}
{उत्थाय चासनं दद्यादेष धर्मः सनातनः}
{प्रत्युत्थायाभिगमनं कुर्यान्न्यायेन चार्चनाम्}


\twolineshloka
{अग्निहोत्रमनड्वांश्च ज्ञातयोऽतिथिवान्धवाः}
{पुत्रा दाराश्च भृत्याश्च निर्दहेयुरपूजिताः}


\twolineshloka
{आत्मार्थं पाचयेन्नान्नं न वृथा घातयेत्पशून्}
{न चैकः स्वयमश्नीयाद्विधिवर्जं न निर्वपेत्}


\twolineshloka
{श्वभ्यश्च श्वपचेभ्यश्च वयोभ्यश्चावपेद्भुवि}
{वैश्वदेवं हि नामैतत्सायं प्रातश्च दीयते}


\twolineshloka
{विघसाशी भवेत्तस्मान्नित्यं चामृतभोजनः}
{विघसो भुक्तशेषं तु यज्ञशेषं तथाऽमृतम्}


\twolineshloka
{चक्षुर्दद्यान्मनो दद्याद्वाचं दद्याच्च सूनृताम्}
{अनुव्रजेदुपासीत स यज्ञः पञ्चदक्षिणः}


\twolineshloka
{यो दद्यादपरिक्लिष्टमन्नमद्वनि वर्तते}
{श्रान्तायादृष्टपूर्वाय तस्य पुण्यफलं महत्}


\threelineshloka
{एवं यो वर्तते वृत्तिं वर्तमानो गृहाश्रमे}
{तस्य धर्मं परंप्राहुः कथं वा विप्र मन्यसे ॥शौनक उवाच}
{}


\twolineshloka
{अहो बत महत्कष्टं विपरीतमिदं जगत्}
{येनापत्रपते साधुरसाधुस्तेन तुष्यति}


\twolineshloka
{शिश्नोदरकृतेऽप्राज्ञः करोति विषसं बहु}
{मोहरागवशाक्रान्त इन्द्रियार्थवशानुगः}


\twolineshloka
{ह्रियते बुध्यमानोपि नरो हारिभिरिन्द्रियैः}
{विमूढसंज्ञो दुष्टाश्वैरुद्धान्तैरिव सारथिः}


\twolineshloka
{षडिन्द्रियाणि विषयं समागच्छन्ति वै यदा}
{तद्रा प्रादुर्भवत्यषां पूर्वसंकल्पजं मनः}


\twolineshloka
{मनो यस्येन्द्रियस्येह विषयान्याति सेवितुम्}
{वस्यौत्सुक्यं संभवति प्रवृत्तिश्चोपजायते}


\twolineshloka
{ततः संकल्पवीर्येण कामेन विषयेषुभिः}
{विद्धः पतति लोभाग्नौज्योतिर्लोभात्पतङ्गवत्}


\twolineshloka
{ततो दारैर्विहारैश्च मोहितश्च यथेप्सया}
{महामोहमुखे मग्नो नात्मानमवबुध्यते}


\twolineshloka
{एवं पतति संसारे तासुतास्विह योनिषु}
{अविद्याकर्मतृष्णाभिर्भ्राम्यमाणोऽथ चक्रवत्}


\twolineshloka
{ब्रह्मादिषु तृणान्तेषु भूतेषु परिवर्तते}
{जले भुवि तथाऽऽकाशे जायमानः पुनःपुनः}


\twolineshloka
{अबुधानां गतिस्त्वेषा बुधानामपि मे शृणु}
{ये धर्मे श्रेयसि रता विमोक्षरतयो जनाः}


\twolineshloka
{तदिदं वेदवचनं कुरु कर्म त्यजेति च}
{तस्माद्धर्मानिमान्सर्वान्नाभिमानात्समाचरेत्}


\twolineshloka
{इज्याध्ययनदानानि तपः सत्यं क्षमा दमः}
{अलोभ इतिमार्गोऽयं धर्ममस्याष्टविधः स्मृतः}


\twolineshloka
{अत्र पूर्वश्चतुर्वर्गः पितृयाणपथे स्थितः}
{कर्तव्यमिति यत्कार्यं नाभिमानात्समाचरेत्}


\twolineshloka
{उत्तरो देवयानस्तु सद्भिराचरितः सदा}
{अष्टाङ्गेनैव मार्गेण विशुद्धात्मा समाचरेत्}


\twolineshloka
{सम्यक्संकल्पसंबन्धात्सम्यक्चेन्द्रियनिग्रहात्}
{सम्यग्द्व्रतविशेषाच्च सम्यक्च गुरुसेवनात्}


\twolineshloka
{सम्यगाहारयोगाच्च सम्यक्चाध्ययनागमात्}
{सम्यक्कर्मोपसंन्यासात्सम्यक्चित्तनिरोधनात्}


\twolineshloka
{एवं कर्माणि कुर्वन्ति संसारविजिगीषवः}
{रागद्वेषविनिर्मुक्ता ऐश्वर्यवशमागताः}


\twolineshloka
{रुद्राः साध्यास्तथाऽऽदित्या वसवोऽथ तथाश्विनौ}
{योगैश्वर्येण संयुक्ता धारयन्ति प्रजा इमाः}


\twolineshloka
{तथा त्वमपि कौन्तेय शममास्थाय पुष्कलम्}
{तपसा सिद्धिमन्विच्छ योगसिद्धिं च भारत}


\twolineshloka
{पितृमातृमयी सिद्धिः प्राप्ता कर्ममयी च ते}
{तपसा सिद्धिमन्विच्छ द्विजानां भरणाय वै}


\twolineshloka
{सिद्धा हि यद्यदिच्छन्ति कुर्वते तदनुग्रहम्}
{तस्मात्तपः समास्थाय कुरुष्वात्ममनोरथम्}


\chapter{अध्यायः ३}
\twolineshloka
{वैशंपायन उवाच}
{}


\threelineshloka
{शौनकेनैवमुक्तस्तु कुन्तीपुत्रो युधिष्ठिः}
{`प्रणम्य द्विजशार्दूलं पूज्यवाक्यं सुभाषितम्}
{'पुरोहितमुपागम्य भ्रातृमध्येऽब्रवीषिदम्}


\twolineshloka
{प्रस्थिताननुयातारो ब्राह्मणा वेदपारनाः}
{न चास्मि पोषणे शक्तो बहुदुःखसमन्वितः}


\threelineshloka
{परित्यक्तुं न शक्नोमि दानशक्तिश्च नास्ति मे}
{कथमत्र मया कार्यं तद्बूहि भगवन्मम ॥वैशंपायन उवाच}
{}


\twolineshloka
{मुहूर्तमिव स ध्यात्वा धर्मेणान्वीक्ष्य तां गतिम्}
{युधिष्ठिरमुवाचैदं धौम्यो धर्मभृतां वरः}


\twolineshloka
{पुरा सृष्टानि भूतानि पीड्यन्ते क्षुधया भृशम्}
{ततोऽनुकम्पंयां तेषां सविता स्वपिता यथा}


\twolineshloka
{गत्वोत्तसयणं तेजो रसानुद्धृत्य रश्मिभिः}
{दक्षिणायनमावृत्तो महीं वर्षति वारिणा}


\twolineshloka
{क्षेत्रभूते ततस्तस्मिन्नोषधीरोषधीयतिः}
{रवेस्तेजः समुद्धृत्य जनयामास वारिणा}


\twolineshloka
{निषिक्तश्चन्द्तेजोभिः सूयते जगतो रविः}
{ओषध्यः षड्रसा मेध्यास्तदन्नं प्राणिनां भुवि}


\twolineshloka
{एवं भानुमयं ह्यन्नं भूतानां प्राणधारणम्}
{नाथोऽयं सर्वभूतानां तस्मात्तं शरणं व्रज}


\twolineshloka
{राजानो हि महात्मानो योनिकर्मविशोधिताः}
{उद्धरन्ति प्रजाः सर्वास्तप आस्थाय पुष्कलम्}


\twolineshloka
{भौमेन कार्तवीर्येण वैन्येन नहुषेण च}
{तपोयोगसमाधिस्थैरुद्धृता ह्यापदः प्रजाः}


\threelineshloka
{तथा त्वमपि धर्मात्मन्कर्मणा च विशोधितः}
{तप आस्थाय धर्मेण द्विजातीन्भर भारत ॥वैशंपायन उवाच}
{}


\threelineshloka
{एवमुक्तस्तु धौम्येन तत्कालसदृशं वचः}
{ततस्त्वध्यापयामास मन्त्रं सर्वार्थसाधकम्}
{अष्टाक्षरं परं मन्त्रमार्तस्य सततं प्रियम्}


\twolineshloka
{[विप्रत्यागसमाधिस्थः संयतात्मा दृढव्रतः}
{धर्मराजो विशुद्धात्मा तप आतिष्ठदुत्तमम् ॥]}


\twolineshloka
{पुष्पोपहारैर्बलिभिरर्चयित्वा दिवाकरम्}
{सोऽवगाह्य जलं राजा देवस्याभिमुखोऽभवत्}


\threelineshloka
{गाङ्गेयं वार्युपस्पृश्य प्राणायामेन तस्थिवान्}
{[शुचिः प्रयतवाग्भूत्वा स्तोत्रमारब्धवांस्ततःयुधिष्ठिर उवाच}
{}


\twolineshloka
{त्वं भानो जगतश्चक्षुस्त्वमात्मा सर्वदेहिनाम्}
{त्वं योनिः सर्वभूतानां त्वमाचारः क्रियावतां}


\twolineshloka
{त्वं गतिः सर्वसाङ्ख्यानां योगिनां त्वं परायणम्}
{अनावृतार्गलद्वारं त्वं गतिस्त्वं मुमुक्षताम्}


\twolineshloka
{त्वया संधार्यते लोकस्त्वया लोकः प्रकाश्यते}
{त्वयापवित्रीक्रियते निर्व्याजं पाल्यते त्वया}


\twolineshloka
{त्वामुपस्थाय काले तु ब्राह्मणा वेदपारगाः}
{स्वशाखाविहितैर्मन्त्रैरर्चन्त्यृषिगणार्चित}


\twolineshloka
{तव दिव्यं रथं यान्तमनुयान्ति वरार्थिनः}
{सिद्धचारणगन्धर्वा यक्षगुह्यकपन्नगाः}


\twolineshloka
{त्रयस्त्रिंशच्च वै देवास्तथा वैमानिका गणाः}
{सोपेन्द्राः समहेन्द्राश्च त्वामिष्ट्वासिद्धिमागताः}


\twolineshloka
{उपयान्त्यर्चयित्वा तु त्वां वै प्राप्तमनोरथाः}
{दिव्यमन्दारमालाभिस्तूर्णं विद्याधरोत्तमाः}


\twolineshloka
{गुह्याः पितृगणाः सप्त ये दिव्या ये च मानुषाः}
{ते पूजयित्वा त्वामेव गच्छन्त्याशु प्रधानताम्}


\twolineshloka
{वसवो मरुतो रुद्रा ये च साध्या मरीचिपाः}
{वालखिल्यादयः सिद्धाः श्रेष्ठत्वं प्राणिनां गताः}


\twolineshloka
{सब्रह्मकेषु लोकेषु सप्तस्वप्यखिलेषु च}
{न तद्भूतमहं मन्ये यदर्कादतिरिच्यते}


\twolineshloka
{सन्ति चान्यानि सत्वानिवीर्यवन्ति महान्ति च}
{न तु तेषां तथा दीप्तिः प्रभावो वायथा तव}


\twolineshloka
{ज्योतीषित्वयि सर्वाणि त्वं सर्वज्योतिषां पतिः}
{त्वयिसत्त्वं च सत्त्वं च सर्वेभावाश्च सात्त्विकाः}


\twolineshloka
{त्वत्तेजसा कृतंचक्रं सुनाभं विश्वकर्मणा}
{देवारीणां मदो येन नाशितः शार्ङ्गधन्वना}


\twolineshloka
{त्वमादायांशुभिस्तेजो निदाघे सर्वदेहिनाम्}
{सर्वौषधिरसानां च पुनर्वर्षासु मुञ्चसि}


\twolineshloka
{तपन्त्यन्ते दहन्त्यन्ये गर्जन्त्यन्ये तथा घनाः}
{विद्योतन्ते प्रवर्षन्ति तव प्रावृषि रश्मयः}


\twolineshloka
{न तथा सुखयत्यग्निर्न प्रावारा न कम्बलाः}
{शीतवातार्दितं लोकं यथा तव मरीचयः}


\twolineshloka
{त्रयोदशद्वीपवतीं गोभिर्भासयसे महीम्}
{त्रयाणामपि लोकानां हितायैकः प्रवर्तसे}


\twolineshloka
{तव यद्युदयो न स्यादन्धं जगदिदं भवेत्}
{न च धर्मार्थकामेषु प्रवर्तेरन्मनीषिणः}


\twolineshloka
{आधानपशुबन्धेष्टिमन्त्रयज्ञतपःक्रियाः}
{त्वत्प्रसादादवाप्यन्ते ब्रह्मक्षत्रविशां गणैः}


\twolineshloka
{यदहर्ब्रह्मणः प्रोक्तं सहस्रयुगसंमितम्}
{तस्य त्वमादिरन्तश्च कालज्ञैः परिकीर्तितः}


\twolineshloka
{मनूनां मनुपुत्राणां जगतोऽमानवस्य च}
{मन्वन्तराणां सर्वेषामीश्वराणां त्वमीश्वरः}


\twolineshloka
{संहारकाले संप्राप्ते तव क्रोधविनिःसृतः}
{संवर्तकाग्निस्त्रैलोक्यं भस्मीकृत्यावतिष्ठते}


\twolineshloka
{त्वद्दीधितिसमुत्पन्ना नानावर्ण महाघनाः}
{सैरावताः साशनयः कुर्वन्त्याभूतसंप्लवम्}


\twolineshloka
{कृत्वा द्वादशधाऽऽत्मानं द्वादशादित्यतां गतः}
{संहृत्यैकार्णवं सर्वं त्वं शोषयसि रश्मिभिः}


\twolineshloka
{त्वामिन्द्रमाहुस्त्वं रुद्रस्त्वं विष्णुस्त्वं प्रजापतिः}
{त्वमग्निस्त्वं मनः सूक्ष्मं प्रभुस्त्वं ब्रह्मशाश्वतं}


\twolineshloka
{त्वं हंसः सविता भानुरंशुमाली वृषाकपिः}
{विवस्वान्मिहिरः पूषा मित्रो धर्मस्तथैव च}


\twolineshloka
{सहस्ररश्मिरादित्यस्तपनस्त्वं गवां पतिः}
{मार्तण्डोऽर्को रविः सूर्यः शरण्यो दिनकृत्तथा}


\twolineshloka
{दिवाकरः सप्तसप्तिर्धामकेशी विरोचनः}
{आशुगामी तमोघ्नश्च हरिताश्वश्च कीर्त्यसे}


\twolineshloka
{सप्तम्यागथवा षष्ठ्यां भक्त्या पूजां करोति यः}
{अनिर्विण्णोऽनहंकारी तं लक्ष्मीर्भजते नरम्}


\twolineshloka
{न तेषामापदः सन्ति नाधयो व्याधयस्तथा}
{ये तवानन्यमनसा कुर्वन्त्यर्चनवन्दनम्}


\twolineshloka
{सर्वरोगैर्विरहिताः सर्वपापाविवर्जिताः}
{त्वद्भावभक्ताः सुखिनो भवन्ति चिरजीविनः}


\twolineshloka
{त्वं ममापन्नकामस्य सर्वातिथ्यं चिकीर्षतः}
{अन्नमन्नपते दातुमभितः श्रद्धयाऽर्हसि}


\twolineshloka
{येच तेऽनुचराः सर्वे पादोपान्तं समाश्रिताः}
{माठरारुणदण्डाद्यास्तांस्तान्वन्देऽशनिक्षुभान्}


\threelineshloka
{क्षुभया सहिता मैत्री याश्चान्या भूतमातरः}
{ताश्च सर्वा नमस्यामि पान्तु मां शरणागतम् ॥]वैशंपायन उवाच}
{}


\fourlineindentedshloka
{कण्ठदघ्ने जले स्थित्वा मन्त्रैः स्तोत्रैश्च तोषितः}
{ततो दिवाकरः प्रीतो दर्शयामास पाण्डवम्}
{दीप्यमानः स्ववपुषा ज्वलन्निव हुताशनः ॥विवस्वानुवाच}
{}


\twolineshloka
{यत्तेऽभिलषितं किंचित्तत्त्वं सर्वमवाप्स्यसि}
{अहमन्नं प्रदास्यामि सप्त पञ्च च ते समाः}


\twolineshloka
{गृह्णीष्व पिठरं ताम्रं मया दत्तं नराधिप}
{यावद्वर्त्स्यति पाञ्चाली पात्रेणानेन सुव्रत}


\fourlineindentedshloka
{फलमूलामिषं शाकं संस्कृतं यन्महानसे}
{चतुर्विधं तदन्नाद्यमक्षय्यं ते भविष्यति}
{इतश्चतुर्दशे वर्षे भूयो राज्यमवाप्स्यसि ॥वैशंपायन उवाच}
{}


% Check verse!
एवमुक्त्वा तु भगवांस्तत्रैवान्तरधीयत
\twolineshloka
{लब्ध्वा वरं तु कौन्तेयो जलादुत्तीर्य धर्मवित्}
{जग्राह पादौ धौम्यस्य भ्रातॄंश्च परिषस्वजे}


\twolineshloka
{द्रौपद्या सह-संगम्य पश्यमानोऽपयात्प्राभुः}
{महानसे तदान्नं तु साधयामास पाण्डवः}


\twolineshloka
{संस्कृतं प्रसवं याति स्वल्पमन्नं चतुर्विधम्}
{अक्षय्यं वर्धते चान्नं तेनाभोजयत द्विजान्}


\twolineshloka
{भुक्तवत्सु च विप्रेषु भोजयित्वाऽनुजानपि}
{शेषं विघससंज्ञं तुपश्चाद्भुङ्क्ते युधिष्ठिरः}


\twolineshloka
{युधिष्ठिरं भोजयित्वा शेषमश्नाति पार्षती}
{[द्रौपद्यां भुज्यमानायां तदन्नं क्षयमेति च ॥]}


\twolineshloka
{एवं दिवाकरात्प्राप्य दिवाकरसमप्रभः}
{कामान्मनोभिलषितान्ब्राह्मणेभ्योऽददात्प्रभुः}


\twolineshloka
{पुरोहितपुरोगाश्च तिथिनक्षत्रपर्वसु}
{इज्यार्थे संप्रवर्तन्ते विधिमन्त्रप्रमाणतः}


\threelineshloka
{ततः कृतस्वस्त्ययना धौम्येन सह पाण्डवाः}
{द्विजसङ्घैः परिवृताः प्रययुः काम्यकं वनम् ॥`जनमेजय उवाच}
{}


\twolineshloka
{पुष्पोपहारबलिभिर्बहुशश्च यथाविधि}
{सर्वात्मभूतं संपूज्य यतप्राणो जितेन्द्रियः}


\twolineshloka
{स्तवेन केन विप्रर्षे स तु राजा युधिष्ठिरः}
{विप्रार्थमाराधितवान्सूर्यमद्भुतविक्रमम्}


\threelineshloka
{मयि स्नेहोऽस्ति चेद्ब्रह्मन्यद्यनुग्रहभागहम्}
{भगवन्नास्ति चेद्गुद्यं तच्च मे ब्रूहि सांप्रतम् ॥'वैशंपायन उवाच}
{}


\twolineshloka
{शृणुष्वांवहितो राजञ्शुचिर्भूत्वा समाहितः}
{क्षणं च कुरु राजेन्द्र गुह्यं वक्ष्यामि ते हितम्}


\twolineshloka
{धौम्येन तु यथाप्रोक्तं पार्थाय सुमहात्मने}
{नाम्नामष्टोत्तरं पुण्यं शतं तच्छृणु भूपते}


\twolineshloka
{सूर्योऽर्यभा भगस्त्वष्टा पूषाऽर्कः सविता रविः}
{गभस्तिमानजः कालो मृत्युर्धाता प्रभाकरः}


\twolineshloka
{पृथिव्यापश्च तेजश्च खं वायुश्च परायणम्}
{सोमो बृहस्पतिः शुक्रो बुधोऽङ्गारक एव च}


\twolineshloka
{इन्द्रो विवस्वान्दीप्तांशुः शुचिः शौरिः शनैश्चरः}
{ब्रह्मा विष्णुश्च रुद्रश्च स्कन्दो वैश्रवणो यमः}


\twolineshloka
{वैद्युतो जाठरश्चाग्निरैन्धनस्तेजसां पतिः}
{धर्मध्वजो वेदकर्ता वेगाङ्गो वेदवाहनः}


\twolineshloka
{कृतं त्रेता द्वापरश्च कलिः सर्वामराश्रयः}
{कला काष्ठा मुहूर्ताश्च पक्षा मासा ऋतुस्तथा}


\twolineshloka
{संवत्सरकरोऽश्वत्थः कालचक्रो विभावसुः}
{पुरुषः शाश्वतो योगी व्यक्ताव्यक्तः सनातनः}


\twolineshloka
{लोकाध्यक्षः सुराध्यक्षो विश्वकर्मा तमोनुदः}
{वरुणः सागरोंशश्च जीमूतो जीवनोऽरिहा}


\threelineshloka
{भूताश्रयो भूतपति- सर्वभूतनिषेवितः}
{`मणिः सुवर्णो भूतात्मा कामदः सर्वतोमुखः}
{'[स्रष्टा संवर्तको वह्निः सर्वस्यादिरलोलुपः}


\twolineshloka
{अनन्तः कपिलो भानुः कामदः सर्वतोमुखः}
{]जयो विशालो वरदः सर्वधातुनिषेचिता}


\twolineshloka
{मनः सुपर्णो भूतादिः शीघ्रगः प्राणधारकः}
{धन्वन्तरिर्धूमकेतुरादिदेवोऽदितेः सुतः}


\twolineshloka
{द्वादशात्माऽरविन्दाक्षः पिता माता पितामहः}
{स्वर्गद्वारं प्रजाद्वारं मोक्षद्वारं त्रिविष्टपम्}


\twolineshloka
{देवकर्ता प्रशान्तात्मा विश्वात्मा विश्वतोमुखः}
{चराचरात्मा सूक्ष्मात्मा मैत्रेण वपुषाऽन्वितः}


\twolineshloka
{एतद्वै कीर्तनीयस्य सूर्यस्यामिततेजसः}
{नाम्नामष्टशतं पुण्यं शक्रेणोक्तं महात्मना}


\twolineshloka
{शक्राच्च नारदः प्राप्तो धौम्यश्च तदन्तरम्}
{धौम्याद्युधिष्ठरः प्राप्य सर्वान्कमानवाप्तवान्}


\twolineshloka
{सुरगणपितृयक्षसेवितंह्यसुरनिशाचरसिद्धवन्दितम्}
{वरकनकहुताशनप्रभंत्वमपि मनस्यभिधेहि भास्करम्}


\twolineshloka
{सूर्योदये यः सुसमाहितः पठे-त्त पुत्रदारान्धनरत्नसंचयान्}
{लभेत जातिस्मरतां नरः सदाधृतिं च मेधां च स विन्दते पुमान्}


\twolineshloka
{[इमं स्तवं देववरस्य यो नरःप्रकीर्तयेच्छुचिसुमनाः समाहितः}
{स मुच्यते शोकदवाग्निसागरा-ल्लभेत कामान्मनसा यथेप्सितान्}


\twolineshloka
{[इमं स्तवं प्रयतमनाः समाधिनापछेदिहान्योऽपि वरं समर्थयन्}
{तत्त्स्य दद्याच्च रविर्मनीषितंतदाऽऽप्नुयाद्यद्यपि तत्सुदुर्लभम्}


\threelineshloka
{यश्चेदं धारयेन्नित्यं शृणुयाद्वाप्यभीक्ष्णशः}
{पुत्रार्थी लभते पुत्रं धनार्थी लभते धनम्}
{विद्यार्थी लभते विद्यां पुरुषोप्यथवा स्त्रियः}


\twolineshloka
{उभे संध्ये पठेन्नित्यं नारी वा पुरुषो यदि}
{आपदं प्राप्य मुच्येत बद्धो मुच्येत बन्धनात्}


\twolineshloka
{संग्रामे च जयेन्नित्यं विपुलं चाप्नुयाद्वसु}
{मुच्यते सर्वपापेभ्यः सूर्यलोकं स गच्छति ॥]}


\chapter{अध्यायः ४}
\twolineshloka
{वैशंपायन उवाच}
{}


\threelineshloka
{वनं प्रविष्टेष्वथ पाण्डवेषुप्रज्ञाचक्षुस्तप्यमानोऽम्बिकेयः}
{धर्मात्मानं विदुरमगाधबुद्धिं 3-4-1da सुखासीनो वाक्यमुवाच राजा ॥ धृतराष्ट्र उवाच}
{}


\twolineshloka
{प्रज्ञा च ते भार्गवस्येव शुद्धाधर्मं च त्वं परमं वेत्थ सूक्ष्मम्}
{समश्च त्वं संमतः कौरवाणांपथ्यं चैषां मम चैव ब्रवीहि}


\twolineshloka
{एवं गते विदुरयदद्य कार्यंपौराश्चेमे कथमस्मान्भजेरन्}
{ते चाप्यस्मान्नोद्धरेयुः समूला-न्न कामये तांश्च विनश्यमानान्}


\twolineshloka
{`सौबलेनैव पापेन दुर्योधनहितैषिणा}
{क्रूरमाचरितं क्षत्तर्न मे प्रियमनुष्ठितम्}


\twolineshloka
{तथैवाङ्गीकृते तत्र तद्भवान्वक्तुमर्हति}
{उत्तरं प्राप्तकालं च किमन्यन्मन्यते क्षमम्}


\threelineshloka
{नास्ति धर्मे सहायत्वमिति मे दीर्यते मनः}
{यत्र पाण्डुसुताः सर्वे क्लिश्यन्ति वानमागताः' ॥विदुर उवाच}
{}


\twolineshloka
{त्रिवर्गोऽयं धर्ममूलो नरेन्द्रराज्यं चेदं धर्ममूलं वदन्ति}
{धर्मे राजन्वर्तमानः स्वशक्त्यापुत्रान्सर्वान्पाहि कुन्तीसुतांश्च}


\twolineshloka
{स वै ध्रमो विप्रलब्धः सभायांपापात्मभिः सौबलेयप्रधानैः}
{आहूय कुन्तीसुतमक्षवत्यांपराजैषीत्सत्यसन्धं सुतस्ते}


\twolineshloka
{एतस्य ते दुष्प्रणीतस्य राज-न्द्वेषस्याहं परिपश्याम्युपायम्}
{यथा पुत्रस्तव कौरव्य पापा-न्मुक्तो लोके प्रतितिष्ठेत साधु}


\twolineshloka
{तद्वै सर्वं पाण्डुपुत्रा लभन्तांयत्तद्राजन्नतिसृष्टं त्वयाऽऽसीत्}
{एष धर्मः परमो यत्स्वकेनराजा तुष्येन्न परस्वेषु गृद्ध्येत्}


\twolineshloka
{[यशो ने नश्येज्ज्ञातिभेदश्च न स्या-द्धर्मो न स्यान्नैव चैवं कृते त्वाम्}
{]एतत्कार्ये तव सर्वप्रधानंतेषां तुष्टिः शकुनेश्चावमानः}


\twolineshloka
{एवं शेषं यदि पुत्रेषु ते स्या-देतद्राजंस्त्वरमाणः कुरुष्व}
{अथैतदेवं न करोषि राजन्ध्रुवं कुरूणां भविता विनाशः}


\twolineshloka
{नहि क्रुद्धो भीमसेनोऽर्जुनो वाशेषं कुर्याच्छात्रवाणामनीके}
{येषां योद्धा सव्यसाची कृतास्त्रोधनुर्येषां गाण्डिवं लोकसारम्}


\twolineshloka
{येषां भीमो बाहुशाली च योद्धातेषां लोके किंनु न प्राप्यमस्ति}
{उक्तं पूर्वं जातमात्रे सुते तेमया यत्ते हितमासीत्तदानीम्}


\twolineshloka
{पुत्रं त्यजेममहितं कुलस्यहितं परं न च तत्त्वं चकर्थ}
{इदानीं ते हितमुक्तं न चेत्त्व-मेवं कर्ता परितप्ताऽसि पश्चात्}


\twolineshloka
{यद्येतदेवमनुमन्ता सुतस्तेसंप्रीयमाणः पाण्डवैरेकराज्यम्}
{तापो न ते भविता प्रीतियोगा-त्त्वं चेन्न गृह्णासि सुतं सहायैः}


\twolineshloka
{दुर्योधनं त्वहितं वै निगृह्यपाण्डोः पुत्रं प्रकुरुष्वाधिपत्ये}
{`ध्रुवं विनाशस्तव पुत्रेण धीमन्सबन्धुवर्गेण सहैव राजभिः}


\twolineshloka
{चतुर्दशे चैव वर्षे नरेन्द्रकुलक्षयं प्राप्स्यसि राजसिंह}
{तस्मात्कुरुष्वाधिपत्ये नरेन्द्रयुधिष्ठिरं धर्मभृतां वरिष्ठम् ॥'}


\twolineshloka
{अजातशत्रुर्हि विमुक्तरागोधर्मोणेमां पृथिवीं शास्तु राजन्}
{ततो राजन्पार्थिवाः सर्व एववैश्या इवास्मानुपतिष्ठन्तु सद्यः}


\twolineshloka
{दुर्योधनः शकुनिः सूतपुत्रःप्रीत्या राजन्पाण्डुपुत्रान्भजन्तु}
{दुःशासनो याचतु भीमसेनंसभामध्ये द्रुपदस्यात्मजां च}


\threelineshloka
{युधिष्ठिरं त्वं परिसान्त्वयस्वराज्ये चैनं स्थापयस्वाभिपूज्य}
{त्वया पृष्टः किमहमन्यद्वदेय-मेतत्कृत्वा कृतकृत्योसि राजन् ॥धृतराष्ट्र उवाच}
{}


\twolineshloka
{एतद्वाक्यं विदुर यत्ते सभाया-मिह प्रोक्तं पाण्डवान्प्राप्य मां च}
{हितं तेषामहितं मामकाना-मेतत्सर्वं मम नावैदि चेतः}


\twolineshloka
{इदं त्विदानीं गत एव निश्चितंतेषामर्थे पाण्डवानां यदात्थ}
{तेनाद्य मन्ये नासि हितो ममेतिकथं हि पुत्रं पाण्डवार्थे त्यजेयम्}


\twolineshloka
{असंशयं तेऽपि ममैव पुत्रादुर्योधनस्तु मम देहात्प्रमूतः}
{स्वं वै देहं परहेतोस्त्यजेतिकोनु ब्रूयात्समतामन्ववेक्षन्}


\threelineshloka
{स मां जिह्मं विदुर सर्वं ब्रवीपिमन्युं तेऽहमधिकं धारयामि}
{यथेच्छकं गच्छ वा तिष्ठ वा त्वंसुसान्त्व्यमानाऽप्यसती स्त्री जहाति ॥वैशेपायन उवाच}
{}


\twolineshloka
{एतावदुक्त्वा धृतराष्ट्रोऽन्वपद्य-दन्तर्वेश्म सहसोत्थाय राजन्}
{नेदमस्तीत्यथ विदुरो भापमाप्पःसंप्राद्रवद्यत्र पार्था बभूवुः}


\chapter{अध्यायः ५}
\twolineshloka
{वैशंपायन उवाच}
{}


\twolineshloka
{पाण्डवास्तु वने वासमुद्दिश्य भरतर्पभाः}
{प्रययुर्जाह्नवीकूलात्कुरुक्षेत्रं सहानुगाः}


\twolineshloka
{सरस्वतीदृपद्वत्यौ यमुनां च निषेव्य ते}
{ययुर्वननैव वनं सततं पश्चिमां दिशम्}


\twolineshloka
{ततः सरस्वतीकूले समेषु मरुधन्वसु}
{काम्यकं नाम ददृशुर्वनं मुनिजनप्रियम्}


\twolineshloka
{तत्र ते न्यवसन्वीरा वने बहुमृगद्विजे}
{अन्वास्यमाना मुनिभिः सान्त्व्यमानाश्च भारत}


\twolineshloka
{विदुरस्त्वथ पाण्डूनां सदा दर्शनलालसः}
{जगामैकरथेनैव काम्यकं वनमृद्धिमत्}


\twolineshloka
{ततो गत्वा विदुरः काम्यकं त-च्छीघ्रैरश्वैर्वाहितः स्यन्दनेन}
{ददर्शासीनं धर्मात्मानं विविक्तेसार्धं द्रौपद्या भ्रातृभिर्ब्राह्मणैश्च}


\twolineshloka
{ततोऽपश्यद्विदुरं तूर्णमारा-दभ्यायान्तं सत्यसन्धः स राजा}
{अथाब्रवीद्धातरं भीमसेनंकिंनु क्षत्ता वक्ष्यतिनः समेत्य}


\twolineshloka
{कच्चिन्नायं वचनात्सौबलस्यसमाह्वातुं देवनायोपयाति}
{कच्चित्क्षुद्रः कुशली चाऽयुधानिजेष्यत्यस्माकं पुनरेवाक्षवत्याम्}


\threelineshloka
{समाहूतः केनचिदाहवायनाहं शक्तो भीमसेनापयातुम्}
{गाण्डीवे च संशयिते कथं नुराज्यप्राप्तिः संशयिता भवेन्नः ॥वैशंपायन उवाच}
{}


\twolineshloka
{तत उत्थाय विदुरं पाण्डवेयाःप्रत्यगृह्णन्नृपते सर्व एव}
{तैः सत्कृतः स च तानाजमीढोयथोचितं पाण्डुपुत्रान्समेयात्}


\threelineshloka
{समाश्वस्तं विदुरं ते नरर्षभा-स्ततोऽपृच्छन्नागमनाय हेतुम्}
{स चापि तेभ्यो विस्तरतः शशंसयथावृत्तं धृतराष्ट्रेऽम्बिकेये ॥विदुर उवाच}
{}


\twolineshloka
{अवोचन्मां धृतराष्ट्रोऽभिगुप्त-मजातशत्रो परिगृह्याभिपूज्य}
{एवं गते समतामभ्युपेत्यपथ्यं तेषां मम चैव ब्रवीहि}


\twolineshloka
{मयाप्युक्तं यत्क्षमं कौरवाणांहितं पथ्यं धृतराष्ट्रस्य चैव}
{तद्वै पथ्यं तन्मनो नाभ्युपैतिततश्चाहं क्षममन्यन्न मन्ये}


\twolineshloka
{परं श्रेयः पाण्डवेया मयोक्तंन मे तच्च शुतवानाम्बिकेयः}
{यथाऽऽतुरस्येव हि पथ्यमौपधंन रोचनते स्मास्य तदुच्यमानम्}


\twolineshloka
{न श्रेयसे यततेऽजातशत्रोस्त्री श्रोत्रियस्येव गृहे प्रदुष्टा}
{ध्रुवं न रोचेद्भरतर्षभस्यपतिः कुमार्या इव षष्टिवर्षः}


\twolineshloka
{ध्रुवं विनाशो नृप कौरवाणांन वै श्रेयो धृतराष्ट्रः परैति}
{यथा च पर्णे पुष्करस्यावसिक्तंजलं न तिष्ठेत्पथ्यमुक्तं तथाऽस्मिन्}


\twolineshloka
{ततः क्रुद्धो धृतराष्ट्रोऽब्रवीन्मांयस्मिञ्श्रद्धा तव तत्र प्रयाहि}
{नाहं भूयः कामये त्वां सहायंमहीमिमां पालयितुं पुरं वा}


\twolineshloka
{सोहं त्यक्तो धृतराष्ट्रेण राज्ञाप्रशासितुं त्वामुपयातो नरेन्द्र}
{तद्वै सर्वं यन्मयोक्तं सभायांतद्धार्यतां यत्प्रवक्ष्यामि भूयः}


\twolineshloka
{क्लेशैस्तीर्व्रर्युज्यमानः सपत्नैःक्षमां कुर्वन्कालमुपासते यः}
{संवर्धयन्स्तोकमिवाग्रिमात्मवान्स वै भुङ्क्ते पृथिवीमेक एव}


\twolineshloka
{यस्याविभक्तं वसु राजन्सहायै-स्तस्य दुःखस्यांशभाजः सहायाः}
{सहायानामेष संग्रगणेऽभ्युपायःसहायाप्तौ पृथिवीप्राप्तिमाहुः}


\threelineshloka
{सत्यं श्रेष्टं पाण्डवा निष्प्रलापंतुल्यं चान्नं सह भोज्यं सहायैः}
{आत्मा चैषामग्रतो नातिवर्तेदेवं वृत्तिर्वर्धते भूमिपाल ॥युधिष्ठिर उवाच}
{}


\twolineshloka
{एवं करिष्यामि यथा ब्रवीपिपरां बुद्धिमुपगम्याप्रमत्तः}
{यच्चाप्यन्यद्देशकालोपपन्नंतद्वै वाच्यं तत्करिष्यामि कृत्स्नम्}


\chapter{अध्यायः ६}
\twolineshloka
{वैशंपायन उवाच}
{}


\twolineshloka
{गते तु विदुरे राजन्नाश्रमं पाण्डवान्प्रति}
{धृतराष्ट्रो महाप्राज्ञः पर्यतप्यत दुर्मनाः}


\twolineshloka
{[विदुरस्य प्रभावं च सन्धिविग्रहकारितम्}
{विवृद्धिं च परां मत्वा पाण्डवानां भविष्यति ॥]}


\twolineshloka
{स सभाद्वारमागम्य विदुरस्मारमोहितः}
{समक्षं पार्थिवेनद्राणां विसंज्ञः प्रापतद्भुवि}


\twolineshloka
{स तु लब्ध्वा चिरात्संज्ञां समुत्थाय महीतलात्}
{समीपोपस्थितं राजा संजयं वाक्यमब्रवीत्}


\twolineshloka
{भ्राता मम सुहृच्चैव साक्षाद्धर्म इवापरः}
{तस्य स्मृत्याऽद्य सुभृशं हृदयं दीर्यतीवमे}


\twolineshloka
{तमानय स्वधर्मज्ञं मम भ्रातरमाशु वै}
{इति ब्रुवन्स नृपतिः कृपणं पर्यदेवयत्}


\twolineshloka
{पश्चात्तापाभिसंतप्तो विदुरस्मारमोहितः}
{भ्रातृश्नेहादिदं राजा संजयं वाक्यमब्रवीत्}


\twolineshloka
{गच्छ संजय जानीहि भ्रातरं विदुरं मम}
{यदि जीवति रोषेण मया पापेन निर्धुतः}


\twolineshloka
{न हि तेन मम भ्रात्रा सुसूक्ष्ममपि किंचन}
{व्यलीकं कृतपूर्वं वै प्राज्ञेनामितबुद्धिना}


\twolineshloka
{स व्यलीकं कथं प्राप्तो मत्तः परमबुद्धिमान्}
{न जह्माज्जीवितं प्राज्ञ तं गच्छानय संजय}


\twolineshloka
{तस्य तद्वचनं श्रुत्वा राज्ञस्तमनुमान्य च}
{संजयो बाढमित्युक्त्वा प्राद्रवत्काम्यकं प्रति}


\twolineshloka
{सोऽचिरेण समासाद्य तद्वनं यत्र पाण्डवाः}
{रौरवाजिनसंवीतं ददर्शाथ युधिष्ठिरम्}


\twolineshloka
{विदुरेण सहासीनं ब्राह्मणैश्च सहस्रशः}
{भ्रातृभिश्चाभिसंगुप्तं देवैरिव पुरंदरम्}


\twolineshloka
{युधिष्ठिरमुपागम्य पूजयामास संजयः}
{भीमार्जुनयमाश्चापि तद्युक्तं प्रतिपेदिरे}


\threelineshloka
{राज्ञा पृष्टः स कुशलं सुखासीनश्च संजयः}
{शशंसागमने हेतुमिदं चैवाब्रवीद्वचः ॥संजय उवाच}
{}


\twolineshloka
{राजा स्मरति ते क्षत्तर्धृतराष्ट्रोऽम्बिकासुतः}
{तं पश्य गत्वा त्वं क्षिप्रं संजीवय च पार्थिवम्}


\twolineshloka
{सोऽनुमान्य नरश्रेष्ठान्पाण्डवान्कुरुनन्दनान्}
{नियोगाद्राजसिंहस्य गन्तुमर्हसि सत्तम्}


\twolineshloka
{एवमुक्तस्तु विदुरो धीमान्स्वजनवत्सलः}
{युधिष्ठिरस्यानुमते पुनरायाद्गजाह्वयम्}


\twolineshloka
{तमब्रवीन्महातेजा धृतराष्ट्रोऽम्बिकासुतः}
{दिष्ट्या प्राप्तोसि धर्मज्ञ दिष्ट्या स्मरसि मेऽनघ}


\threelineshloka
{अद्य चाहं दिवारात्रौ त्वत्कृते भरतर्षभ}
{प्रजागरे प्रपश्यामि विचित्रं देहमात्मनः ॥वैशंपायन उवाच}
{}


\threelineshloka
{सोङ्कमानीय विदुरं मूर्धन्याघ्राय चैव ह}
{क्षम्यतामिति चोवाचयदुक्तोसि मयाऽनघ ॥विदुर उवाच}
{}


\twolineshloka
{क्षान्तमेव मया राजन्गुरुर्मे परमो भवान्}
{तथाऽहमागतः शीघ्रं त्वद्दर्शनपरायणः}


\twolineshloka
{भवन्ति हि नरव्याघ्र पुरुषा धर्मचेतनाः}
{दीनानुकम्पिनो राजन्नात्र कार्या विचारणा}


\threelineshloka
{पाण्डोः पुत्रा यादृशास्ते तादृशा मे सुतास्तव}
{दीना इतीव मे बुद्धिरभिपन्नाऽद्य तान्प्रति ॥वैशंपायन उवाच}
{}


\twolineshloka
{अन्योन्यमनुनीयैवं भ्रातरौ द्वौ महाद्युती}
{विदुरो धृतराष्ट्रश्च लेभाते परमां मुदम्}


\chapter{अध्यायः ७}
\twolineshloka
{वैशंपायन उवाच}
{}


\twolineshloka
{श्रुत्वा च विदुरं प्राप्तं राज्ञा च परिसान्त्वितम्}
{धृताराष्ट्रात्मजो राजा पर्यतप्यत दुर्मतिः}


\twolineshloka
{स सौबलेयमानाय्य कर्णदुःशासनौ तथा}
{अब्रवीद्वचनं राजा प्रविश्याबुद्धिजं तमः}


\twolineshloka
{एष प्रत्यागतो मन्त्री धृतराष्ट्रस्य संमतः}
{विदुरः पाण्डुपुत्राणां सुहृद्विद्वान्हिते रतः}


\twolineshloka
{यावदस्य पुनर्बुद्धिं विदुरो नापकर्षति}
{पाण्डवानयने तावन्मन्त्रयध्वं हितं मम}


\twolineshloka
{अथ पश्याम्यहं पार्थान्प्राप्तानिह कथंचन}
{पुनः शोषं गमिष्यामि निरसुर्निष्परिग्रहः}


\threelineshloka
{विषमुद्बन्धनं चैव शस्त्रमग्निप्रवेशनम्}
{करिष्ये न हि तानृद्धान्पुनर्द्रष्टुमिहोत्सहे ॥सुकुनिरुवाच}
{}


\twolineshloka
{किं बालिशमतिं राजन्नास्थितोसि विशांपते}
{गतास्ते समयं कृत्वा नैतदेवं भविष्यति}


\twolineshloka
{सत्यवाक्येः स्विताः सर्वे पाण्डवा भरतषभ}
{पितुस्ते वचनं तात न ग्रहीष्यन्ति कर्हिचिति}


\twolineshloka
{अथवा ते ग्रहीष्यन्ति पुनरेष्यन्ति वा पुरम्}
{निरस्य समयं सर्वे पणोऽस्माकं भविष्यति}


\fourlineindentedshloka
{सर्वे भवामो मध्यस्था राज्ञश्छन्दानुवर्तिनः}
{छिद्रं बहु प्रपश्यन्तः पाण्डवानां सुसंवृताः}
{दुःशासन उवाच}
{}


\threelineshloka
{एवमेतन्महाप्राज्ञ यथा वदसि मातुल}
{नित्यं हि मे कथयतस्तव बुद्धिर्विरोचते ॥कर्ण उवाच}
{}


\twolineshloka
{काममीक्षामहे सर्वे दुर्योधन तवेप्सितम्}
{ऐकमत्यं हि नो राजन्सर्वेषामेव लक्षये}


\threelineshloka
{[नागमिष्यन्ति ते धीरा अकृत्वा कालसंविदम्}
{आगमिष्यन्ति चेन्मोहात्पुनर्द्यूतेन ताञ्जय ॥]वैशंपायन उवाच}
{}


\twolineshloka
{एवमुक्तस्तु कर्णेन राजा दुर्योधनस्तदा}
{नातिहृष्टमनाः क्षिप्रमभवत्स पराङ्मुखः}


\twolineshloka
{उपलभ्य ततः कर्णो विवृत्य नयने शुभे}
{रोषाद्दुःशासनं चैव सौबलं च तमेव च}


\twolineshloka
{उवाच परमक्रुद्ध उद्यम्यात्मानमात्मना}
{अथो मम मतं यत्तु तन्निबोधत भूमिपाः}


\twolineshloka
{प्रियं सर्वे करिष्यामो राज्ञः किं करवामहे}
{न चास्य शक्नुमः स्थातुं प्रिये सर्वे ह्यतन्द्रिताः}


\twolineshloka
{वयं तु शस्त्राण्यादाय रथानास्थाय दंशिताः}
{गच्छामः सहिता हन्तुं पाण्डवान्वनगोचरान्}


\twolineshloka
{तेषु सर्वेषु शान्तेषु गतेष्वविदितां गतिम्}
{निर्विवादा भविष्यन्ति धार्तराष्ट्रास्तथा वयम्}


\twolineshloka
{यावदेव परिद्यूना यावच्छोकपरायणाः}
{यावन्मित्रविहीनाश्च तावद्गच्छाम माचिरम्}


\twolineshloka
{तस्यतद्वचनं श्रुत्वा पूजयन्तः पुनः पुनः}
{प्रहृष्टमनसः सर्वे प्रत्यूचुः सूतजं तदा}


\twolineshloka
{एवमुक्त्वा सुसंरब्धा रथैः सर्वे पृथक्पृथक्}
{निर्ययुः पाण्डवान्हन्तुं सहिताः कृतनिश्चयाः}


\twolineshloka
{तान्प्रस्थितान्परिज्ञाय कृष्णद्वैपायनः प्रभुः}
{आजगाम विशुद्धात्मा दृष्ट्वा दिव्येन चक्षुषा}


\twolineshloka
{प्रतिषिध्याथ तान्सर्वान्भगवाँल्लोकपूजितः}
{प्रज्ञाचक्षुषमासीनमुवाचाभ्येत्य सत्वरम्}


\chapter{अध्यायः ८}
\twolineshloka
{व्यास उवाच}
{}


\twolineshloka
{धतराष्ट्र महाप्राज्ञ निबोध वचनं मम}
{वक्ष्यामि त्वां कौरवाणां सर्वेषां हितमुत्तमम्}


\twolineshloka
{न मे प्रियं महाबाहो यद्गताः पाण्डवा वनम्}
{निकृत्त्या निकृताश्चैव दुर्योधनपुरोगमैः}


\twolineshloka
{ते स्मरन्तः परिक्लेशान्वर्षे पूर्णे त्रयोदशे}
{विमोक्ष्यन्ति विषं क्रुद्धाः कौरवेयेषु भारत}


\twolineshloka
{तदयं किंनु पापात्मा तव पुत्रः सुमन्दधीः}
{पाण्डवान्नित्यसंक्रुद्धो राज्यहेतोर्जिघांसति}


\twolineshloka
{वार्यतां साध्वयं मूढः शमं गच्छतु ते सुतः}
{वनस्थांस्तानयं हन्तुमिच्छन्प्राणान्विमोक्ष्यति}


\twolineshloka
{यथाऽऽह विदुरः प्राज्ञो यथा भीष्मो यथा वयम्}
{यथा कृपश्च द्रोणश्च तथा साधु विधीयताम्}


\twolineshloka
{विग्रहो हि महाप्राज्ञ स्वजनेन विगर्हितः}
{अधर्म्यमयशस्यं च मा राजन्प्रतिपद्यथाः}


\twolineshloka
{समीक्षा यादृशी ह्मस्य पाण्डवान्प्रति भारत}
{उपेक्ष्यमाणा सा राजन्महान्तमनयं व्रजेत्}


\twolineshloka
{अथवाऽयं सुमन्दात्मा वनं गच्छतु ते सुतः}
{पाण्डवैः सहितो राजन्नेक एवासहायवान्}


\twolineshloka
{ततः संसर्गजः स्नेहः पुत्रस्य तव पाण्डवैः}
{यदि स्यात्कृतकृत्यस्त्वं भवेथाः पुरषर्षभ}


\twolineshloka
{अथवा जायमानस्य यच्छीलमनुजायते}
{श्रूयते तन्महाराज नामृतस्यापसर्पति}


\twolineshloka
{कथं वा मन्यते भीष्मो द्रोणोऽथ विदुरोपिवा}
{भवान्वाऽत्र परं कार्यं पुरा चार्थो निवर्तते}


\chapter{अध्यायः ९}
\twolineshloka
{धृतराष्ट्र उवाच}
{}


\twolineshloka
{भगवन्नाहमप्येतद्रोचये द्यूतसंभवम्}
{मन्ये तद्विधिनाऽऽकृष्य कारितोस्मीति वै मुने}


\twolineshloka
{नैतद्रोचयते भीष्मो न द्रोणो विदुरो न च}
{गान्धार्या नेष्यते द्यूतं तत्र मोहात्प्रवर्तितम्}


\threelineshloka
{परित्यक्तुं न शक्नोमि दुर्योधनमचेतनम्}
{पुत्रस्नेहेन भगवञ्जानन्नपि यतव्रत ॥व्यास उवाच}
{}


\twolineshloka
{वैचित्रवीर्य नृपते सत्यमाह यथा भवान्}
{दृढं विझः परं पुत्रं परं पुत्रान्न विद्यते}


\threelineshloka
{इन्द्रोप्यश्रुनिपातेन सुरभ्या प्रतिबोधितः}
{अन्यैः समृद्धैरप्यर्थैर्न सुतानमन्यते परम्}
{}


\twolineshloka
{अत्र ते वर्तयिष्यामि गहदाख्यानमुत्तमम्}
{सुरभ्याश्चैव संवादमिन्द्रस्य च विशांपते}


\threelineshloka
{त्रिविष्टपगता राजन्सुरभी प्रारुदत्किल}
{गवां माता पुरा तात तामिन्द्रोऽन्वकृपायत ॥इन्द्र उवाच}
{}


\threelineshloka
{किमिदं रोदिषि शुभे कच्चित्क्षेमं दिवौकसाम्}
{मनुष्येष्वथवा गोषु नैतदल्पं भविष्यति ॥सुरभिरुवाच}
{}


\twolineshloka
{विनिपातो न वः कश्चिद्दृश्यते त्रिदशाधिप}
{अहं तु पुत्रं शोचामि तेन रोदिमि कौशिक}


\threelineshloka
{पश्यैनं कर्षकं क्षुद्रं दुर्बलं मम पुत्रकम्}
{प्रतोदेनाभिनिघ्नन्तं लाङ्गलेन च पीडितम्}
{निषीदमानं सोत्कण्ठं वध्यमानं सुराधिप}


\twolineshloka
{`एनं दृष्ट्वा भृशं श्रान्तं वध्यमानं सुराधिप}
{'कृपाविष्टाऽस्मि देवेन्द्र मनश्चोद्वेपते मम}


\threelineshloka
{एकस्तत्रबलोपेतो धुरमुद्वहतेऽधिकाम्}
{अपरोप्यबलप्राणः कृशो धमनिसंततः}
{कृच्छ्रादुद्वहते भारं तं वै शोचामि वासव}


\twolineshloka
{वध्यमानः प्रतोदेन तुद्यमानः पुनः पुनः}
{नैव शक्नोति तं भारमुद्वोढुं पश्य वासव}


\threelineshloka
{ततोऽहं तस्य शोकार्ता विरौमि भृशदुःखिता}
{अश्रूण्यावर्तयन्ती च नेत्राभ्यां करुणायती ॥शक्र उवाच}
{}


\threelineshloka
{तव पुत्रसहस्रेषु पीड्यमानेषु शोभने}
{किं कृपायितवत्यत्र पुत्र एको निपीड्यते ॥सुरभिरुवाच}
{}


\threelineshloka
{यदि पुत्रसहस्राणि सर्वत्रसमतैव मे}
{दीनस्य तु सतः शक्र पुत्रस्याभ्यधिकाकृपा ॥व्यास उवाच}
{}


\twolineshloka
{तदिन्द्रः सुरभेर्वाक्यं निशम्य भृशविस्मितः}
{जीवितेनापि कौरव्य मेनेऽभ्यधिकमात्मजम्}


\twolineshloka
{प्रववर्ष च तत्रैव सहसा तोयमुल्वणम्}
{कर्षकस्याचरन्विघ्नं भगवान्पाकशासनः}


\twolineshloka
{तद्यथा सुरभिः प्राह सममेवास्तु ते तथा}
{सुतेषु राजन्सर्वेषु हीनेष्वभ्यधिका कृपा}


\twolineshloka
{यादृशो मे सुतः पाण्डुस्तादृशो मेऽसि पुत्रक}
{विदुरश्च महाप्राज्ञः स्नेहादेतद्ब्रवीम्यहम्}


\twolineshloka
{चिराय तव पुत्राणां शतमेकश्च भारत}
{पाण्डौः पञ्चैव लक्ष्यन्ते तेऽपि मन्दाःसुदुःखिताः}


\twolineshloka
{कथं जीवेयुरत्यन्तं कथं वर्धेयुरित्यपि}
{इति दीनेषु पार्थेषु मनो मे परितप्यते}


\twolineshloka
{यदि पार्थिव कौख्याञ्जीवमानानिहेच्छसि}
{दुर्योधनस्तव सुतः शमं गच्छतु पाण्डवैः}


\chapter{अध्यायः १०}
\twolineshloka
{धृतराष्ट्र उवाच}
{}


\twolineshloka
{एवमेतन्महाप्राज्ञ यथा वदसि मां मुने}
{अहं चैव विजानामि सर्वे चेमे नराधिपाः}


\twolineshloka
{भवांश्च मन्यते साधु यत्कुरूणां सुखोदयम्}
{तदेव विदुरोऽप्याह भीष्मो द्रोणश्च मां मुने}


\twolineshloka
{यदित्वहमनुग्राह्यः कौरव्येषु दया यदि}
{अन्वशाधि दुरात्मानं पुत्रं दुर्योधनं मम ॥व्यास उवाच.}


\twolineshloka
{अयमायाति वै राजन्मैत्रेयो भगवानृषिः}
{अन्वीक्ष्य पाण्डवान्भ्रातृनिहैवाऽस्मद्दिदृक्षया}


\twolineshloka
{एष दुर्योधनं पुत्रं तव राजन्महानृषिः}
{अनुशास्ता यथान्यायं शमायास्य कुलस्य च}


\threelineshloka
{ब्रूयाद्यदेष कौरव्य तत्कार्यमविशङ्कया}
{अक्रियायां तु कार्यस्य पुत्रं ते शप्स्यते रुषा ॥वैशंपायन उवाच}
{}


\twolineshloka
{एवमुक्त्वा ययौ व्यासो मैत्रेयः प्रत्यदृश्यत}
{पूजया प्रतिजग्राह सपुत्रस्तं नराधिपः}


% Check verse!
कृत्वाऽर्ध्याद्याः क्रियाः सर्वा विश्रान्तं मुनिसत्तमम्प्रश्रयेणाब्रवीद्राजा धृतराष्ट्रोऽम्बिकासुतः
\threelineshloka
{सुखेनागमनं कच्चिद्भगवन्कुरुजाङ्गले}
{कच्चित्कुशलिनो वीरा भ्रातरः पञ्च पाण्डवाः}
{}


\threelineshloka
{समये स्थातुमिच्छन्ति कच्चिच्च भरतर्षभाः}
{कच्चित्कुरूणां सौभ्रात्रमव्युच्छिन्नं भविष्यति ॥मैत्रेय उवाच}
{}


\twolineshloka
{तीर्थयात्रामनुक्रामन्प्राप्तोऽस्मि कुरुजाङ्गलान्}
{यदृच्छया धर्मराजं दृष्ट्ववान्काम्यके वने}


\twolineshloka
{तं जटाजिनसंवीतं तपोवननिवासिनम्}
{समाजग्मुर्महात्मानं द्रष्टुं मुनिगणाः प्रभो}


\twolineshloka
{तत्राश्रौषं महाराज पुत्राणां तव विग्रहम्}
{अनयं द्यूतरूपेण महाभयमुपस्थितम्}


% Check verse!
ततोऽहं त्वामनुप्राप्तः कौरवाणामवेक्षया ॥सदा ह्यभ्यधिकः स्नेहः प्रीतिश्च त्वयि मे प्रभो
\twolineshloka
{नैतदौपयिकं राजंस्त्वयि भीष्मे च जीवति}
{यदन्योन्येन ते पुत्रा विरुध्यन्ते कथंचन}


\twolineshloka
{मेढीभूतः स्वयं राजन्निग्रहे प्रग्रहे भवान्}
{किमर्थमनयं घोरमुत्पतन्तमुपेक्षसे}


\threelineshloka
{दस्यूनामिव यद्वृत्तं सभायां कुरुनन्दन}
{तेन न भ्राजसे राजंस्तापसानां समागमे ॥वैशंपायन उवाच}
{}


\threelineshloka
{ततो व्यावृत्य राजानं दुर्योधनममर्षणम्}
{उवाच श्लक्ष्णया वाचा मैत्रेयो भगवानृपिः ॥मैत्रेय उवाच}
{}


\twolineshloka
{दुर्योधन महाबाहो निबोध वदतां वर}
{वचनं मे महाभाग ब्रुवतो यद्धितं तव}


\twolineshloka
{मा द्रुहः पाण्डवान्राजन्कुरुष्व हितमात्मनः}
{पाण्डवानां कुरूणां च लोकस्य च नरर्षभ}


\twolineshloka
{पाण्डवान्प्राप्य तान्रात्रौ किर्मीरो नाम राक्षसः}
{अवृत्य मार्गं रौद्रात्मा तस्थौ गिरिवाचलः}


\twolineshloka
{तं भीमः समरस्लाघी बलेन बलिनां वरः}
{जघान पशुमारेण व्याघ्रः क्षुद्रमृगं यथा}


\twolineshloka
{पश्य दिग्विजये राजन्यथा भीमेन पातितः}
{जरासन्धो महेष्वासो नागायुतबलो युधि}


\twolineshloka
{संबन्धी वासुदेवश्च श्यालाः सर्वे च पार्षताः}
{ते हि सर्वे नरव्याघ्राः शूरा विक्रान्तयोधिनः}


\twolineshloka
{सर्वे नागायुतप्राणा वज्रसंहनना दृढः}
{सत्यव्रतधराः सर्वेसर्वे पुरुषमानिनः}


\threelineshloka
{हन्तारो देवशत्रूणां रक्षसां कामरूपिणाम्}
{हिडिम्बबकमुख्यानां किर्मीरस् च रक्षसः}
{कस्तान्युधि समासीत जरामरणवान्नरः}


\threelineshloka
{तस्य ते शम एवास्तु पाण्डवैर्भरतर्षभ}
{कुरु मे वचनं राजन्मा मृत्युवशमन्वगाः ॥वैशंपायन उवाच}
{}


\twolineshloka
{एवं तु ब्रुवतस्तस्य मैत्रेयस्य विशांपते}
{ऊरुं करिकराकारं करेणाभिजघान तः}


\twolineshloka
{दुर्योधनः स्मितं कृत्वा चरणेनोल्लिखन्महीम्}
{नकिंचिदुक्त्वा दुर्मेधास्तस्थौ किंचिदवाङ्मुखः}


\twolineshloka
{तमशुश्रूषमाणं तु विलिखन्तं वसुंधराम्}
{दृष्ट्वा दुर्योधनं राजन्मैत्रेयं कोप आविशत्}


\twolineshloka
{स कोपवशमापन्नो मैत्रेयो मुनिसत्तमः}
{विधिना संप्रयुक्तश्च शापायास्य मनो दधे}


\twolineshloka
{ततः स वार्युपस्पृश्य कोपसंरक्तलोचनः}
{मैत्रेयो धार्तराष्ट्रं तमशपद्दुष्टचेतसम्}


\twolineshloka
{यस्मात्त्वं मामनादृत्य नेमां वाचं चिकीर्षसि}
{तस्मादस्यातिमानस्य सद्यः फलमवाप्स्यसि}


\twolineshloka
{त्वदभिद्रोहसंयुक्तं युद्धमुत्पत्स्यते महत्}
{यत्र भीमो गदाघातैस्तवोरुं भेत्स्यते बली}


\threelineshloka
{इत्येवमुक्ते वचने धृतराष्ट्रो महीपतिः}
{प्रसादयामास मुनिं नैतदेवं भवेदिति ॥मैत्रेय उवाच}
{}


\threelineshloka
{शमं यास्यति चेत्पुत्रस्तव राजन्यदा तदा}
{शायो न भविता तात विपरीते भविष्यति ॥वैशंपायन उवाच}
{}


\threelineshloka
{सविलक्षस्तु राजेन्द्रो दुर्योधनपिता तदा}
{मैत्रेयं प्राह किर्मीरः कथं भीमेन पातितः ॥मैत्रेय उवाच}
{}


\threelineshloka
{नाहं वक्ष्याम्यसूया ते न ते शुश्रूषते सुतः}
{एष ते विदुरः सर्वमाख्यास्यति गते मयि ॥वैशंपायन उवाच}
{}


\twolineshloka
{इत्येवमुक्त्वा मैत्रेयः प्रातिष्ठत यथागतम्}
{किर्मीरवधसंविग्नो भयं दुर्योधनो ययौ}


\chapter{अध्यायः ११}
\twolineshloka
{धृतराष्ट्र उवाच}
{}


\threelineshloka
{किर्मीरस्य वधं क्षत्तः श्रोतुमिच्छामि कथ्यताम्}
{रक्षसा भीमसेनस्य कथमासीत्समागमः ॥विदुर उवाच}
{}


\twolineshloka
{शृणु भीमस्य कर्मेदमतिमानषकर्मणः}
{श्रुतपूर्वं मया तेषां कथान्तेषु पुनः पुनः}


\twolineshloka
{इतः प्रयाता राजेन्द्र पाण्डवा द्यूतनिर्जिताः}
{जग्मुस्त्रिभिरहोरात्रैः काम्यकं नाम तद्वनम्}


\twolineshloka
{रात्रौ निशीथे त्वाभीले गतेऽर्धसमये नृप}
{प्रचारे पुरुषादानां रक्षसां घोरकर्मणाम्}


\twolineshloka
{तद्वनं तापसा नित्यमशेषा वनचारिणः}
{दूरात्परिहरन्ति स्म पुरुषादभयात्किल}


\twolineshloka
{तेषां प्रविशतां तत्र मार्गमावृत्य भारत}
{दीप्ताक्षं भीषणं रक्षः मोल्मुकं प्रत्यपद्यत}


\twolineshloka
{बाहू महान्तौ कृत्वा च तथास्यं च भयानकम्}
{स्थितमावृत्यपन्थानं येन यान्ति कुरूद्वहाः}


\twolineshloka
{दष्टोष्ठदंष्ट्रं ताम्राक्षं प्रदीप्तोर्ध्वशिरोरुहम्}
{सार्करश्मितडिच्चक्रं सबलाकमिवाम्बुदम्}


\twolineshloka
{सृजन्तं राक्षसीं मायां महानादनिनादितम्}
{मुञ्चन्तं विपुलान्नादान्सतोयमिव तोयदम्}


\twolineshloka
{तस्य नादेन संत्रस्ताः पक्षिणः सर्वतोदिशम्}
{विमुक्तनादाः संपेतुः स्थलजा जलजैः सह}


\twolineshloka
{संप्रद्रुतमृगद्वीपिमहिषर्क्षसमाकुलम्}
{तद्वनं तस्य नादेन संप्रस्थितमिवाभवत्}


\twolineshloka
{तस्योरुवाताभिहतास्ताम्रपल्लवबाहवः}
{विदूरजाताश्च लताः समाश्लिष्यन्ति पादपान्}


\twolineshloka
{तस्मिन्क्षणेऽथ प्रववौ मारुतो भृशदारुणः}
{रजसा संवृतं तेन नष्टर्क्षमभवन्नभः}


\twolineshloka
{पञ्चानां पाण्डुपुत्राणामविज्ञातो महारिपुः}
{पञ्चानामिन्द्रियाणां तु शोकावेश इवातुलः}


\twolineshloka
{स दृष्ट्वा पाण्डवान्दूरात्कृष्णाजिनसमावृतान्}
{आवृणोत्तद्वनद्वारं मैनाक इव पर्वतः}


\twolineshloka
{तं समासाद्य वित्रस्ता कृष्णा कमललोचना}
{अदृष्टपूर्वं संत्रासान्न्यमीलयत लोचने}


\twolineshloka
{दुःशासनकरोत्सृष्टविप्रकीर्णशिरोरुहा}
{पञ्चपर्वतमध्यस्था नदीवाकुलतां गता}


\twolineshloka
{विमुह्यमानां तां तत्र जगृहुः पञ्च पाण्डवाः}
{इन्द्रियाणि प्रसक्तानि विषयेषु यथारतिम्}


\twolineshloka
{अथ तां राक्षसीं मायामुत्थितां घोरदर्शनाम्}
{रक्षोघ्नैर्विविधैर्मन्त्रैर्धौम्यः सम्यक्प्रयोजितैः}


\twolineshloka
{पश्यतां पाण्डुपुत्राणां नाशयामास वीर्यवान्}
{स नष्टमायोऽतिबलः क्रोधविस्फारितेक्षणः}


\twolineshloka
{काममूर्तिधरः क्रूरः कालकल्पो व्यदृश्यत}
{तमुवाच ततो राजा दीर्घप्रज्ञो युधिष्ठिरः}


\twolineshloka
{को भवान्कस्य वा किं ते क्रियतां कार्यमुच्यताम्}
{प्रत्युवाचाथ तद्रक्षो धर्मराजं युधिष्ठिरम्}


\twolineshloka
{अहं वकस्य वै भ्राता किर्मीर इति विश्रुतः}
{वनेऽस्मिन्काम्यके शून्ये निवसामि गतज्वरः}


\fourlineindentedshloka
{युधि निर्जित्य पुरुषानाहारं नित्यमाचरन्}
{के यूयमभिसंप्राप्ता भक्ष्यभूता ममान्तिकम्}
{युधि निर्जित्यवः सर्वान्भक्षयिष्ये गतज्वरः ॥विदुर उवाच}
{}


\twolineshloka
{युधिष्ठिरस्तु तच्छ्रुत्वा वचस्तस्य दुरात्मनः}
{आचचक्षे ततः सर्वं गोत्रनामादि भारत}


\twolineshloka
{पाण्डवो धर्मराजोऽहं यदि ते श्रोत्रमागतः}
{सहितो भ्रातृभिः सर्वैर्भीमसेनार्जुनादिभिः}


\twolineshloka
{हृतराज्यो वने वासं वस्तुं कृतमतिस्ततः}
{वनमभ्यागतो घोरमिदं तव परिग्रहम्}


\threelineshloka
{`विस्मयं परमं गत्वा राक्षसो घोरदर्शनः}
{'किर्मीरस्त्वब्रवीदेनं दिष्ट्या देवैरिदं मम}
{उपपादितमद्येह चिरकालान्मनोरथम्}


\twolineshloka
{भीमसेनवधार्थं हि नित्यमभ्युद्यतायुधः}
{चरामि पृथिवीं कृत्स्नां नैनं चासादयाम्यहम्}


\twolineshloka
{सोयमासादितो दिष्ट्या भ्रातृहा काङ्क्षितश्चिरम्}
{अनेन हि मम भ्राता बको विनिहतः प्रियः}


\twolineshloka
{वैत्रकीयवने राजन्ब्राह्मणच्छद्मरूपिणा}
{विद्याबलमुपाश्रित्य न ह्मस्त्यस्यौरसं बलम्}


\twolineshloka
{हिडिम्बश्च सखा मह्मं दयितो वनगोचरः}
{हतो दुरात्मनाऽनेन स्वसा चास्य हृता पुरा}


\twolineshloka
{सोयमभ्यागतो मूढो ममेदं गहनं वनम्}
{प्रचारसमयेऽस्माकमर्द्धरात्रे स्थिते समे}


\twolineshloka
{अद्यास्य यातयिष्यामि तद्वैरं चिरसंभृतम्}
{तर्पयिष्यामि च बकं रुधिरेणास्य भूरिणा}


\twolineshloka
{अद्याहमनृणो भूत्वा भ्रातुः सख्युस्तथैव च}
{शान्तिं लब्धास्मि परमां हत्वा राक्षसकण्टकम्}


\twolineshloka
{यदि तेन पुरा मुक्तो भीमसेनो बकेन वै}
{अद्यैनं भक्षयिष्यामि पश्यतस्ते युधिष्ठिर}


\twolineshloka
{एनं हि विपुलप्राणमद्य हत्वा वृकोदरम्}
{संभक्ष्य जरयिष्यामि यथाऽगस्त्यो महासुरम्}


\twolineshloka
{एवमुक्तस्तु धर्मात्मा सत्यसन्धो युधिष्ठिरः}
{नैतदस्तीति सक्रोधो भर्त्सयामास राक्षसम्}


\twolineshloka
{ततो भीमो महाबाहुरारुज्यतरसा द्रुमम्}
{दशव्याममथोद्विद्धं निष्पत्रमकरोत्तदा}


\twolineshloka
{चकार सज्यं गाण्डीवं वज्रनिष्पेषगौरवम्}
{निमेषान्तरमात्रेण तथैव विजयोऽर्जुनः}


\twolineshloka
{निवार्य भीमो जिष्णुं तं तद्रक्षो मेघनिःस्वनम्}
{अभिद्रुत्याब्रवीद्वाक्यं तिष्ठतिष्ठेति भारत}


\threelineshloka
{इत्युक्त्वैनमतिक्रुद्धः कक्ष्यामुत्पीड्य पाण्डवः}
{निष्पिष्य पाणिना पाणिं संदष्टौष्ठपुटो बली}
{तमभ्यधावद्वेगेन भीमो वृक्षायुधस्तदा}


\twolineshloka
{यमदण्डप्रतीकाशं ततस्तं तस्य मूर्धनि}
{पातयामास वेगेन कुलिशं मघवानिव}


\twolineshloka
{असंभ्रान्तं तु तद्रक्षः समरे प्रत्यदृश्यत}
{चिक्षेप चोल्मुकं दीप्तमशनिं ज्वलितामिव}


\twolineshloka
{तदुदस्तमलातं तु भीमः प्रहरतां वरः}
{पदा सव्येन चिक्षेप तद्रक्षः पुनराव्रजत्}


\twolineshloka
{किर्मीरश्चापि सहसा वृक्षमुत्पाट्य पाण्डवम्}
{दण्डपाणिरिव क्रुद्धः समरे प्रत्यधावत}


\twolineshloka
{तद्वृक्षयुद्धमभवन्महीरुहविनाशनम्}
{वालिसुग्रीवयोर्भ्रात्रोर्यथा स्त्रीकाङ्क्षिणोः पुरा}


\twolineshloka
{शीर्षयोः पतिता वृक्षा विभिदुर्नैकधा तयोः}
{यथैवोत्पलपत्राणि मत्तयोर्द्विपयोस्तथा}


\twolineshloka
{मूर्ध्निजर्झरभूतास्तु बहवस्तत्र पादपाः}
{चीराणीव व्युदस्तानि रेजुस्तत्र महावने}


\twolineshloka
{तद्वृक्षयुद्धमभवन्महूर्तं भरतर्षभ}
{राक्षसानां च मुख्यस्य नराणामुत्तमस्य च}


\twolineshloka
{ततः शिलां समुत्क्षिप्य भीमस्य युधि तिष्ठतः}
{प्राहिणोद्रासः क्रुद्धो भीमसेनश्चचाल ह}


\twolineshloka
{तं शिलाताडनजडं पर्यधावत राक्षसः}
{बाहुविक्षिप्तकिरणः स्वर्भानुरिव भास्करम्}


\twolineshloka
{तावन्योन्यं समाश्लिश्य प्रकर्षन्तौ परस्परम्}
{उभावपि चकाशेते प्रवृद्धौ वृषभाविव}


\twolineshloka
{तयोरासीत्सुतुमुलः संप्रहारः सुदारुणः}
{नखदंष्ट्रायुधवतोर्व्याघ्रयोरिव दृप्तयोः}


\twolineshloka
{दुर्योधननिकाराच्च बाहुवीर्याच्च दर्पितः}
{कृष्णानयनदृष्टश्च व्यवर्धत वृकोदरः}


\twolineshloka
{अभिहत्याथ बाहुभ्यां प्रत्यगृह्णादमर्षितः}
{मातङ्गमिव मातङ्गः प्रभिननकरटामुखम्}


\twolineshloka
{स चाप्येनं ततो रक्षः प्रतिजग्राह वीर्यबान्}
{तमाक्षिपद्भीमसेनो बलेन बलिनां वरः}


\twolineshloka
{तयोर्भुजविनिष्पेषादुभयोर्बलिनोस्तदा}
{शब्दः समभवद्घोरो वेणुस्फोटसमो युधि}


\twolineshloka
{अथैनमाक्षिप्य बलाद्गृद्य मध्ये वृकोदरः}
{धूनयामास वेगेन वायुश्चण्ड इव द्रुमम्}


\twolineshloka
{स भीमेन परामृष्टो दुर्बलो बलिनां रणे}
{व्यस्पन्दत यथाप्राणं विचकर्ष च पाण्डवम्}


\twolineshloka
{तत एनं परिश्रान्तमुपलक्ष्य वृकोदरः}
{योक्रयामास बाहुभ्यां पशुं रशनया यथा}


\twolineshloka
{विनदन्तं महानादं भिन्नभिरीस्वनं बली}
{भ्रामयामास सुचिरं विस्फुरन्तमचेतसम्}


\threelineshloka
{तं विषीदन्तमाज्ञाय राक्षसं पाण्डुनन्दनः}
{प्रगृह्य तरसा दोर्भ्यां पशुमारममारयत्}
{}


\twolineshloka
{आक्रम्य च कटीदेशे जानुना राक्षसाधमम्}
{पीडयामास पाणिभ्यां तस्य कण्ठंवृकोदरः}


\twolineshloka
{अथ जर्जरसर्वाङ्गं व्यावृत्तनयनोल्बणम्}
{भूतले भ्रामयामास वाक्यं चेदमुवाच ह}


\twolineshloka
{हिडिम्बबकयोः पाप न त्वमश्रुप्रमार्जनम्}
{करिष्यसि गतश्चापि यमस्य सदनं प्रति}


\twolineshloka
{इत्येवमुक्त्वा पुरुषप्रवीर-स्तं राक्षसं क्रोधपरीतचेताः}
{विस्रस्तवस्त्राभरणं स्फुरन्त-मुद्धान्तचित्तं व्यसुमुत्ससर्ज}


\threelineshloka
{तस्मिन्हते तोयदतुल्यरूपेकृष्णां पुरस्कृत्य नरेन्द्रपुत्राः}
{भीमं प्रशस्याथ गुणैरनेकै-र्हृष्टास्ततो द्वैतवनाय जग्मुः ॥विदुर उवाच}
{}


\twolineshloka
{एवं विनिहतः सङ्ख्ये किर्मोरो मनुजाधिप}
{भीमेन वचनात्तस् धर्मराजस्य कौरव}


\twolineshloka
{ततो निष्कष्टकं कृत्वा वनं तदयराजितः}
{द्रौपद्या सह धर्मज्ञो वसतिं तामुवास ह}


\twolineshloka
{समाश्वास्य च ते सर्वे द्रौपदीं भरतर्षभाः}
{प्रहृष्टमनसः प्रीत्या प्रसशंसुर्वृकोदरम्}


\twolineshloka
{भीमबाहुबलोत्पिष्टे विनष्टे राक्षसे ततः}
{विविशुस्ते वनं वीराः क्षेमं निहतकण्टकम्}


\twolineshloka
{स मया गच्छता मार्गे विनिकीर्णो भयावहः}
{वने महति दुष्टात्मा दृष्टो भीमबलाद्धतः}


\threelineshloka
{तत्राश्रौषमहं चैतत्कर्म भीमस् भारत}
{ब्राह्मणानां कथयतां ये तत्रासन्समागताः ॥वैशंपायन उवाच}
{}


\twolineshloka
{एवं विनिहतं सङ्ख्ये किर्मीरं रक्षसां वरम्}
{श्रुत्वा ध्यानपरो राजा निशश्वासार्तवत्तदा}


\chapter{अध्यायः १२}
\twolineshloka
{वैशंपायन उवाच}
{}


\twolineshloka
{भोजाः प्रव्रजिताञ्श्रुत्वा वृष्णयश्चान्धकैः सह}
{पाण्डवान्दुःखसंतप्तान्समाजग्मुर्महावने}


\twolineshloka
{पाञ्चालस्य च दायादो धृष्टकेतुश्च चेदिपः}
{केकयाश्च महावीर्या भ्रातरो लोकविश्रुताः}


\twolineshloka
{वने द्रष्टुं ययुः पार्थान्क्रोधामर्षसमन्विताः}
{गर्हयन्तो धार्तराष्ट्रान्किं कुर्म इति चाब्रुवन्}


\fourlineindentedshloka
{वासुदेवं पुरस्कृत्य सर्वे ते क्षत्रियर्षभाः}
{परिवार्योपविविशुर्धर्मराजं युधिष्ठिरम्}
{अभिवाद्य कुरुश्रेष्ठं विषण्णः केशवोऽब्रवीत् ॥वासुदेव उवाच}
{}


\twolineshloka
{दुर्योधनस्य कर्णस् शकुनेश्च दुरात्मनः}
{दुःशासनचतुर्थानां भूमिः पास्यति शोणितम्}


\twolineshloka
{एतान्निहत्य समरे ये च तेषां पदानुगाः}
{तांश्च सर्वान्विनिर्जित्य सहितान्स नराधिपान्}


\threelineshloka
{ततः सर्वेऽभिषिञ्चामो धर्मराजं युधिष्ठिरम्}
{निकृत्याऽभिचरन्वध्य एष धर्मः सनातनः ॥वैशंपायन उवाच}
{}


\twolineshloka
{पार्तानामभिषङ्गेण तथा क्रुद्धं जनार्दनम्}
{अर्जुनः शमयामास दिधक्षन्तमिव प्रजाः}


\twolineshloka
{संक्रुद्धं केशवं दृष्ट्वा पूर्वदेवेषु फल्गुनः}
{कीर्तयामास कर्माणि सत्यकीर्तेर्महात्मनः}


\threelineshloka
{पुरुषस्याप्रमेयस्य सत्यस्यामिततेजसः}
{प्रजापतिपतेर्विष्णोर्लोकनाथस्य धीमतः ॥अर्जुन उवाच}
{}


\twolineshloka
{दशवर्षसहस्राणि यत्रसायंगृहो मुनिः}
{व्यचरस्त्वं पुरा कृष्ण पर्वते गन्धमादने}


\twolineshloka
{दशवर्षसहस्राणि दशवर्षशतानि च}
{पुष्करेष्ववसः कृष्णं त्वमपो भक्षयन्पुरा}


\twolineshloka
{ऊर्ध्वबाहुर्विशालायां बदर्यां मधुसूदन}
{अतिष्ठ एकपादेन वायुभक्षः शतं समाः}


\twolineshloka
{कृष्णाजिनोत्तरासङ्गः कृशो धमनिसंततः}
{आसीः कृष्ण सरस्वत्यां सत्रेद्वादशवार्षिके}


\twolineshloka
{प्रभासभप्यथासाद्य तीर्थं पुण्यजनार्चितम्}
{तत्र कृष्ण महातेजा दिव्यं वर्षसहस्रकम्}


\twolineshloka
{आतिष्ठस्त्वमथैकेन पादेन नियमस्थितः}
{लोकप्रवृत्तिहेतोस्त्वमिति व्यासो ममाब्रवीत्}


\twolineshloka
{क्षेत्रज्ञः सर्वभूतानामादिरन्तश्च केशवं}
{निधानं तपसां कृष्ण यज्ञस्त्वं च सनातनः}


\twolineshloka
{`योगकर्ता हृषीकेशः सांख्यकर्ता सनातनः}
{शीलस्त्वं सर्वयोगानां करणं नियमस्य च'}


\twolineshloka
{निहत्य नरकं भौममाहृत्य मणिकुण्डले}
{प्रथमोत्पादितं कृष्ण मेध्यमश्वमवासृजः}


\twolineshloka
{कृत्वातत्कर्म लोकानामृषभः सर्वलोकजित्}
{अवधीस्त्वं रणे सर्वान्समेतान्दैत्यदानवान्}


\twolineshloka
{ततः सर्वेश्वरत्वं च संप्रदाय शचीपतेः}
{मानुषेषु महाबाहो प्रादुर्भूतोऽसि केशव}


\twolineshloka
{स त्वं नारायणो भूत्वा हरिरासीः परंतप}
{ब्रह्मा सोमश्च सूर्यश्च धर्मो धाता यमोऽनलः}


\twolineshloka
{वायुर्वैश्रवणो रुद्रः कालः खं पृथिवी दिशः}
{अजश्चराचरगुरुः स्रष्टा त्वं पुरुषोत्तम}


\twolineshloka
{तुरायणादिभिर्देव क्रतुभिर्भूरदक्षिणैः}
{अयजो भूरितेजा वै कृष्ण चैत्ररथे वने}


\twolineshloka
{शतं शतसहस्राणि सुवर्णस्य जनार्दन}
{एकैकस्मिंस्तदा यज्ञे परिपूर्णानि दत्तवान्}


\twolineshloka
{अदितेरपि पुत्रत्वमेत्य यादवनन्दन}
{त्वं विष्णुरिति विख्यात इन्द्रादवरजो विभुः}


\threelineshloka
{शिशुर्भूत्वा दिवं खं च पृथिवीं च परंतप}
{त्रिभिर्विक्रमणैः कृष्ण क्रान्तवानसि तेजसा}
{}


\twolineshloka
{संप्राप्य दिवमाकाशमादित्यस्यन्दने स्थितः}
{अत्यरोचश्च भूतात्मन्भास्करं स्वेन तेजसा}


\twolineshloka
{प्रादुर्भावसहस्रेषु तेषुतेषु त्वया विभो}
{अधर्मरुचयः कृष्ण निहताः शतशोऽसुराः}


\twolineshloka
{सादिता मौखाः पाशा निशुम्भनरकौ हतौ}
{कृतः क्षेमः पुनः पन्थाः पुरं प्राग्ज्योतिषं प्रति}


\twolineshloka
{जारूथ्यामाहुतिः क्राथः शिशुपालो नृपैः सह}
{जरासन्धश्च शैब्यश्च शतधन्वा च निर्जितः}


\twolineshloka
{तथा पर्जन्यघोषेण रथेनादित्यवर्चसा}
{अहार्षी रुक्मिणीं भैष्मीं रणे निर्जित्य रुक्मिणे}


\twolineshloka
{इन्द्रद्युम्नो हतः कोपाद्यवनश्च कशेरुमान्}
{हतः सौभपतिः साल्वस्त्वया सौभं च पातितम्}


\threelineshloka
{एवमेते युधि हता भूयश्चान्याञ्शृमुष्व ह}
{इरावत्यां हतो भोजः कार्तवीर्यसमो युधि}
{गोपतिस्तालकेतुश्च त्वया विनिहतावुभौ}


\twolineshloka
{तां च भोगवतीं पुण्यामृषिकान्तां जनार्दन}
{द्वारकामात्मसात्कृत्वा समुद्रं गमयिष्यसि}


\twolineshloka
{न क्रोधो न च मात्सर्यं नानृतं मधुसूदन}
{त्वयि तिष्ठति दाशार्ह न नृशंस्यं कुतोऽनृजु}


\twolineshloka
{आसीनं चैत्यमध्ये त्वां दीप्यमानं स्वतेजसा}
{आगम्य ऋषयः सर्वेऽयाचन्ताभयमच्युत}


\twolineshloka
{युगान्ते सर्वभूतानि संक्षिप्य मधुसूदन}
{आत्मनैवात्मसात्कृत्वा जगदासीः परंतप}


\threelineshloka
{युगादौ तव वार्ष्णेय नाभिपद्मादजायत}
{ब्रह्मा चराचरगुरुर्यस्येदं सकलं जगत्}
{तं हन्तुमुद्यतौ घोरौ दानवौ मधुकैटभौ}


\twolineshloka
{तयोर्व्यतिक्रमं दृष्ट्वा क्रुद्धस्य भवतो हरेः}
{ललाटाज्जातवाञ्शंभुः शूलपाणिस्त्रिलोचनः}


\twolineshloka
{इत्थं तावपि देवेशौ त्वच्छरीरसमुद्भवौ}
{त्वन्नियोगकरावेताविति मे नारदोऽब्रवीत्}


\twolineshloka
{तथा नारायण पुरा क्रतुभिर्भूरिदक्षिणैः}
{इष्टवांस्त्वं महासत्रं कृष्ण चैत्ररथे वने}


\twolineshloka
{नैवं परे नापरे वा करिष्यन्ति कृतानि वा}
{यानि कर्माणि देव त्वं बाल एव महाबलः}


\threelineshloka
{कृतवान्पुण्डरीकाक्ष बलदेवसहायवान्}
{वैराजभवने चापि ब्रह्मणा न्यवसः सह ॥वैशंपायन उवाच}
{}


\twolineshloka
{एवमुक्त्वा महात्मानमात्मा कृष्णस्य पाण्डवः}
{तूष्णीमासीत्ततः पार्थमित्युवाच जनार्दनः}


\twolineshloka
{ममैव त्वं तवैवाहं ये मदीयास्तवैव ते}
{यस्त्वां द्वेष्टि स मां द्वेष्टि यस्त्वामनु स मामनु}


\twolineshloka
{नरस्त्वमसि दुर्धर्ष हरिर्नारायणो ह्यहम्}
{काले लोकमिमं प्राप्तौ नरनारायणावृषी}


\threelineshloka
{अनन्यः पार्थ मत्तस्त्वं त्वत्तश्चाहं तथैव च}
{नावयोरन्तरं शक्यं वेदितुं भरतर्षभ ॥वैशंपायन उवाच}
{}


\twolineshloka
{`इत्युक्त्वा पुण्डरीकाक्षः पाण्डवं सुप्रियं वचः}
{प्रीयमाणो हृषीकेशस्तूष्णीं तत्र बभूव सः ॥'}


\twolineshloka
{एवमुक्ते तु वचने केशेवेन महात्मना}
{तस्मिन्वीरसमावाये संरब्धेष्वथ राजसु}


\fourlineindentedshloka
{धृष्टद्युम्नमुखैर्वीरैर्भ्रातृभिः परिवारिता}
{पाञ्चाली पुण्डरीकाक्षमासीनं यादवैः सह}
{अभिगम्याब्रवीत्कृष्णा शरण्यं शरणैषिणी ॥द्रौपद्युवाच}
{}


\twolineshloka
{`वासुदेव हृषीकेश वासवावरजाच्युत}
{देवदेवोसि देवानामिति द्वैपायनोऽब्रवीत्'}


\twolineshloka
{पूर्वे प्रजाभिसर्गे त्वामाहुरेकं प्रजापतिम्}
{स्रष्टारं सर्वलोकानामसितो देवलोऽब्रवीत्}


\twolineshloka
{विष्णुस्त्वमसि दुर्धर्ष त्वं यज्ञो मधुसूदन}
{यष्टात्वमसि त्वमसि यष्टव्यो जामदग्न्यो यथाऽब्रवीत्}


\twolineshloka
{ऋषयस्त्वां क्षमामाहुः सत्यं च परुषोत्तम}
{सत्याद्यज्ञोसि संभूतः कश्यपस्त्वां यथाऽब्रवीत्}


\twolineshloka
{साध्यानामपि देवानां शिवानामीश्वरेश्वर}
{लोकभावन लोकेश यथा त्वां नारदोऽब्रवीत्}


\twolineshloka
{ब्रह्मशंकरशक्राद्यैर्देववृन्दैः पुनः पुनः}
{क्रीडसे त्वं नरव्याघ्र बालः क्रीडनकैरिव}


\twolineshloka
{द्यौश्च ते शिरसा व्याप्ता पद्भ्यां च पृथिवीप्रभो}
{जठरे खमिमे लोकाः पुरुषोसि सनातनः}


\twolineshloka
{विद्यातपोभितप्तानां तपसा भावितात्मनाम्}
{आत्मदर्शनतृप्तानामृषीणामसि सत्तमः}


\threelineshloka
{राजर्षीणां पुण्यकृतामाहवेष्वनिवर्तिनाम्}
{सर्वधर्मोपपन्नानां त्वं गतिः पुरुषर्षभ}
{त्वं प्रभुस्त्वं विभुश्च त्वं भूतात्मा त्वं विचेष्टसे}


\twolineshloka
{लोकपालाश्च लोकाश्च नक्षत्राणि दिशे दश}
{नभश्चन्द्रश्च सूर्यश्च त्वयि सर्वं प्रतिष्ठितम्}


\twolineshloka
{मंर्त्यता चैव भूतानाममरत्वं दिवौकसाम्}
{त्वयि सर्वं महाबाहो लोककार्यं प्रतिष्ठितम्}


\twolineshloka
{सा तेऽहं दुःखमाख्यास्ये प्रणयान्मधुसूदन}
{ईशस्त्वं सर्वभूतानां ये दिव्या ये च मानुषाः}


\twolineshloka
{कथं नु भार्या पार्थानां तव कृष्ण सखी विभो}
{धृष्टद्युम्नस्य भगिनी सभां कृष्येत मादृशी}


\twolineshloka
{स्त्रीधर्मिणी वेपमाना शोणितेन समुक्षिता}
{एकवस्त्रा विकृष्टाऽस्मि दुःखिता कुरुसंसदि}


\twolineshloka
{राज्ञां मध्ये सभायां तु रजसाऽतिपरिप्लुता}
{दृष्ट्वा च मां धार्तराष्ट्राः प्राहसन्पापचेतसः}


\twolineshloka
{दासीभावेन मां भोक्तुमीष्सुते मधुसूदन}
{जीवत्सु पाण्डुपुत्रेषु पाञ्चालेषु च वृष्णिषु}


\twolineshloka
{नन्वहं कृष्ण भीष्मस्य धृतराष्ट्रस्य चोभयोः}
{स्नुषा भवामि धर्मेण साऽहं दासीकृता बलात्}


\twolineshloka
{गर्हयेपाण्डवांस्त्वेव युधि श्रेष्ठान्महाबलान्}
{यत्क्लिश्यमानां प्रेक्षन्ते धर्मपत्नीं यशस्विनीम्}


\twolineshloka
{धिग्बलं भीमसेनस्य धिक्पार्थस् च गाण्डिवम्}
{यौ मां विप्रकृतां क्षुद्रैर्मर्षयेतां जनार्दन}


\twolineshloka
{शाश्वतोऽयं धर्मपथः सद्भिराचरितः सदा}
{यद्भार्यां परिरक्षन्ति भर्तारोऽल्पबला अपि}


\twolineshloka
{भार्यायां रक्ष्यमाणायां प्रजा भवति रक्षिता}
{प्रजायां रक्ष्यमाणायामात्मा भवति रक्षितः}


\twolineshloka
{आत्मा हि जायेत तस्यां तस्मज्जाया भवत्युत}
{भर्ता च भार्यया रक्ष्यः कथं जायान्ममोदरे}


\twolineshloka
{नन्विमे शरणं प्राप्तं न त्यजन्ति कदाचन}
{ते मां शरणमापन्नां नान्वपद्यन्त पाण्डवाः}


\twolineshloka
{पञ्चभिः पतिभिर्जाताः कुमासरा मे महौजसः}
{एतेषामप्यवेक्षार्थं त्रातव्याऽस्मि जनार्दन}


\twolineshloka
{प्रतिविन्ध्यो युधिष्ठिरात्सुतसोमो वृकोदरात्}
{अर्जुनाच्छ्रुतकीर्तिश्च शतानीकस्तु नाकुलिः}


\twolineshloka
{कनिष्ठाच्छ्रुतकर्मा च सर्वे सत्यपराक्रमा}
{प्रद्युम्नो यादृशः कृष्ण तादृशास्ते महारथाः}


\twolineshloka
{नन्विमे धनुषि श्रेष्ठा अजेया युधि शात्रवैः}
{किमर्थं धार्तराष्ट्राणां सहन्ते दुर्बलीयसाम्}


\twolineshloka
{अधर्मेण हृतं राज्यं सर्वे दासाः कृतास्तथा}
{सभायां परिकृष्टाऽहमेकवस्त्रा रजस्वला}


\twolineshloka
{नाधिज्यमपि यच्छक्यं कर्तुमन्येन गाण्डिवम्}
{अन्यत्रार्जुनभीमाभ्यां त्वया वा मधुसूदन}


\twolineshloka
{धिग्बलं भीमसेनस्य धिक्पार्थस्य च पौरुषम्}
{यत्र दुर्योधनः कृष्ण मुहूर्तमपि जीवति}


\twolineshloka
{य एतानाक्षिपद्राष्टात्सह मात्राऽविहिंसकान्}
{अधीयानान्पुरा बालान्व्रतस्थान्मधुसूदन}


\twolineshloka
{भोजने भीमसेनस्य पापः प्राक्षिपयद्विषम्}
{कालकूटं नवं तीक्ष्णं संभृतं रोमहर्षणम्}


\twolineshloka
{तज्जीर्णमविकारेण सहान्नेन जनार्दन}
{सशेषत्वान्महाबाहो भीमस्य पुरुषोत्तम}


\twolineshloka
{प्रमाणकोट्यां विश्वस्तं तथा सुप्तं वृकोदरम्}
{बद्ध्वैनं कृष्ण गङ्गायां प्रक्षिप्य पुरमाव्रजत्}


\twolineshloka
{यदा विबुद्धः कौन्तेयस्तदा संछिद्य बन्धनम्}
{उदतिष्ठन्महाबाहुर्भीमसेनो महाबलः}


\twolineshloka
{आशीविषैः कृष्णसर्पैः सुप्तंचैनमदंशयत्}
{सर्वेष्वेवाङ्गदेशेषु न ममार च शत्रुहा}


\twolineshloka
{प्रतिबुद्धस्तु कौन्तेयः सर्वान्सर्पानपोथयत्}
{सारथिं चास्य दयितमपहस्तेन जघ्निवान्}


\twolineshloka
{पुनः सुप्तानुपाधाक्षीद्बालकान्वारणावते}
{शयानानार्यया सार्धं को नु तत्कर्तुमर्हति}


\twolineshloka
{यत्रार्या रुदती भीता पाण्डवानिदमब्रवीत्}
{महद्व्यसनमापन्ना शिखिना परिवारिता}


\twolineshloka
{हा हतास्मि कुंतोन्वद्य भवेच्छान्तिरिहानलात्}
{अनाथा विनशिष्यामि बालकैः पुत्रकैः सह}


\twolineshloka
{तत्र भीमो महाबाहुर्वायुवेगपराक्रमः}
{आर्यामाश्वासयामास भ्रातॄंश्चापि वृकोदरः}


\twolineshloka
{वैनतेयो यथा पक्षी गरुत्मान्पततांवरः}
{तथैवाभिपतिष्यामि भयंवो नेह विद्यते}


\twolineshloka
{आर्यामङ्केन वामेन राजानं दक्षिणेन च}
{अंसयोश्च यमौ कृत्वा पृष्ठे बीभत्सुमेव च}


\twolineshloka
{सहसोत्पत्य वेगेन सर्वानादाय वीर्यवान्}
{भ्रातॄनार्यां च बलवान्मोक्षयामास पावकात्}


\twolineshloka
{ते रात्रौ प्रस्थितां सर्वे सह मात्रा यशस्विनः}
{अभ्यगच्छन्महारण्ये हिडिम्बवनमन्तिकात्}


\twolineshloka
{श्रान्ताः प्रसुप्तास्तत्रेमे मात्रा सह सुदुःखिताः}
{सुप्तांश्चैनानभ्यगच्छद्धिडिम्बा नाम राक्षसी}


\twolineshloka
{सा दृष्ट्वा पाण्डवांस्तत्र सुप्तान्मात्रा सह क्षितौ}
{हृच्छयेनाभिभूतात्मा भीमसेनमकामयत्}


\twolineshloka
{भीमस्य पादौ कृत्वा तु स्वउत्सङ्गे ततोऽबला}
{पर्यमर्दत संहृष्टा कल्याणी मृदुपाणिना}


\twolineshloka
{तामबुद्ध्यदमेयात्मा बलवान्सत्यविक्रमः}
{पर्यपृच्छत तां भीमः किमिहेच्छस्यनिन्दिते}


\twolineshloka
{एवमुक्ता तु भीमेन राक्षसी कामरूपिणी}
{भीमसेनं महात्मानमाह चैवमनिन्दिता}


\twolineshloka
{पलायध्वमितः क्षिप्रं मम भ्रातैष वीर्यवान्}
{आगमिष्यति वो हन्तुं तस्माद्गच्छत माचिरम्}


\twolineshloka
{अथ भीमोऽभ्युवाचैनां साभिमानमिदं वचः}
{नोद्विजेयमहं तस्मान्निहनिष्येऽहमागतम्}


\threelineshloka
{तयोः श्रुत्वा तु संजल्पमागच्छद्राक्षसाधमः}
{भीमरूपो महानादान्विसृजन्भीमदर्शनः ॥राक्षस उवाच}
{}


\twolineshloka
{केन सार्धं कथयसि आनयैनं ममान्तिकम्}
{हिडिम्बे भक्षयिष्यामो न चिरं कर्तुमर्हसि}


\twolineshloka
{सा कृपासंगृहीतेन हृदयेन मनस्विनी}
{नैनमैच्छत्तदाकर्तुमनुक्रोशादनिन्दिता}


\twolineshloka
{स नादान्विनदन्घोरान्राक्षसः पुरुषादकः}
{अभ्यद्रवत वेगेन भीमसेनं तदा किल}


\twolineshloka
{तमभिद्रुत्य संक्रुद्धो वेगेन महता बली}
{अगृह्णात्पाणिना पाणिं भीमसेनस्य राक्षसः}


\twolineshloka
{इन्द्राशनिसमस्पर्शं वज्रसंहननं दृढम्}
{संहत्य भीमसेनाय व्याक्षिपत्सहसा करम्}


\twolineshloka
{गृहीतं पाणिना पाणिं भीमसेनस्य रक्षसा}
{नामृष्यत महाबाहुस्तत्राक्रुद्ध्यद्वृकोदरः}


\twolineshloka
{तदासीत्तुमुलं युद्धं भीमसेनहिडिम्बयोः}
{सर्वास्त्रविदुषोर्घोरं वृत्रवासवयोरिव}


\twolineshloka
{विक्रीड्य सुचिरं भीमो राक्षसेन सहानघ}
{निजघान महावीर्यस्तं तदा निर्बलं बली}


\twolineshloka
{हत्वा हिडिम्बं भीमोऽथ प्रस्थितो भ्रातृभिः सह}
{हिडिम्बामग्रतः कृत्वा यस्यां जातो घटोत्कटः}


\twolineshloka
{ततः संप्राद्रवन्सर्वे सह मात्रा परंतपाः}
{एकचक्रामभिमुखाः संवृता ब्राह्मणव्रजैः}


\twolineshloka
{प्रस्थाने व्यास एषां च मन्त्री प्रियहिते रतः}
{ततोऽगच्छन्नेकचक्रां पाणअडवाः संशितव्रताः}


\twolineshloka
{तत्राप्यासादयामासुर्बकं नाम महाबलम्}
{पुरुषादं प्रतिभयं हिडिम्बेनैव संमितम्}


\twolineshloka
{तं चापि विनिहत्योग्रं भीमः प्रहरतां वरः}
{सहितो भ्रातृभिः सर्वैर्द्रुपदस्य पुरं ययौ}


\twolineshloka
{लब्धाऽहमपि तत्रैव वसता सव्यसाचिना}
{यथा त्वया जिता कृष्णरुक्मिणी भीष्मकात्मजा}


\twolineshloka
{एवं सुयुद्धे पार्थेन जिताऽहं मधुसूदन}
{स्वयंवरे महत्कर्म कृत्वा न सुकरं परैः}


\twolineshloka
{एवं क्लेशैः सुबहुभिः क्लिश्यमाना सुदुःखिता}
{निवसाम्यार्यया हीना कृष्ण धौम्यपुरःसुरा}


\threelineshloka
{त इमे सिंहविक्रान्ता वीर्येणाभ्यधिकाः परैः}
{निहीनैः परिक्लिश्यन्ति समुपेक्षन्ति मां कथं}
{}


\twolineshloka
{एतादृशानि दुःखानि महन्ती दुर्बलीयसाम्}
{दीर्घकालं प्रदीप्ताऽस्मि पापानां पापकर्मणाम्}


\twolineshloka
{कुले महतिजाताऽस्मि दिव्येन विधिना किल}
{पाण्डवानां प्रिया भार्या स्नुषा पाण्डोर्महात्मनः}


\twolineshloka
{केशग्रहमनुप्राप्ता का नु जीवेत मादृशी}
{पञ्चानामिन्द्रकल्पानां प्रेक्षतां मधुसूदन}


\twolineshloka
{इत्युक्त्वा प्रारुदत्कृष्णा मुखं प्रच्छाद्य पाणिना}
{पद्मकोशप्रकाशेन मृदुना मृदुभाषिणी}


\twolineshloka
{स्तनावपतितौ पीनौ सुजातौ शुभलक्षणौ}
{अभ्यवर्षत पाञ्चाली दुःखजैरश्रुबिन्दुभिः}


\twolineshloka
{चक्षुषी परिमार्जन्ती निःश्वसन्ती पुनःपुनः}
{बाष्पपूर्णेन कणअठेन क्रुद्धा वचनमब्रवीत्}


\twolineshloka
{नैव मे पतयः सन्ति न पुत्रा न च बान्धवाः}
{न भ्रातरो न च पिता नैव त्वं मधुसूदन}


\twolineshloka
{ये मां विप्रकृतां क्षुद्रैरुपेक्षध्वं विशोकवत्}
{न च मे शाम्यते दुःखं कर्णो यत्प्राहसत्तदा}


\threelineshloka
{चतुर्भिः कारणैः कृष्ण त्वया रक्ष्याऽस्मि नित्यशः}
{संबन्धाद्गौरवात्सख्यात्प्रभुत्वेनैव केशव ॥वैशंपायन उवाच}
{}


\twolineshloka
{अथ तामब्रवीत्कृष्णस्तस्मिन्वीरसमागमे}
{रोदिष्यन्ति स्त्रियो ह्येवं येषां क्रुद्धासि नामिनि}


\twolineshloka
{बीभत्सुशरसंछन्नाञ्शोणितौघपरिप्लुतान्}
{निहतान्वल्लभान्वीक्ष्य शयानान्वसुधातले}


\twolineshloka
{यत्समर्थं पाण्डवनां तत्करिष्यामि मा शुचः}
{सत्यं ते प्रतिजानामि राज्ञां राज्ञी भविष्यसि}


\twolineshloka
{पतेद्द्यौर्हिमवाञ्शीर्येत्पृथिवी शकली भवेत्}
{शुष्येत्तोयनिधिः कृष्णे न मे मोघं वचो भवेत्}


\twolineshloka
{तच्छ्रुत्वा द्रौपदी वाक्यं प्रतिवाक्यमथाच्युतात्}
{साचीकृतमवैक्षत्सा पाञ्चाली मध्यमं पतिम्}


\fourlineindentedshloka
{आबभाषे महाराज द्रौपदीमर्जुनस्तदा}
{मा रोदीः शुभताम्राक्षि यदाह मधुसूदनः}
{तथा तद्भविता देवि नान्यथा वरवर्णिनि ॥धृष्टद्युम्न उवाच}
{}


\twolineshloka
{अहं द्रोणं हनिष्यामि शिखण्डी तु पितामहम्}
{दुर्योधनं भीमसेनः कर्णं हन्ता धनिजयः}


\threelineshloka
{रामकृष्णौ व्यपाश्रित्य अजेयाः स्म रणए स्वसः}
{अपि वृत्रहणा युद्धे किं पुनर्धृतराष्ट्रजैः ॥वैशंपायन उवाच}
{}


\twolineshloka
{इत्युक्तेऽभिमुखा वीरा वासुदेवमुपास्थितः}
{तेषां मध्ये महाबाहुः केशवो वाक्यमब्रवीत्}


\chapter{अध्यायः १३}
\twolineshloka
{वासदेव उवाच}
{}


\twolineshloka
{नैतत्कृच्छ्रमनुप्राप्तो भवान्स्याद्वसुधाधिप}
{यद्यहं द्वारकायां स्यां राजन्सन्निहितः पुरा}


\threelineshloka
{आगच्छेयमहं द्यूतमनाहूतोऽपि कौरवैः}
{आम्बिकेयेन दुर्धर्ष राज्ञा दुर्योधनेन च}
{वारयेयमहं द्यूतं बहून्दोपान्प्रदर्शयन्}


\twolineshloka
{भीष्मद्रोणौ समानाय्य कृपं बाह्लीकमेव च}
{वैचित्रवीर्यं राजानमलं द्यूतेन कौरव}


\twolineshloka
{पुत्राणां तव राजेन्द्र त्वन्निमित्तमिति प्रभो}
{तत्राचक्षमहं दोषान्यैर्भवान्व्यतिरोपितः}


\twolineshloka
{वीरसेनसुतो यैस्तु राज्यात्प्रभ्रंशितः पुरा}
{अतर्कितविनाशश्च देवनेन विशांपते}


% Check verse!
सातत्यं च प्रसङ्गस्य वर्णयेयं यथातथम्
\twolineshloka
{स्त्रियोऽक्षा मृगया पानमेतत्कामसमुत्थितम्}
{दुःखं चतुष्टय प्रोक्तं यैर्नरो भ्रश्यते श्रियः}


\twolineshloka
{तत्रसर्वत्रवक्तव्यं मन्यन्ते शास्त्रकोविदाः}
{विशेषतश्च वक्तव्यं द्यूते पश्यन्ति तद्विदः}


\twolineshloka
{एकाहाद्द्रव्यनाशोऽत्र ध्रुवं व्यसनमेव च}
{अभुक्तनाशश्चार्थानां वाक्पारुष्यं च केवलम्}


\twolineshloka
{एतच्चान्यच्च कौरव्य प्रसङ्गिकटुकोदयम्}
{द्यूते ब्रूयां महाबाहो समासाद्याम्बिकासुतम्}


\twolineshloka
{एवमुक्तो यदि मया गृह्णीयाद्वचनं मम}
{अनामयं स्याद्धर्मश्च कुरूणां कुरुवर्दन}


\twolineshloka
{न चेत्स मम राजेन्द्र गृह्णीयान्मधुरं वचः}
{पथ्यं च भरतश्रेष्ठ निगृह्णीयां बलेन तम्}


\twolineshloka
{अथैनमविनीतेन सुहृदो नाम दुर्हृदः}
{सभासदोऽनुवर्तेरंस्तांश्च इन्यां दुरोदरान्}


\twolineshloka
{असान्निध्यं तु कौरव्य ममानर्तेष्वभूत्तदा}
{येनेदं व्यसनं प्राज्ञा भवन्तो द्यूतकारितम्}


\twolineshloka
{सोहमेत्य कुरुश्रेष्ठ द्वारकां पाण्डुनन्दन}
{अश्रौषं त्वां व्यसनिनं युयुधानाद्यथातथम्}


\twolineshloka
{श्रुत्वैव चाहं राजेन्द्र परमोद्विग्रमानसः}
{तूर्णमभ्यागतोस्मित्वां द्रष्टुकामो विशंपते}


\twolineshloka
{अहो कृच्छ्रमनुप्राप्ताः सर्वे स्म भरतर्षभ}
{सोऽहं त्वां व्यसने मग्नं पश्यामि सह सोदरैः}


\chapter{अध्यायः १४}
\twolineshloka
{युधिष्ठिर उवाच}
{}


\threelineshloka
{असान्निध्यं कथं कृष्ण तवासीद्वृष्णिनन्दन}
{क्व चासीद्विप्रवासस्ते किंचाकार्षीः प्रवासतः ॥कृष्ण उवाच}
{}


\twolineshloka
{साल्वस्य नगरं सौभं गतोऽहं भरतर्षभ}
{निहन्तुं कौरवश्रेष्ठ तत्र मे शृणु कारणम्}


\twolineshloka
{महातेजा महांवाहुर्यः स राजा महायशाः}
{दमघोषात्मजो वीरः शिशुपालो मया हतः}


\twolineshloka
{यज्ञे ते भरतश्रेष्ठ राजसूयेऽर्हणां प्रति}
{स रोषवशमापन्नो नामृष्यत दुरात्मवान्}


\twolineshloka
{श्रुत्वा तं निहतं साल्वस्तीव्ररोषसमन्वितः}
{उपायाद्द्वारकां शून्यामिहस्थे मयि भारत}


\twolineshloka
{स तत्र युद्धमकरोद्बालकैर्वृष्णिपुङ्गवैः}
{निवृत्तः कामगं सौभमारुह्यैव नृशंसवत्}


\threelineshloka
{`चिरजीवी नृपः सोऽपि प्रसादात्पद्मजन्मनः}
{'ततो वृष्णिप्रवीरांस्तान्बालान्हत्वा बहूंस्तदा}
{पुरोद्यानानि सर्वाणि भेदयामास दुर्भतिः}


\twolineshloka
{उक्तवांश्च महाबाहो क्वासौ वृष्णिकुलाधमः}
{वासुदेवः स मन्दात्मा वसुदेवसुतो गतः}


\twolineshloka
{तस्य युद्धार्थिनो दर्पं युद्धेनाशयिताऽस्म्यहम्}
{आनर्ताः सत्यमाख्यात तत्र गन्ताऽस्मि यत्र सः}


\threelineshloka
{तं हत्वा विनिवर्तिष्ये कंसकेशिनिषूदनम्}
{अहत्वा न निवर्तिष्ये सत्येनायुधमालभे}
{}


\twolineshloka
{क्वासौ क्वासाविति पुनस्तत्रतत्र प्रधावति}
{मया सह रणे योद्धुं काङ्क्षमाणः स सौभराट्}


\twolineshloka
{अद्यतं पापकर्माणं क्षुद्रं विश्वासघातिनम्}
{शिशुपालवधामर्षाद्गमयिष्ये यमक्षयम्}


\twolineshloka
{मम पापस्वभावेन भ्राता येन निपातितः}
{शिशुपालो महीपालस्तं वधिष्ये महीतले}


\threelineshloka
{भ्राता बालश्च राजा च न च संग्रामकोविदः}
{प्रमत्तश्च हतो वीरस्तं हनिष्ये जनार्दनम्}
{}


\twolineshloka
{एकमादि महाराज विलप्यदिवमास्थितः}
{कामगेन स सौभेन क्षिप्त्वा मां कुरुनन्दन}


\twolineshloka
{तमश्रौषमहं गत्वा यथा वृत्तः स दुर्मतिः}
{मयि कौरव्य दुष्टात्मा मार्तिकावतको नृपः}


\twolineshloka
{ततोऽहमपि कौरव्य रोषव्याकुललोचनः}
{निश्चित्य मनसा राजन्वधायास्य मनो दधे}


\twolineshloka
{आनर्तेषु विमर्दं च क्षेपं चात्मनि कौरव}
{प्रवृद्धमवलेपं च तस्य दुष्कृतकर्मणः}


\twolineshloka
{ततः सौभवधायाहं प्रतस्थे पृथिवीपते}
{स मया सागरावर्ते दृष्ट आसीत्परीप्सता}


\twolineshloka
{ततः प्रध्माप्य जलजं पाञ्चजन्यमहे नृप}
{आहूय साल्वं समरे युद्धाय समवस्थितः}


\twolineshloka
{तन्मुहूर्तमभूद्युद्धं तत्र मे दानवैः सह}
{शवीभूताश्च मे सर्वे भूतले च निपातिताः}


\threelineshloka
{एतत्कार्यं महाबाहो येनाहं नागमं तदा}
{श्रुत्वैव हास्तिनपुरे द्यूतं चाविनयोत्थितम्}
{द्रुतमागतवान्युष्मान्द्रष्टुकामः सुदुःखितान्}


\chapter{अध्यायः १५}
\twolineshloka
{युधिष्ठिर उवाच}
{}


\threelineshloka
{वासुदेव महाबाहो विस्तरेण महामते}
{सौभस्य वधमाचक्ष्व न हि तृष्यामि कथ्यतः ॥वासुदेव उवाच}
{}


\twolineshloka
{हतं श्रुत्वा महाबाहौ मया श्रौतश्रवं नृप}
{उपायाद्भरतश्रेष्ठ साल्वो द्वारवतीं पुरीम्}


\twolineshloka
{अरुणत्तां सुदुष्टात्मा सर्वतः पाण्डुनन्दन}
{साल्वो वैहायसं चापि तत्पुरं व्यूह्य धिष्ठितः}


\twolineshloka
{तत्रस्थोऽथ महीपालो योधयामास तां पुरीम्}
{अभिसारेण सर्वेण तत्र युद्धमवर्तत}


\twolineshloka
{पुरी समन्ताद्विहिता सपताका सतोरणा}
{सचक्रा सहुडा चैव सयन्त्रखनका तथा}


\twolineshloka
{सोपशल्यप्रतोलीका साट्टाट्टालकगोपुरा}
{सचक्रग्रहिणी चैव सोल्कालातावपोथिका}


\twolineshloka
{सोष्टिका भरतश्रेष्ठ सभेरीपणवानका}
{सतोमराङ्कुशा राजन्सशतघ्नीकलाङ्गला}


\threelineshloka
{सभुशुण्ड्यश्मगुडका सायुधा सपरश्वधा}
{लोहचर्मवती चापि साग्निः सगुडशृङ्गिका}
{शास्त्रदृष्टेन विधिना सुयुक्ता भरतर्षभ}


\twolineshloka
{रथैरनेकैर्विविधैर्गदसाम्बोद्धवादिभिः}
{पुरुषैः कुरुशार्दूल समर्थैः प्रतिवारणे}


\threelineshloka
{अतिख्यातकुलैर्वीरैर्दृष्टवीर्यैश्च संयुगे}
{मध्यमेन च गुल्मेन रक्षिभिः सा सुरक्षिता}
{उत्क्षिप्तगुल्मैश्च तथा हयैश्च सपताकिभिः}


\twolineshloka
{आघोषितं च नगरे न पातव्या सुरेति वै}
{प्रमादं परिरक्षद्भिरुग्रसेनोद्धवादिभिः}


\twolineshloka
{प्रमत्तेष्वभिघातं हि कुर्यात्साल्वो नराधिपः}
{इति कृत्वाऽप्रमत्तास्ते सर्वेवृष्ण्यन्धकाः स्थिताः}


\twolineshloka
{आनर्ताश्च तथा सर्वे नटनर्तकगायकाः}
{बहिर्निर्वासिताः सर्वे रक्षद्भिर्वित्तसंचयम्}


\twolineshloka
{संक्रमा भेदिताः सर्वे नावश्च प्रतिषेधिताः}
{परिखाश्चापि कौरव्य कीलैः सुनिचिताः कृताः}


\threelineshloka
{उदपानाः कुरुश्रेष्ठ तथैवाप्यम्बरीषकाः}
{समन्तात्क्रोशमात्रं च कारिता विषमा च भूः}
{`संक्रमा भेदिताः सर्वे प्राकाराश्च नवीकृताः ॥'}


\twolineshloka
{प्रकृत्या विषमं दुर्गं प्रकृत्या च सुरक्षितम्}
{प्रकृत्या चायुधोपेतं विशेषेण तदाऽनघ}


\twolineshloka
{सुरक्षितं सुगुप्तं च सर्वायुधसमन्वितम्}
{तत्पुरं भरतश्रेष्ठ यथेन्द्रभवनं तथा}


\twolineshloka
{न चामुद्रोऽभिनिर्याति न चामुद्रः प्रवेश्यते}
{वृष्ण्यन्धकपुरे राजंस्तदा सौभसमागमे}


\twolineshloka
{अनुरध्यासु सर्वासु चत्वरेषु च कौरव}
{बलं बभूव राजेन्द्र प्रभूतगजवाजिमत्}


\twolineshloka
{दत्तवेतनभक्तं च दत्तायुधपरिच्छदम्}
{कृतोपधानं च तदा बलमासीद्विशांपते}


\twolineshloka
{न कृप्यवेतनी कश्चिन्न चातिक्रान्तवेतनी}
{नानुग्रहभृतः कश्चिन्न चादृष्टपराक्रमः}


\twolineshloka
{एवं सुविहिता राजन्द्वारका भूरिदक्षिणैः}
{आहुकेन सुगुप्ता च राज्ञा राजीवलोचन}


\chapter{अध्यायः १६}
\twolineshloka
{वासुदेव उवाच}
{}


\twolineshloka
{तां तूपयातो राजेन्द्र साल्वः सौभपतिस्तदा}
{प्रभूतनरनागेन बलेनोपविवेश ह}


\twolineshloka
{समे निविष्टां सा सेना प्रभूतसलिलाशये}
{चतुरङ्गबलोपेता साल्वराजाभिपालिता}


\twolineshloka
{वर्जयित्वा श्मशानानि देवतायतनानि च}
{वल्मीकांश्चैव चैत्यांश्च संनिविष्टमभूद्बलम्}


\twolineshloka
{अनीकानां विभागेन पन्थानः सुकृतास्तथा}
{प्रथमा नवमाश्चैव साल्वस्य शिबिरे नृप}


\twolineshloka
{सर्वायुधसमोपेतं सर्वशस्त्रविशारदम्}
{रथनागाश्वकलिलं पदातिजनसंकुलम्}


\twolineshloka
{तुष्टपुष्टबलोपेतं वीरलक्षणलक्षितम्}
{विचित्रध्वजसन्नाहं विचित्ररथकार्मुकम्}


\twolineshloka
{संनिवेश्य च कौरव्य द्वारकायां नरर्षभ}
{अभिसारयामास तदा वेगैन पतगेन्द्रवत्}


\twolineshloka
{तदापतन्तं संदृश्य बलं साल्वपतेस्तदा}
{निर्याय योधयामासुः कुमारा वृष्णिनन्दनाः}


\twolineshloka
{असहन्तोऽभियानं तत्साल्वराजस्य कौरव}
{चारुदेष्णश्च साम्बश्च प्रद्युम्नश्च रमहारथः}


\twolineshloka
{ते रथैर्दंशिताः सर्वेविचित्राभरणध्वजाः}
{संसक्ताः साल्वराजस्य बहुभिर्योधपुङ्गवैः}


\twolineshloka
{गृहीत्वा कार्मुकं साम्बः साल्वस्य सचिवं रणे}
{यीधयामास संहृष्टः क्षेमधूर्तिं चमूपतिम्}


\twolineshloka
{तस्य बाणमयं वर्षं जाम्बवत्याः सुतो महत्}
{मुमोच भरतश्रेष्ठ यथावर्षं सहस्रदृक्}


\twolineshloka
{तद्बाणवर्षं तुमुलं विषेहे स चमूपतिः}
{क्षेमधूर्तिर्महाराज हिमवानिव निश्चलः}


\twolineshloka
{ततः साम्बाय राजेन्द्र क्षेमधूर्तिरपि स्वयम्}
{मुमोच मायाविहितं शरजालं महत्तरम्}


\twolineshloka
{ततो मायामयं जालं माययैव विदार्यसः}
{साम्बः शरसहस्रेण रथमस्याभ्यवर्षत}


\twolineshloka
{ततः स विद्धः साम्बेन क्षेमधूर्तिश्चमूपतिः}
{अपायाज्जवनैरश्वैः साम्बबाणप्रपीडितः}


\twolineshloka
{तस्मिन्विप्रद्रुते शरे साल्वस्याथ चमूपतौ}
{वेगवान्नाम दैतेयः सुतं मेऽभ्यद्रवद्बली}


\twolineshloka
{अभिपन्नस्तु राजेन्द्र साम्बो वृष्णिकुलोद्वहः}
{वेगं वेगवतो राजंस्तस्थौ वीरो विधारयन्}


\twolineshloka
{स वेगवति कौन्तेय साम्बो वेगवतीं गदाम्}
{चिक्षेप तरसा वीरो व्याविद्ध्यन्सत्यविक्रमः}


\twolineshloka
{तया त्वभिहतो राजन्वेगवान्न्यपतद्भुवि}
{वातरुग्णं इव क्षुण्णो जीर्णमूलोवनस्पतिः}


\twolineshloka
{तस्मिन्निपतिते वीरे गदारुग्णे महासुरे}
{प्रविश्य महतीं सेनां योधयामास मे सुतः}


\twolineshloka
{चारुदेष्णेन संसक्तो विविन्ध्यो नाम दानवः}
{महारथः समाज्ञातो महाराज महाधनुः}


\twolineshloka
{ततः सुतुमुलं युद्धं चारुदेष्णविविन्ध्ययोः}
{वृत्रवासवयो राजन्यथापूर्वं तथाऽभवत्}


\twolineshloka
{अन्योन्यस्याभिसंक्रुद्धावन्योन्यं जघ्नतुः शरैः}
{विनदन्तौ महाराज सिंहाविव महाबलौ}


\twolineshloka
{रौक्मिणेयस्ततो बाणमग्न्यर्कोपमवर्चसम्}
{अभिमन्त्र्य महास्त्रेण संदधे शत्रुनाशनम्}


\twolineshloka
{स विविन्ध्याय सक्रोधः समाहूय महारथः}
{चिक्षेप मे सुतो राजन्स गतासुरथापतत्}


\twolineshloka
{विविन्ध्यं निहृतं दृष्ट्वा तां च विक्षोभितां चमूम्}
{कामगेन स सौभेन साल्वः पुनरुपागमत्}


\twolineshloka
{ततो व्याकुलितं सर्वं द्वारकावासि तद्बलम्}
{दृष्ट्वा साल्वं महाबाहो सौभस्थं पृथिवीगतम्}


\twolineshloka
{ततो निर्याय कौरव्य अवस्थाप्य च तद्बलम्}
{आनर्तानां महाराज प्रद्युम्नो वाक्यमब्रवीत्}


\twolineshloka
{सर्वे भवन्तस्तिष्ठन्तु सर्वे पश्यन्तु मां युधि}
{निवारयन्तं संग्रामे बलात्सौभं सराजकम्}


\twolineshloka
{अयं सौभपतेः सेनामायसैर्भुजगैरिव}
{धनुर्भुजविनिर्मुक्तैर्नाशयम्यद्य यादवाः}


\twolineshloka
{आश्वसध्वं न भीः कार्या सौभराडद्य नश्यति}
{मयाऽभिपन्नो दुष्टात्मा ससौभो विनशिष्यति}


\twolineshloka
{एवं ब्रुवति संहृष्टे प्रद्युम्ने पाण्डुनन्दन}
{धिष्टितं तद्बलं द्वारि युयुधे च यथासुखम्}


\chapter{अध्यायः १७}
\twolineshloka
{वासुदेव उवाच}
{}


\twolineshloka
{एवमुक्त्वा रौक्मिणेयो यादवान्यादवर्षभ}
{दंशितैर्हरिभिर्युक्तं रथमास्थाय काञ्चनम्}


\twolineshloka
{उच्छ्रित्य मकरं केतुं व्यात्ताननमलङ्कृतम्}
{उत्पतद्भिरिवाकाशं तैर्हयैरन्वयात्परान्}


\twolineshloka
{विक्षिपन्नादयंश्चापि धनुः श्रेष्ठं महाबलैः}
{तूणखङ्गधरः शूरो बद्धगोधाङ्गुलित्रवान्}


\twolineshloka
{सविद्युच्छुरितं चापं विहरन्वै तलात्तलम्}
{मोहयामास दैतेयान्सर्वान्सौभनिवासिनः}


\twolineshloka
{तस्य विक्षिपतश्चापं संदधानस्य चासकृत्}
{नान्तरं ददृशे कश्चिन्नघ्नितः शात्रवान्रणे}


\twolineshloka
{मुखस्य वर्णो न विकल्पतेऽस्यचेलुश्च गात्राणि न चापि तस्य}
{सिंहोन्नतं चाप्यभिगर्जतोऽस्यशुश्राव लोकोऽद्भुतवीर्यमग्र्यम्}


\twolineshloka
{जलेचरः काञ्चनयष्टिसंस्थोव्यात्ताननः शत्रुबलप्रमाथी}
{वित्रासयन्राजति वाहमुख्येसाल्वस्य सेनाप्रमुखे ध्वजाग्र्यः}


\twolineshloka
{ततस्तूर्णं विनिष्पत्य प्रद्युम्नः शत्रुकर्शनः}
{साल्वमेवाभिदुद्राव विधित्सुः कलहं नृप}


\twolineshloka
{अभियानं तु वीरेण प्रद्युम्नेन महारणे}
{नामर्षयत संक्रुद्वः साल्वः कुरुकुलोद्वह}


\twolineshloka
{सरोषमहमत्तो वै कामगादवरुह्य च}
{प्रद्युम्नं योधयामास साल्वः परपुरंजयः}


\twolineshloka
{तयोः सुतुमुलं युद्धं साल्ववृष्णिप्रवीरयोः}
{समेता ददृशुर्लोका बलिवासवयोरिव}


\twolineshloka
{तस्य मायामयो वीर रथो हेमपरिष्कृतः}
{सपताकः सध्वजश्च सानुकर्षः स तूणवान्}


\twolineshloka
{स तं रथवरं श्रीमान्समारुह्य किल प्रभो}
{मुमोच बाणान्कौख्य प्रद्युम्नाय महाबलः}


\twolineshloka
{ततो बाणमयं वर्षं व्यसृजत्तरसा रणे}
{प्रद्युम्नो भुजवेगेन साल्वं संमोहयन्निव}


\twolineshloka
{स तैरभिहतः सङ्ख्ये नामर्यत सौभराट्}
{शरान्दीप्ताग्निसंकाशान्मुमोच तनये मम}


\twolineshloka
{तमापतन्तं बाणौधं स चिच्छेद महाबलः}
{ततश्चान्याञ्शरान्दीप्तान्प्रचिक्षेप सुते मम}


\twolineshloka
{स साल्वबाणै राजेन्द्र विद्धो रुक्मिणिनन्दनः}
{मुमोच बाणं त्वरितो मर्मभेदिनमाहवे}


\twolineshloka
{तस्य वर्म विभिद्याशु स बाणो मत्सुतेरितः}
{विवेश हृदयं पत्री स पपात भृशहतः}


\twolineshloka
{तस्मिन्निपतिते वीरे साल्वराजे विचेतसि}
{संप्रद्रबन्दानवेन्द्रा दारयन्तो वसुंधराम्}


\twolineshloka
{हाहाकृतमभूत्सैन्यं साल्वस्य पृथिवीपते}
{नष्टसंज्ञे निपतिते तदा सौभपतौ नृपे}


\twolineshloka
{तत उत्थाय राजेन्द्र प्रतिलभ्य च चेतनाम्}
{मुमोच बाणान्सहसा प्रद्युम्नाय महाबलः}


\twolineshloka
{तेन बाणेन महता प्रद्युम्नः समरे स्थितः}
{जत्रुदेशे भृशं विद्धो व्यथितो ददृशे तदा}


\twolineshloka
{तं स विद्ध्वा महाराज साल्वो रुक्मिणिनन्दनम्}
{ननाद सिंहनादं वै नादेनापूरयन्महीम्}


\twolineshloka
{ततो मोहं समापन्ने तनये मम भारत}
{मुमोच बाणांस्त्वरितः पुनरन्यान्दुरात्मवान्}


\twolineshloka
{स तैरभिहतो बाणैर्बहुभिस्तेन मोहितः}
{निश्चेष्टः कौरवश्रेष्ठ प्रद्युम्नोऽभूद्रणाजिरे}


\chapter{अध्यायः १८}
\twolineshloka
{वासुदेव उवाच}
{}


\twolineshloka
{साल्वबाणार्दिते तस्मिन्प्रद्युम्ने बलिनांवरे}
{वृष्णयो भग्नसंकल्पा विव्यथुः पृतनामुखे}


\twolineshloka
{हाहाकृतमभूत्सर्वं वृष्ण्यन्धकबलं ततः}
{प्रद्युम्ने मोहिते राजन्साल्वः प्रमुदितोऽभवत्}


\twolineshloka
{तं तथा मोहितं दृष्ट्वा सारथिर्जवनैर्हयैः}
{रणादपाहरत्तूर्णं शिक्षितो दारुकिस्तदा}


\twolineshloka
{नातिदूरापयाते तु रथे रथवरप्रणुत्}
{धनुर्गृहीत्वा यन्तारं लब्धसज्ञोऽब्रवीदिदम्}


\twolineshloka
{सौते किं ते व्यवसितं कस्माद्यासि पराङ्मुखः}
{नैष वृष्णिप्रवीराणामाहवे धर्म उच्यते}


\threelineshloka
{कच्चित्सौते न ते मोहः साल्वं दृष्ट्वा महाहवे}
{विषादो वा रणँ दृष्ट्वा ब्रूहि मे त्वं यथातथम् ॥सौतिरुवाच}
{}


\twolineshloka
{जानार्दने न मे मोहो नापि मां भयमाविशत्}
{अतिभारं तु ते मन्ये साल्वं केशवनन्दन}


\twolineshloka
{सोभियाति शनैर्वीर बलवानेष पापकृत्}
{मोहितश्च रणे शूरो रक्ष्यः सारथिना रथी}


\twolineshloka
{आयुष्मंस्त्वं मया नित्यं रक्षितव्यस्त्वयाऽप्यहम्}
{रक्षितव्यो रणे नित्यमिति कृत्वाऽपयाम्यहम्}


\threelineshloka
{एकश्चासि महाबाहो बहवश्चापि दानवाः}
{न समं रौक्मिणेयाहं रणए मत्वाऽपयामि वा ॥वासुदेव उवाच}
{}


\twolineshloka
{एवं ब्रुवति सूते तु तदा मकरकेतुमान्}
{उवाच सूतं कौरव्य संनिवर्त्य रथं पुनः}


\twolineshloka
{दारुकात्मज मैवं त्वं पुनः कार्षीः कथंचन}
{न्यपयानं रणात्सौते जीवतो मम कर्हिचित्}


\twolineshloka
{न स वृष्णिकुले जातो यो वै त्यजति संगरम्}
{यो वा निपतितं हन्ति तवास्मीति च वादिनम्}


\twolineshloka
{तथा स्त्रियं च यो हन्ति बालं वृद्धं तथैव च}
{विरथं मुक्तकेशं च भग्नशस्त्रायुधं तथा}


\twolineshloka
{त्वं च सूतकुले जातो विदितः सूतकर्मणि}
{धर्मज्ञश्चासि वृष्णीनामाहवेष्वपि दारुके}


\twolineshloka
{स जानंश्चरितं कृत्स्नं वृष्णीनां पृतनामुखे}
{अपयानं पुन सौते मैवं कार्षीः कथंचन}


\twolineshloka
{अपयातं हतं पृष्ठे भ्रान्तं रणपलायितम्}
{गदाग्रजो दुराधर्षः किं मां वक्ष्यति माधवः}


\twolineshloka
{केशवस्याग्रजो वाऽपि नीलवासा मदोत्कटः}
{किं वक्ष्यति महाबाहुर्बलदेवः समागतः}


\twolineshloka
{किं वक्ष्यतिशिनेर्नप्ता रणसिंहो महारथः}
{अपयातं रणात्सूत साम्बश्च समितिंजयः}


\twolineshloka
{चारुदेष्णश्च दुर्धर्षस्तथैव गदसारणौ}
{अक्रूरश्च महाबाहुः किं मां वक्ष्यति सारथे}


\twolineshloka
{शूरं संभावितं शान्तं नित्यं पुरुषमानिनम्}
{स्त्रियश्च वृष्णिवीराणां किं मांवक्ष्यन्ति संगताः}


\twolineshloka
{प्रद्युम्नोऽयमुपायाति भीतस्त्यक्त्वा महाहवम्}
{धिगेनमिति वक्ष्यन्ति न तु वक्ष्यन्ति साध्विति}


\twolineshloka
{धिग्वाचा परिहासोपि मम वा मद्विधस्य वा}
{मृत्युनाऽभ्यधिकः सौते स त्वं माव्यपयाः पुनः}


\threelineshloka
{भारं हि मयि संन्यस्य यातो मधुनिहा हरिः}
{यज्ञं बारतसिंहस्य न हि शक्योऽद्यमर्षितुम्}
{}


\twolineshloka
{कृतवर्मा मया वीरो निर्यास्यन्नेव वारितः}
{साल्वं निवारयिष्येऽहं तिष्ठ त्वमिति सूतज}


\twolineshloka
{स च संभावयन्मां वै निवृत्तो हृदिकात्मजः}
{तं समेत्य रणं त्यक्त्वा किं वक्ष्यामि महारथं}


\twolineshloka
{उपयान्तं दुराधर्षं शङ्खचक्रगदाधरम्}
{पुरुषं पुण्डरीकाक्षं किं वक्ष्यामि महाभुजम्}


\twolineshloka
{सात्यकिं बलदेवं च ये चान्येऽन्धकवृष्णयः}
{मया स्पर्धन्ति सततं किं नु वक्ष्यामि तानहं}


\twolineshloka
{त्यक्त्वा रणमिमं सौते पृष्ठतोऽभ्याहतः शरैः}
{त्वयाऽपनीती विवशो न जीवेयं कथंचन}


\twolineshloka
{संनिवर्त रथेनाशु पुनर्दारुकनन्दन}
{न चैतदेवं कर्तव्यमथापत्सु कथंचन}


\twolineshloka
{न जीवितमहं सौते बहु मन्ये कथंचन}
{अपयातो रणाद्भीतः पृष्ठतोऽभ्याहतः शरैः}


\twolineshloka
{कदाऽपि सूतपुत्र त्वं जानीषे मां भयार्दितम्}
{अपयातं रणं हित्वा यथा कापुरुषं तथा}


\twolineshloka
{अयुक्तं तु मया त्यक्तुं संग्रामं दारुकात्मज}
{मयि युद्धार्थिनि भृशं स त्वं याहि यतो रणम्}


\chapter{अध्यायः १९}
\twolineshloka
{वासुदेव उवाच}
{}


\twolineshloka
{एवमुक्तस्तु कौन्तेय सूतपुत्रस्ततो मृधे}
{प्रद्युम्नमब्रवीच्छ्लक्ष्णं मधुरं वाक्यमञ्जसा}


\twolineshloka
{न मे भयं रौक्मिणेय संग्रामे यच्छतो हयान्}
{युद्धज्ञोस्मि च वृष्णीनां नात्र किंचिदतोऽन्यथा}


\twolineshloka
{आयुष्मन्नुपदेशस्तु सारथ्ये वर्ततां स्मृतः}
{सर्वार्थेषु रथी रक्ष्यस्त्वं चापि भृशपीडितः}


\twolineshloka
{त्वं हि साल्वप्रयुक्तेन शरेणाभिहतो भृशम्}
{कश्मलाभिहतो वीर ततोऽहमपयातवान्}


\twolineshloka
{सत्वं सात्वतमुख्याद्य लब्धसंज्ञो यदृच्छया}
{पश्य मे हयसंयाने शिक्षां केशवनन्दन}


\threelineshloka
{दारुकेणाहमुत्पन्नो यथावच्चैव शिक्षितः}
{वीतभीः प्रविशाम्येतां साल्वस्य महतीं चमूम् ॥वासुदेव उवाच}
{}


\twolineshloka
{एवमुक्त्वा ततो वीरो हयान्संचोद्य संगरे}
{रश्मिभिस्तु समुद्यम्य जवेनाभ्यपतत्तदा}


\twolineshloka
{मण्डलानि विचित्राणि यमकानीतराणि च}
{सव्यानि च विचित्राणि दक्षिणानि च सर्वशः}


\twolineshloka
{प्रतोदेनाहता राजन्रश्मिभिश्च समुद्यताः}
{उत्पतन्त इवाकाशं विबभुस्ते हयोत्तमाः}


\twolineshloka
{ते हस्तलाघवोपेतं विज्ञाय नृप दारुकिम्}
{उह्यमाना इव तदा नास्पृशंश्चरणैर्महीम्}


\twolineshloka
{सोपसव्यां चमूं तस्य साल्वस्य भरतर्षभ}
{चकार नातियत्नेन तदद्भुतमिवाभवत्}


\twolineshloka
{अमृष्यमाणोपसव्यं साल्वः समितिदारुणः}
{यन्तारमस्य सहसा त्रिभिर्बाणैः समार्दयत्}


\twolineshloka
{दारुकस्य सुतस्तं तु बाणवेगमचिन्तयन्}
{भूय एव महाबाहो प्रययावपसव्यतः}


\twolineshloka
{ततो बाणान्बहुविधान्पुनरेव स सौभराट्}
{मुमोच तनये वीर मम रुक्मिणिनन्दने}


\twolineshloka
{तानप्राप्ताञ्शितैर्ब्राणैश्चिच्छेद परवीरहा}
{रौक्मिणेयः स्मितं कृत्वा दर्शयन्हस्तलाघवं}


\twolineshloka
{छिन्नान्दृष्ट्वा तु तान्बाणान्प्रद्युम्नेन च सौभराट्}
{आसुरीं दारुणीं मायामास्थाय व्यसृजच्छरान्}


\twolineshloka
{प्रयुज्यमानमाज्ञाय दैतेयास्त्रं महाबलम्}
{ब्रह्मास्त्रेणान्तरा च्छित्त्वामुमोचान्यन्पतस्त्रिणः}


\twolineshloka
{ते तदस्त्रं विधूयाशु विव्यधू रुधिराशनाः}
{शिरस्युरसि वक्रे च स मुमोह पपात च}


\twolineshloka
{तस्मिन्निपतिते क्षुद्रे साल्वे बाणप्रपीडिते}
{रौक्मिणेयोऽपरं बाणं संदधे शत्रतापनः}


\twolineshloka
{तमर्चितं सर्वदाशार्हपूगै-राशीर्भिरर्कज्वलनप्रकाशम्}
{दृष्ट्वा शरं ज्यामभिनीयमानंबभूव हाहाकृतमन्तरिक्षम्}


\twolineshloka
{ततो देवगणाः सर्वे सेन्द्राः सहधनेश्वराः}
{नारदं प्रेषयामासु- श्वसनं च मनोजवम्}


\twolineshloka
{तौ रौक्मिणेयमागम्य वचोऽब्रूतां दिवौकसाम्}
{नैव वध्यस्त्वया वीर साल्वराजः कथंचन}


\twolineshloka
{संहरस्व पुनर्बाणमवध्योऽयं त्वया रणे}
{एतस्य च शरस्याजौ नावध्योस्ति पुमान्क्वचित्}


\twolineshloka
{मृत्युरस्य महाबाहो रणे देवकिनन्दनः}
{कृष्णः संकल्पितो धात्रा तन्मिथ्या न भवेदिति}


\twolineshloka
{ततः परमसंहृष्टः प्रद्युम्नः शरमुत्तमम्}
{संजहार धनुःश्रेष्ठात्तूणे चैव न्यवेशयत्}


\twolineshloka
{तत उत्थाय राजेन्द्र साल्वः परमदुर्मनाः}
{व्यपायात्सबलस्तूर्णं प्रद्युम्नशरपीडितः}


\twolineshloka
{स द्वारकां परित्यज्य साल्वो वृष्णिभिरार्दितः}
{सौभमास्थाय राजेन्द्र दिवमाचक्रमे तदा}


\chapter{अध्यायः २०}
\twolineshloka
{वासुदेव उवाच}
{}


\twolineshloka
{आनर्तनगरं मुक्तं ततोऽहमगमं तदा}
{महाक्रतौ राजसूये निवृत्ते नृपते तव}


\twolineshloka
{अपश्यं द्वारकां चाहं महाराज हतत्विषम्}
{निःस्वाध्यायवषट्कारां निर्भूषणवरस्त्रियम्}


\twolineshloka
{अनभिज्ञेयरूपाणि द्वारकोपवनानि च}
{दृष्ट्वा शङ्कोपपन्नोऽहमपृच्छं हृदिकात्मजम्}


\twolineshloka
{अस्वस्थनरनारीकमिदं वृष्णिकुलं भृशम्}
{किमिदं नरशार्दूल श्रोतुमिच्छामि तत्त्वतः}


\twolineshloka
{एवमुक्तः स तु मया विस्तरेणेदमब्रवीत्}
{रोधं मोक्षं च साल्वेन हार्दिक्यो राजसत्तम}


\twolineshloka
{ततोऽहं भरतश्रेष्ठ श्रुत्वा सर्वमशेषतः}
{विनाशे साल्वराजस्य तदैवाकरवं मतिम्}


% Check verse!
ततोऽहं भरतश्रेष्ठ समाश्वास्य पुरे जनम् ॥राजानमाहुकं चैव तथैवाकनदुन्दुभिम्
\twolineshloka
{सर्वान्वृष्णिप्रवीरांश्च हर्षयन्नब्रुवं तदा}
{अप्रमादः सदा कार्यो नगरे यादवर्षभाः}


\twolineshloka
{साल्वराजविनाशाय प्रयातं मां निबोधत}
{नाहत्वा तं निवर्तिष्ये पुरीं द्वारवतीं प्रति}


\twolineshloka
{ससाल्वं सौभनगरं हत्वा द्रष्टास्मि वः पुनः}
{त्रिःसामा हन्यतामेषा दुन्दुभिः शत्रुभीषणा}


\twolineshloka
{ते मयाऽऽश्वासिता वीरा यथावद्भरतर्षभ}
{सर्वे मामब्रुवन्हृष्टाः प्रयाहि जहिशात्रवान्}


\twolineshloka
{तैः प्रहृष्टात्मभिर्वीरैराशीर्भिरभिनन्दितः}
{वाचयित्वा द्विजश्रेष्ठान्प्रणम्य शिरसा भवम्}


\twolineshloka
{शैब्यसुग्रीवयुक्तेन रथेनानादयन्दिशः}
{प्रध्माय शङ्खप्रवरं पाञ्चजन्यमहं नृप}


\twolineshloka
{प्रयातोस्मि नरव्याघ्र बलेन महता वृतः}
{क्लृप्तेन चतुरङ्गेण यत्तेन जितकाशिना}


\twolineshloka
{समतीत्य बहून्देशान्गिरींश्च बहुपादपान्}
{सरांसि सरितश्चैव मार्तिकावतमासदम्}


\twolineshloka
{तत्राश्रौषं नरव्याघ्र साल्वं सागरमन्तिकात्}
{प्रयान्तं सौभमास्थाय तमहं पृष्ठतोऽन्वगाम्}


\threelineshloka
{`दृष्टवानस्मि राजेन्द्र साल्वराजमथान्तिके}
{'ततः सागरमासाद्य कुक्षौ तस्य महोर्मिणः}
{समुद्रनाभ्यां साल्वोऽभूत्सौभमास्थाय शत्रुहन्}


\twolineshloka
{`स मामालोक्य सहसा सेनां स्वां प्राहिणोन्मृधे}
{मद्बाहुना च सेनायां शिष्टायां किंचिदेव च'}


\twolineshloka
{स समालोक्य सहसा स्मयन्निव युधिष्ठिर}
{आह्वयामास दुष्टात्मा युद्धायैव मुहुर्मुहुः}


\twolineshloka
{तस्य शार्ङ्गविनिर्मुक्तैर्बहुभिर्मर्मभेदिभिः}
{पुरं नासाद्यत शरैस्ततो मां रोष आविशत्}


\twolineshloka
{स चापि पापप्रकृतिर्दैतेयापशदो नृप}
{मय्यवर्षत दुर्धर्षः शरधाराः सहस्रशः}


\twolineshloka
{सैनिकान्मम सूतं च हयांश्च स समाकिरत्}
{अचिन्तयन्तस्तु शरान्वयं युध्याम भारत}


\twolineshloka
{ततः शतसहस्राणि शराणां नतपर्वणाम्}
{चिक्षिपुः समरे वीरा मयि साल्वपदानुगाः}


\twolineshloka
{ते हयांश्च रथं चैव ध्वजं दारुकमेव च}
{छादयामासुरसुरास्तैर्बाणैर्मर्मभेदिभिः}


\twolineshloka
{न हया न रथो वीर न ध्वजो न च दारुकः}
{अदृश्यन्त शरैश्छन्नास्तथाऽहं सैनिकाश्च मे}


\twolineshloka
{ततोऽहमपि कौन्तेय शराणामयुतान्बहून्}
{आमन्त्रितानां धनुषा दिव्येन विधिनाऽक्षिपम्}


\twolineshloka
{न तत्रविषयस्त्वासीन्मम सैन्यस्य भारत}
{खे विषक्तं हि तत्सौभं क्रोशमात्र इवाभवत्}


\twolineshloka
{ततस्ते प्रेक्षकाः सर्वे देवा वै दिवमास्थिताः}
{हर्षयामासुरुच्चैर्मां सिंहनादतलस्वनैः}


\twolineshloka
{मत्कार्मुकविनिर्मुक्ता दानवानां महारणे}
{अङ्गेषु रुधिराक्तास्ते विविशुः शलभा इव}


\twolineshloka
{ततो हलहलाशब्दः सौभमध्ये व्यवर्धत}
{वध्यतां विशिखैस्तीक्ष्णैः पततां च महार्णवे}


\threelineshloka
{ते निकृत्तभुजस्कन्धाः कबन्धाकृतिदर्शनाः}
{नदन्तो भैरवान्नादान्निपतन्ति स्म दानवाः}
{पतितास्तेऽपि भक्ष्यन्ते समुद्रांभोनिवासिभिः}


\twolineshloka
{ततो गोक्षीरकुन्देन्दुमृणालरजतप्रभम्}
{जलजं पाञ्चजन्यं वै प्राणेनाहमपूरयम्}


\twolineshloka
{तान्दृष्ट्वा पतितांस्तत्र साल्वः सौभपतिस्ततः}
{मायायुद्धेन महता योधयामास मां युधि}


\threelineshloka
{ततो गदा हलाः प्रासाः शूलशक्तिपरश्चथाः}
{असयः शक्तिकुलिशपाशर्ष्टिकनपाः शराः}
{पट्टसाश्च भुशुण्ड्यश्च प्रपतन्त्यनिशं मयि}


\twolineshloka
{तामहं माययैवाशु प्रतिगृह्य व्यनाशयम्}
{तस्यां हतायां मायायां गिरिशृङ्गैरयोधयत्}


\twolineshloka
{ततोऽभवत्तम इव प्रकाश इव चाभवत्}
{दुर्दिनं सुदिनं चैव शीतमुष्णं च भारत}


\twolineshloka
{अङ्गारपांसुवर्षं च शस्त्रवर्षं च भारत}
{एवं मायां प्रकुर्वाणो योधयामास मां रिपुः}


\twolineshloka
{विज्ञाय तदहं सर्वं माययैव व्यनाशयम्}
{यथाकालं तु युद्धेन व्यधमं सर्वतः शरैः}


\twolineshloka
{`ततो हतायां च मया मायायां युधि दानवः}
{मायामन्यां महाराज चकार मतिमोहिनीम् ॥'}


\twolineshloka
{ततो व्योम महाराज शतसूर्यमिवाभवत्}
{शतचन्द्रं च कौन्तेय सहस्रायुततारकम्}


\twolineshloka
{ततो नाज्ञायत तदा दिवारात्रं तथा दिशः}
{ततोऽहं मोहमापन्नः प्रज्ञास्त्रं समयोजयम्}


\threelineshloka
{तदस्त्रमस्तमस्त्रेण विधूतं शरतूलवत्}
{तथा तदभवद्युद्धं तुमुलं रोमहर्षणम्}
{लब्धालोकस्तु राजेन्द्र पुनः शत्रुमयोधयम्}


\chapter{अध्यायः २१}
\twolineshloka
{वासुदेव उवाच}
{}


\twolineshloka
{एवं स पुरुषव्याघ्रः साल्वो राज्ञं महारिपुः}
{युध्यमानो मया सङ्ख्ये वियदभ्यगमत्पुनः}


\twolineshloka
{ततः शतघ्नीश्च महागदाश्चदीप्ताश्च शूलान्मुसलानसींश्च}
{चिक्षेप रोषान्मयि मन्दबुद्धिःसाल्वो महाराज जयाभिकाङ्क्षी}


\twolineshloka
{तानाशुगैरापततोऽहमाशुनिवार्य तूर्णं खगमान्ख एव}
{द्विधा त्रिता चाच्छिनमाशु मुक्तै-स्ततोऽन्तरिक्षे निनदो बभूव}


\twolineshloka
{ततः शतसहस्रेण शराणां नतपर्वणाम्}
{दारुकं वाजिनश्चैव रथं च समवाकिरत्}


\threelineshloka
{ततो मामब्रवीद्वीर दारुको विह्वलन्निव}
{स्थातव्यमिति तिष्ठामि साल्वबाणप्रपीडितः}
{अवस्थातुं न शक्नोमि अङ्गं मे व्यवसीदति}


\twolineshloka
{इति तस्य निशम्याहं सारथेः करुणं वचः}
{अवेक्षमाणो यन्तारमपश्यं शरपीडितम्}


\twolineshloka
{न तस्योरसि नो मूर्ध्नि न काये न भुजद्वये}
{अन्तरं पाण्डवश्रेष्ठ पश्याम्यनिचितं शरैः}


\twolineshloka
{स तु बाणवरोत्पीडाद्विस्रवत्यसृगुल्बणम्}
{अभिवृष्टो यथा मेघैर्गिरिर्गैरिकधातुमान्}


\twolineshloka
{अभीशुहस्तं तं दृष्ट्वा सीदन्तं सारथिं रणे}
{अस्तम्भयं महाबाहो साल्वबाणप्रपीडितम्}


\twolineshloka
{अथ मां पुरुषः कश्चिद्द्वारकानिलयोऽब्रवीत्}
{त्वरितो रथमारोप्य सौहृदादिव भारत}


\twolineshloka
{आहुकस्य वचो वीर तस्यैव परिचारकः}
{विषण्णः सन्नकण्ठेन तन्निबोध युधिष्ठिर}


\twolineshloka
{द्वारकाधिपतिर्वीर आह त्वामाहुको वचः}
{केशवैहि विजानीष्व यत्त्वां पितृसखोऽब्रवीत्}


\twolineshloka
{उपयात्वा तु साल्वेन द्वारकां वृष्णिनन्दन}
{विषक्ते त्वयि दुर्धर्ष इतः शूरसुतो बलात्}


\twolineshloka
{तदलं साधुयुद्धेन निवर्तस्व जनार्दन}
{द्वारकामेव रक्षस्व कार्यमेतन्महत्तव}


\twolineshloka
{इत्यहं तस्य वचनं श्रुत्वा परमदुर्मनाः}
{निश्चयं नाधिगच्छामि कर्तव्यस्येतरस्य च}


\twolineshloka
{सात्यकिं बलदेवं च प्रद्युम्नं च महांरथम्}
{जगर्हे मनसा वीर तच्छ्रुत्वा महदप्रियम्}


\twolineshloka
{अहं हि द्वारकायाश्च पितुश्च कुरुनन्दन}
{तेषु रक्षां समाधाय प्रयातः सौभयातने}


\threelineshloka
{बलदेवो महाबाहुः कच्चिज्जीवति शत्रुहा}
{सात्यकी रौक्मिणेयश्च चारुदेष्णश्च वीर्यवान्}
{साम्बप्रभृतयश्चैवेत्यहमासं सुदुर्मनाः}


\twolineshloka
{एतेषु हि नरव्याघ्र जीवत्सु न कथंचन}
{शक्यः शूरसुतो हन्तुमपि वज्रभृता स्वयम्}


\twolineshloka
{हतः शूरसुतो व्यक्तं व्यक्तं चैते परासवः}
{बलदेवमुखाः सर्व इति मे निश्चिता मतिः}


\twolineshloka
{सोहं सर्वविनाशं तं चिन्तयानो मुहुर्मुहुः}
{अविह्वलो महाराज पुनः साल्वमयोधयम्}


\twolineshloka
{तदोऽपश्यं महाराज प्रपतन्तमहं तदा}
{सौभाच्छूरसुतं वीर ततो मां मोह आविशत्}


\twolineshloka
{तस्य रूपं प्रपततः पितुर्मम नराधिप}
{ययातेः क्षीणपुण्यस्य स्वर्गादिव भहीतलम्}


\twolineshloka
{विशीर्णमलिनोष्णीपप्रकीर्णाम्बरमूर्धजः}
{प्रपतन्दृश्यते ह स्म क्षीणपुण्य इव ग्रहः}


\twolineshloka
{ततः शार्ङ्गं धनुःश्रेष्ठं करात्प्रपतितं मम}
{मोहापन्नश्च कौन्तेय रथोपस्थ उपाविशम्}


\twolineshloka
{ततो हाहाकृतं सर्वं सैन्यं मे गतचेतनम्}
{मां दृष्ट्वा रथनीडस्थं गतासुमिव भारत}


\twolineshloka
{प्रसार्य बाहू पततः प्रसार्य चरणावपि}
{रूपं पितुर्मे विबभौ शकुने- पततो यथा}


\twolineshloka
{तं पतन्तं महाबाहो शूलपट्टसपाणयः}
{अभिघ्नन्तो भृशं वीर मम चोतोह्यकम्पयन्}


\twolineshloka
{ततो मुहूर्तात्प्रतिलभ्य संज्ञा-महं तदा वीर महाविमर्दे}
{न तत्रसौमं न रिपुं च साल्वंपश्यामि वृद्धं पितरं न चापि}


\twolineshloka
{ततो ममासीन्मनसि मायेयमिति निश्चितम्}
{प्रबुद्धोस्मि ततो भूयः शतशो वाऽकिरं शरान्}


\chapter{अध्यायः २२}
\twolineshloka
{वासुदेव उवाच}
{}


\twolineshloka
{ततोऽहं भरतश्रेष्ठ प्रगृह्यरुचिरं धनुः}
{शरैरपातयं सौभाच्छिरांसि विबुधद्विषाम्}


\twolineshloka
{शरांश्चाशीविषाकारानूर्ध्वगांस्तिग्मतेजसः}
{प्रैषयं साल्वराजाय शार्ङ्गमुक्तान्सुवाससः}


\twolineshloka
{ततो नादृश्यत तदा सौभं कुरुकुलोद्वह}
{अन्तर्हितं माययाऽभूत्ततोऽहं विस्मितोऽभवम्}


\twolineshloka
{अथ दानवसङ्घास्ते विकृताननमूर्धजाः}
{उदक्रोशन्महाराज धिष्ठिते मयि भारत}


\twolineshloka
{ततोऽस्त्रं शब्दसाहं वै त्वरमाणो महारणे}
{अयोजयं तद्वधाय ततः शब्द उपारमत्}


\twolineshloka
{हतास्ते दानवाः सर्वे यैः स शब्द उदीरितः}
{शरैरादित्यसंकशैर्ज्वलितैः शब्दसाधनैः}


\twolineshloka
{तस्मिन्नुपरते शब्दे पुनरेवान्यतोऽभवत्}
{शब्दोऽपरो महाराज तत्रापि प्राहरं शरैः}


\twolineshloka
{एवं दश दिशः सर्वास्तिर्यगूर्ध्वं च भारत}
{नादयामासुरसुरास्ते चापि निहता मया}


\twolineshloka
{ततः प्राग्ज्योतिषं गत्वा पुनरेव व्यदृश्यत}
{सौभं कामगमं वीर मोहयन्मम चक्षुषी}


\twolineshloka
{ततो लोकान्तकरणो दानवो दारुणाकृतिः}
{शिलावर्षेण महता सहसा मां समावृणोत्}


\twolineshloka
{सोहं पर्वतवर्षेण वध्यमानः समन्ततः}
{वल्मीक इव राजेन्द्र पर्वतोपचितोऽभवम्}


\twolineshloka
{ततोऽहं पर्वतचितः सहयः सहसारथिः}
{अप्रख्यातिमियां राजन्सध्वजः पर्वतैश्चितः}


\twolineshloka
{ततो वृष्णिप्रवीरा ये ममासन्सैनिकास्तदा}
{ते भयर्ता दिशः सर्वे सहसा विप्रदुद्रुवुः}


\twolineshloka
{ततो हाहाकृतमभूत्सर्वं किल विशांपते}
{द्यौश्च भूमिश्च खं चैवादृश्यमाने तथा मयि}


\twolineshloka
{ततो विषण्णमनसो मम राजन्सुहृज्जनाः}
{रुरुदुश्चुक्रुशुश्चैव दुःखशोकसमन्विताः}


\twolineshloka
{द्विषतां च प्रहर्षोऽभूदार्तिश्च सुहृदामपि}
{एवं विजितवान्वीर पश्चादश्रौषमच्युत}


\twolineshloka
{ततोऽहमिन्द्रदयितं सर्वपाषाणभेदनम्}
{वज्रमुद्यम्य तान्सर्वान्पर्वतान्समशातयम्}


\twolineshloka
{ततः पर्वतभारार्ता मन्दप्राणविचेष्टिताः}
{हया मम महाराज वेपमाना इवाभवन्}


\twolineshloka
{मेघजाल इवाकाशे विदार्याभ्युदितं रविम्}
{दृष्ट्वा मां बान्धवाः सर्वे हर्षमाहारयन्पुनः}


\twolineshloka
{ततः पर्वतभारार्तान्मन्दप्राणविचेष्टितान्}
{हयान्संदृश्य मां सूतः प्राह तात्कालिकं वचः}


\twolineshloka
{साधु संपश्य वार्ष्णेय साल्वं सौभपतिं स्थितम्}
{अलं कृष्णावमन्यैनं साधु यत्नं समाचर}


\twolineshloka
{मार्दवं सखितां चैव साल्वादद्य व्यपाहर}
{जहि साल्वं महाबाहो मैनं जीवय केशव}


\threelineshloka
{सर्वैः पराक्रमैर्वीर वध्यः शत्रुरमित्रहन्}
{न शत्रुरवमन्तव्यो दुर्बलोऽपि बलीयसा}
{योपिस्यात्पीठगः कश्चित्किं पुनः समरे स्थितः}


\twolineshloka
{स त्वं पुरुषशार्दूल सर्वयत्नैरिमं प्रभो}
{जहि वृष्णिकुलश्रेष्ठ मा त्वां कालोऽत्यगात्पुनः}


\twolineshloka
{जितवाञ्जामदग्न्यं यः कोटिवर्षगणान्बहून्}
{स एष नान्यैर्वध्यो हि त्वामृते नास्ति कश्चन}


\twolineshloka
{नैष मार्दवसाध्यो वै मतो नापि सखा तव}
{येन त्वं योधितो वीर द्वारका चावमर्दिता}


\twolineshloka
{एवमादि तु कौन्तेय श्रुत्वाऽहं सारथेर्वचः}
{तत्वमेतदिति ज्ञात्वा युद्धे मतिमधारयम्}


\twolineshloka
{वधाय साल्वराजस्य सौभस्य च निपातने}
{दारुकं चाब्रुवं वीर मुहूर्तं स्थीयतामिति}


\threelineshloka
{ततोऽप्रतिहतं दिव्यमभोद्यमतिवीर्यवत्}
{आग्नेयमस्त्रं दयितं सर्वसाहं महाप्रभम्}
{योजयं तत्रधनुषा दानवान्तकरं रणे}


\twolineshloka
{यक्षाणां राक्षसानां च दानवानां च संयुगे}
{राज्ञां च प्रतिलोमानां भस्मान्तकरणं महत्}


\twolineshloka
{क्षुरान्तममलं चक्रं कालान्तकयमोपमम्}
{अनुमन्त्र्याहमतुलं द्विषतां विनिबर्हणम्}


\twolineshloka
{जहि सौभं स्ववीर्येण ये चात्र रिपवो मम}
{इत्युक्त्वा भुजवीर्येण तस्मै प्राहिणवं रुषा}


\twolineshloka
{रूपं सुदर्शनस्यासीदाकाशे पततस्तदा}
{द्वितीयस्येव सूर्यस् युगान्ते प्रतपिष्यतः}


\twolineshloka
{तत्समासाद्य नगरं सौभं व्यपगतत्विषम्}
{मध्येन पाटयामास क्रकचो दार्विवोच्छ्रितम्}


\twolineshloka
{द्विधा कृतं ततः सौभं सुदर्शनबलाद्धतम्}
{महेश्वरशरोद्धूतं पपात त्रिपुरं यथा}


\twolineshloka
{तस्मिन्निपतिते सौभे चक्रमागात्करं मम}
{पुनश्चादाय वेगेन साल्वायेत्यहमब्रुवम्}


\twolineshloka
{ततः साल्वं गदां गुर्वीमाविध्यन्ते महाहवे}
{द्विधा चकार सहसा प्रजज्वाल च तेजसा}


\twolineshloka
{तस्मिन्विनिहते वीरे दानवास्त्रस्तचेतसः}
{हाहाभूता दिशो जग्मुरर्दिता मम सायकैः}


\twolineshloka
{ततोऽहंसमवस्थाप्यरथं सौभसमीपतः}
{शङ्खं प्रध्माप्य हर्षेण सुहृदः पर्यहर्षयम्}


\twolineshloka
{तन्मेरुशिखराकारं विध्वस्ताट्टालगोपुरम्}
{दह्यमानमभिप्रेक्ष्य स्त्रियस्ताः संप्रदुद्रुवुः}


\twolineshloka
{एवं निहत्य समरे सौभं साल्वं निपात्य च}
{आनर्तात्पुनरागम्य सुहृदां प्रीतिमावहम्}


\twolineshloka
{तदेतत्कारणं राजन्नागमं नागसाह्वयम्}
{यद्यागां परवीरघ्न न हि जीवेत्सुयोधनः}


\threelineshloka
{मय्यागतेऽथवा वीर द्यूतं न भविता तथा}
{अद्याहं किं करिष्यामि भिन्नसेतुरिवोदकम् ॥वैशंपायन उवाच}
{}


\twolineshloka
{एवमुक्त्वा महाबाहुः कौरवं पुरुषोत्तमः}
{आमन्त्र्य प्रययौ श्रीमान्पाण्डवान्मधुसूदनः}


\twolineshloka
{अभिवाद्य महाबाहुर्धर्मराजं युधिष्ठिरम्}
{राज्ञा मूर्धन्युपाघ्रातो भीमेन च महाभुजः}


\twolineshloka
{परिष्वक्तश्चार्जुनेन यमाभ्यां चाभिवादितः}
{संमानितश्च धौम्येन द्रौपद्या चार्चितोश्रुभिः}


\twolineshloka
{सुभद्रामभिमन्युं च रथमारोप्य काञ्चनम्}
{आरुरोह रथं कृष्णः पाण्डवैरभिपूजितः}


\twolineshloka
{शैब्यसुग्रीवयुक्तेन रथेनादित्यवर्चसा}
{द्वारकां प्रययौ कृष्णः समाश्वास्य युधिष्ठिरम्}


\twolineshloka
{ततः प्रयते दाशार्हे धृष्टद्युम्नोपि पार्षतः}
{द्रौपदेयानुपादाय प्रययौ स्वपुरं तदा}


\twolineshloka
{धृष्टकेतुः स्वसारं च समादायाथ चेदिराट्}
{जगाम पाण्डवान्दृष्ट्वा रम्यां शुक्तिमतीं पुरीम्}


\twolineshloka
{केकयाश्चाप्यनुज्ञाताः कौन्तेयेनामितौजसा}
{आमन्त्र्य पाण्डवान्सर्वान्प्रययुस्तेऽपि भारत}


\twolineshloka
{ब्राह्मणाश्च विशश्चैव तथाविषयवासिनः}
{विसृज्यमानाः सुभृशं न त्यजन्ति स्म पाण्डवान्}


\twolineshloka
{समवायः स राजेन्द्र सुमहाद्भुतदर्शनः}
{आसीन्महात्मनां तेषां काम्यके भरतर्षभ}


\twolineshloka
{युधिष्ठिरस्तु विप्रांस्ताननुमान्य महामनाः}
{शशास पुरुषान्काले रथान्योजयतेति वै}


\chapter{अध्यायः २३}
\twolineshloka
{वैशंपायन उवाच}
{}


\twolineshloka
{तस्मिन्दशार्हाधिपतौ प्रयातेयुधिष्ठिरो भीमसेनार्जुनौ च}
{यमौ च कृष्ण च पुरोहितश्चरथान्महार्हान्परमाश्वयुक्तान्}


\twolineshloka
{अस्थाय वीराः सहिता वनायप्रतस्थिरे भूतपतिप्रकाशाः}
{हिरण्यनिष्कान्वसनानि गाश्चगदाय शिक्षाक्षरमन्त्रविद्भ्यः}


\twolineshloka
{प्रेष्याः पुरो विंशतिरात्तशस्त्राधनूंषि शस्त्राणि शरांश्च दीप्तान्}
{मौर्वीश्च यन्त्राणि च सायकांश्चसर्वे समादाय जघन्यमीयुः}


\twolineshloka
{तत्सतु वासांसि च राजपुत्र्याधात्र्यश्च दास्यश्च विभूषणं च}
{तदिन्द्रसेनस्त्वरितः प्रगृह्यजघन्यमेवोपययौ रथेन}


\twolineshloka
{ततः कुरुश्रेष्ठमुपेत्य पौराःप्रदक्षिणं चक्रुरदीनसत्वाः}
{तं ब्राह्मणाश्चाभ्यवदन्प्रसन्नामुख्याश्च सर्वे कुरुजाङ्गलानाम्}


\twolineshloka
{स चापि तानभ्यवदत्प्रसन्नःसहैव तैर्भ्रातृभिर्धर्मराजः}
{तस्थौ च तत्राधिपतिर्महात्मादृष्ट्वा जनौघं कुरुजाङ्गलानाम्}


\twolineshloka
{पितेव पुत्रेषु स तेषु भावंचक्रे कुरूणामृषभो महात्मा}
{ते चापि तस्मिन्भरतप्रबर्हेतथा बभूवुः पितरीव पुत्राः}


\twolineshloka
{ततस्तमासाद्य महाजनौघाःकुरुप्रवीरं परिवार्य तस्थुः}
{हानाथ हाधर्म इति ब्रुवाणाहीताश्च सर्वेऽऽश्रुमुखा बभूवुः}


\twolineshloka
{वरः कुरूणामधिपः प्रजानांपितेव पुत्रानपहाय चास्मान्}
{पौरानिमाञ्जानपदांश्च सर्वान्हित्वा प्रयातः क्वनु धर्मराजः}


\twolineshloka
{धिग्धार्तराष्ट्रं सुनृशंसबुद्धिंधिक्सौबलं पापमतिं च कर्णम्}
{अनर्थमिच्छन्ति नरेन्द्र पापाये धर्मनित्यस्य सतस्तवोग्राः}


\twolineshloka
{स्वयं निवेश्याप्रतिमं महात्मापुरं महादेवपुरप्रकाशम्}
{शतक्रतुप्रस्थममोघकर्माहित्वा प्रयातः क्वनु धर्मराजः}


\twolineshloka
{चकार यामप्रतिमां महात्मासभां मयो देवसभाप्रकाशाम्}
{तां देवगुप्तामिव देवमायांहित्वाप्रयातः क्वनु धर्मराजः}


\twolineshloka
{तान्धर्मकामार्थविदुत्तमौजाबीभत्सुरुच्चैः सहितानुवाच}
{आदास्यते वासमिमं निरुष्यरवनेषु राजा द्विषतां यशांसि}


\twolineshloka
{द्विजातिमुख्याः सहिताः पृथक्वभवद्भिरासाद्य तपस्विनश्च}
{प्रसाद्य धर्मार्थविदश्च वाच्यायथार्थसिद्धिः परमा भवेन्नः}


\twolineshloka
{इत्येवमुक्ते वचनेऽऽर्जुनेनते ब्राह्मणाः सर्ववर्णाश्च राजन्}
{मुदाऽभ्यनन्दन्सहिताश्च चक्रुःप्रदक्षिणं धर्मभृतांवरिष्ठम्}


\twolineshloka
{आमन्त्र्य पार्थं च वृकोदरं चधनंजयं याज्ञसेनीं यमौ च}
{प्रतस्थिरे राष्ट्रमरपेतहार्षायुधिष्ठिरेणानुमता यथास्वम्}


\chapter{अध्यायः २४}
\twolineshloka
{वैशंपायन उवाच}
{}


\twolineshloka
{ततस्तेषु प्रयातेषु कौन्तेयः सत्यसंगरः}
{अभ्यभाषत धर्मात्मा भ्रातॄन्सर्वान्युधिष्ठिरः}


\twolineshloka
{द्वादशेमाः समाऽस्माभिर्वस्तव्यं निर्जने वने}
{समीक्षध्वं महारण्ये देशं बहुमृगद्विजम्}


\twolineshloka
{बहुपुष्पफलं रम्यं शिवं पुण्यजनोचितम्}
{यत्रेमा द्वादश समाः सुखं प्रतिवसेमहि}


\twolineshloka
{एवमुक्ते प्रत्युवाच धर्मराजं धनंजयः}
{गुरुवन्मानववरं मानयित्वा मनस्विनम्}


\twolineshloka
{भवानेव महर्षीणां वृद्धानां पर्युपासिता}
{अज्ञातं मानुषे लोके भवतो नास्ति किंचन}


\twolineshloka
{त्वया ह्युपासिता नित्यं ब्राह्मणा वेदपारगाः}
{द्वैपायनप्रभृतयो नारदश्च महातपाः}


\twolineshloka
{यः सर्वलोकद्वाराणि नित्यं संचरते वशी}
{देवलोकाद्ब्रह्मलोकं गन्धर्वाप्सरसामपि}


\twolineshloka
{अनुभावांश्च जानासि ब्राह्मणानां न संशयः}
{प्रभावांश्चैव वेत्थ त्वं सर्वेषामेव पार्थिव}


\twolineshloka
{त्वमेव राजञ्जानासि श्रेयःकारणमेव च}
{यत्रेच्छसि महाराज निवासं तत्र कुंर्महे}


\twolineshloka
{इदं द्वैतवनं नाम सरः पुण्यजनोषितम्}
{बहुपुष्पफलं रम्यं नानाद्विजनिषेवितम्}


\threelineshloka
{अत्रेमा द्वादश समा विहरेमेति रोचये}
{यदि तेऽनुमतं राजन्किमन्यन्मन्यते भवान् ॥युधिष्ठिर उवाच}
{}


\threelineshloka
{ममाप्येतन्मतं पार्थ त्वया यत्समुदाहृतम्}
{गच्छामः पुण्यविख्यातं महद्द्वैतवनं सरः ॥वैशंपायन उवाच}
{}


\twolineshloka
{ततस्ते प्रययुः सर्वे पाण्डवा धर्मचारिणः}
{ब्राह्मणैर्बहुभिः सार्धं पुण्यं द्वैतवनं सरः}


\twolineshloka
{ब्राह्मणाः साग्निहोत्राश्च तथैव च निरग्नयः}
{स्वाध्यायिनो भिक्षवश्च तथैव वनवासिनः}


\twolineshloka
{बहवो ब्राह्मणास्तत्र परिवव्रुर्युधिष्ठिरम्}
{तपस्विनः सत्यशीलाः शतशः संशितव्रताः}


\twolineshloka
{ते यात्वा पाण्डवास्तत्रब्राह्मणैर्बहुभिः सह}
{पुण्यं द्वैतवनं रम्यं विविशुर्बरतर्षभाः}


\twolineshloka
{तत्सालतालाम्रमधूकनीप-कदम्बसर्जार्जुनकर्णिकारैः}
{तपात्यये पुष्पधरैरुपेतंमहावनं राष्ट्रपतिर्ददर्श}


\twolineshloka
{महाद्रुमाणां शिखरेषु तस्थु-र्मनोरमां वाचमुदीरयन्तः}
{मयूरदात्यूहचकोरसङ्घा-स्तस्मिन्वने बर्हिणकोकिलाश्च}


\twolineshloka
{करेणुयूथैः सह यूथपानांमदोत्कटानामचलप्रभाणाम्}
{महान्ति यूथानि महाद्विपानांतस्मिन्वने राष्ट्रपतिर्ददर्श}


\twolineshloka
{मनोरमां भोगवतीमुपेत्यपूतात्मनां चीरजटाधराणाम्}
{तस्मिन्वने धर्मभृतां निवासेददर्श सिद्धर्षिगणाननेकान्}


\twolineshloka
{ततः स यानादवरुह्य राजासभ्रातृकः सजनः काननं तत्}
{विवेश धर्मात्मवतां बरिष्ठ-स्त्रिविष्टपं शक्र इवामितौजाः}


\twolineshloka
{तं सत्यसन्धं सहिताऽभिपेतु-र्दिदृक्षवश्चारणसिद्धसङ्घाः}
{वनौकसश्चापि नेरन्द्रसिंहंमनस्विनं तं परिवार्य तस्थुः}


\twolineshloka
{स तत्रवृद्धानभिवाद्य सर्वान्प्रत्यर्चितो राजवद्देववच्च}
{विवेश सर्वैः सहितो द्विजाग्र्यैःकृताञ्जलिर्धर्मभृतां वरिष्ठः}


\twolineshloka
{स पुण्यशीलः पितृवन्महात्मातपस्विभिर्मपरैरुपेत्य}
{प्रत्यर्चितः पुष्पधरस्य मूलेमहाद्रुमस्योपविवेश राजा}


\twolineshloka
{भीमश्च कृष्णा च धनंजयश्चयमौ च ते चानुचरा नरेन्द्रम्}
{विमुच्यवाहानवशाश्च सर्वेतत्रोपतस्थुर्भरतप्रबर्हाः}


\twolineshloka
{लतावतानावनतः स पाण्डवै-र्महाद्रुमः पञ्चभिरेव धन्विभिः}
{बभौ निवासोपगतैर्महात्मभि-र्महागिरिर्वारणयूथपैरिव}


\chapter{अध्यायः २५}
\twolineshloka
{वैशंपायन उवाच}
{}


\twolineshloka
{तत्काननं प्राप्य नरेन्द्रपुत्राःसुखोचिता वासमुपेत्य कृच्छ्रम्}
{विजह्रुरिनद्रप्रतिमाः शिवेषुसरस्वतीसालवनेषु तेषु}


\twolineshloka
{यतींश्च राजा स मुनींश्च सर्वांस्तस्मिन्वने मूलफलैरुदग्रैः}
{द्विजातिमुख्यानृषभः कुरूणांसंतर्पयामास महानुभावः}


\twolineshloka
{इष्टीश्च पित्र्याणि तथा क्रियाश्चमहावने वसतां पाण्डवानाम्}
{पुरोहितस्तत्र समृद्धतेजा-श्चकार धौम्यः पितृवन्नृपाणाम्}


\twolineshloka
{अपेत्य राष्ट्राद्वसतां तु तेषा-मृषिः पुराणोऽतिथिराजगाम}
{तमाश्रमं तीव्रसमृद्धतेजामार्कण्डेयः श्रीमतां पाण्डवानाम्}


\twolineshloka
{तमागतं ज्वलितहुताशनप्रभंमहामनाः कुरुवृषभो युधिष्ठिरः}
{अपूजयत्सुरऋषिमानवार्चितंमहामुनिं ह्यनुपमसत्ववीर्यवान्}


\twolineshloka
{स सर्वविद्द्रौपदीं वीक्ष्य दीनांयुधिष्टिरं भीमसेनार्जुनौ च}
{संस्मृत्यरामं मनसा महात्मातपस्विमध्येऽस्मयतामितौजाः}


\threelineshloka
{तं धर्मराजो विमना इवाब्रवी-त्सर्वे ह्रिया सन्ति तपस्विनोऽमी}
{भवानिदं किं स्मयतीव हृष्ट-स्तपस्विनां पश्यतां मामुदीक्ष्य ॥मार्कण्डेय उवाच}
{}


\twolineshloka
{न तात हृष्यामि न च स्मयामिप्रहर्षजो मां भजतेन दर्पः}
{तवापदं त्वद्य समीक्ष्य रामंसत्यव्रतं दाशरथिं स्मरामि}


\twolineshloka
{स चापि राजा सह लक्ष्मणेनवने निवासं पितुरेव शासनात्}
{धन्वी चरन्पार्थ मयैव दृष्टोगिरे पुरा ऋष्यमूकस्य सानौ}


\twolineshloka
{सहस्रनेत्रप्रतिमो महात्मायमस्य नेता नमुचेश्च हन्ता}
{पितुर्निदेशादनघः स्वधर्मंचरन्वने दाशरथिश्चकार}


\twolineshloka
{स चापि शक्रस् समप्रभावोमहानुभावः समरेष्वजेयः}
{विहाय भोगानचरद्वनेषुनेशे बलस्येति चरेदधर्मम्}


\twolineshloka
{भापाश्च नाभागभगीरथादयोमहीमिमां सागरान्तां विजित्य}
{सत्येन तेऽप्यजयस्तात लोका-न्नेशे बलस्येति चरेदधर्मम्}


\twolineshloka
{अलर्कमाहुर्नवर्य सन्तंसत्यव्रतं काशिकरूशराजम्}
{विहाय राज्यानि वसूनि चैवनेशे बलस्येति चरेदधर्मम्}


\twolineshloka
{धात्रा विधिर्यो विहितः पुराणै-स्तं पूजयन्तो नरवर्य सन्तः}
{सप्तर्षयः पार्थ दिवि प्रभान्तिनेशे बलस्येति चरेदधर्मम्}


\twolineshloka
{महाबलान्पर्वतकूटमात्रा-न्विषाणिनः पश्य गजान्नरेन्द्र}
{स्थितान्निदेशे नरवर्य धातु-र्नेशे बलस्येति चरेदधर्मम्}


\twolineshloka
{सर्वाणि भूतानि नरेन्द्र पश्यतथा यथावद्विहितं विधात्रा}
{स्वयोनितः कर्म सदाचरन्तिनेसे बलस्येति चरेदधर्मम्}


\twolineshloka
{सत्येनं धर्मेण यथार्हवृत्त्याह्रिया तथा सर्वभूतान्यतीत्य}
{यशश्च तेजश्च तवापि दीप्तंविभावसोर्भास्करस्येव पार्त}


\threelineshloka
{यथाप्रतिज्ञं च महानुभावकृच्छ्रं वने वासमिमं निरूष्य}
{ततः श्रियं तेजसा तेन दीप्ता-मादास्वसे पार्थिव कौरेवभ्यः ॥वैशंपायन उवाच}
{}


\twolineshloka
{तमेवमुक्त्वा वचनं महर्षि-स्तपस्विमध्ये सहितं सुहृद्धिः}
{आमन्त्र्य धौम्यं सहितांश्च पार्थां-स्ततः प्रतस्थे दिशमुत्तरां सः}


\chapter{अध्यायः २६}
\twolineshloka
{वैशंपायन उवाच}
{}


\twolineshloka
{वसत्सु वै द्वैतवने पाण्डवेषु महात्मसु}
{अनुकीर्णं महारण्यं ब्राह्मणैः समपद्यत}


\twolineshloka
{ईर्यमाणेन सततं ब्रह्मघोषेण सर्वशः}
{ब्रह्मलोकसमं पुण्यमासीद्द्वैतवनं सरः}


\twolineshloka
{यजुषामृचां साम्नां च गद्यानां चैव सर्वशः}
{आसीदुच्चार्यमाणानां निःस्वनो हृदयंगमः}


\twolineshloka
{ज्याघोषश्चैव पार्थानां ब्रह्मघोषश्च धीमताम्}
{संसृष्टं ब्रह्मणा क्षत्रं भूय एव व्यरोचत}


\twolineshloka
{अथाब्रवीद्बको दाल्भ्यो धर्मराजं युधिष्ठिरम्}
{सन्ध्यां कौन्तेयमासीनमृषिभिः परिवारितम्}


\twolineshloka
{पश्य द्वैतवने पार्थ ब्राह्मणानां तपस्विनाम्}
{होमवेलां कुरुश्रेष्ठ संप्रज्वलितपावकाम्}


\twolineshloka
{चरन्ति धर्मं पुण्येऽस्मिंस्त्वया गुप्ता धृतव्रताः}
{भृगवोऽङ्गिरसश्चैव वासिष्ठाः काश्यपैः सह}


\twolineshloka
{आगस्त्याश्च महाभागा आत्रेयाश्चोत्तमव्रताः}
{सर्वस्य जगतः श्रेष्ठा ब्राह्मणाः संगतास्त्वया}


\twolineshloka
{इदं तु वचनं पार्थ शृण्वेकाग्रमना मम}
{भ्रातृभिः सह कौन्तेय यत्त्वां वक्ष्यामि कौरव}


\twolineshloka
{ब्राह्म क्षत्रेण संसृष्टं क्षत्रं च ब्रह्मणा सह}
{उदीर्णे दहतः शत्रून्वनानीवाग्निमारुतौ}


\twolineshloka
{नाब्राह्मणस्तात चिरं बुभूषे-दिच्छन्नमं लोकममुं च जेतुम्}
{विनीतधर्मार्थमपेतमोहंलब्ध्वा द्विजं हन्ति नृपः सपत्नान्}


\twolineshloka
{चरन्नैश्रेयसं धर्मं प्रजापालनकारितम्}
{नाध्यगच्छद्बलिर्लोके तीर्थमन्यत्र वै द्विजात्}


\twolineshloka
{अनूनमासीदसुरस्य कामै-वैरोचनेः श्रीरपि चाक्षयाऽऽसीत्}
{लब्ध्वा महीं ब्राह्मणसंप्रयोगा-त्तेष्वाचरन्दुष्टमयथो व्यनश्यत्}


\twolineshloka
{नाब्राह्मणं भूमिरियं सभूति-र्वर्णं द्वितीयं भजते चिराय}
{समुद्रनेमिर्नमते तु तस्मैयं ब्राह्मणः शास्ति नयैर्विनीतम्}


\twolineshloka
{कुञ्चरस्येव संग्रामे परिगृह्याङ्कुशग्रहम्}
{ब्राह्माणैर्विप्रहीनस्य क्षत्रस्य क्षीयते बलम्}


\twolineshloka
{ब्राह्मण्यनुपमा दृष्टिः क्षात्रमप्रतिमं बलम्}
{तौ यदा चरतः सार्धं तदा लोकः प्रसीदती}


\twolineshloka
{यथा हि सुमहानग्निः कक्षं दहति सानिलः}
{तथा दहति राजन्यो ब्राह्मणेन समं रिपुम्}


\twolineshloka
{ब्राह्मणेष्वेव मेधावी बुद्धिपर्येषणं चरेत्}
{अलब्धस्य च लाभाय लब्धस्य परिवृद्धये}


\threelineshloka
{अलब्धलाभाय च लब्धवृद्धये}
{यथार्हतीर्थप्रतिपादनाय}
{यशस्विनं वेदविदं विपश्चितंबहुश्रुं ब्राह्मणमेव वासयेत्}


\threelineshloka
{ब्राह्मणेषूत्तमा भक्तिस्तव नित्यं युधिष्ठिर}
{तेन ते सर्वलोकेषु दीप्यते प्रथितं यशः ॥वैशंपायन उवाच}
{}


\twolineshloka
{ततस्ते ब्राह्मणाः सर्वे बकं दाल्भ्यमंपूजयन्}
{युधिष्ठिरं स्तूयमाने भूयः सुमनसोऽभवन्}


\twolineshloka
{द्वैपायनो नारदश्च जामदग्न्यः पृथुश्रवाः}
{इन्द्रद्युम्नो भालुकिश्च कृतचेताः सहस्रपात्}


\twolineshloka
{कर्णश्रवाश्च मुञ्जश्च लवणाश्वश्च काश्यपः}
{हारीतः स्थूणकर्णश्च अग्निवेश्योऽथ शौनकः}


\twolineshloka
{कृतवाक्च सुवाश्चैव बृहदश्चो विभावसुः}
{ऊर्ध्वरेता वृषामित्रः सुहोत्रो होत्रवाहनः}


\twolineshloka
{एते चान्ये च बहवो ब्राह्मणाः संशितव्रताः}
{अजातशत्रुमानर्चुः पुरंदरमिवर्षयः}


\chapter{अध्यायः २७}
\twolineshloka
{वैशंपायन उवाच}
{}


\twolineshloka
{ततो वनगताः पार्थाः सायाह्ने सह कृष्णया}
{उपविष्टाः कथाश्चक्रुर्दुःखशोकपरायणाः}


\threelineshloka
{प्रिया च दर्शनीया च पण्डिता च पतिव्रता}
{अथ कृष्णा धर्मराजमिदं वचनमब्रवीत् ॥द्रौपद्युवाच}
{}


\twolineshloka
{न नूनं तस्य पापस्य दुःखमस्मासु किंचन}
{विद्यते धार्तराष्ट्रस्य नृशंसस्य दुरात्मनः}


\twolineshloka
{यस्त्वां राजन्मया सार्धमजिनैः प्रतिवासितम्}
{वनं प्रस्थाप्य दुष्टात्मा नान्वतप्यत दुर्मतिः}


\threelineshloka
{आयसं हृदयं नूनं तस्य दुष्कृतकर्मणः}
{यस्त्वां धर्मपरं श्रेष्ठं रूक्षाण्यश्रावयत्तदा}
{`वचनान्यपरोक्षाणि दुर्वाच्यानि च संसदि ॥'}


\twolineshloka
{सुखोचितमदुःखार्हं दुरात्मा ससुहृद्गणः}
{ईदृशं दुःखमानीय मोदते पापपूरुषः}


\twolineshloka
{चतुर्णामेव पापानामस्रं न पतितं तदा}
{त्वयि भारत निष्क्रान्ते वनायाजिनवाससि}


\twolineshloka
{दुर्योधनस्य कर्णस्य शकुनेश्च दुरात्मनः}
{दुर्भ्रातुस्तस्य चोग्रस्य राजन्दुःखशासनस्य च}


\twolineshloka
{इतरेषां तु सर्वेषां कुरूणां कुरुसत्तम}
{दुःखेनाभिपरीतानां नेत्रेभ्यः प्रापतज्जलम्}


\twolineshloka
{इदं च शयनं दृष्ट्वा यच्चासीत्ते पुरातनम्}
{शोचामित्वां महाराजदुःखानर्हंसुखोचितम्}


\twolineshloka
{दान्तं यच्च सभामध्य आसनं रत्नभूषितम्}
{दृष्ट्वा कुशबृशीं चेमां शोको मां दारयत्ययम्}


\twolineshloka
{यदपश्यं सभायां त्वां राजभिः परिवारितम्}
{तच्च राजन्नपश्यन्त्याः का सान्तिर्हृदयस्यमे}


\twolineshloka
{या त्वाऽह्रं चन्दनादिग्धमपश्यं सूर्यवर्चसम्}
{सा त्वां पङ्कमलादिग्धं दृष्ट्वा मुह्यामि भारत}


\twolineshloka
{या त्वाऽहं कौशिकैर्वस्त्रैः शुभ्रैराच्छादितं पुरा}
{दृष्टवत्यस्मिराजेनद््र साऽद्य पश्यामि चीरिणम्}


\twolineshloka
{यच्च तद्रुक्मपात्रीभिर्ब्राह्मणेभ्यः सहस्रशः}
{ह्रियते ते गृहादन्नं संस्कृतंसार्वकामिकम्}


\twolineshloka
{यतीनामगृहाणां ते तथैव गृहमेधिनाम्}
{दीयते भोजनं राजन्नतीव गुणवत्प्रभो}


\threelineshloka
{सत्कृतानि सहस्राणि सर्वकामैः पुरा गृहे}
{सर्वकामैः सुविहितैर्यदपूजयथा द्विजान्}
{तच्च राजन्नपश्यन्त्याः का शान्तिर्हृदयस्य मे}


\twolineshloka
{यत्ते भ्रातॄन्महाराज युवानो मृष्टकुण्डलाः}
{अभोजयन्त मृष्टान्नैः सूदाः परमसंस्कृतैः}


\twolineshloka
{सर्वांस्तानद्य पश्यामि वने वन्येन जीविनः}
{अदुःखार्हान्मनुष्येन्द्र नोपशाम्यति मे मनः}


\twolineshloka
{भीमसेनमिमं चापि दुःखितं वनवासिनम्}
{ध्यायतः किं न मन्युस्ते प्राप्ते काले विवर्धते}


\twolineshloka
{भीमसेनं हि कर्माणि स्वयं कुर्वाणमच्युतम्}
{सुखार्हं दुःखितं दृष्ट्वा कस्माद्राजन्नुपेक्षसे}


\twolineshloka
{सत्कृतं विविधैर्यानैर्वस्त्रैरुच्चावचैस्तथा}
{तं ते वनगतं दृष्ट्वा कस्मान्मन्युर्न वर्धते}


\twolineshloka
{अयं कुरून्रणे सर्वान्हन्तुमुत्सहते प्रभुः}
{त्वत्प्रतिज्ञां प्रतीक्षंस्तु सहतेऽयं वृकोदरः}


\twolineshloka
{योऽर्जुनेनार्जुनस्तुल्यो द्विबाहुर्बहुबाहुना}
{शरातिसर्गे शीघ्रत्वात्कालान्तकयमोपमः}


\twolineshloka
{यस् शस्त्रप्रतापेन प्रणताः सर्वपार्थिवाः}
{यज्ञे तव महाराज ब्राह्मणानुपतस्थिरे}


\twolineshloka
{तमिमं पुरुषव्याघ्रं पूजितं देवदानवैः}
{ध्यायन्तमर्जुनं दृष्ट्वा कस्माद्राजन्न कुप्यसि}


\twolineshloka
{दृष्ट्वा वनगतं पार्थमदुःखार्हं सुखोचितम्}
{न च तेवर्धते मन्युस्तेन मुह्यामि भारत}


\twolineshloka
{यो देवांश्च मनुष्यांश्च सर्पांश्चैकरथोऽजयत्}
{तं ते वनगतं दृष्ट्वा कस्मान्मन्युर्न वर्धते}


\twolineshloka
{यो यानैरद्भुताकारैर्हयैर्नागैश्च संवृतः}
{प्रसह्य वित्तान्यादत्त पार्थिवेभ्यः परंतप}


\twolineshloka
{क्षिपत्येकेन वेगेन पञ्चबाणशतानि यः}
{तं ते वनगतं दृष्ट्वा कस्मान्मन्युर्न वर्धते}


\twolineshloka
{श्यामं बृहन्तं तरुणं चर्मिणामुत्तमं रणे}
{नकुलं ते वने दृष्ट्वा कस्मान्मन्युर्न वर्धते}


\twolineshloka
{दर्शनीयं च शूरं च माद्रीपुत्रं युधिष्ठिर}
{3-27-32bसहदेवं वनेदृष्ट्वा कस्मात्क्षमसि पार्थिव}


\twolineshloka
{नकुलं सहदेवं च दृष्ट्वा ते दुःखितावुभौ}
{अदुःखार्हौ मनुष्येन्द्र कस्मान्मन्युर्न वर्धते}


\threelineshloka
{द्रुपदस्य कुले जातां स्नुषां पाण्डोर्महात्मनः}
{धृष्टद्युम्नस्य भगिनीं वीरपत्नीमनुव्रताम्}
{मां वै वनगतां दृष्ट्वा कस्मात्क्षमसि पार्थिव}


\twolineshloka
{नूनं च तव नैवास्ति मन्युर्भरतसत्तम}
{यत्ते भ्रातॄंश्च मां चैव दृष्ट्वा नव्यथते मनः}


\twolineshloka
{न निर्मन्युःक्षत्रियोऽस्ति लोके निर्वचनं स्मृतम्}
{तदद्य त्वयि पश्यामि क्षत्रिये विपरीतवत्}


\twolineshloka
{यो न दर्शयते तेजः क्षत्रियः काल आगते}
{सर्वभूतानि तं पार्थ सदा परिभवन्त्युत}


\twolineshloka
{तत्त्वया न क्षमा कार्या शत्रून्प्रति कथंचन}
{तेजसैव हि ते शक्या निहन्तुं नात्र संशयः}


\twolineshloka
{तथैव यः क्षमाकाले क्षत्रियो नोपशाम्यति}
{अप्रियः सर्वभूतानां सोमुत्रेह च नश्यति}


\chapter{अध्यायः २८}
\twolineshloka
{द्रौपद्युवाच}
{}


\twolineshloka
{अत्राप्युदाहरन्तीममितिहासं पुरातनम्}
{प्रह्लादस्य च संवादं बलेर्वैरोचनस्य च}


\twolineshloka
{असुरेन्द्रं महाप्राज्ञं धर्माणामागतागमम्}
{बलिः पप्रच्छ दैत्येनद्रं प्रह्लादं पितरं पितुः}


\twolineshloka
{क्षमा स्विच्छ्रेयसी तात उताहो तेज इत्युत}
{एतन्मे संशयं तात यथावद्ब्रूहि पृच्छते}


\twolineshloka
{श्रेयो यदत्र धर्मज्ञ ब्रूहि मे तदसंशयम्}
{करिष्यामि हि तत्सर्वं यथावदनुशासनम्}


\threelineshloka
{तस्मै प्रोवाच तत्सर्वमेवं पृष्टः पितामहः}
{सर्वनिश्चयवित्प्राज्ञः संशयं परिपृच्छते ॥प्रह्लाद उवाच}
{}


\twolineshloka
{न श्रेयः सततं तेजो न नित्यं श्रेयसी क्षमा}
{इति तात विजानीहि द्वयमेतदसंशयम्}


\twolineshloka
{यो नित्यं क्षमते तात बहून्दोषान्स विन्दति}
{भृत्याः परिभवन्त्येनमुदासीनास्तथाऽरयः}


\twolineshloka
{सर्वभूतानि चाप्यस् न नमन्ते कदाचन}
{तस्मान्नित्यं क्षमा तात पण्डितैरपि वर्जिता}


\twolineshloka
{अवज्ञाय हितं भृत्या भजन्ते बहुदोषताम्}
{आदातुं चास्य वित्तानि प्रार्थयन्तेऽल्पचेतसः}


\twolineshloka
{यानं वस्त्राण्यलंकाराञ्शयनान्यासनानि च}
{भोजनान्यथ पानानि सर्वोपकरणानि च}


\twolineshloka
{आददीरन्नधिकृता यथाकाममचेतसः}
{प्रदिष्टानि च देयानि न दद्युर्भर्तृशासनात्}


\twolineshloka
{न चैनं भर्तृपूजाभिः पूजयन्ति कदाचन}
{अव्रज्ञानं हि लोकेऽस्मिन्मरणादपि गर्हितम्}


\twolineshloka
{क्षमिणं तादृशं तांत ब्रुवन्ति कटुकान्यपि}
{प्रेष्याः पुत्राश्च भृत्याश्च तथोदासीनवृत्तयः}


\twolineshloka
{अप्यस्य दारानिच्छन्ति परिभूय क्षमावतः}
{दाराश्चास्य प्रवर्तन्ते यथाकाममचेतसः}


\threelineshloka
{तथा च नित्यमुदिता यदि नाल्पमपीश्वरात्}
{`अकृतोपद्रवः कश्चिन्महानपि न पूज्यते}
{}


\twolineshloka
{पूजयन्ति नरा नागान्न तार्क्ष्यं नामघातिनम् ॥'एते चान्ये च बहवो नित्यं दोषाः क्षमावताम्}
{}


% Check verse!
अथ वैरोचने दोषानिमान्विद्ध्यक्षमांवताम् ॥अथ वैरोचने दोषानिमान्विद्ध्यक्षमावताम्
\twolineshloka
{अस्थाने यदि वा स्थाने सततं रजसावृतः}
{क्रुद्धो दण्डान्प्रणयति विविधान्स्वेन तेजसा}


\twolineshloka
{मित्रैः सह विरोधं च प्राप्नुते तेजसाऽऽवृतः}
{आप्नोति द्वेष्यतां चैव लोकात्स्वजनतस्तथा}


\twolineshloka
{सोऽवमानादर्थहानिमुपालम्भमनादरम्}
{संतापद्वेषमोहांश्च शत्रूंश्च लभते नरः}


\twolineshloka
{क्रोधाद्दण्डान्मनुष्येषु विविधान्पुरुषोऽनयात्}
{भ्रश्यते शीघ्रमैश्वर्यात्प्राणेभ्यः स्वजनादपि}


\twolineshloka
{योपकर्तॄश्च हर्तॄश्च तेजसैवोपगच्छति}
{तस्मादुद्विजते लोकः सर्पाद्वेश्मगतादिव}


\twolineshloka
{यस्मादुद्विजतेलोकः कथं तस्य भवो भवेत्}
{अन्तरं तस्य दृष्ट्वैव लोको विकुरुते ध्रुवम्}


\twolineshloka
{तस्मान्नात्युत्सृजेत्तेजो न च नित्यं मृदुर्भवेत्}
{कालेकाले तु संप्राप्ते मृदुस्तीक्ष्णोपि वा भवेत्}


\twolineshloka
{काले मृदुर्यो भवति काले भवति दारुणः}
{स वै सुखमवाप्नोति लोकेऽमुष्मिन्निहैव च}


\twolineshloka
{क्षमाकालांस्तु वक्ष्यामि शृणु मे विस्तरेण तान्}
{ये ते नित्यमसंत्याज्या यथा प्राहुर्मनीषिणः}


\twolineshloka
{पूर्वोपकारी यस्ते स्यादपराधे गरीयसि}
{उपकारेण तत्तस्य क्षन्तव्यमपराधिनः}


\twolineshloka
{अबुद्धिमाश्रितानां तु क्षन्तव्यमपराधिनाम्}
{न हि सर्वत्र पाण्डित्वं सुलभं पुरुषेण वै}


\twolineshloka
{अथ चेद्बुद्धिजं कृत्वा ब्रूयुस्ते तदबुद्धिजम्}
{पापान्स्वल्पेऽपि तान्हन्यादपराधे तथाऽनृजून्}


\twolineshloka
{सर्वस्यैकोपराधस्ते क्षन्तव्यः प्राणिनो भवेत्}
{द्वितीये सति वध्यस्तु स्वल्पेऽप्यपकृते भवेत्}


\twolineshloka
{अजानता भवेत्कश्चिदपराधः कृतो यदि}
{क्षन्तव्यमेव तस्याहुः सुपरीक्ष्य परीक्षया}


\twolineshloka
{मृदुना दारुणं हन्ति मृदुना हन्त्यदारुणम्}
{नासाध्यं मृदुना किंचित्तस्मात्तीव्रतरं मृदु}


\twolineshloka
{देशकालौ तु संप्रेक्ष्य बलाबलमथात्मनः}
{अन्वीक्ष्यकारणं चैव कार्यं तेजः क्षमापि वा ॥'}


\twolineshloka
{नादेशकाले किंचित्स्याद्देशकालौ प्रतीक्षताम्}
{तथा लोकभयाच्चैव क्षन्तव्यमपराधितम्}


\twolineshloka
{एत एवंविधाः कालाः क्षमायाः परिकीर्तिताः}
{अतोऽन्यथाऽनुवर्तत्सु तेजसः काल उच्यते}


\twolineshloka
{तदहं तेजसः कालं तव मन्ये नराधिप}
{धार्तराष्ट्रेषु लुब्धेषु सततं चापकारिषु}


\twolineshloka
{न हि कश्चित्क्षमाकालो विद्यतेऽद्य कुरून्प्रति}
{तेजसश्चागते काले तेज उत्स्रष्टुमर्हसि}


\twolineshloka
{मृदुर्भवत्यवज्ञातस्तीक्ष्णादुद्विजते जनः}
{काले प्राप्ते द्वयं चैतद्यो वेद स महीपतिः}


\chapter{अध्यायः २९}
\twolineshloka
{`वैशंपायन उवाच}
{}


\twolineshloka
{द्रौपद्या वचनं श्रुत्वा श्लक्ष्णाक्षरपदं शुभम्}
{उवाच द्रौपदीं राजा स्मयमानो युधिष्ठिरः}


\fourlineindentedshloka
{कारणे भवती क्रुद्धा धार्तराष्ट्रस्य दुर्मतेः}
{येन क्रोधं महाप्रज्ञे बहुधा बहुमन्यसे}
{क्रोधमूलं हरं शत्रुं कारणैः शृणु तं मम' ॥युधिष्ठिर उवाच}
{}


\threelineshloka
{क्रोधो हन्ता मनुष्याणां क्रोधो भावयिता पुनः}
{इति विद्धि महाप्रात्रे क्रोधमूलौ भवाभवौ}
{}


\threelineshloka
{यो हि संहरते क्रोधं भावस्तस्य सुशोभने}
{`यो न संहरते क्रोधं तस्याभावो भवत्युत}
{अभावकारणं तस्मात्क्रोधो भवति शोभने'}


\twolineshloka
{यः पुनः पुरुषः क्रोधं नित्यं विसृजते शुभे}
{तस्याभावाय भवति क्रोधः परमदारुणः}


\twolineshloka
{क्रोधमूलो विनाशो हि प्रजानामिह दृश्यते}
{तत्कथं मादृशः क्रोधमुत्सृजेल्लोकनाशनम्}


\twolineshloka
{क्रुद्धः पापं नरः कुर्यात्क्रुद्धो हन्याद्गुरूनपि}
{क्रुद्धः परुषया वाचा श्रेयसोऽप्यवमन्यते}


\twolineshloka
{वाच्यावाच्ये हि कुपितो न प्रजानाति कर्हिचित्}
{नाकार्यमस्ति क्रुद्धस्य नावाच्यं विद्यते तथा}


\twolineshloka
{हिंस्यात्क्रोधादवध्यांस्तु वध्यान्संपूजयीत च}
{आत्मानमपि च क्रुद्धः प्रेषयेद्यमसादनम्}


\twolineshloka
{एतान्दोषान्प्रपश्यद्भिर्जितः क्रोधो मनीषिभिः}
{इच्छद्भिः परमं श्रेय इह चामुत्र चोत्तमम्}


\twolineshloka
{तं क्रोधं वर्जितं धीरैः कथमस्मद्विधश्ररेत्}
{एतद्द्रौपदि संस्मृत्यन मे मन्युः प्रवर्धते}


\twolineshloka
{आत्मानं च परांश्चैव त्रायते महतो भयात्}
{क्रुध्यन्तमप्रतिक्रुध्यन्दूयोरेष चिकित्सकः}


\twolineshloka
{मूढो यदि क्लिश्यमानः क्रुध्यतेऽशक्तिमान्नरः}
{बलीयसां मनुष्याणां त्यजत्यात्मानमन्ततः}


\twolineshloka
{तस्यात्मानं संत्यजतो लोका नश्यन्त्यनात्मनः}
{तस्माद्द्रौपद्यशक्तस् मन्योर्नियमनं स्मृतम्}


\twolineshloka
{विद्वांस्तथैव यः शक्तः क्लिश्यमानः प्रकुप्यति}
{स नाशयित्वा द्वेष्टारं परलोके न नन्दति}


\twolineshloka
{तस्माद्बलवता चैव दुर्बलेन च नित्यदा}
{क्षन्तव्यं पुरुषेणाहुरापत्स्वपि विजानता}


\twolineshloka
{मन्योर्हि विजयं कृष्णे प्रशंसन्तीह साधवः}
{क्षमावतो जयो नित्यं साधोरिह सतां मतम्}


\threelineshloka
{सत्यंचानृततः श्रेयो नृशंसाच्चानृशंसता}
{तमेवं बहुदोषं त क्रोधं सद्भिर्विवर्जितम्}
{मादृशो विसृजेत्तस्मात्सर्वलोकविनाशनम्}


\twolineshloka
{तेजस्वीति यमाहुर्वै पण्डिता दीर्गदर्शिनः}
{न क्रोधोऽभ्यन्तरस्तस् भवतीति विनिश्चेतम्}


\twolineshloka
{वस्तु क्रोधं समुत्पन्नं प्रज्ञया परिबाधते}
{तेजस्विनं तं विदुषो मन्यन्ते तत्त्वदर्शिनः}


\twolineshloka
{क्रुद्धो हि कार्यं सुश्रोणि न यथावत्प्रपश्यति}
{नाकार्यं न च मर्यादां नरः क्रुद्धोऽनुपश्यति}


\twolineshloka
{हन्त्यवध्यानपि क्रुद्धो गुरून्रूक्षैस्तुदत्यपि}
{तस्मात्तेजसि कर्तव्ये क्रोधो दूरात्प्रतिष्ठितः}


\twolineshloka
{दाक्ष्यं ह्यमर्षः शौर्यं च शीघ्रत्वमिति तेजसः}
{गुणाः क्रोधाभिभूतेन न शक्याः प्राप्तुमञ्जसा}


\twolineshloka
{क्रोधं त्यक्त्वा तु पुरुषः सम्यक्तेजोऽभिपद्यते}
{कालयुक्तं महाप्राज्ञैः क्रुद्धैस्तेजः सुदुर्लभम्}


\twolineshloka
{क्रोधस्त्वपण्डितैः शश्वत्तेज इत्यभिनिश्चितम्}
{रजस्तु लोकनाशाय विहितं मानुषान्प्रति}


\twolineshloka
{तस्माच्छश्वत्त्यजेत्क्रोधं पुरुषः सम्यगाचरन्}
{श्रेयान्स्वधर्मानपगो न क्रुद्ध इति निश्चितम्}


\twolineshloka
{यदि सर्वमबुद्धीनामतिक्रान्तमचेतसाम्}
{अतिक्रमो मद्विधस्य कथं स्वित्स्यादनिन्दिते}


\twolineshloka
{यदि न स्युर्मानुषेषु क्षमिणः पृथिवीसमाः}
{न स्यात्सन्धिर्मनुष्याणां क्रोधमूलो हि विग्रहः}


\twolineshloka
{अभिषक्तो ह्यभिषजेदाहन्याद्गुरुणा हतः}
{एवं विनाशो भूतानाममधर्मः प्रथितो भवेत्}


\twolineshloka
{आक्रुष्ट पुरुषः सर्वं प्रत्याक्रोशेदनन्तरम्}
{प्रतिहन्याद्धतश्चैव तथा हिंस्याच्च हिंसितः}


\twolineshloka
{हन्युर्हि पितरः पुत्रान्पुत्राश्चापि तथा पितॄन्}
{हन्युश्च पतयो भार्याः पतीन्भार्यास्तथैव च}


\twolineshloka
{एवं संकुपिते लोके जन्म कृष्णो न विद्यते}
{प्रजानां सन्धिमूलं हि जन्म विद्धिशुभानने}


\twolineshloka
{ताः क्षीयेरन्प्रजाः सर्वाः क्षिप्रं द्रौपदि तादृशाः}
{तस्मान्मन्युर्विनाशाय प्रजानामभवाय च}


\twolineshloka
{यस्मात्तु लोके दृश्यन्ते क्षमिणः पृथिवीसमाः}
{तस्माज्जन्म च भूतानां भवश्च प्रतिपद्यते}


\twolineshloka
{क्षन्तव्यं पुरुषेणेह सर्वापत्सु सुशोभने}
{क्षमावतो हि भूतानां जन्म चैव प्रकीर्तितम्}


\threelineshloka
{आक्रुष्टस्ताडितः क्रुद्धः क्षमते यो बवीयसा}
{यश्च नित्यं जितक्रोधो विद्वानुत्तमपूरुषः}
{प्रभाववानपि नरस्तस्य लोकाः सनातनाः}


% Check verse!
क्रोधनस्त्वल्पविज्ञानः प्रेत्य चेह च नश्यति
\twolineshloka
{अत्राप्युदाहरन्तीमा गाथा नित्यं क्षमावताम्}
{गीताः क्षमावतां कृष्णे काश्यपेन महात्मना}


\twolineshloka
{क्षमा धर्मः क्षमा यज्ञः क्षमा वेदाः क्षमा श्रुतम्}
{यस्तमेवं विजानाति स सर्वं क्षन्तुर्महति}


\twolineshloka
{क्षमा ब्रह्म क्षमा सत्यं क्षमा भूतं च भावि च}
{क्षमा तपः क्षमा शौचं क्षमयेदं धृतं जगत्}


\twolineshloka
{अति यज्ञविदां लोकान्क्षमिणः प्राप्नुवन्ति च}
{अति ब्रह्मविदां लोकानति चापि तपस्विनाम्}


\twolineshloka
{अन्ये वै यजुषां लोकाः कर्मिणामपरे तथा}
{क्षमावतां ब्रह्मलोके लोकाः परमपूजिताः}


\twolineshloka
{क्षमा तेजस्विनां तेजः क्षमा ब्रह्म तपस्विनाम्}
{क्षमा सत्यं सत्यवतां क्षमा यज्ञः क्षमा शभः}


\twolineshloka
{तां क्षमां तादृशीं कृष्णे कथमस्मद्विधस्त्यजेत्}
{यत्रब्रह्म च सत्यं च यज्ञा लोकाश्च धिष्ठिताः}


\threelineshloka
{`इज्यन्ते यज्वनां लोकाः क्षमिणामपरे तथा}
{'क्षन्तव्यमेव सततं पुरुषेण विजानता}
{यदा हि क्षमते सर्वं ब्रह्म संपद्यते तदा}


\twolineshloka
{क्षमावतामयं लोकः परश्चैव क्षमावताम्}
{इह सन्मानमृच्छन्ति परत्र च परा गतिम्}


\twolineshloka
{येषां मन्युर्मनुष्याणां क्षमयाऽभिहतः सदा}
{तेषां परतरे लोकास्तस्मात्क्षान्तिः परा मता}


\twolineshloka
{इति गीताः काश्यपेन गाथानित्यं क्षमावताम्}
{श्रुत्वा गाथाः क्षमायास्त्वं तुष्यद्रौपदि मा क्रुधः}


\twolineshloka
{पितामहः शान्तनवः रशमं संपूजयिष्यति}
{कृष्णश्च देवकीपुत्रः शमं संपूजयिष्यति}


\twolineshloka
{आचार्यो विदुरः क्षत्ता शममेव वदिष्यतः}
{कृपश्च संजयश्चैव शममेव वदिष्यतः}


\twolineshloka
{सोमदत्तो युयुत्सुश्च द्रोणपुत्रस्तथैव च}
{पितामहश्च नो व्यासः शमं वदति नित्यशः}


\twolineshloka
{एतैर्हि राजा नियतं चोद्यमानः शमं प्रति}
{राज्यं दातेति मे बुद्धिर्न चेल्लोभान्नशिष्यति}


\twolineshloka
{कालोऽयं दारुणः प्राप्तो भरतानामभूतये}
{निश्चितं मे सदैवैतत्पुरस्तादपि भामिति}


\twolineshloka
{सुयोधनो नार्हतीति क्षमा मामेव विन्दति}
{अर्हस्तत्राहमित्येवं तस्मान्मां विन्दते क्षमा}


\twolineshloka
{एतदात्मवतां वृत्तमेष धर्मः सनातनः}
{क्षमा चैवानृशंस्यं च तत्कर्ताऽस्म्यहमञ्जसा}


\chapter{अध्यायः ३०}
\twolineshloka
{द्रौपद्युवाच}
{}


\twolineshloka
{नमो धात्रे विधात्रे च यौ मोहं चक्रतुस्तव}
{पितृपैतामहे राज्ये वोढव्ये तेऽन्यथा मतिः}


\threelineshloka
{[कर्मभिश्चिन्तितो लोको गत्यांगत्यां पृथग्विधः}
{तस्मात्कर्माणि नित्यानि लोभान्मोक्षं यियासति}
{]}


\twolineshloka
{नेह धर्मानृशंस्याभ्यां न क्षान्त्या नार्जवेन च}
{पुरुषः श्रियमाप्नोति न घृणित्वेन कर्हिचित्}


\twolineshloka
{त्वां चेद्व्यसनमभ्यागादिदं भारत दुःसहम्}
{स त्वं नार्हसि नापीमे भ्रातरस्ते महौजसः}


\twolineshloka
{न हि तेऽध्यगमञ्जातु तदानीं नाद्य भारत}
{धर्मात्प्रियतरं किंचिदपि चेज्जीवितादपि}


\twolineshloka
{धर्मार्थमेव ते राज्यं धर्मार्थं जीवितं च ते}
{ब्राह्मणा गुरवश्चैव जानन्त्यपि च देवताः}


\twolineshloka
{भीमसेनार्जुनौ चोभौ माद्रेयौ च मया सह}
{त्यजेस्त्वमिति मे बुद्धिर्न तु धर्मं परित्यजेः}


\twolineshloka
{राजानं धर्मगोप्तारं धर्मो रक्षति रक्षितः}
{इति मे श्रुतमार्याणां त्वां तु मन्ये न रक्षति}


\twolineshloka
{त्यक्त्वाऽन्यान्हि नरव्याघ्र नित्यदा धर्म एव ते}
{बुद्धिः सततमन्वेति छायेवं पुरुषं निजा}


\twolineshloka
{नावमंस्था हि सदृशान्नावराञ्श्रेयसः कुतः}
{अवाप्य पृथिवीं कृत्स्नां न ते शृङ्गमवर्धत}


\twolineshloka
{स्वाहाकारैः स्वधाभिश्च पूजाभिरपि च द्विजान्}
{देवताश्च पितॄंश्चैव सततं पार्थ सेवसे}


\twolineshloka
{ब्राह्मणाः सर्वकामैस्ते सततं पार्थ तर्पिताः}
{यतयो मोक्षिणश्चैव गृहस्ताश्चैव भारत}


\threelineshloka
{भुञ्जते रुक्मपात्रीभिर्यत्राहं परिचारिका}
{आरण्यकेभ्यो वन्यानि भोजनानि प्रयच्छसि}
{नादेयं ब्राह्मणेभ्यस्ते गृहे किंचन विद्यते}


\twolineshloka
{यदिदं वैश्वदेवान्ते सायं प्रातः प्रदीयते}
{तद्दत्त्वाऽतिथिभृत्येभ्योराजञ्शिष्टेनजीवसि}


\twolineshloka
{इष्टयः पशुबन्धाश्च काम्यनैमित्तिकाश्च ये}
{वर्तन्ते पाकयज्ञाश्च सदा यज्ञाश्च तेऽनघ}


\twolineshloka
{अस्मिन्नपि महारण्ये विजने दस्युसेविते}
{राष्ट्रादपेत्य वसतो धर्मस्ते नावसीदति}


\twolineshloka
{अश्वमेधो राजसूयः पौण्डरीकोऽथ गोसवः}
{इष्टास्त्वया महायज्ञाबहवोऽन्ये सदक्षिणाः}


\twolineshloka
{राजन्परीतया बुद्ध्या विषमेऽक्षपराजये}
{पार्थ मित्राणि चास्मांश्च वसूनि च पराजितः}


\twolineshloka
{ऋजोर्मृदोर्बदान्यस्य धीमतः सत्यवादिनः}
{कथमक्षव्यसनजा बुद्धिरापतिता तव}


\twolineshloka
{अतीव मोह आयाति मनश् परिदूयते}
{निशाम्य व्यसनं पार्थ तवेदमतिदुःसहम्}


\twolineshloka
{अत्राप्युदाहरन्तीममितिहासं पुरातनम्}
{ईश्वरस्य वशे लोकस्तिष्ठते नात्मनो यथा}


\twolineshloka
{धातैव खलु भूतानां सुखदुःखे प्रियाप्रिये}
{ददाति सर्वमीशानः पुरस्ताच्छुक्रमुच्चरन्}


\twolineshloka
{यथा दारुमयीं योषां नरो धीरः समाहितः}
{इङ्ग्यत्यङ्गमङ्गानि तथा राजन्निमाः प्रजाः}


\twolineshloka
{आकाश इव भूतानि व्याप्य सर्वाणि भारत}
{ईश्वरो विदधातीह कल्याणं यच्च पापकम्}


\twolineshloka
{शकुनिस्तन्तुबद्धो वा नीयतेऽयमनीश्वरः}
{ईश्वरस्य वशे तिष्ठन्नान्येषामात्मनः प्रभुः}


\twolineshloka
{मणिः सूत्र इव प्रोतो नस्योत इव गोवृषः}
{धातुरादेशमन्वेति तन्मयो हि तदर्पणः}


\twolineshloka
{नात्माथीनो मनुष्योऽयं कालं भजतिं कंचन}
{स्रोतसो मध्यमापन्नः कूलवृक्ष इव च्युतः}


\twolineshloka
{अज्ञो जन्तुरनीशोऽयमात्मनः सुखदुःखयोः}
{ईश्वरप्रेरितो गच्छेत्स्वर्गं नरकमेव च}


\twolineshloka
{यथा वायोस्तृणाग्राणि वशं यान्ति बलीयसः}
{धातुरेवं वशं यान्ति सर्वभूतानि भारत}


\twolineshloka
{आर्ये कर्मणि युञ्जानः पापे वा पुनरीश्वः}
{व्याप्य भूतानि चरते न चायमिति लक्ष्यते}


\twolineshloka
{हेतुमात्रमिदं धातुः शरीरं क्षेत्रसंज्ञितम्}
{येन कारयते कर्म शुभाशुभफलं विभुः}


\twolineshloka
{पश्य मायाप्रभावोऽयमीश्वरेण यथा कृतः}
{यो हन्ति भूतैर्भूतानि मोहयित्वाऽऽत्ममायया}


\twolineshloka
{अन्यथा परिदृष्टानि मुनिभिस्तत्त्वदर्शिभिः}
{अन्यथा परिवर्तन्ते वेगा इव नभस्वतः}


\twolineshloka
{अन्यथैव हि वर्तन्ते पुरुषास्तानि तानि च}
{अन्यथैव प्रभुस्तानि करोति विकरोति च}


\twolineshloka
{यथा काष्ठेन वा काष्ठमश्मानं चाश्मना पुनः}
{अयसा चाप्ययश्छिन्द्यान्निर्विचेष्टमचेतनम्}


\twolineshloka
{एवं स भगवान्देवः स्वयंभूः प्रपितामहः}
{निहन्ति भूतैर्भूतानि छद्म कृत्वा युधिष्ठिर}


\twolineshloka
{संप्रयोज्य वियोज्यायं कामकारकरः प्रभुः}
{क्रीडते भगवान्भूतैर्बालः क्रीडनकैरिव}


\twolineshloka
{न मातृपितृवद्राजन्धाता भूतेषु वर्तते}
{रोषादिव प्रवृत्तोऽयं यथाऽयमितरो जनः}


\twolineshloka
{आर्याञ्शीलवो दृष्ट्वा हीमतो वृत्तिकर्शितान्}
{अनार्यान्सुखिनश्चैव विह्वलानिव चिन्तये}


\twolineshloka
{तवेमामापदं दृष्ट्वा समृद्धिं च सुयोधने}
{धातारं गर्हये पार्थ विषमं योऽनुपश्यति}


\twolineshloka
{आर्यशास्त्रातिगे क्रूरे लुब्धे धर्मापचायिनि}
{धार्तराष्ट्रे श्रियं दत्त्वा धाता किं फलमश्नुते}


\twolineshloka
{कर्मचेत्कृतमन्वेति कर्तारं नान्यमृच्छति}
{कर्मणा तेन पापेन लिप्यते नूनमीश्वरः}


\twolineshloka
{अथ कर्मकृतं पापं न चेत्कर्तारमृच्छति}
{कारणं बलमेवेह जनाञ्शोचामि दुर्बलान्}


\chapter{अध्यायः ३१}
\twolineshloka
{युधिष्ठिर उवाच}
{}


\twolineshloka
{वल्गु चित्रपदं श्लक्ष्णं याज्ञसेनि त्वया वचः}
{उक्तं तच्छ्रुतमस्माभिर्नास्तिक्यं तु प्रभाषसे}


\twolineshloka
{नाहं धर्मफलाकाङ्क्षी राजपुत्रि चराम्भुत}
{ददामि देयमित्येव यजे यष्टव्यमित्युत}


\twolineshloka
{अस्तु वाऽत्र फलं मा वा कर्तव्यं पुरुषेण यत्}
{गृहे निवसता कृष्णे यथाशक्ति करोमि तत्}


\twolineshloka
{धर्मं चरामि सुश्रोणि न धर्मफलकारणात्}
{आगमाननतिक्रम्य सतां वृत्तमवेक्ष्य च}


\twolineshloka
{धर्म एव मनः कृष्णे स्वभावाच्चैव मे धृतम्}
{[धर्मवाणिज्यको हीनो जघन्यो धर्मवादिनाम्]}


\twolineshloka
{न धर्मफलमाप्नोति यो धर्मं दोग्धुमिच्छति}
{यश्चैनं शङ्कते कृत्वा नास्तिक्यात्पापचेतनः}


\twolineshloka
{अतिवादान्मदाच्चैव मा धर्ममभिशङ्कथाः}
{धर्मातिशङ्की पुरुषस्तिर्यग्गतिपरायणः}


\twolineshloka
{धर्मो यस्यातिशङ्क्यः स्यादार्षं वा दुर्बलात्मनः}
{वेदाच्छूद्र इवापेयात्स लोकादजरामरात्}


\twolineshloka
{वेदाध्यायी धर्मपरः कुले जातो मनस्विनि}
{स्थविरेषु स योक्तव्यो राजभिर्धर्मचारिभिः}


\twolineshloka
{पापीयान्स हि शूद्रेभ्यस्तस्करेभ्यो विशिष्यते}
{शास्त्रातिगो मन्दबुद्धिर्यो धर्ममतिशङ्कते}


\twolineshloka
{प्रत्यक्षं हि त्वया दृष्ट ऋषिर्गच्छन्महातपाः}
{मार्कण्डेयोऽप्रमेयात्मा धर्मेण चिरजीविता}


\twolineshloka
{व्यासो वसिष्ठो मैत्रेयो नारदो लोमशः शुकः}
{अन्ये च ऋषयः सर्वे धर्मेणैव सुचेतसः}


\twolineshloka
{प्रत्यक्षं पश्यसि ह्येतान्दिव्ययोगसमन्वितान्}
{शापानुग्रहणे शक्तान्देवेभ्योऽपि गरीयसः}


\twolineshloka
{एते हि धर्ममेवादौ वर्णयन्ति सदाऽनधे}
{कर्तव्यममरप्रख्याः प्रत्यक्षागमबुद्धयः}


\twolineshloka
{अतो नार्हसि कल्याणि धातारं धर्ममेव च}
{रजोमूढेन मनसा क्षेप्तुं शङ्कितुमेव च}


\twolineshloka
{उन्मत्तान्मन्यते बालः सर्वानागतनिश्चयान्}
{धर्मातिशङ्की नान्यस्मिन्प्रमाणमधिगच्छति}


\twolineshloka
{इन्द्रियप्रीतिसंबद्धं यदिदं लोकसाक्षिकम्}
{एतावन्मन्यते बालो मोहमन्यत्र गच्छति}


\twolineshloka
{प्रायश्चित्तं न तस्यास्ति यो धर्ममतिशङ्कते}
{ध्यायन्स कृपणः पापो न लोकान्प्रतिपद्यते}


\twolineshloka
{प्रमाणाद्धि निवृत्तो हि वेदशास्त्रार्थनिन्दकः}
{कामलोभानुगो मूढो नरकं प्रतिपद्यते}


\twolineshloka
{आर्षं प्रमाणमुत्क्राम्य धर्मं न प्रतिपालयन्}
{सर्वशास्त्रातिगो मूढः शं जन्मसु न विन्दति}


\twolineshloka
{यस्तु नित्यं कृतमतिर्धर्ममेवाभिपद्यते}
{अशङ्कमानः कल्याणि सोऽमुत्रानन्त्यमश्नुते}


\twolineshloka
{यस्य नार्षं प्रमाणं स्याच्छिष्टाचारश्च भामिनि}
{न वै तस्य परो लोको नायमस्तीति निश्चयः}


\twolineshloka
{शिष्टैराचरितं धर्मं कृष्णे मास्मातिशङ्कथाः}
{पुराणमृषिभिः प्रोक्तं सर्वज्ञैः सर्वदर्शिभिः}


\twolineshloka
{धर्म एव प्लवो नान्यः स्वर्गं द्रौपदि गच्छताम्}
{सैवः नौः सागरस्येव वणिजः पारमिच्छतः}


\twolineshloka
{अफलो यदि धर्मः स्याच्चरितो धर्मचारिभिः}
{अप्रतिष्ठे तमस्येतज्जगन्मज्जेदनिन्दिते}


\twolineshloka
{निर्वाणं नाधिगच्छेयुर्जीवेयुः पशुजीविकाम्}
{विघातेनैव युज्येयुर्न चार्थं कंचिदाप्नुयुः}


\twolineshloka
{तपश्च ब्रह्मचर्यं च यज्ञः स्वाध्याय एव च}
{दानमार्जवमेतानि यदि स्युरफलानि वै}


\twolineshloka
{नाचरिष्यन्परे धर्मं परे परतरेऽपि च}
{दानमार्जवमेतानि यदि स्युरफलानि वै}


\twolineshloka
{ऋषयस्चैव देवाश्च गन्धर्वासुरराक्षसाः}
{ईश्वराः कस्य हेतोस्ते चरेयुर्धर्ममादृताः}


\twolineshloka
{फलदं त्विह विज्ञाय धातारं श्रेयसि स्थितम्}
{धर्मं तेऽव्यचरन्कृष्णे स हि धर्म सनातनः}


\twolineshloka
{स चायं सफलो धर्मो न धर्मोऽफल उच्यते}
{दृश्यन्तेऽपि हि विद्यानां फलानि तपसां तथा}


\twolineshloka
{त्वय्येतद्वै विजानीहि जन्म कृष्णे यथाश्रुतम्}
{वेत्थ चापि यथा जातो धृष्टद्युम्नः प्रतापवान्}


\twolineshloka
{एतावदेव पर्याप्तमुपमानं शुचिस्मिते}
{कर्मणां फलमस्तीति धीरोऽल्पेनापि तुष्यति}


\twolineshloka
{बहुनाऽपि ह्यविद्वांसो नैव तुष्यन्त्यबुद्धयः}
{तेषां न ध्मजं किंचित्प्रेत्य कर्मास्ति शर्म वा}


\twolineshloka
{कर्मणामुत पुण्यानां पापानां च फलोदयः}
{प्रभवश्चाप्ययश्चैव देवगुह्यानि भामिनि}


\twolineshloka
{नैतानि वेद यः कश्चिन्मुह्यन्त्यत्र प्रजा इमाः}
{[अपि कल्पसहस्रेण न च श्रेयोऽधिगच्छति ॥]}


\threelineshloka
{रक्ष्याण्येतानि देवानां गूढमाया हि देवताः}
{कृशाङ्गाः सुब्रताश्चैव तपसा दग्धकिल्विषाः}
{प्रसन्नैर्मानसैर्युक्ताः पश्यन्त्येतानि वै द्विजाः}


\twolineshloka
{न फलादर्सनाद्धर्मः शङ्कितव्यो न देवताः}
{यष्टव्यं च प्रयत्नेन दातव्यं चानमूयता}


\twolineshloka
{कर्मणां फलमस्तीह तथैतद्धर्मशासनम्}
{ब्रह्मा प्रोवाच पुत्राणां यदृषिर्वेद काश्यपः}


\twolineshloka
{तस्मात्ते संशयः कृष्णे नीहार इव नश्यतु}
{विमृश्य सर्वमस्तीति नास्तिक्यं भावमुत्सृज}


\twolineshloka
{ईश्वरं चापि भूतानां धातारं मा च वै क्षिप}
{शिक्षस्वैनं नमस्वैनं मा तेऽभूद्बुद्धिरीदृशी}


\twolineshloka
{यस्य प्रसादात्तद्भक्तो मर्त्यो गच्छत्यमर्त्यताम्}
{उत्तमां देवतां कृष्णे मातिवोचः कथंचन}


\chapter{अध्यायः ३२}
\twolineshloka
{द्रौपद्युवाच}
{}


\twolineshloka
{नावमन्ये न गर्हे च धर्मं पार्थ कथंचन}
{ईश्वरं कुत एवाहमवमंस्ये प्रजापतिम्}


\twolineshloka
{आर्ताऽहं प्रलपामीदमिति मां विद्धि भारत}
{भूयश्च विलपिष्यामि सुमनास्त्वं निबोध मे}


\twolineshloka
{कर्म खल्विह कर्तव्यं जातेनामित्रकर्शन}
{अकर्माणो हि जीवन्ति स्थावरा नेतरे जनाः}


\twolineshloka
{आमातृस्तन्यपानाच्च यावच्छय्योपसर्पणम्}
{जङ्गमाः कर्मणा वृत्तिमाप्नुवन्ति युधिष्ठिर}


\twolineshloka
{जङ्गमेषु विशेषेण मनुष्या भरतर्षभ}
{इच्छन्ति कर्मणा वृत्तिमवाप्तुं प्रेत्य चेह च}


\twolineshloka
{उत्थानमभिजानन्ति सर्वभूतानि भारत}
{प्रत्यक्षं फलमश्नन्ति कर्मणां लोकसाक्षिकम्}


\twolineshloka
{पश्यन्तः स्वं समुत्थानमुपजीवन्ति जन्तरः}
{अपि धाता विधाता च यथाऽयमुदके बकः}


\twolineshloka
{[अकर्मणां वै भूतानां वृत्तिः स्वान्न हि काचन}
{तदेवाभिप्रपद्येत न विहन्यात्कदाचन ॥]}


\twolineshloka
{स्वकर्म कुरु माहासीः कर्मणा भव दंशितः}
{कृत्यं हि योऽभिजानाति सहस्रे सोस्ति नास्ति वां}


\twolineshloka
{तस्य चापि भवेत्कार्वं विवृद्धौ रक्षणे तथा}
{भक्ष्यमाणो द्यनावाषः क्षीयेत हिमवानपि}


\twolineshloka
{उत्सीदेरन्प्रजाः सर्वा न कुर्युः कर्म चेद्यदि}
{[तथा ह्येता न वर्धेरन्कर्म चेदफलं भवेत् ॥]}


\twolineshloka
{दृष्ट्वाऽपिच फलं कर्म पश्यामः कुर्वतो जनान्}
{नान्यथा ह्यपि जानन्ति वृत्तिं लोकाः कथंचन}


\twolineshloka
{यश्च दिष्टपरो लोके यश्चापि हठवादकः}
{उभावपि शठावेतौ कर्मबुद्धिः प्रशस्यते}


\twolineshloka
{यो हि दिष्टमुपासीत निर्विचेष्टः सुखं स्वपन्}
{अवसीदेत्सुदुर्बुद्धिरामो घट इवोदके}


\twolineshloka
{तथैव हठबुद्धिर्यः शक्तः कर्मण्यकर्मकृत्}
{आसीत न चिरं जीवेदनाथ इव दुर्बलः}


\twolineshloka
{अकस्मादिह यः कश्चिदर्थं प्राप्नोति पूरुषः}
{तं हठेनेति मन्यन्ते स हि यत्नो न कस्यचित्}


\twolineshloka
{वश्चापि कश्रित्पुरुषो दिष्टं नाम भजत्युत}
{दैवेन विधिना पार्थ तद्दैवमिति निश्चितम्}


\twolineshloka
{यत्तावस्कर्मणा किंचित्फलमाप्नोति पूरुषः}
{प्रत्यक्षं चक्षुषा दृष्टं तत्पौरुषमिति श्रुतम्}


\twolineshloka
{स्वभावतः प्रवृत्तो यः प्राप्नोत्यर्थं न कारणात्}
{तत्स्वभावात्प्रकं विद्धि फलं पुरुषसत्तम}


\twolineshloka
{एवं हठाच्च दैवाच्च स्वभावात्कर्मणस्तथा}
{यानि प्राप्नोति पुरुषस्तत्फलं पूर्वकर्मणाम्}


\twolineshloka
{धातापि हि स्वकर्मैव तैस्तैर्हेतुभिरीश्वरः}
{विदधाति विभज्येह फलं पूर्वकतं नृणाम्}


\twolineshloka
{यद्धि यः पुरुषः किंचित्कुरुते वै शुभाशुभम्}
{तद्धातृविहितं विद्धि पूर्वकर्मफलोदयम्}


\twolineshloka
{कारणं तस्य देहोऽयं धातुः कर्मणि कर्मणि}
{स यथा प्रेरयत्येनं तथाऽयं कुरुतेऽवशः}


\twolineshloka
{तेषुतेषु हि कृत्येषु विनियोक्ता महेश्वरः}
{सर्वभूतानि कौन्तेय कारयत्यवशान्यपि}


\twolineshloka
{मनसाऽर्थान्विनिश्चित्य पश्चात्प्राप्नोति कर्मणा}
{बुद्धिपूर्वं स्वयं वीर पुरुषस्तत्र कारणम्}


\twolineshloka
{संख्यातुं नैव शक्यानि कर्माणि पुरुषर्षभ}
{अगारनगराणां हि सिद्धिः पुरुषहैतुकी}


\twolineshloka
{तिले तैलं गवि क्षीरं काष्ठे पावकमन्ततः}
{एवं धीरो विजानीयादुपायं चास्य सिद्धये}


\twolineshloka
{ततः प्रवर्तते पश्चात्कारणेष्वस्व सिद्धये}
{तां सिद्धिमुपजीवन्ति कर्मणामिह जन्तवः}


\twolineshloka
{कुशलेन कृतं कर्म कर्त्रा साधु स्वनुष्ठितम्}
{इदं त्वकुशलेनेति विशेषादुपलभ्यते}


\twolineshloka
{इष्टापूर्त्तफलं न स्यान्न शिष्यो न गुरुर्भवेत्}
{पुरुषः कर्मसाध्येषु स्वाच्चेदयमकारणम्}


\twolineshloka
{कर्तृत्वादेव पुरुषः कर्मसिद्धौ प्रशस्यते}
{असिद्धौ निन्द्यते चापि कर्मनाशः कथं त्विह}


\twolineshloka
{सर्वमेव हठेनैके दैवेनैके वदन्त्युत}
{पुंसः प्रयत्नजं केचिद्दैवमेतद्विशिष्यते}


\twolineshloka
{न चैवैतावता कार्यं मन्यन्त इति चापरे}
{अस्ति सर्वमदृश्यं तु दिष्टं चैव तथा हठः}


\twolineshloka
{दृश्यते हि हठाच्चैव दिष्टाच्चार्थस्य संततिः}
{किंचिद्दैवाद्धठात्किंचित्किंचिदेव स्वकर्मतः}


\twolineshloka
{पुरुषः फलमाप्नोति चतुर्थं नात्र कारणम्}
{कुशलाः प्रतिजानन्ति ये वै तत्त्वविदोजनाः}


\twolineshloka
{तथैव धाता भूतानामिष्टानिष्टफलप्रदः}
{यदि न स्वानन भूतानां कृपणो नाम कश्चन}


\twolineshloka
{यंयमर्थमभिप्रेप्सुः कुरुते कर्म पूरुषः}
{तत्तत्सफलमेव स्याद्यदि न स्यात्पुरा कृतम्}


\twolineshloka
{त्रिद्वारामर्थसिद्धिं तु नानुपश्यन्ति ये नराः}
{तथैवानर्थसिद्धिं च यथा लोकास्तथैव ते}


\twolineshloka
{कर्तव्यमेव कर्मेति मनोरेष विनिश्चयः}
{एकान्तेन ह्यनीहोऽयं वर्ततेऽस्मासु संप्रति}


\twolineshloka
{कुर्वतो हि भवत्येव प्रायेणेह युधिष्ठिर}
{एकान्तफलसिद्धिं तु न विन्दत्यलसः क्वचित्}


\twolineshloka
{असंभवे त्वस्य हेतुः प्रायश्चित्तं तु लक्ष्यते}
{कृतेकर्मणि राजेन्द्र तथाऽनृण्यमवाप्नुते}


\twolineshloka
{अलक्ष्मीराविशत्येनं शयानमलसं नरम्}
{निःसंशयं फलं लब्ध्वा दक्षो भूतिमुपाश्नुते}


\twolineshloka
{अनर्थं संशयावस्थं गृणन्त्यामुक्तसशंयाः}
{धीरा नराः कर्मरता न तु निःसंशयाः क्वचित्}


\twolineshloka
{एकान्ते नह्यनर्थोऽयं वर्ततेऽस्मासु सांप्रतम्}
{स तु निःसंशयं न स्यात्त्वयि कर्मण्यवस्थिते}


\twolineshloka
{अथवाऽसिद्धिरेव स्यान्महिमा तु तदेव ते}
{वृकोदरस्य बीभत्सोर्भात्रोश्च यमयोरपि}


\twolineshloka
{अन्येषां कर्म सफलमस्माकमपि वा पुनः}
{विप्रकर्षेण बुद्ध्येत कृतकर्मा यथा फलम्}


\twolineshloka
{पृथिवीं लाङ्गलेनेह कृष्ट्वा बीजं वपत्युत}
{आस्तेऽथ कर्षकस्तूष्णीं पर्जन्यस्तत्र कारणम्}


\twolineshloka
{वृष्टिश्चेन्नानुगृह्णीयादनेनास्तत्र कर्षकः}
{यदन्यः पुरुषः कुर्यात्तत्कृतं सफलं मया}


\twolineshloka
{तच्चेदं फलमस्माकमपराधो न मे क्वचित्}
{इति धीरोऽन्ववेक्ष्यैव नात्मानं तत्रगर्हयेत्}


\twolineshloka
{कुर्वतो नार्थसिद्धिर्मे भवतीति ह भारत}
{निर्वेदो नात्र गन्तव्यो द्वावन्तौ यस्य कर्मणः}


\twolineshloka
{सिद्धिर्वाऽप्यथवाऽसिद्धिरप्रवृत्तिरतोऽन्यथा}
{बहूनां समवाये हि भावानां कर्म सिध्यति}


\twolineshloka
{गुणाभावे फलं न्यूनं भवत्यफलमेव च}
{अनारम्भे हि न फलं न गुणो दृश्यते क्वचित्}


\twolineshloka
{देशकालावुपायांश्च मङ्गलं स्वस्तिसिद्धये}
{युनक्ति मेधया धीरो यथायोगं यथाबलम्}


\twolineshloka
{अप्युपायेन तत्कार्यमुपदेष्टा पराक्रमः}
{भूयिष्ठं कर्मयोगेषु सर्व एव पराक्रमः}


\twolineshloka
{यत्रधीमानवेक्षेत श्रेयासं बहुभिर्गुणैः}
{साम्नैवार्थं ततो लिप्सेत्कर्म चास्मै प्रयोजयेत्}


\twolineshloka
{व्यसनं नाभिकाङ्क्षेत्त विनाशं वा युधिष्ठिर}
{अपि सिन्धोर्गिरेर्वाऽपि किं पुनर्मर्त्यधर्मिणः}


\twolineshloka
{उत्थानयुक्तः सततं परेषामन्तरैषणे}
{आनृण्यमाप्नोति नरः परस्यात्मन एव च}


\twolineshloka
{न त्वेवात्माऽवमन्तव्यः पुरुषेण कदाचन}
{न ह्यात्मपरिभूतस् भूतिर्भवति शोभना}


\twolineshloka
{एवं संस्थितिका सिद्धिरियं लोकस्य भारत}
{चित्रा सिद्धिगतिः प्रोक्ता कालावस्थाविभागशः}


\twolineshloka
{ब्राह्मणं मे पिता पूर्वं वासयामास पण्डितम्}
{सर्वं चार्थमिमं प्राह पित्रे मे भरतर्षभ}


\twolineshloka
{नीतिं बृहस्पतिप्रोक्तां भ्रातृन्मेऽग्राहयन्पुरा}
{तेषां सकाशादश्रौषमहमेतां तदा गृहे}


\twolineshloka
{स मां राजन्कर्मवतीमागतामाह सान्त्वयन्}
{शुश्रूषमाणामासीनां पितुरङ्के युधिष्ठिर}


\chapter{अध्यायः ३३}
\twolineshloka
{वैशंपायन उवाच}
{}


\twolineshloka
{याज्ञसेन्या वचः श्रुत्वा भीमसेनो ह्यमर्षणः}
{निश्वसन्नुपसंगम्य क्रुद्धो राजानमब्रवीत्}


\twolineshloka
{राजन्सत्पदवीं धर्म्यां व्रज सत्पुरुषोचिताम्}
{धर्मकामार्थहीनानां किं नो वस्तुं तपोवने}


\twolineshloka
{नैव धर्मेण तद्राज्यं नार्जवेण न चौजसा}
{अक्षकूटमधिष्ठाय हृतं दुर्योधनेन वै}


\twolineshloka
{गोमायुनेव सिंहानां दिर्बलेन बलीयसाम्}
{आमिपं विघसाशेन तद्वद्राज्यं हि नो हृतम्}


\twolineshloka
{धर्मलेशप्रतिच्छन्नः प्रभवं धर्मकामयोः}
{अर्थमुत्सृज्य किं राजन्दुःखेन परितप्यसे}


\twolineshloka
{भवतोऽनुविधानेन राज्यं नः पश्यतां हृतम्}
{अहार्यमपि शक्रेण गुप्तं गाण्डीवधन्वना}


\twolineshloka
{कुणीनामिव बिल्वानि पङ्गूनामिव धेनवः}
{हृतमैश्वर्यमस्माकं जीवतां भवतः कृते}


\twolineshloka
{भवतः प्रियमित्येवं महद्व्यसनमीदृशम्}
{धर्मकामे प्रतीतस्य प्रतिपन्नाः स्म भारत}


\twolineshloka
{कर्शयामः स्वमित्राणि नन्दयामश्च शात्रवान्}
{आत्मानं भवतः शास्त्रैर्नियम्य भरतर्षभ}


\twolineshloka
{यद्वयं न तदैवैतान्धार्तराष्ट्रान्निहन्म हि}
{भवतः शास्त्रमादाय तन्नस्तपति दुष्कृतम्}


\twolineshloka
{अथैनामन्ववेक्षस्व मृगचर्यामिवात्मनः}
{अवीराचरितां राजन्नबलस्थैर्निषेविताम्}


\twolineshloka
{यां न कृष्णो न बीभत्सुर्नाभिमन्युर्न सृञ्जयाः}
{न चाहमभिनन्दामि न च माद्रीसुतावुभौ}


\twolineshloka
{भवान्धर्मो धर्म इति सततं व्रतकर्शितः}
{कच्चिद्राजन्न निर्वेदादापन्नः क्लीबजीविकाम्}


\twolineshloka
{दुर्मनुष्या हि निर्वेदमफलं सर्वघातकम्}
{अशक्ताः श्रियमाहर्तुमात्मनः कुर्वते प्रियम्}


\twolineshloka
{स भवान्दृष्टिमाञ्शक्तः पश्यन्नस्मासु पौरुषम्}
{आनृशंस्यपरो राजन्नात्मार्थमवबुध्यसे}


\twolineshloka
{अस्मानमी धार्तराष्ट्राः क्षममाणानहिंसतः}
{अशक्तानेव मन्यन्ते तद्दुःखं नाहवे वधः}


\twolineshloka
{तत्रचेद्युध्यमानानामजिह्ममनिवर्तिनाम्}
{सर्वशो हि वधः श्रेयान्प्रेत्य लोकांल्लभेमहि}


\twolineshloka
{अधवा वयमेवैतान्निहत्य भरतर्षभ}
{आददीमहि गां सर्वां तथापि श्रेय एव नः}


\twolineshloka
{सर्वथा कार्यमेतन्नः स्वधर्ममनुतिष्ठताम्}
{काङ्खतां विषुलां कीर्ति वैरं प्रतिचिकीर्षताम्}


\twolineshloka
{आत्मार्थं युध्यमानानां जीविते कृत्यलक्षणे}
{अन्यैरपि हृते राज्ये प्रशंसैव न गर्हणा}


\twolineshloka
{कर्शनार्थो हि यो धर्मो विप्राणामात्मनस्तथा}
{व्यसनं नाम तद्राजन्न धर्मः स कुवर्त्मतत्}


\twolineshloka
{सर्वथा धर्मनित्यं तु पुरुषं धर्मदुर्बलम्}
{जहतस्तात धर्मार्थौ प्रेतं दुःखसुखे यथा}


\twolineshloka
{यस्य धर्मो हि धर्मार्थँ क्लेशभाङ्ग स पण्डितः}
{न स धर्मस्य वेदार्थं सूर्यस्यान्धः प्रभामिव}


\twolineshloka
{यस्य चार्थार्थ एवार्थः स च नार्थस्य कोविदः}
{रक्ष्यते भृतकः पुण्यं यथा स्यात्तादृगेव सः}


\twolineshloka
{अतिवेलं हि योऽर्थार्थी नेतरावनुतिष्ठति}
{स वध्यः सर्वभूतानां ब्रह्महेव जुगुप्सितः}


\twolineshloka
{सततं यश्च कामार्थी नेतरावनुतिष्ठति}
{मित्राणि तस्य नश्यन्ति धर्मार्थाभ्यां च हीयते}


\twolineshloka
{तस्य धर्मार्थहीनस्य कामान्ते निधनं ध्रुवम्}
{कामतो रममाणस्य मीनस्येवाम्भसः क्षये}


\twolineshloka
{तस्माद्धर्मार्थयोर्नित्यं न प्रमान्द्यन्ति पण्डिताः}
{प्रकृतिः सा हि कामस्य पावकस्यारणिर्यथा}


\twolineshloka
{सर्वथा धर्ममूलोऽर्थो धर्मश्चार्थपरिग्रहः}
{इतरेतरयोर्नीतौ विद्धि मेघोदधी यथा}


\twolineshloka
{द्रव्यार्थस्पर्शसंयोगे या प्रीतिरुपजायते}
{स कामश्चित्तसंकल्पः शरीरं नास्य दृश्यते}


\twolineshloka
{अर्थार्थी पुरुषो राजन्बृहन्तं धर्ममिच्छति}
{अर्थमिच्छति कामार्थीन कामादन्यमिच्छति}


\twolineshloka
{न हि कामेन कामोऽन्यः सिध्यते फलमेव तत्}
{उषयोगात्फलस्यैव काष्ठाद्भस्मेव पण्डितः}


\twolineshloka
{इमाञ्शकुनकान्राजन्हन्ति वैतंसिको यथा}
{एतद्रूपमधर्मस्य भूतेषु च विहिंसनम्}


\twolineshloka
{कामाल्लोभाच्च धर्मस्य प्रवृत्तिं यो न पश्ति}
{स वध्यः सर्वभूतानां प्रेत्य चेहैच दुर्मतिः}


\twolineshloka
{व्यक्तं ते विदितो राजन्नर्थो द्रव्यपरिग्रहः}
{प्रकृतिं चापि वेत्थास्य विकृतिं चापि भूयसीम्}


\twolineshloka
{तस्य नाशं विनाशं वा जरया मरणेन वा}
{अनर्थ इति मन्यन्ते सोयमस्मासु वर्तते}


\twolineshloka
{इन्द्रियाणां च पञ्चानां मनसो हृदयस्य च}
{विपये वर्तमानानां या प्रीतिरुपजायते}


\twolineshloka
{स काम इति मे बुद्धिः कर्मणां फलमुत्तमम्}
{एवमेव पृथग्दृष्ट्वा धर्मार्थौ काममेव च}


\twolineshloka
{न धर्मपर एव स्यान्न चार्थपरमो नरः}
{न कामपरमो वा स्यात्सर्वान्सेवेत सर्वदा}


\twolineshloka
{धर्मं पूर्वे धनं मध्ये जघन्ये काममाचरेत्}
{अहन्यनुचरेदेवमेष शास्त्रकृतो विधिः}


\twolineshloka
{कामं पूर्वे धनं मध्ये जघन्ये धर्ममाचरेत्}
{वयस्यनुचरेदेवमेष शास्त्रकृतो विधिः}


\twolineshloka
{धर्मं चार्थं च कामं च यथावद्वदतांवर}
{विभज्य काले कालज्ञः सर्वान्सेवेत पण्डितः}


\twolineshloka
{मोक्षो वा परमं श्रेय एष राजन्सुखार्थिनाम्}
{प्राप्तिं वा बुद्धिमास्थाय सोपायां कुरुनन्दन}


\twolineshloka
{तद्वाऽऽशु क्रियतां राजन्प्राप्तिर्वाप्यधिगम्यताम्}
{जीवितं ह्यातुरस्येव दुःखमन्तरवर्तिनः}


\twolineshloka
{विदितश्चैव मे धर्मः सततं चरितश्च ते}
{जानन्तस्त्वयि शंसन्ति सुहृदः कर्मचोदनाम्}


\twolineshloka
{दानं यज्ञाः सतां पूजा वेदधारणमार्जवम्}
{एष धर्मः परो राजन्फलवान्प्रेत्य चेंह च}


\twolineshloka
{एष नार्थविहीनेन शक्यो राजन्निपेवितुम्}
{अखिलाः पुरुषव्याघ्र गुणाः स्युर्यद्यपीनरे}


\twolineshloka
{धर्ममूलं जगद्राजन्नान्यद्धर्माद्विशिष्यते}
{धर्मश्चार्थेन महता शक्यो राजन्निषेवितुम्}


\twolineshloka
{न चार्थो भैक्ष्यचर्येण नापि क्लैब्येन कर्हिचित्}
{वेत्तुं शक्यस्तथा राजन्केवलं धर्मबुद्धिना}


\twolineshloka
{प्रतिष्द्धा हि ते याच्ञा यया सिध्यति वै द्विजः}
{तेजसैवार्थलिप्सायां यतस्व पुरुषर्षभ}


\twolineshloka
{भैक्ष्यचर्या न विहिता न च विट्शूद्रजीविका}
{क्षत्रियस् विशेषेण धर्मस्तु बलमौरसम्}


\twolineshloka
{स्वधर्मं प्रतिपद्यस्व जहि शत्रून्समागतान्}
{धार्तराष्ट्रवनं पार्थ मया पार्थेन नाशय}


\twolineshloka
{उदारमेव विद्वांसो धर्मं प्राहुर्मनीषिणः}
{उदारं प्रतिपद्यस्व नावरे स्थातुमर्हसि}


\twolineshloka
{अनुबुध्यस्व राजेन्द्र वेत्थ धर्मान्सनातनान्}
{क्रूरकर्माऽभिजातोसि यस्मादुद्विजते जनः}


\twolineshloka
{प्रजापालनसंभूतं फलं तव न गर्हितम्}
{एष ते विहितो राजन्धात्रा धर्मः सनातनः}


\twolineshloka
{तस्माद्विचलितः पार्थ लोके हास्यं गमिष्यसि}
{स्वधर्माद्धि मनुष्याणां चलनं न प्रशस्यते}


\twolineshloka
{स क्षात्रं हृदयं कृत्वा त्यक्त्वेदं शिथिलं मनः}
{वीर्यमास्थाय कौरव्य धुरमुद्वह धुर्यवत्}


\twolineshloka
{न हि केवलधर्मात्मा पृथिवीं जातु कश्चन}
{पार्थिवो व्यजयद्राजन्न भूतिं न पुनः श्रियम्}


\twolineshloka
{जिह्वां दत्ता बहूनां हि क्षुद्राणां लुब्धचेतसाम्}
{निकृत्या लभते राज्यमाहारमिव शल्यकः}


\twolineshloka
{भ्रातर पूर्वजाताश्च सुसमृद्धाश्च सर्वशः}
{निकृत्या निर्जिता देवैरसुराः पाण्डवर्षभ}


\twolineshloka
{एवं बलवतः सर्वमिति बुद्ध्वा महीपते}
{जहि शत्रून्महाबाहो परां निकृतिमास्थितान्}


\twolineshloka
{न ह्यर्जुनसमः कश्चिद्युधि योद्धा धनुर्धरः}
{भविता वा पुमान्कश्चिन्मत्समो वा गदाधरः}


\twolineshloka
{सत्वेन कुरुते युद्धं राजन्सुबलवानपि}
{न प्रमाणेन नोत्साहात्सत्वस्थो भव पाण्डव}


\twolineshloka
{सत्वं हि मूलमर्थस् यवितथं यदतोऽन्यथा}
{न तु प्रसक्तं भवति वृक्षच्छायेन हैमनी}


\twolineshloka
{अर्थत्यागोऽपि कार्यः स्यादर्थं श्रेयांसमिच्छता}
{बीजौपम्यन कौन्तेय मा ते भूदत्र संशयः}


\twolineshloka
{अर्थेन तु समोऽनर्थो यत्र लब्धो महोदयः}
{न तत्र विपणः कार्यः खरकण्डूयनं हि तत्}


\twolineshloka
{एवमेव मनुष्येन्द्र धर्मं त्यक्त्वाऽल्पकं नरः}
{बृहन्तं धर्ममाप्नोति स बुद्ध इति निश्चितम्}


% Check verse!
अमित्रं मित्रसंपन्नं मित्रैर्भिन्दन्ति पण्डितः ॥भिन्नैर्मित्रैः परित्यक्तं दुर्बलं कुरुते वशे
\twolineshloka
{सत्वेन कुरुते युद्धं राजन्सुबलवानपि}
{नोद्यमेन न होत्राभिः सर्वाः स्वीकुरुते प्रजाः}


\twolineshloka
{सर्वथा संहतैरेव दुर्बलैर्बलवानपि}
{अमित्रः शक्यते हन्तुं मधुहा भ्रमरैरिव}


\twolineshloka
{यथा राजन्प्रजाः सर्वाः सूर्यः पाति गभस्तिभिः}
{हन्ति चैव तथैव त्वं सदृशः सवितुर्भव}


\twolineshloka
{एतच्चापि तपो पाजन्पुराणमिति नः श्रुतम्}
{विधिना पालनं भूमेर्यत्कृतं नः पितामहैः}


\twolineshloka
{न तथा तपसा राजँल्लोकान्प्राप्नोति क्षत्रियः}
{यथा सृष्टेन युद्धेन विजयेनेतरेण वा}


\twolineshloka
{अपेयात्किल भा सूर्याल्लक्ष्मीश्चन्द्रमसस्तथा}
{इतिलोको व्यवसितो दृष्ट्वेमां भवतोव्यथाम्}


\twolineshloka
{भवतश्च प्रशंसाभिर्निन्दाभिरितरस्य च}
{कथयन्त्यः परिषदः पृथग्राजन्समागताः}


\twolineshloka
{इदमभ्यधिकं राजन्ब्राह्मणाः कुरवश्च ते}
{समेताः कथयन्तीह मुदिताः सत्यसन्धताम्}


\twolineshloka
{यन्न मोहान्न कार्पण्यान्न लोभान्न भयादपि}
{अनृतं किंचिदुक्तं ते न कामान्नार्थकारणात्}


\twolineshloka
{यदेनः कुरुते किंचिद्राजा भूमिमवाप्नुवन्}
{सर्वं तन्नुदते पश्चाद्यज्ञैर्विपुलदक्षिणैः}


\twolineshloka
{ब्राह्मणेभ्यो ददद्ग्रामान्गाश्च राजन्सहस्रशः}
{मुच्यते वीर पापेभ्यस्तमोभ्य इव चन्द्रमाः}


\twolineshloka
{पौरजानपदाः सर्वे प्रायशः कुरुनन्दन}
{सवृद्धबालाः सहिताः शंसन्ति त्वां युधिष्ठिर}


\twolineshloka
{श्वदृतौ क्षीरमासक्तं ब्रह्म वा वृषले यथा}
{सत्यं स्तेने बलं नार्यां राज्यं दुर्योधने तथा}


\twolineshloka
{इतिलोके निर्वचनं पुरश्चरति भारत}
{अपि चैताः स्त्रियो बालाः स्वाध्यायमधिकुर्वते}


\twolineshloka
{इमामवस्थां च गते सहास्माभिररिंदम्}
{हन्त नष्टाः स्म सर्वे वै भवतोपद्रवे सति}


\twolineshloka
{स भवान्रथमास्थाय सर्वोपकरणान्वितम्}
{त्वरमाणोऽभिनिर्यातु विप्रेभ्योऽर्थविभावकः}


\twolineshloka
{वाचयित्वा द्विजश्रेष्ठानद्यैव गजसाह्वयम्}
{अस्त्रविद्भिः परिवृतोभ्रातृभिर्दृढधन्विभिः}


\threelineshloka
{आशीविषसमैर्वीरैर्मरुद्भिरिव वृत्रहा}
{अमित्रांस्तेजसा मृद्गन्नसुरानिव वृत्रहा}
{श्रियमादत्स्व कौन्तेय धार्तराष्ट्रान्महाबल}


\twolineshloka
{न हि गाण्डीवमुक्तानां शराणां गार्ध्रवाससाम्}
{स्पर्शमाशीविषाभानां मर्त्यः कश्चन संसहेत्}


\twolineshloka
{न स वीरो न मातङ्गो न च सोऽश्वोस्ति भारत}
{यः सहेत गदावेगं मम क्रुद्धस्य संयुगे}


\twolineshloka
{सृञ्जयैः सहकैकेयैर्वृष्णीनां वृषभेण च}
{कथंस्विद्युधि कौन्तेय न राज्यं प्राप्नुयामहे}


\twolineshloka
{शत्रुहस्तगतां राजन्कथंस्विन्नाहरेर्महीम्}
{इह यत्नमुपाहृत्य बलेन महताऽन्वितः}


\chapter{अध्यायः ३४}
\twolineshloka
{वैशंपायन उवाच}
{}


\twolineshloka
{स एवमुक्तस्तु महानुभावःसत्यव्रतो भीमसेनेन राजा}
{अजातशत्रुस्तदनन्तरं वैधैर्यान्वितो वाक्यमिदं बभाषे}


\twolineshloka
{असंशयं भारत सत्यमेत-द्यन्मां तुदन्वाक्यशल्यैः क्षिणोषि}
{न त्वां विगर्हे प्रतिकूलमेत-न्ममानयाद्धि व्यसनं व आगात्}


\twolineshloka
{अहं ह्यक्षानन्वपद्यं जिहीर्षन्राज्यंसराष्ट्रं धृतराष्ट्रस्य पुत्रात्}
{तन्मां शठः कितव प्रत्यदेवी-त्सुयोधनार्थं सुबलस्य पुत्रः}


\twolineshloka
{महामायः शकुनिः पार्वतीयःसभामध्ये प्रवपन्नक्षपूगान्}
{अमायिनं मायया प्रत्यजैषी-त्ततोऽपश्यं वृजिनं भीमसेन}


\twolineshloka
{अक्षांश्च दृष्ट्वा शकुनेर्यथावत्कामानुकूलानयुजो युजश्च}
{शक्यो नियन्तुमभविष्यदात्मामन्युस्तु हन्यात्पुरुषस्य धैर्यम्}


\twolineshloka
{यन्तुं नात्मा शक्यते पौरुषेणमानेन वीर्येण च तात नद्धः}
{न ते वाचो भीमसेनाभ्यसूयेमन्ये तथा तद्भवितव्यमासीत्}


\twolineshloka
{स नो राजा धृतराष्ट्रस्य पुत्रोन्यपातयद्व्यसने राज्यमिच्छन्}
{दास्यं च नोऽगमयद्भीमसेनयत्राभवच्छरणं द्रौपदी नः}


\twolineshloka
{त्वं चापि तद्वेत्थ धनंजयश्चपुनर्द्यूतायागतास्तां सभां नः}
{यन्माऽब्हवीद्धृतराष्ट्रस्य पुत्रएकग्लहार्थं भरतानां समक्षम्}


\twolineshloka
{वने समा द्वादश राजपुत्रयथाकामं विदितमजातशत्रो}
{अथापरं चाविदितश्चरेथाःसर्वैः सह भ्रातृभिश्छद्मरूपः}


\twolineshloka
{त्वां चेच्छ्रुत्वा तात तथा चरन्त-मवभोत्स्यन्ते भारतानां चराश्च}
{अन्यांश्चरेथास्तावतोऽब्दांस्तथात्वंनिश्चित्य तत्प्रतिजानीहि पार्थ}


\twolineshloka
{चरैश्चेन्नो विदितः कालमेतंयुक्तो राजन्मोहयित्वा मदीयान्}
{ब्रवीमि सत्यं कुरुसंसदीहतवैव ता भारत पञ्च नद्यः}


\twolineshloka
{वयं चैतद्भारत सर्व एवत्वया जिताः कालमपास्य भोगान्}
{चरेम इत्याह पुरा स राजामध्ये कुरूणां स मयोक्तस्तथेति}


\twolineshloka
{तत्र द्यूतमभवन्नो जघन्यंयस्मिञ्जिताः प्रव्रजिताश् सर्वे}
{इत्थं तु देशाननुसंचरामोवनानि कृच्छ्राणि च कृच्छ्ररूपाः}


\twolineshloka
{सुयोधनश्चापि न शान्तिमिच्छत्भूयः स मन्योर्वशमन्वगच्छत्}
{उद्योजयामास कुरूंश्च सर्वान्येचास्य केचिद्वशमन्वगच्छन्}


\twolineshloka
{तां सिद्धिमास्थाय सतां सकाशेको नाम जह्यादिह राज्यहेतोः}
{आर्यस्य मन्ये मरणाद्गरीयोयद्धर्ममुत्क्रम्य महीं प्रशासेत्}


\twolineshloka
{तदैव चेद्वीरकर्माकरिष्योयदा द्यूते परिघं पर्यमृक्षः}
{बाहू दिधक्षन्वारितः फल्गुनेनकिं दुष्कृतंभीम तदाऽभविष्यत्}


\twolineshloka
{प्रागेव चैवं समयक्रियायाःकिं नाब्रवीः पौरुषमाविदानः}
{प्राप्तं तु कालं त्वभिपद्यपश्चात्किं मामिदानीमतिवेलमात्थ}


\twolineshloka
{भूयोपि दुःखं मम भीमसेनदूये विषस्येव रसं हि पीत्वा}
{यद्याज्ञसेनीं परिक्लिश्यमानांसंदृश्य तत्क्षान्तमिति स्म भीम}


\twolineshloka
{न त्वद्य शक्यं भरतप्रवीरकृत्वा यदुक्तं कुरुवीरमध्ये}
{कालं प्रतीक्षस्व सुखोदयायपाकं फलानामिव बीजवापः}


\twolineshloka
{यदा हि पूर्वं निकृतोनिकृन्ते-द्वैरं सपुष्पं सफलं विदित्वा}
{महागुणं हरति हि पौरुषेणतदा वीरो जीवति जीवलोके}


\twolineshloka
{श्रियं च लोके लभते समग्रांमन्ये चास्मै शत्रवः सन्नमन्ते}
{मित्राणि चैनमतिरागाद्भजन्तेदेवा इवेनद््रमुपजीवन्ति चैनम्}


\twolineshloka
{मम प्रतिज्ञां च निबोध सत्यांवृणे सत्यममृतीज्जीविताच्च}
{राज्यं च पुत्राश्च यशो धनं चसर्वं न सत्यस्य कलामुपैति}


\chapter{अध्यायः ३५}
\twolineshloka
{भीमसेन उवाच}
{}


\twolineshloka
{सन्धिं कृत्वैव कालेन ह्यन्तकेन पतत्रिणा}
{अनन्तेनाप्रमेयेण स्रोतसा सर्वहारिणा}


\twolineshloka
{प्रत्यक्षं मन्यसे कालं मर्त्यः सन्कालबन्धनः}
{फेनधर्मा महाराज फलधर्मा तथैव च}


\twolineshloka
{निमेषादपि कौन्तेय यस्यायुरपचीयते}
{सूच्येवाञ्जनचूर्णानि किमिति प्रतिपालयेत्}


\twolineshloka
{यो नूनममितायुः स्यादथवाऽऽयुःप्रमाणवित्}
{स कालं वै प्रतीक्षेत सर्वप्रत्यक्षदर्शिवान्}


\twolineshloka
{प्रतीक्ष्यमाणः कालो नः समा राजंस्त्रयोदश}
{आयुषोपचयं कृत्वा मरणायोपनेष्यति}


\twolineshloka
{शरीरिणां हि मरणं शरीरे नित्यमाश्रितम्}
{प्रागेव मरणात्तस्माद्राज्यायैव घटामहे}


\twolineshloka
{यो न याति प्रसंख्यानमस्पष्टो भूमिवर्धनः}
{अयातयित्वा वैराणि सोऽवसीदति गौरिव}


\twolineshloka
{यो न यातयते वैरमल्पसत्वोद्यमः पुमान्}
{अफलं जन्म तस्याहं मन्ये दुर्जातजीविनः}


\twolineshloka
{हैरण्यौ भवतो बाहू श्रुतिर्भवति पार्थिवी}
{हत्वा द्विषन्तं संग्रामे भुङ्क्ष्व बाहुजितं वसु}


\twolineshloka
{हत्वा वै पुरुषो राजन्निकर्तारमरिंदम्}
{अह्नाय नरकं गच्छेत्स्वर्गेणास्य स संमितः}


\twolineshloka
{अमर्षजो हि संतापः पावकाद्दीप्तिमत्तरः}
{येनाहमभिसंतप्तो न नक्तं न दिवा शये}


\twolineshloka
{अयं च पार्थो बीभत्सुर्वरिष्ठो ज्याविकर्षणे}
{आस्ते परमसंतप्तो नूनं सिंह इवाशये}


\twolineshloka
{योऽयमेको नुदेत्सर्वांल्लोके राजन्धनुर्भृतः}
{सोयमात्मजमूष्माणं ग्रहाहस्तीव यच्छति}


\twolineshloka
{नकुलः सहदेवश्च वृद्धा माता च वीरसूः}
{तवैव प्रियमिच्छन्त आसते जडमूकवत्}


\twolineshloka
{सर्वे तेऽप्रियमिच्छन्ति बान्धवाः सह सृञ्जयैः}
{अहमेकश्च संतप्तो माता च प्रतिबिन्ध्यतः}


\twolineshloka
{प्रियमेव तु सर्वेषां यद्ब्रवीम्युत किंचन}
{सर्वे हि व्यसनं प्राप्ताः सर्वे युद्धाभिनन्दिनः}


\twolineshloka
{नातः पापीयसी काचिदापद्राजन्भविष्यति}
{यन्नो नीचैरल्पबलै राज्यमाच्छिद्य भुज्यते}


\twolineshloka
{शीलदोषाद्धृणाविष्ट आनृशंस्यात्परंतप}
{क्लेशांस्तितिक्षसे राजन्नान्यः कश्चित्प्रशंसति}


\twolineshloka
{श्रोत्रियस्येव ते राजन्मन्दकस्याविपश्चितः}
{अनुवाकहता बुद्धिर्नैषा तत्त्वार्थदर्शिनी}


\twolineshloka
{घृणी ब्राह्मणरूपोसि कथं क्षत्रेष्वजायथाः}
{अस्यां हि योनौ जायन्ते प्रायशः क्रूरबुद्धयः}


\twolineshloka
{अश्रौषीस्त्वं राजधर्मान्यथा वै मनुरब्रवीत्}
{क्रूरान्निकृतिसंपन्नान्विहितानशमात्मकान्}


\twolineshloka
{धार्तराष्ट्रान्महाराज क्षमसे किं दुरात्मनः}
{कर्तव्ये पुरुषव्याघ्र किमास्से पीठसर्पवत्}


\threelineshloka
{बुद्ध्या वीर्येण संयुक्तः श्रुतेनाभिजनेन च}
{तृणानां मुष्टिनैकेन हिमवन्तं च पर्वतम्}
{छन्नमिच्छसि कौन्तेय योऽस्मान्संवर्तुमिच्छसि}


\twolineshloka
{अज्ञातचर्यागूढेन पृथिव्यां विश्रुतेन च}
{दिवीव पार्थ सूर्येण न शक्याऽऽचरितुं त्वया}


\twolineshloka
{बृहत्साल इवानूपे शाखापुष्पपलावान्}
{हस्ती श्वेत इवाज्ञातः कथं जिष्णुश्चरिष्ति}


\twolineshloka
{इमौ च सिंहसंकाशौ भ्रातरौ सहितौ शिशू}
{नकुलः सहदेवश्च कथं पार्थ चरिष्यतः}


\twolineshloka
{पुण्यकीर्ती राजपुत्री द्रौपदी वीरसूरियम्}
{विश्रुता कथमज्ञाता कृष्णा पार्थ चरिष्यति}


\twolineshloka
{मां चापि राजञ्जानन्ति ह्याकुमारमिमाः प्रजाः}
{अज्ञातचर्यां पश्यन्ति मेरोरिव निगूहनम्}


\twolineshloka
{तथैव बहव्रोऽस्माभी राष्ट्रेभ्यो विप्रवासिताः}
{राजानो राजपुत्राश्च धार्तराष्ट्रमनुव्रताः}


\twolineshloka
{न हि तेऽप्युपशाम्यन्ति निकृता वा निराकृताः}
{अवश्यं तैर्निकर्तव्यमस्माकं तत्प्रियैषिभिः}


\twolineshloka
{तेऽप्यस्मासु प्रयुञ्जीरन्प्रच्छन्नान्सुबहूंश्चरान्}
{आचक्षीरंश्च नो ज्ञात्वा ततः स्यात्सुमहद्भयम्}


\twolineshloka
{अस्माभिरुषिताः सम्यग्वने मासास्त्रयोदश}
{परिमाणएन तान्पश्य तावतः परिवत्सरान्}


\twolineshloka
{अस्ति मासः प्रतिनिधिर्यथा प्राहुर्मनीषिणः}
{पूतिकानिव सोमस्य तथेदं क्रियतामिति}


\twolineshloka
{अथवाऽनडुहे राजन्साधने साधुवाहिने}
{सौहित्यदानादेतस्मादेनसः प्रतिमुच्यते}


\twolineshloka
{तस्माच्छत्रुवधे राजन्क्रियतां निश्चयस्त्वया}
{क्षत्रियस्य हि सर्वस्य नान्यो धर्मोस्ति संयुगात्}


\chapter{अध्यायः ३६}
\twolineshloka
{वैशंपायन उवाच}
{}


\twolineshloka
{भीमसेनवचः श्रुत्वा कुन्तीपुत्रो युधिष्ठिरः}
{निःश्वस्य पुरुषव्याघ्रः संप्रदध्यौ परंतपः}


\twolineshloka
{श्रुता मे राजधर्माश्च वर्णानां च पृथक्पृथक्}
{आयत्यां च तदात्वे च यः पश्यति स पश्यति}


\twolineshloka
{धर्मस्य जानमानोऽहं गतिमग्र्यां सुदुर्विदाम्}
{कथं बलात्करिष्यामि मेरोरिव विवर्तनम्}


\twolineshloka
{स मुहूर्तमिव ध्यात्वा विनिश्चित्येतिकृत्यताम्}
{भीमसेनमिद वाक्यमपदान्तरमब्रवीत्}


\twolineshloka
{एवमेतन्महाबाहो यथा वदसि भारत}
{इदमन्यत्समादत्स्व वाक्यं मे वाक्यकोविद}


\twolineshloka
{महापापानि कर्माणि यानि केवलसाहसात्}
{आरभ्यन्ते भीमसेन व्यथन्ते तानि भारत}


\twolineshloka
{सुमन्त्रिते सुविक्रान्ते सुकृतेसुविचारिते}
{सिद्ध्यन्त्यर्था महाबाहो दैवं चात्र प्रदक्षिणम्}


\twolineshloka
{यत्तु केवलचापल्याद्बलदर्पोत्थितः स्वयम्}
{आरब्धव्यमिदं कार्यं मन्यसे शृणु तत्रमे}


\twolineshloka
{भूरिश्रवाः शलश्चैव जलसन्धश्च वीर्यवान्}
{भीष्मो द्रोणश्च कर्णशच् द्रोणपुत्रश्च वीर्यवान्}


\twolineshloka
{धार्तराष्ट्रा दुराधर्षा दुर्योधनपुरोगमाः}
{सर्व एव कृताशस्त्राश्च सततं चाततायिनः}


\twolineshloka
{राजानः पार्थिवाश्चैव येऽस्माभिरुपतापिताः}
{ते श्रिताः कौरवं पक्षं जातस्नेहाश्च सांप्रतम्}


\twolineshloka
{दुर्योधनहिते युक्ता न तथाऽस्मासु भारत}
{पूर्णकोशा बलोपेताः प्रयतिष्यन्ति रक्षणे}


\twolineshloka
{सर्वे कौरवसैन्यस् सपुत्रामात्यसैनिकाः}
{संविभक्ता हि मात्राभिर्भोगैरपि च सर्वशः}


\twolineshloka
{दुर्योधनेन ते वीरा मानिताश्च विशेषतः}
{प्राणांस्त्यक्ष्यन्ति संग्रामे इति मे निश्चिता मतिः}


\twolineshloka
{समा यद्यपि भीष्मस्य वृत्तिरस्मासु तेषु च}
{द्रोणस् च महाबाहो कृषस्य च महात्मनः}


\twolineshloka
{अवश्यं राजपिण्डस्तैर्निर्वेश्य इति मे मतिः}
{तस्मात्त्यक्ष्यन्ति संग्रामे प्राणानपि सुदुस्त्यजान्}


\twolineshloka
{सर्वेदिव्यास्त्रविद्वांसः सर्वे धर्मपरायणाः}
{अजेयाश्चेति मे बुद्धिरपि देवैः सवासवैः}


\twolineshloka
{अमर्षी नित्यसंरब्धस्तत्र कर्णो महारथः}
{सर्वास्त्रविदनाधृष्यो ह्यभेद्यकवचावृतः}


\twolineshloka
{अनिर्जित्यरणे सर्वानेतान्पुरुषसत्तमान्}
{अशक्यो ह्यसहायेन हन्तुं दुर्योधनस्त्वया}


\threelineshloka
{न निद्रामधिगच्छामि चिन्तयानो वृकोदर}
{अतिसर्वान्धनुर्ग्राहान्सूतपुत्रस्य लाघवम् ॥वैशंपायन उवाच}
{}


\twolineshloka
{एतद्वचनमाज्ञाय भीमसेनोऽत्यमर्षणः}
{बभूव शान्तिसंयुक्तो गुरोर्वचनवारितः}


\twolineshloka
{तयोः संवदतोरेवं तदा पाण्डवयोर्द्वयोः}
{आजगाम महायोगी व्यासः सत्यवतीसुतः}


\twolineshloka
{सोऽभिगम्य यथान्यायं पाण्डवैः प्रतिपूजितः}
{युधिष्ठिरमिदं वाक्यमुवाच वदतांवरः}


\twolineshloka
{युधिष्ठिर महाबाहो वेद्मि ते हृदयस्थितम्}
{मनीषया तत क्षिप्रमागतोस्मि नरर्षभ}


\twolineshloka
{भीष्माद्द्रोणात्कृपात्कर्णाद्द्रोणपुत्राच्च भारत}
{दुर्योधनान्नृपसुतात्तथा दुःशासनादपि}


\twolineshloka
{त्ते भयममित्रघ्न हृदि संपरिवर्तते}
{तत्तेऽहं नाशयिष्यामि विधिदृष्टेन हेतुना}


\twolineshloka
{तच्छ्रुत्वा धृतिमास्थाय कर्मणा प्रतिपादय}
{प्रतिपाद्य तु राजेन्द्र ततः क्षिप्रं ज्वरं जहि}


\twolineshloka
{तत एकान्तमानाय्य पाराशर्यो युधिष्ठिरम्}
{अब्रवीदुपपन्नार्थमिदं वाक्यविशारदः}


\twolineshloka
{श्रेयसस्तेऽपरः कालः प्राप्तो भरतसत्तम}
{येनाभिभविता शत्रून्रणे पार्थो धनुर्धरः}


\twolineshloka
{गृहाणेमां मया प्रोक्तां सिद्धिं मूर्तिमर्तामिव}
{विद्यां प्रतिस्मृतिं नाम प्रपन्नाय ब्रवीमि ते}


\twolineshloka
{यामवाप्यमहाबाहुरर्जुनः साधयिष्यति}
{अस्त्रहेतोर्महेन्द्रं च रुद्रं चैवाभिगच्छतु}


\twolineshloka
{वरुणं च कुबेरं च धर्मराजं च पाण्डव}
{शक्तो ह्येष सुरान्द्रष्टुं तपसा विक्रमेण च}


\twolineshloka
{ऋषिरेष मेहातेजा नारायणसहायवान्}
{पुराणः शास्वतो देवस्त्वजेयो जिष्णुरच्युतः}


\twolineshloka
{अस्त्राणीनद्राच्च रुद्राच्च लोकपालेभ्य एव च}
{समादाय महाबाहुर्महत्कर्म करिष्यति}


\twolineshloka
{वनादस्माच्च कौन्तेय वनमन्यद्विचिन्त्यताम्}
{निवासार्थाय यद्युक्तं भवेद्वः पृथिवीपते}


\twolineshloka
{एकत्रचिरवासो हि न प्रीतिजननो भवेत्}
{तापसानां च शान्तानां भवेदुद्वेगकारकः}


\threelineshloka
{मृगाणामुपरोधश्च वीरुदोषधिसंक्षयः}
{बिभर्षि च बहून्विप्रान्वेदवेदाङ्गपारगान् ॥वैशंपायन उवाच}
{}


\twolineshloka
{एवमुक्त्वा प्रपन्नाय शुचये भगवान्प्रभुः}
{प्रोवाच योगतत्त्वज्ञो योगविद्यामनुत्तमाम्}


\twolineshloka
{धर्मराजाय धीमान्स व्यासः सत्यवतीसुतः}
{अनुज्ञाय च कौन्तेयं तत्रैवान्तरधीयत}


\twolineshloka
{युधिष्ठिरस्तु धर्मात्मा तद्ब्रह्म मनसा यतः}
{धारयामास मेधावी कालेकाले सदाऽभ्यसन्}


\twolineshloka
{स व्यासवाक्यमुदितो वनाद्द्वैतवनात्ततः}
{ययौ सरस्वतीकूले काम्यकं नाम काननम्}


\twolineshloka
{तमन्वयुर्महाराज शिक्षाक्षरविशारदाः}
{ब्राह्मणास्तपसा युक्ता देवेन्द्रमृषयो यथा}


\twolineshloka
{ततः काम्यकमासाद्य पुनस्ते भरतर्षभ}
{न्यविशन्त महात्मानः सामान्याः सपदानुगाः}


\twolineshloka
{तत्रते न्यवसन्राजन्कंचित्कालं मनस्विनः}
{धनुर्वेदपरा वीरा गृणाना वेदमुत्तमम्}


\twolineshloka
{चरन्तो मृगयां नित्यं शुद्धैर्बाणैर्मृगार्थिनः}
{पितृदैवतविप्रेभ्यो निर्वपन्तो यथाविधि}


\chapter{अध्यायः ३७}
\twolineshloka
{वैशंपायन उवाच}
{}


\twolineshloka
{कस्यचित्त्वथ कालस्य धर्मराजो युधिष्ठिरः}
{संस्मृत्य मुनिसंदेशमिदं वचनमब्रवीत्}


\twolineshloka
{विविक्ते विदितप्रज्ञमर्जुनं पुरुषर्षभ}
{सान्त्वपूर्वं स्मितं कृत्वापाणिना परिसंस्पृशन्}


\twolineshloka
{स मुहूर्तमिव ध्यात्वा वनवासमरिंदमः}
{धनंजयंधर्मराजो रहसीदमुवाच ह}


\twolineshloka
{भीष्मे द्रोणे कृपे कर्णे द्रोणपुत्रे च भारत}
{धनुर्वेदश्चतुष्पाद एतेष्वद्य प्रतिष्ठितः}


\twolineshloka
{दैवं ब्राह्मं चासुरं च सप्रयोगचिकित्सितम्}
{सर्वास्त्राणां प्रयोगं च अभिजानन्ति कृत्स्नशः}


\threelineshloka
{ते सर्वे धृतराष्ट्रस्य पुत्रेण परिसान्त्विताः}
{संविभक्ताश्च तुष्टाश्च गुरुवत्तेषु वर्तते}
{सर्वयोधेषु चैवास्य सदा प्रीतिरनुत्तमा}


\twolineshloka
{आचार्या मानितास्तुष्टाः शान्तिं व्यवहरन्न्युत}
{शक्तिं न हापयिष्यन्ति ते काले प्रतिपूजिताः}


\twolineshloka
{अद्य चेयं मही कृत्स्ना दुर्याधनवशानुगा}
{सग्रामनगरा पार्थ ससागरवनाकरा ॥भवानेव प्रियोऽस्माकं त्वयि भारः समाहितः}


\twolineshloka
{अत्र कृत्यं प्रपश्यामि प्राप्तकालमरिंदम}
{कृष्णद्वैपायनात्तात गृहीतोपनिषन्मया}


\twolineshloka
{गृहीतया तया सम्यग्जगत्सर्वं प्रकाशते}
{तेन त्वं ब्रह्मणा तात संयुक्तः सुसमाहितः}


\twolineshloka
{देवतानां यथाकालं प्रसादं प्रतिपालय}
{तपसा योजयात्मानमुग्रेण भरतर्षभ}


\twolineshloka
{धनुष्मान्कवची खङ्गी मुनिः साधुव्रते स्थितः}
{न कस्यचिद्ददन्मार्गं गच्छ तातोत्तरां दिशम्}


\twolineshloka
{इनद्रे ह्यस्त्राणि दिव्यानि समस्तानि धनंजय}
{वृत्राद्भीतैर्बलं देवैस्तदा शक्रे समर्पितम्}


\twolineshloka
{तान्येकस्थानि सर्वाणि ततस्त्वं प्रतिपत्स्यसे}
{`अनेन ब्रह्मणा तात सर्वं संप्रतिपद्यते ॥'}


\threelineshloka
{शक्रमेव प्रपद्यस्व स तेऽस्त्राणि प्रदास्यति}
{दीक्षितोऽद्यैव गच्छ त्वं द्रष्टुं देवं पुरंदरम् ॥वैशंपायन उवाच}
{}


% Check verse!
एवमुक्त्वा धर्मराजस्तमध्यापयत प्रभुः
\twolineshloka
{दीक्षितं विधिना तेन यतवाक्कायमानसम्}
{अनुजज्ञे तदा वीरं भ्राता भ्रातरमग्रजः}


\twolineshloka
{निदेशाद्धर्मराजस्य द्रष्टुकामः पुरंदरम्}
{धनुर्गाण्डीवमादाय तथाऽक्षय्ये महेषुधी}


\twolineshloka
{कवची सतलत्राणो बद्धगोधाङ्गुलित्रवान्}
{हत्वाऽग्निं ब्राह्मणान्निष्कैः स्वस्तिवाच्य मह्यभुज}


\twolineshloka
{प्रातिष्ठत महाबाहुः प्रगृहीतशरासनः}
{वधाय धार्तराष्ट्राणां निःश्वस्योर्ध्वमुदैक्षत}


\twolineshloka
{तं दृष्ट्वा तत्रकौन्तेयं प्रगृहीतशरासनम्}
{अब्रुवन्ब्रह्मणाः सिद्धा भूतान्यन्तर्हितानि च}


\twolineshloka
{क्षिप्रमाप्नुहि कौन्तेय मनसा यद्यदिच्छसि}
{संसाधयस्व कौन्तेय ध्रुवोस्तु विजयस्तव}


\twolineshloka
{अब्रुवन्ब्राह्मणाः पार्थमिति कृत्वा जयाशिषः}
{`सिद्धचारणसङ्घाश्च गन्धर्वाश्च तमब्रुवन्}


\twolineshloka
{स्वस्तिव्रतमुपाधत्स्व सङ्कल्पस्तव सिद्ध्यताम्}
{मनोरथाश्च ते सर्वे समृद्ध्यन्तां महारथ}


\twolineshloka
{एवमुक्तोऽभिवाद्यैतान्बद्धाञ्जलिपुटस्तथा}
{तपोयोगमनाः पार्थः पुरोहितमवन्दत}


\twolineshloka
{ततः प्रीतमना जिष्णुस्तावुभावभ्यवन्दत}
{सहोदरावतिरथौ युधिष्ठिरवृकोदरौ}


\twolineshloka
{तं क्लान्तमनसौ पूर्णमभिगम्य महारथौ}
{यमौ गाण्डीवधन्वानमभ्यवादयतामुभौ}


\threelineshloka
{अभिवाद्य तु तौ वीरावूचतुः पाकशासनिम्}
{अवाप्तव्यानि सर्वाणि दिव्यान्यस्त्राणि वासवात्}
{3-37-29cअस्त्राण्याप्नुहि कौन्तेय मनसा यद्यदिच्छसि}


\twolineshloka
{गिरो ह्यशिथिलाः सर्वा निर्दोषाः संमताः सताम्}
{त्वमेकः पाण्डवेष्वद्य संप्राप्तोसि धनञ्जय}


\twolineshloka
{न चाधर्मविदं देवा नासिद्धं नातपस्विनम्}
{द्रष्टुमिच्छन्ति कौन्तेय चलचित्तं शठं न च}


\twolineshloka
{रोरूयमाणः कटुकमीर्ष्यकः कटुकाक्षरः}
{शठकः श्लाघकः क्षेप्ता हन्ता च विचिकित्सिता}


\twolineshloka
{विश्वस् हन्ता मायावी क्रोधनोऽनृतभाषिता}
{अत्याशी नास्तिकोऽदाता मित्रध्रुक्सर्वशङ्कितः}


\twolineshloka
{आक्रोष्टा चातिमानी च रौद्रो लुब्धोऽथ लोलुपः}
{स्तेनश्च मद्यपश्चैव भ्रूणहा गुरुतल्पगः}


\twolineshloka
{संभावितात्मा चात्यर्थं नृशंसः पुरुषश्च यः}
{नैते लोकानाप्नुवन्ति निर्लोकास्ते धनञ्जय}


\twolineshloka
{आनृशंस्यमनुक्रोशः सत्यं करुणवेदिता}
{दमः स्थितिर्धृतिर्धर्मः क्षमा कृत्यमनुत्तमम्}


\threelineshloka
{दया शमश्च धर्मश्च गुरुपूजा कृतज्ञता}
{मैत्रता द्विजभक्तिश्च वसन्ति त्वयि फल्गुन}
{व्यपेक्षा सर्वभूतेषु कृपा दानं मतिः स्मृतिः}


\twolineshloka
{तस्मात्कौरव्य शक्रेण समेष्यसि धनञ्जय}
{त्वादृशेन हि देवानां श्लाघनीयः समागमः}


\twolineshloka
{सुहृदां सोदराणां च सर्वेषां भरतर्षभ}
{त्वं गतिः परमा तात वृत्रहा मरुतामिव}


\twolineshloka
{तस्मिंस्त्रयोदशे वर्षे भ्रातरः सुहृदश्च ते}
{सर्वे हिसंश्रयिष्यन्ति बाहुवीर्यं महाबल}


\twolineshloka
{स पार्थ पितरं गच्छ सहस्राक्षमरिन्दम}
{मुष्टिग्रहणमादत्स्व सर्वाण्यस्त्राणि वासवात्}


\twolineshloka
{शतशृङ्गे महाबाहो मघवानिदमब्रवीत्}
{शृण्वतां सर्वभूतानां त्वामुपाघ्राय मूर्धनि}


\twolineshloka
{विदितः सर्वभूतानां दिवं तात गमिष्यसि}
{प्राप्य पुण्यकृतां लोकान्रंस्यसे जयतांवर}


\twolineshloka
{मानितस्त्रिदशैः पार्थ विहृत्य सुसुखं दिवि}
{अवाप्यपरमास्त्राणि पृथिवीं पुनरेष्यसि}


\twolineshloka
{गुणांस्ते वासवस्तात खाण्डवं दहति त्वयि}
{शृण्वतां सर्वभूतानां पुनः पुनरभाषत}


\twolineshloka
{तां प्रतिज्ञां नरश्रेष्ठ कर्तुमर्हसि वासवीम्}
{कंचिद्देशमितः प्राप्य तपोयोगमना भव}


\twolineshloka
{कर्तुमर्हसि कौरव्य मघवद्वचनं हितम्}
{दीक्षित्वाऽद्यैव गच्छ त्वं द्रष्टाऽसि त्वं पुरन्दरम्}


\twolineshloka
{तौ परिष्वज्य बीभत्सुः कृष्णामामन्त्र्य चाप्युभौ}
{मनांस्यादाय सर्वेषां प्रयातः पुरुषर्षभः ॥'}


\threelineshloka
{तं तथा प्रस्थितं वीरं सिंहस्कन्धोरसं तदा}
{प्राञ्जलिः पाण्डवंकृष्णा देवानां कुर्वती नमः}
{वाग्भिः परमशुद्धाभिर्मङ्गलाभिरभाषत्}


\threelineshloka
{नमो धात्रे विधात्रे च स्वस्ति गच्छ वनाद्वनम्}
{`धर्मस्त्वां जुषतां पार्थ भास्करश्च विभावसुः}
{ब्रह्मा त्वां ब्राह्मणाश्चैव पालयन्तु धनञ्जय ॥'}


\threelineshloka
{ज्येष्ठापचायी ज्येष्ठस्य भ्रातुर्वचनमास्थितः}
{प्रपद्येथा वसूव्रुद्रानादित्यान्समरुद्गणान्}
{विश्वेदेवांस्तथा साध्याञ्शान्त्यर्थं भरतर्षभ}


\twolineshloka
{स्वस्ति तेस्त्वान्तरिक्षेभ्यो दिव्येभ्यो भरतर्षभ}
{पार्थिवेभ्यश्च सर्वेभ्यो ये केचित्परिपन्थिनः}


\twolineshloka
{अवरोधाद्वने वासात्सर्वस्वहरणादपि}
{इदं दुःखतरं मन्ये पुत्रेभ्यश्च विवासनात्}


\twolineshloka
{मास्माकं क्षत्रियकुले जातुचित्पुनरुद्भवः}
{ब्राह्मणेभ्यो नमस्यामि येषां नायुधजीविका}


\twolineshloka
{इदं मे परमं दुःखं यः स पापः सुयोधनः}
{दृष्ट्वा मां गौरिति प्राह प्रहसन्राजसंसदि}


\twolineshloka
{तस्माद्दुःखादिदं दुःखं गरीय इतिमे मतिः}
{यत्तत्परिषदो मध्ये बह्वयुक्तमभाषत}


\twolineshloka
{`ध्वंसितः स्वगृहेभ्यश्च राष्ट्राच्च भरतर्षभ}
{वने प्रतिष्ठितो भूत्वा सौहार्दादवतिष्ठसे ॥'}


\twolineshloka
{जेता यः सर्वशत्रूणां यः पावकमतर्पयः}
{जनस्त्वां पश्यतीदानीं गच्छन्तं भरतर्षभ ॥'}


\twolineshloka
{अस्मिन्नूनं महारण्ये भ्रातरः सुहृदश्च ते}
{त्वत्कथाः कथयिष्यन्ति चारणा ऋषयस्तथा}


\twolineshloka
{यत्ते कुन्ती महाबाहो जातस्यैच्छद्धनञ्जय}
{तत्तेऽस्तु सर्वं कौन्तेय यथा च स्वयमिच्छसि}


\twolineshloka
{`वसुदेवस्वसा देवी त्वामार्या पुनरागतम्}
{पश्यतु त्वां पृथा पार्थ सहस्राक्षमिवादितिः'}


\twolineshloka
{नूनं ते भ्रातरः सर्वे त्वत्कथाभिः प्रजागरे}
{रंस्यन्ते तव कर्माणि कीरत्यन्तः पुनः पुनः}


\twolineshloka
{न च नः पार्थ भोगेषु न धने नोत जीविते}
{तुष्टिर्बुद्धिर्भवित्री वा त्वयि दीर्घप्रवासिनि}


\twolineshloka
{त्वयि नः पार्थ सर्वेषां सुखदुःखे प्रतिष्ठते}
{जीवितं मरणं चैव राज्यमैश्वर्यमेव च}


\twolineshloka
{आपृष्टो नोसि कौन्तेय स्वस्ति प्राप्नुहि पाण्डव}
{कृतास्त्रं स्वस्तिमन्तं त्वां द्रक्ष्यामि पुनरागतम्}


\twolineshloka
{बलवद्भिर्विरुद्धं न कार्यमेतत्त्वयाऽनघ}
{प्रयाह्यविघ्नेयैवाशु विजयाय महाबल}


\threelineshloka
{ह्रीः श्रीः कीर्तिर्धृतिः पुष्टिरुमा लक्ष्मीः सरस्वती}
{इमा वै तव पान्थस् पालयन्तु धनञ्जय ॥वैशंपायन उवाच}
{}


\twolineshloka
{एवमुक्त्वाऽऽशिषः कृष्णा विरराम यशस्विनी}
{}


\twolineshloka
{ततःप्रदक्षिणं कृत्वा भ्रातॄन्धौम्यं च पाण्डवः}
{प्रातिष्ठत महाबाहुः प्रगृह्य रुचिरं धनुः}


\threelineshloka
{`शनैरिव दिशं वीर उदीचीं भरतर्षभ}
{संहरंस्तरसा वृक्षाँल्लता वल्लीश्च भारत}
{असज्जमानो वृक्षेषु जगाम सुमहाबलः ॥'}


\twolineshloka
{तस्य मार्गादपाक्रामन्सर्वभूतानि गच्छतः}
{युक्तस्यैन्द्रेण योगेन पराक्रान्तस्य शुष्मिणः}


\twolineshloka
{सोऽगच्छत्पर्वतांस्तात तपोधननिषेवितान्}
{दिव्यं हैमवतं पुण्यं देवजुष्टं परंतपः}


\twolineshloka
{अगच्छत्पर्वतं पुण्यमेकाह्नैव महामनाः}
{मनोजवगतिर्भूत्वा योगयुक्तो यथाऽनिलः}


\twolineshloka
{हिमवन्तमतिक्रम्य गन्धमादनमेव च}
{अत्यक्रामत्स दुर्गाणि दिवारात्रमतन्द्रितः}


\twolineshloka
{इन्द्रिकीलं समासाद्य ततोऽतिष्ठद्धनंजयः}
{`गत्वा स षडहोरात्रान्सप्तमेऽहनि पाण्डवः}


\twolineshloka
{प्रस्थेन्द्रकीलस्य शुभे तपोयोगपरोऽभवत्}
{ऊर्ध्वबाहुर्न चाङ्गानि प्रास्पन्दयत किंचन}


\twolineshloka
{समाहितात्मा नियतः सहस्राक्षसुतोऽच्युतः'}
{अन्तरिक्षे स शुश्राव तिष्ठेति स वचस्तदा}


\twolineshloka
{तच्छ्रुत्वा सर्वतो दृष्टिं चारयामास रपाण्डवः}
{अथापश्यत्सव्यसाची वृक्षमूले तपस्विनम्}


\twolineshloka
{ब्राह्म्या श्रिया दीप्यमानं पिङ्गलं जटिलं कृशम्}
{सोऽब्रवीदर्जुनं तत्रस्थितं दृष्ट्वा महातपाः}


\twolineshloka
{कस्त्वं तातेह संप्राप्तो धनुष्मान्कवची शरी}
{निबद्धाऽसितलत्राणः क्षत्रधर्ममनुव्रतः}


\twolineshloka
{नेह शस्त्रेण कर्तव्यं शान्तानामेष आलयः}
{विनीतक्रोधहर्षाणां ब्राह्मणानां तपस्विनाम्}


\twolineshloka
{नेहास्ति धनुषा कार्यं न संग्रामोऽत्र कर्हिचित्}
{निक्षिपैतद्धनुस्तात प्राप्तोसि परमां गतिम्}


\threelineshloka
{इत्यनन्तौजसं वीरं यथा चान्यं पृथग्जनम्}
{तथा हसन्निवाभीक्ष्णं ब्राह्मणोऽर्जुनमब्रवीत्}
{3-3783c न चैनंचालयामास धैर्यात्सुधृतनिश्चयम्}


\twolineshloka
{तमुवाच ततः प्रीतः स द्विजः प्रहसन्निव}
{वरं वृणीष्व भद्रं ते शक्रोऽहमरिसूदन}


\twolineshloka
{एवमुक्तः सहस्राक्षं प्रत्युवाच धनंजयः}
{प्राञ्जलिः प्रणतो भूत्वा सूरः कुरुकुलोद्वहः}


\twolineshloka
{ईप्सितो ह्येष वै कामो वरं चैनं प्रयच्छ मे}
{त्वत्तोऽद्य भगवन्नस्त्रंकृत्स्नमिच्छामि वेदितम्}


\twolineshloka
{प्रत्युवाच महेन्द्रस्तं प्रीतात्मा प्रहसन्निव}
{इह प्राप्तस्य किं कार्यमस्त्रैस्तव धनंजय}


\twolineshloka
{कामान्वृणीष्व लोकांस्त्वं प्राप्तोसि परमां गतिम्}
{एवमुक्तः प्रत्युवाच सहस्राक्षं धनंजयः}


\twolineshloka
{न लोभान्न पुनः कामान्न देवत्वं पुनः सुखम्}
{न च सर्वामरैश्वर्यं कामये त्रिदशाधिप}


\twolineshloka
{भ्रातॄंस्तान्विपिने त्यक्त्वा वैरमप्रतियात्य च}
{अकीर्तिं सर्वलोकेषु गच्छेयं शाश्वतीः समाः}


\twolineshloka
{एवमुक्तः प्रत्युवाच वृत्रहा पाण्डुनन्दनम्}
{सान्त्वयञ्श्लक्ष्णया वाचा सर्वलोकनमस्कृतः}


\twolineshloka
{यदा द्रक्ष्यसि भूतेशं त्र्यक्षं शूलधरं शिवम्}
{तदा दाताऽस्मि ते तात दिव्यान्यस्त्राणि सर्वशः}


\twolineshloka
{क्रियतां दर्शने यत्नो देवस्य परमेष्ठिनः}
{दर्शनात्तस्य कौन्तेय संसिद्धः स्वर्गमेष्यसि}


\twolineshloka
{इत्युक्त्वा फल्गुनं शक्रो जगामादर्शनं पुनः}
{अर्जुनोप्यथ तत्रैव तस्थौ योगसमन्वितः}


\chapter{अध्यायः ३८}
\twolineshloka
{जनमेजय उवाच}
{}


\twolineshloka
{भगवञ्श्रोतुमिच्छासमि पार्थस्याक्लिष्टकर्मणः}
{विस्तरेण कथामेतां यथाऽस्त्राण्युपलब्धवान्}


\twolineshloka
{कथं च पुरुषव्याघ्रो दीर्घबाहुर्धनंजयः}
{वनं प्रविष्टस्तेजस्वी निर्मनुष्यमभीतवत्}


\twolineshloka
{किंचानेन कृतंतत्रवसता ब्रह्मवित्तम}
{कथं च भगवान्स्थाणुर्देवराजश्च तोषितः}


\twolineshloka
{एतदिच्छाम्यहं श्रोतुं त्वत्प्रसादाद्द्विजोत्तम}
{त्वं हि सर्वज्ञ दिव्यं च मानुषं चैव वेत्थ}


\threelineshloka
{अत्यद्भुतं महाप्राज्ञ रोमहर्षणमर्जुनः}
{भवेन सह संग्रामं चकाराप्रतिमं किल}
{पुरा प्रहरतांश्रेष्ठः संग्रामेष्वपराजितः}


\twolineshloka
{यच्छ्रुत्वा नरसिंहानां दैन्यहर्षातिविस्मयात्}
{शूराणामपि पार्थानां हृदयानि चकम्पिरे}


% Check verse!
यद्यच्च कृतवानन्यत्पार्थस्तदस्विलं वद
\threelineshloka
{न ह्यस् निन्दितं जिष्णोः सुसूक्ष्ममपि लक्षये}
{चरितं यस्य शूरस्य तन्मे सर्वं प्रकीर्तय ॥वैशंपायन उवाच}
{}


\twolineshloka
{कथयिष्यामि ते तात कथामेतां महात्मनः}
{दिव्यां कौरवशार्दूल महतीमद्भुतोपमाम्}


\twolineshloka
{गात्रसंस्पर्शसंबद्धां त्र्यम्बकेण महाहवे}
{पार्थस्य देवदेवेन शृणु सम्यक्समागमम्}


\twolineshloka
{युधिष्ठिरनियोगात्स जगामामितविक्रमः}
{शक्रं सुरेश्वरं द्रष्टुं देवदेवं च शंकरम्}


\twolineshloka
{दिव्यं तद्धनुरादाय खङ्गं च पुरुषर्षभः}
{महाबलो महाबाहुरर्जुनः कार्यसिद्धये}


\twolineshloka
{दिशं ह्युदीचीं कौरव्यो हिमवच्छिखरं प्रति}
{ऐन्द्रिः स्थिरमना राजन्सर्वलोकमहारथः}


\twolineshloka
{त्वरया परया युक्तस्तपसे धृतनिश्चयः}
{वनं कण्टकितं घोरमेक एवान्वपद्यत}


\twolineshloka
{नानापुष्पफलोपेतं नानापक्षिनिनादितम्}
{नानामृगगणाकीर्णं सिद्धचारणसेवितम्}


\twolineshloka
{ततः प्रयाते बीभत्सौ वनं मानुषवर्जितम्}
{शङ्खानां पटहानां च शब्दः समभवद्दिवि}


\twolineshloka
{पुष्पवर्षं च सुमहन्निपपात महीतले}
{मेघजालं च सततं छादयामास सर्वतः}


\twolineshloka
{सोतीत्य वनदुर्गाणि सन्निकर्षे महागिरेः}
{शुशुभे हिमवत्पृष्ठे वसमानोऽर्जुनस्तदा}


\twolineshloka
{तत्रतत्र द्रुमान्फुल्लान्विहगैर्वल्गुनादितान्}
{नदीश्च विपुलावर्ता वैदूर्यनीलसंनिभाः}


\twolineshloka
{हंसकारण्डवोद्गीताः सारसाभिरुतास्तथा}
{पुंस्कोकिलरुताश्चैव क्रौञ्चबर्हिणनादिताः}


\twolineshloka
{मनोहरवनोपेतास्तस्मिन्नतिरथोऽर्जुनः}
{पुण्यशीतामलजलाः पश्यन्प्रीतमनाऽभवत्}


\twolineshloka
{रमणीये वनोद्देशे रममाणोऽर्जुनस्तदा}
{तपस्युग्रे वर्तमान उग्रतेजा महामनाः}


\twolineshloka
{दर्भचीरं निवस्याथ दण्डाजिनबिभूषितः}
{शीर्णं च पतितं भूमौ पर्णं समुपभुक्तवान्}


\twolineshloka
{पूर्णेपूर्णे त्रिररात्रे तु मासमेकं फलाशनः}
{द्विगुणेनैव कालेन द्वितीयं मासमत्ययात्}


\twolineshloka
{तृतीयमपि मासं स क्रमेणाहारमाचरन्}
{चतुर्थे त्वथ संप्राप्ते मासे भरतसत्तमः}


\twolineshloka
{वायुभक्षो महाबाहुरभवत्पाण्डुनन्दनः}
{ऊर्ध्वबाहुर्निरालम्बः पादाङ्गुष्ठाग्रधिष्ठितः}


\twolineshloka
{सदोपस्पर्शनाच्चास्य बभूवुरमितौजसः}
{विद्युदम्भोरुहनिभा जटास्तस्य महात्मनः}


\twolineshloka
{ततो महर्षयः सर्वे जग्मुर्देवं पिनाकिनम्}
{विवेदयिषवः पार्थं तपस्युग्रे समास्थितम्}


\twolineshloka
{नीलकण्ठं महादेवमभिवाद्य प्रणम्य च}
{सर्वे निवेदयामासुः कर्म तत्फल्गुनस्य ह}


\twolineshloka
{एकः पार्थो महातेजा हिमवत्पृष्ठमाश्रितः}
{उग्रे तपरसि दुष्पारे स्थितो धूमापयन्दिशः}


\twolineshloka
{तस्य देवेश न वयं विद्मः सर्वे चिकीर्षितम्}
{संतापयति नः सर्वानसौ साधु निवार्यताम्}


\twolineshloka
{तेषां तद्वचनं श्रुत्वा मुननां भावितात्मनाम्}
{उमापतिर्भूतपतिर्वाक्यमेतदुवाच ह}


\twolineshloka
{न वो विषादः कर्तव्यः फल्गुनं प्रति सर्वशः}
{शीघ्रं गच्छत संहृष्टा यथागतमतन्द्रिताः}


\fourlineindentedshloka
{अहमस्य विजानामि संकल्पं मनसि स्थितम्}
{नास्य स्वर्गस्पृहा काचिन्नैश्वर्यस्य न चायुषः}
{यत्तस्य काङ्क्षितं प्राप्तुं तत्करिष्येऽहमद्य वै ॥वैशंपायन उवाच}
{}


\twolineshloka
{तच्छ्रुत्वा शर्ववचनमृषयः सत्यवादिनः}
{प्रहृष्टमनसो जग्मुर्यथास्वं पुनराश्रमम्}


\chapter{अध्यायः ३९}
\twolineshloka
{वैशंपायन उवाच}
{}


\twolineshloka
{गतेषु तेषु सर्वेषु तपस्विषु महात्मसु}
{पिनाकपाणिर्भगवान्सर्वपापहरो हरः}


\twolineshloka
{कैरातं वेषमास्थाय काञ्चनद्रुमसन्निभम्}
{विभ्राजमानो वपुषा गिरिर्मेरुरिवापरः}


\twolineshloka
{श्रीगद्धनुरुपादाय शरांश्चाशीविषोपमान्}
{निष्पपात महार्चिष्मान्दहन्कक्षमिवानलः}


\twolineshloka
{देव्या सहोमया श्रीमान्समानव्रतवेषया}
{नानावेषधरैर्हृष्टैर्भूतैरनुगतस्तदा}


\twolineshloka
{किरातवेषसंछन्नः स्त्रीभिश्चापि सहस्रशः}
{अशोभत तदा राजन्स देशोऽतीव भारत}


\twolineshloka
{क्षणेन तद्वनं सर्वं निःशब्दमभवत्तदा}
{नादः प्रस्रवणानां च पक्षिणां चाप्युपारमत्}


\twolineshloka
{स सन्निकर्षमागम्य पार्थस्याक्लिष्टकर्मणः}
{मूकं नाम दितेः पुत्रं ददर्शाद्भुतदर्शनम्}


\twolineshloka
{वाराहं रूपमास्थाय तर्जयन्तमिवार्जुनम्}
{हन्तुं परमदुष्टात्मा तमुवाचाथ फल्गुनः}


\twolineshloka
{गाण्डीवं धनुरादाय शरांश्चाशीविषोपमान्}
{सज्यं धनुर्वरं कृत्वा ज्याघोषेण निनादयन्}


\twolineshloka
{यन्मां प्रार्थयसे हन्तुमनागसमिहागतम्}
{तस्मात्त्वां पूर्वमेवाहं नेष्यामि यमसादनम्}


\twolineshloka
{दृष्ट्वा तं प्रहरिष्यन्तं फल्गुनं दृढधन्विनम्}
{किरातरूपी सहसा वारयामास शंकरः}


\twolineshloka
{मयैष प्रार्थितः पूर्वं नीलमेघसमप्रभः}
{अनादृत्यैव तद्वाक्यं प्रजहाराथ फल्गुनः}


\twolineshloka
{किरातश्च समं तस्मिन्नेकलक्ष्ये महाद्युतिः}
{प्रमुमोचाशनिप्रख्यं शरमग्निशिखोपमम्}


\twolineshloka
{तौ मुक्तौ सायकौ ताभ्यां समं तत्र निपेततुः}
{मूकस्य गात्रे विस्तीर्णे शैलपृष्ठनिभे तदा}


\twolineshloka
{यथाऽशनैर्विनिष्पेषो वज्रस्येव च पर्वते}
{तथा तयोः सन्निपातः शरयोरभवत्तदा}


\twolineshloka
{स विद्धो बहुभिर्बाणैर्दीप्तास्यैः पन्नगैरिव}
{ममार राक्षसं रूपं भूयः कृत्वा सुदारुणम्}


\twolineshloka
{स ददर्श ततो जिष्णुः पुरुषं काञ्चनप्रभम्}
{किरातवेषसंछन्नं स्त्रीसहायममित्रहा}


\twolineshloka
{तमब्रवीत्प्रीतमना कौन्तेयः प्रहसन्निव}
{को भवानटते शून्ये वने रस्त्रीगणसंवृतः}


\twolineshloka
{न त्वमस्मिन्वने घोरे बिभेषि कनकप्रभ}
{किमर्थं च त्वया विद्धो मृगोऽयं मत्परिग्रहः}


\twolineshloka
{मयाऽभिपन्नः पूर्वं हि राक्षसोऽयमिहागतः}
{कामात्परिभवाद्वाऽपि न मे जीवन्विमोक्ष्यसे}


\twolineshloka
{न ह्येष मृगयाधर्मो यस्त्वयाऽद्य कृतो मयि}
{तेन त्वां भ्रंशयिष्यामि जीवितात्पर्वताश्रयम्}


\twolineshloka
{इत्युक्तः पाण्डवेयेन किरातः प्रहसन्निव}
{उवाच श्लक्ष्णया वाचा पाण्डवं सव्यसाचिनं}


\twolineshloka
{न मत्कृते त्वया वीरः भीः कार्या वनमन्तिकात्}
{इयं भूमिः सदाऽस्माकमुचिता वसतां वने}


\twolineshloka
{त्वया तु दुष्करः कस्मादिह वासः प्ररोचितः}
{वयं तु बहुसत्त्वेऽस्मिन्निवसामस्तपोधन}


\threelineshloka
{भवांस्तु कृष्णवत्मार्भः सुकुमारः सुखोचितः}
{कथं शून्यमिमं देशमेकाकी विचरिष्यति ॥अर्जुन उवाच}
{}


\twolineshloka
{गाण्डीवमाश्रयं कृत्वा नाराजांश्चाग्निसन्निभान्}
{निवसामि महारण्ये द्वितीय इव पावकः}


\threelineshloka
{एष चापि मया जन्तुर्मृगरूपं समाश्रितः}
{राक्षसो निहतो घोरो हन्तुं मामिह चागतः ॥किरात उवाच}
{}


\twolineshloka
{मयैष धन्वनिर्मुक्तैस्ताडितः पूर्वमेव हि}
{बाणैरभिहतः शेते नीतश्च यमसादनम्}


\twolineshloka
{ममैवायं लक्ष्यभूतः पूर्वमेव परिग्रहः}
{ममैव च प्रहारेण जीविताद्व्यपरिपितः}


\twolineshloka
{दोषान्स्वान्नार्हसेऽन्यस्मै वक्तुं स्वबलदर्पितः}
{अभिषक्तोस्मि मन्दात्मन्न मे जीवन्विमोक्ष्यसे}


\twolineshloka
{स्थिरो भव विमोक्ष्यामि सायकानशनीनिव}
{घटस्व परया शक्त्या मुञ्च त्वमपि सायकान्}


\twolineshloka
{तस्य तद्वचनं श्रुत्वा किरातस्यार्जुनस्तदा}
{रोषमाहारयामास ताडयामास चेषुभिः}


\twolineshloka
{ततो हृष्टेन मनसा प्रतिजग्राह सायकान्}
{भूयोभूय इति प्राह मन्दमन्देत्युवाच ह}


\twolineshloka
{प्रहरस्व शरानेतान्नाराचान्मर्मभेदिनः}
{इत्युक्तो बाणवर्षं स मुमोच सहसाऽर्जुनः}


\twolineshloka
{ततस्तौ तत्रसंरब्धौ गर्जमानौ मुहुर्मुहुः}
{शरैराशीविषाकारैस्ततक्षाते परस्परम्}


\twolineshloka
{ततोऽर्जुनः शरवर्षं किराते समवासृजत्}
{तत्प्रसन्नेन मनसा प्रतिजग्राह शंकरः}


\twolineshloka
{मुहूर्तं शरवर्षं तु प्रतिगृह्य पिनाकधृक्}
{अक्षतेन शरीरेण तस्थौ गिरिवाचलः}


\twolineshloka
{स दृष्ट्वा बाणवर्षं तु मघीभूतं धनंजयः}
{परमं विस्मयं चक्रे साधुसाध्विति चाब्रवीत्}


\twolineshloka
{अहोऽयं सुकुमाराङ्गो हिमवच्छिखराश्रयः}
{गाण्डीवमुक्तान्नाराचान्प्रतिगृह्णात्यविह्वलः}


\twolineshloka
{कोऽयं देवो भवेत्साक्षाद्रुद्रो यक्षः सुरोऽसुरः}
{विद्यते हि गिरिश्रेष्ठे त्रिदशानां समागमः}


\twolineshloka
{न हि मद्वाणजालानामुत्सृष्टानां सहस्रशः}
{शक्तोऽन्यः सहितुं वेगमृते देवं पिनाकिनम्}


\twolineshloka
{देवो वा यदि वा यक्षो रुद्रादन्यो व्यवस्थितः}
{अहमेनं शरैस्तीक्ष्णैर्नयामि यमसादनम्}


\twolineshloka
{ततो हृष्टमना जिष्णुर्नाराचान्मर्मभेदिनः}
{व्यसृजच्छतधा राजन्मयूखानिव भास्करः}


\twolineshloka
{तान्प्रसन्नेन मनसा भगवाँल्लोकभावनः}
{शूलपाणिः प्रत्यगृह्णाच्छिलावर्षमिवाचलः}


\twolineshloka
{क्षणेन क्षीणबाणोऽथ संवृत्तः फल्गुनस्तदा}
{वित्रासं च जगामाथ तं दृष्ट्वा शरसंक्षयम्}


\twolineshloka
{चिन्तयामास जिष्णुस्तु भगवन्तं हुताशनम्}
{पुरस्तादक्षयौ दत्तौ तूणौ येनास्य खाण्डवे}


\twolineshloka
{किं नु मोक्ष्यामि धनुषा यन्मे बाणाः क्षयं गताः}
{अयं च पुरुषः कोपि बाणान्ग्रसति सर्वशः}


\twolineshloka
{अहमेनं धनुष्कोट्या रशूलाग्रेणेव कुञ्जरम्}
{नयामि दण्डधारस्य यमस्य सदनं प्रति}


\twolineshloka
{प्रगृह्याथ धनुष्कोट्या ज्यापाशेनावकृष्य च}
{मुष्टिभिश्चापि हतवान्वज्रकल्पैर्महाद्युतिः}


\twolineshloka
{संप्रायुध्यद्धनुष्कोट्या कौन्तेयः परवीरहा}
{तदप्यस्य धनुर्दिव्यं जग्राह गिरिगोचरः}


\twolineshloka
{ततोऽर्जुनो ग्रस्तधनुः खङ्गपाणिरतिष्ठत}
{युद्धस्यान्तमभीप्सन्वै वेगेनाभिजगाम तम्}


\twolineshloka
{तस्य मूर्ध्नि शितं खङ्गमसक्तं पर्वतेष्वपि}
{मुमोय भुजवीर्येण विक्रम्य कुरुनन्दनः}


\twolineshloka
{तकस्य मूर्धानमासाद्य पफालासिवरो हि सः}
{ततो वृक्षैः शिलाभिश्च योधयामास फल्गुनः}


\twolineshloka
{तदा वृक्षान्महाकायः प्रत्यगृह्णादथो शिलाः}
{किरातरूपी भगवांस्ततः पार्थो महाबलः}


\twolineshloka
{मुष्टिभिर्वज्रसंस्पर्शैर्धूममुत्पादयन्मुखे}
{प्रजहार दुराधर्षे किरातसमरूपिणि}


\twolineshloka
{ततः शक्राशनिसमैर्मुष्टिभिर्भृशदारुणैः}
{किरातरूपी भगवानर्दयामास फल्गुनम्}


\twolineshloka
{ततश्चटचटाशब्दः सुधोरः समजायत}
{पाण्डवस्य च मुष्टीनां किरातस् च युध्यतः}


\twolineshloka
{सुमुहूर्तं तयोर्युद्धमभवल्लोमहर्षणम्}
{भुजप्रहारसंयुक्तं वृत्रवासवयोरिव}


\twolineshloka
{महाराज ततो जिष्णुः किरातमुरसा बली}
{पाण्डवं च विचेष्टन्तं किरातोप्यहनद्बलात्}


\twolineshloka
{तयोर्भुजविनिष्पेषात्संधर्षेणोरसोस्तथा}
{समजायत गात्रेषु पावकोऽङ्गारधूमवान्}


\twolineshloka
{तत एनं महादेवः पीड्य गात्रैः सुपीडितम्}
{तेजसा व्क्रमद्रोषाच्चेतस्तस्य विमोहयन्}


\twolineshloka
{ततोऽभिपीडितैर्गात्रैः पिण्डीकृत इवाबभौ}
{फल्गुनो गात्रसंरुद्धो देवदेवेन भारत}


\twolineshloka
{निरुच्छ्वासोऽभवच्चैव सन्निरुद्धो महामनाः}
{ततः पपात संमूढस्ततः प्रीतोऽभवद्भवः}


\twolineshloka
{स मुहूर्तं तथा भूत्वा सचेताः पुनरुत्थितः}
{रुधिरेणाप्लुताङ्गस्तु पाण्डवो भृशदुःखितः}


\twolineshloka
{शरण्यं शरणं गत्वा भगवन्तं पिनाकिनम्}
{मृन्मयं स्ण्डिलं कृत्वामाल्येनापूजयद्भवम्}


\twolineshloka
{तच्च माल्यं तदा पार्थः किरातशिरसि स्थितम्}
{अपश्यत्पाण्डवश्रेष्ठो हर्षेण प्रकृतिं गतः}


\fourlineindentedshloka
{पपात पादयोस्तस्य ततः प्रीतोऽभवद्भवः}
{[उवाच चैनं वचसा मेघगम्भीरगीर्हसः}
{जातविस्मयमालोक्य ततः क्षीणाङ्गसंहतिम् ॥भव उवाच}
{}


\twolineshloka
{भोभो फल्गुन तुष्टोस्मि कर्मणाऽप्रतिमेन ते}
{शौर्येणानेन धृत्या च क्षत्रियो नास्ति ते समः}


\twolineshloka
{समं तेजश्च वीर्यं च ममाद्य तव चानघ}
{प्रीतस्तेऽहं महाबाहो पश्य मां भरतर्षभ}


\twolineshloka
{ददामि ते विशालाक्ष चक्षुः पूर्वं मुनिर्भवान्}
{विजेष्यसि रणे शत्रूनपि सर्वान्दिवौकसः}


\threelineshloka
{[प्रीत्या च तेऽहं दास्यामि यदस्त्रमनिवारितम्}
{त्वं हि शक्तो मदीयं तदस्त्रं धारयितुं क्षणात् ॥]वैशंपायन उवाच}
{}


\twolineshloka
{ततो देवं महादेवं गिरिशं शूनपाणिनम्}
{ददर्श फल्गुनस्तत्र सह देव्या महाद्युतिम्}


\threelineshloka
{स जानुभ्यां महीं गत्वा शिरसा प्रणिपत्य च}
{प्रसादयामास हरं पार्थः परपुरंजयः ॥अर्जुन उवाच}
{}


\twolineshloka
{कपर्दिन्सर्वभूतेश भगनेत्रनिपातन}
{[देवदेव महादेव नीलग्रीव जटाधर}


\twolineshloka
{कारणानां च परमं जाने त्वां त्र्यम्बकं विभुम्}
{देवानां च गतिं देवं त्वत्प्रसूतमिदं जगत्}


\twolineshloka
{अजेयस्त्वं त्रिभिर्लोकैः सदेवासुरमानुषैः}
{शिवाय विष्णुरूपाय विष्णवे शिवरूपिणे}


\twolineshloka
{दक्षियज्ञविनाशाय हरिरूपाय ते नमः}
{ललाटाक्षाय शर्वाय मीढुषे शूलपाणये}


\twolineshloka
{पिनाकगोप्त्रे सूर्याय मङ्गल्याय च वेधसे}
{प्रसादये त्वां भगवन्सर्वभूतमहेश्वर}


\twolineshloka
{गणेशं जगतः शम्भुं लोककारणकारणम्}
{प्रधानपुरुषातीतं परं सूक्ष्मतरं हरम्}


\twolineshloka
{व्यतिक्रमं मे भगवन्क्षन्तुमर्हसि शंकर}
{भगवन्दर्शनाकाङ्क्षी प्राप्तोस्मीमं महागिरिम्}


\twolineshloka
{दयितं तव देवेश तापसालयमुत्तमम्}
{प्रसादये त्वां भगवन्सर्वलोकनमस्कृतम्}


\threelineshloka
{कृतो मयाऽयमज्ञानाद्विमर्दो यस्त्वया सह}
{शरणं प्रतिपन्नाय तत्क्षमस्वाद्य शंकर ॥वैशंपायन उवाच}
{}


\twolineshloka
{तमुवाच महातेजाः प्रहसन्वृषभध्वजः}
{प्रगृह्य रुचिरं बाहुं क्षान्तमित्येव फल्गुनम्}


\twolineshloka
{परिष्वज्य च बाहुभ्यां प्रीतात्मा भगवान्हरः}
{पुनः पार्थं सान्त्वपूर्वमुवाच वृषभध्वजः}


\twolineshloka
{`गङ्गाङ्गितजटः शर्वः पार्थस्यामिततेजसः}
{प्रगृह्य रुचिरं बाहुं वृत्तं ताम्रतलाङ्गुलिम् ॥'}


\chapter{अध्यायः ४०}
\twolineshloka
{देवदेव उवाच}
{}


\twolineshloka
{नरस्त्वं पूर्वदेहे वै नारायणसहायवान्}
{बदर्यां तप्तवानुग्रं तपो वर्षायुतान्बहून्}


\twolineshloka
{त्वयि वा परमं तेजो विष्णौ वा पुरुषोत्तमे}
{युवाभ्यां पुरुषाग्र्याभ्यां तेजसा धार्यते जगत्}


\twolineshloka
{शक्राभिषेके सुमहद्धनुर्जलदनिःस्वनम्}
{प्रगृह्य दानवाः शस्तास्त्वया कृष्णेन च प्रभो}


\twolineshloka
{तदेतदेव गाण्डीवं तव पार्थ करोचितम्}
{भायामास्थाय तद्ग्रस्तं मया पुरुषसत्तम}


\twolineshloka
{तूणौ चाप्यक्षयौ भूयस्तव पार्थ करोचितौ}
{भविष्यति शरीरं च नीरुजं कुरुनन्दन}


\twolineshloka
{प्रीतिमानस्मि वै पार्थ भवान्सत्यपराक्रमः}
{गृहाण वरमस्मत्तः काङ्क्षितं यन्नरर्षभ}


\threelineshloka
{न त्वया सदृशः कश्चित्पुमान्मर्त्येषु भारत}
{दिवि वा वर्तते त्रं त्वत्प्रधानमरिंदम ॥अर्जुन उवाच}
{}


\twolineshloka
{वरं ददासि चेन्मह्यं कामं प्रीत्या वृषध्वज}
{कामये दिव्यामस्त्रं तद्धोरं पाशुपतं प्रभो}


\twolineshloka
{यत्तद्ब्रह्मशिरो नाम रौद्रं भीमपराक्रमम्}
{युगान्ते दारुणे प्राप्ते कृत्स्नं संहरते जगत्}


\twolineshloka
{[कर्णभीष्मकृपद्रोणैर्भविता तु महाहवः}
{त्वत्प्रसादान्महादेव जयेयं तान्यथा युधि ॥]}


\twolineshloka
{जयेयं येन संग्रामे दानवान्राक्षसांस्तथा}
{राज्ञश्चैव पिशाचांश्च गन्धर्वानथपन्नगान्}


\twolineshloka
{यस्मिञ्शूलसहस्राणि गदाश्चोग्रप्रदर्शनाः}
{शराश्चाशीविषाकाराः संभवन्त्यनुमन्त्रिताः}


\twolineshloka
{युध्येयं येन भीष्मेण द्रोणेन च कृपेण च}
{सूतपुत्रेण च रणे नित्यं कटुकभाषिणा}


\threelineshloka
{एष मे प्रथमः कामो भगवन्भगनेत्रहन्}
{त्वत्प्रसादाद्विनिर्बृत्तः समर्थः स्यामहं यथा ॥भव उवाच}
{}


\twolineshloka
{ददामि तेऽस्त्रं दयितमहं पाशुपतं विभो}
{समर्थो धारणे मोक्षे संहारेऽपि च पाण्डव}


\twolineshloka
{न तद्वेद महेन्द्रोपि न यमो न च यक्षराट्}
{वरुणोप्यथवा वायुः कुतो वेत्स्यन्ति मानवाः}


\twolineshloka
{न त्वया सहसा पार्थ मोक्तव्यं पुरुषे क्वचित्}
{जगद्विनिर्दहेत्सर्वमल्पतेजसि पातितम्}


\threelineshloka
{अबध्यो नाम नास्त्यस्य त्रैलोक्ये सचराचरे}
{मनसा चक्षुषा वाचा धनुषा च निपात्यते ॥वैशंपायन उवाच}
{}


\twolineshloka
{तच्छ्रुत्वा त्वरितः पार्थः शुचिर्भूत्वा समाहितः}
{उपसंगृह्य विश्वेशमधीष्वेत्यथ सोऽब्रवीत्}


\twolineshloka
{ततस्त्वध्यापयामास सरहस्यनिवर्तनम्}
{तदस्त्रं पाण्डवश्रेष्ठं मूर्तिमन्तमिवान्तकम्}


\twolineshloka
{उपतस्थे महात्मानं यथा त्र्यक्षमुमापतिम्}
{प्रतिजग्राह तच्चापि प्रीतिमानर्जुनस्तदा}


\twolineshloka
{ततश्चचाल पृथिवी सपर्वतवनद्रुमा}
{ससागरवनोद्देशा सग्रामनगराकरा}


\twolineshloka
{शङ्खदुन्दुभिघोषाश्च भेरीणां च सहस्रशः}
{तस्मिनमुहूर्ते संप्राप्ते निर्घातश्च महानभूत्}


\twolineshloka
{अथास्त्रं जाज्वलद्धोरं पाण्डवस्यामितौजसः}
{मूर्तिमद्विष्ठितं पार्श्वे ददृशुर्देवदानवाः}


\twolineshloka
{स्पृष्टस्य त्र्यम्बकेणाथ फल्गुनस्यामितौजसः}
{यत्किं गच्छेत्यनुज्ञातस्त्र्यम्बकेण तदाऽर्जुनः}


\threelineshloka
{स्वर्गं गच्छेत्यनुज्ञातस्त्र्यम्बकेण तदाऽर्जुनः}
{प्रणम्य शिरसा राजन्प्राञ्जलिर्भवमैक्षत}
{}


\twolineshloka
{ततः प्रभुस्त्रिदिवनिवासिनां वसीमहामतिर्गिरिश उमापतिः शिवः}
{धनुर्महद्दितिजपिशाचसूदनंददौ भवः पुरुषवराय गाण्डिवम्}


\twolineshloka
{ततः शुभं गिरिवरमीश्वरस्तदासहोमयाऽसिततटसानुकन्दरम्}
{विहाय तं पतगमहर्षिसेवितंजगाम स्वं पुरुषवरस्य पश्यतः}


\chapter{अध्यायः ४१}
\twolineshloka
{वैशंपायन उवाच}
{}


\twolineshloka
{तस्य संपश्यतस्त्वेव पिनाकी गोवृषध्वजः}
{जगामादर्शनं भानुर्लोकस्येवास्तमीयिवान्}


\twolineshloka
{ततोऽर्जुनः परं चक्रे विस्मयं परवीरहा}
{मया साक्षान्महादेवो दृष्ट इत्येव भारत}


\twolineshloka
{धन्योस्म्यनुगृहीतोस्मि यन्मया त्र्यम्बको हरः}
{पिनाकी वरदो रूपी दृष्टः स्पृष्टश्ट पाणिना}


\twolineshloka
{कृतार्थं चावगच्छामि परमात्मानमात्मना}
{शत्रूंश्च विजितान्सर्वान्निर्वृत्तं च प्रयोजनम्}


\threelineshloka
{इत्येवं चिन्तयानस्य पार्थस्यामिततेजसः}
{ततो वैडूर्यवर्णाभो भासयन्सर्वतो दिशः}
{यादोगणवृतः श्रीमानाजगाम जलेश्वरः}


\twolineshloka
{नागैर्नदैर्नदीभिश्च दैत्यैः साध्यैर्मरुद्गणैः}
{वरुणो यादसांभर्ता वशी तं देशमागमत्}


\twolineshloka
{अथ जाम्बूनदवपुर्विमानेन महार्चिषा}
{कुबेरः समनुप्राप्तो यक्षैरनुगतः प्रभुः}


\twolineshloka
{विद्योतयन्निवाकाशमद्भुतोपमदर्शनः}
{धनानामधिपः श्रीमानर्जुनं द्रष्टुमागतः}


\twolineshloka
{तथा लोकान्तकः श्रीमान्यमः साक्षात्प्रतापवान्}
{मर्त्यमूर्तिधरैः सार्धं पितृभिर्लोकभावनैः}


\twolineshloka
{दण्डपाणिरचिन्त्यात्मा सर्वभूतविनाशकृत्}
{वैवस्वतो धर्मराजो विमानेनावभासयन्}


\twolineshloka
{त्रील्लोकान्गुह्यकांश्चैव गन्धर्वाश्चैव पन्नगान्}
{द्वितीय इव मार्ताण्डो युगान्ते समुपस्थिते}


\twolineshloka
{भानुमन्ति विचित्राणि शिखराणि महागिरेः}
{समास्थायार्जुनं तत्रददृशुस्तपसान्वितम्}


\twolineshloka
{ततो मुहूर्ताद्भगवानैरावतशिरोगतः}
{आजगाम सहेन्द्राण्या शक्रः सुरगणैर्वृतः}


\twolineshloka
{पाण्डुरेणातपत्रेण ध्रियमाणेन मूर्धनि}
{शुशुभे नागराजस्थः सितमभ्रमिव स्थितः}


\twolineshloka
{संस्तूयमानो गन्धर्वैर्ऋषिभिश्च तपोधनैः}
{शृङ्गं गिरेः समासाद्य तस्थौ सूर्य इवोदितः}


\twolineshloka
{अथ मेघस्वनो धीमान्व्याजहार शुभां गिरम्}
{यमऋ परमधर्मज्ञो दक्षिणां दिशमास्थितः}


\twolineshloka
{अर्जुनार्जुन पश्यास्माँल्लोकपालान्समागतान्}
{दृष्टिं ते वितरामोऽद्य भवानर्हति दर्शनम्}


\twolineshloka
{पूर्वर्षिरमितात्मा त्वं नरो नाम महाबलः}
{नियोगाद्ब्रह्मणस्तात मर्त्यतां समुपागतः}


\twolineshloka
{त्वया च वसुसंभूतो महावीर्यः पितामहः}
{भीष्मः परमधर्मात्मा जेतव्यश्च रणेऽनघ}


\twolineshloka
{क्षत्रं चाग्निसमस्पर्शं भारद्वाजेन रक्षितम्}
{निवातकवचाश्चैव संसाध्याः कुरुनन्दन}


\twolineshloka
{पितुर्ममांशो देवस्य सर्वलोकप्रतापिनः}
{कर्णश्व सुमहावीर्यस्त्वया वध्यो धनंजय}


\threelineshloka
{अंशाश्च क्षितिसंप्राप्ता देवगन्धर्वरक्षसाम्}
{त्वया निपातिता युद्धेस्वकर्मफलनिर्जिताम्}
{गतिं प्राप्स्यन्ति कौन्तेय यथास्वमरिकर्शन}


\twolineshloka
{अक्षया तव कीर्तिश्च लोके स्थासय्ति फल्गुन}
{त्वया साक्षान्महादेवस्तोषितो हि महामृधे}


% Check verse!
लघ्वी वसुमती चापि कर्तव्या विष्णुना सह
\threelineshloka
{गृहाणास्त्रं महाबाहो दण्डमप्रतिवारणम्}
{अनेनास्त्रेण सुमहत्त्वं हि कर्म करिष्यसि ॥वैशंपायन उवाच}
{}


\twolineshloka
{प्रतिजग्राह पार्थोपि विधिवत्कुरुनन्दनः}
{समन्त्रं सोपरोधं च समोक्षविनिवर्तनम्}


\twolineshloka
{ततो जलधरश्यामो वरुणो यादसांपतिः}
{पश्चिमां दिशमास्थाय गिरमुच्चारयन्प्रभुः}


\twolineshloka
{पार्थ क्षत्रियमुख्यस्त्वं क्षत्रधर्मे व्यवस्थितः}
{पश्य मां पृथुताम्राक्ष वरुणोस्मि जलेश्वरः}


\twolineshloka
{मया समुद्यतान्पाशान्वारुणाननिवारितान्}
{प्रतिगृह्णीष्व कौन्तेय सहरहस्यनिवर्तनान्}


\twolineshloka
{एभिस्तदा मया वीर संग्रामे तारकामये}
{दैतेयानां सहस्राणि संयतानि महात्मनाम्}


\twolineshloka
{तस्मादिमान्महासत्व मत्प्रसादसमुत्थितान्}
{गृहाण न हि ते मुच्येदन्तकोप्याततायिनः}


\twolineshloka
{अनेन त्वं यदाऽस्त्रेण संग्रामे विचरिष्यसि}
{तदा निःक्षत्रिया भूमिर्भविष्यति न संशयः}


\twolineshloka
{`ततस्तान्वारुणानस्त्रान्दिव्यानस्त्रविदांवरः}
{प्रतिजग्राह विधिवद्वरुणाद्वासविस्तदा ॥'}


\twolineshloka
{ततः कैलासनिलयो धनाध्यक्षोऽभ्यभापत}
{दत्तेष्वस्त्रेषु दिव्येषु वरुणेन यमेन च}


\twolineshloka
{प्रीतोऽहमपि ते प्राज्ञ पाण्डवेय महाहल}
{त्वया सह समागम्य अजितेन तथैव च}


\twolineshloka
{सव्यसाचिन्महाबाहो पूर्वदेव सनातन}
{सहास्माभिर्भवाञ्श्रान्तः पुराकल्पेषु नित्यशः}


\twolineshloka
{दर्शनं ते त्विदं दिव्यं प्रदिशामि नरर्षभ}
{अमानुषान्महाबाहो दुर्जयानपि जेष्यसि}


\twolineshloka
{मत्तश्चैव भवानाशु गृह्णात्वस्त्रमनुत्तमम्}
{अनेन त्वमनीकानि धार्तराष्ट्रस्य धक्ष्यसि}


\twolineshloka
{मत्तोऽपि त्वं गृहाणास्त्रमन्तर्धानं प्रियं भम}
{ओजस्तेजोद्युतिकरं प्रस्वापनमरातिनुत्}


\twolineshloka
{महात्मना शंकरेण त्रिपुरं निहतं यादा}
{तदैतदस्त्रं निर्मुक्तं येन दग्धा महासुराः}


\threelineshloka
{त्वदर्थमुद्यतं चेदं मया सत्यपराक्रम}
{त्वमर्हो धारणे चास्य मेरुप्रतिमगौरव}
{}


\twolineshloka
{ततोऽर्जुनो महाबाहुर्विधिवत्कुरुनन्दनः}
{कौबेरमधिजग्राह दिव्यमस्त्रं महाबलः}


\twolineshloka
{ततोऽब्रवीद्देवराजः पार्थमक्लिष्टकारिणम्}
{सान्त्वयञ्श्लक्ष्णयावाचादिव्यदुन्दुभिनिःस्वनः}


\twolineshloka
{कुन्तीमातर्महाबाहो त्वमीशानः पुरातनः}
{परां सिद्धिमनुप्राप्तः साक्षाद्देवगतिं गतः}


\twolineshloka
{देवकार्यं तु सुमहत्त्वया कार्यमरिंदम}
{आरोढव्यस्त्वया स्वर्गः सज्जीभव महाद्युते}


\twolineshloka
{रथो मातलिसंयुक्त आगतस्त्वत्कृते मम}
{तत्र तेऽहंप्रदास्यामि सर्वाण्यस्त्राणि कौरव}


\twolineshloka
{तान्दृष्ट्वा लोकपालांस्तु समेतान्गिरिमूर्धनि}
{जगाम विस्मयं धीमान्कुन्तीपुत्रो धनंजयः}


\twolineshloka
{ततोऽर्जुनो महातेजा लोकपालान्समागतान्}
{पूजयामास विधिवद्वाग्भिरद्भिः फलैरपि}


\twolineshloka
{ततः प्रतिययुर्देवाः प्रतिपूज्य धनंजयम्}
{यथागतेन विबुधाः सर्वे कामं मनोजवाः}


\twolineshloka
{ततो ऽर्जुनो मुदं लेभे लब्धास्त्रः पुरुषर्पभः}
{कृतार्थमथ चात्मानं स मेने पूर्णमानसम्}


\chapter{अध्यायः ४२}
\twolineshloka
{वैशंपायन उवाच}
{}


\twolineshloka
{गतेषु लोकपालेषु पार्थः शत्रुनिबर्हणः}
{चिन्तयामास राजेन्द्र देवराजरथागमम्}


\twolineshloka
{तस्य चिन्तयमानस्य गुडाकेशस्य धीमतः}
{रथो मातलिसंयुक्त आजगाम महाप्रभः}


\threelineshloka
{नभो वितिमिरं कुर्वञ्जलदान्पाटयन्निव}
{दिशः संपूरयन्नादैर्महामेघरवोपमैः}
{}


\twolineshloka
{असयः शक्तयो भीमा गदाश्चोग्रप्रदर्शनाः}
{दिव्यप्रभावाः प्रासाश्च विद्युतश्च महाप्रभाः}


\twolineshloka
{तथैवाशनयश्चैव चक्रयुक्तास्तुलागुडाः}
{वायुस्फोटाः सनिर्घाताः शङ्खमेघस्वनास्तथा}


\twolineshloka
{तत्रनागा महाकाया ज्वलितास्याः सुदारुणाः}
{सिताभ्रकूटप्रतिमाः संहताश्च यथोपलाः}


\twolineshloka
{दशवाजिसहस्राणि हरीणां वातरंहसाम्}
{वहन्ति यं नेत्रमुषं दिव्यं मायामयं रथम्}


\twolineshloka
{तत्रापश्यन्महानीलं वैजयन्तं महाप्रभम्}
{ध्वजमिन्दीवरश्यामं वंशं कनकभूषणम्}


\twolineshloka
{तस्मिन्रथे स्थितं सूतं तप्तहेमविभूषितम्}
{दृष्ट्वा पार्थो महाबाहुर्देवराजमतर्कयत्}


\twolineshloka
{तथा तर्कयतस्तस्य फल्गुनस्याथ मातलिः}
{सन्नतः प्रस्थितो भूत्वा वाक्यमर्जुनमब्रवीत्}


\twolineshloka
{भोभो शक्रात्मज श्रीमञ्शक्रस्त्वां द्रष्टुमिच्छति}
{आरोहतु भवाञ्शीघ्रं रथमिन्द्रस्य संभतम्}


\twolineshloka
{आह माममरश्रेष्ठः पिता तव शतक्रतुः}
{कुन्तीसुतमिह प्राप्तं पश्यन्तु त्रिदशालयाः}


\twolineshloka
{एष शक्रः परिवृतो देवैर्ऋषिगणैस्तथा}
{गन्धर्वैरप्सरोभिश्च त्वां दिदृक्षुः प्रतीक्षते}


\threelineshloka
{अस्माल्लोकाद्देवलोकं पाकशासनशासनात्}
{मारोह त्वं मया सार्धं लब्धास्त्रः पुनरेष्यसि ॥अर्जुन उवाच}
{}


\twolineshloka
{मातले गच्छ शीघ्रं त्वमारोहस्व रथोत्तमम्}
{राजसूयाश्वमेधानां शतैरपि सुदुर्लभम्}


\twolineshloka
{पार्थिवैः सुमहाभागैर्यज्वभिर्भूरिदक्षिणैः}
{दैवतैर्वा दुरारोहं दानवैर्वा रथोत्तमम्}


\twolineshloka
{नातप्ततपसा शक्य एष दिव्यो महारथः}
{द्रष्टुं वाऽप्यथवा स्प्रष्टुमारोढुं कुत एव च}


\threelineshloka
{त्वयि मतिष्ठिते साधो रथस्थे स्थिरवाजिनि}
{पश्चादहमथारोक्ष्ये मुकृती सत्पथं यथा ॥वैशंपायन उवाच}
{}


\twolineshloka
{तस्यतद्वचनं श्रुत्वा मातलिः शक्रसारथिः}
{आरुरोह रथं शीघ्रं हयाञ्जग्राह रश्मिभिः}


\twolineshloka
{ततोऽर्जुनो हृष्टमना गङ्गायामाप्लुतः शुचिः}
{जजाप जप्यं कौन्तेयो विधिवत्कुरुनन्दनः}


\twolineshloka
{ततः पितृन्यथान्यायं तर्पयित्वा यथाविधि}
{मन्दरं शैलराजं तमाप्रष्टुमुपचक्रमे}


\twolineshloka
{साधूनां पुण्यशीलानां मुनीनां पुण्यकर्मणाम्}
{त्वं सदा संश्रयः शैल स्वर्गमार्गाभिकाङ्क्षिणाम्}


\twolineshloka
{त्वत्प्रसादात्सदा शैल ब्राह्मणाः क्षत्रिया विशः}
{स्वर्गं प्राप्ताश्चरन्ति स्म देवैः सह गतव्यथाः}


\twolineshloka
{अद्रिराज महाशैल मुनिसंश्रय तीर्थवन्}
{गच्छाम्यामन्त्रयित्वा त्वां सुखमस्म्युषितस्त्वयि}


\twolineshloka
{तव सानूनि कुञ्जाश्च नद्यः प्रस्रवणानि च}
{तीर्थानि च सुपुण्यानि मया दृष्टान्यनेकशः}


\twolineshloka
{सुसुगन्धाश्च वैर्योघास्त्वच्छरीरविनिःसृताः}
{अमृतास्वादसृशाः पीताः प्रस्रवणोदकाः}


\twolineshloka
{शिशुर्यथा पितुश्चाङ्के सुखं शेते तटे तथा}
{मया तवाङ्के ललितं शैलराज महाप्रभो}


\twolineshloka
{अप्सरोगणसंकीर्णए ब्रह्मघोषानुनादिते}
{सुखमस्म्युषितः शैल तव सानुषु नित्यदा}


\twolineshloka
{एवमुक्त्वार्जुनः शैलमामन्त्र्य परवीरहा}
{आरुरोह रथं दिव्यं द्योतयन्निव भास्करः}


\twolineshloka
{स तेनादित्यरूपेण दिव्येनाद्भुतकर्मणा}
{ऊर्ध्वमाचक्रमे धीमान्प्रहृष्टः कुरुनन्दनः}


\twolineshloka
{सोऽदर्शनपथं यातो मर्त्यानां धर्मचारिणाम्}
{ददर्शाद्भुतरूपाणि विमानानि सहस्रशः}


\twolineshloka
{न तत्र सूर्यः सोमो वा द्योतते न च पावकः}
{स्वयैव प्रभया तत्र द्योतन्ते पुण्यलब्धया}


\twolineshloka
{तारारूपाणि यानीह दृश्यन्ते द्युतिमन्ति वै}
{आकाशे विप्रकृष्टत्वात्तनूनि सुमहान्त्यपि}


\twolineshloka
{तानि तत्र प्रभास्वन्ति रूपवन्ति च पाण्डवः}
{ददर्श स्वेषु धिष्ण्येषु दीप्तिमन्ति स्वयाऽर्चिषा}


\twolineshloka
{तत्र राजर्षयः सिद्धा वीराश्च निहता युधि}
{तपसा च जितं स्वर्गं संपेतुः शतसङ्घशः}


\twolineshloka
{गन्धर्वाणां सहस्राणि मूर्यज्वलिततेजसाम्}
{गुह्यकारनामृषीणां च तथैवाप्सरसां गणान्}


\twolineshloka
{लोकानात्मप्रभान्पश्यन्फल्गुनो विस्मयान्वितः}
{पप्रच्छ मातलिं प्रीत्या स चाप्येनमुवाच ह}


\twolineshloka
{एते सुकृतिनः पार्थ स्वेषु धिष्ण्येष्ववस्थिताः}
{तान्दृष्टवानसि विभो तारारूपाणि भूतले}


\twolineshloka
{ततोऽपश्यत्स्थितं द्वारि मत्तं विजयिनं गजम्}
{ऐरावतं चतुर्दन्तं कैलासमिव शृङ्गिणम्}


\twolineshloka
{स सिद्धमार्गमाक्रम्य कुरुपाण्डवसत्तमः}
{व्यरोचत यथापूर्वं मान्धाता पार्थिवोत्तमः}


% Check verse!
अभिचक्राम लोकान्स राज्ञां राजीवलोचनः
\twolineshloka
{[एवं स संक्रमंस्तत्र स्वर्गलोके महायशाः}
{]ततो ददर्श शक्रस्य पुरीं ताममरावतीम्}


\chapter{अध्यायः ४३}
\twolineshloka
{वैशंपायन उवाच}
{}


\twolineshloka
{ददर्श स पुरीं रम्यां सिद्धचारणसेविताम्}
{सर्वर्तुकुसुमैः पुण्यैः पादपैरुपशोभिताम्}


\twolineshloka
{तत्र सौगन्धिकानां च पुष्पाणां पुण्यगन्धिनाम्}
{उपवीज्यमानो मिश्रेण वायुना पुण्यगन्धिना}


\twolineshloka
{नन्दनं च वनं दिव्यमप्सरोगणसेवितम्}
{ददर्श दिव्यकुसुमैराह्वयद्भिरिव द्रुमैः}


\twolineshloka
{नातप्ततपसा शक्यो द्रष्टुं नानाहिताग्निना}
{स लोकः पुण्यकर्तॄणां नापि युद्धे पराङ्युखैः}


\twolineshloka
{नायज्वभिर्नाव्रतिकैर्न वेदश्रुतिवर्जितैः}
{नानाप्लुताङ्गैस्तीर्थेषु यज्ञदानबहिष्कृतैः}


\twolineshloka
{नापि यज्ञहनैः क्षुद्रैर्द्रष्टुं शक्यः कथंचन}
{पानपैर्गुरुतल्पैश्च मांसादैर्वा दुरात्मभिः}


\twolineshloka
{स तद्दिव्यं वनं पश्यन्दिव्यगीतनिनादितम्}
{प्रविवेश महाबाहुः शक्रस्य दयितां पुरीम्}


\twolineshloka
{तत्र देवविमानानि कामगानि सहस्रशः}
{संस्थितान्यभियातानि ददर्शायुतशस्तदा}


\twolineshloka
{संस्तूयमानो गन्धर्वैरप्सरोभिश्च पाण्डवः}
{पुष्पगन्धवहैः पुण्यैर्वायुभिश्चानुवीजितः}


\twolineshloka
{ततो देवाः सगन्धर्वाः सिद्धाश्च परमर्षयः}
{हृष्टाः संपूजयामासुः पार्थमक्लिष्टकारिणम्}


\twolineshloka
{आसीर्वादैः स्तूयमानो दिव्यवादित्रनिःस्वनैः}
{प्रतिपेदे महाबाहुः शङ्खदुन्दुभिनादितम्}


\twolineshloka
{नक्षत्रमार्गं विपुलं सुरवीथीति विश्रुतम्}
{इन्द्राज्ञया ययौ पार्थः स्तूयमानः समन्ततः}


\twolineshloka
{तत्र साध्यास्तथा विश्वे मरुतोऽथाश्विनौ तथा}
{आदित्या वसवो रुद्रास्तथा ब्रह्मर्षयोऽमलाः}


\twolineshloka
{राजर्षयश्च बहवो दिलीपप्रमुखा नृपाः}
{तम्बुरुर्नारदश्चैव गन्धर्वौ च हाहा हूहूः}


\twolineshloka
{तान्स सर्वान्समागम्य विधिवत्कुरुनन्दनः}
{ततोऽपश्यद्देवराजं शतक्रतुमरिंदमः}


\twolineshloka
{ततः पार्थो महाबाहुरवतीर्य रथोत्तमात्}
{ददर्श साक्षाद्देवेशं पितरं पाकशासनम्}


\twolineshloka
{पाण्डुरेणातपत्रेण हेमदण्डेन चारुणा}
{दिव्यगन्धाधिवासेन व्यजनेन विधूयता}


\twolineshloka
{विश्वावसुप्रभृतिनिर्गन्धर्वैः स्तुतिवन्दिभिः}
{स्तूयमानं द्विजाग्र्यैश्च ऋग्यजुःसामसंस्तवैः}


\twolineshloka
{ततोऽभिगम्य कौन्तेयः शिरसाऽभ्यगमद्बली}
{स चैनं वृत्तपीनाभ्यां बाहुभ्यां प्रत्यगृह्णत}


\twolineshloka
{ततः शक्रासने पुण्ये देवर्षिगणसेविते}
{शक्रः पाणौ गृहीत्वैनमुपावेशयदन्तिके}


\twolineshloka
{मूर्धि चैनमुपाघ्राय देवेन्द्रः परवीरहा}
{अङ्कमारोपयामास प्रश्रयावनतं तदा}


\twolineshloka
{सहस्राक्षनियोगात्स पार्थः शक्रासनं तदा}
{आरुरुक्षुरमेयात्मा द्वितीय इववासवः}


\twolineshloka
{ततः प्रेम्णा वृत्रशत्रुरर्जुनस्य शुभं मुखम्}
{पस्पर्श पुण्यगन्धेन करेण परिसान्त्वयन्}


\twolineshloka
{प्रमार्जमानः शनकैर्बाहू चास्यायतौ शुभौ}
{ज्याशरक्षेपकिनौ स्तम्भाविव हिरण्मयौ}


\twolineshloka
{वज्रग्रहणचिह्नेन करेण परिसान्त्वयन्}
{मुहुर्मुहुर्वज्रधरो बाहू चास्फोटयञ्शनैः}


\twolineshloka
{स्मयन्निव गुडाकेशं प्रेक्षमाणः सहस्रदृक्}
{हर्षेणोत्फुल्लनयनो न चातृप्यत वृत्रहा}


\twolineshloka
{एकासनोपविष्टौ तौ शोभयांसचक्रतुः सभाम्}
{सूर्याचन्द्रमसौ व्योम चतुर्दश्यामिवोदितौ}


\twolineshloka
{तत्र स्म गाथा गायन्ति साम्ना परमवल्गुना}
{गन्धर्वास्तुम्बुरुश्रेष्ठाः कुशला गीतसामसु}


\twolineshloka
{घृताची मेनका रम्भा पूर्वचित्तिः स्वयंप्रभा}
{उर्वशी मिश्रकेशी च दण्डगौरी वरूथीनि}


\twolineshloka
{गोपाली सहजन्या च कुम्भयोनिः प्रजागरा}
{चित्रसेना चित्रलेखा सहा च मधुरस्वरा}


\twolineshloka
{एताश्चान्याश्च ननृतुस्तत्रतत्र शुचिस्मिताः}
{चत्तप्रमथने युक्ताः सिद्धानां पद्मलोचनाः}


\twolineshloka
{महाकटितटश्रोण्यः कम्पमानैः पयोधरैः}
{कटाक्षहावमाधुर्यैश्चेतोबुद्धिमनोहरैः}


\twolineshloka
{[ततो देवाः सगन्धर्वाः समादायार्घ्यमुत्तमम्}
{शक्रस्य मतमाज्ञाय पार्थमानर्चुरञ्जसा}


\twolineshloka
{पाद्यमाचमनीयं च प्रतिग्राह्य नृपात्मजम्}
{प्रवेशयामासुरथो पुरंदरनिवेशनम्}


\twolineshloka
{एवं संपूजितो जिष्णुरुवास भवने पितुः}
{उपशिक्षन्महास्राणि ससंहाराणि पाण्डवः}


\twolineshloka
{शक्रस्य हस्ताद्दयितं वज्रमस्त्रं च दुःसहम्}
{अशनीश्च महानादा मेघबर्हिणलक्षणाः}


\twolineshloka
{गृहीतास्त्रस्तु कौन्तेयो भ्रातॄन्सस्मार पाण्डवः}
{पुरंदरनियोगाच्च पञ्चाब्दानवसत्सुखी}


\twolineshloka
{ततः शक्रोऽब्रवीत्पार्थं कृतास्त्रं काल आगते}
{नृत्यं गीतं च कौन्तेय चित्रसेनादवाप्नुहि}


\twolineshloka
{वादित्रं देवविहितं नृलोके यन्न विद्यते}
{तदर्जयस्व कौन्तेय श्रेयो वै ते भविष्यति}


\twolineshloka
{सखायं प्रददौ चास्य चित्रसेनं पुरंदरः}
{स तेन सह संगम्य रेमे पार्थो निरामयः}


\twolineshloka
{गीतवादित्रनृत्यानि भूय एवादिदेश ह}
{तथाऽपि नालभच्छर्म तरस्वी द्यूतकारितम्}


\threelineshloka
{दुःशासनवधामर्षी शकुनेः सौबलस्य च}
{ततस्तेनातुलां प्रीतिमुपागम्य क्वचित्क्वचित्}
{गान्धऱ्वमतुलं नृत्यं वादित्रं चोपलब्धवान्}


\twolineshloka
{स शिक्षितो नृत्यगुणाननेका-न्वादित्रगीतार्थगुणांश्च सर्वान्}
{न शर्ण लेभे परवीरहन्ताभ्रातॄन्स्मरन्मातरं चैव कुन्तीम्}


\chapter{अध्यायः ४४}
\twolineshloka
{वैसंपायन उवाच}
{}


\twolineshloka
{कदाचित्स हि देवेन्द्रश्चित्रसेनं रहोऽब्रवीत्}
{पार्थस्य चक्षुरुर्वश्यां सक्तं विज्ञाय वासवः}


\twolineshloka
{गन्धर्वराज गच्छाद्य प्रहितोऽप्सरसांवराम्}
{उर्वशीं पुरुषव्याघ्रं सोपातिष्ठतु फल्गुनम्}


\twolineshloka
{यथा न तामभिसृतां विद्यादस्मन्नियोगतः}
{तथा त्वया विधातव्यं स्त्रीसंसर्गविशारद}


\twolineshloka
{एवमुक्तस्तथेत्युक्त्वा सोऽनुज्ञां प्राप्य वासवात्}
{गन्धर्वराजोऽप्सरसमभ्यगादुर्वशीं वराम्}


\twolineshloka
{तां दृष्ट्वा विदितो हृष्टः स्वागतेनार्चितस्तया}
{सुखासीनः सुखासीनां स्मितपूर्वं वचोऽब्रवीत्}


\twolineshloka
{विदितं तेऽस्तु सुश्रोणि प्रेषितोऽहमिहागतः}
{त्रिदिवस्यैकराजेन त्वत्प्रसादाभिनन्दिना}


\threelineshloka
{यः स देवमनुष्येषु प्रख्यातः सहजैर्गुणैः}
{श्रिया शीलेन रूपेण श्रुतेन च बलेन च}
{प्रख्यातः शौर्यवीर्याभ्यां प्रपन्नः प्रतिभानवान्}


\twolineshloka
{तेजस्वी सौम्यीलश्च क्षमावाञ्जितमत्सरः}
{साङ्गोपनिषदान्वेदांश्चतुराख्यानपञ्चमान्}


\twolineshloka
{योऽधीते गुरुशुश्रूषां मेधां चाष्टगुणाश्रयाम्}
{ब्रह्मचर्येण दाक्ष्येण प्रसवैर्वयसाऽपि च}


\twolineshloka
{एको वै रक्षिता चैव त्रिदिवं मघवानिव}
{अकत्थनो मानयिता स्थूललक्ष्यः प्रियंवदः}


\twolineshloka
{सुहृदश्चान्नपानेन विविधेनाभिवर्षति}
{सत्यवागूर्जितो वक्ता रूपवाननहंकृतः}


\twolineshloka
{भक्तानुकम्पी कान्तश्च प्रियश्च स्थिरसंगरः}
{प्रार्थनीयैर्गुणगणैर्महेन्द्रवरुणोपमः}


\threelineshloka
{विदितस्तेऽर्जुनो वीरः स स्वर्गफलमाप्तवान्}
{तव शक्राभ्यनुज्ञातः पादावद्यप्रपद्यताम्}
{}


\twolineshloka
{एवमुक्ता स्मितं कृत्वा स्वात्मानं बहुमान्य च}
{प्रत्युवाचोर्वशी प्रीता चित्रसेनमनिन्दिता}


\twolineshloka
{यत्त्वस्य कथितः सत्यो गुणोद्देशस्त्वयाऽनघ}
{तं श्रुत्वाऽन्यं प्रियं नारी वृणुयात्किमतोऽर्जुनं}


\threelineshloka
{तस्य चाहं गुणौघेन फल्गुने जातमन्मथा}
{गच्छत्वं हि याथाकाममागमिष्याम्यहं सुखम् ॥वैशंपायन उवाच}
{}


\twolineshloka
{ततो विसृज्यगन्धर्वं कृतकृत्या शुचिस्मिता}
{उर्वशी चाकरोत्स्नानं पार्थप्रार्थनलालसा}


\twolineshloka
{स्नानालङ्कारनेपथ्यैर्गन्धमाल्यैश्च शौभनैः}
{धनंजयस्य रूपेण शरैर्मन्मथचोदितैः}


\twolineshloka
{अतिविद्धेन मनसा मन्मथेन प्रदीपिता}
{दिव्यास्तरणसंस्तीर्णे विस्तीर्णे शयनोत्तमे}


\twolineshloka
{चित्तसंकल्पभावेन सुचित्ताऽनन्यमानसा}
{मनोरथेन संप्राप्तं रमयन्तीव फल्गुनम्}


\twolineshloka
{निशाम्य चन्द्रोदयनं विगाढे रजनीमुखे}
{प्रस्थिता सा पृथुश्रोणी पार्थस्य भवनं महत्}


\twolineshloka
{मृदुकुञ्चितदीर्घेण कुसुमोत्तमधारिणा}
{केशपाशेन ललना गच्छमाना व्यराजत}


\twolineshloka
{भ्रूक्षेपालापमाधुर्यैः कान्त्या सौम्यतयाऽपि च}
{शशिनं वक्रचन्द्रेण साऽऽह्वयन्तीव गच्छती}


\twolineshloka
{दिव्याङ्गरागौ सुमुखौ दिव्यचन्दनरूषितौ}
{गच्छन्त्या हाररुचिरौ स्तनौ तस्या ववल्गतुः}


\twolineshloka
{स्तनोद्वहनसंक्षोभात्ताम्यमाना पदेपदे}
{त्रिवलीदामचित्रेण मध्येनातीव शेभिना}


\twolineshloka
{रथकूबरविस्तीर्णं नितम्बोन्नतपीवरम्}
{मन्मथायतनं शुभ्रं रशनादामभूषितम्}


\twolineshloka
{ऋषीणामपि दिव्यानां मनोव्याघातकारणम्}
{सूक्ष्मवस्त्रधरं भाति जघनं निरवद्यवत्}


\twolineshloka
{गूडगुल्फधरौ पादौ ताम्रायततलाङ्गुली}
{कूर्मपृष्ठोन्तौ चास्याः शोभेते किङ्किणीकिणौ}


\twolineshloka
{शीधुपानेन चाल्पेन तुष्टा च मदनेन च}
{विलासितैश्च विविधैः प्रेक्षणीयतराऽभवत्}


\twolineshloka
{सिद्धचारणगन्धर्वैः साध्यैर्याति विलासिनी}
{बब्वाश्चर्येऽपि वै स्वर्गे दर्शनीयतमाकृतिः}


\twolineshloka
{सुमूक्ष्मेणोत्तरीयेण मेघवर्णेन राजता}
{तन्वभ्रप्रावृता व्योम्नि चन्द्रलेखेन गच्छति}


\twolineshloka
{ततः प्राप्तातिदुष्प्रापा मनसाऽपि विकर्मभिः}
{भवनं पाण्डुपुत्रस्य फल्गुनस्य शुचिस्मिता}


\twolineshloka
{तत्र द्वारमनुप्राप्ता द्वारस्थैश्च निवेदिता}
{अर्जुनस्य नरश्रेष्ठ उर्वशी शुभलोचना}


\twolineshloka
{उपातिष्ठत तद्वेश्म निर्मलं सुमनोहरम्}
{स शङ्कितमना राजन्प्रत्यगच्छत तां निशि}


\threelineshloka
{दृष्ट्वैव चोर्वशीं पार्थो लज्जासंवृलोचनः}
{तदाऽभिवादनं कृत्वा गुरुपूजां प्रयुक्तवान् ॥अर्जुन उवाच}
{}


\threelineshloka
{अभिवादये त्वां शिरसा प्रवराप्सरसांवरे}
{किं चागमनकृत्यं ते ब्रूहि सर्वं यथातथम्}
{किमाज्ञापयसे देवि प्रेष्यस्तेऽहमुपस्थितः}


\threelineshloka
{अकामं फल्गुनं ज्ञात्वा इङ्गितज्ञा तदोर्वशी}
{गन्धर्ववचनं सर्वं श्रावयामास फल्गुनम् ॥उर्वश्युवाच}
{}


\twolineshloka
{यथा मे चित्रसेनेन कथितं मनुजोत्तम}
{नत्तेऽहं संप्रवक्ष्यामि यथा चाहमिहागता}


\twolineshloka
{उपस्ताने महेन्द्रस् वर्तमाने मनोरमे}
{तवागमनतुष्ट्या च स्वर्गस्य परमोत्सवे}


\twolineshloka
{रुद्राणां चैव सान्निध्यमादित्यानां च सर्वशः}
{समागमेऽश्विनोश्चैव वसूनां च नरोत्तम}


\twolineshloka
{महर्षीणां च सङ्घेषु राजर्षिप्रवरेषु च}
{सिद्धचारणयक्षेषु महोरगगणेषु च}


\twolineshloka
{उपविष्टेषु सर्वेषु स्थानमानप्रभावतः}
{ऋद्ध्या प्रज्वलमानेषु अग्निसोमार्कवर्ष्मसु}


\twolineshloka
{वीणासु वाद्यमानासु गन्धर्वैः शक्रनन्दन}
{दिव्ये मनोरमे गीते प्रवृत्ते पृथुलोचन}


\twolineshloka
{सर्वाप्सरःसु मुख्यासु प्रनृत्तासु कुरूद्वह}
{त्वं किलानिमिषः पार्थ मामेकां तत्र दृष्टवान्}


\twolineshloka
{तत्र चावभृथे तस्मिन्नुपस्थाने दिवौकसाम्}
{तव पित्राऽभ्यनुज्ञाता गताः स्वनिलयान्सुराः}


\twolineshloka
{तथैवाप्सरसः सर्वाविशिष्टाः स्वगृहं गताः}
{अपि चान्याश्च शत्रुघ्न तव पित्रा विसर्जिताः}


\twolineshloka
{ततः शक्रेण संदिष्टश्चित्रसेनो ममान्तिकम्}
{प्राप्तः कमलपत्राक्ष स च मामब्रवीत्स्वयम्}


\twolineshloka
{त्वत्कृतेऽहं सुरेशेन प्रेषितो वरवर्णिनि}
{प्रियं कुरु महेन्द्रस्य मम चैवात्मनश्च ह}


\twolineshloka
{शक्रतुल्यं रणे शूरं रूपौदार्यगुणान्वितम्}
{पार्थं प्रार्थय सुश्रोणि त्वमित्येवं तदाऽब्रवीत्}


\twolineshloka
{ततोऽहं समनुज्ञाता तेन पित्रा च तेऽनघ}
{तवान्तिकमनुप्राप्ता शुश्रूषितुमरिंदम}


\threelineshloka
{न केवलं हि चक्रेण प्रेषिता चाहमागता}
{चिराभिलषितो वीर ममाप्येष मनोरथः ॥वैशंपायन उवाच}
{}


\twolineshloka
{तां तथा ब्रुवतीं श्रुत्वा भृशं लज्जान्वितोऽर्जुनः}
{उवाच कर्णौ हस्ताभ्यां पिधाय त्रिदशोपमः}


\twolineshloka
{दुःश्रुतं मेऽस्तु सुभगे यन्मां वदसि भामिनि}
{गुरुदारैः समाना मे निश्चयेन वरानने}


\twolineshloka
{[यथा कुन्तीमहाभागा यथेन्राणी शची मम}
{तथा त्वमपि कल्याणीनात्र कार्या विचारणा]}


\twolineshloka
{यच्चेक्षिताऽसि विस्पष्टं विशेषेण मया शुभे}
{तच्च मे कारणं सर्वं शृणु सत्येन सुस्मिते}


\twolineshloka
{इयं पौरववंशस् जननी सुदतीति ह}
{त्वामहं दृष्टवांस्तत्र विस्मयोत्फुल्ललोचनः}


\threelineshloka
{न मामर्हसि कल्याणि अन्यथा ध्यात्रमप्सरः}
{गुरोर्गरुतरा मे त्वं मम त्वं वंशवर्धिनी ॥उर्वश्युवाच}
{}


\twolineshloka
{अनावृताश्च सर्वाः स्म देवराजाभिनन्दन}
{गुरुस्थाने न मां वीर नियोक्तुं त्वमिहार्हसि}


\twolineshloka
{पितरः सोदराः पुत्रा नप्तारो वा त्विहागताः}
{तपसा रमयन्त्यस्मान्न च तेषां व्यतिक्रमः}


\threelineshloka
{तत्प्रसीद न मामार्तां विसर्जयितुमर्हसि}
{हृच्छयेन च संतप्तां भक्तां च भज मानद ॥अर्जुन उवाच}
{}


\twolineshloka
{शृणु सत्यंवरारोहे यत्त्वां वक्ष्याम्यनिन्दिते}
{शृण्वन्तु मे दिशश्चैव विदिशश्च सदेवताः}


\twolineshloka
{यथा कुन्ती च माद्री च शची चैव समा इह}
{तथा च वंशजननी त्वं हि मेऽद्य गरीयसी}


\threelineshloka
{गच्छ मूर्ध्ना प्रपन्नोस्मि पादौ ते वरवर्णिनि}
{त्वं हि मे मातृवत्पूज्या रक्ष्योऽहं पुत्रवत्त्वया ॥वैशंपायन उवाच}
{}


\threelineshloka
{ततोऽवधूता पार्थेन उर्वशी क्रोधमूर्च्छिता}
{वेपन्ती भ्रुकुटीकक्रा शशापाथ धनंजयम् ॥उर्वश्युवाच}
{}


\twolineshloka
{तव पित्राऽभ्यनुज्ञातां स्वयं च गृहमागताम्}
{यस्मान्मां नाभिनन्देथाः कामबाणवशंगताम्}


\twolineshloka
{तस्मात्त्वं नर्तकः पार्थ स्त्रीमध्ये मानवर्जितः}
{अपुंस्त्वेन च विख्यातः पण्ढवद्विचरिष्यसि}


\twolineshloka
{एवं दत्त्वाऽर्जुने शापं स्फुरितोष्ठी श्वसन्त्यथ}
{पुनः प्रत्यागता क्षिप्रमुर्वशी स्वं निवेशनम्}


\twolineshloka
{पार्थोपि लब्ध्वा शापं तं तां निशां दुःखितोऽवसम्}
{विवक्षुश्चित्रसेनाय प्रातः सर्वमहृष्टवत्}


\twolineshloka
{ततः प्रभाते विमले गन्धर्वाय यथातथम्}
{निवेदयामास तदा चित्रसेनाय पाण्डवः}


\twolineshloka
{तच्च सर्वं यथावृत्तं शापं चैव यथातथम्}
{अवेदयच्च शक्रस्य चित्रसेनोऽपि सर्वशः}


\twolineshloka
{तदा त्वानाय्य तनयं विविक्ते हरिवाहनः}
{सान्त्वयित्वा शुभैर्वाक्यैः स्मयमानोऽभ्यभाषत}


\twolineshloka
{सुपुत्राद्यपृथा तात त्वया पुत्रेण सत्तम}
{ऋषयोपि हि धैर्येण जिता वै ते महाभुज}


\twolineshloka
{यं च दत्तवती शापमुर्वशी तव मानद}
{स चापि तेऽर्थकृत्तात साधकश्च भविष्यति}


\twolineshloka
{अज्ञातवासो वस्तव्यो भवद्भिर्भूतलेऽनघ}
{वर्पे त्रयोदशे वीर तत्र त्वं गमयिष्यसि}


\twolineshloka
{तेन नर्तकवेपेण अपुंस्त्वेन तथैव च}
{वर्षमेकं विहृत्यैव ततः पुंस्त्वमवाप्स्यसि}


\twolineshloka
{एवमुक्तस्तु शक्रेण फल्गुनः परवीरहा}
{मुदं परमिकां लेभे न च शापं व्यचिन्तयत्}


\twolineshloka
{चित्रसेनेन सहितो गन्धर्वेण यशस्विना}
{रेमे स स्वर्गभवने पाण्डुपुत्रो धनंजयः}


\twolineshloka
{य इमां शृणुयान्नित्यं धृतिं पाण्डुसुतस्य वै}
{न तस्य कामः कामेषु पापकेषु प्रवर्तते}


\twolineshloka
{इदममरवरात्मजस्य घोरंसुचि चरितं विनिशाम्य फल्गुनस्य}
{व्यपगतमददम्भरागदोषा-स्त्रिदिवगताऽभिरमन्ति मानवेन्द्राः}


\chapter{अध्यायः ४५}
\twolineshloka
{`वैशंपायन उवाच}
{}


\twolineshloka
{ततो देवाः सगन्धर्वाः समादायार्घ्यमुत्तमम्}
{शक्रस्य मतमाज्ञाय पार्थमानर्चुर्जसा}


\twolineshloka
{पाद्यमाचमनीयं च प्रतिग्राह्य नृपात्मजम्}
{प्रवेशयामासुरथो पुरन्दरनिवेशनम्}


\twolineshloka
{एवं संपूजितो जिष्णुरुवास भवने पितुः}
{उपशिक्षन्महास्त्राणि ससंहाराणि पाण्डवः}


\twolineshloka
{स शक्रहस्ताद्दयितं वज्रमस्त्रं दुरुत्सहम्}
{अशनिं च महानादां मेघबृंहितलक्षणाम्}


\twolineshloka
{गृहीतास्त्रस्तु कौन्तेयो भ्रातॄन्सस्मार पाण्डवः}
{पुरन्दरनियोगाच्च पञ्चाब्दमवसत्सुखम्}


\twolineshloka
{ततः शक्रोऽब्रवीत्पार्थं कृतास्त्रं काल आगते}
{नृत्तं गीतं च कौन्तेय चित्रसेनादवाप्नुहि}


\twolineshloka
{वादित्रं दैवविहितं नृलके यन्न विद्ते}
{मदाज्ञया च कौन्तेय श्रेयो वै ते भविष्ति}


\twolineshloka
{सखायं प्रददौ चास्य चित्रसेनं पुरन्दरः}
{स तेन सह संगम्य रेमे पार्थो निरामयः ॥'}


\twolineshloka
{कदाचिदटमानस्तु महर्षिरथ लोमशः}
{जगाम शक्रभवनं पुरंदरदिदृक्षया}


\twolineshloka
{स समेत्य नमस्कृत्य देवराजं महामुनिः}
{ददर्शार्धासनगतं पाण्डवं वासवस्य हि}


\twolineshloka
{ततः शक्राभ्यनुज्ञात आसने विष्टरोत्तरे}
{निषसाद द्विजश्रेष्ठः पूज्यमानो महर्षिभिः}


\twolineshloka
{तस्य दृष्ट्वाऽभवद्बुद्धिः पार्थमिन्द्रासने स्थितम्}
{कथं नु क्षत्रियः पार्थः शक्रासनमवाप्तवान्}


\twolineshloka
{किं त्वस्य सुकृतं कर्म के लोका वै विनिर्जिताः}
{स एवमनुसंप्राप्तः स्थानं देवनमस्कृतम्}


\twolineshloka
{तस्य विज्ञाय संकल्पं शक्रो वृत्रविमर्दनः}
{लोमशं प्रहसन्वाक्यमिदमाह शचीपतिः}


\twolineshloka
{देवर्षे श्रूयतां यत्ते मनसैतद्विवक्षितम्}
{नायं केवलमर्त्योऽभूत्क्षत्रियत्वमुपागतः}


\twolineshloka
{महर्षे मम पुत्रोऽयं कुन्त्यां जातो महाभुजः}
{अस्त्रहेतोरिह प्राप्तः कस्माच्चित्कारणान्तरात्}


\twolineshloka
{अहो नैनं भवान्वेत्ति पुराणमृषिसत्तमम्}
{शृणु मे वदतो ब्रह्मन्योऽयं यच्चास्य कारणम्}


\twolineshloka
{नरनारायणौ यौ तौ पुराणावृषिसत्तमौ}
{ताविमावभिजानीहि हृषीकेशधनंजयौ}


\twolineshloka
{विख्यातौ त्रिषु लोकेषु नरनारायणावृषी}
{कार्यार्थमवतीर्णौ तौ पृथ्वीं पुण्यप्रतिश्रयाम्}


\twolineshloka
{यन्न शक्यं शुरैर्द्रष्टुमृषिभिर्वा महात्मभिः}
{तदाश्रमपदं पुण्यं बदरीनाम विश्रुतम्}


\twolineshloka
{स निवासोऽभवद्विप्र विष्णोर्जिष्णोस्तथैव च}
{यतः प्रववृते गङ्गा सिद्धचारणसेविता}


\twolineshloka
{तौ मन्नियोगाद्ब्रह्मर्षे क्षितौ जातौ महाद्युती}
{भूमेर्भारावतरणं महावीर्यौ करिष्यतः}


\twolineshloka
{उद्वृत्ता ह्यसुराः केचिन्निवातकवचा इति}
{विप्रियेषु स्थिताऽस्माकं वरदानेन मोहिताः}


\twolineshloka
{तर्कयन्ते सुरान्हन्तुं बलदर्पसमन्विताः}
{देवान्न गणयन्त्येते तथा दत्तवरा हि ते}


\twolineshloka
{पातालवासिनो रौद्रा दनोः पुत्रा महाबलाः}
{सर्वे देवनिकाया हि नालं योधयितुं हि तान्}


\twolineshloka
{योसौ भूमिगतः श्रीमान्विष्णुर्मुधुनिषूदनः}
{कपिलो नाम देवोसौ भगवानजितो हरिः}


\twolineshloka
{येन पूर्वंमहात्मानः खनमाना रसातलम्}
{दर्शनादेव निहताः सगरस्यात्मजा विभो}


\twolineshloka
{तेन कार्यं महत्कार्यमस्माकं द्विजसत्तम}
{पाथेन च महायुद्धे समेताभ्यामसंशयम्}


\twolineshloka
{सोऽसुरान्दर्शनादेव शक्तो हन्तुं सहानुगान्}
{निवातकवचान्सर्वान्नागानिव महाह्रदे}


\twolineshloka
{किंतु नाल्पेन कार्येण प्रबोध्यो मधुसूदनः}
{तेजसः सुमहाराशिः प्रबुद्धः प्रदहेज्जगत्}


\twolineshloka
{अयं तेषां समस्तानां शक्तः प्रतिसमासने}
{तान्निहत्यरणे शूरः पुनर्यास्यति मानुषान्}


\twolineshloka
{भवानस्मन्नियोगेन यातु तावन्महीतलम्}
{काम्यके द्रक्ष्यसे वीरं निवसन्तं युधिष्ठिरम्}


\twolineshloka
{सवाच्यो मम संदेशाद्धर्मात्मा सत्यसंगरः}
{नोत्कणअठा फल्गुने कार्या कृतास्त्रः शीघ्रमेष्यति}


\twolineshloka
{नाशुद्बाहुवीर्येण नाकृतास्त्रेण वा रणे}
{भीष्मद्रोणादयो युद्धे शक्याः प्रतिसमासितुम्}


\twolineshloka
{गृहीतास्त्रो गुडाकेशो महाबाहुर्महामनाः}
{नृत्तवादित्रगीतानां दिव्यानां पारमीयिवान्}


\twolineshloka
{भवानपि विविक्तानि तीर्थानि मनुजेश्वर}
{भ्रातृभिः सहितः सर्वैर्द्रष्टुमर्हत्यरिंदम}


\twolineshloka
{तीर्थेष्वाप्लुत्य पुण्येषु विपाप्मा विगतज्वरः}
{राज्यं भोक्ष्यसि धर्मेण सुखी विगतकल्मपः}


\twolineshloka
{भवांश्चैनं द्विजश्रेष्ठ पर्यटन्तं महीतलम्}
{त्रातुमर्हति विप्राग्र्य तपोबलसमन्वितः}


\twolineshloka
{गिरिदुर्गेषु च सदा देशेषु विषमेषु च}
{वसन्ति रासा रौद्रास्तेभ्योरक्षां विधास्यति}


\twolineshloka
{एवमुक्ते महेन्द्रेण बीभत्सुरपि लोमशम्}
{उवाच प्रयतो वाक्यं रक्षेथाः पाण्डुनन्दनम्}


\threelineshloka
{[यथा गुप्तस्त्वया राजा चरेत्तीर्थानि सत्तम}
{दानं दद्याद्यथा चैव तथा कुरु महामुने ॥]वैशंपायन उवाच}
{}


\twolineshloka
{तथेति संप्रतिज्ञाय लोमशः सुमहातपाः}
{कामय्कं वनमुद्दिश्य समुपायान्महीतलम्}


\twolineshloka
{ददर्श तत्र कौन्तेयं धर्मराजमरिंदमम्}
{तापसैर्भ्रातृभिश्चैव सर्वतः परिवारितम्}


\chapter{अध्यायः ४६}
\twolineshloka
{जनमेजय उवाच}
{}


\threelineshloka
{अत्यद्भुतमिदं कर्म पार्थस्यामिततेजसः}
{धृतराष्ट्रो महातेजाः श्रुत्वा तत्र किमब्रवीत् ॥वैशंपायन उवाच}
{}


\twolineshloka
{शक्रलोकगतं पार्थं श्रुत्वा राजाऽम्बिकासुतः}
{द्वैपायनादृषिश्रेष्ठात्संजयं वाक्यमब्रवीत्}


\twolineshloka
{श्रुतं मे सूत कार्त्स्न्येन कर्म पार्थस् धीमतः}
{कच्चित्तवापि विदितं याथातथ्येन सारथे}


\twolineshloka
{प्रमत्तो ग्राम्यधर्मेषु मन्दात्मा पापनिश्चयः}
{मम पुत्रः सुदुर्बुद्धिः पृथिवीं घातयिष्यति}


\twolineshloka
{यस्य नित्यमृता वाचः स्वैरेष्वपि महात्मनः}
{त्रैलोक्यमपि तस्य स्याद्योद्धा यस्य धनंजयः}


\twolineshloka
{अस्यतः कर्णिनाराचांस्तीक्ष्णाग्रांश्च शिलाशितान्}
{नार्जुनस्याग्रतस्तिष्ठेदपि मृत्युर्जरातिगः}


\twolineshloka
{मम पुत्रा दुरात्मानः सर्वे मृत्युवशंगताः}
{येषां युद्धं दुराधर्षैः पाण्डवैः समुपस्थितम्}


\twolineshloka
{तस्यैव च न पश्यामि युधि गाण्डीवधन्वनः}
{अनिशं चिन्तयानोऽपि य एनमुदियाद्युधि}


\twolineshloka
{द्रोणकर्णौ प्रतीयातां यदि भीष्मोऽपि वा रणे}
{महान्स्यात्संशयोलोके न तु पश्यामि नो जयम्}


\twolineshloka
{घृणी कर्णः प्रमादी च आचार्यः स्थविरो गुरुः}
{अमर्षी बलवान्पार्थः संरम्भी दृढविक्रमः}


\twolineshloka
{भवेत्सुतुमुलं युद्धं सर्वस्याप्यपराजितम्}
{सर्वे ह्यस्त्रविदः शूराः सर्वे प्राप्ता महद्यशः}


\twolineshloka
{अपि सर्वेश्वरत्वं हि न वाञ्छेरन्पराजिताः}
{वधे नूनं भवेच्छान्तिरेतेषां फल्गुनस्य वा}


\twolineshloka
{न तु हन्ताऽर्जुनस्यास्ति जेता वाऽस्य न विद्यते}
{मन्युस्तस्य कथं शाम्येन्मन्दान्प्रति समुत्थितः}


\twolineshloka
{त्रिदशेशसमो वीरः खाण्डवेऽग्निमतर्पयत्}
{जिगा पार्थिवान्सर्वान्राजसूये महाक्रतौ}


\twolineshloka
{शेषं कुर्याद्गिरेर्वज्रो निपतन्मूर्ध्नि संजय}
{न तु कुर्युः शराः शेषं क्षिप्तास्तात किरीटिना}


\twolineshloka
{यथा हि किरणा भानोस्तपन्तीह चराचरम्}
{तथा पार्थभुजोत्सृष्टाः शरास्तप्स्यन्ति मत्सुतान्}


\twolineshloka
{अपि तद्रथघोषेण भयार्था सव्यसाचिनः}
{प्रतिभाति विदीर्णेव सर्वतो भारती चमूः}


\threelineshloka
{समुद्धरन्प्रवपंश्चैव बाणान्स्ताताऽऽततायी समरे किरीटि}
{सृष्टोऽन्तकः सर्वहरो विधात्राभवेद्यथा तद्वदवारणीयः ॥संजय उवाच}
{}


\twolineshloka
{यदतत्कथितं राजंस्त्वया दुर्योधनं प्रति}
{सर्वमेतद्यथातत्त्वं नतु मिथ्या महीपते}


\twolineshloka
{मन्युना हि समाविष्टाः पाण्डवास्त्वमितौजसः}
{दृष्ट्वा कृष्णां सभां नीतां धर्मपत्नीं यशस्विनीं}


\twolineshloka
{दुःशासनस्य ता वाचः श्रुत्वा वै कटुकोदयाः}
{कर्णस्य च महाराज न स्वप्स्यन्तीति मे मतिः}


\twolineshloka
{श्रुतं हि ते महाराज यथा पार्थेन संयुगे}
{एकादशतनुः स्थाणुर्धेनुषा परितोषितः}


\twolineshloka
{कैरातं वेषमास्थाय योधयामास फल्गुनम्}
{जिज्ञासुः सर्वदेवेशः कपर्दी भगवान्स्वयम्}


\threelineshloka
{`लेभे पाशुपतं चापि परमास्त्रं महाद्युतिः}
{'तत्रैनं लोकपालास्ते दर्शयामासुरर्जुनम्}
{अस्त्रहेतोः पराक्रान्तं तपसा कौरवर्षभम्}


\twolineshloka
{नैतदुत्सहते चान्यो लब्धुमन्यत्र फल्गुनात्}
{साक्षाद्दर्शनमेतेषामीश्वराणां नरो भुवि}


\twolineshloka
{महेश्वरेण यो राजन्न जीर्णो ग्रस्तमूर्तिमान्}
{कस्तमुत्सहते वीरो युद्धे जरयितुं पुमान्}


\twolineshloka
{आसादितमिदं घोरं तुमुलं रोमहर्षणम्}
{द्रौपदीं परिकर्षद्भिः कोपयद्भिश्च पाण्डवान्}


\twolineshloka
{यत्र विस्फुरमाणौष्ठो भीमः प्राह वचोऽर्थवत्}
{दृष्ट्वा दुर्योधनेनोरू द्रौपद्या दर्शितावुभौ}


\twolineshloka
{ऊरुं भेत्स्यामि ते पाप गदया वज्रकल्पया}
{त्रयोदशानां वर्षाणामन्ते दुर्द्यूतदेविनः}


\twolineshloka
{सर्वे प्रहरतां श्रेष्ठाः सर्वेचामिततेजसः}
{सर्वेसर्वास्त्रविद्वांसो देवैरपि सुदुर्जयाः}


\threelineshloka
{मन्ये मन्युसमुद्भूताः पुत्राणां तव संयुगे}
{अन्तं पार्थाः करिष्यन्ति वीर्यामर्षसमन्विताः ॥धृतराष्ट्र उवाच}
{}


\twolineshloka
{किं कृतं सूत कर्णेन वदता परुषं वचः}
{पर्याप्तं वैरमेतावद्यत्कृष्णा सा सभां गता}


\twolineshloka
{अपीदानीं मम सुतास्तिष्ठेरन्मन्दचेतसः}
{येषां भ्राता गुरुर्ज्येष्ठो विनये नावतिष्ठते}


\twolineshloka
{ममापि वचनं सूत न शुश्रूषति मन्दभाक्}
{दृष्ट्वा मां चक्षुषा हीनं निर्विचेष्टमचेतसम्}


\twolineshloka
{ये चास्य सचिवा मन्दाः कर्णसौबलकादयः}
{ते तस्य भूयसो दोषान्वर्धयन्ति विचेतसः}


\twolineshloka
{स्वैरं मुक्ता ह्यपि शराः पार्थेनामिततेजसा}
{निर्दहेयुर्मम सुतान्किंपुनर्मन्युनेरिताः}


\twolineshloka
{पार्थबाहुबलोत्सृष्टा महाचापविनिःसृताः}
{दिव्यास्त्रमन्त्रमुदिताः सादयेयुः सुरानपि}


\twolineshloka
{यस्य मन्त्री च गोप्ता च सुहृच्चैव जनार्दनः}
{हनिस्त्रैलोक्यनाथः स किंनु तस् न निर्जितम्}


\twolineshloka
{इदं हि सुमहच्चित्रमर्जुनस्येह संजय}
{महादेवेन बाहुभ्यां यत्समेत इति श्रुतिः}


\twolineshloka
{प्रत्यक्षं सर्वलोकस्य खाण्डवे यत्कृतं पुरा}
{फल्गुनेन सहायार्थे वह्नेर्दामोदरेण च}


\twolineshloka
{सर्वथा न हि मे पुत्राः सहामात्याः सबान्धवाः}
{क्रुद्धे पार्थे च भीमे च वासुदेवे च सात्वते}


\chapter{अध्यायः ४७}
\twolineshloka
{जनमेजय उवाच}
{}


\twolineshloka
{यदिदं शोचितं राज्ञा धृतराष्ट्रेण वै मुने}
{प्रव्राज्यपाण्डवान्वीरान्सर्वमेतन्निर्रथकम्}


\twolineshloka
{कथं च राजपुत्रं तमुपेक्षेताल्पचेतसम्}
{दुर्योधनं पाण्डुपुत्रान्कोपयानं महारथान्}


\threelineshloka
{किमासीत्पाण्डुपुत्राणां वने भोजनमुच्यताम्}
{वन्यं वाऽप्यवा कृष्टमेतदाख्यातु नो भवान् ॥वैंशंपायन उवाच}
{}


\twolineshloka
{वानेयं च मृगांश्चैव शुद्धैर्बाणैर्निपातितान्}
{ब्राह्मणानां निवेद्याग्रमभुञ्जता महारथाः}


\twolineshloka
{तांस्तु शूरान्महेष्वासांस्तदा निवसतो वने}
{अन्वयुर्ब्राह्मणा राजन्साग्नयोऽनग्नयस्तथा}


\twolineshloka
{ब्राह्मणानां सहस्राणि स्नातकानां महात्मनाम्}
{दश मोक्षविदां तद्वद्यान्बिभर्ति युधिष्ठिरः}


\twolineshloka
{रुरून्कृष्णमृगांश्चैव मेध्यांश्चान्यान्मनोरमान्}
{बाणैरुन्मथ्य विविधैर्ब्राह्मणेभ्यो न्यवेदयत्}


\twolineshloka
{न तत्र कश्चिद्दुर्वर्णो व्याधितो वाऽपि दृश्यते}
{कृशो वा दुर्बलो वाऽपि दीनो भीतोपि वा पुनाः}


\threelineshloka
{पुत्रानिव प्रियान्भ्रातॄन्ज्ञातीनिव सहोदरान्}
{पुरोष कौरवश्रेष्ठो धर्मेराजो युधिष्ठिरः}
{}


\twolineshloka
{पतीश्च द्रौपदी सर्वाञ्द्विजातीश्च यशस्विनी}
{मातेव भोजयित्वाऽग्रे शिष्टमाहारयत्तदा}


\twolineshloka
{प्राचीं राजा दक्षिणां भीमसेनोयमौ प्रतीचीमथवाऽप्युदीचीम्}
{धनुर्धरा मांसहेतोर्मृगाणांक्षयं चक्रुर्नित्यमेवोपगम्य}


\twolineshloka
{तथा तेषां कवसतां काम्यके वैविहीनानामर्जुनेनोत्सुकानाम्}
{पञ्चैव वर्षाणि तथा व्यतीयु-रथीयतां जपतां जुह्वतां च}


\chapter{अध्यायः ४८}
\twolineshloka
{वैशंपायन उवाच}
{}


\twolineshloka
{तेषां तच्चरितं श्रुत्वा मनुष्यातीतमद्भुतम्}
{चिन्ताशोकपरीतात्मा मन्युनाभिपरिप्लुतः}


\twolineshloka
{दीर्घमुष्णं च निःश्वस्य धृतराष्ट्रोऽम्बिकासुतः}
{अब्रवीत्संजयं सूतमामन्त्र्य भरतर्षभ}


\twolineshloka
{न रात्रौ न दिवा सूत शान्तिं प्राप्नोमि वैक्षणम्}
{संचिन्त्य दुर्णयं घोरमतीतं द्यूतजं हि तत्}


\twolineshloka
{तेषामसह्यवीर्याणां शौर्यं धैर्यं धृतिं पराम्}
{अन्योन्यमनुरागं च भ्रातॄणामतिमानुषम्}


\twolineshloka
{देवपुत्रौ महाभागौ देवराजसमद्युती}
{नकुलः सहदेवश्च पाण्डवौ युद्धदुर्मदौ}


\twolineshloka
{दृढायुधौ दूरपातौ युद्धे च कृतनिश्चयौ}
{शीघ्रहस्तौ दृढक्रोधौ नित्ययुक्तौ रथे स्थितौ}


\twolineshloka
{भीमार्जुनौ पुरोधाय यदा तौ रणमूर्धनि}
{स्थास्येते सिंहविक्रान्तावश्विनाविव दुःसहौ}


% Check verse!
निःशेषमिह पश्यामि मम सैन्यस्य संजय
\twolineshloka
{तौ ह्यप्रतिरथौ युद्धे देवपुत्रौ महारथौ}
{द्रौपद्यास्तं परिक्लेशं न क्षंस्येतेऽत्यमर्षिणौ}


\threelineshloka
{वृष्णयोऽथ महेष्वासाः पाञ्चाला वा महौजसः}
{युधि सत्याभिसन्धेन वासुदेवेन रक्षिताः}
{प्रधक्ष्यन्ति रणे पार्थाः पुत्राणां मम वाहिनीम्}


\twolineshloka
{रामकृष्णप्रणीतानां वृष्णीनां सूतनन्दन}
{न शक्यः सहितुं वेगः पर्वतैरपि दुःसहः}


\twolineshloka
{तेषां मध्ये महेष्वासो भीमो भीमपराक्रमः}
{शैक्यया वीरघातिन्या गदया विचरिष्यति}


\twolineshloka
{तथा गाण्डीवनिर्घोषं विस्फूर्जितमिवाशनेः}
{गदावेगं च भीमस्य नालं सोढुं नराधिपाः}


\threelineshloka
{ततोऽहं सुहृदां वाचो दुर्योधनवशानुगः}
{स्मरणीयाः स्मरिष्यामि मया या न कृताः पुरा ॥संजय उवाच}
{}


\twolineshloka
{व्यतिक्रमोऽयं सुमहांस्त्वया राजन्नुपेक्षितः}
{समर्थेनापि यन्मोहात्पुत्रस्ते न निवारितः}


\twolineshloka
{श्रुत्वाऽयं निर्जितान्द्यूते पाण्डवान्मधुसूदनः}
{त्वरितः कामय्के पार्थान्समभावयदच्युतः}


\twolineshloka
{द्रुपदस्य तथा पुत्रा धृष्टद्युम्नपुरोगमाः}
{विराटो धृष्टकेतुश्च केकयाश्च महारथाः}


\twolineshloka
{तैश्च यत्कथितं राजन्दृष्ट्वा पार्थान्पराजितान्}
{चारेण विदितं सर्वं तन्मया वेदितं च ते}


\twolineshloka
{समागम्य वृतस्तत्र पाण्डवैर्मधुसूदनः}
{सारथ्ये फल्गुनस्याजौ तथेत्याह च तान्हरिः}


\twolineshloka
{अमर्षितो हि कृष्णोपि दृष्ट्वा पार्थांस्तदा गतान्}
{कृष्णाजिनोत्तरासङ्गानब्रवीच्च युधिष्ठिरम्}


\twolineshloka
{या सा समृद्धिः पार्थानामिन्द्रप्रस्थे बभूव ह}
{राजसूये मया दृष्टा नृपैरन्यैः सुदुर्लभा}


\twolineshloka
{यत्रसर्वान्महीपालाञ्शस्त्रतेजोभयार्दितान्}
{सवङ्गाङ्गान्सपौण्ड्रौढ्रान्सचोलद्रविडान्ध्रकान्}


\twolineshloka
{सागरानूपकांश्चैव ये च पत्तनवासिनः}
{सिंहलान्बर्बरान्म्लेच्छान्ये च लङ्कानिवासिनः}


\twolineshloka
{पश्चिमानि चराष्ट्राणि शतशः सागरान्तिकान्}
{पह्लवान्दरदान्सर्वान्किरातान्यवनाञ्शकान्}


\twolineshloka
{हारहूणांश्च चीनांश्च तुषारान्सैन्धवांस्तथा}
{जागुडान्रामठान्मुण्डान्स्त्रीराज्यमथ तङ्गणान्}


\twolineshloka
{केकयान्मालवांश्चैव तथा काश्मीरकानपि}
{अद्राक्षमहमाहूतान्यज्ञे ते परिवेषकान्}


\twolineshloka
{सा ते समृद्धिर्यैरात्ता चपला प्रतिसारिणी}
{आदाय जीवितं तेषामाहरिष्यामि तामहम्}


\twolineshloka
{रामेण सह कौरव्य भीमार्जुनवयैस्तथा}
{अक्रूरगदसाम्बैश्च प्रद्युम्नेनाहुकेन च}


\threelineshloka
{धृष्टद्युम्नेन वीरेण शिशुपालात्मजेन च}
{दुर्योधनं रणे हत्वा सद्यः कर्णं च भारत}
{दुःशासनं सौबलेयं यश्चान्यः प्रतियोत्स्यति}


\twolineshloka
{ततस्त्वं हास्तिनपुरे भ्रातृभिः सहितो वसन्}
{धार्तराष्ट्रीं श्रियं प्राप्य प्रशाधि पृथिवीमिमां}


\twolineshloka
{अथैनमब्रवीद्राजा तस्मिन्वीरसमागमे}
{शृण्वत्स्वेतेषु वीरेषु धृष्टद्युम्नमुखेषु च}


\twolineshloka
{प्रतिगृह्णामि ते वाचमिमां सत्यां जनार्दन}
{अमित्रान्मे महाबाहो सानुबन्धान्हनिष्यसि}


\twolineshloka
{वर्षात्रयोदशादूर्ध्वं सत्यं मां कुरु केशव}
{प्रतिज्ञातो वने वासो राज्ञांमध्ये मया ह्ययम्}


\twolineshloka
{धर्मराजस्य वचनं प्रतिश्रुत्य सभासदः}
{धृष्टद्युम्नपुरोगास्ते समयामासुऱञ्जसा}


\twolineshloka
{केशवं मधुरैर्वाक्यैः कालयुक्तैरमर्षितम्}
{पाञ्चालीं प्राहुरक्लिष्टां वासुदेवस्य शृण्वतः}


\twolineshloka
{दुर्योधनस्तव क्रोधाद्देवि त्यक्ष्यति जीवितम्}
{प्रतिजानीम ते सत्यं मा शुचो वरवर्णिनि}


\twolineshloka
{ये स्म ते कुरवः कृष्णे दृष्ट्वा त्वां प्राहसंस्तदा}
{मांसानि तेषां खादन्तो हरिष्यन्ति वृकद्विजाः}


\twolineshloka
{पास्यन्ति रुधिरं तेषां गृध्रा गोमायवस्तथा}
{उत्तमाङ्गानि कर्षन्तो यैः कृष्टाऽसि सभातले}


\twolineshloka
{तेषां द्रक्ष्यसि पाञ्चालि गात्राणि पृथिवीतले}
{क्रव्यादैः कृष्यमाणानि भक्ष्यमाणानि चासकृत्}


\twolineshloka
{परिक्लिष्टाऽसियैस्तत्रयैश्चासि समुपेक्षिता}
{तेषामुत्कृत्तशिरसां भूमिः पास्यति शोणितम्}


\twolineshloka
{एवं बहुविधा वाचस्त ऊचुः पुरुषर्षभाः}
{सर्वेतेजस्विनः शूराः सर्वेचाहतलक्षणाः}


\twolineshloka
{ते धर्मराजेन वृतावर्षादूर्ध्वं त्रयोदशात्}
{पुरस्कृत्योपयास्यन्ति वासुदेवं महारथाः}


\twolineshloka
{रामश्च कृष्णश्च धनंजयश्चप्रद्युम्नसाम्बौ युयुधानभीमौ}
{माद्रीसुतौ केकयराजपुत्राःपाञ्चालपुत्राः सह मत्स्यराज्ञा}


\threelineshloka
{एतान्सर्वाल्लोँकवीरानजेया-न्महात्मनः सानुबन्धान्ससैन्यान्}
{को जीवितार्थी समरेऽभ्युदीया-त्क्रुद्धान्सिंहान्केसरिणो यथैव ॥धृतराष्ट्र उवाच}
{}


\twolineshloka
{यन्माऽब्रवीद्विदुरो द्यूतकालेत्वं पाण्डवाञ्जेषय्सि चेन्नरेन्द्र}
{ध्रुवं कुरूणामयमन्तकालोमहाभयो भविता शोणितौधः}


\twolineshloka
{मन्ये यथा तद्भवितेति सूतयथा क्षत्ता प्राह वचः पुरा माम्}
{असंशयं भविता युद्धमेत-द्गते काले पाण्डवानां यथोक्तम्}


\chapter{अध्यायः ४९}
\twolineshloka
{जनमेजय उवाच}
{}


\threelineshloka
{अस्त्रहेतोर्गते पार्थे शक्रलोकं महात्मनि}
{युधिष्ठिरप्रभृतयः किमकुर्वत पाण्डवाः ॥वैशंपायन उवाच}
{}


\twolineshloka
{अस्त्रहेतोर्गते पार्थे शक्रलोकं महात्मनि}
{न्यवसन्कृष्णया सार्धं काम्यके भरतर्षभाः}


\twolineshloka
{ततः कदाचिदेकान्ते विविक्ते मृदुशाद्वले}
{दुःखार्ता भरतश्रेष्ठा निषेदुः सह कृष्णया}


\twolineshloka
{धनंजयं शोचमानाः साश्रुकण्ठाः सुदुःखिताः}
{तद्वियोगार्दितान्सर्वाञ्शोकः समभिपुप्लुवे}


\twolineshloka
{धनंजयवियोगाच्च राज्यभ्रंशाच्च दुःखिताः}
{अथ भीमो महाबाहुर्युधिष्ठिरमभाषत}


\twolineshloka
{निदेशात्ते महाराज गतोऽसौ भरतर्षभः}
{अर्जुनः पाण्डुपुत्राणां यस्मिन्प्राणाः प्रतिष्ठिताः}


\twolineshloka
{यस्मिन्विनष्टे पाञ्चालाः सह पुत्रैस्तथा वयम्}
{सात्यकिर्वासुदेवश्च विनश्येयुर्न संशयाः}


\twolineshloka
{योसौ गच्छति धर्मात्मा बहून्क्लेशान्विचिन्तयन्}
{भवन्नियोगाद्बीभत्सुस्ततो दुःखतरं नु किम्}


\twolineshloka
{यस्य बाहू समाश्रित्य वयं सर्वे महात्मनः}
{मन्यामहे जितानाजौ परान्प्राप्तां च मेदिनीम्}


\twolineshloka
{यस्य प्रभावाद्धि वयं सभामध्ये धनुष्मतः}
{`जितान्मन्यामहे सर्वान्धार्तराष्ट्रात्ससौबलान्}


\twolineshloka
{यस्य प्रभावान्न मया सभामध्ये महाबलाः'}
{नीता लोकममुं सर्वे धार्तराष्ट्राः ससौबलाः}


\twolineshloka
{तेवयं बाहुबलिनः क्रोधमुत्थितमात्मनः}
{सहामहे भवन्मूलं वासुदेवेन पालिताः}


\twolineshloka
{मया हि सह कृष्णेन हत्वा कर्णमुखान्परान्}
{स्वबाहुविजितां कृत्स्नां प्रशाधीनां वसुंधराम्}


\twolineshloka
{भतो द्यूतदोषेण सर्वेवयमुपप्लुताः}
{अहीनपौरुषा राजन्बलिभिर्बलवत्तराः}


\twolineshloka
{क्षात्रं धर्मं महाराज त्वमवेक्षितुमर्हसि}
{ग हिधर्मो महाराज त्रियस्य वनाश्रयः}


\twolineshloka
{राज्यमेव परंधर्मं क्षत्रियस्य विदुर्बुधाः}
{स क्षत्रधर्मविद्राजन्माधर्म्यान्नीनशः पथः}


\twolineshloka
{प्राग्द्वादश समा राजन्धार्तराष्ट्रान्निहन्महि}
{निवर्त्य च वनात्पार्थमानाय्य च जनार्दनम्}


\twolineshloka
{व्यूढानीकान्महाराज जवेनैव महाहवे}
{धार्तराष्ट्रानमुं लोकं गमयाम विशांपते}


\twolineshloka
{सर्वानहं हनिष्यामि धार्तराष्ट्रान्ससौबलान्}
{दुर्योधनं च कर्णं च यो वाऽन्यः प्रतियोत्स्यते}


\twolineshloka
{मया प्रशमिते पश्चात्त्वमेष्यसि वनं पुनः}
{एवं कृते न ते दोषो भविष्यति विशांपते}


\twolineshloka
{यज्ञैश्च विविधैस्तात कृतंपापमरिंदम}
{अवधूय महाराजगच्छेम स्वर्गमुत्तमम्}


\twolineshloka
{एवमेतद्भवेद्राजन्यदि राजा न बालिशः}
{अस्माकं दीर्घसूत्रः स्याद्भवान्धर्मपरायणः}


\threelineshloka
{निकृत्या निकृतिप्रत्रा हन्तव्या इति निश्चयः}
{न हि नैकृतिकं हत्वा निकृत्या पापमुच्यते}
{तथा भारत धर्मेषु धर्मज्ञैरिह दृश्यते}


\threelineshloka
{अहोरात्रं महाराज तुल्यं संवत्सरेण ह}
{तथैव वेदवचनं श्रूयते नित्यदा विभो}
{संवत्सरो महाराज पूर्णो भवति कृच्छ्रतः}


\twolineshloka
{यदि वेदाः प्रमाणं ते दिवसादूर्ध्वमच्युत}
{तयोदशादितः कालो ज्ञायतां परिनिष्ठितः}


\twolineshloka
{कालो दुर्योधनं हन्तुं सानुबन्धमरिंदम}
{एकाग्रां पृथिवीं सर्वां पुरा राजन्करोति सः}


\twolineshloka
{द्यूतप्रियेण राजेन्द्र तथा तद्भवता कृतम्}
{प्रायेणाज्ञातच्रयायां वयं सर्वेनिपातिताः}


\twolineshloka
{न तं देशं प्रपश्यामि यत्र सोऽस्मान्सुदुर्जनः}
{न विज्ञास्यति दुष्टात्मा चारैरिति सुयोधनः}


\threelineshloka
{अधिगम्य च सर्वान्नो वनवासमिमं ततः}
{प्रव्राजयिष्यतिपुनर्निकृत्याऽधमपूरुषः}
{यद्यस्मानभिगच्छेत पापः स हि कथंचन}


\twolineshloka
{अज्ञातचर्यामुत्तीर्णान्दृष्ट्वा च पुनराह्वयेत्}
{द्यूतेन ते महाराज पुनर्युद्धं प्रवर्तते}


\twolineshloka
{भवांश्च पुनराहूतो द्यूतेनैवापनेष्यति}
{स तथाऽक्षेषु कुशलो निश्चितो गतचेतनः}


% Check verse!
चरिष्यसि महाराज वनेषु वसतीः पुनः
\threelineshloka
{यद्यस्मान्सुमहाराज कृपणान्कर्तुमर्हसि}
{यावज्जीवमवेक्षस्व वेदधर्मांश्च कृत्स्नशः}
{निकृत्या निकृतिप्रज्ञो हन्तव्य इति निश्चयः}


\fourlineindentedshloka
{अनुज्ञातस्त्वया गत्वायावच्छक्ति सुयोधनम्}
{यथैव कक्षमुत्सृष्टो दहेदनिलसारथिः}
{हनिष्यामि तथा मन्दमनुजानातु मे भवान् ॥वैशंपायन उवाच}
{}


\twolineshloka
{एवं ब्रुवाणं भीमं तु धर्मराजो युधिष्ठिरः}
{उवाच सान्त्वयन्राजा मूर्न्ध्युपाघ्राय पाण्डवम्}


\twolineshloka
{असशयं महाबाहो हनिष्यसि सुयोधनम्}
{वर्षात्रयोदशादूर्ध्वं सह गाण्डीवधन्वना}


\twolineshloka
{यत्त्वमाभाषसे पार्थ प्राप्तः काल इति प्रभो}
{अनृतं नत्सहे वक्तुं न ह्येतन्मयि विद्यते}


\twolineshloka
{अन्तरेणापि कौन्तेय निकृतिंपापनिश्चयम्}
{हन्ता त्वमसि दुर्धर्ष सानुबन्धं सुयोधनम्}


\twolineshloka
{एवं ब्रुवति भीमं तु धर्मराजे युधिष्ठिरे}
{आजगाम महाभागो बृहदश्वो महानृषिः}


\twolineshloka
{तमभिप्रेक्ष्यधर्मात्मा संप्राप्तं धर्मचारिणम्}
{शास्त्रवन्मधुपर्केण पूजयामास धर्मराट्}


\twolineshloka
{आश्वस्तं चैनमासीनमुपासनो युधिष्ठिरः}
{अभिप्रेक्ष्य महाबाहुः कृपणं बह्वभाषत}


\twolineshloka
{अक्षद्यूते च भगवन्धनं राज्यं च मे हृतम्}
{आहूय निकृतिप्रज्ञैः कितवैरक्षकोविदैः}


\twolineshloka
{अनक्षज्ञस्य हि सतो निकृत्या पापनिश्चयैः}
{भार्या च मे सभां नीता प्राणेभ्योपि गरीयसी}


\twolineshloka
{पुनर्द्यूतेन मां जित्वा वनवासां सुदारुणम्}
{प्राव्राजयन्मंहारण्यमजिनैः परिवारितम्}


\twolineshloka
{अहं वने दुर्वसतीर्वसन्परमदुःखितः}
{अक्षद्यूताभिषङ्गेण गिरः शृण्वन्सुदारुणाः}


\twolineshloka
{आर्तानां सुहृदां वाचो द्यूतप्रभृति शंसताम्}
{अहं हृदि श्रिताः स्मृत्वा सर्वरात्रीर्विचिन्तयन्}


\twolineshloka
{यस्मिंश्च वयमायत्ताः सदा गाण्डीवधन्वनि}
{`स चेन्द्रलोकं गतवानस्त्रहेतोर्महाबलः ॥'}


\twolineshloka
{विना महात्मना तेन गतसत्व इवास्महे}
{कदा द्रक्ष्यामि बीभत्सुं कृतास्त्रं पुनरागतम्}


\twolineshloka
{`इति सर्वे महेष्वासं चिन्तयाना धनंजयम्}
{अनेन तु विषण्णोऽहं कारणेन सहानुजः}


\threelineshloka
{वनवासान्निवृत्तं मां पुनस्ते पापबुद्धयः}
{जानन्तः प्रीयमाणा वै देवने भ्रातृभिः सह}
{द्यूतेनैवाह्वयिष्यन्ति बलादक्षेषु तद्विदः}


\twolineshloka
{आहूतश्च पुनर्द्यूते नास्मि शक्तो निवर्तितुम्}
{पणे च मम नात्यर्थं वसु किंचन विद्यते}


\twolineshloka
{एतत्सर्वमनुध्यायंश्चिन्तयानो दिवानिशम्}
{न मत्तो दुःखिततरः पुमानस्तीह कश्चन'}


\threelineshloka
{नास्ति राजा मया कश्चिदल्पभाग्यतरो भुवि}
{भवता दृष्टपूर्वो वा श्रुतपूर्वोपि वा क्वचित्}
{न मत्तो दुःखिततरः पुमानस्तीति मे मतिः}


\threelineshloka
{`एवं ब्रुवन्तं दुःखार्तमुवाच भगवानृषिः}
{शोकं व्यपनुदन्राज्ञो धर्मराजस्य धीमतः ॥बृहदश्व उवाच}
{}


\twolineshloka
{न विषादे मनः त्वया बुद्धिमतांवर}
{आगमिष्यति बीभत्सुरमित्रांश्च विजेष्यते'}


\twolineshloka
{यद्ब्रवीषि महाराज न मत्तो विद्ते क्वचित्}
{अल्पभाग्यतरः कश्चित्पुमानस्तीति पाण्डव}


\threelineshloka
{अत्र ते वर्णयिष्यामि यदि शुश्रूषसेऽनघ}
{यस्त्वत्तो दुःखिततरो राजाऽऽसीत्पृथिवीपते ॥वैशंपायन उवाच}
{}


\threelineshloka
{अथैनमब्रवीद्राजा ब्रवीतु भगवानिति}
{इमामवस्थां संप्राप्तं श्रोतुमिच्छामि पार्थिवम् ॥बृहदश्व उवाच}
{}


\twolineshloka
{शृणु राजन्नवहितः सह भ्रातृभिरच्युत}
{यस्त्वत्तो दुःखिततरो राजाऽऽसीत्पृथिवोपते}


\twolineshloka
{निषधेषु महीपालो वीरसेन इति श्रुतः}
{तस्य पुत्रोऽभवन्नाम्ना नलो धर्मार्तकोविदः}


\twolineshloka
{स निकृत्याजतो राजा पुष्करेणेति नः श्रुतम्}
{वनवासं सुदुःखार्तो भार्यया न्यवसत्सह}


\twolineshloka
{न तस्य दासा न रथो न भ्राता न च भान्धवाः}
{वने निवसतो राजन्नश्रूयन्त कदाचन}


\threelineshloka
{भवान्हि संवृतो वीरैर्भ्रातृभिर्देवसंमितैः}
{ब्रह्मकल्पैर्द्विजाग्र्यैश्च तस्मान्नार्हसि शोचितुम् ॥युधिष्ठिर उवाच}
{}


\twolineshloka
{विस्तरेणाहमिच्छामि नलस्य सुमहात्मनः}
{चरितं वदतांश्रेष्ठ तन्ममाख्यातुमर्हसि}


\chapter{अध्यायः ५०}
\twolineshloka
{बृहदश्व उवाच}
{}


\twolineshloka
{आसीद्राजा नलो नाम वीरसेनसुतो बली}
{उपपन्नो गुणैरिष्टै रूपवानश्वकोविदः}


\threelineshloka
{`यज्वा दानपतिर्दक्षः सदा शीलपुरस्कृतः'}
{अतिष्ठन्मनुजेन्द्राणां मूर्ध्नि देवपतिर्यथा}
{उपर्युपरि सर्वेषामादित्य इव तेजसा}


\twolineshloka
{ब्रह्मण्यो वेदविच्छ्ररो निषधेषु महीपतिः}
{अक्षप्रियः सत्यवादी महानक्षौहिणीपतिः}


\twolineshloka
{ईप्सितो वरनारीणामुदारः संयतेन्द्रियः}
{रक्षिताधन्विनां श्रेष्ठः साक्षादिव मनुः स्वयम्}


\twolineshloka
{तथैवासीद्विदर्भेषु भीमो भीमपराक्रमः}
{शूरः सर्वगुणैर्युक्तः प्रजाकामः स चाप्रजाः}


\twolineshloka
{स प्रजार्थे परं यत्नमकरोत्सुसमाहितः}
{तमभ्यगच्छद्ब्रह्मर्षिर्दमनो नाम भारत}


\twolineshloka
{तं स भीमः प्रजाकामस्तोषयामास धर्मवित्}
{महिष्या सह राजेनद््र सत्कारेण सुवर्चसम्}


\twolineshloka
{तस्मै प्रसन्नो दमनः सभार्याय वरं ददौ}
{कन्यारत्नं कुमारांश्च त्रीनुदारान्महायशाः}


\twolineshloka
{दमयन्तीं दमं दान्तं दमनं च सुवर्चसम्}
{उपपन्नान्गुणैः सर्वैर्भीमान्भीमपराक्रमान्}


\twolineshloka
{दमयन्ती तु रूपेण तेजसा वपुषा श्रिया}
{सौभाग्येन च लोकेषु यशः प्राप सुमध्यमा}


\twolineshloka
{अथ तां वसि प्राप्ते दासीनां समलंकृतम्}
{शतं सखीनां च तदा पर्युपास्ते शचीमिव}


\twolineshloka
{तत्र स्म राजते भैमी सर्वाभरणभूषिता}
{सखीमध्येऽनवद्याङ्गी विद्युत्सौदामनी यथा}


\twolineshloka
{अतीव रूपसंपन्ना श्रीरिवायतलोचना}
{न देवेषु न यक्षेषु तादृग्रूपवती क्वचित्}


\twolineshloka
{मानुषेष्वपि चान्येषु दृष्टपूर्वाऽथवा श्रुता}
{चित्तप्रमाथिनी बाला देवानामपि सुन्दरी}


\twolineshloka
{नलश्च नरशार्दूलो रूपेणाप्रतिमो भुवि}
{कंदर्प इव रूपेण मूर्तिमानभवत्स्वयम्}


\twolineshloka
{तस्याः समीपे तु नलं प्रशशंसुः कुतूहलात्}
{नैषधस्य समीपे तु दमयन्तीं पुनः पुनः}


\twolineshloka
{तयोरदृष्टः कामोऽभूच्छृण्वतोः सततं गुणान्}
{अन्योन्यं प्रतिकौन्तेय स व्वर्धत हृच्छयः}


\twolineshloka
{अशक्नवन्नलः कामं तदा धारयितुं हृदा}
{अन्तःपुरसमीपस्थे वन आस्ते रहोगतः}


\twolineshloka
{स ददर्शै ततो हंसाञ्जातरूपपरिच्दान्}
{वने विचरतां तेषामेकं जग्राह पक्षिणम्}


\threelineshloka
{ततो}
{डन्तरिक्षगो वाचं व्याजहार नलं तदा}
{हन्तव्योस्मि न तेराजन्करिष्यामि तवप्रियम्}


\twolineshloka
{दमयन्तीसकाशे त्वां कथयिष्यामि नैषध}
{यथा त्वदन्यं पुरुषं न सा मंस्यति कर्हिचित्}


\twolineshloka
{`तव चैव यथा भार्या भविष्यति तथाऽनघ}
{विधास्यामि नरव्याघ्र सोऽनुजानातु मां भवान् ॥'}


\twolineshloka
{एवमुक्तस्ततो हंसमुत्ससर्ज महीपतिः}
{ते तु हंसाः समुत्पत्य विदर्भानगमंस्ततः}


\twolineshloka
{विदर्भनगरीं गत्वा दमयन्त्यास्तदान्तिके}
{निपेतुस्ते गरुत्मन्तः सा ददर्शाथ तान्खगान्}


\twolineshloka
{सा तानद्भुतरूपान्वै दृष्ट्वा सखिगणावृता}
{हृष्टा ग्रहीतुं खगमांस्त्वरमाणोपचक्रमे}


\twolineshloka
{अथ हंसा विससृपुः सर्वतः प्रमदावने}
{एकैकशस्तदा कन्यास्तान्हंसान्समुपाद्रवन्}


\twolineshloka
{दमयन्ती तु यं हंसं समुपाधावदन्तिके}
{स मानुषीं गिरं कृत्वा दमयन्तीमथाब्रवीत्}


\twolineshloka
{दमयन्ति नलो नाम निषधेषु महीपतिः}
{अश्विनोः सदृशो रूपे न समास्तस्य मानुषाः}


\threelineshloka
{[कन्दर्प इव रूपेण मूर्तिमानभवत्स्वयम्}
{]तस्य वै यदिभार्या त्वं भवेथा वरवर्णिनि}
{सफलंते भवेज्जन्म रूपं चेदं सुमध्यमे}


\twolineshloka
{वयं हि देवगन्धर्वमनुष्योरगराक्षसान्}
{दृष्टवन्तो न चास्माभिर्दृष्टपूर्वस्तथाविधः}


\twolineshloka
{त्वं चापि रत्नं नारीणां नरेषु च नलो वरः}
{विशिष्टाया विशिष्टेन संगमो गुणवान्भवेत्}


\twolineshloka
{एवमुक्ता तु हंसेन दमयन्ती विशांपते}
{अब्रवीत्तत्र तं हंसं त्वमप्येवं नलं वद}


\twolineshloka
{तथेत्युक्त्वाऽण्डजः कन्यां विदर्भस्य विशांपते}
{पुनरागम्य निषधान्नले सर्वं न्यवेदयत्}


\chapter{अध्यायः ५१}
\twolineshloka
{बृहदश्व उवाच}
{}


\twolineshloka
{दमयन्ती तु तच्छ्रुत्वा वचो हंसस्य भारत}
{तदा प्रभृति न स्वस्था नलं प्रति बभूव सा}


\twolineshloka
{ततश्चिन्तापरा दीना विवर्णवदना कृशा}
{बभूवदमयन्ती तु निःश्वासपरमा तदा}


\twolineshloka
{ऊर्ध्वदृष्टिर्ध्यानपरा बभूवोन्मत्तदर्शना}
{पाण्डुवर्णा क्षणेनाथ हृच्छयाविष्टचेतना}


\twolineshloka
{न शय्यासनभोगेषु रतिं विन्दति कर्हिचित्}
{न नक्तं न दिवा शेते हाहेति रुदती मुहुः}


% Check verse!
तामस्वस्थां तदाकारां सख्यस्ता जज्ञिरिङ्गितैः
\twolineshloka
{ततो विदर्भपतये दमयन्त्याः सखीजनः}
{न्यवेदयत्तामस्वस्थां दमयन्तीं नरेश्वरः}


\twolineshloka
{तच्छ्रुत्वा नृपतिर्भीमो दमयन्तीसखीगणात्}
{किमर्थं दुहिता मेऽद्यनातिस्वस्थेव लक्ष्यते}


\twolineshloka
{स समीक्ष्य महीपालः स्वां सुतां प्राप्तयौवनाम्}
{अपश्यदात्मना कार्यं दमयन्त्याः स्वयंवरम्}


\twolineshloka
{स सन्निपातयामास महीपालान्विशांपतिः}
{एषोऽनुभूयतां चीराः स्वयंवर इति प्रभो}


\twolineshloka
{श्रुत्वा तु पार्थिवाः सर्वे दमयन्त्याः स्वयंवरम्}
{अभिजग्मुस्ततो वीरा राजानो भीमशासनात्}


\twolineshloka
{हस्त्यश्वरथघोषेण नादयन्तो वसुंधराम्}
{विचित्रमाल्याभरणैर्बलैर्दृश्यैः स्वलंकृतैः}


\twolineshloka
{तेषां भीमो महाबाहुः पार्थिवानां महात्मनाम्}
{यथार्हमकरोत्पूजां तेऽवसंस्तत्र पूजिताः}


\twolineshloka
{एतस्मिन्नेव काले तु सुराणामृषिसत्तमौ}
{अटमानौ महात्मानाविन्द्रलोकमितो गतौ}


\twolineshloka
{नारदः पर्वतश्चैव महाप्राज्ञौ महाव्रतौ}
{देवराजस् भवनं विविशाते सुपूजितौ}


\threelineshloka
{तावर्चयित्वा मघवा ततः कुशलमव्ययम्}
{पप्रच्छानामयं चापि तयोः सर्वगतं विभुः ॥नारद उवाच}
{}


\threelineshloka
{आवयोः कशलं देव सर्वत्रगतमीश्वर}
{लोके च मघवन्कृत्स्ने नृपाः कुशलिनो विभो ॥बृहदश्व उवाच}
{}


\twolineshloka
{नारदस्य वचः श्रुत्वा पप्रच्छ बलवृत्रहा}
{धर्मज्ञाः पृथिवीपालास्त्यक्तजीवितयोधिनः}


\twolineshloka
{शस्त्रेण निधनं काले ये गच्छन्त्यपराड्युखाः}
{अयं लोकोऽक्षयस्तेषां यथैव मम कामधुक्}


\twolineshloka
{क्वनु ते क्षत्रियाः शूरा नहि पश्यामि तानहम्}
{आगच्छतो महीपालान्दयितामनिथीन्मम}


\twolineshloka
{एवमुक्तस्तु शक्रेण नारदः प्रत्यभाषत}
{शृणु मे मघवन्येन न दृश्यन्ते महीक्षितः}


\twolineshloka
{विदर्भराज्ञो दुहिता दमयन्तीति विश्रुता}
{रूपेण समतिक्रान्ता पृथिव्यां सर्वयोषितः}


\twolineshloka
{तस्याः स्वयंवरः शक्र भविता नचिरादिव}
{तत्र गच्छन्ति राजानो राजपुत्राश्च सर्वशः}


\twolineshloka
{तां रत्नभूतां लोकस्य प्रार्थयन्तो महीक्षितः}
{काङ्क्षन्ति स्म विशेषेण बलवृत्रनिषूदन}


\twolineshloka
{एतस्मिन्कथ्यमाने तु लोकपालाश्च साग्निकाः}
{आजग्मुर्देवराजस्य समीपममरोत्तमाः}


\twolineshloka
{ततस्ते शुश्रुवुः सर्वे नारदस्य वचो महत्}
{श्रुत्वैव चाब्रुवन्हृष्टा गच्छामो वयमप्युत}


\twolineshloka
{ततः सर्वे महाराज सगणाः सहवाहनाः}
{विदर्भानभिजग्मुस्ते यतः सर्वे महीक्षितः}


\twolineshloka
{नलोपि राजा कौन्तेय श्रुत्वा राज्ञां समागमम्}
{अभ्यगच्छददीनात्मा दमयन्तीमनुव्रतः}


\twolineshloka
{अथ देवाः पथि नलं ददृशुर्भूतले स्तितम्}
{साक्षादिव स्थितं मूर्त्या मन्मथं रूपसंपदा}


\twolineshloka
{तं दृष्ट्वा लोकपालास्ते भ्राजमानं यथा रविम्}
{तस्थुर्विगतसंकल्पा विस्मिता रूपसंपदा}


\twolineshloka
{ततोऽन्तरिक्षे विष्टभ्य विमानानि दिवौकसः}
{अब्रुवन्नैषधं राजन्नवतीर्य नभस्तलात्}


\twolineshloka
{भोभो निषधराजेन्द्र नल सत्यव्रतो भवात्}
{अस्माकं कुरु साहाय्यं दूतो भव नरोत्तम}


\chapter{अध्यायः ५२}
\twolineshloka
{बृहदश्व उवाच}
{}


\twolineshloka
{तेभ्यः प्रतिज्ञाय नलः करिष्य इति भारत}
{अथैतान्परिपप्रच्छ कृताञ्जलिरुपस्थितः}


\twolineshloka
{के वै भवन्तः कश्चासौ यस्याहं दूत ईप्सितः}
{किंच तत्रमया कार्यं कथयध्वं यथातथम्}


\twolineshloka
{एवमुक्ते नैषधेन मघवानभ्यभाषत}
{अमरान्वै निबोधास्मान्दमयन्त्यर्थमागतान्}


\twolineshloka
{अहमिन्द्रोऽयमग्निश्च तथैवायमपांपतिः}
{शरीरान्तकरो नॄणां यमोऽयमपि पार्थिव}


\twolineshloka
{त्वं वै समागतानस्मान्दमयन्त्यै निवेदय}
{लोकपाला महेन्द्राद्याः समायान्ति दिदृक्षवः}


\twolineshloka
{प्राप्तुमिच्छन्ति देवास्त्वां शक्रोऽग्निर्वरुणो यमः}
{तेषामन्यतमं देवं पतित्वे वरयस्व ह}


\twolineshloka
{एवमुक्तः स शक्रेण नलः प्राञ्जलिरब्रवीत्}
{एकार्थसमवेतं मां न प्रेषयितुमर्हथ}


\twolineshloka
{कथ हि जातसंकल्पः स्त्रियमुत्सृजते पुमान्}
{परार्थमीदृशं वक्तं तद्वै पश्यामरेश्वर}


\threelineshloka
{`एवमुक्तो नैषधेन मघवान्पुनरब्रवीत्}
{'करिष्य इतिसंश्रुत्य पूर्वमस्मासु नैषध}
{न करिष्यसि कस्मात्त्वं व्रज नैषध माचिरम्}


\threelineshloka
{`स वै त्वमागतानस्मान्दमयन्त्यै निवेदय}
{श्रेयसा योक्ष्यसे हि त्वं कुर्वन्नमरशासनम् ॥'बृहदश्व उवाच}
{}


\twolineshloka
{एवमुक्तः स देवैस्तैर्नैषधः पुनरब्रवीत्}
{सुरक्षितानि वेश्मानि प्रवेष्टुं कथमुत्सहे}


\twolineshloka
{प्रवेक्ष्यसीति तं शक्रः पुनरेवाभ्यभाषत}
{जगाम स तथेत्युक्त्वा दमयन्त्या निवेशनम्}


\twolineshloka
{ददर्श तत्र वैदर्भीं सखीगणसमावृताम्}
{देदीप्यमानां वपुषा श्रिया च वरवर्णिनीम्}


\twolineshloka
{अतीव सुकुमाराङ्गीं तनुमध्यां सुलोचनाम्}
{आक्षिपन्तीमिव च तां शशिनं स्वेन तेजसा}


\twolineshloka
{तस्य दृष्ट्वैव ववृधे कामस्तां चारुहासिनीम्}
{सत्यं चिकीर्षमाणस्तुधारयामास हृच्छयम्}


\twolineshloka
{ततस्ता नैषधं दृष्ट्वा संभ्रान्ताः परमाङ्गनाः}
{आसनेभ्यः समुत्पेतुस्तेजसा तस्य धर्षिताः}


\twolineshloka
{प्रशशंसुश्च मुप्रीता नलं ता विस्मयान्विताः}
{न चैनमभ्यभाषन्त मनोभिस्त्वभ्यपूजयन्}


\twolineshloka
{अहो रूपमहो कान्तिरहो धैर्यं महात्मनः}
{कोऽयं देवोऽथवा यक्षो गन्धर्वो वा भविष्यति}


\twolineshloka
{न तास्तं शक्नुवन्ति स्म व्याहर्तुमपि किंचन}
{तेजसा धर्षितास्तस्य लज्जावत्यो वराङ्गनाः}


\twolineshloka
{अथैनं स्मयमानेव स्मितपूर्वाभिभाषिणी}
{दमयन्ती नलं वीरमभ्यभाषत विस्मिता}


\twolineshloka
{कस्त्वं सर्वानवद्याङ्ग मम हृच्छयवर्धन}
{प्राप्तोस्यमरवद्वीर ज्ञातुमिच्छामि तेऽनघ}


\twolineshloka
{कथमागमनं चेह कथं चासि न लक्षितः}
{सुरक्षितं हि मे वेश्म राजा चैवोग्रशासनः}


\twolineshloka
{एवमुक्तस्तु वैदर्भ्या नलस्तां प्रत्युवाच ह}
{नलं मां विद्धि कल्याणि देवदूतमिहागतम्}


\twolineshloka
{देवास्त्वां प्राप्तुमिच्छन्ति शक्रोऽग्निर्वरुणो यमः}
{तेषामन्यतमं देवं पतिं वरय शोभने}


\twolineshloka
{तेषामेव प्रभावेण प्रविष्टोऽहमलक्षितः}
{प्रविशन्तं न मां कश्चिदपश्यन्नाप्यवारयत्}


\twolineshloka
{एतदर्थमहं भद्रे प्रेषितः सुरसत्तमैः}
{एतच्छ्रुत्वा शुभे बुद्धिं प्रकुरुष्व यथेच्छसि}


\chapter{अध्यायः ५३}
\twolineshloka
{बृहदश्व उवाच}
{}


\twolineshloka
{सा नमस्कृत्य देवेभ्यः प्रहस्य नलमब्रवीत्}
{प्रणयस्व यथाश्रद्धं राजन्किं करवाणि ते}


\twolineshloka
{अहं चैव हि यच्चान्यन्ममास्ति वसु किंचना}
{तत्सर्वं तव विस्रब्धं कुरु प्रणयमीश्वर}


\twolineshloka
{हंसानां वचनं यत्तु तन्मां दहति पार्थिव}
{त्वत्कृते हि मया वीर राजानः सन्निपातिताः}


\twolineshloka
{यदि त्वं भजमानां मां प्रत्याख्यास्यसि मानद}
{विषमग्निं जलं रज्जुमास्थास्ये तव कारणात्}


\twolineshloka
{एवमुक्तस्तु वैदर्भ्या नलस्तां प्रत्युवाच ह}
{तिष्ठत्सु लोकपालेषु कथं मानुषमिच्छसि}


\twolineshloka
{येषामहं लोककृतामीश्वराणां महात्मनाम्}
{न पादरजसा तुल्यो मनस्ते तेषु वर्तताम्}


\twolineshloka
{विप्रियं ह्याचरन्मर्त्यो देवानां मृत्युच्छति}
{त्राहि मामनवद्याङ्गि वरयस्व सुरोत्तमान्}


\twolineshloka
{विरजांसि च वासांसि दिव्याश्चित्राः स्रजस्तथा}
{भूषणानि तु दिव्यानि देवान्प्राप्य तु भुङ्क्ष्व वै}


\twolineshloka
{य इमां पृथिवीं कृत्स्नां संक्षिप्य ग्रसते पुनः}
{हुताशमीशं देवानां का तं न वरयेत्पतिम्}


\twolineshloka
{यस् दण्डभयात्सर्वे भूतग्रामाः समागताः}
{धर्ममेवानुरुध्यन्ति का तं न वरयेत्पतिम्}


\twolineshloka
{धर्मात्मानं महात्मानं दैत्यदानवमर्दनम्}
{महेन्द्रं सर्वदेवानां का तं न वरयेत्पतिम्}


\twolineshloka
{क्रियतामविशङ्केन मनसा यदि मन्यसे}
{वरुणं लोकपालानां सुहृद्वाक्यमिदं शृणु}


\twolineshloka
{नैषधेनैवमुक्ता सा दमयन्ती बचोऽब्रवीत्}
{समाप्लुताभ्यां नेत्राभ्यां शोकजेनाथ वारिणा}


\twolineshloka
{देवेभ्योऽहं नमस्कृत् यसर्वेभ्यः पृथिवीपते}
{वृणे त्वामेव भर्तारं सत्यमेतद्ब्रवीमि ते}


\twolineshloka
{तामुवाच ततो राजा वेषमानां कृताञ्जलिम्}
{दौत्येनागत्य कल्याणि नोत्सहे स्वार्थमीप्सितं}


\twolineshloka
{कथं ह्यहं प्रतिश्रुत्य देवतानां विशेषतः}
{परार्थे यत्नमारभ्य कथं स्वार्थमिहोत्सहे}


\twolineshloka
{एष धर्मो यदि स्वार्थो ममापि भविता ततः}
{एवं स्वार्थं करिष्यामि तथा भद्रे विधीयताम्}


\twolineshloka
{ततो बाष्पाकुलांवाचं दमयन्ती शुचिस्मिता}
{प्रत्याहरन्ती शनकैर्नलं राजानमब्रवीत्}


\twolineshloka
{अस्त्युपायो मया दृष्टो निरपायो नरेश्वर}
{येन दोषो न भविता तव राजन्कथंचन}


\twolineshloka
{त्वं चैव हि नरश्रेष्ठ देवाश्चेन्द्रपुरोगमाः}
{आयान्तु सहिताः सर्वे मम यत्र स्वयंवरः}


\twolineshloka
{ततोऽहं लोकपालानां सन्निधौ त्वां नरेश्वर}
{वरयिष्ये नरव्याघ्र नैवं दोषो भविष्यति}


\twolineshloka
{एवमुक्तस्तु वैदर्भ्या नलो राजा विशंपते}
{आजगाम पुनस्तत्र यत्र देवाः समागताः}


\twolineshloka
{तमपश्यंस्तथाऽऽयान्तं लोकपाला महेश्वराः}
{दृष्ट्वा चैनं ततोऽपृच्छन्वृत्तान्तं सर्वमेव तम्}


\threelineshloka
{कच्चिद्दृष्टा त्वया राजन्दमयन्ती शुचिस्मिता}
{किमब्रवीच्च नः सर्वान्वद भूमिपतेऽनघ ॥नल उवाच}
{}


\twolineshloka
{भवद्भिरहमादिष्टो दमयन्त्या निवेशनम्}
{प्रविष्टः सुमहाकक्ष्यं दण्डिभिः स्थविरैर्वृतम्}


\twolineshloka
{प्रविशन्तं च मां तत्र न कश्चिद्दृष्टवान्नरः}
{ऋते तां पार्तिवसुतां भवतामेव तेजसा}


\twolineshloka
{सख्यश्चास्या मया दृष्टास्ताभिश्चाप्युपलक्षितः}
{विस्मिताश्चाभवन्सर्वा दृष्ट्वा मां विबुधेश्वराः}


\twolineshloka
{वर्ण्यमानेषु च मया भवत्सु रुचिराननां}
{मामेव गतसंकल्पा वृणीते सा सुरोत्तमाः}


\twolineshloka
{अब्रवीच्चैव मां बाला आयान्तु सहिताः सुराः}
{त्वया सह नरव्याघ्र मम यत्रस्वयंवरः}


\twolineshloka
{तेषामहं संनिधौ त्वां वरयिष्यामि नैषध}
{एवं तव महाबाहो दोषो न भवितेति ह}


\twolineshloka
{एतावदेव विबुधा यथावृत्तमुपाहृतम्}
{मया शेषे प्रमाणं तु भवन्तस्त्रिदशेश्वराः}


\chapter{अध्यायः ५४}
\twolineshloka
{बृहदश्व उवाच}
{}


\twolineshloka
{अथ काले शुभे प्राप्ते तथौ पुण्ये क्षणे तथा}
{आजुहाव महीपालान्भीमो राजा स्वयंवरे}


\twolineshloka
{तच्छ्रुत्वा पृथिवीपालाः सर्वे हृच्छयपीडिताः}
{त्वरिताः समपाजग्मुर्दमयन्तीमभीप्सवः}


\twolineshloka
{कनकस्तम्भरुचिरं तोरणेन विराजितम्}
{विविशुस्ते नृपा रङ्गं महासिंह इवाचलम्}


\twolineshloka
{तत्रासनेषु विविधेष्वासीनाः पृथिवीक्षितः}
{सुरभिस्रग्धराः सर्वे प्रमृष्टमणिकुण्डलाः}


\twolineshloka
{संपूर्णां पुरुषव्याघ्रैर्व्याघ्रैर्गिरिगुहामिव}
{`प्रविवेश नलो देवैः पुण्यश्लोको नराधिप'}


\twolineshloka
{तत्र स्म पीना दृश्यन्ते बाहवः परिघोपमाः}
{आकारवर्णसुश्लक्ष्णाः पञ्चशीर्षा इवोरगाः}


\twolineshloka
{सुकेशान्तानि चारूणि सुनासानि शुभानि च}
{मुखानि राज्ञां शोभन्ते नक्षत्राणि यथा दिवि}


\twolineshloka
{दमयन्ती ततो रङ्गं प्रविवेश शुभानना}
{मुष्णन्ती प्रभया राज्ञां चक्षूषि च मनांसि च}


\twolineshloka
{तस्या गात्रेषु पतिता तेषां दृष्टिर्महात्मनाम्}
{तत्रतत्रैव सक्ताऽभून्न चचाल च पश्यताम्}


\twolineshloka
{ततः संकीर्त्यमानेषु राज्ञां नामसु भारत}
{ददर्श भैमी पुरुषान्पञ्च तुल्याकृतीनिह}


\twolineshloka
{तान्समीक्ष्य ततः सर्वान्निर्विशेषाकृतीन्स्थितान्}
{संदेहादथ वेदर्भी नाभ्यजानान्नलं नृपम्}


\threelineshloka
{`निर्विशेषवयोवेषरूपाणां तत्र सा शुभा}
{'यंयं हि ददृशे तेषां तंतं मेने नलं नृपम्}
{साचिन्तयन्ती बुद्ध्याऽथ तर्कयामास भामिनी}


\twolineshloka
{कथं नु देवाञ्जानीयां कथं विद्यां नलं नृपम्}
{एवं संचिन्तयन्ती सा वैदर्भी भृशदुःखिता}


\twolineshloka
{श्रुतानि देवलिङ्गानि तर्कयामास भारत}
{देवानां यानि लिङ्गानि स्थविरेभ्यः श्रुतानि मे}


\twolineshloka
{तानीह तिष्ठतां भूमावेकस्यापि न लक्षये}
{एवं विचिन्त्य बहुधा विचार्य च पुनः पुनः}


\twolineshloka
{शरणं प्रति देवानां प्राप्तकालममन्यत}
{वाचा च मनसा चैव नमस्कारं प्रयुज्य सा}


\threelineshloka
{देवेभ्यः प्राञ्जलिर्भूत्वा वेपमानेदमब्रवीत्}
{हंसानां वचनं श्रुत्वा यथा मे नैषधो वृतः}
{पतित्वे तेन सत्येन देवास्तं प्रदिशन्तु मे}


\twolineshloka
{मनसा वचसा चैव यथा नातिचराम्यहम्}
{तेन सत्येन विबुधास्तमेव प्रदिशन्तु मे}


\twolineshloka
{यथा देवैः स मे भर्ता विहितो निषधाधिषः}
{तेन सत्येन मे देवास्तमेव प्रदिशन्तु मे}


\twolineshloka
{यथेदं व्रतमारब्धं नलस्याराधने मया}
{तेन सत्येन मे देवास्तमेव प्रदिशन्तु मे}


\twolineshloka
{स्वं चैव रूपं पुष्यन्तु लोकपाला महेश्वराः}
{यथाऽहमभिजानीयां पुण्यश्लोकं नराधिपम्}


\twolineshloka
{निशम्य दमयन्त्यास्तत्करुणं प्रतिदेवितम्}
{निश्चयंपरमं तथ्यमनुरागं च नैषधे}


\twolineshloka
{मनोविशुद्धिं बुद्धिं च भक्तिं रागं च नैषधे}
{यथोक्तं चक्रिरे देवाः सामर्थ्यं लिङ्गधारणे}


\twolineshloka
{साऽपश्यद्विबुधान्सर्वानस्वेदान्स्तब्धलोचनान्}
{अम्लानस्रग्रजोहीनान्स्थितानस्पृशतः क्षितिम्}


\twolineshloka
{छायाद्वितीयो म्लानस्रग्रजःस्वेदसमन्वितः}
{भूमिष्ठो नैषधश्चैव निमेषेण च सूचितः}


\twolineshloka
{सा समीक्ष्य तु तान्देवान्पुण्यश्लोकं च भारत}
{नैषधं वरयामास भैमी धर्मेण पाण्डव}


\twolineshloka
{विलज्जमाना वस्त्रान्तं जग्राहायतलोचना}
{स्कन्धदेशेऽसृजत्तस्य स्रजं परमशोभनाम्}


\twolineshloka
{वरयामास चैवैनं पतित्वे वरवर्णिनी}
{ततो हाहेति सहसा मुक्तः शब्दो नराधिपैः}


\twolineshloka
{देवैर्महर्षिभिस्तत्र साधुसाध्विति भारत}
{विस्मितैरीरितः शब्दः प्रशंसद्भिर्नलं नृपम्}


\twolineshloka
{दमयन्तीं तु कौरव्य वीरसेनसुतो नृपः}
{आश्वासयद्वरारोहां प्रहृष्टेनान्तरात्मना}


\twolineshloka
{यत्त्वं भजसि कल्याणि पुमांसं देवसन्निधौ}
{तस्मान्मां विद्धि भर्तारमेतत्ते वचने रतम्}


\twolineshloka
{यावच्च मे धरिष्यन्ति प्राणा देहे शुचिस्मिते}
{तावत्त्वयि भविष्यामि सत्यमेतद्ब्रवीमि ते}


% Check verse!
दमयन्ती तथा वाग्भिरभिनन्द्य कृताञ्जलिः
\twolineshloka
{तौ परस्परतः प्रीतौ दृष्ट्वा त्वग्निपुरोगमान्}
{तानेव शरणं देवाञ्जग्मतुर्मनसा तदा}


\twolineshloka
{वृते तु नैषधे भैम्या लोकपाला महौजसः}
{प्रहृष्टमनसः सर्वे नलायाष्टौ वरान्ददुः}


\twolineshloka
{प्रत्यक्षदर्शनं यज्ञे गतिं चानुत्तमां शुभाम्}
{नैषधाय ददौ शक्रः प्रीयमाणः शचीपतिः}


\twolineshloka
{अग्निरात्मभवं प्रादाद्यत्र वाञ्छति नैषधः}
{लोकानात्मप्रभांश्चैव ददौ तस्मै हुताशनः}


\twolineshloka
{यमस्त्वन्नरसं प्रादाद्धर्मे च परमां स्थितिम्}
{अपांपतिरपां भावं यत्रवाञ्छति नैपधः ॥स्रजश्चोत्तमगन्धाढ्याः सर्वेच मिथुनं ददुः}


\threelineshloka
{वरानेवं प्रदायास्य देवास्ते त्रिदिवं गताः}
{`एतत्सर्वं नलोऽपश्यद्दमयन्ती च भारत}
{यथा स्वप्नं महाराज तथैव ददृशुर्जनाः}


\twolineshloka
{ततः स्वयवरं चक्रे भीमो राजाऽतिमानुषम्}
{समागतेषु सर्वेषु भूपालेसु विशांपते}


\twolineshloka
{दमयन्त्यपि तद्दृष्ट्वाराजमण्डलमृद्धिमत्}
{अन्वीक्ष्यनैषधं वव्रे भैमी धर्मेण भारत}


\twolineshloka
{वृते च नैषधे भैम्या निवृत्ते च स्वयंवरे}
{सर्व एव महीपालाः प्रतिजग्मुर्यथागतम्'}


\twolineshloka
{पार्थिवाश्चानुभूयास्य विवाहं विस्मयान्विताः}
{दमयन्त्याश्च मुदिताः प्रतिजग्मुर्यथागतम्}


\twolineshloka
{गतेषु पार्थिवेन्द्रेषु भीमः प्रीतो महामनाः}
{विवाहं कारयामास दमयन्त्या नलस्य च}


\threelineshloka
{उष्य तत्र तथाकामं नैषधो द्विपदांवरः}
{भीमेन समनुज्ञातो जगाम नगरं स्वकम्}
{अवाप्य नारीरत्नं तु पुण्यश्लोकोपि पार्थिवः}


\twolineshloka
{रेमे सह तया राजञ्छच्येव बलवृत्रहा}
{अतीव मुदितो राजा भ्राजमानोंशुमानिव}


\threelineshloka
{अरञ्जयत्प्रजा वीरो धर्मेण परिपालयन्}
{ईजे चाप्यश्वमेधेन ययातिरिव नाहुषः}
{अन्यैश्च बहुभिर्धीमान्क्रतुभिश्चाप्तदक्षिणैः}


\twolineshloka
{पुगश्च रमणीयेषु वनेषूपवनेषु च}
{दमयन्त्या सह नलो विजहारामरोपमः}


\twolineshloka
{जनयामास च ततो दमयन्त्यां महामनाः}
{इन्द्रसेनं सुतं चापि इन्द्रसेनां चकन्यकाम्}


\twolineshloka
{एवं स यजमानश्च विहरंश्च नराधिपः}
{ररक्ष वसुसंपूर्णां वसुधां वसुधाधिपः}


\chapter{अध्यायः ५५}
\twolineshloka
{बृहदश्व उवाच}
{}


\twolineshloka
{वृते तु नैषधे भैम्या लोकपाला महौजसः}
{यान्तो ददृशुरायान्तं द्वापरं कलिना सह}


\twolineshloka
{अथाब्रवीत्कलिं शक्रः संप्रेक्ष्य बलवृत्रहा}
{द्वापरेण सहायेन कले ब्रूहि क्व यास्यसि}


\twolineshloka
{ततोऽब्रवीत्कलिः शक्रं दमयन्त्याः स्वयंवरम्}
{गत्वा हि वरयिष्ये तां मनो हि मम तां गतम्}


\twolineshloka
{तमब्रवीत्प्रहस्येन्द्रो निर्वृत्तः स स्वयंवरः}
{वृतस्तया नलो राजा पतिरस्मत्समीपतः}


\twolineshloka
{एवमुक्तस्तु शक्रेण कलिः कोपसमन्वितः}
{देवानामन्त्र्य तान्सर्वानुवाचेदं वचस्तदा}


\twolineshloka
{देवानां मानुषं मध्ये यत्सा पतिमविन्दत}
{ननु तस्या भवेन्न्याय्यं विपुलं दण्डधारणम्}


\twolineshloka
{एवमुक्ते तु कलिना प्रत्यूचुस्ते दिवौकसः}
{अस्माभिः सभनुज्ञाते दमयन्त्या नलो वृतः}


\twolineshloka
{का हि सर्वगुणोपेतं नाश्रयेत नलं नृपम्}
{यो वेद धर्मानखिलान्यथावच्चरितव्रतः}


\twolineshloka
{योऽधीते चतुरो वेदान्सर्वानाख्यानपञ्चमान्}
{अहिंसानिरतो यश्च सत्यवादी दृढव्रतः}


\twolineshloka
{यस्मिन्दाक्ष्यं धृतिर्ज्ञानं तपः शौचं दमः शमः}
{ध्रुवाणि पुरुषव्याघ्रे लोकपालसमे नृपे}


\twolineshloka
{एवंरूपं नलं यो वै कामयेच्छपितुं कले}
{आत्मानं स शपेन्मूढो हन्यादात्मानमात्मना}


\twolineshloka
{एवंगुणं नलं यो वै कामयेच्छपितुं कले}
{कृच्छ्रे स नरके मञ्जेदगाधे विपुले ह्रदे}


\twolineshloka
{एवमुक्त्वा कलिं देवा द्वापरं च दिवं ययुः}
{ततो गतेषु देवेषु कलिर्द्वापरमब्रवीत्}


\twolineshloka
{संयन्तुं नोत्सहे कोपं नले वत्स्यामि द्वापर}
{भ्रंशयिष्यामि तं राज्यान्न भैम्या सह रंस्यते}


\twolineshloka
{त्वमप्यक्षान्समाविश्य साहाय्यं कर्तुमर्हसि}
{`मम प्रियकृते ह्यस्मन्कृतवांश्च भविष्यसि'}


\chapter{अध्यायः ५६}
\twolineshloka
{बृहदश्व उवाच}
{}


\twolineshloka
{एवं स समयं कृत्वा द्वापरेण कलिः सह}
{आजगाम ततस्तत्र यत्रराजा स नैषधः}


\twolineshloka
{स नित्यमन्तरप्रेक्षी निषधेष्ववसच्चिरम्}
{अथास्य द्वादशे वर्षे ददर्श कलिरन्तरम्}


\twolineshloka
{कृत्वा मूत्रमुपस्पृश्य संध्यामन्वास्त नैषधः}
{अकृत्वा पादयोः शौचं तत्रैनं कलिराविशत्}


\twolineshloka
{स समाविश्य च नलं समीपं पुष्करस्य च}
{गत्वा पुष्करमाहेदमेहि दीव्य नलेन वै}


\twolineshloka
{अक्षद्यूते नलं जेता भवान्हि सहितो मया}
{निषधान्प्रतिपद्यस्व जित्वा राज्यं नलं नृपम्}


\twolineshloka
{एवमुक्तस्तु कलिना पुष्करो नलमभ्ययात्}
{कलिश्चैव वृषो भूत्वा तं वै पुष्करमन्वयात्}


\twolineshloka
{आसाद्य तु नलं वीरं पुष्करः परवीरहा}
{दीव्यावेत्यब्रवीद्धाता वृषेणेति मुहुर्मुहुः}


\twolineshloka
{न चक्षमे ततो राजा समाह्वानं महामनाः}
{वैदर्भ्याः प्रेक्षमाणायाः प्राप्तकालममन्यत}


% Check verse!
`ततः स राज्ञा सहसा देवितुं संप्रचक्रमे
\threelineshloka
{भ्रात्रा देवाभिभूतेन दैवाविष्टो जनाधिपः}
{'हिरण्यस्य सुवर्णस्य यानयुग्यस् वाससाम्}
{आविष्टः कलिना द्यूते जीयते स्म नलस्तदा}


\twolineshloka
{तमक्षमदसंमत्तं सुहृदां न तु कश्चन}
{निवारणेऽभवच्छक्तो दीव्यमानमरिंदमम्}


\twolineshloka
{ततः पौरजनाः सर्वे मन्त्रिभिः सह भारत}
{राजानं द्रष्टुमागच्छन्निवारयितुमातुरम्}


\twolineshloka
{ततः सूत उपागम्य दमयन्त्यै न्यवेदयत्}
{एष पौरजनो देवि द्वारि तिष्ठति कार्यवान्}


\twolineshloka
{निवेद्यतां नैषधाय सर्वाः प्रकृतयः स्थिताः}
{अमृष्यमाणा व्यसनं राज्ञो धर्मार्थदर्शिनः}


\twolineshloka
{ततः सा बाष्पकलया वाचा दुखेन कर्शिता}
{उवाच नैषधं भैमी शोकोपहतचेतना}


\twolineshloka
{राजन्पौरजनो द्वारि त्वां दिदृक्षुरवस्थितः}
{`वृद्धैर्ब्राह्मणमुख्यैश्च वणिग्भिश्च समन्वितः}


\twolineshloka
{आगतं सहितं राजंस्त्वत्प्रसादावलम्बनम्}
{'तं द्रष्टुमर्हसीत्येवं पुनः पुनरभाषत}


\twolineshloka
{तां तथा रुचिरापाङ्गीं विलपन्तीं तथाविधाम्}
{आविष्टः कलिना राजा नाभ्यभाषत किंचन}


\twolineshloka
{ततस्ते मन्त्रिणः सर्वे ते चैव पुरवासिनः}
{नायमस्तीति दुःखार्ता व्रीडिता जग्मुरालयान्}


\twolineshloka
{तथा तदभवद्द्यूतं पुष्करस्य नलस्य च}
{युधिष्ठिर बहून्मासान्पुण्यश्लोकस्त्वजीयत}


\chapter{अध्यायः ५७}
\twolineshloka
{बृहदश्व उवाच}
{}


\twolineshloka
{दमयन्ती ततो दृष्ट्वा पुण्यश्लोकं नराधिपम्}
{उन्मत्तवदनुन्मत्ता देवने कृतचेतसम्}


\twolineshloka
{भयशोकसमाविष्टा राजन्भीमसुता ततः}
{चिन्तयामास तत्कार्यं सुमहत्पार्थिवं प्रति}


\twolineshloka
{सा शङ्कमाना तत्पापं चिकीर्षन्ती च तत्प्रियम्}
{नलं च हृतसर्वस्वमुपलभ्येदमब्रवीत्}


\twolineshloka
{बृहत्सेनामतियशां तां धात्रीं परिचारिकाम्}
{हितां सर्वार्थकुशलामनुरक्तां सुभाषिताम्}


\twolineshloka
{बृहन्सेने व्रजामात्यानानाय्य नलंशासनात्}
{आचक्ष्व यद्धृतं द्रव्यमवशिष्टं च यद्वसु}


\twolineshloka
{`इत्येवं सा समादिष्टा बृहत्सेना नरेश्वर}
{उवाच देव्या वचनं मन्त्रिणां सा समीपत ॥'}


\twolineshloka
{ततस्ते मन्त्रिणः सर्वे विज्ञाय नलशासनम्}
{अपि नो भागधेयं स्यादित्युक्त्वा पुनराव्रजन्}


\twolineshloka
{तास्तु सर्वाः प्रकृतयो द्वितीयं समुपस्थिताः}
{न्यवेदयद्भीमसुता न च तत्प्रत्यनन्दत}


\twolineshloka
{वाक्यमप्रतिनन्दन्तं भर्तारमभिवीक्ष्य सा}
{दमयन्ती पुनर्वेश्म व्रीडिता प्रविवेश ह}


\twolineshloka
{निशाम्य सततं चाक्षान्पुण्यश्लोकपराङ्युखान्}
{नलं च हृतसर्वस्वं धात्रीं पुनरुवाच ह}


\twolineshloka
{बृहत्सेने पुनर्गच्छ वार्ष्णेयं नलशासनात्}
{सूतमानय कल्याणि महत्कार्यमुपस्थितम्}


\twolineshloka
{बृहत्सेना तु सा श्रुत्वा दमयन्त्याः प्रभापितम्}
{वार्ष्णेयमानयामास पुरुषैराप्तकारिभिः}


\twolineshloka
{वार्ष्णेयं तु ततो भैमी सान्त्वयच्छ्लक्ष्णया गिरा}
{उवाच देशकालज्ञा प्राप्तकालमनिन्दिता}


\twolineshloka
{जानषे त्वं यथा राजा सम्यग्वृत्तः सदा त्वयि}
{तस्य त्वं विपमस्थस्य साहाय्यं कर्तुमर्हसि}


\twolineshloka
{यथायथा हि नृपतिः पुष्करेणैव जीयते}
{तथातथाऽस्य वै द्यूते रागो भूयोऽभिवर्धते}


\twolineshloka
{यथा च पुष्करस्याक्षाः पतन्ति वशवर्तिनः}
{तथा विपर्ययश्चापि नलस्याक्षेषु दृश्यते}


\twolineshloka
{सुहृत्स्वजनवाक्यानि यथाऽयं न शृणोति च}
{ममापि च तथा वाक्यं नाभिनन्दति नैषध}


\twolineshloka
{यथा राज्ञः प्रदीप्तानां भाग्यानामद्य सारथे}
{नूनं मन्ये न शेषोस्ति नैषधस्य महात्मनः}


\threelineshloka
{यत्तु मे वचनं राजा नाभिनन्दति मोहितः}
{शरणं त्वां प्रपन्नाऽस्मि सारथे कुरु मद्वचः}
{न हि मे शुध्ते भावो विनाशं प्रति सारथे}


\twolineshloka
{नलस्य दयितानश्वान्योजयित्वा मनोजवान्}
{रथमारोप्य मिथुनं कुण्डिनं यातुर्महसि}


\twolineshloka
{मम ज्ञातिषु निक्षिप्य दारकौ स्यन्दनं तथा}
{अश्वांश्चेमान्यथाकामं वस वाऽन्यत्र गच्छवा}


\twolineshloka
{दमयन्त्यास्तु तद्वाक्यं वार्ण्येयो नलसारथिः}
{न्यवेदयदशेषेण नलामात्येषु मुख्यशः}


\twolineshloka
{तैः समेत्य विनिश्चित्य सोऽनुज्ञातो महीपते}
{ययौ मिथुनमारोप्य विदर्भांस्तेन वाहिना}


\twolineshloka
{हयांस्तत्र विनिक्षिप्य सूतो रथवरं च तम्}
{इन्द्रसेनां च तां कन्यामिन्द्रसेनं च बालकम्}


\threelineshloka
{आमन्त्र्य भीमं राजानमार्तः शोचन्नलं नृपम्}
{क्व नु यास्यामि मनसा चिन्तयानो मुहुर्मुहुः}
{अटमानस्ततोऽयोध्यां जगाम नगरीं तदा}


\twolineshloka
{ऋतुपर्णं स राजानमुपतस्थे सुदुःखितः}
{भृतिं च स ददौ चास्य सारथ्येन नियोजितः}


\chapter{अध्यायः ५८}
\twolineshloka
{बृहदश्व उवाच}
{}


\twolineshloka
{ततस्तु याते वार्ष्णेये पुण्यश्लोकस्य दीव्यतः}
{पुष्करेण हृतं राज्यं यच्चापि वसु किंचन}


\twolineshloka
{हृतराज्यं नलं राजन्प्रहसन्पुष्करोऽब्रवीत्}
{द्यूतं प्रवर्ततां भूयः प्रतिपाणोऽस्ति कस्तव}


\twolineshloka
{शिष्टा ते दमयन्त्येका सर्वमन्यद्धृतं मया}
{दमयन्त्याः पणः साधु वर्ततां यदि मन्यसे}


\twolineshloka
{पुष्रेणैवमुक्तस्य पुण्यश्लोकस्य मन्युना}
{व्यदीर्यतेव हृदयं न चैनं किंचिदब्रवीत्}


\twolineshloka
{ततः पुष्करमालोक्य नलः परममन्युमान्}
{`उवाच विद्यतेऽन्यच्च धनं मम नराधम}


\threelineshloka
{षणरूपेण निक्षिप्य पुण्यश्लोकः सुदुर्मनाः}
{उत्तरीयं तथा वस्त्रं तस्याश्चाभरणानि च'}
{उन्सृज्य सर्वगात्रभ्यो भूषणानि सहायशाः}


\twolineshloka
{सुकवासा ह्यसंवीतः सुहृच्छोकविवर्धनः}
{निश्चक्राम ततोराजा त्यक्त्वा सुविपुलां श्रियम्}


\twolineshloka
{दमयन्त्येकवस्त्राऽथ गच्छन्तं पृष्ठतोऽन्वगात्}
{स तया नगराभ्याशे त्रिरात्रं नैपधोऽवसत्}


\twolineshloka
{पुष्करस्तु महाराज धोषयामास वै पुरे}
{नले यः सम्यगातिष्ठेत्स गच्छेद्वध्यतां मम}


\twolineshloka
{पुष्करस्य तु वाक्येन तस्य विद्वेषणेन च}
{पौरा न तस्य सत्कारं कृतवन्तो युधिष्ठिर}


\twolineshloka
{स तथा नगराभ्याशे सत्कारार्हो न सन्कृतः}
{त्रिगत्रमुपितो राजा जलमात्रेण वर्तयन्}


\twolineshloka
{[पीड्यमानः क्षुधा तत्र फलमृलानि कर्पयन्}
{प्रातिष्ठत ततोराजा दमयन्ती तमन्वगात् ॥]}


\twolineshloka
{क्षुधया पीड्यमानम्तु नलो बहुतिथेऽहनि}
{अपश्यच्छकुनान्कांश्रिद्धिरण्यसदृशच्छदान्}


\twolineshloka
{स चिन्तयामास तदा निपधाधिपतिर्बली}
{अस्ति भक्ष्यो ममाद्यापि वसु चेदं भविष्यति}


\twolineshloka
{ततस्तानन्तरीयेण वाससा स समावृणोत्}
{तस्य तद्वस्त्रमादाय सर्वे जग्मुर्विहायसा}


\twolineshloka
{उन्पतन्तः खगा वाक्यमेतदाहुस्ततो नलम्}
{दृष्ट्वा दिग्वाससं भूमौ स्थितं दीनमधोमुखम्}


\twolineshloka
{वयमक्षाः सुदुर्बुद्धे तववासो जिहीर्षवः}
{आगता न हि नः प्रीतिः सवाससि गते त्वयि}


\twolineshloka
{नान्समीक्ष्य गतानक्षानात्मानं च विवाससम्}
{पुण्यश्लोकस्तदा राजा दमयन्तीमथाब्रवीत्}


\twolineshloka
{येषां प्रकोपादैश्वर्यान्प्रच्युतोऽहमनिन्दिते}
{प्राणयात्रां न विन्देयं दुःखितः क्षुधयाऽन्वितः}


\twolineshloka
{येषां कृते न सत्कारमकुर्वन्मयि नैषधाः}
{इमे ते शकुना भूत्वा वासश्चाषहरन्ति मे}


\twolineshloka
{वैषम्यं परमं प्राप्तो दुःखितो गतचेतनः}
{भर्ता तेऽहं निबोधेदं वचनं हितमात्मनः}


\twolineshloka
{एते गच्छन्ति वहवः पन्थानो दिणापथम्}
{अवन्तीमृक्षवन्तं च समतिक्रम्य पर्वतम्}


\twolineshloka
{एष विन्ध्यो महाशैलः पयोष्णी च समुद्रगा}
{आश्रमाश्च महर्षीणां बहुमूलफलान्विताः}


\twolineshloka
{एष पन्था विदर्भाणामेष यास्यति कोसलान्}
{अतःपरं च देशोऽयं दक्षिणो दक्षिणापथः}


\twolineshloka
{एतद्वाक्यं नलो राजा दमयन्तीं समाहितः}
{उवाचासकृदार्तो हि भैमीमुद्दिश्य भारत}


\twolineshloka
{ततः सा वाष्पकलया वाचा दुःखेन कर्शिता}
{उवाच दमयन्ती तं नैषधं करुणं वचः}


\twolineshloka
{उद्वेषते मे हृदयं सीदन्त्यङ्गानि सर्वशः}
{तव पार्थिव संकल्पं चिन्तयन्त्याः पुनः पुनः}


\twolineshloka
{हृतराज्यं हृतद्रव्यं विवस्त्रं क्षुच्छ्रमान्वितम्}
{कथमुत्सृज्य गच्छेयमहं त्वां निर्जने वने}


\twolineshloka
{श्रान्तस्य ते क्षुधार्तस्य चिन्तयानस्य तत्सुखम्}
{वने घोरे महाराज नाशयिष्याम्यहं क्लमम्}


\threelineshloka
{न च भार्यासमं किंचिन्नरस्यार्तस्य भेषजम्}
{नित्यं हि सर्वदुःखेषु सत्यमेतद्व्रवीमि ते ॥नल उवाच}
{}


\twolineshloka
{एवमेतद्यथाऽऽन्थ त्वं दमयन्ति सुमध्यमे}
{नास्ति बार्यासमं मित्रं नरस्यार्तस्य भेषजम्}


\threelineshloka
{न चाहं त्यक्तुकामस्त्वां किमलं भीरु शङ्कसे}
{त्यजेयमहमात्मानं न चैव त्वामनिन्दिते ॥दमयन्त्युवाच}
{}


\twolineshloka
{यदि मां त्वं महाराज न विहातुमिहच्छसि}
{तत्किमर्थं विदर्भाणां पन्थाः समुपदिश्यते}


\twolineshloka
{अवैमि चाहं नृपते न तु मां त्यक्तुमर्हसि}
{चेतसा त्वपकृष्टेन मां त्यजेथा महीपते}


\twolineshloka
{पन्थानं हि ममाभीक्ष्णमाख्यासि च नरोत्तम}
{अतो निमित्तं शोकं मे वर्धयस्यमरोपम}


\twolineshloka
{यदि चायमभिप्रायस्तव राजन्भविष्यति}
{सहितावेव गच्छावो विदर्भान्यदि मन्यसे}


\twolineshloka
{विदर्भराजस्तत्र त्वां पूजयिष्यति मानद}
{तेन त्वं पूजितो राजन्सुखं वत्स्यसि नो गृहे}


\chapter{अध्यायः ५९}
\twolineshloka
{बृहदश्व उवाच}
{}


\threelineshloka
{`इत्युक्तः स तदा देव्या नलो वचनमब्रवीत्}
{'यथा राज्यंतव पितुस्तथा मम नसंशयः}
{न तु तत्र गमिष्यामि विषमस्थः कथंचन}


\twolineshloka
{कथं समृद्धो गत्वाऽहं तव हर्षविवर्धनः}
{परिद्यूनो गमिष्यामि तव शोकविवर्धनः}


\twolineshloka
{इति ब्रुवन्नलो राजा दमयन्तीं पुनः पुनः}
{सान्खयामास कल्याणीं वाससोर्धेन संवृताम्}


\twolineshloka
{तावेकवस्त्रसंवीतावटमानावितस्ततः}
{क्षुत्पिपासापरिश्रान्तौ सभां कांचिदुपेयतुः}


\twolineshloka
{तां सभामुपसंप्राप्य तदा स निषधाधिपः}
{वैदर्भ्या सहितो राजा निषसाद महीतले}


\twolineshloka
{स वै विवस्त्रो मलिनो विवर्णः पांसुगुण्ठितः}
{दमयन्त्या सह श्रान्तः सुष्वाप धरणीतले}


\twolineshloka
{`ततः स्ववसनस्यान्तं दमयन्ती विशांपते}
{भूमावास्तीर्य सुष्वाप पतिमालिङ्ग्य शोभना ॥'}


\twolineshloka
{दमयनत्यपि कल्याणी निद्रयाऽपहृता ततः}
{सहसा दुःखमासाद्य सुकुमारी तपस्विनी}


\twolineshloka
{सुप्तायां दमयन्त्यां तु नलो राजा विशांपते}
{शोकोन्मथितचित्तः सन्न स्म शेते यथा पुरा}


\twolineshloka
{स तद्राज्यापहरणं सुहृत्त्यागं च सर्वशः}
{वने च तं परिध्वंसं प्रेक्ष्य चिन्तामुपेयिवान्}


\twolineshloka
{किं नु मे स्यादिदं कृत्वा किंनु मे स्यादकुर्वतः}
{किंनु मे मरणं श्रेयः परित्यागो जनस्य वा}


\twolineshloka
{मामियं ह्यनुरक्तैवं दुःखमाप्नोति मत्कृते}
{मद्विहीना त्वियं गच्छेत्कदाचित्स्वजनं प्रति}


\threelineshloka
{मया निःसंशयं दुखमियं प्राप्स्यत्यनुव्रता}
{उत्सर्गेसंशयः स्यात्तु विन्देतापि सुखं क्वचित् ॥बृहदश्च उवाच}
{}


\twolineshloka
{स विनिश्चित्य बहुधा विचार्य च पुनःपुनः}
{उत्सर्गेऽमन्यत श्रेयो दमयन्त्या नराधिप}


\twolineshloka
{न चैषा तेजसा शक्या कैश्चिद्धर्षयितुं पथि}
{यशस्विनी महाभागा मद्भक्तेयं पतिव्रता}


\twolineshloka
{एवं तस् तदा बुद्धिर्दमयन्त्यां न्यवर्तत}
{कलिना दुष्टभावेन दमयन्त्या विसर्जने}


\twolineshloka
{सोऽवस्त्रतामात्मनश्च तस्याश्चाप्येकवस्त्रताम्}
{चिन्तयित्वाऽध्यगाद्राजा वस्त्रार्धस्यावकर्तनम्}


\twolineshloka
{कथं वासो विकर्तेयं न च बुद्ध्येत मे प्रिया}
{विचिन्त्यैवं नलो राजा सभां पर्यचरत्तदा}


\twolineshloka
{परिधावन्नथ नल इतश्चेतश्च भारत}
{आससाद सभोद्देशे विकोशं स्वङ्गमुत्तमम्}


\twolineshloka
{तेनार्धं वाससश्छित्त्वा निवस् च परंतपः}
{सुप्तामुत्सृज्य वैदर्भीं प्राद्रवद्गतचेतनाम्}


\twolineshloka
{ततो निबद्धहृदयः पुनरागम् तां सभाम्}
{दमयन्तीं तदा दृष्ट्वा रुरोद निषधाधिपः}


\twolineshloka
{यां न वायुर्न चादित्यः पुरा पश्यति मे प्रियाम्}
{सेयमद्य सभामध्ये शेते भूमावनाथवत्}


\twolineshloka
{इयं वस्त्रावकर्तेन संवीता चारुहासिनी}
{उन्मत्तेव मया हीना कथं बुद्ध्वा भविष्यति}


\twolineshloka
{कथमेका सती भमी मया विरहिता शुभा}
{चरिष्यति वने घोरे मृगव्यालनिषेविते}


\twolineshloka
{आदित्या वसवो रुद्रा अश्विनौ समरुद्गणौ}
{रक्षन्तु त्वां महाभागे धर्मेणासि समावृता}


\twolineshloka
{एवमुक्त्वा प्रियां भार्यां रूपेणाप्रतिमां भुषि}
{कलिनाऽपहृतज्ञानो नलः प्रातिपष्ठदुद्यतः}


\twolineshloka
{गत्वागत्वा नलो राजा पुनरेति सभां मुहुः}
{आकृष्यमाणः कलिना सौहृदेनावकृष्यते}


\twolineshloka
{द्विधेव हृदयं तस्य दुःखितस्याभवत्तदा}
{दोलेव मुहुरायाति याति चैव मुहुर्मुहुः}


\twolineshloka
{अवकृष्टस्तु कलिना मोहितः प्राद्रवन्नलः}
{सुप्तामुत्सृज्य तां भार्यां विलप्य करुणं बहु}


\twolineshloka
{नष्टात्मा कलिनाऽऽविष्टस्तत्तद्विगणयन्मुहुः}
{जगामैकां वने शून्ये भार्यामुत्सृज्य मोहितः}


\chapter{अध्यायः ६०}
\twolineshloka
{बृहदश्व उवाच}
{}


\twolineshloka
{अपक्रान्ते नले राजन्दमयन्ती गतक्लमा}
{अबुध्यत वरारोहा संत्रस्ता विजने वने}


\twolineshloka
{अपश्यमाना भर्तारं शोकदुःखसमन्विता}
{प्राक्रोशदुच्चैः संत्रस्ता महाराजेति नैषधम्}


\twolineshloka
{हा नाथ हा महाराज हा स्वामिन्किं जहासि माम्}
{हा हताऽस्मि विनष्टाऽस्मि भीताऽस्मि विजने वने}


\twolineshloka
{ननु नाम महारज धर्मज्ञः सत्यवागसि}
{कथंविधस्त्वंहि तथा सुप्तामुत्सृज्य मां गतः}


\twolineshloka
{कथमुत्सृज्यगन्ताऽसि वश्यां भार्यामनुव्रताम्}
{विशेषतो नापकृतः परेणापकृतो ह्यसि}


\twolineshloka
{शक्यसे न गिरः सत्याः कर्तुं मयि नरेश्वर}
{यास्त्वया लोकपालानां सन्निधौ कथिताः पुरा}


\twolineshloka
{नाकाले विहितो मृत्युर्मर्त्यानां पुरुषर्षभ}
{यत्र कान्ता त्वयोत्सृष्टा मुहूर्तमपि जीवति}


\twolineshloka
{पर्याप्तः परिहासोऽयमेतावान्पुरुषर्षभ}
{भृशं भीतास्मि दुर्धर्ष दर्शयात्मानमीश्वर}


\twolineshloka
{दृश्यसे दृश्यसे राजन्नेष तिष्ठसि नैषध}
{आवार्य गुल्मैरात्मानं किं मां न प्रतिभाषसे}


\twolineshloka
{नृशंसां वत राजेन्द्र यन्मामेवंगतामिह}
{विलपन्तीं समालिङ्ग्य नाश्वासयसि पार्थिव}


\twolineshloka
{न शोचाम्यहमात्मानं न चान्यदपि किंचन}
{कथं नु भवितास्येक इति त्वां नृप शोचये}


\twolineshloka
{कथं नु राजंस्तृषितः क्षुक्षितः श्रमकर्शितः}
{सायाह्ने वृक्षमूलेषु मामपश्यन्भविष्यसि}


\twolineshloka
{ततः सा तीव्रशोकार्ता प्रदीप्तेव च मन्युना}
{इतश्चेतश्च रुदती पर्यधावत दुःखिता}


\twolineshloka
{मुहुरुत्पतते बाला मुहुः पतति विह्वला}
{मुहुरालीयते भीता मुहुः क्रोशति रोदिति}


\twolineshloka
{अतीव शोकसंतप्ता मुहुर्निःश्वस् विह्वला}
{उवाच भैमी निःश्वस् रोदमाना पतिव्रता}


\twolineshloka
{यस्याभिशापाद्दुःखार्तो नैष नन्दति नैषधः}
{तस्य भूतस्य तद्दुःखाद्दुःखमभ्यधिकं भवेत्}


\twolineshloka
{अपापचेतसं पापो य एवं कृतवान्नलम्}
{तस्माद्दुःखतरं प्राप्य जीवत्वमसुखजीविकाम्}


\twolineshloka
{एवं तु विलपन्ती सा राज्ञो भार्या महात्मनः}
{अन्वेषति स्म भर्तारं वने श्वापदसेविते}


\twolineshloka
{उन्मत्तवद्भीमसुता विलपन्ती ततस्ततः}
{हाहा राजन्निति मुहुरितश्चेतश्च धावति}


\threelineshloka
{तां शुष्यमाणामत्यर्थं कुररीमिव वाशतीम्}
{करुणं बहुशोचन्तीं विलपन्तीं सुमध्यमाम्}
{}


\twolineshloka
{सहसाऽभ्यागतां भैमीमभ्याशपरिवर्तिनीम्}
{जग्राहाजगरो ग्राहो महाकायः क्षुधान्वितः}


\twolineshloka
{सा ग्रस्यमाना ग्रहेण शोकेन च पराजिता}
{नात्मानं शोचति तथा यथा शोचति नैषधम्}


\twolineshloka
{द्वा नाथ मामिह वने ग्रस्यमानामनागसम्}
{ग्राहेणानेन विजने किमर्थं नानुधावसि}


\threelineshloka
{कथं भविष्यसि पुनर्मामनुस्मृत्य नैषध}
{[कथं भवाञ्जगामाद्य मामुत्सृज्य वने प्रभो}
{]पापान्मुक्तः पुनर्लब्ध्वा बुद्धिं चोतो धनानि च}


\threelineshloka
{श्रान्तस्य ते क्षुधार्तस् परिग्लानस्य नैषध}
{कः श्रमं राजशार्दूल नाशयिष्यति तेऽनघ ॥बृहदश्च उवाच}
{}


\twolineshloka
{तां तु दृष्ट्वा तथा ग्रस्तामुरगेणायतेक्षणाम्}
{आक्रन्दमानां संश्रुत्य जवेनाभिससार ह}


\threelineshloka
{तां तु दृष्ट्वा तथा ग्रस्तामुरगेणायतेक्षणाम्}
{त्वरमाणो मृगव्याधः संचरन्गहने वने}
{समभिक्रम्य वेगेन सत्वरः स वनेचरः}


\twolineshloka
{मुखे तं पाटयामास शस्त्रेण निशितेन च}
{निर्विचेष्टं भुजङ्गं तं विशस् मृगजीवनः}


\twolineshloka
{मोक्षयित्वा स तांव्याधः प्रक्षाल्य सलिलेन ह}
{समाश्वास्य कृताहारामथ पप्रच्छ भारत}


\twolineshloka
{कस्य त्वं मृगशावाक्षि कथं चास्यागता वनम्}
{कथं चेदं महत्कृच्छ्रं प्राप्तवत्यसि भामिनि}


\twolineshloka
{दमयन्ती तथा तेन पृच्छ्यमाना विशांपते}
{सर्वमेतद्यथावृत्तमाचचक्षेऽस्य भारत}


\twolineshloka
{तामर्धवस्त्रसंवीतां पीनश्रोणिपयोधराम्}
{सुकुमारानवद्याङ्गीं पूर्णचन्द्रनिभाननाम्}


\twolineshloka
{अरालपक्ष्मनयनां तथा मधुरभाषिणीम्}
{लक्षयित्वा मृगव्याधः कामस्य वशमीयिवान्}


\twolineshloka
{तामथ श्लक्ष्णया वाचा लुब्धको मृदुपूर्वया}
{सान्त्वयामास कामार्तस्तदबुध्यत भामिनी}


\twolineshloka
{दमयन्त्यपि तं दुष्टमुपलभ्य पतिव्रता}
{तीव्ररोषसमाविष्टा प्रजज्वालेव मन्युना}


\twolineshloka
{स तु पापमतिः क्षुद्रः प्रधर्षयितुमातुरः}
{दुर्धर्षां तर्कयामास दीप्तामग्निशिखामिव}


\twolineshloka
{दमयन्ती तु दुःखार्था पतिराज्यविनाकृता}
{अतीतवाक्यथे काले शशापैनं रुषा किल}


\twolineshloka
{यथाऽहं नैषधादन्यं मनसाऽपि न चिन्तये}
{तथाऽयं पततां क्षुद्रः परासुर्मृगजीवनः}


\twolineshloka
{उक्तमात्रे तु वचने तया स मृगजीवनः}
{व्यसुः पपात मेदिन्यामग्निदग्ध इव द्रुमः}


\chapter{अध्यायः ६१}
\twolineshloka
{बृहदश्व उवाच}
{}


\twolineshloka
{सा निहत्य मृगव्याधं प्रतस्थे कमलेक्षणा}
{वनं प्रतिभयं शून्यं झिल्लिकागणनादितम्}


\twolineshloka
{सिंहद्वीपिरुरुव्याघ्रवराहर्क्षगणैर्युतम्}
{नानापक्षिगणाकीर्णां म्लेच्छतस्करसेवितम्}


\twolineshloka
{सालवेणुधवाश्वत्थतिन्दुकेङ्गुदिकिंशुकैः}
{अर्जुनारिष्टसंछन्नं स्यन्दनैश्च सशाल्मलैः}


\twolineshloka
{जम्ब्वाम्रलोध्रखदिरशाकवेत्रसमाकुलम्}
{पद्मकामलकप्लक्षकदम्बोदुम्बरावृतम्}


\twolineshloka
{बदरीबिल्वसंछन्नं न्यग्रोधैश्च समाकुलम्}
{प्रियालतालखर्जूरहरीतकिबिभीतकैः}


\twolineshloka
{नानाधातुशतैर्नद्धान्विविधानपि चाचलान्}
{निकुञ्जान्पक्षिसंघुष्टान्दरीश्चाद्भुतदर्शनाः}


\twolineshloka
{नदी सरांसि वापीश्च विविधांश्चरतो द्विजान्}
{सा बहून्भीमरूपांश्च पिशाचोरगराक्षसान्}


\twolineshloka
{पल्वलानि तटाकानि गिरिकूटानि सर्वशः}
{सरित्सरांसि च तदा ददर्शाद्भुतदर्शनाम्}


\twolineshloka
{यूथशो ददृशे चात्र विदर्भाधिपनन्दिनी}
{महिषान्वराहान्गोमायूनृक्षवानरपन्नगान्}


\twolineshloka
{तेजसा यशसा लक्ष्म्या स्थित्या च परया युता}
{वैर्दभी विचचारैका नलमन्वेषती तदा}


\twolineshloka
{नाबिभ्यत्सा नृपसुता भैमी तत्राथ कस्यचित्}
{दारुणामटवीं प्राप्य भर्तृव्यसनकर्शिता}


\threelineshloka
{विदर्भतनया राजन्विललाप सुदुःखिता}
{भर्तृशोकपरीताङ्गी शिलातलमथाश्रिता ॥दमयन्त्युवाच}
{}


\twolineshloka
{सिह्योरस्क महाबाहो निषधानां जनाधिप}
{क्वनु राजन्गतोसि त्वं त्यक्त्वा मां विजने वने}


\twolineshloka
{अश्वमेधादिभिर्वीर क्रतुभिश्चाप्तदक्षिणैः}
{कथमिष्ट्वा नरव्याघ्र मयि मिथ्या प्रवर्तसे}


\twolineshloka
{यत्त्वयोक्तं नरश्रेष्ठ मत्समक्षं महाद्युते}
{कर्तुमर्हसि कल्याण तदृतं पार्थिवर्षभ}


\twolineshloka
{यच्चोक्तं विहगैर्हंसैः समीपे तव भूमिप}
{मत्सकाशे च तैरुक्तं तदवेक्षितुमर्हसि}


\twolineshloka
{चत्वार एकतो वेदाः साङ्गोपाङ्गाः सविस्तराः}
{स्वधीता मनुजव्याघ्र सत्यमेकं किलैकतः}


\twolineshloka
{तस्मादर्हसि शत्रुघ्न सत्यं कर्तुं नरेश्वर}
{उक्तवानसि यद्वीर मत्सकाशे पुरा वचः}


\twolineshloka
{हा वीर ननुनामाहमिष्टा किल तवानघ}
{अस्यामटव्यां घोरायां किं मां न प्रतिभाषसे}


\twolineshloka
{भर्त्सयत्येष मां रौद्रो व्यात्तास्यो दारुणाकृतिः}
{अरण्यराट् क्षुधाविष्टः किं मां न त्रातुमर्हसि}


\twolineshloka
{न मे त्वदन्या काचिद्धि प्रियाऽस्तीत्यब्रवीस्तदा}
{तामृतां कुरु कल्याण पुरोक्तां भारतीं नृप}


\twolineshloka
{उन्मत्तां विलपन्तीं मां भार्यामिष्टां नराधिप}
{ईप्सितामीप्सितोसि त्वं किं मां न प्रतिभाषसे}


\twolineshloka
{कृशां दीनां विवर्णां च मलिनां वसुधाधिप}
{वस्त्रार्धप्रावृतामेकां विलपन्तीमनाथवत्}


\twolineshloka
{यूथभ्रष्टामिवैकां मां हरिणीं पृथुलोचन}
{न मानयसि मामार्य रुदन्तीमरिकर्शन}


\twolineshloka
{महाराज महारण्ये मामिहैकाकिनीं सतीम्}
{आभाषमाणां स्वां पत्नीं किं मां न प्रतिभाषसे}


\twolineshloka
{कुलशीलोपसंपन्नां चारुसर्वाङ्गशोभनाम्}
{`अनुव्रतां महाराज किं मां न प्रतिभाषसे ॥'}


\twolineshloka
{नाद्यत्वां प्रतिपश्यामि गिरावस्मिन्नरोत्तम}
{वने चास्मिन्महाधोरे सिंहव्याघ्रनिषेविते}


\twolineshloka
{शयानमुपविष्टं वा स्थितं वा निषधाधिप}
{प्रस्थितं वा नरश्रेष्ठ मम शोकनिबर्हण}


\twolineshloka
{कं नु पृच्छामि दुःखार्ता त्वदर्थे शोककर्शिता}
{कच्चिद्दृष्टस्त्वयाऽरण्ये संगत्येति नलो नृपः}


\twolineshloka
{को नु मे कथयेदद्य वनेऽस्मिन्विष्ठितं नृपम्}
{अभिरूपं महात्मानं परव्यूहविनाशनम्}


\twolineshloka
{यमन्वेपसि राजानं नलं पद्मनिभेक्षणम्}
{अयंस इति कस्याद्य श्रोष्यामि मधुरां गिरम्}


\twolineshloka
{अरण्यराडयं श्रीमांश्चतुर्दंष्ट्रो महाहनुः}
{शार्दूलोऽभिमुखोऽभ्येति पृच्छाम्येनमशङ्किता}


\twolineshloka
{भवान्मृगाणामधिपस्त्वमस्मिन्कानने प्रभुः}
{विदर्भराजतनयां दमयन्तीति विद्धि माम्}


\twolineshloka
{निषधाधिपतेर्भार्यां नलस्यामित्रघातिनः}
{पतिमन्वेषतीमेकां कृपणां शोककर्शिताम्}


\twolineshloka
{आश्वासय मृगेन्द्रेह यदि दृष्टस्त्वया नलः}
{`सिंहस्कन्धो महाबाहुः पद्मपत्रनिभेक्षणः ॥'}


\twolineshloka
{अथवाऽरण्यनृपते नलं यदि न शंससि}
{मां खादय मृगश्रेष्ठ दुःखादस्माद्विमोचय}


\twolineshloka
{श्रुत्वाऽरण्ये विलपितं ममैष मृगराट् स्वयम्}
{यात्येष मृष्टसलिलामापगां सागरोपमाम्}


\twolineshloka
{इमं शिलोच्चयं पृच्छे शृङ्गैर्बहुभिरुच्छितैः}
{विराजितं दिवस्पृग्भिर्नैकवर्णैर्मनोरमैः}


\twolineshloka
{नानाधातुसमाकीर्णं विविधोपलभूषितम्}
{आस्यारण्यस्य महतः केतुभूतमिवोच्छितम्}


\twolineshloka
{सिंहशार्दूलमातङ्गवराहर्क्षमृगायुतम्}
{पतत्रिभिर्बहुविधैः समन्तादनुनादितम्}


\twolineshloka
{किंशुकाशोकबकुलपुन्नागैरुपशोभितम्}
{[कर्णिकारधवप्लक्षैः सुपुष्पैरुपशोभितम् ॥]}


\threelineshloka
{सरिद्भिः सविहङ्गाभिः शिखरैश्च समाकुलम्}
{`पृथिव्या रुचिराकारं चूडामणिमिव स्थितम्'}
{निरिराजमिमं तावत्पृच्छामि नृपतिं प्रति}


\twolineshloka
{भगवन्नचलश्रेष्ठ दिव्यद्रशन विश्रुत}
{शरण्य बहुकल्याण नमस्तेऽस्तु महीधर}


\twolineshloka
{प्रणमे त्वाऽभिगम्याहं राजपुत्रीं निबोध माम्}
{राजस्नुषां राजभार्यां दमयन्तीति विश्रुताम्}


\twolineshloka
{राजा विदर्भाधिपतिः पिता मम महारथः}
{भीमो नाम क्षितिपतिश्चातुर्वर्ण्यस्य रक्षिता}


\twolineshloka
{राजसूयाश्वमेधानां क्रतूनां दक्षिणावताम्}
{आहर्ता पार्थिवश्रेष्ठः पृथुचार्वञ्चितेक्षणः}


\twolineshloka
{ब्रह्मण्यः साधुवृत्तश्च सत्यवागनसूयकः}
{शीलवान्वीर्यसंपन्नः पृथुश्रीर्धर्मविच्छुचिः}


\twolineshloka
{सम्यग्गोप्ता विदर्भाणां निर्जितारिगणः प्रभुः}
{तस्य मां विद्धि तनयां भगवंस्त्वामुपस्थिताम्}


\twolineshloka
{निषधेषु महाराजः श्वशुरो मे नरोत्तमः}
{गृहीतनामा विख्यातो वीरसेन इति स्म ह}


\twolineshloka
{तस्य राज्ञः सुतो वीरः श्रीमान्सत्यपराक्रमः}
{क्रमप्राप्तं पितुः स्वं यो राज्यं समनुशास्ति ह}


\twolineshloka
{नलो नामारिहा श्यामः पुण्यल्को इति श्रुतः}
{ब्रह्मण्यो वेदविद्वाग्मी पुण्यकृत्सोमपोऽग्निचित्}


\twolineshloka
{यष्टा दाता च योद्धा च सम्यक्चैव प्रशासिता}
{तस्य मामचलश्रेष्ठ विद्धि भार्यामिहागताम्}


\twolineshloka
{त्यक्तश्रियं भर्तृहीनामनाथां व्यसनान्विताम्}
{अन्वेषमाणां भर्तारं नलं नरवरोत्तमम्}


\twolineshloka
{खमुल्लिखद्भिरेतैर्हि त्वया शृङ्गशतैर्नृपः}
{कच्चिद्दृष्टोऽचलश्रेष्ठ वनेऽस्मिन्दारुणे नलः}


\twolineshloka
{विक्रान्तः सत्त्ववान्वीरो भर्ता मम महायशाः}
{निषधानामधिपतिः कच्चिद्दृष्टस्त्वया नलः}


\twolineshloka
{कं मीं विलपतीमेकां पर्वतश्रेष्ठ विह्वलाम्}
{गिरा नाश्वासयस्यद्यस्वां सुतामिव दुखितां}


\twolineshloka
{वीर विक्रान्त धर्मज्ञ सत्यसंध महीपते}
{यद्यस्यस्मिन्वने राजन्दर्शयात्मानमात्मना}


\twolineshloka
{कदा सुस्निग्गम्भीरां जीमूतस्वनसन्निभाम्}
{श्रोष्यामि नैषधस्याहं वाचं ताममृतोपमाम्}


\twolineshloka
{वैदर्भ्येहीति विस्पष्टां शुभां राज्ञो महात्मनः}
{आत्माभिधायिनीमृद्धां मम शोकविनाशिनीं}


\twolineshloka
{इतिसा तं गिरिश्रेष्ठमुक्त्वा पार्थिवनन्दिनी}
{दमयन्ती ततो भूयो जगाम दिशमुत्तराम्}


\twolineshloka
{सा गत्वा त्रीनहोरात्रान्ददर्श परमाङ्गना}
{तापसारण्यमतुलं दिव्यकाननशोभितम्}


\twolineshloka
{वसिष्ठभृग्वत्रिसमैस्तापसैरुपशोभितम्}
{नियतैः संयताहारैर्दमशौचसमन्वितैः}


\twolineshloka
{अब्भक्षैर्वायुभक्षैश्च पत्राहारैस्तथैव च}
{जितेन्द्रियैर्महाभागैः स्वर्गमार्गदिदृक्षुभिः}


\twolineshloka
{वल्कलाजिनसंवीतैर्मुनिभिः संयतेन्द्रियैः}
{तापसाध्युषितं रम्यं ददर्शाश्रममण्डलम्}


\twolineshloka
{नानामृगगणैर्जुष्टं शाखामृगगणायुतम्}
{तापसैः समुपेतं च सा दृष्ट्वैव समाश्वसत्}


\twolineshloka
{सुभ्रूः सुकेशी सुश्रोणी सुकुचा सुद्विजानना}
{वर्चस्विनी सुप्रतिष्ठा स्वसितायतलोचना}


\twolineshloka
{सा विवेशाश्रमपदं वीरसेनसुतप्रिया}
{योषिद्रत्नं महाभागा दमयन्ती तपस्विनी}


\twolineshloka
{साऽभिवाद्य तपोवृद्धान्विनयावनता स्थिता}
{स्वागतं त इतिप्रोक्ता तैः सर्वैस्तापसोत्तमैः}


\twolineshloka
{पूजां चास्या यथान्यायं कृत्वा तत्रतपोधनाः}
{आस्यतामित्यथोचुस्ते ब्रूहि किं करवामहे}


\twolineshloka
{तानुवाच वरारोहा कच्चिद्भगवतामिह}
{तपःस्वग्निषु धर्मेषु मृगपक्षिषु चानघाः}


\twolineshloka
{कुशलं वो महाभागाः स्वधर्माचरणेषु च}
{तैरुक्ता कुशलं भद्रे सर्वत्रेति यशस्विनी}


\twolineshloka
{ब्रूहि सर्वानवद्याङ्गि का त्वं किंच चिकीर्षसि}
{दृष्ट्वैव ते परंरूपं द्युतिं च परमामिह}


\threelineshloka
{विस्मयो नः समुत्पन्नः समाश्वसिहि मा शुचः}
{अस्यारण्यस्य देवी त्वमुताहोऽस्य महीभृतः}
{अस्याश्च नद्याः कल्याणि वद सत्यमनिन्दिते}


\twolineshloka
{साऽब्रवीत्तानृपीन्नाहमरण्यस्यास्य देवता}
{न चाप्यस्य गिरर्विप्रा नैव नद्याश्च देवता}


\twolineshloka
{मानुषीं मां विजानीत यूयं सर्वेतपोधनाः}
{विस्तरेणाभिधास्यामि तन्मे शृणुत सर्वशः}


\twolineshloka
{विदर्भेपु महीपालो भीमो नाम महीपतिः}
{तस्य मां तनयां सर्वे जानीत द्विजसत्तमाः}


\twolineshloka
{निपधाधिपतिर्धीमान्नलो नाम महायशाः}
{वीरः संग्रामजिद्विद्वान्मम भर्ता विशांपतिः}


\twolineshloka
{देवताभ्यर्चनपरो द्विजातिजनवत्सलः}
{गोप्ता निपधवंशस्य महातेजा महाबलः}


\twolineshloka
{सत्यवान्धर्मवित्प्राज्ञः सत्यसंधोऽरिमर्दनः}
{ब्रह्मण्यो दैवतपरः श्रीमान्परपुरंजयः}


\twolineshloka
{नलो नाम नृपश्रेष्ठो देवराजसमद्युतिः}
{मम भर्ता विशालाक्षः पूर्णेन्दुवदनोऽरिहा}


\twolineshloka
{आहर्ता क्रतुमुख्यानां वेदवेदाङ्गपारगः}
{सपत्नानां मृधे हन्ता रविसोमसमप्रभः}


\threelineshloka
{स कैश्चिन्निकृतिप्रज्ञैरनार्यैरकृतात्मभिः}
{आहृय पृथिवीपालः सत्यधर्मपरायणः}
{देवनेकुशलैर्जिह्मैर्जितो राज्यं वसूनि च}


\twolineshloka
{तस्य मामवगच्छध्वं भार्यां राजर्पभस्य वै}
{दमयन्तीति विख्यातां भर्तुर्दर्शनलालसाम्}


\twolineshloka
{सा वनानि गिरींश्चैव सरांसि सरितस्तथा}
{पल्वलानि च सर्वाणि तथाऽरण्यानि सर्वशः}


\twolineshloka
{अन्वेपमाणआ भर्तारं नलं रणविशारदम्}
{महात्मानं कृतास्त्रं च विचरामीह दुःखिता}


\twolineshloka
{कच्चिद्भगवतां रम्यं तपोवनमिदं नृपः}
{भवेत्प्राप्तो नलो नाम निपधानां जनाधिपः}


\twolineshloka
{यत्कृतेऽहमिदं दुःखं प्रपन्ना भृशदारुणम्}
{वनं प्रतियं धोरं शार्दूलमृगसेवितम्}


\twolineshloka
{यदि कैश्चिदहोरात्रैर्न द्रक्ष्यामि नलं नृपम्}
{आत्मानं श्रेयसा योक्ष्ये देहस्यास्य विमोक्षणात्}


\twolineshloka
{कोनु मे जीवितेनार्थस्तमृते पुरुषर्षभम्}
{कथं भविष्याम्यद्याहं भर्तृशोकाभिपीडिता}


\twolineshloka
{तथा विलपतीमेकामरण्ये भीमनन्दिनीम्}
{दमयन्तीमथोचुस्ते तापसाः सत्यवादिनः}


\twolineshloka
{उदर्कस्तव कल्याणि कल्याणो भविता शुभे}
{वयं पश्याम तपसा क्षिप्रं द्रक्ष्यसि नैषधम्}


\twolineshloka
{निपधानामधिपतिं नलं रिपुनिपातिनम्}
{भैमि धर्मभृतां श्रेष्ठं द्रक्ष्यसे विगतज्वरम्}


\twolineshloka
{विमुक्तं सर्वपापेभ्यः सर्वरत्नसमन्वितम्}
{तदेव नगरं श्रेष्ठं प्रशासतमरिन्दमम्}


\twolineshloka
{द्विपतां भयकर्तारं सुहृदां शोकनाशनम्}
{पतिमेष्यसि कल्याणि कल्याणाभिजनं नृपम्}


\twolineshloka
{एवमुक्त्वा नलस्येष्टां महिषीं पार्थिवात्मजाम्}
{अन्तर्हितास्तापसास्ते साग्निहोत्राश्रमास्तथा}


\twolineshloka
{सा दृष्ट्वा महदाश्चर्यं विस्मिता ह्यभवत्तदा}
{दमयन्त्यनवद्याङ्गी वीरसेननृपस्नुषा}


% Check verse!
`चिन्तयामास वैदर्भी किमेतद्दृष्टवत्यहम्'
\twolineshloka
{किंनु स्वप्नो मया दृष्टः कोऽयंविधिरिहाभवत्}
{क्वनु ते तापसाः सर्वे क्व तदाश्रममण्डलम्}


\twolineshloka
{क्व सा पुण्यजला रम्या नदी द्विजनिषेविता}
{क्वनु ते ह नगा हृद्याः फलपुष्पोपशोभिताः}


\threelineshloka
{`इत्येवं नरशार्दूल विस्मिता कमलेक्षणा}
{'ध्यात्वा चिरं भीमसुता दमयन्ती शुचिस्मिता}
{भर्तृशोकपरा दीनाविवर्णवदनाऽभवत्}


\twolineshloka
{सा गत्वाऽथापरां भूमिं बाष्पसंदिग्धया गिरा}
{विललापाश्रुपूर्णाक्षी दृष्ट्वाऽशोकतरुं ततः}


\twolineshloka
{उपगम्य तरुश्रेष्ठमशोकं पुष्पितं वने}
{पल्लवापीडितं हृद्यं विहङ्गैरनुनादितम्}


\twolineshloka
{अहो बतायमगमः श्रीमानस्मिन्वनान्तरे}
{आपीडैर्बहुभिर्भाति श्रीमान्पर्वतराडिव}


\twolineshloka
{विशोकां कुरु मां क्षिप्रमशोक प्रियदर्शन}
{वीतशोकभयाबाधं कच्चित्त्वं दृष्टवान्नृपम्}


\twolineshloka
{नलं नामारिदमनं नमयन्त्याः प्रियं पतिम्}
{निमधानामधिपतिं दृष्टवानसि मे प्रियम्}


\twolineshloka
{एकवस्त्रार्धसंवीतं सुकुमारतनुत्वचम्}
{व्यसनेनार्दितं वीरमरण्यमिदमागतम्}


\twolineshloka
{यथा विशोका गच्छेयमशोकनग तत्कुरु}
{सत्यनामा भवाशोक मम शोकविनाशनात्}


\twolineshloka
{एवंसाऽशोकवृत्तं तमार्ता वै परिगम्य ह}
{जगाम दारुणतरं देशं भैमी वराङ्गना}


\twolineshloka
{सा ददर्श नागान्नैकान्नैकाश्च सरितस्तथा}
{नैकांश्च पर्वतान्रम्यान्नैकांश्च मृगपक्षिणः}


\twolineshloka
{कन्दरांश्च नितम्बांश्च नदीश्चाद्भुतदर्शनाः}
{ददर्श तान्भीमसुता पतिमन्वेषती तदा}


\twolineshloka
{गत्वा प्रकृष्टमध्वानं दमयन्ती शुचिस्मिता}
{ददर्शाथ महासार्थं हस्त्यश्वरथसंकुलम्}


\twolineshloka
{उत्तरन्तं नदीं रम्यां प्रसन्नसलिलां शुभाम्}
{सुशीततोयां विस्तीर्णां ह्रदिनीं वेतसैर्वृताम्}


\twolineshloka
{प्रोद्धुष्टां क्रौञ्चकुररैश्चक्रबाकोपकूजिताम्}
{कूर्मग्राहझवाकीर्णां विपुलद्वीपशोभिताम्}


\twolineshloka
{सा दृष्ट्वैव महासार्थं नलपत्नी यशस्विनी}
{उपसर्प्य वरारोहा जनमध्यं विवेश ह}


\twolineshloka
{उन्मत्तरूपा शोकार्ता तथा वस्त्रार्धसंवृता}
{कृशा विवर्णा मलिना पांसुध्वस्तशिरोरुहा}


\twolineshloka
{तां दृष्ट्वा तत्रमनुजाः केचिद्भीताः प्रदुद्रुवुः}
{केचिच्चिन्तापरास्तस्थुः केचित्तत्रविचुक्रुशुः}


\twolineshloka
{प्रहसन्ति स्म तां केचिदभ्यसूयन्ति चापरे}
{कुर्वन्त्यस्यांदयांकेचित्पप्रच्छुश्चापि भारत}


\twolineshloka
{काऽसि कस्यासि कल्याणि किं वा मृगयसे वने}
{त्वां दृष्ट्वा व्यथिताः स्मेह कच्चित्त्वमसिं मानुषी}


\twolineshloka
{वद सत्यं वनस्यास्य पर्वतस्याथवा दिशः}
{देवता त्वं हि कल्याणि त्वां वयं शरणं गताः}


\twolineshloka
{यक्षी वा राक्षसी वा त्वमुताहोसि सुराङ्गना}
{सर्वथाकुरु नः स्वस्ति रक्ष चास्माननिन्दिते}


\twolineshloka
{यथाऽयंसर्वथा सार्थः क्षेमी शीघ्रमितो व्रजेत्}
{तथा विधत्स्व कल्याणि यथा श्रेयो हि नो भवेत्}


\twolineshloka
{तथोक्ता तेन सार्थेन दमयन्ती नृपात्मजा}
{प्रत्युवाच ततः साध्वी भर्तृव्यसनपीडिता}


\twolineshloka
{सार्तवाहं च सार्थं च जना ये चात्र केचन}
{युवस्थविरबालाश् सार्थस्य च पुरोगमाः}


\twolineshloka
{मानुषीं मां विजानीत मनुजाधिपतेः सुताम्}
{नृपस्नुषां राजभार्यां भर्तृदर्शनलालसाम्}


\twolineshloka
{विदर्भराण्मम पिता भर्ता राजा च नैषधः}
{नलोनाम महाभागस्तं मार्गाम्यपराजितम्}


\twolineshloka
{यदि जानीत नृपतिं क्षिप्रं शंसत मे प्रियम्}
{नलं पुरुषशार्दूलममित्रगणसूदनम्}


\twolineshloka
{तामुवाचानवद्याङ्गीं सार्थस्य महतः प्रभुः}
{सार्थवाहः शुचिर्नाम शृणु कल्याणि मद्वचः}


\twolineshloka
{अहं सार्थस्य नेता वै सार्थवाहः शुचिस्मिते}
{मनुष्यं नलनामानं न पश्यामि यशस्विनि}


\twolineshloka
{कुञ्जरद्वीपिमहिषशार्दूलर्क्षमृगानपि}
{पश्याम्यस्मिन्वने कृत्स्ने ह्यमनुष्यनिषेविते}


\twolineshloka
{ऋते त्वां मानुषीं मर्त्यं न पश्यामि महावने}
{तथा नो यक्षराडद्य मणिभद्रः प्रसीदतु}


\threelineshloka
{साऽब्रवीद्वणिजः सर्वान्सार्थवाहं च तं ततः}
{क्वनु यास्यति सार्थोऽयमेतदाख्यातुमर्हसि ॥सार्तवाह उवाच}
{}


\twolineshloka
{सार्थोऽयं चेदिराजस्य सुबाहोः सत्यदर्शिनः}
{क्षिप्रं जनपदं गन्ता लाभाय मनुजात्मजे}


\chapter{अध्यायः ६२}
\twolineshloka
{बृहृदश्व उवाच}
{}


\twolineshloka
{सा तच्छ्रुत्वाऽनवद्याङ्गी सार्थवाहवचस्तदा}
{जगाम सह तेनैव सार्थेन पतिलालसा}


\twolineshloka
{अथ काले बहुतिथे वने महति दारुणे}
{तटाकं सर्वतोभद्रं पद्मसौगन्धिकायुतम्}


\twolineshloka
{ददृशुर्वणिजो रम्यं प्रभूतयवसेन्धनम्}
{बहुमूलफलोपेतं नानापक्षिनिषेवितम्}


\twolineshloka
{तं दृष्ट्वा मृष्टसलिलं मनोहारि सुशीतलम्}
{सुपरिश्रान्तवाहास्ते निवेशाय मनो दधुः}


\twolineshloka
{संमते सार्थवाहस्य विविशुर्वनमुत्तमम्}
{उवास सार्थः सुमहान्निशामासाद्य पद्मिनीम्}


\threelineshloka
{अथार्धरात्रसमये निःशब्दे तिमिरे तदा}
{सुप्ते सार्थे परिश्रान्ते हस्तियूथमुपागमत्}
{पानीयार्थं गिरितटान्मदप्रस्रवणाविलम्}


% Check verse!
[अथापश्यत सार्थं तं सार्थजान्सुबहून्गजान्
\twolineshloka
{ते तान्ग्राम्यगजान्दृष्ट्वा सर्वे वनगजास्तदा}
{समाद्रवन्त वेगेन जिघांसन्तो मदोत्कटाः}


\threelineshloka
{तेषामापततां वेगः करिणां दुःसहोऽभवत्}
{नगाग्रादिव शीर्णानां शृङ्गाणां पततां क्षितौ}
{स्पन्दतामपि नागानां मार्गा नष्टा वनोद्भवाः ॥]}


\twolineshloka
{मार्गं संरुद्ध्य संसुप्तं पद्मिन्याः सार्थमुत्तमम्}
{ते तं ममर्दुः सहसा वेष्टमानं महीतले}


\twolineshloka
{महारवं प्रमुञ्चतो वद्ध्यन्ते शरणार्थिनः}
{वनगुल्मांश्च धावन्तो निद्रया महतो भयात्}


% Check verse!
केचिद्दन्तैः करैः केचित्केचित्पद्भ्यां हता गजैः
\twolineshloka
{निहतोष्ट्राश्वबहुलाः पदातिजसंकुलाः}
{भयादाधावमानाश् परस्परहतास्तदा}


\twolineshloka
{घोरान्नादान्विमुञ्चन्तो निपेतुर्धरणीतले}
{वृक्षेष्वासज्यसंभग्नाः पतिता विषमेषु च}


\twolineshloka
{एवंप्रकारैर्बहुभिर्दैवेनाक्रम्य हस्तिभिः}
{राजन्विनिहतं सर्वं समृद्धं सार्थमण्डलम्}


\twolineshloka
{आरावः सुमहांश्चासीत्रैलोक्यभयकारकः}
{एषोऽग्निरुत्थितः कष्टस्त्रायध्वं धावताऽधुना}


\twolineshloka
{रत्नराशिर्विशीर्णोऽयं गृह्णीध्वं किं प्रधावत}
{सामान्यमेतद्द्रविणं न मिथ्या वचनं मम}


\twolineshloka
{पुनरेवाभिधास्यामि चिन्तयध्वं सुकातराः}
{एवमेवाभिभाषन्तो विद्रवन्ति भयात्तदा}


\threelineshloka
{तस्मिंस्तथा वर्तमाने दारुणे जनसंक्षये}
{दमयन्ती च बुबुधे भयसंत्रस्तमानसा}
{अपश्यद्वैशसं तत्र सर्वलोकभयंकरम्}


\twolineshloka
{अदृष्टपूर्वं तद्दृष्ट्वा बाला पद्मनिभेक्षणा}
{संसक्तवदनाश्वासा उत्तस्थौ भयविह्वला}


\twolineshloka
{ये तु तत्र विनिर्मुक्ताः सार्थात्केचिदविक्षताः}
{तेऽब्रुवन्सहिताः सर्वे कस्येदं कर्मणः फलम्}


\twolineshloka
{नूनं न पूजितोऽस्माभिर्मणिभद्रो महायशाः}
{तथा यक्षाधिपः श्रीमान्न वै वैश्रवणः प्रभुः}


% Check verse!
न पूजा विघ्रकर्तॄणामथवा प्रथमं कृता
\twolineshloka
{शकुनानां फलं वाऽथ विपरीतिमिदं ध्रुवम्}
{ग्रहा न विपरीतास्तु किमन्यदिदमागतम्}


\twolineshloka
{अपरे त्वब्रुवन्दीना ज्ञातिद्रव्यविनाकृता}
{याऽसावद्य महासार्थे नारी ह्युन्मत्तदर्शना}


\twolineshloka
{प्रविष्टा विकृताकारा कृत्वा रूपममानुषम्}
{तयैवं विहिता पूर्वं माया परमदारुणा}


\twolineshloka
{राक्षसी वा ध्रुवं यक्षी पिशाची वा भयंकरी}
{तस्याः सर्वमिदं पापं नात्र कार्या विचारणा}


\threelineshloka
{यदि पश्याम तां पापां सार्थघ्नीं नैकदुःखदाम्}
{लोष्टभिः पांसुभिश्चैव तृणैः काष्ठैश्च मुष्टिभिः}
{अवश्यमेव हन्यामः सार्थस्य किल कृत्यकाम्}


\twolineshloka
{दमयन्ती तु तच्छ्रुत्वा वाक्यं तेषां सुदारुणम्}
{ह्रीता भीता च संविग्रा प्राद्रवद्यत्र काननम्}


% Check verse!
आशङ्कमाना तत्पापमात्मानं पर्यदेवयत्
\twolineshloka
{अहो ममोपरि विधेः संरम्भो दारुणो महान्}
{नानुबध्नाति कुशलं कस्येदं कर्मणः फलम्}


\twolineshloka
{न स्मराम्यशुभं किंचित्कृतं कस्यचिदण्वपि}
{कर्मणा मनसा वाचा कस्येदं कर्मणः फलम्}


\twolineshloka
{नूनं जन्मान्तरकृतं पापां माऽऽपतितं महत्}
{अपश्चिमामिमां कष्टामापदं प्राप्तवत्यहम्}


\threelineshloka
{भर्तृराज्यापहरणं स्वजनाच्च रपराजयः}
{भर्त्रा सह वियोगश्च तनयाभ्यां च विच्युतिः}
{निर्नाथता वने वासो बहुव्यालनिषेविते}


\threelineshloka
{अथापरेद्युः संप्राप्ते हतशिष्टा जनास्तदा}
{वनगुल्माद्विनिष्क्रम्य शोचन्ते वैशसं कृतम्}
{भ्रातरं पितरं पुत्रं सखायं च नराधिप}


\twolineshloka
{`हन्यमाने तदा सार्थे दमयन्ती शुचिस्मिता}
{ब्राह्मणैः सहिता तत्रवने तु न विनाशिता ॥'}


\twolineshloka
{अशोचत्तत्र वैदर्भी किंनु मे दुष्कतं कृतम्}
{योपि मे निर्जनेऽरण्ये संप्राप्तोऽयं जनार्णवः}


\twolineshloka
{स हतो हस्तियूथेन मन्दभाग्यान्ममैव तत्}
{प्राप्तव्यं सुचिरं दुःखं नूनमद्यापि वै मया}


\twolineshloka
{नाप्राप्तकालो म्रियते श्रुतं वृद्धानुशासनम्}
{या नाहमद्य मृदिता हस्तियूथेन दुःखिता}


\twolineshloka
{न ह्यदैवकृतंकिंचिन्नराणामिह विद्यते}
{न च मे बालभावेऽपि किंचित्पापकृतं कृतम्}


\twolineshloka
{कर्मणा मनसा वाचा यदिदं दुःखमागतम्}
{मन्ये स्वयंवरकृतेलोकपाला समागताः}


\twolineshloka
{प्रत्याख्याता मया तत्र नलस्यार्थाय देवताः}
{नूनं तेषां प्रभावेन वियोगं प्राप्तवत्यहम्}


\twolineshloka
{एवमादीनि दुःखार्ता सा विलप्य वराङ्गना}
{प्रलापानि तदा तानि दमयन्ती पतिव्रता}


\twolineshloka
{हतशेषैः सह तदा ब्राह्मणैर्वेदपारगैः}
{अगच्छद्राजशार्दूल दुःखशोकपरायणा}


\twolineshloka
{गच्छन्ती सा चिराद्बाला पुरमासादयन्महत्}
{सायाह्नि चेदिराजस् सुबाहोः सत्यवादिनः}


\twolineshloka
{`सा तु तच्चारुसर्वाङ्गी सुबाहोस्तुङ्गगोपुरम्}
{'वस्त्रार्धेन च संवीता प्रविवेश पुरोत्तमम्}


\twolineshloka
{तां विह्वलां कृशां दीनां मुक्तकेशीममार्जिताम्}
{उन्मत्तामिव गच्छन्तीं ददृशुः पुरवासिनः}


\twolineshloka
{प्रविशन्तीं तु तां दृष्ट्वा चेदिराजपुरीं तदा}
{अनुजग्मुस्तत्र बाला ग्रामिपुत्राः कुतूहलात्}


\twolineshloka
{सा तैः परिवृताऽगच्छत्समीपं राजवेश्मनः}
{तां प्रासादगताऽपश्यद्राजमाता जनैर्वृताम्}


\twolineshloka
{धात्रीमुवाच च्छैनामानयेह ममान्तिकम्}
{जनेन क्लिश्यते बाला दुःखिता शरणार्थिनी}


\twolineshloka
{तादृग्रूपं च पश्यामि विद्योतयति मे गृहम्}
{उन्मत्तवेषा कल्याणी श्रीरिवायतलोचना}


\twolineshloka
{सा जनं वारयित्वा तं प्रासादतलमुत्तमम्}
{आरोप्य विस्मिता राजन्दमयन्तीमपृच्छत}


\twolineshloka
{एवमप्यसुखाविष्टा बिभर्षि परमं वपुः}
{भासि विद्युदिवाभ्रेषु शंस मे काऽसिकस्य वा}


\twolineshloka
{न हि ते मानुषं रूपं भूषणैरपि वर्जितम्}
{असहाया नरेभ्यश्च नोद्विजस्यमरप्रभे}


\twolineshloka
{तच्छ्रुत्वा वचनं तस्या भैमी वचनमब्रवीत्}
{मानुषीं मां विजानीहि भर्तारं समनुव्रताम्}


\twolineshloka
{सैरन्ध्रीं जातिसंपन्नां भुजिष्यां कामवासिनीम्}
{फलमूलाशनामेकां यत्रसायंप्रतिश्रयाम्}


\twolineshloka
{असङ्ख्येयगुणो भर्ता मां च नित्यमनुव्रतः}
{भक्ताऽहमपि तं वीरं छायेवानुगता पथि}


\twolineshloka
{तस्य दैवात्प्रसङ्गोऽभूदतिमात्रं सुदेवने}
{द्यूते स निर्जितश्चैव वनमेक उपेयिवान्}


\twolineshloka
{तमेकवसनच्छन्नमुन्म्तमिव विह्वलम्}
{आश्वासयन्ती भर्तारमहमप्यगमं वनम्}


\twolineshloka
{स कदाचिद्वने वीरः कस्मिंश्चित्कारणान्तरे}
{क्षुत्परीतस्तु विमना वासश्चैकं व्यसर्जयत्}


\twolineshloka
{तमेकवसना नग्नमुन्मत्तवदचेतसम्}
{अनुव्रजन्ती बहुला न स्वपामि निशाः सदा}


\twolineshloka
{ततो बहुतिथे काले सुप्तामुत्सृज्य मां क्वचित्}
{वाससोऽर्धं परिच्छिद्य त्यक्तवान्मामनागसम्}


\twolineshloka
{तं मार्गमाणा भर्तारं दह्यमाना दिवानिशम्}
{न विन्दाम्यमरप्रख्यं प्रियं प्राणेश्वरं प्रभुम्}


\twolineshloka
{`इत्युक्त्वा साऽनवद्याङ्गी राजमातरमप्युत}
{स्थिताऽश्रुपरिपूर्णाक्षी वेपमाना सुदुःखिता'}


\twolineshloka
{तामश्रुपरिपूर्णाक्षीं विलपन्तीं तथा बहु}
{राजमाताऽब्रवीदार्ता भैमीमार्तस्वरां स्वयम्}


\twolineshloka
{वस त्वमिह कल्याणि प्रीतिर्मे परमा त्वयि}
{मृगयिष्यन्ति ते भद्रे भर्तारं पुरुषा मम}


\twolineshloka
{अपि वा स्वयमागच्छेत्परिधावन्नितस्ततः}
{इहैव वसती भद्रे भर्तारमुपलप्स्यसे}


\twolineshloka
{राजमातुर्वचः श्रुत्वा दमयन्ती वचोऽब्रवीत्}
{समयेनोत्सहे वस्तुं त्वि वीरप्रजायिनि}


\twolineshloka
{उच्छिष्टं नैव भुञ्जीयां न कुर्यां पादधावनम्}
{न चाहं पुरुषानन्यान्प्रभाषेयं कथंचन}


\twolineshloka
{प्रार्थयेद्यदि मां कश्चिद्दण्ड्यस्ते स पुमान्भवेत्}
{वध्यश्च ते सकृन्मन्द इतिमे व्रतमाहितम्}


\threelineshloka
{भर्तुरन्वेषणार्थं तु पश्येयं ब्राह्मणानहम्}
{यद्येवमिह वत्स्यामि त्वत्सकाशे न संशयः}
{अतोऽन्यथा न मे वासो वर्तते हृदये क्वचित्}


\twolineshloka
{इत्युक्ता दमयन्त्या तु राजमातेदमब्रवीत्}
{सर्वमेतत्करिष्यामि दिष्ट्या ते व्रतमीदृशम्}


\twolineshloka
{एवमुक्त्वा ततो भैमीं राजमाता विशांपते}
{उवाचेदं दुहितरं सुनन्दां नाम भारत}


\threelineshloka
{सैरन्ध्रीमभिजानीष्व सुनन्दे देवरूपिणीम्}
{वयसा तुल्यतां प्राप्ता सखी तव भवत्वियम्}
{एतया सह मोदस्व निरुद्विग्रमना सदा}


\twolineshloka
{ततः परमसंहृष्टा सुनन्दा गृहणागमत्}
{दमयन्तीमुपादाय सखीभिः परिवारिता}


\twolineshloka
{सहसा न्यवसद्राजन्राजपुत्र्या सुनन्दया}
{चिन्तयन्ती नलं वीरमनिशं वामलोचना}


\chapter{अध्यायः ६३}
\twolineshloka
{बृहदश्व उवाच}
{}


\twolineshloka
{उत्सृज्य दमयन्तीं तु नलो राजा विशांपते}
{ददर्श दावं दह्यन्तं महान्तं गहने वने}


\twolineshloka
{तत्र सुश्राव शब्दंवै मध्ये भूतस्य कस्यचित्}
{अभिधाव नलेत्युच्चैः पुण्यश्लोकेति चासकृत्}


\twolineshloka
{मा भैरिति नलश्चोक्त्वा मध्यमग्नेः प्रविश्य तम्}
{दद्रश नागराजानं शयानं कुण्डलीकृतम्}


\twolineshloka
{स नागः प्रञ्जलिर्भूत्वा वेपमानो नलं तदा}
{उवाच मां विद्धिराजन्नागं कर्कोटकं नृप}


\twolineshloka
{मया प्रलब्धो ब्रह्मर्षिरनागाः सुमहातपाः}
{तेन मन्युपरीतेन शप्तोस्मि मनुजाधिप}


\twolineshloka
{तिष्ठ त्वं स्थावर इव यावदेति नलः क्वचित्}
{इतो नेतासि तत्स त्वं शापान्मोक्ष्यसि मत्कृतात्}


\twolineshloka
{तस्य शापान्न शक्नोमि पदाद्विचलितुं पदम्}
{उपदेक्ष्यामि ते श्रेयस्त्रातुमर्हति मां भवान्}


\twolineshloka
{सखा च ते भविष्यामि मत्समो नास्ति पन्नगः}
{लघुश्च ते भविष्यामि शीघ्रमादाय गच्छ माम्}


\twolineshloka
{एवमुक्त्वा स नागेन्द्रो बभूवाङ्गुष्ठमात्रकः}
{दं गृहीत्वा नलः प्रायाद्देशं दावविवर्जितम्}


\twolineshloka
{आकाशदेशमासाद्य विमुक्तं कृष्णवर्त्मना}
{उत्स्रष्टुकामं तं नागः पुनः कर्कोटकोऽब्रवीत्}


\twolineshloka
{पदानि गणयन्गच्छ स्वानि नैषध कानिचित्}
{तत्रतेऽहं महाबाहो श्रेयो धास्यामि यत्परम्}


\twolineshloka
{ततः संख्यातुमारब्धमदशद्दशमे पदे}
{तस्य दष्टस्य तद्रूपं क्षिप्रमन्तरधीयत}


\twolineshloka
{स दृष्ट्वा विस्मितस्तस्थावात्मानं विकृतं नलः}
{स्वरूपधारिणं नागं ददर्श स महीपतिः}


\twolineshloka
{ततः कर्कोटको नागः सान्त्वयन्नलमब्रवीत्}
{मया तेऽन्तर्हितं रूपं न त्वां विद्युर्जना इति}


\twolineshloka
{यत्कृते चासि निकृतो दुःखेन महता नल}
{विषेण स मदीयेन त्वयि दुःखं निवत्स्यति}


\twolineshloka
{विषेण संवृतैर्गात्रैर्यावत्त्वां न विमोक्ष्यति}
{तावत्तु त्वां महाराज क्लेशेऽस्मिन्स नियोक्ष्यति}


\twolineshloka
{अनागा येन निकृतस्त्वमनर्हो जनाधिप}
{क्रोधादसूययित्वा तं रक्षा मे भवतः कृता}


\twolineshloka
{न ते भयं नरव्याघ्र दंष्ट्रिभ्यः शत्रुतोपि वा}
{ब्रह्मवित्त्वं च भविता मत्प्रसादान्नराधिप}


\twolineshloka
{राजन्विषनिमित्ता च न ते पीडा भविष्यति}
{संग्रामेषु न राजेन्द्रशश्वज्जयमवाप्स्यसि}


\twolineshloka
{गच्छ राजन्नितः सूतो बाहुकोऽहमिति ब्रुवन्}
{समीपमृतुपर्णस्य स हि वेदाक्षनैपुणम्}


\twolineshloka
{अयोध्यां नगरीं रम्यामद्य वै निषधेश्वर}
{स तेऽक्षहृदयं दाता राजाऽश्वहृदयेन वै}


\twolineshloka
{इक्ष्वाकुकुलजः श्रीमान्मित्रं चैव भविष्यति}
{भविष्यसि यदाऽक्षज्ञः श्रेयसा योक्ष्यसे तदा}


\twolineshloka
{संयोक्ष्यसे स्वदारैस्त्वं मा स्म शोके मनः कृथाः}
{राज्येन तनयाभ्यां च सत्यमेतद्ब्रवीमि ते}


\twolineshloka
{स्वं रूपं च यदा द्रष्टुमिच्छेथास्त्वं नराधिप}
{संस्मर्तव्यस्तदा तेऽहंवासश्चदं विवासयेः}


\twolineshloka
{अनेन वाससाच्छन्नः स्वं रूपं प्रतिपत्स्यसे}
{इत्युक्त्वा प्रददौ तस्मै दिव्यं वासोयुगं तदा}


\twolineshloka
{एवं नलं च संदिश्य वासो दत्ताव च कौरव}
{नागराजस्ततो राजंस्तत्रैवान्तरधीयत}


\chapter{अध्यायः ६४}
\twolineshloka
{बृहदश्व उवाच}
{}


\twolineshloka
{तस्मिन्नन्तर्हिते नागे प्रययौ नैषधो नलः}
{ऋतुपर्णस्य नगरं प्राविशद्दशमेऽहनि}


\twolineshloka
{स राजानमुपातिष्ठद्बाहुकोऽहमिति ब्रुवन्}
{अश्वानां वाहने युक्तः पृथिव्यां नास्ति मत्समः}


\twolineshloka
{अर्थकृच्छ्रेषु चैवाहं प्रष्टव्यो नैपुणेषु च}
{अन्नसंस्कारमपि च जानाम्यन्यैर्विशेषतः}


\twolineshloka
{यानि शिल्पानि लोकेऽस्मिन्यच्चैवान्यत्सुदुष्करम्}
{सर्वं यतिष्ये तत्कर्तुमृतुपर्ण भरस्व माम्}


\twolineshloka
{`इत्युक्तः स नलेनाथ ऋतुपर्णो नराधिपः}
{उवाच सुप्रीतमनास्तं प्रेक्ष्य च महीपते'}


\twolineshloka
{वसबाहुक भद्रं ते सर्वमेतत्करिष्यसि}
{शीघ्रयाने सदा बुद्धिर्ध्रियते मे विशेषतः}


\twolineshloka
{स त्वमातिष्ठ योगं तं येन शीध्रां हया मम}
{भवेयुरश्वाध्यक्षोसि वेतनं ते शतंशतम्}


\threelineshloka
{त्वामुपस्थास्यतश्चैव नित्यं वार्ष्णेयजीवलौ}
{एताभ्यां रंस्यसे सार्धं वस वै मयि बाहुक ॥बृहदश्व उवाच}
{}


\twolineshloka
{एवमुक्तो नलस्तेन न्यवसत्तत्रपूजितः}
{ऋतुपर्णस्य नगरे सहवार्ष्णेयजीवलः}


\twolineshloka
{स वै तत्रावसद्राजा वैदर्भीमनुचिन्तयन्}
{सायंसायं सदा चेमं श्लोकमेकं जगाद ह}


\twolineshloka
{क्व नु सा क्षुत्पिपासार्ता श्रान्ता शेते तपस्विनी}
{स्मरन्ती तस्य मन्दस्य कं वाऽऽसाद्योपतिष्ठति}


\threelineshloka
{एवं ब्रुवन्तं राजानं निशायां जीवलोऽब्रवीत्}
{कामेनां शोचसे नित्यं श्रोतुमिच्छामि बाहुक}
{आयुष्मन्कस्य वा नारी यामेवमनुशोचसि}


\twolineshloka
{तमुवाच नलो राजा मन्दप्रज्ञस्य कस्यचित्}
{आसीद्बहुमता नारी तस्या दृढतरश्च सः}


\twolineshloka
{स वै केनचिदर्थेन तया मन्दो व्ययुज्यत}
{विप्रयुक्तः स मन्दात्मा भ्रमत्यसुखपीडितः}


\twolineshloka
{दह्यमानः स शोकेन दिवारात्रमतन्द्रितः}
{निशाकाले स्मरंस्तस्याः श्लोकमेकं स्म गायति}


\twolineshloka
{स विभ्रमन्महीं सर्वां क्वचिदासाद्य किंचन}
{वसत्यनर्हस्तद्दुःखं भूय एवानुसंस्मरन्}


\twolineshloka
{सा तु तं पुरुषं नारी कृच्छेऽप्यनुगता वने}
{त्यक्ता तेनाल्पभाग्येन दुष्करं यदि जीवति}


\twolineshloka
{एका बालाऽनभिज्ञा च मार्गमाणाऽतथोचिता}
{क्षुत्पिपासापरीताङ्गी दुष्करं यदि जीवति}


\twolineshloka
{श्वापदाचरिते नित्यं वने महति दारुणे}
{त्यक्ता तेनाल्पभाग्येन मन्दप्रज्ञेन मारिष}


\twolineshloka
{इत्येवं नैषधो राजा दमयन्तीमनुस्मरन्}
{अज्ञातवासं न्यवसद्राज्ञस्तस्य नवेशने}


\chapter{अध्यायः ६५}
\twolineshloka
{बृहद्श्व उवाच}
{}


\twolineshloka
{हृतराज्ये नले भीमः सभार्येऽदर्शनं गते}
{`चिन्तयामास बहुशः सहामात्यैर्नराधिपः}


\twolineshloka
{समाहूय द्विजान्सर्वानिदं वचनमब्रवीत्}
{अग्रहारं च दास्यामिग्रामं नगरसंमितम्}


\twolineshloka
{दमयन्तीं नलं चैव पर्यन्वेषति यो द्विजः}
{गवां शतसहस्राणि दाता तस्मै द्विजातये}


\twolineshloka
{इत्युक्त्वा ब्राह्मणान्सर्वान्समाहूय द्विजोत्तमान्}
{प्रस्थापयामास तदा तयोर्दर्शनकाङ्क्षया'}


\twolineshloka
{संदिदेश च तान्भीमो वसु दत्ता च पुष्कलम्}
{मृगयध्वं नलं चैव दमयन्तीं च मे सुताम्}


\twolineshloka
{अस्मिन्कर्मणि निष्पन्ने विज्ञाते निषधाधिपे}
{गवां सहस्रं दास्यामि यो वस्तावानयिष्यति}


\twolineshloka
{अग्रहारांश्च दास्यामि ग्रामं नगरसंमितम्}
{हिरण्यंच सुवर्णं च दासीदासं तथैव}


\twolineshloka
{न चेच्छक्याविहानेतुं दमयन्ती नलोऽपि वा}
{ज्ञातमात्रेऽपि दास्यामि गवां दशशतं धनम्}


\threelineshloka
{इत्युक्तास्ते ययुर्हृष्टा ब्राह्मणाः सर्वतो दिशम्}
{पुरराष्ट्राणि चिन्वन्तो नैषधं सह भार्यया}
{नैव क्वापि प्रपश्यन्ति नलं वा भीमपुत्रिकाम्}


\twolineshloka
{ततश्चेदिपुरीं रम्यां सुदेवो नाम वै द्विजः}
{विचिन्वानोऽथ वैदर्भीमपश्यद्राजवेश्मनि}


\twolineshloka
{तयैव राजमाता च ब्राह्मणान्पर्यवेषयत्}
{भोजनार्थे सुदेवोऽपि तत्रैव प्रविवेश ह}


\twolineshloka
{कृशां विवर्णां मलिनां भर्तृशोकपरायणाम्}
{पुण्याहवाचने राज्ञः सुनन्दासहितां स्थिताम्}


\twolineshloka
{मन्दं प्रख्यायमानेन रूपेणाप्रतिमेन ताम्}
{निबद्धां धूमजालेन प्रभामिव विभावसोः}


\threelineshloka
{तां समीक्ष्य विशालाक्षीमधिकं मलिनां कृशाम्}
{तर्कयामास भैमीति कारणैरुपपादयन् ॥सुदेव उवाच}
{}


\twolineshloka
{यथेयं मे पुरा दृष्टा तथारूपेयमङ्गना}
{कृतार्थोस्म्यद्य दृष्ट्वेमां लोककान्तामिव श्रियं}


\twolineshloka
{पूर्णचन्द्राननां श्यामां चारुवृत्तपयोधराम्}
{कुर्वन्तीं प्रभया देवीं सर्वावितिमिरा दिशः}


\twolineshloka
{चारुपद्मविशालाक्षीं मन्मथस्य रतीमिव}
{इष्टां समस्तलोकस्य पूर्णचन्द्रप्रभामिव}


\twolineshloka
{विदर्भसरसस्तस्माद्दैवदोषादिवोद्धृताम्}
{मलपङ्कानुलिप्ताङ्गीं प्रम्लानां नलिनीमिव}


\twolineshloka
{पौर्णमासीमिव निशां राहुग्रस्तनिशाकराम्}
{पतिशोकाकुलां दीनां कृशस्त्रोतां नदीमिव}


\twolineshloka
{विध्वस्तपर्णकमलां वित्रासितविहंगमाम्}
{हस्तिहस्तपरिक्लिष्टां व्याकुलामिव पद्मिनीम्}


\twolineshloka
{सुकुमारीं सुजाताङ्गीं रत्नगर्भगृहोचिताम्}
{दह्यमानामिवार्केण मृणालीमिव चोद्धृताम्}


\twolineshloka
{रूपौदार्यगुणोपेतां मण्डनार्हाममण्डिताम्}
{चन्द्रलेखामिव नवांव्योम्नि नीलाभ्रसंवृताम्}


\twolineshloka
{कामभोगैः प्रियैर्हीनां हीनां बन्धुजनेन च}
{देहं धारयतीं दीनं भर्तृदर्शनकाङ्क्षया}


\twolineshloka
{भर्ता नाम परं नार्या भूषणं भूषणैर्विना}
{एषा हि रहिता तेन शोभमाना न शोभते}


\twolineshloka
{दुष्करं कुरुतेऽत्यन्तं हीनो यदनया नलः}
{धारयत्यात्मनो देहं न शोकेनापि सीदति}


\twolineshloka
{इमामसितकेशान्तां शतपत्रायतेक्षणाम्}
{सुखार्हां दुःखितां दृष्ट्वा ममापि व्यथते मनः}


\twolineshloka
{कदा नु खलुः दुःखस्य पारं यास्यति वै शुभा}
{भर्तुः समागमात्साध्वीरोहिणी शशिनो यथा}


\twolineshloka
{अस्या नूनं पुनर्लाभान्नैषधः प्रीतिमेष्यति}
{राजा राज्यपरिभ्रष्टः पुनर्लब्ध्वेव मेदिनीम्}


\twolineshloka
{तुल्यशीलवयोयुक्तां तुल्याभिजनसंवृतम्}
{नैषधोऽर्हति वैदर्भीं तं चेयमसितेक्षणा}


\twolineshloka
{युक्तं तस्याप्रमेयस्य वीर्यसत्ववतो मया}
{समाश्वासयितुं भार्यां पतिदर्शनलालसाम्}


\threelineshloka
{अहमाश्वासयाम्येनां पूर्णचन्द्रनिभाननाम्}
{अदृष्टदुःखां दुःखार्तां ध्यानरोदनतत्पराम् ॥बृहदश्व उवाच}
{}


\twolineshloka
{एवं विमृश्य विविधैः कारणैर्लक्षणैश्च ताम्}
{उपगम्य ततो भैमीं सुदेवो ब्राह्मणोऽब्रवीत्}


\twolineshloka
{अहं सुदेवो वैदर्भि भ्रातुस्ते दयितः सखा}
{भीमस्य वचनाद्राज्ञस्त्वामन्वेष्टुमिहागतः}


\twolineshloka
{कुशली ते पिता राज्ञि जननी भ्रातरश्च ते}
{आयुष्मन्तौ कुशलिनौ तत्रस्थौ दारकौ च ते}


\threelineshloka
{त्वत्कृतेबन्धुवर्गाश्च गतसत्वा इवासते}
{अन्वेष्टारो ब्राह्मणाश्च भ्रमन्ति शतशो महीम् ॥बृहदश्व उवाच}
{}


\twolineshloka
{अभिज्ञाय सुदेवं तं दमयन्ती युधिष्ठिर}
{पर्यपृच्छत तान्सर्वान्क्रमेण सुहृदः स्वकान्}


\twolineshloka
{रुरोद च भृशं राजन्वैदर्भी शोककर्शिता}
{दृष्ट्वा सुदेवं सहसा भ्रातुरिष्टं द्विजोत्तमम्}


\twolineshloka
{ततो रुदन्तीं तां दृष्ट्वा सुनन्दा शोककर्शिता}
{सुदेवेन सहैकान्ते कथयन्तीं च भारत}


\twolineshloka
{जनित्र्याः प्रेषयामास सैरन्ध्री रुदते भृशम्}
{ब्राह्मणेन सहागम्य तांविद्धि यदि मन्यसे}


\twolineshloka
{अथ चेदिपतेर्माता राज्ञश्चान्तःपुरात्तदा}
{जगाम यत्रसा बाला ब्राह्मणेन सहाभवत्}


\twolineshloka
{ततः सुदेवमानाय्य राजमाता विशांपते}
{पप्रच्छ भार्या कस्येयं सुता वा कस्य भामिनी}


\twolineshloka
{कथं च नष्टा ज्ञातिभ्यो भर्तुर्वा वामलोचना}
{त्वया च विदिता विप्र कथमेवंगता सती}


\twolineshloka
{एतदिच्छाम्यहं श्रोतुं त्वत्तः सर्वमशेषतः}
{तत्त्वेन हि ममाचक्ष्व पृच्छन्त्यादेवरूपिणीम्}


\twolineshloka
{एवमुक्तस्तया राजन्सुदेवो द्विजसत्तमः}
{सुखोपविष्ट आचष्ट दमयन्त्या यथातथम्}


\chapter{अध्यायः ६६}
\twolineshloka
{सुदेव उवाच}
{}


\twolineshloka
{विदर्भराजो धर्मात्मा भीमो भीमपराक्रमः}
{सुतेयं तस्य कल्याणी दमयन्तीति विश्रुता}


\twolineshloka
{राजा तु नैषधो नाम वीरसेनसुतो नलः}
{भार्येयं तस्य कल्याणी पुण्यश्लोकस्य धीमतः}


\twolineshloka
{स द्यूतेन जितो भ्रात्रा हृतराज्यो महामनाः}
{दमयन्त्या गतः सार्धं न प्राज्ञायत कस्यचित्}


\twolineshloka
{ते वयं दमयन्त्यर्थे चरामः पृथिवीमिमाम्}
{सेयमासादिता बाला तव देवि निवेशने}


\twolineshloka
{अस्या रूपेण सदृशी मानुषी न हि विद्यते}
{अस्या ह्येष भ्रुवोर्मध्ये सहजः पिप्लुरुत्तमः}


\twolineshloka
{श्यामायाः पद्मसंकाशो लक्षितोऽन्तर्हितो मया}
{मलेन संवृतोह्यस्याश्छन्नोऽभ्रेणेव चन्द्रमाः}


\twolineshloka
{चिह्नभूतो विभूत्यर्थमयं धात्रा विनिर्मितः}
{प्रतिपत्कलुषेवेन्दोर्लेखा नातिविराजते}


\twolineshloka
{न चास्या नश्यते रूपं वपुर्मलसमाचितम्}
{असंस्कृतमपि व्यक्तं भाति काञ्चनसन्निभम्}


\threelineshloka
{अनेन वपुषा बाला पिप्लुनाऽनेन सूचिता}
{लक्षितेयं मया देवी पिहितोऽग्निरिवोष्मणा ॥बृहदश्व उवाच}
{}


\twolineshloka
{तच्छ्रुत्वा वचनं तस्य सुदेवस्य विशांपते}
{सुनन्दा शोधयामास पिप्लुप्रच्छादनं मलम्}


\twolineshloka
{स मलेनापकृष्टन पिप्लुस्तस्या व्यरोचत}
{दमयन्त्या यथा व्यभ्रे नभसीव निशाकरः}


\twolineshloka
{पिप्लुं दृष्ट्वा सुनन्दा च राजमाता च भारत}
{रुदन्त्यौ तां परिष्वज्यमुहूर्तमिव तस्थतुः}


\twolineshloka
{उत्सृज्य बाष्पं शनकै राजमातेदमब्रवीत्}
{भगिन्या दुहिता मेऽसि पिप्लुनानेन सूचिता}


\twolineshloka
{अहं च तव माता च राजन्यस्य महात्मनः}
{सुते दशार्णाधिपतेः सुदाम्नश्चारुदर्शने}


\twolineshloka
{भीमस्य राज्ञः सा दत्ता वीरबाहोरहं पुनः}
{त्वं तु जाता मया दृष्टा दशार्णेषु पितुर्गृहे}


\twolineshloka
{यथैव ते पितुर्गेहं तथेदमपि भामिति}
{यथैव च ममैश्वर्यं दमयन्ति तथा तव}


\twolineshloka
{तां प्रहृष्टेन मनसा दमयन्ती विशांपते}
{प्रणम्य मातुर्भगिनीमिदं वचनमब्रवीत्}


\twolineshloka
{अज्ञायमानाऽपिसती सुखमस्म्युषिता त्वयि}
{सर्वकामैः सुविहिता रक्ष्यमाणा सदा त्वया}


\twolineshloka
{सुखात्सुखतरो वासो भविष्यति न संशयः}
{चिरविप्रोषितां मातर्मामनुज्ञातुमर्हसि}


\twolineshloka
{दारकौ च हि मे नीतौ वसतस्तत्र बालकौ}
{पित्रा विहीनौ शोकार्तौ मया चैव कथं नु तौ}


\twolineshloka
{यदि चापि प्रियं किंचिन्मयि कर्तुमिहेच्छसि}
{विदर्भान्यातुमिच्छामि शीघ्रं मे यानमादिश}


\twolineshloka
{बाढमित्येव तामुक्त्वा हृष्ट्वा मातृष्वसा नृप}
{गुप्तां बलेन महता पुत्रस्यानुमते ततः}


\twolineshloka
{प्रास्थापयद्राजमाता श्रीमतीं नरवाहिना}
{यानेन भरतश्रेष्ठ स्वन्नपानपरिच्छदाम्}


\twolineshloka
{ततः सा नचिरादेव विदर्भानगमच्छुभा}
{तां तु बन्धुजनः सर्वः प्रहृष्टः समपूजयत्}


\twolineshloka
{सर्वान्कुशलिनो दृष्ट्वा बान्धवान्दारकौ च तौ}
{मातरं पितरं चोभौ सर्वं चैव सखीजनम्}


\twolineshloka
{देवताः पूजयामास ब्राह्मणांश्च यशस्विनी}
{परेण विधिना देवी दमयन्ती विशांपते}


\twolineshloka
{अतर्पयत्सुदेवं च गोसहस्रेण पार्थिवः}
{प्रीतो दृष्ट्वैव तनयां ग्रामेण द्रविणेन च}


\twolineshloka
{सा व्युष्टा रजनीं तत्रपितुर्वेश्मनि भामिनी}
{विश्रान्ता मातरं राजन्निदं वचनमब्रवीत्}


\chapter{अध्यायः ६७}
\twolineshloka
{दमयन्त्यवाच}
{}


\twolineshloka
{मां चेदिच्छसि जीवन्तीं मातः सत्यं ब्रवीमिते}
{नलस्य नरवीरस्य यतस्वानयने पुनः}


\twolineshloka
{दमयन्त्या तथोक्ता तु सा देवी भृशदुःखिता}
{बाष्पेणापिहिता राज्ञी नोत्तरं किंचिदब्रवीत्}


\twolineshloka
{तदवस्थां तु तां दृष्ट्वा सर्वमन्तःपुरं तदा}
{हाहाभूतमतीवासीद्भृशं च प्ररुरोद ह}


\twolineshloka
{ततो भीमं महाराजं भार्या वचनामब्रवीत्}
{दमयन्ती नृप भृशं भर्तारमनुशोचति}


\threelineshloka
{अपकृष्य च लज्जां सा स्वयमुक्तवती विभो}
{प्रयतन्तु तवप्रेष्याः पुण्यश्लोकस्य दर्शने ॥बृहदश्व उवाच}
{}


\twolineshloka
{तयाप्रयोदितो राजा ब्राह्मणान्वशवर्तिनः}
{प्राश्थापयद्दिशः सर्वा यतध्वं नलदर्शने}


\twolineshloka
{ततो विदर्भाधिपतेर्नियोगाद्ब्राह्मणास्तदा}
{दमयन्तीमथापृच्छ्य प्रस्थितास्ते तथाऽव्रुवन्}


\twolineshloka
{अथ तानब्रवीद्भैमी सर्वराष्ट्रेष्विदं वचः}
{ब्रूत वै जनसंसत्सु तत्रतत्र पुनः पुनः}


\twolineshloka
{क्वनु त्वं कितव च्छित्त्वा वस्त्रार्धं प्रस्थितो मम}
{उत्सृज्य वपिने सुप्तामनुरक्तां प्रियां प्रिय}


\twolineshloka
{सा वै यथा त्वया दृष्टा तथाऽऽस्ते त्वत्प्रतीक्षिणी}
{दह्यमाना भृशं बाला वस्त्रार्धेनाभिसंवृता}


\twolineshloka
{तस्या रुदन्त्याः सततं तेन शोकेन पार्थिव}
{प्रसादं कुरु वै देव प्रतिवाक्यं वदस्व च}


\twolineshloka
{एवमन्यच्च वक्तव्यं कृपां कुर्याद्यथा मयि}
{वायुना धूयमानो हि वनं दहति पावकः}


\twolineshloka
{भर्तव्या रक्षणीया त्व पत्नी पत्या हि नित्यदा}
{तन्नष्टमुभयं कस्माद्धर्मज्ञस्य सतस्तव}


\twolineshloka
{ख्यातः प्राज्ञः कुलीनश् सानुक्रोशो भवान्सदा}
{संवृत्तो निरनुक्रोशः शङ्के मद्भाग्यसंक्षयात्}


\twolineshloka
{तत्कुरुष्व नरव्याघ्र दयां मयि नरर्पभ}
{आनृशंस्यं परो धर्मस्त्वत्त एव हि मे श्रुतः}


\twolineshloka
{एवंब्रुवाणान्यदि वः प्रतिब्रूयाद्धि कश्चन}
{स नरः सर्वथा ज्ञेयः कश्चासौ क्वनु वर्तते}


\twolineshloka
{यश्चैवं वचनं श्रुत्य्वा बूयात्प्रतिवचो नरः}
{तदादाय वचस्तस्य ममावेद्यं द्विजोत्तमाः}


\twolineshloka
{यथा च वो न जानीयाच्चरतो भीमशासनात्}
{पुनरागमनं चेह तथा कार्यमतन्द्रितैः}


\twolineshloka
{यदि वाऽसौ समृद्धः स्याद्यदि वाऽप्यधनो भवेत्}
{यदिऽवाप्यसमर्थः स्याज्ज्ञेयमस्य चिकीर्षितम्}


\twolineshloka
{एवमुक्तास्त्वगच्छंस्ते ब्राह्मणाः सर्वतो दिशम्}
{नलं मृगयितुं राजंस्तदा व्यसनिनं तथा}


\twolineshloka
{ते पुराणि सराष्ट्राणि ग्रामान्धोपांस्तथाऽऽश्रमान्}
{अन्वेषन्तो नलं राजन्नाधिजग्मुर्दविजातयः}


\twolineshloka
{तच्च वाक्यं तथा सर्वेतत्रतत्र विशांपते}
{श्रावयांचक्रिरे विप्रा दमयन्त्या यथेरितम्}


\chapter{अध्यायः ६८}
\twolineshloka
{बृहदश्व उवाच}
{}


\twolineshloka
{अथ दीर्घस्य कालस्य पर्णादो नाम वै द्विजः}
{प्रत्येत्य नगरं भैमीमिदं वचनमब्रवीत्}


\twolineshloka
{नैषधं मृगयाणेन दमयन्ति दिवानिशम्}
{अयोध्यां नगरीं गत्वा भागस्वरिरुपस्थितः}


\twolineshloka
{श्रावितश्च मया वाक्यं त्वदीयं स महाजने}
{ऋतुपर्णो महाभागो यथोक्तं वरवर्णिनि}


\twolineshloka
{तच्छ्रुत्वा नाब्रवीत्किंचिदृतुपर्णो नराधिपः}
{न च पारिषदः कश्चिद्भाष्यमाणो मयाऽसकृत्}


\twolineshloka
{अनुज्ञातं तु मां राज्ञा विजने कश्चिदब्रवीत्}
{ऋतुपर्णस्य पुरुषो बाहुको नाम नामतः}


\twolineshloka
{सूतस्तस्य नरेन्द्रस्य विरूपो ह्रस्वबाहुकः}
{शीघ्रयानेषु कुशलो मृष्टकर्ता च भोजने}


\twolineshloka
{स विनिःश्वस्य बंहुशो रुदित्वा च षुनःपुनः}
{कुशलं चैव मां पृष्ट्वा पश्चादिदमभाषत}


\twolineshloka
{वैषम्यमपि संप्राप्ता गोपायन्ति कुलस्त्रियः}
{आत्मानमात्मना सत्यो जितः स्वर्गो न संशयः}


\twolineshloka
{रहिता भर्तृभिश्चैव न कुप्यन्ति कदाचन}
{3-68-9bप्राणांश्चारित्रकवचान्धारयन्ति वरस्त्रियः}


\twolineshloka
{विषमस्थेन मूढेन परिभ्रष्टसुखेन च}
{यत्सा तेन परित्यक्ता तत्र न क्रोद्धुमर्हति}


\twolineshloka
{प्राणयात्रां परिप्रेप्सोः शकुनैर्हृतवाससः}
{आधिभिर्दह्यमानस्य श्यामा न क्रोद्धुमर्हति}


\twolineshloka
{सत्कृताऽसत्कृता वाऽपि पतिं दृष्ट्वा तथागतम्}
{भ्रष्टराज्यं श्रित्या हीनं श्यामा न क्रोद्धुमर्हति}


\twolineshloka
{तस्य तद्वचनं श्रुत्वा त्वरितोऽहमिहागतः}
{श्रुत्वा प्रमाणं भवती राज्ञश्चैव निवेदयः}


\twolineshloka
{एतच्छ्रुत्वाऽश्रुपूर्णाक्षी पर्णादस्य विशांपते}
{दमयन्ती रहोऽभ्येत्य मातरं प्रत्यभापत}


\twolineshloka
{अयमर्थौ न संवेद्यो भीमे मातः कदाचन}
{त्वत्सन्निधौ नियोक्ष्येऽहंसुदेवं द्विजसत्तमम्}


\twolineshloka
{यथा न नृपतिर्भीमः प्रतिपद्येत मे मतम्}
{यथा त्वया प्रकर्तव्यं मम चेत्प्रियमिच्छसि}


\twolineshloka
{यथा चाहं समानीता सुदेवेनाशु बान्धवान्}
{तेनैव मङ्गलेनाद्य सुदेवो यातु मा चिरम्}


\twolineshloka
{समानेतुं नलं मातरयोध्यां नगरीमितः}
{`ऋतुपर्णस्य नगरे निवसन्तमरिन्दमम्'}


\twolineshloka
{विश्रान्तं तु ततः पश्चात्पर्णादं द्विजसत्तमम्}
{अर्चयामास वैदर्भी धनेनातीव भामिनी}


\twolineshloka
{`लवाच चैनं महता संपूज्य द्रविणेन वै}
{'नले चेहागते विप्र भूयो दास्याभि ते वसु}


\twolineshloka
{त्वया हि मे बहुकृतंयदन्यो न करिष्यति}
{यद्भर्त्राऽहं समेष्यामि शीघ्रमेव द्विजोत्तम}


\twolineshloka
{स एवमुक्तोऽथाश्वास्य आशीर्वादैः सुमङ्गलैः}
{गृहानुपययौ चापि कृतार्थः सुमहामनाः}


\twolineshloka
{ततः सुदेवमानाय्य दमयन्ती युधिष्ठिर}
{अब्रवीत्सन्निदौ मातुर्दुःखशोकसमन्विता}


\twolineshloka
{गत्वा सुदेव नगरीमयोध्यावासिनं नृपम्}
{ऋतुपर्णं वचो ब्रूहि पतिमन्यं चिकीर्षती}


\twolineshloka
{आस्थास्यति पुनर्भैमी दमयन्ती स्वयंवरम्}
{तत्र गच्छन्ति राजानो राजपुत्राश्च सर्वशः}


\twolineshloka
{तथा च गणितः कालः श्वोभूते स भविष्यति}
{यदि संभाविनीयं ते गच्छ शीघ्रमरिंदम}


\twolineshloka
{सूर्योदये द्वतीयं सा भर्तारं वरयिष्यति}
{न हि स ज्ञायते वीरो नलो जीवन्मृतोपि वा}


\twolineshloka
{एवं तथा यथोक्तो वै गत्वा राजानमब्रवीत्}
{ऋतुपर्णं महाराज सुदेवो ब्राह्मणस्तदा}


\chapter{अध्यायः ६९}
\twolineshloka
{बृहदश्व उवाच}
{}


\threelineshloka
{श्रुत्वा वचः सुदेवस्य ऋतुपर्णो नराधिपः}
{`सारथीन्स समानीय वार्ष्णेयप्रभृतीन्नृपः}
{कथयामास यद्वृत्तं ब्राह्मणेन श्रुतं तथा}


\twolineshloka
{बाहुकं च समाहूय दमयन्त्याः स्वयंवरम्'}
{सान्त्यञ्श्र्लक्ष्णया वाचा बाहुकं प्रत्यभाषत}


\twolineshloka
{विदर्भान्यातुमिच्छामि दमयन्त्याः स्वयंवरम्}
{एकाह्ना हयतत्त्वज्ञ मन्यसे यदि बाहुक}


\twolineshloka
{एवमुक्तस्य कौन्तेय तेन राज्ञा नलस्य ह}
{व्यदीर्यत मनो दुःखात्प्रदध्यौ च महामनाः}


\twolineshloka
{दमयन्ती भवेदेवं किंनु दुःखेन मोहिता}
{अस्मदर्थे भवेद्बाऽयमुपायश्चिन्तितो महान्}


\threelineshloka
{नृशंसं बत वैदर्भी कर्तुकामा तपस्विनी}
{यया क्षुद्रेण निकृताकृपणा पापबुद्धिना}
{स्त्रीस्वभावश्चलो लोके मम दोषश्च दारुणः}


\twolineshloka
{मम शोकेन संविग्ना नैराश्यात्तनुमध्यमा}
{नैवं सा कर्हिचित्कुर्यात्सापत्या च विशेषतः}


\twolineshloka
{यदत्र सत्यं वाऽसत्यं गत्वा वेत्स्यामि निश्चयम्}
{ऋतुपर्णस्य वै काममात्मार्थं च करोम्यहम्}


\twolineshloka
{इति निश्चित्य मनसा बाहुको दीनमानसः}
{कृतालुरुवाचेदमृतुपर्णं जनाधिपम्}


\threelineshloka
{प्रतिजानामि ते वाक्यं गमिष्यामि नराधिप}
{एकाह्ना पुरुषव्याघ्र विदर्भनगरीं नृप}
{`तत्रादित्योदये काले श्वो विदर्भान्गमिष्यसि}


\threelineshloka
{एवमुक्तोऽब्रवीद्राजा बाहुकं प्रहसन्निव}
{किं ते कामं करोम्यद्य तुष्टोऽस्मि तव बाहुक ॥बाहुक उवाच}
{}


% Check verse!
यावद्यानमिदं सज्जमृतुपर्ण करोम्यहम्'
\twolineshloka
{ततः परीक्षामश्वानां चक्रे राजन्स बाहुकः}
{अश्वशालामुपागम्य भागस्वरिनृपाज्ञया}


\threelineshloka
{स त्वर्यमाणो बहुश ऋतुपर्णन बाहुकः}
{अश्वाञ्जिज्ञासमानो वै विचार्य च पुनःपुनः}
{अध्यगच्छत्कृशानश्वान्समर्थानध्वनि क्षमान्}


\twolineshloka
{तेजोबलसमायुक्तान्कुलशीलसमन्वितान्}
{वर्जिताँल्लक्षणैर्हीनैः पृथुप्रोथान्महाहनून्}


\twolineshloka
{शुद्धान्दशभिरावर्तैः सिन्धुजान्वातरंहसः}
{`दृश्यमानान्कृशानङ्गैर्जवेनाप्रतिमान्पथि'}


\twolineshloka
{तान्दृष्ट्वा दुर्बलान्नाजा प्राह कोपसमन्वितः}
{किमिदं प्रार्थितं कर्तुं प्रलब्धव्या न ते वचम्}


\threelineshloka
{कथमल्पबलप्राणा वक्ष्यन्तीमे हया रथम्}
{महानध्वा स चैकाह्ना गन्तव्यः कथमीदृशैः ॥बाहुक उवाच}
{}


\twolineshloka
{[एको ललाटे द्वे मूर्ध्नि द्वौद्वौ पार्श्वोपपार्श्वयोः}
{द्वौद्वौ वक्षसि विज्ञैर्यौ प्रयाणे चैक एव तु ॥]}


\threelineshloka
{एते हया गमिष्यन्ति विदर्भान्नात्र संशयः}
{यानन्यान्मन्यसे राजन्ब्रूहि तान्योजयामिते ॥ऋतुपर्ण उवाच}
{}


\twolineshloka
{त्वमेव हयतत्त्वज्ञः कुशलो ह्यसि बाहुक}
{यान्मन्यसे समर्थास्त्वं क्षिप्रं तानेव योजय}


\twolineshloka
{ततः सदश्वांश्चतुरः कुलशीलसमन्वितान्}
{योजयामास कुशलो जवयुक्तान्रथे नलः}


\twolineshloka
{ततो युक्तं रथं राजा समारोहत्त्वरान्वितः}
{अथ पर्यपतन्भूमौ जानुभिस्ते हयोत्तमाः}


\twolineshloka
{ततो नरवरः श्रीमान्नलो राजा विशांपते}
{सान्त्वयामास तानश्वांस्तेजोबलसमन्वितान्}


\twolineshloka
{रश्मिभिश्च समुद्यम्य नलो यातुमियेष सः}
{सूतमारोप्य वार्ष्णेयं जवमास्थाय वै परम्}


\twolineshloka
{ते चोद्यमाना विधिवद्बाहुकेन हयोत्तमाः}
{समुत्पेतुरिवाकाशं रथिनं मोहयन्ति च}


\twolineshloka
{तथा तु दृष्ट्वा तानश्वान्वहतो वातरंहसः}
{अयोध्याधिपतिः श्रीमान्विस्मयं परमं ययौ}


\twolineshloka
{रथघोषं तु तं श्रुत्वा हयसंग्रहणं च तत्}
{वार्ष्णेयश्चिन्तयामास बाहुकस्य हयज्ञताम्}


\twolineshloka
{किंनु स्यान्मातलिरयं देवराजस्य सारथिः}
{तथा तल्लक्षणं वीरे बाहुके दृश्यते महत्}


\twolineshloka
{शालिहोत्रोऽथ किंतु स्याद्धयानां कुलतत्त्ववित्}
{मानुपं समनुप्राप्तो वपुः परमशोभनम्}


\twolineshloka
{उताहोस्विद्भवेद्राजा नलः परपुरंजयः}
{सोयं नृपतिरायात इत्येवं समचिन्तयत्}


\twolineshloka
{अथवाऽयंनलात्प्राप्तो विद्यां तामेव बाहुकः}
{तुल्यं हि लक्षये ज्ञानं बाहुकस्य नलस्य च}


\twolineshloka
{अपिचेदं वयस्तुल्यं बाहुकस्य नलस्य च}
{नायं नलो महावीर्यस्तद्विद्यश्च भविष्यति}


\twolineshloka
{प्रच्छन्ना हि महात्मानश्चरन्ति पृथिवीमिमाम्}
{दैवेन विधिना युक्ताः शास्त्रोक्तैश्च निरूपणैः}


\twolineshloka
{भवेन्न मतिभेदो मे गात्रवैरूप्यता प्रति}
{प्रमाणात्परिहीनस्तु भवेदिति मतिर्मम}


\twolineshloka
{वयःप्रमाणं तत्तुल्यं रूपेण तु विपर्ययः}
{नलं सर्वगुणैर्युक्तं मन्ये बाहुकमन्ततः}


\twolineshloka
{एवं विचार्य बहुशो वार्ष्णोयः पर्यचिन्तयत्}
{हृदयेन महाराज पुण्यश्लोकस्य सारथिः}


\twolineshloka
{ऋतुपर्णश्च राजेनद्रो बाहुकस्य हयज्ञताम्}
{चिन्तयन्मुमुदे राजा सहवार्ष्णेयसारथिः}


\twolineshloka
{ऐकाग्र्यं च तथोत्साहं हयसंग्रहणं च तत्}
{कौशलं चापि संप्रेक्ष्य परां सुदमवाप ह}


\chapter{अध्यायः ७०}
\twolineshloka
{बृहदश्व उवाच}
{}


\twolineshloka
{स नदीः पर्वतांश्चैव वनानि च सरांसि च}
{अचिरेणातिचक्राम स्वेचरः खे चरन्निव}


\twolineshloka
{तथा प्रयाते तु रथे तदा भागस्वरिर्नृपः}
{उत्तरीयमधोपश्यद्धष्टं परपुरंजयः}


\twolineshloka
{ततः स त्वरमाणस्तु पटे निपतिते तदा}
{ग्रहीष्यामीति तं राजा नलमाह महामनाः}


\twolineshloka
{निगृह्णीष्व महाबुद्धे हयानेतान्महाजवान्}
{वार्ष्णेयो यावदेनं मे पटमानयतामिह}


\twolineshloka
{नलस्तं प्रत्युवाचाथ दूरे भ्रष्टः पटस्तव}
{योजनं समतिक्रान्तो नाहर्तुं शक्यते पुनः}


\twolineshloka
{एवमुक्ते नलेनाथ नातिप्रीतमना नृपः}
{आससाद वने राजन्फलवन्तं बिभीतकम्}


\twolineshloka
{तं दृष्ट्वा बाहुकं राजा त्वरमाणोऽभ्यभापत}
{ममापि सूत पश्य त्वं संख्याने परमं बलम्}


\twolineshloka
{सर्वः सर्वं न जानाति सर्वज्ञो नास्ति कश्चन}
{नैकत्र परिनिष्ठाऽस्ति ज्ञानस्य पुरुषे क्वचित्}


\twolineshloka
{वृक्षेऽस्मिन्यानि पर्णानि फलामेकं च बाहुक}
{पतितान्यपि यान्यत्रतत्रैकमधिकं कृतम्}


\threelineshloka
{एवपत्राधिकं चात्र फलमेकं च बाहुक}
{फलकोट्यपि पत्राणां द्वयोरपि च शाखयोः}
{`प्रवक्ष्यामि फलान्यत्र यानि संख्यास्यते भवान् ॥'}


\twolineshloka
{प्रचिनुह्यस्य शाखे द्वे याश्चाप्यन्याः प्रशाखिकाः}
{आभ्यां फलसहस्रे द्वे पञ्चोनं शतमेव च}


\twolineshloka
{ततो रथादवप्लुत्य राजानं बाहुकोऽब्रवीत्}
{परोक्षमिव मे राजन्कत्थसे शत्रुकर्शन}


\twolineshloka
{प्रत्यक्षमेतत्कर्ताऽस्मि शातयित्वा विभीतकम्}
{अथ ते गणिते राजन्द्विजामाम्यपरोक्षताम्}


\twolineshloka
{प्रत्यक्षं ते महाराज् गणयिष्ये विभीतकम्}
{अहं हि नाभिजानामि भवेदेव नवेति वा}


\twolineshloka
{संख्यास्यामि फलान्यस्य पशय्तस्ते जनाधिप}
{मुहूर्तमपि वार्ष्णेयो रश्मीन्यच्छतु वाजिनाम्}


\twolineshloka
{तमब्रवीन्नृपः सूतं नायं कालो विलम्बितुम्}
{बाहुकस्त्वब्रवीदेनं परं यत्नं समास्थितः}


\twolineshloka
{प्रतीक्षस्व मुहूर्तं त्वमथवा त्वरते भवान्}
{एष याति शिवः पन्था याहि वार्ष्णेयसारथिः}


\twolineshloka
{अब्रवीदृतुपर्णस्तं सान्त्वयन्कुरुनन्दन}
{त्वभेव यन्ता नान्योस्ति पृथिव्यामपि बाहुक}


\twolineshloka
{त्वत्कृतेयातुमिच्छामि विदर्भान्हयकोविद}
{शरणं त्वांप्रपन्नोस्मि न विघ्नं कर्तुमर्हसि}


\threelineshloka
{कामं च ते करिष्यामि यन्मां वक्ष्यसि बाहुक}
{विदर्भान्यदि यात्वाऽद्य सूर्यं दर्शयितासि मे}
{}


\twolineshloka
{अथाब्रवीद्बाहुकस्तं संख्यायच बिभीतकम्}
{ततो विदर्भान्यास्यामि कुरुष्वैवं वचो मम}


\twolineshloka
{अकाम इव तं राजा गणयस्वेत्युवाच ह}
{एकदेशं च शाखायाः समादिष्टं मयाऽनघ}


\twolineshloka
{गणयस्वाश्वतत्वज्ञ ततस्त्वंप्रीतिमावह}
{सोऽवतीर्य रथात्तूर्णं शातयामास तं द्रुमम्}


\twolineshloka
{ततः स विस्मयाविष्टो राजानमिदमब्रवीत्}
{गणयित्वायथोक्तानि तावन्त्येव फलानि तु}


\twolineshloka
{अत्युद्भुतमिदं राजन्दृष्टवानस्मि ते बलम्}
{श्रोतुमिच्छामि तां विद्यां ययैतज्ज्ञायते नृप}


\twolineshloka
{तमुवाच ततो राजा त्वरितो गमने नृप}
{विद्ध्यक्षहृदयज्ञं मां संख्याने च विशारदम्}


\twolineshloka
{बाहुकस्तमुवाचाथ देहि विद्याद्वयं च मे}
{मत्तोऽपि चाश्वहृदयं गृहाण पुरुपर्पभ}


\twolineshloka
{ऋतुपर्णस्ततो राजा बाहुकं कार्यगौरवात्}
{हयज्ञानस्य लोभाच् तं तथेत्यब्रवीद्वचः}


\threelineshloka
{यथोक्तं त्वं गृहाणेदमक्षाणां हृदयं परम्}
{निक्षेपो मेऽश्वहृदयं त्वयि तिष्ठतु बाहुक}
{एवमुक्त्वा ददौ विद्यामृतुपर्णो नलाय वै}


\threelineshloka
{तस्याक्षहृदयज्ञस्य शरीरान्निःसृतः कलिः}
{कर्कोटकविषं तीक्ष्णं मुखात्सततमुद्वमन्}
{कलेस्तस्य तदार्तस्य शापाग्निः स विनिःसृतः}


% Check verse!
स तेन कर्शितो राजादीर्घकालमनात्मवान्
\threelineshloka
{`तं भ्राम्तरूपं निःशोभं संक्लिष्टमकरोत्कलिः'}
{ततो विषविमुक्तात्मा स्वंरूपमकरोत्कलिः}
{तं शप्तुमैच्छत्कुपितो निषधाधिपतिर्नलः}


\twolineshloka
{तमुवाच कलिर्भीतो वेपमानः कृताञ्जलिः}
{कोपं संयच्छ नृपते कीर्तिं दास्यामि ते पराम्}


\twolineshloka
{इन्द्रसेनस्य जननी कुपिता माऽशपत्पुरा}
{यदा त्वया परित्यक्ता ततोऽहं भृशपीडितः}


\twolineshloka
{अवसं त्वयि राजेन्द्र सुदुःखभपराजित}
{विषेण नागराजस्य दह्यमानो दिवानिशम्}


\threelineshloka
{शरणं त्वां प्रपन्नोस्मि शृणु चेदं वचो मम}
{ये च त्वां मनुजा लोके कीर्तयिष्यन्त्यतन्द्रिताः}
{मत्प्रत्सूतं भयं तेषां न कदाचिद्भविष्यति}


\threelineshloka
{`न तेषां मानसं किंचिच्छारीरं वाचिकं तथा}
{भविष्यति महाराज कीर्तयिष्यन्ति ये नलम्'}
{भयार्तं शरणं यातं यदि मां त्वं न शप्स्यसे}


\threelineshloka
{एवमुक्तो नलो राजा न्ययच्छत्कोपमात्मनः}
{ततो बीतः कलिः क्षिप्रं प्रविवेश विभीतकम्}
{कलिस्त्वन्येन नो दृष्टः कथयन्नैषधेन वै}


\twolineshloka
{ततो गतत्वरो राजा नैषधः परवीरहा}
{संप्रनष्टे कलौ राजा संख्यायास्य फलान्युत}


\twolineshloka
{मुदा परमया युक्तस्तेजसाऽथ परेण वै}
{रथमारुह्य तेजस्वी प्रययौ जवनैर्हयैः}


\twolineshloka
{बिभीतकश्चाप्रशस्तः संवृत्तः कलिसंश्रयात्}
{`ततः प्रभृतिराजेन्द्र लोकेऽस्मिन्पाण्डुनन्दन'}


\twolineshloka
{हयोत्तभानुत्पततो द्विजानिव पुनः पुनः}
{नलः संचोदयामास प्रहृष्टेनान्तरात्मना}


\twolineshloka
{विदर्भाभिमुखो राजा प्रययौ स महायशाः}
{नले तु समतिक्रान्ते कलिरप्यगमद्गृहम्}


\twolineshloka
{ततो गतज्वरो राजा नलोऽभूत्पृथिवीपतिः}
{विमुक्तः कलिना राजन्रूपमात्रवियोजितः}


\chapter{अध्यायः ७१}
\twolineshloka
{बृहदश्व उवाच}
{}


\twolineshloka
{ततो विदर्भान्संप्राप्तं सायाह्ने सत्यविक्रमम्}
{ऋतुपर्णं जना राज्ञे भीमाय प्रत्यवेदयन्}


\twolineshloka
{स भीमवचनाद्राजा कुण्डिनं प्राविशत्पुरम्}
{नादयन्रथघोषेण सर्वाः स विदिशो दिशः}


\twolineshloka
{ततस्तं रथनिर्घोषं नलाश्वास्तत्र शुश्रुवुः}
{श्रुत्वा तु समहृष्यन्त पुरेव नलसन्निधौ}


\twolineshloka
{दमयन्ती तु शुश्राव रथघोषं नलस्य तम्}
{यथा मेघस्य नदतो गम्भीरं जलदागमे}


\threelineshloka
{परं विस्मयमापन्ना श्रुत्वा नादं महास्वनम्}
{नलेन संगृहीतेषु पुरेव नलवाजिषु}
{सदृशं रथनिर्घोषं मेने भैमी तथा हयाः}


\twolineshloka
{प्रासादस्थाश्च शिखिनः शालास्थाश्चैव वारणाः}
{हयाश्च शुश्रुवुस्तस्य रथघोषं महीपतेः}


\threelineshloka
{तच्छ्रुत्वा रथनिर्घोषं वारणाः शिखिनस्तथा}
{प्रवेदुरुन्मुखा राजन्दृष्ट्वेव जलदोदयम् ॥दमयन्त्युवाच}
{}


\twolineshloka
{यथाऽसौ रथनिर्घोषः पूरयन्निव मेदिनीम्}
{ममाह्लादयते चेतो नल एव महीपतिः}


\twolineshloka
{अद्य नन्द्राभवक्रं तं न पश्यामि नलं यदि}
{असंख्येयगुणं वीरं विनङ्क्ष्यामि न संशयः}


\twolineshloka
{यदि वै तस्य वीरस्य हाह्वोर्नाद्याहमन्तरम्}
{प्रविशामि सुखस्पर्शं नभविष्याम्यसंशयम्}


\twolineshloka
{यदि मां मेघनिर्घोषो नोपगच्छति नैषधः}
{अद्य चामीकरप्रख्यं प्रवेक्ष्यामि हुताशनम्}


\twolineshloka
{यदि मां सिंहविक्रान्तो मत्तवारणविक्रमः}
{नाभिगच्छतिराजेन्द्रो विनङ्क्ष्यामि न संशयः}


\twolineshloka
{न स्मराम्यनृतं किंचिन्न स्मराम्यपकारताम्}
{न च पर्युषितं वाक्यं स्वैरेष्वपि महात्मनः}


\twolineshloka
{प्रभुः क्षमावान्वीरश्च दाता चाप्यधिको नृपैः}
{अहो नीचानुवर्ती च क्लीबवन्मम नैषध}


\twolineshloka
{गुणांस्तस्य स्मरन्त्या मे तत्पराया दिवानिशम्}
{हृदयं दीर्यत इदं शोकात्प्रियविनाकृतम्}


\twolineshloka
{एवं विलपमाना सा नष्टसंज्ञेव भारत}
{आरुरोह महद्वेश्म पुण्यश्लोकदिदृक्षया}


\twolineshloka
{ततो मध्यमकक्षायां ददर्श रथमास्थितम्}
{ऋतुपर्णं महीपालं सहवार्ष्णेयबाहुकम्}


\twolineshloka
{ततोऽवतीर्य वार्ष्णएयो बाहुकश्च रथोत्तमात्}
{हयांस्तानवमुच्याथ स्थापयामासतू रथम्}


\twolineshloka
{सोऽवतीर्य रथोपस्थादृतुपर्णो नराधिपः}
{उपतस्थे महाराजं भीमं भीमपराक्रमम्}


\twolineshloka
{तं भीमः प्रतिजग्राह पूजया परया मुदा}
{स तेन पूजितो राज्ञा ऋतुपर्णो नराधिपः}


\threelineshloka
{स तत्र कुण्डिने रम्ये वसमानो महीपतिः}
{न च किंचित्तदाऽपश्यत्प्रेक्षमाणो मुहुर्मुहुः}
{स तु राज्ञा समागम्य विदर्भपतिना तदा}


\twolineshloka
{किं कार्यं स्वागतं तेऽस्तु राज्ञा पृष्टः स भारत}
{नाभिजज्ञे स नृपतिर्दुहित्रर्थे समागतम्}


\twolineshloka
{ऋतुपर्णोपि राजा स धीमान्सत्यपराक्रमः}
{राजानं राजपुत्रं वा न स्म पश्यति कंचन}


\twolineshloka
{नैव स्वयंवरकथां न च विप्रसमागमम्}
{`न चान्यं कंचिदारम्भं स्वयंवरविधिं प्रति'}


\twolineshloka
{ततो विगणयद्राजा मनसा कोसलाधिपः}
{आगतोस्मीत्युवाचैनं भवन्तमभिवादुकः}


\threelineshloka
{राजापि च स्मयन्भीमो मनसा समचिन्तयत्}
{अधिकं योजनशतं तस्यागमनकारणम्}
{ग्रामान्बहूनतिक्रम्य नाध्यगच्छद्यथातथम्}


\twolineshloka
{अल्पकार्यं विनिर्दिष्टं तस्यागमनकारणम्}
{पश्चादुदर्के ज्ञास्याप्रि कारणं यद्भविष्यति}


\twolineshloka
{नैतदेवं स नृपतिस्तं सत्कृत्य व्यसर्जयत्}
{विश्राम्यतामित्युवाच क्लान्तोसीति पुनःपुनः}


\twolineshloka
{स सत्कृतः प्रहृष्टात्मा प्रीतः प्रीतेन पार्थिवः}
{राजप्रेष्यैरनुगतो दिष्टं वेश्म समाविशत्}


\twolineshloka
{ऋतुपर्णे गते राजन्वार्ष्णेयसहिते नृपे}
{बाहुको रथमादाय रथशालामुपागमत्}


\twolineshloka
{स मोचयित्वा तानश्वानुपचर्य च शास्त्रतः}
{स्वयं चैतान्समाश्वास्य रथोपस्थ उपाविशत्}


\twolineshloka
{दमयन्त्यपि शोकार्ता दृष्ट्वा भागस्वरिं नृपम्}
{सूतपुत्रं च वार्ष्णेयं बाहुकं च तथाविधम्}


\twolineshloka
{चिन्तयामास वैदर्भी कस्यैष रथनिःस्वनः}
{नलस्येव महानासीन्न च पश्यामि नैषधम्}


\twolineshloka
{वार्ष्णोयेन भवेन्नूनं विद्या सैवोपशिक्षिता}
{तेनाद्य रथनिर्घोषो नलस्येव महानभूत्}


\twolineshloka
{कआहोस्विदृतुपर्णोऽपि यथा राजा नलस्यथा}
{यथाऽयंरथनिर्घोषो नैषधस्येव लक्ष्यते}


\twolineshloka
{एवं सा तर्कयित्वा तु दमयन्ती विशांपते}
{दूतीं प्रस्थापयामास नैषधान्वेषणे शुभा}


\chapter{अध्यायः ७२}
\twolineshloka
{दमयन्त्युवाच}
{}


\twolineshloka
{गच्छ केशिनि जानीहि क एष रथवाहकः}
{उपविष्टो रथोपस्थे विकृतो ह्रस्वबाहुकः}


\twolineshloka
{अभ्येत्य कुशलं भद्रे मृदुपूर्वंसमाहिता}
{पृच्छेथाः पुरुषे ह्येनं यथातत्त्वमनिन्दिते}


\twolineshloka
{अत्र मे महती शङ्का भवेदेष नलो नृपः}
{यथाच मनसस्तुष्टिर्हृदयस्य च निर्वृतिः}


\twolineshloka
{ब्रूयाश्चैनं कथान्ते त्वं पर्णादवचनं यथा}
{प्रतिवाक्यं च सुश्रोणि बुद्ध्येथास्त्वमनिन्दिते}


\threelineshloka
{एवं समाहिता गत्वा दूती बाहुकमब्रवीत्}
{दमयन्त्यपि कल्याणी प्रासादस्थाऽनव्वैक्षत ॥केशिन्युवाच}
{}


\twolineshloka
{स्वागतं ते मनुष्येन्द्र कुशलं ते ब्रवीम्यहम्}
{दमयन्त्या वचः साधु निबोध पुरुषर्षभ}


\threelineshloka
{कदा वै प्रस्थिता यूयं किमर्थमिह चागताः}
{तत्त्वं ब्रूहि यथान्यायं वैदर्भी श्रोतुमिच्छति ॥बाहुक उवाच}
{}


\twolineshloka
{श्रुतः स्वयंवरो राज्ञा कोसलेन महात्मना}
{द्वितीयो दमयन्त्या वै भविता श्व इति द्विजात्}


\threelineshloka
{श्रुत्वैतत्प्रस्थितो राजा शतयोजनयायिभिः}
{हयैर्वातजवैर्मुख्यैरहमस्य च सारथिः ॥केशिन्युवाच}
{}


\threelineshloka
{अथ योसौ तृतीयो वः स कुतः कस्य वा पुनः}
{त्वं च कस्य कथं चेदंत्वयि कर्म समाहितम् ॥बाहुक उवाच}
{}


\twolineshloka
{पुण्यश्लोकस्य वै सूतो वार्ष्णेय इति विश्रतः}
{स नले विद्रुते भद्रेभागस्वरिमुपस्थितः}


\threelineshloka
{अहमप्यश्वकुशलः सूतत्वे च प्रतिष्ठितः}
{ऋतुपर्णेन सारथ्ये भोजने च वृतः स्वयम् ॥केशिन्युवाच}
{}


\threelineshloka
{अथ जानाति वार्ष्णेयः क्वनु राजा नलो गतः}
{कथं च त्वयि वा तेन कथितं स्यात्तु बाहुक ॥बाहुक उवाच}
{}


\twolineshloka
{इहैव पुत्रौ निक्षिप्य नलस्य प्रियदर्शनौ}
{गतस्ततो यथाकामं नैष जानाति नैषधम्}


\twolineshloka
{न चान्यः पुरुषः कश्चिन्नलं वेत्ति यशस्विनि}
{गूढश्चरति लोकेऽस्मिन्नष्टरूपो महीपतिः}


\threelineshloka
{आत्मैव तु नलं वेद या चास्य तदनन्तरा}
{न हि वै स्वानि लिङ्गानि नलं शंसन्ति कर्हिचित् ॥केशिन्युवाच}
{}


\twolineshloka
{योसावयोध्यां प्रथमं गतोसौ ब्राह्मणस्तदा}
{इमानि नारीवाक्यानि कथयानः पुनःपुनः}


\twolineshloka
{क्वनु त्वं कितव च्छित्त्वा वस्त्रार्धं प्रस्थितो मम}
{उत्सृज्य विपिने सुप्तामनुरक्तां प्रियां प्रिय}


\twolineshloka
{सा वै यथा समादिष्टा तथाऽऽस्ते त्वत्प्रतीक्षिणी}
{दह्यमाना दिवारात्रौ वस्त्रार्धेनाभिसंवृता}


\twolineshloka
{तस्या रुदनत्याः सततं तेन दुःखेन पार्थिव}
{प्रसादं कुरु मे वीर प्रतिवाक्यं वदस्व च}


\twolineshloka
{तस्यास्तत्प्रियमाख्यानं प्रवदस्व महामते}
{तदेव वाक्यं वैदर्भी श्रोतुमिच्छन्त्यनिन्दिता}


\threelineshloka
{एतच्छ्रुत्वा प्रतिवचस्तस्य दत्तं त्वया किल}
{यत्पुरा तत्पुनस्त्वत्तो वैदर्भी श्रोतुमिच्छति ॥बृहदश्व उवाच}
{}


\twolineshloka
{एवमुक्तस्य केशिन्या नलस्य कुरुनन्दन}
{हृदयं व्यथितं चासीदश्रुपूर्णे च लोचने}


\twolineshloka
{स निग्राह्यात्मनो दुःखं दह्यमानो महीपतिः}
{बाष्पसंदिग्धया वाचा पुनरेवेदमब्रवीत्}


\twolineshloka
{वैषम्यमपि संप्राप्ता गोपायन्ति कुलस्त्रियः}
{आत्मानमात्मना सत्यो जितः स्वर्गो न संशयः}


\twolineshloka
{रहिता भर्तृभिश्चापि न क्रुध्यन्ति कदाचन}
{प्राणांश्चारित्रकवचान्धारयन्ति वरस्त्रियः}


\twolineshloka
{विषमस्थेन मूढेन परिभ्रष्टसुखेन च}
{यत्सा तेन परित्यक्ता तत्रन क्रोद्धुमर्हति}


\twolineshloka
{प्राणयात्रां परिप्रेप्सोः शकुनैर्हतवाससः}
{आधिभिर्दह्यमानस्य श्यामा न क्रोद्धुमर्हति}


\twolineshloka
{सत्कृताऽसत्कृता वाऽपि पतिं दृष्ट्वा तथाविधम्}
{राज्यभ्रष्टं श्रिया हीनं क्षिधितं व्यसनाप्लुतम्}


\twolineshloka
{एवं ब्रुवाणस्तद्वाक्यं नलः परमदुर्मनाः}
{न बाष्पमशकत्सोढुं प्ररुरोद च भारत}


\twolineshloka
{ततः सा केशिनी गत्वा दमयन्त्यै न्यवेदयत्}
{तत्सर्वं कथितं चैव विकारं तस्य चैव तम्}


\chapter{अध्यायः ७३}
\twolineshloka
{बृहदश्व उवाच}
{}


\twolineshloka
{केशिन्यास्तद्वचः श्रुत्वा दमयन्ती विशांपते}
{शङ्कमाना नलं तं वै केशिनीमिदमब्रवीत्}


\twolineshloka
{गच्छ केशिनि भूयस्त्वं परीक्षां कुरु बाहुके}
{अब्रुवाणा समीपस्था चरितान्यस्य लक्षय}


\twolineshloka
{यदा च किंचित्कुर्यात्स कारणं तत्र भामिनि}
{तत्रसंचेष्टमानस्य संलक्षेथा विचेष्टितम्}


\threelineshloka
{न चास्य प्रतिबन्धेन देयोऽग्निरपि केशिनि}
{याचते न जलं देयं सकृच्चात्वरमाणया}
{एतत्सर्वं समीक्ष्यत्वं चरितं मे निवेदय}


\twolineshloka
{निमित्तं यत्त्वया दृष्टं बाहुके दैवमानुषम्}
{यच्चान्यदपि पश्येथास्तच्चाख्येयं त्वया मम}


\twolineshloka
{दमयन्त्यैवमुक्ता सा जगामाथ च केशिनी}
{निशाम्याथ हयज्ञस्य लिङ्गानि पुनरागमत्}


\threelineshloka
{सा तत्सर्वं यथावृत्तं दमयन्त्यै न्यवेदयत्}
{निमित्तं यत्तया दृष्टं बाहुके दिव्यमानुषम् ॥केशिन्युवाच}
{}


\twolineshloka
{दृढं शुच्यपदानोसौ न मया मानुषः क्वचित्}
{दृष्टपूर्वः श्रुतो वाऽपि दमयन्ति तथाविधः}


\threelineshloka
{ह्रस्वमासाद्य तु द्वारं नासौ विनमते क्वचित्}
{तं तु दृष्ट्वा यथासङ्गमुत्सर्पति यथासुखम्}
{संकटेऽप्यस्य सुमहद्विवरं जायतेऽधिकम्}


\threelineshloka
{ऋतुपर्णस्य चार्थाय भोजनीयमनेकशः}
{ऋतुपर्णस्य चार्थाय भोजनीयमनेकशः}
{प्रेषितं तत्रराज्ञा तु मांसं बहु च पाशवम्}


\twolineshloka
{तस्य प्रक्षालनार्थाय कुम्भास्तत्रोपकल्पिताः}
{ते तेनावेक्षिताः कुम्भाः पूर्णा एवाभवंस्ततः}


\twolineshloka
{ततः प्रक्षालनं कृत्वा समधिश्रित्य बाहुकः}
{तृणमुष्टिं समादाय सवितुस्तं समादधत्}


\twolineshloka
{अथ प्रज्वलितस्तत्र सहसा हव्यवाहनः}
{तदद्भुततमं दृष्ट्वा विस्मिताऽमिहागता}


\twolineshloka
{अन्यच्च तस्मिन्सुमहदाश्चर्यं लक्षितं मया}
{यदग्निमपि संस्पृश्य नैवासौ दह्यते शुभे}


\twolineshloka
{छन्देन चोदकं तस्य वहत्यावर्जितं द्रुतम्}
{अतीव चान्यत्सुमहदाश्चर्यं दृष्टवत्यहम्}


\threelineshloka
{यत्स पुष्पाण्युपादाय हस्ताभ्यां ममृदे शनैः}
{मृद्यमानानि पाणिभ्यां तेन पुष्पाणि नान्यथा}
{भूय एव सुगन्धीनि हृपितानि भवन्ति हि}


\threelineshloka
{एतान्यद्भुतकल्पानि दृष्ट्वाऽहं भृशविस्मिता}
{चेष्टितानि विशालाक्षि बाहुकस्य समीपतः ॥बृहदश्व उवाच}
{}


\twolineshloka
{दमयन्ती तु तच्छ्रुत्वा पुण्यश्लोकस्य चेष्टितम्}
{अमन्यत नलं प्राप्तं कर्मचेष्टाभिसूचितम्}


\twolineshloka
{सा शङ्कमाना भर्तारं नलं बाहुकरूपिणम्}
{केशिनीं श्लक्ष्णया वाचा रुदन्ती पुनरब्रवीत्}


\twolineshloka
{पुनर्गच्छ प्रमत्तस्य बाहुकस्योपसंस्कृतम्}
{महानसाच्छ्रितं मांसमानयस्वेह भामिनि}


\threelineshloka
{सा दृष्ट्वाबाहुके व्यग्रे तन्मांसमपकृष्य च}
{अत्युष्णमेव त्वरिता तत्क्षणात्प्रियकारिणी}
{दमयन्त्यै ततः प्रादात्केशिनी कुरुनन्दन}


\twolineshloka
{साऽशिता नलसिद्धस् मांसस्य बहुशः पुरा}
{प्राश्य मत्वा नलं सूतं प्राक्रोशद्भृशदुःखिता}


\twolineshloka
{वैक्लव्यं परमं गत्वाप्रक्षाल्य च मुखं ततः}
{मिथुनं प्रेषयामास केशिन्या सह भारत्}


\twolineshloka
{इन्द्रसेनां सह भ्रात्रा समभिज्ञाय बाहुकः}
{अभिद्रुत्य तदा राजा परिष्वज्याङ्कमानयत्}


\twolineshloka
{बाहुकस्तु समासाद्य सुतौ सुरसुतोपमौ}
{भृशं दुःखपरीतात्मा सुस्वरं प्ररुरोद ह}


\twolineshloka
{नैषधो दर्शयित्वा तु विकारमसकृत्तदा}
{उत्सृज्य सहसा पुत्रौ केशिनीमिदमब्रवीत्}


\twolineshloka
{इदं च मदृशं भद्रे मिथुनं मम पुत्रयोः}
{अतो दृष्ट्वैव सहसा बाष्पमुत्सृष्टवानहम्}


\twolineshloka
{बहुशः संपतन्तीं त्वां जनः शङ्केत दोषतः}
{वयंच देशातिथयो गच्छ भद्रे यथासुखम्}


\chapter{अध्यायः ७४}
\twolineshloka
{बृहदश्व उवाच}
{}


\twolineshloka
{सर्वं विकारं दृष्ट्वा तु पुण्यश्लोकस्य धीमतः}
{आगत्य केशिनी क्षिप्रं दमयन्त्यै न्यवेदयत्}


\twolineshloka
{दमयन्ती ततो भूयः प्रेषयामास केशिनीम्}
{मातुः सकाशं दुःखार्ता नलशङ्कासमुत्सुका}


\twolineshloka
{परीक्षितो मे बहुशो बाहुको नलशङ्कया}
{रूपे मे संशयस्त्वेकः स्वयमिच्छमि वेदितुम्}


\twolineshloka
{स वा प्रवेश्यतां मातर्मां वाऽनुज्ञातुमर्हसि}
{विदितं वाऽथवाऽज्ञातं पितुर्मे संविधीयताम्}


\twolineshloka
{एवमुक्ता तु वैदर्भ्या सा देवी भीममब्रवीत्}
{दुहितुस्तमभिप्रायमन्वजानात्स पार्थिवः}


\twolineshloka
{सा चै पित्राऽभ्यनुज्ञाता मात्रा च भरतर्षभ}
{`तयोर्नियोगात्कौरव्य केशिनीमिदमब्रवीत्'}


\twolineshloka
{गच्छ केशिनि शीघ्रं त्वंबाहुकं पितृशासनात्}
{आनयस्व यथा माता त्वं तथा कुरु मे प्रियम्}


\twolineshloka
{गत्वा तु केशिनी शिघ्रं बाहुकं वाक्यमब्रवीत्}
{भीमस्य शासनात्सूतागताहं विद्धि बाहुक}


\threelineshloka
{प्रविश्यतां राजवेश्म इत्युक्तो भरतर्षभ}
{बाहुकस्तु चिरं ध्यात्वा केशिन्या सह भारत}
{प्रविवेश महाबाहुर्दमयन्तीनिवेशनम्'}


% Check verse!
नलं प्रवेशयामास यत्रतस्याः प्रतिश्रयः
\twolineshloka
{तां स्म दृष्ट्वैव सहसा दमयन्तीं नलो नृपः}
{आविष्टः शोकदुःखाभ्यां बभूवाश्रुपरिप्लुतः}


\twolineshloka
{तं तु दृष्ट्वातथायुक्तं दमयन्ती नलं तदा}
{तीव्रशोकसमाविष्टा बभूव वरवर्णिनी}


\twolineshloka
{ततः काषायवसना जटिला मलपङ्किनी}
{दमयन्ती महाराज बाहुकं वाक्यमब्रवीत्}


\twolineshloka
{पूर्वं दृष्टस्त्वया कश्चिद्धर्मज्ञो नाम बाहुक}
{सुप्तामुत्सृज्यविपिने गतो यः पुरुषः स्त्रियम्}


\twolineshloka
{अनागसं प्रियां भार्यां विजने श्रममोहिताम्}
{अपहाय तु को गच्चेत्पुण्यश्लोकमृतेनलम्}


\twolineshloka
{किमु तस्य मया बाल्यादपराद्धं महीपतेः}
{यो मामुत्सृज्य विपिने गतवान्निद्रयाऽर्दिताम्}


\twolineshloka
{साक्षाद्देवानपाहाय वृतो यः स पुरा मया}
{अनुव्रतामभिमतां पुत्रिणीं त्यक्तवान्कथम्}


\threelineshloka
{अग्नौ पाणिगृहीता च हंसानां वचने स्थिताम्}
{भरिष्यामीति सत्यं तु प्रतिश्रुत्य क्व तद्गतम् ॥बृहदश्व उवाच}
{}


\twolineshloka
{दमयन्त्या ब्रुवन्त्यास्तु सर्वमेतदरिदम}
{शोकजं वारिनेत्राभ्यामसुखं प्रास्रवद्बहु}


\twolineshloka
{अतीव कृष्णताराभ्यां रक्तान्ताभ्यां जलंतु तत्}
{परिस्रवन्नलो राजा शोकार्तइदमब्रवीत्}


\twolineshloka
{`नलोऽहं विपुलश्रोणि त्वामुत्सृज्य यतो गतः}
{आविष्टः कलिना भद्रे तेन मोहवशं गतः ॥'}


\twolineshloka
{मम राज्यं प्रनष्टं यन्नाहं तत्कृतवान्स्वयम्}
{कलिना तत्कृतं भीरु यच्च त्वामहमत्यजम्}


\twolineshloka
{यत्त्वया धर्मकृच्छ्रे तु शापेनाभिहतः पुरा}
{वनस्थया दुःखितया शोचन्त्या मांदिवानिशम्}


\twolineshloka
{स मच्छरीरे त्वच्छापाद्दह्यमानोऽवसत्कलिः}
{त्वच्छापदग्धः सततं सोऽग्नावग्निरिवाहितः}


\twolineshloka
{मम च व्यवसायेन तपसा चैव निर्जितः}
{दुःखस्यान्तेन चानेन भवितव्यं हि नौ शुभे}


\twolineshloka
{विमुच्य मां गतः पापस्ततोऽहमिह चागतः}
{त्वदर्थं विपुलश्रोणि न हि मेऽन्यत्प्रयोजनम्}


\twolineshloka
{कथं नु नारी भर्तारमनुरक्तमनुव्रतम्}
{उत्सृज्य वरयेदन्यं यथा त्वं भीरु कर्हिचित्}


\twolineshloka
{दूताश्चरन्ति पृथिवीं कृत्स्नां नृपतिशासनात्}
{भैमी किल स्म भर्तारं द्वितीयं वरयिष्यति}


\twolineshloka
{स्वैरवृत्ता यथाकाममनुरूपमिवात्मनः}
{श्रुत्वैव चैवं त्वरितो भागस्वरिरुपस्थितः}


\twolineshloka
{दमयन्ती तु तच्छ्रुत्वा नलस्य परिदेवितम्}
{प्राञ्जलिर्वेपमाना च भीता वचनमब्रवीत्}


\twolineshloka
{न मामर्हसि कल्याण पापेन परिशङ्कितुम्}
{मया हि देवानुत्सृज्य वृतस्त्वं निषधाधिप}


\twolineshloka
{तवाभिगमनार्थं तु सर्वतो ब्राह्मणा गताः}
{वाक्यानि मम गाथाभिर्गायमाना दिशो दश}


\twolineshloka
{ततस्त्वां ब्राह्मणो विद्वान्पर्णादो नाम पार्थिव}
{अभ्यगच्छत्कोसलायामृतुपर्णनिवेशने}


\twolineshloka
{तेन वाक्ये कृतेसम्यक्प्रतिवाक्ये तथा हृते}
{उपायोऽयंमया दृष्टो नैषधानयने तव}


\twolineshloka
{त्वामृते नहि लोकेऽन्य एकाह्ना पृथिवीपते}
{समर्थो योजनशतं गन्तुमश्वैर्नराधिप}


\threelineshloka
{`तथापि मां महीपाल भजेतां चरणौ तव}
{'स्पृशेयं तेन सत्येन पादावेतौ महीपते}
{यथा नासत्कृतं किंचिन्मनसाऽपि चराम्यहम्}


\twolineshloka
{अयं चरति लोकेऽस्मिन्भूतसाक्षी सदागतिः}
{एष मे मुञ्चतु प्राणान्यदि पापं चराम्यहम्}


\twolineshloka
{यथा चरति तिग्मांशुः परितो भुवनं सदा}
{स मुञ्चतु मम प्राणान्यदि पापं चराम्यहम्}


\twolineshloka
{चन्द्रमाः सर्वभूतानामन्तश्चरति साक्षिवत्}
{स मुञ्चतु मम प्राणान्यदि पापं चराम्यहम्}


\twolineshloka
{एते देवास्त्रयः कृत्स्नं त्रैलोक्यं धारयन्ति वै}
{विब्रुवन्तु यथा सत्यमेतद्देवास्त्यजन्तु माम्}


\twolineshloka
{एवमुक्ते ततो वायुरन्तरिक्षादभाषत}
{नैषा कृतवती पापं नल सत्यं ब्रवीमि ते}


\twolineshloka
{राजञ्शीलनिधिः स्फीतो दमयन्त्या सुरक्षितः}
{साक्षिणो रक्षिणश्चास्या वयं त्रीन्परिवत्सरान्}


\twolineshloka
{उपायो विहितश्चायं त्वदर्थमतुलोऽनया}
{न ह्येकाह्ना शतं गन्ता त्वामृतेऽन्यः पुमानिह}


\twolineshloka
{उपपन्ना त्वया भैमी त्वं च भैम्या महीपते}
{नात्र शङ्का त्वया कार्या संगच्छ सह भार्यया}


\twolineshloka
{तथा ब्रुवति वायौ तु पुष्पवृष्टिः पपात ह}
{देवदुन्दुभयो नेदुर्ववौ च पवनः शिवः}


\twolineshloka
{तदद्भुतमयं दृष्ट्वा नलो राजाऽथ भारत}
{दमयन्त्यां विशङ्कां तामुपाकर्षदरिंदमः}


\twolineshloka
{ततस्तद्वस्त्रमरजः प्रावृणोद्वसुधाधिपः}
{संस्मृत्य नागराजं तं ततो लेभे स्वकं वपुः}


\twolineshloka
{स्वरूपिणं तु भर्तारं दृष्ट्वा भीमसुता तदा}
{प्राक्रोशदुच्चैरालिङ्ग्य पुण्यश्लोकमनिन्दिता}


\twolineshloka
{भैमीमपि नलो राजा भ्राजमानो यथा पुरा}
{सस्वजे स्वसुतौ चापि यथावत्प्रत्यनन्दत}


\twolineshloka
{ततः स्वोरसि विन्यस्य वक्रं तस्य शुभानना}
{परीता तेन दुःखेन निशश्वासायतेक्षणा}


\twolineshloka
{तथैव मलदिग्धाङ्गीं परिष्वज्य शुचिस्मिताम्}
{सुचिरं पुरुषव्याघ्रस्तस्थौ शोकपरिप्लुतः}


\twolineshloka
{ततः सर्वं यथावृत्तं दमयन्त्या नलस्य च}
{भीमायाकथयत्प्रीत्या वैदर्भ्या जननी नृप}


\twolineshloka
{ततोऽब्रवीन्महाराजः कृतशौचमहंनलम्}
{दमयन्त्या सहोपेतं कल्ये द्रष्टा सुखोषितम्}


\twolineshloka
{ततस्तौ सहितौ रात्रिं कथयन्तौ पुरातनम्}
{वने विचरितं सर्वमूषतुर्मुदितौ नृप}


\twolineshloka
{गृहे भीमस्य नृपतेः परस्परसुखैषिणौ}
{वसेतां हृष्टसंकल्पौ वैदर्भी च नलश् ह}


\twolineshloka
{स चतुर्थे ततो वर्षे संगम्य सह भार्यया}
{सर्वकामैः सुसिद्धार्थो लब्धवान्परमां मुदम्}


\twolineshloka
{दमयन्त्यपि भर्तारमासाद्याप्यायिता भृशम्}
{अर्धसंजातसस्येव तोयं प्राप्य वसुंधरा}


\twolineshloka
{सैवं समेत्य व्यपनीय तन्द्रांशान्तज्वरा हर्षविवृद्धसत्त्वा}
{रराज भैमी समवाप्तकामाशीतांशुना रात्रिरिवोदितेन}


\chapter{अध्यायः ७५}
\twolineshloka
{बृहदश्व उवाच}
{}


\twolineshloka
{अथ तां व्युषितो रात्रिं नलो राजा स्वलंकृतः}
{वैदर्भ्या सहितः काल्यं ददर्श वसुधाधिपम्}


\twolineshloka
{ततोऽभिवादयामास प्रयतः श्वशुरं नलः}
{ततो नु दमयन्ती च ववन्दे पितरं शुभा}


\twolineshloka
{तं भीमः प्रतिजग्राह पुत्रवत्परया मुदा}
{यथार्हं पूजयित्वा च समाश्वासयत प्रभुः}


\twolineshloka
{नलेन सहितां तत्रदमयन्तीं पतिव्रताम्}
{`अनुजग्राह महता सत्कारेण क्षितीश्वरः'}


\twolineshloka
{तामर्हणां नलो राजा प्रतिगृह्य यथाविधि}
{परिचर्यां स्वकां तस्मै यथावत्प्रत्यवेदयत्}


\twolineshloka
{ततो बभूव नमरे सुमहान्हर्षजः स्वनः}
{जनस्य संप्रहृष्टस्य नलं दृष्ट्वा तथाऽऽगतम्}


\twolineshloka
{अशोभयच्च नगरीं पताकाध्वजमालिनीम्}
{सिक्ताः सुमृष्टपुष्पाढ्या राजमार्गाः स्वलंकृताः}


\twolineshloka
{द्वारिद्वारि च पौराणां पुष्पभङ्गः प्रकल्पितः}
{अर्चितानि च सर्वाणि देवतायतनानि च}


\twolineshloka
{ऋतुपर्णोऽपिशुश्राव बाहुकच्छद्मिनं नलम्}
{दमयन्त्या समायुक्तं जहृषे च नराधिपः}


\twolineshloka
{तमानाय्य नलं राजा क्षमयामास पार्थिवः}
{स च तं क्षमयामास हेतुभिर्बुद्धिसंमितः}


\twolineshloka
{स सत्कृतोमहीपालो नैषधं विस्मिताननः}
{दिष्ट्या समेतो दारैः स्वैर्भवानित्यभ्यनन्दत}


\twolineshloka
{कच्चित्तु नापराधं ते कृतवानस्मि नैषध}
{अज्ञातवासे वसतो मद्गृहेवसुधाधिप}


\threelineshloka
{यदि वाऽबुद्धिपूर्वाणि यदि बुद्ध्याऽपि कानिचित्}
{मया कृतान्यकार्याणि तानि त्वं क्षन्तुमर्हसि ॥नल उवाच}
{}


\twolineshloka
{न मेऽपराधं कृतवांस्त्वं स्वल्पमपि पार्थिव}
{कृतेऽपि च न मे कोपः क्षन्तव्यं हि मया तव}


\twolineshloka
{पूर्वं ह्यापि सखा मेऽसि संबन्धी च जनाधिप}
{अत ऊर्ध्वं तु भूयस्त्वं प्रीतिमाहर्तुमर्हसि}


\twolineshloka
{सर्वकामैः सुविहितैः सुखमस्म्युषितस्त्वयि}
{न तथा स्वगृहेराजन्यथा तव गृहे सदा}


\twolineshloka
{इदं चैव हयज्ञानं त्वदीयं मयि तिष्ठति}
{तदुपाकर्तुमिच्छामि मन्यसे यदि पार्थिव}


\twolineshloka
{एवमुक्त्वा ददौ विद्यामृतुपर्णाय नैषधः}
{स च तां प्रतिजग्राह विधिदृष्टेन क्रमणा}


\threelineshloka
{गृहीत्वा चाश्वहृदयं प्रीतो भागस्वरिर्नृप}
{निषधाधिपतेश्चापि दत्ताऽक्षहृदयं नृपः}
{सूतमन्यमुपादाय ययौ स्वपुरमेव ह}


\twolineshloka
{ऋतुपर्णे गते राजन्नलो राजा विशांपते}
{नगरे कुण्डिने कालं नातिदीर्गमिवावसत्}


\chapter{अध्यायः ७६}
\twolineshloka
{बृहदश्व उवाच}
{}


\twolineshloka
{स मासमुष्य कौन्तेय भीममामन्त्र्य नैषधः}
{पुरादल्पपरीवारो जगाम निषधान्प्रति}


\twolineshloka
{रथेनैकेन शुभ्रेण दन्तिभिः परिषोडशै}
{पञ्चाशद्भिर्हयैश्चैव षट्शतैश्च पदातिभिः}


\twolineshloka
{स कम्पयन्निव महीं त्वरमाणो महापुरीम्}
{प्रविवेशाथ संरब्धस्तरसैव महामनाः}


\twolineshloka
{ततः पुष्करमासाद्य वीरसेनसुतो नलः}
{उवाच दीव्याव पुनर्बहुवित्तं मयाऽर्जितम्}


\twolineshloka
{दमयन्ती च यच्चान्यन्मम किंचन विद्यते}
{एष वै मम संन्यासस्तव राज्यंतु पुष्कर}


\twolineshloka
{पुनः प्रवर्ततां द्यूतमिति मे निश्चिता मतिः}
{एकपाणेन भद्रं ते प्राणयोश्च पणावहे}


\twolineshloka
{जित्वा परस्वमाहृत् राज्यंवा यदि वा वसु}
{प्रतिपाणः प्रदातव्यः प्राणो हि पणमुच्यते}


\twolineshloka
{न चेद्वाञ्छसि तद्द्यूतं युद्धद्यूतं प्रवर्तताम्}
{द्वैरथेनास्तु वै शान्तिस्तव वा मम वा नृप}


\twolineshloka
{वंशभोज्यमिदं राज्यमर्थितव्यं यथा तथा}
{येनकेनाप्युपायेन वृद्धानामिति शासनम्}


\twolineshloka
{द्वयोरेकतरे बुद्धिः क्रियतामद्य पुष्कर}
{कैतवेनाक्षवत्यां वा युद्धे वा नाम्यतां धनुः}


\twolineshloka
{नैषधेनैवमुक्तस्तु पुष्करः प्रहसन्निव}
{ध्रुवमात्मजयं मत्वा प्रत्याह निषधाधिपम्}


\twolineshloka
{दिष्ट्या त्वयाऽर्जितं वित्तं प्रतिपाणाय नैषध}
{दिष्ट्या च दुष्कृतंकर्म दमयन्त्याः क्षयं गतम्}


\twolineshloka
{दिष्ट्या वै प्रीसे राजन्मम लाभाय नैषध}
{पुनर्द्यूते च ते बुद्धिर्दिष्ट्या पुरुषसत्तम}


\twolineshloka
{धनेनानेन वै भैमी जितेन समलंकृता}
{मामुपस्थास्यति व्यक्तं दिवि शक्रमिवाप्सराः}


\twolineshloka
{नित्यशो हि स्मरामि त्वां प्रतीक्षेऽपि च नैषध}
{देवने च मभ प्रीतिर्भवत्येवासुहृद्गणैः}


\twolineshloka
{जित्वात्वद्य वरारोहां दमयन्तीमनिन्दिताम्}
{कतकृत्यो भविष्यामि सा हिमे नित्यशो हृदि}


\twolineshloka
{श्रुत्वा तु तस् ता वाचो बह्वबद्धप्रलापिनः}
{इयेष स शिरश्छेत्तुं खङ्गेन कुपितो नलः}


\twolineshloka
{स्मयंस्तु रोषताम्राक्षस्तमुवाच नलो नृपः}
{पणावः किं व्याहरसे जितो न व्याहरिष्यसि}


\twolineshloka
{ततः प्रावर्तत द्यूतं पुष्करस्य नलस्य च}
{एकपाणेन भद्रं ते नलेन स पराजितः}


\twolineshloka
{स रत्नकोशनिचयैः प्राणेन पणितोपि च}
{जित्वा च पुष्करं राजा प्रहसन्निदमब्रवीत्}


\threelineshloka
{मम सर्वमिदं राज्यमव्यग्रं हतकण्ठकम्}
{वैदर्भी न त्वया शक्या राजापशद वीक्षितम्}
{तस्यास्त्वं सपरीवारो मूढ दासत्वमागतः}


\twolineshloka
{न त्वया तत्कृतंकर्म येनाहं विजितः पुरा}
{कलिना तत्कृतं कर्म त्वं च मूढ न बुध्यसे}


\twolineshloka
{नाहं परकृतं दोषं त्वय्याधास्ये कथंचन}
{यथासुखं वै जीवत्वंप्राणानवसृजामि ते}


\twolineshloka
{तथैव सर्वसंभारं स्वमंशं वितरामि ते}
{तथैव च मम प्रीतिस्त्वयि वीर न संशयः}


\twolineshloka
{सौहार्दं चापि मे त्वत्तो न कदाचित्प्रहास्यति}
{पुष्कर त्वं हि मे भ्राता संजीव शरदः शतम्}


\twolineshloka
{एवं नलः सान्त्वयित्वा भ्रातरं सत्यविक्रमः}
{वचनैस्तोषयामास परिष्वज्य पुनः पुनः}


\twolineshloka
{सान्त्वितो नैषधेनैवं पुष्ररः प्रत्युवाच तम्}
{पुण्यश्लोकं तदा राजन्नभिवाद्य कृताञ्जलिः}


\twolineshloka
{कीर्तिरस्तु तवाक्षय्या जीव वर्षायुतं सुखी}
{यो मे वितरसि प्राणानधिष्ठानं च पार्थिव}


\twolineshloka
{स तथा सत्कृतो राज्ञा मासमुष्य तदा नृपः}
{प्रययौ पुषरो हृष्टः स्वपुरं स्वजनावृतः}


\twolineshloka
{महत्या सेनया सार्धंविनीतैः परिचारकैः}
{भ्राजमान इवादित्यो वपुषा पुरुषर्षभ}


\twolineshloka
{प्रस्थाप्य पुष्करं राजा वित्तवन्तमनामयम्}
{प्रविवेश पुरं श्रीमानत्यर्थमुपशोभिताम्}


\twolineshloka
{प्रविश्य सान्त्वयामास पौरांश्च निषधाधिपः}
{`हितेषु चैषां सततं पितेवावहितोऽभवत्'}


\twolineshloka
{पौरा जानपदाश्चापि संप्रहृष्टतनूरुहाः}
{ऊचुः प्राञ्जलयः सर्वे सामात्यप्रमुखा जनाः}


\twolineshloka
{अद्यस्म निर्वृता राजन्पुरे जनपदेऽपि च}
{उपासितुं पुनः प्राप्ता देवा इव शतक्रतुम्}


\chapter{अध्यायः ७७}
\twolineshloka
{बृहदश्व उवाच}
{}


\twolineshloka
{प्रशान्ते तु पुरे हृष्टे संप्रवृत्ते महोत्सवे}
{महत्या सेनया राजा दमयन्तीमुपानयत्}


\twolineshloka
{`पुण्यश्लोकं तु राज्यस्थं श्रुत्वा भीमो महीपतिः}
{मुदा परमया युक्तो बभूव भरतर्षभ}


\twolineshloka
{अथ हृष्टमना राजा महत्या सेनया सह}
{सुतां प्रस्थापयामास पुण्यश्लोकाय धीमते'}


\twolineshloka
{दमयन्तीमपि पिता सत्कृत्यपरवीरहा}
{प्रास्थापयदमेयात्मा भीमो भीमपराक्रमः}


\twolineshloka
{आगतायां तु वैदर्भ्यां सपुत्रायां नलो नृपः}
{वर्तयामास मुदितो दोवराडिव नन्दने}


\twolineshloka
{तथा प्रकाशतां यातो जम्बूद्वीपे स राजसु}
{पुनः स्वे चावसद्राज्ये प्रत्याहृत्य महायशाः}


\twolineshloka
{ईजे च विविधैर्यज्ञैर्विधिवच्चाप्तदक्षिणैः}
{तथा त्वमपि राजेन्द्र ससुहृद्वक्ष्यसे चिरात्}


\twolineshloka
{दुःखमेतादृशंप्राप्तो नलः परपुरंजयः}
{देवनेन नरश्रेष्ठ सभार्यो भरतर्षभ}


\twolineshloka
{एकाकिनैव सुमहन्नलेन पृथिवीपते}
{दुःखमासादितं घोरं प्राप्तश्चाभ्युदयः पुनः}


\twolineshloka
{त्वं पुनर्भ्रातृसहितः कृष्णया चैव पाण्डव}
{कथाश्चापि समाकर्ण्य धर्ममेवानुचिन्तयन्}


\twolineshloka
{ब्राह्मणैश्च महाभागैर्वेदवेदाङ्गपारगैः}
{नित्यमन्वास्यसे राजंस्तत्र का परिदेवना}


\twolineshloka
{कर्कोटकस्य नागस् दमयन्त्या नलस्य च}
{ऋतुपर्णस्य राजर्षेः कीर्तनं कलिनाशनम्}


\twolineshloka
{इतिहासमिमं चापि कलिनाशनमच्युत}
{शक्यमाश्वसितुं श्रुत्वा त्वद्विधेन विशांपते}


\twolineshloka
{अस्थिरत्वं च संचिन्त्य पुरुषार्थस् नित्यदा}
{तस्योदये व्यये चापि न चिन्तयितुमर्हसि}


\twolineshloka
{श्रुत्वेतिहासं नृषते समाश्वसिहि मा शुचः}
{व्यसने त्वं महाराज न विषीदितुमर्हसि}


\twolineshloka
{विषमावस्थिते दैवे पौरुषेऽफलतां गते}
{विषादयन्ति नात्मानं सत्त्वापाश्रयिणो नराः}


\twolineshloka
{ये चेदं कथयिष्यन्ति नलस् चरितं महत्}
{श्रोष्यन्ति चाप्यभीक्ष्णं वै नालक्ष्मीस्तान्भजिष्यति}


\twolineshloka
{अर्थास्तस्योपपत्स्यन्ते धन्यतां च गमिष्यति}
{इतिहासमिमं श्रुत्वा पुराणं शश्वदुत्तमम्}


\twolineshloka
{पुत्रान्पौत्रान्पशूंश्चापि लभते नृषु चाग्र्यताम्}
{आरोग्यप्रीतिमांश्चैव भविष्ति न संशयः}


\twolineshloka
{भयात्रस्यसि यच्च त्वमाह्वयिष्यति मां पुनः}
{अक्षज्ञ इति तत्तेऽहं नाशयिष्यामि पार्थिव}


\threelineshloka
{वेदाक्षहृदयं कृत्स्नमहं सत्यपराक्रम}
{उपपद्यस्व कौन्तेय प्रसन्नोऽहं ब्रवीमि ते ॥वैशंपायन उवाच}
{}


\twolineshloka
{ततो हृष्टमना राजा बृहदश्वमुवाच ह}
{भगवन्नक्षहृदयं ज्ञातुमिच्छामि तत्त्वतः}


\fourlineindentedshloka
{`कौन्तेयेनैवमुक्तस्तु बृहदश्वो महामुनिः'}
{ततोऽक्षहृदयं प्रादात्पाण्डवाय महात्मने}
{`लब्ध्वा च पाण्डवो राजा विशोकः समपद्यत ॥बृहदश्व उवाच}
{}


\twolineshloka
{पुनरेव तु वक्ष्यामि यस्त्वत्तो दुःखितो नृपः}
{तं शृणुष्व महाराज सर्वदुःखापनुत्तये}


\twolineshloka
{इक्ष्वाकूणां कुले जातो महात्मा पृथिवीपतिः}
{त्रिशङ्कुरिति विख्यातोराजराजो महाद्युतिः}


\twolineshloka
{हरिश्चन्द्रस्ततो जज्ञे गुणरत्नाकरो नृपात्}
{ततो विशेषैर्विविधैर्यज्ञैर्विपुलदक्षिणैः}


\twolineshloka
{स तु लोके वरः पुंसां पुण्यश्लोको महायशाः}
{सत्यवादी मधुरवाक्सत्येन बहुभाषिता}


\twolineshloka
{तस्य भार्याऽभवद्भूमौ सौशील्यसमलंकृता}
{उशीनरस्य राजर्षेर्दुहिता पुण्यलक्षणा}


\twolineshloka
{स्वयंवरे महाभागं वरयामास भामिनी}
{हरिश्चन्द्रं समेतानां राज्ञां मद्ये पतिं विभुम्}


\twolineshloka
{तया सह महीपालः सत्यवत्या मनोज्ञया}
{रेमे च सुचिरं कालं राजा राज्यमवाप्य च}


\twolineshloka
{तस्यां देव्यां हरिश्चन्द्राज्जज्ञे राजीवलोचनः}
{पुत्रः पुण्यवतां श्रेष्ठो लोहिताश्व इति श्रुतः}


\twolineshloka
{देव्या पुत्रेण सहितः पुण्यश्लोको महायशाः}
{वसिष्ठयाज्यो नृपतिरीजे शुण्यैर्महाध्वरैः}


\twolineshloka
{एतस्मिन्नेव काले तु विश्वामित्रो दिवं गतः}
{पुरुहूतपुरीं रम्यामाजगामेन्द्रसेवया}


\twolineshloka
{उपस्थाने च संवृत्ते देवेन्द्रस्य महात्मनः}
{आजगाम वसिष्ठोऽपि वामदेवसहायवान्}


\twolineshloka
{उपस्थाने च संवृत्ते सुखासीने पुरंदरे}
{वर्ण्यमानेषु च तदा सत्यवादिषु राजसु}


\twolineshloka
{तस्यां संसदिसर्वस्माद्धरिश्चन्द्रोऽपि पप्रथे}
{यज्ञदानतपःशीलसत्यवाक्यदृझव्रतैः}


\twolineshloka
{वसिष्ठः परमप्रीतः स्वयाज्यपरिकीर्तनात्}
{तथापि विश्वामित्रस्तं न सेहे सत्यभूषितम्}


\twolineshloka
{हरिश्चन्द्रं प्रति तदा विश्वामित्रवसिष्ठयोः}
{पणः कृतस्तदा पश्चाद्विश्वामित्रेण पार्थिवः}


\threelineshloka
{राज्याच्चापि सुखाच्चापि सहसा चावरोपितः}
{अवाप परमं दुःखं मरणादमनोहरम् ॥वैशंपायन उवाच}
{}


\threelineshloka
{तच्छ्रुत्वा परमप्रीतो धर्मराजो युधिष्ठिरः}
{भ्रातृभिर्ब्राह्मणैश्चैव द्रौपद्या च समन्वितः}
{विस्मयं परमं गत्वा साधुसाध्वित्यभाषत}


\twolineshloka
{ततो हरिश्चन्द्रकथां च सर्वेश्रुत्वा तु राजा मनुजेन्द्रकेतुः}
{विहाय शोकं विजहार भूयःस्मरन्हरिश्चन्द्रमनन्तकीर्तिम्}


\threelineshloka
{कथामेवं तथा कृत्वा हरिश्चन्द्रनलाश्रयाम्}
{आमन्त्र्य पाण्डवान्सर्वान्बृहदश्वो जगाम ह'}
{उक्त्वा चाशु सरोऽगच्छदुपस्प्रष्टुं महातपाः}


\twolineshloka
{बृहदश्व गते पार्थमश्रौषीत्सव्यसाचिनम्}
{वर्तमनं तपस्युग्रे वायुभक्षं मनीषिणम्}


\twolineshloka
{ब्राह्मणएभ्यस्तपस्विभ्यः संपतद्भ्यस्ततस्ततः}
{तीर्थशैलवनेभ्यश्च समेतेभ्यो दृढव्रतः}


\twolineshloka
{इतिपार्थो महाबाहुर्दुरापं तप आस्थितः}
{न तथा दृष्टपूर्वोऽन्यः कश्चिदुग्रतपा इति}


\twolineshloka
{यथा धनंजयः पार्थस्तपस्वी नियतव्रतः}
{मुनिरेकचरः श्रीमान्धर्मो विग्रहवानिव}


\twolineshloka
{तं श्रुत्वा पाण्डवो राजंस्तप्यमानं महावने}
{अन्वशोचत कौन्तेयः प्रियं वै भ्रातरं जयम्}


\twolineshloka
{दह्यमानेन तु हृदा शरणार्थी महावने}
{ब्राह्मणान्विविधज्ञानान्पर्यपृच्छद्युधिष्ठिरः}


\twolineshloka
{`प्रतिगृह्याक्षहृदयं कुन्तीपुत्रो युधिष्ठिरः}
{आसीद्धृष्टमना राजन्भीमसेनादिभिर्युतः}


\twolineshloka
{स्वभ्रातॄन्सहितान्पश्यन्कुन्तीपुत्रो युधिष्ठिरः}
{अपश्यन्नर्जुनं तत्रबभूवाश्रुपरिप्लुतः}


\twolineshloka
{संतप्यमानः कौन्तेयो भीमसेनमुवाच ह}
{कदा द्रक्ष्यामि वै भीम पार्तमत्र तवानुजम्}


\twolineshloka
{मत्कृते हि कुरुश्रेष्ठ तष्यते परमं तपः}
{तस्याक्षहृदयज्ञानमाख्यास्यामि कदा न्बहम्}


\twolineshloka
{स हि श्रुत्वाऽक्षहृदयं समुपात्तं मया विभो}
{प्रहृष्टः पुरुषव्याघ्रो भविष्यति न संशयः'}


\chapter{अध्यायः ७८}
\twolineshloka
{जनमेजय उवाच}
{}


\twolineshloka
{भगवान्काम्यकात्पार्थे गते मे प्रपितामहे}
{पाण्डवाः किमकुर्वंस्ते तमृते सव्यसाचिनम्}


\twolineshloka
{स हि तेषां महेष्वासो गतिरासीदनीकजित्}
{आदित्यानां यथा विष्णुस्तथैव प्रतिभाति मे}


\threelineshloka
{तेनेन्द्रसमवीर्येण संग्रामेष्वनिवर्तिना}
{विनाभूता वने वीराः कथमासन्पितामहाः ॥वैशंपायन उवाच}
{}


\twolineshloka
{गते तु पाण्डवेतात काम्यकात्सव्यसाचिनि}
{बभूवुः कौरवेयास्ते दुःखशोकपरायणाः}


\twolineshloka
{आक्षिप्तसूत्रा मणयश्छिन्नपक्षा इवाण्डजाः}
{अप्रीतमनसः सर्वे बभूवुरथ पाण्डवाः}


\twolineshloka
{वनं तु तदभूत्तेन हीनमक्लिष्टकर्मणा}
{कुबेरेण यथाहीनं वनं चैत्ररथं तथा}


\twolineshloka
{तमृते ते नरव्याघ्राः पाण्डवा जनमेजय}
{मुदमप्राप्नुवन्तो वै काम्यके न्यवसंस्तदा}


\twolineshloka
{ब्राह्मणार्थे पराक्रान्ताः शुद्धैर्वाणैर्महारथाः}
{निघ्नन्तो भरतश्रेष्ठ मेध्यान्बहुविधान्मृगान्}


\twolineshloka
{नित्यंहि पुरुषव्याघ्रा वन्याहारमरिंदमाः}
{प्रविसृत्य समाहृत्य ब्राह्मणेभ्यो न्यवेदयन्}


\twolineshloka
{एवं ते न्यवसंस्तत्रसोत्कणअठाः पुरुषर्षभाः}
{अहृष्टमनसः सर्वेगते राजन्धनंजये}


\twolineshloka
{अथ विप्रोषितं राजन्पाञ्चाली मध्यमं पतिम्}
{स्मरन्ती पाण्डवश्रेष्ठमिदं वचनमब्रवीत्}


\twolineshloka
{योऽर्जुनेनार्जुनस्तुल्यो द्विबाहुर्बहुबाहुना}
{तमृते पाण्डवश्रेष्ठं वनं न प्रतिभाति मे}


\twolineshloka
{शून्यामिव प्रपश्यामि तत्रतत्र महीमिमाम्}
{बह्वाश्चर्यमिदं चापि वनं कुसुमितद्रुमम्}


\twolineshloka
{न तथा रमणीयं वै तमृते सव्यसाचिनम्}
{नीलाम्बुदसमप्रख्यं मत्तमातङ्गगामिनम्}


\threelineshloka
{तमृते पुण्डरीकाक्षं काम्यकं नातिभाति मे}
{यस्य वा धनुषो घोषः श्रूयते चाशनिस्वनः}
{न लभे शर्म वै राजन्स्मरन्ती सव्यसाचिनम्}


\twolineshloka
{तथा लालप्यमानां तां निशाम्य परवीरहा}
{भीमसेनो महाराज द्रौपदीमिदमब्रवीत्}


\twolineshloka
{मनःप्रीतिकरं भद्रे यद्ब्रवीषि सुमध्यमे}
{तन्म प्रीणाति हृदयममृतप्राशनोपमम्}


\twolineshloka
{यस्य दीर्घौ समौ पीनौ भुजौ परिघसन्निभौ}
{मौर्वीकृतकिणौ वृत्तौ खङ्गायुधधनुर्धरौ}


\twolineshloka
{निष्काङ्गदकृतापीडौ पञ्चशीर्षाविवोरगौ}
{तमृते पुरुषव्याघ्रं नष्टसूर्यमिवाम्बरम्}


\twolineshloka
{यमाश्रित्य महाबाहुं पाञ्चालाः कुरवस्तथा}
{सुराणामपि यत्तानां पृतनासु न बिभ्यति}


\twolineshloka
{यस्य बाहू समाश्रित्य वयं सर्वेमहात्मनः}
{मन्यामहे जितानाजौ परान्प्राप्तां च मेदिनीम्}


\twolineshloka
{तमृते फल्गुनं वीरं न लभे काम्यके धृतिम्}
{पश्यामि च दिशः सर्वास्तिमिरेणावृता इव}


\threelineshloka
{ततोऽब्रवीत्साश्रुकण्ठो नकुलः पाण्डुनन्दनः}
{यस्मिन्दिव्यानि कर्माणि कथयन्ति रणाजिरे}
{देवा अपि युधांश्रेष्ठं तमृतेका रतिर्वने}


\twolineshloka
{उदीचीं चो दिशं गत्वा जित्वा युधि महाबलान्}
{गन्धर्वमुख्याञ्शतशो हर्याँल्लेबे महाद्युतिः}


\twolineshloka
{राज्ञे तित्तिरिकल्माषाञ्श्रीमतोऽनिलरंहसः}
{प्रादाद्भात्रे प्रियः प्रेम्णा राजसूये महाक्रतौ}


\threelineshloka
{तमृतेभीमधन्वानं भीमादवरजं वने}
{कामये काम्यके वासं नेदानीममरोपमम् ॥सहदेव उवाच}
{}


\threelineshloka
{यो धनानिन कन्याश्च युधि हित्वा महाबलान्}
{`शतशो घातयित्वाऽरीन्पृतनामध्यगस्तदा'}
{आजहार पुरा राज्ञे राजसूये महाक्रतौ}


\twolineshloka
{यः समेतान्मृधे जित्वायादवानमितद्युतिः}
{सुभद्रामाजहारैको वासुदेवस्य संमते}


\twolineshloka
{`येनार्धराज्यमाच्छिद्य द्रुपदस्य महात्मनः}
{आचार्यदक्षिणा दत्ता रणे द्रोणश्यभारत'}


\twolineshloka
{तस्य जिष्णोर्बृसीं दृष्ट्वा सून्यत्रेव निवेशने}
{हृदयं मे महाराज न शाम्यात्रकदाचन}


\twolineshloka
{विनादस्माद्विवासं तु रोचयेऽहमरिंदम}
{न हि नस्तमृते वीरं रमणीयमिदं वनम्}


\chapter{अध्यायः ७९}
\twolineshloka
{वैशंपायन उवाच}
{}


\twolineshloka
{`धनञ्जयोत्सुकास्ते तु वने तस्मिन्महारथाः}
{न्यवसन्त महाभागा द्रौपद्या सह कृष्णया'}


\twolineshloka
{धनं जयोत्सुकानां तु भ्रातॄणां कृष्णया सह}
{श्रुत्वा वाक्यानि विमना धर्मराजोप्यजायत}


\twolineshloka
{अथापश्यन्महात्मानं देवर्षिं तत्र नारदम्}
{दीप्यमानं श्रिया ब्राह्म्या दीप्ताग्निसमतेजसम्}


\twolineshloka
{तमागतमभिप्रेक्ष्य भ्रातृभिः सह धर्मराट्}
{प्रत्युत्थाय यथान्यायं पूजां चक्रे महात्मने}


\twolineshloka
{स तैः परिवृतः श्रीमान्भ्रातृभिः कुरुसत्तमः}
{विबभावतिदीप्तौजा देवैरिव शतक्रतुः}


\twolineshloka
{यथा च वेदान्सावित्री याज्ञसेनी तथा पतीन्}
{न जहौ धर्मतः पार्थान्मेरुमर्कप्रभा यथा}


\twolineshloka
{`अर्ध्यं पाद्यमथानीय त्वभ्यवायदच्युतः}
{नारदस्तु महातेजाः स्वस्त्यस्त्वित्यभ्यभाषत}


\twolineshloka
{ततो युधिष्ठिरो राजा दृष्ट्वा देवर्षिसत्तमम्}
{यथार्हं पूजयामास विधिवत्कुरुनन्दनः'}


\twolineshloka
{प्रतिगृह्य च तां पूजां नारदो भगवानृषिः}
{आश्वासयद्धर्मसुतं युक्तरूपमिवानघ}


\twolineshloka
{उवाच च महात्मानं धर्मराजं युधिष्ठिरम्}
{ब्रूहि धर्मभृतां श्रेष्ठ केनार्थः किं ददानि ते}


\twolineshloka
{अथ धर्मसुतो राजा प्रणम्य भ्रातृभिः सह}
{उवाच प्राञ्जलिर्भूत्वा नारदं देवसंमितम्}


\twolineshloka
{न्वयि तुष्टे महाभाग सर्वलोकाभिपूजिते}
{कृतमित्येव मन्येऽहं प्रसादात्तव सुव्रत}


\twolineshloka
{यदि त्वहमनुग्राह्यो भ्रातृभिः सहितोऽनघ}
{संदेहं मे सुनिश्रेष्ठ तत्वतश्छेत्तुमर्हसि}


\threelineshloka
{प्रदक्षिणां यः कुरुते पृथिवीं तीर्थतत्परः}
{किं फलं तस्य कार्त्स्न्येन तद्भवान्वक्तुमर्हति ॥नारद उवाच}
{}


\twolineshloka
{शृणु राजन्नवहितो यथा भीष्मेण धीमता}
{पुलस्त्यस्य सकाशाद्वै सर्वमेतदुपश्रुतम्}


\twolineshloka
{पुरा भागीरथीतीरे भीष्मो धर्मभृतांवरः}
{पित्र्यं व्रतं समास्थाय न्यवसन्मुनिभिः सह}


\twolineshloka
{शुभे देशे तथा राजन्पुण्ये देवर्षिसेविते}
{गङ्गाद्वारे महाभाग देवगन्धर्वसेविते}


\twolineshloka
{स पितॄंस्तर्पयामास देवांश्च परमद्युतिः}
{ऋषींश्च तर्पयामास विधिदृष्टेन कर्मणा}


\twolineshloka
{कस्य चित्त्वथ कालस्य जपन्नेव महायशाः}
{ददर्शाद्भुतसंकाशं पुलस्त्यमृषिसत्तमम्}


\twolineshloka
{स तं दृष्ट्वोग्रतपसं दीप्यमानमिव श्रिया}
{प्रहर्षमतुलं लेभे विस्मयं परमं ययौ}


\twolineshloka
{उपस्थितं महाभागं पूजयामास भारत}
{भीष्मो धर्मभृतां श्रेष्ठो विधिदृष्टेन कर्मणा}


\twolineshloka
{शिरसा चाघमादाय शुचिः प्रयतमानसः}
{नाम संकीर्तयामास तस्मिन्ब्रह्मर्षिसत्तमे}


\twolineshloka
{भीष्मोऽहमस्मि भद्रं ते दासोऽस्मि तव सुव्रत}
{तव संदर्शनादेव मुक्तोऽहं सर्वकिल्वपैः}


\twolineshloka
{रएवमुक्त्वा महाराज भीष्मो धर्मभृतांवरः}
{वाग्यतः प्राञ्जलिर्भूत्वातूष्णीमासीद्युधिष्ठिर}


\twolineshloka
{तं दृष्ट्वा नियमेनाथ स्वाध्यायाम्नायकर्शितम्}
{भीष्मं कुरुकुलश्रेष्ठं मुनिः प्रीतमनाऽभवत्}


\twolineshloka
{`ततः स मधुरेणाथ स्वरेण सुमहातपाः}
{उवाच वाक्यं धर्मज्ञः पुलस्त्यः प्रीतमानसः'}


\chapter{अध्यायः ८०}
\twolineshloka
{पुलस्त्य उवाच}
{}


\twolineshloka
{अनेन तव धर्मज्ञ प्रश्रयेण दमेन च}
{सत्येन च महाभाग तुष्टोस्मि तव सुव्रत}


\twolineshloka
{यस्येदृशस्ते धर्मोऽयं पितृवाक्याश्रितोऽनघ}
{तेन पश्यसि मां पुत्र प्रीतिश्च परमा त्वयि}


\threelineshloka
{अमोघदर्शी भीष्माहं ब्रूहि किं करवाणि ते}
{यद्वक्ष्यसि कुरुश्रेष्ठ तस्य दाताऽस्मि तेऽनघ ॥भीष्म उवाच}
{}


\twolineshloka
{प्रीते त्वयि महाभाग सर्वलोकाभिपूजिते}
{कृतमित्येव मन्येऽहं यदहं दृष्टवान्प्रभुम्}


\twolineshloka
{यदि त्वहमनुग्राह्यस्तव धर्मभृतांवर}
{संदेहं ते प्रवक्ष्यामि तन्मे त्वं छेत्तुमर्हसि}


\twolineshloka
{अस्ति मे भगवन्कश्चित्तीर्थानि प्रति संशयः}
{तमहं श्रोतुमिच्छामि तद्भवान्वक्तुमर्हति}


\threelineshloka
{प्रदक्षिणां यः पृथिवीं करोत्यमरसन्निभ}
{किं फलं तस्य विप्रर्षे तन्मे ब्रूहि तषोधन ॥पुलस्त्य उवाच}
{}


\twolineshloka
{हन्त तेऽहंप्रवक्ष्यामि यदृषीणां परायणम्}
{तदेकाग्रमनास्तात शृणु तीर्थेषु यत्फलम्}


\twolineshloka
{यस्य हस्तौ च पादौ च मनश्चैव सुसंयतम्}
{विद्या तपश्च कीर्तिश्च स तीर्थफलमश्नुते}


\twolineshloka
{प्रतिग्रहादपावृत्तः संतुष्टो येन केनचित्}
{अहंकारनिवृत्तश्च स तीर्थफलमश्नुते}


\twolineshloka
{अकल्कको निरारम्भो लध्वाहारो जितेन्द्रियः}
{विमुक्तः सर्वपापेभ्यः स तीर्थफलमश्नुते}


\twolineshloka
{अक्रोधनश्च राजेन्द्र सत्यशीलो दृढव्रतः}
{आत्मोपमश्च भूतेषु स तीर्धफलमश्नुते}


\twolineshloka
{ऋषिभिः क्रतवः प्रोक्ता देवेष्विह यथाक्रमम्}
{फलं चैव यथातत्त्वं प्रेत्य चेह च सर्वशः}


\twolineshloka
{न ते शक्या दरिद्रेण यज्ञाः प्राप्तुं महीपते}
{बहूपकरणा यज्ञा नानासंभारविस्तराः}


\twolineshloka
{प्राप्यन्ते पार्थिवैरेतैः समृद्धैर्वा नरैः क्वचित्}
{नार्थन्यूनोपकरणैरेकात्मभिरसंहतैः}


\twolineshloka
{यो दरिद्रैरपि विधिः शक्यः प्राप्तुं नरेश्वर}
{तुल्यो यज्ञफलैः पुण्यैस्तं निबोध युधांवर}


\twolineshloka
{ऋषीणां परमं गुह्यमिदं भरतसत्तम}
{तीर्थाभिगमनं पुण्यं यज्ञैरपि विशिष्यते}


\twolineshloka
{अनपोष्य त्रिरात्राणि तीर्थान्यनभिगम्य च}
{अदत्त्वा काञ्चनं गाश्च दरिद्रो नाम जायते}


\twolineshloka
{अग्निष्टोमादिभिर्यज्ञैरिष्ट्वा विपुलदक्षिणैः}
{न तत्फलमवाप्नोति तीर्थाभिगमनेन यत्}


\twolineshloka
{सर्वतीर्थेषु राजेन्द्र तीर्थं त्रैलोक्यविश्रुतम्}
{पुष्करं नाम विख्यातं महाभागः समाविशेत्}


\twolineshloka
{दशकोटिसहस्राणि तीर्थानां वै महामते}
{सान्निध्यं पुष्करे येषां त्रिसन्ध्यं कुरुनन्दन}


\twolineshloka
{आदित्या वसवो रुद्राः साध्याश्च समरुद्गणाः}
{गन्धर्वाप्सरसश्चैव नित्यं सन्निहिता विभो}


\threelineshloka
{यत्र देवास्तपस्तप्त्वा दैत्या ब्रह्मर्षयस्तथा}
{`तपोविशेषैर्बहुभिः स्थानान्यापुर्महौजसः'}
{दिव्ययोगा महाराज पुण्येन महताऽन्विताः}


\twolineshloka
{मनसाभ्येतुकामस्य पुष्कराणि मनस्विनः}
{पूयन्ते सर्वपापानि नाकपृष्ठे च पूज्यते}


\twolineshloka
{तस्मिंस्तीर्थे महाभागो नित्यमेव पितामहः}
{उवास परमप्रीतो देवदानवसत्तमः}


\twolineshloka
{पुष्करेषु महाभाग देवाः सर्पिगणाः पुरा}
{सिद्धिं समभिसंप्राप्ताः पुण्येन महताऽन्विताः}


\twolineshloka
{तत्राभिषेकं यः कुर्यात्पितृदेवांश्च तर्पयेत्}
{सर्वपापविनिर्मुक्तो ब्रह्मलोके च पूज्यते}


\twolineshloka
{अप्येकं भोजयेद्विप्रं पुष्करारण्यमाश्रितः}
{तेनासौ कर्मणा भीष्म प्रेत् यचेह च मोदते}


\threelineshloka
{शाकैर्मूलैः फलैर्वाऽपिचेन वर्तयते स्वयम्}
{तद्वै दद्याद्ब्राह्मणाय श्रद्धावाननसूयकः}
{तेनैव प्राप्नुयात्प्राज्ञो हयमेधफलं नरः}


\twolineshloka
{`अपि वाऽप्युदपात्रेण ब्राह्मणान्स्वस्ति वाचयेत्}
{तेनापि पूजनेनाशु प्रेत्यानन्त्याय कल्पते'}


\twolineshloka
{ब्राह्मणाः क्षत्रिया वैश्याः शूद्रा वा राजसत्तम}
{न वै योनौ प्रजायन्ते स्नातास्तीर्थे महात्मनः}


\twolineshloka
{कार्तिक्यां तु विशेषेण योऽभिगच्छति पुष्करम्}
{`फलं तत्राक्षयं तेन लभते भरतर्षभ'}


\threelineshloka
{सायंप्रातः स्मरेद्यस्तु पुष्कराणि कृताञ्जलिः}
{उपस्पृष्टं भवेत्तेन सर्वतीर्थेषु भारत}
{प्राप्नुयाच्च नरो लोकान्ब्रह्मणः सदनेऽक्षयान्}


\twolineshloka
{जन्मप्रभृतियत्पापं स्त्रिया वा पुरुषस् वा}
{पुष्करे स्नातमात्रस्य सर्वमेव प्रणश्यति}


\twolineshloka
{यथा सुराणां सर्वेषामादिस्तु मधुसूदनः}
{तथैव पुष्करं राजंस्तीर्थानामादिरुच्यते}


\twolineshloka
{उष्य द्वादशवर्षाणि पुष्करे नियतः शुचिः}
{क्रतून्सर्वानवाप्नोति ब्रह्मलोकं स गच्छति}


\twolineshloka
{यस्तु वर्षशतं पूर्णमग्निहोत्रमुपासते}
{कार्तिकीं वा वसेदेकां पुष्करे सममेव तत्}


\twolineshloka
{[त्रीणि शृङ्गाणि शुभ्राणि त्रीणि प्रस्रवणानि च}
{पुष्कराण्यादिसिद्धानि न विद्मस्तत्र कारणम् ॥]}


\twolineshloka
{दुष्करं पुष्करं गन्तुं दुष्करं पुष्करे तपः}
{दुष्करं पुष्करे दानं वस्तुं चैव सुदुष्करम्}


\twolineshloka
{उष्य द्वादशरात्रं तु नियतो नियताशनः}
{प्रदक्षिणमुपावृत्य जम्बूमार्गं समाविशेत्}


\twolineshloka
{जम्बूमार्गं समाक्श्यि देवर्षिपितृसेवितम्}
{अश्वमेधमवाप्नोति विष्णुलोकं च गच्छति}


\twolineshloka
{तत्रोष्य रजनीः पञ्च कषष्ठकालक्षमी नरः}
{न दुर्गतिमवाप्नोति विष्णुलोकं च गच्छति}


\twolineshloka
{`तत्रगत्वा महाप्राज्ञः कुर्याच्छ्राद्धं दृढव्रतः}
{वाजपेयमवाप्नोति दुष्कृतं चास्य नश्यति'}


\twolineshloka
{जम्बूमार्गादुपावृकत्यगच्छेत्स्थण्डिलकाश्रमम्}
{न दुर्गतिमवाप्नोति ब्रह्मलोकं च गच्छति}


\twolineshloka
{आगस्त्यं सर आसाद्य पितृदेवार्चने रतः}
{त्रिरात्रोपोषितो राजन्नग्निष्टोमफलं लभेत्}


\twolineshloka
{शाकवृत्तिः फलैर्वाऽपि कौमारं विन्दते पदम्}
{कण्वाश्रमं ततो गच्छेच्ध्रीजुष्टं लोकपूजितम्}


\twolineshloka
{धर्मारण्यं हि तत्पुण्यमाद्यं च भरतर्षभ}
{यत्र प्रविष्टमात्रो वै सर्वपापैः प्रमुच्यते}


\twolineshloka
{अर्चयुत्वा पितॄन्देवान्नियतो नियताशनः}
{सर्वकामसमृद्धः स्याद्यज्ञस्य फलमश्नुते}


\twolineshloka
{प्रदक्षिणं ततः कृत्वा ययातिपतनं व्रजेत्}
{हयमेधस्य यज्ञस्य फंल प्राप्नोति तत्र वै}


\twolineshloka
{महाकालं ततो गच्छेन्नियतो नियताशनः}
{कोटितीर्थमुपस्पृश्य हयमेधफलं लभेत्}


\twolineshloka
{ततो गच्छेत धर्मज्ञः स्थाणोस्तीर्थमुमापतेः}
{नाम्ना भद्रवटं नाम त्रिषु लोकेषु विश्रुतम्}


\twolineshloka
{तत्राभिगम्य चेशानं गोसहस्रफलं लभेत्}
{महादेवप्रसादाच्च गाणपत्यं च विन्दति}


\twolineshloka
{समृद्धमसपत्नं च श्रिया युक्तं नरोत्तमः}
{`राज्ञां चैवाधिपत्यं हि तत्र गत्वा समाप्नुयात् ॥'}


\twolineshloka
{नर्मदां स समासाद्य नदीं त्रैलोक्यविश्रुताम्}
{तर्पयित्वा पितृन्देवानग्निष्टोमफलं लभेत्}


\twolineshloka
{दक्षिमं सिन्धुमासाद्य ब्रह्मचारी जितेन्द्रियः}
{अग्निष्टोममवाप्नोति विमानं चाधिरोहति}


\twolineshloka
{चर्मण्वतीं समासाद्य नियतो नियताशनः}
{रन्तिदेवाभ्यनुज्ञातमग्निष्टोमफलं लभेत्}


\twolineshloka
{ततो गच्छेत धर्मज्ञं हिमवत्सुतमर्बुदम्}
{पृथिव्यां यत्रवै छिद्रं पूर्वमासीद्युधिष्ठिर}


\twolineshloka
{तत्राश्रमो वसिष्ठस्य त्रिषु लोकेषु विश्रुतः}
{तत्रोष्य रजनीमेकां गोसहस्रफलं लभेत्}


\twolineshloka
{पिङ्गतीर्थमुपस्पृश्य ब्रह्मचारी जितेन्द्रियः}
{कपिलानां नरश्रेष्ठ शतस् फलमश्नुते}


\twolineshloka
{ततो गच्छेत राजेन्द्र प्रभासं लोकविश्रुतम्}
{`तीर्थं देवगणैः पूज्यमृषिभिश्च निषेवितम्'}


\twolineshloka
{देवतानां मुखं वीर ज्वलनोऽनिलसारथिः}
{देवतानां मुखं वीर ज्वलनोऽनिलसारथिः}


\twolineshloka
{तस्मिंस्तीर्थे नरः स्नात्वा शुचिः प्रयतमानसः}
{अग्निष्टोमातिरात्राभ्यां फलं प्राप्नोति मानवः}


\threelineshloka
{ततो गत्वा सरस्वत्याः सागरस्य च संगमे}
{गोसहस्रफलं तस्य स्वर्गलोकं च विन्दति}
{प्रभया दीप्यते नित्यमग्निवद्भरतर्षभ}


\threelineshloka
{वरदानं ततो गच्छेत्तीर्थं भरतसत्तम}
{विष्णोर्दुर्वाससा यत्र वरो दत्तो युधिष्ठिर}
{प्रभासते यथा सोमः सोश्वमेधं च विन्दति}


\threelineshloka
{वरदानं ततो गच्छेत्तीर्थं भरतसत्तम}
{विष्णोर्दुर्वाससा यत्र वरो दत्तो युधिष्ठिर}
{वरदाने नरः स्नात्वा गोसहस्रफंल लभेत्}


\twolineshloka
{ततो द्वारवतीं गच्छेन्नियतो नियताशनः}
{पिण्डारके नरः स्नात्वा लभेद्बहुसुवर्णकम्}


\twolineshloka
{तस्मिंस्तीर्थे महाभाग पद्मलक्षणलक्षिताः}
{अद्यापि मुद्रा दृश्यन्ते तदद्भुतमरिंदम}


\twolineshloka
{त्रिशूलाङ्कानि पद्मानि दृश्यन्ते कुरुनन्दन}
{महादेवस्य सांनिध्यं तत्र वै पुरुषर्षभ}


\twolineshloka
{सागरस्य च सिन्धोश्च संगमं प्राप्य दुर्लभम्}
{तीर्थे सलिलराजस्य स्नात्वा प्रयतमानसः}


\twolineshloka
{तर्पयित्वा पितॄन्देवानृषींश्च भरतर्षभ}
{प्राप्नोति वारुणं लोकं दीप्यमानं स्वतेजसा}


\twolineshloka
{शङ्कुकर्णेश्वरं देवमर्चयित्वा युधिष्ठिर}
{अश्वमेधाद्दशगुणं प्रवदन्ति मनीषिणः}


\twolineshloka
{प्रदक्षिणमुपावृत्य गच्छेत भरतर्षभ}
{तीर्थं कुरुवरश्रेष्ठ त्रिषु लोकेषु विश्रुतम्}


\twolineshloka
{शमीति नाम्ना विख्यातं सर्वपापप्रणाशनम्}
{तत्रब्रह्मादयो देवा उपासन्ते महेश्वरम्}


\twolineshloka
{तत्रस्नात्वाऽर्चयित्वा च रुद्रं देवगणैर्वृतम्}
{जन्मप्रभृतियत्पापं तत्स्नातस्य प्रणश्यति}


\twolineshloka
{शमी चात्रनरश्रेष्ठ सर्वदेवैरभिष्टुता}
{तत्र स्नात्वा नरव्याघ्र हयमेधमवाप्नुयात्}


\twolineshloka
{गत्वा यत्रमहाप्राज्ञ विष्णुना प्रभविष्णुना}
{पुरा शौचं कृतंराजन्हत्वा दैवतकण्टकान्}


\twolineshloka
{ततो गच्छेत धर्मज्ञ वसोर्धारामभिष्टुताम्}
{गमनादेवतस्यां हि हयमेधफलं लभेत्}


\twolineshloka
{स्नात्वा कुरुवरश्रेष्ठ प्रयतात्मा समाहितः}
{तर्प्य देवान्पितॄंश्चैव विष्णुलोके महीयते}


\twolineshloka
{तीर्थे चात्र सर पुण्यं वसूनां भरतर्षभ}
{तत्र स्नात्वा च पीत्वा च वसूनां संमतोभवेत्}


\twolineshloka
{सिन्धूत्तममिति ख्यातं सर्वपापप्रणाशनम्}
{तत्र स्नात्वा नरश्रेष्ठ लभेद्बहुसुवर्णकम्}


\twolineshloka
{ब्रह्मतीर्थं समासाद्य शुचिः शीलसमन्वितः}
{ब्रह्मलोकमवाप्नोति गतिं च परमां व्रजेत्}


\twolineshloka
{कुमारिकाणां शक्रस्य तीर्थं सिद्धनिषेवितम्}
{तत्र स्नात्वा नरः क्षिप्रं स्वर्गलोकमवाप्नुयात्}


\twolineshloka
{रेणुकायाश्च तत्रैव तीर्थं सिद्धनिषेवितम्}
{तत्र स्नात्वा भवेद्विप्रो निर्मलश्चन्द्रमा यथा}


\twolineshloka
{अथ पञ्चनदं गत्वा नियतो नियताशनः}
{पञ्चयज्ञानवाप्नोति क्रमशो येऽनुकीर्तिताः}


\twolineshloka
{ततो गच्छेत राजेन्द्र भीमायाः स्थानमुत्तमम्}
{तत्रस्नात्वा न योन्यां वै भवेद्भरतसत्तम}


\twolineshloka
{देव्याः पुत्रो भवेद्राजंस्तप्तकुण्डलभूषणः}
{गवां शतसहस्रस्य फलं प्राप्नोति मानवः}


\twolineshloka
{गिरिकुञ्जं समासाद्य त्रिषु लोकेषु विश्रुतम्}
{पितामहं नमस्कृत्य गोसहस्रफलं लभेत्}


\twolineshloka
{ततो गच्छेत धर्मज्ञ विमलं तीर्थमुत्तमम्}
{अद्यापि यत्रदृश्यन्ते मत्स्याः सौवर्णराजताः}


\twolineshloka
{तत्र स्नात्वा नरः क्षिप्रं वासवं लोकमाप्नुयात्}
{सर्वपापविशुद्धात्मा गच्छेत परमां गतिम्}


\twolineshloka
{वितस्तां च समासाद्य संतर्प्य पितृदेवताः}
{नरः फलमवाप्नोति वाजपेयस्य भारत}


\twolineshloka
{काश्मीरष्वेव नागस्य भवनं तक्षकस्य च}
{वितस्ताख्यमिति ख्यातं सर्वपापप्रमोचनम्}


\twolineshloka
{तत्रस्नात्वा नरो नूनं वाजपेयमवाप्नुयात्}
{सर्वपापविशुद्धात्मा गच्छेच्च परमां गतिम्}


\twolineshloka
{ततो गच्छेत मलदां त्रिषु लोकेषु विश्रुताम्}
{पश्चिमायां तु संध्यायामुपस्पृश्य यथाविधि}


\twolineshloka
{चरुं सप्तार्चिषे राजन्यथाशक्ति निवेदयेत्}
{पितॄणामक्षयं दानं प्रवदन्ति मनीषिणः}


\twolineshloka
{ऋषयः पितरो देवा गन्धर्वाप्सरसां गणाः}
{गुह्यकाः किन्नरा यक्षाः सिद्धा विद्याधरा नराः}


\twolineshloka
{राक्षसा दितिजा रुद्रा ब्रह्मा च मनुजाधिप}
{नियतः परमां दीक्षामास्थायाब्दसहस्रिकीम्}


\twolineshloka
{विष्णोः प्रसादनं कुर्वंश्चरुं च श्रपयंस्तथा}
{सप्तभिः सप्तभिश्चैव ऋग्भिस्तुष्टाव केशवम्}


\twolineshloka
{ददावष्टगुणैश्वर्यं तेषां तुष्टस्तु केशवः}
{यथाभिलषितानन्यान्कामान्दत्वा महीपते}


\twolineshloka
{तत्रैवान्तर्दधे देवो विद्युदभ्रेषु वै यथा}
{नाम्ना सप्तचरुं तेन ख्यातं लोकेषु भारत}


\twolineshloka
{गवां शतसहस्रेण राजसूयशतेन च}
{अश्वमेधसहस्रेण श्रेयान्सप्तार्चिषे चरुः}


\threelineshloka
{ततो निवृत्तो राजेन्द्र रुद्रं पदमथाविशेत्}
{अर्चयित्वामहादेवमश्वमेधफलं लभेत्}
{}


\twolineshloka
{मणिमन्तं समासाद्य ब्रह्मचारी समाहितः}
{एकरात्रोपितो राजन्नग्निष्टोमफलं लभेत्}


\twolineshloka
{अथ गच्छेत राजेन्द्र देविकां लोकविश्रुतम्}
{प्रसूतिर्यत्र विप्राणां श्रूयते भरतर्षभ}


\twolineshloka
{त्रिशूलपाणएः स्थानं च त्रिषु लोकेषु विश्रुतम्}
{देविकायां नरः स्नात्वा समभ्यर्च्य महेश्वरम्}


\twolineshloka
{यथाशक्ति चरुं तत्र निवेद्य भरतर्षभ}
{सर्वकामसमृद्धस्य यज्ञस्य लभते फलम्}


\twolineshloka
{कामाख्यं तत्र रुद्रस्य तीर्थं देवनिषेवितम्}
{तत्रस्नात्वा नरः क्षिप्रं सिद्धिं प्राप्नोति भारत}


\twolineshloka
{यजनं याजनं चैवतथैव ब्रह्मबालुकम्}
{पुष्पाम्भश्च उपस्पृश्य न शोचेन्मरणं गतः}


\twolineshloka
{अर्धयोजनविस्तारां पञ्चयोजनमायताम्}
{एतां हि देविकामाहुः पुण्यां देवर्षिसेविताम्}


\threelineshloka
{ततो गच्छेत धर्मज्ञ दीर्घसत्रं यथाक्रमम्}
{तत्र ब्रह्मादयो देवाः सिद्धाश्च परमर्षपः}
{दीर्घसत्रमुपासन्ते दीक्षिता नियतव्रताः}


\twolineshloka
{गमनादेव राजेन्द्र दीर्घसत्रमरिंदम}
{राजसूयाश्वमेधाभ्यां फलं प्राप्नोति भारत}


\twolineshloka
{ततो विनशनं गच्छेन्नियतो नियताशनः}
{गच्छत्यन्तर्हिता यत्रमेरुपृष्ठे सरस्वती}


\threelineshloka
{चमसेऽथ शिवोद्भेदे नागोद्भेदे च दृश्यते}
{`तत्र स्नात्वा नरव्याघ्र द्योतते शशिवत्वदा}
{'स्नात्वा तु चमसोद्भेदे अग्निष्टोमफलं लभेत्}


\twolineshloka
{शिवोद्भेदे नरः स्नात्वा गोसहस्रफंल लभेत्}
{नागोद्भेदे नरः स्नात्वा नागलोकमवाप्नुयात्}


\twolineshloka
{शशयानं च राजेन्द्रतीर्थमासाद्य दुर्लभम्}
{शतरूपप्रतिच्छन्नाः पुष्करा यत्र भारत}


\twolineshloka
{सरस्वत्यां महाराज शतं संवत्सरं च ते}
{दृश्यन्ते भरतश्रेष्ठ वृत्तां वै कार्तिकीं सदा}


\twolineshloka
{तत्रस्नात्वा नरव्याघ्र द्योतते शशिवत्सदा}
{गोसहस्रफलं चैव प्राप्नुयाद्भरतर्षभ}


\twolineshloka
{कुमारकोटीमासाद्य नियतः कुरुनन्दन}
{तत्राभिषेकं कुर्वीत पितृदेवार्चने रतः}


\twolineshloka
{गवामयुतमाप्नोति कुलं चैव समुद्धरेत्}
{ततो गच्छेत धर्मज्ञ रुद्रकोटिं समाहितः}


\twolineshloka
{पुरा यत्र महाराज मुनिकोटिः समागता}
{हर्षेण महताऽऽविष्टा रुद्रदर्शनकाङ्क्षया}


\twolineshloka
{अहंपूर्वमहंपूर्वं द्रक्ष्यामि वृषभध्वजम्}
{एवं संप्रस्थिता राजन्नृषयः किल भारत}


\twolineshloka
{ततो योगीश्वरेणापि योगमास्थाय भूपते}
{तेषां मन्युप्रणाशार्थमृषीणां भावितात्मनाम्}


\twolineshloka
{स्रष्टा कोटीस्तु रुद्राणामृषीणामग्रतः स्थिता}
{मया पूर्वतरं दृष्ट इतिते मेनिरे पृथक्}


\twolineshloka
{तेषां तुष्टो महादेवो मुनीनां भावितात्मनाम्}
{भक्त्या परमया राजन्वरं तेषां प्रदिष्टवान्}


\twolineshloka
{अद्यप्रभृति युष्माकं धर्मवृद्धिर्भविष्यति}
{तत्रस्नात्वा नरव्याघ्र रुद्रकोट्यां नरः शुचिः}


\twolineshloka
{अश्वमेधमवाप्नोति कुलं चैव समुद्धरेत्}
{ततो गच्छेत राजेन्द्र संगमं लोकविश्रुतम्}


\twolineshloka
{सरस्वत्या महापुण्यं केशवं समुपासते}
{यत्रब्रह्माहयो देवा ऋषयश्च तपोधनाः}


\threelineshloka
{अभिगच्छन्ति राजेन्द्र चैत्रशुक्लचतुर्दशीम्}
{तत्रस्नात्वा नरव्याघ्र विन्देद्बहुसुवर्णकम्}
{सर्वपापविशुद्धात्मा ब्रह्मलोकं च गच्छति}


\threelineshloka
{ऋषीणां यत्र सत्राणि समाप्तानि नराधिप}
{तत्रावसानमासाद्य गोसहस्रफलं लभेत्}
{}


\chapter{अध्यायः ८१}
\twolineshloka
{पुलस्त्य उवाच}
{}


\twolineshloka
{ततो गच्छेत राजेन्द्र कुरुक्षेत्रमभिष्टुतम्}
{पापेभ्योविप्रमुच्यन्ते तद्गताः सर्वजन्तवः}


\twolineshloka
{कुरुक्षेत्रं गमिष्यामि कुरुक्षेत्रे वसाम्यहम्}
{य एवं सततं ब्रूयात्सोऽपि पापैः प्रमुच्यते}


\twolineshloka
{पांसवोऽपिकुरुक्षेत्रे वायुना समुदीरिताः}
{अपिदुष्कृतकर्माणं नयन्ति परमां गतिम्}


\twolineshloka
{दक्षिणेन सरस्वत्या दृषद्वत्युत्तरेण च}
{ये वसन्ति कुरुक्षेत्रे ते वसन्ति त्रिविष्टपे}


\twolineshloka
{तत्र मासं वसेद्धीरः सरस्वत्यां युधांवर}
{यत्रब्रह्मादयो देवा ऋषयः सिद्धचारणाः}


\twolineshloka
{गन्धर्वाप्सरसो यक्षाः पन्नगाश्च महीपते}
{ब्रह्मक्षेत्रं महापुण्यमभिगच्छन्ति भारत}


\twolineshloka
{मनसाऽप्यभिकामस्य कुरुक्षेत्रं युधांवर}
{पापानि विप्रणश्यन्ति ब्रह्मलोकं च गच्छति}


\twolineshloka
{गत्वा हि श्रद्धया युक्तः करुक्षेत्रं कुरूद्वह}
{राजसूयाश्वमेधाभ्यां फलमाप्नोति मानवः}


\twolineshloka
{ततश्च मन्तुकं राजन्द्वारपालं महाबलम्}
{यक्षं समभिवाद्यैव गोसहस्रफलं लभेत्}


\twolineshloka
{ततो गच्छेत धर्मज्ञ विष्णो स्थानमनुत्तमम्}
{सततं नाम राजेन्द्र यत्रसन्निहितो हरिः}


\twolineshloka
{तत्रस्नात्वाऽर्चयित्वा च त्रिलोकप्रभवं हरिम्}
{अश्वमेधमवाप्नोति विष्णुलोकं च गच्छति}


\twolineshloka
{ततः परिप्लवं गच्छेत्तीर्थं त्रैलोक्यविश्रुतम्}
{अग्निष्टोमातिरात्राभ्यां फलं प्राप्नोति भारत}


\twolineshloka
{पृथिवीतीर्थमासाद्य गोसहस्रफलं लभेत्}
{ततः शालूकिनी गत्वा तीर्थसेवी नराधिप}


\twolineshloka
{दशाश्वमेधे स्नात्वा च तदेव फलमाप्नुयात्}
{सर्पदर्वीं समासाद्य नागानां तीर्थमुत्तमम्}


\twolineshloka
{अग्निष्टोममवाप्नोति नागलोकं च विन्दति}
{ततो गच्छेत धर्मज्ञ द्वारपालमरुन्तुकम्}


\twolineshloka
{तत्रोष्य रजनीमेकां गोसहस्रफलं लभेत्}
{ततः पञ्चनदं गत्वा नियतो नियताशनः}


\twolineshloka
{कोटितीर्थमुपस्पृश्य हयमेधफंल लभेत्}
{अश्विनोस्तीर्थमासाद्य रूपवानमिजायते}


\twolineshloka
{ततो गच्छेत धर्मज्ञ वाराहं तीर्थमुत्तमम्}
{विष्णुर्वाराहरूपेण पूर्वं यत्र स्थितो विभुः}


\twolineshloka
{तत्रस्नात्वा नरश्रेष्ठ अग्निष्टोमफलं लभेत्}
{ततो जयन्त्यां राजेन्द्र सोमतीर्थं समाविशेत्}


\twolineshloka
{स्नात्वा फलमवाप्नोति राजसूयस्य मानवः}
{एकहंसे नरः स्नात्वा गोसहस्रफलं लभेत्}


\twolineshloka
{शतशौचं समासाद्य तीर्थसेवी नराधिप}
{पौण्डरीकमवाप्नोति कृतशौचो भवेच्च सः}


\twolineshloka
{ततो मुञ्जवटं नाम स्थाणोः स्थानं महात्मनः}
{उपोष्य रजनीमेकां गाणपत्यमवाप्नुयात्}


\twolineshloka
{तत्रैव च महाराज यक्षिणी लोकविश्रुता}
{तां चाभिगम्यराजेन्द्र सर्वान्कामानवाप्नुयात्}


\twolineshloka
{कुरुक्षेत्रस् तद्द्वारं विश्रुतं भरतर्षभ}
{प्रदक्षिणमुपावृत्य तीर्थसेवी समाहितः}


\twolineshloka
{संमिते पुष्कराणां च स्नात्वाऽर्च्य पितृदेवताः}
{जामदग्न्येन रामेण कृतं तत्सुमहात्मना}


\twolineshloka
{कृतकृत्यो भवेद्राजन्नश्वमेधं च विन्दति}
{ततो रामह्रदानार्च्छेत्तीर्थसेवी समाहितः}


\twolineshloka
{तत्ररामेण राजेनद््रतरसा दीप्ततेजसा}
{क्षत्रमुत्साद् वीरेण ह्रदाः पञ्च निवेशिताः}


\twolineshloka
{पूरयित्वा नरव्याघ्र रुधिरेणेति नः श्रुतम्}
{पितरस्तर्पिताः सर्वे तथैव प्रपितागहाः}


\twolineshloka
{ततस्ते पितरः प्रीता राममूचुर्नराधिप}
{रामराम महाभाग प्रीताः स्म तव भार्गव}


\twolineshloka
{अनया पितृभक्त्या च विक्रमेण च ते विभो}
{वरं वृणीष्व भद्रं ते किमिच्छसि महाद्युते}


\twolineshloka
{एवमुक्तः स राजेन्द्र राम प्रहरतांवरः}
{अब्रवीत्प्राञ्जलिर्वाक्यं पितॄन्स गगने स्थितान्}


\twolineshloka
{भवन्तो यदि मे प्रीता यद्यनुग्राह्यता मयि}
{पितृप्रसादादिच्छेयं तपःस्वाध्ययनं पुनः}


\twolineshloka
{यच्च रोषाभिभूतेन क्षत्रमुत्सादितं मया}
{ततश्च पापान्मुच्येयं युष्माकं तेजसाऽप्यहम्}


\twolineshloka
{ह्रदाश्च तीर्थभूता मे भवेयुर्भुवि विश्रुताः}
{एतच्छ्रुत्वा शुभं वाक्यं रामस्य पितरस्तदा}


\twolineshloka
{प्रत्यूचुः परमप्रीता रामं हर्पसमन्विताः}
{तपस्ते वर्धतां भूयः पितृभक्त्या विशेषतः}


\twolineshloka
{यच्च रोपाभिभूतेन क्षत्रमुत्सादितं त्वया}
{ततश्च पापान्मुक्तस्त्वं पातितास्ते स्वकर्मभिः}


\twolineshloka
{ह्रदाश्च तव तीर्थत्वं गमिष्यन्ति न संशयः}
{ह्रदेषु तेषु यः स्नात्वा पितॄन्संतर्पयिष्यति}


\twolineshloka
{पितरस्तस् वै प्रीता दास्यन्ति भुवि दुर्लभम्}
{ईप्सितं च मनःकामं स्वर्गलोकं च शाश्वतम्}


\twolineshloka
{एवं दत्त्वा वरान्राजन्रामस्य पितरस्तदा}
{आमन्त्र्य भारग्वं प्रीत्या तत्रैवान्तर्हितास्ततः}


\twolineshloka
{एवं रामह्रदाः पुण्या भार्गवस्य महात्मनः}
{स्नात्वा ह्रहेषु रामस्य ब्रह्मचारी शुचिव्रतः}


\twolineshloka
{राममभ्यर्च्य राजेन्द्र लभेद्बहु सुवर्णकम्}
{वंशमूलकमासाद्य तीर्थसेवी कुरूद्वह}


\threelineshloka
{स्ववंशमुद्धरेद्राजन्स्नात्वा वै वंशमूलके}
{कायशोधनमासाद्य तीर्थं भरतसत्तम ॥शरीरशुद्धिः स्नातस्य तस्मिंस्तीर्थे न संशयः}
{}


\twolineshloka
{शरीरशुद्धिः स्नातस्य तस्मिंस्तीर्थे न संशयः}
{शुद्धदेहश्च संयाति युभाँल्लोकाननुत्तमान्}


\twolineshloka
{ततो गच्छेत धर्मज्ञ तीर्थं त्रैलोक्यविश्रुतम्}
{लोका यत्रोद्धृताः पूर्वं विष्णुना प्रभविष्णुना}


\twolineshloka
{लोकोद्धारं समासाद्य तीर्थं त्रैलोक्यपूजितम्}
{स्नात्वा तीर्थवरे राजँल्लोकानुद्धरते स्वकान्}


\twolineshloka
{श्रीतीर्थं च समासाद्य स्नात्वा नियतमानसः}
{अर्चयित्वा पितॄन्देवान्विन्दते श्रियमुत्तमाम्}


\twolineshloka
{कपिलातीर्थमासाद्य ब्रह्मचारी समाहितः}
{तत्र स्नात्वाऽर्चयित्वा च पितॄन्स्वान्दैवतान्यपि}


\twolineshloka
{कपिलानां सहस्रस्य फलंविन्दति मानवः}
{सूर्यतीर्थं समासाद्य स्नात्वा नियतमानसः}


\twolineshloka
{अर्चयित्वा पितॄन्देवानुपवासपरायणः}
{अग्निष्टोममवाप्नोति सूर्यलोकं च गच्छति}


\twolineshloka
{गवां भगवनासाद्य तीर्थसेवी यथाक्रमम्}
{तत्राभिषेकं कुर्वाणो गोसहस्रफलं लभेत्}


\twolineshloka
{शङ्खिनीतीर्थमासाद्य तीर्थसेवी कुरूद्वह}
{देव्यास्तीर्थे नरः स्नात्वा लभते रूपमुत्तमम्}


\twolineshloka
{ततो गच्छेत राजेन्द्रद्वारपालमरुन्तुकम्}
{तच्च तीर्थं सरस्वत्यां यक्षेन्द्रस्य महात्मनः}


\twolineshloka
{तत्रस्नात्वा नरो राजन्नग्निष्टोमफलं लभेत्}
{ततो गच्छेत राजेन्द्र ब्रह्मावर्तं नरोत्तमः}


\twolineshloka
{ब्रह्मावर्ते नरः स्नात्वा ब्रह्मलोकमवाप्नुयात्}
{ततो गच्छेत राजेन्द्र सुतीर्थकमनुत्तमम्}


\twolineshloka
{तत्रसन्निहिता नित्यं पितरो दैवतैः सह}
{तत्राभिषेकं कुर्वीत पितृदेवार्चने रतः}


\twolineshloka
{अश्वमेधमावाप्नोति पितृलोकं च गच्छति}
{ततोम्बुमत्यां धर्मज्ञ सुतीर्थकमनुत्तमम्}


\twolineshloka
{कोशेश्वरस्य तीर्थेषु स्नात्वा भरतसत्तम}
{सर्वव्याधिविनिर्मुक्तो ब्रह्मलोके महीयते}


\twolineshloka
{मातृतीर्थं च तत्रैव यत्रस्नातस्य भारत}
{प्रजा विवर्धते राजन्न तन्वीं श्रियमश्नुते}


\twolineshloka
{ततः शीतवनं गच्छेन्नियतो नियताशनः}
{तीर्थं तत्रमहाराज महदन्यत्र दुर्लभम्}


\twolineshloka
{पुनाति गमनादेव कुलमेकं नराधिप}
{केशानभ्युक्ष् वै तस्मिन्पूतो भवति भारत}


\twolineshloka
{तीर्थं तत्र महाराज श्वानलोमापहं स्मृतम्}
{यत्र विप्रा नरव्याघ्र विद्वांसस्तीर्थतत्पराः}


\twolineshloka
{गतिं गच्छन्ति परमां स्नात्वा भरतसत्तम}
{श्वानलोमापनयने तीर्थे भरतसत्तम}


\twolineshloka
{प्राणायामैर्निर्हरन्ति स्वलोमानि द्विजोत्तमाः}
{पूतात्मानश्च राजेन्द्रप्रयान्ति परमां गतिम्}


\twolineshloka
{दशाश्वमेधिकं चैव तस्मिंस्तीर्थे महीपते}
{तत्र स्नात्वा नरव्याघ्र गच्छेत परमां गतिम्}


\twolineshloka
{ततो गच्छेत राजेन्द्र मानुषं लोकविश्रुतम्}
{यत्र कृष्णमृगा राजन्व्याधेन परिपीडिताः}


\twolineshloka
{विगाह्य तस्मिन्सरसि मानुषत्वमुपागताः}
{तस्मिंस्तीर्थे नरः स्नात्वा ब्रह्मचारी समाहितः}


\twolineshloka
{सर्वपापविशुद्धात्मा स्वर्गलोके महीयते}
{मानुषस्य तु पूर्वेण क्रोशमात्रे महीपते}


\twolineshloka
{आपगा नाम विख्याता नदीसिद्धनिषेविता}
{श्यामाकभोजनं तत्रयः प्रयच्छति मानवः}


\twolineshloka
{देवान्पितॄन्समुद्दिश्य तस्य धर्मफलं महत्}
{एकस्मिन्भोजिते विप्रे कोटिर्भवति भोजिता}


\twolineshloka
{तत्रस्नात्वाऽर्चयित्वा च पितृन्वै दैवतानि च}
{उषित्वा रजनीमेकामग्निष्टोमफलं लभेत्}


\twolineshloka
{ततो गच्छेत राजेन्द्र ब्रह्मणः स्थानमुत्तमम्}
{ब्रह्मोदुम्बरमित्येव प्रकाशं भुवि भारत}


\twolineshloka
{तत्रसप्तर्षिकुण्डेषु स्नातस् यनरपुङ्गव}
{केदारे चैवराजेन्द्रकपिञ्जलमहामुनेः}


\twolineshloka
{ब्रह्माणमधिगत्वा च शुचिः प्रयतमानसः}
{सर्वपापविशुद्धात्मा ब्रह्मलोकं प्रपद्यते}


\twolineshloka
{कपिञ्जलस्य केदारं समासाद्य सुदुर्लभम्}
{अन्तर्धानमवाप्नोति तपसा दग्धकिल्विषः}


\twolineshloka
{ततो गच्छेत राजेन्द्र शङ्करं लोकविश्रुतम्}
{कृष्णपक्षे चतुर्दश्यामभिगम्य वृषध्वजम्}


\twolineshloka
{लभेत सर्वकामान्हि स्वर्गलोकं च गच्छति}
{तिस्रः कोट्यस्तु तीर्थानां शङ्करे कुरुनन्दन}


\twolineshloka
{रुद्रकोट्यां तथा कूपे ह्रदेषु च महीपते}
{शुद्धास्पदं च तत्रैव तीर्थं भरतसत्तम}


\twolineshloka
{तत्रस्नात्वाऽर्चयित्वा च दैवतानि पितॄनथ}
{न दुर्गतिमवाप्नोति वाजपेयं च विन्दति}


\twolineshloka
{किंदाने च नरः स्नात्वा किंजप्ये च महीपते}
{अप्रमेयमवाप्नोति दानं जप्यं च भारत}


\twolineshloka
{कलश्यां वार्युपस्पृश्य श्रद्दधानो जितेन्द्रियः}
{अग्निष्टोमस् यज्ञस्य फलं प्राप्नोति मानवः}


\twolineshloka
{शंकरस्य तु पूर्वेण नारदस्य महात्मनः}
{तीर्थं कुरुकुलश्रेष्ठ अजानन्तेति विश्रुतम्}


\twolineshloka
{तत्रतीर्थे नरः स्नात्वा प्राणानुत्सृज्य भारत}
{नारदेनाभ्यनुज्ञातो लोकान्प्राप्नोत्यनुत्तमानन्}


\twolineshloka
{शुक्लपक्षे दशम्यां च पुण्डरीकं समाविशेत्}
{तत्र स्नात्वा नरो राजन्पौण्डरीकफलं लभेत्}


\twolineshloka
{ततस्त्रिविष्टपं गच्छेत्रिषु लोकेषु विश्रुतम्}
{तत्र वैतरणी पुण्या नदी पापप्रणाशिनी}


\twolineshloka
{तत्र स्नात्वाऽर्चयित्वा च शूलपाणिं वृषध्वजम्}
{सर्वपापविशुद्धात्मा गच्छेत परमां गतिम्}


\twolineshloka
{ततो गच्छेत राजेन्द्र फलकीवनमुत्तमम्}
{तत्र देवाः सदा राजन्फलकीवनमाश्रिताः}


\twolineshloka
{तपश्चरन्ति विपुलं बहुवर्षसहस्रकम्}
{दृषद्वत्यां नरः स्नात्वा तर्पयित्वा च देवताः}


\twolineshloka
{अग्निष्टोमातिरात्राभ्यां फलं विन्दति भारत}
{तीर्ते च सर्वदेवानां स्नात्वा भरतसत्तम}


\twolineshloka
{गोसहस्रस्य राजेन्द्र फलं विन्दति मानवः}
{पाणिस्वाते नरः स्नात्त्वा तर्पयित्वा च देवताः}


\twolineshloka
{अग्निष्टोमातिरात्राभ्यां फलं विन्दति भारत}
{राजसूयमावाप्नोति ऋषिलोकं च विन्दति}


\twolineshloka
{ततो गच्छेत राजेन्द्र मिश्रकं तीर्थमुत्तमम्}
{तत्र तीर्थानि राजेन्द्र मिश्रितानि महात्मना}


\twolineshloka
{व्यासेन नृपशार्दूल द्विजार्थमिति नः श्रुतम्}
{सर्वतीर्थेषु स स्नाति मिश्रके स्नाति यो नरः}


\twolineshloka
{ततो व्यासवनं गच्छेन्नियतो नियताशनः}
{मनोजवे नरः स्नात्वा गोसहस्रफलं लभेत्}


\twolineshloka
{गत्वा मधुवटीं चैव देव्याः स्थानं नरः शुचिः}
{तत्र स्नात्वाऽर्चयित्वा च पितॄन्देवांश्च पूरुषः}


\twolineshloka
{स देव्या समनुज्ञातो गोसहस्रफलं लभेत्}
{कौशिक्याः संगमे यस्तु दृषद्वत्याश्च भारत}


\twolineshloka
{स्नाति वै नियताहार सर्वपापैः प्रमुच्यते}
{ततो व्यासस्थली नाम यत्रव्यासेन धीमता}


\twolineshloka
{पुत्रशोकाभितप्तेन देहत्यागे कृता मतिः}
{ततो देवैस्तु राजेन्द्र पुनरुत्थापितस्तदा}


\twolineshloka
{अभिगत्वा स्थलीं तस्य गोसहस्रफलं लभेत्}
{किंदत्तं कूपमासाद्य तिलप्रस्थं प्रदाय च}


\twolineshloka
{गच्छेत परमां सिद्धिमृणैर्मुक्तः कुरूद्वह}
{वेदीतीर्थे नरः स्नात्वा गोसहस्रफलं लभेत्}


\twolineshloka
{अहस्छ सुदिनं चैव द्वे तीर्थे लोकविश्रुते}
{तयोः स्नात्वा नरव्याघ्र सूर्यलोकमवाप्नुयात्}


\twolineshloka
{मृगधूमं ततो गच्छेत्रिषु लोकेषु विश्रुतम्}
{तत्राभिषेकं कुर्वीत गङ्गायां नृपसत्तम}


\twolineshloka
{अर्चयित्वा महादेवमश्वमेधफलं लभेत्}
{देव्यास्तीर्थे नरः स्नात्वा गोसहस्रफलं लभेत्}


\twolineshloka
{ततो वामनकं गच्छेत्रिषु लोकेषु विश्रुतम्}
{तत्रविष्णुपदे स्नात्वा अर्चयित्वा च वामनम्}


\twolineshloka
{सर्वपापविशुद्धात्मा विष्णुलोकं स गच्छति}
{कुलपुने नरः स्नात्वा पुनाति स्वकुलं ततः}


\twolineshloka
{पवनस्य ह्रदे स्नात्वा मरुतां तीर्थमुत्तमम्}
{तत्रस्नात्वा नख्याघ्र विष्णुलोके महीयते}


\twolineshloka
{अमराणां ह्रदे स्नात्वा समभ्यर्च्यामराधिपम्}
{अमराणां प्रभावेन स्वर्गलोके महीयते}


\twolineshloka
{शालिहोत्रस्य तीर्थे च शालिशूर्पे यथाविधि}
{स्नात्वा नरवरश्रेष्ठ गोसहस्रफलं लभेत्}


\twolineshloka
{श्रीकुञ्जं च सरस्वत्यास्तीर्थं भरतसत्तम}
{तत्र स्नात्वा नरश्रेष्ठ अग्निष्टोमफलं लभेत्}


\twolineshloka
{ततो नैमिपकुञ्जं च समासाद्य कुरूद्वह}
{ऋषयः किल राजेन्द्र नैमिषेयास्तपस्विनः}


\threelineshloka
{तीर्थयात्रां पुरस्कृत्य कुरुक्षेत्रं गताः पुरा}
{`तत्र तीर्थे नरः स्नात्वा वाजिमेधफलं लभेत्}
{'ततः कुञ्जः सरस्वत्याः कृतो भरतसत्तम}


\twolineshloka
{ऋपीणामवकाशः स्याद्यथा तुष्टिकरो महान्}
{कन्यातीर्थे नरः स्नात्वागोसहस्रफलं लभेत्}


\twolineshloka
{ततो गच्छेत धर्मज्ञ कन्यातीर्थमनुत्तमम्}
{कन्यातीर्थे नरः स्नात्वा अग्निष्टोमफलं लभेत्}


\twolineshloka
{ततो गच्छेत राजेन्द्र ब्रह्मणस्तीर्थमुत्तमम्}
{तत्रवर्णावरः स्नात्वा ब्राह्मण्यं लभते नरः}


\twolineshloka
{ब्राह्मणश्च विशुद्धात्मा गच्छेत परमां गतिम्}
{ततो गच्छेन्नरश्रेष्ठ सोमतीर्थमनत्तमम्}


\twolineshloka
{तत्र स्नात्वा नरो राजन्सोमलोकमवाप्नुयात्}
{सप्तसारस्वतं तीर्थं ततो गच्छेन्नराधिप}


\twolineshloka
{यत्र मङ्कणकः सिद्धो महर्षिर्लोकविश्रुतः}
{पुरा मङ्कणको राजन्कुशाग्रेणेति नःश्रुतम्}


\twolineshloka
{क्षतः किल करे राजंस्तस्य शाकरसोऽस्रवत्}
{स वै शाकरसं दृष्ट्वा हर्षाविष्टः प्रनृत्तवान्}


\twolineshloka
{ततस्तस्मिन्प्रनृत्ते तु स्थावरं जङ्गमं च यत्}
{प्रनृत्तमुभयं वीर तेजसा तस्य मोहितम्}


\twolineshloka
{ब्रह्मादिभिः सुरै राजनृषिभिश्च तपोधनैः}
{क्षिप्तो वै महादेव ऋषेरर्थे नराधिप}


\threelineshloka
{नायं नृत्येद्यथा देव तथा न्वं कर्तुमर्हसि}
{तं प्रनृत्तं समासाद्य हर्षाविष्टेन चेतसा}
{सुराणां हितकामार्थमृषिं देवोऽभ्यभाषत}


\threelineshloka
{भोभो महर्षे धर्मज्ञ किमर्थं नृत्यते भवान्}
{हर्षस्थानं किमर्थं वा तवाद्य मुनिपुङ्गव ॥ऋषिरुवाच}
{}


\twolineshloka
{तपस्विनो धर्मपथे स्थितस्य द्विजसत्तम}
{किं न पश्यसि मे ब्रह्मन्कराच्छाकरसं स्रुतम्}


\twolineshloka
{यं दृष्ट्वाऽहं प्रनृत्तोऽहं हर्षेण महताऽन्वितः}
{तं प्रहस्याब्रवीद्देव ऋषिं रागेण मोहितम्}


\twolineshloka
{अहं तु विस्मयं विप्र न गच्छामीति पश्य माम्}
{एवमुक्त्वा नरश्रेष्ठ महादेवेन धीमता}


\twolineshloka
{अङ्गुल्यग्रेण राजेन्द्रस्वाङ्गुष्ठस्ताडितोऽनघ}
{ततो भस्म क्षताद्राजन्निर्गतं हिमसन्निभम्}


\twolineshloka
{तद्दृष्ट्वा व्रीडितो राजन्स मुनिः पादयोर्गतः}
{नान्यद्देवात्परं मेने रुद्रात्परतरं महत्}


\twolineshloka
{सुरासुरस्य जगतो गतिस्त्वमसि शूलधृत्}
{त्वया सर्वमिदं सृष्टं त्रैलोक्यं सचराचरम्}


\twolineshloka
{त्वमेव सर्वान्ग्रससि पुनरेव युगक्षये}
{देवैरपि न शक्यस्त्वं परिज्ञातुं कुतो मया}


\twolineshloka
{त्वयि सर्वे प्रदृश्यन्ते सुरा ब्रह्मादयोऽनघ}
{सर्वस्त्वमसि लोकानां कर्ताकारयिता च ह}


\twolineshloka
{त्वत्प्रसादात्सुराः सर्वे निवसन्त्यकुतोभयाः}
{एवं स्तुत्वा महादेवमृषिर्वचनमब्रवीत्}


\twolineshloka
{त्वत्प्रसादान्महादेव तपो मे न क्षरेत वै}
{ततो देवः प्रहृष्टात्मा ब्रह्मर्षिमिदमब्रवीत्}


\twolineshloka
{तपस्ते वर्धतां विप्र मत्प्रसादात्सहस्रशः}
{आश्रमे चेह वत्स्यामि त्वया सह महामुने}


\twolineshloka
{सप्तसारस्वते स्नात्वा अर्चयिष्यन्ति ये तु माम्}
{न तेषां दुर्लभं किंचिदिह लोके परत्र च}


\threelineshloka
{सारस्वतं च ते लोकं गमिष्यन्ति न संशयः}
{एवमुक्त्वा महादेवस्तत्रैवान्तरधीयत ॥पुलस्त्य उवाच}
{}


\twolineshloka
{ततस्त्वौशनसं गच्छेत्रिषु लोकेषु विश्रुतम्}
{यत्रब्रह्मादयो देवा ऋषयश्च तपोधनाः}


\twolineshloka
{कार्तिकेयश्च भगवांस्त्रिसंध्यं किल भारत}
{सान्निध्यमकरोन्नित्यं भार्गवप्रियकाम्यया}


\twolineshloka
{कपालमोचनं तीर्थं सर्वपापप्रमोचनम्}
{तत्र स्नात्वा नरव्याघ्र सर्वपापैः प्रमुच्यते}


\twolineshloka
{अग्नितीर्थं ततो गच्छेत्तत्र स्नात्वा नरर्षभ}
{अग्निलोकमवाप्नोति कुलं चैव सुमुद्धरेत्}


\twolineshloka
{विश्वामित्रस्य तत्रैव तीर्थं भरतसत्तम}
{तत्र स्नात्वा नरश्रेष्ठ ब्राह्मण्यमधिगच्छति}


\twolineshloka
{ब्रह्मयोनिं समासाद्य शुचिः प्रयतमानसः}
{तत्र स्नात्वा नरव्याघ्र ब्रह्मलोकं प्रपद्यते}


\twolineshloka
{पुनात्यासप्तमं चैव कुलं नास्त्यत्रसंशयः}
{ततो गच्छेत राजेन्द्र तीर्थं त्रैलोक्यविश्रुतम्}


\twolineshloka
{पृथूदकमिति ख्यातं कार्तिकेयस्य वै नृप}
{तत्राभिषेकं कुर्वीत पितृदेवार्चने रतः}


\twolineshloka
{अज्ञानाज्ज्ञानतो वाऽपिस्त्रिया वा पुरुषेण वा}
{यत्किंचिदशुभं कर्म कृतं मानुषबुद्धिना}


\twolineshloka
{तत्सर्वं नश्यते तत्र स्नातमात्रस्य भारत}
{अश्वमेधफलं चास्य स्वर्गलोकं च गच्छति}


\twolineshloka
{पुण्यमाहुः कुरुक्षेत्रं कुरुक्षेत्रात्सरस्वती}
{सरस्वत्याश्च तीर्थानि तीर्थेभ्यश् पृथूदकम्}


\twolineshloka
{उत्तमे सर्वतीर्थानां यस्त्यजेदात्मनस्तनुम्}
{पृथूदके जप्यपरो नैव श्वो मरणं तपेत्}


\twolineshloka
{गीतं सनत्कुमारेण व्यासेन च महात्मना}
{वेदे च नियतं राजन्नभिगच्छेत्पृथूदकम्}


\twolineshloka
{पृथूदकात्तीर्थतमं नान्यत्तीर्थं कुरूद्वह}
{तन्मेध्यं तत्पवित्रं च पावनं च न संशयः}


\twolineshloka
{तत्रस्नात्वा दिवं यान्ति येऽपि पापकृतो नराः}
{पृथूदके नरश्रेष्ठ एवमाहुर्मनीषिणः}


\twolineshloka
{मधुस्रवं च तत्रैव तीर्थं भरतसत्तम}
{तत्र स्नात्वा नरो राजन्गोसहस्रफलं लभेत्}


\twolineshloka
{ततो गच्छेत राजेन्द्र तीर्थं देव्या यथाक्रमम्}
{सरस्वत्यारुणायाश्च संगमे लोकविश्रुते}


\twolineshloka
{त्रिरात्रोपोषितः स्नात्वा मुच्यते ब्रह्महत्यया}
{अग्निष्टोमातिरात्राभ्यां फलं विन्दति मानवः}


\twolineshloka
{आसप्तमं कुलं चैव पुनाति भरतर्षभ}
{अवतीर्णं च तत्रैव तीर्थं कुरुकुलोद्वह}


\twolineshloka
{विप्राणामनुकम्पार्थं शार्ङ्गिणा निर्मिनं पुरा}
{व्रतोपनयनाभ्यां चाप्युपवासेन वाऽप्युत}


\threelineshloka
{क्रियामन्त्रैश्च संयुक्तो ब्राह्मणः स्यान्न संशयः}
{क्रियामन्त्रविहीनोऽपि तत्र स्नात्वा नरर्षभ}
{चीर्णव्रतो भवेद्विद्वान्दृष्टमेतत्पुरातनैः}


\twolineshloka
{समुद्राश्चापि चत्वारः समानीताश् दर्भिणा}
{तेषु स्नातो नरश्रेष्ठ न दुर्गतिमवाप्नुयात्}


\twolineshloka
{फलानि गोसहस्राणां चतुर्णां विन्दते च सः}
{ततो गच्छेत धर्मज्ञ तीर्थं शतसहस्रकम्}


\twolineshloka
{साहस्रकं च तत्रैव द्वे तीर्थे लोकविश्रुते}
{उभयोर्हि नरः स्नात्वा गोसरस्रफंल लभेत्}


\twolineshloka
{दानं वाऽप्युपवासो वा सहस्रगुणितं भवेत्}
{ततो गच्छेत राजेन्द्र रेणुकातीर्थमुत्तमम्}


\twolineshloka
{तीर्ताभिषेकं कुर्वीत पितृदेवार्चने रतः}
{सर्वपापविशुद्धात्मा अग्निष्टोमफलं लभेत्}


\twolineshloka
{विमोचनमुपस्पृश्य जितमन्युर्जितेन्द्रियः}
{प्रतिग्रहकृतैर्दोषैः सर्वैः स परिमुच्यते}


\twolineshloka
{ततः पञ्चवटीं गत्वा ब्रह्मचारी जितेन्द्रियः}
{पुण्येन महता युक्तः सतां लोके महीयते}


\twolineshloka
{यत्रयोगेश्वरः स्थाणुः स्वयमेव वृषध्वजः}
{तमर्चयित्वा देवेशं गमनादेव सिद्ध्यति}


\twolineshloka
{तैजसं वारुणं तीर्थं दीप्यमानं स्वतेजसा}
{यत्र रब्रह्मादिभिर्दैर्वैर्ऋषिभिश्च तपोधनैः}


\twolineshloka
{सैनापत्येन देवानामभिषिक्तो गुहस्तदा}
{तैजसस्य तु पूर्वेण कुरुतीर्थं कुरूद्वह}


\twolineshloka
{कुरुतीर्थे नरः स्नात्वा ब्रह्मचारी जितेन्द्रियः}
{सर्वपापविशुद्धात्मा ब्रह्मलोकं प्रपद्यते}


\twolineshloka
{स्वर्गद्वारं ततो गच्छेन्नियतो नियताशनः}
{स्वर्गलोकमवाप्नोति ब्रह्मलोकं च गच्छति}


\twolineshloka
{ततो गच्छेदनरकं तीर्थसेवी नराधिप}
{तत्र स्नात्वा नरो राजन्न दुर्गतिमवाप्नुयात्}


\twolineshloka
{तत्रब्रह्मा स्वयं नित्यं देवैः सह महीपते}
{अन्वास्ते पुरुषव्याघ्र नारायणपुरोगमैः}


\twolineshloka
{सान्निध्यं तत्रराजेन्द्र रुद्रपत्न्याः कुरूद्वह}
{अभिगम्य च तां देवीं न दुर्गतिमवाप्नुयात्}


\twolineshloka
{तत्रैव च महाराज विश्वेश्वरमुमापतिम्}
{अभिगम्य महादेवं मुच्यते सर्वकिल्विषैः}


\twolineshloka
{नारायणं चाभिगम्य पद्मनाभमरिंदम}
{राजमानो महाराज विष्णुलोकं च गच्छति}


\twolineshloka
{तीर्थेषु सर्वदेवानां स्नातः स पुरुषर्षभ}
{सर्वदुःखैः परित्यक्तो द्योतते शशिवन्नरः}


\twolineshloka
{ततः स्वस्तिपुरं गच्छेत्तीर्थसेवी नराधिप}
{प्रदक्षिणमुपावृत्य गोसहस्रफलं लभेत्}


\twolineshloka
{पावनं तीर्थमासाद्य तर्पयेत्पितृदेवताः}
{अग्निष्टोमस्य यत्रस्य फलं प्राप्नोति भारत}


\twolineshloka
{गङ्गाह्रदश्च तत्रैव कूपश्च भरतर्षभ}
{तिस्रः कोट्यस्तु तीर्थानां तस्मिन्कूपे महीपते}


\twolineshloka
{तत्र स्नात्वा नरो राजन्स्वर्गलोकं प्रपद्यते}
{आपगायां नरः स्नात्वा अर्चयित्वा महेश्वरम्}


\twolineshloka
{गाणपत्यमवाप्नोति कुलं चैव समुद्धरेत्}
{ततः स्थाणुवटं गच्छेत्रिषु लोकेषु विश्रुतम्}


\twolineshloka
{तत्र स्नात्वा स्थितो रात्रिं रुद्रलोकमवाप्नयात्}
{बदरीकाननं गच्छेद्वसिष्ठस्याश्रमं गतः}


\twolineshloka
{बदरीं भक्षयेत्तत्र त्रिरात्रोपोषितो नरः}
{सम्यग्द्वादशवर्षाणि बदरीं भक्षयेत्तु यः}


\twolineshloka
{त्रिरात्रोपोषितस्तेन भवेत्तुल्यो नराधिप}
{इन्द्रमार्गं समासाद्य तीर्थसेवी नराधिप}


\twolineshloka
{अहोरात्रोपवासेन शक्रलोके महीयते}
{एकरात्रं समासाद्य एकरात्रोषितो नरः}


\twolineshloka
{नियतः सत्यवादी च ब्रह्मलोके महीयते}
{ततो गच्छेत राजेन्द्र तीर्थं त्रैलोक्यविश्रुतम्}


\twolineshloka
{आदित्यस्याश्रमो यत्र तेजोराशेर्महात्मनः}
{तस्मिंस्तीर्थे नरः स्नात्वा पूजयित्वा विभावसुं}


\twolineshloka
{आदित्यलोकं व्रजति कुलं चैव समुद्धरेत्}
{सोमतीर्थे नरः स्नात्वा तीर्थसेवी नराधिप}


\twolineshloka
{सोमलोकमवाप्नोति नरो नास्त्यत्रसंशयः}
{ततो गच्छेत धर्मज्ञ दधीचस्य महात्मनः}


\twolineshloka
{तीर्थं पुण्यतमं राजन्पावनं लोकविश्रुतम्}
{यत्र सारस्वतो यातः सोङ्गिरास्तपसो निधिः}


\twolineshloka
{तस्मिस्तीर्थे नरः स्नात्वा वाजिमेधफलं लभेत्}
{सारस्वतीं गतिं चैव लभते नात्र संशयः}


\twolineshloka
{ततः कन्याश्रमं गच्छेन्नियतो ब्रह्मचर्यवान्}
{त्रिरात्रोपोषितो राजन्नियतो नियताशनः}


\twolineshloka
{लभेत्कन्याशतं दिव्यं स्वर्गलोकं च गच्छति}
{ततो गच्छेत धर्मज्ञ तीर्थं सन्निहतीमपि}


\twolineshloka
{तत्रब्रह्मादयो देवा ऋषयश्च तपोधनाः}
{मासिमासि समायान्ति पुण्येन महाताऽन्विताः}


\twolineshloka
{सन्निहत्यामुपस्पृश्य राहुग्रस्ते दिवाकरे}
{अश्वमेधशतं तेन तत्रेष्टं शाश्वतं भवेत्}


\twolineshloka
{पृथिव्यां यानि तीर्थानि अन्तरिक्षचराणि च}
{नद्यो ह्रदास्तडागाश्च सर्वप्रस्रवणानि च}


\twolineshloka
{उदपानानि वाप्यश्च तीर्थान्यायतनानि च}
{निःसंशयममावास्यां समेष्यन्ति नराधिप}


\twolineshloka
{मासिमासि नरव्याघ्र सन्निहत्यां न संशयः}
{तीर्थसन्निहनादेव सन्निहत्येति विश्रुता}


\twolineshloka
{तत्र स्नात्वा च पीत्वा च स्वर्गलोके महीयते}
{अमावास्यां तु तत्रैव राहुग्रस्ते रदिवाकरे}


\twolineshloka
{यः श्राद्धं कुरुते मर्त्यस्तस्य पुण्यफंल शृणु}
{अश्वमेधसहस्रस्य सम्यगिष्टस्य यत्फलम्}


\twolineshloka
{स्नात एव समाप्नोति कृत्वा श्राद्धं च मानवः}
{यत्किंचिद्दुष्कृतंकर्म स्त्रिया वा पुरुषेण वा}


\twolineshloka
{स्नातमात्रस्य तत्सर्वं नश्यते नात्र संशयः}
{पद्मवर्णेन याने ब्रह्मलोकं प्रपद्यते}


\twolineshloka
{अभिवाद्य ततो यक्षं द्वारपालं मचक्रुकम्}
{कोटितीर्थमुपस्पृश्य लभेद्बहुसुवर्णकम्}


\twolineshloka
{गङ्गाह्रदश्च तत्रैव तीर्थं भरतसत्तम}
{तत्रस्नायीत धर्मज्ञ ब्रह्मचारी समाहितः}


\twolineshloka
{राजसूयाश्वमेधाभ्यां फलं विन्दति मानवः}
{पृथिव्यां नैमिषं तीर्थमन्तरिक्षे च पुष्करम्}


\twolineshloka
{त्रयाणामपि लोकानां कुरुक्षेत्रं विशिष्यते}
{पांसवोपि कुरुक्षेत्राद्वायुना समुदीरिताः}


\twolineshloka
{अपि दुष्कृतकर्माणं नयन्ति परमां गतिम्}
{दक्षिणेन सरस्वत्या उत्तरेण दृषद्वतीम्}


\threelineshloka
{ये वसन्ति कुरुक्षेत्रे तेवसन्ति त्रिविष्टपे}
{कुरुक्षेत्रं गमिष्यामि कुरुक्षेत्रे वसाम्यहम्}
{अप्येकां वाचमुत्सृज्य सर्वपापैः प्रमुच्यते}


\twolineshloka
{ब्रह्मवेदी कुरुक्षेत्रं पुण्यं ब्रह्मर्षिसेवितम्}
{तस्मिन्वसन्ति ये मर्त्या न ते शोच्याः कथंचन}


\twolineshloka
{तरन्तुकारन्तुकयोर्यदन्तरंरामह्रदानां च मचक्रुकस्य च}
{एतत्कुरुक्षेत्रसमन्तपञ्चकंपितामहस्योत्तरवेदिरुच्यते}


\chapter{अध्यायः ८२}
\twolineshloka
{पुलस्त्य उवाच}
{}


\twolineshloka
{ततो गच्छेन्महाराज धर्मतीर्थमनुत्तमम्}
{[यत्र धर्मो महाभागस्तप्तवानुत्तमं तपः ॥तेन तीर्थं कृतंपुण्यं स्वेन नाम्ना च विश्रुतम् ॥]}


\twolineshloka
{तत्रस्नात्वा नरो राजन्धर्मशील समाहितः}
{आसप्तमं कुलं चैव पुनीते नात्र संशयः}


\twolineshloka
{ततो गच्छेत राजेन्द्र ज्ञानपावनमुत्तमम्}
{अग्निष्टोममवाप्नोति मुनिलोकं च गच्छति}


\twolineshloka
{सौगन्धिकवनं राजंस्ततो गच्छेत मानवः}
{तत्रब्रह्मादयो देवा ऋषयश्च तपोधनाः}


\twolineshloka
{सिद्धचारणगन्धर्वाः किन्नराश्च महोरगाः}
{कतद्वनं प्रविशन्नेव सर्वपापैः प्रमुच्यते}


\twolineshloka
{ततश्चापि सरिचच्छ्रेष्ठा नदीनामुत्तमा नदी}
{प्लक्षा देवी स्मृता राजन्पुण्या देवी सरस्वती}


\twolineshloka
{तत्राभिषेकं कुर्वीत वल्मीकान्निः सृतेजले}
{अर्चयित्वा पितॄन्देवानश्वमेधफलं लभेत्}


\twolineshloka
{कवेराध्युषितं नाम तत्र तीर्थं सुदुर्लभम्}
{पट्सु शम्यानिपातेषु वल्मीकादिति निश्चयः}


\twolineshloka
{कपिलानां सहस्रं च वाजिमेधं च विन्दति}
{तत्र स्नात्वा नरव्याघ्र दृष्टमेतत्पुरातने}


\twolineshloka
{सुगन्धां शतकुम्भां च पञ्चयक्षां च भारत}
{अभिगम्य नरश्रेष्ठ स्वर्गलोके महीयते}


\threelineshloka
{त्रिशूलखातं तत्रैव तीर्थमासाद्य भारत}
{तत्राभिषेकं कुर्वीत पितृदेवार्चने रतः}
{गाणपत्यं च लभते देहं त्यक्त्वा न संशयः}


\twolineshloka
{ततो गच्छेत राजेन्द्र देव्याः स्थानं सुदुर्लभम्}
{शाकम्भरीति विख्याता त्रिषु लोकेषु विश्रुता}


\twolineshloka
{दिव्यं वर्षसहस्रं हि शाकेन किल सुव्रता}
{आहारं सा कृतवती मासिमासि नराधिप}


\threelineshloka
{ऋषयोऽभ्यागतास्तत्रदेव्या भक्त्या तपोधनाः}
{आतिथ्यं च कृतं तेषां शाकेन किल भारत}
{ततः शाकम्भरीत्येव नाम तस्याः प्रतिष्ठितम्}


\twolineshloka
{शाकम्भरीं समासाद्य ब्रह्मचारी समाहितः}
{त्रिरात्रमुषितः शाकं भक्षयित्वा नरः शुचिः}


\twolineshloka
{शाकाहारस्य यत्किंचिद्वर्षैर्द्वादशभिः कृतम्}
{तत्फलं तस्य भवति देव्याश्छन्देन भारत}


\twolineshloka
{ततो गच्छेत्सुवर्णाख्यं त्रिषु लोकेषु विश्रुतम्}
{तत्र विष्णुः प्रसादार्थं रुद्रकमाराधयत्पुरा}


\twolineshloka
{वरांश्च सुबर्हूल्लेभे दैवतेषु सुदुर्लभान्}
{उक्तश्च त्रिपुरघ्नेन परितुष्टेन भारत}


\twolineshloka
{अपिच त्वं प्रियतरो लोके कृष्ण भविष्यसि}
{त्वन्मुखं च जगत्सर्वं भविष्ति न संशयः}


\twolineshloka
{तत्राभिगम्य राजेन्द्र पूजयित्वा वृषध्वजम्}
{अश्वमेधमवाप्नोति गाणपत्यं च विन्दति}


\twolineshloka
{धूमावतीं ततो गच्छेत्रिरात्रोपोषितो नरः}
{मनसा प्रार्थितान्कार्माल्लभते नात्र संशयः}


\threelineshloka
{देव्यास्तु दक्षिणार्धेन रथावर्तो नराधिप}
{तत्रारोहेत धर्मज्ञ श्रद्दधानो जितेन्द्रियः}
{महादेवप्रसादाद्धि गच्छेत परमां गतिम्}


\twolineshloka
{प्रदक्षिणमुपावृत्य गच्छेत भरतर्षभ}
{धारां नाम महाप्रात्र सर्वपापप्रमोचनीम्}


\twolineshloka
{तत्र स्नात्वा नरव्याघ्र न शोचति नराधिप}
{ततो गच्छेत धर्मज्ञ नमस्कृत् महागिरिम्}


\twolineshloka
{`अशीतियोजनशतं पुष्करं स्वर्गमुच्यते}
{अशीतिं धर्मपृष्ठात्तु प्रवदन्ति मनीषिणः}


\twolineshloka
{षष्टिं प्रयागाद्राजेन्द्र कुरुक्षेत्रात्तु द्वादश}
{संयुक्तमेव राजेन्द्र गङ्गाद्वारं त्रिविष्टपम्'}


\twolineshloka
{स्वर्गद्वारेण यत्तुल्यं गङ्गाद्वारं न संशयः}
{तत्राभिषेकं कुर्वीत कोटितीर्थे समाहितः}


\twolineshloka
{पौण्डरीकमवाप्नोति कुलं चैव समुद्धरेत्}
{उष्यैकां रजनीं तत्र गोसहस्रफलं लभेत्}


\twolineshloka
{सप्तगङ्गे त्रिगङ्गे च शक्रावर्ते च तर्पयन्}
{देवान्पितॄंश्च विधिवत्पुण्ये लोके महीयते}


\twolineshloka
{ततः कनस्वले स्नात्वा त्रिरात्रोपोषितो नरः}
{अश्वमेधमवाप्नोति स्वर्गलोकं च गच्छति}


\twolineshloka
{कपिलावटं ततो गच्छेत्तीर्थसेवी नराधिप}
{उपोष्य रजनीं तत्र गोसहस्रफलं लभेत्}


\twolineshloka
{नागराजस्य राजेन्द्र कपिलस्य महात्मनः}
{तीर्थं कुरुवरश्रेष्ठ सर्वलोकेषु विश्रुतम्}


\twolineshloka
{तत्राभिषेकं कुर्वीत नागतीर्थे नराधिप}
{कपिलानां सहस्रस्य फलं विन्दति मानवः}


\twolineshloka
{ततो ललितकं गच्छेच्छन्तनोस्तीर्थमुत्तमम्}
{तत्र स्नात्वा नरो राजन्न दुर्गतिमवाप्नुयात्}


\twolineshloka
{गङ्गायमुनयोर्मध्ये स्नाति यः संगमे नरः}
{दशाश्वमेधानाप्नोति कुलं चैव समुद्धरेत्}


\twolineshloka
{ततो गच्छेत राजेन्द्र सुगन्धं लोकविथुतम्}
{सर्वपापविशुद्धात्मा ब्रह्मलोके महीयते}


\twolineshloka
{रुद्रावर्तं ततो गच्छेत्तीर्थसेवी नराधिप}
{तत्रस्नात्वा नरो राजन्स्वर्गलोके च गच्छति}


\twolineshloka
{गङ्गायाश्च नरश्रेष्ठ सरस्वत्याश्च संगमे}
{स्नात्वाऽश्वमेधं प्राप्नोति स्वर्गलोकं च गच्छति}


\twolineshloka
{भद्रकर्णएश्वरं गत्वा देवमर्च्य यथाविधि}
{न दुर्गतिमवाप्नोति नाकपृष्ठे च पूज्यते}


\twolineshloka
{तत कुब्जावतीं गच्छेत्तीर्थसेवी नराधिप}
{रगोसहस्रमवाप्नोति स्वर्गलोकं च नच्छति}


\twolineshloka
{अरुन्धतीवटं गच्छेत्तीर्थसेवी नराधिप}
{सामुद्रकमुपस्पृश्य ब्रह्मचारी समाहितः}


\twolineshloka
{अश्वमेधमवाप्नोति त्रिरात्रोपोषितो नरः}
{गोसहस्रफलं विद्यात्कुलं चैव समुद्धरेत्}


\twolineshloka
{कब्र्हमावर्तं ततो गच्छेद्ब्रह्मचारी समाहितः}
{अश्वमेधमवाप्नोति सोमलोकं च गच्छति}


\twolineshloka
{यमुनाप्रभवं गत्वा समुपस्पृश्य यामुनम्}
{अश्वमेधफंल लब्ध्वा स्वर्गलोके महीयते}


\twolineshloka
{दर्वीसंक्रमणं प्राप्य तीर्थं त्रैलोक्यपूजितम्}
{अश्वमेधमवाप्नोति स्वर्गलोकं च गच्छति}


\twolineshloka
{सिन्धोश्च प्रभवं गत्वा सिद्धगन्धर्वसेवितम्}
{तत्रोष्य रजनीः पञ्च विन्देद्बहु सुवर्णकम्}


\twolineshloka
{अथ वेदीं समासाद्य नरः परमदुर्गमाम्}
{अश्वमेधमवाप्नोति गच्छेदौशनसीं गतिम्}


\twolineshloka
{ऋषिकुल्यां समासाद्य वासिष्ठं चैवभारत}
{वासिष्ठीं समतिक्रम्य सर्वे वर्णा द्विजातयः}


\twolineshloka
{ऋषिकुल्यां समासाद्य नरः स्नात्वा विकल्मषः}
{देवान्पितॄंश्चार्चयित्वा ऋषिलोकं प्रपद्यते}


\twolineshloka
{यदि तत्रवसेन्मासं शाकाहारो नराधिप}
{`द्वादशाहस्य यज्ञस्य फलं स लभते नरः'}


\twolineshloka
{भृगुतुङ्गं समासाद्य वाजिमेधफलं लभेत्}
{गत्वा वीरप्रमोक्षं च सर्वपापैः प्रमुच्यते}


\twolineshloka
{कृत्तिकामघयोश्चैव तीर्थमासाद्य भारत}
{अग्निष्टोमातिरात्राभ्यां फलमाप्नोति मानवः}


\twolineshloka
{तत्र संध्यां समासाद्य विद्यातीर्थमनुत्तमम्}
{उपस्पृश्य च वै विद्यां यत्र तत्रोपपद्यते}


\twolineshloka
{महाश्रमे वसेद्रात्रिं सर्वपापप्रमोचने}
{एककालं निराहारो लोकानावसते शुभान्}


\threelineshloka
{षष्ठाकालोपवासेन मासमुष्य महालये}
{सर्वपापविशुद्दात्मा विन्देद्बहु सुवर्णकम्}
{दशापरान्दशपूर्वान्नरानुद्धरते कुलम्}


\twolineshloka
{अथ वेतसिकां गत्वा पितामहनिषेविताम्}
{अश्वमेधमवाप्नोति गच्छेदौशनसीं गतिम्}


\twolineshloka
{अथ सुन्दरिकातीर्थं प्राप्य सिद्धनिषेवितम्}
{रूपस्य भागी भवति दृष्टमेतत्पुरातनैः}


\twolineshloka
{ततो वै ब्राह्मणं गत्वा ब्रह्मचारी जितेन्द्रियः}
{पद्मवर्णेन यानेन ब्रह्मलोकं प्रपद्यते}


\twolineshloka
{ततस्तु नैमिषं गच्छेत्पुण्यं सिद्धनिषेवितम्}
{तत्र नित्यं निवसति ब्रह्मा देवगणैः सह}


\twolineshloka
{नैमिषं मृगयाणस् पापस्यार्धं प्रणश्यति}
{प्रविष्टमात्रस्तु नरः सर्वपापैः प्रमुच्यते}


\twolineshloka
{तत्र मासं वसेद्धीरो नैमिषे तीर्थतत्परः}
{पृथिव्यां यानि तीर्थानि तानि तीर्थानि नैंमिषे}


\threelineshloka
{कृताभिषेकस्तत्रैव नियतो नियताशनः}
{गवां मेधस्य यज्ञस्य फलं प्राप्नोति भारत}
{पुनात्यासप्तमं चैव कुलं भरतसत्तम}


\twolineshloka
{यस्त्यजेन्नैमिषे प्राणानुपवासपरायणः}
{स मोदेत्सर्वलोकेषु एवमाहुर्मनीषिणः}


% Check verse!
नित्यं मेध्यं च पुण्यं च नैमिषं नृपसत्तम
\twolineshloka
{गङ्गोद्भेदं समासाद्य त्रिरात्रोपोषितो नरः}
{वाजपेयमवाप्नोति ब्रह्मभूतो भवेत्सदा}


\twolineshloka
{सरस्वतीं समासाद्य तर्पयेत्पितृदेवताः}
{सारस्वतेषु लोकेषु मोदते नात्र संशयः}


\threelineshloka
{ततश्च बाहुदां गच्छेद्ब्रह्मचारी समाहितः}
{तत्रोष्य रजनीमेकां स्वर्गलोके महीयते}
{देवसत्रस्य यज्ञस्य फलंप्राप्नोति कौरव}


\twolineshloka
{ततः क्षीरवतीं गच्छेत्पुण्यां पुण्यतरैर्वृताम्}
{पितृदेवार्चनपरो वाजपेयमवाप्नुयात्}


\twolineshloka
{विमलाशोकमासाद्य ब्र्हमचारी समाहितः}
{तत्रोष्य रजनीमेकां स्वर्गलोके महीयते}


\threelineshloka
{गोप्रतारं ततो गच्छेत्सरय्वास्तीर्थमुत्तमम्}
{यत्र रामो गतः स्वर्गं सभृत्यबलवाहनः}
{देहं त्यक्त्वा महाराज तस्य तीर्थस्य तेजसा}


\threelineshloka
{रामस्य च प्रसादेन व्यवसायाच्च भारत}
{तस्मिंस्तीर्थे नरः स्नात्वा गोप्रतारे नराधिप}
{सर्वपापविशुद्धात्मा स्वर्गलोके महीयते}


\twolineshloka
{रामतीर्थे नरः स्नात्वा गोमत्यां कुरुनन्दन}
{अश्वमेधमवाप्नोति पुनाति च कुलं नरः}


% Check verse!
शतसाहस्रकं तीर्थं तत्रैव भरतर्षभ
\twolineshloka
{तत्रोपस्पर्शनं कृत्वा नियतो नियताशनः}
{गोसहस्रफलं पुण्यं प्राप्नोति भरतर्षभ}


\twolineshloka
{रततो गच्छेत राजेनद््र भर्तृस्थानमनुत्तमम्}
{अश्वमेधस्य यज्ञस्य फलं प्राप्नोति मानवः}


\twolineshloka
{कोटितीर्थे नरः स्नात्वा अर्चयित्वा गुहं नृप}
{गोसहस्रफंल विद्यात्तेजस्वी च भवेन्नरः}


\twolineshloka
{ततो वाराणसीं गत्वा अर्चयित्वा नृषध्वजम्}
{कपिलाह्रदे नरः स्नात्वाराजसूयमवाप्नुयात्}


\twolineshloka
{अविमुक्तं समासाद्य तीर्थसेवी कुरूद्वह}
{दर्शनाद्देवदेवस्य मुच्यते ब्रह्महत्यया}


\twolineshloka
{प्राणानुत्सृज्यतत्रैव मोक्षं प्राप्नोति मानवः}
{मार्कण्डेयस्य राजेन्द्र तीर्थमासाद्य दुर्लभम्}


\twolineshloka
{गोमतीगङ्गयोश्चैव संगमे लोकविश्रुते}
{अग्निष्टोममवाप्नोति कुलं चैव समुद्धरेत्}


\twolineshloka
{ततो गयां समासाद्य ब्रह्मचारी समाहितः}
{अश्वमेधमवाप्नोति कुलं चैव समुद्धरेत्}


\twolineshloka
{तत्राक्षयवटो नाम त्रिषु लोकेषु विश्रुतः}
{तत्र दत्तं पितृभ्यस्तु भवत्यक्षयमुच्यते}


\twolineshloka
{महानद्यामुपस्पृश्य तर्पयेत्पितृदेवताः}
{अक्षयान्प्राप्नुयाल्लोकान्कुलं चैव समुद्धरेत्}


\twolineshloka
{ततो ब्रह्मसरो गत्वा धर्मारण्योपशोभितम्}
{ब्रह्मलोकमवाप्नोति प्रभातामेव शर्वरीम्}


\twolineshloka
{ब्रह्मणा तत्र सरसि यूपश्रेष्ठः समुच्छ्रितः}
{यूपं प्रदक्षिणीकृत्य वाजपेयफलं लभेत्}


\twolineshloka
{ततो गच्छेत राजेन्द्र धेनुकं लोकविश्रुतम्}
{एकरात्रोषितो राजन्प्रयच्छेत्तिलधेनुकाम्}


\twolineshloka
{सर्वपापविशुद्धात्मा सोमलोकं ब्रजेद्भुवम्}
{तत्र चिह्निं महद्राजन्नद्यापि सुमहद्भृशम्}


\twolineshloka
{कपिलायाः सवत्सायाश्चरन्त्याः पर्वते कृतम्}
{सवत्सायाः पदानि स्म दृश्यन्तेऽद्यापि भारत}


\twolineshloka
{तेषूपस्पृश्य राजेन्द्रपदेषु नृपसत्तम}
{यत्कंचिदशुभं कर्म तत्प्रणश्यति भारत}


\twolineshloka
{ततो गृध्रवटं गच्छेत्स्थानं देवस्य धीमतः}
{स्नायीत भस्मना तत्र अभिगम्य वृषध्वजम्}


\twolineshloka
{ब्राह्मणेन भवेच्चीर्णं व्रतं द्वादशवार्षिकम्}
{इतरेषां तु वर्णानां सर्वपापं प्रणश्यति}


\twolineshloka
{उद्यन्तं च ततो गच्छेत्पर्वतं गीतनादितम्}
{सावित्र्यास्तु पदं तत्र दृश्यते भरतर्षभ}


\twolineshloka
{तत्रसंध्यामुपासीत ब्राह्मणः संशितव्रतः}
{उपासिता भवेत्संध्या तेन द्वादशवार्षिकी}


\twolineshloka
{योनिद्वारं च तत्रैव विश्रुतं भरतर्षभ}
{तत्राभिगम्य मुच्येत पुरुषो योनिसंकरात्}


\twolineshloka
{कृष्णशुक्लावुभौ पक्षौ गयायां यो वसेन्नरः}
{पुनात्यासप्तमं राजन्कुलं नास्त्यत्र संशयः}


\twolineshloka
{एष्टव्या बहवः पुत्रा यद्येकोपि गयां व्रजेत्}
{गौरीं वा वरयेत्कन्यां नीलं वा वृषमुत्सृजेत्}


\twolineshloka
{ततः फल्गुं व्रजेद्राजंस्तीर्थसेवी नराधिप}
{अश्वमेधमवाप्नोति सिद्धिं च महतीं व्रजेत्}


\twolineshloka
{ततो गच्छेत राजेन्द्र धर्मप्रस्थं समाहितः}
{तत्र धर्मो महाराज नित्यमास्ते युधिष्ठिर}


\twolineshloka
{तत्र कूपोदकं कृत्वा तेन स्नातः शुचिस्तथा}
{पितॄन्देवांस्तु संतर्प्य मुक्तपापो दिवं व्रजेत्}


% Check verse!
मतङ्गस्याश्रमस्तत्रमहर्षेर्भावितात्मनः
\twolineshloka
{तं प्रविश्याश्रमं श्रीमच्छ्रमशोकविनाशनम्}
{गवामयनयज्ञस्य फलंप्राप्नोति मानवः}


% Check verse!
धर्मं तत्राभिसंस्पृश्य वाजिमेधमवाप्नुयात्
\threelineshloka
{ततो गच्छेत राजेन्द्र ब्रह्मस्थानमनुत्तमम्}
{तत्राभिगम्य राजेन्द्र ब्रह्माणं पुरुषर्षभ}
{राजसूयाश्वमेधाभ्यां फलं विन्दति मानवः}


\twolineshloka
{ततो राजगृहं गच्छेत्तीर्थसेवी नराधिप}
{उपस्पृश्य ततस्तत्रकक्षीवानिव मोदते}


\twolineshloka
{यक्षिण्या नैत्यकं तत्र प्राश्नीन पुरुषः शुचिः}
{यक्षिण्यास्तु प्रसादेन मुच्यते ब्र्हमहत्यया}


% Check verse!
मणइनागं ततो गत्वा गोसहस्रफलं लभेत्
\twolineshloka
{तैर्थिकं भुञ्जते यस्तु मणिनागस् भारत}
{दष्टस्याशीविषेणापि न तस्य क्रमते विपम्}


\twolineshloka
{तत्रोष् रजनीभेकां गोसहस्रफंल लभेत्}
{ततो गच्छेत ब्रह्मर्पेर्गौतमस्य वनं प्रियम्}


\twolineshloka
{अहल्याया ह्रदे स्नात्वा व्रजेत परमां गतिम्}
{अभिगम्याश्रमं राजन्विनद्ते श्रियमात्मनः}


\twolineshloka
{तत्रोदपानं धर्मज्ञ त्रिपु लोकेषु विश्रुतम्}
{तत्राभिषेकं कृत्वा तु वाजिमेधमवाप्नुयात्}


\twolineshloka
{जनकस् तु राजर्षेः कूपस्त्रिदशपूजितः}
{तत्राभिषेकं कृत्वा तु विष्णुलोकमवाप्नुयात्}


\twolineshloka
{ततो विनशनं गच्छेत्सर्वपापप्रमोचनम्}
{वाजपेयमवाप्नोति सोमलोकं च गच्छति}


\twolineshloka
{गण्डकीं तु समासाद्य सर्वतीर्थजलोद्भवाम्}
{वाजपेयमवाप्नोति सूर्यलोकं च गच्छति}


\twolineshloka
{कततो विशल्यामासाद्य नदीं त्रैलोक्यविश्रुताम्}
{अग्निष्टोममवाप्नोति स्वर्गलोकं च गच्छति}


\twolineshloka
{ततोऽधिवङ्गं धर्मज्ञ समाविश्य ततो वनम्}
{गुह्यकेषु महाराज मोदते नात्र संशयः}


\twolineshloka
{कम्पनां तु समासद्य नदीं सिद्धनिवेषिताम्}
{पुण्डरीकमवाप्नोति स्वर्गलोकं च गच्छति}


\twolineshloka
{अथ माहेश्वरीं धारां समासाद्य धराधिप}
{अश्वमेधमवाप्नोति कुलं चैव समुद्धरेत्}


\twolineshloka
{दिवौकसां पुष्करिणीं समासाद्य नराधिप}
{न दुर्गतिमवाप्नोति वाजिमेधं च विन्दति}


\twolineshloka
{अथ सोमपदं गच्छेद्ब्रह्मचारी समाहितः}
{माहेश्वरपदे स्नात्वा वाजिमेधफलं लभेत्}


\twolineshloka
{तत्रकोटी तु तीर्थानां विश्रुता भरतर्षभ}
{कूर्मरूपेण राजेन्द्र ह्यसुरेण दुरात्मना}


\twolineshloka
{ह्रियमाणा हृता राजन्विष्णुना प्रभविष्णुना}
{तत्राभिषेकं कुर्वीत तीर्थकोट्यां युधिष्ठिर}


\twolineshloka
{पुण्डरीकमवाप्नोति विष्णुलोकं च गच्छति}
{ततो गच्छेत राजेन्द्रस्थानं नारायणस्य च}


\twolineshloka
{सदा सन्निहितो यत्रविष्णुर्वसति भारत}
{यत्रब्रह्मादयो देवा ऋषयश्च तपोधनाः}


\twolineshloka
{आदित्या वसवो रुद्रा जनार्दनमुपासते}
{शालग्राम इति ख्यातो विष्णुरद्भुतकर्मकः}


\twolineshloka
{अभिगम्य त्रिलोकेशं वरदं विष्णुमव्ययम्}
{अश्वमेधमवाप्नोति विष्णुलोकं च गच्छति}


\threelineshloka
{तत्रोपदानं धर्मज्ञ सर्वपापप्रमोचनम्}
{समुद्रास्तत्रचत्वारः कूपे संनिहिताः सदा}
{तत्रोपस्पृश् राजेन्द्र न दुर्गतिमवाप्नुयात्}


\twolineshloka
{अभिगम्य त्रिलोकेशं वरदं विष्णुमव्ययम्}
{विराजति यथासोमो मेघैर्मुक्तो नराधिप}


\twolineshloka
{जातिस्मरमुपस्पृश्य शुचिः प्रयतमानसः}
{जातिस्मरत्वमाप्नोति स्नात्वा तत्र न संशयः}


\twolineshloka
{वदेश्वरपुरं गत्वा अर्चयित्वा तु केशवम्}
{ईप्सिताँल्लभते कामानुपवासान्न संशयः}


\twolineshloka
{ततस्तु वामनं गत्वा सर्वपापप्रमोचनम्}
{अभिगम्य हरिं देवं न दुर्गतिमवाप्नुयात्}


% Check verse!
भरतस्याश्रमं गच्छेत्सर्वपापप्रमोचनम्
\twolineshloka
{कौशिकीं तत्र गच्छेत महापापप्रणाशिनीम्}
{राजसूयस्य यज्ञस्य फलं प्राप्नोति मानवः}


\twolineshloka
{ततो गच्छेत राजेन्द्र चम्पकारण्यमुत्तमम्}
{तत्रोष्य रजनीमेकां गोसहस्रफलं लभेत्}


\twolineshloka
{अथ जेष्ठिलमासाद्य तीर्धं परमदुर्लभम्}
{तत्रोष्य रजनीमेकां गोसहस्रफलं लभेत्}


\threelineshloka
{तत्र विश्वेश्वरं दृष्ट्वा देव्या सह महाद्युतिम्}
{मित्रावरुणयोर्लोकानाप्नोति पुरुषर्षभ}
{त्रिरात्रोपोषितस्तत्रअग्निष्टोमफलं लभेत्}


\twolineshloka
{कन्यासंवेद्यमासाद्य नियतो नयिताशनः}
{मनोः प्रजापतेर्लोकानाप्नोति पुरुषर्षभ}


\twolineshloka
{कन्यायां ये प्रयच्छन्ति दानमण्वपि भारत}
{तदक्षय्यमिति प्राहुर्ऋषयः संशितव्रताः}


\twolineshloka
{ततो निर्वीरमासाद्य त्रिपु लोकेपु विश्रुतम्}
{अश्वमेधमवाप्नोति विष्णुलोकं च गच्छति}


\twolineshloka
{ये त्विन्धनं प्रयच्छन्ति निर्वीरासंगमे नराः}
{ते यान्ति नरशार्दूल शक्रलोकमनामयम्}


\twolineshloka
{तत्राश्रमो वसिष्ठस् त्रिषु लोकेषु विश्रुतः}
{तत्राभिषेकं कुर्वाणो वाजपेयमवाप्नुयात्}


\twolineshloka
{देवकूटं समासाद्य ब्रह्मर्षिगणसेवितम्}
{अश्वमेधमवाप्नोति कुलं चैव समुद्धरेत्}


\twolineshloka
{ततो गच्छेत राजेन्द्र कौशिकस्य मुनेर्ह्रदम्}
{यत्र सिद्धिं परां प्राप्तो विश्वामित्रोथ कौशिकः}


\twolineshloka
{तत्र मासं वसेद्वीर कौशिक्यां भरतर्षभ}
{अश्वमेधस्य यत्पुण्यं तन्मासेनाधिगच्छति}


\twolineshloka
{सर्वतीर्थवरे चैव यो वसेत महाह्रदे}
{न दुर्गतिमवाप्नोति विन्द्याद्बहु सुवर्णकम्}


\twolineshloka
{कुमारमभिगम्याथ वीराश्रमनिवासिनम्}
{अश्वमेधमवाप्नोति नरो नास्त्यत्र संशयः}


\twolineshloka
{अग्निधारां समासाद्य त्रिषु लोकेषु विश्रुताम्}
{तत्राभिषेकं कुर्वाणो ह्यग्निष्टोममवाप्नुयात्}


\threelineshloka
{अभिगम्य महादेवं वरदं विष्णुमव्ययम्}
{पितामहसरो गत्वा शैलराजसमीपतः}
{तत्राभिषेकं कुर्वाणो ह्यग्निष्टोममवाप्नुयात्}


\twolineshloka
{पितामहस्य सरसः प्रस्रुता लोकपावनी}
{कुमारधारा तत्रैव त्रिषु लोकेषु विश्रुता}


\twolineshloka
{यत्रस्नात्वा कृतार्थोस्मीत्यात्मानमवगच्छति}
{षष्ठकालोपवासेन मुच्यते ब्रह्महत्यया}


\twolineshloka
{ततो गच्छेत धर्मज्ञ तीर्थसेवनतत्परः}
{शिखरं वै महादेव्या गौर्यास्त्रैलोक्यविश्रुतम्}


\twolineshloka
{समारुह्य नरश्रेष्ठ स्तनकुण्डेषु संविशेत्}
{स्तनकुण्डमुपस्पृश्य वाजपेयफलं लभेत्}


\twolineshloka
{तत्राभिषेकं कुर्वाणः पितृदेवार्चने रतः}
{हयमेधमवाप्नोति शक्रलोकं च गच्छति}


\twolineshloka
{ताम्रारुणं समासाद्य ब्रह्मचारी समाहितः}
{अश्वमेमवाप्नोति ब्रह्मलोकं च गच्छति}


\twolineshloka
{नन्दिन्यां च समासाद्य कूपं देवनिषेवितम्}
{नरमेधस् यत्पुण्यं तदाप्तोति नराधिप}


\twolineshloka
{कालिकासंगमे स्नात्वा कौशिक्यरुणयोर्गतः}
{त्रिरात्रोपोषितो राजन्सर्वपापैः प्रमुच्यते}


\twolineshloka
{उर्वशीतीर्थमासाद्य ततः सोमाश्रमं बुधः}
{कुम्भकर्णाश्रमं गत्वा पूज्यते भुविमानवः}


\twolineshloka
{कोकामुखमुपस्पृश् ब्रह्मचारी यतव्रतः}
{जातिस्मरत्वमाप्नोति दृष्टमेतत्पुरातनैः}


\twolineshloka
{प्राङ्गदीं च समासाद्य कृतात्मा भवति द्विजः}
{सर्वपापविशुद्धात्मा शक्रलोकं च गच्छति}


\twolineshloka
{ऋषभद्वीपमासाद्य मेध्यं क्रौञ्चनिषूदनम्}
{सरस्वत्यामुपस्पृश्य विमानस्थो विराजते}


\twolineshloka
{औद्दालकं महाराज तीर्थं मुनिनिषेवितम्}
{तत्राभिषेकं कृत्वा वै सर्वपापैः प्रमुच्यते}


\twolineshloka
{धर्मतीर्थं समासाद्य पुण्यं ब्रह्मर्षिसेवितम्}
{वाजपेयमवाप्नोति विमानस्थश्च पूज्यते}


\twolineshloka
{अथ पम्पां समासाद्य भागीरथ्यां कृतोदकः}
{दण्डार्तमभिगत्वा तु गोसहस्रफलं लभेत्}


\twolineshloka
{`ततो लेवलिकां गच्छेत्पुण्यां पुण्योपसेविताम्}
{वाजपेयमवाप्नोति विमानस्थश्च पूज्यते'}


\chapter{अध्यायः ८३}
\twolineshloka
{पुलस्त्य उवाच}
{}


\twolineshloka
{अथ सन्ध्यां समासाद्य संवेद्ये तीर्थ उत्तमे}
{उपस्पृश्य नरो विद्यां लभते नात्र संशयः}


\twolineshloka
{रामस्य च प्रसादेन तीर्थं राजन्कृतं पुरा}
{तल्लौहित्यं समासाद्य विन्द्याद्बहु सुवर्णकम्}


\twolineshloka
{करतोयां समासाद्य त्रिरात्रोपोषितो नरः}
{अश्वमेधमवाप्नोति प्रजापतिकृतो विधिः}


\twolineshloka
{गङ्गायास्तत्र राजेन्द्र सागरस्य च सङ्गमे}
{अश्वमेधं दशगुणं प्रवदन्ति मनीषिणः}


\twolineshloka
{गङ्गायास्त्वपरं पारं प्राप्य यः स्नाति मानवः}
{त्रिरात्रमुषितो राजन्सर्वान्कामानवाप्नुयात्}


\twolineshloka
{ततो वैतरणीं गत्वा सर्वपापप्रमोचनीम्}
{विरजातीर्थमासाद्य विराजति यथा शशी}


\twolineshloka
{प्रभवेच्च कुले पुण्ये सर्वपापं व्यपोहति}
{गोसहस्रफंल लब्ध्वा रपुनाति स्वकुलं नरः}


\twolineshloka
{शोणस्य ज्योतिरथ्यायाः संगमे नियतः शुचिः}
{तर्पयित्वा पितॄन्देवानग्निष्टोमफलं लभेत्}


\twolineshloka
{शोणस्य नर्मदायाश् प्रभेदे कुरुनन्दन}
{वंशगुल्म उपस्पृश्य वाचजिमेधफलं लभेत्}


\twolineshloka
{वाजपेयमवाप्नोति त्रिरात्रोपोषितो नरः}
{गोसहस्रफलं विन्द्यात्कुलं चैव समुद्धरेत्}


\twolineshloka
{कोसलां तु समासाद्य कालतीर्थमुपस्पृशेत्}
{वृषभैकादशफलं लभते नात्र संशयः}


\twolineshloka
{पुष्पवत्यामुपस्पृश्य त्रिरात्रोपोषितो नरः}
{गोसहस्रफलं लब्ध्वा पुनाति स्वकुलं नृप}


\twolineshloka
{ततो बदरिकातीर्थे स्नात्वा भरतसत्तम}
{दीर्घमायुरवाप्नोति स्वर्गलोकं च गच्छति}


\twolineshloka
{तथा चम्पां समासाद्य भागीरथ्यां कृतोदकः}
{दण्डाख्यमभिगम्यैव गोसहस्रफलं लभेत्}


\twolineshloka
{लपेटिकां ततो गच्छेत्पुण्यां पुण्योपशोभिताम्}
{वाजपेयमवाप्नोति देवैः सर्वैश्च पूज्यते}


\twolineshloka
{ततो महेन्द्रमासाद्य जामदग्न्यनिपेवितम्}
{रामतीर्थे नरः स्नात्वा वाजिमेधफलं लभेत्}


\twolineshloka
{मतङ्गस्य तु केदारस्तत्रैव कुरुनन्दन}
{तत्र स्नात्वा कुरुश्रेष्ठ गोसहस्रफलं लभेत्}


\twolineshloka
{श्रीपर्वतं समासाद्य नदीतीरमुपस्पृशेत्}
{रअश्वमेधमवाप्नोति पूजयित्वा वृषध्वजम्}


\twolineshloka
{श्रीपर्वते महादेवो देव्या सह महाद्युतिः}
{न्यवसत्परमप्रीतो ब्रह्मा च त्रिदशैः सह}


\twolineshloka
{तत्र देवह्रदे स्नात्वा शुचिः प्रयतमानसः}
{अश्वमेधमवाप्नोति कुलं चैव समुद्धरेत्}


\twolineshloka
{ऋषभं पर्वतं गत्वा पाण्ड्येषु नृपपूजितम्}
{वाजपेयमवाप्नोति नाकपृष्ठे च मोदते}


\twolineshloka
{ततो गच्छेत कावेरीं वृतामप्सरसां गणैः}
{तत्रस्नात्वा नरो राजन्गोसहस्रफलं लभेत्}


\twolineshloka
{ततस्तीरे समुद्रस्य कन्यातीर्थमुपस्पृशेत्}
{तत्तोयं स्पृश्य राजेन्द्र सर्वपापैः प्रमुच्यते}


\twolineshloka
{अथ गोकर्णमासाद्य त्रिषु लोकेषु विश्रुतम्}
{समुद्रमध्ये राजेन्द्र सर्वलोकनमस्कृतम्}


\twolineshloka
{रयत्र ब्रह्मादयो देवा ऋषयश्च तपोधनाः}
{भूतयक्षपिशाचाश्च किन्नराः समहोरगाः}


\twolineshloka
{सिद्धचारणगन्धर्वमानुषाः पन्नगास्तथा}
{सरितः सागराः शैला उपासन्त उमापतिम्}


\twolineshloka
{तत्रेशानं समभ्यर्च्य त्रिरात्रोपोषितो नरः}
{अश्वमेधमवाप्नोति गाणपत्यं च विन्दति}


% Check verse!
उष्य द्वादशरात्रं तु पूतात्मा च भवेन्नरः
\threelineshloka
{तत एव च गायत्र्याः स्थानं त्रैलोक्यपूजितम्}
{त्रिरात्रमुषितस्तत्र गोसहस्रफलं लभेत्}
{निदर्शनं च प्रत्यक्षं ब्राह्मणानां नराधिप}


\twolineshloka
{गायत्रीं पठते यस्तु योनिसंकरजस्तथा}
{गाथा च गाथिका चापि तस्य संपद्यते नृप}


% Check verse!
अब्राह्मणस्य सावित्रीं पठतस्तु प्रणश्यति
\twolineshloka
{संवर्तस्य तु विप्रर्षेर्वापीमासाद्य दुर्लभाम्}
{रूपस्य भागी भवति सुभगश्च प्रजायते}


\twolineshloka
{ततो वेणां समासाद्य त्रिरात्रोपोषितो नरः}
{मयूरहंससंयुक्तं विमानं लभते नरः}


\twolineshloka
{ततो गोदावरीं प्राप्य नित्यं सिद्धनिषेविताम्}
{गवां मेधमवाप्नोति वासुकेर्लोकमुत्तमम्}


\twolineshloka
{वेणायाः संगमे स्नात्वा वाजिमेधफलं लभेत्}
{वरदासंगमे स्नात्वा गोसहस्रफलं लभेत्}


\twolineshloka
{ब्रह्मस्थानं समासाद्य त्रिरात्रोपोषितो नरः}
{गोसहस्रफलं विन्द्यात्स्वर्गलोकं च गच्छति}


\twolineshloka
{कुशप्लवनमासाद्य ब्रह्मचारी समाहितः}
{त्रिरात्रमुषितः स्नात्वा अश्वमेधफंल लभेत्}


\threelineshloka
{ततो देवह्रदेऽरण्ये कृष्णवेणाजलोद्भवे}
{जातिस्मरह्रदे स्नात्वा भवेज्जातिस्मरो नरः}
{यत्र क्रतुशतैरिष्ट्वा देवराजो दिवं गतः}


\twolineshloka
{अग्निष्टोमफलं विन्द्याद्गमनादेव भारत}
{ततः सर्वह्रदे स्नात्वा गोसहस्रफलं लभेत्}


\twolineshloka
{ततो वापीं महापुण्यां पयोष्णीं सरितां वराम्}
{पितृदेवार्चनरतो गोसहस्रफलं लभेत्}


\twolineshloka
{दण्डकारण्यमासाद्य पुण्यं राजन्नुपस्पृशेत्}
{गोसहस्रफलं तस्य स्नातमात्रस्य भारत}


\twolineshloka
{शरभङ्गाश्रमं गत्वा शुकस्य च महात्मनः}
{न दुर्गतिमवाप्नोति पुनाति च कुलं नरः}


\twolineshloka
{ततः शूर्पारकं गच्छेज्जामदग्न्यनिषेवितम्}
{रामतीर्थे नरः स्नात्वा विन्द्याद्बहु सुवर्णकम्}


\twolineshloka
{सप्तगोदावरे स्नात्वा नियतो नियताशनः}
{महत्पुण्यमवाप्नोति देवलोकं च गच्छति}


\twolineshloka
{ततो देवपथं गत्वा नियतो नियताशनः}
{देवसत्रस्य यत्पुण्यं तदेवाप्नोति मानवः}


\twolineshloka
{तुङ्गकारण्यमासाद्य ब्र्हमचारी जितेन्द्रियः}
{वेदानध्यापयत्तत्र ऋषिः सारस्वतः पुरा}


\twolineshloka
{तत्र वेदेषु नष्टेषु मुनेरङ्गिरसः सुतः}
{ऋषीणआमुत्तरीयेषु सूपविष्टो यथासुखम्}


\twolineshloka
{ओंकारेण यथान्यायं सम्यगुच्चारितेन ह}
{येन यत्पूर्वमभ्यस्तं तत्सर्वं समुपस्थितम्}


\twolineshloka
{ऋषयस्तत्र देवाश्च वरुणोऽग्निः प्रजापतिः}
{हरिर्नारायणस्तत्र महादेवस्तथैव च}


\twolineshloka
{पितामहश्च भगवान्देवैः सह महाद्युतिः}
{भृगुं नियोजयामास याजनार्थे महाद्युतिम्}


\twolineshloka
{ततः स चक्रे भगवानृषीणां विधिवत्तदा}
{सर्वेषां पुनराधानं विधिदृष्टेन कर्मणा}


\twolineshloka
{आज्यभागेन तत्राग्नीं तर्पयित्वा यथाविधि}
{देवाः स्वभवनं याता ऋषयश्च यथाक्रमम्}


\twolineshloka
{तदरण्यं प्रविष्टस्य तुङ्गकं राजसत्तम}
{पापं प्रणश्यत्यखिलं स्त्रिया वा पुरुषस्य वा}


\twolineshloka
{तत्रमासं वसेद्धीरो नियतो नियताशनः}
{ब्रह्मलोकं व्रजेद्राजन्कुलं चैव समुद्धरेत्}


\twolineshloka
{मेधाविकं समासाद्य पितॄन्देवांश्च तर्पयेत्}
{अग्निष्टोममवाप्नोति स्मृतिं मेधां च विन्दति}


\twolineshloka
{अत्र कालञ्चरं गत्वा पर्वतं लोकविश्रुतम्}
{तत्र देवह्रदे स्नात्वा गोसहस्रफलं लभेत्}


\twolineshloka
{आत्मानं स्नापयेत्तत्रगिरौ कालञ्जरे नृप}
{स्वर्गलोके महीयेत नरो नास्त्यत्र संशयः}


\twolineshloka
{ततो गिरिवरश्रेष्ठे चित्रकूटे विशांपते}
{मन्दाकिनीं समासाद्य सर्वपापप्रणाशिनीम्}


\twolineshloka
{तत्राभिषेकं कुर्वाणः पितृदेवार्चने रतः}
{अश्वमेधमवाप्नोति गतिं च परमां व्रजेत्}


\twolineshloka
{ततो गच्छेत धर्मज्ञ भर्तृस्थानमनुत्तमम्}
{यत्र नित्यं महासेनो गुहः सन्निहितो नृप}


\twolineshloka
{तत्र गत्वा नृपश्रेष्ठ गमनादेव सिध्यति}
{कोटितीर्थे नरः स्नात्वा गोसहस्रफलं लभेत्}


\twolineshloka
{प्रदक्षिणमुपावृत्य ज्येष्ठस्थानं व्रजेन्नरः}
{अभिगम्य महादेवं विराजति यथा शशी}


\twolineshloka
{तत्र कूपे महाराज विश्रुता भरतर्षभ}
{समुद्रास्तत्र चत्वारो निवसन्ति युधिष्ठिर}


\twolineshloka
{तत्रोपस्पृश्य राजेन्द्र पितृदेवार्चने रतः}
{नियतात्मा नरः पूतो गच्छेत परमां गतिम्}


\twolineshloka
{ततो गच्छेत राजेन्द्र शृङ्गबेरपुरं महत्}
{यत्रतीर्णो महाराज रामो दाशरथिः पुरा}


\twolineshloka
{रकतस्मिंस्तीर्थे महाबाहो स्नात्वा पापैः प्रमुच्यते}
{गङ्गायां तु नरः स्नात्वा ब्रह्मचारी समाहितः}


\twolineshloka
{विधूतपाप्मा भवति वाजपेयं च विन्दति}
{ततो मुञ्जवटं गच्छेत्स्थानं देवस्य धीमतः}


\twolineshloka
{अभिगम्य महादेवमभिवाद्य च भारत}
{प्रदक्षिणमुपावृत् यगाणपत्यमवाप्नुयात्}


\twolineshloka
{तस्मिंस्तीर्थे तु जाह्नव्यां स्नात्वा पापैः प्रमुच्यते}
{ततो गच्छेत राजेन्द्र प्रयागमृषिसंस्तुतम्}


\twolineshloka
{यत्र ब्रह्मादयो देवा दिशश्च सदिगीश्वराः}
{लोकपालाश्च साध्याश्च पितरो लोकसंमताः}


\twolineshloka
{सनत्कुमारप्रमुखास्तथैव परमर्षयः}
{अङ्गिरःप्रमुखाश्चैव तथा ब्रह्मर्षयोऽमलाः}


\twolineshloka
{तथा नागाः सुपर्णश्च सिद्धाश्चक्रचरास्तथा}
{सरितः सागराश्चैव गन्धर्वाप्सरसोऽपि च}


\twolineshloka
{हरिश्च भगवानास्ते प्रजापतिपुरस्कृतः}
{तत्र त्रीण्यग्निकुण्डानि येषां मध्येन जाह्नवी}


\twolineshloka
{प्रयागादभिनिष्क्रान्ता सर्वतीर्थपुरस्कृता}
{तपनस्य सुता देवी त्रिषु लोकेषु विश्रुता}


\twolineshloka
{यमुना गङ्गया सार्धं संगता लोकपावनी}
{गङ्गायमुनयोर्मध्यं पृथिव्या जघनं स्मृतम्}


\twolineshloka
{प्रयागं जघनस्थानमुपस्थमृषयो विदुः}
{प्रयागं सप्रतिष्ठानं कम्बलाश्वतरौ तथा}


\twolineshloka
{तीर्थं भोगवती चैव वेदिरेषा प्रजापतेः}
{तत्रवेदाश्च यज्ञाश्च मूर्तिमन्तो युधिष्ठिर}


\twolineshloka
{प्रजापतिमुपासन्ते ऋषयश्च तपोधनाः}
{यजन्ते क्रतुभिर्देवास्तथा चक्रधरा नृपाः}


\twolineshloka
{ततः पुण्यतमं नाम त्रिषु लोकेषु भारत}
{प्रयागं सर्वतीर्थेभ्यः प्रवदन्त्यधिकं विभो}


\twolineshloka
{श्रवणात्तस्य तीर्थस् नामसंकीर्तनादपि}
{मृत्युकालभयाच्चापि नरः पापात्प्रमुच्यते}


\twolineshloka
{तत्राभिषेकं यः कुर्यात्संगमे लोकविश्रुते}
{पुण्यं स फलमाप्नोति राजसूयाश्वमेधयोः}


\twolineshloka
{एषा यजनभूमिर्हि देवानामभिसंस्कृता}
{तत्र रदत्तं सूक्ष्ममपि महद्भवति भारत}


\twolineshloka
{न वेदवचनात्तात न लोकवचनादपि}
{मतिरुत्क्रमणीया ते प्रयागमरणं प्रति}


\twolineshloka
{दशतीर्तसहस्राणि षष्टिः कोट्यस्तथाऽपराः}
{येषां सान्निध्यमत्रैव कीर्तितं कुरुन्दन}


\twolineshloka
{चतुर्विद्ये च यत्पुण्यं सत्यवादिषु चैव यत्}
{स्नात एव तदाप्नोति गङ्गायमुनसंगमे}


\twolineshloka
{तत्र भोगवती नाम वासुकेस्तीर्थमुत्तमम्}
{तत्राभिषेकं यः कुर्यात्सोऽश्वमेधफलं लभेत्}


\twolineshloka
{तत्र हंसप्रपतनं तीर्थं त्रैलोक्यविश्रुतम्}
{दशाश्वमेधिकं चैव गङ्गायां कुरुनन्दन}


\twolineshloka
{कुरुक्षेत्रसमा गङ्गा यत्रतत्रावगाहिता}
{विशेषो वै कनखले प्रयागे परमं महत्}


\twolineshloka
{यद्यकार्यशतं कृत्वा कृतंगङ्गावसेचनम्}
{सर्वं तत्तस्य गङ्गापो दहत्यग्निरिवेन्धनम्}


\twolineshloka
{सर्वं कृतयुगे पुण्यंत्रेतायां पुष्करं स्मृतम्}
{द्वापरेऽपि कुरुक्षेत्रं गङ्गा कलियुगे स्मृता}


\twolineshloka
{पुष्करे तु तपस्तप्येद्दानं दद्यान्महालये}
{मलये त्वग्निमारोहेद्भृगुतुन्दे त्वनाशनम्}


\twolineshloka
{पुष्करे तु कुरुक्षेत्रे रगङ्गायां मगधेषु च}
{स्नात्वा तारयते जन्तुः सप्तसप्तावरांस्तथा}


\twolineshloka
{पुनाति कीर्तिता पापं दृष्टा भद्रं प्रयच्छति}
{अवगाढा च पीता च पुनात्यासप्तमं कुलम्}


\twolineshloka
{यावदस्थि मनुष्यस् गङ्गायाः स्पृशते जलम्}
{तावत्स पुरुषो राजन्स्वर्गलोके महीपते}


\twolineshloka
{यथा पुण्यानि तीर्थानि पुण्यान्यायतनानि च}
{उपास्य पुण्यं लब्ध्वा च भवत्यमरलोकभाक्}


\twolineshloka
{न गङ्गासदृशं तीर्थं न देवः केशवात्परः}
{ब्राह्मणेभ्यः परं नास्ति एवमाह पितामहः}


\twolineshloka
{यत्र गङ्गा महाराज स देशस्तत्तपोकवनम्}
{सिद्धिक्षेत्रं च तज्ज्ञेयं गङ्गातीरसमाश्रितम्}


\twolineshloka
{इदं सत्यंद्विजातीनां साधूनामात्मनोपि च}
{सुहृदां च जपेत्कर्मे शिष्यस्यानुगतस्य च}


\twolineshloka
{इदं धन्यमिदं मेध्यमिदं स्वर्ग्यमिदं सुखम्}
{इदं पुण्यमिदं रम्यं पावनं धर्म्यमुत्तमम्}


\twolineshloka
{महर्षीणामिदं गुह्यं सर्वपापप्रमोचनम्}
{अधीत्यद्विजमध्ये च निर्मलः स्वर्गमाप्नुयात्}


\twolineshloka
{श्रीमत्स्वर्ग्यं तथा पुण्यं सपत्नशमनं शिवम्}
{मेधाजननमग्र्यं वै तीर्थवंशानुकीर्तनम्}


\threelineshloka
{अपुत्रो लभते पुत्रमधनो धनमाप्नुयात्}
{शूद्रो यथेप्सितान्कामान्ब्राह्मणः पारगः पठन्}
{महीं विजयते राजा वैश्यो धनमवाप्नुयात्}


\twolineshloka
{यश्चेदं शृणुयान्नित्यं तीर्थपुण्यं नरः शुचिः}
{जातीः स स्मरते बह्वीर्नाकपृष्ठे च मोदते}


\twolineshloka
{गम्यान्यपि च तीर्थानि कीर्तितान्यगमानि च}
{मनसा तानि गच्छेत सर्वतीर्थसमीक्षया}


\twolineshloka
{एतानि वसुभिः साध्यैरादित्यैर्मरुदश्विभिः}
{ऋषिभिर्देवकल्पैश्च स्नातानि सुकृतैषिभिः}


\twolineshloka
{एवं त्वमपि कौरव्य विधिनाऽनेन सुव्रत}
{व्रत तीर्थानि नियतः पुण्यं पुण्येन वर्धयन्}


\threelineshloka
{भाषितैः करणैः पूर्वमास्तिक्याच्छ्रुतिदर्शनात्}
{प्राप्यन्ते तानि तीर्तानि सद्भिः शास्त्रार्थदर्शिभिः}
{सद्भिः शास्त्रार्थतत्वज्ञैर्ब्राह्मणैः सह गम्यताम्}


\twolineshloka
{नाव्रती नाकृतात्मा च नाशुचिर्न च तस्करः}
{स्नाति तीर्थेषु कौरव्य न च वक्रमतिर्नरः}


\twolineshloka
{त्वया तु सम्यग्वृत्तेन नित्यं धर्मार्थदर्शिना}
{`पितरस्तात सर्वे च तारिताः प्रपितामहाः'}


\threelineshloka
{पिता पितामहस्चैव सर्वे च प्रपितामहाः}
{पितामहपुरोगाश्च देवाः सर्षिगणा नृप}
{तव धर्मेण धर्मज्ञ नित्यमेवाभितोषितः}


\threelineshloka
{अवाप्स्यसि त्वं लोकान्वै वसूनां वासवोपम}
{कीर्तिं च महातीं भीष्म प्राप्स्यसे भुविशाश्वतीं ॥नारद उवाच}
{}


\twolineshloka
{एवमुक्त्वाऽभ्यनुज्ञाय पुलस्त्यो भगवानृषिः}
{प्रीतः प्रीतेन मनसा तत्रैवान्तरधीयत}


\twolineshloka
{भीष्मश्च कुरुशार्दूल शास्त्रतत्त्वार्थदर्शिवान्}
{पुलस्त्यवचनाच्चैव पृथिवीं परिचक्रमे}


\twolineshloka
{एवमेषा महाभागा प्रतिष्ठाने प्रतिष्ठिता}
{तीर्थयात्रा महापुण्या सर्वपापप्रमोचनी}


\twolineshloka
{अनेन विधिना यस्तु पृथिवीं संचरिष्यति}
{अश्वमेधशतं साग्रं फलं प्रेत्य स भोक्ष्यति}


\twolineshloka
{ततश्चाष्टगुणं पार्थ प्राप्स्यसे धर्ममुत्तमम्}
{भीष्मः कुरूणां प्रवरो यथा पूर्वमवाप्तवान्}


\threelineshloka
{नेता च त्वमृषीन्यस्मात्तेन तेऽष्टगुणं फलम्}
{रक्षोगणबिकीर्णानि तीर्थान्येतानि भारत}
{अगम्यानि मनुष्येन्द्रैस्त्वामृते कुरुनन्दन}


\twolineshloka
{इदं देवर्षिचरितं सर्वतीर्ताभिसंवृतम्}
{यः पठेच्छृणुयाद्वाऽपिसर्वपापैः प्रमुच्यते}


\twolineshloka
{ऋषिमुख्याः सदा यत्रवाल्मीकिस्त्वथ कश्यपः}
{आत्रेयः कुण्डजठरो विश्वामित्रोऽथ गौतमः}


\twolineshloka
{असितो देवलश्चैव मार्कण्डेयोऽथ गालवः}
{भरद्वाजो वसिष्ठश्च मुनिरुद्दालकस्तथा}


\twolineshloka
{शौनकः सह पुत्रेण व्यासश्च तपतांवरः}
{दुर्वासाश्च मुनिश्रेष्ठो जाबालिश्च महातपाः}


\twolineshloka
{एते ऋषिवराः सर्वे त्वत्प्रतीक्षास्तपोधनाः}
{एभिः सह महाराज तीर्थान्येतान्यनुव्रज}


\twolineshloka
{एष ते लोमशो नाम महर्षिरमितद्युतिः}
{समेष्यति महाराज तेन सार्धमनुव्रज}


\twolineshloka
{मयाऽपिसह धर्मज्ञ तीर्थान्येतान्यनुक्रमात्}
{3-83-b124प्राप्स्यसे महतीं कीर्तिं यथा राजा महाभिषक्}


\twolineshloka
{यथा ययातिर्धर्मात्मा यथा राजा पुरूरवाः}
{तथा त्वं कुरुशार्दूल स्वेन धर्मेण शोभसे}


\twolineshloka
{यथा भगीरथो राजा यथा रामश्च विश्रुतः}
{तथा त्वं सर्वराजभ्यो भ्राजसे रश्मिवानिव}


\twolineshloka
{यथा मनुर्यथेक्ष्वाकुर्यथा पूरुर्महायशाः}
{यथा वैन्यो महाराज तथा त्वमपि विश्रुतः}


\twolineshloka
{यथा च वृत्रहा सर्वान्सपत्नान्निर्दहन्पुरा}
{त्रैलोक्यं पालयामास देवराड्विगतज्वरः}


\fourlineindentedshloka
{तथा शत्रुक्षयं कृत्वा त्वं प्रजाः पालयिष्यसि}
{स्वधर्मविजितामुर्वीं प्राप्य राजीवलोचन}
{ख्यातिं यास्यसि धर्मेण कार्तवीर्यार्जुनो यथा ॥वैशंपायन उवाच}
{}


\twolineshloka
{एवमाश्वास्य राजानं नारदो भगवानृषिः}
{अनुज्ञाप्य महाराज तत्रैवान्तरधीयत}


\twolineshloka
{युधिष्ठिरोषि धर्मात्मा तमेवार्थं विचिन्तयन्}
{तीर्थयात्राश्रितं पुण्यमृषीणां प्रत्यवेदयत्}


\chapter{अध्यायः ८४}
\twolineshloka
{वैशंपायन उवाच}
{}


\twolineshloka
{भ्रातॄणां मतमाज्ञाय नारदस्य च धीमतः}
{पितामहसमं धौम्यं प्राह राजा युधिष्ठिरः}


\twolineshloka
{मया स पुरुषव्याघ्रो जिष्णुः सत्यपराक्रमः}
{अस्त्रहेतोर्महाबाहुरमितात्मा विवासितः}


\twolineshloka
{स हि वीरोऽनुरक्तश् समर्थश्च तपोधनः}
{कृती च भृशमप्यस्त्रे वासुदेव इव प्रभुः}


\twolineshloka
{अहं ह्येतावुभौ ब्रह्मन्कृष्णावरिविघातिनौ}
{अभिजानामि विक्रान्तौ तथा व्यासः परन्तपौ}


\twolineshloka
{त्रियुगौ पुण्डरीकाक्षौ वासुदेवधनंजयौ}
{नारदोऽपितथा वेद योष्यशंसत्सदा मम}


\twolineshloka
{तथाऽहमपि जानामि नरनारायणावृषी}
{शक्तोऽयमित्यतो मत्वा मया स प्रेषितोऽर्जुनः}


\twolineshloka
{इन्द्रादनवरः शक्रं सुरसूनुः सुराधिपम्}
{द्रष्टुमस्त्राणि चादातुमिन्द्रादिति विवासितः}


\twolineshloka
{भीष्मद्रोणावतिरथौ कृपो द्रौणिश्च दुर्जयः}
{धृतराष्ट्रस्य पुत्रेण सुधृताः सुमहाबलाः}


\threelineshloka
{सर्वे वेदविदः शूराः सर्वेऽस्त्रकुशलास्तथा}
{`सर्वे महारथा मुख्याः सर्वे जितपरिश्रमाः'}
{योद्धुकामाश्च पार्थेन सततं ये महाबलाः}


\threelineshloka
{स च दिव्यास्त्रवित्कर्णः सूतपुत्रो महारथः}
{योऽस्त्रवेगानिलबलः शरार्चिस्तलनिःस्वनः}
{रजोधूमोऽस्त्रसंपातो धार्तराष्ट्रानिलोद्धतः}


\twolineshloka
{निसृष्ट इव कालेन युगान्तज्वलनो यथा}
{मम सैन्यमयं कक्षं प्रधक्ष्यति न संशयः}


\twolineshloka
{तं स कुष्णानिलोद्धूतो दिव्यास्त्रज्वलनो महान्}
{श्वेतवाजिबलाकाभृद्गण्डीवेन्द्रायुधोल्बणः}


\twolineshloka
{संरब्धः शरधाराभिः सुधीप्तं कर्णपावकम्}
{उदीर्णोऽर्जुनमेघोऽयं शमयिष्यति संयुगे}


\twolineshloka
{स साक्षादेव सर्वाणि शक्रात्परपुरंजयः}
{दिव्यान्यस्त्राणि वीभत्सुर्यथावत्प्रतिपत्स्यते}


\twolineshloka
{अलं स तेषां सर्वेषामिति मे धीयते मतिः}
{नास्ति त्वतिक्रिया तस् रणेऽरीणां प्रतिक्रिया}


\twolineshloka
{ते वयं पाण्डवं सर्वे गृहीतास्त्रमरिंदमम्}
{द्रष्टारो न हि बीभत्सुर्भारमुद्यम्य सीदति}


\twolineshloka
{वयं तु तमृते वीरं वनेऽस्मिन्द्विपदांवर}
{अवधानं न गच्छामः काम्यके सह कृष्णया}


\twolineshloka
{भवानन्यद्वनं साधु बह्वन्नं फलवच्छुचि}
{आख्यातु रमणीयं च सेवितं पुण्यक्रमभिः}


\twolineshloka
{यत्रकंचिद्वयं कालं वसन्तः सत्यविक्रमम्}
{प्रतीक्षामोऽर्जुनं वरं वृष्टिकामा इवाम्बुदम्}


\twolineshloka
{विविधानाश्रमान्कांश्चिद्द्विजातिभ्यः प्रतिश्रुतान्}
{सरांसि सरितश्चैव रमणीयांश्च पर्वतान्}


\twolineshloka
{आचक्ष्व न हि मे ब्रह्मन्रोचते तमृतेऽर्जुनम्}
{वनेऽस्मिन्काम्यके वासो गच्छामोऽन्यां दिशंप्रति}


\chapter{अध्यायः ८५}
\twolineshloka
{वैशंपायन उवाच}
{}


\twolineshloka
{तान्सर्वानुत्सुकान्दृष्ट्वा पाण्डवान्दीनचेतसः}
{अश्वासयंस्तथा धौम्यो बृहस्पतिसमोऽब्रवीत्}


\twolineshloka
{ब्राह्मणानुमतान्पुण्यानाश्रमान्भरतर्षभ}
{दिशस्तीर्थानि शैलांश्च शृणु मे वदतोऽनघ}


\twolineshloka
{याञ्श्रुत्वा गदतो राजन्विशोको भवितासि ह}
{द्रौपद्या चानया सार्धं भ्रातृभिश्च नरेश्वर}


\twolineshloka
{श्रवणाच्चैव तेषां त्वं पुण्यमाप्स्यसि पाण्डव}
{गत्वा शतगुणं चैव तेभ्य एव नरोत्तम}


\twolineshloka
{शृणु पूर्वां दिशं राजन्देवर्षिगणसेविताम्}
{रम्यां ते कथयिष्यामि युधिष्ठिर यथास्मृति}


\twolineshloka
{तस्यां रदेवर्षिजुष्टायां नैमिषं नाम भारत}
{यत्रतीर्थानि देवानां पुण्यानि च पृथक् पृथक्}


\twolineshloka
{यत्र सा गोमती पुण्या रम्या देवर्षिसेविता}
{यज्ञभूमिश्च देवानां शामित्रं च विवस्वतः}


\twolineshloka
{तस्यां गिरिवरः पुण्यो गयो राजर्षिसत्कृतः}
{शिवं ब्रह्मसरो यत्रसेवितं त्रिदशर्षिभिः}


\twolineshloka
{यदर्थे पुरुषव्याघ्र कीर्तयन्ति पुरातनाः}
{एष्टव्या बहवः पुत्रा यद्येकोपि गयां व्रजेत्}


\twolineshloka
{गौरीं वा वरयेत्कन्यां नीलं वा वृषमुत्सृजेत्}
{उत्तारयति संतत्या दश पूर्वान्दशावरान्}


\twolineshloka
{महानदी च तत्रैव तथा गयशिरो नृप}
{यत्रासौ कीर्त्यते विप्रैरक्षय्यकरणो वटः}


\twolineshloka
{यत्रदत्तं पितृभ्योऽन्नमक्षय्यं भवति प्रभो}
{सा च पुण्यजला तत्र फल्गुनामा महानदी}


\twolineshloka
{बहुमूलफला चापि कौशिकी भरतर्षभ}
{विश्वामित्रोऽध्यगाद्यत्र ब्राह्मणत्वं तपोधनः}


\twolineshloka
{गङ्गा यत्रनदी पुण्या यस्यास्तीरे भगीरथः}
{अयजत्तत्रबहुभिः क्रतुभिर्भूरिदक्षिणैः}


\twolineshloka
{पाञ्चालेषु च कौरव्य कथयन्त्युत्पलावतम्}
{विश्वामित्रोऽयजद्यत्र शक्रेण सह कौशिकः}


\twolineshloka
{यत्रानुवंशं भगवाञ्जामदग्न्यस्तथा जगौ}
{विश्वामित्रस्य तां दृष्ट्वा चिभूतिमतिमानुषीम्}


\twolineshloka
{कान्यकुब्जेऽपिबत्सोममिन्द्रेण सह कौशिकः}
{ततः क्षत्रादपाक्रामद्ब्राह्मणोस्मीति चाब्रवीत्}


\twolineshloka
{पवित्रमृषिभिर्जुष्टं पुण्यं पावनमुत्तमम्}
{गङ्गायमुनयोर्वीर संगमं लोकविश्रुतम्}


\twolineshloka
{यत्रायजत भूतात्मा पूर्वमेव पितामहः}
{प्रयागमिति विख्यातं तस्माद्भरतसत्तम}


\twolineshloka
{अगस्त्यस्य तु राजेन्द्र तत्राश्रमवरो नृप}
{तत्तथा तापसारण्यं तापसैरुपशोभितम्}


\twolineshloka
{हिरण्यबिन्दुः कथितो गिरौ कालञ्जरे महान्}
{आगस्त्यपर्वतोरभ्यः पुण्यो गिरिवः शिवः}


\twolineshloka
{महेन्द्रो नाम कौरव्य भार्गवस्य महात्मनः}
{अयजत्तत्रकौन्तेय पूर्वमेव पितामहः}


\threelineshloka
{यत्रभागीरथी पुण्यां सरस्यासीद्युधिष्ठिर}
{यत्र सा ब्रह्मशालेति पुण्याख्याता विशांपते}
{धूतपाप्मभिराकीर्णा पुण्यं तस्याश्च दर्शनम्}


\twolineshloka
{पवित्रो मङ्गलीयश्च ख्यातो लोके सनातनः}
{केदारश्च मतङ्गस्य महानाश्रम उत्तमः}


\twolineshloka
{कुण्डोदः पर्वतो रम्यो बहुमूलफलोदकः}
{नैषधस्तृषितो यत्र जलं शर्म च लब्धवान्}


\twolineshloka
{यत्र देववनं पुण्यं तापसैरुपशोभितम्}
{बाहुदा च नदी यत्रनन्दा च गिरिमूर्धनि}


\twolineshloka
{तीर्थानि सरितः शैलाः पुण्यान्यायतनानि च}
{प्राच्यां दिशि महाराज कीर्तितानि मया तव}


\twolineshloka
{तिसृष्वन्यासु पुण्यानि दिक्षु तीर्थानि मे शृणु}
{सरितः पर्वतांश्चैव पुण्यान्यायतनानि च}


\chapter{अध्यायः ८६}
\twolineshloka
{धौम्य उवाच}
{}


\twolineshloka
{दक्षिणस्यां तु पुण्यानि शृणु तीर्थानि भारत}
{विस्तरेण यथाबुद्धि कीर्त्यमानानि तानि वै}


\twolineshloka
{यस्यामाख्यायते पुण्या दिशि गोदावरी नदी}
{बह्वारामा बहुजला तापसाचरिता शिवा}


\twolineshloka
{वेणा भीमरथी चैव नद्यौ पापभयापहे}
{मृगद्विजसमाकीर्णे तापसालयभूषिते}


\twolineshloka
{राजर्षेस्तस् च सरिन्नृगस्य भरतर्षभ}
{रम्यतीर्था बहुजला पयोष्णी द्विजसेविता}


\twolineshloka
{अपि चात्र महायोगी मार्कण्डेयो महायशाः}
{अनुवंश्यां जगौ गाथां नृगस्य धरणीपतेः}


\twolineshloka
{नृगस् यजमानस्य प्रत्यक्षमिति नः श्रुतम्}
{`मरुतः परिवेष्टारः सदस्याश्च दिवौकसः'}


\threelineshloka
{पयोष्ण्यां यजमानस्य वाराहे तीर्थ उत्तमे}
{उद्धृतं भूतलस्थं वावायुना समुदीरितम्}
{पयोष्ण्या हरते तोयं पापमामरणान्तिकम्}


\twolineshloka
{स्वर्गादुत्तुङ्गममलं विषाणं यत्र शूलिनः}
{स्वमात्मविहितं दृष्ट्वा मर्त्यः शिवपुरं व्रजेत्}


\twolineshloka
{एकतः सरितः सर्वा गङ्गाद्याः सलिलोच्चयाः}
{पयोष्णी चैकतः पुण्या तीर्थेभ्यो हि मता मम}


\twolineshloka
{माठरस्य वनं पुण्यं बहुमूलफलं शिवम्}
{यूपश्च भरतश्रेष्ठ वरुणस्रोतसे गिरौ}


\twolineshloka
{प्रवेण्युत्तरपार्स्वे तु पुण्ये कण्वाश्रमे तथा}
{तापसानामरण्यानि कीर्तितानि यथाश्रुति}


\twolineshloka
{वेदी शूर्पारके तात जमदग्नेर्महात्मनः}
{रम्या पाषाणतीर्था च पुरश्चन्द्रा च भारत}


\twolineshloka
{अशोकतीर्थं तत्रैव कौन्ते य बहुलाश्रमम्}
{अगस्त्यतीर्थं पाण्ड्येषु वारुणं च युधिष्ठिर}


\twolineshloka
{कुमार्यः कथिताः पुण्याः पाण्ड्येष्वेव नरर्षभ}
{ताम्रपर्णी तु कौन्तेय कीर्तयिष्यामि तां शृणु}


\twolineshloka
{यत्र देवैस्तपस्तप्तं महदिच्छद्भिराश्रमे}
{गोकर्णमिति विख्यातं त्रिषु लोकेषु भारत}


\twolineshloka
{शीततोयो बहुजलः पुण्यस्तात शिवश्च सः}
{ह्रदः परमदुष्प्रापो मानुषैरकृतात्मभिः}


\twolineshloka
{तत्र वृक्षतृणाद्यैश्च संपन्नः फलमूलवान्}
{आश्रमोऽगस्त्यशिष्यस्य पुण्यो देवसहे गिरौ}


\twolineshloka
{वैडूर्यपर्वतस्तत्र श्रीमान्मणिमयः शिवः}
{अगस्त्यस्याश्रमश्चैव बहुमूलफलोदकः}


\twolineshloka
{सुराष्ट्रेष्वपि वक्ष्यामि पुण्यान्यायतनानि च}
{आश्रमान्सरितश्चैव सरांसि च नराधिप}


\twolineshloka
{चमसोद्भेदनं विप्रास्तत्रापि कथयन्त्युत}
{प्रभासं चोदधौ तीर्थं त्रिषु लोकेषु विश्रुतम्}


\twolineshloka
{तत्र पिण्डारकं नाम तापसाचरितं शिवम्}
{उज्जयन्तश्च शिखरी क्षिप्रं सिद्धिकरो महान्}


\twolineshloka
{तत्र देवर्षिवीरेण नारदेनानुकीर्तितः}
{पुराणः श्रूयते श्लोकस्तं निबोध युधिष्ठिर}


\twolineshloka
{पुण्ये गिरौ सुराष्ट्रेषु मृगपक्षिनिषेविते}
{उज्जयन्ते स्म तप्ताङ्गो नाकपृष्ठे महीयते}


\twolineshloka
{`एष नारायणः श्रीमान्क्षीरार्णवनिकेतनः}
{नागपर्यङ्कमुत्सृज्यह्यागतो मधुरां पुरीम्'}


\twolineshloka
{पुण्या द्वारवती तत्रयत्रासौ मधुसूदनः}
{साक्षाद्देवः पुराणोसौ स हि धर्मः सनातनः}


\twolineshloka
{ये च कवेदविदो विप्रा ये चाध्यात्मविदो जनाः}
{ते वदन्ति महात्मानं कृष्णं धर्मं सनातनम्}


\threelineshloka
{पवित्राणां हि गोविन्दः पवित्रं परमुच्यते}
{पुण्यानामपि पुण्योसौ मङ्गालानां च मङ्गलम्}
{त्रैलोक्ये पुण्डरीकाक्षो देवदेवः सनातनः}


\twolineshloka
{अव्ययात्मा व्ययात्मा च क्षेत्रज्ञः परमेश्वरः}
{आस्ते हरिरचिन्त्यात्मा तत्रैव मधुसूदनः}


\chapter{अध्यायः ८७}
\twolineshloka
{धौम्य उवाच}
{}


\twolineshloka
{अवन्तीषु प्रतीच्यां वै कीर्तयिष्यामि ते दिशि}
{यानि तत्रपवित्राणि पुण्यान्यायतनानि च}


\twolineshloka
{प्रियङ्ग्वाम्रवणोपेता वानीरफलमालिनी}
{प्रत्यक्स्रोता नदी पुण्या नर्मदा तत्र भारत}


\twolineshloka
{त्रैलोक्ये यानि तीर्थानि पुण्यान्यायतनानि च}
{सरिद्वनानि शैलेन्द्रा देवाश्च सपितामहाः}


\twolineshloka
{नर्मदायां कुरुश्रेष्ठ सहसिद्धर्षिचारणैः}
{स्नातुमायान्ति पुण्यौधैः सदा वारिषु भारत}


\twolineshloka
{निरेतः श्रूयते पुण्यो यत्र विश्रवसो मुनेः}
{जज्ञे धनपतिर्यत्र कुबेरो नरवाहनः}


\twolineshloka
{वैढूर्यशिखरो नाम पुण्यो गिरिवरः शिवः}
{नित्यपुष्पफलास्तत्र पादपा हरितच्छदाः}


\twolineshloka
{तस्य शैलस्य शिखरे सरः पुण्यं महीपते}
{फुल्लपद्मं महाराज देवगन्धर्वसेवितम्}


\twolineshloka
{बह्वाश्चर्यं महाराज दृश्यते तत्र पर्वते}
{पुण्ये स्वर्गोपमे चैव देवर्षिगणसेविते}


\twolineshloka
{ह्रदिनी पुण्यतीर्था च राजर्षेस्तत्र वै सरित्}
{विश्चामित्रेण तपसा निर्मिता सर्वपावनी}


\twolineshloka
{यस्यास्तीरे सतां मध्ये ययातिर्नहुषात्मजः}
{पपात स पुनर्लोकाँल्लेभे धर्मान्सनातनान्}


\twolineshloka
{तत्रपुण्यो ह्रदः ख्यातो मैनाकश्चैव पर्वतः}
{बहुमूलफलोपेतस्त्वमितो नाम पर्वतः}


\threelineshloka
{आश्रमः कक्षसेनस्य पुण्यस्तत्रयुधिष्ठिरः}
{च्यवनस्याश्रमश्चैव विख्यातस्तत्रपाण्डव}
{तत्राल्पेनैव सिद्ध्यन्ति मानवास्तपसा विभो}


\twolineshloka
{जम्बूमार्गो महाराज ऋषीणां भावितात्मनाम्}
{आश्रम शाम्यतां श्रेष्ठ मृगद्विजनिषेवितः}


\threelineshloka
{ततः पुण्यतमा राजन्सततं तापसैर्युता}
{केतुमाला च मेध्या च गङ्गाद्वारं च भूमिप}
{ख्यातं च सैन्धवारण्यं पुण्यं द्विजनिषेवितम्}


\twolineshloka
{पितामहसरः पुण्यं पुष्करं नाम नामतः}
{वैखानसानां सिद्धानामृषीणामाश्रमः प्रियः}


\twolineshloka
{अप्यत्र संश्रयार्थाय प्रजापतिरथो जगौ}
{पुष्करेषु कुरुश्रेष्ठ गाथां सुकृतिनांवर}


\twolineshloka
{मनसाऽप्यभिकामस्य पुष्कराणि मनखिनः}
{विप्रणश्यन्ति पापानि नाकपृष्ठे च मोदते}


\chapter{अध्यायः ८८}
\twolineshloka
{धौम्य उवाच}
{}


\twolineshloka
{उदीच्यां राजशार्दूल दिशि पुण्यानि यानि वै}
{तानि ते कीर्तयिष्यामि पुण्यान्यायतनानि च}


\twolineshloka
{शृणुष्वावहितो भूत्वा मम मन्त्रयतः प्रभो}
{कथाप्रतिग्रहो वीर श्रद्धां जनयते शुभाम्}


\twolineshloka
{सरस्वती महापुण्या ह्रदिनी तीर्थमालिनी}
{समुद्रगा महावेगा यमुना यत्र पाण्डव}


\twolineshloka
{यत्र पुण्यतरं तीर्थं प्लक्षावतरणं शुभम्}
{यत्रसारस्वतैरिष्ट्वा गच्छन्त्यवभृथं द्विजाः}


\twolineshloka
{पुण्यं चाख्यायते दिव्यं शिवमग्निशिरोऽनघ}
{सहदेवोऽयजद्यत्रशम्याक्षेपेण भारत}


\twolineshloka
{एतस्मिन्नेव चार्थेऽसौ इन्द्रगीता युधिष्ठिर}
{गाथा चरति लोकेऽस्मिन्गीयमाना द्विजातिभिः}


\twolineshloka
{अग्नयः सहदेवेन ये चिता यमुनामनु}
{ते तस्य कुरुशार्दूल सहस्रशतदक्षिणाः}


\twolineshloka
{तत्रैव भरतो राजा चक्रवर्ती महायशाः}
{विंशतीं सप्त चाष्टौ च हयमेधानुपाहरत्}


\twolineshloka
{कामकृद्यो द्विजातीनां श्रुतस्तात यथा पुरा}
{अत्यन्तमाश्रमः पुण्यः शरभङ्गस्य विश्रुतः}


\twolineshloka
{सरस्वती नदी सद्भिः सततं पार्थ पूजिता}
{वालखिल्यैर्महाराज यत्रेष्टमृषिभिः पुरा}


\threelineshloka
{दृषद्वती महापुण्या यत्र रख्याता युधिष्ठिर}
{न्यग्रोधाख्यस्तु पाञ्चाल्यः पाञ्चाल्योद्विपदांवर}
{दाल्भ्यघोषश्च दाल्भ्याश् धरणीस्थो महात्मनः}


\twolineshloka
{कौन्तेयानन्तयशसः सुव्रतस्यामितौजसः}
{आश्रमः ख्यायते पुण्यस्त्रिषु लोकेषु विश्रुतः}


\twolineshloka
{एतावर्णाववर्णौ च विश्रुतौ मनुजाधिप}
{ईजाते क्रतुभिर्मुख्यैः पुण्यैर्भरतसत्तम}


\twolineshloka
{समेत्य बहुशो देवाः सेन्द्राः सवरुणाः पुरा}
{विशाखयूपेऽतप्यन् तेन पुण्यतमश्च सः}


\twolineshloka
{ऋषिर्महान्महाभागो जमदग्निर्महायशाः}
{पलाशकेषु पुण्येषु रम्येष्वयजत प्रभुः}


\twolineshloka
{यत्रसर्वाः सरिच्छ्रेष्टाः साक्षात्तमृषिसत्तमम्}
{स्वं स्वं तोयमुपादाय परिवार्योपतस्थिरे}


\twolineshloka
{अपि चात्र महाराज स्वयं विश्वावसुर्जगौ}
{इमं श्लोकं तदा वीर प्रेक्ष्य दीक्षां महात्मनः}


\twolineshloka
{यजमानस्य वै देवाञ्जमदग्नेर्महात्मनः}
{आगम्य सरितो विप्रान्मधुना समतर्पयन्}


\twolineshloka
{गन्धर्वयक्षरक्षोबिरप्सरोभिश्च सेवितम्}
{किरातकिन्नरावासं शैलं शिखरिणांवरम्}


\twolineshloka
{बिभेद तरसा गङ्गा गङ्गाद्वारं युधिष्ठिर}
{पुण्यं तत्ख्यायते राजन्ब्रह्मर्षिगणसेवितम्}


\twolineshloka
{सनत्कुमारः कौरव्य पुण्यं कनखलं तथा}
{पर्वतश्च पुरुर्नाम यत्रयातः पुरूरवाः}


\twolineshloka
{भृगुर्यत्रतपस्तेपे महर्षिगणसेविते}
{राजन्स आश्रमः ख्यातो भृगुतुन्दो महागिरिः}


\twolineshloka
{यः स भूतं भविष्यच्च भवच् भरतर्षभ}
{नारायणः प्रभुर्विष्णुः शाश्वतः पुरुषोत्तमः}


\twolineshloka
{तस्यातियशसः पुण्यां विशालां बदरीमनु}
{आश्रमः ख्यायते पुण्यस्त्रिषु लोकेषु विश्रुतः}


\twolineshloka
{उष्णतोयवहा गङ्गा शीततोयवहा पुरा}
{सुवर्णसिकता राजन्विशालां बदरीमनु}


\twolineshloka
{ऋषयो यत्रदेवाश्च महाभागा महौजसः}
{प्राप्य नित्यं नमस्यन्ति देवं नारायणं प्रभुम्}


\twolineshloka
{यत्रनाराजणो देवः परमात्मा सनातनः}
{तत्र कृत्स्नं जगत्सर्वं तीर्थान्यायतनानि च}


\twolineshloka
{तत्पुण्यं परमं ब्रह्म तत्तीर्थं त्तपोवनम्}
{तत्परं परमं देवं भूतानां परमेश्वरम्}


\twolineshloka
{शाश्वतं परमं चैव धातारं परमं पदम्}
{यं विदित्वान शोचन्ति विद्वांसः शास्त्रदृष्टयः}


\twolineshloka
{तत्र देवर्षयः सिद्धाः सर्वे चैव तपोधनाः}
{आदिदेवो महायोगी यत्रास्ते मधुसूदनः}


\threelineshloka
{पुण्यानामपि तत्पुण्यमत्र ते संशयेस्तु मा}
{एतानि राजन्पुण्यानि पृथिव्यां पृथिवीपते}
{कीर्तितानि नरश्रेष्ठ तीर्थान्यायतनानि च}


\twolineshloka
{एतानि वसुभिः साध्यैरादित्यैर्मरुदश्विभिः}
{ऋषिभिर्देवकल्पैश्च सेवितानि महात्मभिः}


\twolineshloka
{चरन्नेतानि कौन्तेय सहितो ब्राह्मणर्षभैः}
{भ्रातृभिश्च महाभागैरुत्कण्ठां विजयिष्यसि}


\chapter{अध्यायः ८९}
\twolineshloka
{वैशंपायन उवाच}
{}


\twolineshloka
{एवं संभाषमाणे तु धौम्ये कौरवनन्दन}
{लोमशः स महातेजा ऋषिस्तत्राजगाम ह}


\twolineshloka
{तं पाण्डवाग्रजो राजा सगणो ब्राह्मणाश्च ते}
{उपातिष्ठन्महाभागं दिवि शक्रमिवामराः}


\twolineshloka
{समभ्यर्च्य यथान्यायं धर्मपुत्रो युधिष्ठिरः}
{पप्रच्छागमने हेतुमटने च प्रयोजनम्}


\twolineshloka
{स पृष्टः पाण्डुपुत्रेण प्रीयमाणो महामनाः}
{उवाच श्लक्ष्णया वाचा हर्षयन्निव पाण्डवान्}


\twolineshloka
{संचरन्नस्मि कौन्तेय सर्वांल्लोकान्यदृच्छया}
{गतः शक्रस्य भवनं तत्रापश्यं सुरेश्वरम्}


\twolineshloka
{तव च भ्रातरं वीरमपश्यं सव्यसाचिनम्}
{शक्रस्यार्धासनगतं तत्र मे विस्मयो महान्}


\twolineshloka
{आसीत्पुरुषशार्दूल दृष्ट्वा पार्थं तथागतम्}
{आह मां तत्र देवेशो गच्छ पाण्डुसुतान्प्रति}


\twolineshloka
{सोऽहमभ्यागतः क्षिप्रं दिदृक्षस्त्वां सहानुजम्}
{वचनात्पुरुहूतस्य पार्थस् च महात्मनः}


\twolineshloka
{आख्यास्ये ते प्रियं तात महत्पाण्डवनन्दन}
{भ्रातृभिः सहितो राजन्कृष्णया चैव तच्छृणु}


\twolineshloka
{यत्त्वयोक्तो महाबाहुरस्त्रार्थं भरतर्षभ}
{तदस्त्रमाप्तं पार्थेन रुद्रादप्रतिमं विभो}


\twolineshloka
{यत्तद्ब्रह्मशिरो नाम तपसा रुद्रमागमत्}
{अमृतादुत्थितं रौद्रं तल्लब्धं सव्यसाचिना}


\twolineshloka
{तत्समन्त्रं ससंहारं सप्रायश्चित्तमङ्गलम्}
{वज्रमस्त्राणि चान्यानि दण्डादीनि युधिष्ठिर}


\twolineshloka
{यमात्कुबेराद्वरुणादिन्द्राच्च कुरुनन्दन}
{अस्त्राण्यधीतवान्पार्थो दिव्यान्यमितविक्रमः}


\twolineshloka
{विश्वावसोस्तु तनयाद्गीतं नृत्यं च साम च}
{वादित्रं च यथान्यायं प्रत्यविन्दद्यथाविधि}


\twolineshloka
{एवं कृतास्त्रः कौन्तेयो गान्धर्वं वेदमात्मवान्}
{सुखं वसति बीभत्सुरनुजस्यानुजस्तव}


\twolineshloka
{यदर्थं मां सुरश्रेष्ठ इदं वचनमब्रवीत्}
{तच्च ते कथयिष्यामि युधिष्ठिर निबोध मे}


\twolineshloka
{भवान्मनुष्यलोकेऽपि गमिष्यति न संशयः}
{ब्रूयाद्युधिष्ठिरं तत्रवचनान्मे द्विजोत्तम}


\twolineshloka
{आगमिष्यति ते भ्राता कृतास्त्रः क्षिप्रमर्जुनः}
{सुरकार्यं महत्कृत्वा यदशक्यं दिवौकसैः}


\twolineshloka
{तपसाऽपि त्वमात्मानं योजय भ्रातृभिः सह}
{तपसो हि परं नास्ति तपसा विन्दते महत्}


\twolineshloka
{अहं च कर्णं जानामि यथावद्भरतर्षभ}
{सत्यसन्धं महोत्साहं महावीर्यं महाबलम्}


\twolineshloka
{महाहवेष्वप्रतिमं महायुद्धविशारदम्}
{महाधनुर्धरं वीरं महास्त्रं वरवर्णिनम्}


\twolineshloka
{महेश्वरसुतप्रख्यमादित्यतनयं प्रभुम्}
{तथाऽर्जुनमतिस्कन्धं सहजोल्वणपौरुणम्}


% Check verse!
न स पार्तस्य संग्रामे कलामर्हति षोडशीम्
\twolineshloka
{यच्चापि ते भयं कर्णान्मनसिस्थमरिंदम}
{तच्चाप्यपहरिष्यामि सव्यसाचिन्युपागते}


\twolineshloka
{यच्च ते मानसं वीर तीर्थयात्रामिमां प्रति}
{स महर्षिर्लोमशस्ते कथयिष्यत्यसंशयम्}


\twolineshloka
{यच्च किंचित्तपोयुक्तं फलं तीर्थेषु भारत}
{ब्रह्मर्षिरेष ब्रूयात्ते न तच्छ्रद्धेयमन्यथा}


\chapter{अध्यायः ९०}
\twolineshloka
{लोमश उवाच}
{}


\twolineshloka
{धनंजयेन चाप्युक्तं यत्तच्छृणु युधिष्ठिर}
{युधिष्ठिरं भ्रातरं मे योजयेर्धर्म्यया धिया}


\twolineshloka
{त्वं हि धर्मान्परान्वेत्थ तपांसि च तपोधन}
{श्रीमतां चापि जानासि धर्मं राज्ञां सनातनम्}


\twolineshloka
{स भवान्परमं वेद रपावनं पुरुषं प्रति}
{तेन संयोजयेथास्त्वं तीर्थपुण्येन पाण्डवान्}


\twolineshloka
{यथा तीर्थानि गच्छेत गाश्च दद्यात्स पार्थिवः}
{तथा सर्वात्मना कार्यमिति मामर्जुनोऽब्रवीत्}


\twolineshloka
{भवता चानुगुप्तोऽसौ चरेत्तीर्थानि सर्वशः}
{रक्षोभ्यो रक्षितव्यश्च दुर्गेषु विषमेषु च}


\twolineshloka
{दधीच इव देवेन्द्रं यथा चाप्यङ्गिरा रविम्}
{तथा रक्षस्व कौन्तेयान्राक्षसेभ्यो द्विजोत्तम}


\twolineshloka
{यातुधाना हि बहवो राक्षसाः पर्वतोपमाः}
{त्वयाऽभिगुप्तान्कौन्तेयान्न विवर्तेयुरन्तिकम्}


\twolineshloka
{सोऽहमिन्द्रस्य वचनान्नियोगादर्जुनस्य च}
{रक्षमाणो भयेभ्यस्त्वां चरिष्यामि त्वा सह}


\twolineshloka
{द्विस्तीर्थानि मया पूर्वं दृष्टानि कुरुनन्दन}
{इदं तृतीयं द्रक्ष्यामि तान्येव भवता सह}


\twolineshloka
{इयं राजर्षिभिर्याता पुण्यकृद्भिर्युधिष्ठिर}
{मन्वादिभिर्महाराज तीर्थयात्रा भयापहा}


\twolineshloka
{नानृजुर्नाकृतात्मा च नाविद्यो न च पापकृत्}
{स्नाति तीर्थेषु कौरव्य न च वक्रमतिर्नरः}


\twolineshloka
{त्वं तु धर्ममतिर्नित्यं धर्मज्ञः सत्यसंगरः}
{विमुक्तः सर्वपापेभ्यो भूय एव भविष्यसि}


\threelineshloka
{यथा भगीरथो राजा राजानश्च गयादयः}
{यथा ययातिः कौन्तेय तथा त्वमपि पाण्डव ॥युधिष्ठिर उवाच}
{}


\twolineshloka
{न हर्षात्संप्रपश्यामि वाक्यस्यास्योत्तरं क्वचित्}
{यन्मां स्मरति देवेशः किं नामाभ्यधिकं ततः}


\twolineshloka
{भवता संगमो यस् भ्राता चैव धनंजयः}
{वासवः स्मरते यस् को नामाभ्यधिकस्ततः}


\twolineshloka
{यच्च मां भगवानाह तीर्थानां गमनं प्रति}
{धौम्यस्य वचनादेषा बुद्धिः पूर्वं कृतैव मे}


\threelineshloka
{तद्यदा मन्यसे ब्रह्मन्गमनं तीर्तदर्शने}
{तदैव गन्तास्मि तीर्तान्येष मे निश्चयः परः ॥वैशंपायन उवाच}
{}


\threelineshloka
{गमने कृतबुद्धिं तं पाण्डवं लोमशोऽब्रवीत्}
{लघुर्भव महाराज लघुः स्वैरं गमिष्यसि ॥युधिष्ठिर उवाच}
{}


% Check verse!
भिक्षाभुजो निवर्तन्तां ब्राह्मणा यतयश्च ये
\twolineshloka
{क्षुत्तृडध्वश्रमायासशीतार्तिमसहिष्णवः}
{ते सर्वे विनिवर्तन्तां ये च मृष्टभुजो द्विजाः}


\threelineshloka
{पक्वान्नलेह्यपानानां मांसानां च विकल्पकाः}
{तेऽपि सर्वे निवर्तन्तां ये च सूदानुयायिनः}
{मया यथोचिताऽऽजीव्यौः संविभक्ताश्च वृत्तिभिः}


\threelineshloka
{ये चाप्यनुगताः पौरा राजभक्तिपुरःसराः}
{धृतराष्ट्रं महाराजमभिगच्छन्तु ते च वै}
{स दास्यति यथाकालमुचिता यस्य या भृतिः}


\threelineshloka
{स चेद्यथोचितां वृत्तिं न दद्यान्मनुजेश्वरः}
{अस्मत्प्रियहितार्थायपाञ्चाल्यो वः प्रदास्यति ॥वैशंपायन उवाच}
{}


\twolineshloka
{ततो भूयिष्ठशः पौरा गुरौ भारे समाहिते}
{विप्राश्च यतयो मुख्या जग्मुर्नागपुरं प्रति}


\twolineshloka
{तान्सर्वान्धर्मराजस्य प्रेम्णा राजाऽम्बिकासुतः}
{प्रतिजग्राह विधिवद्धनैश्च समतर्पयत्}


\twolineshloka
{ततः कुन्तीसुतो राजा लघुभिर्ब्राह्मणैः सह}
{लोमशेन च सुप्रीतस्त्रिरात्रं काम्यकेऽवसत्}


\chapter{अध्यायः ९१}
\twolineshloka
{वैशंपायन उवाच}
{}


\twolineshloka
{ततः प्रयान्तं कौन्तेयं ब्राह्मणा वनवासिनः}
{अभिगम्य तदा राजन्निदं वचनमब्रुवन्}


\twolineshloka
{राजंस्तीर्थानि गन्तासि पुण्यानि भ्रातृभिः सह}
{देवर्षिणा च सहितो लोमशेन महात्मना}


\twolineshloka
{अस्मानपि महाराज नेतुमर्हसि पाण्डव}
{अस्माभिर्हि न शक्यानि त्वदृते तानि कौरव}


\twolineshloka
{श्वापरदैरुपसृष्टानि दुर्गाणि विषमाणि च}
{अगम्यानि नरैरल्पैस्तीर्थानि मनुजेश्वर}


\twolineshloka
{भवतो भ्रातरः शूरा धनुर्धरवराः सदा}
{भवद्भिः पालिताः शूरैर्गच्छामो वयमप्युत}


\twolineshloka
{भवत्प्रसादाद्धि वयं प्राप्नुयामः सुखं फलम्}
{तीर्थानां पृथिवीपाल वनानां च विशांपते}


\twolineshloka
{तव वीर्यपरित्राताः शुद्धास्तीर्थपरिप्लताः}
{भवेम धूतपाप्मानस्तीर्थसंदर्शनान्नृप}


\twolineshloka
{भवानपि नरेन्द्रस्य कार्तवीर्यस्य भारत}
{अष्टकस्य च राजर्षेर्लोमपादस् चैव ह}


\twolineshloka
{भरतस्य च वीरस्य सार्वभौमस्य पार्तिव}
{ध्रुवं प्राप्स्यति दुष्प्रापाँल्लोकांस्तीर्थपरिप्लुतः}


\twolineshloka
{प्रभासादीनि तीर्थानि महेन्द्रादींश्च पर्वतान्}
{गङ्गाद्याः सरितश्चैव प्लक्षादींश्च पर्वतान्}


\twolineshloka
{त्वया सह महीपाल द्रष्टुमिच्छामहे वयम्}
{`भवद्भिः पालिताः शूरैस्तीर्थान्यायतनानि च'}


\twolineshloka
{यदि ते ब्राह्मणेष्वस्ति काचित्प्रीतिर्जनाधिप}
{कुरुक्षिप्रं वचोऽस्माकं ततः श्रेयोऽभिपत्स्यसे}


\twolineshloka
{तीर्तानि हि महाबाहो तपोविघ्नकरैः सदा}
{अनुकीर्णानि रक्षोभिस्तेभ्यो नस्त्रातुमर्हसि}


\twolineshloka
{तीर्थान्युक्तानि धौम्येन नारदेन च धीमता}
{यान्युवाच च देवर्षिर्लोमशः सुमहातपाः}


\twolineshloka
{विधिवत्तानि सर्वाणि पर्यटस्व नराधिप}
{धूतपाप्मा सहास्माभिर्लोमशेनाबिपालितः}


\twolineshloka
{स राजा पूज्यमानस्तैर्हर्षादश्रुपरिप्लुतः}
{भीमसेनादिभिर्वीरैर्भ्रातृभिः परिवारितः}


\twolineshloka
{बाढमित्यब्रवीत्सर्वांस्तानृषीन्पाण्डवर्षभः}
{लोमशं समनुज्ञाप्य धौम्यं चैव पुरोहितम्}


\twolineshloka
{ततः स पाण्डवश्रेष्ठो भ्रातृभिः सहितो वशी}
{द्रौपद्या चानवद्याङ्ग्या गमनाय मनो दधे}


\twolineshloka
{अथ व्यासो महाभागस्तथा पर्वतनारदौ}
{दाम्यके पाण्डवंद्रष्टुं समाजग्मुर्मनीषिणः}


\twolineshloka
{तेषां युधिष्ठिरो राजा पूजां चक्रे यथाविधि}
{सत्कृतास्ते महाभागा युधिष्ठिरमथाब्रुवन्}


\twolineshloka
{युधिष्ठिरयमौ भीम मनसा कुरुतार्जवम्}
{मनसा कृतशौचा वै शद्धास्तीर्थानि यास्यथ}


\twolineshloka
{शरीरनियमं प्राहुर्ब्राह्मणा मानुषं व्रतम्}
{मनोविशुद्धां बुद्धिं च दैवमाहुर्व्रतं द्विजाः}


\twolineshloka
{मनो ह्यदुष्टं शौचाय पर्याप्तं वै नराधिप}
{मैत्रीं बुद्धिं समास्थाय शुद्धास्तीर्थानि गच्छत}


\twolineshloka
{ते यूयं मानसैः शुद्धाः शरीरनियमव्रतैः}
{दैवं व्रतं समास्थाय यथोक्तं फलमाप्स्यथ}


\twolineshloka
{ते तथेति प्रतिज्ञाय कृष्णया सह पाण्डवाः}
{कृतस्वस्त्ययनाः सर्वे मुनिभिर्देवमानुषैः}


\twolineshloka
{लोमशस्योपसंगृह्य पादौ द्वैपायनस् च}
{नारदस्य च राजेन्द्र देवर्षेः पर्वतस्य च}


\twolineshloka
{धौम्येन सहिता वीरास्तथा तैर्वनवासिभिः}
{मार्गशीर्ष्यामतीतायां पुष्येण प्रययुस्ततः}


\twolineshloka
{कठिनानि समादाय चीराजिनजटाधराः}
{अभेद्यैः कवचैर्युक्तास्तीर्थान्यन्वचरंस्ततः}


\twolineshloka
{इन्द्रसेनादिभिर्भृत्यै रथैः परिचतुर्दशैः}
{महानसव्यापृतैश्च तथाऽन्यैः परिचारकैः}


\twolineshloka
{सायुधा बद्धनिस्त्रिंशास्तूणवन्तः समार्गणाः}
{प्राङ्युखाः प्रययुर्वीराः पाण्डवा जनमेजय}


\chapter{अध्यायः ९२}
\twolineshloka
{युधिष्ठिर उवाच}
{}


\twolineshloka
{न वै निर्गुणमात्मानं मन्ये देवर्षिसत्तम}
{तथाऽस्मि दुःखसंतप्तो यथा नान्यो महीपतिः}


\threelineshloka
{परांश्च निर्गुणान्मन्ये न च ध्रमगतानपि}
{ते च लोमश लोकेऽस्मिन्नृध्यन्ते के न हेतुना ॥लोमश उवाच}
{}


\twolineshloka
{नात्र दुःखं त्वया राजन्कार्यं पार्थ कथंचन}
{यदधर्मेण वर्धेयुरधर्मरुचयो जनाः}


\twolineshloka
{वर्धत्यधर्मेण नरस्ततो भद्राणि पश्यति}
{ततः सपत्नाञ्जयति समूलस्तु विनश्यति}


\twolineshloka
{`यत्र धर्मेण वर्धन्ते राजानो राजसत्तम}
{सर्वान्सपत्नान्वाधन्ते राज्यं चैषां विवर्धते'}


\twolineshloka
{मया हि दृष्टा दैतेया दानवाश्च महीपते}
{वर्धमाना ह्यधर्मेण क्षयं चोपगताः पुनः}


\twolineshloka
{पुरा देवयुगे चैव दृष्टं सर्वं मया विभो}
{अरोचयन्सुरा धर्मं धर्मं तत्यजिरेऽसुराः}


\twolineshloka
{तीर्तानि देवा विविशुर्नाविशन्भारतासुराः}
{तानधर्मकृतो दर्पः पूर्वमेव समाविशत्}


\twolineshloka
{दर्पान्मानः समभवन्मानात्क्रोधो व्यजायत}
{क्रोधादहीस्ततोऽलञ्जा वृत्तं तेषां ततोऽनशत्}


\twolineshloka
{तानलज्जान्गतश्रीकान्हीनवृत्तान्वृथाव्रतान्}
{क्षमा लक्ष्मीः स्वधर्मश्च नचिरात्प्रजहुस्ततः}


% Check verse!
लक्ष्मीस्तु देवानगमदलक्ष्मीरसुरान्नृप
\twolineshloka
{तानलक्ष्मीसमाविष्टान्दर्पोपहतचेतसः}
{दैतेयान्दानवांश्चैव कलिरप्याविशत्ततः}


\twolineshloka
{तानलक्ष्मीसमाविष्टान्दानवान्कलिनाहतान्}
{दर्पाभिभूतान्कौन्तेय क्रियाहीनानचेतसः ॥मानाभिभूतानचिराद्विनाशः समपद्यत}


\twolineshloka
{निर्यशस्कास्तथा दैत्याः कृत्स्नशो विलयं गताः}
{`अधर्मरुचयोराजन्नलक्ष्म्या समधिष्ठिताः'}


\twolineshloka
{देवास्तु सागरांश्चैव सरितश्च सरांसि च}
{अभ्यगच्छन्धर्मशीलाः पुण्यान्यावतनानि च}


\twolineshloka
{तपोभिः क्रतुभिर्दानैराशीर्वादैश्च पाण्डव}
{प्रजहुः सर्वपापानि श्रेयश्च प्रतिपेदिरे}


\twolineshloka
{एवमादानवन्तश् निरादानाश्च सर्वशः}
{तीर्थान्यगच्छन्विबुधास्तेनापुर्भूतिमुत्तमाम्}


\twolineshloka
{तथा त्वमपि राजेन्द्र स्नात्वा तीर्थेषु सानुजः}
{पुनर्वेत्स्वसि तां लक्ष्मीमेष पन्थाः सनातनः}


\twolineshloka
{यथैव हि नृगो राजा शिविरौशीनरो यथा}
{भगीरयो वसुमना गयः पूरुः पुरूरवाः}


\twolineshloka
{चरमाणास्तपो नित्यंस्पर्शनादम्भसश्च ते}
{तीर्थामिगमनात्पूता दर्शनाच्च महात्मनाम्}


\twolineshloka
{अलभन्त यशः पुण्यं धनानि च विशांपते}
{तथा त्वमपि राजेन्द्र लब्ध्वा सुविपुलां श्रियम्}


\twolineshloka
{यथा चेक्ष्वाकुरभवत्सपुत्रजनबान्धवः}
{मुचुकुन्दोऽथ मांधाता मरुत्तश्च महीपतिः}


\twolineshloka
{कीर्तिं पुण्यामविन्दन्त यथा देवास्तपोबलात्}
{देवर्षयश्च कार्त्स्न्येन तथा त्वमपि वेत्स्यसि}


\twolineshloka
{धार्तराष्ट्रास्त्वधर्मेण मोहिन च वशीकृताः}
{नचिराद्वै विनङ्क्ष्यन्ति दैत्या इव न संशयः}


\chapter{अध्यायः ९३}
\twolineshloka
{वैशंपायन उवाच}
{}


\twolineshloka
{ने तथा सहिता वीरा वसन्तस्तत्रतत्र ह}
{क्रमेण पृथिवीपाल नैमिषारण्यमागताः}


\twolineshloka
{तत्स्तीर्थेषु पुण्येषु गोमत्याः पाण्डवा नृप}
{कृताभिषेकाः प्रददुर्गाश्च वित्तं च भारत}


\twolineshloka
{तत्र देवान्पितॄन्विप्रांस्तर्पयित्वा पुनःपुनः}
{कन्यातीर्थेऽश्वतीर्थे च गवां तीर्थे च भारत}


\twolineshloka
{कालकोट्यां विपप्रस्थे गिरावुष्य च कौरवाः}
{बाहुदायां महीपाल चक्रुः सर्वेऽभिषेचनम्}


\twolineshloka
{प्रयागे देवयजने देवानां पृथिवीपते}
{ऊषुगप्लुत्य गात्राणि तपश्चातस्थुरुत्तमम्}


\twolineshloka
{गङ्गायमुनयोश्चैव संगमे सत्यसंगराः}
{विपाप्मानो महात्मानो विप्रेभ्यः प्रददुर्वसु}


\twolineshloka
{तपस्विजनजुष्टां च ततो वेदीं प्रजापतेः}
{जग्मुः पाण्डुसुता राजन्ब्राह्मणैः सह भारत}


\twolineshloka
{तत्र ते न्यवसन्वीरास्तपश्चातस्थुरुत्तमम्}
{संतर्पयन्तः सततं वन्येन हविषा द्विजान्}


\twolineshloka
{ततो महीधरं जग्मुर्धर्मज्ञेनाभिसत्कृतम्}
{राजर्षिणा पुण्यकृता गयेनानुपमद्युते}


\twolineshloka
{नगो गयशिरो यत्र पुण्या चैव महानदी}
{वानीरमालिनी रम्या नदी पुलिनशोभिता}


\twolineshloka
{दिव्यं पवित्रकूटं च पवित्रधरणीधरम्}
{ऋषिजुष्टं सुपुण्यं तत्तीर्थं ब्रह्मसरोतुलम्}


\twolineshloka
{अगस्त्यो भगवान्यत्र गतो वैवस्वतं प्रति}
{रउवास च स्वयं तत्र धर्मराजः सनातनः}


\twolineshloka
{सर्वासां सरितां चैव समुद्भेदो विशांपते}
{यत्रसंनिहितो नित्यं महादेवः पिनाकधृक्}


\twolineshloka
{`परिपूर्णः परंज्योतिः परमात्मा सनातनः}
{ब्रह्मादिभिरुपास्योऽयं भगवान्परमेश्वरः}


\twolineshloka
{तं प्रणम्य महादेवं चतुर्वर्गफलप्रदम्}
{रसिद्धिक्षेत्रमिदं मत्वासर्वेषांमोक्षकाङ्क्षिणाम्}


\threelineshloka
{तत्र ते पाण्डवा वीराश्चातुर्मास्यैस्तदेजिरे}
{ऋषियज्ञेन महता यत्राक्षयवटो महान्}
{अक्षये देवयजने अक्षयं यत्रवै फलम्}


\twolineshloka
{ये तु तत्रोपवासांस्तु चक्रुर्निश्चितमानसाः}
{ब्राह्मणास्तत्रशतशः समाजग्मुस्तपोधनाः}


\threelineshloka
{चातुर्मास्येनायजन्त आर्षेण विधिना तदा}
{तत्र विद्यातपोवृद्धा ब्राह्मणा वेदपारगाः}
{कथां प्रचक्रिरे पुण्यां सदसिस्था महात्मनाम्}


\threelineshloka
{तत्र विद्याव्रतस्नातः कौमारं व्रतमास्थितः}
{शमठोऽकथयद्राजन्नाधूर्तरजसं गयम् ॥शमठ उवाच}
{}


\twolineshloka
{आधूर्तरजसः पुत्रो गयो राजर्षिसत्तमः}
{पुण्यानि तस्य कर्माणि तानि मे शृणु भारत}


\twolineshloka
{यस्य यज्ञो बभूवेह बह्वन्नो बहुदक्षिणः}
{यत्रान्नपर्वता राजञ्शतशोऽथ सहस्रशः}


\twolineshloka
{घृतकुल्याश्च दध्नश्च नद्यो बहुशतास्तथा}
{व्यञ़्जनानां प्रवाहाश्च महार्हाणां सहस्रशः}


\twolineshloka
{अहन्यहनि चाप्येवं याचतां संप्रदीयते}
{अन्ये च ब्राह्मणा राजन्भुञ्जतेऽन्नं सुसंस्कृतम्}


\twolineshloka
{तत्र वै दक्षिणाकाले ब्रह्मघोषो दिवं गतः}
{न च प्रज्ञायते किंचिद्ब्रह्मशब्देन भारत}


\twolineshloka
{पुण्येन चरता राजन्भूर्दिशः खं नभस्तथा}
{आपूर्णमासीच्छब्देन तदप्यासीन्महाद्भुतम्}


\twolineshloka
{यत्रस्म गाथा गायन्ति मनुष्या मरतर्षभ}
{अन्नपानैः शुभैस्तृप्ता देशे देशे सुवर्चसः}


\twolineshloka
{गयस्य यज्ञे के त्वद्य प्राणिनो भोक्तुमीप्सवः}
{तत्र भोजनशिष्टस्य पर्वताः पञ्चविंशतिः}


\twolineshloka
{न तत्पूर्वे जनाश्चक्रुर्न करिष्यन्ति चापरे}
{गयो यदकरोद्यज्ञे राजर्षिरमितद्युतिः}


\twolineshloka
{कथं तु देवा हविषा गयेन परितर्पिताः}
{पुन शक्ष्यन्त्युपादातुमन्यैर्दत्तानि कानिचित्}


\threelineshloka
{सिकता वा यथा लोके यथा वा दिवि तारकाः}
{यथा वा वर्षतोधारा असंख्येयाः स्म केनचित्}
{तथा गणयितुं शक्या गययज्ञे न दक्षिणाः}


\twolineshloka
{एवंविधाः सुबहवस्तस्य यज्ञा महीपतेः}
{बभूवुरस्य सरसः समीपे कुरुनन्दन}


\chapter{अध्यायः ९४}
\twolineshloka
{वैशंपायन उवाच}
{}


\twolineshloka
{ततः संप्रस्थितो राजा कौन्तेयो भूरिदक्षिणः}
{अगस्त्याश्रममासाद्य दुर्जयायामुवास ह}


\twolineshloka
{तत्रैव लोमशं राजा पप्रच्छ वदतांवरः}
{अगस्त्येनेह वातापिः किमर्थमुपशामितः}


\threelineshloka
{आसीद्वा किंप्रभावश्च स दैत्यो मानवान्तकः}
{किमर्थं चोदितो मन्युरगस्त्यस्य महात्मनः ॥लोमश उवाच}
{}


\twolineshloka
{इल्वलो नाम दैतेय आसीत्कौरवनन्दना}
{मणिमत्यां पुरि पुरा वातापिस्तस्य चानुजः}


\twolineshloka
{स ब्राह्मणं तपोयुक्तमुवाच दितिनन्दनः}
{पुत्रं मे भगवानेकमिन्द्रतुल्यं प्रयच्छतु}


\twolineshloka
{तस्मै स ब्राह्मणो नादात्पुत्रं वासवसंमितम्}
{चुक्रोध सोऽसुरस्तस्य ब्राह्मणस्य ततो भृशम्}


\twolineshloka
{तदाप्रभृतिराजेन्द्र इल्वलो ब्रह्महाऽसुरः}
{मन्युमान्भ्रातरं छागं मायावी ह्यकरोत्ततः}


\twolineshloka
{मेषरूपी च वातापिः कामरूप्यभवत्क्षणात्}
{संस्कृत्यतं भोजयति ततो विप्रं जिधांसति}


\twolineshloka
{स चाह्वयति यं वाचा गतं वैवस्वतक्षयम्}
{स पुनर्देहमास्थाय जीवन्स्म प्रत्यदृश्यत}


\twolineshloka
{ततो वातापिमसुरं छागं कृत्वा सुसंस्कृतम्}
{तं ब्राह्मणं भोजयित्वा पुनरेव समाह्वयत्}


\twolineshloka
{तामिल्वलेन महता स्वरेण गिरमीरिताम्}
{श्रुत्वाऽतिमायो बालवान्क्षिप्रं ब्राह्मणकण्टकः}


\twolineshloka
{तस्य पार्श्वं विनिर्भिद्य ब्राह्मणस्य महासुरः}
{वातापिः प्रहसन्राजन्निश्चक्राम विशांपते}


\twolineshloka
{एवं स ब्राह्मणान्राजन्भोजयित्वा पुनःपुनः}
{हिंसयामायस दैतेय इल्वलो दुष्टचेतनः}


\twolineshloka
{अगस्त्यश्चापि भगवानेतस्मिन्काल एव तु}
{पितॄन्ददर्श गर्ते वै लम्बमानानधोमुखान्}


\threelineshloka
{सोऽपृच्छल्लम्बमानांस्तान्भगवन्तश्च किंपराः}
{`किंमर्थं वेह लम्बध्वे गर्ते यूयमधोमुखाः'}
{संतानहेतोरिति ते प्रत्यूचुर्ब्रह्मवादिनः}


\twolineshloka
{ते तस्मै कथयामासुर्वयं ते पितरः स्वकाः}
{गर्तमेतमनुप्राप्ता लम्बामः प्रसवार्थिनः}


\twolineshloka
{यदि नो जनयेथास्त्वमगस्त्यापत्यमुत्तमम्}
{स्यान्नोस्मान्निरयान्मोक्षस्त्वं च पुत्राप्नुया गतिम्}


\twolineshloka
{स तानुवाच तेजस्वी सत्यधर्मपरायणः}
{करिष्ये पितरः कामं व्येतु वो मानसो ज्वरः}


\twolineshloka
{ततः प्रसवसन्तानं चिन्तयन्भगवानृषिः}
{आत्मनः प्रसवस्यार्थे नापश्यत्सदृशीं स्त्रियम्}


\twolineshloka
{स तस्य तस् सत्वस्य तत्तदङ्गमनुत्तमम्}
{संगृह्यतत्समैरङ्गैर्निर्ममे स्त्रियमुत्तमाम्}


\twolineshloka
{स तां विदर्भराजाय पुत्रकामाय ताम्यते}
{निर्मितामात्मनोऽर्थाय मुनिः प्रादान्महातपाः}


\twolineshloka
{सा तत्र जज्ञे सुभगा विद्युत्सौदामनी यथा}
{विभ्राजमाना वपुषा व्यवर्धत शुभानना}


\twolineshloka
{जातमात्रां च तां दृष्ट्वा वैदर्भः पृथिवीपतिः}
{प्रहर्षेण द्विजातिभ्यो न्यवेदयत भारत}


\twolineshloka
{अभ्यनन्दन्त तां सर्वे ब्राह्मणा वसुधाधिप}
{लोपामुद्रेति तस्याश्च चक्रिरे नाम ते द्विजाः}


\twolineshloka
{ववृधे सा महाराज बिभ्रती रूपमुत्तमम्}
{अप्स्विवोत्पलिनी शीघ्रमग्नेरिव शिखा शुभा}


\twolineshloka
{तां यौवनस्थां राजेन्द्रशतं कन्याः स्वलंकृताः}
{दास्यः शतं च कल्याणीमुपातस्थुर्वशानुगाः}


\twolineshloka
{सा स्म दासीशतवृता मध्ये कन्याशतस्य च}
{आरस्ते तेजस्विनी कन्या रोहिणीव दिविप्रभा}


\twolineshloka
{यौवनस्थामपि च तां शीलाचारसमन्विताम्}
{न वव्रे पुरुषः कश्चिद्भयात्तस्य महात्मनः}


\twolineshloka
{सा तु सत्यवती कन्या रूपेणाप्सरसोप्यति}
{तोषयामास पितरं शीलेन स्वजनं तथा}


\twolineshloka
{वैदर्भी तु तथायुक्तां युवतीं प्रेक्ष्य वै पिता}
{मनसा चिन्तयामास कस्मै दद्यामिमांसुताम्}


\chapter{अध्यायः ९५}
\twolineshloka
{लोमश उवाच}
{}


\twolineshloka
{यदात्वमन्यतागस्त्यो गार्हस्थ्ये तां क्षमामिति}
{तदाऽभिगम्य प्रोवाच वैदर्भं पृथिवीपतिम्}


\twolineshloka
{राजन्निवेशे बुद्धिर्मे वर्तते पुत्रकारणात्}
{वरयेत्वां महीपाल लोपामुद्रां प्रयच्छ मे}


\twolineshloka
{एवमुक्तः स मिनिना महीपालो विचेतनः}
{प्रत्याख्यानाय चाशक्तः प्रदातुं चैव नैच्छत}


\twolineshloka
{ततः स भार्यामभ्येत्य प्रोवाच पृथिवीपतिः}
{महर्षिर्वीर्यवानेष क्रुद्धः शापाग्निना दहेत्}


\twolineshloka
{तं तथा दुःखितं दृष्ट्वा सभार्यं पृथिवीपतिम्}
{लोपामुद्राऽभिगम्येदं काले वचनमब्रवीत्}


\twolineshloka
{न मत्कृते महीपाल पीडामभ्येतुमर्हसि}
{प्रयच्छ मामगस्त्याय त्राह्यात्मानं मया पितः}


\twolineshloka
{दुहितुर्वचनाद्राजा सोऽगस्त्याय महात्मने}
{लोपामुद्रां ततः प्रादाद्विधिपूर्वं विशांपते}


\twolineshloka
{प्राप्य भार्यामगस्त्यस्तु लोपामुद्रामभाषत}
{महार्हाण्युत्सृजैतानि वासांस्याभरणानि च}


\twolineshloka
{ततः सा दर्शनीयानि महार्हाणि तनूनि च}
{समुत्ससर्ज रम्भोरूर्वसनान्यायतेक्षणा}


\twolineshloka
{ततश्चीराणि जग्राह वल्कलान्यजिनानि च}
{समानव्रतचर्या च बभूवायतलोचना}


\twolineshloka
{गङ्गाद्वारमथागम्य भगवानृषिसत्तमः}
{उग्रमातिष्ठत तपः सह पत्न्याऽनुकूलया}


\twolineshloka
{सा प्रीत्या बहुमानाच्च पतिं पर्यचरत्तदा}
{अगस्त्यश्च परां प्रीतिं भार्यायामगमत्प्रभुः}


\twolineshloka
{ततो बहुतिथे काले लोपामुद्रां विशांपते}
{तपसा द्योतितां स्नातां ददर्श भगवानृषिः}


\twolineshloka
{स तस्याः परिचारेण शौचेन च दमेन च}
{श्रिया रूपेण च प्रीतो मैथुनायाजुहाव ताम्}


\twolineshloka
{ततः सा प्राञ्जलिर्भूत्वा लज्जमानेव भामिनी}
{तदा सप्रणयं वाक्यं भगवन्तमथाब्रवीत्}


\twolineshloka
{असंशयं प्रजाहेतोर्भार्यां पतिरविन्दत}
{पा तु त्वयि मम प्रीतिस्तामृषे कर्तुमर्हसि}


\twolineshloka
{यथा पितुर्गृहे विप्र प्रासादे शयनं मम}
{तथाविधे त्वं शयने मामुपैतुमिहार्हसि}


\twolineshloka
{इच्छामि त्वां स्रग्विणं च भूषणैश्च विभूषितम्}
{उपसर्तुं यथाकामं दिव्याभरणभूषिता}


\threelineshloka
{अन्यथा नोपतिष्ठेयं चीरकाषायवासिनी}
{नैवापवित्रो विप्रर्षे भूषणोयं कथंचन ॥अगस्त्य उवाच}
{}


\threelineshloka
{न ते धनानि विद्यन्ते लोपामुद्रे तथा मम}
{यथाविधानि कल्याणि पितुस्तव सुमध्यमे ॥लोपामुद्रोवाच}
{}


\threelineshloka
{ईशोसि तपसा सर्वं समाहर्तुं तपोधन}
{क्षणेन जीवलोके यद्वसु किंचन विद्यते ॥अगस्त्य उवाच}
{}


\threelineshloka
{एवमेतद्यथाऽऽत्थ त्वं तपोव्ययकरं तु तत्}
{यथा तु मे न नश्येत तपस्तन्मां प्रयोदय ॥लोपामुद्रोवाच}
{}


\twolineshloka
{अल्पावशिष्टः कालोऽयमृतोर्मम तपोधन}
{न चान्यथाऽहमिच्छामि त्वामुपैतुं कथंचन}


\threelineshloka
{न चापि धर्ममिच्छामि विलोप्तुं ते कथंचन}
{एवं तु मे यथाकामं संपादयितुमर्हसि ॥अगस्तय उवाच}
{}


\twolineshloka
{यद्येष काम सुभगे तव बुद्ध्या विनिश्चितः}
{हर्तुं गच्छाम्यहं भद्रे चर काममिह स्थिता}


\chapter{अध्यायः ९६}
\twolineshloka
{लोमश उवाच}
{}


\twolineshloka
{ततो जगाम कौरव्य सोऽगस्त्यो भिक्षितुं वसु}
{श्रुतर्वाणं महीपालं यं वेदाभ्यधिकं नृपैः}


\twolineshloka
{स विदित्वा तु नृपतिः कुम्भयोनिमुपागतम्}
{विषयान्ते सहामात्यः प्रत्यगृह्णात्सुसत्कृतम्}


\threelineshloka
{तस्मै चार्घ्यं थान्यायमानीय पृथिवीपतिः}
{प्राञ्जलिः प्रयतो भूत्वापप्रच्छागमनेऽर्थिताम् ॥अगस्त्य उवाच}
{}


\threelineshloka
{वित्तार्थिनमनुप्राप्तं विद्धि मां पृथिवीपते}
{थाशक्त्यविहिंस्यान्यान्संविभागं प्रयच्छ मे ॥लोमश उवाच}
{}


\twolineshloka
{तत आयव्ययौ पूर्णो तस्मै राजा न्यवेदयत्}
{अतो विद्वन्नुपादत्स्व यदत्रव्यतिरिच्यते}


\twolineshloka
{तत आयव्ययौ दृष्ट्वा समौ सममतिर्द्विजः}
{सर्वथा प्राणिनां पीडामुपादानादमन्यत}


\twolineshloka
{स श्रुतर्वाणमादाय ब्रध्नश्वमगमत्ततः}
{स च तौ विषयस्यान्ते प्रत्यगृह्णाद्यथाविधि}


\fourlineindentedshloka
{तयोरर्ध्यं च पाद्यं च ब्रध्नश्वः प्रत्यवेदयत्}
{अनुज्ञाप्यच पप्रच्छ प्रयोजनमुपक्रमे}
{`वद कामं मुनिश्रेष्ठ धन्योस्म्यागमनेन ते' ॥अगस्त्य उवाच}
{}


\threelineshloka
{वित्तकामाविह प्राप्तौ विद्ध्यावां पृथिवीपते}
{यथाशक्त्यविहिंस्यान्यान्संविभागं प्रयच्छ नौ ॥लोमश उवाच}
{}


\twolineshloka
{तत आयव्ययौ पूर्णौ ताभ्यां राजा न्यवेदयत्}
{अतो ज्ञात्वा तु गृह्णीतं यदत्रव्यतिरिच्यते}


\twolineshloka
{तत आयव्ययौ दृष्ट्वासमौ सममतिर्द्विजः}
{सर्वथा प्राणिनां पीडामुपादानादमन्यत}


\twolineshloka
{पौरुकुत्सं ततो जग्मुस्त्रसदस्युं महाधनम्}
{अगस्त्यश्च श्रुतर्वा च ब्रध्नश्वश्च महीपतिः}


\twolineshloka
{त्रसदस्युस्तु तान्दृष्ट्वा प्रत्यगृह्णाद्यथाविधि}
{अभिगम्य महाराज विषयान्ते महामनाः}


\threelineshloka
{अर्चयित्वा यथान्यायमैक्ष्वाको राजसत्तमः}
{समस्तांश्च ततोऽपच्छत्प्रयोजनमुपक्रमे ॥अगस्त्य उवाच}
{}


\threelineshloka
{वित्तकामानिह प्राप्तान्विद्धि नः पृथिवीपते}
{यथाशक्त्यवीहिंस्यान्यान्संविभागं प्रयच्छ नः ॥लोमश उवाच}
{}


\twolineshloka
{तत आयव्ययौ पूर्णौ तेषां राजा न्यवेदयत्}
{एतज्ज्ञात्वा ह्युपादद्ध्वं यदत्रव्यतिरिच्यते}


\twolineshloka
{तत आयव्ययौ दृष्ट्वासमौ सममतिर्द्विजः}
{सर्वथा प्राणिनां पीडामुपादानादमन्यत}


\twolineshloka
{ततः सर्वे समेत्याथ ते नृपास्तं महामुनिम्}
{इदमूचुर्महाराज समवेक्ष्य परस्परम्}


\threelineshloka
{अयं वै दानवो ब्रह्मन्निल्वलो वसुमान्भुवि}
{तमतिक्रम्य सर्वेऽद्यवयं चार्तामहे वसु ॥लोमश उवाच}
{}


\twolineshloka
{तेषां तदासीदुचितमिल्वलस्यैव भिक्षणम्}
{ततस्ते सहिता राजन्निल्वलं समुपाद्रवन्}


\chapter{अध्यायः ९७}
\twolineshloka
{लोमश उवाच}
{}


\twolineshloka
{इल्वलस्तान्विदित्वा तु महर्षिसहितान्नृपान्}
{उपस्थितान्सहामात्यो विषयान्ते ह्यपूजयत्}


\twolineshloka
{तेषां ततोऽसुरश्रेष्ठस्त्वातिथ्यमकरोत्तदा}
{सुसंस्कृतेन कौरव्य भ्रात्रा वातापिना तदा}


\twolineshloka
{ततो राजर्षयः सर्वे विषण्णा गतचेतसः}
{वातापिं संस्कृतं दृष्ट्वा मेषभूतं महासुरम्}


\twolineshloka
{अथाब्रवीदगस्त्यस्तान्राजर्षीनृषिसत्तमः}
{विषादो वो न कर्तव्यो ह्यहं भोक्ष्ये महासुरम्}


\twolineshloka
{धुर्यासनमथासाद्य निषसाद महानृषिः}
{तं पर्यवेषद्दैत्येन्द्र इल्वलः प्रहसन्निव}


\twolineshloka
{अगस्त्य एव कृत्स्नं तु वातापिं बुभुजे ततः}
{`बह्वन्नाशापि ते मेऽस्तीत्यवदद्भक्षयन्स्वयम्'}


\threelineshloka
{भुक्तवत्यसुरोऽह्वानमकरोत्तस्य चेल्वलः}
{`वातापे प्रतिबुध्यस्व दर्शयन्बलतेजसी}
{तपसा दुर्जयो यावदेष त्वां नातिवर्तते}


\twolineshloka
{ततस्तस्योदरं भेत्तुं वातापिर्वेगमाहरत्}
{तमबुध्यत तेजस्वी कुम्भयोनिर्महातपाः}


\twolineshloka
{स वीर्यात्तपसोग्रस्तु ननर्द भगवानृषिः}
{एष जीर्णोसि वातापे मया लोकस्य शान्तये}


\twolineshloka
{इत्युक्त्वा स्वकराग्रेण उदरं समताडयत्}
{त्रिरेवं प्रतिसंरब्धस्तेजसा प्रज्वलन्निव'}


\twolineshloka
{ततो वायुः प्रादुरभूदधस्तस्य महात्मनः}
{शब्देन महता तात गर्जन्निव यथा घनः}


\threelineshloka
{वातापे निष्क्रमस्वेति पुनः पुनरुवाच ह}
{तं प्रहस्याब्रवीद्राजन्नगस्त्यो मुनिसत्तमः}
{कुतो निष्क्रमितुं शक्तो मया जीर्णस्तु सोसुरः}


\threelineshloka
{इल्वलस्तु विषण्णोऽभूद्दृष्ट्वा जीर्णं महासुरम्}
{प्राञ्जलिश्च सहामात्यैरिदं वचनमब्रवीत्}
{किमर्थमुपयाताः स्थ ब्रूत किं करवाणि वः}


\twolineshloka
{प्रत्युवाच ततोऽगस्त्यः प्रहसन्निल्वलं तदा}
{ईशोस्यसुर विद्मस्त्वां वयं सर्वे धनेश्वरम्}


\twolineshloka
{एते च नातिधनिनो धनाशा महती मम}
{यथाशक्त्यविहिंस्यान्यान्संविभागं प्रयच्छ नः}


\threelineshloka
{ततोऽवमत्य तमृषिमिल्वलो वाक्यमब्रवीत्}
{दित्सितं यदि वेत्सि त्वंततो दास्यामि ते वसु ॥अगस्त्य उवाच}
{}


\twolineshloka
{गवां दशसहस्राणि राज्ञामेकैकशोऽसुर}
{तावदेव सुवर्णस्य दित्सितं ते महासुर}


\threelineshloka
{मह्यं ततो वै द्विगुणं रथश्चैव हिरण्मयः}
{मनोजवौ वाजिनौ च दित्सितं ते महासुर ॥`लोमश उवाच}
{}


\fourlineindentedshloka
{उल्वलस्तु मुनिं प्राह सर्वमस्ति यथाऽऽत्थ माम्}
{सर्वमेतत्प्रदास्यामि हिरण्यं गाश्च यद्धनम्}
{रथं तु यदवोचो मां नैतं विद्म हिरण्मयम् ॥आगस्त्य उवाच}
{}


\threelineshloka
{न मे वागनृता काचिदुक्तपूर्वा महाऽसुर}
{विज्ञायतां रथः साधु व्यक्तमस्ति हिरण्मयः ॥लोमश उवाच}
{}


\twolineshloka
{विज्ञायमानः स रथः कौन्तेयासीद्धिरण्मयः'}
{ततः प्रव्यथितो दैत्यो ददावभ्यधिकं वसु}


\threelineshloka
{विवाजी च सुवाजी च तस्मिन्युक्तौ रथे हयौ}
{ऊहतुः स वसूनाशु तावगस्त्याश्रमं प्रति}
{सर्वान्राज्ञः सहागस्त्यान्निमेषादिव भारत}


\twolineshloka
{`इल्वलस्त्वनुगम्यैनमगस्त्यं हन्तुमैच्छत}
{भस्म चक्रे महातेजा हुङ्कारेण महाऽसुरम्'}


\threelineshloka
{अगस्त्येनाभ्यनुज्ञाता जग्मू राजर्षयस्तदा}
{कृतवांश् मुनिः सर्वं लोपामुद्राचिकीर्षितम् ॥लोपामुद्रोवाच}
{}


\threelineshloka
{कृतवानसि तत्सर्वं भगवन्मम काङ्क्षितम्}
{उत्पादय सकृन्मह्यमपत्यं वीर्यवत्तरम् ॥अगस्त्य उवाच}
{}


\twolineshloka
{तुष्टोऽहमस्मि कल्याणि तव वृत्तेन शोभने}
{विचारणामपत्ये तु तव वक्ष्यामि तां शृणु}


\threelineshloka
{सहस्रं तेऽस्तु पुत्राणआं शतं वा तत्समं तव}
{दश वा शततुल्याः स्युरेको वाऽपि सहस्रजित् ॥लोपामुद्रोवाच}
{}


\twolineshloka
{सहस्रसंमितः पुत्र एकोप्यस्तु तपोधन}
{एको हि बहुभिः श्रेयान्विद्वान्साधुरसाधुभिः}


\twolineshloka
{स तथेति प्रतिज्ञाय तया समगमन्मुनिः}
{समये समशीलिन्या श्रद्धावान्श्रद्दधानया}


\twolineshloka
{तत आधाय गर्भं तमगमद्वनमेव सः}
{तस्मिन्वनगते गर्भो ववृधे सप्त शारदान्}


\twolineshloka
{सप्तमेऽब्दे गते चापि प्राच्यवत्स महाकविः}
{ज्वलन्निव प्रभावेन दृढस्युर्नाम भारत}


\twolineshloka
{साङ्गोपनिषदान्वेदाञ्जपन्नेव महातपाः}
{तस्य पुत्रोऽभवदृषेः स तेजस्वी महानृषिः}


\twolineshloka
{स बाल एवतेजस्वी पितुस्तस्य निवेशने}
{इध्मानां भारमाजह्रे इध्मवाहस्ततोऽभवत्}


\threelineshloka
{तथायुक्तं तु तं दृष्ट्वा मुमुदे स मुनिस्तदा}
{एवं स जनयामास भारतापत्यमुत्तमम्}
{लेभिरे पितरश्चास्य लोकान्राजन्यथेप्सितान्}


\twolineshloka
{अगस्त्यस्याश्रमश्चायमत ऊर्ध्वं विशांपते}
{ख्यातो भुवि महाराज तेजसा तस्य धीमतः}


\threelineshloka
{प्राह्लादिरेवं वातापिर्ब्रह्मघ्नो दुष्टचेतनः}
{एवं विनाशितो राजन्नगस्त्येन महात्मना}
{तस्यायमाश्रमो राजन्रमणीयैर्गुणैर्युतः}


\twolineshloka
{एषा भागीरथी पुण्या देवगन्धर्वसेविता}
{वातेरिता पताकेव विराजति नभस्तले}


\twolineshloka
{प्रतार्यमाणा कूटेषु यथा निम्नेषु नित्यशः}
{शिलातलेषु संत्रस्ता पन्नगेन्द्रवधूरिव}


\threelineshloka
{दक्षिणां वै दिशं सर्वां प्लावयन्ती च मातृवत्}
{पूर्वं शंभोर्जटाभ्रष्टा समुद्रमहिषी प्रिया}
{अस्यां नद्यां मुपुण्यायां यथेष्टमवगाह्यताम्}


\chapter{अध्यायः ९८}
\twolineshloka
{[लोमश उवाच}
{}


\twolineshloka
{युधिष्ठिर निबोधेदं त्रिषु लोकेषु विश्रुतम्}
{भृगोस्तीर्थं महाराज महर्षिगणसेवितम्}


\twolineshloka
{यत्रोपस्पृष्टवान्रामो हृतंतेजस्तदाप्तवान्}
{अत्र त्वंभ्रातृभिः सार्धं कृष्णया चैव पाण्डव}


\threelineshloka
{दुर्योधनहृतंतेजः पुनरादातुमर्हसि}
{कृतवैरेण रामेण यथा कचोपहृतं पुनः ॥वैशंपायन उवाच}
{}


\twolineshloka
{स तत्रभ्रातृभिश्चैव कृष्णया चैव पाण्डवः}
{स्नात्वा देवान्पितॄंश्चैव तर्पयामास भारत}


\twolineshloka
{तस्य तीर्थस्य रूपं वै दीप्ताद्दीप्ततरं बभौ}
{अप्रवृष्यतरश्चासीच्छात्रवाणां नरर्षभ}


\fourlineindentedshloka
{अपृच्छच्चैव राजेन्द्र लोमशं पाण्डुनन्दनः}
{भगवन्किमर्थं रामस्य हृतमासीद्वपुः प्रभो}
{कथं प्रत्याहृतंचैव एतदाचक्ष्व पृच्छतः ॥लोमश उवाच}
{}


\twolineshloka
{शृणु रामस्य राजेन्द्र भार्गवस्य च धीमतः}
{जातो दशरथस्यासीत्पुत्रो रामो महात्मनः}


\twolineshloka
{विष्णुः स्वेन शरीरेण रावणस्य वधाय वै}
{पश्यामस्तमयोध्यायां जातं दाशरथिं ततः}


\twolineshloka
{ऋचीकनन्दनो राभो भार्गवो रेणुकासुतः}
{तस् दाशरथेः श्रुत्वा ररामस्याक्लिष्टकर्मणः}


\twolineshloka
{कौतूहलान्वितो रामस्त्वयोध्यामगमत्पुनः}
{जिज्ञासमानो रामस्य वीर्यं दाशरथेस्तदा}


\twolineshloka
{तं वै दशरथः श्रुत्वा वियान्तमुपागतम्}
{प्रेषयामास रामस्य रामं पुत्रं पुरस्कृतम्}


\twolineshloka
{स तमभ्यागतं दृष्ट्वा उद्यतास्त्रमवस्थितम्}
{प्रहसन्निव कोन्तेय रामो वचनमब्रवीत्}


\twolineshloka
{कृतकालं हि राजेन्द्र धनुरेतन्मया विभो}
{समारोपय यत्नेन यदि शक्नोषि पार्तिव}


\threelineshloka
{इत्युक्तस्त्वाह भगवंस्त्वं नाधिक्षेप्तुमर्हसि}
{नाहमप्यधमो धर्मे क्षत्रियाणां द्विजातिषु}
{इश्र्वाकूणां विशेषेण बाहुवीर्ये न कत्थनम्}


\twolineshloka
{तमेवं वादिनं तत्र रामो वचनमब्रवीत्}
{अलं वै व्यपदेशेन धनुरायच्छ राघव}


\twolineshloka
{ततो जग्राह रोषेण क्षत्रियर्षभमूदनम्}
{रामो दाशरथिर्दिव्यं हस्ताद्रामस्य कार्मुकम्}


\threelineshloka
{धनुरारोपयामास सलील इव भारत}
{ज्याशब्दमकरोच्चैव स्मयमानः स वीर्यवान्}
{तस्य शब्दस् भूतानि वित्रसन्त्यशनेरिव}


\twolineshloka
{अथाब्रवीत्तदा रामो रामं दाशरथिस्तदा}
{इदमारोपितं ब्र्हमन्किमन्यत्करवाणि ते}


\threelineshloka
{तस्य रामो ददौ दिव्यं जामदग्न्यो महात्मनः}
{शरमाकर्णदेशान्तमयमाकृष्यतामिति ॥लोमश उवाच}
{}


\twolineshloka
{एतच्छ्रुत्वाऽब्रवीद्रामः प्रदीप्त इव मन्युना}
{श्रूयते क्षम्यते चैव दर्पपूर्णोसि भार्गव}


\threelineshloka
{त्वया ह्यधिगतं तेजः क्षत्रियेभ्यो विशेषतः}
{पितामहप्रसादेन तेन मां क्षिपसि ध्रुवम्}
{पश्य मां स्वेन रूपेण चक्षुस्ते वितराम्यहम्}


\twolineshloka
{ततो रामशरीरे वै रामः पश्यति भार्गवः}
{आदित्यान्सवसून्रुद्रान्साध्यांश्च समरुद्गणान्}


\twolineshloka
{पितरो हुताशनश्चैव नक्षत्राणि ग्रहास्तथा}
{गन्धर्वा राक्षसा यक्षा नद्यस्तीर्थानि यानि च}


\twolineshloka
{ऋषयो वालखिल्याश्च ब्र्हमभूताः सनातनाः}
{देवर्षयश्च कार्त्स्न्येन समुद्राः पर्वतास्तथा}


\threelineshloka
{वेदाश्च सोपनिषदो वषट्कारैः सहाध्वरैः}
{चेतोमन्ति च सामानि धनुर्वेदश्च भारत}
{मेघवृन्दानि वर्षाणि विद्युतश्च युधिष्ठिर}


\twolineshloka
{ततः स भगवान्विष्णुस्तं वै बाणं मुमोच ह}
{शुष्काशनिसमाकीर्णं महोल्काभिश्च भारत}


\twolineshloka
{पांसुवर्षेण महता मेघवर्षैश्च भूतलम्}
{भूमिकम्पैश्च निर्घातैर्नादैश्च विपुलैरपि}


\twolineshloka
{स रामं विह्वलं कृत्वा तेजश्चाक्षिप्य केवलम्}
{आगच्छज्ज्वलितो बाणो रामबाहुप्रचोदितः}


\twolineshloka
{स तु विह्वलतां गत्वा प्रतिलभ्य च चेतनाम्}
{रामः प्रत्यागतप्राणः प्राणमद्विष्णुतेजसम्}


\twolineshloka
{विष्णुना सोभ्यनुज्ञातो महेन्द्रमगमत्पुनः}
{भीतस्तु तत्रन्यवसद्ब्रीडितस्तु महातपाः}


\twolineshloka
{ततः संवत्सरेऽतीते हृतौजसमवस्थितम्}
{निर्मदं दुःखितं दृष्ट्वा पितरो राममब्रुवन्}


\twolineshloka
{न वै सम्यगिदं पुत्र विष्णुमासाद्य वैकृतम्}
{स हि पूज्यश्च मान्यश्च त्रिषु लोकेषु सर्वदा}


\twolineshloka
{गच्छ पुत्रनदीं पुण्यां वधूसरकृताह्वयाम्}
{तत्रोपस्पृश्य तीर्थेषु पुनर्वपुरवाप्स्यसि}


\twolineshloka
{दीप्तोदं नाम तत्तीर्थं यत्रते प्रतितामहः}
{भृगुर्देवयुगे राम तप्तवानुत्तमं तपः}


\twolineshloka
{तत्तथा कृतवान्रामः कौन्तेय वचनात्पितुः}
{प्राप्तवांश्च पुनस्तेजस्तीर्थेऽस्मिन्पाण्डुनन्दन}


\twolineshloka
{एतदीदृशकं तात रामेणाक्लिष्टकर्मणा}
{प्राप्तमासीन्महाराज विष्णुमासाद्य वै पुरा}


\chapter{अध्यायः ९९}
\twolineshloka
{युधिष्ठिर उवाच}
{}


\threelineshloka
{भूय एवाहमिच्छामि महर्षेस्तस्य धीमतः}
{कर्मणां विस्तरं श्रोतुमगस्त्यस्य द्विजात्तम ॥लोमश उवाच}
{}


\twolineshloka
{शृणु राजन्कथां दिव्यामद्भुतामतिमानुषीम्}
{अगस्त्यस्य महाराज प्रभावममितौजसः}


\twolineshloka
{आसन्कृतयुगे घोरा दानवा युद्धदुर्मदाः}
{कालकेया इतिख्याता गणाः परमदारुणाः}


\twolineshloka
{ते तु वृत्रं समाश्रित्य नानाप्रहरणोद्यताः}
{समन्तात्पर्यधावन्त महेन्द्रप्रमुखान्सुरान्}


\twolineshloka
{ततो वृत्रवधे यत्नमकुर्वंस्त्रिदशाः पुरा}
{पुरंदरं पुरस्कृत्य ब्रह्माणमुपतस्थिरे}


\twolineshloka
{कृताञ्जलींस्तु तान्सर्वान्परमेष्ठीत्युवाच ह}
{विदितं मे सुराः सर्वं यद्वः कार्यं चिकीर्षितम्}


\twolineshloka
{तमुपायं प्रवक्ष्यामि यथा वृत्रं वधिष्यथ}
{दधीच इतिविख्यातो महानृषिरुदारधीः}


\twolineshloka
{तं गत्वा सहिताः सर्वेवरं वै संप्रयाचत}
{स वो दास्यति धर्मात्मा सुप्रीतेनान्तरात्मना}


\twolineshloka
{स वाच्यः सहितैः सर्वैर्भवद्भिर्जयकाङ्क्षिभिः}
{स्वान्यस्थीनि प्रयच्छेति त्रैलोक्यस् हिताय वै}


\twolineshloka
{स शरीरं समुत्सृज्य स्वान्यस्थीनि प्रदास्यति}
{तस्यास्थिभिर्महाघोरं वज्रं संस्क्रियतां दृढम्}


\twolineshloka
{महच्छत्रुहणं घोरं ष़डश्चं भीमनिःस्वनम्}
{तेन वज्रेण वै वृत्रं वधिष्यति शतक्रतुः}


\twolineshloka
{एतद्वः सर्वमाख्यातं तस्माच्छीघ्रं विधीयताम्}
{एवमुक्तास्ततो देवा अनुज्ञाप्य पितामहम्}


\twolineshloka
{नारायणं पुरस्कृत्य दधीचस्याश्रमं ययुः}
{सरस्वत्याः परे पारे नानाद्रुमलतावृतम्}


\twolineshloka
{षट्पदोद्गीतनिनदैर्विघुष्टं सामगैरिव}
{पुंस्कोकिलरवोन्मिश्रं जीवंजीवकनादितम्}


\twolineshloka
{महिषैश्च वराहैश्च सृमरैश्चमरैरपि}
{तत्र तत्रानुचरितं शार्दूलभयवर्जितैः}


\twolineshloka
{करेणुभिर्वारणैश्च प्रभिन्नकरटामुखैः}
{सरोवगाढैः क्रीडद्भिः समन्तादनुनादितम्}


\twolineshloka
{सिंहव्याघ्रैर्महानादान्नदद्भिरनुनादितम्}
{अपरैश्चापि संलीनैर्गुहाकन्दरशायिभिः}


\twolineshloka
{तेषु तेष्ववकाशेषु शोभितं सुमनोरमम्}
{त्रिविष्टपसमप्रख्यं दधीचाश्रममागमन्}


\twolineshloka
{तत्रापश्यन्दधीचं ते दिवाकरसमद्युतिम्}
{जाज्वल्यमानं वपुषा यथा साक्षात्पितामहम्}


\twolineshloka
{तस्य पादौ सुरा राजन्नभिवाद्य प्रणम्य च}
{अयाचन्त वरं सर्वे यथोक्तं परमेष्ठिना}


\twolineshloka
{ततो दधीचः परमप्रतीतःसुरोत्तमांस्तानिदमभ्युवाच}
{करोमि यद्वो हितमद्य देवाःस्वं चापि देहंस्वयमुत्सृजामि}


\twolineshloka
{स एवमुक्त्वा द्विपदांवरिष्ठःप्राणान्वशी स्वान्सहसोत्ससर्ज}
{ततः सुरास्ते जगृहुः परासो-रस्थीनि तस्याथ यथोपदेशम्}


\twolineshloka
{प्रहृष्टरूपाश्च जयाय देवा-स्त्वष्टारमागम् तमर्थमूचुः}
{त्वष्टा तु तेषां वचनं निशम्यप्रहृष्टरूपः प्रयतः प्रयत्नात्}


\twolineshloka
{चकार वज्रं भृशमुग्ररूपंकृत्वा च शक्रं स उवाच हृष्टः}
{अनेन वज्रप्रवरेण देवभस्मीकुरुष्वाद्य सुरारिमुग्रम्}


\twolineshloka
{ततो हतारिः सगणः सुखं वैप्रशाधिकृत्स्नं त्रिदिवं दिविष्ठः}
{त्वष्ट्रा तथोक्तस्तु पुरंदरस्त-द्वज्रं प्रहृष्टः प्रयतो ह्यगृह्नात्}


\chapter{अध्यायः १००}
\twolineshloka
{लोमश उवाच}
{}


\twolineshloka
{ततः स वज्री बलभिद्दैवतैरभिरक्षितः}
{आससाद ततो वृत्रं स्थितमावृत्य रोदसी}


\twolineshloka
{कालकेयैर्महाकायै समन्तादभिरक्षितम्}
{समुद्यतप्रहरणैः सशृङ्गैरिव पर्वतैः}


\twolineshloka
{ततो युद्धं समभवद्देवानां दानवैः सह}
{मुहूर्तं भरतश्रेष्ठ लोकत्रासकरं महत्}


\twolineshloka
{उद्यतप्रतिविद्धानां स्वङ्गानां वीरबाहुभिः}
{आसीत्सुतुमुः शब्दः शरीरेष्वभिपात्यताम्}


\twolineshloka
{शिरोभिः प्रपतद्भिश्चाप्यन्तरिक्षान्महीतलम्}
{तालैरिव महाराज वृन्ताद्भ्रष्टैरदृश्यत}


\twolineshloka
{ते हेमकवचा भूत्वा कालेयाः परिघायुधाः}
{त्रिदशानभ्यवर्तन्त दावदग्धा इवाद्रयः}


\twolineshloka
{तेषां वेगवतां वेगं सहितानां प्रधावताम्}
{न शेकुस्त्रिदशाः सोढुं ते भग्नाः प्राद्रवन्भयात्}


\twolineshloka
{तान्दृष्ट्वा द्रवतो भीतान्सहस्राक्षः पुरंदरः}
{वृत्रे विवर्धमाने च कश्मलं महदाविशत्}


\twolineshloka
{तं शक्तं कश्मलाविष्टं दृष्ट्वा विष्णुः सनातनः}
{जगाम शरणं शीघ्रं तं तु नारायणं प्रभुम्}


\twolineshloka
{तं शक्रं कश्मलाविष्टं दृष्ट्वा विष्णुः सनातनः}
{स्वतेजो व्यदधच्छक्रे बलमस्य विवर्धयन्}


\twolineshloka
{विष्णुना गोपितं शक्रं दृष्ट्वादेवगणास्ततः}
{स्वं स्वंतेजः समादध्युस्तथा ब्रह्मर्षयोऽमलाः}


\twolineshloka
{स समाप्यायितः शक्रो विष्णुना दैवतैः सह}
{ऋषिभिश्च महाभागैर्बलवान्समपद्यत}


\twolineshloka
{ज्ञात्वा बलस्थं त्रिदशाधिपं तुननाद वृत्रो महतो निनादान्}
{तस् प्रणादेन धरा दिशश्चखं द्यौर्नगाश्चापि चचाल सर्वम्}


\twolineshloka
{ततो महेन्द्रः परमाभितप्तःश्रुत्वा रवं घोररूपं महान्तम्}
{भये निमग्नस्त्वरितो मुमोचवज्रं महत्तस्य वधाय राजन्}


\twolineshloka
{स चक्रवज्राभिहतः पपातमहासुरः काञ्चनमाल्यधारी}
{यथा महाशैलवरः पुरस्ता-त्स मन्दरो विष्णुकराद्विमुक्तः}


\twolineshloka
{तस्मिन्हते दैत्यवरे भयार्तःशक्रः प्रदुद्राव सरः प्रवेष्टुम्}
{वज्रं न मेने स्वकराद्विमुक्तंवृत्रं भयाच्चापि हतं न मेने}


\twolineshloka
{सर्वे च देवा मुदिताः प्रहृष्टामहर्षयश्चन्द्रमभिष्टुवन्तः}
{`वृत्रं हतं संददृशुः पृथिव्यांवज्राहतं शैलमिवावदीर्णम्}


\twolineshloka
{सर्वांश्च दैत्यांस्त्वरिताः समेत्यजघ्नुः सुरा वृत्रवधाभितप्तान्}
{ते वध्यमानास्त्रिदशैः समेतैःसमुद्रमेवाविविशुर्भयार्ताः}


\twolineshloka
{प्रविश्य चैवोदधिमप्रमेयंझषाकुलं नक्रसमाकुलं च}
{तदा स्म मन्त्रं सहिताः प्रचक्रु-स्त्रैलोक्यनाशार्थमभिप्रयत्नात्}


\twolineshloka
{तत्रस्म केचिन्मतिनिश्चयज्ञा-स्तांस्तानुपयानुपवर्णयन्ति}
{तेषां तु तत्रक्रमकालयोगा-द्धोरा मतिश्चिन्तयतां बभूव}


\twolineshloka
{ये सन्ति विद्यातपसोपपन्ना-स्तेषां विनाशः प्रथमं तु कार्यः}
{लोका हि सर्वे तपसा ध्रियन्तेतस्मात्त्वरध्वं तपसः क्षयाय}


\twolineshloka
{येसन्ति केचिच्च वसुंधरायांतपस्विनो धर्मविदश् तज्ज्ञाः}
{तेषां वधः क्रियतां क्षिप्रमेवतेषु प्रनष्टेषु जगत्प्रनष्टम्}


\twolineshloka
{एवं हि सर्वे गतबुद्धिभावाजगद्विनाशे परमप्रहृष्टाः}
{दुर्गं समाश्रित्य महोर्मिमन्तंरत्नाकरं वरुणस्यालयं स्म}


\chapter{अध्यायः १०१}
\twolineshloka
{लोमश उवाच}
{}


\twolineshloka
{समुद्रं ते समाश्रित्य वरुणं निधमम्भसः}
{कालेयाः संप्रवर्तन्त त्रैलोक्यस्य विनाशने}


\twolineshloka
{ते रात्रौ समभिक्रुद्धा भक्षयन्ति सदा मुनीन्}
{आश्रमेषु च ये सन्ति पुण्येष्वायतनेषु च}


\twolineshloka
{वसिष्ठस्याश्रमे विप्रा भक्षितास्तैर्दुरात्मभिः}
{अशीतिः शतमष्टौ च नव चान्ये तपस्विनः}


\twolineshloka
{च्यवनस्याश्रमं गत्वा पुण्यं द्विजनिषेवितम्}
{फलमूलाशनानां हि मुनीनां भक्षितं शतम्}


\twolineshloka
{एवं रात्रौ स्म कुर्वन्ति विविशुश्चार्णवं दिवा}
{`कालेयास्ते दुरात्मानो भक्षयन्तस्तपोधनान्'}


\twolineshloka
{भरद्वाजाश्रमे चैव नियता ब्र्हमचारिणः}
{वाय्वाहाराम्बुभक्षाश्च विंशतिः संनिषूदिताः}


\threelineshloka
{एवं क्रमेण सर्वांस्तानाश्रमान्दानवास्तदा}
{निशायां परिबाधन्ते समुद्राम्बुबलाश्रयात्}
{कालोपसृष्टाः कालेया घ्नन्तो द्विजगणान्बहून्}


\twolineshloka
{न चैनानन्वबुध्यन्त मनुजा मनुजोत्तम}
{एवं प्रवृत्तान्दैत्यांस्तांस्तापसेषु तपस्विषु}


% Check verse!
`क्षयाय जगतः क्रूराः पर्यटन्ति स्म मेदिनीम्
\twolineshloka
{प्रभाते समदृश्यन्त नियताहारकर्शिताः}
{महीतलस्था मुनयः शरीरैर्गतजीवितैः}


\twolineshloka
{क्षीणमांसैर्विरुधिरैर्विमज्जान्त्रैर्विसन्धिभिः}
{आकीर्णैराचिता भूमिः शङ्खानामिव राशिभिः}


\twolineshloka
{लशैर्विप्रविद्धैश्च स्रुवैर्भग्नैस्तथैव च}
{विकीर्णैरग्निहोत्रैश्च भूर्बभूव समावृता}


\twolineshloka
{निःस्वाध्यायवपट्कारं नष्टयज्ञोत्सवक्रियम्}
{जगदासीन्निरुत्साहं कालेयभयपीडितम्}


\twolineshloka
{एवं संक्षीयमाणाश्च मानवा मनुजेश्वर}
{आत्मत्राणपपा भीताः प्राद्रवन्त दिशो भयात्}


\twolineshloka
{केचिद्गुहाः प्रविविशुर्निर्भरांश्चापरे श्रिताः}
{अपरे मरणोद्विग्ना भयात्प्राणान्समुत्सृजन्}


\twolineshloka
{केचिदत्रमहेष्वासाः शूराः परमहर्षिताः}
{मार्गमाणआः परं यत्नं दानवानां प्रचक्रिरे}


\twolineshloka
{न चैतानधिजग्मुस्ते समुद्रं समुपाश्रितान्}
{श्रमं जग्मुश्च परममाजग्मुः क्षयमेव च}


\twolineshloka
{जगत्युपशमं याते नष्टयज्ञोत्सवक्रिये}
{आजग्मुः परमामार्तिं त्रिदशा मनुजेश्वर}


% Check verse!
समेत्य समहेन्द्राश्च भयान्मन्त्रं प्रचक्रिरे
\threelineshloka
{शरण्यं शरणं देवं नारायणमजं विभुम्}
{तेऽभिगम्य नमस्कृत्य वैकुण्ठमपराजितम्}
{ततो देवाः समस्तास्ते तदोचुर्मधुसूदनम्}


\threelineshloka
{त्वंनः स्रष्टा च भर्ता च हर्ता च जगतः प्रभो}
{त्वया सृष्टमिदं विश्वं यच्चेङ्गं यच्च नेङ्गति}
{`त्वय्येव पुण्डरीकाक्ष पुनस्तत्प्रविलीयते'}


\twolineshloka
{त्वया भूमिः पुरा नष्टा समुद्रात्पुष्करेक्षण}
{वाराहं वपुराश्रित्यजगदर्थे समुद्धृता}


\twolineshloka
{आदिदैत्यो महावीर्यो हिरण्यकशिपुः पुरा}
{नारसिंहं वपुः कृत्वा सूदितः पुरुषोत्तम}


\twolineshloka
{अवध्यः सर्वभूतानां बलिश्चापि महासुरः}
{वामनं वपुराश्रित्य त्रैलोक्याद्धंशितस्त्वया}


\twolineshloka
{असुरश्च महेष्वासो जम्भ इत्यभिविश्रुतः}
{यज्ञक्षोभकरः क्रूरस्त्वयैव विनिपातितः}


\twolineshloka
{एवमादीनि कर्माणि येषां संख्या न विद्यते}
{अस्माकं भयभीतानां त्वं गतिर्मधुसूदन}


\threelineshloka
{तस्मात्त्वां देवदेवेश लोकार्थं ज्ञापयामहे}
{रक्ष लोकांश्च देवांश्च शक्रं च महतो भयात्}
{`शरणागतसंत्राणे त्वमेकोऽसि दृढव्रतः'}


\chapter{अध्यायः १०२}
\twolineshloka
{देवा ऊचुः}
{}


\twolineshloka
{तव प्रसादाद्वर्धन्ते प्रजाः सर्वाश्चतुर्विधाः}
{ता भाविता भावयनति हव्यकव्यैर्दिवौकसः}


\twolineshloka
{लोका ह्येवं विवर्धन्ते ह्यन्योन्यं समुपाश्रिताः}
{त्वत्प्रसादान्निरुद्विग्नास्त्वयैव परिरक्षिताः}


\twolineshloka
{इदं च समनुप्राप्तं लोकानां भयमुत्तमम्}
{न च जानीम केनेति रात्रौ वध्यन्ति ब्राह्मणाः}


\twolineshloka
{क्षीणेषु च ब्राह्मणेषु पृथिवी क्षयमेष्यति}
{तत पृथिव्यां क्षीणायां त्रिदिवं क्षयमेष्यति}


\threelineshloka
{त्वत्प्रसादान्महाबाहो लोकाः सर्वे जगत्पते}
{विनाशं नाधिगच्छेयुस्त्वया वै परिरक्षिताः ॥विष्णुरुवाच}
{}


\twolineshloka
{विदितं मे सुराः सर्वं प्रजानां क्षयकारणम्}
{भवतां चापि वक्ष्यामि शृणुध्वं विगतज्वराः}


\twolineshloka
{कालेया इति विख्याता गणाः परमदारुणाः}
{तैश्च वृत्रं समाश्रित्य जगत्सर्वं प्रमाथितम्}


\twolineshloka
{ते वृत्रं निहतं दृष्ट्वा सहस्राक्षेण धीमता}
{जीवितं परिरक्षन्तः प्रविष्टा वरुणालयम्}


\twolineshloka
{ते प्रविश्योदधिं घोरं नक्रग्राहसमाकुलम्}
{उत्सादार्थं लोकानां रात्रौ घ्नन्ति ऋषीनिह}


\twolineshloka
{न तु शक्याः क्षयं नेतुं समुद्राश्रयिणो हि ते}
{समुद्रस्य क्षये बुद्धिर्भवद्भिः संप्रधार्यताम्}


\twolineshloka
{`एतच्छ्रुत्वा वचो देवा विष्णुना समुदीरितम्}
{विष्णुमेव पुरस्कृत्यब्रह्माणं समुपस्थिताः}


\twolineshloka
{ततस्ते प्रणता भूत्वा तमेवार्धं न्यवेदयन्}
{सर्वलोकविनाशार्थं कालेया कृतनिश्चयाः}


\twolineshloka
{एषां तद्वचनं श्रुत्वा पद्मयोनिः सनातनः}
{उवाच परमप्रीतस्त्रिदशानर्थवद्वचः}


\twolineshloka
{विदितं मे सुराः सर्वे दानवानां विवेष्टितम्}
{मनुष्यादेश्च निधनं कालेयैः कालचोदितैः}


\twolineshloka
{क्षयस्तेपामनुप्राप्तः कालेनोपहताश्च ये}
{उपायं संप्रवक्ष्यामि समुद्रस्य विशोषणे}


\twolineshloka
{अगस्त्य इतिविग्व्यातो वारुणिः सुसमाहितः}
{तमुपागम्य सहिता इममर्थं प्रयाचत}


\twolineshloka
{स हि शक्तो महातेजाः क्षणात्पातुं महोदधिम्'}
{अगस्त्येन विना को हि शक्तोऽन्योऽर्णवशोपणे}


\twolineshloka
{अन्यथा हि न शक्यास्ते विना सागरशोपणम्}
{समुद्रे च क्षयं नीते कालेयान्निहनिष्यथ}


\twolineshloka
{एवं श्रुत्वा वचो देवा ब्रह्मणः परमेष्ठिनः}
{परमेष्ठिनमाज्ञाप्यअगस्त्यस्याश्रमं ययुः}


\twolineshloka
{तत्रापश्यन्महात्मानं वारुणिं दीप्ततेजसम्}
{उपास्यमानमृपिभिर्देवैरिव पितामहम्}


\threelineshloka
{तेऽभिगम्य महात्मानं मैत्रावरुणिमच्युतम्}
{आश्रमश्थं तपोराशिं कर्मभिः स्वैस्तु तुष्टुवुः ॥देवा ऊचुः}
{}


\twolineshloka
{नहुपेणाभितप्तानां त्वं लोकानां गतिः पुरा}
{भ्रंशितश्च सुरैश्वर्याल्लोकार्थं लोककण्टकः}


\twolineshloka
{क्रोधात्प्रवृद्धस्तरणं भास्करस्य नभोगतः}
{वचस्तवानतिक्रामन्विन्ध्यः शैलो न वर्धते}


\twolineshloka
{तमसा चावृतेलोके मृत्युनाऽभ्यर्दिताः प्रजाः}
{त्वामेव नाथमासाद्य निर्वृतिं परमां गताः}


\twolineshloka
{अस्माकं भयभीतानां नित्यशो भगवान्गतिः}
{ततस्त्वार्ताः प्रयाचामो वरं त्वां वरदो ह्यसि}


\chapter{अध्यायः १०३}
\twolineshloka
{युधिष्ठिर उवाच}
{}


\threelineshloka
{किमर्थं सहसा विन्ध्यः प्रवृद्धः क्रोधमूर्च्छितः}
{एतदिच्छाम्यहं श्रोतुं विस्तरेण महामुने ॥लोमश उवाच}
{}


\twolineshloka
{अद्रिराजं महाघोरं रमेरुं कनकपर्वतम्}
{उदयास्तमने भानुः प्रदक्षिणमवर्तत}


\twolineshloka
{तं तु दृष्ट्वा तथा विन्ध्य शैलः सूर्यमथाब्रवीत्}
{यथा हि मेरुर्भवता नित्यशः परिगम्यते}


\twolineshloka
{प्रदक्षिणश्च क्रियते मामेवं करु भास्कर}
{एवमुक्तस्ततः सूर्यः शैलेन्द्रं प्रत्यभापत}


\twolineshloka
{नाहमात्मेच्छया शैल करोम्येनं प्रदक्षिणम्}
{एष मार्गः प्रदिष्टो मे येनेदं निर्मितं जगत्}


\twolineshloka
{एवमुक्तस्ततः क्रोधात्प्रवृद्धः सहसाऽचलः}
{सूर्याचन्द्रमसोर्मार्गं रोद्धुमिच्छन्परंतप}


\threelineshloka
{ततो देवाः सहिताः सर्व एव}
{सेन्द्राः समागम्य महाद्रिराजम्}
{निवारयामासुरुपायतस्तंन च स्म तेषां वचनं चकार}


\twolineshloka
{अथाभिजग्मुर्मुनिमाश्रमस्थंतपस्विनं धर्मभृतां वरिष्ठम्}
{अगस्त्यमत्यद्भुतवीर्यदीप्तंतं चार्थमूचुः सहिताः सुरास्ते}


\twolineshloka
{सूर्याचन्दर्मसोर्मार्गं नक्षत्राणां गतिं तथा}
{शैलराजो वृणोत्येप विन्ध्यः क्रोधवशानुगः}


\twolineshloka
{तं निवारयितुं शक्तो नान्यः कश्चिद्द्विजोत्तम}
{ऋते त्वां हि महाभाग तस्मादेनं निवारय}


\twolineshloka
{तच्छ्रुत्वा वचनं विप्रः सुराणां शैलमभ्यगात्}
{सोभिगम्याब्रवीद्विन्ध्यं सदारः समुपस्थितः}


\twolineshloka
{मार्गमिच्छाम्यहं दत्तं भवता पर्वतोत्तम}
{दक्षिणामभिगन्तास्मि दिशे कार्येण केनचित्}


\twolineshloka
{यावदागमनं मह्यं तावत्त्वं प्रतिपालय}
{निवृत्ते मयि शैलेन्द्र ततो वर्धस्व कामतः}


\twolineshloka
{एवं स समयं कृत्वाविन्ध्येनामित्रकर्शन}
{अद्यापि दक्षिणाद्देशाद्वारुणिर्न निवर्तते}


\twolineshloka
{विन्ध्योऽपितद्भयाद्राजन्कुञ्चिताङ्गो न वर्धते}
{अगस्त्यस्य प्रभावेण यन्मां त्वं परिपृच्छसि}


\twolineshloka
{कालेयास्तु यथा राजन्सुरैः सर्वैर्निषूदिताः}
{अगस्त्याद्वरमासाद्य तन्मे निगदतः शृणु}


\twolineshloka
{त्रिदशानां वचः श्रुत्वा मैत्रावरुणिरब्रवीत्}
{किमर्थमभियाता स्थ वरं मत्तः कमिच्छथ}


\twolineshloka
{एवमुक्तास्ततस्तेन देवता मुनिमब्रुवन्}
{`सर्वे प्राञ्जलयो भूत्वा पुरन्दरपुरोगमाः'}


\twolineshloka
{एवं त्वयोच्छाम कृतं हि कार्यंमहार्णवं पीयमानं महात्मन्}
{ततो वधिष्याम सहानुबन्धा-न्कालोपसृष्टान्सुरविद्विषस्तान्}


\twolineshloka
{त्रिदशानां वचः श्रुत्वा तथेति मुनिरब्रवीत्}
{करिष्ये भवतां कामं लोकानां च महत्सुखम्}


\twolineshloka
{एवमुक्त्वा ततोऽगच्छत्समुद्रं सरितांपतिम्}
{ऋषिभिश्च तपःसिद्धै सार्धं देवैश्च सुव्रत}


\twolineshloka
{मनुष्योरगगन्धर्वयक्षकिंपुरुषास्तथा}
{अनुजग्मुर्महात्मानं द्रष्टुकानास्तदद्भुतम्}


\twolineshloka
{ततोऽभ्यगच्छन्सहिताः समुद्रं भीमनिःखनम्}
{नृत्यन्तमिव चोर्मीभिर्वल्गन्तमिव वायुना}


\twolineshloka
{हसन्तमिव फेनौघैः स्खलन्तं कन्दरेषु च}
{तानाग्राहसमाकीर्णं नानाद्विजगणान्वितम्}


\twolineshloka
{अगस्त्यसहिता देवाः सगन्धर्वमहारगाः}
{ऋषयश्च महाभागाः समासेदुर्महोदधिम्}


\chapter{अध्यायः १०४}
\twolineshloka
{लोमश उवाच}
{}


\twolineshloka
{समुद्रं स समासाद्य वारुणिर्भगवानृषिः}
{उवाच सहितान्देवानृषींश्चैव समागतान्}


\twolineshloka
{एष लोकहितार्थं वै पिबामि वरुणालयम्}
{भवद्भिर्यदनुष्ठेयं तच्छीघ्रं संविधीयताम्}


\twolineshloka
{एतावदुक्त्वा वचनं मैत्रावरुणिरच्युतः}
{समुद्रमपिबत्क्रुद्धः सर्वलोकस्य पश्यतः}


\twolineshloka
{पीयमानं समुद्रं तं दृष्ट्वा सेन्द्रास्तदाऽमराः}
{विस्मयं परमं जग्मुः स्तुतिभिश्चाप्यपूजयन्}


\twolineshloka
{त्वं नस्त्राता विधाता च लोकानां लोकभावन}
{त्वत्प्रसादात्समुच्छेदं न गच्छेत्सामरं जगत्}


\twolineshloka
{स पूज्यमानस्त्रिदशैर्महात्मागन्धर्वतूर्येषु नदत्सु सर्वशः}
{दिव्यैश्च पुष्पैरवकीर्यमाणोमहार्णवं निःसलिलं चकार}


\twolineshloka
{दृष्ट्वा कृतंनिःसलिलं महार्णवंसुराः समस्ताः परमप्रहृष्टाः}
{3-104-7cप्रगृह्यदिव्यानिवरायुधानितान्दानवाञ्जघ्रुरदीनसत्वाः}


\twolineshloka
{ते वध्यमानास्त्रिदशैर्महात्मभि-र्महाबलैर्वेगिभिरुन्नदद्भिः}
{न सेहिरेवेगवतां महात्मनांवेगं तदा धारयितुं दिवौकसाम्}


\twolineshloka
{ते वध्यमानास्त्रिदशैर्दानवा भीमनिःस्वनाः}
{रचक्रु सुतुमुलं युद्धं मुहूर्तमिव भारत}


\twolineshloka
{ते पूर्वं तपसा दग्धा मुनिभिर्भावितात्मभिः}
{यतमानाः परं शक्त्या त्रिदशैर्विनिषूदिताः}


\twolineshloka
{तेहेमनिष्काभरणाः कुण्डलाङ्गदधारिणः}
{निहता बह्वशोभन्त पुष्पिता इव किंशुकाः}


\twolineshloka
{हतशेषास्ततः केचित्कालेया मनुजोत्तम}
{विदार्य वसुधां देवीं पातालतलमास्थिताः}


\twolineshloka
{निहतान्दानवान्दृष्ट्वा त्रिदशा मुनिपुह्गवम्}
{तुष्टुवुर्विविधैर्वाक्यैरिदं वचनमब्रुवन्}


\twolineshloka
{त्वत्प्रसादान्महाबाहो लोकैः प्राप्तं महत्सुखम्}
{त्वत्तेजसा च निहताः कालेयाः क्रूरविक्रमाः}


\twolineshloka
{पूरस्व महाबाहो समुद्रं लोकभावन}
{यत्त्वया सलिलं पीतं तदस्मिन्पुनरुत्सृज}


\twolineshloka
{एवमुक्तः प्रत्युवाच भगवान्मुनिपुङ्गवः}
{`तांस्तदा सहितान्देवानगस्त्यःसपुरन्दरान्'}


\twolineshloka
{जीर्णं तद्धि मया तोयमुपायोऽन्यः प्रचिन्त्यताम्}
{पूरणार्थं समुद्रस्य भवद्भिर्यत्नमास्थितैः}


\twolineshloka
{केतच्छ्रुत्वा तु वचनं महर्षेर्भावितात्मनः}
{विस्मिताश्च विषण्णाश्च बभूवुः सहिताः सुराः}


\twolineshloka
{परस्परमनुज्ञाप्यप्रणम्य मुनिपुङ्गवम्}
{प्रजाः सर्वा महाराज विप्रजग्मुर्यथागतम्}


\twolineshloka
{त्रिदशा विष्णुना सार्धमुपजग्मुः पितामहम्}
{पूरणार्थं समुद्रस्य मन्त्रयित्वा पुनः पुनः}


\twolineshloka
{ते धातारमुपागम्य त्रिदशाः सह विष्णुना}
{ऊचुः प्राञ्जलयः सर्वे सागरस्याभिपूरणम्}


\chapter{अध्यायः १०५}
\twolineshloka
{लोमश उवाच}
{}


\twolineshloka
{तानुवाच समेतांस्तु ब्रह्मा लोकपितामहः}
{`निर्ह्रादिन्या गिरा राजन्देवानाश्वासयंस्तदा'}


\twolineshloka
{गच्छध्वं विबुधाः सर्वे यथाकामं यथेप्सितम्}
{महता कालयोगेन प्रकृतिं यास्यतेऽर्णवः}


\twolineshloka
{ज्ञातींश्च कारणं कृत्वा महाराजो भगीथः}
{`पूरयिष्यति तोयौघैः समुद्रं निधिमम्भसाम्'}


\threelineshloka
{पितामहवचः श्रुत्वा सर्वे विबुधसत्तमाः}
{कालयोगं प्रतीक्षन्तो जग्मुश्चापि यथागतम् ॥युधिष्ठिर उवाच}
{}


\twolineshloka
{कथं वै ज्ञातयोब्रह्मन्कारणं चात्र किं मुने}
{कथं समुद्रः पूर्णश्च भगीरथपरिश्रमात्}


\threelineshloka
{एतदिच्छाम्यहं श्रोतुं विस्तरेण तपोधन}
{कथ्यमानं त्वया विप्र राज्ञां चरितमुत्तमम् ॥वैशंपायन उवाच}
{}


\threelineshloka
{एवमुक्तस्तु विप्रेन्द्रो धर्मराज्ञा महात्मना}
{कथयामास माहात्म्यं सगरस्य महात्मनः ॥लोमश उवाच}
{}


\twolineshloka
{इक्ष्वाकूणां कुले जातः सगरो नाम पार्थिवः}
{रूपसत्वबलोपेतः स चापुत्रः प्रतापवान्}


\twolineshloka
{स हैहयान्समुत्साद्य तालजङ्घांश्च भारत}
{वशे च कृत्वा राजन्यान्स्वराज्यमनुशिष्टवान्}


\twolineshloka
{तस्य भार्ये त्वभवतां रूपयौवनदर्पिते}
{वैदर्भी भरतश्रेष्ठ शैव्या च भरतर्षभ}


\twolineshloka
{सपुत्रकामो नृपतिस्तप्यते स्म महत्तपः}
{पत्नीभ्यां सह राजेन्द्र कैलासं गिरिमाश्रितः}


\twolineshloka
{स तप्यमानः सुमहत्तपोयोगसमन्वितः}
{आससाद महात्मानं त्र्यक्षं त्रिपुरमर्दनम्}


\twolineshloka
{शंकरं भवमीशानं पिनाकिं शूलपाणिनम्}
{त्र्यम्बकं शिवमुग्रेशं बहुरूपमुमापतिम्}


\twolineshloka
{स तं दृष्ट्वैव वरदं पत्नीभ्यां सहितो नृपः}
{प्रणिपत्य महाबाहुः पुत्रार्थे समयाचत}


\twolineshloka
{तं प्रीतिमान्हरः प्राह सभार्यं नृपसत्तमम्}
{यस्मिन्वृतो मुहूर्तेऽहं त्वयेह नृपते वरम्}


\twolineshloka
{षष्टिः पुत्रसहस्राणि शूराः परमदर्पिताः}
{एकस्यां संभविष्यन्ति पत्न्यां नरवरोत्तम}


\twolineshloka
{ते चैवसर्वे सहिताः क्षयं यास्यन्ति पार्थिव}
{एको वंशधरः शूर एकस्यां संभविष्यति}


\twolineshloka
{एवमुक्त्वा तु तं रुद्रस्तत्रैवान्तरधीयत}
{स चापि सगरो राजा जगाम स्वं निवेशनम्}


\twolineshloka
{पत्नीभ्यां सहितस्तत्र होऽतिहृष्टमनास्तदा}
{`कालं शंभुवरप्राप्तं प्रतीक्षन्सगरोऽनयत्'}


\twolineshloka
{तस्य तेमनुजश्रेष्ठ भार्ये कललोचने}
{वैदर्भी चैव शैब्या च गर्भिण्यौ संबभूवतुः}


\twolineshloka
{ततः कालेन वैदर्भी गर्भालाबुं व्यजायत}
{शैव्या च सुषुवे पुत्रं कुमारं देवरूपिणम्}


\twolineshloka
{तदाऽलाबुं समुत्स्रष्टुं मनश्चक्रे स पार्थिवः}
{अथान्तरिक्षाच्छुश्राव वाचं गंभीरनिःस्वनाम्}


\twolineshloka
{राजन्मा साहसंकार्षीः पुत्रान्न त्यक्तुमर्हसि}
{अलाबुमध्यान्निष्कृष्य वीजं यत्नेन गोप्यताम्}


\twolineshloka
{सोपस्वेदेषु पात्रेषु घृतपूर्णेषु भागशः}
{ततः पुत्रसहस्राणि षष्टिं प्राप्स्यसि पार्थिव}


\twolineshloka
{महादेवेन दिष्टं ते पुत्रजन्म नराधिप}
{अनेन क्रमयोगेन मा ते बुद्धिरतोऽन्यथा}


\chapter{अध्यायः १०६}
\twolineshloka
{लोमश उवाच}
{}


\twolineshloka
{एतच्छ्रुत्वान्तरिक्षाच्च स राजा राजसत्तमः}
{यथोक्तं तच्चराकाथ श्रद्दधद्भरतर्षभ}


\twolineshloka
{[एकैकशस्ततः कृत्वा बीजं बीजं नराधिपः}
{घृतपूर्णेषु कुम्भेषु तान्भागान्विदधे ततः}


\twolineshloka
{धात्रीश्चैकैकशः प्रादात्पुत्ररक्षणतत्परः}
{ततः कालेन महता समुत्तस्थुर्महाबलाः ॥]}


\twolineshloka
{षष्टिः पुत्रसहस्रणि तस्याप्रतिमतेजसः}
{रुद्रप्रसादाद्राजर्षेः समजायन्त पार्थिव}


\twolineshloka
{ते घोराः क्रूरकर्माण आकाशपरिसर्पिणः}
{बहुत्वाच्चावजानन्तः सर्वाल्लोँकान्सहामरान्}


\twolineshloka
{त्रिदशांश्चाप्यबाधन्त तथा गन्धर्वराक्षसान्}
{सर्वाणि चैव भूतानि शूरा समरशालिनः}


\twolineshloka
{वध्यमानास्तदा लोकाः सागरैर्मन्दबुद्धिभिः}
{ब्रह्माणं शरणं जग्मुः सहिताः सर्वदैवतैः}


\twolineshloka
{तानुवाच महाभागः सर्वलोकपितामहः}
{गच्छध्वं त्रिदशाः सर्वे लोकैः सार्धं यथागतम्}


\twolineshloka
{नातिदीर्घेण कालेन सागराणां क्षयो महान्}
{विष्यति महाघोरः स्वकृतैः कर्मभिः सुराः}


\twolineshloka
{एवमुक्तास्तु ते देवा लोकाश् मनुजेश्वर}
{पितामहमनुज्ञाप्य विप्रजग्मुर्यथागतम्}


\twolineshloka
{ततः काले बहुतिथे व्यतीते भरतर्षभ}
{दीक्षितः सगरो राजा हयमेधेन वीर्यवान्}


\twolineshloka
{तस्याश्वो व्यचरद्भूमिं पुत्रैः सुपरिरक्षितः}
{`सर्वैरेव महोत्साहैः स्वच्छन्दप्रचरो नृप'}


\twolineshloka
{समुद्रं स समासाद्य निस्तोयं भीमदर्शनम्}
{रक्ष्यमाणः प्रयत्नेन तत्रैवान्तरधीयत}


\threelineshloka
{ततस्ते सागरास्तात हृतं मत्वा हयोत्तमम्}
{आगम्य पितुराचख्युरदृश्यं तुरगं हृतम्}
{तेनोक्ता दिक्षु सर्वासु पुत्रा मार्गत वाजिनम्}


\twolineshloka
{ततस्ते पितुराज्ञाय दिक्षु सर्वासु तं हयम्}
{अमार्गन्त महाराज सर्वंच पृथिवीतलम्}


\twolineshloka
{ततस्ते सागराः सर्वे समुपेत्य परस्परम्}
{नाध्यगच्छन्त तुरगमश्वहर्तारमेव च}


% Check verse!
आगम्य पितरं चोचुस्ततः प्राञ्जलयोऽग्रतः
\twolineshloka
{ससमुद्रवनद्वीपा सनदीनदकन्दरा}
{सपर्वतवनोद्देशा निखिलेन मही नृप}


\twolineshloka
{अस्माभिर्विचिता राजञ्शासनात्तव पार्थिव}
{न चाश्वमधिगच्छामो नाश्वहर्तारमेव च}


\twolineshloka
{श्रुत्वा तु वचनं तेषां स राजा क्रोधमूर्चितः}
{उवाच वचनं सर्वांस्तदा दैववशान्नृप}


\twolineshloka
{अनागमाय गच्छध्वं भूयो मार्गत वाजिनम्}
{याज्ञीयं तं विनाह्यश्वं नागन्तव्यं हि पुत्रकाः}


\twolineshloka
{प्रतिगृह्य तु संदेशं पितुस्ते सगरात्मजाः}
{भूय एव महीं कृत्स्नां विचेतुमुपचक्रमुः}


\twolineshloka
{अथापश्यन्त ते वीराः पृथिवीमवदारिताम्}
{`समुद्रे पृथिवीपाल पदं मार्गं च वाजिनः'}


\twolineshloka
{समासाद्य बिलं तच्चाप्यखनन्सगरात्मजाः}
{कुद्दालैर्मुसलैश्चैव समुद्रं यत्नमास्थिताः}


\twolineshloka
{स खन्यमानः सहितैः सागरैर्वरुणालयः}
{अगच्छत्परमामार्तिं दीर्यमाणः समन्ततः}


\twolineshloka
{असुरोरगरक्षांसि सत्वानि विविधानि च}
{आर्तनादमकुर्वन्त वध्यमानानि सागरैः}


\twolineshloka
{छिन्नशीर्षा विदेहाश्च भिन्नजान्वस्थिमस्तकाः}
{प्राणिनः समदृश्यन्त शतशोथ सहस्रशः}


\twolineshloka
{एवं हि खनतां तेषां समुद्रं वरुणालयम्}
{व्यतीतः सुमहान्कालो न चाश्वः समदृश्यत}


\twolineshloka
{ततः पूर्वोत्तरे देशे समुद्रस्य महीपते}
{विदार्य पातालमथ संक्रुद्धाः सगरात्मजाः}


\twolineshloka
{अपश्यन्त हयं यत्र विचरन्तं महीतले}
{कपिलं च महात्मानं तेजोराशिमनुत्तमम्}


\twolineshloka
{तेजसा दीप्यमानं तु ज्वालाभिरिव पावकम्}
{`दृष्ट्वा हि विस्मिताः सर्वे बभूवुः सगरात्मजाः'}


\chapter{अध्यायः १०७}
\twolineshloka
{लोमश उवाच}
{}


\twolineshloka
{*तेतं दृष्ट्वा हयं राजन्संप्रहृष्टतनूरुहाः}
{अनादृत् महात्मानं कपिलं कालचोदिताः}


\twolineshloka
{संफुद्धाः संप्रधावन्त वाजिग्रहणकाङ्क्षिणः}
{ततः क्रुद्धो महाराज कपिलो मुनिसत्तमः}


\twolineshloka
{वासुदेवेति यं प्राहुः कपिलं मुनिपुङ्गवम्}
{स चक्षुर्विकतं कृत्वा तेजस्तेषु समुत्सृजन्}


\twolineshloka
{ददाह सुमहातेजा मन्दबुद्धीन्स सागरान्}
{षष्टिं रतानि सहस्राणि युगपन्मुनिसत्तमः}


\twolineshloka
{तान्दृष्ट्वा भस्मसाद्भूतान्नारदः सुमहातपाः}
{सगरान्तिकमागत्य तच्च तस्मै न्यवेदयत्}


\twolineshloka
{स तच्छ्रुत्वा वचो घोरं राजा मुनिमुखोद्गतम्}
{महूर्तं विमना भूत्वा स्थाणोर्वाक्यमचिन्तयत्}


\twolineshloka
{`स पुत्रनिधनोत्थेन दुःखेन समभिप्लुतः}
{आत्मानमात्मनाऽऽश्वास्य हयमेवान्वचिन्तयत्'}


\twolineshloka
{अशुमन्तं समाहूय असमञ्जसुतं तदा}
{पौत्रं भरतशार्दूल इदं वचनमब्रवीत्}


\twolineshloka
{षष्टिस्तात सहस्राणि पुत्राणाममितौजसाम्}
{कापिलं तेज आसाद्य मत्कृतेनिधनं गताः}


\threelineshloka
{तव चापि पिता तात परित्यक्तो मयाऽनघ}
{धर्मं संरक्षमाणेन पौराणां हितमिच्छता ॥युधिष्ठिर उवाच}
{}


\threelineshloka
{किमर्थं राजशार्दूलः सगरः पुत्रमात्मजम्}
{त्यक्तवान्दुस्त्यजं वीरं तन्मे ब्रूहि तपोधन ॥लोमश उवाच}
{}


\twolineshloka
{असमञ्ज इति ख्यातः सगरस्य सुतो ह्यभूत्}
{यं शैव्या जनयामास पौराणां स हि दारकान्}


\twolineshloka
{`क्रीडतः सहसाऽऽसाद्य तत्रतत्र महीपते'}
{चूडासु क्रोशतो गृह्यनद्यां चिक्षेप दुर्बलान्}


\twolineshloka
{ततः पौराः समाजग्मुर्भयशोकपरिप्लुताः}
{सगरं चाभ्यभाषन्त सर्वे प्राञ्जलयः स्थिताः}


\twolineshloka
{त्वं नस्त्राता महाराज परचक्रादिभिर्भयात्}
{असमञ्जभयाद्धोरात्ततो नस्त्रातुनर्हसि}


\twolineshloka
{पौराणां वचनं श्रुत्वा घोरं नृपतिसत्तमः}
{मुहूर्तं विमना भूत्वासचिवानिदमब्रवीत्}


\twolineshloka
{असमञ्ज पुरादद्य सुतो मे विप्रवास्यताम्}
{यदि वो मत्प्रियं कार्यमेतच्छीध्रं विधीयताम्}


\twolineshloka
{एवमुक्ता नरेन्द्रेण सचिवास्ते नराधिप}
{यथोक्तं त्वरिताश्चक्रुर्यथाऽऽज्ञापितवान्नृपः}


\twolineshloka
{एतत्ते सर्वमाख्यातं यथा पुत्रो महात्मना}
{पौराणां हितकामेन सगरेण विवासितः}


\threelineshloka
{अंशुमांस्तु महेष्वासो यदुक्तः सगरेण हि}
{तत्ते सर्वंप्रवक्ष्यामि कीर्त्यमानं निबोध मे ॥सगर उवाच}
{}


\twolineshloka
{पितुश् तेऽहंत्यागेन पुत्राणां निधनेन च}
{अलाभेन तथाऽश्वस् परितप्यामि पुत्रक}


\threelineshloka
{तस्माद्दुःखाभिसंतप्तं यज्ञविघ्नाच्च मोहितम्}
{हयस्यानयनात्पौत्र नरकान्मां समुद्धर ॥लोमश उवाच}
{}


\twolineshloka
{अंशुमानेवमुक्तस्तु सगरण महात्मना}
{जगाम दुःखात्तं देशं यत्रवै दारिता मही}


\twolineshloka
{स तु तेनैव मार्गेण समुद्रं प्रविवेश ह}
{अपश्यच्च महात्मानं कपिलं तुरगं च तम्}


\twolineshloka
{स दृष्ट्वा तेजसो राशिं पुराणमृपिसत्तमम्}
{प्रणम्य शिरसा भूमौ कार्यमस्मै न्यवेदयत्}


\twolineshloka
{ततः प्रीतो महाराज कपिलोंऽशुमतोऽभवत्}
{उवाच चैनं धर्मात्मा वरदोस्मीति भारत}


\twolineshloka
{स वव्रे तुरगं तत्र प्रथमं यज्ञकारणात्}
{द्वितीयमुदकं वव्रे पितॄणां पावनेच्छया}


\twolineshloka
{तमुवाच महातेजाः कपिलो मुनिपुङ्गवः}
{ददानि तव भद्रं ते यद्यत्प्रार्थयसेऽनघ}


\twolineshloka
{त्वयि क्षमा च धर्मश्च सत्यं चापि प्रतिष्ठितम्}
{त्वया कृतार्थः सगरः पुत्रवांश्च त्वया पिता}


\twolineshloka
{तव चैव प्रभावेन स्वर्गं यास्यन्ति सागराः}
{`शलभत्वं गता ये ते मम क्रोधहुताशने'}


\twolineshloka
{पौत्रश्च ते त्रिपथगां त्रिदिवादानयिष्यति}
{पावनार्थं सागराणां तोषयित्वा महेश्वरम्}


\twolineshloka
{हयं नयस्व भद्रं ते याज्ञिं नरपुङ्गव}
{यज्ञः समाप्यतां तात सगरस्य महात्मनः}


\twolineshloka
{अंशुमानेवमुक्तस्तु कपिलेन महात्मना}
{आजगाम हयं गृह्य यज्ञवाटं महात्मनः}


\twolineshloka
{सोभिवाद्य ततः पादौ सगरस्य महात्मनः}
{मूर्ध्नि तनाप्युपाघ्रातस्तस्मै सर्वंनय्वेदयत्}


\twolineshloka
{यथा दृष्टं श्रुतं चापि सागराणां क्षयं तथा}
{तं चास्मै हयमाचष्ट यज्ञवाटमुपागतम्}


\twolineshloka
{तच्छ्रुत्वा सगरो राजा पुत्रजं दुःखमत्यजत्}
{अंशुमन्तं च संपूज्यसमापयत तं क्रतुम्}


\twolineshloka
{समाप्तयज्ञः सगरो देवैः सर्वैः सभाजितः}
{पुत्रत्वे कल्पयामास समुद्रं वरुणालयम्}


\twolineshloka
{प्रशास्य सुचिरं कालं राज्यं राजीवलोचनः}
{पौत्रे भारं समवेश्य जगाम त्रिदिवं तदा}


\twolineshloka
{अंशुमानपि धर्मात्मा महीं सागरमेखलाम्}
{प्रशशास महाराज यथैवास् पितामहः}


\twolineshloka
{तस्य पुत्रः समभवद्दिलीपो नाम धर्मवित्}
{तस्मिन्राज्यं समाधाय अंशुमानपि संस्थितः}


\twolineshloka
{दिलीपस्तु ततः श्रुत्वा पितॄणां निधनं महत्}
{पर्यतप्यत दुःखेन तेषां गतिमचिन्तयत्}


\twolineshloka
{गङ्गावतरणए यत्नं सुमहच्चाकरोन्नृपः}
{न चावतारयामास चेष्टमानो यथाबलम्}


\twolineshloka
{तस् पुत्र समभवच्छ्रीमान्धर्मपराणः}
{भगीरथ इति ख्यातः सत्यवागनसूयकः}


% Check verse!
अभिषिच्य तु तं राज्येदिलीपो वनमाश्रितः
\twolineshloka
{तपः सिद्धिसमायोगात्स राजा भरतर्षभ}
{वनाज्जगाम त्रिदिवं कालयोगेन भारत}


\chapter{अध्यायः १०८}
\twolineshloka
{लोमश उवाच}
{}


\twolineshloka
{स तु राजा महेष्वासश्चक्रवर्ती महारथः}
{बभूव सर्वलोकस् मनोनयननन्दनः}


\twolineshloka
{स शुश्राव महाबाहुः कपिलेन महात्मना}
{पितॄणआं निधनं घोरमप्राप्तिं त्रिदिवस्य च}


\twolineshloka
{स राज्यं सचिवे न्यस्य हृदयेन विदूयता}
{जगाम हिमवत्पार्श्वं तपस्तप्तुं नरेश्वरः}


\twolineshloka
{आरिराधयिषुर्गङ्गां तपसा दग्धकिल्विषः}
{सोऽपश्यत नर्रेष्ठ हिमवन्तं नगोत्तमम्}


\twolineshloka
{शृङ्गैर्बहुविधाकारैर्धातुमद्भिरलंकृतम्}
{पवनालम्बिभिर्मेघैः परिपिक्तं समन्ततः}


\twolineshloka
{नदीकुञ्जनितम्बैश्च सोदकैरुपशोभितम्}
{गुहाकन्दरसंलीनसिंहव्याघ्रनिषेवितम्}


\twolineshloka
{शकुनैश्च विचित्राङ्गैः कूजद्भिर्विविधा गिरः}
{भृङ्गराजैस्तथा हंसैर्दात्यूहैर्जलकुक्कुटैः}


\twolineshloka
{मयूरैः शतपत्रैश्च जीवंजीवककोकिलैः}
{चकोरैरसितापाङ्गैस्तथा पुत्रप्रियैरपि}


\twolineshloka
{जलस्थानेषु रम्येषु पद्मिनीभिश्च संकुलम्}
{सारसानां च मधुरैर्व्याहृतैः समलंकृतम्}


\twolineshloka
{किन्नरैरप्सरोभिश्च निषेवितशिलातलम्}
{दिग्वारणविषाणाग्रैः समन्ताद्धृष्टपादपम्}


\twolineshloka
{विद्याधरानुचरितं नानारत्नसमाकुलम्}
{विषोल्वणैर्भुजङ्गैश्च दीप्ताजिह्वैर्निषेवितम्}


\twolineshloka
{क्वचित्कनकसंकाशं क्वचिद्रजतसंनिभम्}
{क्वचिदञ्जनपुञ्जाभं हिमवनतमुपागमत्}


\twolineshloka
{स तु तत्रनरश्रेष्ठस्तपो घोरं समाश्रितः}
{फलमूलाम्बुसंभक्षः सहस्रपरिवत्सरान्}


\threelineshloka
{संवत्सरसहस्रे तु गते दिव्ये महानदी}
{दर्शयामास तं गङ्गा तदा मूर्तिमती स्वयम् ॥गङ्गोवाच}
{}


\twolineshloka
{किमिच्छसि महाराज म्तः किंच ददानि ते}
{तद्ब्रवीहि नरश्रेष्ठ करिष्यामि वचस्तव}


\threelineshloka
{एवमुक्तः प्रत्युवाच राजा हैमवतीं नदीम्}
{`तदा भगीरथो राजन्प्रणिपत्य कृताञ्जलिः'}
{पितामहा मे वरदे कपिलेन महानदि}


\twolineshloka
{अन्वेषमाणास्तुरगं नीता वैवस्वतक्षयम्}
{षष्टिस्तानि सहस्राणि सागराणां महात्मनाम्}


\twolineshloka
{कपिलं देवमासाद्य क्षणेन निधनं गताः}
{तेषामेवं विनष्टानां स्वर्गे वासो न विद्यते}


\twolineshloka
{यावत्तानि शरीराणि त्वं जलैर्नाभिषिञ्चसि}
{तावत्तेषां गतिर्नास्ति सागराणां महानदि}


\threelineshloka
{स्वर्गं नय महाभागे मत्पितॄन्सगरात्मजान्}
{तेषामर्थेऽभियाचामि त्वामहं वै महानदि ॥लोमश उवाच}
{}


\twolineshloka
{एतच्छ्रुत्वा वचो राज्ञो गङ्गा लोकनमस्कृता}
{भगीरथमिदं वाक्यं सुप्रीता समभाषत}


\twolineshloka
{करिष्यामि माहाराज वचस्ते नात्र संशयः}
{वेगं तु मम दुर्धार्यं पतन्त्या गगनाच्च्युतम्}


\twolineshloka
{न शक्तस्त्रिषु लोकेषु कश्चिद्धारयितुं नृप}
{अन्यत्र विबुधश्रेष्ठान्नीलकण्ठान्महेश्वरात्}


\twolineshloka
{तं तोषय रमहाबाहो तपसा वरदं हरम्}
{स तु मां प्रच्युतां देवः शिरसा धारयिष्यति}


\twolineshloka
{स करिष्यति ते रकामं पितॄणां हितकाम्यया}
{`तपसाऽऽराधितः शंभुर्भगर्वाँल्लोकभावनः'}


\twolineshloka
{एतच्छ्रुत्वा ततो राजन्महाराजो भगीरथः}
{कैलासं पर्वतं गत्वा तोषयामास शंकरम्}


\twolineshloka
{ततस्तेन समागम्य कालयोगेन केनचित्}
{`गह्गावतरणं राजन्नयाचत महीपतिः}


\twolineshloka
{अगृह्णाच्च वरं तस्माद्गङ्गाया धारणे नृप}
{स्वर्गे वासं समुद्दिश् पितॄणां स नरोत्तमः}


\chapter{अध्यायः १०९}
% Check verse!

% Check verse!

% Check verse!

% Check verse!

% Check verse!

% Check verse!

% Check verse!

% Check verse!

% Check verse!

% Check verse!

% Check verse!

% Check verse!

% Check verse!

% Check verse!

% Check verse!

% Check verse!

% Check verse!

% Check verse!

% Check verse!

% Check verse!

% Check verse!

% Check verse!

% Check verse!

\chapter{अध्यायः ११०}
% Check verse!

% Check verse!

% Check verse!

% Check verse!

% Check verse!

% Check verse!

% Check verse!

% Check verse!

% Check verse!

% Check verse!

% Check verse!

% Check verse!

% Check verse!

% Check verse!

% Check verse!

% Check verse!

% Check verse!

% Check verse!

% Check verse!

% Check verse!

% Check verse!

% Check verse!

% Check verse!

\chapter{अध्यायः १११}
\twolineshloka
{लोमश उवाच}
{}


\twolineshloka
{एषा देवनदी पुण्या कौशिकी भरतर्षभ}
{विश्वामित्राश्रमो रम्य एष चात्र प्रकाशते}


\twolineshloka
{आश्रमश्चैव पुण्याख्यः काश्यपस्य महात्मनः}
{ऋश्यशृङ्गः सुतो यस् तपस्वी संयतेन्द्रियः}


\twolineshloka
{तपसो यः प्रभावेन वर्षयामास वासवम्}
{अनावृष्ठ्यां भयाद्यस्य ववर्ष बलवृत्रहा}


\twolineshloka
{मृग्यां जातः स तेजस्वी काश्यपस्य सुतः प्रभुः}
{विषये लोमपादस्य यश्चकाराद्भुतं महत्}


\threelineshloka
{निर्वर्तितेषु सस्येषु यस्मै शान्तां ददौ नृपः}
{लोमपादो दुहितरं सावित्रीं सविता यथा ॥युधिष्ठिर उवाच}
{}


\twolineshloka
{ऋश्यशृङ्गः कथं मृग्यामुत्पन्नः काश्यपात्मजः}
{विरुद्धयोनिसंसर्गः कथं च तपसा युतः}


\twolineshloka
{किमर्थं च भयाच्छक्रस्तस्य बालस्य धीमतः}
{अनावृष्ट्यां प्रवृत्तायां ववर्ष बलवृत्रहा}


\twolineshloka
{कथंरूपा च सा शान्ता राजपुत्री पतिव्रता}
{लोभयामास या चेतो मृगभूस्य तस्य वै}


\twolineshloka
{लोमपादश्च राजर्षिर्यदाऽश्रूयत धार्मिकः}
{कथं वै विषये तस्य नावर्षत्पाकशासनः}


\threelineshloka
{एतन्मे भगवन्सर्वं विस्तरेण यथातथम्}
{वक्तुमर्हसि शुश्रूषोर्ऋश्यशृङ्गस् चेष्टितम् ॥लोमश उवाच}
{}


\twolineshloka
{विभाण्डकस्य ब्रह्मर्षेस्तपसा भावितात्मनः}
{अमोघवीर्यस्य सतः प्रजापतिसमद्युतेः}


\twolineshloka
{शृणु पुत्रो यथा जात ऋश्यशृङ्गः प्रतापवान्}
{महाह्रदे महातेजा बालः स्थविरसंभतः}


\twolineshloka
{महाह्रदं समासाद्य काश्यपस्तपसि स्थितः}
{दीर्घकालं रिश्रान्त ऋषिर्देवर्षिसंमितः}


\threelineshloka
{तस्य रेतः प्रचस्कन्द दृष्ट्वाऽप्सरसमुर्वशीम्}
{अप्सूपस्पृशतो राजन्मृगी तच्चापिबत्तदा}
{सह तोयेन तृषिता गर्भिणी चाभवत्ततः}


\twolineshloka
{सा पुरोक्ता भगवता ब्रह्मणा लोककर्तृणा}
{देवकन्या मृगी भूत्वा मुनिं सूय विमोक्ष्यसे}


\threelineshloka
{अमोघत्वाद्विधेश्चैव भावित्वाद्दैवनिर्मितात्}
{तस्यां मृग्यां समभवत्तस्व पुत्रो महानृषिः}
{ऋश्यशृङ्गस्तपोनित्यो वन एवाभ्यवर्धत}


\twolineshloka
{तस्य शृङ्गं शिरसि वै राजन्नासीन्महात्मनः}
{तेनर्श्यशृङ्ग इत्येवं तदा स प्रथितोऽभवत्}


\twolineshloka
{न तेन दृष्टपूर्वोऽन्यः पितुरन्यत्रमानुषः}
{तस्मात्तस्य मनो नित्यं ब्रह्मचर्येऽभवन्नृप}


\twolineshloka
{एतस्मिन्नेव काले तु सखा दशरशस्य वै}
{कलोमपाद इतिख्यातो ह्यङ्गानामीश्वरोऽभवत्}


\threelineshloka
{तेन कामः कृतो मिथ्या ब्राह्मणेभ्य इति श्रुतिः}
{`दैवोपहतसत्त्वेन धर्मज्ञेनापि भारत'}
{स ब्राह्मणैः परित्यक्तस्तदा भरतसत्तम}


\twolineshloka
{पुरोहितापचाराच्च तस्य राज्ञो यदच्छया}
{न ववर्ष सहस्राक्षस्ततोऽपीड्यन्त वै प्रजाः}


\twolineshloka
{स ब्राह्मणान्पर्यपृच्छत्तपोयुक्तान्मनीषिणः}
{प्रवर्षणे सुरेन्द्रस्य समर्थान्पृथिवीपते}


\twolineshloka
{कथं प्रवर्षेत्पर्जन्य उपायः परिमृश्यताम्}
{तमूचुश्चोदितास्ते तु स्वमतानि मनीषिणः}


\twolineshloka
{तत्र त्वेको रमुनिवरस्तं राजानमुवाच ह}
{कुपितास्तव राजेन्द्र ब्राह्मणा निष्कृतिं चर}


\twolineshloka
{ऋश्यशृङ्गं मुनिसुतमानयस्व च पार्थिव}
{ऐणेयमनभिज्ञं च नारीणामार्जवे रतम्}


\twolineshloka
{स चेदवतरेद्राजन्विषयं ते महातपाः}
{सद्यः प्रवर्षेत्पर्जन्य इति मे नास्ति संशयः}


\twolineshloka
{एतच्छ्रुत्वा वचो राजन्कृत्वा निष्कृतिमात्मनः}
{स गत्वा पुनरागच्छत्प्रसन्नेषु द्विजातिषु}


\twolineshloka
{राजानमागतं ज्ञात्वा प्रतिसंजगृहुः प्रजाः}
{`स च ता प्रतिजग्राह पितेव हितकृत्सदा'}


\twolineshloka
{ततोऽङ्गपतिराहूय सचिवान्मन्त्रकोविदान्}
{ऋश्यशृङ्गागमे यत्नमकरोन्मन्त्रनिश्चये}


\twolineshloka
{सोऽध्यगच्छदुपायं तु तैरमात्यैः सहाच्युतः}
{शास्त्रज्ञैरलमर्थज्ञौर्नीत्यां च परिनिष्ठितैः}


\twolineshloka
{ततश्चानाययामास वारमुख्या महीपतिः}
{वेश्याः सर्वत्रनिष्णातास्ता उवाच स पार्थिवः}


\twolineshloka
{ऋश्यशृङ्गमृषेः पुत्रमानयध्वमुपायतः}
{लोभयित्वाऽभिविश्चास्य विषयं मम शोभनाः}


\twolineshloka
{ता राजभयभीताश्च शापभीताश्च योषितः}
{अशक्यमूचुस्तत्कार्यं विषण्णा गतचेतसः}


\twolineshloka
{तत्र त्वेका जरद्योषा राजानमिदमब्रवीत्}
{प्रयतिष्ये महाराज तमानेतुं तपोधनम्}


\twolineshloka
{अभिप्रेतांस्तु मे कामांस्त्वमनुज्ञातुमर्हसि}
{ततः शक्ष्याम्यानयितुमृश्यशृङ्गमृषेः सुतम्}


\twolineshloka
{तस्याः सर्वमभिप्रेतमन्वजानात्स पार्थिवः}
{धनं च प्रददौ भूरि रत्नानि विविधानि च}


\twolineshloka
{ततो रूपेण सपन्ना वयसा च महीपते}
{स्त्रिय आदाय काश्चित्सा जगाम वनमञ्जसा}


\chapter{अध्यायः ११२}
\twolineshloka
{लोमश उवाच}
{}


\twolineshloka
{सा तु नाव्याश्रमं चक्रे राजकार्यार्थसिद्धये}
{संदेशाच्चैव नृपतेः स्वबुद्ध्या चैव भारत}


\twolineshloka
{नानापुष्पफलैर्वृक्षैः कृत्रिमैरुपशोभितैः}
{नानागुल्मलतोपेतैः स्वादुकामफलप्रदैः}


\twolineshloka
{अतीव रमणीयं तदतीव च मनोहरम्}
{चक्रे नाव्याश्रमं रम्यमद्भुतोपमदेर्शनम्}


\twolineshloka
{ततो निवध्य सा नावमदूरे काश्यपाश्रमात्}
{चारयामास पुरुषैर्विहारं तस्य वै मुनेः}


\twolineshloka
{ततो दुहितरं वेश्यां समाधायेतिकार्यताम्}
{दृष्ट्वाऽन्तरं काशय्पस्य प्राहिणोद्बुद्धिसंमताम्}


\threelineshloka
{सा तत्र गत्वा कुशला तपोनित्यस्य संनिधौ}
{आश्रमं तं समासाद्य ददर्श तमृषेः सुतम् ॥वेश्योवाच}
{}


\twolineshloka
{कच्चिन्मुने कुशलं तापसानांकच्चिच्च वो मूलफलं प्रभूतम्}
{कच्चिद्भवान्रमते चाश्रेऽस्मिं-स्त्वां वै द्रष्टुं सांप्रतमागतास्मि}


\threelineshloka
{कच्चित्तपो वर्धते तापसानांपिता च ते कच्चिदहीनतेजाः}
{कच्चित्त्वया प्रीयतेचैव विप्रकच्चित्स्वाध्यायः क्रियते चर्श्यशृङ्गः ॥ऋश्यशृङ्ग उवाच}
{}


\twolineshloka
{ऋद्ध्या भवाञ्ज्योतिरिव प्रकाशतेमन्ये चाहं त्वामभिवादनीयम्}
{पाद्यं वै ते संप्रदास्यामि कामा-द्यथाधर्मं फलमूलानि चैव}


\threelineshloka
{कौश्यां बृस्यामास्स्व यथोपजोषंकृष्णाजिनेनावृतायां सुखाय}
{क्व चाश्रमस्तव किं नाम चेदंव्रतंब्रह्मंश्चरसि हि देववत्त्वम् ॥वेश्योवाच}
{}


\twolineshloka
{ममाश्रमः काश्यपपुत्र रम्य-स्त्रियोजनं शैलमिमं परेण}
{तत्रस्वधर्मोऽनभिवादनं नोन चोदकं पाद्यमुपस्पृशामः}


\threelineshloka
{भवता नाभिवाद्योऽहमभिवाद्यो भवान्मया}
{व्रतमेतादृशं ब्रह्मन्परिष्वज्यो भवान्मया ॥ऋश्यशृङ्ग उवाच}
{}


\threelineshloka
{फलानि पक्वानि ददानि तेऽहंभल्लातकान्यामलकानि चैव}
{करूषकानीङ्गुदधन्वनानिप्रियालानां काङ्क्षितं वै कुरुष्व ॥लोमश उवाच}
{}


\twolineshloka
{सा तानि सर्वाणि विसर्जयित्वाभक्ष्याण्यनर्हाणि ददौ ततोऽस्मै}
{तान्यृश्यशृङ्गाय महारसानिभृशं सुरूपाणि च मोदकानि}


\twolineshloka
{ददौ च माल्यानि सुगन्धवन्तिचित्राणि वासांसि च भानुमन्ति}
{पेयानि चाग्र्याणि ततो मुमोदचिक्रीड चैव प्रजहास चैव}


\twolineshloka
{सा कन्दुकेनारमतास्य मूलेविभज्यमाना फलिता लतेव}
{गात्रैश्च गात्राणि निषेवमाणासमाश्लिषच्चासकृदृश्यशृङ्गम्}


\twolineshloka
{सर्जानशोकांस्तिलकांश्च वृक्षा-न्सुपुष्पितानवनाम्यावभज्य}
{विलज्जमानेव मदाभिभूताप्रोभयामास सुतं महर्षेः}


\twolineshloka
{अथर्श्यशृङ्गं विकृतंसमीक्ष्यपुनः पुनः पीञ्य च कायमस्य}
{अवेक्ष्यमाणा शनकैर्जगामकृत्वाऽग्निहोत्रस्य तदाऽपदेशम्}


\twolineshloka
{तस्यां गतायां मदनेन मत्तोविचेतनश्चाभवदृश्यशृङ्गः}
{तामेव भावेन गतेन शून्येविनिःश्वसन्नार्तरूपो बभूव}


\twolineshloka
{ततो मुहूर्ताद्धरिपिङ्गलाक्षःप्रवेष्टितो रोमभिरानखाग्रात्}
{स्वाध्यायवान्वृत्तसमाधियुक्तोविभाण्डकः काश्यपः प्रादुरासीत्}


\twolineshloka
{सोऽपश्यदासीनमुपेत्य पुत्रंध्यायन्तमेकं विपरीतचित्तम्}
{विनिःश्वसन्तं मुहुरूर्ध्वदृष्टिंविभाण़्कः पुत्रमुवाच दीनम्}


\twolineshloka
{न कल्पिताः समिधः किंनु तातकच्चिद्धुतं चाग्निहोऽत्रं त्वयाऽद्य}
{`न संसृष्टं क्रियते द्वारभागेसुवृक्षाणा खण्डने कः प्रवृत्तः'}


\twolineshloka
{सुनिर्णिक्तं स्रुक्स्रुवं होमधेनुःकच्चित्सवत्साद्यकृता त्वया च}
{`कोप्यागतः शुश्रूषणायेह पुत्रकुतश्चित्रं माल्यमिदं प्रवृद्धम्'}


\twolineshloka
{न वै यथापूर्वमिवासि पुत्रचिन्तापरश्चासि विचेतनश्च}
{दीनोतिमात्रं किमिवाद्य खिन्नःपृच्छामि त्वां क इहाद्यागतोऽभूत्}


\chapter{अध्यायः ११३}
\twolineshloka
{ऋश्यशृङ्ग उवाच}
{}


\twolineshloka
{इहागतो जटिलो ब्रह्मचारीन वै ह्रस्वो नातिदीर्घो मनस्वी}
{सुवर्णवर्णः कमलायताक्षःसुतः सुराणामवि शोभमानः}


\twolineshloka
{समृद्धरूपः सवितेव दीप्तःसुश्लक्ष्णवाक्कृष्णतारश्चकोरात्}
{नीलाः प्रसन्नाश्च जटाः सुगन्धाहिरण्यरज्जुग्रथिताः सुदीर्घाः}


\twolineshloka
{आधारभूतः पुनरस्य कण्ठेविभ्राजते विद्युदिवान्तरिक्षे}
{द्वौ चास् पिण्डावधरेण कण्ठा-दजातरोमौ सुमनोहरौ च}


\twolineshloka
{विलग्नमध्यश्च स नाभिदेशेकटिश्च तस्यातिकृतप्रमाणा}
{तथाऽस् चीरान्तरतः प्रभातिहिरण्मयी मेखला मे यथेयम्}


\twolineshloka
{अन्यच्च तस्याद्भुतदर्शनीयंविकूजितं पादयोः संप्रभाति}
{पाण्योश्च तद्वत्स्वनवन्निबद्धौकलापकावक्षमाला यथेयम्}


\twolineshloka
{विचेष्टमानस्य च तस्य तानिकूजन्ति हंसाः सरसीव मत्ताः}
{चीराणि तस्याद्भुतदर्शनानिनेमानि तद्वन्मम रूपवन्ति}


\twolineshloka
{वक्रं च तस्याद्भुतदर्सनीयंप्रव्याहृतं ह्लादयतीव चेतः}
{पिंस्कोकिलस्येव च तस्य वाणितां शृण्वतो मे व्यथितोऽन्तरात्मा}


\twolineshloka
{यथा वनं माधवमासिमध्येसमीरीतं श्वसनेनेव भाति}
{तथा स भात्युत्तमपुण्यगन्धीनिषेव्यमाणः पवनेन तात}


\twolineshloka
{सुसंवताश्चापि जटा विभक्ताद्वैधीकृता भातिललाटदेशे}
{कर्णौ च चित्रैरिव चक्रवाकैःसमावृतौ तस्य सुरूपवद्भिः}


\twolineshloka
{तथा फलं वृत्तमथो विचित्रंसमाहतं पाणिना दक्षिणेन}
{तद्भूमिमासाद्य पुनःपुनश्चसमुत्पतत्यद्भुतरूपमुच्चैः}


\twolineshloka
{तच्चाभिहत्वा परिवर्ततेऽसौवातेरितो वृक्ष इवावघूर्णम्}
{तं प्रेक्षतः पुत्रमिवामराणांप्रीतिः परा तात रतिश्च जाता}


\twolineshloka
{स मे समाश्लिष्य पुनः शरीरंजटासु गृह्याभ्यवनाम्य वक्रम्}
{वक्रेण वक्रं प्रणिधाय शब्दंचकार तन्मेऽजनयत्प्रहर्षम्}


\twolineshloka
{न चापि पाद्यं बहुमन्यतेऽसौफलानि चेमानि मया हृतानि}
{एवंव्रतोस्मीति च मामवोच-त्फलानि चान्यानि नवान्यदान्मे}


\twolineshloka
{मयोपयुक्तानि फलानि यानिनेमानि तुल्यानि रसेन तेषाम्}
{न चापि तेषां त्वगियं यथैषांसाराणि नैषामिव सन्ति तेषाम्}


\twolineshloka
{तोयानि चैवातिरसानि मह्यंप्रादात्म वै पातुमुदाररूपः}
{पीत्वैव यान्यभ्यधिकः प्रहर्षोममाभवद्भूश्चलितेव चासीत्}


\twolineshloka
{इमानि चित्राणि च गन्धवन्तिमाल्यानि तस्योद्ग्रथितानि पट्टैः}
{यानि प्रकीर्येह गतः स्वमेवस आश्रमं तपसा द्योतमानः}


\twolineshloka
{गतेन तेनास्मिकृतो विचेतागात्रं च मे संपरिदह्यतीव}
{इच्छामि तस्यान्तिकमाशु गन्तुंतं चेह नित्यं परिवर्तमानम्}


\twolineshloka
{गच्छामि तस्यान्तिकमेव तातका नाम सा ब्रह्मचर्या च तस्य}
{इच्छाम्यहं चरितुं तेन सार्धंयथा तपः स चरत्यार्यधर्मा}


% Check verse!
चर्तुं तथेच्छा हृदये ममास्तिदुनोति चित्तं यदि तं न पश्ये
\chapter{अध्यायः ११४}
\twolineshloka
{विभाण्डक उवाच}
{}


\twolineshloka
{रक्षांसि चैतानि चरन्ति पुत्ररूपेण तेनाद्भुतदर्शनेन}
{अतुल्यवीर्याण्यभिरूपवन्तिविघ्रं सदा तपसश्चिन्तयन्ति}


\twolineshloka
{सुरूपरूपाणि च तानि तातप्रलोभयन्ते विविधैरुपायैः}
{सुखाच्च लोकाच्च निपातयन्तितान्युग्ररूपाणि मुनीन्वनेषु}


\twolineshloka
{न तानि सेवेत मुनिर्यतात्मासतां लोकान्प्रार्थयानः कथंचित्}
{कृत्वा विघ्रनं तापसानां रमन्तेपापाचारास्तापसस्तान्न पश्येत्}


\twolineshloka
{असज्जनेनाचरितानि पुत्रपानान्यपेयानि मधूनि तानि}
{माल्यानि चैतानि न वै मुनीनांस्मृतानि चित्रोज्ज्वलगन्धवन्ति}


\twolineshloka
{रक्षांसि तानीति निवार्य पुत्रंविभाण्डकस्तां मृगयांबभूव}
{नासादयामास यदा त्र्यहेणतदा स पर्याववृते श्रमाय}


\twolineshloka
{यदा पुन काश्यपो वै जगामफलान्याहर्तुं विधिनाश्रमेऽसौ}
{तदा पुनर्लोभयितुं जगामसा वेशायोषा मुनिमृश्यशृह्गम्}


\threelineshloka
{दृष्ट्वैव तामृश्यशृङ्गः प्रहृष्टःसंभ्रान्तरूपोऽभ्यपतत्तदानीम्}
{प्रोवाच चैनां भवतः श्रमायगच्छाव यावन्न पिता ममैति ॥लोमश उवाच}
{}


\twolineshloka
{ततो राजन्काश्यपस्यैकपुत्रंप्रवेश्य वेगेन विमुच्य नावम्}
{प्रलोभयन्त्यो विविधैरुपायै-राजग्मुरङ्गाधिपतेः समीपम्}


\twolineshloka
{संस्थाप्यतामाश्रमदर्शने तुसंतारितां नावमथातिशुभ्राम्}
{तीरादुपादाय तथैव चक्रेराजाश्रमं नाम वनं विचित्रम्}


\twolineshloka
{अन्तःपुरे तं तु निवेश्य राजाविभाण्डकस्यात्मजमेकपुत्रम्}
{ददर्श मेघैः सहसा प्रवृष्ट-मापूर्यणाणं च जगज्जलेन}


\twolineshloka
{स रोमपाद परिपूर्णकामःसुतां ददावृश्यशृङ्गाय शान्ताम्}
{क्रोधप्रतीकारकरं च चक्रेगोभिश्च मार्गेष्वभिकर्षणं च}


\twolineshloka
{विभाण्डकस्याव्रजतः स राजापशून्प्रभूतान्पशुपांश्च वीरान्}
{समादिशत्पुत्रगृद्धी महर्षि-र्विभाण्डकः परिपृच्छेद्यदा वः}


\twolineshloka
{स क्तव्यः प्राञ्जलिभिर्भवद्भिःपुत्रस्य ते पशवः कर्षणं च}
{किं ते प्रियं वै क्रियतां महर्षेदासाः स्म सर्वे तव वाचि बद्धाः}


\twolineshloka
{अथोपायात्स मुनिश्चण्डकोपःस्वमाश्रमं मूलफलं गृहीत्वा}
{अन्वेषमाणश्च न तत्र पुत्रंददर्श चुक्रोध ततो भृशं सः}


\twolineshloka
{ततः स कोपेन विदीर्यमाणआशङ्कमानो नृपतेर्विधानम्}
{जगाम चम्पां प्रतिधक्ष्यमाण-स्तमङ्गराजं सपुरं सराष्ट्रम्}


\twolineshloka
{स वै श्रान्तः क्षुधितः काश्यपस्ता-न्घोषान्समासादितवान्समृद्धान्}
{गोपैश्च तैर्विधिवत्पूज्यमानोराजेव तां रात्रिमुवास तत्र}


\twolineshloka
{अवाप्य सत्कारमतीव हृष्टःप्रोवाच कस्य प्रथिताः स्थ गोपाः}
{ऊचुस्ततस्तेऽभ्युपगम्य सर्वेधनं तवेदं विहितं सुतस्य}


\twolineshloka
{देशेषु देशेषु स पूज्यमान-स्तांश्चैव शृण्वन्मधुरान्प्रलापान्}
{प्रशान्तभूयिष्ठरजाः प्रहृष्टःसमाससादाङ्गपतिं पुरस्थम्}


\twolineshloka
{स पूजितस्तेन नरर्षभेणददर्श पुत्रं दिवि देवं यथेन्द्रम्}
{शान्तां स्नुषां चैव ददर्श तत्रसौदामनीमुच्चरन्तीं यथैव}


\twolineshloka
{ग्रामां श्च घोषांश्च सुतस्य दृष्ट्वाशान्तां च शान्तोऽस्य परः स कोपः}
{चकार तस्यैव परं प्रसादंविभाण्डको भूमिपतेर्नरेद्र}


\twolineshloka
{स तत्रनिक्षिप्य सुतं महर्षि-रुवाच सूर्याग्निसमप्रभावः}
{जाते च पुत्रे वनमेवाव्रजेथाराज्ञः प्रियाण्यस्य सर्वाणि कृत्वा}


\twolineshloka
{स तद्वचः कृतवानृश्यशृङ्गोययौ च यत्रास्य पिता बभूव}
{शान्ता चैनं पर्यचरन्नरेन्द्रस्वे रोहिणी सोममिवानुकूला}


\twolineshloka
{अरुन्धती वा सुभगा वसिष्ठंलोपामुद्रा वा यथा ह्यगस्त्यम्}
{नलस्य वै दमयन्ती यथा भूद्यथा शची वज्रधरस्य चैव}


\threelineshloka
{नालायनी चेन्द्रसेना बभूववस्या नित्यं मुद्गलस्याजमीढ}
{`यथा सीता दाशरथेर्महात्मनोयथा तव द्रौपदी पाण्डुपुत्र'}
{तथा शान्ता ऋश्यशृङ्गं वनस्थंप्रीत्या युक्ता पर्यचरन्नरेन्द्र}


\twolineshloka
{तस्याश्रमः पुण्य एषोऽवभातिमहाह्रदं शोभयन्पुण्यकीर्तिः}
{अत्रस्नातः कृतकृत्यो विशुद्ध-स्तीर्थान्यन्यान्यनुसंयाहि राजन्}


\chapter{अध्यायः ११५}
\twolineshloka
{वैशंपायन उवाच}
{}


\twolineshloka
{ततः प्रयातः कौशिक्याः पाण्डवो जनमेजय}
{आनुपूर्व्येण सर्वाणि जगामायतनान्यथ}


\twolineshloka
{स सागरं समासाद्य गङ्गायाः संगमे नृप}
{नदीशतानां पञ्चानां मध्ये चक्रे समाप्लवम्}


\threelineshloka
{ततः समुद्रतीरेण जगाम वसुधाधिपः}
{भ्रातृभिः सहितो वीरः कलिङ्गान्प्रति भारत ॥लोमश उवाच}
{}


\twolineshloka
{एते कलिङ्गाः कौन्तेय यत्र वैतरणी नदी}
{यत्रायजत धर्मोपि देवाञ्शरणमेत्य वै}


\twolineshloka
{ऋषिभिः समुपायुक्तं यज्ञियं गिरिशोभितम्}
{उत्तरं तीरमेतद्धि सततं द्विजसेवितम्}


\twolineshloka
{समेन देवयानेन पथा स्वर्गमुपेयुषः}
{अत्रवै ऋषयोऽन्येऽपि पुरा क्रतुभिरीजिरे}


\twolineshloka
{अत्रैव रुद्रो राजेन्द्र पशुमादत्तवान्मखे}
{पशुमादाय राजेन्द्र भागोयमिति चाब्रवीत्}


\twolineshloka
{हृते पशौ तदा देवास्तमूचुर्भरतर्षभ}
{मा परस्वमभिद्रोग्धा माधर्म्यान्नीनशः पथः}


\twolineshloka
{`स्वयं यज्ञेश्वरो भूत्वा कर्मणां फलदायकः}
{यज्ञं विहन्तुं भगवान्नार्हसे जगदीश्वर'}


\twolineshloka
{इतिकल्याणरूपाभिर्वाग्भिस्ते रुद्रमस्तुवन्}
{इष्ट्या चैनं तर्पयित्वा मानयांचक्रिरे तदा}


\twolineshloka
{ततः स पशुमुत्सृज्य देवयानेन जग्मिवान्}
{तत्रानुवंशो रुद्रस्यतं निबोध युधिष्ठिर}


\twolineshloka
{अयातयामं सर्वेभ्यो भागेभ्यो भागमुत्तमम्}
{देवाः संकल्पयामासुर्भयाद्रुद्रस्य शाश्वतम्}


\threelineshloka
{इमां गाथामत्र गायन्नपः स्पृशति यो नरः}
{देवयानोऽस्य पन्थाश्च चक्षुषाऽभिप्रकाशते ॥वैशंपायन उवाच}
{}


\threelineshloka
{ततो वैतरणीं सर्वे पाण्डवा द्रौपदी तथा}
{अवतीर्य महाभागास्तर्पयांचक्रिरे पितृन् ॥युधिष्ठिर उवाच}
{}


\twolineshloka
{उपस्पृश्येह विधिवदस्यां नद्यां तपोबलात्}
{मानुषादस्मि विषयादपेतः पश्य लोमश}


\threelineshloka
{सर्वाल्लोँकान्प्रपश्यामि प्रसादात्तव सुव्रत}
{वैखानसानां जपतामेष शब्दो महात्मनाम् ॥लोमश उवाच}
{}


\twolineshloka
{त्रिशतं वै सहस्राणि योजनानां युधिष्ठिर}
{यत्रध्वनिं शृणोष्येनं तूष्णीमास्स्व विशांपते}


\twolineshloka
{एतत्स्वयंभुवो राजन्वनं दिव्यं प्रकाशते}
{यत्रायजत राजेन्द्र विश्वकर्मा प्रतापवान्}


\twolineshloka
{यस्मिन्यज्ञे हि भूर्दत्ता कश्यपाय महात्मने}
{सपर्वतवनोद्देशा दक्षिणार्थे स्वयंभुवा}


\twolineshloka
{अवासीदच्च कौन्तेय दत्तमात्रा मही तदा}
{उवाच चापि कुपिता लोकश्वरमिदं प्रभुम्}


\twolineshloka
{न मां मर्त्याय भगवन्कस्मैचिद्दातुमर्हसि}
{प्रदानं मोघमेतत्ते यास्याम्येषा रसातलम्}


\twolineshloka
{विषीदन्तीं तु तां दृष्ट्वा कश्यपो भगवानृषिः}
{प्रसादयांबभूवाथ ततो भूमिं विशांपते}


\twolineshloka
{ततः प्रसन्ना पृथिवी तपसा तस्य पाण्डव}
{पुनरुन्नह्य सलिलाद्देदीरूपास्थिता बभौ}


\twolineshloka
{सैषा प्रकाशते राजन्वेदीसंस्थानलक्षणा}
{आरुह्यात्र महाराज वीर्यवान्वै भविष्यसि}


\twolineshloka
{सैषा सागरमास्राद्य राजन्वेदी समाश्रिता}
{एतामारुह्य भद्रं ते त्वमेकस्तर सागरम्}


\twolineshloka
{अहं च ते स्वस्त्ययनं प्रयोक्ष्येयथा त्वमेनामधिरोहसेऽद्य}
{स्पृष्टा हि मर्त्येन ततः समुद्र-मेषा वेदी प्रविशत्याजमीढ}


\twolineshloka
{ओंनमो विश्वगुप्ताय नमो विश्वपराय ते}
{सान्निध्यं कुरु देवेश सागरे लवणाम्भसि}


\twolineshloka
{अग्निर्मित्रो योनिरापोऽथ देव्योविष्णो रेतस्त्वममृतस्य नाभिः}
{एवं ब्रुवन्पाण्डव सत्यवाक्यंवेदीमिमां त्वं तरसाऽधिरोह}


\twolineshloka
{अग्निश्च ते योनिरिडा च देहोरेतोधा विष्णोरमृतस्य नाभिः}
{एवं ब्रुवन्पाण्डव सत्यवाक्यंततोऽवगाहेत पतिं नदीनाम्}


\threelineshloka
{अन्यथा हि कुरुश्रेष्ठ देवयोनिरपांपतिः}
{कुशाग्रेणापि कौन्तेय न स्प्रष्टव्यो महोदधिः ॥वैशंपायन उवाच}
{}


\twolineshloka
{ततः कृतस्वस्त्ययनो महात्मायुधिष्ठिरः सागरमभ्यगच्छत्}
{कृत्वा च तच्छासनमस्य सर्वंमहेन्द्रमासाद्य निशामुवास}


\chapter{अध्यायः ११६}
\twolineshloka
{वैशंपायन उवाच}
{}


\twolineshloka
{स तत्र तामुषित्वैकां रजनीं पृथिवीपतिः}
{तापसानां परं चक्रे सत्कारं भ्रातृभिः सह}


\twolineshloka
{लोमशस्तस्य तान्सर्वानाचख्यौ तत्र तापसान्}
{भृगूनङ्गिरसश्चैव वसिष्ठानथ काश्यपान्}


\twolineshloka
{तान्समेत्य सा राजर्षिरभिवाद्य कृताञ्जलिः}
{रामस्यानुचरं वीरमपृच्छदकृतव्रणम्}


\threelineshloka
{कदा नु रामो भगवांस्तापसो दर्शयिष्यति}
{तमहं तपसा युक्तं द्रष्टुमिच्छामि भार्गवम् ॥अकृतव्रण उवाच}
{}


\twolineshloka
{आयानेवासि विदितो रामस्य विदितात्मनः}
{प्रीतिस्त्वयि च रामस्य क्षिप्रं त्वां दर्शयिष्यति}


\twolineshloka
{चतुर्दशीमष्टमीं च रामं पश्यन्ति तापसाः}
{अस्यां रात्र्यां व्यतीतायां भवित्री श्वश्चतुर्दशी}


\fourlineindentedshloka
{`ततो द्रक्ष्यसि रामं त्वं कृष्णाजिनजटाधरम्'}
{युधिष्ठिर उवाच}
{भवाननुगतो रामं जामदग्न्यं महाबलम्}
{प्रत्यक्षदर्शी सर्वस्य पूर्ववृत्तस्य कर्मणः}


\threelineshloka
{स भवान्कथयत्वद्य यथा रामेण निर्जिताः}
{आहवे क्षत्रियाः सर्वे कथं केन च हेतुना ॥अकृतव्रण उवाच}
{}


\twolineshloka
{[हन्त ते कथयिष्यामि महदाख्यानमुत्तमम्}
{भृगूणां राजशार्दूल वंशे जातस्य भारत}


\twolineshloka
{रामस्य जामदग्न्यस्य चरितं देवसंमितम्}
{हैहयाधिपतेश्चैव कार्तवीर्यस् भारत}


\twolineshloka
{रामेण चार्जुनो नाम हैहयाधिपतिर्हतिः}
{तस्य बाहुशतान्यासंस्त्रीणि सप्त च पाण्डव}


\twolineshloka
{दत्तात्रेयप्रसादेन विमानं काञ्चनं तथा}
{ऐश्वर्यं सर्वभूतेषु पृथिव्यां पृथिवीपते}


\twolineshloka
{अव्घाहतगतिश्चैव रथस्तस्य महात्मनः}
{रथेन तेन तु सदावरदानेन वीर्यवान्}


\twolineshloka
{ममर्द देवान्यक्षांश्च ऋषींश्चैव समन्ततः}
{भूतांश्चैव स सर्वांस्तु पीडयामास सर्वतः}


\threelineshloka
{ततो देवाः समेत्याहुर्ऋषयश्च महाव्रताः}
{देवदेवं सुरारिघ्नं विष्णुं सत्यपराक्रमम्}
{भगवन्भूतरार्थमर्जुनं जहि वै प्रभो}


\twolineshloka
{विमानेन च दिव्येन हैहयाधिपतिः प्रभुः}
{शचीसहायं क्रीडन्तं धर्षयामास वासवम्}


\twolineshloka
{ततस्तु भगवान्देवः शक्रेण सहितस्तदा}
{कार्तवीर्यविनाशार्थं मन्त्रयामास भारत}


\threelineshloka
{यत्तद्भूतहितं कार्यं सुरेन्द्रेण निवेदितम्}
{संप्रतिश्रुत्य तत्सर्वं भगवाँल्लोकपूजितः}
{जगाम बदरीं रम्यां स्वमेवाश्रममण्डलम्}


\threelineshloka
{एतस्मिन्नेव काले तु पृथिव्यां पृथिवीपतिः}
{]कान्यकुब्जे महानासीत्पार्थिवः सुमहाबलः}
{गाधीति विश्रुतो लोके वनवासं जगाम ह}


\twolineshloka
{वने तु तस्य वसतः कन्या जज्ञेऽप्सरःसमा}
{ऋचीको भार्गवस्तां च वरयामास भारत}


\twolineshloka
{तमुवाच ततो गाधिर्ब्राह्मणं संशितव्रतम्}
{उचितं नः कुले किंचित्पूर्वैर्यत्संप्रवर्तितम्}


\twolineshloka
{एकतः श्यामकर्णानां पाण्डुराणां तरस्विनाम्}
{सहस्रं वाजिनां शुक्लमिति विद्धि द्विजोत्तम}


\threelineshloka
{न चापि भगवान्वाच्योदीयतामिति भार्गव}
{देया मे दुहिता चैव त्वद्विधाय महात्मने ॥ऋचीक उवाच}
{}


\threelineshloka
{एकतः श्यामकर्णानां पाण्डुराणां तरस्विनाम्}
{दास्याम्यश्वसहस्रं ते मम भार्या सुताऽस्तु ते ॥`गाधिरुवाच}
{}


\twolineshloka
{ददास्यश्वसहस्रं मे तव भार्या सुताऽस्तु मे' ॥अकृतव्रण उवाच}
{}


\threelineshloka
{स तथेति प्रतिज्ञाय राजन्वरुणमब्रवीत्}
{एकतः श्यामकर्णानां पाण्डुराणां तरस्विनाम्}
{सहस्रं वाजिनामेकं शुल्कार्थं प्रतिदीयताम्}


\twolineshloka
{तस्मै प्रादात्सहस्रं वै वाजिनां वरुणस्तदा}
{तदश्वतीर्थं विख्यातमुत्थिता यत्र ते हयाः}


\twolineshloka
{गङ्गायां कान्यकुब्जे वै ददौ सत्यवतीं तदा}
{ततो गाधिः सुतां चास्मै जन्याश्चासन्सुरास्तदा}


\twolineshloka
{लब्धं हयसहस्रं तु तां च दृष्ट्वा दिवौकसः}
{`विस्मयं परमं जग्मुस्तमेव दिवि संस्तुवन्'}


\twolineshloka
{धर्मेण लब्ध्वा तां भार्यामृचीको द्विजसत्तमः}
{यथाकामं यथाजोषं तया रेमे सुमध्यया}


\twolineshloka
{तं विवाहे कृतेराजन्सभार्यमवलोककः}
{आजगाम भृगुश्रेष्ठः पुत्रं दृष्ट्वा ननन्द च}


\twolineshloka
{भार्यापती तमासीनं भृगुं सुरगणार्चितम्}
{अर्चित्वा पर्युपासीनौ प्राञ्जली तस्थतुस्तदा}


\twolineshloka
{ततः स्नुषां स भगवान्प्रहृष्टो भृगुरब्रवीत्}
{वरं वृणीष्व सुभगे दाता ह्यस्मि तवेप्सितम्}


\threelineshloka
{सा वै प्रसादयाभास तं गुरुं पुत्रकारणात्}
{आत्मनश्चैव मातुश्च प्रसादं च चकार सः ॥भृगुरुवाच}
{}


\twolineshloka
{ऋतौ त्वं चैव माता च स्नाते पुंसवनाय वै}
{आलिङ्गेतां पृथग्वृक्षौ साऽस्वत्थं त्वमुदुम्बरम्}


\twolineshloka
{चरुद्वयमिदं भद्रे जनन्याश्च तवैव च}
{विश्वमावर्तयित्वा तु मया यत्नेन साधितम्}


\twolineshloka
{प्राशितव्यं प्रयत्नेन तेत्युक्त्वाऽदर्शनं गतः}
{आलिङ्गने चरौ चैव चक्रतुस्ते विपर्ययम्}


\twolineshloka
{ततः पुन स भगवान्काले बहुतिथे गते}
{दिव्यज्ञानाद्विदित्वा तु भगवानागतः पुनः}


% Check verse!
अथोवाच महातेजा भृगुः सत्यवतीं श्नुषाम्
\twolineshloka
{उपयुक्तश्चरुर्भद्रे वृक्षे चालिङ्गनं कृतम्}
{विपरीतेन ते सुभ्रूर्मात्रा चैवासि वञ्चिता}


\twolineshloka
{क्षत्रियो ब्राह्मणाचारो मातुस्तव सुतो महान्}
{भविष्यति महावीर्यः साधूनां मार्गमास्थितः}


\twolineshloka
{ततः प्रसादयामास श्वशुरं सा पुनःपुनः}
{न मे पुत्रो भवेदीदृक्कामं पौत्रो भवेदिति}


\twolineshloka
{एवमस्त्विति सा तेन पाण्डव प्रतिनन्दिता}
{क्रालं प्रतीक्षती गर्भं धारयामास यत्नतः}


\twolineshloka
{जमदग्निं ततः पुत्रं जज्ञे सा काल आगते}
{तेजसा वर्चसा चैव युक्तं भार्गवनन्दनम्}


\twolineshloka
{स वर्धमानस्तेजस्वी वेदस्याध्ययनेन च}
{बहूनृषीन्महातेजाः पाण्डवेयात्यवर्तत}


\twolineshloka
{तं तु कृत्स्नो धनुर्वेदः प्रत्यभाद्भरतर्षभ}
{चतुर्विधानि चास्त्राणि भास्करोपमवर्चसम्}


\chapter{अध्यायः ११७}
\twolineshloka
{अकृतव्रण उवाच}
{}


\twolineshloka
{स वेदाध्ययने युक्तो जमदग्निर्महातपाः}
{तपस्तेपे ततो देवान्नियमाद्वशमानयत्}


\twolineshloka
{स प्रसेनजितं राजन्नधिगम्य नराधिपम्}
{रेणुकां वरयामास स च तस्मै ददौ नृपः}


\twolineshloka
{रेणुकां त्वथ संप्राप्य भार्यां भार्गवनन्दनः}
{आश्रमस्थस्तया सार्धं तपस्तेपेऽनुकूलया}


\twolineshloka
{तस्याः कुमाराश्चत्वारो जज्ञिरे रामपञ्चमाः}
{सर्वेषामजघन्यस्तु राम आसीञ्जघन्यजः}


\twolineshloka
{फलाहारेषु सर्वेषु गतेष्वथ सुतेषु वै}
{रेणुका स्नातुमगमत्कदाचिन्नियतव्रता}


\twolineshloka
{सा तु चित्ररथं नाम मार्तिकावतकं नृपम्}
{ददर्श रेणुका राजन्नागच्छन्ती यदृच्छया}


\twolineshloka
{क्रीडन्तं सलिले दृष्ट्वा सभार्यं पद्ममालिनम्}
{ऋद्धिमन्तं ततस्तस्य स्पृहयामास रेणुका}


\twolineshloka
{व्यभिचाराच्च तस्मात्सा क्लिन्नाम्भसि विचेतना}
{`अन्तरिक्षान्निपतिता नर्मदायां महाह्रदे}


\twolineshloka
{उतीर्य चापि सा यत्नाज्जगाम भरतर्षभ'}
{प्रविवेशाश्रमं त्रस्ता तां वै भर्तान्वबुध्यत}


\twolineshloka
{स तां दृष्ट्वाच्युतांधैर्याद्ब्राह्म्या लक्ष्म्या विवर्जिताम्}
{श्रिक््शब्देन महातेजा गर्हयामास वीर्यवान्}


\twolineshloka
{ततो ज्येष्ठो जामदग्न्यो रुमण्वान्नाम नामतः}
{आजगाम सुषेणश्च वसुर्विश्वावसुस्तथा}


\twolineshloka
{तानानुपूर्व्याद्भगवान्वधे मातुरचोदयत्}
{न च ते जातसंमोहाः किंचिदूचुर्विचेतसः}


\twolineshloka
{ततः शशाप तान्क्रोधात्ते शप्ताश्चैतनां जहुः}
{मृगपक्षिसधर्माणः क्षिप्रमासञ्जडोपमाः}


\twolineshloka
{ततो रामोऽभ्ययात्पश्चादाश्रमं परवीरहा}
{तमुवाच महामन्युर्जमदग्निर्महातपाः}


\twolineshloka
{जहीमां मातरं पापां मा च पुत्र व्यथां कृथाः}
{तत आदाय परशुं रामो मातु शिरोऽहरत्}


\twolineshloka
{ततस्तस्य महाराज जमदग्नेर्महात्मनः}
{कोपोऽभ्यगच्छत्सहसा प्रसन्नश्चाब्रवीदिदम्}


\twolineshloka
{ममेदं वचनात्तात कृतं ते कर्म दुष्करम्}
{वृणीष्व कामान्धर्मज्ञ यावतो वाञ्छसे हृदा}


\twolineshloka
{स वव्रे मातुरुत्थानमस्मृतिं च वधस्य वै}
{पापेन तेन चास्पर्शं भ्रातॄणां प्रकृतिं तथा}


\twolineshloka
{अप्रतिद्वन्द्वतां युद्धे दीर्घमायुश्च भारत}
{ददौ च सर्वान्कामांस्ताञ्जमदग्निर्महातपाः}


\twolineshloka
{कदाचित्तु तथैवास्य विनिष्क्रान्ताः सुताः प्रभो}
{अथानूपपतिर्वीरः कार्तवीर्योऽभ्यवर्तत}


\twolineshloka
{तमाश्रमपदं प्राप्तमृषिरर्ध्यात्समार्चयत्}
{स युद्धमदसंमत्तो नाभ्यनन्दत्तथाऽर्चनम्}


\twolineshloka
{प्रमथ्य चाश्रमात्तस्माद्धोमधेनोस्ततोबलात्}
{जहार वत्सं क्रोशन्त्या बभञ्ज च महाद्रुमान्}


\threelineshloka
{आगताय च रामाय तदाचष्ट पिता स्वयम्}
{गां च रोरुदतीं दृष्ट्वा कोपो रामं समाविशत्}
{स मन्युवशमापन्नः कार्तवीर्यमुपाद्रवत्}


\twolineshloka
{तस्याथ युधि विक्रम्य भार्गवः परवीरहा}
{चिच्छेद निशितैर्भल्लैर्बाहून्परिघसंनिभान्}


\twolineshloka
{सहस्रसंमितान्राजन्प्रगृह्य रुचिरं धनुः}
{अभिभूतः स रामेण संयुक्तः कालधर्मणा}


\twolineshloka
{अर्जुनस्याथ दायादा रामेण कृतमन्यवः}
{आश्रमस्थं विना रामं जमदग्निमुपाद्रवन्}


\twolineshloka
{ते तं जघ्नुर्महावीर्यमयुध्यन्तं तपस्विनम्}
{असकृद्रामरामेति विक्रोशन्तमनाथवत्}


\twolineshloka
{कार्तवीर्यस्य पुत्रास्तु जमदग्निं युधिष्ठिर}
{घातयित्वा शरैर्जग्मुर्यथागतमरिंदमाः}


\twolineshloka
{अपक्रान्तेषु वै तेषु जमदग्नौ तथा गते}
{समित्पाणिरुपागच्छदाश्रमं भृगुनन्दनः}


\twolineshloka
{स दृष्ट्वा पितरं वीरस्तदा मृत्युवशं गतम्}
{अनर्हं तं तथाभूतं विललाप सुदुःखितः}


\chapter{अध्यायः ११८}
\twolineshloka
{राम उवाच}
{}


\twolineshloka
{ममापराधात्तैः क्षुद्रैर्हतस्त्वं तात बालिशैः}
{कार्तवीर्यस्य दायादैर्वने मृग इवेषुभिः}


\twolineshloka
{धर्मज्ञस्य कथं तात वर्तमानस्य सत्पथे}
{मृत्युरेवंविधो युक्तः सर्वभूतेष्वनागसः}


\twolineshloka
{किंनु तैर्न कृतंपापं यैर्बवांस्तपसि स्तितः}
{अयुध्यमानो वृद्धः सन्हतः शरशतैः शितैः}


\threelineshloka
{किंनु ते तत्र वक्ष्यन्ति सचिवेषु सुहृत्सु च}
{अयुध्यमानं धर्मज्ञमेकं हत्वाऽनपत्रपाः ॥अकृतव्रण उवाच}
{}


\twolineshloka
{लालप्यैवं सकुस्न्णं बहु नानाविधं नृप}
{प्रेतकार्याणि सर्वाणि पितुश्चक्रे महातपाः}


\twolineshloka
{ददाह पितरं चाग्नौ रामः परपुरंजयः}
{प्रतिजज्ञे वधं चापि सर्वक्षत्रस्य भारत}


\twolineshloka
{संक्रुद्धोऽतिबलः सङ्ख्ये शस्त्रमादाय वीर्यवान्}
{जघ्निवान्कार्तवीर्यस्य सुतानेकोऽन्तकोपमः}


\twolineshloka
{तेषां चानुगता ये च क्षत्रियाः क्षत्रियर्षभ}
{तांश्च सर्वानवामृद्गाद्रामः प्रहरतांवरः}


\twolineshloka
{त्रिःसप्तकृत्वः पृथिवीं कृत्वा निःक्षित्रियां प्रभुः}
{समन्तपञ्चके पञ्च चकार रुधिरह्रदान्}


\twolineshloka
{स तेषु तर्पयामास पितॄन्भृगुकुलोद्वहः}
{साक्षाद्ददर्श चर्चीकं स च रामं न्यवारयत्}


\twolineshloka
{ततो यज्ञेन महता जामदग्न्यः प्रतापवान्}
{तर्पयामास देवेन्द्रमृत्विग्भ्यः प्रददौ महीम्}


\twolineshloka
{वेदीं चाप्यददद्धैमीं कश्यपाय महात्मने}
{दशव्यामायतां कृत्वा नवोत्सेधां विशांपते}


\twolineshloka
{तां कश्यपस्यानुमते ब्राह्मणाः खण्डशस्तदा}
{व्यभजंस्ते तदा राजन्प्रख्याताः खाण्डवायनाः}


\threelineshloka
{स प्रदाय महीं तस्मै कश्यपाय महात्मने}
{`तपः सुमहदास्थाय महाबलपराक्रमः'}
{अस्मिन्महेन्द्रे शैलेन्द्रे वसत्यमितविक्रमः}


\threelineshloka
{एवं वैरमभूत्तस्य क्षत्रियैर्लोकवासिभिः}
{पृथिवी चापि विजिता रामेणामिततेजसा ॥वैशंपायन उवाच}
{}


\twolineshloka
{ततश्चतुर्दशीं रामः समयेन महामनाः}
{दर्शयामास तान्विप्रान्धर्मराजं ज सानुजम्}


\twolineshloka
{स तमानर्च राजेन्द्र भ्रातृभिः सहितः प्रभुः}
{द्विजानां च परां पूजां चक्रे नृपतिसत्तमः}


\twolineshloka
{अर्जित्वा जामदग्न्यं स पूजितस्तेन चोदितः}
{महेन्द्र उष्य तां रात्रिं प्रययौ दक्षिणामुखः}


\chapter{अध्यायः ११९}
\twolineshloka
{वैशंपायन उवाच}
{}


\threelineshloka
{गच्छन्स तीर्थानि महानुभावः}
{पुण्यानि रम्याणि ददर्श राजा}
{सर्वाणि वैप्रैरुपशोभितानिक्वचित्क्वचिद्भारत सागरम्य}


\twolineshloka
{स वृत्तवांस्तेषु कृताभिषेकःसहानुजः पार्थिवपुत्रपौत्रः}
{समुद्रगां पुण्यतमां प्रशस्तांजगाम पारिक्षित पाण्डुपुत्रः}


\twolineshloka
{तत्रापि चाप्लुत्य महानुभावसंतर्पयामास पितृडन्सुरांश्च}
{द्विजातिमुख्येषु धनं विसृज्यगोदावरीं सागरगामगच्छत्}


\twolineshloka
{ततो विपाप्मा द्रविडेषु राज-न्समुद्रमासाद्य च लोकपुण्यम्}
{अगस्त्यतीर्थं च महापवित्रं-नारीतीर्थान्यथा वीरो ददर्श}


\twolineshloka
{तत्रार्जुनस्याग्र्यधनुर्धरस्यनिशम्य तत्कर्म परैरशक्यम्}
{संपूज्यमानः परमर्षिसङ्घैःपरां मुदं पाणअडुसुतः स लेभे}


\twolineshloka
{स तेषु तीर्थेष्वभिषिक्तगात्रःकृष्णासहायः सहितोऽनुजैश्च}
{संपूजयन्विक्रममर्जुनस्यरमे महीपालपतिः पृथिव्याम्}


\twolineshloka
{ततः सहस्राणि गवां प्रदायतीर्थेषु तेष्वम्बुधरोत्तमस्य}
{हृष्टः सह भ्रातृभिरर्जुनस्यसंकीर्तयामास गवां प्रदानम्}


\twolineshloka
{स तानि तीर्थानि च सागरस्यपुण्यानि चान्यानि बहूनि राजन्}
{क्रमेण गच्छन्परिपूर्णकामःशूर्पारकं पुण्यतमं ददर्श}


\twolineshloka
{तत्रोदधेः कंचिदतीत्य देशंख्यातं पृथिव्यां वनमाससाद}
{तप्तं सुरैरत्र तपः पुरस्ता-दिष्टुं कतथा पुण्यपरैर्नरेन्द्रैः}


\twolineshloka
{स तत्र तामग्र्यधनुर्धरस्यवेदीं ददर्शायतपीनबाहुः}
{ऋचीकपुत्रस्य तपस्विसङ्घैःसमावृतां पुण्यकृदर्चनीयाम्}


\twolineshloka
{ततो वसूनां वसुधाधिपः समरुद्गणानां च तथाश्विनोश्च}
{वैवस्वतादित्यधनेश्वराणा-मिन्द्रस्य विष्णोः सवितुर्विभोश्च}


\twolineshloka
{भगस्य चन्द्रस्य दिवाकरस्यपतेरपां साध्यगणस्य चैव}
{धातुः पितॄणां च तथा महात्मारुद्रस्य राजन्सगणस्य चैव}


\twolineshloka
{सरस्वत्याः सिद्धगणस्य चैवपूष्णश्च ये चाप्यमरास्तथाऽन्ये}
{पुण्यानि चाप्यायतनानि तेषांददर्श राजा सुमनोहराणि}


\twolineshloka
{तेषूपवासान्विबुधानुपोष्यदुत्त्वा च रत्नानि महान्ति राजा}
{तीर्थेषु सर्वेषु परिप्लुताङ्गःपुनः स शूर्पारकमाजगाम}


\twolineshloka
{स तेन तीरेण तु सागरस्यपुनः प्रयातः सह सोदरीयैः}
{द्विजैः पृथिव्यां प्रथितं महद्भि-स्तीर्थं प्रभासं समुपाजगाम}


\twolineshloka
{तत्राभिषिक्तः पृथुलोहिताक्षःसहानुजैर्देवगणान्पितॄंश्च}
{संतर्पयामास तथैव कृष्णाते चापि विप्राः सह लोमशेन}


\twolineshloka
{कस द्वादशाहं जलवायुभक्षःकुर्वन्क्षपाहःसु तदाऽभिषेकम्}
{समन्ततोऽग्नीनुपदीपयित्वातेपे तपो धर्मभृतांवरिष्ठः}


\twolineshloka
{तमुग्रमास्थाय तपश्चरन्तंशुश्राव रामश्च जनार्दनश्च}
{तौ सर्ववृष्णिप्रवरौ ससैन्यौयुधिष्ठिरं जग्मतुराजमीढम्}


\twolineshloka
{ते वृष्णयः पाण्डुसुतान्समीक्ष्यभूमौ शयानान्मलदिग्धगात्रान्}
{अनर्हतीं द्रौपदीं चापि दृष्ट्वासुदुःखिताश्चुक्रुशुरार्तनादान्}


\twolineshloka
{ततः स रामं च जनार्दनं चकार्ष्णि च साम्बं च शिनेश्च पौत्रम्}
{अन्यांश्च वृष्णीनुपगम्य पूजांचक्रे यथाधर्ममहीनसत्वः}


\twolineshloka
{ते चापि सर्वान्प्रतिपूज्य पार्थां-स्तैः सत्कृताः पाण्डुसुतैस्तथैव}
{युधिष्ठिरं संपरिवार्य राज-न्नुपाविशन्देवगणा यथेन्द्रम्}


\twolineshloka
{तेषां स सर्वं चरितं परेषांवने च वासं परमप्रतीतः}
{अस्त्रार्थमिन्द्रस्य गतं च पार्थंकृष्णे शशंसामरराजसूनुम्}


\twolineshloka
{श्रुत्वा तु ते तस्य वचः प्रतीता-स्तांश्चापि दृष्ट्वा सुकृशनतीव}
{नेत्रोद्भवं संमुमुचुर्महार्हादुःखार्तिजं वारि महानुभावाः}


\chapter{अध्यायः १२०}
\twolineshloka
{जनमेजय उवाच}
{}


\twolineshloka
{प्रभासतीर्थमासाद्यपाण्डवा वृष्णयस्तथा}
{किमकुर्वन्कथाश्चैषां कास्तत्रासंस्तपोधन}


\threelineshloka
{ते हि सर्वे महात्मानः सर्वशास्त्रविशारदाः}
{वृष्णयः पाण्डवाश्चैव सुहृदश्चपरस्परम् ॥वैशंपायन उवाच}
{}


\twolineshloka
{प्रभासतीर्थं संप्राप्य पुण्यं तीर्थं महोदधेः}
{वृष्णयः पाण्डवान्वीराः परिवार्योपतस्थिरे}


\twolineshloka
{ततो गोक्षीरकुन्देन्दुमृणआलरजतप्रभः}
{वनमाली हली रामो बभाषे पुष्करेक्षणम्}


\twolineshloka
{न कृष्ण धर्मश्चरितो भवायजन्तोरधर्मश्च पराभवाय}
{युधिष्ठिरो यत्रजटी महात्मावनाश्रयः क्लिश्यति चीरवासाः}


\twolineshloka
{दुर्योधनश्चापि महीं प्रशास्तिन चास्य भूमिर्विवरं ददाति}
{धर्मादधर्मश्चरितो वरीया-नितीव मन्येत नरोऽल्पबुद्धिः}


\threelineshloka
{दुर्योधने चापि विवर्धमानेयुधिष्ठिरे चासुखमात्तराज्ये}
{`हृतस्वराज्यायतनार्थभार्येदुर्योधनेनाल्पधिया च पार्थे'}
{किंन्वत्र कर्तव्यमिति प्रजाभिःशङ्का मिथः संजनिता नराणाम्}


\twolineshloka
{अयं स धर्मप्रभवो नरेन्द्रोधर्मे धृतः सत्यधृतिः प्रदाता}
{चलेद्धि राज्याच्च सुखाच्च पार्थोधर्मादपेतस्तु कथं विवर्धेत्}


\twolineshloka
{कथं नु भीष्मश्च कृपश्च विप्रोद्रोणश्च राजा च कुलस्य वृद्धः}
{प्रव्राज्य पार्थान्सुमाप्नुवन्तिधिक्पापबुद्धीन्भरतप्रधानान्}


\twolineshloka
{किंनाम वक्ष्यत्यवनिप्रधानःपितृडन्समागम्य परत्र पापः}
{पुत्रेषु सम्यक्वरितं मयेतिपुत्रानपापान्व्यपरोप्य राज्यात्}


\twolineshloka
{नासौ धिया संप्रति पश्यति स्मकिंनाम कृत्वाऽहमचक्षुरेवम्}
{जातः पृथिव्यामिति पार्थिवेषुप्रव्राज्यकौन्तेयमिति स्म राज्यात्}


\twolineshloka
{नूनं समृद्धान्पितृलोकभूमौचामीकराभान्क्षितिजान्प्रफुल्लान्}
{विचित्रवीर्यस्य सुतः सपुत्रःकृत्वा नृशंसं वत पश्यति स्म}


\twolineshloka
{व्यूढोत्तरांसान्पृथुलोहिताक्षा-निमाम्स्म पृच्छन्स शृणोति नूनम्}
{प्रास्थापयद्यत्स वनं सशङ्कोयुधिष्ठिरं सानुजमात्तशस्त्रम्}


\twolineshloka
{योऽयं परषां पृतनां समृद्धांनिरायुधो दीर्घभुजो निहन्यात्}
{श्रुत्वैव शब्दं हि वृकोदरस्यमुञ्चन्ति सैन्यानि शकृत्समूत्रम्}


\twolineshloka
{स क्षुत्पिपासाध्वकृशस्तरस्वीसमेत्य नानायुधबाणपाणिः}
{वने स्मरन्वासमिमं सुघोरंशेषं न कुर्यादिति निश्चितं मे}


\twolineshloka
{न ह्यस् वीर्येण बलेन कश्चि-त्समः पृथिव्यामपि विद्यतेऽन्यः}
{स शीतवातातपकर्शिताङ्गोन शेषमाजावसुहृत्सु कुर्यात्}


\twolineshloka
{प्राच्यां नृपानेकरथेन जित्वावृकोदरः सानुचरान्रणेषु}
{स्वस्त्यागमद्योऽतिरथस्तरस्वीसोयं वने क्लिश्यति चीरवासाः}


\twolineshloka
{यः सिन्धुकूले व्यजयन्नृदेवा-न्समागतान्दाक्षिणात्यान्महीपान्}
{तं पश्यतेमं सहदेवमद्यतरस्विनं तापसवेषरूपम्}


\twolineshloka
{यः पार्थिवानेकरथेन वीरोदिशं प्रतीचीं प्रत्नियुद्धशौण्डः}
{जिग्ये रणे तं नकुलं वनेऽस्मि-न्संपश्यतैनं मलदिग्धगात्रम्}


\twolineshloka
{सत्रे समृद्धेऽतिरथस्य राज्ञोवेदीतलादुत्पतिता सुता या}
{सेयं वने वासमिमं सुदुःखंकथं सहत्यद्य सती सुखार्हा}


\twolineshloka
{त्रिवर्गमुख्यस् समीरणस्यदेवेश्वरस्याप्यथवाऽश्विनोश्च}
{एषां सुराणां तनयाः कथंनुवने चरन्त्यस्तसुखाः सुखार्हाः}


\twolineshloka
{जिते हि धर्मस्य सुते सभार्येसभ्रातृके सानुचरे निरस्ते}
{दुर्योधने चापि विवर्धमानेकथं न सीदत्यवनिः सशैला}


\chapter{अध्यायः १२१}
\twolineshloka
{सात्यकिरुवाच}
{}


\twolineshloka
{न राम कालः परिदेवनाययदुत्तरं त्वत्र तदेव सर्वे}
{समाचरामो ह्यनतीतकालंयुधिष्ठिरो यद्यपि नाह किंचित्}


\twolineshloka
{ये नाथवन्तोऽद्य भवन्ति लोकेते नात्मना कर्म समारन्ते}
{तेषां तु कार्येषु भवन्ति नाथाःशैब्यादयो राम यथा ययातेः}


\twolineshloka
{येषां तथा राम समारभन्तेकार्याणि नाथाः स्वमतेन लोके}
{रते नाथवन्तः पुरुषप्रवीरानानाथवत्कृच्छ्रमवाप्नुवन्ति}


\twolineshloka
{कस्मादिमौ रामजनार्दनौ चप्रद्युम्नसाम्बौ च मया समेतौ}
{वसन्त्यरण्ये सह सोदरीयै-स्त्रैलोक्यनाथानभिगम्य पार्थाः}


\twolineshloka
{निर्यातु साध्यद्य दशार्हसेनाप्रभूतनानायुधचित्रवर्मा}
{यमक्षयं गच्छतु धार्तराष्ट्रःसबान्धवो वृष्णिबलाभिभूतः}


\twolineshloka
{त्वं ह्येव कोपात्पृथिवीमपीमांसंवेष्टयेस्तिष्ठतु शार्ह्गधन्वा}
{स धार्तराष्ट्रं जहि सानुबन्धंवृत्रं यथा देवपतिर्महेन्द्रः}


\twolineshloka
{भ्राता च मे यः स सखा गुरुश्चजनार्दनस्यात्मसमश्च पार्थः}
{तदर्थमेको हि य उद्यमन्वैकरोति कर्णोऽस्त्रमवारणीयम्}


\twolineshloka
{[यदर्थमैच्छन्मनुजाः सुपुत्रंशिष्यं गुरुं चाप्रतिकूलवादम्}
{यदर्थमभ्युद्यतमुत्तमं त-त्करोति कर्माग्र्यमपारणीयम् ॥]}


\twolineshloka
{तस्यास्त्रवर्षाण्यहमुत्तमास्त्रै-र्विहत्य सर्वाणइ रणेऽभिभूय}
{कायाच्छिरः सर्पविषाग्निकल्पैःशरोत्तमैरुन्मथिताऽस्मि राम}


\twolineshloka
{खङ्गेन चाहं निशितेन सङ्ख्येकायाच्छिरस्तस्य बलात्प्रमत्य}
{ततोऽस् सर्वाननुगान्हनिष्येदुर्योधनं चापि कुरूश्च सर्वान्}


\twolineshloka
{आत्तायुधं मामिह रौहिणएयपश्यन्तु भैमा युधि जातहर्षाः}
{निघ्नन्तमेकं कुरुयोधमुख्या-नग्निं महाकक्षमिवान्तकाले}


\threelineshloka
{प्रद्युम्नमुक्तान्निशितान्न शक्ताः}
{सोढुं कृपद्रोणविकर्णकर्णाः}
{जानासि पीर्यं च तवात्मजस्यकार्ष्णिर्भवत्येव यथा रणस्थः}


\twolineshloka
{साम्बः ससूतं सरथं भुजाभ्यांदुःशासनं शास्तु बलात्प्रमथ्य}
{न विद्यतेजाम्बवतीसुतस्यरणेऽविषह्यं हि रणोत्कटस्य}


\twolineshloka
{एतेन बालेन हि शम्बरस्यदैत्यस्य सैन्यं सहसा प्रणुन्नम्}
{`हतः स पापो युधि केवलेनयुद्धेऽद्वितीयो हरितुल्यवीर्यः'}


\twolineshloka
{वृत्तोरुरत्यायतपीनबाहु-रेतेन सङ्ख्ये निहतोऽश्वचक्रः}
{को नाम साम्बस्य महारथस्यरणे समक्षं रथमभ्युदीयात्}


\twolineshloka
{यथा प्रविश्यान्तरमन्तकस्यकाले मनुष्यो न विनिष्क्रमेत}
{तथा प्रविश्यान्तरमस्य सङ्ख्येको नाम जीवन्पुनराव्रजेच्च}


\twolineshloka
{द्रोणं कच भीष्मं च महारथौ तौसुतैर्वृतं चाप्यथ सोमदत्तम्}
{सर्वाणइ सैन्यानि च वासुदेवःप्रधक्ष्यते सायकवह्निजालैः}


\twolineshloka
{किंनाम लोकेषु विषह्ममस्ति-कृष्णस्य सर्वेषु सदेवकेषु}
{आत्तायुधस्योत्तमबाणपाणे-श्चक्रायुधस्याप्रतिमस्य युद्धे}


\twolineshloka
{ततो निरुद्धोऽप्यसिचर्मपाणि-र्महीमिमां धार्तराष्ट्रैर्विसंज्ञैः}
{हृतोत्तमाङ्गैर्निहतैः करोतुकीर्णाङ्कुशैर्वेदिमिवाध्वरेषु}


\twolineshloka
{गदोल्मुकौ बाहुकभानुनीथाःशूरश्र सङ्ख्ये निशठः कुमारः}
{रणोत्कटौ सारणचारुदेष्णौकुलोचितं विप्रथयन्तु कर्म}


\twolineshloka
{सवृष्णिभोजान्धकयोधमुख्यासमागता सात्वतशूरसेना}
{हत्वा रणे तान्धृतराष्ट्रपुत्रा-न्लोके यशः स्फीतमुपाकरोतु}


\twolineshloka
{ततोऽभिमन्युः पृथिवीं प्रशास्तुयावद्व्रतं धर्मभृतांवरिष्ठः}
{युधिष्ठिरः पारयते महात्माद्यूते यथोक्तं कुरुसत्तमेन}


\threelineshloka
{अस्मत्प्रमुक्तैर्विशिखैर्जितारि-स्ततो महीं भोक्ष्यति धर्मराजः}
{निर्धार्तराष्ट्रां हतसूतपुत्रा-मेतद्धि नः कृत्यतमं यशस्यम् ॥वासुदेव उवाच}
{}


\twolineshloka
{असंशयं माधव सत्यमेत-द्गृह्णीम ते वाक्यमदीनसत्व}
{स्वाभ्यां भुजाभ्यामजितां तु भूमिंनेच्छेत्कुरूणामृषभः कथंचित्}


\twolineshloka
{न ह्येष कामान्न भयान्न लोभा-द्युधिष्ठिरो जातु जह्यात्स्वधर्मम्}
{भीमार्जुनौ चातिरथौ यमौ चतथैव कृष्णा द्रुपदात्मजेयम्}


\twolineshloka
{उभौ हि युद्धेऽप्रतिमौ पृथिव्यांवृकोदरश्चैव धनंजयश्च}
{कस्मान्न कृत्स्नां पृथिवीं प्रशासे-न्माद्रीसुताभ्यां च पुरस्कृतोऽयम्}


\threelineshloka
{यदा तु पाञ्चालपतिर्महात्मासकेकयश्चेदिपतिर्वयं च}
{युध्येम विक्रम्य रणे समेता-स्तदैव सर्वे रिपवो हि न स्युः ॥युधिष्ठिर उवाच}
{}


\twolineshloka
{नेदं चित्रं माधव यद्ब्रवीषिसत्यं तु मे रक्ष्यतमं न राज्यम्}
{कृष्णस्तु मां वेद यथावदेकःकृष्णं च वेदाहमथो यथावत्}


\twolineshloka
{यदैव कालं पुरुषप्रवीरोवेत्स्यत्ययं माधव विक्रमस्य}
{तदा रणे त्वं च शिनिप्रवीरसुयोधनं जेष्यसि केशवश्च}


\threelineshloka
{प्रतिप्रयान्त्वद्य दशार्हवीरादृष्टोस्मि नाथैर्नरलोकनाथैः}
{धर्मेऽप्रमादं कुरुताप्रमेयाद्रष्टाऽस्मि भूयः सुखिनः समेतान् ॥वैशंपायन उवाच}
{}


\twolineshloka
{तेऽन्योन्यमामन्त्र्य तथाऽभिवाद्यवृद्धान्परिष्वज्य शिशूंश्च सर्वान्}
{यदुप्रवीराः स्वगृहाणि जग्मु-स्ते चापि तीर्थान्यमुसंविचेरुः}


\twolineshloka
{विसृज्य वृष्णीननु धर्मराजौविदर्भराजोपचितां सुतीर्थाम्}
{जगाम पुण्यां सरितं पयोष्णींसभ्रातृभृत्यः सह लोमशेन}


\twolineshloka
{सुतेन सोमेन विमिश्रतोयांपयः पयोष्णीं प्रति सोध्युवास}
{द्विजातिमुख्यैर्मुदितैर्महात्मासंस्तूयमानः स्तुतिभिर्वराभिः}


\chapter{अध्यायः १२२}
\twolineshloka
{लोमश उवाच}
{}


\twolineshloka
{गयेन यजमानेन सोमेनेह पुरंदरः}
{तर्पित श्रूयते राजन्स तृप्तो मुदमभ्यगात्}


\twolineshloka
{इह देवैः सहेन्द्रैश्च प्रजापतिभिरेव च}
{इष्टं बहुविधैर्यज्ञैर्महद्भिर्भूरिदक्षिणैः}


\twolineshloka
{आधूर्तरजसश्चेह राजा वज्रधरं प्रभुः}
{तर्पयामास सोमेन हयमेधेषु सप्तसु}


\twolineshloka
{तस्य सप्तसु यज्ञेषु सर्वमासीद्धिरण्ययम्}
{वानप्रस्थं च भौमं च यद्द्रव्यं नियतं मखे}


\twolineshloka
{चषालयूपचमसाः स्थाल्यः पात्र्यः स्रुचः स्रुवाः}
{तेष्वेव चास्य यज्ञेषु प्रयोगाः सप्त विश्रुताः}


\threelineshloka
{सप्तैकैकस्ययूपस्य चषालाश्चोपरिश्थिताः}
{तस्य स्म यूपान्यज्ञेषु भ्राजमानान्हिरण्मयान्}
{स्वयमुत्थापयामासुर्देवाः सेन्द्रा युधिष्ठिर}


\twolineshloka
{तेषु तस्य मखाग्र्येषु गयस्य पृथिवीपतेः}
{अमाद्यदिन्द्रः सोमेन दक्षिणाभिर्द्विजातयः}


% Check verse!
प्रसङ्ख्यानानसङ्ख्येयान्प्रत्यगृह्णन्द्विजातयः
\twolineshloka
{सिकता वा यथा लोके यथावा दिवि तारकाः}
{यथा वा वर्षतो धारा असङ्ख्येयाः स्म केनचित्}


\twolineshloka
{तथैव तदसङ्ख्येयं धनं यत्प्रददौ गयः}
{सदस्येभ्यो महाराज तेषु यज्ञेषु सप्तसु}


\twolineshloka
{भवेत्सङ्ख्येयमेतद्धि यदेतत्परिकीर्तितम्}
{न तस्य शक्याः सङ्ख्यातुं दक्षिणादक्षिणावतः}


\twolineshloka
{हिरण्मयीभिर्गोभिश्च कृताभिर्विश्वकर्मणा}
{ब्राह्मणांस्तर्पयामास नानादिग्भ्यः समागतान्}


\twolineshloka
{अल्पावशेषा पृथिवी चैत्यैरासीत्समाचिता}
{गयस्य यजमानस्य तत्रतत्र विशांपते}


\twolineshloka
{स लोकान्प्राप्तवैनैन्द्रान्कर्मणा तेन भरत}
{सलोकतां तस्य गच्छेत्पयोष्ण्यां य उपस्पृशेत्}


\threelineshloka
{तरस्मात्त्वमत्र राजेनद्र भ्रातृभिः सहितोच्युत}
{उपस्पृश्य महीपाल धूतपाप्मा भविष्यसि ॥वैशंपायन उवाच}
{}


\twolineshloka
{स पयोष्ण्यां नरश्रेष्ठः स्नात्वा वै भ्रातृभिः सह}
{वैदूर्यपर्वतं चैव नर्मदां च महानदीम्}


\twolineshloka
{`उद्दिश्य पाण्डवश्रेष्ठः स प्रतस्थे महीपतिः'}
{समागमत तेजस्वी भ्रातृभिः सहितो नघ}


\twolineshloka
{तत्रास्य सर्वाण्याचख्यौ लोमशो भगवानृषिः}
{तीर्थानि रमणीयानि पुण्यान्यायतनानि च}


\threelineshloka
{यथायोगं यथाप्रीति प्रययौ भ्रातृभिः सह}
{तत्रतत्राददद्वित्तं ब्राह्मणेभ्यः सहस्रशः ॥लोमश उवाच}
{}


\twolineshloka
{देवानामेति कौन्तेय तथा राज्ञां सलोकताम्}
{वेदूर्यपर्वतं दृष्ट्वा नर्मदामवतीर्य च}


\twolineshloka
{सन्धिरेष नरश्रेष्ठ त्रेताया द्वापरस्य च}
{एनमासाद्य कौन्तेय सर्वपापैः प्रमुच्यते}


\twolineshloka
{एष शर्यातियज्ञस्य देशस्तात प्रकाशते}
{साक्षाद्यत्रापिबत्सोममश्विभ्यां सह वासवः}


\twolineshloka
{चुकोप भार्गवश्चापि महेन्द्रस् महातपाः}
{संस्तम्भयामास च तं वासवं च्यवनः प्रभुः}


\threelineshloka
{सुकन्यां चापि भार्यां स राजपुत्रीमवाप्तवान्}
{`नासत्यौ च महाभाग कृतवान्सोमपीथिवौ' ॥युधिष्ठिर उवाच}
{}


\twolineshloka
{कथं विष्टम्भितस्तेन भगवान्पाकशासनः}
{किमर्थं भार्गवश्चापि कोपं चक्रे महातपाः}


\twolineshloka
{नासत्यौ च कथं ब्रह्मन्कृतवान्सोमपीथिनौ}
{एतत्सर्वं यथावृत्तमाख्यातु भगवान्मम}


\chapter{अध्यायः १२३}
\twolineshloka
{लोमश उवाच}
{}


\twolineshloka
{भृगोर्महर्षेः पुत्रोऽभूच्चयवनो नाम भारत}
{समीपे सरसस्तस्य तपस्तेपे महाद्युतिः}


\twolineshloka
{स्थाणुभूतो महातेजा वीरस्थानेन पाण्डव}
{अतिष्ठत चिरं कालमेकदेशे विशांपते}


\twolineshloka
{सवल्मीकोऽभवदृषिर्लताभिरिव संवृतः}
{कालेन महता राजन्समाकीर्णः पिपीलिकैः}


\twolineshloka
{तथा स संवृतो धीमान्मृत्पिण्ड इव सर्वशः}
{तप्यते स्म तपो घोरं वल्मीकेन समावृतः}


\twolineshloka
{अथ दीर्घस्य कालस्य शर्यातिर्नाम पार्थिवः}
{आजगाम सरो रम्यं विहर्तुमिदमुत्तमम्}


\twolineshloka
{तस्य स्त्रीणां सहस्राणि चत्वार्यासन्परिग्रहः}
{एकैव च सुता सुभ्रूः सुकन्या नाम भारत}


\twolineshloka
{सा सखीभिः परिवृता दिव्याभरणभूषिता}
{चङ्क्रम्यमाणा वल्मीकं भार्गवस्य समासदत्}


\twolineshloka
{सा वै वसुमतीं तत्र पश्यन्ती सुमनोरमाम्}
{वनस्पतीन्प्रचिन्वन्ती विजहार सखीवृता}


\twolineshloka
{रूपेण वयसा चैव भदनेन मदेन च}
{बभञ्ज वनवृक्षाणां शाखाः परमपुष्पिताः}


\twolineshloka
{तां सखीरहितामेकामेकवस्त्रामलंकृताम्}
{ददर्श भार्गवो धमांश्चरन्तीमिव विद्युतम्}


\threelineshloka
{तां पश्यमानो विजने स रेमे परमद्युतिः}
{क्षामकण्ठश्च विप्रर्षिस्तपोबलसमन्वितः}
{तामाबभाषेकल्याणीं सा चास्य न शृणोति वै}


\threelineshloka
{ततः सुकन्या वल्मीके दृष्ट्वा भार्गवचक्षुषी}
{कौतूहलात्कण्टकेन बुद्धिमोहबलात्कृता}
{किंनु खल्विदमित्युक्त्वा निर्बिभेदास्य लोचने}


\twolineshloka
{अक्रुद्ध्यत्स तथा विद्धे नेत्रे परममन्युमान्}
{ततः शर्यातिसैन्यस्य शकृन्मूत्रे समावृणोत्}


\twolineshloka
{ततो रुद्धे शकृन्मूत्रे सैन्यमासीत्सुदुःखितम्}
{तथागतमभिप्रेक्ष्य पर्यपृच्छत्स पार्थिवः}


\threelineshloka
{तपोनित्यस्य वृद्धस्य रोषणस्य विशेषतः}
{केनापकृतमद्येह भार्गवस्य महात्मनः}
{ज्ञातं वा यदि वाऽज्ञातं तद्द्रुतं ब्रूत माचिरम्}


\twolineshloka
{तमूचुः सैनिकाः सर्वे न विद्मोऽपकृतं वयम्}
{सर्वोपायैर्यथाकामं भवांस्तदधिगच्छतु}


\twolineshloka
{ततः स पृथिवीपालः साम्ना चोग्रेण च स्वयम्}
{पर्यपृच्छत्सुहृद्वर्गं पर्यजानन्न चैव ते}


\twolineshloka
{आनाहार्तं ततो दृष्ट्वा तत्सैन्यमनुखार्दितम्}
{पितरं दुःखितं दृष्ट्वा सुकन्येदमथाब्रवीत्}


\twolineshloka
{मयाऽटन्त्येह वल्मीके दृष्टं सत्वमभिज्वलत्}
{खद्योतवदभिज्ञातं तन्मया विद्धमन्तिकात्}


\twolineshloka
{एतच्छ्रुत्वा तु वल्मीकं शर्यातिस्तूर्णमभ्ययात्}
{तत्रापशय्त्तपोवृद्धं चन्द्रादित्यसमप्रभम्}


\twolineshloka
{अयाचदथ सैन्यार्थं प्राञ्जलिः पृथिवीपतिः}
{अज्ञानाद्बालया यत्ते कृतं तत्क्षन्तुमर्हसि}


\twolineshloka
{ततोऽब्रवीन्महीपालं च्यवनो भार्गवस्तदा}
{अपमानादहं विद्धो ह्यनया दर्पपूर्णया}


\fourlineindentedshloka
{रूपौदार्यसमायुक्तां लोभमोहबलात्कृताम्}
{तामेव प्रतिगृह्याहं राजन्दुहितरं तव}
{क्षंस्यामीति महीपाल सत्यमेतद्ब्रवीमि ते ॥लोमश उवाच}
{}


\twolineshloka
{ऋषेर्वचनमाज्ञाय शर्यातिरविचारयन्}
{ददौ दुहितरं तस्मै च्यवनाय महात्मने}


\twolineshloka
{प्रतिगृह्य च तां कन्यां भगवान्प्रससाद ह}
{प्राप्तप्रसादौ राजा वै ससैन्यः पुरमाव्रजत्}


\twolineshloka
{सुकन्याऽपि पतिं लब्ध्वा तपस्विनमनिन्दिता}
{नित्यं पर्यचरत्प्रीत्या तपसा नियमेन च}


\twolineshloka
{अग्नीनामतीथीनां च शुश्रूषुनसूयिका}
{समाराधयत क्षिप्रं च्यवनं सा शुभानना}


\chapter{अध्यायः १२४}
\twolineshloka
{लोमश उवाच}
{}


\twolineshloka
{कस्य चित्त्वथ कालस्य त्रिदशावश्विनौ नृप}
{कृताभिषेकां विवृतां सुकन्यां तामपश्यताम्}


\twolineshloka
{तां दृष्ट्वा दर्शनीयाङ्गीं देवराजसुतामिव}
{ऊचतुः समभिद्रुत्य नासत्यावश्विनाविदम्}


\twolineshloka
{कस्य त्वमसि वामोरु वनेऽस्मिन्किं करोषि च}
{इच्छाव भद्रे ज्ञातुं त्वां तत्वमाख्याहि शोभने}


\twolineshloka
{ततः सुकन्या संवीता तावुवाच सुरोत्तमौ}
{शर्यातितनयां वित्तं भार्यां मां च्यवनस्य च}


\twolineshloka
{`नाम्ना चाहं सुकन्येति नृलोकेऽस्मिन्प्रतिष्ठिता}
{साऽहंसर्वात्मना नित्यं भर्तारमनुवर्तिनी'}


\twolineshloka
{अथाश्विनौ प्रहस्यैतामब्रूतां पुनरेव तु}
{कथं त्वमसि कल्याणि पित्रा दत्ताऽऽगता वने}


\twolineshloka
{भ्राजसेऽस्मिन्वने भीरु विद्युत्सौदामनी यथा}
{न देवेष्वपि तुल्यां हि त्वया पश्याव भामिनि}


\twolineshloka
{अनाभरणसंपन्ना परमाम्बरवर्जिता}
{शोभयस्यधिकं भद्रे वनमप्यनलंकृता}


\twolineshloka
{सर्वाभरणसंपनना परमाम्वरधारिणी}
{शोभेथास्त्वनवद्याङ्गि न त्वेवं मलपङ्किनी}


\twolineshloka
{कस्मादेवंविधा भूत्वा जराजर्जरितं पतिम्}
{त्वमुपास्से ह कल्याणि कामभोगबहिष्कृतम्}


\twolineshloka
{असमर्थं परित्राणे पोषणे तु शुचिस्मिते}
{सा त्वं च्यवनमुत्सृज्यवरयस्वैकमावयोः}


\twolineshloka
{पत्यर्थं देवगर्भाभे मा वृथा यौवनं कृथाः}
{एवमुक्ता सुकन्याऽपि सुरौ ताविदमब्रवीत्}


\twolineshloka
{रताऽहं च्यवने पत्यौ मैवं मां पर्यशङ्कतम्}
{तावब्रूतां पुनस्त्वेनामावां देवभिषग्वरौ}


\threelineshloka
{युवानं रूपसंपन्नं करिष्यावः पतिं तव}
{ततस्तस्यावयोश्चैव वृणीष्वान्यतमं पतिम्}
{एतेन समयेनैनमामन्त्रय पतिं शुभे}


\twolineshloka
{सा तयोर्वचनाद्राजन्नुपसंगम्य भार्गवम्}
{उवाच वाक्यं यत्ताभ्यामुक्तं भृगुसुतं प्रति}


\twolineshloka
{तच्छ्रुत्वा च्यवनो भार्यामुवाच क्रियतामिति}
{`सा भर्त्रा समनुज्ञाता क्रियतामित्यथाब्रवीत्}


\twolineshloka
{श्रुत्वा तदश्विनौ वाक्यं तस्यास्तत्क्रियतामिति'}
{ऊचतू राजपुत्रीं तां पतिस्तव विशत्वपः}


\twolineshloka
{ततोऽम्भश्च्यवनः शीघ्रं रूपार्थी प्रविवेश ह}
{अश्विनावपि तद्राजन्सरः प्राविशतां तदा}


\threelineshloka
{ततो मुहूर्तादुत्तीर्णाः सर्वे ते सरसस्त्रयः}
{दिव्यरूपधराः सर्वे युवानो मृष्टकुण्डलाः}
{तुल्यवेषधराश्चैव मनसः प्रीतिवर्धनाः}


\twolineshloka
{तेऽब्रुवन्सहिताः सर्वे वृणीष्वान्यतमं शुभे}
{अस्माकमीप्सितं भद्रे पतित्वे वरवर्णिनि}


\twolineshloka
{`त्वमश्विनोरन्यतरं च्यवनं वा यशस्विनि'}
{यत्र वाऽप्यभिकामाऽसि तं वृणीष्व सुशोभने}


\twolineshloka
{सा समीक्ष्य तु तान्सर्वांस्तुल्यरूपधरान्स्थितान्}
{निश्चित्य मनसा बुद्ध्या देवी वव्रे स्वकं पतिम्}


\twolineshloka
{लब्ध्वा तु च्यवनो भार्यांवयोरूपं च वाञ्छितम्}
{हृष्टोऽब्रवीन्महातेजास्तौ नासत्याविदं वचः}


\twolineshloka
{यथाऽहं रूपसंपन्नो वयसा च समन्वितः}
{कृतो भवद्भ्यां वृद्धः सन्भार्यां च प्राप्तवानिमाम्}


\twolineshloka
{तस्माद्युवां करिष्यामि प्रीत्याऽहं सोमपीथिनौ}
{मिषतो देवराजस्य सत्यमेतद्ब्रवीमि वाम्}


\twolineshloka
{तच्छ्रुत्वा हृष्टमनसौ दिवं तौ प्रतिजग्मतुः}
{च्यवनश्च सुकन्या च सुराविव विजह्रतुः}


\chapter{अध्यायः १२५}
\twolineshloka
{लोमश उवाच}
{}


\twolineshloka
{ततः श्रुत्वा तु शर्यातिर्वयस्थं च्यवनं कृतम्}
{सुहृष्टः सेनया सार्धमुपायाद्भार्गवाश्रमम्}


\twolineshloka
{च्यवनं च सुकन्यां च दृष्ट्वा देवसुताविव}
{रेमे सभार्यः शर्यातिः कृत्स्नां प्राप्य महीमिव}


\twolineshloka
{ऋषिणा सत्कृतस्तेन सभार्यः पृथिवीपतिः}
{उपोपविष्टः कल्याणीः कथाश्चक्रे मनोरमाः}


\twolineshloka
{अथैनं भार्गवो राजन्नुवाच परिसान्त्वयन्}
{याजयिष्यामि राजंस्त्वां संभारानुपकल्पय}


\twolineshloka
{ततः परमसंहृष्टः शर्यातिरवनीपतिः}
{च्यवनस्य महाराज तद्वाक्यं प्रत्यपूजयत्}


\twolineshloka
{प्रशस्तेऽहनि यज्ञीये सर्वकामसमृद्धिमत्}
{कारयामास शर्यातिर्यज्ञायतनमुत्तमम्}


\twolineshloka
{तत्रैनं च्यवनो राजन्याजयामास भार्गवः}
{अद्भुतानि च तत्रासन्यानि तानि निबोध मे}


\threelineshloka
{अगृह्णाच्च्यवनः सोममश्विनोर्देवयोस्तदा}
{तमिन्द्रो वारयामास गृह्णानं स तयोर्ग्रहम् ॥इन्द्र उवाच}
{}


\threelineshloka
{उभावेतौ न सोमार्हौ नासत्याविति मे मतिः}
{भिषजौ दिवि देवानां कर्मणा तेन नार्हतः ॥च्यवन उवाच}
{}


\twolineshloka
{मावमंस्था महात्मानौ रूपद्रविणवत्तरौ}
{यौ चक्रतुर्मां मधवन्वृन्दारकमिवाजरम्}


\threelineshloka
{ऋते त्वां विबुधांश्चान्यान्कथं वै नार्हतः सवम्}
{अश्विनावपि देवेन्द्र देवौ विद्धि पुरंदर ॥इन्द्र उवाच}
{}


\threelineshloka
{चिकित्सकौ कर्मकरौ कामरूपसमन्वितौ}
{लोके चरन्तौ मर्त्यानां कथं सोममिहार्हतः ॥लोमश उवाच}
{}


\twolineshloka
{एतदेव तदा वाक्यमाम्रेडयति वासवे}
{अनादृत्यततः शक्रं ग्रहं जग्राह भार्गवः}


\twolineshloka
{ग्रहीष्यन्तं तु तं सोममश्विनोरुत्तमं तदा}
{समीक्ष्य बलभिद्देव इदं वचनमब्रवीत्}


\twolineshloka
{आभ्यामर्ताय सोमं त्वं ग्रहीष्यसि यदि स्वयम्}
{वज्रं ते प्रहरिष्यामि घोररूपमनुत्तमम्}


\twolineshloka
{एवमुक्तः स्मयन्निन्द्रमभिवीक्ष्यस भार्गवः}
{जग्राह विधिवत्सोममश्विभ्यामुत्तमं ग्रहम्}


\twolineshloka
{ततोऽस्मै प्राहरद्वज्रं घोररूपं शचीपतिः}
{तस्य प्रहरतो बाहुं स्तम्भयामास भार्गवः}


\twolineshloka
{तं स्तम्भयित्वा च्यवनो जुहुवे मन्त्रतोऽनलम्}
{कृत्यार्थी सुमहातेजा देवं हिंसितुमुद्यतः}


\twolineshloka
{ततः कृत्याऽथ संजज्ञे मुनेस्तस्य तपोबलात्}
{मदो नाम महावीर्यो बृहत्कायो महासुरः}


\twolineshloka
{शरीरं यस् निर्देष्टुमशक्यं तु सुरासुरैः}
{तस्यास्यमभवद्धोरं तीक्ष्णाग्रदशनं महत्}


\twolineshloka
{हनुरेका स्थिता त्वस्य भूमावेका दिवं गता}
{चतस्रश्चायता दंष्ट्रा योजनानां शतंशतम्}


\twolineshloka
{इतरे तस्य दशना बभूवुर्दशयोजनाः}
{प्रासादशिखराकाराः शूलाग्रसमदर्शनाः}


\twolineshloka
{बाहू परिघसंकाशावायतावयुतं समौ}
{नेत्रे रविशशिप्रख्ये वक्रं कालाग्निसंनिभम्}


\twolineshloka
{लेलिहञ्जिह्वया वक्रं विद्युच्चपललोलया}
{व्यात्ताननो घोरदृष्टिर्ग्रसन्निव जगद्बलात्}


\twolineshloka
{स भक्षयिष्यन्संक्रुद्धः शतक्रतुमुपाद्रवत्}
{महता घोररूपेण लोकाञ्शब्देन नादयत्}


\chapter{अध्यायः १२६}
\twolineshloka
{लोमश उवाच}
{}


\twolineshloka
{तं दृष्ट्वा घोरवदनं मदं देवः शतक्रतुः}
{आयान्तं भक्षयिष्यन्तं व्यात्ताननमिवान्तकम्}


\twolineshloka
{भयात्संस्तम्भितभुजः सृक्विणी लेलिहन्मुहुः}
{ततोऽव्रवीद्देवराजश्च्यवनं भयपीडितः}


\twolineshloka
{सोमार्हावश्विनावेतावद्य प्रभृति भार्गव}
{भविष्यतः सत्यमेतद्वचो विप्र प्रसीद मे}


\twolineshloka
{न ते मिथ्या समारम्भो भवत्वेष परो विधिः}
{जानामि चाहं विप्रर्षे न मिथ्या त्वं करिष्यसि}


\twolineshloka
{सोमपावश्विनावेतौ यथा वाद्य कृतौ त्वया}
{`तथैव मामपि ब्रह्मञ्श्रेयसा योक्तुमर्हसि'}


\twolineshloka
{भूय एव तु ते वीर्यं प्रकाशेदिति भार्गव}
{सुकन्यायाः पितुश्चास्य लोके कीर्तिः प्रथेदिति}


\twolineshloka
{अतो मयैतद्विहितं तव वीर्यप्रकाशनम्}
{तस्मान्प्रसादं कुरु मे भवत्वेवं यथेच्छसि}


\twolineshloka
{एवमुक्तस्य शक्रेण च्यवनस्य महात्मनः}
{स मन्युर्व्यगमच्छीघ्रं सुमोच च पुरंदरम्}


\twolineshloka
{मदं च व्यभजद्राजन्पाने स्त्रीषु च वीर्यवान्}
{अक्षेषु मृगयायां च पूर्वसृष्टं पुनः पुनः}


\twolineshloka
{तदा मदं विनिक्षिप्य शक्रं संतर्प्य चेन्दुना}
{अश्विभ्यां सहितान्देवान्याजयित्वा च तं नृपम्}


\twolineshloka
{विख्याप्य वीर्यं लोकेषु सर्वेषु वदतांवरः}
{सुकन्यया सहारण्ये विजहारानुकूलया}


\twolineshloka
{तस्यैतद्द्विजसंघुष्टं सरो राजन्प्रकाशते}
{अत्रत्वं सह सोदर्यैः पितृडन्देवांश्च तर्पय}


\twolineshloka
{एतद्दृष्ट्वा महीपाल सिकताक्षं च भारत}
{सैन्धवारण्यमासाद्य कुल्यानां कुरु दर्शनम्}


\twolineshloka
{पुष्करेषु महाराज सर्वेषु च जलं स्पृश}
{स्थाणोर्मन्त्राणि च जपन्सिद्धिं प्राप्स्यसि भारत}


\threelineshloka
{संधिर्द्वयोर्नरश्रेष्ठ त्रेताया द्वापरस्य च}
{अयं हि दृश्यते पार्थ सर्वपापप्रणाशनः}
{अत्रोपस्पृश चैव त्वं सर्वपापप्रणाशने}


\twolineshloka
{आर्चीकपर्वतश्चैव निवासौ वै मनीषिणाम्}
{सदाफलः सदास्रोतो मरुतां स्तानमुत्तमम्}


\threelineshloka
{चैत्याश्चैते बहुविधास्त्रिदशानां युधिष्ठिर}
{एतच्चन्द्रमसस्तीर्थमृषयः पर्युपासते}
{वैखानसप्रभृतयो वालखिल्यास्तथैव च}


\twolineshloka
{शृङ्गाणि त्रीणि पुण्यानि त्रीणि प्रसर्वणानि च}
{सर्वाण्यनुपरिक्रम्य यथाकाममुपस्पृश}


\threelineshloka
{शान्तनुश्चात्र राजेन्द्र शुनकश्च नराधिपः}
{नरनारायणौ चोभौ तपस्तप्त्वा चिरं नृप}
{स्थानं सनातनं प्राप्तावीश्वरध्यानतत्परौ}


\twolineshloka
{इह नित्याश्रया देवाः पितरश्च महर्षिभिः}
{आर्चीकपर्वते तेपुस्तान्यजस्व युधिष्ठिर}


\twolineshloka
{इह ते वै चरून्प्राश्नन्नृषयश्च विशांपते}
{यमुना चाक्षयस्रोताः कृष्णश्चेह तपोरतः}


\twolineshloka
{यमौ च भीमसेनश्च कृष्णा चामित्रकर्शन}
{सर्वे चात्र गमिष्यामस्त्वयैव सह पाण्डव}


\twolineshloka
{एतत्प्रस्रवणं पुण्यमिन्द्रस्य मनुजेश्वर}
{यत्रधाता विधाता च वपरुणश्चोर्ध्वमागताः}


\twolineshloka
{इह तेऽप्यवसन्राजञ्शान्ताः परमधर्मिणः}
{मैत्राणामृजुबुद्धीनामयं गिरिवरः शुभः}


\twolineshloka
{एषा सा यमुना राजन्महर्षिगणसेविता}
{नानायज्ञचिता राजन्पुण्या पापभयापहा}


\twolineshloka
{अत्रराजा महेष्वासो मांधाताऽयजत स्वयम्}
{साहदेविश्च कौन्तेय सोमको ददतांवरः}


\chapter{अध्यायः १२७}
\twolineshloka
{युधिष्ठिर उवाच}
{}


\twolineshloka
{मान्धाता राजशार्दूलस्त्रिषु लोकेषु विश्रुतः}
{कथं जातो महाब्रह्मन्यौवनाश्वो नृपोत्तमः}


\twolineshloka
{कथं चैनां परां ख्यातिं प्राप्तवानमितद्युतिः}
{यस्य लोकास्त्रयो वश्या विष्णोरिव महात्मनः}


\twolineshloka
{एतदिच्छाम्यहं श्रोतुं चरितं तस्य धीमतः}
{`सत्यकीर्तेर्हि मान्धातुः कथ्यमानं त्वयाऽनघ'}


\threelineshloka
{यथा मान्धातृशब्दश्च तस्य शक्रसमद्युते}
{जन्म चाप्रतिवीर्यस् कुशलो ह्यसि भाषितुम् ॥लोमश उवाच}
{}


\twolineshloka
{शृणुष्वावहितो राजन्राज्ञस्तस्य महात्मनः}
{यथा मान्धातृशब्दो वै लोकेषु परिगीयते}


\twolineshloka
{इक्ष्वाकुवंशप्रभवो युवनाश्वो महीपतिः}
{सोऽयजत्पृथिवीपालः क्रतुभिर्भूरिदक्षिणैः}


\twolineshloka
{अश्वमेधसहस्रं च प्राप्य धर्मभृतांवरः}
{अन्यैश्च क्रतुभिर्मुख्यैरयजत्स्वाप्तदक्षिणैः}


\threelineshloka
{अनपत्यस्तु राजर्षिः स महात्मा महाव्रतः}
{मन्त्रिष्वाधाय तद्राज्यं वनित्यो बभूव ह}
{शास्त्रदृष्टेन विधिना संयोज्यात्मानमात्मवान्}


\twolineshloka
{स कदाचिन्नृपो राजन्नुपवासेन दुःखितः}
{पिपासाशुष्कहृदयः प्रविवेशाश्रमं भृगोः}


\twolineshloka
{तामेव रात्रिं राजेन्द्र महात्मा भृगुनन्दनः}
{इष्टिं चकार सौद्युम्नेर्महर्षिः पुत्रकारणात्}


\twolineshloka
{संभृतो मन्त्रपूतेन वारिणा कलशो महान्}
{तत्रातिष्ठत राजेन्द्र पूर्वमेव समाहितः}


\threelineshloka
{यत्प्राश्य प्रसवेत्तस्य पत्नी शक्रसमं सुतम्}
{`तद्वारि विहितं राजन्यस्मिन्नासीत्सुसंस्कृतम्'}
{तं न्यस्य वेद्यां कलशं सुषुपुस्ते महर्षयः}


\threelineshloka
{रात्रिजागरणाच्छ्रान्तान्सौद्युम्निः समतीत्य तान्}
{शुष्ककण्ठः पिपासार्तः पानीयार्थी भृशं नृपः}
{तं प्रविश्याश्रमं शान्तः पानीयं सोऽभ्ययाचत}


\twolineshloka
{तस्य श्रान्तस्य शुष्केण कण्ठेन क्रोशतस्तदा}
{नाश्रौषीत्कश्चन तदा शकुनेरिव वाशतः}


\twolineshloka
{ततस्तं कलशं दृष्ट्वा जलपूर्णं स पार्थिवः}
{अभ्यद्रवत वेगेन पीत्वा चाम्भो व्यवासृजत्}


\twolineshloka
{स पीत्वा शीतलं तोयं पिपासार्तो महीपतिः}
{निर्वाणमगमद्धीमान्सुसुखी चाभवत्तदा}


\twolineshloka
{ततस्ते प्रत्यबुध्यन्त मुनयः सतपोधनाः}
{निस्तोयं तं च कलशं ददृशुः सर्व एव ते}


\twolineshloka
{कस्य कर्मेदमिति ते पर्यपृच्छन्समागताः}
{युवनाश्वो ममेत्येवं सत्यं समभिपद्यत}


\twolineshloka
{न युक्तमिति तं प्राह भगवान्भार्गवस्तदा}
{सुतार्थं स्थापिता ह्यापस्तपसा चैव संभृताः}


\twolineshloka
{मया ह्यत्राहितं ब्रह्म तप आस्थाय दारुणम्}
{पुत्रार्थं तव राजर्षे महाबलपराक्रम}


\twolineshloka
{महाबलो महावीर्यस्तपोबलसमन्वितः}
{यः शक्रमपि वीर्येण गमयेद्यमसादनम्}


\twolineshloka
{अनेन विधिना राजन्मयैतदुपपादितम्}
{अब्भक्षणं त्वया राजन्न युक्तं कृतमद्य वै}


\twolineshloka
{न त्वद्य शक्यमस्माभिरेतत्कर्तुमतोऽनय्था}
{नूनं दैवकृतं ह्येतद्यदेवं कृतवानसि}


\twolineshloka
{पिपासितेन याः पीता विधिमन्त्रपुरस्कृताः}
{आपस्त्वया महाराज मत्तपोवीर्यसंभृताः}


\twolineshloka
{ताभ्यस्त्वमात्मना पुत्रमीदृशं जनयिष्यसि}
{विधास्यामो वयं तत्र तवेष्टिं परमाद्भुताम्}


\twolineshloka
{यथा शक्रसमं पुत्रं जनयिष्यसि वीर्यवान्}
{`न च प्राणैर्महाराज वियोगस्ते भविष्यति'}


\twolineshloka
{मा खिदस्त्वं हि राजेनद्र दैवं हि बलवत्तरम्'}
{गर्भधारणजं वाऽपि न खेदं समवाप्स्यसि}


\twolineshloka
{ततो वर्षशते पूर्णे तस्य राज्ञो महात्मनः}
{वामं पार्श्वं विनिर्भिद्य सुतः सूर्य इव स्थितः}


\twolineshloka
{निश्चक्राम महातेजा न च तं मृत्युराविशत्}
{युवनाश्वं नरपतिं तदद्भुतमिवाभवत्}


\twolineshloka
{ततः शक्रो महातेजास्तं दिदृक्षुरुपागमत्}
{ततो देवा महेन्द्रं तमपृच्छन्धास्यतीति किम्}


\threelineshloka
{प्रदेशिनीं ततोऽस्यास्ये शक्रः समभिसंदधे}
{मामयं धास्यतीत्येवं भाषिते चैव वज्रिणा}
{मांधातेति च नामास्य चक्रुः सेन्द्रा दिवौकसः}


\twolineshloka
{प्रदेशिनीं शक्रदत्तामाखाद्य स शिशुस्तदा}
{अवर्धत महातेजाः किष्कून्राजंस्त्र्योदश}


\twolineshloka
{वेदास्तं सधनुर्वेदा दिव्यान्स्त्राणि चेश्वरम्}
{उपतस्थुर्महाराजं ध्यातमात्राणि सर्वशः}


\twolineshloka
{धनुराजगवं नाम शराः शृङ्गोद्भवाश्च ये}
{अभेद्यं कवचं चैव सद्यस्तमुपशिश्रियुः}


\twolineshloka
{सोऽभिषिक्तो भगवता स्वयं शक्रेण भारत}
{धर्मेण व्यजयल्लोकांस्त्रीन्विष्णुरिव विक्रमैः}


\twolineshloka
{तस्याप्रतिहतं चक्रं प्रावर्तत महात्मनः}
{रत्नानि चैव राजर्षिं स्वयमेवोपतस्थिरे}


\twolineshloka
{तस्येयं वसुसंपूर्णा वसुधा वसुधाधिप}
{तेनेष्टं विविधैर्यज्ञैर्बहुभिः स्वाप्तदक्षिणैः}


\twolineshloka
{चितचैत्यो महातेजा धर्मान्प्राप्य च पुष्कलान्}
{शक्रस्यार्धासनं राजँल्लब्धवानमितद्युतिः}


\twolineshloka
{एकाह्ना पृथिवी तेन धर्मनित्येन धीमता}
{विजिता शासनादेव सरत्नाकरपत्तना}


\twolineshloka
{तस्य चैत्यैर्महाराज क्रतूनां दक्षिणावताम्}
{चतुरन्ता मही व्याप्ता नासीत्किंचिदनावृतम्}


\twolineshloka
{तेन पद्मसहस्राणि गवां दश महात्मना}
{ब्राह्मणेभ्यो महाराज दत्तानीति प्रचक्षते}


\twolineshloka
{तेन द्वादशवार्षिक्यामनावृष्ट्यां महात्मना}
{वृष्टं सस्यविवृद्ध्यर्थं मिषतो वज्रपाणिनः}


\twolineshloka
{तेन सोमकुलोत्पन्नो गान्धाराधिपतिर्महान्}
{शर्जन्निव महामेघः प्रमथ्य निहतः शरैः}


\twolineshloka
{प्रजाश्चतुर्विधास्तेन त्राता राजन्कृतात्मना}
{तेनात्मतपसा लोकास्तापिताश्चातितेजसा}


\twolineshloka
{तस्यैतद्देवयजनं स्थानमादित्यवर्चसः}
{यस्य पुण्यतमे देशे कुरुक्षेत्रेस्य मध्यतः}


\twolineshloka
{`तथा त्वमपि राजेन्द्र मान्धातेव महीपतिः}
{धर्मं कृत्वा महीं रक्ष स्वर्गलोकं गमिष्यसि'}


\threelineshloka
{एतत्ते सर्वमाख्यातं मान्धातुश्चरितं महत्}
{जन्म चाग्र्यं महीपाल यन्मां त्वं परिपृच्छसि ॥वैशंपायन उवाच}
{}


\twolineshloka
{एवमुक्तः स कौन्तेयो लोमशेन महर्षिणा}
{पप्रच्छानन्तरं भूयः सोमकं प्रति भारत}


\chapter{अध्यायः १२८}
\twolineshloka
{युधिष्ठिर उवाच}
{}


\threelineshloka
{कथंवीर्यः स राजाऽभूत्सोमको ददतांवरः}
{कर्माण्यस्य प्रभावं च श्रोतुमिच्छामि तत्त्वतः ॥लोमश उवाच}
{}


\twolineshloka
{युधिष्ठिरासीन्नृपतिः सोमको नाम धार्मिकः}
{तस्य भार्याशतं राजन्सदृशीनामभूत्तदा}


\twolineshloka
{स वै यत्नेन महता तासु पुत्रं महीपतिः}
{कंचिन्नासादयामास कालेन महता ह्यपि}


\twolineshloka
{कदाचित्तस् वृद्धस्य यतमानस्य धीमतः}
{जन्तुर्नाम सुतस्तस्य ज्येष्ठायां समजायत}


\twolineshloka
{तं जातं मातरः सर्वाः परिवार्य समासते}
{सततं पृष्ठतः कृत्वा कामभोगान्विशांपते}


\twolineshloka
{ततः पिपीलिका जन्तुं कदाचिददशत्स्फिचि}
{स दष्टो ह्यरुदद्राजंस्तेन दुःखेन बालकः}


\twolineshloka
{ततस्ता मातरः सर्वाः प्राक्रोशन्भृशदुःखिताः}
{प्रवार्य जनतुं सहसा स शब्दस्तुमुलोऽभवत्}


\twolineshloka
{तमार्तनादं सहसा शुश्राव स महीपतिः}
{अमात्यपर्षदो मध्ये उपविष्टः सहर्त्विजा}


\twolineshloka
{ततः प्रस्थापयामास किमेतदिति पार्थिवः}
{तस्मै क्षत्ता यथावृत्तमाचचक्षे सुतं प्रति}


\twolineshloka
{त्वरमाणः स चोत्थाय सोमकः सह मन्त्रिभिः}
{प्रविश्यान्तःपुरं पुत्रमाश्वासयदरिंदमः}


\threelineshloka
{सान्त्वयित्वा तु तं पुत्रं निष्क्रम्यान्तःपुरान्नृपः}
{ऋत्विजा सहितो राजन्सहामात्य उपाविशत् ॥सोमक उवाच}
{}


\twolineshloka
{धिगस्त्विहैकपुत्रत्वमपुत्रत्वं वरं भवेत्}
{नित्यातुरत्वाद्भूतानां शोक एवैकपुत्रता}


\twolineshloka
{इदं भार्याशतं ब्रह्मन्परीक्ष्यसदृशं प्रभो}
{पुत्रार्थिना मया वोढं न तासां विद्यते प्रजा}


\twolineshloka
{एकः कथंचिदुत्पन्नः पुत्रो जन्तुरयं मम}
{यतमानासु सर्वासु किंनु दुःखमतः परम्}


\twolineshloka
{वयश्च समतीतं मे सभार्यस्यं द्विजोत्तम}
{आसां प्राणाः समायत्ता मम चात्रैकपुत्रके}


\threelineshloka
{स्यात्तु कर्म तथा युक्तं येन पुत्रशतं भवेत्}
{महता लघुना वाऽपि कर्मणा दुष्करेण वा ॥ऋत्विगुवाच}
{}


\threelineshloka
{अस्ति चैतादृशं कर्म येन पुत्रशतं भवेत्}
{यदि शक्नोषि तत्कर्तुमथ वक्ष्यामि सोमक ॥सोमक उवाच}
{}


\threelineshloka
{कार्यं वा यदि वाऽकार्यं येन पुत्रशतं भवेत्}
{कृतमेवेति तद्विद्धि भगवान्प्रब्रवीतु मे ॥ऋत्विगुवाच}
{}


\twolineshloka
{यजस्व जन्तुना राजंस्त्वं मया वितते क्रतौ}
{ततः पुत्रशतं श्रीमद्भविष्यत्यचिरेण ते}


\twolineshloka
{वपार्या हूयमानायां धूममाघ्राय मातरः}
{ततस्ताः सुमहावीर्याञ्जनयिष्यन्ति ते सुतान्}


\twolineshloka
{तस्यामेव तु ते जन्तुर्भविता पुनरात्मजः}
{उत्तरे चास्य सौवर्णं लक्ष्म पार्श्वे भविष्यति}


\chapter{अध्यायः १२९}
\twolineshloka
{सोमक उवाच}
{}


\threelineshloka
{ब्रह्मन्यद्यद्यथा कार्यं तत्कुरुष्व तथातथा}
{पुत्रकामतया सर्वंकरिष्यामि वचस्तव ॥लोमश उवाच}
{}


\twolineshloka
{ततः स याजयामास सोमकं तेन जन्तुना}
{मातरस्तु बलातपुत्रमपाकार्षुः कृपान्विताः}


\twolineshloka
{हा हताः स्मेति वाशन्त्यस्तीव्रशोकसमाहताः}
{तं मातरः प्त्यकर्षन्गृहीत्वा दक्षिणे करे}


\twolineshloka
{सव्ये पाणौ गृहीत्वा तु याजकोपि स्म कर्षति}
{कुररीणामिवार्तानामपाकृष्य तु तं सुतम्}


\twolineshloka
{विशस्य चैनं विधिवद्वपामस्य जुहाव सः}
{वपायां हूयमानायां गन्धमाघ्राय मातरः}


\twolineshloka
{आर्ता निपेतुः सहसा पृथिव्यां कुरुनन्दन}
{सर्वाश्च गर्भानलबंस्ततस्ताः पार्थिवाङ्गनाः}


\twolineshloka
{ततो दशसु मासेषु सोमकस्य विशांपते}
{जज्ञे पुत्रशतं पूर्णं तासु सर्वासु भारत}


\twolineshloka
{जन्तुर्ज्येष्ठः समभवज्जनित्र्यामेव पूर्ववत्}
{स तासामिष्ट एवासीन्न तथाऽन्ये निजाः सुताः}


\twolineshloka
{तच्च लक्षणमस्यासीत्सौवर्णं पार्श्व उत्तरे}
{तस्मिन्पुत्रशते चाग्र्यः स बभूव गुणैरपि}


\twolineshloka
{ततः स लोकमगमत्सोमकस्य गुरुः परम्}
{अन्वक्षमेव पश्चात्तु सोमकोप्यगमत्परम्}


\twolineshloka
{अथ तं नरके घोरे पच्यमानं ददर्श सः}
{तमपृच्छत्किमर्थं त्वं नरके पच्यसे द्विज}


\twolineshloka
{तमब्रवीद्गुरुः सोऽथ पच्यमानोऽग्निना भृशम्}
{त्वं मया याजितो राजंस्तस्येदं कर्मणः फलम्}


\twolineshloka
{एतच्छ्रुत्वा स राजर्षिर्धर्मराजमथाब्रवीत्}
{अहमत्र प्रवेक्ष्यामि मुच्यतां मम याजकः}


\threelineshloka
{मत्कृते हि महाभागः पच्यते नरकाग्निना}
{`सोहमात्मानमाधास्ये नरके मुच्यतां गुरुः' ॥धर्म उवाच}
{}


\threelineshloka
{नान्य कर्तुः फलं राजन्नुपभुङ्क्ते कदाचन}
{इमानि तव दृश्यन्ते फलानि वदतांवर ॥सोमक उवाच}
{}


\twolineshloka
{पुण्यान्न कामये लोकानृतेऽहं ब्रह्मवादिनम्}
{इच्छाम्यहमननैव सह वस्तुं सुरालये}


\threelineshloka
{नरके वा धर्मराज कर्मणाऽस्य समो ह्यहम्}
{पुण्यापुण्यफलं देव सममस्त्वावयोरिदम् ॥धर्मराज उवाच}
{}


\threelineshloka
{यद्येवमीप्सितं राजन्भुङ्खास्य सहितः फलम्}
{तुल्यकालं सहानेन पश्चात्प्राप्स्यसि सद्गतिम् ॥लोमश उवाच}
{}


\twolineshloka
{स चकार तथा सर्वं राजा राजीवलोचनः}
{क्षीणपापश्च तस्मात्स विमुक्तो गुरुणा सह}


\twolineshloka
{लेभे कामाञ्शुभान्राजन्कर्मणा निर्जितान्स्वयम्}
{सह तेनैव विप्रेण गुरुणा स गुरुप्रियः}


\twolineshloka
{एष तस्याश्रमः पुण्यो य एषोग्रे विराजते}
{क्षान्त उष्यात्र षड्रात्रं प्राप्नोति सुगतिं नरः}


\twolineshloka
{एतस्मिन्नपि राजेन्द्र वत्स्यामो विगतज्वराः}
{षड्रात्रं नियतात्मानः स़ज्जीभव कुरूद्वह}


\chapter{अध्यायः १३०}
\twolineshloka
{`वैशंपायन उवाच}
{}


\threelineshloka
{सोमकस्याश्रमे पुण्ये धर्मराजो युधिष्ठिरः}
{षड्रात्रमुष्य नियतो भ्रात्रादिभिररिंदमः}
{तस्मान्निर्गम्य सहसा दिशं प्रायात्तथोत्तराम्}


\twolineshloka
{बहुदूरं ततो गत्वा वनं तत्र मनोहरम्}
{बहुपष्पफलाकीर्णं महानद्युपशोभितम्}


\twolineshloka
{दृष्ट्वा पप्रच्छ राजाऽसौ लोमशं मुनिसत्तमम्}
{किमिदं दृश्यते रम्यं वनं बहुमृगद्विजम्}


\twolineshloka
{बहुपुष्पफलोपेतं मुनिसङ्घैर्निषेवितम्}
{स्रवन्त्या च समायुक्तं महत्या पुण्यतोयया}


\fourlineindentedshloka
{ऋषीणामाश्रमाः पुण्या दृश्यन्ते विविधा मुने}
{कस्यायमाश्रमः पुण्यः कस्येमे मुनयोऽमलाः}
{एतदिच्छाम्यहं श्रोतुं वद त्वं वदतांवर ॥लोमश उवाच}
{}


\twolineshloka
{शृणु राजेनद््र भद्रं ते वनस्यास्य पुरातनम्}
{वृत्तान्तं निखिलेनाद्य प्रोच्यमानं मयाऽनघ}


\twolineshloka
{मृकण्डुपुत्रो मेधावी मार्कण्डेयो महामुनिः}
{बाल एव महाबुद्धिः सर्वविद्याविशारदः}


\twolineshloka
{मातापित्रोः प्रियं कुर्वंस्तपोर्थं वनमाविशत्}
{अत्राश्रमपदं कृत्वा तपस्तेपे सुदारुणम्}


\twolineshloka
{ग्रीष्मे पञ्चतपा भूत्वा वर्षास्वाकाशसंश्रयः}
{जलस्थः शिशिरे योगी बहुकालमवर्तत}


\threelineshloka
{ऊर्ध्वबाहुर्निरालम्बः पादाङ्गुष्ठाग्रविष्ठितः}
{जितेन्द्रियो जितप्राणश्चिन्तयन्दृढमव्ययम्}
{अनाहारो जितक्रोधश्चिरमेवमवर्तत}


\twolineshloka
{एतस्मिन्नन्तरे राजन्ननावृष्टिः सुदारुणा}
{संभूता सर्वसंहर्त्री तया दग्धं चराचरम्}


\twolineshloka
{अनावृष्ट्यां प्रवृत्तायां सर्वे च निधनं गताः}
{केचिदन्ये महात्मानो मुनयो द्विजपुङ्गवाः}


\threelineshloka
{ब्राह्मणाः क्षत्रिया वैश्याः शूद्रा सर्वाश्च योषितः}
{पशुपक्षिमृगाः सर्वे क्षुत्पिपासासमाकुलाः}
{कृशाः शुष्कोष्ठकण्ठाश्च श्रान्ता भ्रान्ता विचेतसः}


\twolineshloka
{केनचित्पुण्यशेषेण मार्कण्डेयस्य धीमतः}
{आश्रमं शनकैः प्राप्ता मूर्च्छिताः सहसाऽपतन्}


\twolineshloka
{समाधिविरतो योगी मार्कण्डेयो महातपाः}
{तद्वनं निबिडं दृष्ट्वा जनैः क्षुत्तृट्समाकलैः}


\threelineshloka
{दयार्द्रहृदयो योगी शिवं ध्यात्वा हृदम्बुजे}
{तेषां संरक्षणार्थाय वेदगन्धिस्वरेण सः}
{आजुहाव तदा गङ्गां तपोयोगेन भारत}


\twolineshloka
{गङ्गा समागता शीघ्रं तेनाहूताऽतिपावना}
{नाम्ना वेदनदीत्येव प्रख्याता लोकपावना}


\threelineshloka
{पर्जन्यश्च समाहूतः सुखं वर्षति भारत}
{सस्यानि च समृद्धानि फलमूलान्यनेकशः}
{संभूतान्यत्रराजेनद्र सर्वे ते रक्षिता जनाः}


\twolineshloka
{मार्कण्डेयं प्रशंसन्तो जनाः सर्वे द्विजात्तयः}
{चिरं सुखमवर्तन्त तेन संरक्षिता नृप}


\twolineshloka
{एवं विधाय रक्षां स सर्वेषां पुण्यकर्मणाम्}
{पुनश्चचार च तपः परमेश्वरतुष्टये}


% Check verse!
एवं बहुतिथे काले प्रादुरासीन्महेश्वरः
\twolineshloka
{दृष्ट्वा च सर्वदेवेशं चन्द्रमौलिमुमापतिम्}
{ब्रह्मविष्ण्वादिभिर्देवैः सिद्धविद्याधरोरगैः}


\threelineshloka
{गन्धर्वयक्षप्रवरैः सकिन्नरपतत्रिभिः}
{स्तूयमानं महादेवमव्ययं निष्कलं शिवम्}
{प्रणनाम मुनिर्भक्त्या साष्टाङ्गं च पुनः पुनः}


\twolineshloka
{प्रणम्योत्थाय सहसा बद्धाञ्जलिपुटो मुनिः}
{तुष्टाव विविधैः स्तोत्रैर्महादेवं जगत्पतिम्}


\twolineshloka
{तमुवाच महादेवो मार्कण्डेयं महामुनिम्}
{वरं वरय भद्रं ते वरदोस्मि मुने तव}


\twolineshloka
{एवं संबोधितस्तेन शिवेन परमात्मना}
{सगद्गदमिदं वाक्यमुवाच परमेश्वरम्}


\twolineshloka
{नान्यं वरं वृणे शंभो ---त्वत्पादपङ्कजे}
{भक्तिंह्यनन्यसुलभां स्व व्यभिचारिणीम्}


\twolineshloka
{एवमुक्तोऽथ मुनिना भरतगीश्वरेश्वरः}
{पुनरेवाब्रवीद्वाक्यं मार्कण्डेय महामुनिम्}


% Check verse!
सम्यगाराधितः पित्रा तव पुत्रार्थमादरात्
\twolineshloka
{शतायुर्निर्गुणः पुत्रः शुभः षोडशवार्षिकः}
{उभयोरन्यमिच्छ त्वमित्युक्तः सोऽब्रवीच्च माम्}


\twolineshloka
{निर्गुणो मास्तु देवेश शतायुः षोडशाब्दकः}
{सुगुणोऽस्तु सुतो मेऽद्य इति पित्रावृत पुरा}


\twolineshloka
{त्वया तप्तेन तपसा तोषितोऽहं भृशं मुने}
{दीर्घमायुर्मया दत्तं मृत्युश्च प्रतिषेधितः}


\twolineshloka
{इत्युक्त्वा भगवानीशस्तत्रैवान्तरधीयत}
{तस्यायमाश्रमः पुण्यस्तस्येमे मुनयोऽमलाः}


\twolineshloka
{अत्रैकरात्रमुषिताः सर्वे मृत्युं तरन्ति वै}
{अत्रैव भरतश्रेष्ठ प्रयतो वस भूमिप'}


\chapter{अध्यायः १३१}
\twolineshloka
{लोमश उवाच}
{}


\twolineshloka
{अस्मिन्किल स्वयं राजन्निष्टवान्वै प्रजापतिः}
{सत्रमिष्टीकृतं नाम पुरा वर्षसहस्रिकम्}


\twolineshloka
{अम्बरीषश्च नाभाग इष्टवान्यमुनामनु}
{यत्रेष्ट्वा दशपद्मानि सदस्येभ्योऽभिसृष्टवान्}


\twolineshloka
{यज्ञैश्च तपसा चैव परां सिद्धिमवाप सः}
{देशश्च नाहुषस्यायं य़ज्वनः पुण्यकर्मणः}


\twolineshloka
{सार्वभौमस्य कौन्तेय ययातेरमितौजसः}
{स्पर्धमानस्य शक्रेण तस्येदं यज्ञवास्त्विह}


\twolineshloka
{पश्य नानाविधाकारैरग्निभिर्निचितां महीम्}
{मज्जन्तीमिव चाक्रान्तां ययातेर्यज्ञकर्मभिः}


\twolineshloka
{एषा शम्येकपत्रा सा शकटं चैतदुत्तमम्}
{पश्य रामह्रदानेतान्पश्य नारायणाश्रमम्}


\twolineshloka
{एतच्चर्चीकपुत्रस्य योगैर्विचरतो महीम्}
{प्रसर्पणं महीपाल रौप्यायाममितौजसः}


\twolineshloka
{अत्रानुवंशं पठतः शृणु मे कुरुनन्दन}
{उलूखलैराभरणैः पिशची यदभाषत}


\twolineshloka
{युगन्धरे दधि प्राश्य उषित्वा चाच्युतस्थले}
{तद्वद्भूतलये स्नात्वा सपुत्रा वस्तुमर्हसि}


\twolineshloka
{एकरात्रमुवित्वेह द्वितीयं यदि वत्स्यसि}
{एतद्वै ते गदेवावृत्तं रात्रौ वृत्तमितोऽन्यथा}


\twolineshloka
{अद्य चात्र निवत्स्यामः क्षपां भरतसत्तम}
{द्वारमेतत्तु कौन्तेय कुरुक्षेत्रस्य भारत}


\twolineshloka
{अत्रैव नाहुषो राजा राजन्क्रतुभिरिष्टवान्}
{ययातिर्बहुरत्नौर्घर्यत्रेन्द्रो मुदमभ्यगात्}


\twolineshloka
{एतत्प्लक्षावतरणं यमुनातीर्थमुत्तमम्}
{एतद्वै नाकपृष्ठस्य द्वारमाहुर्मनीषिणः}


\twolineshloka
{अत्रसारस्वतैर्यज्ञैरीजानाः परमर्षयः}
{यूपोलूखलिकास्तात गच्छन्त्यवभृथप्लवम्}


\twolineshloka
{अत्रवै भरतो राजा राजन्क्रतुभिरिष्टवान्}
{हयमेधेन यज्ञेन मेध्यमश्वमवासृजत्}


\threelineshloka
{असकृत्कृष्णसारङ्गं धर्मेणाप्य च मेदिनीम्}
{अत्रैव पुरुषव्याघ्र मरुत्तः सत्रमुत्तमम्}
{प्राप चैवर्षिमुख्येन संवर्तेनाभिपालितः}


\threelineshloka
{अत्रोपस्पृश्य राजेन्द्र सर्वाल्लोँकान्प्रपश्यति}
{पूयते दुष्कृताच्चैव अत्रापि समुपस्पृश ॥वैशंपायन उवाच}
{}


\twolineshloka
{तत्र सभ्रातृकः स्नात्वा स्तूयमानो महर्षिभिः}
{लोमशं पाण्डवश्रेष्ठ इदं वचनमब्रवीत्}


\threelineshloka
{सर्वाँल्लोकान्प्रपश्यामि तपसा सत्यविक्रम}
{इहस्तः पाण्डवश्रेष्ठं पश्यामि श्वेतवाहनम् ॥लोमश उवाच}
{}


\twolineshloka
{एवमेतन्महाबाहो पश्यन्ति परमर्षयः}
{इह स्नात्वा तपोयुक्तांस्त्रील्लोँकान्सचराचरान्}


\twolineshloka
{सरस्वतीमिमां पुण्यां पुण्यैकशरणावृताम्}
{यत्र स्नात्वा नरश्रेष्ठ धूतपाप्मा भविष्यसि}


\twolineshloka
{इह सारस्वतैर्यज्ञैरिष्टवन्तः सुरर्षयः}
{ऋषयश्चैव कौन्तेय तथा राजर्षयोपि च}


\twolineshloka
{वेदी प्रजापतेरेषा समन्तात्पञ्चयोजना}
{कुरोर्वै यज्ञशीलस्य क्षेत्रमेतन्महात्मनः}


\chapter{अध्यायः १३२}
\twolineshloka
{लोमश उवाच}
{}


\twolineshloka
{इह मर्त्यास्तनूस्त्यक्त्वा स्वर्गं गच्छन्ति भारत}
{मर्तुकामा नरा राजन्निहायान्ति सहस्रशः}


\twolineshloka
{एवमाशीः प्रयुक्ता हि दक्षेण यजता पुरा}
{इह ये वै मरिष्यन्ति ते वै स्वर्गजितो नराः}


\twolineshloka
{एषा सरस्वती रम्या दिव्या चौघवती नदी}
{एतद्विनशनं नाम सरस्वत्या विशांपते}


\twolineshloka
{द्वारं निषादराष्ट्रस्य येषां दोषात्सरस्वती}
{प्रविष्टा पृथिवीं वीर मा निषादा हि मां विदुः}


\twolineshloka
{एष वै चमसोद्भेदो यत्र दृश्या सरस्वती}
{यत्रैनामभ्यवर्तन्त दिव्याः पुण्याः समुद्रगाः}


\twolineshloka
{एतत्सिन्धोर्महत्तीर्थं यत्रागस्त्यमरिंदम}
{लोपामुद्रा समागम्य भर्तारमवृणीत वै}


\twolineshloka
{एतत्प्रकाशते तीर्थं प्रभासं भास्करद्युते}
{इन्द्रस्य दयितं पुण्यं पवित्रं पापनाशनम्}


\twolineshloka
{एतद्विष्णुपदं नाम दृश्यते तीर्थमुत्तमम्}
{एषां रम्या विपाशा च नदी परमपावनी}


\twolineshloka
{अत्र वै पुत्रशोकेन वसिष्ठो भगवानृषिः}
{बद्ध्वाऽत्मानं निपतितो विपाशः पुनरुत्थितः}


\twolineshloka
{काश्मीरमण्डलं चैतत्सर्वपुण्यमरिंदम}
{महर्षिभिश्चाध्युषितं पश्येदं भ्रातृभिः सह}


\twolineshloka
{यत्रौत्तराणां सर्वेषामृषीणां नाहुषस्य च}
{अग्नेश्चैवात्र संवादः काश्यपश्य च भारत}


\twolineshloka
{एतद्द्वारं महाराज मानसस्य प्रकाशते}
{वर्षमस्य गिरेर्मध्ये रामेण श्रीमता कृतम्}


\twolineshloka
{एष वातिकषण्डो वै प्रख्यातः सत्यविक्रमः}
{नात्यवर्तत यद्द्वारं विदेहादुत्तरं च यः}


\threelineshloka
{इदमाश्चर्यमपरं देशेऽस्मिन्पुरुषर्षभ}
{क्षीणे युगे तु कौन्तेय शर्वस्य सह पार्षदैः}
{सहोमया च भवति दर्शनं कामरूपिणः}


\twolineshloka
{अस्मिन्सरसि सत्रैर्वै चैत्रे मासि पिनाकिनम्}
{यजन्ते याजकाः सम्यक् परिवारं शुभार्थिनः}


\twolineshloka
{अत्रोपस्पृश्य सरसि श्रद्दधानो जितेन्द्रियः}
{क्षीणपापः शुभाँल्लोकान्प्राप्नुते नात्र संशयः}


\twolineshloka
{एष उज्जानको नाम पावकिर्यत्र शान्तवान्}
{अरुन्धतीसहायश्च वसिष्ठो भगवानृषिः}


\twolineshloka
{ह्रदश्च कुशवानेष यत्र पद्मं कुशेशयम्}
{आश्रमश्चैव रुक्मिण्या यत्राशाम्यदकोपना}


\twolineshloka
{समाधीनां समासस्तु पाण्डवेय श्रुतस्त्वया}
{तं द्रक्ष्यसि महाराज भृगुतुन्दं महागिरिम्}


\twolineshloka
{वितस्तां पश्य राजेन्द्र सर्वपापप्रमोचनीम्}
{महर्षिभिश्चाध्युषितां शीततोयां सुनिर्मलाम्}


\twolineshloka
{जलां चोपजलां चैव यमुनामभितो नदीम्}
{उशीनरो वै यत्रेष्ट्वा वासवादत्यरिच्यत}


\twolineshloka
{तां देवसमितिं तस्य वासवश्च विशांपते}
{अभ्यागच्छन्नृपवरं ज्ञातुमग्निश्च भारत}


\twolineshloka
{जिज्ञासमानौ वरदौ महात्मानमुशीनरम्}
{इन्द्रः श्येनः कपोतोऽग्निर्भूत्वा यज्ञेऽभिजग्मतुः}


\twolineshloka
{उरुं राज्ञः समासाद्य कपोतः श्येनजाद्भयात्}
{शरणार्थी तदा राजन्निलिल्ये भयपीडितः}


\chapter{अध्यायः १३३}
\twolineshloka
{श्येन उवाच}
{}


\twolineshloka
{धर्मात्मानं त्वाहुरेकं सर्वे राजन्महीक्षितः}
{स वै धर्मविरुद्धं त्वं कस्मात्कर्म चिकीर्षसि}


\threelineshloka
{विहितं भक्षणं राजन्पीड्यमानस्य मे क्षुधा}
{माहिंसीर्धर्मलोभेन धर्ममुत्सृज्य मा नशः ॥राजोवाच}
{}


\twolineshloka
{संत्रस्तरूपस्त्राणार्थी त्वत्तो भीतो महाद्विज}
{मत्सकाशमनुप्रप्तः प्राणगृध्नुरयं द्विजः}


\twolineshloka
{एवमभ्यागतस्येह कपोतस्याभयार्थिनः}
{अप्रदाने परो धर्मः किं त्वं श्येनेह पश्यसि}


\twolineshloka
{प्रस्पन्दमानः संभ्रान्तः कपोतः श्येन लक्ष्यते}
{मत्सकाशं जीवितार्थी तस्य त्यागो विगर्हितः}


\threelineshloka
{[यो हि कश्चिद्द्विजान्हन्याद्गां वा लोकस्य मातरम्}
{शरणागतं च त्यजते तुल्यं तेषां हि पातकम्]श्येन उवाच}
{}


\twolineshloka
{आहारत्सर्वभूतानि संभवन्ति महीपते}
{आहारेण विवर्धन्ते तेन जीवन्ति जन्तवः}


\twolineshloka
{शक्यते दुस्त्यजेऽप्यर्थे चिररात्राय जीवितुम्}
{न तु भोजनमुत्सृज्य शक्यं वर्तयितुं चिरम्}


\twolineshloka
{भक्ष्याद्विलोपितस्याद्य मम प्राणा विशांपते}
{विसृज्यकायमेष्यन्ति पन्थानमपुनर्भवम्}


\twolineshloka
{प्रमृते मयि धर्मात्मन्पुत्रदारादि नङ्क्ष्यति}
{रक्षमाणः कपोतं त्वं बहून्प्राणान्न रक्षसि}


\twolineshloka
{बहून्यो बाधते धर्मो न स धर्मः कुवर्त्म तत्}
{अविरोधी तु यो धर्मः स धर्मः सत्यविक्रम}


\twolineshloka
{विरोधिषु महीपाल निश्चित्य गुरुलाघवम्}
{न बाधा विद्यते यत्र तं धर्मं समुपाचरेत्}


\threelineshloka
{गुरुलाघवमाज्ञाय धर्माधर्मविनिश्चये}
{यतो भूयांस्ततो राजन्कुरु धर्मविनिश्चयम् ॥राजोवाच}
{}


\twolineshloka
{बहुकल्याणसंयुक्तं भाषसे विहगोत्तम}
{सुपर्णः पक्षिराट् किं त्वं धर्मं ज्ञात्वाऽभिभाषसे}


\threelineshloka
{तथाहि धर्मसंयुक्तं बहुचित्रं च भाषसे}
{न तेऽस्त्यविदितं किंचिदिति त्वां लक्षयाम्यहम्}
{शरणैषिपरित्यागं कथं साध्विति मन्यसे}


\twolineshloka
{आहारार्थं समारम्भस्तव चायं विहंगम}
{शक्यश्चाप्यन्यथा कर्तुमाहारोऽप्यधिकस्त्वया}


\threelineshloka
{गोवृषो वा वराहो वा मृगो वा महिषोपि वा}
{त्वदर्थमद्य क्रियतां यच्चान्यदिह काङ्क्षसि ॥श्येन उवाच}
{}


\twolineshloka
{न वराहं न चोक्षाणं न मृगान्विविधांस्तथा}
{भक्षयामि महाराज किं ममान्येन केचचित्}


\twolineshloka
{यस्तु मे दैवविहितो भक्षः क्षत्रियपुङ्गव}
{तमुत्सृज महीपाल कपोतमिममेव मे}


\threelineshloka
{श्येनाः कपोतान्स्वादन्ति श्रुतिरेषा सनातनी}
{मा राजन्सारमज्ञात्वा कदलीस्कन्धमासज ॥राजोवाच}
{}


\twolineshloka
{राष्ट्रं शिवीनामृद्धं वै शाधि पक्षिभिरर्चितः}
{कृत्स्नमेतन्मया दत्तं राजवद्विहगोत्तम}


\twolineshloka
{यं वा कामयसे कामं श्येन सर्वं ददानि ते}
{विनेमं पक्षिणं श्यन शरणार्थिनमागतम्}


\threelineshloka
{येनेमं स्थापयेथास्त्वं कर्मणा पक्षिसत्तम}
{तदाचक्ष्व करिष्यामि न हि दास्ये कपोतकम् ॥श्येन उवाच}
{}


\twolineshloka
{उशीनर कपोते ते यदि स्नेहो नराधिप}
{आत्मनो मांसमुत्कृत्य कपोततुलया धृतम्}


\threelineshloka
{यदा समं कपोतेन तव मांसं नृपोत्तम}
{त्वया प्रदेयं तन्मह्यं सा मे तुष्टिर्भविष्यति ॥राजोवाच}
{}


\threelineshloka
{अनुग्रहमिमं मन्ये श्येन यन्माभियाचसे}
{तस्मात्तेऽद्य प्रदास्यामि स्वमांसं तुलया धृतम् ॥लेमश उवाच}
{}


\twolineshloka
{अथोत्कृत्य स्वमांसं तु राजा परमधर्मवित्}
{तुलयामास कौन्तेय कपोतेन समं विभो}


\twolineshloka
{ध्रियमाणः कपोतस्तु मांसेनात्यतिरिच्यते}
{पुनश्चोत्कृत्यमांसानि राजा प्रादादुशीनरः}


\threelineshloka
{न विद्यते यदा मांसं कपोतेन समं धृतम्}
{तत उत्कृत्तमांसोऽसावारुरोह स्वयं तुलाम् ॥श्येन उवाच}
{}


\twolineshloka
{इन्द्रोऽहमस्मि धर्मज्ञ कपोतो हव्यवाडयम्}
{जिज्ञासमानौ धर्म त्वां यज्ञवाटमुपागतौ}


\twolineshloka
{यत्ते मांसानि गात्रेभ्य उक्तृत्तानि विशांपते}
{एषा ते शाश्वती कीर्तिर्लोकानभिभविष्यति}


\twolineshloka
{यावल्लोके मनुष्यास्त्वां कथयिष्यन्ति पार्थिव}
{तावत्कीर्तिश्च लोकाश्च स्थास्यन्ति तव शाश्वताः}


\twolineshloka
{इत्युक्त्वा भूमिपतये तस्मै दत्त्वा यथेप्सितम्}
{प्रशस्य जग्मतू राजन्निन्द्राग्री तुष्टमानसौ}


\twolineshloka
{उशीनरोऽपिधर्मात्मा धर्मेणावृत्यरोदसी}
{विभ्राजमानो वपुषाऽप्यारुरोह त्रिविष्टपम्}


\twolineshloka
{तदेतत्सदनं राजन्राज्ञस्तस्य महात्मनः}
{पश्यस्वैतन्मया सार्धं पुण्यं पापप्रमोचनम्}


\twolineshloka
{तत्र वै सततं देवा मुनयश्च सनातनाः}
{दृश्यन्ते ब्राह्मणै राजन्पुण्यवद्भिर्महात्मभिः}


\chapter{अध्यायः १३४}
\twolineshloka
{लोमश उवाच}
{}


\twolineshloka
{यः कथ्यते मन्त्रविदग्र्यबुद्धि-रौद्दालकिः श्वेतकेतुः पृथिव्याम्}
{तस्याश्रमं पश्यत पाण्डवेयासदाफलैरुपपन्नं महीजैः}


\twolineshloka
{साक्षादत्र श्वेतकेतुर्ददर्शसरस्वतीं मानुषदेहरूपाम्}
{वेत्स्यामि वाणीमिति संप्रवृत्तांसरस्वतीं श्वेतकेतुर्बभाषे}


\twolineshloka
{अस्मिन्युगे ब्रह्मकृतां वरिष्ठा-वास्तां मुनी मातुलभागिनेयौ}
{अष्टावक्रश्चैव कहोलसूनु-रौद्दालकिः श्वेतकेतुः पृथिव्याम्}


\twolineshloka
{विदेहराजस् समीपतस्तौघीरावुभौ मातुलभागिनेयौ}
{प्रविश्य यज्ञायतनं विवादेवन्दिं निजग्राहतुरप्रमेयौ}


\fourlineindentedshloka
{उपास्स्व कौन्तेय सहानुजस्त्वंतस्याश्रमं पुण्यतमं प्रविश्य}
{अष्टावक्रं यस्य दौहित्रमाहु-र्योऽसौ वन्दिं जनकस्याथ यज्ञे}
{वादि विप्राग्न्यो बाल एवाभिगम्यवादे भङ्क्त्वा मज्जयामास नद्याम् ॥युधिष्ठिर उवाच}
{}


\twolineshloka
{कथंप्रभावः स बभूव विप्र-स्तथाभूतं यो निजग्राह वन्दिम्}
{`किंचाधिकृत्याथ तयोर्विवादोविदेहराजस् समीप आसीत्'}


\twolineshloka
{अष्टावक्रः केन चासौ बभूवतत्सर्वं मे लोमश शंस तत्त्वम् ॥लोमश उवाच}
{}


\twolineshloka
{उद्दालकस्य नियतः शिष्य एकोनाम्ना कहोळेति बभूव राजन्}
{शुश्रूषुराचार्यवशानुवर्तीदीर्घ कालं सोऽध्ययनं चकार}


\twolineshloka
{तं वै विप्राः पर्यभवंस्तु शिखा-स्तं च ज्ञात्वा विप्रकारं गुरुः सः}
{तस्मै प्रादात्सद्य एव श्रुतं चभार्यां च वै दुहितरं स्वां सुजाताम्}


\twolineshloka
{तस्यां गर्भः समभवदग्निकल्पःसोऽधीयानं पितरमथाभ्युवाच}
{सर्वां रात्रिमध्ययनं करोषिनेदं पितः सम्यगिवोपवर्तते}


\twolineshloka
{उपालब्धः शिष्यमध्ये महर्षिःस तं कोपादुदरस्थं शशाप}
{यस्माद्वक्रं वर्तमानो ब्रवीषितस्माद्वक्रो भवितास्यष्टधैव}


\twolineshloka
{स वै तथा वक्र एवाभ्यजाय-दष्टावक्रः प्रथितो मानवेषु}
{अस्यासीद्वै मातुलः श्वेतकेतुःस तेन तुल्यो वयसा बभूव}


\twolineshloka
{संपीड्यमाना तु तदा सुजाताविवर्धमानेन सुतेन कुक्षौ}
{उवाच भर्तारमिदं रहोगताप्रसाद्य हीना वसुना धनार्थिनी}


\twolineshloka
{कथं करिष्याम्यधुना महर्षेमासश्चायं दशमो वर्तते मे}
{नैवास्ति मे वसु किंचित्प्रदातायेंनाहमेतामापदं निस्तरेयम्}


\twolineshloka
{उक्तस्त्वेवं भार्यया वै कहोळोवित्तस्यार्थे जनकमथाभ्यगच्छत्}
{स वै तदा वादविदा निगृह्यनिमज्जितो वन्दिनेहाप्सु विप्रः}


\twolineshloka
{उद्दालकस्तं तु तदा निशम्यसूतेन वादेऽप्सु निमज्जितं तथा}
{उवाच तां तत्रततः सुजात-मष्टवक्रे गूहितव्योऽयमर्थः}


\twolineshloka
{ररक्ष सा चापि तमस् मन्त्रंजातोऽप्यसौ नैव शुश्राव विप्रः}
{उद्दालकं पितरं सोऽभिमेनेतथाऽष्टावक्रो भ्रातरं श्वेतकेतुम्}


\twolineshloka
{ततो वर्षे द्वादशे श्वेतकेतु-रष्टावक्रं पितुरङ्के निषण्णम्}
{अपाकर्षद्गृह्यपाणौ रुदन्तंनायं तवाङ्कः पितुरित्युक्तवांश्च}


\twolineshloka
{यत्तेनोकतं दुरुक्तं तत्तदानींहृदि स्थितं तस्य सुदुःखमासीत्}
{गृहं गत्वा मातरं सोऽथ विग्रःपप्रच्छेदं क्व नु तातो ममेति}


\twolineshloka
{ततः सुजाता परमार्तरूपाशापाद्भीता तत्त्वमस्याचचक्षे}
{तद्वै तत्त्वंसर्वमाज्ञाय रात्रा-वित्यब्रवीच्छ्वेतकेतुं स विप्रः}


\twolineshloka
{गच्छाव यज्ञं जनकस्य राज्ञोबह्वाश्चर्यः श्रूयते तस्य यज्ञः}
{श्रोष्यावोऽत्र ब्राह्मणानां विवाद-मन्नं चाग्र्यं तत्रभोक्ष्यावहे च}


% Check verse!
विचक्षणत्वं च भविष्यते नौशिवश्च सौम्यश्च हि ब्रह्मघोषः
\twolineshloka
{तौ जग्मतुर्मातुलभागिनेयौयज्ञं समृद्धं जमकस्य राज्ञः}
{अष्टावक्रः पथि राज्ञा समेत्यप्रोत्सार्यमाणो वाक्यमिदं जगाद}


\chapter{अध्यायः १३५}
\twolineshloka
{अष्टावक्र उवाच}
{}


\threelineshloka
{अन्धस्य पन्था बधिरस्य पन्थाःस्त्रियः पन्था भारवाहस्य पन्थाः}
{राज्ञः पन्था ब्राह्मणेनासमेत्यसमेत्य तु ब्राह्मणस्यैव पन्थाः ॥राजोवाच}
{}


\threelineshloka
{पन्था अयं तेऽद्य मया निसृष्टोयेनेच्छसे तेन कामं व्रजस्व}
{न पावको विद्यते वै लघीया-निन्द्रोपि नित्यं नमते ब्राह्मणानाम् ॥लोमश उवाच}
{}


\twolineshloka
{`स एवमुक्तो मातुलेनैव सार्धंयथेष्टमार्गो यज्ञनिवेशनं तत्}
{संप्राप्य धर्मेण निवारितः सन्द्वारि द्वाःस्थं वाक्यमिदं बभाषे ॥'}


\twolineshloka
{यज्ञं द्रष्टुं प्राप्तवन्तौ स्म तातकौतूहलं बलवद्वै विवृद्धम्}
{आवां प्राप्तावतिथी संप्रवेशंकाङ्क्षावहे द्वारपते तवाज्ञाम्}


\threelineshloka
{ऐन्द्रद्युम्नेर्यज्ञदृशाविहावांविवक्षू वै जनकेन्द्रं दिदृक्षू}
{न वै क्रुध्यो वन्दिना चोत्तमेनसंयोजय द्वारपाल क्षणेन ॥द्वारपाल उवाच}
{}


\threelineshloka
{वन्देः समादेशकरा वयं स्मनिबोध वाक्यं च मयेर्यमाणम्}
{न वै बालाः प्रविशन्त्यत्र विप्रावृद्धा विदग्धाः प्रविशन्ति द्विजाग्र्याः ॥अष्टावक्र उवाच}
{}


\twolineshloka
{यद्यत्रवृद्धेषु कृतः प्रवेशोयुक्तं मया द्वारपाल प्रवेष्टुम्}
{वयं हि वृद्धाश्चरितव्रताश्चवेदप्रभावेन समन्विताश्च}


\threelineshloka
{शुश्रूषवश्चापि जितेन्द्रियाश्चज्ञानागमे चापि गताः स्म निष्ठाम्}
{न बाल इत्येव मन्तव्यमाहु-र्बालोऽप्यग्निर्दहति स्पृश्यमानः ॥द्वारपाल उवाच}
{}


\threelineshloka
{सरस्वतीमीरय वेद जुष्टा-मेकाक्षरां बहुरूपां विराजम्}
{अङ्गात्मानं समवेक्षस्व बालंकिं श्लाघसे दुर्लभा वादसिद्धिः ॥अष्टावक्र उवाच}
{}


\threelineshloka
{न ज्ञायते कायवृद्ध्या विवृद्धि-र्यथाऽष्ठीला शाल्मलेः संप्रवृद्धाः}
{ह्रस्वोऽल्पकायः फलितो विवृद्धोयश्चाफलस्तस्य न वृद्धभावः ॥द्वारपाल उवाच}
{}


\threelineshloka
{वृद्धेभ्य एवेह मतिं स्म बालागृह्णन्ति कालेन भवन्ति वृद्धाः}
{न हि ज्ञातुमल्पकालेन शक्यंकस्माद्बालः स्थविर इव प्रभाषसे ॥अष्टावक्र उवाच}
{}


\twolineshloka
{न तेन स्थविरो भवति येनास्य पलितं शिरः}
{बालोपि यः प्रजानाति तं देवाः स्थविरं विदुः}


\twolineshloka
{न हायनैर्न पलितैर्न वित्तैर्न च बन्धुभिः}
{ऋषयश्चक्रिरे धर्मं योऽनूचानः स नो महान्}


\twolineshloka
{दिदृक्षुरस्मि संप्राप्तो बन्दिनं राजसंसदि}
{निवेदयस्व मां द्वाःस्थ राज्ञे पुष्करमालिने}


\twolineshloka
{द्रष्टास्यद्य वदतोऽस्मान्द्वारपाल मनीषिभिः}
{सह वादे विवृद्धे तु वन्दिनं चापि निर्जितम्}


\threelineshloka
{पश्यन्तु विप्राः परिपूर्णविद्याःसहैव राज्ञा सपुरोधमुख्याः}
{उताहो वाऽप्युच्चतां नीचतां वातूष्णींभूतेष्वेव सर्वेष्वथाद्य ॥द्वारपाल उवाच}
{}


\twolineshloka
{कथं यज्ञं दशवर्षो विशेस्त्वंविनीतानां विदुषां संप्रवेशम्}
{उपायतः प्रयतिष्ये तवाहंप्रवेशने कुरु यत्नं यथावत्}


\threelineshloka
{`एष राजा संश्रवणे स्थितस्तेस्तुह्येनं त्वं वचसा संस्कृतेन}
{स चानुज्ञां दास्यति प्रीतियुक्तःप्रवेशने यच्च किंचित्तवेष्टम्' ॥अष्टावक्र उवाच}
{}


\twolineshloka
{भोभो राजञ्जनकानां वरिष्ठत्वं वै सम्राट् त्वयि सर्वं समृद्धम्}
{त्वं वा कर्ता कर्मणां यज्ञियानांययातिरेको नृपतिर्वा पुरस्तात्}


\twolineshloka
{विद्वान्वन्दी वादविदो निगृह्यवादे भग्नानप्रतिशङ्कमानः}
{त्वयाऽभिसृष्टैः पुरुषैराप्तकृद्भि-र्जले सर्वान्मज्जयतीति नः श्रुतम्}


\threelineshloka
{सोऽहं श्रुत्वा ब्राह्मणानां सकाशेब्रह्माद्य वै कथयितुमागतोस्मि}
{क्वासौ वनदी यावदेनं समेत्यनक्षत्राणीव सविता नाशयामि ॥राजोवाच}
{}


\twolineshloka
{आशंससे वन्दिनं वै विजेतु-मविज्ञाय त्वं वाक्यबलं परस्य}
{विज्ञातवीर्यैः शक्यमेवं प्रवक्तुंदृष्टश्चासौ ब्राह्मणैर्वादशीलैः}


\twolineshloka
{आशंसमाना वन्दिनं वै बिजेतु-मविज्ञात्वा तु बलं वन्दिनोऽस्य}
{समागता ब्राह्मणास्तेन पूर्वंन शोभन्ते भास्करेणेव ताराः}


\threelineshloka
{आशानुबन्धो हि तवात्र यत्नःस वन्दिमासाद्यतथा विनश्यति}
{विज्ञानवन्तो निकृतास्तु तातकथं सदस्तर्तुमिदं समर्थः ॥अष्टावक्र उवाच}
{}


\threelineshloka
{विवादितोऽसौ न हि मादृशैर्हिसिंहीकृतस्तेन वदन्यभीतः}
{समेत्य मां निहतः शेष्यतेऽद्यमार्गे भग्नं शकटमिवाबलाक्षम् ॥राजोवाच}
{}


\threelineshloka
{षण्नाभेर्द्वादशाक्षस्य चतुर्विंशतिपर्वणः}
{यस्त्रिषष्टिशतारस्य वेदार्थं स परः कविः ॥अष्टावक्र उवाच}
{}


\threelineshloka
{चतुर्विंशतिपर्व त्वां षणअनाभि द्वादशप्रधि}
{तत्रिषष्टिशतारं वै चक्रं पातु सदागति ॥राजोवाच}
{}


\threelineshloka
{बडबे इव संयुक्ते श्येनपाते दिवौकसाम्}
{कस्तयोर्गर्भमाधत्ते गर्भं सुषुवतुश्च कम् ॥अष्टावक्र उवाच}
{}


\threelineshloka
{मा स्म भूः स्वगृहे राजञ्शात्रवाणामपि ध्रुवम्}
{वातसारथिराधत्ते गर्भं सुषुवतुश्च तम् ॥राजोवाच}
{}


\threelineshloka
{किंस्वित्स्तुप्तं न निमिषति किंस्विज्जातं न चोपति}
{कस्यखिद्धृदयं नास्ति किंस्विद्वेगेन वर्धते ॥अष्टावक्र उवाच}
{}


\threelineshloka
{मत्स्यः सुप्तो न निमिषत्यण्डं जातं न चोपति}
{अश्मनो हृदयं नास्ति नदी वेगेन वर्धते ॥राजोवाच}
{}


\twolineshloka
{न त्वां मन्ये मानुषं देवसत्वन त्वं बालः स्थविरः संमतो मे}
{न ते तुल्यो विद्यते वाक्प्रलापेतस्माद्द्वारं वितराम्येष विद्वन्}


\chapter{अध्यायः १३६}
\twolineshloka
{अष्टावक्र उवाच}
{}


\twolineshloka
{अत्रोग्रसेनसमितेषु राजन्समागतेष्वप्रतिमेषु राजसु}
{न मे विवित्सान्तरमस्ति वादिनांमहाजने हंसविवादिनामिव}


\threelineshloka
{न मोक्ष्यसे वै वदमानो निमज्जन्जलं प्रपन्नः सरितामिवाधनः}
{हुताशनस्येव समिद्धतेजसःस्थिरो भवस्वेह ममाद्य वन्दिन् ॥वन्द्युवाच}
{}


\twolineshloka
{व्याघ्रं शयानं प्रति मा प्रबोधआशीविषं सृक्विणी संलिहानम्}
{पदा हतस्येह शिरोभिहत्यनादष्टो वै मोक्ष्यसे तन्निबोध}


\threelineshloka
{यो वै दर्पात्संहननोपपन्नःसुदुर्बलः पर्वतमाविहन्ति}
{तस्यैव पाणिः सनखो विदीर्यतेन चैव शैलस्य हि दृश्यते व्रणः ॥अष्टावक्र उवाच}
{}


\twolineshloka
{सर्वे राज्ञो मैथिलस्य मैनाकस्येव पर्वताः}
{निकृष्टभूता राजानो वत्सा ह्यनडुहो यथा}


\threelineshloka
{यथा महेन्द्रः प्रवरः सुराणांनदीषु गङ्गा प्रवरा यथैव}
{तथा नृपाणां प्रवरस्त्वमेकोवन्दिं समभ्यानय मत्सकाशम् ॥लोमश उवाच}
{}


\threelineshloka
{एवमष्टावक्रः समितौ हि गर्ज-ञ्जातक्रोधो वन्दिनमाह राजन्}
{उक्ते वाक्ये चोत्तरं मे ब्रवीहिवाक्यस्य चाप्युत्तरं ते ब्रवीमि ॥वन्द्युवाच}
{}


\threelineshloka
{एक एवाग्निर्बहुधा समिध्यतेएकः सूर्यः सर्वमिदं विभाति}
{एकोऽवीरो देवराजोऽरिहन्तायमः पितृणामीश्वरश्चैक एव ॥अष्टावक्र उवाच}
{}


\threelineshloka
{द्वाविन्द्राग्नी चरतो वै सखायौद्वौ देवर्षी नारदपर्वतौ च}
{द्वावश्विनौ द्वे रथस्यापि चक्रेभार्यापती द्वौ विहितौ विधात्रा ॥वन्द्युवाच}
{}


\threelineshloka
{त्रिः सूयते कर्मणा वै प्रजेयंत्रयो युक्ता वाजपेयं वहन्ति}
{अध्वर्यवस्त्रिसवनानि तन्वतेत्रयो लोकास्त्रीणि ज्योतींषि चाहुः ॥अष्टावक्र उवाच}
{}


\threelineshloka
{चतुष्टयं ब्राह्मणानां निकेतंचत्वारो वर्णा यज्ञमिमं वहन्ति}
{दिशश्चतस्रो वर्णचतुष्टयं चचतुष्पदा गौरपि शश्वदुक्ता ॥वन्द्युवाच}
{}


\threelineshloka
{पञ्चाग्नयः पञ्चपदा च पङ्क्ति-र्यज्ञाः पञ्चैवाप्यथ पञ्चेनद्रियाणि}
{दृष्ट्वा वेदे पञ्चचूडाप्सराश्चलोके ख्यातं पञ्चनदं च पुण्यम् ॥अष्टावक्र उवाच}
{}


\threelineshloka
{षडाधाने दक्षिणामाहुरेकेषट् चैवेमे ऋतवः कालचक्रम्}
{षडिन्द्रियाण्युत षट् कृत्तिकाश्चषट् साद्यस्काः सर्ववेदेषु दृष्टाः ॥वन्द्युवाच}
{}


\threelineshloka
{सप्त ग्राम्याः पशवः सप्त वन्याःसप्त च्छन्दांसि क्रतुमेकं वहन्ति}
{सप्तर्षयः सप्त चाप्यर्हणानिसप्तन्त्री प्रथिता चैव वीणा ॥अष्टावक्र उवाच}
{}


\threelineshloka
{अष्टौ शाणाः शतमानं वहन्तितथाष्टपादः शरभः सिंहघाती}
{अष्टौ वसूञ्शुश्रुम देवतासुयूपश्चाष्टास्रिर्विहितः सर्वयज्ञे ॥वन्द्युवाच}
{}


\threelineshloka
{नवैवोक्ताः सामिधेन्यः पितॄणांतथा प्राहुर्नवयोगं विसर्गम्}
{नवाक्षरा बृहती संप्रदिष्टानवैव योगो गणनामेति शश्वत् ॥अष्टावक्र उवाच}
{}


\threelineshloka
{दिशो दशोक्ताः पुरुषस्य लोकेसहस्रमाहुर्दशपूर्णं शतानि}
{दशैव मासान्बिभ्रति भर्गवत्योदशैरका दशदाशा दशार्हाः ॥वन्द्युवाच}
{}


\threelineshloka
{एकादशैकादशिनः पशूना-मेकादशैवात्र भवन्ति यूपाः}
{एकादश प्राणभृतां विकाराएकादशोक्ता दिवि देवेषु रुद्राः ॥अष्टावक्र उवाच}
{}


\twolineshloka
{संवत्सरं द्वादशमासमाहु-र्जगत्याः पादो द्वादशैवाक्षराणि}
{द्वादशाहः प्राकृतो यज्ञ उक्तोद्वादशादित्यान्कथयन्तीह घीराः}


% Check verse!
वन्द्युवाच

त्रयोदशी तिथिरुक्ता महोग्रात्रयोदशद्वीपवती मही च

लोमश उवाच

एतावदुक्त्वा विरराम वन्दीश्लोकस्यार्धं व्याजहाराष्टवक्रः

अष्टावक्र उवाच

त्रयोदशाहानि ससार केशीत्रयोदशादीन्यतिच्छन्दांसि चाहुः ॥लोमश उवाच


\twolineshloka
{ततो महानुदतिष्ठन्निनाद-स्तूष्णींभूतं सूतपुत्रं निशम्य}
{अधोमुखं ध्यानपरं तदानी-मष्टावक्रं चाप्युदीर्यन्तमेव}


\threelineshloka
{तस्मिंस्तथा संकुले वर्तमानेस्फीते यज्ञे जनकस्योत राज्ञः}
{अष्टावक्रं पूजयन्तोऽभ्युपेयु-र्विप्राः सर्वे प्राञ्जलयः प्रतीताः ॥अष्टावक्र उवाच}
{}


\threelineshloka
{अनेनैव ब्राह्मणाः शुश्रुवांसोवादे जित्वा सलिले मज्जिताः प्राक्}
{तानेव धर्मानयमद्य वन्दीप्राप्नोतु गृह्याशु निमज्जयैनम् ॥वन्द्युवाच}
{}


\twolineshloka
{अहं पुत्रो वरुणस्योत राज्ञ-स्तत्रास सत्रं द्वादशवार्षिकं वै}
{सत्रेण ते जनक तुल्यकालंतदर्थं ते प्रहिता मे द्विजाग्र्याः}


\threelineshloka
{ते तु सर्वे वरुणस्योत यज्ञंद्रष्टुं गता इम आयान्ति भूयः}
{अष्टावक्रं पूजये पूजनीयंयस्य हेतोर्जनितारं समेष्ये ॥अष्टावक्र उवाच}
{}


\twolineshloka
{विप्राः समुद्राम्भसि मज्जिता येवाचा जिता मेधया वा विदानाः}
{तां मेधया वाचमथोज्जहारयथा वाचमवचिन्वन्ति सन्तः}


\twolineshloka
{अग्निर्दहञ्जातवेदाः सतां, गृहान्विसर्जयंस्तेजसा न स्म धाक्षीत्}
{बालेषु पुत्रेषु कृपणं वदत्सुतथा वाचमवचिन्वन्ति सन्तः}


\fourlineindentedshloka
{श्लेष्मातकी क्षीणवर्चाः शृणोषिउताहो त्वां स्तुतयो मादयन्ति}
{हस्तीव त्वं स्तुतयो मादयन्ति}
{न मामिकां वाचमिमां शृणोषि ॥जनक उवाच}
{}


\threelineshloka
{शृणोमि वाचं तव दिव्यरूपा-ममानुषीं दिव्यरूपोऽसि साक्षात्}
{अजैषीर्यद्वन्द्विनं त्वं विवादेनिसृष्ट एष तव कामोऽद्य वन्दी ॥अष्टावक्र उवाच}
{}


\threelineshloka
{नानन जीवता कश्चिदर्थो मे वन्दिना नृप}
{पिता यद्यस्य वरुणो मज्जयैनं जलाशये ॥वन्द्युवाच}
{}


\threelineshloka
{अहं पुत्रो वरुणस्योत राज्ञोन मे भयं विद्यते मज्जितस्य}
{इमं मुहूर्तं पितरं द्रक्ष्यतेऽय-मष्टावक्रश्चिरनष्टं कहोळम् ॥लोमश उवाच}
{}


\threelineshloka
{ततस्ते पूजिता विप्रा वरुणेन महात्मना}
{उदतिष्ठंस्ततः सर्वे जनकस्य समीपतः ॥कहोळ उवाच}
{}


\twolineshloka
{इत्यर्थमिच्छन्ति सुताञ्जना जननकर्मणा}
{यदहं नाशकं कर्तुं तत्पुत्रः कृतवान्मम}


\threelineshloka
{उताबलस्य क्लवानुत बालस्य पण्डितः}
{उत वाऽविदुषो विद्वान्पुत्रो जनक जायते ॥वन्द्युवाच}
{}


\twolineshloka
{शितेन ते परशुना स्वयमेवान्तको नृप}
{शिरांस्यपाहरन्नाजौ रिपूणां भद्रमस्तु ते}


\threelineshloka
{महदैक्थ्यं गीयते साम चाग्र्यंसम्यक्सोमः पीयते चात्र सत्रे}
{शुचीन्भागान्प्रतिजगृहुश्च हृष्टाःसाक्षाद्देवा जनकस्येह राज्ञः ॥लोमश उवाच}
{}


\twolineshloka
{समुत्थितेष्वथ सर्वेषु राजन्विप्रेषु तेष्वधिकं सुप्रभेषु}
{अनुज्ञातो जनकेनाथ राज्ञाविवेश तोयं सागरस्योत वन्दी}


\twolineshloka
{अष्टावक्रः पितरं पूजयित्वासंपूजितो ब्राह्मणैस्तैर्यथावत्}
{प्रत्याजगामाश्रममेव चाग्र्यंजित्वा वन्दिं सहितो मातुलेन}


\twolineshloka
{ततोऽष्टावक्रं मातुरथान्तिके पितानदीं समङ्गां शीघ्रमिमां विशस्व}
{प्रोवाच चैनं स तथा विवेशसमैरङ्गैश्चापि बभूव पुण्या}


\twolineshloka
{नदी समङ्गा च बभूव पुण्यायस्यां स्नातो मुच्यते किल्बिषाद्धि}
{त्वमप्येनां स्नानपानावगाहैःसभ्रातृकः सहभार्यो विशश्व}


\twolineshloka
{अत्र कौन्तेय सहितो भ्रातृभिस्त्वंसुखोषितः सह विप्रैः प्रतीतः}
{पुण्यान्यन्यानि शुचिकर्मैकभक्ति-र्मया सार्धं चरितास्याजमीढ}


\chapter{अध्यायः १३७}
\twolineshloka
{लोमश उवाच}
{}


\twolineshloka
{एषा मधुविला नाम समङ्गा संप्रकाशते}
{एतत्कर्दमिलं नाम भरतस्याभिषेचनम्}


\twolineshloka
{अलक्ष्म्या किल संयुक्तो वृत्रं हत्वा शचीपतिः}
{आप्लुतः सर्वपापेभ्यः समङ्गायां व्यमुच्यत}


\twolineshloka
{एतद्विनशनं कुक्षौ मैनाकस्य नरर्षभ}
{अदितिर्यत्रपुत्रार्थं तदन्नमपचत्पुरा}


\twolineshloka
{एनं पर्वतराजानमारुह्य भरतर्षभाः}
{अयशस्यामसंशब्द्यामलक्ष्मीं व्यपनोत्स्यथ}


\twolineshloka
{एते कनखला राजन्नृषीणां दयिता नगाः}
{एषा प्रकाशते गङ्गा युधिष्ठिर यशस्विनी}


\twolineshloka
{सनत्कुमारो भगवानत्र सिद्धिमगात्पुरा}
{आजमीढावगाह्यैनां सर्वपापैः प्रमोक्ष्यसे}


\twolineshloka
{अपां ह्रदं च पुण्याख्यं भृगुतुन्दं च पर्वतम्}
{तूष्णीं गङ्गां च कौन्तेय सानुजः समुपस्पृश}


\twolineshloka
{आश्रमः स्थूलशिरसो रमणीयः प्रकाशते}
{अत्र मानं च कौन्तेय क्रोधं चैव विवर्जय}


\threelineshloka
{एष रैभ्याश्रमः श्रीमान्पाण्डवेय प्रकाशते}
{भारद्वाजो यत्रकविर्यवक्रीतो व्यनश्यत ॥युधिष्ठिर उवाच}
{}


\twolineshloka
{कथं युक्तोऽभवदृषिर्भरद्वाजः प्रतापवान्}
{किमर्थं च यवक्रीतः पुत्रोऽनश्यत वै मुनेः}


\threelineshloka
{एतत्सर्वं यथावृत्तं श्रोतुमिच्छामि तत्त्वतः}
{कर्मभिर्देवकल्पानां कीर्त्यमानैर्भृशं रमे ॥लोमश उवाच}
{}


\twolineshloka
{भरद्वाजश्च रैभ्यश्च सखायौ संबभूवतुः}
{तावूषतुरिहात्यन्तं प्रीयमाणौ वनान्तरे}


\twolineshloka
{रैभ्यस्य तु सुतावास्तामर्वावसुपरावसू}
{आसीद्यवक्रीः पुत्रस्तु भरद्वाजस्य भारत}


\twolineshloka
{रैभ्यो विद्वान्सहापत्यस्तपस्वी चेतरोऽभवत्}
{तयोश्चाप्यतुला प्रीतिरभवद्भरतर्षभ}


\twolineshloka
{यवक्रीः पितरं दृष्ट्वा तपस्विनमसत्कृतम्}
{दृष्ट्वा च सत्कृतं विप्रै रैभ्यं पुत्रैः सहानघः}


\twolineshloka
{पर्यतप्यत तेजस्वी मन्युनाऽभिपरिप्लुतः}
{तपस्तेपे ततो घोरं वेदज्ञानाय पाण्डव}


\twolineshloka
{सुसमिद्धे महत्यग्नौ शरीरमुपतापयन्}
{जनयामास संतापमिन्द्रस्य सुमहातपाः}


\threelineshloka
{तत इन्द्रो यवक्रीतमुपगम्य युधिष्ठिर}
{अब्रवीत्कस्य हेतोस्त्वमास्थितस्तप उत्तमम् ॥यवक्रीत उवाच}
{}


\twolineshloka
{द्विजानामनधीता वै वेदाः सुरगणार्चित}
{प्रतिभान्त्विति तप्येहमिदं परमकं तपः}


\twolineshloka
{स्वाध्यायार्थं समारम्भो ममायं पाकशासन}
{तपसा ज्ञातुमिच्छामि सर्वज्ञानानि कौशिक}


\threelineshloka
{कालेन महता वेदाः शक्या गुरुमुखाद्विभो}
{प्राप्तुं तस्मादयं यत्नः परमो मे समास्थितः ॥इन्द्र उवाच}
{}


\threelineshloka
{अमार्ग एष विप्रर्षे येन त्वं यातुमिच्छसि}
{किं विघातेन ते विप्र गच्छाधीहि गुरोर्मुखात् ॥लोमश उवाच}
{}


\twolineshloka
{एवमुक्त्वा गतः शक्रो यवक्रीरपि भारत}
{भूय एवाकरोद्यत्नं तपस्यमितविक्रमः}


\twolineshloka
{घोरेण तपसा राजंस्तप्यमानो महत्तपः}
{संतापयामास भृशं देवेन्द्रमिति नः श्रुतम्}


\twolineshloka
{तं तथा तप्यमानं तु तपस्तीव्रं महामुनिम्}
{उपेत्य बलभिद्देवो वारयामास वै पुनः}


\threelineshloka
{अशक्योऽर्थः समारब्धो नैतद्बुद्धिकृतं तव}
{प्रतिभास्यन्ति वै वेदास्तव चैव पितुश्च ते ॥यवक्रीत उवाच}
{}


\twolineshloka
{न चैतदेवं क्रियते देवराजसमीप्सितम्}
{महता नियमेनाहं तप्स्ये घोरतरं तपः}


\threelineshloka
{समिद्धेऽग्नावुपकृत्याङ्गमङ्गंहोष्यामि वा मघवंस्तन्निबोध}
{यद्येतदेवं न करोषि कामंममेप्सितं देवराजेह सर्वम् ॥लोमश उवाच}
{}


\twolineshloka
{निश्चयं तमभिज्ञाय मुनेस्तस्य महात्मनः}
{प्रतिवारणहेत्वर्थं बुद्ध्या संचिन्त्य बुद्धिमान्}


\twolineshloka
{तत इन्द्रोऽकरोद्रूपं ब्राह्मणस्य तपस्विनः}
{अनेकशतवर्षस्य दुर्बलस्य सयक्ष्मणः}


\twolineshloka
{यवक्रीतस्य यत्तीर्थमुचितं शैचकर्मणि}
{भागीरथ्यां तत्र सेतुं वालुकाभिश्चकार सः}


\twolineshloka
{यदाऽस्य वदतो वाक्यं न स चक्रे द्विजोत्तमः}
{वालुकाभिस्ततः शक्रो गङ्गां समभिपूरयन्}


\twolineshloka
{वालुकामुष्टिमनिशं भागीरथ्यां व्यसर्जयत्}
{स्नातुमभ्यागतं शक्रो यवक्रीतमदर्शयत्}


\twolineshloka
{तं ददर्श यवक्रीतो यत्नवन्तं निबन्धने}
{प्रहसंश्चाब्रवीद्वाक्यमिदं स मुनिपुङ्गवः}


\threelineshloka
{किमिदं वर्तते ब्रह्मन्किंच ते ह चिकीर्षितम्}
{अतीव हि महान्यत्नः क्रियतेऽयं निरर्थकः ॥इन्द्र उवाच}
{}


\threelineshloka
{बन्धिष्ये सेतुना गङ्गां सुखः पन्था भविष्यति}
{क्लिश्यते हि जनस्तात तरमाणः पुनःपुनः ॥यवक्रीत उवाच}
{}


\threelineshloka
{नायं शक्यस्त्वया बद्धुं महानोघस्तपोधन}
{अशक्याद्विनिवर्तस्व शक्यमर्थं समारभ ॥इन्द्र उवाच}
{}


\threelineshloka
{यथैव भवता चेद तपो वेदार्थमुद्यतम्}
{अशक्यं तद्वदस्माभिरयं भारः समाहितः ॥यवक्रीत उवाच}
{}


\twolineshloka
{यथा तव निरर्थोऽयमारम्भस्त्रिदशेश्वर}
{तथा यदि ममापीदं मन्यसे पाकशासन}


\threelineshloka
{क्रियतां यद्भवेच्छक्यं त्वया सुरगणेश्वर}
{वरांश्च मे प्रयच्छान्यासन्यैर्विद्वान्भवितास्म्यहम् ॥लोमश उवाच}
{}


\twolineshloka
{तस्मै प्रादाद्वरानिन्द्र उक्तवान्यान्महातपाः}
{प्रतिभास्यन्ति ते वेदाः पित्रा सह यथेप्सिताः}


\threelineshloka
{यच्चान्यत्काङ्क्षसे कामं यवक्रीर्गम्यतामिति}
{स लब्धकाम पितरं समेत्याथेदमब्रवीत् ॥यवक्रीत उवाच}
{}


\threelineshloka
{प्रतिभास्यन्ति वै वेदा मम तातस्य चोभयोः}
{अपि चान्यान्भविष्यावो वरा लब्धास्तथा मया ॥भरद्वाज उवाच}
{}


\twolineshloka
{दर्पस्ते भविता तात वराँल्लब्ध्वा यथेप्सितान्}
{स दर्पपर्णः कृपणः क्षिप्रमेव विनङ्क्ष्यसि}


\twolineshloka
{अत्राप्युदाहरन्तीमा गाथा देवैरुदाहृताः}
{मुनिरासीत्पुरा पुत्र बालधिर्नाम विश्रुतः}


\twolineshloka
{स पुत्रशोकादुद्विग्रस्तपस्तेपे सुदुष्करम्}
{भवेन्मम सुतोऽमर्त्य इतितं लब्धवांश्च सः}


\threelineshloka
{तस्य प्रसादो वै देवैः कृतो न त्वमरैः समः}
{नामर्त्यो विद्यते मर्त्यो निमित्तायुर्भविष्यति ॥बालधिरुवाच}
{}


\threelineshloka
{यथेमे पर्वताः शश्वत्तिष्ठन्ति सुरसत्तमाः}
{तावज्जीवेन्मम् सुतो निर्वाणमुत मे मतः ॥भरद्वाज उवाच}
{}


\twolineshloka
{तस्य पुत्रस्तदा जत्रे मेधावी क्रोधनस्तदा}
{स तु लब्धवरो दर्पादृषींश्चैवावमन्यत}


\twolineshloka
{विकुर्वाणो मुनीनां च व्यचरत्स महीमिमाम्}
{आससाद महावीर्यं धनुषाक्षं मनीषिणम्}


\twolineshloka
{तस्यापचक्रे मेधावी तं शशाप स वीर्यवान्}
{भव भस्मेति चोक्तः स न भस्म समपद्यत}


\threelineshloka
{धनुषाक्षस्तु तं दृष्ट्वा मेधाविनमनामयम्}
{`मुनिस्तत्कारणं ज्ञात्वा स्वयं महिषरूपधृत्}
{शृङ्गेणाद्रीनचलयत्ततोऽयंभस्मसादभूत्'}


\twolineshloka
{निमित्तमस्य महर्षिर्भेदयामास पर्वतान्}
{स निमित्ते विनष्टे तु ममार सहसा शिशुः}


% Check verse!
तं मृतं पुत्रमादाय विललाप ततः पिता
\twolineshloka
{लालप्यमानं तं दृष्ट्वा मुनयः परमार्तवत्}
{ऊचुर्वेदविदः सर्वै गाथां यां तां निबोध मे}


\twolineshloka
{न दिष्टमर्थमत्येतुमीशोऽमर्त्यः कथंचन}
{महर्षिर्भेदयामास धनुषाक्षो महीधरान्}


\twolineshloka
{एवं लब्ध्वा वरान्बाला दर्पपूर्णास्तपस्विनः}
{क्षिप्रमेव विनश्यन्ति यथा न स्यात्तथा भवान्}


\twolineshloka
{एष रैभ्यो महावीर्यः पुत्रौ चास्य तथाविधौ}
{तं यथा पुत्र नाभ्येषि तथा कुर्यास्त्वतन्द्रितः}


\threelineshloka
{स हि क्रुद्धः समर्थस्त्वां पुत्र पीडयितुं रुषा}
{रैभ्यश्चापि तपस्वी च कोपनश्च महानृषिः ॥यवक्रीत उवाच}
{}


\threelineshloka
{एव करिष्ये मा तापं तात कार्षीः कथंचन}
{यथा हि मे भवान्मान्यस्तथा रैभ्यः पिता मम ॥लोमश उवाच}
{}


\twolineshloka
{उक्त्वा स पितरं श्लक्ष्ण यवक्रीरकुतोभयः}
{विप्रकुर्वन्नृषीनन्यानतुष्यत्परया मुदा}


\chapter{अध्यायः १३८}
\twolineshloka
{लोमश उवाच}
{}


\twolineshloka
{चङ्क्रम्यमाणः स तदा यवक्रीरकुतोभयः}
{जगाम माधवे मासि रैभ्याश्रमपदं प्रति}


\twolineshloka
{स ददर्शाश्रमे रम्ये पुष्पतद्रुमभूषिते}
{विचरन्तीं स्नुषां तस्य किन्नरीमिव भारत}


\twolineshloka
{यवक्रीस्तामुवाचेदमुपतिष्ठस्व मामिनि}
{निर्लज्जो लज्जया युक्तां कामेन हृतचेतनः}


\twolineshloka
{सा तस्य शीलमाज्ञाय तस्माच्छापाच्च बिभ्यती}
{तेजस्वितां च रैभ्यस् तथेत्युक्त्वा जगाम ह}


\twolineshloka
{तत एकान्तमानीय लज्जयामास भारत}
{आजगाम तदा रैभ्यः स्वमाश्रममरिंदम}


\twolineshloka
{रुदतीं च स्नुषां दृष्ट्वा भार्यामार्ता परावसोः}
{सान्त्वयञ्श्लक्ष्णया वाचा पर्यपृच्छद्युधिष्ठिर}


\twolineshloka
{सा तस्मै सर्वमाचष्ट यवक्रीभाषितं शुभा}
{प्रत्युक्तं च यवक्रीतं प्रेक्षापूर्वं तथाऽऽत्मना}


\twolineshloka
{शृण्वानस्यैव रैभ्यस्य यवक्रेस्तद्विचेष्टनम्}
{दहन्निव तदा चेतः क्रोधः समभवन्महान्}


\twolineshloka
{स तदा मन्युनाऽऽविष्टस्तपस्वी कोपनो भृशम्}
{अवलुप्य जटामेकां जुहावाग्नौ सुसंस्कृते}


\twolineshloka
{ततः समभवन्नारी तस्या रूपेण संमिता}
{अवलुप्यापरां चापि जुहावाग्नौ जटां पुनः}


\twolineshloka
{ततः समभवद्रक्षो दीप्तास्यं घोरदर्शनम्}
{अब्रूतां तौ तदा रैभ्यं किं कार्यं करवामहे}


\twolineshloka
{तावब्रवीदृषिः क्रुद्धो यवक्रीर्वध्यतामिति}
{जग्मतृस्तौ तथेत्युक्त्वा यवक्रीतजिघांसया}


\twolineshloka
{ततस्तं समुपास्थाय कृत्या सृष्टा महात्मना}
{कमण्डलुं जहारास्य मोहयित्वा तु भारत}


\twolineshloka
{उच्छिष्टं तु यवक्रीतमपकृष्टकमण्डलुम्}
{तत उद्यतशूलः स राक्षसः समुपाद्रवत्}


\twolineshloka
{तमाद्रवन्तं संप्रेक्ष्य शूलहस्तं जिघांसया}
{यवक्रीः सहसोत्थाय प्राद्रवद्यत्रवै सरः}


\twolineshloka
{जलहीनं सरो दृष्ट्वा यवक्रीस्त्वरितः पुनः}
{जगाम सरितः सर्वास्ताश्चाप्यासन्विशोषिताः}


\twolineshloka
{स काल्यमानो घोरेण शूलहस्तेन रक्षसा}
{अग्निहोत्रे पितुर्भीतः सहसा प्रविवेश ह}


\twolineshloka
{स वै प्रविशमानस्तु शुद्रेणान्धेन रक्षिणा}
{निगृहीतो बलाद्द्वारि सोऽवातिष्ठत पार्थिव}


\twolineshloka
{निगृहीतं तु शूद्रेण यवक्रीतं स राक्षसः}
{ताडयामास शूलेन स भिन्नहृदयोऽपतत्}


\twolineshloka
{यवक्रीतं स हत्वा तु राक्षसो रैभ्यमागमत्}
{अनुज्ञातस्तु रैभ्येण तया नार्या सहावसत्}


\chapter{अध्यायः १३९}
\twolineshloka
{लोमश उवाच}
{}


\twolineshloka
{भरद्वाजस्तु कौन्तेय कृत्वा स्वाध्यायमाह्निकम्}
{समित्कलापमादाय प्रविवेश स्वमाश्रमम्}


\twolineshloka
{तं स्म दृष्ट्वा पुरा सर्वे प्रत्युत्तिष्ठिन्ति पावकाः}
{न त्वेनमुपतिष्ठन्ति हतपुत्रं तदाऽग्नयः}


\twolineshloka
{वैकृतंत्वग्निहोत्रे स लक्षयित्वा महातपाः}
{तमन्धं शूद्रमासीनं गृहपालमथाब्रवीत्}


\twolineshloka
{किंनु मे नाग्नयः शूद्र प्रतिनन्दन्ति दर्शनम्}
{त्वं चापि न यथापूर्वं कच्चित्क्षेममिहाश्रमे}


\threelineshloka
{कच्चिन्न रैभ्यं पुत्रो मे गतवानल्पचेतनः}
{एतदाचक्ष्व मे शीघ्रं न हि शुद्ध्यति मे मनः ॥शूद्र उवाच}
{}


\twolineshloka
{रैभ्यं यातो नूनमयं पुत्रस्ते मन्दचेतनः}
{तथाहि निहतः शेते राक्षसेन महात्मना}


\twolineshloka
{प्रकाल्यमानस्तेनायं शूलहस्तेन रक्षसा}
{अग्न्यगारं प्रतिद्वारि मया दोर्भ्यां निवारितः}


\threelineshloka
{ततः स विहताशोऽत्रजलकामोशुचिर्ध्रुवम्}
{निहतः सोऽतिवेगेन शूलहस्तेन रक्षसा ॥लोमश उवाच}
{}


\threelineshloka
{भरद्वाजस्तु तच्छ्रुत्वा शूद्रस्य विप्रियं महत्}
{गतासुं पुत्रमादाय विललाप सुदुःखितः ॥भरद्वाज उवाच}
{}


\twolineshloka
{रब्राह्मणानां किलार्थाय ननु त्वं तप्तवांस्तपः}
{द्विजानामनधीता वै वेदाः संप्रतिभान्त्विति}


\twolineshloka
{तथा कल्याणशीलस्त्वं ब्राह्मणेषुमहात्मसु}
{अनागाः सर्वभूतेषु कर्कशत्वमुपेयिवान्}


\twolineshloka
{प्रतिषिद्धो मया तात रैभ्यावसथदर्शनात्}
{गतवानेव तं क्षुद्रं कालान्तकयमोपमम्}


\twolineshloka
{यः स जानन्महातेजा वृद्धस्यैकं ममात्मजम्}
{गतवानेव कोपस्य वशं परमदुर्मतिः}


\twolineshloka
{पुत्रशोकमनुप्राप्त एष रैभ्यस्य कर्मणा}
{त्यक्ष्यामि त्वामृतेपुत्र प्राणानिष्टतमान्भुवि}


\twolineshloka
{यथाऽहं पुत्रशोकेन देहं त्यक्ष्यामि किल्विषी}
{तथा ज्येष्ठः सुतो रैभ्यं हिंस्याच्छीघ्रमनागसम्}


\twolineshloka
{सुखिनो वै नरा येषां जाया पुत्रो न विद्यते}
{ये पुत्रशोकमप्राप्य विचरन्ति यथासुखम्}


\twolineshloka
{ये तु पुत्रकृताच्छोकाद्भृशं व्याकुलचेतसः}
{शपन्तीष्टान्सखीनार्तास्तेभ्यः पापतरो नु कः}


\threelineshloka
{परासुश्च सुतो दृष्टः शप्तश्चैष्टः सखा मया}
{ईदृशीमापदं कोत्र द्वितीयोऽनुविष्यति ॥लोमश उवाच}
{}


\twolineshloka
{विलप्यैवं बहुविधं भरद्वाजोऽदहत्सुतम्}
{सुसमिद्धं ततः पश्चात्प्रविवेश हुताशनम्}


\chapter{अध्यायः १४०}
\twolineshloka
{लोमश उवाच}
{}


\twolineshloka
{एतस्मिन्नेव काले तु बृहद्द्युम्नो महीपतिः}
{सत्रं तेने महाभागो रैभ्ययाज्यः प्रतापवान्}


\twolineshloka
{तेन रैभ्यस्य वै पुत्रावर्वावसुपरासू}
{वृतौ सहायौ सत्रार्थं बृहद्द्युम्नेन धीमता}


\twolineshloka
{तत्र तौ समनुज्ञातौ पित्रा कौन्तेय जग्मतुः}
{आश्रमे त्वभवद्रैभ्यो भार्या चैव परावसोः}


\twolineshloka
{अथावलोककोऽगच्छद्गृहानकः परावसुः}
{कृष्णाजिनेन संवीतं ददर्श पितरं वने}


\twolineshloka
{जघन्यरात्रे निद्रान्धः सावशेषे तमस्यापि}
{चरन्तं गहनेऽरण्ये मेने स पितरं मृगम्}


\twolineshloka
{मृगं तु मन्यमानेन पिता वै तेन हिंसितः}
{अकामयानेन तदा शरीरत्राणमिच्छता}


\twolineshloka
{तस्य स प्रेतकार्याणि कृत्वा सर्वाणि भारत}
{पुनरागम्य तत्सत्रमब्रवीद्धातरं वचः}


\twolineshloka
{इदं कर्म न शक्तस्त्वं वोढुमेकः कथंचन}
{मया च हिंसितस्तातो मन्यमानेन वै मृगम्}


\threelineshloka
{सोऽस्मदर्थे व्रतं तात चर त्वं ब्रह्मघातिनाम्}
{समर्थो ह्यहमेकाकी कर्म कर्तुमिदं मुने ॥अर्वावसुरुवाच}
{}


\threelineshloka
{करोतु वै भवान्सत्रं बृहद्द्युम्नस्य धीमतः}
{ब्रह्महत्यां चरिष्येऽहं त्वदर्थं नियतेन्द्रियः ॥लोमश उवाच}
{}


\twolineshloka
{स तस्यब्रह्महत्यायाः पारं गत्वा युधिष्ठिर}
{अर्वावसुस्तदा सत्रमाजगाम पुनर्मुनिः}


\twolineshloka
{ततः परावसुर्दृष्ट्वा भ्रातरं समुपस्थितम्}
{बृहद्द्युम्नमुवाचेदं वचनं हर्षगद्गदम्}


\threelineshloka
{एष ते ब्रह्महा यज्ञं मा द्रष्टुं प्रविशेदिति}
{ब्रह्महा प्रेक्षितेनापि पीडयेत्त्वामसंशयम् ॥लोमश उवाच}
{}


\threelineshloka
{तच्छ्रुत्वैव तदा राजा प्रेष्यानाह स विट््पते}
{प्रेष्यैरुत्सार्यमाणस्तु राजन्नर्वावसुस्तदा}
{न मया ब्रह्महत्येयं कृतेत्याह पुनःपुनः}


\threelineshloka
{उच्यमानोऽसकृत्प्रेष्यैर्ब्रह्महा इति भारत}
{नैवस्म प्रतिजानामि ब्र्हमहत्यां स्वयंकृताम्}
{मम भ्रात्रा कृतमिदं मया स परिमोक्षितः}


\twolineshloka
{स तथा प्रवदन्क्रोधात्तैश्च प्रेष्यैः प्रभाषितः}
{तूष्णीं जगाम ब्रह्मर्षिर्वनमेव महातपाः}


\twolineshloka
{उग्रं तपः समास्थाय दिवाकरमथाश्रितः}
{रहस्यवेदं कृतवान्सूर्यस्य द्विजसत्तमः}


\twolineshloka
{मूर्तिमांस्तं ददर्शाथ स्वयमग्रभुगव्ययः ॥ 3-140-19aप्रीतास्तस्याभवन्देवाः कर्मणाऽर्वावसोर्नृप}
{तं ते प्रवरयामासुर्निरासुश्च परावसुम्}


% Check verse!
ततो देवा वरं तस्मै ददुरग्निपुरोगमाः
\twolineshloka
{स चापि वरयामास पितुरुत्थानमात्मनः}
{अनागस्त्वं ततो भ्रातुः पितुश्चास्मरणं वधे}


\twolineshloka
{भरद्वाजस्य चोत्थानं यवक्रीतस्य चोभयोः}
{प्रतिष्ठां चापि वेदस्य सौरस्य द्विजसत्तमः}


\twolineshloka
{एवमस्त्विति तं देवाः प्रोचुश्चापि वरान्ददुः}
{ततः प्रादुर्बभूवुस्ते सर्व एव युधिष्ठिर}


\twolineshloka
{अथाब्रवीद्यवक्रीतो देवानग्निपुरोगमान्}
{समधीतं मया ब्रह्म व्रतानि चरितानि च}


\threelineshloka
{कथं च रैभ्यः शक्तो मामदीयानं तपस्विनम्}
{तथायुक्तेन विधिना निहन्तुममरोत्तमाः ॥देवा ऊचुः}
{}


\twolineshloka
{मैवं कृथा यवक्रीत यथा वदसि वै मुने}
{ऋते गुरुमधीता हि स्वयं वेदास्त्वया पुरा}


\threelineshloka
{अनन तु गुरूनदुःखात्तोषयित्वाऽऽत्मक्रमणा}
{कालेन महता क्लेशाद्ब्रह्माधिगतमुत्तमम् ॥लोमश उवाच}
{}


\twolineshloka
{यवक्रीतमथोक्त्वैवं देवाः साग्निपुरोगमाः}
{संजीवयित्वा तान्सर्वान्पुनर्जग्मुस्त्रिविष्टपम्}


\twolineshloka
{`ततो वै स यवक्रीतो ब्रह्मचर्यं चचार ह}
{अष्टादश च वर्षाणि त्रिंशतं च युधिष्ठिर'}


\twolineshloka
{आस्रमस्तस्य पुण्योऽयं सदापुष्पफलद्रुमः}
{अत्रोष्य राजशार्दूल सर्वपापैः प्रभोक्ष्यसे}


\chapter{अध्यायः १४१}
\twolineshloka
{लोमश उवाच}
{}


\twolineshloka
{उशीरबीजं मैनाकं गिरिं श्वेतं च भारत}
{समतीतोऽसि कौन्तेय कालशैलं च पार्थिव}


\twolineshloka
{एषा गङ्गा सप्तविधा राजते भरतर्षभ}
{स्थानं विरजसं पुण्यं यत्राग्निर्नित्यमिध्यते}


\twolineshloka
{एतद्वै मानुषेणाद्य न शक्यं द्रष्टुमद्भुतम्}
{समाधिं कुरुताव्यग्रास्तीर्थान्येतानि द्रक्ष्यथ}


\twolineshloka
{एतद्द्रक्ष्यसि देवानामाक्रीडं च रणाङ्कितम्}
{अतिक्रान्तोसि कौन्तेय कालशैलं च पर्वतम्}


\twolineshloka
{श्वेतं गिरिं प्रवेक्ष्यामो मन्दरं चैव पर्वतम्}
{यत्र माणिवरो यक्षः कुबेरश्चैव यक्षराट्}


\twolineshloka
{अष्टाशीतिसहस्राणि गन्धर्वाः शीघ्रगामिनः}
{तथा किंपुरुषा राजन्यक्षाश्चैव चतुर्गुणाः}


\twolineshloka
{अनेकरूपसंस्थाना नानाप्रहरणाश्च ते}
{यक्षेन्द्रं मनुजश्रेष्ठ माणिभद्रमुपासते}


\twolineshloka
{तेषामृद्धिरतीवात्र गतौ वायुसमाश्च ते}
{यक्षेन्द्रं मनुजश्रेष्ठ माणिभद्रमुपासते}


\twolineshloka
{तैस्तात बलिभिर्गुप्ता यातुधानैश्च रक्षिताः}
{दुर्गमाः पर्वताः पार्थ समाधिं परमं कुरु}


\twolineshloka
{कुबेरसचिवाश्चान्ये रौद्रा मैत्राश्च राक्षसाः}
{तैः समेष्याम कौन्तेय यत्तो विक्रमणे भव}


\twolineshloka
{कैलासः पर्वतो राजन्पड्योजनशतोच्छ्रितः}
{यत्रदेवाः समायान्ति विशाला यत्र भारत}


\twolineshloka
{असङ्ख्येयास्तु कौन्तेय यक्षराक्षसकिन्नराः}
{नागाः सुपर्णा गन्धर्वाः कुबेरसदनं प्रति}


\twolineshloka
{तान्विगाहस्व पार्थाद्य तपसा च दमेन च}
{रक्ष्यमाणो मया राजन्भीमसेनबलेन च}


\twolineshloka
{स्वस्ति ते वरुणो राजा यमश्च समितिंजयः}
{गङ्गा च यमुना चैव पर्वताश्च दिशन्तु ते}


\twolineshloka
{मरुतश्च सहाश्विभ्यां सरितश्च सरांसि च}
{स्वस्ति देवासुरेभ्यश्च वसुभ्यश् महाद्युते}


\twolineshloka
{इन्द्रस्य जाम्बूनदपर्वताद्वैशृणोमि घोषं तव देवि गङ्गे}
{गोपाययेमं सुभगे गिरिभ्यःसर्वाजमीढापचितं नरेन्द्रम्}


\fourlineindentedshloka
{ददस्व शर्म प्रविविक्षतोऽस्यशैलानिमाञ्छैलसुते नृपस्य}
{`शिवप्रदा सर्वसरित्प्रधानेस भ्रातृकस्येह युधिष्ठिरस्य'}
{युधिष्ठिर उवाच}
{}


\threelineshloka
{अपूर्वोऽयं संभ्रमो लोमशस्यकृष्णां च सर्वे रक्षत मा प्रमादः}
{देशो ह्ययं दुर्गतमो मतोऽस्यतस्मात्परं शौचमिहाचरध्वम् ॥वैशंपायन उवाच}
{}


\twolineshloka
{ततोऽब्रवीद्भीममुदारवीर्यंकृष्णां यत्तः पालय भीमसेन}
{शून्येऽर्जुनेऽसन्निहिते च तातत्वामेव कृष्णा भजते भयेषु}


\twolineshloka
{ततो महात्मां स यमौ समेत्यमूर्धन्युपाघ्राय विमृज्यगात्रे}
{उवाच तौ बाष्पकलं स राजामा बैष्टमागच्छतमप्रमत्तौ}


\chapter{अध्यायः १४२}
\twolineshloka
{युधिष्ठिर उवाच}
{}


\twolineshloka
{अन्तर्हितानि भूतानि रक्षांसि बलवन्ति च}
{अग्निना तपसा चैव शक्यं गन्तुं वृकोदर}


\twolineshloka
{सन्निवर्तय कौन्तेय क्षुत्पिपासे बलाश्रयात्}
{ततो बलं च दाक्ष्यं च संश्रयस्व वृकोदर}


\twolineshloka
{ऋषेस्त्वया श्रुतं वाक्यं कैलासं पर्वतं प्रति}
{बुद्ध्या प्रपश्य कौन्तेय कथं कृष्णा गमिष्यति}


\twolineshloka
{अथवा सहदेवेन धौम्येन च समं विभो}
{सूतैः पौरोगवैश्चैव सर्वैश्च परिचारकैः}


\twolineshloka
{रथैरश्वैश्च ये चान्ये विप्राः क्लेशासहाः पथि}
{सर्वैस्त्वं सहितो भीम निवर्तस्वायतेक्षण}


\twolineshloka
{त्रयो वयं गमिष्यामो लध्वाहारा यतव्रताः}
{अहं च नकुलश्चैव लोमशश्च महातपाः}


\threelineshloka
{ममागमनमाकाङ्क्षन्ग्गाद्वारे समाहितः}
{वसेह द्रौपदीं रक्षन्यावदागमनं मम ॥भीम उवाच}
{}


\twolineshloka
{राजपुत्री श्रमेणार्ता दुःखार्ता चैव भारत}
{व्रजत्येव हि कल्याणी श्वेतवाहदिदृक्षया}


\threelineshloka
{तवचाप्यरतिस्तीव्रा वर्तते तमपश्यतः}
{गुडाकेशं महात्मानं संग्रामेष्वपलायिनम्}
{किं पुनः सहदेवं च मां च कृष्णां च भारत}


\twolineshloka
{द्विजाः कामं निवर्तन्तां सर्वे च परिचारकाः}
{सूताः पौरोगवाश्चैव यं च मन्येत नो भवान्}


\twolineshloka
{न ह्यहं हातुमिच्छामि भवन्तमिह कर्हिचित्}
{शैलेऽस्मिन्राक्षसाकीर्णे दुर्गेषु विपमेषु च}


\twolineshloka
{इयं चापि महाभागा राजपुत्री पतिव्रता}
{त्वामृते पुरुषव्याघ्र नोत्सहेद्विनिवर्तितुम्}


\twolineshloka
{तथैवसहदेवोऽयं सततं त्वामनुव्रतः}
{न जातु विनिवर्तेत मतज्ञो ह्यहमस्य वै}


\twolineshloka
{अपिचात्र महाराज सव्यसाचिदिदृक्षया}
{सर्वे लालसभूताः स्म तस्माद्यास्यामहे सह}


\twolineshloka
{यद्यशक्यो रथैर्गन्तुं शैलोऽयं बहुकन्दरः}
{पद्भिरेव गमिष्यामो मा राजन्विमना भव}


\twolineshloka
{अहं वहिष्ये पाञ्चालीं यत्रयत्र न शक्ष्यति}
{इति मे वर्तते बुद्धिर्मा राजन्विमना भव}


\threelineshloka
{सुकुमारौ तथा वीरौ माद्रीनन्दिकरावुभौ}
{दुर्गे संतारयिष्यामि यत्राशक्तौ भविष्यतः ॥युधिष्ठिर उवाच}
{}


\twolineshloka
{एवं ते भाषमाणस्य बलं भीमाभिवर्धताम्}
{यत्त्वमुत्सहसे वोढुं पाञ्चालीं विपुलेऽध्वनि}


\twolineshloka
{यमजौ चापि भद्रं ते नैतदन्यत्र विद्यते}
{बलं तव यशश्चैवधर्मः कीर्तिश्च वर्धताम्}


\threelineshloka
{यत्त्वमुत्सहसे नेतुं भ्रातरौ सह कृष्णया}
{मा ते ग्लानिर्महाबाहो मा च तेऽस्तु पराभवः ॥वैशंपायन उवाच}
{}


\threelineshloka
{ततः कृष्णाऽब्रवीद्वाक्यं प्रहसन्ती मनोरमा}
{गमिष्यामि न संताप कार्यो मां प्रति भारत ॥लोमश उवाच}
{}


\twolineshloka
{तपसा शक्यते गन्तुं पर्वतो गन्धमादनः}
{तपसा चैव कौन्तेय सर्वे योक्ष्यामहे वयम्}


\threelineshloka
{नकुलः सहदेवश्च भीमसेनश् पार्थिव}
{अहं च त्वं च कौन्तेय द्रक्ष्यामः श्वेतवाहनम् ॥वैशंपायन उवाच}
{}


\twolineshloka
{एवं संभाषमाणास्ते सुबाहुविषयं महत्}
{ददृशुर्मुदिता राजन्प्रभूतगजवाजिमत्}


\threelineshloka
{किराततङ्गणाकीर्णं पुलिन्दशतसंकुलम्}
{हिमवत्यमरैर्जुष्टं बह्वाश्चर्यसमाकुलम्}
{सुबाहुश्चापि तान्दृष्ट्वा पूजयाप्रत्यगृह्णत}


\twolineshloka
{विषयान्ते कुलिन्दानामीश्वरः प्रीतिपूर्वकम्}
{तत्र ते पूजितास्तेन सर्व एव सुखोपिताः}


\twolineshloka
{प्रतस्थुर्विमले सूर्ये हिमवन्तं गिरिं प्रति}
{इन्द्रसेनमुखांश्चापि भृत्यान्पौरोगवांस्तथा}


\twolineshloka
{सूदांश्च पारिबर्हांश्च द्रौपद्याः सर्वशो नृप}
{राज्ञः कुलिन्दाधिपतेः परिदाय महारथाः}


\threelineshloka
{पद्भिरेव महावीर्या ययुः कौरवनन्दनाः}
{ते शनैः प्राद्रवन्सर्वे कृष्णया सह पाण्डवाः}
{तस्माद्देशात्सुसंहृष्टा द्रष्टुकामा धनंजयम्}


\chapter{अध्यायः १४३}
\twolineshloka
{युधिष्ठिर उवाच}
{}


\twolineshloka
{भीमसेनयमौ चोभौ पाञ्चालि च निबोधत}
{नास्ति भूतस्य नाशो वै पश्यतास्मान्वनेचरान्}


\twolineshloka
{दुर्बलाः क्लेशिताः स्मेति यद्ब्रुवामेतरेतरम्}
{अशक्येऽपि व्रजामो यद्नंजयदिदृक्षया}


\twolineshloka
{तन्मे दहति गात्राणि तूलराशिमिवानलः}
{यच्च वीरं न पश्यामि धनंजयमुपान्तिके}


\twolineshloka
{तस्यादर्शनतप्तं मां सानुजं वनमास्थितम्}
{याज्ञसेन्याः परामर्शः स च वीर दहत्युत}


\twolineshloka
{नकुलात्पूर्वजं पार्थं न पश्याम्यमितौजसम्}
{अजेयमुग्रधन्वानं तेन तप्ये वृकोदर}


\twolineshloka
{तीर्थानि चैव रम्याणि वनानि च सरांसि च}
{चरामि सह युष्माभिस्तस्य दर्शनकाङ्क्षया}


\twolineshloka
{पञ्चवर्षाण्यहं वीरं सत्यसन्धं धनंजयम्}
{यन्न पश्यामि बीभत्सुं तेन तप्ये वृकोदर}


\twolineshloka
{तं वै श्यामं गुडाकेशं सिंहविक्रान्तगामिनम्}
{न पश्यामि महाबाहुं तेन तप्ये वृकोदर}


\twolineshloka
{कृतास्त्रं निपुणं युद्देऽव्रतिमानं धनुष्मताम्}
{न पश्यामि कुरुश्रेष्ठ तेन तप्ये वृकोदर}


\twolineshloka
{चरन्तमरिसङ्घेषु क्रुद्धं कालमिवानघ}
{प्रभिन्नमिव मातङ्गं सिंहस्कन्धं धनंजयम्}


\twolineshloka
{यः स शक्रादनवरो वीर्येण च बलेन च}
{यमयोः पूर्वजः पार्थः श्वेताश्वोऽमितविक्रमः}


\twolineshloka
{`नारायणसमो युद्धे सत्यसन्धो दृढव्रतः}
{तं ममापश्यतो भीम न शान्तिर्हृदयस्य वै'}


\twolineshloka
{दुःखेन महताऽऽविष्टश्चिन्तयामि दिवानिशम्}
{अजेयमुग्रधन्वानं तेन तप्ये वृकोदर}


\twolineshloka
{सततं यः क्षमाशीलः क्षिप्यमाणोऽप्यणीयसा}
{ऋजुमार्गप्रपन्नस्य शर्मदाता भयस्य च}


\twolineshloka
{स तु जिह्मप्रवृत्तस्य माययाऽभिजिघांसतः}
{अपि वज्रधरस्यापि भवेत्कालविषोपमः}


\twolineshloka
{शत्रोरपि प्रपन्नस्य सोऽनृशंसः प्रतापवान्}
{दाताऽभयस् बीभत्सुरमितात्मा महाहलः}


\twolineshloka
{सर्वेषामाश्रयोऽस्माकं रणेऽरीणां प्रमर्दिता}
{आहर्ता सर्वरत्नानां सर्वेषां नः सुखावहः}


\twolineshloka
{रत्नानि यस्य वीर्येण दिव्यान्यासत्पुरा मम}
{बहूनि बहुजातीनि यानि प्राप्तः सुयोधनः}


\twolineshloka
{यस्य बाहुबलाद्वीर सभा चासीत्पुरा मम}
{सर्वरत्नमयी ख्याता त्रिषु लोकेषु पाण्डव}


\twolineshloka
{वासुदेवसमं वीर्ये कार्तवीर्यसमं युधि}
{अजेयममितं युद्धे तं न पश्यामि फल्गुनम्}


\twolineshloka
{संकर्षणं महावीर्यं त्वां च भीमापराजितम्}
{अनुयातः स्ववीर्येण वासुदेवं च शत्रुहा}


\twolineshloka
{यस् बाहुबले तुल्यः प्रभावे च पुरंदरः}
{जवे वायुर्मुखे सोमः क्रोधे मृत्युः सनातनः}


\twolineshloka
{ते वयं तं नरव्याघ्रं सर्वेवीर दिदृक्षवः}
{प्रवेक्ष्यामो महाबाहो पर्वतं गन्धमादनम्}


\twolineshloka
{विशाला बदरी यत्रनरनारायणाश्रमः}
{तं सदाऽध्युषितं यक्षैर्द्रक्ष्यामो गिरिसुत्तमम्}


\twolineshloka
{कुबेरनलिनीं रम्यां राक्षसैरभिसेविताम्}
{पद्भिरेव गमिष्यामस्तप्यभाना महत्तपः}


\twolineshloka
{नातप्ततपसा शक्यो गन्तुं देशो वृकोदर}
{न नृशंसेन लुब्धेन नाप्रशान्तेन भारत}


\twolineshloka
{तत्र सर्वेगमिष्यामो भीमार्जुपदैषिणः}
{सायुधा बद्धनिस्त्रिंशाः सार्धं विप्रैर्महाव्रतैः}


\twolineshloka
{मक्षिकादंशमशकान्सिंहान्व्याघ्रान्सरीसृपान्}
{प्राप्नोत्यनियतः पार्थ नियतस्तान्न पश्यति}


\twolineshloka
{ते वयं नियतात्मानः पर्वतं गन्धमादनम्}
{प्रवेक्ष्यामो मिताहारा धनंजयदिदृक्षवः}


\chapter{अध्यायः १४४}
\twolineshloka
{[लोमश उवाच}
{}


\twolineshloka
{द्रष्टारः पर्वताः सर्वे नद्यः सपुरकाननाः}
{तीर्थानि चैव श्रीमन्ति स्पृष्टं च सलिलं करैः}


\twolineshloka
{पर्वतं मन्दरं दिव्यमेष पन्थाः प्रयास्यति}
{समाहीता निरुद्विग्नाः सर्वे भवत पाण्डवाः}


\twolineshloka
{अयं देवनिवासो वै गन्तव्यो वो भविष्यति}
{ऋषीणां चैव दिव्यानां निवासः पुण्यकर्मणां}


\twolineshloka
{एषा शिवजला पुण्या याति सौम्य महानदी}
{बदरीप्रभवा राजन्देवर्षिगणसेविता}


\twolineshloka
{एषा वैहायसैर्नित्यं वालखिल्यैर्महात्मभिः}
{अर्चिता चोपयाता च गन्धर्वैश्च महात्मभिः}


\twolineshloka
{अत्रसाम स्म गायन्ति सामगाः पुण्यनिःस्वनाः}
{मरीचिः पुलहश्चैव भृगुश्चैवाङ्गिरास्तथा}


\twolineshloka
{अत्राह्निकं सुरश्रेष्ठो जपते समरुद्गणः}
{साध्याश्चैवाश्विनौ चैव परिधावन्ति तं तदा}


\twolineshloka
{चन्द्रमाः सह सूर्येण ज्योतीषि च ग्रहैः सह}
{अहोरात्रविभागेन नदीमेनामनुव्रजन्}


\twolineshloka
{एतस्याः सलिलं मूर्ध्नि वृषाङ्कः पर्यधारयत्}
{गङ्गाद्वारे महाभाग येन लोकस्थितिर्भवेत्}


\twolineshloka
{एतां भगवतीं देवीं भवन्तः सर्व एव हि}
{प्रयतेनात्मना तात प्रतिगम्याभिवादत}


\twolineshloka
{तस्य तद्वचनं श्रुत्वा लोमशस्य महात्मनः}
{आकाशगङ्गां प्रयताः पाण्डवास्तेऽभ्यवादयन्}


\twolineshloka
{अभिवाद्य च ते सर्वे पाण्डवा धर्मचारिणः}
{पुनः प्रयाताः संहृष्टाः सर्वैर्ऋषिगणैः सह}


\twolineshloka
{ततो दूरात्प्रकाशन्तं पाण्डुरं मेरुसंनिभम्}
{ददृशुस्ते नरश्रेष्ठा विकीर्णं सर्वतोदिशम्}


\twolineshloka
{तान्प्रष्टुकामान्विज्ञाय पाण्डवान्स तु लोमशः}
{उवाच वाक्यं वाक्यज्ञः शृणुध्वं पाण्डुनन्दनाः}


\twolineshloka
{एतद्विकीर्णं सुश्रीमत्कैलासशिखरोपमम्}
{यत्पश्यसि नरश्रेष्ठ पर्वतप्रतिमं स्थितम्}


\twolineshloka
{एतान्यस्थीनि दैत्यस्य नरकस्य महात्मनः}
{पर्वतप्रतिमं भाति पर्वतप्रस्तराश्रितम्}


\twolineshloka
{पुरातनेन देवेन विष्णुना परमात्मना}
{रदैत्यो विनिहतस्तेन सुरराजहितैषिणा}


\twolineshloka
{दशवर्षसहस्राणि तपस्तप्यन्महामनाः}
{ऐनद्रं प्रार्थयते स्थानं तपःस्वाध्यायविक्रमात्}


\twolineshloka
{तपोबलेन महता बाहुवेगबलेन च}
{नित्यमेव दुराधर्षो धर्षयन्स दितेः सुतः}


\twolineshloka
{स तु तस्य बलं ज्ञात्वा धर्मे च चरितव्रतम्}
{भयाभिभूतः संविग्नः शक्र आसीत्तदाऽनघ}


\twolineshloka
{तेन संचिन्तितो देवो मनसा विष्णुरव्ययः}
{सर्वत्रगः प्रभुः श्रीमानागतश्च स्थितो बभौ}


% Check verse!
ऋषयश्चापि तं सर्वे तुष्टुवुश्च दिवौकसः
\twolineshloka
{तं दृष्ट्वा ज्वलमानश्रीर्भगवान्हव्यवाहनः}
{नष्टतेजाः समभवत्तस्य तेजोभिभर्त्सितः}


\fourlineindentedshloka
{तं दृष्ट्वा वरदं देवं विष्णुं देवगणेश्वरम्}
{प्राञ्जलिः प्रणतो भूत्वा नमस्कृत्य च वज्रभृत्}
{प्राह वाक्यं ततस्तत्त्वं यतस्तस्य भयं भवेत् ॥विष्णुरुवाच}
{}


\twolineshloka
{जानामि ते भयं शक्र दैत्येन्द्रान्नरकात्ततः}
{ऐन्द्रं रप्रार्थयते स्थानं तपःसिद्धेन कर्मणा}


\threelineshloka
{सोहमेनं तव प्रीत्या तपःसिद्धमपि ध्रुवम्}
{वियुनज्मि देहाद्देवेनद्र मुहूर्तं प्रतिपालय ॥लोमश उवाच}
{}


\threelineshloka
{तस्य विष्णुर्महातेजाः पाणिना चेतनां हरत्}
{स पपात ततो भूमौ गिरिराज इवाहतः}
{तस्यैतदस्थिसंघातं मायाविनिहतस्य वै}


% Check verse!
इदं द्वितीयमपरं विष्णो कर्म प्रकाशते
\threelineshloka
{नष्टा वसुमती कृत्स्ना पाताले चैव मज्जिता}
{पुनरुद्धरिता तेन वाराहेणैकशृङ्गिणा ॥युधिष्ठिर उवाच}
{}


% Check verse!
मगवन्विस्तरेणेमां कथां कथय तत्त्वतः
\twolineshloka
{कथं तेन सुरेशेन नष्टा वसुमती तदा}
{योजनानां शतं ब्रह्मन्पुनरुद्धरिता तदा}


\threelineshloka
{केन चैव प्रकारेण जगतो धरणी ध्रुवा}
{शिवा देवी महाभागा सर्वसस्यप्ररोहिणी}
{कस्य चैव प्रभावाद्धि योजनानां शतं गता}


\twolineshloka
{श्रोतुं विस्तरशः सर्वं त्वं हि तस्य प्रतिश्रयः ॥लोमश उवाच}
{}


\twolineshloka
{यत्तेऽहं परिपृष्टोऽस्मि कथामेतां युधिष्ठिर}
{तत्सर्वमखिलेनेह श्रूयतां मम भाषतः}


\twolineshloka
{पुरा कृतयुगे तात वर्तमाने भयंकरे}
{यमत्वं कारयामास आदिदेवः पुरातनः}


\twolineshloka
{यमत्वं कुर्वतस्तस्य देवदेवस्य धीमतः}
{न तत्र म्रियते कश्चिज्जायते वा तथाऽच्युत}


\twolineshloka
{वर्धन्ते पक्षिसङ्घाश्च तथा पशुगवेडकम्}
{गवाश्वं च मृगाश्चैव सर्वे ते पिशिताशनाः}


\twolineshloka
{तथा पुरुषशार्दूल मानुपाश्च परंतप}
{सहस्रशो ह्ययुतशो वर्धन्ते सलिलं यथा}


\twolineshloka
{एतस्मिन्संकुले तात वर्तमाने भयंकरे}
{अतिभाराद्वसुमती योजनानां शतं गता}


\threelineshloka
{सा वै व्यथितसर्वाङ्गी भारेणाक्रान्तचेतना}
{नारायणं वरं देवं प्रपन्ना शरणं गता ॥पृथिव्युवाच}
{}


\twolineshloka
{भगवंस्त्वत्प्रसादाद्धि तिष्ठेयं सुचिरं त्विह}
{भारेणास्मि समाक्रान्ता न शक्नोस्मि स्म वर्तितुम्}


\twolineshloka
{ममेमं भगवन्भारं व्यपनेतुं त्वमर्हसि}
{शरणागताऽस्मि ते देव प्रसादं कुरु मे विभो}


\twolineshloka
{तस्यास्तद्वचनं श्रुत्वा भगवानक्षरः प्रभुः}
{प्रोवाच वचनं हृष्टः श्राव्यारसमीरितम्}


\threelineshloka
{न ते महि भयं कार्यं भारार्ते वसुधारिणि}
{अयमेवं तथा कुर्मि यथा लध्वी भविष्यसि ॥लोमश उवाच}
{}


\twolineshloka
{स तां विसर्जयित्वा तु वसुधां शैलकुण्डलाम्}
{ततो वराहः संवृत्त एकशृङ्गो महाद्युतिः}


\twolineshloka
{रक्ताभ्यां नयनाभ्यां तु भयमुत्पादयन्निव}
{धूमं च ज्वलयँल्लक्ष्म्या तत्र देशे व्यवर्धत}


\twolineshloka
{स गृहीत्वा वसुमतीं शृङ्गेणैकेन भास्वता}
{योजनानां शतं वीर समुद्धरति सोऽक्षरः}


\twolineshloka
{तस्यां चोद्धार्यमाणायां संक्षोभः समजायत}
{देवाः संक्षुभिताः सर्वे ऋषयश्च तपोधनाः}


\twolineshloka
{हाहाभूतमभूत्सर्वं त्रिदिवं व्योम भूस्तथा}
{न पर्यवस्थितः कश्चिद्देवो वा मानुषोपि वा}


\twolineshloka
{ततो ब्रह्माणमासीनं ज्वलमानमिव श्रिया}
{देवाः सर्षिगणाश्चैव उपतस्थुरनेकशः}


\twolineshloka
{उपसर्प्य च देवेशं ब्रह्माणं लोकसाक्षिकम्}
{भूत्वा प्राञ्जलयः सर्वे वाक्यमुच्चारयंस्तदा}


\twolineshloka
{लोकाः संक्षुभिताः सर्वे व्याकुलं च चराचरम्}
{समुद्राणां च संक्षोभस्त्रिदशेश प्रकाशते}


\fourlineindentedshloka
{सैषा वसुमती कृत्स्ना योजनानां शतं गता}
{किमेतत्किंप्रभावेण येनेदं रव्याकुलं जगत्}
{आख्यातु नो भवाञ्शीघ्रं विसंज्ञाः स्मेह सर्वशः ॥ब्रह्मोवाच}
{}


\twolineshloka
{असुरेभ्यो भयं नास्ति युष्माकं कुत्रचित्क्वचित्}
{श्रूयतां यत्कृते त्वेष संक्षोभो जायतेऽमराः}


\twolineshloka
{योसौ सर्वत्रगः श्रीमानक्षरात्मा व्यवस्थितः}
{तस्य प्रभावात्संक्षोभस्त्रिदिवस्य प्रकाशते}


\twolineshloka
{यैषा वसुमती कृत्स्ना योजनानां शतं गता}
{समुद्धृता पुनस्तेन विष्णुना परमात्मना}


\threelineshloka
{तस्यामुद्धार्यमाणायां संक्षोभः समजायत}
{एवं भवन्तो जानन्तु च्छिद्यतां संशयश्च वः ॥देवा ऊचुः}
{}


\threelineshloka
{क्व तद्भूतं वसुमतीं समुद्धरति हृष्टवत्}
{तं देशं भगवन्ब्रूहि तत्र यास्यामहे वयम् ॥ब्रह्मोवाच}
{}


\twolineshloka
{हन्त गच्छत भद्रं वो नन्दने पशय्त स्थितम्}
{एषोत्र भगवाञ्श्रीमान्मुपर्णः संप्रकाशते}


\twolineshloka
{वारादेणैव रूपेण भगवाँल्लोकभावनः}
{कालानल इवाभाति पृथिवीतलमुद्धरन्}


\threelineshloka
{एतस्योरसि सुव्यक्तं श्रीवत्समभिराजते}
{पश्यध्वं विबुधाः सर्वे भूतमेतदनामयम् ॥लोमश उवाच}
{}


\threelineshloka
{ततो दृष्ट्वा महात्मानं श्रुत्वा चामन्त्र्य चामराः}
{पितामहं पुरस्कृत्य जग्मुर्देवा यथागतम् ॥वैशंपायन उवाच}
{}


\twolineshloka
{श्रुत्वा तु तां कथां सर्वे पाण्डवा जनमेजय}
{लोमशादेशितेनाशु यथा जग्मुः प्रहृष्टवत्}


\chapter{अध्यायः १४५}
\twolineshloka
{वैशंपायन उवाच}
{}


\twolineshloka
{ते शूरा सज्जधन्वानस्तूणवन्तः समार्गणाः}
{बद्धगोधाङ्गुलित्राणाः खङ्गवन्तोऽमितौजसः}


\twolineshloka
{परिगृह्य द्विजश्रेष्ठाञ्ज्येष्ठाः सर्वधनुष्मताम्}
{पञ्चालीसहिता राजन्प्रययुर्गन्धमादनम्}


\twolineshloka
{सरांसि सरितश्चैव पर्वतांश्च वनानि च}
{वृक्षांश्च बहुलच्छायान्ददृशुर्गिरिमूर्धनि}


\twolineshloka
{नित्यपुष्पफलान्देशान्देवर्षिगणसेवितान्}
{आत्मन्यात्मानमाधाय वीरा मूलफलाशिनः}


\twolineshloka
{चेरुरुच्चावचाकारान्देशान्विषमसंकटान्}
{पश्यन्तो मृगजातानि बहूनि विविधानि च}


\twolineshloka
{ऋषिसिद्धामरयुतं गन्धर्वाप्सरसां प्रियम्}
{विविशुस्ते महात्मानः किन्नराचरितं गिरिम्}


\twolineshloka
{प्रविशत्स्वथ वीरेषु पर्वतं गन्धमादनम्}
{चणअडवातं महद्वर्षं प्रादुरासीद्विशांपते}


\twolineshloka
{ततो रेणुः समुद्भूतः सपत्रबहुलो महान्}
{पृथिवीं चान्तरिक्षं च द्यां चैव सहसाऽवृणोत्}


\twolineshloka
{न स्म प्रज्ञायते किंचिदावृते व्योम्नि रेणुना}
{न चापि शेकुस्तत्कर्तुमन्योन्यस्याभिभाषणम्}


\twolineshloka
{न चापश्यंस्ततोऽन्योन्यं तमसाऽऽवृतचक्षुषः}
{आकृष्यमाणा वातेन साश्मचूर्णेन भारत}


\twolineshloka
{द्रुमाणां वातरुग्णआनां पततां भूतलेऽनिशम्}
{अन्येषां च महीजानां शब्दः समभवन्महान्}


\twolineshloka
{द्यौः स्वित्पतति किं भूमिर्दीर्यते पर्वतोनु किम्}
{इति ते मेनिरे सर्वेपवनेन विमोहिताः}


\twolineshloka
{ते पथाऽनन्तरांन्वृक्षान्वल्मीकान्विषमाणि च}
{पाणिभिः परिमार्गन्तो भीता वायोर्निलिल्यिरे}


\twolineshloka
{ततः कार्मुकमादाय भीमसेनो महाबलः}
{कृष्णामादाय संगम्य तस्थावाश्रित्य पादपम्}


\twolineshloka
{धर्मराजश्च धौम्यश् निलिल्याते महावने}
{अग्निहोत्राण्युपादाय सहदेवस्तु पर्वते}


\twolineshloka
{नकुलो ब्राह्मणाश्चान्ये लोमशश्च महातपाः}
{वृक्षानासाद्य संत्रस्तास्तत्रतत्र निलिल्यिरे}


\twolineshloka
{मन्दीभूते तु पवने तस्मिन्रजसि शाम्यति}
{महद्भिर्जलधारौर्घर्वर्षमभ्याजगाम ह}


\twolineshloka
{भृशं चटचटाशब्दो वज्राणां क्षिप्यतामिव}
{ततस्ताश्चञ्चलाभासश्चेरुरभ्रेषु विद्युतः}


\twolineshloka
{ततोऽश्मसहिता धाराः संवृण्वन्त्यः समन्ततः}
{प्रपेतुरनिशं तत्र शीघ्रवातसमीरिताः}


\twolineshloka
{तत्र सागरगा ह्यापः कीर्यमाणाः समन्ततः}
{प्रादुरासन्सकलुषाः फेनवत्यो विशांपते}


\twolineshloka
{वहन्त्यो वारि बहुलं फेनोडुपपरिप्लुतम्}
{परिसमस्रुर्महाशब्दाः प्रकर्षन्त्यो महीरुहान्}


\twolineshloka
{तस्मिन्नुपरते शब्देबाते च समतां गते}
{गते ह्यम्भसि निम्नानि प्रादुर्भूते दिवाकरे}


\twolineshloka
{निर्जग्मुस्ते शनैः सर्वे समाजग्मुश्च भारत}
{प्रतस्थिरे पुनर्वीराः पर्वतं गन्धमादनम्}


\chapter{अध्यायः १४६}
\twolineshloka
{वैशंपायन उवाच}
{}


\twolineshloka
{ततः प्रयातमात्रेषु पाण्डवेषु महात्मसु}
{पद्भ्यामनुचिता गन्तुं द्रौपदी समुपाविशत्}


\twolineshloka
{श्रान्ता दुःखपरीता च वातवर्षेण तेन च}
{सौकुमार्याच्च पाञ्चाली संमुमोह तपस्विनी}


\twolineshloka
{सा कम्पमाना मोहेन बाहुभ्यामसितेक्षणा}
{रवृत्ताभ्यामनुरूपाभ्यामूरू समवलम्बत}


\twolineshloka
{आलम्बमाना सहितावूरू गजकरोपमौ}
{रपपात सहसा भूमौ वेपन्ती कदली यथा}


\threelineshloka
{तां पतन्तीं वरारोहां भज्यमानां लतामिव}
{नकुलः समभिद्रुत्य परिजग्राह वीर्यवान् ॥नकुल उवाच}
{}


\twolineshloka
{राजन्पाञ्चालराजस्य सुतेयमसितक्षणा}
{श्रान्ता निपतिता भूमौ तामवेक्षस्व भारत}


\threelineshloka
{अदुःखार्हा परं दुःखं प्राप्तेयं मृदुगामिनी}
{आश्वासय महाराज तामिमां श्रमकर्शिताम् ॥वैशंपायन उवाच}
{}


\twolineshloka
{राजा तु वचनात्तस्य भृशं दुःखसमन्वितः}
{भीमश्च सहदेवश्च सहसा समुपाद्रवन्}


\threelineshloka
{तामवेक्ष्यतु कौन्तेयो विवर्णवदनां कृशाम्}
{अङ्कमानीय धर्मात्मा पर्यदेवयदातुरः ॥युधिष्ठिर उवाच}
{}


\twolineshloka
{कथं वेश्मसु गुप्तेषु स्वास्तीर्णशयनोचिता}
{भूमौ निषतिता शेते सुखार्हा वरवर्णिनी}


\twolineshloka
{सुकुमारौ कथं पादौ मुखं च कमलप्रभम्}
{मत्कृतेऽद्य वरार्हायाः श्यामतां समुपागतम्}


\twolineshloka
{किमिदं द्यूतकामेन मया कृतमबुद्धिना}
{आदाय कृष्णां चरता वने मृगगणाकुले}


\twolineshloka
{सुखं प्राप्स्यसि कल्याणि पाण्डवान्प्राप्य वै पतीन्}
{इतिद्रुवदराजेन पित्रा दत्ताऽयतेक्षणा}


\threelineshloka
{तत्सर्वमानवाप्येयं श्रमशोकाद्विकर्शिता}
{शेते निपतिता भूमौ पापस्य मम कर्मभिः ॥वैशंपायन उवाच}
{}


\twolineshloka
{तथा लालप्यमाने तु धर्मराजे युधिष्ठिरे}
{धौम्यप्रभृतयः सर्वे तत्राजग्मुर्द्विजोत्तमाः}


\twolineshloka
{ते समाश्वासयामासुराशीर्भिश्चाप्यपूजयन्}
{राक्षोघ्नांश्च तथा मन्त्राञ्जेपुश्चक्रुश्च ते क्रियाः}


\twolineshloka
{पठ्यमानेषु मन्त्रेषु शान्त्यर्थं परमर्षिभिः}
{स्पृश्यमाना करैः शीतैः पाण्डवैश्च मुहुर्मुहुः}


\twolineshloka
{सेव्यमाना च शीतेन जलमिश्रेण वायुना}
{पाञ्चाली सुखमासाद्य लेभे चेतः शनैः शनैः}


\twolineshloka
{परिगृह्यच तां दीनां कृष्णामजिनसंस्तरे}
{पार्था विश्रामयामासुर्लब्धसंज्ञां तपस्विनम्}


\twolineshloka
{तस्या यमौ रक्ततलौ पादौ पूजितलक्षणौ}
{कराभ्यां किणजाताभ्यां शनकैः संववाहतुः}


\twolineshloka
{पर्याश्वासयदप्येनां धर्मराजो युधिष्ठिरः}
{उवाच च कुरुश्रेष्ठो भीमसेनमिदं वचः}


\threelineshloka
{बहवः पर्वता मीम विषमा हिमदुर्गमाः}
{तेषु कृष्णा महाबाहो कथं नु विचरिष्यति ॥भीमसेन उवाच}
{}


\twolineshloka
{त्वां राजन्राजपुत्रीं च यमौ च पुरुषर्षभ}
{स्वयं नेष्यामि राजेनद्र मा विषादे मनुः कृथाः}


\twolineshloka
{अथवा यो मया जातो विहगो मद्बलोपमः}
{वहेदनघ सर्वान्नो वचनात्ते घटोत्कचः}


\chapter{अध्यायः १४७}
\twolineshloka
{युधिष्ठिर उवाच}
{}


\twolineshloka
{धर्मज्ञो बलवाञ्शूरः सद्यो राक्षसपुङ्गवः}
{भक्तोऽस्मानौरसः पुत्रो नेतुमर्हति मातरम्}


\threelineshloka
{तव भीम सुतेनाहं नीतो भीमपराक्रम}
{अक्षतः सह पाञ्चाल्या गच्छेयं गन्धमादनम् ॥वैशंपायन उवाच}
{}


\twolineshloka
{भ्रातुर्वचनमाज्ञाय भीमसेनो घटोत्कचम्}
{`चिन्तयामास बलवान्महाबलपराक्रमम्}


\threelineshloka
{घटोत्कचश्च धर्मात्मा स्मृतमात्रः पितुस्तदा}
{कृताञ्जलिरुषातिष्ठदभिवाद्याथ पाण्डवान्}
{ब्राह्मणांश्च महाबाहुः स च तैरमिनन्दितः}


% Check verse!
उवाच भीमसेनं स पितरं सत्यविक्रमः
\twolineshloka
{स्मृतोऽस्मि भवता शीघ्रं शुश्रूषुरहमागतः}
{आज्ञापय महाबाहो सर्वं कर्ताऽस्म्यसंशयम्}


\twolineshloka
{तच्छ्रुत्वा भीमसेनस्तु राक्षसं परिषस्वजे}
{चिन्ताया समनप्राप्तमित्युवाच वृकोदरः'}


\twolineshloka
{हैडिम्बेय परिश्रान्ता तव माताऽपराजिता}
{त्वं कच कामगमस्तात बलवान्वह तां खग}


\threelineshloka
{स्कन्धमारोप्य भद्रं ते मध्येऽस्माकं विहायसा}
{गच्छ नीचिकया गत्या यथा चैनां न पीडयेः ॥घटोत्कच उवाच}
{}


\twolineshloka
{धर्मराजं च धौम्यं च कृष्णां च यमजौ तथा}
{एकोप्यहमलं वोढुं किमुताद्य सहायवान्}


\fourlineindentedshloka
{[अन्ये च शतशः शूरा विहङ्गाः कामरूपिणः}
{सर्वान्वो ब्राह्मणैः सार्धं वक्ष्यन्ति सहिताऽनघ}
{`मन्दंमन्दं गमिष्यामि वहन्द्रुपदनन्दिनीम्' ॥वैश्पायन उवाच}
{}


\twolineshloka
{एवमुक्त्वा ततः कृष्णामुवाह स घटोत्कचः}
{पाण्डूनां मध्यगो वीरः पाण्डवानपि चापरे}


\twolineshloka
{लोमशः सिद्धमार्गेण जगामानुपमद्युतिः}
{स्वेनैव स प्रभावेण द्वीतीय इव भास्करः}


\twolineshloka
{ब्राह्मणांश्चापि तान्सर्वान्समुपादाय राक्षसाः}
{नियोगाद्राक्षसेनद्रस्य जग्मुर्भीमपराक्रमाः}


\twolineshloka
{एवं सुरमणीयानि वनान्युपवनानि च}
{आलोकयन्तस्ते जग्मुर्विशालां बदरीमनु}


\twolineshloka
{ते त्वाशुगतिभिर्वीरा राक्षसैस्तैर्महाजवैः}
{उह्यमाना ययुः शीघ्रं महदध्वानमल्पवत्}


\twolineshloka
{देशान्म्लेच्छजनाकीर्णान्नानारत्नाकरायुतान्}
{ददृशुर्गिरिपादांश्च नानाधातुसमाचितान्}


\twolineshloka
{विद्याघरमाकीर्णआन्युतान्वानरकिन्नरैः}
{तथा किंपुरुषैश्चैव गन्धर्वैश्चसमन्ततः}


\twolineshloka
{मयूरैश्चभरैश्चैव हरिणै रुरुभिस्तथा}
{वराहैर्गवयैश्चैव महिषैश्च समावृतान्}


\threelineshloka
{नदीजालसमाकीर्णान्नानापक्षियुतान्बहून्}
{नानाविधमृगैर्जुष्टांश्चारणैश्चोपशोभितान्}
{समदैश्च्यशि विहगैः उपैरन्वितांस्तथा}


\twolineshloka
{तेऽवतीर्य बहून्देशानुत्तरांश्च कुरूनपि}
{ददृशुर्विविधाश्चर्यं कैलासं पर्वतोत्तमम्}


\twolineshloka
{तस्याभ्याशे तु ददृशुर्नरनारायणाश्रमम्}
{उपेतं पादपैर्दिव्यैः सदापुष्पफलोपगैः}


\twolineshloka
{ददृशुस्तां च बदरीं वृत्तस्कन्धां मनोरमाम्}
{स्निग्धामविरलच्छायां श्रिया परमया युताम्}


\twolineshloka
{पत्रैः स्निग्धैरविरलैरुपेतां मृदुभिः शुभाम्}
{विशालशाखां विस्तीर्णामतिद्युतिसमन्विताम्}


\threelineshloka
{फलैरुपचितैर्दिव्यैराचितां स्वादुभिर्भृशम्}
{मधुस्रवैः सदा दिव्यां महर्षिगणसेविताम्}
{मदप्रमुदितैर्नित्यं नानाद्विजगणैर्युताम्}


\twolineshloka
{अदंशमशके देशे बहुमूलफलोदके}
{नीलशाद्वलसंछन्ने देवगन्धर्वसेविते}


\twolineshloka
{सुभूमिभागविशदे स्वभावविदिते शुभे}
{जातां हिममृदुस्पर्शे देशेऽपहतकण्टके}


\twolineshloka
{तामुपेत्य महात्मानः सह तैर्ब्राह्मणर्षभैः}
{अवतेरुस्ततः सर्वे राक्षसस्कन्धतः शनैः}


\twolineshloka
{ततस्तमाश्रमं पुण्यं नरनारायणाश्रितम्}
{ददृशुः पाण्डवा राजन्सहिता द्विजपुङ्गवैः}


\twolineshloka
{तमसा रहितं पुण्यमनामृष्टं रवेः करैः}
{क्षुत्तृट्शीतोष्णदोषैश्च वर्जितं शोकनाशनम्}


\twolineshloka
{महर्षिगणसंबाधं ब्राह्म्या लक्ष्म्या समन्वितम्}
{दुष्प्रवेशं महाराज नरैर्धर्मबहिष्कृतैः}


\twolineshloka
{बलिहोमार्चितं दिव्यं सुसंमृष्टानुलेपनम्}
{दिव्यपुष्पोपहारैश्च सर्वतोऽभिविराजितम्}


\twolineshloka
{विशालैरग्निशरणैः स्रुग्भाण्डैराचितं शुभैः}
{महद्भिस्तोयकलशैः कठिनैश्चोपशोभितम्}


\threelineshloka
{शरण्यं सर्वभूतानां ब्रह्मघोपनिनादितम्}
{दिव्यमाश्रयणीयं तमाश्रमं श्रमनाशनम्}
{श्रिया युतमनिर्देश्यं देववर्योपशोभितम्}


\twolineshloka
{फलमूलाशनैर्दान्तैश्चारुकृष्णाजिनाम्बरैः}
{सूर्यवैश्वानरसमैस्तपसा भावितात्मभिः}


\twolineshloka
{महर्षिभिर्मोक्षपरैर्यतिभिर्नियतेन्द्रियैः}
{ब्रह्मभूतैर्महाभागैरुपेतं ब्रह्मवादिभिः}


\twolineshloka
{सोऽभ्यगच्छन्महातेजास्तानृषीन्नियतः शुचिः}
{भ्रातृभिः सहितो धीमान्धर्मपुत्रो युधिष्ठिरः}


\threelineshloka
{दिव्यज्ञानोपपन्नास्ते दृष्ट्वा प्राप्तं युधिष्ठिरम्}
{अभ्यगच्छन्त सुप्रीता दिव्या देवमहर्षयः}
{}


\twolineshloka
{प्रीतास्ते तस्य सत्कारं विधिना पावकापमाः}
{उपाजह्रुश्च सलिलं पुष्पमूलफलं शुचि}


\twolineshloka
{स तैः प्रीत्याऽथ सत्कारमुपनीतं महर्षिभिः}
{प्रयतः प्रतिगृह्याथ धर्मराजो युधिष्ठिरः}


\twolineshloka
{तं शक्रसदनप्रख्यं दिव्यगन्धं मनोरमम्}
{प्रीतः स्वर्गोपमं पुण्यं पाण्डवः सह कृष्णया}


\twolineshloka
{रविवेश शोभया युक्तं भ्रातृभिश्च सहानघ}
{ब्राह्मणैर्वेदवेदाङ्गपारगैश्च सहार्चितः}


\twolineshloka
{तत्रापश्यत्स धर्मात्मा देवदेवर्पिपूजितम्}
{नरनारायणस्थानं भागीरथ्योपशोभितम्}


\twolineshloka
{तस्मिन्मधृस्रवफलां ब्रह्मर्षिगणभाविनीम्}
{वदरीं तामुपाश्रित्य पाण्डवो भ्रातृभिः सह}


\threelineshloka
{मुदा युक्ता महात्मानो रेमिरे तत्र ते तदा}
{आलोकयन्तो मैनाकं नानाद्विजगणायुतम्}
{हिरण्यशिखरं चैव मध्ये विन्दुसरः शिवम्}


\twolineshloka
{भागीरथीं सुतीर्थां च शीतामलजलां शिवाम्}
{मणिप्रवालप्रस्तारां पादपैरुषशोभिताम्}


\twolineshloka
{दिव्यपुष्पसमाकीर्णां मनःप्रीतिविवर्धनीम्}
{वीक्षमाणा महात्मानो विजह्रुस्तत्र पाण्डवाः}


\threelineshloka
{तस्मिन्देवर्षिचरिते देशे परमदुर्गमे}
{भागीरथीपुण्यजलेतर्पयांचक्रिरे पितॄन्}
{देवानृषींश्च कौन्तेयाः परमं शौचमास्थिताः}


\twolineshloka
{तत्र ते तर्पयन्तश्च जपन्तश्च कुरूद्वहाः}
{ब्राह्मणैः सहिता वीरा ह्यवसन्पुरुपर्षभाः}


\twolineshloka
{कृष्णायास्तत्रपशय्न्तः क्रीडितान्यमरप्रभाः}
{विचित्राणि नरव्याघ्रा रेमिरे तत्र पाण्डवाः}


\chapter{अध्यायः १४८}
\twolineshloka
{वैशंपायन उवाच}
{}


\twolineshloka
{तत्र ते पुरुषव्याघ्राः परमं शौचमास्थिताः}
{षड्रात्रमवसन्वीरा धनंजयदिदृक्षया}


\twolineshloka
{तस्मिन्विहरमाणाश्च रममाणाश्च पाण्डवाः}
{मनोज्ञे काननवरे सर्व भूतमनोरमे}


\threelineshloka
{पादपैः पुष्पविकचैः फलभारावनामितैः}
{शोभितः पर्वतो रम्यः पुंस्कोकिलकुलाकुलैः}
{स्निग्धपत्रैरविरलैः शीतच्छायैर्मनोरमैः}


\threelineshloka
{सरांसि च विचित्राणि प्रसन्नसलिलानि च}
{कमलैः सोत्पलैस्तत्रभ्राजमानानि सर्वशः}
{पश्यन्तश्चारुरूपाणइ रेमिरे तत्र पाण्डवाः}


\twolineshloka
{पुण्यगन्धः सुखस्पर्शो ववौ तत्र समीरणः}
{ह्लादयन्पाण्डवान्सर्बान्सकृष्णान्सद्विजर्षभान्}


\twolineshloka
{ततः पूर्वोत्तरे वायुः प्लवमानो यदृच्छया}
{सहस्रपत्रमर्काभं दिव्यं पद्ममुपाहरत्}


\twolineshloka
{तदवैक्षत पाञ्चाली दिव्यगन्धं मनोरमम्}
{अनिलेनाहृतंभूमौ पतितं जलजं शुचि}


\twolineshloka
{तच्छुभा शुभमासाद्य सौगन्धिकमनुत्तमम्}
{अतीव मुदिता राजन्भीमसेनमथाब्रवीत्}


\twolineshloka
{पश्य दिव्यं सुरुचिरं भीम पुष्पमनुत्तमम्}
{गन्धसंस्थानसंपन्नं मनसो मम नन्दनम्}


\threelineshloka
{इदं च धर्मराजाय प्रदास्यामि परंतप}
{`गृह्यापराणि पुष्पाणि बहूनि पुरुषर्षभ'}
{हरेरिदं मे कामाय काम्यके पुनराश्रमे}


\twolineshloka
{यदि तेऽहं प्रिया पार्थ बहूनीमान्युपाहर}
{तान्यहं नेतुमिच्छामि काम्यकं पुनराश्रमम्}


\twolineshloka
{एवमुक्त्वा तु पाञ्चली भीमसेनमनिन्दिता}
{जगाम पुष्पमादाय धर्मराजाय तत्तदा}


\twolineshloka
{अभिप्रायं तु विज्ञाय महिष्याः पुरुषर्षभः}
{प्रियायाः प्रियकामः स प्रायाद्भीमो महाबलः}


\twolineshloka
{वातं तमेवाभिमुखो यतस्तत्पुष्पमागतम्}
{आजिहीर्षुर्जगामाशु सपुष्पाण्यपराण्यपि}


\threelineshloka
{रुक्मपृष्ठं धनुर्गृह्य शरांश्चाशीविषोपमान्}
{मृगराडिव संक्रुद्धः प्रभिन्न इव कुञ्जरः}
{[ददृशुः सर्वभूतानि महाबाणधनुर्धरम्}


\twolineshloka
{न ग्लानिर्न च वैक्लब्यं न भयं न च संभ्रमः}
{कदाचिज्जुषते पार्थमात्मजं मातरिश्वनः ॥]}


\twolineshloka
{द्रौपद्याः प्रियमन्विच्छन्स बाहुबलमाश्रितः}
{व्यपेतभयसमोहः शैलमभ्यपतद्बली}


\twolineshloka
{स तं द्रुमलतागुल्मच्छन्नं नीलशिलातलम्}
{गिरिं चचारारिहरः किन्नराचरितं शुभम्}


\twolineshloka
{नानावर्णधरैश्चित्रं धातुद्रुममृगाण्डजैः}
{सर्वभूषणसंपूर्णं भूमेर्भुजमिवोच्छ्रितम्}


\twolineshloka
{सर्वर्तुरमणीयेषु गन्धमादनसानुषु}
{सक्तचक्षुरभिप्रायान्हृदयेनानुचिन्तयन्}


\twolineshloka
{पुंस्कोकिलनिनादेषु षट्पदाचरितेषु च}
{बद्धश्रोत्रमनश्चक्षुर्जगामामितविक्रमः}


\twolineshloka
{आजिघ्रन्स महातेजाः सर्वर्तुकुसुमोद्भवम्}
{गन्धमुद्धतमुद्दामो वने मत्त इव द्विपः}


\twolineshloka
{वीज्यमानः सुपुण्येन नानाकुसुमगन्धिना}
{पितुः संस्पर्शशीतेन गन्धमादनवायुना}


% Check verse!
ह्रियमाणश्रमः पित्रा संप्रहृष्टतनूरुहः
\twolineshloka
{स यक्षगन्धर्वसुरब्रह्मर्षिगणसेवितम्}
{विलोकयामास तदा पुष्पहेतोररिंदमः}


\twolineshloka
{विषमच्छदैरचितैरनुलिप्त इवाङ्गुलैः}
{विमलैर्धातुविच्छेदैः काञ्चनाञ्जनराजतैः}


\twolineshloka
{सपक्षमिव नृत्यन्तं पार्श्वलग्नैः पयोधरैः}
{मुक्ताहारैरिव चितं च्युतैः प्रस्रवणोदकैः}


\twolineshloka
{अभिरामदरीकुञ्जनिर्झरोदककन्दरम्}
{अप्सरोनूपुररवैः प्रनृत्तवरबर्हिणम्}


\twolineshloka
{दिग्वारणविषाणाग्रैर्घृष्टोपलशिलातलम्}
{स्रस्तांशुकमिवाश्रोभ्यैर्निम्नगानिःसृतैर्जलैः}


\twolineshloka
{सशष्पकबलैः स्वस्थैरदूरपरिवर्तिभिः}
{भयानभिक्षज्ञैर्हरिणैः कौतूहलनिरीक्षितः}


\twolineshloka
{चालयानः स्ववेगेन लताजालान्यनेकशः}
{आक्रीडमानः कौन्तेयः श्रीमान्वायुसुतो ययौ}


\twolineshloka
{प्रियामनोरथं कर्तुमुद्यतश्चारुलोचनः}
{प्रांशुः कनकवर्णाभः सिंहसंहननो युवा}


\twolineshloka
{मत्तवारणविक्रन्तो मत्तवारणवेगवान्}
{मत्तवारणताम्राक्षो मत्तवारणवारणः}


\twolineshloka
{प्रियपार्श्वोपविष्टाभिर्वायवृत्ताभिर्विचेष्टितैः}
{यक्षगन्धर्वयोषाभिरदृश्याभिर्निरीक्षितः}


\twolineshloka
{नवावतारं रूपस् विक्रीडन्निव पाण्डवः}
{चचार रमणीयेषु गन्धमादनसानुषु}


\twolineshloka
{संस्मरन्विविधान्क्लेशान्दुर्योधनकृतान्बहून्}
{द्रौपद्या वनवासिन्याः प्रियं कर्तुं समुद्यतः}


\twolineshloka
{सोऽचिन्तयत्तथा पार्थे मयि त्वतिविलम्बिते}
{पुष्पहेतोः कथं त्वार्यः करिष्यति युधिष्ठिरः}


\twolineshloka
{स्नेहान्नरवरो नूनमविश्वासाद्बलस्य च}
{नकुलं सहदेवं च न मोक्ष्यति युधिष्ठिरः}


\twolineshloka
{कथं नु कुसुमावाप्तिः स्याच्छीघ्रमिति चिन्तयन्}
{प्रतस्थे नरशार्दूलः पक्षिराडिव वेगितः}


\twolineshloka
{[सज्जमानमनोदृष्टिः फुल्लेषु गिरिसानुषु}
{द्रौपदीवाक्यपाथेयो भीमः शीघ्रतरं ययौ ॥]}


\twolineshloka
{कम्पयन्मेदिनीं पद्भ्यां निर्घात इव पर्वसु}
{त्रासयन्गजयूथानि वातरंहा वृकोदरः}


\twolineshloka
{सिंहव्याघ्रमृगांश्चैव मर्दयानो महाबलः}
{उन्मूलयन्महावृक्षान्पोथयंश्चोरसा बली}


\twolineshloka
{लतावल्लीश्च वेगेन विकर्षन्पाण्डुनन्दनः}
{उपर्युपरि शैलाग्रमारुरुक्षुरिव द्विपः}


\twolineshloka
{जलावलम्बोऽतिभृशं सविद्युदिव तोयदः}
{`व्यनदत्स महानादं भीमसेनो महाबलः}


\twolineshloka
{तेन शब्देन महता भीमस्य प्रतिबोधिताः}
{गुहां संतत्यजुर्व्याघ्रा निलिल्युर्वनवासिनः}


\threelineshloka
{समुत्पेतुः खगास्त्रस्ता मृगयूथानि दुद्रुवुः}
{ऋक्षाश्चोत्ससृजर्वृक्षांस्तत्यजुर्हरयो गुहाम्}
{व्यजृम्भन्त महासिंहा महिषाश्च वनेचराः}


\twolineshloka
{तेन वित्रासिता नागाः करेणुपरिवारिताः}
{तद्वनं संपरित्यज्य जग्मुरन्यन्महावनम्}


\twolineshloka
{वराहमृगसङ्घाश्च महिषाश्च वनेचराः}
{व्याघ्रगोमायुसङ्घाश्च प्रणेदुर्गवयैः सह}


\twolineshloka
{रथाङ्गसाह्वदात्यूहा हंसकारण्डवप्लवाः}
{शुकाः पारावताः कौञ्चा विसंज्ञा भेजिरे दिशः}


\twolineshloka
{तथाऽन्ये दर्पिता नागाः करेणुशरपीडिताः}
{सिंहव्याघ्राश्च संक्रुद्धा भीमसेनमथाद्रवन्}


\twolineshloka
{शकृन्मूत्रं च मुञ्चाना भयविभ्रान्तमानसाः}
{व्यादितास्या महारौद्र व्यनदन्भीषणान्रवान्}


\threelineshloka
{ततो वायुसुतः क्रोधात्स्वबाहुबलमाश्रितः}
{गजेनान्यान्गजान्श्रीमान्सिंहं सिंहेन वा विभुः}
{तलप्रहारैरन्यांश्च व्यहनत्पाण्डवो बली}


\twolineshloka
{ते वध्यमाना भीमेन सिंहव्याघ्रतरक्षवः}
{भयाद्विससृजुर्भीमं शकृन्मूत्रं च सुस्रुवुः}


\twolineshloka
{प्रविवेश ततः क्षिप्रं तानपास्य महाबलः}
{वनं पाण्डुसुतः श्रीमाञ्शब्देनापूरयन्दिशः}


\twolineshloka
{अथापश्यन्महाबाहुर्गनधमादनसानुषु}
{सुरम्यं कदलीषणअडं बहुयोजनविस्तृतम्}


\twolineshloka
{तमभ्यगच्छद्वेगेन क्षोभयिष्यन्महाबलः}
{महागज इवास्रावी प्रभञ्जन्विविधान्द्रुमान्}


\threelineshloka
{उत्पाट्य कदलीस्तम्भान्बहुतालसमुच्छ्रयान्}
{चिक्षेप तरसा भीमः समन्ताद्बलिनां वरः}
{विमर्दन्सुमहातेजा नृसिंह इव दर्पितः}


\twolineshloka
{ततः संन्यपतंस्तत्रसुबहूनि महान्ति च}
{रुरुवारणयूथानि महिषाश्च जलाश्रयाः}


\twolineshloka
{`प्रविवेश ततः क्षिप्रं तानपास्य महाबलः}
{वनं पाण्डुसुतः श्रीमान्नादेनापूरयन्दिशः'}


\twolineshloka
{तेन शब्देन चैवाथ भीमसेनरवेण च}
{वनान्तरगताश्चापि वित्रेसुर्मृहपक्षिः}


\twolineshloka
{तस्मिननथ प्रवृत्ते तु संक्षोभे मृगपक्षिणाम्}
{जलार्द्रपक्षा विहगाः समुत्पेतुः सहस्रशः}


\twolineshloka
{तानौदकान्पक्षिगणान्निरीक्ष्य भरतर्षभः}
{तानेवानुसरन्रम्यं ददर्श सुमहत्सरः}


\twolineshloka
{काञ्चनैः कदलीषण्डैर्मन्दमारुतकम्पितैः}
{वीज्यमानमिवाक्षोभ्यं तीरात्तीरविसर्पिभिः}


\twolineshloka
{तत्सरोऽथावतीर्याशु प्रभूतनलिनोत्पलम्}
{महागज इवोद्दामश्चिक्रीड बलवद्बली}


\twolineshloka
{विक्रीड्यतस्मिन्रुचिरमुत्ततारामितद्युतिः}
{`क्षोभयन्सलिलं भीमः प्रभिन्न इववारणः'}


\threelineshloka
{ततो जगाहे वेगेन तद्वनं बहुपादपम्}
{दध्मौ च शङ्खं स्वनवत्सर्वप्राणेन पाण्डवः}
{[आस्फोटयच्च बलवान्भीमः संनादयन्दिशः ॥]}


\twolineshloka
{तस्य शङ्खस्य शब्दन भीमसेनरवेण च}
{बाहुशब्दन चोग्रेण नदन्तीव गिरेर्गुहाः}


\twolineshloka
{तं वज्रनिष्पेषसममास्फोटितमहारवम्}
{श्रुत्वा शैलगुहासुप्तैः सिंहैर्मुक्तो महास्वनः}


\twolineshloka
{सिंहनादभयत्रस्तैः कुञ्जरैरपि भारत}
{मुक्तो विरावः सुमहान्पर्वतो येन पूरितः}


\twolineshloka
{तं तु नादं ततः श्रुत्वा सुप्तो वानरपुङ्गवः}
{[भ्रातरं भीमसेनं तु विज्ञाय हनुमान्कपिः}


\twolineshloka
{दिवंगमं रुरोधाथ मार्गं भीमस्य कारणात्}
{अनेन हि पथा मा वै गच्छेदिति विचार्य सः}


\twolineshloka
{आस्त एकायने मार्गे कदलीषण्डमण्डिते}
{भ्रातुर्भीमस्य रक्षार्थं तं मार्गमवरुध्य वै}


\twolineshloka
{माऽत्र प्राप्स्यति शापं वा धर्षणां वेति पाण्डवः}
{कदलीषण्डमध्यस्थो ह्येवं संचिन्त्य वानरः}


\twolineshloka
{प्राजृम्भत महाकायो हनूमान्नाम वानरः}
{कदलीषण्डमध्यस्थो निद्रावशगतस्तदा}


\threelineshloka
{`तेन शब्देन महता व्यबुध्यत महाकपिः'}
{जृम्भमाणः सुविपुलं शक्रध्वजमिवोच्छ्रितम्}
{आस्फोटयच्चलाङ्गूलमिन्द्राशनिसमस्वनम्}


\twolineshloka
{तस्य लाङ्गूलनिनदं पर्वतः सुगुहामुखैः}
{उद्गारमिव गौर्नर्दन्नुत्ससर्ज समन्ततः}


\twolineshloka
{लाङ्गूलास्फोटशब्दाच्च चलितः स महागिरिः}
{विघूर्णमानशिखरः समन्तात्पर्यशीर्यत}


\twolineshloka
{स लाङ्गूलरवस्तस्य मत्तवारणनिस्वनम्}
{अन्तर्धायविचित्रेषु चचार गिरिसानुषु}


\twolineshloka
{स भीमसेनस्तच्छ्रुत्वा संप्रहृष्टतनूरुहः}
{शब्दप्रभवमन्विच्छंश्चचार कदलीवनम्}


\twolineshloka
{कदलीवनमध्यस्थमथ पीने शिलातले}
{ददर्श सुमहाबाहुर्वानराधिपतिं तदा}


\twolineshloka
{विद्युत्संपातदुष्प्रेक्षं विद्युत्संपातपिङ्गलम्}
{विद्युत्संपातनिनदं विद्युत्संपातचञ्चलम्}


\twolineshloka
{बाहुस्वस्तिकविन्यस्तपीनवृत्तशिरोधरम्}
{स्कन्धभूयिष्ठकायत्वात्तनुमध्यकटीतटम्}


\twolineshloka
{किंचिच्चाभुग्नशीर्षेण दीर्घरोमाञ्चितेन च}
{लाङ्गूलेनोर्ध्वगतिना ध्वजेनेव विराजितम्}


\twolineshloka
{ह्रस्वौष्ठं ताम्रजिह्वास्यं रक्तकर्णं चलद्धुवम्}
{अपश्यद्वदनं तस्य रश्मिवन्तमिवोडुपम्}


\twolineshloka
{वदनाभ्यन्तरगतैः शुक्लैर्दन्तैरलंकृतम्}
{केसरोत्करसंमिश्रमशोकानामिवोत्करम्}


\threelineshloka
{हिरण्मयीनां मध्यस्थं कदलीनां महाद्युतिम्}
{दीप्यमानेन वपुषा स्वर्चिष्मन्तमिवानलम्}
{निरीक्षन्तमपत्रस्तं लोचनैर्मधुपिङ्गलैः}


\twolineshloka
{तं वानरवरं धीमानतिकायं महाबलम्}
{स्वर्गपन्थानमावृत्य हिमवन्तमिव स्तितम्}


\twolineshloka
{दृष्ट्वा चैनं महाबाहुरेकं तस्मिन्महावने}
{अथोपसृत्यतरसा भीमो भीमपराक्रमः}


\twolineshloka
{सिंहनादं चकारोग्रं वज्राशनिसमं बली}
{तेन शब्देन भीमस्य वित्रेसुर्मृगपक्षिणः}


\twolineshloka
{हनूमांश्च महासत्व ईषदुन्मील्य लोचने}
{दृष्ट्वा तमथ सावज्ञं लोचनैर्मधुपिङ्गलैः}


\twolineshloka
{`ततः पवनजः श्रीमानन्तिकस्थं महौजसम्'}
{स्मितेन चैनमासाद्यहनूमानिदमब्रवीत्}


\twolineshloka
{किमर्थं सरुजस्तेऽहं सुखसुप्तः प्रबोधितः}
{ननु नाम त्वया कार्या दया भूतेषु जानता}


\twolineshloka
{वयं धर्मं न जानीमस्तिर्यग्योनिमुपाश्रिताः}
{नरास्तु बुद्धिसंपन्ना दयां कुर्वन्ति जन्तुषु}


\twolineshloka
{क्रूरेषु कर्मसु कथं देहवाक्चित्तदूषिषु}
{धर्मघातिषु सज्जन्ते बुद्दिमन्तो भवद्विधाः}


\twolineshloka
{न त्वं धर्मं विजानासि वृद्धा नोपासितास्त्वया}
{अल्पबुद्धितया बाल्यादुत्सादयसि यन्मृगान्}


\twolineshloka
{ब्रूहि कस्त्वं किमर्थं वा किमिदं वनमागतः}
{वर्जितं मानुषैर्भावैस्तथैव पुरुषैरपि}


\twolineshloka
{क्व च त्वयाऽद्यगन्तव्यं प्रब्रूहि पुरुषर्षभ}
{अतः परमगम्योऽयं पर्वतः सुदुरारुहः}


\twolineshloka
{विना सिद्धगतिं वीर गतिरत्र न विद्यते}
{[देवलोकस्य मार्गोऽयमगम्यो मानुषैः सदा]}


\twolineshloka
{कारुण्यात्त्वामहं वीर वारयामि निबोध मे}
{नातः परं त्वया शक्यं गन्तुमाश्वसिहि प्रभो}


\twolineshloka
{स्वागतं सर्वथैवेह तवाद्य मनुजर्षभ}
{इमान्यमृतकल्पानि मूलानि च फलानि च}


\twolineshloka
{भक्षयित्वा निवर्तस्व मा वृथा प्राप्स्यसे वधम्}
{ग्राह्यं यदि वचो मह्यं हितं मनुजपुङ्गव}


\chapter{अध्यायः १४९}
\twolineshloka
{वैशंपायन उवाच}
{}


\threelineshloka
{एतच्छ्रुत्वा वचस्तस्य वानरेन्द्रस्य धीमतः}
{भीमसेनस्तदा वीरः प्रोवाचामितविक्रमः ॥भीम उवाच}
{}


\twolineshloka
{को भवान्किंनिमित्तं वा वानरं वपुराश्रितः}
{ब्राह्मणानन्तरो वर्णः क्षत्रियस्त्वाऽनुपृच्छति}


\twolineshloka
{कौरवः सोमवंशीयः कुन्त्या गर्भेण धारितः}
{पाण्डवो वायुतनयो भीमसेन इति श्रुतः}


\twolineshloka
{स वाक्यं भीमसेनस्य स्मितेन प्रतिगृह्य तत्}
{हनुमान्वायुतनयो वायुपुत्रमभाषत}


\threelineshloka
{वानरोऽहं न ते मार्गं प्रदास्यामि यथोप्सितम्}
{साधु गच्छ निवर्तस्व मा त्वं प्राप्स्यसि वैशसम् ॥भीमसेन उवाच}
{}


\threelineshloka
{वैशसं वाऽस्तु यद्वान्यन्न त्वां रपृच्चामि वानर}
{प्रयच्छ मार्गमुत्तिष्ठ मा मत्तः प्राप्स्यसे व्यथाम् ॥हनूमानुवाच}
{}


\threelineshloka
{नास्ति शक्तिर्ममोत्थातुं जरया क्लेशितो ह्यहम्}
{यद्यवश्यं प्रयातव्यं लङ्घयित्वा प्रयाहि माम् ॥भीम उवाच}
{}


\twolineshloka
{निर्गुणः परमात्मा तु देहं ते व्याप्य तिष्ठते}
{तमहं ज्ञानविज्ञेयं नावमन्ये न लङ्घये}


\threelineshloka
{यद्यागमैर्न विद्यां च तमहं भूतभावनम्}
{क्रमेयं त्वां गिरिं चैव हनीमानिव सागरम् ॥हनूमानुवाच}
{}


\threelineshloka
{क एष हनुमान्नाम सागरो येन लङ्घितः}
{पृच्चामि त्वां नरश्रेष्ठ कथ्यतां यदि शक्यते ॥भीम उवाच}
{}


\twolineshloka
{भ्राता मम गुणश्लाघ्यो बुद्धिसत्वबलान्वितः}
{रामायणेऽतिविख्यातः श्रीमान्वानरपुङ्गवः}


\twolineshloka
{रामपत्नीकृते येन शतयोजनविस्तृतः}
{सागरः प्लवगेन्द्रेण क्रमेणैकेन लङ्घितः}


\twolineshloka
{स मे भ्राता महावीर्यस्तुल्योऽहं तस् तेजसा}
{बले पराक्रमे युद्धे शक्तोऽहं तव निग्रहे}


\threelineshloka
{`इमं देशमनुप्राप्तः कारणेनास्मि केनचित्}
{'उत्तिष्ठ देहि मे मार्गं पश्य मे चाद्य पौरुषम्}
{मच्छासनमकुर्वाणं त्वां वा नेष्ये यमक्षयम् ॥वैशंपायन उवाच}


\twolineshloka
{विज्ञाय तं बलोन्मत्तं बाहुवीर्येण दर्पितम्}
{हृदयेनावहस्यैनं हनूमान्वाक्यमब्रवीत्}


\threelineshloka
{प्रसीद नास्ति मे शक्तिरुत्थातुं जरयाऽनघ}
{ममानुकम्पया त्वेतत्पुच्छमुत्सार्य गम्यताम् ॥वैशंपायन उवाच}
{}


\twolineshloka
{[एवमुक्ते हनुमता हीनवीर्यपराक्रमम्}
{मनसाऽचिनतयद्भीमः स्वबाहुबलदर्पितः}


\twolineshloka
{पुच्छे प्रगृह्य तरसा हीनवीर्यपराक्रमम्}
{सालोक्यमन्तकस्यैनं नयाम्यद्येह वानरम् ॥]}


\twolineshloka
{सावज्ञमथ वामेन स्मयञ्जग्राह पाणिना}
{न चाशकच्चालयितुं भीमः पुच्छं महाकपेः}


\twolineshloka
{उच्चिक्षेप पुनर्दोर्भ्यामिन्द्रायुधमिवोच्छ्रितम्}
{नोद्धर्तुमशकद्भीमो दोर्भ्यामपि महाबलः}


\twolineshloka
{उत्क्षिप्तभ्रूर्विवृत्ताक्षः संहतभ्रकुटीमुखः}
{स्विन्नगात्रोऽभवद्भीमो न चोद्धर्तुंशशाक तम्}


\twolineshloka
{यत्नवानपि तु श्रीमाँल्लाङ्गूलोद्धरणे ततः}
{कपेः पार्श्वगतो भीमस्तस्थौ व्रीडडानताननः}


\twolineshloka
{प्रणिपत्य च कौन्तेयः प्राञ्जलिर्वाक्यमब्रवीत्}
{प्रसीद कपिशार्दूल दुरुक्तं क्षम्यतां मम}


\twolineshloka
{सिद्धो वा यदि वा देवो गन्धर्वो वाऽथ गुह्यकः}
{पृष्टः सन्को मया ब्रूहि कस्त्वं वानररूपधृत्}


\threelineshloka
{न चेद्गुह्यं महाबाहो श्रोतव्यं चेद्भवेन्मम}
{शिष्यवत्त्वां तु पृच्छामि उपपन्नोऽस्मि तेऽनघ ॥हनूमानुवाच}
{}


\twolineshloka
{यत्ते मम परिज्ञाने कौतूहलमरिंदम}
{तत्सर्वमखिलेन त्वं शृणु पाण्डवनन्दन}


\twolineshloka
{अहं केसरिणः क्षेत्रे वायुना जगदायुषा}
{जातः कमलपत्राक्ष हनूमान्नाम वानरः}


\twolineshloka
{सूर्यपुत्रं च सुग्रीवं शक्रपुत्रं च वालिनम्}
{सर्ववानरराजानौ सर्ववानरयूथपाः}


\twolineshloka
{उपतस्थुर्महावीर्या मम चामित्रकर्शन}
{सुग्रीवेणाभवत्प्रीतिरनिलस्याग्निना यथा}


\twolineshloka
{निकृतः स ततो भ्रात्रा कस्मिंश्चित्कारणान्तरे}
{ऋश्यमूके मया सार्धं सुग्रीवो न्यवसच्चिरम्}


\twolineshloka
{अथ दाशरथिर्वीरो रामो नाम महाबलः}
{विष्णुर्मानुषरूपेण चचार वसुधातलम्}


\twolineshloka
{स पितुः प्रियमन्विच्छन्सहभार्यः सहानुजः}
{सधनुर्धन्विनांश्रेष्ठो दण्डकारण्यमाश्रितः}


\twolineshloka
{तस्य भार्या जनस्थानाच्छलेनापहृता बलात्}
{राक्षसेन्द्रेण बलिना रावणेन दुरात्मना}


\twolineshloka
{सुवर्णरत्नचित्रेण मृगरूपेण रक्षसा}
{वञ्चयित्वा नरव्याघ्रं मारीचेन तदाऽनघ}


\chapter{अध्यायः १५०}
\twolineshloka
{हनूमानुवाच}
{}


\twolineshloka
{हृतदारः सह भ्रात्रा पत्नीं मार्गन्स राघवः}
{दृष्टवाञ्शैलशिखरे सुग्रीवं वानरर्षभम्}


\twolineshloka
{तेन तस्याभवत्सख्यं राघवस्य महात्मनः}
{स हत्वा वालिनं राज्ये सुग्रीवं प्रत्यपादयत्}


\twolineshloka
{स राज्यं प्राप्य सुग्रीवः सीतायाः परिमार्गणे}
{वानरान्प्रेषयामास शतशोऽथ सहस्रशः}


\twolineshloka
{ततो वानरकोटीभिः सहितोऽहं नरर्षभ}
{सीतां मार्गन्महाबाहो प्रस्थितो दक्षिणां दिशम्}


\twolineshloka
{ततः प्रवृत्तिः सीताया गृध्रेण सुमहात्मना}
{संपातिना समाख्याता रावणस्य निवेशने}


\twolineshloka
{ततोऽहं कार्यसिद्ध्यर्थं रामस्याक्लिष्टकर्मणः}
{शतयोजनविस्तारमर्णवं सहसा प्लुतः}


\twolineshloka
{अहं स्ववीर्यादुत्तीर्य सागरं मकरालयम्}
{सुतां जनकराजस्य सीतां सुररसुतोपमाम्}


\twolineshloka
{दृष्टवान्भरतश्रेष्ठ रावणस्य निवेशने}
{समेत्य तामहं देवीं वैदेहीं राघवप्रियाम्}


\twolineshloka
{दग्ध्वा लङ्कामशेषेण सादृप्राकारतोरणाम्}
{प्रत्यागतश्चास्य पुनर्नाम तत्रप्रकाश्य वै}


\threelineshloka
{मद्वाक्यं चावधार्याशु रामो राजीवलोचनः}
{अबद्धपूर्वमन्यैश्च बद्ध्वा सेतुं महोदधौ}
{वृतो वानरकोटीभिः समुत्तीर्णो महार्णवम्}


\twolineshloka
{ततो ररामेण वीर्येण हत्वा तान्सर्वराक्षसान्}
{रणे तु राक्षसगणं रावणं लोकरावणम्}


\twolineshloka
{निशाचरेनद्रं हत्वा तु सभ्रातृसुतबान्धवम्}
{राज्येऽभिषिच्य लङ्कायां राक्षसेन्द्रं विभीषणम्}


\twolineshloka
{धार्मिकं भक्तिमन्तं च भक्तानुगतवत्सलः}
{प्रत्याहृत्य ततः सीतां नष्टां वेदश्रुतिं यथा}


\threelineshloka
{तयैव सहितः साध्व्या पत्न्या रामो महायशाः}
{गत्वा ततोऽतित्वरितः स्वां पुरीं रघुनन्दनः}
{अध्यावसत्ततोऽयोध्यामयोध्यां द्विषतां प्रभुः}


\twolineshloka
{ततः प्रतिष्ठितो राज्ये रामो नृपतिसत्तमः}
{वरं मया याचितोऽसौ रामो राजीवलोचनः}


\twolineshloka
{यावद्रामकथेयं ते भवेल्लोकेषु शत्रुहन्}
{तावज्जीवेयमित्येवं तथाऽस्त्विति च सोब्रवीत्}


\twolineshloka
{सीताप्रसादाच्च सदा मामिहस्थमरिंदम}
{उपतिष्ठन्ति दिव्या हि भोगा भीम यथेप्सिताः}


\twolineshloka
{दशवर्षसहस्राणि दशवर्षशतानि च}
{राज्यं कारितवान्रामस्ततः स्वभवनं गतः}


\twolineshloka
{तदिहाप्सरसस्तात गन्धर्वाश्च सदाऽनघ}
{तस्य वीरस्य चरितं गायन्त्यो रमयन्ति माम्}


\twolineshloka
{अयं च मार्गो मर्त्यानामगम्यः कुरुनन्दन}
{ततोऽहं रुद्धवान्मार्गं तवेमं देवसेवितम्}


\twolineshloka
{`त्वामनेन पथा यान्तं यक्षो वा राक्षसोपि वा'}
{धर्षयेद्वा शपेद्वाऽपि मा कश्चिदिति भारत}


\twolineshloka
{दिव्यो देवपथो ह्येष नात्र गच्छन्ति मानुषाः}
{यदर्थमागतश्चासि अत एव सरश्च तत्}


\chapter{अध्यायः १५१}
\twolineshloka
{वैशंपायन उवाच}
{}


\twolineshloka
{एवमुक्तो महाबाहुर्भीमसेनः प्रतापवान्}
{प्रणिपत्य ततः प्रीत्या भ्रातरं हृष्टमानसः}


\twolineshloka
{उवाच श्लक्ष्णया वाचा हनूमन्तं कपीश्वरम्}
{मत्तो धन्यतरो नास्ति यदार्यं दृष्टवानहम्}


\twolineshloka
{अनुग्रहो मे सुमहांस्तृप्तिश्च तव दर्शनात्}
{एतत्तु कृतमिच्छामि त्वयाऽऽर्येण प्रियं मम}


\twolineshloka
{यत्त्वेतदासीत्प्लवतः सागरं मकरालयम्}
{रूपमप्रतिमं वीर तदिच्छामि नीरीक्षितुम्}


\twolineshloka
{एवं तुष्टो मभिष्यामि श्रद्धास्यामि च ते वचः}
{एवमुक्तः स तेजस्वी प्रहस्य हरिरब्रवीत्}


\threelineshloka
{न तच्छक्यं त्वया द्रष्टुं रूपं नान्येन केनचित्}
{कालावस्था तदा ह्यन्या वर्तते सा न सांप्रतम्}
{`ततोऽद्य दुष्करं द्रष्टुं मम रूपं नरोत्तम'}


\twolineshloka
{अन्यः कृतयुगे कालश्रेतायां द्वापरे परः}
{अयं प्रध्वंसनः कालो नाद्य तद्रूपमस्ति मे}


\twolineshloka
{भूमिर्नद्यो नगाः शैलाः सिद्धा देवा महर्षयः}
{कालं समनुवर्तन्ते यथा भावा युगेयुगे}


\twolineshloka
{`कालंकालं समासाद्य नराणां नरपुङ्गव'}
{बलवर्ष्मप्रभावा हि प्रहीयन्त्युद्भवन्ति च}


\threelineshloka
{तदलं बत तद्रूपं द्रष्टुं कुरुकुलोद्वह}
{युगं समनुवर्तामि कालो हि दुरतिक्रमः ॥भीम उवाच}
{}


\threelineshloka
{युगसङ्ख्यां समाचक्ष्व आचारं च युगेयुगे}
{धर्मकामार्थभावांश्च कर्मवीर्ये भवाभवौ ॥हनूमानुवाच}
{}


\twolineshloka
{कृतं नाम युगं श्रेष्ठं यत्रधर्मः सनातनः}
{कृतमेव न कर्तव्यं तस्मिन्काले युगोत्तमे}


\twolineshloka
{न तत्र धर्माः सीदन्ति क्षीयन्ते न च वै प्रजाः}
{ततः कृतयुगं नाम कालेन गुणतां गतम्}


\twolineshloka
{देवदानवगन्धर्वयक्षराक्षसपन्नगाः}
{नासन्कृतयुगे तात तदा न क्रयविक्रयः}


\twolineshloka
{न सामऋग्यजुर्वर्णाः क्रिया नासीच्च मानवी}
{अभिध्याय फलं तत्र धर्मः संन्यास एव च}


\twolineshloka
{न तस्मिन्युगसंसर्गे व्याधयो नेन्द्रियक्षयः}
{नासूया नापि रुदितं न दर्पो नापि वैकृतम्}


\twolineshloka
{न विग्रहः कुतस्तन्द्री न द्वेषो न च पैशुनम्}
{न भयं नापि संतापो न चेर्ष्या न च मत्सरः}


\twolineshloka
{ततः परमकं ब्रह्म सा गतिर्योगिनां परा}
{आत्मा च सर्वभूतानां शुक्लो नारायणस्तदा}


\twolineshloka
{ब्राह्मणआः क्षत्रिया वैश्याः शूद्राश्च कृतलक्षणाः}
{कृते युगे समभवन्स्वकर्मनिरताः प्रजाः}


\twolineshloka
{समाश्रयं समाचारं समज्ञानं च केवलम्}
{तदा हि समकर्माणो वर्णा धर्मानवाप्नुवन्}


\twolineshloka
{एकदेवसदायुक्ता एकमन्त्रविधिक्रियाः}
{पृथग्धर्मास्त्वेकवेदा धर्ममेकमनुव्रताः}


\twolineshloka
{चातुराश्रम्ययुक्तेन कर्मणा कालयोगिना}
{अकामफलसंयोगात्प्राप्नुवन्ति परां गतिम्}


\twolineshloka
{आत्मयोगसमायुक्तो धर्मोऽयं कृतलक्षणः}
{कृते युगे चतुष्पादश्चातुर्वर्ण्यस्य शाश्वतः}


\twolineshloka
{`कामः कामयमानेषु ब्राह्मणेषु तिरोहितः'}
{एतत्कृतयुगं नाम त्रैगुण्यपरिवर्जितम्}


\twolineshloka
{त्रेतामपि निबोध त्वं यस्मन्सत्रं प्रवर्तते}
{पादेन ह्रसते धर्मो रक्ततां याति चाच्युतः}


\twolineshloka
{सत्यप्रवृत्ताश्च नराः क्रिया धर्मपरायणाः}
{ततो यज्ञाः प्रवर्तन्ते धर्माश्चविविधाः क्रियाः}


\threelineshloka
{त्रेतायां भावसंकल्पाः क्रियादानफलोपगाः}
{प्रचलन्ति न वै धर्मात्तपोदानपरायणाः}
{स्वधर्मस्थाः क्रियावन्तो नरास्त्रेतायुगेऽभवन्}


\twolineshloka
{द्वापरे च युगे धर्मो द्विभागो नः प्रवर्तते}
{विष्णुर्वै पीततां याति चतुर्धा वेद एव च}


\twolineshloka
{ततोऽन्ये च चतुर्वेदास्त्रिवेदाश्च तथा परे}
{द्विवेदाश्चैकवेदाश्चाप्यनृचश्च तथा परे}


\twolineshloka
{एवं शास्त्रेषु भिननेषु बहुधा नीयते क्रिया}
{तपोदानप्रवृत्ता च राजसी भवति प्रजा}


\twolineshloka
{एकवेदस्य चाज्ञानाद्वेदास्ते बहवः कृताः}
{सत्यस्य चेह विभ्रंशात्सत्ये कश्चिदवस्थितः}


\twolineshloka
{सत्यात्प्रच्यवमानानां व्याघयो बहवोऽभवन्}
{कामाश्चोपद्रवाश्चैव तदा वै दैवकारिताः}


\twolineshloka
{यैरर्द्यभानाः सुभृशं तपस्तप्यन्ति मानवाः}
{कामकामाः स्वर्गकामा यज्ञांस्तन्वन्ति चापरे}


\twolineshloka
{एवं द्वापरमासाद्य प्रजाः क्षीयन्त्यधर्मतः}
{पादेनैकेन कौन्तेय धर्मः कलियुगे स्थितः}


\twolineshloka
{तामसं युगमासाद्य कृष्णो भवति केशवः}
{वेदाचाराः प्रशाम्यन्ति धर्मयज्ञक्रियास्तथा}


\twolineshloka
{ईतयो व्याधयस्तन्द्री दोषाः क्रोधादयस्तथा}
{उपद्रवाः प्रवर्तन्ते आधयो व्याधयस्तथा}


\twolineshloka
{युगेष्वावर्तमानेषु धर्मो व्यावर्तते पुनः}
{धर्मे व्यावर्तमाने तु लोको व्यावर्तते पुनः}


\twolineshloka
{लोके क्षीणे क्षयं यानति भावा लोकप्रवर्तकाः}
{युगक्षयकृता धर्माः प्रार्थनानि विकुर्वते}


\twolineshloka
{एतत्कलियुगं नाम अचिराद्यत्प्रवर्तते}
{युगानुवर्तनं त्वेतत्कुर्वन्ति चिरजीविनः}


\twolineshloka
{यच्च ते मत्परिज्ञाने कौतूहलमरिंदम}
{अनर्थकेषु को भावः पुरुषस्य विजानतः}


\twolineshloka
{एतत्ते सर्वमाख्यातं यन्मां त्वं परिपृच्छसि}
{युगसङ्ख्यां महाबाहो स्वस्ति प्राप्नुहि गम्यताम्}


\chapter{अध्यायः १५२}
\twolineshloka
{भीमसेन उवाच}
{}


\threelineshloka
{पूर्वरूपमदृष्ट्वा ते न यास्यामि कथंचन}
{यदि तेऽहमनुग्राह्यो दर्शयात्मानमात्मना ॥वैशंपायन उवाच}
{}


\twolineshloka
{एवमुक्तस्तु भीमेन स्मितं कृत्वा प्लवंगमः}
{तद्रूपं दर्शयामास यद्वै सागरलङ्घने}


\twolineshloka
{भ्रातुः प्रियमभीप्सन्वै चकार सुमहद्वपुः}
{`तद्रूपं यत्पुरा तस्य बभूवोदधिलङ्घने'}


\twolineshloka
{देहस्तस्य ततोऽतीव वर्धत्यायामविस्तरैः}
{सद्रुमं कदलीषण्डं छादयन्नमितद्युतिः}


\twolineshloka
{गिरिश्चोच्छ्रयमाक्रम्य तस्थौ तत्र च वानरः}
{समुच्छ्रितमहाकायो द्वितीय इव पर्वतः}


\twolineshloka
{ताम्रेक्षणस्तीक्ष्णदंष्ट्रो भृकुटीकृतलोचनः}
{दीर्घं लाङ्गूलमाविध्य दिशो व्याप्य स्थितः कपिः}


\twolineshloka
{तद्रूपं महदालक्ष्यभ्रातुः कौरवनन्दनः}
{विसिष्मिये तदा भीमो जहृषे च पुनः पुनः}


\twolineshloka
{तमर्कमिव तेजोभिः सौवर्णमिव पर्वतम्}
{प्रदीप्तमिव चाकाशं दृष्ट्वा भीमो न्यमीलयत्}


\twolineshloka
{आबभाषे च हनुमान्भीमसेनं स्मयन्निव}
{एतावदिह शक्तस्त्वं द्रुष्टुं रूपं ममानघ}


\threelineshloka
{वर्धेऽहं चाप्यतो भूयो यावन्मे मनसेप्सितम्}
{भीम शत्रुषु चात्यर्थं वर्धते मूर्तिरोजसा ॥वैशंपायन उवाच}
{}


\twolineshloka
{तदद्भुतं महारौद्रं विन्ध्यपर्वतसन्निभम्}
{दृष्ट्वा हनूमतो वर्ष्म संभ्रान्तः पवनात्मजः}


\twolineshloka
{प्रत्युवाच ततो भीमः संप्रहृष्टतनूरुहः}
{कृताञ्जलिरदीनात्मा हनूमन्तमवस्थितम्}


\twolineshloka
{दृष्टं प्रमाणं विपुलं शरीरस्यास्य ते विभो}
{संहरस्व महावीर्य स्वयमात्मानमात्मना}


\twolineshloka
{न हि शक्नोमि त्वां द्रष्टुं दिवाकरमिवोदितम्}
{अप्रमेयमनाधृष्यं मैनाकमिव पर्वतम्}


\twolineshloka
{विस्मयश्चैव मे वीर सुमहान्मनसोऽद्य वै}
{यद्रामस्त्वयि पार्श्वश्थे स्वयं रावणमभ्यगात्}


\twolineshloka
{त्वमेव शक्तस्तां लङ्कां सयोधां सहरावणाम्}
{स्वबाहुबलमाश्रित् विनाशयितुमञ्जसा}


\twolineshloka
{न हि ते किंचिदप्राप्यं मारुतात्मज विद्यते}
{न चैव तव पर्याप्तो रावयणः सगणो युधि}


\twolineshloka
{एवमुक्तस्तु भीमेन हनूमान्प्लवगोत्तमः}
{प्रत्युवाच ततो वाक्यं स्निग्धगम्भीरया गिरा}


\twolineshloka
{एवमेतन्महाबाहो यथा वदसि भारत}
{भीमसेन न पर्याप्तो ममासौ राक्षसाधमः}


\twolineshloka
{मया तु निहते तस्मिन्रावणे लोककण्टके}
{कीर्तिर्नश्येद्राघवस्य तत एतदुपेक्षितम्}


\twolineshloka
{तेन वीरेण तं हत्वा सगणं राक्षसाधमम्}
{आनीता स्वपुरं सीता कीर्तिश्च स्थापिता नृषु}


\twolineshloka
{तद्गच्छ विपुलप्रज्ञ भ्रातुः प्रियहिते रतः}
{अरिष्टं क्षेममध्वानं वायुना परिरक्षितः}


\twolineshloka
{एष पन्थाः कुरुश्रेष्ठ सौगन्धिकवनाय ते}
{द्रक्ष्यसे धनदोद्यानं रक्षितं यक्षराक्षसैः}


\twolineshloka
{न च ते तरसा कार्यः कुसुमापचयः स्वयम्}
{दैवतानि हि मान्यानि पुरुषेण विशेषतः}


\twolineshloka
{बलिहोमनमस्कारैर्मन्त्रैश्च भरतर्षभ}
{दैवतानि प्रसादं हि भक्त्या कुर्वन्ति भारत}


\twolineshloka
{मा तात साहसं कार्षीः स्वधर्मं परिपालय}
{स्वधर्मस्थापनं धर्मं बुध्यस्वागमयस्व च}


\twolineshloka
{न हि धर्ममविज्ञाय वृद्धाननुपसेव्य च}
{धर्मो वै वेदितुं शक्यो बृहस्पतिसमैरपि}


\twolineshloka
{अधर्मो यत्र धर्माख्यो धर्मश्चाधर्मसंज्ञितः}
{स विज्ञेयो विभागेन यत्रमुह्यन्त्यबुद्धयः}


\twolineshloka
{आचारसंभवो धर्मो धर्माद्वेदाः समुत्थिताः}
{वेदैर्यज्ञाः समुत्पन्ना यज्ञैर्देवाः प्रतिष्ठिताः}


\twolineshloka
{वेदाचारविधानोक्तैर्यज्ञैर्धार्यन्ति देवताः}
{बृहस्पत्युशनःप्रोक्तैर्नयैर्धार्यन्ति मानवाः}


\twolineshloka
{बल्याकरवणिज्याभिः कृष्यर्थैर्योनिपोषणैः}
{वार्तया धार्यते सर्वं धर्मैरेतैर्द्विजातिभिः}


\twolineshloka
{त्रयी वार्ता दण्डनीतिस्तिस्रो विद्या विजानताम्}
{ताभिः सम्यक्प्रवृत्ताभिर्लोकयात्रा विधीयते}


\twolineshloka
{सा चेद्धर्मकृता न स्यात्रयीधर्ममृते भुवि}
{दण्डनीतिमृतेचापि निर्मर्यादमिदं भवेत्}


\twolineshloka
{वार्ताधर्मे ह्यवर्तिन्यो विनश्येयुरिमाः प्रजाः}
{सुप्रवृत्तैस्त्रिभिर्ह्येतैर्धर्मैः सूयन्ति वै प्रजाः}


\twolineshloka
{द्विजानाममृतं धर्मो ह्येकश्चैवैकवर्णकः}
{यज्ञाध्ययनदानानि त्रयः साधारणाः स्मृताः}


\twolineshloka
{याजनाध्यापने चोभे ब्राह्मणानां परिग्रहः}
{पालनं क्षत्रियाणां वै वैश्यधर्मश्च पोषणम्}


\twolineshloka
{शुश्रूषा च द्विजातीनां शूद्राणआं धर्म उच्यते}
{भैक्षहोमव्रतैर्हीनास्तथैव गुरुवासिताः}


\twolineshloka
{क्षत्रधर्मोऽत्रकौन्तेय तव धर्मोऽत्र रक्षणम्}
{स्वधर्मं प्रतिपद्यस्व विनीतो नियतेन्द्रियः}


\twolineshloka
{वृद्धैः संमन्त्र्य सद्भिश्च बुद्धिमद्भिः श्रुतान्वितैः}
{आस्थितः शास्तिदण्डेन व्यसनी परिभूयते}


\twolineshloka
{निग्रहानुग्रहैः सम्यग्यदा नेता प्रवर्तते}
{तदा भवन्ति लोकस्य मर्यादाः सुव्यवस्थिताः}


\twolineshloka
{तस्माद्देशे च दुर्गे च शत्रुमित्रबलेषु च}
{नित्यं चारेण बोद्धव्यं स्थानं वृद्धिः क्षयस्तथा}


\twolineshloka
{राज्ञामुपायश्चारश्च बुद्धिमन्त्रपराक्रमाः}
{निग्रहप्रग्रहौ चैव दाक्ष्यं वै कार्यसाधकम्}


\twolineshloka
{साम्ना दानेन भेदेन दण्डेनोपेक्षणेन च}
{साधनीयानि कार्याणि समासव्यासयोगतः}


\twolineshloka
{मन्त्रमूला नयाः सर्वेचाराश्च भरतर्षभ}
{सुमन्त्रितनयैः सद्भिस्तद्विधैः सह मन्त्रयेत्}


\twolineshloka
{स्त्रिया मूढेन बालेन लुब्धेन लघुनाऽपि वा}
{न मन्त्रयीत गुह्यानि येषु चास्पनदलक्षणम्}


\twolineshloka
{मन्त्रयेत्सह विद्वद्भिः शक्तैः कर्माणि कारयेत्}
{स्निग्धैश्च नीतिविन्यासैर्मूर्खान्सर्वत्र वर्जयेत्}


\twolineshloka
{धार्मिकान्धर्मकार्येषु अर्थकार्येषु पण्डितान्}
{स्त्रीषु क्लीबान्नियुञ्जीत क्रूरान्क्रूरेषु कर्मसु}


\twolineshloka
{स्वेभ्यश्चैव परेभ्यश्च कार्याकार्यसमुद्भवा}
{बुद्धिः कर्मसु विज्ञेया रिपूणां च बलाबलम्}


\twolineshloka
{बुद्ध्या स्वप्रतिपन्नेषु कुर्यात्साधुष्वनुग्रहम्}
{निग्रहं चाप्यशिष्टेषु निर्मर्यादेषु कारयेत्}


\twolineshloka
{निग्रहे प्रग्रहे सम्यग्यदा राजा प्रवर्तते}
{तदा भवति लोकस्य मर्यादा सुव्यवस्थिता}


\twolineshloka
{एष तेऽभिहितः पार्थ घोरो धर्मो दुरन्वयः}
{तं स्वधर्मविभागेन विनयस्थोऽनुपालय}


\twolineshloka
{तपोधर्मदमेज्याभिर्विप्रा यान्ति यथा दिवम्}
{दानातिथ्यक्रियाधर्मैर्यान्ति वैश्याश्च सद्गतिम्}


\twolineshloka
{`द्विजशुश्रूषया शूद्रा लभन्ते गतिमुत्तमाम्'}
{क्षत्रं याति तथा स्वर्गं भुवि निग्रहपालनैः}


\twolineshloka
{सम्यक्प्रणीतदण्डा हि कामद्वेपविवर्जिताः}
{अलुब्धा विगतक्रोधाः सतां यान्ति सलोकतां}


\chapter{अध्यायः १५३}
\twolineshloka
{वैशंपायन उवाच}
{}


\twolineshloka
{ततः संहृत्य विपुलं तद्वपुः कामवर्धितम्}
{भीमसेनं पुनर्दोर्भ्यां पर्यष्वजत वानरः}


\twolineshloka
{परिष्वक्तस्य तस्याशु भ्रात्रा भीमस्य भारत}
{श्रमो नाशमुपागच्छत्सर्वं चासीत्प्रदक्षिणम्}


\twolineshloka
{[बलं चातिबलो मेने न मेऽस्ति सदृशो महान्}
{]ततः पुनरथोवाच पर्यश्रुनयनो हरिः}


\twolineshloka
{भीममाभाष्य सौहार्दाद्बाष्पगद्गदया गिरा}
{गच्छ वीर स्वमावासं स्मर्तव्योऽस्मि कथान्तरे}


\twolineshloka
{इहस्थश्च कुरुश्रेष्ठ न निवेद्योस्मि कस्यचित्}
{धनदस्यालयाच्चापि विसृष्टानां महाबल}


\twolineshloka
{एष काल इहायातुं देवगन्धर्वयोषिताम्}
{ममापि सफलं च क्षुः स्मारितश्चास्मि राघवम्}


\twolineshloka
{[रामाभिधानं विष्णुं हि जगद्धृदयनन्दनम्}
{सीतावक्रारविन्दार्कं दशास्यध्वान्तभास्करम् ॥]}


\twolineshloka
{मानुषं गात्रसंस्पर्शं गत्वा भीम त्वया सह}
{तदस्मद्दर्शनं वीर कौन्तेयामाघमस्तु ते}


\twolineshloka
{भ्रातृत्वं त्वं पुरस्कृत्य वरं वरय भारत}
{यदि तावन्मया क्षुद्रा गत्वा वारणसाह्वयम्}


\twolineshloka
{धार्तराष्ट्रा निहन्तव्या यावदेतत्करोम्यहम्}
{शिलया नगरं वा तन्मर्दितव्यं मया यदि}


\twolineshloka
{[बद्ध्वा दुर्योधनं चाद्य आनयामि तवान्तिकम्}
{]यावदेतत्करोम्यद्यकामं तव महाबल}


\twolineshloka
{भीमसेनस्तु तद्वाक्यं श्रुत्वा तस् महात्मनः}
{प्रत्युवाच हनून्तं प्रहृष्टेनान्तरात्मना}


\twolineshloka
{कृतमेव त्वया सर्वं मम वानरपुङ्गव}
{स्वस्ति तेऽस्तु महाबाहो कामये त्वां प्रसीदमे}


\twolineshloka
{सनाथाः पाण्डवाः सर्वे त्वया नाथेन वीर्यवन्}
{तवैव तेजसा सर्वान्विजेष्यामो वयं परान्}


\twolineshloka
{एवमुक्तस्तु हनुमान्भीमसेनमभाषत}
{भ्रातृत्वात्सौहृदाच्चैव करिष्यामि प्रियं तव}


\twolineshloka
{चमूं विगाह्य शत्रूणां परशक्तिसमाकुलाम्}
{यदा सिंहरवं वीर करिष्यसि महाबल}


\twolineshloka
{तदाहं बृंहयिष्यामि स्वरवेण रवं तव}
{`यं श्रुत्वैव भविष्यन्ति व्यसवस्तेऽरयो रणे'}


\twolineshloka
{विजयस्य ध्वजस्थश्च नादान्मोक्ष्पामि दारुणान्}
{शत्रूणां ये प्राणहराः सुखं येन हनिष्यथ}


\twolineshloka
{एवमाभाष्य हनुमांस्तदा पाण्डवनन्दनम्}
{मार्गमाख्याय भीमाय तत्रैवान्तरधीयत}


\twolineshloka
{गते तस्मिन्हरिवरे भीमोपि बलिनांवरः}
{तेन मार्गेण विपुलं व्यचरद्गन्धमादनम्}


\twolineshloka
{अनुस्मरन्वपुस्तस्य श्रियं चाप्रतिमां भुवि}
{माहात्म्यमनुभावं च स्मरन्दाशरथेर्ययौ}


\twolineshloka
{स तानि रमणीयानि वनान्युपवनानि च}
{विलोलयामास तदा सौगन्धिकवनेप्सया}


\twolineshloka
{फुल्लपद्मविचित्राणि सरांसि सरितस्तथा}
{नानाकुसुमचित्राणि पुष्पितानि वनानि च}


\twolineshloka
{मत्तवारणयुथानि पङ्कक्लिन्नानि भारत}
{वर्षतामिव मेघानां वृन्दानि ददृशे तदा}


\twolineshloka
{हरिणैश्चपलापाङ्गैर्हरिणीसहितैर्वनम्}
{सशष्पकवलैः श्रीमान्पथि दृष्ट्वा द्रुतं ययौ}


\twolineshloka
{महिषैश्च वराहैश्च शार्दूलैश्च निषेवितम्}
{व्यपेतभीर्गिरिं शौर्याद्भीमसेनो व्यगाहत}


\twolineshloka
{कुसुमानतशाखैश्च ताम्रपल्लवकोमलैः}
{याच्यमान इवारण्ये द्रुमैर्मारुतकम्पितैः}


\twolineshloka
{कृतपद्माञ्जलिपुटा मत्तषट्पदसेविताः}
{प्रियतीर्थवना मार्गे पद्मिनीः समतिक्रमन्}


\twolineshloka
{सज्जमानमनोदृष्टिः फुल्लेषु गिरिसानुषु}
{द्रौपदीवाक्यपाथेयो भीमो भीमपराक्रमः}


\twolineshloka
{परिवृत्तेऽहनि ततः प्रकीर्णहरिणे वने}
{काञ्चनैर्विलैः पद्मैर्ददर्श विपुलां नदीम्}


\twolineshloka
{हंसकारण्डवयुतां चक्रवाकोपशोभिताम्}
{रचितामिव तस्याद्रेर्भालां विमलपङ्कजाम्}


\twolineshloka
{तस्यां नद्यां महासत्वः सौगन्धिकवनं महत्}
{अपश्यत्प्रीतिजननं बालार्कसदृशद्युति}


\twolineshloka
{तद्दृष्ट्वा लब्धकामः स मनसा पाण्डुनन्दनः}
{वनवासपरिक्लिष्टां जगाम मनसा प्रियाम्}


\chapter{अध्यायः १५४}
\twolineshloka
{वैशंपायन उवाच}
{}


\twolineshloka
{स गत्वानलिनीं रम्यां राक्षसैरभिरक्षिताम्}
{कैलासशिरे रम्ये ददर्श शुभकानने}


\twolineshloka
{कुबेरभुवनाभ्याशे जातां पर्वतनिर्झरैः}
{सुरम्यां विपुलच्छायां नानाद्रुमलताकुलाम्}


\twolineshloka
{हरिताम्बुजसंछन्नां दिव्यां कनकपुष्कराम्}
{नानापक्षिजनाकीर्णां सूपतीर्थामकर्दमाम्}


\twolineshloka
{अतीव रम्यां सुजलां जातां पर्वतसानुषु}
{विचित्रभूतां लोकस्य शुभामद्भुतदर्शनाम्}


\twolineshloka
{तत्रामृतरसं शीतं लघु कुन्तीसुतः शुभम्}
{ददर्शविमलं तोयं पिबंश्च बहु पाण्डवः}


\twolineshloka
{तां तु कपुष्करिणीं रम्यां दिव्यसौगन्धिकावृताम्}
{जातरूपमयैः पद्मैश्छन्नां परमगन्धिभिः}


\twolineshloka
{वैडूर्यवरनालैश्च बहुचित्रैर्मनोरमैः}
{हंसकारण्डवोद्धूतैः सृजद्भिरमलं रजः}


\twolineshloka
{आक्रीडं राजराजस् कुबेरस्य महात्मनः}
{गन्धर्वैरप्सरोभिश्च देवैश्च परमार्चिताम्}


\twolineshloka
{सेवितामृषिभिर्दिव्यैर्यक्षैः किंपुरुषैस्तथा}
{राक्षसैः किंनरैश्चापि गुप्तां वैश्रवणेन च}


\twolineshloka
{तां च दृष्ट्वैव कौन्तेयो भीमसेनो महाहलः}
{बभूव परमप्रीतो दिव्यंप्रेक्ष्य सरो महत्}


\twolineshloka
{तच्च क्रोवशा नाम राक्षसा राजशासनात्}
{रक्षन्ति शतसाहस्राश्चित्रायुधपरिच्छदाः}


\twolineshloka
{ते तु दृष्ट्वैव कौन्तेयमजिनैः परिवारितम्}
{रुक्माङ्गदधरं वीरं भीमं भीमपराक्रमम्}


\twolineshloka
{सायुधं बद्धनिस्त्रिंशमशङ्कितमरिंदमम्}
{पुष्करप्सुमुपायान्तमन्योन्यमभिचुक्रुशुः}


\twolineshloka
{अयं पुरुषशार्दूलः सायुधोऽजिनसंवृतः}
{यच्चिकीर्षुरिह प्राप्तस्तत्संप्रष्टुमिहार्हथ}


\twolineshloka
{ततः सर्वे महाबाहुं समासाद्यवृकोदरम्}
{तेजोयुक्तमपृच्छन्त कस्त्वमाख्यातुमर्हसि}


\twolineshloka
{मुनिवेषधरश्चैव सायुधश्चैव लक्ष्यसे}
{यदर्थमाभिसंप्राप्तस्तदाचक्ष्यमहामते}


\chapter{अध्यायः १५५}
\twolineshloka
{भीम उवाच}
{}


\twolineshloka
{पाण्डवो भीमसेनोऽहं धर्मराजादनन्तरः}
{विशालां बदरीं प्राप्तो भ्रातृभिः सह राक्षसाः}


\twolineshloka
{अपश्यत्तत्रपाञ्चाली सौगन्धिकमनुत्तमम्}
{अनिलोढमितो नूनं सा बहूनि परीप्सति}


\threelineshloka
{तस्या मामनवद्याङ्ग्या धर्मपत्न्याः प्रिये स्थितम्}
{पुष्पाहारमिह प्राप्तं निबोधत निशाचराः ॥राक्षसा ऊचुः}
{}


\twolineshloka
{आक्रीडोऽयं कुबेरस्य दयितः पुरुषर्षभ}
{नेह शक्यं मनुष्येण विहर्तुं मर्त्यधर्मणा}


\twolineshloka
{देवर्षयस्तथा यक्षा देवाश्चात्र वृकोदर}
{आमन्त्र्य यक्षप्रवरं पिबन्ति च हरन्ति च}


\twolineshloka
{गन्धर्वाप्सरसश्चैव विहरन्त्यत्र पाण्डव}
{`यक्षाधिपस्यानुमते कुबेरस् महात्मनः'}


\twolineshloka
{अन्यायेनेह यः कश्चिदवमन्य धनेश्वरम्}
{विहर्तुमिच्छेद्दुर्वृत्तः स विनश्येन्न संशयः}


\twolineshloka
{तमनादृत्य पद्मानि जिहीर्षसि बलादिह}
{धर्मराजस्य चात्मानं ब्रवीषि भ्रातरं कथम्}


\threelineshloka
{[आमन्त्र्य यक्षराजं वै ततः पिव हरस्व च}
{नातोऽन्यथा त्वया शक्यं किंचित्पुष्करमीक्षितुं ॥भीमसेन उवाच}
{}


\twolineshloka
{राक्षसास्तं न पश्यामि धनेश्वरमिहान्तिके}
{दृष्ट्वाऽपिच महाराजं नाहं याचितुमुत्सहे}


\twolineshloka
{न हि याचन्ति राजान एष धर्मः सनातनः}
{न चाहं हातुमिच्छामि क्षात्रधर्मं कथंचन}


\twolineshloka
{इयं च नलिनी रम्या जाता पर्वतनिर्झरे}
{नेयं भवनमासाद्यकुबेरस्य महात्मनः}


\twolineshloka
{तुल्या हि सर्वभूतानामियं वैश्रवणस्य च}
{एवं गतेषु द्रव्येषु कः कं याचितुमर्हति}


\twolineshloka
{इत्युक्त्वा राक्षसान्सर्वान्भीमसेनो व्यगाहत}
{तां तु पुष्करिणीं वीरः प्रभिन्न इव कुञ्जरः}


\twolineshloka
{ततः स राक्षसैर्वाचा प्रतिषिद्धः प्रतापवान्}
{मा मैवमिति सक्रोधैर्भर्त्सयद्भिः समन्ततः}


\twolineshloka
{कदर्थीकृत्यतु स तान्राक्षसान्भीमविक्रमः}
{व्यगाहत महातेजास्ते तं सर्वे न्यवारयन्}


\twolineshloka
{गृह्णीत बध्नीत विकर्ततेमंपचाम खादाम च भीमसेनम्}
{क्रुद्धा ब्रुवन्तोऽभिययुर्द्रतं तेशस्त्राणि चोद्यम्य विवृत्तनेत्राः}


\twolineshloka
{प्रगृह्यतानभ्यपतत्तरस्वीततोऽब्रवीत्तिष्ठत तिष्ठतेति ॥ते तं तदा तोमरपट्टसाद्यै-र्व्याविद्धशस्त्रैः सहसा निपेतुः}
{}


\twolineshloka
{जिघांसवः क्रोधवशाः सुभीमाभीमं समन्तात्परिवव्रुरुग्राः}
{जिघांसवः क्रोधवशाः सुभीमाभीमं समन्तात्परिवब्रुरुग्राः}


\twolineshloka
{वातेन कुन्त्यां बलवान्सुजातःशूरस्तरस्वी द्विषतां निहन्ता}
{सत्ये च धर्मे च रतः सदैवपराक्रमे शत्रुभिरप्रधृष्यः}


\twolineshloka
{तेषां स मार्गान्विविधान्महात्मानिहत्य शस्त्राणि च शास्त्रवाणाम्}
{यथा प्रवीरान्निजघान भीमःपरश्शतान्पुष्करिणीसमीपे}


\twolineshloka
{ते तस्य वीर्यं च बलं च दृष्ट्वाविद्याबलंबाहुबलं तथैव}
{अशक्नुवन्तः सहितं समन्ता-द्द्रुतं प्रवीरा सहसा निवृत्ताः}


\twolineshloka
{विदीर्यमाणास्तत एव तूर्ण-माकाशमास्थाय विमूढसंज्ञाः}
{कैलासशृङ्गाण्यभिदुद्रुवुस्तेभीमार्दिताः क्रोधवशाः प्रभग्नाः}


\twolineshloka
{स शक्रवद्दानवदैत्यसङ्घान्विक्रम्य जित्वा च रणेऽरिसङ्घान्}
{विगाह्यतां पुष्करीणीं जितारिःकामं स जग्राह ततोऽम्बुजानि}


\twolineshloka
{ततः स पीत्वाऽमृतकल्पमम्भोभूयो बभूवोत्तमवीर्यतेजाः}
{उत्पाट्य जग्राह ततोऽम्बुजानिसौगन्धिकान्युत्तमगन्धवन्ति}


\twolineshloka
{ततस्तु ते क्रोधवशाः समेत्यधनेश्वरं भीमबलप्रणुन्नाः}
{भीमस्य वीर्यं च बलं च संख्येयथावदाचख्युरतीव दीनाः}


\twolineshloka
{तेषां वचस्तत्तु निशाम्य देवःप्रहस्य रक्षांसि ततोऽभ्युवाच}
{गृह्णातु भीमो जलजानि कामंकृष्णानिमित्तं विदितं ममैतत्}


\twolineshloka
{ततोऽभ्यनुज्ञाय धनेश्वरं तेजग्मुः कुरूणां प्रवरं विरोषाः}
{भीमं च तस्यां ददृशुर्नलिन्यांयथोपजोषं विहरन्तमेकम्}


\chapter{अध्यायः १५६}
\twolineshloka
{वैशंपायन उवाच}
{}


\twolineshloka
{ततस्तानि महार्हाणि दिव्यानि भरतर्षभ}
{बहूनि बहुरूपाणि विरजांसि समाददे}


\twolineshloka
{ततो वायुर्महाञ्शीघ्रो नीचैः शर्करकर्षणः}
{प्रादुरासीद्वरस्पर्शः संग्राममभिचोदयन्}


\twolineshloka
{पपात महती चोल्का सनिर्घाता महाभया}
{निष्प्रभश्चाभवत्सूर्यंश्छन्नरश्मिस्तमोवृतः}


\twolineshloka
{निर्घातश्चाभवद्भीमो भीमे विक्रममास्थिते}
{चचाल पृथिवी चापि पांसुवर्षं पपात च}


\threelineshloka
{सलोहिता दिशश्चासन्खरवाचो मृगद्विजाः}
{तमोवृतमभूत्सर्वं न प्राज्ञायत किंचन}
{}


\twolineshloka
{[अन्ये च बहवो भीमा उत्पातास्तत्र जज्ञिरे ॥]तदद्भुतमभिप्रेक्ष्य धर्मपुत्रो युधिष्ठिरः}
{उवाच वदतां श्रेष्ठः कोऽस्मानभिभविष्यति}


\twolineshloka
{सज्जीभवत भद्रं वः पाण्डवा युद्धदुर्मदाः}
{यथा रूपाणि पश्यामि सुव्यक्तो नः पराक्रमः}


\twolineshloka
{एवमुक्त्वा ततो राजा वीक्षांचक्रे समन्ततः}
{अपश्यमानो भीमं तु धर्मपुत्रो युधिष्ठिरः}


\twolineshloka
{ततः कृष्णां यमौ चापि समीपस्थानरिंदमः}
{पप्रच्छ भ्रातरं भीमं भीमकर्माणमाहवे}


\twolineshloka
{कच्चिन्न भीमः पाञ्चालि किंच कृत्यं चिकीर्षति}
{कृतवानपि वा वीर साहसं साहसप्रियः}


\twolineshloka
{इमे ह्यकस्मादुत्पाता महासमरशंसिनः}
{दर्शयन्तो भयं तीव्रं प्रादुर्भूताः समन्ततः}


\twolineshloka
{तं तथावादिनं कृष्णा प्रत्युवाच मनस्विनी}
{प्रिया प्रियं चिकीर्षन्ती महिषी चारुहासिनी}


\twolineshloka
{यत्तत्सौगन्धिकं राजन्नाहृतं मातरिश्वना}
{तन्मया भीमसेनस्य प्रीतयाऽद्योपपादितम्}


\twolineshloka
{अपि चोक्तो मया वीरो यदिपश्येर्बहून्यपि}
{तानि सर्वाणअयुपादाय शीघ्रमागम्यतामिति}


\twolineshloka
{स तु नूनं महाबाहुः प्रियार्थं मम पाण्डवः}
{प्रागुदीचीं दिशं राजंस्तान्याहर्तुमितो गतः}


\twolineshloka
{उक्तस्त्वेवं तया राजा यमाविदमथाब्रवीत्}
{गच्छाम सहितास्तूर्णं येन यातो वृकोदरः}


\twolineshloka
{वहन्तु राक्षसा विप्रान्यथाश्रान्तान्यथाकृशान्}
{त्वमप्यमरसंकाश वह कृष्णां घटोत्कच}


\twolineshloka
{व्यक्तं दूरमितो भीमः प्रविष्ट इतिमे मतिः}
{चिरं चतस्य कालोऽयं स च वायुसमो जवे}


\twolineshloka
{तरस्वी वैनतेयस्य सदृशो भुवि लङ्घने}
{उत्पतेदपिचाकाशं निपतेच्चयथेच्छकम्}


\twolineshloka
{तमन्वियाम भवतां प्रभावाद्रजनीचराः}
{पुरा स नापराध्नोति सिद्धानां ब्रह्मवादिनाम्}


\twolineshloka
{तथेत्युक्त्वा तु ते सर्वेहैडिम्बप्रमुखास्तदा}
{उद्देशज्ञाः कुबेरस्य नलिन्या भरतर्षभ}


\twolineshloka
{आदाय पाण्डवांश्चैव तांश्च विप्राननेकशः}
{लोमशेनैव सहिताः प्रययुः प्रीतमानसाः}


\twolineshloka
{ते सर्वे त्वरिता गत्वा ददृशुस्तत्र कानने}
{पद्मसौगन्धिकवतींनलिनीं सुमनोरमाम्}


\twolineshloka
{तं च भीमं महात्मानं तस्यास्तीरे व्यवस्थितम्}
{ददृशुर्निहतांश्चैव यक्षांश्च विपुलेक्षणान्}


\twolineshloka
{भिन्नकायाक्षिबाहूरुन्संचूर्णितशिरोधरान्}
{तं च भीमं महात्मानं तस्यास्तीरे व्यवस्थितम्}


\twolineshloka
{सक्रोधं स्तव्धनयनं संदष्टदशनच्छदम्}
{उद्यम्य च गदां दोर्भ्यां नदीतीरे व्यवस्थितम्}


\twolineshloka
{प्रजासंक्षेपसमये दण्डहस्तमिवान्तकम्}
{तं दृष्ट्वा धर्मराजस्तु परिष्वज्याथ भारत}


\threelineshloka
{उवाच श्लक्ष्णया वाचा कौन्तेय किमिदं कृतम्}
{साहसं वत भद्रं ते देवानामपि चाप्रियम्}
{पुनरेवं न कर्तव्यं मम चेदिच्छसि प्रियम्}


\twolineshloka
{अनुशिष्य तु कौन्तेयं पद्मानि परिगृह्य च}
{तस्यामेव नलिन्यां तु विजह्ररमरोपमाः}


\twolineshloka
{एतस्मिन्नैव काले तु प्रगृहीतशिलायुधाः}
{प्रादुरासन्महाकायास्तस्योद्यानस्य रक्षिणः}


\twolineshloka
{ते दृष्ट्वाधर्मराजानं महर्षिं चापि लोमशम्}
{नकुलं सहदेवं च तथाऽन्यान्ब्राह्मणर्षभान्}


\twolineshloka
{विनयेन नताः सर्वेप्रणिपत्य च भारत}
{सान्त्विता धर्मराजेन प्रसेदुः क्षणदाचराः}


\threelineshloka
{विदिताश्चकुबेरस्य तत्रते कुरुपुङ्गवाः}
{ऊषुर्नातिचिरं कालं रममाणाः कुरूद्वहाः}
{प्रतीक्षमाणआ बीभत्सुं गन्धमादनसानुषु}


\chapter{अध्यायः १५७}
\twolineshloka
{वैशंपायन उवाच}
{}


\twolineshloka
{तस्मिन्निवसमानोऽथ धर्मराजो युधिष्ठिरः}
{आमन्त्र्य सहितान्भ्रातृनित्युवाच सहद्विजान्}


\twolineshloka
{दृष्टानि तीर्थान्यस्माभिः पुण्यानि च शिवानि च}
{मनसो ह्लादनीयानि वनानि च पृथक्पृथक्}


\twolineshloka
{देवैः पूर्वं विचीर्णानि मुनिभिश्च महात्मभिः}
{यथाक्रममशेषेण द्विजैः संपूजितानि च}


\twolineshloka
{ऋषीणां पूर्वचरितं तपोधर्मविचेष्टितम्}
{राजर्षीणां च चरितं कथाश्च विविधाः शुभाः}


\twolineshloka
{शृण्वानास्तत्रतत्र स्म आश्रमेषु शिवेषु च}
{अभिषेकं द्विजैः सार्धं कृतवन्तो विशेषतः}


\twolineshloka
{अर्चिताः सततं देवाः पुष्पैरद्भिः सदा च वः}
{यथालब्धैर्मूलफलैः पितरश्चापि तर्पिताः}


\twolineshloka
{पर्वतेषु च रम्येषु सर्वेषु च सरस्सु च}
{उदधौ च महापुण्ये सूपस्पृष्टं महात्मभिः}


\twolineshloka
{इला सरस्वती सिन्धुर्यमुना नर्मदा तथा}
{नानातीर्थेषु रम्येषु सूपस्पृष्टं सह द्विजैः}


\twolineshloka
{गङ्गाद्वारमतिक्रम्य बहवः पर्वताः शुभाः}
{हिमवान्पर्वतश्चैव नानाद्विजगणायुतः}


\twolineshloka
{विशाला बदरी दृष्टा नरनारायणाश्रमः}
{दिव्यपुष्करिणी दृष्टा सिद्धदेवर्षिपूजिता}


\twolineshloka
{यथाक्रमविशेषेण सर्वाण्यायतनानि च}
{दर्शितानि द्विजेनद्रेण लोमशेन महात्मना}


\threelineshloka
{इमं वैश्रवणावासं दुर्गमं गन्धमादनम्}
{कथं भीम गमिष्यामो मतिरत्र विधीयताम् ॥वैशंपायन उवाच}
{}


\twolineshloka
{एवं ब्रुवति राजेन्द्रे वागुवाचाशरीरिणी}
{न शक्यो दुर्गमो गन्तुं पर्वतो गन्धमादनः}


\twolineshloka
{अननैव पथा राजन्प्रतिगच्छ यथागतम्}
{नरनारायणस्थानं बदरीत्यभिविश्रुतम्}


\twolineshloka
{तस्माद्यास्यसि कौन्तेय सिद्धचारणसेवितम्}
{बहुपुष्पफलं रम्यमाश्रमं वृषपर्वणः}


\twolineshloka
{अतिक्रम्य च तं पार्थ त्वार्ष्टिषेणाश्रमे वसेः}
{ततो द्रक्ष्यसि कौन्तेय निवेशं धनदस्य च}


\twolineshloka
{एतस्मिन्नन्तरे वायुर्दिव्यगन्धवहः शुभः}
{भनःप्रह्लादनः शीतः पुष्पवर्षं ववर्ष वै}


\twolineshloka
{तच्छ्रुत्वा दिव्यमाकाशाद्विस्मयः समपद्यत}
{ऋषीणां ब्राह्मणानां च पार्थिवानां विशेषतः}


\twolineshloka
{श्रुत्वा तन्महदाश्चर्यं द्विजो धौम्यस्त्वभाषत}
{न शक्यमुत्तरं गन्तुं प्रतिगच्छाम पाण्डव}


\twolineshloka
{ततो युधिष्ठिरो राजा विस्मयोत्फुल्ललोचनः}
{[प्रत्यागम्य पुनस्तं तु नरनारायणाश्रमम् ॥]}


\twolineshloka
{भीमसेनादिभिः सर्वैर्भ्रातृभिः परिवारितः}
{पाञ्चाल्या ब्राह्मणैश्चैव न्यवसत्सुसुखं तदा}


\chapter{अध्यायः १५८}
\twolineshloka
{वैशंपायन उवाच}
{}


\twolineshloka
{ततस्तान्परिविश्वस्तान्वसतस्तत्र पाण्डवान्}
{[पर्वतेन्द्रे द्विजैः सार्धं पार्तागमनकाङ्क्षया ॥]}


\fourlineindentedshloka
{गतेषु तेषु रक्षःसु भीमसेनात्मजेऽपि च}
{रहितान्मीमसेनेन कदाचित्तान्यदृच्छया}
{जहार धर्मराजं वै यमौ कृष्णां च राक्षसः ॥`जनमेजय उवाच}
{}


\fourlineindentedshloka
{ब्रह्मन्कथं धर्मराजं यमौ कृष्णां च राक्षसः}
{जहार चित्रं भीमश्च गतो राक्षसकण्टकः}
{वक्तुमर्हसि विप्राग्र्य व्यक्तमेतन्ममाऽनघ ॥वैशंपायन उवाच}
{'}


\threelineshloka
{ब्राह्मणो मन्त्रकुशलः सर्वास्त्रेष्वस्त्रवित्तमः}
{इति ब्रुवन्पाण्डवेयान्पर्युपास्ते स्म नित्यदा}
{परीप्समानः पार्थानां कलापांश्च धनूंषि च}


\twolineshloka
{अन्तरं संपरिप्रेप्सुर्द्रौपद्या हरणं प्रति}
{दुष्टात्मा पापबुद्धिः स नाम्ना ख्यातो जटासुरः}


\twolineshloka
{[पोषणं तस् राजेनद्र चक्रे पाण्डवनन्दनः}
{बुबुधे न च तं पापं भस्मच्छन्नमिवानलम्}


\twolineshloka
{स भीमसेने निष्क्रान्ते मृगयार्थमरिंदमे}
{[घटोत्कचं सानुचरं दृष्ट्वा विप्रद्रुतं दिशः}


\twolineshloka
{लोमशप्रभृतींस्तांस्तु महर्षीश्च समाहितान्}
{स्नातुं विनिर्गतान्दृष्ट्वा पुष्पार्थं च तपोधनान्]}


\threelineshloka
{रूपमन्यत्समास्थाय विकृतं भैरवं महत्}
{गृहीत्वा सर्वशस्त्राणि द्रौपदीं परिगृह्य च}
{प्रातिष्ठत सुदुष्टात्मा त्रीन्गृहीत्वा च पाण्डवान्}


\twolineshloka
{[विक्रम्य कौशिकं खङ्गं मोक्षयित्वा ग्रहं रिपोः}
{]आक्रन्दद्भीमसेन वै येन यातो महाबलः}


\twolineshloka
{तमब्रवीद्धर्मराजो ह्रियमाणो युधिष्ठिरः}
{धर्मस्ते हीयते मूढ न चैनं समवेक्षसे}


\twolineshloka
{येऽन्ये क्वचिन्मनुष्येषु तिर्यग्योनिगताश्च ये}
{धर्मं ते समवेक्षन्ते रक्षांसि च विशेषतः}


\twolineshloka
{धर्मस्यराक्षसा मूलं धर्मं ते विदुरुत्तमम्}
{एतत्परीक्ष्यसर्वं त्वं समीपे स्थातुमर्हसि}


\twolineshloka
{देवाश्च ऋषयः सिद्धाः पितरश्चापि राक्षस}
{गन्धर्वोरगरक्षांसि वयांसि पशवस्तथा}


\twolineshloka
{तिर्यग्योनिगताश्चैव अपि कीटपिपीलिकाः}
{मनुष्यानुपजीवन्ति ततस्त्वमपि जीवसि}


\threelineshloka
{समृद्ध्या यस्य लोकस्य लोको युष्माकमृध्यति}
{इमं च लोकं शोचन्तमनुशोचन्ति देवताः}
{पूज्यमानाश्च वर्धन्ते हव्यकव्यैर्यथाविधि}


\twolineshloka
{वयं राष्ट्रस्य गोप्तारो रक्षितारश्च राक्षस}
{राष्ट्रस्यारक्ष्यमाणस्य कुतो भूतिः कुतः सुखम्}


\twolineshloka
{न च राजाऽवमन्तव्यो रक्षसा जात्वनागसि}
{अनुरप्यपचारश्च नास्त्यस्माकं नराशन}


\twolineshloka
{विघसाशान्यथाशक्त्या कुर्महे देवतादिषु}
{गुरूंश्च ब्राह्मणांश्चैव प्रमाणप्रवणाः सदा}


\twolineshloka
{द्रोग्धव्यं न च मित्रेषु न विश्वस्तेषु कर्हिचित्}
{येषां चान्नानि भुञ्जीत यत्रच स्यात्प्रतिश्रयः}


\twolineshloka
{स त्वं प्रतिश्रयेऽस्माकं पूज्यमानः सुखोषितः}
{भुक्त्वा चान्नानि दुष्प्रज्ञ कथमस्माञ्जिहीर्षसि}


\twolineshloka
{एवमेव वृथाचारो वृथा वृद्धो वृथामतिः}
{वृथामरणमर्हस्त्वं वृथाऽद्य नमविष्यसि}


\twolineshloka
{अथ चेद्दुष्टबुद्धिस्त्वं सर्वैर्धर्मैर्विवर्जितः}
{प्रदाय शस्त्राण्यस्माकं युद्धेन द्रौपदीं हर}


\twolineshloka
{अथ चेत्त्वमविज्ञाय इदं कर्म करिष्यसि}
{अधर्मं चाप्यकीर्तिं च लोके प्राप्स्यसि केवलं}


\twolineshloka
{एतामद्य परामृश्य स्त्रियं राक्षस मानुषीम्}
{विषमेतत्समालोड्य कुम्भेन प्राशितं त्वया}


\twolineshloka
{ततो युधिष्ठिरस्तस्य भारिकः समपद्यत}
{स तु भाराभिभूतात्मा न तथा शीघ्रगोऽभवत्}


\twolineshloka
{अथाब्रवीद्द्रौपदीं च नकुलं च युधिष्ठिरः}
{मा भैष्टं राक्षसान्मूढाद्गतिरस्य मया हृता}


\twolineshloka
{नातिदूरे महबाहुर्भविता पवनात्मजः}
{अस्मिन्मुहूर्ते संप्राप्ते नभविष्यति राक्षसः}


\twolineshloka
{सहदेवस्तु तं दृष्ट्वा राक्षसं मूढचेतसम्}
{उवाच वचनं राजन्कुन्तीपुत्रं युधिष्ठिरम्}


\twolineshloka
{राजन्किं नाम सत्कृत्यं क्षत्रियस्यास्त्यतोऽधिकम्}
{यद्युद्धेऽभिमुखः प्राणांस्त्यजेच्छत्रुं जयेत वा}


\twolineshloka
{एष वास्मान्वयं वैनं युध्यमानाः परंतप}
{सूदयेम महाबाहो देशः कालो ह्ययं नृप}


\twolineshloka
{क्षत्रधर्मस्य संप्राप्तः कालः सत्यपराक्रम}
{जयन्तो हन्यमाना वा प्राप्तुमर्हाम सद्गतिम्}


\twolineshloka
{राक्षसे जीवमानेऽद्य रविरस्तमियाद्यदि}
{नाहं ब्रूयां पुनर्जातु क्षत्रियोस्मीति भारत}


\twolineshloka
{भोभो राक्षस तिष्ठस्व सहदेवोस्मि पाण्डवः}
{हत्वा रवा मां नयस्वैनां हतो वा स्वप्स्यसीह वै}


\twolineshloka
{तदा ब्रुवति माद्रेये भीमसेनो यदृच्छया}
{प्रादृश्यत महाबाहुः सवज्र इव वासवः}


\twolineshloka
{सोऽपश्यद्धातरौ तत्र द्रौपदीं च यशस्विनीम्}
{क्षितिस्थं सहदेवं च क्षिपन्तं राक्षसं तदा}


\twolineshloka
{मार्गाच्च राक्षसं मूढं कालोपहतचेतसम्}
{भ्रमन्तं तत्रतत्रैव दैवेन परिमोहितम्}


\twolineshloka
{हृतान्संदृश्य तान्भ्रातृन्द्रौपदीं च महाबलः}
{क्रोधमाहारयद्भीमो राक्षसंचेदमब्रवीत्}


\twolineshloka
{विज्ञातोऽसि मया पूर्वं चेष्टञ्शस्त्रपरीक्षणे}
{आस्था तु त्वयि मे नास्ति यतोसि न हतस्तदा}


\threelineshloka
{ब्रह्मरूपप्रतिच्छन्नो न नो वदसि चाप्रियम्}
{प्रियेषु रममाणं त्वां न चैवाप्रियकारिणम्}
{ब्रह्मरूपेण विहितं नैव हन्यामनागसम्}


\twolineshloka
{राक्षसं जानमानोऽपि यो हन्यान्नरकं व्रजेत्}
{अपक्वस्य च कालेन वधस्तव न विद्यते}


\threelineshloka
{नूनमद्यासि संपक्वो यथा ते मतिरीदृशी}
{दत्ता कृष्णापहरणे कालेनाद्भुतकर्मणा}
{`सोपि कालं समासाद्य तथाऽद्य नभविष्यसि'}


\twolineshloka
{बडिशोऽयंत्वया ग्रस्तः कालसूत्रेण लम्बितः}
{मत्स्योऽम्भसीव स्यूतास्यः कथमद्य गमिष्यसि}


\twolineshloka
{यं चासि प्रस्थितो देशं मनः पूर्वं गतं च ते}
{न तं गन्तासि गन्तासि मार्गं बकहिडिम्बयोः}


\twolineshloka
{एवमुक्तस्तु भीमेन राक्षसः कालचोदितः}
{भीत उत्सृज्य तान्सर्वान्युद्धाय समुपस्थितः}


\twolineshloka
{अब्रवीच्च पुनर्भीमं रोषात्प्रस्फुरिताधरः}
{न मे मूढा दिशः पाप त्वदर्थं मे विलम्बनम्}


\twolineshloka
{श्रुता मे राक्षसा ये ये त्वया विनिहता रणे}
{तेषामद्यकरिष्यामि तवास्रेणोदकक्रियाम्}


\twolineshloka
{`एवमुक्त्वातदा भीमं राक्षसो योद्धुमाययौ}
{करालवदनः क्रोधात्कालसर्प इव श्वसन्'}


\twolineshloka
{एवमुक्तस्ततो भीमः सृक्विणी परिलेलिहन्}
{स्मयमान इव क्रोधात्साक्षात्कालान्तकोपमः}


\twolineshloka
{`ब्रुवन्वै तिष्ठतिष्ठेति क्रोधसंरक्तलोचनः'}
{बाहुसंरम्भमेवेच्छन्नभिदुद्राव राक्षसम्}


\twolineshloka
{मुहुर्मुहुर्व्याददानः सृक्विणी परिसंलिहन्}
{]अभिदुद्राव संरब्धो बलो वज्रधरं यथा}


\twolineshloka
{भीमसेनोऽप्यवष्टब्धो नियुद्धायाभवत्स्थितः}
{राक्षसोऽपि च विस्रब्धो बाहुयुद्धमकाङ्क्षत'}


\twolineshloka
{वर्तमाने तयो राजन्बाहुयुद्धे सुदारुणे}
{माद्रीपुत्रावतिक्रुद्धावुभावप्यभ्यधावताम्}


\twolineshloka
{न्यवारयत्तौ प्रहसन्कुन्तीपुत्रो वृकोदरः}
{शक्तोऽहं राक्षसस्येति प्रेक्षध्वमिति चाब्रवीत्}


\twolineshloka
{आत्मना भ्रातृभिश्चैव धर्मेण सुकृतेन च}
{इष्टेन च शपे राजन्सूदयिष्यामि राक्षसम्}


\twolineshloka
{इत्येवमुक्त्वा तौ वीरौ स्पर्धमानौ परस्परम्}
{बाहुभिः समसज्जेतामुभौ रक्षोवृकोदरौ}


\twolineshloka
{तयोरासीत्संप्रहारः क्रुद्धयोर्भीमरक्षसोः}
{अमृष्यमाणयोः सङ्ख्ये शक्रशम्बरयोरिव}


\threelineshloka
{तौ वीरौ समभिक्रुद्धावन्योन्यं पर्यधावताम्}
{आरुज्यारुज्यतौ वृक्षानन्योन्यमभिजघ्नतुः}
{जीमूताविव धर्मान्ते विनदन्तौ महाबलौ}


\twolineshloka
{बभञ्जतुर्महावृक्षानूरुभिर्बलिनां वरौ}
{अन्योन्येनाभिसंरब्धौ परस्परवधैषिणौ}


\twolineshloka
{तद्वृक्षयुद्धमभवनमहीरुहविनाशनम्}
{वालिसुग्रीवयोर्भ्रात्रोः पुरेव कपिसिंहयोः}


\twolineshloka
{आविध्याविध्य तौ वृक्षान्मुहूर्तमितरेतरम्}
{ताडयामासतुरुभौ विनदन्तौ मुहुर्मुहुः}


\twolineshloka
{तस्मिन्देशे यदा वृक्षाः सर्व एव निपातिताः}
{पुगीकृताश्च शतशः परस्परवधेप्सया}


\twolineshloka
{ततः शिलाः समादाय मुहूर्तमिव भारत}
{महाभ्रैरिव शैलेन्द्रौ युयुधाते महाबलौ}


\twolineshloka
{शिलाभिरुग्ररूपाभिर्बृहतीभिः परस्परम्}
{वज्रैरिव महावेगैराजघ्नतुरमर्षणौ}


\twolineshloka
{अभिद्रुत्य च भूयस्तावन्योन्यबलदर्पितौ}
{भुजाभ्यां परिगृह्याथ चकर्षाते गजाविव}


\twolineshloka
{मुष्टिभिश् महाघोरैरन्योन्यमभिपेततुः}
{ततः कटकटाशब्दो बभूव शुमहात्मनोः}


\twolineshloka
{ततः संहृत्यमुष्टिं तु पञ्चशीर्षमिवोरगम्}
{वेगेनाभ्यहनद्भीमो राक्षसस्य शिरोधराम्}


\twolineshloka
{ततः श्रान्तं तु तद्रक्षो भीमसेनभुजाहतम्}
{सुपरिभ्रान्तमालक्ष्य भीमसेनोऽभ्यवर्तत}


\twolineshloka
{तत एनं महाबाहुर्बाहुभ्याममरोपमः}
{समुत्क्षिप्य बलाद्भीमो निष्पिपेष महीतले}


\threelineshloka
{`ततः संपीड्य बलवद्भुजाभ्यां क्रोधमूर्च्छितः}
{'तस्य गात्राणि सर्वाणि चूर्णयामास पाण्डवः}
{अरत्निना चाभिहत्य शिरः कायादपाहरत्}


\threelineshloka
{संदष्टौष्ठं विवृत्ताक्षं फलं वृक्षादिव च्युतम्}
{जटासुरस्य तु शिरो भीमसेनबलाद्धृतम्}
{पपात रुधिरादिग्धं संदष्टदशनच्छदम्}


\twolineshloka
{तं निहत्यमहेष्वासो युधिष्ठिरमुपागमत्}
{स्तूयमानो द्विजाग्र्यैस्तु मरुद्भिरिव वासवः}


\chapter{अध्यायः १५९}
\twolineshloka
{वैशंपायन उवाच}
{}


\twolineshloka
{निहते राक्षसे तस्मिन्पुनर्नारायणाश्रमम्}
{अभ्येत्य राजा कौन्तेयो निवासमकरोत्प्रभुः}


\twolineshloka
{स समानीय तान्सर्वान्भ्रातॄनित्यब्रवीद्वचः}
{द्रौपद्या सहितः काले संस्मरन्भ्रातरं जयम्}


\twolineshloka
{समाश्चतस्रोऽभिगताः शिवेन चरतां वने}
{कृतोद्देशः स बीभत्सुऋ पञ्चमीमभितः समाम्}


\twolineshloka
{प्राप्य पर्वतराजानं श्वेतं शिखरिणांवरम्}
{[पुष्पितैर्द्रुमषण्डैश्च मत्तकोकिलाषट्पदैः}


\twolineshloka
{मयूरैश्चातकैश्चापि नित्योत्सवविभूषितम्}
{व्याघ्रैर्वरहैर्महिषैर्गवयैर्हरिणैस्तथा}


\twolineshloka
{श्वापदैर्व्यालरूपैश्च रुरुभिश्च निषेवितम्}
{फुल्लैः सहस्रपत्रैश्च शतपत्रैस्तथोत्पलैः}


\threelineshloka
{प्रफुल्लैः कमलैश्चैव तथा नीलोत्पलैरपि}
{महापुणअयं पवित्रं च सुरासुरनिषेवितम्}
{]तत्रापि च कृतोद्देशः समागमदिदृक्षुभिः}


\twolineshloka
{कृतश्च समयस्तेन पार्थेनामिततेजसा}
{पञ्चवर्षाणि वत्स्याभि विद्यार्थीति पुरा मयि}


\twolineshloka
{अत्रगाण्डीवधन्वानमवाप्तास्त्रमरिंदमम्}
{देवलोकादिमं लोकं द्रक्ष्यामः पुनरागतम्}


\twolineshloka
{इत्युक्त्वा ब्राह्मणान्सर्वानामन्त्रयत पाण्डवः}
{कारणं चैव तत्तेवामाचचक्षे तपस्विनाम्}


\twolineshloka
{तमुग्रतपसः प्रीताः कुत्वापार्थं प्रदक्षिणम्}
{ब्राह्मणास्तेऽन्वमोदन्त शिवेन कुशलेन च}


\twolineshloka
{सुखोदर्कमिमं क्लेशमचिराद्भरतर्षभ}
{क्षत्रधर्मेण धर्मज्ञ तीर्त्वा गां पालयिष्यसि}


\twolineshloka
{तत्तु राजा वचस्तेषां प्रतिगृह्य तपस्विनाम्}
{प्रतस्थे सह विप्रैस्तैर्भ्रातृभिश्च परंतपः}


\twolineshloka
{`द्रौपद्या सहितः श्रीमान्हैडिम्बेयादिभिस्तदा'}
{राक्षसैरनुयातो वै लोमशेनाभिरक्षितः}


\twolineshloka
{क्वचित्पद्भ्यां ततोऽगच्छद्राक्षसैरुदह्यते क्वचित्}
{तत्रतत्र महातेजा भ्रातृभिः सह सुव्रतः}


\twolineshloka
{ततो युधिष्ठिरो राजा बहुन्क्लेशान्विचिन्तयन्}
{सिंहव्याघ्रगजाकीर्णामुदीचीं प्रययौ दिशम्}


\twolineshloka
{अवेमाण कैलासं मैनाकं चैव पर्वतम्}
{गन्धमादनपादांश्च मेरुं चापि शिलोच्चयम्}


\twolineshloka
{उपर्युपरि शैलस्य बहीश्च सरितः शिवाः}
{पृष्ठं हिमवतः पुण्यं ययौ सप्तदशेऽहनि}


\twolineshloka
{ददृशुः पाण्डवा राजन्गन्धमादनमन्तिकम्}
{पृष्ठे हिमवतः पुण्ये नानाद्रुमलतावृते}


\twolineshloka
{सलिलावर्तसंजातैः पुष्पितैश्च महीरुहैः}
{समावृतं पुण्यतममाश्रमं वृषपर्वणः}


\twolineshloka
{तमुपक्रम्य राजर्षिं धर्मात्मानमरिंदमाः}
{पाण्डवा वृषपर्वाणमवदन्त गतक्लमाः}


\twolineshloka
{अभ्यनन्दत्स राजर्षिः पुत्रवद्भरतर्षभान्}
{पूजिताश्चावसंस्तत्रसप्तरात्रमरिंदमाः}


\twolineshloka
{अष्टमेऽहनि संप्राप्ते तमृषिं लोकविश्रुतम्}
{आमन्त्र्य वृषपर्वाणं प्रस्थानं प्रत्यरोचयन्}


\threelineshloka
{एकैकशश्च तान्विप्रान्निवेद्य वृषपर्वणि}
{न्यासभूतान्यथाकालं बन्धूनिव सुसत्कृतान्}
{पारिबर्हं चतं शेषं परिदाय महात्मने}


\twolineshloka
{ततस्ते यज्ञपात्राणि रत्नान्याभरणानि च}
{न्यदधुःपाण्डवा राजन्नाश्रमे वृषपर्वणः}


\twolineshloka
{अतीतानागते विद्वान्कुशलः सर्वधर्मवित्}
{अन्वशासत्स धर्मज्ञः पुत्रवद्भरतर्षभान्}


\twolineshloka
{तेऽनुज्ञाता महात्मानः प्रययुर्दिशमुत्तराम्}
{`कुष्णया सहिता वीरा ब्राह्मणैश्च महात्मभिः}


\twolineshloka
{तान्प्रस्थितानन्वगच्छद्वृषपर्वा महीपतीन्}
{उपन्यस् महातेजा विप्रेभ्यः पाण्डवांस्तदा}


\twolineshloka
{अनुसंसार्य कौन्तेयानाशीर्भिरभिनन्द्य च}
{वृषपर्वा निववृतेपन्थानमुपदिश्य च}


\twolineshloka
{नानामृगगणैर्जुष्टं कौन्तेयः सत्यविक्रमः}
{पदातिर्भ्रातृभिः सार्धं प्रातिष्ठत युधिष्ठिरः}


\twolineshloka
{नानाद्रुमनिरोधेषु वसन्तः शैलसानुषु}
{पर्वतं विविशुस्ते तं चतुर्थेऽहनि पाण्डवाः}


\twolineshloka
{महाभ्रघनसंकाशं सलिलोपहितं शुभम्}
{मणिकाञ्चनरम्यं च शैलं नानासमुच्छ्रयम्}


\twolineshloka
{`रम्यं हिमवतः प्रस्थं बहुकन्दरनिर्झरम्}
{शिलाविभङ्गविकटं लतापादपसङ्कुलम्'}


\twolineshloka
{ते समासाद्यपन्थानं यतोक्तं वृषपर्वणा}
{अनुसस्रुर्यथोद्देशं पश्यन्तो विविधान्नगान्}


\twolineshloka
{उपर्युपरि शैलस्य गुहाः परमदुर्गमाः}
{सुदुर्गमांस्ते सुबहून्सुखेनैवाभिचक्रमुः}


\twolineshloka
{धौम्यः कृष्णा च पार्ताश्च लोमशश्च महानृषिः}
{आगच्छन्सहितास्तत्रन कश्चिदपि हीयते}


\twolineshloka
{ते मृगद्विजसंघुष्टं नानाद्रुमलतायुतम्}
{शाखामृगगणैश्चैव सेवितं सुमनोरमम्}


\twolineshloka
{पुण्यं पद्मसरोपेतं सपल्वलमहावनम्}
{उपतस्थुर्महाभागा माल्यवतं महागिरिम्}


\twolineshloka
{ततः किंपुरुषावासं सिद्धचारणसेवितम्}
{ददृशुर्हृषरोमाणाः पर्वतं गन्धमादनम्}


\twolineshloka
{विद्याधरानुचरितं किन्नरीभिस्तथैव च}
{गजसङ्घसमावासं सिंहव्याघ्रगणायुतम्}


% Check verse!
शरभोन्नादसंघुष्टं नानमृगनिषेवितम्
\twolineshloka
{ते गन्धमादनवनं तन्नन्दनवनोपमम्}
{मुदिताः पाण्डुतनया मनोनयननन्दनम्}


\twolineshloka
{विविशुः क्रमशो वीरा अरण्यं शुभकाननम्}
{द्रौपदीसहिता वीरास्तैश्च विप्रैर्महात्मभिः}


\twolineshloka
{शृण्वन्तः प्रीतिजननान्वल्गून्मन्दकलाञ्शुभान्}
{श्रोत्ररम्यान्सुमधुराञ्शब्दान्खगमुखेरितान्}


\twolineshloka
{सर्वर्तुफलभाराञ्यान्सर्वर्तुकुसुमोज्ज्वलाम्}
{पश्यन्तः पादपांस्चापि फलभारावनामितान्}


\twolineshloka
{आम्रानाम्रातकान्फुल्लान्नारिकेलान्सतिन्दुकान्}
{मुञ्जातकांस्तथाञ्जीरान्दाडिमान्बीजपूरकान्}


\twolineshloka
{पनसाँल्लिकुचान्मोचाः खर्जूरानम्लवेतसान्}
{परावतांस्तथा क्षौद्रानीपांश्चापि मनोरमान्}


\twolineshloka
{बेल्वान्कपित्थाञ्जम्बूश्च काश्मरीर्ब्रदरीस्तथा}
{पुक्षानुदुम्बरवटानश्वत्थान्क्षीरिकास्तथा}


\twolineshloka
{मल्लातकानामलकीर्हरीतकबिभीतकान्}
{इङ्गुदान्करमर्दांश्च तिन्दुकांश् महाफलान्}


\twolineshloka
{एतानन्यांश्च विविधान्गन्धमादनसानुषु}
{फलैरमृतकल्पैस्तानाचितान्स्वादुभिस्तरूपन्}


\threelineshloka
{तथैव चम्पकाशोकान्केतकान्बकुलांस्तथा}
{पुन्नागान्सप्तपर्णांश्च कर्णिकारान्सकेतकान्}
{पाटलान्कुटजान्रम्यान्मन्दारेन्दीवरांस्तथा}


\threelineshloka
{पारिजातान्कोविदारान्देवदारुद्रुमांस्तथा}
{पारिजातान्कोविदारान्देवदारुद्रुमांस्तथा}
{शालांस्तालांस्तमालांश्च पिप्पलान्हिङ्गुकांस्तथा}


\twolineshloka
{चकोरैः शतपत्रैश् भृङ्गराजैस्तथा शुकैः}
{कोकिलैः कलविङ्कैश्च हारितैर्जीवजीवकैः}


\twolineshloka
{प्रियकैश्चातकैश्चैव तथाऽन्यैर्विविधैः खगैः}
{श्रोत्ररम्यं सुमधुरं कूजद्भिश्चाप्यधिष्ठितान्}


\threelineshloka
{सरांसि च मनोज्ञानि समन्ताज्जलचारिभिः}
{कुमुदैः पुण्डरीकैश्च तृथा कोकनदोत्पलैः}
{कहारैः कमलैश्चैव आचितानि समन्ततः}


\twolineshloka
{कादम्बैश्चक्रवाकैश्च कुररैर्जलकुक्कुटैः}
{कारण्डवैः प्लवैर्हंसैर्बकैर्मद्गुभिरेव च}


\twolineshloka
{एतैश्चान्यैश्च कीर्णानि समन्ताज्जलचारिभिः}
{हृष्टैस्तथा तामरसरसासवमदालसैः}


\twolineshloka
{पद्मोदरच्युतरजःकिञ्जल्कारुणरञ्जितैः}
{मञ्जुस्वरैर्मधुकरैर्विरुतान्कमलाकरान्}


\twolineshloka
{अपश्यंस्ते नरव्याघ्रा गन्धमादनसानुषु}
{तथैव पद्मषण्डैश्च मण्डितांश्च समन्ततः}


\twolineshloka
{शिखण्डिनीभिः सहिताँल्लतामण्डलकेषु च}
{मेघतूर्यरवोद्दाममदनाकुलितान्भृशम्}


\twolineshloka
{कृत्वैव केकामधुरं संगीतं मधुरस्वरम्}
{चित्रान्कलापान्विस्तीर्य सविलासान्मदालसान्}


\twolineshloka
{मयूरान्ददृशुर्हृष्टान्नृत्यतो वनलालसान्}
{कांश्चित्प्रियाभिः सहितान्रममाणान्कलापिनः}


\twolineshloka
{वल्लीलतासंकटेषु कुटजेषु स्थितांस्तथा}
{कांश्चिच्च कुटजानां तु विटपेषूत्कटानिव}


\twolineshloka
{कलापरुचिराटोपनिचितान्मुकुटानिव}
{विवरेषु तरूणां च रुचिरान्ददृशुश्च ते}


\threelineshloka
{सिन्धुवारांस्तथोदारान्मन्मथस्येव तोमरान्}
{सुवर्णवर्णकुसुमान्गिरीणां सिखरेषु च}
{कर्णिकारान्विकसितान्कर्णपूरानिवोत्तमान्}


\twolineshloka
{तथाऽपश्यन्कुरबकान्वनराजिषु पुष्पितान्}
{कामवश्यौत्सुक्यकरान्कामस्येव शरोत्करान्}


\twolineshloka
{तथैव वनराजीनामुदारान्रचितानिव}
{विराजमानांस्तेऽपश्यंस्तिलकांस्तिलकानिव}


\twolineshloka
{तथानङ्गशराकारान्सहकारान्मनोरमान्}
{अपश्यन्भ्रामरान्राजन्मञ्जरीभिर्विराजितान्}


\threelineshloka
{हिरण्यसदृशैः पुष्यैर्दावाग्निसदृशैरपि}
{लोहितैरञ्जनाभैश्च वैदूर्यसदृशैरपि}
{अतीव वृक्षा राजन्ते पुष्पिताः शैलसानुषु}


\twolineshloka
{तथा सालांस्तमालांश्च पाटलाबकुलानपि}
{माला इव समासक्ताः शैलानां शिखरेषु च}


\twolineshloka
{विमलस्फाटिकाभानि पाण्डुरच्छदनैर्द्विजैः}
{कलहंसैरुपेतानि सारसाभिरुतानि च}


\twolineshloka
{सरांसि बहुशः पार्था पश्यन्तः शैलसानुषु}
{पद्मोत्पलविमिश्राणि सुखशीतजलानि च}


\twolineshloka
{एवं क्रमेण ते वीरा वीक्षमाणाः समन्ततः}
{गन्वन्त्यथ माल्यानि रसवन्ति फलानि च}


\twolineshloka
{सरांसि च मनोज्ञानि वृक्षांश्चातिमनोरमान्}
{विविशुः पाण्डवाः सर्वेविस्मयोत्फुल्ललोचनाः}


\twolineshloka
{कमलोत्पलकह्लारपुण्डरीकसुगन्धिना}
{सेव्यमाना वने तस्मन्सुखस्पर्शेन वायुना}


\twolineshloka
{ततो युधिष्ठिरो भीममाहेदं प्रीतिमद्वचः}
{अहो श्रीमदिदं भीम गन्धमादनकाननम्}


\threelineshloka
{वने ह्मस्मिन्मनोरम्ये दिव्या काननजा द्रुमाः}
{लताश्च विविधाकाराः पत्रपुष्पफलोपगाः}
{भान्त्येते पुष्पविकचाः पुंस्कोकिलकुलाकुलाः}


\twolineshloka
{नात्र कण्टकिनः केचिन्न च विद्यन्त्यपुष्पिताः}
{स्निग्धपत्रफला वृक्षा गन्धमादनसानुषु}


\twolineshloka
{भ्रमरारावमधुरा नलिनीः फुल्लपङ्कजाः}
{विलोड्यमाना पश्येमाः करिभिः सकरेणुभिः}


\twolineshloka
{पश्येमां नलिनीं चान्यां कमलोत्पलमालिनीम्}
{स्रग्धरां विग्रहवतीं साक्षाच्छ्रियमिवापराम्}


\twolineshloka
{नानाकुसुमगन्धाढ्यास्तस्येमाः काननोत्तमे}
{उपगीयमाना भ्रमरै राजन्ते वनराजयः}


\twolineshloka
{पश्य भीम शुभान्देशान्देवाक्रीडान्समन्ततः}
{अमानुषगतिं प्राप्ताः संसिद्धाः स्म वृकोदर}


\twolineshloka
{लताभिः पुष्पिताग्राभिः पुष्पिताः पादपोत्तमाः}
{संश्लिष्टाः पार्थ शोभन्ते गन्धमादनसानुषु}


\twolineshloka
{शिखण्डिनीभिश्चरतां सहितानां शिखण्डिनाम्}
{नदतां शृणु निर्घोषं भीम पर्वतसानुषु}


\twolineshloka
{चकोराः शतपत्राश्च मत्तकोकिलशारिकाः}
{पत्रिणः पुष्पितानेतान्संपतन्ति महाद्रुमान्}


\twolineshloka
{रक्तपीतारुणाः पार्थ पादपाग्रगताः खगाः}
{परस्परमुदीन्ते बहवो जीवजीवकाः}


\twolineshloka
{हरितारुणवर्णानां शाड्वलानां समीपतः}
{सारसाः प्रतिदृश्यन्ते शैलप्रस्रवणेष्वपि}


\twolineshloka
{वदन्ति मधुरा वाचः सर्वभूतमनोरमाः}
{भृङ्गराजोपचक्राश्च लोहपृष्ठाः पतत्रिणः}


\twolineshloka
{चतुर्विषाणाः पद्माभाः कुञ्जराः सकरेणवः}
{एते वैडूर्यवर्णाभं क्षोभयन्ति महत्सरः}


\twolineshloka
{बहुतालसमुत्सेघाः शैलशृङ्गपरिच्युताः}
{नानाप्रस्रवणेभ्यश्च वारिधाराः पतन्ति च}


\twolineshloka
{भास्कराभप्रभा भीमाः शारदाभ्रघनोपमाः}
{शोभयन्ति महाशैलं नानारजतधातवः}


\twolineshloka
{क्वचिदञ्जनवर्णाभाः क्वचित्काञ्चनसन्निभाः}
{धातवो हरितालस्य क्वचिद्धिङ्गुलकस्य च}


\twolineshloka
{मनःशिलागुहाश्चैव संध्याभ्रनिकरोपमाः}
{शशलोहितवर्णाभाः क्वचिद्गैरिकधातवः}


\twolineshloka
{सितासिताभ्रप्रतिमा बालसूर्यसमप्रभाः}
{एते बहुविधाः शैलं शोभयन्ति महाप्रभाः}


\twolineshloka
{गन्धर्वाः सह कान्ताभिर्यथेष्टं भीमविक्रमाः}
{दृश्यन्ते शैलशृङ्गेषु पार्थ किंपुरुषैः सह}


\twolineshloka
{गीतानां समतालानां तथासाम्नां च निःखनः}
{श्रूयते बहुधा भीम सर्वभूतमनोहरः}


\twolineshloka
{महागङ्गामुदीक्षख पुण्यां देवनदीं शुभाम्}
{कलहंसगणैर्जुष्टामृषिकिन्नरसेविताम्}


\twolineshloka
{धातुभिश्च सरिद्भिश्च किन्नरैर्मृगपक्षिभिः}
{गन्धर्वैरप्सरोभिश्च काननैश्च मनोरमैः}


\threelineshloka
{व्यालैश्च विविधाकारैः शतशीर्षैः समन्ततः}
{उपेतं पश्य कौन्तेय शैलराजमरिंदम ॥वैशंपायन उवाच}
{}


\twolineshloka
{ते प्रीतमनसः शूराः प्राप्ता गतिमनुत्तमाम्}
{नातृप्यन्पर्वतेन्द्रस्य दर्शनेन परंतपाः}


\twolineshloka
{उपेतमथ माल्यैश्च फलवद्भिश्च पादपैः}
{आर्ष्टिपेणस् राजर्षेराश्रमं ददृशुस्तदा}


\twolineshloka
{ततस्ते तिग्मतपसं कृशं धमनिसंततम्}
{पारगं सर्वविद्यानामार्ष्टिषेणमुपागमन्}


\chapter{अध्यायः १६०}
\twolineshloka
{वैशंपायन उवाच}
{}


\twolineshloka
{युधिष्ठिरस्तमासाद्य तपसा दग्धकिल्बिषम्}
{अभ्यवादयत प्रीतः शिरसा नाम कीर्तयन्}


\twolineshloka
{ततः कृष्णा च भीमश्च यमौ च सुतपस्विनौ}
{शिरोभिः प्राप्य राजर्षिं परिवार्योपतस्थिरे}


\twolineshloka
{तथैव धौम्यो धर्मज्ञः पाण्डवानां पुरोहितः}
{यथान्यायमुपाक्रान्तस्तमृषिं संशितव्रतम्}


\twolineshloka
{अन्वजानात्स धर्मज्ञो मुनिर्दिव्येन चक्षुषा}
{पाण्डोः पुत्रान्कुरुश्रेष्ठानास्यतामिति चाब्रवीत्}


\twolineshloka
{कुरूणामृपभं प्राज्ञं पूजयित्वा महातपाः}
{सह भ्रातृभिरसीनं पर्यपृच्छदनामयम्}


\twolineshloka
{नानृते कुरुपे भावं कच्चिद्धर्मे प्रवर्तसे}
{मातापित्रोश्च ते वृत्तिः कच्चित्पार्थ न सीदति}


\twolineshloka
{कच्चित्ते गुरवः सर्वे वृद्धा वैद्याश्च पूजिताः}
{कच्चिन्न कुरुपे भावं पार्थ पापेषु कर्मसु}


\twolineshloka
{सुकृतंप्रतिकर्तुं च कच्चिद्धातुं च दुष्कृतम्}
{यथान्यायं कुरुश्रष्ठ जानासि न विकत्थसे}


\twolineshloka
{यथार्हं मानिताः कच्चित्त्वया नन्दन्ति साधवः}
{वनेष्वपि वसन्कच्चिद्धर्ममेवानुवर्तसे}


\twolineshloka
{कच्चिद्धौम्यस्त्वदाचारैर्न पार्थ परितप्यते}
{दानधर्मतपःशौचैरार्जवेन तितिक्षया}


\twolineshloka
{पितृपैतामहं वृत्तं कच्चित्पार्थानुवर्तसे}
{कच्चिद्राजर्षियातेन पथा गच्छसि पाण्डव}


\twolineshloka
{स्वेस्वे किल कुले जाते पुत्रे नप्तरि वा पुनः}
{पितरः पितृलोकस्थाः शोचन्ति च रमन्ति च}


\twolineshloka
{किं तस्य दुष्कृतेऽस्माभिः संप्राप्तव्यं भविष्यति}
{किं चास्य सुकृतेऽस्माभिः प्राप्तव्यमिति शोभनं}


\threelineshloka
{पिता माता तथैवाग्निर्गुरुरात्मा च पञ्चमः}
{यस्यैतेपूजिताः पार्थ तस् लोकावुभौ जितौ ॥[युधिष्ठिर उवाच}
{}


\threelineshloka
{भगवन्नार्य माऽऽहैतद्यथावद्धर्मनिश्चयम्}
{यथाशक्ति यथान्यायं क्रियते विधिवन्मया ॥आर्ष्टिषेण उवाच}
{]}


\twolineshloka
{अब्भक्षा वायुभक्षाश्च प्लवमाना विहायसा}
{जुषन्ते पर्वतश्रेष्ठमृषयः पर्वसन्धिषु}


\twolineshloka
{कामिनः सह कान्ताभिः परस्परमनुव्रताः}
{दृश्यन्ते शैलशृङ्गस्था यथा किंपुरुषा नृप}


\twolineshloka
{अरजांसि च वासांसि वसानाः कौशिकानि च}
{दृश्यन्ते बहवः पार्थ गन्धर्वाप्सरसां गणाः}


\twolineshloka
{विद्याधरगणाश्चैव स्रग्विणः प्रियदर्शनाः}
{महोरगगणाश्चैव सुपर्णाश्चारणादयः}


\twolineshloka
{अस्य चोपरि शैलस्य श्रूयते पर्वसन्धिषु}
{भेरीपणवशङ्खानां मृदङ्गानां च निःखनः}


\twolineshloka
{इहस्थैरेव तत्सर्वं श्रोतव्यं भरतर्षभाः}
{न कार्या वः कथंचित्स्यात्तत्राभिगमने मतिः}


\twolineshloka
{न चाप्यतः परं शक्यं गन्तुं भरतसत्तमाः}
{विहारस्तत्र देवानाममानुषगतिस्तु सा}


\twolineshloka
{ईषच्चपलकर्माणं मनुष्यमिह भारत}
{द्विषन्ति सर्वभूतानि ताडयन्ति च राक्षसाः}


\twolineshloka
{अस्यातिक्रम्य शिखरं कैलासस्य युधिष्ठिर}
{गतिः परमसिद्धानां देवर्षीणां प्रकाशते}


\twolineshloka
{चापलाद्धि न गन्तव्यं पार्त यानैस्ततः परम्}
{अयःशूलादिभिर्घ्नन्ति राक्षसाः शत्रुसूदन}


\twolineshloka
{अप्सरोभिः परिवृतः समृद्ध्या नरवाहनः}
{इह वैश्रवणस्तात पर्वसन्धिषु दृश्यते}


\twolineshloka
{शिखरे तं समासीनमधिपं यक्षरक्षसाम्}
{प्रेक्षन्ते सर्वभूतानि भानुमन्तमिवोदितम्}


\twolineshloka
{देवदानवसिद्धानां तथा वैश्रवणस्य च}
{गिरेः शिखरमुद्यानमिदं भरतसत्तम}


\twolineshloka
{उपासीनस्य धनदं तुम्बुरोः पर्वसन्धिषु}
{गीतसामस्वनस्तात श्रूयते गन्धमादने}


\twolineshloka
{एतदेवंविधं चित्रमिह तात युधिष्ठिर}
{प्रेक्षन्ते सर्वभूतानि बहुशः पर्वसन्धिषु}


\twolineshloka
{भुञ्जाना मुनिभोज्यानि रसवन्ति फलानि च}
{वसध्वं पाण्डवश्रेष्ठा यावदर्जुनदर्शनात्}


\twolineshloka
{न तात चपलैर्भाव्यमिह प्राप्तैः कथंचन}
{`चपलः सर्वभूतानां द्वेष्यो भवति मानवः'}


\twolineshloka
{उषित्वेह यथाकामं यथाश्रद्धं विहृत्य च}
{ततः शस्त्रजितां श्रेष्ठ पृथिवीं पालयिष्यसि}


\chapter{अध्यायः १६१}
\twolineshloka
{जनमेजय उवाच}
{}


\twolineshloka
{आर्ष्टिषेणाश्रमे तस्मिन्मम पूर्वपितामहाः}
{पाण्डोः पुत्रा महात्मान सर्वे दिव्यपराक्रमाः}


\twolineshloka
{कियन्तं कालमवसन्पर्वते गन्धमादने}
{किंच चक्रुर्महावीर्याः सर्वेऽतिबलपौरुषाः}


\twolineshloka
{कानि चाभ्यवहार्याणि तत्र तेषां महात्मनाम्}
{वसतां लोकवीराणामासंस्तद्ब्रूहि सत्तम}


\twolineshloka
{विस्तरेण च मे शंस भीमसेनपराक्रमम्}
{यद्यच्चक्रे महाबाहुस्तस्मिन्हैमवते गिरौ}


\twolineshloka
{न खल्वासीत्पुनर्युद्धं तस् यक्षैर्द्विजोत्तम}
{`धनदाध्युषिते नित्यं वसतस्तत्र पर्वते'}


\twolineshloka
{कच्चित्समागमस्तेषामासीद्वैश्रवणस्य च}
{तत्र ह्यायाति धनद आर्ष्टिषेणो यथाऽब्रवीत्}


\threelineshloka
{एतदिच्छाम्यहं श्रोतुं रविस्तरेण तपोधन}
{न हि मे शृण्वतस्तृप्तिरस्ति तेषां विचेष्टितम् ॥वैशंपायन उवाच}
{}


\twolineshloka
{एतदात्महितं श्रुत्वा तस्याप्रतिमतेजसः}
{शासनं सततं चक्रुस्तथैव भरतर्षभाः}


\twolineshloka
{भुञ्जाना मुनिभोज्यानि रसवन्ति फलानि च}
{शुद्धबाणहतानां च मृगाणां पिशितान्यपि}


\twolineshloka
{मेध्यानि हिमवत्पृष्ठे मधूनि विविधानि च}
{एवं ते न्यवसंस्तत्र पाण्डवा भरतर्षभाः}


\twolineshloka
{तथा निवसतां तेषां पञ्चमं वर्षमभ्यगात्}
{शृण्वतां लोमशोक्तानि वाक्यानि विविधान्युत}


\twolineshloka
{कृत्यकाल उपस्तास्य इति चोक्त्वा घटोत्कचः}
{राक्षसैः सह सर्वैश्च पूर्वमेव गतः प्रभो}


\twolineshloka
{आर्ष्टिषेणाश्रमे तेषां वसतां वै महात्मनाम्}
{अगच्छन्बहवो मासाः पश्यतां महदद्भुतम्}


\twolineshloka
{तैस्तत्रविहरद्भिश्च रममाणैश्च पाण्डवैः}
{प्रीतिमन्तो महाभागा मुनयश्चारणास्तथा}


\twolineshloka
{आजग्मुः पाण्डवान्द्रष्टुं शुद्धात्मानो यतव्रताः}
{ते तैः सह कथां चक्रुर्द्रिव्यां भरतसत्तमाः}


\twolineshloka
{ततः कतिपयाहस्सु महाह्रदनिवासिनम्}
{ऋद्धिमन्तं महानागं सुपर्णः सहसाऽहरत्}


\twolineshloka
{प्राकम्पत महाशैलः प्रामृद्यन्त महाद्रुमाः}
{ददृशुः सर्वभूतानि पाणअडवाश्च तदद्भुतम्}


\twolineshloka
{ततः शैलोत्तमस्याग्रात्पाण्डवान्प्रति मारुतः}
{अवहत्सर्वमाल्यानि गन्धवन्ति शुभानि च}


\twolineshloka
{तत्रपुष्पाणि दिव्यानि सुहृद्भिः सह पाण्डवाः}
{ददृशुः पञ्चवर्णानि द्रौपदी च यशस्विनी}


\twolineshloka
{भीमसेनं ततः कृष्णा काले वचनमब्रवीत्}
{विविक्ते पर्वतोद्देशे सुखासीनं महाभुजम्}


\twolineshloka
{सुपर्णानिलवेगेन श्वसनेन महाबलात्}
{पञ्चवर्णानि पात्यन्ते पुष्पाणि भरतर्षभ}


\twolineshloka
{`दिव्यावर्णानि दिव्यानि दिव्यगन्धवहानि च}
{मदयन्तीव गन्धेन मनो मे भरतर्षभ}


\twolineshloka
{येषां तु दर्शनात्स्पर्शान्सौरभ्याच्च तथैव च}
{नश्यतीव मनोदुःखं ममेदं शत्रुतापन}


\twolineshloka
{ईदृशैः कुसुमैर्दिव्यैर्दिव्यगन्धवहैः शुभैः}
{देवतान्यर्चयित्वाऽहमिच्छेयं सङ्गमं त्वया}


\twolineshloka
{इदं तु पुरुषव्याघ्र विशेषेणाम्बुजं शुभम्}
{गन्धसंस्थानसंपन्नं मम मानसवर्धनम्'}


% Check verse!
प्रत्यक्षं सर्वलोकस् नह्यदृश्यत मां प्रति
\twolineshloka
{`वासुदेवसहायेन वासुदेवप्रियेण च'}
{खाण्डवे सत्यसन्धेन भ्रात्रा तव महात्मना}


\twolineshloka
{गन्धर्वोरगरक्षांसि वासवश्च पराजितः}
{हता मायाविनश्चोग्रा धनुः प्राप्तं च गाण्डिवं}


\twolineshloka
{तवापि सुमहत्तेजो महद्बाहुबलं च ते}
{अविषह्यमनाधृष्यं शक्रतुल्यपराक्रम}


\twolineshloka
{त्वद्बाहुबलवेगेन त्रासिताः सर्वराक्षसाः}
{हित्वा शैलं प्रपद्यन्तां भीमसेन दिशो दश}


\twolineshloka
{ततः शैलोत्तमस्याग्रं चित्रमाल्यधरं शिवम्}
{व्यपेतभयसंमोहाः पश्यन्तु सुहृदस्तव}


\twolineshloka
{एवं प्रणिहितं भीम चिरात्प्रभृतिमे मनः}
{द्रष्टुमिच्छामि शैलाग्रं त्वद्बाहुबलमाश्रिता}


\twolineshloka
{`इच्छामि च नरव्याघ्र पुष्पं प्रत्यक्षमीदृशम्}
{आनीयमानं क्षिप्रं वै त्वया भरतसत्तम'}


\twolineshloka
{ततः क्षिप्तमिवात्मानं द्रौपद्या स परंतपः}
{नामृष्यत महाबाहुः प्रहारमिवसद्गजः}


\twolineshloka
{सिंहर्षभगतिः श्रीमानुदारः कनकप्रभः}
{मनस्वी बलवान्दृष्टो मानी भूरश्च पाण्डवः}


\twolineshloka
{लोहिताक्षः पृथुव्यंसो मत्तवारणविक्रमः}
{सिंहदंष्ट्रो वृषस्कन्धः सालपोत इवोद्गतः}


\twolineshloka
{महात्मा चारुसर्वाङ्ग कम्बुग्रीवो महाहनुः}
{रुक्मपृष्ठं धनुः खङ्गं तूणांश्चापि परामृशन्}


\twolineshloka
{सकेसरीव चोत्सिक्तः प्रभिन्न इव वारणः}
{व्यपेतभयसंमोहः शैलमभ्यपतद्बली}


\twolineshloka
{तं मृगेन्द्रमिवायान्तं प्रभिन्नमिव वारणम्}
{ददृशुः सर्वभूतानि पार्थं खङ्गधनुर्धरम्}


\twolineshloka
{द्रौपद्या वर्धयन्हर्षं गदामादाय पाण्डवः}
{व्यपेतभयसंमोहः शैलराजं समाविशत्}


\twolineshloka
{न ग्लानिर्न च कातर्यं न वैक्लव्यं न मत्सरः}
{कदाचिज्जुषते पार्थमात्मजं मातरिश्वनः}


\twolineshloka
{तदेकायनमासाद्य विषमं भीमदर्शनम्}
{बहुतालोच्छ्रयं शृङ्गमारुरोह महाबलः}


\twolineshloka
{स किन्नरमहानागमुनिगन्धर्वराक्षसान्}
{हर्षयन्पर्वतस्याग्रमारुरोह महाबलः}


\twolineshloka
{ततो वैश्रवणावासं ददर्श भरतर्षभः}
{काञ्चनैः स्फाटिकैश्चैव वेश्मभिः समलंकृतम्}


\twolineshloka
{[प्राकारेण परिक्षिप्तं सौवर्णेन समन्ततः}
{सर्वरत्नद्युतिमता सर्वोद्यानवता तथा}


\twolineshloka
{शैलादभ्युच्छ्रयवता चयाट्टालकशोभिना}
{द्वारतोरणनिर्व्यूहध्वजसंवाहशोभिना}


\twolineshloka
{विलासिनीभिरत्यर्थं नृत्यन्तीभिः समन्ततः}
{वायुना धूयमानाभिः पताकाभिरलंकृतम्}


\twolineshloka
{धनुष्कोटिमवष्टभ्य वक्रभावेन बाहुना}
{पश्यमानः सखेदेन द्रविणाधिपतेः पुरम् ॥]}


\twolineshloka
{मोदयनसर्वभूतानि गन्धमादनसंभवः}
{सर्वगन्धवहस्तत्र मारुतः सुसुखो ववौ}


\twolineshloka
{चित्रा विविधवर्णाभाश्चित्रमञ्जरिधारिणः}
{अचिन्त्या विविधास्तत्र द्रुमाः परमशोभिनः}


\twolineshloka
{रत्नजालपरिक्षिप्तं चित्रमाल्यविभूषितम्}
{राक्षसाधिपतेः स्थानं ददृशे भरतर्षभः}


\twolineshloka
{गदाखङ्गधनुष्पाणिः समभित्यक्तजीवितः}
{भीमसेनो महाबाहुस्तस्थौ गिरिवाचलः}


\twolineshloka
{ततः शङ्खमुपाध्माय द्विषतां रोमहर्षणम्}
{ज्याघोषं तलशब्दं च कृत्वा भूतान्यमोहयत्}


\twolineshloka
{ततः प्रहृष्टरोमाणस्तं शब्दमभिदुद्रुवुः}
{यक्षराक्षसगन्धर्वाः पाण्डवस्य समीपतः}


\twolineshloka
{गदापरिघनिस्त्रिंशशूलशक्तिपरश्वथाः}
{प्रगृहीता व्यरोचन्त यक्षराक्षसबाहुभिः}


% Check verse!
ततः प्रववृते युद्धं तेषां तस्य च भारत
\twolineshloka
{तैः प्रयुक्तान्महामायैः शूलशक्तिपरश्वथान्}
{भल्लैर्भीमः प्रचिच्छेद भीमवेगतरैस्ततः}


\twolineshloka
{अन्तरिक्षगतानां च भूमिष्ठानां च गर्जताम्}
{शरैर्विव्याध गात्राणइ राक्षसानां महाबलः}


\threelineshloka
{`शोणितस्य ततः पेतुर्घनानामिव भारत}
{'गात्रेभ्यः प्रच्युता धारा राक्षसानां समन्ततः'}
{स लोहितमहावष्टिमभ्यवर्षन्महाबलः}


\twolineshloka
{गदापरिघपाणीनां रक्षसां कायसंभवा}
{कायेभ्यः प्रच्युता धारा राक्षसानां समन्ततः}


\twolineshloka
{भीमबाहुबलोत्सृष्टैरायुधैर्यक्षरक्षसाम्}
{विनिकृत्तानि दृश्यन्ते शरीराणि शिरांसि च}


\twolineshloka
{प्रच्छाद्यमानं रक्षोभिः पाण्डवं प्रियदर्शनम्}
{ददृशुः सर्वभूतानि सूर्यमभ्रगणैरिव}


\twolineshloka
{स रश्मिभिरिवादित्यः शरैररिनिपातिभिः}
{सर्वानार्च्छन्महाबाहुर्बलवान्सत्यविक्रमः}


\twolineshloka
{अभितर्जयमानाश्च रुवन्तश् महारवान्}
{सन्नाहं भीमसेनस्य ददृशुः सर्वराक्षसाः}


\twolineshloka
{ते हि विक्षतसर्वाङ्गा भीमसेनभयार्दिताः}
{भीममार्तस्वरं चक्रुर्विप्रकीर्णमहायुधाः}


\twolineshloka
{उत्सृज्य ते गदाशूलानसिशक्तिपरश्वथान्}
{दक्षिणां दिशमाजग्मुस्त्रासिता दृढधन्वना}


\twolineshloka
{तत्र शूलगदापाणिर्व्यूढोरस्को महाभुजः}
{सखा वैश्रवणस्यासीन्मणिमान्नाम राक्षसः}


\twolineshloka
{दर्शयन्स प्रतीकारं पौरुषं च महाबलः}
{स तान्दृष्ट्वा परावृत्तान्स्मयमान इवाब्रवीत्}


\twolineshloka
{एकेन बहवः सङ्ख्ये मानुषेण पराजिताः}
{प्राप्य वैश्रवणावासं किं वक्ष्यथ धनेश्वरम्}


\twolineshloka
{एवमाभाष् तान्सर्वानभ्यवर्तत राक्षसः}
{शक्तिशूलगदापाणिरभ्यधावत्स पाण्डवम्}


\twolineshloka
{तमापतन्तं वेगेन प्रभिन्नमिव वारणम्}
{वत्सदन्तैस्त्रिभिः पार्श्वे भीमसेनः समार्दयत्}


\twolineshloka
{मणिमानपि संक्रुद्धः प्रगृह्य महतीं गदाम्}
{प्राहिणोद्भीमसेनाय परिगृह्य महाबलः}


\twolineshloka
{विद्युद्रूपां महाघोरामाकाशे महतीं गदाम्}
{शरैर्बहुभिरानर्छद्भीमसेनः शिलाशितैः}


\twolineshloka
{प्रत्यहन्यन्त ते सर्वेगदामासाद्य सायकाः}
{न वेगं धारयामासुर्गदावेगस्य वेगिताः}


\twolineshloka
{गदायुद्धसमाचारं बुध्यमानः स वीर्यवान्}
{व्यंसयामास तं तस्य प्रहारं भीमविक्रमः}


\twolineshloka
{ततः शक्तिं महाघोरां रुक्मदण्डामयस्मयीम्}
{तस्मिन्नेवान्तरे धीमान्प्रचिक्षेप स राक्षसः}


\twolineshloka
{सा भुजं भीमनिर्ह्रादा भित्त्वाभीमस् दक्षिणम्}
{साग्निज्वाला महारौद्रा पपात सहसा भुवि}


\twolineshloka
{सोऽतिविद्धो महेष्वासः शक्त्याऽमितपराक्रमः}
{गदां जग्राह कौन्तेयो गदायुद्धविशारदः}


\threelineshloka
{रुक्मपट्टपिनद्धां तां शत्रूणां भयवर्धिनीम्}
{प्रगृह्याथ नदन्भीमः शैक्यां सर्वायसीं गदाम्}
{तरसा चाभिदुद्राव मणिमन्तं महाबलम्}


\twolineshloka
{दीप्यमानं महाशूलं प्रगृह्य मणिमानपि}
{प्राहिणोद्भीमसेनाय वेगेन महता नदन्}


\twolineshloka
{भङ्क्त्वा शूलं गदाग्रेण गदायुद्धविभागवित्}
{अभिदुद्राव तं तूर्णं गरुत्मानिव पन्नगम्}


\twolineshloka
{सोऽन्तरिक्षमवप्लुत्य विधूय सहसा गदाम्}
{प्रचिक्षेप महाबाहुर्विनद्य रणमूर्धनि}


\twolineshloka
{सेन्द्राशनिरिवेन्द्रेण विसृष्टा वातरहसा}
{हत्वा रक्षः क्षितिं प्राप्य कृत्येव निपपात ह}


\twolineshloka
{तं राक्षसं बीमबलं भीमसेनबलाहतम्}
{ददृशुः सर्वभूतानि सिंहेनेव गवांपतिम्}


\twolineshloka
{तं प्रेक्ष्य निहतं भूमौ हतशेषा निशाचराः}
{भीममार्तस्वरं कृत्वा जग्मुः प्राचजीं दिशं प्रति}


\chapter{अध्यायः १६२}
\twolineshloka
{वैशंपायन उवाच}
{}


\twolineshloka
{श्रुत्वा बहुविधैः शब्दैर्नाद्यमानां गिरेर्गुहाम्}
{अजातशत्रुः कौन्तेयो माद्रीपुत्रावुभावपि}


\twolineshloka
{धौम्यः कृष्णा च विप्राश्च सर्वे च सुहृदस्तथा}
{भीमसेनमपश्यन्तः सर्वे विमनसोऽभवन्}


\twolineshloka
{द्रौपदीमार्ष्टिषेणाय संप्रधार्य महारथाः}
{सहिताः सायुधाः शूराः शैलमारुरुहुस्तदा}


\twolineshloka
{ततः संप्राप्य शैलाग्रं वीक्षमाणा महारथाः}
{ददृशुस्ते महेष्वासा भीमसेनमरिंदमम्}


\twolineshloka
{स्फुरतश्च महाकायान्गतसत्वांश्च राक्षसान्}
{महाबलान्महासत्वान्भीमसेनेव पातितान्}


\twolineshloka
{शुशुरतश्च महाकायान्गतसत्वांश्च राक्षसान्}
{महाबलान्महासत्वान्भीमसेनेन पातितान्}


\twolineshloka
{ततस्ते समतिक्रम्य परिष्वज्य वृकोदरम्}
{तत्रोपविविशुः पार्थाः प्राप्ता गतिमनुत्तमाम्}


\twolineshloka
{तैश्चतुर्भिर्महेष्वासैर्गिरिशृङ्गमशोभत}
{लोकपालैर्महाभागैर्दिवं देववरैरिव}


\twolineshloka
{कुबेरसदनं दृष्ट्वा राक्षसांश्च निपातितान्}
{भ्राता भ्रातरमासीनमथोवाच युधिष्ठिरः}


\twolineshloka
{साहसाद्यदिवा मोहाद्बीम पापमिदं कृतम्}
{नैतत्ते सदृशं वीर मुनेरिव मृषा वधाः}


\twolineshloka
{राजद्विष्टं न कर्तव्यमिति धर्मविदो विदुः}
{त्रिदशानामिदं द्विष्टं भीमसेन त्वया कृतम्}


\twolineshloka
{अर्थधर्मावनादृत्य यः पापे कुरुते मनः}
{कर्मणां पार्थ पापानां स फलं विन्दते ध्रुवम्}


\threelineshloka
{`साहसंवत भद्रं ते देवानामपि चाप्रियम्'}
{पुनरेवं न कर्तव्यं मम चेदिच्छसि प्रियम् ॥वैशंपायन उवाच}
{}


\twolineshloka
{एवमुक्त्वा स धर्मात्मा भ्राता भ्रातरमच्युतम्}
{`भीमसेनं महाबाहुमप्रधृष्यपराक्रमम्'}


\twolineshloka
{अर्थतत्त्वविभागज्ञः कुन्तीपुत्रो युधिष्ठिरः}
{विराम महातेजास्तमेवार्थं विचिन्तयन्}


\twolineshloka
{ततस्ते हतशिष्टा ये भीमसेनेन राक्षसाः}
{सहिताः प्रत्यपद्यन्त कुबेरसदनं प्रति}


\twolineshloka
{ते जवेन महावेगाः प्राप्य वैश्रवणालयम्}
{भीममार्तस्वरं चक्रुर्भीमसेनभयार्दिताः}


\twolineshloka
{न्यस्तशस्त्रायुधाः क्लान्ताः शोणिताक्तपरिच्छदाः}
{प्रकीर्णमूर्धजा राजन्यक्षाधिपतिमब्रुवन्}


\twolineshloka
{गदापरिघनिस्त्रिंसतोमरप्रासयोधिनः}
{राक्षसा निहताः सर्वे तव देवपुरःसराः}


\twolineshloka
{प्रमृद्यतरसा शैलं मानुषेण धनेश्वर}
{एकेन निहताः सङ्ख्ये गताः क्रोधवशा गणाः}


\twolineshloka
{प्रवरा राक्षसेन्द्राणां यक्षाणां च नराधिप}
{शेरते निहता देव गतसत्वाः परासवः}


\twolineshloka
{भग्नः शैलो वयं भग्ना मणिमांस्ते सखा हतः}
{मानुषेणं कृतं कर्म विधत्स्व यदनन्तरम्}


\twolineshloka
{स तच्छ्रुत्वा तु संक्रुद्धः सर्वयक्षगणाधिपः}
{कोपसंरक्तनयनः कथमित्यब्रवीद्वचः}


\twolineshloka
{द्वितीयमपराध्यन्तं भीमं श्रुत्वा धनेश्वरः}
{चुक्रोध यक्षाधिपतिर्युज्यतामिति चाब्रवीत्}


\twolineshloka
{अथाभ्रघनसंकाशं गिरिकूटमिवोच्छ्रितम्}
{रथं संयोजयामासुर्गर्न्धैवर्हेममालिभिः}


\threelineshloka
{तस्य सर्वगुणोपेता विमलाक्षा हयोत्तमाः}
{तेजोबलगुणोपेता नानारत्नविभूषिताः}
{शोभमाना रथे युक्तास्तरिष्यन्त इवाशुगाः}


\twolineshloka
{`ततस्ते तु महायक्षाः क्रुद्धं दृष्ट्वा धनेश्वरम्'}
{हर्षयामासुरन्योन्यमिङ्गितैर्विजयावहैः}


\twolineshloka
{स तमास्थाय भगवान्राजराजो महारथम्}
{प्रययौ देवगनधर्वैः स्तूयमानो महाद्युतिः}


\twolineshloka
{तं प्रयान्तं महात्मानं सर्वे यक्षा धनाधिपम्}
{`अनुजग्मुर्महात्मानं धनदं घोरदर्शनाः'}


\twolineshloka
{रक्ताक्षा हेमसंकाशा महाकाया महाबलाः}
{सायुधा बद्धनिस्त्रिंशा यक्षा बहुशतायुधाः}


\twolineshloka
{ते जवेन महावेगाः प्लवमाना विहायसा}
{गन्धमादनमाजग्मुः प्रकर्षन्त इवाम्बरम्}


\twolineshloka
{तत्केसरिमहाजालं धनाधिपतिपालितम्}
{`रम्यं चैव गिरेः शृङ्गमासेदुर्यत्रपाण्डवाः'}


\twolineshloka
{कुबेरं च महात्मानं यक्षरक्षोगणावृतम्}
{ददृशुर्हृष्टरोमाणः पाण्डवाः प्रियदर्शनम्}


\twolineshloka
{कुबेरस्तु महासत्वान्पाण्डोः पुत्रान्महारथान्}
{आत्तकार्मुकनिस्त्रिंशान्दृष्ट्वा प्रीतोऽभवत्तदा}


\twolineshloka
{`सर्वे चेमे नरव्याघ्राः पुरन्दरसमौजसः}
{'देवकार्यं करिष्यन्ति हृदयेन तुतोष ह}


\twolineshloka
{ते पक्षिण इवापेतुर्गिरिशृङ्गं महाजवाः}
{तस्थुस्तेषां सकाशे वै धनेश्वरपुरःसराः}


\twolineshloka
{ततस्तं हृष्टमनसं पाण्डवान्प्रति भारत}
{समीक्ष्ययक्षगन्धर्वा निर्विकारमवस्थिताः}


\twolineshloka
{पाण्डवाश्च महात्मानः प्रणम्य धनदं प्रभुम्}
{नकुलः सहदेवश् धर्मपुत्रश्च धर्मवित्}


\twolineshloka
{अपराद्धमिवात्मानं मन्यमाना महारथाः}
{तस्थुः प्राञ्जलयः सर्वे परिवार्य धनेश्वरम्}


\twolineshloka
{शय्यासनयुतं श्रीमत्पुष्पकं विश्वकर्मणा}
{विहितं चित्रपर्यन्तमातिष्ठत धनाधिपः}


\twolineshloka
{तमासीनं महाकायाः शङ्कुकर्णा महाजवाः}
{उपोपविविशुर्यक्षा राक्षसाश् सहस्रशः}


\twolineshloka
{शतशश्चापि गन्धर्वास्तथैवाप्सरसां गणाः}
{परिवार्योपतिष्ठन्ति यथा देवाः शतक्रतुम्}


\twolineshloka
{काञ्जनीं शिरसा बिभ्रद्भीमसेनः स्रजं शुभाम्}
{बाणखङ्गधनुष्पाणिरुदैक्षत धनाधिपम्}


\twolineshloka
{न भीर्भीमस्य न ग्लानिर्विक्षतस्यापि राक्षसैः}
{आसीत्तस्यामवस्तायां कुबेरमपि पश्यतः}


\twolineshloka
{आददानं शितान्बाणान्योद्धुकाममवस्थितम्}
{दृष्ट्वा भीमं धर्मसुतमब्रवीन्नरवाहनः}


\twolineshloka
{विदुस्त्वां सर्वभूतानि पार्थ भूतहिते रतम्}
{निर्भयश्चापिशैलाग्रे वस त्वं सह बन्धुभिः}


\twolineshloka
{न च मन्युस्त्वया कार्यो भीमसेनस्य पाण्डव}
{कालेनैते हताः पूर्वं निमित्तमनुजस्तव}


\twolineshloka
{व्रीडा चात्रन कर्तव्या साहसं यदिदं कृतम्}
{दृष्टश्चापि सुरैः पूर्वंविनाशो यक्षरक्षसाम्}


\threelineshloka
{न भीमसेने कोपो मे प्रीतोस्मि भरतर्षभ}
{कर्मणा भीमसेनस्य मम तुष्टिरभूत्पुरा ॥वैशंपायन उवाच}
{}


\twolineshloka
{स एवमुक्त्वा राजानं भीमसेनमभाषत}
{नैतन्मनसि मे तात वर्तते कुरुसत्तम}


\twolineshloka
{यदिदं साहसं भीम कृष्णार्थे कृतवानसि}
{मामनादृत्य देवांश्च विनाशं यक्षरक्षसाम्}


\twolineshloka
{स्वबाहुबलमाश्रित्य तेनाहं प्रीतिमांस्त्वयि}
{शापादद्य विनिर्मुक्तो घोरादस्माद्वृकोदर}


\twolineshloka
{अहं पूर्वमगस्त्येन क्रुद्धेन परमर्षिणा}
{शप्तोऽपराधे कस्मिंश्चित्तस्यैषा निष्कृतिर्ध्रुवम्}


\threelineshloka
{दृष्टो हि मम संक्लेशः पुरा पाण्डवनन्दन}
{न तवात्रापराधोस्ति कथंचिदपि पाण्डव ॥युधिष्ठिर उवाच}
{}


\twolineshloka
{कथं शप्तोसि भगवन्नगस्त्येन महात्मना}
{श्रोतुमिच्छाम्यहं देव यथैतच्छापकारणम्}


\threelineshloka
{इदं चाश्चर्यभूतं मे यत्क्रोधात्तस् धीमतः}
{तदैव त्वं न निर्दग्धः सबलः सपदानुगः ॥धनेश्वर उवाच}
{}


% Check verse!
देवतानामभून्मन्त्रः कुशावत्यां नरेश्वर
\twolineshloka
{वृतस्तत्राहमगमं महापद्मशतैस्त्रिभिः}
{यक्षाणआं घोररूपाणां विविधायुधधारिणाम्}


\threelineshloka
{अध्वन्यहमथापश्यमगस्त्यमृपिसत्तमम्}
{उग्रं तपस्तप्यभानं यमुनातीरमाश्रितम्}
{नानापक्षिगणाकीर्णं पुष्पितद्रुमशोभितम्}


\twolineshloka
{तमूर्द्वबाहुं दृष्ट्वैव सूर्यस्याभिमुखे स्थितम्}
{तेजोराशिं दीप्यमानं हुताशनमिवैधितम्}


\twolineshloka
{राक्षसाधिपतिः श्रीमान्मणिमान्नाम मे सखा}
{मौर्क्यादज्ञानभावाच्च दर्पान्मोहाच्च पार्थिव}


\twolineshloka
{न्यष्ठीवदाकाशगतो महर्षेस्तस्य मूर्धनि}
{ततः क्रुद्धः स भगवानुवाचेदं तपोधनः}


\twolineshloka
{मामवज्ञाय दुष्टात्मा यस्मादेष सखा तव}
{धर्षणां कृतवानेतां पश्यतस्ते धनेश्वर}


\twolineshloka
{त्वं चाप्येभिर्हतैः सैन्यैः क्लेशं प्राप्नुहि दुर्भते}
{तमेव मानुषं दृष्ट्वाकिल्विषाद्विप्रमोक्ष्यसे}


\twolineshloka
{सैन्यानां तु तवैतेषां पुत्रपौत्रबलान्वितम्}
{न शापं प्राप्यते घोरं गच्छ तेऽऽज्ञां करिष्यति}


\twolineshloka
{एष शापो मया प्राप्तः प्राक्तस्मादृषिसत्तमात्}
{स भीमेन महाराज भ्रात्रा तव विमोक्षितः}


\chapter{अध्यायः १६३}
\twolineshloka
{धनद उवाच}
{}


\twolineshloka
{युधिष्ठिर धृतिर्दाक्ष्यं देशकालपराक्रमाः}
{लोकतन्त्रविधानानामेष पञ्चविधो विधिः}


\twolineshloka
{धृतिमन्तश्च दक्षाश्च स्वे स्वे कर्मणि भारत}
{पराक्रमविधानज्ञा नराः कृतयुगेऽभवन्}


\twolineshloka
{धृतिमान्देशकालज्ञः सर्वधर्मविधानवित्}
{क्षत्रियः क्षत्रियश्रेष्ठ शास्ति वै पृथिवीमनु}


\twolineshloka
{य एवं वर्तते पार्थ पुरुषः सर्वकर्मसु}
{स लोके लभते वीर यशः प्रेत्य च सद्गतिम्}


\twolineshloka
{देशकालान्तरप्रेप्सुः कृत्वा शक्रः पराक्रमम्}
{संप्राप्तस्त्रिदिवे राज्यं वृत्रहा वसुभिः सह}


\threelineshloka
{[यस्तु केवलसंरम्भात्प्रपातं न निरीक्षते}
{]पापात्मा पापबुद्धिर्यः पापमेवानुवर्तते}
{कर्मणामविभागज्ञः प्रेत्य चेह विनश्यति}


\twolineshloka
{अकालज्ञः सुदुर्मेधाः कार्याणामविशेषवित्}
{वृथाचारसमारम्भः प्रेत्य चेह विनश्यति}


\twolineshloka
{साहसे वर्तमानानां निकृतीनां दुरात्मनाम्}
{सर्वेषामर्थलिप्सूनां पापो भवति निश्चयः}


\twolineshloka
{अधर्मज्ञोऽवलिप्तश्च बालबुद्धिरमर्षणः}
{निर्भयो भीमसेनोऽयं तं शाधि पुरुषर्षभ}


\twolineshloka
{आर्ष्टिषेणस् राजर्षेः प्राप्य भूयस्त्वमाश्रमम्}
{तमिस्रां प्रथमां तत्र वीतशोकभयो वस}


\twolineshloka
{अलकां सह गन्धर्वैर्यक्षैश्च सह किन्नरैः}
{`गमिष्यामि महाबाहो त्वं चापि बदरीं व्रज'}


\twolineshloka
{मन्नियुक्ता मनुष्येन्द्र सर्वे च गिरिवासिनः}
{रक्षन्तु त्वां महाबाहो सहितं द्विजसत्तमैः}


\twolineshloka
{साहसेषु च संतिष्ठंस्त्वया शैले वृकोदरः}
{वार्यतां साध्वयं राजंस्त्वया धर्मभृतांवर}


\twolineshloka
{इतः परं च वो राजन्द्रक्ष्यन्ति वनगोचराः}
{उपश्थास्यन्ति वो राजन्रक्षिष्यन्ते च वः सदा}


\twolineshloka
{तथैव चान्नपानानि स्वादूनि च बबूनि च}
{आहरिष्यन्ति मत्प्रेष्याः सदा वः पुरुषर्षभाः}


\twolineshloka
{यथा जिष्णुर्महेन्द्रस्य यथा वायोर्वृकोदरः}
{धर्मस्य त्वं यथा तात योगोत्पन्नो निजः सुतः}


\twolineshloka
{आत्मजावात्मसंपन्नौ यमौ चोभौ यथाश्विनोः}
{रक्ष्यास्तद्वन्ममापीह यूयं सर्वे युधिष्ठिर}


\twolineshloka
{अर्थतत्त्वविधानज्ञः सर्वधर्मविधानवित्}
{भीमसेनादवरजः पल्गुनः कुशली दिवि}


\twolineshloka
{याः काश्चन मता लोके स्वर्ग्याः परमसंपदः}
{जनमप्रभृति ताः सर्वाः स्थितास्तात धनंजये}


\twolineshloka
{द्रमो दानं बलं बुद्धिर्ह्रीर्धृतिस्तेज उत्तमम्}
{एतान्यपि महासत्वे स्थितान्यमिततेजसि}


\twolineshloka
{न मोहात्कुरुते जिष्णुः कर्म पाण्डव गर्हितम्}
{न पार्थस् मृषोक्तानि कथयन्ति नरा नृषु}


\twolineshloka
{स देवपितृगन्धर्वैः कुरूणआं कीर्तिवर्धनः}
{आनीतः कुरुतेऽस्त्राणि शक्रसद्मनि भारत}


\twolineshloka
{योऽसौ सर्वान्महीपालान्धर्मेण वशमानयत्}
{स शन्तनुर्महातेजाः पितुस्तव पितामहः}


\twolineshloka
{प्रीयते पार्थ पार्थेन दिवि गाण्डीवधन्वना}
{सम्यक्वासौ महावीर्यः कुलधुर्य इव स्थितः}


\twolineshloka
{पितॄन्देवानृषीन्विप्रान्पूजयित्वा महायशाः}
{सप्त मुख्यान्महामेधानाहरद्यमुनां प्रति}


\threelineshloka
{अधिराजः स राजंस्त्वां शन्तनुः प्रपितामहः}
{स्वर्गजिच्छक्रलोकस्थः कुशलं परिपृच्छति ॥वैशंपायन उवाच}
{}


\twolineshloka
{एतच्छ्रुत्वा तु वचनं धनदेन प्रभाषितम्}
{पाण्डवाश्च ततस्तेन बभूवः संप्रहर्षिताः}


\twolineshloka
{ततः शक्तिं गदां खङ्गं धनुश्च भरतर्षभः}
{प्राध्वंकृत्वा नमश्चक्रे कुबेराय वृकोदरः}


\twolineshloka
{ततोऽब्रवीद्धनाध्यक्षः शरण्यः शरणागतम्}
{मानहा भव शत्रूणां सुहृदां नन्दिवर्धनः}


\twolineshloka
{`विभयस्ताप शैलाग्रे वसानः सह बन्धुभिः}
{सुपर्णपितृदेवानां सततं मानकृद्भव}


\twolineshloka
{ऋजुं पश्यत मा वक्रं सत्यं वदत माऽनृतम्}
{दीर्घं पश्यत मा ह्रस्वं परं पश्यत माऽपरम्'}


\twolineshloka
{स्वेषु वेश्मसु रम्येषु वसतामित्रतापनाः}
{कामान्न परिहास्यन्ति यक्षा वोभरतर्षभाः}


\twolineshloka
{शीघ्रमेव गुडाकेशः कृतास्त्रः पुरुषर्षभः}
{साक्षान्मघवतोत्सृष्टः संप्राप्स्यति धनंजयः}


\twolineshloka
{एवमुत्तमकर्माणमनुशिष्य युधिष्ठिरम्}
{अस्तं गिरिमिवादित्यः प्रययौ गुह्यकाधिपः}


\twolineshloka
{तं परिस्तोमसंकीर्णैर्नानारत्नविभूषितैः}
{यानैरनुययुर्यक्षा राक्षसाश्च सहस्रशः}


\twolineshloka
{पक्षिणामिव कनिर्घोषः कुबेरसदनं प्रति}
{बभूव परमाश्वानामैरावतपथे यथा}


\twolineshloka
{ते जग्मुस्तूर्णमाकाशं धनाधिपतिवाजिनः}
{प्रकर्षन्त इवाभ्राणि पिबन्त इवमारुतम्}


\twolineshloka
{ततस्तानि शरीराणि गतसत्वानि रक्षसाम्}
{अपाकृष्यन्त शैलाग्राद्धनाधिपतिशासनात्}


\twolineshloka
{तेषां हि शापकालः स कृतोऽगस्त्येन धीमता}
{समरे निहतास्तस्माच्छापस्यान्तोऽभवत्तदा}


\twolineshloka
{पाण्डवाश्महात्मानस्तेषु वेश्मसु तां क्षपाम्}
{सुखमूषुर्गतोद्वेगाः पूजिताः सह राक्षसैः}


\chapter{अध्यायः १६४}
\twolineshloka
{वैशंपायन उवाच}
{}


\twolineshloka
{ततः सूर्योदये धौम्यः पाञ्चालीसहितांस्च तान्}
{आर्ष्टिणेन सहितः पाण्डवानभ्यवर्तत}


\twolineshloka
{तेऽभिवाद्यार्ष्टिषेणस्य पादौ धौम्यस्य चैव ह}
{ततः प्राञ्जलयः सर्वे ब्राह्मणांस्तानपूजयन्}


\threelineshloka
{`आर्ष्टिषेणं परिष्वज्य पुत्रवद्भरतर्षभ'}
{धर्मराजं स्पृशन्पाणौ पाणिना स महातपाः}
{प्राचीं दिशमभिप्रेक्ष्य महर्षिरिदमब्रवीत्}


\threelineshloka
{असौ सागरपर्यन्तां भूमिमावृत्य तिष्ठति}
{शैलराजो महाराज मन्दरोऽभिविराजयन्}
{इन्द्रवैश्रवणोपेतां दिशं पाण्डव रक्षति}


\threelineshloka
{पर्वतैश्च वनान्तैश्च काननैश्चैव शोभितम्}
{एतमाहुर्महेन्द्रस्य राज्ञो वैश्रवणस्य च}
{ऋषयः सर्वधर्मज्ञाः सर्वे तात मनीषिणः}


\twolineshloka
{अतश्चोद्यन्तमादित्यमुपतिष्ठन्ति वै प्रजाः}
{ऋषयश्चापि धर्मज्ञाः सिद्धाः साध्याश्च देवताः}


\twolineshloka
{यमस्तु राजा धर्मज्ञः सर्वप्राणभृतां प्रभुः}
{प्रेतसत्वगतीमेतां दक्षिणामाश्रितो दिशम्}


\twolineshloka
{एतत्संयमनं पुण्यमतीवाद्भुतदर्शनम्}
{प्रेतराजस्य भवनमृद्ध्या परमया युतम्}


\twolineshloka
{यं प्राप्य सविता राजन्सत्येन प्रतितिष्ठति}
{अस्तं पर्वतराजानमेतमाहुर्मनीषिणः}


\twolineshloka
{एवं पर्वतराजानं समुद्रं च महोदधिम्}
{अवसन्वरुणो राजा भूतानि परिरक्षति}


\twolineshloka
{उदीचीं दीपयन्नेष दिशं तिष्ठति कीर्तिमान्}
{महामेरुर्महाभाग शिवो ब्रह्मविदांगतिः}


\twolineshloka
{यस्मिन्ब्रह्मसदश्चैव भूतात्मा चावतिष्ठते}
{प्रजापतिः सृजन्सर्वं यत्किंचिज्जङ्गमागमम्}


\twolineshloka
{यानाहुर्ब्रह्मणः पुत्रान्मानसान्दक्षसप्तमान्}
{तेषामपि महामेरुः शिवं स्थानमनामयम्}


\twolineshloka
{अत्रैव प्रतितिष्ठन्ति पुनरेषोदयन्ति च}
{सप्त देवर्षयस्तात वसिष्ठप्रमुखास्तदा}


\twolineshloka
{देशं विरजसं पश्य मेरोः शिखरमुत्तमम्}
{यत्रात्मतृप्तैरध्यास्ते देवैः सह पितामहः}


\twolineshloka
{यमाहुः सर्वभूतानां प्रकृतेः प्रकृतिं ध्रुवम्}
{अनादिनिधनं देवं प्रभुं नारायणं परम्}


\twolineshloka
{ब्रह्मणः सदनात्तस्य परं स्थानं प्रकाशते}
{देवाश्च यत्नात्पश्यन्ति सर्वतेजोमयं शुभम्}


\twolineshloka
{अत्यर्कानलदीप्तं तत्स्थानं विष्णोर्महात्मनः}
{स्वयैव प्रभया राजन्दुष्प्रेक्ष्यं देवदानवैः}


% Check verse!
प्राच्यां नारायणस्थानं मेरावतिविराजते
\twolineshloka
{यत्र भूतेश्वरस्तात सर्वप्रकृतिरात्मभूः}
{भासयन्सर्वभूतानि सुश्रियाऽभिविराजते}


\twolineshloka
{नात्र ब्रह्मर्षयस्तात कुत एव महर्षयः}
{प्राप्नुवन्ति गतिं ह्येतां यतीनां भावितात्मनाम्}


\twolineshloka
{न तं ज्योतींपि सर्वाणि प्राप्य भासन्ति पाण्डव}
{स्वयंप्रभुरचिन्त्यात्मा तत्र ह्यतिविराजते}


\threelineshloka
{यतयस्तत्र गच्छन्ति भक्त्या नारायणं हरिम्}
{परेण तपसा युक्ता भाविताः कर्मभिः शुभैः}
{योगसिद्धा महात्मानस्तमोमोहविवर्जिताः}


\twolineshloka
{तत्र गत्वा पुनर्नेमं लोकमायान्ति भारत}
{स्वयंभुवं महात्मानं देवदेवं सनातनम्}


\twolineshloka
{स्थानमेतन्महाभाव ध्रुवमक्षयमव्ययम्}
{ईश्वरस्य सदा ह्येतत्प्रणमात्र युधिष्ठिर}


\twolineshloka
{[एनं त्वहरहर्मेरुं सूर्याचन्द्रमसौ ध्रुवम्}
{प्रदक्षिणमुपावृत्यकुरुतः कुरुनन्दन}


\twolineshloka
{ज्योतींषि चाप्यशेषेण सर्वाण्यनघ सर्वतः}
{परियान्ति महाराज गिरिराजं प्रदक्षिणम् ॥]}


\twolineshloka
{एतं ज्योतींषि सर्वाणि प्रकर्षन्भगवानपि}
{कुरुते वितमस्कर्मा आदित्योऽभिप्रदक्षिणम्}


\twolineshloka
{अस्तं प्राप्य ततः सन्ध्यामतिक्रम्य दिवाकरः}
{उदीचीं भजते काष्ठां बहुधा पर्वसन्धिषु}


\twolineshloka
{सुमेरुमनुवृत्तः स पुनर्गच्छति पाण्डव}
{प्राङ्युखः सविता देवः सर्वभूतहिते रतः}


\twolineshloka
{स सम्यग्विभजन्कालान्बहुधा पर्वसन्धिषु}
{तथैव भगवान्सोमो नक्षत्रैः सह गच्छति}


\threelineshloka
{एवमेतं त्वतिक्रम्य महामेरुमतन्द्रितः}
{सोमश्च विभजन्कालं बहुधा पर्वसन्धिषु}
{भासयन्सर्वभूतानि पुनर्गच्छति सागरम्}


\twolineshloka
{तथा तमिस्रहा देवो मयूखैर्भासयञ्जगत्}
{मार्गमेतमसंबाधमादित्यः परिवर्तते}


\twolineshloka
{सिसृक्षुः शिशिराण्येव दक्षिणां भजते दिशम्}
{ततः सर्वाणि भूतानि कालं शिशिरमृच्छति}


\twolineshloka
{स्थावराणां च भूतानां जङ्गमानां च तेजसा}
{तेजांसि समुपादत्ते निवृत्तः स विभावसुः}


\twolineshloka
{ततः स्वेदः क्लमस्तन्द्री ग्लानिश्च भजते नरान्}
{प्राणिभिः सततं स्वप्नोह्यभीक्ष्णं च निपेव्यते}


\twolineshloka
{एवमेतमनिर्देश्यं मार्गमावृत्य भानुमान्}
{पुनः सृजति वर्षाणि भगवान्भासयन्प्रजाः}


\twolineshloka
{वृष्टिमारुतसंतापैः सुखैः स्थावरजङ्गमान्}
{वर्धयनसुमहातेजाः पुनः प्रतिनिवर्तते}


\twolineshloka
{एवमेष चरन्पार्थ कालचक्रमतन्द्रितः}
{प्रकर्षन्सर्वभूतानि सविता परिवर्तते}


\twolineshloka
{संतता गतिरेतस्य नैष तिष्ठति पाण्डव}
{आदायैव तु भूतानां तेजो विसृजते पुनः}


\twolineshloka
{विभजन्सर्वभूतानामायुः कर्म च भारत}
{अहोरात्रं कलाः काष्ठाः सृजत्येष सदा विभुः}


\chapter{अध्यायः १६५}
\twolineshloka
{वैशंपायन उवाच}
{}


\twolineshloka
{तस्मिन्नगेन्द्रे वसतां तु तेषांमहात्मनां सद्व्रतमास्थितानाम्}
{रतिः प्रमोदश्च बभूव तेषा-माकाङ्क्षतां दर्शनमर्जुनस्य}


\twolineshloka
{तान्वीर्ययुक्तान्सुविशुद्धसत्वां-स्तेजस्विनः सत्यधृतिप्रधानान्}
{संप्रीयमाणा बहवोऽबिजग्मु-र्गन्धर्वसङ्घाश्च महर्षयश्च}


\twolineshloka
{तं पादपैः पुष्पफलैरुपेतंनगोत्तमं प्राप्य महारथानाम्}
{मनःप्रसादः परमो बभूवयथा दिवं प्राप्य मरुद्गणानाम्}


\twolineshloka
{मयूरहंसस्वननादितानिपुष्पोपकीर्णानि महाचलस्य}
{शृङ्गाणि सानूनि च पश्यमानागिरेः परं हर्षमवाप्य तस्थुः}


\twolineshloka
{साक्षात्कुबेरेण कृताश्च तस्मि-न्नगोत्तमे संवृतकूटगुल्माः}
{कादम्बकारण्डवहंसजुष्टाःपद्माकुलाः पुष्करिणीरपश्यन्}


\twolineshloka
{क्रीडाप्रदेशांस्च समृद्धरूपा-न्सवेदिकांस्ते न्यवसन्सुवेशान्}
{मणिप्रकीर्णांश्च मनोरमांश्चयथा भवेयुर्नदस्य राज्ञः}


\twolineshloka
{अनेकवर्णैश् सुगन्धिभिश्चमहाद्रुमैः संततमभ्रजालैः}
{तपःप्रधानाः सततं चरन्तःशृङ्गं गिरेश्चिन्तयितुं न शेकुः}


\twolineshloka
{स्वतेजसा तस्य नगोत्तमस्यमहौषधीनां च तथा प्रभावात्}
{विभक्तरूपः सविता बभूवनिशामुखं प्राप्य नरर्षभाणाम्}


\twolineshloka
{यमास्थितः स्थावरजङ्गमानांविभावसुर्भाविता हरीशः}
{तस्योदयं चास्तमनं च वीर-स्तत्रस्थितास्ते ददृशुर्नृसिंहाः}


\twolineshloka
{रवेस्तमिस्रागमनिर्गमांस्तेतथोदयं चास्तमनं च वीराः}
{समावृताः प्रेक्ष्य तमोनुदस्यगभस्तिजालैः प्रदिशो दिशश्च}


\twolineshloka
{स्वाध्यायवन्तः सततक्रियाश्चधर्मप्रधानाश्च शुचिव्रताश्च}
{सत्ये स्थितास्तस्य महारथस्यसत्यव्रतस्यागमनप्रतीक्षाः}


\twolineshloka
{इहैव हर्षोऽस्तु समागतानांक्षिप्रं कृतास्त्रेण धनंजयेन}
{इति ब्रुवन्तः परमाशिपस्तेपार्थास्तपोयोगपरा बभूवुः}


\twolineshloka
{दृष्ट्वा विचित्राणि गिरौ वनानिकिरीटिनं चिन्तयतामभीक्ष्णम्}
{बभूव राविर्दिवसश्च तेषांसंवत्सरेणैव समानरूपः}


\twolineshloka
{यदैव धौम्यानुमते महात्माकृत्वा जटां प्रव्रजितः स जिष्णुः}
{तदैव तेषां नबभूव हर्षःकुतो रतिस्तद्गतमानसानाम्}


\twolineshloka
{भ्रातुर्नियोगात्तु युधिष्ठिरस्यवनादसौ वारणमत्तगामी}
{यत्काम्यकात्प्रव्रजितः स जिष्णु-स्तदैव ते शोकहता बभूवुः}


\twolineshloka
{तथैव तं चिन्तयतां सिताश्व-मस्त्रार्थिनं वासवमभ्युपेतम्}
{कालः स कृच्छ्रेण महानतीत-स्तस्मिन्नगे भारत भारतानाम्}


\twolineshloka
{[उषित्वा पञ्चवर्षाणि सहस्राक्षनिवेशने}
{अवाप्य दिव्यान्यस्त्राणि सर्वाणि विबुधेश्वरात्}


\twolineshloka
{आग्नेयं वारुणं सौम्यं वायव्यमथ वैष्णवम्}
{ऐन्द्रं पाशुपतं ब्राह्मं पारमेष्ठ्यं प्रजापतेः}


\twolineshloka
{यमस्य धातुः सवितुस्त्वष्टुर्वैश्रवणस्य चं}
{तानि प्राप्य सहस्राक्षादभिवाद्य शतक्रतुम्}


\twolineshloka
{अनुज्ञातस्तदा तेन कृत्वा चापि प्रदक्षिणम्}
{आगच्छदर्जुनः प्रीतः प्रहृष्टो गन्धपादनम्}


\chapter{अध्यायः १६६}
\twolineshloka
{वैशंपायन उवाच}
{}


\twolineshloka
{ततः कदाचिद्धरिसंप्रयुक्तंमहेन्द्रवाहं सहसोपयातम्}
{विद्युत्प्रभं प्रेक्ष्य महारथानांहर्षोऽर्जुनं चिन्तयतां बभूव}


\twolineshloka
{स दीप्यमानः सहसाऽन्तरिक्षंप्रकाशयन्मातलिसंगृहीतः}
{बभौ महोल्केव घनान्तरस्थाशिखेव चाग्नेर्ज्वलिता विधूमा}


\twolineshloka
{तमास्थितः संददृशे किरीटीस्रग्वी नवान्याभरणानि बिभ्रत्}
{धनंजयो वज्रधरप्रभावःश्रिया ज्वलन्पर्वतमाजगाम्}


\twolineshloka
{स शैलमासाद्य किरीटमालीमहेन्द्रवाहादवरुह्य तस्मात्}
{धौम्यस्य पादावभिवाद्य पूर्व-मजातशत्रोस्तदनन्तरं च}


\twolineshloka
{वृकोदरस्यापि च वन्द्य पादौमाद्रीसुताभ्यामभिवादितश्च}
{समेत्य कृष्णां परिसान्त्व्य चैनांप्रह्वोऽभवद्भ्रातुरुपह्वरे सः}


\twolineshloka
{बभूव तेषां परमः प्रहर्ष-स्तेनाप्रमेयेण समेत्य तत्र}
{स चापि तान्प्रेक्ष्य किरीटमालीननन्द राजानमभिप्रशंसन्}


\twolineshloka
{यमास्थितः सप्त जघान पूगान्दितेः सुतानां नमुचेर्निहन्ता}
{तमिन्द्रवाहं समुपेत्य पार्थाःप्रदक्षिणं चक्रुरदीनसत्वाः}


\twolineshloka
{ते मातलेश्चक्रुरतीव हृष्टाःसत्कारमग्र्यं सुरराजतुल्यम्}
{सर्वं यथावच्च दिवौकसङ्घंपप्रच्युरेनं कुरुराजपुत्राः}


\twolineshloka
{तानप्यसौ मातलिरभ्यनन्द-त्पितेव पुत्राननुशिष्य पार्थान्}
{ययौ रथेनाप्रतिमप्रभेणपुनः सकाशं त्रिदिवेश्वरस्य}


\twolineshloka
{गते तु तस्मिन्वरदेववाहेशक्रात्मजः सर्वरिपुप्रमाथी}
{`साक्षात्सहस्राक्ष इव प्रतीतःश्रीमान्स्वदेहादवमुच्य जिष्णुः'}


\twolineshloka
{शक्रेण दत्तानिददौ महात्मामहाधनान्युत्तमरूपवन्ति}
{दिवाकराभाणि विभूषणानिप्रीतः प्रियायै सुतसोममात्रे}


\twolineshloka
{ततऋः स तेषां कुरुपुङ्गवानांतेषां च सूर्याग्निसमप्रभाणाम्}
{विप्रर्षभाणामुपविश्य मध्येसर्वं यथावत्कथयांबभूव}


\twolineshloka
{एवं मयाऽस्त्राण्युपशिक्षितानिशक्राच्च वायोश्च शिवाच्च साक्षात्}
{तथैव शीलेन समाधिना चप्रीताः सुरा मे सहिताः सहेन्द्राः}


\twolineshloka
{संक्षेपतो वै स विशुद्धकर्मातेभ्यः समाख्याय दिवःप्रवेशम्}
{माद्रीसुताभ्यां सहितः किरीटीसुष्वाप तामावसतिं प्रतीतः}


\chapter{अध्यायः १६७}
\twolineshloka
{वैशंपायन उवाच}
{}


\twolineshloka
{[ततो रजन्यां व्युष्टायां धर्मराजं युधिष्ठिरम्}
{भ्रातृभिः सहितः सर्वैरवन्दत धनंजयः ॥]}


\twolineshloka
{एतस्मिन्नेव काले तु सर्ववादित्रनिःखनः}
{बभूव तुमुलः शब्दस्त्वन्तरिक्षे दिवौकसाम्}


\twolineshloka
{रथनेमिखनश्चैव घण्टाशब्दश्च भारत}
{पृथग्व्यालमृगाणां च पक्षिणां चैव सर्वशः}


\twolineshloka
{`रवोन्मुखास्ते ददृशुः प्रीयमाणाः कुरूद्वह}
{मरुद्भिरन्वितं शक्रमापतन्तं विहायसा'}


\twolineshloka
{ते समन्तादनुययुर्गन्धर्वाप्सरसस्तथा}
{विमानैः सूर्यसंकाशैर्देवराजमरिंदमम्}


\twolineshloka
{ततः स हरिभिर्युक्तं जाम्बूनदपरिष्कृतम्}
{मेघनादिनमारुह्य श्रिया परमया ज्वलन्}


\twolineshloka
{पार्थानभ्याजपामाशु देवराजः पुरंदरः}
{आगत्य च सहस्राक्षो रथादवरुरोह वै}


\twolineshloka
{तं दृष्ट्वैव महात्मानं धर्मराजो युधिष्ठिरः}
{भ्रातृभिः सहितः श्रीमान्देवराजमुपागमत्}


\twolineshloka
{पूजयामास चैवाथ विधिवद्भूरिदक्षिणः}
{यथार्हममितात्मानं विधिदृष्टेन कर्मणा}


\twolineshloka
{धनंजयश्च तेजस्वी प्रणिपत्य पुरंदरम्}
{भृत्यवत्प्रणतस्तस्थौ देवराजसमीपतः}


\twolineshloka
{आघ्राय तं महातेजाः कुन्तीपुत्रो युधिष्ठिरः}
{धनंजयभिप्रेक्ष्यविनीतं स्थितमन्तिके}


\threelineshloka
{जटिलं देवराजस्य तपोयुक्तमकल्मषम्}
{हर्षेण महताऽऽविष्टः फल्गुनस्याथ दर्शनात्}
{बभूव परमप्रीतो देवराजं च पूजयन्}


\twolineshloka
{तं तथाऽदीनमनसं राजानं हर्,संप्लुतम्}
{उवाच वचनं धीमान्धर्मराजं पुरंदरः}


\twolineshloka
{त्वमिमां पृथिवीं राजन्प्रशासिष्यसि पाण्डव}
{स्वस्ति प्राप्नुहि कौन्तेय काम्यकं पुनराश्रमम्}


\twolineshloka
{अस्त्राणि लब्धानि च पाण्डवेनसर्वाणि मत्तः प्रयतेन राजन्}
{कृतप्रियश्चास्मि धनंजयेनजेतुं न शक्यस्त्रिभिरेष लोकैः}


\twolineshloka
{एवमुक्त्वा सहस्राक्षः कुन्तीपुत्रं युधिष्ठिरम्}
{जगाम त्रिदिवं हृष्टः स्तूयमानो महर्षिभिः}


\twolineshloka
{धनेश्वरगृहस्थानां पाण्डवानां समागमम्}
{शक्रेण य इदं विद्वानधीयीत समाहितः}


\twolineshloka
{संवत्सरं ब्रह्मचारी नियतः संशितव्रतः}
{स जीवेद्धि निराबाधः सुसुखी शरदां शतम्}


\chapter{अध्यायः १६८}
\twolineshloka
{वैशपायन उवाच}
{}


\twolineshloka
{यथागतं गते शक्रे भ्रातृभिः सह संगतः}
{कृष्णया चैव बीभत्सुर्धर्मराजमपूजयत्}


\twolineshloka
{अभिवादयमानं तं मूर्ध्न्युपाघ्राय पाण्डवम्}
{हर्षगद्गदया वाचा प्रहृष्टोऽर्जुनमब्रवीत्}


\twolineshloka
{कथमर्जुन कालोऽयं स्वर्गे रव्यतिगतस्तव}
{कथं चास्त्राण्यवाप्तानि देवराजश्च तोषितः}


\twolineshloka
{सम्यग्वा ते गृहीतानि कच्चिदस्त्राणि पाण्डव}
{कच्चित्सुराधिपः प्रीतो रुद्रश्चास्त्राण्यदात्तव}


\twolineshloka
{यथा दृष्टश्च ते शक्रो भगवान्वा पिनाकधृत्}
{यथैवास्त्राण्यवाप्तानि यथैवाराधिताश्च ते}


\twolineshloka
{यथोक्तवांस्त्वां भगवाञ्शतक्रतुररिंदम}
{कृतप्रियस्त्वयाऽस्मीति तस्य ते किं प्रियं कृतम्}


\twolineshloka
{एतदिच्छाम्यहं श्रोतुं विस्तरेण महाद्युते}
{यथा तुष्टो महादेवो देवराजस्तथाऽनघ}


\threelineshloka
{यच्चापि वज्रपाणेस्तु प्रियं कृतमरिंदम}
{एतदाख्याहि मे सर्वमखिलेन धनंजय ॥अर्जुन उवाच}
{}


\twolineshloka
{शृणु हन्त महाराज विधिना येन दृष्टवान्}
{शतक्रतुमहं देवं भगवन्तं च शंकरम्}


\twolineshloka
{विद्यामधीत्य तां राजंस्त्वयोक्तामरिमर्दन}
{भवता च समादिष्टस्तपसे प्रस्थितो वनम्}


\twolineshloka
{भृगुतुन्दमथो गत्वा काम्यकादास्थितस्तपः}
{एकरात्रोषितः कंचिदपश्यं ब्राह्मणं पथि}


\twolineshloka
{स मामपृच्छत्कौन्तेय क्वासि गन्ता ब्रवीहि मे}
{तस्मा अवितथं सर्वमब्रवं कुरुनन्दन}


\twolineshloka
{स तथ्यं मम तच्छ्रुत्वा ब्राह्मणो राजसत्तम}
{अपूजयत मां राजन्प्रीतिमांश्चाभवन्मयि}


\twolineshloka
{ततो मामब्रवीत्प्रीतस्तप आतिष्ठ भारत}
{तपस्वी नचिरेण त्वं द्रक्ष्यसे विबुधाधिषम्}


\twolineshloka
{ततोऽहं वचनात्तस्य गिरिमारुह्य शैशिरम्}
{तपोऽतप्यं महाराज मासं मूलफलाशनः}


\twolineshloka
{द्वितीयश्चापि मे मासो जलं भक्षयतो गतः}
{निराहारस्तृतीयेऽथ मासे पाण्डवनन्दन}


\twolineshloka
{ऊर्ध्वबाहुश्चतुर्थं तु मासमस्मि स्थितस्तदा}
{न च मे हीयते प्राणस्तदद्भुतमिवाभवत्}


\twolineshloka
{पञ्चमे त्वथ संप्राप्ते प्रथमे दिवसे गते}
{वराहसंस्थितं भूतं मत्समीपं समागमत्}


\twolineshloka
{निघ्नन्प्रोथेन पृथिवीं विलिखंश्चरणैरपि}
{संमार्जञ्जठरेणोर्वीं विवर्तंश्च मुहुर्मुहुः}


\twolineshloka
{अनु तस्यापरं भूतं महत्कैरातसंस्थितम्}
{धनुर्बाणासिमत्प्राप्तं स्त्रीगणानुगतं तदा}


\twolineshloka
{ततोऽहं धनुरादाय तथाऽक्षय्ये महेषुधी}
{अताडयं शरेणाथ तद्भूतं रोमहर्षणम्}


\twolineshloka
{युपत्तं किरातस्तु विकृष्य बलवद्धनुः}
{अभ्याजघ्ने दृढतरं कम्पयन्निव मेदिनीम्}


\twolineshloka
{सतु मामब्रवीद्राजन्मम पूर्वपरिग्रहः}
{मृगयाधर्ममुत्सृज्य किमर्थं ताडितस्त्वया}


\twolineshloka
{एष ते निशितैर्बाणैर्दर्पं हन्मि स्थिरो भव}
{संघर्षवान्महाकायस्ततो मामभ्यधावत}


\twolineshloka
{ततो गिरिमिवात्यर्थमावृणोन्मां महाशरैः}
{तं चाहं शरवर्षेण महता समवाकिरम्}


\twolineshloka
{ततः शरैर्दीप्तमुखैर्यन्त्रितैरनु यन्त्रितैः}
{प्रत्यविध्यमहं तं तु वज्रैरिव शिलोच्चयम्}


\twolineshloka
{तस्य तच्छतधा रूपमभवच्च सहस्रधा}
{तानि चास्य शरीराणि शरैरहमताडयम्}


\twolineshloka
{पुनस्तानि शरीराणि एकीभूतानि भारत}
{अदृश्यन्त महाराज तान्यहं व्यधमं पुनः}


\twolineshloka
{अणुर्बृहच्छिरा भूत्वा बृहच्चाणुशिराः पुनः}
{एकीभूतस्तदा राजन्सोऽभ्यवर्तत मां युधि}


\twolineshloka
{यदाऽभिभवितुं बाणैर्न च शक्नोमि तं रणे}
{ततो महास्त्रमातिष्ठं वायव्यं भरतर्षभ}


\twolineshloka
{न चैनमशकं हन्तुं तदद्भुतमिवाभवत्}
{तस्मिन्प्रतिहते चास्त्रे विस्मयो मे महानभूत्}


\twolineshloka
{तत्रापि च महाराज सविशेषमहं ततः}
{अस्त्रपूगेन महता रणे भूतमवाकिरम्}


\twolineshloka
{स्थूणाकर्णमथो जालं शरवर्षं शरोल्वणम्}
{शलभास्त्रमश्मवर्षं समास्थायाहमभ्ययाम्}


\twolineshloka
{जग्रास प्रसभं तानि सर्वाण्यस्त्राणि मे नृप}
{तेषु सर्वेषु जग्धेषु ब्रह्मास्त्रं महदादिशम्}


\twolineshloka
{ततः प्रज्वलितैर्बाणैः सर्वतश्चोपचीयत}
{उपचीयमानश्च तदा महास्त्रेण व्यवर्धत}


\twolineshloka
{ततः संतापिता लोका मत्प्रसूतेन तेजसा}
{क्षणएन हि दिशः स्वं च सर्वतो हि विदीपितं}


\twolineshloka
{तदप्यस्त्रं महातेजाः क्षणेनैव व्यशामयत्}
{ब्र्हमास्त्रे तु हते राजन्भयं मां महदाविशत्}


\twolineshloka
{ततोऽहं धनुरादाय तथाऽक्षय्ये महेषुधी}
{सहसाऽभ्यहनं भूतं तान्यप्यस्त्राण्यभक्षयत्}


\twolineshloka
{एतेष्वस्त्रेषु भूतेन भक्षितेष्वायुधेषु च}
{मम तस् च भूतस्य बाहुयुद्धमवर्तत}


\twolineshloka
{व्यायामं मुष्टिभिः कृत्वा तलैरभिसमाहतौ}
{अपायच्च तद्भूतमहं चापातयं महीम्}


\twolineshloka
{ततः प्रहस्य तद्भूतं तत्रैवान्तरधीयत}
{सह स्त्रीभिर्महाराज पश्यतो मेऽद्भुतोपमम्}


\twolineshloka
{मुख्यं कृत्वा स भगवांस्ततोऽन्यद्रूपमात्मनः}
{दिव्यमेव महाराज वरानोऽद्भुतमम्बरम्}


\twolineshloka
{हित्वा किरातरूपं च भगवांस्त्रिदशेश्वरः}
{स्वरूपं दिव्यमास्थाय तस्यौ महेश्वरः}


\twolineshloka
{सोऽदृश्यत ततः साक्षाद्भगवान्गोवृषध्वजः}
{धनुर्गृह्यतदा पाणौ बहुरूपः पिनाकधृत्}


\twolineshloka
{स मामभ्येत्य समरे तथैवाभिमुखं स्थितम्}
{शूलपाणिरथोवाच तुष्टोस्मीति परंतप}


\twolineshloka
{ततस्तद्धनुरादाय तूणौ चाक्षय्यसायकौ}
{प्रादान्ममैव भगवान्वरयस्वेति चाब्रवीत्}


\threelineshloka
{तुष्टोस्मि तव कौन्तेय ब्रूहि किं करवाणि ते}
{मनोगतं वीर यत्ते तद्ब्रूहि वितराम्यहम्}
{अमरत्वमपाहाय ब्रूहि यत्ते मनोगतम्}


\twolineshloka
{ततः प्राञ्डलिरेवाहमस्त्रेषु कृतमानसः}
{प्रणम्य शिरसा शर्वं ततो वचनमाददे}


\twolineshloka
{भगवानमे प्रसन्नश्चेदीप्सितोऽयं वरो मम}
{अस्त्राणीच्छाम्यहं ज्ञातुं यानि देवेषु कानिचित्}


\twolineshloka
{ददानीत्येव भगवानब्रवीत्र्यम्बकश्च माम्}
{रौद्रमस्त्रं मदीयं त्वामुपस्थास्यति पाण्डव}


\twolineshloka
{प्रददौ च मम प्रीतः सोऽस्त्रं पाशुपतं प्रभुः}
{उवाच च महादेवो दत्त्वा मेऽस्त्रं सनातनम्}


\twolineshloka
{न प्रयोज्यंभवेदेतन्मानुषेषु कथंचन}
{[जगद्विनिर्दहेदेवमल्पतेजसि पातितम्}


\twolineshloka
{पीड्यमानन बलवत्प्रयोज्यंस्याद्धनंजय}
{अस्त्राणां प्रतिघाते च सर्वथैव प्रयोजयेः}


\twolineshloka
{ततोऽप्रतिहतं दिव्यं सर्वास्त्रप्रतिषेधनम्}
{मूर्तिमन्मे स्थितं पार्श्वे प्रसन्ने गोवृषध्वजे}


\twolineshloka
{उत्सादनममित्राणां परसेनानिकर्तनम्}
{दुरासदं दुष्प्रसहं सुरदानवराक्षसैः}


\twolineshloka
{अनुज्ञातस्त्वहं तेन तत्रैव समुपाविशम्}
{प्रेक्षतश्चैवमे देवस्तत्रैवान्तरधीयत}


\chapter{अध्यायः १६९}
\twolineshloka
{अर्जुन उवाच}
{}


\twolineshloka
{ततस्तामवसं प्रीतो रजनीं तत्र भारत}
{प्रसादाद्देवदेवस्यत्र्यम्बकस्य महात्मनः}


\twolineshloka
{व्युषितो रजनीं चाहंकृत्वापौर्वाह्णिकीः क्रियाः}
{अपश्यं तं द्विजश्रेष्ठं दृष्टवानस्मि यं पुरा}


\twolineshloka
{तस्मै चाहं यथावृत्तं सर्वमेव न्यवेदयम्}
{भगवन्तं महादेवं समेतोस्मीति भारत}


\twolineshloka
{स मामुवाच राजेन्द्र प्रीयमाणो द्विजोत्तमः}
{दृष्टस्त्वया महादेवो यथा नान्येन केनचित्}


\twolineshloka
{समेतं लोकपालैस्तु सर्वैर्वैवस्वतादिभिः}
{द्रष्टास्यनघ देवेन्द्रं स च तेऽस्त्राणि दास्यति}


\twolineshloka
{एवमुक्त्वा स मां राजन्नाश्लिष्य च पुनः पुनः}
{अगच्छत्स यथाकामं ब्राह्मणः सूर्यसन्निभः}


\twolineshloka
{अथापराह्णे तस्याह्नः प्रावात्पुण्यः समीरणः}
{पुनर्नवमिमं लोकं कुर्वन्निव सपत्नहन्}


\twolineshloka
{दिव्यानि चैव माल्यानि सुगन्धीनि नवानि च}
{शैशिरस्य गिरे पादे प्रादुरासन्समीपतः}


\twolineshloka
{वादित्राणिच दिव्यानि सुघोषाणि समन्ततः}
{स्तुतयश्चेन्द्रसंयुक्ता अश्रूयन्त मनोहराः}


\twolineshloka
{गणाश्चाप्सरसां तत्रगन्धर्वाणां तथैव च}
{पुरस्ताद्देवदेवस्य जगुर्गीतानि सर्वशः}


\twolineshloka
{मरुतां च गणास्तत्र देवयानैरुपागमन्}
{महेन्द्रानुचरा ये च देवसद्मनिवासिनः}


\twolineshloka
{ततो मरुत्वानहरिभिर्युक्तैर्वाहैः स्वलंकृतैः}
{शचीसहायस्तत्रायात्सह सर्वैस्तदाऽमरैः}


\twolineshloka
{एतस्मिन्नैव काले तु कुबेरो नरवाहनः}
{दर्शयामास मां राजँल्लक्ष्म्या परमया युतः}


\twolineshloka
{दक्षिणस्यां दिशि यमं प्रत्यपश्यं व्यस्थितम्}
{वरुणं देवराजं च यथास्थानमवस्थितम्}


\twolineshloka
{ते मामूचुर्महाराज सान्त्वयित्वा सुरर्षभाः}
{सव्यसाचिन्निरीक्षास्माँल्लोकपालानवस्थितान्}


\twolineshloka
{सुरकार्यार्थसिद्ध्यर्थं दृष्टवानसि शंकरम्}
{अस्मत्तोऽपि गृहाण त्वमस्त्राणीति समन्ततः}


\twolineshloka
{ततोऽहं प्रयतो भूत्वा प्रणिपत्य सुरर्षभान्}
{प्रत्यगृह्णां तदाऽस्त्राणि महान्ति विविधानि च}


\twolineshloka
{गृहीतास्त्रस्ततो देवैरनुज्ञातोस्मि भारत}
{अथ देवा ययुः सर्वेयथागतमरिंदम}


\twolineshloka
{मघवानपि मां देवो रथमारोप्य सुप्रभम्}
{उवाच भगवान्वाक्यं स्मयन्निव महायशाः}


\twolineshloka
{पुरैवागमनादस्माद्वेदाहं त्वां धनंजय}
{अतः परं त्वहं वै त्वां दर्शये भरत्रषभ}


\twolineshloka
{त्वया हि तीर्थेषु पुरा समाप्लावः कृतोऽसकृत्}
{तपश्चेदं महत्त्प्तं स्वर्गं गन्तासि पाण्डव}


\twolineshloka
{भूयश्चैव च तप्तव्यं तपश्चरणमुत्तमम्}
{`दुश्चरं घोरमस्त्राणां तपोबलकरं तव'}


\twolineshloka
{स्वर्गस्त्ववश्यं गन्तव्यस्त्वया शत्रुनिषूदन}
{मातलिर्मन्नियोगात्त्वां त्रिदिवं प्रापयिष्यति}


\twolineshloka
{विदितस्त्वंहि देवानां मुनीनां च महात्मनाम्}
{`इहस्थः पाण्डवश्रेष्ठ तपः कुर्वन्सुदुष्करम्'}


\threelineshloka
{ततोऽहमब्रुवं शक्रं प्रसीद भगवन्मम}
{आचार्यं वरयेऽहं त्वामस्त्रार्थं त्रिदशेश्वर ॥इन्द्र उवाच}
{}


\twolineshloka
{क्रूरकर्माऽस्त्रवित्तात भविष्यसि परंतप}
{यदर्थमस्त्राणीप्सुस्त्वं तं कामं पाण्डवाप्नुहि}


\twolineshloka
{ततोऽहमब्रुवं नाहं दिव्यान्यस्त्राणि शत्रुहन्}
{मानुषेषु प्रयोक्ष्यामि विनाऽस्त्रप्रतिघातनात्}


\threelineshloka
{तानि दिव्यानि मेऽस्त्राणि प्रयच्छ विबुधाधिप}
{लेकांश्चास्त्रजितान्पश्चाल्लभेयं सुरपुङ्गव ॥इन्द्र उवाच}
{}


\twolineshloka
{परीक्षार्थं मयैतत्ते वाक्यमुक्तं धनंजय}
{ममात्मजस्य वचनं सूपपन्नमिदं तव}


\twolineshloka
{शिक्ष मे भवनं गत्वासर्वाण्यस्त्राणि भारत}
{वायोरग्नेर्वसुभ्योऽपि वरुणात्समरुद्गणात्}


\twolineshloka
{साध्यं पैतामहं चैव गन्धर्वोरगरक्षसाम्}
{वैष्णवानि च सर्वाणि नैर्ऋतानि तथैव च}


\twolineshloka
{मद्गतानि च जानीहि सर्वास्त्राणि कुरूद्वह}
{एवमुक्त्वा तु मां शक्रस्तत्रैवान्तरधीयत}


\twolineshloka
{अथापश्यं हरियुजं रथमैन्द्रमुपस्थितम्}
{दिव्यं मायामयं पुण्यं यत्तं मातलिना नृप}


\twolineshloka
{लोकपालेषु यातेषु मामुवाचाथ मातलिः}
{द्रष्टुमिच्छति शक्रस्त्वां देवराजो महाद्युते}


\twolineshloka
{संसिद्धस्त्वं महाबाहो कुरु कार्यमनुत्तमम्}
{पश्य पुण्यकृतां लोकान्सशरीरो दिवं व्रज}


\twolineshloka
{देवराजः सहस्राक्षस्त्वां दिदृक्षति भारत}
{इत्युक्तोऽहं मातलिना गिरिमामन्त्र्य शैशिरम्}


\twolineshloka
{प्रदक्षिणमुपावृत्य समारोहं रथोत्तमम्}
{चोदयामास स हयान्मनोमारुतरंहसः}


\threelineshloka
{[मातलिर्हयतत्त्वज्ञो यथावद्भूरिदक्षिणः}
{अवैक्षत च मे वक्रृंस्तितस्याथ ससारथिः}
{तथा भ्रान्ते रथे राजन्विस्मितश्चेदमब्रवीत्}


\twolineshloka
{अत्यद्भुतमिदं त्वद्यविचित्रं प्रतिभाति मे}
{यदास्थितो रथं दिव्यं पदान्न चलितः पदम् ॥]}


\twolineshloka
{देवराजोऽपिहि मया नित्यमत्रोपलक्षितः}
{विचलन्प्रथमोत्पाते हयानां भरतर्षभ}


\twolineshloka
{त्वं पुनः स्थित एवात्ररथे भ्रान्ते कुरूद्वह}
{अतिशक्रमिदं सर्वं तवेति प्रतिभाति मे}


\twolineshloka
{इत्युक्त्वाऽऽकाशमाविश्य मातलिर्विबुधालयान्}
{दर्शयामास मे राजन्विमानानि च भारत}


\twolineshloka
{[स रथो हरिभिर्युक्तो ह्यूर्ध्वमाचक्रमे ततः}
{ऋषयो देवताश्चैव पूजयन्ति नरोत्तम}


\twolineshloka
{ततः कामगमाँल्लोकानपश्यं वै सुरर्षिणाम्}
{गन्धर्वाप्सरसां चैव प्रभावममितौजसाम् ॥]}


\twolineshloka
{नन्दनादीनि देवानां वनान्युपवनानि च}
{दर्शयामास मे शीघ्रं मातलिः शक्रसारथिः}


\twolineshloka
{ततः शक्रस्य भवनमपश्यभमरावतीम्}
{दिव्यैः कामफलैर्वृक्षै रत्नैश्च समलंकृताम्}


\twolineshloka
{न तां भासयते सूर्यो न शीतोष्णे न च क्लमः}
{न बाधते तत्ररजस्तत्रास्ति न जरा नृप}


\threelineshloka
{न तत्रशोको दैन्यं वा वैवर्ण्यं चोपलक्ष्यते}
{दिवौकसां महाराज न ग्लानिररिमर्दन}
{न क्रोधलोभौ तत्रास्तामशुबं वा विशांपते}


\twolineshloka
{नित्यं तुष्टाश् ते राजन्प्राणिनः सुरवेश्मनि}
{नित्यपुष्पफलास्तत्र पादपा हरितच्छदाः}


\twolineshloka
{पुष्करिण्यश्च विविधाः पद्मसौगन्धिकायुताः}
{शीतस्तत्रववौ वायुः सुगन्धो वीजते शुभः}


\threelineshloka
{सर्वरत्नविचित्रा च भूमिः पुष्पविभूषिता}
{मृगद्विजाश्च बहवो रुचिरा मधुरस्वराः}
{विमानगामिनश्चात्रदृश्यन्ते बहवोऽमराः}


\twolineshloka
{ततोऽपश्यं वसून्रुद्रान्साध्यांश्च समरुद्गणान्}
{आदित्यानश्विनौ चैव तान्सर्वान्प्रत्यपूजयम्}


\twolineshloka
{ते मां वीर्येण यशसा तेजसा च बलेन च}
{अस्त्रैश्चाप्यन्वजानन्त संग्रामे विजयेन च}


\twolineshloka
{प्रविश्य तां पुरीं दिव्यां देवगन्धर्वपूजिताम्}
{देवराजं सहस्राक्षमुपातिष्ठं कृताञ्जलिः}


\twolineshloka
{ददावर्धासनं प्रीतः शक्रो मे ददतांवरः}
{बहुमानाच्च गात्राणि पस्पर्श मम वासवः}


\twolineshloka
{तत्राहं देवगन्धर्वैः सहितो भूरिदक्षिणैः}
{अस्त्रार्तमवसं स्वर्गे शिक्षाणोऽस्त्राणि भारत}


\twolineshloka
{विश्वावसोश्च वै पुत्रश्चित्रसेनोऽभवत्सखा}
{स च गान्धर्वमखिलं ग्राहयामास मां नृप}


\twolineshloka
{तत्राहमवसं राजन्गृहीतास्त्रः सुपूजितः}
{सुखं शक्रस् भवने सर्वकामसमन्वितः}


\twolineshloka
{शृण्वन्वै गीतशब्दं च तूर्यशब्दं च पुष्कलम्}
{पश्यंश्चाप्सरसः श्रेष्ठा नृत्यन्तीर्भरतर्षभ}


\twolineshloka
{तत्सर्वमनवज्ञाय तथ्यं विज्ञाय भारत}
{अत्यर्थं प्रतिगृह्याहमस्त्रेष्वेव व्यवस्थितः}


\twolineshloka
{ततोऽतुष्यत्सहस्राक्षस्तेन कामेन मे विभुः}
{एवं मे वसतो राजन्नेष कालोऽत्यगाद्दिवि}


\chapter{अध्यायः १७०}
\twolineshloka
{अर्जुन उवाच}
{}


\twolineshloka
{कृतास्त्रमतिविश्वस्तमथ मां हरिवाहनः}
{रसंस्पृश्य मूर्ध्नि पाणिब्यामिदं वचनमब्रवीत्}


\twolineshloka
{न त्वमद्य युधा जेतुं शक्यः सुरगणैरपि}
{किं पुनर्मानुषे लोके मानुषैरकृतात्मभिः}


\twolineshloka
{अप्रमेयोऽप्रधृष्यश्च युद्धेष्वप्रतिमस्तथा}
{`अजेयस्त्वं हि सङ्ग्रामे सर्वैरपिसुरासुरैः'}


\twolineshloka
{अथाब्रवीत्पुनर्देवः संप्रहृष्टतनूरुहः}
{अस्त्रयुद्धे समो वीर न ते कश्चिद्भविष्यति}


\twolineshloka
{अप्रमत्तः सदा दक्षः सत्यवादी जितेन्द्रियः}
{ब्रह्मण्यश्चास्त्रविच्चासि शूरश्चासि कुरूद्वह}


\twolineshloka
{अस्त्राणि समवाप्तानि त्वया दश च पञ्च च}
{पञ्चभिर्विधिभिः पार्थ विद्यते न त्वया समः}


\twolineshloka
{प्रयोगमुपसंहारमावृत्तिं च धनञ्जय}
{प्रायश्चित्तं च वेत्थ त्वं प्रतीघातं च सर्वशः}


\twolineshloka
{तव गुर्वर्थकालोऽयं समुत्पन्नः परंतप}
{प्रतिजानीष्व तं कर्तुं ततो वेत्स्याम्यहं परम्}


\twolineshloka
{ततोऽहमब्रवं राजन्देवराजमिदं वचः}
{विषह्यं यन्मया कर्तुं कृतमेव निबोध तत्}


\twolineshloka
{ततो मामब्रवीद्राजन्प्रहसन्बलवृत्रहा}
{नाविपह्यं तवाद्यास्ति त्रिषु लोकेषु किंचन}


\twolineshloka
{निवातकवचा नाम दानवा मम शत्रवः}
{समुद्रकुक्षिमाश्रित्य दुर्गे प्रतिवसन्त्युत}


\twolineshloka
{तिस्रः कोट्यः समाख्यातास्तुल्यरूपबलप्रभाः}
{तांस्तत्रजहि कौन्तेय गुर्वर्थस्ते भविष्यति}


\twolineshloka
{ततो मातलिसंयुक्तं मयूरसमरोमभिः}
{हयैरुपेतं प्रादान्मे रथं दिव्यं महाप्रभम्}


\twolineshloka
{बबन्ध चैव मे मूर्ध्नि किरीटमिदमुत्तमम्}
{स्वरूपसदृशं चैव प्रादादङ्गविभूषणम्}


\twolineshloka
{अभेद्यं कवचं चेदं स्पर्सरूपवदुत्तमम्}
{अजरां ज्यामिमां चापि गाण्डीवे समयोजयत्}


\twolineshloka
{ततः प्रायामहं तेन स्यन्दनेन विराजता}
{येनाजयद्देवपतिर्बलिं वैरोचनिं पुरा}


\twolineshloka
{ततो देवाः सर्व एव तेन घोषेण बोधिताः}
{मन्वाना देवराजं मां समाजग्मुर्विशांपते}


\twolineshloka
{दृष्ट्वा च मामपृच्छन्त किं करिष्यसि फल्गुन}
{तानब्रुवं यथाभूतमिदं कर्ताऽस्मि संयुगे}


\twolineshloka
{निवातकवचानां तु प्रस्थितं मां तधैपिणम्}
{निबोधत महाभागा शिवं चाशास्त मेऽनघाः}


\twolineshloka
{`ततो वाग्भिः प्रशस्ताभिस्त्रिदशाः पृथिवीपते'}
{तुष्टुवुर्मां प्रसन्नास्ते यथा देवं पुरंदरम्}


\twolineshloka
{रथेनानेन मघवा जितवाञ्शम्बरं युधि}
{नमुचिं बलवृत्रौ च प्रह्लादनरकावपि}


\twolineshloka
{बहूनि च सहस्राणि प्रयुतान्यर्बुदान्यपि}
{रथेनानेन दैत्यानां जितवान्मघवा युधि}


\twolineshloka
{न्वमप्यनेन कौन्तेय निबातकवचान्रणे}
{विजेता युधि विक्रम्य पुरेव मघवा वशी}


\twolineshloka
{अयंच शङ्खप्रवरो येन जेतासि दानवान्}
{अनेन विजिता लोका शक्रेणापि महात्मना}


\twolineshloka
{प्रदीयमानं देवैस्तं देवदत्तं जलोद्भवम्}
{प्रत्यगृह्णां जयायैनं स्तूयमानस्तदाऽमरैः}


\twolineshloka
{स शङ्खी कवची वाणी प्रगृहीतशरासनः}
{दानवालयमत्युग्रं प्रयातोस्मि युयुत्सया}


\chapter{अध्यायः १७१}
\twolineshloka
{अर्जुन उवाच}
{}


\twolineshloka
{ततोऽहं स्तूयमानस्तु तत्रतत्र सुरपिभिः}
{अपश्यमुदधिं भीममपांपतिमथाव्ययम्}


\twolineshloka
{फेनवत्यः प्रकीर्णाश्च संहताश्च समुच्छिताः}
{ऊर्मयश्चात्र दृश्यन्ते चलन्तइव पर्वताः}


\twolineshloka
{नावः सहस्रशस्तत्र रत्नपूर्णाः समन्ततः}
{`नभसीव विमानानि विचरन्त्यो विरेजिरे'}


\twolineshloka
{तिमिङ्गिलाः कच्छपाश् तथा तिमितिभिङ्गिलाः}
{मकराश्चात्र दृश्यन्ते जले मप्रा इवाद्रयः}


\twolineshloka
{शङ्खानां च महबाणि मग्रान्यप्सु ममन्ततः}
{दृश्यन्ते स्म यथा रात्रौ तारास्तन्वभ्रमंवृताः}


\twolineshloka
{तथा सहस्रशस्तत्ररत्नसङ्घाः प्लवन्त्युत}
{वायुश् घृर्णते भीमस्तदद्भुतमिवाभवत्}


\twolineshloka
{तमतीत्य महावेगं सर्वाम्भोनिधिमुत्तमम्}
{अपश्यं दानवाकीर्णं तद्दैत्यपुरमन्तिकात्}


\twolineshloka
{तत्रैव मातलिस्तृणं निपात्य पृथिवीतले}
{दानवान्रथघोपेण तन्पुरं समुपाद्रवत्}


\twolineshloka
{रथघोषं तु तं श्रुत्वा स्तनयित्नोग्विम्बरे}
{मन्वाना देवराजं मामाविग्ना दानवाऽभवन्}


\twolineshloka
{सर्वे संभ्रान्तमनसः शग्चापधराः स्थिताः}
{तथाऽसिशूलपरशुगदामुसलपाणयः}


\twolineshloka
{ततो द्वाराणि पिदधुर्दानवास्त्रस्तचेतसः}
{संविधाय पुरे रक्षां न स्म कश्चन दृश्यते}


\twolineshloka
{ततः शङ्खमुपादाय देवदत्तं महास्वनम्}
{परमां मुदमाश्रित्य प्राधमं तं शनैरहम्}


\twolineshloka
{स तु शब्दो दिवं स्तब्ध्वा प्रतिशब्दमजीजनत्}
{वित्रेसुश्च निलिल्युश्च भूतानि सुमहान्त्यपि}


\twolineshloka
{ततो निवातकवचाः सर्व एव समन्ततः}
{दंशिता विविधैस्त्राणैर्विचित्रायुपाणयः}


\twolineshloka
{आयसैश्र महाशूलैर्गदाभिर्मुसलैरपि}
{पट्टसैः करवालैश्च रथचक्रैश्च भारत}


\twolineshloka
{शतघ्नीभिर्भुशुण्डीभिः खङ्गैश्चित्रैः स्वलंकृतैः}
{प्रगृहीतैर्दितेः पुत्राः प्रादुरासन्सहस्रशः}


\twolineshloka
{ततो विचार्य बहुशो रथमार्गेषु तान्हयान्}
{प्राचोदयत्समे देशे मातलिर्भरतर्षभ}


\twolineshloka
{तेन तेषां प्रणुन्नानासाशुत्वाच्छीघ्रगामिनाम्}
{नान्वपश्यत्तदा कश्रित्तन्मेऽद्भुतमिवाभवत्}


\twolineshloka
{ततस्ते दानवास्तत्रयोधवातान्यनेकशः}
{विकृतस्वररूपाणि भृशं सर्वाण्यचोदयन्}


\twolineshloka
{तेन शब्देन सहसा समुद्रे पर्वतोपमाः}
{आप्लवन्त गतैः सत्वैर्मत्स्याः शतसहस्रशः}


\chapter{अध्यायः १७२}
\twolineshloka
{अर्जुन उवाच}
{}


\twolineshloka
{ततो निवातकवचाः सर्वेवेगेन भारत}
{अभ्यद्रवन्मां सहिताः प्रगृहीतायुधा रणे}


\twolineshloka
{आच्छाद्य रथपन्थानमुत्क्रोशन्तो महारथाः}
{अवृत्य सर्वतस्ते मां शरवर्षैरवाकिरन्}


\twolineshloka
{ततोऽपरे महावीर्याः शूलपट्टसपाणयः}
{शूलानि च भुशुण्डीश्च मुमुचुर्दानवा मयि}


\twolineshloka
{सुमहत्तुमुलं वर्षं गदाशक्तिसमाकुलम्}
{अनिशं सृज्यमानं तैरपतन्मद्रथोपरि}


\twolineshloka
{अन्ये मामभ्यधावन्त निवातकवचा सुधि}
{शितशस्त्रायुधा रौद्राः कालरूपाः प्रहारिणः}


\twolineshloka
{तानहं विविधैर्बाणैर्वेगवद्भिरजिह्मगैः}
{गाण्डीवमुक्तैरभ्यघ्नमेकैकं दशभिर्मृधे}


\twolineshloka
{ते कृता विमुखाः सर्वे मत्प्रयुक्तैः शिलाशितैः}
{ततो मातलिना तूर्णं हयास्ते संप्रयोदिताः}


\twolineshloka
{अथ मार्गान्बहूस्तत्र विचेरुर्वातरंहसः}
{सुसंयता मातलिना प्रामथ्नन्त दितेःसुतान्}


\twolineshloka
{शतं शतास्ते हरयस्तस्मिन्युक्ता महारथे}
{तदा मातलिना यथ्ता व्यचरन्नल्पका इव}


\twolineshloka
{तेषां चरणपातेन रथनेमिस्वनेन च}
{मम बाणनिपातैश्च हतास्ते शतशोऽसुराः}


\twolineshloka
{गतासवस्तथैवान्ये दानवाः पाण्डवर्षभ}
{हतसारथयस्तत्र व्यकृष्यन्त तुरंगमैः}


\twolineshloka
{ते दिशो विदिशः सर्वे प्रतिरुध्य प्रहारिणः}
{अभ्यघ्नन्विविधैः शस्त्रैस्ततो मे व्यथितं मनः}


\twolineshloka
{ततोऽहं मातलेर्वीर्यमद्भुतं समदर्शयम्}
{अश्वांस्तथा वेगवतो यदयत्नादधारयत्}


\twolineshloka
{ततोऽहं लघुभिश्चित्रैरस्त्रैस्तानसुरान्रणे}
{चिच्छेद सायुधान्राजञ्छतशोऽथ सहस्रशः}


\twolineshloka
{एवं मे चरतस्तत्र सर्वयत्नेन शत्रुहन्}
{प्रीतिमानभवद्वीरो मातलिः शक्रसारथिः}


\twolineshloka
{वध्यमानास्ततस्तैस्तु हयैस्तेन रथेन च}
{अगमन्प्रक्षयं केचिन्न्यवर्तन्त तथाऽपरे}


\twolineshloka
{स्पर्धमाना इवास्माभिर्निवातकवचा रणे}
{शरवर्षैः शरार्तं मां महद्भिः प्रत्यवारयन्}


\twolineshloka
{शरवेगैर्निहत्याहमस्त्रैः शरविघातिभिः}
{ज्वलद्भिः परमैः शीघ्रैस्तानविध्यं सहस्रशः}


\twolineshloka
{ततः संपीड्यमानास्ते क्रोधाविष्टा महासुराः}
{अपीडयन्मां सहिताः शक्तिशूलासिवृष्टिभिः}


\twolineshloka
{ततोऽहमस्त्रं प्रायुञ्जं गान्धर्वं नाम भारत}
{दयितं देवराजस् माधवं नाम भारत}


\twolineshloka
{ततः खङ्गांस्त्रिशूलांश्च तोमरांशच् सहस्रशः}
{अस्त्रवीर्येण शतधा तैर्मुक्तानहमच्छिदम्}


\twolineshloka
{छित्त्वा प्रहरणान्येषां ततस्तानपि सर्वशः}
{प्रत्यविध्यमहं रोषाद्दशभिर्दशभिः शरैः}


\twolineshloka
{गाण्डीवाद्धि तदा सङ्ख्ये यथा भ्रमरपङ्क्तयः}
{निष्पतन्ति महाबाणास्तन्मातलिरपूजयत्}


\twolineshloka
{तेषामपि तु बाणास्ते बहुत्वाच्छलभा इव}
{अवाकिरन्मां बलवत्तानहं व्यधमं शरैः}


\twolineshloka
{वध्यमानास्ततस्ते तु निवातकवचाः पुनः}
{शरवर्षैर्महद्भिर्मां समन्तात्पर्यवारयन्}


\twolineshloka
{शरवेगान्निहत्याहमस्त्रैरस्त्रविघातिभिः}
{ज्वलद्भिः परमैः शीघ्रैस्तानविध्यं सहस्रशः}


\twolineshloka
{तेषां छिन्नानि गात्राणि विसृजन्तिस्म शोणितम्}
{प्रावृषीवाभिवृष्टानि शृङ्गाण्यथ धराभृताम्}


\twolineshloka
{इन्द्राशनिसमस्पर्शैर्वेगवद्भिरजिह्मगैः}
{मद्वाणैर्वध्यमानास्ते समुद्विग्राः स्म दानवाः}


\twolineshloka
{शतधा भिन्नदेहास्ते क्षीणप्रहरणौजसः}
{ततो निवातकवचा मामयुध्यन्त मायया}


\chapter{अध्यायः १७३}
\twolineshloka
{अर्जुन उवाच}
{}


\twolineshloka
{ततोऽश्मवर्षं सुमहत्प्रादुरासीत्समन्ततः}
{नगमात्रैः शिलाखण्डैस्तन्मां दृढमपीडयत्}


\twolineshloka
{तदहं वज्रसंकाशैः शरैरिनद्रास्त्रचोदितैः}
{अचूर्णयं वेगवद्भिः शतधैकैकमाहवे}


\twolineshloka
{चूर्ण्यमानेऽश्मवर्षे तु यावकः समजायत}
{तत्राश्मचूर्णा न्यपतन्पावकप्रकरा इव}


\twolineshloka
{ततोऽश्मवर्षे विहते जलवर्षं महत्तरम्}
{धाराभिरक्षमात्राभिः प्रादुरासीनममान्तिके}


\twolineshloka
{नभस प्रच्युता धारास्तिग्मवीर्याः सहस्रशः}
{आवृण्वन्सर्वतो व्योम दिशश्चोपदिशस्तथा}


\twolineshloka
{धाराणां च निपातेन वायोर्विष्फूर्जितेन च}
{गर्जितेन च मेघानां न प्राज्ञायत किंचन}


\twolineshloka
{धारा दिवि च संबद्धा वसुधायां च सर्वशः}
{व्यामोहयन्त मां तत्रनिपतन्त्योऽनिशं भुवि}


\twolineshloka
{तत्रोपदिष्टमिन्द्रेण दिव्यमस्त्रं विशोषणम्}
{दीप्तं प्राहिणवं घोरमशुप्यत्तेन तज्जलम्}


\twolineshloka
{हतेऽश्मवर्षे च मया जलवर्षे च शोषिते}
{मुमुचुर्दानवा मायामग्निं वायुं च भारत}


\twolineshloka
{ततोऽहमग्निं व्यधमं सलिलास्त्रेण सर्वशः}
{शैलेन च महास्त्रेण वायोरवेगमधारयम्}


\twolineshloka
{तस्यां प्रतिहतायां ते दानवा युद्धदुर्मदाः}
{प्राकुर्वन्विविधां मायां यौगपद्येन भारत}


\twolineshloka
{ततो वर्षं प्रादुरभूत्सुमहद्रोमहर्षणम्}
{अस्त्राणां घोररूपाणामग्नेर्वायोस्तथाऽश्मनाम्}


\twolineshloka
{सा तु मायामयी वृष्टिः पीडयामास मां युधि}
{अथ घोरं तमस्तीव्रं प्रादुरासीत्समन्ततः}


\twolineshloka
{तमसा संवृतेलोके घोरेण परुषेण च}
{हरयो विमुखाश्चासन्प्रास्खलच्चापि मातलिः}


\twolineshloka
{हस्ताद्धिरण्मयश्चास्य प्रतोदः प्रापतद्भुवि}
{असकृच्चाह मां भीतः किं करिष्याव इत्यपि}


\twolineshloka
{मां च भीराविशत्तीव्रा तस्मिन्विगतचेतसि}
{स च मां विगतज्ञानः संत्रस्तमिदमब्रवीत्}


\twolineshloka
{सुराणामसुराणां च संग्रामः सुमहानभूत्}
{अमृतार्थं पुरा पार्त स च दृष्टो मयाऽनघ}


\twolineshloka
{शम्बरस्य वधे घोरः संग्रामः सुमहानभूत्}
{सारथ्यं देवराजस्य तत्रापि कृतवानहम्}


\twolineshloka
{तथैव वृत्रस्य वधे संगृहीता हया मया}
{वैरोचनेर्मया युद्धं दृष्टं चापि सुदारुणम्}


\twolineshloka
{एते मया महाघोरः संग्रामाः पर्युपासिताः}
{न चापि विगतज्ञानो भूतपूर्वोस्मि पाण्डव}


\twolineshloka
{पितामहेन संहारः प्रजानां विहितो ध्रुवम्}
{न हि युद्धमिदं युक्तमन्यत्र जगतः क्षयात्}


\twolineshloka
{तस्य तद्वचनं श्रुत्वा संस्तभ्यात्मानमात्मना}
{मोहयिष्यन्दानवानामहं मायाबलं महत्}


\twolineshloka
{अब्रुवं मातलिं भीतं पश्य मे बुजयोर्बलम्}
{अस्त्राणां च प्रभावं वै धनुषो गाण्डिवस्य च}


\twolineshloka
{अद्यास्त्रमाययैतैषां मायामेतां सुदारुणाम्}
{विनिहन्मि तमश्चोग्रं मा भैः सूत स्तिरो भव}


\twolineshloka
{एवमुक्त्वाहमसृजमस्त्रमायां नराधिप}
{मोहनीं सर्वभूतानां हिताय त्रिदिवौकसाम्}


\twolineshloka
{पीड्यमानासु मायासु तासु तास्वसुरोत्तमाः}
{पुनर्बहुविधा मायाः प्राकुर्वन्नमितौजसः}


\twolineshloka
{पुनः प्रकाशमभवत्तमसा ग्रस्यते पुनः}
{भवत्यदर्शनो लोकः पुनरप्सु निमज्जति}


\twolineshloka
{सुसंगृहीतैर्हरिभिः प्रकाशे सति मातलिः}
{व्यचरत्स्यन्दनाग्र्येण संग्रामे रोमहर्षणे}


\twolineshloka
{ततः पर्यपतन्नुग्रा निवातकवचा मयि}
{तानहं विवरं दृष्ट्वा प्राहण्वं यमसादनम्}


\twolineshloka
{वर्तमाने तथा युद्धे निवातकवचान्तके}
{नापश्यं सहसा सर्वान्दानवान्मायया वृतान्}


\chapter{अध्यायः १७४}
\twolineshloka
{अर्जुन उवाच}
{}


\twolineshloka
{अदृश्यमानास्ते दैत्या योधयन्ति स्म मायया}
{अदृश्यनास्त्रवीर्येण तानप्यहमयोधयम्}


\twolineshloka
{गाण्डीवमुक्ता विशिखाः सम्यगस्त्रप्रचोदिताः}
{अच्छिन्दन्नुत्तमाङ्गानि यत्रयत्र स्म तेऽभवन्}


\twolineshloka
{ततो निवातकवचा वध्यमाना मया युधि}
{संहृत्य रमायां सहसा प्राविशन्पुरमात्मनः}


\twolineshloka
{व्यपयातेषु दैत्येषु प्रादुर्भूते च दर्शने}
{अपश्यं दानवांस्तत्र हताञ्शतसहस्रशः}


\twolineshloka
{विनिष्पिष्टानि तत्रैषां शस्त्राण्याभरणानि च}
{शतशः स्म प्रदृश्यन्ते गात्राणि कवचानि च}


\twolineshloka
{हयानां नान्तरं ह्यासीत्पदाद्विचलितुं पदम्}
{उत्पत्य सहसा तस्थुरन्तरिक्षगमास्ततः}


\twolineshloka
{ततो निवातकवचा व्योम संछाद्य केवलम्}
{अदृश्या ह्यभ्यवर्तन्त विसृजन्तः शिलोच्चयान्}


\twolineshloka
{अन्तर्भूमिगताश्चान्ये हयानां चरणानथ}
{व्यगृह्णन्दानवा घोरा रथचक्रे च भारत}


\twolineshloka
{विनिगृह्य हयांश्चान्ये रथं च मम युध्यतः}
{सर्वतो मामविध्यन्त सरथं धरणीधरैः}


\twolineshloka
{पर्वतैरुपचीयद्भिः पतद्भिश्च तथाऽपरैः}
{स देशो यत्रवर्तामि गुहेव समपद्यत}


\twolineshloka
{पर्वतैश्चाद्यमानोऽहं निगृहीतैश्च वाजिभिः}
{अगच्छं परमामार्तिं मातलिस्तदलक्षयत्}


\twolineshloka
{लक्षयित्वा च मां भीतमिदं वचनमब्रवीत्}
{आर्जुनार्जुन मा भैस्त्वं वज्रमस्त्रमुदीरय}


\twolineshloka
{ततोऽहं तस्य तद्वाक्यं श्रुत्वा वज्रमुदीरयम्}
{देवराजस् दयितं वज्रमस्त्रं नराधिप}


\twolineshloka
{अचलं स्थानमासाद्य गाण्डीवमनुमन्त्र्य च}
{अमुञ्चं वज्रसंस्पर्शानायतान्निशिताञ्सरान्}


\twolineshloka
{ततो मायाश् ताः सर्वा निवातकवचांश्च तान्}
{ते वज्रचोदिता बाणा वज्रभूताः रसमाविशन्}


\twolineshloka
{ते वज्रवेगविहता दानवाः पर्वतोपमाः}
{इतरेतरमाश्लिष्य न्यपतन्पृथिवीतले}


\twolineshloka
{अन्तर्भूमौ च येऽगृह्णन्दानवा रथवाजिनः}
{अनुप्रविश्य तान्वाणाः प्राहिण्वन्यमसादनम्}


\twolineshloka
{हतैर्निवातकवचैर्निरस्तैः पर्वतोपमैः}
{समाच्छाद्यत देशः स विकीर्णैरिव पर्वतैः}


\twolineshloka
{न हयानां क्षतिः काचिन्न रथस् न मातलेः}
{मम चादृश्यत तदा तदद्भुतमिवाभवत्}


\twolineshloka
{ततो मां प्रहसन्राजन्मातलिः प्रत्यभाषत}
{नैतदर्जुन देवेषु त्वयि वीर्यं यदीक्ष्यते}


\twolineshloka
{हतेष्वसुरसङ्घेषु दारास्तेषां तु सर्वशः}
{प्राक्रोशन्नगरे तस्मिन्यथा शरदि लक्ष्मणाः}


\twolineshloka
{ततो मातलिना सार्धमहं तत्पुरमभ्ययाम्}
{त्रासयन्रथघोषेण निवातकवचस्त्रियः}


\twolineshloka
{तान्दृष्ट्वा दशसाहस्रान्मयूरसदृशान्हयान्}
{रथं च रविसंकाशं प्राद्रवन्गणशः स्त्रियः}


\twolineshloka
{ताभिराभरणैः शब्दस्त्रासिताभिः समीरितः}
{शिलानामिव शैलेषु पतन्तीनामभूत्तदा}


\twolineshloka
{वित्रस्ता दैत्यनार्यस्ताः स्वानि वेश्मान्यथाविशन्}
{बहुरत्नविचित्राणि शातकुम्भमयानि च}


\twolineshloka
{तदद्भुताकारमहं दृष्ट्वा नगरमुत्तमम्}
{विशिष्टं देवनगरादपृच्छं मातलिं ततः}


\threelineshloka
{इदमेवंविधं कस्माद्देवता न विशन्त्युत}
{पुरंदरपुराद्धीदं विशिष्टमिति लक्षये ॥मातलिरुवाच}
{}


\twolineshloka
{आसीदिदं पुरा पार्थ देवराजस्य नः पुरम्}
{ततो निवातकवचैरितः प्रच्याविताः सुराः}


\twolineshloka
{तपस्तप्त्वा महत्तीव्रं प्रसाद्य च पितामहम्}
{इदं वृतं निवासाय देवेभ्यश्चाभयं युधि}


\twolineshloka
{ततः शक्रेण भगवान्स्वयंभूरभिचोदितः}
{विधत्तां भगवानत्रेत्यात्मनो हितकाम्यया}


\twolineshloka
{तत उक्तो भगवता दिष्टमत्रेति भारत}
{भवितान्तस्त्वमप्येषां देहेनान्येन वृत्रहन्}


\twolineshloka
{तत एषां वधार्थाय शक्रोऽस्त्राणि ददौ तव}
{न हि शक्याः सुरैर्हन्तुं य एते निहतास्त्वया}


\twolineshloka
{कालस्य परिणामेन ततस्त्वमिह भारत}
{एषामन्तकरः प्राप्तस्तत्त्वया च कृतं तथा}


\threelineshloka
{दानवानां विनाशार्थं महास्त्राणां महद्बलम्}
{ग्राहितस्त्वं महेन्द्रेण पुरुषेन्द्र तदुत्तमम् ॥उर्जुन उवाच}
{}


\twolineshloka
{ततः प्रशाम्य नगरं दानवांश्च निहत्य तान्}
{पुनर्मातलिना सार्धमगमं देवसद्म तत्}


\chapter{अध्यायः १७५}
\twolineshloka
{अर्जुन उवाच}
{}


\twolineshloka
{निवर्तमानेन मया महद्दृष्टं ततोऽपरम्}
{पुरं कामगमं दिव्यं पावकार्कसमप्रभम्}


\twolineshloka
{रत्नद्रुममयैश्चित्रैर्भास्वरैश्च पतत्रिभिः}
{पौलोमैः कालकेयैश्च रनित्यहृष्टैरधिष्ठिरतम्}


\twolineshloka
{गोपुराट्टालकोपेतं चतुर्द्वारं दुरासदम्}
{सर्वरत्नमयं दिव्यमद्भुतोपमदर्शनम्}


\twolineshloka
{द्रुमैः पुष्पलोपेतैः सर्वरत्नमयैर्वृतम्}
{तथा पतत्रिभिर्दिव्यैरुपेतं सुमनोहरैः}


\twolineshloka
{असुरैर्नित्यमुदितैः शूलर्ष्टिमुसलायुधैः}
{चापमुद्गरहस्तैश्च स्रग्विभिः सर्वतो वृतम्}


\threelineshloka
{तदहं प्रेक्ष्यदैत्यानां पुरमद्भुतदर्शनम्}
{अपृच्छं मातलिं राजन्किमिदं दृश्यतेति वै ॥मातलिरुवाच}
{}


\twolineshloka
{पुलोमा नाम दैतेयी कालका च महासुरी}
{दिव्यंवर्षसहस्रं ते चेरतुः परमं तपः}


\twolineshloka
{तपसोन्ते ततस्ताभ्यां स्वयंभूरददाद्वरम्}
{अगृह्णीतां वरं ते तु सुतानामल्पदुःखताम्}


\twolineshloka
{अवध्यतां च राजेन्द्र सुरराक्षसपन्नगैः}
{पुरं सुरमणीयं च स्वचरं सुकृतप्रभम्}


\twolineshloka
{सर्वरत्नैः समुदितं दुर्धर्षममरैरपि}
{महर्षियक्षगन्धर्वपन्नगासुरराक्षसैः}


\twolineshloka
{सर्वकामगुणोपेतं वीतशोकमनामयम्}
{ब्र्हमणो भवनाच्छ्रेष्ठं कालकेयकृतं विभो}


\twolineshloka
{तदेतत्खपुरं दिव्यं चरत्यमरवर्जितम्}
{पौलोमाध्युषितं वीर कालकेयैश्च दानवैः}


\twolineshloka
{हिरण्यपुरमित्येवं ख्यायते नगरं महत्}
{रक्षितंकालकेयैश्च पौलोमैश्च महासुरैः}


\twolineshloka
{त एतेमुदिता राजन्नवध्याः सर्वदैवतैः}
{निवसन्त्यत्रराजेन्द्र गतोद्वेगा निरुत्सुकाः}


\twolineshloka
{`सुरासुरैरवध्यानां दानवानां धनञ्जय'}
{मानुषान्मृत्युरेतेषां निर्दिष्टो ब्रह्मणा पुरा}


\threelineshloka
{[एतानपि रणे पार्त कालकेयान्दुरासदान्}
{वज्रास्त्रेण नयस्वाशु विनाशं सुमहाबलान्] ॥अर्जुन उवाच}
{}


\twolineshloka
{सुरासुरैरवध्यं तदहं ज्ञात्वा विशांपते}
{अब्रुवं मातलिं हृष्टो याह्येतत्पुरमञ्जसा}


\twolineshloka
{त्रिदशेशद्विषो यावत्क्षयमस्त्रैर्नयाम्यहम्}
{न कथंचिद्धि मे पापा न वध्या ये सुरद्विषः}


\twolineshloka
{उवाह मां ततः शीघ्रं हिरण्यपुरमन्तिकात्}
{रथेन तेन दिव्येन हरियुक्तेन मातलिः}


\twolineshloka
{ते मामालक्ष्य दैतेया विचित्राभरणाम्बराः}
{समुत्पेतुर्महावेगा रथानास्थाय दंशिताः}


\twolineshloka
{ततो नालीकनाराचैर्भल्लैः शक्त्यृष्टितोमरैः}
{प्रत्यघ्नन्दानवेन्द्रा मां क्रुद्धास्तीव्रपराक्रमाः}


\twolineshloka
{तदहं शस्त्रवर्षेण महता प्रत्यवारयम्}
{शस्त्रवर्षं महद्राजन्विद्याबलमुपाश्रितः}


\twolineshloka
{व्यामोहयं च तान्सर्वान्रथमार्गैश्चरन्रणे}
{तेऽन्योन्यमभिसंमूढाः पातयन्ति स्म दानवाः}


\twolineshloka
{तेषामेवं विमूढानामन्योन्यमभिधावताम्}
{शिरांसि विशिखैर्दीप्तैरच्छिन्दं शतसङ्घशः}


\twolineshloka
{ते वध्यमाना दैतेयाः पुरमास्थाय तत्पुनः}
{खमुत्पेतुः सनगरा मायामास्थाय दानवीम्}


\twolineshloka
{ततोऽहं शरवर्षेण महता कुरुनन्दन}
{मार्गमावृत्य दैत्यानां गतिं चैषामवारयम्}


\twolineshloka
{तत्पुरं खचरं दिव्यं कामगं दिव्यवर्चसम्}
{दैतेयैर्वरदानेन धार्यते स्म यथासुखम्}


\twolineshloka
{अन्तर्भूमौ निपतति पुनरूर्ध्वं प्रतिष्ठते}
{पुनस्तिर्यक् प्रयात्याशु पुनरप्सु निमज्जति}


\twolineshloka
{अमरावतिसंकाशं तत्पुरं कामगं महत्}
{अहमस्त्रैर्बहुविधैः प्रत्यगृह्णां परंतप}


\twolineshloka
{ततोऽहं शरजालेन दिव्यास्त्रनुदितेन च}
{व्यगृह्णां सह दैतेयैस्तत्पुरं पुरुषर्षभ}


\twolineshloka
{विक्षत चायसैर्बाणैर्मत्प्रयुक्तैरजिह्मगैः}
{महीमभ्यपतद्राजन्प्रभग्नं पुरमासुरम्}


\twolineshloka
{ते वध्यमाना मद्बाणैर्वज्रवेगैरयस्मयैः}
{पर्यभ्रमन्त वै राजन्नसुराः कालचोदिताः}


\twolineshloka
{ततो मातलिरारुह्यपुरस्तान्निपतन्निव}
{महीमवातरत्क्षिप्रं रथेनादित्यवर्चसा}


\threelineshloka
{ततो रथसहस्राणि षष्टिस्तेषाममर्षिणाम्}
{युयुत्सूनां मया सार्धं पर्यवर्तन्त भारत}
{तान्यहं निशितैर्बाणैर्व्यधमं गार्ध्रराजितैः}


\twolineshloka
{ते युद्धे संन्यवर्तन्त समुद्रस्य यथोर्मयः}
{नेमे शक्या मानुषेण युद्धेनेति प्रचिन्त्य तत्}


% Check verse!
ततोऽहमानुपूर्व्येण दिव्यान्यस्त्राण्ययोजयम्
\twolineshloka
{ततस्तानि सहस्राणइ रथिनां चित्रयोधिनाम्}
{अस्त्राणि मम दिव्यानि प्रत्यघ्नञ्छनकैरिव}


\twolineshloka
{रथमार्गान्विचित्रांस्ते विचरन्तो महाबलाः}
{प्रत्यदृश्यन्त संग्रामे शतशोऽथ सहस्रशः}


\twolineshloka
{विचित्रमुकुटापीडा विचित्रकवचध्वजाः}
{विचित्राभरणाश्चैव मम प्रीतिकराऽभवन्}


\twolineshloka
{अहं तुशरवर्षैस्तानस्त्रप्रचुदितै रणे}
{नाशक्रुवं पीडयितुं ते तु मां प्रत्यपीडयन्}


\twolineshloka
{तैः पीड्यमानो बहुभिः कृतास्त्रैः कुशलैर्युधि}
{व्यथितोस्मि महायुद्धे भयं चागान्महन्मम}


% Check verse!
`ततोहं परमायत्तो मातलिं परिपृष्टवान्
\twolineshloka
{किमेते मम बाणौघैर्दिव्यास्त्रप्रतिमन्त्रितैः}
{न वध्यन्ते महाघोरैस्तत्त्वमाख्याहि पृच्छतः}


\twolineshloka
{स मामुवाच पर्याप्तस्त्वमेषां भरतर्षभ}
{तानुद्दिश्याथ मर्माणि प्रतिघातं तदाचर}


\twolineshloka
{एतच्छ्रुत्वा तु राजेनद्रसंप्रहृष्टस्तमूचिवान्}
{निवर्तय रथं शीघ्रं पश्य चैतान्निपातितान्}


% Check verse!
एवमुक्तो रथं तत्र मातलिः पर्यवर्तयत्
\twolineshloka
{ततो मत्वा रणे भग्नं दानवाः प्रतिहर्षिताः}
{विचुक्रुशुर्महाराज स्वरेण महता मुहुः}


\twolineshloka
{अभग्नः कैश्चिदप्येष पाण्डवो रणमूर्धनि}
{अस्माभिः समरे भग्न इत्येवं संघशस्तदा}


\threelineshloka
{ततोहं देवदेवाय रुद्रायाव्यक्तमूर्तये}
{प्रयतः प्रणतो भूत्वा नमस्कृत्य महात्मने'}
{यत्तद्रौद्रमिति ख्यातं सर्वामित्रविनाशनम्}


\twolineshloka
{यत्तद्रौद्रमिति ख्यातं सर्वामित्रविनाशनम्}
{`महत्पाशुपतं दिव्यं सर्वलोकनमस्कृतम्'}


\threelineshloka
{`ततोऽपश्यं त्रिशिरसं पुरुषं नवलोचनम्}
{त्रिमुखं षङ्भुजं दीप्तमर्कज्वलनमूर्धजम्}
{लेलिहानैर्महानागैः कृतचिह्नममित्रहन्}


\twolineshloka
{विभीस्ततस्तदस्त्रं तु घोरं रौद्रं सनातनम्}
{दृष्ट्वा गाण्डीवसंयोगमानीय भरतर्षभ}


\twolineshloka
{नमस्कृत्वा त्रिनेत्राय शर्वायामिततेजसे}
{मुक्तवान्दानवेन्द्राणां पराभावाय भारत}


\twolineshloka
{मुक्तमात्रे ततस्तस्मिन्रूपाण्यासन्सहस्रशः}
{मृगाणामथ सिंहानां व्याग्राणां च विशांपते}


\threelineshloka
{ऋक्षाणां महिषाणां च पन्नगानां तथा गवाम्}
{शरभाणां गजानां च वानराणां च सङ्घशः}
{ऋषभाणां वराहाणां मार्जाराणां तथैव च}


\twolineshloka
{सालावृकाणां प्रेतानां भुरुण्डानां च सर्वशः}
{गृध्राणां गरुडानां च चमराणां तथैव च}


\twolineshloka
{देवानां च ऋषीणां रच गन्धर्वाणां च सर्वशः}
{पिशाचानां सयक्षाणां तथैव च सुरद्विषाम्}


\twolineshloka
{गुह्यकानां च संग्रामे नैर्ऋतानां तथैव च}
{झषाणां गजवक्राणामुलूकानां तथैव च}


\twolineshloka
{मीनवाजिसरूपाणां नानाशस्त्रासिपाणिनाम्}
{तथैव यातुधानानां गदामुद्गरधारिणाम्}


\twolineshloka
{एतैस्चान्यैश्च बहुभिर्नानारूपधरैस्तदा}
{सर्वमासीज्जगद्व्याप्तं तस्मिन्नस्त्रे विसर्जिते}


\twolineshloka
{त्रिशिरोभिश्चतुर्दंष्ट्रैश्चतुरास्यैश्चतुर्भुजैः}
{अनेकरूपसंयुक्तैर्मांसमेदोवसास्तिभिः}


\twolineshloka
{अभीक्ष्णं वध्यमानास्ते दानवा ये समागताः}
{अर्कज्वलनतेजोभिर्वज्राशनिसमप्रभैः}


\twolineshloka
{अद्रिसारमयैश्चान्यैर्बाणैरपि निबर्हणैः}
{न्यहनं दानवान्सर्वान्मुहूर्तेनैव भारत}


\twolineshloka
{गाण्डीवास्त्रप्रणुन्नांस्तान्गतासून्नभसश्च्युतान्}
{दृष्ट्वाऽहं प्राणमं भूयस्त्रिपुरघ्नाय वेधसे}


\twolineshloka
{तथा रौद्रास्त्रनिष्पिष्टान्दिव्याभरणभूषितान्}
{निशाम्य परमं हर्षमगमद्देवसारथिः}


\twolineshloka
{तदसह्यं कृतं कर्म देवैरपि दुरासदम्}
{दृष्ट्वा मां पूजयामास मातलिः शक्रसारथिः}


\twolineshloka
{उचाच वचनं चेदंप्रीयमाणः कृताञ्जलिः}
{`हतांस्तान्दानवान्दृष्ट्वा मया सङ्ख्ये सहस्रशः}


\threelineshloka
{सुरासुरैरसह्यं हि कर्म यत्साधितं त्वया}
{न ह्येतत्संयुगे कर्तुमपि शक्तः सुरेश्वरः}
{`ध्रुवं धनञ्जय प्रीतस्त्वयि शक्रः पुरार्दन'}


\threelineshloka
{सुरासुरैरवध्यं हि पुरमेतत्खगं महत्}
{त्वया विमथितं वीर स्ववीर्यतपसो बलात् ॥अर्जुन उवाच}
{}


\twolineshloka
{विद्वस्ते खपुरे तस्मिन्दानवेषु हतेषु च}
{विनदन्त्यः स्त्रियः सर्वा निष्पेतुर्नगराद्बहिः}


\twolineshloka
{प्रकीर्णकेश्यो व्यथिताः कुरर्य इव दुःखिताः}
{पेतुः पुत्रान्पितृन्भ्रातॄञ्शोचमाना महीतले}


\twolineshloka
{रुदत्यो दीनकण्ठ्यस्तु निनदन्त्यो हतेश्वराः}
{उरांसि पाणिभिर्घ्नन्त्यो विस्रस्तस्रग्विभूषणाः}


\twolineshloka
{तच्छोकयुक्तमश्रीकं दुःखदैन्यसमाहतम्}
{न बभौ दानवपुरं हतत्विट्कं हतेश्वरम्}


\twolineshloka
{गन्धर्वनगराकारं हृतनागमिव ह्रदम्}
{शुष्कवृक्षमिवारण्यमदृश्यमभवत्पुरम्}


\twolineshloka
{मां तु संहृष्टमनसं क्षिप्रं मातलिरानयत्}
{देवराजस्य भवनं कृतकर्माणमाहवात्}


\twolineshloka
{हिरण्यपुरमुत्सृज्य निहत्य च महासुरान्}
{निवातकवचांश्चैव ततोऽहंशक्रमागमम्}


\twolineshloka
{मम कर्म च देवेन्द्रं मातलिर्विस्तरेण तत्}
{सर्वं विश्रावयामास यथाभूतं महाद्युते}


\threelineshloka
{हिरण्यपुरघातं च मायानां च निवारणम्}
{निवातकवचानां च वधं सङ्ख्ये महौजसाम्}
{`कालकेयवधं चैव अद्भुतं रोमहर्षणम्'}


\threelineshloka
{तच्छ्रुत्वा भगवान्प्रीतः सहस्राक्षः पुरंदरः}
{मरुद्भिः सहितः श्रीमान्साधुसाध्वित्यथाब्रवीत्}
{`परिष्वज्य च मां प्रेम्णा मूर्ध्नि चाघ्राय सस्मितम्'}


\twolineshloka
{ततो मां देवराजो वै समाश्वास्य पुनः पुनः}
{अब्रवीद्विबुधैः सार्धमिदं समधुरं वचः}


\twolineshloka
{अतिदेवासुरं कर्म कृतमेव त्वया रणे}
{गुर्वर्थश्च कृतः पार्थ महाशत्रून्घ्नता मम}


\twolineshloka
{एवमेव सदा भाव्यं स्थिरेणाजौ धनंजय}
{असंमूढेन चास्त्राणां कर्तव्यं प्रतिपादनम्}


\twolineshloka
{अविषह्यो रणे हि त्वं देवदानवराक्षसैः}
{सयक्षासुरगन्धर्वैः सपक्षिगणपन्नगैः}


\twolineshloka
{वसुधां चापि कौन्तेय त्वद्बाहुबलनिर्जिताम्}
{पालयिष्यति धर्मात्मा कुन्तीपुत्रो युधिष्ठिरः}


\chapter{अध्यायः १७६}
\twolineshloka
{अर्जुन उवाच}
{}


\twolineshloka
{ततो मामिविश्वस्तं संरूढशरविक्षतम्}
{देवराजोऽनुगृह्येदं काले वचनमब्रवीत्}


\twolineshloka
{दिव्यान्यस्त्राणि सर्वाणि त्वयि तिष्ठन्ति भारत}
{न त्वाऽभिभवितुं शक्तो मानुषो भुवि कश्चन}


\twolineshloka
{भीष्मो द्रोणः कृपः कर्णः शकुनिः सह राजभिः}
{संग्रामस्थस्य ते पुत्र कलां नार्हन्ति षोडशीम्}


\twolineshloka
{इदं च मे तनुत्राणं प्रायच्छन्मघवान्प्रभुः}
{अभेद्यं कवचं दिव्यं स्रजं चैव हिरण्मयीम्}


\twolineshloka
{देवदत्तं च मे शङ्खं देवः प्रादान्महारवम्}
{दिव्यं चेदं किरीटं मे स्वयमिन्द्रो युयोज ह}


\twolineshloka
{ततो दिव्यानि वस्त्राणि दिव्यान्याभरणानि च}
{प्रादाच्छक्रो ममैतानि रुचिराणि बृहन्ति च}


\twolineshloka
{एवं संपूजितस्तत्रसुखमस्म्युषितो नृप}
{इन्द्रस् भवने पुण्ये गन्धर्वशिशुभिः सह}


\twolineshloka
{ततो मामब्रवीच्छक्रः प्रीतिमानमरैः सह}
{समयोऽर्जुन गन्तुं ते भ्रातरो हि स्मरन्ति ते}


\twolineshloka
{एवमिन्द्रस्य भवने पञ्चवर्षाणि भारत}
{उषितानि मया राजन्स्मरता द्यूतजं कलिम्}


\threelineshloka
{ततो भवन्तमद्राक्षं भ्रातृभिः परिवारितम्}
{गन्धमादनमासाद्य पर्वतस्यास्य मूर्धनि ॥युधिष्ठिर उवाच}
{}


\twolineshloka
{दिष्ट्या धनंजयास्त्राणि त्वया प्राप्तानि भारत}
{दिष्ट्या चाराधितो राजा देवानामीश्वरः प्रभुः}


\twolineshloka
{दिष्ट्या च भगवन्स्थाणुर्देव्या सह परंतपः}
{साक्षाद्दृष्टः स्वयुद्धेन तोषितश् त्वयाऽनघ}


\twolineshloka
{दिष्ट्या च लोकपालैस्त्वं समेतो भरतर्षभ}
{दिष्ट्या वर्धामहे पार्थ दिष्ट्याऽसि पुनरागतः}


\threelineshloka
{अद्य कृत्स्नामिमां देवीं विजितां पुरमालिनीम्}
{`पश्यामि भूमिं कौन्तेय त्वया मे प्रतिपादिताम्'}
{मन्ये च धृतराष्ट्रस्य पुत्रानपि वशीकृतान्}


\threelineshloka
{इच्छामि तानि चास्त्राणि द्रष्टुं दिव्यानि भारत}
{यैस्तथा वीर्यवन्तस्ते निवातकवचा हताः ॥अर्जुन उवाच}
{}


\threelineshloka
{सः प्रभाते भवान्द्रष्टा दिव्यान्यस्त्राणि सर्वशः}
{निवातकवचा घोरा यैर्मया विनिपातिताः ॥वैशंपायन उवाच}
{}


\twolineshloka
{एवमागमनं तत्र कथयित्वा धनंजयः}
{भ्रातृभिः सहितः सर्वै रजनीं तामुवास ह}


\chapter{अध्यायः १७७}
\twolineshloka
{वैशंपायन उवाच}
{}


\twolineshloka
{तस्यां रात्र्यां व्यतीतायां धर्मराजो युधिष्ठिरः}
{उत्थायावश्यकार्याणि कृतवान्भ्रातृभिः सह}


\twolineshloka
{ततः संचोदयामास सोऽर्जुनं मातृनन्दनम्}
{दर्शयास्त्राणि कौन्तेय यैर्जिता दानवास्त्वया}


\twolineshloka
{ततो धनंजयो राजन्देवैर्दत्तानि पाण्डवः}
{अस्त्राणि तानि दिव्यानि दर्शयामास भारत}


\twolineshloka
{यथान्यायं महातेजाः शौचं परममास्थितः}
{`नमस्कृत्य त्रिनेत्राय वासवाय च पाण्डवः'}


\twolineshloka
{गिरिकूबरपादाक्षं शुभवेणु त्रिवेणुमत्}
{पार्थिवं रथमास्थाय शोभमानो धनंजयः}


\twolineshloka
{दिव्येन संवृतस्तेन कवचेन सुवर्चसा}
{धनुरादाय गाण्डीवं देवदत्तं स वारिजम्}


\twolineshloka
{शोशुभ्यमानः कौन्तेय आनुपूर्व्यान्महाभुजः}
{अस्त्राणि तानि दिव्यानि दर्शनायोपचक्रमे}


\twolineshloka
{अथ प्रयोक्ष्यमाणेषु दिव्येष्वस्त्रेषु तेषु वै}
{समाक्रान्ता मही पद्भ्यां समकम्पत सद्रुमा}


\twolineshloka
{क्षुभिताः सरितश्चैव तथैव च महोदधिः}
{शैलाश्चापि व्यदीर्यन्त न ववौ च समीरणः}


\twolineshloka
{न बभासे सहस्रांशुर्न जज्वाल च पावकः}
{न वेदाः प्रतिभान्ति स्म द्विजातीनां कथंचन}


\twolineshloka
{अन्तर्भूमिगता ये च प्राणिनो जनमेजय}
{पीड्यमानाः समुत्थाय पाण्डवं पर्यवारयन्}


\twolineshloka
{वेपमानाः प्राञ्जलयस्ते सर्वे विकृताननाः}
{दह्यमानास्तदाऽस्त्रैस्ते याचन्ति स्म धनंजयम्}


\twolineshloka
{ततो ब्रह्मर्षयश्चैव सिद्धा ये च महर्षयः}
{जङ्गमानि च भूतानि सर्वाण्येवावतस्थिरे}


\threelineshloka
{देवर्षयश्च प्रवरास्तथैव च दिवौकसः}
{यक्षराक्षसगन्धर्वास्तथैव च पतत्रिणः}
{खेचराणि च भूतानि सर्वाण्येवावतस्थिरे}


\twolineshloka
{ततः पितामहश्चैव लोकपालाश्च सर्वशः}
{भगवांश्च महादेवः सगणोऽभ्याययौ तदा}


\twolineshloka
{ततो वायुर्महाराज दिव्यैर्माल्यैः समन्वितः}
{अभितः पाण्डवंचित्रैरवचक्रे समन्ततः}


\twolineshloka
{जगुश्च गाथा विविधा गन्धर्वाः सुरचोदिताः}
{ननृतुः सङ्घशश्चैव राजन्नप्सरसां गणाः}


\twolineshloka
{तस्मिंश्च तादृशे काले नारदश्चोदितः सुरैः}
{आगम्याह वचः पार्थं श्रवणीयमिदं नृप}


\twolineshloka
{अर्जुनार्जुन मा युङ्क्ष दिव्यान्यस्त्राणि भारत}
{नैतानि निरधिष्ठाने प्रयुज्यन्ते कथंचन}


\twolineshloka
{अधिष्ठाने न वाऽनार्तः प्रयुञ्जीत कदाचन}
{प्रयोगेषु महान्दोषो ह्यस्त्राणां कुरुनन्दन}


\twolineshloka
{एतानि रक्ष्यमाणानि धनंजय यथागमम्}
{बलवन्ति सुखार्हाणि भविष्यन्ति न संशयः}


\twolineshloka
{अरक्ष्यमाणान्येतानि त्रैलोक्यस्यापि पाण्डव}
{भवन्ति स्म विनाशाय मैवं भूयः कृथाः क्वचित्}


\threelineshloka
{अजातशत्रो त्वं चैव द्रक्ष्यसे तानि संयुगे}
{योज्यमानानि पार्थे नद्विषतामवमर्दने ॥वैशंपायन उवाच}
{}


\twolineshloka
{निवार्याथ ततः पार्थं सर्वे देवा यथागतम्}
{जग्मुरन्ये च ये तत्र रसमाजग्मुर्नरर्षभ}


\twolineshloka
{तेषु सर्वेषु कौरव्य प्रतियातेषु पाण्डवाः}
{तस्मिननेव वने हृष्टास्त ऊषुः सह कृष्णया}


\chapter{अध्यायः १७८}
\twolineshloka
{जनमेजय उवाच}
{}


\fourlineindentedshloka
{तस्मिन्कृतास्त्रे रथिनां प्रवीरे}
{प्रत्यागते भवनाद्वृत्रहन्तुः}
{अतः परं किमकुर्वन्त पार्थाःसमेत्य शूरेण धनंजयेन ॥वैशंपायन उवाच}
{}


\twolineshloka
{वनेषु तेष्वेव तु ते नरेन्द्राःसहार्जुनेनेन्द्रसमेन वीराः}
{तस्मिंश्च शैलप्रवरे सुरम्येधनेश्वराक्रीडगता विजह्रुः}


\twolineshloka
{वेश्मानि तान्यप्रतिमानि पश्यन्क्रीडाश्च नानाद्रुमसन्निबद्धाः}
{चचार धन्वी बहुधा नरेन्द्रःसोऽस्त्रेषु यत्तः सततं किरीटी}


\twolineshloka
{अवाप्य वासं नरदेवपुत्राःप्रसादजं वैश्रवणस्य राज्ञः}
{न प्राणिनां ते स्पृहयन्ति राज-ञ्शिवश् कालः स बभूव तेषाम्}


\twolineshloka
{समेत्य पार्थेन यथैकरात्र-मूषुः समास्तत्रतथा चतस्रः}
{पूर्वाश्च षट् ता दश पाण्डवानांशिवा बभूवुर्वसतां वनेषु}


\twolineshloka
{ततोऽब्रवीद्वायुसुतस्तरस्वीजिष्णुश्च राजानमुपोपविश्य}
{यमौ च वीरौ सुरराजकल्पा-वेकान्तमास्थाय हितं प्रियं च}


\twolineshloka
{तव प्रतिज्ञां कुरुराज सत्यांचिकीर्षमाणास्त्वदनुप्रियं च}
{ततो न गच्छाम वनान्यपास्यसुयोधनं सानुचरं निहन्तुम्}


\twolineshloka
{एकादशं वर्षमिदं वसामःसुयोधनेनात्तसुखाः सुखार्हाः}
{तं वञ्चयित्वाऽधमबुद्धिशील-मज्ञातवासं सुखमाप्नुयाम}


\twolineshloka
{तवाज्ञया पार्थिव निर्विशङ्काविहाय मानं विचरामो वनानि}
{समीपवासेन विलोभितास्तेज्ञास्यन्ति नास्मानपकृष्टदेशान्}


\twolineshloka
{संवत्सरं तत्र विहृत्य गूढंनराधमं तं सुखमुद्धरेम}
{निर्यात्य वैरं सफलं सपुष्पंतस्मै नरेन्द्राधमपूरुषाय}


\twolineshloka
{सुयोधनायानुचरैर्वृतायततो महीं प्राप्नुहि धर्मराज}
{स्वर्गोपमं देशमिमं चरद्भिःशक्यो विहन्तुं नरदेव शोकः}


\twolineshloka
{कीर्तिस्तु ते भारत पुण्यगन्धानश्येद्धि लोकेषु चराचरेषु}
{तत्प्राप्य राज्यंकुरुपुङ्गवानांशक्यं महत्प्राप्तुमथ क्रियाश्च}


\twolineshloka
{इदं तु शक्यं सततं नरेन्द्रप्राप्तुं त्वया यल्लभसे कुबेरात्}
{कुरुष्व बुद्धिं द्विषतां वधायकृतागसां भारत निग्रहे च}


\twolineshloka
{तेजस्तवोग्रं न सहेत राजन्समेत्य साक्षादपि वज्रपाणिः}
{न हि व्यथां जातु करिष्यतस्तौसमेत्य देवैरपि धर्मराज}


\twolineshloka
{तवार्थसिद्ध्यर्थमपि प्रवृत्तौसुपर्णकेतुश्च शिनेश्च नप्ता}
{यथैव कृष्णोऽप्रतिमो बलेनतथैव राजन्स शिनिप्रवीरः}


\twolineshloka
{तवार्थसिद्ध्यर्थमभिप्रपन्नोयथैव कृष्णः सह यादवैस्तैः}
{तथैव चेमौ नरदेववर्ययमौ च वीरौ कृतिनौ प्रयोगे}


% Check verse!
त्वदर्थयोगप्रभवप्रधानाःशमं करिष्याम परान्समेत्य ॥वैशंपायन उवाच
\twolineshloka
{ततस्तदाज्ञाय मतं महात्मातेषां च धर्मस्य सुतो वरिष्ठः}
{प्रदक्षिणं स्थानमुपेत्य राजापर्याक्रमद्वैश्रवणस्य राज्ञः}


\threelineshloka
{आमन्त्र्य वेश्मानि नदीः सरांसिसर्वाणि रक्षांसि च धर्मराजः}
{यथागतं मार्गमवेक्षमाणःपुनर्गिरिं चैव निरीक्षमाणः}
{[ततो महात्मा स विशुद्धबुद्धिःसंप्रार्थयामास नगेन्द्रवर्यम् ॥]}


\twolineshloka
{समाप्तकर्मा सहितः सुहृद्भि-र्जित्वा सपत्नान्प्रतिलभ्य राज्यम्}
{शैलेन्द्र भूयस्तपसे जितात्माद्रष्टा तवास्मीति मतिं चकार}


\twolineshloka
{वृतश्च सर्वैरनुजैर्द्विजैश्चतेनैव मार्गेण पुनर्निवृत्तः}
{उवाह चैनान्गणशस्तथैवघटोत्कचः पर्वतनिर्झरेषु}


\twolineshloka
{तान्प्रस्थितान्प्रीतमना महर्षिःपितेव पुत्राननुशिष्य सर्वान्}
{स लोमशो दिवमेवोर्जितश्री-र्जगाम तेषां विजयं तदोक्त्वा}


\twolineshloka
{तेनार्ष्टिषेणेन तथानुशिष्टा-स्तीर्तानि रम्याणि तपोवनानि}
{महान्ति चान्यानि सरांसि पार्थाःकसंपश्यमानाः प्रययुर्नराग्र्याः}


\chapter{अध्यायः १७९}
\twolineshloka
{वैशंपायन उवाच}
{}


\twolineshloka
{नगोत्तमं प्रस्रवणैरुपेतंदिशां गजैः किन्नरपक्षिभिश्च}
{सुखं निवासं जहतां हि तेषांन प्रीतिरासीद्भरतर्षभाणाम्}


\twolineshloka
{ततस्तु तेषां पुनरेव हर्षःकैलासमालोक्य महान्बभूव}
{कुबेरकान्तं भरतर्षभाणांमहीधरं वारिधरप्रकाशम्}


\twolineshloka
{समुच्छ्रयान्पर्वतसंनिरोधान्गोष्ठान्हरीणां गिरिगह्वराणि}
{बहुप्रकाराणि समीक्ष्यवीराःस्थलानि निम्नानि च तत्र तत्र}


\twolineshloka
{तथैव चान्यानि महावनानिमृगद्विजानेकपसेवितानि}
{आलोकयन्तोऽभिययुः प्रतीता-स्ते धन्विनाः खङ्गधरा नराग्र्याः}


% Check verse!
वनानि रम्याणि नदीः सरांसिगुहा गिरीणां गिरिगह्वराणिएते निवासाः सततं बभूवु-र्निशामुखं प्राप्य नरर्षभाणाम्
\twolineshloka
{ते दुर्गवासं बहुधा निरुष्यव्यतीत्य कैलासमचिन्त्यरूपम्}
{आसेदुरत्यर्थमनोरमं तेतमाश्रमाग्र्यं वृषपर्वणस्तु}


\twolineshloka
{समेत्य राज्ञा वृषपर्वणा तेप्रत्यर्चितास्तेन च वीतमोहाः}
{शशंसिरे विस्तरशः प्रवासंशिवं यथावद्वृषपर्वणस्ते}


\twolineshloka
{सुखोषितास्तस्य त एकरात्रंपुण्याश्रमे देवमहर्षिजुष्टे}
{अभ्याययुस्ते बदरीं विशालांसुखेन वीराः पुनरेव वासम्}


\twolineshloka
{ऊषुस्ततस्तत्र महानुभावानारायणस्थानगताः समग्राः}
{कुबेरकान्तां नलिनीं विशोकाःसंपश्यमानाः सुरसिद्धजुष्टाम्}


\twolineshloka
{तां चाथ दृष्ट्वा नलिनीं विशोकाःपाण्डोः सुताः सर्वनरप्रधानाः}
{ते रेमिरे नन्दनवासमेत्यद्विजर्षयो वीतमला यथैव}


\twolineshloka
{ततः क्रमेणोपययुर्नृवीरायथागतेनैव पथा समग्राः}
{विहृत्य मासं सुखिनो बदर्यांकिरातराज्ञो विषयं सुबाहोः}


\twolineshloka
{चीनांस्तुषारान्दरदांश्च सर्वान्देशान्कुलिन्दस्य च भूरिरम्यान्}
{अतीत्य दुर्गं हिमवत्प्रदेशंपुरं सुबाहोर्ददृशुर्नृवीराः}


\twolineshloka
{श्रुत्वा च तान्पार्थिव पुत्रपौत्रा-न्प्राप्तान्सुबाहुर्विषये समग्रान्}
{प्रत्युद्ययौ प्रीतियुतः स राजातं चाभ्यनन्दन्वृषभाः कुरूणाम्}


\twolineshloka
{समेत्य राज्ञा तु सुबाहुना तेसुतैर्विशोकप्रमुखैश्च सर्वे}
{सहेनद्रसेनैः परिचारकैश्चपौरोगवैर्ये च रमहानसस्थाः}


\twolineshloka
{सुखोपितास्तत्रत एकरात्रंसूतान्समादाय रथांश्च सर्वान्}
{घटोत्कचं सानुचरं विसृज्यततोऽभ्ययुर्यामुनमद्रिराजम्}


\twolineshloka
{तस्मिन्गिरौ प्रस्रवणोपपन्न-हिमोत्तरीयारुणपाण्डुसानौ}
{विशाखयूपं समुपेत्य चक्रु-स्तदा निवासं पुरुषप्रवीराः}


\twolineshloka
{वराहनानामृगपक्षिजुष्टंमहावनं चैत्ररथप्रकाशम्}
{शिवेन यात्वा मृगयाप्रधानाःसंवत्सरं तत्रवने विजह्रुः}


\twolineshloka
{तत्राससादातिबलं भुजङ्गंक्षुधार्दितं मृत्युमिवोग्ररूपम्}
{वृकोदरः पर्वतकन्दरायांविषादमोहव्यथितान्तरात्मा}


\twolineshloka
{द्वीपोऽभवद्यत्र वृकोदरस्ययुधिष्ठिरो धर्मभृतां वरिष्ठः}
{अमोक्षयद्यस्तमनन्ततेजाग्राहेण संवेष्टितसर्वगात्रम्}


\twolineshloka
{ते द्वादशं वर्षमथोपयान्तंवने विहुर्तुं कुरवः प्रतीताः}
{तस्माद्वनाच्चैत्ररथप्रकाशा-च्छ्रिया ज्वलन्तस्तपसा च युक्ताः}


\twolineshloka
{ततश्च यात्वा मरुधन्वपार्श्वंसदा धनुर्वेदरतिप्रधानाः}
{सरस्वतीमेत्य निवासकामाःसरस्ततो द्वैतवनं प्रतीयुः}


\twolineshloka
{समीक्ष् तान्द्वैतवने निविष्टा-न्निवासिनस्तत्रततोऽभिजग्मुः}
{तपोदमाचारसमाधियुक्ता-स्तृणोदपात्रावरणाश्मकुट्टाः}


\twolineshloka
{प्लक्षाक्षरौहीतकवेतसाश्चतथा बदर्यः खदिराः शिरीषाः}
{बिल्वेङ्गुदाः पीलुशमीकरीराःसरस्वतीतीररुहा बभूवुः}


\twolineshloka
{तां यक्षगन्धर्वमहर्षिकान्ता-मावासभूतामिव देवतानाम्}
{सरस्वतीं प्रीतियुताश्चरन्तःसुखं विजह्रुर्नरदेवपुत्राः}


\chapter{अध्यायः १८०}
\twolineshloka
{जनमेजय उवाच}
{}


\twolineshloka
{कथं नागायुतप्राणो भीमो भीमपराक्रमः}
{भयमाहारयत्तीव्रं तस्मादजगरान्मुने}


\twolineshloka
{पौलस्त्यं धनदं युद्दे य आह्वयति दर्पितः}
{नलिन्यां कदनं कृत्वा निहन्ता यक्षरक्षसाम्}


\threelineshloka
{तं संससि भयाविष्टमापन्नमरिसूदनम्}
{एतदिच्छाम्यहं श्रोतुं परं कौतूहलं हि मे ॥वैशंपायन उवाच}
{}


\twolineshloka
{बह्वाश्चर्ये वने तेषां वसतामुग्रधन्विनाम्}
{प्राप्तानामाश्रमं राजन्राजर्षेर्वृषपर्वणः}


\twolineshloka
{यदृच्छया धनुष्पाणिर्बद्धखङ्गो वृकोदरः}
{ददर्श तद्वनं रम्यं देवगन्धर्वसेवितम्}


\twolineshloka
{स ददर्श शुभान्देशान्गिरेर्हिमवतस्तदा}
{देवर्षिसिद्धचरितानप्सरोगणसेवितान्}


\twolineshloka
{चकोरैरुपचक्रैशच् पक्षिभिर्जीवजीवकैः}
{कोकिलैर्भृङ्गराजैश्च तत्र तत्र निनादितान्}


\twolineshloka
{नित्यपुष्पफलैर्वृक्षैर्हिमसंस्पर्शकोमलैः}
{उपेतान्बहुलच्छायैर्मनोनयननन्दनैः}


\twolineshloka
{स संपश्यन्गिरिनदीर्वैडूर्यमणिसंनिभैः}
{सलिलैर्हिमसंकाशैर्हंसकारण्डवायुतैः}


\twolineshloka
{वनानि देवदारूणां मेघानामिव वागुराः}
{हरिचन्दनमिश्राणि तुङ्गकालीयकान्यपि}


\twolineshloka
{मृगयां परिधावन्स समेषु मरुधन्वसु}
{विध्यन्मृगाञ्शरैः शुद्धैश्चचार स महाबलः}


\twolineshloka
{भीमसेनस्तु विख्यातो महान्तं दंष्ट्रिणं बलात्}
{निघ्नन्नागशतप्राणो वने तस्मिन्महाबलः}


\twolineshloka
{मृगाणां सवराहाणां महिषाणां महाभुजः}
{विनिघ्नंस्तत्रतत्रैव भीमो भीमपराक्रमः}


\twolineshloka
{स मातङ्गशतप्राणो मनुष्यशतवारणः}
{सिंहशार्दूलविक्रान्तो वने तस्मिन्महाबलः}


\twolineshloka
{वृक्षानुत्पाटयामास तरसा वै बभञ्ज च}
{पृथिव्याश्च प्रदेशान्वै नादयंस्तु वनानि च}


\twolineshloka
{पर्वताग्राणि वै मृद्गन्नादयानश्च विज्वरः}
{प्रक्षिपन्पादपांश्चापि नादेनापूरयन्महीम्}


\threelineshloka
{वेगेन न्यपतद्भीमो निर्भयश्च पुनः पुनः}
{आस्फोटयन्क्ष्वेडयंश्च तलतालांश्च वादयन्}
{चिरसंबद्धदर्पस्तु भीमसेनो वने तदा}


\twolineshloka
{गजेनद्राश्च महासत्वा मृगेन्द्राश्च महाबलाः}
{भीमसेनस्य नादेन व्यमुञ्चन्त गुहा भयात्}


\twolineshloka
{क्वचित्प्रधावंस्तिष्ठंश्च क्वचिच्चोपविशंस्तथा}
{मृगप्रेप्सुर्महारौद्रे वने चरति निर्भयः}


\twolineshloka
{स तत्रमनुजव्याघ्रो वने वनचरोपमः}
{पद्भ्यामभिसमापेदे भीमसेनो महाबलः}


\twolineshloka
{स प्रविष्टो महारण्ये नादान्नदति चाद्भुतान्}
{त्रासयन्सर्वभूतानि महासत्वपराक्रमः}


\threelineshloka
{ततो भीमस्य शब्देन भीताः सर्पा गुहाशयाः}
{अतिक्रान्तास्तु वेगेन जगामानुसृतः शनैः}
{ततोऽमरवरप्रख्यो भीमसेनो महाबलः}


\twolineshloka
{स ददर्श महाकायं भुजङ्गं रोमहर्षणम्}
{गिरिदुर्गे समापन्नं कायेनावृत्य कन्दरम्}


\twolineshloka
{पर्वताभोगवर्ष्माणं भोगैश्चन्द्रार्कमण्डलैः}
{चित्राङ्गमङ्गजैश्चित्रैर्हरिद्रासदृशच्छविम्}


\twolineshloka
{गुहाकारेण वक्रेण चतुर्दंष्ट्रेण राजता}
{दीप्ताक्षेणातिताम्रेण लिहानं सृक्विणी मुहुः}


\twolineshloka
{त्रासनं सर्वभूतानां कालान्तकयमोपमम्}
{निःश्वासक्ष्वेडनादेन भर्त्सयन्तमिव स्थितम्}


\twolineshloka
{स भीमं सहसाऽभ्येत्य पृदाकुः क्षुधितो भृशम्}
{जग्राहाजगरो ग्राहो भुजयोरुभयोर्बलात्}


\twolineshloka
{तेन संस्पृष्टगात्रस्य भीमसेनस्य वै तदा}
{संज्ञा मुमोह सहसा वरदानेन तस्य हि}


\twolineshloka
{दशनागसहस्राणि धारयन्ति हि यद्बलम्}
{इद्रलं भीमसेनस्य भुजयोरसमं परैः}


\twolineshloka
{स तेजस्वी तथा तेन भुजगेन वशीकृतः}
{विम्फुरञ्शनकैर्भीमो न शशाक विचेष्टितुम्}


\twolineshloka
{नागायुतसमप्राणः सिंहस्कन्धो महाभुजः}
{गृहीतो व्यजहात्सत्वं वरदानविमोहितः}


\twolineshloka
{स हि प्रयत्नमकरोत्तीव्रमात्मविमोक्षणे}
{न चैनमशकद्वीरः कथंचित्प्रतिबाधितुम्}


\chapter{अध्यायः १८१}
\twolineshloka
{वैशंपायन उवाच}
{}


\twolineshloka
{स भीमसेनस्तेजस्वी तदा सर्पवशं गतः}
{चिन्तयामास सर्पस्य वीर्यमत्यद्भुतं महत्}


\twolineshloka
{उवाच च महासर्पं कामयाब्रूहि पन्नग}
{कस्त्वं भो भुजगश्रेष्ठ किं मया च करिष्यसि}


\twolineshloka
{पाण्डवो भीमसेनोऽहं धर्मराजादनन्तरः}
{नागायुतसमप्राणस्त्वया नीतः कथं वशम्}


\twolineshloka
{सिंहाः केसरिणो व्याघ्रा महिषा वारणास्तथा}
{समागताश्च बहुशो निहताश्च मया युधि}


\twolineshloka
{राक्षसाश्च पिशाचाश्च पन्नगाश्च महाबलाः}
{भुजवेगमशक्ता मे सोढुं पन्नगसत्तम}


\twolineshloka
{किंनु विद्याबलं किंनु वरदानमथो तव}
{उद्योगमपि कुर्वाणो वशगोस्मि कृतस्त्वया}


\twolineshloka
{अनुत्यो विक्रमो नॄणामिति मे धीयते मतिः}
{यथेदं मे त्वया नाग बलं प्रतिहतं महत्}


\twolineshloka
{इत्येवंवादिनं वीरं भीममक्लिष्टकारिणम्}
{भोगेन महता गृह्य समन्तात्पर्यवेष्टयत्}


\twolineshloka
{निगृह्यैनं महाबाहुं ततः स भुजगस्तदा}
{विमुच्यास्य भुजौ पीनाविदं वचनमब्रवीत्}


\twolineshloka
{दिष्टस्त्वं क्षुधितस्याद्य देवैर्भक्षो महाभुज}
{दिष्ट्या कालस्य महतः प्रियाः प्राणा हि देहिनां}


\twolineshloka
{यथा त्विदं मया प्राप्तं भुजङ्गत्वमरिंदम}
{तदवश्यं त्वया मत्तः श्रोतव्यं शृणु यन्मम}


\twolineshloka
{इमामवस्थां संप्राप्तो ह्यहं कोपान्महर्षिणा}
{शापस्यान्तं परिप्रेप्सुः सर्वस् कथयामि तत्}


\twolineshloka
{नहुषो नाम राजर्षिर्व्यक्तं ते श्रोत्रमागतः}
{तवैव पूर्वः पूर्वेषामायोर्वंशधरः सुतः}


\twolineshloka
{सोहं शापादगस्त्यस्य ब्राह्मणानवमत्य च}
{इमामवस्थामापन्नः पश्य दैवमिदं मम}


\twolineshloka
{त्वां चेदवध्यमायान्तमतीव प्रियदर्शनम्}
{अहमद्योपयोक्ष्यामि विधानं पश्य यादृशम्}


\twolineshloka
{न हि मे मुच्यते कश्चित्कथंचित्प्रग्रहं गतः}
{गचजोवा महिषो वाऽपि षष्ठे काले नरोत्तम}


\twolineshloka
{नासि केवलसर्पेण तिर्यग्योनिषु वर्तता}
{गृहीतः कौरवश्रेष्ठ वरदानमिदं मम}


\twolineshloka
{पतता हि विमानाग्रान्मया शक्रासनाद्द्रुतम्}
{कुरु शापान्तमित्युक्तो भगवान्मुनिसत्तमः}


\twolineshloka
{`यस्त्वया वेष्टितो राजन्मोहमेति महाबलः'}
{मोक्षस्ते भविता राजन्कस्माच्चित्कालपर्ययात्}


\twolineshloka
{ततोस्मि पतितो भूमौ न च मामजहात्स्मृतिः}
{स्मार्तमस्ति पुराणं मे यथैवाधिगतं तथा}


\twolineshloka
{यस्तु ते व्याहृतान्प्रश्नान्प्रतिब्रूयाद्विभागवित्}
{स त्वां मोक्षयिता शापादिति मामब्रवीदृषिः}


\twolineshloka
{गृहीतस्य त्वया राजन्प्राणिनोपि बलीयसः}
{सत्वभ्रंशोऽधिकस्यापि सर्वस्याशु भविष्यति}


\twolineshloka
{इतिचाप्यहमश्रौषं वचस्तेषां दयावताम्}
{मयि संजातहार्दानामथ तेऽन्तर्हिता द्विजाः}


\twolineshloka
{सोहं परमदुष्कर्मा वसामि निरयेऽशुचौ}
{सर्पयोनिमिमां प्राप्यकालाकाङ्क्षी महाद्युते}


\twolineshloka
{तमुवाच महाबाहुर्भीमसेनो भुजङ्गमम्}
{न ते कुप्ये महासर्प नात्मने द्विजसत्तम}


\twolineshloka
{यस्मादभावी भावी वा मनुष्यः सुखदुःखयोः}
{आगमे यदि वाऽपाये न तत्र ग्लपयेन्मनः}


\twolineshloka
{दैवं पुरुषकारेण को वञ्चयितुमर्हति}
{दैवमेव परं मन्ये पौरुषं तु निरर्थकम्}


\twolineshloka
{पश्य दैवोपघाताद्धि भुजवीर्यव्यपाश्रयम्}
{इमामवस्थां संप्राप्तमनिमित्तमिहाद्य माम्}


\twolineshloka
{किंतु नाद्यानुशोचामि तथाऽऽत्मानं विनाशितम्}
{तथा तु विपिने न्यस्तान्भ्रातॄन्राज्यपरिच्युतान्}


\twolineshloka
{हिमवांश्च सुदुर्गोऽयं यक्षराक्षससंकुलः}
{मां समुद्वीक्षमाणास्ते प्रयतिष्यन्ति विह्वलाः}


\twolineshloka
{विनष्टमथ मां श्रुत्वा भविष्यन्ति निरुद्यमाः}
{धर्मशीला मया ते हि बाध्यन्ते राज्यगृद्धिना}


\twolineshloka
{अथवा नार्जुनो धीमान्विषादमुपयास्यति}
{सर्वास्त्रविदनाधृष्यो देवगन्धर्वराक्षसैः}


\twolineshloka
{समर्थः स महाबाहुरेकाङ्ना सुमहाबलः}
{देवराजमपि स्थानात्प्रच्यावयितुमञ्जसा}


\twolineshloka
{किं पुनर्धृतराष्ट्रस् पुत्रं दुर्द्यूतदेविनम्}
{विद्विष्टं सर्वलोकस्य दम्भमोहपरायणम्}


\twolineshloka
{मातरं चैव शोचामि कृपणां पुत्रगृद्धिनीम्}
{याऽस्माकं नित्यमाशास्ते महत्त्वमधिकं परैः}


\twolineshloka
{तस्याः कथं त्वनाथाया मद्विनाशाद्भुजङ्गम्}
{सफलास्ते भविष्यन्ति मयि सर्वे मनोरथाः}


\twolineshloka
{नकुलः सहदेवश्च यमौ च गुरुवर्तिनौ}
{मद्वाहुबलसंगुप्तौ नित्यं पुरुषमानिनौ}


\twolineshloka
{भविष्यतो निरुत्साहौ भ्रष्टवीर्यपराक्रमौ}
{मद्विनाशात्परिद्यूनाविति मे वर्तते मतिः}


\twolineshloka
{एवंविधं बहु तदा विललाप वृकोदरः}
{भुजङ्गभोगसंरुद्धो नाशकच्च विचेष्टितुम्}


\twolineshloka
{युधिष्ठिरस्तु कौन्तेयो बभूवास्वस्तचेतनः}
{अनिष्टदर्शनान्घोरानुत्पातान्परिचिन्तयन्}


\twolineshloka
{दारुणं ह्यशिवं नादं शिवा दक्षिणतः स्थिता}
{दीप्तायां दिशि वित्रस्ता रौति तस्याश्रमस्य ह}


\twolineshloka
{एकपक्षाक्षिचरणा वर्तिका घोरदर्शना}
{रक्तं वमन्ती ददृशे प्रत्यादित्यमभासुरा}


\twolineshloka
{प्रववौ चानिलो रूक्षश्चण्डः सर्करकर्षणः}
{अपसव्यानि सर्वाणि मृगपक्षिरुतानि च}


\twolineshloka
{पृष्ठतो वायसः कृष्णो याहियाहीति वाशति}
{मुहुर्मुहुः स्फुरति च दक्षिणोऽस्य भुजस्तथा}


\twolineshloka
{हृदयं चरणश्चापि वामोऽस्य परितप्यति}
{सव्यस्याक्ष्णओ विकारश्चाप्यनिष्टः समपद्यत}


\twolineshloka
{धर्मराजोपि मेधावी शङ्कमानो महद्भयम्}
{द्रौपदीं परिपप्रच्छ क्व भीम इति भारत}


\twolineshloka
{शंस तस्मै पाञ्चाली चिरयातं वृकोदरम्}
{स प्रतस्थे महाबाहुर्धौम्येन सहितो नृपः}


\twolineshloka
{द्रौपद्या रक्षणं कार्यमित्युवाच धनंजयम्}
{नकुलं सहदेवं च व्यादिदेश द्विजान्प्रति}


\twolineshloka
{स तस्य पदमुन्नीय तस्मादेवाश्रमात्प्रभुः}
{मृगयामास कौन्तेयो भीमसेनं महावने}


\twolineshloka
{स प्राचीं दिशमास्थाय महतो गजयूथपान्}
{ददर्श पृथिवीं चिह्नैर्भीमस्य परिचिह्निताम्}


\twolineshloka
{ततो मृगसहस्राणि मृगेनद्राणां शतानि च}
{पतितानि वने दृष्ट्वा मार्गं तस्याविशन्नृपः}


\twolineshloka
{धावतस्तस्य वीरस् मृगार्थं वातरंहसः}
{ऊरुवातविनिर्भग्नान्द्रुमान्व्यावर्जितान्पथिः}


\twolineshloka
{स गत्वा तैस्तदा चिह्नैर्ददर्श गिरिगह्वरे}
{[रूक्षमारुतभूयिष्ठे निष्पत्रद्रुमसंकुले}


\threelineshloka
{ईरिणे निर्जले देशे कण्टकिद्रुमसंकुले}
{अश्मस्थाणुक्षुपाकीर्णे सुदुर्गे विषमोत्कटे}
{]गृहीतं भुजगेन्द्रेण निश्चेष्टमनुजं तदा}


\chapter{अध्यायः १८२}
\twolineshloka
{वैशंपायन उवाच}
{}


\twolineshloka
{युधिष्ठिरस्तमासाद्य सर्पभोगेन वेष्टितम्}
{दयितं भ्रातरं धीमानिदं वचनमब्रवीत्}


\twolineshloka
{कुन्तीमातः कथमिमामापदं त्वमवाप्तवान्}
{कश्चायं पर्वताभोगप्रतिमः पन्नगोत्तमः}


\twolineshloka
{स धर्मराजमालक्ष्य भ्राता भ्रतरमग्रजम्}
{कथयामास तत्सर्वं ग्रहणादिविचेष्टितम्}


\threelineshloka
{[अयमार्य महासत्वो भक्षार्थं मां गृहीतवान्}
{नहुषो नाम राजर्षिः प्राणवानिव संस्थितः ॥युधिष्ठिर उवाच}
{}


\threelineshloka
{मुच्यतामयमायुष्मन्भ्राता मेऽमितविक्रमः}
{वयमाहारमन्यं ते दास्यामः क्षुन्निवारणम् ॥सर्प उवाच}
{}


\twolineshloka
{मुच्यतामयमायुष्मन्भ्राता मेऽमितविक्रमः}
{गम्यतां नेह स्थातव्यं श्वो भवानपि मे भवेत्}


\twolineshloka
{व्रतमेतन्महाबाहो विषयं मम यो ब्रजेत्}
{स म भक्षो भवेत्तात त्वं चापि विषये मम}


\threelineshloka
{चिरेणाद्य मयाऽऽहारः प्राप्तोऽयमनुजस्तव}
{नाहमेनं विमोक्ष्यामि न चान्यमभिकाङ्क्षये ॥]युधिष्ठिर उवाच}
{}


\twolineshloka
{देवो वा यदि वा दैत्य उरगो वा भवान्यदि}
{सत्यं सर्प वचो ब्रूहि पृच्छति त्वां युधिष्ठिरः}


\fourlineindentedshloka
{किमर्थं च त्वया ग्रस्तो भीमसेनो भुजंगम}
{किमाहृत्यविदित्वा वा प्रीतिस्ते स्याद्भुजंगम}
{किमाहारं प्रयच्छामि कथं मुञ्चेद्भवानिमम् ॥सर्प उवाच}
{}


\twolineshloka
{नहुषो नाम राजाऽहमासं पूर्वस्तवानघ}
{प्रथितः पञ्चमः सोमादायोः पुत्रो नराधिप}


\twolineshloka
{क्रतुभिस्तपसा चैवस्वाध्यायेन दमेन च}
{त्रैलोक्यैश्वर्यमव्यग्रं प्राप्तोऽहंविकर्मेण च}


\twolineshloka
{तदैश्वर्यं समासाद्य दर्पो मामगमत्तदा}
{सहस्रं हि द्विजातीनामुवाह शिविकां मम}


\twolineshloka
{ऐश्वर्यमदमत्तोऽहमवमत्य ततो द्विजान्}
{इमामगस्त्येन दशामानीत इतिमे स्मृतिः}


\twolineshloka
{न तु मामजहात्प्रज्ञा यावदन्वेति पाण्डवः}
{तस्यैवानुग्रहाद्राजन्नगस्त्यस्य महात्मनः}


\twolineshloka
{षष्ठे काले मयाऽऽहारः प्राप्तोऽयमनुजस्तव}
{नाहमेनं विमोक्ष्यामि न चान्यदपि कामये}


\threelineshloka
{प्रश्नानुच्चारितानद्य वायहरिष्यसि चेन्मम}
{तथापश्चाद्विमोक्ष्यामि भ्रातरं ते वृकोदरम् ॥युधिष्ठिर उवाच}
{}


\twolineshloka
{ब्रूहि सर्प यथाकामं प्रतिवक्ष्यामि ते वचः}
{अपि चच्छक्नुयां प्रीतिमाहर्तुं ते भुजङ्गम्}


\threelineshloka
{वेद्यंच ब्राह्मणेनेह तद्भवान्वेत्ति केवलम्}
{सर्पराज ततः श्रुत्वा प्रतिवक्ष्यामि ते वचः ॥सर्प उवाच}
{}


\threelineshloka
{ब्राह्मणः को भवेद्राजन्वेद्यं किंच युधिष्ठिर}
{ब्रवीह्यतिमतिं त्वां हि वाक्यैरनिमिनोमि ते ॥युधिष्ठिर उवाच}
{}


\twolineshloka
{सत्यं दानं क्षमा शीलमानृशंस्यं तपो घृणा}
{दृश्यन्ते यत्रनागेन्द्रस ब्राह्मण इति स्मृतः}


\threelineshloka
{वेद्यं सर्प परं ब्र्हम निर्दुःखमसुखं च यत्}
{यत्रगत्वा न शोचन्ति भवतः किं विवक्षितं ॥सर्प उवाच}
{}


\twolineshloka
{चातुर्वर्ण्यं प्रमाणं च सत्यं च ब्रह्म चैव हि}
{आनृशंस्यमहिंसा च घृणा चैव युधिष्ठिर}


\threelineshloka
{वेद्यं यच्चात्र निर्दुःखमसुखं च नराधिप}
{ताभ्यां हीनं पदं चान्यन्न तदस्तीति लक्षये ॥युधिष्ठिर उवाच}
{}


\twolineshloka
{शूद्रे तु यद्भवेल्लक्ष्म द्विजे तच्च न विद्यते}
{न वै शूद्रो भवेच्छूद्रो ब्राह्मणो न च ब्राह्मणः}


\twolineshloka
{यत्रैतल्लक्ष्यते सर्प वृत्तं स ब्राह्मणः स्मृतः}
{तत्रैतन्न वेत्सर्प तं शूद्रमिति निर्दिशेत्}


\twolineshloka
{यत्पुनर्भवता प्रोक्तं न वेद्यं विद्यतेति च}
{ताभ्यां हीनमतोऽन्यत्र पदं नास्तीति चेदपि}


\twolineshloka
{एवमेतन्मतं सर्प ताभ्यां हीनं तु विद्यते}
{यथा शीतोष्णयोर्मध्ये भवेन्नोष्णं न शीतता}


\threelineshloka
{एवं वै सुखदुःखाभ्यां हीनमस्ति पदं क्वचित्}
{एषा मम मतिः सर्प यथा वा मन्यते भवान् ॥सर्प उवाच}
{}


\threelineshloka
{यदि ते वृत्ततो राजन्ब्राह्मणः प्रसमीक्षितः}
{वृथा जातिस्तदायुष्मन्कृतिर्यावन्न विद्यते ॥युधिष्ठिर उवाच}
{}


\twolineshloka
{जातिरत्रमहासर्प मनुष्यत्वे महामते}
{संकरात्सर्ववर्णानां दुष्परीक्ष्येति मे मतिः}


\twolineshloka
{सर्वे सर्वास्वपत्यानि जनयन्ति यदा नराः}
{वाङ्मैथुनमथो जन्म मरणं च समं नृणाम्}


\twolineshloka
{इदमार्षं प्रमाणं च ये यजामह इत्यपि}
{तस्माच्छीलं प्रधानेष्टं विदुर्ये तत्त्वदर्शिनः}


\threelineshloka
{प्राङ्गाभिवर्धनात्पुंसो जातकर्म विधीयते}
{`तथोपनयनं प्रोक्तं द्विजातीनां यथाक्रमम्'}
{तत्रास्य माता सावित्री पिता त्वाचार्य उच्यते}


\twolineshloka
{वृत्त्या शूद्रसमो ह्येष यावद्वेदेन जायते}
{तस्मिन्नेवं मतिद्वैधे मनुः स्वायंभुवोऽब्रवीत्}


\twolineshloka
{कृतकृत्याः सर्ववर्णा यदि वृत्तं न पश्यति}
{संकरस्त्वत्र नागेन्द्र बलवान्प्रसमीक्षितः}


\threelineshloka
{यत्रेदानीं महासर्प संस्कृतं वृत्तमिष्यते}
{तं ब्राह्मणमहं पूर्वमुक्तवान्भुजगोत्तम ॥सर्प उवाच}
{}


\twolineshloka
{श्रुतं विदितवेद्यस्य तव वाक्यं युधिष्ठिर}
{भक्षयेयमहं कस्माद्भ्रातरं ते वृकोदरम्}


\chapter{अध्यायः १८३}
\twolineshloka
{युधिष्ठिर उवाच}
{}


\threelineshloka
{भवानेतादृशो लोके वेदवेदाङ्गपारगः}
{ब्रूहि किं कुर्वतः कर्म भवेद्गतिरनुत्तमा ॥ 1 ॥सर्प उवाच}
{}


\threelineshloka
{पात्रे दत्त्वा प्रियाण्युक्त्वा सत्यमुक्त्वा च भारत}
{अहिंसानिरतः स्वर्गं गच्छेदिति मतिर्मम ॥युधिष्ठिर उवाच}
{}


\threelineshloka
{दानाद्वा सर्प सत्याद्वा किमतो गुरु दृश्यते}
{अहिंसाप्रिययोश्चैव गुरुलाघवमुच्यताम् ॥सर्प उवाच}
{}


\twolineshloka
{दानं च सत्यं तत्त्वं वा अहिंसा प्रियमेव च}
{एषां कार्यगरीयस्त्वाद्दृश्यते गुरुलाघवम्}


\twolineshloka
{क्वचिद्दानप्रयोगाद्धि सत्यमेव विशिष्यते}
{सत्यवाक्याच्च राजेनद्र क्वचिद्दानं विशिष्यते}


\twolineshloka
{एवमेव महेष्वास प्रियवाक्यान्महीपते}
{अहिंसा दृश्यते गुर्वी ततश्च प्रियमिष्यते}


\threelineshloka
{एवमेतद्भवेद्राजन्कार्यापेक्षमनन्तरम्}
{यदभिप्रेतमन्यत्ते ब्रूहि यावद्ब्रवीम्यहम् ॥युधिष्ठिर उवाच}
{}


\threelineshloka
{कथं स्वर्गे गतिः सर्प कर्मणां च फलं ध्रुवम्}
{अशरीरस्य दृश्येत प्रब्रूहि विषयांश्च मे ॥सर्प उवाच}
{}


\twolineshloka
{तिस्रो वै गतयो राजन्परिदृष्टाः स्वकर्मभिः}
{मानुषं स्वर्गवासश्च तिर्यग्योनिश्च तत्रिधा}


\twolineshloka
{तत्र वै मानुषाल्लोकाद्दानादिभिरनादिभिः}
{अहिंसार्थसमायुक्तैः कारणैः स्वर्गमश्नुते}


\twolineshloka
{विपरीतैश्च राजेन्द्र कारणैर्मानुषो भवेत्}
{तिर्यग्योनिस्तथा तात विशेषश्चात्र वक्ष्यते}


\twolineshloka
{क्रामक्रोधसमायुक्तो हिंसालोभसमनवितः}
{मनुष्यत्वात्परिब्रष्टस्तिर्यग्योनौ प्रसूयते}


\twolineshloka
{तिर्यग्योन्याः पृथग्भावो मनुष्यार्थे विधीयते}
{गवादिभ्यस्तथाऽश्वेभ्यो देवत्वमपि दृश्यते}


\twolineshloka
{सोयमेता गतीस्तात जन्तुश्चरति कार्यवान्}
{निम्ने महति चात्मानमवस्थाप्य च वै नृप}


\threelineshloka
{जातो जातश्च बलवान्भुङ्क्ते नाम्नाऽथ देहवान्}
{फलार्थस्तात निष्पृक्तः प्रजालक्षणभावनः ॥युधिष्ठिर उवाच}
{}


\twolineshloka
{शब्दे स्पर्शे च रूपे च तथैव रसगन्धयोः}
{तस्याधिष्ठानमव्यग्रो ब्रूहि सर्प यथातथम्}


\threelineshloka
{किं न गृह्णासि विषयान्युगपत्त्वं महामते}
{एतावदुच्यतां चोक्तं सर्वं पन्नगसत्तम ॥सर्प उवाच}
{}


\twolineshloka
{तदात्मद्रव्यमायुष्मन्देहसंश्रयणान्वितम्}
{करणाधिष्ठितं भोगानुपभुङ्क्ते यथाविथि}


\twolineshloka
{ज्ञानं चैवात्र बुद्धिश्च कमश्च भरतर्षभ}
{तस्य भोगाधिकरणे करणानि निबोध मे}


\twolineshloka
{मनसा तात पर्येति क्रमशो विषयानिमान्}
{विषयायतनत्वेन भूतात्मा क्षेत्रविष्ठितः}


\twolineshloka
{तत्रचापि नरव्याघ्र मनो जन्तोर्विधीयते}
{तस्माद्युगपदत्रास्य ग्रहणं नोपपद्यते}


\twolineshloka
{स आत्मा पुरुषव्याघ्र भ्रुवोरन्तरमाश्रितः}
{बुद्धिं द्रव्येषु सृजति विविधेषु परावराम्}


\threelineshloka
{बुद्धेरुत्तरकालं च वेदना दृश्यते बुधैः}
{एष वै राजशार्दूल विधिः क्षेत्रज्ञभावनः ॥युधिष्ठिर उवाच}
{}


\threelineshloka
{मनसश्चापि बुद्धेश्च ब्रूहि मे लक्षणं परम्}
{एतदध्यात्मविदुषां परं कार्यं विधीयते ॥सर्प उवाच}
{}


\twolineshloka
{बुद्धिरात्मानुगाऽतीव उत्पातेन विधीयते}
{तदाश्रिता हि सा ज्ञेया बुद्धिस्तस्यैषिणी भवेत्}


\twolineshloka
{बुद्धिरुत्पद्यते कार्यान्मनस्तूत्पन्नमेव हि}
{बुद्धेर्गुणविधानेन मनस्तद्गुणवद्भवेत्}


\threelineshloka
{एतद्विशेषणं तात मनोबुद्ध्योर्मयेरितम्}
{त्वमप्यत्राभिसंबुद्धः कथं वा मनय्से स्वयम् ॥युधिष्ठिर उवाच}
{}


\twolineshloka
{अहो बुद्धिमतांश्रेष्ठ शुभा बुद्धिरियं तव}
{विदितं वेदितव्यं ते कस्मान्नहुष पृच्छसि}


\threelineshloka
{सर्वज्ञं त्वां कथं मोह आविशत्स्वर्गवासिनम्}
{एवमद्भुतकर्माणमिति मे संशयो महान् ॥सर्प उवाच}
{}


\twolineshloka
{सुप्रज्ञमपि चेच्छूरमृद्धिर्मोहयते नरम्}
{वर्तमानः सुखे सर्वो नावैतीति मतिर्मम}


\twolineshloka
{सोहमैश्वर्यमोहेन मदाविष्टो युधिष्ठिर}
{पतितः प्रतिसंबुद्धस्त्वां तु संबोधयाम्यहम्}


\twolineshloka
{कृतंकार्यं महाराज त्वया मम परंतप}
{क्षीणः शाप सुकृच्छ्रो मे त्वया संभाष्य साधुना}


\twolineshloka
{अहं हि दिवि दिव्येन विमानन चरन्पुरा}
{अभिमानेन मत्तः सन्कंचिन्नान्यमचिन्तयम्}


\twolineshloka
{ब्रह्मर्षिदेघगन्धर्वयक्षराक्षसपन्नगाः}
{वरं मम प्रयच्छन्ति सर्वे त्रैलोक्यवासिनः}


\twolineshloka
{चक्षुषा यं प्रपश्यामि प्राणिनं पृथिवीपते}
{तस्य तेजो हराम्याशु तद्धि दृष्टेर्बलं मम}


\twolineshloka
{ब्रह्मर्षीणां सरस्रं हि उवाह शिबिकां मम}
{कस मामपनयो राजन्भ्रंशयामास वै श्रियः}


\twolineshloka
{तत्र ह्यगस्त्यः पादेन वहन्स्पृष्टो महामुनिः}
{अगस्त्येन ततोस्म्युक्तो ध्वंस सर्पेति वैरुषा}


\twolineshloka
{ततस्तस्माद्विमानाग्र्यात्प्रच्युतश्च्युतलक्षणः}
{प्रपतन्बुबुधेऽऽत्मानं व्यालीभूतमधोमुखम्}


\twolineshloka
{अयाचं तमहं विप्रं शापस्यान्तो भवेदिति}
{प्रमादात्संप्रमूढस्य भगवन्क्षन्तुमर्हसि}


\twolineshloka
{ततः स मामुवाचेदं प्रपतन्तं कृपानवितः}
{युधिष्ठिरो धर्मराजः शापात्त्वां मोक्षयिष्यति}


\twolineshloka
{अभिमानस्य वै तस्य बलस्य च नराधिप}
{फले क्षीणे महाराज फलं पुण्यमवाप्स्यसि}


\twolineshloka
{ततो मे विस्मयो जातस्तद्दृष्ट्वा तपसो बलम्}
{ब्रह्म च ब्राह्मणत्वं च येन त्वाऽहमचूचुदम्}


\twolineshloka
{सत्यं दमस्तपो योगमहिंसा ज्ञाननित्यता}
{साधकानि सतां पुंसां न जातिर्नकुलं नृप}


\twolineshloka
{अरिष्ट कएष ते भ्राता मुक्तो बीमो महाभुज}
{स्वस्ति तेऽस्तु महाराज गमिष्यामि दिवं पुनः}


\threelineshloka
{`स चायं पुरुषव्याघ्र कालः पुण्य उपागतः}
{तदस्मात्कारणात्पार्थ कार्यं तन्मे महत्कृतम्' ॥वैशंपायन उवाच}
{}


\twolineshloka
{`ततस्तस्मिन्मुहूर्ते तु विमानं कामगामि वै}
{अवपातेन महता तत्रावाप तदुत्तमम्'}


\twolineshloka
{इत्युक्त्वाऽऽजगरं देहं मुक्त्वा स नहुषो नृपः}
{दिव्यं वपुः समास्थाय गतस्त्रिदिवमेव ह}


\twolineshloka
{युधिष्ठिरोपि धर्मात्मा भ्रात्रा भीमेन संगतः}
{धौम्येन सहितः श्रीमानाश्रमं पुनरागमत्}


\twolineshloka
{ततो द्विजेभ्यः सर्वेभ्यः समेतेभ्यो यथातथम्}
{कथयामास तत्सर्वं धर्मराजो युधिष्ठिरः}


\twolineshloka
{तच्छ्रुत्वा ते द्विजाः सर्वे भ्रातरश्चास्य ते त्रयः}
{आसन्सुव्रीडिता राजन्द्रौपदी च यशस्विनी}


\twolineshloka
{ते तु सर्वेद्विजश्रेष्ठाः पाण्डवानां हितेप्सया}
{मैवमित्यब्रुवन्भीमं गर्हयन्तोऽस् साहसम्}


\twolineshloka
{पाण्डवास्तु भयान्पुक्तं प्रेक्ष्य भीमं महाबलम्}
{हर्षमाहारयांचक्रुर्विजह्रुश्च मुदा युताः}


\chapter{अध्यायः १८४}
\twolineshloka
{वैशंपायन उवाच}
{}


\twolineshloka
{निदाघान्तकरः कालः सर्वभूतसुखावहः}
{तत्रैव वसतां तेषां प्रावृट् समभिपद्यत}


\twolineshloka
{छादयन्तो महाघोषाः स्वं दिशश्च बलाहकाः}
{प्रववर्षुर्दिवारात्रमसिताः सततं तदा}


\twolineshloka
{तपात्ययनिकेताश्च शतशोऽथ सहस्रशः}
{अपेतार्कप्रभाजालाः सविद्युन्मण्डप्रभाः}


\twolineshloka
{विरूढशष्पा धरणी मत्तदंशसरीसृपा}
{बभूव पयसा सिक्ता शान्तधूमरजोगणा}


\twolineshloka
{न स्म प्रज्ञायते किंचिदम्भसा समवस्तृते}
{समं वा विषमं वाऽपि नद्यो वा स्तावराणि वा}


\twolineshloka
{क्षुब्धतोया महावेगाः श्वसमाना इवाशुगाः}
{सिन्धवः शोभयांचक्रुः काननानि तपात्यये}


\twolineshloka
{नदतां काननान्तेषु श्रूयन्ते विविधाः स्वनाः}
{वृष्टिभिस्ताड्यमानानां वराहमृगपक्षिणाम्}


\twolineshloka
{स्तोककाः शिखिनश्चैव पुंस्कोकिलगणैः सह}
{मत्ताः परिपतन्ति स्म दर्दुरश्चैव दर्पिताः}


\twolineshloka
{तदा बहुविधाकारा प्रावृण्मेघानुनादिता}
{अभ्यतीता शिवा तेषां चरतां मरुधन्वसु}


\twolineshloka
{क्रौञ्चहंससमाकीर्णा शरत्प्रमुदीताऽभवत्}
{रूढकक्षवनप्रस्था प्रसन्नजलनिम्नगा}


\twolineshloka
{विमलाकाशनक्षत्रा शरत्तेषां शिवाऽभषत्}
{मृगद्विजसमाकीर्णा पाण्डवानां महात्मनाम्}


\twolineshloka
{दृश्यन्ते शान्तरजसः क्षपाजलदशीतलाः}
{ग्रहनक्षत्रसङ्घैश्च सोमेन च विराजिताः}


\twolineshloka
{कुमुदैः पुण्डरीकैश्च शीतवारिधराः शिवाः}
{नदीः पुष्करिणीश्चैव ददृशुः समलंकृताः}


\twolineshloka
{ते वै मुमुदिरे वीराः प्रसन्नसलिलां शिवाम्}
{पश्यन्तो दृढधन्वानः परिपूर्णां सरस्वतीम्}


\twolineshloka
{तेषां पुण्यतमा रात्रिः पर्वसन्धौ स्म शारदी}
{तत्रैव वसतामासीत्कार्तिकी जनमेजय}


\twolineshloka
{पुण्यकृद्भिर्महासत्वैस्तापसैः सह पाण्डवाः}
{तत्सर्वं भरतश्रेष्ठ समूहुर्योगमुत्तमम्}


\twolineshloka
{तमिस्राभ्युदये तस्मिन्धौम्येन सह पाण्डवाः}
{सूतैः पौरोगवैः साकं काम्यकं प्रययुर्वनम्}


\chapter{अध्यायः १८५}
\twolineshloka
{वैशंपायन उवाच}
{}


\twolineshloka
{काम्यकं प्राप्य कौरव्य युधिष्ठिरपुरोगमाः}
{कृतातिथ्या मुनिगणैर्निषेदुः सह कृष्णया}


\twolineshloka
{ततस्तान्परिविश्वस्तान्वसतः पाण्डुनन्दनान्}
{ब्राह्मणा बहवस्तत्र समन्तात्पर्यवारयन्}


\twolineshloka
{अथाब्रवीद्द्विजः कश्चिदर्जुनस्य प्रियः सखा}
{स एष्यति महाबाहुर्वशी शौरिरुदारधीः}


\twolineshloka
{विदिता हि हरेर्यूयमिहायाताः कुरूद्वहाः}
{सदा हि दर्शनाकाङ्क्षी श्रेयोन्वेषी च वो हरिः}


\twolineshloka
{बहुवत्सरजीवी च मार्कण्डेयो महातपाः}
{स्वाध्यायतपसा युक्तः क्षिप्रं युष्मान्समेष्यति}


\twolineshloka
{तथैव ब्रुवतस्तस्य प्रत्यदृश्यत केशवः}
{शैब्यसुग्रीवयुक्तेन रथेन रथिनारः}


\twolineshloka
{मघवानिव पौलोम्या सहितः सत्यभामया}
{उपायाद्देवकीपुत्रो दिदृक्षुः कुरुसत्तमान्}


\twolineshloka
{अवतीर्य रथात्कृष्णो धर्मराजं यथाविधि}
{ववन्दे मुदितो धीमान्भीमं च बलिनां वरम्}


\twolineshloka
{पूजयामास धौम्यं च यमाभ्यामभिवादितः}
{परिष्वज्यमहाभागां द्रौपदीं पर्यसान्त्वयत्}


\twolineshloka
{स दृष्ट्वा फल्गुनं वीरं चिरस्य प्रियमागतम्}
{पर्यष्वजत दाशार्हः पुनः पुनररिंदमः}


\twolineshloka
{तथैव सत्यभामाऽपि द्रौपदीं परिषस्वजे}
{पाण्डवानां प्रियं भार्यां कृष्णस्य महिषी प्रिया}


\twolineshloka
{ततस्ते पाण्डवाः सर्वेसभार्याः सपुरोहिताः}
{आनर्चुः पुण्डरीकाक्षं परिवव्रुश्च सर्वशः}


\twolineshloka
{कृष्णस्तु पार्थेन समेत्य विद्वान्धनंजयेनासुरमर्दनेन}
{बभौ यथा भूतपतिर्महात्मासमेत्य साक्षाद्भगवान्गुहेन}


\twolineshloka
{ततः समस्तानि किरीटमालीवनेषु वृत्तानि गदाग्रजाय}
{उक्त्वा यथावत्पुनरन्वपृच्छ-त्तस्मिन्सुभद्रां च तथाऽभिमन्युम्}


\twolineshloka
{धौम्यं च कृष्णां च युधिष्ठिरं चयमौ च भीमं च दशार्हसिंहः}
{उवाच दिष्ट्या भवतां शिवेनप्राप्तः किरीटी मुदितः कृतास्त्रः}


\twolineshloka
{स पूजयित्वा मधुहा यथाव-त्पार्थांश्च कृष्णां च पुरोहितं च}
{उवाच राजानमभिप्रशंसन्युधिष्ठिरं तेन सहोपविश्य}


\twolineshloka
{धर्मः परः पाण्डव राज्यलाभा-त्तस्यादिमाहुस्तप एव राजन्}
{सत्यार्जवाभ्यां चरता स्वधर्मंजितस्त्वयाऽचं परश्च लोकः}


\twolineshloka
{अधीतमग्रे चरता व्रतानिसम्यग्धनुर्वेदमवाप्य कृत्स्नम्}
{क्षात्रेण धर्मेण वसूनि लब्ध्वासर्वे ह्यवाप्ताः क्रतवः पुराणाः}


\twolineshloka
{न ग्राम्यधर्मेषु रतिस्तवास्तिकामान्न किंचित्कुरुषे नरेन्द्र}
{न चार्तलोभात्प्रजहासि धर्मंतस्मात्प्रभावादसि धर्मराजः}


\twolineshloka
{दानं च सत्यं च तपश्च राजन्क्षमा च शान्तिश्च दमो धृतिश्च}
{अवाप्य राष्ट्राणि वसूनि भोगा-नेषा परा पार्थ सदा रतिस्ते}


\twolineshloka
{हृतं च देशं कुरुजाङ्गलानांकृष्णां सभायामशां च पश्यन्}
{अपेतधर्मव्यवहारवृत्तंसहेत तत्पाण्डव कस्त्वदन्यः}


\twolineshloka
{असंशयं सर्वसमृद्धकामःक्षिप्रं प्रजाः पालयितासि सम्यक्}
{अद्यैव तन्निग्रहणं कुरूणांयदि प्रतिज्ञा भवतः समाप्ता}


\twolineshloka
{प्रोवाच कृष्णामपि याज्ञसेनींदशार्हभर्ता सहितः सुहृद्भिः}
{कदिष्ट्या समग्राऽसि धनंजयेनसमागतेत्येवमुवाच कृष्णः}


\twolineshloka
{कृष्णे धनुर्वेदरतिप्रदाना-स्तवात्मजास्ते शिशवः शुशीलाः}
{सद्भिः सदैवाचरितं समाधिंचरन्ति पुत्रास्तव याज्ञसेनि}


\twolineshloka
{राज्ये नियुक्तैश्च निमन्त्र्यमाणाःपित्रा च कृष्णे तव सोदरैश्च}
{न यज्ञसेनस्य न मातुलानांगृहेषु बाला रतिमाप्नुवन्ति}


\twolineshloka
{आनर्तमेवाभिमुखाः शिवेनगत्वाधनुर्वेदरतिप्रधानाः}
{तवात्मजा वृष्णिपुरं प्रविश्यन दैवतेभ्यः स्पृहयन्ति कृष्णे}


\twolineshloka
{यथा त्वमेवार्हसि तेषु वृत्तंप्रयोक्तुमार्या च यथैव कुन्ती}
{तेष्वप्रमादेन तथा करोतितथैव भूयश्च तथा सुभद्रा}


\twolineshloka
{यथाऽनिरुद्धस्य यथाऽभिमन्यो-र्यथा सुनीथस्य यथैव भानोः}
{तथा विनेता च गतिश्च कृष्णेतवात्मजानामपि रौक्मिणेयः}


\twolineshloka
{गदासिचर्मग्रहणेषु शूरा-नस्त्रेषु शिक्षासु रथाश्वयाने}
{सम्यग्विनेता विनयेदतन्द्री-स्तांश्चाभिमन्युं च सदा कुमारान्}


\twolineshloka
{स चापि सम्यक्प्रणिधाय शिक्षांशस्त्राणइ चैषां विधिवत्प्रदाय}
{तवात्मजानां च तथाऽभिमन्योःपराक्रमैस्तुष्यति रौक्मिणेयः}


\twolineshloka
{यथा विहारं प्रसमीक्षमाणाःप्रयान्ति पुत्रास्तव याज्ञसेनि}
{एकैकमेषामनुयान्ति यत्ररथाश्च यानानि च दन्तिनश्च}


\twolineshloka
{अथाब्रवीद्धर्मराजं तु कृष्णोदशार्हयोधाः कुकुरान्धकाश्च}
{एते निदेशं तव पालयन्त-स्तिष्ठन्तु यत्रेच्छसि तत्र राजन्}


\twolineshloka
{आवर्ततां कार्मुकवेगवाताहलायुधप्रग्रणा मधूनाम्}
{सेना तवार्थेषु नरेन्द्र यत्ताससादिपत्त्यश्वरथा सनागा}


\twolineshloka
{प्रस्थाप्यतां पाण्डव धार्तराष्ट्रःसुयोधनः पापकृतां वरिष्ठः}
{स सानुबन्धः ससुहद्गणश्चभौमस्य सौबाधिपतेश्च मार्गम्}


\twolineshloka
{कामं तथा तिष्ठ नरेन्द्र तस्मि-न्यथा कृतस्ते समयः सभायाम्}
{दाशार्हयोधैस्तु हतारियोधंप्रतीक्षतां नागपुरं प्रभग्नम्}


\twolineshloka
{व्यपेतमन्युर्व्यपनीतपाप्माविहृत् यत्रेच्छसि तत्र कामम्}
{ततः समृद्धिप्रभवं विशोकःप्रपत्स्यसे नागपुरं सराष्ट्रम्}


\twolineshloka
{ततस्तदाज्ञाय मतं महात्मायथावदुक्तं पुरुषोत्तमेन}
{प्रशस्य विप्रेक्ष्य च धर्मराजःकृताञ्जलिः केशवमित्युवाच}


\twolineshloka
{असंशयं केशव पाण्डवानांभवान्गतिस्त्वच्छरणा हि पार्थाः}
{कालोदये तच्च ततश्च भूयःकर्ता भवान्कर्म न संशयोस्ति}


\twolineshloka
{यथाप्रतिज्ञं विहृतश्च कालःसर्वाः समा द्वादश निर्जनेषु}
{अज्ञातचर्यां विधिवत्समाप्यभवद्गताः केशव पाण्डवेयाः}


\threelineshloka
{एषैव बुद्धिर्जुषतां सदा त्वांसत्ये स्थिताः केशव पाण्डवेयाः}
{सदानधर्माः सजलाः सदाराःसबान्धवास्त्वच्छरणा हि पार्थाः ॥वैशंपायन उवाच}
{}


\threelineshloka
{तथा वदतिवार्ष्णेये धर्मराजे च भारत}
{अथ पश्चात्तपोवृद्धो बहुवर्षसहस्रधृक्}
{प्रत्यदृश्यत धर्मात्मा मार्कण्डेयो महातपाः}


\twolineshloka
{अजरश्चामरंश्चैव रूपौदार्यगुणान्वितः}
{व्यदृश्यत तथा युक्तो यथा स्यात्पञ्चविंशकः}


\twolineshloka
{तमागतमृषिं वृद्धं बहुवर्षसहस्रिणम्}
{उपातिष्ठन्त ते सर्वेपाण्डवाः सहयादवाः}


\twolineshloka
{तमर्चितं सुविश्वस्तमासीनमृषिसत्तमम्}
{ब्राह्मणानां मतेनाह पाण्डवानां च केशवः}


\twolineshloka
{शुश्रूपवः पाण्डवास्ते ब्राह्मणाश्च समागताः}
{द्रौपदी सत्यभामा च तथाऽहंपरमं वचः}


\threelineshloka
{पुरावृत्ताः कथाः पुण्याः सदाचारान्सनातनान्}
{राज्ञांस्त्रीणामृषीणां च मार्कणअडेय प्रचक्ष्व नः ॥वैशंपायन उवाच}
{}


\twolineshloka
{तेषु तत्रोपविष्टेषु देवर्षिरपि नारदः}
{आजगाम विशुद्धात्मा पाण्डवानवलोककः}


\twolineshloka
{तमप्यथ महात्मानं सर्वे ते पुरुषर्षभाः}
{पाद्यार्ध्याभ्यां यथान्यायमुपतस्थुर्मनीषिणः}


\twolineshloka
{नारदस्त्वथ देवर्षिर्ज्ञात्वा तांस्तु कृतक्षणान्}
{मार्कण्डेयस्य वदतस्तां कथामन्वमोदत}


\twolineshloka
{उवाच चैनं कालज्ञं स्मयन्निव सनारदः}
{ब्रह्मर्षे कथ्यतां यत्ते पाण्डवेषु विवक्षितम्}


\twolineshloka
{एवमुक्तः प्रत्युवाच मार्कण्डेयो महातपाः}
{क्षणं कुरुध्वंविपुलमाख्यातव्यं भविष्यति}


\twolineshloka
{एवमुक्ताः क्षणं चक्रुः पाण्डवाः सह तैर्द्विजैः}
{मध्यंदिने यथादित्यं प्रेक्षन्तस्ते महामुनिम्}


\chapter{अध्यायः १८६}
\twolineshloka
{वैशंपायन उवाच}
{}


\twolineshloka
{तं विवक्षन्तमालक्ष्यकुरुराजो महामुनिम्}
{कथासंजननार्थाय चोदयामास पाण्डवः}


\twolineshloka
{भवान्दैवतदैत्यानामृषीणां च महात्मनाम्}
{राजर्षीणां च सर्वेषां चरितज्ञः पुरातनः}


\twolineshloka
{सेव्यश्चोपासितव्यश् मतो नः काङ्क्षितश्चिरम्}
{अयं च देवकीपुत्रः प्राप्तोऽस्मानवलोककः}


\twolineshloka
{भ्रमत्येव हि मे बुद्धिर्दृष्ट्वाऽऽत्मानं सुखाच्च्युतम्}
{धार्तराष्ट्रांश्च रदुर्त्तानृद्ध्यतः प्रेक्ष्य सर्वशः}


\twolineshloka
{कर्मणः पुरुषः कर्ता शुभस्याप्यशुभस्य वा}
{स्वफलं तदुपाश्नापि कथं कर्ता स्विदीश्वरः}


\twolineshloka
{अथवा सुखदुःखेषु नृणां ब्रह्मविदांवर}
{इह वा कृतमन्वेति परदेहेऽथवा पुनः}


\twolineshloka
{देही च देहं संत्यज्य मृग्यमाणः शुभाशुभैः}
{कथं संयुज्यतेप्रेत्य इह वा द्विजसत्तम}


\threelineshloka
{ऐहलौकिकमेवैतदुताहो पारलौकिकम्}
{क्व वा कर्माणि तिष्ठन्ति जन्तोः प्रेतस्य भार्गव ॥मार्कण्डेय उवाच}
{}


\twolineshloka
{त्वद्युक्तोऽयमनुप्रश्नो यथावद्वदतांवर}
{विदितं वेदितव्यं ते स्थित्यर्थं त्वं तु पृच्छसि}


\twolineshloka
{अत्र ते कथयिष्यामि तदिहैकमनाः शृणु}
{यथेहामुत्र च नर सुखदुःखमुपाश्नुते}


\twolineshloka
{निर्मलानि शरीराणि विशुद्धानि शरीरिणाम्}
{ससर्ज धऱ्मतन्त्राणि पूर्वोत्पन्नः प्रजापतिः}


\twolineshloka
{अमोघफलसंकल्पाः सुव्रताः सत्यवादिनः}
{ब्रह्मभूता नराः पुण्याः पुराणाः कुरुसत्तम}


\twolineshloka
{सर्वे देवैः समायान्ति स्वच्छन्देन नभस्तलम्}
{ततश्च पुनरायान्ति सर्वे स्वच्छन्दचारिणः}


\twolineshloka
{स्वच्चन्दमरणाश्चासन्नराः स्वच्छन्दजीविनः}
{अल्पबाधा निरातङ्काः सिद्धार्था निरुपद्रवाः}


\twolineshloka
{द्रष्टारोदेवसङ्घानामृषीणां च महात्मनाम्}
{प्रत्यक्षाः सर्वधर्माणां दान्ता विगतमत्सराः}


\twolineshloka
{आसन्वर्षसहस्राणि तथा पुत्रसहस्रिणः}
{ततः कालान्तरे तस्मिन्पृथिवीतलचारिणः}


\twolineshloka
{कामक्रोधाभिभूतास्ते मायाव्याजोपजीविनः}
{लोभमोहाभिभूताश्च त्यक्ता देवैस्ततो नराः}


\twolineshloka
{अशुभैः कर्मभिः पापास्तिर्यङ्गिरयगामिनः}
{संसारेषु विचित्रेषु पच्यमानाः पुनः पुनः}


\twolineshloka
{मोघेष्टा मोघसंकल्पा मोघज्ञाना विचेतसः}
{`काङ्क्षिणः सर्वकामानां नास्तिका भिन्नसेतवः'}


\twolineshloka
{सर्वाभिशङ्खिनश्चैव संवृत्ताः क्लेदायिनः}
{अशुभैः कर्मभिश्चापि प्रायशः परिचिह्निताः}


\threelineshloka
{दौष्कुल्या व्याधिबहुला दुरात्मानोऽभितापिनः}
{भवन्त्यल्पायुषः पापा रौद्रकर्मफलोदयाः}
{नाथन्तः सर्वकामानां नास्तिका भिन्नचेतसः}


\twolineshloka
{जन्तोः प्रेतस्य कौन्तेय गतिः स्वैरिह कर्मभिः}
{प्राज्ञस्य हीनबुद्धेश्च कर्मकोशः क्व तिष्ठति}


\twolineshloka
{क्वस्थस्तत्समुपाश्नाति सुकृतं यदि वेतरत्}
{इति ते दर्शनं यच्च तत्राप्यनुनयं शृणु}


\twolineshloka
{अयमादिशरीरेण देवसृष्टेन मानवः}
{शुभानामशुभानां च कुरुते संचयं महत्}


\twolineshloka
{आयुषोऽन्ते प्रहायेदं क्षीणप्रायं कलेबरम्}
{संभवत्येव युगपद्योनौ नास्त्यन्तराऽभवः}


\twolineshloka
{तत्रास्य स्वकृतं कर्म च्छायेवानुगतं सदा}
{फलत्यथ सुखार्हो वा दुःखार्होवाऽथ जायते}


\twolineshloka
{कृतान्तविधिसंयुक्तः स जन्तुर्लक्षणैः शुभैः}
{अशुभैर्वा निरादानो लक्ष्यते ज्ञानदृष्टिभिः}


\twolineshloka
{एषा तावदबुद्धीनां गतिरुक्ता युधिष्ठिर}
{अतः परं ज्ञानवतां निबोध गतिमुत्तमाम्}


\twolineshloka
{मनुष्यास्तप्ततपसः सर्वागमपरायणाः}
{स्थिरव्रताः सत्यपरा गुरुशुश्रूषणे रताः}


\twolineshloka
{सुशीलाः शुक्लजातीयाः क्षान्ता दान्ताः सुतेजसः}
{शुचियोन्यन्तरगताः प्रायशः शुभलक्षणाः}


\twolineshloka
{जितेनद्रियत्वाद्वशिनः सत्कृत्यान्मन्दरागिणः}
{अल्पाबाधपरित्रासा भवन्ति निरुपद्रवाः}


\twolineshloka
{च्यवन्तं जायमानं च गर्भस्यं चैव सर्वशः}
{स्वमात्मानं परं चैव बुध्यन्ते ज्ञानचक्षुषः}


\threelineshloka
{ऋषयस्ते महात्मानः प्रत्यक्षागमबुद्धयः}
{कर्मभूमिमिमां प्राप्य पुनर्यान्ति सुरालयम्}
{`कृत्वा शुभानि कर्माणि ज्ञानेन भरतर्षभ'}


\twolineshloka
{किंचिद्दैवाद्धठात्किंचित्किंचिदेव स्वकर्मभिः}
{प्राप्नुवन्ति नरा राजन्मा तेऽस्त्वन्याविचारणा}


\twolineshloka
{इमामत्रोपमां चापि निबोध वदतांवर}
{मनुष्यलोके यच्छ्रेयः परंमन्ये युधिष्ठिर}


\twolineshloka
{इहैवैकस्य नामुत्र अमुत्रैकस्य नो इह}
{इह चामुत्र चैकस् नामुत्रैकस्य नो इह}


\twolineshloka
{धनानि येषां विपुलानि सन्तिनित्यंरमन्ते सुविभूषिताङ्गाः}
{तेषामयं शत्रुवरघ्नलोकोनासौ सदा ग्राम्यसुखे रतानाम्}


\twolineshloka
{ये योगयुक्तास्तपसि प्रसक्ताःस्वाध्यायशीला जरयन्ति देहान्}
{जितेन्द्रिया भूतहिते निविष्टास्तेषामसौ नायमरिघ्नलोकः}


\twolineshloka
{ये धर्ममेव प्रथमं चरन्तिधर्मेण लब्ध्वा च धनानि काले}
{दारानवाप्य क्रतुभिर्यजन्तेतेषामयं चैव परश्च लोकः}


\twolineshloka
{ये नैव विद्यां न तपो न दानंन चापि मूढाः प्रजने यतन्ते}
{न चापिगच्छन्ति शुभान्यभाग्या-स्तेषामयं चैव परश्च नास्ति}


\twolineshloka
{सर्वे भवन्तस्त्वतिवीर्यसत्वादिव्यौजसः संहननोपपन्नाः}
{लोकादमुष्मादवनिं प्रपन्नाःस्वधीतविद्याः सुरकार्यहेतोः}


\twolineshloka
{कृत्वैव कर्माणइ महान्ति शूरा-स्तपोदमाचारविहारशीलाः}
{देवानृषीन्प्रेतगतांसच् सर्वा-न्संतर्पयित्वा विधिना परेण}


\twolineshloka
{स्वर्गं परं पुण्यकृतां निवासंक्रमेण संप्राप्स्यथ कर्मभिः स्वैः}
{माभूद्विशङ्का तव कौरवेन्द्रदृष्ट्वाऽऽत्मनः क्लेशमिमं सुखार्थम्}


\chapter{अध्यायः १८७}
\twolineshloka
{वैशंपायन उवाच}
{}


\twolineshloka
{मार्कण्डेयं महात्मानमूचुः पाण्डुसुतास्तदा}
{माहात्म्यं द्विजमुख्यानां श्रोतुमिच्छाम कथ्यताम्}


\twolineshloka
{एवमुक्तः स भगवान्मार्कण्डेयो महातपाः}
{उवाच सुमहातेजाः सर्वशास्त्रविशारदः}


\twolineshloka
{हेहयानां कुलकरो राजा परपुरंजयः}
{कुमारो रूपसंपननो मृगयां व्यचरद्बली}


\twolineshloka
{चरमाणस्तु सोऽरण्ये तृणवीरुत्समावृते}
{कृष्णाजिनोत्तरासङ्गं ददर्श मुनिमन्तिके}


\twolineshloka
{स तेन हिंसितोऽरण्ये मन्यमानेन वै मृगम्}
{व्यथितः कर्म तत्कृत्वा शोकोपहतचेतनः}


\twolineshloka
{जगाम हेहयानां वै सकाशं प्रथितात्मनाम्}
{राज्ञां राजीवनेत्रोसौ कुमारः पृथिवीपते}


\twolineshloka
{तेषां च तद्यथावृत्तं कथयामास वै तदा}
{`स कुमारो महीपालो हेहयानां महीभृताम्'}


\twolineshloka
{तं चापि हिंसितं तात मुनिं मूलफलाशिनम्}
{श्रुत्वा दृष्ट्वा च ते तत्र बभूवुर्दीनमानसाः}


\twolineshloka
{कस्यायमिति ते सर्वे मार्गमाणास्ततस्ततः}
{जग्मुश्चारिष्टनाम्नोऽथ तार्क्ष्यस्याश्रममञ्जसा}


\twolineshloka
{तेऽभिवाद्य महात्मानं तं मुनिं संशितव्रतम्}
{तस्थुः सर्वे स तु मुनिस्तेषां पूजामथाहरत्}


\twolineshloka
{ते तमूचुर्महात्मानं न वयं सक्रियां मुने}
{त्वत्तोऽर्हाः कर्मदोषेण ब्राह्मणो हिंसितो हि नः}


\twolineshloka
{तानब्रवीत्स विप्रर्षिः कथं वो ब्राह्मणो हतः}
{क्व चासौ ब्रूत सहिताः पश्यध्वं मे रतपोबलम्}


\twolineshloka
{ते तु तत्सर्वमखिलमाख्यायास्मै यथातथम्}
{नापश्यंस्तमृषिं तत्र गतासुं ते समागताः}


\twolineshloka
{अन्वेषमाणाः सव्रीडाः सुप्तवद्गतमानसाः}
{तानब्रवीत्तत्रमुनिस्तार्क्ष्यः परपुरंजयः}


\twolineshloka
{स्यादयं ब्राह्मणः सोऽथयुष्माभिर्यो विनाशितः}
{पुत्रो ह्ययं मम नृपास्तपोबलसमन्वितः}


\twolineshloka
{ते च दृष्ट्वैव तमृषिं विस्मयं परमं गताः}
{महदाश्चर्यमिति वै ते ब्रुवाणा महीपते}


\threelineshloka
{मृतो ह्ययमितो दृष्टः कथं जीवितमाप्तवान्}
{किमेतत्तपसो वीर्यं येनायं जीवितः पुनः}
{श्रोतुमिच्छामहे विप्र यदि श्रोतव्यमित्युत}


\threelineshloka
{स तानुवाच नास्माकं मृत्युः प्रभवते नृपाः}
{कारणं च प्रवक्ष्यामि हेतुयोगं समासतः}
{`मृत्युः प्रभवते येन नास्माकं नृपसत्तमाः}


\twolineshloka
{शुद्धाचारादनलसाः साध्योपासनतत्पराः}
{शुद्धान्नाः शुद्धसदना ब्रह्मचर्यव्रतान्विताः'}


\twolineshloka
{सत्यमेवाभिजानीमो नानृते कुर्महे मनः}
{स्वधर्ममनुतिष्ठामस्तस्मान्मृत्युभयं न नः}


\twolineshloka
{यद्ब्राह्मणानां कुशलं तदेषां कथयामहे}
{तेषां हि चरितं ब्रूमस्तस्मान्मृत्युभयं न नः}


\twolineshloka
{अतिथीनन्नपानेन भृत्यानत्यशनेन च}
{संभोज्य शेषमश्नीमस्तस्मान्मृत्युभयं न नः}


\threelineshloka
{क्षान्ता दान्ताः क्षमाशीलास्तीर्थदानपरायणाः}
{पुण्यदेशनिवासाच्च तस्मान्मृत्युभयं न नः}
{तेजस्विदेशवासाच्च तस्मान्मृत्युभयं न नः}


\twolineshloka
{एतद्वै लेशमात्रं वः समाख्यातं विमत्सराः}
{गच्छध्वं सहिताः सर्वे न पापाद्भयमस्ति वः}


\twolineshloka
{एवमस्त्विति ते सर्वे प्रतिपूज्य महामुनिम्}
{स्वदेशमगमन्हृष्टा राजानो भरतर्षभ}


\chapter{अध्यायः १८८}
\twolineshloka
{मार्कण्डेय उवाच}
{}


\twolineshloka
{भूय एव तु माहात्म्यं ब्राह्मणआनां निबोध मे}
{वैन्यो नामेह राजर्षिरश्वमेधाय दीक्षितः}


\twolineshloka
{तमत्रिर्गन्तुमारेबे वित्तार्थमिति नः श्रुतम्}
{भूयोऽर्थं नानुरुध्यत्स धर्मव्यक्तिनिदर्शनात्}


\twolineshloka
{स विचिन्त्य महातेजा वनमेवान्वरोचयत्}
{धर्मपत्नीं समाहूय पुत्रांश्चेदमुवाच ह}


\twolineshloka
{प्राप्स्यामः फलमत्यन्तं बहुलं निरुपद्रवम्}
{अरण्यगमनं क्षिप्रं रोचतां वो गुणाधिकम्}


\twolineshloka
{तं भार्या प्रत्युवाचाथ धनमेवानुरुन्धती}
{वैन्यं गत्वा महात्मानमर्थयस्व धनं बहुः}


\twolineshloka
{स ते दास्यति राजर्षिर्यजमानोऽर्थिने धनम्}
{तत आदाय विप्रर्षे प्रतिगृह्य धनं बहु}


\threelineshloka
{भृत्यान्सुतान्संविभज्य ततो व्रज यथेप्सितम्}
{एष वै परमो धर्मो धर्मविद्भिरुदाहृतः ॥अत्रिरुवाच}
{}


\twolineshloka
{कथितो मे महाभागे गौतमेन महात्मना}
{वैन्यो धर्मार्थसंयुक्तः सत्यव्रतसमन्वितः}


\twolineshloka
{किंत्वस्ति तत्र द्वेष्टारो निवसन्ति हि मे द्विजाः}
{यथा मे गौतमः प्राह ततो न व्यवसाम्यहम्}


\twolineshloka
{तत्र स्म वाचं कल्याणीं धर्मकामार्थसंहिताम्}
{मयोक्तामन्यथा ब्रूयुस्ततस्ते वै निरर्थिकाम्}


\twolineshloka
{गमिष्यामि महाप्राज्ञे रोचते मे वचस्तव}
{गाश्च मे दास्यते वैन्यः प्रभूतं चार्थसंचयम्}


\twolineshloka
{एवमुक्त्वा जगामाशु वैन्ययज्ञं महातपाः}
{गत्वा च यज्ञायतनमत्रिस्तुष्टाव तं नृपम्}


\threelineshloka
{वाक्यैर्मङ्गलसंयुक्तैः पूजयानोऽब्रवीद्वचः}
{राजन्धन्यस्त्वमीशश्च भुवि त्वं प्रथमो नृपः}
{स्तुवन्ति त्वां मुनिगणास्त्वदन्यो नास्ति धर्मवित्}


\twolineshloka
{तमब्रवीदृषिः क्रुद्धो वचनं वै महातपाः ॥गौतम उवाच}
{}


\threelineshloka
{मैवमत्रे पुनर्ब्रूया न ते प्रज्ञा समाहिता}
{अत्र नः प्रथमो धाता महेन्द्रो वै प्रजापतिः ॥वैशंपायन उवाच}
{}


\fourlineindentedshloka
{अथात्रिरपि राजेन्द्र गौतमं प्रत्यभाषत}
{अयमेव विधाता च तथैवेन्द्रः प्रजापतिः}
{त्वमेव मुह्यसे मोहान्न प्रज्ञानं तवास्ति ह ॥गौतम उवाच}
{}


\twolineshloka
{जानामि नाहं मुह्यामि त्वंविवक्षुर्विमुह्यसे}
{स्तौषि त्वं दर्शनप्रेप्सू राजानं जनसंसदि}


\threelineshloka
{न वेत्थ परमं धर्मं न चास्त्यत्र प्रयोजनम्}
{बालस्त्वमसि मूढश्च वृद्धः केनापि हेतुना ॥वैशंपायन उवाच}
{}


\twolineshloka
{विवदन्तौ तथा तौ तु मुनीनां द्रशने स्थितौ}
{ये तस्य यज्ञे संवृत्तास्तेऽपृच्छन्त कथं त्विमौ}


\twolineshloka
{प्रवेशः केन दत्तोऽयमनयोर्वैन्यसंसदि}
{उच्चैः समभिभाषन्तौ केन कार्येण धिष्ठितौ}


\twolineshloka
{ततः परमधर्मात्मा काश्यपः सर्वधऱ्मवित्}
{विवादिनावनुप्राप्तौ तावुभौ प्रत्यवेदयत्}


\threelineshloka
{अथाब्रवीत्सदस्यांस्तु गौतमो मुनिसत्तमान्}
{आवयोर्व्याहृतं प्रश्नं शृणुत द्विजसत्तमाः}
{वैन्यं विधातेत्याहात्रिरत्र नौ संशयो महान्}


\twolineshloka
{`ततस्तु गौतमेनोक्तं वाक्यं वैन्यस्य संसदि'}
{श्रुत्वैव तु महात्मानो मुनयोऽभ्यद्रवन्दुतम्}


\twolineshloka
{सनत्कुमारं धर्मज्ञं संशयच्छेदनाय वै}
{`पप्रच्छुः प्रणताः सर्वे ब्रह्माणमिव सोमपाः'}


\twolineshloka
{स च तेषां वचः श्रुत्वा यथातत्त्वं महातपाः}
{प्रत्युवाचाथ तानेवं धर्मार्थसहितं वचः}


\twolineshloka
{ब्रह्म क्षत्रेण सहितं क्षत्रं च ब्रह्मणा सह}
{संयुक्तौ दहतः शत्रून्वनानीवाग्निमारुतौ}


\twolineshloka
{राजा वै प्रथितो धर्मः प्रजानां पतिरेव च}
{स एव शक्रः शुक्रश्च स धाता च बृहस्पतिः}


\twolineshloka
{प्रजापतिर्विराट् सम्राट् क्षत्रियो भूपतिर्नृपः}
{य एभिः स्तूयते शब्दैः कस्तं नार्चितुमर्हति}


\threelineshloka
{पुरायोनिर्युधाजिच्च अभीष्टानचितो भवः}
{स्वर्णेता सहजिद्बभ्रुरिति राजाऽभिधीयते}
{सत्ययोनिः पुराविच्च सत्यधर्मप्रवर्तकः}


\twolineshloka
{अधर्मादृषयो भीता बलं क्षत्रे समादधन्}
{`तस्माद्धि ब्रह्मणा क्षत्रं क्षत्रेण ब्रह्म चाव्ययम्'}


\twolineshloka
{आदित्यो दिवि देवेषु तमो नुदति तेजसा}
{तथैव नृपतिर्भूमावधर्मान्नुदते भृशम्}


\threelineshloka
{ततो राज्ञः प्रधानत्वं शास्त्रप्रामाण्यदर्शनात्}
{उत्तरः सिद्ध्यते पक्षो येन राजेति भाषितम् ॥मार्कण्डे उवाच}
{}


\twolineshloka
{ततः स राजा संहृष्टः सिद्धे पक्षे महामनाः}
{तमत्रिमब्रवीत्प्रीतः पूर्वं येनाभिसंस्तुतः}


\twolineshloka
{यस्मात्पूर्वं मनुष्येषु ज्यायांसं मामिहाब्रवीः}
{सर्वदेवैश्च विप्रर्षे संमितं श्रेष्ठमेव च}


\twolineshloka
{तस्मात्तेऽहंप्रदास्यामि विविधं वसु भूरि च}
{दासीसहस्रं श्यामानां सुवस्त्राणामलकृतम्}


\twolineshloka
{दशकोटीर्हिरण्यस्य रुक्मभारांस्तथा दश}
{एतद्ददामि विप्रर्षे सर्वज्ञस्त्वं मतो हि मे}


\twolineshloka
{तदत्रिनर्न्यायतः सर्वं प्रतिगृह्याभिसत्कृतः}
{प्रत्युज्जगाम तेजस्वी गृहानेव महातपाः}


\twolineshloka
{प्रदाय च धनं प्रीतः पुत्रेभ्यः प्रयतात्मवान्}
{ततः समभिसंधाय वनमेवान्वपद्यत}


\chapter{अध्यायः १८९}
\twolineshloka
{मार्कण्डेय उवाच}
{}


\threelineshloka
{अत्रैव च सरस्वत्या गीतं परपुरंजय}
{पृष्टया मुनिना वीर शृणु तार्क्ष्येण धीमता ॥तार्क्ष्य उवाच}
{}


\twolineshloka
{किंन श्रेयः पुरुषस्येह भद्रेकथं कुर्वन्न च्यवते स्वधर्मात्}
{आचक्ष्व मे चारुसर्वाङ्गि सर्वंत्वया शिष्टो न च्यवेयं स्वधर्मात्}


\threelineshloka
{कथं चाग्निं जुहुयां पूजये वाकस्मिन्काले केन धर्मो न नश्येत्}
{एतत्सर्वं सुभगे प्रब्रवीहियथा लोकान्विरजाः संचरेयम् ॥मार्कण्डेय उवाच}
{}


\twolineshloka
{एवं पृष्टा प्रीतियुक्तेन तेनशुश्रूषुणा चोत्तमबुद्दियुक्तम्}
{तार्क्ष्यं विप्रं धर्मयुक्तं हितं चसरस्वती वाक्यमिदं बभाषे}


\twolineshloka
{यो ब्रह्म जानाति यथोपदेशंस्वाध्यायनित्यः शुचिरप्रमत्तः}
{स वै पुरो देवलोकस्य गन्तासहामरैः प्राप्नुयात्प्रीतियोगम्}


\twolineshloka
{तत्र स्म रम्या विपुला विशोकाःसुपुष्पिताः पुष्करिण्यः सुपुण्याः}
{अकर्दमा मीनवत्यः सुतीर्थाहिरण्मयैरावृताः पुण्डरीकैः}


\twolineshloka
{तासां तीरेष्वासते पुण्यभाजोमहीयमानाः पृथगप्सरोभिः}
{सुपुण्यगन्धाभिरलंकृताभि-र्हिरण्यवर्णाभिरतीव हृष्टाः}


\twolineshloka
{परं लोकं गोप्रदास्त्वाप्नुवन्तिदत्त्वाऽनड्वाहं सूर्यलोकं व्रजन्ति}
{वासो दत्त्वा चान्द्रमसं तु लोकंदत्त्वा हिरण्यममरत्वमेति}


\twolineshloka
{धेनुं दत्त्वा सुप्रमां साधुदोहांकल्याणवत्सामपलायिनीं च}
{यावन्ति रोमाणि भवन्ति तस्या-स्तावद्वर्षाण्यासते देवलोके}


\twolineshloka
{अनड्वाहं सुव्रतं यो ददातिहलस्य वोढारमनन्तवीर्यम्}
{धुरंधरं बलवन्तं युवानंप्राप्नोति लोकान्दशधेनुदस्य}


\twolineshloka
{ददाति यो वै कपिलां सचैलांकांस्योपदोहां द्रविणैरुत्तरीयैः}
{तैस्तैर्गुणैः कामदुहाऽथ भूत्वानरं प्रदातारमुपैति नाकैः}


\twolineshloka
{यावन्ति रोमाणि भवन्ति धेन्वा-स्तावत्फलं लभते गोप्रदाने}
{पुत्रांसच् पौत्रांश्च कुलं च सर्व-मासप्तमं तारयते परत्र}


\twolineshloka
{सदक्षिणां काञ्चनचारुशृङ्गींकांस्योपदोहां द्रविणैरुत्तरीयैः}
{धेनुं तिलानां ददतो द्विजायलोका वसूनां सुलभा भवन्ति}


\twolineshloka
{स्वकर्मभिर्दानवसंनिरुद्धेतीव्रान्धकारे नरके पतन्तम्}
{महार्णवे नौरिव वातयुक्तादानं गवां तारयते परत्र}


\twolineshloka
{यो ब्राह्मदेयां तु ददाति कन्यांभूमिप्रदानं च करोति विप्रे}
{ददाति दानं विधिना च यश्चस लोकमाप्नोति पुरंदरस्य}


\threelineshloka
{यः सप्तवर्षाणि जुहोति तार्क्ष्यहव्यं त्वग्नौ नियतः साधुशीलः}
{सप्तावरान्सप्तपूर्वान्पुनातिपितामहानात्मना कर्मभिः स्वैः ॥तार्क्ष्य उवाच}
{}


\threelineshloka
{किमग्निहोत्रस्य व्रतं पुराण-माचक्ष्व मे पृच्तश्चारुरूपे}
{त्वयाऽनुशिष्टोऽहमिहाद्य विद्यांयदग्निहोत्रस्य व्रतं पुराणम् ॥सरस्वत्युवाच}
{}


\twolineshloka
{न चाशुचिर्नाप्यनिर्णिक्तपाणि-र्नाब्रह्मविज्जुहुयान्नाविपश्चित्}
{बुभुत्सवः शुचिकामा हि देवानाश्रद्दधानाद्धि हविर्जुषन्ति}


\twolineshloka
{नाश्रोत्रियं देवहव्ये नियुञ्ज्या-न्मोघं पुरा सिञ्चति तादृशो हि}
{अपूर्णमश्रोत्रियमाह तार्क्ष्यन वै तादृग्जुहुयादग्निहोत्रम्}


\threelineshloka
{रये वै कृशानुं जुह्वति श्रद्दधानाःसत्यव्रता हुतशिष्टाशिनश्च}
{गवां लोकं प्राप्यते पुण्यगन्धंपश्यन्ति देवं परमं चापि सत्यम् ॥तार्क्ष्य उवाच}
{}


\threelineshloka
{क्षेत्रज्ञभूतां परलोकभावेकर्मोदये बुद्दिमभिप्रविष्टाम्}
{प्रज्ञां च देवीं सुभगे विमृश्यपृच्छामि त्वां का ह्यसि चारुरूपे ॥सरस्वत्युवाच}
{}


\threelineshloka
{अग्निहोत्रादहमभ्यागताऽस्मिविप्रर्षभाणां संशयच्छेदनाय}
{त्वत्संयोगादहमेतदब्रुवंभावे स्थिता तथ्यमर्थं यथावत् ॥तार्क्ष्य उवाच}
{}


\threelineshloka
{न हि त्वया सदृशी काचिदस्तिविभ्राजसे ह्यतिमात्रं यथा श्रीः}
{रूपं च ते दिव्यमनन्तकान्तिप्रज्ञां च देवीं सुभगे बिभर्षि ॥सरस्वत्युवाच}
{}


\twolineshloka
{श्रेष्ठानि यानि द्विपदांवरिष्ठयज्ञेषु विद्वन्नुपपादयन्ति}
{तैरेव चाहं संप्रवृद्धा भवामिचाप्यायिता रूपवती च विप्र}


\threelineshloka
{यच्चापि पात्रमुपयुज्यते हवानस्पत्यमायसं पार्थिवं वा}
{दिव्येन रूपेण प्रज्ञया चतेनैव सिद्धिरिति विद्धि विद्वन् ॥तार्क्ष्य उवाच}
{}


\fourlineindentedshloka
{इदं श्रेयः परमं मन्यमानाव्यायच्छन्ते मुनयः संप्रतीताः}
{आचक्ष्व मे तं परमं विशोकंमोक्षं परं यं प्रविशन्ति धीराः}
{साङ्ख्या योगाः परमं यं विदन्तिपरं पुराणं तमहं न वेद्मि ॥सरस्वत्युवाच}
{}


\twolineshloka
{तं वै परं वेदविदः प्रपन्नाःपरं परेभ्यः प्रथितं पुराणम्}
{स्वाध्यायदानव्रतपुण्ययोगै-स्तपोधना वीतशोका विमुक्ताः}


\twolineshloka
{तस्याथ मध्ये वेतसः पुण्यगन्धःसहस्रशाखो विपुलो विभाति}
{तस्य मूलात्सरितः प्रस्रवन्तिमधूदकप्रस्रवणाः सुपुण्याः}


\twolineshloka
{शाखांशाखां महानद्यः संयान्ति सिकताशयाः}
{धानापूपा मांसशाकाः सदापायसकर्दमाः}


\twolineshloka
{यस्मिन्नग्निमुखा देवाः सेन्द्राः सहमरुद्गणाः}
{ईजिरे क्रतुभिः श्रेष्ठैस्तत्पदं परमं मुने}


\chapter{अध्यायः १९०}
\twolineshloka
{वैशंपायन उवाच}
{}


\threelineshloka
{ततः स पाण्डवो भूयो मार्कण्डेयमुवाच ह}
{कथयस्वति चरितं मनोर्वैवस्वतस्य च ॥मार्कण्डेय उवाच}
{}


\twolineshloka
{विवस्वतः सुतो राजन्महर्षिः सुप्रतापवान्}
{बभूव नरशार्दूल प्रजापतिसमद्युतिः}


\twolineshloka
{ओजसातेजसा लक्ष्म्या तपसा च विशेषतः}
{अतिचक्राम पितरं मनुः स्वंच पितामहम्}


\twolineshloka
{ऊर्ध्वबाहुर्विशालायां बदर्यां स नराधिपः}
{एकपादस्थितस्तीव्रं चकार सुमहत्तपः}


\twolineshloka
{अवाकूशिरास्तथा चापि नेत्रैरनिमिषैर्दृढम्}
{सोऽतप्यत तपो घोरं वर्षाणामयुतं तदा}


\twolineshloka
{तं कदाचित्तपस्यन्तमार्द्रचीरजटाधरम्}
{चीरिणीतीरमागम्य मत्स्यो वचनमब्रवीत्}


\twolineshloka
{भगवन्क्षुद्रमत्स्येस्मि बलवद्भ्यो भयंमम}
{मत्स्येभ्यो हि ततो मां त्वं त्रातुमर्हसि सुव्रत}


\twolineshloka
{दुर्बलं बलवन्तो हि मत्स्या मत्स्यं विशेषतः}
{भक्षयन्ति सदा वृत्तिर्विहिता नः सनातनी}


\twolineshloka
{तस्माद्भयौघान्महतो मज्जन्तं मां विशेषतः}
{त्रातुमर्हसि कर्तास्मि कृतेप्रतिकृतं तव}


\twolineshloka
{स मत्स्यवचनं श्रुत्वा कृपयाऽभिपरिप्लुतः}
{मनुर्वैवस्वतोऽगृह्णात्तं मत्स्यं पाणिना स्वयम्}


\twolineshloka
{उदकान्तमुपानीय मत्स्यं वैवस्वतो मनुः}
{अलिञ्जरे प्राक्षिपत्तं चन्द्रांशुसदृशप्रभम्}


\twolineshloka
{स तत्र ववृधे राजन्मत्स्यः परमसत्कृतः}
{पुत्रवत्स्वीकरोत्तस्मै मनुर्भावं विशेषतः}


\twolineshloka
{अथ कालेन महता स मत्स्यः सुमहानभूत्}
{अलिञ्जरजले चैव नासौ समभवत्किल}


\twolineshloka
{अथ मत्स्यो मनुं दृष्ट्वा पुनरेवाभ्यभाषत}
{भगवन्साधु मेऽद्यान्यत्स्थानं संप्रतिपादय}


\twolineshloka
{उद्धृत्यालिञ्जरात्तस्मात्ततः स भगवान्मनुः}
{तं मत्स्यमनयद्वापीं महतीं स मनुस्तदा}


\twolineshloka
{तत्रतं प्राक्षिपच्चापि मनुः परपुरंजय}
{अथावर्धत मत्स्यः स पुनर्वर्षगणान्बहून्}


\twolineshloka
{द्वियोजनायता वापी विस्तृताचापि योजनम्}
{तस्यां नासौ समभवन्मत्स्यो राजीवलोचन}


\twolineshloka
{विचेष्टितुं च कौन्तेय मत्स्यो वाप्यां विशांपते}
{मनुं मत्स्यस्ततो दृष्ट्वा पुनरेवाभ्यभाषत}


\twolineshloka
{नय मां भगवन्साधो समुद्रमहिषीं प्रियाम्}
{गङ्गां नांत्रापि शक्तोस्मि वस्तुं मतिमतांवर}


\twolineshloka
{निदेशे हि मया तुभ्यं स्थातव्यमनसूयता}
{वृद्धिर्हि परमा प्राप्ता त्वत्कृते हि मयाऽनघ}


\twolineshloka
{एवमुक्तो मनुर्मत्स्यमनयद्भगवान्वशी}
{नदीं गङ्गां तत्रचैनं स्वयं प्राक्षिपदेव च}


\twolineshloka
{स तत्र ववृधे मत्स्यः कंचित्रालमरिंदम}
{गतः पुनर्मनुं दृष्ट्वा मत्स्यो वचनमब्रवीत्}


\twolineshloka
{गङ्गायां हि न शक्नोमि बृहत्त्वाच्चेष्टितुं प्रभो}
{समुद्रं नय मामाशु प्रसीद भगवन्निति}


\twolineshloka
{उद्धृत्य गङ्गासलिलात्ततो मत्स्यं मनुः स्वयम्}
{समुद्रमनयत्पार्त तत्रचैनमवासृजत्}


\twolineshloka
{सुमहानपि मत्स्यस्तु स मनोर्नयतस्तदा}
{आसीद्यथेष्टहार्यश्च स्पर्शगन्धसुखश्च वै}


\twolineshloka
{यदा समुद्रे प्रक्षिप्तः स मत्स्यो ननुना तदा}
{तत एनमिदं वाक्यं स्मयमान इवाब्रवीत्}


\twolineshloka
{भगवन्हि कृता रक्षा त्वया सर्वा विशेषतः}
{प्राप्तकालं तु यत्कार्यं त्वया तच्छ्रूयतां मम}


\twolineshloka
{अचिराद्भगवन्बौममिदं स्थावरजङ्गमम्}
{सर्वमेव महाभाग प्रलयं वै गमिष्यति}


\twolineshloka
{संप्रक्षालनकालोऽयंलोकानां समुपस्थितः}
{तस्मात्त्वां बोधयाम्यद्ययत्ते हितमनुत्तमम्}


\twolineshloka
{त्रसानां स्तावराणां च यच्चेङ्गं यच्च नेङ्गति}
{तस्य सर्वस् संप्राप्तः कालः परमदारुणः}


\twolineshloka
{नौश्च कारयितव्या ते दृढा युक्तवटारका}
{तत्र सप्तर्षिभिः सार्धमारुहेथा महामुने}


\twolineshloka
{बीजानि चैव सर्वाणि यथोक्तानि मया पुरा}
{तस्यामारोहयेर्नावि सुसंगुप्तानि भागशः}


\twolineshloka
{नौस्थश्च मां प्रतीक्षेथास्ततो मुनिजनप्रिय}
{आगमिष्याम्यहं शृङ्गी विज्ञेयस्तेन तापस}


\threelineshloka
{एवमेतत्त्वया कार्यमापृष्टोसि व्रजाम्यहम्}
{ता न शक्या महत्यो वै आपस्तर्तुं मया विना}
{नातिशङ्क्यमिदं चापि वचनं मे त्वया विभो}


\twolineshloka
{एवं करिष्य इति तं स मत्स्यं प्रत्यभाषत}
{जग्मतुश्च यथाकाममनुज्ञाप्य परस्परम्}


\threelineshloka
{ततो मनुर्महाराज यथोक्तं मत्स्यकेन ह}
{बीजान्यादाय सर्वाणि सागरं पुप्लुवे तदा}
{नौकया शुभयावीर महोर्मिणमरिंदम}


\threelineshloka
{चिन्तयामास च मनुस्तं मत्स्यं पृथिवीपते}
{सच तच्चिन्तितं ज्ञात्वा मत्स्यः परपुरंजय}
{शृङ्गी तत्राजगागाशु तदा भरतसत्तम}


\twolineshloka
{तं दृष्ट्वा मनुजव्याघ्र मनुर्मत्स्यं जलार्णवे}
{शृङ्गिणं तं यथोक्तेन रूपेणाद्रिमिवोच्छ्रितम्}


\twolineshloka
{वटारकमयं पाशमथ मत्स्यस्य मूर्धनि}
{मनुर्मनुजशार्दूल तस् शृङ्गे न्यवेशयत्}


\twolineshloka
{संयतस्तेन पाशेन मत्स्यः परपुरंजय}
{वेगेन महता नावं प्राकर्षल्लवणाम्भसि}


\twolineshloka
{स ततार तया नावा समुद्रं मनुजेश्वर}
{नृत्यमानमिवोर्मीभिर्गर्जमानमिवाम्भसा}


\twolineshloka
{क्षोभ्यमाणा महावातैः सा नौस्तस्मिन्महोदधौ}
{घूर्णते चपलेव स््री मत्ता परपुरंजय}


\twolineshloka
{नैव भूमिर्न च दिशः प्रदिशो वा चकाशिरे}
{सर्वं सलिलमेवासीत्खं द्यौश्च नरपुङ्गव}


\twolineshloka
{एवंभूते तदा लोके संकुले भरतर्षभ}
{अदृश्यन्त सप्तर्षयो मनुर्मत्स्यस्तथैव च}


\twolineshloka
{एवं बहून्वर्षगणांस्तां नावं सोऽथ मत्स्यकः}
{चकर्षातन्द्रितो राजंस्तस्मिन्सलिलसंचये}


\twolineshloka
{ततो हिमवतः शृङ्गं यत्परं भरतर्षभ}
{तत्राकर्षत्ततो नावं स मत्स्यः कुरुनन्दन}


\twolineshloka
{अथाब्रवीत्तदा मत्स्यस्तानृषीन्प्रहसञ्शनैः}
{अस्मिन्हिमवतः शृङ्गे नावं बध्नीत माचिरम्}


\twolineshloka
{सा बद्धा तत्र तैस्तूर्णमृषिभिर्भरतर्षभ}
{नौर्मत्स्यस्य वचः श्रुत्वा शृङ्गे हिमवतस्तदा}


\twolineshloka
{तच्च नौबन्धनं नाम शृङ्गं हिमवतः परम्}
{ख्यातमद्यापि कौन्तेय तद्विद्धि भरतर्षभ}


\threelineshloka
{अथाब्रवीदनिमिषस्तानृषीन्सहितस्तदा}
{अहं प्रजापतिर्ब्रह्मा मत्परं नाधिगम्यते}
{मत्स्यरूपेण यूयं च मयाऽस्मान्मोक्षिता भयात्}


\twolineshloka
{मनुना च प्रजाः सर्वाः सदेवासुरमानुपाः}
{स्रष्टव्याः सर्वलोकाश्चयच्चेङ्गं यच्चा नेङ्गति}


\threelineshloka
{तपसा चापि तीव्रेण प्रतिभाऽस्य भविष्यति}
{मत्प्रसादात्प्रजासर्गे न च मोहं गमिष्यति}
{इत्युक्त्वा वचनं मत्स्यः क्षणेनादर्शनं गतः}


\twolineshloka
{स्रष्टुकामः प्रजाश्चापि मनुर्वैवस्वतः स्वयम्}
{प्रमूढोऽभूत्प्रजासर्गे तपस्तेपे महत्ततः}


\twolineshloka
{तपसा महता युक्तः सोऽथ स्रष्टुं प्रचक्रमे}
{सर्वाः प्रजा मनुः साक्षाद्यथावद्भरतर्षभ}


\twolineshloka
{इत्येतन्मात्स्यकं नाम पुराणं परिकीर्तितम्}
{आख्यानमिदमाख्यातं सर्वपापहरं मया}


\twolineshloka
{य इदं शृणुयान्नित्यं मनोश्चरितमादितः}
{स सुखी सर्वपूर्णार्थः सर्वलोकमियान्नरः}


\chapter{अध्यायः १९१}
\twolineshloka
{वैशंपायन उवाच}
{}


\twolineshloka
{ततः स पुनरेवाथ मार्कण्डेयं तपस्विनम्}
{पप्रच्छ विनयोपेतो धर्मराजो युधिष्ठिरः}


\twolineshloka
{नैके युगसहस्रान्तास्त्वया दृष्टा महामुने}
{न चापीह समः कश्चिदायुष्मान्दृश्यते तव}


\twolineshloka
{वर्जयित्वा महात्मानं ब्रह्माणं परमेष्ठिनम्}
{न तेऽस्ति सदृशः कश्चिदायुषा ब्रह्मसत्तम}


\twolineshloka
{अथाऽन्तरिक्षे लोकेऽस्मिन्देवदानववर्जिते}
{त्वमेव प्रलये विप्र ब्रह्माणमुपतिष्ठसे}


\twolineshloka
{प्रलये चापि निर्वृत्ते प्रबुद्धे च पितामहे}
{त्वमेकः सृज्यमानानि भूतानीह प्रपश्यसि}


\twolineshloka
{चतुर्विधानि विप्रर्षे यथावत्परमेष्ठिना}
{वायुभूता दिशः कृत्वा विक्षिप्यापस्ततस्ततः}


\twolineshloka
{त्वया लोकगुरुः साक्षात्सर्वलोकपितामहः}
{आराधितो द्विजश्रेष्ठ तत्परेण समाधिना}


\twolineshloka
{स्वप्रमाणमथो विप्र त्वया कृतमनेकशः}
{घोरेणाविश्य तपसा वेधसो निर्जितास्त्वया}


% Check verse!
नारायणाङ्कप्रख्यस्त्वं सांपरायेऽतिपठ्यसे
\threelineshloka
{भगवाननेकशः कृत्वा त्वया विष्णोश्च विश्वकृत्}
{कर्णिकोद्धरणं दिव्यं ब्रह्मणः कामरूपिणः}
{रत्नालंकारयोगाभ्यां दृग्भ्यां दृष्टस्त्वया पुरा}


\twolineshloka
{तस्मात्तवान्तको मृत्युर्जरा वा देहनाशिनी}
{न त्वां विशति विप्रर्षे प्रसादात्परमेष्ठिनः}


\twolineshloka
{यदा नैव रविर्नाग्निर्न वायुर्न च चन्द्रमाः}
{नैवान्तरिक्षं नैवोर्वी शेषं भवति किंचन}


\twolineshloka
{तस्मिन्नेकार्णवे लोके नष्टे स्थावरजङ्गमे}
{नष्टे देवासुरगणे समुत्सन्नमहोरगे}


\twolineshloka
{शयानममितात्मानं पद्मे पद्मनिकेतनम्}
{त्वमेकः सर्वभूतेशं ब्रह्माणमुपतिष्ठसि}


\twolineshloka
{एतत्प्रत्यक्षतः सर्वं पूर्वं वृत्तं द्विजोत्तम}
{तस्मादिच्छाम्यहं श्रोतुं सर्वांहेत्वात्मिकां कथां}


\threelineshloka
{अनुभूतं हि बहुशस्त्वयैकेन द्विजोत्तम}
{न तेऽस्त्यविदितं किंचित्सर्वलोकेषु नित्यदा ॥मार्कण्डेय उवाच}
{}


\threelineshloka
{हन्त ते कथयिष्यामि नमस्कृत्वा स्वयंभुवे}
{पुरुषाय पुराणाय शाश्वतायाव्ययाय च}
{अव्यक्ताय सुसूक्ष्माय निर्गुणाय गुणात्मने}


\twolineshloka
{य एष पृथुदीर्घाक्षः पीतवासा जनार्दनः}
{एष कर्ता विकर्ता च भूतात्मा भूतकृत्प्रभुः}


\twolineshloka
{अचिन्त्यं महदाश्चर्यं पवित्रमिति चोच्यते}
{अनादिनिधनं भूतं विश्वमव्ययमक्षयम्}


\twolineshloka
{एष कर्ता न क्रियते कारणं चापि पौरुषे}
{को ह्येनं परुषं वेत्ति देवा अपि न तं पौरषे}


\twolineshloka
{सर्वमाश्चर्यमेवैतन्निर्वृत्तं राजसत्तम}
{आदितो मनुजव्याघ्र कृत्स्नस्य जगतः क्षये}


\twolineshloka
{चत्वार्याहुः सहस्राणि वर्षाणां तत्कृतं युगम्}
{तस्य तावच्छती सन्ध्या सन्ध्यांशश्च तथाविधः}


\twolineshloka
{वीणि वर्षसहस्राणि त्रेतायुगमिहोच्यते}
{तस्य तावच्छती सन्ध्या सन्ध्यांशश्च ततः परं}


\twolineshloka
{तथा वर्षसहस्रे द्वे द्वापरं परिमाणतः}
{तस्यापि द्विशती सन्ध्या सन्ध्यांशश्च तथाविधः}


\twolineshloka
{सहस्रमेकं वर्षाणां ततः कलियुगं स्मृतम्}
{तस्य वर्षशतं सन्ध्या सन्ध्यांशश्च ततः परम्}


\twolineshloka
{सन्ध्यासंध्यांशयोस्तुल्यं प्रमाणमुपधारय}
{क्षीणे कलियुगे चैव प्रवर्तति कृतं युगम्}


\twolineshloka
{एषा द्वादशसाहस्री युगाख्या परिकीर्तिता}
{एतत्सहस्रपर्यन्तमहो ब्राह्ममुदाहृतम्}


\twolineshloka
{विश्वं हि ब्रह्मभवने सर्वतः परिवर्तते}
{लोकानां मनुजवन्याघ्र प्रलयं तं विदुर्बुधाः}


\twolineshloka
{अल्पावशिष्टे तु तदा युगान्ते भरतर्षभ}
{सहस्रान्ते नराः सर्वे प्रायशोऽनृतवादिनः}


\twolineshloka
{यज्ञप्रतिनिधिः पार्थ दानप्रतिनिधिस्तथा}
{व्रतप्रतिनिधिश्चैव तस्मिन्काले प्रवर्तते}


\twolineshloka
{ब्राह्मणाः शूद्रकर्माणस्तथा शूद्रा धनार्जकाः}
{क्षत्रधर्मेण वाऽप्यत्र वर्तयन्ति युगक्षये}


\twolineshloka
{निवृत्तयज्ञस्वाध्याया दण्डाजिनविवर्जिताः}
{ब्राह्मणा सर्वभक्षाश्च भविष्यन्ति कलौ युगे}


\twolineshloka
{अजपा ब्राह्मणास्तात शूद्रा जपपरायणाः}
{विपरीते तदा लोके पूर्वरूपं क्षयस्य तत्}


\twolineshloka
{बहवो म्लेच्छराजानः पृथिव्यां मनुजाधिप}
{मृषानुशासिनः पापा मृषावादपरायणाः}


\twolineshloka
{आन्ध्राः शकाः पुलिन्दाश्च यवनाश्च नराधिपाः}
{काम्भोजा बाह्लिकाः शूरास्तथाऽऽभीरा नरोत्तमा}


\twolineshloka
{न तदा ब्राह्मणः कश्चित्स्वधर्ममुपजीवति}
{क्षत्रियाश्चापि वैश्याश्च विकर्मस्था नराधिप}


\twolineshloka
{अल्पायुषः स्वल्पबलाः स्वल्पवीर्यपराक्रमाः}
{अल्पसाराल्पदेहाश्च तथा सत्याल्पभाषिणः}


\twolineshloka
{बहुशून्या जनपदा मृगव्यालावृता दिशः}
{युगान्ते समनुप्राप्ते वृथा च ब्रह्मवादिनः}


\twolineshloka
{भोवादिनस्तथा शूद्रा ब्राह्मणाश्चार्यवादिनः}
{युगान्ते मनुजव्याघ्र भवन्ति बहुजन्तवः}


\twolineshloka
{न तथा घ्राणयुक्ताश्च सर्वगन्धा विशांपते}
{रसाश्च मनुजव्याघ्र न तथा स्वादुयोगिनः}


\twolineshloka
{बहुप्रजा ह्रस्वदेहा शीलाचारविवर्जिताः}
{मुखेभगाः स्त्रियो राजन्भविष्यन्ति युगक्षये}


\twolineshloka
{अट्टशूला जनपदाः शिवशूलाश्चतुष्पथाः}
{केशशूलाः स्त्रियो राजन्भविष्यन्ति युगक्षये}


\twolineshloka
{अल्पक्षीरास्तथा गावो भविष्यन्ति जनाधिप}
{अल्पपुष्पफलाश्चापि पादपा बहुवायसाः}


\twolineshloka
{ब्रह्मवध्यानुलिप्तानां तथा मिथ्याभिशंसिनाम्}
{नृपाणां पृथिवीपाल प्रतिगृह्णन्ति वै द्विजाः}


\twolineshloka
{लोभमोहपरीताश्च मिथ्याधर्मध्वजावृताः}
{भिक्षार्थं पृथिवीपाल चञ्चूर्यन्ते द्विजैर्दिशः}


\twolineshloka
{कराभारभयाद्भीता गृहस्थाः परिमोषकाः}
{मुनिच्छद्माकृतिच्छन्ना वाणिज्यमुपभुञ्जते}


\twolineshloka
{मित्या च नखरोमाणि धारयन्ति तदा द्विजाः}
{अर्थलोभान्नरव्याघ्र वृथा च ब्रह्मचारिणः}


\twolineshloka
{आश्रमेषु वृथाचार पारपा गुरुतल्पगाः}
{ऐहलौकिकमीहन्ते मांसशोणितवर्धनम्}


\threelineshloka
{`पारलौकिककार्येषु प्रमत्ता भृशनास्तिकाः'}
{बहुपाषण्डसंकीर्णाः परान्नगुणवादिनः}
{आश्रमा मनुजव्याघ्र न भवन्ति युगक्षये}


\twolineshloka
{यथर्तुवर्षी भगवान्न तथा पाकशासनः}
{न चापि सर्वभीजानि सम्यग्रोहन्ति भारत}


\threelineshloka
{फलं धर्मस्य राजेन्द्र सर्वत्र परिहीयते}
{हिंसाभिरामश्च जनस्तथा संपद्यतेऽशुचिः}
{अधर्मफलमत्यर्थं तदा भवति चानघ}


\twolineshloka
{तदा च पृथिवीपाल यो भवेद्धर्मसंयुतः}
{अल्पायुः स हि मन्तव्यो न हि धर्मोस्ति कश्चन}


\twolineshloka
{भूयिष्ठं कूटमानैश्च पण्यं विक्रीणते जनाः}
{वणिजश्च नरव्याघ्र बहुमाया भवन्त्युत}


\twolineshloka
{धर्मिष्ठाः परिहीयन्ते पापीयान्वर्धते जनः}
{धर्मस्य बलहानिः स्यादधर्मश्च बलायते}


\twolineshloka
{अल्पायुषो दरिद्राश्च धर्मिष्ठा मानवास्तथा}
{दीर्घायुषः समृद्धाश्च विधर्माणो युगक्षये}


\twolineshloka
{नगराणआं विहारेषु विधर्माणो युगक्षये}
{अधर्मिष्ठैरुपायैश् प्रजा व्यवहरन्त्युत}


\twolineshloka
{संचयेन तथाऽल्पेन भवन्त्याढ्यमदान्विताः}
{धनं विश्वासतो न्यस्तं मिथो भूयिष्ठशो नराः}


\twolineshloka
{हर्तुं व्यवसिता राजन्पापाचारसमनविताः}
{नैतदस्तीति मनुजा वर्तन्ते निरपत्रपाः}


\twolineshloka
{पुरुषादानि सत्वानि पक्षिणोऽथ मृगास्तथा}
{नगराणां विहारेषु चैत्येष्वपि च शेरते}


\twolineshloka
{सप्तवर्षाष्टवर्षाश्च स्त्रियो गर्भधरा नृप}
{दशद्वादशवर्षाणां पुंसां पुत्रः प्रजायते}


\twolineshloka
{भवन्ति षोडशे वर्षे नराः पलितिनस्तथा}
{आयुःक्षयो मनुष्याणां क्षिप्रमेव प्रपद्यते}


\twolineshloka
{क्षीणायुषो महाराज तरुणा वृद्धशीलिनः}
{तरुणानां च यच्छीलं तद्वृद्धेषु प्रजायते}


\twolineshloka
{विपरीतास्तदा नार्यो वञ्चयित्वाऽर्हतः पतीन्}
{व्युच्चरन्त्यपि दुःशीला दासैः पशुभिरेव च}


\twolineshloka
{वीरपत्न्यस्तथा नार्यः संश्रयन्ति नरान्नृप}
{भर्तारमपि जीवन्तमन्यान्व्यभिचरन्त्युत}


\twolineshloka
{तस्मिन्युगसहस्रान्ते संप्राप्ते चायुषः क्षये}
{अनावृष्टिर्महाराज जायते बहुवार्षिकी}


\twolineshloka
{ततस्तान्यल्पसाराणि सत्वानि क्षुधितानि वै}
{प्रलयं यान्ति भूयिष्ठं पृथिव्यां पृथिवीपते}


\twolineshloka
{ततो दिनकरैर्दीप्तैः सप्तभिर्मनुजाधिप}
{पीयते सलिलं सर्वं समुद्रेषु सरित्सु च}


\twolineshloka
{यच्च रकाष्ठं तृणं चापि शुष्कं चार्द्रं च भारत}
{सर्वं तद्भस्मसाद्भूतं दृश्यते भरतर्षभ}


\twolineshloka
{ततः संवर्तको वह्निर्वायुना सह भारत}
{लोकमाविशते पूर्वमादित्यैरुपशोषितम्}


\twolineshloka
{ततः स पृथिवीं भित्त्वा प्रविश्य च रसातलम्}
{रदेवदानवयक्षाणां भयं जनयते महत्}


\twolineshloka
{निरदहन्नागलोकं च यच्च किंचित्क्षिताविह}
{अधस्तात्पृथिवीपाल सर्वं नाशयते क्षणात्}


\twolineshloka
{ततो योजनविंशानां सहस्राणि शतानि च}
{निर्दहत्यशिवो वायुः स च संवर्तकोऽनलः}


\twolineshloka
{सदेवासुरगन्धर्वं सयक्षोरगराक्षसम्}
{ततो दहति दीप्तः स सर्वमेव जगद्विभुः}


\twolineshloka
{ततो गजकुलप्रख्यास्तजिन्मालाविभूषिताः}
{उत्तिष्ठन्ति महामेघा नभस्यद्भुतदर्शनाः}


\twolineshloka
{केचिन्नीलोत्पलश्यामाः केचित्कुमुदसन्निभाः}
{केचित्किञ्जल्कसंकाशाः केचित्पीताः पयोधराः}


\twolineshloka
{केचिद्धारिद्रसंकाशाः कारण्डवनिभास्तथा}
{केचित्कमलपत्राभाः केचिद्धिङ्गुलसप्रभाः}


\twolineshloka
{केचित्पुरवराकाराः केचिद्गजकुलोपमाः}
{केचिदञ्जनसंकाशाः केचिन्मकरसन्निभाः}


\twolineshloka
{विद्युन्मालापिनद्धाङ्गाः समुत्तिष्ठन्ति वै घनाः}
{घोररूपा महाराज घोरस्वननिनादिताः}


\twolineshloka
{ततो जलघराः सर्वं व्याप्नुवनति नभस्तलम्}
{`गर्जन्तः पृथिवीपाल पृथिवीधरसन्निभाः'}


\twolineshloka
{तैरियं पृथिवी सर्वा सपर्वतवनाकरा}
{आपूर्यते महाराज सलिलौघपरिप्लुता}


\twolineshloka
{ततस्ते जलदा घोरा राविणः पुरुषर्षभ}
{पर्वतान्प्लावयन्त्याशु चोदिताः परमेष्ठिना}


\twolineshloka
{वर्षमाणा महत्तोयं पूरयन्तो वसुंधराम्}
{सुघोरमशिवं रौद्रं नाशयनति च पावकम्}


\twolineshloka
{ततो द्वादशवर्षाणि पयोदास्त उपप्लवे}
{धाराभिः पूरयन्तो वै चोद्यमाना महात्मना}


\twolineshloka
{ततः समुद्रः स्वां वेलामतिक्रामति भारत}
{पर्वताश्च विदीर्यन्ते मही चापि विदीर्यते}


\twolineshloka
{सर्वतः सहसा भ्रान्तास्ते पयोदा नभस्तलम्}
{संवेष्टयित्वा नश्यन्ति वायुवेगपराहताः}


\twolineshloka
{कततस्तं मारुतं घोरं स्वयंभूर्मनुजाधिप}
{आदिः पद्मालयो देवः पीत्वा स्वपिति भारत}


\twolineshloka
{तस्मिन्नेकार्णवे घोरे नष्टे स्थावरजङ्गमे}
{नष्टे देवासुरगणए यक्षराक्षसवर्जिते}


\twolineshloka
{निर्मनुष्ये महीपाल निःश्वापदमहीरुहे}
{अनन्तरिक्षे लोकेऽस्मिन्भ्रमाम्येकोऽहमातुरः}


\twolineshloka
{एकार्णवे जले घोरे विचरन्पार्थिवोत्तम}
{अपश्यन्सर्वभूतानि वैक्लब्यमगमं ततः}


\twolineshloka
{ततः सुदीर्घं गत्वाऽहं प्लवमानो धराधिप}
{श्रान्तः क्वचिन्न शरणं लब्धवानस्म्यतन्द्रितः}


\twolineshloka
{ततः कदाचित्पश्यामि तस्मिन्सलिलसंनिघौ}
{न्यग्रोधं सुमहान्तं वै विशालं पृथिवीपते}


\twolineshloka
{शाखायां तस्य वृक्षस्य विस्तीर्णायां नराधिप}
{पर्यङ्के पृथिवीपाल दिव्यास्तरणसंस्तृते}


\twolineshloka
{उपविष्टं महाराज पद्मेन्दुसदृशाननम्}
{फुल्लपद्मविशालाक्षं बालं पश्यामि भारत}


\twolineshloka
{ततो मे पृतिवीपाल विस्मयः सुमहाभूत्}
{कथं त्वयं शिशुः शेते लोके नाशमुपागते}


\twolineshloka
{तपसा चिन्तयंश्चापि तं शिशुं नोपलक्षये}
{भूतं भव्यं भविष्यं च जानन्नपि नराधिप}


\twolineshloka
{अतसीपुष्पवर्णाभः श्रीवत्सकृतभूषणः}
{साक्षाल्लक्ष्म्या इवावास स तदा प्रतिभाति मे}


\twolineshloka
{ततो मामब्रवीद्बालः स पद्मनिभलोचनः}
{श्रीवत्सधारी द्युतिमान्वाक्यं श्रुतिसुखावहम्}


\twolineshloka
{जानामि त्वां परिश्रान्तं तात विश्रामकाङ्क्षिणम्}
{मार्कण्डेय महासत्वं यावदिच्छसि भार्गव}


\twolineshloka
{अभ्यन्तरं शरीरं मे प्रविश्य मुनिसत्तम}
{आस्खेह विहितो वासः प्रसादस्ते कृतो मया}


\twolineshloka
{ततो बालेन तेनैव मुक्तस्यासीत्तदा मम}
{निर्वेदो जीविते दीर्घे मनुष्यत्वे च भारत}


\twolineshloka
{ततो बालेन तेनास्यं सहसा विवृतं कृतम्}
{तस्याहमवशो वक्रे दैवयोगात्प्रवेशितः}


\twolineshloka
{ततः प्रविष्टस्तत्कुक्षिं सहसा मनुजाधिप}
{सराष्ट्रनगराकीर्णां कृत्स्नां पश्यामि मेदिनीम्}


\twolineshloka
{गङ्गां शतद्रुं सीतां च यमुनामथ कौशिकीम्}
{चर्मण्वतींवेत्रवतीं चन्द्रभागां सरस्वतीम्}


\twolineshloka
{सिन्धुं चैव विपाशां च नदीं गोदावरीमपि}
{वस्वोकसारां नलिनीं नर्मदां चैव भारत}


\twolineshloka
{नदीं ताम्रां च वेणां च पुण्यतोयां शुभावहाम्}
{सुवेणां कृष्णवेणां च इरामां च महानदीम्}


\threelineshloka
{वितस्तां च महाराज कावेरीं च महानदीम्}
{`तुङ्गभद्रां कृष्णवेणीं कमलां च महानदीम्'}
{शोणं च पुरुषव्याघ्र विशल्यां किंपुनामपि}


\twolineshloka
{एताश्चान्याश्च नद्योऽहं पृथिव्यां या नरोत्तम}
{परिक्रामन्प्रपश्यामि तस्य कुक्षौ महात्मनः}


\twolineshloka
{ततः समुद्रं पश्मामि यादोगणनिषेवितम्}
{रत्नाकरममित्रघ्न पयसोनिधिमुत्तमम्}


\twolineshloka
{ततः पश्यामि गगनं चन्द्रसूर्यविराजितम्}
{जाज्वल्यमानं तेजोभिः पावकार्कसमप्रभम्}


\twolineshloka
{पश्यामि च महीं राजन्काननैरुपशोभिताम्}
{`सपर्वतवनद्वीपां निम्नगाशतसंकुलाम्}


\twolineshloka
{यजन्ते हि ततो राजन्ब्राह्मणा बहुभिर्मखैः}
{क्षत्रियाश् प्रवर्तन्ते सर्ववर्णानुरञ्जनैः}


\twolineshloka
{वैश्याः कृषिं यथान्यायं कारयन्ति नराधिप}
{शुश्रूषायां च निरता द्विजानां वृषलास्तथा}


\twolineshloka
{ततः परिपतन्राजंस्तस्य कुक्षौ महात्मनः}
{हिमवन्तं च पश्यामि हेमकूटं च पर्वतम्}


\twolineshloka
{निषधं चापि पश्यामि श्वेतं च रजतान्वितम्}
{पश्यामि च महीपाल पर्वतं गन्धमादनम्}


\twolineshloka
{मन्दरं मनुजव्याघ्र नीलं चापि महागिरिम्}
{पश्यामि च महाराज मेरु कनकपर्वतम्}


\twolineshloka
{महेन्द्रं चैव पश्यामि विन्ध्यं च गिरिमुत्तमम्}
{मलयं चापि पश्यामि पारियात्रं च पर्वतम्}


\twolineshloka
{एते चान्ये च बहवो यावन्तः पृथिवीधराः}
{तस्योदरे मया दृष्टाः सर्वे रत्नविभूषिताः}


\threelineshloka
{सिंहान्व्याघ्रान्वराहांश्च पश्यामि मनुजाधिप}
{पृथिव्यां यानि चान्यानि सत्त्वानि जगतीपते}
{तानि सर्वाण्यहं तत्र पश्यन्पर्यचरं तदा}


\twolineshloka
{कुक्षौ तस् नरव्याघ्र प्रविष्टः संचरन्दिशः}
{शक्रादींश्चापि पश्यामि कृत्स्नान्देवगणानहम्}


\twolineshloka
{साध्यान्रुद्रांस्तथाऽऽदित्यान्गुह्यकान्पितरस्तथा}
{सर्पान्नागान्सुपर्णांश्च वसूनप्यश्विनावपि}


\twolineshloka
{गन्धर्वाप्सरसो यक्षानृषींश्चैव महीपते}
{दैत्यदानवसङ्घांश्च नागांश्च मनुजाधिप}


\twolineshloka
{सिंहिकातनयांश्चापि ये चान्ये सुरशत्रवः}
{यच्च किंचिन्मया लोके दृष्टं स्थावरजङ्गमम्}


\twolineshloka
{सर्वं पश्याम्यहं राजंस्तस्य कुक्षौ महात्मनः}
{त्वरमाणः फलाहारः कृत्स्नं जगदिदं विभो}


\twolineshloka
{अन्तःशरीरे तस्याहं वर्षाणामधिकं शतम्}
{3-191-b124 न च पश्यामियस्याहं देहस्यान्तं कदाचन}


\threelineshloka
{सततं धावमानश्च चिन्तयानो विशांपते}
{`भ्रमंस्तत्र महीपाल यदा वर्षगणान्बहून्'}
{आसादयामि नैवान्तं तस्य राजन्महात्मनः}


\twolineshloka
{ततस्तमेव शरणं गतोस्मि विधिवत्तदा}
{वरेण्यं वरदं देवं मनसा कर्मणैव च}


\twolineshloka
{ततोऽहं सहसा राजन्वायुवेगेन निःसृतः}
{महात्मनो मुखात्तस्य विवृतात्पुरुषोत्तम}


\twolineshloka
{ततस्तस्यैव शाखायां न्यग्रोधस्य विशांपते}
{आस्ते मनुजशार्दूल कृत्स्नमादाय वै जगत्}


\twolineshloka
{ते नैव बालवेषेण श्रीवत्सकृतलक्षणम्}
{आसीनं तं नरव्याघ्र पश्याम्यमिततेजसम्}


\twolineshloka
{ततो मामब्रवीद्बालः स प्रीतः प्रहसन्निव}
{श्रीवत्सधारी द्युतिमान्पीतवासा महाद्युतिः}


\twolineshloka
{अपीदानीं शरीरेऽस्मिन्मामके मुनिसत्तम}
{उषितस्त्वं परिश्रान्तो मार्कण्डेय ब्रवीहि मे}


\twolineshloka
{मुहूर्तादथ मे दृष्टिः प्रादुर्भूता पुनर्नवा}
{मायानिर्मुक्तमात्मनमपश्यं लब्धचेतसम्}


\twolineshloka
{तस्य ताम्रतलौ तात चरणौ सुप्रतिष्ठितौ}
{सुजातौ मृदुरक्ताभिरङ्गुलीभिर्विराजितौ}


\twolineshloka
{प्रयत्नेन मया मूर्ध्ना गृहीत्वा ह्यभिवनदितौ}
{दृष्ट्वाऽपरिमितं तस् प्रभावममितौजसः}


\twolineshloka
{विनयेनाञ्जलिं कृत्वाप्रयत्नेनोपगम्य ह}
{दृष्टो मया स भूतात्मा देवः कमललोचनः}


\twolineshloka
{तमहं प्राञ्जलिर्भूत्वा नमस्कृत्येदमब्रवम्}
{ज्ञातुमिच्छामि देव त्वां मायां चैतां तवोत्तमाम्}


\twolineshloka
{आस्येनानुप्रविष्टोऽहं शरीरे भगवंस्तव}
{दृष्टवानखिलाँल्लोकान्समस्तान्जठरे हि ते}


\twolineshloka
{तव देव शरीरस्था देवदानवराक्षसाः}
{यक्षगन्धर्वनागाश्च जगत्स्थावरजङ्गमम्}


\twolineshloka
{त्वत्प्रसादाच्च मे देव स्मृतिर्न परिहीयते}
{द्रुतमन्तःशरीरे ते सततं परिवर्तिनः}


\twolineshloka
{निर्गतोऽहमकामस्तु इच्छया ते महाप्रभो}
{यतिष्ये पुण्डरीकाक्ष ज्ञातुं त्वाऽहमनिन्दितं}


\twolineshloka
{इह भूत्वा शिशुः साक्षात्किं भवानवतिष्ठते}
{पीत्वा जगदिदं सर्वमेतदाख्यातुमर्हसि}


\twolineshloka
{किमर्थं च जगत्सर्वं शरीरस्थं तवानघ}
{कियन्तं च त्वया कालमिह स्थेयमरिंदम}


\threelineshloka
{एतदिच्छामि देवेश श्रोतुं ब्राह्मणकाम्यया}
{त्वत्तः कमलपत्राक्षं विस्तरेण यथातथम्}
{महद्ध्येतदचिन्त्यं च यदहं दृष्टवान्प्रभो}


\twolineshloka
{इत्युक्तः स मया श्रीमान्देवदेवो महाद्युतिः}
{सान्त्वयन्मामिदं वाक्यमुवाच वदतांवरः}


\chapter{अध्यायः १९२}
\twolineshloka
{देव उवाच}
{}


\twolineshloka
{कामं देवाऽपि मां विप्र न हि जानन्ति तत्त्वतः}
{त्वत्प्रीत्या तु प्रवक्ष्यामि यथेदं विमृजाम्यहम्}


\twolineshloka
{पितृभक्तोसि विप्रर्षे मां चैव शरणं गतः}
{ततो दृष्टोस्मि ते साक्षाद्ब्रह्यचर्यं च ते महत्}


\twolineshloka
{आपो नारा इति प्रोक्तास्तासां नाम कृतं मया}
{तेन नारायणप्युक्तो मम तत्त्वयनं सदा}


\twolineshloka
{अहंनारायणो नाम प्रभवः शाश्वतोऽव्ययः}
{विधाता सर्वभूतानां संहर्ता च द्विजोत्तम्}


\twolineshloka
{अहं विष्णुरहं ब्रह्मा शक्रश्चाहं सुराधिपः}
{अहं वैश्रवणो राजा यमः प्रेताधिपस्तथा}


\twolineshloka
{अहं शिवश्च सोमश्च कश्यपोऽथ प्रजापतिः}
{अहं धाता विधाता च यज्ञश्चाहं द्विजोत्तम}


\threelineshloka
{अग्निरास्यं क्षिति पादौ चन्द्रादित्यौ च लोचने}
{द्यौर्मूर्धा मे दिशः श्रोत्रे तथाऽऽपः स्वेदसंभवाः}
{सकलं च नभः कायो वायुर्मनसि मे स्थितः}


\twolineshloka
{मया क्रतुशतैरिष्टं बहुभिस्त्वाप्तदक्षिणैः}
{यजन्ते वेदविदुषो मां देवसदने स्थितम्}


\twolineshloka
{पृथिव्यां क्षत्रियेन्द्राश् पार्तिवाः स्वर्गकाङ्क्षिणः}
{यजन्ते मां तथा वैश्याः स्वर्गलोकजिगीषया}


\twolineshloka
{चतुःसमुद्रपर्यन्तां मेरुमन्दरभूषणाम्}
{शेषो भूत्वाऽहमेवैतां धारयामि वसुंधराम्}


\twolineshloka
{वाराहं रूपमास्थाय मयेयं जगती पुरा}
{मज्जमाना जले विप्र वीर्येणासीत्समुद्धृता}


\twolineshloka
{अग्निश्च बडबावक्रे भूत्वाऽहं द्विजसत्तम}
{पिबाम्यापः सदा विद्वंस्ताश्चैव विसृजाम्यहम्}


\twolineshloka
{ब्रह्म वक्रं भुजौ क्षत्रमूरू मे संस्थिता विशः}
{पादौ शूद्रा भवन्तीमे विक्रमेण क्रमेण च}


\twolineshloka
{ऋग्वेदः सामवेदश्च यजुर्वेदोऽप्यथर्वणः}
{मत्तः प्रादुर्भवन्त्येते मामेव प्रविशन्ति च}


\twolineshloka
{यतयः शान्तिपरमा यतात्मानो मुमुक्षवः}
{कामक्रोधद्वेषमुक्ता निःसंज्ञा वीतकल्मषाः}


\twolineshloka
{सत्वस्था निरहंकारा नित्यमध्यात्मकोविदाः}
{मामेव सततं विप्राश्चिन्तयन्त उपासते}


\twolineshloka
{अहं संवर्तको वह्निरहं संवर्तको यमः}
{अहं संवर्तकः सूर्यस्त्वहं संवर्तकोऽनिलः}


\twolineshloka
{तारारूपाणि दृश्यन्ते यान्येतानि नभस्तले}
{मम रूपाण्यथैतानि विद्धि त्वं द्विजसत्तम}


\twolineshloka
{रत्नाकराः समुद्राश्च सर्व एव चतुर्दिशः}
{वसनं शयनं चैव विलयं चैव विद्धि मे}


\threelineshloka
{मयैव सुविभक्तास्ते देवकार्यार्थसिद्धये}
{कामं क्रोधं च हर्षं च भयं मोहं तथैव च}
{ममैव विद्धि रोमाणि सर्वाण्येतानि सत्तम}


\twolineshloka
{प्राप्नुवन्ति नरा विप्र यत्कृत्वा कर्म शोभनम्}
{सत्यं दानं तपश्चोग्रमहिंसा चैव जन्तुषु}


\twolineshloka
{मद्विधानेन विहिता मम देहविहारिणः}
{मयाऽभिभूतविज्ञाना विचेष्टन्ते न कामतः}


\twolineshloka
{सम्यग्वेदमधीयाना यजन्ते विविधैर्मखैः}
{शान्तात्मानो जितक्रोधाः प्राप्नुवन्ति द्विजातयः}


\twolineshloka
{---न शक्यो यो विद्वन्नरैर्दुष्कृतकर्मभिः}
{---भाभिभूतैः कृपणैरनार्यैरकृतात्मभिः}


\twolineshloka
{तस्मान्महाफलंविद्धि पदं सुकृतकर्मणः}
{सुदुष्प्रापं विमूढानां मार्गं योगैर्निषेवितम्}


\twolineshloka
{यदा यदा च धर्मस्य ग्लानिर्भवति सत्तम}
{अभ्युत्थानमधर्मस्य तदाऽऽत्मानं सृजाम्यहम्}


\twolineshloka
{दैत्या हिंसानुरक्ताश्च अवध्याः सुरसत्तमैः}
{राक्षसाश्चापि लोकेऽस्मिन्यदोत्पत्स्यन्ति दारुणाः}


\twolineshloka
{तदाऽहंसंप्रसूयामि गृहेषु शुभकर्मणाम्}
{प्रविष्टो मानुषं देहं सर्वं प्रशमयाम्यहम्}


\twolineshloka
{सृष्ट्वा देवमनुष्यांस्तु गन्धर्वोरगराक्षसान्}
{स्थावराणि च भूतानि संहराम्यात्ममायया}


\twolineshloka
{कर्मकाले पुनर्देहमनुचिन्त्य सृजाम्यहम्}
{आविश्य मानषं देहं मर्यादाबन्धकारणात्}


\twolineshloka
{श्वेतः कृतयुगे वर्णः पीतस्त्रेतायुगे मम}
{श्यामो द्वापरमासाद्य कृष्णः कलियुगे तथा}


\twolineshloka
{त्रयो भागा ह्यधर्मस् तस्मिन्काले भवनति च}
{`यदा भवति मे वर्णः कृष्णो वै द्विजसत्तम्'}


\twolineshloka
{अन्तकाले च संप्राप्ते कालो भूत्वाऽतिदारुणः}
{त्रैलोक्यं नाशयाम्येकः कृत्स्नं स्थावरजङ्गमम्}


\threelineshloka
{अहं त्रिवर्त्मा विश्वात्मा सर्वलोकसुखावहः}
{अजितः सर्वगोऽनन्तो हृषीकेश उरुक्रमः}
{कालचक्रं नयाम्येको ब्रह्मन्नहमरूपकम्}


\threelineshloka
{शमनं सर्वभूतानां सर्वकालकृतोद्यमम्}
{एवं प्रणिहितः सम्यङ्मायया मुनिसत्तम}
{सर्वभूतेषु विप्रेन्द्र न च मां वेत्ति कश्चन}


% Check verse!
सर्वलोके च मां भक्ताः पूजयन्ति च सर्वशः
\twolineshloka
{यच्च किंचित्त्वया प्राप्तं मयि क्लेशात्मकं द्विज}
{सुखोदयाय तत्सर्वं श्रेयसे च तवानघ}


\twolineshloka
{यच्च किंचित्त्वया लोके दृष्टं स्थावरजङ्गमम्}
{विहित सर्वथैवासौ ममात्मा भूतभावनः}


\twolineshloka
{अर्धं मम शरीरस्य सर्वलोकपितामहः}
{अहं नारायणो नाम शङ्खचक्रगदाधरः}


\twolineshloka
{यावद्युगानां विप्रर्षे सहस्रपरिवर्तनम्}
{तावत्स्वपिमि विश्वात्मा सर्वलोकपितामहः}


\twolineshloka
{एवं सर्वमहं कालमिहासे मुनिसत्तम}
{अशिशुः शिशुरूपेण यावद्ब्रह्मा न बुध्यते}


\twolineshloka
{मया च दत्तो विप्राग्र्य वरस्ते ब्रह्मरूपिणा}
{असकृत्परितष्टेन विप्रर्षिगणपूजित}


\twolineshloka
{सर्वमेकार्णवं दृष्ट्वा नष्टं स्थावरजङ्गमम्}
{विक्लबोसि मया ज्ञातस्ततस्ते दर्शितं जगत्}


\twolineshloka
{अभ्यन्तरं शरीरस्य प्रविष्टोसि यदा मम}
{दृष्ट्वा लोकं समस्तं च विस्मितो नावबुध्यसे}


\twolineshloka
{ततोसि वक्राद्विप्रर्षे द्रुतं निःसारितो मया}
{आख्यातस्ते मया चात्मा दुर्ज्ञे योपि सुरासुरैः}


\twolineshloka
{यावत्स भगवान्ब्रह्मा न बुध्येत महातपाः}
{तावत्त्वमिह विप्रर्षे विस्रब्धश्चर वै सुखम्}


\twolineshloka
{ततो विबुद्धे तस्मिंस्तु सर्वलोकपितामहे}
{एकीभूतः प्रवेक्ष्यामि शरीराणि द्विजोत्तम}


\threelineshloka
{आकाशं पृथिवीं ज्योतिर्वायुं सलिलमेव च}
{लोके यच्च भवेच्छेषमिह स्थावरजङ्गमम् ॥मार्कण्डेय उवाच}
{}


\twolineshloka
{इत्युक्त्वान्तर्हितस्तात स देवः परमाद्भुतः}
{प्रजाश्चेमाः प्रपश्यामि विचित्रा विविधाः कृताः}


\twolineshloka
{एवं दृष्टं मया राजंस्तस्मिन्प्राप्ते युगक्षये}
{आश्चर्यं भरतश्रेष्ठ सर्वधर्मभृतांवरः}


\twolineshloka
{यः स देवो मया दृष्टः पुरा पद्मायतेक्षणः}
{स एष पुरुषव्याघ्र संबन्धी ते जनार्दनः}


\twolineshloka
{अस्यैव वरदानाद्धि स्मृतिर्न प्रजहाति माम्}
{दीर्गमायुश्च कौन्तेय स्वच्छन्दमरणं मम}


\twolineshloka
{स एष कृष्णो वार्ष्णेय पुराणपुरुषो विभुः}
{आस्ते हरिरचिन्त्यात्मा क्रीडन्निव महाभुजः}


\twolineshloka
{एष धाता विधाता च संहर्ता चैव शाश्वतः}
{श्रीवत्सवक्षा गोविन्दः प्रजापतिपतिः प्रभुः}


\twolineshloka
{दृष्ट्वेमं वृष्णिप्रवरं स्मृतिर्मामियमागता}
{आदिदेवमयं जिष्णुं पुरुषं पीतवाससम्}


\threelineshloka
{सर्वेषामेव भूतानां पिता माता च माधवः}
{गच्छध्वमेनं शरणं शरण्यं कौरवर्षभाः ॥वैशंपायन उवाच}
{}


\twolineshloka
{एवमुक्तास्च ते पार्ता यमौ च पुरुषर्षभौ}
{द्रौपद्या सहिताः सर्वे नमश्चक्रुर्जनार्दनम्}


\twolineshloka
{स चैतान्पुरुषव्याघ्र साम्ना परमवल्गुना}
{सान्त्वयामास मानार्हो मन्यमानो यथाविधि}


\chapter{अध्यायः १९३}
\twolineshloka
{वैशंपायन उवाच}
{}


\threelineshloka
{युधिष्ठिरस्तु कौन्तेयो मार्कण्डेयं महामुनिम्}
{पुन- पप्रच्छ सामात्यो भविष्यां जगतो गतिम् ॥युधिष्ठिर उवाच}
{}


\twolineshloka
{आश्चर्यभूतं भवतः श्रुतं नो वदतांवर}
{मुने भार्गव यद्वृत्तं युगादौ प्रभवाप्ययौ}


\twolineshloka
{अस्मिन्कलियुगे त्वस्ति पुनः कौतूहलं मम}
{समाकुलेषु धर्मेषु किंनु शेषं भविष्यति}


\twolineshloka
{किंवीर्या मानवास्तत्रकिमाहारविहारिणः}
{किमायुषः किंवसना भविष्यन्ति युगक्षये}


\twolineshloka
{कां च काष्ठां समासाद्य पुनः संपत्स्यते कृतम्}
{विस्तरेण मुने ब्रूहि विचित्राणीह भाषसे}


\twolineshloka
{इत्युक्तः स मुनिश्रेष्ठः पुनरेवाभ्यभाषत}
{रमयन्वृष्णिशार्दूलं पाण्डवांस्च महानृषिः}


\twolineshloka
{शृणु राजन्मया दृष्टं यत्पुरा श्रुतमेव च}
{अनुभूतं च राजेनद््रदेवदेवप्रसादजम्}


\twolineshloka
{भविष्यं सर्वलोकस्य वृत्तान्तं भरतर्षभ}
{कलुषं कालमासाद्य कथ्यमानं निबोध मे}


\twolineshloka
{कृते चतुष्पात्सकलो निर्व्याजोप्राधिवर्जितः}
{वृषः प्रतिष्ठितो धर्मो मनुष्ये भरतर्षभ}


\twolineshloka
{अधर्मपादविद्धस्तु त्रिभिरंशैः प्रतिष्ठितः}
{त्रेतायां द्वापरेऽर्धेन व्यामिश्रो धर्म उच्यते}


\twolineshloka
{त्रिभिरंशैरधर्मस्तु लोकानाक्रम्य तिष्ठति}
{तामसं युगमासाद्य तदा भरतसत्तम}


\twolineshloka
{चतुर्थांशेन धर्मस्तु मनुष्यानुपतिष्ठति}
{आयुर्वीर्यं मनो बुद्धिर्बलं तेजश्च पाण्डव}


\twolineshloka
{मनुष्याणआमनुयुगं ह्रप्तन्तीति निबोध मे}
{ब्राह्मणाः क्षत्रिया वैश्याः शूद्राश्चैव युधिष्ठिर}


\twolineshloka
{व्याजैर्धर्मं चरिष्यन्ति धर्मवैतंसिका नराः}
{सत्यं संक्षेप्स्यते लोके नरैः पण्डितमानिभिः}


\twolineshloka
{सत्यहान्या ततस्तेषामायुरल्पं भविष्यति}
{आयुषः प्रक्षयाद्विद्यां न शक्ष्यन्त्युपशिक्षितुम्}


\twolineshloka
{विद्याहीनानविज्ञानाल्लोभोप्यभिभविष्यति}
{लोभमोहपरा मूढाः कामासक्ताश् मानवाः}


\twolineshloka
{वैरबद्धा भविष्यन्ति परस्परवधैषिणः}
{ब्राह्मणाः क्षत्रिया वैश्याः संकीर्यन्ते परस्परम्}


\twolineshloka
{शूद्रतुल्या भविष्यन्ति तपःसत्यविवर्जिताः}
{अन्त्या मध्या भविष्यन्ति मध्याश्चान्त्या न संशयः}


\twolineshloka
{ईदृशो भविता लोको युगान्ते पर्युपस्थिते}
{वस्त्राणां प्रवरा शाणी धान्यानां कोरदूपकः}


\twolineshloka
{भार्यामित्राश्च पुरुषा भविष्यन्ति युगक्षये}
{मत्स्यामिषेण जीवन्तो दुहन्तश्चाप्यजैडकम्}


\twolineshloka
{गोषु नष्टासु पुरुषा येऽपि नित्यं धृतव्रताः}
{तेऽपिलोभसमायुक्ता भविष्यन्ति युगक्षये}


\twolineshloka
{अन्योन्यं परिमुष्णन्तो हिंसयन्तश्च मानवाः}
{अजपा नास्तिकाः स्तेना भविष्यन्ति युगक्षये}


\twolineshloka
{सरित्तीरेषु कुद्दालैर्वापयिष्यनति चौपधीः}
{ताश्चाप्यल्पफलास्तेषां भविष्यन्ति युगक्षये}


\twolineshloka
{श्राद्धे दैवे च पुरुषा येऽपि नित्यं धृव्रताः}
{तेऽपिलोभसमायुक्ता भोक्ष्यन्तीह परस्परम्}


\twolineshloka
{पिता पुत्रस् भोक्ता च पितुः पुत्रस्तथैव च}
{अतिक्रान्तानि भोज्यानि भविष्यन्ति युगक्षये}


\threelineshloka
{न व्रतानि चरिष्यन्ति ब्राह्मणा वेदनिन्दकाः}
{न यक्ष्यन्ति न होष्यन्ति हेतुवादविमोहिताः}
{निम्नेष्वीहां करिष्यनति हेतुवादविमोहिताः}


\twolineshloka
{निम्ने कृषिं करिष्यन्ति योक्ष्यन्ति धुरि धेनुक्राः}
{एकहायनवत्सांश्च वाहयिष्यन्ति मानवाः}


\threelineshloka
{पुत्रः पितृवधं कृत्वा पिता पुत्रवधं तथा}
{`स्त्रियोऽपि पतिपुत्रादीन्वधिष्यनति युगक्षये'}
{निरुद्वेगो बृहद्बादी न निन्दामुपलप्स्यते}


\twolineshloka
{म्लेच्छभूतं जगत्सर्वं निष्क्रियं दानवर्जितम्}
{भविष्यति निरानन्दमनुत्सवमथो तथा}


\twolineshloka
{प्रायशः कृपणानां हि तथा बन्धुमतामपि}
{विधवानां च वित्तानि हरिष्यन्तीह मानवाः}


\twolineshloka
{स्वल्पवीर्यबलाः स्वाधा लोभमोहपरायणाः}
{तत्कथादानसंतुष्ट शिष्टानामपि बान्धवाः}


\twolineshloka
{परिग्रहंकरिष्यति मायाचारपरिग्रहाः}
{संघातयन्तः तय राजानः पापबुद्धयः}


\twolineshloka
{परस्परवधोयुक्ता मूर्खाः पण्डितमानिनः}
{भविष्यन्ति युगस्यान्ते क्षत्रिया लोककण्टकाः}


\twolineshloka
{अरक्षितारो लुब्धाश् मानाहंकारदर्पिताः}
{केवलं दण्डरुचयो भविष्यन्ति युगक्षये}


\twolineshloka
{आक्रम्याक्रम्य साधूनां दारांश्चापि धनानि च}
{भोक्ष्यन्ते निरनुक्रोसा रुदतामपि भारत}


\twolineshloka
{न कन्यां याचे कश्चिन्नापि कन्या प्रदीयते}
{स्वयंग्रहा भविष्यन्ति युगान्ते समुपस्थिते}


\twolineshloka
{राजानश्चाप्यसंतुष्टाः परार्तान्मूढचेतसः}
{सर्वोपायैर्हरिष्यन्ति युगान्ते पर्युपस्थिते}


\twolineshloka
{म्लेच्छीभूतं जगत्सर्वं भविष्यति न संशयः}
{हस्तो हस्तं परिमुषेद्युगान्ते समुपस्थिते}


\twolineshloka
{सत्यं संक्षिप्यते लोके नरैः पण्डितमानिभिः}
{स्थविरा बालमतयो बालाः स्थविरबुद्धयः}


\twolineshloka
{भीरुस्था शूरमानी शूरा भीरुविषादिनः}
{न विश्वसन्ति चान्योन्यं युगान्ते पर्युपस्थिते}


\twolineshloka
{नैकभार्यं जगत्सर्वं लोभमोहव्यवस्थितम्}
{अधर्मो वर्धते तत्र न तु धर्मः प्रवर्तते}


\twolineshloka
{ब्राह्मणाः क्षत्रिया वैश्या नशिष्यन्ति जनाधिप}
{एकवर्णस्तदा लोको भविष्यति युगक्षये}


\twolineshloka
{न क्षंस्यति पिता पुत्रं पुत्रश्च पितरं तथा}
{बार्याश्च पतिशुश्रूषां न करिष्यन्ति संक्षये}


\twolineshloka
{ये यवान्ना जनपदा गोधूमान्नास्तथैव च}
{तान्देशान्संश्रयिष्यन्ति युगान्ते पर्युपस्थिते}


\twolineshloka
{स्वैराहाराश्च पुरुषा योपितश्च विशांपते}
{अन्योन्यं न सहिष्यन्ति युगान्ते पर्युपस्थिते}


\twolineshloka
{म्लेच्छभूतं जगत्सर्वं भविष्यति युधिष्ठिर}
{श्राद्धे न देवान्न पितॄस्तर्पयिष्यन्ति मानवाः}


\twolineshloka
{न कश्चित्कस्यचिच्छ्रोता न कश्चित्कस्यचिद्गुरुः}
{तमोग्रस्तस्तदा लोको भविष्यति जनाधिप}


\twolineshloka
{परमायुश्च भविता तदा वर्षाणि षोडश}
{ततः प्राणान्विमोक्ष्यन्ति युगान्ते समुपस्थिते}


\twolineshloka
{पञ्चमे वाऽथ षष्ठे वा वर्षे कन्या प्रसूयते}
{सप्तवर्षाष्टवर्षाश्च प्रजास्यन्ति नरास्तदा}


\twolineshloka
{पत्यौ स्त्री तु तदा राजन्पुरुषो वा स्त्रियं प्रति}
{युगान्ते राजशार्दूल न तोषमुपयास्यति}


\twolineshloka
{अल्पद्रव्या वृथालिङ्गा हिंसा च प्रभविष्यति}
{न कश्चित्कस्यचिद्दाता भविष्यति युगक्षये}


\twolineshloka
{अट्टशूला जनपदाः शिवशूलाश्चतुष्पथाः}
{केशशूलाः स्त्रियश्चापि भविष्यनति युगक्षये}


\twolineshloka
{म्लेच्छाचाराः सर्वमभक्षा दारुणा सर्वकर्मसु}
{भाविन पश्चिमे काले मनुष्या नात्र संशयः}


\twolineshloka
{क्रयविक्रयकाले च सर्वः सर्वस्य वञ्चनम्}
{युगान्ते भरतश्रेष्ठ वित्तलोभात्करिष्यति}


\twolineshloka
{ज्ञानानि चाप्यविज्ञाय करिष्यन्ति क्रियास्तथा}
{आत्मच्छन्देन वर्तन्ते युगान्ते समुपस्थिते}


\twolineshloka
{स्वभावात्क्रूरकर्माणश्चान्योन्यमभिशंसिनः}
{भवितारो जनाः सर्वे संप्राप्ते तु युगक्षये}


\twolineshloka
{आरामांश्चैव वृक्षांश्च नाशयिष्यन्ति निर्व्यथाः}
{भविता संशयो लोके जीवितस्य हि देहिनाम्}


\twolineshloka
{तथा लोभाभिभूताश्च भविष्यन्ति नरा नृप}
{ब्राह्मणांश्च हनिष्यन्ति ब्राह्मणस्वोपभोगिनः}


\twolineshloka
{हाहाकृता द्विजाश्चैव भयार्ता वृषलार्दिताः}
{त्रातारमलाभन्तो वै भ्रमिष्यनति महीमिमाम्}


\twolineshloka
{जीवितान्तकराः क्रूरा रौद्राः प्राणिविहिंसकाः}
{यदा भविष्यन्ति नरास्तदा संक्षेप्स्यते युगम्}


\twolineshloka
{आश्रयिष्यन्ति च नदी पर्वतान्विषमाणि च}
{प्रधावमाना वित्रस्ता द्विजाः कुरुकुलोद्वह}


\twolineshloka
{दस्युभिः पीडिता राजन्काका इव द्विजोत्तमाः}
{कुराजभिश्च सततं करभारप्रपीडिताः}


\twolineshloka
{धैर्यं त्यक्त्वा महीपाल दारुणे युगसंक्षये}
{विकर्माणि करिष्यन्ति शूद्राणां परिचारकाः}


\twolineshloka
{शूद्रा धर्मं प्रवक्ष्यन्ति ब्राह्मणाः पर्युपासकाः}
{श्रोतारश्च भविष्यन्ति प्रामाण्येन व्यवस्थिताः}


\threelineshloka
{विपरीतश्च लोकोऽयं भविष्यत्यधरोत्तरः}
{एडूकान्पूजयिष्यन्ति वर्जयिष्यन्ति देवताः}
{शूद्राश्च प्रभविष्यन्ति न द्विजा युगसंक्षये}


\twolineshloka
{आश्रमेषुमहर्षीणां ब्राह्मणावसथेषु च}
{देवस्थानेषु चैत्येषु नागानामालयेषु च}


\twolineshloka
{एडूकचिह्ना पृथिवी न देवगृहभूषिता}
{भविष्यति युगे क्षीणे तद्युगान्तस्य लक्षणम्}


\twolineshloka
{दा रौद्रा धर्महीना मांसादाः पानपास्तथा}
{भविष्यन्ति नरा नित्यं तदा संक्षेप्स्यते युगम्}


\twolineshloka
{पुष्पं पुष्पे यदा राजन्फले वा फलमाश्रितम्}
{प्रजास्यति महाराज तदा संक्षेप्स्यते युगम्}


\twolineshloka
{अकालवर्षी पर्जन्यो भविष्यति गते युगे}
{अक्रमेण मनुष्याणां भविष्यनति तदा क्रियाः}


\twolineshloka
{विरोधमथ यास्यनति वृषला ब्राह्मणैः सह}
{मही म्लेच्छजनाकीर्णा भविष्यति ततोऽचिरात्}


\twolineshloka
{करभारभयाद्विप्रा भजिष्यन्ति दिशो दश}
{`अन्यायवर्तिनश्चापि भविष्यनति नराधिपाः'}


\twolineshloka
{निर्विशेषा जनपदास्तथा विष्टिकरार्दिताः}
{आश्रमानुपलप्स्यन्ति फलमूलोपजीविनः}


\twolineshloka
{एवं पर्याकुले लोके मर्यादा न भविष्यति}
{`ब्राह्मणःक्षत्रियावैश्याः परित्यक्ष्यन्ति सत्क्रियाम्'न स्थास्यन्त्युपदेशे च शिष्या विप्रियकारिणः}


\twolineshloka
{आचार्योपनिधिश्चैव भर्त्स्यते तदनन्तरम्}
{अर्थयुक्त्या प्रवात्स्यन्ति मित्रसंबन्धिबान्धवाः}


\twolineshloka
{अभावः सर्वभूतानां युगान्ते संभविष्यति}
{दिशः प्रज्वलिता सर्वा नक्षत्राण्यप्रभाणि च}


\twolineshloka
{प्रधूपितानि ज्योतींषि वाताः पर्याकुलास्तथा}
{उल्कापाताश्च बहवो महाभयनिदर्शकाः}


\threelineshloka
{षङ्भिरन्यैश्च सहितो भास्करः प्रतपिष्यति}
{तुमुलाश्चापि निर्ह्रादा दिग्दाहाश्चापि सर्वशः}
{कबन्धान्तर्हितो भानुरुदयास्तमने तदा}


\twolineshloka
{अकालवर्षी भगवान्भविष्यति सहस्रदृक्}
{सस्यानि च न रोक्ष्यन्ति युगान्ते पर्युपस्थिते}


\twolineshloka
{अभीक्ष्णं क्रूरवादिन्यः परुषा रदितप्रियाः}
{भर्तॄणां वचने चैव न स्थास्यन्ति ततः स्त्रियः}


\twolineshloka
{पुत्राश्च मातापितरौ हनिष्यन्ति युगक्षये}
{सूदयिष्यन्ति च पतीन्स्त्रियः पुत्रानपाश्रिताः}


\twolineshloka
{अपर्वणि महाराज सूर्यं राहुरुपैष्यति}
{युगान्ते हुतभुक्वापि सर्वतः प्रज्वलिष्यति}


\twolineshloka
{पानीयं भोजनं चापि याचमानास्तदाऽध्वगाः}
{न लप्स्यन्ते निवासं च निरस्ताः पथि शेरते}


\twolineshloka
{निर्घातवायसा नागाः शकुनाः समृगद्विजाः}
{रूक्षा वाचो विमोक्ष्यन्ति युगान्ते पर्युपस्थिते}


\twolineshloka
{मित्रसंबन्धिनश्चापि संत्यक्ष्यनति नरास्तदा}
{जनं परिजनं चापि यागान्ते पर्युपस्थिते}


\twolineshloka
{अथ देशान्दिशश्चापि पत्तनान्यापणानि च}
{क्रमशः संलयिष्यन्ति युगान्ते पर्युपस्थिते}


\twolineshloka
{हा तात हा सुतेत्येवं तदा वाच सुदारुणाः}
{विक्रोशमानश्चान्योन्यं जनो गां पर्यटिष्यति}


\twolineshloka
{`मोवादिनस्तथा शूद्रा ब्राह्मणाः प्राकृतप्रियाः}
{पाषण्डजनसंकीर्णा भविष्यन्ति युगक्षये'}


\twolineshloka
{ततस्तुमुलसंघाते वर्तमाने युगक्षये}
{द्विजातिपूर्वको लोकः क्रमेण प्रभविष्यति}


\twolineshloka
{ततः कालान्तरेऽन्यस्मिन्पुनर्लोकविवृद्धये}
{भविष्यति पुनर्दैवमनुकूलं यदृच्छया}


\twolineshloka
{यदा सूर्यश्च चन्द्रश्च तथा तिष्यबृहस्पती}
{एकराशौ समेष्यन्ति प्रपत्स्यति तदा कृतम्}


\threelineshloka
{कालवर्षी च पर्जन्यो नक्षत्राणि शुभानि च}
{प्रदक्षिणा ग्रहाश्चापि भविष्यन्त्यनुलोमगाः}
{क्षेमं सुभिक्षमारोग्यं भविष्यति निरामयम्}


\twolineshloka
{कल्की विष्णुयशा नाम द्विजः कालप्रचोदितः}
{उत्पत्स्यते महावीर्यो महाबुद्धिपराक्रमः}


\twolineshloka
{संभूतः संभलग्रामे ब्राह्मणावसथे शुभे}
{`महात्मा वृत्तसंपन्नः प्रजानां हितकृन्नृप'}


\twolineshloka
{मनसा तस्य सर्वाणि वाहनान्यायुधानि च}
{उपस्तास्यन्ति योधाश्च शस्त्राणि कवचानि च}


\twolineshloka
{स धर्मविजयी राजा चक्रवर्ती भविष्यति}
{सचेमं संकुलं लोकं प्रसादमुपनेष्यति}


\twolineshloka
{उत्थितो ब्राह्मणो दीप्तः क्षयान्तकृदुदारधीः}
{संक्षेपको हि सर्वस्य युगस्य परिवर्तकः}


\twolineshloka
{स सर्वत्र गतान्क्षुद्रान्ब्राह्मणैः परिवारितः}
{उत्सादयिष्यति तदा सर्वम्लेच्छगणान्द्विजः}


\chapter{अध्यायः १९४}
\twolineshloka
{मार्कण्डेय उवाच}
{}


\twolineshloka
{ततश्चोरक्षयंकृत्वाद्विजेभ्यः पृथिवीमिमाम्}
{वाजिमेधे महायज्ञे विधिवत्कल्पयिष्यति}


\twolineshloka
{स्थापयित्वा च मर्यादाः स्वयंभुविहिताः शुभाः}
{पुनः पुण्ययशःकर्म जरया संश्रयिष्यति}


\twolineshloka
{तच्छीलमनुवर्स्यन्ति मनुष्या लोकवासिनः}
{विप्रैश्चोरक्षये चैव कृतेक्षेमं भविष्यति}


\twolineshloka
{कृष्णाजिनानि शक्तीश्च त्रिशूलान्यायुधानि च}
{स्थापयन्द्विजशार्दूलो देशेषु विजितेषु च}


\twolineshloka
{संस्तूयमानो विप्रेन्द्रैर्मानयानो द्विजोत्तमान्}
{कल्कीचरिष्यति महीं सदा दस्युवधे रतः}


\twolineshloka
{हा मातस्तात पुत्रेति तास्ता वाचः सुदारुणाः}
{विक्रोशमानान्सुभृशं दस्यूननेष्यति संक्षयम्}


\twolineshloka
{ततोऽधऱ्मविनाशो वै धर्मवृद्धिश्च भारत}
{भविष्यति कृतेप्राप्ते क्रियावांश्च जनस्तथा}


\twolineshloka
{आरामाश्चैव चैत्याश्च तटाकावसथास्तथा}
{पुष्करिण्यश्च विविधा देवतायतनानि च}


\twolineshloka
{यज्ञक्रियाश्च विविधा भविष्यन्ति कृते युगे}
{ब्राह्मणाः साधवश्चैव मुनयश्च तपस्विनः}


\twolineshloka
{आश्रमा हतपाषण्डाः स्थिताः सत्ये जनास्तदा}
{जायन्ति सर्वभूतानि शुध्यमानानि चैव हि}


\twolineshloka
{सर्वेष्वृतुषु राजेन्द्र सर्वं सस्यं भविष्यति}
{नरा दानेषु निरता व्रतेषु नियमेषु च}


\twolineshloka
{जपयज्ञपरा विप्रा धर्मकामा मुदा युताः}
{पालयिष्यन्ति राजानो धर्मेणेमां वसुंधराम्}


\twolineshloka
{व्यवहाररता वैश्या भविष्यन्ति कृते युगे}
{षट्कर्मनिरता विप्राः क्षत्रिया रक्षणे रताः}


\twolineshloka
{शुश्रूषायां रताः शूद्रास्तथा वर्णत्रयस्य च}
{एष धर्मः कृतयुगे त्रेतायां द्वापरे तथा}


\twolineshloka
{पश्चिमे युगकाले च यः स ते संप्रकीर्तितः}
{सर्वलोकस् विदिता युगसह्ख्या च पाण्डव}


\twolineshloka
{एतत्ते सर्वमाख्यातमतीतानागतं मया}
{वायुप्रोक्तमनुस्मृत्यपुराणमृषिसंस्तुतम्}


\twolineshloka
{एवं संसारमार्गा मे बहुशश्चिरजीविनः}
{दृष्टाश्चैवानुभूताश्च किं भूयः श्रोतुमिच्छसि}


\twolineshloka
{इदं चैवापरं भूयः सह भ्रातृभिरच्युत}
{धर्मसंशयमोक्षार्थं निबोध वचनं मम}


\twolineshloka
{न तेऽन्यथाऽत्र विज्ञेयो धर्मो धर्मभृतांवर}
{धर्मात्मा हि सुखं राजन्प्रेत्य चेह च नन्दति}


\threelineshloka
{न ब्राह्मणे परिभवः कर्तव्यस्ते कदाचन}
{ब्राह्मणः कुपितोऽहन्यादपि लोकान्प्रतिज्ञया ॥वैशंपायन उवाच}
{}


\twolineshloka
{मार्कण्डेयवचः श्रुत्वा कुरूणां प्रवरो नृपः}
{उवाच वचनं धीमान्परमं परमद्युतिः}


\twolineshloka
{`एतच्छ्रुत्वा मया किं स्यात्कर्तव्यं मुनिसत्तम}
{कथं चायं जितोलोको रक्षितव्यो भविष्यति'}


\threelineshloka
{कस्मिन्धर्मे मया स्थेयं प्रजाः संरक्षता मुने}
{कथंच वर्तमानो वै न च्यवेयं स्वधर्मतः ॥मार्कण्डेय उवाच}
{}


\twolineshloka
{दयावान्सर्वबूतेषु हितो रक्तोऽनसूयकः}
{उपत्यानामिव स्नेहात्प्रजानां रक्षणे रतः}


\twolineshloka
{चर धर्मं त्यजाधर्मं पितॄन्पूर्वाननुस्मर}
{प्रमादाद्यत्कृतं तेऽभूत्सम्यग्ज्ञानेन तज्जय}


\twolineshloka
{अलं ते मानमाश्रित्य सततं प्रियवाग्भव}
{विजित्य पृथिवीं सर्वां मोदमानः सुखी भव}


\twolineshloka
{एष भूतो भविष्यश्च धर्मस्ते समुदीरितः}
{न तेऽस्त्यविदितं किंचिदतीतानागतं भुवि}


\twolineshloka
{तस्मादिमं परिक्लेशं त्वं तात हृदि मा कृथाः}
{प्राज्ञास्तात न मुह्यन्ति कालेनापि प्रपीडिताः}


\twolineshloka
{एष कालो महाबाहो अपि सर्वदिबौकसाम्}
{मुह्यन्ति हि प्रजास्तात कालेनापि प्रचोदिताः}


\twolineshloka
{मा च तेऽत्र विशङ्का भूद्यन्मयोक्तं तवानघ}
{अतिशङ्क्य वचो ह्येतद्धर्मलोपो भवेत्तव}


\threelineshloka
{जातोसि प्रथिते वंशे कुरूणां भरतर्षभ}
{कर्मणा मनसा वाचा सर्वमेतत्समाचर ॥युधिष्ठिर उवाच}
{}


\twolineshloka
{यत्त्वयोक्तं द्विजश्रेष्ठ वाक्यं श्रुतिमनोहरम्}
{तथा करिष्ये यत्नेन भवतः शासनं विभो}


\threelineshloka
{न मे लोभोस्ति विप्रेन्द्रन भयं न च मत्सरः}
{करिष्यामि हि तत्सर्वमुक्तं यत्ते मयि प्रभो ॥वैशंपायन उवाच}
{}


\twolineshloka
{श्रुत्वा तु वचनं तस्य पाण्डवस्य यशस्विनः}
{संहृष्टः पाण्डवा राजन्सहिता शार्ङ्गधन्वना}


\threelineshloka
{विप्र्रषभाश्च ते सर्वे ये तत्रासन्समागताः}
{तथा कथां शुभां श्रुत्वा रमार्कण्डेयस्य धमतः}
{विस्मिता समपद्यन्त पुराणस् निवेदनात्}


\chapter{अध्यायः १९५}
\twolineshloka
{वैशंपायन उवाच}
{}


% Check verse!
भूय एव ब्राह्मणमाहात्म्यं वक्तुमर्हसीत्यब्रवीत्पाण्डवेयोमार्कण्डेयम्
% Check verse!
अथाचष्ट मार्कण्डेयोऽपूर्वमिदं श्रूयतां ब्राह्मणानां चरितम्
% Check verse!
अयोध्यायामिक्ष्वाकुकुलोद्वहः पार्थिवः परीक्षिन्नाम मृगयामगमत्
% Check verse!
स एकोऽश्वेन मृगमन्वसरत् मृगो दूरमपाहरत्
% Check verse!
अध्वनि जातश्रमः क्षुत्तृष्णाभिभूतश्चैकस्मिन्देशे नीलं गहनंवनषण्डमपश्यत्
% Check verse!
तस्याविशेषतो वनषण्डस्य मध्येऽतीव रमणीयं सरो दृष्ट्वा साश्व एवव्यगाहत
% Check verse!
अथाश्वस्तः स बिसमृणालमश्वायाग्रतो निक्षिप्य पुष्करिणीतीरेसंविवेश ततः शयानो मधुरंगीतमशृणोत्
% Check verse!
स श्रुत्वाऽचिन्तयन्नेह मनुष्यगतिं पश्यामि कस्य स्वल्वयंगीतंशब्द इति
% Check verse!
अथापश्यत्कन्यां परमरूपदर्शनीयां पुष्पाण्यवचिन्वतीं गायन्तीं चअथ सा राज्ञः समीपे पर्यक्रामत्
% Check verse!
तामब्रवीद्राजा कस्यासि भद्रे का वा त्वमिति सा प्रत्युवाचकन्याऽस्मीति तां राजोवाचार्थी त्वयाऽहमिति
% Check verse!
अथोवाच कन्या समयेनाहं शक्या त्वया लब्धुं नान्यथेति राजा तांसमयमपृच्छत् मण्डूकराजस्य कः समय इति ततः कन्येदमुवाच नोदकं मेदर्शयितव्यमिति
% Check verse!
स राजा तां बाढमित्युक्त्वा तां समागम्य तयासह तत्र तस्थौ
% Check verse!
ततस्तत्रैवासीने राजनि सा सेनाऽन्वगच्छत्
% Check verse!
पदेनानुपदं दृष्ट्वा राजानं परिवार्यातिष्ठत अथ पर्याश्वस्तश्चराजा तयैव सह शिविकया प्रायादविषादः स्वनगरमनुप्राप्य रहसि तया रममाणोन कांश्चिदपश्यत्
% Check verse!
अथ प्रधानामात्योऽभ्याशचरास्तस्य स्त्रियोऽपृच्छत्
% Check verse!
किमत्र प्रयोजनं वर्तते इत्यथाब्रुवंस्ताः स्त्रियः
% Check verse!
अपूर्वमिव पश्याम उदकं नात्र नीयत इत्यथामात्योऽनुदकं वनंकारयित्वोदारवृक्षं बहुपुष्पफलमूलं तस्य मध्ये मुक्ताजालमयीं पार्श्वेवापीं गूढां सुधोपलिप्तां स रहस्युपगम्य राजानमब्रवीत्
% Check verse!
वनमिदमुदारमनुदकं साध्वत्र सहैतया गम्यतामिति
% Check verse!
स तस्य वचनात्तयैव सह देव्या तद्वनं प्राविशत्सकदाचित्तस्मिन्कानने रम्ये तयैव स व्यवाहरदथक्षुत्तृष्णार्दितःश्रान्तोऽतिमुक्तागारमपश्यत्
% Check verse!
तत्प्रविश्यराजा सह प्रियया सुसंस्कृतां विमलां सलिलपूर्णांवापीमपश्त्
% Check verse!
दृष्ट्वैव च तां तस्याश्च तीरे सहैव तया देव्याऽवातिष्ठत्
% Check verse!
अथ ता देवीं स राजाऽब्रवीत्साध्ववतर वापीसलिलमिति सा तद्वचश्रुत्वाऽवतीर्य वापीं न्यमज्जन्न पुनरुदकादुदमज्जत्
% Check verse!
तां स मृगयमाणो राजा नापश्यद्वापीमथ निःस्राव्य मण्डूकंश्वभ्रमुखे दृष्ट्वा क्रुद्ध आज्ञापयामास स राजा
% Check verse!
सर्वमण्डूकवधः क्रियतामिति योमयाऽर्थी स मांमृतमण्डूकोपायनमादायोपतिष्ठेदिति
% Check verse!
अथ मण्डूकवधे घोरे क्रियमाणे दिक्षु सर्वासु रमण्डूकान्भयमाविवेशते भीता मण्डूकराज्ञे यथावृत्तं न्यवेदयन्
% Check verse!
ततो मण्डूकराट् तापसवेषधारी राजानमभ्यगच्छदुपेत्य चैनमुवाच
% Check verse!
मा राजन्क्रोधवशं गमः प्रसादं कुरु नार्हसि मण्डूकानामनपराधिनांवधं कर्तुमिति श्लोकौ चात्र भवतः
\twolineshloka
{मा मण्डूकाञ्जिघांसीस्त्वं कोपं संधारयाच्युत}
{प्रक्षिपन्तिवधाद्भेका जनानां परिजानताम्}


\twolineshloka
{प्रतिजानीहि नैतांस्त्वं प्राप्य क्रोधं विमोक्ष्यसे}
{अलं कृत्वातवाधर्मं मण्डूकैः किं हतैर्हि ते}


% Check verse!
तमेवंवादिनमिष्टजनवियोगशोकपरीतात्मा राजाऽथोवाच
% Check verse!
न हि क्षम्यते तन्मया हनिष्याम्येतानेतैर्दुरात्मभिः प्रिया मेभक्षिता सर्वथैव मे वध्या मण्डूका नार्हसि विद्वन्मामुपरोद्धुमिति
% Check verse!
स तद्वाक्यमुपलभ्यव्यथितेन्द्रियमनाः प्रोवाच प्रसीदराजन्नहमायुर्नाम मण्डूकराजो मम सा दुहिता सुशोभना नाम तस्या हिदौःशील्यमेतद्बहवस्तया राजानो विप्रलब्धपूर्वा इति
% Check verse!
तमब्रवीद्राजा या सा तव रदुहिता तया समर्थी सा मे दीयतामिति
% Check verse!
अथैनां राज्ञे पिताऽदादब्रवीच्चैनामेनं राजानं शुश्रूषस्वेति
% Check verse!
स एवमुक्त्वा दुहितरं क्रुद्धः शशाप यस्मात्त्वया राजानोविप्रलब्धा बहवस्तस्मादब्रह्मण्यानि तवापत्यानिभविष्यन्त्यानृतिकत्वात्तवेति
% Check verse!
स च राजा तामुपलभ्य तस्यां सुरतगुणनिबद्धहृदयोलोकत्रयैश्वर्यमिवोपलभ्य हर्षबाष्पसन्दिग्धया वाचा प्रणिपत्याभिपूज्यमण्डूकराजमब्रवीदनुगृहीतोस्मीति
% Check verse!
स च मण्डूकराजो दुहितरमनुज्ञाप्य यथागतमगच्छत्
% Check verse!
अथ कस्यचित्कालस्य तस्यां कुमारास्त्रयस्तस्य राज्ञः संबभूवुःशलो दलो बलश्चेति ततस्तेषां ज्येष्ठं शलं समये पिता राज्येऽभिषिच्यतपसि धृतात्मा वनं जगाम
% Check verse!
अथ कदाचिच्छलो मृगयामनुचरन्मृगमासाद्य रथेनान्वधावत्
% Check verse!
सूतं चोवाच शीघ्रं मां वाहयस्वेति स तथोक्त सूतो राजानमब्रवीत्
% Check verse!
न क्रियतामनुबन्धो नैष शक्यस्त्वया मृगोऽयं ग्रहीतुं यद्यपि तेरथे युक्तौ वाम्यौ स्यातामिति ततोऽब्रवीद्राजा सूतमाचक्ष्व मे वाभ्यौवाजिनौ त्वा पृच्छामीति ॥ स एवमुक्तो राजभयभीतः सूतोवामदेवशापभयभीतश्चास्योपाचख्यौ ॥ वामदेवस्याश्वौ वाम्यौ मनोजवाविति
% Check verse!
अथैनमेवं ब्रुवाणमब्रवीद्राजा वामदेवाश्रमं प्रयादहीति स गत्वावामदेवाश्रमं तमृषिमब्रवीत्
% Check verse!
भगवन्मृगो मे विद्धः पलायते संभावयितुमर्हसि वाम्यौ दातुमितितमब्रवीदृषिर्ददानि ते वाम्यौ कृतकार्येण भवता ममैव वाम्यौनिर्यातयितव्यौ क्षिप्रमिति स च तावश्वौ प्रतिगृह्यानुज्ञाप्य ऋषिंप्रायाद्वाजियुक्तेन रथेन मृगं प्रतिगच्छंश्चाब्रवीत्सूतमश्वरत्नद्वयमावयोर्योग्यं नैतौ प्रतिदेयौवामदेवायेत्युक्त्वा मृगमवाप्य स्वनगरमेत्याश्वावन्तःपुरेऽस्थापयत्
% Check verse!
अथर्षिश्चिन्तयामास तरुणो राजपुत्रः कल्याणं पत्रमासाद्य रमते नमे वाम्यौ प्रतिनिर्यातयत्यहो कष्टमिति
% Check verse!
स मनसा विचिन्त्य मासि पूर्णे शिष्यमब्रवीत्
% Check verse!
गच्छात्रेय राजानं ब्रूहि यदि पर्याप्तंनिर्यातयोगाध्यायवाम्याविति स गत्वैवं तं राजानमब्रवीत्तराजाप्रत्युवाच राज्ञामेतद्वाहनमनर्हा ब्राह्मणा रत्नानामेवंविधानां नकिंचित् ब्राह्मणानामश्वैः कार्यं साधु गम्यतामिति
\twolineshloka
{स गत्वैतदुपाध्यायायाचष्ट तच्छ्रुत्वा वचनमप्रियं वामदेवःक्रोधपरीतात्मा स्वयमेव राजानमभिगम्याश्वार्थमचोदयन्न चाददद्राजा ॥वामदेव उवाच}
{}


\threelineshloka
{प्रपच्छ वाम्यौ मम पार्थिव त्वंकृतं हि ते कार्यमाभ्यामशक्यम्}
{मा त्वा वधीद्वरुणो घोरपाशै-र्ब्रह्मक्षत्रस्यान्तरे वर्तमानः ॥राजोवाच}
{}


\threelineshloka
{अनड्वाहौ सुव्रतौ साधु दान्ता-वेतद्विप्राणां वाहनं वामदेव}
{ताभ्यां याहि त्वं यत्र कामो महर्षे-च्छन्दांसि वै त्वादृशं संवहन्ति ॥वामदवे उवाच}
{}


\threelineshloka
{छन्दांसि वै मादृशं संवहन्तिलोकेऽमुष्मिन्पार्थिव यानि सन्ति}
{अस्मिस्तु लोके मम यानमेत-दस्मद्विधानामपरेषां च राजन् ॥राजोवाच}
{}


\threelineshloka
{चत्वारस्त्वां वा गर्दभाः संवहन्तुश्रेष्ठाश्वतर्यो हरयो वातरंहाः}
{तैस्त्वं याहि रक्षत्रियस्यैष वाहोममैव वाम्यौ न तवैतौ हि विद्धि ॥वामदेव उवाच}
{}


\fourlineindentedshloka
{घोरं व्रतं ब्राह्मणस्यैतदाहु-रेतद्राजन्यदिहाजीवमानः}
{अयस्मया घोररूपा महान्त-श्चत्वारो वा यातुधानाः सुरौद्राः}
{मया प्रयुक्तास्त्वद्वधमीप्समानावहन्तु त्वां शितशूलाश्चतुर्धा ॥राजोवाच}
{}


\threelineshloka
{ये त्वां विदुर्ब्राह्मणं वामदेववाचा हन्तुं मनसा कर्मणा वा}
{ते त्वां सशिष्यमिह पातयन्तुमद्वाक्यनुन्नाः शितशूलासिहस्ताः ॥[वामदेव उवाच}
{}


\threelineshloka
{ममैतौ वाम्यौ प्रतिगृह्य राजन्पुनर्ददानीति प्रपद्य मे त्वम्}
{प्रयच्छ शीघ्रं मम वाम्यौ त्वमश्वौयद्यात्मानं जीवितुं ते क्षमं स्यात् ॥राजोवाच}
{}


\threelineshloka
{न ब्राह्मणेभ्यो मृगया प्रसूतान त्वाऽनुशास्म्यद्यप्रभृति ह्यसत्यम्}
{तवैवाज्ञां संप्रणिधाय सर्वांतथा ब्रह्मन्पुण्यलोकं लभेयम् ॥]वामदेव उवाच}
{}


\threelineshloka
{नानुयोगा ब्राह्मणानां भवन्तिवाचा राजन्मनसा कर्मणा वा}
{यस्त्वेवं ब्रह्म तपसाऽन्वेति विद्वां-स्तेन श्रेष्ठो भवति हि जीवमानः ॥मार्कण्डेय उवाच}
{}


\twolineshloka
{एवमुक्ते वामदेवेन राजन्समुत्तस्थू राक्षसा घोररूपाः}
{तैः शूलहस्तैर्वध्यमानः स राजाप्रोवाचेदं वाक्यमुच्चैस्तदानीम्}


\twolineshloka
{इक्ष्वाकवो वा यदि मां त्यजेयु-र्ये ते विधेया भुवि चान्ये महीपाः}
{नोत्स्रक्ष्येऽहं वामदेवस् वाम्यौनैवंविधा धर्मशीला भवन्ति}


\twolineshloka
{एवं ब्रुवन्नेव स यातुधानै-र्हतो जगामाशु महीं क्षितीशः}
{ततो विदित्वा नृपतिं निपातित-मिक्ष्वाकवो वै दलमभ्यषिञ्चन्}


\twolineshloka
{राज्ये तदा तत्र गत्वा स विप्रःप्रोवाचेदं वचनं वामदेवः}
{दलं राजानं ब्राह्मणानां हि देय-मेवं राजन्सर्वधर्मेषु दृष्टम्}


\twolineshloka
{बिभेषि चेत्त्वमधर्मान्नरेन्द्रप्रयच्छ मे शीघ्रमेवाद्य वाम्यौ}
{एतच्छ्रुत्वा वामदेवस् वाक्यंस पार्थिवः सूतमुवाच रोषात्}


\threelineshloka
{एकं हि मे सायकं चित्ररूपंदुग्धं विषेणाहरसंगृहीतम्}
{येन विद्धो वामदेवः शयीतसंदश्यमानः श्वभिरार्तरूपः ॥वामदेव उवाच}
{}


\threelineshloka
{जानामि पुत्रं दशवर्षं तवाहंजातं महिष्यां श्येनजितं नरेन्द्र}
{एतं जहि त्वं मद्वचनात्प्रणुन्न-स्तूर्णं प्रियं सायकैर्घोररूपैः ॥मार्कण्डेय उवाच}
{}


\twolineshloka
{एवमुक्तो वामदेवेन राजन्न-न्तःपुरे राजपुत्रं जघान}
{स सायकस्तिग्मतेजा विसृष्टःश्रुत्वा दलस्तत्र वाक्यं बभाषे}


\threelineshloka
{इक्ष्वाकवो हन्त चरामि वः प्रियंनिहन्मीमं विप्रमद्य प्रमथ्य}
{आनीयतामपरस्तिग्मतेजाःपश्यध्वं मे वीर्यमद्य क्षितीशाः ॥वामदेव उवाच}
{}


\threelineshloka
{यत्त्वमेनं सायकं घोररूपंविषेण दिग्धं मम संदधासि}
{न त्वेतं त्वं शरवर्षं विमोक्तुंसंधातुं वा शक्ष्यसे मानवेन्द्र ॥राजोवाच}
{}


\twolineshloka
{इक्ष्वाकवः पश्त मां गृहीतंन वै शक्नोम्येष शरं विमोक्तुम्न चास्य कर्तुं नाशमभ्युत्सहामिआयुष्मान्वै जीवतु वामदेवः ॥वैमदेव उवाच}
{}


\twolineshloka
{संस्पृश्यैनां महीषीं सायकेनततस्तस्मादेनसो मोक्ष्यसे त्वम्}
{ततस्तथा कृतवान्पार्थिवस्तुततो मुनिं राजपुत्री बभाषे}


\threelineshloka
{यथा युक्ता वामदेवाहमेनंदिनेदिने संदिशन्ती नृशंसम्}
{ब्राह्मणेभ्यो मृगयती सूनृतानितथा ब्रह्मन्पुण्यलोकं लभेयम् ॥वामदेव उवाच}
{}


\threelineshloka
{त्वया त्रातं राजकुलं शुभेक्षणेवरं वृणीष्वाप्रतिमं ददानि ते}
{प्रशाधीमं स्वजनं राजपुत्रिइक्ष्वाकुराज्यं सुमहच्चाप्यनिन्द्ये ॥राजपुत्र्युवाच}
{}


\threelineshloka
{वरं वृणे भगवंस्त्वेवमेषविमुच्यतां किल्विषादद्य भर्ता}
{शिवेन जीवामि सपुत्रबान्धवंवरो वृतो ह्येष मया द्विजाग्र्य ॥मार्कण्डेय उवाच}
{}


\twolineshloka
{श्रुत्वा वच स मुनी राजपुत्र्या-स्तथाऽस्त्विति प्राह कुरुप्रवीर}
{ततः स राजा मुदितो बभूववाम्यौ चास्मै प्रददौ संप्रणम्य}


\chapter{अध्यायः १९६}
\twolineshloka
{वैशंपायन उवाच}
{}


\twolineshloka
{मार्कण्डेयमृषयो ब्राह्मणा युधिष्ठिरश्च}
{पर्यपृच्छन्नृषिः केन दीर्घायुरासीद्बकः}


\twolineshloka
{मार्कण्डेयस्तु तान्सर्वान्प्रत्युवाच महातपाः}
{दीर्घायुश्च बकोराजन्नृषिर्नात्र विचारणा}


\twolineshloka
{एतच्छ्रुत्वा तु कौन्तेयो भ्रातृभिः सह भारत}
{मार्कण्डेयं पर्यपृच्छद्धर्मराजो युधिष्ठिरः}


\twolineshloka
{बकदाल्भ्यौ महात्मानौ श्रूयेते चिरजीविनौ}
{सखायौ देवराजस् यतावृषी लोकसंमतौ}


\threelineshloka
{एतदिच्छामि भगवन्बकशक्रसमागमम्}
{सुखदुःखसमायुक्तं तत्त्वेन कथितं त्वया ॥मार्कण्डेय उवाच}
{}


\twolineshloka
{वृत्ते देवासुरे राजन्संग्रामे रोमहर्षणे}
{त्रयाणामपि लोकानामिन्द्रो लोकाधिपोऽभवत्}


\threelineshloka
{सम्यग्वर्षति पर्जन्ये सस्यसंपद उत्तमाः}
{निरामयाः सुधर्मिष्ठाः प्रजा धर्मपरायणाः}
{मुदितश्च जनः सर्वः स्वधर्मेषु व्यवस्थितः}


\twolineshloka
{ताः प्रजा मुदिताः सर्वा दृष्ट्रा बलनिषूदनः}
{ततस्तु मुदितोराजन्देवराजः शतक्रतुः}


\twolineshloka
{ऐरावतं समास्थाय अपश्यन्मुदिताः प्रजाः}
{आश्रमांश्च विचित्रांश् नदीश्च विविधाः शुभाः}


\twolineshloka
{नगराणि समृद्धानि घेटाञ्जनपदांस्तथा}
{प्रजापालनदक्षांश्च नरेन्द्रान्धर्मचारिणः}


\twolineshloka
{उदपानं प्रपा वापी तटाकानि सरांसि च}
{नानाब्रह्रह्मसमाचारैः सेवितानि द्विजोततमैः}


\twolineshloka
{ततोऽवतीर्य रम्यायां पृथ्व्यां राजञ्छतक्रतुः}
{तत्र रम्ये शिवे देशे बहुवृक्षसमाकुले}


\threelineshloka
{पूर्वस्यां दिशि रम्यायां समुद्राभ्याशतो नृप}
{तत्राश्रमपदं रम्यं मृगद्विजनिषेवितम्}
{तत्राश्रमपदे रम्ये बकं पश्यति देवराट्}


\twolineshloka
{बकस्तु दृष्ट्वा देवेन्द्रं दृढं प्रीतमनाऽभवत्}
{पाद्यासनार्थदानेन फलमूलैरथार्चयत्}


\twolineshloka
{सुखोपविष्टो वरदस्ततस्तु बलसूदनः}
{ततः प्रश्नं बकं देवं उवाच त्रिदशेश्वरः}


\threelineshloka
{शतं वर्षसहस्राणि मुने जातस्य तेऽनघ}
{समाख्याहि मम ब्रह्मन्किं दुःखं चिरजीविनाम् ॥बक उवाच}
{}


\twolineshloka
{अप्रियैः सह संवासः प्रियैश्चापि विनाभवः}
{असद्भिः संप्रयोगश्च तद्दुःखं चिरजीविनाम्}


\twolineshloka
{पुत्रदारविनाशोऽत्र ज्ञातीनां सुहृदामपि}
{परेष्वायत्तता कृच्छ्रं किंनु दुःखतरं ततः}


\twolineshloka
{नान्यद्दुःखतरं किंचिल्लोकेषु प्रतिभाति मे}
{अर्थैर्विहीनः पुरुषः एरैः संपरिभूयते}


\twolineshloka
{अकुलानां कुले भावं कुलीनानां कुलक्षयम्}
{संयोगं विप्रयोगं च पश्यन्ति चिरजीविनः}


\twolineshloka
{अपिप्रत्यक्षमेवैतत्तव देघ शतक्रतो}
{अकुलानां समृद्धानां कथं कुलविपर्ययः}


\twolineshloka
{देवदानवगन्धर्वमनुष्योरगराक्षसाः}
{प्राप्नुवन्ति विपर्यासं किंनु दुःखतरं ततः}


\twolineshloka
{कुले जाताश् क्लिश्यन्ते दौष्कुलेयवशानुगाः}
{आढ्यैर्दरिद्राऽवमताः किंनु दुःखतरं ततः}


\fourlineindentedshloka
{लोके वैधर्म्यमेतत्तु दृश्ते बहुविस्तरम्}
{हीनज्ञानाश्च हृष्यन्ते क्लिश्यन्ते प्राज्ञकोविदाः}
{बहुदुःखपरिक्लेशं मानुष्यमिह दृश्यते ॥इन्द्र उवाच}
{}


\threelineshloka
{पुनरेव महाभाग देवर्षिगणसेवित}
{समाख्याहि मम ब्रह्मन्किं सुखं चिरजीविनाम् ॥बक उवाच}
{}


\twolineshloka
{अष्टमे द्वादशे वाऽपि शाकं यः पचते गृहे}
{कुमित्राण्यनपाश्रित् किं वै सुखतरं ततः}


\twolineshloka
{यत्राहानि न गण्यन्ते नैनमाहुर्महाशनम्}
{अपि शाकं पचानस्य सुखं वै मघवन्गृहे}


\twolineshloka
{आर्जितं स्वेन वीर्येण नाप्यपाश्रित्य कंचन}
{फलशाकमपि श्रेयो भोक्तुं ह्यकृपणं गृहे}


\twolineshloka
{परस्य तु गृहे भोक्तुः परिभूतस् नित्यशः}
{सुमृष्टमपि न श्रेयो विकल्पोऽयमतः सताम्}


\twolineshloka
{श्ववद्धि लोलुपो यस्तु परान्नं भोक्तुमिच्छति}
{धिगस्तु तस्य तद्भुक्तं कृपणस् दुरात्मनः}


\twolineshloka
{यो दत्त्वाऽतिथिभृत्येभ्यः पितृभ्यश्च द्विजोत्तमः}
{शिष्टान्यन्नानि यो भुङ्क्ते किं वै सुखतरं ततः}


\twolineshloka
{अतो मृष्टतरं नान्यत्पूतं किंचितच्छतक्रतो}
{दत्त्वा यस्त्वतिथिभ्योऽन्नं भुङ्क्ते तेनैव नित्यशः}


\twolineshloka
{तावतां गोसहस्राणां फलं प्राप्नोति दायकः}
{यदेनो यौवनकृतंतत्सर्वं नश्यते ध्रुवम्}


\twolineshloka
{रसदक्षिणस् भुक्तस् द्विजस्य तु करे गतम्}
{यद्वारि वारिणा सिञ्चेत्तद्ध्येनस्तरते क्षणात्}


\twolineshloka
{एताश्चान्याश्च वै बह्वीः कथयित्वा कथाः शुभाः}
{बकेन सह देवेन्द्र आपृच्छ्य त्रिदिवं गतः}


\chapter{अध्यायः १९७}
\twolineshloka
{वैशंपायन उवाच}
{}


% Check verse!
ततः पाण्डवाः पुनर्मार्कण्डेयमूचुः
% Check verse!
कथितं ब्राह्मणमाहात्म्यं राजन्यमाहात्म्यमिदानीं शुश्रूषामह इतितानुवाच मार्कण्डेयो महर्षिः श्रूयतामिदानीं राजन्यानां माहात्म्यमिति ॥ कुरूणामन्यतमः सुहोत्रो नाम राजा महर्षीनभिगम्य निवृत्यरथस्थमेवराजानमौशीनरं शिबिं ददर्शाभिमुखं तौ समेत्य परस्परेण यथावयः पूजांप्रयुज्य गुणसाम्येन परस्परेण तुल्यात्मानौ विदित्वाऽन्योन्यस्यपन्थानं न ददतुस्तत्र नारदः प्रादुरासीत्किमिदं भवन्तौ परस्परस्यपन्थानमावृत्य तिष्ठत इति
% Check verse!
तावूचतुर्नारदं नैतद्भगवन्पूर्वकर्मकर्त्रादिभिर्विशिष्टस् पन्था उपदिश्यतेसमर्थाय वा आवां च सख्यं परस्परेणोपगतौ तच्चावधानतोऽत्युत्कृष्टमधरोत्तरंपरिभ्रष्टम् ॥नारदस्त्वेवमुक्तः श्लोकत्रयमपठत्
\twolineshloka
{क्रूरः कौरव्य मृदवे मृदुः क्रूरे च कौरव}
{साधुश्चासाधवे साधुः साधवे नाप्नुयात्कथम्}


\twolineshloka
{कृतं शतगुणं कुर्यान्नास्ति देवेषु निर्णयः}
{औशीनरः साधुशीलो भवतो वै महीपतिः}


\twolineshloka
{जलेत्कदर्यं दानेन सत्येनानृतवादिनम्}
{क्षमया क्रूरकर्माणमसाधुं साधुना जयेत्}


% Check verse!
तदुभावेव भवन्तावुदारौ य इदानीं भवद्भ्यामन्यतमः सोपसर्पतुएतद्वै निदर्शनमित्युक्ताव तूष्णीं नारदो बभूव ॥ एतच्छ्रुत्वा तुकौरव्यः शिबिं प्रदक्षिणं कृत्वा पन्थानं दत्त्वा बहुकर्मभिः प्रशस्यप्रययौ
% Check verse!
तदेतद्राज्ञो महाभाग्यमप्युक्तवान्नारदः
\chapter{अध्यायः १९८}
\twolineshloka
{[*मार्कण्डेय उवाच}
{}


% Check verse!
इदमन्यच्छ्रूतां ययातिर्नाहुषो राजा राज्यस्थः पौरजनावृतआसांचक्रे ॥ गुर्वर्थी ब्राह्मण उपेत्याब्रवीत् भो राजन्गुर्वर्थंभिक्षेयं समयादिति राजोवाच
\twolineshloka
{ब्रवीतु भगवान्समयमिति ॥ब्राह्मण उवाच}
{}


\threelineshloka
{विद्वेषणं परमं जीवलोकेकुर्यान्नरः पार्थिव याच्यमानः}
{तं त्वां पृच्छामि कथं तु राज-न्दद्याद्वान्दयितं च मेऽद्य ॥राजोवाच}
{}


\twolineshloka
{नचानुकीर्तये दद्य दत्त्वाअयाच्यमर्थं न च संशृणोमि}
{प्राप्यमर्थं च संश्रुत्यतं चापि दत्त्वा सुसुखी भवामि}


\twolineshloka
{ददामि ते रोहिणीनां सहस्रंप्रियो हि मे ब्राह्मणो याचमानः}
{न मे मनः कुप्यति याचमानेदत्तं न शोचामि कदाचिदर्थम्}


\twolineshloka
{इत्युक्त्वा ब्राह्मणा राजा गोसहस्रं ददौ}
{प्राप्तवांश्च गवां सहस्रं ब्राह्मण इति}


\chapter{अध्यायः १९९}
\twolineshloka
{वैशंपायन उवाच}
{}


% Check verse!
भूय एव महाभाग्यं कथ्यतामित्यब्रवीत्पाण्डवः
% Check verse!
अथाचष्ट मार्कण्डेयो महाराज वृषदर्भसेदुकनामानौ राजानौनतिमार्गरतावस्त्रोपास्त्रकृतिनौ
% Check verse!
सेदुको वृषदर्भस् बालस्यैव उपांशुव्रतमभ्यजानात् कुप्यमदेयंब्राह्मणस्य
% Check verse!
अथ तं सेदुकं ब्राह्मणः कश्चिद्वेदाध्ययनसंपन्न आशिषं दत्त्वागुर्वर्थी भिक्षितवान्
% Check verse!
अश्वसहस्रं मे भवान्ददात्विति ॥ तं सेदुको ब्राह्मणमब्रवीत्
% Check verse!
नास्ति संभवो गुर्वर्थं दातुमिति
% Check verse!
स त्वं गच्छ वृषदर्भसकाशम् ॥ राजा परमधर्मज्ञो ब्राह्मण तंभिक्षस्व ॥ स ते दास्यति तस्यैतदुपांशुव्रतमिति
% Check verse!
अथ ब्राह्मणो वृषदर्भसकाशं गत्वा अश्वसहस्रमयाचत ॥ स राजा तंकशेनाताडयत्
% Check verse!
तं ब्राह्मणोऽब्रवीत् ॥ किं हिंस्यनागसं मामिति
\twolineshloka
{एवमुक्त्या तं शपन्तं राजाऽऽह ॥ विप्र किं यो न ददातितुभ्यमुताहोस्विद्ब्राह्मण्यमेतत् ॥कब्राह्मण उवाच}
{}


% Check verse!
राजाधिराज तव समीपं सेदुकेन प्रेषितो भिक्षितुमागतःतेनानुशिष्टेन मया त्वं भिक्षितोसि
% Check verse!
पूर्वाह्णे ते दास्यामि यो मेऽद्य बलिरागमिष्यति ॥ यो हन्यतेकशया कथं मोघं क्षेपणं तस्य स्यात्
% Check verse!
इत्युक्त्वा ब्राह्मणाय दैवसिकामुत्पत्तिं प्रदात् ॥अधिकस्याश्वसहस्रस्य मूल्यमेवादादिति
\chapter{अध्यायः २००}
\twolineshloka
{मार्कण्डेय उवाच}
{}


% Check verse!
देवानां कथा संजाता महीतलं गत्वा महीपतिं शिबिमौशीनरं साध्वेनंशिबिं जिज्ञास्याम इति ॥ एवं भो इत्युक्त्वा अग्नीन्द्रावुपतिष्ठेताम्
% Check verse!
अग्निः कपोतरूपेण तमभ्यधावदामिषार्थमिन्द्रः श्येनरूपेण
% Check verse!
अथ कपोतो राज्ञो दिव्यासनासीनस्योत्सङ्गं न्यपतत्
% Check verse!
अथ पुरोहितो राजानमब्रवीत् ॥ प्राणरक्षार्थं श्येनाद्भीतोभवन्तं प्राणार्थी प्रपद्यते
% Check verse!
वसु ददातु अन्तवान्पार्थिवोऽस्य निष्कृतिं कुर्यात् घोरं कपोतस्यनिपातमाहुः
% Check verse!
अथ कपोतो राजानमब्रवीत् ॥ प्राणरक्षणार्थं श्येनाद्भीतो भवन्तंप्राणार्थी प्रपद्ये अङ्गैरङ्गानि प्राप्यार्थी मुनिर्भूत्वाप्राणांस्त्वां प्रपद्ये
% Check verse!
स्वाध्यायेन कर्शितं ब्रह्मचारिणं मां विद्धि ॥ तपसा दमेनयुक्तमाचार्यस्याप्रतिकूलभाषिणम् ॥ एवंयुक्तमपापं मां विद्धि
\twolineshloka
{गदामि वेदानविचिनोमि छन्दःसर्वेवेदा अक्षरशो मे अधीताः}
{न साधु दानं श्रोत्रियस्य प्रदानंमा प्रादाः श्येनाय न कपोतोऽस्मि}


% Check verse!
अथ श्येनो राजानमब्रवीत्
\threelineshloka
{पर्यायेण वसतिर्वा भवेषुसर्गे जातः पूर्वमस्मात्कपोतात्}
{त्वमाददानोऽथ कपोतमेनंमा त्वं राजन्विघ्नकर्ता भवेथाः ॥राजोवाच}
{}


\twolineshloka
{केनेदृशी जातु परा हि दृष्टावागुच्यमाना शकनेन संस्कृता}
{यां वै कपोतो वदते यां च श्येनउभौ विदित्वा कथमस्तु साधु}


\twolineshloka
{नास्य वर्षं वर्षति वर्षकालेनास्य बीजं रोहति काल उप्तम्}
{भीतं प्रपन्नं यो हि ददाति शत्रवेन त्राणं लभेत्राणमिच्छन्स काले}


\twolineshloka
{जाता ह्रस्वा प्रजा प्रमीयतेसदा न वासं पितरोऽस्य कुर्वते}
{भीतं प्रपन्नं यो हि ददाति शत्रवेनास्य देवाः प्रतिगृह्णन्ति हव्यम्}


\twolineshloka
{मोघमन्नं विन्दति चाप्रचेताःस्वर्गाल्लोकाद्धश्यति शीघ्रमेव}
{भीतं प्रपन्नं यो हि ददाति शत्रवेसेन्द्रा देवाः प्रहरन्त्यस्य वज्रम्}


\threelineshloka
{उक्षाणं पक्त्वा सह ओदनेनअस्मात्कपोतात्प्रति ते नयन्तु}
{यस्मिन्देशे रमसेऽतीव श्येनतत्रमांसं शिवयस्ते वहन्तु ॥श्येन उवाच}
{}


\threelineshloka
{नोक्षाणो राजन्प्रार्थयेयं न चान्यदस्मान्मांसमधिकं वा कपोतात्}
{देवैर्दत्तः सोऽद्य ममैष भक्ष-स्तन्मे ददस्व शकुनानामभावात् ॥राजोवाच}
{}


\twolineshloka
{उक्षाणं वेहतमनूनं नयन्तुते पश्यन्तु पुरुषा ममैव}
{भयाहितस् दायं ममान्तिकात्त्वांप्रत्याम्नायं तु त्वं ह्येनं मा हिंसीः}


\twolineshloka
{त्यजे प्राणाननैव दद्यां कपोतंसौम्यो ह्ययं किं न जानासि श्येन}
{यथा क्लेशं मा कुरुष्वेह सौम्यनाहं कपोतमर्पयिष्ये कथंचित्}


\threelineshloka
{यथा मां वै साधुवादैः प्रसन्नाःप्रशंसेयुः शिबयः कर्मणा तु}
{यथा श्येन प्रियमेव कुर्यांप्रशाधि मां यद्वदेस्तत्करोमि ॥श्येन उवाच}
{}


% Check verse!
ऊरोर्दक्षिणादुत्कृत्य स्वपिशितं तावद्राजन्यावन्मांसं कपोतेनसमम् ॥ तथा तस्मात्साधु त्रातः कपोतः प्रशंसेयुश्च शिबय कृतं च प्रियंस्यान्ममेति
% Check verse!
अथ स दक्षिणादूरोरुत्कृत्य स्वमांसपेशीं तुलयाधारयत् ॥ गुरुतरएव कपोत आसीत्
% Check verse!
पुनरन्यमुच्चकर्त गुरुतर एव कपोतः ॥ एवं सर्वं समधिकृत्य शरीरंतुलायामारोपयामास रतत्तथापि गुरुतर एव कपोत आसीत्
% Check verse!
अथ राजा स्वयमेव तुलामारुरोह ॥ न च व्यलीकमासीद्राज्ञएतद्वृत्तान्तं दृष्टाव त्रात इत्युक्त्वा प्रालीयत श्येनः अथ राजाअब्रवीत्
\threelineshloka
{कपोतं विद्युः शिवयस्त्वां कपोतपृच्छामि ते शकुने को नु श्येनः}
{नानीश्वर ईदृशं जातु कुर्या-देतं प्रश्नं भगवन्मे विचक्ष्व ॥कपोत उवाच}
{}


\twolineshloka
{वैश्वानरोऽहं ज्वलनो धूमकेतु-रथैव श्येनो वज्रहस्तः शचीपतिः}
{साधु ज्ञातुं त्वामृषभं सौरथेयनौ जिज्ञासया त्वत्सकाशंप्रपन्नौ}


\twolineshloka
{यामेतां पेशीं मम निष्क्रयायप्रादाद्भवानसिनोत्कृत्य राजन्}
{एतद्वो लक्ष्म शिवं करोमिहिरण्यवर्णं रुचिरं पुण्यगन्धम्}


\twolineshloka
{एतासां प्रजानां पालयिता यशस्वीसुरर्षीणामथ संमतो भृशम्}
{एतस्मात्पार्श्वात्पुरुषो जनिष्यतिकपोतरोमेति च तस्य नाम}


% Check verse!
कपोतरोमाणं शिबिनौद्भिदं पुत्रं प्राप्स्यसि नृपवृषसंहननंयशोदीप्यमानं द्रष्टासि शूरमृषभं सौरथानाम्
\chapter{अध्यायः २०१}
\twolineshloka
{वैशंपायन उवाच}
{}


% Check verse!
भूय एव महाभाग्यं कथ्यतामित्यब्रवीत्पाण्डवो मार्कण्डेयम् ॥अथाचष्ट मार्कण्डेयः ॥ अष्टकस्य वैश्वामित्रेरश्वमेधे सर्वे राजानःप्रागच्छन्
% Check verse!
भ्रातरश्चास्य प्रतर्दनो वमुमनाः शिबिरौशीनर रइति स चसमाप्तयज्ञो भ्रातृभिः सह रथेन प्रायात्ते चनारदमागच्छन्तमभिवाद्यारोहतु भवान्रथमित्यब्रुवन्
% Check verse!
तांस्तथेत्युक्त्वा रथमारुरोह ॥ अथ तेषामेकः सुरर्षिंनारदमब्रवीत् ॥ प्रसाद्य भगवन्तं किंचिदिच्छेयं प्रष्टुमिति
% Check verse!
पृच्छेत्यब्रवीदृषिः ॥ रसोऽब्रवीदायुष्मन्तः सर्वगुणप्रमुदिताः ॥ अथायुष्मन्तं स्वर्गस्थानं चतुर्भिर्यातव्यं स्यात्कोऽवतरेत् ॥अयमष्टकोऽवतरेदित्यब्रवीदृषिः
% Check verse!
किंकारणमित्यपृच्छत् ॥ अथाचष्टाष्टकस्य गृहे मया उपितं स मांरथेनानुप्रावहदथापश्यमनेकानि गोसहस्राणि वर्णशो विविक्तानि तमहमपृच्छंकस्येमा गाव इति सोब्रवीत् ॥ मयानिसृष्टाइत्येतास्तेनैवस्वयं श्लाघतिकथितेन ॥ एपोवतरेदथ त्रिभिर्यातव्यं सांप्रतं कोऽवतरेत्
% Check verse!
प्रतर्दन इत्यत्रवीदृषिः ॥ तत्र किं कारणं प्रतर्दनस्यापि गृहेमयोपितं स मां रथेनानुप्रावहत्
% Check verse!
अथैनं ब्राह्मणोऽभिक्षेताश्वं मे ददातु भवान्निवृत्तोदास्यामीत्यब्रवीद्ब्राह्मणं त्वरितमेवदीयतामित्यब्रवीद्ब्राह्मणस्त्वरितमेव स ब्राह्मणस्यैव मुक्त्वादक्षिणं पार्श्वमददत्
% Check verse!
अथान्योष्यश्वार्थी ब्राह्मण आगच्छत् ॥ तथैव चैनमुक्त्वावामपार्ष्णिमभ्यदादथ प्रायात्पुनरपि चान्योप्यश्वार्थी ब्राह्मणआगच्छत् त्वरितोथ तस्मै अपनह्य वामं धुर्यमददत्
% Check verse!
अथ प्रायात्पुनरन्य आगच्छदश्वार्थी ब्राह्मणस्तमब्रवीदतियातोदास्यामि त्वरितमेव मे दीयतामित्यब्रवीद्ब्राह्मणस्तस्मै दत्त्वाऽश्वंरथधुरं गृह्णता व्याहृतं ब्राह्मणानां सांप्रतं नास्ति किंचिदिति
% Check verse!
य एष ददाति चासूयति च तेन व्याहृतेन तथाऽवतरेत् ॥ अथ द्वाभ्यांयातव्यमिति कोऽवतरेत्
% Check verse!
वसुमना अवतरेदित्यब्रवीदृषिः
% Check verse!
किंकारणमित्यपृच्छदथाचष्ट नारदः ॥ अहं परिभ्रमन्वसुमनसोगृहमुपस्थितः
% Check verse!
स्वस्तिवचनमासीत्पुष्परथस्य प्रयोजनेन तमहमन्वगच्छंस्वस्तिवाचितेषु ब्राह्मणेषु रथो ब्राह्मणानां दर्शितः
% Check verse!
तमहं रथं प्राशंसमथ राजाऽब्रवीद्भगवतारथः प्रशस्तः ॥ एष भगवतोभगवतोरथ इति
% Check verse!
अथ कदाचित्पुनरप्यहमुपस्थितः पुनरेव च रथप्रयोजनमासीत् ॥सम्यगयमेष भगवत इत्येवं राजाऽब्रवीदिति पुनरेव च तृतीयं स्वस्तिवाचनंसमभावयमथ राजा ब्राह्मणानां दर्शयन्मामभिप्रेक्ष्याब्रवीत् ॥ अथोभगवता पुष्परथस्य स्वस्तिवाचनानि सुष्ठु संभावितानि एतेनद्रोहवचनेनावतरेत्
% Check verse!
अथैकेन यातव्यं स्यात्कोऽवतरेत्पुनर्नारद आह ॥शिबिर्यायादहमवतरेयं अत्र किं कारणमित्यब्रवीत् ॥ असावहं शिविना समोनास्मि यतो] ब्राह्मणः कश्चिदेनमब्रवीत्
% Check verse!
शिबे अन्नार्थ्यस्मीति तमब्रवीच्छिबिः किं क्रियतामाज्ञापयतुभवानिति
% Check verse!
अथैनं ब्राह्मणोऽब्रवीत् य एष ते पुत्रो बृहद्गर्भो नाम एषप्रमातव्य इति तमेनं संस्कुरु अन्नं चोपपादय ततोऽहं प्रतीक्ष्य इति ॥ततः पुत्रं प्रमाथ्य संस्कृत्य विधिना साधयित्वा पात्र्यामर्पयित्वाशिरसा प्रतिगृह्य ब्राह्मणममृगयत्
% Check verse!
अथास्य मृगयमाणस्य कश्चिदाचष्ट एष ते ब्राह्मणो नगरं प्रविश्यदहति ते गृहं कोशागारमायुधागारं स्त्र्यगारमश्वशालां हस्तिशालां चक्रुद्ध इति
% Check verse!
अथ शिबिस्तथैवाविकृतमुखवर्णो नगरं प्रविश्य ब्राह्मणंतमब्रवीत्सिद्धं भगवन्नन्नमिति ब्राह्मणो न किंचिद्व्याजहारविस्मयादधोमुखश्चासीत्
% Check verse!
ततः प्रासादयद्ब्राह्मणं भगवन्भुज्यतामिति ॥मुहूर्तादुद्वीक्ष्य शिबिमब्रवीत्
% Check verse!
त्वमेवैतदशानेति तत्राह तथेति शिबिस्तथैवाविमना महित्वाकपालमभ्युद्धार्य भोक्तुमैच्छत्
% Check verse!
अथास्य ब्राह्मणो हस्तमगृह्णात् ॥ अब्रवीच्चैनं जितक्रोधोसि नते किंचिदपरित्याज्यं ब्राह्मणार्थे ब्राह्मणोऽपि तं महाभागं सभाजयत्
% Check verse!
स ह्युद्वीक्षमाणः पुत्रमपश्यदग्रे तिष्ठन्तं देवकुमारमिवपुण्यगन्धान्वितमलंकृतं सर्वं च तमर्थं विधाय ब्राह्मणोऽन्तरधीयत ॥स ह्युद्वीक्षमाणः पुत्रमपश्यदग्रे तिष्ठन्तं देवकुमारमिवपुण्यगन्धान्वितमलंकृतं सर्वंच तमर्थं विधाय ब्राह्मणोऽन्तरधीयत
\twolineshloka
{तस्य राजर्षेर्विधाता तेनैव वेषेण परीक्षार्थमागतइतितस्मिन्नन्तर्हिते अमात्या राजानमूचुः ॥ किं प्रेप्सुना भवताइदमेवं जानता कृतमिति ॥शिबिरुवाच}
{}


\twolineshloka
{नैवाहमेतद्यशसे ददानिन चार्यहेतोर्न च भोगतृष्णया}
{पापैरनासेवित एष मार्गइत्येवमेतत्सकलं करोमि}


\twolineshloka
{सद्भिः सदाऽध्यासितं तु प्रशस्तंतस्मात्प्रशस्तं श्रयते मतिर्मे}
{एतन्महाभाग्यवरं शिबेस्तुतस्मादहं वेद यथावदेतत्}


\chapter{अध्यायः २०२}
\twolineshloka
{वैशंपायन उवाच}
{}


% Check verse!
मार्कण्डेयमृषिं पाण्डवाः पर्यपृच्छन्नस्ति कश्चिद्भवतश्चिरजाततरइति
% Check verse!
स तानुवाचास्ति खलु राजर्षिरिन्द्रद्युम्नो नामक्षीणपुण्यस्त्रिदिवात्प्रच्युतः कीर्तिर्मे व्युच्छिन्नेति समामुपातिष्ठदथ प्रत्यभिजानाति मां भवानिति
% Check verse!
तमहमब्रवं नाभिजानीमो भवन्तमिति रकार्यचेष्टाकुलत्वानन वयंरासायनिका ग्रामैकरात्रवासिनो नप्रत्यभिजानीमोऽप्यात्मनोऽर्थानामनुष्टानं न शरीरोपतापेनात्मनःसमारभामोऽर्थानामनुष्ठानम्
\twolineshloka
{`एवमुक्तो राजर्षिरिन्द्रद्युम्नः पुनर्मामब्रवीत् ॥ अथास्तिकश्चित्त्वत्तश्चिरजाततर इति}
{तं पुनः प्रत्यब्रवम् ॥' अस्ति खलुहिमवति प्रावारकर्णो नामोलूकः प्रतिवासति ॥ स मत्तश्चिरजातो भवन्तंयदि जानीयाद्विप्रकृष्टे चाध्वनि हिमवांस्त त्रासौ प्रतिवसतीति}


% Check verse!
तत स मामश्वो भूत्वा तत्रावहद्यत्र बभूवो लूकः ॥ अथैनं सराजर्षिः पर्यपृच्छत्प्रतिजानाति मां भवानिति
% Check verse!
स मुहूर्तमिव ध्यात्वाऽब्रवीदेनं नाभिजानामि भवन्तमिति स एवमुक्तइन्द्रद्युम्नः पुनस्तमुलङ मब्रवीद्राजर्षिः
% Check verse!
अथास्ति कश्चिद्भवतः सकाशाच्चिरजात इति सएवमुक्तोऽब्रवीदस्तिखल्विन्द्रद्युम्नं नाम सरस्त स्मिन्नालिजङ्घो नामबकः प्तिवसति सोस्मत्तश्चिरजाततरस्तं पृच्छेति तत इन्द्रद्युम्नो मांचोलूकमादाय तत्सरोऽगच्छद्यत्रासौ नालिजङ्घो नाम बको बभूव
% Check verse!
सोस्माभिः पृष्टो भवानिममिनद्रद्युम्नं राजानमभिजानातीति स एवंमुहूर्तं ध्यात्वाऽब्रवीन्नाभिजानाम्यहमिन्द्रद्युम्नं राजानमिति ॥ततः सोस्माभिः पृष्टः कश्चिद्भवतोऽन्यश्चिरजाततरोस्तीति सनोऽब्रवीदस्ति खल्वस्मिन्नेव सरस्यकूपारो नाम रकच्छपः प्रतिवसति ॥ समत्तश्चिरजाततरः स यदि कथंचिदभिजानीयादिमं राजानं तमकूपारंपृच्छध्वमिति
% Check verse!
ततः स बकस्तमकूपारं कच्छपं विज्ञापयामास अस्माकमभिप्रेतं भवन्तंकंचिदर्थमभिप्रष्टं साध्वागम्यतां तावदिति तच्छ्रुत्वाकच्छपस्तस्मात्सरस उत्थायाभ्यागच्छद्यत्र तिष्ठामो वयं तस्य सरसस्तीरेआगतं चैनं वयमपृच्छाम भवानिन्द्रद्युम्नं राजानमभिजानातीति
% Check verse!
स मूहूर्तं ध्यात्वा बाष्पसंपूर्णनयन उद्विग्नहृदयो वेपमानोविसंज्ञकल्पः प्राञ्जलिरब्रवीत् किमहमेनं न प्रत्यभिज्ञास्यामीह ह्यनेनसहस्रकृत्वः पूर्वमग्निचितिषूपहिताः
% Check verse!
सरश्चेदमस्य दक्षिणोदकदत्ताभिर्गोभिरतिक्रममाणाभिः कृतम् ॥अत्रचाहं प्रतिवसामीति
\twolineshloka
{अथैतत्सकलं कच्छपेनोदाहृतं श्रुत्वा तदनन्तरं देवलोकाद्देवरथःप्रादुरासीद्वाचश्चाश्रूयन्तेन्द्रद्युम्नं प्रति प्रस्तुतस्ते स्वर्गोयथोचितं स्थानं प्रतिपद्यस्व कीर्तिमानस्यव्यग्रो याहीति ॥ 3-20213xभवन्ति चात्र श्लोकाः}
{}


\twolineshloka
{दिवं स्पृशति भूमिं च शब्दः पुण्यस्य कर्मणः}
{यावत्स शब्दो भवतितावत्स्वर्गे महीयते}


\twolineshloka
{अकीर्तिः कीर्त्यते लोके यस्य भूतस्य कस्यचित्}
{सपतत्यधमाँल्लोकान्यावच्छब्दः प्रकीर्त्यते}


\twolineshloka
{तस्मात्कल्याणवृत्तः स्याद्दाता तावन्नरो भुवि}
{विहायवृत्तंपापिष्ठं धर्ममेव समाश्रयेत्}


% Check verse!
इत्येतच्छ्रुत्वा स राजाऽब्रवीत्तिष्ठ तावद्यावदिमौ वृद्धौयथास्थानं प्रतिपादयामीति
% Check verse!
स मां प्रावारकर्णं चोलूकं यथोचिते स्थाने प्रतिपाद्य तेनदेवयानेन संसिद्धं यथोचितं स्थानं प्रतिपेदे तन्मयाऽनुभूतंचिरजीविनेदृशमिति पाण्डवानुवाच मार्कण्डेयः
% Check verse!
तमृषिं पाण्डवाश्चोचुः साधु शोभनं भवता कृतंराजानमिन्द्रद्युम्नंस्वर्गलोकाच्च्युतं स्वे स्थाने प्रतिपादयतेत्यथैतानब्रवीदसौ ननुदेवकीपुत्रेणापि कृष्णेन नरके मज्जमानोराजर्षिर्नृगस्तस्मात्कृच्छ्रात्पुनः समुद्धृत्यस्वर्गं प्रापित इति
\chapter{अध्यायः २०३}
\twolineshloka
{वैशंपायन उवाच}
{}


\twolineshloka
{श्रुत्वा स राजा राजर्षिरिन्द्रद्युम्नस्य तत्तदा}
{मार्कण्डेयान्महाभागात्स्वर्गस्य प्रतिपादनम्}


\threelineshloka
{युधिष्ठिरो महाराज पुनः पप्रच्छ तं मुनिम्}
{कीदृशीषु ह्यवस्थासु दत्त्वा दानं महामुने}
{इन्द्रलोकं त्वनुभवेत्पुरुषस्तद्ब्रवीहि मे}


\threelineshloka
{गार्हस्थ्येऽप्यथवा बाल्ये यौवने स्थाविरेऽपि वा}
{यथा फलं समश्नाति तथा त्वं कथयस्व मे ॥मार्कण्डेय उवाच}
{}


\twolineshloka
{वृथा जन्मानि चत्वारि वृथा दानानि षोडश}
{वृथा जन्म ह्यपुत्रस् ये च धर्मबहिष्कृताः}


\twolineshloka
{परपाकं च येऽश्नन्ति आत्मार्थं च पचेत्तु यः}
{पर्यश्नन्ति वृथा यत्र तदसत्यं प्रकीर्त्यते}


\twolineshloka
{आरूढपतिते दत्तमन्यायोपहृतं च यत्}
{व्यर्थं तु पतिते दानं ब्राह्मणे तस्करे तथा}


\twolineshloka
{गुरौ चानृतिके पापे कृतघ्ने ग्रामयाजके}
{वेदविक्रयिणे दत्तं तथा वृषलयाजके}


\twolineshloka
{ब्रह्मबन्धुषु यद्दत्तं यद्दत्तं वृषलीपतौ}
{स्त्रीजितेषु च यद्दत्तं व्यालग्राहे तथैव च}


\twolineshloka
{परिचारकेषु यद्दत्तं वृथा दानानि षोडश}
{तमोवृतस्तु यो दद्याद्भयात्क्रोधात्तथैव च}


\twolineshloka
{भुङ्क्ते च दानं तत्सर्वं गर्भस्थस्तु नरः सदा}
{ददद्दानं द्विजातिभ्यो वृद्धभावेन मानवः}


\threelineshloka
{तस्मात्सर्वास्ववस्थासु सर्वदानानि पार्थिव}
{दातव्यानि द्विजातिभ्यः स्वर्गमार्गजिगीषया ॥युधिष्ठिर उवाच}
{}


\threelineshloka
{चातुर्वर्ण्यस्य सर्वस्य वर्तमानाः प्रतिग्रहे}
{केन विप्रा विशेषेण तारयन्ति तरन्ति च ॥मार्कण्डेय उवाच}
{}


\twolineshloka
{जपैर्मन्त्रैश्च होमैश्च स्वाध्यायाध्यायनेन च}
{नावं वेदमयीं कृत्वातारयन्ति तरन्ति च}


\twolineshloka
{ब्राह्मणांस्तोषयेद्यस्तु तुष्यन्ते तस्य देवताः}
{वचनाच्चापि विप्राणां स्वर्गलोकमवाप्नुयात्}


\twolineshloka
{अनन्तं पुण्यलोकं तु गन्ताऽसौ तु न संशयः}
{श्लेष्मादिभिर्व्याप्ततनुर्म्रियमाणोऽविचेतनः}


\twolineshloka
{ब्राह्णा एव संपूज्याः पुण्यं स्वर्गमभीप्सता}
{श्राद्धकाले तु यत्नेन भोक्तव्या ह्यजुगुप्सिताः}


\twolineshloka
{दुर्बलः कुनस्वी कुष्ठी मायावी कुण्डगोलकौ}
{वर्जनीयाः प्रयत्नेन काण्डपृष्ठाश्च देहिनः}


% Check verse!
जुगुप्सितं हि यच्छ्राद्धं दहत्यग्निरिवेन्धनम्
\twolineshloka
{ये ये श्राद्धे न पूज्यन्ते मूकान्धबधिरादयः}
{तेऽपिसर्वे नियोक्तव्या मिश्रिता वेदपारगैः}


\twolineshloka
{प्रतिग्रहश्च वै देयः शृणु यस् युधिष्ठिर}
{प्रदातारं तथाऽऽत्मानं यस्तारयति शक्तिमान्}


\twolineshloka
{तस्मिन्देयं द्विजे दानं सर्वागमविजानता}
{प्रदातारं तथाऽऽत्मानं तारयेद्यः स शक्तिमान्}


\twolineshloka
{न तथा हविषो होमैर्न पुष्पैर्नानुलेपनैः}
{अग्नयः पार्थ तुष्यन्ति यथा ह्यतिथिभोजने}


% Check verse!
तस्मात्त्वंसर्वयत्नेन यतस्वातिथिभोजने
\twolineshloka
{पादोदकंपादघृतंदीपमन्नं प्रतिश्रयम्}
{प्रयच्छन्ति तु ये राजन्नोपसर्पन्ति ते यमम्}


\threelineshloka
{देवमाल्यापनयनं द्विजोच्छिष्टावमार्जनम्}
{आकल्पपरिचर्या च गात्रसंवाहनानि च}
{अत्रैकैकं नृपश्रेष्ठ गोदानाद्व्यतिरिच्यते}


\twolineshloka
{कपिलायाः प्रदानात्तु मुच्यते नात्र संशयः}
{तस्मादलंकृतां दद्यात्कपिलां तु द्विजातये}


\twolineshloka
{श्रोत्रियाय दरिद्राय गृहस्थायाग्निहोत्रिणे}
{पुत्रदाराभिभूताय तथा ह्यनुपकारिणे}


\twolineshloka
{एवंविधेषु दातव्या न समृद्धेषु भारत}
{को गुणो भरतश्रेष्ठ समृद्धेष्वभिवर्जितम्}


\threelineshloka
{एकस्यैका प्रदातव्या न बहूनां कदाचन}
{सा गौर्विक्रयमापन्ना हन्यात्रिपुरुषं कुलम्}
{न तारयति दातारं ब्राह्मणं नैव नैव तु}


\twolineshloka
{ब्राह्मणस्य विशुद्धस्य सुवर्णं यः प्रयच्छति}
{सुवर्णानां शतं तेन दत्तं भवति शाश्वतम्}


\twolineshloka
{अनड्वाहं तु यो दद्याद्बलवन्तं धुरंधरम्}
{स निस्तरति दुर्गाणि स्वर्गलोकं च गच्छति}


\twolineshloka
{वसुंधरां तु यो दद्याद्द्विजाय विदुषात्मने}
{दातारं ह्यनुगच्छन्ति सर्वे कामाभिवाञ्छिताः}


\twolineshloka
{पृच्छन्ति चान्नदातारं वदन्ति पुरुषा भुवि}
{अध्वनि क्षीणगात्राश्च पांसुना चावकुण्ठिताः}


\twolineshloka
{तेषामेव श्रमार्तानां यो ह्यन्नं कथयेद्बुधः}
{अन्नदातृसमः सोपि कीर्त्यते नात्र संशयः}


\twolineshloka
{तस्मात्त्वं सर्वदानानि हित्वाऽन्नं संप्रयच्छ ह}
{न हीदृशं पुण्यफलं विचित्रामिह विद्यते}


\twolineshloka
{यथाशक्ति च यो दद्यादन्नं विप्रेषु संस्कृतम्}
{स तेन कर्मणाऽऽप्नोति प्रजापतिसलोकताम्}


\twolineshloka
{अन्नमेव विशिष्टं हि तस्मात्परतरं न च}
{अन्नं प्रजापतिश्चोक्तः स च संवत्सरो मतः}


\twolineshloka
{संवत्सरस्तु यज्ञोऽसौ सर्वं यज्ञे प्रतिष्ठितम्}
{तस्मात्सर्वाणि भूतानि स्थावराणि चराणि च}


% Check verse!
तस्मादन्नं विशिष्टं हि सर्वेभ्यइति विश्रुतम्
\twolineshloka
{येषां तटाकानि महोदकानिवाप्यश्चकूपाश्चप्रतिश्रयाश्च}
{अन्नस्य दानं मधुरा च वाणीयमस्य ते निर्वचना भवन्ति}


\twolineshloka
{धान्यं श्रमेणार्जितवित्तसंचितंविप्रे सुशीले प्रतियच्छते यः}
{वसुंधरा तस्य भवेत्सुतुष्टाधारां वसूनां प्रतिमुञ्चतीव}


\threelineshloka
{अन्नदाः प्रथमं यान्ति सत्यवाक्यदनन्तरम्}
{अयाचितप्रदाता च समं यान्ति त्रयो जनाः ॥वैशंपायन उवाच}
{}


\twolineshloka
{कौतूहलसमुत्पन्नः पर्यपृच्छद्युधिष्ठिरः}
{मार्कण्डेयं महात्मानं पुनरेव सहानुजः}


\fourlineindentedshloka
{यमलोकस्य चाध्वानमन्तरं मानुषस्य च}
{कीदृशं किंप्रमाणं वा कथं वा तन्महामुने}
{तरन्ति पुरुषाश्चैव केनोपायेन शंस मे ॥मार्कण्डेय उवाच}
{}


\twolineshloka
{सर्वगुह्यतनं प्रश्नं पवित्रमृषिसंस्तुतम्}
{कथयिष्यामि ते राजन्धर्मं धर्मभृतांवर}


\twolineshloka
{षडशीतिसहस्राणि योजनानां नराधिप}
{यमलोकस्य चाध्वानमन्तरं मानुषस्य च}


\twolineshloka
{आकाशं तदपानीयं घोरं कान्तारदर्शनम्}
{न तत्र वृक्षच्छाया वा पानीनं केतनानि च}


\twolineshloka
{विश्रमेद्यत्र वै श्रान्तः पुरुषोऽध्वनि कर्शितः}
{नीयते यमदूतैस्तु यमस्वाज्ञाकरैर्बलात्}


\twolineshloka
{नराः स्त्रियस्तथैवान्ये पृथिव्यां जीवसंज्ञिताः}
{ब्रह्मणेभ्यः प्रदानानि नानारूपाणि पार्थिव}


\twolineshloka
{हयादीनां प्रकृष्टानि तेऽध्वानं यान्ति वै नराः}
{सन्निवार्यातपं यानति च्छत्रेणैव हि च्छत्रदाः}


\twolineshloka
{तृप्ताश्चैवान्नदातारो ह्यतृप्ताश्चाप्यनन्नदाः}
{वस्त्रिणो वस्त्रदा यान्ति अवस्त्रा यान्त्यवस्त्रदाः}


\twolineshloka
{हिरण्यदाः सुखं यान्ति पुरुषास्त्वभ्यलंकृताः}
{भूमिदास्तुसुखं यान्ति सर्वैः कामैः सुतर्पिताः}


\twolineshloka
{यान्ति चैवापरिक्लिष्टा नरा सस्यप्रदावकाः}
{नराः सुखतरं यान्ति विमानेषु गृहप्रदाः}


\twolineshloka
{पानीयदा ह्यतृषिताः प्रहृष्टमनसो नराः}
{पन्थानं द्योतयन्तश्च यान्ति दीपप्रदाः सुखम्}


\twolineshloka
{गोप्रदास्तु सुखं यान्ति निर्मुक्ताः सर्वपातकैः}
{विमानैर्हंससंयुक्तैर्यान्ति मासोपवासिनः}


\twolineshloka
{तथा बर्हिप्रयुक्तैश्च षष्ठरात्रोपवासिनः}
{त्रिरात्रं क्षपते यस्तु एकभक्तेन पाण्डव}


\twolineshloka
{अन्तरा चैव नाश्नाति तस्य लोका ह्यनामयाः}
{पानीयस्य गुणा दिव्याः प्रेतलोकसुखावहाः}


\twolineshloka
{तत्र पुष्पोदका नाम नदी तेषां विधीयते}
{शीतलं सलिलं तत्रपिबन्ति ह्यमृतोपमम्}


\twolineshloka
{ये चदृष्कृतकर्माणः पूयं तेषां विधीयते}
{एवं नदी महाराज सर्वकामप्रदा हि सा}


\twolineshloka
{तस्मात्त्वमपि राजेन्द्र पूजयैनान्यथाविधि}
{अध्वनिं क्षीणगात्रश्च पथि पांशुसमन्वितः}


\twolineshloka
{पृच्छते ह्यन्नदातारं गृहमायाति चाशया}
{तं पूजयाथ यत्नेन सोऽतिथिर्ब्राह्यणश्च सः}


\twolineshloka
{तं यान्तमनुगच्छन्ति देवाः सर्वे सवासवाः}
{तस्मिन्संपूजिते प्रीता निराशा यान्त्यपूजिते}


\threelineshloka
{तस्मात्त्वमपि राजेन्द्र पूजयैनं यथाविधि}
{एतत्ते शतशः प्रोक्तं किं भूयः श्रोतुमिच्छसि ॥युधिष्ठिर उवाच}
{}


\threelineshloka
{पुनः पुनरहं श्रोतुं कथां धर्मसमाश्रयाम्}
{पुण्यामिच्छामि धर्मज्ञ कथ्यमानां त्वया विभो ॥मार्कण्डेय उवाच}
{}


\twolineshloka
{धर्मान्तरं प्रति कथां कथ्यमानां मया नृप}
{सर्वपापहरां नित्यं शृणुष्वावहितो मम}


\twolineshloka
{कपिलायां तु दत्तायां यत्फलंज्येष्ठपुष्करे}
{तत्फलंभरतश्रेष्ठ विप्राणां पादधावने}


\twolineshloka
{द्विजपादोदकक्लिन्ना यावत्तिष्ठति मेदिनी}
{तावत्पुष्करपर्णेन पिबन्ति पितरो जलम्}


\twolineshloka
{स्वागतेनाग्नयस्तृप्ता आसनेन शतक्रतुः}
{पितरः पादशौचेन अन्नाद्येन प्रजापतिः}


\twolineshloka
{यावद्वत्सस्य पादौ द्वौ शिरश्चैव प्रदृश्यते}
{तस्मिन्काले प्रदातव्या प्रयतेनान्तरात्मना}


\twolineshloka
{अन्तरिक्षगतो वत्सो यावद्योन्यां प्रदृश्यते}
{तावद्गौः पृथिवी ज्ञेया यावद्गर्भं न मुञ्चति}


\twolineshloka
{यावन्ति तस्यां रोमाणि वत्सस्य च युधिष्ठिर}
{तावद्युगसहस्राणि स्वर्लोके महीयते}


\twolineshloka
{सुवर्णनासां यः कृत्वा सखुरां कृष्णधेनुकाम्}
{तिलैः प्रच्छादितां दद्यात्सर्वरत्नैरलंकृताम्}


\twolineshloka
{प्रतिग्रहं गृहीत्वा यः पुनर्ददति साधवे}
{फलानां फलमश्नाति तदा दत्त्वा च भारत}


\twolineshloka
{ससमुद्रगुहा तेन सशैलवनकानना}
{चतुरन्ता भवेद्दत्ता पृथिवी नात्र संशयः}


\twolineshloka
{अन्तर्जानुशयो यस्तु भुञ्जते सक्तभाजनः}
{यो द्विजः शब्दरहितं संयन्तुस्तारणाय वै}


\twolineshloka
{ये पानीयानि ददति तथाऽन्ये ये द्विजातयः}
{जपन्ति संहितां सम्यक्ते नित्यं तारणक्षमाः}


\twolineshloka
{हव्यंकव्यं च यत्किंचित्सर्वं तच्छ्रोत्रियोऽर्हति}
{दत्तं हि श्रोत्रिये साधौज्यलितेऽग्नौ यथा हुतम्}


\twolineshloka
{मन्युप्रहरणा विप्रा न विप्राः शस्त्रयोधिनः}
{निहन्युर्मन्युना विप्रा वज्रपाणिरिवासुरान्}


\twolineshloka
{धर्माश्रितेयं तु कथा कथिता हि तवानघ}
{यां श्रुत्वा मुनयः प्रीता नैमिषारण्यवासिनः}


\threelineshloka
{वीतशोकभयक्रोधा विपाप्मानस्तथैव च}
{श्रुत्वेमां तु कथां राजन्भवन्तीह तु मानवाः ॥युधिष्ठिर उवाच}
{}


\threelineshloka
{किं तच्छौचं भवेद्येन विप्रः शुद्धः सदा भवेत्}
{तदिच्छामि महाप्राज्ञ श्रोतुं धर्मभृतांवर ॥मार्कण्डेय उवाच}
{}


\twolineshloka
{वाक्शौचं क्रमशौचं च यच्च शौचं जलात्मकम्}
{त्रिभिः शौचैरुपेतो यः स स्वर्गी नात्र संशयः}


\twolineshloka
{सायं प्रातश्च संध्यां यो ब्राह्मणोऽभ्युपसेवते}
{प्रजपन्पावनीं देवीं गायत्रीं वेदमातरम्}


\twolineshloka
{स तया पावितो देव्या ब्राह्मणो नष्टकिल्बिषः}
{न सीदेत्प्रतिगृह्णानो महीमपि ससागराम्}


\twolineshloka
{ये चास् दारुणा केचिद्ग्रहाः सूर्यादयो दिवि}
{ते चास्य सौम्या जायन्ते शिवाः शिवतराः सदा}


\twolineshloka
{सर्वेनानुगतं चैनं दारुणाः पिशिताशनाः}
{घोररूपा महाकाया धर्षयन्ति द्विजोत्तमम्}


\twolineshloka
{नाध्यापनाद्याजनाद्वा अन्यायाद्वा प्रतिग्रहात्}
{दोषो भवति विप्राणां ज्वलिताग्निससा द्विजाः}


\twolineshloka
{दुर्वेदा वा सुवेदा वा प्राकृताः संस्कृतास्तथा}
{ब्राह्मणा नावमन्तव्या भस्मच्छन्ना इवाग्नयः}


\twolineshloka
{यथा श्मशाने दीप्तौजाः पावको नैव दुष्यति}
{एवं विद्वानविद्वान्वा ब्राह्मणो दैवतं महत्}


\twolineshloka
{प्राकरैश्च पुरद्वारैः प्रासादैश्च पृथग्विधैः}
{नगराणि न शोभन्ते हीनानि ब्राह्मणोत्तमैः}


\twolineshloka
{वेदाढ्ञा वृत्संपन्ना ज्ञानवन्तस्तपस्विनः}
{यत्रतिष्न्ति वै विप्रास्तन्नाम नगरं नृप}


\twolineshloka
{व्रजे वाऽप्यथवाऽरण्ये यत्रसन्ति बहुश्रुताः}
{तत्तन्नगरमित्याहुः पार्थ तीर्थं च तद्भवेत्}


\twolineshloka
{रक्षितारं च राजानं ब्राह्मणं च तपस्विनम्}
{अभिगम्याभिपूज्याथ सद्यः पापात्प्रमुच्यते}


\twolineshloka
{पुण्यतीर्थाभिषेकं च पवित्राणां च कीर्तनम्}
{सद्भिः संभाषणं चैव प्रशस्तं कीर्त्यते बुधैः}


\twolineshloka
{साधुसङ्गमपूतेन वाक्सुभाषितवारिणा}
{पवित्रीकृतमात्मानं सन्तो मन्न्ति नित्यशः}


\twolineshloka
{त्रिदण्डधारणं मौनं जटाभारोऽथ मुण्डनम्}
{वल्कलाजिनसंवेष्टं व्रतचर्याऽभिषेचनम्}


\twolineshloka
{अग्निहोत्रं वने वासः शरीरपरिशोषणम्}
{सर्वाण्येतानि मिथ्या स्युर्यदि भावो न निर्मलः}


\twolineshloka
{विशुद्धिं चक्षुरादीनां षण्णामिन्द्रियगामिनाम्}
{विकारि तेषां राजेन्द्र सुदुष्करतरं मनः}


\twolineshloka
{ये पापानि न कुर्वन्ति मनोवाक्कर्मबुद्धिभिः}
{ते तपन्ति महात्मानो न शरीरस्य शोषणम्}


\twolineshloka
{न ज्ञातिभ्यो दया यस्य शुक्लदेहोऽविकल्मषः}
{हिंसा सा तपसस्तस्य नानाशित्वं तपः स्मृतम्}


\twolineshloka
{तिष्ठन्गृहे चैव मुनिर्नित्यं शुचिरलंकृतः}
{यावज्जीवं दयावांश् सर्वपापैः प्रमुच्यते}


\twolineshloka
{न हि पापानि कर्माणि शुद्ध्यन्त्यनशनादिभिः}
{सीदत्यनशनादेव मांसशोणितलेपनः}


\twolineshloka
{अज्ञातं कर्म कृत्वा च क्लेशो नान्यत्प्रहीयते}
{नाग्निर्दहति कर्माणि भावशून्यस्य देहिनः}


\twolineshloka
{पुण्यादेव प्रव्रजन्ति शुध्यन्त्यनशनानि च}
{न मूलफलभक्षित्वान्न मौनान्नानिलाशनात्}


\threelineshloka
{शिरसो मुण्डनाद्वाऽपि न स्थानकुटिकासनात्}
{न जटाधारणाद्वाऽपिन तु स्थानकुटिकासनात्}
{}


\twolineshloka
{नित्यं ह्यनशनाद्वाऽपि नाग्निशुश्रूषणादपि}
{न चोदकप्रवेशेन न च क्ष्माशयनादपि}


\twolineshloka
{ज्ञानेन कर्मणा वापि जरामरणमेव च}
{व्याधयश्च प्रहीयन्ते प्राप्यते चोत्तमं पदम्}


\twolineshloka
{बीजानि ह्यग्निदग्धानि न रोहन्ति पुनर्यथा}
{ज्ञानदग्धैस्तधा क्लेशैर्नात्मा संडुज्यते पुनः}


\twolineshloka
{आत्मना विप्रहीणानि काष्ठकुण्ठोपमानि च}
{विनश्यन्ति न संदेहः फेनानीव महार्णवे}


\twolineshloka
{आत्मानं विन्दते येन सर्वभूतगुहाशयम्}
{श्लोकेन यदि वाऽर्धेन क्षीणं तस्य प्रयोजनम्}


\twolineshloka
{द्व्यक्षरादभिसंधाय केचिच्छ््लोकपदाङ्कितैः}
{शतैरन्यैः सहस्रैश्च प्रत्ययो मोक्षलक्षणम्}


\twolineshloka
{नायं लोकोस्ति न परो न सुखं संशयात्मनः}
{ऊटुर्ज्ञानविदो वृद्धाः प्रत्ययो मोक्षलक्षणम्}


\twolineshloka
{विदितार्थस्तु वेदानां परिवेद प्रयोजनम्}
{उद्विजेत्स तु देवेभ्यो दावाग्नेरिव मानवः}


\threelineshloka
{शुष्कं तर्कं परित्यज्यआश्रयस्व श्रुतिं स्मृतिम्}
{एकाराभिसंबद्धं तत्त्वं हेतुभिरिच्छसि}
{बुद्धिर्न तस्य सिद्ध्येत साधनस्य विपर्ययात्}


\twolineshloka
{वेद्यं पूर्वं वेदितव्यं प्रयत्ना-त्तद्धीर्वेदस्तस्य वेदः शरीरम्}
{वेदस्तत्त्वं तत्समासोपलब्धिःक्लीबस्त्वात्मा न स वेद्यं न वेदः}


\twolineshloka
{वेदोक्तमायुर्देवानामाशिषश्चैव कर्मणाम्}
{फलत्यनुयुगं लोके प्रभावश्च शरीरिणाम्}


\twolineshloka
{इन्द्रियाणां प्रसादेन तदेतत्परिवर्जयेत्}
{तस्मादनशनं दिव्यं निरुद्धेन्द्रियगोचरम्}


\threelineshloka
{तपसा स्वर्गगमनं भोगो दानेन जायते}
{ज्ञानेन मोक्षो विज्ञेयस्तीर्थस्नानादघक्षयः ॥वैशंपायन उवाच}
{}


\threelineshloka
{एवमुक्तस्तु राजेन्द्र प्रत्युवाच महायशाः}
{भगवञ्श्रोतुमिच्छामि प्रदानविधिमुत्तमम् ॥मार्कण्डेय उवाच}
{}


\twolineshloka
{यत्तत्पृच्छसि राजेन्द्रदानधर्मं युधिष्ठिर}
{इष्टं चेदं सदा मह्यं राजन्गौरवतस्तथा}


\threelineshloka
{शृणु दानरहस्यानि श्रुतिस्मृत्युदितानि च}
{छायायां करिणः श्राद्धं तत्कर्म परिवीजितम्}
{दशकल्पायुतानीह न क्षीयेत युधिष्ठिर}


\twolineshloka
{जीवनाय समाक्लिन्नं वसु दत्त्वा महीपते}
{विप्रं तु वासयेद्यस्तु सर्वयज्ञैः स इष्टवान्}


\threelineshloka
{प्रतिस्रोतश्चित्रवाहाः पर्जन्योऽन्नानुसंचरन्}
{महाधुरि यथा नावा महापापैः प्रमुच्यते}
{विषुवे विप्रदत्तानि दधिमध्वक्षयाणि च}


\twolineshloka
{पर्वसु द्विगुणं दानमृतौ दशगुणं भवेत्}
{3-203-b124 `अब्दे दशगुणंप्रोक्तमनन्तं विषुवे भवेत्'}


\twolineshloka
{अयने विषुवे चैवषडशीतिमुखेषु च}
{चन्द्रसूर्योपरागे च दत्तमक्षयमुच्यते}


\twolineshloka
{ऋतुषु दशगुणं वदन्ति दत्तंशतगुणमृत्वयनादिषु ध्रुवम्}
{भवति सहस्रगुणं दिनस्य राहो-र्विषुवति चाक्षयमश्नुते फलम्}


\twolineshloka
{नाभूमिदो भूमिमश्नाति राज-न्नायानदो यानमारुह्य याति}
{यान्यान्कामान्ब्राह्मणेभ्यो ददातितांस्तान्कामाञ्जायमानः स भुङ्क्ते}


\twolineshloka
{अग्नेरपत्यं प्रथमं सुवर्णंभूर्वैष्णवी सूर्यसुताश्च गावः}
{लोकाश्त्रयस्तेन भवन्ति दत्तायः काञ्चनं गाश्च महीं च दद्यात्}


\twolineshloka
{परंहि दानान्न बभूव शाश्वतंभव्यं त्रिलोके भवते कुतः पुनः}
{तस्मात्प्रधानं परमं हि दानंवदन्ति लोकेषु विशिष्टबुद्धयः}


\chapter{अध्यायः २०४}
\twolineshloka
{वैशंपायन उवाच}
{}


\twolineshloka
{[श्रुत्वा तु राजा राजर्षेरिन्द्रद्युम्नस्य तत्तथा}
{मार्कण्डेयान्महाभागात्स्वर्गस्य प्रतिपादनम् ॥]}


\twolineshloka
{युधिष्ठिरो महाराज पप्रच्छ भरतर्षभ}
{मार्कण्डेयं तपोवृद्धं दीर्घायुषमकल्मषम्}


\twolineshloka
{विदितास्तव धर्मज्ञ देवदानवरक्षसाम्}
{राजवंशाश्च विविधा ऋषिवंशाश्च शाश्वताः}


\twolineshloka
{न तेऽस्त्यविदितं किंचिदस्मिँल्लोके द्विजोत्तम}
{अथ वेत्सि मुने वंशान्मनुष्योरगरक्षसाम्}


\twolineshloka
{देवगन्धर्वयक्षाणां किन्नराप्सरसां तथा}
{इदमिच्छाम्यहं श्रोतुं तत्त्वे द्विजसत्तम}


\twolineshloka
{कुवलाश्व इति ख्यात इक्ष्वाकुकुलसंभवः}
{कथं नाम विपर्यासाद्धुन्धुमारत्वमागतः}


\threelineshloka
{एतदिच्छामि तत्त्वेन ज्ञातुं भार्गवसत्तम}
{विपर्यस्तं यथा नाम कुवलाश्वस् धीमतः ॥[वैशंपायन उवाच}
{}


\threelineshloka
{युधिष्ठिरेणैवमुक्तो मार्कण्डेयो महामुनिः}
{धौन्धुमारमुपाख्यानं कथयामास भारत ॥]मार्कण्डेय उवाच}
{}


\twolineshloka
{हन्त ते कथयिष्यामि शृणु राजन्युधिष्ठिर}
{धर्मिष्ठमिदमाख्यानं धुन्धुमारस्य तच्छृणु}


\twolineshloka
{यथा स राजा ऐक्ष्वाकः कुवलाश्वो महीपतिः}
{धुन्धुमारत्वमगमत्तच्छृणुष्व महीपते}


\twolineshloka
{महर्षिर्विश्रुतस्तात उदङ्क इति भारत}
{मरुधन्वसु रम्येषु आश्रमस्तस्य कौरव}


\twolineshloka
{उदङ्कस्तु महाराज तपोतप्यत्सुदुश्चरम्}
{आरिराधयिषुर्विष्णुं बहून्वर्षगणान्विभुः}


\threelineshloka
{तस्य प्रीतः स भगवान्साक्षाद्दर्शनमेयिवान्}
{दृष्ट्वा महर्षिस्तद्ब्रह्म तुष्टाव विविधैः स्तवैः ॥उदङ्क उवाच}
{}


\twolineshloka
{त्वया देव प्रजाः सर्वाः ससुरासुरमानवाः}
{स्थावराणि च भूतानि जङ्गमानि तथैव च}


\twolineshloka
{ब्रह्म वेदाश्च वेद्यं च त्वया सृष्टं महाद्युते}
{शिरस्ते गगनं देव नेत्रे शशिदिवाकरौ}


\twolineshloka
{निःश्वासः पवनश्चापि तेजोऽग्निश्च तवाच्युत}
{बाहवस्ते दिशः सर्वाः कुक्षिश्चापि महार्णवः}


\twolineshloka
{ऊरू ते पर्वता देव स्वं नाभिर्मधुसूदन}
{पादौ ते पृथिवी चैव रोमाण्योषधयस्तथा}


\twolineshloka
{इन्द्रसोमाग्निवरुणा देवासुरमहोरगाः}
{प्रह्वास्त्वामुपतिष्ठन्ति स्तुवन्तो विविधैः स्तवैः}


\twolineshloka
{त्वया व्याप्तानि सर्वाणि भूतानि भुवनेश्वर}
{योगिनः सुमहावीर्याः स्तुवन्ति त्वां महर्षयः}


\twolineshloka
{त्वयि तुष्टे जगच्छान्तं त्वयि क्रुद्धे महद्भयम्}
{भयानामपनेतासि त्वमेकः पुरुषोत्तम}


\twolineshloka
{देवानां मानुषाणां च सर्वभूतसुखावहः}
{त्रिभिर्विक्रमणैर्देव त्रयो लोकास्त्वया वृताः}


\twolineshloka
{असुराणां समृद्धानां विनाशश्च त्वया कृतः}
{तव विक्रमणैर्देवा निर्वाणमगमन्परम्}


\threelineshloka
{पराभवं च दैत्येन्द्रास्त्वयि क्रुद्धे महाद्युते}
{त्वं हि कर्ता विकर्ता च भूतानामिह सर्वशः}
{आराधयित्वा त्वां देवाः सुखमेधन्ति नित्यशः}


\threelineshloka
{एवं स्तुतो हृषीकेश उदङ्केन महात्मना}
{उदङ्कमब्रवीद्विष्णुः प्रीतस्तेऽहं वरं वृणु ॥उदङ्क उवाच}
{}


\threelineshloka
{पर्याप्तो मे वरो ह्येष यदहं दृष्टवान्हरिम्}
{पुरुषं शाश्वतं दिव्यं स्रष्टारं जगतः प्रभुम् ॥विष्णुरुवाच}
{}


\twolineshloka
{प्रीतस्तेऽहमलौल्येन भक्त्या तव च सत्तम}
{अवश्यं हि त्वया ब्रह्मन्मत्तो ग्राह्यो वरो द्विज}


\twolineshloka
{एवं स च्छन्द्यमानस्तु वरेण हरिणा तदा}
{उदङ्कः प्राञ्जलिर्वव्रे वरं भरतसत्तम}


\fourlineindentedshloka
{यदि मे भगवन्प्रीतः पुण्डरीकनिभेक्षण}
{धर्मे सत्ये दमे चैव बुद्धिर्भवतु मे सदा}
{अभ्यासश्च भवेद्भक्त्या त्वयि नित्यं ममेश्वर ॥भगवानुवाच}
{}


% Check verse!
सर्वमेतद्धि भविता मत्प्रसादात्तव द्विज
\twolineshloka
{प्रतिभास्यति योगश्च येन युक्तो दिवौकसाम्}
{त्रयाणामपि लोकानां महत्कार्यं करिष्यसि}


\twolineshloka
{उत्सादनार्थं लोकानां धुन्धुर्नाम महासुरः}
{तपस्यति तपो घोरं शृणु यस्तं हनिष्यति}


% Check verse!
[राजा हि वीर्यवांस्तात इक्ष्वाकुरपराजितः]बृहदश्व इति ख्यातो भविष्यति महीपतिः
\threelineshloka
{तस्य पुत्रः शुचिर्दान्तः कुवलाश्व इति श्रतः}
{स योगबलामास्थाय मामकं पार्तिवोत्तमः}
{शासनात्तव विप्रर्षे धुन्धुमारो भविष्यति}


% Check verse!
उदङ्कमेवमुक्त्वा तु विष्णुरन्तरधीयत
\chapter{अध्यायः २०५}
\twolineshloka
{मार्कण्डेय उवाच}
{}


\twolineshloka
{इक्ष्वाकौ संस्तिते राजञ्शशादः पृथिवीमिमाम्}
{प्राप्तः परमधर्मात्मा सोऽयोध्यायां नृपोऽभवत्}


\twolineshloka
{शशादस्य तु दायाद ककुत्स्थो नाम वीर्यवान्}
{अनेनाश्चापि काकुत्स्थः पृथुश्चानेनसः सुतः}


\twolineshloka
{विष्वगश्वः पृथोः पुत्रस्तस्मादार्द्रश्च जज्ञिवान्}
{आर्द्रस्य युवनाश्वस्तु श्रावस्तस्तस्य चात्मजः}


\twolineshloka
{जज्ञे श्रावस्तको राजा श्रावस्ती येन निर्मिता}
{श्रावस्तस्य तु दायादो बृहदश्वो महाबलः}


\twolineshloka
{बृहदश्वस् दायादः कुवलाश्व इति स्मृतः}
{कुवलाश्वस्य पुत्राणां सहस्राण्येकविंसतिः}


\twolineshloka
{सर्वे विद्यासु निष्णाता बलवन्तो दुरासदाः}
{कुवलाश्वश्च पितृतोगुणैरभ्यधिकोऽभवत्}


\twolineshloka
{समये तं तदा राज्ये बृहदश्वोऽभ्यषेचयत्}
{कुवलाश्वं महाराज शूरमुत्तमधार्मिकम्}


\twolineshloka
{पुत्रसंक्रामितश्रीस्तु बृहदश्वो महीपतिः}
{जगाम तपसे धीमांस्तपोवनममित्रहा}


\twolineshloka
{अथ शश्वाव राजर्षिं तमुदङ्को नराधिप}
{वनं संप्रस्थितं राजन्बृहदश्वं द्विजोत्तमः}


\threelineshloka
{तमुदङ्को महातेजाः सर्वास्त्रविदुषांवरम्}
{न्यवारयदमेयात्मा समासाद्य पुरोत्तमे ॥उदङ्क उवाच}
{}


\twolineshloka
{भवता रक्षणं कार्यं तत्तावत्कर्तुमर्हसि}
{निरुद्विग्रा वयं राजंस्त्वत्प्रदाद्वसेमहि}


\twolineshloka
{त्वया हि पृथिवी राजन्रक्ष्यमाणा महात्मना}
{भविष्यति निरुद्विग्ना नारण्यं गन्तुमर्हसि}


\twolineshloka
{पालन हि महान्धर्मः प्रजानामिह दृश्यते}
{न तथा दृश्यतेऽरण्ये माभूत्ते बुद्धिरीदृशी}


\twolineshloka
{ईदृशो न हि राजेन्द्र धर्मः क्वचन दृश्यते}
{प्रजानां पालने यत्नः पुरा राजर्षिभिः कृतः}


\twolineshloka
{रक्षितव्याः प्रजा राज्ञा तास्त्वं रक्षितुमर्हसि}
{निरुद्विग्नस्तपस्तप्तुं न हि शक्नोमि पार्थिव}


\threelineshloka
{ममाश्रमसमीपे वै समेषु मरुधन्वसु}
{समुद्रवालुकापूर्ण उज्जालक इति स्मृतः}
{बहुयोजनविस्तीर्णो बहुयोजनमायतः}


\threelineshloka
{तत्र रौद्रो दानवेन्द्रो महावीर्यपराक्रमः}
{मधुकैटभयोः पुत्रो धुन्धुर्नाम सुदारुणः}
{अन्तर्भूमिगतो राजन्वसत्यमितविक्रमः}


% Check verse!
तं निहत्य महाराज वनं त्वं गन्तुमर्हसि
\twolineshloka
{शेते लोकविनाशाय तप आस्थाय दारुणम्}
{त्रिदशानां विनाशाय लोकानां चापि पार्थिव}


\threelineshloka
{अवध्यो दैवतानां हि दैत्यानामथ रक्षसाम्}
{नागानामथ यक्षाणां गन्धर्वाणां च सर्वशः}
{अवाप्यस वरं राजन्सर्वलोकपितामहात्}


\twolineshloka
{तं विनाशय भद्रं ते मा ते बुद्धिरतोऽन्यथा}
{प्राप्स्यसे महतीं कीर्तिं शाश्वतीमव्ययां ध्रुवाम्}


\twolineshloka
{क्रूरस्य तस्य स्वपतो वालुकान्तर्हितस्य च}
{संवत्सरस्य पर्यन्ते निःश्वासः संप्रवर्तते}


\twolineshloka
{यदा तदा भूश्चलति सशैलवनकानना}
{तस्य निःश्वासवातेन रज उद्धूयते महत्}


\twolineshloka
{आदित्यरथमाश्रित्य सप्ताहं भूमिकम्पनम्}
{सविस्फुलिङ्गं सज्वालं धूममिश्रं सुदारुणम्}


% Check verse!
तेन राजन्न शक्नोमि तस्मिन्स्तातुं स्वआंश्रमे
\twolineshloka
{तं विनाशय राजेन्द्र लोकानां हितकाम्यया}
{लकाः स्वस्था भविष्यन्ति तस्मिन्विनिहतेऽसुरे}


\twolineshloka
{`धुन्धुनामानमत्युग्रं दानवं धोरविग्रहम्}
{समरे धोरतुमुले विनाशय महेषुणा'}


\twolineshloka
{त्वं हि तस्य विनाशाय पर्याप्त इति मे मतिः}
{तेजसा तव तेजश्च विष्णुराप्यायविष्यति}


% Check verse!
विष्णुना च वरो दत्तः पूर्वं मम महीपते
\twolineshloka
{यस्तं महासुरं रौद्रं वधिष्यति महीपतिः}
{तेजस्तद्वैष्णवमिति प्रवेक्ष्यति दुरासदम्}


\twolineshloka
{तत्तेजस्त्वंसमाधाय राजेन्द्र भुवि दुःसहम्}
{तं निषूदय संदिष्टो दैत्यं रौद्रपराक्रमम्}


\twolineshloka
{न हि धुन्धुर्महातेजास्तेजसाऽल्पेन शक्यते}
{निर्दग्धुं पृथिवीपाल स हि वर्षशतैरपि}


\chapter{अध्यायः २०६}
\twolineshloka
{मार्कण्डेय उवाच}
{}


\twolineshloka
{स एवमुक्तो राजर्षिरुदङ्केनापराजितः}
{उदङ्कं कौरवश्रेष्ठ कृताञ्जलिथाब्रवीत्}


\twolineshloka
{न हि मे गमनं ब्रह्मन्मोघमेतद्भविष्यति}
{पुत्रो ममायं भगवन्कुवलाश्व इति स्मृतः}


\twolineshloka
{धृतिमान्क्षिप्रकारी च वीर्येणाप्रतिमो भुवि}
{प्रियं च ते सर्वमेतत्करिष्यति न संशयः}


\threelineshloka
{पुत्रैः परिवृतः सर्वैः शूरैः परिघबाहुभिः}
{`हनिष्यति महाबाहुस्तं वै धुन्धुं महाऽसुरम्'}
{विसर्जयस्व मां ब्रह्मन्न्यस्तशस्त्रोस्मि सांप्रतम्}


\fourlineindentedshloka
{तथाऽस्त्विति च तेनोक्तो मुनिनाऽमिततेजसा}
{स तमादिश्य तनयमुदङ्काय महात्मने}
{क्रियतामिति राजर्षिर्जगाम वनमुत्तमम् ॥युधिष्ठिर उवाच}
{}


\twolineshloka
{क एष भगवन्दैत्यो महावीर्यस्तपोधन}
{कस्य पुत्रोऽथ नप्ता वा एतदिच्छामि वेदितुम्}


\twolineshloka
{एवं महाबलो दैत्यो न श्रुतो मे तपोधन}
{`यस्य निश्वासवातेन कम्पते भूः सपर्वता'}


\threelineshloka
{एतदिच्छामि भगवन्याथातथ्येन वेदितुम्}
{सर्वमेव महाप्राज्ञ विस्तरेण तपोधन ॥मार्कण्डेय उवाच}
{}


\twolineshloka
{शृणु राजननिदं सर्वं यथावृत्तं नराधिप}
{कथ्यमानं महाप्राज्ञ विस्तरेण यथातथम्}


\twolineshloka
{एकार्णवे निरालोके नष्टे स्थावरजङ्गमे}
{प्रनष्टेषु च भूतेषु सर्वेषु भरतर्षभ}


\threelineshloka
{प्रभवं लोककर्तारं विष्णुं शाश्वतमव्ययम्}
{यमाहुर्मुनयः सिद्धाः सर्वलोकमहेश्वरम्}
{`चतुर्भुजमुदाराङ्गं दृष्टवानस्मि भारत'}


\twolineshloka
{सुष्वाप भगवान्विष्णुरप्शय्यामेक एव हि}
{नागस् भोगे महति शेपस्यामिततेजसः}


\twolineshloka
{लोककर्ता महाभाग भगवानच्युतो हरिः}
{नागभगेन महता परिरभ्य महीमिमाम्}


\twolineshloka
{स्वपतस्तस्य देवस्य पद्मं सूर्यसमप्रभम्}
{नाभ्या विनिःसृतं दिव्यं तत्रोत्पन्नः पितामहः}


\twolineshloka
{साक्षाल्लोकगुरुर्ब्रह्मा पद्मे सूर्यसमप्रभः}
{चतुर्वेदश्चतुर्मूर्तिश्चतुर्वर्गश्चतुर्मुखः}


\twolineshloka
{स्वप्रभावाद्दुराधर्षो महाबलपराक्रमः}
{कस्यचित्त्वथ कालस्य दानवौ वीर्यवत्तमौ}


\twolineshloka
{मधुश्च कैटभश्चैव दृष्टवन्तौ हरिं प्रभुम्}
{शयानं शयने दिव्ये नागभोगे महाद्युतिम्}


\twolineshloka
{बहुयोजनविस्तीर्णे बहुयोजनमायते}
{किरीटकौस्तुभधरं पीतकौशेयवाससम्}


\twolineshloka
{दीप्यमानं श्रिया राजंस्तेजसा वपुपा तथा}
{सहस्रसूर्यप्रतिममद्भुतोपमदर्शनम्}


% Check verse!
विस्मय सुमहानासीन्मधुकैटभयोस्तदा
\twolineshloka
{दृष्ट्वा पितामहं चापि पद्मे पद्मनिभेक्षणम्}
{वित्रासयेतामथ तौ ब्रह्माणममितौजसम्}


\twolineshloka
{वित्रास्यमानो बहुधा ब्रह्मा ताभ्यां महायशाः}
{अकम्पयत्पद्मनालं ततोऽबुध्यत केशवः}


% Check verse!
अथापश्त गोविन्दो दानवौ वीर्यवत्तरौ
\twolineshloka
{दृष्ट्वा तावब्रवीद्देवः स्वागतं वां महाबलौ}
{ददामि वां वरं श्रेष्ठं रप्रीतिर्हि मम जायते}


\twolineshloka
{तौ प्रहस्य हृषीकेशं महादर्पौ महाबलौ}
{प्रत्यब्रूतां महाराज सहितौ मधुसूदनम्}


\threelineshloka
{आवां वरय देव त्वं वरदौ स्वः सुरोत्तम}
{दातारौ स्वो वरं तुभ्यं तद्ब्रवीह्यविचारयन् ॥भगवानुवाच}
{}


\twolineshloka
{प्रतिगृह्णे वरं वीरावीप्सितश्च वरो मम}
{युवां हि वीर्यसंपन्नौ न वामस्ति समः पुमान्}


\threelineshloka
{वध्यत्वमुपगच्छेतां मम सत्यपराक्रमौ}
{एतदिच्छाम्यहं कामं प्राप्तुं लोकहिताय वै ॥मधुकैटभावूचतुः}
{}


\twolineshloka
{अनृतंनोक्तपूर्वं नौ स्वैरेष्वपि कुतोऽन्यथा}
{सत्ये धर्मे च निरतौ विद्ध्यावां पुरुषोत्तम}


\twolineshloka
{बले रूपे च शौर्ये च शमे न च समोस्ति नौ}
{धर्मे तपसि दाने च शीलसत्वदमेषु च}


\twolineshloka
{उपप्लवो महानस्मानुपावर्तत केशव}
{उक्तं प्रतिकुरुष्व त्वं कालो हि दुरतिक्रमः}


\twolineshloka
{आवामिच्छावहे देव कृतमेकं त्वया विभो}
{आनावृतेऽवकाशे त्वं जह्यावां सुरसत्तम}


\twolineshloka
{पुत्रत्वमधिगच्छाव तव चापि सुलोचन}
{वर एष वृतोदेव तद्विद्धि सुरसत्तम}


\twolineshloka
{अनृतं मा भवेद्देव यद्धि नौ संश्रुतं तदा ॥भगवानुवाच}
{}


% Check verse!
बाढमेवं करिष्यामि सर्वमेतद्भविष्यति
\twolineshloka
{सविचिन्त्याथ गोविन्दो नापश्यद्यदनावृतम्}
{अवकाशंपृथिव्यां वा दिवि वा मधुसूदनः}


\threelineshloka
{स्वकावनावृतावूरू दृष्ट्वा देववरस्तदा}
{मधुकैटभयो राजञ्शिरसी मधुसूदनः}
{चक्रेण शितधारेण न्यकृन्तत महायशाः}


\chapter{अध्यायः २०७}
\twolineshloka
{मर्कण्डेय उवाच}
{}


\twolineshloka
{धुन्धुर्नाम महाराज तयोः पुत्रो महाद्युतिः}
{स तपोऽतप्यत महन्महावीर्यपराक्रमः}


\twolineshloka
{अतिष्ठदेकपादेन कृशो धमनिसंततः}
{तस्मै ब्रह्मा ददौ प्रीतो वरं वव्रे स च प्रभुम्}


\twolineshloka
{देवदानवयक्षाणां सर्पगन्धर्वरक्षसाम्}
{अवध्योऽहं भवेयं वै वर एष वृतो मया}


\twolineshloka
{एवं भवतु गच्छेति तमुवाच पितामहः}
{स एवमुक्तस्तत्पादौ मूर्ध्ना स्पृष्ट्वा जगाम ह}


\twolineshloka
{सतु धुन्धुर्वरं लब्ध्वा महवीर्यपराक्रमः}
{अनुस्मरन्पितृवधं द्रुतं विष्णुमुपागमत्}


\twolineshloka
{सतु देवान्सगन्धर्वाञ्जित्वा धुन्धुरमर्षणः}
{बबाध सर्वानसकृद्विष्णुं देवांश्च वै भृशम्}


\threelineshloka
{समुद्रवालुकापूर्णे उज्जानक इति स्मृते}
{आगम्य च स दुष्टात्मा तं देशं भरतर्षभ}
{बाधतेस्म परं शक्त्या तमुदङ्काश्रमं विभो}


\twolineshloka
{अन्तर्भूमिगतस्तत्र वालुकान्तर्हितस्तथा}
{मधुकैटभयोः पुत्रो धुन्धुर्भीमपराक्रमः}


\twolineshloka
{शेते लोकविनाशाय तपोबलमुपाश्रितः}
{उदङ्कस्याश्रमाभ्याशे निःश्वसन्पावकार्चिषः}


\twolineshloka
{एतस्मिन्नेव काले तु राजा सबलवाहनः}
{कुवलाश्वो नरपतिरन्वितो बलशालिनाम्}


\twolineshloka
{सहस्रैरेकविंशत्या पुत्राणामरिमर्दनः}
{प्रायादुदङ्कसहितो धुन्धोस्तस्य वधाय वै}


\twolineshloka
{तमाविशत्ततो विष्णुर्भगवांस्तेजसा प्रभुः}
{उदङ्कस् नियोगेन लोकानां हितकाम्यया}


\twolineshloka
{तस्मिन्प्रयाते दुर्धर्षे दिवि शब्दो महानभूत्}
{एष श्रीमान्नृपसुतो धुन्धुमारो भविष्यति}


\twolineshloka
{दिव्यैश्च पुष्पैस्तं देवाः समन्तात्पर्यवाकिरन्}
{देवदुन्दुभयश्चापि नेदुः स्वयमनीरिताः}


\threelineshloka
{शीतश्च वायुः प्रववौ प्रयाणे तस्य धीमतः}
{विपांसुलां महीं कुर्वन्ववर्ष च सुरेश्वरः}
{`प्रदक्षिणाश्चाप्यभवन्वन्यास्तं समृगद्विजाः'}


\twolineshloka
{अन्तरिक्षे विमानानि देवतानां युधिष्ठिर}
{तत्रैव समदृश्यन्त धुन्धुर्यत्र महासुरः}


\twolineshloka
{कुवलाश्वस्य धुन्योश्च युद्धकौतूहलान्विताः}
{देवगन्धर्वसहिताः समवैक्षन्महर्षयः}


\twolineshloka
{नारायणेन कौरव्य तेजसाऽऽप्यायितस्तदा}
{स गतो नृपतिः क्षिप्रं पुत्रैस्तैः सर्वतो दिशम्}


\twolineshloka
{अर्णवं स्वानयामास कुवलाश्वो महीपतिः}
{`हितार्तं सर्वलोकानामुदङ्कस्य वशे स्थितः'}


\twolineshloka
{कुवलाश्वस् पुत्रैश्च तस्मिन्वै वालुकार्णवे}
{सप्तभिर्दिवसैः स्वात्वा दृष्टो धुन्धुर्महाबलः}


\twolineshloka
{आसीद्धोरं वपुस्तस्य वालुकान्तर्हितं महत्}
{दीप्यमानं यथा सूर्यस्तेजसा भरतर्षभ}


\twolineshloka
{ततो धुन्धुर्महाराज दिशमावृत्य पश्चिमाम्}
{सुप्तोऽभूद्राजशार्दूल कालानलसमद्युतिः}


\twolineshloka
{कुवलाश्वस्य पुत्रैस्तु सर्वतः परिवारितः}
{अभिद्रुतः शरैस्तीक्ष्णैर्गदाभिर्मुसलैरपि}


\twolineshloka
{प्टसैः परिधैः प्रासैः खङ्गैश् विमलैः शितैः}
{सवध्यमानः संक्रुद्धः समुत्तस्थौ महाबलः}


\twolineshloka
{क्रुद्धश्चाभक्षयत्तेषां शस्त्राणि विविधानि च}
{आस्याद्वमन्पावकं स संवर्तकसमं तदा}


\twolineshloka
{तान्सर्वान्नृपतेः पुत्रानदहत्स्वेन तेजसा}
{मुखजेनाग्निना क्रुद्धो लोकानुद्वर्तयन्निव}


\twolineshloka
{क्षणेन राजशार्दूल पुरेव कपिलः प्रभुः}
{सगरस्यात्मजान्क्रुद्धस्तदद्भुतमिवाभवत्}


\threelineshloka
{तेषु क्रोधाग्निदग्धेषु तदा भरतसत्तम}
{तं प्रबुद्धं महात्मानं कुम्भकर्णमिवापरम्}
{आससाद महातेजाः कुवलाश्वो महीपतिः}


\twolineshloka
{तस्य वारिमहाराज सुस्राव बहु देहतः}
{तत्तदापीयतेतेजो राज्ञा वारिमयं नृप}


\threelineshloka
{योगी योगेन सृष्टं स समयित्वा च वारिणा}
{ब्रह्मास्त्रेण ततो राजा दैत्यंक्रूरपराक्रमम्}
{ददाह भरतश्रेष्ठ सर्वलोकभवाय वै}


\twolineshloka
{सोऽस्त्रेण दग्ध्वा राजर्षिः कुवलाश्वो महासुरम्}
{सुरशत्रुममित्रघ्नस्त्रैलोक्येश इवापरः}


\twolineshloka
{धुन्धोर्वधात्तदा राजा कुवलाश्वो महामनाः}
{धुन्धुमार इतिख्यातो नाम्ना समभवत्ततः}


\threelineshloka
{प्रीतैश्च त्रेदशैः सर्वैर्महर्षिमसितैस्तदा}
{परं वृणीष्वत्युक्तः स प्राञ्जलिः प्रणतस्तदा}
{अतीव मुदितो राजन्निदंवचनमब्रवीत्}


\threelineshloka
{दद्यां वित्तं द्विजाग्र्येभ्यः शत्रूणां चापि दुर्जयः}
{सख्यं च विष्णुना मे स्याद्भूतेष्वद्रोह एव च}
{धर्मे रतिश् सततं स्वर्गे वासस्तथाऽक्षयः}


\twolineshloka
{तथाऽस्त्विति ततो देवैः प्रीतैरुक्तः स पार्थिवः}
{ऋषिभिश्च सगन्धर्वैरुदङ्केन च धीमता}


\twolineshloka
{संभाष्य चैनं विविधैराशीर्वादैस्ततो नृषम्}
{देवा महर्षयश्चापि स्वानि स्तानानि भेजिरे}


\twolineshloka
{तस्य पुत्रास्त्रयः शिष्टा युधिष्ठिर तदाऽभवन्}
{दृढाश्वः कपिलाश्वश्च भद्राश्वश्चैव भारत}


\twolineshloka
{तेभ्यः परम्परा राजन्निक्ष्वाकूणां महात्मनाम्}
{[वंशस्य सुमहाभाग राज्ञाममिततेजसाम्]}


\twolineshloka
{एवंसं निहतस्तेन कुवलाश्वेन सत्तम}
{धुन्धुर्नाम महादैत्यो मधुकैटभयोः सुतः}


\twolineshloka
{कुवलाश्वश्च नृपतिर्धुन्धुमार इति स्मृतः}
{नाम्ना च गुणयुक्तेन तदाप्रभृति सोऽभवत्}


\twolineshloka
{एतत्ते सर्वमाख्यातं यन्मां त्वं परिपृच्छसि}
{धौन्धुमारमुपाख्यानं प्रथितं यस्य कर्मणा}


\twolineshloka
{इदं तु पुण्यमाख्यानं विष्णोः समनुकीर्तनम्}
{शृणुयाद्यः स धर्मात्मा पुत्रवांश्च भवेन्नरः}


\twolineshloka
{आयुष्मान्भूतिमांश्चैव श्रुत्वा भवति पर्वसु}
{न च व्याधिभंयकिंचित्प्राप्नोति विगतज्वरः}


\chapter{अध्यायः २०८}
\twolineshloka
{वैशंपायन उवाच}
{}


\twolineshloka
{ततो युधिष्ठिरो राजा मार्कण्डेयं महाद्युतिम्}
{प्रपच्छ भरतश्रेष्ठ धर्मप्रश्नं स दुर्वचम्}


\twolineshloka
{श्रोतुमिच्छामि भगवन्स्त्रीणां माहातम्यमुत्तमम्}
{कथ्यमानं त्वया विप्र सूक्ष्मं धर्म्यं च तत्त्वतः}


\twolineshloka
{प्रत्यक्षमिह विप्रर्षे देवा दृश्यन्ति सत्तम}
{सूर्याचन्द्रमसौ वायुः पृथिवी वह्निरेव च}


\twolineshloka
{पिता माता च भगवान्गाव एव च सत्तम}
{यच्चान्यदेव विहितं तच्चापि भृगुनन्दन}


\threelineshloka
{मन्येऽहं गुरुवत्सर्वमेकपत्न्यस्तथा स्त्रियः}
{पतिव्रतानां शुश्रूषा दुष्करा प्रतिभाति मे}
{पतिव्रतानां महात्म्यं वक्तुमर्हसि नः प्रभो}


\twolineshloka
{निरुध्य चेन्द्रियग्रामं मनः संरुध्य चानघ}
{पतिं दैवतवच्चापि चिन्तयन्त्य स्थिता हि या}


\twolineshloka
{भगवन्दुष्करं त्वेतत्प्रतिभाति मम प्रभो}
{मातापित्रोश्च शुश्रूषा स्त्रीणां भर्तरि च द्विज}


\twolineshloka
{स्त्रीणां धर्मात्सुघोराद्धि नान्यं पश्यामि दुष्करम्}
{साध्वाचाराः स्त्रियो ब्र्हमन्कुर्वन्तीह सदादृताः}


\twolineshloka
{दुष्करं खलु कुर्वन्ति पितरो मातरश्च वै}
{एकपत्न्यश्च या नार्यो याश् सत्यं वदन्त्युत}


\twolineshloka
{कुक्षिणा दशमासांश्च गर्भं संधारयन्ति याः}
{नार्यः कालेन संबूय किमद्भुततरं ततः}


\twolineshloka
{संशयं परमं प्राप्य वेदनामतुलामपि}
{प्रजायन्ते सुतान्नार्यो दुःखेन महता विभो}


\twolineshloka
{पुष्णन्ति चापि महता स्नेहेन द्विजपुङ्गव}
{`चिन्तयन्ति ततश्चापि किंशीलोऽयंभविष्यति'}


\twolineshloka
{याश्च क्रूरेषु सर्वेषु वर्तमाना जुगुप्सिताः}
{स्वकर्म कुर्वन्ति सदा दुष्करं तच्च मे मतम्}


\twolineshloka
{क्षत्रधर्मसमाचारतत्त्वं व्याख्याहि मे द्विज}
{धर्मः सुदुर्लभो विप्र नृशंसेन महात्मना}


\threelineshloka
{एतदिच्छामि भगवन्प्रश्नं प्रश्नविदांवर}
{श्रोतुं भृगुकुलश्रेष्ठ शुश्रूषे तव सुव्रत ॥मार्कण्डेय उवाच}
{}


\twolineshloka
{हन्त तेऽहं समाख्यास्ये प्रश्नमेतं सुदुर्वचम्}
{तत्त्वेन भरतश्रेष्ठ गदतस्तन्निबोध मे}


\twolineshloka
{मातरंश्रेयसीं तात पितृनन्ये तु मेनिरे}
{दुष्करं कुरुते माता विवर्धयति या प्रजाः}


\twolineshloka
{तपसा देवतेज्याभिर्वन्दनेन तितिक्षया}
{सुप्रशस्तैरुपायैश्चापीहन्ते पितरः सुतान्}


\twolineshloka
{एवं कृच्छ्रेण महता पुत्रं प्राप्य सुदुर्लभम्}
{चिन्तयन्ति सदा वीर कीदृशोऽयं भविष्यति}


\twolineshloka
{आशंसते हि पुत्रेषु पिता माता च भारत}
{यशः कीर्तिमथैश्वर्यं तेजो धर्मं तथैव च}


\twolineshloka
{`मातुः पितुश्च राजेन्द्र सततं हितकारिंणोः'}
{तयोराशां तु सफलां यः करोति स धर्मवित्}


\twolineshloka
{पिता माता च राजेन्द्र तुष्यतो यस् नित्यशः}
{इह प्रेत्य च तस्याथ कीर्तिर्धर्मश्र शाश्वतः}


\twolineshloka
{नैव यज्ञक्रियाः काश्चिन्न श्राद्धं नोपवासकम्}
{या तु भर्तरि शुश्रूषा तया स्वर्गं जयत्युत}


\twolineshloka
{एतत्प्रकरणं राजन्नधिकृत्य युधिष्ठिर}
{पतिव्रतानां नियतं धर्मं चावाहितः शृणु}


\chapter{अध्यायः २०९}
\twolineshloka
{मार्कण्डेय उवाच}
{}


\twolineshloka
{कश्चिद्द्विजातिप्रवरो वेदाध्यायी तपोधनः}
{तपस्वी धर्मशीलश्च कौशिको नाम भारत}


\twolineshloka
{साङ्गोपनिषदान्वेदानधीते द्विजसत्तमः}
{सवृक्षमूले कस्मिंश्चिद्वेदानुच्चारयन्स्थितः}


\twolineshloka
{उपरिष्टाच्च वृक्षस्य बलाका संन्यलीयत}
{तया पुरीषमुत्सृष्टं ब्राह्मणस्य तदोपरि}


\twolineshloka
{समवेक्ष्यततः क्रुद्धः सममध्यायत द्विजः}
{`तां बलकां महाराज निलीनां नगमूर्धनि}


\twolineshloka
{भृशं क्रोधाभिभूतेन बलाका सा निरीक्षिता}
{अपध्याता च विप्रेण न्यपतद्धरणीतले}


\threelineshloka
{बलाकां पतितां दृष्ट्वा गतसत्वामचेतनाम्}
{कारुण्यादभिसंतप्तः पर्यशोचत तां द्विजः}
{अकार्यं कृतवानस्मि द्वेषरागबलात्कृतः}


\twolineshloka
{इत्युक्ताव बहुशो विद्वान्ग्रामं भैक्षाय संश्रितः}
{ग्रामे शुचीनि प्रचरन्कुलानि भरतर्षभ}


\twolineshloka
{देहीति याचमानोऽसौ तिष्ठेत्युक्तः स्त्रिया ततः}
{शौचं तु यावत्कुरुते भाजनस्य कुटुम्बिवी}


\twolineshloka
{एतस्मिन्नन्तरे राजन्क्षुधासंपीडितो भृशम्}
{भर्ता प्रविष्टः सहसा तस्या भरतसत्तम}


\twolineshloka
{सा तु दृष्ट्वा पतिं साध्वी ब्राह्मणं व्यपहाय तम्}
{पाद्यमाचमनीयं वै ददौ भर्तुस्तथाऽऽसनम्}


\twolineshloka
{प्रह्वा पर्यचरच्चापि भर्तारमसितेक्षणा}
{आहारेणाथ भक्ष्यैश्च वाक्यैः सुमधुरैस्तथा}


\twolineshloka
{उच्छिष्टं भाविता भर्तुर्भुङ्क्ते नित्यं युधिष्ठिर}
{दैवतं च पतिं मेने भर्तुश्चित्तानुसारिणी}


\twolineshloka
{कर्मणा मनसा वाचा नात्यश्नान्नापि चापिवत्}
{तं सर्वभावोपगता पतिशुश्रूषणे रता}


\twolineshloka
{साध्वाचारा शुचिर्दक्षा कुटुम्बस्य हितैषिणी}
{भर्तुश्चापि हितं यत्तत्सततं साऽनुवर्तते}


\twolineshloka
{देवतातिथिभृत्यानां श्वश्रूश्वशुरयोस्तथा}
{शुश्रूषणपरा नित्यं सततं संयतेन्द्रिया}


\twolineshloka
{सा ब्राह्मणं तदा दृष्ट्वा संस्थितं भैक्षकाङ्क्षिणम्}
{कुर्वती पतिशुश्रूषां सस्माराथ शुभेक्षणा}


\threelineshloka
{व्रीडिता साऽभवत्साध्वी तदा भरतसत्तम}
{भिक्षामादाय विप्राय निर्जगाम यशस्विनी ॥ब्राह्मण उवाच}
{}


\threelineshloka
{किमिदं भवति त्वं मां तिष्ठेत्युक्त्वा वराङ्गने}
{उपरोधं कृतवती न विसर्जितवत्यसि ॥मार्कण्डेय उवाच}
{}


\twolineshloka
{ब्राह्मणं क्रोधसंतप्तं ज्वलन्तमिव तेजसा}
{दृष्ट्वा साध्वी मनुष्येन्द्रसान्त्वपूर्वं वचोऽब्रवीत्}


\twolineshloka
{`क्षमस्वविप्रप्रवर क्षमस्व स्त्रीजडात्मताम्}
{प्रसीद भगवन्मह्यं कृपां कुरु मयि द्विज'}


\threelineshloka
{क्षन्तुमर्हसि मे विद्वन्भर्ता मे दैवतं महत्}
{स चापि क्षुधितः श्रान्तः प्राप्तः शुश्रूषितोमया ॥ब्राह्मण उवाच}
{}


\twolineshloka
{ब्राह्मणआ न गरीयांसो गरीयांस्ते पतिः कृतः}
{गृहस्थधर्मे वर्तन्ती ब्राह्मणानवमन्यसे}


\twolineshloka
{इन्द्रोऽप्येषां प्रणमते किं पुनर्मानवो भुवि}
{अवलिप्ते न जानीषे वृद्धानां न श्रुतं त्वया}


\threelineshloka
{ब्राह्मणा ह्यग्निसदृशा दहेयुः पृथिवीमपि}
{`सपर्वतवनद्वीपां क्षिप्रमेवावमानिताः' ॥स्त्र्युवाच}
{}


\twolineshloka
{[नाहं बलाका विप्रर्षे त्यज क्रोधं तपोधन}
{अनया क्रुद्धया दृष्ट्या क्रुद्धः किं मां करिष्यसि]}


\twolineshloka
{नावजानाम्यहं विप्रान्देवैस्तुल्यान्मनस्विनः}
{अपराधमिमं विप्र क्षन्तुमर्हसि मेऽनघ}


\twolineshloka
{जानामि तेजो विप्राणआं महाभाग्यं च धीमताम्}
{अपेयः सागरः क्रोधात्कृतो हि लवणोदकः}


\threelineshloka
{तथैव दीप्ततपसां मुनीनां भावितात्मनाम्}
{येषां क्रोधाग्निरद्यापि समुद्रे नोपशाम्यति}
{`कस्तान्परिभवेन्मूढो ब्राह्मणानमितौजसः'}


\twolineshloka
{ब्राह्मणानां परिभवाद्वातापिः सुदुरात्मवान्}
{अगस्त्यमृषिमासाद्य जीर्णः क्रूरो महासुरः}


\threelineshloka
{बहुप्रभावाः श्रूयन्ते ब्राह्मणानां महात्मनाम्}
{क्रोधः सुविपुलो ब्रह्मन्प्रसादश्च महात्मनाम्}
{अस्मिंस्त्वतिक्रमे ब्रह्मन्क्षन्तुमर्हसि मेऽनघ}


\twolineshloka
{पतिशुश्रूषया धर्मो य स मे रोचते द्विज}
{दैवतेष्वपि सर्वेषु भर्ता मे दैवतं परम्}


\twolineshloka
{अविशेषेण तस्याहं कुर्यां धर्मं द्विजोत्तम}
{शुश्रूषायाः फलं पश्य पत्युर्ब्राह्मण यादृशम्}


\twolineshloka
{बलाका हि त्वया दग्धा रोषात्तद्विदितं मया}
{क्रोधः शत्रुः शरीरस्थो मनुष्याणां द्विजोत्तम}


\twolineshloka
{`मास्म क्रुध्यो बलाकेव न वध्याऽस्मि पतिव्रता'}
{यः क्रोधमोहौ त्यजति तं देवा ब्राह्मणं विदुः}


\twolineshloka
{यो वदेदिह सत्यानि गुरुं संतोषयेत च}
{हिंसितश्च त हिंसेत तं देवा ब्राह्मणं विदुः}


\twolineshloka
{जितेन्द्रियो धर्मपरः स्वाध्यायनिरतः शुचिः}
{कामक्रोधौ वशौ यस्य तं देवा ब्राह्मणं विदुः}


\twolineshloka
{यस्य चात्मसमो लोको धर्मज्ञस्य मनस्विनः}
{सर्वधर्मेषु चरतस्तं देवा ब्राह्मणं विदुः}


\twolineshloka
{योऽध्यापयेदधीयीत यजेद्वा याजयीत वा}
{दद्याद्वाऽपि यथाशक्ति तं देवा ब्राह्मणं विदुः}


\twolineshloka
{ब्राह्मचारी वदान्यो योप्यधीयाद्द्विजपुङ्गवः}
{स्वाध्यायवानमत्तो वै तं देवा ब्राह्मणं विदुः}


\twolineshloka
{यद्ब्राह्मणानां कुशलं तदेषां परिकीर्तयेत्}
{सत्यं तथा व्याहरतां नानृते रमते मनः}


\twolineshloka
{धनं तु ब्राह्मणस्याहुः स्वाध्यायं दममार्जवम्}
{इन्द्रियाणां निग्रहं च शाश्वतं द्विजसत्तम}


\twolineshloka
{सत्यार्जवे धर्ममाहुः परं धर्मविदो जनाः}
{दुर्ज्ञेयः शाश्वतो धर्मः स च सत्ये प्रतिष्ठितः}


\twolineshloka
{श्रुतिप्रमाणो धर्मः स्यादिति वृद्धानुशासनम्}
{बहुधा दृश्यते धर्मः सूक्ष्म एव द्विजोत्तम}


\twolineshloka
{भगवानपि धर्मज्ञः स्वाद्यायनिरतः शुचिः}
{न तु तत्त्वेन भगवन्दर्मं वेत्सीति मे मतिः}


\twolineshloka
{यदि विप्र न जानीषे धर्मं परमकं द्विज}
{धर्मव्याधं तत पृच्छ गत्वा तु मिथिलां पुरीम्}


\twolineshloka
{मातापितृभ्यां शुश्रूपुः सत्यवादी जितेन्द्रियः}
{मिथिलायां वसेद्व्याधः स ते धर्मान्प्रवक्ष्यति}


\twolineshloka
{तत्र गच्छस्व भद्रं ते यथाकामं द्विजोत्तम}
{`व्याधः परमधर्मात्मा स ते छेत्स्यति संशयम्'}


\threelineshloka
{अत्युक्तमपि मे सर्वं क्षन्तुमर्हस्यनिन्दित}
{स्त्रियो ह्यवध्याः सर्वेषां ये धर्ममभिविन्दते ॥ब्राह्मण उवाच}
{}


\twolineshloka
{प्रीतोस्मि तव भद्रं ते गतः क्रोधश्च शोभने}
{उपालम्भस्त्वया प्रोक्तो मम निश्रेयसं परम्}


\fourlineindentedshloka
{स्वस्ति तेऽस्तु गमिष्यामि साधयिष्यामि शोभने}
{`धन्या त्वमसि कल्याणि यस्याः स्याद्वृत्तमीदृशम्}
{मारक्ण्डेय उवाच}
{}


\twolineshloka
{तया विसृष्टो निर्गत्य स्वमेव भवनं ययौ}
{विनिन्दन्स स्वमात्मानं कौशिको द्विजसत्तमः}


\chapter{अध्यायः २१०}
\twolineshloka
{मार्कण्डेय उवाच}
{}


\twolineshloka
{चिन्तयित्वा तदाश्चर्यं स्त्रिया प्रोक्तमशेपतः}
{विनिन्दन्स द्विजोऽऽत्मानमागस्कृत इवाबभौ}


\twolineshloka
{चिन्तयानः स धर्मस्य सूक्ष्मां गतिमथाब्रवीत्}
{श्रद्दधानेन वै भाव्ये गच्छामि मिथिलामहम्}


\twolineshloka
{कृतात्मा धर्मवित्तस्यां व्याधो निवसते किल}
{तं गच्चाम्यहमद्यैव धर्मं प्रष्टुं तपोधनम्}


\twolineshloka
{इतिसंचिन्त्य मनसा श्रद्दधानः स्त्रिया वचः}
{बलाकाप्रत्ययेनासौ धर्म्यैश्च वचनैः शुभैः}


\twolineshloka
{संप्रतस्थे स मिथिलां कौतूहलसमन्वितः}
{अतिक्रामन्नरण्यानि ग्रामांश्च नगराणि च}


\twolineshloka
{ततो जगाम मिथिलां जनकेन सुरक्षिताम्}
{धर्मसेतुसमाकीर्णां यज्ञोत्सववतीं शुभाम्}


\twolineshloka
{गोपुराट्टालकवतीं हर्म्यप्राकारशोभनाम्}
{प्रविश्य नगरीं रम्यां विमानैर्बहुभिर्युताम्}


\twolineshloka
{पण्यैश्च बहुभिर्युक्तां सुविभक्तमहापथाम्}
{अश्वै रथैस्तथा नागैर्योधैश्च बहुभिर्युताम्}


\twolineshloka
{हृष्टपुष्टजनाकीर्णां नित्योत्सवसमाकुलाम्}
{सोऽपस्यद्बहुवृत्तान्तां ब्राह्मणः समतिक्रमन्}


\twolineshloka
{धर्मव्याधमपृच्छच्च स चास्य कथितो द्विजैः}
{अपश्यत्तत्रगत्वा तं सूनामध्ये व्यवस्थितम्}


\twolineshloka
{मार्गमाहिपमांसानि विक्रीणन्तं तपस्विनम्}
{आकुलत्वाच्च क्रेतृणामेकान्ते संस्थितो द्विजः}


\threelineshloka
{स तु ज्ञाता द्विजं प्राप्तं सहसा संभ्रमोत्थितः}
{आजगाम यतो विप्रः स्थित एकान्तआसने ॥व्याध उवाच}
{}


\twolineshloka
{अभिवादये त्वां भगवन्स्वागतं ते द्विजोत्तम}
{अहं व्याधो हि भद्रं ते किं करोमि प्रशाधि माम्}


\twolineshloka
{एकपत्न्या यदुक्तोसि गच्छ त्वं मिथिलामिति}
{जानाम्येतदहं सर्वं यदर्थं त्वमिहागतः}


\twolineshloka
{श्रुत्वा च तस्य तद्वाक्यं स विप्रो भृशविस्मितः}
{द्वितीयमिदमाश्चर्यमित्यचिन्तयत द्विजः}


\threelineshloka
{अदेशस्थं हि ते स्तानमिति व्याघोऽब्रवीद्द्विजम्}
{गृहं गच्छाव भगवन्यदि ते रोचतेऽनघ ॥मार्कण्डेय उवाच}
{}


\twolineshloka
{बाढमित्येव तं विप्रो हृष्टो वचनमब्रवीत्}
{अग्रतस्तु द्विजं कृत्वा स जगाम गृहं प्रति}


\twolineshloka
{प्रविश्य च गृहंरम्यमासनेनाभिपूजितः}
{`अर्ध्येण च स वै तेन व्याधेन द्विजसत्तमः'}


\twolineshloka
{पाद्यमाचमनीयं च प्रतिगृह्य द्विजोत्तमः}
{ततः सुखोपविष्स्तं व्याधं वचनमब्रवीत्}


\threelineshloka
{कर्मैतद्वै न सदृशं भवतः प्रतिभाति मे}
{अनुतप्ये भृशं तात तव घोरेण कर्मणा ॥व्याध उवाच}
{}


\twolineshloka
{कुलोचितमिदं कर्म पितृपैतामहं परम्}
{वर्तमानस्य मे धर्मे स्वे मन्युं मा कृथा द्विज}


\twolineshloka
{विधात्रा विहितं पूर्वं कर्म स्वमनुपालयन्}
{प्रयत्नाच्च गुरू वृद्धौ शुश्रूषेऽहं द्विजोत्तम}


\twolineshloka
{सत्यं वदे नाभ्यसूये यथाशक्ति ददामि च}
{देवतातिथिभृत्यानामवशिष्टेन वर्तये}


\twolineshloka
{न कुत्सयाम्यहं किंचिन्न गर्हे बलवत्तरम्}
{कृतमन्वेति कर्तारं पुरा कर्म द्विजेत्तम}


\twolineshloka
{कृषिगोरक्ष्यवाणिज्यमिह लोकस्य जीवनम्}
{दण्डनीतिस्त्रयो विद्या तेन लोको भवत्युत}


\twolineshloka
{कर्म शूद्रे कृषिर्वैश्ये संग्रामः क्षत्रिये स्मृतः}
{ब्रह्मचर्यतपोमन्त्राः सत्यं च ब्राह्मणे सदा}


\twolineshloka
{राजा प्रशास्ति धर्मेण स्वकर्मनिरताः प्रजाः}
{विकर्माणश्च ये केचित्तान्युनक्ति स्वकर्मसु}


\twolineshloka
{भेतव्यं हि सदा राज्ञां प्रजानामधिपा हिते}
{मारयन्ति विकर्मस्थं लुब्धा मृगमिवेषुभिः}


\twolineshloka
{जनकस्येह विप्रर्षे विकर्मस्थो न विद्यते}
{स्वकर्मनिरता वर्णाश्चत्वारोपि द्विजोत्तम}


\twolineshloka
{स एष जनको राजा दुर्वृत्तमपि चेत्सुतम्}
{दण्ड्यं दण्डे निक्षिपति यथा न ग्लाति धार्मिकम्}


\twolineshloka
{सुयुक्तचारो नृपतिः सर्वं धर्मेण पश्यति}
{श्रीश्च राज्यं च दण्डश्च क्षत्रियाणां द्विजोत्तम}


\twolineshloka
{राजानो हि स्वधर्मेण श्रियमिच्छन्ति भूयसीम्}
{सर्वेषामेव वर्णानां त्राता राजा भवत्युत}


\twolineshloka
{परेण हि हतान्ब्रह्मन्वराहमहिषानहम्}
{न स्वयं हन्मि विप्रर्षे विक्रीणामि सदा त्वहम्}


\twolineshloka
{न भक्षयामि मांसानि ऋतुगामी तथाह्यहम्}
{सदोपवासी च तथा नक्तभोजी सदा द्विज}


\twolineshloka
{अशीलश्चापि पुरुषो भूत्वा भवति शीलवान्}
{प्राणिहिंसारतिश्चापि भवते धार्मिकः पुनः}


\twolineshloka
{व्यभिचाराननरेन्द्राणां धर्मः संकीर्यते महान्}
{अधर्मो वर्तते चापि संकीर्यन्ते ततः प्रजाः}


\twolineshloka
{भेरुण्डा वामनाः कुब्जाः स्थूलशीर्षास्तथैव च}
{क्लीबाश्चान्धाश्च बधिरा जायन्तेऽत्युच्चलोचनाः}


\twolineshloka
{पार्थिवानामधर्मत्वात्प्रजानामभवः सदा}
{स एष राजा जनकः प्रजा धर्मेण पश्यति}


\twolineshloka
{अनुगृह्णन्प्रजाः सर्वाः स्वधर्मनिरताः सदा}
{`पात्येष राजा जनकः पितृवद्द्विजसत्तम'}


\twolineshloka
{येचैव मां प्रशंसन्ति येच निन्दन्ति मानवाः}
{सर्वान्सुपरिणीतेन कर्माणा तोषयाम्यहम्}


\twolineshloka
{ये जीवन्ति स्वधर्मेण संयुञ्जन्ति च पार्थिवाः}
{न किंचिदुपजीवनति दान्ता उत्थानसीलिनः}


\twolineshloka
{शक्त्याऽन्नदानं सततं तितिक्षा धर्मनित्यता}
{यथार्हं प्रतिपूजा च सर्वभूतेषु वै दया}


\twolineshloka
{त्यागान्नान्यत्र मर्त्यानां गुणास्तिष्ठान्ति पूरुषे}
{मृषावादं परिहरेत्कुर्यात्प्रियमयाचितः}


\twolineshloka
{न च कामान्न संरम्भान्न द्वेषाद्धर्ममुत्सृजेत्}
{प्रिये नातिभृशं हृष्येदप्रिये न च संज्वरेत्}


\twolineshloka
{न मुह्येदर्थकृच्छ्रेषु न च धऱ्मं परित्यजेत्}
{कर्म चेत्किंचिदन्यत्स्यादितरन्न तदाचरेत्}


\twolineshloka
{यत्कल्याणमभिध्यायेत्तत्रात्मानं नियोजयेत्}
{न पापं प्रति पापः स्यात्साधुरेव सदा भवेत्}


\twolineshloka
{आत्मनैव हतः पापो यः पापं कर्तुमिच्छति}
{कर्म चैतदसाधूनां वृजिनानामसाधुकम्}


\twolineshloka
{न धर्मोस्तीति मन्वानाः शुचीनवहसन्ति ये}
{अश्रद्दधाना धर्मस् ते नश्यन्ति न संशयः}


\twolineshloka
{महादृतिरिवाध्मातः पापो भवति नित्यदा}
{`साधुः सन्नतिमानेव सर्वत्रद्विजसत्तम'}


\twolineshloka
{मूढानामवलिप्तानामसारं भाषितं भवेत्}
{दर्शयन्त्यन्तरात्मानं दिवा रूपमिवांशुमान्}


\twolineshloka
{न लोके राजते मूर्खः केवलात्मप्रशंसया}
{अपिचेह मृजाहीनः कृतविद्यः प्रकाशते}


\twolineshloka
{अब्रुवन्कस्यचिन्निन्दामात्मपूजामवर्णयन्}
{न कश्चिद्गुणसंपन्नः प्रकाशो भुवि दृश्यते}


\twolineshloka
{विकर्मणा तप्यमानः पापाद्विपरिमुच्यते}
{न तत्कुर्यां पुनरिति द्वितीयात्परिमुच्यते}


\twolineshloka
{कर्मणा येन केनापि पापात्मा द्विजसत्तम}
{एवं श्रुतिरियं ब्रह्मन्धर्मेषु प्रतिदृश्यते}


\twolineshloka
{पापनि बुद्ध्वेह पुरा कृतानिस्वधर्मशीलो विनिहन्ति पश्चात्}
{धर्मो ब्रह्मन्नुदते ब्राह्मणानांयत्कुर्वते पापमिह प्रमादात्}


\twolineshloka
{पापं कृत्वा हि मन्येत नाहमस्मीति पूरुषः}
{[तं तु देवाः प्रपश्यन्ति स्वस्यैवान्तरपूरुषः]}


% Check verse!
चिकीर्षेदेव कल्यायणं श्रद्दधानोऽनसूयकः
\twolineshloka
{वसनस्येव च्छिद्राणइ साधूनां विवृणोति यः}
{`अपश्यन्नात्मनो दोषान्स पापः प्रेत्य नश्यति'}


\twolineshloka
{पापं चेत्पुरुषः कृत्वा कल्याणमभिपद्यते}
{मुच्यते सर्वपापेभ्यो महाभ्रेणेव चन्द्रमाः}


\twolineshloka
{यथाऽऽदित्यः समुद्यन्वै तमः सर्वं व्यपोहति}
{एवं कल्याणमातिष्ठन्सर्वपापैः प्रमुच्यते}


\twolineshloka
{पापानां विद्ध्यधिष्ठानं लोभमोहौ द्विजोत्तम}
{`तस्मात्तौ विदुषा विप्र वर्जनीयौ विशेषतः'}


\twolineshloka
{लुबधाः पापं व्यवस्यन्ति नरा नातिबहुश्रुताः}
{अधर्म्या धर्मरूपेण तृणैः कूपा इवावृताः}


\twolineshloka
{येषां पञ् पवित्राणि प्रलापा धर्मसंश्रितः}
{सर्वं हि विद्यते तेषु शिष्टाचारः सुदुर्लभः}


\chapter{अध्यायः २११}
\twolineshloka
{मार्कण्डेय उवाच}
{}


\twolineshloka
{स तु विप्रो महाप्राज्ञ धर्मव्याधमपृच्छत}
{शिष्टाचारं कथमहं विद्यामिति नरोत्तम}


\fourlineindentedshloka
{`पञ्च कानि पवित्राणि शिष्टाचारेषु नित्यदा'}
{एतदिच्छामि भद्रं ते श्रोतुं धर्मभृतांवर}
{त्वत्तो महामते व्याध तद्ब्रवीहि यथातथम् ॥व्याध उवाच}
{}


\twolineshloka
{यज्ञो दानं तपो वेदाः सत्यं च द्विजसत्तम}
{पञ्चैतानि पवित्राणि शिष्टाचारेषु नित्यदा}


\twolineshloka
{कामक्रोधौ वशे कृत्वा दम्भं लोभमनार्जवम्}
{धर्ममित्येवं संतुष्टास्ते शिष्टाः शिष्टसंमताः}


\twolineshloka
{न तेषां भिद्यते वृत्तं यज्ञस्वाध्यायशीलिनाम्}
{आचारपालनं चैव द्वितीयं शिष्टलक्षणम्}


\twolineshloka
{गुरुशुश्रूषणं सत्यमक्रोधो दानमेव च}
{एतच्चतुष्टयं ब्रह्मञ्शिष्टाचारेषु नित्यदा}


\twolineshloka
{शिष्टाचारे मनः कृत्वा प्रतिष्ठाप्य च सर्वशः}
{यामयं लभते तुष्टिं सा न शक्या ह्यतोऽन्यथा}


\twolineshloka
{वेदस्योपनिषत्सत्यं सत्यस्योपनिषद्दमः}
{दमस्योपनिषत्त्यागः शिष्टाचारेषु नित्यदा}


\twolineshloka
{ये तु धर्मानसूयन्ते बुद्धिमोहान्विता नराः}
{अपथा गच्छतां तेषामनुयाता च पीड्यते}


\twolineshloka
{ये तु शिष्टाः सुनियताः श्रुतित्यागपरायणाः}
{धर्मपन्थानमारूढाः सत्यधर्मपरायणाः}


\twolineshloka
{नियच्छन्ति परां बुद्धिं शिष्टाचारान्विता जनाः}
{उपाध्यायमते युक्ताः स्थित्या धर्मार्थदर्शिनः}


\twolineshloka
{नास्तिकान्भिन्नमर्यादान्क्रूरान्पापमतौ स्थितान्}
{त्यज ताञ्ज्ञानमाश्रित्य धार्मिकानुपसेव्य च}


\twolineshloka
{कामलोभग्रहाकीर्णं पञ्चेन्द्रियजलां नदीम्}
{नावं धृतिमयीं कृत्वा जन्मदुर्गाणि संतर}


\twolineshloka
{क्रमेण संचितो धर्मो बुद्धियोगमयो महान्}
{शिष्टाचारे भवेत्साधू रागः शुक्ले व वाससि}


\threelineshloka
{अहिंसा सत्यवचनं सर्वभूतहितं परम्}
{अहिंसा परमो धर्मः स च सत्ये प्रतिष्ठितः}
{सत्यं कृत्वा प्रतिष्ठां तु प्रवर्तन्ते प्रवृत्तयः}


\twolineshloka
{सत्यमेव गरीयस्तु शिष्टाचारनिषेवितम्}
{आचारश्च सतां धर्मः सन्तो ह्याचारलक्षणाः}


\twolineshloka
{यो यथा प्रकृतिर्जन्तुः स स्वां प्रकृतिमश्नुते}
{पापात्मा क्रोधकामादीन्दोषानाप्नोत्यनात्मवान्}


\twolineshloka
{आरम्भो न्याययुक्तो यः स हि धऱ्म इति स्मृतः}
{अनाचारस्त्वध्रमेति एतच्छिष्टानुशासनम्}


\twolineshloka
{अक्रुध्यन्तोऽनसूयन्तो निरहंकारमत्सराः}
{ऋजवः शमसंपन्नाः शिष्टाचारा भवन्ति ते}


\twolineshloka
{त्रैविद्यवृद्धाः शुचयो वृत्तवन्तो मनस्विनः}
{गुरुशुश्रूषवो दान्ताः शिष्टाचारा भवन्त्युत}


\twolineshloka
{तेषामहीनसत्वानां दुष्कराचारकर्मणाम्}
{स्वैः कर्मभिः सत्कृतानां घोरत्वं संप्रणश्यति}


\twolineshloka
{तं तदाचारमाश्चर्यं पुराणं शाश्वतं ध्रुवम्}
{धर्म्यं धर्मेण पश्यन्तः स्वर्गं यान्ति मनीषिणः}


\twolineshloka
{आस्तिका मानहीनाश्च द्विजातिजनपूजकाः}
{श्रुतवृत्तोपसंपन्नास्ते सन्तः स्वर्गगामिनः}


\twolineshloka
{वेदोक्तः प्रथमो धर्मो धर्मशास्त्रेषु चापरः}
{शिष्टाचीर्णश् शिष्टानां त्रिविधं धर्मलक्षणम्}


\twolineshloka
{धारणं चापि वेदानां तीर्थानामवगाहनम्}
{क्षमा सत्यार्जवं शौचं शिष्टाचारनिदर्शनम्}


\twolineshloka
{सर्वभूतदयावन्तो ह्यहिंसानिरताः सदा}
{परुषं च न भाषन्ते सदा सन्तो द्विजप्रियाः}


\twolineshloka
{शुभानामशुभानां च कर्मणां सबलाश्रयम्}
{विपाकमभिजानन्ति ते शिष्टाः शिष्टसंमताः}


\threelineshloka
{न्यायोपेता गुणोपेताः सर्वलोकहितैषिणः}
{सन्तः स्वर्गजितः शक्त्या सन्निविष्टाश्च सत्पथे}
{दातारः संविभक्तारो दीनानुग्रहकारिणः}


\twolineshloka
{सर्वपूज्याः श्रुतधनास्तथैव च तपस्विनः}
{सर्वभूतदयावन्तस्ते शिष्टाः शिष्टसंमताः}


\twolineshloka
{दाननित्याः सुखान्याशु प्राप्नुवन्त्यपि च श्रियम्}
{पीडया च कलत्रस्य भृत्यानां च समाहिताः}


\threelineshloka
{अतिशक्त्या प्रयच्छन्ति सन्तः सद्भिः समागताः}
{लोकयात्रां च पशय्न्तो धर्ममात्महितानि च}
{एवं सन्तोवर्तमानास्त्वेधन्ते शाश्वतीः समाः}


\twolineshloka
{अहिंसा सत्यवचनमानृशंस्यवथार्जवम्}
{अद्रोहो नातिमानश्च ह्रीस्तितिक्षा दमः शमः}


\twolineshloka
{धीमन्तो धृतिमन्तश्च भूतानामनुकम्पकाः}
{अकामद्वेषसंयुक्तास्ते सन्तो लोकसत्कृताः}


\twolineshloka
{त्रीण्येव तु पदान्याहुः सतां वृत्तमनुस्मरन्}
{न चैव द्रुह्येद्दद्याच्च सत्यं चैव सदा वदेत्}


\twolineshloka
{सर्वत्र च दयावन्तः सन्तः करुणवेदिनः}
{गच्छन्तीह सुसंतुष्टा धर्म्यं पन्थानमुत्तमम्}


\twolineshloka
{शिष्टाचारा महात्मानो येषां धर्मः सुनिश्चितः}
{अनसूया क्षमा शान्तिः संतोषः प्रियवादिता}


\twolineshloka
{कामक्रोधपरित्यागः शिष्टाचारनिषेवणम्}
{कर्म च श्रुतसंपन्नं सतां मार्गमनुत्तमम्}


\twolineshloka
{शिष्टाचारं निषेवन्ते नित्यं ध्रममनुव्रताः}
{प्रज्ञाप्रासादमारूह्य मुह्यतो महतो जनान्}


\twolineshloka
{प्रेक्षन्ते लोकवृत्तानि विविधानि द्विजोत्तमाः}
{अतिषुण्यानि दानानि तानि द्विजवरोत्तम}


\twolineshloka
{एतत्ते सर्वमाख्यातं यथाप्रज्ञं यथाश्रुतम्}
{शिष्टाचारगुणान्ब्रह्मन्पुरस्कृत्य द्विजर्षभ}


\chapter{अध्यायः २१२}
\twolineshloka
{मार्कण्डेय उवाच}
{}


\twolineshloka
{स तु विप्रमथोवाच धर्मव्याधो युधिष्ठिर}
{यदहं ह्याचरे कर्म घोरमेतदसंशयम्}


\twolineshloka
{विधिस्तु बलवान्ब्रह्मन्दुस्तरं हि पुरा कृतम्}
{पुरा कृतस् पापस्य कर्मदोषो भवत्ययम्}


\twolineshloka
{दोपस्यैतस्य वै ब्रह्मन्विघाते यत्नवानहम्}
{विधिना हि हते पूर्वं निमित्तं घातको भवेत्}


% Check verse!
निमित्तभूता हि वयं कर्मणोऽस्य द्विजोत्तम
\threelineshloka
{येषां हतानां मांसानि विक्रीणीमो वयं द्विज}
{तेषामपि भवेद्धर्म उपयोगेन भक्षणात्}
{देवतातिथिभृत्यानां पितृणां चापि पूजनात्}


\twolineshloka
{ओषध्यो वीरुधश्चैव पशवो मृगपक्षिणः}
{अन्नाद्यभूता लोकस्य इत्यपि श्रूयते श्रुतिः}


\twolineshloka
{आत्ममांसप्रसादेन शिबिरौशीनरो नृपः}
{स्वर्गं सुदुर्लभं प्राप्तः क्षमावान्द्विजसत्तम}


\threelineshloka
{राज्ञो महानसे पूर्वं रन्तिदेवस्य वै द्विज}
{[द्वे सहस्रे तु पच्छेते पशूनामन्वहं तदा}
{]अहन्यहनि पच्येते द्वे सहस्रे गवां तथा}


\twolineshloka
{स मासं ददतो ह्यन्नं रन्तिदेवस्य नित्यशः}
{अतुला कीर्तिरभवन्नृपस्य द्विजसत्तम}


\twolineshloka
{चातुर्मास्ये च पशवो वध्यन्त इति नित्यशः}
{अग्नयो मांसकामाश्चइत्यपि श्रूयते श्रुतिः}


\twolineshloka
{यज्ञेषु पशवो ब्रह्मन्वध्यन्ते सततं द्विजैः}
{संस्कृताः किल मन्त्रैश्च तेऽपि स्वर्गमवाप्नुवन्}


\twolineshloka
{यदि नैवाग्नयो ब्रह्मन्मांसकामाऽभवन्पुरा}
{भक्ष्यं नैवाभवन्मांसं कस्यचिद्द्विजसत्तम}


% Check verse!
अत्रापि विधिरुक्तश् मुनिभिर्मांसभक्षणे
\twolineshloka
{देवतानां पितृणां च शुङ्क्ते दत्त्वाऽपियः सदा}
{यथाविधि यथाश्रद्धं न स दुष्येत भक्षणात्}


\twolineshloka
{अमांसाशी भवत्येवमित्यपि श्रूयते श्रुतिः}
{भार्यां गच्छन्ब्रह्मचारी ऋतौ भवति ब्राह्मणः}


\threelineshloka
{सत्यानृते विनिश्चित्य अत्रापि विधिरुच्यते}
{सौदासेन तदा राज्ञा मानुषा भक्षिता द्विज}
{शापाभिभूतेन भृशमत्र किं प्रतिभाति ते}


\twolineshloka
{स्वधर्म इतिकृत्वा तु न त्यजामि द्विजोत्तम}
{पुरा कृतमिति ज्ञात्वा रजीवाम्येतेन कर्मणा}


\twolineshloka
{स्वधर्मं त्यजतो ब्रह्मन्नधर्म इह दृश्यते}
{स्वकर्मनिरतो यस्तु धर्मः स इति निश्चयः}


\twolineshloka
{कुले हि विहितं कर्म देही तं न विमुञ्चति}
{धात्रा विधिरयं दृष्टो बहुधा कर्मनिर्मये}


\twolineshloka
{द्रष्टव्यस्तु भवेद्ब्रह्मन्धर्मो धर्मविनिश्चये}
{कथं कर्म शुभं कुर्यां कथं मुच्ये पराभवात्}


\threelineshloka
{कर्मणस्तस्य घोरस् वसुधा निर्णयो भवेत्}
{दाने च सत्यवाक्ये च गुरुशुश्रूणे तथा}
{द्विजातिपूजने चाहं धर्मे च निरतः सदा}


\twolineshloka
{अतिमानातिवादाभ्यां निवृत्तोस्मि द्विजोत्तम}
{कृषिं साध्वीति मन्यन्ते तत्र हिंसा परा स्मृता}


\twolineshloka
{कर्षन्तो लाङ्गलैरुर्वीं घ्नन्ति भूमिशयान्बहून्}
{जीवानन्यांश्च बहुशस्तत्रकिं प्रतिभाति ते}


\twolineshloka
{धान्यबीजानि यान्याहुर्व्रीह्यादीनि द्विजोत्तम}
{सर्वाण्येतानि जीवा हि तत्र किं प्रतिभाति ते}


\twolineshloka
{अध्याक्रम् पशूंश्चापि घ्नन्ति वै भक्षयन्ति च}
{वृक्षांस्तथौषधीश्चापि छिन्दन्ति पुरुषा द्विज}


\twolineshloka
{जीवा हि बहवो ब्रह्मन्वृक्षेषु च फलेषु च}
{उदके बहवश्चापि तत्रकिं प्रतिभाति ते}


\twolineshloka
{सर्वं व्याप्तमिदं ब्रह्मन्प्राणिभिः प्राणिजीवनैः}
{मत्स्यान्ग्रसन्ते मत्स्याश्च तत्रकिं प्रतिभाति ते}


\twolineshloka
{सत्वैः सत्वानि जीवन्ति बहुधा द्विजसत्तम}
{प्राणिनोऽन्योन्यभक्षाश्च तत्रकिं प्रतिभाति ते}


\twolineshloka
{चङ्क्रम्यमाणा जीवांश्च धरणीसंश्रितान्बहून्}
{पद्भ्यां घ्नन्ति नरा विप्र तत्र किं प्रतिभाति ते}


\twolineshloka
{उपविष्टाः शयानाश्च घ्नन्ति जीवाननेकशः}
{अज्ञानादथवा ज्ञानात्तत्रकिं प्रतिभाति ते}


\twolineshloka
{जीवैर्ग्रस्तमिदं सर्वमाकाशं पृथिवी तथा}
{अविज्ञानाच्च हिंसन्ति तत्र किं प्रतिभाति ते}


\twolineshloka
{अहिंसेति यदुक्तं हि पुरुषैर्विस्मितैः पुरा}
{के न हिंसन्ति जीवान्वै लोकेऽस्मिन्द्विजसत्तम्}


% Check verse!
बहु संचिन्त्य इह वै नास्ति कश्चिदहिंसकः
\twolineshloka
{अहिंयासां तु निरता यतयो द्विजसत्तम}
{कुर्वन्त्येव हि हिंसां ते यत्नादल्पतरा भवेत्}


\twolineshloka
{आलक्ष्याश्चैव पुरुषाः कुले जाता महागुणाः}
{महाघोराणि कर्माणि कृत्वा लज्जन्ति वै न च}


\twolineshloka
{सुहृदः सुहृदोऽन्यांस्च दुर्हृदश्चापि दुर्हृदः}
{सम्यक्प्रवृत्तान्पुरुषानन सम्यगनुपश्यति}


\twolineshloka
{समृद्धैश्चन नन्दन्ति बान्धवा बान्धवैरपि}
{गुरूंश्चैव विनिदन्ति मूढा निश्चितमानिनः}


\twolineshloka
{बहु लोके विपर्यस्तं दृश्यते द्विजसत्तम}
{धर्मयुक्तमधर्मं च तत्र किं प्रतिभाति ते}


\twolineshloka
{वक्तुं बहुविधं शक्यं धर्माधर्मेषु कर्मसु}
{स्वकर्मनिरतो यो हि स यशः प्राप्नुयान्महत्}


\chapter{अध्यायः २१३}
\twolineshloka
{मार्कण्डेय उवाच}
{}


\twolineshloka
{धर्मव्याधस्तु निपुणं पुनरेव युधिष्ठिर}
{विप्रर्षभमुवाचेदं सर्वधर्मभृतांवर}


\twolineshloka
{श्रुतिप्रमाणो धर्मोऽयमिति वृद्धानुशासनम्}
{सूक्ष्मा गतिर्हि धर्मस् बहुशाखा ह्यनन्तिका}


\twolineshloka
{प्राणान्तिके विवाहे च वक्तव्यमनृतं भवेत्}
{अनृतेन भवेत्सत्यं सत्येनैवानृतं भवेत्}


\twolineshloka
{यद्भूतहितमत्यन्तं तत्सत्यमिति धारणा}
{विपर्ययकृतोऽधर्मः पश्य धर्मस्य सूक्ष्मताम्}


\twolineshloka
{यत्करोत्यशुभं कर्म शुभं वा यदि सत्तम}
{अवश्यं तत्समाप्नोति पुरुषो नात्र संशयः}


\twolineshloka
{विषमां च दशां प्राप्तो देवान्गर्हति वै भृशम्}
{आत्मृनः कर्मदोषेण न विजानात्यपण्डितः}


\twolineshloka
{मूढो नैकृतिकश्चापि चपलश्च द्विजोत्तम}
{`न शुभं कर्म बध्नाति पुरुषं पाषनिश्चयम्'}


\twolineshloka
{सुखदुःखविपर्यासो यदा समुपपद्यते}
{नैनं प्रज्ञा सुनीतं वा त्रायते नैव पौरुषम्}


\twolineshloka
{यो यमिच्छेद्यथा कामं तं तं कामं स आप्नुयात्}
{यदि स्यादपराधीनं पौरुषस् क्रियाफलम्}


\twolineshloka
{संयताश्चापि दक्षाश्च मतिमन्तश्च मानवाः}
{दृश्यन्ते निष्फलाः सन्तः प्रहीणाः सर्वकर्मभिः}


\twolineshloka
{भूतानामपरः कश्चिद्धिंसायां सततोत्थितः}
{वञ्चनायां च लोकस्य स सुखेनैव युज्यते}


\twolineshloka
{अचेष्टमपि चासीनं श्रीः कंचिदुपतिष्ठति}
{कश्चित्कर्माणि कुर्वन्हि न प्राप्यमधिगच्छति}


\twolineshloka
{देवानिष्ट्वा तपस्तप्त्वा कृपणैः पुत्रगृध्नुभिः}
{दशमासधृता गर्भा जायन्ते कुलपांसनाः}


\twolineshloka
{अपरे धनधान्यैश्च भोगैश्च पितृसंचितैः}
{विपुलैरभिजायन्ते लब्धास्तैरेव मङ्गलैः}


\twolineshloka
{`न देहजा मनुष्याणां व्याधयो द्विजसत्तम'}
{कर्मजा हि मनुष्याणां रोगा नास्त्यत्र संशयः}


\twolineshloka
{आधिभिश्चैव बाध्यन्ते व्यालैः क्षुद्रमृगा इव}
{व्याधयो विनिवार्यन्ते मृगा व्याधैरिव द्विज}


\twolineshloka
{येषामस्ति च भोक्तव्यं ग्रहणीरोगपीडिताः}
{न शक्नुवन्ति ते भोक्तुं चेष्टितं पूर्वकर्मया}


\twolineshloka
{अपरे बाहुबलिनः क्लिश्यन्ति बहवो जनाः}
{दुःखेन चाधिगच्छनति भोजनं द्विजसत्तम}


\twolineshloka
{इति लोकमनाक्रन्दं देहशङ्कापरिप्लुतम्}
{स्रोतसाऽसकृदाक्षिप्तं ह्रियमाणं बलीयसा}


\twolineshloka
{न म्रियेयुर्न जीर्यैयुः सर्वे स्युः सर्वकामिकाः}
{नाप्रियं प्रतिपश्येयुर्विधिश्च यदि नो भवेत्}


\twolineshloka
{उपर्युपरि लोकस्य सर्वो गन्तुं समीहते}
{यतते च यथाथक्ति न च तद्वर्तते तथा}


\twolineshloka
{बहवः संप्रदृश्यन्ते तुल्यनक्षत्रमङ्गलाः}
{महत्तु फलवैषम्यं दृश्यते कर्मसिद्धिषु}


\twolineshloka
{न केचिदीशते ब्रह्मन्स्वयंग्राह्यस्य सत्तम}
{कर्मणां प्राकृतानां वै इह सिद्धिः प्रदृश्यते}


\twolineshloka
{तथा श्रुतिरियंब्रह्मञ्जीवः किल सनातनः}
{शरीरमध्रुवं लोके सर्वेषां प्राणिनामिह}


\threelineshloka
{वध्यमाने शरीरे तु देहनाशो भवत्युत}
{जीवः संक्रमतेऽन्यत्रकर्मबनधनिबन्धनः ॥ब्राह्मण उवाच}
{}


\threelineshloka
{कथं धर्मविदांश्रेष्ठ जीवो भवति शाश्वतः}
{एतदिच्छाम्यहं ज्ञातुं तत्त्वेन वदतांवर ॥व्याध उवाच}
{}


\twolineshloka
{न जीवनाशोस्ति हि देहभेदेमिथ्यैतदाहुर्म्रियतीति मूढाः}
{जीवस्तु देहान्तरितः प्रयातिदशार्धतैवास् शरीरभेदः}


\twolineshloka
{अन्यो हि नाश्नाति कृतं हि कर्ममनुष्यलोके मनुजस् कश्चित्}
{यत्तेन किंचिद्धि कृतंहि कर्मतदश्नुते नास्ति कृतस्य नाशः}


\threelineshloka
{सुपुण्यशीला हि भवन्ति पुण्यानराधमाः पापकृतो भवन्ति}
{नरोऽनुयातस्त्विह कर्मभिः स्वै-स्ततः समुत्पद्यति भावितस्तैः ॥ब्राह्मण उवाच}
{}


\threelineshloka
{कथं संभवते योनौ कथं वा पुण्यपापयोः}
{जातीः पुण्या ह्यपुण्याश् कथं गच्छति सत्तम ॥व्याध उवाच}
{}


\twolineshloka
{गर्भाधानसमायुक्तं कर्मेदं संप्रदृश्यते}
{समासेन तु ते क्षिप्रं प्रवक्ष्यामि द्विजोत्तम}


\twolineshloka
{यथा संभृतसंभारः पुनरेव प्रजायते}
{शुभकृच्छुभयोनीषु पापकृत्पापयोनिषु}


\twolineshloka
{शुभैः प्रयोगैर्देवत्वंव्यामिश्रैर्मानुषो भवेत्}
{मोहनीयैर्वियोनीषु त्वधोगामी च किल्बिषैः}


\twolineshloka
{जातिमृत्युजरादुःखैः सततं समभिद्रुतः}
{संसारे पच्यमानश्च दोषैरात्मकृतैर्नरः}


\twolineshloka
{तिर्यग्योनिसहस्राणि गत्वा नरकमेव च}
{जीवाः संपरिवर्तन्ते कर्मबन्धनिबन्धनाः}


\twolineshloka
{जन्तुस्तु कर्मभिस्तैस्तैः स्वकृतैः प्रेत्य दुःखितः}
{तद्दुःखप्रतिघातार्थमपुण्यां योनिमाप्नुते}


\twolineshloka
{ततः कर्म समादत्ते पुनरन्यन्नवं बहु}
{पच्यते तु पुनस्तेन भुक्त्वाऽपथ्यमिवातुरः}


\twolineshloka
{अजस्रमेव दुःखार्तोऽदुःखितः सुखिसंज्ञितः}
{ततो निवृत्तबन्धत्वात्कर्मणामुदयादपि}


\twolineshloka
{परिक्रामति संसारे चक्रवद्बहुवेदनः}
{स चेन्निवृत्तबन्धस्तु विशुद्धश्चापि कर्मभिः}


\twolineshloka
{तपोयोगसमारम्भं कुरुते द्विजसत्तम}
{कर्मभिर्बहुभिश्चापि लोकानश्नाति कर्मभिः}


\twolineshloka
{[स चेन्निवृत्तबन्धस्तु विशुद्धश्चापि कर्मभिः}
{]प्राप्नोति सुकृताँल्लोकान्यत्रगत्वा न शोचति}


\threelineshloka
{पापं कुर्वनपुण्यवृत्तः पुण्यस्यान्तं न गच्छति}
{`पुण्यं कुर्वन्पुण्यवृत्तः पुण्यस्यान्तं न गच्छति'}
{तस्मात्पुण्यं यतेत्कर्तुं वर्जयीत च पापकम्}


\twolineshloka
{अनसूयुः कृतज्ञश्च कल्याणान्येव सेवते}
{सुखानि धर्ममर्थं च स्वर्गं च लभते नरः}


\twolineshloka
{संस्कृतस्य च दान्तस् नियतस्य यतात्मनः}
{प्राज्ञास्यानन्तरा वृत्तिरिहि लोके परत्र च}


\twolineshloka
{सतां धऱ्मेण वर्तेत क्रियां शिष्टवदाचरेत्}
{असंक्लेशेन लोकस्य वृत्तिं लिप्सेत वै द्विजः}


\twolineshloka
{सन्ति ह्यागमविज्ञानाः शिष्टाः शास्त्रे विचक्षणाः}
{स्वधर्मेण क्रिया लोके कुर्वाणास्ते ह्यसंकराः}


\twolineshloka
{प्राज्ञो धर्मेण रमते धर्मं चैवोपजीवति}
{तस्माद्धर्मादवाप्तेन धनन द्विजसत्तम}


\threelineshloka
{तस्यैव सिंचते मूलं गुणान्पश्यति यत्र वै}
{धर्मात्मा भवति ह्येवं चित्तं चास्य प्रसीदति}
{स मित्रजनसंतुष्ट इह प्रेत्य च नन्दति}


\twolineshloka
{शब्दं स्पर्शं तथा रूपं गन्धानिष्टांस्च सत्तम}
{प्रभुत्वं लभते चापि धर्मस्यैतत्फलं विदुः}


\twolineshloka
{धरमस्य च पलं लब्ध्वा न तुष्यति महाद्विज}
{अतुष्यमाणो निर्वेदमादत्ते ज्ञानचक्षुषा}


\twolineshloka
{प्रज्ञाचक्षुर्नर इह दोषं नैवानुरुध्यते}
{विरज्ये यथाकामं न च धर्मं विमुञ्चति}


\twolineshloka
{फलत्यागे च यतते दृष्ट्वा लोकं क्रियात्मकम्}
{ततो मोक्षे प्रयतते नानुपायादुपायतः}


\twolineshloka
{एवं निर्वेदमादत्ते पापं कर्म जहाति च}
{धार्मिकश्चापि भवति मोक्षं च लभते परम्}


\twolineshloka
{तपो निःश्रेयसं जन्तोस्तस्य मूलं शमो दमः}
{तेन सर्वानवाप्नोति कामान्यान्मनसेच्छति}


\threelineshloka
{इन्द्रियाणां निरोधेन सत्येन च दमेन च}
{ब्रह्मणः पदमाप्नोति यत्परं द्विजसत्तम ॥ब्राह्मण उवाच}
{}


\twolineshloka
{इन्द्रियाणीति यान्याहुः कानि तानि यतव्रत}
{निग्रहश्च कथं कार्यो निग्रहस्य च किं फलम्}


\twolineshloka
{कथं च फलमाप्नोति तेषां धर्मभृतांवर}
{एतदिच्छामि तत्त्वेन धर्मं ज्ञातुं सुधार्मिक}


\chapter{अध्यायः २१४}
\twolineshloka
{मार्कण्येय उवाच}
{}


\threelineshloka
{एवमुक्तस्तु विप्रेण धर्मव्याधो युधिष्ठिर}
{प्रत्युवाच यथा विप्रं तच्छृणुष्व नराधिप ॥व्याध उवाच}
{}


\twolineshloka
{विज्ञानार्थं मनुष्याणां मनः पूर्वं प्रवर्तते}
{तत्प्राप्य कामं भजतेक्रोधं च द्विजसत्तम}


\twolineshloka
{ततस्तदर्थं यतते कर्म चारभते महत्}
{इष्टानां रूपगन्धानामभ्यासं च निषेवते}


\twolineshloka
{ततो रागः प्रभवति द्वेषश्च तदनन्तरम्}
{ततो लोभः प्रभवति मोहश्च तदनन्तरम्}


\twolineshloka
{तस् लोभाभिभूतस्य रागद्वेषहतस्य च}
{न धर्मे जायते बुद्धिर्व्याजाद्धर्मं करोति च}


\twolineshloka
{व्याजेन चरते धर्ममर्थं व्याजेन रोचते}
{व्याजेन सिध्यमानेषु धनेषु द्विजसत्तम}


\twolineshloka
{तत्रैव रमते बुद्धिस्ततः पापं चिकीर्षति}
{सुहृद्भिर्वार्यमाणश्च पण्डितैश्च द्विजोत्तम}


\twolineshloka
{उत्तरं श्रुतिसंबद्धं ब्रवीत्यश्रुतियोजितम्}
{अधर्मस्त्रिविधस्तस्य वर्तते रागदोषजः}


\twolineshloka
{पापं चिन्तयते चैव ब्रवीति च करोति च}
{तस्याधर्मप्रवृत्तस्य गुणा नश्यन्ति साधवः}


\twolineshloka
{एकशीलाश्च मित्रत्वं भजन्ते पापकर्मिणः}
{स तेन दुःखमाप्नोति परत्र च विपद्यते}


\twolineshloka
{पापात्मा भवति ह्येवं धर्मलाभं तु मे शृणु}
{यस्त्वेतान्प्रज्ञाया दोषान्पूर्वमेवानुपश्यति}


\threelineshloka
{कुशलः सुखदुःखेषु सांधूंश्चाप्युपसेवते}
{तस्य साधुसमारम्भाद्बुद्ध्रिधर्मेषु राजते ॥ब्राह्मण उवाच}
{}


\threelineshloka
{ब्रवीषि सूनृतंधर्मं यस्य वक्ता न विद्यते}
{दिव्यप्रभावः सुमहानृषिरेव मतोसि मे ॥व्याध उवाच}
{}


\twolineshloka
{ब्राह्मणा वै महाभागाः पितरोऽग्रभुजः सदा}
{तेषां सर्वात्मना कार्यं प्रियं लोके मनीषिणा}


\twolineshloka
{यत्तेषां च प्रियं तत्ते वक्ष्यामि द्विजसत्तम}
{नमस्कृत्वा ब्राह्मणेभ्यो ब्राह्मीं विद्यां निबोध मे}


\twolineshloka
{इदं विश्वं जगत्सर्वमजगच्चापि सर्वशः}
{महाभूतात्मकं ब्रह्मन्नातः परतरं भवेत्}


\twolineshloka
{महाभूतानि खं वायुरग्निरापस्तथा च भूः}
{शब्दः स्पर्शश्च रूपं च रसो गन्धश् तद्गुणाः}


\twolineshloka
{तेषामपि गुणाः सर्वे गुणवृत्तिः परस्परम्}
{पूर्वपूर्वगुणाः सर्वे क्रमशो गुणिषु त्रिषु}


\twolineshloka
{षष्ठी तु चेतना नाम मन इत्यभिधीयते}
{सप्तमी तु भवेद्बुद्धिरहंकारस्ततः परम्}


\twolineshloka
{इन्द्रियाणि च पञ्चात्मा रजः सत्वं तमस्तथा}
{इत्येष सप्तदशको राशिरव्यक्तसंज्ञकः}


\threelineshloka
{सर्वैरिहेन्द्रियार्थैस्तु व्यक्ताव्यक्तैः सुसंवृतैः}
{चतुर्विंसक इत्येष व्यकत्वाव्यक्तमयो गुणः}
{एतत्ते सर्वमाख्यातं किं भूयः श्रोतुमिच्छसि}


\chapter{अध्यायः २१५}
\twolineshloka
{मार्कण्डेय उवाच}
{}


\threelineshloka
{एवमुक्तः स विप्रस्तु धर्मव्याधेन भारत}
{कथामकथयद्भूयो मनसः प्रीतिवर्धनीम् ॥ब्राह्मण उवाच}
{}


\threelineshloka
{महाभूतानि यान्याहुः पञ्च धर्मविदांवर}
{एकैकस्य गुणान्सम्यक्पञ्चानामपि मे वद ॥व्याध उवाच}
{}


\twolineshloka
{भूमिरापस्तथा ज्योतिर्वायुराकाशमेव च}
{गुणोत्तराणइ सर्वाणि तेषां वक्ष्यामि ते गुणान्}


\twolineshloka
{भूमिः पञ्चगुणा ब्रह्मननुदकं च चतुर्गुणम्}
{गुणास्त्रयस्तेजसि च त्रयश्चाकाशवातयोः}


\twolineshloka
{शब्दः स्पर्शश्च रूपं च रसो गन्धश्च पञ्चमः}
{एतेगुणाः पञ्च भूमेः सर्वेभ्यो गुणवत्तराः}


\twolineshloka
{शब्दः स्पर्शश्च रूपं च सरश्चापि द्विजोत्तम}
{अपामेते गुणा ब्रह्मन्कीर्तितास्तव सुव्रत}


\twolineshloka
{शब्दः स्पर्शश्च रूपं च तेजसोऽथ गुणास्त्रयः}
{शब्दः स्पर्शश्च वायौ तु शब्दश्चाकाश एव तु}


\twolineshloka
{एते पञ्चदश ब्रह्मन्गुणा भूतेषु पञ्चसु}
{वर्तन्ते सर्वभूतेषु येषु लोकाः प्रतिष्ठिताः}


\twolineshloka
{अन्योन्यं नातिवर्तन्ते सम्यक्व भवति द्विज}
{यदा तु विषमं भावमाचरन्ति चराचराः}


\twolineshloka
{तदा देही देहमन्यं व्यतिरोहति कालतः}
{प्रातिलोम्याद्विनश्यन्ति जायन्ते चानुपूर्वशः}


\twolineshloka
{तत्र तत्रहि दृश्यन्ते धातवः पाञ्चभौतिकाः}
{यैरावृतमिदं सर्वं जगत्स्थावरजङ्गमम्}


\twolineshloka
{इन्द्रियैर्गृह्यते यद्यत्तत्तद्व्यक्तमिति स्मृतम्}
{तदव्यक्तमिति ज्ञेयं लिङ्गग्राह्यमतीन्द्रियम्}


\twolineshloka
{यथास्वं ग्राहकान्येषां शब्दादीनामिमानि तु}
{इन्द्रियाणि तथा देही धारयन्निह तप्यते}


\twolineshloka
{लोके विततमात्मानं लोकं चात्मनि पश्यति}
{परापरज्ञः सक्तः सन्स तु भूतानि पश्यति}


\twolineshloka
{पश्यतः सर्वभूतानि सर्वावस्थासु सर्वदा}
{ब्रह्मभूतस् संयोगो नाशुभेनोपपद्यते}


\twolineshloka
{ज्ञानमूलात्मकं क्लेशमतिवृत्तस्य मोहजम्}
{लोकबुद्धिप्रकाशेन ज्ञेयमार्गेण गम्यते}


\twolineshloka
{अनादिनिधनं जन्तुमात्मयोनिं सदाव्ययम्}
{अनौपम्यममूर्तं च भगवानाह बुद्धिमान्}


\threelineshloka
{तपोमूलमिदं सर्वं यन्मां विप्रानुपृच्छसि}
{`तपसा हि समाप्नोति यद्यदेवाभिवाञ्छति'}
{इन्द्रियाण्येव संयम्य तपो भवति नान्यथा}


\twolineshloka
{इन्द्रियाण्येव तत्सर्वंयत्स्वर्गनरकावुभौ}
{निगृहीतविसृष्टानि स्वर्गाय नरकाय च}


\twolineshloka
{एष योगविधिः कृत्स्नो यावदिन्द्रियधारणम्}
{एतन्मूलं हि तपसः स्वर्गस्य नरकस्य च}


\twolineshloka
{इन्द्रियाणां प्रसङ्गेन दोमार्च्छत्यसंशयम्}
{सन्नियम्य तु तान्येव ततः सिद्धिं समाप्नुयात्}


\twolineshloka
{षण्णामात्प्रनि योज्यानामैश्वर्यं योऽधितिष्ठति}
{न स पापैः कुतोऽनर्थैर्युज्यते विजितेन्द्रियः}


\twolineshloka
{रथः शरीरं पुरुषस्य दृष्ट-मात्मा नियन्तेन्द्रियाण्याहुरश्वान्}
{तैरप्रमत्तः कुशली सदश्वै-र्दान्त सुखं याति रथीव धीरः}


\twolineshloka
{षण्णामात्मनियुक्तानामिन्द्रियाणां प्रमाथिनाम्}
{यो धीरो धारयेद्रश्मीन्स स्यात्परमसारथिः}


\twolineshloka
{इन्द्रियाणां प्रसृष्टानां हयानामिव वर्त्मसु}
{धृतिं कुर्वीत सारथ्ये धृत्या तानि जयेद्ध्रुवम्}


\twolineshloka
{इन्द्रियाणां विचरतां यन्मनोऽनुविधीयते}
{तदस् हरते बुद्धिं नावं वायुरिवाम्भसि}


\twolineshloka
{येषु विप्रतिपद्यन्ते षट्स्वमोहात्फलागमम्}
{तेष्वध्यवसिताध्यायी विन्दते ध्यानजं फलम्}


\chapter{अध्यायः २१६}
\twolineshloka
{मार्कण्डेय उवाच}
{}


\twolineshloka
{एवं तु सूक्ष्मे कथिते धर्मव्याधेन भारत}
{ब्राह्मणः स पुन सूक्ष्मं पप्रच्छ सुसमाहितः}


\threelineshloka
{वृत्त्वस्य रजसश्चैव तमसश्च यथातथम्}
{वृणांस्तत्त्वेन मे ब्रूहि यथावदिह पृच्छतः ॥व्याध उवाच}
{}


\twolineshloka
{दृन्त ते कथयिष्यामि यन्मां त्वं परिपृच्छसि}
{एतान्गुणान्पृथक्त्वेन निबोध गदतो मम}


\twolineshloka
{मोहात्मकं तमस्तेषां रज एषां प्रवर्तकम्}
{प्रकाशबहुलत्वाच्च सत्वं ज्याय इहोच्यते}


\twolineshloka
{अविद्याबहुलो मूढः स्वप्नशीलो विचेतनः}
{दृर्हृपीकस्तमोध्यस्तः सक्रोधस्तामसोऽलसः}


\twolineshloka
{सुवृत्तवाक्यो मन्त्री च यो नराग्र्योऽनमूयकः}
{विवित्समानो विप्रर्षे स्तब्धो मानी स राजसः}


\twolineshloka
{प्रकाशबहुलो धीरो निर्विवित्सोऽनसूयकः}
{अक्रोधनो नरो धीमान्दान्तश्चैवस सात्विकः}


\twolineshloka
{सात्विकस्त्वथ संबुद्धो लोकवृत्तैर्न लिप्यते}
{यदा बुध्यति बोध्धव्यं लोकवृत्तं जुगुप्सते}


\twolineshloka
{वैराग्यस् च रूपं तु पूर्वमेव प्रवर्तते}
{मृदुर्भवत्यहंकारः प्रसीदत्यार्जवं च यत्}


\twolineshloka
{ततोऽस् सर्वद्वन्द्वानि प्रशाम्यन्ति परस्परम्}
{न चास्यासंयमो नाम क्वचिद्भवति कश्चन}


\twolineshloka
{शूद्रयोनौ हि जातस्य सद्गुणानुपतिष्ठतः}
{वैश्यत्वं भवति ब्रह्मन्क्षत्रियत्वं तथैव च}


\threelineshloka
{आर्जवे वर्तमानस्य ब्राह्मण्यमभिजायते}
{गुणास्ते कीर्तिताः सर्वे किं भूयः श्रोतुमिच्छसि ॥ब्राह्मण उवाच}
{}


\threelineshloka
{पार्थिवं धातुमासाद्य शारीरोऽग्निः कथं भवेत्}
{अवकाशविशेषेण कथं वर्तयतेऽनिलः ॥मार्कण्डेय उवाच}
{}


\twolineshloka
{प्रश्नमेतं समुद्दिष्टं ब्राह्मणेन युधिष्ठिर}
{व्याधस्तु कथयामास ब्राह्मणाय महात्मने}


\twolineshloka
{मूर्धानमाश्रितो वह्निः शरीरं परिपालयन्}
{प्राणो मूर्धनि चाग्नौ च वर्तमानो विचेष्टते}


\twolineshloka
{भूं भव्यं भविष्यं च सर्वं प्राणे प्रतिष्ठितम्}
{श्रेष्ठं तदेव भूतानां ब्रह्मज्योतिरुपास्महे}


\twolineshloka
{स जन्तुः सर्वभूतात्मा पुरुषः स सनातनः}
{मनोबुद्धिरहंकारो भूतानां विषयश्च सः}


\twolineshloka
{`अव्यक्तं सस्वसंज्ञं च जीवः कालः स चैव हि}
{प्रकृतिः पुरुषश्चैव प्राण एव द्विजोत्तम}


\twolineshloka
{जागर्ति स्वप्नकाले च स्वप्ने स्वप्नायते च सः}
{जाग्रत्सु बलमाधत्ते चेष्टत्सु चेष्टयत्यपि}


\twolineshloka
{तस्मिन्निरुद्दे विप्रेन्द्र मृत इत्यभिधीयते}
{त्यक्त्वा शरीरं भूतात्मा पुनरन्यत्प्रपद्यते}


\twolineshloka
{एष त्वग्निरपानन प्राणेन परिपाल्यते}
{पृष्ठतस्तु समानन स्वांस्वां गतिमुपाश्रितः}


\twolineshloka
{वस्तुमूले गुदे चैव पावकं समुपाश्रितः}
{वहन्मूत्रं पुरीषंवाऽप्यपानः परिवर्तते}


\twolineshloka
{प्रयत्ने कर्मणि बले य एकस्त्रिषु वर्तते}
{उदान इति तं प्राहुरध्यात्मविदुषो जनाः}


\twolineshloka
{सन्धौसन्धौ संनिविष्टः सर्वेष्वपि तथाऽनिलः}
{शरीरेषु मनुष्याणां व्यान इत्युपदिश्ते}


\twolineshloka
{धातुष्वग्निस्तु विततः स तु वायुसमीरितः}
{रसान्धातूंश्च दोषांश्च वर्तयन्परिधावति}


\twolineshloka
{प्राणानां संनिपातात्तु सन्निपातः प्रजायते}
{सोष्मा सोग्निरितिज्ञेयो योऽन्नं पचतिदेहिनां}


\twolineshloka
{अपानोदानयोर्मध्ये प्राणन्यानौ समाहितौ}
{समन्वितस्त्वधिष्ठानं सम्यक्पचति पावकः}


\twolineshloka
{अस्यापि पायुपर्यन्तस्तथा स्याद्गुदसंज्ञितः}
{स्रोतांशि तस्माज्जायन्ते सर्वप्राणेषु देहिनाम्}


\twolineshloka
{अग्निवेगवहः प्राणो गुदान्ते प्रतिहन्यते}
{स ऊर्ध्वमागम्य पुनः समुत्क्षिपति पावकम्}


\twolineshloka
{पक्वाशयस्त्वधोनाभ्या ऊर्ध्वमामाशयः स्थितः}
{नाभिमध्ये शरीरस्य प्राणाः सर्वे प्रतिष्ठिताः}


\twolineshloka
{प्रवृत्ता हृदयात्सर्वे तिर्यगूर्ध्वमधस्तथा}
{वहन्त्यन्नरसान्नाड्यो दशप्राणप्रचोदिताः}


\twolineshloka
{योगिनामेष मार्गस्तु येन गच्छन्ति तत्परम्}
{जितक्लमासनो धीरो मूर्धन्यात्मानमादत्}


\twolineshloka
{एवं सर्वेषु विततौ प्राणापानौ हि देहिषु}
{`तौ तावदग्निसहितौ विद्धि वै प्राणमात्मनि'}


\threelineshloka
{एकादशविकारात्मा कलासंभारसंभृतः}
{मूर्तिमन्तं हि तं विद्धि नित्यं कर्माजेतात्मकम्}
{तस्मिन्यः संस्थितो ह्यग्निर्नित्यं स्थाल्यामिवाहितः}


% Check verse!
आत्मानं तं विजानीहि नित्यं त्यागजितात्मकं
\twolineshloka
{देवो यः संस्थितस्तस्मिन्नब्बिन्दुरिव पुष्करे}
{क्षेत्रज्ञं तं विजानीहि नित्यं त्यागजितात्मकं}


\twolineshloka
{जीवात्मकं विजानीहि रजः सत्वं तमस्तथा}
{जीवमात्मगुणं विद्धि तथाऽऽत्मानं परात्मकं}


\twolineshloka
{अचेतनं जीवगुणं वदन्तिसचेष्टते चेष्टयते च सर्वम्}
{ततः परं क्षेत्रविदो वदन्तिप्राकल्पयद्यो भुवनानि सप्त}


\twolineshloka
{एष सर्वेषु भूतेषु भूतात्मा न प्रकाशते}
{दृश्यते त्वग्र्यया बुद्ध्या सूक्ष्मया ज्ञानवेदिभिः}


\twolineshloka
{चित्तस्य हि प्रसादेन हनति कर्म शुभाशुभम्}
{प्रसन्नात्मात्मनि स्थित्वा सुखमनन्त्यमश्नुते}


\threelineshloka
{लक्षणं तु प्रसादस्य यथा तृप्तः सुखं स्वपेत्}
{`सुखदुःखे हि संत्यज्य निर्द्वन्द्वो निष्परिग्रहः'}
{निवाते वा यथा दीपो दीप्येत्कुशलदीपितः}


\twolineshloka
{पूर्वरात्रेऽपरे चैव युञ्जानः सततं मनः}
{लध्वाहारो विशुद्धात्मा पश्यत्यात्मानमात्मनि}


\twolineshloka
{प्रदीप्तेनेव दीपेन मनोदीपेन पश्यति}
{दृष्ट्वाऽऽत्मानं निरात्मानं स तदा विप्रमुच्यते}


\twolineshloka
{सर्वोपायैस्तु लोभस् क्रोधस्य च विनिग्रहः}
{एतत्पवित्रं यञ्ज्ञानं तपो वै संक्रमो मतः}


\twolineshloka
{नित्यं क्रोधात्तपो रक्षेन्छ्रियं रक्षेच्च मत्सरात्}
{विद्यां मानापमानाभ्यामात्मानं तु प्रमादतः}


\twolineshloka
{आनृशंस्यं परो धर्मः क्षमा च परमं बलम्}
{आत्मज्ञानं परं ज्ञानं परं सत्यव्रतव्रतम्}


\twolineshloka
{सत्यस् वचनं श्रेयः सत्यं ज्ञानं हितं भवेत्}
{यद्बूतहितमत्यन्तं तद्वै सत्यं परं मतम्}


\twolineshloka
{यस् सर्वे समारम्भा निराशीर्बन्धनाः सदा}
{त्यागे यस् हुतं सर्वं स त्यागी स च बुद्धिमान्}


\twolineshloka
{यदा न गुरुतां चैनं च्यावयेदुपपादयन्}
{तं विद्याद्ब्राह्मणो योगमयोगं योगसंज्ञितम्}


\twolineshloka
{न हिंस्यात्सर्वभूतानि मैत्रायणगतश्चरेत्}
{नेदं जीवितमासाद्यवैरं कुर्वीत केनचित्}


\twolineshloka
{आकिंचन्यं सुसंतोषो निराशित्वमचापलम्}
{एतदेव परं ज्ञानं सदात्मज्ञानमुत्तमम्}


\twolineshloka
{परिग्रहं परित्यज्य भवेद्बुद्ध्या यतव्रतः}
{अशोकं स्थानमाश्रित् निश्चलं प्रेत्य चेह च}


\twolineshloka
{तपोनित्येन दान्तेन मुनिना संयतात्मना}
{अजितं जेतुकामेन भाव्यं सङ्गेष्वसङ्गिना}


\twolineshloka
{गुणागुणमनासङ्गमेककार्यमनन्तरम्}
{एतत्तद्ब्रह्मणो वृत्तमाहुरेकपदं सुखम्}


\twolineshloka
{परित्यजति यो दुःखं सुखं चाप्युभयं नरः}
{ब्रह्म प्राप्नोति सोत्यन्तमासङ्गं च न गच्छति}


\twolineshloka
{यथाश्रुतमिदं सर्वं समासेन द्विजोत्तम}
{एतत्ते सर्वमाख्यातं किं भूयः श्रोतुमिच्छसि}


\chapter{अध्यायः २१७}
\twolineshloka
{मार्कण्डेय उवाच}
{}


\twolineshloka
{एवं संकथिते कृत्स्ने मोक्षधर्मे युधिष्ठिर}
{दृढप्रीतमना विप्रो धर्मव्याधमुवाच ह}


\threelineshloka
{न्याययुक्तमिदं सर्वं भवता परिकीर्तितम्}
{न तेऽस्त्यविदितं किंचिद्धर्मेष्वभिसमीक्ष्यते ॥व्याध उवाच}
{}


\twolineshloka
{प्रत्यक्षं मम यो धर्मस्तं च पश्य द्विजोत्तम}
{येन सिद्धिरियं प्राप्ता मया ब्राह्मणपुङ्गव}


\twolineshloka
{उत्तिष्ठ भगवन्क्षिप्रं प्रविश्याभ्यन्तरं गृहम्}
{द्रष्टुमर्हसि धर्मज्ञ मातरं पितरं च मे}


\twolineshloka
{इत्युक्तः स प्रविश्याथ ददर्श परमार्चितम्}
{सौधं हृद्यं चतुःशालमतीव च मनोरमम्}


\twolineshloka
{देवतागृहसंकाश दैवतैश्च सुपूजितम्}
{शयनासनसंबाधं गन्धैश्च परमैर्युतम्}


\twolineshloka
{तत्रशुक्लाम्बरधरौ पितरावस्य पूजितौ}
{कृताहारौ तु संतुष्टावुपविष्टौ वरासने}


\threelineshloka
{`तस्य व्याधस्य पितरौ ब्राह्मणः संददर्श ह'}
{धर्मव्याधस्तु तौ दृष्ट्वा पादेषु शिरसाऽपतत् ॥वृद्धावूचतुः}
{}


\threelineshloka
{उत्तिष्ठोत्तिष्ठ धर्मज्ञ धर्मस्त्वामभिरक्षतु}
{प्रीतौ स्वस्तव शौचेन दीर्घमायुरवाप्नुहि}
{[गतिमिष्टां तपो ज्ञानं मेधां च परमां गतः]}


\twolineshloka
{सत्पुत्रेण त्वया पुत्र नित्यं काले सुपूजितौ}
{`सुखमेव वसावोऽत्र देवलोकगताविव'}


\twolineshloka
{न तेऽन्यद्दैवतं किंचिद्दैवतेष्वपि वर्तते}
{प्रयतसत्वाद्द्विजातीनांदमेनासि समनवितः}


\twolineshloka
{पितुः पितामहा ये च तथैव प्रपितामहाः}
{प्रीतास्ते सततं पुत्र दमेनावां च पूजया}


\twolineshloka
{मनसा कर्मणा वाचा शुश्रूषा नैव हीयते}
{न चान्या हि तथा बुद्धिर्दृश्यते सांप्रतं तव}


\threelineshloka
{जामदग्न्येन रामेण यथा वृद्धौ सुपूजितौ}
{तथा त्वया कृतंसर्वंतद्विशिष्टं च पुत्रक ॥मार्कण्डेय उवाच}
{}


\twolineshloka
{ततस्तं ब्राह्मणं ताभ्यां धर्मव्याधो न्यवेदयत्}
{तौ स्वागतेन तं विप्रमर्चयामासतुस्तदा}


\fourlineindentedshloka
{प्रतिगृह्यच तां पूजां द्विजः पप्रच्छ तावुभौ}
{सपुत्राभ्यां सभृत्याभ्यां कच्चिद्वां कुशलं गृहे}
{अनामय च वां कच्चित्सुखं वेह शरीरयोः ॥वृद्धावूचतुः}
{}


\threelineshloka
{कुशलं नौ गृहे विप्र भृत्यवर्गे च सर्वशः}
{कच्चित्त्वमप्यविघ्नेन संप्राप्तो भगवन्निति ॥मार्कण्डेय उवाच}
{}


\twolineshloka
{हाढमित्येव तौ विप्रः प्रत्युवाच मुदान्वितः}
{धर्मव्याधस्तु तं विप्रमर्थवद्वाक्यमब्रवीत्}


\twolineshloka
{पिता माता च भगवन्नेतौ मे दैवतं परम्}
{यद्दैवतेभ्यः कर्तव्यं तदेताभ्यां करोम्यहम्}


\twolineshloka
{त्रयस्त्रिंशद्यथा देवाः सर्वे शक्रपुरोगमाः}
{संपूज्याः सर्वलोकस्य तथा वृद्धाविमौ मम}


\twolineshloka
{उपाहारानाहरन्तो देवतानां यथा द्विजः}
{कुर्वन्ति तद्वदेताभ्यां करोम्यहमतन्द्रितः}


\twolineshloka
{एतौ मे परमं ब्रह्मन्पिता माता च दैवतम्}
{एतौ पुष्पैः फलैरन्नैस्तोषयामि सदा द्विज}


\twolineshloka
{एतावेवाग्नयो मह्यं यान्वदन्ति मनीषिणः}
{यज्ञा वेदाश्च चत्वारः सर्वमेतौ मम द्विज}


\twolineshloka
{एतदर्थं मम प्राणा भार्या पुत्रः सुहृज्जनः}
{सपुत्रदारः शुश्रूषां नित्यमेव करोम्यहम्}


\twolineshloka
{स्वयं च स्नापयाम्येतौ तथा पादौ प्रधावये}
{आहारं च प्रयच्छामि स्वयंच द्विजसत्तम}


\twolineshloka
{अनुकूलाः कथा वच्मि विप्रियं परिवर्जये}
{अधर्मेणापि संयुक्तं प्रियमाभ्यां करोम्यहम्}


\twolineshloka
{धर्ममेव गुरुं मत्वा साक्षादेतौ द्विजोत्तम}
{अतन्द्रितः सदा विप्र शुश्रूषां वै करोम्यहम्}


\twolineshloka
{पञ्चैव गुरवो ब्रह्मनपुरुषस्य बुभूषतः}
{पिता माताऽग्निरात्मा च गुरुश्च द्विजसत्तम}


\threelineshloka
{एतेषु यस्तु वर्तेत सम्यगेव द्विजोत्तम}
{भवेयुरप्रयस्तेन परिचीर्णास्तु नित्यशः}
{गार्हस्थ्ये वर्तमानस्य एष धर्मः सनातनः}


\chapter{अध्यायः २१८}
\twolineshloka
{मार्कण्डेय उवाच}
{}


\twolineshloka
{गुरू निवेद्य विप्राय तौ मातापितरावुभौ}
{पुनरेव स धर्मात्मा व्याधो ब्राह्मणमब्रवीत्}


\twolineshloka
{प्रवृत्तचक्षुर्जातोस्मि संपश्य तपसो बलम्}
{यदर्थमुक्तोसि तया गच्छ त्वं मिथिलामिति}


\threelineshloka
{पतिशुश्रूषपरया दान्तया सत्यशीलया}
{मिथिलायां वसन्व्याधः स ते धर्मान्प्रवक्ष्यति ॥ब्राह्मण उवाच}
{}


\threelineshloka
{पतिव्रतायाः सत्यायाः शीलाढ्याया यतव्रत}
{संस्मृत्य वाक्यं धर्मज्ञ गुणवानसि मे मतः ॥व्याध उवाच}
{}


\twolineshloka
{यत्तया त्वं द्विजश्रेष्ठ नियुक्तो मां प्रति प्रभो}
{दृष्टमेव तया सम्यगेकपत्न्या न संशयः}


\twolineshloka
{त्वदनुग्रहबुद्ध्या तु विप्रैतद्दर्शितं मया}
{वाक्यं च शृणु मे तात यत्ते वक्ष्ये हितं द्विज}


\twolineshloka
{त्वया न पूजिता माता पिता च द्विजसत्तम}
{अनिसृष्टोसि निष्क्रान्तो गृहात्ताभ्यामनिनदित}


\twolineshloka
{वेदोच्चारणकार्यार्थमयुक्तं तत्त्वया कृतम्}
{तव शोकेन वृद्धौ तावन्धीभूतौ तपस्विनौ}


\twolineshloka
{तौ प्रसादयितुं गच्छ मा त्वां धर्मोऽत्यगादयम्}
{तपस्वी त्वं महात्मा च धर्मे च निरतः सदा}


\twolineshloka
{सर्वमेतदपार्थं ते क्षिप्रं तौ संप्रसादय}
{`तौ प्रसाद्य द्विजश्रेष्ठ यच्छ्रेयस्तदवाप्स्यसि'}


\threelineshloka
{श्रद्दधस्व मम ब्रह्मन्नान्यथा कर्तुमर्हसि}
{यम्यतामद्यविप्रर्षे श्रेयस्ते कथयाम्यहम् ॥ब्राह्मण उवाच}
{}


\threelineshloka
{यदेतदुक्तं भवता सर्वं सत्यमसंशयम्}
{प्रीतोस्मि तव भद्रं ते धर्माचारगुणान्वित ॥व्याध उवाच}
{}


\twolineshloka
{दैवतप्रतिमो हि त्वं यस्त्वं धर्ममनुव्रतः}
{पुराणं शाश्वतं दिव्यं दुष्प्रापमकृतात्मभिः}


\fourlineindentedshloka
{मातापित्रोः सकाशं हि गत्वात्वं द्विजसत्तम}
{अतन्द्रितः कुरु क्षिप्रंमातापित्रोर्हि पूजनम्}
{अतः परमहं धर्मं नान्यं पश्यामि कंचन ॥ब्राह्मण उवाच}
{}


\twolineshloka
{इहाहमागतो दिष्ट्या दिष्ट्या मे संगतं त्वया}
{ईदृशा दुर्लभा लोके नरा धर्मप्रदर्शकाः}


\twolineshloka
{एकोनरसहस्रेषु धर्मवानविद्यते न वा}
{प्रीतोस्मि तवसत्येन भद्रं ते पुरुषर्षभ}


\twolineshloka
{पतमानोऽद्यनरके भवताऽस्मि समुद्धृतः}
{भवितव्यमथैवं च यद्दृष्टोसि मयाऽनघ}


\twolineshloka
{राजा ययातिर्दौहित्रैः पतितस्तारितो यथा}
{सद्भिः पुरुषशार्दूल तथाऽहं भवता त्विह}


\twolineshloka
{मातापितृभ्यां शुश्रूषां करिष्ये वचनात्तव}
{नाकृतात्मा वेदयति धर्माधर्मविनिश्चयम्}


\twolineshloka
{दुर्ज्ञेयः शाश्वतो धर्मः शूद्रयोनौ हि वर्तता}
{न त्वां शूद्रमहं मन्ये भवितव्यं हि कारणम्}


\fourlineindentedshloka
{येन कर्मविशेषेण प्राप्तेयं शूद्रता त्वया}
{एतामिच्छामि विज्ञातुं तत्त्वेन तव शूद्रताम्}
{कामयानस् मे शंस सर्वं त्वं प्रयतात्मवान् ॥व्याघ उवाच}
{}


\twolineshloka
{अनतिक्रमणीया वै ब्राह्मणा मे द्विजोत्तम}
{शृणु सर्वमिदं वृत्तं पूर्वदेहे ममानघ}


\threelineshloka
{अहं हि ब्राह्मणः पूर्वमासं द्विजवरात्मजः}
{वेदाध्यायी सुकुशलो वेदाङ्गानां च पारगः}
{आत्मदोषकृतैर्ब्रह्मन्नवस्थामाप्तवानिमाम्}


\twolineshloka
{कश्चिद्राजा मम सखा धनुर्वेदपरायणः}
{संसर्गाद्धनुषि श्रेष्ठस्ततोऽहमभवं द्विज}


\threelineshloka
{एतस्मिन्नेव काले तु मृगयां निर्गतो नृपः}
{सहितो योधमुख्यैश् मन्त्रिभिश्च सुसंवृतः}
{ततोऽभ्यहन्मृगांस्तत्र सुबहूनाश्रमं प्रति}


\twolineshloka
{अथ क्षिप्तः शरो घोरो मयापि द्विजसत्तम}
{ताडितश्च ऋषिस्तन शरेणानतपर्वणा}


\twolineshloka
{भूमौ निपतितो ब्रह्मन्नुवाच प्रतिनादयन्}
{नापराध्याम्यहं किंचित्केन पापमिदं कृतम्}


\threelineshloka
{मन्वानस्तं मृगं चाहं संप्राप्तः सहसा मुनिम्}
{अपश्यं तमृषिं विद्धं शरेणानतपर्वणा}
{तमुग्रतपसं विप्रं निष्टनन्तं महीतले}


\twolineshloka
{अकार्यकरणाच्चापि भृशं मे व्यथितं मनः}
{अजानता कृतमिदं मयेत्यहमथाब्रुवम्}


% Check verse!
क्षन्तुमर्हसि मे सर्वमिति चोक्तो मया मुनिः
\twolineshloka
{ततः प्रत्यब्रवीद्वाक्यमृषिर्मां क्रोधमूर्च्छितः}
{व्याधस्त्वं भविता क्रूर शूद्रयोनाविति द्विज}


\chapter{अध्यायः २१९}
\twolineshloka
{व्याध उवाच}
{}


\twolineshloka
{एवं शप्तोऽहमृषिणा तदा द्विजबरोत्तम}
{अहं प्रासादयमृषिं गिरा वाक्यविशारदम्}


\threelineshloka
{अजानता मयाऽकार्यमिदमद्य कृतं मुने}
{क्षन्तुमर्हसि तत्सर्वं प्रसीद भगवन्निति ॥ऋषिरुवाच}
{}


\twolineshloka
{नान्यथा भविता शाप एवमेतदसंशयम्}
{आनृशंस्यात्त्वंहकिंचित्कर्ताऽनुग्रहमद्य ते}


\twolineshloka
{शूद्रयोन्यां वर्तमानो धर्मज्ञो हि भविष्यसि}
{मातापित्रोश् शुश्रूषां करिष्यसि न संशयः}


\twolineshloka
{तयोः शुश्रूषया सिद्धिं महतीं समवाप्स्यसि}
{जातिस्मरश् भविता स्वर्गं चैव गमिष्यसि}


\twolineshloka
{`भूत्वाच धार्मिको व्याधः पित्रोः शुश्रूषणे रतः}
{शापक्षये तु निर्वृत्ते भविताऽसि पुनर्द्विजः}


\twolineshloka
{एवं शप्तः पुरा तेन ऋषिणाऽस्म्युग्रतेजसा}
{प्रसादश्च कृतस्तेन ममैव द्विपदांवर}


\twolineshloka
{शरं चोद्धृतवानस्मि तस्य वै द्विजसत्तम}
{आश्रमं च मया नीतो न च प्राणैर्व्ययुज्यत}


\threelineshloka
{एतत्ते सर्वमाख्यातं यथा मम पुराऽभवत्}
{अभितश्चापि गन्तव्यो मया स्वर्गो द्विजोत्तम ॥ब्राह्मण उवाच}
{}


\twolineshloka
{एवमेतानि पुरुषा दुःखानि च सुखानि च}
{आप्नुवन्ति महाबुद्धे नोत्कण्ठां कर्तुमर्हसि}


\twolineshloka
{दुष्करं हि कृतंकर्म जानता जातिमात्मनः}
{[लोकवृत्तान्ततत्त्वज्ञ नित्यं धर्मपरायण ॥]}


\twolineshloka
{कर्मदोषाच्च वै विद्वन्नात्मजातिकृतेन वै}
{कंचित्कालं मृष्यतां वै ततोसि भविता द्विजः}


% Check verse!
सांप्रतं च मतो मेऽसि ब्राह्मणो नात्र संशयः
\twolineshloka
{ब्राह्मणः पतनीयेषु वर्तमानो विकर्मसु}
{दाम्भिक्वो दुष्कृतप्रायः शूद्रेण सदृशो भवेत्}


\twolineshloka
{यस्तु शूद्रो दमे सत्ये धर्मे च सततोत्थितः}
{तं ब्राह्मणमहं मन्ये वृत्तेन हि भवेद्द्विजः}


\twolineshloka
{कर्मदोषेण विषमां गतिमाप्तोसि दारुणाम्}
{क्षीणदोषमहं मन्ये चाभितस्त्वां नरोत्तम}


\threelineshloka
{कर्तुमर्हसि नोत्कण्ठां त्वद्विधा ह्यविषादिनः}
{लोकवृत्तान्ततत्वज्ञा नित्यं धर्मपरायणाः ॥व्याध उवाच}
{}


\twolineshloka
{प्रज्ञया मानसं दुःखं हन्याच्छारीरमौषधैः}
{एतद्विज्ञानसामर्थ्यं न बालै समतामियात्}


\twolineshloka
{अनिष्टसंप्रयोगाच्च विप्रयोगात्प्रयिस्य च}
{मनुष्या मानसैर्दुःखैर्युज्यन्ते चाल्पबुद्धयः}


\twolineshloka
{गुणैर्भूतानि युज्यन्ते वियुज्यन्ते तथैव च}
{सर्वाणि नैतदेकस्य शोकस्थानं हि विद्यते}


\twolineshloka
{अनिष्टनान्वितं पश्यंस्तथा क्षिप्रं विरज्यते}
{ततश्च प्रतिकुर्वन्ति यदि पश्यन्त्युपक्रमम्}


\twolineshloka
{शोचतो न भवेत्किंचित्केवलं परितप्यते}
{परित्यजन्ति ये दुःखं सुखं वाऽप्युभयं नराः}


\twolineshloka
{त एव सुखमेधन्ते ज्ञानतृप्ता मनीषिणः}
{असंतोषपरा मूढाः संतोषं यानति पण्डिताः}


\twolineshloka
{असंतोषस्य नास्त्यन्तस्तुष्टिस्तु परमं सुखम्}
{न शोचन्ति गताध्वानः पश्यन्तः परमां गतिं}


\twolineshloka
{न विषादे मनः कार्यं विषादो विषमुत्तमम्}
{मारयत्यकृतप्रज्ञं बालं क्रुद्ध इवोरगः}


\twolineshloka
{यंविषादोऽभिभवति विषमे समुपस्थिते}
{तेजसा तस् हीनस्य पुरुषार्थो न विद्यते}


\twolineshloka
{अवश्यं क्रियमाणस् कर्मणो दृश्यते फलम्}
{न हि निर्वेदमागम्य किंचित्प्राप्नोति शोभनम्}


\twolineshloka
{अथाप्युपायं पश्येत दुःखस् परिमोक्षणे}
{अशोचन्नारभेतैव युक्तश्चाव्यसनी भवेत्}


\twolineshloka
{भूतेष्वभावं संचिन्त्य ये तु बुद्धेः परं गताः}
{न शोचन्ति कृतप्रज्ञाः पश्यन्तः परमां गतिम्}


\threelineshloka
{न शोचामि रमे विद्वन्कालाकाङ्क्षी स्थितोस्म्यहम्}
{एतैर्निदर्शनैर्ब्रह्मन्नावसीदामि सत्तम ॥ब्राह्मण उवाच}
{}


\twolineshloka
{कृतप्रज्ञोसि मेधावी बुद्धिश्च विपुला तव}
{पापान्निवृत्तोसि सदा ज्ञानवृद्धोसि धर्मवित्}


\threelineshloka
{आपृच्छे त्वां स्वस्ति तेऽस्तु धर्मस्त्वां परिरक्षतु}
{अप्रमादस्तु कर्तव्यो धर्मे धर्मभृतांवर ॥मार्कण्डेय उवाच}
{}


\twolineshloka
{बाढमित्येव तं व्याधः कृताञ्जलिरुवाच ह}
{प्रदक्षिणमथो कृत्वा प्रस्थितो द्विजसत्तमः}


\twolineshloka
{स तु गत्वाद्विजः सर्वां शुश्रूषां कृतवांस्तदा}
{मातापितृभ्यामन्धाभ्यां यथान्यायं सुसंशितः}


\twolineshloka
{एतत्ते सर्वमाख्यातं निखिलेन युधिष्ठिर}
{पृष्टवानसि यं तात धर्मं धर्मभृतांवर}


\threelineshloka
{पतिव्रताया माहात्म्यं ब्राह्मणस्य च सत्तम}
{मातापित्रोश्च शुश्रूषा धर्मव्याधेन कीर्तिता ॥युधिष्ठिर उवाच}
{}


\twolineshloka
{अत्यद्भुतमिदं ब्रह्मन्धर्माख्यानमनुत्तमम्}
{सर्वधर्मविदांश्रेष्ठ कथितं मुनिसत्तम}


\twolineshloka
{सुखश्राव्यतया विद्वन्मुहूर्त इव मे गतः}
{न हि तृप्तोस्मि भगवञ्शृण्वानो धर्ममुत्तमम्}


\chapter{अध्यायः २२०}
\twolineshloka
{वैशंपायन उवाच}
{}


\twolineshloka
{श्रुत्वेमां धर्मसंयुक्तां धर्मराजः कथां शुभाम्}
{पुनः पप्रच्छ तमृषिं मार्कण्डेयमिदं तदा}


\twolineshloka
{कथमग्निर्वनं यातः कथं चाप्यङ्गिराः पुरा}
{नष्टेऽग्नौ हव्यमहवदग्निर्भूत्वा महाद्युतिः}


\twolineshloka
{अग्निर्यदा चैक एव बहुत्वं चास्य कर्मसु}
{दृश्यते भगवन्सर्वमेतदिच्छामि वेदितुम्}


\twolineshloka
{कुमारश्च यथोत्पन्नो यथा चाग्नेः सुतोऽभवत्}
{यथा रुद्राच्च संभूतो गङ्गायां कृत्तिकासु च}


\threelineshloka
{एतदिच्छाम्यहं त्वत्तः श्रोतुं भार्गवसत्तम}
{कौतूहलसमाविष्टो याथातथ्यं महामुने ॥मार्कण्डेय उवाच}
{}


\twolineshloka
{अत्राप्युदाहरन्तीममितिहासं पुरातनम्}
{यथा क्रुद्धो हुतवहस्तपस्तप्तुं वनं गतः}


\twolineshloka
{यथा च भगवानग्निः स्वयमेवाङ्गेराऽभवत्}
{संतापयंश्च प्रभया नाशयंस्तिमिरावलिम्}


\threelineshloka
{पुराङ्गिरा महाबाहो चचार तप उत्तमम्}
{आश्रमस्थो महाभागो हव्यवाहं विशेषयन्}
{यथाग्निर्भूत्वा तु तदा जगत्सर्वं व्यकाशयत्}


\twolineshloka
{तपश्चरंस्तु हुतभुक्संतप्तस्तस्य तेजसा}
{भृशं ग्लानश्चतेजस्वी न च किंचित्प्रजज्ञिवान्}


\twolineshloka
{अथ संचिन्तयामास भगवान्हव्यवाहनः}
{अन्योऽग्निरिव लोकानां ब्रह्मणा संप्रकल्पितः}


\twolineshloka
{अग्नित्वं विप्रनष्टं हि तप्यमानस् मे तपः}
{कथमग्निः पुनरहं भवेयमिति चिन्त्य सः}


\twolineshloka
{अपश्यदग्निवल्लोकांस्तापयन्तं महामुनिम्}
{सोपासर्पच्छनैर्भीतस्तमुवाच तदाङ्गिराः}


\twolineshloka
{शीघ्रमेव भवस्वाग्निस्त्वं पुनर्लोकभावनः}
{विज्ञातश्चासि लोकेषु त्रिषु संस्थानचारिषु}


\threelineshloka
{त्वमग्ने प्रथमः सृष्टो ब्र्हमणा तिमिरापर्हः}
{स्वस्थानं प्रतिपद्यस्व शीघ्रमेव तमो नुद ॥अग्निरुवाच}
{}


\twolineshloka
{नष्टकीर्तिरहं लोके भवाञ्जातो हुताशनः}
{भवन्तमेव ज्ञास्यन्ति पावकं न तु मां जनाः}


\threelineshloka
{निक्षिपम्यहमग्नित्वं त्वमग्निः प्रथमो भव}
{भविष्यामि द्वितीयोऽहं प्राजापत्यक एव च ॥अङ्गिरा उवाच}
{}


\threelineshloka
{कुरु पुण्यं प्रजास्वर्ग्यं भवाग्निस्तिमिरापहः}
{मां च देव कुरुष्वाग्ने प्रथमं पुत्रमञ्जसा ॥मार्कण्डेय उवाच}
{}


\twolineshloka
{तच्छ्रुत्वाङ्गिरसो वाक्यं जातवेदास्तथाऽकरोत्}
{राजन्वृहस्पतिर्नाम तस्याप्यङ्गिरसः सुतः}


\twolineshloka
{ज्ञात्वा प्रथमजं तं तु वह्नेरङ्गिरसं सुतम्}
{उपेत्य देवाः पप्रच्छुः कारणं तत्र भारत}


\twolineshloka
{स तु पृष्टस्तदा देवैस्तत कारणमब्रवीत्}
{प्रत्यगृह्णन्त देवाश्च तद्वचोऽङ्गिरसस्तदा}


\twolineshloka
{तत्रनानाविधानग्नीन्प्रवक्ष्यामि महाप्रभान्}
{कर्मभिर्वहुभिः ख्याताननानार्थान्ब्राह्मणेष्विह}


\chapter{अध्यायः २२१}
\twolineshloka
{मार्कण्डेय उवाच}
{}


\twolineshloka
{ब्रह्मणो यस्तृतीयस्तु पुत्रः कुरुकुलोद्वह}
{तस्यापि वसुदाभार्या प्रजास्तस्यां च मे शृणु}


\twolineshloka
{बृहत्कीर्तिर्बृहज्ज्योतिर्बृहद्ब्रह्मा बृहन्मनाः}
{बृहन्मन्त्रो बृहद्भासस्तथा राजन्बृहस्पतिः}


\twolineshloka
{प्रजासु तासु सर्वासु रूपेणाप्रतिमाऽभवत्}
{देवी भानुमती नाम प्रथमाऽङ्गिरसः सुता}


\twolineshloka
{भूतानामेव सर्वेषां यस्यां रागस्तदाऽभवत्}
{रागाद्रागेति यामाहुर्द्वितीयाऽङ्गिरसः सुता}


\twolineshloka
{यां कपर्दिसुतामाहुर्दृश्यादृश्येति देहिनः}
{तनुत्वात्सा सिनीवाली तुतीयाऽङ्गिरसः सुता}


\twolineshloka
{यां तु दृष्ट्वा भगवतीं जनः कुहकुहायते}
{एकानकेति यामाहुश्चतुर्थ्यङ्गिरसः सुता}


\twolineshloka
{पञ्चम्यर्चिष्मती नाम्ना हविर्भिश्च हविष्मती}
{षष्ठीम्गिरसः कन्यां पुण्यामाहुर्महिष्मतीम्}


\twolineshloka
{महामखेष्वाङ्गिरसी दीप्तिमत्सु महीयती}
{महामतीति विख्याता सप्तमी कथ्यते सुता}


\twolineshloka
{बृहस्पतेश्चान्द्रमसी भार्याऽऽसीद्या यशस्विनी}
{अग्नीन्साऽजनयत्पुण्यान्षडेकां चापि पुत्रिकाम्}


\twolineshloka
{आहुतिष्वेव यस्याग्नेर्हविराज्यं विधीयते}
{सोग्निर्बृहस्पतेः पुत्र शंयुर्नाम महाप्रभः}


\twolineshloka
{चातुर्मास्येषु यस्येष्टमश्वमेधाग्रभागभूत्}
{दीप्तिज्वालैरनेकाग्रैरग्निष्टोमोऽथ वीर्यवान्}


\twolineshloka
{शंयोरप्रतिमा भार्या सत्या सत्याऽथ धर्मजा}
{अग्नितस्य सुतो दीप्तस्तिस्रः कन्याश्च सुव्रताः}


\twolineshloka
{प्रथमेनाज्यभागेन पूज्यते योऽग्निरध्वरे}
{अग्निस्तस्य भरद्वाजः प्रथमः पुत्र उच्यते}


\twolineshloka
{पौर्णमासेषु सर्वेषु हविराज्यं ध्रुवोद्यतम्}
{भरतो नामतः सोग्निर्द्वितीयः शंयुतः सुतः}


\twolineshloka
{तिस्रः कन्या भवन्त्यन्या यासां स भरत पतिः}
{भारतस्तु सुतस्तस्य भारत्येका च पुत्रिका}


\twolineshloka
{भारतो भरतस्याग्नेः पावकस्तु प्रजायते}
{महानत्यर्थमहितस्तथा भरतसत्तम}


\twolineshloka
{भरद्वाजस्य भार्या तु वीरा वीरश्च पिण्डदः}
{प्राहुराज्येन तस्येज्यां सोमस्येव द्विजाः शनैः}


\twolineshloka
{हविषा यो द्वितीयेन सोमेन सह युज्यते}
{रथप्रभू रथध्वानः कुम्भरेताः स उच्यते}


\twolineshloka
{सरय्वां जनयत्सिद्धिं भानुं भाभिः समावृणोत्}
{आग्नेयमनयन्मिथ्यं मिथ्यो नामैव कथ्यते}


\twolineshloka
{यस्तु न च्यवते नित्यं शयसा वर्चसा श्रिया}
{अग्निर्निश्च्यवनो नाम पृथिवीं स्तौति केवलम्}


\twolineshloka
{विपाप्मा कलुषैर्मुक्तो विशुद्धश्चार्चिषा ज्वलन्}
{विपापोऽग्निः सुतस्तस्य योज्यः समयकर्मसु}


\twolineshloka
{अक्रोशतां हि भूतानां यः करोति हि निष्कृतिम्}
{अग्निः स निष्कृतिर्नाम शोभयत्यभिसेवितः}


\twolineshloka
{अनुकूजन्ति येनेह वेदनार्ताः स्वयं जनाः}
{तस्य पुत्रः स्वनो नाम पावकः सरुजस्करः}


\twolineshloka
{यस्तु विश्वस् जगतो बुद्धिमाक्रम्य तिष्ठति}
{तं प्राहुरध्यात्मविदो विश्वजिन्नाम पावकम्}


\twolineshloka
{अन्तराग्निः स्मृतो यस्तु भुक्तं पचति देहिनाम्}
{स यज्ञे विश्वभुङ्वाम सर्वलोकेषु भारत}


\twolineshloka
{ब्रह्मचारी यतात्मा च सततं विपुलप्रभः}
{ब्राह्मणाः पूजयन्त्येनं पाकयज्ञेषु पावकम्}


\twolineshloka
{प्रथितो गोपतिर्नाम नदी यस्याभवत्प्रिया}
{तस्मिन्कर्माणि सर्वाणि क्रियन्ते धर्मकर्तृभिः}


\twolineshloka
{बडबाग्निः पिबत्यम्भो योसौ परमदारुणः}
{ऊर्द्वभागूर्ध्वभाङ्नाम कविः प्राणाश्रितस्तु यः}


\twolineshloka
{उदग्द्वारं हविर्यस्य गृहे नित्यं प्रदीयते}
{ततस्तुष्टो भवेद्ब्रह्मा स्विष्टकृत्परमः स्मृतः}


\twolineshloka
{यः प्रशान्तेषु भूतेषु आविर्भवति पावकः}
{क्रोधस्य तु रसो जज्ञे मन्यती चाथ पुत्रिका}


\threelineshloka
{स्वाहेति दारुणा क्रूरा सर्वभूतेषु तिष्ठति}
{त्रिदिवे यस्य सदृशो नास्ति रूपेण कश्चन}
{अतुलत्वात्कृतो देवैर्नाम्ना कामस्तु पावकः}


\twolineshloka
{संहर्षाद्धारयन्क्रोधं धन्वी स्रग्वी रथे स्थितः}
{समरे नाशयेच्छत्रूनमोघो नाम पावकः}


\twolineshloka
{उक्थ्यो नाम महाभाग त्रिभिरुक्थैरभिष्टुतः}
{महावर्षं त्वजनयत्सकामाश्वं हि यं विदुः}


\chapter{अध्यायः २२२}
\twolineshloka
{मार्कण्डेय उवाच}
{}


\twolineshloka
{काश्यपो ह्यथ वासिष्ठः प्राणश्च प्राणपुत्रकः}
{अग्निराङ्गिरसश्चैव च्यवनस्तीव्रवर्चकः}


\twolineshloka
{अचरत्स तपस्तीव्रं पुत्रार्थे बहुवार्षिकम्}
{पुत्रं लभेयं धर्मिष्ठं यशसा ब्रह्मणा समम्}


\twolineshloka
{महाव्याहृतिभिर्ध्यातः पञ्चभिस्तैस्तदा त्वथ}
{जज्ञे तेजोमयार्चिष्मान्पञ्चवर्णः प्राकरः}


\twolineshloka
{समिद्धोऽग्निः शिरस्तस्य बाहू सूर्यनिभौ तथा}
{त्वङ्नेत्रे च सुवर्णाभे कृष्णे जङ्घे च भारत}


\twolineshloka
{पञ्चवर्षः स तपसा कृतस्तैः पञ्चभिर्जनैः}
{पाञ्चजन्यः श्रुतो देवः पञ्चवंशकरस्तु सः}


\twolineshloka
{दशवर्षसहस्राणि तपस्तप्त्वा महातपाः}
{जनयत्पावकं घोरं पितॄणां स प्रजाः सृजन्}


\twolineshloka
{बृहद्रथंतरौ मूर्ध्ना वक्राच्च तपसा हरिम्}
{शिवं नाभ्यां बलादिन्द्रं प्राणाद्वायुं च भारत}


\twolineshloka
{बाहुभ्यामनुदात्तौ च विश्वे भूतानि चैव ह}
{एतान्सृष्ट्वा ततः पञ्च पितॄणामसृजत्सुतान्}


\twolineshloka
{बृहद्रथस्य प्रणिधिः काश्यपश्य महत्तरः}
{भानुरङ्गिरसो धीरः पुत्रो वर्चस्य सौरभः}


\twolineshloka
{प्राणस्य चानुदात्तस्तु व्याख्याताः पञ्च वंशजाः}
{देवान्यज्ञमुषश्चान्यान्सृजन्पञ्चदशोत्तरान्}


\twolineshloka
{सुभीममतिभीमं च भीमं भीमबलाबलम्}
{एतान्यज्ञमुषः पञ्च देवानप्यसृजत्ततः}


\twolineshloka
{सुमित्रं मित्रवन्तं च मित्रज्ञं मित्रवर्धनम्}
{मित्रधर्माणमित्येतान्देवानभ्यसृजत्ततः}


\twolineshloka
{सुरप्रवीरं वीरं च सुरेशं च सुवर्चसम्}
{सुराणामपि भर्तारं पञ्चैतानसृजत्ततः}


\twolineshloka
{त्रिविधं संस्थिता ह्येते पञ्चपञ्च पृथक्पृथक्}
{मुष्णन्त्यत्र स्थिता ह्येते स्वर्गतान्यज्ञयाजिनः}


\twolineshloka
{तेषामिष्टं हरन्त्येते निघ्नन्ति च महद्धविः}
{स्पर्धयाहव्यवाहानां निघ्नन्त्येते हरन्ति च}


\twolineshloka
{बहिर्वेद्यां तदादानं कुशलैः संप्रवर्तितम्}
{तत्रैते नोपसर्पन्ति यत्र चाग्निः स्थितो भवेत्}


\twolineshloka
{चितोऽग्निरुद्वहन्यज्ञं पक्षाभ्यां तान्प्रबाधते}
{मन्त्रै प्रशमिताह्येते नेष्टं मुष्णन्ति यज्ञियम्}


\twolineshloka
{तपस्ये बृदुक्थस्य पुत्रो भूमिमुपाश्रितः}
{अग्निहोत्रे हूयमाने पृथिव्यां सद्भिरीड्यते}


\twolineshloka
{रथन्तरश्चतपसः पुत्रोऽग्निः परिपठ्यते}
{मित्रविन्दा तथा भार्या हविरध्वर्यवो विदुः}


\twolineshloka
{`एतैः सह महाभाग तपस्तेजस्विभिर्नृप'}
{मुमुदे परमप्रीतः सह पुत्रैर्भहायशाः}


\chapter{अध्यायः २२३}
\twolineshloka
{मार्कण्डेय उवाच}
{}


\twolineshloka
{गुरुभिर्नियमैर्युक्तो भरतो नाम पार्थिव}
{अग्निः पुष्टिमग्निमि तुष्टः पुष्टिं प्रयच्छति}


\threelineshloka
{`सततं भरतश्रेष्ठ पावकोयं महाप्रभः'}
{अग्निर्यश्च शिवो नाम शक्तिपूजायनिश्च सः}
{दुःखार्तानां स सर्वेषां शिवकृत्सततं शिवः}


\twolineshloka
{तपसस्तु फलं दृष्ट्वासंप्रवृत्तं तपोमयम्}
{उद्धर्तुकामो मतिमान्पुत्रो जज्ञे पुरंदरः}


\twolineshloka
{उष्मा चैवोष्मणो जज्ञे सोऽग्निर्भूतेषु लक्ष्यते}
{अग्निश्चापि मनुर्नाम प्राजापत्यं सकारणम्}


\twolineshloka
{शंभुमग्निमथ प्राहुर्ब्राह्मणा वेदपारगाः}
{आवसथ्यं देविजाः प्राहुर्दीप्तमग्निं महाप्रभम्}


\twolineshloka
{ऊर्जस्करान्हव्यवाहान्सुवर्णसदृशप्रभान्}
{अग्निस्तपो ह्यजनयत्पञ्च यज्ञसुतानिह}


\twolineshloka
{प्रततोऽग्निर्महाभाग परिश्रान्तो गवांपतिः}
{असुराञ्जनयन्धोरान्मत्यांश्चैव पृथिग्विधान्}


\twolineshloka
{तपसश्च मनुं भानुं चाप्यङ्गिराः सृजत्}
{बृहद्भानुं तु तं प्राहुर्ब्राह्मणा वेदपारघाः}


\threelineshloka
{भानोर्भार्या महाराज बृहद्भासा तु सोमजा}
{षट्पुत्राञ्जनयामास तदा सा कन्यया सह}
{भानोराङ्गिरसस्याथ शृणु तस्य प्रजाविधिम्}


\twolineshloka
{दुर्बलानां तु भूतानामसून्यः संप्रयच्छति}
{तमग्निं बलदं प्राहुः प्रथमं भानुजं सुतम्}


\twolineshloka
{यः प्रशान्तेषु भूतेषु मन्युर्भवति दारुणः}
{अग्निः स मन्युमान्नाम द्वितीयो भानुजः सुतः}


\twolineshloka
{दर्शे च पौर्णमासे च यस्येह हविरुच्यते}
{विष्णुर्नामेह योऽग्निस्तु धृतिमान्नाम सोङ्गिराः}


\twolineshloka
{इन्द्रेण सहितं यस्य हविराग्रयणं स्मृतम्}
{अग्निराग्रयणो नाम भानोरेवान्वयस्तु सः}


\twolineshloka
{चातुर्मास्येषु नित्यानां हविषां यो निरग्रहः}
{चतुर्भिः सहितः पुत्रैर्भानोरेवान्वयस्तु सः}


\twolineshloka
{निशां त्वजनयत्कन्यामग्नीषोमावुभौ तथा}
{मनोरेवाभवद्भार्या सुषुवे पञ्च पावकान्}


\twolineshloka
{पूज्यते हविषा योऽग्रे चातुर्मास्येषु पावकः}
{पर्जन्यसहितः श्रीमानग्निर्वैश्वानरस्तु सः}


\twolineshloka
{अस्य लोकस्य सर्वस्य यः प्रभुः परिषठ्यते}
{सोऽग्निर्विश्वपतिर्नाम द्वितीयस्तापसः सुतः}


\twolineshloka
{कन्या या हरिणी नाम हिरण्यकशिपोः सुताः}
{}


\threelineshloka
{कर्मणाऽसौ बभौ भार्या स वह्निः स प्रजापतिः}
{प्राणानाश्रित्य यो देहं प्रवर्तयति देहिनाम्}
{तस्य सन्निहितो नाम शब्दरूपस् साधनः}


\twolineshloka
{शुक्लं कृष्णं वपुर्देवो यो बिभर्ति हुताशनः}
{अकल्मषः कल्मषाणां कर्ता क्रोधाश्रितस्तुसः}


\twolineshloka
{कपिलं परमर्षिं च यं प्राहुर्यतयः सदा}
{अग्निः स कपिलो नाम साङ्ख्ययोगप्रवर्तकः}


\twolineshloka
{योऽन्तर्यच्छति भूतानि येन चेष्टन्ति नित्यदा}
{कर्मस्विह विचित्रेषु सोऽग्रणीर्वह्निरुच्यते}


\twolineshloka
{इमामन्यान्समसृजत्पावकान्प्रथितौजसः}
{अग्निहोत्रस्य दुष्टस्य प्रायश्चित्तार्थमुल्वणान्}


\twolineshloka
{संस्पृशेयुर्यदाऽन्योन्यं कथंचिद्वायुनाऽग्नयः}
{इष्टिरष्टाकपालेन कार्या वै शुचयेऽग्नये}


\twolineshloka
{दक्षिणाग्निर्यदा द्वाभ्यां संसृजेत तदा किल}
{इष्टिरष्टाकपालेन कार्या वै वीतयेऽग्नये}


\twolineshloka
{यद्यग्नयो हि स्पृश्येयुर्निवेशस्था दवाग्निना}
{इष्टरष्टाकपालेन कार्या तु शुचयेऽग्नये}


\twolineshloka
{अग्निं रजस्वला वै स्त्री संस्पृशेदाग्निहौत्रिकम्}
{इष्टिरष्टाकपालेन कार्या वसुमतेऽग्नये}


\twolineshloka
{मृतः श्रूयेत यो जीवन्परासुरशुचिर्यथा}
{इष्टिरष्टाकपालेन कार्या सुरभिमतेऽग्नये}


\twolineshloka
{आर्तो न जुहुयादग्निं त्रिरात्रं यस्तु ब्राह्मणः}
{इष्टिरष्टाकपालेन कार्या तन्तुमतेऽग्नये}


\twolineshloka
{दर्शश्च पौर्णमासश्च यस् तिष्ठेत्प्रतिष्ठितम्}
{इष्टिरष्टाकपालेन कार्या पथिकृतेऽग्नये}


\twolineshloka
{सूतिकाग्निर्यदा चाग्निं संस्पृशेदाग्निहौत्रिकम्}
{इष्टिरष्टाकपालेन कार्यां चाग्निमतेऽग्नये}


\chapter{अध्यायः २२४}
\twolineshloka
{मार्कण्डेय उवाच}
{}


\twolineshloka
{आपस्य मुदिता भार्या सा चास्य परमाप्रिया}
{भूपतिं भूतकर्तारं जनयामास पावकम्}


\twolineshloka
{भूतानां साऽपि सर्वेषां यं प्राहुः पावकं पतिम्}
{आत्मा भुवनकर्ता च सोऽध्वरेषु द्विजातिभिः}


\twolineshloka
{महतां चैव भूतानां सर्वेषामिह यः पतिः}
{भगवान्स महातेजा नित्यं चरति पावकः}


\twolineshloka
{अग्निर्गृहपतिर्नाम नित्यं यज्ञेषु पूज्यते}
{आज्यं वहतियो हव्यं सर्वलोकस् पावकः}


\twolineshloka
{अपांगर्भो महाभागः सत्वभुग्यो महाद्भुतः}
{भूपतिर्भूतकर्ता च महतः पतिरुच्यते}


\twolineshloka
{मता वहन्तो हव्यानि तस्याग्नेरप्रजाऽभवन्}
{अग्निष्टोमस्तु नियतः क्रतुश्रेष्ठो भवत्युत}


\twolineshloka
{[स वह्निः प्रथमो नित्यं देवैरन्विष्यते प्रभुः]}
{आयान्तं नियतं दृष्ट्वाप्रविवेशार्णवं भयात्}


\twolineshloka
{देवास्तं नाधिगच्छन्ति मार्गमाणा यथातथम्}
{दृष्ट्वा त्वग्निरथर्वाणं ततो वचनमब्रवीत्}


\twolineshloka
{देवानां वह हव्यं त्वमहं वीर सुदुर्बलः}
{अथ त्वं गच्छ मध्वक्षं प्रियमेतत्कुरुष्व मे}


\threelineshloka
{प्रेष्य चाग्निरथर्वाणमन्यं देशं ततोऽगमत्}
{मत्स्यास्तस्य समाचख्युः क्रुद्धस्तानग्निरब्रवीत्}
{भक्ष्या वै विविधैर्भावैर्भविष्यथ शरीरिणाम्}


% Check verse!
अथर्वाणं तथा चापि हव्यवाहोऽब्रवीद्वचः
\twolineshloka
{अनुनीयमानो हि भृशं देववाक्यान्वितेन सः}
{नैच्छद्वोढुं हवि सर्वं शरीरं चापि सोऽत्यजत्}


\twolineshloka
{स तच्छरीरं संत्यज्य प्रविवेश धरां तदा}
{भूमिं स्पृष्ट्वाऽसृजद्भ्रातून्पृथक्पृथगतीव हि}


\twolineshloka
{आस्यात्सुगन्धं तेजश्च अस्थिभ्यो देवदारु च}
{श्लेष्मणः स्फाटिकं तस् पित्तान्मारकतं तथा}


\twolineshloka
{वातात्कृष्णायसं तस्य त्रिभिरेतैर्बहु प्रजाः}
{त्वचस्तस्याभ्रपटलं स्नायुजं चापि विद्रुमम्}


\twolineshloka
{शरीरास्थिविधाश्चान्ये धातवोऽस्याभवन्नृप}
{एवं कृत्वा शरीरं च परमे तपसि स्थितः}


\twolineshloka
{भृग्वङ्गिरादिभिर्भूयस्तपसोत्थापितस्तदा}
{भृशं जज्वाल तेजस्वी तपसाऽऽप्याधितः शिखी}


\threelineshloka
{दृष्ट्वा ऋषीनभयाच्चापि प्रविवेश महार्णवम्}
{तस्मिन्नष्टे जगद्भीतमथर्वाणमथाश्रितम्}
{अर्चयामासुरेवैनमथर्वाणं सुरादयः}


\twolineshloka
{अथर्वाणं श्रिताँल्लोकानात्मन्यालोच्य पावकः}
{मिषतां सर्वभूतानामुन्ममज्ज महार्णवात्}


\twolineshloka
{एवमग्निर्भगवता नष्टः पूर्वमथर्वणा}
{आहूतः सर्वूतानां हव्यं वहति सर्वदा}


\twolineshloka
{एवं त्वजनयद्धिष्ण्यान्वेदोक्तान्विबुधान्बहून्}
{विचरन्विविधान्देशान्भ्रममाणस्तु तत्र वै}


\twolineshloka
{सिन्धुवर्ज्याः पञ्च नद्यो देविकाऽथ सरस्वती}
{गङ्गा च शतकुम्भा च सरयू गण्डसाह्वया}


\twolineshloka
{चर्मण्वती मही चैव मेध्या मेधासृतिस्तदा}
{इरावती वेत्रवती नद्यतिस्रोऽथ कौशिकी}


\twolineshloka
{तमसा नर्मदा चैव नदी गोदावरी तथा}
{वेण्णा प्रवेणी भीमा च मरुदा चैव भारत}


\twolineshloka
{भारती सुप्रयोगा च कावेरी मज्जुरा तथा}
{तुङ्गवेणा कृष्णवेणा कपिला शोण एव च}


% Check verse!
एता नद्यस्तु धिष्ण्यानां मातरो याः प्रकीर्तिताः
\twolineshloka
{अद्भुतस्य प्रिया भार्या तस्याः पुत्रो विडूरथः}
{यावन्तः पावकाः प्रोक्ताः सोमास्तावन्त एव तु}


\twolineshloka
{अत्रेश्चाप्यन्वये जाता ब्रह्मणो मानसाः प्रजाः}
{अग्निः पुत्रान्स्रष्टुकामस्तानेवात्मन्यधारयत्}


\twolineshloka
{तस्य तद्ब्रह्मणः कायान्निर्हरन्ति हुताशनाः}
{एवमेते महात्मानः कीर्तितास्तेऽग्नयो यथा}


\twolineshloka
{अप्रमेया यथोत्पन्नाः श्रीमन्तस्तिमिरापहाः}
{अद्भुतस्त तु माहात्म्यं यथा वेदेषु कीर्तितम्}


\twolineshloka
{तादृशं विद्धि सर्वेषामेको ह्येषु हुताशनः}
{एक एवैष भगवान्विज्ञेयः प्रथमोऽङ्गिराः}


\threelineshloka
{बहुधा निःसृतः कायाज्ज्योतिष्टोमः क्रतुर्वथा}
{इत्येष वंशः सुमहानग्नीनां कीर्तितो मया}
{पावको विविधैर्मन्त्रैर्हव्यं वहति देहिनाम्}


\chapter{अध्यायः २२५}
\twolineshloka
{मार्कण्डेय उवाच}
{}


\twolineshloka
{अग्नीनां विविधा वंशाः कीर्तितास्ते मयाऽनघ}
{शुणु जन्म तु कौरव्य कार्तिकेयस् धीमतः}


\twolineshloka
{अद्भुतस्याद्भुतं पुत्रं प्रवक्ष्याम्यमितौजसम्}
{जातं सप्तर्षिभार्याभिर्ब्रह्मण्यं कीर्तिवर्धनम्}


\twolineshloka
{देवासुराः पुरा यत्ता विनिघ्नन्तः परस्परम्}
{तत्राजयन्सदा देवान्दानवा घोररूपिणः}


\twolineshloka
{वध्यमानं बलं दृष्ट्वा बहुशस्तैः पुरंदरः}
{स सैन्यनायकार्थाय चिन्तयामास वासवः}


\twolineshloka
{देवसेनां दानवैर्हि भग्नां दृष्ट्वा महाबलः}
{पालयेद्वीर्यमाश्रित्य स ज्ञेयः पुरुषो मया}


\twolineshloka
{स शैलं मानसं गत्वा ध्यायन्नर्थमिमं भृशम्}
{शुश्रावार्तस्वरं घोरमथ मुक्तं स्त्रिया तदा}


\twolineshloka
{अभिधावतु मां कश्चित्पुरुषस्त्रातु चैव ह}
{पतिं च मे प्रदिशतु स्वयं वा पतिरस्तु मे}


\twolineshloka
{पुरंदरस्तु तामाह मा भैर्नास्ति भयं तव}
{एवमुक्त्वा ततोऽपश्यत्केशिनं स्थितमग्रतः}


\twolineshloka
{किरीटिनं गदापाणिं धातुमन्तमिवाचलम्}
{हस्ते गृहीत्वा कन्यां तामथैनं वासवोऽब्रवीत्}


\threelineshloka
{अनार्यकर्मन्कस्मात्त्वमिमां कन्यां जिहीर्षसि}
{वज्रिणं मांविजानीहि विरमास्याः प्रबाधनात् ॥केश्युवाच}
{}


\twolineshloka
{विसृजस्व त्वमेवैनां शक्रैषा प्रार्थिता मया}
{क्षमं तेजीवतो गनतुं स्वपुरं पाकशासन}


\twolineshloka
{एवमुक्त्वा गदां केशी चिक्षेपेन्द्रवधाय वै}
{तामापतन्तीं चिच्छेद मध्ये वज्रेण वासवः}


\twolineshloka
{अथास्य शैलशिखरं केशी क्रुद्धो व्यवासृजत्}
{`महामेघप्रतीकाशं चलत्पावकसंकुलम्}


\twolineshloka
{तदापन्ततं संप्रेक्ष्य शैलशृङ्गं शतक्रतुः}
{बिभेद राजन्वज्रेण भुवि तन्निपपात ह}


\twolineshloka
{पतता तु तदा केशी तेन शृङ्गेण ताडितः}
{हित्वा कन्यां महाभागां प्राद्रवद्भृशपीडितः}


\twolineshloka
{अपयातेऽसुरे तस्मिंस्तां कन्यां वासवोऽब्रवीत्}
{कासि कस्यसि किंचेह कुरुषे त्वं शुभानने}


\chapter{अध्यायः २२६}
\twolineshloka
{कन्योवाच}
{}


\twolineshloka
{अहं प्रजापतेः कन्या देवसेनेति वुश्रुता}
{भगिनी मे दैत्यसेना सा पूर्वं कोशिना हृता}


\twolineshloka
{सहैवावां भगिन्यौ तु सखीभिः सह मानसम्}
{आगच्छावो विहारार्थमनुज्ञाप्य प्रजापतिम्}


\twolineshloka
{नित्यं चावां प्रार्थयते हर्तुं केशी महासुरः}
{इच्छत्येनं दैत्यसेना न चाहं पाकशासन}


\threelineshloka
{सा हृताऽनेन भगवान्मुक्ताऽहं त्वद्बलेन तु}
{त्वया देवेन्द्र निर्दिष्टं पतिमिच्छामि दुर्जयम् ॥इन्द्र उवाच}
{}


\threelineshloka
{मम मातृष्वसेयी त्वं माता दाक्षायणी मम}
{आख्यातुं त्वहमिच्छामि स्वयमात्मबलं त्वया ॥कन्योवाच}
{}


\threelineshloka
{अबलाऽहं महाबाहो पतिस्तु बलवान्मम}
{वरदानात्पितुर्भावी सुरासुरनमस्कृतः ॥इन्द्र उवाच}
{}


\threelineshloka
{कीदृशं तु बलंदेवि पत्युस्तव भविष्यति}
{एतदिच्छाम्यहं श्रोतुं तव वाक्यमनिन्दिते ॥कन्योवाच}
{}


\twolineshloka
{देवदानवयक्षाणां किन्नरोरगरक्षसाम्}
{जेता यो दुष्टदैत्यानां महावीर्यो महाबलः}


\threelineshloka
{यस्तु सर्वाणि भूतानि त्वया सह विजेष्यति}
{स हि मे भविता भर्ता ब्रह्मण्यः कीर्तिवर्धनः ॥मार्कण्डेय उवाच}
{}


\twolineshloka
{इन्द्रस्तस्या वचः श्रुत्वा दुःखितोऽचिन्तयद्भृशम्}
{अस्या देव्याः पतिर्नास्ति यादृशं संप्रभाषते}


\twolineshloka
{अथापश्यत्स उदये भास्करं भास्करद्युतिः}
{सोमं चैव महाभागं प्रविशन्तं दिवाकरम्}


\twolineshloka
{अमावास्यां प्रवृत्तायां मुहूर्ते रौद्र एव तु}
{देवासुरं च संग्रामं सोऽपश्यदुदये गिरौ}


\twolineshloka
{लोहितैश्च घनैर्युक्तां पूर्वां संध्यां शतक्रतुः}
{अपश्यल्लोहितोदं च मघवान्वरुणालयम्}


\twolineshloka
{भृगुभिश्चाङ्गिरोभ्यश्च हुतं मन्त्रैः पृथग्विधैः}
{हव्यं गृहीत्वा वह्निं च प्रविशन्तं दिवाकरम्}


\twolineshloka
{पर्व चैव चतुर्विंशं तदा सूर्यमुपस्थितम्}
{तथा सूर्यं वह्निगतं सोमं सूर्यगतं च तम्}


\twolineshloka
{समालोक्यैकतामेव शशिनो भास्करस्य च}
{समवायं तु तं रौद्रं दृष्ट्वा शक्रोऽन्वचिन्तयत्}


\twolineshloka
{सूर्याचन्द्रमसोर्घोरं दृश्यते परिवेषणम्}
{एतस्मिन्नेव रात्र्यन्ते महद्युद्धं तु शंसति}


\twolineshloka
{सरित्सिन्धुरपीयं तु प्रत्यसृग्वाहिनी भृशम्}
{शृगालिनयग्निवक्रा च प्रत्यादित्यं विराविणी}


\twolineshloka
{एष रौद्रश्चसंघातो महान्युक्तश्च तेजसा}
{सोमस्य वह्निसूर्याभ्यामद्भुतोऽयं समागमः}


\twolineshloka
{जनयेद्यं सुतं सोमः सोऽस्या देव्याः पतिर्भवेत्}
{अथानेकैर्गुणैश्चाग्निरग्निः सर्वाश्च देवताः}


\twolineshloka
{एष चेञ्जनयेद्गर्भं सोऽस्या देव्याः पतिर्भवेत्}
{एवं संचिन्त्य भगवान्ब्रह्मलोकं तदा गतः}


\threelineshloka
{गृहीत्वा देवसेनां तां ववन्दे स पितामहम्}
{उवाच चास्या देव्यास्त्वं साधु शूरं पतिं दिशा ॥ब्रह्मोवाच}
{}


\twolineshloka
{यथैतच्चिन्तितंकार्यं त्वया दानवसूदन}
{तथा स भविता गर्भो बलवानुरुविक्रमः}


\twolineshloka
{स विष्यति सेनानीस्त्वया सह शतक्रतो}
{अस्या देव्याः पतिश्चैव स भविष्यति वीर्यवान्}


\twolineshloka
{एतच्छ्रुत्वा नमस्तस्मै कृत्वाऽसौ मह कन्यया}
{तत्राभ्यगच्छद्देवेन्द्रो यत्र सप्तर्षयोऽभवन्}


\twolineshloka
{वसिष्ठप्रमुखा मुख्या विप्रेन्द्राः सुमहाप्रभाः}
{भागार्थं तपसोपात्तं तेषां सोमं तथाऽध्वरे}


\twolineshloka
{पिपासवो ययुर्देवाः शतक्रतुपुरोगमाः}
{इष्टिं कृत्वयतान्यायं सुसमिद्धे हुताशने}


\twolineshloka
{जुहुवुस्ते महात्मानो हव्यं सर्वदिवौकसाम्}
{समाहूतो हुतवहःसोऽद्भुतः सूर्यमण्डलात्}


\twolineshloka
{विनिःसृत्य ययौ वह्निः पार्श्वतो विधिवत्प्रभुः}
{आगम्याहवनीयं वै तैर्द्विजैर्मन्त्रतो हुतम्}


\twolineshloka
{सतत्र विविधं हव्यं प्रतिगृह्यहुताशनः}
{ऋषिभ्यो भरतश्रेष्ठ प्रायच्छत दिवौकसाम्}


\twolineshloka
{निष्क्रामंश्चाप्यपश्यत्स पत्नीस्तेषां महात्मनाम्}
{स्वेष्वासतेषूपविष्टाः स्नायन्तीश्च यथासुस्वम्}


\twolineshloka
{रुक्मवेदिनिभास्तास्तु चन्द्रलेखा इवामलाः}
{हुताशनार्चिःप्रतिमाः सर्वास्तारा इवाद्भुताः}


\twolineshloka
{स तत्रतेन मनसा बभूव क्षुभितेन्द्रियः}
{पत्नीर्दृष्ट्वा द्विजेन्द्राणां वह्निः कामवशं ययौ}


\twolineshloka
{भूयः संचिन्तयामास न न्याय्यं क्षुभितो ह्यहम्}
{साध्व्यः पत्न्यो द्विजेन्द्राणामकामा कामयाम्यहम्}


\threelineshloka
{नैताः शक्या मया द्रष्टुं स्प्रष्टुं वाऽप्यनिमित्ततः}
{गार्हपत्यं समाविश्य तस्मात्पश्याम्यभीक्ष्णशः ॥मार्कण्डेय उवाच}
{}


\twolineshloka
{संस्पृशन्निव सर्वास्ताः शिखाभिः काञ्चनप्रभाः}
{पश्यमानश्च मुमुदे गार्हपत्यं समाश्रितः}


\twolineshloka
{निरुध्य तत्रसुचिरमेवं वह्निर्वनं गतः}
{मनस्तासु विनिक्षिप्य कामयानो वराङ्गनाः}


\twolineshloka
{कामसंतप्तहृदयो देहत्यागे विनिश्चितः}
{अलाभे ब्राह्मणस्त्रीणामग्निर्वनमुपागमत्}


\threelineshloka
{स्वाहा तं दक्षदुहिता प्रथमाऽकामयत्तदा}
{सा तस्य च्छिद्रमन्वैच्छच्चिरात्प्रभृतिभामिनी}
{}


\twolineshloka
{सा तं ज्ञात्वा यथावत्तु वर्हिं वनमुपागतम्}
{तत्त्वतः कामसंतप्तं चिन्तयामास भामिनी}


\threelineshloka
{अहं सप्तर्षिपत्नीनां कृत्वा रूपाणि पावकम्}
{कामयिष्यामि कामार्तं तासां रूपेण मोहितम्}
{एवं कृते प्रीतिरस् कामावाप्तिश्च मे भवेत्}


\chapter{अध्यायः २२७}
\twolineshloka
{मार्कण्डेय उवाच}
{}


\twolineshloka
{शिवा भार्या त्वङ्गिरसः शीलरूपगुणान्विता}
{तस्याः सा प्रथमं रूपं कृत्वादेवी जनाधिप}


\twolineshloka
{जगाम पावकाभ्याशं तं चोवाच वराङ्गना}
{मामग्ने कामसंतप्तां त्वं कामयितुमर्हसि}


\twolineshloka
{करिष्यसि न चेदेवं मृतां मामुपधारय}
{`तवाप्यधर्मः सुमहान्भविता वै हुताशन'}


\threelineshloka
{अहमङ्गिरसो भार्या शिवा नाम हुताशन}
{सखीभिः सहिता प्राप्ता मन्त्रयित्वा विनिश्चयम् ॥अग्निरुवाच}
{}


\threelineshloka
{कथं मां त्वं विजानीषे कामार्तमितराः कथम्}
{यास्त्वया कीर्तिताः सर्वाः सप्तर्षीणां प्रियाः स्त्रियः ॥शिवोवाच}
{}


\twolineshloka
{अस्माकं त्वं प्रियो नित्यं बिभीमस्तु वयं तव}
{त्वच्चित्तमिङ्गितैर्ज्ञात्वा प्रेषिताऽस्मि तवान्तिकम्}


\fourlineindentedshloka
{मैथुनायेह संप्राप्ता कामाच्चैव द्रुतं च माम्}
{`उपयन्तुं महावीर्य पूर्वमेव त्वमर्हसि'}
{[जामयो मां प्रतीक्षन्ते गमिष्यामि हुताशन] ॥मार्कण्डेय उवाच}
{}


\twolineshloka
{ततोऽग्निरुपयेमे तां शिवां प्रीतिमुदायुतः}
{प्रीत्या देवी समायुक्ता शुक्रं जग्राह पाणिना}


\twolineshloka
{साऽचन्तयन्ममेदं ये रूपं द्रक्ष्यन्ति कानने}
{ते ब्राह्मणीनामनृतं दोषं वक्ष्यन्ति पावके}


\threelineshloka
{तस्मादेतद्रक्षमाणआ गरुडी संभवाम्यहम्}
{वनान्निर्गमनं चैव सुखं मम भविष्यति ॥मार्कण्डेय उवाच}
{}


\twolineshloka
{सुपर्णी सा तदा भूत्वा निर्जगाम महावनात्}
{उपश्यत्पर्वतं श्वेतं शरस्तम्बैः सुसंवृतम्}


\twolineshloka
{दृष्टीविषैः सप्तशीर्षैर्गुप्तं भोगिभिरद्भुतैः}
{रक्षोभिश्च पिशाचैश्च रौद्रैर्भूतगणैस्तथा}


\twolineshloka
{राक्षसीभिश्च संपूर्णमनेकैश्च मृगद्विजैः}
{`नदीप्रस्रवणोपेतं नानातरुलताचितम्'}


\twolineshloka
{सा तत्र सहसा गत्वा शैलपृष्ठं सुदुर्गमम्}
{प्राक्षिपत्काञ्चने कुण्डे शुक्रं सा त्वरिता शुभा}


\twolineshloka
{शिष्टानामपि सा देवी सप्तर्षीणां महात्मनाम्}
{पत्नीसरूपतां कृत्वा रमयामास पावकम्}


\twolineshloka
{दिव्यरूपमरुन्धत्याः कर्तुं न शकितं तया}
{तस्यास्तपःप्रभावेण भर्तृशुश्रूषणेन च}


\twolineshloka
{षट्कृत्वस्तत्तु निक्षिप्तमग्ने रेतः कुरुत्तम}
{तस्मन्कुण्डे प्रतिपदि कामिन्या स्वाहया तदा}


\twolineshloka
{तत्स्कन्नं तेजसा तत्रसंवृतंजनयत्सुतम्}
{ऋषिभिः पूजितं स्कन्नमनयन्स्कन्दतां ततः}


\twolineshloka
{षट््शिरा द्विगुणश्रोत्रो द्वादशाक्षिभूजक्रमः}
{एकग्रीवस्त्वेककायः कुमारः समपद्यत}


\twolineshloka
{द्वितीयायामभिव्यक्तस्तृतीयायां शिशुर्बभौ}
{अङ्गप्रत्यङ्गसंभूतश्चतुर्थ्यामभवद्गुहः}


\twolineshloka
{लोहिताभ्रेण महता संवृतः सह विद्युता}
{लोहिताभ्रे सुमहति भाति सूर्य इवोदितः}


\twolineshloka
{गृहीतं तु धनुस्तेन विपुलं रोमहर्षणम्}
{न्यस्तं यत्रिपुरघ्नेन सुरारिविनिकृन्तनम्}


\twolineshloka
{तद्गृहीत्वा धनुःश्रेष्ठं ननाद बलवांस्तदा}
{संमोहयन्निमाँल्लोकान्गुहस्त्रीन्सचराचरान्}


\twolineshloka
{तस् तं निनदं श्रुत्वा महामेघौघनिःस्वनम्}
{उत्पेततुर्महानागौ चित्रश्चैरावतश्च ह}


\twolineshloka
{तावापतन्तौ संप्रेक्ष्यस बालोऽर्कसमद्युतिः}
{द्वाभ्यां गृहीत्वा पाणिभ्यां शक्तिं चान्येन पाणिना}


\threelineshloka
{अपरेणाग्निदायादस्ताम्रचूडं भुजेन सः}
{महाकायमुपश्लिष्टं कुक्कुटं बलिनां वरम्}
{गृहीत्वा व्यनदद्भीमं चिक्रीड च महाभुजः}


\fourlineindentedshloka
{द्वाभ्यां भुजाभ्यां बलवान्गृहीत्वा शङ्खमुत्तमम्}
{प्राध्मापयत् भूतानां त्रासनं बलिनामपि}
{द्वाभ्यां भुजाभ्यामाकाशं बहुशो निजघान ॥ 3-227-28aक्रीडन्भातिमहासेनस्त्रींल्लोकान्वदनैः पिवन्}
{पर्वताग्रेऽप्रमेयात्मा रश्मिमानुदये यथा}


\twolineshloka
{स तस् पर्वतस्याग्रे निषण्णोऽद्भुतविक्रमः}
{व्यलोकयदमेयात्मा मुखैर्नानाविधैर्दिशः}


\threelineshloka
{स पश्यन्विविधान्भावांश्चकार निनदं पुनः}
{तस्य तं निनदं श्रुत्वा न्यपतन्बहुधा जनाः}
{भीताश्चोद्विग्रमनसस्तमेव शरणं ययुः}


\twolineshloka
{ये तुं संश्रिता देवं नानावर्णास्तदा जनाः}
{तानप्याहुः पारिषदान्ब्राह्मणाः सुमहाबलान्}


\twolineshloka
{स तूत्थाय महाबाहुरुपसान्त्व्य च ताञ्जनान्}
{धनुर्विकृष्य व्यसृजद्बाणाञ्श्वेते महागिरौ}


\twolineshloka
{विभेद स शरैः शैलं क्रौञ्चं हिमवतः सुतम्}
{तेन हंसाश्च गृध्राश्च मेरुं गच्छन्ति पर्वतम्}


% Check verse!
स विशीर्णोऽपतच्छैलो भृशमार्तस्वराव्रुवन्
\twolineshloka
{तस्मिन्निपतिते त्वन्ये नेदुः शैला भृशं भयात्}
{`घोरमार्तस्वरं चक्रुर्दृष्ट्वा क्रौञ्चं विदारितम्'}


\twolineshloka
{सतं नादं भृशार्तानां श्रुत्वाऽपि बलिनांवरः}
{न प्राव्यधदमेयात्मा शक्तिमुद्यम्य चानदत्}


\twolineshloka
{सा तदा विमला शक्तिः क्षिप्ता तेन महात्मना}
{बिभेद शिखरं घोरं श्वेतस् तरसा गिरेः}


\twolineshloka
{स तेनाभिहतो दीर्णो गिरिः श्वेतोऽचलैः सह}
{पलायत महीं त्यक्त्वा भीतस्तस्मान्महात्मनः}


\twolineshloka
{ततः प्रव्यथिता भूमिर्व्यशीर्यत समन्ततः}
{आर्ता स्कन्दं समासाद्य पुनर्बलवती बभौ}


\twolineshloka
{पर्वताश्च नमस्कृत्यतमेव पृथिवीं गताः}
{अथैनमभजल्लोकः स्कन्दं शुक्लस्य पञ्चमीम्}


\chapter{अध्यायः २२८}
\twolineshloka
{मार्कण्डेय उवाच}
{}


\twolineshloka
{तस्मिञ्जाते महासत्त्वे महासेने महाबले}
{समुत्तस्थुर्महोत्पाता घोररूपाः पृथग्विधाः ॥ 1}


\twolineshloka
{स्त्रीपुंसोर्विपरीतं च तथा द्वन्द्वानि यानि च}
{ग्रहा दीप्ता दिशः खं च ररास च मही भृशम्}


\twolineshloka
{ऋषयश्च महाघोरान्दृष्ट्वोत्पातान्समन्ततः}
{अकुर्वञ्शान्तिमुद्विग्ना लोकानां लोकभावनाः}


\threelineshloka
{निवसन्ति वने ये तु तस्मिंश्चैत्ररथे जनाः}
{तेऽब्रुवन्नेष नोऽनर्थः पावकेनाहृतो महान्}
{संगम्य षड्भिः पत्नीभिः सप्तर्षीणामिति स्म ह}


\twolineshloka
{अपरे गरुडीमाहुस्तयाऽनर्थोऽयमाहृतः}
{यैर्दृष्टा सा तदा देवी तस्या रूपेण गच्छती}


% Check verse!
न तु तत्स्वाहया कर्म कृतंजानाति वै जनः
\twolineshloka
{सुपर्णी तु वचः श्रुत्वा ममायं तनयस्त्विति}
{उपगम्य शनैः स्कन्दमाहाहं जननी तव}


\threelineshloka
{अथ सप्तर्षयः श्रुत्वा जातं पुत्रं महौजसम्}
{तत्यजुः षट् तदा पत्नीर्विना देवीमरुन्धतीम्}
{षड्भिरेव तदा जातमाहुस्तद्वनवासिनः}


\twolineshloka
{सप्तर्षीनाह च स्वाहा मम पुत्रोऽयमित्युत}
{अहं हेतुर्नैतदेवमिति राजन्पुनः पुनः}


\threelineshloka
{विश्वामित्रस्तु कृत्वेष्टिं सप्तर्षीणां महामुनिः}
{पावकं कामसंतप्तमदृष्टः पृष्ठतोऽन्वगात्}
{तत्तेन निखिलं सर्वमवबुध्य यथातथम्}


\twolineshloka
{विश्वामित्रस्तु प्रथमं कुमारं शरणं गतः}
{स्तवं दिव्यं संप्रचक्रे महासेनस् चापि सः}


\twolineshloka
{मङ्गलानि च सर्वाणि कौमाराणि त्रयोदश}
{जातकर्मादिकास्तस् क्रियाश्चक्रे महामुनिः}


\twolineshloka
{पड्वक्रस्य तु मांहात्म्यं कुक्कुटस्य तु साधनम्}
{शक्त्या देव्याः साधनं च तथा पारिषदामषि}


\twolineshloka
{विश्वामित्रश्चकारैतत्कर्म लोकहिताय वै}
{तस्मादृषिः कुमारस् विश्वामित्रोऽभवत्प्रियः}


\fourlineindentedshloka
{अन्वजानाच्च स्वाहाया रूपान्यत्वं महामुनिः}
{अब्रवीच्च मुनीन्सर्वाननापराध्यन्ति वै स्त्रियः}
{श्रुत्वा तु तत्वतस्तस्मात्ते पत्नीः सर्वतोत्यजन् ॥मार्कण्डेय उवाच}
{}


\twolineshloka
{स्कन्दं श्रुत्वा तदा देवा वासवं सहिताऽब्रुवन्}
{अविषह्यं वलं स्कन्दं जहि शक्राशु माचिर्}


\twolineshloka
{यदि वा न निहंस्येनमद्येन्द्रोऽयं भविष्यति}
{त्रैलोक्यं सन्निगृह्यास्मांस्त्वां च शक्र महाबला}


\threelineshloka
{स तानुवाच व्यथितो वालोऽयं सुमहाबलः}
{स्रष्टारमपि लोकानां युधि विक्रम्य नाशयेत्}
{[न बालमुत्सहे हन्तुमिति शक्रः प्रभाषते}


\twolineshloka
{तेऽब्रुवन्नास्ति ते वीर्यं यत एवं प्रभासे}
{]सर्वास्त्वद्याभिगच्छन्तु स्कन्दं लोकस्य मातरः}


\twolineshloka
{कामवीर्या घ्नन्तु चैनं तथेत्युक्त्वा च ता ययुः}
{तमप्रतिबलं दृष्ट्वा विषण्णवदनास्तु ताः}


\twolineshloka
{अशक्योऽयंविचिन्त्यैवं तमेव शरणं ययुः}
{ऊचुश्चैनं त्वमस्माकं पुत्रोऽस्माभिर्धृतंजगत्}


\twolineshloka
{अभिनन्द्य ततः सर्वाः प्रस्नुताः स्नेहविक्लबाः}
{[तासां तद्वचनं श्रुत्वा पातुकामः स्तनान्प्रभुः]}


\twolineshloka
{ताः संपूज्य महासेनःकामांश्चासां प्रदाय सः}
{अपश्यदग्निमायान्तं पितरं बलिनां बली}


\twolineshloka
{स तु संपूजितस्तेन सह मातृगणेन ह}
{परिवार्य महासेनं रक्षमाणः स्थितः शिवः}


\twolineshloka
{सर्वासां या तु मातॄणां नारी क्रोधसमुद्भवा}
{धात्री स्वपुत्रवत्स्कन्दं शूलहस्ताऽभ्यरक्षत}


\twolineshloka
{लोहितस्योदधेः कन्या क्रूरा लोहितभोजना}
{परिष्वज्य महासेनं पुत्रवत्पर्यरक्षत}


\twolineshloka
{अग्निर्भूत्वा नैगमेयश्छागवक्रो बहुप्रजः}
{रमयामास शैलस्तं बालं क्रीडनकैरिव}


\twolineshloka
{ब्रहाः सोपग्रहाश्चैव ऋषयो मातरस्तथा}
{हुताशनमुखाश्चैव दृप्ताः पारिषदां गणाः}


\twolineshloka
{एते चान्ये च बहवो घोरास्त्रिदिववासिनः}
{परिवार्य महासेनं स्थिता मातृगणैः सह}


\twolineshloka
{संदिग्धं विजयं दृष्ट्वा विजयेप्सुः सुरेश्वरः}
{आरुह्यैरावतस्कन्धं प्रययौ दैवतैः सह}


\twolineshloka
{आदाय वज्रं बवान्सर्वैर्देवगणैर्वृतः}
{विजिघांसुर्महासेनमिन्द्रस्तूर्णतरं ययौ}


\twolineshloka
{`इन्द्रस्तस्य महावेगं दृष्ट्वाऽद्भुतपराक्रमम्}
{विस्मितश्चाभवद्राजन्देवानीकमचोदयत्}


\twolineshloka
{उग्रं तं च महानादं देवानीकं महाप्रभम्}
{विचित्रध्वजसन्नाहं नानावाहनकार्मुकम्'}


\twolineshloka
{प्रवराम्बरसंवीतं श्रिया जुष्टमलंकृतम्}
{विजिघांसुं तमायान्तं कुमारः शक्रमन्वयात्}


\twolineshloka
{वियत्पतिः स शक्रस्तु द्रुतमायान्महाबलः}
{संहर्पयन्देवसेनां जिघांसुः पावकात्मजम्}


\twolineshloka
{संपूज्यमानस्त्रिदशैस्तथैव परमर्षिभिः}
{समीपमथ संप्राप्तो वह्निपुत्रस्य वासवः}


\twolineshloka
{सिंहनादं ततश्चक्रे देवेशः सहितैः सुरैः}
{गुहोऽपिशब्दं तं श्रुत्वा व्यनदात्सागरो यथा}


\twolineshloka
{तस्य शब्देन महता समुद्धूतोदधिप्रभम्}
{वभ्राम तत्रतत्रैव दैवसैन्यमचेतनम्}


\twolineshloka
{जिधांसूनुपसंप्राप्तान्देवान्दृष्ट्वा सपावकिः}
{विससर्ज मुखात्क्रुद्धः प्रवृद्धाः पावकार्चिषः}


% Check verse!
अदहद्देवसैन्यानि वेपमानानि भूतले
\twolineshloka
{ते प्रदीप्तशिरोदेहाः प्रदीप्तायुधवाहनाः}
{प्रच्युताः सहसा भान्ति चित्रास्तारागणा इव}


\twolineshloka
{दह्यमानाः प्रपन्नास्ते शरणं पावकात्मजम्}
{देवा वज्रधरं त्यक्त्वाततः शानतिमुपागताः}


% Check verse!
त्यक्तो देवैस्ततः स्कन्दे वज्रं शक्रो न्यपातयत्
\twolineshloka
{तद्विसृष्टं जघानाशु पर्श्वं स्कन्दस्य दक्षिणम्}
{विभेद च महाराज पार्श्वं तस्य महात्मनः}


\twolineshloka
{वज्रप्रहारात्स्कन्दस्य संजातः पुरुषोऽपरः}
{युवा काञ्चनसन्नाहः शक्तिधृग्दिव्यकुण्डलः}


% Check verse!
यद्वज्रविशनाज्जातो विशाखस्तेन सोऽभवत्
\twolineshloka
{तं जातमपरं दृष्ट्वा कालानलसमद्युतिम्}
{भयादिनद्रस्तुत तं स्कन्दं प्राञ्जलिः शरणं गतः}


\twolineshloka
{तस्याभयंददौ स्कन्दः सहसैन्यस् सत्तम}
{ततः प्रहृष्टास्त्रिदशा वादित्राण्यभ्यवादयन्}


\threelineshloka
{स्कन्दपारिषदान्धोराञ्छृणुष्वाद्भुतदर्शनान्}
{वज्रप्हहारात्स्कन्दस्य जज्ञुस्तत्र कुमारकाः}
{ये हरन्ति शिशूञ्जातान्गर्भस्थाश्चैव दारुणाः}


% Check verse!
वज्रप्रहारात्कन्याश्च जज्ञिरेऽस् महाबलाः
\twolineshloka
{कुमारान्स विशखं च पुत्रत्वे समकल्पयत्}
{स भूत्वा भगवान्सङ्ख्ये रक्षंश्छागमुखस्तदा}


\twolineshloka
{वृतः कन्यागणैः सर्वैरात्मीयैः सह पुत्रकैः}
{मातृणां प्रेषितानां च भद्रशाखश्चकोमलैः}


\twolineshloka
{ततः कुमारं संजातं स्कन्दमाहुर्जना भुवि}
{रुद्रमग्निमुखां स्वाहां प्रदेशेषु महाबलाः}


\twolineshloka
{यजन्ति पुत्रकामाश्च पुत्रिणश्च सदा जनाः}
{यास्तास्त्वजनयत्कन्यास्तपो नाम हुताशनः}


% Check verse!
किं करोमीति ताः स्कन्दं संप्राप्ताः समभाषयन्
\twolineshloka
{भवेम सर्वलोकस्य मातरो वयमुत्तमाः}
{प्रसादात्तव पूज्याश्च प्रियमेतत्कुरुष्व नः}


\twolineshloka
{सोऽब्रवीद्बाढमित्येवं भविष्यध्वं पृथिग्विधाः}
{शिवाश्चैवाशिवाश्चैव पुनःपुनःरुदारधीः}


\twolineshloka
{`अग्निर्भूत्वा ततश्चैनं छागवक्रो बहुप्रजः}
{रमयामास शैलस्थं बलं क्रीडनकैरिव'}


\chapter{अध्यायः २२९}
\twolineshloka
{मार्कण्डेय उवाच}
{}


\threelineshloka
{ततः प्रकल्प्य पुत्रत्वे स्कन्दं मातृगणोऽगमत्}
{काकी च हलिमा चैव माता चाथ हली तथा}
{आर्या बाला च धात्री च सप्तैता शिशुमातरः}


\twolineshloka
{एतासां वीर्यसंपन्नः शिशुर्नामातिदारुणः}
{स्कन्दप्रसादजः पुत्रो लोहिताक्षो भयंकरः}


\twolineshloka
{एष वीरोऽष्टमः प्रोक्तः स्कन्दो मातृगणोद्भवः}
{छागवक्रेण सहितो नवमः परिकीर्त्यते}


\twolineshloka
{षष्ठं छागमयं वक्रं स्कन्दस्यैवेति विद्धि तम्}
{षट््शिरोभ्यन्तरं राजन्नित्यं मातृगणार्चितम्}


\twolineshloka
{षण्णां तु प्रवरं तस्य शीर्षाणामिह शब्द्यते}
{शक्तिं येनासृजद्दिव्यां भद्रशाख इति स्म ह}


\twolineshloka
{इत्येतद्द्विविधाकारं वृत्तं शुक्लस्य पञ्चमीम्}
{तत्र युद्धं महाघोरं वृत्तं षष्ठ्यां जनाधिप}


\twolineshloka
{उपविष्टं तु तं स्कन्दमामुक्तकवचस्रजम्}
{हिरण्यचूडमुकुटं हिरण्यांक्षं महाप्रभम्}


\twolineshloka
{लोहिताम्बरसंवीतं तीक्ष्णदंष्ट्रं मनोरमम्}
{सर्वलक्षणसंपन्नं त्रैलोक्यस्यापि सुप्रियम्}


\twolineshloka
{ततस्तं वरदं शूरं युवानं मृष्टकुण्डलम्}
{अभजत्पद्मरूपा श्रीः स्वयमेव शरीरिणी}


\twolineshloka
{श्रिया जुष्टः पृथुयशाः स कुमारो वरस्तदा}
{निषण्णो दृश्यते भूतैः पौर्णमास्यां यथा शशी}


% Check verse!
अपूजयन्महात्मानो ब्राह्मणास्तं महाबलम्
% Check verse!
इदमाहुस्तदा चैव स्कन्दं तत्र महर्षयः
\twolineshloka
{हिरण्यवर्ण भद्रं ते लोकानां शंकरो भव}
{त्वया षड्रात्रजातेन सर्वे लोका वशीकृताः}


\threelineshloka
{अभयंच पुनर्दत्तं त्वयैवैषां सुरोत्तम}
{तस्मादिन्द्रो भवानस्तु त्रैलोक्यस्याभयंकरः ॥स्कन्द उवाच}
{}


\threelineshloka
{किमिन्द्रः सर्वलोकानां करोतीह तपोधनाः}
{कथं देवगणांश्चैव पाति नित्यं सुरेश्वरः ॥ऋषय ऊचुः}
{}


\twolineshloka
{इन्द्रो दधाति भूतानां बलं तेजः प्रजाः सुखम्}
{तुष्टः प्रयच्छति तथा सर्वान्कामान्सुरेश्वरः}


\twolineshloka
{दुर्वृत्तानां संहरति व्रतस्थानां प्रयच्छति}
{अनुशास्ति च भूतानि कार्येषु बलसूदनः}


\twolineshloka
{असूर्ये च भवेत्सूर्यस्तथाऽचनद््रे च चन्द्रमाः}
{भवत्यग्निश्च वायुश्च पृथिव्यापश्च स्वं तथा}


\threelineshloka
{एतदिन्द्रेण कर्तव्यमिन्द्रे हि विपुलं बलम्}
{त्वं च वीर बली श्रेष्ठस्तस्मादिन्द्रो भवस्व न ॥शक्र उवाच}
{}


\threelineshloka
{भवस्वेन्द्रो महाबाहो सर्वेषां नः सुखावहः}
{अभिषिच्यस्व चैवाद्य प्राप्तरूपोऽसि सत्तम ॥स्कन्द उवाच}
{}


\threelineshloka
{शाधि त्वमेव त्रैलोक्यमव्याग्रे निजये रतः}
{अहं ते किंकरः शक्र न ममेन्द्रत्वमीप्सितम् ॥शक्र उवाच}
{}


\twolineshloka
{बलंतवाद्भुतं वीर त्वं देवानामरीञ्जहि}
{अवज्ञास्यन्ति मां लोका वीर्यण तव विस्मिताः}


\twolineshloka
{इन्द्रत्वे तु स्थितं वीर बलहीनं पराजितम्}
{`त्वत्तेजसाऽवमंस्यन्ति लोका मां सुरसत्तम'}


\twolineshloka
{आवयोश्च मिथो भेदे प्रयतिष्यन्त्यतन्द्रिताः}
{भेदिते च त्वयि विभो लोको द्वैधमुपेष्यति}


\twolineshloka
{द्विधाभूतेषु लोकेषु निश्चितेष्वावयोस्तथा}
{विग्रहः संप्रवर्तेत भूतभेदान्महाबल}


\threelineshloka
{तत्र त्वं मां रणे तात यथाश्रद्धं विजेष्यसि}
{तस्मादिन्द्रो भवानेव भविता मा विचारय ॥स्कन्द उवाच}
{}


\threelineshloka
{त्वमेव राजा भद्रं ते त्रैलोक्यस्य ममैव च}
{करोमि किं च तेशक्र शासनात्तद्ब्रवीहि मे ॥इन्द्र उवाच}
{}


\twolineshloka
{अहमिन्द्रो भविष्यामि तव वाक्यान्महाबल}
{यदि सत्यमिदं वाक्यं निश्चयाद्भाषितं त्वया}


\threelineshloka
{यदि वा शासनं स्कन्द कर्तुमिच्छसि मे शृणु}
{अभिषिच्यस्व देवानां सैनापत्ये महाबल ॥स्कन्द उवाच}
{}


\threelineshloka
{दानवानां विनाशाय देवानामर्थसिद्धये}
{गोब्राह्मणहितार्थाय सैनापत्येऽभिषिञ्च माम् ॥मार्कण्डेय उवाच}
{}


\twolineshloka
{सोऽभिषिक्तो मघवता सर्वैर्देवगणैः सह}
{अतीव शुशुभे तत्र पूज्यमानो महर्षिभिः}


\twolineshloka
{तत्र तत्काञ्चनं छत्रं ध्रियमाणं व्यरोचत}
{तथैव सुसमिद्धस्य पावकस्यात्ममण्डलम्}


\twolineshloka
{विश्वकर्मकृता चास्य दिव्या माला हिरण्मयी}
{आबद्धा त्रिपुरघ्नेन स्वयमेव यशस्विना}


\twolineshloka
{आगम्य मनुजव्याघ्र सह देव्या परंतप}
{अर्चयामास सुप्रीतो भगवान्गोवृषध्वजः}


\twolineshloka
{रुद्रमग्निं द्विजाः प्राहू रुद्रसूनुस्ततस्तु सः}
{`कीर्त्यते सुमहातेजाः कुमारोऽद्भुतदर्शनः'}


\twolineshloka
{रुद्रेण शुक्रमुत्सृष्टं तच्छ्वेतः पर्वतोऽभवत्}
{पावकस्येनद्रियं श्वेते कृत्तिकाभिः कृतं नगे}


\twolineshloka
{पूज्यमानं तु रुद्रेण दृष्ट्वा सर्वेदिवौकसः}
{रुद्रसूनुं ततः प्राहुर्गुहं गुणवतांवरम्}


\twolineshloka
{अनुप्रविश्य रुद्रेण वह्निं जातो ह्ययं शिशुः}
{तत्र जातस्ततः स्कन्दो रुद्रसूनुस्ततोऽभवत्}


\twolineshloka
{रुद्रस्य वह्नेः स्वाहायाः षण्णां स्त्रीणां च तेजसा}
{जातः स्कन्दः सुरश्रेष्ठो रुद्रसूनुस्ततोऽभवत्}


\twolineshloka
{अरजे वाससी रक्ते वसानः पावकात्मजः}
{भाति दीप्तवपुः श्रमान्रक्ताभ्राभ्यामिवांशुमान्}


\twolineshloka
{कुक्कुटश्चाग्निना दत्तस्तस् केतुरलंकृतः}
{रथे समुच्छितो भाति कालाग्निरव लोहितः}


\twolineshloka
{या चेष्टा सर्वभूतानां प्रभा शक्तिर्बलं तथा}
{अग्रतस्तस्य सा शक्तिर्देवानां जयवर्धनी}


\twolineshloka
{विवेश कवचं चास्य शरीरं सहजं तथा}
{युध्यमानस्य देवस्य प्रादुर्भवति तत्सदा}


\twolineshloka
{शक्तिर्धर्मो बलं तेजः कान्तत्वं सत्यमुन्नतिः}
{ब्रह्मण्यत्वमसंमोहो भक्तानां परिरक्षणम्}


\twolineshloka
{निकृन्तनं च शत्रूणां लोकानां चाभिरक्षणम्}
{स्कन्देन सह जातानि सर्वाण्येव जनाधिप}


\twolineshloka
{एवं देवगणैः सर्वैः सोऽभिषिक्तः स्वलंकृतः}
{बभौ प्रतीतः सुमनाः परिपूर्णेन्दुदर्शनः}


\twolineshloka
{इष्टैः स्वाध्यायघोषैश्च देवतूर्यवरैरपि}
{देवगन्धर्वगीतैश्च सर्वैरप्सरसां गणैः}


\threelineshloka
{एतैश्चान्यैश्च बहुभिस्तुष्टैर्हृष्टैः स्वलंकृतः}
{[सुसंवृतः पिशाचानां गणैर्देवगणैस्तथा}
{]क्रीडन्भाति तदा देवैरभिषिक्तश्च पावकिः}


\twolineshloka
{अभिषिक्तं महासेनमपश्यन्त दिवौकसः}
{विनिहत्य तमः सूर्यं यथैवाभ्युदितं तथा}


\twolineshloka
{अथैनमभ्ययुः सर्वा देवसेनाः सहस्रशः}
{अस्माकं त्वं पतिरिति ब्रुवाणाः सर्वतो दिशः}


\twolineshloka
{ताः समासाद्य भगवान्सर्वभूतगणैर्वृतः}
{अर्चितस्तु स्तुतश्चैव सान्त्वयामास ता अपि}


\twolineshloka
{शतक्रतुश्चाभिषिच्य स्कन्दं सेनापतिं तदा}
{सस्मार तां देवसेनां या सा तेन विमोक्षिता}


\twolineshloka
{अयं तस्याः पतिर्नूनं विहितो ब्रह्मणा स्वयम्}
{संचिन्त्य त्वानयामास देवसेनां ह्यलंकृताम्}


\twolineshloka
{स्कन्दं प्रोवाच बलभिदियं कन्या सुरोत्तम}
{अजाते त्वयि निर्दिष्टा तव पत्नी स्वयंभुवा}


\twolineshloka
{तस्मात्त्वमस्या विधिवत्पाणिं मन्त्रपुरस्कृतम्}
{गृहाण दक्षिणं देव्याः पाणिना पद्मवर्चसा}


\twolineshloka
{एवमुक्तः स जग्राह तस्याः पाणिं यथाविधि}
{बृहस्पतिर्मन्त्रविद्धि जजाप च जुहाव च}


\twolineshloka
{एवं स्कन्दस्य महिषीं देवसेनां विदुर्जनाः}
{षष्ठीं यां ब्राह्मणाः प्राहुर्लक्ष्मीमासां सुखप्रदाम्}


\twolineshloka
{सिनीबालीं कुहूं चैव सद्वृत्तिमपराजिताम्}
{`इत्येवमादिभिर्देवी नामभिः परिकीर्त्यते'}


\twolineshloka
{यदा स्कन्दः पतिर्लब्धः शाश्वतो देवसेनया}
{तदा तमाश्रयल्लक्ष्मीः स्वयं देवी शरीरिणी}


\twolineshloka
{श्रीजुष्टः पञ्चमीस्कन्दस्तस्माच्छ्रीः पञ्चमी स्मृता}
{षष्ठ्यां कृतार्थोऽभूद्यस्थात्तस्मात्षष्ठी महातिधिः}


\chapter{अध्यायः २३०}
\twolineshloka
{मार्कण्डेय उवाच}
{}


\twolineshloka
{श्रिया जुष्टं महासेनं देवसेनापतीकृतम्}
{सप्तर्षिपत्नयः षड् देव्यस्तत्सकाशमथागमन्}


\twolineshloka
{ऋषिभिः संपरित्यक्ता धर्मयुक्ता महाव्रताः}
{द्रुतमागम्य चोचुस्ता देवसेनापतिं प्रभुम्}


\twolineshloka
{वयं पुत्र परित्यक्तां भर्तृभिर्देवसंमितैः}
{अकारणाद्रुषा तैस्तु पुण्यस्थानात्परिच्युताः}


\twolineshloka
{अस्माभिः किल जातस्त्वमिति केनाप्युदाहृतम्}
{तत्सत्यमेतत्संश्रुत्य तस्मान्नस्त्रातुमर्हसि}


\threelineshloka
{अक्षयश्च भवेत्स्वर्गस्त्वत्प्रसादाद्धि नः प्रभो}
{त्वां पुत्रं चाप्यभीप्सामः कृत्वैतदनृणो भव ॥स्कन्द उवाच}
{}


\threelineshloka
{मातरो हि भवत्यो मे सुतो वोऽहमनिन्दिताः}
{यद्वापीच्छत तत्सर्वं संभविष्यति वस्तथा ॥मार्कण्डेय उवाच}
{}


\twolineshloka
{विवक्षन्तं ततः शक्रं किं कार्यमिति सोऽब्रवीत्}
{उक्तः स्कन्देन ब्रूहिति सोऽब्रवीद्वासवस्ततः}


\twolineshloka
{अभिजित्स्पर्धमाना तु रोहिण्या कन्यसी स्वसा}
{इच्छन्ती ज्येष्ठतां देवी तपस्तप्तुं वनं गता}


\twolineshloka
{तत्र मूढोस्मि भद्रं ते नक्षत्रं गगनाच्च्युतम्}
{कालं त्विमं परं स्कन्द ब्रह्मणा सह चिन्तय}


\twolineshloka
{धनिष्ठादिस्तदा कालो ब्रह्मणा परिकल्पितः}
{रोहिणो ह्यभवत्पूर्वमेवं सङ्ख्या समाभवत्}


\twolineshloka
{एवमुक्ते तु शक्रेण त्रिविदं कृत्तिका गताः}
{नक्षत्रं शकटाकारं भाति तद्वह्निदैवतम्}


\threelineshloka
{विनता चाब्रवीत्स्कन्दं मम त्वं पिण्डदः सुतः}
{इच्छामि नित्यमेवाहं त्वया पुत्र सहासितुम् ॥स्कन्द उवाच}
{}


\threelineshloka
{एवमस्तु नमस्तेऽस्तु पुत्रस्नेहात्प्रशाधि माम्}
{स्नुषया पूज्यमाना वै देवि वत्स्यसि नित्यदा ॥मार्कण्डेय उवाच}
{}


\threelineshloka
{अथ मातृगणः सर्वः स्कन्दं वचनमब्रवीत्}
{वयं सर्वस्य लोकस्य मातरः कविभिः स्तुताः}
{इच्छामो मातरस्तुभ्यं भवितुं पूजयस्व नः}


\fourlineindentedshloka
{`तासां तु वचनं श्रुत्वास्कन्दो वचनमब्रवीत्'}
{मातरो हि भवत्यो मे भवतीनामहं सुतः}
{उच्यतां यन्मया कार्यं भवतीनामथेप्सितम् ॥मातर ऊचुः}
{}


\twolineshloka
{यास्तु ता मातरः पूर्वं लोकस्यास्य प्रकल्पिताः}
{अस्माकं तु भवेत्स्थानं तासां चैव न तद्भवेत्}


\threelineshloka
{भवेम पूज्या लोकस्य न ताः पूज्याः सुरर्षभ}
{प्रजाऽस्माकं हृतास्ताभिस्त्वत्कृते ताः प्रयच्छ नः ॥स्कन्द उवाच}
{}


\threelineshloka
{वृत्ताः प्रजा न ताः प्रक्या भवतीभिर्निषेवितुम्}
{अन्प्रां वः कां प्रयच्छामि प्रजां यां मनसेच्छथा ॥मातर ऊचुः}
{}


\threelineshloka
{इच्चाम तासां मातॄणां प्रजा भोक्तुं प्रयच्छ नः}
{त्वया सह पृथग्भूता ये च तासामथेश्वराः ॥स्कन्द उवाच}
{}


\threelineshloka
{प्रजा वो दद्मि कष्टं तु भवतीभिरुदाहृतम्}
{परिरक्षत भद्रं वः प्रजा साधुनमस्कृताः ॥मातर ऊचुः}
{}


\threelineshloka
{परिरक्षाम भद्रं ते प्रजाः स्कन्द यथेच्छसि}
{त्वया नो रोचते स्कन्द सहवासश्चिरंप्रभो ॥स्कन्द उवाच}
{}


\twolineshloka
{यावत्षोडश वर्षाणि भवन्ति तरुणाः प्रजाः}
{प्रबाधत मनुष्याणां तावद्रूपैः पृथग्विधैः}


\threelineshloka
{अहं च वः प्रदास्यामि रौद्रमात्मानमव्ययम्}
{परमं तेन सहिताः सुखं वत्स्यथ पूजिताः ॥मार्कण्डेय उवाच}
{}


\twolineshloka
{ततः शरीरात्स्कन्दस्य परुषः पावकप्रभः}
{भोक्तुं प्रजाः स मर्त्यानां निष्पपात महाबलः}


\twolineshloka
{अपतत्सहसा भूमौ विसंज्ञोऽथ क्षुधार्दितः}
{स्कन्देन सोऽभ्यनुज्ञातो रौद्ररूपोऽभवद्ग्रहः}


\twolineshloka
{स्कन्दापस्मार इत्याहुर्गृहं तं द्विजसत्तमाः}
{विनता तु महारौद्रा कथ्यते शकुनिग्रहः}


\twolineshloka
{मातॄणां राक्षसंप्राहुस्तं विद्यात्पूतनाग्रहम्}
{कष्टा दारुणरूपेण घोररूपा निशाचरी}


\twolineshloka
{पशाची रदारुणाकारा कथ्यते शीतपूतना}
{गर्भान्सा मानुषीणां तु हरते घोरदर्शना}


\twolineshloka
{अदितिं रेवतीं प्राहुर्ग्रहस्तस्यास्तु रैवतः}
{सोऽपि बालान्महागोरो बाधते वै महाग्रहः}


\twolineshloka
{दैत्यानां या दितिर्माता तामाहुर्मुखमण्डिकाम्}
{अत्यर्थं शिशुमांसेन संप्रहृष्टा दुरासदा}


\twolineshloka
{कुमाराश्च कुमार्यश् ये प्रोक्ताः स्कन्दसंभवाः}
{तेऽपि गर्भभुजः सर्वे कौरव्यसुमहाग्रहाः}


\twolineshloka
{तासामेव तु पत्नीनां पतयस्ते प्रकीर्तिताः}
{आजायमानान्गृह्णन्ति बालकान्रौद्रकर्मणः}


\twolineshloka
{गवां माता तु या प्राज्ञैः कथ्यते सुरभिर्नृप}
{शकुनिस्तामथारुह्यसह भुङ्क्ते शिशून्भुवि}


\twolineshloka
{सरमा नाम या माता शुनां देवी जनाधिप}
{साऽपिगर्भान्समादत्ते मानुषीणां सदैव हि}


\twolineshloka
{पादपानां च या माता करञ्जनिलया हि सा}
{वरदा सा हि सौम्या च नित्यं भूतानुकम्पिनी}


\twolineshloka
{करञ्जे तां नमस्यन्ति तस्मात्पुत्रार्थिनो नराः}
{इमे त्वष्टादशान्ये वै ग्रहा मांसमधुप्रियाः}


\twolineshloka
{द्विपञ्चरात्रं तिष्ठन्ति सततं सूतिकागृहे}
{कद्रूः सूक्ष्मवपुर्भूत्वा गर्भिणीं प्रविशत्यथ}


\twolineshloka
{भुङ्क्ते सा तत्र तं गर्भं सा तु नागं प्रसूयते}
{गन्धर्वाणां तु या माता सा गर्भं गृह्य गच्छति}


\twolineshloka
{ततो विलीनगर्भा सा मानुषी भुवि दृश्यते}
{या जनित्री त्वप्सरसां गर्भमास्ते प्रगृह्य सा}


\twolineshloka
{उपविष्टं ततो गर्भं कथयनति मनीषिणः}
{लोहितस्योदधेः कन्या धात्री स्कन्दस्य सा स्मृता}


\twolineshloka
{लोहितायनिरित्येवं कदम्बे सा हि पूज्यते}
{पुरुषे तु यथा रुद्रस्तथाऽऽर्या प्रमदास्वपि}


\twolineshloka
{आर्या माता कुमारस्य पृथक्कामार्थमिज्यते}
{एवमेते कुमाराणां मया प्रोक्ता महाग्रहाः}


\twolineshloka
{यावत्षोडश वर्षाणि शिशूनां ह्यशिवास्ततः}
{ये च मातृगणाः प्रोक्ताः पुरुषाश्चैवये ग्रहाः}


\threelineshloka
{सर्वे स्कन्दग्रहा नाम ज्ञेया नित्यं शरीरिभिः}
{तेषां प्रशमनं कार्यं स्नानं धूपमथाञ्जनम्}
{बलिकार्गेपहाराश् स्कन्दस्येज्या विशेषतः}


\twolineshloka
{एवमभ्यर्चिताः सर्वे प्रयच्छन्ति शुभं नृणाम्}
{आयुर्दीर्घं च राजेनद्रसम्यक्पूजानमस्कृताः}


\twolineshloka
{ऊर्ध्वं तु षोडशाद्वर्षाद्ये भवन्ति ग्रहा नृणाम्}
{तानहं संप्रवक्ष्यामि नमस्कृत्य महेश्वरम्}


\twolineshloka
{यः पश्यति नरो देवाञ्जाग्रद्वा शयितोपि वा}
{उन्माद्यति स तु क्षिप्रं तं तु देवग्रहं विदुः}


\twolineshloka
{आसीनश्च शयानश्च यः पश्यति नरः पितॄन्}
{उन्माद्यतिस तु क्षिप्रं स ज्ञेयस्तु पितृग्रहः}


\twolineshloka
{अवमन्यति यः सिद्धान्क्रुद्धाश्चापि शपन्ति यम्}
{उन्माद्यति स तु क्षिप्रं ज्ञेयः सिद्धग्रहस्तु सः}


\twolineshloka
{उपाघ्राति च यो गन्धान्रसांश्चापि पृथग्विधान्}
{उन्माद्यति स तु क्षिप्रं स ज्ञेयो राक्षसो ग्रहः}


\twolineshloka
{गन्धर्वाश्चापि यं दिव्याः संविशन्ति नरं भुवि}
{उन्माद्यति स तु क्षिप्रं ग्रहो गान्धर्व एव सः}


\twolineshloka
{अधिरोहन्ति यं नित्यं पिशाचाः पुरुषं प्रति}
{उन्माद्यति स तु क्षिप्रं ग्रहः पैशाच एव सः}


\twolineshloka
{आविशन्ति च यं यक्षाः पुरुषं कालपर्यये}
{उनमाद्यति स तु क्षिप्रं ज्ञेयो यक्षग्रहस्तु सः}


\twolineshloka
{यस् दोषैः प्रकुपितं चित्तं मुह्यति देहिनः}
{उनमाद्यति स तु क्षिप्रं साधनं तस्य शास्त्रतः}


\twolineshloka
{वैक्लव्याच्च भयाच्चैव घोराणां चापि दर्शनात्}
{उन्माद्यति स तु क्षिप्रं सान्त्वं तस्य तु साधनम्}


\twolineshloka
{कश्चित्क्रीडितुकामो वै भोक्तुकामस्तथाऽपरः}
{अभिकामस्तथैवान्य इत्येष त्रिविधो ग्रहः}


\twolineshloka
{यावत्सप्ततिवर्षाणि भवन्त्येते ग्रहा नृणाम्}
{अतः परं देहिनां तु ग्रहतुल्या भवेञ्जरा}


\twolineshloka
{अप्रकीर्णेन्द्रियं दान्तं शुचिं नित्यमतन्द्रितम्}
{आस्तिकं श्रद्दधानं च वर्जयन्ति तदा ग्रहाः}


\twolineshloka
{इत्येष ते ग्रहोद्देशो मानुषाणां प्रकीर्तितः}
{न स्पृशन्ति ग्रहा भक्तान्नरान्देवं महेश्वरम्}


\chapter{अध्यायः २३१}
\twolineshloka
{मार्कण्डेय उवाच}
{}


\twolineshloka
{यदा स्कन्देन मातृणामेवमेतत्प्रियं कृतम्}
{अथनैमब्रवीत्स्वाहा मम पुत्रस्त्वमौरसः}


\threelineshloka
{इच्छाम्यहं त्वया दत्तां प्रीतिं परमदुर्लभाम्}
{तामब्रवीत्ततः स्कन्दः प्रीतिमिच्छसि कीदृशीम् ॥स्वाहोवाच}
{}


\twolineshloka
{दक्षस्याहं प्रिया कन्या स्वाहा नाम महाभुज}
{बाल्यात्प्रभृति नित्यं च जातकामा हुताशने}


\threelineshloka
{न स मां कामिनीं पुत्र सम्यग्जनाति पावकः}
{इच्छामि शाश्वतं वासं वस्तु पुत्र सहाग्निना ॥स्कन्द उवाच}
{}


\twolineshloka
{हव्यं कव्यंच यत्किंचिद्द्विजा मन्त्रसुसंस्तुतम्}
{होष्यन्त्यग्नौ सदा देवि स्वाहेत्युक्त्वासमुद्धृतम्}


\threelineshloka
{अद्यप्रभृति दास्यन्ति सुवृत्ताः सत्पथे स्थिताः}
{एवमग्निसत््वया सार्धं सदा वत्स्यति शोभने ॥मार्कण्डेय उवाच}
{}


\twolineshloka
{एवमुक्ता ततः स्वाहा तुष्टास्कन्देन पूजिता}
{पावकेन समायुक्ता भर्त्रा स्कन्दमपूजयत्}


\twolineshloka
{ततो ब्रह्मा महासेनं प्रजापतिरथाब्रवीत्}
{अभिगच्छ महादेवं पितरं त्रिपुरार्दनम्}


\twolineshloka
{रुद्रेणाग्निं समाविश्य स्वाहामाविश्य चोमया}
{हितर्थं सर्वलोकानां जातस्त्वमपराजितः}


\twolineshloka
{उमायोन्यां च रुद्रेण शुक्रं सिक्तं महात्मना}
{अस्मिन्गिरौ निपतितं मिञ्जिकमिञ्जिकं यतः}


\twolineshloka
{`मिथुनं वै महाभाग तत्र तद्रुद्रसंभवम्'}
{संभूतं लोहितोदे तु शुक्रशेषमवापतत्}


\twolineshloka
{सूर्यरश्मिषु चाप्यनयदन्यच्चैवापतद्भुवि}
{आसक्तमन्यद्वृक्षेषु तदेवं प्चधाऽपतत्}


\twolineshloka
{तत्र ते विविधाकारा गणा ज्ञेया मनीषिभिः}
{तव पारिषदा घोरा य एते पिशिताशिनः}


\threelineshloka
{एवमस्त्विति चाप्युक्त्वा महासेनो महेश्वरम्}
{अपूजयदमेयात्मा पितरं पितृवत्सलः ॥मार्कण्डेय उवाच}
{}


\twolineshloka
{अर्कपुष्पैस्तु ते पञ् गणाः पूज्या धनार्थिभिः}
{व्यादिप्रशमनार्थं चतेषां पूजां समाचरेत्}


\twolineshloka
{मिञ्जिकामिञ्जिकं चैव मिथुनं रुद्रसंभवम्}
{नमस्कार्यं सदैवेह बालानां हितमिच्छता}


\twolineshloka
{स्त्रियो मानुषमांसादा वृक्षका नाम नामतः}
{वृक्षेषु जातास्ता देव्यो नमस्कार्याः प्रजार्थिभिः}


\twolineshloka
{एषामेव पिशाचानामसङ्ख्येयगणाः स्मृताः}
{घण्टायाः सपताकायाः शृणु मे संभवं नृप}


\twolineshloka
{ऐरावतस्य घण्टे द्वे वैजयन्त्याविति श्रुते}
{गुहस्य ते स्वयं दत्ते क्रमेणानाय्य धीमता}


\twolineshloka
{एका तत्रविशाखस्य घण्टा स्कन्दस्य चापरा}
{पताका कार्तिकेयस् विशाखस्य च लोहिता}


\twolineshloka
{यानि क्रीडनकान्यस्य देवैर्दत्तानि वै तदा}
{तैरेव रमते देवो महासेनो महाबलः}


\twolineshloka
{स संवृतः पिशाचानां गणैर्देवगणैस्तथा}
{शुशुभे काञ्चने शैले दीप्यमानः श्रिया वृत्तः}


\twolineshloka
{तेन वीरेण शुशुभे स शैलः शुभकाननः}
{आदित्येनेवांशुमता मन्दरश्चारुकन्दरः}


\twolineshloka
{संतानकवनैः फुल्लैः करवीरवनैरपि}
{पारिजातवनैश्चैव जपाशोकवनैस्तथा}


\twolineshloka
{कदम्बतरुषण्डैश्च दिव्यैर्मृगगणैरपि}
{दिव्यैः पक्षिगणैश्चैव शुशुभे श्वेतपर्वतः}


\twolineshloka
{तत्रदेवगणाः सर्वेसर्वे देवर्षयस्तथा}
{मेघतूर्यरवाश्चैव क्षुब्धोदधिसमस्वनाः}


\twolineshloka
{तत्रदेवाश्च गन्धर्वा नृत्यन्तेऽप्सरसस्तथा}
{हृष्टानां तत्रभूतानां श्रूयते निनदो महान्}


\twolineshloka
{एवं सेन्द्रं जगत्सर्वं श्वेतपर्वतसंस्थितम्}
{प्रहृष्टं प्रेक्षते स्कन्दं न च ग्लायति दर्शनात्}


\chapter{अध्यायः २३२}
\twolineshloka
{मार्कण्डेय उवाच}
{}


\twolineshloka
{यदाऽभिषिक्तो भगवान्सैनापत्ये तु पावकिः}
{तदा संप्रस्थितः श्रीमान्हृष्टो भद्रवटं हरः}


\twolineshloka
{रथेनादित्यवर्णेन पार्वत्या सहितः प्रभुः}
{`अनुयातः सुरैः सर्वैः सहस्राक्षपुरोगमैः'}


\twolineshloka
{सहस्रं तस्य सिंहानां तस्मिन्युक्तं रथोत्तमे}
{उत्पपात दिवं शुभ्रं कालेनाभिप्रचोदितम्}


\twolineshloka
{तेपिबन्त इवाकाशं त्रासयन्तक्षराचरान्}
{सिंहा नभस्यगच्छन्त नदन्तश्चारुकेसराः}


\twolineshloka
{तस्मिन्रथे पशुपतिः स्थितो भात्युमया सह}
{विद्युता सहितः सूर्यः सेन्द्रचापे घने यथा}


\twolineshloka
{अग्रतस्तस्य भगवान्धनेशो गुह्यकैः सह}
{आस्थाय रुचिरं याति पुष्पकं नरवाहनः}


\twolineshloka
{ऐरावतं समास्थाय शक्रश्चापि सुरैः सह}
{पृष्ठतोऽनुययौ यान्तं वरदं वृषभध्वजम्}


\twolineshloka
{जृम्भकैर्यक्षरक्षोभिः स्रग्विभिः समलंकृतः}
{यात्यमोघो महायक्षो दक्षिणं पक्षमास्थितः}


\twolineshloka
{तस्य दक्षिणतो देवा बहवश्चित्रयोधिनः}
{गच्छन्ति वसुभिः सार्धं रुद्रैश्च सह संगताः}


\twolineshloka
{यमश्च मृत्युना सार्धं सर्वतः परिवारितः}
{घोरैर्व्याधिशतैर्याति घोररूपवपुस्तथा}


\twolineshloka
{यमस्य पृष्ठतश्चैव रघोरस्त्रिशिखरः शितः}
{विजयो नाम रुद्रस्य याति शूलः स्वलंकृतः}


\twolineshloka
{तमुग्रपाशो वरुणो भगवान्सलिलेश्वरः}
{परिवार्य शनैर्याति यादोभिर्विविधैर्वृतः}


\twolineshloka
{पृष्ठतोविजयस्यापि याति रुद्रस्य पट्टसः}
{गदामुसलशक्त्याद्यैर्वृतः प्रहरणोत्तमैः}


\twolineshloka
{पट्टसं त्वन्वगाद्राजंश्छत्रं रौद्रं महाप्रभम्}
{कमण्डलुश्चाप्यनु तं महर्षिगणसेवितः}


\twolineshloka
{तस्य दक्षिणतो भाति दण्डो गच्छञ्श्रिया वृतः}
{भृग्वङ्गिरोभिः सहितो दैधतैश्चानुपूजितः}


\twolineshloka
{एषां तु पृष्ठतो रुद्रो विमले स्यन्दने स्थितः}
{याति संहर्षयन्सर्वांस्तेजसा त्रिदिवौकसः}


\twolineshloka
{ऋषयश्चापि देवाश्च गन्धर्वा भुजगास्तथा}
{नद्यो ह्रदाःसमुद्राश्चतथैवाप्सरसां गणाः}


\twolineshloka
{नक्षत्राणि ग्रहाश्चैवदेवानां शिशवश्च ये}
{स्त्रियश्च विविधाकारा यान्ति रुद्रस्य पृष्ठतः}


\twolineshloka
{सृजन्त्यः पुष्पवर्षाणि चारुरूपा वराङ्गनाः}
{पर्जन्यश्चाप्यनुययौ नमस्कृत्य पिनाकिनम्}


\twolineshloka
{छत्रं च पाण्डुरं सोमस्तस्य मूर्धन्यधारयत्}
{चामरे चापि वायुश्च गृहीत्वाऽग्निश्च धिष्ठितौ}


\twolineshloka
{शक्रश्च पृष्ठतस्तस्य याति राजञ्छ्रिया वृतः}
{सहराजर्षिभिः सर्वैः स्तुवानो वृषकेतनम्}


\twolineshloka
{गौरी विद्याऽथ गान्धारी केशिनी मित्रसाह्वया}
{सावित्र्या सह सर्वास्ताः पार्वत्या यान्ति पृष्ठतः}


\twolineshloka
{तत्र विद्यागणाः सर्वेये केचित्कविभिः स्मृताः}
{तस्य कुर्वन्ति वचनं सेन्द्रा देवाश्चमूमुखे}


\threelineshloka
{गृहीत्वा तु पताकां वै यात्यग्रे राक्षसो ग्रहः}
{व्यापृतस्तु श्मशाने यो नित्यं रुद्रस् वै सखा}
{पिङ्गलो नाम यक्षेन्द्रो लोकस्यानन्ददायकः}


\twolineshloka
{एबिश्च सहितो देवस्तत्रयाति यथासुखम्}
{अग्रतः पृष्ठतस्चैवं न हि तस्य गतिर्ध्रुवा}


\twolineshloka
{रुद्रं सत्कर्मभिर्मर्त्याः पूजयन्तीह दैवतम्}
{शिवमित्येव यं प्राहुरीशं रुद्रं पिनाकिनम्}


\twolineshloka
{`एवं सर्वे सुरगणास्तदा वै प्रीतमानसाः'}
{भावैस्तु विविधाकारैः पूजयन्ति महेश्वरम्}


\twolineshloka
{कदेवसेनापतिस्त्वं देवसेनाभिरावृतः}
{अनुदच्छति देवेशं ब्रह्मण्यः कृत्तिकासुतः}


\threelineshloka
{अथाब्रवीन्महासेनं महादेवो बृहद्वचः}
{सप्तमं मारुतस्कन्धं रक्ष नित्यमतन्द्रितः ॥स्कन्द उवाच}
{}


\threelineshloka
{सप्तमं मारुतस्कन्धं पालयिष्याम्यहं प्रभो}
{यदन्यदपि मे कार्यं देव तद्वद माचिरम् ॥रुद्र उवाच}
{}


\threelineshloka
{कार्येष्वहं त्वया पुत्र संद्रष्टव्यः सदैव हि}
{दर्शनान्मम भक्त्या च श्रेयः परमवाप्स्यसि ॥मार्कण्डेय उवाच}
{}


\twolineshloka
{इत्युक्त्वा विससर्जैनं परिष्यज्य महेश्वरः}
{`स्कन्दं सहोमया प्रीतो ज्वलन्तमिव तेजसा'}


\twolineshloka
{विसर्जिते ततः स्कन्दे बभूवौत्पातिकं महत्}
{सहसैव महाराजदेवान्सर्वान्प्रमोहयत्}


\twolineshloka
{जज्वाल स्वं सनक्षत्रं प्रमूढं भुवनं भृशम्}
{चचाल व्यनदच्चोर्वी तमोभूतं जगत्प्रभो}


\twolineshloka
{ततस्तद्दारुणं दृष्ट्वा क्षुभितः शंकरस्तदा}
{उमा चैवमहाभाग देवाश्च समहर्षयः}


\twolineshloka
{ततस्तेषु प्रमूढेषु पर्वताम्बुदसन्निभम्}
{नानाप्रहरणं घोरमदृश्यत महद्बलम्}


\twolineshloka
{तद्वै घोरमसङ्ख्येयं गर्जच्च विविधा गिरः}
{अभ्यद्रवद्रणए देवान्भगवन्तं च शंकरम्}


\twolineshloka
{तैर्विसृष्टान्यनीकेषु बाणजालान्यनेकशः}
{पर्वताश्च शतघ्न्यश्च प्रासासिपरिघा गदाः}


\twolineshloka
{निपतद्भिश्च तैर्घोरैर्देवानीकं महायुधैः}
{क्षणेन व्यद्रवत्सर्वं विमुखं चाप्यदृश्यत}


\twolineshloka
{निकृत्तयोधनागाश्वं कृत्तायुधमहारथम्}
{दानवैरर्दितं सैन्यं देवानां विमुखं बभौ}


\twolineshloka
{असुरैर्वध्यमानं तत्पावकैरिव काननम्}
{अपतद्दग्धभूयिष्ठं महाद्रुमवनं तथा}


\twolineshloka
{ते विभिन्नशिरोदेहाः प्राद्रवन्तो दिवौकसः}
{न नाथमधिगच्छन्ति वध्यमाना महारणे}


\twolineshloka
{अथ तद्विद्रुतं सैन्यं दृष्ट्वा देवः पुरंदरः}
{आश्वासयन्नुवाचेदं बलभिद्दानवार्दितम्}


\twolineshloka
{भयं त्यजत भद्रं व शूराः शस्त्राणि गृह्णत}
{कुरुध्वंविक्रमे बुद्धिं मा वः काचिद्व्यथा भवेत्}


\twolineshloka
{जयतैनाम्सुदुर्वृत्तान्दानवान्घोरदर्शनान्}
{अभिद्रवत भद्रं वो मया सह महासुरान्}


\twolineshloka
{शक्रस् वचनं श्रुत्वा समाश्वस्ता दिवौकसः}
{दानवान्प्रत्ययुध्यन्त शक्रं कृत्वा व्यपाश्रयम्}


\twolineshloka
{ततस्ते त्रिदशाः सर्वेमरुतश्च महाबलाः}
{प्रत्युद्ययुर्महाभागाः साध्याश्च वसुभिः सह}


\twolineshloka
{तैर्विसृष्टान्यनीकेषु क्रुद्धैः शस्त्राणि संयुगे}
{शराश्च दैत्यकायेषु पिबन्ति रुधिरं बहु}


\twolineshloka
{तेषां देहान्विनिर्भिद्य शरास्ते निशितास्तदा}
{निपतन्तोऽभ्यदृश्यन्त नगेभ्य इव पन्नगाः}


\twolineshloka
{तानि दैत्यशरीराणि निर्भिन्नानिस्म सायकैः}
{अपतन्भूतलेराजंश्छिन्नाभ्राणीव सर्वशः}


\twolineshloka
{ततस्तद्दानवं सैन्यं सर्वैर्देवगणैर्युधि}
{त्रासितं विविधैर्बाणैः कृतंचैव पराङ्युखम्}


\twolineshloka
{अथोत्क्रुष्टं तदा हृष्टैः सर्वैर्देवैरुदायुधैः}
{संहतानि च तूर्याणि प्रावाद्यन्त ह्यनेकशः}


\twolineshloka
{एवमन्योन्रसंयुक्तं युद्धमासीत्सुदारुणम्}
{देवानां दानवानां च मांसशोणितकर्दमम्}


\twolineshloka
{अनयो देवलोकस्य सहसैवाभ्यदृश्यत}
{तथहि दानवा घोरा विनिघ्नन्ति दिवौकसः}


\twolineshloka
{ततस्तूर्यप्रणादाश्च भेरीणां च महास्वनाः}
{बभूवुर्दानवेन्द्राणां सिंहनादाश्च दारुणाः}


\twolineshloka
{अथ दैत्यबलाद्घोरान्निष्पपात महाबलः}
{दानवो महिषो नाम प्रगृह्य विपुलं गिरिम्}


\twolineshloka
{ते तं धनैरिवादित्यं दृष्ट्वासंपरिवारितम्}
{तमुद्यतगिरिं राजन्व्यद्रवन्त दिवौकसः}


\twolineshloka
{अथाभिद्रुत्य महिषो देवांश्चिक्षेप तं गिरिम्}
{`महाकायं महाराज सतोयमिव तोयदम्'}


\twolineshloka
{पतता तेन गिरिणा देवसैन्यस् पार्थिव}
{भीमरूपेण निहतमयुतं प्रापतद्भुवि}


\twolineshloka
{अथ तैर्दानवैः सार्धं महिषस्त्रासयन्सुरान्}
{अभ्यद्रवद्रणे तूर्णं सिंहः क्षुद्रमृगानिव}


\twolineshloka
{तमापतन्तं महिषं दृष्ट्वा सेन्द्रा दिवौकसः}
{व्यद्रवन्तरणे बीता विकीर्णायुधकेतनाः}


\twolineshloka
{ततःस महिषः क्रुद्धस्तूर्णं रुद्ररथं ययौ}
{अभिद्रुत्य च जग्राह रुद्रस् रथकूवरम्}


\twolineshloka
{यदा रुद्ररथं क्रुद्धो महिषः सहसा गतः}
{रेसतू रोदसी गाढं मुमुहुश् महर्षयः}


\twolineshloka
{अनदंश्चमहाकाया दैत्या जलधरोपमाः}
{आसीच्चनिश्चितं तेषां जितमस्माभिरित्युत}


\threelineshloka
{तथाभूते तु भगवानाहूय गुहमात्मजम्}
{`दौरात्म्यं पश्य पुतर् त्वं दानवस्य दुरात्मनः}
{जहि शीघ्रं दुराचारं द्रष्टुमिच्छामि ते बलम्}


\threelineshloka
{इत्युक्ताव भगवान्स्कन्दं परिपूज्य महेश्वरः}
{अयोजयन्निग्रहार्थं महिषस् गतायुषः'}
{सस्मार च तदा स्कन्दं मृत्युं तस्य दुरात्मनः}


\twolineshloka
{महिषोऽपि रथं दृष्ट्वा रौद्रं रुद्रस् चानदत्}
{देवान्संत्रासयंश्चापि दैत्यांस्चापिप्रहर्षयत्}


\twolineshloka
{ततस्तस्मिनभये घोरे देवानां समुपस्थिते}
{आजगाम महासेनः क्रोधात्सूर्य इव ज्वलन्}


\twolineshloka
{लोहिताम्बरसंवीतो लोहितस्रग्विभूषणः}
{लोहिताश्वो महाबाहुर्हिरण्यकवचः प्रभुः}


\twolineshloka
{रथमादित्यसंकाशमास्थितः कनकप्रभम्}
{तं दृष्ट्वा दैत्यसेना सा व्यद्रवत्सहसा रणे}


\twolineshloka
{स चापि तां प्रज्वलितां महिषस् विदारिणीम्}
{मुमोच शक्तिं राजेनद््र महासेनो महाबलः}


\twolineshloka
{सा मुक्ताऽभ्यहरत्तस्य महिषस्य शिरो महत्}
{पपात भिन्ने शिरसि महिषस्त्यक्तजीवितः}


\threelineshloka
{पतता शिरसा तेन द्वारं षोडशयोजनम्}
{पर्वताभेन पिहितं तदाऽगम्यं ततोऽभवत्}
{उत्तराः कुरवस्तेन गच्छन्त्यद्य यथासुम्}


\twolineshloka
{क्षिप्ताक्षिप्ता तु सा शक्तिर्हत्वा शत्रून्सहस्रशः}
{स्कन्दहस्तमनुप्राप्ता दृश्यते देवदानवैः}


\threelineshloka
{प्रायः शरैर्विनिहता महासेनेन धीमता}
{शेषा दैत्यगणा घोरा भीतास्त्रस्ता दुरासदैः}
{स्कन्दपारिषदैर्हत्वा भक्षिताश्च सहस्रशः}


\twolineshloka
{दनवान्भक्षयन्तस्ते प्रपिबन्तश्च शोणितम्}
{क्षणान्निर्दानवं सर्वमकार्षुर्भृशहर्षिताः}


\twolineshloka
{तमांसीवयथा सूर्यो वृक्षानग्निर्घनान्खगः}
{तथास्कन्दोऽजयच्छत्रून्स्वेन वीर्येण कीर्तिमान्}


\twolineshloka
{संपूज्यमानस्त्रिदशैरभिवाद्य महेश्वरम्}
{शुशुभे कृतिकापुत्रः प्रकीरणांशुरिवांशुमान्}


\twolineshloka
{नष्टशत्रुर्यदा स्कन्दः प्रयातस्तु महेश्वरम्}
{तदाऽब्रवीन्महासेनं परिष्वज्य पुरंदरः}


\twolineshloka
{ब्रह्मदत्तवरः स्कन्द त्वयाऽयं महिषो हतः}
{`हजय्यो युधि देवानां दानवः स महाबलः'}


\twolineshloka
{देवास्तृणसमा यस्य वबूवुर्जयतांवर}
{सोऽयं त्वया महाबाहो शमितो देवकण्टकः}


\twolineshloka
{शतं महिषतुल्यानां दानवानां त्वय रणे}
{निहतंदेवशत्रूणां यैर्वयं पूर्वतापिताः}


\twolineshloka
{तावकैर्भक्षिताश्चान्ये दानवाः शतसङ्घशः}
{अजेयस्त्वं रणेऽरीणामुमापतिरिव प्रभुः}


\threelineshloka
{एतत्ते प्रथमं देव ख्यातं कर्म भविष्यति}
{त्रिषु लोकेषु कीर्तिश्च तवाक्षय्या भविष्यति}
{वशगाश्च भविष्यनति सुरास्तव महाभुज}


\twolineshloka
{महासेनमेवमुक्त्वा निवृत्तः सह दैवतैः}
{अनुज्ञातो भगवता त्र्यम्बकेण शचीपतिः}


\twolineshloka
{गतो भद्रवटंरुद्रो निवृत्ताश्च दिवौकसः}
{उक्ताश्च देवा रुद्रेण स्कन्दं पश्यत मामिव}


\twolineshloka
{स हत्वा दानवगणान्पूज्यमानो महर्षिभिः}
{एकाह्नैवाजयत्सर्वं त्रैलोक्यं वह्निनन्दनः}


\twolineshloka
{स्कन्दस् य इदं विप्रः पठेज्जन्म समाहितः}
{`शृणुयाद्ब्राह्मणेभ्यो यः श्रावयेद्वाविचेतनम्}


\twolineshloka
{धनमायुर्यशो दीप्तिं पुत्राञ्शत्रुजयंतथा'}
{सपुष्टिमिहसंप्राप्य स्कन्दसालोक्यमाप्नुयात्}


\chapter{अध्यायः २३३}
\twolineshloka
{युधिष्ठिर उवाच}
{}


\threelineshloka
{भगवञ्श्रोतुमिच्छामि नामानि च महात्मनः}
{तरिषु लोकेषु यान्यस्य विख्यातानि द्विजोत्तम ॥वैशंपायन उवाच}
{}


\twolineshloka
{इत्युक्तः पाण्डवेयेन महात्मा ऋषिसन्निधौ}
{उवाच भगवांस्तत्र मार्कण्डेयो महातपाः}


\twolineshloka
{आगेयश्चैव स्कनदश्च दीप्तकीर्तिरनामयः}
{मयूरकेतुर्धर्मात्मा भूतेशो महिषार्दनः}


\twolineshloka
{कामजित्कामदः कान्तः सत्यवाग्भुवनेश्वरः}
{शिशुः शीघ्रः शुचिश्चण्डो दीप्तवर्णः शुभाननः}


\twolineshloka
{अमोघस्त्वनघो रौद्रः प्रियश्चन्द्राननस्तथा}
{दीप्तशक्तिः प्रशान्तात्मा नद्रकुक्कुटमोहनः}


\twolineshloka
{षष्ठीप्रियश्च धर्मात्मा पवित्रो मातृवत्सलः}
{कन्याभर्ता विभक्तश्च स्वाहेयो रेवतीसुतः}


\twolineshloka
{प्रभुर्नेता विशाखश्च नैगमेयः सुदुश्चरः}
{सुव्रतो ललितश्चैवबालक्रीडनकप्रियः}


\threelineshloka
{खचारी ब्रह्मचारी च शूरः शरवणोद्भवः}
{विश्वामित्रप्रियश्चैव देवसेनाप्रियस्तथा}
{वासुदेवप्रियश्चैव प्रियः प्रियकृदेव तु}


\threelineshloka
{नामान्येतानि दिव्यानि कार्तिकेयस्य यः पठेत्}
{स्वर्गं कीर्तिं धनं चैव स लभेन्नात्र संशयः ॥मार्कण्डेय उवाच}
{}


\twolineshloka
{स्तोष्यामि देवैर्ऋषिभिश्च जुष्टंशक्त्या गुहंनामभिरप्रमेयम्}
{षडाननं शक्तिधरं सुवीरंनिबोध चैतानि कुरुप्रवीर}


\twolineshloka
{ब्रह्मण्यो वै ब्रहमजो ब्रह्मविच्चब्रह्मेशयो ब्रह्मवतांवरिष्ठः}
{ब्रह्मप्रियो ब्राह्मणसर्वमन्त्रीत्वं ब्रह्मणां ब्राह्मणानांच नेता}


\twolineshloka
{स्वाहा स्वधा त्वंपरमं पवित्रंमन्त्रस्तुतस्त्वंप्रथितः षडर्चिः}
{संवत्सरस्त्वमृतवश्च षड्वैमासार्धमासाश्चदिनं दिशश्च}


\twolineshloka
{त्वंपुष्कराक्षस्त्वरविन्दवक्रःसहस्रचक्षोसि सहस्रबाहुः}
{त्वं लोकपालः परमं हविश्चत्वं भावनः सर्वसुरासुराणाम्}


\twolineshloka
{त्वमेव सेनाधिपतिः प्रचण्डःप्रभुर्विभुश्चाप्यथ शक्रजेता}
{सहस्रपात्त्वं धरणी त्वमेवसहस्रतुष्टिश्च सहस्रभुक्व}


\threelineshloka
{सहस्रशीर्षस्त्वमनन्तरूपः}
{सहस्रपात्त्वंदशशक्तिधारी}
{गङ्गासुतस्त्वं स्वमतेन देवस्वाहामहीकृत्तिकानां तथैव}


\twolineshloka
{त्वं क्रीडसे षण्मुख कुक्कुटेनयथेष्टनानाविधकामरूपी}
{दीक्षाऽसि सोमो मरुतः सदैवधर्मोऽसि वायुरचलेन्द्र इन्द्रः}


\twolineshloka
{सनातनानामपि शाश्वतस्त्वंप्रभुः प्रभूणामपि चोग्रधन्वा}
{ऋतस्य कर्ता दितिजान्तकस्त्वंजता रिपूणां प्रवरः सुराणाम्}


\twolineshloka
{सूक्ष्मं तपस्तत्परमं त्वमेवपरावरज्ञोसि परावरस्त्वम्}
{धर्मस्य कामस्य परस्य चैवत्वत्तेजसा कत्स्नमिदं महात्मन्}


\twolineshloka
{व्याप्तं जगत्सर्वसुरप्रवीरशक्त्या मया संस्तुत लोकनाथ}
{नमोस्तु ते द्वादशनेत्रबाहोअतः परं वेद्मि गतिं न तेऽहम्}


\twolineshloka
{स्कन्दस्य य इदं विप्रः पठेज्जन्म समाहितः}
{श्रावयेद्ब्राह्मणेभ्यो यः शृणुयाद्वा द्विजेरितम्}


\twolineshloka
{धनमायुर्यशो दीप्तं पुत्राञ्शत्रुजयं तथा}
{स पुष्टितुष्टी संप्राप्य स्कन्दसालोक्यमाप्नुयात्}


\chapter{अध्यायः २३४}
\twolineshloka
{वैशंपायन उवाच}
{}


\twolineshloka
{उपासीनेषु विप्रेषु पाण्डवेषु च भारत}
{द्रौपदी सत्यभामा च विविशाते तदा समम्}


\twolineshloka
{`प्रविश्य चाश्रमं पुण्यमुभे ते परमस्त्रियौ'}
{जाहस्यमाने सुप्रीते सुखं तत्र निषीदतुः}


\twolineshloka
{चिरस् दृष्ट्वा राजेन्द्र तेऽन्योन्यस्य प्रियंवदे}
{कथयामासतुश्चित्राः कथाः कुरुयदूचिताः}


\twolineshloka
{अथाब्रवीत्सत्यभामा कृष्णस् महिषी प्रिया}
{सात्राजिती याज्ञसेनीं रहसीदं सुमध्यमा}


\twolineshloka
{केन द्रौपदि वृत्तेन पाण्डवानधितिष्ठसि}
{लोकपालोपमान्वीरान्नूनं परमसंमतान्}


\twolineshloka
{कथं च वशगस्तुभ्यं न कुप्यन्ति च ते शुभे}
{तव वश्या हि सतत पाण्डवाः प्रियदर्शने}


\twolineshloka
{`न चान्योन्यमसूयन्ते कथं वा ते सुमध्यमे'}
{मुखप्रेक्षाश्च ते सर्वे तत्त्वमेतद्ब्रवीहि मे}


\twolineshloka
{व्रतचर्या तपो वाऽपि स्नानमन्त्रौषधानि वा}
{विद्यावीर्यं मूलवीर्यंजपहोमागदास्तथा}


\twolineshloka
{ममाद्याचक्ष्वपाञ्चालि यशस्यं भगवेतनम्}
{येन कृष्णे भवेन्नित्यं मम कृष्णो वशानुगः}


\twolineshloka
{एवमुक्त्वासत्यभामा विरराम यशस्विनी}
{पतिव्रता महाभागा द्रौपदी प्रत्युवाच ताम्}


\twolineshloka
{असत्स्त्रीणां समाचरं सत्ये मामनुपृच्छसि}
{असदाचरिते मार्गे कथं स्यादनुकीर्तनम्}


\twolineshloka
{अनुप्रश्नः संशयो वा नैष त्वय्युपपद्यते}
{कथं ह्युपेता बुद्ध्या त्वंकृष्णस् महिषी प्रिया}


\twolineshloka
{यदैव भर्ता जानीयानमन्त्रमूलपरां स्त्रियम्}
{उद्विजेत तदैवास्याः सर्पाद्वेश्मगतादिव}


\twolineshloka
{उद्विग्नस्य कुतः शान्तिरशान्तस्य कुतः सुखम्}
{न जातु वशगो भर्ता स्त्रियाः स्यान्मन्त्रकारणात्}


\twolineshloka
{किमत्रप्रहिताश्चापि गदाः परमदारुणाः}
{मूलप्रवादैर्हि विषं प्रयच्छन्ति जिघांसवः बब}


\twolineshloka
{जिह्वया यानि पुरुषस्त्वचा वाप्युपसेवते}
{तत्र चूर्णानि दत्तानि हन्युः क्षिप्रमसंशयम्}


\twolineshloka
{जलोदरसमायुक्ताः श्वित्रिणः पलितास्तथा}
{अपुमांसः कृताः स्त्रीभिर्जडान्धवधिरास्तथा}


\twolineshloka
{पापानुगास्तु पापास्ताः पतीनुपसृजन्त्युत}
{न जातु विप्रियं भर्तुः स्त्रिया कार्यं कथंचन}


\twolineshloka
{वर्ताम्यहं तु यां वृत्तिं पाण्डवेषु महात्मसु}
{तां सर्वां शृणु मे सत्यां सत्यभामे यशस्विनि}


\twolineshloka
{अहंकारं विहायाहं कामक्रोधौ च सर्वदा}
{सदारान्पाण्डवान्नित्यं प्रयतोपचराम्यहम्}


\twolineshloka
{प्रणयं प्रतिसंहृत्य निधायात्मानमात्मनि}
{शुश्रूषुर्निरभीमाना पतीनां चित्तरक्षिणी}


\twolineshloka
{दुर्व्याहृताच्छङ्कमाना दुस्थिताद्दुरवेक्षितात्}
{दुरासिताद्दुर्व्रजितादिङ्गिताध्यासितादपि}


\twolineshloka
{सूर्यवैश्वानरसमान्सोमकल्पान्महारथान्}
{सेवे चक्षुर्हणः पार्थानुग्रवीर्यप्रतापिनः}


\twolineshloka
{देवो मनुष्यो गन्धर्वो युवा चापि स्वलंकृतः}
{द्रव्यवानभिरूपो वा न मेऽन्यः पुरुषो मतः}


\twolineshloka
{न भुक्तवति न स्नाते नासंविष्टे च भर्तरि}
{न संविशामि नाश्नामि न स्नाये कर्म कुर्वती}


\twolineshloka
{क्षेत्राद्वनाद्वा ग्रामाद्वा भर्तारं गुहमागतम्}
{अभ्युत्थायाभिनन्दामि आसनेनोदकेन च}


\twolineshloka
{प्रसन्नभाण्डा मृष्टान्ना काले भोजनदायिनी}
{संयता गुप्तधान्या च सुसंमृष्टनिवेशना}


\twolineshloka
{अतिरस्कृतसंभाषा दुःस्त्रियो नानुसेवती}
{अनुकूलवती नित्यं भवाम्यनलसा सदा}


\twolineshloka
{अनर्म चापि हसितं द्वारि स्थानमभीक्ष्णशः}
{अवस्करे चिरस्थानं निष्कुटेषु च वर्जये}


\twolineshloka
{`अत्यालापमसन्तोषं परव्यापारसंकथाः'}
{अतिहासातिरोषौ च क्रोधस्थानं च वर्जये}


\twolineshloka
{निरताऽहं सदा सत्ये पापानां च विवर्जने}
{सर्वथा भर्तुरहितं न ममेष्टं कथंचन}


\twolineshloka
{यदा प्रवसते भर्ता कुटुम्बार्थेन केनचित्}
{सुमनोवर्णकापेता भवामि व्रतचारिणी}


\twolineshloka
{यच्च भर्ता न पिबति यच्च भर्ता न सेवते}
{यच्च नाश्नाति मे भर्ता सर्वं तद्वर्जयाम्यहम्}


\twolineshloka
{यथोपदेशं नियता वर्तमाना वराङ्गने}
{स्वलंकृता सुप्रयता भर्तुः प्रियहिते रता}


\twolineshloka
{ये च धर्माः कुटुम्बेषु श्वश्र्वामे कथिताः पुरा}
{`अनुतिष्ठामि तान्सत्ये नित्यकालमतन्द्रिता'}


\twolineshloka
{भिक्षाबलिश्राद्धविधिस्थालीपाकाश्च पर्वसु}
{मान्यानां मानसत्कारा ये चान्ये विदिता मम}


\twolineshloka
{तान्सर्वाननुवर्तामि दिवारात्रमतन्द्रिता}
{विनयाननियमांश्चैव सदा सर्वात्मना श्रिता}


\twolineshloka
{मृदून्सतः सत्यशीलान्सत्यधर्मानुपालिनः}
{स देवः सा गतिर्नार्यास्तस्य का विप्रियं चरेत्}


\twolineshloka
{पत्याश्रयो हि मे धर्मो मतः स्त्रीणां सनातनः}
{स देवः सा गतिर्नार्यास्तस्य काविप्रियं चरेत्}


\twolineshloka
{अहं पतीन्नातिशये नात्यश्ने नातिभूषये}
{नापि श्वश्रूं परिवदे सर्वदा परियन्त्रिता}


\twolineshloka
{अवधानेन सुभगे नित्योत्थिततयैव च}
{भर्तारो वशगा मह्यं गुरुशुश्रूषयैव च}


\twolineshloka
{नित्यमार्यामहं कुन्तीं वीरसूं सत्यवादिनीम्}
{स्वयं परिचराम्यतां पानाच्छादनभोजनैः}


\twolineshloka
{नैतामतिशये जातु वस्त्रभूषणभोजनैः}
{न वदे चाप्यतिवाचा तां पृथां पृथिवीसमाम्}


\twolineshloka
{अष्टावग्रे ब्राह्मणानां सहस्राणि स्म नित्यदा}
{भुञ्जते रुक्मपात्रीषु युधिष्ठिरनिवेशने}


\twolineshloka
{अष्टाशीतिसहस्राणि स्नातका गृहमेधिनः}
{त्रिंशद्दासीक एकैको यान्विभर्ति युधिष्ठिरः}


\twolineshloka
{दशान्यानि सहस्राणि येषामन्नं सुसंस्कृतम्}
{ह्रियते रुक्मपात्रीभिर्यतीनामूर्ध्वरेतसाम्}


\twolineshloka
{तान्सर्वानग्रहारेण ब्राह्मणान्वेदवादिनः}
{यथार्हं पूजयामि स्म पानाच्छादनभोजनैः}


\twolineshloka
{शतं दासीसहस्राणि कौन्तेयस्य महात्मनः}
{कम्बुकेयूरधारिण्यो निष्ककण्ठ्यः स्वलंकृताः}


\twolineshloka
{महार्हमाल्याभरणाः सुवर्णाश्चन्दनोक्षिताः}
{मणीन्हेम च विभ्रत्यो नृत्तगीतविशारदाः}


\twolineshloka
{तासां नाम च रूपंच भोजनाच्छादनानि च}
{सर्वासामेव वेदाहं कर्म चैव कृताकृतम्}


\twolineshloka
{शतं दासीसहस्राणि कुन्तीपुत्रस्य धीमतः}
{पात्रीपस्ता दिवारात्रमतिथीन्भोजयन्त्युत}


\twolineshloka
{शतमश्वसहस्राणि दशनागायुतानि च}
{युधिष्ठिरस्यानुयात्रमिन्द्रप्रस्थनिवासिनः}


\twolineshloka
{एतदासीत्तदा राज्ञो यन्महीं पर्यपालयत्}
{येषां सङ्ख्याविधिं चैव प्रदिशामि शृणोमि च}


\twolineshloka
{अन्तःपूराणां सर्वेषां भृत्यानां चैव सर्वशः}
{आगोपालाविपालेभ्यः सर्वं वेद कृताकृतम्}


\twolineshloka
{सर्वं राज्ञः समुदयमायं च व्ययमेव च}
{एकाऽहंवेद्मि कल्याणि पाण्डवानां यशस्विनि}


\twolineshloka
{मयि सर्वं समासज्यकुटुम्बं भरतर्षभाः}
{उपासनरताः सर्वे घटयन्ति वरानने}


\twolineshloka
{तमहं भारमासक्तमनाधृष्यं दुरात्मभिः}
{सुखं सर्वंपरित्यज्यरात्र्यहानि घटामि वै}


\twolineshloka
{अधृष्यं वरुणस्येव निधिपूर्णमिवोदधिम्}
{एकाहं वेद्मि कोशं वै पतीनां धर्मचारिणाम्}


\twolineshloka
{अनिशायां निशायां च विहाय क्षुत्पिपासयोः}
{आराधयन्त्याः कौरव्यांस्तुल्या रात्रिरहश्च मे}


\twolineshloka
{प्रथमं प्रतिबुध्यामि चरमं संविशामि च}
{नित्यकालमहं सत्ये एतत्संवननं मम}


\threelineshloka
{एतज्जानाम्यहं कर्तुं भर्तृसंवननं महत्}
{असत्स्त्रीणां समाचारं नाहं कुर्यां न कामये ॥वैशंपायन उवाच}
{}


\twolineshloka
{तच्छ्रुत्वा धर्मसहितं व्याहृतं कृष्णया तदा}
{उवाच सत्या सत्कृत्य पाञ्चालीं धर्मचारिणीं}


\twolineshloka
{अभिपन्नाऽस्मि पाञ्चालि याज्ञसेनि क्षमस्व मे}
{कामकारः सखीनं हि सोपहासं प्रभाषितम्}


\chapter{अध्यायः २३५}
\twolineshloka
{द्रौपद्युवाच}
{}


\twolineshloka
{इमं तु ते मार्गमपेतदोषंवक्ष्यामि चित्तग्रहणाय भर्तुः}
{अस्मिन्यथावत्सखि वर्तमानाभर्तारमाच्छेत्स्यसि कामिनीभ्यः}


\twolineshloka
{नैतादृशं दैवतमस्ति किंचि-त्सर्वेषुलोकेषु सदेवकेषु}
{यथा पतिस्तस्य तु सर्वकामालभ्याः प्रसादे कुपितश् हन्यात्}


\twolineshloka
{तस्मादपत्यं विविधाश् भोगाःशय्यासनान्यद्भुतदर्शनानि}
{वस्त्राणि माल्यानि तथैव गन्धाःस्वर्गश्चलोको विपुला च कीर्तिः}


\twolineshloka
{सुखं सुखनेह न जातु लभ्यंदुःखेन साध्वी लभते सुखानि}
{सा कृष्णमाराधय सौहृदेनप्रेम्णा च नित्यं प्रतिकर्मणा च}


\twolineshloka
{स्नानासनैश्चारुभिरग्रमाल्यै-र्दाक्षिण्ययोगैर्विविधैश्च गन्धैः}
{अस्याः प्रियोस्मीति यथा विदित्वात्वामेव संश्लिष्यति सर्वभावैः}


\twolineshloka
{श्रुत्वा स्वरं द्वारगतस् भर्तुःप्रत्युत्थिता तिष्ठ गृहस्य मध्ये}
{दृष्ट्वा प्रविष्टं त्वरिताऽसनेनपाद्येन चैनं प्रतिपूजयस्व}


\twolineshloka
{संप्रेषितायामथ चैव दास्या-मुत्थाय सर्वं स्वयमेव कार्यम्}
{जानातु कृष्णस्तव भावमेतंसर्वात्मना मां भजतीति सत्ये}


\twolineshloka
{त्वत्संनिधौ यत्कथयेत्पतिस्तेयद्यप्यगुह्यं परिरक्षितव्यम्}
{काचित्सपत्नी तव वासुदेवंप्रत्यादिशेत्तेन भवेद्विरागः}


\twolineshloka
{प्रियांश्च रम्यांश्च हितांश् भर्तु-स्तान्भोजयेथा रविविधैरुपायैः}
{द्वेष्यैरुपेक्ष्यैरहितैश् तस्यभिद्स्वनित्यं कुहकोद्धतैश्च}


\twolineshloka
{मदं प्रमादं पुरुषेषु हित्वासंयच्छ मानं प्रतिगृह् वाचम्}
{प्रद्युम्नसाम्बावपि ते कुमारौनोपासितव्यौ रहिते कदाचित्}


\twolineshloka
{महाकुलीनाभिरपापिकाभिःस्त्रीभिः सतीभिस्तव सख्यमस्तु}
{चण्डाश् शौण्डाश्च महाशनाश्चचोराश्च दुष्टाश्चपलाश्च वर्ज्याः}


\twolineshloka
{एतद्यशस्यं भगवेतनं चस्वार्थं तदा शत्रुनिबर्हणं च}
{महार्हमाल्याभरणाङ्गरागाभर्तारमाराधय पुण्यगन्धैः}


\chapter{अध्यायः २३६}
\twolineshloka
{वैशंपायन उवाच}
{}


\twolineshloka
{मार्कण्डेयादिभिर्विप्रैः पाण्डवैश्च महात्मभिः}
{कथाभिरनुकूलाभिः सह स्तित्वा जनार्दनः}


\twolineshloka
{ततस्तैः संविदं कृत्वा यथावन्मधुसूदनः}
{आरुरुक्षू रथं सत्यामाह्वयामास भारत}


\twolineshloka
{सत्यभामा ततस्तत्र स्वजित्वा द्रुपदात्मजाम्}
{उवाच वचनं हृद्यं यथाभावं समाहितम्}


\twolineshloka
{कृष्णे माभूत्तवोत्कण्ठा मा व्यथा मा प्रजागरः}
{भर्तृभिर्देवसंकाशैर्जितां प्राप्स्यसि मेदिनीम्}


\twolineshloka
{न ह्येवं शीलसंपन्ना नैवं पूजितलक्षणाः}
{प्राप्नुवन्ति चिरं क्लेशं यथा त्वमसितेक्षणे}


\twolineshloka
{अवश्यं च त्वया भूमिरियं निहतकण्टका}
{भर्तृभिः सहभोक्तव्या निर्द्वन्द्वेति श्रुतं मया}


\twolineshloka
{धार्तराष्ट्रवधं कृत्वावैराणि प्तियात्य च}
{युधिष्ठिरस्थां पृथिवीं द्रष्टासि द्रुपदात्मजे}


\twolineshloka
{यास्ताः प्रव्राजपानां त्वां प्राहसन्दर्पमोहिताः}
{ताः क्षिप्रं हतसंकल्पा द्रक्ष्यसि त्वं कुरुस्त्रियः}


\twolineshloka
{तव दुःखोपपन्नाया यैराचरितमप्रियम्}
{विद्धि संप्रस्थितान्सर्वांस्तान्कृष्णे यमसादनम्}


\twolineshloka
{पुत्रस्ते प्रतिविन्ध्यश्च सुतसोमस्तथाविधः}
{श्रुतकर्माऽर्जुनिश्चैव शतानीकश्च नाकुलिः}


\twolineshloka
{सहदेवाच्च यो जातः श्रुतसेनस्तवात्मजः}
{सर्वेकुशलिनो वीराः कृतास्त्राश्च सुतास्तव}


\twolineshloka
{अभिमन्युरिव प्रीता द्वारवत्यां रता भृशम्}
{त्वमिवैषां सुभद्रा च प्रीत्या सर्वात्मना स्थिता}


\twolineshloka
{प्रीयते तव निर्द्वन्द्वा तेभ्यश्च विगतज्वरा}
{दुःखिता तेन दुःखेन सुखेन सुखिता तथा}


\twolineshloka
{भजेत्सर्वात्मना चैव प्रद्युम्नजननी तथा}
{भानुप्रभृतिभिश्चैनान्विशिनष्टि च केशवः}


\twolineshloka
{भोजनाच्छादने चैषां नित्यं मे श्वशुरः स्थितः}
{रामप्रभृतयः सर्वे भजन्त्यन्धकवृष्णयः}


\twolineshloka
{तुल्यो हिप्रणयस्तेषां प्रद्युम्नस्य च भामिनि}
{एवमादि प्रियं सत्यंहृद्यमुक्त्वा मनोनुगम्}


\twolineshloka
{गमनाय मनश्चक्रेवासुदेवरथं प्रति}
{तां कृष्णां कृष्णमहिषी चकाराभिप्रदक्षिणम्}


\threelineshloka
{आरुरोह रथं शौरेः सत्यभामाऽथ भामिनी}
{स्मयित्वातु यदुश्रेष्ठो द्रौपदीं परिसान्त्व्य च}
{उपावर्त्य ततः शीघ्रैर्हयैः प्रायात्परंतपः}


\chapter{अध्यायः २३७}
\twolineshloka
{जनसेजय उवाच}
{}


\threelineshloka
{एवं वने वर्तमाना नराग्र्याःशीतोष्णवातातपकर्शिताङ्गाः}
{सरस्तदासाद्यवनं च पुण्यंततः परंकिमकुर्वन्त पार्थाः ॥वैशंपायन उवाच}
{}


\twolineshloka
{सरस्तदासाद्य तु पाण्डुपुत्राजनं समुत्सृज्य विधाय चेष्टम्}
{वनानि रम्याण्यथ पर्वतांश्चनदीप्रदेशांश्च तदा विचेरुः}


\twolineshloka
{तथा वने तान्वसतः प्रवीरान्स्वाध्यायवन्तश्च तपोधनाश्च}
{अभ्याययुर्वेदविदः पुराणा-स्तान्पूजयामासुरथो नराग्र्याः}


\twolineshloka
{ततः कदाचित्कुशलः कथासुविप्रोऽभ्यगच्छद्भुवि कौरवेयान्}
{स तैः समेत्याथ यदृच्छयैववैचित्रवीर्यं नृपमभ्यगच्छत्}


\twolineshloka
{अथोपविष्टः प्रतिसत्कृतश्चवृद्धेन राज्ञा कुरुसत्तमेन}
{प्रचोदितः संकथयांबभूवधर्मानिलेन्द्रप्रभवान्यमौ च}


\twolineshloka
{कृशांश्च वातातपकर्शिताङ्गान्दुःखस्य चोग्रस् मुखे प्रपन्नान्}
{तां चाप्यनाथामिव वीरनाथांकृष्णां परिक्लेशगुणेन युक्ताम्}


\twolineshloka
{ततः कथास्तस्य निशम्य राजावैचित्रवीर्यः कृपयाऽभितप्तः}
{वने तथा पार्थिवपुत्रपौत्रा-ञ्श्रुत्वा तथा दुःखनदींप्रपन्नान्}


\twolineshloka
{प्रोवाच दैन्याभिहतान्तरात्मानिश्वासवातोपहतस्तदानीम्}
{वाचं कथंचित्स्थिरतामुपेत्यतत्सर्वमात्मप्रभवं विचिन्त्य}


\twolineshloka
{कथंनु सत्यः शुचिरार्यवृत्तःश्रेष्ठः सुतानां मम धर्मराजः}
{अजातशत्रुः पृथिवीतले स्मशेते पुरा राङ्कवकूटशायी}


\twolineshloka
{प्रबोध्यते मागधसूतपुत्रै-र्नित्यं स्तुवद्भिः स्वयमिन्द्रकल्पः}
{पतत्रिसङ्घैः स जघन्यरात्रेप्रबोध्यते नूनमिलातलस्थः}


\twolineshloka
{कथंनु वातातपकर्शिताङ्गोवृकोदरः कोपपरिप्लुताङ्गः}
{शेते पृथिव्यामतथोचिताङ्गःकृष्णासमक्षं वसुधातलस्थः}


\twolineshloka
{तथाऽर्जुनः सुकुमारो मनस्वीवशे स्थितो धर्मसुतस्य राज्ञः}
{विदूयमानैरिव सर्वगात्रै-र्ध्रुवं न शेते वसतीरमर्षात्}


\twolineshloka
{यमौ च कृष्णां च युधिष्ठिरं चभीमं च दृष्ट्वा सुखविप्रयुक्तान्}
{विनिःश्वसन्सर्प इवोग्रतेजाध्रुवं न शेते वसतीरमर्षात्}


\twolineshloka
{तथा यमौ चाप्यसुखौ सुखार्हौसमृद्धरीपावमरौ दिवीव}
{प्रजागरस्थौ ध्रुवमप्रशान्तौक्रोधेन सत्येन च वार्यमाणौ}


\twolineshloka
{समीरणेनाथ समो बलेनसमीरणस्यैवसुतो बलीयान्}
{स धर्मपासेन सितोऽग्रजेनध्रुवं विनिःश्वस्य सहत्यमर्षम्}


\twolineshloka
{स चापिभूमौ परिवर्तमानोवधं सुतानां मम काङ्क्षमाणः}
{सत्येन धर्मेण च वार्यमाणःकालंप्रतीक्षत्यधिको रणेऽन्यैः}


\twolineshloka
{अजातशत्रौ तु जिते निकृत्यादुःशासनो यत्परुषाण्यवोचत्}
{तानि प्रविष्टानि वृकोदराङ्गंदहन्ति कक्षाग्निरिवेन्धनानि}


\twolineshloka
{न पापकं ध्यास्यति धर्मपुत्रोधनंजयश्चाप्यनुवर्त्स्यते तम्}
{अरण्यवासेन विवर्धते तुभीमस्य कोपोऽग्निरिवानिलेन}


\twolineshloka
{स तेन कोपेन विदीर्यमाणःकरं करेणाभिनिपीड्यवीरः}
{विनिःश्वसत्युष्णमतीव घोरंदहन्निवेमां मम पुत्रसेनाम्}


\twolineshloka
{गाण्डीवधन्वा च वृकोदरश्चसंरम्भिणावन्तककालकल्पौ}
{न शेषयेतां युधि शत्रुसेनांशरान्किरन्तावनिप्रकाशान्}


\twolineshloka
{दुर्योधनः शकुनिः सूतपुत्रोदुःशासनश्चापि सुमन्दचेताः}
{मधु प्रपश्यन्ति न तु प्रपातंवृकोदरं चैव धनंजयं च}


\twolineshloka
{शुभाशुभं कर्म नरोहि कृत्वाप्रतीक्षतेचेत्स फलंविपाके}
{सतेन युज्यत्यवशः फलेनमोक्षः कथं स्यात्पुरुषस्य तस्मात्}


\twolineshloka
{क्षेत्रे सुकृष्टे ह्युपिते च बीजेदेवेच वर्षत्यृतुकालयुक्तम्}
{न स्यात्फलंतस्य कुतः प्रसिद्धि-रन्यत्रदैवादिति नास्ति हेतुः}


\twolineshloka
{कृतं मताक्षेण यथा न साधुसाधुप्रवृत्तेन च पाण्डवेन}
{मया च दुष्पुत्रवशानुगेनकृतः कुरूणामयमन्तकालः}


\twolineshloka
{ध्रुवं प्रशाम्यत्यसमीरितोऽग्नि-र्ध्रुवं प्रजास्यत्युत गर्भिणी या}
{ध्रुवं दिनादौ रजनीप्रणाश-स्तथा क्षपादौ च दिनप्रणाशः}


\twolineshloka
{कृतेच कस्मान्न परेच कुर्यु-र्दत्ते च दद्युः पुरुषाः कथंस्वित्}
{प्राप्यार्थकालं च भवेदनर्थःकथंचन स्यादितितत्कुतः स्यात्}


\twolineshloka
{कथं न भिद्येत न च स्रवेतन च प्रसिच्येदितिरक्षितव्यम्}
{अरक्ष्यमाणं शतधा प्रकीर्ये-द्भ्रुवं न नाशोऽस्ति कृतस्य लोके}


\twolineshloka
{गतो ह्यरण्यादपिशक्रलोकंधनंजयः पशय्त वीर्यमस्य}
{अस्त्राणि दिव्यानि चतुर्विधानिज्ञात्वा पुनर्लोकमिमं प्रपन्नः}


\twolineshloka
{स्वर्गं हि गत्वा सशरीर एवको मानुषः पुनरागन्तुमिच्छेत्}
{अन्यत्रकालोपहताननेका-न्समीक्षमाणस्तुकुरून्मुमूर्षून्}


\twolineshloka
{धनुर्ग्राहश्चार्जुनः सव्यसाचीधनुश्च तद्गाण्डिवं भीमवेगम्}
{अस्त्राणि दिव्यानि च तानि तस्यत्रयस्य तेजः प्रसहेत कोऽत्र}


\twolineshloka
{निशम्य तद्वचनं पार्थिवस्यदुर्योधनं रहिते सौबलोऽथ}
{अबोधयत्कर्णमुपेत्य सर्वंस चाप्यहृष्टोऽभवदल्पचेताः}


\chapter{अध्यायः २३८}
\twolineshloka
{वैशंपायन उवाच}
{}


\twolineshloka
{धृतराष्ट्रस्य तद्वाक्यं निशम्य शकुनिस्तदा}
{दुर्योधनमिदं काले कर्णेन सहितोऽब्रवीत्}


\twolineshloka
{प्रव्राज्य पाण्डवान्वीरान्स्वेन वीर्येण भारत}
{भुङ्क्ष्वेमां पृथिवीमेको दिवि शम्बरहायथा}


\twolineshloka
{`तवाद्यपृथिवी राजन्नखिला सागराम्बरा}
{सपर्वतवनाकारा सहस्थावरजङ्गमा'}


\twolineshloka
{प्राच्याश्च दाक्षिणात्याश्च पतीच्योदीच्यवासिनः}
{कृताः करप्रदाः सर्वे राजानस्ते नाराधिप}


\twolineshloka
{या हि सा दीप्यमानेव पाण्डवान्भजडते पुरा}
{साऽद्यलक्ष्मीस्त्वया राजन्नवाप्ता भ्रातृभिः सह}


\twolineshloka
{इन्द्रप्रस्थगते यां तां दीप्यमानां युधिष्ठिरे}
{अपश्याम श्रियं राजन्सुचिरं शोककर्शिताः}


\twolineshloka
{सा तु बुद्धिबलेनेयं राज्ञस्तस्मात्तथाविधात्}
{त्वयाक्षिप्ता महाबाहो दीप्यमानेव दृश्यते}


\twolineshloka
{तथैव तव राजेन्द्रराजानः परवीरहन्}
{शासनेऽधिष्ठिताः सर्वेकिं कुर्म इति वादिनः}


\twolineshloka
{ते वयं पृथिवी राजन्निखिला सागराम्बरा}
{सपर्वतवना देवी सग्रामनगराकरा}


\twolineshloka
{नानावनोद्देशवती पत्तनैरुपशोभिता}
{`नानाजनपदाकीर्णा स्फीतराष्ट्रा महाहला'}


\twolineshloka
{नन्द्यमानो द्विजै राजन्भासि नक्षत्रराडिव}
{पौरुषाद्दिवि देवेषु भ्राजसे रश्मिवानिव}


\twolineshloka
{रुद्रैरिव यमो राजा मरुद्भिरिव वासवः}
{कुरुभिस्त्वं वृतो राजन्भासि नक्षत्रराडिव}


\twolineshloka
{यैः स्म ते नाद्रियेताज्ञा न च ये शासने स्थिताः}
{पश्यामस्ताञ्श्रिया हीनान्पाण्डवान्वनवासिनः}


\twolineshloka
{श्रूयते हि महाराजसरो द्वैतवनं प्रति}
{वसन्तः पाण्डवाः सार्धं ब्राह्मणैर्वनवासिभिः}


\twolineshloka
{सप्रयाहि महाराज श्रिया परमया युतः}
{तापयन्पाण्डुपुत्रांस्त्वं रश्मिवानिव तेजसा}


% Check verse!
स्तितोराज्येऽच्युतान्राज्याच्छियाहीनाञ्छ्रियावृतःअसमृद्धान्समृद्धार्थः पश्य पाण्डुसुतान्नृप
\twolineshloka
{महाभिजनसंपन्नं भद्रे महति संस्थितम्}
{पाण्डवास्त्वाऽभिवीक्षन्तु ययातिमिव नाहुषां}


\twolineshloka
{यां श्रियं सुहृदश्चैव दुर्हृदश्च विशांपते}
{पश्यन्ति पौरुषैर्दीप्तां सा समर्था भवत्युत}


\twolineshloka
{समस्थो विषमस्थान्हि दुर्हृदो योऽभिवीक्षते}
{जगतीस्थनिवाद्रिस्थः किमतः परमं सुखम्}


\twolineshloka
{न पुत्रधनलाभेन न राज्येनापि विन्दति}
{प्रीतिं नृपतिशार्दूल याममित्राधदर्शनात्}


\twolineshloka
{किंनु तस्य सुखं न स्यादाश्रमे यो धनंजयम्}
{अभिवीक्षेत सिद्धार्थो वल्कलाजिनवाससम्}


\twolineshloka
{सुवाससो हि ते भार्या वल्कलाजिनसंवृताम्}
{पश्यन्तु दुःखितां कृष्णां सा च निर्विद्यतां पुनः}


\twolineshloka
{विनिन्दतां तथाऽत्मानं जीवितं च धनच्युतम्}
{`दाराणां ते श्रियं दृष्ट्वा दीप्तामद्य जनाधिपा'}


\threelineshloka
{न तथा हिसभामध्ये तस्या भवितुमर्हति}
{वैमनस्यं यथा दृष्ट्वा तव भार्याः स्वलंकृताः ॥वैशंपायन उवाच}
{}


\twolineshloka
{एवमुक्त्वा तु राजानं कर्णः शकुनिना सह}
{तूष्णीं बभूवतुरुभौ दाक्यान्ते जनमेजय}


\chapter{अध्यायः २३९}
\twolineshloka
{वैशंपायन उवाच}
{}


\twolineshloka
{कर्णस्य वचनं श्रुत्वा राजा दुर्योधनस्ततः}
{हृष्टो भूत्वापुनर्दीनो राधेयमिदमब्रवीत्}


\twolineshloka
{ब्रवीषि यदिदं कर्ण सर्वं मनसि मे स्थितम्}
{न त्वभ्यनुज्ञां लप्स्यामि गमने यत्रपाण्डवाः}


\twolineshloka
{परिदेवति तान्वीरान्धृतराष्ट्रो महीपतिः}
{मन्यतेऽभ्यधिकांश्चापि तपोयोगेन पाण्डवान्}


\twolineshloka
{अथवाऽप्यनुबुध्येत नृपोऽस्माकं चिकीर्षितम्}
{एतामप्यायतिं रक्षन्नाभ्यनुज्ञातुमर्हति}


\twolineshloka
{न हि द्वैतवने किंचिद्विद्यतेऽन्यत्प्रयोजनम्}
{उन्माथनमृते तेषां वनस्थानामपि द्विषाम्}


\twolineshloka
{जानासिहि यथा क्षत्ता द्यूतकाल उपस्थिते}
{अब्रवीद्यच्च मां त्वां च सौबलं वचनं तदा}


\twolineshloka
{तानिसर्वाणि वाक्यानि यच्चान्यत्परिदेवितम्}
{विचिन्त्य निश्चय गच्छे गमनायेतराय वा}


\twolineshloka
{ममापि हिमहान्हर्षो यदहं भीमफल्गुनौ}
{क्लिष्टावरण्ये पश्येयं कृष्णया सहिताविति}


\twolineshloka
{न तथा ह्याप्नुयां प्रीतिमवाप्य वसुधामिमाम्}
{दृष्ट्वा यथा पाण्डुसुतान्वल्कलाजिनवाससः}


\twolineshloka
{किंनु स्यादधिकं तस्माद्यदहं द्रुपदात्मजाम्}
{द्रौपदीं कर्ण पश्येयं काषायवसनां वने}


\twolineshloka
{यदि मां दर्मराजश्च भीमसेनश्च पाण्डवः}
{युक्तं रपरमया लक्ष्म्या पश्येतां जीवितं भवेत्}


\twolineshloka
{उपायं न तु पश्यामि येन गच्छेम तद्वनम्}
{यथाचाभ्यनुजानीयाद्गच्छन्तं मां महीपतिः}


\twolineshloka
{स सौबलेन सहितस्तथा दुःशासनेन च}
{उपायं पश्य रनिपुणं येन गच्छेम तद्वनम्}


\twolineshloka
{अहमप्यनुगच्छामि गमनायेतराय च}
{कल्यमेव गमिष्यामि समीपं पार्थिवस्य ह}


\twolineshloka
{मयि तत्रोपविष्टे तु भीष्मे च कुरुसत्तमे}
{उपायो यो भवेद्दृष्टस्तं ब्रूयाः सहसौबलः}


\twolineshloka
{ततो भीष्मस्य राज्ञश्च निशम्य रगमनं प्रति}
{व्यवसायं करिष्येऽहमनुनीय पितामहम्}


\twolineshloka
{तथेत्युक्त्वा तु ते सर्वेग्मुरावसथान्प्रति}
{व्युषितायां रजन्यां तु कर्णो राजानमभ्ययात्}


\twolineshloka
{ततो दुर्योधनं कर्णः प्रहसन्निदमब्रवीत्}
{उपायः परिदृष्टोऽयं तं निबोधजनेश्वर}


\twolineshloka
{घोषा द्वैतवने सर्वेत्वत्प्रतीक्षा नराधिप}
{घोषयात्रापदेशेन गमिष्यामो न संशयः}


\twolineshloka
{उचितं हि सदा गन्तुं घोषयात्रां विशांपते}
{एवं चेत्त्वां पिता राजन्समनुज्ञातुमर्हति}


\twolineshloka
{तथा कथयमानौ तु घोषयात्राविनिश्चयम्}
{गान्धारराजः शकुनिरित्युवाच हसन्निव}


\twolineshloka
{उपायोऽयंमया दृष्टो गमनाय निरामयः}
{अनुज्ञास्यतिनो राजा चोदयिष्यति चाप्युत}


\twolineshloka
{घोषा द्वैतव्रने सर्वेत्वत्प्रतीक्षा नराधिप}
{घोषयात्रापदेशेन गमिष्यामः सरः प्रति}


\twolineshloka
{ततः प्रहसिताः सर्वे तेऽन्योन्यस्य तलान्ददुः}
{तदेव च विनिश्चेत्य ददृशुः कुरुसत्तमम्}


\chapter{अध्यायः २४०}
\twolineshloka
{वैसंपायन उवाच}
{}


\twolineshloka
{धृतराष्ट्रं ततः सर्वेददृशुर्जनमेजय}
{दृष्ट्वा सुखमथो राज्ञः पृष्टा राज्ञा च भारत}


\twolineshloka
{ततस्तैर्विहितः पूर्वं संगवो नाम वल्लवः}
{समीपस्थास्तदा गावो धृतराष्ट्रे न्यवेदयत्}


\twolineshloka
{अनन्तरं च राधेयः शकुनिश्च विशांपते}
{आहतुः पार्थिवश्रेष्ठं धृतराष्ट्रं जनाधिपम्}


\twolineshloka
{रमणीयेषु देशेषु घोषाः संप्रति कौरव}
{स्मारणे समयः प्राप्तो वत्सानामपि चाङ्कनम्}


\threelineshloka
{मृगया चोचिता राजन्नस्मिन्काले सुतस्यते}
{दुर्योधनस्य गमनं त्वमनुज्ञातुमर्हसि ॥धृतराष्ट्र उवाच}
{}


\twolineshloka
{मृगया शोभना तात गवां हि समवेक्षणम्}
{विस्रम्भस्तु न गन्तव्यो वल्लवानामिति स्मरे}


\twolineshloka
{ते तु तत्रनरव्याघ्राः समीप इति नः श्रुतम्}
{अतो नाभ्यनुजानामि गमनं तत्र वः स्वयम्}


\twolineshloka
{छद्मना निर्जितास्ते तु कर्शिताश्च महावने}
{तपोनित्याश्च राधेय समर्थाश्च महारथाः}


\twolineshloka
{धर्मराजो न संक्रुद्ध्येद्भीमसेनस्त्वमर्षणः}
{यज्ञसेनस् दुहिता तेज एवतु केवलम्}


\twolineshloka
{यूयंचाप्यपराध्येयुर्दर्पमोहसमन्विताः}
{ततो विनिर्दहेयुस्ते तपसा हि समन्विताः}


\twolineshloka
{अथवा सायुधावीरा मन्युनाऽभिपरिप्लुताः}
{सहिता बद्धनिस्त्रिशा दहेयुः शस्त्रतेजसा}


\twolineshloka
{अथ यूयं बहुत्वात्तान्नारभध्वं कथंचन}
{अनार्यं परमं तत्स्यादशक्यं तच्च वै मतम्}


\twolineshloka
{उषितो हि महाबाहुरिन्द्रलोके धनंजयः}
{दिव्यान्यस्त्राण्यवाप्याथ ततः प्रत्यागतो वनं}


\twolineshloka
{अकृतास्त्रेण पृथिवी जिता बीभत्सुना पुरा}
{किं पुनः सकृतास्त्रोऽद्य न हन्याद्वो महारथः}


\twolineshloka
{अथवा मद्वचः श्रुत्वा तत्र यत्ता भविष्यथ}
{उद्विग्रवासा विस्रब्धा दुःखं तत्रगमिष्यथ}


\twolineshloka
{अथवा सैनिकाः केचिदपकुर्युर्युधिष्ठिरे}
{तदबुद्धिकृतंकर्म दोषमुत्पादयेच्च वः}


\threelineshloka
{तस्मादन्ये नरा यान्तु स्मारणायाप्तकारिणः}
{न स्वयं तत्रगमनं रोचये तव भारत ॥शकुनिरुवाच}
{}


\twolineshloka
{धर्मज्ञः पाण्डवो ज्येष्ठः प्रतिज्ञातं च संसदि}
{तेन द्वादशवर्षाणि वस्तव्यानीति भारत}


\twolineshloka
{अनुवृत्ताश्च रतं सर्वे पाण्डवा धर्मचारिणः}
{युधिष्ठिरस्तु कौन्तेयो न नः कोपं करिष्यति}


\twolineshloka
{मृगयां चैव नो गन्तुमिच्छा संवर्धते भृशम्}
{स्मारणं तु चिकीर्षामो न तु पाण्डवदर्सनम्}


\threelineshloka
{न चानार्यसमाचारः कश्चित्तत्र भविष्यति}
{न च तत्र गमिष्यामो यत्र तेषां प्रतिश्रयः ॥वैशंपायन उवाच}
{}


\twolineshloka
{एवमुक्तः शकुनिना धृतराष्ट्रो जनेश्वरः}
{दुर्योधनं सहामात्यमनुजज्ञे न कामतः}


\twolineshloka
{अनुज्ञातस्तु गान्धारिः कर्णेन सहितस्तदा}
{निर्ययौ भरतश्रेष्ठो बलेन महता वृतः}


\twolineshloka
{दुःशासनेन च तथा सौबलेन च धीमता}
{संवृतो भ्रातृभिश्चान्यैः स्त्रीभिश्चापि सहस्रशः}


\twolineshloka
{तं निर्यान्तं महाबाहुं द्रष्टुं द्वैतवनं सरः}
{पौराश्चानुययुः सर्वेसहदारा वनं च तत्}


\twolineshloka
{अष्टौ रथसहस्राणि त्रीणि नागायुतानि च}
{पत्तयो बहुसाहस्रा हयाश्च नवतिः शताः}


\twolineshloka
{शकटापणवेशाश्च वणिजो वन्दिनस्तथा}
{नराश्च मृगयाशीलाः शतशोऽथ सहस्रशः}


\twolineshloka
{ततः प्रयाणे नृपतेः सुमहानभवत्स्वनः}
{प्रावृषीव महावायोरुत्थितस्य विशांपते}


\twolineshloka
{गव्यूतिमात्रेन्यवसद्राजा दुर्योधनस्तदा}
{प्रयातो वाहनैः सर्वैस्ततो द्वैतवनं सरः}


\chapter{अध्यायः २४१}
\twolineshloka
{वैशंपायन उवाच}
{}


\twolineshloka
{अथ दुर्योधनो राजा तत्रतत्र वने वसन्}
{जगाम घोषानभितस्तत्र चक्रे निवेशनम्}


\twolineshloka
{रमणीये समाज्ञाते सोदके समहीरुहे}
{देशे सर्वगुणोपेते चक्रुरावसथान्नराः}


\twolineshloka
{तथैव तत्समीपस्थान्पृथगावसथान्बहून्}
{कर्णस्य शकुनेश्चैव भ्रातॄणां चैव सर्वशः}


\twolineshloka
{पश्यन्तस्ते तदा गावः शतशोऽथ सहस्रशः}
{अङ्कर्लक्षैश्च ताः सर्वा लक्षयामास पार्थिवः}


\twolineshloka
{अङ्कयामास वत्सांश्च जज्ञे चोपसृतांस्त्वपि}
{बालवत्साश्च यां गावः कालयामास ता अपि}


\twolineshloka
{अथ स स्मारणं कृत्वा लक्षयित्वा त्रिहायनान्}
{वृतो गोपालकैः प्रीतो व्याहरत्कुरुनन्दनः}


\twolineshloka
{स च पौरजनः सर्वः सर्वः सैनिकाश्च सहस्रशः}
{यथोपजोषं चिक्रीडुर्वने तस्मिन्यथाऽमराः}


\twolineshloka
{ततोऽध्वगमनाच्छ्रान्तं कुशला नृत्यवादितैः}
{धार्तराष्ट्रमुपातिष्ठन्कन्याश्चैव स्वलंकृताः}


\twolineshloka
{स स्त्रीगणवृतो राजा प्रहृष्टः प्रददौ वसु}
{तेभ्यो यथार्हमन्नानि पानानि विविधानि च}


\twolineshloka
{ततस्ते सहिताः सर्वे तरक्षून्महिषान्मृगान्}
{गवयर्क्षवराहांश्च समन्तात्पर्यवारयन्}


\twolineshloka
{स ताञ्छरैर्विनिर्भिद्य गजांश्च सुबहून्वने}
{रमणीयेषु देशेषु ग्राहयामास वै मृगान्}


\twolineshloka
{गोरसानुपयुञ्जान उपभोगांशच् भारत}
{पश्यन्स रमणीयानि वनान्युपवनानि च}


\twolineshloka
{मत्तभ्रमरजुष्टानि बर्हिणाभिरुतानि च}
{अगच्छदानुपूर्व्येण पुण्यं द्वैतवनं सरः}


\twolineshloka
{मत्तभ्रमरसंजुष्टं नीलकण्ठरवाकुलम्}
{सप्तच्छदसमाकीर्णं पुन्नागवकुलैर्युतम्}


\twolineshloka
{ऋद्ध्या परमया युक्तो महेन्द्र इव वज्रभृत्}
{यदृच्छया च तत्रस्थो धर्मपुत्रो युधिष्ठिरः}


\twolineshloka
{ईजे राजर्षियज्ञेन साद्यस्केन विशांपते}
{दिव्येन विधिना चैव वन्येन कुरुसत्तम}


\threelineshloka
{`विद्वद्भिः सहितो धीमान्ब्राह्मणैर्वनवासिभिः'}
{कृत्वा निवेशमभितः सरसस्तस्य कौरव}
{द्रौपद्या सहितो धीमान्धर्मपत्न्या नराधिपः}


\twolineshloka
{ततो दुर्योधनः प्रेष्यानादिदेश सहानुजः}
{आक्रीडावसथाञ्शीघ्रं कुरुध्वं सरसोऽभितः}


\twolineshloka
{ते तथेत्येव कौरव्यमुक्त्वा वचनकारिणः}
{चिकीर्षन्तस्तदाक्रीडाञ्जग्मुर्द्वैतवनं सरः}


% Check verse!
सेनाग्र्यं धार्तराष्ट्रस् प्राप्तं द्वैतवनं सरः
\threelineshloka
{प्रविशन्तं वनद्वारि गन्धर्वाः समवारयन्}
{तत्र गन्धऱ्वराजो वै पूर्वमेव विशांपते}
{कुबेरभवनाद्राजन्नाजगाम गणावृतः}


\twolineshloka
{गणैरप्सरसां चैव त्रिदशानां तथाऽऽत्मजैः}
{विहारशीलैः क्रीडार्थं तेन तत्संवृतं सरः}


\twolineshloka
{तेन तत्संवृतं दृष्ट्वा ते राजपरिचारकाः}
{प्रतिजग्मुस्ततो राजन्यत्र दुर्योधनो नृपः}


\twolineshloka
{स तु तेषां वचः श्रुत्वा सैनिकान्युद्धदुर्मदान्}
{प्रेषयामास कौरव्य उत्सारयत तानिति}


\twolineshloka
{तस्य तद्वचनं श्रुत्वा राज्ञः सेनाग्रयायिनः}
{सरो द्वैनवनं गत्वा गन्धर्वानिदमब्रुवन्}


\twolineshloka
{राजा दुर्योधनो नाम धृतराष्ट्रसुतो बली}
{चिक्रीडिषुरिहायाति तदर्थमपसर्पत}


\twolineshloka
{एवमुक्तास्तु गन्धर्वाः रप्रहसन्तो विशांमपते}
{प्रत्यब्रुवंस्तान्पुरुषानिदं हि परुषं वचः}


\twolineshloka
{न चेतयति वो राजा मन्दबुद्धिः सुयोधनः}
{योऽस्मानाज्ञापयत्येवं वश्यानिव दिवौकसः}


\twolineshloka
{यूयं मुमूर्षवश्चापि मन्दप्रज्ञा न संशयः}
{ये तस् वचनादेवमस्मान्ब्रूथ विचेतसः}


\twolineshloka
{गच्छध्वं त्वरिताः सर्वे यत्र राजा स कौरवः}
{न चेदद्यैव गच्छध्वं धर्मराजनिवेशनम्}


\twolineshloka
{एवमुक्तास्तु गन्धर्वै राज्ञः सेनाग्रयायिनः}
{संप्राद्रवन्यतो राजा धृतराष्ट्रसुतोऽभवत्}


\chapter{अध्यायः २४२}
\twolineshloka
{वैशंपायन उवाच}
{}


\twolineshloka
{ततस्ते सहिताः सर्वे दुर्याधनमुपागमन्}
{अब्रुवंश्च महाराज यदूचुः कौरवं प्रति}


\twolineshloka
{गन्धर्वैर्वारिते सैन्ये धार्तराष्ट्रः प्रतापवान्}
{अमर्षपूर्णः सैन्यानि प्रत्यभाषत भारत}


\threelineshloka
{शासतैनानधर्मज्ञान्मम विप्रियकारिणः}
{यदि प्रक्रीडते सर्वैर्देवैः सह शचीपतिः}
{`वयमत्र यथा प्रीता क्रीडिष्यामो निरङ्कुशं'}


\twolineshloka
{दुर्योधनवचः श्रुत्वा धार्तराष्ट्रा महाबलाः}
{सर्व एवाभिसन्नद्धा योधाश्चापि सहस्रशः}


\twolineshloka
{ततः प्रमथ्य सर्वांस्तांस्तद्वनं विविशुर्बलात्}
{सिंहनादेन महता पूरयन्तो दिशो दश}


\twolineshloka
{ततोऽपरैरवार्यन्त गन्धर्वैः कुरुसैनिकाः}
{`साम्नैव रतत्र विक्रान्ता मा साहसमिति प्रभो'}


\twolineshloka
{ते वार्यमाणा गन्धर्वैः साम्नैव वसुधायिप}
{ताननादृत्य गन्धर्वांस्तद्वनं विविशुर्महत्}


\twolineshloka
{यदि वाता न तिष्ठन्ति धार्तराष्ट्राः सराजकाः}
{ततस्ते खेचराः सर्वे चित्रसेने न्यवेदयन्}


\twolineshloka
{गन्धर्वराजस्तान्सर्वानब्रवीत्कौरवान्प्रति}
{अनार्याञ्शासतेत्येतांश्चित्रसेनोऽत्यमर्षणः}


\twolineshloka
{अनुज्ञाताश्च गन्धर्वाश्चित्रसेनेन भारत}
{प्रगृहीतायुधाः सर्वे धार्तराष्ट्रानभिद्रवन्}


\twolineshloka
{तान्दृष्ट्वाऽऽपततः शीघ्रान्गन्धर्वानुद्यतायुधान्}
{प्राद्रवंस्ते दिशः सर्वे धार्तराष्ट्रस्य पश्यतः}


\twolineshloka
{तान्दृष्ट्वा द्रवतः सर्वान्धार्तराष्ट्रान्पराङ्मुखान्}
{राधेयस्तु तदा वीरो नासीत्तत्रपराङ्मुखः}


\twolineshloka
{आपतन्तीं तु संप्रेक्ष्य गन्धर्वाणां महाचमूम्}
{महता शरवर्षेण राधेयः प्रत्यवारयत्}


\twolineshloka
{क्षुरप्रैर्विशिखैर्भल्लैर्वत्सदन्तैस्तथाऽऽयसैः}
{गन्धर्वाञ्शतशोऽभिघ्नँल्लघुत्वात्सूतनन्दनः}


\twolineshloka
{पातयन्नुत्तमाङ्गानि गन्धर्वाणां महारथः}
{क्षणएन व्यधमत्सर्वां चित्रसेनस्य वाहिनीम्}


\twolineshloka
{ते वध्यमाना गन्धर्वाः सूतपुत्रेण धीमता}
{भूय एवाभ्यवर्तन्त शततशोऽथ सहस्रशः}


\twolineshloka
{गन्धर्वभूता पृथिवी क्षणेन समपद्यत}
{आपतद्भिर्महावेगैश्चित्रसेनस्य सैनिकैः}


\threelineshloka
{अथ दुर्योधनो राजा शकुनिश्चापि सौबलः}
{दुःशासनो विकर्णश्च ये चान्ये धृतराष्ट्रजाः}
{न्यहनंस्तत्तदा सैन्यं रथैर्गरुडनिःखनैः}


% Check verse!
सैन्यमायोधितं दृष्ट्वाकर्णो राजन्न मृष्यत
\threelineshloka
{महता रथसङ्घेन रथचारेण चाप्युत}
{वैकर्तनं परीप्सन्तो गन्धर्वाः प्रत्यवारयन्}
{ततः संन्यपतन्सर्वे गन्धर्वाः कौरवं प्रति}


\twolineshloka
{तदा सुतुमुलं युद्धमभवद्रोमहर्षणम्}
{ततस्ते मृदवोऽभूवनगन्धर्वाः शरपीडिताः}


% Check verse!
उच्चुक्रुशुश्च कौरव्यागन्धर्वान्प्रेक्ष्य पीडितान्
\twolineshloka
{गन्धर्वांस्त्रासितान्दृष्ट्वा चित्रसेनो ह्यमर्षणः}
{उत्पपातासनात्क्रुद्धो वधे तेषां समाहितः}


\twolineshloka
{ततो मायास्त्रमास्थाय युयुधे चित्रमार्गवित्}
{`वियत्संछादयामास न ववौ तत्र मारुतः'}


\twolineshloka
{हस्त्यारोहा हताः पेतुर्हस्तिभिः सह भारत}
{हयारोहाः सह हयै रथैश्च रथिनस्तदा}


\twolineshloka
{पत्तयश्च तथापेतुर्विशस्ताः शरवृष्टिभिः'}
{तयाऽमुह्यन्त कौरव्याश्चित्रसेनस्य मायया}


\twolineshloka
{एकैको हि तदा योधो धार्तारष्ट्रस्य भारत}
{पर्यवार्यत गन्धर्वैर्दशभिर्दशभिर्युधि}


\twolineshloka
{ततः संपीड्यमानास्ते बलेन महता तदा}
{प्राद्रवन्त रणे भीता यत्रराजा युधिष्ठिरः}


\twolineshloka
{भज्यमानेष्वनीकेषु धार्तराष्ट्रेषु सर्वशः}
{कर्णो वैकर्तनो राजंस्तस्थौ गिरिरिवाचलः}


\twolineshloka
{दुर्योधनश्च तेजस्वी शकुनिश्चापि सौबलः}
{गन्धर्वान्योधयामासुः समरे भृशविक्षताः}


\twolineshloka
{सर्व एव तु गन्धर्वाः शतशोऽथ सहस्रशः}
{जिघांसमानाः संरब्धाः कर्णमभ्यद्रवत्रणे}


\twolineshloka
{असिभिः पट्टसैः शूलैर्गदाभिश्च महाबलाः}
{सूतपुत्रं जिघांसन्तः समन्तात्पर्यवारयन्}


\twolineshloka
{अन्येऽस्य युगमच्छिन्दन्ध्वजमन्ये न्यपातयन्}
{ईषामन्ये हयानन्ये सूतमन्ये न्यपातयन्}


\twolineshloka
{अन्ये च्छत्रं वरूथं च बन्धुरं च तथा परे}
{`अन्ये संचूर्णयामासुश्छत्रे चाक्षौ तथा परे ॥'}


\twolineshloka
{गन्धर्वा बहुसाहस्रास्तिलशो व्यधमन्रथम् ॥ततो रथादवप्लुत्य सूतपुत्रोऽसिचर्मभृत्}
{}


\twolineshloka
{`अंसावलम्बितधनुर्धावमानो महाबलः'}
{विकर्णरथमास्थाय मोक्षायाश्वानचोदयत्}


\chapter{अध्यायः २४३}
\twolineshloka
{वैशंपायन उवाच}
{}


\twolineshloka
{गन्धर्वैस्तु महाराज भग्ने कर्णे महारथे}
{संप्राद्रवच्चमूः सर्वा धार्तराष्ट्रस्य पश्यतः}


\twolineshloka
{तान्दृष्ट्वा द्रवतः सर्वान्धार्तराष्ट्रान्पराङ्मुखान्}
{दुर्योधनो महाराजो नासीत्तत्र पराङ्मुखः}


\twolineshloka
{तामापतन्तीं संप्रेक्ष्य गन्धर्वाणां महाचमूम्}
{महता शरवर्षेण सोऽभ्यवर्षदरिंदमः}


\twolineshloka
{अचिन्त्य शरवर्षं तु गन्धर्वास्तस्य तं रथम्}
{दुर्योधनं जिघांसन्तः समन्तात्पर्यवारयन्}


\twolineshloka
{युगमीषां वरूथं च तथैव ध्वजसारथी}
{अश्वांस्त्रिवेणुमक्षं च तिलशो व्यधमञ्छरैः}


\twolineshloka
{दुर्योधनं चित्रसनो विरथं पतितं भुवि}
{अभिद्रुस्य महाबाहुर्जीवग्राहमथाग्रहीत्}


\twolineshloka
{`तस्य बाहू महाराज बद्ध्वा रज्ज्वा महारथम्}
{आरोप्यस महाबाहुश्चित्रसेनो ननाद ह'}


\twolineshloka
{तस्मिन्गृहीते राजेन्द्र स्थितं दुःशासनं रथे}
{पर्यगृह्णन्त गन्धर्वाः परिवार्य समन्ततः}


\twolineshloka
{विधिशतिं चित्रसेनमादायान्ये विदुद्रुवुः}
{विन्दानुविन्दावपरे राजदारांश्च सर्वशः}


\twolineshloka
{सेनास्तु धार्तराष्ट्रस्य गन्धर्वैः समभिद्रुताः}
{पूर्वं प्रभग्नैः सहिताः पाण्डवानभ्ययुस्तदा}


\threelineshloka
{शकटापणवेशाश्च यानयुग्यं च सर्वशः}
{शरणं पाण्डवाञ्जग्मुर्हियमाणे महीपतौ ॥सैनिका ऊचुः}
{}


\twolineshloka
{प्रियदर्शी महाबाहो धार्तराष्ट्रो महाबलः}
{गन्धर्वैर्ह्रियते राजा पार्थास्तमनुधावत}


\twolineshloka
{दुःशासनो दुर्विषहो दुर्मुखो दुर्मुखो दुर्जयस्तथा}
{बद्ध्वा ह्रियन्ते गन्धर्वै राजदाराश्च सर्वशः}


\twolineshloka
{इतिदुर्योधनामात्याः क्रोशन्तो राजगृद्धिनः}
{आर्ता दीनास्ततः सर्वे युधिष्ठिरमुपागमन्}


\twolineshloka
{तांस्तथा व्यथितान्दीनान्भिक्षमाणान्युधिष्ठिरम्}
{वृद्धान्दुर्योधनामात्यान्भीमसेनोऽभ्यभाषत}


\twolineshloka
{महता हि प्रयत्नेन संनह्य गजवाजिभिः}
{अन्यथा वर्तमानानामर्थो जातोऽयमन्यथा}


% Check verse!
अस्माभिर्यदनुष्ठेयं गन्धर्वैस्तदनुष्ठितम्
\twolineshloka
{दुर्मन्त्रितमिदं तावद्राज्ञो दुर्द्यूतदेविनः}
{`दीनान्दुर्योधनस्यास्मान्द्रष्टुकामस्य दुर्मतेः'}


\twolineshloka
{द्वेष्टारमन्ये क्लीवस्य घातयन्तीति नः श्रुतम्}
{इदं कृतं नः प्रत्यक्षं गन्धर्वैरतिमानुषम्}


\twolineshloka
{दिष्ट्या लोके पुमानस्ति कश्चिदस्मन्प्रिये स्थितः}
{येनास्माकं हृतो भार आसीनानां सुखावहः}


\twolineshloka
{शीतवातातपसहांस्तपसा चैव कर्शितान्}
{समस्थो विषमस्थान्हि द्रष्टुमिच्छति दुर्मतिः}


\twolineshloka
{अधर्मचारिणस्तस्य कौरव्यस्य दुरात्मनः}
{ये शीलमनुवर्तन्ति ते पश्यन्ति पराभवम्}


\twolineshloka
{अधर्मो हि कृतस्तेन येनैतदुपलक्षितम्}
{अनृशंसास्तु कौन्तेयास्तत्प्रत्यक्षं ब्रवीमि वः}


\twolineshloka
{एवं ब्रुवाणं कौन्तेयं भीमसेनमपस्वरम्}
{न कालः परुषस्यायमिति राजाऽभ्यभाषत}


\chapter{अध्यायः २४४}
\twolineshloka
{यधिष्ठिर उवाच}
{}


\twolineshloka
{अस्मानभिगतांस्तात भयार्ताञ्छरणैषिणः}
{कौरवान्विषमप्राप्तान्कथं ब्रूयास्त्वमीदृशम्}


\twolineshloka
{भवन्ति भेदा ज्ञातीनां कलहाश्च वृकोदर}
{प्रसक्तानि च वैराणि ज्ञातिधर्मो न नश्यति}


\twolineshloka
{यदा तु कश्चिज्ज्ञातीनां बाह्यः प्रार्थयते कुलम्}
{न मर्षयन्ति तत्सन्तो बाह्येनाभिप्रधर्षणम्}


\twolineshloka
{जानात्येष हि दुर्बुद्धिरस्मानिह चिरोषितान्}
{स एवं परिभूयास्मानकार्षीदिदमप्रियम्}


\twolineshloka
{दुर्योधनस्य ग्रहणाद्गन्धर्वेण बलाद्रणे}
{स्त्रीणां बाह्याभिमर्शाच्च हतं भवति नः कुलम्}


\twolineshloka
{श्चरणं च प्रपन्नानां त्राणार्थं च कुलस्य च}
{उत्तिष्ठध्वं नरव्याघ्राः सज्जीभवत मा चिरम्}


\twolineshloka
{अर्जुनश्च यमौ चैव त्वं च भीमापराजितः}
{मोक्षयध्वं नरव्याघ्रा ह्रियमाणं सुयोधनम्}


\twolineshloka
{एते रथा नरव्याघ्राः सर्वशस्त्रसमन्विताः}
{धृतराष्ट्रस्य पुत्राणां विमला काञ्चनध्वजाः}


\twolineshloka
{सस्वनानधिरोहध्वं नित्यसज्जानिमान्रथान्}
{इन्द्रसेनादिभिः सूतैः कृतशस्त्रैरधिष्ठितान्}


\twolineshloka
{एतानास्थाय वै यत्ता गन्धर्वान्योद्धुमाहवे}
{सुयोधनस्य मोक्षाय प्रयतध्वमतनद्रिताः}


\twolineshloka
{`परैः परिभवे प्राप्ते वयं पञ्चोत्तरं शतम्}
{परस्परविरोधे तु वयं पञ्चैव ते शतम् ॥'}


\twolineshloka
{य एव कश्चिद्राजन्यः शरणार्थमिहागतम्}
{परं शक्त्याभिरक्षेत किं पुनस्त्वं वृकोदर}


\twolineshloka
{`एवमुक्तस्तु कौन्तेयः पुनर्वाक्यमभाषत}
{कोपसंरक्तनयनः पूर्ववैरमनुस्मरन्}


\twolineshloka
{पुरा जतुगृहेऽनेन दग्धुमस्मान्युधिष्ठिर}
{दुर्बुद्धिर्हि कृता वीर तदा दैवेन रक्षिताः}


\twolineshloka
{कालकूटविषं तीक्ष्णं भोजने मम भारत}
{उप्त्वा गङ्गां लतापाशैर्वैद्ध्वा च प्राक्षिपत्प्रभो}


\twolineshloka
{रसातलं च संप्राप्य तदा वासुकिमञ्जसा}
{तत्र दृष्ट्वा तु राजेन्द्रपुनः प्राप्तो महीतलम्}


\threelineshloka
{द्यूतकालेऽपिकौन्तेय वृजिनानि कृतनि वै}
{द्रौपद्याश्च पराभर्शः केशग्रहणमेव च}
{वस्त्रापहरणं चैव सभामध्ये कृतानि वै}


\twolineshloka
{राज्यं चाच्छिद्य राजेन्द्र उक्तवान्परुषाणि नः}
{पुरा कृतानां पापानां फलं भुङ्क्ते सुयोधनः}


\fourlineindentedshloka
{अस्माबिरेवकर्तव्यं धार्तराष्ट्रस्य निग्रहम्}
{अन्येन तु कृतं तद्वै मैत्र्यमस्माकमिच्छता}
{उपकारी तु गन्धर्वो मा राजन्विमना भव ॥वैशंपायन उवाच}
{}


\twolineshloka
{एतस्मिन्नन्तरे राजंश्चित्रसेनेन वै हृतः}
{विललाप सुदुःखार्तो नीयमानः सुयोधनः}


\twolineshloka
{युधिष्ठिर महाबाहो सर्वधर्मभृतांवर}
{सपुत्रान्सहदारांश्च गन्धर्वेण हृतान्बलात्}


\twolineshloka
{पाण्डुपुत्र महाबाहो कौरवाणां यशस्कर'सर्वधर्मभृतां श्रेष्ठ गन्धर्वेण हृतं बलात्}
{रक्षस्व पुरुषव्याघ्र युधिष्ठिर महायशः}


\twolineshloka
{भ्रातरं ते महाबाहो बद्ध्वा नयति मामयम्}
{दुश्शासनं दुर्विषहं दुर्मुखं दुर्जयं तथा}


\twolineshloka
{बद्ध्वा हरन्ति गन्धर्वा अस्मान्दारांश्च सर्वशः}
{अनुधावत मां क्षिप्रं रक्षध्वं पुरुषोत्तमाः}


\threelineshloka
{यमौ मामनुधावेतां रक्षार्थं मम सायुधौ}
{कुरुवंशस्य सुमहदयशः प्राप्तमीदृशम्}
{व्यपोहयध्वं गन्धर्वाञ्जित्वा वीर्येण पाण्डवाः}


\twolineshloka
{एवं विलपमानस्य कौरवस्यार्तया गिरा}
{श्रुत्वा विलापं संभ्रान्तो घृणयाऽभिपरिप्लुतः}


\twolineshloka
{युधिष्ठिरः पुनर्वाक्यं भीमसेनमथाब्रवीत्}
{सुयोधनस्य मोक्षाय प्रयतध्वमतन्द्रिताः'}


\twolineshloka
{क इवार्यो दयेत्प्राणानभिधावेति चोदितः}
{प्राञ्जलं शरणापन्नं दृष्ट्वा शत्रुमपि ध्रुवम्}


\twolineshloka
{वरप्रदानं राज्यं च पुत्रजन्म च पाण्डवाः}
{शत्रोश्च मोक्षणं क्लेशास्त्रीणि चैकं च तत्समम्}


\twolineshloka
{न ह्यस्त्यधिकमेतस्माद्यदापन्नः सुयोधनः}
{त्वद्बाहुबलमाश्रित्य जीवितं परिमार्गते}


\twolineshloka
{स्वयमेव प्रधावेयं यदि न स्याद्वृकोदर}
{विततो मे क्रतुर्वीर न हि मेऽत्र विचारणा}


\twolineshloka
{साम्नैव तु यथा भीम मोक्षयेथाः सुयोधनम्}
{तथा सर्वैरुपायैस्त्वं यतेथाः कुरुनन्दन}


\twolineshloka
{न साम्ना प्रतिपद्येत यदि गन्धर्वराडसौ}
{पराक्रमेण मृदुना मोक्षयेथाः सुयोधनम्}


\twolineshloka
{अथासौ मृदुयुद्धेन न मुञ्चेद्भीम कौरवान्}
{सर्वोपायैर्पिमोच्यास्ते निगृह्य परिपन्थिनः}


\fourlineindentedshloka
{एतावद्धि मया शक्यं संदेष्टुं वै वृकोदर}
{वैताने कर्मणि तते वर्तमाने च भारत}
{`वरप्रदानं सुमहद्याचकस्य प्रकीर्तितम्' ॥वैशंपायन उवाच}
{}


\threelineshloka
{अजातसत्रोर्वचनं तच्छ्रुत्वा तु धनंजयः}
{प्रतिजज्ञे गुरोर्वाक्यं कौरवाणां विमोक्षणम् ॥अर्जुन उवाच}
{}


\twolineshloka
{यदि ककसाम्ना न मोक्ष्यन्ति गन्धर्वा धृतराष्ट्रजान्}
{अद्य गन्धर्वराजस्य भूमिः पास्यति शोणितम्}


\twolineshloka
{अर्जुनस्य तु तां श्रुत्वा प्रतिज्ञां सत्यवादिनः}
{कौरवाणां तदा राजन्पुनः प्रत्यागतं मनः}


\chapter{अध्यायः २४५}
\twolineshloka
{वैशंपायन उवाच}
{}


\twolineshloka
{युधिष्ठिरवचः श्रुत्वा भीमसेनपुरोगमाः}
{प्रहृष्टवदनाः सर्वे समुत्तस्थुर्नरर्षभाः}


\threelineshloka
{अभेद्यानि ततः सर्वे समनह्यन्त भारत}
{जाम्बूनदविचित्राणि कवचानि महारथाः}
{आयुधानि च दिव्यानि विविधानि समादधुः}


\twolineshloka
{ते दंशिता रथैः सर्वे ध्वजिनः सशरासनाः}
{पाण्डवाः प्रत्यदृश्यन्त ज्वलिता इवपावकाः}


\twolineshloka
{तान्रथान्साधुसंपन्नान्संयुक्ताञ्जवनैर्हयैः}
{आस्थाय रथशार्दूलाः शीघ्रमेव ययुस्ततः}


\twolineshloka
{ततः कौरवसैन्यानां प्रादुरासीन्महास्वनः}
{प्रयातान्सहितान्दृष्ट्वा पाण्डुपुत्रान्महारथान्}


\twolineshloka
{जितकाशिनश्च स्वचरास्त्वरिताश्च महारथाः}
{क्षणेनैव वने तस्मिन्समाजग्मुरभीतवत्}


\twolineshloka
{न्यवर्तन्त ततः सर्वे गन्धर्वा जितकाशिनः}
{दृष्ट्वा रथगतान्वीरान्पाण्डवांश्चतुरो रणे}


\twolineshloka
{तांस्तु विभ्राजितान्दृष्ट्वा लोकपालानिवोद्यतान्}
{व्यूढानीका व्यतिष्ठन्त गन्धमादनवासिनः}


\twolineshloka
{राज्ञस्तु वचनं श्रुत्वा धर्मपुत्रस्य धीमतः}
{क्रमेण मृदुना युद्धमुपक्रान्तं च भारत}


\twolineshloka
{न तु गन्धर्वराजस्य सैनिका मन्दचेतसः}
{शक्यन्ते मृदुना श्रेयः प्रतिपादयितुं तदा}


\twolineshloka
{ततस्तान्युधि दुर्धर्षान्सव्यसाची परंतपः}
{सान्त्वपूर्वमिदं वाक्यमुवाच खचरान्रणे}


\twolineshloka
{नैतद्गन्धर्वराजस्य युक्तं कर्म जुगुप्सितम्}
{परदाराभिमर्शश्च मानुषैश्च समागमः}


\twolineshloka
{उत्सृजध्वं महावीर्यान्धृतराष्ट्रसुतानिमान्}
{दारांश्चैषां प्रमुञ्चध्वं धर्मराजस्य शासनात्}


\twolineshloka
{त एवमुक्ता गन्धर्वाः पाण्डवेन यशस्विना}
{उत्स्मयन्तस्तदा पार्थमिदं वचनमब्रुवन्}


\twolineshloka
{एकस्यैव वयं तात कुर्याम वचनं भुवि}
{यस्य शासनमाज्ञाय चरामो विगतज्वराः}


\twolineshloka
{तेनैकेन रकयथादिष्टं तथा वर्ताम भारत}
{न शास्ता विद्यतेऽस्माकमन्यस्तस्मात्सुरेश्वरात्}


\twolineshloka
{एवमुक्तः स गन्धर्वैः कुन्तीपुत्रो धनंजयः}
{गन्धर्वान्पुनरेवैतान्वचनं प्रत्यभाषत}


\twolineshloka
{यदि साम्ना न मुञ्चध्वं गन्धर्वा धृतराष्ट्रजान्}
{मोक्षयिष्यामि विक्रम्य स्वयमेव सुयोधनम्}


\twolineshloka
{एवमुक्त्वा ततः पार्थः सव्यसाची धनंजयः}
{ससर्ज निशितान्बाणान्खचरान्खचरान्प्रति}


\twolineshloka
{तथैव शरवर्षेण गन्धर्वास्ते बलोत्काटाः}
{पाण्डवानभ्यवर्तन्त पाण्डवाश्च दिवौकसः}


\twolineshloka
{ततः सुतुमुलं युद्धं गन्धर्वाणां तरस्विनाम्}
{बभूव भीमवेगानां च भात}


\chapter{अध्यायः २४६}
\twolineshloka
{वैशंपायन उवाच}
{}


\twolineshloka
{ततो दिव्यास्त्रसंपन्ना गन्धर्वा हेममालिनः}
{विसृजन्तः शरान्दीप्तान्समन्तात्पर्यवारयन्}


\twolineshloka
{चतुरः पाण्डवान्वीरान्गन्धर्वाश्च सहस्रशः}
{रणे संन्यपतन्राजंस्तदद्भुतमिवाभवत्}


\twolineshloka
{यथा कर्णस्य च रथो धार्तराष्ट्रस् चोभयोः}
{गन्धर्वैः शतशश्छिन्नौ तथा तेषां प्रचक्रितरे}


\twolineshloka
{तान्समापततो राजन्गधर्वाञ्छतशो रणे}
{प्रत्यगृह्णन्नरव्याघ्राः शरवर्षैरनेकशः}


\twolineshloka
{ते कीर्यमाणाः खगमाः शरवर्षैः समन्ततः}
{न शेकुः पाण्डुपुत्राणां समीपे परिवर्तितुम्}


\twolineshloka
{अभिक्रुद्धानभिक्रुद्धो गन्धर्वानर्जुनस्तदा}
{लक्षयित्वाऽथ दिव्यानि महास्त्राण्युपचक्रमे}


\twolineshloka
{सहस्राणां सहस्राणि प्राहिणोद्यमसादनम्}
{अजेयानर्जुनः सङ्ख्ये गन्धर्वाणां बलोत्कटः}


\twolineshloka
{तथा भीमो महेष्वासः संयुगे बलिनांवरः}
{गन्धर्वाञ्शतशो राजञ्चघान निशितैः शरैः}


\twolineshloka
{माद्रीपुत्रावपि तथा युध्मानौ बलोत्कटौ}
{परिगृह्याग्रतो राजञ्जघ्नतुः शतशः परान्}


\twolineshloka
{तेवध्यमाना गन्धर्वा दिव्यैरस्त्रैर्महारथैः}
{उत्पेतुः खमुपादाय धृतराष्ट्रसुतांस्ततः}


\twolineshloka
{सतानुत्पतितानदृष्ट्वा कुन्तीपुत्रो धनंजयः}
{महता शरजालेन समन्तात्पर्यवारयत्}


% Check verse!
ते ब्रद्धाः शरजालेन शकुन्ता इव पञ्जरेववर्पुरर्जुनं क्रोधाद्गदाशक्त्यृष्टिवृष्टिभिः
\twolineshloka
{गदाशक्त्यृष्टिवृष्टीस्ता निहत्य परमास्त्रवित्}
{गात्राणि चाहनद्भल्लैर्गन्धर्वाणां धनंजयः}


\twolineshloka
{शिरोभिः प्रपतद्भिश्च चरणैर्बाहुभिस्तथा}
{अश्मवृष्टिरिवाभाति परेषामभवद्भयम्}


\twolineshloka
{ते वध्यमाना गन्धर्वाः पाण्डवेन महात्मना}
{भूमिष्ठमन्तरिक्षस्थाः शरवर्षैरवाकिरन्}


\twolineshloka
{तेषां तु शरवर्षाणि सव्यसाची परंतपः}
{अस्त्रैः संवार्य तेजस्वी गन्धर्वान्प्रत्यविध्यत}


\twolineshloka
{स्थूणाकर्णेन्द्रजालं च सौरं चापि तथाऽर्जुनः}
{आग्नेयं चापि सौम्यं च ससर्ज कुरुननदनः}


\twolineshloka
{ते दह्यमाना गन्धर्वाः कुन्तीपुत्रस्य सायकैः}
{दैतेया इव शक्रेण विषादमगमन्परम्}


\twolineshloka
{ऊर्ध्वमाक्रममाणाश्च शरजालेन वारिताः}
{विसर्पमाणा भल्लैश्च वार्यन्ते सव्यसाचिना}


\twolineshloka
{गन्धर्वांस्त्रासितान्दृष्ट्वा कुन्तीपुत्रेण भारत}
{चित्रसेनो गदां गृह्य सव्यसाचिनमनाद्रवत्}


\twolineshloka
{तस्याभिपततस्तूर्णं गदाहस्तस्य संयुगे}
{गदां सर्वायसीं पार्थः शरैश्चिच्छेद सप्तधा}


\twolineshloka
{स गदां बहुधा दृष्ट्वा कृत्तां बाणैस्तरस्विना}
{संवृत्य विद्ययाऽऽत्मानं योधयामास पाण्डवं}


\twolineshloka
{अस्त्राणइ तस्य दिव्यानि संप्रयुक्तानि सर्वशः}
{दिव्यैरस्त्रैस्तदा वीरः पर्यवारयदर्जुनः}


\twolineshloka
{स वार्यमाणस्तैरस्त्रैरर्जुनेन महात्मना}
{गन्धर्वराजो बलवान्माययाऽन्तर्हितस्तदा}


\twolineshloka
{अन्तर्हितं तमालक्ष्य प्रहरन्तमथार्जुनः}
{ताडयामास खचरैर्दिव्यास्त्रप्रतिमन्त्रितैः}


\twolineshloka
{अन्तर्धानवधं चास्य चक्रे क्रुद्धोऽर्जुनस्तदा}
{शब्दवेषं समाश्रित्य बहुरूपो धनंजयः}


\twolineshloka
{रस वध्यमानस्तैरस्त्रैरर्जुनेन महात्मना}
{ततोऽस्य दर्शयामास तदाऽऽत्मानं प्रियः सखा}


\twolineshloka
{चित्रसेनस्तथोवाच सखायं युधि विद्धि माम्}
{चित्रसेनमथालक्ष्य सखायमिति विस्मितः}


% Check verse!
संजहारास्त्रमथं तत्प्रसृष्टं पाण्डवर्षभः
\twolineshloka
{दृष्ट्वा तु पाण्डवाः सर्वे संहृतास्त्रं धनंजयम्}
{संजह्रुः प्रद्रुतानश्वाञ्शरवेगान्धनूंषि च}


\twolineshloka
{चित्रसेनश्च भीमश्च सव्यसाची यमावपि}
{पृष्ट्वा कौशलमन्योन्यं रथेष्वेवावतस्थिरे}


\chapter{अध्यायः २४७}
\twolineshloka
{वैशंपायन उवाच}
{}


\twolineshloka
{ततोऽर्जुनश्चित्रसेनं प्रहसन्निदमब्रवीत्}
{मध्ये गन्धर्वसैन्यानां महेष्वासो महाद्युतिः}


\threelineshloka
{किं ते व्यवसितं वीर कौरवाणां विनिग्रहे}
{किमर्थं च सदारोऽयं निगृहीतः सुयोधनः ॥चित्रसेन उवाच}
{}


\twolineshloka
{विदितोऽयमभिप्रायस्तत्रस्थेन दुरात्मनः}
{इन्द्रेण धार्तराष्ट्रस्य सकर्णस्य धनंजय}


\twolineshloka
{वनस्थान्भवतो ज्ञात्वा क्लिश्यमानाननर्हवत्}
{समस्थो विषमस्थांस्तान्द्रक्ष्यामीत्यनवस्थितान्}


\twolineshloka
{इमेऽवहसितुं प्राप्ता द्रौपदीं च यशस्विनीम्}
{ज्ञात्वा चिकीर्षितं चैषां मामुवाच सुरेश्वरः}


\twolineshloka
{गच्छ दुर्योधनं बद्ध्वा सहामात्यमिहानय}
{धनंजयश्च ते रक्ष्यः रसह भ्रातृभिराहवे}


\twolineshloka
{स च प्रियः सखा तुभ्यं शिष्यश् तव पाण्डवः}
{वचनाद्देवराजस् ततोऽस्मीहागतो द्रुतम्}


\threelineshloka
{अयं दुरात्मा बद्धश्च गमिष्यामि सुरालयम्}
{नेष्याम्येनं कदुरात्मानं पाकशासनशासनात् ॥अर्जुन उवाच}
{}


\threelineshloka
{उत्सृज्यतां चित्रसेन भ्राताऽस्माकं सुयोधनः}
{धर्मराजस्य संदेशान्मम चेदिच्छसि प्रियम् ॥चित्रसेन उवाच}
{}


\twolineshloka
{पापोऽयं नित्यसंदुष्टो न विमोक्षणमर्हति}
{प्रलब्धा धर्मराजस्य कृष्णायाश्च धनंजय}


\threelineshloka
{नेदं चिकीर्षितं तस् कुन्तीपुत्रो युधिष्ठिरः}
{जानाति धर्मराजो हि श्रुत्वा कुरु यथेच्छसि ॥वैशंपायन उवाच}
{}


\twolineshloka
{ते सर्व एव राजनमभिजग्मुर्युधिष्ठिरम्}
{अभिगम्य च तत्सर्वं शशंसुस्तस्य चेष्टितम्}


\twolineshloka
{अजातशत्रुस्तच्छ्रुत्वा गन्धर्वस्य वचस्तदा}
{मोक्षयामास तान्सर्वान्गन्धर्वान्प्रशशंस च}


\twolineshloka
{`चित्रसेनस्तदा वाक्यमुवाच प्रौढया गिरा}
{मुञ्चध्वंसानुजामात्यं सदारं च सुयोधनम्}


\twolineshloka
{गन्धर्वास्तु वचः श्रुत्वा चित्रसेनस्य वै द्रुतम्}
{राजानं मोचयामासुर्बद्धं निगडबन्धनैः}


\twolineshloka
{सदारं सानुगामात्यं बालजालमयेन ये}
{लुछन्तश्चापि ते सर्वे युधिष्ठिरसमीपतः}


\twolineshloka
{पतिता लज्जिताश्चैव तस्थुश्चाधोमुखास्तदा}
{युधिष्ठिरोपि दयया तान्समीक्ष्य तथागतान्}


\twolineshloka
{दिष्ट्या भवद्भिर्बलिभिः शक्तैः सर्वैर्न हिंसितः}
{दुर्वृत्तो धार्तराष्ट्रोऽयं सामात्यज्ञातिबान्धवः}


\twolineshloka
{उपकारो मसांस्तात कृतोऽयं मम खेचर}
{कुलं न परिभूतं मे मोक्षेणास्य दुरात्मनः}


\twolineshloka
{आज्ञापयध्वमिष्टानि प्रीतं मां दर्शनेन वः}
{प्राप्य सर्वानभिप्रायांस्ततो व्रजत मा चिरम्}


\twolineshloka
{अनुज्ञातास्तु गन्धर्वाः पाण्डुपुत्रेण धीमता}
{सहाप्सरोभिः संहृष्टाश्चित्रसेनमुखा ययुः}


\twolineshloka
{`देवलोकं ततो गत्वा गन्धर्वैः सहितस्तदा}
{न्यवेदयच्च तत्सर्वं चित्रसेनः शतक्रतोः'}


\twolineshloka
{देवराडपि गन्धर्वान्मृतांस्तान्समजीवयत्}
{दिव्येनामृतवर्षेण ये हताः कौरवैर्युधि}


\twolineshloka
{ज्ञातींस्तानवमुच्याथ राजदारांश्च सर्वशः}
{कृत्वा च रदुष्करं कर्म प्रीतियुक्ताश्च पाण्डवाः}


\twolineshloka
{सस्त्रीकुमारैः कुरुभिः पूज्यमाना महारथाः}
{बभ्राजिरे महात्मानः क्रतुमध्ये यथाऽग्नयः}


\twolineshloka
{ततो दुर्योधनं मुक्तं भ्रातृभिः सहितस्तदा}
{युधिष्ठिरस्तु प्रणयादिदं वचनमब्रवीत्}


\twolineshloka
{रमा स्म तात पुनः कार्षीरीदृशं साहसं क्वचित्}
{न हि साहसकर्तारः सुखमेघन्ति भारत}


\threelineshloka
{स्वस्तिमान्सहितः सर्वैर्ब्रातृभिः कुरुनन्दन}
{गृहान्व्रज यथाकामं वैमनस्यं च मा कृथाः ॥वैशंपायन उवाच}
{}


\threelineshloka
{पाण्डवेनाभ्यनुज्ञानो राजा दुर्योधनस्तदा}
{अभिवाद्य धर्मपुत्रं गतेन्द्रिय इवातुरः}
{विदीर्यमाणो व्रीडावाञ्जगाम नगरं प्रति}


\twolineshloka
{तस्मिन्गते कौरवेये कुन्तीपुत्रो युधिष्ठिरः}
{भ्रातृभिः सहितो वीरः पूज्यमानो द्विजातिभिः}


\twolineshloka
{तपोधनैश्च तैः सर्वैर्वृतः शक्र इवामरैः}
{तथा द्वैतवने तस्मिन्विजहार मुदा युतः}


\chapter{अध्यायः २४८}
\twolineshloka
{जनमेजय उवाच}
{}


\twolineshloka
{शत्रुभिर्जितबद्धस् पाण्डवैश्च महात्मभिः}
{मोक्षितस्य युधा पश्चान्मानिनः सुदुरात्मनः}


\twolineshloka
{कत्थनस्यावलिप्तस्य गर्वितस् च नित्यशः}
{सदा च पौरुषौदार्यैः पाण्डवानवमन्यतः}


\twolineshloka
{दुर्योधनस् पापस्य नित्याहंकारवादिनः}
{प्रवेशो हास्तिनपुरे दुष्करः प्रतिभाति मे}


\threelineshloka
{तस्य लज्जानवितस्यैव शोकव्याकुलचेतसः}
{प्रवेशं विस्तरेण रत्वं वैशंपायन कीर्तय ॥वैशंपायन उवाच}
{}


\twolineshloka
{धर्मराजनिसृष्स्तु धार्तराष्ट्रः सुयोधनः}
{लज्जयाऽधोमुखः सीदन्नुपासर्पत्सुदुःखितः}


\twolineshloka
{स्वपुरं प्रययौ राजा चतुरङ्गबलानुगः}
{शोकोदहतया बुद्ध्या चिन्तयानः पराभवम्}


\threelineshloka
{विमुच्य पथि यानानि देशे सुयवसोदके}
{सन्निविष्टः शुभे रम्ये भूमिभागे यथेप्सितम्}
{हस्त्यश्वरथपादातं यथास्थानं न्यवेशयत्}


\threelineshloka
{अथोपविष्टं राजानं पर्यङ्के ज्वलनप्रभे}
{उपप्लुतं यथा सोमं राहुणा रात्रिसंक्षये}
{उपागम्याब्रवीत्कर्णो दुर्योधनमिदं तदा}


\twolineshloka
{दिष्ट्या जीवसि गान्धारे दिष्ठ्या नः संगमः पुनः}
{दिष्ट्या त्वया जिताश्चैव गन्धर्वाः कामरूपिणः}


\twolineshloka
{दिष्ट्या समग्रान्पश्यामि भ्रातॄस्ते कुरुनन्दन}
{विजिगीषून्रणे युक्तान्निर्जितारीन्महारथान्}


\threelineshloka
{अहं त्वभिद्रुतः सर्वैर्गन्धर्वैः पश्यतस्व}
{नाशक्नुवं स्थापयितुं दीर्यमाणां च वाहिनीम्}
{शरक्षताङ्गश्च भृशं व्यपयातोऽभिपीडितः}


\threelineshloka
{इदं त्वत्यद्भुतं मन्ये यद्युष्मानिह भारत}
{अरिष्टानक्षसांश्चापि सदारबलवाहनान्}
{विमुक्तान्संप्रपश्यामि युद्धात्तस्मादमानुषात्}


\twolineshloka
{नैतस्य कर्ता लोकेऽस्मिन्पुमान्विद्यति भारत}
{यत्कृतं ते महाराज सह भ्रातृभिराहवे}


\twolineshloka
{एवमुक्तस्तु कर्णेन राजा दुर्योधनस्तदा}
{उवाचावाक्शिरा राजन्बाष्पगद्गदया गिरा}


\chapter{अध्यायः २४९}
\twolineshloka
{दुर्योधन उवाच}
{}


\twolineshloka
{अजानतस्ते राधेय नाभ्यसूयाम्यहं वचः}
{जानासि त्वं जिताञ्शत्रून्गन्धर्वां स्तेजसा मया}


\twolineshloka
{आयोधितास्तु गन्धर्वाः सुचिरं सोदरैर्मम}
{मया सह महाबाहो कृतश्चोभयतः क्षयः}


\twolineshloka
{मायाधिकास्त्वयुध्यन्त यदा शूरा वियद्गताः}
{तदा नो न समं युद्धमभवत्खेचरैः सह}


\threelineshloka
{पराजयं च प्राप्ताः स्मो रणे बन्धनमेव च}
{सभृत्यामात्यपुत्राश्च सदारबलवाहनाः}
{उच्चैराकाशमार्गेण ह्रियमाणाः सुदुःखिताः}


\twolineshloka
{अथ नः सैनिकाः केचिदमात्याश्च महारथाः}
{उपगम्याब्रुवन्दीनाः पाण्डवाञ्शरणप्रदान्}


\twolineshloka
{एष दुर्योधनो राजा धार्तराष्ट्रः सहानुजः}
{सामात्यदारो ह्रियते गन्धर्वैर्दिवमाश्रितैः}


\threelineshloka
{तं मोक्षयत भद्रं वः सहदारं नराधिपम्}
{परामर्शो मा भविष्यत्कुरुदारेषु सर्वशः}
{`इत्यब्रुवन्रणआन्मुक्ता धर्मराजमुपागताः'}


\twolineshloka
{एवमुक्ते तु धर्मात्मा ज्येष्ठः पाण्डुसुतस्तदा}
{प्रसाद्य सोदरान्सर्वानाज्ञापयत मोक्षणे}


\twolineshloka
{अथागम्य तमुद्देशं पाण्डवाः पुरुषर्षभाः}
{सान्त्वपूर्वमयाचन्त शक्ताः सन्तो महारथाः}


\twolineshloka
{यदा चास्मान्न मुमुचुर्गन्धर्वाः सान्त्विता अपि}
{`आकाशचारिणो वीरा नदन्तो जलदा इव'}


\twolineshloka
{ततोऽर्जुनश्च भीमश्च यमजौ च बलोत्कटौ}
{मुमुचुः शरवर्षाणि गन्धर्वान्प्रत्यनेकशः}


\twolineshloka
{अथ सर्वे रणं मुक्त्वा प्रयाताः खेचरा दिवम्}
{अस्मानेवाभिकर्षन्तो दीनान्मुदितमानसाः}


\twolineshloka
{ततः समन्तात्पश्यामः शरजालेन वेष्टितम्}
{अमानुषाणि चास्त्राणि प्रयुञ्जानं घनंजयम्}


\twolineshloka
{समावृता दिशो दृष्ट्वा पाण्डवेन शितैः शरैः}
{धनंजयसखाऽऽत्मानं दर्शयामास वै तदा}


\twolineshloka
{चित्रसेनः पाण्डवेन समाश्लिष्य परस्परम्}
{कुशलं परिपप्रच्छ तैः पृष्टश्चाप्यनामयम्}


\twolineshloka
{ते समेत्य तथाऽन्योन्यं सन्नाहान्विप्रमुच्य च}
{एकीभूतास्ततो वीरा गन्धर्वाः सह पाण्डवैः}


\twolineshloka
{`परस्परं समागम्य प्रीत्या परमया युतौ'}
{अपूजयेतामन्योन्यं चित्रसेनधनंजयौ}


\chapter{अध्यायः २५०}
\twolineshloka
{दुर्योधन उवाच}
{}


\twolineshloka
{चित्रसेनं समागम्य प्रहसन्नर्जुनस्तदा}
{इदं वचनमक्लीवमब्रवीत्परवीरहा}


\twolineshloka
{भ्रातृनर्हसि मे वीर मोक्तुं गन्धर्वसत्तम}
{अनर्हधर्षणा हीमे जीवमानेषु पाण्डुषु}


\twolineshloka
{एवमुक्तस्तु गन्धर्वः पाण्डवेन महात्मना}
{उवाच यत्कर्ण वयं मन्त्रयन्तो विनिर्गताः}


% Check verse!
`स्थितोराज्येच्युतान्स्थानाच्छ्रियाहीनांश्रियावृतः'द्रष्टास्मि निःसुखान्वीरान्सदारान्पाण्डवानिति
\twolineshloka
{तस्मिन्नुच्चार्यमाणे तु गन्धर्वेण वचस्तथा}
{भूमेर्विवरमन्वैच्छं प्रवेष्टुं व्रीडयाऽन्वितः}


\twolineshloka
{युधिष्ठिरमथागम् गन्धर्वाः सह पाण्डवैः}
{अस्मद्दुर्मन्त्रितं तस्मै बद्धांश्चास्मान्न्यवेदयन्}


\twolineshloka
{स्त्रीसमक्षमहं दीनो बद्धः शत्रुवशं गतः}
{युधिष्ठिरस्योपहृतः किंनु दुःखमतः परम्}


\twolineshloka
{ये मे निराकृता नित्यं रिपुर्येषामहं सदा}
{तैर्मोक्षितोऽहं दुर्बुद्धिर्दत्तं तैरेव जीवितम्}


\twolineshloka
{प्राप्तः स्यां यद्यहं वीर वधं तस्मिन्महारणे}
{श्रेयस्तद्भविता मह्यं नैवंभूतस्य जीवितम्}


\twolineshloka
{भवेद्यशः पृथिव्यां मे ख्यातं गन्धर्वतो वधात्}
{प्राप्ताश्च पुण्यलोकाः स्युर्महेन्द्रसदनेऽक्षयाः}


\threelineshloka
{यत्त्वद्य मे व्यवसितं तच्छृणुध्वं नरर्षभाः}
{इह प्रायमुपासिष्ये यूयं व्रजत वै गृहान्}
{भ्रातरश्चैव मे सर्वे यान्त्वद्य स्वपुरं प्रति}


\twolineshloka
{कर्णप्रभृतयश्चैव सुहृदो बान्धवाश्च ये}
{दुःशासनं पुरस्कृत्य प्रयान्त्वद्य पुरं प्रति}


\twolineshloka
{न ह्यहं संप्रयास्यामि पुरं शत्रुनिराकृतः}
{शत्रुमानापहो भूत्वा सुहृदां मानकृत्तथा}


\threelineshloka
{`कामं रणशिरस्यद्य शत्रुभिर्वै विमानितः'}
{स सुहृच्छोकदो जातः शत्रूणां हर्वर्धनः}
{वारणाह्वयमासाद्य किं वक्ष्यामि जनाधिपम्}


\twolineshloka
{भीष्मद्रोणौ कृपद्रौणी विदुरः संजयस्तथा}
{बाह्लीकः सौमदत्तिश्च ये चान्ये वृद्धसंमताः}


\twolineshloka
{ब्राह्मणाः श्रेणिमुख्याश्च तथोदासीनवृत्तयः}
{किं मां वक्ष्यंति किं चापि प्रतिवक्ष्यामि तानहं}


\twolineshloka
{रिपूणां शिरसि स्थित्वा तथा विक्रम्य चोरसि}
{आत्मदोषात्परिभ्रष्टः कथं वक्ष्यामि तानहम्}


\twolineshloka
{दुर्विनीताः श्रियं प्राप्य विद्यामैश्वर्यमेव च}
{तिष्ठन्ति न चिरं भद्रे यथाऽहं मदगर्वितः}


\twolineshloka
{अहो वत यथेदं मे कष्टं दुश्चरितं कृतम्}
{स्वयं दुर्बुद्धिना मोहाद्येन प्राप्तोस्मि संशयम्}


\twolineshloka
{तस्मात्प्रायमुपासिष्ये न हिशक्ष्यामि जीवितुम्}
{चेतयानो हि को जीवेत्कृच्छ्राच्छत्रुभिरुद्धृतः}


\threelineshloka
{शत्रुभिश्चावहसितो मानी पौरुषवर्जितः}
{पाण्डवैर्विक्रमाढ्यैशच् सावमानमवेक्षितः ॥वैशंपायन उवाच}
{}


\twolineshloka
{एवं चिन्तापरिगतो दुःशासनमथाब्रवीत्}
{दुःशासन निबोधेदं वचनं मम भारत}


\twolineshloka
{प्रतीच्छ त्वं मया दत्तमभिषेकं नृपो भव}
{प्रशाधि पृथिवीं स्फीतांकर्णसौबलपालिताम्}


\twolineshloka
{भ्रातॄन्पालय विस्रब्धं मरुतो वृत्रहा यथा}
{बान्धवाश्चोपजीवन्तु देवा इव शतक्रतुम्}


\twolineshloka
{ब्राह्मणेषु सदा वृत्तिं कुर्वीथाश्चाप्रमादतः}
{बन्धूनां सुहृदां चैव भवेथास्त्वं गतिः सदा}


\twolineshloka
{ज्ञातींश्चाप्यनुपश्येथा विष्णुर्देवगणान्यथा}
{गुरवः पालनीयास्ते गच्छ पालय मेदिनीम्}


\twolineshloka
{नन्दयन्सुहृदः सर्वाञ्शात्रवांश्चावभर्त्सयन्}
{कण्ठे चैनं परिष्वज्यगम्यतामित्युवाच ह}


\twolineshloka
{तस्य तद्वचनं श्रुत्वा दीनो दुशासनोऽब्रवीत्}
{अश्रुकण्ठः सुदुःखार्तः प्राञ्जलिः प्रणिपत्य च}


\twolineshloka
{सगद्गदमिदं वाक्यं भ्रातरं ज्येष्ठमात्मनः}
{कप्रसीदेत्यपतद्भूमौ दूयमानेन चेतसा}


\twolineshloka
{दुःखितः पादयोस्तस्य नेत्रजं जलमुत्सृजन्}
{उक्तवांश्च नरव्याघ्रो नैतदेवं भविष्यति}


\twolineshloka
{विदीर्यत्सकला भूमिर्द्यौश्चापि शकलीभवेत्}
{रविरात्मप्रभां जह्यात्सोमः शीतांशुतां त्यजेत्}


\twolineshloka
{वायुः शैघ्र्यमथो जह्याद्धिमवांश्च हिमं त्यजेत्}
{शुष्येत्तोयं समुद्रेषु वह्निरप्युष्णतां त्यजेत्}


\twolineshloka
{न चाहं त्वदृते राजन्प्रशासेयं वसुंधराम्}
{पुनःपुनः प्रसीदेति वाक्यं चेदमुवाच ह}


% Check verse!
`शत्रूणां शोककृद्राजन्सुहृदां शोकनाशनः'त्वमेव नःकुलेराजा भविष्यसि शतं समाः
\twolineshloka
{एवमुक्त्वा स राजानं सुश्वरं प्ररुरोद ह}
{पादौ संस्पृश्य मानार्हौ भ्रातुर्ज्येष्ठस्य भारत}


\twolineshloka
{तथा तौ दुःखितौ दृष्ट्वा दुःशासनसुयोधनौ}
{अभिगम्य व्यथाविष्टः कर्णस्तौ प्रत्यभाषत}


\twolineshloka
{विषीदथः किं कौरव्यौ बालिश्यात्प्राकृताविव}
{न शोकः शोचमानस्य विनिवर्तेत कर्हिचित्}


\twolineshloka
{यदा च शोचतः शोकोव्यसनं नापकर्षति}
{सामर्थ्यं किं ततः शोके शोचमानौ प्रपश्यथः}


\twolineshloka
{धृतिं गृह्णीतं मा शत्रूञ्शोचन्तौ नन्दयिष्यथः}
{कर्तव्यं हि कृतं राजन्पाण्डवैस्तव मोक्षणम्}


\twolineshloka
{नित्यमेव प्रियं कार्यं राज्ञो विपयवासिभिः}
{पाल्यमानास्त्वया ते हि निवसन्ति गतज्वराः}


\twolineshloka
{नार्हस्येवंगते मन्युं कर्तुं प्राकृतवत्स्वयम्}
{विषण्णास्तव सोदर्यास्त्वयि प्रायं समास्थिते}


\twolineshloka
{`तदलं दुःखितानेतान्कर्तुं सर्वान्नराधिप'}
{उत्तिष्ठ व्रज भद्रं ते समाश्वासय सोदरान्}


\chapter{अध्यायः २५१}
\twolineshloka
{कर्ण उवाच}
{}


\twolineshloka
{राजन्नद्यावगच्छामि तवैव लघुसत्वताम्}
{`अल्पत्वं च तथा बुद्धेः कार्याणामविवेकिताम्'}


\twolineshloka
{किमत्र चित्रं धर्मज्ञ मोक्षितः पाण्डवैरसि}
{सद्यो वशं समापन्नः शत्रूणां शत्रुकर्शन}


\twolineshloka
{सेनाजीवैश्च कौरव्य तथा विषयवासिभिः}
{अज्ञातैर्यदि वा ज्ञातैः कर्तव्यं नृपतेः प्रियम्}


\twolineshloka
{प्रायः प्रधानाः पुरुषाः क्षोभयित्वाऽरिवाहिनीम्}
{निगृह्यन्ते चयुद्धेषु मोक्ष्यन्ते च स्वसैनिकैः}


\twolineshloka
{सेनाजीवाश्च ये राज्ञां विषये सन्ति मानवाः}
{तैः संगम्य नृपार्थाय यतितव्यं यथातथम्}


\twolineshloka
{यद्येवं पाण्डवै राजन्भवद्विषयवासिभिः}
{यदृच्छया मोक्षितोऽसि कतत्रका परिदेवना}


\twolineshloka
{न चैतत्साधु यद्राजन्पाण्डवास्त्वां नृपोत्तमम्}
{स्वसेनया संप्रयान्तं नानुयान्ति स्म पृष्ठतः}


\twolineshloka
{शूराश्च बलवन्तश्च संयुगेष्वपलायिनः}
{भवतस्ते सभायां वै प्रेष्यतां पूर्वमागताः}


\twolineshloka
{पाण्डवेयानि रत्नानि त्वमद्याप्युपभुञ्जसे}
{सत्वस्थान्पाण्डवान्पश्य न ते प्रायमुपाविशन्}


\twolineshloka
{`तदलं ते महाबाहो विषादं कर्तुमीदृशम्'}
{उत्तिष्ठ रराजन्भद्रं ते न चिन्तां कर्तुमर्हसि}


\twolineshloka
{अवश्यमेव नृपते राज्ञो विषयवासिभिः}
{प्रियाण्याचरितव्यानि तत्र का परिदेवना}


\twolineshloka
{मद्वाक्यमेतद्राजेन्द्र यद्येवं न करिष्यसि}
{स्थास्यामीह भवत्पादौ शुश्रूषन्नरिमर्दन}


\threelineshloka
{नोत्सहे जीवितुमहं त्वद्विहीनो नरर्षभः}
{प्रायोपविष्टस्तु तथा राज्ञां हास्यो भविष्यसि ॥वैशंबायन उवाच}
{}


\twolineshloka
{एवमुक्तस्तु कर्णेन राजा दुर्योधनस्तदा}
{नैवोत्थातुं मनश्चक्रे स्वर्गाय कृतनिश्चयः}


\chapter{अध्यायः २५२}
\twolineshloka
{वैशंपायन उवाच}
{}


\twolineshloka
{प्रायोपविष्टं राजानं दुर्योधनममर्षणम्}
{उवाच सान्त्वयन्राजञ्शकुनिः सौबलस्तदा}


\twolineshloka
{सम्यगुक्तं हि कर्णेन तच्छ्रुतं कौरव त्वया}
{मया हृतां श्रियं स्फीतां तां मोहादपहासि किम्}


\twolineshloka
{त्वमबुद्ध्या नृपवर प्राणानुत्स्रष्टुमिच्छसि}
{अद्य वाऽप्यवगन्छामि न वृद्धाः सेवितास्त्वया}


\twolineshloka
{यः समुत्पतितं हर्षं दैन्यं वा न नियच्छति}
{स नश्यति श्रियं प्राप्य पात्रमाममिवाम्भसि}


\twolineshloka
{अतिभीरुं मृदुं क्लीबं दीर्घसूत्रं प्रमादिनम्}
{व्यसनाद्विषयाक्रान्तं न भजन्ति नृपं श्रिताः}


\twolineshloka
{सत्कृतस्य हि ते शोको विपरीते कथं भवेत्}
{मा कृतं शोभनं पार्थैः शोकमालम्ब्य नाशय}


\twolineshloka
{यत्र हर्षस्त्वया कार्यः सत्कर्तव्याश्च पाण्डवाः}
{तत्र शोचसि राजेन्द्रविपरीतमिदं तव}


\threelineshloka
{प्रसीदमा त्यजात्मानं तुष्टश्च सुकृतं स्मर}
{कप्रयच्छ राज्यं पार्थानां यशो धर्ममवाप्नुहि}
{क्रियामेतां समाज्ञाय कृतघ्नो न भविष्यसि}


\threelineshloka
{सौभ्रात्रं पाण्डवैः कृत्वा समवस्ताप्य चैव तान्}
{पित्र्यं राज्यं प्रयच्छैषां ततः सुखमवाप्स्यसि ॥वैशंपायन उवाच}
{}


\twolineshloka
{शकुनेस्तु वचः श्रुत्वा दुःशासनमवेक्ष्य च}
{पादयोः पतितं वीरं विकृतं भ्रातृसौहृदात्}


\twolineshloka
{बाहुभ्यां साधुजाताभ्यां दुःशासनमरिंदमम्}
{उत्थाप्य संपरिष्वज्य प्रीत्याऽजिघ्रत मूर्धनि}


\threelineshloka
{कर्णसौबलयोश्चापि संश्रुत्य वचनान्यसौ}
{निर्वेदं परमं गत्वा राजा दुर्योधनस्तदा}
{व्रीडयाऽभिपरीतात्मा नैराश्यमगमत्परम्}


\twolineshloka
{सुहृदां चैव तच्छ्रुत्वा समन्युरिदमब्रवीत्}
{न धर्मधनसौख्येन नैश्वर्येण न चाज्ञया}


\twolineshloka
{नैव भोगैश्च मे कार्यं मा विहन्यत गच्छत}
{निश्चितेयं मम मतिः स्थिता प्रायोपवेशने}


\twolineshloka
{गच्छध्वं नगरं सर्वे पूज्याश्च गुरवो मम}
{त एवमुक्ताः प्रत्यूच् राजानमरिमर्दनम्}


\threelineshloka
{या गतिस्तव राजेन्द्र साऽस्माकमपि भारत}
{कथं वा संप्रवेक्ष्यामस्त्वद्विहीनाः पुरं वयम् ॥वैशंपायन उवाच}
{}


\twolineshloka
{स सुहृद्भिरमात्यैश्च भातृभिः स्वजनेन च}
{बहुप्रकारमप्युक्तो निश्चयान्न व्यचाल्यत}


\twolineshloka
{दर्भास्तरणमास्तीर्य निश्चयाद्धृतराष्ट्रजः}
{संस्पृश्यापः शुचिर्भूत्वा भूतले समुपस्थितः}


\twolineshloka
{कुशचीराम्बरधरः परं नियममास्थितः}
{वाग्यतो राजशार्दूलः स स्वर्गगतिकाम्यया}


\twolineshloka
{मनसोपचितिं कृत्वा निरस् च बहिःक्रियाः}
{`तस्थौप्रायोपवेशेऽथमतिं कृत्वा सुनिश्चयाम्'}


\twolineshloka
{अथ तं निश्चयं तस्य बुद्ध्वा दैतेयदानवाः}
{पातालवासिनो रौद्राः पूर्वं देवैर्विनिर्जिताः}


\twolineshloka
{ते स्वपक्षक्षयं तं तु ज्ञात्वा दुर्योधनस्य वै}
{आह्वानाय तदा चक्रुः कर्म वैतानसंभवम्}


\threelineshloka
{बृहस्पत्युशनोक्तैश्च मन्त्रैर्मन्त्रविशारदाः}
{अथर्ववेदप्रोक्तैश्च याश्चौपनिषदाः क्रियाः}
{मन्त्रजप्यसमायुक्तास्तास्तदा समवर्तयन्}


\twolineshloka
{जुह्वत्यग्नौ हविः क्षीरं मन्त्रवत्सुसमाहिताः}
{ब्राह्मणा वेदवेदाङ्गपारगाः सुदृढव्रताः}


% Check verse!
`अध्वर्यवो दानवानां कर्म प्रावर्तयंस्ततः'
\twolineshloka
{कर्मसिद्धौ तदा तत्र जृम्भमाणा महाद्भुता}
{कृत्या समुत्थिता राजन्किं करोमीति चाब्रवीत्}


\twolineshloka
{आहुर्दैत्याश्च तां तत्र सुप्रीतेनान्तरात्मना}
{प्रायोपविष्टं राजानं धार्तराष्ट्रमिहानय}


\twolineshloka
{तथेति च प्रतिश्रुत्य सा कृत्या प्रययौ तदा}
{निमेषादगमच्चापि यत्र राजा सुयोधनः}


\twolineshloka
{समादाय च राजानं प्रविवेश रसातलम्}
{दानवानां मुहूर्ताच्च पुरतस्तं न्यवेदयत्}


\twolineshloka
{तमानीतं नृपं दृष्ट्वा रात्रौ संगत्य दानवाः}
{प्रहृष्टमनसः सर्वे किंचिदुत्फुल्ललोचनाः}


\twolineshloka
{`दृढमेनं परिष्यज्य दृष्ट्वा च कुशलं तदा'}
{साभिमानमिदं वाक्यं दुर्योधनमथाब्रुवन्}


\chapter{अध्यायः २५३}
\twolineshloka
{दानवा ऊचुः}
{}


\twolineshloka
{भोः सुयोधन राजेन्द्र भरतानां कुलोद्वह}
{शूरैः परिवृतो नित्यं तथैव च महात्मभिः}


\twolineshloka
{अकार्षीः साहसमिदं कस्मात्प्रायोपवेशनम्}
{आत्मत्यागी ह्यधो याति वाच्यतां चायशस्करीम्}


\twolineshloka
{न हि कार्यविरुद्धेषु बहुपापेषु कर्मसु}
{मूलघातिषु सज्जन्ते बुद्धिमन्तो भवद्विधाः}


\twolineshloka
{नियच्छैनां मतिं राजन्धर्मार्थसुखनाशिनीम्}
{यशःप्रतापवीर्यघ्नीं शत्रूणां हर्षवर्धनीम्}


\twolineshloka
{श्रूयतां तु प्रभो तत्त्वं दिव्यतां चात्मनो नृप}
{निर्माणं च रशरीरस्य ततो धैर्यमवाप्नुहि}


\twolineshloka
{पुरा त्व तपसाऽस्माभिर्लब्धो राजन्महेश्वरात्}
{पूर्वकायश्च ते सर्वो निर्मितो वज्रसंचयैः}


\twolineshloka
{अस्त्रैरभेद्यः शस्त्रैश्चाप्यधःकायश्च तेऽनघ}
{कृतः पुष्पमयो देव्या रूपतस्त्रीमनोहरः}


\twolineshloka
{एवमीश्वरसंयुक्तस्तव देहो नृपोत्तम}
{देव्या च राजशार्दूल दिव्यस्त्वं हि न मानुषः}


\twolineshloka
{क्षत्रियाश्च महावीर्या भगदत्तपुरोगमाः}
{दिव्यास्त्रविदुषः शूराः क्षपयिष्यन्ति ते रिपून्}


\twolineshloka
{तदलं ते विषादेन भयं तव न विद्यते}
{सहायार्थं च ते वीराः संभूता बुवि दानवाः}


\twolineshloka
{भीष्मद्रोणकृपादींश्च प्रवेक्ष्यन्त्यपरेऽसुराः}
{यैराविष्टा घृणां त्यक्त्वा योत्स्यन्ते तव वैरिभिः}


\threelineshloka
{नैव पुत्रान्न च भ्रातॄन्न पितॄन्न च बान्धवान्}
{नैव शिष्यान्न च ज्ञातीन्न बालान्त्यविरान्न च}
{युधि संप्रहरिष्यन्तो मोक्ष्यन्ति कुरुसत्तम}


\twolineshloka
{निःस्नेहा दानवाविष्टाः समाक्रान्तेऽन्तरात्मनि}
{प्रहरिष्यन्ति बन्धुभ्यः स्नेहमुत्सृज्य दूरतः}


\threelineshloka
{हृष्टाः पुरुषशार्दूलाः कलुषीकृतमानसाः}
{अनभिज्ञातमूलाश्च दैवाच्च विधिनिर्मितात्}
{व्याभाषमाणाश्चान्योन्यं न मे जीवन्विमोक्ष्यसे}


\twolineshloka
{सर्वे शस्त्रास्त्रमोक्षेण पौरुषे समवस्थिताः}
{श्लाघमानाः कुरुश्रेष्ठ करिष्यन्ति जनक्षयम्}


\twolineshloka
{तेपि पञ्च महात्मानः प्रतियोत्सन्ति पाण्डवाः}
{वधं चैषां करिष्यन्ति दैवयुक्ता महाबलाः}


\twolineshloka
{दैत्यरक्षोगणाश्चैव संभूताः क्षत्रयोनिषु}
{योत्स्यन्ति युधि विक्रम्य शत्रुभिस्तव पार्थिव}


\twolineshloka
{गदाभिर्मुसलैः शूलैः शस्त्रैरुच्चावचैस्तथा}
{`प्रहरिष्यनति ते वीरास्तवारिषु महाबलाः'}


\twolineshloka
{यच्च तेऽन्तर्गतं वीर भयमर्जुनसंभवम्}
{तत्रापि विहितोऽस्माभिर्वधोपायोऽर्जुनस्य वै}


\twolineshloka
{हतस्य नरकस्यात्मा कर्णमूर्तिमुपाश्रितः}
{तद्वैरं संस्मरन्वीर योत्स्यते केशवार्जुनौ}


\twolineshloka
{स ते विक्रमशौण्डीरो रणे पार्थं विजेष्यति}
{कर्णः प्रहरतांश्रेष्ठः सर्वांश्चारीन्महारथः}


\twolineshloka
{ज्ञात्वैतच्छद्मना वज्री रक्षार्थं सव्यसाचिनः}
{कुण्डले कवचं चैव कर्णस्यापहरिष्यति}


\threelineshloka
{तस्मादस्माभिरप्यत्र दैत्याः शतसहस्रशः}
{नियुक्ता राक्षसाश्चैव ये ते संशप्तका इति}
{प्रख्यातास्तेऽर्जुनं वीरं युधि हिंस्यन्ति मा शुचः}


\twolineshloka
{असपत्ना त्वया हीयं भोक्तव्या वसुधा नृप}
{मा विषादं गमस्तस्मान्नैतत्त्वय्युपपद्यते}


\fourlineindentedshloka
{विनष्टे त्वयि चास्माकं पक्षो हीयेत कौरव}
{गच्छ वीर न ते बुद्धिरन्या कार्या कथंचन}
{त्वमस्माकं गतिर्नित्यं देवतानां च पाण्डवाः ॥वैशंपायन उवाच}
{}


\twolineshloka
{एवमुक्त्वा परिष्वज्य दैत्यास्तं राजकुञ्जरम्}
{समाश्वास्य च दुर्धर्षं पुत्रवद्दानवर्षभाः}


\twolineshloka
{स्थिरां कृत्वा बुद्धिमस्य प्रियाण्युक्त्वा च भारत}
{गम्यतामित्यनुज्ञाय जयमाप्नुहि चेत्यथ}


\twolineshloka
{तैर्विसृष्टं महाबाहुं कृत्या सैवानयत्पुरः}
{तमेव देशं यत्रासौ तदा प्रायमुपाविशत्}


\twolineshloka
{प्रतिनिक्षिप्यतं वीरं कृत्या समभिपूज्य च}
{अनुज्ञाता च राज्ञा सा तत्रैवान्तरधीयत}


\twolineshloka
{गतायामथ तस्यां तु राजा दुर्योधनस्तदा}
{स्वप्नभूतमिदं सर्वमचिन्तयत भारत}


\twolineshloka
{`संमृश्य तानि वाक्यानि दानवोक्तानि दुर्मतेः'}
{विजेष्यामि रणे पाण्डूनिति चास्याभवनमतिः}


\twolineshloka
{कर्णं संशप्तकांश्चैव पार्थस्यामित्रघातिनः}
{अमन्यत वधे युक्तान्समर्थांश्च सुयोधनः}


\twolineshloka
{एवमाशा दृढा तस् धार्तराष्ट्रस् दुर्मतेः}
{विनिर्जये पाण्डवानामभवद्भरतर्षभ}


\twolineshloka
{कर्णोऽप्याविष्टचित्तात्मा नरकस्यान्तरात्मना}
{अर्जुनस्य वधे क्रूरां करोति स्म तदा मतिम्}


\twolineshloka
{संशप्तकाश्च ते वीरा राक्षसाविष्टचेतसः}
{रजस्तमोभ्यामाक्रान्ताः फाल्गुनस्य वधैषिणः}


\twolineshloka
{भीष्मद्रोणकृपाद्याश्च दानवाक्रान्तचेतसः}
{न तथा पाण्डुपुत्राणां स्नेहवन्तोऽभवंस्तदा}


\twolineshloka
{न चाचचक्षे कस्मैचिदेतद्राजा सुयोधनः}
{`कृत्ययाऽऽनाय्यकथितं यदस्य निशि दानवैः'}


\twolineshloka
{दुर्योधनं निशन्ते च कर्णो वैकर्तनोऽब्रवीत्}
{स्मयन्निवाञ्जलिं कृत्वा पार्थिवं हेतुमद्वचः}


\twolineshloka
{न मृतो जयते शत्रूञ्जीवन्भद्राणि पश्यति}
{मृतस्य भद्राणि कुतः कौरवेय कुतो जयः}


\twolineshloka
{न कालाऽद्य विषादस्य भयस्य मरणस्य वा}
{परिष्वज्याब्रवीच्चैवनं भुजाभ्यां स महाभुजः}


\twolineshloka
{उत्तिष्ठ राजन्किं शेषे कस्माच्छोचसि शत्रुहन्}
{शत्रून्प्रताप्य वीर्येण स कथं मर्तुमर्हसि}


\twolineshloka
{अथवा ते भयं जातं दृष्ट्वाऽर्जुनपराक्रमम्}
{सत्यंते प्रतिजानामि वधिष्यामि रणेऽर्जुनम्}


\twolineshloka
{गते त्रयोदशे वर्षे सत्येनायुधमालभे}
{आनयिष्याम्यहं पार्तान्वशं तव जनाधिप}


\twolineshloka
{एवमुक्तस्तु कर्णेन दैत्यानां वचनात्तथा}
{प्रणिपातेन चाप्येषामुदतिष्ठत्सुयोधनः}


\threelineshloka
{दैत्यानां तद्वचः श्रुत्वा हृदि कृत्वा स्थिरां मतिम्}
{ततो मनुजशार्दूलो योजयामास वाहिनीम्}
{रथनागाश्वफलिलां पदातिजनसंकुलाम्}


% Check verse!
गङ्गौघप्रतिमा चास्य प्रयाणे शुशुभे चमूः
\twolineshloka
{श्वेतच्छत्रैः पताकाभिश्चामरैश्च सुपाण्डुरैः}
{रथैर्नागैः पदातैश्च शुशुभेऽतीव संकुला}


\twolineshloka
{व्यपेताभ्रघने काले द्यौरिवासीत्तु शारदी}
{`हंसपङ्क्तिसमाकीर्णा भ्रमत्सारसशोभिता'}


\twolineshloka
{जयाशीर्भिर्द्विजेन्द्रैः स स्तूयमानोऽधिराजवत्}
{गृह्णन्नञ्जलिमालाश्च धार्तराष्ट्रो जनाधिपः}


\twolineshloka
{सुयोधनो ययावग्रे श्रिया परमया ज्वलम्}
{कर्णेन सार्धं राजेन्द्र सौबलेन च देविना}


\twolineshloka
{दुःशासनादयश्चास्य भ्रातरः सर्व एव ते}
{भूरिश्रवाः सोमदत्तो माराजश्च बाह्लिकः}


\twolineshloka
{रथैर्नानाविधाकारैर्हयैर्गजवरैस्तथा}
{प्रयान्तं नृपसिंहं तमनुजग्मुः कुरूद्वहाः}


% Check verse!
`प्रहृष्टमनसः सर्वे दुर्योधनपुरोगमाः' ॥कालेनाल्पेन राजेन्द्र स्वपुरं विविशुस्तदा
\chapter{अध्यायः २५४}
\twolineshloka
{जनमेजय उवाच}
{}


\twolineshloka
{एवं गतेषु पार्थेषु वने तस्मिन्महात्मसु}
{धार्तराष्ट्रा महेष्वासाः किमकुर्वत सत्तमाः}


\threelineshloka
{कर्णो वैकर्तनश्चैव शकुनिश्च महाबलः}
{भीष्मद्रोणकृपाश्चैव तन्मे शंसितुमर्हसि ॥वैशंपायन उवाच}
{}


\threelineshloka
{एवं गतेषु पार्थेषु विसृष्टे च सुयोधने}
{आगते हास्तिनपुरं मोक्षिते पाण्डुनन्दनैः}
{भीष्मोऽब्रवीन्महाराज धार्तराष्ट्रमिदं वच}


\twolineshloka
{उक्तं तात मया पूर्वं गच्छतस्ते तपोवनम्}
{वचनं ते न रुचितं मम तन्न कृतं च ते}


\twolineshloka
{ततः प्राप्तं त्वया वीर ग्रहणं शत्रुभिर्बलात्}
{मोक्षितश्चासि धर्मज्ञैः पाण्डवैर्न च लज्जसे}


\twolineshloka
{प्रत्यक्षं तव गान्धारे ससैन्यस् विशांपते}
{सूतपुत्रोऽपयाद्भीतो गन्धर्वाणां तदा रणात्}


\twolineshloka
{क्रोशतस्तव राजेन्द्र ससैन्यस्य नृपात्मज}
{`व्यपायात्पृष्ठतस्तस्मात्प्रेक्षमाणः पुनःपुनः ॥'}


\twolineshloka
{दृष्टस्ते विक्रमश्चैव पाण्डवानां महात्मनाम्}
{कर्णस्य च महाबाहो सूतपुत्रस्य दुर्मते}


\twolineshloka
{न चापि पादभाक्कर्णः पाण्डवानां महात्मनाम्}
{धनुर्वेदे च शौर्ये च धर्मे वा धर्मवत्सल}


\twolineshloka
{तस्मादहं क्षमं मन्ये पाण्डवैस्तैर्महात्मभिः}
{रसन्धिं सन्धिविदांश्रेष्ठ कुलस्यास्य विवृद्धये}


\twolineshloka
{एवमुक्तश्च भीष्मेण धार्तराष्ट्रो जनेश्वरः}
{प्रहस्य सहसा राजन्विप्रतस्थे ससौबलः}


\twolineshloka
{तं तु प्रस्थितमाज्ञाय कर्णदुःशासनादयः}
{अनुजग्मुर्महेष्वासा धार्तराष्ट्रं महाबलम्}


\twolineshloka
{तांस्तु संप्रस्थितान्दृष्ट्वा भीष्मः कुरुपितामहः}
{लज्जया व्रीडितो राजञ्जगाम स्वं निवेशनम्}


\twolineshloka
{गते भीष्मे महाराज धार्तराष्ट्रो जनेश्वरः}
{पुनरागम्य तं देशममन्त्रयत मन्त्रिभिः}


\threelineshloka
{किमस्माकं भवेच्छ्रेयः किं कार्यमवशिष्यते}
{कथं च सुकृतं तत्स्यान्मन्त्रयामास भारत ॥[*कर्ण उवाच}
{}


\twolineshloka
{दुर्योधन निबोधेदं यत्त्वां वक्ष्यामि कौरव}
{भीष्मोस्मान्निन्दति सदा पाण्डवांश्च प्रशंसति}


\twolineshloka
{त्वद्वेषाच्च महाबाहो मामपि द्वेष्टुमर्हति}
{विगर्हते च मां नित्यं त्वत्समीपे नरेश्वर}


\twolineshloka
{सोऽहं भीष्मवचस्तद्वै न मृष्यामीह भारत}
{त्वत्समं यदुक्तं च भीष्मेणामित्रकर्शन}


\twolineshloka
{पाण्डवानां यशो राजंस्तव निन्दां च भारत}
{अनुजानीहि मां राजन्सभृत्यबलवाहनम्}


\twolineshloka
{जेष्यामि पृथिवीं राजन्सशैलवनकाननाम्}
{जिता च पाण्डवैर्भूमिश्चतुर्भिर्बलशालिभिः}


\twolineshloka
{तामहं ते विजेष्यामि एक एव न संशयः}
{संपश्यतु सुदुर्बुद्धिर्भीष्मः कुरुकुलाधमः}


\twolineshloka
{अनिन्द्यं निन्दते यो हि अप्रशंस्यं प्रशंसति}
{स पश्यतु बलं मेऽद्य आत्मानं तु विगर्हतु}


\twolineshloka
{अनुजानीहि मां राजन्ध्रुवो हि विजयस्तव}
{प्रतिजानामि ते सत्यं राजन्नायुधमालभे}


\twolineshloka
{तच्छ्रुत्वा तु वचो राजन्कर्णस्य भरतर्षभ}
{प्रीत्या परमया युक्तः कर्णमाह नराधिपः}


\twolineshloka
{धन्योस्म्यनुगृहीतोस्मि यस्य मे त्वं महाबलः}
{हितेषु वर्तसे नित्यं सफलंजन्म चाद्य मे}


\twolineshloka
{यदा च मन्यसे वीर सर्वशत्रुनिबर्हणम्}
{तदा निर्गच्छ भद्रं ते ह्यनुशाधि च मामिति}


\twolineshloka
{एवमुक्तस्तदा कर्णो धार्तराष्ट्रेण धीमता}
{सर्वमज्ञापयामास प्रायात्रिकमरिंदम}


\twolineshloka
{प्रययौ च महेष्वासो नक्षत्रे शुभदैवते}
{शुभेतिथौ मुहूर्ते च पुज्यमानो द्विजातिभिः}


\twolineshloka
{मङ्गलैश्च शुभैः स्नातो वाग्भिश्चापि प्रपूजितः}
{नादयन्रथघोषेण त्रैलोक्यं सचराचरम्}


\chapter{अध्यायः २५५}
\twolineshloka
{वैशंपायन उवाच}
{}


\twolineshloka
{ततः कर्णो महेष्वासो बलेन महता वृतः}
{द्रुपदस्य पुरं रम्यं रुरोध भरतर्षभ}


\threelineshloka
{युद्धेन महता चैनं चक्रे वीरं वशानुगम्}
{सुवर्णं रजतं चापि रत्नानि विविधानि च}
{करं च दापयांमास द्रुपदं नृपसत्तम}


\twolineshloka
{तं विनिर्जित्य राजेन्द्र राजानस्तस्य येऽनुगाः}
{तान्सर्वान्वशगांश्चक्रे करं चैनानदापयत्}


\twolineshloka
{अथोत्तरां दिशं गत्वा वशे चक्रे नराधिपान्}
{भगदत्तं च निर्जित्य राधेयो गिरिमारुहत्}


\twolineshloka
{हिमवन्तं महाशैलं युध्यमानश्च शत्रुभिः}
{प्रययौ च दिशः सर्वान्नृपतीन्वशमानयत्}


\twolineshloka
{स हैमवतिकाञ्जित्वा करं सर्वानदापयत्}
{नेपालविषये ये च राजानस्तानवाजयत्}


\twolineshloka
{अवतीर्य ततः शैलात्पूर्वां दिशमभिद्रुतः}
{अङ्गान्वङ्गान्कलिङ्गांश्च शुण्डिकान्मिथिलानथ}


\twolineshloka
{मागधान्कर्कखण्डांश्च निवेश्य विषयेऽऽत्मनः}
{आवशीरांश्च योध्यांश्च अहिक्षत्रांश्च सोऽजयत्}


% Check verse!
पूर्वां दिशं विनिर्जित्य वत्सभूमिं तथाऽगमत्
\twolineshloka
{वत्सभूमिं विनिर्जित्य केवलां मृत्तिकावतीम्}
{मोहनं पत्तनं चैव त्रिपुरीं कोसलां तथा}


\threelineshloka
{एतान्सर्वान्विनिर्जित्य करमादाय सर्वशः}
{दक्षिणां दिशमास्थाय कर्णो जित्वा महारथान्}
{रुक्मिणं दाक्षिणात्येषु योधयामास सूतजः}


\twolineshloka
{स युद्धं तुमुलं कृत्वा रुक्मी प्रोवाच सूतजम्}
{प्रीतोस्मि तव राजेन्द्र विक्रमेण बलेन च}


\twolineshloka
{न ते विघ्नं करिष्यामि प्रतिज्ञां समपालयम्}
{प्रीत्या चाहं प्रयच्छामि हिरण्यं यावदिच्छसि}


% Check verse!
समेत्य रुक्मिणा कर्णः पाण्ड्यं शैलं च सोगमत्
\threelineshloka
{स केवलं रणए चैव नीलं चापि महीपतिम्}
{वेणुदारिसुतं चैव ये चान्ये नृपसत्तमाः}
{दक्षिणस्यां दिशि नृपान्करान्सर्वानदापयत्}


\twolineshloka
{शैशुपालिं ततो गत्वा विजिग्ये सूतनन्दनः}
{पार्श्वस्थांश्चापि नृपतीन्वशे चक्रे महाबलः}


\twolineshloka
{आवन्त्यांश्च वशे कृत्वा साम्ना च भरतर्षभ}
{वृष्णिभिः सह संम्य पश्चिमामपि निर्जयत्}


\twolineshloka
{वारुणीं दिशमागम्य यावनान्वर्बरांस्तथा}
{नृपान्पश्चिमभूमिस्थान्दापयामास वै करान्}


\twolineshloka
{विजित्य पृथिवीं सर्वां सपूर्वापरदक्षिणाम्}
{सम्लेच्छाटविकान्वीरः सपर्वतनिवासिनः}


\twolineshloka
{भद्रान्रोहितकांश्चैव आग्रेयान्मालवानपि}
{गणान्सर्वान्विनिर्जित्य नीतिकृत्प्रहसन्निव}


\twolineshloka
{शशकान्यवनांश्चैव विजिग्ये सूतनन्दनः}
{नग्नजित्प्रमुखांश्चैव गणाञ्जित्वा महारथान्}


\twolineshloka
{एवं स पृथिवीं सर्वां वशे कृत्वा महारथः}
{विजित्य पुरुषव्याघ्रो नागसाह्वयमागमत्}


\twolineshloka
{तमागतं महेष्वासं धार्तराष्ट्रो जनाधिपः}
{प्रत्युद्गत्य महाराज सभ्रातृपितृबान्धवः}


\twolineshloka
{अर्चयामास विधिना कर्णमाहवशोभिनम्}
{आश्रावयच्च तत्कर्म प्रीयमाणो जनेश्वरः}


\twolineshloka
{यन्न भीष्मान्न च द्रोणान्न कृपान्न च वाह्लिकात्}
{प्राप्तवानस्मि भद्रं ते त्वत्तःप्राप्तं मया हि तत्}


\twolineshloka
{बहुना च किमुक्तेन शृणु कर्ण वचो मम}
{सनाथोस्मि महाबाहो त्वया नाथेन सत्तम}


\twolineshloka
{न हि ते पाण्डवाः सर्वे कलामर्हन्ति षोडशीम्}
{अन्येवा पुरुषव्याघ्र राजानोऽभ्युदितोदिताः}


\twolineshloka
{स भवान्धृतराष्ट्रं तं गान्धारीं च यशस्विनीम्}
{पश्य कर्ण महेष्वास अदितिं वज्रभृद्यथा}


\twolineshloka
{ततो हलहलाशब्दः प्रादुरासीद्विशांपते}
{हाहाकाराश्च बहवो नगरे नागसाह्वये}


\twolineshloka
{रकेचिदेनं प्रशंसन्ति निन्दन्ति स्म तथा परे}
{तूष्णीमासंस्तथा चान्ये नृपास्तत्र जनाधिप}


\twolineshloka
{एवं विजित्य राजेन्द्र कर्णः शस्त्रभृतांवरः}
{सपर्वतवनाकाशां ससमुद्रां सनुष्कुटाम्}


\twolineshloka
{देशैरुच्चावचैः पूर्णां पत्तनैर्नगरैरपि}
{द्वीपैश्चानूपसंपूर्णैः पृथिवीं पृथिवीपते}


\twolineshloka
{कालेन नातिदीर्घेण वशे कृत्वा तु पार्थिवान्}
{अक्षयं धनमादाय सूतजो नृपमभ्ययात्}


\twolineshloka
{प्रविश्य च गृहं राजन्नभ्यन्तरमरिंदम}
{गान्धारीसहितं वीरो धृतराष्ट्रं ददर्श सः}


\twolineshloka
{पुत्रवच्च नरव्याघ्र पादौ जग्राह धर्मवित्}
{धृतराष्ट्रेण चाश्लिष्य प्रेम्णा चापि विसर्जितः}


\twolineshloka
{तदाप्रभृति राजा च शकुनिश्चापि सौबलः}
{जानाते निर्जितान्पार्थान्कर्णेन युधि भारत}


\chapter{अध्यायः २५६}
\twolineshloka
{वैशंपायन उवाच}
{}


\twolineshloka
{जित्वा तु पृथिवीं राजन्सूतपुत्रो जनाधिप}
{अब्रवीत्परवीरघ्नो दुर्योधनमिदं वचः}


\twolineshloka
{दुर्योधन निबोधेदं यत्त्वां वक्ष्यामि कौरव}
{श्रुत्वा वाचं तथा सर्वं कर्तुमर्हस्यरिंदम}


\threelineshloka
{तवाद्य पृथिवी वीर निःसपत्ना नृपोत्तम}
{तां पालय यथा शक्रो हतशत्रुर्महामनाः ॥वैशंपायन उवाच}
{}


\twolineshloka
{एवमुक्तस्तु कर्णेन कर्णं राजाऽब्रवीत्पुनः}
{न किंचिद्दुर्लभं तस्य यस् त्वं पुरुषर्षभ}


\twolineshloka
{सहायश्चानुरक्तश्च मदर्थं च समुद्यतः}
{अभिप्रायस्तु मे कश्चित्तं वै शृणु यथातथम्}


\twolineshloka
{राजसूयं पाण्डवस्य दृष्ट्वा क्रतुवरं तदा}
{कमम स्पृहा समुत्पन्ना तां संपादय सूतज}


\twolineshloka
{एवमुक्तस्ततः कर्णो राजानमिदमब्रवीत्}
{तवाद्य पृथिवीपाला वश्याः सर्वे नृपोत्तम}


\twolineshloka
{आहूयन्तां द्विजवराः संभाराश्च यथाविधि}
{संभ्रियन्तां कुरुश्रेष्ठ यज्ञोपकरणानि च}


\twolineshloka
{ऋत्विजश्च समाहूता यथोक्तं वेदपारगाः}
{क्रियां कुर्वन्तु ते राजन्यथाशास्त्रमरिंदम}


\twolineshloka
{बह्वन्नपानसंयुक्तः सुसमृद्धगुणान्वितः}
{प्रवर्ततां महायज्ञस्तवापि भरतर्षभ}


\twolineshloka
{एवमुक्तस्तु कर्णेन धार्तराष्ट्रो विशांपते}
{पुरोहितं समानाय्य वचनं चेदमब्रवीत्}


\twolineshloka
{राजसूयं क्रतुश्रेष्ठं समाप्तवरदक्षिणम्}
{आहरस्व यथाशास्त्रं यथान्यायं यथाक्रमम्}


% Check verse!
स एवमुक्तो नृपतिमुवाच द्विजसत्तमः
\threelineshloka
{`ब्राह्मणैः सहितो राजन्ये तत्रासन्समागताः'}
{न स शक्यः क्रतुश्रेष्ठो जीवमाने युधिष्ठिरे}
{आहर्तुं कौरवश्रेष्ठ कुले तव नृपोत्तम}


\twolineshloka
{दीर्घायुर्जीवति च ते धृतराष्ट्रः पिता नृप}
{अतश्चापि विरुद्धस्ते क्रतुरेणष नृपोत्तमः}


\twolineshloka
{अस्ति त्वन्यन्महत्सत्रं राजसूयसमं प्रभो}
{तेन त्वं यज राजेन्द्र शृणु चेदं वचो मम}


\twolineshloka
{य इमे पृथिवीपालाः करदास्तव पार्थिव}
{ते करान्संप्रयच्छन्तु सुवर्णं च कृताकृतम्}


\twolineshloka
{तेन ते क्रियतामद्यलाङ्गलं नृपसत्तम}
{यज्ञवाटस् ते भूमिः कृष्यतां तेन भारत}


\twolineshloka
{तत्र यज्ञो नृपश्रेष्टः प्रभूतान्नः सुसंस्कृतः}
{प्रवर्ततां यथान्यायं सर्वतो ह्यनिवारितः}


\twolineshloka
{एष ते वैष्णवो नाम यज्ञः सत्पुरुषोचितः}
{एतेन नेष्टवान्कश्चिदृतेविष्णुं पुरातनम्}


\twolineshloka
{राजसूयं क्रतुश्रेष्ठं स्पर्धत्येष महाक्रतुः}
{अस्माकं रोचते चैव श्रेयश्च तव भारत}


\twolineshloka
{निर्विघ्नश्च भवत्येष सफला स्यात्स्पृहा तव}
{`तस्मादेष महाबाहो तव यज्ञः प्रवर्तताम्'}


\twolineshloka
{एवमुकत्स्तु तैर्विप्रैर्धार्तराष्ट्रो महीपतिः}
{कर्णं च सौबलं चैव भ्रातॄश्चैवेदमब्रवीत्}


\twolineshloka
{रोचते मे वचः कृत्स्नं ब्राह्मणानां न संशयः}
{रोचते यदि युष्माकं तस्मात्प्रब्रूत माचिरम्}


\twolineshloka
{एवमुक्तास्तु ते सर्वे तथेत्यूचुर्नराधिपम्}
{संदिदेश ततोराजाव्यापारस्थान्यथाक्रमम्}


\twolineshloka
{हलस्य करणे चापि व्यादिष्टाः सर्वशिल्पिनः}
{यथोक्तं च नृपश्रेष्ठ कृतं सर्वं यथाक्रमम्}


\chapter{अध्यायः २५७}
\twolineshloka
{वैशंपायन उवाच}
{}


\twolineshloka
{ततस्तु शिल्पिनः सर्वं कृतमूर्चुर्नराधिपम्}
{विदुरश्च महाप्राज्ञो धृतराष्ट्रे न्यवेदयत्}


\twolineshloka
{सज्जं क्रतुवरं राजन्कालप्राप्तं च भारत}
{सौवर्णं च कृतं सर्वं लाङ्गलं च महाधनम्}


\twolineshloka
{एवच्छ्रुत्वा नृपश्रेष्ठो धृतराष्ट्रो विशंपते}
{आज्ञापयामास नृपः क्रतुराजप्रवर्तनम्}


\twolineshloka
{ततः प्रववृते यज्ञः प्रभूतार्थः सुसंस्कृतः}
{दीक्षितश्चापि गान्धारिर्यथाशास्त्रं यथाक्रमम्}


\twolineshloka
{प्रहृष्टो धृतराष्ट्रश्च विदुरश्च महायशाः}
{भीष्मो द्रोणः कृपः कर्णो गान्धारी च यशस्विनी}


\twolineshloka
{निमन्त्रणार्थं दूर्तांश्च प्रेषयामास शीघ्रगान्}
{पार्थिवानां च राजेन्द्र ब्राह्मणानां तथैव च}


\twolineshloka
{ते प्रयाता यथोद्दिष्टा दूतास्त्वरितवाहनाः}
{तत्र कचित्प्रयान्तं तु दूतं दुःशासनोऽब्रवीत्}


\twolineshloka
{गच्छ द्वैतवनं शीघ्रं पाण्डवान्पापपूरुषान्}
{निमन्त्रय यथान्यायं विप्रांस्त स्मिन्वने तदा}


\twolineshloka
{स गत्वा पाण्डवान्सर्वानुवाचाभिप्रणम्य च}
{दुर्योधनो महाराज यजते नृपसत्तमः}


\twolineshloka
{स्ववीर्यार्जितमर्थौघमवाप्य कुरुसत्तमः}
{तत्र गच्छन्ति राजानो ब्राह्मणाश्च ततस्ततः}


\threelineshloka
{अहं तु प्रेषितो राजन्कौरवेण महात्मना}
{आमन्त्रयति वो राजा धार्तराष्ट्रो जनेश्वरः}
{मनोभिलषितं राज्ञस्तं क्रतुं द्रष्टुमर्हथ}


\threelineshloka
{ततो युधिष्ठिरो राजा तच्छ्रुत्वा दूतभाषितम्}
{अब्रवीन्नृपशार्दूलो दिष्ट्या राजा सुयोधनः}
{यजते क्रतुमुख्येन पूर्वेषां कीर्तिवर्धनः}


\twolineshloka
{वयमप्युपयास्यामो न त्विदानीं कथंचन}
{समयः परिपाल्यो नो यावद्वर्षं त्रयोदशम्}


\twolineshloka
{श्रुत्वैतद्धर्मराजस्य भीमो वचनमब्रवीत्}
{तदा तु नृपतिर्गन्ता धर्मराजो युधिष्ठिरः}


\twolineshloka
{अस्त्रशस्त्रप्रदीप्तेऽग्नौ यदा तं पातयिष्यति}
{वर्षात्रयोदशादूर्ध्वं रणसत्रे नराधिपः}


\twolineshloka
{यदा क्रोधहविर्मोक्ता धार्तराष्ट्रेषु पाण्डवः}
{आगन्तारस्तदा स्मेति वाच्यस्ते स सुयोधनः}


\twolineshloka
{शेषास्तु पाण्डवा राजन्नैवोचुः किंचिदप्रियम्}
{दूतश्चापि यथावृत्तं धार्तराष्ट्रे न्यवेदयत्}


\twolineshloka
{अथाजग्मुर्नरश्रेष्ठा नानाजनपदेश्वराः}
{ब्राह्मणाश्च महाभाग धार्तराष्ट्रपुरं प्रति}


\twolineshloka
{ते त्वर्चिता यथाशास्त्रं यथाविधि यथाक्रमम्}
{मुदा परमया युक्ताः प्रीताश्चापि नरेश्वराः}


\twolineshloka
{धृतराष्ट्रोऽपि राजेन्द्र संवृतः सर्वकौरवैः}
{हर्षेण महता युक्तो विदुरं प्रत्यभाषत}


\twolineshloka
{यथा सुखी जनः सर्वः क्षत्तः स्यादन्नसंयुतः}
{तुष्येत्तु यज्ञसदने तथा नीतिर्विधीयताम्}


\twolineshloka
{विदुरस्तु तदाज्ञाय सर्ववर्णानरिंदम}
{यथा प्रमाणतो विद्वान्पूजयामास धर्मवित्}


\twolineshloka
{भक्ष्यपेयान्नपानेन माल्यैश्चापि सुगन्धिभिः}
{वासोभिर्विविधैश्चैव योजयामास हृष्टवत्}


\threelineshloka
{कृत्वा ह्यवभृथं वीरो यथाशास्त्रं यथाक्रमम्}
{सान्त्वयित्वा च राजेन्द्रो दत्त्वा च विविधं वसु}
{विसर्जयामास नपान्ब्राह्मणांश्च सहस्रशः}


\twolineshloka
{विसृज्यच नृपान्सर्वान्भ्रातृभिः परिवारितः}
{विवेश हास्तिनपुरं सहितः कर्णसौबलैः}


\chapter{अध्यायः २५८}
\twolineshloka
{वैशंपायन उवाच}
{}


\twolineshloka
{प्रविशन्तं महाराज सूतास्तुष्टुवुरच्युतम्}
{जनाश्चापि महेष्वासं तुष्टुवू राजसत्तमम्}


\twolineshloka
{लाजैश्चन्दनचूर्णैश्च विकीर्य च जनास्ततः}
{ऊटुर्दिष्ट्या नृपाविघ्नः समाप्तोयं क्रतुस्तव}


\twolineshloka
{अपरे त्वब्रुवंस्तत्रवादिकास्तं महीपतिम्}
{यौधिष्ठिरस्य यज्ञस्य न समो ह्येष ते क्रतुः}


\twolineshloka
{नैव तस्य क्रतोरेष कलामर्हति षोडशीम्}
{एवं तत्राब्रुवन्केचिद्वातिकास्तं जनेश्वरम्}


\twolineshloka
{सुहृदस्त्वब्रुवंस्तत्र अति सर्वानयं क्रतुः}
{`प्रवर्तितो ह्ययं राज्ञा धार्तराष्ट्रेण धीमता'}


\twolineshloka
{ययातिर्नहुषश्चापि मान्धाता भरतस्तथा}
{क्रतुमेनं समाहृत्य पूताः सर्वे दिवं गताः}


\twolineshloka
{एता वाचः शुभाः शृण्वन्सुहृदां भरतर्षभ}
{प्रविवेश पुरं हृष्टः स्ववेश्म च नराधिपः}


\twolineshloka
{अभिवाद्य ततः पादान्मतापित्रोर्विशांपते}
{भीष्मद्रोणकृपादीनां विदुरस्य च धीमतः}


\twolineshloka
{अभिवादितः कनीयोभिर्भ्रातृभिर्भ्रातृनन्दनः}
{निषसादासने मुख्ये भ्रातृभिः परिवारितः}


\twolineshloka
{तमुत्थाय महाराजं सूतपुत्रोऽब्रवीद्वचः}
{दिष्ट्या ते भरतश्रेष्ठ समाप्तोऽयं महाक्रतुः}


\twolineshloka
{हतेषु युधि पार्थेषु राजसूये तथा त्वया}
{आहृतेऽहं नरश्रेष्ठ त्वां सभाजयिता पुनः}


\twolineshloka
{तमब्रवीन्महाराजो धार्तराष्ट्रो महायशाः}
{}


\threelineshloka
{सत्यमेतत्त्वयोक्तं हि पाण्डवेषु दुरात्मसु}
{निहतेषु नरश्रेष्ठ प्राप्ते चापि महाक्रतौ}
{राजसूये पुनर्वीर त्वमेवं वर्धयिष्यसि}


\twolineshloka
{एवमुक्त्वा महाराज कर्णमाश्लिष्य भारत}
{राजसूयं क्रतुश्रेष्ठं चिन्तयामास कौरवः}


\twolineshloka
{सोऽब्रवीत्कौरवांश्चापि पार्श्वस्थान्नृपसत्तमः}
{`राधेयसौबलादीन्वै धार्तराष्ट्रो महीपतिः'}


\twolineshloka
{कदा तु तं क्रतुवरं राजसूयं महाधनम्}
{निहत्य पाण्डवान्सर्वानाहरिष्यामि कौरवाः}


\twolineshloka
{तमब्रवीत्तदा कर्णः शृणु मे राजकुञ्जर}
{पादौ न धावये तावद्यावन्न निहतोऽर्जुनः}


\twolineshloka
{कीलालजं न खादेयं करिष्ये चासुरव्रतम्}
{नास्तीति नैव वक्ष्यामि याचितो येन केचनित्}


\twolineshloka
{अथोत्क्रुष्टं महेष्वासैर्धार्तराष्ट्रैर्महारथैः}
{प्रतिज्ञाते फल्गुनस्य वधे कर्णेन संयुगे}


\twolineshloka
{विजितांश्चाप्यमन्यन्त पाण्डवान्धृतराष्ट्रजाः}
{`तदा प्रतिज्ञामारुह्य सूतपुत्रेण भाषिते'}


\twolineshloka
{दूर्योधनोऽपि राजेन्द्र विसृज्यनरपुङ्गवान्}
{प्रविवेश गृहंश्रीमान्यथा चैत्ररथं प्रभुः}


\twolineshloka
{तेऽपिसर्वे महेष्वासा जग्मुर्वेश्मानि भारत}
{`स्वानिस्वनि महाराज भीष्मद्रोणादयो नृपा'}


\twolineshloka
{पाण्डवाश्च महेष्वासा दूतवाक्यप्रचोदिताः}
{चिन्तयन्तस्तमेवार्थं नालभन्त सुखं क्वचित्}


\twolineshloka
{भूयश्च चारै राजेन्द्र प्रवृत्तिरुपपादिता}
{प्रतिज्ञा सूतपुतर्स्य विजयस्य वधं प्रति}


\twolineshloka
{एतच्छ्रुत्वा धर्मसुतः समुद्विग्नो नराधिप}
{`अधोमुखश्चिरं तस्थौ किं कार्यमिति चिन्तयन्'}


\twolineshloka
{अभेद्यकवचं मत्वा कर्णमद्भुतविक्रमम्}
{अनुस्मरंश् संक्लेशान्न शान्तिमुपजग्मिवान्}


\twolineshloka
{तस्य चिन्तापरीतस्य बुद्धिर्जज्ञे महात्मनः}
{बहुव्यालमृगाकीर्णं त्यक्तुं द्वैतवनं वनम्}


\twolineshloka
{धार्तराष्ट्रोऽपि नृपतिः प्रशशास वसुंधराम्}
{भ्रातृभिः सहितो वीरैर्भीष्मद्रोणकृपैस्तथा}


\twolineshloka
{संगम्य सूतपुत्रेण कर्णेनाहवशोभिना}
{`सततं प्रीयमाणो वै देविना सौबलेन च'}


\twolineshloka
{दुर्योधनः प्रिये नित्यं वर्तमानो महीभृताम्}
{पूजयामास विप्रेन्द्रान्क्रतुभिर्भूरिदक्षिणैः}


\twolineshloka
{भ्रातॄणां च प्रियं राजन्स चकार परंतपः}
{निश्चित्य मनसा वीरो दत्तभुक्तफलं धनम्}


\chapter{अध्यायः २५९}
\twolineshloka
{जनमेजय उवाच}
{}


\threelineshloka
{दुर्योधनं मोक्षयित्वा पाण्डुपुत्रा महाबलाः}
{किमकार्षुर्वने तस्मिंस्तन्ममाख्यातुमर्हसि ॥वैशंपायन उवाच}
{}


\twolineshloka
{ततः शयानं कौन्तेयं रात्रौ द्वैतवन मृगाः}
{स्वप्नान्ते दर्शयामासुर्बाष्पकण्ठा युधिष्ठिरम्}


\twolineshloka
{तानब्रवीत्स राजेन्द्रो वेपमानान्कृताञ्जलीन्}
{ब्रूत यद्वक्तुकामाः स्थ के भवन्तः किमिष्यते}


\twolineshloka
{एवमुक्ताः पाण्डवेन कौन्तेयेन यशस्विना}
{प्रत्यब्रुवन्मृगास्तत्रहतशेषा युधिष्ठिरम्}


\twolineshloka
{वयं मृगा द्वैतवने हतशिष्टास्तु भारत}
{नोत्सीदेम महाराज क्रियतां वासपर्ययः}


\twolineshloka
{भवन्तो भ्रातरः शूराः सर्व एवास्त्रकोविदाः}
{कुलान्यल्पावशिष्टानि कृतवन्तो वनौकसाम्}


\twolineshloka
{बीजभूता वयं केचिदवशिष्टा महामते}
{विवर्धेमहि राजेन्द्र प्रसादात्ते युधिष्ठिर}


\twolineshloka
{तान्वेपमानान्वित्रस्तान्बीजमात्रावशेषितान्}
{मृगान्दृष्ट्वा सुदुःखार्तो धर्मराजो युधिष्ठिरः}


\twolineshloka
{तांस्तथेत्यब्रवीद्राजा सर्वभूतहिते रतः}
{यथा भवन्तो ब्रुवते करिष्यामि च तत्तथा}


\twolineshloka
{इत्येवं प्रतिबुद्धः स रात्र्यन्ते राजसत्तमः}
{अब्रवीत्सहितान्भ्रातॄन्दयापन्नो मृगान्प्रति}


\twolineshloka
{उक्तो रात्रौ मृगैरस्मि स्वप्नान्ते हतशेषितैः}
{तनुभूताः स्म भद्रं ते दया नः क्रियतामिति}


\twolineshloka
{ते सत्यमाहुः कर्तव्या दयाऽस्माभिर्वनौकसाम्}
{साष्टमासं हि नो वर्षं यदेनानुपयुंक्ष्महे}


\threelineshloka
{पुनर्वहुमृगं रम्यं काम्यकं काननोत्तमम्}
{तत्रेमां वसतिं शिष्टां विहरन्तो रमेमहि ॥वैशंपायन उवाच}
{}


% Check verse!
ततस्ते पाण्डवाः शीघ्रं प्रययुर्धर्मकोविदाः
\twolineshloka
{ब्राह्मणैः सहिता राजन्ये च तत्रसहोषिताः}
{इन्द्रसेनादिभिश्चैव प्रेष्यैरनुगतास्तदा}


\twolineshloka
{ते यात्वा सुसुखैर्मार्गैः स्वन्नैः शुचिजलान्वितैः}
{ददृशुः काम्यकं पुण्यमाश्रमं तापसान्वितम्}


\twolineshloka
{विविशुस्ते स्म कौरव्या वृता विप्रर्षभैस्तदा}
{तद्वनं भरतश्रेष्ठाः स्वर्गं सुकृतिनो यथा}


\chapter{अध्यायः २६०}
\twolineshloka
{वैशंपायन उवाच}
{}


\twolineshloka
{वने निवसतां तेषां पाण्डवानां महात्मनाम्}
{वर्षाण्येकादशातीयुः कृच्छ्रेण भरतर्षभ}


\twolineshloka
{फलमूलाशनास्ते हि सुखार्हा दुःखमुत्तमम्}
{प्राप्तकालमनुध्यान्तः सेहिरे वरपूरुषाः}


\twolineshloka
{युधिष्ठिरस्तु राजर्षिरात्मकर्मापराधजम्}
{चिन्तयन्स महाबाहुर्भ्रातॄणां दुःखमुत्तमम्}


\twolineshloka
{न सुष्वाप सुखं राजा हृदि शल्यैरिवार्पितैः}
{दौरात्म्यमनुपश्यंस्तत्काले द्यूतोद्भवस्य हि}


\twolineshloka
{संस्मरन्परुषा वाचः सूतपुत्रस्य पाण्डवः}
{निःश्वासपरमो दीनो दध्रे कोपविषं महत्}


\twolineshloka
{अर्जुनोयमजौ चोभौ द्रौपदी च यशस्विनी}
{स च भीमो महातेजाः सर्वेषामुत्तमो बले}


\twolineshloka
{`चिरस्य जातं धर्मज्ञं सासूयमिव ते तदा'}
{युधिष्ठिरमुदीक्षन्तः सेहुर्दुखमनुत्तमम्}


\twolineshloka
{अवशिष्टं त्वल्पकालं मन्वानाः पुरुषर्षभाः}
{वपुरन्यदिवाकार्पुरुत्साहामर्षचेष्टितैः}


\twolineshloka
{कस्यचित्त्वथ कालस्य व्यासः सत्यवतीसुतः}
{आजगाम महायोगी पाण्डवानवलोककः}


\twolineshloka
{तमागतमभिप्रेक्ष्यकुन्तीपुत्रो युधिष्ठिरः}
{प्रत्युद्गम्य महात्मानं प्रत्यगृह्णाद्यथाविधि}


\twolineshloka
{तमासीनमुपासीनः शुश्रूषुर्नियतेन्द्रियः}
{तोषयामास शौचेन व्यासं पाण्डवनन्दनः}


\twolineshloka
{तानवेक्ष्यकृशान्पौत्रान्वने वन्येन जीवतः}
{महर्षिरनुकम्पार्थमब्रवीद्बाष्पगद्गदम्}


\threelineshloka
{युधिष्ठिर महाबाहो शृणु धर्मभृतांवर}
{नातप्ततपसो लोके प्राप्नुवन्ति महत्सुखम्}
{सुखदुःखे हि पुरुषः पर्यायेणोपसेवते}


\twolineshloka
{नात्यन्तमसुखं कश्चित्प्राप्नोति पुरुषर्षभ}
{प्रज्ञावांस्त्वेव पुरुषः संयुक्तः परया धिया}


\twolineshloka
{उदयास्तमयज्ञो हि न हृष्यति न शोचति}
{सुखमापतितं विन्दन्दुःखमापतितं सहन्}


\twolineshloka
{कालप्राप्तमुपासीत सस्यानामिव कर्षकः}
{तपसो हि परं नास्ति तपसा विन्दते महत्}


\twolineshloka
{नासाध्यं तपसः किंचिदिति बुध्यस् भारत}
{सत्यमार्जवमक्रोधः संविभागो दमः शमः}


\twolineshloka
{अनसूयाऽविहिंसा च शौचमिन्द्रियसंयमः}
{साधनानि महाराज नराणां पुण्यकर्मणाम्}


\twolineshloka
{अधर्मरुचयो मूढास्तिर्यग्गतिपरायणाः}
{कृच्छ्रां योनिमनुप्राप्ता न सुखं विन्दते अनाः}


\threelineshloka
{इह यत्क्रियते कर्म तत्परत्रोपभुज्यते}
{`मूलसिक्तस्य वृक्षस्य फलं शाखासु दृश्यते'}
{तस्माच्छरीरं युञ्जीत तपसा नियमेन च}


\twolineshloka
{यथाशक्ति प्रयच्छेत संपूज्याभिप्रणम्य च}
{काले प्राप्ते च हृष्टात्मा राजन्विगतमत्सरः}


\twolineshloka
{सत्यवादी लभेतायुरनायासमथार्जवम्}
{अक्रोधनोऽनसूयश्च निर्वृतिं लभते पराम्}


\twolineshloka
{दान्तः शमपरः शश्वत्परिक्लेशं न विन्दति}
{न च तप्यति दान्तात्मा दृष्ट्वा परगतां श्रियम्}


\twolineshloka
{संविभक्ता च दाता च भोगवान्सुखवान्नरः}
{भवत्यहिंसकश्चैव परमारोग्यमश्नुते}


\threelineshloka
{मान्यं मानयिता जन्म कुले महति विन्दति}
{`विन्दते सुखमत्यर्थमिह लोके परत्र च'}
{व्यसनैर्न तुसंयोगं प्राप्नोति विजितेन्द्रियः}


\threelineshloka
{शुभानुशयबुद्धिर्हि संयुक्तः कालधर्मणा}
{प्रादुर्भवति तद्योगात्कल्याणमतिरेव सः ॥युधिष्ठिर उवाच}
{}


\threelineshloka
{भगवन्दानधर्माणआं तपसो वा महामुने}
{किंस्विद्बहुगुणं प्रेत्य किं वा दुष्करमुच्यते ॥व्यास उवाच}
{}


\twolineshloka
{दानान्न दुष्करं तात पृथिव्यामस्ति किंचन}
{अर्थे च महती तृष्णा स च दुःखेन लभ्यते}


\threelineshloka
{`राजन्प्रत्यक्षमेवैतद्दृश्यते लोकसाक्षिकम्'}
{परित्यज्य प्रियान्प्राणान्प्रविशन्ति रणाजिरम्}
{तथैव प्रतिपद्यन्ते समुद्रमटवीं तथा}


\twolineshloka
{कृषिगोरक्ष्यमित्येके प्रतिपद्यन्ति मानवाः}
{पुरुषाः प्रेष्यतामेके निर्गच्छन्ति धनार्थिनः}


\twolineshloka
{तस्माद्दुःखार्जितस्यैव परित्यागः सुदुष्करः}
{सुदुष्करतरं दानं तस्माद्दानं मतं मम}


\twolineshloka
{विसेषस्त्वत्र विज्ञेयो न्यायेनोपार्जितं धनम्}
{पात्रे काले च देशे च प्रयतः प्रतिपादयेत्}


\twolineshloka
{अन्यायात्समुपात्तेन दानधर्मौ धनन यः}
{कुरुते न स कर्तारं त्रायते महतो भयात्}


\twolineshloka
{पात्रे दानं स्वल्पमपि काले दत्तं युधिष्ठिर}
{मनसा हि विशुद्धेन प्रेत्यानन्तफलं स्मृतम्}


\twolineshloka
{`श्रद्धा धर्मानुगा देवी पावनी विश्वधारिणी}
{'सवित्री प्रसवित्री च संसारार्णवतारिणी}


\twolineshloka
{श्रद्धया धार्यते धर्मो महद्भिर्नार्थदर्शिभिः}
{सधना अपिराजानो निःश्रद्धा नरकं गताः}


\threelineshloka
{निष्किंचनाश्च मुनयः श्रद्धावन्तो दिवं गताः}
{देशे काले च पात्रे च मुद्गलः श्रद्धयाऽन्वितः}
{व्रीहिद्रोणं प्रदायाथ परं पदमवाप्तवान्'}


\twolineshloka
{अत्राप्युदाहरन्तीमतिहासं पुरातनम्}
{व्रीहिद्रोणपरित्यागाद्यत्फलं प्राप मुद्गलः}


\chapter{अध्यायः २६१}
\twolineshloka
{युदिष्ठिर उवाच}
{}


\twolineshloka
{व्रीहिद्रोणः परित्यक्तः कथं तेन महात्मना}
{कस्मै दत्तश्च भगवन्विधिना केन चात्थ मे}


\threelineshloka
{प्रत्यक्षधर्मा भगवान्यस् तुष्टो हि कर्मभिः}
{सफलं तस्य जन्माहं मन्ये सद्धर्मचारिणः ॥व्यास उवाच}
{}


\twolineshloka
{शिलोञ्छवृत्तिर्धर्मात्मा मुद्गलः संयतेन्द्रियः}
{आसीद्राजन्कुरुक्षेत्रे सत्यवागनसूयकः}


\twolineshloka
{अतिथिव्रती क्रियावांश्च कापोतीं वृत्तिमास्थितः}
{सत्रमिष्टीकृतंनाम समुपास्ते महातपाः}


\twolineshloka
{सपुत्रदारो हि मुनिः पक्षाहारो बभूव ह}
{कपोतवृत्त्या पक्षेण व्रीहिद्रोयणमुपार्जयत्}


\twolineshloka
{दर्शं च पौर्णमासं च कुर्वन्विगतमन्सरः}
{देवतातिथिशेषेण कुरुते देहयापनम्}


\twolineshloka
{तत्रेन्द्रः सहितो देवैः साक्षात्रिभुवनेश्वरः}
{प्रत्यृह्णान्महाराज भागं पर्वणिपर्वणि}


\twolineshloka
{स पर्वकाल्यं कृत्वा तु मुनिवृत्त्या समन्वितः}
{अतिथिभ्यो ददावन्नं प्रहृष्टेनान्तरात्मना}


\twolineshloka
{व्रीहिद्रोणस्य तत्प्रीत्या ददतोऽन्नं महात्मनः}
{ऋषेर्मात्सर्यहीनस् वर्धत्यतिथिदर्शनात्}


\twolineshloka
{तच्छतान्यपि भुञ्जन्ति ब्राह्मणानां मनीषिणाम्}
{मुनेस्त्यागविशुद्ध्या तु तदन्नं वृद्धिमृच्छति}


\twolineshloka
{तं तु रशुश्राव धर्मिष्ठं मुद्गलं संशितव्रतम्}
{दुर्वासा नृप दिग्वासास्तमथाभ्याजगाम ह}


\twolineshloka
{विभ्रच्चानियतं वेषमुन्मत्त इव पाण्डव}
{विकचः परुषा वाचो व्याहरन्विविधा मुनिः}


\twolineshloka
{अभिगम्याथ तं विप्रमुवाच मुनिसत्तमः}
{अन्नार्थिनमनुप्राप्तं विद्धि मां मुनिसत्तम}


\twolineshloka
{स्वागतं तेऽस्त्विति मुनिं मुद्गलः प्रत्यभाषत}
{पाद्यमाचमनीयं च प्रतिवेद्यान्नमुत्तमम्}


\twolineshloka
{प्रादात्स तपसोपात्तं क्षुधितायातिथिप्रियः}
{उन्मत्ताय परां श्रद्धामास्थाय स धृतव्रतः}


\twolineshloka
{ततस्तदन्नं रसवत्स एव क्षुधयाऽन्वितः}
{बुभुजे कृत्स्नमुन्मत्तः प्रादात्तस्मै च मुद्गलः}


\twolineshloka
{भुक्त्वा चान्नं ततः सर्वमुच्छिष्टेनात्मनस्ततः}
{अथाङ्गं लिलिपेऽन्नेन यथागतमगाच्च सः}


\twolineshloka
{`तेनैवात्मानमालिप्य हसन्गायन्प्रधावति}
{नृत्यते धावते चैव बुद्ध्या तत्क्रोशते तथा'}


\twolineshloka
{एवं द्वितीये संप्राप्ते पर्वकाले मनीषिणः}
{आगम्य बुभुजे सर्वमन्नपुञ्छोपजीविनः}


\twolineshloka
{निराहारस्तु स मुनिरुञ्छमार्जये पुनः}
{न चैनं विक्रियां नेतुमशकन्मुद्गलं क्षुधा}


\twolineshloka
{न क्रोधो न च मात्सर्यं नावमानो न संभ्रमः}
{सपुत्रदारमुञ्छन्तमाविवेश द्विजोत्तमम्}


\twolineshloka
{तथा तमुञ्छधर्माणं दुर्वासा मुनिसत्तमम्}
{उपतस्थे यथाकालं षट्कृत्वः कृतनिश्चयः}


\twolineshloka
{न चास् मनसः कश्चिद्विकारो दृश्यते मुनेः}
{शुद्धसत्वस्य शुद्धं स ददृशे निर्मलं मनः}


\twolineshloka
{तमुवाच ततः प्रीतः स मुनिर्मुद्गलं तदा}
{त्वत्समो नास्ति लोकेऽस्मिन्दाता मात्सर्यवर्जितः}


\twolineshloka
{क्षुद्धर्मसंज्ञां प्रणुदत्यादत्ते धैर्यमेव च}
{विषयानुसारिणी जिह्वा कर्षत्येव रसान्प्रति}


\twolineshloka
{आहारप्रभवाः प्राणा मनो दुर्निग्रहं चलम्}
{मनसश्चेन्द्रियाणां चाप्यैकाग्र्यं निश्चितं तपः}


\twolineshloka
{श्रमेणोपार्जितं त्यक्तं न च दुःखेन चेतसा}
{तत्सर्वं भवता साधो यथावदुपपादितम्}


\threelineshloka
{प्रीताः स्मोऽनुगृहीताश्च समेत्य भवता सह}
{इन्द्रियाभिजयो धैर्यं संविभागो दमः शमः}
{दया सत्यं च धर्मश्च त्वयि सर्वं प्रतिष्ठितम्}


\twolineshloka
{`लोकाः समस्ता धर्मेण धार्यन्ते सचराचराः}
{धर्मोपि धार्यते येन धृतियुक्त त्वयाऽऽत्मना}


\twolineshloka
{विशुद्धसत्वसंपन्नो न त्वदन्योस्ति कश्चन'}
{जितास्ते कर्मभिर्लोकाः प्राप्तोसि परमां गतिम्}


\twolineshloka
{अहो दानं विघुष्टं ते सुमहत्स्वर्गवासिभिः}
{सशरीरो भवान्गन्ता स्वर्गं सुचरितव्रत}


\twolineshloka
{इत्येवं वदतस्तस्य तदा दुर्वाससो मुनेः}
{देवदूतो विमानेन मुद्गलं प्रत्युपस्थितः}


\twolineshloka
{हंससारसयुक्तेन किङ्किणीजालमालिना}
{कामगेन विचित्रेण दिव्यगन्धवता तथा}


\twolineshloka
{उवाच चैनं विप्रर्षिं विमानं कर्मभिर्जितम्}
{समुपारोह संसिद्धिं प्राप्तोसि परमां मुने}


\twolineshloka
{तमेवंवादिनमृषिर्देवदूतमुवाच ह}
{इच्छामि भवता प्रोक्तान्गुणान्स्वर्गनिवासिनां}


\twolineshloka
{के गुणास्तत्रवसतां किं तपः कश्च निश्चयः}
{स्वर्गे तत्रसुखं किं च दोषो वा देवदूतक}


\twolineshloka
{सतां साप्तपदं मित्रमाहुः सन्तः कुलोचिताः}
{मित्रतां च पुरस्कृत्य पृच्छामि त्वामहं विभो}


\twolineshloka
{यदत्र तथ्यं पथ्यं च तद्ब्रवीह्यविचारयन्}
{श्रुत्वा तथा करिष्यामि व्यवसायं गिरा तव}


\chapter{अध्यायः २६२}
\twolineshloka
{देवदूत उवाच}
{}


\twolineshloka
{महर्षेऽकार्यबुद्धिस्त्वं यः स्वर्गसुखमुत्तमम्}
{संप्राप्तं प्रतिपत्तव्यं विमृशस्यबुधो यथा}


\twolineshloka
{उपरिष्टादयं लोको योऽयं स्वरिति संज्ञितः}
{ऊर्ध्वगः सत्पथः शश्वद्देवयानचरो मुने}


\twolineshloka
{नातप्ततपसः पुंसो नामहायज्ञयाजिनः}
{नानृता नास्तिकाश्चैव तत्रगच्छन्ति मुद्गल}


\twolineshloka
{धर्मात्मानो जितात्मानः शान्ता दान्ता विमत्सराः}
{दानधर्मरताः पुंसः शूराश्चाहितलक्षणाः}


\twolineshloka
{तत्रगच्छन्ति धर्माग्र्यं कृत्वा शमदमात्मकम्}
{लोकान्पुण्यकृतां ब्रह्मन्सद्भिराचरितान्नृभिः}


\twolineshloka
{देवाः साध्यास्तथा विश्वे तथैव च महर्षयः}
{यामा धामाश्च मौद्गल्य गन्धर्वाप्सरसस्तथा}


\twolineshloka
{एषां देवनिकायानां पृथक््पृथगनेकशः}
{भास्वन्तः कामसंपन्ना लोकास्तेजोमयाः शुभाः}


\twolineshloka
{त्रयस्त्रिंशत्सहस्राणि योजनानि हिरण्मयः}
{मेरुः पर्वतराड्यत्रदेवोद्यानानि मुद्गल}


\twolineshloka
{`नन्दनान्यतिरम्याणि तत्रोद्यानानि मुद्गल}
{सर्वकामफलैर्वृक्षैः शोभितानि समन्ततः'}


\twolineshloka
{नन्दनादीनि पुण्यानि विहाराः पुण्यकर्मणाम्}
{न क्षुत्पिपासे न ग्लानिर्न शीतोष्णे भयं तथा}


\twolineshloka
{बीभत्समशुभं वाऽपि रोगो वा तत्र कश्चन}
{मनोज्ञाः सर्वतो गन्धाः सुखस्पर्शाश्च सर्वशः}


\twolineshloka
{शब्दाः श्रुतिमनोग्राह्याः सर्वतस्तत्रवै मुने}
{न शोको न जरा तत्र नायासपरिदेवने}


\twolineshloka
{ईदृशः स मुने लोकः स्वकर्मफलहेतुकः}
{सुकृतैस्तत्रपुरुषाः संभवन्त्यात्मकर्मभिः}


\twolineshloka
{तैजसानि शरीराणि भवन्त्य्रोपपद्यताम्}
{कर्मजान्येव मौद्गल्य न मातृपितृजान्युत}


\twolineshloka
{रन संस्वेदो न दौर्गन्ध्यं पुरीषं मूत्रमेव च}
{तेषां न च रजोवस्त्रं बाधते तत्रवै मुने}


\twolineshloka
{न म्लायन्ति स्रजस्तेषां दिव्यगन्धा मनोरमाः}
{समूह्यन्ते विमानैश्च ब्रह्मन्नेवंविधा हि ते}


\twolineshloka
{ईर्ष्याशोकक्लमापेता मोहमात्सर्यवर्जिताः}
{सुखं संर्गजितस्तत्रवर्तयन्ते महामुने}


\twolineshloka
{तेषां तथाविधानां तु लोकानां मुनिपुङ्गव}
{उपर्युपरि शक्रस् लोका दिव्या गुणान्विताः}


\twolineshloka
{परतो ब्रह्मणस्तस्य लोकस्तेजोमयः शुभः}
{यत्र यान्त्यृषयो ब्रह्मन्पूताः स्वैः कर्मभिः शुभैः}


\twolineshloka
{ऋभवो नाम तत्रान्ये देवानामपि देवताः}
{तेषां लोकाः परतरे यान्यजन्तीह देवताः}


\twolineshloka
{स्वयंप्रभास्ते भास्वन्तो लोकाः कामदुघाः परे}
{न तेषां स्त्रीकृतस्तापो न भोगैश्वर्यमत्सरः}


\twolineshloka
{न वर्तयन्त्याहुतिभिस्ते नाप्यमृतभोजनाः}
{तथा दिव्यशरीरास्ते न च विग्रहमूर्तयः}


\twolineshloka
{नासुखाः सुखकामास्ते देवदेवाः सनातनाः}
{न कल्पपरिवर्तेषु परिवर्तन्ति ते तथा}


\twolineshloka
{रजरा मृत्युः कुतस्तेषां हर्षः प्रीतिः सुखं न च}
{न दुःखं न सुखं चापि रागद्वेषौ कुतो मुने}


\twolineshloka
{देवानामपि मौद्गल्यकाङ्क्षिता सा गतिः परा}
{दुष्प्रापा परमा सिद्धिरगम्या कामगोचरैः}


\twolineshloka
{त्रयस्त्रिंशदिमे लोकाः शेषा लोका मनीषिभिः}
{गम्यन्ते नियमैः श्रेष्ठैर्दानैर्वा विधिपूर्वकैः}


\twolineshloka
{सेयं दानकृता व्युष्टिरनुप्राप्ता सुखं त्वया}
{तां रभुङ्क्ष्व सुकृतैर्लब्धां तपसा द्योतितप्रभः}


\twolineshloka
{एतत्स्वर्गसुखं विप्र लोका नानाविधास्तथा}
{गुणाः स्वर्गस्य प्रोक्तास्ते दोषानपि निबोध मे}


\twolineshloka
{कृस्य कर्मणस्तत्रभुज्यते यत्फलं दिवि}
{न चान्त्क्रियते कर्म मूलच्छेदेन भुज्यते}


\twolineshloka
{सोऽत्रदोषो मम मतस्तस्यान्ते पतनं च यत्}
{सुखव्याप्तमनस्कानां पतनं यच्च मुद्गल}


\twolineshloka
{असंतोषः परीतापो दृष्ट्वा दीप्ततराः श्रियः}
{यद्भवत्यवरे स्थाने स्थितानां तत्सुदुष्करम्}


\twolineshloka
{संज्ञामोहश्चपततां रजसा च प्रधर्षणम्}
{प्रम्लानेषु च माल्येषु ततः पिपतिषोर्भयम्}


\twolineshloka
{आब्रह्मभवनादेते दोषा मौद्गल्य दारुणाः}
{नाकलोकेसुकृतिनां गुणास्त्वयुतशो नृणाम्}


\twolineshloka
{अयं त्वन्यो गुणः श्रेष्ठश्च्युतानां स्वर्गतो मुने}
{शुभानुशययोगेन मनुष्येषूपजायते}


\twolineshloka
{तत्रापि स महाभागः कुले महति जायते}
{न चेत्संबुध्यते तत्रगच्छत्यधमतां ततः}


\threelineshloka
{इह यत्क्रियते कर्म तत्परत्रोपभुज्यते}
{कर्मभूमिरियं ब्रह्मन्फलभूमिरसौ मता ॥[मुद्गल उवाच}
{}


\threelineshloka
{महान्तस्तु अमी दोपास्त्वया स्वर्गस्य कीर्तिताः}
{निर्दोष एव यस्त्यन्यो लोकं तं प्रवदस्व मे ॥देवदूत उवाच}
{}


\twolineshloka
{ब्रह्मणः सदनादूर्ध्वं तद्विष्णोः परमं पदम्}
{शुद्धं सनातनं ज्योतिः परं ब्रह्मेति यद्विदुः}


\twolineshloka
{न तत्रविप्र गच्छन्ति पुरुषा विषयात्मकाः}
{दम्भलोभमहाक्रोधमोहद्रोहैरभिद्रुताः}


\twolineshloka
{निर्ममा निरहंकारा निर्द्विन्द्वाः संयतेन्द्रियाः}
{ध्यानयोगपराश्चैव तत्रगच्छन्ति मानवाः ॥]}


\threelineshloka
{एतत्ते सर्वमाख्यतं यन्मां पृच्छसि मुद्गल}
{तवानुकम्पया साधो साधु गच्छाम माचिरम् ॥व्यास उवाच}
{}


\twolineshloka
{एतच्छ्रुत्वा तु मौद्गल्यो वाक्यं विममृशे धिया}
{विमृश्य च मुनिश्रेष्ठो देवदूतमुवाचह}


\twolineshloka
{देवदूत नमस्तेऽम्तु गच्छ तात यथासुखम्}
{महादोषेण मे कार्यं न स्वर्गेण सुखेन चा}


\twolineshloka
{पतनान्ते महादुःखं परितापः सुदारुणः}
{स्वर्गभाजः पतन्तीह तस्मात्स्वर्गं न कामये}


\twolineshloka
{यत्रगत्वान शोचन्ति न व्यथन्तिचलन्ति वा}
{तदहं स्थानमत्यन्तं मार्गयिष्यामि केवलम्}


\twolineshloka
{इत्युक्त्वा स मुनिर्वाक्यं देवदूतंविसृज्य तम्}
{शिलोञ्छवृत्तिमुत्सृज्य शममातिष्ठदुत्तमम्}


\twolineshloka
{तुल्यनिन्दास्तुतिर्भूत्वासमलोष्टाश्मकाञ्चनः}
{ज्ञानयोगेन शुद्धेन ध्याननित्यो बभूव ह}


\twolineshloka
{`निगृहीतेन्द्रियग्रामः समयोजयदात्मनि}
{युक्तचित्तं यथाऽऽत्मानं युयोज परमेश्वरे'}


\twolineshloka
{ध्यानयोगाद्बलं लब्ध्वा प्राप्य बुद्धिमनुत्तमाम्}
{जगाम शाश्वतीं सिद्धिं परां निर्वाणलक्षणाम्}


\twolineshloka
{तस्मात्त्वमपिकौन्तेय न शोकं कर्तुमर्हसि}
{राज्यात्स्फीतात्परिभ्रष्टस्तपसा तदवाप्स्यसि}


\twolineshloka
{सुखस्यानन्तरं दुःखं दुःखस्यानन्तरं सुखम्}
{पर्यायेणोपसर्पन्ते नरं नेमिमरा इव}


\threelineshloka
{पितृपैतामहं राज्यंप्राप्स्यस्यमितविक्रम}
{वर्षात्रयोदशादूर्ध्वंव्येतु ते मानसो ज्वरः ॥वैशंपायन उवाच}
{}


\twolineshloka
{स एवमुक्त्वाभगवान्व्यासः पाण्डवनन्दनम्}
{जगाम तपसे धीमान्पुनरेवाश्रमं प्रति}


\chapter{अध्यायः २६३}
\twolineshloka
{जनमेजय उवाच}
{}


\twolineshloka
{वसत्स्वेवं वने तेषु पाण्डवेषु महात्मसु}
{रममाणएषु चित्राबिः कथाभिर्मुनिभिः सह}


\threelineshloka
{सूर्यदत्ताक्षयान्नेन कृष्णाया भोजनावधि}
{ब्राह्मणांस्तर्पमाणेषु ये चान्नार्थमुपागताः}
{आरण्यानां मृगाणां च मांसैर्नानाविधैरपि}


\twolineshloka
{धार्तराष्ट्रा दुरात्मानः सर्वे दुर्योधनादयः}
{कथं तेष्वन्ववर्तन्त पापाचारा महामुने}


\threelineshloka
{दःशासनस्य कर्णस्य शकुनेश्च मते स्थिताः}
{एतदाचक्ष्व भगवन्वैशंपायन पृच्छतः ॥वैशंपायन उवाच}
{}


\twolineshloka
{श्रुत्वा तेषां तथा वृत्तिं नगरे वसतामिव}
{दुर्योधनो महाराज तेषु पापमरोचयत्}


\twolineshloka
{तथा तैर्निकृतिप्रज्ञैः कर्णदुःशासनादिभिः}
{नानोपायैरधं तेषु चिन्तयत्सु दुरात्मसु}


\twolineshloka
{अभ्यागच्छत्स धर्मात्मा तपस्वी कसुमहायशाः}
{शिष्यायुतसमोपेतो दुर्वासा नाम कामतः}


\twolineshloka
{तमागतमभिप्रेक्ष्य मुनिं परमकोपनम्}
{सहितो भ्रातृभिः श्रीमानातिथ्येन न्यमन्त्रयत्}


\twolineshloka
{विधिवत्पूजयामास स्वयं किंकरवत्स्थितः}
{अहानि कतिचित्तत्र तस्थौ स मुनिसत्तमः}


\twolineshloka
{तं च पर्यचरद्राजा दिवारात्रमतन्द्रितः}
{दुर्योधनो महाराज शापात्तस्य विशङ्कितः}


\twolineshloka
{क्षुधितोऽस्मि ददस्वान्नं शीघ्रं मम नराधिप}
{इत्युक्त्वा गच्छति स्नातुं प्र्यागच्चति वै चिरात्}


\twolineshloka
{न भोक्ष्याम्यद्यमे नास्ति क्षुधेत्युक्त्वैत्यदर्शनम्}
{अकस्मादेत्य च ब्रूते भोजनयास्मांस्त्वरान्वितः}


\twolineshloka
{कदाचिच्च निशीथे स उत्थाय निकृतौ स्थितः}
{पूर्ववत्कारयित्वाऽन्नं न भुङ्क्ते गर्हयन्स्म सः}


\twolineshloka
{वर्तमाने तथा तस्मिन्यदा दुर्योधनो नृपः}
{विकृतिं नैति न क्रोधं तदा तुष्टोऽभवन्मुनिः}


% Check verse!
आहचैनं दुराधर्षो वरदोऽस्मीति भारत
\threelineshloka
{वरं वरय भद्रं ते यत्ते मनसि वर्तते}
{मयि प्रीते तु यद्धर्म्यं नालभ्यं विद्यते तव ॥वैशंपायन उवाच}
{}


\twolineshloka
{एतच्छ्रुत्वा वचस्तस्य महर्षेर्भावितात्मनः}
{अमन्यत पुनर्जातमात्मानं स सुयोधनः}


\twolineshloka
{प्रागेव मन्त्रितं चासीत्कर्णदुःशासनादिभिः}
{याचनीयं मुनेस्तुष्टादिति निश्चित्य दुर्मतिः}


\twolineshloka
{अतिहर्षान्वितो राजन्वरमेनमयाचत}
{शिष्यैः सह मम ब्रह्मन्यथा जातोऽतिथिर्भवान्}


\threelineshloka
{अस्मत्कुले महाराजो ज्येष्ठः श्रेष्ठो युधिष्ठिरः}
{वने वसति धर्मात्मा भ्रातृभिः परिवारितः}
{गुणवाञ्शीलसंपन्नस्तस्य त्वमतिथिर्भव}


\twolineshloka
{यदा च राजपुत्री सा सुकुमारी यशस्विनी}
{भोजयित्वा द्विजान्सर्वान्पतींश्च वरवर्णिनी}


\twolineshloka
{विश्रान्ता च स्वयं भुक्त्वा सुखासीना भवेद्यदा}
{तदा त्वं तत्रगच्छेथा यद्यनुग्राह्यता मयि}


\twolineshloka
{तथा करिष्ये त्वत्प्रीत्येत्येवमुक्त्वा सुयोधनम्}
{दुर्वासा अपिविप्रेन्द्रो यथागतमगात्ततः}


\twolineshloka
{कृतार्थमपि चात्मानं तदा मेने सुयोधनः}
{करेण च करं गृह्य कर्णस्य मुदितो भृशम्}


\threelineshloka
{कर्णोपि भ्रातृसहितमित्युवाच नृपं मुदा}
{दुष्ट्या कामः सुसंवृत्तोदिष्ट्या कौरव वर्धसे}
{दिष्ट्या ते शत्रवो मग्ना दुस्तरे व्यसनार्णवे}


\threelineshloka
{दुर्वासःक्रोधजे वह्नौ पतिताः पाण्डुनन्दनाः}
{स्वैरेव ते महापापैर्गता वै दुस्तरं तमः ॥वैशंपायु उवाच}
{}


\twolineshloka
{इत्थं ते निकृतिप्रज्ञा राजन्दुर्योधनादयः}
{हसन्तः प्रीतमनसो जग्मुः स्वंस्वं निकेतनम्}


\chapter{अध्यायः २६४}
\twolineshloka
{वैशंपायन उवाच}
{}


\threelineshloka
{ततः कदाचिद्दुर्वासाः सुखासीनांस्तु पाण्डवान्}
{भुक्त्वा चावस्थितां कृष्णां ज्ञात्वा तस्मिन्वने मुनिः}
{अभ्यागच्छत्परिवृतः शिष्यैरयुतसंमितैः}


\twolineshloka
{दृष्ट्वा यान्तं तमतिथिं स च राजा युधिष्ठिरः}
{जगामाभिमुखः श्रीमान्सह भ्रातृभिरच्युतः}


\threelineshloka
{तस्मै बद्ध्वाञ्जलिं सम्यगुपवेश्य वरासने}
{विधिवत्पूजयित्वा तमातिथ्यन न्यमन्त्रयत्}
{आह्निकं भगवन्कृत्वा शीघ्रमेहीति चाब्रवीत्}


\twolineshloka
{जगाम च मुनिः सोपि स्नातुं शिष्यैः सहानघः}
{भोजयेत्सहशिष्यं मां कथमित्यविचिन्तयन्}


% Check verse!
न्यमज्जत्सलिले चापि मुनिसङ्घः समाहितः
\twolineshloka
{एतस्मिन्नन्तरे राजन्द्रौपदी योपितां वरा}
{चिन्तामवाप परमामन्नहेतोः पतिव्रता}


\twolineshloka
{सा चिन्तयन्ती च यदा नान्नहेतुमविन्दत}
{मनसा चिन्तयामास कृष्णं कंसनिषूदनम्}


\twolineshloka
{कृष्णकृष्ण महाबाहो देवकीनन्दनाव्यय}
{वासुदेव जगन्नाथ प्रणतार्तिविनाशन}


\threelineshloka
{विश्वात्मन्विश्वजनक विश्वहर्तः प्रभोऽव्यय}
{प्रपन्नपाल गोपाल प्रजापाल परात्पर}
{आकूतीनां च चित्तीनां प्रवर्तक नताऽस्मि ते}


\twolineshloka
{वरेण्य वरदानन्त अगतीनां गतिप्रद}
{पुराणपुरुष प्राणमनोवृत्त्याद्यगोचर}


\twolineshloka
{सर्वाध्यक्ष पराध्यक्ष त्वामहं शरणं गता}
{पाहि मां कृपया देव शरणागतवत्सल}


\twolineshloka
{नीलोत्पलदलश्याम पद्मगर्भारुणेक्षण}
{पीताम्बरपरीधान लसत्कौस्तुभभूषण}


\twolineshloka
{त्वमादिरन्तो भूतानां त्वमेव च परायणम्}
{परात्परतरं ज्योतिर्विश्वात्मा सर्वतोमुखः}


\twolineshloka
{त्वामेवाहुः परं बीजं निधानं सर्वसंपदाम्}
{त्वया नाथेन देवेश सर्वापद्म्यो भयं न हि}


\threelineshloka
{दुःशासनादहं पूर्वं सभायां मोचिता यथा}
{तथैव संकटादस्मान्मामुद्धर्तुमिहार्हसि ॥वैशंपायन उवाच}
{}


\twolineshloka
{एवं स्तुतस्तदा देवः कृष्णया भक्तवत्सलः}
{द्रौपद्याः संकटं ज्ञात्वा देवदेवो जगत्पतिः}


\twolineshloka
{पार्स्वस्थां शयने त्यक्त्वा रुक्मिणीं केशवः प्रभुः}
{तत्राजगाम त्वरितो ह्यचिन्त्यगतिरीश्वरः}


\twolineshloka
{ततस्तं द्रौपदी दृष्ट्वाप्रणम्य परया मुदा}
{अब्रवीद्वासुदेवाय मुनेरागमनादिकम्}


\twolineshloka
{ततस्तामब्रवीत्कृष्णः क्षुधितोस्मि भृशातुरः}
{शीघ्रं भोजय मां कृष्णे पश्चात्सर्वं करिष्यसि}


\twolineshloka
{निशम्य तद्वचः कृष्णा लज्जिता वाक्यमब्रवीत्}
{स्थाल्यां भास्करदत्तायामन्नं मद्भोजनावधि}


\twolineshloka
{भुक्तवत्यस्म्यहं देव तस्मादन्नं न विद्यते}
{ततः प्रोवाच भगवान्कृष्णां कमललोचनः}


\twolineshloka
{कृष्णे न नर्मकालोऽयं क्षुच्छ्रमेणातुरे मयि}
{शीघ्रं गच्छ मम स्थालीमानयित्वा प्रदर्शय}


\twolineshloka
{इति निर्बन्धतः स्थालीमानाय्य स यदूद्वहः}
{स्थाल्याः कण्ठेऽथ संलग्नं शाकान्नं वीक्ष्यकेशवः}


\twolineshloka
{उपयुज्याब्रवीदेनामनेन हरिरीश्वरः}
{विश्वात्मा प्रीयतां देवस्तुष्टश्चास्त्विति यज्ञभूक्}


\twolineshloka
{आकारय मुनीञ्शीघ्रं भोजनायेति चाब्रवीत्}
{भीमसेनं महाबाहुः कृष्णः क्लेशविनाशनः}


\threelineshloka
{ततो जगाम त्वरितो भीमसेनो महायशाः}
{आकारितुं तु तान्सर्वान्भोजनार्थं नृपोत्तम}
{स्नातुं गतान्देवनद्यां दुर्वासःप्रभृतीन्मुनीन्}


\threelineshloka
{ते चावतीर्णाः सलिले कृतवन्तोऽघमर्षणम्}
{दृष्ट्वोद्गारान्सान्नरसांस्तृप्त्या परमया युताः}
{उत्तीर्य सलिलात्तस्माद्दृष्टवन्तः परस्परम्}


\twolineshloka
{दुर्वाससमभिप्रेक्ष्यते सर्वे मुनयोऽब्रुवन्}
{राज्ञा हिकारयित्वाऽन्नं वयं स्नातुं समागताः}


\threelineshloka
{आकण्ठतृप्ता विप्रर्षे किंस्विद्भुञ्जामहे वयम्}
{वृथा पाकः कृतोस्माभिस्तत्र किं करवामहे ॥दुर्वासा उवाच}
{}


\twolineshloka
{वृथा पाकेन राजर्षेरपराधः कृतो महान्}
{माऽस्मानधाक्षुर्दृष्ट्वैव पाण्डवाः क्रूरचक्षुषा}


\twolineshloka
{स्मृत्वाऽनुभावं राजर्षेरम्बरीषस्य धीमतः}
{बिभेमि सुतरां विप्रा हरिपादाश्रयाज्जनात्}


\twolineshloka
{पाण्डवाश्च महात्मानः सर्वे धर्मपरायणाः}
{शूराश्चकृतविद्याश्च व्रतिनस्तपसि स्थिताः}


\fourlineindentedshloka
{सदाचाररता नित्यं वासुदेवपरायणाः}
{क्रुद्धास्ते निर्दहेयुर्वै तूलराशिमिवानलः}
{तत एतानपृष्ट्वैव शिष्याः शीघ्रं पलायत ॥वैशंपायन उवाच}
{}


\twolineshloka
{इत्युक्तास्ते द्विजाः सर्वे मुनिना गुरुणा तदा}
{पाण्डवेभ्यो भृशं भीता दुद्रुवुस्ते दिशो दश}


\twolineshloka
{भीमसेनो देवनद्यामपश्यन्मुनिसत्तमान्}
{तीर्थे ष्वितस्ततस्तस्या विचचार गवेषयन्}


\twolineshloka
{तत्रस्थेभ्यस्तापसेभ्यः श्रुत्वा ताश्चैव विद्रुतान्}
{युधिष्ठिरमथाभ्येत्य तं वृत्तान्तं न्यवेदयत्}


\twolineshloka
{ततस्ते पाण्डवाः सर्वे प्र्यागमनकाङ्क्षिणः}
{प्रतीक्षनतः कियत्कालं जितात्मानोऽवतस्थिरे}


\twolineshloka
{निशीथेऽभ्येत्य चाकस्मादस्मान्स च्छलयिष्यति}
{कथं च निस्तरेमास्मात्कृच्छ्राद्दैवोपसादितात्}


\twolineshloka
{इति चिन्तापरान्दृष्ट्वा निःश्वसन्तो मुहुर्मुहुः}
{उवाच वचनं श्रीमान्कृष्णः प्रत्यक्षतां गतः}


\twolineshloka
{भवतामापदं ज्ञात्वा ऋषेः परमकोपनात्}
{द्रौपद्या चिन्तितः पार्था अहं सत्वरमागतः}


\twolineshloka
{न भयं विद्यतेतस्मादृषेर्दुर्वाससोऽल्पकम्}
{तेजसा भवतां भीतः पूर्वमेव पलायितः}


\threelineshloka
{धर्मनित्यास्तु ये केचिन्न ते सीदन्ति कर्हिचित्}
{आपृच्छे वो गमिष्यामि नियतं भद्रमस्तु वः ॥वैशंपायन उवाच}
{}


\twolineshloka
{श्रुत्वेरितं केशवस्य बभूवुः स्वस्थामानसाः}
{द्रौपद्या सहिताः पार्थास्तमूचुर्विगतज्वराः}


\twolineshloka
{त्वया नाथेन गोविन्द दुस्तरामापदं विभो}
{तीर्णाः प्लवमिवासाद्य मज्जमाना महार्णवे}


% Check verse!
स्वस्ति साधय भद्रं ते इत्याज्ञातो ययौ पुरीम्
\threelineshloka
{पाण्डवाश्च महाभाग द्रौपद्या सहिताः प्रभो}
{ऊषुः प्रहृष्टमनसो विहरन्तो वनाद्वनम्}
{इति तेऽभिहितं राजन्यत्पृष्टोऽहमिह त्वया}


\twolineshloka
{एवंविधान्यलीकानि धार्तराष्ट्रैर्दुरात्मभिः}
{पाण्डवेषु वनस्थेषु प्रयुक्तानि वृथाऽभवन्}


\chapter{अध्यायः २६५}
\twolineshloka
{वैशंपायन उवाच}
{}


\twolineshloka
{तस्मिन्बहुमृगेऽरण्ये अटमाना महारथाः}
{काम्यके भरतश्रेष्ठा विजह्वुस्ते यथामराः}


\twolineshloka
{प्रेक्षमाणा बहुविधान्वनोद्देशान्समन्ततः}
{यथर्तुकालरम्याश्चवनराजीः सुषुष्पिताः}


\twolineshloka
{पाण्डवा मृगयाशीलाश्चरन्तस्तन्महद्वनम्}
{विजह्नरिन्द्रप्रिमाः कंचित्कालमरिंदमाः}


\twolineshloka
{ततस्ते यौगपद्येन ययुः सर्वे चतुर्दिशम्}
{मृगयां पुरुषव्याघ्रा ब्राह्मणार्थे परतपाः}


\twolineshloka
{द्रौपदीमाश्रमे न्यस् तृणबिन्दोरनुज्ञया}
{महर्षेर्दीप्ततपसो घौम्यस्य च पुरोधसः}


\twolineshloka
{तस्तु राजा सिंधूनां वार्धक्षत्रिर्महायशाः}
{विवाहकामः साल्वेयान्प्रयातः सोऽभवत्तदा}


\twolineshloka
{महता परिबर्हेण राजयोग्येन संवृतः}
{राजभिर्बहुभिः सार्धमुपायात्काम्यकं च सः}


\twolineshloka
{तत्रापश्यत्प्रियां भार्यां पाण्डवानां यशस्विनीम्}
{तिष्ठन्तीमाश्रमद्वारि द्रौपदीं निर्जने वने}


\twolineshloka
{विभ्राजमानां वपुषा बिभ्रतीं रूपमुत्तमम्}
{भ्राजयन्तीं वनोद्देशं नीलाभ्रमिव विद्युतम्}


\twolineshloka
{अप्सरा देवकन्या वा माया वा देवनिर्मिता}
{इतिकृत्वाञ्जलिं सर्वे ददृशृस्तामनिन्दिताम्}


\twolineshloka
{तः स राजा सिन्धूनां वार्धक्षत्रिर्जयद्रथः}
{विस्मितस्त्वनवद्याङ्गीं दृष्ट्वा तां दुष्टमानसः}


\twolineshloka
{स कोटिकाश्यं राजानमब्रवीत्काममोहितः}
{कस्य त्वेषाऽनवद्याङ्गी यदिवाऽपिन मानुषी}


\twolineshloka
{विवाहार्थो न मे कश्चिदिमां दृष्ट्वाऽतिमुन्दरीम्}
{एतामेवाहमादाय गमिष्यामि स्वमालयम्}


\twolineshloka
{गच्छ जानीहि सौम्येमां कस्य वाऽत्र कुतोपि वा}
{किमर्थमागता सुभ्रूरिदं कष्टकितं वनम्}


\twolineshloka
{अपि नाम वरारोहा मामेषा लोकसुन्दरी}
{भजेदद्यायतापाङ्गी सुदती तनुमध्यमा}


\twolineshloka
{अप्यहं कृतकामः स्यामिमां प्राप्य वरस््रियम्}
{गच्छजानीहि कोन्वस्या नाथ इत्येव कोटिक}


\twolineshloka
{स कोटिकाश्यस्तच्छ्रुत्वा रथात्प्रस्कन्द्य कुण्डली}
{उपेत्यपप्रच्छ तदा क्रोष्टा व्याघ्रवधूमिव}


\chapter{अध्यायः २६६}
\twolineshloka
{कोटिक उवाच}
{}


\twolineshloka
{का त्वं कदम्बस्य विनम्य शाखांकिमाश्रमे तिष्ठसि शोभमाना}
{देदीप्यमानाऽग्निशिखेव नक्तंव्याधूयमाना पवनेन सूभ्रूः}


\twolineshloka
{अतीव रूपेण समन्विता त्वंन चाप्यरण्येषु बिभेषि किंनु}
{देवी नु यक्षी यदिदानवी वावराप्सरा दैत्यवराङ्गना वा}


\twolineshloka
{वपुष्मती वोरगराजकन्यावनेचरी वा क्षणदाचरस्त्री}
{त्वं देवराज्ञो वरुणस्य पत्नीयमस्य सोमस्य धनेश्वरस्य}


\twolineshloka
{धातुर्विधातुः सवितुर्विभोर्वाशक्रस्य वा त्वं सदनात्प्रपन्ना}
{न ह्येव नः पृच्छसि ये वयं स्मन चापि जानीम तवेह नाथम्}


\twolineshloka
{वयं हि मानं तव वर्धयन्तःपृच्छाम भद्रेप्रभवं प्रभुं च}
{आचक्ष्व बन्धूंश्च पतिं कुलं चजातिं च यच्चेगह करोषि कार्यम्}


\twolineshloka
{अहं तु राज्ञः सुरथस्य पुत्रोयं कोटिकाश्येति विदुर्मनुष्याः}
{`वश्येन्द्रियः सत्यरतिर्वरोरुवृद्धोपसेवी गुरुपूजकश्च'}


\twolineshloka
{असौ तु यस्तिष्ठतिकाञ्चनाङ्गेतथे हुतोऽग्निश्चयने यथैव}
{तरिगर्तराजः कमलायताक्षःक्षेमंकरो नाम स एष वीरः}


\twolineshloka
{अस्मात्परस्त्वेष महाधनुष्मा-न्पुत्रः कुलिन्दाधिपतेर्वरिष्ठः}
{निरीक्षते त्वां विपुलायताक्षःसुपुष्पितः पर्वतवासनित्यः}


\twolineshloka
{असौ तु यः पुष्करिणीसमीपेश्यामो युवा तिष्ठति दर्शनीयः}
{इक्ष्वाकुराजः सुबलस्य पुत्रःस एव हन्ता द्विषतां सुगात्रि}


\twolineshloka
{यस्यानुयात्रं ध्वजिनः प्रयान्तिसौवीरका द्वादशराजपुत्राः}
{शोणाश्वयुक्तेषु रथेषु सर्वेमस्वेषु दीप्ता इव इव्यवाहाः}


\twolineshloka
{अङ्गारकः कुञ्जरो गुप्तकश्चश्रुंजयः संजयसुप्रवृद्धौ}
{प्रभंकरोऽथ भ्रमरो रविश्चशूरः प्रतापःकुहनश्च नाम}


\twolineshloka
{यं षट्सहस्रा रथिनोऽनुयान्तिनागा हयाश्चैव पदातिनश्च}
{जयद्रथो नाम यदि श्रुतस्तेसौवीरराजः सुभगे स एषः}


\twolineshloka
{तस्यापरे भ्रातरोऽदीनसत्वाबलाहकानीकविदारणाद्याः}
{सौवीरवीराः प्रवरा युवानोराजानमेते बलिनोऽनुयान्ति}


\twolineshloka
{एतैः सहायैरुपयाति राजामरुद्गणैरिन्द्र इवाभिगुप्तः}
{अजानतां ख्यापय नः सुकेशिकस्यासि भार्या दुहिता च कस्य}


\chapter{अध्यायः २६७}
\twolineshloka
{वैशंपायन उवाच}
{}


\twolineshloka
{अथाब्रवीद्द्रौपदी राजपुत्रीपृष्टा शिबीनां कप्रवरेण तेन}
{अवेक्ष्य मन्दं प्रविमुच्य शाखांसंगृह्णती कौशिकमुत्तरीयम्}


\twolineshloka
{बुद्ध्याऽभिजानामि नरेनद््रपुत्रन मादृशी त्वामभिभाष्टुमर्हति}
{न त्वेह वक्ताऽस्ति तवेह वाक्य-मन्यो नरो वाऽप्यथवाऽपि नारी}


\twolineshloka
{एका ह्यहं संप्रति तेन वाचंददानि वै भद्र निबोध चेदम्}
{अहं ह्यरण्येकथमेकमेकात्वामालपेयं निरता स्वधर्मे}


\twolineshloka
{जानामि च त्वां सुरथस्य पुत्रंयं कोटिकाश्येति विदुर्मनुष्याः}
{तस्मादहं शैब्य तथैव तुभ्य-माख्यामि बन्धून्प्रथितं कुलं च}


\twolineshloka
{अपत्यमस्मि द्रुपदस्य राज्ञःकृष्णेति मां शैब्य विदुर्मनुष्याः}
{साऽहं वृणे पञ्च जनान्पतित्वेये खाण्डवप्रस्थगताः श्रुतास्ते}


\twolineshloka
{युधिष्ठिरो भीमसेनार्जुनौ चमाद्र्याश्च पुत्रौ पुरुषप्रवीरौ}
{ते मां निवेश्यह दिशश्चतस्रोविबज्य पार्था मृगयां प्रयाताः}


\twolineshloka
{प्राचीं राजा दक्षिणां भीमसेनोजयः प्रतीचीं यमजावुदीचीम्}
{मन्ये तु तेषां रथसत्तमानांकालो बहुः प्राप्ता इहोपयातुम्}


\twolineshloka
{संमानिता यास्यथ तैर्यथेष्टंविमुच्य वाहानवरोहयध्वम्}
{प्रियातिथिर्धर्मसुतो महात्माप्रीतो भविष्यत्यभिवीक्ष्य युप्मान्}


\twolineshloka
{एतावदुक्त्वा द्रुपदात्मजा साशैव्यात्मजं चन्द्रसुखी प्रतीता}
{विवेश तां पर्णशालां प्रशस्तांसंचिन्त्य तेषामतिथिस्वधर्मम्}


\chapter{अध्यायः २६८}
\twolineshloka
{वैशंपायन उवाच}
{}


\twolineshloka
{तथाऽऽसीनेषु सर्वेषु तेषु राजसु भारत}
{`कोटिकाश्यो जगामाशु सिन्धुराजनिवेशनम्}


\twolineshloka
{यदुक्तं कृष्णया सार्धं तत्सर्वं प्रत्यवेदयत्}
{कोटिकाश्यवचः श्रुत्वा शैब्यं सौवीरकोऽब्रवीत्}


\twolineshloka
{यदा वाचं व्याहरन्त्यामस्यां मे रमते मनः}
{सीमन्तिनीनां सुख्यायां विनिवृत्तः कथं भवान्}


\twolineshloka
{एतां दृष्ट्वा स्त्रियो मेऽन्या यथा शाखामृगस्त्रियः}
{प्रतिभान्ति महाबाहो सत्यमेतद्ब्रवीमि ते}


\threelineshloka
{दर्शनादेव हि मनस्तया मेऽपहृतं भृशम्}
{तां समाचक्ष्व कल्याणीं यदि स्याच्छैब्य मानुषी ॥कोटिक उवाच}
{}


\twolineshloka
{एषा वै द्रौपदी कृष्णा राजपुत्री यशस्विनी}
{पञ्चानां पाण्डुपुत्राणां महिषी संमता भृशम्}


\threelineshloka
{सर्वेषां चैव पार्थानां प्रिया बहुमता सती}
{तया समेत्य सौवीर सौवीराभिमुखो व्रज ॥वैशंपायन उवाच}
{}


\twolineshloka
{एवमुक्तः प्रत्युवाच पश्यामि द्रौपदीमिति}
{पतिः सौवीरसिन्धूनां दुष्टभावो जयद्रथः}


\twolineshloka
{स प्रविश्याश्रमं पुण्यं सिंहगोष्ठं वृको यथा}
{आत्मना सप्तमः कृष्णामिदं वचनमब्रवीत्}


\threelineshloka
{कुशलं ते वरारोहे भर्तारस्तेऽप्यनामयाः}
{येषां कुशलकामासि तेऽपिकच्चिदनामयाः ॥द्रौपद्युवाच}
{}


\threelineshloka
{अपि ते कुशलं राजन्राष्ट्रे कोशे बले तथा}
{कच्चिदेकः शिबीनाढ्यान्सौवीरान्सह सिन्धुभिः}
{अनुतिष्ठसि धर्मेण ये चान्ये विदितास्त्वया}


\twolineshloka
{कौरव्यः कुशली राजा कुन्तीपुत्रो युधिष्ठिरः}
{अहं च भ्रातरश्चास्य यांश्चान्यान्परिपृच्छसि}


\twolineshloka
{पाद्यं प्रतिगृहाणेदमासनं च नृपात्मज}
{मृगान्विता मृगीश्चैव प्रातराशं ददानि ते}


\twolineshloka
{ऐणेयान्पृषतान्न्यङ्कून्हरिणाञ्शरभाञ्शशान्}
{ऋक्षाव्रुरूञ्शम्बरांश्च गवयांश्च मृगान्बहून्}


\threelineshloka
{वराहामहिषांश्चैव याश्चान्या मृगजातयः}
{प्रदास्यति स्वयं तुभ्यं कुन्तीपुत्रो युधिष्टिरः ॥जयद्रथ उवाच}
{}


\twolineshloka
{कुशलं प्रातराशस्य सर्वं मे दित्सितं त्वया}
{एहि मे रथमारोह सुखमाप्नुहि केवलम्}


\twolineshloka
{गतश्रीकान्हृतराज्यान्कृपणान्गतचेतसः}
{अरण्यवासिनः पार्थान्नानुरोद्धुं त्वमर्हसि}


\twolineshloka
{नैव प्राज्ञा गतश्रीकं भर्तारमुपयुञ्जते}
{युञ्जानमनुयुञ्जीत न शरियः संक्षये वसेत्}


\twolineshloka
{श्रिया विहीना राष्ट्राच्च विनष्टाः शाश्वतीः समाः}
{अलं ते पाण्डुपुत्राणां भक्त्या क्लेशमुपासितुम्}


\threelineshloka
{भार्या मे भव सुश्रोणि त्यजैनान्मुखमाप्नुहि}
{अखिलान्सिन्धुसौवीरानाप्नुहि त्वं मया सह ॥वैशंपायन उवाच}
{}


\twolineshloka
{इत्युक्ता सिन्धुराजेन वाक्यं हृदयकम्पनम्}
{कृष्णा तस्मादपाक्रामद्देशात्सभ्रुकुटीमुखी}


\twolineshloka
{अवमत्यास्य तद्वांक्यमाक्षिप्य च सुमध्यमा}
{मैवमित्यब्रवीत्कृष्णा लज्जस्वेति च सैन्धवम्}


\threelineshloka
{सा काङ्क्षमाणा भर्तृणामुपयातमनिन्दिता}
{विलम्बयामास परं वाक्यैर्वाक्यानि युञ्जती ॥द्रैपद्युवाच}
{}


\twolineshloka
{नैवं वद महाबाहो न्याय्यं त्वं न च मन्यसे}
{पाण्डूनां धार्तराष्ट्राणां स्वसा चैव कनीयसी}


\twolineshloka
{दुश्शला नाम तस्यास्त्वं भर्ता राजकुलोद्वह}
{मम भ्राता च न्याय्येन त्वया रक्ष्या महारथ}


\twolineshloka
{धर्मिष्ठानां कुले जातो न धर्मं त्वमवेक्षसे}
{इत्युक्तः सिन्धुराजोथ वाक्यमुत्तरमब्रवीत्}


\twolineshloka
{राज्ञां धर्मं न जानीषे स्त्रियो रत्नानि चैव हि}
{साधारणानि लोकेऽस्मिन्प्रवदन्ति मनीषिणः}


\twolineshloka
{स्वसा च स्वस्रिया चैव भ्रातृभार्या तथैव च}
{सुखं गृह्णन्ति राजानस्ताश्च तत्र नृपोद्भवाः'}


\chapter{अध्यायः २६९}
\twolineshloka
{वैशंपायन उवाच}
{}


\twolineshloka
{सरोषरागोपहतेन वल्गुनासरागनेत्रेण नतोन्नतभ्रुवा}
{मुखेन विस्फूर्य सुवीरराष्ट्रपंततोऽब्रवीत्तं द्रुपदात्मजा पुनः}


\twolineshloka
{यशस्विनस्तीक्ष्णविषान्महारथा-नभिब्रुवन्मूढ न लज्जसे कथम्}
{महेन्द्रकल्पान्निरतान्स्वकर्मसुस्तितान्समूहेष्वपि यक्षरक्षसाम्}


\twolineshloka
{न किंचिदार्याः प्रवदन्ति पापंवनेचरंवा गृहमेधिनं वा}
{तपस्विनं संपरिपूर्णविद्यंभजन्ति चैवं सुनराः सुवीर}


\twolineshloka
{अहं तु मन्ये तव नास्ति कश्चि-देतादृशे क्षत्रियसंनिवेशे}
{यस्त्वद्य पातालमुखे पतन्तंपाणौ गृहीत्वा प्रतिसंहरेत}


\twolineshloka
{नागं प्रमिन्नं गिरिकूटकल्प-मुपत्यकां हैमवतीं चरन्तम्}
{दण्डीव यूथादपसेधसि त्वंयो जेतुमाशंससि धर्मराजम्}


\twolineshloka
{बाल्यात्प्रसुप्तस्य महाबलस्यसिंहस् पक्ष्माणि मुखाल्लुनासि}
{पदा समाहत्य पलायमानःक्रुद्धं यदा द्रक्ष्यसि भीमसेनम्}


\twolineshloka
{महाबलं घोरतरं प्रवृद्धंजातं हरिं पर्वतकन्दरेषु}
{प्रसुप्तमुग्रं प्रपदेन हंसियः क्रुद्धमायोत्स्यसि जिष्णुमुग्रम्}


\twolineshloka
{कृष्णोरगौ तीक्ष्णमुखौ द्विजिह्वौमत्तः पदा क्रामसि पुच्छदेशे}
{यः पाण्डवाभ्यां पुरुषोत्तमाभ्यांजघन्यजाभ्यां प्रयुयुत्ससे त्वम्}


\threelineshloka
{यथा च वेणुः कदली नलो वाफलन्त्यभावाय न भूतयेऽऽत्मनः}
{तथैव मां तैः परिरक्ष्यमाणा-मादास्यसे कर्कटकीव गर्भम् ॥जयद्रथ उवाच}
{}


\twolineshloka
{जानामि कृष्णे विदितं ममैत-द्यथाविधास्ते नरदेवपुत्राः}
{न त्वेवमेतेन विभीषणेनशक्या वयं त्रासयितुं त्वयाऽद्य}


\twolineshloka
{वयं पुनः सप्तदशेषु कृष्णेकुलेषु सर्वेऽनवमेषु जाताः}
{षड्भ्यो गुणेभ्योऽभ्यधिका विहीनान्मन्यामहे द्रौपदी पाण्डुपुत्रान्}


\threelineshloka
{सा क्षिप्रमातिष्ठ गजं रथं वान वाक्यमात्रेण वयं हि शक्याः}
{आशंस वा त्वं कृपणं वदन्तीसौवीरराजस्य पुनः प्रसादम् ॥द्रौपद्युवाच}
{}


\twolineshloka
{महाबला किंत्विह दुर्बलेवसौवीरराजस्य मताऽहमस्मि}
{नाहं प्रमाथादिह संप्रतीतासौवीरराजं कृपणं वदेयम्}


\twolineshloka
{यस्या हि कृष्णौ पदवीं चरेतांसमास्थितावेकरथे समेतौ}
{इन्द्रोऽपितां नापहरेत्कथंचि-न्मनुष्यमात्रः कृपणः कुतोऽन्यः}


\twolineshloka
{यथा विकीटी परवीरघातीनिघ्नन्रथस्थो द्विषतां मनांसि}
{मदन्तरे त्वद्ध्वजिनीं प्रवेष्टावक्षं दहन्नग्निरिवोष्णगेषु}


\twolineshloka
{जनार्दनः सान्धकवृष्णिवीरोमहेष्वासाः केकयाश्चापि सर्वे}
{एते हि सर्वे मम राजपुत्राःप्रहृष्टरूपाः पदवीं चरेयुः}


\twolineshloka
{मौर्वीविसृष्टाः स्तनयित्नुघोषागाण्डीवमुक्तास्त्वतिवेगवन्तः}
{हस्तं समाहत्य धनंजयस्यभीमाः शब्दं घोरतरं नदनति}


\twolineshloka
{गाण्डीवमुक्तांश्च महाशरौघान्पतङ्गसङ्घानिव शीघ्रवेगान्}
{यदा द्रष्टास्यर्जुनं वीर्यशालिनंतदा स्वबुद्धिं प्रतिनिन्दितासि}


\twolineshloka
{सशङ्खघोषः सतलत्रघोषोगाण्डीवधन्वा मुहुरुद्वहंश्च}
{यदा शरानर्पयिता तवोरसितदा मनस्ते किमिवाभविष्यत्}


\twolineshloka
{गदाहस्तं भीममभिद्रवन्तंमाद्रीपुत्रौ संपतन्तौ दिशश्च}
{अमर्षजं क्रोधविषं वमन्तौदृष्ट्वा चिरं तापमुपैष्यसेऽधम}


\twolineshloka
{यथा वाऽहं नातिचरे कथंचि-त्पतीन्महार्हान्मनसाऽपिजातु}
{तेनाद्य सत्येन वशीकृतंत्वांद्रष्टास्मि पार्थैः परिकृष्यमाणम्}


\threelineshloka
{न संभ्रमं गन्तुमहं हि शक्ष्येत्वया नृशंसेन विकृष्यमाणा}
{समागताऽहं हि कुरुप्रवीरैःपुनर्वनं काम्यकमागताऽस्मि ॥वैशंपायन उवाच}
{}


\twolineshloka
{`इत्येवमुक्तस्तु स सिन्धुनाथ-स्तां द्रौपदीमाहविशालनेत्राम्}
{आरुह्यतामाशु रथं मदीयंमा त्वां वलाद्दौपदिकर्षयेहम्'}


\twolineshloka
{सा ताननुप्रेक्ष्य विशालनेत्राजिघृक्षमाणानवभर्त्सयन्ती}
{प्रोवाच मा मा स्पृशतेति भीताधौम्यं प्रचुक्रोश पुरोहितं सा}


\twolineshloka
{जग्राह तामुत्तरवस्त्रदेशेजयद्रथस्तं समवाक्षिपत्सा}
{तया समाक्षिप्ततनुः स पापःपपात शाखीव निकृत्तमूलः}


\threelineshloka
{प्रगृह्यमाणा तु महाजवेनमुहुर्विनिःश्वस्य च राजपुत्री}
{सा मृष्यमाणा रथमारुरोहधौम्यस्य पादावभिवाद्य कृष्णा ॥धौम्य उवाच}
{}


\twolineshloka
{नेयं शक्या त्वया नेतुमविजित्य महारथान्}
{धर्मं क्षत्रस् पौराणमवेक्षस्व जयद्रथ}


\threelineshloka
{क्षुद्रं कृत्वाफलंपापं त्वं प्राप्स्यसि न संशयः}
{आसाद्य पाण्डवान्वीरान्धर्मराजपुरोगमान् ॥वैशंपायन उवाच}
{}


\twolineshloka
{इत्युक्तवाह्रियमाणां तां राजपुत्रीं यशस्विनीम्}
{अन्वगच्छत्तदा धौम्यः पदातिगणमध्यगः}


\chapter{अध्यायः २७०}
\twolineshloka
{वैशंपायन उवाच}
{}


\twolineshloka
{ततो दिशः संपरिवृत्य पार्थामृगान्वराहान्महिपांस्च हन्वा}
{धनुर्धराः श्रेष्ठतमाः पृथिव्यांपृथक्चरन्तः सहिता बभूवुः}


\twolineshloka
{ततो मृगव्यालजनानुकीर्णंमहावनं तद्विहगोपघुष्टम्}
{भ्रातॄंश्च तानभ्यवदद्युधिष्ठिरःश्रुत्वा गिरो व्याहरतां मृगाणाम्}


\twolineshloka
{आदित्यदीप्तां दिशमभ्युपेत्यमृगा द्विजाः क्रूरमिमे वदन्ति}
{आयासमुग्रं प्रतिवेदयन्तोमहाभयं शत्रुभिर्वाऽवमानम्}


\twolineshloka
{क्षिप्रं निवर्तध्वमलं मृगैर्नोमनो हि मे दूयति दह्यते च}
{बुद्धिं समाच्छाद्य च मे समन्यु-रुद्धूयते प्राणपतिः शरीरे}


\twolineshloka
{सरः सुपर्णन हृतोरगेन्द्र-मराजकं राष्ट्रमिवेह शान्तम्}
{एवंविधं मे प्रतिभाति काम्यकंशौण्डैर्यथा पीतसुरश्च कुम्भः}


\twolineshloka
{ते सैन्धवैरग्न्यनिलोग्रवेगै-र्महाजवैर्वाजिभिरुह्यमानाः}
{युक्तैर्बृहद्भिः सुरथैर्नृवीरा-स्तदाश्रमायाभिमुखा बभूवुः}


\twolineshloka
{तेषां तु गोमायुरनल्पघोषोनिवर्ततां वाममुपेत्य पार्श्वम्}
{प्रव्याहरत्तत्प्रविमृश्य राजाप्रोवाच भीमं च धनंजयं च}


\twolineshloka
{यथा वदत्येप विहीनयोनिःसालावृकोवाममुपेत्य पार्स्वम्}
{सुव्यक्तमस्मानवमत्य पापैःकृतोऽभिमर्दः कुरुभिः प्रसह्य}


\twolineshloka
{एत्याथ ते तद्वनमाविशन्तोमहत्यरण्ये मृगयां चरित्वा}
{बालामपश्यन्त तदा रुदन्तींधात्रेयिकां प्रेष्यवधूं प्रियायाः}


\twolineshloka
{तामिन्द्रसेनस्त्वरितोऽभिसृत्यरथादवप्लुत्य ततोऽभ्यधावत्}
{प्रोवाच चैनां वचनं नरेन्द्रधात्रेयिकामार्ततरस्तदानीम्}


\threelineshloka
{किं रोदिषि त्वं पतिता धरण्यांकिं त मुखं शुष्यति दीनवर्णम्}
{कच्चिन्न पापैः सुनृशंसकृद्भिःप्रमाथिता द्रौपदी राजपुत्री}
{`गतेष्वरण्यं हि सुतेषु पाण्डोःकच्चित्परैर्नापकृतं वनेऽस्मिन्}


\threelineshloka
{पर्याकुला साधु समीक्ष्यसूत-मभ्यापतन्तं द्रुतमिन्द्रसेनम्}
{उरो घ्नती कष्टतरं तदानी-मुच्चैः प्रचुक्रोश हृतेति देवी ॥इन्द्रसेन उवाच}
{}


\twolineshloka
{अनिन्द्यरूपा तु विशालनेत्राशरीरतुल्या कुरुपुङ्गवानाम्}
{`केनात्मनाशाय यदापनीताछिद्रं समासाद्य नरेन्द्रपत्नी'}


\twolineshloka
{यद्येव देवीं पृथिवीं प्रविष्टादिवं प्रपन्नाऽप्यथवा समुद्रम्}
{तस्या गमिष्यन्ति पदे हि पार्था-स्तथा हि संतप्यति धर्मराजः}


\twolineshloka
{को हीदृशानामरिमर्दनानांक्लेशक्षमाणामपराजितानाम्}
{प्राणैः समामिष्टतमां जिहीर्षे-दनुत्तमं रत्नमिव प्रमूढः}


\twolineshloka
{न बुध्यते नाथवतीमिहाद्यबहिश्चरं हृदयं पाण्डवानाम्}
{कस्याद्य कायं प्रतिभिद्य घोरामहीं प्रवेक्ष्यन्ति शिताः शराग्र्याः}


\twolineshloka
{मा त्वं शुचस्तां प्रति भीरु विद्धियथाऽद्य कृष्णा पुनरेष्यतीति}
{निहत्य सर्वान्द्विषतः समग्रा-न्पार्थाः समेष्यनत्यथ याज्ञसेन्या}


\twolineshloka
{अथाब्रवीच्चारुमुखं प्रसृज्यधात्रेयिका सारथिमिन्द्रसेनम्}
{जयद्रथेनापहृताप्रमथ्यपञ्चेन्द्रकल्पान्परिभूय कृष्णा}


\twolineshloka
{तिष्ठन्ति वर्त्मानि नवान्यमूनिवृक्षाश्च न म्लान्ति तथैव भग्नाः}
{आवर्तयध्वं ह्यनुयात शीघ्रंन दूरयातैव हि राजपुत्री}


\twolineshloka
{सन्नह्यध्वं सर्व एवेन्द्रकल्पामहान्ति चारूणि च दंशनानि}
{गृह्णीत चापानि महाधनानिशरांश्च शीघ्रं पदवीं व्रजध्वम्}


\twolineshloka
{पुरा हि निर्भर्त्सनदण्डमोहिताप्रमूढचित्ता वदनेन शुष्यता}
{ददाति कस्मैचिदनर्हते तनुंवराज्यपूर्णामिव भस्मनि स्रुचम्}


\twolineshloka
{पुरा तुषाग्नाविव हूयते हविःपुरा श्मशाने स्रगिवापविद्ध्यते}
{पुरा च सोमोऽध्वरगोऽवलिह्यतेशुना यथा विप्रजने प्रमोहिते}


\twolineshloka
{`पुरा हि पार्थाश्च दृतौ च कापिलीप्रसिच्छते क्षीरधारा यतध्वम्'}
{महत्यरण्ये मृगयां चरित्वापुरा सृगालो नलिनीं विगाहते}


\twolineshloka
{`पुरा हि मन्त्राहुतिपूजितायांहुताग्निवेद्यां बलिभुङ्गिलीयते}
{श्रुतिं च सम्यक्प्रसृतां महाध्वरेग्राम्यो जनो यद्वदसौ न नाशयेत्'}


\twolineshloka
{मा वः प्रियायाः सुनसं सुलोचनंचन्द्रप्रभाच्छं वदनं प्रसन्नम्}
{स्पृश्याच्छुभं कश्चिदकृत्यकारीश्वा वै पुरोडाशमिवाध्वरस्थम्}


% Check verse!
एतानि वर्त्मान्यनुयात शीघ्रंमा वः कालः क्षिप्रमिहात्यगाद्वै
\threelineshloka
{`शीघ्रं प्रधावध्वमितो नरेन्द्रायावन्न दूरं व्रजतीति पापः}
{प्रत्याहरध्वं द्विषतां सकाशा-ल्लक्ष्मीमिव स्वां दयितां नृसिंहाः' ॥युधिष्ठिर उवाच}
{}


\threelineshloka
{भद्रे प्रतिक्राम नियच्छ वाचंमाऽस्मत्सकाशे परुषाण्यवोचः}
{राजानो वा यदि वा राजपुत्राबलेन मत्ताः पञ्चतां प्राप्नुवन्ति ॥वैशंपायन उवाच}
{}


\twolineshloka
{एतावदुक्त्वा प्रययुर्हि शीघ्रंतान्येव वर्त्मान्यनुवर्तमानाः}
{मुहुर्मुहुर्व्यालवदुच्छ्वसन्तोज्यां विक्षिपन्तश्च महाधनुर्भ्यः}


\twolineshloka
{ततोऽपश्यंस्तस्य सैन्यस्य रेणु-मुद्धूतं वै वाजिस्वुरप्रणुन्नम्}
{पदातीनां मध्यगतं च धौम्यंविक्रोशन्तं भीम पार्थेत्यभीक्ष्णम्}


\twolineshloka
{ते सान्त्व्य धौम्यं परिदीनसत्वाःसुखं भवानेत्विति राजपुत्राः}
{श्येना यथैवामिषसंप्रयुक्ताजवेन तत्सैन्यमथाभ्यधावन्}


\twolineshloka
{तेषां महेन्द्रोपमविक्रमाणांसंरब्धानां धर्षणाद्याज्ञसेन्याः}
{क्रोधः प्रजज्वाल जयद्रथं चदृष्ट्वा प्रियां तस्य रथे स्थितां च}


\twolineshloka
{प्रचुक्रुशुश्चाप्यथ सिन्धुराजंवृकोदरश्चैव धनंजयश्च}
{यमौ च राजा च महाधनुर्धरा-स्ततो दिशः संमुमुहुः परेषाम्}


\chapter{अध्यायः २७१}
\twolineshloka
{वैशंपायन उवाच}
{}


\twolineshloka
{ततो घोरतरः शब्दो वने समभवत्तदा}
{भीमसेनार्जुनौ दृष्ट्वाक्षत्रियणाममर्षिणाम्}


\twolineshloka
{तेषां ध्वजाग्राण्यभिवीक्ष्य राजास्वयंदुरात्मा कुरुपुङ्गवानाम्}
{जयद्रथो याज्ञसेनीमुवाचरथे स्थितां भानुमतीं हतौजाः}


\threelineshloka
{आयान्तीमे पञ्चरथा महान्तोमन्ये च कृष्णे पतयस्तवैते}
{सा जानती ख्यापय नः सुकेशिपरंपरं पाण्डवानां रथस्थम् ॥द्रौपद्युवाच्}
{}


\twolineshloka
{किं ते ज्ञातैर्मूढ महाधनुर्धरै-रनायुष्यं कर्म कुत्वाऽतिघोरम्}
{एते वीराः पतयो मे समेतान वः शेषः कश्चिदिहास्ति युद्धे}


\twolineshloka
{आख्यातव्यं त्वेव सर्वं मुमूर्षो-र्मया तुभ्यं पृष्टया धर्म एषः}
{न मे व्यथा विद्यते त्वद्भयं वासंपश्यन्त्याः सानुजं धर्मराजम्}


\twolineshloka
{यस् ध्वजाग्रे नदतो मृदङ्गौनन्दोपनन्दौ मधुरौ सुयुक्तौ}
{एनं स्वधर्मार्थविनिश्चयज्ञंसदा जनाः कृत्यवन्तोऽनुयान्ति}


\threelineshloka
{य एष जाम्बूनदशुद्धगौरः}
{प्रचण्डघोणस्तनुरायताक्षः}
{एतं कुरुश्रेष्ठतमं वदन्तियुधिष्ठिरं धर्मसुतं पतिं मे}


\twolineshloka
{अप्येष शत्रोः शरणागतस्यदद्यात्प्राणान्धर्मचारी नृवीरः}
{परैह्येनं मूढ जवेन भूतयेत्वमात्मनः प्राञ्जलिर्न्यस्शस्त्रः}


\twolineshloka
{अथाप्येनं पश्यसि यं रथस्यंमहाभुजं सालमिव प्रवृद्धम्}
{संदष्टौष्ठं भ्रुकुटीसंहतभ्रुवंवृकोदरो नाम पतिर्ममैषः}


\twolineshloka
{आजानेया बलिनः साधुदान्तामहाबलाः शूरमुदावहन्ति}
{एतस्य कर्माण्यतिमानुषाणिभीमेति शब्दोऽस्य ततः पृथिव्याम्}


\twolineshloka
{नास्यापराद्धाः शेषमवाप्नुवन्तिनायं वैरं विस्मरते कदाचित्}
{वैरस्यान्तं संविधायोपयातिपश्चाच्छान्तिं न च तत्तप्यतीव}


\twolineshloka
{धनुर्धराग्र्यो धृतिमान्यशस्वीजितेन्द्रियो वृद्धसेवी नृवीरः}
{भ्राता च शिष्यश्च युधिष्ठिरस्यधनंजयो नाम पतिर्ममैषः}


\twolineshloka
{यो वै न कामान्न भयान्न लोभा-त्त्यजेद्धर्मं न नृशंसं च कुर्यात्}
{स एष वैश्वानरतुल्यतेजाःकुन्तीसुतः शत्रुसहः प्रमाथी}


\twolineshloka
{यः सर्वधर्मार्तविनिश्चयज्ञोभयार्तानां भयहर्ता मनीषी}
{`बन्धुप्रियः शस्त्रभृतां वरिष्ठोमहाहवेष्वप्रतिवार्यवीर्यः'}


\twolineshloka
{यस्योत्तमं रूपमाहुः पृथिव्यांयं पाण्डवाः परिरक्षन्ति सर्वे}
{प्राणैर्गरीयांसमनुव्रतं वैस एष वीरो नकुलः पतिर्मे}


\twolineshloka
{यः खङ्गयोधी लघुचित्रहस्तोमहांश्च धीमान्सहदेवोऽद्वितीयः}
{यस्याद्यकर्म द्रक्ष्यसे मूढसत्वशतक्रतोर्वा दैत्यसेनासु सङ्ख्ये}


\twolineshloka
{शूरः कृतास्त्रो मतिमान्मनस्वीप्रियंकरो धर्मसुतस्य राज्ञः}
{हुताशचन्द्रार्कसमानतेजाजघन्यजः पाण्डवानां प्रियश्च}


\twolineshloka
{बुद्ध्या समो यस् नरो न विद्यतेवक्ता तथा सत्सु विनिश्चयज्ञः}
{सएष शूरो नित्यममर्षणश्चधीमान्प्राज्ञः सहदेवः पतिर्मे}


\twolineshloka
{त्यजेत्प्राणान्प्रविशेद्धव्यवाहंन त्वेवैष व्याहरेद्धर्मबाह्यम्}
{सदा मनस्वी क्षत्रधर्मे रतश्चकुन्त्याः प्राणैरिष्टतमो नृवीरः}


\twolineshloka
{विशीर्यनतीं नावमिवार्णवान्तेरत्नाभिपूर्णां मकरस्य पृष्ठे}
{सेनां तवेमां हतसर्वयोधांविक्षोभितां द्रक्ष्यसि पाण्डुपुत्रैः}


\threelineshloka
{इत्येते वै कथिताः पाण्डुपुत्रायांस्त्वं मोहादवमत्य प्रवृत्तः}
{यद्येतेभ्यो मुच्यसे रिष्टदेहःपुनर्जन्म प्राप्स्यसे जीवितं च ॥वैशंपायन उवाच}
{}


\twolineshloka
{ततः पार्थाः पञ्चपञ्चेन्द्रकल्पा-स्त्यक्त्वा त्रस्तान्प्राञ्जलींस्तान्पदातीन्}
{यथाऽनीकं शरवर्षान्धकारंचक्रुः क्रुद्धाः सर्वतस्ते निगृह्य}


\chapter{अध्यायः २७२}
\twolineshloka
{वैशंपायन उवाच}
{}


\twolineshloka
{संतिष्ठध्वं प्रहरत तूर्णं विपरिधावत}
{इति स्म सैन्धवो राजा चोदयामास तान्नृपन्}


\twolineshloka
{ततो घोरतमः शब्दोरणे समभवत्तदा}
{भीमार्जुनयमान्दृष्ट्वा सैन्यानां सयुधिष्ठिरान्}


\twolineshloka
{शिबिसिंधुत्रिगर्तानां विषादश्चाप्यजायत}
{तान्दृष्ट्वापुरुषव्याघ्रान्व्याघ्रानिव बलोत्कटान्}


\twolineshloka
{हेमबिन्दुं महोत्सेधां सर्वशैक्यायसीं गदाम्}
{प्रगृह्याभ्यद्रवद्भीमः सैन्धवं कालचोदितम्}


\twolineshloka
{तदन्तरमथावृत्य कोटिकाश्योऽभ्यहारयत्}
{महता रथवंशेन परिवार्य वृकोधरम्}


\twolineshloka
{शक्तितोमरनाराचैर्वीरबाहुप्रचोदितैः}
{कीर्यमाणोपि बहुभिर्न स्म भीमोऽभ्यकम्पत}


\twolineshloka
{गजं तु सगजारोहं पदातींश्च चतुर्दश}
{जघान गदया भीमः सैन्धवध्वजंनीमुखे}


\twolineshloka
{पार्थः पञ्चशताञ्शूरान्पार्वतीयान्महारथान्}
{परीप्समानः सौवीरं जघान ध्वजिनीमुखे}


\twolineshloka
{राजा स्वयं सुवीराणां प्रवराणां प्रहारिणाम्}
{निमेषमात्रेण शतं जघान समरे तदा}


\twolineshloka
{ददृशे नकुलस्तत्ररथात्प्रस्कन्द्य खङ्गधृत्}
{शिरांसि पादरक्षाणां बीजवत्प्रवपन्मुहुः}


\twolineshloka
{सहदेवस्तु संयाय रथेन गजयोधिनः}
{पातयामास नाराचैर्द्रुमेभ्य इव बर्हिणः}


\twolineshloka
{ततस्त्रिगर्तः सधनुरवतीर्य महारथात्}
{गदया चतुरो वाहान्राज्ञस्तस् तदाऽवधीत्}


\twolineshloka
{तमथाभ्यागतं राजा पदातिं कुन्तिनन्दनः}
{अर्धचन्द्रेण बाणएन विव्याधोरसि धर्मराट्}


\twolineshloka
{स भिन्नहृदयो वीरो वक्राच्छोणितमुद्वमन्}
{पपाताभिमुखं प्राप्तश्चिन्नमूल इव द्रुमः}


\twolineshloka
{इन्द्रसेनद्वितीयस्तु रथात्प्रस्कन्द्य धर्मराट्}
{हताश्वः सहदेवस्य प्रतिपेदे महारथम्}


\twolineshloka
{नकुलं त्वभिसंवार्य क्षेमंकरमहामुखौ}
{उभावुभतस्तीक्ष्णैः शरवर्षैरवर्षताम्}


\twolineshloka
{तौ शरैरभिवर्षन्तौ जीमूताविव वार्षिकौ}
{एकैकेन विपाठेन जघ्ने माद्रवतीसुतः}


\twolineshloka
{त्रिगर्तराजः सुरथस्तस्याथ गजूर्गतः}
{रथमाक्षेपयामास गजेन गजयानवित्}


\twolineshloka
{नकुलस्त्वपभीस्तस्माद्रथाच्चर्मासिपाणिमान्}
{उद्धाम्य स्तानमास्थाय तस्थौ गिरिवाचलः}


\twolineshloka
{सुरथस्तं गजवरं वधाय नकुलस्य तु}
{प्रेषयामास सक्रोधमत्युच्छ्रितकरं ततः}


\twolineshloka
{नकुलस्तस्य नागस्य समीपपरिवर्तिनः}
{सविषाणं भुजं मूले खङ्गेन निरकृन्तत}


\twolineshloka
{स विनद्यमहानादं गजः किङ्किणिभूषणः}
{पतन्नवाक्शिरा भूमौ हस्त्यारोहमपोथयत्}


\twolineshloka
{स तत्कर्म महत्कृत्वा शूरो माद्रवतीसुतः}
{भीमसेनरथं प्राप्य शर्म लेभे महारथः}


\twolineshloka
{भीमस्त्वापततो राज्ञः कोटिकाश्यस् संगरे}
{सूतस्य नुदतो वाहान्क्षुरेणापाहरच्छिरः}


\twolineshloka
{न बुबोध हतं सूतं स राजा बाहुशालिना}
{तस्याश्वा व्यद्रवन्सङ्ख्ये हतमसूतास्ततस्ततः}


\twolineshloka
{विरथं हतसूतं तं भीमः प्रहरतांवरः}
{जघान तलयुक्तेन प्रासेनाभ्येत्य पाण्डवः}


\twolineshloka
{द्वादशानां तु सर्वेषां सौवीराणां धनंजयः}
{चकर्त निशितैर्भल्लैर्धनूंषि च शिरांसि च}


\twolineshloka
{शिबीनिक्ष्वाकुमुख्यांश्च त्रिगर्तान्सैन्धवानपि}
{जघानातिरथः सङ्ख्ये बाणगोचरमागतान्}


\twolineshloka
{सूदिताः प्रत्यदृश्यन्त बहवः सव्यसाचिना}
{सपताकाश्च मातङ्गाः सध्वजाश्च महारथाः}


\twolineshloka
{प्रच्छाद्य पृथिवीं तस्थुः सर्वमायोधनं प्रति}
{शरीराण्यशिरस्कानि विदेहानि शिरांसि च}


\twolineshloka
{श्वगृध्रकङ्ककाकोलभासगोमायुवायसाः}
{अतृप्यंस्तत्रवीराणां हतानां मासशोणितैः}


\twolineshloka
{`एवं तीक्ष्णशरज्वालैर्गाण्डीवानिलचोदितैः}
{सेनेन्धनं ददाहाशु सरोषः पार्थपावकः}


\twolineshloka
{चक्राणां पतितानां च युगानां च महीपते}
{तूणीराणआं पताकानां ध्वजानां च रथैः सह}


\twolineshloka
{ईषाणामनुकर्षाणां त्रिवेणूनां तथैव च}
{अक्षाणामथ योक्राणां प्रतोदानां च राशयः}


\twolineshloka
{शिरसां पतितानां च कुण्डलोष्णीषधारिणाम्}
{भुजानां मकुटानां च हाराणामङ्गदैः सह}


\twolineshloka
{छत्राणां व्यजनानां च चर्मणआं वर्मणां तथा}
{छिन्नानां कार्मुकाणां च पट्टसानां तथैव च}


\twolineshloka
{शक्तीनामथ खङ्गानां दण्डानां सह तेमरैः}
{राशयश्चात्रदृश्यन्ते तत्रतत्र विशांपते}


\twolineshloka
{पतितैश्चैव मातङ्गैः सयोधैः पर्वतोपमैः}
{हयैर्द्विधाकृतैः सार्धं सादिभिः सायुधैस्तथा}


\twolineshloka
{विप्रविद्धै रथैश्चैव निहतैश्चपदातिभिः}
{अगम्यरूपा पृथिवी मांसशोणितकर्दमा'}


\twolineshloka
{हतेषु तेषु वीरेषु सिन्धुराजो जयद्रथः}
{विमुच्य कृष्णां संत्रस्तः पलायनपरोऽभवत्}


\twolineshloka
{स तस्मिन्संकुले सैन्ये द्रौपदीमवतार्य ताम्}
{प्राणप्रेप्सुरुपाधावद्वनं तत्र नराधमः}


\twolineshloka
{द्रौपदीं धर्मराजस्तु दृष्ट्वा धौम्यपुरस्कृताम्}
{माद्रीपुत्रेण वीरेण रथमारोपयत्तदा}


\twolineshloka
{ततस्तद्विद्रुतं सैन्यमपयाते जयद्रथे}
{आदिश्यादिश्य नाराचैराजघान वृकोदरः}


\threelineshloka
{सव्यसाची तु तं दृष्ट्वा पलायन्तं जयद्रथम्}
{वारयामास निघ्नन्तं भीमं सैन्धवसैनिकान् ॥अर्जुन उवाच}
{}


\twolineshloka
{यस्यापचारात्प्राप्तोऽयमस्मान्क्लेशो दुरासदः}
{तमस्मिन्समरोद्देशे न पश्यामि जयद्रथम्}


\fourlineindentedshloka
{`मूढं नैकृतिकं दुष्टं द्रौपद्याः क्लेशकारिणम्'}
{तमेवान्विष भद्रं ते किं ते योधैर्निपातितैः}
{अनामिषमिदं कर्म कथं वा मन्यते भवान् ॥वैशंपायन उवाच}
{}


\twolineshloka
{इत्युक्तो भीमसेनस्तु गुडाकेशेन धीमता}
{युधिष्ठिरमभिप्रेत्य वाग्मी वचनमब्रवीत्}


\twolineshloka
{हतप्रवीरा रिपवो भूयिष्ठं विद्रुता दिशः}
{गृहीत्वा द्रौपदीं राजन्निवर्ततु भवानितः}


\twolineshloka
{यमाभ्यां सह राजेन्द्र धौम्येन च महात्मना}
{प्राप्याश्रमपदं राजन्द्रौपदीं परिसान्त्वय}


\threelineshloka
{न हि मे मोक्ष्यतेजीवन्मूढः सैन्धवको नृपः}
{पातालतलसंस्थोपि यदि शक्रोस्य सारथिः ॥युधिष्ठिर उवाच}
{}


\threelineshloka
{न हन्तव्यो महाबाहो दुरात्माऽपिस सैन्धवः}
{दुःशलामभिसंस्मृत्य गान्धारीं च यशस्विनीम् ॥वैशंपायन उवाच}
{}


\twolineshloka
{तच्छ्रुत्वा द्रौपदी भीममुवाच व्याकुलेन्द्रिया}
{कुपिता ह्रीमती प्राज्ञा पती भीमार्जुनावुभौ}


\twolineshloka
{कर्तव्यं चेत्प्रियं मह्यं वध्यः स पुरुषाधमः}
{सैन्धवापशदः पापो दुर्मतिः कुलपांसनः}


\twolineshloka
{भार्याभिहर्ता वैरी यो यश्च राज्यहरो रिपुः}
{याचमानोऽपिसंग्रामे न मोक्तव्यः कथंचन}


\twolineshloka
{इत्युक्तौ तौ नरव्याघ्रौ ययतुर्यत्र सैन्धवः}
{राजा निववृतेकृष्णामादाय सपुरोहितः}


\twolineshloka
{स प्रविश्याश्रमपदमपविद्धवृसीकटम्}
{मार्कण्डेयादिभिर्विप्रैरनुकीर्णं ददर्श ह}


\twolineshloka
{द्रौपदीमनुशोचद्भिर्ब्राह्मणैस्तैः समाहितैः}
{समेयाय महाप्राज्ञः सभार्य भ्रातृमध्यगः}


\twolineshloka
{ते स्म तं मुदिता दृष्ट्वा पुनरप्यागतं नृपम्}
{जित्वा तान्सिनधुसौवीरान्द्रौपदीं चाहृतां पुनः}


\twolineshloka
{स तैः परिवृतो राजातत्रैवोपविवेश ह}
{प्रविवेशाश्रमं कृष्णा यमाभ्यां सह भामिनी}


\twolineshloka
{भीमसेनार्जुनौ चापि श्रुत्वा क्रोशगतं रिपुम्}
{स्वयमश्वांस्तुदन्तौ तौ जवेनैवाभ्यधावताम्}


\twolineshloka
{इदमत्यद्भुतं चात्र चकारातिरथोऽर्जुनः}
{क्रोशमात्रगतानश्वान्सैन्धवस्य जघान यत्}


\twolineshloka
{स हि दिव्यास्त्रसंपन्नः कृच्छ्रकालेऽप्यसंभ्रमः}
{अकरोद्दुष्करं कर्म शरैरस्त्रानुमन्त्रितैः}


\twolineshloka
{ततोऽभ्यधावतां वीरावुभौ भीमधनंजयौ}
{हताश्वं सैन्धवं भीतमेकं व्याकुलचेतसम्}


\twolineshloka
{सैन्धवस्तु हतान्दृष्ट्वा तथाऽश्वान्स्वान्सुदुःखितः}
{`रथात्प्रस्कन्द्य पद्भ्यां वै पलायनपरोऽभवत्'}


\twolineshloka
{दृष्ट्वा रविक्रमकर्माणि कुर्वाणं च धनंजयम्}
{पलायनकृतोत्साहः प्राद्रवद्येन वै वनम्}


\twolineshloka
{सैन्धवं त्वमिसंप्रेक्ष्यपराक्रान्तं पलायने}
{अनुयाय महाबाहुः फल्गुनो वाक्यमब्रवीत्}


\twolineshloka
{अनन वीर्येण कथं स्त्रियं प्रार्थयसे बलात्}
{राजपुत्र निवर्तस्व न ते युक्तं पलायनम्}


\twolineshloka
{कथं ह्यनुचरान्हित्वा शत्रुमध्ये पलायसे}
{इत्युच्यमानः पार्थेन सैन्धवो न न्यवर्तत}


\twolineshloka
{तिष्ठतिष्ठेति तं भीमः सहसाऽभ्यद्रवद्बली}
{मा वधीरिति पार्थस्तं दयावान्प्रत्यभाषत}


\chapter{अध्यायः २७३}
\twolineshloka
{वैशंपायन उवाच}
{}


\twolineshloka
{जयद्रथस्तु संप्रेक्ष्यभ्रातरावुद्यतायुधौ}
{प्रादावत्तूर्णमव्यग्रो जीवितेप्सुः सुदुःखितः}


\twolineshloka
{`लताभिः संवृते कक्षे किरीटं रत्नभास्वरम्}
{अपविध्यप्रधावन्तं निलीयन्तं वनान्तरे}


\threelineshloka
{भीमसेनस्तु तं कक्षे लीयमानं भयाकुलम्}
{मार्गमाणोऽवतीर्याशु रथाद्रत्नविभूपितात्'}
{अभिद्रुत्य निजग्राहकेशपक्षे ह्यमर्षणः}


\twolineshloka
{समुद्यम्य च तं भीमो निष्पिपेप महीतले}
{गले गृहीत्वाराजानं पातयामास चैव ह}


\twolineshloka
{पुनः स जीवमानस्य तस्योत्पतितुमिच्छतः}
{पदा मूर्ध्नि महाबाहुः प्राहरद्विलपिष्यतः}


\twolineshloka
{तस्य जानू ददौ भमो जघ्ने चैनमरत्निना}
{स मोहमगमद्राजा प्रहारवरपीडितः}


\threelineshloka
{सरोषं भीमसेनं तु वारयामास फल्गुनः}
{दुःशलायाः कृतेराजा यत्त्वामाहेति कौरव ॥भीम उवाच}
{}


\twolineshloka
{नायं पापसमाचारो मत्तो जीवितुमर्हति}
{कृष्णायास्तदनर्हायाः परिक्लेष्टा नराधमः}


\twolineshloka
{किंनु शक्यं मया कर्तुं यद्राजा सततं घृणी}
{त्वं च बालिशया बुद्ध्या सदैवास्मान्प्रबाधसे}


\twolineshloka
{एवमुक्त्वा सटास्तस्य पञ्चचक्रे वृकोदरः}
{अर्धचन्द्रेण वाणेन किंचिदब्रुवतस्तदा}


\twolineshloka
{विकल्पयित्वा राजानं ततः प्राह वृकोदरः}
{जीवितुं चेच्छसे मूढ हेतुं मे गदतः शृणु}


\twolineshloka
{दासोस्मीति त्वया वाच्यं संसत्सु च सभासु च}
{एवं चेज्जीवितं दद्यामेष युद्धजितो विधिः}


\twolineshloka
{एवमस्त्विति तं राजा कृच्छ्रमाणो जयद्रथः}
{प्रोवाच पुरुषव्याघ्रं भीममाहवशोभिनम्}


\twolineshloka
{तत एनं विचेष्टन्तं वद्ध्वा पार्थो वृकोदरः}
{रथमारोपयामास विसंज्ञं पांसुकुण्ठितम्}


\twolineshloka
{ततस्तं रथमास्थाय भीमः पार्तानुगस्तदा}
{अग्र्यमाश्रममध्यस्थामभ्यगच्छद्युधिष्ठिरम्}


\twolineshloka
{दर्शयामास भीमस्तु तदवस्थं जयद्रथम्}
{तं राजा प्राहसद्दृष्ट्वा मुच्यतामिति चाब्रवीत्}


\twolineshloka
{राजानं चाब्रवीद्भीमो द्रौपद्यै कथयेति वै}
{दासभावं गतो ह्येष पाण्डूनां पापचेतनः}


\twolineshloka
{तमुवाच च ततो ज्येष्ठो भ्राता सप्रणयं वचः}
{मुञ्चेममधमाचारं प्रमाणा यदि ते वयम्}


\twolineshloka
{द्रौपदी चाब्रवीद्भीममभिप्रेक्ष्य युधिष्ठिरम्}
{दासोऽयंमुच्यतां रात्रस्त्वयापञ्चशिखः कृतः}


\twolineshloka
{`एवमुक्तः स भीमस्तु भ्रात्रा चैव च कृष्णया}
{मुमोच तं महापापं जयद्रथमचेतनम्'}


\twolineshloka
{स मुक्तोऽभ्येत्यराजानमभिवाद्य युधिष्ठिरम्}
{ववन्दे विह्वलोराजातांश्च वृद्धान्मुनींस्तदा}


\twolineshloka
{तमुवाच धृणी राजा धर्मपुत्रो युधिष्ठिरः}
{तथाजयद्रथं दृष्ट्वा गृहीतं सव्यसाचिना}


\twolineshloka
{अदासो गच्छ मुक्तोसि मैवं कार्षीः पुनः क्वचित्}
{स्त्रीकामुक धिगस्तु त्वां क्षुद्रः क्षत्रसहायवान्}


\twolineshloka
{एवंविधं हि क कुर्यात्त्वदनयः पुरुषाधमः}
{`कर्म धर्मविरुद्धं वै लोकदुष्टं च दुर्मते'}


\twolineshloka
{गतसत्त्वमिव ज्ञात्वा कर्तारमशुभस्य तम्}
{संप्रेक्ष्यभरतश्रेष्ठः कृपां चक्रे नराधिपः}


\twolineshloka
{धर्मे ते वर्धतां वुद्धिर्मा चाधर्मे मनः कृथाः}
{साश्च साथपादातः स्वस्ति गच्छ जयद्रथ}


\twolineshloka
{एवमुक्तस्तु सव्रीजं तूष्णीं किंचिदवाङ्युखः}
{जगाम राजा दुःखार्तो गङ्गाद्वाराय भारत}


\twolineshloka
{स देवं शरणं गत्वा विरूपाक्षमुमापतिम्}
{तपश्चचार विपुलं तस्य प्रीतो वृषध्वजः}


\twolineshloka
{बलिं स्वयं प्रत्यगृह्णात्प्रीयमाणस्त्रिलोचनः}
{वरं चास्मै ददौ देवः स जग्राह च तच्छृणु}


\twolineshloka
{समस्तान्सरथान्पञ्चजयेयं युधि पाण्डवान्}
{इति राजाऽब्रवीद्देवं नेति देवस्तमब्रवीत्}


\twolineshloka
{अजय्यांश्चाप्यवध्यांश्च वारयिष्यसि तान्युधि}
{ऋतेऽर्जुनं महाबाहुं नरं नाम सुरेश्वरम्}


\twolineshloka
{बदर्यां तप्ततपसं नारायणसहायकम्}
{अजितं सर्वलोकानां देवैरपि दुरासदम्}


\twolineshloka
{मया दत्तं पाशुपतं दिव्यमप्रतिमं शरम्}
{अवाप लोकपालेभ्यो वज्रादीन्स महाशरान्}


\twolineshloka
{देवदेवो ह्यनन्तात्मा विष्णुः सुरगुरुः प्रभुः}
{प्रधानपुरुषोऽव्यक्तोविश्वात्माविश्वमूर्तिमान्}


\threelineshloka
{युगान्तकाले संप्राप्ते कालाग्निर्दहते जगत्}
{सपर्वतार्णवद्वीपं सशैलवनकाननम्}
{निर्दहन्नागलोकांश्चपातालतलचारिणः}


\threelineshloka
{अथान्तरिक्षे सुमहन्नानावर्णाः पयोधराः}
{घोरस्वरा विनदिनस्तटिन्मालावलम्बिनः}
{समुत्तिष्ठन्दिशः सर्वाविवर्षन्तः समन्ततः}


\twolineshloka
{ततोऽग्नीं शमयामासुः संवर्ताग्निनियामकाः}
{अक्षमात्रैश्च धाराभिस्तिष्ठन्त्यापूर्य सर्वशः}


\threelineshloka
{एकार्णवे तदातस्मिन्नुपशान्तचराचरे}
{नष्टचन्द्रार्कपवने ग्रहनक्षत्रवर्जिते}
{चतुर्युगसहस्रान्ते सलिलेनाप्लुता मही}


\twolineshloka
{ततो नारायणाख्यस्तु सहस्राक्षः सहस्रपात्}
{सहस्रशीर्पा पुरुषः स्वप्नुकामस्त्वतीन्द्रियः}


\threelineshloka
{फणासहस्रविकटं शेषं पर्यङ्कभोगिनम्}
{सहस्रमिव तिग्मांशुसंघातममितद्युतिम्}
{कुन्देन्दुहारगोक्षीरमृणालकुमुदप्रभम्}


\twolineshloka
{तत्रासौ भगवान्देवः स्वपञ्जलनिधौ तदा}
{नैशेन तमसा व्याप्तां स्वां रात्रिंकुरुते विभुः}


\twolineshloka
{सत्वोद्रेकात्प्रबुद्धस्तु शून्यं लोकमपश्यत}
{इमं चोदाहरन्त्यत्रश्लोकं नारायणं प्रति}


\twolineshloka
{आपो नारास्तत्तनव इत्यपां नाम शुश्रुमः}
{अयनं तेन चैवास्ते तेन नारायणः स्मृतः}


\twolineshloka
{प्रध्यानसमकालं तुप्रजाहेतोः सनातनः}
{ध्यातमात्रे तु भगवन्नाभ्यां पद्मः समुत्थितः}


\twolineshloka
{ततश्चतुर्मुखो ब्रह्मा नाभिपद्माद्विनिःसृतः}
{तत्रोपविष्टः सहसा ब्रह्मा लोकपितामहः}


\twolineshloka
{शून्यं दृष्ट्वा जगत्कृत्स्नं मानसानात्मनः समान्}
{ततो मरीचिप्रमुखान्महर्षीनसृजन्नव}


\twolineshloka
{तेऽसृजन्सर्वभूतानि त्रसानि स्थावराणि च}
{यक्षराक्षसभूतानि पिशाचोरगमानुषान्}


\twolineshloka
{सृजते ब्रह्ममूर्तिस्तु रक्षते पौरुषी तनुः}
{रौद्री भावेन शमयेत्तिस्रोऽवस्थाः प्रजापतेः}


\twolineshloka
{न श्रुतं ते सिन्धुपते विष्णोरद्भुतकर्मणः}
{कथ्यमानानि मुनिभिर्ब्राह्मणैर्वेदपारगैः}


\twolineshloka
{जलेन समनुप्राप्ते सर्वतः पृथिवीतले}
{तदा चैकार्णवे तस्मिननेकाकाशे प्रभुश्चरन्}


\twolineshloka
{निशायामिव खद्योतः प्रावृट््काले समन्ततः}
{प्रतिष्ठानाय पृथिवीं मार्गमाणस्तदाऽभवत्}


\twolineshloka
{जले निमग्नां गां दृष्ट्वाचोद्धर्तुं मनसेच्छति}
{किंनु रूपमहं कृत्वासलिलादुद्धरे महीम्}


\twolineshloka
{एवं संचिन्त्य मनसा दृष्ट्वा दिव्येन चक्षुषा}
{जलक्रीडाभिरुचितं वाराहं रूपमस्मरत्}


\twolineshloka
{कृत्वा वराहवपुपं वाङ्मयं वेदसंमितम्}
{दशयोजनविस्तीर्णमायतं शतयोजनम्}


\twolineshloka
{महापर्वतवर्ष्माभं तीक्ष्णदंष्ट्रंप्रदीप्तिमत्}
{महामेघौघनिर्घोपं नीलजीमूतसन्निभम्}


\twolineshloka
{भूत्वा यज्ञवराहो वै अपः संप्राविशत्प्रभुः}
{दंष्ट्रेणैकेन चोद्धृत्यस्वे स्थाने न्यविशन्महीम्}


\twolineshloka
{पुनरेव महाबाहुरपूर्वां तनुमाश्रितः}
{नरस्य कृत्वाऽर्धतनुं सिंहस्यार्धतनुं प्रभुः}


\twolineshloka
{दैत्येन्द्रस्य सभां गत्वापाणिं संस्पृश्यपाणिना}
{दैत्यानामादिपुरुषः सुरारिर्दितिनन्दनः}


\twolineshloka
{दृष्ट्वा चापूर्ववपुषं क्रोधात्संरक्तलोचनः}
{शूलोद्यतकरः स्रग्वी हिरण्यकशिपुस्तदा}


\twolineshloka
{मेघस्तनितनिर्घोषो नीलाभ्रचयसन्निभः}
{देवारिर्दितिजो वीरो नृसिंहं समुपाद्रवत्}


\twolineshloka
{समुत्पत्य ततस्तीक्ष्णैर्मृगेन्द्रेण बलीयसा}
{नारसिंहेन वपुषादारितः करजैर्भृशम्}


\threelineshloka
{एवं निहत्य भगवान्दैत्येनद््रं रिपुघातिनम्}
{भूयोऽन्यः पुण्डरीकाक्षः प्रभुर्लोकहिताय च}
{कश्यपस्यात्मजः श्रीमानदित्या गर्भधारितः}


% Check verse!
पूर्णे वर्षसहस्रे तु प्रसूत भर्गमुत्तमम्
\threelineshloka
{दुर्दिनाम्भोदसदृशो दीप्ताक्षो वामनाकृतिः}
{दण्डी कमण्डलुधरः श्रीवत्सोरसि भूषितः}
{जटी यज्ञोपवीती च भगवान्बालरूपधृक्}


\twolineshloka
{यज्ञवाटं गतः श्रीमान्दानवेन्द्रस्य वै तदा}
{बृहस्पतिसहायोऽसौ प्रविष्टो बलिनो मखे}


\twolineshloka
{तं दृष्ट्वावामनतनुं प्रहृष्टो बलिरब्रवीत्}
{प्रीतोस्मि दर्शने विप्र ब्रूहि त्वं किं ददानिते}


% Check verse!
एवमुक्तस्तु बलिना वामनः प्रत्युवाच ह
\twolineshloka
{स्वस्तीत्युक्त्वा बलिं देवः स्मयमानोऽभ्यभाषत}
{मेदिनीं दानवपते देहि मे विक्रमत्रयम्}


\twolineshloka
{बलिर्ददौ प्रसन्नात्मा विप्रायामिततेजसे}
{ततो दिव्याद्भुततमं रूपं विक्रमतोहरेः}


\twolineshloka
{विक्रमैस्त्रिभिरक्षोभ्यो जहाराशु स मेदिनीम्}
{ददौ शक्राय च महीं विष्णुर्देवः सनातनः}


\twolineshloka
{एष ते वामनो नाम प्रादुर्भावः प्रकीर्तितः}
{तेन देवाः प्रादुरासन्वैष्णवं चोच्यते जगत्}


\threelineshloka
{असतां निग्रहार्थाय धर्मसंरक्षणाय च}
{अवतीर्णो मनुष्याणामजायत यदुक्षये}
{यं देवंविदुषोगान्ति तस्य कर्माणि सैन्दव}


\twolineshloka
{अनाद्यन्तमजं देवं प्रभुंलोकनमस्कृतम्}
{यं देवं विदुषोगान्तितस् कर्माणि सैन्धव}


\twolineshloka
{यमाहुरजितंकृष्णं शङ्खचक्रगदाधरम्}
{श्रीवत्सधारिणं देवं पीतकौशेयवाससम्}


% Check verse!
प्रधानं सोऽस्त्रविदुषां तेन कृष्णन रक्ष्यते
\twolineshloka
{सहायः पुण्डरीकाक्षः श्रीमानतुलविक्रमः}
{समानस्यन्दने पार्थमास्थाय परवीरहा}


\twolineshloka
{न शक्यते तेन जेतुं त्रिदशैरपि दुःसहः}
{कः पुनर्मानुषो भावो रणे पार्थं विजेष्यति}


\twolineshloka
{तमेकं वर्जयित्वातु सर्वंयौधिष्ठिरं बलम्}
{चतुरः पाण्डवात्राजन्दिनैकं जेप्यसे रिपून्}


\threelineshloka
{`तस्मात्त्वंपार्थरहितान्पाण्डवान्वारयिष्यसि}
{एतद्धि पुरुषव्याघ्र मया दत्तो वरस्तव' ॥वैशंपायन उवाच}
{}


\twolineshloka
{इत्येवमुक्त्वा नृपतिं सर्वपापहरो हरः}
{उमापतिः पशुपतिर्यज्ञहा त्रिपुरार्दनः}


\twolineshloka
{वामनैर्विकटैः कुब्जैरुग्रश्रवणदर्शनैः}
{वृतः पारिषदैर्घोरैर्नानाप्रहरणोद्यतैः}


\twolineshloka
{त्र्यम्बको राजशार्दूल भगनेत्रनिपातनः}
{उमासहायो भगवांस्तत्रैवान्तरधीयत}


\twolineshloka
{एवमुक्तस्तु नृपतिः स्वमेव भवनं ययौ}
{पाण्डवाश्च वने तस्मिन्न्यवसन्काम्यके तथा}


\chapter{अध्यायः २७४}
\twolineshloka
{जनमेजय उवाच}
{}


\threelineshloka
{एवं हृतायां कृष्णायां प्राप्य क्लेशमनुत्तमम्}
{अत ऊर्द्वं नरव्याघ्राः किमकुर्वत पाण्डवाः ॥वैशंपायन उवाच}
{}


\twolineshloka
{एवं कृष्णां मोक्षयित्वाविनिर्जित्य जयद्रथम्}
{आसांचक्रे मुनिगणैर्धर्मराजो युधिष्टिरः}


\twolineshloka
{तेषां मध्येमहर्षीणां शृण्वतामनुशोचताम्}
{मार्कण्डेयमिदं वाक्यमब्रवीत्पाण्डुनन्दनः}


\twolineshloka
{भगवन्देवर्षीणां त्वं ख्यातो भूतभविष्यवित्}
{संशयं परिपृच्छमि च्छिन्धि मे हृदि संस्थितम्}


\twolineshloka
{द्रुपदस्य सुता ह्येषा वेदिमध्यात्समुत्थिता}
{अयोनिजा महाभागास्नुषा पाण्डोर्महात्मनः}


\twolineshloka
{मन्ये कालश्च भगवान्दैवं च दुरतिक्रमम्}
{भवितव्यं च भूतानां यस् नास्ति व्यतिक्रमः}


\twolineshloka
{कथं हि पत्नीमस्माकं धर्मज्ञां धर्मचारिणीम्}
{संस्पृशेदीदृशो भावः शुचिं स्तैन्यमिवानृतम्}


\twolineshloka
{न हि पापं कृतंकिंचित्कर्म वा निन्दितं क्वचित्}
{द्रौपद्या ब्राह्मणेष्वेव धर्मः सुचरितो महान्}


\twolineshloka
{तां जहार बलाद्राजा मूढबुद्धिर्जयद्रथः}
{तस्याः संहरणात्पापः शिरसः केशवापनम्}


\twolineshloka
{पराजयं च संग्रामे ससहाः समाप्तवान्}
{प्रत्याहृता तथाऽस्माभिर्हत्वा तत्सैन्धवं बलम्}


\twolineshloka
{तद्दारहरणं प्राप्तमस्माभिरवितर्कितम्}
{दुःखश्चायं वने वासो मृगयायां च जीविका}


\twolineshloka
{हिंसा च मृगजातीनां वनौकोभिर्वनौकसाम्}
{ज्ञातिभिर्विप्रवासश्च मिथ्याव्यवसितैरियम्}


\twolineshloka
{अस्ति नूनं मया कश्चिदल्पभाग्यतरो नरः}
{भवता दृष्टपूर्वो वा श्रुतपूर्वोऽपि वा क्वचित्}


\chapter{अध्यायः २७५}
\twolineshloka
{मार्कण्डेय उवाच}
{}


\twolineshloka
{प्राप्तमप्रतिमं दुःखं रामेण भरतर्षभ}
{रक्षसा जानकी तस्य हृता भार्या बलीयसा}


\twolineshloka
{आश्रमाद्राक्षसेन्द्रेण रावणेन दुरात्मना}
{मायामास्थाय तरसा हत्वा गृध्रं जटायुषम्}


\threelineshloka
{प्रत्याजहार तां रामः सुग्रीवबलमाश्रितः}
{बद्ध्वा सेतुं समुद्रस्य दग्ध्वा लङ्कां शितैः शरैः ॥युधिष्ठिर उवाच}
{}


\twolineshloka
{कस्मिन्नामः कुले जातः किंवीर्यः किंपराक्रमः}
{रावणः कस्य पुत्रो वा किं वैरं तस् तेन ह}


\fourlineindentedshloka
{एतन्मे भगवन्सर्वं सम्यगाख्यातुमर्हसि}
{`त्वया प्रत्यक्षोदृष्टं यथासर्वमशेषतः}
{'श्रोतुमिच्छामि चरितं रामस्याक्लिष्टकर्मणः ॥मार्कण्डेय उवाच}
{}


\twolineshloka
{अजोनामाभवद्राजा महानिक्ष्वाकुवंशजः}
{तस् पुत्रो दशरथःशश्वत्स्वाध्यायवाञ्छुचिः}


\twolineshloka
{अभवंस्तस्य चत्वारः पुत्रा धर्मार्थकोविदाः}
{रामलक्ष्मणशत्रुघ्ना भरतश्च महाबलः}


\twolineshloka
{रामस्य माता कौसल्या कैकेयी भरतस्य तु}
{सुतौ लक्ष्मणशत्रुघ्नौ सुमित्रायाः परंतपौ}


\twolineshloka
{विदेहराजो जनकः सीता तस्यात्मजा विमो}
{यां चकार स्वयं त्वष्टा रामस्य महिषीं प्रियाम्}


\twolineshloka
{एतद्रामस्य ते जन्म सीतायाश्च प्रकीर्तितम्}
{रावणस्यापि ते जन्म व्याख्यास्यामि रजनेश्वर}


\twolineshloka
{पितामहो रावणस्य साक्षाद्देवः प्रजापतिः}
{स्वयंभूः सर्वलोकानां प्रभुः स्रष्टा महातपाः}


\twolineshloka
{पुलस्त्यो नाम तस्यासीन्मानसो दयितः सुतः}
{तस्य वैश्रवणो नाम गवि पुत्रोऽभवत्प्रभुः}


\twolineshloka
{पितरं स समुत्सृज्य पितामहमुपस्थितः}
{तस्य कोपात्पिता राजन्ससर्जात्मानमात्मना}


\twolineshloka
{स जज्ञे विश्रवा नाम तस्यात्मार्धेन वै द्विजः}
{प्रतीकाराय सक्रोधस्ततो वैश्रवणस्य वै}


\twolineshloka
{पितामहस्तुप्रीतात्मा ददौ वैश्रवणस् ह}
{अमरत्वं धनेशत्वं लोकपालत्वमेव च}


\twolineshloka
{ईशानन तथा सख्यं पुत्रं च नलकूवरम्}
{राजधानीनिवेसं च लङ्कांरक्षोगणान्विताम्}


\twolineshloka
{विमानं पुष्पकं नाम कामगं च ददौ प्रभुः}
{यक्षाणामाधिपत्यंच राजराजत्वमेव च}


\chapter{अध्यायः २७६}
\twolineshloka
{मार्कण्डेय उवाच}
{}


\threelineshloka
{पुलस्त्यस् तु यः क्रोधादर्धदेहोऽभवन्मुनिः}
{विश्रवानाम सक्रोधं पितरं राक्षसेश्वरः}
{}


\twolineshloka
{बुबुधे तं तु सक्रोधं पितरं राक्षसेश्वरः}
{कुबेरस्तत्प्रसादार्थं यतते स्म सदा नृप}


\twolineshloka
{स राजराजो लङ्कायां न्यवसन्नरवाहनः}
{राक्षसीः प्रददौ तिस्रः पितुर्वै परिचारिकाः}


\twolineshloka
{ताः सदा तं महात्मानं संतोषयितुमुद्यताः}
{ऋषिं भरतशार्दूल नृत्यगीतविशारदाः}


\twolineshloka
{पुष्पोत्कटा च राका च मालिनी च विशांपते}
{अन्योन्यस्पर्धयाराजञ्श्रेयस्कामाः सुमध्यमाः}


\twolineshloka
{स तासां भगवांस्तुष्टो महात्मा प्रददौ वरान्}
{लोकपालोपमान्पुत्रानकैकस्या यथेप्सितान्}


\twolineshloka
{पुष्पोत्कटायां जज्ञाते द्वौ पुत्रौ राक्षसेश्वरौ}
{कुम्भकर्णदशग्रीवौ बलेनाप्रतिमौ भुवि}


\twolineshloka
{मालिन जनयामास पुत्रमेकं विभीषणम्}
{राकार्या मिथुनं जज्ञे खरः शूर्पणखा तथा}


\twolineshloka
{विभीषणस्तु रूपेण सर्वेभ्योऽभ्यधिकोऽभवत्}
{स बभूव महाभागो धर्मगोप्ता क्रियारतिः}


\twolineshloka
{दशग्रीवस्तु सर्वेषां श्रेष्ठो राक्षसपुङ्गवः}
{महोत्साहो महावीर्यो महासत्वपराक्रमः}


\twolineshloka
{कुम्भकर्णओ बलेनासीत्सर्वेभ्योऽभ्यधिको युधि}
{मायावी रणशौण्डश्चरौद्रश्च रजनीचरः}


\twolineshloka
{खरो धनुषि विक्रान्तो ब्रह्मद्विट् पिशिताशनः}
{सिद्धविघ्नकरी चापि रौद्री शूर्पणखा तदा}


\twolineshloka
{सर्वे वेदविदः शूराः सर्वेसुचरितव्रताः}
{ऊषुः पित्रा सह रता गन्धमादनपर्वते}


\twolineshloka
{ततो वैश्रवणं तत्र ददृशुर्नरवाहनम्}
{पित्रा सार्धं समासीनमृद्ध्या परमया युतम्}


\twolineshloka
{जातामर्षास्ततस्ते तु तपसे धृतनिश्चयाः}
{ब्रह्माणं तोषयामासुर्घोरेण तपसा तदा}


\twolineshloka
{अतिष्ठदेकपादेन सहस्रं परिवत्सरान्}
{वायुभक्षो दशग्रीवः पञ्चाग्निः सुसमाहितः}


\twolineshloka
{अधःशायी कुम्भकर्णो यताहारो यतव्रतः}
{विभीषणः शीर्णपर्णमेकमभ्यवहारयन्}


\twolineshloka
{उपवासरतिर्धीमान्सदा जप्यपरायणः}
{तमेव कालमातिष्ठत्तीव्रं तप उदारधीः}


\twolineshloka
{स्वरः शूर्पणखा चैव तेषां वै तप्यतां तपः}
{परिचर्यां च रक्षां च चक्रतुर्हष्टमानसौ}


\twolineshloka
{पूर्णे वर्षसहस्रेतु शिरश्छित्त्वा दशाननः}
{जुहोत्यग्नौ दुराधर्षस्तेनातुष्यज्जगत्प्रभुः}


\threelineshloka
{ततो ब्रह्मा स्वयं गत्वा तपसस्तान्न्यवारयत्}
{प्रलोभ्यवरदानेन सर्वानेवपृथक्पृथक् ॥ब्राह्मोवाच}
{}


\twolineshloka
{प्रीतोऽस्मि वो निवर्तध्वं वरान्वृणुत पुत्रकाः}
{यद्यदिष्टमृते त्वेकममरत्वं तथाऽस्तु तत्}


\twolineshloka
{यद्यदग्नौ हुतं सर्वं शिरस्ते महदीप्सया}
{तथैव तानि ते देहे भविष्यन्ति यथेप्सया}


\threelineshloka
{वैरूप्यं च न ते देहे कामरूपधरस्तथा}
{भविष्यसि रणेऽरीणां विजेता न च संशयः ॥रावण उवाच}
{}


\threelineshloka
{गन्धर्वदेवासुरतो यक्षराक्षसतस्तथा}
{सर्पकिंनरभूतेभ्यो न मे भूयात्पराभवः ॥ब्रह्मोवाच}
{}


\threelineshloka
{य एते कीर्तिताः सर्वे न तेभ्योस्ति भयं रतव}
{ऋते मनुष्याद्भद्रं ते तथा तद्विहितं मया ॥मार्कण्डेय उवाच}
{}


\twolineshloka
{एवमुक्तो दशग्रीवस्तुष्टः समभवत्तदा}
{अवमेने हि दुर्बुद्धिर्मनुष्यान्पुरुषादकः}


\threelineshloka
{कुम्भकर्णमथोवाच तथैव प्रपितामहः}
{`वरं वृणीष्व भद्रं ते प्रीतोस्मीति पुनःपुनः'}
{स वव्रे महतीं निद्रां तमसा ग्रस्तचेतनः}


\threelineshloka
{तथाभविष्यतीत्युक्त्वा विभीषणमुवाच ह}
{वरं वृणीष्व पुत्र त्वं प्रीतोऽस्मीति पुनःपुनः ॥विभीषण उवाच}
{}


\threelineshloka
{परमापद्गतस्यापि नाधर्मे मे मतिर्भवेत्}
{अशिक्षितं च भगवन्ब्रह्मास्त्रं प्रतिभातु मे ॥ब्रह्मोवाच}
{}


\threelineshloka
{यस्माद्राक्षसयोनौ ते जातस्यामित्रकर्शन}
{नाधर्मे धीयते बुद्धिरमरत्वं ददानि ते ॥मार्कण्डेय उवाच}
{}


\twolineshloka
{राक्षसस्तु वरंलब्ध्वा दशग्रीवो विशांपते}
{लङ्कायाश्च्यावयामास युधि जित्वा धनेश्वरम्}


\twolineshloka
{हित्वास भगवाँल्लङ्कामाविशद्गन्धमादनम्}
{गन्धर्वयक्षानुगतो रक्षःकिंपुरुषैः सह}


\twolineshloka
{विमानं पुष्पकं तस्य जहाराक्रम्य रावणः}
{शशाप तं वैश्रवणो न त्वामेतद्वहिष्यति}


\twolineshloka
{यस्तु त्वां समरे हन्ता तमेवैतद्वहिष्यति}
{अवमत्य गुरुं मां च क्षिप्रं त्वंनभविष्यसि}


\twolineshloka
{विभीषणस्तु धर्मात्मा सतां मार्गमनुस्मरन्}
{अन्वगच्छन्महाराज श्रिया परमया युतः}


\twolineshloka
{तस्मै स भगवांस्तुष्टो भ्राता भ्रात्रे धनेश्वरः}
{सैनापत्यं ददौ धीमान्यक्षराक्षससेनयोः}


\twolineshloka
{राक्षसाः पुरुषादाश्च पिशाचाश्च महाबलाः}
{सर्वे समेत्य राजानमभ्यषिञ्चन्दशाननम्}


\twolineshloka
{दशग्रीवश्चदैत्यानां दानवानां बलोत्कटः}
{आक्रम्य रत्नान्यहरत्कामरूपी विहंगम}


\twolineshloka
{रावयामास लोकान्यत्तस्माद्रावण उच्यते}
{दशग्रीवः कामबलो देवानां भयमादधत्}


\chapter{अध्यायः २७७}
\twolineshloka
{मार्कण्डेय उवाच}
{}


\threelineshloka
{ततो ब्रह्मर्षयः सर्वे सिद्धा देवर्षयस्तथा}
{हव्यवाहं पुरस्कृत्य ब्रह्माणं शरणं गताः ॥अग्निरुवाच}
{}


\twolineshloka
{योसौ विश्रवसः पुत्रो दशग्रीवो महाबलः}
{अवध्यो वरदानेन कृतो भगवता पुरा}


\threelineshloka
{स बाधते प्रजाः सर्वा विप्रकारैर्महाबलः}
{ततो नस्त्रातु भगवान्नान्यस्त्राता हि विद्यते ॥ब्रह्मोवाच}
{}


\twolineshloka
{न स देवासुरैः शक्यो युद्धे जेतुं विभावसो}
{विहितं तत्रयत्कार्यमभितस्तस्य निग्रहः}


\threelineshloka
{तदर्थमवतीर्णोऽसौ मन्नियोगाच्चतुर्भुजः}
{विष्णुः प्रहरतां श्रेष्ठः स तत्कर्म करिष्यति ॥मार्कण्डेय उवाच}
{}


\twolineshloka
{पितामहस्ततस्तेषां संनिधौ शक्रमब्रवीत्}
{सर्वैर्देवगणैः सार्धं संभव त्वं महीतले}


\twolineshloka
{विष्णोः सहायानृक्षीषु वानरीषु च सर्वशः}
{जनयध्वं सुतान्वीरान्कामरूपबलान्वितान्}


\twolineshloka
{`ते यथोक्ता भगवता तत्प्रतिश्रुत्य शासनम्}
{ससृजुर्देवगन्धर्वाः पुत्रान्वानररूपिणः'}


\twolineshloka
{ततो भागानुभागेन देवगन्धर्वदानवाः}
{अवतर्तुं महीं सर्वे मन्त्रयामासुरञ्जसा}


\threelineshloka
{`अवतेरुर्महीं स्वर्गादंशैश्च सहिताः सुराः}
{ऋषयश्च महात्मानः सिद्धाश्च सह किन्नरैः}
{चारणाश्चासृजन्घोरान्वानरान्वनचारिणः}


\twolineshloka
{यस्य देवस्य यद्रूपं वेषस्तेजश्च यद्विधम्}
{अजायन्त समास्तेन तस्य तस्य सुतास्तदा'}


\twolineshloka
{तेषां समक्षं गन्धर्वी दुन्दुभीं नाम नामतः}
{शशास वरदो देवो गच्छ कार्यार्थसिद्धये}


\twolineshloka
{पितामहवचः श्रुत्वा गन्धर्वी दुन्दुभी ततः}
{मन्थरा मानुषे लोके कुब्जा समभवत्तदा}


\twolineshloka
{शक्रप्रभृतयश्चैव सर्वे ते सुरसत्तमाः}
{वानरर्क्षवरस्त्रीषु जनयामासुरात्मजान्}


\twolineshloka
{तेऽन्ववर्तन्पितॄन्सर्वे यशसा च बलेन च}
{भेत्तारो गिरिशृङ्गाणां सालतालशिलायुधाः}


\twolineshloka
{वज्चसंहननाः सर्वेसर्वेऽमोघवलास्तथा}
{कामवीर्यबलाश्चैवसर्वे बुद्धिविशारदाः}


\twolineshloka
{नागायुतसमप्राणा वायुवेगसमा जवे}
{यथेच्छविनिपाताश्च केचिदत्र वनौकसः}


\twolineshloka
{एवं विधाय तत्सर्वं भगवाँल्लोकभावनः}
{मन्थरां बोधयामास यद्यत्कार्यं त्वया तथा}


\twolineshloka
{सा तद्वच समाज्ञाय तथा चक्रे मनोजवा}
{इतश्चेतश्च गच्छन्ती वैरसंधुक्षणे रता}


\chapter{अध्यायः २७८}
\twolineshloka
{युधिष्ठिर उवाच}
{}


\twolineshloka
{उक्तं भगवता जन्म रामादीनां पृथक्पृथक्}
{प्रस्थानकारणं ब्रह्मञ्श्रोतुमिच्छामि कथ्यताम्}


\threelineshloka
{कथं दाशरथी वीरौ भ्रातरौ रामलक्ष्मणौ}
{प्रस्तापितौ वने ब्रह्मन्मैथिली च यशस्विनी ॥मार्कण्डेय उवाच}
{}


\twolineshloka
{जातपुत्रो दशरथः प्रीतिमानभवन्नृप}
{क्रियारतिर्धर्मरतः सततं वृद्धसेविता}


\twolineshloka
{क्रमेण चास्य ते पुत्रा व्यवर्धन्त महौजसः}
{वेदेषु सरहस्येषु धनुर्वेदेषु पारगाः}


\twolineshloka
{चरितब्रह्मचर्यास्ते कृतदाराश्च पार्तिव}
{दृष्ट्वा रामं दशरथः प्रीतिमानभवत्सुखी}


\twolineshloka
{ज्येष्ठो रामोऽभवत्तेषां रमयामास हि प्रजाः}
{मनोहरतया धीमान्पितुर्हृदयनन्दनः}


\twolineshloka
{ततः स राजा मतिमान्मत्वाऽऽत्मानं वयोधिकम्}
{मन्त्रयामास सचिवैर्मन्त्रज्ञैश्च पुरोहितैः}


\twolineshloka
{अभिषेकाय रामस्य यावैराज्येन भारत}
{प्राप्तकालं च ते सर्वे मेनिरे मन्त्रिसत्तमाः}


\twolineshloka
{लोहिताक्षं महाबाहुं मत्तमातङ्गगामिनम्}
{कम्बुग्रीवं महोरस्कं नीलकुञ्चितमूर्धजम्}


\twolineshloka
{दीप्यमानं श्रिया वीरं शक्रादनवरं बले}
{पारगं सर्वधर्माणां बृहस्पतिसमं मतौ}


\twolineshloka
{सर्वानुरक्तप्रकृतिं सर्वविद्याविशारदम्}
{जितेन्द्रियममित्राणामपि दृष्टिमनोहरम्}


\twolineshloka
{नियन्तारमसाधूनां गोप्तारं धर्मचारिणाम्}
{धृतिमन्तमनाधृष्यं जेतारमपराजितम्}


\twolineshloka
{पुत्रं राजा दशरथः कौसल्यानन्दवर्धनम्}
{संदृश्यपरमां प्रीतिमगच्छत्कुलनन्दनम्}


\twolineshloka
{चिन्तयंश्च महातेजा गुणान्रामस्य वीर्यवान्}
{अभ्यभाषत भद्रं ते प्रीयमाणः पुरोहितम्}


\twolineshloka
{अद् पुष्यो निशि ब्रह्मन्पुण्यं योगमुपैष्यति}
{संभाराः संभ्रियन्तां मे रामश्चोपनिमन्त्र्यताम्}


\twolineshloka
{`श्व एवपुष्यो भविता यत्ररामः सुतो मया}
{यौवराज्येऽभिषेक्तव्यः पौरेषु सहमन्त्रिभिः'}


\twolineshloka
{इति तद्राजवचनं प्रतिश्रुत्याथ मन्थरा}
{कैकेयीमभिगम्येदं काले वचनमब्रवीत्}


\twolineshloka
{अद्य कैकेयि दौर्भाग्यं राज्ञा ते ख्यापितं महत्}
{आशीविषस्त्वां संक्रुद्धश्छन्नो दशति दुर्भगे}


\twolineshloka
{सुभगा खलु कौसल्या यस्याः पुत्रोऽभिषेक्ष्यते}
{कुतो हि तव सौभाग्यं यस्याः पुत्रो न राज्यभाक्}


\twolineshloka
{सा तद्वचनमाज्ञाय सर्वाभरणभूषिता}
{वेदी विलग्नमध्येन बिभ्रती रूपमुत्तमम्}


\twolineshloka
{वविक्ते पतिमासाद्य हसन्तीव शुचिस्मिता}
{राजानं तर्जयन्तीव मधुरं वाक्यमब्रवीत्}


\fourlineindentedshloka
{सत्यप्रतिज्ञ यन्मे त्वं काममेकं विसृष्टवान्}
{उपाकुरुष्व तद्राजंस्तस्मान्मुञ्चस्व संकटात्}
{`तदद्य कुरु सत्यं मे वरं वरद भूपते' ॥राजोवाच}
{}


\twolineshloka
{वरं ददानि ते हन्त तद्गृहाण यदिच्छसि}
{अवध्यो वध्यतां कोद्य वध्यः कोऽद्य विमुच्यतां}


\twolineshloka
{धनं ददानि कस्याद् ह्रियतां कस्यरवापुनः}
{ब्राह्मणस्वादिहान्यत्रयत्किंचिद्वित्तमस्ति मे}


\twolineshloka
{पृथिव्यां राजराजोस्मि चातुर्वर्ण्यस् रिता}
{यस्तेऽभिलपितः कामो ब्रूहि कल्याणि माचिरं}


\twolineshloka
{सातद्वचनमाज्ञाय परिगृह्य नराधिपम्}
{आत्मनो बलमाज्ञाय तत एनमुवाच ह}


\twolineshloka
{आभिषेचनिकं यत्ते रामार्थमुपकल्पितम्}
{भरतस्तदवाप्नोतु वनं गच्छतु राघवः}


\twolineshloka
{`नव पञ्च च वर्षाणि दण्डकारण्यमाश्रितः}
{चीराजिनजटाधारी रामो भवतु तापसः'}


\twolineshloka
{स तं राजा वरं श्रुत्वा विप्रियं दारुणोदयम्}
{दुःखार्तो भरतश्रेष्ठ न किंचिद्व्याजहार ह}


\twolineshloka
{ततस्तथोक्तं पितरं रामो विज्ञाय वीर्यवान्}
{वनं प्रतस्थे धर्मात्मा राजा सत्यो भवत्विति}


\twolineshloka
{तमन्वगच्छल्लक्ष्मीवान्धनुष्माँल्लक्ष्मणस्तदा}
{सीता च भार्या भद्रं ते वैदेही जनकात्मजा}


\twolineshloka
{ततो वनं गतेरामे राजा दशरथस्तदा}
{समयुज्यत देहस्य कालपर्यायधर्मणा}


\twolineshloka
{रामं तु गतमाज्ञाय राजानं च तथागतम्}
{अनार्या भरतं देवी कैकेयी वाक्यमब्रवीत्}


\twolineshloka
{गतोदशरथः स्वर्गं वनस्थौ रामलक्ष्मणौ}
{गृहाण राज्यंविपुलं क्षेमं निहतकण्टकम्}


\twolineshloka
{तामुवाच स धर्मात्मा नृशंसं बत ते कृतम्}
{पतिं हत्वाकुलं चेदमुत्साद्य धनलुब्धया}


\twolineshloka
{अयशः पातयित्वा मे मूर्ध्नि त्वं कुलपांसने}
{सकामा भव मे मातरित्युक्त्वा प्ररुरोद ह}


\twolineshloka
{स चारित्रं विशोध्याथ सर्वप्रकृतिसन्निधौ}
{अन्वयाद्धातरं रामं विनिवर्तनलालसः}


\twolineshloka
{कौसल्यां च सुमित्रां च कैकेयीं च सुदुःखितः}
{अग्रे प्रस्थाप्य यानैः स शत्रुघ्नसहितो ययौ}


\twolineshloka
{वसिष्ठवामदेवाभ्यां विप्रैश्चान्यैः सहस्रशः}
{पौरजानपदैः सार्धं रामानयनकाङ्क्षया}


\twolineshloka
{ददर्श चित्रकूटस्थं स रामं सहलक्ष्मणम्}
{तापसानामलंकारं धारयन्तं धनुर्धरम्}


\twolineshloka
{`उवाच प्राञ्जलिर्भूत्वाप्रणिपत्य रघूत्तमम्}
{शशंस मरणं राज्ञः सोऽनाथांश्चापि कोसलान्}


% Check verse!
नाथ त्वं प्रतिपद्यस्व स्वराज्यमिति चोक्तवान्
\twolineshloka
{स तस्य वचनं श्रुत्वा रामः परमदुःखितः}
{चकार देवकल्पस्य पितुः स्नात्वोदकक्रियाम्}


\threelineshloka
{अब्रवीत्स तदारामो भ्रातरं भ्रातृवत्सलम्}
{पादुके मे भविष्येते राज्यगोप्त्र्यौ परंतप}
{एवमस्त्विति तं प्राह भरतः प्रणतस्तदा'}


\twolineshloka
{विसर्जितः स रामेण पितुर्वचनकारिणा}
{नन्दिग्रामेऽकरोद्राज्यं पुरस्कृत्यास्य पादुके}


\twolineshloka
{रामस्तु पुनराशङ्क्य पौरजानपदागमम्}
{प्रविवेश महारण्यं शरभङ्गाश्रमं प्रति}


\twolineshloka
{सत्कृत्य शरभङ्गं स दण्डकारण्यमाश्रितः}
{नदीं गोदावरीं रम्यामाश्रित्य न्यवसत्तदा}


\twolineshloka
{वसतस्तस्य रामस्य ततः शूर्पणखाकृतम्}
{खरेणासीन्महद्वैरं जनस्थाननिवासिना}


\twolineshloka
{रक्षार्थं तापसानां तु राघवो धर्मवत्सलः}
{चतुर्दशसहस्राणि जघान भुवि राक्षसान्}


\twolineshloka
{दूषणं च स्वरं चैवनिहत्य सुमहाबलौ}
{चक्रे क्षेमं पुनर्धीमान्धर्मारण्यं स राघवः}


\twolineshloka
{हतेषु तेषु रक्षःसु ततः शूर्पणखा पुनः}
{ययौ निकृत्तनासोष्ठी लङ्कां भ्रातुर्निवेशनम्}


\twolineshloka
{ततो रावणमभ्येत्य राक्षसी दुःखमूर्च्छिता}
{पपात पादयोर्भ्रातुः संशुष्करुधिरानना}


\twolineshloka
{तां तथा विकृतां दृष्ट्वा रावणः क्रोधमूर्च्छितः}
{उत्पपातासनात्क्रुद्धो दन्तैर्दन्तानुपस्पृशन्}


\twolineshloka
{स्वानमात्यान्विसृज्याथ विविक्ते तामुवाच सः}
{केनास्येवं कृता भद्रे मामचिन्त्यावमत्य च}


\twolineshloka
{कः शूलं तीक्ष्णमासाद्य सर्वगात्रेषु सेवते}
{कः शिरस्यग्निमाधाय विश्वस्तः स्वपते सुखम्}


\twolineshloka
{आशीविषं घोरतरं पादेन स्पृशतीह कः}
{सिंहं केसरिणं मत्तः स्पृष्ट्वा दंष्ट्रासु तिष्ठति}


\twolineshloka
{इत्येवं ब्रुवतस्तस्य नेत्रेभ्यस्तेजसोऽर्चिषः}
{निश्चेरुर्दह्यतो रात्रौ वृक्षस्येव स्वरन्ध्रतः}


\twolineshloka
{तस्य तत्सर्वमाचख्यौ भगिनी रामविक्रमम्}
{खरदूषणसंयुक्तं राक्षसानां पराभवम्}


\twolineshloka
{`ततो ज्ञातिवधं श्रुत्वा रावणः कालचोदितः}
{रामस्य वधमाकाङ्क्षन्मारीचं मनसागमत्'}


\twolineshloka
{स निश्चित्यततः कृत्यं सागरं लवणाकरम्}
{ऊर्ध्वमाचक्रमे राजा विधाय नगरे विधिम्}


\twolineshloka
{त्रिकूटं समतिक्रम्य कालपर्वतमेव च}
{ददर्श मकरावासं गम्भीरोदं महोदधिम्}


\twolineshloka
{तमतीत्याथ गोकर्णमभ्यगच्छद्दशाननः}
{दयितं स्तानमव्यग्रं शूलपाणेर्महात्मनः}


\twolineshloka
{तत्राभ्यगच्छन्मारीचं पूर्वामात्यं दशाननः}
{पुरा रामभयादेव तापसं प्रियजीवितम्}


\chapter{अध्यायः २७९}
\twolineshloka
{मार्कण्डेय उवाच}
{}


\twolineshloka
{मारीचस्त्वथ संभ्रान्तो दृष्ट्वा रावणमागतम्}
{पूजयामास सत्कारैः फलमूलादिभिस्ततः}


\twolineshloka
{विश्रान्तं चैनमासीनमन्वासीनः स राक्षसः}
{उवाच प्रश्रितं वाक्यं वाक्यज्ञो वाक्यकोविदम्}


\twolineshloka
{न ते प्रकृतिमान्वर्णः कच्चित्क्षेमं पुरे तव}
{कच्चित्प्रकृतयः सर्वा भजन्ते त्वां यथा पुरा}


\twolineshloka
{किमिहागमने चापि कार्यं ते राक्षसेश्वर}
{कृतमित्येव तद्विद्धि यद्यपि स्यात्सुदुष्करम्}


\twolineshloka
{शशंस रावणस्तस्मै तत्सर्वं रामचेष्टितम्}
{समासेनैव कार्याणि क्रोधामर्षसमन्वितः}


\twolineshloka
{मारीचस्त्वब्रवीच्छ्रत्वा समासेनैव रावणम्}
{अलं ते राममासाद्य वीर्यज्ञो ह्यस्मि तस्य वै}


\threelineshloka
{बाणवेगं हि कस्तस्य शक्तः सोढुं महात्मनः}
{प्रव्रज्यायां हि मे हेतुः स एव पुरुषर्षभः}
{विनाशमुखमेतत्ते केनाख्यातं दुरात्मना}


\twolineshloka
{तमुवाचाथ सक्रोधो रावणः परिभर्त्सयन्}
{अकुर्वतोऽस्मद्वचनं स्यान्मृत्युरपि ते ध्रुवम्}


\twolineshloka
{मरीचश्चिन्तयामास विशिष्टान्मरणं वरम्}
{अवश्यं मरणे प्राप्ते करिष्याम्यस्य यन्मतम्}


\twolineshloka
{ततस्तं प्रत्युवाचाथ मारीचो रक्षसांवरम्}
{किं ते साह्यां मया कार्यं करिष्याम्यवशोपि तत्}


\twolineshloka
{तमब्रवीद्दशग्रीवो गच्छ सीतां प्रलोभय}
{रत्नशृङ्गो मृगो भूत्वा रत्नचित्रतनूरुहः}


\twolineshloka
{ध्रुवं सीता समालक्ष्यत्वां रामं चोदयिष्यति}
{अपक्रान्ते च काकुत्स्थे सीता वश्या भविष्यति}


\twolineshloka
{तामादायापनेष्यामि ततः स नभविष्यति}
{भार्यावियोगाद्दुर्बुद्धिरेतत्साह्यं कुरुष्व मे}


\twolineshloka
{इत्येवमुक्तोमारीचः कृत्वोदकमथात्मनः}
{रावणं पुरतो यान्तमन्वगच्छत्सुदुःखितः}


\twolineshloka
{ततस्तस्याश्रमं गत्वारामस्याक्लिष्टकर्मणः}
{चक्रतुस्तद्यथा सर्वमुभौ यत्पूर्वमन्त्रितम्}


\twolineshloka
{रावणस्तु यतिर्भूत्वा मुण्डः कुण्डीत्रिदण्डधृत्}
{मृगश्चभूत्वामारीचस्तं देशमुपजग्मतुः}


\twolineshloka
{दर्शयामास मारीचो वैदेहीं मृगरूपधृत्}
{चोदयामास तस्यार्थे सा रामं विधिचोदिता}


\twolineshloka
{रामस्तस्याः प्रियं कुर्वन्धनुरादाय सत्वरः}
{रक्षार्थे लक्ष्मणं न्यस्य प्रययौ मृगलिप्सया}


\twolineshloka
{स धन्वी बद्धतूणीरः खङ्गगोधाङ्गुलित्रवान्}
{अन्वधावन्मृगं रामो रुद्रस्तारामृगं यथा}


\twolineshloka
{सोऽन्तर्हितः पुनस्तस्य दर्शनं राक्षसो व्रजन्}
{चकर्ष महदध्वानं रामस्तं वुबुधे ततः}


\twolineshloka
{निशाचरं विदित्वा तं राघवः प्रतिभानवान्}
{अमोघं शरमादाय जघान मृगरूपिणम्}


\twolineshloka
{स रामवाणाभिहतः कृत्वा रामस्वरं तदा}
{हा सीते लक्ष्मणेत्येवं चुक्रोशार्तस्वरेण ह}


\twolineshloka
{शुश्राव तस्य वैदेही ततस्तां करुणां गिरम्}
{साप्रापतत्ततः सीता तामुवाचाथ लक्ष्मणः}


\twolineshloka
{अलं ते शङ्कया भीरु को रामं प्रहरिष्यति}
{मुहूर्ताद्द्रक्ष्यसे रामं भर्तारं त्वं शुचिस्मितम्}


\twolineshloka
{इत्युक्ता सा प्ररुदती पर्यशङ्कत लक्ष्मणम्}
{हता वै स्त्रीस्वभावेन शुद्धचारित्रभूषणा}


\twolineshloka
{सा तं परुषमारब्धा वक्तुं साध्वी पतिव्रता}
{नैष कामो भवेन्मूढ यं त्वं प्रार्थयसे हृदा}


\twolineshloka
{अप्यहंशस्त्रमादाय हन्यामात्मानमात्मना}
{पतेयं गिरिशृङ्गाद्वा विशेयं वा हुताशनम्}


\twolineshloka
{रामं भर्तारमुत्सुज्यन त्वहं त्वां कथञ्चन}
{निहीनमुपतिष्ठेयं शार्दूली क्रोष्टुकं यथा}


\twolineshloka
{एतादृशं वचः श्रुत्वा लक्ष्मणः प्रियराघ्नव}
{पिधायकर्णौ सद्वृत्तः प्रस्थितो येन राघवः}


\twolineshloka
{स रामस्य पदंगृह्य प्रससार धनुर्धरः}
{अवीक्षमाणो विम्बोष्ठीं प्रययौ लक्ष्मणस्तदा}


\twolineshloka
{एतस्मिन्नन्तरे रक्षो रावणः प्रत्यदृश्यत}
{अभव्यो भव्यरूपेण भस्मच्छन्न इवानलः}


\twolineshloka
{यतिवेपप्रतिच्छन्नो जिहीर्षुस्तामनिन्दिताम्}
{`उपागच्छत्स वैदेहीं रावणः पापनिश्चयः'}


\twolineshloka
{सा तमालक्ष्यसंप्राप्तं धर्मज्ञा जनकात्मजा}
{निमन्त्रयामास तदा फलमूलाशनादिभिः}


\twolineshloka
{अवमत्यततः सर्वं स्वं रूपं प्रत्यपद्यत}
{सान्त्वयामास वैदेहीं कामी राक्षसपुङ्गवः}


\twolineshloka
{सीते राक्षसराजोऽहंरावणो नाम विश्रुतः}
{मम लङ्कापुरी नाम्ना रम्या पारे महोदधेः}


\twolineshloka
{तत्र त्वं नरनारीषु शोभिष्यसि मया सह}
{भार्या मे भव सुश्रोणि तापसं त्यज राघवम्}


\twolineshloka
{एवमादीनि वाक्यानि श्रुत्वा तस्याथ जानकी}
{पिधाय कर्णौ सुश्रोणी मैवमित्यब्रवीद्वचः}


\twolineshloka
{प्रपतेद्द्यौः सनक्षत्रा पृथिवी शकलीभवेत्}
{`शुष्येत्तोयनिधौ तोयं चन्द्रः शीतांशुतां त्यजेत्}


\twolineshloka
{उष्णांशुत्वमथो जह्यादादित्यो वह्निरुष्णताम्'}
{त्यक्त्वाशैत्यं भजेन्नाहं त्यजेयंरघुनन्दनम्}


\twolineshloka
{कथं हि भिन्नकरटं पद्मिनं वनगोचरम्}
{उपस्थाय महानागं करेणुः सूकरं स्पृशेत्}


\twolineshloka
{कथं हि पीत्वा माध्वीकं पीत्वा च मधुमाधवीम्}
{लोभं सौवीरके कुर्यान्नारी काचिदिति स्मरेः}


\twolineshloka
{इति सा तं समाभाष्य प्रविवेशाश्रमं ततः}
{क्रोधात्प्रस्फुरमाणौष्ठी विधुन्वाना करौ मुहुः}


% Check verse!
तामधिद्रुत्य सुश्रोणीं रावणः प्रत्यषेधयत्
\twolineshloka
{भर्त्सयित्वातु रूक्षेण स्वरेण गतचेतनाम्}
{मूर्धजेषु निजग्राह ऊर्ध्वमाचक्रमे ततः}


\twolineshloka
{तां ददर्श ततो गृध्रो जटायुर्गिरिगोचरः}
{रुदतीं रामरामेति हियमाणां तपस्विनीम्}


\chapter{अध्यायः २८०}
\twolineshloka
{गार्कण्डेय उवाच}
{}


\twolineshloka
{सखा दशरथस्यासीज्जटायुररुणात्मजः}
{गृध्रराजो महावीरः संपातिर्यस् सोदरः}


\twolineshloka
{स ददर्श तदा सीतां रावणाङ्कगतां स्नुषाम्}
{सक्रोधोऽभ्यद्रवत्पक्षी रावणं राक्षसेश्वरम्}


\twolineshloka
{अथैनमब्रवीद्गृध्रो मुञ्चमुञ्चेति मैथिलीम्}
{ध्रियमाणे मयि कथं हरिष्यसि निशाचर}


\twolineshloka
{न हिमे मोक्ष्यसे जीवन्यदि नोत्सृजसे वधूम्}
{उक्त्वैवं राक्षसेन्द्रं तं चकर्त नखरैर्भृशम्}


\twolineshloka
{पक्षतुण्डप्रहारैश्च शतशो जर्जरीकृतम्}
{चक्षार रुधिरं भूरि गिरिः प्रस्रवणैरिव}


\twolineshloka
{स वध्यमानो गृध्रेण रामप्रियहितैषिणा}
{खङ्गमादाय चिच्छेद भुजौ तस् पतत्रिणः}


\twolineshloka
{निहत्य गृध्रराजं सभिन्नाभ्रशिखरोपमम्}
{ऊर्ध्वमाचक्रमे सीतां गृहीत्वाऽङ्केन राक्षसः}


\twolineshloka
{यत्रयत्रतु वैदेही पश्यत्याश्रममण्डलम्}
{सरोवा सरितो वाऽपि तत्र मुञ्चति भूषणम्}


\twolineshloka
{सा ददर्श गिरिप्रस्थे पञ्च वानरपुङ्गवान्}
{तत्र वासो महद्दिव्यमुत्ससर्ज मनस्विनी}


\twolineshloka
{तत्तेषां वानरेन्द्राणां पपात पवनोद्धतम्}
{मध्ये सुपीतं पञ्चानां विद्युन्मेघान्तरे यथा}


\twolineshloka
{अचिरेणातिचक्राम खेचरः खे चरन्निव}
{ददर्शाथ पुरीं रम्यां बहुद्वारां मनोरमाम्}


\twolineshloka
{प्राकारवप्रसंबाधां निर्मितां विश्वकर्मणा}
{प्रविवेशपुरीं लङ्कां ससीतो राक्षसेश्वरः}


\twolineshloka
{एवं हृतायां वैदेह्यां रामो हत्वा महामृगम्}
{निवृत्तो ददृशे दूराद्भ्रातरं लक्ष्मणं तदा}


\twolineshloka
{कथमुत्सृज्य वैदेहीं वने राक्षससेविते}
{इति तं भ्रातरं दृष्ट्वा प्राप्तोऽसीति व्यगर्हयत्}


\twolineshloka
{मृगरूपधरेणाथ रक्षसासोपकर्षणम्}
{भ्रातुरागमनं चैवचिन्तयन्पर्यतप्यत}


\twolineshloka
{गर्हयन्नेव रामस्तु त्वरितस्तं समासदत्}
{अपि जीवति वैदेहीमिति पश्यामि लक्ष्मण}


\twolineshloka
{तस् तत्सर्वमाचख्यौ सीताया लक्ष्मणो वचः}
{यदुक्तवत्यसदृशं वैदेही पश्चिमं वचः}


\twolineshloka
{दह्यमानेन तु हृदा रामोऽभ्यपतदाश्रमम्}
{स ददर्श रतदा गृध्रं निहतं पर्वतोपमम्}


\twolineshloka
{राक्षसं शङ्कमानस्तं विकृष्य बलवद्धनुः}
{अभ्यधावत काकुत्स्थस्ततस्तं सहलक्ष्मणः}


\twolineshloka
{स तावुवाच तेजस्वी सहितौ रामलक्ष्मणौ}
{गृध्रराजेस्मि भद्रंवां सखा दशरथस् वै}


\twolineshloka
{तस्य तद्वचनं श्रुत्वा संगृह्य धनुषी शुभे}
{कोयं पितरमस्माकं नाम्नाऽऽहेत्यूचतुश्च तौ}


\twolineshloka
{ततो ददृशतुस्तौ तं छिननपक्षद्वयं खगम्}
{तयोः शशंस गृध्रस्तु सीतार्थे रावणाद्वधम्}


\twolineshloka
{अपृच्छद्राघवो गृध्रं रावणः कां दिशं गतः}
{तस् गृध्रः शिरःकम्पैराचचक्षे ममार च}


\twolineshloka
{दक्षिणामिति काकुत्स्थो विदित्वाऽस्य तदिङ्गितम्}
{संस्कारं लम्भयामास सखायं पूजयन्पितुः}


\twolineshloka
{ततो दृष्ट्वाऽऽश्रमपदं व्यपविद्धबृसीकटम्}
{विध्वस्तकलशं शून्यं गोमायुशतसंकुलम्}


\twolineshloka
{दुःखशोकसमाविष्टौ वैदेहीहरणार्दितौ}
{जग्मतुर्दण्डकारण्यं दक्षिणेन परंतपौ}


\twolineshloka
{वने महति तस्मिंस्तु रामः सौमित्रिणा सह}
{ददर्श मृगयूथनि द्रवमाणानि सर्वशः}


\twolineshloka
{शब्दं च घोरं सत्वानां दावाग्नरिववर्धतः}
{अपश्येतां मुहूर्ताच्च कबन्धं घोरदर्शनम्}


\twolineshloka
{मेघपर्वतसंकाशं सालस्कन्धं महाभूजम्}
{उरोगतविशालाक्षं महोदरमहामुखम्}


\twolineshloka
{यदृच्छयाथ तद्रक्षः करे जग्राह लक्ष्मणम्}
{विषादमगमत्सद्यः सौमित्रिरथ भारत}


\twolineshloka
{स राममभिसंप्रेक्ष्य कृष्यते येन तन्मुखम्}
{विषण्णश्चाब्रवीद्रामं पश्यावस्थामिमां मम}


\twolineshloka
{हरणं चैववैदेह्या मम चायमुपप्लवः}
{राज्यभ्रंशश्च भवतस्तातस्य मरणं तथा}


\twolineshloka
{नाहं त्वां मह वैदेह्या समेतं कोसलागतम्}
{द्रक्ष्यामि प्रथिते राज्येपितृपैतामहे स्थितम्}


\twolineshloka
{द्रक्ष्यन्त्यार्यस्य धन्या ये कुशलाजशमीदलैः}
{अभिषिक्तस् वदनं सोमं शान्तघनं यथा}


\twolineshloka
{एवं बहुविधं धीमान्विललाप स लक्ष्मणः}
{तमुवाचाथकाकुत्स्थः संभ्रमेष्वप्यसंभ्रमः}


\threelineshloka
{मा विषीद नरव्याघ्र नैष कश्चिन्मयि स्थिते}
{`शक्तो धर्षयितुं वीर सुमित्रानन्दवर्धन'}
{छिन्ध्यस्य दक्षिणं बाहुं छिन्नः सव्यो मया भुजः}


\twolineshloka
{इत्येवं वदता तस् भुजो रामेण पातितः}
{खङ्गेन भृशतीक्ष्णेन निकृत्तस्तिलकाण्डवत्}


\twolineshloka
{ततोऽस्य दक्षिणं बाहुं स्वङ्गेनाजघ्निवान्बली}
{सौमित्रिरपि संप्रेक्ष्यभ्रातरं राघवं स्थितम्}


\twolineshloka
{पुनर्जघान पार्श्वे वै तद्रक्षो लक्ष्मणो भृशम्}
{गतासुरपतद्भूमौ कबन्धः सुमहांस्ततः}


\twolineshloka
{तस्य देहाद्विनिःसृत्य पुरुषो दिव्यदर्शनः}
{ददृशे दिवमास्थाय दिवि सूर्य इव ज्वलन्}


\twolineshloka
{पप्रच्छ रामस्तं वाग्मी कस्त्वं प्रब्रूहि पृच्छतः}
{कामया किमिदं चित्रमाश्चर्यं प्रतिभाति मे}


\twolineshloka
{तस्याचचक्षेगन्धर्वोविश्वावसुरहं नृप}
{प्राप्तो ब्राह्मणशपेन योनिं राक्षससेविताम्}


\twolineshloka
{रावणेन हृतासीता लङ्कायां संनिवेशिता}
{सुग्रीवमभिगच्छस्वस ते साह्यं करिष्यति}


\twolineshloka
{एषा पम्पा शिवजला हंसकारण्डवायुता}
{ऋश्यमूकस्य शैलस्य संनिकर्षे तटाकिनी}


\twolineshloka
{वसते तत्रसुग्रीवश्चतुर्भिः सचिवैः सह}
{भ्राता बानरराजस् वालिनो हेममालिनः}


\twolineshloka
{तेन त्वं सहसंगम्य दुःखमूलं निवेदय}
{समानशीलो भवतः साहाय्यं स करिष्यति}


\twolineshloka
{एतावच्छक्यमस्माभिर्वक्तुं द्रष्टासि जानकीम्}
{ध्रुवं वानरराजस् विदितो रावणालयः}


\twolineshloka
{इत्युक्त्वाऽन्तर्हितो दिव्यः पुरुषः स महाप्रभः}
{विस्मयं जग्मतुश्चोभौ प्रवीरौ रामलक्ष्मणौ}


\chapter{अध्यायः २८१}
\twolineshloka
{मार्कण्डेय उवाच}
{}


\twolineshloka
{ततोऽविदूरे नलिनीं रप्रभूतकमलोत्पलाम्}
{सीताहरणदुःखार्तः पम्पां रामः समासदत्}


\twolineshloka
{मारुतेन सुशीतेन सुखेनामृतगन्धिना}
{सेव्यमानो वने तस्मिञ्जगाम मनसा प्रियाम्}


\twolineshloka
{विललाप सराजेन्द्रस्तत्रकान्तानुस्मरन्}
{कामबाणाभिसंतप्तं सौमित्रिस्तमथाब्रवीत्}


\twolineshloka
{न त्वामेवंविधो भावः स्प्रष्टुमर्हति मानद}
{आत्मवन्तमिव व्याधिः पुरुषंवृद्धसेविनम्}


\twolineshloka
{प्रवृत्तिरुपलब्धा ते वैदेह्या रावणस्य च}
{तां त्वं पुरुषकारेण बुद्ध्या चैवोपपादय}


\twolineshloka
{अभिगच्छाव सुग्रीवं शैलस्थं हरिपुङ्गवम्}
{मयि शिष्ये च भृत्ये च सहाये च समाश्वस}


\twolineshloka
{एवं बहुविधैर्वाक्यैर्लक्ष्मणेन स राघवः}
{उक्तः प्रकृतिमापेदे कार्ये चानन्तरोऽभवत्}


\twolineshloka
{निषेव्य वारि पम्पायास्तर्पयित्वा पितृनपि}
{प्रतस्थतुरभौ वीरौ भ्रातरौ रामलक्ष्मणौ}


\twolineshloka
{तावृश्यमूकमभ्येत्य बहुमूलफलद्रुमम्}
{गिर्यग्रे वानरान्पञ्वीरौ ददृशतुस्तदा}


\twolineshloka
{सुग्रीवः प्रेषयामास सचिवं वानरं तयोः}
{बुद्धिमन्तं हनूमन्तं हिमवन्तमिव स्थितम्}


\twolineshloka
{तेन संभाष्य पूर्वं तौ सुग्रीवमभिजग्मतुः}
{रसख्यं वानरराजेन चक्रे रामस्तदा नृप}


\twolineshloka
{`ततः सीतां हृतां श्रुत्वा सुग्रीवो वालिना कृतम्}
{दुःखमाख्यातवान्सर्वं रामायामिततेजसे'}


\twolineshloka
{तद्वासो दर्शयामास तस् कार्ये निवेदिते}
{वानराणां तु यत्सीता ह्रियमाणा व्यपासृजत्}


\twolineshloka
{तत्प्रत्ययकरं लब्ध्वा सुग्रीवं प्लवगाधिपम्}
{पृथिव्यां वानरैश्वर्ये स्वयंरामोऽभ्यषेचयत्}


\twolineshloka
{प्रतिजज्ञे चकाकुत्स्थः समरे वालिनो वधम्}
{सुग्रीवश्चापि वैदेह्याः पुनरानयनं नृप}


\twolineshloka
{इत्येवं समयं कृत्वाविश्वास्य च परस्परम्}
{अभ्येत्य सर्वकिष्किन्धां तस्थुर्युद्धाभिकाङ्क्षिणः}


\twolineshloka
{सुग्रीवः प्राप्यकिष्किन्धां ननादौघनिभस्वनः}
{नसाय् तन्ममृषे वाली तारा तं प्रत्यषेधयत्}


\twolineshloka
{यथानदतिसुग्रीवो बलवानेष वानरः}
{मन्ये चाश्रयवान्प्राप्तो न त्वं निष्क्रान्तुमर्हसि}


\twolineshloka
{हेममाली ततो वाली तारां ताराधिपाननाम्}
{प्रोवाच वचनं वाग्मी तां वानरपतिः पतिः}


% Check verse!
सर्वभूतरुतज्ञा शृणु सर्वं कपीश्वर ॥केन चाश्रयवान्प्राप्तो ममैप भ्रातृगन्धिकः
\twolineshloka
{चिन्तयित्वा मुहूर्तं तु तारा ताराधिपप्रभा}
{पतिमित्यब्रवीत्प्राज्ञा शृणु सर्वं कपीश्वर}


\twolineshloka
{हृतदारो महासत्वोरामो दशरथात्मजः}
{रतुल्यारिमित्रतां प्राप्तः सुग्रीवेण धनुर्धरः}


\twolineshloka
{भ्राता चास्य महाबाहुः सौमित्रिरपराजितः}
{लक्ष्मणो नाम मेधावी स्थितः कार्यार्थसिद्धये}


\twolineshloka
{मैन्दश्च द्विविदश्चापि हनूमांश्चानिलात्मजः}
{जाम्बवानृक्षराजश्च सुग्रीवसचिवाः स्थिताः}


\twolineshloka
{सर्व एते महात्मानो बुद्धिमन्तो महाबलाः}
{अलं तव विनाशाय रामवीर्यव्यपाश्रयाः}


\twolineshloka
{तस्यास्तदाक्षिप्य वचो हितमुक्तं कपीश्वरः}
{पर्यशङ्कत तामीर्षुः सुग्रीवगतमानसाम्}


\twolineshloka
{तारां परुषमुक्त्वा तु निर्जगाम गुहामुखात्}
{स्थितं माल्यवतोऽभ्याशे सुग्रीवं सोभ्यभाषत}


\twolineshloka
{असकृत्त्वं मया क्लीव निर्जितो जीवितप्रियः}
{मुक्तो गच्छसि दुर्बुद्धे कथंकारं रणे पुनः}


\twolineshloka
{इत्युक्तः प्राहसुग्रीवो भ्रातरं हेतुमद्वचः}
{प्राप्तकालममित्रघ्नं रामं सम्बोधयन्निव}


\twolineshloka
{हृतराज्यस्य मे राजन्हृतदारस्य च त्वया}
{किं मे जीवितसामर्थ्यमिति विद्धि समागतम्}


\twolineshloka
{एवमुक्त्वाबहुविधं ततस्तौ सन्निपेततुः}
{समरे वालिसुग्रीवौ सालतालशिलायुधौ}


\twolineshloka
{उभौ जघ्नतुरन्योन्यमुभौ भूमौ निपेततुः}
{उभौ ववल्गतुश्चित्रं मुष्टिभिश्च निजघ्नतुः}


\twolineshloka
{उभौ रुधिरसंसिक्तौ नखदन्तपरिक्षतौ}
{शुशुभाते तदा वीरौ पुष्पिताविव किंशुकौ}


\twolineshloka
{न विशेषस्तयोर्युद्धे यदा कश्चन दृश्यते}
{सुग्रीवस् तदा मालां हनुमान्कण्ठ आसजत्}


\twolineshloka
{स मालया तदा वीरः शुशुभे कण्ठसक्तया}
{श्रीमानिव महाशैलो मलयो मेघमालया}


\twolineshloka
{कृतचिह्नं तु सुग्रीवं रामो दृष्ट्वा महाधनुः}
{विचकर्ष धनुःश्रेष्ठं वालिमुद्दिश्य लक्षयन्}


\twolineshloka
{विष्फारस्तस् धनुषो यन्त्रस्येव तदा बभौ}
{वितत्रास तदा वाली शरेणाभिहतो हृदि}


\twolineshloka
{स भिन्नहृदयो वाली वक्राच्छोणितमुद्वमन्}
{ददर्शावस्थितं रामं ततः सौमित्रिणा सह}


\twolineshloka
{गर्हयित्वास काकुत्स्थं पपात भुवि मूर्च्छितः}
{तारा ददर्श तं भूमौ तारापतिमिव च्युतम्}


\twolineshloka
{हते वालिनि सुग्रीवः किष्किन्धां प्रत्यपद्यत}
{तां तारापतिमुखीं तारां निपतितेश्वराम्}


\twolineshloka
{रामस्तु चतुरो मासान्पृष्ठे माल्यवतः शुभे}
{निवासमकरोद्धीमान्सुग्रीवेणाभ्युपस्थितः}


\twolineshloka
{रावणोऽपिपुरीं गत्वालङ्कां कामबलात्कृतः}
{सीतां निवेशयामास भवने नन्दनोपमे}


\twolineshloka
{अशोकवनिकाभ्यासे तापसास्रमसन्निभे}
{भर्तृस्मरणतन्वङ्गी तापसीवेषधारिणी}


\twolineshloka
{उपवासतपःशीला ततः सा पृथुलेक्षणा}
{उवास दुःखवसतिं फलमूलकृताशना}


\twolineshloka
{दिदेश राक्षसीस्तत्ररक्षणे राक्षसाधिपः}
{प्रासासिशूलपरशुमुद्गरालातधारिणीः}


\twolineshloka
{द्व्यक्षीं त्र्यक्षीं ललाटक्षीं दीर्घजिह्वामजिह्विकाम्}
{त्रिस्तनीमेकपादां च त्रिजटामेकलोचनाम्}


\twolineshloka
{एताश्चान्याश्च दीप्ताक्ष्यः करभोत्कटमूर्धजाः}
{परिवार्यासते सीतां दिवारात्रमतन्द्रिताः}


\twolineshloka
{तास्तु तामायतापाङ्गीं पिशाच्यो दारुणस्वराः}
{तर्जयन्ति सदा रौद्राः परुषव्यञ्जनस्वराः}


\twolineshloka
{खादाम पाटयामैनां तिलशः प्रविभज्यताम्}
{येयं भर्तारमस्माकमवमत्येह जीवति}


\twolineshloka
{इत्येवं परिभर्त्सन्तीस्त्रासयानाः पुनः पुनः}
{भर्तृशोकसमाविष्टा निःश्वस्येदमुवाच ताः}


\twolineshloka
{आर्याः खादत मां शीघ्रं न मे लोभोस्ति जीविते}
{विना तं पुण्डरीकाक्षं नीलकुञ्चितमूर्धजम्}


\twolineshloka
{अद्यैवाहं निराहारा जीवितप्रियवर्जिता}
{शोषयिष्यामि गात्राणि बल्ली तलगता यथा}


\twolineshloka
{न त्वन्यमभिगच्छेयं पुमांसं राघवादृते}
{इति जानीत सत्यं मेक्रियतां यदनन्तरम्}


\twolineshloka
{तस्यास्तद्वचनं श्रुत्वा राक्षस्यस्ताः खरस्वनाः}
{आख्यातुं राक्षसेन्द्राय जन्मुस्तत्सर्वमादितः}


\twolineshloka
{गतासु तासु सर्वासु त्रिजटा नाम राक्षसी}
{सान्त्वयामास वैदेहीं धर्मज्ञा प्रियवादिनी}


\twolineshloka
{सीते वक्ष्यामि ते किंचिद्विश्वासं करु मे सखि}
{भयं त्वं त्यज वामोरु शृणु चेदं वचो मम}


\twolineshloka
{अविन्ध्यो नाम मेधावी वृद्धो राक्षसपुङ्गवः}
{स रामस्य हितान्वेषी त्वदर्थे मामचूचुदत्}


\twolineshloka
{सीता मद्वचनाद्वाच्या समाश्वास्य प्रसाद्य च}
{भर्ता तेकुशली रामोलक्ष्मणानुगतो बली}


\twolineshloka
{सख्यं वानरराजेन शक्रप्रतिमतेजसा}
{कृतवान्राघवः श्रीमांस्त्वदर्थे च समुद्यतः}


\twolineshloka
{मा च ते भूद्भयं भीरु रावणाल्लोकगर्हितात्}
{नलकूबरशापेन रक्षिता ह्यसि नन्दिनि}


\twolineshloka
{शप्तो ह्येष पुरा पापो वधूं रम्भां परामृशन्}
{न शक्रोत्यवशां नारीमुपैतुमजितेन्द्रियः}


\twolineshloka
{क्षिप्रमेष्यति ते भर्ता सुग्रीवेणाभिरक्षितः}
{सौमित्रिसहितो धीमांस्त्वां चेतो मोक्षयिष्यति}


\twolineshloka
{स्वप्ना हि सुमहाघोरा दृष्टा मेऽनिष्टदर्शनाः}
{विनाशायास्य दुर्बुद्धेः पौलस्त्यस्य कुलस्य च}


\twolineshloka
{दारुणो ह्येष दुष्टात्मा क्षुद्रकर्मा निशाचरः}
{स्वभावाच्छीलदोषेण सर्वेषां भयवर्धनः}


\twolineshloka
{स्पर्धते सर्वदेवैर्यः कालोपहतचेतनः}
{मया विनासलिङ्गानि स्वप्ने दृष्टानि तस्य वै}


\twolineshloka
{तैलाभिषिक्तो विकचो मज्जनप्के दशाननः}
{असकृत्स्वरयुक्ते तु रथे नृत्यन्निव स्थितः}


\twolineshloka
{कुम्भकर्णादयश्चेमे नग्नाः पतितमूर्धजाः}
{गच्छन्ति दक्षिणामाशां रक्तमाल्यानुलेपनाः}


\twolineshloka
{श्वेतातपत्रः सोष्णीषः शुक्लमाल्यानुलेपनः}
{श्वेतपर्वतमारूढ एक एव विभीषणः}


\twolineshloka
{सचिवाश्चास्य चत्वारः शुक्लमाल्यानुलेपनाः}
{श्वेतपर्वतमारूढा मोक्ष्यन्तेऽस्मान्महाभयात्}


\twolineshloka
{रामस्यास्त्रेण पृथिवी परिक्षिप्ता ससागरा}
{यशसा पृथिवीं कृत्स्नां पूरयिष्यति ते पतिः}


\twolineshloka
{हस्तिसक्थिसमारूढो भुञ्जानो मधुपायसम्}
{लक्ष्मणश्च मया दृष्टो दिधक्षुः सर्वतो दिशम्}


\twolineshloka
{रुदती रुधिरार्द्राङ्गी व्याघ्रेण परिरक्षिता}
{असकृत्त्वं मया दृष्टा गच्छन्ती दिशमुत्तराम्}


\twolineshloka
{हर्षमेष्यसि वैदेहि क्षिप्रं भर्त्रा समन्विता}
{राघवेण सहभ्रात्रा सीते त्वमचिरादिव}


\twolineshloka
{इत्येतन्मृगशावाक्षी तच्छ्रुत्वा त्रिजटावचः}
{बभूवाशावती बाला पुनर्भर्तृसमागमे}


\twolineshloka
{तावदभ्यागता रौद्र्यः पिशाच्यस्ताःसुदारुणाः}
{ददृशुस्तां त्रिजटया सहासीनां यथापुरम्}


\chapter{अध्यायः २८२}
\twolineshloka
{मार्कण्डेय उवाच}
{}


\twolineshloka
{ततस्तां भर्तृशोकार्तां दीनां मलिनवाससम्}
{मणिशेषाभ्यलंकारां रुदतीं च पतिव्रताम्}


\twolineshloka
{राक्षसीभिरुपास्यन्तीं समासीनां शिलातले}
{रावणःकामबाणार्तो ददर्शोपससर्प च}


\twolineshloka
{देवदानवगन्धर्वयक्षकिंपुरुषैर्युधि}
{अजितोशोकवनिकां ययौ कन्दर्पपीडितः}


\twolineshloka
{दिव्याम्बरधरः श्रीमन्सुमृष्टमणिकुण्डलः}
{विचित्रमाल्यमुकुटो वसन्त इव मूर्तिमान्}


\twolineshloka
{न कल्पवृक्षसदृशोयत्नादपि विभूषितः}
{श्मशानचैत्यद्रुमवद्भूषितोऽपि भयंकरः}


\twolineshloka
{स तस्यास्तनुमध्यायाः समीपे रजनीचरः}
{ददृशे रोहिणीमेत्य शनैश्चर इव ग्रैहः}


\twolineshloka
{स तामामन्त्र्य सुश्रोणीं पुष्पकेतुशराहतः}
{इदमित्यब्रवीद्वाक्यं त्रस्तां रौहीमिवाबलाम्}


\twolineshloka
{सीते पर्याप्तमेतावत्कृतोभर्तुरनुग्रहः}
{प्रसादं कुरु तन्वङ्गि क्रियतां परिकर्म ते}


\twolineshloka
{भजस्वमां वरारोहे महार्हाभरणाम्बरा}
{भवमे सर्वनारीणामुत्तमा वरवर्णिनी}


\threelineshloka
{सन्ति मे देवक्न्याश्च गन्धर्वाणआं च योषितः}
{सन्ति दानवन्याश् दैत्यानां चापि योषितः}
{`तासामद्यविशालाक्षि सर्वासां मे भवोत्तमा}


\twolineshloka
{चतुर्दश पिशाचीनां कोट्यो मे वचने स्थिताः}
{द्विस्तावत्पुरुषादानां रक्षसां भीमकर्मणाम्}


\twolineshloka
{ततो मे त्रिगुणा यक्षा ये मद्वचनकारिणः}
{केचिदेव धनाध्यक्षं भ्रातरं मे समाश्रिताः}


\twolineshloka
{गन्दर्वाप्सरसो भद्रे मामापानगतं सदा}
{उपतिष्ठन्ति वामोरु यथैव भ्रातरं मम}


\twolineshloka
{पुत्रोऽहमपि विप्रर्षेः साक्षाद्विश्रवसो मुनेः}
{पञ्चमो लोकपालानामिति मे प्रथितं यशः}


\twolineshloka
{दिव्यानि भक्ष्यभोज्यानि पानानि विविधानि च}
{यथैव त्रिदशेशस्यतथैव मम भामिनि}


\twolineshloka
{क्षीयतां दुष्कृतं कर्म वनवासकृतं तव}
{भार्या मे भवसुश्रोणि यथा मण्डोदरीतथा}


\twolineshloka
{इत्युक्ता तेन वैदेही परिवृत्य सुभानना}
{तृणमनतरतः कृत्वातमुवाच निशाचरम्}


\twolineshloka
{अशिवेनातिवामोरूरजस्रं नेत्रवारिणा}
{स्तनावपतितौ बाला संहतावभिवर्षती}


\twolineshloka
{`व्यवस्थाप्यकथंचित्सा विषादादतिमोहिता'}
{उवाच वाक्यं तं क्षुद्रं वैदेही पतिदेवता}


\threelineshloka
{असकृद्वदतो वाक्यमीदृशं राक्षसेश्वर}
{विषादयुक्तमेतत्ते मया श्रुतमभाग्यया}
{तद्भद्रमुख भद्रं ते मानसं विनिवर्त्यताम्}


\twolineshloka
{परदाराऽस्म्यलभ्या च सततं च पतिव्रता}
{न चैवौपयिकी भार्य मानुषी तव राक्षस}


\twolineshloka
{विवशां धर्षयित्वच कां त्वं प्रीतिमवाप्स्यसि}
{न च पालयसे धर्मं लोकपालसमः कथम्}


\twolineshloka
{भ्रातरं राजराजं तं महेश्वरसस्वं प्रभुम्}
{धनेश्वरं व्यपदिशन्कथं त्विह न लज्जसे}


\twolineshloka
{इत्युक्त्वा प्रारुदत्सीता कम्पयन्ती पयोधरौ}
{शिरोधरां च तन्वङ्गी मुस्वं प्रच्छाद्यवाससा}


\twolineshloka
{तस्य रुदत्या भामित्या दीर्घा वेणी सुसयता}
{ददृशे स्वसिता स्निग्धा काली व्यालीव मूर्धनि}


\twolineshloka
{श्रुत्वा तद्रावणो वाक्यं सीतयोक्तं सुनिषुरम्}
{प्रत्याख्यातोऽपिदुर्मेधाः पुनरेवाब्रवीद्वचः}


\twolineshloka
{काममङ्गनि मे सीते दुनोतु मकरध्वजः}
{नत्वामकामां सुश्रोणीं समेप्ये चारुहासिनीं}


\twolineshloka
{किंनु शक्यं मया कर्तुं यत्त्वमद्यापिमानुषम्}
{आहारभूतमस्माकं राममेवानुरुध्यसे}


\twolineshloka
{इत्युक्त्वा तामनिन्द्याङ्गीं स राक्षसमहेश्वरः}
{तत्रैवान्तर्हितो भूत्वा जगामाभिमतां दिशम्}


\twolineshloka
{राक्षसीभिः परिवृतावैदेही शोककशिंता}
{सेव्यमाना त्रिजटया तत्रैव न्यवसत्तदा}


\chapter{अध्यायः २८३}
\twolineshloka
{मार्कण्डेय उवाच}
{}


\twolineshloka
{राघवः सहसौमित्रिः सुग्रीवेणाभिपालितः}
{वसनमाल्यवतः पृष्ठे ददर्श विमलं नभः}


\twolineshloka
{सदृष्ट्वाविमले व्योम्नि निर्मलं शसलक्षणम्}
{ग्रहनक्षत्रताराभिरनुयान्तममित्रहा}


\twolineshloka
{कुमुदोत्पलपद्मानां गन्धमादाय वायुना}
{महीधरस्थः शीतेन सहसाप्रतिबोधितः}


\twolineshloka
{प्रभाते लक्ष्मणं वीरमभ्यभाषत दुर्मनाः}
{सीतां संस्मृत् यधर्मात्मा रुद्धां राक्षसवेश्मनि}


\twolineshloka
{गच्छ लक्ष्मण जानीहि किष्किंदायां कपीश्वरम्}
{प्रमत्तं ग्राम्यधर्मेषु कृतघ्नं स्वार्थपण्डितम्}


\twolineshloka
{योसौ कुलाधमो मूढो मया राज्येऽभिषेचितः}
{सर्ववानरगोपुच्छा यमृक्षाश्च भजन्ति वै}


\twolineshloka
{यदर्थं निहतो बाली मया रघुकुलोद्वह}
{त्वया सहमहाबाहो किष्किन्धोपवने तदा}


\twolineshloka
{कृतघ्नं तमहं मन्ये वानरापशदं भुवि}
{यो मामेवंगतो मूढो न जानीतेऽद्य लक्ष्मण}


\twolineshloka
{असौ मन्ये न जानीते समयप्रतिपालनम्}
{कृतोपकारं मां नूनमवमत्याल्पया धिया}


\twolineshloka
{यदितावदनुद्युक्तः शेते कामसुखात्मकः}
{नेतव्यो वालिमार्गेण सर्वभूतगतिं त्वया}


\twolineshloka
{अथापि घटतेऽस्माकमर्ते वानरपुङ्गवः}
{तमादायैव काकुत्स्थ त्वरावान्भव माचिरम्}


\twolineshloka
{इत्युक्तो लक्ष्मणो भ्रात्रा गुरुवाक्यहिते रतः}
{प्रतस्थे रुचिरं गृह्य समार्गणगुणं धनुः}


\twolineshloka
{किष्किन्धाद्वारमासाद्यप्रविवेशानिवारितः}
{सक्रोध इतितं मत्वाराजा प्रत्युद्ययौ हरिः}


\twolineshloka
{तं सदारोविनीतात्मा सुग्रीवः प्लवगाधिपः}
{पूजया प्रतिजग्राह प्रीयमाणस्तदर्हया}


\twolineshloka
{तमब्रवीद्रामवचः सौमित्रिरकुतोभयः}
{स तत्सर्वमशेषेण श्रुत्वा प्रह्वः कृताञ्जलिः}


\twolineshloka
{सभृत्यदारो राजेन्द्रसुग्रीवो वानराधिपः}
{इदमाह वचः प्रीतो लक्ष्मणं नरकुञ्जरम्}


\twolineshloka
{नास्मि लक्ष्मण दुर्मेधा नाकृतज्ञो न निर्घृणः}
{श्रूयतां यः प्रयत्नो मे सीतापर्येषणे कृतः}


\twolineshloka
{दिशः प्रस्थापिताः सर्वेविनीता हरयो मया}
{सर्वेषां च कृतः कालो मासेऽभ्यागमने पुनः}


\twolineshloka
{यैरियं सवना साद्रिः सपुरा सागराम्बरा}
{विचेतव्या मही वीर सग्रामनगराकरा}


\twolineshloka
{स मासः पञ्चरात्रेण पूर्णो भवितुमर्हति}
{ततः श्रोष्यसि रामेण सहितः सुमहत्प्रियम्}


\twolineshloka
{इत्युक्तो लक्ष्मणत्तेन कवानरेनद्रेण धीमता}
{त्यक्त्वारोषमदीनात्मा सुग्रीवं प्रत्यपूजयत्}


\twolineshloka
{सरामं सहसुग्रीवो माल्यवत्पुष्ठमास्थितम्}
{भिगम्योदयं तस्य कार्यस्य प्रत्यवेदयत्}


\twolineshloka
{इत्येवंवानरेनद्रास्ते समाजन्मुः सहस्रशः}
{दिशस्तिस्रो विचित्याथ न तु ये दक्षिणां गताः}


\twolineshloka
{आचख्युस्तत्र रामाय महीं सागरमेखलाम्}
{विचितां न तु वैदेह्या दर्शनं रावणस् वा}


\twolineshloka
{गतास्तु दक्षिणामाशां ये वै वानरपुङ्गवाः}
{आशावांस्तेषु काकुत्स्थः प्राणानार्तोऽभ्यधारयत्}


\twolineshloka
{द्विमासोपरमे काले व्यतीते प्लवगास्ततः}
{सुग्रीवमभिगम्येदं त्वरिता वाक्यमब्रुवन्}


\twolineshloka
{रक्षितंवालिना यत्तत्स्फीतं मधुवनं महत्}
{त्वया च प्लवगश्रेष्ठ तद्भुङ्क्ते पवनात्मजः}


\twolineshloka
{वालिपुत्रोऽङ्गदश्चैव ये चान्ये प्लवगर्षभाः}
{विचेतुं दक्षिणामाशां राजन्प्रस्थापितास्त्वया}


\twolineshloka
{तेषामपनयं श्रुत्वा मेने सकृतकृत्यताम्}
{कृतार्थानां हि भृत्यानामेतद्भवति चेष्टितम्}


\twolineshloka
{स तद्रामाय मेधावी शशंस प्लवगर्षभः}
{रामश्चाप्यनुमानेन मेने दृष्टां तु मैथिलीम्}


\twolineshloka
{हनुमत्प्रमुखाश्चापि विश्रान्तास्ते प्लवङ्गमाः}
{अभिजग्मुर्हरीन्द्रं तं रामलक्ष्मणसन्निधौ}


\twolineshloka
{गतिं च मुखवर्णं च दृष्ट्वारामो हनूमतः}
{अगमत्प्रत्ययं भूयो दृष्टा सीतेति भारत}


\twolineshloka
{हनूमत्प्रमुखास्ते तु वानराः पूर्णमानसाः}
{प्रणेमुर्विधिवद्रामं सुग्रीवं लक्ष्मणं तथा}


\twolineshloka
{तानुवाचानतान्रामः प्रगृह्य सशरं धनुः}
{अपि मां जीवयिष्यध्वमपि वः कृतकृत्यता}


\twolineshloka
{अपि राज्यमयोध्यायां कारयिष्याम्यहं पुनः}
{निहत्यसमरे शत्रूनाहृत्यजनकात्मजाम्}


\twolineshloka
{अमोक्षयित्वावैदेहीमहत्वा च रणे रिपून्}
{हृतदारोऽवधूतश्चनाहं जीवितुमुत्सहे}


\twolineshloka
{इत्युक्तवचनं रामं प्रत्युवाचानिलात्मजः}
{प्रियमाख्यामि ते राम दृष्टा सा जानकी मया}


\twolineshloka
{विचित्य दक्षिणामाशां सपर्वतवनाकराम्}
{श्रान्ताः काले व्यतीते स्म दृष्टवन्तो महागुहां}


\twolineshloka
{प्रविशामो वयं तां तु बहुयोजनमायताम्}
{अन्धकारां सुविपिनां गहनां कीटसेविताम्}


\twolineshloka
{गत्वा सुमहदध्वानमादित्यस् प्रभां ततः}
{दृष्टवन्तः स्म तत्रैवभवनं दिव्यमन्तरा}


\twolineshloka
{गयस् किल दैत्यस् तदा सद्वेश्म राघव}
{तत्रप्रभावती नाम तपोऽतप्यत तापसी}


\twolineshloka
{तया दत्तानि भोज्यानिपानानिविविधानि च}
{भुकत््वा लब्धबलाः सन्तस्तयोक्तेन पथा ततः}


\twolineshloka
{निर्याय तस्मादुद्देशात्पश्यामो लवणाम्भसः}
{समीपे सह्यमलयौ दर्दुरं च महागिरिम्}


\twolineshloka
{ततो मलयमारुह्य पश्यन्तो वरुणालयम्}
{कविषण्णा व्यथिताः खिन्ना निराशा जीविते भृशम्}


\twolineshloka
{अनेकशतविस्तीर्णं योजनानां महोदधिम्}
{तिमिनक्रझषावासं चिन्तयन्तः सुदुःखिताः}


\twolineshloka
{तत्रानशनसंकल्पं कृत्वाऽऽसीना वयं तदा}
{ततः कथान्ते गृध्रस्य जटायोरभवत्कथा}


\twolineshloka
{ततः पर्वतशृङ्गाभं घोररूपं भयावहम्}
{पक्षिणं दृष्टवन्तः स्म वैनतियेमिवापरम्}


\twolineshloka
{सोऽस्मानतर्कयद्भोक्तुमथाभ्येत्य वचोऽब्रवीत्}
{भोः क एष मम भ्रातुर्जटायोः कुरुते कथाम्}


\twolineshloka
{संपातिर्नाम तस्याहं ज्येष्ठो भ्राता खगाधिपः}
{अन्योन्यस्पर्धया रूढावावामदित्यसत्पदम्}


\twolineshloka
{ततो दग्धाविमौ पक्षौ न दग्धौ तु जटायुषः}
{तस्मान्मे चिरदृष्टः स भ्राता गृध्रपतः प्रियः}


\twolineshloka
{निर्दग्धपक्षः पतितो ह्यहमस्मिन्महागिरौ}
{`द्रष्टुं वीरं न शक्नोमि भ्रातरं वै जटायुषम्'}


\twolineshloka
{तस्यैवं वदतोऽस्माभिर्हतो भ्राता निवेदितः}
{व्यसनं भवतश्चेदं संक्षेपाद्वै निवेदितम्}


\twolineshloka
{स सम्पातिस्तदा राजञ्श्रुत्वासुमहदप्रियम्}
{विषण्णचेताः पप्रच्छ पुनरस्मानरिंदम}


\twolineshloka
{कः सरामः कथं सीता जटायुश्च कथं हतः}
{इच्छामि सर्वमेवैतच्छ्रोतुं प्लवगसत्तमाः}


\twolineshloka
{तस्याहं सर्वमेवैतद्भवतो व्यसनागमम्}
{प्रायोपवेशने चैवहेतुं विस्तरशोऽब्रुवम्}


\twolineshloka
{सोऽस्मानाश्वासयामास वाक्येनानेन पक्षिराट्}
{रावणो विदितो मह्यं लङ्का चास्य महापुरी}


\twolineshloka
{दृष्टापारे समुद्रस्य त्रिकूटगिरिकन्दरे}
{भवित्री तत्र वैदेही न रमेऽस्त्यत्रविचारणा}


\twolineshloka
{इतितस्य वचः श्रुत्वा वयमुत्थाय सत्वराः}
{सागरक्रमणे मन्त्रं मन्त्रयामः परंतप}


\threelineshloka
{नाध्यवास्यद्यदा कश्चित्सागरस्य विलङ्घनम्}
{ततः पितरमाविश्य पुप्लुवेऽहंमहार्णवम्}
{शतयोजनविस्तीर्णं निहत्य जलराक्षसीम्}


\twolineshloka
{उपवासतपःशीला भर्तृदर्शनलालसा}
{जटिला मलदिग्धाङ्गीकृश दीना तपस्विन}


\twolineshloka
{निमित्तैस्तामहं सीतामुपलभ्य पृथग्विधैः}
{उपसृत्याब्रवं चार्यामभिगम्य रहोगताम्}


\twolineshloka
{सीते रामस्य दूतोऽहंवानरोमारुतात्मजः}
{त्वद्दर्शनमभिप्रप्सुरिह प्राप्तो विहायसा}


\twolineshloka
{राजपुत्रौ कुशलिनौ भ्रातरौ रामलक्ष्मणौ}
{सर्वशाखामृगेन्द्रेण सुग्रीवेणाभिपालितौ}


\twolineshloka
{कुशलंत्वाब्रवीद्रामःसीते सौमित्रिणा सह}
{सखिभावाच् सुग्रीवः कुशलं त्वाऽनुपृच्छति}


\twolineshloka
{क्षिप्रमेष्यति ते भर्ता सर्वशाखामृगैः सह}
{प्रत्ययं कुरु मे देवि वानरोऽस्मि न राक्षसः}


\twolineshloka
{मुहूर्तमिवच ध्यात्वा सीता मां प्रत्युवाच ह}
{अवैमि त्वांहनूमन्तमविन्ध्यवचनादहम्}


\twolineshloka
{अविन्ध्यो हि महाबाहो राक्षसो वृद्धसंमतः}
{कथितस्तेन सुग्रीवस्त्वद्विधैः सचिवैर्वृतः}


\twolineshloka
{गम्यतामिति चोक्त्वा मां सीता पादादिमं मणिम्}
{घारिता येन वैदेही कालमेतमनिन्दिता}


\twolineshloka
{प्रत्ययार्थं कथां चेमां कथयामास जानकी}
{क्षिप्तामिषीकां काकाय चित्रकूटे महागिरौ}


\twolineshloka
{भवता पुरुषव्याघ्र प्रत्यभिज्ञानकारणात्}
{`एकाक्षिविकलः काकः सुदुष्टात्मा कृतश्चवै'}


\twolineshloka
{ग्राहयित्वाऽहमात्मानं ततो दग्ध्वाच तां पुरीम्}
{संप्राप्त इतितं रामः प्रियवादिनमार्चयत्}


\chapter{अध्यायः २८४}
\twolineshloka
{मार्कण्डेय उवाच}
{}


\twolineshloka
{ततस्तत्रैवरामस्य समासीनस्य तैः सह}
{समाजग्मुः कपिश्रेष्ठाः सुग्रीववचनात्तदा}


\twolineshloka
{वृतः कोटिसहस्रेण वानराणां तरस्विनाम्}
{श्वशुरो वालिनः श्रीमान्सुषेणो राममभ्ययात्}


\twolineshloka
{कोटीशतवृतोवाऽपिगजो गवय एव च}
{वानरेन््रौ महावीर्यौ पृथक्पृथगदृश्यताम्}


\twolineshloka
{षष्टिकोटिसहस्राणि प्रकर्षन्प्रत्यदृश्यत}
{गोलाङ्गूलो महाराज गवाक्षो भीमदर्शनः}


\twolineshloka
{गन्धमादनवासी तु प्रथितो गन्धमादनः}
{कोटीशतसहस्राणि हरीणां समकर्षत}


\twolineshloka
{पनसो नाम मेधावी वानरःसुमहाबलः}
{कोटीर्दश द्वादश च त्रिंशत्पञ्च प्रकर्षति}


\twolineshloka
{श्रीमान्दधिमुखो नाम हरिवृद्धोऽतिवीर्यवान्}
{प्रचकर्ष महासैनयं हरीणां भीमतेजसाम्}


\twolineshloka
{कृषणानां मुखपुण्ड्राणामृक्षाणां भीमकर्मणाम्}
{कोटीर्दश द्वादश च त्रिंशत्पञ्च प्रकर्षति}


\twolineshloka
{एते चान्ये च बहवो हरियूथपयूथपाः}
{असङ्ख्येया महाराज समीयू रामकारणात्}


\twolineshloka
{गिरिकूटनिभाङ्गानां सिंहानामिव गर्जताम्}
{श्रूयते तुमुलः शब्दस्तत्रतत्रप्रधावताम्}


\twolineshloka
{गिरिकूटनिभाः क्नचित्केचिन्महिषसन्निभाः}
{शरदभ्रप्रतीकाशाः केचिद्धिङ्गुलकाननाः}


\twolineshloka
{उत्पतन्तः पतन्तश्च प्लवमानाश्च वानराः}
{उद्धुन्वन्तोऽपरे रेणून्समाजग्मुः समन्ततः}


\twolineshloka
{सवानरमहासैन्यः पूर्णसागरसन्निभः}
{निवेशमकरोत्तत्रसुग्रीवानुमते तदा}


\twolineshloka
{ततस्तेषु हरीन्द्रेषु समावृत्तेषु सर्वशः}
{तिथौ प्रशस्ते नक्षत्रे मुहूर्ते चाभिपूजिते}


\twolineshloka
{तेन व्यूढेन सैन्येन लोकानुद्वर्तयन्निव}
{प्रययौ राघवः श्रीमान्सुग्रीवसहितस्तदा}


\twolineshloka
{मुखमासीत्तु सैन्यस्य हनूमान्मारुतात्मजः}
{जघनं पालयामास सौमित्रिरकुतोभयः}


\twolineshloka
{बद्धगोधाङ्गुलित्रणौ राघवौ तत्रजग्मतुः}
{वृतौ हरिमहामात्रैश्चन्द्रसूर्यौ ग्रहैरिव}


\twolineshloka
{प्रबभौ हरिसैन्यं तत्सालतालशिलायुधम्}
{सुमहच्छालिभवनं यथा सूर्योदयं प्रति}


\twolineshloka
{नलनीलाङ्गदक्राथमैन्दद्विविदपालिता}
{ययौ सुमहती सेना राघवस्यार्थसिद्धये}


\twolineshloka
{विविधेषु प्रशस्तेषु बहुमूलफलेषु च}
{प्रभूतमधुमांसेषु वारिमत्सु विवेषु च}


\twolineshloka
{निवसन्ती निराबाधा तथैवगिरिसानुषु}
{उपायाद्धिरिसेना सा क्षारोदमथ मागरम्}


\twolineshloka
{द्वितीयसागरनिमं तद्बलंबहुलध्वजम्}
{वेलावनं समासाद् निवासमकरोत्तदा}


\twolineshloka
{ततो दाशरथिः श्रीमान्सुग्रीवं प्रत्यभाषत}
{मध्ये वानरमुख्यानां प्राप्तकालमिदं वचः}


\twolineshloka
{उपायः कोनु भवतां मतः सागरलङ्घने}
{इयं हि महती सेना सागरश्चातिदुस्तरः}


\twolineshloka
{तत्रान्ये व्याहरन्ति स्म वानराः पटुमानिनः}
{समर्था लङ्घने सिन्दोर्न तत्कृत्स्नस्य वानराः}


\twolineshloka
{केचिन्नौभिर्व्यवस्यन्ति केचिच्च विविधैः प्लवैः}
{नेति रामस्तु तानसर्वान्सान्त्वयन्प्रत्यभाषत}


\twolineshloka
{शतयोजनविस्तारं न शक्ताः सर्ववानराः}
{क्रान्तुं तोयनिधिं वीरानैषा वो नैष्ठिकी मतिः}


\twolineshloka
{नावो न सन्ति सेनाया बह्व्यस्तारयितुं तथा}
{वणिजामुपघातं च कथमस्मद्विधश्चरेत्}


\twolineshloka
{विस्तीर्णं चैव नः सैन्यं हन्याच्छिद्रेण वै परः}
{प्लवोडुपप्रतारश्चनैवात्रमम रोचते}


\twolineshloka
{अहं त्विमं जलनिधिं समारप्स्याम्युपायतः}
{प्रतिशेष्याम्युपवसन्दर्शयिष्ति मां ततः}


\twolineshloka
{न चेद्दर्शयिता मार्गं धक्ष्याम्यनमहं ततः}
{महास्त्रैरप्रतिहतैरत्यग्निपवनोज्ज्वलैः}


\twolineshloka
{इत्युक्त्वा सहसौमित्रिरुपस्पृश्याथ राघवः}
{प्रतिशिस्ये जलनिधं विधिवत्कुशसंस्तरे}


\twolineshloka
{सागरस्तु ततः स्वप्ने दर्शयामास राघवम्}
{देवो नदनदीमर्ता श्रीमान्यादोगणैर्वृतः}


\twolineshloka
{कौसल्यामातरित्येवमाभाष्य मधुरं वचः}
{इदमित्याह रत्नानामाकरैः शतशो वृतः}


\threelineshloka
{ब्रूहि किं तेकरोम्यत्रसाहाय्यं पुरुषर्षभ}
{ऐक्ष्वाको ह्यस्मि ते ज्ञाती राम सत्यपराक्रमः}
{एवमुक्तः समुद्रेण रामो वाक्यमथाब्रवीत्}


\threelineshloka
{मार्गमिच्छामि सैन्यस्य दत्तं नदनदीपते}
{येन गत्वादशग्रीवं हन्याम कुलपांसनम्}
{`राक्षसंसानुबन्धं तं मम भार्यापहारिणम्'}


\twolineshloka
{यद्येवं याचतो मार्गं न प्रदास्यति मे भवान्}
{शरैस्त्वां शोषयिष्यामि दिव्यास्त्रयतिमन्त्रितैः}


\twolineshloka
{इत्येवंब्रुवतः श्रुत्वारामस्य वरुणालयः}
{उवाचव्यथितोवाक्यमितिबद्धाञ्जलिःस्थितः}


\twolineshloka
{नेच्छामि प्रतिघातं ते नास्मि विघ्नकरस्तव}
{शृणु चेदं वचोराम श्रुत्वा कर्तव्यमाचर}


\twolineshloka
{यदि दास्यामि ते मार्गं सैन्यस् व्रजतोऽऽज्ञया}
{अन्येऽप्याज्ञापयिष्यन्ति मामेवं धनुषोबलात्}


\twolineshloka
{अस्तित्वत्रनलो नाम वानरः शिल्पिसंमतः}
{त्वष्टुः काकुत्स्थ तनयो बलवान्विश्वकर्मणः}


\twolineshloka
{स यत्काष्ठं तृणं वाऽपिशिलां वा क्षेप्स्यते मयि}
{सर्वं तद्धारयिष्यामि स ते सेतुर्भविष्यति}


\twolineshloka
{इत्युक्त्वाऽन्तर्हिते तस्मिन्रामो नलमुवाच ह}
{कुरु सेतुं समुद्रे त्वंशक्तो ह्यसि मतो मम}


\twolineshloka
{तेनोपायेन काकुत्स्थः सतुबन्धमकारयत्}
{दशयोजनविस्तारमायतं शतयोजनम्}


\twolineshloka
{नलसेतुरिति ख्यातो योऽद्यापि प्रथितो भुवि}
{रामस्याज्ञां पुरस्कृत्य धार्यते गिरिसंनिभः}


\twolineshloka
{तत्रस्थं स तु धर्मात्मा समागच्चद्विभीषणः}
{भ्राता वै राक्षसेन्द्रस्य चतुर्भिः सचिवैः सह}


\twolineshloka
{प्रतिजग्राह रामस्तं स्वागतेन महामनाः}
{सुग्रीवस्य तु शङ्काऽभूत्प्रणिधिः स्यादिति स्मह}


\twolineshloka
{राघवः सत्यचेष्टाभिः सम्यक्व चरितेङ्गितैः}
{यदा तत्त्वेन तुष्टोऽभूत्तत एनमपूजयत्}


\twolineshloka
{सर्वराक्षसराज्येचाप्यभ्यपिञ्चद्विभीषणम्}
{चक्रे च मन्त्रसचिवं सहृदंलक्ष्मणस् च}


\twolineshloka
{विभीषणमते चैव सोऽत्यक्रामन्महार्णवम्}
{ससैन्यः सेतुना तेन मार्गेणैव नराधिपः}


\twolineshloka
{ततो गत्वासमासाद्य लङ्कोद्यानान्यनेकशः}
{भेदयामास कपिभिर्महान्ति च बहूनि च}


\twolineshloka
{तत्रास्तां रावणामात्यौ राक्षसौ शुकसारणौ}
{चरौ वानररूपेण तौ जग्राह विभीषणः}


\twolineshloka
{प्रतिपन्नौ यदा रूपं राक्षसं तौ निशाचरौ}
{दर्शयित्वा ततः सैन्यं रामः पश्चादवासृजत्}


\twolineshloka
{निवेश्योपवने सैन्यं स शूरः प्राज्यवानरम्}
{प्रेषयामास दुत्येन रावणस्य ततोऽङ्गदम्}


\chapter{अध्यायः २८५}
\twolineshloka
{मार्कण्डेय उवाच}
{}


\twolineshloka
{प्रभूतान्नोदकेतस्मिन्बहुमूलफले वने}
{सेनां निवेश्य काकुत्स्थो विधिवत्पर्यरक्षत}


\twolineshloka
{रावणः संविधं चक्रे लङ्कायां शास्त्रनिर्मिताम्}
{प्रकृत्यैवदुराधर्षा दृढप्राकारतोरणा}


\twolineshloka
{अगाधतोयाः परिखा मीननक्रसमाकुलाः}
{बभूवुः सप्त दुर्धर्षाः स्वादिरैः शङ्कुभिश्चिताः}


\twolineshloka
{कर्णाटयन्त्रा दुर्धर्षा बभूवुः सहुडोपलाः}
{साशीविषघटायोधाः ससर्जरसपांसवः}


\twolineshloka
{मुसलालातनाराचतोमरासिपरश्वथैः}
{अन्विताश्चशतघ्नीभिः समधूच्छिष्टमुद्गराः}


\twolineshloka
{पुरद्वारेषु सर्वेषु गुल्माः स्थावरजङ्गमाः}
{बभूवुः पत्तिबहुलाः प्रभूतगजवाजिनः}


\twolineshloka
{अङ्गदस्त्वथ लङ्कायां द्वारदेशमुपागतः}
{विदितो रराक्षसेन्द्रस्य प्रविवेशगतव्यथः}


\twolineshloka
{मध्ये राक्षसकोटीनां बह्वीनां सुमहाबलः}
{शुशुभे मेघमालाभिरादित्य इव संवृतः}


\twolineshloka
{ससमासाद्य पौलस्त्यममात्यैरभिसंवृतम्}
{रामसंदेशमामन्त्र्य वाग्मी वक्तुं प्रचक्रमे}


\threelineshloka
{आह त्वां राघवो राजन्कोसलेन्द्रो महायशाः}
{प्राप्तकालमिदं वाक्यं तदादत्स्व सुदुर्मते}
{}


\twolineshloka
{अकृतात्मानमासाद्य राजानमनये रतम्}
{विनश्यन्त्यनयाविष्टा देशाश्च नगराणि च}


\twolineshloka
{त्वयैकेनापराद्धं मे सीतामाहरता बलात्}
{वधायानपराद्धानामन्येषां तद्भविष्यति}


\twolineshloka
{ये त्वया बलदर्पाभ्यामाविष्टेन वनेचराः}
{ऋषयोहिंसिताः पूर्वंदेवाश्चाप्यवमानिताः}


\twolineshloka
{राजर्षयश्च निहता रुदत्यश्चाहृताः स्त्रियः}
{तदिदं समनुप्राप्तं फलंतस्यानयस्य ते}


\twolineshloka
{हन्तास्मि त्वां सहामात्यैर्युध्यस्व पुरुषो भव}
{पश्य मे धनुषो वीर्यं मानुषस् निशाचर}


\twolineshloka
{मुच्यतां जानकी सीता न मे मोक्ष्यमसि कर्हिचित्}
{अराक्षसमिमं लोकंकर्ताऽस्मि निशितैः शरैः}


\twolineshloka
{इतितस् ब्रुवाणस् दूतस् परुषं वचः}
{श्रुत्वा न ममृषे राजा रावणः क्रोधमूर्च्छितः}


\twolineshloka
{हङ्गितज्ञास्ततो भर्तुश्चत्वारो रजनीचराः}
{चतुर्ष्वङ्गेषु जगृहुः शार्दूलमिव पक्षिणः}


\twolineshloka
{तांस्तथाङ्गेषु संसक्तानङ्गदो रजनीचरान्}
{आदायैव खमुत्पत्य प्रासादतलमाविशत्}


\twolineshloka
{वेगेनोत्पततस्तस्य पेतुस्ते रजनीचराः}
{भुवि संभिन्नहृदयाः प्रहारवरपीडिताः}


\twolineshloka
{संसक्तोहर्म्यशिखरात्तस्मात्पुनरवापतत्}
{लङ्घयित्वा पुरं लङ्कां सुवेलस्य समीपतः}


\twolineshloka
{कोसलेन्द्रमथागम्य सर्वमावेद्य वानरः}
{विशश्राम स तेजस्वी राघवेणाभिनन्दितः}


\twolineshloka
{ततः सर्वाभिसारेण हरीणां वातरंहसाम्}
{भेदयामास लङ्कायाः ग्राकारं रघुनन्दनः}


\twolineshloka
{विभीषणर्क्षाधिपती पुरस्कृत्याथ लक्ष्मणः}
{दक्षिणं नगरद्वारमवामृद्गाद्दुरासदम्}


\twolineshloka
{करभारुणगात्राणां हरीणां युद्धशालिनाम्}
{कोटीशतसहस्रेण लङ्कामभ्यपतत्तदा}


\twolineshloka
{प्रलम्बबाहूरुकरजङ्घान्तरविलम्बिनाम्}
{ऋक्षाणआं धूम्रवर्णानां तिस्रः कोठ्यो व्यवस्थिताः}


\twolineshloka
{उत्पतद्भिः पतद्भिश्च निपतद्भिश्च वानरैः}
{नादृश्यत तदा सूर्यो रजसा नाशितप्रभः}


\twolineshloka
{शालिप्रसूनसदृशैः शिरीपकुसुमप्रभैः}
{तरुणादित्यसदृशैः शणगौरैश्च वैनरैः}


\twolineshloka
{प्राकारं ददृशुस्ते तु समन्तात्कपिलीकृतम्}
{राक्षसा विस्मिता राजन्सस्त्रीवृद्धाः समन्ततः}


\twolineshloka
{बिभिदुस्ते मणिस्तम्भान्कर्णाट्टशिखराणि च}
{भग्नोन्मथितशृङ्गाणि यन्त्राणि च विचिक्षिपुः}


\twolineshloka
{परिगृह्य शतघ्नीश्च सचक्राः सगुडोपलाः}
{चिक्षिपुर्भुजवेगेन लङ्कामध्येमहास्वनाः}


\twolineshloka
{प्राकारस्थाश्चये केचिन्निशाचरगणास्तथा}
{प्रदुद्रुवुस्ते शतशः कपिभिः समभिद्रुताः}


\twolineshloka
{ततस्तु राजवचनाद्राक्षसाः कामरूपिणः}
{निर्ययुर्विकृताकाराः सहस्रशतसङ्घशः}


\twolineshloka
{शखवर्षाणि वर्षन्तो द्रावयित्वा वनौकसः}
{प्राकारं शोभयन्तस्ते परं विस्मयमास्थिताः}


\twolineshloka
{स मापराशिसदृशैर्बभूव क्षणादाचरैः}
{कृतो निर्वानरो भूयः प्राकारो भीमदर्शनैः}


\twolineshloka
{पेतुः शलविभिन्नाङ्गा बहवो वानरर्पभाः}
{स्तम्भतोरणभग्नाश्चपेतुस्तत्रनिशाचराः}


\twolineshloka
{केशाकेश्यभवद्युद्धं रक्षसां वानरैः सह}
{नखैर्दन्तैश्च वीराणां खादतां वै परस्परम्}


\twolineshloka
{निष्टनन्तो ह्युभयतस्तत्र वानरराक्षसाः}
{हतानिपतिता भूमौ न मुञ्चन्ति परस्परम्}


\twolineshloka
{रामस्तु शरजालानिववर्ष जलदो यथा}
{तानिलङ्कां समासाद्य जघ्रुस्तान्रजनीचरान्}


\twolineshloka
{सौमित्रिरपि नाराचैर्दृढधन्वा जितक्लमः}
{आदिश्यादिश्य दुर्गस्थान्पातयामास राक्षसान्}


\twolineshloka
{ततः प्रत्यवहारोऽभूत्सैन्यानां राधवाज्ञया}
{कृते विमर्दे लङ्कायां लब्धलक्ष्योजयोत्तरः}


\chapter{अध्यायः २८६}
\twolineshloka
{मार्कण्डेय उवाच}
{}


\twolineshloka
{ततो निविशमानांस्तान्सैनिकान्रावणानुगाः}
{अभिजग्मुर्गणाऽनके पिशाचक्षुद्ररक्षसाम्}


\twolineshloka
{पर्वणः पतनो जम्भः खरः क्रोधवशो हरिः}
{प्ररुजश्चारुजश्चैव प्रघसश्चैवमादयः}


\twolineshloka
{ततोऽभिपततां तेषामदृश्यानां दुरात्मनाम्}
{अन्तर्धानवधं तज्ज्ञश्चकार स विभीषणः}


\twolineshloka
{ते दृश्यमाना हरिभिर्बलिभिर्दूरपातिभिः}
{निहताः सर्वशो राजन्महीं जग्मुर्गतासवः}


\twolineshloka
{अमृष्यमाणः सबलो रावणो निर्ययावथ}
{राक्षसानां बलैर्घोरैः पिशाचानांच संवृतः}


\twolineshloka
{युद्धशास्त्रविधानज्ञ उशना इव चापरः}
{व्यूह्यचौशनसं व्यूहं हरीनभ्यवहारयत्}


\twolineshloka
{राघवस्तु विनिर्यान्तं व्यूढानीकं दशाननम्}
{बार्हस्पत्यं विधं कृत्वा प्रतिव्यूह्य ह्यदृश्यत}


\twolineshloka
{समेत्य युयुधे तत्र ततो रामेण रावणः}
{युयुधे लक्ष्मणश्चापि तथैवेन्द्रजिता सह}


\twolineshloka
{विरूपाक्षेण सुग्रीवस्तारेण च निस्वर्वटः}
{पौण्ड्रेण च नलस्तत्र पदुशः पनसेन च}


\twolineshloka
{विषह्यं यं हि यो मेने स स तेन समेयिवान्}
{युयुधे युद्धवेलायां स्वबाहुबलमाश्रितः}


\twolineshloka
{स संप्रहारो ववृधे भीरूणां भयवर्धनः}
{रोमसंहर्षणो घोरः पुरा देवासुरे यथा}


\twolineshloka
{रावणो राममानर्च्छच्छक्तिशूलासिवृष्टिभिः}
{निशितैरायसैस्तीक्ष्णै रावणं चापि राघवः}


\twolineshloka
{तथैवेन्द्रजितं यत्तं लक्ष्मणो मर्मभेदिभिः}
{इन्द्रजिच्चापि सौमित्रिं बिभेद बहुभिः शरैः}


\twolineshloka
{विभीषणः प्रहस्तं च प्रहस्तश्च विभीषणम्}
{खगपत्रैः शरैस्तीक्ष्णैरभ्यवर्षद्गतव्यथः}


\twolineshloka
{तेषां बलवतामासीन्महास्त्राणां समागमः}
{विव्यथुः सकला येन त्रयो लोकाश्चराचराः}


\chapter{अध्यायः २८७}
\twolineshloka
{मार्कण्डेय उवाच}
{}


\twolineshloka
{ततः प्रहस्तः सहसा समभ्येत्य विभीषणम्}
{गदया ताडयामास विनद्य रणकर्कशम्}


\twolineshloka
{स तयाऽभिहतो धीमान्गदया भीमवेगया}
{नाकम्पत महाबाहुर्हिमवानिव सुस्थिरः}


\twolineshloka
{ततः प्रगृह्यविपुलां शतघण्टां विभीषणः}
{अनुमन्त्र्य महाशक्तिं चिक्षेपास् शिरः प्रति}


\twolineshloka
{पतन्त्या स तया वेगाद्राक्षसोऽशनिवेगया}
{हृतोत्तामङ्गो ददृशे वातरुग्ण इव द्रुमः}


\twolineshloka
{तं दृष्ट्वा निहतं सङ्ख्ये प्रहस्तं क्षणदाचरम्}
{अभिदुद्राव धूम्राक्षो वेगेन महता कपीन्}


\twolineshloka
{तस् मेघोपमं सैन्यमापतद्भीमदर्शनम्}
{दृष्ट्वैव सहसा दीर्णा रणे वानरपुङ्गवाः}


\twolineshloka
{ततस्तान्सहसा दीर्णान्दृष्ट्वा वानरपुङ्गवान्}
{निर्ययौ कपिशार्दूलो हनूमान्मारुतात्मजः}


\twolineshloka
{तं दृष्ट्वाऽवस्थितं सङ्ख्ये हरयः पवनात्मजम्}
{महत्या त्वरया राजत्संन्यवर्तन्त सर्वशः}


\twolineshloka
{ततः शब्दो महानासीत्तुमुलो रोमहर्षणः}
{रामरावणसैन्यानामन्योन्यमभिधावताम्}


\twolineshloka
{तस्मिन्प्रवृत्ते संग्रामे घोरे रुधिरकर्दमे}
{क्षूम्राक्षः कपिसैन्यं तद्द्रावयामास पत्रिभिः}


\twolineshloka
{तं स रक्षोमहामात्रमापतन्तं सपत्नजित्}
{प्रतिजग्राह हनुमांस्तरसा पवनात्मजः}


\twolineshloka
{तयोर्युद्धमभूदधोरं हरिराक्षसवीरयोः}
{जीगीषतोर्युधाऽन्योन्यमिन्द्रप्रह्लादयोरिवं}


\twolineshloka
{कगदाभिः परिघैश्चैव राक्षसो जघ्निवान्कपिम्}
{कपिश्च जघ्निवान्रः सस्कन्धविटपैर्द्रुमैः}


\twolineshloka
{ततस्तमतिकोपेन साश्वं सरथसारथिम्}
{धूम्राक्षमवधीत्क्रुद्धो हनूमान्मारुतात्मजः}


\twolineshloka
{ततस्तं निहतं दृष्ट्वा धूम्राक्षं राक्षसोत्तमम्}
{हरयो जातविश्रम्भा जघ्नुरन्ये च सैनिकान्}


\twolineshloka
{ते वध्यमाना हरिभिर्बलिभिर्जितकाशिभिः}
{राक्षसा भग्नसंकल्पा लङ्कामभ्यपतन्भयात्}


\twolineshloka
{तेऽभिपत्य पुरं भग्ना हतशेषा निशाचराः}
{सर्वं राज्ञे यथावृत्तं रावणाय न्यवेदयन्}


\twolineshloka
{श्रुत्वा तु रावणस्तेभ्यः प्रहस्तं निहतं युधि}
{धूम्राक्षं च महेष्वासं ससैन्यं सहराक्षसैः}


\twolineshloka
{सुदीर्घमिव निःश्वस्य समुत्पत्य वरासनात्}
{उवाच कुम्भकर्णस्य कर्मकालोऽयमागतः}


\twolineshloka
{इत्येवमुक्त्वा विविधैर्वादित्रैः सुमहास्वनैः}
{शयानमतिनिद्रालुं कुम्भकर्णमबोधयत्}


\threelineshloka
{प्रबोध्य महता चैनं यत्नेनाऽऽगतसाध्वसः}
{स्वस्थमासीनमव्यग्रं विनिद्रं राक्षसाधिपः}
{ततोऽब्रवीद्दशग्रीवः कुम्भकर्णं महाबलम्}


\twolineshloka
{धन्योसि यस्य ते निद्रा कुम्भकर्णेयमीदृशी}
{य इदं दारुणं कालं न जानीषे महाभयम्}


\twolineshloka
{एष तीर्त्वाऽर्णवं रामः सेतुना हरिभिः सह}
{अवमत्येह नः सर्वान्करोति कदनं महत्}


\twolineshloka
{मया त्वपहृता भार्या सीता नामास्य जानकी}
{तां नेतुं स इहायातो बद्ध्वा सेतुं महार्णवे}


\twolineshloka
{तेन चैव प्रहस्तादिर्महान्नः स्वजनो हतः}
{तस्य नान्यो निहन्ताऽस्ति त्वामृतेशत्रुकर्शन}


\twolineshloka
{सदंशितोऽभिनिर्याहि त्वमद्य बलिनांवर}
{रामादीन्समरे सर्वाञ्जहि शत्रूनरिंदम}


\twolineshloka
{दूषणावरजौ चैव वज्रवेगप्रमाथिनौ}
{तौ त्वां बलेन महता सहितावनुयास्यतः}


\twolineshloka
{इत्युक्त्वा राक्षुसपतिः कुम्भकर्णं तरस्विनम्}
{संदिदेशेतिकर्तव्ये वज्रवेगप्रमाथिनौ}


\twolineshloka
{तथ्त्युक्त्वा युतौ वीरौ रावणं दूषाणानुजौ}
{कुम्भकर्णं पुरस्कृत्य तूर्णं निर्ययतुः पुरात्}


\chapter{अध्यायः २८८}
\twolineshloka
{मार्कण्डेय उवाच}
{}


\twolineshloka
{ततो निर्याय स्वपुरात्कुम्भकर्णः सहानुगः}
{अपश्यत्कपिसैन्यं रतज्जितकाश्यग्रतः स्थितम्}


\twolineshloka
{स वीक्षमाणस्तत्सैन्यं रामदर्शनकाङ्क्षया}
{अपश्यच्चापि सौमित्रिं धनुष्पाणिं व्यवस्थितम्}


\twolineshloka
{तमभ्येत्याशु हरयः परिवब्रुः समन्ततः}
{`शैलवृक्षायुधा नादानमुञ्चन्भीषणास्ततः'}


\twolineshloka
{अभ्यघ्नंश्च महाकायैर्बहुभिर्जगतीरुहैः}
{करजैरतुदंश्चान्ये विहाय भयमुत्तमम्}


\twolineshloka
{बहुधा युध्यमानास्ते युद्धमार्गैः प्लवंगमाः}
{नानाप्रहरणैर्भीमै राक्षसेन्द्रमताडयन्}


\twolineshloka
{स ताड्यमानः प्रहसन्भक्षयामास वानरान्}
{बलं चण्डबलाख्यं च वज्रबाहुं च वानरम्}


\twolineshloka
{तद्दृष्ट्वा व्यथनं कर्म कुम्भकर्णस्य रक्षसः}
{उदक्रोशन्परित्रस्तास्तारप्रभृतयस्तदा}


\twolineshloka
{तानुच्चैः क्रोशतः सैन्याञ्श्रुत्वा स हरियूथपान्}
{अभिदुद्राव सुग्रीवः कुम्भकर्णमपेतभीः}


\twolineshloka
{ततो निपत्य वेगेन कुम्भकर्णं महामना}
{सालेन जघ्निवान्मूर्ध्निं बलेन कपिकुञ्जरः}


\twolineshloka
{स महात्मा महावेगः कुम्भकर्णस् मूर्धनि}
{बिभेद सालं सुग्रीवो न चैवाव्यथयत्कपिः}


\twolineshloka
{ततो विनद्यसहसा सालस्पर्शविबोधितः}
{दोर्भ्यामादाय सुग्रीवं कुम्भकर्णोऽहरद्बलात्}


\twolineshloka
{ह्रियमाणं तु सुग्रीवं कुम्भकर्णेन रक्षसा}
{अवेक्ष्याभ्यद्रवद्वीरः सौमित्रिर्मित्रनन्दनः}


\twolineshloka
{सोऽभिपत्य महर्वेगं रुक्मपुङ्खं महाशरम्}
{प्राहिणोत्कुम्भकर्णाय लक्ष्मणः परवीरहा}


\twolineshloka
{स तस्य देहावरणं भित्त्वा देहं च सायकः}
{जगाम दारयन्भूमिं रुधिरेण समुक्षितः}


\twolineshloka
{तथा स भिन्नहृदयः समुत्सृज्य कपीश्वरम्}
{`वेगेन महताऽऽविष्टस्तिष्ठतिष्ठेति चाब्रवीत्}


\twolineshloka
{कुम्भकर्णो महेष्वासः प्रगृहीतशिलायुधः}
{अभिदुद्राव सौमित्रिमुद्यम्य महतीं शिलाम्}


\twolineshloka
{तस्याभिपततस्तूर्णं क्षुराभ्यामुच्छितौ करौ}
{चिच्छेद निशिताग्राभ्यां स बभूव चतुर्भुजः}


\twolineshloka
{तानप्यस् भुजान्सर्वान्प्रगृहीतशिलायुधान्}
{क्षुरैश्चिच्छेदलघ्वस्त्रं सौमित्रिः प्रतिदर्शयन्}


\twolineshloka
{स बभूवातिकायश्च बहुपादशिरोभुजः}
{तं ब्रह्मास्त्रेण सौमित्रिर्ददाराद्रिचयोपमम्}


\twolineshloka
{स पपात महावीर्यो दिव्यास्त्राभिहतो रणे}
{महाशनिविनिर्दग्धः पादपोऽङ्कुरवानिव}


\twolineshloka
{तं दृष्ट्वा वृत्रसंकाशं कुम्भकर्णं तरस्विनम्}
{गतासुं पतितं भूमौ राक्षसाः प्राद्रवन्भयात्}


\twolineshloka
{तथातान्द्रवतो योधान्दृष्ट्वा तौ दूषणानुजौ}
{अवस्थाप्याथ सौमित्रिं संक्रुद्धावभ्यधावताम्}


\twolineshloka
{तावाद्रवन्तौ संक्रुद्धौ वज्रवेगप्रमाथिनौ}
{अभिजग्राह सौमित्रिर्विनद्योभौ पतत्रिभिः}


\twolineshloka
{ततः सुतुमुलं युद्धमभवद्रोमहर्षणम्}
{दूषणानुजयोः पार्थ लक्ष्मणस् च धीमतः}


\twolineshloka
{महता शरवर्षेण राक्षसौ सोऽभ्यवर्पत}
{तं चापिवीरौ संक्रुद्धावुभौ तौ समवर्षताम्}


\twolineshloka
{मुहूर्तमेवमभवद्वज्रवेगप्रमाथिनोः}
{सौमित्रेश्च महाबाहोः संप्रहारः सुदारुणः}


\twolineshloka
{अथाद्रिशृङ्गमादाय हनुमान्मारुतात्मजः}
{अभिद्रुत्याददे प्राणान्वज्रवेगस्य रक्षसः}


\twolineshloka
{नीलश्च महता ग्राव्णा दूपणावरजं हरिः}
{प्रमाथिनमभिद्रुत्य प्रममाथ महाबलः}


\twolineshloka
{ततः प्रावर्तत पुनः संग्रामः कटुकोदयः}
{रामरावणसैन्यानामन्योन्यमभिधावताम्}


\twolineshloka
{शतसो नैर्ऋतान्वन्या जघ्नुर्वन्यांश्च नैर्ऋताः}
{नैर्ऋतास्तत्रवध्यन्ते प्रायेण न तु वानराः}


\chapter{अध्यायः २८९}
\twolineshloka
{मार्कण्डेय उवाच}
{}


\twolineshloka
{ततः श्रुत्वाहतं सङ्ख्ये कुम्भकर्णं सहानुगम्}
{प्रहस्तं च महेष्वासं धूम्राक्षं चातितेजसम्}


\twolineshloka
{पुत्रमिनद्रजितं वीरं रावणः प्रत्यभाषत}
{जहिरामममित्रघ्न सुग्रीवं च सलक्ष्मणम्}


\twolineshloka
{त्वया हि मम सत्पुत्र यशो दीप्तमुपार्जितम्}
{जित्वावज्रधरं सङ्ख्ये सहस्राक्षं शचीपतिम्}


\twolineshloka
{अन्तर्हितः प्रकाशो वा दिव्यैर्दत्तवरैः शरैः}
{जहि शत्रूनमित्रघ्न मम शस्त्रभृतांवर}


\twolineshloka
{रामलक्ष्मणसुग्रीवाः शरस्पर्शं न तेऽनघ}
{समर्थाः प्रतिसोढुं च कुतस्तदनुयायिनः}


\twolineshloka
{अगता या प्रहस्तेन कुम्भकर्णेन चानघ}
{खरस्यापचितिः सङ्ख्ये तां गच्छ त्वे महाभुज}


\twolineshloka
{त्वमद्य निशितैर्बाणैर्हत्वा शत्रून्ससैनिकान्}
{प्रतिनन्दय मां पुत्र पुरा जित्वेव वासवम्}


\twolineshloka
{इत्युक्तः स तथेत्युक्त्वा रथमास्थाय दंशिथः}
{प्रययाविन्द्रजिद्राजंस्तूर्णमायोधनं प्रति}


\twolineshloka
{ततो विश्राव्य विस्पष्टं नाम राक्षसपुङ्गवः}
{आह्वयामास समरे लक्ष्मणं शुभलक्षणम्}


\twolineshloka
{तं लक्ष्मणोऽभ्यधावच्च प्रगृह्य सशरं धनुः}
{त्रासयंस्तलघोषेण सिंहः क्षुद्रमृगं यथा}


\twolineshloka
{तयोः समभवद्युद्धं सुमहज्जयगृद्धिनोः}
{दिव्यास्त्रविदुपोस्तीव्रमन्योन्यस्पर्धिनोस्तदा}


\twolineshloka
{रावणिस्तु यदा नैनं विशेषयति सायकैः}
{ततो गुरुतरं यत्नमातिष्ठद्बलिनां वरः}


\twolineshloka
{तत एवं महावेगैरर्दयामास तोमरैः}
{तानागतान्स चिच्छेद सौमित्रिर्निशितैः शरैः}


\twolineshloka
{ते निकृत्ताः शरैस्तीक्ष्णैर्न्यपतन्धरणीतले}
{`साधका रावणेराजौ शतशः शकलीकृताः}


\twolineshloka
{तमङ्गदो वालिसुतः श्रीमानुद्यम्य पादपम्}
{अभिद्रुत्य महावेगस्ताडयामास मूर्धनि}


\twolineshloka
{तस्येन्द्रजिदसंभ्रान्तः प्रासेनोरसि वीर्यवान्}
{प्रहर्तुमैच्छत्तं चास्य प्रासं चिच्छेद लक्ष्मणः}


\twolineshloka
{तमभ्याशगतं वीरमङ्गदं रावणात्मजः}
{गदयाऽताडयत्सव्ये पार्श्वेवानरपुङ्गवम्}


\twolineshloka
{तमचिन्त्य प्रहारं स बलवान्वालिनः सुतः}
{ससर्जेन्द्रजितः क्रोधात्सालस्कन्धं तथाङ्गदः}


\twolineshloka
{सोऽङ्गदेन रुपोत्सृष्टो वधायेन्द्रजितस्तरुः}
{जघानेन्द्रजितः पार्थ रथं साश्वं ससारथिम्}


\twolineshloka
{ततो हताश्वात्प्रस्कन्द्य रथात्स हतसारथिः}
{तत्रैवान्तर्दधे राजन्मायया रावणात्मजः}


\twolineshloka
{अन्तर्हितं विदित्वा तं बहुमायं च राक्षसम्}
{रामस्तं देशमागम्य तत्सैन्यं पर्यरक्षत}


\twolineshloka
{स राममुद्दिश्य शरैस्ततो दत्तवरैस्तदा}
{विव्याध सर्वगात्रेषु लक्ष्मणं च महाबलम्}


\twolineshloka
{तमदृश्यंशरैः शूरौ माययाऽन्तर्हितं तदा}
{योधयामासतुरुभौ रावणिं रामलक्ष्मणौ}


\twolineshloka
{स रुषा सर्वगात्रेषु तयोः पुरुषसिंहयोः}
{व्यसृजत्सायकान्भूयः शतशोऽथ सहस्रशः}


\twolineshloka
{तमदृश्यं विचिन्वन्तः सृजन्तमनिशं शरान्}
{हरयो विविशुर्व्योम प्रगृह्य महतीः शिलाः}


\twolineshloka
{तांश्च तौ चाप्यदृश्यः सशरैर्विव्याध राक्षसः}
{स भृशं ताडयामास रावणिर्मायया वृतः}


\twolineshloka
{तौ शरैरर्दितौ वीरौ भ्रारौ रामलक्ष्मणौ}
{पेततुर्गगनाद्भूमिं सूर्याचन्द्रमसाविव}


\chapter{अध्यायः २९०}
\twolineshloka
{मार्कण्डेय उवाच}
{}


\twolineshloka
{तावुभौ पतितौ दृष्ट्वा भ्रातरौ रामलक्ष्मणौ}
{बबन्ध रावणिर्भूयः शरैर्दत्तवरैस्तदा}


\twolineshloka
{तौ वीरौ शरजालेन बद्धाविन्द्रजिता रणे}
{रेजतुः पुरुषव्याघ्रौ शकुन्ताविव पञ्जरे}


\twolineshloka
{दृष्ट्वा निपतितौ भूमौ सर्वाङ्गेषु शराचितौ}
{सुग्रीवः कपिभिः सार्धं परिवार्योपतस्तिवान्}


\twolineshloka
{सुषेणमैन्दद्विविदैः कुमुदेनाङ्गदेन च}
{हनुमननीलतारैश्च नलेन च कपीश्वरः}


\twolineshloka
{ततस्तं देशमागम्य कृतकर्मा विभीषणः}
{बोधयामास तौ वीरौ प्रज्ञास्त्रेण प्रमोहितौ}


\twolineshloka
{विशल्यौ चापि सुग्रीवः क्षणेनैतौ चकार ह}
{विशल्यया महौषध्या दिव्यमन्त्रप्रयुक्तया}


\twolineshloka
{तौ लब्धसंज्ञौ नृवरौ विशल्यावुदतिष्ठताम्}
{उभौ गतक्लमौ चास्तां णेनैतौ महारथौ}


\twolineshloka
{ततो विभीषणः पार्थ राममिक्ष्वाकुनन्दनम्}
{उवाच विज्वरं दृष्ट्वा कृताञ्जलिरिदं वचः}


\twolineshloka
{अयमम्भो गृहीत्वातु राजराजस् शासनात्}
{गुह्कोऽभ्यागतः श्लेतात्त्वत्सकाशमरिंदम}


\twolineshloka
{इदमम्भः कुबेरस्ते महाराज प्रयच्छति}
{अन्तर्हितानां भूतानां दर्शनार्थं परंतप}


\twolineshloka
{अनेन मृष्टनयनो भूतान्यन्तर्हितान्युत}
{भवान्द्रक्ष्यति यस्मै च भवानेतत्प्रदास्यति}


\twolineshloka
{तथेति रामस्तद्वारि प्रतिगृह्याभिसंस्कृतम्}
{चकार नेत्रयोः शौचं लक्ष्मणश्च महामनाः}


\twolineshloka
{सुग्रीवजाम्बवन्तौ चहनुमानङ्गदस्तथा}
{मैन्दद्विविदनीलाश्च प्रायः प्लवगसत्तमाः}


\twolineshloka
{तथासमभवच्चापि यदुवाच विभीषणः}
{क्षणेनातीन्द्रियाण्येषां चक्षुंष्यासन्युधिष्ठिर}


\twolineshloka
{इन्द्रजित्कृतकर्मा तु पित्रे कर्म तदाऽऽत्मनः}
{निवेद्य पुनरागच्छत्त्वरयाऽऽजिशिरःप्रति}


\twolineshloka
{तमागतं तु संक्रुद्धं पुनरेव युयुत्सया}
{अभिदुद्राव सौमित्रिर्विभीषणमते स्थितः}


\twolineshloka
{अकृताह्निकमेवैनं जिघांसुर्जितकाशिनम्}
{शरैर्जघान संक्रुद्धः कृतसंज्ञोऽथ लक्ष्मणः}


\twolineshloka
{तयोः समभवद्युद्धं तदाऽन्योन्यं जीगीषतोः}
{अतीव चित्रमाश्चर्यं शक्रप्रह्लादयोरिव}


\twolineshloka
{अविध्यदिन्द्रजित्तीक्ष्णैः सौमित्रिं मर्मभेदिभिः}
{सौमित्रिश्चानलस्पर्शैरविध्यद्रावणिं शरैः}


\twolineshloka
{सौमित्रिशरसंस्पर्शाद्रावणिः क्रोधमूर्च्छितः}
{असृजल्लक्ष्मणायाष्टौ शरानाशीविषोपमान्}


\twolineshloka
{तस्येषून्पावकस्पर्शैः सौमित्रिः पत्रिभिस्त्रिभिः}
{`वारयामास नाराचैः सौमित्रिर्मित्रनन्दनः}


\twolineshloka
{असृजल्लक्ष्मणश्चाष्टौ राक्षसाय शरान्पुनः'}
{तथा तं न्यहनद्वीरस्तन्मे निगदतः शृणु}


\twolineshloka
{एकेनास्य धनुष्मन्तं बाहुं देहादपातयत्}
{द्वितीयेन तु बाणेन भुजमन्यमपातयत्}


\twolineshloka
{तृतीयेन तु बाणेन शितधारेण भास्वता}
{जहार सुनसं चापि शिरो ज्वलितकुण्डलम्}


\twolineshloka
{विनिकृत्तभुजस्कन्धः कबन्धाकृतिदर्शनः}
{`पपात वसुधायां तु छिन्नमूल इवद्रुमः'}


\twolineshloka
{तं हत्वासूतमप्यस्त्रैर्जघान बलिनंवरः}
{लङ्कां प्रवेशयामासुस्तं रथं वाजिनस्तदा}


\threelineshloka
{ददर्श रावणस्तं च रथं पुत्रविनाकृतम्}
{स पुत्रं निहतं श्रुत्वा त्रासात्संभ्रान्तमानसः}
{}


% Check verse!
रावणः शोकमोहार्तो वैदेहीं हन्तुमुद्यतः ॥ङ्गमादाय दुष्टात्मा जवेनाभिपपात ह
\twolineshloka
{तं दृष्ट्वातस्य दुर्बुद्देरविन्ध्यः पापनिश्चयम्}
{शमयामास संक्रुद्धं श्रूयतां येन हेतुना}


\twolineshloka
{महाराज्येस्थितो दीप्ते न स्त्रियं हन्तुमर्हसि}
{हतैवैषा यदा स्त्री च कबन्धनस्था च ते वशे}


\twolineshloka
{न चैषा दहभेदेन हतास्यादिति मे मतिः}
{जहि भर्तारमेवास्या हते तस्मिन्हता भवेत्}


\twolineshloka
{न हि ते विक्रमे तुल्यः साक्षादपि शतक्रतुः}
{असकृद्धि त्वया सन्द्रास्त्रासितास्त्रिदसा युधि}


\twolineshloka
{एवं बहुविधैर्वाक्यैरविन्ध्यो रावणं तदा}
{क्रुद्धं संशमयामास जगृहे च स तद्वचः}


\twolineshloka
{निर्याणे स मतिं कृत्वा नियन्तारं क्षपाचरः}
{आज्ञापयामास तदारथो मे कल्प्यतामिति}


\chapter{अध्यायः २९१}
\twolineshloka
{मार्कण्डेय उवाच}
{}


\twolineshloka
{ततः क्रुद्धो दशग्रीवः प्रिये पुत्रे निपातिते}
{निर्ययौ रथमास्थाय हेमरत्नविभूषितम्}


\twolineshloka
{संवृतोराक्षसैर्घेरैर्विविधायुधपाणिभिः}
{अभिदुद्राव रामं स पोथयन्हरियूथपान्}


\twolineshloka
{तमाद्रवन्तं संक्रुद्ध मैन्दनीलनलाङ्गदाः}
{हनुमाञ्जाम्बवांश्चैव ससैन्याः पर्यवारयन्}


\twolineshloka
{ते दशग्रीवसैन्यं तदृक्षवानरपुङ्गवाः}
{द्रुमैर्विध्वंसयांचक्रुर्दशग्रीवस्य पश्यतः}


\twolineshloka
{ततः स्वसैन्यमालोक्य वध्यमानमरातिभिः}
{मायावी चासृजन्मायां रावणो राक्षसाधिपः}


\twolineshloka
{तस्य देहविनिष्क्रान्ताः शतशोऽथ सहस्रशः}
{राक्षसाः प्रत्यदृश्यन्त शरशक्त्यृष्टिपाणयः}


\twolineshloka
{तान्रामो जघ्निवान्सर्वान्दिव्येनास्त्रेण राक्षसान्}
{अथ भूयोपि मायां स व्यदधाद्राक्षसाधिपः}


\twolineshloka
{कृत्वा रामस् रूपाणि लक्ष्मणस्य च भारत}
{अभिदुद्राव रामं च लक्ष्मणं च दशाननः}


\twolineshloka
{ततस्ते राममर्च्छन्तो लक्ष्मणं च क्षपाचराः}
{अभिपेतुस्तदा रामं प्रगृहीतशरासनाः}


\twolineshloka
{तां दृष्ट्वाराक्षसेन्द्रस् मायामिक्ष्वाकुनन्दनः}
{उवाच रामः सौमित्रिमसंभ्रान्तो बृहद्वचः}


\twolineshloka
{जहीमान्राक्षसान्पापानात्मनः प्रतिरूपकान्}
{इत्युक्त्वाऽभ्यहनद्रामो लक्ष्मणश्चात्मरूपकान्}


\threelineshloka
{ततो हर्यश्वयुक्तेन रथेनादित्यवर्चसा}
{उपतस्थे रणे रामं मातलिः शक्रसारथिः ॥मातलिरुवाच}
{}


\twolineshloka
{अयं हर्यश्वयुग्जैत्रो मघोनः स्यन्दनोत्तमः}
{`त्वदर्थमिह संप्राप्तः संदेशाद्वै शतक्रतोः'}


\twolineshloka
{अनेन शक्रः काकुत्स्थ समरे दैत्यदानवान्}
{शतशः पुरुषव्याघ्र रथोदारेण जघ्निवान्}


\twolineshloka
{तदनन नरव्याघ्र मया यत्तेन संयुगे}
{स्यन्दनेन जहिक्षिप्रं रावणं मा चिरं कृथाः}


\twolineshloka
{इत्युक्तो राघवस्तथ्यं वचोऽशङ्कत मातलेः}
{मायैषाराक्षसस्येति तमुवाच विबीषणः}


\twolineshloka
{नेयं माया नरव्याघ्ररावणस् दुरात्मनः}
{तदातिष्ठ रथंशीघ्रमिमसैन्द्रं महाद्युते}


\twolineshloka
{ततः प्रहृष्टः काकुत्स्थस्तथेत्युक्त्वा विभीषणम्}
{रथेनाभिपपाताथ दशग्रीवं रुषाऽन्वितः}


\twolineshloka
{हाहाकुतानि भूतानि रावणे समभिद्रुते}
{सिंहनादाः सपटहादिति दिव्यास्तथाऽनदन्}


\twolineshloka
{[दशकन्धरराजसून्वोस्तथा युद्धमभून्महत्}
{अलब्धोपममन्यत्रतयोरेव तथाऽभवत् ॥]}


\twolineshloka
{सरामाय महाघोरं विससर्ज निशाचरः}
{शूलमिन्द्राशनिप्रख्यं ब्रह्मदण्डभिवोद्यतम्}


\twolineshloka
{तच्छूलं सत्वरं रामश्चच्छेद निशितैः शरैः}
{तद्दृष्ट्वा दुष्करं कर्म रावणं भयमाविशत्}


\twolineshloka
{ततः क्रुद्धः ससर्जाशु दशग्रीवः शिताञ्छरान्}
{सहस्रायुतशो रामे शस्त्राणि विविधानि च}


\twolineshloka
{ततो भुशुण्डीः शूलानि मुसलानि परश्वथान्}
{शक्तीश्च विविधाकाराः शतघ्नीश्च शितान्क्षुरान्}


\twolineshloka
{तां मायांविविधां दृष्ट्वा दशग्रीवस्य रक्षसः}
{भयात्प्रदुद्रुवुः सर्वे वानराः सर्वतोदिशम्}


\twolineshloka
{ततः सुपत्रं सुमुखंहेमपुङ्गं शरोत्तमम्}
{तूणादादाय काकुत्स्थो ब्रह्मास्त्रेण युयोज ह}


\twolineshloka
{तं प्रेक्ष्यबाणं रामेण ब्रह्मास्त्रेणानुमन्त्रितम्}
{जहृषुर्देवगन्धर्वा दृष्ट्वा शक्रपुरोगमाः}


\twolineshloka
{अल्पावशेषमायुश्च ततोऽमन्यन्त रक्षसः}
{ब्रह्मास्त्रोदीरणाच्छत्रोर्देवदानवकिंनराः}


\twolineshloka
{ततः ससर्ज तं रामः शरमप्रतिमौजसम्}
{रावणान्तकरं घोरं ब्रह्मदण्डमिवोद्यतम्}


\threelineshloka
{मुक्तमात्रेण रामेण दूराकृष्टेन भारत}
{स तेन राक्षसश्रेष्ठः सरथः साश्वसारथिः}
{प्रजज्वाल महाज्वालेनाग्निनाभिपरिप्लुतः}


\twolineshloka
{ततः प्रहृष्टास्त्रिदशाः सहगन्धर्वचारणाः}
{निहतं रावणं दृष्ट्वा रामेणाक्लिष्टकर्मणा}


\twolineshloka
{तत्यजुस्तं महाभागं पञ्चभूतानि रावणम्}
{भ्रंशितः सर्वलोकेषु स हि ब्रह्मास्त्रतेजसा}


\twolineshloka
{शरीरधातवो ह्यस् मासं रुधिरमेव च}
{नेशुर्ब्रह्मास्त्रनिर्दग्दा न च भस्माप्यदृश्यत}


\chapter{अध्यायः २९२}
\twolineshloka
{मार्कण्डेय उवाच}
{}


\twolineshloka
{स हत्वा रावयणं क्षुद्रं राक्षसेनद्रं सुरद्विषम्}
{बभूव हृष्टः ससुहृद्रामः सौमित्रिणा सह}


\twolineshloka
{ततो हते दशग्रीवे देवाः सर्षिपुरोगमाः}
{आशीर्भिर्जययुक्ताभिरानर्चुस्तं महाभुजम्}


\twolineshloka
{रामं कमलपत्राक्षं तुष्टुवुः सर्वदेवताः}
{गन्धर्वाः पुष्पवर्षैश्च वाग्भिश्च त्रिदशालयाः}


\twolineshloka
{पूजयित्वा रणे रामं प्रतिजग्मुर्यथागतम्}
{तन्महोत्सवसंकाशमासीदाकाशमच्युत}


\twolineshloka
{ततो हत्वा दशग्रीवं लङ्कां रामो महायशाः}
{विभीषणाय प्रददौ प्रभुः परपुरंजयः}


\twolineshloka
{ततः सीतां पुरस्कृत्य विभीषणपुरस्कृताम्}
{अविन्ध्यो नाम सुप्रज्ञो वृद्धामात्यो विनिर्ययौ}


\twolineshloka
{उवाच च महात्मानं काकुत्स्थं दैन्यमास्थितम्}
{प्रतीच्छ देवीं सद्वृत्तां महात्मञ्जानकीमिति}


\twolineshloka
{एतच्छ्रुत्वा वचस्तस्मादवतीर्य रथोत्तमात्}
{बाष्पेणापिहितां सीतां ददर्शेक्ष्वाकुनन्दनः}


\twolineshloka
{तां दृष्ट्वा चारुसर्वाङ्गीं यानस्थां शोककर्शिताम्}
{मलोपचितसर्वाङ्गीं जटिलां कृष्णवाससम्}


\twolineshloka
{उवाच रामो वैदेहीं परामर्शविशङ्कितः}
{`लक्षयित्वेङ्गितं सर्वं प्रियं तस्यै निवेद्य सः'}


\threelineshloka
{गच्छ वैदेहि मुक्ता त्वं यत्कार्यं तनमया कृतम्}
{मामासाद्यपतिं भद्रे न त्वं राक्षसवेश्मनि}
{जरां व्रजेथा इतिमे निहतोसौ निशाचरः}


\twolineshloka
{कथं ह्यस्मद्विधो जातु जानन्धर्मविनिश्चयम्}
{परहस्तगतां नारीं मुहूर्तमपि धारयेत्}


\twolineshloka
{सुवृत्तामसुवृत्तां वाऽप्यहं त्वामद्य मैथिलि}
{नोत्सहे परिभोगाय श्वावलीढं हविर्यथा}


\twolineshloka
{ततः सा सहसा बाला तच्छ्रुत्वा दारुणं वचः}
{पपात देवी व्यथिता निकृत्ता कदली यथा}


\twolineshloka
{योप्यस्या हर्षसंभूतो मुखरागः पुराऽभवत्}
{क्षणेन सपुनर्नष्टो निःश्वासादिव दर्पणे}


\twolineshloka
{ततस्ते हरयः सर्वे तच्छ्रुत्वा रामभाषितम्}
{गतासुकल्पा निश्चेष्टा बभूवुः सहलक्ष्मणाः}


\twolineshloka
{ततो देवो विशुद्धात्मा विमानेन चतुर्मुखः}
{पद्मयोनिर्जगत्स्रष्टा दर्शयामास राघवम्}


\twolineshloka
{शक्रश्चाग्निश्च वायुश्चयमो वरुण एव च}
{यक्षाधिपश्च भगवांस्तथा सप्तर्षयोऽमलाः}


\twolineshloka
{राजा दशरथश्चैव दिव्यभास्वरमूर्तिमान्}
{विमानेन महार्हेण हंसयुक्तेन भास्वता}


\twolineshloka
{ततोऽन्तरिक्षं तत्सर्वंदेवगन्धर्वसंकुलम्}
{शुशुभे तारकाचित्रं शरदीव नभस्तलम्}


\twolineshloka
{तत उत्थाय वैदेही तेषां मध्ययशस्विनी}
{उवाच वाक्यं कल्याणी रामं पृथुलवक्षसम्}


\twolineshloka
{राजपुत्र न ते कोपं करोमि विदिताहि मे}
{गतिः स्त्रीणां नराणां च शृणु चदं वचो मम}


\twolineshloka
{अन्तश्चरतिभूतानां मातरिश्वा सदागतिः}
{स मे विमुञ्चतु प्राणान्यदि पापं चराम्यहम्}


\twolineshloka
{अग्निरापस्तथाऽऽकाशं पृथिवी वायुरेव च}
{विमुञ्चन्तु मम प्राणान्यदि पापं चराम्यहम्}


\twolineshloka
{यथाऽहं त्वदृतेवीर नान्यंस्वप्नेऽप्यचिन्तयम्}
{तथा मे देव निर्दिष्टस्त्वमेव हि पतिर्भव}


\threelineshloka
{ततोऽन्तरिक्षे वागारीत्सुभगा लोकसाक्षिणी}
{पुण्यासंहर्षणी तेषां वानराणां महात्मनाम् ॥वायुरुवाच}
{}


\threelineshloka
{बोभो राघव सत्यं वै वायुरस्मि सदागतिः}
{अपापा मैथिली राजन्संगच्छसहभार्यया ॥अग्निरुवाच}
{}


\threelineshloka
{अहमन्तःशरीरस्थो भूतानां रघुनन्दन}
{सुसूक्ष्ममपि काकुत्स्थ मैथिलीनापराध्यति ॥वरुण उवाच}
{}


\threelineshloka
{रसावै मत्प्रसूता हि भूतदेहेषु राघव}
{अहंवै त्वां प्रब्रवीमि मैथिली प्रतिगृह्यताम् ॥यम उवाच}
{}


\threelineshloka
{`धर्मोऽहमस्मि काकुत्स्थ साक्षी लोकस्य कर्मणाम्}
{शुभाशुभानां सीतेयमपापा प्रतिगृह्यताम्' ॥ब्र्हमोवाच}
{}


\twolineshloka
{पुत्र नैतदिहाश्चर्यं त्वयि राजर्षिधर्मणि}
{साधो सद्वृत्त काकुत्स्थ शृणु चेदं वचो मम}


\twolineshloka
{शत्रुरेष त्वया वीर देवगनधर्वभोगिनाम्}
{यक्षाणां दानवानां च महर्षीणां च पातितः}


\twolineshloka
{अवध्यः सर्वभूतानां मत्प्रसादात्पुराऽभवत्}
{कस्माच्चित्कारणात्पापः कंचित्कालमुपेक्षितः}


\twolineshloka
{वधार्थमात्मनस्तेन हृता सीता दुरात्मना}
{नलकूबरशापेन रक्षा चास्याः कृता मया}


\twolineshloka
{यदि ह्यकामामासेवेत्स्तरियमन्यामपि ध्रुवम्}
{शतधाऽस्य फलेन्मूर्धा इत्युक्तः सोभवत्पुरा}


\threelineshloka
{नात्रशङ्का त्वया कार्या प्रतीच्छेमां महामते}
{कृतं त्वया महत्कार्यं देवानाममितप्रभ ॥दशरथ उवाच}
{}


\threelineshloka
{प्रीतोस्मि वत्स भद्रं ते पिता दशरथोस्मि ते}
{अनुजानामि राज्यं च प्रशाधि पुरुषोत्तम ॥राम उवाच}
{}


\threelineshloka
{अभिवादयेत्वां राजेन्द्र यदि त्वं जनको मम}
{गमिष्यामि पुरीं रम्यामयोध्यां शासनात्तव ॥मार्कण्डेय उवाच}
{}


\threelineshloka
{तमुवाच पिता भूयः प्रहृष्टो भरतर्षभ}
{गच्छायोध्यां प्रशाधि त्वंराम रक्तान्तलोचन}
{संपूर्णानीहवर्षाणि चतुर्दश महाद्युते}


\twolineshloka
{ततो देवान्नमस्कृत्य मुहृद्भिरभिनन्दितः}
{महेन्द्रइव पौलोम्या भार्यया स समेयिवान्}


\twolineshloka
{ततो वरं ददौ तस्मै ह्यविन्ध्याय परंतपः}
{त्रिजटां चार्थमानाभ्यां योजयामास राक्षसीम्}


\twolineshloka
{तमुवाच ततो ब्रह्मा देवैः शक्रषुरोगमैः}
{कौसल्यामातरिष्टांस्ते वरानद्य ददानि कान्}


\twolineshloka
{वव्रेरामः स्थितिं धर्मे शत्रुभिश्चापराजयम्}
{राक्षसैर्निहतानां च वानराणां समुद्भवम्}


\twolineshloka
{ततस्ते ब्रह्मणा प्रोक्ते तथेतिवचने तदा}
{समुत्तस्थुर्महाराज वानरा लब्धचेतसः}


\twolineshloka
{सीता चापि महाभागा वरं हनुमते ददौ}
{रामकीर्त्या समं पुत्र जीवितं ते भविष्यति}


\twolineshloka
{दिव्यास्त्वामुपभोगाश्च मत्प्रसादकृताः सदा}
{उपस्थास्यन्ति हनुमन्निति स्म हरिलोचन}


\twolineshloka
{ततस्ते प्रेक्षमाणानां तेपामक्लिष्टकर्मणाम्}
{अन्तर्धानं ययुर्देवाः सर्वे शक्रपुरोगमाः}


\twolineshloka
{दृष्ट्वा रामं तु जानक्या संगतं शक्रसारथिः}
{उवाच परमप्रीतसुहृन्मध्य इदं वचः}


\twolineshloka
{देवगन्धर्वयक्षाणां मानुषासुरभोगिनाम्}
{अपनीतं त्वया दुःखमिदं सत्यपराक्रम}


\twolineshloka
{सदेवासुरगनधर्वा यक्षराक्षसपन्नगाः}
{कथयिष्यन्ति लोकास्त्वां यावद्भूमिर्धरिष्यति}


\twolineshloka
{इत्येवमुक्त्वाऽनुज्ञाप्यरामं शस्त्रभृतांवरम्}
{संपूज्यापाक्रमत्तेन रथेनादित्यवर्चसा}


\twolineshloka
{ततःसीतां पुरस्कृत्य रामः सौमित्रिणा सह}
{सुग्रीवप्रमुखैश्चैव सहितः सर्ववानरैः}


\twolineshloka
{विधाय रक्षां लङ्कायां विभीषणपुरस्कृतः}
{संततार पुनस्तेन सेतुना मकरालयम्}


\twolineshloka
{पुष्पकेण विसानन खेचरेण विराजता}
{कामगेन यथामुख्यैरमात्यैः संवृतो वसी}


\twolineshloka
{ततस्तीरे समुद्रस्यं यत्रशिश्य स पार्थिवः}
{तत्रैवोवास धर्मात्मा सहितः सर्ववानरैः}


\twolineshloka
{अथैनान्राघवः काले समानीयाभिपूज्य च}
{विसर्जयामास तदा रत्नैः संतोष्य सर्वशः}


\twolineshloka
{गतेषु वानरेन्द्रेषु गोपुच्छर्क्षेषु तेषु च}
{सुग्रीवसहितो रामः किष्किन्दां पुनरागमत्}


\twolineshloka
{विभीषणेनानुगतः सुग्रीवसहितस्तदा}
{पुष्पकेण विमानेन वैदेह्या दर्शयन्वनम्}


\twolineshloka
{किष्किन्धां तु समासाद्यरामः प्रहरतांवरः}
{अङ्गदं कृतकर्माणं यौवराज्येऽभ्यषेचयत्}


\twolineshloka
{ततस्तैरेव सहितो रामः सौमित्रिणा सह}
{यथागतेन मार्गेण प्रययौ स्वपुरं प्रति}


\twolineshloka
{अयोध्यां स समासाद्यपुरीं राष्ट्रपतिस्ततः}
{भरताय हनूमन्तं दूतं प्रास्थापयद्द्रुतम्}


\twolineshloka
{लक्षयित्वेङ्गितं सर्वंप्रियं तस्मै निवेद्य वै}
{वायुपुत्रे पुनः प्राप्ते नन्दिग्राममुपाविशत्}


\threelineshloka
{सतत्रमलदिग्धाङ्गं भरतं चीरवाससम्}
{`नन्दिग्रामगतंरामः सशत्रुघ्नं सराघवः'}
{अग्रतःपादुके कृत्वा ददर्शासीनमासने}


\twolineshloka
{समेत्यभरतेनाथ शत्रुघ्नेन च वीर्यवान्}
{राघवः सहसौमित्रिर्मुमुदे भरतर्षभ}


\twolineshloka
{ततो भरतशत्रुघ्नौ समेतौ गुरुणा तदा}
{वैदेह्या दर्शनेनोभौ प्रहर्षं समवापतुः}


\twolineshloka
{तस्मै तद्भरतो राज्यमागतायातिसत्कृतम्}
{न्यासं निर्यातयामास युक्तः परमया मुदा}


\twolineshloka
{ततस्तं वैष्णवे शूरं नक्षत्रेऽभिजितेऽहनि}
{वसिष्ठो वामदेवश्च सहितावभ्यषिञ्चताम्}


\twolineshloka
{सोभिषिक्तः कपिश्रेष्ठं सुग्रीवं ससुहृज्जनम्}
{विभीषणं च पौलस्त्यमन्वजानाद्गृहान्प्रति}


\twolineshloka
{अभ्यर्च्य विविधै रत्नैः प्रीतियुक्तौ मुदा युतौ}
{समाधायेतिकर्तव्यं दुःखेन विससर्ज ह}


\twolineshloka
{पुष्पकं च विमानं तत्पूजयित्वा स राघवः}
{प्रादाद्वैश्रवणायैव प्रीत्या स रघुनन्दनः}


\twolineshloka
{ततो देवर्षिसहितः सरितं गोमतीमनु}
{शताश्वमेधानाजह्रे जारूथ्यान्स निरर्गलान्}


\chapter{अध्यायः २९३}
\twolineshloka
{मार्कण्डेय उवाच}
{}


\twolineshloka
{एवमेतन्महाबाहो रामेणामिततेजसा}
{प्राप्तं व्यसनमत्युग्रं वनवासकृतं पुरा}


\twolineshloka
{मा शुचः परुषव्याघ्र क्षत्रियोसि परंतप}
{बाहुवीर्याश्रयेमार्गे वर्तसे दीप्तनिर्णये}


\twolineshloka
{न हि ते वृजिनं किंचिद्दृश्यते परमण्वपि}
{अस्मिन्मार्गे निपीदेयुः सेन्द्रा अपि सुरासुराः}


\twolineshloka
{संहत्य निहतोवृत्रो मरुद्भिर्वज्रपाणिना}
{नमुचिश्चैवदुर्धर्षो दीर्गजिह्वा चराक्षसी}


\twolineshloka
{सहायवति सर्वार्थाः सतिष्ठन्तीह सर्वशः}
{किंनु तस्याजितं सङ्ख्ये यस् भ्राता धनंजयः}


\twolineshloka
{अयं च बलिनांश्रेष्ठो भीमो भीमपराक्रमाः}
{युवानौ च महेष्वासौ वीरौ माद्रवतीसुतौ}


\twolineshloka
{एभिः सहायैः कस्मात्त्वं विषीदसि परंतप}
{य इमे वज्रिणः सेनां जयेयुः समरुद्गणाम्}


\twolineshloka
{त्वमप्येभिर्महेष्वासैः सहायैर्देवरूपिभिः}
{विजेष्यसि रणे सर्वानमित्रान्भरतर्षभ}


\twolineshloka
{इतश्च त्वमिमां पश्यसैन्धवेन दुरात्मना}
{बलिना वीर्यमत्तेन हृतामेभिर्महात्मभिः}


\twolineshloka
{आनीतां द्रौपदीं कृष्णां कृत्वा कर्म सुदुष्करम्}
{जयद्रथं च राजानं विजितं वशमागतम्}


\twolineshloka
{असहायेन रामेण वैदेही पुनराहृता}
{हत्वासङ्ख्ये दशग्रीवं राक्षसं भीमविक्रमम्}


\twolineshloka
{यस् शाखामृगामित्राण्यृक्षाः कालमुखास्तथा}
{जात्यन्तरगता राजन्नेतद्बुद्ध्याऽनुचिन्तय}


\threelineshloka
{तस्मात्सर्वं कुरुश्रेष्ठ मा शुचो भरतर्षभ}
{त्वद्विधा हि महात्मानो न शोचन्ति परंतप ॥वैशंपायन उवाच}
{}


\twolineshloka
{एवमाश्वासितो राजामार्कण्डेयेन धीमता}
{त्यक्त्वा दुःखमदीनात्मा पुनरप्येनमब्रवीत्}


\chapter{अध्यायः २९४}
\twolineshloka
{युधिष्ठिर उवाच}
{}


\twolineshloka
{नात्मानमनुशोचामि नेमान्भ्रातॄन्महामुने}
{हरणं चापि राज्यस् यथेमां द्रुपदात्मजाम्}


\twolineshloka
{द्यूते दुरात्मभिः क्लिष्टाः कृष्णया तारिता वयम्}
{जयद्रथेन चपुनर्वनाच्चापि हृता बलात्}


\threelineshloka
{अस्ति सीमन्तिनी काचिद्दृष्टपूर्वाऽपिवा श्रुता}
{पतिव्रता महाभागा यथेयं द्रुपदात्मजा ॥मार्कण्डेय उवाच}
{}


\twolineshloka
{शृणु राजन्कुलस्त्रीणां महाभाग्यं युधिष्ठिर}
{सर्वमेतद्यथाप्राप्तं सावित्र्या राजकन्यया}


\twolineshloka
{आसीन्मद्रेषु धर्मात्मा राजा परमधार्मिकः}
{ब्रह्मण्यश्चमहात्मा च सत्यसन्धो जितेन्द्रियः}


\twolineshloka
{यज्वा दानपतिर्दक्षः पौरजानपदप्रियः}
{पार्थिवोऽश्वपतिर्नाम सर्वभूतहिते रतः}


\twolineshloka
{क्षमावाननपत्यश्च सत्यवाग्विजितेन्द्रियः}
{अतिक्रान्तेन वयसा संतापमुपजग्मिवान्}


\twolineshloka
{अपत्योत्पादनार्थं च तीव्रं नियममास्थितः}
{काले परिमिताहारो ब्रहमचारी जितेन्द्रियः}


\twolineshloka
{हुत्वा शतसहस्रं स सावित्र्या राजसत्तम}
{षष्ठेषष्ठे तदाकाले बभूव मितमोजनः}


\twolineshloka
{एतेन नियमेनासीद्वर्षाण्यष्टादशैव तु}
{पूर्णे त्वष्टादशे वर्षे सावित्री तुष्टिमभ्यगात्}


\twolineshloka
{रूपिणी तु तदा राजन्दर्शयामास तं नृपम्}
{अग्निहोत्रात्समुत्थाय हर्षेण महताऽन्विता}


\twolineshloka
{उवाच चैनं वरदा वचनं पार्थिवं तदा}
{सा तमश्वपतिं राजन्सावित्री नियमे स्थितम्}


\twolineshloka
{ब्रह्मचर्येण शुद्धेन दमेन नियमेन रच}
{सर्वात्मना च भक्त्या च तुष्टाऽस्मि तव पार्थिवाः}


\threelineshloka
{वरं वृणीष्वाश्वपते मद्रराज यदीप्सितम्}
{न प्रामादश्च धर्मेषु कर्तव्यस्ते कथंचन ॥अश्वपतिरुवाच}
{}


\twolineshloka
{अपत्यार्थः समारम्भः कृतो धर्मेप्सया मया}
{पुत्रा मे बहवो देवि भवेयुः कुलभावनाः}


\threelineshloka
{तुष्टाऽसि यदि मे देवि वरमेतं वृणोम्यहम्}
{संतां परमो धर्म इत्याहुर्मां द्विजातयः ॥सावित्र्युवाच}
{}


\twolineshloka
{पूर्वमेव मया राजन्नभिप्रायमिमं तव}
{ज्ञात्वा पुत्रार्थमुक्तो वै भगवांस्ते पितामहः}


\twolineshloka
{प्रसादाच्चैव तस्मात्ते स्वयं विहितवत्यहम्}
{कन्या तेजस्विनी सौम्य क्षिप्रमेव भविष्यति}


\twolineshloka
{उत्तरं च न ते किंचिद्व्याहर्तव्यं कथंचन}
{पितामहनियोगेन तुष्टा ह्येतद्ब्रवीमि ते}


\twolineshloka
{स तथेति प्रतिज्ञाय सावित्र्या वचनं नृपः}
{प्रसादयामास पुनः क्षिप्रमेतद्भविष्यति}


\twolineshloka
{अन्तर्हितायां सावित्र्यां जगाम स्वपुरं नृपः}
{स्वराज्ये चावसद्बीरः प्रजा धर्मेण पालयन्}


\twolineshloka
{कस्मिंश्चित्तु गते काले स राजा नियतब्रतः}
{ज्येष्ठायां धर्मचारिण्यां महिष्यां कगर्भमादधे}


\twolineshloka
{राजपुत्र्यास्तु गर्भः स मानव्या भरतर्षभ}
{व्यर्धतं तदा शुक्ले तारापतिरिवाम्बरे}


\threelineshloka
{प्राप्ते काले तु सुषुवे कन्यां राजीवलोचनाम्}
{क्रियाश्च तस्या मुदितश्चक्रे च नृपसत्तमः ॥ 3-294-25aसावित्र्या प्रीतया दत्ता साव्त्र्या द्दुतया ह्यपि}
{सावित्रीत्येव नामास्याश्चक्रुर्विप्रास्तथा पिता}


\twolineshloka
{सा विग्रहवतीव श्रीव्यवर्धत नृपात्मजा}
{कालेन चापि सा कन्या यौवनस्ता बभूव ह}


\twolineshloka
{तां सुमध्यां पृथुश्रोणीं प्रतिमां काञ्चनीमिव}
{प्राप्तेयं रदेवकन्येति दृष्ट्वा संमेनिरे जनाः}


\twolineshloka
{तां तु पद्मपलाशाक्षीं ज्वलन्तीमिव तेजसा}
{न कश्चिद्वरयामास तेजसा प्रतिवारितः}


\twolineshloka
{अथोपोष्य शिरःस्नाता देवतामभिगम्य सा}
{हुत्वाग्निं विधिवद्विप्रान्वाचयामास पर्वणि}


\twolineshloka
{ततः सुमनसः शेषाः प्रतिगृह्य महात्मनः}
{पितुः समीपमगमद्देवी श्रीरिव रूपिणी}


\twolineshloka
{साऽभिवाद्य पितुः पादौ शेषाः पूर्वं निवेद्य च}
{कृताञ्जलिर्वरारोहा नृपतेः पार्श्वमास्थिता}


\threelineshloka
{यौवनस्थां तु तां दृष्ट्वा स्वां सुतां देवरूपिणीम्}
{अयाच्यमानां च वरैर्नृपतिर्दुःखितोऽभवत् ॥राजोवाच}
{}


\twolineshloka
{पुत्रि प्रदानकालस्ते न च कश्चिद्वृणोति माम्}
{स्वयमन्विच्छ भर्तारं गुणैः सदृशमात्मनः}


\twolineshloka
{प्रार्थितः पुरुषो यश्च स निवेद्यस्त्वया मम}
{विमृश्याहं प्रदास्यामि वरय त्वं यथेप्सितम्}


\twolineshloka
{श्रुतं हि धर्मशास्त्रेषु पठ्यमानं द्विजातिभिः}
{तथा त्वमपिकल्याणि गदतो मे वचः शृणु}


\twolineshloka
{अप्रदाता पिता वाच्यो वाच्यश्चानुपयन्पतिः}
{मृते पितरि पुत्रश्च वाच्यो मातुररक्षिता}


\twolineshloka
{इदं मे वचनं क्षुत्वा भर्तुरन्वेषणे न्वर}
{देवतानां यथा याच्यो न भवेयं तथा कुरु}


\twolineshloka
{एवमुक्त्वा दुहितरं तथा वृद्धांश्च मन्त्रिणः}
{व्यादिदेशानुयात्रं च गम्यतां चेत्यचोदयत्}


\twolineshloka
{साऽभिवाद्य पितुः पादौ व्रीडितेव मनस्विनी}
{पितुर्वचनमाज्ञाय निर्जगामाविचारितम्}


\twolineshloka
{सा हैमं रथमास्थाय स्थविरैः सचिवैर्वृता}
{तपोवनानिरम्याणि राजर्षीणां जगाम ह}


\twolineshloka
{मान्यानां तत्र वृद्धानां कृत्वा पादाभिवादनम्}
{वनानि क्रमशस्तात सर्वाण्येवाभ्यगच्छत}


\twolineshloka
{एवं तीर्थेषु सर्वेषु धनोत्सर्गं नृपात्मजा}
{कुर्वी द्विजमुख्यानां तंतं देशं जगाम ह}


\chapter{अध्यायः २९५}
\twolineshloka
{मार्कण्डेय उवाच}
{}


\twolineshloka
{अथ मद्राधिपो राजा नारदेन समागतः}
{उपविष्टः सभामध्ये कथायोगेन भारत}


\twolineshloka
{ततोऽभिगम्य तीर्थानि सर्वाण्येवाश्रमांस्तथा}
{आजगाम पितुर्वेश्म सावित्री सह मन्त्रिभिः}


\threelineshloka
{नारदेन सहासीनं सा दृष्ट्वा पितरं शुभा}
{उभयोरेव शिरसा चक्रे पादाभिवादनम् ॥नारद उवाच}
{}


\threelineshloka
{क्व गताऽभूत्सुतेयं ते कुतश्चैवागता नृप}
{किमर्थं युवतीं भद्र न चैनां संप्रयच्छसि ॥अश्वपतिरुवाच}
{}


\threelineshloka
{कार्येण खल्वनेनैव प्रेषिताद्यैव चागता}
{एतस्याः शृणु देवर्षे भर्तारं योऽनया वृतः ॥मार्कण्डेय उवाच}
{}


\twolineshloka
{सा ब्रूहि विस्तरेणेति पित्रा संयोदिता शुभा}
{तदैव तस्य वचनं प्रतिगृह्येदमब्रवीत्}


\twolineshloka
{आसीत्साल्वेषु धर्मात्मा क्षत्रियः पृथिवीपतिः}
{द्युमत्सेन इति ख्यातः पश्चाच्चान्धो बभूव ह}


\twolineshloka
{विनष्टचक्षुषस्तस्य बालपुत्रस्य धीमतः}
{सामीप्येन हृतं राज्यं छिद्रेऽस्मिन्पूर्ववैरिणा}


\twolineshloka
{स बालवत्सया सार्धं भार्यया प्रस्थितो वनम्}
{महारण्यं गतश्चापि पस्तेषे महाव्रतः}


\threelineshloka
{तस्य पुत्रः पुरे जातः संवृद्धश्च तपोवने}
{सत्यवाननुरूपो मे भर्तेति मनसा वृतः ॥नारद उवाच}
{}


\twolineshloka
{अहो वत महत्पापं सावित्र्या नृपते कृतम्}
{अजानन्त्या यदनया गुणवान्सत्यवान्वृतः}


\twolineshloka
{सत्यं वदत्यस्य पिता सत्यं माता प्रभाषते}
{तथाऽस् ब्राह्मणाश्चक्रुर्नामैतत्सत्यवानिति}


\threelineshloka
{बालस्याश्वाः प्रियाश्चास्य करोत्यश्वांश्च मृन्मयान्}
{चित्रेऽपि विलिखत्यश्वांश्चित्राश्व इति चोच्यते ॥राजोवाच}
{}


\threelineshloka
{अपीदानीं स तेजस्वी बुद्धिमान्वा नृपात्मजः}
{क्षमावानपि वा शृरः सत्यवान्पितृवत्सलः ॥नारद उवाच}
{}


\threelineshloka
{विवस्वानिव तेजस्वी वृहस्पतिसमो मतौ}
{महेन्द्र इवन वीरश्च वसुधेव क्षमान्वितः ॥अश्वपतिरुवाच}
{}


\threelineshloka
{अपि राजात्मजो दाता ब्रह्मण्यश्चापि सत्यवान्}
{रूपवानप्युदारो वाऽप्यथवा प्रियदर्शनः ॥नारद उवाच}
{}


\twolineshloka
{सांकृते रन्तिदेवस् स्वशक्त्या दानतः समः}
{ब्रह्मण्यः सत्यवादी च शिबिरौशीनरो यथा}


\twolineshloka
{ययातिरिव चोदारः सोमवत्प्रियदर्शनः}
{रूपेणान्यतमोऽश्विभ्यां द्युमत्सेनसुतो बली}


\threelineshloka
{`स वदान्यः स तेजस्वीधीमांश्चैव क्षमान्वितः'}
{स दान्तः स मृदुः शूरः स सत्यः संयतेन्द्रियः}
{सन्मैत्रः सोनसूयश्च स ह्रीमान्द्युतिमांश्च सः}


\threelineshloka
{नित्यशश्चार्जवं तस्मिन्धृतिस्तत्रैव च ध्रुवा}
{संक्षेपतस्तपोवृद्धैः शीलवृद्धैश्च कथ्यते ॥अश्वपतिरुवाच}
{}


\threelineshloka
{गुणैरुपेतं सर्वैस्तं भगवन्प्रब्रवीषि मे}
{दोषानप्यस्य मे ब्रूहि यदि सन्तीह केचन ॥नारद उवाच}
{}


\twolineshloka
{एक एवास्य दोषो हि गुणानाक्रम्य तिष्ठति}
{स च दोषः प्रयत्नेन न शक्यमतिवर्तितुम्}


\threelineshloka
{एको दोषोऽस्ति नान्योऽस्य सोद्यप्रभृति सत्यवान्}
{संवत्सरेण क्षीणायुर्देहन्यासं करिष्यति ॥राजोवाच}
{}


\twolineshloka
{एहि सावित्रि गच्छस्व अन्यं वरय शोभने}
{तस्य दोषो महानेको गुणानाक्रम्य च स्थितः}


\threelineshloka
{यथा मे भगवानाह नारदो देवसत्कृतः}
{संवत्सरेण सोऽल्पायुर्देहन्यासं करिष्यति ॥सावित्र्युवाच}
{}


\twolineshloka
{सकृदंशो निपतति सकृत्कन्या प्रदीयते}
{स कृदाह ददानीति त्रीण्येतानि सकृत्सकृते}


\twolineshloka
{दीर्घायुरथवाऽल्पायुः सगुणो निर्गुणोऽपि वा}
{सकृद्वृतो मया भर्ता न द्वितीयं वृणोम्यहम्}


\threelineshloka
{मनसा निश्चयं कृत्वाततो वाचाऽभिधीयते}
{क्रियते कर्मणा पश्चात्प्रमाणं मे मनस्ततः ॥नारद उवाच}
{}


\twolineshloka
{स्थिरा बुद्धिर्नरश्रेष्ठ सावित्र्या दुहितुस्तव}
{नैषा वारयितुं शक्या धर्मादस्मात्कथंचन}


\threelineshloka
{नान्यस्मिन्पुरुषे सन्ति ये सत्यवति वै गुणाः}
{प्रदानमेव तस्मान्मे रोचते दुहितुस्तव ॥राजोवाच}
{}


\threelineshloka
{अविचाल्यमेतदुक्तं तथ्यं च भवता वचः}
{करिष्याम्येतदेवं च गुरुर्हि भगवान्मम ॥नारद उवाच}
{}


\threelineshloka
{अविघ्नमस्तु सावित्र्याः प्रदाने दुहितुस्तव}
{साधयिष्याम्यहं तावत्सर्वेषां भद्रमस्तु वः ॥मार्कण्डेय उवाच}
{}


\twolineshloka
{एवमुक्त्वा स्वमुत्पत्य नारदस्त्रिदिवं गतः}
{राजाऽपि दुहितुः सज्जं वैवाहिकमकारयत्}


\chapter{अध्यायः २९६}
\twolineshloka
{मार्कण्डेय उवाच}
{}


\twolineshloka
{अथ कन्याप्रदाने स तमेवार्थं विचिन्तयन्}
{समानित्ये च तत्सर्वंभाण्डं वैवाहिकं नृपः}


\twolineshloka
{ततो वृद्धान्द्विजान्सर्वानृत्विक्सभ्यपुरोहितान्}
{समाहूय दिने पुण्ये प्रययौ सह कन्यया}


\twolineshloka
{मेध्यारण्यं स गत्वा च द्युमत्सेनाश्रमं नृपः}
{पद्भ्यामेव द्विजैः सार्धं राजर्षिं तमुपागमत्}


\twolineshloka
{तत्रापश्यन्महाभागं सालवृक्षमुपाश्रितम्}
{कौश्यां बृस्यां समासीनं चक्षुर्हीनं नृपं तदा}


\twolineshloka
{स राजा तस्य राजर्षेः कृत्वापूजां यथाऽर्हतः}
{वाचा सुनियतो भूत्वा चकारात्मनिवेदनम्}


\twolineshloka
{तस्यार्ध्यमासं चैव गां चावेद्यस धर्मवित्}
{किमागमनमित्येवं राजा राजानमब्रवीत्}


\twolineshloka
{तस्य सर्वमभिप्रायमितिकर्तव्यतां च ताम्}
{सत्यवन्तं समुद्दिश्य सर्वमेव न्यवेदयत्}


\threelineshloka
{सावित्री नाम राजर्षे कन्येयं मम शोभना}
{तां स्वधर्मेण धर्मज्ञ स्नुषार्थे त्वं गृहाण मे ॥द्युमत्सेन उवाच}
{}


\threelineshloka
{च्युताः स्म राज्याद्वनवासमाश्रिता-श्चराम धर्मं नियतास्तपस्विनः}
{कथं त्वनर्हा वनवासमाश्रमेसहिष्यति क्लेशमिमं सुता तव ॥अश्वमतिरुवाच}
{}


\twolineshloka
{सुखं च दुःखं च भवाभवात्मकंयदा विजानाति सुताऽहमेव च}
{न मद्विधे युज्यतेवाक्यमीदृशंविनिश्चयेनाभिगतोस्मि ते नृप}


\twolineshloka
{आशां नार्हसि मे हन्तुं सौहृदात्प्रणतस्य च}
{अभितश्चागतं प्रेम्णा प्रत्याख्यातुं न माऽर्हसि}


\threelineshloka
{अनुरूपो हि युक्तश्च त्वं ममाहं तवापि च}
{स्नुषां प्रतीच्छ मे कन्यां भार्यां सत्यवतस्ततः ॥द्युमत्सेन उवाच}
{}


\twolineshloka
{पूर्वमेवाभिलवितः संबन्धो मे त्वया सह}
{भ्रष्टराज्यस्त्वहमिति तत एतद्विचारितम्}


\twolineshloka
{अभिप्रायस्त्वयं यो मे पूर्वमेवाभिकाङ्क्षितः}
{स निर्वर्ततु मेऽद्यैव काङ्क्षितो ह्यसि मेऽतिथिः}


\twolineshloka
{ततः सर्वान्समानाय्य द्विजानाश्रमवासिनः}
{यथाविधि समुद्वाहं कारयामासतुर्नृपौ}


\twolineshloka
{दत्त्वा सोऽश्वपतिः कन्यां यथार्हं सपरिच्छदम्}
{ययौ स्वमेव भवनं युक्तः परमया मुदा}


\twolineshloka
{सत्यवानपि तां भार्यां लब्ध्वा सर्वगुणान्विताम्}
{मुमुदे सा रच रतं लब्ध्वा भर्तारं मनसेप्सितम्}


\twolineshloka
{गते पितरि सर्वाणि संन्यस्याभरणानि सा}
{जगृहेवल्कलान्येव वस्त्रं काषायमेव च}


\twolineshloka
{परिचारैर्गुणैश्चैव प्रश्रयेण दमेन च}
{सर्वकामक्रियाभिश्च सर्वेषां तुष्टिमादधे}


\twolineshloka
{श्वश्रूं शरीरसत्कारैः सर्वैराच्छादनादिभिः}
{श्वशुरं देवसत्कारैर्वाचः संयमनेन च}


\twolineshloka
{तथैव प्रियवादेन नैषुणेन शमेन च}
{रहश्चैवोपचारेण भर्तारं पर्यतोषयत्}


\twolineshloka
{एवं तत्राश्रमे तेषां तदा निवसतां सताम्}
{कालस्तपस्यतां कश्चिदपाक्रामत भारत}


\twolineshloka
{सावित्र्याग्लायमानायास्तिष्ठन्त्यास्तु दिवानिशम}
{नारदेन यदुक्तं तद्वाक्यं मनसि वर्तते}


\chapter{अध्यायः २९७}
\twolineshloka
{मार्कण्डेय उवाच}
{}


\twolineshloka
{ततः काले बहुतिथे व्यतिक्रान्ते कदाचन}
{प्राप्तः स कालो मर्व्यं यत्रसत्यवता नृप}


\twolineshloka
{गणयन्त्याश्च सावित्र्या दिवसदिवसे गते}
{यद्वाक्यं नारदेनोक्तं वर्तते हृदि नित्यशः}


\twolineshloka
{चतुर्थेऽहनि मर्तव्यमिति संचिन्त्य भामिनी}
{व्रतं त्रिरात्रमुद्दिश्य दिवारात्रं स्थिताऽभवत्}


\twolineshloka
{`त्रयोदश्यां चोपवासं प्रतिपत्सु च पारणम्}
{आयुष्यं वर्धते भर्तुर्व्रतेनानि भारत'}


\twolineshloka
{तं श्रुत्वा रकनियमं तस्या भृशं दुःखान्वितो नृपः}
{उत्थाय वाक्यं सावित्रीमब्रवीत्परिसान्त्वयन्}


\threelineshloka
{अतितीव्रोऽयमारम्भस्त्वयाऽऽरब्धो नृपात्मजे}
{तिसृणां वसतीनां हि स्तानं परमदुश्चरम् ॥सावित्र्युवाच}
{}


\threelineshloka
{न कार्यस्तात संतापः पारयिष्याम्यहं व्रतम्}
{व्यसंसायकृतं हीदं व्यवसायश्च कारणम् ॥द्युमत्सेन उवाच}
{}


\threelineshloka
{व्रतं भिन्धीति वक्तुं त्वां नास्मि शक्तः कथंचन}
{पारयस्वेति वचनं युक्तमस्मद्विधो वदेत् ॥मार्कण्डेय उवाच}
{}


\twolineshloka
{एवमुक्त्वा द्युमत्सेनो विरराम महामनाः}
{तिष्ठन्ती चैव सावित्री काण्ठभूतेव लक्ष्यते}


\twolineshloka
{श्वोभूते भर्तृमरणे सावित्र्या भरतर्षभ}
{दुःखान्वितायास्तिष्ठन्त्याः सा रात्रिर्व्यत्यवर्तत}


\twolineshloka
{अद्य तद्दिवसं चेति हुत्वा दीप्तं हुताशनम्}
{युगमात्रोदिते सूर्येकृत्वा पौर्वाङ्णिकीः क्रियाः}


\threelineshloka
{`व्रतं समाप्यसावित्री स्नात्वा शुद्धा यशस्विनी'}
{ततः सर्वान्द्विजान्वृद्धाञ्श्वश्रूं श्वशुरमेव च}
{अभिवाद्यानुपूर्व्येण प्राञ्जलिर्नियता स्थिता}


\twolineshloka
{अवैधव्याशिषस्ते तु सावित्र्यर्थं हिताः शुभाः}
{ऊचुस्तपस्विनः सर्वे तपोवननिवासिनः}


\twolineshloka
{एवमस्त्विति सावित्री ध्यानयोगपरायणा}
{मनसा ता गिरः सर्वाः प्रत्यगृह्णात्तपस्विनी}


\twolineshloka
{तं कालं तं मुहूर्तं च प्रतीक्षन्ती नृपात्मजा}
{यथोक्तं नारदवचश्चिन्तयन्ती सुदुःखिता}


\twolineshloka
{ततस्तु श्वश्रूश्वशुरावूचतुस्तां नृपात्मजाम्}
{एकान्तमास्थितां वाक्यं प्रीत्या भरतसत्तम}


\threelineshloka
{व्रतं यथोपदिष्टं तु तथा तत्पारितं त्वया}
{आहारकालः संप्राप्तः क्रियतां यदनन्तरम् ॥सावित्र्युवाच}
{}


\threelineshloka
{अस्तं गते मयाऽऽदित्ये भोक्तव्यं कृतकामया}
{एष मे हृदि संकल्पः समयश्च कृतो मया ॥मार्कण्डेय उवाच}
{}


\twolineshloka
{एवं संभाषमाणायाः सावित्र्या भोजनं प्रति}
{स्कन्धे परशुमादाय सत्यवान्प्रस्थितो वनम्}


\threelineshloka
{सावित्री त्वाह भर्तारं नैकस्त्वं गन्तुमर्हसि}
{सह त्वया गमिष्यामि न हित्वां हातुमुत्सहे ॥सत्यवानुवाच}
{}


\threelineshloka
{वनं न गतपूर्वं ते दुःख पन्थाश्च भामिनि}
{व्रतोपवासक्षामा च कथं पद्भ्यां गमिष्यसि ॥सावित्र्युवाच}
{}


\threelineshloka
{उपवासान्न मे ग्लानिर्नास्ति चापि परिश्रमः}
{गमने च कृतोत्साहां प्रतिषेद्धुं न माऽर्हसि ॥सत्यवानुवाच}
{}


\threelineshloka
{यदि ते गमनोत्साहः करिष्यामि तव प्रयम्}
{मम त्वामन्त्रय गुरून्न मां दोषः स्पृशेदयम् ॥मार्कण्डेय उवाच}
{}


\twolineshloka
{साऽभिवाद्याब्रवीच्छ्वश्रूं श्वशुरं च महाव्रता}
{अयं गच्छति मे भर्ता फलाहारो महावनम्}


\twolineshloka
{इच्छेयमभ्यनुज्ञाता आर्यया श्वशुरेण ह}
{अनेन सह निर्गन्तुं न मेऽद्य विरहः क्षमः}


\twolineshloka
{गुर्वग्निहोत्रार्तकृतेप्रस्थितश्च सुतस्तव}
{न निवार्यो निवार्यः स्यादन्यथा प्रस्थितो वनम्}


\threelineshloka
{संवत्सरः किंचिदूनो न निष्क्रान्ताऽहमाश्रमात्}
{वनं कुसुमितं द्रष्टुं परं कौतूहलं हि मे ॥द्युमत्सेन उवाच}
{}


\twolineshloka
{यदा प्रभृति सावित्री पित्रा दत्ता स्नाषु मम}
{नानयाऽभ्यर्थनायुक्तमुक्तपूर्वं स्मराम्यहम्}


\threelineshloka
{तदेषा लभतां कामं यथाभिलषितं वधूः}
{अप्रमादश्च कर्तव्यः पुत्रि सत्यवतः पथि ॥मार्कण्डेय उवाच}
{}


\twolineshloka
{उभाभ्यामभ्यनुज्ञाता सा जगाम यशस्विनी}
{सहभर्त्रा हसन्तीव हृदयेन विदूयता}


\twolineshloka
{सा वनानि विचित्राणि रमणीयानि सर्वशः}
{मयूरगणजुष्टानि ददर्श विपुलेक्षणा}


\twolineshloka
{नदीः पुण्यवहाश्चैव पुष्पितांश्च नगोत्तमान्}
{सत्यवानाह पश्येति सावित्रीं मधूरं वचः}


\twolineshloka
{नीरीक्षमाणा भर्तारं सर्वावस्थमनिन्दिता}
{मृतमेव हिं मेने काले मुनिवचः स्मरन्}


\twolineshloka
{अनुव्रजन्ती भर्तारं जगाम मृदुगामिनी}
{द्विधेव हृदयं कृत्वा तं च कालमवेक्षती}


\chapter{अध्यायः २९८}
\twolineshloka
{मार्कण्डेय उवाच}
{}


\twolineshloka
{अथ भार्यासहायः स फलान्यादाय वीर्यवान्}
{कठिनं पूरयामास ततः काण्ठान्यपाटयत्}


\twolineshloka
{तस्य पाटयतः काष्ठं स्वेदो वै समजायत}
{व्यायामेन च तेनास्य जज्ञे शिरसि वेदना}


\twolineshloka
{सोऽभिगम्य प्रियां भार्यामुवाच श्रमपीडितः}
{व्यायामेन ममानेन जाता शिरसि वेदना}


\twolineshloka
{अङ्गानि चैव सावित्रि हृदयं दूयतीव च}
{अस्वस्थमिव चात्मानं लक्षये मितभाषिणि}


\threelineshloka
{शूलैरिव शिरो विद्धमिदं संलक्षयाम्यहम्}
{`भ्रमन्तीव दिशः सर्वाश्चक्रारूढं मनो मम'}
{तत्स्वप्तुमिच्छे कल्याणि न स्तातुं शक्तिरस्ति मे}


\twolineshloka
{सा समासाद्य सावित्री भर्तारमुपगम्य च}
{उत्सङ्गेऽस्य शिर कृत्वा निषसाद महीतले}


\twolineshloka
{ततः सा नारदवचो विमृशन्ती तपस्विनी}
{तं मुहूर्तं क्षणं वेलां दिवसं च युयोज ह}


\threelineshloka
{`हन्त प्राप्तः स कालोऽयमिति चिन्तापरा सती'}
{मुहूर्तादेव चापश्य्पुरुषं रक्तवाससम्}
{वद्धमौलिं वपुष्मन्तमादित्यसमतेजसम्}


\twolineshloka
{श्यामावदातं रक्ताक्षं पाशहस्तं भयावहम्}
{स्थितं सत्यवतः पार्श्वे निरीक्षन्तं तमेव च}


\twolineshloka
{तं दृष्ट्वासहसोत्थाय भर्तुन्यस्य शनैः शिरः}
{कृताञ्जलिरुवाचार्ता हृदयेन प्रवेपती}


\threelineshloka
{दैवतंत्वाभिजानामि वपुरेतद्ध्यमानुषम्}
{कामया ब्रूहि देवेश कस्त्वं किंच चिकीर्षसि ॥यम उवाच}
{}


\twolineshloka
{पतिव्रताऽसि सावित्रि तथैव च तपोन्विता}
{अस्त्वामभिभाषामि विद्धि मां त्वं शुभे यमम्}


\threelineshloka
{अयं ते रसत्यवान्भर्ता क्षीणायुः पार्थिवात्मजः}
{नेष्यामि तमहं बद्ध्वा विद्ध्येतन्मे चीकिर्षितं ॥सावित्र्युवाच}
{}


\threelineshloka
{श्रूयते भगवन्दूतास्तवागच्छन्ति मानवान्}
{नेतुं किल भवान्कस्मादागतोसि स्वयं प्रभो ॥मार्कण्डेय उवाच}
{}


\twolineshloka
{इत्युक्तः पितृराजस्तां भगवान्स्वचिकीर्षितम्}
{यथावत्सर्वमाख्यातुं तत्प्रियार्थं प्रचक्रमे}


\twolineshloka
{अयं च धर्मसंयुक्तो रूपवान्गुणसागरः}
{नार्हो मत्पुरुषैर्नेतुमतोस्मि स्वयमागतः}


\twolineshloka
{ततः सत्यवतः कायात्पाशबद्धं वशंगतम्}
{अङ्गुष्ठमात्रं पुरुषं निश्चकर्ष यमो बलात्}


\twolineshloka
{ततः समुद्धृतप्राणं गतश्वासं हतप्रभम्}
{निरविचेष्टंशरीरं तद्बभूवाप्रियदर्सनम्}


\twolineshloka
{यमस्तु तं ततो बद्ध्वा प्रयातो दक्षिणामुखः}
{सावित्री चैव दुःखार्ता यममेवान्वगच्छत}


\fourlineindentedshloka
{`भर्तुः शरीररां च विधाय हि तपस्विनी}
{भर्तारमनुगच्छन्ती तथावस्थं सुमध्यमा'}
{नियमव्रतसंसिद्धा महाभागा पतिव्रता ॥यम उवाच}
{}


\threelineshloka
{निवर्त गच्छ सावित्रि कुरुष्वास्यौर्ध्वदैहिकम्}
{कृतंभर्तुस्त्वयाऽऽनृण्यं यावद्गम्यं गतं त्वया ॥सावित्र्युवाच}
{}


\twolineshloka
{यत्र मे नीयते भर्ता स्वयं वा यत्र गच्छति}
{मया च तत्र गन्तव्यमेष धर्मः सनातनः}


\twolineshloka
{तपसा गुरुभक्त्या च भर्तुः स्नेहाद्व्रतेन च}
{तव चैव प्रसादेन न मे प्रतिहता गतिः}


\twolineshloka
{प्राहुः साप्तपदं मैत्रं बुधास्तत्त्वार्थदर्शिनः}
{मित्रतां च पुरस्कृत्य किंचिद्वक्ष्यामि तच्छृणु}


\twolineshloka
{नानात्मवन्तस्तु वने चरन्तिधर्मं च वासं च परिश्रमं च}
{विज्ञानतो धर्ममुदाहरन्तितस्मात्सन्तो धर्ममाहुः प्रधानम्}


\threelineshloka
{एकस्य धर्मेण सतां मतेनसर्वेस्म तं मार्गमनुप्रपन्नाः}
{मा वै द्वितीयं मा तृतीयं च वाञ्छेतस्मात्सन्तो धर्ममाहुः प्रधानम् ॥यम उवाच}
{}


\threelineshloka
{निवर्त तुष्टोस्मि तवानया गिरास्वराक्षरव्यञ्जनहेतुयुक्तया}
{वरं वृणीष्वेहविनाऽस्य जीवितंददानि ते सर्वमनिन्दिते वरम् ॥सावित्र्युवाच}
{}


\threelineshloka
{च्युतः स्वराज्याद्वनवासमाश्रितोविनष्टचक्षुः श्वशुरो ममाश्रमे}
{स लब्धचक्षुर्बलवान्भवेन्नृप-स्तव प्रसादाज्ज्वलनार्कसंनिभः ॥यम उवाच}
{}


\threelineshloka
{ददानि तेऽहं तमनिन्दिते वरंयथा त्वयोक्तं भविता च तत्तथा}
{तवाध्वना ग्लानिमिवोपलक्षयेनिवर्त गच्छस्व न ते श्रमो भवेत् ॥सावित्र्युवाच}
{}


\twolineshloka
{श्रमः कुतो भर्तृसमीपतो हिमेयतो हि भर्ता मम सा गतिर्ध्रुवा}
{यतः पतिं नमेष्यसि तत्र मे गतिःसुरेश भूयश्च वचो निबोध मे}


\threelineshloka
{सतां सकृत्संगतमीप्सितं परंततः परं मित्रमिति प्रचक्षते}
{न चाफलं सत्पुरुषेण संगतंततः सतां संनिवसेत्समागमे ॥यम उवाच}
{}


\threelineshloka
{मनोनुकूलं बुधबुद्धिवर्धनंत्वया यदुक्तं वचनं हिताश्रयम्}
{विना पुनः सत्यवतोस्य जीवितंवरं द्वितीयं वरयस्व भामिनि ॥सावित्र्युवाच}
{}


\threelineshloka
{हृतंपुरा मे श्वशुरस्य धीमतःस्वमेव राज्यंलभतां स पार्थिवः}
{कजह्यात्स्वधर्मान्न च मे गुरुर्यथाद्वितीयमेतद्वरयामि ते वरम् ॥यम उवाच}
{}


\threelineshloka
{स्वमेवं राज्यं प्रतिपत्स्यतेऽचिरा-न्न च स्वधर्मात्परिहीयते नृपः}
{कृतेन कामेन मया नृपात्मजेनिवर्त गच्छस्व न ते श्रमो भवेत् ॥सावित्र्युवाच}
{}


\twolineshloka
{प्रजास्त्वयैता नियमेन संयतानियम्य चैता नयसे निकामया}
{ततो यमत्वं तव देव विश्रुतंनिबोध चेमां गिरमीरितां मया}


\twolineshloka
{अद्रोहः सर्वभूतेषु कर्मणा मनसा गिरा}
{अनुग्रहश्च दानं च सतां धर्मः सनातनः}


\threelineshloka
{एवंप्रायश्च लोकोऽयं मनुष्याः शक्तिपेशलाः}
{सन्तस्त्वेवाप्यमित्रेषु दयां प्राप्तेषु कुर्वते ॥यम उवाच}
{}


\threelineshloka
{पिपासितस्येव भवेद्यथा पय-स्तथा त्वया वाक्यमिदं समीरितम्}
{विना पुनः सत्यवतोऽस्य जीवितंवरं वृणीष्वेह शुभे यदिच्छसि ॥सावित्र्युवाच}
{}


\threelineshloka
{ममानपत्यः पृथिवीपतिः पिताभवत्पितुः पुत्रशतं तथौरसम्}
{कुलस्य संतानकरं च यद्भवे-त्तृतीयमेतद्वरयामि ते वरम् ॥यम उवाच}
{}


\threelineshloka
{कुलस्य संतानकरं सुवर्चसंशतं सुतानां पितुरस्तु ते शुभे}
{कृतेन कामेन नराधिपात्मजेनिवर्त दूरं हि पथस्त्वमागता ॥सावित्र्युवाच}
{}


\twolineshloka
{न दूरमेतन्मम भर्तृसन्निधौमनो हि मे दूरतरं प्रधावति}
{अथ व्रजन्नेव गिरं समुद्यतांमयोच्यमानां शृणु भूय एव च}


\twolineshloka
{विवस्वतस्त्वं तनयऋ प्रतापवां-स्ततो हि वैवस्वत उच्यसे बुधैः}
{समेन धर्मेण चरन्ति ताः प्रजा-स्ततस्तवेहेवर धर्मराजता}


\twolineshloka
{आत्मन्यपि न विश्वासस्तथा भवति सत्सु यः}
{तस्मात्सत्सु विशेपेण सर्वः प्रणयमिच्छति}


\threelineshloka
{सौहदात्सर्वभूतानां विश्वासो नाम जायते}
{तस्मात्सत्सु विशेपेण विश्वासं कुरुते जनः ॥यव उवाच}
{}


\threelineshloka
{उदाहृतंते वचनं यदङ्गनेशुभे न तादृक् त्वदृते श्रुतं मया}
{अनेन तुष्टोस्मि विनाऽस्य जीवितंवरं चतुर्थं वरयस्व गच्छ च ॥सावित्र्युवाच}
{}


\threelineshloka
{ममात्मजं सत्यवतस्तथौरसंभवेदुभाभ्यामिह यत्कुलोद्वहम्}
{शतं सुतानां बलवीर्यशालिना-मिदंचतुर्थं वरयामि ते वरम् ॥यम उवाच}
{}


\threelineshloka
{शतं सुतानां बलवीरय्शालिनांभविष्यति प्रीतिकरं तवाबले}
{परिश्रमस्ते न भवेन्नृपात्मजेनिवर्त दूरं हि पथस्त्वमागता ॥सावित्र्युवाच}
{}


\twolineshloka
{सतां सदा शाश्वतधर्मवृत्तिःसन्तो न सीदन्ति न च व्यथन्ति}
{सतां सद्भिर्नाफलः संगमोस्तिसद्भ्यो भयंनानुवर्तन्ति सन्तः}


\twolineshloka
{सन्तो हि सत्येन नयन्ति सूर्यंसन्तो भूमिं तपसा धारयन्ति}
{सन्तो गतिर्भूतभव्यस् राज-न्सतां मध्ये नावसीदन्ति सन्तः}


\twolineshloka
{आर्यजुष्टमिदं वृत्तमिति विज्ञाय शाश्वतम्}
{सन्तः परार्थं कुर्वाणा नावेक्षन्ति प्रतिक्रियाः}


\threelineshloka
{न च प्रसादः सत्पुरुषेषु मोघोन चाप्यर्थो नश्यति नापि मानः}
{यस्मादेतन्नियतं सत्सु नित्यंतस्मात्सन्तो रक्षितारो भवन्ति ॥यम उवाच}
{}


\threelineshloka
{यथायथा भाषसि धर्मसंहितंमनोनुकूलं सुपदं महार्थवत्}
{तथातथा मे त्वयि भक्तिरुत्तमावरं वृणीष्वाप्रतिमं पतिव्रते ॥सावित्र्युवाच}
{}


\twolineshloka
{न तेऽपवर्गः सुकृताद्विना कृत-स्तथा यथाऽन्येषु वरेषु मानद}
{वरं वृणे जीवतु सत्यवानयंयथा मृता ह्येवमहं पतिं विना}


\twolineshloka
{न कामये भर्तविनाकृता सुखंन कामये भर्तृविनाकृता दिवम्}
{नकामये भर्तविनाकृता श्रियंन भर्तृहीना व्यवसामि जीबितुम्}


\threelineshloka
{वरातिसर्गः शतपुत्रता ममत्वयैव दत्तो ह्रियते च मे पतिः}
{वरं वृणे जीवतु सत्यवानयंतवैव सत्यं वचनं भविष्यति ॥मार्कण्डेय उवाच}
{}


\twolineshloka
{तथेत्युक्त्वा तु तं पाशं मुक्त्वा वैवस्वतो यमः}
{धर्मराजः प्रहृष्टात्मा सावित्रीमिदमब्रवीत्}


\twolineshloka
{एष भद्रे मया मुक्तो भर्ता ते कुलनन्दिनि}
{`तोषितोऽहं त्वया साध्वि वाक्यैर्धर्मार्तसंहितैः'}


\twolineshloka
{अरोगस्व नेयश्च सिद्धार्थः स भविष्यि}
{चतुर्वर्षशतायुश्च त्वया सार्धमवाप्स्यति}


\twolineshloka
{इष्ट्वा यज्ञैश्च धर्मेण ख्यातिं लोके गमिष्यति}
{त्वयि पुत्रशतं चैव सत्यवाञ्जनयिष्यति}


\twolineshloka
{ते चापि सर्वे राजानः क्षत्रियाः पुत्रपौत्रिणः}
{ख्यातास्त्वन्नामधेयाश्चभविष्यन्तीह शाश्वताः}


\threelineshloka
{पितुश्च ते पुत्रशतं भविता तव मातरि}
{मालव्यां मालवा नाम शाश्वताः पुत्रपौत्रिणः}
{भ्रातरस्ते भविष्यन्ति क्षत्रियास्त्रिदशोपमाः}


\twolineshloka
{एवं तस्यै वरं दत्त्वा धर्मराजः प्रतापवान्}
{निवर्तयित्वा सावित्रीं स्वमेव भवनं ययौ}


\twolineshloka
{सावित्र्यपि यमे याते भर्तारं प्रतिलभ्य च}
{जगाम तत्र यत्रास्या भर्तुः शावं कलेवरम्}


\twolineshloka
{सा भूमौ प्रेक्ष्यभर्तारमुपसृत्योपगृह्य च}
{उत्सङ्गे शिर आरोप्य भूमावुपविवेश ह}


\twolineshloka
{संज्ञां चस पुनर्लब्ध्वा सावित्रीमभ्यभाषत}
{प्रोष्यागत इव प्रेम्णा पुनःपुनरुदीक्ष्यवै}


\threelineshloka
{सुचिरं बत सुप्तोस्मि किमर्थं नावबोधितः}
{क्व चासौ पुरुषः श्यामो योसौ मां संचकर्षह ॥सावित्र्युवाच}
{}


\twolineshloka
{सुचिरं त्वंप्रसुप्तोसि ममाह्के पुरुषर्षभ}
{गतः स भगवान्देवः प्रजासंयमनो यमः}


\threelineshloka
{विश्रान्तोसि महाभाग विनिद्रश्च नृपात्मज}
{यदि शक्यं समुत्तिष्ठ विगाढां पश्य शर्वरीम् ॥मार्कण्डेय उवाच}
{}


\twolineshloka
{उपलभ्यततः संज्ञां सुखसुप्त इवोत्थितः}
{दिशः सर्वा वनान्तांश्च निरीक्ष्योवाच सत्यवान्}


\twolineshloka
{फलाहारोस्मि निष्क्रान्तस््वया सह सुमध्यमे}
{ततः पाटयतः काष्ठं शरिसो मे रुजाऽभवत्}


\twolineshloka
{शिरोभितापसंतप्तः स्थातुं चिरमशक्नुवन्}
{तवोत्सङ्गे प्रसुप्तोस्मि इति सर्वं स्मरे शुभे}


\twolineshloka
{त्वयोपगूढस्य च मे निद्रयाऽपहृतं मनः}
{ततोऽपश्यं तमो घोरं पुरुषं च महौजसम्}


\twolineshloka
{तद्यदि त्वं विजानासि किं तद्ब्रूहि सुमध्यमे}
{स्वप्नो मे यदिवा दृष्टो यदि वा सत्यमेव तत्}


\twolineshloka
{तमुवाचाथ सावित्री रजनी व्यवगाहते}
{श्वस्ते सर्वंयथावृत्तमाख्यास्यामि नृपात्मज}


\twolineshloka
{उत्तिष्ठोत्तिष्ठ भद्रं ते पितरौ पश्य सुव्रत}
{विगाढा रजनी चेयं निवृत्तश्च दिवाकरः}


\twolineshloka
{नक्तंचराश्चरन्त्येते हृष्टाः क्रूराभिभाषिणः}
{श्रूयन्ते पर्णशब्दाश्च मृगाणां चरतां वने}


\threelineshloka
{एता घोरं शिवा नादान्दिशं दक्षिणपश्चिमाम्}
{आस्थाय विरुवन्त्युग्राः कम्पयन्त्यो मनो मम ॥सत्यवानुवाच}
{}


\threelineshloka
{वनं प्रतिभयाकारं घनेन तमसा वृतम्}
{न विज्ञास्यसि पन्थानं गन्तुं चैव न शक्ष्यसि ॥सावित्र्युवाच}
{}


\twolineshloka
{अस्मिन्न वने दग्धे शुष्कवृक्षः स्थितो ज्वलन्}
{वायुना धम्यमानोत्र दृश्यतेऽग्निः क्वचित्क्वचित्}


\twolineshloka
{ततोऽग्निमानयित्वेह ज्वालयिप्यामि सर्वतः}
{काष्ठानीमानि सन्तीह जहि संतापमात्मनः}


\twolineshloka
{यदि नोत्सहसे गन्तुं सरुजं त्वां हि लक्षये}
{न च ज्ञास्यसि पन्थानं तमसा संवृते वने}


\threelineshloka
{श्वः प्रभाते वने दृश्ये यास्यावोऽनुमते तव}
{वसावेह क्षपामेकां रुचितं यदि तेऽनघ ॥सत्यवानुवाच}
{}


\twolineshloka
{शिरोरुजा निवृत्ता मे स्वस्थान्यङ्गानि लक्षये}
{मातापितृभ्यामिच्छामि संयोगं त्वत्प्रसादजम्}


\twolineshloka
{न कदाचिद्विकाले हि गतपूर्वोहमाश्रमात्}
{अनागतायां सन्ध्यायां माता मे प्ररुणद्धि माम्}


\twolineshloka
{दिवाऽपिमयि निष्क्रान्ते सन्तप्येते गुरू मम}
{विचिनोति हि मां तातः सहैवाश्रमवासिभिः}


\twolineshloka
{मात्रा पित्रा च सुभृशं दुःखिताभ्यामहं पुरा}
{उपालब्धश्च बहुशश्चिरेणागच्छसीति हि}


\twolineshloka
{कात्ववस्था तयोरद्य मदर्थमिति चिन्तये}
{तयोरदृश्ये मयि च महद्दुःखं भविष्यति}


\twolineshloka
{पुरा मामूचतुश्चैव रात्रावस्रायमाणकौ}
{भृशं सुदुःखितौ वृद्धौ बहुशः प्रीतिसंयुतौ}


\twolineshloka
{त्वया हीनौ न जीवाव मुहूर्तमपि पुत्रक}
{यावद्धरिष्यसे पुत्र तावन्नौ जीवितं ध्रुवम्}


\twolineshloka
{वृद्धयोरन्धयोर्दृष्टिस्त्वयि वंशः प्रतिष्ठिः}
{त्वयि पिण्डश्च कीर्तिश्च सन्तानश्चावयोरिति}


\twolineshloka
{माता वृद्धा पिता वृद्धस्तयोर्यष्टिरहं किल}
{तौ रात्रौ मामपश्यन्तौ कामवस्थां गमिष्यतः}


\twolineshloka
{निद्रायाश्चाभ्यसूयामि यस्या हेतोः पिता मम}
{माता च संशयं प्राप्ता मत्कृतेऽनपकारिणी}


\twolineshloka
{अहं च संशयं प्राप्तः कृच्छ्रामापदमास्थितः}
{मातापितृभ्यां हि विना नाहं जीवितुमुत्सहे}


\twolineshloka
{व्यक्तमाकुलया बुद्ध्या प्रज्ञाचक्षुः पिता मम}
{एकैकमस्यां वेलायां पृच्छत्याश्रमवासिनम्}


\twolineshloka
{नात्मानमनुशोचामि यथाऽहंपितरं शुभे}
{भर्तारं चाप्यनुगतां मातरं भृशदुःखिताम्}


\twolineshloka
{मत्कृते न हि तावद्य सन्तापं परमेष्यतः}
{जीवन्तावनुजीवामि भर्तव्यौ तौ मयेति ह}


\fourlineindentedshloka
{तयोः प्रियं मे कर्व्यमिति जीवामि चाप्यहम्}
{`परमं दैवतं तौ मे पूजनीयौ सदा मया}
{तयोस्तु मे सदाऽस्त्येवं व्रतमेतत्पुरातनम्' ॥मार्कण्डेय उवाच}
{}


\twolineshloka
{एवमुक्त्वा स धर्मात्मा गुरुभक्तो गुरुप्रियः}
{उच्छ्रित्य बाहू दुःखार्तः सुस्वरं प्ररुरोद ह}


\twolineshloka
{ततोऽब्रवीत्तथा दृष्ट्वाभर्तारं शोककर्शितम्}
{प्रमृज्याश्रूणि पाणिभ्यां सावित्री धर्मचारिणी}


\twolineshloka
{यदि मेऽस्ति तपस्तप्तं यदि दत्तं हुतं यदि}
{श्वश्रूश्वशुरभर्तॄणां मम पुण्याऽस्तु शर्वरी}


\threelineshloka
{न स्मराम्युक्तपूर्वं वै स्वैरेष्वप्यनृतां गिरम्}
{तेन सत्येन तावद्य ध्रियेतां श्वशुरौ मम ॥सत्यवानुवाच}
{}


\twolineshloka
{कामये दर्शनं पित्रोर्याहि सावित्रि माचिरम्}
{`अपिनाम गुरू तौ हि पश्येयं ध्रियमाणकौ'}


\twolineshloka
{पुरा मातुः पितुर्वाऽपियदि पश्यामि विप्रियम्}
{न जीविष्ये वरारोहे सत्येनात्मानमालभे}


\threelineshloka
{यदि धर्मे च ते बुद्धिर्मां चेज्जीवन्तमिच्छसि}
{मम प्रियं वा कर्तव्यं गच्छावाश्रममन्तिकात् ॥मार्कण्डेय उवाच}
{}


\twolineshloka
{सावित्री तत उत्थाय केशान्संम्य भामिनी}
{पतिमुत्थापयामास बाहुभ्यां परिगृह्य वै}


\twolineshloka
{उत्ताय सत्यवांश्चापि प्रमृज्याङ्गानि पाणिना}
{सर्वा दिशः समालोक्य कठिने दृष्टिमादधे}


\twolineshloka
{तमुवाचाथसावित्री श्वः फलानि हरिष्यसि}
{योगक्षेमार्थमेतं ते नेष्यामि परशुं त्वहम्}


\twolineshloka
{कृत्त्वा कठिनभारं सा वृक्षशाखावलम्बिनम्}
{गृहीत्वा परशुं भर्तुः सकाशे पुनरागमत्}


\threelineshloka
{वामे स्कन्धे तु वामोरूर्भर्तुर्बाहुं निवेश्य च}
{दक्षिणएन परिष्वज्य जगाम गजगामिनी ॥ सत्यवानुवाच}
{}


\twolineshloka
{अभ्यासगमनाद्भीरु पन्थानो विदिता मम}
{वृक्षान्तरालोकितया ज्योत्स्नया चापि लक्षये}


\twolineshloka
{आगतौ स्वः पथा येन फलान्यवचितानि च}
{यथागतं शुभे गच्छ पन्थानं मा विचारय}


\twolineshloka
{पलाशखण्डे चैतस्मिन्पन्था व्यावर्तते द्विधा}
{तस्योत्तरेण यः पन्थास्तेन गच्छ त्वरस्व च}


\twolineshloka
{स्वस्थोस्मि बलवानस्मि दिदृक्षुः पितरावुभौ}
{ब्रुवन्नेव त्वरायुक्तः सम्प्रायादाश्रमं प्रति}


\chapter{अध्यायः २९९}
\twolineshloka
{मार्कण्डेय उवाच}
{}


\twolineshloka
{एतस्मिन्नेव काले तु द्युमत्सेनो महाबलः}
{लब्धचक्षुः प्रसन्नायां दृष्ट्यां सर्वं ददर्श ह}


\twolineshloka
{स सर्वानाश्रमान्गत्वा शैब्यया सह भार्यया}
{पुत्रहेतोः परामार्तिं जगाम भरतर्षभ}


\twolineshloka
{तावाश्रमान्नदीश्चैववनानि च सरांसि च}
{तस्यां निशि विचिन्वन्तौ दम्पती परिजग्मतुः}


\twolineshloka
{श्रुत्वा शब्दं तु यं कंचिदुन्मुखौ सुतशङ्कया}
{सावित्रीसहितोऽभ्येति सत्यवानित्यभाषताम्}


\twolineshloka
{भिन्नैश्च परुषैः पादैः सव्रणैः शोणितोक्षितैः}
{कुशकणअटकविद्धाङ्गावुनमत्ताविव धावतः}


\twolineshloka
{ततोऽभिसृत्य तैर्विप्रैः सर्वैराश्रमवासिभिः}
{परिवार्य समाश्वास्य तावानीतौ स्वमाश्रमम्}


\twolineshloka
{तत्रभार्यासहायः स वृतो वृद्धैस्तपोधनैः}
{आश्वासितोपि चित्रार्थैः पूर्वराजकथाश्रयैः}


\threelineshloka
{ततस्तौ पुनराश्वस्तौ वृद्धौ पुत्रदिदृक्षया}
{बाल्यवृत्तानि पुत्रस्य सावित्र्या दर्शनानि च}
{शोकं जग्मतुरन्योन्यं स्मरन्तौ भृशदुःखितौ}


\threelineshloka
{हापुत्र हासाध्वि वधु क्वासिक्वासीत्यरोदताम्}
{ब्राह्मणः सत्यवाक्येषामुवाचेदं तयोर्वचः ॥सुवर्चा उवाच}
{}


\threelineshloka
{यथास्य भार्या सावित्री तपसा च दमेन च}
{आचारेण च संयुक्ता तथा जीवति सत्यवान् ॥गौतम उवाच}
{}


\twolineshloka
{वेदाः साङ्गा मयाऽधीतास्तपो मे संचितं महत्}
{कौमारब्रह्मचर्यं च गुरवोऽग्निश्च तोषिताः}


\twolineshloka
{समाहितेन चीर्णानि सर्वाण्येव व्रतानि मे}
{वायुभक्षोपवासश्च कृतोमे विधिवत्सदा}


\threelineshloka
{अनेन तपसा वेद्मि सर्वं परचीकीर्षितम्}
{सत्यमेतन्निबोधध्वं ध्रियते सत्यवानिति ॥शिष्य उवाच}
{}


\threelineshloka
{उपाध्यायस्य मे वक्राद्यथा वाक्यं विनिःसृतम्}
{नैव जातु भवेन्मिथ्या तथा जीवति सत्यवान् ॥ऋषय ऊचुः}
{}


\threelineshloka
{यथाऽस्य भार्या सावित्री सर्वैरेव सुलक्षणैः}
{अवैधव्यकरैर्युक्ता तथा जीवति सत्यवान् ॥भारद्वाज उवाच}
{}


\threelineshloka
{यथाऽस्य भार्या सावित्री तपसा च दमेन च}
{आचारेण च संयुक्ता तथा जीवति सत्यवान् ॥दाल्भ्या रउवाच}
{}


\threelineshloka
{यथा दृष्टिः प्रवृत्ता ते सावित्र्याश्च यथा व्रतम्}
{गताऽऽहारमकृत्वैव तथा जीवति सत्यवान् ॥आपस्तम्ब उवाच}
{}


\threelineshloka
{यथा वदन्ति शान्तायां दिशि वै मृगपक्षिणः}
{पार्थिवीं चैववृद्धिं ते तथा जीवति सत्यवान् ॥धौम्य उवाच}
{}


\threelineshloka
{सर्वैर्गुणैरुपेतस्ते यथा पुत्रो जनप्रियः}
{दीर्घायुर्लक्षणोपेतस्तथा जीवति स्यवान् ॥मार्कण्डेय उवाच}
{}


\twolineshloka
{एवमाश्वासितस्तैस्तु सत्यवाग्भिस्तपस्विभिः}
{तांस्तान्विगणयन्सर्वांस्ततः स्थिर इवाभवत्}


\twolineshloka
{ततो मुहूर्तात्सावित्री भर्त्रा सत्वता सह}
{आजगामाश्रमं रात्रौ प्रहृष्टा प्रविवेश ह}


\threelineshloka
{`दृष्ट्वा चोत्पतिताः सर्वेहर्षं जग्मुश्च ते द्विजाः}
{कण्ठं माता पिता चास्य समालिङ्ग्याभ्यरोदतां' ॥ब्राह्मणा ऊचुः}
{}


\twolineshloka
{पुत्रेण संगतं त्वां तु चक्षुष्मन्तं निरीक्ष्य च}
{सर्वे वयं वै पृच्छामो वृद्धिं वै पृथिवीपते}


\twolineshloka
{समागमेन पुत्रस्य सावित्र्या दर्शनेन च}
{चक्षुषश्चात्मनो लाभात्रिभिर्दिष्ठ्या विवर्धसे}


\threelineshloka
{सर्वैरस्माभिरुक्तं यत्तथा तन्नात्रसंशयः}
{भूयोभूयः समृद्धिस्ते क्षिप्रमेव भविष्यति ॥मार्कण्डेय उवाच}
{}


\twolineshloka
{ततोऽग्निं तत्र संज्वाल्य द्विजास्ते सर्व एव हि}
{उपासांचक्रिरे पार्थ द्युमत्सेनं महीपतिम्}


\twolineshloka
{शैव्या च सत्यवांश्चैव सावित्री चैकतः स्थिताः}
{सर्वैस्तैरभ्यनुज्ञाता विशोका समुपाविशन्}


\twolineshloka
{ततो राज्ञा सहासीनाः सर्वे ते वनवासिनः}
{जातकौतूहलाः पार्थ पप्रच्छुर्नृपतेः सुतम्}


\twolineshloka
{प्रागेव नागतं कस्मात्सभार्येण त्वया विभो}
{विरात्रे चागतं कस्मात्कोनु बन्धस्तवाभवत्}


\threelineshloka
{संतापितः पिता माता वयं चैव नृपात्मज}
{कस्मादिति न जानीमस्तत्सर्वं वक्तुमर्हिसि ॥सत्यवानुवाच}
{}


\twolineshloka
{पित्राऽहमभ्यनुज्ञातः सावित्रीसहितो गतः}
{अथ मेऽभूच्छिरोदुःखं वने काष्ठानि भिन्दतः}


\twolineshloka
{सुप्तश्चाहं वेदनया चिरमित्युपलक्षये}
{तावत्कालं न च मया सुप्तपूर्वं कदाचन}


\threelineshloka
{सर्वेषामेव भवतां संतापो मा भवेदिति}
{अतो विरात्रागमनं नान्यदस्तीह कारणम् ॥गौतम उवाच}
{}


\twolineshloka
{अकस्माच्चक्षुः प्राप्तिर्द्युमत्सेनस्य ते पितुः}
{नास्य त्वं कारणं वेत्सि सावित्री वक्तुमर्हति}


\twolineshloka
{श्रोतुमिच्छामि सावित्रि त्वं हि वेत्थ परावरम्}
{त्वां हि जानामि सावित्रि सावित्रीमिव तेजसा}


\threelineshloka
{त्वमत्र हेतुं जानीषे तस्मात्सत्यं निरुच्यताम्}
{रहस्यं यदि ते नास्ति किंचिदत्र वदस्व नः ॥सावित्र्युवाच}
{}


\twolineshloka
{एवमेतद्यथा वेत्थ संकल्पो नान्यथा हि वः}
{न हि किंचिद्रहस्यं मे श्रूयतां तथ्यमेव यत्}


\twolineshloka
{मृत्युर्मे पत्युराख्यातो नारदेन महात्मना}
{स चाद्य दिवसः प्राप्तस्ततो नैनं जहाम्यहम्}


\twolineshloka
{सुप्तं चैनं साक्षादुपागच्छत्सकिंकरः}
{स एनमनयद्बद्ध्वा दिशं पितृनिषेविताम्}


\twolineshloka
{अस्तौषं तमहं देवं सत्येन वचसा विभुम्}
{पञ्च वै तेन मे दत्ता वराः शृणुत तान्मम}


\twolineshloka
{चक्षुषी च स्वराज्यंच द्वौ वरौ श्वशुरस्य मे}
{लब्धं पितुः पुत्रशतं पुत्राणां चात्मनः शतम्}


\threelineshloka
{चतुर्वर्षशतायुर्मे भर्ता लब्धश्च सत्यवान्}
{भर्तुर्हि जीवितार्थं तु मया चीर्णं त्विदं व्रतम्}
{}


\threelineshloka
{एतत्सर्वं मयाऽऽख्यातं कारणं विस्तरण वः}
{यथावृत्तं सुखोदर्कमिदं दुःखं महन्मम ॥ऋषय ऊचुः}
{}


\threelineshloka
{निमज्ज्मानं व्यसनैरभिद्रुतंकुलं नरन्द्रस्य तमोमये ह्रदे}
{त्वया सुशीलव्रतपुण्यया कुलंरसमुद्धृतं साध्वि पुनः कुलीनया ॥मार्कण्डेय उवाच}
{}


\twolineshloka
{तथा प्रशस्य ह्यभिपूज्य चैववरस्त्रियं तामृषयः समागताः}
{नरेन्द्रमामन्त्र्य सपुत्रभञ्जसाशिवेन रजग्मुर्मुदिताः स्वमालयम्}


\chapter{अध्यायः ३००}
\twolineshloka
{मार्कण्डेय उवाच}
{}


\twolineshloka
{तस्यां रात्र्यां व्यतीतायामुदिते सूर्यमण्डले}
{कृतपौर्वाह्णिकाः सर्वे समेयुस्ते तपोधनाः}


\twolineshloka
{तदेव सर्वं सावित्र्या महाभाग्यं महर्षयः}
{द्युम्सेनाय नातृप्यन्कथयन्तः पुनः पुनः}


\twolineshloka
{ततः प्रकृतयः सर्वाः साल्वेभ्योऽभ्यागा नृपम्}
{आचख्युर्निहतं चैव स्वेनामात्येन तं द्विषम्}


\twolineshloka
{तं मन्त्रिणा हतं प्रोच्य ससहायं सबान्धवम्}
{न्यवेदयन्यथावृत्तं विद्रुतं च द्विषद्बलम्}


\twolineshloka
{ऐकमत्यं च सर्वस्य जनस्य स्वं नृपं प्रति}
{सचक्षुर्वाऽप्यचक्षुर्वा स नो राजा भवत्विति}


\twolineshloka
{अनेन निश्चयेनेह वयं प्रस्थापिता नृप}
{प्राप्तानीमानि यानानि चतुरङ्गं च ते बलम्}


\threelineshloka
{प्रयाहि राजन्भद्रं ते घुष्टस्ते नगरे जयः}
{अध्यास्स्व चिररात्राय पितृपैतामहं पदम् ॥मार्कण्डेय उवाच}
{}


\twolineshloka
{चक्षुष्मन्तं च तं दृष्ट्वा राजानं वपुषाऽन्वितम्}
{मूर्ध्ना निपतिताः सर्वेविस्मयोत्फुल्ललोचनाः}


\twolineshloka
{तोऽभिवाद्य तान्वृद्धान्द्विजानाश्रमवासिनः}
{तैश्चाभिपूजितः सर्वैः प्रययौ नगरं प्रति}


\twolineshloka
{शैव्या च सह सावित्र्या स्वास्तीर्णेन सुवर्चसा}
{नरयुक्तेन यानेन प्रययौ सेनया वृता}


\twolineshloka
{ततोऽभिषिषिचुः प्रीत्या द्युमत्सेनं पुरोहिताः}
{पुत्रं चास्य महात्मानं यौवराज्येऽभ्यषेचयन्}


\twolineshloka
{ततः कालेन महता सावित्र्याः कीर्तिवर्धनम्}
{तद्वै पुत्रशतं जज्ञे शूराणामनिवर्तिनाम्}


\twolineshloka
{भ्रातृणां सोदराणां च तथैवास्याभवच्छतम्}
{मद्राधिपस्याश्वपतेर्मालव्यां सुमहाबलम्}


\twolineshloka
{एवमात्मा पिता माता श्वश्रूः श्वशुर एव च}
{भर्तुः कुलं च सावित्र्या सर्वं कृच्छ्रात्समुद्धृतं}


\threelineshloka
{तथैवैषा हि कल्याणी द्रौपदी शीलसंमता}
{तारयिष्यति वः सर्वान्सावित्रीव कुलाङ्गना ॥वैशंपायन उवाच}
{}


\twolineshloka
{एवं स पाण्डवस्तेन अनुनीतो महात्मना}
{विशोको विज्वरो राजन्काम्यके न्यवसत्तदा}


\twolineshloka
{यश्चेदं शृणुयाद्भक्त्या सावित्र्याख्यानमुत्तमम्}
{स सुखी सर्वसिद्धार्थो न दुःखं प्राप्नुयान्नरः}


\chapter{अध्यायः ३०१}
\twolineshloka
{जनमेजय उवाच}
{}


\twolineshloka
{यत्तत्तदा महद्ब्रह्मँल्लोमशो वाक्यमब्रवीत्}
{इन्द्रस्य वचनादेव पाण्डुपुत्रं युधिष्ठिरम्}


\twolineshloka
{यच्चापि ते भयं तीव्रं न च कीर्तयसे क्वचित्}
{तच्चाप्यपहरिष्यामि धनंजय इतो गते}


\threelineshloka
{किंनु रतज्जपतांश्रेष्ठ कर्णं प्रति महद्भयम्}
{आसीन्न च स धर्मात्मा कथयामास कस्यचित् ॥वैशंपायन उवाच}
{}


\twolineshloka
{अहं ते राजशार्दूल कथयामि कथामिमाम्}
{पृच्छतो भरतश्रेष्ठ शुश्रूषस्व गिरं मम}


\twolineshloka
{द्वादशे समतिक्रान्ते वर्षे प्राप्ते त्रयोदशे}
{पाण्डूनां हितकृच्छक्रः कर्णं भिक्षितुमुद्यतः}


\twolineshloka
{अभिप्रायमथो ज्ञात्वा महेन्द्रस्य विभावसुः}
{कुण्डलार्थे महाराज सूर्यः कर्णमुपागतः}


\twolineshloka
{महार्हे शयने वीरं स्पर्द्ध्यास्तरणसंवृते}
{शयानमतिविश्वस्तं ब्रह्मण्यं सत्यवादिनम्}


\twolineshloka
{स्वप्नान्ते निशि राजेन्द्र दर्शयामास रश्मिवान्}
{कृपया परयाऽऽविष्टः पुत्रस्नेहाच्च भारत}


\twolineshloka
{ब्राह्मणो वेदविद्भूत्वा सूर्यो योगर्द्धिरूपवान्}
{हितार्थमब्रवीत्कर्णं सान्त्वपूर्वमिदं वचः}


\twolineshloka
{कर्ण मद्वचनं तात शृणु सत्यभृतांवर}
{ब्रुवतोऽद्य महाबाहो सौहृदात्परमं हितम्}


\twolineshloka
{उपायास्यति शक्रस्त्वां पाण्डवानां हितेप्सया}
{ब्राह्मणच्छद्मना कर्ण कुण्डलोपजिहीर्षया}


\twolineshloka
{विदितं तेन शीलं ते सर्वस्य जगतस्तथा}
{यथा त्वं भिक्षितः सद्भिर्ददास्येव न याचसे}


\twolineshloka
{त्वं हि तात ददास्येव ब्राह्मणेभ्यः प्रयाचितम्}
{वित्तं यच्चान्यदप्याहुर्न प्रत्याख्यासि कस्यचित्}


\twolineshloka
{त्वां तु चैवंविधं ज्ञात्वा स्वयं वै पाकशासनः}
{आगन्ता रकुण्डलार्थाय कवचं चैव भिक्षितुम्}


\twolineshloka
{तस्मै प्रयाचमानाय न देये कुण्डले त्वया}
{अनुनेयः परं शक्त्या श्रेय एतद्धि ते परम्}


\twolineshloka
{कुण्डलार्थेऽब्रुवंस्तात कारणैर्बहुभिस्त्वया}
{अन्यैर्बहुविधैर्वित्तैः सन्निवार्यः पुनःपुनः}


\twolineshloka
{रत्नैः स्त्रीभिस्तथा गोभिर्धनैर्बहुविधैरपि}
{निदर्शनैश्च बहुभिः कुण्डलेप्सुः पुरंदरः}


\twolineshloka
{यदि दास्यसि कर्ण त्वं सहजे कुण्डले शुभे}
{आयुषः प्रक्षयं गत्वा मृत्योर्वशमुपैष्यसि}


\twolineshloka
{कवचेन समायुक्तः कुण्डलाभ्यां च मानद}
{अवध्यस्त्वं रणेऽरीणामिति विद्धि वचो मम}


\threelineshloka
{अमृतादुत्थितं ह्येतदुभयं रत्नसंमितम्}
{तस्माद्रक्ष्यं त्वया कर्ण जीवितं चेत्प्रियं तव ॥कर्ण उवाच}
{}


\threelineshloka
{को मामेवं भवान्प्राह दर्शयन्सौहृदं परम्}
{कामया भगवन्ब्रूहि को भवान्द्विजवेषधृक् ॥ब्रह्मण उवाच}
{}


\threelineshloka
{अहं तात सहस्रांशुः सौहृदात्त्वां निदर्शये}
{कुरुष्वैतद्वयो मे त्वमेतच्छ्रेयः परं हि ते ॥कर्ण उवाच}
{}


\twolineshloka
{श्रेय एव ममात्यन्तं यस्य मे गोपतिः प्रभुः}
{प्रवक्ताऽद्य हितान्वेषी शृणु चेदं वचो मम}


\twolineshloka
{प्रसादये त्वां वरदं प्रणयाच्च ब्रवीम्यहम्}
{न निवार्यो व्रतादस्मादहं यद्यस्मि ते प्रियः}


\twolineshloka
{व्रतं वै मम लोकोऽयं वेत्ति कृत्स्नं विभावसो}
{यथाऽहं द्विजमुख्येभ्यो दद्यां प्राणानपिध्रुवम्}


\twolineshloka
{यद्यागच्छति मां शक्रो ब्राह्मणच्छद्मना वृतः}
{हितार्थं पाण्डुपुत्राणां खेचरोत्तम भिक्षितुम्}


\twolineshloka
{दास्यामि विबुधश्रेष्ठ कुण्डले वर्म चोत्तमम्}
{न मे कीर्तिः प्रणश्येत त्रिषु लोकेषु विश्रुता}


\twolineshloka
{मद्विधस्यायशस्यं हि न युक्तं प्राणरक्षणम्}
{युक्तं हि यशसा युक्तं मरणं लोकसंमतम्}


\twolineshloka
{सोहमिन्द्राय दास्यामि कुण्डले सह वर्मणा}
{यदि मां बलवृत्रघ्नो भिक्षार्थमुपयास्यति}


\twolineshloka
{हितार्थं पाण्डुपुत्राणां कुण्डले मे प्रयाचितुम्}
{तन्मे कीर्तिकरं लोके तस्याकीर्तिर्भविष्यति}


\twolineshloka
{वृणोमि कीर्तिं लोके हि जीवितेनापि भानुमन्}
{कीर्तिमानश्नुते स्वर्गं हीनकीर्तिस्तु नश्यति}


\twolineshloka
{कीर्तिर्हि पुरुषं लोके संजीवयति मातृवत्}
{अकीर्तिर्जीवितं हन्ति जीवतोपि शरीरिणः}


\twolineshloka
{अयंपुराणः श्लोको हि स्वयं गीतो विभावसो}
{धात्रा लोकेश्वर यथा कीर्तिरायुर्नरस्य ह}


\twolineshloka
{पुरुषस्य परे लोके कीर्तिरेव परायणम्}
{इह लोके विशुद्धा च कीर्तिरायुर्विवर्धनी}


\twolineshloka
{सोहं शरीरजे दत्त्वा कीर्तिं प्राप्स्यामि शाश्वतीम्}
{दत्त्वा च विधिवद्दानं ब्राह्मणेभ्यो यथाविधि}


\twolineshloka
{हुत्वा शरीरं रसंग्रामे कृत्वा कर्म सुदुष्करम्}
{विजित्य च परानाजौ यशः प्राप्स्यामि केवलम्}


\twolineshloka
{भीतानामभयं दत्त्वा संग्रामे जीवितार्थिनाम्}
{वृद्धान्वालान्द्विजातींश्च मोक्षयित्वा महाभयात्}


\twolineshloka
{प्राप्स्यामि परमं लोके यशः स्वर्ग्यमनुत्तमम्}
{जीवितेनापि मे रक्ष्या कीर्तिस्तद्विद्वि मे व्रतम्}


\twolineshloka
{सो हं दत्त्वा मघवते भिक्षामेतामनुत्तमाम्}
{ब्राह्मणच्छद्मिने देव लोके गन्ता परां गतिम्}


\chapter{अध्यायः ३०२}
\twolineshloka
{सूर्य उवाच}
{}


\twolineshloka
{माऽहितं कर्ण कार्षीस्त्वमात्मनः सुहृदां तथा}
{पुत्राणामथ भार्याणामथो मातुरथो पितुः}


\twolineshloka
{शरीरस्याविरोधेन प्राणिनां प्राणभृद्वर}
{इष्यते यशसः प्राप्तिः कीर्तिश्च त्रिदिवे स्थिरा}


\twolineshloka
{यस्त्वं प्राणविरोधेन कीर्तिमिच्छसि शाश्वतीम्}
{सा ते प्राणान्समादाय गमिष्यति न संशयः}


\twolineshloka
{जीवतां कुरुते कार्यं पिता माता सुतास्तथा}
{ये चान्ये बान्धवाः केचिल्लोकेऽस्मिन्पुरुषर्षभ}


\twolineshloka
{राजानश्च नरव्याघ्र पौरुषेण निबोध तत्}
{कीर्तिश्च जीवतः साध्वी पुरुषस्य महाद्युते}


\twolineshloka
{मृतस्य कीर्त्या किं कार्यं भस्मीभूतस्य देहिनः}
{मृतः कीर्तिं न जानीते जीवन्कीर्ति समश्नुते}


\twolineshloka
{मृतस्य कीर्तिर्मर्त्यस्य यथा माला गतायुषः}
{अहं तु त्वां ब्रवीम्येतद्भक्तोसीति हितेप्सया}


\twolineshloka
{भक्तिमन्तो हि मे रक्ष्या इत्येतेनापि हेतुना}
{भक्तोयं परया भक्त्या मामित्येव महाभुज}


\threelineshloka
{ममापि भक्तिरुत्पन्ना स त्वं कुरु वचो मम}
{अस्ति चात्र परं किंचिदध्यात्मं देवनिर्मितम्}
{अतश्च त्वां ब्रवीम्येतत्क्रियतामविशङ्कया}


\twolineshloka
{देवगुह्यं त्वया ज्ञातुं न शक्यं पुरुषर्षभ}
{नस्मान्नाख्यामि ते गुह्यं काले वेत्स्यति तद्भवान्}


\twolineshloka
{पुनरुक्तं च वक्ष्यामि त्वं राधेय निबोध तत्}
{माऽस्मै ते कुण्डले दद्या भिक्षिते वज्रापाणिना}


\twolineshloka
{शोभसे कुण्डलाभ्यां च रुचिराभ्यां महाद्युते}
{विशाखयोर्मध्यगतः शशीव विमले दिवि}


\twolineshloka
{कीर्तिश्च जीवतः साध्वी पुरुषस्येति विद्धि तत्}
{प्रत्याख्येयस्त्वया तात कुण्डलार्थे सुरेश्वरः}


\threelineshloka
{`पाण्डवानां हिते युक्तो भिक्षन्ब्राह्मणवेषधृत्'}
{शक्त्या बहुविधैर्वाक्यैः कुण्डलेप्सा त्वयाऽनघ}
{विहन्तुं देवराजस् हेतुयुक्तैः पुनःपुनः}


\twolineshloka
{उपपत्त्युपपन्नार्थैर्माधुर्यकृतभूषणैः}
{पुरंदरस्य कर्ण त्वं बुद्धिमेतामपानुद}


\twolineshloka
{त्वं हि नित्यं नरव्याघ्र स्पर्धसे सव्यसाचिना}
{सव्यसाची त्वया चेह युधि शूरः समेष्यति}


\twolineshloka
{न तु त्वामर्जुनः शक्तः कुण्डलाभ्यां समन्वितम्}
{विजेतुं युधि यद्यस्य स्वयमिन्द्रः शरो भवेत्}


\twolineshloka
{तस्मान्न देये शख्राय त्वयैते कुण्डले शुभे}
{संग्रामे यदि निर्जेतुं कर्ण कामयसेऽर्जुनम्}


\chapter{अध्यायः ३०३}
\twolineshloka
{कर्ण उवाच}
{}


\twolineshloka
{भगवन्तमहं भक्तो यथा मां वेत्थ गोपते}
{तथा परमतिग्मांशो नास्त्यदेयं कथंचन}


\twolineshloka
{न मे दारा न मेपुत्रा न चात्मा सुहृदो न च}
{तथेष्टा वै सदा भक्त्या यथा त्वं गोपते मम}


\twolineshloka
{इष्टानां च महात्मानो भक्तानां च न संशयः}
{कुर्वन्ति भक्तिमिष्टां च जानीषे त्वं च भास्कर}


\twolineshloka
{इष्टो भक्तश्च मे कर्णो न चान्यद्दैवतं दिवि}
{जानीत इतिवै कृत्वा भगवानाह मद्धितम्}


\twolineshloka
{भूयश्च शिरसा याचे प्रसाद्य च पुनःपुनः}
{इति ब्रवीमि तिग्मांशो त्वं तु मे क्षन्तुमर्हसि}


\twolineshloka
{बिभेमि न तथा मृत्योर्यथा बिभ्येऽनृतादहम्}
{विशेषेण द्विजातीनां सर्वेषां सर्वदा सताम्}


\twolineshloka
{प्रदाने जीवितस्यापि न मेऽत्रास्ति विचारणा}
{यच्च मामात्थ देव त्वं पाण्डवं फल्गुनं प्रति}


\twolineshloka
{व्येतु संतापजं दुःखं तव भास्कर मानसम्}
{अर्जुनप्रतिमं चैव विजेष्यामि रणेऽर्जुनम्}


\twolineshloka
{तवापि विदितं देव ममाप्यस्त्रबलं महत्}
{जामदग्न्यादुपात्तं यत्तथा द्रोणान्महात्मनः}


\threelineshloka
{इदं त्वमनुजानीहि सुरश्रेष्ठ व्रतं मम}
{भिक्षते वज्रिणे दद्यामपि जीवितमात्मनः ॥सूर्य उवाच}
{}


\twolineshloka
{यदि तात ददास्येते वज्रिणे कुण्डले शुभे}
{त्वमप्येनमथो ब्रूया विजयार्थं महाबल}


\twolineshloka
{नियमेन प्रदद्यास्त्वं कुण्डलेवै शतक्रतोः}
{अवध्यो ह्यसि भूतानां कुण्डलाभ्यां समन्वितः}


\twolineshloka
{अर्जुनन विनाशं हि तव दानवसूदनः}
{प्रार्तयानो रणे वत्स कुण्डले ते जिहीर्षति}


\twolineshloka
{स त्वमप्येनमाराध्य सूनृताभिः पुनः पुनः}
{अभ्यर्थयेथा देवेशममोघार्थं पुरंदरम्}


\twolineshloka
{अमोघां देहि मे शक्तिममित्रविनिबर्हिणीम्}
{दास्यामि ते सहस्राक्ष कुण्डले वर्म चोत्तमम्}


\twolineshloka
{इत्येव नियमेन त्वं दद्याः शक्राय कुण्डले}
{तया त्वं कर्ण संग्रामे हनिष्यसि रणे रिपून्}


\threelineshloka
{नाहत्वा हि महाबाहो शत्रूनेति करं पुनः}
{सा शक्तिर्देवराजस्य शतशोऽथ सहस्रशः ॥वैशंपायन उवाच}
{}


\twolineshloka
{एवमुक्त्वा सहस्रांशुः सहसाऽन्तरधीयत}
{`कर्णस्तु बुबुधे राजन्स्वप्नान्ते प्रव्यथन्निव}


\twolineshloka
{प्रतिबुद्धस्तु राधेयः स्वप्नं संचिन्त्य भारत}
{चकार निश्चयं राजञ्शक्त्यर्थं वदतांवर}


\twolineshloka
{यदि मामिन्द्र आयाति कुण्डलार्थं परन्तप}
{शक्त्या तस्मै प्रदास्यामि कुण्डले वर्म चैव ह}


\twolineshloka
{स कृत्वा प्रातरुत्थाय कार्याणि भरतर्षभ}
{ब्राह्मणान्वाचयित्वा च यथाकार्यमुपाक्रमत्}


\twolineshloka
{विधिना राजशार्दूल मुहूर्तमजपत्तदा'}
{ततः सूर्याय जप्यान्ते कर्णः स्वप्नं न्यवेदयत्}


\twolineshloka
{यथा दृष्टं यथातत्त्वं यथोक्तमुभयोर्निसि}
{तत्सर्वमानुपूर्व्येण शशंसास्मै वृषस्तदा}


\twolineshloka
{तच्छ्रुत्वा भगवान्देवो भानुः स्वर्भानुसूदनः}
{उवाच तं तथेत्येव कर्णं सूर्यः स्मयन्निव}


\twolineshloka
{ततस्तत्त्वमिति ज्ञात्वा राधेयः परवीरहा}
{शक्तिमेवाभिकाङ्क्षन्वै वासवं प्रत्यपालयत्}


\chapter{अध्यायः ३०४}
\twolineshloka
{जनमेजय उवाच}
{}


\twolineshloka
{किं तद्गुह्यं न चाख्यातं कर्णायेहोष्णरश्मिना}
{कीदृशेकुण्डले ते च कवचं रचैव कीदृशम्}


\threelineshloka
{कुतश्च कवचं तस्यं कुण्डलेचैव सत्तम}
{एतदिच्छाम्यहं श्रोतुं तन्मे ब्रूहि तपोधन ॥वैशंपायन उवाच}
{}


\twolineshloka
{अहं राजन्ब्रवीम्येतत्तस्य गुह्यं विभावसोः}
{यादृशे कुण्डले ते च कवचं वैव यादृशम्}


\twolineshloka
{कुन्तिभोजं पुरा राजन्ब्राह्मणः पर्युपस्थितः}
{तिग्मतेजा महान्प्रांशुः श्मश्रुदण्डजटाधरः}


\twolineshloka
{दर्सनीयोऽनवद्याङ्गस्तेजसा प्रज्वलन्निव}
{मधुपिङ्गो मधुरवाक्तपःस्वाध्यायभूषणः}


\twolineshloka
{स राजानं कुन्तिभोजमब्रवीत्सुमहातपाः}
{भिक्षामिच्छामि वै भोक्तुं तव गेहे विमत्सर}


\twolineshloka
{न मे व्यलीकं कर्तव्यं त्वया वा तव चानुगैः}
{एवं वत्स्यामि ते गेहे यदि ते रोचतेऽनघ}


\twolineshloka
{यथाकामं च गच्छेयमागच्छेयं तथैव च}
{शय्यासने च मे राजन्नापराध्येत कश्चन}


\twolineshloka
{तमब्रवीत्कुन्तिभोजः प्रीतियुक्तमिदं वचः}
{एवमस्तु परंचेति पुनश्चैवमथाब्रवीत्}


\twolineshloka
{मम कन्या महाप्राज्ञ पृथा नाम यशस्विनी}
{शीलवृत्तान्विता साध्वी नियताऽनवमानिनी}


\twolineshloka
{उपस्थास्यति सा त्वां वै पूजयाऽनवमत्य च}
{तस्याश्च शीलवृत्तेन तुष्टिं समुपयास्यसि}


\twolineshloka
{एवमुक्त्वा तु तं विप्रमभिपूज्य यथाविधि}
{उवाच कन्यामभ्येत्य पृथां पृथुललोचनाम्}


\twolineshloka
{अयं वत्से महाभागो ब्राह्मणो वस्तुमिच्छति}
{मम गेहे मया चास्य तथेत्येवं प्रतिश्रुतम्}


\twolineshloka
{त्वयि वत्से परायत्तं ब्राह्मणस्याभिराधनम्}
{तन्मे वाक्यममिथ्या त्वं कर्तुमर्हसि कर्हिचित्}


\twolineshloka
{अयं तपस्वी भगवान्स्वाध्यायनियतो द्विजः}
{यद्यद्ब्रूयान्महातेजास्तत्तद्देयमात्सरात्}


\twolineshloka
{ब्राह्मणा हि परं तेजो ब्राह्मणा हि परं तपः}
{ब्राह्मणानां नमस्कारैः सूर्यो दिवि विराजते}


\threelineshloka
{अमानयन्हि दाण्डक्यो वातापिश्च महासुरः}
{निहतो ब्रह्मदण्डेन तालजङ्घस्तथैव च}
{`वैन्ध्याद्रिश्च समुद्रश्च नद्दुषश्च विहिंसितः'}


\twolineshloka
{सोयं वत्से महाभार आहितस्त्वयि सांप्रतम्}
{त्वं सदा नियता कुर्या ब्राह्मणस्याभिराधनम्}


\twolineshloka
{जानामि प्रणिधानं ते बाल्यात्प्रभृति नन्दिनि}
{ब्राह्मणेष्विह सर्वेषु गुरुबन्धुषु चैव ह}


\twolineshloka
{तथा प्रेष्येषु सर्वेषु मित्रसंबन्धिमातृषु}
{मयि चैव यथावत्त्वंसर्वमावृत्य वर्तसे}


\twolineshloka
{न ह्यतुष्टो जनोऽस्तीह पुरे चान्तःपुरे च ते}
{सम्यग्वृत्त्याऽनवद्याङ्गि तव भृत्यजनेष्वपि}


\twolineshloka
{संदेष्टव्यां तु मन्ये त्वां द्विजातिं कोपनं प्रति}
{पृथे बालेति कृत्वा वै सुता चासिममेति च}


\twolineshloka
{वृष्णीनां च कुले जाता शूरस् दयिता सुता}
{दत्ता प्रीतिमता मह्यं पित्रा बाला पुरास्वयम्}


\twolineshloka
{वसुदेवस् भगिनी सुतानां प्रवरा मम}
{अग्र्यमग्रे प्रतिज्ञाय तेनासि दुहिता मम}


\twolineshloka
{तादृशे हि कुले जाता कुले मम विवर्धिता}
{सुखात्सुखमनुप्राप्ता ह्रदाद्ध्रदमिवापगा}


\twolineshloka
{दौष्कुलेया विशेषेण कथंचित्प्रग्रहं गताः}
{बालभावाद्विकुर्वन्ति प्रायशः प्रमदाः शुभे}


\twolineshloka
{पृथे राजकुले जन्म रूपं चापि तवाद्भुतम्}
{तेन तेनासि संपन्ना समुपेता च भामिनि}


\twolineshloka
{सा त्वं दर्पं परित्यज्य दम्भं मानं च भामिनि}
{आराध्यवरदं विप्रं श्रेयसा योक्ष्यसे पृथे}


\twolineshloka
{एवं प्राप्स्यसि कल्याणि कल्याणमनघे ध्रुवम्}
{कोपिते च द्विजश्रेष्ठे कुत्स्नं दह्येत मे कुलम्}


\chapter{अध्यायः ३०५}
\twolineshloka
{कुन्त्युवाच}
{}


\twolineshloka
{ब्राह्मणं यन्त्रिता राजन्नुपस्थास्यामि पूजया}
{यथाप्रतिज्ञं राजेन्द्रन च मिथ्या ब्रवीम्यहम्}


\twolineshloka
{एष चैव स्वभावो मे पूजयेयं द्विजानिति}
{तव चैव प्रियं कार्यं श्रेयश्च परमं मम}


\twolineshloka
{यद्येवैष्यति सायाह्ने यदि प्रातरथो निशि}
{यद्यर्धरात्रे भगवान्न मे कोपं करिष्यति}


\twolineshloka
{लाभो ममैष राजेन्द्र यद्वैपूजयितुं द्विजान्}
{आदेशे तव तिष्ठन्ती हितं कुर्यां नरोत्तम}


\twolineshloka
{विस्रब्धो भवराजेन्द्र न व्यलीकं द्विजोत्तमः}
{वसन्प्राप्स्यति ते गेहे सत्यमेतद्ब्रवीमि ते}


\twolineshloka
{यत्प्रियं च द्विजस्यास्य हितं चैव तवानघ}
{यतिष्यामि तथा राजन्व्येतु ते मानसो ज्वरः}


\twolineshloka
{ब्राह्मणा हि महाभागाः पूजिताः पृथिवीपते}
{तारणाय समर्थाः स्युर्विपरीते वधाय च}


\twolineshloka
{रसाऽहमेतद्विजानन्ती तोषयिष्ये द्विजोत्तमम्}
{न मत्कृतेव्यथां राजन्प्राप्स्यसि द्विजसत्तमात्}


\twolineshloka
{अपराधेऽपि राजेन्द्र राज्ञामश्रेयसे द्विजाः}
{भवन्ति च्यवनो यद्वत्सुकन्यायाः कृते पुरा}


\twolineshloka
{नियमेन परेणाहमुपस्थास्ये द्विजोत्तमम्}
{यथा त्वया नरेन्द्रेदं भापितं ब्राह्मणं प्रति}


\twolineshloka
{एवं ब्रुवन्तीं बहुशः परिष्वज्य समर्थ्य च}
{इतिचेति च क्रतव्यं राजा सर्वमथादिशत्}


\twolineshloka
{एवमेतत्त्वया भद्रे कर्तव्यमविशङ्कया}
{मद्धितार्थं तथाऽऽत्मार्थंकुलार्थं चाप्यनिन्दिते}


\twolineshloka
{एवमुक्त्वा तु तां कन्यां कुन्तिभोजो महायशाः}
{पृथां परिददौ तस्मै द्विजाय द्विजवत्सलः}


\twolineshloka
{इयं ब्रह्मन्मम सुता बाला सुखविवर्धिता}
{अपराध्येत यत्किंचिन्न कार्यं हृदि तत्त्वया}


\twolineshloka
{द्विजातयो महाभागा वृद्धबालतपस्विषु}
{भवन्त्यक्रोधनाः प्रायो ह्यपराद्धेषु नित्यदा}


\twolineshloka
{सुमहत्यपराधेऽपि क्षान्तिः कार्या द्विजातिभिः}
{यथाशक्ति यथोत्साहं पूजा ग्राह्या द्विजोत्तम}


\twolineshloka
{तथेति ब्राह्मणेनोक्ते स राजा प्रीतमानसः}
{हंसचन्द्रांशुसंकाशं गृहमस्मै न्यवेदयत्}


\twolineshloka
{तत्राग्निशरणे क्लृप्तमासनं तस्य भानुमत्}
{आहारादि च सर्वं तत्तथैव प्रत्यवेदयत्}


\twolineshloka
{निक्षिप्य राजपुत्री तु तन्द्रीं मानं तथैव च}
{आतस्थे परमं यत्नं ब्राह्मणस्याभिराधने}


\twolineshloka
{तत्रसा ब्राह्मणं गत्वा पृथा शौचपरा सती}
{विधिवत्परिचारार्हं देववत्पर्यतोषयत्}


\chapter{अध्यायः ३०६}
\twolineshloka
{वैशंपायन उवाच}
{}


\twolineshloka
{सा तु कन्या महाराज ब्राह्मणं संशितव्रतम्}
{तोषयामास शुद्धेन मनसा संशितव्रता}


\twolineshloka
{प्रातरेष्याम्यथेत्युक्त्वा कदाचिद्द्विजसत्तमः}
{तत आयाति राजेन्द्र सायं रात्रावथो पुनः}


\twolineshloka
{तं च सर्वासु वेलासु भक्ष्यभोज्यप्रतिश्रयैः}
{पूजयामास सा कन्या वर्धमानैस्तु सर्वदा}


\twolineshloka
{अन्नादिसमुदाचारः शय्यासनकृतस्तथा}
{दिवसेदिवसे तस्य वर्धते न तु हीयते}


\twolineshloka
{निर्भर्त्सनापवादैश्च तथैवाप्रियया गिरा}
{ब्राह्मणस्य पृथा राजन्न चकाराप्रियं तदा}


\twolineshloka
{स्वप्नकाले पुनश्चैति न चैति बहुशो द्विजः}
{सुदुर्लभमपि ह्यन्नं दीयतामिति सोऽब्रवीत्}


\twolineshloka
{कृतमेव च तत्सर्वं यथा तस्मै न्यवेदयत्}
{शिष्यवत्पुत्रवच्चैव स्वसृवच्च सुसंयता}


\twolineshloka
{यथोपजोषं राजेन्द्र द्विजातिप्रवरस्य सा}
{प्रीतिमुत्पादयामास कन्यारत्नमनिन्दिता}


\twolineshloka
{तस्यास्तु शीलवृत्तेन तुतोप द्विजसत्तमः}
{अवधानेन भूयोऽस्याः परं यत्नमथाकरोत्}


\twolineshloka
{तां प्रभाते च सायं च पिता पप्रच्छ भारत}
{अपितुष्यतिते पुत्रि ब्राह्मणः परिचर्यया}


\twolineshloka
{तं सा परममित्येवप्रत्युवाच यशस्विनी}
{ततः प्रीतिमवापाग्र्यां कुन्तिभोजो महामनाः}


\twolineshloka
{ततः संवत्सरे पूर्णे यदाऽसौ जपतांवरः}
{नापश्यद्दुष्कृतंकिंचित्पृथायाः सौहृदे रतः}


\twolineshloka
{ततः प्रीतमना भूत्वा स एनां ब्राह्मणोऽब्रवीत्}
{प्रीतोस्मि परमं भद्रे परिचारेण ते शुभे}


\threelineshloka
{वरान्वृणीष्व क्रल्याणि दूरापान्मानुपैरिह}
{यैस्त्वं सीमन्तिनीः सर्वायशसाऽभिमविष्यसि ॥कुन्त्युवाच}
{}


\threelineshloka
{कृतानि मम सर्वाणि सस्या मे वेदवित्तम}
{त्वं प्रसन्नः पिता चैव कृतं विप्र वरैर्मम ॥ब्राह्मण उवाच}
{}


\twolineshloka
{यदि नेच्छसि मत्तस्त्वं वरं भद्रे शुचिस्मिते}
{इमं मन्त्रं गृहाण त्वमाह्रानाय दिवौकसाम्}


\twolineshloka
{यंयं देवं त्वमेतेन मन्त्रेणावाहयिष्यसि}
{तेनतेन वशे भद्रे स्थातव्यं ते भविष्यति}


\threelineshloka
{अकामो वा सकामो वा स समेष्यति ते वशे}
{विबुधो मन्त्रसंभ्रान्तो वाक्यैर्भृत्य इवानतः ॥वैशंपायन उवाच}
{}


\twolineshloka
{न शशक द्वितीयं सा प्रत्याख्यातुमनिन्दिता}
{तं वै द्विजातिप्रवरं तदा शापभयान्नृप}


\twolineshloka
{ततस्तामनवद्याङ्गीं ग्राहयामास स द्विजः}
{मन्त्रग्रामं तदा राजन्नथर्वशिरसि श्रुतम्}


\twolineshloka
{तं प्रदाय तु राजेन्द्र कुन्तिभोजमुवाच ह}
{उपितोस्मि सुखं राजन्कन्यया परितोपितः}


\twolineshloka
{तवगेहे सुविहितः सदा सुप्रतिपूजितः}
{साधयिष्यामहे तावदित्युक्त्वाऽन्तरधीयत}


\twolineshloka
{स तु राजा द्विजं दृष्ट्वा तत्रैवान्तर्हितं तदा}
{बभूव विस्मयाविष्टः पृथां च समपूजयत्}


\chapter{अध्यायः ३०७}
\twolineshloka
{वैशंपायन उवाच}
{}


\twolineshloka
{गते तस्मिन्द्विजश्रेष्ठे कस्मिंश्चित्कालपर्यये}
{चिन्तयामास सा कन्या मन्त्रग्रामबलाबलम्}


\twolineshloka
{अयं वै कीदृशस्तेन मम दत्तो हमात्मना}
{मन्त्रग्रामो बलं तस्य ज्ञास्ये नातिचिरादिति}


\twolineshloka
{एवं संचिन्तयन्ती सा ददर्शर्तुं यदृच्छया}
{व्रीडिता साऽभवद्बाला कन्याभावे रजस्वला}


\twolineshloka
{ततो हर्म्यतलस्था सा महार्हशयनोचिता}
{प्राच्यां दिशि समुद्यन्तं ददर्शादित्यमण्डलम्}


\twolineshloka
{तत्र बद्धमनोदृष्टिरभवत्सा सुमध्यमा}
{न चातप्यत रूपेण भानोः सन्ध्यागतस्य सा}


\twolineshloka
{तस्या दृष्टिरभूद्दिव्या साऽपश्यद्दिव्यदर्शनम्}
{आमुक्तकवचं देवं कुण्डलाभ्यां विभूषितम्}


\twolineshloka
{तस्याः कौतूहरलं त्वासीन्मन्त्रं प्रति नराधिप}
{आह्वानमकरोत्साऽथ तस्य देवस्य भामिनी}


\twolineshloka
{प्राणानुपस्पृश्य तदा ह्याजुहाव दिवाकरम्}
{आजगाम ततो राजंस्त्वरमाणो दिवाकरः}


\twolineshloka
{मधुपिङ्गो महाबाहुः कम्बुग्रीवो हसन्निव}
{अङ्गदी बद्धमुकुटो दिशः प्रज्वालयन्निव}


\twolineshloka
{योगात्कृत्वा द्विधाऽऽत्मानमाजगाम तताप च}
{आबभाषे ततः कुन्तीं साम्ना परमबल्गुना}


\threelineshloka
{आगतोस्मि वशं भद्रे तव मन्त्रबलात्कृतः}
{किं करोमि वशो राज्ञि ब्रूहि कर्ता तदस्मि ते ॥कुन्त्युवाच}
{}


\threelineshloka
{गम्यतां भगवंस्तत्र यत एवागतो ह्यसि}
{कौतूहलात्समाहूतः प्रसीद भगवन्निति ॥सूर्य उवाच}
{}


\twolineshloka
{गमिष्येऽहं यथा मा त्वं ब्रवीषि तनुमध्यमे}
{न तु देवं समाहूय न्याय्यं प्रेषयितुं वृथा}


\twolineshloka
{तवाभिसन्धि सुभगे रसूर्यात्पुत्रो भवेदिति}
{वीर्येणाप्रतिमो लोके कवची कुण्डलीति च}


\threelineshloka
{सा त्वमात्मप्रदानं वै कुरुष्व गजगामिनि}
{उत्पत्स्यति हि पुत्रस्ते यथासंकल्पमङ्गने}
{अथ गच्छाम्यहं भद्रे त्वया संगम्य सुस्मिते}


\twolineshloka
{यदि त्वंवचनं नाद्य करिष्यसि मम प्रियम्}
{शप्स्ये कन्येऽन्यथा क्रुद्धो ब्राह्मणं पितरं च ते}


\twolineshloka
{त्वत्कृते तान्प्रधक्ष्यामि सर्वानपि न संशयः}
{पितरं चैव ते मूढं यो न वेत्ति तवानयम्}


\twolineshloka
{तस् च ब्राह्मणस्याद्य योसौ मन्त्रमदात्तव}
{शीलवृत्तमविज्ञाय धास्यामि विनयं परम्}


\twolineshloka
{एते हि विबुधाः सर्वेपुरंदरमुखा दिवि}
{त्वया प्रलब्धं पश्यन्ति स्मयन्त इव मां शुभे}


\threelineshloka
{पश्य चैनान्सुरगणान्दिव्यं चक्षुरिदं हि ते}
{पूर्वमेव मया दत्तं दृष्टवत्यसि येन माम् ॥वैशंपायन उवाच}
{}


\twolineshloka
{ततोऽपश्यत्रिदशान्राजपुत्रीसर्वानव स्वेषु धिष्ण्येषु खस्थान्}
{प्रभावन्तं भानुमन्तं महान्तंयथाऽऽदित्यं रोचमानांस्तथैव}


\twolineshloka
{सा तान्दृष्ट्वा त्रिदशानेव बालासूर्यं देवी वचनं प्राह भीता}
{गच्छ त्वं वै गोपते स्वं विमानंकन्याभावाद्दुःख एवापचारः}


\twolineshloka
{पिता माता गुरवश्चैवयेऽन्येदेहस्यास्य प्रभवन्ति प्रदाने}
{नाहं धर्मं लोपयिष्यामि लोकेस्त्रीणां वृत्तं पूज्यते देहरक्षा}


\threelineshloka
{मया मन्त्रबलं ज्ञातुमाहूतस्त्वं विभावसो}
{बाल्याद्बालेति तत्कृत्वा क्षन्तुमर्हसि मे विभो ॥सूर्य उवाच}
{}


\twolineshloka
{बालेति कृत्वाऽनुनयं तवाहंददानि नान्यानुनयं लभेत}
{आत्मप्रदानं कुरु कुन्तिकन्येशान्तिस्तवैवं हि भवेच्च भीरु}


\twolineshloka
{न चापि गन्तुं युक्तं हि मया मिथ्याकृतेन वै}
{असमेत्य त्वया भीरु मन्त्राहूतेन भामिनि}


\threelineshloka
{गमिष्याम्यनवद्याङ्गि लोके समवहास्यताम्}
{`गच्छेयमेव सुश्रोणि गतोऽहं वै निराकृतः'}
{सर्वेषां विबुधानां च वक्तव्यः स्यां तथा शुभे}


\twolineshloka
{सा त्वं मया समागच्छ पुत्रं लप्स्यसि माध्शम्}
{विशिष्टा सर्वलोकेषु भविष्यसि न संशय}


\chapter{अध्यायः ३०८}
\twolineshloka
{वैशंपायन उवाच}
{}


\twolineshloka
{सा तु कन्या बहुविधं ब्रुवन्ती मधुरं वचः}
{अनुनेतुं सहस्रांशुं न शशाक मनस्विनी}


\twolineshloka
{न शशाक यदा बाला प्रत्याख्यातुं तमोनुदम्}
{भीता शापात्ततो राजन्दध्यौ दीर्घमथान्तरम्}


\twolineshloka
{अनागसः पितुः शापो ब्राह्मणस्य तथैव च}
{मन्निमित्तः कथं न स्यात्क्रुद्धादस्माद्विभावसोः}


\twolineshloka
{बालेनापि सता मोहाद्भृशं सापह्नवान्यपि}
{नाऽभ्यासादयितव्यानि तेजांसि च तपांसि च}


\twolineshloka
{साहमद्य भृशं भीता गृहीत्वा च करे भृशम्}
{कथं त्वकार्यं कुर्यां वै प्रदानं ह्यात्मनः स्वयम्}


\twolineshloka
{सा वै शापपरित्रस्ता बहु चिन्तयती हृदा}
{मोहेनाभिपरीताङ्गी स्मयमाना पुनःपुनः}


\twolineshloka
{तं देवमब्रवीद्भीता बन्धूनां राजसत्तम}
{व्रीडाविह्वलया वाचा शापत्रस्ता विशांपते}


\twolineshloka
{पिता मे ध्रियते देव माता चान्ये च बान्धवाः}
{न तेषु ध्रियमाणएषु विदिलोपो भवेदयम्}


\twolineshloka
{त्वया तु संगमो देव यदि स्याद्विधिवर्जितः}
{मन्निमित्तं कुलस्यास्य लोकेऽकीर्तिर्न संशयः}


\twolineshloka
{अथवा धर्ममेतं त्वं मन्यसे तपतांवर}
{ऋते प्रदानाद्बन्धुभ्यस्तव कामं करोम्यहम्}


\threelineshloka
{आत्मप्रदानं दुर्धर्ष तव कृत्वासती त्वहम्}
{त्वयि धर्मो यशश्चैव कीर्तिरायुश्च देहिनाम् ॥सूर्य उवाच}
{}


\twolineshloka
{न ते पिता न ते माता गुरवो वा शुचिस्मिते}
{प्रभवन्ति प्रदाने ते भद्रं ते शृणु मे वचः}


\twolineshloka
{सर्वान्कामयते यस्मात्कनेर्धातोश्च भामिनि}
{तस्मात्कन्येह सुश्रोणी स्वतन्त्रा वरवर्णिनि}


\twolineshloka
{नाधर्मश्चरितः कश्चित्त्वया भवति भामिनि}
{अधर्मंकुत एवाहं वरेयं लोककाम्यया}


\twolineshloka
{अनावृताः स्त्रियः सर्वा नराश्च वरवर्णिनि}
{स्वभाव एष लोकानां विकारोऽन्य इति स्मृतः}


\threelineshloka
{सा मया सहसंगम्य पुनः कन्या भविष्यसि}
{पुत्रश्च ते महावाद्दुर्भविष्यति न संशयः ॥कुन्त्युवाच}
{}


\fourlineindentedshloka
{यदि पुत्रो मम भवेत्त्वत्तः सर्वतमोनुद}
{कुण्डली कवची शूरो महाबाहुर्महाबलः}
{`अस्तु मे सङ्गमो देव अनेन समयेन ते' ॥सूर्य उवाच}
{}


\threelineshloka
{भविष्यति महाबाहुः कुण्डली दिव्यवर्मभृत्}
{उभयं चामृतमयं तस् भद्रे भविष्यति ॥कुन्त्युवाच}
{}


\twolineshloka
{यद्यतदमृतादस्ति कुण्डले वर्म चोत्तमम्}
{मम पुत्रस्य यं वै त्वं मत्त उत्पादयिष्यसि}


\threelineshloka
{अस्तु मे सङ्गमो देव यथोक्तं भगवंस्त्वया}
{त्वद्वीर्यरूपसत्वौजा धऱ्मयुक्तो भवेत्स च ॥सूर्य उवाच}
{}


\threelineshloka
{अदित्या कुण्डले राज्ञि दत्ते मे मत्तकाशिनि}
{तस्मै दास्यामि वामोरु वर्म चैवेदमुत्तमम् ॥वैशंपायन उवाच}
{}


\threelineshloka
{परमं भगवन्नेवं सङ्गमिष्ये त्वया सह}
{यदि पुत्रो भवेदेवं यथा वदसि गोपते ॥वैशंपायन उवाच}
{}


\twolineshloka
{तथेत्युक्त्वा तु तां कुन्तीमाविवेश विहंगमः}
{स्वर्भानुशत्रुर्योगात्मा नाभ्यां पस्पर्श चैव ताम्}


\threelineshloka
{तत सा विह्वलेवासीत्कन्या सूरय्स् तेजसा}
{पपात चाथ सा देवी शयने मूढचेतना ॥सूर्य उवाच}
{}


\threelineshloka
{साधयिष्यामि सुश्रोणि पुत्रं वै जनयिष्यसि}
{सर्वशस्त्रभृतांश्रेष्ठं कन्या चैव भविष्यसि ॥वैशंपायन उवाच}
{}


\twolineshloka
{ततः सा व्रीडिता बाला तदा सूर्यमथाब्रवीत्}
{एवमस्त्विति राजेन्द्रप्रस्थितं भूरिवर्चसम्}


\twolineshloka
{इतिस्मोक्ता कुन्तिराजात्मजा साविवस्वन्तं याचमाना सलज्जा}
{तस्मिन्पुण्ये शयनीये पपातमोहाविष्टा वेपमाना लतेव}


\twolineshloka
{तिग्मांशुस्तां तेजसा मोहयित्वायोगेनाविश्यात्मसंस्थां चकार}
{न चैवैनां दूषयामास भानुःसंज्ञां लेभे भूय एवात बाला}


\chapter{अध्यायः ३०९}
\twolineshloka
{वैशंपायन उवाच}
{}


\twolineshloka
{ततो गर्भः समभवत्पृथायाः पृथिवीपते}
{शुक्ले दशोत्तरे पक्षे तारापतिरिवाम्बरे}


\twolineshloka
{सा बान्धवभयाद्बाला गर्भं तं विनिगूहती}
{धारयामास सुश्रोणी न चैनां बुबुधे जनः}


\twolineshloka
{न हि तां वेद नार्यन्या काचिद्धात्रेयिकामृते}
{कन्यापुरगतां बालां निपुणां परिरक्षणे}


\twolineshloka
{ततः कालेन सा गर्भं सुषुवे वरवर्णिनी}
{कन्यैव तस्य देवस्य प्रसादादमरप्रभम्}


\twolineshloka
{तथैवाबद्धकवचं कनकोज्ज्वलकुण्डलम्}
{हर्यक्षं वृषभस्कन्धं यथास्य पितरं तथा}


\threelineshloka
{जातमात्रं च तं गर्भं धात्र्या संमन्त्र्य भामिनी}
{`उत्स्रष्टुकामा तं गर्भं कारयामास भारत}
{}


\twolineshloka
{मञ्जूषां शिल्पिभिस्तूर्णं सुनद्धां सुप्रतिष्ठिताम् ॥प्लवैर्बहुविधैर्बद्धां प्लवनार्थं जले नृप}
{अजिनैर्मृदुभिश्चैवं संस्तीर्णशयनां तथा'}


\threelineshloka
{मञ्जूषायां समाधाय स्वास्तीर्णायां समन्ततः}
{मधूच्चिष्टस्थितायां तं सुखायां रुदती तथा}
{श्लक्ष्णायां सुपिधानायामश्वनद्यामवासृजत्}


\twolineshloka
{जानती चाप्यकर्तव्यं कन्याया गर्भधारणम्}
{पुत्रस्नेहेन सा राजन्करुणं पर्यदेवयत्}


\twolineshloka
{समुत्सृजन्ती मञ्जूपामश्वनद्यां तदा जले}
{उवाच रुदतीकुन्ती यानि वाक्यानि तच्छृणु}


\twolineshloka
{स्वस्ति तेऽस्त्वान्तरिक्षेभ्यः पार्थिवेभ्यश्च पुत्रक}
{दिव्येभ्यश्चैव भूतेभ्यस्तथा तोयचराश्च ये}


\twolineshloka
{शिवास्ते सन्तु पन्थानो मा च ते परिपन्थिनः}
{आगताश्च तथा पुत्र भवन्त्यद्रोहचेतसः}


\twolineshloka
{पातु त्वां वरुणो राजा सलिले सलिलेश्वरः}
{अन्तरिक्षेऽन्तरिक्षस्थः पवनः सर्वगस्तथा}


\twolineshloka
{पिता त्वां पातु सर्वत्र तपनस्तपतांवरः}
{येन दत्तोसि मे पुत्र दिव्येन विधिना किल}


\twolineshloka
{आदित्या वसवो रुद्राः साध्या विश्वे च देवताः}
{मरुतश्च सहेन्द्रेण दिशश्च सदिदीश्वराः}


\twolineshloka
{रक्षन्तु त्वां सुराः सर्वे समेषु विषमेषु च}
{वेत्स्यामि त्वांविदेशेपि कवचेनाभिसूचितम्}


\twolineshloka
{धन्यस्ते पुत्र जनरको देवो भानुर्विभावसुः}
{स्त्वां द्रक्ष्यति दिव्येन चक्षुषा वाहिनीगतम्}


\twolineshloka
{धन्या सा प्रमदा या त्वां पुत्रत्वे कल्पयिष्यति}
{यस्यास्त्वं तृषितः पुत्र स्तनं पास्यसि देवज}


\twolineshloka
{कोनु स्वप्नस्तया दृष्टो या त्वामादित्यवर्चसम्}
{दिव्यवर्मसमायुक्तं दिव्यकृण्डलभूषितम्}


\twolineshloka
{पद्मायतविशालाक्षं पद्मताम्रदलोज्ज्वलम्}
{सुललाटं सुकेशान्तं पुत्रत्वे कल्पयिष्यति}


\twolineshloka
{धन्या द्रक्ष्यन्ति पुत्र त्वां भूमौ संसर्पमाणकम्}
{अव्यक्तकलवाक्यानि वदन्तं रेणुगुण्ठितम्}


\twolineshloka
{धन्या द्रक्ष्यन्ति पुत्र त्वां पुनर्यौवनगोचरम्}
{हिमवद्वनसंभूतं सिंहं केसरिणं यथा}


\twolineshloka
{एवं बहुविधं राजन्विलप्य करुणं पृथा}
{अवामृजतमज्जूषामश्वनद्यां तदा जले}


\twolineshloka
{रुदती पुत्रशोकार्ता निशीथे कमलेक्षणा}
{धात्र्या सह पृथा राजन्पुत्रदर्शनलालसा}


\twolineshloka
{विसर्जयित्वा मञ्जूषां संबोधनभयात्पितुः}
{विवेश राजभवनं पुनः शोकातुरा ततः}


\twolineshloka
{मञ्जूषा त्वश्वनद्याः सा ययौ चर्मयण्वतीं नदीम्}
{चर्मण्वत्याश्चयमुनां ततो गङ्गां जगाम ह}


\twolineshloka
{गङ्गायाः सूतविषयं चम्पामनुययौ पुरीम्}
{स मञ्जूषागतो गर्भस्तरङ्गैरुह्यमानकः}


\twolineshloka
{अमृतादुत्थितं दिव्यं तनुवर्म सकुण्डलम्}
{धारयामास तं गर्भं दैवं च विधिनिर्मितम्}


\twolineshloka
{एतद्गुह्यं महाराज सूर्यस्यासीन्महात्मनः}
{स सूर्यसंभवो गर्भः कुन्त्या गर्भेण धारितः'}


\chapter{अध्यायः ३१०}
\twolineshloka
{वैशंपायन उवाच}
{}


\twolineshloka
{एतस्मिन्नेव काले तु धृतराष्ट्रस्य वै सखा}
{सूतोऽधिरथ इत्येव सदारो जाह्नवीं ययौ}


\threelineshloka
{तस्य भार्याऽभवद्राजन्रूपेणासदृशी भुवि}
{राधा नाम महाभागा न सा पुत्रमविन्दत}
{अपत्यार्थे परं यत्नमकरोच्च विशेषतः}


\twolineshloka
{सा ददर्शाथ मञ्जूषामुह्यमानां यदृच्छया}
{दत्तरक्षाप्रतिसरामन्वालम्भनशोभनाम्}


\twolineshloka
{ऊर्मीतरङ्गैर्जाह्नव्याः समानीतामुपह्वरम्}
{`विवर्तमानां बहुशः पुनःपुनरितस्ततः}


\twolineshloka
{ततः सा वायुना राजन्स्रोतसा च बलीयसा}
{उपानीतो यतः सूतः सभार्यो जलमाश्रितः'}


\twolineshloka
{सा तु कौतूहलात्प्राप्तां ग्राहयामास भामिनी}
{ततो निवेदयामास सूतस्याधिरथस्य वै}


\twolineshloka
{स तामुद्धृत्य मञ्जूषामुत्सार्य जलमन्तिकात्}
{यन्त्रैरुद्घाटयामास सोऽपश्यत्तत्रबालकम्}


\twolineshloka
{मृष्टकुण्डलयुक्तेन वदनेन विराजितम्}
{`परिम्लानमुखं बालं रुदन्तं क्षुधितं भृशम्}


\threelineshloka
{स तु तं परया लक्ष्म्या दृष्ट्वा युक्तं वरात्मजम्'}
{स सूतो भार्यया सार्धं विस्मयोत्फुल्ललोचनः}
{अङ्कमारोप्य तं बालं भार्यां वचनमब्रवीत्}


\twolineshloka
{इदमत्यद्भुतं भीरु यतो जातोस्मि भामिनि}
{दृष्टवान्देवगर्भोऽयं मन्येऽस्माकमुपागतः}


\twolineshloka
{अनपत्यस्य पुत्रोऽयं देवैर्दत्तो ध्रुवं मम}
{इत्युक्त्वा तं ददौ पुत्रं राधायै स महीपते}


\twolineshloka
{रप्रतिजग्राह तं राधा विधिवद्दिव्यरूपिणम्}
{पुत्रं कमलगर्भाभं देवगर्भं श्रिया वृतम्}


\threelineshloka
{`स्तन्यं समस्रवच्चास्य दैवादित्थ निश्चयः'}
{पुपोष चैनं विधिवद्ववृधे स च वीर्यवान्}
{ततः प्रभृति चाप्यन्ये प्राभवन्नौरसाः सुताः}


\twolineshloka
{`नामकर्म च चक्रुस्ते कुण्डले तस् दृश्यते}
{कर्ण इत्येव तं बालं दृष्ट्वा कर्णं सकुण्डलम्'}


\twolineshloka
{वसुवर्मधरं दृष्ट्वा तं बालं हेमकुण्डलम्}
{नामास्य वसुषेणेति ततश्चक्रुर्द्विजातयः}


\twolineshloka
{एवं स सूतपुत्रत्वं जगामामितविक्रमः}
{वसुषेण इतिख्यातो वृष इत्येव च प्रभुः}


\twolineshloka
{सूतस्य ववृधेऽङ्गेषु ज्येष्ठः पुत्रः स वीर्यवान्}
{चारेण विदितश्चासीत्पृथया दिव्यवर्मभृत्}


\twolineshloka
{सूतस्त्वधिरथः पुत्रं विवृद्धं समयेन तम्}
{दृष्ट्वा प्रस्थापयामास पुरं वारणसाह्वयम्}


\twolineshloka
{तत्रोपसदनं चक्रे द्रोणस्येष्वस्त्रकर्मणि}
{सख्यं दुर्योधनेनैवमगमत्स च वीर्यवान्}


\twolineshloka
{द्रोणात्कृपाच्च रामाच्च सोऽस्त्रग्रामं चतुर्विधम्}
{लब्ध्वा लोकेऽभवत्ख्यातः परमेष्वासतां गतः}


\twolineshloka
{संधाय धार्तराष्ट्रेण पार्थानां विप्रिये रतः}
{योद्धुमाशंसते नित्यं फल्गुनेन महात्मना}


\twolineshloka
{सदा हि तस् स्पर्धाऽऽसीदर्जुनन विशांपते}
{अर्जुनस्य च कर्णेन यतो द्वन्द्वं बभूव ह}


\twolineshloka
{एतद्गुह्यं महाराज सूर्यस्यासीन्न संशयः}
{यः सूर्यसंभवः कर्णः कुर्यात्प्रतिकुले रतः}


\twolineshloka
{तं तु कुण्डलिनं दृष्ट्वा वर्मणा च समन्वितम्}
{अवध्यं समरे मत्वा पर्यतप्यद्युधिष्ठिरः}


\twolineshloka
{यदा च कर्णो राजेन्द्र भानुमन्तं दिवाकरम्}
{स्तौति मध्यंदिने प्राप्ते प्राञ्जलिः सलिलोत्थितः}


\twolineshloka
{तत्रैनमुपतिष्ठन्ति ब्राह्मणा धनहेतुना}
{नादेयं तस्य तत्काले किंचिदस्ति द्विजातिषु}


\twolineshloka
{तमिन्द्रो ब्राह्मणो भूत्वा भिक्षां देहीत्युपस्थितः}
{स्वागतं चेति राधेयस्तमथ प्रत्यभाषत}


\chapter{अध्यायः ३११}
\twolineshloka
{वैशंपायन उवाच}
{}


\twolineshloka
{देवराजमनुप्राप्तं ब्राह्मणच्छद्मनाऽऽवृतम्}
{दृष्ट्वास्वागतमित्याह न बुबोधास्य मानसम्}


\threelineshloka
{हिरण्यकण्ठीः प्रमदा ग्रामान्वा बहुगोकुलान्}
{किं ददानीति तं विप्रमुवाचाधिरथिस्ततः ॥ब्राह्मण उवाच}
{}


\twolineshloka
{हिरण्यकण्ठ्यः प्रमदा यच्चान्यत्प्रीतिवर्धनम्}
{नाह दत्तमिहेच्छामि तदर्थिभ्यः प्रदीयताम्}


\twolineshloka
{यदेतत्सहजं वर्म कुण्डले च तवानघ}
{एतदुत्कृत्य मे देहि यदि सत्यव्रतो भवान्}


\threelineshloka
{एतदिच्छाम्यहं भिक्षां त्वया दत्तां परंतप}
{एष मे सर्वलाभानां लाभः परमको मतः ॥कर्ण उवाच}
{}


\threelineshloka
{अवनिं प्रमदा गाश्च निर्वापं बहुवार्षिकम्}
{तत्ते विप्र प्रदास्यामि न तु वर्म सकुण्डलम् ॥वैशंपायन उवाच}
{}


\twolineshloka
{एवं बहुविधैर्वाक्यैर्वार्यमाणः स तु द्विजः}
{रकर्णेन भरतश्रेष्ठ नान्यं वरमयाचत}


\twolineshloka
{सान्त्वितश्च यथाशक्ति पूजितश्च यथाविधि}
{न चान्यं स द्विजश्रेष्ठः कामयामास वै वरम्}


\threelineshloka
{यदा नान्य प्रवृणुते वरं वै द्विजसत्तमः}
{`विनाऽस् सहजं वर्म कुण्डले च विशांपते'}
{तदैनमब्रवीद्भूयो राधेयः प्रहसन्निव}


\twolineshloka
{सहजं वर्म मे विप्र कुण्डले चामृतोद्भवे}
{तेनावध्योस्मि लोकेषु ततो नैतज्जहाम्यहम्}


\twolineshloka
{विशालं पृथिवीराज्यं क्षेमं निहतकण्टकम्}
{प्रतिगृह्णीष्व मत्तस्त्वं साधु ब्राह्मणपुङ्गव}


\threelineshloka
{कुण्डलाभ्यां विमुक्तोऽहंवर्मणा सहजेन च}
{दमनीयो भविष्यामि शत्रूणां द्विजसत्तम ॥वैशंपायन उवाच}
{}


\twolineshloka
{यदन्यं न वरं वव्रे भगवान्पाकशासनः}
{ततः प्रहस् कर्णस्तं पुनरित्यब्रवीद्वचः}


\twolineshloka
{विदितो देवदेवेश प्रागेवासि मम प्रभो}
{न तु न्याय्यं मया दातुं तव शक्र वृथा वरम्}


\twolineshloka
{त्वं हि देवेश्वरः साक्षात्त्वया देयो वरो मम}
{अन्येषां चैव भूतानामीश्वरो ह्यसि भूतकृत्}


\twolineshloka
{यदि दास्यामि ते देव कुण्डले कवचं तथा}
{वध्यतामुपयास्यामि त्वं च शक्रावहास्यताम्}


\threelineshloka
{तस्माद्विनिमयं कृत्वा कुण्डलेवर्म चोत्तमम्}
{हरस्व शक्रकामं मे न दद्यामहमन्यथा ॥शक्र उवाच}
{}


\twolineshloka
{विदितोऽहं रवेः पूर्वमायानेव तवान्तिकम्}
{तेन ते सर्वमाख्यातमेवमेतन्न संशयः}


\threelineshloka
{काममस्तु तथा तात तव कर्ण यथेच्छसि}
{वर्जयित्वा तु मे वज्रं प्रवृणीष्व यथेच्छसि ॥वैशंपायन उवाच}
{}


\threelineshloka
{ततः कर्णः प्रहृष्टस्तु उपसंगम्य वासवम्}
{अमोघां शक्तिमभ्येत्य वव्रे सम्पूर्णमानसः ॥क्रण उवाच}
{}


\twolineshloka
{वर्मणा कुण्डलाभ्यां च शक्तिं मे देहि वासव}
{अमोघां शत्रुसङ्घानां घातिनीं पृथनामुखे}


\twolineshloka
{ततः सञ्चिन्त्य मनसा मुहूर्तमिव वासवः}
{शक्त्यर्थं पृथिवीपाल कर्णं वाक्यमथाब्रवीत्}


\twolineshloka
{कुण्डले मे प्रयच्छस्व वर्म चैव शरीरजम्}
{गृहाण कर्ण शक्तिं त्वमनेन समयेन च}


\twolineshloka
{अमोघा हन्ति शतशः शत्रून्मम करच्युता}
{पुनश्च पाणिमभ्येति मम दैत्यान्विनिघ्नतः}


\threelineshloka
{सेयं तव करप्राप्ता हत्वैरकं रिपुमूर्जितम्}
{गर्जन्तं प्रतपन्तं च मामेवैष्यति सूतज ॥कर्ण उवाच}
{}


\threelineshloka
{एकमेवाहमिच्छामि रिपुं हन्तुं महाहवे}
{गर्जन्तंप्रतपन्तं च यतो मम भयं भवेत् ॥इन्द्र उवाच}
{}


\twolineshloka
{एकं हनिष्यसि रिपुं गर्जन्तं बलिनं रणे}
{त्वं तु यं प्रार्थयस्येकं रक्ष्यते स महात्मना}


\threelineshloka
{यमाहुर्वेदविद्वांसो वराहमपराजितम्}
{नारायणमचिन्त्यं च तेन कृष्णेन रक्ष्यते ॥कर्ण उवाच}
{}


\twolineshloka
{`एवमेतद्यथाऽऽत्थ त्वं दानवानां निषूदन}
{वधिष्यामि रणे शत्रुं यो मे स्थाता पुरस्सरः'}


\twolineshloka
{एवमप्यस्तु भगवन्नेकवीरवधे मम}
{अमोघां देहि मे शक्तिं यथा हन्यां प्रतापिनम्}


\threelineshloka
{उत्कृत्य तु प्रदास्यामि कुण्डले कवचं च ते}
{निकृत्तेषु तु गात्रेषु न मे बीभत्सता भवेत् ॥इन्द्र उवाच}
{}


\twolineshloka
{न ते बीभत्सता कर्ण भविष्यति कथञ्चन}
{ब्रणश्चैव न गात्रेषु यस्त्वं नानृतमिच्छसि}


\twolineshloka
{यादृशस्ते पितुर्वर्णस्तेजश्च वदतांवर}
{तादृशेनैव वर्णेन त्वं कर्ण भविता पुनः}


\threelineshloka
{विद्यमानेषु शस्त्रेषु यद्यमोघामसंशये}
{प्रमत्तो मोक्ष्यसे चापि त्वय्येवैषा पतिष्यति ॥कर्ण उवाच}
{}


\threelineshloka
{संशयं परमं प्राप्य विमोक्ष्ये वासवीमिमाम्}
{यथा मामौत्थ शक्र त्वं सत्यमेतद्ब्रवीमि ते ॥वैशंपायन उवाच}
{}


\twolineshloka
{ततः शक्तिं प्रज्वलितां प्रतिगृह् विशांपते}
{शस्त्रं गृहीत्वा निशितं सर्वगात्राण्यकृन्तत}


\twolineshloka
{ततो देवा मानवा दानवाश्चनिकृन्तन्तं कर्णमात्मानमेव}
{दृष्ट्वा सर्वे सिंहनादान्प्रणेदु-र्न ह्यस्यासीन्मुखजो वै विकारः}


\twolineshloka
{ततो दिव्या दुन्दुभयः प्रणेदुःपपातोच्चैः पुष्पवर्षं च दिव्यम्}
{दृष्ट्वा कर्णं शस्त्रसंकृत्तगात्रंमहुश्चापि स्मयमानं नृवीरम्}


\twolineshloka
{ततश्छित्त्वा कवचं दिव्यमङ्गा-त्तथैवार्द्रं प्रददौ वासवाय}
{तथोत्कृत्य प्रददौ कुण्डले तेकर्णात्तस्मात्कर्मणा तेन कर्णः}


\threelineshloka
{`ततो देवो मुदितो वज्रपाणि-र्दृष्ट्वा कर्णं शस्त्रनिकृत्तगात्रम्'}
{ततः शक्रः प्रहसन्वञ्चयित्वाकर्णं लोके यशसा योजयित्वा}
{कृतंकार्यं पाण्डवनां हि भेनेततः पश्चाद्दिवमेवोत्पपात}


\threelineshloka
{श्रुत्वा कर्णं मुषितं धार्तराष्ट्रादीनाः सर्वे भग्नदर्पा इवासन्}
{तां रचावस्थां गमितं सूतपुत्रंश्रुत्वा पार्था जहृपुः काननस्थाः ॥जनमेजय उवाच}
{}


\threelineshloka
{क्वस्ता वीराः पाण्डवास्ते बभूवुःकुतश्चैते श्रुतवन्तः प्रियं तत्}
{किं वाऽकार्षुर्द्वादशेऽब्दे व्यतीतेतन्मे सर्वं भगवान्व्याकरोतु ॥वैशंपायन उवाच}
{}


\twolineshloka
{लब्ध्वा कृष्णां सैन्धवं द्रावयित्वाविप्रैः सार्धं काम्यकादाश्रमात्ते}
{मार्कण्डेयाच्छ्रुतवन्तः पुराणंकदेवर्षीणआं चरितं विस्तरेण}


\twolineshloka
{`प्रत्याजग्मुः सरथाः सानुयात्राःसर्वैः सार्धं सूतपौरोगवैस्ते}
{ततो ययुर्द्वैतवनं नृवीरानिस्तीर्यैवं वनवासं समग्रम्'}


\chapter{अध्यायः ३१२}
\twolineshloka
{जनमेजय उवाच}
{}


\threelineshloka
{एवं हृतायां भार्यायां प्राप्य क्लेशमनुत्तमम्}
{प्रतिपद्य ततः कृष्णां किमकुर्वत पाण्डवाः ॥वैशंपायन उवाच}
{}


\twolineshloka
{एवं हृतायां कृष्णायां प्राप्य क्लेशमनुत्तमम्}
{विहाय काम्यकं राजा सह भ्रातृभिरच्युतः}


\twolineshloka
{पुनर्द्वैतवनं रम्यमाजगाम युधिष्ठिरः}
{स्वादुमूलफलं रम्यं विचित्रबहुपादपम्}


\twolineshloka
{अनुभुक्तफलाहाराः सर्व एव मिताशनाः}
{न्यवसन्पाण्डवास्तत्रकृष्णया सह भार्यया}


\twolineshloka
{वसन्द्वैतवने राजा कुन्तीपुत्रो युधिष्ठिरः}
{भीमसेनोऽर्जुनश्चैव माद्रीपुत्रौ च पाण्डवौ}


\twolineshloka
{ब्राह्मणार्थे पराक्रान्ता धर्मात्मानो यतव्रताः}
{क्लेशमार्च्छन्त विपुलं सुखोदर्कं परंतपाः}


\twolineshloka
{तस्मिन्प्रतिवसन्तस्ते यत्प्रापुः कुरुसत्तमाः}
{वने क्लेशं सुखोदर्कं तत्प्रवक्ष्यामि ते शृणु}


\twolineshloka
{अरणीसहितं भाण्डं ब्राह्मणस्य तपस्विनः}
{मृगस् घर्षणस्य विपाणे समसज्जत}


\twolineshloka
{तदादाय गतो राजंस्त्वरमाणो महामृगः}
{आश्रमान्तरितः शीघ्रं प्लवमानो महाजवः}


\threelineshloka
{ह्रियमाणं तु तं दृष्ट्वा स विप्रः कुरुसत्तम}
{त्वरितोऽभ्यागमत्तत्रअग्निहोत्रपरीप्सया}
{`तेषां तु वसतां तत्र पाण्डवानां महारथम्'}


\twolineshloka
{अजातशत्रुमासीनं भ्रातृभिः सहितं वने}
{आगम्य ब्राह्मणस्तूर्णं संतप्तश्चेदमब्रवीत्}


\twolineshloka
{अरणीसहितं भाण्डं समासक्तं वनस्पतौ}
{मृगस्य घऱ्षमाणस्य विषाणे समसज्जत}


\twolineshloka
{तमादाय गतो राजंस्त्वरमाणो महामृगः}
{आश्रमात्त्वरितः शीघ्रं प्लवमानो महाजवः}


\twolineshloka
{तस्य गत्वा पदं राजन्नासाद्य च महामृगम्}
{अग्निहोत्रं न लुप्येत तदानयत पाण्डवाः}


\twolineshloka
{ब्राह्मणस्य वचः श्रुत्वा सन्तप्तोऽथ युधिष्ठिरः}
{धनुरादाय कौन्तेयः प्राद्रवद्भ्रातृभिः सह}


\twolineshloka
{सन्नद्धा धन्विनः सर्वे प्राद्रवन्नरपुङ्गवाः}
{ब्राह्मणार्थे यतन्तस्ते शीघ्रमन्वगमन्मृगम्}


\twolineshloka
{कर्णिनालीकनाराचानुत्सृजन्तो महारथाः}
{नाविध्यन्पाण्डवास्तत्र पश्यन्तो मृगमन्तिकात्}


\twolineshloka
{तेषां प्रयतमानानां नादृश्यत महामृगः}
{अपश्यन्तोमृगं श्रान्ता दुःखं प्राप्ता मनस्विनः}


\twolineshloka
{शीतलच्छायमागमय् न्यग्रोधं गहने वने}
{क्षुत्पिपासापरीताङ्गाः पाण्डवाः समुपाविशन्}


\twolineshloka
{तेषां समुपविष्टानां नकुलो दुःखितस्तदा}
{अब्रवीद्भ्रातरं श्रेष्ठममर्षात्कुरुनन्दनम्}


\twolineshloka
{नास्मिन्कुले जातु ममज्ज धर्मोन चालस्यादर्थलोपो बभूव}
{अनुत्तराः सर्वभूतेषु भूपसंप्राप्ताः स्मः संशयं किंनु राजन्}


\chapter{अध्यायः ३१३}
\twolineshloka
{युधिष्ठिर उवाच}
{}


\threelineshloka
{नापदामस्ति मर्यादा न निमित्तं न कारणम्}
{धर्मस्तु विभजत्यर्थमुभयोः पुण्यपापयोः ॥भीम उवाच}
{}


\threelineshloka
{प्रातिकाम्यनयत्कृष्णां सभायां प्रेष्यवत्तदा}
{न मया निहतस्तत्रतेन प्राप्ताः स्म संशयम् ॥अर्जुन उवाच}
{}


\threelineshloka
{वाचस्तीक्ष्णास्थिभेदिन्यः सूतपुत्रेण भाषिताः}
{अतितीव्रा मया क्षान्तास्तेन प्राप्ताः स्म संशयं ॥सहदेव उवाच}
{}


\threelineshloka
{शकुनिस्त्वां यदाऽजैषीदक्षद्यूतेन भारत}
{स मया न हतस्तत्रतेन प्राप्ताः स्म संशयम् ॥वैशंपायन उवाच}
{}


\twolineshloka
{ततो युधिष्ठिरो राजा नकुलं वाक्यमब्रवीत्}
{आरुह्य वृक्षं माद्रेय निरीक्षस्व दिशो दश}


\twolineshloka
{पानीयमन्तिके पश्य वृक्षान्वाप्युदकाश्रयान्}
{एते हि भ्रातरः श्रान्तास्तव तात पिपासिताः}


\threelineshloka
{नकुलस्तु तथेत्युक्त्वा भ्रातुर्ज्येऽष्ठस्य शासनात्}
{तत उत्थाय मतिमाञ्शीघ्रमारुह्य पादपम्}
{अब्रवीद्धांतरं ज्येष्मभिवीक्ष्य समन्ततः}


\twolineshloka
{पश्यामि बहुलान्राजन्वृक्षानुदकसंश्रयान्}
{सारसानां च निर्ह्रादस्तत्रोदकमसंशयम्}


\twolineshloka
{ततोऽब्रवीत्सत्यधृतिः कुन्तीपुत्रो युधिष्ठिरः}
{गच्छ सौम्य ततः शीघ्रं तूणैः रपानीयमानय}


\twolineshloka
{नकुलस्तु तथेत्युक्त्वा भ्रातुर्ज्येष्ठस्य शासनात्}
{प्राद्रवद्यत्र पानीयं शीघ्रं चैवान्वपद्यत}


\threelineshloka
{स दृष्ट्वा विमलं तोयं सारसैः परिवारितम्}
{पातुकामस्ततो वाचमन्तरिक्षात्स शुश्रुवे ॥यक्ष उवाच}
{}


\twolineshloka
{मा तात साहसं कार्षीर्मम पूर्वपरिग्रहः}
{प्रश्नानुक्त्वा तु माद्रेय ततः पिब हरस्व च}


\twolineshloka
{अनादृत्य तु तद्वाक्यं नकुलः सुपिपासितः}
{अपिबच्छीतलं तोयं पीत्वा च निपपात ह}


\twolineshloka
{चिरायमाणे नकुले कुन्तीपुत्रो युधिष्ठिरः}
{अब्रवीद्ध्रातरं वीरं सहदेवमरिंदमम्}


\twolineshloka
{भ्राता चिरायते तात सहदेव तवाग्रजः}
{तं चैवानय सोदर्यं पानीयं च त्वमानय}


\twolineshloka
{सहदेवस्तथेत्युक्त्वा तां दिशं प्रत्यपद्यत}
{ददर्श च हतं भूमौ भ्रातरं नकुलं तदा}


\twolineshloka
{भ्रातृशोकाभिसंतप्तस्तृषया च प्रपीडितः}
{अभिदुद्राव रपानीयं ततो वागभ्यभाषत}


\twolineshloka
{मा तात साहसं कार्षीर्मम पूर्वपरिग्रहः}
{प्रश्नानुक्त्वा यथाकामं पिबस्व च हरस्व च}


\twolineshloka
{अनादृत्य तु तद्वाक्यं सहदेवः पिपासितः}
{अपिबच्छीतलं तोयं पीत्वा च निपपात ह}


\twolineshloka
{अथाब्रवीत्स विजयं कुन्तीपुत्रो युधिष्ठिरः}
{भ्रातरौ ते चिरगतौ बीभत्सो शत्रुकर्शन}


\twolineshloka
{तौ चैवानय भद्रं ते पानीयं च त्वमानय}
{त्वं हि नस्तात सर्वेषां दुःखितानामपाश्रयः}


\twolineshloka
{एवमुक्तो गुडाकेशः प्रगृह्य सशरं धनुः}
{आमुक्तखङ्गो मेधावी तत्सरः प्रत्यपद्यत}


\twolineshloka
{यतः पुरुषशार्दूलौ पानीयहरणे गतौ}
{तौ ददर्श हतौ तत्रभ्रातरौ श्वेतवाहनः}


\threelineshloka
{`विगतासू नरव्याघ्रौ शयानौ वसुधातले'}
{प्रसुप्ताविव तौ दृष्ट्वा नरसिंहः सुदुःखितः}
{धनुरुद्यम्य कौन्तेयो व्यलोकयत तद्वनम्}


\twolineshloka
{नापश्यत्तत्रकिंचित्स भूतमस्मिन्महावने}
{सव्यसाची पिपासार्तः पानीयं सोभ्यधावत}


\twolineshloka
{अभिधावंस्ततो वाचमन्तरिक्षात्स शुश्रुवे}
{यत्त्वमिच्छसि पानीयं नैतच्छक्यं बलात्त्वया}


\twolineshloka
{कौन्तेय यदि वै प्रश्नान्मयोक्तान्प्रतिवक्ष्यसि}
{ततः पास्यसि पानीयं हरिष्यसि च भारत}


\twolineshloka
{वारितस्त्वब्रवीत्पार्थो दृश्यमानो निवारय}
{यावद्बाणैर्विनिर्भिन्नः पुनर्नैवं वदिष्यसि}


\fourlineindentedshloka
{एवमुक्त्वा ततः पार्थः शरैरस्त्रानुमन्त्रितैः}
{प्रववर्ष रदिशः कृत्स्नाः शब्दवेधं च दर्शयन्}
{कर्णिनालीकनाराचानुत्सृजन्भरतर्षभः}
{}


\threelineshloka
{स त्वमोघानिषून्मुक्त्वा तृष्णयाऽभिप्रपीडितः}
{अनेकैरिषुसंघातैरन्तरिक्षे ववर्ष ह ॥यक्ष उवाच}
{}


\twolineshloka
{किं विधानेन ते पार्थ प्रश्नानुक्त्वा पयः पिब}
{अनुक्त्वा च पिबन्प्रश्नान्पीत्वैव नभविष्यसि}


\twolineshloka
{स च मोघानिषून्दृष्ट्वातृष्णया च प्रपीडितः}
{अवज्ञायैव तां वाचं पीत्वैव निपपात् ह}


\twolineshloka
{अथाब्रवीद्भीमसेनं कुनतीपुत्रो युधिष्ठिरः}
{नकुलः सहदेवश्च बीभत्सुश्च परंतप}


\twolineshloka
{चिरंगतास्तोयहेतोर्न चागच्छन्ति भारत}
{तांश्चैवानय भद्रं ते पानीयं च त्वमानय}


\twolineshloka
{भीमसेनस्तथेत्युक्त्वा तं देशं प्रत्यपद्यत}
{रयत्रते पुरुषव्याघ्रा भ्रातरोस्य निपातिताः}


\twolineshloka
{तान्दृष्ट्वा दुःखितो भीमस्तृषया च प्रपीडितः}
{अमन्यत महाबाहुः कर्म तद्यक्षरक्षसाम्}


\twolineshloka
{सचिन्तयामास तदा योद्धव्यं ध्रुवमद्य मे}
{पास्यामि तावत्पानीयमिति पार्थो वृकोदरः}


\twolineshloka
{ततोऽभ्यधावत्पानीयं पिपासुः पुरुषर्षभः ॥यक्ष उवाच}
{}


\twolineshloka
{मा तात साहसंकार्षीर्मम पूर्वपरिग्रहः}
{प्रश्नानुक्त्वा तु कौन्तेय ततः पिब हरस्व च}


\twolineshloka
{एवमुक्तस्तदा भीमो यक्षेणामिततेजसा}
{अनुक्त्वैव तु तान्प्रश्नान्पीत्वैव निपपात ह}


\twolineshloka
{ततः कुन्तीसुतो राजा प्रचिन्त्य पुरुषर्षभः}
{`आत्मनाऽऽत्मानमन्विष्य विचारमकरोत्प्रभुः}


\twolineshloka
{ततश्चिरगतान्भ्रातृनथाऽऽज्ञाय युधिष्ठिरः}
{चिरायमाणान्बहुशः पुनः पुनरुवाचह}


\twolineshloka
{किंस्विद्वनमिदं दग्धं किंखिद्दृष्टो मृतो भवेत्}
{प्रहरन्तो महाभूतं शप्तास्तेनाथ तेऽपतन्}


\twolineshloka
{न पश्यन्त्यथवा वीराः पानीयं यत्रते गताः}
{अन्विच्छद्भिर्वने तोयं कालोऽयमतिपातितः}


\twolineshloka
{किंनु तत्कारणं येन नायान्ति पुरुषर्षभाः}
{गच्छाम्येषां पदं द्रष्टुमिति कृत्वा युधिष्ठिरः'}


\twolineshloka
{समुत्थाय महाबुद्धिर्दह्यमानेन चेतसा}
{व्यपेतजननिर्घोषं प्रविवेश महावनम्}


\twolineshloka
{रुरुभिश्च वराहैश्च पक्षिभिश्च निषेवितम्}
{नीलभास्वरवर्णैश्च पादपैरुशोभितम्}


\twolineshloka
{भ्रमरैरुपगीतं च पक्षिभिश् समन्ततः}
{`मृदुशाड्वलसंकीर्णभूमिभागं मनोहरम्'}


\twolineshloka
{स गच्छन्कानने तस्मिन्हेमजालपरिष्कृतम्}
{ददर्श तत्सरः श्रीमान्विश्वकर्मकृतं यथा}


\twolineshloka
{उपेतं नलिनीजालैः सिन्धुवारैः सचेतसैः}
{केतकैः करवीरैश्च पिप्पलैश्चैव संवृतम्}


\twolineshloka
{`ततो धर्मसुतः श्रीमान्भ्रातृदर्शनलालसः'}
{श्रमार्तस्तदुपागम्य सरो दृष्ट्वाऽथ विस्मितः}


\chapter{अध्यायः ३१४}
\twolineshloka
{वैशंपायन उवाच}
{}


\twolineshloka
{स ददर्श हतान्भ्रातृँल्लोकपालानिव च्युतान्}
{युगान्ते समनुप्राप्ते शक्रवैश्रवणोपमान्}


\twolineshloka
{विनिकीर्णधनुर्बाणं दृष्ट्वा निहतमर्जुनम्}
{भीमसेनं यमौ चैव निर्विचेष्टान्गतायुषः}


\threelineshloka
{सदीर्घमुष्णं निःश्वस्य शोकबाष्पपरिप्लुतः}
{तान्दृष्ट्वा पतितान्भ्रातॄन्सर्वांश्चिन्तासमन्वितः}
{}


\twolineshloka
{ननु त्वया महाबाहो प्रतिज्ञातं वृकोदर}
{सुयोधनस्य भेत्स्यामि गदया सक्थिनी रणे}


\twolineshloka
{व्यर्थं तदद्य मे सर्वं त्वयि वीरे निपातिते}
{महात्मनि महाबाहो कुरूणां कीर्तिवर्धने}


\twolineshloka
{मनुष्यसंभवा वाचो विधर्मिण्यः प्रतिश्रुताः}
{भवतांदिव्यवाचस्तु ता भवन्तु कथं मृपा}


\twolineshloka
{देवाश्चापि यदाऽवोचन्मूतके त्वां धनंजय}
{सहस्राक्षादनवरः कुन्ति पुत्रस्तवेति वै}


\twolineshloka
{उत्तरे पारियात्रे च जगुर्भूतानि सर्वशः}
{विप्रनष्टां श्रियं चैषामाहर्ता पुनरोजसा}


\twolineshloka
{नास्य जेता रणे कश्चिदजेता नैष कस्यचित्}
{सोयं मृत्युवशं यातः कथं जिष्णुर्महाबलः}


\twolineshloka
{अयंममाशां संहत्य शेते भूमौ धनंजयः}
{आश्रित्ययं वयं नाथं दुःखान्येतानिसेहिम}


\threelineshloka
{रणे प्रगल्भौ वीरौ चसदा शत्रुनिबर्हणौ}
{कथं रिपुवशं यातौ कुन्तीपुत्रौ महाबलौ}
{यौ सर्वास्त्राप्रतिहतौ भीमसेनधनंजयौ}


\twolineshloka
{अश्मसारमयं नूनं हृदयं मम दुर्हृदः}
{यमौ यदेतौ दृष्ट्वाऽद्य पतितौ नावदीर्यते}


\twolineshloka
{शास्त्रज्ञा देशकालज्ञास्तपोयुक्ताः क्रियान्विताः}
{अकृत्वा सदृशं कर्म किं शेध्वं पुरुषर्षभाः}


\twolineshloka
{अविक्षतशरीराश्चाप्यप्रमृष्टशरासनाः}
{असंज्ञा भुवि संगम्य किं शेष्वमपराजिताः}


\twolineshloka
{सानूनिवाद्रेः संसुप्तान्दृष्ट्वा भ्रातृन्महामतिः}
{सुखं प्रसुप्तान्प्रस्विन्नः खिन्नः कष्टां दशां गतः}


\twolineshloka
{एवमेवेदमित्युक्त्वा धर्मात्मा स नरेश्वरः}
{शोकसागरमध्यस्थो दध्यौ कारणमाकुलः}


\twolineshloka
{इतिकर्तव्यतां चेति देशकालविभागवित्}
{नाभिपेदे महाबाहुश्चिन्तयानो महामतिः}


\threelineshloka
{अथसंस्तभ्य धर्मात्मा तदाऽऽत्मानं तपःसुतः}
{एवंविलप्य बहुधा धर्मपुत्रो युधिष्ठिरः}
{बुद्ध्या विचिन्तयामासवीराः केन निपातिताः}


\twolineshloka
{नैषां शस्त्रप्रहारोस्ति पदं नेहास्ति कस्यचित्}
{भूतं महदेदं मन्ये भ्रातरो येन मे हताः}


\twolineshloka
{एकाग्रं चिन्तयिष्यामि पीत्वा वेत्स्यामि वा जलम्}
{`भ्रातॄणां न्न्यसनं घोरं सममेव महात्मनाम्'}


\twolineshloka
{स्यात्तु दुर्योधनेनेदमुपांशु परिकल्पितम्}
{गान्धारराजरचितं सततं जिह्मवुद्धिना}


\twolineshloka
{यस् कार्यमकार्यं वा सममेव भवत्युत}
{कस्तस्य विश्वसेद्वीरो दुष्कृतेरकृतात्मनः}


\twolineshloka
{अथवा पुरुषैर्गूढैः प्रयोगोऽयंदुरात्मनः}
{भवेदिति महाबुद्धिर्बहुधा समचिन्तयत्}


\twolineshloka
{`आचार्यं किंनु वक्ष्यामि कृपं भीष्ममहं नु किम्}
{विदुरं किंनु वक्ष्यामि बृहस्पतिसमं नये}


\twolineshloka
{अम्बां च किंनु वक्ष्यामि सर्वदा दुःखभागिनीम्}
{दृष्ट्वा मां भ्रातृभिर्हीनं पृच्छन्तीं पुत्रगृद्धिनीम्}


\twolineshloka
{यदा त्वं भ्रातृभिः सर्वैः शक्रतुल्यपराक्रमैः}
{सार्धं वनं गतो वीरैः कथमेकस्त्वमागतः'}


\threelineshloka
{कस्य किंनु विषेणेदमुदकं दूपितं यथा}
{मृतानामपि चैतेषां विकृतं नैव जायते}
{मुखवर्णाः प्रसन्ना मे भ्रातॄणामित्यचिन्तयत्}


\twolineshloka
{एकैकशश्चौघबलानिमान्पुरुपसत्तमान्}
{कोऽन्यः प्रतिसमासेत कालान्तकयमादृते}


\threelineshloka
{एतेन व्यवसायेन तत्तोयं व्यवगाढवान्}
{पातुकामश्च तत्तोयमन्तरिक्षात्स शुश्रुवे ॥यक्ष उवाच}
{}


\twolineshloka
{अहं बकः शैवलमत्स्यभक्षोनीता मया प्रेतवशं तवानुजाः}
{त्वं पञ्चमो भविता राजपुत्रन चेत्प्रश्नान्पृच्छतो व्याकरोपि}


\threelineshloka
{मा तात साहसंकार्पीर्मम पूर्वपरिग्रहः}
{प्रश्नानुक्त्वा तु कौन्तेय ततः पिब हरस्व च ॥युधिष्ठिर उवाच}
{}


\twolineshloka
{रुद्राणां वा वसूनां वामरुतां वा प्रधानभाक्}
{पृच्छामि को भवान्देवो नैतच्छकुनिना कृतम्}


\twolineshloka
{हिमवान्पारियात्रश्च विन्ध्यो भलय एव च}
{चत्वारः पर्वताः केन पातिता भुवि तेजसा}


\twolineshloka
{त्वयाऽतीव महत्कर्म कृतं च बलिनांवर}
{`विनिघ्नता महेष्वासांश्चतुरोपि ममात्मजान्'}


\twolineshloka
{यान्न देवान गन्धर्वानासुराश्च न राक्षसाः}
{विपहेरन्महायुद्धे कृतं ते तन्महाद्भुतम्}


\twolineshloka
{न ते जानामि यत्कार्यं नाभिजानामि काङ्क्षित्तम्}
{कौतूहलं महज्जातं साध्वसं चागतं मम}


\threelineshloka
{येनास्स्युद्विग्नहृदयः समुत्पन्नशिरोज्वरः}
{पृच्छामि भगवंस्तस्मात्को भवानिह तिष्ठति ॥यक्ष उवाच}
{}


\threelineshloka
{यक्षोऽहमस्मि भद्रं ते नास्मि पक्षी जलेचरः}
{मयैते निहता सर्वे भ्रातरस्ते निवारिताः ॥वैशंपायन उवाच}
{}


\twolineshloka
{ततस्तामशिवां श्रुत्वावाचं स परुपाक्षराम्}
{यक्षस् ब्रुवतो राजन्नाकम्पत तदाऽऽस्थितः}


\twolineshloka
{विरूपाक्षं महाकायं यक्षं तालसमुच्छ्रयम्}
{ज्वलनार्कप्रतीकाशमधृष्यं पर्वतोपमम्}


\twolineshloka
{सेतुमाश्रित्य तिष्ठन्तं दद्रश भरतर्षभः}
{मेघगम्भीरनादेन तर्जयन्तं महास्वनम्}


% Check verse!
`उवाच यक्षः कौन्तेयं भ्रातृशोकप्रपीडितम्'
\threelineshloka
{इमे तेभ्रातरो राजन्वार्यमाणआ मयाऽसकृत्}
{बलात्तोयं जिहीर्षन्तस्ततो वै मृदिता मया}
{न पेयमुदकं राजन्प्राणानिह परीप्सता}


\threelineshloka
{पार्थ मा साहसं कार्पीर्मम पूर्वपरिग्रहः}
{प्रश्नानुक्त्वा तु कौन्तेय ततःपिब हरस्व च ॥यूधिष्ठिर उवाच}
{}


% Check verse!
न चाहं कामये यक्ष तव पूर्वपरिग्रहम्
\fourlineindentedshloka
{कामं नैतत्प्रसंसन्ति सन्तो हि पुरुषाः सदा}
{यदात्मना स्वमात्मानं प्रशंसेत्पुरुषर्षभ}
{यथाप्रज्ञं तु ते प्रश्नान्प्रतिवक्ष्यामि पृच्छ माम् ॥यक्ष उवाच}
{}


\threelineshloka
{किंस्विदादित्यमुन्नयति के च तस्याभितश्चराः}
{कश्चैनमस्तं नयतिकस्मिंश्च प्रतितिष्ठति ॥युधिष्ठिर उवाच}
{}


\threelineshloka
{ब्रह्मादित्यमुन्नयति देवास्तस्याभितश्चराः}
{धर्मश्चास्तं नयति च सत्ये च प्रतितिष्ठति ॥यक्ष उवाच}
{}


\threelineshloka
{केन स्विच्छ्रोत्रियो भवति केन स्विद्विन्दते महत्}
{केन स्विद्द्वितीयवान्भवतिराजन्केन च बुद्दिमान् ॥युधिष्ठिर उवाच}
{}


\threelineshloka
{श्रुतेन श्रोत्रियो भति रतपसा विन्दते महत्}
{धृत्या द्वितीयवान्भवति बुद्धिमान्वृद्धसेवया ॥यक्ष उवाच}
{}


\threelineshloka
{किं ब्राह्मणानां देवत्वं कश्च धर्मः सतामिव}
{कश्चैषां मानुषो भावः किमेषामसतामिव ॥युधिष्ठिर उवाच}
{}


\threelineshloka
{स्वाध्याय एषां देवत्वं तप एषां सतामिव}
{मरणं मानुषो भावः परिवादोऽसतामिव ॥यक्ष उवाच}
{}


\threelineshloka
{किं क्षत्रियाणां देवत्वं कश्च धर्मः सतामिव}
{कश्चैषां मानुषो भावः किमेषामसतामिव ॥युधिष्ठिर उवाच}
{}


\threelineshloka
{इष्वस्त्रमेषां देवत्वं यज्ञ एषां सतामिव}
{भयं वै मानुषो भावः परित्यागोऽसतामिव ॥यक्ष उवाच}
{}


\threelineshloka
{किमेकं यज्ञियं साम किमेकं यज्ञियं यजुः}
{का चैषां वृणुते यज्ञं कां यज्ञो नातिवर्तते ॥युधिष्ठिर उवाच}
{}


\threelineshloka
{प्राणो वै यज्ञियंसाम मनो वै यज्ञियं यजुः}
{ऋगेका वृणुते यज्ञं तां यज्ञो नातिवर्तते ॥यक्ष उवाच}
{}


\threelineshloka
{किंस्विदावपतां श्रेष्ठं रकिंस्विन्निवपतां वरम्}
{किंस्वित्प्रतिष्ठमानानां किस्वित्प्रसवतांवरम् ॥युधिष्ठिर उवाच}
{}


\threelineshloka
{वर्षमावपतां श्रेष्ठं बीजं निवपतां वरम्}
{गावः प्रतिष्ठमानानां पुत्रः प्रसवतां वरः ॥यक्ष उवाच}
{}


\threelineshloka
{इन्द्रियार्थाननुभवन्बुद्धिमाँल्लोकपूजितः}
{संमतः सर्वभूतानामुच्छ्वसन्को न जीवति ॥युधिष्ठिर उवाच}
{}


\threelineshloka
{देवतातिथिभृत्यानां पितॄणामात्मनश्च यः}
{न निर्वपति पञ्चानामुच्छ्वसन्न स जीवति ॥यक्ष उवाच}
{}


\threelineshloka
{किंस्विद्गुरुतरं भूमेः किंस्विदुच्चतरं च स्वात्}
{किंस्विच्छीघ्रतरं वायोः किंस्विद्बहुतरं तृणात् ॥युधिष्ठिर उवाच}
{}


\threelineshloka
{माता गुरुतरा भूमेः खात्पितोच्चतरस्तथा}
{मनः शीघ्रतरं वाताच्चिन्ता बहुतरी तृणात् ॥यक्ष उवाच}
{}


\threelineshloka
{किंस्वित्सुप्तं न निमिषति किंस्विज्जातं न चेङ्गते}
{कस्यस्विद्धृदयं नास्तिकास्विद्वेगेन वर्धते ॥युधिष्ठिर उवाच}
{}


\threelineshloka
{मत्स्यः सुप्तो न निमिषत्यण्डं जातं न चेङ्गते}
{अश्मनो हृदयंनास्ति नदी वेगेन वर्धते ॥अक्ष उवाच}
{}


\threelineshloka
{किंस्वित्प्रवसतो मित्रं किंस्विन्मित्रं गृहे सतः}
{आतुरस् च किं मित्रं किंस्विन्मित्रं मरिष्यतः ॥युधिष्ठिर उवाच}
{}


\threelineshloka
{विद्या प्रवसतो मित्रं भार्या मित्रं गृहे सतः}
{आतुरस्य भिषङ्भित्रं दानं मित्रं मरिष्यतः ॥यक्ष उवाच}
{}


\threelineshloka
{कोऽतिथिः सर्वभूतानां किं स्विद्धर्मं सनातनम्}
{अमृतं किंस्विद्राजेन्द्रकिंस्वित्सर्वमिदं जगत् ॥युधिष्ठिर उवाच}
{}


\threelineshloka
{अतिथिः सर्वभूतानामग्निः सोमो गवामृतम्}
{सनातनोऽमृतो धर्मो वायुः सर्वमिदं जगत् ॥यक्ष उवाच}
{}


\threelineshloka
{किंस्विदेको विचरते जातः को जायते पुनः}
{किंस्विद्धिमस्य भैषज्यं किंस्विदावपनं महत् ॥युधिष्ठिर उवाच}
{}


\threelineshloka
{सूर्य एको विचरते यन्द्रमा जायते पुनः}
{अग्निर्हमस्य भैषज्यं भूमिरावपनं महत् ॥यक्ष उवाच}
{}


\threelineshloka
{किंस्विदेकपदं धर्म्यं किंस्विदेकपदं यशः}
{किंस्विदेकपदं स्वर्ग्यं किंस्विदेकपदं सुखम् ॥युधिष्ठिर उवाच}
{}


\threelineshloka
{दाक्ष्यमेकपदं धर्म्यं दानमेकपदं यशः}
{सत्यमेकपदं स्वर्ग्यं शीलमेकपदंसुखम् ॥यक्ष उवाच}
{}


\threelineshloka
{किंस्विदात्मा मनुष्यस् किंस्विद्दैवकृतः सखा}
{उपजीवनं किस्विदस् किंस्विदस्य परायणम् ॥युधिष्ठिर उवाच}
{}


\threelineshloka
{पुत्र आत्मा मनुष्यस्य भार्या दैवकृतः सखा}
{उपजीवनं च पर्जन्यो दानमस् परायणम् ॥यक्ष उवाच}
{}


\twolineshloka
{धन्यानामुत्तमं किंस्विद्धनानां स्यात्किमुत्तमम्}
{लाभानामुत्तमं किंस्यात्सुखानां स्यात्किमुत्तमं}


\threelineshloka
{धन्यानामुत्तमं दाक्ष्यंधनानामुत्तमं श्रुतम्}
{लाभानां श्रेय आरोग्यं सुखानां तुष्टिरुत्तमा ॥यक्ष उवाच}
{}


\threelineshloka
{किंस्विद्धर्मपरंलके कश्च धर्मः सदाफलः}
{किं नियम्य न शोचन्ति कैश् सन्धिर्न जीर्यते ॥युधिष्ठिर उवाच}
{}


\threelineshloka
{आनृशंस्यं परं धर्मात्रेताधर्मः सदाफलः}
{मनो यम्य न शोचन्ति सन्धिः सद्भिर्न जीर्यते ॥यक्ष उवाच}
{}


\threelineshloka
{किंनु हित्वाप्रियो भवति किंनु हित्वा न शोचति}
{किंनु हित्वाऽर्थवान्भवति किंनु हित्वा सुखी भवेत् ॥युधिष्ठिर उवाच}
{}


\threelineshloka
{मानं हित्वाप्रियो भवति क्रोधं हित्वा न शोचति}
{कामं हित्वाऽर्थवान्भवति लोमं हित्वा सुखी भवेत् ॥यक्ष उवाच}
{}


\threelineshloka
{किमर्थं ब्राह्मणे दानं किमर्थं नटनर्तके}
{किमर्थं चैव भृत्येषु किमर्थं चैव राजसु ॥युधिष्ठिर उवाच}
{}


\threelineshloka
{धर्मार्थं ब्राह्मणे दानं यशोर्थं नटनर्तके}
{भृत्येषु सङ्ग्रहार्थं च भयार्थं चैव राजसु ॥यक्ष उवाच}
{}


\threelineshloka
{अज्ञानेनावृतोलोकस्तमसा न प्रकाशते}
{लोभात्त्यजतिमित्राणि सङ्गात्स्वर्गं न गच्छति ॥युधिष्ठिर उवाच}
{}


\threelineshloka
{अज्ञानेनावृतोलोकस्तमसा न प्रकाशते}
{लोभात्त्यजतिमित्राणि सङ्गात्स्वर्गं न गच्छति ॥यक्ष उवाच}
{}


\threelineshloka
{मृत कथं स्यात्पुरुषः कथं राष्ट्रं मृतं भवत्}
{श्राद्धं मृतंकथं वा स्यात्कथं यज्ञा मृतो भवेत् ॥युधिष्ठिर उवाच}
{}


\threelineshloka
{मृतो दरिद्रः पुरुषोमृतंराष्ट्रमराजकम्}
{मृतमश्रोत्रियं श्राद्धं मृतो यज्ञस्त्वदक्षिणः ॥यक्ष उवाच}
{}


\threelineshloka
{का दिक्किमुदकंपार्थ किमन्नं किंच वै विषम्}
{श्राद्धस् कालमाख्याहि ततः पिब हरस्व च ॥युधिष्ठिर उवाच}
{}


\threelineshloka
{सन्तो दिग्जलमाकाशं गौरन्नं ब्राह्मणं विषम्}
{श्राद्धस्य ब्राह्मणः कालः कथं वा यक्ष मन्यसे ॥यक्ष उवाच}
{}


\threelineshloka
{तपः किंलक्षणं प्रोक्तं को दमश्च प्रकीर्तितः}
{क्षमा च का परा प्रोक्ता का च ह्रीः परिकीर्तिता ॥युधिष्ठिर उवाच}
{}


\threelineshloka
{तपः स्वधर्मवर्तित्वं मनसो दमनं दमः}
{क्षमा द्वन्द्वसहिष्णुत्वंहीरकार्यनिवर्तनम् ॥यक्ष उवाच}
{}


\threelineshloka
{किं ज्ञानं प्रोच्यते राजन्कः शमश्च प्रकीर्तितः}
{दया च का परा प्रोक्ता किं चार्जवमुदाहृतम् ॥युधिष्ठिर उवाच}
{}


\threelineshloka
{ज्ञानं तत्त्वार्थसम्बोधः शमश्चित्तप्रशान्तता}
{दयासर्वसुखैपित्वमार्जवं समचित्तता ॥यक्ष उवाच}
{}


\threelineshloka
{कः शत्रुर्दुर्जयः पुंसां कश्चव्याधिरनन्तकः}
{कीदृशश्च स्मृतः साधुरसाधुः कीदृशः स्मृतः ॥युधिष्ठिर उवाच}
{}


\threelineshloka
{क्रोधः सुदुर्जयः शत्रुर्लोभोव्याधिरनन्तकः}
{सर्वभूतहितः साधुरसाधुर्निर्दयः स्मृतः ॥यक्ष उवाच}
{}


\threelineshloka
{को मोहः प्रोच्यते राजन्कश् मानः प्रकीर्तितः}
{किमालस्यं च विज्ञेयं कश्चशोकः प्रकीर्तितः ॥युधिष्ठिर उवाच}
{}


\threelineshloka
{मोहो हिधर्ममूढ्तवंमानस्त्वात्माभिमानिता}
{धर्मनिष्क्रियताऽऽलस्यं शोकस्त्वज्ञानमुच्यते ॥यक्ष उवाच}
{}


\threelineshloka
{किं स्थैर्यमृषिभिः प्रोक्तं किं च धैर्यमुदाहृतम्}
{स्नानं च किं परं प्रोक्तं दानं च किमिहोच्यते ॥युधिष्ठि उवाच}
{}


\threelineshloka
{स्वधर्मे स्थिरता स्थैर्यं धैर्यमिन्द्रियनिग्रहः}
{स्नानं मनोमलत्यागो दानं वै भूतरक्षणम् ॥यक्ष उवाच}
{}


\threelineshloka
{कः पण्डिः पुमान्ज्ञेयो नास्तिकः कश्च उच्यते}
{को मूर्खः कश्चकामः स्यात्को मत्सर इति स्मृतः ॥युधिष्ठिर उवाच}
{}


\threelineshloka
{धर्मज्ञः पण्डितो ज्ञेयो नास्तिको मूर्ख उच्यते}
{कामः संसारहेतुश्च हृत्तापो मत्सरः स्मृतः ॥यक्ष उवाच}
{}


\threelineshloka
{कोऽहंकार यइतिप्रोक्तः कश्च दम्भः प्रकीर्तितः}
{किं तद्दैवं परं प्रोक्तं किं तत्पैशुन्यमुच्यते ॥युधिष्ठिर उवाच}
{}


\threelineshloka
{महाऽज्ञानमहंकारो दम्भो धर्मो ध्वजोच्छ्रयः}
{दैवं रदानफलं प्रोक्तं पैशुन्यं परदूषणम् ॥यक्ष उवाच}
{}


\threelineshloka
{धर्मश्चार्थश्च कामश्च परस्परविरोधिनः}
{एषां नित्यविरुद्धानां कथमेकत्र संगमः ॥युधिष्ठिर उवाच}
{}


\threelineshloka
{यदा धर्मश्भार्या च परस्परवशानुगौ}
{तदा धर्मार्थकामानां त्रयाणामपि संगमः ॥यक्ष उवाच}
{}


\threelineshloka
{अक्षयोनरकः केन प्राप्यते भरतर्षभ}
{एतन्मे पृच्छतः प्रश्नं तच्छीघ्रं वक्तुमर्हसि ॥युधिष्ठिर उवाच}
{}


\twolineshloka
{रब्राह्मणं स्वयमाहूय याचमानमकिंचनम्}
{पश्चान्नास्तीति योब्रूयात्सोक्षयंनरकं व्रजेत्}


\twolineshloka
{वेदेषु धर्मशास्त्रेषु मिथ्या यो वै द्विजातिषु}
{देवेषु पितृध्रमेषु सोऽक्षयंनरकं व्रजेत्}


\threelineshloka
{विद्यमाने धने लोभाद्दानभोगविवर्जितः}
{पश्चान्नास्तीति यो ब्रूयात्सोक्षयं नरकं व्रजेत् ॥यक्ष उवाच}
{}


\threelineshloka
{राजन्कुलेन वृत्तेन स्वाध्यायेन श्रुतेन वा}
{ब्राह्मण्यं केन भवति प्रब्रूह्येतत्सुनिश्चितम् ॥युधिष्ठिर उवाच}
{}


\twolineshloka
{शृणु यक्ष कुलं तात न स्वाध्यायो न च श्रुतम्}
{कारणं हि द्विजत्वेच वृत्तमेव न संशयः}


\twolineshloka
{वृत्तं यत्नेन संरक्ष्यं ब्राह्मणेन विशेषतः}
{अक्षीणवृत्तो न क्षीणो वृत्ततस्तु हतो हतः}


\twolineshloka
{पठकाः पाठकाश्चैव ये चान्ये शास्त्रचिन्तकाः}
{सर्वे व्यसनिनो मूर्खा यः क्रियावान्स पण्डितः}


\threelineshloka
{चतुर्वेदोऽपि दुर्वृत्तः स शूद्रादतिरिच्यते}
{योऽग्निहोत्रपोर दान्तः स ब्राह्मण इति स्मृतः ॥यक्ष उवाच}
{}


\threelineshloka
{प्रियवचनवादी किं लभतेविमृशितकार्यकरः किं लभते}
{बहुमित्रकरः किं लभतेधर्मे रतः किं लभते कथय ॥युधिष्ठिर उवाच}
{}


\threelineshloka
{प्रियवचनवादी प्रीयो भवतिविमृशितकार्यकरोऽधिकं जयति}
{बहुमित्रकरः सुखं वसतयश्च धर्मरतः स गतिं लभते ॥यक्ष उवाच}
{}


\twolineshloka
{कोमोदतेकिमाश्चर्यं कः पन्थाः का च वार्तिका}
{वद मे चतुरः प्रश्नान्मृता जीवन्तु बान्धवाः ॥युधिष्ठिर उवाच}


\twolineshloka
{पञ्चमेऽहनि षष्ठे वा शाकं पचति स्वे गृहे}
{अनृणी रचाप्रवासी चस वारिचर मोदते}


\twolineshloka
{अहन्यहनि भूतानि गच्छन्तीह यमालयम्}
{शेषाः स्थावरमिच्छन्ति किमाश्चर्यमतः परम्}


\twolineshloka
{तर्कोऽप्रतिष्ठः श्रुतयो विभिन्नानैको मुनिर्यस्य मतं प्रमाणम्}
{धर्मस्य तत्त्वं निहितं गुहायांमहाजनो येन गतःस पन्था}


\threelineshloka
{पृथ्वी विभाण्डं गगनं पिघानंसूर्याग्निना रात्रिदिवेन्धनेन}
{मासर्तुदर्वीपरिघट्टनेनभूतानि कालः पचतीति वार्ता ॥यक्ष उवाच}
{}


\threelineshloka
{व्याख्याता मे त्वया प्रश्ना यथातत्वं परंतप}
{पुरुषं त्विदानींव्याख्याहि यश्च सर्वधनी नरः ॥युधिष्टिर उवाच}
{}


\twolineshloka
{दिवं स्पृशति भूमिं च शब्दः पुण्येन कर्मणा}
{यावत्स शब्दो भवति तावत्पुरुष उच्यते}


\twolineshloka
{तुल्ये प्रियाप्रिये यस् सुखदुःखे तथैव च}
{अतीतानागते चोभे सवै पुरुष उच्येत}


\threelineshloka
{`समत्वं यस्य सर्वेषु निस्पृहः शान्तमानसः}
{सुप्रसन्नः सदा योगी स वै सर्वधनी नरः' ॥यक्ष उवाच}
{}


\threelineshloka
{व्याख्यातः पुरुषो राजन्यश्च सर्वधनी नरः}
{तस्मात्त्वमेकं भ्रातृणां यमिच्छसि स जीवतु ॥युधिष्ठिर उवाच}
{}


\threelineshloka
{श्यामो य एष रक्ताक्षो बृहत्साल इवोत्थितः}
{व्यूढोरस्को महाबाहुर्नकुलो यक्ष जीवतु ॥यक्ष उवाच}
{}


\twolineshloka
{प्रियस्ते भीमसेनोऽयमर्जुनो वः परायणम्}
{त्वं कस्मान्नकुलं राजन्सापत्नं जीवमिच्छसि}


\twolineshloka
{यस् नागसहस्रेण दशसङ्ख्येन वै बलम्}
{तुल्यंतं भीममुत्सृज्य नकुलं जीवमिच्छसि}


\twolineshloka
{तथैनं मनुजाः प्राहुर्भीमसेनं प्रियं तव}
{अथ कनानुभावेन सापत्नं जीवमिच्छसि}


\threelineshloka
{यस्य बाहुबलंसर्वेपाण्डवाः समुपासते}
{अर्जुनं तमपाहाय नकुलं जीवमिच्छसि ॥युधिष्ठिर उवाच}
{}


\twolineshloka
{धर्म एव हतो हन्ति ध्रमो रक्षति रक्षितः}
{तस्माद्धऱ्मं न त्यजामि मा नो धर्मो हतोऽवधीत्}


\twolineshloka
{आनृशंस्यं परो धर्मः परमार्थाच्चमे मतम्}
{आनृशंस्यं चिकीर्षामि नकुलो यक्ष जीवतु}


\twolineshloka
{धर्मशीलः सदा राजाइतिमां मानवा विदुः}
{स्वधर्मान्न चलिष्यामि नकुलो यक्ष जीवतु}


\twolineshloka
{कुन्ती चैव तु माद्री च द्वे भार्ये तु पितुर्मम}
{उभे सपुत्रे स्यातां वै इतिमे धीयते मतिः}


\threelineshloka
{यथा कुन्ती तथा माद्री विशेषो नास्ति मे तयोः}
{मातृभ्यां सममिच्छामि नकुलो यक्ष जीवतु ॥यक्ष उवाच}
{}


\twolineshloka
{यस् तेऽर्थाच्च कामच्च आनृशंस्यं परं मतम्}
{तस्मात्ते भ्रातरः सर्वे जीवन्तु भरतर्षभ}


\chapter{अध्यायः ३१५}
\twolineshloka
{वैशंपायन उवाच}
{}


\threelineshloka
{ततस्ते यक्षवचनादुदतिष्ठन्त पाण्डवः}
{क्षुत्पिपासे च सर्वेषां क्षणेन व्यपगच्छताम् ॥युधिष्ठि उवाच}
{}


\twolineshloka
{सरस्येकेन पादेन तिष्ठन्तमपराजितम्}
{पृच्छामि को भवान्देवो न मे यक्षो मतो भवान्}


\twolineshloka
{बसूनां वा भवानेको रुद्राणामथवा भवान्}
{अथवा मरुतां श्रेष्ठो वज्री वा त्रिदशेश्वरः}


\twolineshloka
{मम हि भ्रातर इमे सहस्रशतयोधिनः}
{तं योधं न प्रपश्यामि येन सर्वे निपातिताः}


\threelineshloka
{सुखं प्रति प्रबुद्दानामिन्द्रियाण्युपलक्षये}
{स भवान्सुहृदोस्माकमथवा नः पिता भवान् ॥यक्ष उवाच}
{}


\twolineshloka
{अहंते जनकस्तात धर्मो मृदुपराक्रम}
{त्वां रदिदृक्षुरनुप्राप्तो विद्धि मां भरतर्षभ}


\twolineshloka
{यशः सत्यंदमः शौचमार्जवं ह्रीरचापलम्}
{दानं तपो ब्रह्मचर्यमित्येतास्तनवो मम}


\twolineshloka
{अहिंसा समता शान्तिस्तपः शौचममत्सरः}
{द्वाराण्येतानि मे विद्धि प्रियो ह्यसि सुतो मम}


\twolineshloka
{दिष्ट्या पञ्चसु रक्तोसि दिष्ट्या ते षट्रपदी जिता}
{द्वे पूर्वे मध्यमे द्वे च द्वे शान्ते सांपरायिके}


\twolineshloka
{ध्रमोऽहमिति भद्रं ते जिज्ञासुस्त्वामिहागतः}
{आनृशंस्येन तुष्टोस्मि वरं दास्यामि तेऽनघ}


\threelineshloka
{वरं वृणीष्व राजेन्द्रदाता ह्यस्मि तवानघ}
{ये हि मे पुरुषा भक्ता न तेषामस्ति दुर्गतिः ॥युधिष्ठिर उवाच}
{}


\threelineshloka
{अरणी तु हृतायस्य मृगेण वदतांवर}
{तस्याग्नयो न लुप्येरन्प्रथमोऽस्तु वरो मम ॥धर्म उवाच}
{}


\threelineshloka
{आरणेयमिदं तस्य ब्राह्मणस्य हृतं मया}
{मृगवेषेण कौन्तेय जिज्ञालार्थं तवानघ ॥वैशंपायन उवाच}
{}


\threelineshloka
{ददानीत्येव भगवानुत्तरं प्रत्यपद्यत}
{अन्यं वरय भद्रं ते वरं त्वममरोपम ॥युधिष्टिर उवाच}
{}


\twolineshloka
{वर्षाणि द्वादशारण्ये त्रयोदशमुपस्थितम्}
{तत्रनो नाभिजानीयुर्वसतो मनुजाः क्वचित्}


\twolineshloka
{ददानीत्येव भगवानुत्तरं प्रत्यपद्यत}
{भूयश्चाश्वासयामास कौन्तेयं सत्यविक्रमम्}


\twolineshloka
{यद्यपि स्वेन रूपेण चरिष्यथ महीमिमाम्}
{न वो विज्ञास्यते कश्चित्रिषु लोकेषु भारत}


\twolineshloka
{वर्षंत्रयोदशमिदं मत्प्रसादात्कुरूद्वहाः}
{विराटनगरे गूढा अविज्ञाताश्चरिष्यथ}


\twolineshloka
{यद्वः संकल्पितं रूपं मनसा यस् यादृशम्}
{तादृशं तादृशं सर्वे छन्दतो धारयिष्यथ}


\twolineshloka
{अरणीसहितं भाण्डं ब्राह्मणाय प्रयच्छत}
{जिज्ञासार्थं मया ह्येतदाहृतंमृगरूपिणा}


\twolineshloka
{प्रवृणीष्वापरं सौम्य वरमिष्टं ददानि ते}
{न तृप्यामि नरश्रेष्ठ प्रयच्छन्वै वरांस्तथा}


\threelineshloka
{तृतीयं गृह्यतां पुत्र वरमप्रतिमं महत्}
{त्वं हि मत्प्रभवो राजन्विदुरश्चममांशजः ॥जुधिष्ठिर उवाच}
{}


\twolineshloka
{देवदेवो मया दृष्टो भवान्साक्षात्सनातनः}
{यं ददासि वरं तुष्टस्तं ग्रहीष्याम्यहं पितः}


\threelineshloka
{जयेयं लोभमोहौ च क्रोधं चाहं सदा विभो}
{दाने तपसि सत्ये च मनो मे सततं वेत् ॥धर्म उवाच}
{}


\threelineshloka
{रउपपन्नो गुणैरेतैः स्वभावेनासि पाण्डव}
{भवान्धर्मः पुनश्चैव यथोक्तं ते भविष्यति ॥वैशंवायन उवाच}
{}


\twolineshloka
{इत्युक्त्वान्तऽर्दधे धर्मो भगबाँल्लोकभावनः}
{समेताः पाण्डवाश्चैव सुखसुप्ता मनस्विनः}


\twolineshloka
{उपेत्यचाश्रमं वीराः सर्व एव गतक्लमाः}
{आरणेयं ददुस्तस्मै ब्राह्मणाय तपस्विने}


\twolineshloka
{इदं समुत्थानसमागतं मह-त्पितुश्चपुत्रस्य च कीर्तिवर्धनम्}
{पठन्नरः रस्याद्विजितेन्द्रियो वशीसपुत्रपौत्रः शतपर्षभाग्भवेत्}


\twolineshloka
{न चाप्यधर्मे न सुहृद्विभेदनेपरस्वहारे परदारमर्शने}
{कदर्यभावे न रमेनेमनः सदानृणां सदाख्यानमिदं विजानताम्}


