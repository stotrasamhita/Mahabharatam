\part{उद्यॊगपर्व}
\chapter{अध्यायः १}
\threelineshloka
{श्रीवेदव्यासाय नमः}
{नारायणं नमस्कृत्य नरं चैव नरोत्तमम्}
{देवीं सरस्वतीं चैव ततो जयमुदीरयेत्}


\threelineshloka
{वैशंपायन उवाच}
{कृत्वा विवाहं तु कुरुप्रवीरा-स्तदाऽभिमन्योर्मुदिताः सपक्षाः}
{विश्रम्य रात्रावुषसि प्रतीताःसभां विराटस्य ततोऽभिजग्मुः}


\twolineshloka
{सभा तु सा मत्स्यपतेः समृद्धामणिप्रवेकोत्तमरत्नचित्रा}
{न्यस्तासना माल्यवती सुगन्धातामभ्ययुस्ते नरराजवर्याः}


\twolineshloka
{अथासनान्याविशतां पुरस्ता-दुभौ विराटद्रुपदौ नरेन्द्रौ}
{वृद्धौ च मान्यौ पृथिवीपतीनांपित्रा समं तमजनार्दनौ च}


% Check verse!
पाञ्चालराजस्य समीपतस्तुशिनिप्रवीरः सहरौहिणेयःमत्स्यस्य राज्ञस्तु सुसन्निकृष्टौजनार्दनश्चैव युधिष्ठिरश्च
\twolineshloka
{सुताश्च सर्वे द्रुपदस्य राज्ञोभीमार्जुनौ माद्रवतीसुतौ च}
{प्रद्युम्नसाम्बौ च युधि प्रवीरौविराटपुत्रैश्च सहाभिमन्युः}


\twolineshloka
{सर्वे च शूराः पितृभिः समानावीर्येण रूपेण बलेन चैव}
{उपाविशन्द्रौपदेयाः कुमाराःसुवर्णचित्रेषु वरासनेषु}


% Check verse!
तथोपविष्टेषु महारथेषुविराजमानाभरणाम्बरेषुरराज सा राजवती समृद्धाग्रहैरिव द्यौर्विमलैरुपेता
\twolineshloka
{ततः कथास्ते समवाययुक्ताःकृत्वा विचित्राः पुरुषप्रवीराः}
{तस्थुर्मुहूर्तं परिचिन्तयन्तःकृष्णं नृपास्ते समुदीक्षमाणाः}


\threelineshloka
{कथान्तमासाद्य च माधवेनसंघट्टिताः पाण्डवकार्यहेतोः}
{ते राजसिंहाः सहिता ह्यश्रृण्व-न्वाक्यं महार्थं सुमहोदयं च ॥श्रीकृष्ण उवाच}
{}


\twolineshloka
{सर्वैर्भवद्भिर्विदितं यथाऽयंयुधिष्ठिरः सौबलेनाक्षवत्याम्}
{जितो निकृत्याऽपहृतं च राज्यंवनप्रवासे समयः कृतश्च}


\twolineshloka
{शक्तैर्विजेतुं तरसा महीं चसत्ये स्थितैः सत्यरथैर्यथावत्}
{पाण्डोः सुतैस्त्रद्रुतमुग्ररूपंवर्षाणि षट् सप्त च चीर्णमग्र्यैः}


\threelineshloka
{त्रयोदशश्चैव सुदुस्तरोऽय-मज्ञायमानैर्भवतां समीपे}
{क्लेशानसह्यान्विविधान्सहद्भि-र्महात्मभिश्चापि वने निविष्टम्}
{एतैः परप्रेष्यनियोगयुक्तै-रिच्छद्भिराप्तं स्वकुलेन राज्यम्}


\twolineshloka
{एवं गते धर्मसुतस्य राज्ञोदुर्योधनस्यापि च यद्धितं स्यात्}
{तच्चिन्तयध्वं कुरुपाण्डवानांधर्म्यं च युक्तं च यशस्करं च}


\twolineshloka
{अधर्मयुक्तं न च कामयेतराज्यं सुराणामपि धर्मराजः}
{धर्मार्थयुक्तं तु महीपतित्वंग्रोमेऽपि कस्मिंश्चिदयं बुभूषेत्}


\twolineshloka
{पित्र्यं हि राज्यं विदितं नृपाणांयथाऽपकृष्टं धृतराष्ट्रपुत्रैः}
{मिथ्योपचारेण यथा ह्यनेनकृच्छ्रं महत्प्राप्तमसह्यरूपम्}


\twolineshloka
{न चापि पार्थो विजितो रणे तैःस्वतेजसा धृतराष्ट्रस्य पुत्रैः}
{तथाऽपि राजा सहितः सुहृद्भि-रभीप्सतेऽनामयमेव तेषाम्}


% Check verse!
यत्तु स्वयं पाण्डुसुतैर्विजित्यसमाहृतं भूमिपतीन्प्रपीड्यतत्प्रार्थयन्ते पुरुषप्रवीराःकुन्तीसुता माद्रवतीसुतौ च
\twolineshloka
{बलाभियुक्तैर्विविधैरुपायैःसंप्रार्थिता हन्तुममित्रसङ्घैः}
{राज्यं जिहीर्षद्भिरसद्भिरुग्रैःसर्वं च तद्वो विदितं यथावत्}


\threelineshloka
{तेषां च लोभं प्रसमीक्ष्य वृद्धंधर्मज्ञतां चापि युधिष्ठिरस्य}
{संबन्धितां चापि समीक्ष्य तेषांमतिं कुरुध्वं सहिताः पृथक्व}
{}


\twolineshloka
{इमे च सत्येऽभिरताः सदैवतं पालयित्वा समयं यथावत्}
{अतोऽन्यथा तैरुपचर्यमाणाहन्युः समेतान्धृतराष्ट्रपुत्रान्}


\twolineshloka
{तैर्विप्रकारं च निशम्य कार्येसुहृज्जनास्तान्परिवारयेयुः}
{युद्धेन बाधेयुरिमांस्तथैवंतैर्बाध्यमाना युधि तांश्च हन्युः}


\twolineshloka
{तथाऽपि नेमेऽल्पतयाऽसमर्था-स्तेषां जयायेति भवेन्मतिर्वः}
{समेत्य सर्वे सहिताः सुहृद्भि-स्तेषां विनाशाय यतेयुरेव}


% Check verse!
दुर्योधनस्यापि मतं यथाव-न्न ज्ञायते किं नु करिष्यतीतिअज्ञायमाने च मते परस्यमन्त्रस्य पारं कथमभ्युपेमः
\twolineshloka
{तस्मादितो गच्छतु धर्मशीलःशुचिः कुलीनः पुरुषोऽप्रमत्तः}
{दूतः समर्थः प्रशमाय तेषांराज्यार्धदानाय युधिष्ठिरस्य}


\twolineshloka
{निशम्य वाक्यं तु जनार्दनस्यधर्मार्थयुक्तं मधुरं समं च}
{समाददे वाक्यमथाग्रजोऽस्यसंपूज्य वाक्यं तदतीव राजन्}


\chapter{अध्यायः २}
\twolineshloka
{बलदेव उवाच}
{}


\twolineshloka
{श्रुतं भवद्भिर्गदपूर्वजस्यवाक्यं यथा धर्मवदर्थवच्च}
{अजातशत्रोश्च हितं च युक्तंदुर्योधनस्यापि तथैव राज्ञः}


\twolineshloka
{अर्धं हि राज्यस्य विसृज्य वीराःकुन्तीसुतास्तस्य कृते यतन्ते}
{प्रदाय चार्धं धृतराष्ट्रपुत्रःसुखी सहास्माभिरतीव मोदेत्}


\twolineshloka
{लब्ध्वा हि राज्यं पूरुषप्रवीराःसम्यक्प्रवृत्तेषु परेषु चैव}
{ध्रुवं प्रशान्ताः सुखमाविशेयु-स्तेषां प्रशान्तिश्च हितं प्रजानाम्}


\twolineshloka
{दुर्योधनस्यापि मतं च वेत्तुंवक्तुं च वाक्यानि युधिष्ठिरस्य}
{प्रियं च मे स्याद्यदि तत्र कश्चि-द्व्रजेच्छमार्थं कुरुपाण्डवानाम्}


\twolineshloka
{स भीष्ममामन्त्र्य कुरुप्रवीरंवैचित्रवीर्यं च महानुभावम्}
{द्रोणं सपुत्रं विदुरं कृपं चगान्धारराजं च समूतपुत्रम्}


\twolineshloka
{सवे च येऽन्ये धृतराष्ट्रपुत्राबलप्रधाना निगमप्रधानाः}
{स्थिताश्च धर्मेषु यथा स्वकेषुलोकप्रवीराः श्रुतशीलवृद्धाः}


\twolineshloka
{एतेषु सर्वेषु समागतेषुपौरेषु वृद्धेषु च संगतेषु}
{ब्रवीतु वाक्यं प्रणिपातयुक्तंकुन्तीसुतस्यार्थकरं यथा स्यात्}


\twolineshloka
{सर्वास्ववस्थासु च ते न कोप्याग्रस्तो हि सोऽर्थो बलमाश्रितैस्तैः}
{प्रियाभ्युपेतस्य युधिष्ठिरस्यद्यूते प्रसक्तस्य हृतं च राज्यम्}


\twolineshloka
{निवार्यमाणश्च कुरुप्रवीरःसर्वैः सुहृद्भिर्ह्यमप्यतञ्झः}
{स दीव्यमानः प्रतिदीव्य चैनंगान्धारराजस्य सुतं मताक्षम्}


\twolineshloka
{हित्वा हि कर्णं च सुयोधनं चसमाह्वयद्देवितुमाजमीढः}
{दुरोदरास्तत्र सहस्रशोऽन्येयुधिष्ठिरो यान्विषहेत जेतुम्}


\twolineshloka
{उत्सृज्य तान्सौबलमेव चायंसमाह्वयत्तेन जितोऽक्षवत्याम्}
{स दीव्यमानः प्रतिदेवनेनअक्षेषु नित्यं स्वपराङ्भुखेषु}


\twolineshloka
{संरम्भमाणो विजितः प्रसह्यतत्रापराधः शकुनेर्न कश्चित्}
{तस्मात्प्रणम्यैव वचो ब्रवीतुवैचित्रवीर्यं बहुसामयुक्तम्}


\twolineshloka
{तथा हि शक्यो धृतराष्ट्रपुत्रःस्वार्थें नियोक्तुं पुरुषेण तेन}
{अयुद्धमाकाङ्क्षत कौरवाणांसाम्नैव दुर्योधनमाश्वसध्वम्}


% Check verse!
साम्ना जितोऽर्थोऽर्थकरो भवेतयुद्धेऽनयो भविता नेह सोऽर्थः
% Check verse!
एवं ध्रुवत्येव मधुप्रवीरेशिनिप्रवीरः सहसोत्पपाततच्चापि वाक्यं परिनिन्द्य तस्यसमाददे वाक्यमिदं समुन्युः
% Check verse!

\chapter{अध्यायः ३}
\twolineshloka
{सात्यकिरुवाच}
{}


\twolineshloka
{यादृशः पुरुषस्यात्मा तादृशं संप्रभाषते}
{यथारूपोऽन्तरात्मा ते तथारूपं प्रभाषते}


\twolineshloka
{सन्ति वै पुरुषाः शूराः सन्ति कापुरुषास्तथा}
{उभावेतौ दृढौ पक्षौ दृश्येते पुरुषान्प्रति}


\twolineshloka
{एकस्मिन्नेव जायेते कुले क्लीबमहाबलौ}
{फलाफलवती शाखे यथैकस्मिन्वनस्पतौ}


\twolineshloka
{नाभ्यसूयामि ते वाक्यं ब्रुवतो लाङ्गलध्वज}
{ये तु शृण्वन्ति ते वाक्यं तानसूयामि माधव}


\twolineshloka
{कथं हि धर्मराजस्य दोषमल्पमपि ब्रुवन्}
{लभते परिषन्मध्ये व्याहर्तुमकुतोभयः}


\threelineshloka
{समाहूय महात्मानं जितवन्तोऽक्षकोविदाः}
{अनक्षज्ञं यथाऽश्रद्धं तेषु धर्मजयः कुतः}
{}


\twolineshloka
{यदि कुन्तीसुतं गेहे क्रीडन्तं भ्रातृभिः सह}
{अभिगम्य जयेयुस्ते तत्तेषां धर्मतो भवेत्}


\twolineshloka
{समाहूय तु राजानं क्षत्रधर्मरतं सदा}
{निकृत्या जितवन्तस्ते किं नु तेषां परं शुभम्}


\twolineshloka
{कथं प्रणिपतेच्चायमिह कृत्वा पणं परम्}
{वनवासाद्विमुक्तरतु प्राप्तः पैतामहं पदम्}


\twolineshloka
{यद्ययं परवित्तानि कामयेत युधिष्ठिरः}
{एवमप्ययमत्यन्तं परान्नार्हति याचितुम्}


\twolineshloka
{कथं च धर्मयुक्तास्ते न च राज्यं जिहीर्षवः}
{निवृत्तवनवासांस्तान्य आहुर्विदिता इति}


\twolineshloka
{अनुनीता हि भीष्मेण द्रोणेन विदुरेण च}
{न व्यवस्यन्ति पाण्डूनां प्रदातुं पैतृकं वसु}


\twolineshloka
{अहं तु ताञ्छितैर्बाणैरनुनीय रणे बलात्}
{पादयोः पातयिष्यामि कौन्तेयस्य महात्मनः}


\twolineshloka
{अथ ते न व्यवस्यन्ति प्रणिपाताय धीमतः}
{गमिष्यन्ति सहामात्या यमस्य सदनं प्रति}


\twolineshloka
{न हि ते युयुधानस्य संरब्धस्य युयुत्सतः}
{वेगं समर्थाः संसोद्धुं वज्रस्येव महीधराः}


\twolineshloka
{को हि गाण्डीवधन्वानं कश्च चक्रायुधं युधि}
{मां चापि विषहेत्क्रुद्धं कश्च भीमं दुरासदम्}


\twolineshloka
{यमौ च दृढधन्वानौ यमकल्पौ महाद्युती}
{विराटद्रुपदौ वीरौ यमकालोपमद्युती}


\twolineshloka
{को जिजीविषुरासादेद्धृष्टद्युम्नं च पार्षतम्}
{पञ्चैतान्पाण्डवेयांस्तु द्रौपद्याः कीर्तिवर्धनान्}


\twolineshloka
{समप्रमाणान्पाण्डूनां समवीर्यान्मदोत्कटान्}
{सौभद्रं च महेष्वासममरैरपि दुःसहम्}


\twolineshloka
{गदप्रद्युम्नसाम्बांश्च कालसूर्यानलोपमान्}
{ते वयं धृतराष्ट्रस्य पुत्रं शकुनिना सह}


\twolineshloka
{कर्णं चैव निहत्याजावभिषेक्ष्याम पाण्डवम्}
{नाधर्मो विद्यते कश्चिच्छत्रून्हत्वाऽऽततायिनः}


\twolineshloka
{अधर्म्यमयशस्यं च शात्रवाणां प्रयाचनम्}
{हृद्भतस्तस्य यः कामस्तं कुरुध्वमतन्द्रिताः}


\threelineshloka
{धार्तराष्ट्रो ह्ययुद्धेन न राज्यं दातुमिच्छति}
{अद्य पाण्डुसुतो राज्यं लभतां वा युधिष्ठिरः}
{निहता वा रणे सर्वे स्वप्स्यन्ति वसुधातले}


\chapter{अध्यायः ४}
\twolineshloka
{द्रुपद उवाच}
{}


\twolineshloka
{एवमेतन्महाबाहो भविष्यति न संशयः}
{न हि दुर्योधनो राज्यं मधुरेण प्रदास्यति}


\twolineshloka
{अनुवर्त्स्यति तं चापि धृतराष्ट्रः सुतप्रियः}
{भीष्मद्रोणौ च कार्पण्यान्मौर्ख्याद्राधेयसौबलौ}


\twolineshloka
{बलदेवस्य वाक्यं तु मम ज्ञाने न युज्यते}
{एतद्धि पुरुषेणाग्रे कार्यं सुनयमिच्छता}


\twolineshloka
{न तु वाच्यो मृदुवचो धार्तराष्ट्रः कथंचन}
{नहि मार्दवसाध्योऽसौ पापबुद्धिर्मतो मम}


\twolineshloka
{गर्दभे मार्दवं कुर्याद्गोषु तीक्ष्णं समाचरेत्}
{मृदु दुर्योधने वाक्यं यो ब्रूयात्पापचेतसि}


\twolineshloka
{मृदुं वै मन्यते पापो भाषमाणमशक्तिकम्}
{हितमर्थं न जानीयादबुद्धिर्मार्दवे सति}


\twolineshloka
{एतच्चैव करिष्यामो यत्नश्च क्रियतामिह}
{प्रस्थापयाम मित्रेभ्यो बलान्युद्योजयन्तु नः}


\twolineshloka
{शल्यस्य धृष्टकेतोश्च जयत्सेनस्य वा विभो}
{केकयानां च सर्वेषां दूता गच्छन्तु शीघ्रगाः}


\twolineshloka
{स च दुर्योधनो नूनं प्रेषयिष्यति सर्वशः}
{पूर्वाभिपन्नाः सन्तश्च भजन्ते पूर्वचोदनम्}


\twolineshloka
{तत्वरध्वं नरेन्द्राणां पूर्वमेव प्रचोदने}
{महद्धि कार्यं वोढव्यमिति मे वर्तते मतिः}


\twolineshloka
{शल्यस्य प्रेष्यतां शीघ्रं ये च तस्यानुगा नृपाः}
{भगदत्ताय राज्ञे च पूर्वसागरवासिने}


\twolineshloka
{अमितौजसे तथोग्राय हार्दिक्यायान्धकाय च}
{दीर्घप्रज्ञाय शूराय रोचमानाय वा विभो}


\twolineshloka
{आनीयतां बृहन्तश्च सेनाबिन्दुश्च पार्थिवः}
{सेनजित्प्रतिबिन्ध्यश्च चित्रवर्मा सुवास्तुकः}


\twolineshloka
{बाह्लीको मुञ्जकेशश्च चैद्याधिपतिरेव च}
{सुपार्श्वश्च सुबाहुश्च पौरवश्च महारथः}


% Check verse!
शकानां पह्लवानां कर्णवेष्टश्च पार्थिवः ॥सुरारिश्च नदीजश्च कर्णवेष्टश्च पार्थिवः
\twolineshloka
{नीलश्च वीरधर्मा च भूमिपालश्च वीर्यवान्}
{दुर्जयो दन्तवक्रश्च रुक्मी च जनमेजयः}


\twolineshloka
{आषाढो वायुवेगश्च पूर्वपाली च पार्थिवः}
{भूरितेजा देवकश्च एकलव्यः सहात्मजैः}


\twolineshloka
{कारूषकाश्च राजानः क्षेमधूर्तिश्च वीर्यवान्}
{काम्भोजा ऋषिका ये च पश्चिमानूपकाश्च ये}


% Check verse!
जयत्सेनश्च काश्यश्च तथा पञ्चनदा नृपाः ॥जानकिश्च दुर्धर्षः पार्वतीयाश्च ये नृपाः
\twolineshloka
{जानकिश्च सुशर्मा च मणिमान्योतिमत्सकः}
{पांशुराष्ट्राधिपश्चैव धृष्टकेतुश्च वीर्यवान्}


\twolineshloka
{तुण्डश्च दण्डधारश्च बृहत्सेनश्च वीर्यवान्}
{अपराजितो निषादश्च श्रेणिमान्वसुमानपि}


\twolineshloka
{बृहद्बलो महौजाश्च बाहुः परपुरंजयः}
{समुद्रसेनो राजा च सह पुत्रेण वीर्यवान्}


\twolineshloka
{उद्भवः क्षेमकश्चैव वाटधानश्च पार्थिवः}
{श्रुतायुश्च दृढायुश्च साल्वपुत्रश्च वीर्यवान्}


\twolineshloka
{कुमारश्च कलिङ्गानामीश्वरो युद्धदुर्मदः}
{एतेषां प्रेष्यतां शीघ्रमेतद्धि मम रोचते}


\twolineshloka
{अयं च ब्राह्मणो विद्वान्मम राजन्पुरोहितः}
{प्रेष्यतां धृतराष्ट्राय वाक्यमस्मै प्रदीयताम्}


\twolineshloka
{यथा दुर्योधनो वाच्यो यथा सान्तनवो नृपः}
{धृतराष्ट्रो यथा वाच्यो द्रोणश्च रथिनां वरः}


\chapter{अध्यायः ५}
\twolineshloka
{वैशंपायन उवाच}
{}


\twolineshloka
{द्रुपदेनैवमुक्ते तु वाक्ये वाक्यविदां वरः}
{वसुदेवसुतस्तत्र पृष्णिसिंहोऽब्रवीदिदम्}


\twolineshloka
{उपपन्नमिदं वाक्यं सोमकानां धुरंधरे}
{अर्थसिद्धिकरं राज्ञः पाण्डवस्यामितौजसः}


\twolineshloka
{एतच्च पूर्वं कार्यं नः सुनीतिमभिकाङ्क्षताम्}
{अन्यथा ह्याचरन्कर्म पुरुषः स्यात्सुबालिशः}


\twolineshloka
{किं तु संबन्धकं तुल्यमस्माकं कुरुपाण्डुषु}
{यथेष्टं वर्तमानेषु पाण्डवेषु च तेषु च}


\twolineshloka
{ते विवाहार्थमानीता वयं सर्वे तथा भवान्}
{कृते विवाहे मुदिता गमिष्यामो गृहान्प्रति}


\twolineshloka
{भवान्वृद्धतमो राज्ञां वयसा च श्रुतेन च}
{शिष्यवत्ते वयं सर्वे भवामेह न संशयः}


\twolineshloka
{भवन्तं धृतराष्ट्रश्च सततं बहु मन्यते}
{आचार्ययोः सखा चासि द्रोणस्य च कृपस्व च}


\twolineshloka
{स भवान्प्रेषयत्वद्य पाण्डवार्थकरं वचः}
{सर्वेषां निश्चितं तन्नः प्रेषयिष्यति यद्भवान्}


\twolineshloka
{यदि तावच्छमं कुर्यन्न्यायेन कुरुपुङ्गवः}
{न भवेत्कुरुपाण्डूनां सौभ्रात्रेण महान्क्षयः}


\twolineshloka
{अथ दर्पान्वितो मोहान्न कुर्याद्धृतराष्ट्रजः}
{अन्येषां प्रेषयित्वा च पश्चादस्मान्सभाह्वये}


\threelineshloka
{ततो दुर्योधनो मन्दः सहामात्यः सबान्धवः}
{निष्ठामापत्स्यते मूढः क्रुद्धे गाण्डीवधन्वनि ॥वैशंपायन उवाच}
{}


\twolineshloka
{ततः सत्कृत्य वार्ष्णेयं विराटः पृथिवीपतिः}
{गृहान्प्रस्थापयामास सगणं सहबान्धवम्}


\twolineshloka
{द्वारकं तु गते कृष्णे युधिष्ठिरपुरोगमाः}
{चक्रुः साङ्घ्रामिकं सर्वं विराटश्च महीपतिः}


\twolineshloka
{ततः संप्रेषयामास विराटः सह बान्धवैः}
{सर्वेषां भूमिपालानां द्रुपदश्च महीपतिः}


\twolineshloka
{वचनात्कुरुसिंहानां मत्स्यपाञ्चालयोश्च ते}
{समाजग्मुर्महीपालाः संप्रहृष्टा महाबलाः}


\twolineshloka
{तच्छ्रुत्वा पाण्डुपुत्राणां समागच्छन्महद्बलम्}
{धृतराष्ट्रसुताश्चापि समानिन्युर्महीपतीन्}


\twolineshloka
{समाकुला मही राजन्कुरुपाण्डवकारणात्}
{तदा समभवत्कृत्स्ना संप्रयाणे महीक्षिताम्}


\twolineshloka
{संकुला च तदा भूमिश्चतुरङ्गवलान्विता}
{बलानि तेषां वीराणामागच्छन्ति ततस्ततः}


\threelineshloka
{चालयन्तीव गां देवीं सपर्वतवनामिमाम्}
{ततः प्रज्ञावयोवृद्धं पाञ्चाल्यः स्वपुरोहितम्}
{कुरुभ्यः प्रेषयामास युधिष्ठिरमते स्थितः}


\chapter{अध्यायः ६}
\twolineshloka
{द्रुपद उवाच}
{}


\twolineshloka
{भूतानां प्राणिनः श्रेष्ठाः प्राणिनां बुद्धिजीविनः}
{बुद्धिमत्सु नराः श्रेष्ठा नरेष्वपि द्विजातयः}


\twolineshloka
{द्विजेषु वैद्याः श्रेयांसो वैद्येषु कृतबुद्धयः}
{कृतबुद्धिषु कर्तारः कर्तृषु ब्रह्मवादिनः}


\threelineshloka
{स भवान्कृबुद्धीनां प्रधान इति मे मतिः}
{कुलेन च विशिष्टोऽसि वयसा च श्रुतेन च}
{प्रज्ञया सदृशश्चासि शुक्रेणाङ्गिरसेन च}


\twolineshloka
{विदितं चापि ते सर्वं यथावृत्तः स कौरवः}
{पाण्डवश्च यथावृत्तः कुन्तीपुत्रो युधिष्ठिरः}


\twolineshloka
{धृतराष्ट्रस्य विदिते वञ्चिताः पाण्डवाः परैः}
{विदुरेणानुनीतोऽपि पुत्रमेवानुवर्तते}


\twolineshloka
{शकुनिर्बुद्धिपूर्वं हि कुन्तीपुत्रं समाह्वयत्}
{अनक्षज्ञं मताक्षः सन्क्षत्रवृत्ते स्थितं शुचिम्}


\twolineshloka
{ते तथा वञ्चयित्वा तु धर्मराजं युधिष्ठिरम्}
{न कस्यांचिदवस्थायां राज्यं दास्यन्ति वै स्वयम्}


\twolineshloka
{भवांस्तु धर्मसंयुक्तं धृतराष्ट्रं ब्रुवन्वचः}
{मनांसि तस्य योधानां ध्रुवमावर्तयिष्यति}


\twolineshloka
{विदुरश्चापि तद्वाक्यं साधयिष्यति तावकम्}
{भीष्मद्रोणकृपादीनां भेदं संजनयिष्यति}


\twolineshloka
{अमात्येषु च भिन्नेषु योधेषु विमुखेषु च}
{पुनरेकत्र करणं तेषां कर्म भविष्यति}


\twolineshloka
{एतस्मिन्नन्तरे पार्थाः सुखमेकाग्रबुद्धयः}
{सेनाकर्म करिष्यन्ति द्रव्याणां चैव सञ्चयम्}


\twolineshloka
{भिद्यमानेषु च स्वेषु लम्बमाने तथा त्वयि}
{न तथा ते करिष्यन्ति सेनाकर्म न संशयः}


\twolineshloka
{एतत्प्रयोजनं चात्र प्राधान्येनोपलभ्यते}
{संगत्या धृतराष्ट्रश्च कुर्याद्धर्म्यं वचस्तव}


\twolineshloka
{स भवान्धर्मयुक्तश्च धर्म्यं तेषु समाचरन्}
{कृपालुषु परिक्लेशान्पाण्डवीयान्प्रकीर्तयन्}


\twolineshloka
{वृद्धेषु कुलधर्मं च ब्रुवन्पूर्वैरनुष्ठितम्}
{विभेत्स्यति मनांस्येषमिति मे नात्र संशयः}


\twolineshloka
{न च तेभ्यो भयं तेऽस्ति ब्राह्मणो ह्यसि वेदवित्}
{दूतकर्मणि युक्तश्च स्थविरश्च विशेषतः}


\threelineshloka
{स भवान्पुष्ययोगेन मुहूर्तेन जयेन च}
{कौरवेयान्प्रयात्वाशु कौन्तेयस्यार्थसिद्धये ॥वैशंपायन उवाच}
{}


\twolineshloka
{तथाऽनुशिष्टः प्रययौ द्रुपदेन महात्मना}
{पुरोधा वृत्तसंपन्नो नगरं नागसाह्वयम्}


\twolineshloka
{शिष्यैः परिवृतो विद्वान्नीतिशास्त्रार्थकोविदः}
{पाण्डवानां हितार्थाय कौरवान्प्रतिजग्मिवान्}


\chapter{अध्यायः ७}
\twolineshloka
{वैशंपायन उवाच}
{}


\twolineshloka
{पुरोहितं ते प्रस्थाप्य नगरं नागसाह्वयम्}
{दूतान्प्रस्थापयामासुः पार्थिवेभ्यस्ततस्ततः}


\twolineshloka
{प्रस्थाप्य दूतानन्यत्र द्वारकां पुरुषर्षभः}
{स्वयं जगाम कौरव्यः कुन्तीपुत्रो धनञ्जयः}


\twolineshloka
{गते द्वारवतीं कृष्ण बलदेवे च माधवे}
{सह वृष्ण्यन्धकैः सर्वैर्भोजैश्च शतशस्तदा}


\twolineshloka
{सर्वमागमयामास पाण्डवानां विचेष्टितम्}
{धृतराष्ट्रात्मजो राजा गूढैः प्रणिहितैश्चरैः}


\twolineshloka
{स श्रुत्वा माधवं यान्तं सदश्वैरनिलोपमैः}
{बलेन नातिमहता द्वारकामभ्ययात्पुरीम्}


\twolineshloka
{तमेव दिवसं चापि कौन्तेयः पाण्डुनन्दनः}
{आनन्रतनगरीं रम्यां जगामाशु धनञ्जयः}


\twolineshloka
{तौ यात्वा पुरुषव्याघ्रौ द्वारकां कुरुनन्दनौ}
{सुप्तं ददृशतुः कृष्णं शयानं चाभिजग्मतुः}


\twolineshloka
{दुर्योधनस्तु प्रथमं वासुदेवमुपाश्रयत्}
{उच्छीर्षतश्च कृष्णस्य निषसाद वरासने}


\twolineshloka
{ततः किरीटी तस्यानु प्रविवेश महामनाः}
{पश्चार्धे तु स कृष्णस्य प्रह्वोऽतिष्ठत्कृताञ्जलिः}


\twolineshloka
{प्रतिबुद्धः स वार्ष्णेयो ददर्शाग्रे किरीटिनाम्}
{सिंहासनगतं पश्चात्परिवृत्य च दृष्टवान्}


\twolineshloka
{स तयोः स्वागतं कृत्वा यथावत्प्रतिपूज्य तौ}
{तदागमनजं हेतुं पप्रच्छ मधुसूदनः}


\twolineshloka
{ततो दुर्योधनः कृष्णामुवाच प्रहसन्निव}
{विग्रहेऽस्मिन्भवान्साह्यं मम दातुमिहार्हति}


\twolineshloka
{समं हि भवतः सख्यं मम चैवार्जुनेऽपि च}
{तथा संबन्धकं तुल्यमस्माकं त्वयि माधव}


\twolineshloka
{अहं चाभिगतं सन्तो भजन्ते पूर्वसारिणः ॥त्वं च श्रेष्ठतमो लोके सतामद्य जनार्दन}
{}


\twolineshloka
{सततं संमतश्चैव सद्वृत्तमनुपालय ॥कृष्ण उवाच}
{}


\twolineshloka
{भवानभिगतः पूर्वमत्र मे नास्ति संशयः}
{दृष्टस्तु प्रथमं राजन्मया पार्थो धनञ्जयः}


\twolineshloka
{तव पूर्वाभिगमनात्पूर्वं चाप्यस्य दर्शनात्}
{साहाय्यमुभयोरेव करिष्यामि सुयोधन}


\twolineshloka
{प्रवारणं तु बालानां पूर्वं कार्यमिति श्रुतिः}
{तस्मात्प्रवारणं पूर्वमर्हः पार्थो धनञ्जयः}


\twolineshloka
{मत्संहननतुल्यानां गोपानामर्बुदं महत्}
{नारायणा इति ख्याताः सर्वे संग्रामयोधिनः}


\twolineshloka
{ते वा युधि दुराधर्षा भवन्त्वेकस्य सैनिकाः}
{अयुध्यमानः संग्रामे न्यस्तशस्त्रोऽहमेकतः}


\twolineshloka
{आभ्यामन्यतरं पार्थ यत्ते हृद्यतरं मतम्}
{तद्वृणीतां भवानग्रे प्रवार्यस्त्वं हि धर्मतः}


\threelineshloka
{` एतद्विदित्वा कौन्तेय विचार्य च पुनः पुनः}
{तान्वा वरय साहाय्ये मां साचिव्येऽथवा पुनः 'वैशंपायन उवाच}
{}


\twolineshloka
{एवमुक्तस्तु कृष्णेन कुन्तीपुत्रे धनञ्जयः}
{अयुध्यमानं संग्रामे वरयामास केशवम्}


\twolineshloka
{नारायणममिघ्नं कामाज्जातमजं नृषु}
{सर्वक्षत्रस्य पुरतो देवदानवयोरपि}


\twolineshloka
{दुर्योधनस्तु तत्सैन्यं सर्वमावरयत्तदा}
{सहस्राणां महस्रं तु योधानां प्राप्य भारत}


\twolineshloka
{कृष्णं चापहृतं ज्ञात्वा संप्राप परमां मुदम्}
{दुर्योधनस्तु तत्सैन्यं सर्वमादाय पार्थिवः}


\fourlineindentedshloka
{ततोऽभ्ययाद्भीमबलो रौहिणेयं महाबलम्}
{सर्वं चागमने हेतुं स तस्मै संन्यवेदयत्}
{प्रत्युवाच ततः शौरिर्धार्तराष्ट्रमिदं वचः ॥बलदेव उवाच}
{}


\twolineshloka
{विदितं ते नरव्याघ्र सर्वं भवितुमर्हति}
{यन्मयोक्तं विराटस्य पुरा वैवाहिके तदा}


\twolineshloka
{निगृह्योक्तो हृषीकेशस्त्वदर्थं कुरुनन्दन}
{मया संबन्धकं तुल्यमिति राजन्पुनःपुनः}


\twolineshloka
{न च तद्वाक्यमुक्तं वै केशवं प्रत्यपद्यत}
{न चाहमुत्सहे कृष्णं विना स्थातुमपि क्षणम्}


\twolineshloka
{नाहं सहायः पार्थस्य नापि दुर्योधनस्य वै}
{इति मे निश्चिता बुद्धिर्वासुदेवमवेक्ष्य ह}


\threelineshloka
{जातोऽसि भारते वंशे सर्वपार्थिवपूजिते}
{गच्छ युध्यस्व धर्मेण क्षात्रेण पुरुषर्षभ ॥वैशंपायन उवाच}
{}


\twolineshloka
{इत्येवमुक्तस्तु तदा परिष्वज्य हलायुधम्}
{कृष्णं चापि महाबाहुमामन्त्र्य भरतर्षभ}


\twolineshloka
{सोऽभ्ययात्कृतवर्माणं धृतराष्ट्रसुतो नृपः}
{कृतवर्मा ददौ तस्य सेनामक्षौहिणीं तदा}


\twolineshloka
{स तेन सर्वसैन्येन भीमेन कुरुनन्दनः}
{वृतः परिययौ हृष्टः सुहृदः संप्रहर्षयन्}


\twolineshloka
{ततः पीताम्बरधरो जगत्स्रष्टा जनार्दनः}
{गते दुर्योधने कृष्णः किरीटिनमथाब्रवीत्}


\fourlineindentedshloka
{अयुध्यमानः कां बृद्धिमास्थायाहं वृतस्त्वया}
{अर्जुन उवाच}
{भवान्समर्थस्तान्सर्वान्निहन्तुं नात्र संशयः}
{निहन्तुमहमष्येकः समर्थः पुरुषर्षभः}


\twolineshloka
{भवांस्तु कीर्तिमाँल्लोके तद्यशस्त्वां गमिष्यति}
{यशसां चाहमप्यर्थी तस्मादसि मया वृतः}


\threelineshloka
{सारथ्यं तु त्वया कार्यमिति मे मानसं सदा}
{चिररात्रेप्सितं कामं तद्भवान्कर्तुमर्हति ॥वासुदेव उवाच}
{}


\threelineshloka
{उपपन्नमिदं पार्थ यत्स्पर्धसि मया सह}
{सारथ्यं ते करिष्यामि कामः संपद्यतां तव ॥वैशंपायन उवाच}
{}


\twolineshloka
{एवं प्रमुदितः पार्थः कुष्णेन सहितस्तदा}
{वृतो दाशार्हप्रवरैः पुनरायाद्युधिष्ठिरम्}


\chapter{अध्यायः ८}
\twolineshloka
{वैशंपायन उवाच}
{}


\twolineshloka
{शल्यः श्रुत्वा तु दूतानां सैन्येन महता वृतः}
{अभ्यात्पाण्डवान्राजन्सह पुत्रैर्महारथैः}


\twolineshloka
{तस्य सेनानिवेशोऽभूदध्वर्धमिव योजनम्}
{तथा हि विपुलां सेनां बिभर्ति स नरर्षभः}


\twolineshloka
{अक्षौहिणीपती राजन्महावीर्यपराक्रमाः}
{विचित्रकवचाः शूरा विचित्रध्वजकार्मुकाः}


\twolineshloka
{विचित्राभरणाः सर्वे विचित्ररथवाहनाः}
{विचित्रस्रग्धराः सर्वे विवित्राम्बरभूषणाः}


\twolineshloka
{स्वदेशवेषाभरणा वीराः शतसहस्रशः}
{तस्य सेनाप्रणेतारो बभूवुः क्षत्रियर्षभाः}


\twolineshloka
{व्यथयन्निव भूतानि कम्पयन्निव मेदिनीम्}
{शनैर्विश्रामयन्सेनां स ययौ यत्र पाण्डवः}


\twolineshloka
{ततो दुर्योधनः श्रुत्वा महात्मानं महारथम्}
{उपायान्तमभिद्रुत्य स्वयमानर्च भरत}


\twolineshloka
{कारयामास पूजार्थं तस्य दुर्योधनः सभाः}
{रमणीयेषु देशेषु रत्नचित्राः स्वलङ्कृताः}


\twolineshloka
{शिल्पिभिर्विविधैश्चैव क्रीडास्तत्र प्रयोजिताः}
{तत्र माल्यानि मांसानि भक्ष्यं पेयं च सत्कृतम्}


\twolineshloka
{कूपाश्च विविधाकारा औदकानि गृहाणि च ॥स ताः सभाः समासाद्य पूज्यमानो यथाऽमरः}
{}


\twolineshloka
{दुर्योधनस्य सचिवैर्देशे देशे समन्ततः ॥आजगाम सभामन्यां देवावसथवर्चसम्}
{}


\twolineshloka
{स तत्र विषयैर्युक्तैः कल्याणैरतिमानुषैः ॥मेनेऽभ्यधिकमात्मानमवमेने पुरन्दरम्}
{}


\twolineshloka
{पप्रच्छ स ततः प्रेष्यान्प्रहृष्टः क्षत्रियर्षभः ॥युधिष्ठिरस्य पुरुषाः केऽत्र चक्रुः सभा इमाः}
{}


\twolineshloka
{आनीयन्तां सभाकाराः प्रदेयार्हा हि मे मताः ॥प्रदेयमेषां दास्यामि कुन्तीपुत्रोऽनुमन्यताम्}
{5-8-15a` ततःप्रहृष्टं राजानं ज्ञात्वा ते सचिवास्तदा'दुर्योधनाय तत्सर्वं कथयन्ति स्म विस्मिताः}


\twolineshloka
{संप्रहृष्टो यदा शल्यो दिदित्सुरपि जीवितम्}
{गूढो दुर्योधनस्तत्र दर्शयामास मातुलम्}


\threelineshloka
{तं दृष्ट्वा मद्रराजश्च ज्ञात्वा यत्नं च तस्य तम्}
{परिष्वज्याब्रवीत्प्रीत इष्टोऽर्थो ध्रियतामिति ॥दुर्योधन उवाच}
{}


\twolineshloka
{सत्यवाग्भव कल्याण वरो वै मम दीयताम्}
{सर्वसेनाप्रणेता वै भवान्भवितुमर्हति}


\threelineshloka
{यथैव पाण्डवास्तुभ्यं तथैव भवतो ह्यहम्}
{अनुमान्यं च पाल्यं च भक्तं च भज मां विभो ॥वैशंपायन उवाच}
{}


\threelineshloka
{कृतमित्यब्रवीच्छल्यः किमन्यत्क्रियतामिति}
{कृतमित्येव गान्धारिः प्रत्युवाच पुनःपुनः ॥शल्य उवाच}
{}


\twolineshloka
{गच्छ दुर्योधन पुरं स्वकमेव नरर्षभ}
{अहं गमिष्ये द्रुष्टुं वै युधिष्ठिरमरिंदमम्}


\threelineshloka
{दृष्ट्वा युधिष्ठिरं राजन्क्षिप्रमेष्ये नराधिप}
{अवश्यं चापि द्रष्टव्यः पाण्डवः पुरुषर्षभः ॥दुर्योधन उवाच}
{}


\threelineshloka
{क्षिप्रामागम्यतां राजन्पाण्डवं वीक्ष्य पार्थिव}
{त्वय्यधीनाः स्म राजेन्द्र वरदानं स्मरस्व नः ॥शल्य उवाच}
{}


\threelineshloka
{क्षिप्रमेष्यामि भद्रं ते गच्छस्व स्वपुरं नृप}
{` दृष्ट्वा तु पाण्डवान्राजन्न मिथ्या कर्तुमुत्सहे ॥वैशंपायन उवाच}
{}


% Check verse!
पर्यष्वजेतामन्योन्यं शल्यदुर्योधनावुभौ
\twolineshloka
{स तथा शल्यमामन्त्र्य पुरायात्स्वकं पुरम्}
{शल्यो जगाम कौन्तेयानाख्यातुं कर्म तस्य तत्}


\twolineshloka
{उपप्लाव्यं स गत्वा तु स्कन्धावारं प्रविश्य च}
{पाण्डवानथ तान्सर्वाञ्शल्यस्तत्र ददर्श ह}


\twolineshloka
{चिरात्तु दृष्ट्वा राजानं मातुलं समितिञ्जयम्}
{आसनेभ्यः समुत्पेतुः सर्वे सहयुधिष्ठिराः}


\twolineshloka
{तथा भीमार्जुनौ हृष्टौ स्वस्त्रीयौ च यमावुभौ}
{द्रौपदी च सुभद्रा च अभिमन्युश्च भारत}


\twolineshloka
{समेत्य च महापाहुं शल्यं पाण्डुसुतस्तदा}
{पाद्यमर्घ्यं च गां चैव प्रतिग्राह्य पुरोधसा}


\twolineshloka
{कताञ्जलिरदीनात्मा धर्मात्मा शल्यमब्रवीत्}
{स्वागतं तेऽस्तु वै राजन्नेतदासनमास्यताम्}


\threelineshloka
{ततो न्यषीदच्छल्यश्च काञ्चने परमासने}
{तत्र पाद्यथार्घ्यं च न्यवेदयत पाण्डवः ॥शल्य उवाच}
{}


\twolineshloka
{दुःखस्यैतस्य महतो धार्तराष्ट्रकृतस्य वै}
{अवाप्स्यसि फलं राजन्हत्वा शत्रून्परन्तप}


\threelineshloka
{विदितं ते महाराज लोकतन्त्रं नराधिप}
{तस्माल्लोककृतं किञ्चित्तव तात न विद्यते ॥वैशंपायन उवाच}
{}


\twolineshloka
{निवेद्य चार्घ्यं विधिवन्मद्रराजाय भारत}
{कुशलं पाण्डवोऽपृच्छच्छल्यं सर्वसुखावहम्}


\threelineshloka
{स तैः परिवृतः सर्वैः पाण्डवैर्धर्मचारिभिः}
{आसने चोपविष्टः स शल्यः पार्थमथाब्रवीत् ॥शल्य उवाच}
{}


\twolineshloka
{कुशलं राजशार्दूल सर्वत्र कुरुनन्दन}
{अरण्यवासाद्दिष्ट्याऽसि विमुक्तो जयतां वर}


\twolineshloka
{दुष्करं ते कृतं राजन्निर्जने गहने वने}
{भ्रातृभिः सह कौन्तेय कृष्णाया चानयाऽनघ}


\twolineshloka
{अज्ञातवासं घोरं च वसता दुष्करं कृतम्}
{दुःखमेव कुतः सौख्यं राज्यभ्रष्टस्य भारत}


\twolineshloka
{सत्ये तपसि दाने च तव बुद्धिर्युधिष्ठिर}
{क्षमा दमश्चाहिंसा च सत्यं चैव युधिष्ठिर}


\twolineshloka
{अद्भुतश्च पुनर्लोकस्त्वयि राजन्प्रतिष्ठितः}
{मृदुर्वदान्यो ब्रह्मण्यो दान्तो धर्मपरायणः}


\twolineshloka
{धर्मास्ते विदिता राजन्बहवो लोकसाक्षिकाः}
{सर्वं जगदिदं तात विदितं ते परन्तप}


\twolineshloka
{मुदितश्च पुनर्लोकस्त्वयि राजन्प्रतिष्ठिते}
{दिष्ट्या कृच्छ्रमिदं राजन्पारितं भरतर्षभ}


\threelineshloka
{दिष्ट्या पश्यामि राजेन्द्र धर्मात्मानं सहानुगम्}
{निस्तीर्णदुष्करं राजंस्त्वां धर्मनिचयं प्रभो ॥वैशंपायन उवाच}
{}


\threelineshloka
{ततोऽस्याकथयद्राजा दुर्योधनसमागमम्}
{तच्च शुश्रूषितं सर्वं वरदानं च भारत ॥युधिष्ठिर उवाच}
{}


\twolineshloka
{सुकृतं ते कृतं राजन्प्रहृष्टेनान्तरात्मना}
{दुर्योधनस्य यद्वीर त्वया वाचा प्रतिश्रुतम्}


\twolineshloka
{एकं त्विच्छामि भद्रं ते क्रियमाणं महीपते}
{राजन्नकर्तव्यमपि कर्तुमर्हसि सत्तम}


\twolineshloka
{मम त्ववेक्षया वीर शृणु विज्ञापयामि ते}
{भवानिह महाराज वासुदेवसमो युधि}


\twolineshloka
{कर्णार्जुनाभ्यां संप्राप्ते द्वैरथे राजसत्तम}
{कर्णस्य भवता कार्यं सारथ्यं नात्र संशयः}


\fourlineindentedshloka
{तत्र पाल्योऽर्जुनो राजन्यदि मत्प्रियमिच्छसि}
{तेजोवधश्च ते कार्यः सौतेरस्मञ्जयावह्नः}
{अकर्तव्यमपि ह्येतत्कर्मुमर्हसि मातुल ॥शल्य उवाच}
{}


\twolineshloka
{श्रृणु पाण्डव भद्रं ते यद्ब्रवीषि महात्मनः}
{तेजोवधनिमित्तं मां सूतपुत्रस्य सङ्गमे}


\twolineshloka
{अहं तस्य भविष्यामि सङ्घ्रामे सारथिर्ध्रुवम्}
{वासुदेवेन हि समं नित्यं मां सोऽभिमन्यतो}


\twolineshloka
{तस्याहं कुरुशार्दूल प्रतीपमहितं वचः}
{ध्रुवं संकथयिष्यामि योद्धुकामस्य संयुगे}


\twolineshloka
{यथा स हृतदर्पश्च हृततेजाश्च पाण्डव}
{भविष्यति सुखं हन्तुं सत्यमेतद्ब्रवीमि ते}


\twolineshloka
{एवमेतत्करिष्यामि यथा तात त्वमात्थ माम्}
{यच्चान्यदपि शक्ष्यामि तत्करिष्यामि ते प्रियम्}


\twolineshloka
{यच्च दुःखं त्वया प्राप्तं द्यूते वै कृष्णया सह}
{परुषाणि च वाक्यानि सूतपुत्रकृतानि वै}


\twolineshloka
{जटासुरात्परिक्लेशः कीचकाच्च महाद्युते}
{द्रौपद्याऽधिगतं सर्वं दमयन्त्या यथाऽशुभम्}


\twolineshloka
{सर्वं दुःखमिदं वीर सुखोदर्कं भविष्यति}
{नात्र मन्युस्त्वया कार्यो विधिर्हि बलवत्तरः}


\twolineshloka
{दुःखानि हि महात्मानः प्राप्नुवन्ति युधिष्ठिर}
{देवैरपि हि दुःखानि प्राप्तानि जगतीपते}


\twolineshloka
{इन्द्रेण श्रूयते राजन्सभार्येण महात्मना}
{अनुभूतं महद्दुःखं देवराजेन भारत}


\chapter{अध्यायः ९}
\twolineshloka
{युधिष्ठिर उवाच}
{}


\threelineshloka
{कथमिन्द्रेण राजेन्द्र सभार्येण महात्मना}
{दुःखं प्राप्तं परं घोरमेतदिच्छामि वेदितुम् ॥शल्य उवाच}
{}


\twolineshloka
{शृणु राजन्पुरावृत्तमितिहासं पुरातनम्}
{सभार्येण यथा प्राप्तं दुःखमिन्द्रेण भारत}


\twolineshloka
{त्वष्टा प्रजापतिर्ह्यासीद्देवश्रेष्ठो महातपाः}
{स पुत्रं वै त्रिशिरसमिन्द्रद्रोहात्किलासृजत्}


\twolineshloka
{ऐन्द्रं स प्रार्थयत्स्थानं विश्वरूपो महाद्युतिः}
{तैस्त्रिभिर्वदनैर्घोरैः सूर्येन्दुज्वलनोपमैः}


\twolineshloka
{वेदानेकेन सोऽधीते सुरामेकेन चापिबत्}
{एकेन च दिशः सर्वाः पिबन्निव निरीक्षते}


\twolineshloka
{स तपस्वी मुदुर्दान्तो धर्मे तपसि चोद्यतः}
{तपस्तस्य महत्तीव्रं सुदुश्चरमरिन्दम}


\twolineshloka
{तस्य दृष्ट्वा तपोवीर्यं सत्यं चामिततेजसः}
{विषादमगमच्छक्र इन्द्रोऽयं मा भवेदिति}


\twolineshloka
{कथं सज्जेच्च भोगेषु न च तप्येन्महत्तपः}
{विवर्धमानस्त्रिशिराः सर्वं हि भुवनं ग्रसेत्}


\twolineshloka
{इति संचिन्त्य बहुधा बुद्धिमान्भरतर्षभ}
{आज्ञापयत्सोऽप्सरसस्त्वष्टृपुत्रप्रलोभने}


\twolineshloka
{यथा स सज्जेत्रिशिराः कामभोगेषु वै भृशम्}
{क्षिप्रं कुरुत गच्छध्वं प्रलोभयत माचिरम्}


\twolineshloka
{शृङ्गारवेषाः सुश्रोण्यो हारैर्युक्ता मनोहरैः}
{हावभावसमायुक्ताः सर्वाः सौन्दर्यशोभिताः}


\twolineshloka
{प्रलोभयत भद्रं वः शमयध्वं भयं मम}
{अस्वस्थं ह्यात्मनात्मानं लक्षयामि वराङ्गनाः}


\fourlineindentedshloka
{भयं तन्मे महाघोरं क्षिप्रं नाशयताबलाः}
{अप्सरस ऊचुः}
{तथा यत्नं करिष्यामः शक्र तस्य प्रलोभने}
{यथा नावाप्स्यसि भयं तस्माद्बलनिषूदन}


\twolineshloka
{निर्दहन्निव चक्षुर्भ्यां योऽसावास्ते तपोनिधिः}
{तं प्रलोयितुं देव गच्छामः सहिता वयम्}


\fourlineindentedshloka
{यतिष्यामो वशे कर्तुं व्यपनेतुं च ते भयम्}
{शल्य उवाच}
{इन्द्रेण तास्त्वनुज्ञाता जग्मुस्त्रिशिरसोऽन्तिकम्}
{तत्र ता विविधैर्भावैर्लोभयन्त्यो वराङ्गनाः}


\twolineshloka
{नृत्तं संदर्शयामासुस्तथैवाङ्गेषु सौष्ठवम्}
{नाभ्यगच्छत्प्रहर्षं ताः स पश्यन्सुमहातपाः}


\twolineshloka
{इन्द्रियाणि वशे कृत्वा पूर्णसागरसन्निभः}
{तास्तु यत्नं परं कृत्वा पुनः शक्रमुपस्थिताः}


\twolineshloka
{कृताञ्जलिपुटाः सर्वा देवराजमथाब्रुवन्}
{न स शक्यः सुदुर्धर्षो धैर्याच्चालयितुं प्रभो}


\twolineshloka
{यत्ते कार्यं महाभाग क्रियतां तदनन्तरम्}
{संपूज्याप्सरसः शक्रो विसृज्य च महामतिः}


\twolineshloka
{चिन्तयामास तस्यैव वधोपायं युधिष्ठिर}
{स तूष्णीं चिन्तयन्वीरो देवराजः प्रतापवान्}


\twolineshloka
{विनिश्चितमतिर्धीमान्वधे त्रिशिरसोऽभवत्}
{वज्रमस्य क्षिपाम्यद्य स क्षिप्रं नभविष्यति}


\twolineshloka
{शत्रुः प्रवृद्धो नोपेक्ष्यो दुर्बलोऽपि बलीयसा}
{शास्त्रबुद्ध्या विनिश्चित्य कृत्वा बुद्धिं वधे दृढाम्}


\twolineshloka
{अथ वैश्वानरनिभं घोररूपं भयावहम्}
{मुमोच वज्रं संक्रुद्धः शक्रस्त्रिशिरसं प्रति}


\twolineshloka
{स पपात हतस्तेन वज्रेण दृढमाहतः}
{पर्वतस्येव शिखरं प्रणुन्नं मेदिनीतले}


\twolineshloka
{तं तु वज्रहतं दृष्ट्वा शयानमचलोपमम्}
{न शर्म लेभे देवेन्द्रो दीपिततस्तस्य तेजसा}


\twolineshloka
{हतोऽपि दीप्ततेजाः स जीवन्निव हि दृश्यते}
{घातितस्य शिरांस्याजौ जीवन्तीवाद्भुतानि वै}


\threelineshloka
{` शिरांसि तस्याजायन्त त्रीण्येव शकुनास्त्रयः}
{तित्तिरिः कलविङ्कश्च तथैव च कपिञ्जलः}
{विश्वरूपशिरांस्येव जायन्ते तानि भारत '}


\twolineshloka
{ततोऽतिभीतगात्रस्तु शक्र आस्ते विचारयन्}
{अथाजगाम परशुं स्कन्धेनादाय वर्धकिः}


\twolineshloka
{तदरण्यं महाराज यत्रास्तेऽसौ निपातितः}
{स भीतस्तत्र तक्षाणं घटमानं शचीपतिः}


\threelineshloka
{अपश्यदब्रवीच्चैनं सत्वरं पाकशासनः}
{क्षिप्रं छिन्धि शिरांस्यस्य कुरुष्व वचनं मम ॥तक्षोवाच}
{}


\threelineshloka
{महास्कन्धो भृशं ह्येष परशुर्नभविष्यति}
{कर्तुं चाहं न शक्ष्यामि कर्म राद्भिर्विगर्हितम् ॥इन्द्र उवाच}
{}


\threelineshloka
{मा भैस्त्वं शीघ्रमेतद्वै कुरुष्व वचनं मम}
{मत्प्रसादाद्धि ते शस्त्रं वज्रकल्पं भविष्यति ॥तक्षोवाच}
{}


\threelineshloka
{कं भवन्तमहं विद्यां घोरकर्माणमद्य वै}
{एतदिच्छाम्यहं श्रोतुं तत्त्वेन कथयस्व मे ॥इन्द्र उवाच}
{}


\twolineshloka
{अहमिन्द्रो देवराजस्तक्षन्विदितमस्तु ते}
{कुरुष्वैतद्यथोक्तं मे तक्षन्माऽत्र विचारय}


\threelineshloka
{` मया हि निहतः शेते त्रिशिरास्त्वं च विद्वि वै}
{वैशंपायन उवाच}
{स प्रह्वः प्राञ्जलिर्भूत्वा इदं वचनमब्रवीत्'}


\threelineshloka
{क्रूरेण नापत्रपसे कथं शक्रेह कर्मणा}
{ऋषिपुत्रमिमं हत्वा ब्रह्महत्याभयं न ते ॥शक्र उवाच}
{}


\twolineshloka
{पश्चाद्धर्मं चरिष्यामि पावनार्थ सुदुश्चरम्}
{शत्रुरेष महावीर्यो वज्रेण निहतो मया}


\twolineshloka
{अद्यापि चाहमुद्विग्नस्तक्षन्नस्माद्बिभेमि वै}
{क्षिप्रं छिन्धि शिरांसि त्वं करिष्येऽनुग्रहं तव}


\threelineshloka
{शिरः पशोस्ते दास्यन्ति भागं यज्ञेषु मानवाः}
{एष तेऽनुग्रहस्तक्षन्क्षिप्रं कुरु मम प्रियम् ॥शल्य उवाच}
{}


\twolineshloka
{एतच्छ्रुत्वा तु तक्षा स महेन्द्रवचनात्तदा}
{शिरांस्यथ त्रिशिरसः कुठारेणाच्छिनत्तदा}


\twolineshloka
{निकृत्तेषु ततस्तेषु निष्क्रामन्नण्डजास्त्वथ}
{कपिञ्जलास्तित्तिराश्च कलविङ्काश्च सर्वशः}


\twolineshloka
{येन वेदानधीते स्म पिबते सोममेव च}
{तस्माद्वक्रार्द्विनिश्चेरुः क्षिप्रं तस्य कपिञ्जलाः}


\twolineshloka
{येन सर्वा दिशो राजन्पिबन्निव नीरीक्षते}
{तस्माद्वक्राद्विनिश्चेरुस्तित्तिरास्तस्य पाण्डव}


\twolineshloka
{यत्सुरापं तु तस्यासीद्वक्रं त्रिशिरसस्तदा}
{कलविङ्काः समुत्पेतुः श्येनाश्च भरतर्षभ}


\twolineshloka
{ततस्तेषु निकृत्तेषु विज्वरो मघवानथ}
{`तक्ष्णे तथा वरं दत्त्वा प्रहृष्टस्त्रिदशेश्वरः ॥'}


\twolineshloka
{जगाम त्रिदिवं देवस्तक्षाऽपि स्वगृहान्ययौ}
{मेने कृतार्थमात्मानं हत्वा शत्रुं सुरारिहा}


\twolineshloka
{` तक्षापि स्वगृहं गत्वा नैव शंसति कस्यचित्}
{अथैनं नाभिजानन्ति वर्षमेकं तथागतम्}


\threelineshloka
{अथ संवत्सरे पूर्णे भूताः पशुपतेः प्रभो}
{तमाक्रोशन्त मघवान्नः प्रभुर्ब्रह्महा इति}
{}


\twolineshloka
{तत इन्द्रो व्रतं घोरमाचरत्पाकशासनः}
{तपसा च सुसंयुक्तः सह देवैर्मरुद्गणैः}


\twolineshloka
{समुद्रेषु पृथिव्यां च वनस्पतिषु स्त्रीषु च}
{विभज्य ब्रह्महत्यां च तान्वरैरप्ययोजयत्}


\twolineshloka
{वरदस्तु वरं दत्त्वा पृथिव्यै सागराय च}
{वनस्पतिभ्यः स्त्रीभ्यश्च ब्रह्महत्यां नुनोद ताम्}


\twolineshloka
{ततस्तु शुद्धो भगवान्देवैर्लोकैश्च पूजितः}
{इन्द्रस्थानमुपातिष्ठत्पूज्यमानो महर्षिभिः'}


\threelineshloka
{त्वष्टा प्रजापतिः श्रुत्वा शक्रेणाथ हतं सुतम्}
{क्रोधसंरक्तनयन इदं वचनमब्रवीत् ॥त्वष्टोवाच}
{}


\twolineshloka
{तप्यमानं तपो नित्यं क्षान्तं दान्तं जितेन्द्रियम्}
{विनाऽपराधेन यतः पुत्रं हिंसितवान्मम}


\threelineshloka
{तस्माच्छक्रविनाशाय वृत्रमुत्पादयाम्यहम्}
{लोकाः पश्यन्तु मे वीर्यं तपसश्च बलं महम्}
{स च पश्यतु देवेन्द्रो दुरात्मा पापचेतनः}


\twolineshloka
{उपस्पृश्य ततः क्रुद्धस्तपस्वी सुमहायशाः}
{अग्नौ हुत्वा समुत्पाद्य घोरं वृत्रमुवाच ह}


\twolineshloka
{इन्द्रशत्रो विवर्धस्व प्रभावात्तपसो मम}
{सोऽवर्घत दिवं स्तब्ध्वा सूर्यवैश्वानरोपमः}


\twolineshloka
{किं करोमीति चोवाच कालसूर्य इवोदितः}
{शक्रं जहीति चाप्युक्तो जगाम त्रिदिवं ततः}


\twolineshloka
{ततो युद्धं समभवद्वृत्रवासवयोर्महत्}
{संक्रुद्धयोर्महाघोरं प्रसक्तं कुरुसत्तम}


\twolineshloka
{ततो जग्राह देवेन्द्रं वृत्रो वीरः शतक्रतुम्}
{अपावृत्याक्षिपद्वक्रे शक्रं कोपसमन्वितः}


\twolineshloka
{ग्रस्ते वृत्रेण शक्रे तु संभ्रान्तास्त्रिदिवेश्वराः}
{असृजंस्ते महासत्वा जृम्भिकां वृत्रनाशिनीम्}


\twolineshloka
{विजृम्भमाणस्य ततो वृत्रस्यास्यादपावृतात्}
{स्वान्यङ्गान्यभिसंक्षिप्य निष्क्रान्तो बलनाशनः}


\twolineshloka
{ततः प्रभृति लोकस्य जृम्भिका प्राणिसंश्रिता}
{जहृषुश्च सुराः सर्वे शक्रं दृष्ट्वा विनिःसृतम्}


\twolineshloka
{ततः प्रववृते युद्धं वृत्रवासवयोः पुनः}
{संरब्धयोस्तदा घोरं सुचिरं भरतर्षभ}


\threelineshloka
{यदा व्यवर्धत रणे वृत्रो बलसमन्वितः}
{त्वष्टुस्तेजोबलाविद्धस्तदा शक्रो न्यवर्तत}
{}


\twolineshloka
{निवृत्ते च तदा देवा विषादमगमन्परम्}
{समेत्य सह शक्रेण त्वष्टुस्तेजोविमोहिताः}


\twolineshloka
{अमन्त्रयन्त ते सर्वे मुनिभिः सह भारत}
{किं कार्यमिति वै राजन्विचिन्त्य भयमोहिताः}


\twolineshloka
{जग्मुः सर्वे महात्मानं मनोभिर्विष्णुमव्ययम्}
{उपविष्टा मन्दराग्रे सर्वे वृत्रवधेप्सवः}


\chapter{अध्यायः १०}
\twolineshloka
{इन्द्र उवाच}
{}


\twolineshloka
{सर्वं व्याप्तमिदं देवा वृत्रेण जगदव्ययम्}
{न ह्यस्य सदृशं किञ्चित्प्रतिघाताय यद्भवेत्}


\twolineshloka
{समर्थो ह्यभवं पूर्वमसमर्थोऽस्मि सांप्रतम्}
{कथं नु कार्यं भद्रं वो दुर्धर्षः स हि मे मतः}


\twolineshloka
{तेजस्वी च महात्मा च युद्धे चामितविक्रमः}
{ग्रसेत्रिभुवनं सर्वं सदेवासुरमानुषम्}


\twolineshloka
{तस्माद्विनिश्चयमिमं शृणुध्वं त्रिदिवौकसः}
{विष्णोः क्षयमुपागम्य समेत्य च महात्मना}


\fourlineindentedshloka
{तेन संमन्त्र्य वेत्स्यामो वधोपायं दुरात्मनः}
{शल्य उवाच}
{एवमुक्ते मघवता देवाः शर्षिगणास्तदा}
{शरण्यं शरमं देवं जग्मुर्विष्णुं महाबलम्}


\twolineshloka
{ऊचुश्च सर्वे देवेशं विष्णुं वृत्रभयार्दिताः}
{त्रयो लोकास्त्वया क्रान्तास्त्रिभिर्विक्रमणैः पुरा}


\twolineshloka
{अमृतं चाहृतं विष्णो दैत्याश्च निहता रणे}
{बलिं बद्ध्वा महादैत्यं शक्रो देवाधिपः कृतः}


\twolineshloka
{त्वं प्रभुः सर्वदेवानां त्वया सर्वमिदं ततम्}
{त्वं हि देवो महादेव सर्वलोकनमस्कृतः}


\threelineshloka
{गतिर्भव त्वं देवानां सेन्द्राणाममरोत्तम}
{जगद्व्याप्तमिदं सर्वं वृत्रेणासुरसूदन ॥विष्णुरुवाच}
{}


\twolineshloka
{अवश्यं करणीयं मे भवतां हितमुत्तमम्}
{तस्मादुपायं वक्ष्यामि यथाऽसौ नभविष्यति}


\twolineshloka
{गच्छध्वं सर्षिगन्धर्वा यत्रासौ विश्वरूपधृक्}
{साम तस्य प्रयुञ्जध्वं तत एनं विजेष्यथ}


\twolineshloka
{भविष्यति जयो देवाः शक्रस्य मम तेजसा}
{अदृश्यश्च प्रवेक्ष्यामि वज्रे ह्यस्यायुधोत्तमे}


\threelineshloka
{गच्छध्वमृषिभिः सार्धं गन्धर्वैश्च सुरोत्तमाः}
{वृत्रस्य सह शक्रेण सन्धिं कुरुत माचिरम् ॥शल्य उवाच}
{}


\twolineshloka
{एवमुक्ते तु देवेन ऋषयस्त्रिदशास्तथा}
{ययुः समेत्य सहिताः शक्रं कृत्वा पुरःसरम्}


\twolineshloka
{समीपमेत्य च यदा सर्व एव महौजसः}
{तं तेजसा प्रज्वलितं प्रतपन्तं दिशो दश}


\twolineshloka
{ग्रसन्तमिव लोकांस्त्रीन्सूर्याचन्द्रमसौ यथा}
{ददृशुस्ते ततो वृत्रं शक्रेण सह देवताः}


\twolineshloka
{ऋषयोऽथ ततोऽभ्येत्य वृत्रमूचुः प्रियं वचः}
{व्याप्तं जगदिदं सर्वं तेजसा तव दुर्जय}


\twolineshloka
{न च शक्नोषि निर्जेषुं वासवं बलिनां वर}
{युध्यतोश्चापि वां कालो व्यतीतः सुमहानिह}


\twolineshloka
{पीड्यन्ते च प्रजाः सर्वाः सदेवासुरमानुषाः}
{सख्यं भवतु ते वृत्र शक्रेण सह नित्यदा}


\twolineshloka
{अवाप्स्यसि सुखं त्वं च शक्रलोकांश्च शाश्वतान्}
{ऋषिवाक्यं निशम्याथ वृत्रःस तु महाबलः}


\twolineshloka
{उवाच तानृषीन्सर्वान्प्रणम्य शिरसाऽसुरः}
{सर्वे यूयं महाभागा गन्धर्वाश्चैव सर्वशः}


\fourlineindentedshloka
{यद्ब्रूथ तच्छ्रुतं सर्वं ममापि श्रृणुतानघाः}
{सन्धिः कथं वै भविता मम शक्रस्य चोभयोः}
{तेजसोर्हि द्वयोर्देवाः सख्यं वै भविता कथम् ॥ऋषय ऊचुः}
{}


\twolineshloka
{सकृत्सतां सङ्गतमीप्सितव्यंततः परं भविता भव्यमेव}
{नातिक्रामेत्सत्पुरुषेण सङ्गतंतस्मात्सतां संगतमीप्सितव्यम्}


\twolineshloka
{दृढं सतां सङ्गतं चापि नित्यंब्रूयाच्चार्थं ह्यर्थकृच्छ्रेषु धीराः}
{महार्थवत्सत्पुरुषेण सङ्गतंतस्मात्सन्तं न जिघांसेत धीरः}


\twolineshloka
{इन्द्रः सतां संमतश्च निवासश्च महात्मनाम्}
{सत्यवादी ह्यनिन्द्यश्च धर्मवित्सूक्ष्मनिश्चयः}


\threelineshloka
{तेन ते सह शक्रेण सन्धिर्भवतु नित्यदा}
{एवं विश्वासमागच्छ मा ते भूद्बुद्धिरन्यथा ॥शल्य उवाच}
{}


\twolineshloka
{महर्षिवचनं श्रुत्वा तानुवाच महाद्युतिः}
{अवश्यं भगवन्तो मे माननीयास्तपस्विनः}


\twolineshloka
{ब्रवीमि यदहं देवास्तत्सर्वं क्रियते यदि}
{ततः सर्वं करिष्यामि यदूचुर्मां दिवौकसः}


\twolineshloka
{न शुष्केण न चाद्रेण नाश्मना न च दारुणा}
{न शस्त्रेण न चास्त्रेण न दिवा न तथा निशि}


\twolineshloka
{वध्यो भवेयं विप्रेन्द्राः शक्रस्य सह दैवतैः}
{एवं मे रोचते सन्धिः शक्रेण सह नित्यदा}


\twolineshloka
{बाढमित्येव ऋषयस्तमूचुर्भरतर्षभ}
{एवं वृत्ते तु सन्धाने वृत्रः प्रमुदितोऽभवत्}


\twolineshloka
{` ततः सन्धिं मिथः कृत्वा ऋषयो दीप्ततेजसः}
{शक्रस्य सह वृत्रेण पुनर्जग्मुर्यथागतम् '}


\threelineshloka
{युक्तः सदाऽभवच्चापि शक्रो हर्षसमन्वितः}
{वृत्रस्य वधसंयुक्तानुपायानन्वचिन्तयत् ॥ 5-10-34a`अभिसन्धिर्महेन्द्रस्य सन्धिकर्मणि यः कृतः}
{ऋषिभिस्त्वीरितं यच्च महेन्द्रस्तदनुस्मरन् ॥'}


\twolineshloka
{छिद्रान्वेषी समुद्विग्नः सदा वसति देवराट्}
{स कदाचिन्समुद्रान्ते समपश्यन्महासुरम्}


\twolineshloka
{सन्ध्याकाल उपावृत्ते मुहूर्ते चातिदारुणे}
{ततः संचिन्त्य भगवान्वरदानं महात्मनः}


\twolineshloka
{सन्ध्येयं वर्तते रौद्रा न रात्रिर्दिवसं न च}
{वृत्रश्चावश्यवध्योऽयं मम सर्वहरो रिपुः}


\twolineshloka
{यदि वृत्रं न हन्म्यद्य वञ्चयित्वा महासुरम्}
{महाबलं महाकायं न मे श्रेयो भविष्यति}


\twolineshloka
{एवं संचिन्तयन्नेव शक्रो विष्णुमनुस्मरन्}
{अथ फेनं तदाऽपश्यत् समुद्रे पर्वतोपमप्}


\twolineshloka
{नायं युष्को न चाद्रोऽयं न च शस्त्रमिदं तथा}
{एनं क्षेप्स्यामि वृत्रस्य क्षणादेव नशिष्यति}


\twolineshloka
{सवज्रमथ फेनं तं क्षिप्रं वृत्रे निसृष्टवान्}
{प्रविश्य फेनं तं विष्णुरथ वृत्रं व्यनाशयत्}


\twolineshloka
{निहते तु ततो वृत्रे दिशो वितिमिराऽभवन्}
{प्रववौ च शिवो वायुः प्रजाश्च जहृषुस्तथा}


\twolineshloka
{ततो देवाः सगन्धर्वा यक्षरक्षोमहोरगाः}
{ऋषयश्च महेन्द्रं तमस्तुवन्विविधैः स्तवैः}


\twolineshloka
{नमस्कृतः सर्वभूतैः सर्वभूतान्यसान्त्वयत्}
{हत्वा शत्रुं प्रहृष्टात्मा वासवः सह दैवतैः}


\twolineshloka
{विष्णुं त्रिभुवनश्रेष्ठं पूजयामास धर्मवित्}
{ततो हते महावीर्ये वृत्रे देवभयङ्करे}


\twolineshloka
{अनृतेनाभिभूतोऽभूच्छक्रः परमदुर्मनाः}
{त्रैशीर्षयाऽभिभूतश्च स पूर्वं ब्रह्महत्यया}


\twolineshloka
{` महादेवस्य भूतैश्च स पुनर्ब्रह्महा इति}
{आक्रुष्टो निर्भयैर्भूयो व्रीडितो बलवृत्रहा '}


\twolineshloka
{सोऽन्तमाश्रित्य लोकानां नष्टसंज्ञो विचेतनः}
{न प्राज्ञायत देवेन्द्रस्त्वभिभूतः स्वकर्मभिः}


\twolineshloka
{प्रतिच्छन्नोऽवसच्चाप्सु चेष्टमान इवोरगः}
{ततः प्रनष्टे देवेन्द्रे ब्रह्महत्या भयार्दिते}


\twolineshloka
{भूमिः प्रध्वस्तसंकाशा निर्वृक्षा शुष्ककानना}
{विच्छिन्नस्रोतसो नद्यः सरांस्यनुदकानि च}


\twolineshloka
{संक्षोभश्चापि सत्वानामनावृष्टिकृतोऽभवत्}
{देवाश्चापि भृशं त्रस्तास्तथा सर्वे महर्षयः}


\twolineshloka
{अराजकं जगत्सर्वमभिभूतमुपद्रवैः}
{ततो भीताऽभवन्देवाः को नो राजा भवेदिति}


\twolineshloka
{दिवि देवर्षयश्चापि देवराजविनाकृताः}
{न स्म कश्चन देवानां राज्ये वै कुरुते मतिम्}


\chapter{अध्यायः ११}
\twolineshloka
{शल्य उवाच}
{}


\twolineshloka
{ऋषयोऽथाब्रुवन्सर्वे देवाश्च त्रिदिवेश्वराः}
{अयं वै नहुपः श्रीमान्देवराज्येऽभिषिच्यताम्}


\twolineshloka
{तेजस्वी च यशस्वी च धार्मिकश्चैव नित्यदा}
{ते गत्वा त्वब्रुवन्सर्वे राजा नो भव पार्थिव}


\twolineshloka
{स तानुवाच नहुषो देवानृपिगणांस्तथा}
{पितृभिः सहितान्राजन्परिप्सन्हितमात्मनः}


\twolineshloka
{दुर्बलोऽहं न मे शक्तिर्भवतां परिपालने}
{बलवाञ्ज्ञायतां कश्चिद्बलं शक्रे हि नित्यदा}


\twolineshloka
{तमब्रुवन्पुनः सर्वे देवा ऋषिपुरोगमाः}
{अस्माकं तपसा युक्तः पाहि राज्यं त्रिविष्टपे}


\twolineshloka
{परस्परभयं घोरमस्माकं हि न संशयः}
{अभिषिच्यस्व राजेन्द्र भव राजा त्रिविष्टपे}


\threelineshloka
{देवदानवयक्षाणामृषीणां रक्षसां तथा}
{पितृगन्धर्वभूतानां चक्षुर्विषयवर्तिनाम्}
{}


\twolineshloka
{तेज आदास्यसे पश्यन्बलवांश्च भविष्यसि}
{धर्मं पुरस्कृत्य सदा सरवलोकाधिपो भव}


\threelineshloka
{ब्रह्मर्षीश्चापि देवांश्च गोपायस्व त्रिविष्टपे}
{शल्य उवाच}
{अभिषिक्तः स राजेन्द्र ततो राजा त्रिविष्टपे}


\twolineshloka
{धर्मं पुरस्कृत्य तदा सर्वलोकाधिपोऽभवत्}
{सुदुर्लभं वरं लब्ध्वा प्राप्य राज्यं त्रिविष्टपे}


\twolineshloka
{धर्मात्मा सततं भूत्वा कामात्मा समपद्यत}
{देवोद्यानेषु सर्वेषु नन्दनोपवनेषु च}


\twolineshloka
{कैलासे हिमवत्पृष्ठे मन्दरे श्वेतपर्वते}
{सह्ये महेन्द्रे मलये समुद्रेषु सरित्सु च}


\twolineshloka
{अप्सरोभिः परिवृतो देवकन्यासमावृतः}
{नहुषो देवराजोऽथ क्रीडन्बहुविधं तदा}


\twolineshloka
{श्रृण्वन्दिव्या बहुविधाः कथाः श्रुतिमनोहराः}
{वादित्राणि च सर्वाणि गीतं च मधुरस्वनम्}


\twolineshloka
{विश्वावसुर्नारदश्च गन्धर्वाप्सरसां गणाः}
{ऋतवः षट्च देवेन्द्रं मूर्तिमन्त उपस्थिताः}


\twolineshloka
{मारुतः सुरभिर्वाति मनोज्ञः सत्वशीतलः}
{एवं विक्रीडतस्तस्य नहुषस्य दुरात्मनः}


\twolineshloka
{संप्राप्ता दर्शनं देवी शक्रस्य महिषी प्रिया}
{स तां संदृश्य दुष्टात्मा प्राह सर्वान्सभासदः}


\twolineshloka
{इन्द्रस्य महिषी देवी कस्मान्मां नोपतिष्ठति}
{अहमिन्द्रोऽस्मि देवानां लोकानां च तथेश्वरः}


\twolineshloka
{आगच्छतु शची मह्यं क्षिप्रमद्य निवेशनम्}
{तच्छ्रुत्वा दुर्मना देवी बृहस्पतिमुवाच ह}


\twolineshloka
{रक्ष मां नहुषाद्ब्रह्मंस्त्वामस्मि शरणं गता}
{सर्वलक्षणसंपन्नां ब्रह्मंस्त्वं मां प्रभाषसे}


\twolineshloka
{देवराजस्य दयितामत्यन्तं सुखभागिनीम्}
{अवैधव्येन युक्तां चाप्येकपत्नीं पतिव्रदाम्}


\twolineshloka
{उक्तवानसि मां पूर्वमृतां तां कुरु वै गिरम्}
{नोक्तपूर्वं च भगवन्मृषा ते किंचिदीश्वर}


\twolineshloka
{तस्मादेतद्भवेत्सत्यं त्वयोक्तं द्विजसत्तम}
{बृहस्पतिरथोवाच इन्द्राणीं भयमोहिताम्}


\twolineshloka
{यदुक्तासि मया देवि सत्यं तद्भविता ध्रुवम्}
{द्रक्ष्यसे देवराजं तमिन्द्रं शीघ्रमिहागतम्}


\twolineshloka
{न भेतव्यं च नहुषात्सत्यमेतद्ब्रवीमि ते}
{समानयिष्ये शक्रेण नचिराद्भवतीमहम्}


\twolineshloka
{अथ शुश्राव नहुष इन्द्राणीं शरणीं गताम्}
{बृहस्पतेरङ्गिरसश्रुक्रोध स नृपस्तदा}


\chapter{अध्यायः १२}
\twolineshloka
{शल्य उवाच}
{}


\twolineshloka
{क्रुद्धं तु नहुषं दृष्ट्वा देवा ऋषिपुरोगमाः}
{अब्रुवन्देवराजानं नहुषं घोरदर्शनम्}


\twolineshloka
{देवराज जहि क्रोधं त्वयि क्रुद्धे जगद्विभो}
{त्रस्तं सासुरगन्धर्वं सकिंनरमहोरगम्}


\twolineshloka
{जहि क्रोधमिमं साधो न कुप्यन्ति भवद्विधाः}
{परस्य पत्नी सा देवी प्रसीदस्व सुरेश्वर}


\twolineshloka
{निवर्तय मनः पापात्परदाराभिमर्शनात्}
{देवराजोऽसि भद्रं ते प्रजा धर्मेण पालय}


\twolineshloka
{एवमुक्तो न जग्राह तद्वचः काममोहितः}
{अथ देवानुवाचेदमिन्द्रं प्रति सुराधिपः}


\twolineshloka
{अहल्या धर्षिता पूर्वमृषिपत्नी यशस्विनी}
{जीवतो भर्तुरिन्द्रेण स वः किं न निवारितः}


\twolineshloka
{बहूनि च नृशंसानि कृतानीन्द्रेण वै पुरा}
{वैधर्म्याण्युपधाश्चैव स वः किं न निवारितः}


\threelineshloka
{उपतिष्ठतु देवी मामेतदस्या हितं परम्}
{युष्माकं च सदा देवाः शिवमेवं भविष्यति ॥देवा ऊचुः}
{}


\threelineshloka
{इन्द्राणीमानयिष्यामो यथेच्छसि दिवस्पते}
{जहि क्रोधमिमं वीर प्रीतो भव सुरेश्वर ॥शल्य उवाच}
{}


\twolineshloka
{इत्युक्त्वा तं तदा देवा ऋषिभिः सह भारत}
{तग्मुर्बृहस्पतिं वक्तुमिन्द्राणीं चाशुभं वचः}


\twolineshloka
{जानीमः शरणं प्राप्तामिन्द्राणीं तव वेश्मनि}
{दत्ताभयां च विप्रेन्द्र त्वया देवर्षिसत्तम}


\twolineshloka
{ते त्वां देवाः सगन्धर्वा ऋषयश्च महाद्युते}
{प्रसादयन्ति चेन्द्राणी नहुषाय प्रदीयताम्}


\twolineshloka
{इन्द्राद्विशिष्टो नहुषो देवराजो महाद्युतिः}
{वृणोत्विमं वरारोहा भर्तृत्वे वरवर्णिनी}


\twolineshloka
{एवमुक्ते तु सा देवी बाष्पमुत्सृज्य सस्वनम्}
{उवाच रुदती दीना बृहस्पतिमिदं वचः}


\threelineshloka
{नाहमिच्छामि नहुषं पतिं देवर्षिसत्तम}
{शरणागतास्मि ते ब्रहंस्त्रायस्व महतो भयात् ॥बृहस्पतिरुवाच}
{}


\twolineshloka
{शरणागतं न त्यजेयमिन्द्राणि मम निश्चयः}
{धर्मज्ञां सत्यशीलां च न त्यजेयमनिन्दिते}


\twolineshloka
{नाकार्यं कर्तुमिच्छामि ब्राह्मणः सन्विशेषतः}
{श्रुतधर्मा सत्यशीलो जानन्धर्मानुशासनम्}


\twolineshloka
{नाहमेतत्करिष्यामि गच्छध्वं वै सुरोत्तमाः}
{अस्मिंश्चार्थे पुरा गीतं ब्रह्मणा श्रूयतामिदम्}


\twolineshloka
{न तस्य बीजं रोहति रोहकालेन तस्य वर्षं वर्षति वर्षकाले}
{भीतं प्रपन्नं प्रददाति शत्रवेन स त्रातारं लभते त्राणमिच्छन्}


\twolineshloka
{मोघमन्नं विन्दति चाप्यचेताःस्वर्गाल्लोकाद्धश्यति नष्टचेष्टः}
{भीतं प्रपन्नं प्रददाति यो वैन तस्य हव्यं प्रतिगृह्णन्ति देवाः}


\twolineshloka
{प्रमीयते चास्य प्रजा ह्यकालेसदा विवासं पितरोऽस्य कुर्वते}
{भीतं प्रपन्नं प्रददाति शत्रवेसेन्द्रा देवाः प्रहरन्त्यस्य वज्रम्}


\twolineshloka
{एतदेवं विजानन्वै न दास्यामि शचीमिमाम्}
{इन्द्राणीं विश्रुतां लोके शक्रस्य महिषीं प्रियां}


\threelineshloka
{अस्या हितं भवेद्यच्च मम चापि हितं भवेत्}
{क्रियतां तत्सुरश्रेष्ठा नहि दास्याम्यहं शचीम् ॥शल्य उवाच}
{}


\threelineshloka
{अथ देवाः सगन्धर्वा गुरुमाहुरिदं वचः}
{कथं सुनीतं नु भवेन्मन्त्रयस्व बृहस्पते ॥बृहस्पतिरुवाच}
{}


\twolineshloka
{नहुषं याचतां देवी किंचित्कालान्तरं शुभा}
{इन्द्राणि हितमेतद्धि तथाऽस्माकं भविष्यति}


\threelineshloka
{बहुविघ्नः सुराः कालः कालः कालं नयिष्यति}
{गर्वितो बलवांश्चापि नहुषो वरसंश्रयात् ॥शल्य उवाच}
{}


\twolineshloka
{ततस्तेन तथोक्ते तु प्रीता देवास्तथाब्रुवन्}
{ब्रह्मन्साध्विदमुक्तं ते हितं सर्वं दिवौकसाम्}


\twolineshloka
{एवमेतद्द्विजश्रेष्ठ देवी चेयं प्रसाद्यताम्}
{ततः समस्ता इन्द्राणीं देवाश्चाग्निपुरोगमाः}


\fourlineindentedshloka
{ऊचुर्वचनमव्यग्रा लोकानां हितकाम्यया}
{देवा ऊचुः}
{त्वया जगदिदं सर्वं धृतं स्थावरजङ्गमम्}
{एकपत्न्यसि सत्या च गच्छस्व नहुषं प्रति}


\twolineshloka
{क्षिप्रं त्वामभिकामश्च विनशिष्यति पापकृत्}
{नहुषो देवि शक्रश्च सुरैश्वर्यमवाप्स्यति}


\twolineshloka
{एवं विनिश्चयं कृत्वा इन्द्राणी कार्यसिद्धये}
{अभ्यगच्छत सव्रीडा नहुषं घोरदर्शनम्}


\twolineshloka
{दृष्ट्वा तां नहुषश्चापि वयोरूपसमन्विताम्}
{समहृष्यत दुष्टात्मा कामोपहतचेतनः}


\chapter{अध्यायः १३}
\twolineshloka
{शल्य उवाच}
{}


\twolineshloka
{अथ तामब्रवीद्दृष्ट्वा नहुषो देवराट् तदा}
{त्रयाणामपि लोकानामहमिन्द्रः शुचिस्मिते}


\twolineshloka
{भजस्व मां वरारोहे पतित्वे वरवर्णिनि}
{एवमुक्ता तु सा देवी नहुषेण पतिव्रता}


\twolineshloka
{प्रावेपत भयोद्विग्ना प्रवाते कदली यथा}
{प्रणम्य सा हि ब्रह्माणं शिरसा तु कृताञ्जलिः}


\twolineshloka
{देवराजमथोवाच नहुषं घोरदर्शनम्}
{कालमिच्छाम्यहं लब्धुं त्वत्तः कंचित्सुरेश्वर}


\twolineshloka
{न हि विज्ञायते शक्रः किं वा प्राप्तः क्व वा गतः}
{तत्त्वमेतत्तु विज्ञाय यदि न ज्ञायते प्रभो}


\threelineshloka
{ततोऽहं त्वामुपस्थास्ये सत्यमेतद्ब्रवीमि ते}
{एवमुक्तः स इन्द्राण्या नहुषः प्रीतिमानभूत् ॥नहुष उवाच}
{}


\twolineshloka
{एवं भवतु सुश्रोणि यथा मामिह भाषसे}
{ज्ञात्वा चागमनं कार्यं सत्यमेतदनुस्मरेः}


\twolineshloka
{नहुषेण विसृष्टा च निश्चक्राम ततः शुभा}
{बृहस्पतिनिकेतं च सा जगाम यशस्विनी}


\twolineshloka
{तस्याः संश्रुत्य च वचो देवाश्चाग्निपुरोगमाः}
{चिन्तयामासुरेकाग्राः शक्रार्थं राजसत्तम}


\twolineshloka
{देवदेवेन शङ्गम्य विष्णुना प्रभविष्णुना}
{ऊचुश्चैनं समुद्विग्ना वाक्यं वाक्यविशारदाः}


\twolineshloka
{ब्रह्मवध्याभिभूतो वै शक्रः सुरगणेश्वरः}
{गतिश्च नस्त्वं देवेश पूर्वजो जगतः प्रभुः}


\twolineshloka
{रक्षार्थं सर्वभूतानां विष्णुत्वमुपजग्मिवान्}
{त्वद्वीर्यनिहते वृत्रे वासवो ब्रह्महत्यया}


\twolineshloka
{वृतः सुरगणश्रेष्ठ मोक्षं तस्य विनिर्दिश}
{तेषां तद्वचनं श्रुत्वा देवानां विष्णुरब्रवीत्}


\twolineshloka
{मामेव यजतां शक्रः पावयिष्यामि वज्रिणम्}
{पुण्येन हयमेधेन मामिष्ट्वा पाकशासनः}


\twolineshloka
{पुनरेष्यति देवानामिन्द्रत्वमकुतोभयः}
{स्वकर्मभिश्च नहुषो नाशं यास्यति दुर्मतिः}


\threelineshloka
{किंचित्कालमिदं देवा मर्षयध्वमतन्द्रिताः}
{श्रुत्वा विष्णोः शुभां सत्यां वाणीं ताममृतोपमाम्}
{}


\twolineshloka
{ततः सर्वे सुरगणाः सोपाध्यायाः सहर्षिभिः}
{यत्र शक्रो भयोद्विग्नस्तं देशमुपचक्रमुः}


\twolineshloka
{तत्राश्वमेधः सुमहान्महेन्द्रस्य महात्मनः}
{ववृते पावनार्थं वै ब्रह्महत्यापहो नृप}


\twolineshloka
{विभज्य ब्रह्महत्यां तु वृक्षेषु च नदीषु च}
{पर्वतेषु पृथिव्यां च स्त्रीषु चैव युधिष्ठिर}


\twolineshloka
{संविभज्य च भूतेषु विसृज्य च सुरेश्वरः}
{विज्वरो धूतपाप्मा च वासवोऽभवदात्मवान्}


\twolineshloka
{अकम्प्यं नहुषं स्थानाद्दृष्ट्वा बलनिषूदनः}
{तेजोघ्नं सर्वभूतानां वरदानाच्च दुःसहम्}


\twolineshloka
{ततः शचीपतिर्देवः पुनरेव व्यनश्यत}
{अदृश्यः सर्वभूतानां कालाकाङ्क्षी चचार ह}


\twolineshloka
{प्रनष्टे तु ततः शक्रे शची शोकसमन्विता}
{हा शक्रेति तदा देवी विललाप सुदुःखिता}


\twolineshloka
{यदि दत्तं यदि हुतं गुरवस्तोषिता यदि}
{एकभर्तृत्वमेवास्तु सत्यं यद्यस्ति वा मयि}


\twolineshloka
{पुण्यां चेमामहं दिव्यां प्रवृत्तामुत्तरायणे}
{देवीं रात्रिं नमस्यामि सिध्यतां मे मनोरथः}


\twolineshloka
{प्रयता च निशां देवीमुपातिष्ठत तत्र सा}
{पतिव्रतात्वात्सत्येन सोपश्रुतिमथाकरोत्}


\twolineshloka
{यत्रास्ते देवराजोऽसौ तं देशं दर्शयस्व मे}
{इत्याहोपश्रुतिं देवीं सत्यं सत्येन दृश्यताम्}


\chapter{अध्यायः १४}
\twolineshloka
{शल्य उवाच}
{}


\twolineshloka
{अथैनां रूपिणी साध्वीमुपातिष्ठदुपश्रुतिः}
{तां वयोरूपसंपन्नां दृष्ट्वा देवीमुपस्थिताम्}


\threelineshloka
{इन्द्राणी संप्रहृष्टात्मा संपूज्यैनामथाब्रवीत्}
{इच्छामि त्वामहं ज्ञातुं का त्वं ब्रूहि वरानने ॥उपश्रुतिरुवाच}
{}


\twolineshloka
{उपश्रुतिरहं देवि तवान्तिकमुपागता}
{दर्शनं चैव संप्राप्ता तव सत्येन भामिनि}


\twolineshloka
{पतिव्रता च युक्ता च यमेन नियमेन च}
{दर्शयिष्यामि ते शक्रं देवं वृत्रनिषूदनम्}


\twolineshloka
{क्षिप्रमन्वेहि भद्रं ते द्रक्ष्यसे सुरसत्तमम्}
{ततस्तां प्रस्थितां देवीमिन्द्राणी सा समन्वगात्}


\twolineshloka
{देवारण्यान्यतिक्रम्य पर्वतांश्च बहूंस्ततः}
{हिमवन्तमतिक्रम्य उत्तरं पार्श्वमागमत्}


\twolineshloka
{समुद्रं च समासाद्य बहुयोजनविस्तृतम्}
{आससाद महाद्वीपं नानाद्रुमलतावृतम्}


\twolineshloka
{तत्रापश्यत्सरो दिव्यं नानाशकुनिभिर्वृतम्}
{शतयोजनविस्तीर्णं तावदेवायतं शुभम्}


\twolineshloka
{तत्र दिव्यानि पद्मानि पञ्चवर्णानि भारत}
{षट्पदैरुपगीतानि प्रफुल्लानि सहस्रशः}


\twolineshloka
{सरसस्तस्य मध्ये तु पद्मिनी महती शुभा}
{गौरेणोन्नतनालेन पद्मेन महता वृता}


\twolineshloka
{पद्मस्य भित्त्वा नालं च विवेश सहिता तया}
{बिसतन्तुप्रविष्टं च तत्रापश्यच्छतक्रतुम्}


\twolineshloka
{तं दृष्ट्वा च सुसूक्ष्मेण रूपेणावस्थितं प्रभुम्}
{सूक्ष्मरूपधरा देवी बभूवोपश्रुतिश्च सा}


\twolineshloka
{इन्द्रं तुष्टाव चेन्द्राणी विश्रुतैः पूर्वकर्मभिः}
{स्तूयमानस्ततो देवः शचीमाह पुरन्दरः}


\twolineshloka
{किमर्थमसि संप्राप्ता विज्ञातश्च कथं त्वहम्}
{ततः सा कथयामास नहुषस्य विचेष्टितम्}


\twolineshloka
{इन्द्रत्वं त्रिषु लोकेषु प्राप्य वीर्यसमन्वितः}
{दर्पाविष्टश्च दुष्टात्मा मामुवाच शतक्रतो}


\twolineshloka
{उपतिष्ठेति स क्रूरः कालं च कृतवान्मम}
{यदि न त्रास्यसि विभो करिष्यति स मां वशे}


\twolineshloka
{एतेन चाहं संप्राप्ता द्रुतं शक्र तवान्तिकम्}
{जहि रौद्रं महाबाहो नहुषं पापनिश्चयम्}


\twolineshloka
{प्रकाशयस्व चात्मानं दैत्यदानवसूदन}
{तेजः समाप्नुहि विभो देवराज्यं प्रशाधि च}


\chapter{अध्यायः १५}
\twolineshloka
{शल्य उवाच}
{}


\twolineshloka
{एवमुक्तः स भगवाञ्शच्या तां पुनरब्रवीत्}
{विक्रमस्य न कालोऽयं नहुषो बलवत्तरः}


\twolineshloka
{विवर्द्धितश्च ऋषिभिर्हव्यकव्यैश्च भामिनि}
{नीतिमत्र विधास्यामि देवि तां कर्तुमर्हसि}


\twolineshloka
{गुह्यं चैतत्त्वया कार्यं नाख्यातव्यं शुभे क्वचित्}
{गत्वा नहुषमेकान्ते ब्रवीहि च सुमध्यमे}


\twolineshloka
{ऋषियानेन दिव्येन मामुपैहि जगत्पते}
{एवं तव वशे प्रीता भविष्यामीति तं वद}


\twolineshloka
{इत्युक्ता देवराजेन पत्नी सा कमलेक्षणा}
{एवमस्त्वित्यथोक्त्वा तु जगाम नहुषं प्रति}


\twolineshloka
{नहुषस्तां ततो दृष्ट्वा सस्मितो वाक्यमब्रवीत्}
{स्वागतं ते वरारोहे किं करोमि शुचिस्मिते}


\twolineshloka
{भक्तं मां भज कल्याणि किमिच्छसि मनस्विनि}
{तव कल्याणि यत्कार्यं तत्करिष्ये सुमध्यमे}


\threelineshloka
{न च व्रीडा त्वया कार्या सुश्रोणि मयि विश्वसेः}
{सत्येन वै पशे देवि करिष्ये वचनं तव ॥इन्द्राण्युवाच}
{}


\twolineshloka
{यो मे कृतस्त्वया कालस्तमाकाङ्क्षे जगत्पते}
{ततस्त्वमेव भर्ता मे भविष्यसि सुराधिम}


\twolineshloka
{कार्यं च हृदि मे यत्तद्देवराजावधारय}
{वक्ष्यामि यदि मे राजन्प्रियमेतत्करिष्यसि}


\twolineshloka
{वाक्यं प्रणयसंयुक्तं ततः स्यां वशगा तव}
{इन्द्रस्य वाजिनो वाहा हस्तिनोऽथ रथास्तथा}


\twolineshloka
{इच्छाम्यहमथापूर्वं वाहनं ते सुराधिप}
{यन्न विष्णोर्न रुद्रस्य न सुराणां न रक्षसाम्}


\twolineshloka
{वहन्तु त्वां महाभागा ऋषयः सङ्गता विभो}
{सर्वे शिबिकया राजन्नेतद्धि मम रोचते}


\twolineshloka
{नासुरेषु न देवेषु तुल्यो भवितुमर्हसि}
{सर्वेषां तेज आदत्से स्वेन वीर्येण दर्शनात्}


% Check verse!
न ते प्रमुखतः स्थातुं कश्चिच्छक्नोति वीर्यवान्

शल्य उवाच

एवमुक्तस्तु नहुषः प्राहृष्यत तदा किल

उवाच वचनं चापि सुरेन्द्रस्तामनिन्दिताम् ॥नहुष उवाच


\twolineshloka
{अपूर्वं वाहनमिदं त्वयोक्तं वरवर्णिनि}
{दृढं मे रुचिरं देवि त्वद्वशोऽस्मि वरानने}


\twolineshloka
{न ह्यल्पवीर्यो भवति यो वाहान्कुरुते मुनीन्}
{अहं तपस्वी बलवान्भूतभव्यभवत्प्रभुः}


\twolineshloka
{मयि क्रुद्धे जगन्न स्यान्मयि सर्वं प्रतिष्ठितम्}
{देवदानवगन्धर्वाः किन्नरोरगराक्षसाः}


\twolineshloka
{न मे क्रुद्धस्य पर्याप्ताः सर्वे लोकाः शुचिस्मिते}
{चक्षुषा यं प्रपश्यामि तस्य तेजो हराम्यहम्}


\threelineshloka
{` अहमिन्द्रोऽस्मि देवानां लोकानां च महेश्वरः}
{मयि हव्यं च कव्यं च लोकाश्चैव सनातनाः}
{'तस्मात्ते वचनं देवि करिष्यामि न संशयः}


\threelineshloka
{सप्तर्षयो मां वक्ष्यन्ति सर्वे ब्रह्मर्षयस्तथा}
{पश्य माहात्म्ययोगं मे ऋद्धिं च वरवर्णिनि ॥शल्य उवाच}
{}


\twolineshloka
{एवमुक्त्वा तु तां देवीं विसृज्य च वराननाम्}
{5-15-22b` अथसंचिन्त्य नहुषो बलवीर्येण भारत}


\twolineshloka
{विसृज्य सुप्रतीकं च नागमैरावतं तथा}
{हंसयुक्तं विमानं च हरियुक्तं तथा रथम्}


\twolineshloka
{स तु दर्पेण महता परिभूय महामुनीन्}
{'विमाने योजयित्वा च ऋषीन्नियमामास्थितान्}


\twolineshloka
{अब्रह्मण्यो बलोपेतो मत्तो मदबलेन च}
{कामवृत्तः स दुष्टात्मा वाहयामास तानृषीन्}


\twolineshloka
{नहुषेण विसृष्टा च बृहस्पतिमथाब्रवीत्}
{समयोऽल्पावशेषो मे नहुषेणेह यः कृतः}


\twolineshloka
{शक्रं मृगय शीघ्रं त्वं भक्तायाः कुरु मे दयाम्}
{बाढमित्येव भगवान्बृहस्पतिरुवाच ताम्}


\twolineshloka
{न भेतव्यं त्वया देवि नहुषाद्दुष्टचेतसः}
{न ह्येष स्थास्यति चिरं गत एष नराधमः}


\twolineshloka
{अधर्मज्ञो महर्षीणां वाहनाच्च हतः शुभे}
{इष्टिं चाहं करिष्यामि विनाशायास्य दुर्मतेः}


\twolineshloka
{शक्रं चाधिगमिष्यामि माभैस्त्वं भद्रमस्तु ते}
{ततः प्रज्वाल्य विधिवज्जुहाव परमं हविः}


\twolineshloka
{बृहस्पतिर्महातेजा देवराजोपलब्धये}
{हुत्वाऽग्निं सोब्रवीद्राजञ्छक्र अन्विष्यतामिति}


\twolineshloka
{तस्माच्च भगवान्देवः स्वयमेव हुताशनः}
{स्त्रीवेषमद्भुतं कृत्वा तत्रैवान्तरधीयत}


\fourlineindentedshloka
{स दिशः प्रदिशश्चैव पर्वतांश्च वनानि च}
{पृथिवीं चान्तरिक्षं च विचित्याथ मनोगतिः}
{निमेषान्तरमात्रेण बृहस्पतिमुपागमत् ॥अग्निरुवाच}
{}


\twolineshloka
{बृहस्पते न पश्यामि देवराजमिह क्वचित्}
{आपः शेषाः सदा चापः प्रवेष्टुं नोत्सहाम्यहम्}


\threelineshloka
{न मे तत्र गतिर्ब्रह्मन्किमन्यत्करवाणि ते}
{तमब्रवीद्देवगुरुरपो विश महाद्युते ॥अग्निरुवाच}
{}


\twolineshloka
{नापः प्रवेष्टुं शक्ष्यामि क्षयो मेऽत्र भविष्यति}
{शरणं त्वां प्रपन्नोऽस्मि स्वस्ति तेस्तु महाद्युते}


\twolineshloka
{अद्य्भोऽग्निर्ब्रह्मतः क्षत्रमश्मनो लोहमुत्थितम्}
{तेषां सर्वत्रगं तेजः स्वासु योनिषु शाम्यति}


\chapter{अध्यायः १६}
\twolineshloka
{बृहस्पतिरुवाच}
{}


\twolineshloka
{त्वमग्रे सर्वदेवानां मुखं त्वमसि हव्यवाट्}
{त्वमन्तः सर्वभूतानां गूढश्चरसि साक्षिवत्}


\twolineshloka
{त्वामाहुरेकं कवयस्त्वामाहुस्त्रिविधं पुनः}
{त्वया त्यक्तं जगच्चेदं सद्यो नश्येद्धुताशन}


\twolineshloka
{कृत्वा तुभ्यं नमो विप्राः स्वकर्मविजितां गतिम्}
{गच्छन्ति सह पत्नीभिः सुतैरपि च शाश्वतीम्}


\twolineshloka
{त्वमेवाग्ने हव्यवाहस्त्वमेव परमं हविः}
{यजन्ति सत्रैस्त्वामेव यज्ञैश्च परमाध्वरैः}


\twolineshloka
{सृष्ट्वा लोकांस्त्रीमिमान्हव्यवाहप्राप्ते काले पचसि पुनः समिद्धः}
{त्वं सर्वस्य भुवनस्य प्रसूति-स्त्वमेवाग्रे भवसि पुनः प्रतिष्ठा}


\twolineshloka
{त्वामग्ने जलदानाहुर्विद्युतश्च मनीषिणः}
{वहन्ति सर्वभूतानि त्वत्तो निष्क्रम्य हेतयः}


\twolineshloka
{त्वय्यापो निहिताः सर्वास्त्वयि सर्वमिदं जगत्}
{न तेऽस्त्यविदितं किचित्रिषु लोकेषु पावक}


\twolineshloka
{स्वयोर्नि भजते सर्वो विशस्वापोऽविशङ्कितः}
{अहं त्वां वर्धयिष्यामि ब्राह्मैर्मन्त्रैः सनातनैः}


\fourlineindentedshloka
{एवं स्तुतो हव्यवाट् स भगवान्कविरुत्तमः}
{बृहस्पतिमथोवाच प्रीतिमान्वाक्यमुत्तमम्}
{दर्शयिष्यामि ते शक्रं सत्यमेतद्ब्रवीमि ते ॥शल्य उवाच}
{}


\twolineshloka
{प्रविष्यापस्ततो वह्निः ससमुद्राः सपल्वलाः}
{आससाद सरस्तच्च गूढो यत्र शतक्रतुः}


\twolineshloka
{अथ तत्रापि पद्मानि विचिन्वन्भरतर्षभ}
{अपश्यत्स तु देवेन्द्रं बिसमध्यगतं तदा}


\twolineshloka
{आगत्य च ततस्तूर्णं तमाचष्ट बृहस्पतेः}
{अणुमात्रेण वपुषा पद्मतन्त्वाश्रितं प्रभुम्}


\twolineshloka
{गत्वा देवर्षिगन्धर्वैः सहितोऽथ बृहस्पतिः}
{पुराणैः कर्मभिर्देवं तुष्टाव बलसूदनम्}


\twolineshloka
{महाऽसुरो हतः शक्र नमुचिर्दारुणस्त्वया}
{शम्बरश्च बलश्चैव तथोभौ घोरविक्रमौ}


\twolineshloka
{शतक्रतो विवर्धस्व सर्वाञ्शत्रून्निषूदय}
{उत्तिष्ठ शक्र संपश्य देवर्षीश्च समागतान्}


\threelineshloka
{महेन्द्र दानवान्हत्वा लोकास्त्रातास्त्वया विभो}
{अपां फेनं समासाद्य विष्णुतेजोतिबृंहितम्}
{त्वया वृत्रो हतः पूर्वं देवराज जगत्पते}


\twolineshloka
{त्वं सर्वभूतेषु शरण्य ईड्य-स्त्वया समं विद्यते नेह भूतम्}
{त्वया धार्यन्ते सर्वभूतानि शक्रत्वं देवानां महिमानं चकर्थ}


\twolineshloka
{पाहि सर्वांश्च लोकांश्च महेन्द्र बलमाप्नुहि}
{एवं संस्तूयमानश्च सोऽवर्धत शनैः शनैः}


\twolineshloka
{स्वं चैव वपुरास्थाय बभूव स बलान्वितः}
{अब्रवीच्च गुरं देवो बृहस्पतिमवस्थितम्}


\threelineshloka
{किं कार्यमवशिष्टं वो हतस्त्वाष्ट्रो महासुरः}
{वृत्रश्च सुमहाकायो यो वै लोकाननाशयत् ॥बृहस्पतिरुवाच}
{}


\threelineshloka
{मानुषो नहुषो राजा देवर्षिगणतेजसा}
{देवराज्यमनुप्राप्तः सर्वान्नो बाधते भृशम् ॥इन्द्र उवाच}
{}


\fourlineindentedshloka
{कथं च नहुषो राज्यं देवानां प्राप दुर्लभम्}
{तपसा केन वा युक्तः किंवीर्यो वा बृहस्पते}
{` तत्सर्वं कथय त्वं मे यथेन्द्रत्वमुपेयिवान् ॥बृहस्पतिरुवाच}
{}


\twolineshloka
{त्वयि प्रनष्टे देवेश विश्वं प्रव्यथितं जगत्}
{परस्परभयोद्विग्नं बभूवार्तमराजकम्}


\twolineshloka
{ततो देवैः सगन्धर्वैः सर्षिसङ्घैः सपावकैः}
{मानुषो नहुषो राजा देवराज्येऽभिषेचितः}


\twolineshloka
{देवा भीताः शक्रमकामयन्तत्वया त्यक्तं महदैन्द्रं पदं तत्}
{तदा देवाः पितरोऽथर्षयश्चगन्धर्वमुख्याश्च समेत्य सर्वे}


\twolineshloka
{गत्वाऽब्रुवन्नहुषं तत्र शक्रत्वं नो राजा भव भुवनस्य गोप्ता}
{तानब्रवीन्नहुषो नास्मि शक्तआप्यायध्वं तपसा तेजसा माम्}


\twolineshloka
{एवमुक्तैर्वर्द्धितश्चापि देवैराजाऽभवन्नहुषो घोरवीर्यः}
{त्रैलोक्ये च प्राप्य राज्यं महर्षी-न्कृत्वा वाहान्याति लोकान्दुरात्मा}


\threelineshloka
{तेजोहरं दृष्टिविषं सुघोरंमा त्वं पश्येर्नहुषं वै कदाचित्}
{देवाश्च सर्वे नहुषं भृशार्तान पश्यन्ते गूढरूपाश्चरन्तः ॥शल्य उवाच}
{}


\twolineshloka
{एवं वदत्यङ्गिरसां वरिष्ठेबृहस्पतौ लोकपालः कुबेरः}
{वैवस्वतश्चैव यमः पुराणोदवेश्च सोमो वरुणश्चाजगाम}


\twolineshloka
{ते वै समागम्य महेन्द्रमूचु-र्दिष्ट्या त्वाष्ट्रो निहतश्चैव वृत्रः}
{दिष्ट्या च त्वां कुशलिनमक्षतं चपश्यामो वै निहतारिं च शक्र}


\twolineshloka
{स तान्यथावच्च हि लोकपाला-न्समेत्स वै प्रीतमना महेन्द्रः}
{उवाच चैनान्प्रतिभाष्य शक्रःसंचोदयिष्यन्नहुषस्यान्तरेण}


\twolineshloka
{राजा देवानां नहुषो घोररूप-स्तत्र साह्यं दीयतां मे भवद्भिः}
{ते चाब्रुवन्नहुषो घोररूपोदृष्टीविषस्तस्य बिभीम ईश}


\twolineshloka
{त्वं चेद्राजानं नहुषं पराजये-स्ततो वयं भागमर्हाम शक्र}
{इन्द्रोऽब्रवीद्भवतु भवानपां पति-र्यमः कुबेरश्च मयाभिषेकम्}


\fourlineindentedshloka
{संप्राप्नुवन्त्वद्य सहैव दैवतैरिपुं जयाम तं नहुषं घोरदृष्टिम्}
{ततः शक्रं ज्वलनोऽप्याह भागप्रयच्छ मह्यं तव साह्यं करिष्ये}
{तमाह शक्रो भविताऽग्ने तवापिचेन्द्राग्न्योर्वै भाग एको महाक्रतौ ॥शल्य उवाच}
{}


\twolineshloka
{एवं संचिन्त्य भगवान्महेन्द्रः पाकशासनः}
{कुबेरं सर्वयक्षाणां धनानां च प्रभुं तथा}


\twolineshloka
{वैवस्वतं पितॄणां च वरुणं चाप्यपां तथा}
{आधिपत्यं ददौ शक्रः सत्कृत्य वरदस्तथा}


\chapter{अध्यायः १७}
\twolineshloka
{शल्य उवाच}
{}


\twolineshloka
{अथ संचिन्तयानस्य देवराजस्य धीमतः}
{नहुषस्य वधोपायं लोकपालैः सदैवतैः}


\twolineshloka
{तपस्वी तत्र भगवानगस्त्यः प्रत्यदृश्यत}
{सोऽब्रवीदर्च्य देवेन्द्रं दिष्ट्या वै वर्धते भवान्}


\twolineshloka
{विश्वरूपविनाशेन वृत्रासुरवधेन च}
{दिष्ट्याद्य नहुषो भ्रष्टो देवराज्यात्पुरन्दर}


% Check verse!
दिष्ट्या हतारिं पश्यामि भवन्तं वलसूदन

इन्द्र उवाच

स्वागतं ते महर्षेऽस्तु प्रीतोऽहं दर्शनात्तत

पाद्यमाचमनीयं च गामर्घ्यं च प्रतीच्छ मे ॥शल्य उवाच


\twolineshloka
{पूजितं चोपविष्टं तमासेन मुनिसत्तमम्}
{पर्यपृच्छत देवेशः प्रहृष्टो ब्राह्मणर्षभम्}


\threelineshloka
{श्रोतुमिच्छामि भगवन्कथ्यमानं द्विजोत्तम}
{परिभ्रष्टः कथं स्वर्गान्नहुषः पापनिश्चयः ॥अगस्त्य उवाच}
{}


\twolineshloka
{श्रृणु शक्र प्रियं वाक्यं यथा राजा दुरात्मवान्}
{स्वर्गाद्भ्रष्टो दुराचारो नहुषो बलदर्पितः}


\twolineshloka
{श्रमार्ताश्च वहन्तस्तं नहुषं पापकारिणम्}
{देवर्षयो महाभागास्तथा ब्रह्मर्षयोऽमलाः}


\twolineshloka
{पप्रच्छुर्नहुषं देवं संशयं जयतां वर}
{य इमे ब्रह्मणा प्रोक्ता मन्त्रा वै प्रोक्षणे गवाम्}


\threelineshloka
{एते प्रमाणं भवत उताहो नेति वासव}
{नहुषो नेति तानाह तमसा मूढचेतनः ॥ऋषय ऊचुः}
{}


\threelineshloka
{अधर्मे संप्रवृत्तस्त्वं धर्मं न प्रतिपद्यसे}
{प्रमाणमेतदस्माकं पूर्वं प्रोक्तं महर्षिभिः ॥अगस्त्य उवाच}
{}


\twolineshloka
{ततो विवदमानः स मुनिभिः सह वासव}
{अथ मामस्पृशन्मूर्ध्नि पादेनाधर्मपीडितः}


\twolineshloka
{तेनाभूद्धततेजाश्च निःश्रीकश्च महीपतिः}
{ततस्तं तमसा विग्नमवोचं भृशपीडितम्}


\twolineshloka
{यस्मात्पूर्वैः कृतं राजन्ब्रह्मर्षिभिरनुष्ठितम्}
{अदृष्टं दूषयसि मे यच्च मूर्ध्य्रस्पृशः पदा}


% Check verse!
यच्चापि त्वमृषीन्मूढ ब्रह्मकल्पान्दुरासदान्
\twolineshloka
{वाहान्कृत्व्रा वाहयसि तेन स्वर्गाद्धतप्रभः}
{ध्वंस पाप परिभ्रष्टः क्षीणपुण्यो महीतले}


\twolineshloka
{दशवर्षसहस्राणि सर्परूपधरो महान्}
{विचरिष्यसि पूर्णेषु पुनः स्वर्गमवाप्स्यसि}


\twolineshloka
{` दृष्ट्वा युधिष्ठिरं नाम तव वंशसमुद्भवम्}
{निहतो ब्रह्मशापेन प्रपद्यस्व त्रिविष्टपम् ॥'}


\twolineshloka
{एवं भ्रष्टो दुरात्मा स देवराज्यादरिन्दम}
{दिष्ट्या वर्धामहे शक्र हतो ब्रह्मर्षिकण्टकः}


\threelineshloka
{त्रिविष्टपं प्रपद्यस्व पाहि लोकाञ्शचीपते}
{जितेन्द्रियो जितामित्रः स्तूयमानो महर्षिभिः ॥शल्य उवाच}
{}


\twolineshloka
{ततो देवा भृशं तुष्टा महर्षिगणसंवृताः}
{पितरश्चैव यक्षाश्च भुजगा राक्षसास्तथा}


\twolineshloka
{गन्धर्वा देवकन्याश्च सर्वे चाप्सरसां गणाः}
{सरांसि सरितः शैलाः सागराश्च विशांपते}


\threelineshloka
{उपागम्याब्रुवन्सर्वे दिष्ट्या वर्धसि शक्रुहन्}
{हतश्च नहुषः पापो दिष्ट्यागस्त्येन धीमता}
{दिष्ट्या पापसमाचारः कृतः सर्पो महीतले}


\chapter{अध्यायः १८}
\twolineshloka
{शल्य उवाच}
{}


\twolineshloka
{ततः शक्रः स्तूयमानो गन्धर्वाप्सरसां गणैः}
{ऐरावतं समारुह्य द्विपेन्द्रं लक्षणैर्युतम्}


\twolineshloka
{पावकः सुमहातेजा महर्षिश्च बृहस्पतिः}
{यमश्च वरुणश्चैव कुबेरश्च धनेश्वरः}


\twolineshloka
{सर्वैर्देवैः परिवृतः शक्रो वृत्रनिषूदनः}
{गन्धर्वैरप्सरोभिश्च यातस्त्रिभुवनं प्रभुः}


\twolineshloka
{स समेत्य महेन्द्राण्या देवराजः शतक्रतुः}
{मुदा परमया युक्तः पालयामास देवराट्}


\twolineshloka
{ततः स भगवांस्तत्र अङ्गिराः समदृश्यत}
{अथर्ववेदमन्त्रैश्च देवेन्द्रं समपूजयत्}


\twolineshloka
{ततस्तु भगवानिन्द्रः संहृष्टः समपद्यत}
{वरं च प्रददौ तस्मै अथर्वाङ्गिरसे तदा}


\threelineshloka
{अथर्वाङ्गिरसं नाम वेदेऽस्मिन्वै भविष्यति}
{उदाहरणमेतद्धि यज्ञभागं च लप्स्यसे}
{}


\twolineshloka
{एवं संपूज्य भगवानथर्वाङ्गिरसं तदा}
{व्यसर्जयन्महाराज देवराजः शतक्रतुः}


\twolineshloka
{संपूज्य सर्वांस्त्रिदशानृषींश्चापि तपोधनान्}
{इन्द्रः प्रमुदितो राजन्धर्मेणापालयत्प्रजाः}


\twolineshloka
{एवं दुःखमनुप्राप्तमिन्द्रेण सह भार्यया}
{अज्ञातवासश्च कृतः शत्रूणां वधकाङ्क्षया}


\twolineshloka
{नात्र मन्युस्त्वया कार्यो यत्क्लिष्टोऽसि महावने}
{द्रौपद्या सह राजेन्द्र भ्रातृभिश्च महात्मभिः}


\twolineshloka
{एवं त्वमपि राजेन्द्र राज्यं प्राप्स्यसि भारत}
{वृत्रं हत्वा यथा प्राप्तः शक्रः कौरवनन्दन}


\twolineshloka
{दुराचारश्च नहुषो ब्रह्वद्विट् पापचेतनः}
{अगस्त्यशापाभिहतो विनष्टः शाश्वतीः समाः}


\twolineshloka
{एवं तव दुरात्मानः शत्रवः शत्रुसूदन}
{क्षिप्रं नाशं गमिष्यन्ति कर्णदुर्योधनादयः}


\twolineshloka
{ततः सागरपर्यन्तां भोक्ष्यसे मेदिनीमिमाम्}
{भ्रातृभिः सहितो वीर द्रौपद्या च सहानया}


\threelineshloka
{उपाख्यानमिदं शक्रविजयं वेदसंमितम्}
{राज्ञा व्यूढेष्वनीकेषु श्रोतव्यं जयमिच्छता}
{}


\twolineshloka
{तस्मात्संश्रावयामि त्वां विजयं जयतां वर}
{संस्तूयमाना वर्धन्ते महात्मानो युधिष्ठिर}


\twolineshloka
{क्षत्रियाणामभावोयं युधिष्ठिर महात्मनाम्}
{दुर्योधनापराधेन भीमार्जुनबलेन च ॥सङ्ग्रामे संक्षयो घोरो भविष्यत्यचिरादिव}


\twolineshloka
{आख्यानमिन्द्रविजयं य इदं नियतः पठेत्}
{धूतपाप्मा जितस्वर्गः परत्रेह च मोदते}


\fourlineindentedshloka
{न चारिजं भयं तस्य नापुत्रो वा भवेन्नरः}
{नापदं प्राप्नुयात्कांचिद्दीर्घमायुश्च विन्दति}
{सर्वत्र दयमाप्नोति न कदाचित्पराजयम् ॥वैशंपायन उवाच}
{}


\twolineshloka
{एवमाश्वासितो राजा शल्येन भरतर्षभ}
{पूजयामास विधिवच्छल्यं धर्मभृतां वरः}


\twolineshloka
{श्रुत्वा तु शल्यवचनं कुन्तीपुत्रो युधिष्ठिरः}
{प्रत्युवाच महाबाहुर्मद्रराजमिदं वचः}


\threelineshloka
{भवान्कर्णस्य सारथ्यं करिष्यति न संशयः}
{तत्र तेजोवधः कार्यः कर्णस्यार्जुनसंस्तवैः ॥शल्य उवाच}
{}


\threelineshloka
{एवमेतत्करिष्यामि यथा मां संप्रभाषसे}
{यच्चान्यदपि शक्ष्यामि तत्करिष्याम्यहं तव ॥वैशंपायन उवाच}
{}


\twolineshloka
{ततस्त्वामन्त्र्य कौन्तेयाञ्छल्यो मद्राधिपस्तदा}
{जगाम सबलः श्रीमान्दुर्योधनमरिन्दम}


\chapter{अध्यायः १९}
\twolineshloka
{वैशंपायन उवाच}
{}


\twolineshloka
{युयुधानस्ततो वीरः सात्वतानां महारथः}
{महता चतुरङ्गेण बलेनागाद्युधिष्ठिरम्}


\twolineshloka
{तस्य योधा महावीर्या नानादेशसमागताः}
{नानाप्रहरणा वीराः शोभयांचक्रिरे बलम्}


\twolineshloka
{परश्वथैर्भिण्डिपालैः शूलतोमरमुद्गरैः}
{परिघैर्यष्टिभिः पाशैः करवालैश्च निर्मलैः}


\twolineshloka
{खङ्गकार्मुकनिर्यूहैः शरैश्च विविधैरपि}
{तैलधौतैः प्रकाशद्भिस्तदशोभत वै बलम्}


\twolineshloka
{तस्य मेघप्रकाशस्य सौवर्णौः शोभितस्य च}
{बभूव रूपं सैन्यस्य मेघस्येव सविद्युतः}


\twolineshloka
{अक्षौहिणी तु सा सेना तदा यौधिष्ठिरं बलम्}
{प्रविश्यान्तर्दधे राजन्सागरं कुनदी यथा}


\twolineshloka
{तथैवाक्षौहिणीं गृह्य चेदीनामृषभो बली}
{धृष्टकेतुरुपागच्चत्पाण्डवानमितौजसः}


\twolineshloka
{मागधश्च जयत्सेनो जारासन्धिर्महाबलः}
{अक्षौहिण्यैव सैन्यस्य धर्मराजमुपागमत्}


\twolineshloka
{तथैव पाण्ड्यो राजेन्द्र सागरानूपवासिभिः}
{वृतो बहुविधैर्योधैर्युधिष्ठिरमुपागमत्}


\twolineshloka
{तस्य सैन्यमतीवासीत्तस्मिन्बलसमागमे}
{प्रेक्षणीयतरं राजन्सुवेषं बलवत्तदा}


\twolineshloka
{` केकयाश्च नरव्याघ्राः सोदराः पञ्च पार्थिवाः}
{संहर्षयन्तः कौन्तेयोनक्षौहिण्या समागताः ॥'}


\twolineshloka
{द्रुपदस्याप्यभूत्सेना नानादेशसमागतैः}
{शोभिता पुरुषैः शूरैः पुत्रैश्चास्य महारथैः}


\twolineshloka
{तथैव राजा मत्स्यानां विराटो वाहिनीपतिः}
{पार्वतीयैर्महीपालैः सहितः पाण्डवानयात्}


\twolineshloka
{इतश्चेतश्च पाण्डूनां समाजग्मुर्महात्मनाम्}
{अक्षौहिण्यस्तु सप्तैव विविधध्वजसङ्कुलाः}


\twolineshloka
{युयुत्समानाः कुरुभिः पाण्डवान्समहर्षयन्}
{तथैव धार्तराष्ट्रस्य हर्षं समभिर्धयन्}


\twolineshloka
{भगदत्तो महीपालः सेनामक्षौहिणीं ददौ}
{तस्य चीनैः किरातैश्च काञ्चनैरिव संवृतम्}


\twolineshloka
{बभौ बलमनाधृष्यं कर्णिकारवनं यथा}
{तथा भूरिश्रवाः शूरः शल्यश्च कुरुनन्दन}


\twolineshloka
{दुर्योधनमुपायातावक्षौहिण्या पृथक्पृथक्}
{कृतवर्मा च हार्दिक्यो भोजान्धकुकुरैः सह}


\twolineshloka
{अक्षौहिण्यैव सेनाया दुर्योधनमुपागमत्}
{तस्य तैः पुरुषव्याघ्रैर्वनमालाधरैर्बलम्}


\twolineshloka
{अशोभत यथा मत्तैर्वनं प्रक्रीडितैर्गजैः}
{जयद्रथमुखाश्चान्ये सिन्धुसौवीरवासिनः}


\twolineshloka
{आजग्मुः पृथिवीपालाः कम्पयन्त इवाचलान्}
{तेषामक्षौहिणी सेना बहुला विबभौ तदा}


\twolineshloka
{विधूयमानो वातेन बहुरूप इवाम्बुदः}
{सुदक्षिणश्च काम्भोजो यवनैश्च शकैस्तथा}


\twolineshloka
{उपाजगाम कौरव्यमक्षौहिण्या विशांपते}
{तस्य सेनासमावायः शलभानामिवाबभौ}


\twolineshloka
{स च संप्राप्य कौरव्यं तत्रैवान्तर्दधे तदा}
{तथा महिष्मतीवासी नीलो नीलायुधैः सह}


\twolineshloka
{महीपालो महावीर्यैर्दक्षिणापथवासिभिः}
{आवन्त्यौ च महीपालौ महाबलसुसंवृतौ}


\twolineshloka
{पृथगक्षौहिणीभ्यां तावभियातौ सुयोधनम्}
{ततस्ततस्तु सर्वेषां भूमिपानां महात्मनाम्}


\twolineshloka
{तिस्रोऽन्याः समवर्तन्त वाहिन्यो भरतर्षभ}
{एवमेकादशावृत्ताः सेना दुर्योधनस्य ताः}


\twolineshloka
{युयुत्समानाः कौन्तेयान्नानाध्वजसमाकुलाः}
{न हास्तिनपुरे राजन्नवकाशोऽभवत्तदा}


\twolineshloka
{राज्ञां स्वबलमुख्यानां प्राधान्येनापि भारत}
{ततः प़ञ्चनदं चैव कृत्स्नं च कुरुजाङ्गलम्}


\twolineshloka
{तथा रोहितकारण्यं मरुभूमिश्च केवला}
{अहिच्छत्रं कालकूटं गङ्गाकूलं च भारत}


\twolineshloka
{वारणं वाटधानं च यामुनश्चैव पर्वतः}
{एष देशः सुविस्तीर्णः प्रभूतधनधान्यवान्}


\twolineshloka
{बभूव कौरवेयाणां बलेनातीव संवृतः}
{तत्र सैन्यं तथा युक्तं ददर्श स पुरोहितः}


% Check verse!
यः स पाञ्चालराजेन प्रेषितः कौरवान्प्रति
\chapter{अध्यायः २०}
\twolineshloka
{वैशंपायन उवाच}
{}


\twolineshloka
{स च कौरव्यमासाद्य द्रुपदस्य पुरोहितः}
{सत्कृतो धृतराष्ट्रेण भीष्मेण विदुरेण च}


\twolineshloka
{सर्वकौशल्यमुक्त्वाऽऽदौ पृष्ट्वा चैवमनामयम्}
{सर्वसेनाप्रणेतॄणां मध्ये वाक्यमुवाच ह}


\twolineshloka
{सर्वैर्भवद्भिर्विदितो राजधर्मः सनातनः}
{वाक्योपादानहेतोस्तु वक्ष्यामि विदिते सति}


\twolineshloka
{धृतराष्ट्रश्च पाण्डुश्च सुतावेकस्य विश्रुतौ}
{तयोः समानं द्रविणं पैतृकं नात्र संशयः}


\twolineshloka
{धृतराष्ट्रस्य ये पुत्राः प्राप्तं तैः पैतृकं वसु}
{पाण्डुपुत्राः कथं नाम न प्राप्ताः पैतृकं वसु}


\twolineshloka
{वनं गतैः पाण्डवेयैर्विदितं वः पुरा यथा}
{न प्राप्तं पैतृकं द्रव्यं धृतराष्ट्रेण यद्धृतम्}


\twolineshloka
{प्राणान्तिकैरप्युपायैः प्रयतद्भिरनेकशः}
{शेषवन्तो न शकिता नेतुं वै यमसादनम्}


\twolineshloka
{पुनश्च वर्धितं राज्यं स्वबलेन महात्मभिः}
{छद्मनाऽपहृतं क्षुद्रैर्धार्तराष्ट्रैः ससौबलैः}


\twolineshloka
{तदप्यनुमतं कर्म यथा युक्तमनेन वै}
{वासिताश्च महारण्ये वर्षाणीह त्रयोदश}


\twolineshloka
{सभायां क्लेशितैर्वीरैः सहभार्यैस्तथा भृशम्}
{अरण्ये विविधाः क्लेशाः संप्राप्तास्तैः सुदारुणाः}


\twolineshloka
{तथा विराटनगरे योन्यन्तरगतैरिव}
{प्राप्तः परमसंक्लेशो यथा पापैर्महात्मभिः}


\twolineshloka
{ते सर्वं पृष्ठतः कृत्वा तत्पूर्वं कर्म किल्विषम्}
{साम्नैव कुरुभिः सन्धिमिच्छन्ति कुरुपुङ्गवाः}


\twolineshloka
{तेषां च वृत्तमाज्ञाय वृत्तं दुर्योधनस्य च}
{अनुनेतुमिहार्हन्ति धार्तराष्ट्रं सुहृज्जनाः}


\twolineshloka
{न हि ते विग्रहं वीराः कुर्वन्ति कुरुभिः सह}
{अविनाशेन लोकस्य काङ्क्षन्ते पाण्डवाः स्वकम्}


\twolineshloka
{यश्चापि धार्तराष्ट्रस्य हेतुः स्याद्विग्रहं प्रति}
{स च हेतुर्न मन्तव्यो बलीयांसस्तथा हि ते}


\twolineshloka
{अक्षौहिण्यश्च सप्तैव धर्मपुत्रस्य सङ्गताः}
{युयुत्समानाः कुरुभिः प्रतीक्षन्तेऽस्य शासनम्}


\twolineshloka
{अपरे पुरुषव्याघ्राः सहस्राक्षौहिणीसमाः}
{सात्यकिर्भीमसेनश्च यमौ च सुमहाबलौ}


\twolineshloka
{एकादशैताः पृतना एकतश्च समागताः}
{एकतश्च महाबाहुर्बहुरूपी धनञ्जयः}


\twolineshloka
{यथा किरीटी सर्वाभ्यः सेनाभ्यो व्यतिरिच्यते}
{एवमेव महाबाहुर्वासुदेवो महाद्युतिः}


\twolineshloka
{बहुलत्वं च सेनानां विक्रमं च किरीटिनः}
{बुद्धिमत्त्वं च कृष्णस्य बुद्ध्वा युध्येत को नरः}


\twolineshloka
{ते भवन्तो यथाधर्मं यथासमयमेव च}
{प्रयच्छन्तु प्रदातव्यं मा वः कालोऽत्यगादयम्}


\chapter{अध्यायः २१}
\twolineshloka
{वैशंपायन उवाच}
{}


\twolineshloka
{तस्य तद्वचनं श्रुत्वा प्रज्ञावृद्धो महाद्युतिः}
{संपूज्यैनं यथाकालं भीष्मो वचनमब्रवीत्}


\twolineshloka
{दिष्ट्या कुशलिनः सर्वे सह दामोदरेण ते}
{दिष्ट्या सहायवन्तश्च दिष्ट्या धर्मे च ते रताः}


\twolineshloka
{दिष्ट्या च सन्धिकामास्ते भ्रातरः कुरुनन्दनाः}
{दिष्टा न युद्धमनसः पाण्डवाः सह बान्धवैः}


\twolineshloka
{भवता सत्यमुक्तं तु सर्वमेतन्न संशयः}
{अतितीक्ष्णं तु ते वाक्यं ब्राह्मण्यादिति मे मतिः}


\twolineshloka
{असंशयं क्सेशितास्ते वने चेह च पाण्डवाः}
{प्राप्ताश्च धर्मतः सर्वं पितुर्धनमसंशयम्}


\twolineshloka
{किरीटि बलवान्पार्थः कृतास्त्रश्च महारथः}
{को हि पाण्डुसुतं युद्धे विषहेत धनञ्जयम्}


\threelineshloka
{अपि वज्रधरः साक्षात्किमुतान्ये धनुर्भृतः}
{त्रयाणामपि लोकानां समर्थ इति मे मतिः ॥वैशंपायन उवाच}
{}


\twolineshloka
{भीष्मे ब्रुवति तद्वाक्यं धृष्टमाक्षिप्य मन्युना}
{दुर्योधनं समालोक्य कर्णो वचनमब्रवीत्}


\twolineshloka
{न तत्राविदितं ब्रह्मँल्लोके भूतेन केनचित्}
{पुनरुक्तेन किं तेन भाषितेन पुनः पुनः}


\twolineshloka
{दुर्योधनार्थे शकुनिर्द्यूते निर्जितवान्पुरा}
{समयेन गतोऽरण्यं पाण्डुपुत्रो युधिष्ठिरः}


\twolineshloka
{स तं समयमाश्रित्य राज्यं नेच्छति पैतृकम्}
{बलमाश्रित्य मत्स्यानां पाञ्चालानां च मूर्खवत्}


\twolineshloka
{दुर्योधनो भयाद्विद्वन्न दद्यात्पादमन्ततः}
{धर्मतस्तु महीं कृत्स्नां प्रदद्याच्छत्रवेऽपि च}


\twolineshloka
{यदि काङ्क्षन्ति ते राज्यं पितृपैतामहं पुनः}
{यथाप्रतिज्ञं कालं तं चरन्तु वनमाश्रिताः}


\twolineshloka
{ततो दुर्योधनस्याङ्के वर्तन्तामकुतोभयाः}
{अधार्मिकीं तु मा बुद्धिं मौर्ख्यात्कुर्वन्तु केवलात्}


\threelineshloka
{अथ ते धर्ममुत्सृत्य युद्धमिच्छन्ति पाण्डवाः}
{आसाद्येमान्कुरुश्रेष्ठान्स्मरिष्यन्ति वचो मम ॥भीष्म उवाच}
{}


\threelineshloka
{किं नु राधेय वाचा ते कर्म तत्स्मर्तुमर्हसि}
{एक एव यदा पार्थः षड्रथाञ्जितवान्युधि ॥ 5-21-17a` विराटनगरेधीरः किं त्वं तत्रैव नागतः'}
{बहुशो जीयमानस्य कर्म दृष्टं तदैव ते}


\threelineshloka
{न चेदेवं करिष्यामो यदयं ब्राह्मणोऽब्रवीत्}
{ध्रुवं युधि हतास्तेन भक्षयिष्याम पांसुकान्}
{}


\threelineshloka
{` दुर्योधनः सहामात्यो विनङ्क्ष्यति न संशयः ॥वैशंपायन उवाच}
{धृतराष्ट्रस्ततो भीष्ममनुमान्य प्रसाद्य च}
{अवभर्त्स्य च राधेयमिदं वचनमब्रवीत्}


\twolineshloka
{अस्मद्धितं वाक्यमिदं भीष्मः शान्तनवोऽब्रवीत्}
{पाण्डवानां हितं चैव सर्वस्य जगतस्तथा}


\twolineshloka
{चिन्तयित्वा तु पार्थेभ्यः प्रेषयिष्यामि सञ्जयम्}
{स भवान्प्रतियात्वद्य पाण्डवानेव माचिरम्}


\twolineshloka
{स तं सत्कृत्य कौरव्यः प्रेषयामास पाण्डवान्}
{सभामध्ये समाहूय सञ्जयं वाक्यमब्रवीत्}


\chapter{अध्यायः २२}
\twolineshloka
{धृतराष्ट्र उवाच}
{}


\twolineshloka
{प्राप्तानाहुः सञ्जय पाण्डुपुत्रा-नुपप्लव्ये तान्विजानीहि गत्वा}
{अजातशत्रुं च सभाजयेथादिष्ट्या वनाद्ग्राममुपस्थितस्त्वम्}


\twolineshloka
{सर्वान्वदेः सञ्जय स्वस्तिमन्तःकृच्छ्रं वासमतदर्हा निरुष्य}
{तेषां शान्तिर्विद्यतेऽस्मासु शीघ्रंमिथ्यापेतानामुपकारिणां सताम्}


\twolineshloka
{नाहं क्वचित्सञ्जय पाण्डवानांमिथ्यावृत्तिं कांचन जात्वपश्यम्}
{सर्वां श्रियं ह्यात्मवीर्येण लब्धांपर्याकार्षुः पाण्डवा मह्यमेव}


\twolineshloka
{दोषं ह्येषां नाध्यगच्छं परीच्छ-न्सूक्ष्मं कंचिद्येन गर्हेय पार्थान्}
{धर्मार्थाभ्यां कर्म कुर्वन्ति नित्यंसुखप्रिये नानुरुध्यन्ति कामात्}


\twolineshloka
{धर्मं शीतं क्षुत्पिपासे तथैवनिद्रां तन्द्रीं क्रोधहर्षौ प्रमादम्}
{धृत्या चैव प्रज्ञया चाभिभूयधर्मार्थयोगान्प्रयतन्ति पार्थाः}


\threelineshloka
{त्यजन्ति मित्रेषु धनानि कालेन संवासाञ्जीर्यति तेषु मैत्री}
{यथार्हमानार्थकरा हि पार्था-स्तेषां द्वेष्टा नास्त्याजमीढस्य पक्षे}
{अन्यत्र पापाद्विपमान्मन्दबुद्धे-र्दुर्योधनात्क्षुद्रतराच्च कर्णात्}


\twolineshloka
{` पुत्रो मह्यं मृत्युवशं जगामदुर्योधनः सञ्जय रागबुद्धिः}
{'तेषां हीमौ हीनसुखप्रियाणांमहात्मनां संजनयतो हि तेजः}


\twolineshloka
{उत्थानवीर्यः सुखमेधमानोदुर्योधः सुकृतं मन्यते तत्}
{तेषां भागं यच्च मन्येत बालःशक्यं हर्तुं जीवतां पाण्डवानाम्}


\twolineshloka
{यस्यार्जुनः पदवीं केशवश्चवृकोदरः सात्यकोऽजातशत्रोः}
{माद्रीपुत्रौ सृञ्जयाश्चापि यान्तिपुरा युद्धात्साधु तस्य प्रदानम्}


\twolineshloka
{सह्येवैकः पृथिवीं सव्यसाचीगाण्डीवधन्वा प्रणुदेद्रथस्थः}
{तथा जिष्णुः केशवोऽप्यप्रधृष्योलोकत्रयस्याधिपतिर्महात्मा}


\twolineshloka
{तिष्ठेत कस्तस्य मर्त्यः पुरस्ता-द्यः सर्वलोकेषु वरेण्य एकः}
{पर्जन्यघोषान्प्रवपञ्शरौघा-न्पतङ्गसङ्घानिव शीघ्रवेगान्}


\twolineshloka
{दिशं ह्युदीचीमपि चोत्तरान्कुरून्गाण्डीवधन्वैकरथो जिगाय}
{धनं चैपामाहरत्सव्यसाचीसेनानुगान्द्रविडांश्चैव चक्रे}


\twolineshloka
{यश्चैव देवान्खाण्डवे सव्यसाचीगाण्डीवधन्वा प्रजिगाय सेन्द्रान्}
{उपाहरत्पाण्डवो जातवेदसेयशो मानं वर्धयन्पाण्डवानाम्}


\twolineshloka
{गदाभृतां नास्ति समोऽत्र भीमा-द्धस्त्यारोहो नास्ति समश्च तस्य}
{रथेऽर्जुनादाहुरहीनमेनंबाह्वोर्वलेनायुतनागवीर्यम्}


\twolineshloka
{सुशिक्षितः कृतवैरस्तरस्वीदहेत्क्षुद्रांस्तरसा धार्तराष्ट्रान्}
{सदाऽत्यमर्षी न वलात्स शक्योयुद्धे जेतुं वासवेनापि साक्षात्}


\twolineshloka
{सुतेजसौ वलिनौ शीघ्रहस्तौसुशिक्षितौ भ्रातरौ फाल्गुनेन}
{श्येनौ यथा पक्षिपूगान्रुजन्तौमाद्रीपुत्रौ शेपयेतां न शत्रून्}


\twolineshloka
{एतद्बलं पूर्णमस्माकमेवंयत्सत्यं तान्प्राप्य नास्तीति मन्ये}
{तेषां मध्ये वर्तमानस्तरस्वीधृष्टद्युम्नः पाण्डवानामिहैकः}


\twolineshloka
{सहामात्यः सोमकानां प्रबर्हःसन्त्यक्तात्मा पाण्डवार्थे श्रुतो मे}
{अजातशत्रुं प्रसहेत कोऽन्योयेषां स स्यादग्रणीर्वृष्णिसिंह}


\twolineshloka
{सहोषितश्चरितार्थो वयस्थोमात्स्येयानामधिपो वै विराटः}
{स वै सपुत्रः पाण्डवार्थे च शश्व-द्युधिष्ठिरे भक्त इति श्रुतं मे}


\twolineshloka
{अवरुद्धा रथिनः केकयेभ्योमहेष्वासा भ्रातरः पञ्च सन्ति}
{केकयेभ्यो राज्यमाकाङ्क्षमाणायुद्धार्थिनश्रानुवसन्ति पार्थान्}


\twolineshloka
{सर्वांश्च वीरान्पृथिवीपतीनांसमागतान्पाण्डवार्थे निविष्टान्}
{शूरानहं भक्तिमतः शृणोमिप्रीत्या युक्तान्संश्रितान्धर्मराजन्}


\twolineshloka
{गिर्याश्रया दुर्गनिवासिनश्चयोधाः पृथिव्यां कुलजातिशुद्धाः}
{म्लेच्छाश्च नानायुधवीर्यवन्तःसमागताः पाण्डवार्थे निविष्टाः}


\twolineshloka
{पाण्ड्यश्च राजा समितीन्द्रकल्पोयोधप्रवीरैर्बहुभिः समेतः}
{समागतः पाण्डवार्थे महात्मालोकप्रवीरोऽप्रतिवीर्यतेजाः}


\twolineshloka
{अस्त्रं द्रोणादर्जुनाद्वासुदेवा-त्कृपाद्भीष्माद्येन कृतं शृणोमि}
{यं तं कार्ष्णिप्रतिममाहुरेकंस सात्यकिः पाण्डवार्थे निविष्टः}


\twolineshloka
{उपाश्रिताश्चेदिकरूशकाश्चसर्वोद्योगैर्भूमिपालाः समेताः}
{तेषां मध्ये सूर्यमिवातपन्तंश्रिया वृतं चेदिपतिं ज्वलन्तम्}


\twolineshloka
{अस्तम्भनीयं युधि मन्यमान्योज्यां कर्षतां श्रेष्ठतमं पृथिव्याम्}
{सर्वोत्साहं क्षत्रियाणां निहत्यप्रसह्य कृष्णस्तरसा संममर्द}


\twolineshloka
{यशोमानौ वर्धयन्पाण्डवानांपुराऽभिनच्छिशुपालं समीक्ष्य}
{यस्य सर्वे वर्धयन्ति स्म मानंकरूशराजप्रमुखा नरेन्द्राः}


\twolineshloka
{तमसह्यं केशवं तत्र मत्वासुग्रीवयुक्तेन रथेन कृष्णम्}
{संप्राद्रवंश्चेदिपतिं विहायसिंहं दृष्ट्वा क्षुद्रमृगा इवान्ये}


\twolineshloka
{यस्तं प्रतीपस्तरसा प्रत्युदीया-दाशंसमानो द्वैरथे वासुदेवम्}
{सोऽशेत कृष्णेन हतः परासु-र्वातेनेवोन्मथितः कर्णिकारः}


\twolineshloka
{पराक्रमं मे यदवेदयन्ततेषामर्थे सञ्जय केशवस्य}
{अनुस्मरंस्तस्य कर्माणि विष्णो-र्गावल्गणे नाधिगच्छामि शान्तिम्}


\twolineshloka
{न जातु ताञ्छत्रुरन्यः सहेतयेषां स स्यादग्रणीर्वृष्णिसिंहः}
{प्रवेपते मे हृदयं भयेनश्रुत्वा कृष्णावेकरथे समेतौ}


\twolineshloka
{न चेद्गच्छेत्सङ्गरं मन्दबुद्धि-स्ताभ्यां लभेच्छर्म तदा सुतो मे}
{नो चेत्कुरून्सञ्जय निर्दहेता-मिन्द्राविष्णू दैत्यसेनां यथैव}


\twolineshloka
{मते हि मे शक्रसमो धनञ्जयःसनातनो वृष्णिवीरश्च विष्णुः}
{धर्मारामो ह्रीनिषेवस्तरस्वीकुन्तीपुत्रः पाण्डवोऽजातशत्रुः}


\twolineshloka
{दुर्योधनेन निकृतो मनस्वीनो चेत्क्रुद्धः प्रदहेद्धार्तराष्ट्रान्}
{नाहं तथा ह्यर्जुनाद्वासुदेवा-द्भीमाद्वाऽहं यमयोर्वा बिभेमि}


\twolineshloka
{यथा राज्ञः क्रोधदीप्तस्य सूतमन्योरहं भीततरः सदैव}
{महातपा ब्रह्मचर्येण युक्तःसङ्कल्पोऽयं मानसस्तस्य सिद्ध्येत्}


\twolineshloka
{तस्य क्रोधं सञ्जयाहं समीक्ष्यस्थाने जानन्भृशमस्म्यद्य भीतः}
{स गच्छ शीघ्रं प्रहितो रथेनपाञ्चालराजस्य चमूनिवेशनम्}


\twolineshloka
{अजातशत्रुं कुशलं स्म पृच्छेःपुनः पुनः प्रीतियुक्तं वदेस्त्वम्}
{जनार्दनं चापि समेत्य तातमहामात्रं वीर्यवतामुदारम्}


\twolineshloka
{अनामयं मद्वचनेन पृच्छे-र्धृतराष्ट्रः पाण्डवैः शान्तिमीप्सुः}
{न तस्य किंचिद्वचनं न कुर्यात्कुन्तीपुत्रो वासुदेवस्य सूत}


\twolineshloka
{प्रियश्चैषामात्मसमश्च कृष्णोविद्वांश्चैषां कर्मणि नित्ययुक्तः}
{समानीतान्पाण्डवान्सृञ्जयांश्चजनार्दनं युयुधानं विराटम्}


\twolineshloka
{अनामयं मद्वचनेन पृच्छेःसर्वांस्तथा द्रौपदेयांश्च पञ्च}
{यद्यत्तत्र प्राप्तकलं परेभ्य-स्त्वं मन्येथा भारतानां हितं चतद्भाषेथाः सञ्जय राजमध्येन मूर्च्छयेद्यन्न च युद्धहेतुः}


\chapter{अध्यायः २३}
\twolineshloka
{वैशंपायन उवाच}
{}


\twolineshloka
{राज्ञस्तु वचनं श्रुत्वा धृतराष्ट्रस्य सञ्जयः}
{उपप्लाव्यं ययौ द्रष्टुं पाण्डवानमितौजसः}


\twolineshloka
{स तु राजानमासाद्य कुन्तीपुत्रं युधिष्ठिरम्}
{अभिवाद्य ततः पूर्वं सूतपुत्रोऽभ्यभाषत}


\twolineshloka
{गवल्गणिः सञ्जयः सूतसूनु-रजातशत्रुमवदत्प्रतीतः}
{दिष्ट्या राजंस्त्वामरोगं प्रपश्येसहायवन्तं च महेन्द्रकल्पम्}


\twolineshloka
{अनामयं पृच्छति त्वाम्बिकेयोवृद्धो राजा धृतराष्ट्रो मनीषी}
{कच्चिद्भीमः कुशली पाण्डवाग्र्योधनञ्जयस्तौ च माद्रीतनूजौ}


\threelineshloka
{कच्चित्कृष्णा द्रौपदी राजपुत्रीसत्यव्रता वीरपत्नी सुपुत्रा}
{मनस्विनी यत्र च वाञ्छसि त्व-मिष्टान्कामान्भारत स्वस्तिकामः ॥युधिष्ठिर उवाच}
{}


\twolineshloka
{गावल्गणे सञ्जय स्वागतं तेप्रीतात्माऽहं त्वाऽभिनन्दामि सूत}
{अनामयं प्रतिमाने तवाहंसहानुजैः कुशली चास्मि विद्वन्}


\twolineshloka
{चिरादिदं कुशलं भारतस्यश्रुत्वा राज्ञः कुरुवृद्धस्य सूत}
{मन्ये साक्षाद्दृष्टमहं नरेन्द्रंदृष्ट्वैव त्वां सञ्जय प्रीतियोगात्}


\twolineshloka
{पितामहो नः स्थविरो मनस्वीमहाप्राज्ञः सर्वधर्मोपपन्नः}
{सकौरव्यः कुशली तात भीष्मोयथापूर्वं वृत्तिरस्त्यस्य कच्चित्}


\twolineshloka
{कच्चिद्राजा धृतराष्ट्रः सपुत्रोवैचित्रवीर्यः कुशली महात्मा}
{महाराजो बाह्लिकः प्रातिपेयःकच्चिद्विद्वान्कुशली सूतपुत्र}


\twolineshloka
{स सोमदत्तः कुशली तात कच्चि-द्भूरिश्रवाः सत्यसन्धः शलश्च}
{द्रोणः सपुत्रश्च कृपश्च विप्रोमहेष्वासाः कच्चिदेतेऽप्यरोगाः}


\twolineshloka
{सर्वे कुरुभ्यः स्पृहयन्ति सञ्जयधनुर्धरा ये पृथिव्यां प्रधानाः}
{महाप्रज्ञाः सर्वशास्त्रावदाताधनुर्भृतां मुख्यतमाः पृथिव्याम्}


\twolineshloka
{कच्चिन्मानं तात लभन्त एतेधनुर्भृतः कचिदेतेऽप्यरोगाः}
{येषां राष्ट्रे निवसति दर्शनीयोमहेष्वासः शीलवान्द्रोणपुत्रः}


\twolineshloka
{वैश्यापुत्रः कुशली तात कच्चित्महाप्राज्ञो राजपुत्रो युयुत्सुः}
{कर्णो मानी कुशली तात कच्चित्सुयोधनो यस्य मन्दो विधेयः}


\twolineshloka
{स्त्रियो वृद्धा भारतानां जनन्योमहादास्यो दासभार्याश्च सूत}
{वध्वः पुत्रा भागिनेया भगिन्योदौहित्रा वा कच्चिदप्यव्यलीकाः}


\twolineshloka
{कच्चिद्राजा ब्राह्मणानां यथावत्प्रवर्तते पूर्ववत्तात वृत्तिम्}
{कच्चिद्दायान्मामकान्धार्तराष्ट्रोद्विजातीनां सञ्जय नोपहन्ति}


\twolineshloka
{कच्चिद्राजा धृतराष्ट्रः सपुत्रउपेक्षते ब्राह्मणातिक्रमान्वै}
{स्वर्गस्य कच्चिन्न तथा वर्त्मभूता-मुपेक्षते तेषु सदैव वृत्तिम्}


\twolineshloka
{एतञ्ज्योतिश्चोत्तमं जीवलोकेशुक्लं प्रजानां विहितं विधात्रा}
{ते चेद्दोषं न नियच्छन्ति मन्दाःकृत्स्नो नाशो भविता कौरवाणाम्}


\twolineshloka
{कच्चिद्राजा धृतराष्ट्रः सपुत्रोबुभूषते वृत्तिममात्यवर्गे}
{कच्चिन्न भेदेन जिजीविषन्तिमुहृद्रूपा दुर्हृदश्चैकमत्यात्}


\twolineshloka
{कच्चिन्न पापं कथयन्ति तातते पाण्डवानां कुरवः सर्व एव}
{द्रोणः सपुत्रश्च कृपश्च वीरोनास्मासु पापानि वदन्ति कच्चित्}


\twolineshloka
{कच्चिद्राज्ये धृतराष्ट्रं सपुत्रंसमेत्याहुः कुरवः सर्व एव}
{कच्चिद्दृष्ट्वा दस्युसङ्घान्समेता-न्स्मरन्ति पार्थस्य युधां प्रणेतुः}


\twolineshloka
{मौर्वीभुजाग्रप्रहितान्स्म तातदोधूयमानेन धनुर्धरेण}
{गाण्डीवनुन्नांस्तनयित्नुघोषा-नजिह्मगान्कच्चिदनुस्मरन्ति}


\twolineshloka
{न चापश्यं कंचिदहं पृथिव्यांयोधं समं वाऽधिकमर्जुनेन}
{यस्यैकषष्टिर्निशितास्तीक्ष्णधाराःसुवाससः समंतो हस्तवापः}


\twolineshloka
{गदापाणिर्भीमसेनस्तरस्वीप्रवेपयञ्छत्रुसङ्घाननीके}
{नागः प्रभिन्न इव नङ्घलेषुचंक्रम्यते कच्चिदेनं स्मरन्ति}


\twolineshloka
{माद्रीपुत्रः सहदेवः कलिङ्गान्समागतानजयद्दन्तकूरे}
{वामेनास्यन्दक्षिणेनैव यो वैमहाबलं कच्चिदेनं स्मरन्ति}


\twolineshloka
{पुरा जेतुं नकुलः प्रेषितोऽयंशिबींस्त्रिगर्तान्सञ्जय पश्यतस्ते}
{दिशं प्रतीचीं वशमानयन्मेमाद्रीसुतं कच्चिदेनं स्मरन्ति}


\twolineshloka
{पराभवो द्वैतवने य आसी-द्दुर्मन्विते घोपयात्रागतानाम्}
{यत्र मन्दाञ्छत्रुवशं प्रयाता-नमोचयद्भीमसेनो जयश्च}


\twolineshloka
{अहं पश्चादर्जुनमभ्यरक्षंमाद्रीपुत्रौ भीमसेनोऽप्यरक्षत्}
{गाण्डीवधन्वा शत्रुसङ्घानुदस्यस्वस्त्यागमत्कच्चिदेनं स्मरन्ति}


\twolineshloka
{न कर्मणा साधुनैकेन नूनंसुखं शक्यं वै भवतीह सञ्जय}
{सर्वात्मना परिजेतुं वयं चे-न्न शक्नुमो धृतराष्ट्रस्य पुत्रम्}


\chapter{अध्यायः २४}
\twolineshloka
{सञ्जय उवाच}
{}


\twolineshloka
{यथाऽऽत्थ मे पाण्डव तत्तथैवकुरून्कुरुश्रेष्ठ जनं च पृच्छसि}
{अनामयास्तात मनस्विनस्तेकुरुश्रेष्ठान्पृच्छसि पार्थ यांस्त्वम्}


\twolineshloka
{सन्त्येव वृद्धाः साधवो धार्तराष्ट्रेसन्त्येव पापाः पाण्डव तस्य विद्धि}
{दद्याद्रिपुभ्योऽपि हि धार्तराष्ट्रःकुतो दायांल्लोपयेद्ब्राह्मणानाम्}


\twolineshloka
{यद्युष्मासु वर्ततेऽसावधर्म्य-मद्रुग्धेषु द्रुग्धवत्तन्न साधु}
{मित्रध्रुक् स्याद्धृतराष्ट्रस्य पुत्रोयुष्मान्द्विपन्साधुवृत्तानसाधुः}


\twolineshloka
{स चापि जानाति भृशं च तप्यतेशोचत्यन्तः स्थविरोऽजातशत्रो}
{शृणोति हि ब्राह्मणानां समेत्यमित्रद्रोहः पातकेभ्यो यरीयान्}


\twolineshloka
{स्मरन्ति तुभ्यं नरदेव सङ्गमेयुद्धे च जिष्णोश्च युधां प्रणेतुः}
{समुद्धुष्टे दुन्दुभिशङ्खशब्देगदापाणिं भीमसेनं स्मरन्ति}


\twolineshloka
{माद्रीसुतौ चापि तथाऽऽजिमध्येसर्वा दिशः संपतन्तौ स्मरन्ति}
{सेनां वर्षन्तौ शग्वैपरजमंमहारथौ समरे दुष्प्रकम्पौ}


\threelineshloka
{न त्वेव मन्ये पुरुषस्य राज-न्ननागतं ज्ञायते यद्भविष्यम्}
{त्वं चेत्तथा सर्वधर्मोपपन्नःप्राप्तः क्लेशं पाण्डव कृच्छ्ररूपम्}
{त्वमेवैतत्कृच्छ्रगतश्च भूयः समीकुर्याः प्रज्ञयाऽजातशत्रो}


\twolineshloka
{न कामार्थं सन्त्यजेयुर्हि धर्मंपाण्डोः सुताः सर्व एवेन्द्रकल्पाः}
{त्वमेवैतत्प्रज्ञयाऽजातशत्रोसमीकुर्या येन शर्माप्नुयुस्ते}


\twolineshloka
{धार्तराष्ट्राः पाण्डवाः सृञ्जयाश्चये चाप्यन्ते सन्निविष्टा नरेन्द्राः}
{यन्मां ब्रवीद्धृतराष्ट्रो निशाया-मजातशत्रो वचनं पिता ते}


% Check verse!
सहामात्यः सहपुत्रश्च राजन्समेत्य तां वाचमिमां निबोध
\chapter{अध्यायः २५}
\twolineshloka
{युधिष्ठिर उवाच}
{}


\threelineshloka
{समागताः पाण्डवाः सृञ्जयाश्चजनार्दनो युयुधानो विराटः}
{यत्ते वाक्यं धृतराष्ट्रानुशिष्टंगावल्गणे ब्रूहि तत्सूतपुत्र ॥संजय उवाच}
{}


\twolineshloka
{अजातशत्रुं च वृकोदरं चधनञ्जयं माद्रवतीसुतौ च}
{आमन्त्रये वासुदेवं च शौरिंयुयुधानं चेकितानं विराटम्}


\twolineshloka
{पञ्चालानामधिपं चैव वृद्धंधृष्टद्युम्नं पार्षतं याज्ञसेनिम्}
{सर्वे वाचं शृणुतेमां मदीयांवक्ष्यामि यां भूतिमिच्छन्कुरूणाम्}


\twolineshloka
{शमं राजा धृतराष्ट्रोऽभिनन्द-न्नयोजयत्त्वरमाणो रथं मे}
{सभ्रातृपुत्रस्वजनस्य राज्ञ-स्तद्रोचतां पाण्डवानां शमोऽस्तु}


\twolineshloka
{सर्वैर्धर्मैः समुपेतास्तु पार्थाःसंस्थानेन मार्दनेवार्जवेन}
{जाताः कुले ह्यनृशंसा वदान्याहीनिषेवाः कर्मणां निश्चयज्ञाः}


\twolineshloka
{न युज्यते कर्म युष्मासु हीनंसत्वं हि वस्तादृशं भीमसेनाः}
{उद्भासते ह्यञ्जनबिन्दुवत्त-च्छुभ्रे वस्त्रे यद्भवेत्किल्बिषं वः}


\twolineshloka
{सर्वक्षयो दृश्यते यत्र कृत्स्नःपापोदयो भावसंस्थः कुरूणाम्}
{कस्तत्र कुर्याञ्जातु कर्म प्रजानन्पराजयो यत्र समो जयश्च}


\twolineshloka
{ते वै धन्या यैः कृतं ज्ञातिकार्यंते वै पुत्राः सुहृदो बान्धवाश्च}
{उपक्रुष्टं जीवितं सन्त्यजेयु-र्यतः कुरूणां नियतो वै भवः स्यात्}


\twolineshloka
{ते चेत्कुरूननुशिष्याथ पार्थानिर्णीय सर्वान्द्विषतो निगृह्य}
{समं वस्तज्जीवितं मृत्युन स्या-द्यज्जीवध्वं ज्ञातिवधेन साधु}


\twolineshloka
{को ह्येव युष्मान्सह केशवेनसचेकितानान्पार्षतबाहुगुप्तान्}
{ससात्यकीन्विषहेत प्रजेतुंलब्ध्वाऽपि देवान्सचिवान्सहेन्द्रान्}


\twolineshloka
{को वा कुरून्द्रोणभीष्माभिगुप्ता-नश्वत्थाम्ना शल्यकृपादिभिश्च}
{रणे विजेतुं विषहेत राजन्राधेयगुप्तान्सह भूमिपालैः}


\twolineshloka
{महद्बलं धार्तराष्ट्रस्य राज्ञःको वै शक्तो हन्तुमक्षीयमाणः}
{सोऽहं जये चैव पराजये चनिःश्रेयसं नाधिगच्छामि किंचित्}


\twolineshloka
{कथं हि नीचा इव दौष्कुलेयानिर्धर्मार्थं कर्म कुर्युश्च पार्थाः}
{सोऽहं प्रसाद्य प्रणतो वासुदेवंपञ्चालानामधिपं चैव वृद्धम्}


\twolineshloka
{कृताञ्जलिः शरणं वः प्रपद्येकथं स्वस्ति स्यात्कुरुसृञ्जयानाम्}
{न ह्येवमेवं वचनं वासुदेवोधनञ्जयो वा जातु किंचिन्न कुर्यात्}


\twolineshloka
{प्रणान्दद्याद्याचमानः कुतोऽन्य-देतद्विद्वन्साधनार्थं ब्रवीमि}
{एतद्राज्ञो भीष्मपुरोगमस्यमतं यद्वः शान्तिरिहोत्तमा स्यात्}


\chapter{अध्यायः २६}
\twolineshloka
{युधिष्ठिर उवाच}
{}


\threelineshloka
{कां नु वाचं सञ्जय मे शृणोषि}
{युद्धैषिणीं येन युद्धाद्बिभेषि}
{अयुद्धं वै तात युद्धाद्गरीयःकस्तल्लब्ध्वा जातु युद्ध्येत सूत}


\twolineshloka
{अकुर्वतश्चेत्पुरुषस्य सञ्जयसिद्ध्येत्सङ्कल्पो मनसा यं यमिच्छेत्}
{न कर्म कुर्याद्विदितं ममैत-दन्यत्र युद्धाद्बहु यल्लघीयः}


\twolineshloka
{कुतो युद्धं जातु नरोऽवगच्छे-त्को देवशप्तो हि वृणीत युद्धम्}
{सुखैषिणः कर्म कुर्वन्ति पार्थाधर्मादहीनं यच्च लोकस्य पथ्यम्}


\twolineshloka
{धर्मोदयं सुखमाशंसमानाःकृच्छ्रोपायं तत्त्वतः कर्म दुःखम्}
{सुखं प्रेप्सुर्विजिघांसुश्च दुःखंकर्मारभेद्यच्च धर्मानपेतम्}


\twolineshloka
{क इन्द्रियाणां प्रीतिवशानुरानांकर्माभिज्ञः स्वशरीरं दुनोति}
{यया प्रमुक्तो न करोति दुःखंतृष्णां त्यजेत्सर्वधर्मादपेताम्}


\threelineshloka
{यथेध्यमानस्य समिद्धतेजसोभूयो बलं वर्धते पावकस्य}
{कामार्थलाभेन तथैव भूयोन तृप्यते सर्पिषेवाग्निरिद्धः}
{संपश्येमं भोगचयं महान्तंसहास्माभिर्धृतराष्ट्रस्य राज्ञः}


\twolineshloka
{नाश्रेयानीश्वरो विग्रहाणांनाश्रेयान्वै गीतशब्दं श्रृणोति}
{नाश्रेयान्वै सेवते माल्यगन्धा-न्न चाप्यश्रेयाननुलेपनानि ॥नाश्रेयान्वै प्रावारान्संविवस्तेकथं त्वस्मान्संप्रणुदेत्कुरुभ्यः}


\twolineshloka
{अत्रैव स्यादवुधस्यैव कामः}
{प्रायः शरीरे हृदयं दुनोति}


\twolineshloka
{स्वयं राजा विषमस्थः परेषुसामर्थ्यमन्विच्छति तन्न साधु}
{यथाऽऽत्मनः पश्यति वृत्तमेवतथा परेतामपि सोऽभ्युपैति}


\twolineshloka
{आसन्नमग्निं तु निदाघकालेगम्भीरकक्षे गहने विसृज्य}
{यथा विवृद्धं वायुवशेन शोचे-त्क्षेमं मुमुक्षुः शिशिरव्यपाये}


\twolineshloka
{प्राप्तैश्वर्यो धृतराष्ट्रोऽद्य राजालालप्यते सञ्जय कस्य हेतोः}
{प्रगृह्य दुर्बुद्धिमनार्जवे रतंपुत्रं मन्दं मूढममन्त्रिणं तु}


\twolineshloka
{अनाप्तवच्चाप्ततमस्य वाचःसुयोधनो विदुरस्यावमत्य}
{सुतस्य राजा धृतराष्ट्रः प्रियैषीसंबुध्यमानो विशतेऽधर्ममेव}


\twolineshloka
{मेधाविनं ह्यर्थकामं कुरूणांबहुश्रुतं वाग्मिनं शीलवन्तम्}
{स तं राजा धृतराष्ट्रः कुरुभ्योन सस्मार विदुरं पुत्रकाम्यात्}


\twolineshloka
{मानघ्नस्यासौ मानकामस्य चेर्षोःसंरम्भिणश्चार्थधर्मातिगस्य}
{दुर्भाषिणो मन्युवशानुगस्यकामात्मनो दौर्हृदैर्भावितस्य}


\twolineshloka
{अनेयस्याश्रेयसो दीर्घमन्यो-र्मित्रद्रुहः सञ्जय पापबुद्धेः}
{सुतस्य राजा धृतराष्ट्रः प्रियैषीप्रपश्यमानः प्राजहाद्धर्मकामौ}


\twolineshloka
{तदैव मे सञ्जय दीव्यतोऽभू-न्मतिः कुरूणामागतः स्यादभावः}
{काव्यां वाचं विदुरो भाषमाणोन विन्दते यद्धार्तराष्ट्रात्प्रशंसाम्}


\twolineshloka
{क्षत्तुर्यदा नान्ववर्तन्त बुद्धिंकृच्छ्रं कुरून्सूत तदाऽभ्याजगाम}
{यावत्प्रज्ञामन्ववर्तन्त तस्यतावत्तेषां राष्ट्रवृद्धिर्बभूव}


\twolineshloka
{तदर्थलुब्धस्य निबोध मेऽद्यये मन्त्रिणो धार्तराष्ट्रस्य सूत}
{दुःशासनः शकुनिः सूतपुत्रोगावल्गणे पश्य संमोहमस्य}


\twolineshloka
{सोऽहं न पश्यामि परीक्षमाणःकथं स्वस्ति स्यात्कुरुसृञ्जयानाम्}
{आत्तैश्वर्यो धृतराष्ट्रः परेभ्यःप्रव्राजिते विदुरे दीर्घदृष्टौ}


\twolineshloka
{आशंसते वै धृतराष्ट्रः सपुत्रोमहाराज्यमसपत्नं पृथिव्याम्}
{तस्मिञ्शमः केवलं नोपलभ्यःसर्वं स्वकं मद्गते मन्यतेऽर्थम्}


\twolineshloka
{यत्तत्कर्णो मन्यते पारणीयंयुद्धे गृहीतायुधमर्जुनं वै}
{आसंश्च युद्धानि पुरा महान्तिकथं कर्णो नाभवद्द्वीप एषाम्}


\twolineshloka
{कर्णश्च जानाति सुयोधनश्चद्रोणश्च जानाति पितामहश्च}
{अन्ये च ये कुग्वस्तत्र सन्तियथाऽर्जुनान्नाम्त्यपरो धनुर्धरः}


\twolineshloka
{जानन्त्येतन्कुरवः सर्व एवये चाप्यन्ये भूमिपालाः समेताः}
{दुर्योधने राज्यमिहाभवद्यथाअरिन्दमे फाल्गुनेऽविद्यमाने}


\twolineshloka
{तेनानुबन्धं मन्यते धार्तराष्ट्रःशक्यं हर्तुं पाण्डवानां ममत्वम्}
{किरीटिना तालमात्रायुधेनतद्वेदिना संयुगं तत्र गत्वा}


\twolineshloka
{गाण्डीवविष्फारितशब्दमात्रंश्रुत्वैव ते धार्तराष्ट्रा म्रियन्ते}
{क्रुद्धं न चेदीक्षते भीमसेनंसुयोधनो मन्यते सिद्धमर्थम्}


\twolineshloka
{इन्द्रोऽप्येतन्नोत्सहेत्तात हर्तु-मैश्वर्यं नो जीवति भीमसेने}
{धनञ्जये नकुले चैव सूततथा वीरे सहदेवेऽसहिष्णौ}


\twolineshloka
{सचेदेतां प्रतिपद्येत बुद्धिंवृद्धो राजा सह पुत्रेण सूत}
{एवं रणे पाण्डवकोपदग्धान नश्येयुः सञ्जय धार्तराष्ट्राः}


\twolineshloka
{जानामि त्वं क्लेशमस्मासु वृत्तंत्वां पूजयन्मञ्जयाहं क्षमेयम्}
{वच्चास्माकं कौरवैर्भृतपूर्वंया नो वृत्तिधार्तराष्ट्रे तदाऽऽसीत्}


\twolineshloka
{अद्यापि सा तत्र तथैव वर्ततांयान्ति गमिष्यामि यथा न्वमात्थ}
{द्वन्द्राप्रव्ये भवतु ममैव राज्यंसूयोधनो यच्छतु भाग्नाग्र्यः}


\chapter{अध्यायः २७}
\twolineshloka
{सञ्जय उवाच}
{}


\twolineshloka
{धर्मनित्या पाण्डव ते विचेष्टालोके श्रुता दृश्यते चापि पार्थ}
{महाश्रावं जीवित चाप्यनित्यंसंपश्य त्वं पाण्डव मा व्यनीनशः}


\twolineshloka
{नचेद्भागं कुरवोऽन्यत्र युद्धा-त्प्रयच्छेरंस्तुभ्यमजातशत्रो}
{भैक्षचर्यामन्धकवृष्णिराज्येश्रेयो मन्ये न तु युद्धेन राज्यम्}


\twolineshloka
{अल्पकालं जीवितं यन्मनुष्येमहास्रावं नित्यदुःखं चलं च}
{भूयश्च तद्यशसो नानुरूपंतस्मात्पापं पाण्डव मा कृथास्त्वम्}


\twolineshloka
{कामा मनुष्यं प्रसजन्त एतेधर्मस्य ये विघ्नमूलं नरेन्द्र}
{पूर्वं नरस्तान्मतिमान्प्रणिघ्नन्लोके प्रशंसां लभतेऽनवद्याम्}


\twolineshloka
{निबन्धनी ह्यर्थतृष्णेह पार्थतामिच्छतां बाध्यते धर्म एव}
{धर्मं तु यः प्रणृणीते स बुद्धःकामे गृध्नो हीयतेऽर्थानुरोधात्}


\twolineshloka
{धर्मं कृत्वा कर्मणां तात मुख्यंमहाप्रतापः सवितेव भाति}
{हीनो हि धर्मेण महीमपीमांलब्ध्वा नरः सीदति पापबुद्धिः}


\twolineshloka
{वेदोऽधीतश्चरितं ब्रह्मचर्यंयज्ञैरिष्टं ब्राह्मणेभ्यश्च दत्तम्}
{परं स्थानं मन्यमानेन भूयआत्मा दत्तो वर्षपूगं सुखेभ्यः}


\twolineshloka
{सुखप्रिये सेवमानोऽतिवेलंयोगाभ्यासे यो न करोति कर्म}
{वित्तक्षये हीनसुखोऽतिवेलंदुःखं शेते कामवेगप्रणुन्नः}


\twolineshloka
{एवं पुनर्ब्रह्मचर्याऽप्रसक्तोहित्वा धर्मं यः प्रकरोत्यधर्मम्}
{अश्रद्दधत्परलोकाय मूढोहित्वा देहं तप्यते प्रेत्य मन्दः}


\twolineshloka
{न कर्मणां विप्रणाशोऽस्त्यमुत्रपुण्यानां वाऽप्यथवा पापकानाम्}
{पूर्वं कर्तुर्गच्छति पुण्यपापंपश्चात्त्वेनमनुयात्येव कर्ता}


\twolineshloka
{न्यायोपेतं ब्राह्मणेभ्योऽथ दत्तंश्रद्धापूतं गन्धरसोपपन्नम्}
{अन्वाहार्येषूत्तमदक्षिणेषुतथारूपं कर्म विख्यायते ते}


\twolineshloka
{इह क्षेत्रे क्रियते पार्थ कार्यंन वै किंचित्क्रियते प्रेत्य कार्यम्}
{कृतं त्वया पारलोक्यं च कर्मपुण्यं महत्सद्भिरतिप्रशस्तम्}


\twolineshloka
{जहाति मृत्युं च जरां भयं चन क्षुत्पिपासे मनसोऽप्रियाणि}
{न कर्तव्यं विद्यते तत्र किंचि-दन्यत्र वै चेन्द्रियप्रीणनाद्धि}


\twolineshloka
{एवंरूपं कर्मफलं नरेन्द्रमातापित्रोर्हृदयस्याप्रियेण}
{त्यज क्रोधं पाण्डव हर्षजं चलोकावुभौ मा प्रहासीश्चिराया}


\twolineshloka
{अन्तं गत्वा कर्मणां या प्रशंसासत्यं दमं चार्जवमानृशंस्यम्}
{अश्वमेधं राजसूयं तथेष्ट्वापापस्यान्तं कर्मणो मा पुनर्गाः}


\twolineshloka
{तच्चेदेवं द्वेषरूपेण पार्थाःकरिष्यध्वं कर्म पापं चिराय}
{निवसध्वं वर्षपूगान्वनेषुदुःखं वासं पाण्डवा धर्म एव}


\twolineshloka
{प्रव्रज्यया यातयित्वा पुरस्ता-दात्माधीनं यद्बलं ते तदाऽसीत्}
{नित्यं च वश्याः सचिवास्तवेमेजनार्दनो युयुधाश्च वीरः}


\twolineshloka
{मत्स्यो राजा रुक्मरथः सपुत्रःप्रहारिभिः सहपुत्रैर्विराटः}
{राजानस्ते ये विजिताः पुरस्ता-त्त्वामेव ते संश्रयेयुः समस्ताः}


\twolineshloka
{महासहायः प्रतपन्बलस्थःपुरस्कृतो वासुदेवार्जुनाभ्याम्}
{वरान्हनिष्यन्द्विषतो रङ्गमध्येव्यनेष्यथा धार्तराष्ट्रस्य दर्पम्}


\twolineshloka
{बलं कस्माद्वर्धयित्वा परस्यनिजान्कस्मात्कर्शयित्वा सहायान्}
{निरुष्य कस्माद्वर्षपूगान्वनेषुयुयुत्ससे पाण्डव हीनकाले}


\twolineshloka
{अप्राज्ञो वा पाण्डव युद्ध्यमानो-ऽधर्मज्ञो वा भूतिमथोऽभ्युपैति}
{प्रज्ञावान्वा बुद्ध्यमानोऽपि धर्मंसंस्तम्भाद्वा सोऽपि भूतेरपैति}


\twolineshloka
{नाधर्मे ते धीयते पार्थ बुद्धि-र्न संरम्भात्कर्म चकर्थ पापम्}
{आत्थ किं तत्कारणं यस्य हेतोःप्रज्ञाविरुद्धं कर्म चिकीर्षसीदम्}


\twolineshloka
{अव्याधिजं कटुकं शीर्षरोगियशोमुषं पापफलोदयं वा}
{सतां पेयं यन्न पिबन्त्यसन्तोमन्युं महाराज पिब प्रशाम्य}


\twolineshloka
{पापानुबन्धं को नु तं कामयेतक्षमैव ते ज्यायसी नोत भोगाः}
{यत्र भीष्मः शान्तनवो हतः स्या-द्यत्र द्रोणः सहपुत्रो हतः स्यात्}


\twolineshloka
{कृपः शल्यः सौमदत्तिर्विकर्णोविविंशतिः कर्णदुर्योधनौ च}
{एतान्हत्वा कीदृशं तत्सुखं स्या-द्यद्विन्देथास्तदनुब्रूहि पार्थ}


\twolineshloka
{लब्ध्वाऽपीमां पृथिवीं सागरान्तांजरामृत्यू नैव हि त्वं प्रजह्याः}
{प्रियाप्रिये सुखदुःखे च राज-न्नैवं विद्वान्नैव युद्धं कुरु त्वम्}


\twolineshloka
{अमात्यानां यदि कामस्य हेतो-रेवं युक्तं कर्म चिकीर्षसि त्वम्}
{अपक्रमेः स्वं प्रदायैव तेषांमागास्त्वं वै देवयानात्पथोऽद्य}


\chapter{अध्यायः २८}
\twolineshloka
{युधिष्ठिर उवाच}
{}


\twolineshloka
{असंशयं सञ्जय सत्यमेत-द्धर्मो वरः कर्मणां यत्त्वमात्थ}
{ज्ञात्वा तु मां सञ्जय गर्हयेस्त्वंयदि धर्मं यद्यधर्मं चरामि}


\twolineshloka
{यत्राधर्मो धर्मरूपाणि धत्तेधर्मः कृत्स्नो दृश्यतोऽधर्मरूपः}
{बिभ्रद्धर्मो धर्मरूपं तथा चविद्वांसस्तं संप्रपश्यन्ति बुद्ध्या}


\twolineshloka
{एवं तथैवापदि लिङ्गमेत-द्धर्माधर्मौ नित्यवृत्ती भजेताम्}
{आद्यं लिङ्गं यस्य तस्य प्रमाण-मापद्धर्मं सञ्जय तं निबोध}


\twolineshloka
{लुप्तायां तु प्रकृतौ येन कर्मनिष्पादयेत्तत्परीप्सेद्विहीनः}
{प्रकृतिस्थश्चापदि वर्तमानउभौ गर्ह्यौ भवतः सञ्जयैतौ}


\twolineshloka
{अविनाशमिच्छतां ब्राह्मणानांप्रायश्चित्तं विहितं यद्विधात्रा}
{संपश्येथाः कर्मसु वर्तमानान्विकर्मस्थान्सञ्जय गर्हयेस्त्वम्}


\twolineshloka
{मनीषिणां सत्त्विच्छेदनायविधियते सत्सु वृत्तिः सदैव}
{अब्राह्मणाः सन्ति तु ये न वैद्याःसर्वोत्सङ्गं साधु मन्येत तेभ्यः}


\twolineshloka
{तदध्वानः पितरो ये च पूर्वेपितामहा ये च तेभ्यः परेऽन्ये}
{यज्ञैषिणो ये च हि कर्म कुर्यु-र्नान्यं ततो नास्तिकोऽस्मीति मन्ये}


\twolineshloka
{यत्किंचनेदं वित्तमस्यां पृथिव्यांयद्देवानां त्रिदशानां परं यत्}
{प्राजापत्यं त्रिदिवं ब्रह्मलोकंनाधर्मतः सञ्जय कामयेयम्}


\twolineshloka
{धर्मेश्वरः कुशलो नीतिमांश्चा-प्युपासिता ब्राह्मणानां मनीषी}
{नानाविधांश्चैव महाबलांश्चराजन्यभोजाननुशास्ति कृष्णः}


\twolineshloka
{यदि ह्यहं विसृजत्साम गर्ह्योनियुध्यमानो यदि जह्यां स्वधर्मम्}
{महायशाः केशवस्तद्ब्रवीतुवासुदेवस्तूभयोरर्थकामः}


\twolineshloka
{शैनेयोऽयं चेदयश्चान्धकाश्चवार्ष्णेयभोजाः कुकुराः सृञ्जयाश्च}
{उपासीना वासुदेवस्य बुद्धिंनिगृह्य शत्रून्सुहृदो नन्दयन्ति}


\twolineshloka
{वृष्ण्यन्धका ह्युग्रसेनादयो वैकृष्णप्रणीताः सर्व एवेन्द्रकल्पाः}
{मनस्विनः सत्यपरायणाश्चमहाबला यादवा भोगवन्तः}


\twolineshloka
{काश्यो बभ्रुः श्रियमुत्तमां गतोलब्ध्वा कृष्णं भ्रातरमीशितारम्}
{यस्मै कामान्वर्षति वासुदेवोग्रीष्मात्यये मेघ इव प्रजाऽभ्यः}


\twolineshloka
{ईदृशोऽयं केशवस्तात विद्वान्विद्धि ह्येनं कर्मणां निश्चयज्ञम्}
{प्रियश्च नः साधुतमश्च कृष्णोनातिक्रामे वचनं केशवस्य}


\chapter{अध्यायः २९}
\twolineshloka
{वासुदेव उवाच}
{}


\twolineshloka
{अविनाशं सञ्जय पाण्डवाना-मिच्छाम्यहं भूतिमेषां प्रियं च}
{तथा राज्ञो धृतराष्ट्रस्य सूतसमाशंसे बहुपुत्रस्य वृद्धिम्}


\twolineshloka
{कामो हि मे सञ्जय नित्यमेवनान्यद्ब्रूयां तान्प्रति शाम्यतेति}
{राज्ञश्च हि प्रियमेतच्छृणोमिमन्ये चैतत्पाण्डवानां समक्षम्}


\twolineshloka
{सुदुष्करस्तत्र शमो हि नूनंप्रदर्शितः सञ्जय पाण्डवेन}
{यस्मिन्गृद्धो धृतराष्ट्रः सपुत्रःकस्मादेषां कलहो नावमूर्च्छेत्}


% Check verse!
न त्वं धर्मं विचरं सञ्जयेहमत्तश्च जानामि युधिष्ठिराच्चअथो कस्मात्सज्जय पाण्डवस्यउत्साहिनः पूरयतः स्वकर्म
\twolineshloka
{यथाख्यातमावमतः कुटुम्बेपुरा कस्मान्साधुविलोपमात्थ}
{अस्मिन्विधौ वर्तमाने यथाव-दुच्चावचा मतयो ब्राह्मणानाम्}


\twolineshloka
{कर्मणाऽऽहुः सिद्धिमेके परत्रहिन्वा कर्म विद्यया सिद्धिमेके}
{नाभुञ्जानो भक्ष्यभोज्यस्य तृप्ये-द्विद्वानपीह विहितं ब्राह्मणानाम्}


\twolineshloka
{या वै विद्याः साधयन्तीह कर्मतासां फलं विद्यते नेतरासाम्}
{तत्रेह वै दृष्टफलं तु कर्मपीत्वोदकं शाम्यति तृष्णयाऽऽर्तः}


\twolineshloka
{सोऽयं विधिर्विहितः कर्मणैवसंवर्तते सञ्जय तत्र कर्म}
{तत्र योऽन्यत्कर्मणः साधु मन्ये-न्मोघं तस्यालपितं दुर्बलस्य}


\twolineshloka
{कर्मणाऽमी भान्ति देवाः परत्रकर्मणैवेह प्लवते मातरिश्वा}
{अहोरात्रे विदधत्कर्मणैवअतन्द्रितो नित्यमुदेति सूर्यः}


\twolineshloka
{मासार्धमासानथ नक्षत्रयोगा-नतन्द्रितश्चन्द्रमाश्चाभ्युपैति}
{अतन्द्रितो दहते जातवेदाःसमिद्ध्यमानः कर्म कुर्वन्प्रजाभ्यः}


\twolineshloka
{अतन्द्रिता भारमिमं महान्तंबिभर्ति देवी पृथिवी वलेन}
{अतन्द्रिताः शीघ्रसपो वहन्तिसन्तर्पयन्त्यः सर्वभूतानि नद्यः}


\twolineshloka
{अतन्द्रितो वर्षति भूरितेजाःसन्नादयन्नन्तरिक्षं दिशश्च}
{अतन्द्रितो ब्रह्मचर्यं चचारश्रेष्ठत्वमिच्छन्बलभिद्देवतानाम्}


\twolineshloka
{हित्वा सुखं मनसश्च प्रियाणितेन शक्रः कर्मणा श्रैष्ठ्यमाप}
{सत्यं धर्मं पालयन्नप्रमत्तोदमं तितिक्षां समतां प्रियं च}


\twolineshloka
{एतानि सर्वाण्युपसेवमानोयो देवराज्यं मघवान्प्राप मुख्यम्}
{बृहस्पतिर्ब्रह्मचर्यं चचारसमीहितः संशिमात्मा यथावत्}


\twolineshloka
{हित्वा सुख प्रतिरुध्येन्द्रियाणितेन देवानामगमद्गौरवं सः}
{तथा नक्षत्राणि कर्मणाऽमुत्र भान्तिरुद्रादित्या वसवोऽथापि विश्वे}


\twolineshloka
{यमो राजा वैश्रवणः कुबेरोगन्धर्वयक्षाप्सरसश्च सूत}
{ब्रह्मविद्यां ब्रह्मचर्यं क्रियां चनिपेवमाणा ऋषयोऽमुत्र भान्ति}


\twolineshloka
{जानन्निमं सर्वलोकस्य धर्मंविप्रेन्द्राणां क्षत्रियाणां विशां च}
{स कस्मात्त्वं जानतां ज्ञानवान्सन्व्यायच्छसेक सञ्जय कौरवार्थे}


\twolineshloka
{आम्नायेषु नित्यसंयोगमस्यतथाऽश्वमेधे राजसूये च विद्धि}
{संयुज्यते धनुपा वर्मणा चहस्त्यश्वाद्यै रथशस्त्रैश्च भूयः}


\twolineshloka
{ते चेदिमे कौरवाणाम्रुपाय-मवगच्छेयुवधेनैव पार्थाः}
{धर्मत्राणं पुण्यमेषां कृतं स्या-दार्ये वृत्ते भीमसेनं निगृह्य}


\twolineshloka
{ते चेत्पित्र्ये कर्मणि वर्तमानाआपद्येरन्दिष्टवशेन मृत्युम्}
{यथाशक्त्यापूरयन्तः स्वकर्मतदप्येषां निधनं स्यात्प्रशस्तम्}


\twolineshloka
{उताहो त्वं मन्यसे शाम्यमेवराज्ञां युद्धे वर्तते धर्मतन्त्रम्}
{अयुद्धे वा वर्तते धर्मतन्त्रंतथैव ते वाचमिमां श्रृणोमि}


\twolineshloka
{चातुर्वर्ण्यस्य प्रथमं संविभाग-मवेक्ष्य त्वं सञ्जय स्वं च कर्म}
{निशम्याथो पाण्डवानां च कर्मप्रशंस वा निन्द वा या मतिस्ते}


\twolineshloka
{अधीयीत ब्राह्मणो वै यजेतदद्यादीयात्तीर्थमुख्यानि चैव}
{अध्याप्रयेद्याजयेच्चापि याज्यान्प्रतिग्रहान्वा विहितान्प्रतीच्छेत्}


\twolineshloka
{` अधीयीत क्षत्रियोऽथो यजेतदद्याद्धनं न तु याचेत किंचित्}
{न याजयेन्न तु चाध्यापयीतएवं स्मृतः क्षत्रधर्मः पुराणः ॥'}


\twolineshloka
{तथा राजन्यो रक्षणं वै प्रजानांकृत्वा धर्मेणाप्रमत्तोऽथ दत्त्वा}
{यज्ञैदिष्ट्वा सर्ववेदानधीत्यदारान्कृवा पुण्यकृदावसेद्गृहान्}


\twolineshloka
{स धर्मात्मा धर्ममधीत्य पुण्यंयदृच्छया व्रजति ब्रह्मलोकम्}
{वैश्योऽधीत्य कृषिगोरक्षपण्यै-र्वित्तं चिन्वन्पालयन्नप्रमत्तः}


\threelineshloka
{प्रियं कुर्वन्ब्राह्मणक्षत्रियाणांधर्मशीलः पुण्यकृदावसेद्गृहान्}
{परिचर्या वन्दनं ब्राह्मणानांनाधीयीत प्रतिषिद्धोऽस्य यज्ञः}
{नित्योत्थितो भूतयेऽतन्द्रितः स्या-देवं स्मृतः शूद्रधर्मः पुराणः}


\twolineshloka
{एतान्राजा पालयन्नप्रमत्तोनियोजयन्सर्ववर्णान्स्वधर्मे}
{अकामात्मा समवृत्तिः प्रजासुनाधार्मिकाननुरुध्येत कामात्}


\twolineshloka
{श्रेयांस्तस्माद्यदि विद्येत कश्चि-दभिज्ञातः सर्वधर्मोपपन्नः}
{स तं द्रष्टुमनुशिष्यन्प्रजानांन चैतद्बुध्येदिति तस्मिन्नसाधुः}


\threelineshloka
{यदा गृध्येत्परभूतौ नृशंसोविधिप्रकोपाद्बलमाददानः}
{ततो राज्ञामभवद्युद्धमेत-त्तत्र जातं वर्म शस्त्रं धनुश्च}
{इन्द्रेणैतद्दस्युवधाय कर्मउत्पादितं वर्म शस्त्रं धनुश्च}


\twolineshloka
{तत्र पुण्यं दस्युवधेन लभ्यतेसोऽयं दोषः कुरुभिस्तीव्ररूपः}
{अधर्मज्ञैर्धर्ममबुध्यमानैःप्रादुर्भूतः सञ्जय साधु तन्न}


\twolineshloka
{तत्र राजा धृतराष्ट्रः सपुत्रोधर्म्यं हरेत्पाण्डवानामकस्मात्}
{नावेक्षन्ते राजधर्मं पुराणंतदन्वयाः कुरवः सर्व एव}


\twolineshloka
{स्तेनो हरेद्यत्र धनं ह्यदृष्टःप्रसह्य वा यत्र हरेत दृष्टः}
{उभौ गर्ह्यौ भवतः सञ्जयैतौकिं वै पृथत्कं धृतराष्ट्रस्य पुत्रे}


\twolineshloka
{सोऽयं लोभान्मन्यते धर्ममेतंयमिच्छति क्रोधवशानुगामी}
{भागः पुनः पाण्डवानां निविष्ट-स्तं नः कस्मादाददीरन्परे वै}


\twolineshloka
{अस्मिन्पदे युध्यतां नो वधोऽपिश्लाघ्यः पित्र्यं पपराज्याद्विशिष्टम्}
{एतान्धर्मान्कौरवाणां पुराणा-नाचक्षीथाः सञ्जय राजमध्ये}


\twolineshloka
{एते मदान्मृत्युवशाभिपन्नाःसमानीता धार्तराष्ट्रेण मूढाः}
{इदं पुनः कर्म पापीय एवसभामध्ये पश्य वत्तं कुरूणाम्}


\twolineshloka
{प्रियं भार्यां द्रौपदीं पाण्डवानांयशस्विनीं शीलवृत्तोपपन्नाम्}
{यदुपैक्षन्त कुरवो भीष्ममुख्याःकामानिगेनोपरुद्धां व्रजन्तीम्}


\twolineshloka
{तां चेत्तदा ते सकुमारवृद्धाअवारयिष्यन्कुरवः समेताः}
{मम प्रियं धृतराष्ट्रोऽकरिष्यत्पुत्राणां च कृतमस्याभविष्यत्}


\twolineshloka
{दुःशासनः प्रतिलोम्यान्निनायसभामध्ये श्वशुराणां च कृष्णाम्}
{सा तत्र नीता करुणं व्यपेक्ष्यनान्यं क्षत्तुर्नाथमवाप किंचित्}


\twolineshloka
{कार्पण्यादेव सहितास्तत्र भूपानाशक्नुवन्प्रतिवक्तुं सभायाम्}
{एकः क्षत्ता धर्म्यमर्थं ब्रुवाणोधर्मबुद्ध्या प्रत्युवाचाल्पबुद्धिम्}


\twolineshloka
{अबुद्ध्वा त्वं धर्ममेतं सभाया-अथेच्छसे पाण्डवस्योपदेष्टुम्}
{कृष्णा त्वेतत्कर्म चकार शुद्धंसुदुष्करं तत्र सभां समेत्य}


\twolineshloka
{येन कृच्छ्रात्पाण्डवानुज्जहारतथाऽऽत्मानं नौरिव सागरौघात्}
{यत्राब्रवीत्सूतपुत्रः सभायांकृष्णां स्थितां श्वशुराणां समीपे}


\twolineshloka
{न ते गतिर्विद्यते याज्ञसेनिप्रपद्येथा धार्तराष्ट्रस्य वेश्म}
{पराजितास्ते पतयो न सन्तिपतिं चान्यं भामिनि त्वं वृणीष्व}


\twolineshloka
{यो बीभत्सोर्हृदये प्रोत आसी-दस्थि छिन्दन्मर्मघाती सुघोरः}
{कर्णाच्छरो वाङ्भयस्तिग्मतेजाःप्रतिष्ठितो हृदये फाल्गुनस्य}


\twolineshloka
{कृष्णाजिनानि परिधित्समानान्दुःशासनः कटुकान्यभ्यभाषत्}
{एते सर्वे षण्डतिला विनष्टाःक्षयं गता नरकं दीर्घकालम्}


\threelineshloka
{गान्धारराजः शकुनिर्निकृत्यायदब्रवीद्द्यूतकाले स पार्थम्}
{पराजितो नकुलः किं तवास्तिकृष्णया त्वं दीव्य वै याज्ञसेन्या}
{जानासि त्वं सञ्जय सर्वमेतत्द्यूते वाक्यं गर्ह्यमेवं यथोक्तम्}


\twolineshloka
{स्वयं त्वहं प्रार्थये तत्र गन्तुंसमाधातुं कार्यमेतद्विपन्नम्}
{` जानासि त्वं धार्तराष्ट्रस्य मोहंदुरात्मनः पापवशानुगस्य '}


\twolineshloka
{अहापयित्वा यदि पाण्डवार्थंशमं कुरूणामपि चेच्छकेयम्}
{पुण्यं च मे स्याच्चरितं महोदयंमुच्येरंश्च कुरवो मृत्युपाशात्}


\twolineshloka
{अपि मे वाचं भाषमाणस्य काव्यांधर्मारामामर्थवतीमहिंस्राम्}
{अवेक्षेरन्धार्तराष्ट्राः समक्षंमां च प्राप्तं कुरवः पूजयेयुः}


\twolineshloka
{अतोऽन्यथा रथिना फाल्गुनेनभीमेन चैवाहवदंशितेन}
{बलोत्सिक्तान्धार्तराष्ट्रांश्च विद्धिप्रदह्यमानान्कर्मणा स्वेन पापान्}


\twolineshloka
{पराजितान्पाण्डवेयांस्तु वाचोरौद्रा रूक्षा भाषते धार्तराष्ट्रः}
{गदाहस्तो भीमसेनोऽप्रमत्तोदुर्योधनं स्मारयिता हि काले}


\twolineshloka
{सुयोधनो मन्युमयो महाद्रुमःस्कन्धः कर्णः शकुनिस्तस्य शाखाः}
{दुःशासनः पुष्कफले समृद्धेमूलं राजा धृतराष्ट्रोऽमनीषी}


\twolineshloka
{युधिष्ठिरो धर्ममयो महाद्रुमःस्कन्धोऽर्जुनो भीमसेनोऽस्य शाखाः}
{माद्रीपुत्रौ पुष्पफले समृद्धेमूलं त्वहं ब्रह्म च ब्राह्मणाश्च}


\twolineshloka
{` वनं राजा धृतराष्ट्रः सपुत्रःसिंहा वने सञ्जय पाण्डवेयाः}
{सिंहाभिगुप्तं न वनं विनश्येत्सिंहो न नश्येत वनाभिगुप्तः}


\twolineshloka
{निर्वनो वध्यते सिंहो निःसिंहं वध्यते वनम्}
{तस्मात्सिंहो वनं रक्षेद्वनं सिंहं च पालयेत् ॥'}


\twolineshloka
{वनं राजा धृतराष्ट्रः वने व्याघ्राश्च पाण्डवाः}
{मा वनं छिन्धि सव्याघ्रं व्याघ्रा नेशुर्वनं विना}


\twolineshloka
{निर्वनो वध्यते व्याघ्रो निर्व्याघ्रं छिद्यते वनम्}
{तस्माद्व्याघ्रो वनं रक्षेद्वयं व्याघ्रं च पालयेत्}


\twolineshloka
{लताधर्मा धार्तराष्ट्राः सालाः सञ्जय पाण्डवाः}
{न लता वर्धते जातु महाद्रुममनाश्रिता}


\twolineshloka
{स्थिताः शुश्रूषितुं पार्थाः स्थिता योद्धुमरिन्दमाः}
{यत्कृत्यं धृतराष्ट्रस्य तत्करोतु नराधिपः}


\twolineshloka
{स्थिताः शमे महात्मानः पाण्डवा धर्मचारिणः}
{योधाः समर्थास्तद्विद्वन्नाचक्षीथा यथातथम्}


\chapter{अध्यायः ३०}
\twolineshloka
{सञ्जय उवाच}
{}


\twolineshloka
{आमन्त्रये त्वां नरदेवदेवगच्छाम्यहं पाण्डव स्वस्ति तेऽस्तु}
{कच्चिन्न वाचा वृजिनं हि किंचि-दुच्चारितं मे मनसोऽभिषङ्गात्}


\threelineshloka
{जनार्दनं भीमसेनार्जुनौ चमाद्रीसुतौ सात्यकिं चेकितानम्}
{आमन्त्र्य गच्छामि शिवं सुखं वःसौम्येन मां पश्यत चक्षुषा नृपाः ॥युधिष्ठिर उवाच}
{}


\twolineshloka
{अनुज्ञातः सञ्जय स्वस्ति गच्छन नः स्मरस्यप्रियं जातु विद्वन्}
{विद्मश्च त्वां ते च वयं च सर्वेशुद्धात्मानं मध्यगतं सभास्थम्}


\twolineshloka
{आप्तो दूतः सञ्जय सुप्रियोऽसिकल्याणवाक् शीलवांस्तृप्तिमांश्च}
{न मुह्येस्त्वं सञ्जय जातु मत्यान च क्रुद्ध्येरुच्यमानोऽपि तत्त्वम्}


\twolineshloka
{न मर्मगां जातु वक्ताऽसि रूक्षांनोपश्रुतिं कटुकां नोति शुष्काम्}
{धर्मारामामर्थवतीमहिंस्रा-मेतां वाचं तव जानीम सूत}


\twolineshloka
{त्वमेव नः प्रियतमोऽसि दूतइहागच्छेद्विदुरो वा द्वितीयः}
{अभीक्ष्णदृष्टोऽसि पुरातनस्त्वंधनञ्जयस्यात्मसमः सखाऽसि}


\twolineshloka
{इतो गत्वा सञ्जय क्षिप्रमेवउपातिष्ठेथा ब्राह्मणान्ये तदर्हाः}
{विशुद्धवीर्याश्चरणोपपन्नाःकुले जाताः सर्वधर्मापपन्नाः}


\twolineshloka
{स्वाध्यायिनो ब्राह्मणा भिक्षवश्चतपस्विनो ये च नित्या वनेषु}
{अभिवाद्य तान्मद्वचनेन वृद्धां-स्तथैव तान्कुशलं तात पृच्छेः}


\twolineshloka
{पुरोहितं धृतराष्ट्रस्य राज्ञ-स्तथाचार्यानृत्विजो ये च तस्य}
{तांश्चैवत्वं सहितान्वै यथावत्सङ्गच्छेथाः कुशलेनैव सूत}


\twolineshloka
{अश्रोत्रिया ये च वसन्ति वृद्धामनस्विनः शीलबलोपपन्नाः}
{आशंसन्तोऽस्माकमनुस्मरन्तोयथाशक्ति धर्ममात्रां चरन्तः}


\threelineshloka
{श्लाघस्व मां कुशलिनं स्म तेभ्योह्यनामयं तात पृच्छेर्जघन्यम्}
{ये जीव्ति व्यवहारेण राष्ट्रेपशूंश्च ये पालयन्तो वसन्ति}
{` कृषीवला बिभ्रति ये च लोकंतेषां सर्वषां कुशलं स्म पृच्छेः ॥'}


\twolineshloka
{आचार्य इष्टो नयगो विधेयोवेदानभीप्सन्ब्रह्मचर्यं चचार}
{योऽस्त्रं चतुष्पात्पुनरेव चक्रेद्रोणः प्रसन्नोऽभिवाद्यस्त्वयासौ}


\twolineshloka
{अधीतविद्यश्चरणोपपन्नोयोऽस्त्रं चतुष्पात्पुनरेव चक्रे}
{गन्धर्वपुत्रप्रतिमं तरस्विनंतमश्वत्थामानं कुशलं स्म पृच्छेः}


\twolineshloka
{शारद्वतस्यावसथं स्म गत्वामहारथस्यात्मविदां वरस्य}
{त्वं मामभीक्ष्णं परिकीर्तयन्वैकृपस्य पादौ सञ्जय पाणिना स्पृशेः}


\twolineshloka
{यस्मिन्शौर्यमानृशंस्यं तपश्चप्रज्ञा शीलं श्रुतरूपे धृतिश्च}
{पादौ गृहीत्वा कुरुसत्तमस्यभीष्मस्य मां तत्र निवेदयेथाः}


\twolineshloka
{प्रज्ञाचक्षुर्यः प्रणेता कुरूणांबहुश्रुतो वृद्धसेवी मनीषी}
{तस्मै राज्ञे स्थविरायाभिवाद्यआचक्षीथाः सञ्जय मामरोगम्}


\twolineshloka
{ज्येष्ठः पुत्रो धृतराष्ट्रस्य मन्दोमूर्खः शठः सञ्जय पापशीलः}
{यस्यापवादः पृथिवीं याति सर्वांसुयोधनं कुशलं तात पृच्छेः}


\twolineshloka
{भ्राता कनीयानपि तस्य मन्द-स्तथाशीलः सञ्जय सोपि शश्वत्}
{महेष्वासः शूरतमः कुरूणांदुःशासनः कुशलं तात वाच्यः}


\twolineshloka
{यस्य कामो वर्तते नित्यमेवनान्यः शमाद्भारतानामिति स्म}
{स बाह्लिकानामृषभो मनीषीपुनर्यथा माऽभिवदेत्प्रसन्नः}


\twolineshloka
{` भूरिश्रवास्तात निपातयोधीमहेष्वासो रथिनामुत्तमोऽग्र्यः}
{गत्वा स्म तं मद्वचनेन ब्रूयाःशल्यं तथा मद्वचनात्प्रतीतः}


\threelineshloka
{महेष्वासो रथिनामुत्तमोऽग्र्यःसमः शलो रक्षिता पृष्ठमस्य}
{हीनिषेवो देविता यो मताक्षःसत्यव्रतः पुरुमित्रो जयश्च}
{ये प्रस्थानं तत्र मे नाभ्यनन्दं-स्तेषां सर्वेषां कुशलं तात पृच्छेः ॥'}


\twolineshloka
{गुणैरनेकैः प्रवरैश्च युक्तोविज्ञानवान्नैव च निष्ठुरो यः}
{स्नेहादमर्षं सहते सदैवस सोमदत्तः पूजनीयो मतो मे}


\twolineshloka
{अर्हत्तमः कुरुषु सौमदत्तिःस नो भ्राता संजय मत्सखा च}
{महेष्वासो रथिनामुत्तमोऽर्हःसहामात्यः कुशलं तस्य पृच्छेः}


\twolineshloka
{ये चैवान्ये कुरुमुख्या युवानःपुत्राः पौत्रा भ्रातरश्चैव ये नः}
{यंयमेषां मन्यसे येन योग्यंतत्तत्प्रोच्यानामयं सूत वाच्याः}


\twolineshloka
{ये राजानः पाण्डवायोधनायसमानीता धार्तराष्ट्रेण केचित्}
{वशातयः साल्वकाः केकयाश्चतथाम्बष्ठा ये त्रिगर्ताश्च मुख्याः}


\twolineshloka
{प्राच्योदीच्या दाक्षिण्यात्याश्च शूरा-स्तथा प्रतीच्याः पार्वतीयाश्च सर्वे}
{अनृशंसाः शीलवृत्तोपुन्ना-स्तेषां सर्वेषां कुशलं सूत पृच्छेः}


\twolineshloka
{हस्त्यारोहा रथिनः सादिनश्चपदातयश्चार्यसङ्घा महान्तः}
{आख्याय मां कुशलिनं स्म नित्य-मनामयं परिपृच्छेः समग्रान्}


\twolineshloka
{तथा राज्ञो ह्यर्थयुक्तानमात्यान्दौवारिकान्ये च सेनां नयन्ति}
{आयव्ययं ये गमयन्ति नित्य-मर्थांश्च ये महतश्चिन्तयन्ति}


\twolineshloka
{वृन्दारकं कुरुमध्येष्वमूढंमहाप्रज्ञं सर्वधर्मोपपन्नम्}
{न तस्य युद्धं रोचते वै कदाचि-द्वैश्यापुत्रं कुशलं तात पृच्छेः}


\twolineshloka
{निकर्तने देवने योऽद्वितीय-श्छन्नोपधः साधुदेवी मताक्षः}
{यो दुर्जयो देवरथेन सङ्ख्येस चित्रसेनः कुशलं तात वाच्यः}


\twolineshloka
{गान्धारराजः शकुनिः पार्वतीयोनिकर्तने योऽद्वितीयोऽक्षदेवी}
{मानं कुर्वन्धार्तराष्ट्रस्य सूतमिथ्याबुद्धेः कुशलं तात पृच्छेः}


\twolineshloka
{यः पाण्डवानेकरथेन वीरःसमुत्सहत्यप्रधृप्यान्विजेतुम्}
{यो मुह्यतां मोहयिताऽद्वितीयोवैकर्तनः कुशलं तस्य पृच्छेः}


\twolineshloka
{स एव भक्तः स गुरुः स भर्तास वै पिता स च माता सुहृच्च}
{अगाधबुद्धिर्विदुरो दीर्घदर्शीस नो मन्त्री कुशलं तं स्म पृच्छेः}


\twolineshloka
{वृद्धाः स्त्रियो याश्च गुणोपपन्नाज्ञायन्ते नः सञ्जय मातरस्ताः}
{ताभिः सर्वाभिः सहिताभिः समेत्यस्त्रीभिः सवृद्धाभिरभिवादं वदेथाः}


\twolineshloka
{कच्चित्पुत्रा जीवपुत्राः सुसम्य-ग्वर्तन्ते वो वृत्तिमनृशंसरूपाः}
{इति स्मोत्का सञ्जय ब्रूहि पश्चा-दजातशत्रुः कुशली सपुत्रः}


\twolineshloka
{राज्ञो भार्याः सञ्जय वेत्थ तत्रतासां सर्वासां कुशलं तात पृच्छेः}
{सुसंगुप्ताः सुरभयोऽनवद्याःकच्चिद्गृहानावसथाप्रमत्ताः}


\twolineshloka
{कच्चिद्वृत्तिं श्वशुरेषु भद्राःकल्याणीं वर्तध्वमनृशंसरूपाम्}
{यथा च वः स्युः पतयोऽनुकूला-स्तथा वृत्तिमात्मनः स्थापयध्वम्}


\twolineshloka
{या नः स्नुषाः सञ्जय वेत्थ तत्रप्राप्ताः कुलेभ्यश्च गुणोपपन्नाः}
{प्रजावत्यो ब्रूहि समेत्य ताश्चयुधिष्ठिरो वोऽभ्यवदत्प्रसन्नः}


\twolineshloka
{कन्याः स्वजेथाः सदनेषु सञ्जयअनामयं मद्वचनेन पृष्ट्वा}
{कल्याणा वः सन्तु पतयोऽनुकूलायूयं पतीनां भवतानुकूलाः}


\twolineshloka
{अलङ्कृता वस्त्रवत्यः सुगन्धाअबीभत्साः सुखिता भोगवत्यः}
{लघु यासां दर्शनं वाक्व लघ्वीवेशस्त्रियः कुशलं तात पृच्छेः}


\threelineshloka
{दास्यः स्युर्या ये च दासाः कुरूणांतदाश्रया बहवः कुब्जखञ्जाः}
{आख्याय मां कुशलिनं स्म तेभ्यो-ऽप्यनामयं परिपृच्छेर्जघन्यम्}
{}


\twolineshloka
{कच्चिद्वृत्तिं वर्तते वै पुराणींकच्चिद्भोगान्दार्तराष्ट्रो ददाति}
{अङ्गहीनान्कृपणान्वामनान्वायानानृशंस्याद्धार्तराष्ट्रो बिभर्ति}


\twolineshloka
{अन्धाश्च सर्वे बधिरास्तथैवहस्ताजीवा बहवो येऽत्र सन्ति}
{आख्याय मां कुशलिनं स्म तेभ्यो-ऽप्यनामयं परिपृच्छेर्जघन्यम्}


\twolineshloka
{मा भैष्ट दुःखेन कुजीवितेननूनं कृतं परलोकेषु पापम्}
{निगृह्य शत्रून्सुहृदोऽनुगृह्यवासोभिरन्नेन च वो भरिष्ये}


\threelineshloka
{` न चाप्येतच्छक्यमेकेन वक्तुंनानादेशा बहवो जातिसङ्घाः}
{विप्रोषितो बालवद्द्रष्टुमिच्छ-न्नमस्येऽहं सञ्जय भैमसेनान्}
{ते मे यथा वाचमिमां यथोक्तांत्वयोच्यमानां श्रृणुयुक्तथा कुरु}


\twolineshloka
{सन्त्येव मे ब्राह्मणेभ्यः कृतानिभावीन्यथो नो बत वर्तयन्ति}
{तान्पश्यामि युक्तरूपांस्तथैवतामेव सिद्धिं श्रावयेथा नृपं तम्}


\twolineshloka
{ये चानाथा दुर्बलाः सर्वकाल-मात्मन्येव प्रयतन्तेऽथ मूढाः}
{तांश्चापि त्वं कृपणान्सर्वथैवह्यस्मद्वाक्यात्कुशलं तात पृच्छेः}


\twolineshloka
{ये चाप्यन्ये संश्रिता धार्तराष्ट्रा-न्नानादिग्भ्योऽभ्यागताः सूतपुत्र}
{दृष्ट्वा तांश्चैवार्हतश्चापि सर्वा-न्संपृच्छेथाः कुशलं चाव्ययं च}


\twolineshloka
{एवं सर्वानागताभ्यागतांश्चराज्ञो दूतान्सर्वदिग्भ्योभ्युपेतान्}
{पृष्ट्वा सर्वान्कुशलं तांश्च सूतपश्चादहं कुशली तेषु वाच्यः}


\twolineshloka
{न हीदृशाः सन्त्यपरे पृथिव्यांये योधका धार्तराष्ट्रेण लब्धाः}
{धर्मस्तु नित्यो मम धर्म एवमहाबलः शत्रुनिबर्हणाय}


\twolineshloka
{इदं पुनर्वचनं धार्तराष्ट्रंसुयोधनं सञ्जय श्रावयेथाः}
{यस्ते शरीरे हृदयं दुनोतिकामः कुरूनसपत्नोऽनुशिष्याम्}


\twolineshloka
{न विद्यते युक्तिरेतस्य काचि-न्नैवं विधास्यामि यथा प्रियं ते}
{ददस्व वा शक्रपुरीं ममैवयुध्वस्व वा भारतमुख्यवीर}


\chapter{अध्यायः ३१}
\twolineshloka
{युधिष्ठिर उवाच}
{}


\twolineshloka
{उत सन्तमसन्तं वा बालं वृद्धं च सञ्जय}
{उताबलं बलीयांसं धाता प्रकुरुते वशे}


\twolineshloka
{उत बालाय पाण्डित्यं पण्डितायोत बालताम्}
{ददाति सर्वमीशानः पुरस्ताच्छुक्रमुच्चरन्}


\twolineshloka
{बलं जिज्ञासमानस्य आचक्षीथा यथातथम्}
{अथ मन्त्र मन्त्रयित्वा याथातथ्येन हृष्टवत्}


\twolineshloka
{गावल्गणे कुरून्गत्वा धृतराष्ट्रं महाबलम्}
{अभिवाद्योपसंगृह्य ततः पृच्छेरनामयम्}


\twolineshloka
{ब्रूयाश्चैनं त्वमासीनं कुरुभिः परिवारितम्}
{तवैव राजन्वीर्येण सुखं जीवन्ति पाण्डवाः}


\twolineshloka
{तव प्रसादाद्बालास्ते प्राप्ता राज्यमरिन्दम}
{राज्ये तान्स्थापयित्वाऽग्रे नोपेक्षस्व विनश्यतः}


\twolineshloka
{सर्वमप्येतदेकस्य नालं सञ्जय कस्यचित्}
{तात संहत्य जीवामो द्विषतां मा वशं गमः}


\twolineshloka
{तथा भीष्मं शान्तनवं भारतानां पितामहम्}
{शिरसाऽभिवदेथास्त्वं मम नाम प्रकीर्तयन्}


\twolineshloka
{अभिवाद्य च वक्तव्यस्ततोऽस्माकं पितामहः}
{भवता शन्तनोर्वंशो निमग्नः पुनरुद्धृतः}


\twolineshloka
{स त्वं कुरु तथा तात स्वमतेन पितामह}
{यथा जीवन्ति ते पौत्राः प्रीतिमन्तः परस्परम्}


\twolineshloka
{तथैव विदुरं ब्रूयाः कुरूणां मन्त्रधारिणम्}
{अयुद्धं सौम्य भाषस्व हितकामो युधिष्ठिरे}


\twolineshloka
{अथ दुर्योधनं ब्रूया राजपुत्रममर्षणम्}
{मध्ये कुरूणामासीनमनुनीय पुनः पुनः}


\twolineshloka
{अपापां यदुपैक्षस्त्वं कृष्णामेतां सभागताम्}
{तद्दुःखमतितिक्षाम मा वधीष्म कुरूनिति}


\twolineshloka
{एवं पूर्वापरान्क्लेशानतितिक्षन्त पाण्डवाः}
{बलीयांसोऽपि सन्तो यत्तत्सर्वं कुरवो विदुः}


\twolineshloka
{यन्नः प्रव्राजयेः सौम्य अजिनैः प्रतिवासितान्}
{तद्दुःखमतितिक्षाम मा वधीषय कुरूनिति}


\twolineshloka
{यत्कुन्तीं समतिक्रम्य कृष्णां केशेष्वधर्षयत्}
{दुःशासनस्तेऽनुमते तच्चास्माभिरुपेक्षितम्}


\twolineshloka
{अथोचितं स्वकं भागं लभेमहि परन्तप}
{निवर्तय परद्रव्याद्बुद्धिं गृद्धां नरर्षभ}


\twolineshloka
{शान्तिरेवं भवेद्राजन्प्रीतिश्चैव परस्परम्}
{राज्यैकदेशमपि नः प्रयच्छ शममिच्छताम्}


\twolineshloka
{अविस्थलं वृकस्थलं माकन्दीं वारणावतम्}
{अवसानं भवत्वत्र किंचिदेकं च पञ्चमम्}


\twolineshloka
{भ्रातॄणां देहि पञ्चानां पञ्च ग्रामान्सुयोधन}
{शान्तिर्नोस्तु महाप्राज्ञ ज्ञातिभिः सह सञ्जय}


\twolineshloka
{भ्राता भ्रातरमन्वेतु पिता पुत्रेण युज्यताम्}
{स्मयमानाः समायान्तु पाञ्चालाः कुरुभिः सह}


\twolineshloka
{अक्षतान्कुरु पाञ्चालान्पश्येयमिति कामये}
{सर्वे सुमनसस्तात शाम्याय भरतर्षभ}


\twolineshloka
{अलमेव शमायास्मि तथा युद्धाय सञ्जय}
{धर्मार्थयोरलं चाहं मृदवे दारुणाय च}


\chapter{अध्यायः ३२}
\twolineshloka
{` धर्मराजस्य वचनं श्रुत्वा पार्थो धनञ्जयः}
{उवाच सञ्जयं तत्र वासुदेवस्य शृण्वतः}


\twolineshloka
{पितामहं शान्तनवं धृतराष्ट्रं च सञ्जय}
{द्रोणं सपुत्रं शल्यं च महाराजं च बाह्लिकम्}


\threelineshloka
{विकर्णं सोमदत्तं च शकुनिं चैव सौबलम्}
{विविंशतिं चित्रसेनं जयत्सेनं च सञ्जय}
{भगदत्तं तथा चैव शूरं रणकृतां वरम्}


\twolineshloka
{ये चाप्यन्ये कुरवस्तत्र सन्तिराजानश्चेद्भूमिपालाः समेताः}
{युयुत्सवः सैन्धवाः पार्थिवाश्चसमानीता धार्तराष्ट्रेण सूत}


\twolineshloka
{यथान्यायं कुशलं वन्दनं चसमागमे मद्वचनेन वाच्याः}
{ततो ब्रूयाः सञ्जय राजमध्येदुर्योधनं पापकृतां प्रधानम्}


\twolineshloka
{एवं प्रतिष्ठाप्य धनञ्जयस्तंततोऽर्थवद्धर्मवच्चापि पार्थः}
{उवाच वाक्यं स्कवजनप्रहर्षंवित्रासनं धृतराष्ट्रात्मजानाम्}


\twolineshloka
{अर्जुनेन समादिष्टस्तथेत्सुक्त्वा तु सञ्जयः}
{पार्थानामन्त्रयामास केशवं च यशस्विनम् ॥'}


\twolineshloka
{अनुज्ञातः पाण्डवेन प्रययौ सञ्जयस्तदा}
{शासनं धृतराष्ट्रस्यक सर्वं कृत्वा महात्मनः}


\twolineshloka
{` तदा तु सञ्जयः क्षिप्रमेकाहेन परन्तपः}
{याति स्म हास्तिनपुरं निशाकाले परन्तप ॥'}


\twolineshloka
{संप्राप्य हास्तिनपुरं शीघ्रमश्वैर्महाजवैः}
{अन्तःपुरं समास्थाय द्वाःस्थं वचनमब्रवीत्}


\twolineshloka
{आचक्ष्व धृतराष्ट्रास्य द्वाऽस्थ मां समुपागतम्}
{सकाशात्पाण्डुपुत्राणां सञ्जयं माचिरं कृथाः}


\threelineshloka
{.....र्ति चेदभिवदेस्त्वं हि द्वाःस्थ.....विशेयं विदितो भूमिपस्य}
{.....द्यमत्रात्ययिकं हि मेऽस्ति..... स्थोऽथ श्रुत्वा नृपतिं जगाद ॥द्वाःस्थ उवाच}
{}


\threelineshloka
{सञ्जयोऽयं भूमिपते नमस्तेदिदृक्षया द्वारमुपागतस्ते}
{प्राप्तो दूतः पाण्डवानां सकाशात्प्रशाधि राजन्किमयं करोतु ॥धृतराष्ट्र उवाच}
{}


\twolineshloka
{आचक्ष्व मां कुशलिनं कल्पमस्मैप्रवेश्यतां स्वागतं सञ्जयायन चाहमेतस्य भवाम्यकल्पःस मे कस्माद्द्वारि तेष्ठेच्च सक्तः ॥वैशंपायन उवाच}
{}


\threelineshloka
{ततः प्रविश्यानुमते नृपस्यमहद्वेश्म प्राज्ञशूरार्यगुप्तम्}
{सिंहासनस्थं पार्थिवमाससादवैचित्रवीर्यं प्राञ्जलिः सूतपुत्रः ॥संजय उवाच}
{}


\twolineshloka
{सञ्जयोऽहं भूमिपते नमस्तेप्राप्तोऽस्मि गत्वा नरदेव पाण्डवान्}
{अभिवाद्य त्वां पाण्डुपुत्रो मनस्वीयुधिष्ठिरः कुशलं चान्वपृच्छत्}


\threelineshloka
{स ते पुत्रान्पृच्छति प्रीयमाणःकच्चित्पुत्रैः प्रीयसे नप्तृभिश्च}
{तथा सुहृद्भिः सचिवैश्च राजन्ये चापि त्वामुपजीवन्ति तैश्च ॥धृतराष्ट्र उवाच}
{}


\threelineshloka
{अभिनन्द्य त्वां तात वदामि सञ्जयअजातशत्रुं च सुखेन पार्थम्}
{कच्चित्स राजा कुशली सपुत्रःसहामात्यः सानुजः कौरवाणाम् ॥सञ्जय उवाच}
{}


% Check verse!
सहामात्य कुशली पाण्डुपुत्रोबुभूषते यच्च तेऽग्रे त्मनाऽभूत्निर्णीक्तधर्मार्थकरो मनस्वीबहुश्रुतो दृष्टिमाञ्शीलवांश्च
\twolineshloka
{परो धर्मः पाण्डवस्यानृशंस्यंधर्मः परो वित्तचयान्मतोऽस्य}
{सुखप्रियेऽधर्महीनेऽनपार्थेनु रुध्यते भारत तस्य बुद्धिः}


\twolineshloka
{परप्रयुक्तः पुरषो विचेष्टतेसूत्रप्रोता दारुमयीव योषा}
{इमं दृष्ट्वा नियमं पाण्डवस्यमन्ये परं कर्म दैवं मनुष्यात्}


\twolineshloka
{इमं च दृष्ट्वा तव कर्म दोषंपापोदर्कं घोरमवर्णरूपम्}
{यावत्परः कामयतेऽतिवेलंतावन्नरोऽयं लभते प्रशंसाम्}


\twolineshloka
{अजातशत्रुस्तु विहाय पापंजीर्णां त्वचं सर्प इवासमर्थाम्}
{विरोचते ह्यार्यवृत्तेन वीरोयुधिष्ठिरस्त्वयि पापं विसृज्य}


\twolineshloka
{हन्तात्मनः कर्म निबोध राजन्धर्मार्थयुक्तादार्यवृत्तादपेतम्}
{उपक्रोशं चेह गतोऽसि राजन्भूयश्च पापं प्रसजेदमुत्र}


\twolineshloka
{स त्वमर्थं संशयितं विना तै-राशंससे पुत्रवशानुगोऽस्य}
{अधर्मशब्दश्च महान्पृथिव्यांनेदं कर्म त्वत्समं भारताग्र्य}


\twolineshloka
{हीनप्रज्ञो दौष्कुलेयो नृशंसोदीर्घं वैरी क्षत्रविद्यास्वधीरः}
{एवंधर्मानापदः संश्रयेयु-र्हीनवीर्यो यश्च भवेदशिष्टः}


\threelineshloka
{कुले जातो बलवान्यो यशस्वी}
{बहुश्रुतः सुखजीवी यतात्मा}
{धर्माधर्मौ ग्रथितौ यो बिभर्तिस ह्यस्य दिष्टस्य वशादुपैति}


\twolineshloka
{कथं हि मन्त्राग्र्यधरो मनीषीधर्मार्थयोरापदि संप्रणेता}
{एवमुक्तः सर्वमन्त्रैरहीनोनरो नृशंसं कर्म कुर्यादमूढः}


\twolineshloka
{तव ह्यमी मन्त्रविदः समेत्यसमासते कर्मसु नित्ययुक्ताः}
{तेषांमयं बलवान्निश्चयश्चकुरुक्षये नियमेनोदपादि}


% Check verse!
अकालिकं कुरवो नाभविष्यन्पापेन चेत्पापमजातशत्रुःइच्छेञ्जातु त्वयि पापं विसृज्यनिन्दा चेयं तव लोकेऽभविष्यत्
\twolineshloka
{किमन्यत्र विषयादीश्वराणांयत्र पार्थः परलोकं स्म द्रुष्टुम्}
{अत्यक्रामत्स तथा संमतः स्या-न्न संशयो नास्ति मनुष्यकारः}


\twolineshloka
{एतान्गुणान्कर्मकृतानवेक्ष्यभावाभावौ वर्तमानावनित्यौ}
{बलिर्हि राजा पारमविन्दमानोनान्यत्कालात्कारणं तत्र मेने}


\twolineshloka
{चक्षुःश्रोत्रे नासिका त्वक् च जिह्वाज्ञानस्यैतान्यायतनानि जन्तोः}
{तानि प्रीतान्येव तृष्णाक्षयान्तेतान्यव्यथो दुःखहीनः प्रणुद्यात्}


\twolineshloka
{नत्वेव मन्ये पुरुषस्य कर्मसवर्तते सुप्रयुक्तं यथावत्}
{मातुः पितुः कर्मणाभिप्रसूतःसंवर्धते विधिवद्भोजनेन}


\twolineshloka
{प्रियाप्रिये सुखदुःखे च राज-न्निन्दाप्रशंसे च भजन्त एव}
{परस्त्वेनं गर्हयतेऽपराधेप्रशंसते साधुवृत्तं तमेव}


\twolineshloka
{स त्वां गर्हे भारतानां विरोधा-दन्तो नूनं भविताऽयं प्रजानाम्}
{नोचेदिदं तव कर्मापराधात्कुरून्दहेत्कृष्णवर्त्सेव कक्षम्}


\twolineshloka
{त्वमेवैको जातु पुत्रस्य राजन्वशं गत्वा सर्वलोके नेरन्द्र}
{कामात्मनः श्लाघनो द्यूतकालेनागाः शमं पश्य विपाकमस्य}


\twolineshloka
{अनाप्तानां सङ्ग्रहात्त्वं नरेन्द्रतथाऽऽप्तानां निग्रहाच्चैव राजन्}
{भूमिं स्फीतां दुर्बलत्वादनन्ता-मशक्तस्त्वं रक्षितुं कौरवेय}


\threelineshloka
{अनुज्ञातो रथवेगावधूतःश्रान्तोऽभिपद्ये शयनं नृसिंह}
{प्रातः श्रोतारः कुरवः सभाया-मजातशत्रोर्वचनं समेताः ॥धृतराष्ट्र उवाच}
{}


\twolineshloka
{अनुज्ञातोऽस्यावसथं परेहिप्रपद्यस्व शयनं सूतपुत्र}
{प्रातः श्रोतारः कुरवः सभाया-मजातशत्रोर्वचनं त्वयोक्तम्}


\chapter{अध्यायः ३३}
\twolineshloka
{वैशंपानय उवाच}
{}


\twolineshloka
{द्वाःस्थं प्राह महाप्राज्ञो धृतराष्ट्रो महीपतिः}
{विदुरं द्रष्टुमिच्छामि तमिहानय मा चिरम्}


\twolineshloka
{प्रहितो धृतराष्ट्रेण दूतः क्षत्तारमब्रवीत्}
{ईश्वरस्त्वां महाराजो महाप्राज्ञ दिदृक्षति}


\threelineshloka
{एवमुक्तस्तु विदुरः प्राप्य राजनिवेशनम्}
{अब्रवीद्धृतराष्ट्राय द्वाःस्थं मां प्रतिवेदय ॥द्वाःस्थ उवाच}
{}


\threelineshloka
{विदुरोऽयमनुप्राप्तो राजेन्द्र तव शासनात्}
{द्रुष्टुमिच्छति ते पादौ किं करोतु प्रशाधि माम् ॥धृतराष्ट्र उवाच}
{}


\threelineshloka
{प्रवेशय महाप्रज्ञं विदुरं दीर्घदर्शिनम्}
{अहं हि विदुरस्यास्य नाकल्पो जातु दर्शने ॥द्वाःस्थ उवाच}
{}


\threelineshloka
{प्रविशान्तःपुरं क्षत्तर्महारादजस्य धीमतः}
{न हि ते दर्शनेऽकल्पो जातु राजाऽब्रवीद्धि माम् ॥वैशंपायन उवाच}
{}


\twolineshloka
{ततः प्रविश्य विदुरो धृतराष्ट्रनिवेशनम्}
{अब्रवीत्प्राञ्जलिर्वाक्यं चिन्तयानं नराधिपम्}


\threelineshloka
{विदुरोऽहं महाप्राज्ञ संप्राप्तस्तव शासनात्}
{यदि किंचन कर्तव्यमयमस्मि प्रशाधि माम् ॥धृतराष्ट्र उवाच}
{}


\twolineshloka
{सञ्जयो विदुर प्राप्तो गर्हयित्वा च मां गतः}
{अजातशत्रोः श्वो वाक्यं सभामध्ये स वक्ष्यति}


\twolineshloka
{तस्याद्य कुरुवीरस्य न विज्ञातं वचो मया}
{तन्मे दहति गात्राणि तदकार्षीत्प्रजागरम्}


\twolineshloka
{आग्रतो दह्यमानस्य श्रेयो यदनुपश्यसि}
{तद्ब्रूहि त्वं हि नस्तात धर्मार्थकुशलो ह्यसि}


\twolineshloka
{यतः प्राप्तः सञ्जयः पाण्डवेभ्योन मे यथावन्मनसः प्रशान्तिः}
{सर्वेन्द्रियाण्यप्रकृतिं गतानिकिं वक्ष्यतीत्येव मेऽद्य प्रचिन्ता}


\threelineshloka
{` तन्मे ब्रूहि विदुर त्वं यथाव-न्मनीषितं सर्वमजातशत्रोः}
{यथा न नस्तात हितं भवेच्चप्रजाश्च सर्वाः सुखिता भवेयुः ॥'विदुर उवाच}
{}


\twolineshloka
{अभियुक्तं बलवता दुर्बलं हीनसाधनम्}
{हृतस्वं कामिनं चोरमाविशन्ति प्रजागराः}


\threelineshloka
{कच्चिदेतैर्महादोषैर्न स्पृष्टोऽसि नराधिप}
{कच्चिच्च परवित्तेषु गृद्ध्यन्न परितप्यसे ॥धृतराष्ट्र उवाच}
{}


\threelineshloka
{श्रोतुमिच्छामि ते धर्म्यं परं नैश्रेयसं वचः}
{अस्मिन्रादर्षिवंशे हि त्वमेकः प्राज्ञसंमतः ॥विदुर उवाच}
{}


\twolineshloka
{राजा लक्षणसंपन्नस्त्रैलोक्यस्याधिपो भवेत्}
{प्रेष्यस्ते प्रेषितश्चैव धृतराष्ट्र युधिष्ठिरः}


\twolineshloka
{विपरीततरश्च त्वं भागधेये न संमतः}
{अर्चिषां प्रक्षयाच्चैव धर्मात्मा धर्मकोविदः}


\twolineshloka
{आनृशंस्यादनुक्रोशाद्धर्मात्सत्यात्पराक्रमात्}
{गुरुत्वात्त्वयि संप्रेक्ष्य बहून्क्लेशांस्तितिक्षते}


\twolineshloka
{दुर्योधने सौबले च कर्णे दुःशासने तथा}
{एतेष्वैश्वर्यमाधाय कथं त्वं भूतिमिच्छसि}


\twolineshloka
{आत्मज्ञानं समारम्भस्तितिक्षा धर्मनित्यता}
{यमर्थान्नापकर्षन्ति स वै पण्डित उच्यते}


\twolineshloka
{` एकस्माद्वृक्षाद्यज्ञपात्राणि राज-न्स्रुक्व द्रोणी पेठनीपीडने च}
{एतस्माद्राजन्ब्रुवतो मे निबोधएकस्माद्वै जायतेऽसच्च सच्च}


\twolineshloka
{निषेवते प्रशस्तानि निन्दितानि न सेवते}
{अनास्तिकः श्रद्दधान एतत्पण्डितलक्षणम्}


\twolineshloka
{क्रोधो हर्षश्च दर्पश्च ह्रीः स्तम्भो मान्यमानिता}
{यमर्थान्नापकर्षन्ति स वै पण्डित उच्यते}


\twolineshloka
{यस्य कृत्यं न जानन्ति मन्त्रं वा मन्त्रितं परे}
{कृतमेवास्य जानन्ति स वै पण्डित उच्यते}


\twolineshloka
{यस्य कृत्यं न विघ्नन्ति शीतमुष्णं भयं रतिः}
{समृद्धिरसमृद्धिर्वा स वै पण्डित उच्यते}


\twolineshloka
{यस्य संसारिणी प्रज्ञा धर्मार्थावनुवर्तते}
{कामादर्थं वृणीते यः स वै पण्डित उच्यते}


\twolineshloka
{यथाशक्ति चिकीर्षन्ति यथाशक्ति च कुर्वते}
{किंचिदवमन्यन्ते नराः पण्डितबुद्धयः}


\twolineshloka
{क्षिप्रं विजानाति चिरं श्रृणोतिविज्ञाय चार्यं भजते न कामात्}
{नासंपृष्टो व्युपयुङ्क्ते परार्थेतत्प्रज्ञानं प्रथमं पण्डितस्य}


\twolineshloka
{नाप्राप्यमभिवाञ्छन्ति नष्टं नेच्छन्ति शोचितुम्}
{आपत्सु च न मुह्यन्ति नराः पण्डितबुद्धयः}


\twolineshloka
{निश्चित्य यः प्रक्रमते नान्तर्वसति कर्मणः}
{अवन्ध्यकालो वश्यात्मा स वै पण्डित उच्यते}


% Check verse!
आर्यकर्मणि रज्यन्ते भूतिकर्माणि कुर्वतेहितं च नाभ्यसूयन्ति पण्डिता भरतर्षभ
\twolineshloka
{न हृष्यत्यात्मसंमाने नावमानेन तप्यते}
{गाङ्गो ह्रद इवाक्षोभ्यो यः स पण्डित उच्यते}


\twolineshloka
{तत्त्वज्ञः सर्वभूतानां योगज्ञः सर्वकर्मणाम्}
{उपायज्ञो मनुष्याणां नरः पण्डित उच्यते}


\twolineshloka
{प्रवृत्तवाक् चित्रकथ ऊहवान्प्रतिभानवान्}
{आशु ग्रन्थस्य वक्ता च यः स पण्डित उच्यते}


\twolineshloka
{श्रुतं प्रज्ञानुगं यस्य प्रज्ञा चैव श्रुतानुगा}
{असंभिन्नार्यमर्यादः पण्डिताख्यां लभेत सः}


\twolineshloka
{` अर्थं महान्तमासाद्य विद्यामेश्वर्यमेव च}
{विचरत्यसमुन्नद्धो यः स पण्डित उच्यते '}


\twolineshloka
{अश्रुतश्च समुन्नद्धो दरिद्रश्च महामनाः}
{अर्थांश्चाकर्मणा प्रेप्सुर्मूढ इत्युच्यते बुधैः}


\twolineshloka
{स्वमर्थं यः परित्यज्य परार्थमनुतिष्ठति}
{मिथ्या चरति मित्रार्थे यश्च मूढः स उच्यते}


\twolineshloka
{अकामान्कामयति यः कामयानान्परित्यजेत्}
{बलवन्तं च यो द्वेष्टि तमाहुर्मूढचेतसम्}


\twolineshloka
{अमित्रं कुरुते मित्रं मित्रं द्वेष्टि हिनस्ति च}
{कर्म चारभते दुष्टं तमाहुर्मूढचेतसम्}


\twolineshloka
{संसारयति कृत्यानि सर्वत्र विचिकित्सते}
{चिरं करोति क्षिप्रार्थे स मूढो भरतर्षभ}


\twolineshloka
{श्राद्धं पितृभ्यो न ददाति दैवतानि न चार्चति}
{सुहृन्मित्रं न लभते तमाहुर्मूढचेतसम्}


\twolineshloka
{अनाहूतः प्रविशति अपृष्टो बहु भाषते}
{अविश्वस्ते विश्वसिति मूढचेता नराधमः}


\twolineshloka
{परं क्षिपति दोषेण वर्तमानः स्वयं तथा}
{यश्च क्रुध्यत्यनीशानः स च मूढतमो नरः}


\twolineshloka
{आत्मनो बलमाज्ञाय धर्मार्थपरिवर्जितम्}
{अलभ्यमिच्छन्नैष्कर्म्यान्मूढबुद्धिरिहोच्यते}


\twolineshloka
{अशिष्यं शास्ति यो राजन्यश्च शिष्यं न शास्ति च}
{कदर्यं भजते यश्च तमाहुर्मूढचेतसम्}


\twolineshloka
{एकः संपन्नमाश्नाति वस्ते वासश्च शोभनम्}
{योऽसंविभज्य भृत्येभ्यः को नृशंसतरस्ततः}


\twolineshloka
{एकः पापानि कुरुते फलं भुङ्क्ते महाजनः}
{भोक्तारो विप्रमुच्यन्ते कर्ता दोषेण लिप्यते}


\twolineshloka
{एकं हन्यान्न वा हन्यादिषुर्मुक्तो धनुष्मता}
{बुद्धिर्बुद्धिमतोत्सृष्टा हन्याद्राष्ट्रं सराजकम्}


\twolineshloka
{एकया द्वे विनिश्चित्य त्रींश्चतुर्भिर्वशे कुरु}
{पञ्च जित्वा विदित्वा षट् सप्त हित्वा सुखी भवा}


\twolineshloka
{एकं विषरसो हन्ति शस्त्रेणैकश्च वध्यते}
{सराष्ट्रं सप्रजं हन्ति राजानं मन्त्रपिप्लवः}


\twolineshloka
{एकः स्वादु न भुञ्जीत एकश्चार्थान्न चिन्तयेत्}
{एको न गच्छेदध्वानं नैकः सुप्तेषु जागृयात्}


\twolineshloka
{एकमेवाद्वितीयं तद्यद्राजन्नावबुध्यसे}
{सत्यं स्वगस्य सोपानं पारावारस्य नौरिव}


\twolineshloka
{एकः क्षमावतां दोषो द्वितीयो नोपपद्यते}
{यदेनं क्षमया युक्तमशक्तं मन्यते जनः}


\twolineshloka
{सोऽस्य दोषो न मन्तव्यः क्षमा हि परमं बलम्}
{क्षमा गुणो ह्यशक्तानां शक्तानां भूषणं क्षमा}


\twolineshloka
{क्षमा वशीकृतीर्लोके क्षमया किं न साध्यते}
{शान्तिखङ्गः करे यस्य किं करिष्यति दुर्जनः}


\twolineshloka
{अतृणे पतितो वह्निः स्वयमेवोपशाम्यति}
{अक्षमावान्परं दोषैरात्मानं चैव योजयेत्}


\twolineshloka
{एको धर्मः परं श्रेयः क्षमैका शान्तिरुत्तमा}
{विद्यैका परमा तृप्तिरहिंसैका सुखावहा}


\twolineshloka
{द्वाविमौ ग्रसते भूमिः सर्पो बलिशयानिव}
{राजानं चाविरोद्धारं ब्राह्मणं चाप्रवासिनम्}


\twolineshloka
{द्वे कर्मणी नरः कुर्वन्नस्मिँल्लोके विरोचते}
{अब्रुवन्परुषं किंचिदसतोऽनर्चयंस्तथा}


\twolineshloka
{द्वाविमौ पुरुषव्याघ्र परप्रत्ययकारिणौ}
{स्त्रियः कामितकामिन्यो मूर्खाः पूजितपूजकाः}


\twolineshloka
{द्वाविमौ कण्टकौ तीक्ष्णौ शरीरपरिशोषिणौ}
{यश्चाधनः कामयते यश्च कुप्यत्यनीश्वरः}


\twolineshloka
{द्वावेव न विराजेते विपरीतेन कर्मणा}
{गृहस्थश्च निरारम्भः कार्यवांश्चैव भिक्षुकः}


\twolineshloka
{द्वाविमौ पुरुषौ राजन्स्वर्गस्योपरि तिष्ठतः}
{प्रभुश्च क्षमया युक्तो दरिद्रश्च प्रदानवान्}


\twolineshloka
{न्यायागतस्य द्रव्यस्य बोद्धव्यौ द्वावतिक्रमौ}
{अपात्रे प्रतिपत्तिश्च पात्रे चात्प्रतिपादनम्}


\twolineshloka
{द्वावभ्यसि निवेष्टव्यौ गले बध्वा दृढां शिलाम्}
{धनवन्तमदातारं दरिद्रं चातपस्विनम्}


\twolineshloka
{द्वाविमौ पुरुषव्याघ्र सूर्यमण्डलभेदिनौ}
{परिव्राड्योगयुक्तश्च रणे चाभिमुखो हतः}


\twolineshloka
{त्रयो न्याया मनुष्याणां श्रूयन्ते भरतर्षभ}
{कनीयान्मध्यमः श्रेष्ठ इति वेदविदो विदुः}


\twolineshloka
{त्रिविधाः पुरुषा राजन्नुत्तमाधममध्यमाः}
{नियोजयेद्यथावत्तांस्त्रिविधेष्वेव कर्मसु}


\twolineshloka
{त्रय एवाधना राजन्भार्या दासस्तथा सुतः}
{यत्ते समधिगच्छन्ति यस्यैते तस्य तद्धनम्}


\twolineshloka
{हरणं च परस्वानां परदाराभिमर्शनम्}
{सुहृदश्च परित्यागस्त्रयो दोषाः क्षयावहाः}


\twolineshloka
{त्रिविधं नरकस्येदं द्वारं नाशनमात्मनः}
{कामः क्रोधस्तथा लोभस्तस्मादेतत्रयं त्यजेत्}


\twolineshloka
{वरप्रदानं राज्यं च पुत्रजन्म च भारत}
{शत्रोश्च मोक्षणं कृच्छ्रात्रीणि चैकं च तत्समम्}


\twolineshloka
{भक्तं च भजमानं च तवास्मीति च वादिनम्}
{त्रीनेताञ्छरणं प्राप्तान्विषमेऽपि कन सन्त्यजेत्}


\twolineshloka
{चत्वारि राज्ञा तु महाबलेनवर्ज्यान्याहुः पण्डितस्तानि विद्यात्}
{अल्पप्रज्ञैः सह मन्त्रं न कुर्या-न्न दीर्घसूत्रैरलसैश्चारणैश्च}


\twolineshloka
{चतवारि ते तात गृहे वसन्तुश्रियाऽभिजुष्टस्य गृहस्थधर्मे}
{वृद्धो ज्ञातिरवसन्नः कुलीनःसखा दरिद्रो भगिनी चानपत्या}


\twolineshloka
{चत्वार्याह महाराज साद्यस्कानि बृहस्पतिः}
{पृच्छते त्रिदशेन्द्राय तानीमानि निबोध मे}


\twolineshloka
{देवतानां च सङ्कल्पमनुभावं च धीमताम्}
{विनयं कृतविद्यानां विनाशं पापकर्मणाम्}


\twolineshloka
{चत्वारि कर्माण्यभयंकराणिभयं प्रयच्छन्त्ययथाकृतानि}
{मानाग्निहोत्रमुत मानमौनंमानेनाधीतमुत मानयज्ञः}


\twolineshloka
{पञ्चाग्नयो मनुष्येण परिचर्याः प्रयत्नतः}
{पिता माताग्निरात्मा च गुरुश्च भरतर्षभ}


\twolineshloka
{पञ्चैव पूजयँल्लोके यशः प्राप्नोति केवलम्}
{देवान्पितॄन्मनुष्यांश्च भिक्षूनतिथिपञ्चमान्}


\threelineshloka
{पञ्च त्वाऽनुगमिष्यन्ति यत्र यत्र गमिष्यसि}
{मित्राण्यमित्रा मध्यस्था उपजीव्योपजीविनः}
{}


\twolineshloka
{पञ्चेन्द्रियस्य मर्त्यस्य छिद्रं चेदेकमिन्द्रियम्}
{ततोऽस्य स्रवति प्रज्ञा दृतेः पात्रादिवोदकम्}


\threelineshloka
{षड्दोषाः पुरुषेणेह हातव्या भूतिमिच्छता}
{निद्रा तन्द्री भयं क्रोध आलस्यं दीर्घसूत्रता}
{}


\twolineshloka
{षडिमान्पुरुषो जह्याद्भिन्नां नावमिवार्णवे}
{अप्रवक्तारमाचार्यमनधीयानमृत्विजम्}


\twolineshloka
{अरक्षितारं राजानं भार्यां चाप्रियवादिनीम्}
{ग्रामकामं च गोपालं वनकामं च नापितम्}


\twolineshloka
{षडेव तु गुणाः पुंसा न हातव्याः कदाचन}
{सत्यं दानमनालस्यमनसूया क्षमा धृतिः}


\twolineshloka
{अर्थागमो नित्यमरोगिता चप्रिया च भार्या प्रियवादिनी च}
{वश्यश्च पुत्रोऽर्थकरी च विद्याषड्जीवलोकस्य सुखानि राजन्}


\twolineshloka
{षण्णामात्मनि नित्यानामैश्वर्यं योऽधिगच्छति}
{न स पापैः कुतोऽनर्थैर्युज्यते विजितेन्द्रियः}


\twolineshloka
{षडिमे षट्सु जीवन्ति सप्तमो नोपलभ्यते}
{चोराः प्रमत्ते जीवन्ति व्याधितेषु चिकित्सकाः}


\twolineshloka
{प्रमदाः कामयानेषु यजमानेषु याजकाः}
{राजा विवदमानेषु नित्यं मूर्खेषु पण्डिताः}


\twolineshloka
{षडिमानि विनश्यन्ति मुहूर्तमनवेक्षणात्}
{गावः सेवा कृषिर्भार्या विद्या वृषलसङ्गतिः}


\twolineshloka
{षडेते ह्यवमन्यन्ते नित्यं पूर्वोपकारिणम्}
{आचार्यं शिक्षिताः शिष्याः कृतदाराश्च मातरं}


\twolineshloka
{नारीं विगतकामास्तु कृतार्थाश्च प्रयोजकम्}
{नावं निस्तीर्णकान्तारा नातुराश्च चिकित्सकं}


\twolineshloka
{आरोग्यमानृण्यमविप्रवासःसद्भिर्मनुष्यैः सह संप्रयोगः}
{स्वप्रत्यया वृत्तिरभीतवासःषट् जीवलोकस्य सुखानि राजन्}


\twolineshloka
{ईर्षुर्धृणी नसन्तुष्टः क्रोधनो नित्यशङ्कितः}
{परभाग्योपजीवी च षडेते नित्यदुःखिताः}


\twolineshloka
{सप्त दोषाः सदा राज्ञा हातव्या व्यसनोदयाः}
{प्रायशो यैर्विनश्यन्ति कृतमूला अपीश्वराः}


\twolineshloka
{स्त्रियोऽक्षा मृगया पानं वाक्पारुष्यं च पञ्चमम्}
{महच्च दण्डपारुष्यमर्थदूषणमेव च}


\twolineshloka
{अष्टौ पूर्वनिमित्तानि नरस्य विनशिष्यतः}
{ब्राह्मणान्प्रथमं द्वेष्टि ब्राह्मणैश्च विरुध्यते}


\twolineshloka
{ब्राह्मणस्वानि चादत्ते ब्राह्मणांश्च जिघांसति}
{रमते निन्दया चैषां प्रशंसां नाभिनन्दति}


\twolineshloka
{नैनान्स्मरति कृत्येषु याचितश्चाभ्यसूयति}
{एतान्दोषान्नरः प्राज्ञो बुद्ध्या बुध्वा विसर्जयेत्}


\twolineshloka
{अष्टाविमानि हर्षस्य नवनीतानि भारत}
{वर्तमानानि दृश्यन्ते तान्येव स्वसुखान्यपि}


\twolineshloka
{समागमश्च सखिभिर्महांश्चैव धनागमः}
{पुत्रेण च परिष्वङ्गः सन्निपातश्च मैथुने}


\twolineshloka
{समये च प्रियालापः स्वयूथ्येषु समुन्नतिः}
{अभिप्रेतस्य लाभश्च पूजा च जनसंसदि}


\twolineshloka
{अष्टौ गुमाः पुरुषं दीपयन्तिप्रज्ञा च कौल्यं च दमः श्रुतं च}
{पराक्रमश्चाबहुभाषिता चदानं यथाशक्ति कृतज्ञता च}


\twolineshloka
{नवद्वारमिदं वेश्म त्रिस्थूणं पञ्चसाक्षिकम्}
{क्षेत्रज्ञाधिष्ठितं विद्वान्यो वेद स परः कविः}


\twolineshloka
{दश धर्मं न जानन्ति धृतराष्ट्र निबोध तान्}
{मत्तः प्रमत्त उन्मत्तः श्रान्तः क्रुद्धो बुभुक्षितः}


\twolineshloka
{त्वरमाणश्च लुब्धश्च भीतः कामी च ते दश}
{तस्मादेतेषु भावेषु न प्रसञ्जेत पण्डितः}


\twolineshloka
{अत्रैवोदाहरन्तीममितिहासं पुरातनम्}
{पुत्रार्थमसुरेन्द्रेण गीतं चैव सुधन्वना}


\twolineshloka
{यः काममन्यू प्रजहाति राजापात्रे प्रतिष्ठापयते धनं च}
{विशेषविच्छ्रुतवान्क्षिप्रकारीतं सर्वलोकः कुरुते प्रमाणम्}


\twolineshloka
{जानाति विश्वासयितुं मनुष्यान्विज्ञातदोषेषु दधाति दण्डम्}
{जानाति मात्रां च तथा क्षमां चतं तादृशं श्रीर्जुषते समग्रा}


\twolineshloka
{सुदुर्बलं नावजानाति कंचि-द्युक्तो रिपुं सेवते बुद्धिपूर्वम्}
{न विग्रहं रोचयते बलस्थैःकाले च यो विक्रमते स धीरः}


\twolineshloka
{प्राप्याप.. न व्यथते कदाचि-दुद्योगमान्वच्छति चाप्रमत्तः}
{दुःखं च काले सहते महात्माधुरन्धरस्तस्य जिताः सपत्नाः}


\twolineshloka
{अनर्थकं विप्रवासं गृहेभ्यःपापैः सन्धिं परदाराभिमर्शम्}
{दम्भं स्तैन्यं पैशुनं मद्यमानंन सेवते यः स सुखी सदैव}


\twolineshloka
{न संरम्भेणारभते त्रिवर्ग-माकारितः शंसति तत्त्वमेव}
{न मित्रार्थे रोचयते विवादंनापूजितः कुप्यति चाप्यमूढः}


\twolineshloka
{न योऽभ्यसूयत्सयनुकम्पते चन दुर्बलः प्रातिभाव्यं करोति}
{नात्याह किंचित्क्षमते विवादंसर्वत्र तादृग्लभते प्रशंसाम्}


\twolineshloka
{यो नोद्धतं कुरुते जातु वेषंन पौरुषेणापि विकत्थतेऽन्यान्}
{न मूर्च्छितः कटुकान्याह किंचि-त्प्रियं सदा तं कुरुते जनो हि}


\twolineshloka
{न वैरमुद्दिपयति प्रशान्तंन दर्पमारोहति शान्तिमेति}
{न दुर्गतोऽस्मीति करोत्यकार्यंतमार्यशीलं परमाहुरार्याः}


\twolineshloka
{न स्वे सुखे वै कुरुते प्रहर्षंनान्यस्य दुःखे भवति प्रहृष्टः}
{दत्त्वा न पश्चात्कुरुतेऽनुतापंस कथ्यते सत्पुरुषार्यशीलः}


\twolineshloka
{देशाचारान्समयाञ्जातिधर्मान्बुभूषते यः स परावरज्ञः}
{स यत्र तत्राभिगतः सदैवमहाजनस्याधिपत्यं करोति}


\twolineshloka
{दम्भं मोहं मत्सरं पापकृत्यंराजद्विष्टं पैशुनं पूगवैरम्}
{मत्तोन्मत्तैर्दुर्जनैश्चापि वादंयः प्रज्ञावान्वर्जयेत्स प्रधानः}


\twolineshloka
{दमं शौचं दैविकं मङ्गलानिप्रायश्चित्तान्विविधाँल्लोकवादान्}
{एतानि यः कुरुते नैत्यकानितस्योत्थानं देवता राधयन्ति}


\twolineshloka
{समैर्विवाहं कुरुते न हीनैःसमैः सख्यं व्यवहारं कथां च}
{गुणैर्विशिष्टांश्च पुरो दधातिविपश्चितस्तस्य नयाः सुनीताः}


\twolineshloka
{मितं भुङ्क्ते संविभज्याश्रितेभ्योमितं स्वपित्यमितं कर्म कृत्वा}
{ददात्यमित्रेष्वपि याचितः सं-स्तमात्मवन्तं प्रजहत्यनर्थाः}


\twolineshloka
{चिकीर्षितं विप्रकृतं च यस्यनान्ये जनाः कर्म जानन्ति किंचित्}
{मन्त्रे गुप्ते सम्यगनुष्ठिते चनाल्पोऽप्यस्य च्यवते कश्चिदर्थः}


% Check verse!
यः सर्वभूतप्रशमे निविष्टःसत्यो मृदुर्मानकृच्छुद्धभावःअतीव स ज्ञायते ज्ञातिमध्येमहामणिर्जात्य इव सप्रन्नः
% Check verse!
य आत्मनाऽपत्रपते भृशं नरःस सर्वलोकस्य गुरुर्भवत्युतअनन्तजेजाः सुमनाः समाहितःस तेजसा सूर्य इवावभासते
\twolineshloka
{वने जाताः शापदग्धस्य राज्ञःपाण्डोः पुत्राः पञ्च पञ्चेन्द्रकल्पाः}
{त्वयैव बाला वर्धिताः शिक्षिताश्चतवादेशं पालयन्त्याम्बिकेय}


\twolineshloka
{प्रदायैषामुचितं तातराज्यंसुखी पुत्रैः सहितो मोदमानः}
{न देवानां नापि च मानुषाणांभविष्यसि त्वं गर्हणीयो नरेन्द्र}


\chapter{अध्यायः ३४}
\twolineshloka
{धृतराष्ट्र उवाच}
{}


\twolineshloka
{जाग्रतो दह्यमानस्य यत्कार्यमनुपश्यसि}
{तद्ब्रूहि त्वं हि नस्तात धर्मार्थकुशलः शुचिः}


\twolineshloka
{तस्माद्यथावद्विदुर प्रशाधिप्रज्ञापूर्वं सर्वमजातशत्रोः}
{यन्मन्यसे पथ्यमदीनसत्वश्रेयस्करं ब्रूहि तद्वै कुरूणाम्}


\threelineshloka
{पापाशङ्की पापमेवानुपश्यन्पृच्छामि त्वां व्याकुलेनात्मनाहम्}
{कजे तन्मे ब्रूहि तत्पं यथाव-न्मनीषितं सर्वमजातशत्रोः ॥विदुर उवाच}
{}


\twolineshloka
{शुभं वा यदि वा पापं द्वेष्यं वा यदि वा प्रियम्}
{नापृष्टः कस्यचिद्ब्रूयाद्यः स नेच्छेत्पराभवम्}


\twolineshloka
{तस्माद्वक्ष्यामि ते राजन्हितं यत्स्यान्कुरून्प्रति}
{वचः श्रेयस्करं धर्म्यं ब्रुवतस्तन्निबोध मे}


\twolineshloka
{मिथ्योपेतानि कर्माणि सिद्ध्येतादीनि भारत}
{अनुपायप्रयुक्तानि मा स्म ते....मनः कृथाः}


\twolineshloka
{तथैव योगविहितं यत्तु कर्म न सिध्यति}
{उपाययुक्तं मेधावी न तत्र ग्लपयेन्मनः}


\twolineshloka
{अनुबन्धानवेक्षेत सानुबन्धेषु कर्मसु}
{संप्रधार्य च कुर्वीत सहसा न समाचरेत्}


\twolineshloka
{अनुबन्धं च संप्रेक्ष्य विपाकं चैव कर्मणाम्}
{उत्थानमात्मनश्चैव धीरः कुर्वीत वा न वा}


\twolineshloka
{यः प्रमाणं न जानाति स्थाने वृद्धौ तथा क्षये}
{कोशे जनपदे दण्डे न स राज्येऽवतिष्ठते}


\twolineshloka
{यस्त्वेतानि प्रमाणानि यथोक्तान्यनुपश्यति}
{युक्तो धर्मार्थयोर्ज्ञाने स राज्यमधिगच्छति}


\twolineshloka
{न राज्यं प्राप्तमित्येव वर्तितव्यमसांप्रतम्}
{श्रियं ह्यविनयो हन्ति जरा रूपमिवोत्तमम्}


\twolineshloka
{भक्ष्योत्तमप्रतिच्छन्नं मत्स्यो बडिशमायसम्}
{अन्नाभिलाषी ग्रसते नानुबन्धमवेक्षते}


\twolineshloka
{यच्छक्यं ग्रसितुं ग्रस्यं ग्रस्तं परिणमेच्च यत्}
{हितं च परिणामे यत्तदाद्यं भूतिमिच्छता}


\twolineshloka
{वनस्पतेरपक्वानि फलानि प्रचिनोति यः}
{स नाप्नोति रसं तेभ्यो बीजं चास्य विनश्यति}


\twolineshloka
{यस्तु पक्वमुपादत्ते काले परिणतं फलम्}
{फलाद्रसं स लभते बीजाच्चैव फलं पुनः}


\twolineshloka
{यथा मधु समादत्ते रक्षन्पुष्पाणि षट्पदः}
{तद्वदर्थान्मनुष्येभ्य आदद्यादविहिंसया}


\twolineshloka
{पुष्पं पुष्पं विचिन्वीत मूलच्छेदं न कारयेत्}
{मालाकार इवारामे न यथाङ्गारकारकः}


\twolineshloka
{किं नु मे स्यादिदं कृत्वा किं नु मे स्यादकुर्वतः}
{इति कर्माणि सञ्चिन्त्य कुर्याद्वा पुरुषो न वा}


\twolineshloka
{अनारभ्या भवन्त्यर्थाः केचिन्नित्यं यथाऽगताः}
{कृतः पुरुषकारोऽहि भवेद्येषु निरर्थकः}


\twolineshloka
{` अनर्थे चैव निरतमर्थे चैव पराङ्भुखम्}
{न तं भर्तारमिच्छन्ति षण्ढं पतिमिव स्त्रियः}


\twolineshloka
{प्रसादो निष्फलो यस्य क्रोधश्चापि निरर्थकः}
{न तं भर्तारमिच्छन्ति षण्ढं पतिमिव स्त्रियः}


\twolineshloka
{कांश्चिदर्थान्नरः प्राज्ञो लघुमूलान्महाफलान्}
{क्षिप्रमारभते कर्तुं न दीर्घयति तादृशान्}


\twolineshloka
{ऋजु पश्यति यः सर्वं चक्षुषा नु पिबन्निव}
{आसीनमपि तूष्णीकमनुरज्यन्ति तं प्रजाः}


\twolineshloka
{सुपुष्पितः स्यादफलः फलितः स्याद्दुरारुहः}
{अपक्वः पक्वसङ्काशो न तु शीर्येत कर्हिचित्}


\twolineshloka
{चक्षुषा मनसा वाचा कर्मणा च चतुर्विधम्}
{प्रसादयति यो लोकं तं लोकोऽनुप्रसीदति}


\twolineshloka
{यस्मात्रस्यन्ति भूतानि मृगव्याधान्मृगा इव}
{सागरान्तामपि महीं लब्ध्वा स परिहीयते}


\twolineshloka
{पितृपैतामहं राज्यं प्राप्यापि स्वेन कर्मणा}
{वायुरभ्रमिवासाद्य भ्रंशयत्यनये स्थितः}


\twolineshloka
{धर्ममाचरतो राज्ञः सद्भिश्चरितमादितः}
{वसुधा वसुसंपूर्णा वर्धते भूतिवर्धनी}


\threelineshloka
{अथ संत्यजतो धर्ममधर्मं चानुतिष्ठतः}
{प्रतिसंवेष्टते भूमिरग्नौ चर्माहितं यथा}
{}


\threelineshloka
{य एव यत्नः क्रियते परराष्ट्रविमर्दने}
{स एव यत्नः कर्तव्यः स्वराष्ट्रपरिपालने}
{}


\twolineshloka
{धर्मेण राज्यं विन्देत धर्मेण परिपालयेत्}
{धर्ममूलां श्रियं प्राप्य न जहाति न हीयते}


\twolineshloka
{अप्युन्मत्तात्प्रलपतो बालाच्च परिजल्पतः}
{सर्वतः सारमादद्यादश्मभ्य इव काञ्चनम्}


\twolineshloka
{सुव्याहृतानि महतां सुकृतानि ततस्ततः}
{सञ्चिन्वन्धीर आसीत शिलाहारी शिलं यथा}


\twolineshloka
{गन्धेन गावः पश्यन्ति वेदैः पश्यन्ति ब्राह्मणाः}
{चारैः पश्यन्ति राजानश्चक्षुर्भ्यामितरे जनाः}


\twolineshloka
{भूयांसं लभते क्लेशं या गौर्भवति दुर्दुहा}
{अथ या सुदुघा राजन्नैव तां वितुदन्त्यपि}


\twolineshloka
{यदतप्तं प्रणमति न तत्सन्तापमर्हति}
{यच्च स्वयं नतं दारु न तत्संनामयेद्बुधः}


\twolineshloka
{एतयोपमया धीरः सन्नमेत बलीयसे}
{इन्द्राय स प्रणमते नमते यो बलीयसे}


\twolineshloka
{पर्जन्यनाथाः पशवो राजानो मन्त्रिबान्धवाः}
{पतयो बान्धवाः स्त्रीणां ब्राह्मणा वेदबान्धवाः}


\twolineshloka
{सत्येन रक्ष्यते धर्मो विद्या योगेन रक्ष्यते}
{मृजया रक्ष्यते रूपं कुलं वृत्तेन रक्ष्यते}


\twolineshloka
{मानेन रक्ष्यते धान्यमश्वान्रक्षेदनुक्रमात्}
{अभीक्ष्णदर्शनाद्गावः स्त्रियो रक्ष्याः कुचेलतः}


\twolineshloka
{न कुलं वृत्तहीनस्य प्रमाणमिति मे मतिः}
{अन्तेष्वपि हि जातानां वृत्तमेव विशिष्यते}


\twolineshloka
{य ईर्षुः परवित्तेषु रूपे वीर्ये कुलान्वये}
{सुखसौभाग्यसत्कारे तस्य व्याधिरनन्तकः}


\twolineshloka
{अकार्यकरणाद्भीतः कार्याणां च विवर्जनात्}
{अकाले मन्त्रभेदाच्च येन माद्येन्न तत्पिबेत्}


\twolineshloka
{विद्यामदो धनमदस्तृतीयोऽभिजनो मदः}
{मदा एतेऽवलिप्तानामेत एव सतां दमाः}


\twolineshloka
{असन्तोऽभ्यर्थिताः सद्भिः क्वचित्कार्ये कदाचन}
{मन्यन्ते सन्तमात्मानमसन्तमपि विश्रुतम्}


\twolineshloka
{गतिरात्मवतां सन्तः सन्त एव सतां गतिः}
{असतां च गतिः सन्तो न त्वसन्तः सतां गतिः}


\twolineshloka
{जिता सभा वस्त्रवता मिष्टाशा गोमता जिता}
{अध्वा जितो यानवता सर्वं शीलवता जितम्}


\twolineshloka
{शीलं प्रधानं पुरुषे तद्यस्येह प्रणश्यति}
{न तस्य जीवितेनार्थो न धनेन न बन्धुभिः}


\twolineshloka
{आढ्यानां मांसपरमं मध्यानां गोरसोत्तरम्}
{तैलोत्तरं दरिद्राणां भोजनं भरतर्षभ}


\twolineshloka
{संपन्नतरमेवान्नं दरिद्रा भुञ्जते सदा}
{श्रुत्स्वादुतां जनयति सा चाढ्येषु सुदुर्लभा}


\twolineshloka
{प्रायेण श्रीमतां लोके भोक्तुं शक्तिर्न विद्यते}
{जीर्यन्त्यपि हि काष्ठानि दरिद्राणां महीपते}


\twolineshloka
{अवृत्तिर्भयमन्त्यानां मध्यानां मरणाद्भयम्}
{उत्तमानां तु मर्त्यानामवमानात्परं भयम्}


\twolineshloka
{ऐश्वर्यमदपापिष्ठा मदाः पानमदादयः}
{ऐश्वर्यमदमत्तो हि नो पतित्वाऽवबुध्यते}


\twolineshloka
{इन्द्रियौरिन्द्रियार्थेषु वर्तमानैरनिग्रहैः}
{तैरयं ताप्यते लोको नक्षत्राणि ग्रहैरिव}


\twolineshloka
{यो जितः पञ्चवर्गेण सहजेनात्मकर्षिणा}
{आपदस्तस्य वर्धन्ते शुक्लपक्ष इवोडुराट्}


\twolineshloka
{अविजित्य य आत्मानममात्यान्विजिगीषते}
{अमित्रान्वा जितामात्यः सोऽवशः परिहीयते}


\twolineshloka
{आत्मानमेव प्रथमं द्वेष्यरूपेण योजयेत्}
{ततोऽमात्यानमित्रांश्च न मोघं विजिगीषते}


\twolineshloka
{वश्येन्द्रियं जितामात्यं धृतदण्डं विकारिषु}
{परीक्ष्यकारिणं धीरमत्यन्तं श्रीर्निषेवते}


\twolineshloka
{रथः शरीरं पुरुषस्य राज-न्नात्मा नियन्तेन्द्रियाण्यस्य चाश्वाः}
{तैरप्रमत्तः कुशली सदश्वै-र्दान्तैः सुखं याति रथीव धीरः}


\twolineshloka
{एतान्यनिगृहीतानि व्यापादयितुमप्यलम्}
{अविधेया इवादान्ताः सरथं सारथिं हयम्}


\twolineshloka
{अनर्थमर्थतः पश्यन्नर्थं चैवाप्यनर्थतः}
{इन्द्रियैरजितैर्बालः सुदुःखं मन्यते सुखम्}


\twolineshloka
{धर्मार्थौ यः परित्यज्य स्यादिन्द्रियवशानुगः}
{श्रीप्राणधनदारेभ्यः क्षिप्रं स परिहीयते}


\twolineshloka
{अर्थानामीश्वरो यः स्यादिन्द्रियाणामनीश्वरः}
{इन्द्रियाणामनैश्वर्यादैश्वार्याद्भूश्यते हि सः}


\twolineshloka
{आत्मनात्मानमन्विच्छेन्मनोबुद्धीन्द्रियैर्यतैः}
{आत्मा ह्येवात्मनो बन्धुरात्मैव रिपुरात्मनः}


\twolineshloka
{बन्धुरात्मात्मनस्तस्य येनैवात्मात्मना जितः}
{स एव नियतो बन्धुः स एव नियतो रिपुः}


\twolineshloka
{क्षुद्राक्षेणेव जालेन झपावपिहितावुरू}
{कामश्च राजन्क्रोधश्च तौ प्रज्ञानं विलुम्पतः}


\twolineshloka
{समवेक्ष्येह धर्मार्थौ संभारान्योऽधिगच्छति}
{स वै संभृतसंभारः सततं सुखमेधते}


\twolineshloka
{यः पञ्चाभ्यन्तराञ्शत्रूनविजित्य मनोमयान्}
{जिगीषति रिपूनन्यान्रिपवोऽभिभवन्ति तम्}


\twolineshloka
{दृश्यन्ते हि महात्मानो बध्यमानाः स्वकर्मभिः}
{इन्द्रियाणामनीशत्वाद्राजानो राज्यविभ्रमैः}


\twolineshloka
{असंत्यागात्पापकृतामपापां-स्तुल्यो दण्डः स्पृशते मिश्रभावात्}
{शुष्केणार्द्रं दह्यते मिश्रभावा-त्तस्मात्पापैः सह सन्धिं न कुर्यात्}


\twolineshloka
{निजानुत्पततः शत्रून्पश्च पञ्चप्रयोजनान्}
{यो मोहान्न निगृह्णाति तमापद्ग्रसते नरम्}


\twolineshloka
{अनसूयार्जवं शौचं सन्तोषः प्रियवादिता}
{दमः सत्यमनायासो न भवन्ति दुरात्मनाम्}


\twolineshloka
{आत्मज्ञानमानायासस्तितिक्षा धर्मनित्यता}
{वाक्चैव गुप्ता दानं च नैतान्यन्त्येषु भारत}


\twolineshloka
{आक्रोशपरिवादाभ्यां विहिंसन्त्यबुधा बुधान्}
{वक्ता पापमुपादत्ते क्षममाणो विमुच्यते}


\twolineshloka
{हिंसा बलमसाधूनां राज्ञां दण्डविधिर्बलम्}
{शुश्रूपा तु वलं स्त्रीणां क्षमा गुणवतां बलम्}


\twolineshloka
{वाक्संयमो हि नृपते सुदुष्करतमो मतः}
{अर्थवच्च विचित्रं च न शक्यं बहु भाषितुम्}


\twolineshloka
{अभ्यावहति कल्याणं विविधं वाक् सुभाषिता}
{सैव दुर्भाषिता राजन्ननर्थायोपपद्यते}


\twolineshloka
{रोहते सायकैर्विद्धं वनं परशुना हतम्}
{वाचा दुरुक्तं बीभत्सं न संरोहति वाक्क्षतम्}


\twolineshloka
{कर्णिनालीकनाराचान्निर्हरन्ति शरीरतः}
{वाक्छशल्यस्तु न निर्हर्तुं शक्यो हृदिशयो हि सः}


\twolineshloka
{वाक्सायका वदनान्निष्पतन्तियैराहतः शोचति रात्र्यहानि}
{परस्य नामर्मसु ते पतन्तितान्पण्डितो नावसृजेत्परेभ्यः}


\twolineshloka
{यस्मै देवाः प्रयच्छन्ति पुरुषाय पराभवम्}
{बुद्धिं तस्यापकर्षन्ति सोऽवाचीनानि पश्यति}


\twolineshloka
{बुद्धौ कलुषभूतायां विनाशे प्रत्युपस्थिते}
{अनयो नयसङ्काशो हृदयान्नापसर्पति}


\twolineshloka
{सेयं बुद्धिः परीता ते पुत्राणां भरतर्षभ}
{पाण्डवानां विरोधेन न चैनानवबुध्यसे}


\twolineshloka
{राजा लक्षणसंपन्नस्त्रैलोक्यस्यापि यो भवेत्}
{शिष्यस्ते शासिता सोऽस्तु धृतराष्ट्र युधिष्ठिरः}


\twolineshloka
{अतीव सर्वान्पुत्रांस्ते भागधेयपुरस्कृतः}
{तेजसा प्रज्ञया चैव युक्तो धर्मर्थतत्त्ववित्}


\twolineshloka
{अनुक्रोशादानृशंस्याद्योऽसौ धर्मभृतां वरः}
{गौरवात्तव राजेन्द्र बहून्क्लेशांस्तितिक्षति}


\chapter{अध्यायः ३५}
\twolineshloka
{धृतराष्ट्र उवाच}
{}


\threelineshloka
{ब्रूहि भूयो महाबुद्धे धर्मार्थसहितं वचः}
{श्रृण्वतो नास्ति मे तृप्तिर्विचित्राणीह भाषसे ॥विदुर उवाच}
{}


\threelineshloka
{सर्वतीर्थेषु वा स्नानं सर्वभूतेषु चार्जवम्}
{उभे त्वेते समे स्यातामार्जवं वा विशिष्यते}
{}


\twolineshloka
{आर्जवं प्रतिपद्यस्व पुत्रेषु सततं विभो}
{इह कीर्ति परां प्राप्य प्रेत्य स्वर्गमवाप्स्यसि}


\twolineshloka
{यावत्कीर्तिर्मनुष्यस्य पुण्या लोके प्रगीयते}
{तावत्स पुरुषव्याघ्र स्वर्गलोके महीयते}


\twolineshloka
{अत्राप्युदाहरन्तीममितिहासं पुरातनम्}
{विरोचनस्य संवादं केशिन्यर्थे सुधन्वना}


\twolineshloka
{स्वयंवरे स्थिता कन्या केशिनी नाम नामतः}
{रूपेणाप्रतिमा राजन्विशिष्टपतिकाम्यया}


\threelineshloka
{विरोचनोऽथ दैतेयस्तदा तत्राजगाम ह}
{प्राप्तुमिच्छंस्ततस्तत्र दैत्येन्द्रं प्राह केशिनी ॥केशिन्युवाच}
{}


\twolineshloka
{किं ब्राह्मणाः स्विच्छ्रेयांसो दितिजाः स्विद्विरोचनअथ केन स्म पर्यङ्कं सुधन्वा नाधिरोहति ॥विरोचन उवाच}
{}


\threelineshloka
{प्राजापत्यास्तु वै श्रेष्ठा वयं केशिनि सत्तमाः}
{अस्माकं खल्विमे लोकाः के देवाः के द्विजातयःकेशिन्युवाच}
{}


\threelineshloka
{इहैवावां प्रतीक्षाव उपस्थाने विरोचन}
{सुधन्वा प्रातरागन्ता पश्येयं वां समागतौ ॥विरोचन उवाच}
{}


\threelineshloka
{तथा भद्रे करिष्यामि यथा त्वं भीरु भाषसे}
{सुधन्वानं च मां चैव प्रातर्द्रष्टासि सङ्गतौ ॥विदुर उवाच}
{}


\threelineshloka
{अतीतायां च शर्वर्यामुदिते सूर्यमण्डले}
{अथाजगाम तं देशं सुधन्वा राजसत्तम}
{विरोचनो यत्र विभो केशिन्या सहितः स्थितः}


\fourlineindentedshloka
{सुधन्वा च समागच्छत्प्राह्लादिं केशिनीं तथा}
{समागतं द्विजं दृष्ट्वा केशिनी भरतर्षभ}
{प्रत्युत्थायासनं तस्मै पाद्यमर्घ्यं ददौ पुनः ॥ 5-35-14a` इतिहोवाच वचनं विरोचनमनुत्तमम्}
{आस्स्व तल्पे हि सौवर्णे प्राह्लादे ब्राह्मणस्त्वहम् ॥'}


\threelineshloka
{अन्वालभे हिरण्मयं प्राह्लादे ते वरासनम्}
{एकत्वमुपसंपन्नो नत्वासोऽहं त्वया सह ॥विरोचन उवाच}
{}


\threelineshloka
{तवार्हते तु फलकं कूर्चं वाप्यथवा बृसी}
{सुधन्वन्न त्वमर्होऽसि मया सह समासनम् ॥सुधन्वोवाच}
{}


\twolineshloka
{पितापुत्रौ सहासीतां द्वौ विप्रौ क्षत्रियावपि}
{वृद्धौ वैश्यौ च शूद्रौ च न त्वन्यावितरेतरम्}


\threelineshloka
{पिता हि ते समासीनमुपासीतैव मामधः}
{बालः सुखैधितो गेहे न त्वं किञ्चन बुध्यसे ॥विरोचन उवाच}
{}


\threelineshloka
{हिरण्यं च गवाश्वं च यद्वित्तमसुरेषु नः}
{सुधन्वन्विपणे तेन प्रश्नं पृच्छाव ये वुदुः ॥सुधन्वोवाच}
{}


\threelineshloka
{हिरण्यं च गवाश्वं च तवैवास्तु विरोचन}
{प्राणयोस्तु पणं कृत्वा प्रश्नं पृच्छाव ये विदुः ॥विरोचन उवाच}
{}


\threelineshloka
{आवां कुत्र गमिष्यावः प्राणयोर्विपणे कृते}
{न तु देवेष्वहं स्थाता न मनुष्येषु कर्हिचित् ॥सुधन्वोवाच}
{}


\threelineshloka
{पितरं ते गमिष्यावः प्राणयोर्विपणे कृपे}
{पुत्रस्यापि स हेतोर्हि प्रह्लादो नानृतं वदेत् ॥विदुर उवाच}
{}


\threelineshloka
{एवं कृतपणौ क्रुद्धौ तत्राभिजग्मतुस्तदा}
{विरोचनसुधन्वानौ प्रह्लादौ यत्र तिष्ठति ॥प्रह्लाद उवाच}
{}


\twolineshloka
{इमौ तौ संप्रदृश्येते याभ्यां न चरितं सह}
{आशीविषाविव क्रुद्धावेकमार्गाविहागतौ}


\threelineshloka
{किं वै सहैवं चरथो न पुरा चरथः सह}
{विरोचनैतत्पृच्छामि किं ते सख्यं सुधन्वना ॥विरोचन उवाच}
{}


\threelineshloka
{न मे सुधन्वना सख्यं प्राणयोर्विपणावहे}
{प्रह्लाद तत्त्वं पृच्छामि मा प्रश्नमनृतं वदेः ॥प्रह्लाद उवाच}
{}


\threelineshloka
{उदकं मधुपर्कं चाप्यानयन्तु सुधन्वने}
{ब्रह्मन्नभ्यर्चनीयोऽसि श्वेता गौः पीवरी कृता ॥सुधन्वोवाच}
{}


\fourlineindentedshloka
{उदकं मधुपर्कं च प्रश्नवाचार्पितं मम}
{प्रह्लद त्वं तु मे तथ्यं प्रश्नं प्रब्रूहि पृच्छतः}
{किं ब्राह्मणाः स्विच्छ्रेयांस उताहो स्विद्विरोचनः प्रह्लाद उवाच}
{}


\twolineshloka
{` न कल्माषो न कपिलो न कृष्णो न च लोहितः}
{अणीयान्क्षुरधारायाः को धर्मं वक्तुमर्हिति}


\fourlineindentedshloka
{अभिवाद्यो भवान्ब्रह्मन्साक्ष्ये चैव नियोजितः}
{'पुत्र एको मम ब्रह्मंस्त्वं च साक्षादिहास्थितः}
{तयोर्विवदतोः प्रश्नं कथमस्मद्विधो वदेत् ॥सुधन्वोवाच}
{}


\threelineshloka
{` यदेतत्त्वं न वक्ष्यसि यदि वापि विवक्ष्यसि}
{प्रह्लाद प्रश्नमतुलं मूर्धा ते विफलिष्यति ॥विदुर उवाच}
{}


\threelineshloka
{आदित्येन सहायान्तं प्रह्लादो हंसमब्रवीत्}
{धृतराष्ट्र महाप्राज्ञ सर्वज्ञं सर्वदर्शिनम् ॥प्रह्लाद उवाच}
{}


\threelineshloka
{पुत्रो वाऽन्यो भवेद्ब्रह्मन्साक्ष्ये चापि भवेत्स्थितः}
{तयोर्विवदतोर्हंस कथं धर्मः प्रवर्तते ॥हंस उवाच}
{'}


\threelineshloka
{गां प्रदद्यादौरसाय यद्वान्यत्स्यात्प्रियं धनम्}
{द्वयोर्विवदतो राजन्प्रश्नं सत्यं यथा वदेत् ॥प्रह्लाद उवाच}
{}


\twolineshloka
{अथ यो नैव प्रब्रूयात्सत्यं वा यदि वाऽनृतम्}
{हंस तत्वं च पृच्छामि कियदेनः करोति सः ॥`हंस उवाच}


\threelineshloka
{पृष्टो धर्मं न विब्रूयाद्गोकर्णशिथिलं चरन्}
{धर्माद्भ्रश्यति राजंस्तु नास्य लोकोऽस्ति न प्रजाः}
{}


\twolineshloka
{धर्म एतान्संरुजति यथा नद्यस्तु कूलजान्}
{ये धर्ममनुपश्यन्तस्तूष्णीं ध्यायन्त आसते}


\twolineshloka
{श्रेष्ठोऽर्धं तु हरेत्तत्र भवेत्पादश्च कर्तरि}
{पादस्तेषु सभासत्सु यत्र निन्द्यो न निन्द्यते}


\threelineshloka
{अनेना भवति श्रेष्ठो मुच्यन्तेऽपि सभासदः}
{कर्तारमेनो गच्छेद्वा निन्द्यो यत्र हि निन्द्यते ॥प्रह्लाद उवाच}
{}


\threelineshloka
{मोहाद्वा चैव कामाद्वा मिथ्यावादं यदि ब्रुवन्}
{धृतराष्ट्र तत्वं पृच्छामि दुर्विवक्ता तु किं वसेत् ॥हंस उवाच}
{'}


\twolineshloka
{यां रात्रिमधिविन्ना स्त्री यां चैवाक्षपराजितः}
{यां च भाराभितप्ताङ्गो दुर्विवक्ता तु तां वसेत्}


\twolineshloka
{नगरे प्रतिरुद्धः सन्बहिर्द्वारे बुभुक्षितः}
{अमित्रान्भूयसः पश्यन्दुर्विवक्ता तु तां वसेत्}


\twolineshloka
{यां च रात्रिमभिद्रुग्धो यां च मित्रे प्रियेऽनृते}
{सर्वस्वेन च हीनो यो दुर्विवक्ता तु तां वसेत्}


\twolineshloka
{पञ्च पश्वनृते हन्ति दश हन्ति गवानृते}
{शतमश्वानृते हन्ति सहस्रं पुरुषानृते}


\threelineshloka
{हन्ति जातानजातांश्च हिरण्यार्थेऽनृतं वदन्}
{सर्वं भूम्यनृते हन्ति मा स्म भूम्यनृतं वदेः ॥प्रह्लाद उवाच}
{}


\twolineshloka
{मत्तः श्रेयानङ्गिरा वै सुधन्वा त्विद्वरोचन}
{मातास्य श्रेयसी मातुस्तस्मात्त्वं तेन वै जितः}


\threelineshloka
{विरोचन सुधन्वाऽयं प्राणानामीश्वरस्तव}
{सुधन्वन्पुनरिच्छामि त्वया दत्तं विरोचनम् ॥सुधन्वोवाच}
{}


\twolineshloka
{यद्धर्ममवृणीथास्त्वं न कामादनृतं वदीः}
{पुनर्ददामि ते पुत्रं तस्मात्प्रह्लाद दुर्लभम्}


\threelineshloka
{एष प्रह्लाद पुत्रस्ते मया दत्तो विरोचनः}
{पादप्रक्षालनं कुर्यात्कुमार्याः सन्निधौ मम ॥विदुर उवाच}
{}


\twolineshloka
{तस्माद्राजेन्द्र भूम्यर्थे नानृतं वक्तुमर्हसि}
{मा गमः ससुतामात्यो नाशं पुत्रार्थमब्रुवन्}


\twolineshloka
{न देवा यष्टिमादाय रक्षन्ति पशुपालवत्}
{यं तु रक्षितुमिच्छन्ति बुद्ध्या संयोजयन्ति तम्}


\twolineshloka
{यथायथा हि पुरुषः कल्याणे कुरुते मनः}
{तथातथाऽस्य सर्वार्थाः सिद्ध्यन्ते नात्र संशयः}


\twolineshloka
{नैनं छन्दांसि वृजिनात्तारयन्तिमायाविनं मायया वर्तमानम्}
{नीडं शकुन्ता इव जातपक्षा-श्छन्दांस्येनं प्रजहत्यन्तकाले}


\twolineshloka
{मद्यपानं कलहं पूगवैरंभार्यापत्योरन्तरं ज्ञातिभेदम्}
{राजद्विष्टं स्त्रीपुंसयोर्विवादंवर्ज्यान्याहुर्यश्च पन्थाः प्रदुष्टः}


\twolineshloka
{सामुद्रिकं वणिजं चोरपूर्वंशलाकधूर्तं च चिकित्सकं च}
{अरिं च मित्रं च कुशीलवं चनैतान्साक्ष्ये त्वधिकृर्वीत सप्त}


\twolineshloka
{मानाग्निहोत्रमुत मानमौनंमानेनाधीतमुत मानयज्ञः}
{एतानि चत्वार्यभयङ्कराणिभयं प्रयच्छन्त्ययथाकृतानि}


\twolineshloka
{अगारदाही गरदः कुण्डाशी सोमविक्रयी}
{पर्वकारश्च सूची च मित्रध्रुक्पारदारिकः}


\twolineshloka
{भ्रूणहा गुरुतल्पी च यश्च स्यात्पानपो द्विजः}
{अतितीक्ष्णश्च कारुश्च नास्तिको वेदनिन्दकः}


\twolineshloka
{स्रुवप्रग्रहणो व्रात्यः कीनाशश्चात्मवानपि}
{रक्षेत्युक्तश्च यो हिंस्यात्सर्वे ब्रह्महभिः समाः}


\twolineshloka
{तृणोल्कया ज्ञायते जातरूपंवृत्तेन भद्रो व्यवहारेण साधुः}
{शूरो भयेष्वर्थकृच्छ्रेषु धीरःकृच्छ्रेष्वापत्सु सुहृदश्चारयश्च}


\threelineshloka
{जरा रूपं हरति हि धैर्यमाशामृत्युः प्राणान्धर्मचर्यामसूया}
{क्रोधः श्रियं शीलमनार्यसेवाह्रियं कामः सर्वमेवाभिमानः}
{}


\twolineshloka
{न क्रोधिनोऽर्थो न नृशंमस्य मित्रंक्रृगस्य न स्त्री सुखिनो न विद्या}
{न कामिनो ह्रीरलसस्य न श्रीःसर्वं तु न स्यादनवस्थितस्य}


\twolineshloka
{श्रीर्मङ्गलान्प्रभवति प्रागल्भ्यात्संप्रवर्धते}
{दाक्ष्यात्तु कुरुते मूलं संयमान्प्रतितिष्ठति}


\twolineshloka
{अष्टौ गुणाः पुरुषं दीपयन्तिप्रज्ञा च कौल्यं च दमः श्रुतं च}
{पराक्रमश्चाबहुभाषिता चदानं यथाशक्ति कृतज्ञता च}


% Check verse!
एतान्गुणांस्तत महानुभावा-नेको गुणः संश्रयते प्रसह्यराजा यदा सत्कुरुते मनुष्यंसर्वान्गुणानेप गुणोतिभाति
\twolineshloka
{अष्टौ नृपेमानि मनुष्यलोकेस्वर्गस्य लोकस्य निदर्शनानि}
{चत्वार्येपामन्ववेतानि सद्भि-श्चत्वारि चैपामनुयान्ति सन्तः}


\twolineshloka
{यज्ञो दानमध्ययनं तपश्चचत्वार्येतान्यन्ववेतानि सद्भिः}
{दमः सत्यमार्जवमानृशंस्यंचत्वार्येतान्यनुयान्ति सन्तः}


\twolineshloka
{इज्याध्ययनदानानि तपः सत्यं क्षमा घृणा}
{अलोभ इति मार्गोऽयं धर्मस्याष्टविधः स्मृतः}


\twolineshloka
{तत्र पूर्वचतुर्वर्गो दम्भार्थमपि सेव्यते}
{उत्तरश्च चतुर्वर्गो नामहात्मसु तिष्ठति}


\twolineshloka
{न सा सभा यत्र न सन्ति वृद्धान ते वृद्धा ये न वदन्ति धर्मम्}
{नासौ धर्मो यत्र न सत्यमस्तिन तत्सत्यं यच्छलेनानुविद्धम्}


\twolineshloka
{सत्यं रूपं श्रुतं विद्या कौल्यं शीलं बलं धनम्}
{शौर्यं च चित्रभाष्यं च दशेमे स्वर्गयोनयः}


\twolineshloka
{पापं कुर्वन्पापकीर्तिः पापमेवाश्रुते फलम्}
{पुण्यं कुर्वन्पापकीर्तिः पुण्यमत्यन्तमश्रुते}


\twolineshloka
{तस्मात्पापं न कुर्वीत पुरुषः शंसितव्रतः}
{पापं प्रज्ञां नाशयति क्रियमाणं पुःन पुःनः}


\twolineshloka
{वृद्धप्रज्ञः पापमेव नित्यमारभते नरः}
{पुण्यं प्रज्ञां वर्धयति क्रियमाणं पुनः पुनः}


\threelineshloka
{वृद्धप्रज्ञः पुण्यमेव नित्यमारभते नरः}
{पुण्यं कुर्वन्पुण्यकीर्तिः पुण्यं स्थानं स्म गच्छति}
{तस्मात्पुण्यं निषेवेत पुरुषः सुसमाहितः}


\twolineshloka
{असूयको दन्दशूको निष्ठुरो वैरकृच्छठः}
{स कृच्छ्रं महदाप्नोति नचिरात्पापमाचरन्}


\twolineshloka
{अनसूयुः कृतप्रज्ञः शोभनान्याचरन्सदा}
{नकृच्छ्रं महदाप्नोति सर्वत्र च विरोचते}


\twolineshloka
{प्रज्ञामेवागमयति यः प्राज्ञेभ्यः स पण्डितः}
{प्राज्ञो ह्यवाप्य धर्मार्थौ शक्नोति सुखमेधितुम्}


\twolineshloka
{दिवसेनैव तत्कुर्याद्येन रात्रौ सुखं वसेत्}
{अष्टमासेन तत्कुर्याद्येन वर्षाः सुखं वसेत्}


\twolineshloka
{पूर्वे वयसि तत्कुर्याद्येन वृद्धः सुखं वसेत्}
{यावज्जीवं तु तत्कुर्याद्येन प्रेत्य सुखं वसेत्}


\twolineshloka
{जीर्णमन्नं प्रशंसन्ति भार्यां च गतयौवनाम्}
{शूरं विजितसङ्ग्रामं गतपारं तपस्विनम्}


\twolineshloka
{धनेनाधर्मलब्धेन यच्छिद्रमपिधीयते}
{असंवृतं तद्भवति ततोऽन्यदवदीर्यते}


\twolineshloka
{गुरुरात्मवतां शास्ता शास्ता राजा दुरात्मनाम्}
{अन्तः प्रच्छन्नपापानां शास्ता वैवस्वतो यमः}


\twolineshloka
{ऋषीणां च नदीनां च कुलानां च महात्मनाम्}
{प्रभवो नाधिगन्तव्यः स्त्रीणां दुश्चरितस्य च}


\twolineshloka
{द्विजातिपूजाभिरतो दाता ज्ञातिषु चार्जवी}
{क्षत्रियः शीलभाग्राजंश्चिरं पालयते महीम्}


\twolineshloka
{सुवर्णपुष्पां पृथिवीं चिन्वन्ति पुरुषास्त्रयः}
{शूरश्च कृतविद्यश्च यश्च जानाति सेवितुम्}


\twolineshloka
{बुद्धिश्रेष्ठानि कर्माणि बाहुमध्यानि भारत}
{तानि जङ्घाजघन्यानि भारप्रत्यवराणि च}


\twolineshloka
{दुर्योधनेऽथ शकुनौ मूढे दुःशासने तथा}
{कर्मे चैश्वर्यमाधाय कथं त्वं भूतिमिच्छसि}


\twolineshloka
{सर्वैर्गुणैरुपेतास्तु पाण्डवा भरतर्षभ}
{पितृवत्त्वयि वर्तन्ते तेषु वर्तस्व पुत्रवत्}


\chapter{अध्यायः ३६}
\twolineshloka
{अत्रैवोदाहरन्तीममितिहासं पुरातनम्}
{आत्रेयस्य च संवादं साध्यानां चेति नः श्रुतम्}


\threelineshloka
{चरन्तं हंसरूपेण महर्षिं संशितव्रतम्}
{साध्या देवा महाप्राज्ञं पर्यपृच्छन्त वै पुरा ॥साध्या ऊचुः}
{}


\threelineshloka
{साध्या देवा वयमेते महर्षेदृष्ट्वा भवन्तं न शक्नुमोऽनुमातुम्}
{श्रुतेन धीरो बुद्धिमांस्त्वं मतो नःकाव्यां वाचं वक्तुमर्हस्युदाराम् ॥हंस उवाच}
{}


\twolineshloka
{एतत्कार्यममराः संश्रुतं मेधृतिः शमः सत्यधर्मानुवृत्तिः}
{ग्रन्थिं विनीय हृदयस्य सर्वंप्रियाप्रिये चात्मसमं नयीत}


\twolineshloka
{आक्रुश्यमानो नाक्रोशेन्मन्युरेव तितिक्षतः}
{आक्रोष्टारं निर्दहति सुकृतं चास्य विन्दति}


\twolineshloka
{नाक्रोशी स्यान्नावमानी परस्यमित्रद्रोही नोत नीचोपसेवी}
{न चाभिमानी न च हीनवृत्तोरूक्षां वाचं रुशतीं वर्जयीत}


\twolineshloka
{मर्माण्यस्थीनि हृदयं तथासू-न्रूक्षा वाचो निर्दहन्तीह पुंसाम्}
{तस्माद्वाचमुशतीं रूक्षरूपांधर्मारामो नित्यशो वर्जयीत}


\twolineshloka
{अरुन्तुदं परुषं रूक्षवाचंवाक्कण्टकैर्वितुदन्तं मनुष्यान्}
{विद्यादलक्ष्मीकतमं जननांमुखे निबद्धां निर्ऋतिं वै वहन्तम्}


\twolineshloka
{परश्चेदेनमभिविद्ध्येत बाणै-र्भृशं सुतीक्ष्णैरनलार्कदीप्तैः}
{विरिच्यमानोऽप्यतिरिच्यमानोविद्यात्कविः सुकृतं मे दधाति}


\twolineshloka
{यदि सन्तं सेवति यद्यसन्तंतपस्विनं यदि वा स्तेनमेव}
{वासो यथा रङ्गवशं प्रयातितथा स तेषां वशमभ्युपैति}


\twolineshloka
{अतिवादं न प्रवदेन्न वादये-द्यो नाहतः प्रतिहन्यान्न घातयेत्}
{हन्तुं च यो नेच्छति पापकं वैतस्मै देवाः स्पृहयन्त्यागताय}


\twolineshloka
{अव्याहृतं व्याहृताच्छ्रेय आहुःसत्यं वदेद्व्याहृतं तद्द्वितीयम्}
{प्रियं वदेद्व्याहृतं तत्तृतीयंधर्म्यं वदेद्व्याहृतं तच्चतुर्थम्}


\twolineshloka
{यादृशैः सन्निविशते यादृशांश्चोपसेवते}
{यादृगिच्छेच्च भवितुं तादृग्भवति पूरुषः}


\threelineshloka
{यतो यतो निवर्तते ततस्ततो विमुच्यते}
{निवर्तनाद्धि सर्वतो न वेत्ति दुःखमण्वपि ॥न जीयते चानुजिगीषतेऽन्या-न्न वैरकृच्चाप्रतिघातकश्च}
{}


% Check verse!
निन्दाप्रशंसासु समस्वभावोन शोचते हृष्यति नैव चायम्
\twolineshloka
{भावमिच्छति सर्वस्य नाभावे कुरुते मनः}
{सत्यवादी मृदुर्दान्तो यः स उत्तमपूरुषः}


\twolineshloka
{नानर्थकं सान्त्वयति प्रतिज्ञाय ददाति च}
{रन्ध्रं परस्य जानाति यः स मध्यमपूरुषः}


\threelineshloka
{दुःशासनस्तूपहतोऽभिशस्तोनावर्तते मन्युवशात्कृतघ्नः}
{न स्यचिन्मित्रमथो दुरात्माकलाश्चैता अधमस्येह पुंसः ॥न श्रद्दधाति कल्याणं परेभ्योऽप्यात्मशङ्कितः}
{निराकरोति मित्राणि यो वै सोऽधमपूरुषः}


\twolineshloka
{उत्तमानेव सेवेत प्राप्तकाले तु मध्यमान्}
{अधमांस्तु न सेवेत य इच्छेद्भूतिमात्मनः}


\threelineshloka
{प्राग्नोति वै वित्तमसद्बलेननित्योत्थानात्प्रज्ञया पौरुषेण}
{न त्वेव सम्यग्लभते प्रशंसांन वृत्तमाप्नोति महाकुलानाम् ॥धृतराष्ट्र उवाच}
{}


\threelineshloka
{महाकुलेभ्यः स्पृहयन्ति देवाधर्मार्थनित्याश्च बहुश्रुताश्च}
{पृच्छामि त्वां विदुरं प्रश्नमेतंभवन्ति वै कानि महाकुलानि ॥विदुर उवाच}
{}


\threelineshloka
{तपो दमो ब्रह्मवित्त्वं तितिक्षा}
{इज्या विवाहाः सान्त्वनं चान्नदानम्}
{अष्टावेते नित्यमेवं भवन्तिसतां गुणास्तानि महाकुलानि}


\twolineshloka
{येषां न वृत्तं व्यथते न योनि-श्चित्तप्रसादेन चरन्ति धर्मम्}
{ये कीर्तिमिच्छन्ति कुले विशिष्टांत्यक्तानृतास्तानि महाकुलानि}


\twolineshloka
{अनिज्यया कुविवाहैर्वेदस्योत्सादनेन च}
{कुलान्यकुलतां यान्ति ब्राह्मणातिक्रमेण च}


\twolineshloka
{देवद्रव्यनिनाशेन ब्रह्मस्वहरणेन च}
{कुलान्यकुलतां यान्ति ब्राह्मणातिक्रमेण च}


\twolineshloka
{ब्राह्मणानां परिभवात्परिवादाच्च भारत}
{कुलान्यकुलतां यान्ति न्यासापहरणेन च}


\twolineshloka
{कुलानि समुपेताननि गोभिः पुरुषतोऽर्थतः}
{कुलसंख्यां न गच्छन्ति यानि हीनानि वृत्ततः}


\twolineshloka
{वृत्ततस्त्वविहीनानि कुलान्यल्पधनान्यपि}
{कुलसङ्ख्यां च गच्छन्ति कर्षन्ति च महद्यशः}


\twolineshloka
{वृत्तं यत्नेन संरक्षेद्वित्तमेति च याति च}
{अक्षीणो वित्ततः क्षीणो वृत्ततस्तु हतो हतः}


\twolineshloka
{गोभिः पशुभिरश्वैश्च कृष्या च सुसमृद्धया}
{कुलानि न प्ररोहन्ति यानि हीनानि वृत्ततः}


\twolineshloka
{मा नः कुले वैरकृत्कश्चिदस्तुराजा बद्धो मा परस्वापहारी}
{मित्रद्रोही नैकृतिकोऽनृती वापूर्वाशी वा पितृदेवातिथिभ्यः}


\twolineshloka
{यश्च नो ब्राह्मणान्हन्याद्यश्च नो ब्राह्णणान् द्विषेत्}
{न नः स समितिं गच्छेद्यश्च नो निर्वपेत्कृषिम्}


\twolineshloka
{तृणानि भूमिरुदकं वाक्व्रतुर्थी च सूनृता}
{सतामेतानि गेहेषु नोच्छिद्यन्ते कदाचन}


\twolineshloka
{श्रद्धया परया राजन्नुपनीतानि सत्कृतिम्}
{प्रवृत्तानि महाप्राज्ञ धर्मिणां पुण्यकर्मिणाम्}


\twolineshloka
{सूक्ष्मोऽपि भारं नृपते स्यन्दनो वैशक्तो वोढुं न तथाऽन्ये महीजाः}
{एवं युक्ता भारसहा भवन्तिमहाकुलीना न तथान्ये मनुष्याः}


\twolineshloka
{न तन्मित्रं यस्य कोपाद्बिभेतियद्वा मित्रं शङ्कितेनोपचर्यम्}
{यस्मिन्मित्रे पितरीवाश्वसीततद्वै मित्रं सङ्गतानीतराणि}


\twolineshloka
{यः कश्चिदप्यसंबद्धो मित्रसावेन वर्तते}
{स एव बन्धुस्तन्मित्रं सा गतिस्तत्परायणम्}


\twolineshloka
{चलचित्तस्य वै पुंसो वृद्धाननुपसेवतः}
{पारिप्लवमतेर्नित्यमध्रुवो मित्रसङ्ग्रहः}


\twolineshloka
{चलचित्तमनात्मानमिन्द्रियाणां वशानुगम्}
{अर्थाः समभिवर्तते हंसाः शुष्कं सरो यथा}


\twolineshloka
{अकस्मादेव कुप्यन्ति प्रसीदन्त्यनिमित्ततः}
{शीलमेतदसाधूनामभ्रं पारिप्लवं यथा}


\twolineshloka
{सत्कृताश्च कृतार्थाश्च मित्राणां न भवन्ति ये}
{तान्मृतानपि क्रव्यादाः कृतघ्नान्नोपभुञ्जते}


\twolineshloka
{अर्थयेदेव मित्राणि सति वाऽसति वा धने}
{नानर्थयन्प्रजानाति मित्राणां सारफल्गुताम्}


\twolineshloka
{सन्तापाद्भ्रश्यते रूपं सन्तापाद्भ्रश्यते बलम्}
{सन्तापाद्भ्रश्यते ज्ञानं सन्तापाद्व्याधिमृच्छति}


\twolineshloka
{अनवाप्यं च शोकेन शरीरं चोपतप्यते}
{अमित्राश्च प्रहृष्यन्ति मा स्म शोके मनः कृथाः}


\twolineshloka
{पुनर्नरो म्रियते जायते चपुनर्नरो हीयते वर्धते च}
{पुनर्नरो याचति याच्यते चपुनर्नरः शोचति शोच्यते च}


\twolineshloka
{सुखं च दुःखं च भवाभवौ चलाभालाभौ मरणं जीवितं च}
{पर्यायशः सर्वमेते स्पृशन्तितस्माद्धीरो न च हृष्येन्न सोचेत्}


\threelineshloka
{चलानि हीमानि षडिन्द्रियाणितेषां यद्यद्वर्धते यत्रयत्र}
{ततस्ततः स्रवते बुद्धिरस्यच्छिद्रोदकुम्भादिव नित्यमम्भः ॥धृतराष्ट्र उवाच}
{}


\twolineshloka
{तनुरुद्धः शिखी राजा मिथ्योपचरितो मया}
{मन्दानां मम पुत्राणां युद्धेनान्तं करिष्यति}


\threelineshloka
{नित्योद्विग्नमिदं सर्वं नित्योद्विग्रमिदं मनः}
{यत्तत्पदमनुद्विग्नं तन्मे वद महामते ॥विदुर उवाच}
{}


\twolineshloka
{नान्यत्र विद्यातपसो नान्यत्रेन्द्रियनिग्रहात्}
{नान्यत्र लोभसन्त्यागाच्छान्तिं पश्यामि तेऽनघ}


\twolineshloka
{बुद्ध्या भयं प्रणुदति तपसा विन्दते महत्}
{गुरुशुश्रूषया ज्ञानं शान्तिं भोगेन विन्दति}


\twolineshloka
{अनाश्रिता दानपुण्यं वेदपुण्यमनाश्रिताः}
{रागद्वेषविनिर्मुक्ता विचरन्तीह मोक्षिणः}


\twolineshloka
{स्वधीतस्य सुयुद्धस्य सुकृतस्य च कर्मणः}
{तपसश्च सुतप्तस्य तस्यान्ते सुखमेधते}


\twolineshloka
{स्वास्तीर्णानि शयनानि प्रपन्नान वै भिन्ना जातु निद्रां लभन्ते}
{न स्त्रीषु राजन्रतिमाप्नुवन्तिन मागधैः स्तूयमाना न सूतैः}


\twolineshloka
{न वै भिन्ना जातु चरन्ति धर्मंन वै सुखं प्राप्नुवन्तीह भिन्नाः}
{न वै भिन्ना गौरवं प्राप्नुवन्तिन वै भिन्नाः प्रशमं रोचयन्ति}


\twolineshloka
{न वै तेषां स्वदते पथ्यमुक्तंयोगक्षेमं कल्पते नैव तेषाम्}
{भिन्नानां वै मनुजेन्द्र परायणंन विद्यते किञ्चिदन्यद्विनाशात्}


\twolineshloka
{संपन्नं गोषु संभाव्यं संभाव्यं ब्राह्मणे तपः}
{संभाव्यं चापलं स्त्रीषु संभाव्यं ज्ञातितो भयं}


\twolineshloka
{तन्तवोऽप्यायता नित्यं तनवो बहुलाः समाः}
{बहून्बहुत्वादायासान्सहन्तीत्सुपमा सताम्}


\twolineshloka
{धूमायन्ते व्यपेतानि ज्वलन्ति सहितानि च}
{धृतराष्ट्रोल्मुकानीव ज्ञातयो भरतर्षभ}


\twolineshloka
{ब्राह्मणेषु च ये शूराः स्त्रीषु ज्ञातिषु गोषु च}
{वृन्तादिव फलं पक्वं धृतराष्ट्र पतन्ति ते}


\twolineshloka
{महानप्येकजो वृक्षो बलवान्सुप्रतिष्ठितः}
{प्रसह्य एव वातेन सस्कन्धो मर्दितुं क्षणात्}


\twolineshloka
{अथ ये सहिता वृक्षाः सङ्घशः सुप्रतिष्ठिताः}
{ते हि शीघ्रतमान्वातान्सहन्तेन्योन्यसंश्रयात्}


\twolineshloka
{एवं मनुष्यमप्येकं गुणैरपि समन्वितम्}
{शक्यं द्विषन्तो मन्यन्ते वायुर्द्रुममिवैकजम्}


\twolineshloka
{अन्योन्यसमुपष्टम्भादन्योन्यापाश्रयेण च}
{ज्ञातयः संप्रवर्धन्ते सरसीवोत्पलान्युत}


\threelineshloka
{अवध्या ब्राह्मणा गावो ज्ञातयः शिशिवः स्त्रियः}
{येषां चान्नानि भुञ्जीत ये च स्युःक शरणागताः}
{महत्यप्यपराधेऽपि तेषां दण्डो विसर्जनम्}


\twolineshloka
{न मनुष्ये गुणः कश्चिद्राजन्सधनतामृते}
{अनातुरत्वाद्भद्रं ते मृतकल्पा हि रोगिणः}


\twolineshloka
{अव्याधिजं कटुकं शीर्षरोगिपापानुबन्धं परुषं तीक्ष्णमुष्णम्}
{सतां पेयं यन्न पिबन्त्यसन्तोमन्युं महाराज पिब प्रशाम्य}


\twolineshloka
{रोगार्दिता न फलान्याद्रियन्तेन वै लभन्ते विषयेषु तत्त्वम्}
{दुःखोपेता रोगिणो नित्यमेवन बुध्यन्ते धनभोगान्नसौख्यम्}


\twolineshloka
{पुरा ह्युक्तं नाकरोस्त्वं वचो मेद्यूते जितां द्रौपदीं प्रेक्ष्य राजन्}
{दुर्योधनं वारयेत्यक्षवत्यांकितवत्वं पण्डिता वर्जयन्ति}


\twolineshloka
{न तद्बलं यन्मृदुना विरुध्यतेसूक्ष्मो धर्मस्तरसा सेवितव्यः}
{प्रध्वंसिनी क्रूरसमाहिता श्री-र्मृदुप्रौढा गच्छति पुत्रपौत्रान्}


\twolineshloka
{धार्तराष्ट्राः पाण्डवान्पालयन्तुपाण्डोः सुतास्तव पुत्रांश्च पान्तु}
{एकारिमित्राः कुरवो ह्येककार्याजीवन्तु राजन्सुखिनः समृद्धाः}


\twolineshloka
{मेढीभूतः कौरवाणां त्वमद्यत्वय्याधीनं कुरुकुलमाजमीढ}
{पार्थान्बालान्वनवासप्रतप्ता-न्गोपायस्व स्वं यशस्तात रक्षन्}


\twolineshloka
{सन्धत्स्व त्वं कौरव पाण्डुपुत्रै-र्मा नेऽन्तरं रिपवः प्रार्थयन्तु}
{सत्ये स्थितास्ते नरदेव सर्वेदुर्योधनं स्थापय त्वं नरेन्द्र}


\chapter{अध्यायः ३७}
\twolineshloka
{विदुर उवाच}
{}


\twolineshloka
{सप्तदशेमान्राजेन्द्र मनुः स्वायंभुवोऽब्रवीत्}
{वैचित्रवीर्य पुरुषानाकाशं मुष्टिभिर्घ्नतः}


\twolineshloka
{तानेवेन्द्रस्य च धनुरनाम्यं नमतो ब्रवीत्}
{अथो मरीचिनः पादानग्राह्यान्गृह्णतस्तथा}


\twolineshloka
{यश्चाशिष्यं शास्ति वै यश्च तुष्ये-द्यश्चातिवेलं भजते द्विषन्तम्}
{स्त्रियश्च यो रक्षति भद्रमश्रुतेयश्चायाच्यं याचते कत्थते वा}


\threelineshloka
{यश्चाभिजातः प्रकरोत्यकार्यंयश्चाबलो बलिना नित्यवैरी}
{अश्रद्दधानाय च यो ब्रवीतियश्चाकाम्यं कामयते नरेन्द्र}
{}


\twolineshloka
{वध्वाऽवहासं श्वशुरो मन्यते योवध्वाऽवसन्नभयो मानकामः}
{परक्षेत्रे निर्वपति यश्च बीजंस्त्रियं च यः परिवदतेऽतिवेलम्}


\twolineshloka
{यश्चापि लब्ध्वा न स्मरामीति वादीदत्त्वा च यः कत्थति याच्यमानः}
{यश्चासतः सान्त्वमुपानयीतएतान्नयन्ति निरयं पाशहस्ताः}


\twolineshloka
{यस्मिन्यथा वर्तते यो मनुष्य-स्तस्मिंस्तथा वर्तितव्यं स धर्मः}
{मायाचारो मायया वर्तितव्यःसाध्वाचारः साधुना प्रत्युपेयः}


\twolineshloka
{जरा रूपं हरति हि धैर्यमाशामृत्युः प्राणान्धर्मचर्यामसूयाकामो ह्रियं वृत्तमनार्यसेवाक्रोधः श्रियं सर्वमेवाभिमानः ॥धृतराष्ट्र उवाच}
{}


\threelineshloka
{शतायुरुक्तः पुरुषः सर्ववेदेषु वै यदा}
{नाप्नोत्यथ च तत्सर्वमायुः केनेह हेतुना ॥विदुर उवाच}
{}


\twolineshloka
{अतिमानोऽतिवादश्च तथाऽत्यागो नराधिप}
{क्रोधश्चात्मविधित्सा च मित्रद्रोहश्च तानि षट्}


\twolineshloka
{एत एवासयस्तीक्ष्णाः कृन्तन्त्यायूंषि देहिनाम्}
{एतानि मानवान्ध्नन्ति न मृत्युर्भद्रमस्तु ते}


\twolineshloka
{विश्वस्तस्यैति यो दारान्यश्चापि गुरुतल्पगः}
{वृषलीपतिर्द्विजो यश्च पानपश्चैव भारत}


\threelineshloka
{आदेशकृद्वृत्तिहन्ता द्विजानां प्रेषकश्च यः}
{शरणागतहा चैव सर्वे ब्रह्महणः समाः}
{एतैः समेत्य कर्तव्यं प्रायश्चित्तमिति श्रुतिः}


\twolineshloka
{गृहीतवाक्यो नयविद्वदान्यःशेषान्नभोक्ता ह्यविहिंसकश्च}
{नानर्थकृत्याकुलितः कृतज्ञःसत्यो मृदुः स्वर्गमुपैति विद्वान्}


\twolineshloka
{सुलभाः पुरुषा राजन्सततं प्रियवादिनः}
{अप्रियस्य तु पथ्यस्य वक्ता श्रोता च दुर्लभः}


\twolineshloka
{यो हि धर्मं समाश्रित्य हित्वा भर्तुः प्रियाप्रिये}
{अप्रियाण्याह पथ्यानि तेन राजा सहायवान्}


\twolineshloka
{त्यजेत्कुलार्थे पुरुषं ग्रामस्यार्थे कुलं त्यजेत्}
{ग्रामं जनपदस्यार्थे आत्मार्थे पृथिवीं त्यजेत्}


\twolineshloka
{आपदर्थे धनं रक्षेद्दारान्रक्षेद्धनैरपि}
{आत्मानं सततं रक्षेद्दारैरपि धनैरपि}


\twolineshloka
{द्यूतमेतत्पुरा कल्पे दृष्टं वैरकरं नृणाम्}
{तस्माद्द्यूतं न सेवेत हास्यार्थमपि बुद्धिमान्}


\twolineshloka
{उक्तं मया द्यूतकालेऽपि राज-न्नेदं युक्तं वचनं प्रातिपेय}
{तदौषधं पथ्यमिवातुरस्यन रोचते तव वैचित्रवीर्य}


\twolineshloka
{काकैरिमांश्चित्रबर्हान्मयूरान्पराजयेथाः पाण्डवान्धार्तराष्ट्रैः}
{हित्वा सिंहान्क्रोष्टुकान्गूहमानःप्राप्ते काले शोचिता त्वं नरेन्द्रः}


\twolineshloka
{यस्तात न क्रुध्यति सर्वकालंभृत्यस्य भक्तस्य हिते रतस्य}
{तस्मिन्भृत्या भर्तरि विश्वसन्तिन चैनमापत्सु परित्यजन्ति}


\twolineshloka
{न भृत्यानां वृत्तिसंरोधनेनराज्यं धनं सञ्जिघृक्षेदपूर्वम्}
{त्यजन्ति ह्येनं वञ्चिता वै विरुद्धाःस्निग्धा ह्यमात्या परिहीनभोगाः}


\twolineshloka
{कृत्यानि पूर्वं परिसङ्ख्याय सर्वा-ण्यायव्यये चानुरूपां च वृत्तिम्}
{सङ्गृह्णीयादनुरूपान्सहायान्सहायसाध्यानि हि दुष्कराणि}


\twolineshloka
{अभिप्रायं यो विदित्वा तु भर्तुःसर्वाणि कार्याणि करोत्यतन्द्री}
{वक्ता हितानामनुरक्त आर्यःशक्तिज्ञ आत्मेव हि सोऽनुकम्प्यः}


\twolineshloka
{वाक्यं तु यो नाद्रियतेऽनुशिष्टःप्रत्याह यश्चापि नियुज्यमानः}
{प्रज्ञाभिमानी प्रतिकूलवादीत्याज्यः स तादृक् त्वरयैव भृत्यः}


\twolineshloka
{अस्तब्धमक्लीबमदीर्घसूत्रंसानुक्रोशं श्लक्ष्णमहार्यमन्यैः}
{अरोगजातीयमुदारवाक्यंदूतं वदन्त्यष्टगुणोपपन्नम्}


\twolineshloka
{न विश्वासाञ्जातु परस्य गेहेगच्छेन्नरश्चेतयानो विकाले}
{न चत्वरे निशि तिष्ठेन्निगूढोन राजकाम्यां योषितं प्रार्थयीत}


\twolineshloka
{न निह्नवं मन्त्रगतस्य गच्छे-त्संसृष्टमन्त्रस्य कुसङ्गतस्य}
{न च ब्रूयान्नाश्वसिमि त्वयीतिसकारणं व्यपदेशं तु कुर्यात्}


\twolineshloka
{घृणी राजा पुंश्चली राजभृत्यःपुत्रो भ्राता विधवा बालपुत्रा}
{सेनाजीवी चोद्धृतभूरिरेवव्यवहारेषु वर्जनीयाः स्युरेते}


\twolineshloka
{अष्टौ गुणाः पुरुषं दीपयन्तिप्रज्ञा च कौल्यं च श्रुतं दमश्च}
{पराक्रमश्चाबहुभाषिता चदानं यथाशक्ति कृतज्ञता च}


\twolineshloka
{एतान्गुणांस्तात महानुभावा-नेको गुणः संश्रयते प्रसह्य}
{राजा यदा सत्कुरुते मनुष्यंसर्वान्गुणनेष गुणो बिभर्ति}


\twolineshloka
{गुणा दश स्नानशीलं भजन्तेबलं रूपं स्वरवर्णप्रशुद्धिः}
{स्पर्शश्च गन्धश्च विशुद्धता चश्रीः सौकुमार्यं प्रवराश्च नार्यः}


\twolineshloka
{गुणाश्च षण्मितभुक्तं भजन्तेआरोग्यमायुश्च बलं सुखं च}
{अनाविलं चास्य भवत्यपत्यंन चैनमाद्यून इति क्षिपन्ति}


\twolineshloka
{अकर्मशीलं च महाशनं चलोकद्विष्टं बहुमायं नृशंसम्}
{अदेशकालज्ञममिष्टवेष-मेतान्गृहे न प्रतिवासयेत}


\twolineshloka
{कदर्यमाक्रोशकमश्रुतं चवनौकसं धूर्तममान्यमानिनम्}
{निष्ठूरिणं कृतवैरं कृतघ्न-मेतान्भृशार्तोपि न जातु याचेत्}


\twolineshloka
{संक्लिष्टकर्माणमतिप्रमादंनित्यानृतं चादृढभक्तिकं च}
{विसृष्टरागं पटुमानिनं चा-प्येतान्न सेवेत नराधमान्षट्}


\twolineshloka
{सहायबन्धना ह्यर्थाः सहायाश्चार्थबन्धनाः}
{5-37-38अन्योन्यबन्धनावेतौ विनान्योन्यं न सिद्ध्यतः}


\twolineshloka
{उत्पाद्य पुत्राननृणांश्च कृत्वावृत्तिं च तेभ्योऽनुविधाय कांचित्}
{स्थाने कुमारीः प्रतिपाद्य सर्वाअरण्यसंस्थोऽथ मुनिर्बुभूषेत्}


\twolineshloka
{हितं यत्सर्वभूतानामात्मनश्च सुखावहम्}
{तत्कुर्यादीश्वरो ह्येतन्मूलं सर्वार्थसिद्धये}


\twolineshloka
{वृद्धिः प्रभावस्तेजश्च सत्वमुत्थानमेव च}
{व्यवसायश्च यस्य स्यात्तस्यावृत्तिभयं कुतः}


\twolineshloka
{पश्य दोषान्पाण्डवैर्विग्रहे त्वंयत्र व्यथेयुरपि देवाः सशक्रा}
{पुत्रैर्वैरं नित्यमुद्विग्नवासोयशःप्रणाशो द्विषतश्च हर्षः}


\twolineshloka
{भीष्मस्य कोपस्तव चैवेन्द्रकल्पद्रोणस्य राज्ञश्च युधिष्ठिरस्य}
{उत्सादयेल्लोकमिमं प्रवृद्धःश्वेतो ग्रहस्तिर्यगिवापतन्खे}


\twolineshloka
{तव पुत्रशतं चैव कर्णः पञ्च च पाण्डवाः}
{पृथिवीमनुशासेयुरखिलां सागराम्बराम्}


\twolineshloka
{धार्तराष्ट्रा वनं राजन्व्याघ्राः पाण्डुसुता मताः}
{मा वनं छिन्धि सव्याघ्रं मा व्याघ्रा नीनशन्वनात्}


\twolineshloka
{न स्याद्वनमृते व्याघ्रान्व्याघ्रा न स्युर्ऋते वनम्}
{वनं हि रक्ष्यते व्याघ्रैर्व्याघ्रान्रक्षति काननम्}


\twolineshloka
{न तथेच्छन्ति कल्याणान्परेषां वेदितुं गुणान्}
{यथेषां ज्ञातुमिच्छन्ति नैर्गुण्यं पापचेतसः}


\twolineshloka
{अर्थसिद्धिं परामिच्छन्धर्ममेवादितश्चरेत्}
{नहि धर्मादपैत्यर्थः स्वर्गलोकादिवामृतम्}


\twolineshloka
{यस्यात्मा विरतः पापात्कल्याणे च निवेशितः}
{तेन सर्वमिदं बुद्धं प्रकृतिर्विकृतिश्च या}


\twolineshloka
{यो धर्ममर्थं कामं च यथाकालं निषेवते}
{धर्मार्थकामसंयोगं सोऽमुत्रेह च विन्दति}


\twolineshloka
{सन्नियच्छति यो वेगमुत्थितं क्रोधहर्षयोः}
{स श्रियो भाजनं राजन्यश्चापत्सु न मुह्यति}


\twolineshloka
{बलं पञ्चविधं नित्यं पुरुषाणां निबोध मे}
{यत्तु बाहुबलं नाम प्रथमं वलमुच्यते}


\twolineshloka
{अमात्यलाभो भद्रं ते द्वितीयं बलमुच्यते}
{तृतीयं धनलाभं तु बलमाहुर्मनीषिणः}


\twolineshloka
{यत्त्वस्य सहजं राजन्पितृपैतामहं बलम्}
{अभिजातबलं नाम तच्चतुर्थं बलं स्मृतम्}


\twolineshloka
{येन त्वेतानि सर्वाणि सङ्गृहीतानि भारत}
{यद्बलानां बलं श्रेष्ठं तत्प्रज्ञाबलमुच्यते}


\twolineshloka
{महते योऽपकाराय नरस्य प्रभवेन्नरः}
{तेन वैरं समासज्य दूरस्थोऽस्मीति नाश्वसेत्}


\twolineshloka
{स्त्रीषु राजसु सर्पेषु स्वाध्यायप्रभुशत्रुषु}
{भोगेष्वायुषि विश्वासं कः प्राज्ञः कर्तुमर्हति}


\twolineshloka
{प्रज्ञाशरेणाभिहतस्य जन्तो-श्चिकित्सकाः सन्ति न चौषधानि}
{न होममन्त्रा न च मङ्गलानिनाथर्वणा नाप्यगदाः सुसिद्धाः}


\twolineshloka
{सर्पश्चाग्निश्च सिंहश्च कुलपुत्रश्च भारत}
{नावज्ञेया मनुष्येण सर्वे ह्येतेऽतितेजसः}


\twolineshloka
{अग्निस्तेजो महल्लोके गूढस्तिष्ठति दारुषु}
{न चोपयुङ्क्ते तद्दारु यावन्नोद्दीप्यते परैः}


\twolineshloka
{स एव खलु दारुभ्यो यदा निर्मथ्य दीप्यते}
{तद्दारु च वनं चान्यन्निर्दहत्याशु तेजसा}


\twolineshloka
{एवमेव कुले जाताः पावकोपमतेजसः}
{क्षमावन्तो निराकाराः काष्ठेऽग्निरिव शेरते}


\twolineshloka
{लताधर्मा त्वं सपुत्रः सालः पाण्डुसुता मताः}
{न लता वर्धते जातु महाद्रुममनाश्रिता}


\twolineshloka
{वनं राजंस्तव पुत्रोऽम्बिकेयसिंहान्वने पाण्डवांस्तात विद्धि}
{सिंहैर्विहीनं हि वनं विनश्येत्सिंहा विनश्येयुर्ऋते वनेन}


\chapter{अध्यायः ३८}
\twolineshloka
{विदुर उवाच}
{}


\twolineshloka
{ऊर्ध्वं प्राणा ह्युत्क्रामन्ति यूनः स्थविर आयति}
{प्रत्युत्थानाभिवादाभ्यां पुनस्तान्प्रतिपद्यते}


\twolineshloka
{पीठं दत्त्वा साधवेऽभ्यागतायआनीयापः परिनिर्णिज्य पादौ}
{सुखं पृष्ट्वा प्रतिवेद्यात्मसंस्थांततो दद्यादन्नमवेक्ष्य धीरः}


\twolineshloka
{यस्योदकं मधुपर्कं च गां चन मन्त्रवित्प्रतिगृह्णाति गेहे}
{लोभाद्भयादथ कार्पण्यतो वातस्यानर्थं जीवितमाहुरर्याः}


\twolineshloka
{चिकित्सकः शल्यकर्तावकीर्णीस्तेनः क्रूरो मद्यपो भ्रूणहा च}
{सेनाजीवी श्रुतिविक्रायकश्चभृशं प्रियोऽप्यतिथिर्नोदकार्हः}


\twolineshloka
{अविक्रेयं लवणं पक्वमन्नंदधि क्षीरं मधु तैलं घृतं च}
{तिला मांसं फलमूलानि शाकंरक्तं वासः सर्वगन्धा गुडाश्च}


\twolineshloka
{अरोषणो यः समलोष्ठाश्मकाञ्चनःप्रहीणशोको गतसन्धिविग्रहः}
{निन्दाप्रशंसोपरतः प्रियाप्रियेत्यजन्नुदासीनवदेष भिक्षुकः}


\twolineshloka
{नीवारमूलेङ्गुदशाकवृत्तिःसुसंयतात्माग्निकार्येषु चोद्यः}
{वने वसन्नतिथिष्वप्रमत्तोधुरन्धरः पुण्यकृदेष तापसः}


\twolineshloka
{अपकृत्य बुद्धिमतो दूरस्थोऽस्मीति नाश्वसेत्}
{दीर्घौ बुद्धिमतो बाहू याभ्यां हिंसति हिंसितः}


\twolineshloka
{न विश्वसेदविश्वस्ते विश्वस्ते नातिविश्वसेत्}
{विश्वासाद्भयमुत्पन्नं मूलान्यपि निकृन्तति}


\twolineshloka
{अनीर्षुर्गुप्तदारश्च संविभागी प्रियंवदः}
{श्लक्ष्णो मधुरवाक्स्त्रीणां न चासां वशगो भवेत्}


\twolineshloka
{पूजनीया महाभागाः पुण्याश्च गृहदीप्तयः}
{स्त्रियः श्रियो गृहस्योक्तास्तस्माद्रक्ष्या विशेषतः}


\threelineshloka
{पितुरन्तःपुरं दद्यान्मातुर्दद्यान्महानसम्}
{गोषु चात्मसमं दद्यात्स्वयमेव कृषिं व्रजेत्}
{भृत्यैर्वाणिज्यचारं च पुत्रैः सेवेत च द्विजान्}


\twolineshloka
{अद्भ्योऽग्निर्ब्रह्मतः क्षत्रमश्मनो लोहमुत्थितम्}
{तेषां सर्वत्रगं तेजः स्वासु योनिषु शाम्यति}


\twolineshloka
{नित्यं सन्तं कुले जाताः पावकोपमतेजसः}
{क्षमावन्तो निराकाराः काष्ठेऽग्निरिव शेरते}


\twolineshloka
{यस्य मन्त्रं न जानन्ति बाह्याश्चाभ्यन्तराश्च ये}
{स राजा सर्वतश्चक्षुश्चिरमैश्वर्यमश्नुते}


\twolineshloka
{करिष्यन्न प्रभाषेत कृतान्येव तु दर्शयेत्}
{धर्मकामार्थकार्याणि तथा मन्त्रो न भिद्यते}


\twolineshloka
{गिरिपृष्ठमुपारुह्य प्रासादं वा रहोगतः}
{अरण्ये निःशलाके वा तत्र मन्त्रोऽभिधीयते}


\twolineshloka
{नासुहृत्परमं मन्त्रं भारतार्हति वेदितुम्}
{अपण्डितो वापि सुहृत्पण्डितो वाप्यनात्मवान्}


\twolineshloka
{नापरीक्ष्य महीपालः कुर्यात्सचिवमात्मनः}
{अमात्ये ह्यर्थलिप्सा च मन्त्ररक्षणमेव च}


\threelineshloka
{कृतानि सर्वकार्याणि यस्य पारिषदा विदुः}
{धर्मे चार्थे च कामे च स राजा राजसत्तमः}
{}


\twolineshloka
{गूढमन्त्रस्य नृपतेस्तस्य सिद्धिरसंशयम् ॥अप्रशस्तानि कार्याणि यो मोहादनुतिष्ठति}
{}


\twolineshloka
{स तेषां विपरिभ्रंशाद्भ्रश्यते जीवितादपि ॥कर्मणां तु प्रशस्तानामनुष्ठानं सुखावहम्}
{}


\twolineshloka
{तेषामेवाननुष्ठानं पश्चात्तापकरं मतम् ॥अनधीत्य यथा वेदान्न विप्रः श्राद्धमर्हति}
{}


\twolineshloka
{एवमश्रुतषाङ्गुण्यो न मन्त्रं श्रोतुमर्हति ॥स्थानवृद्धिक्षयज्ञस्य षाङ्गुण्यविदितात्मनः}
{}


\twolineshloka
{अनवज्ञातशीलस्य स्वाधीना पृथिवी नृप ॥अमोघक्रोधहर्षस्य स्वयं कृत्वान्ववेक्षिणः}
{}


\twolineshloka
{आत्मप्रत्ययकोशस्य वसुदैव वसुन्धरा ॥नाममात्रेणा तुष्येत छत्रेण च महीपतिः}
{}


\twolineshloka
{भृत्येभ्यो विसृजेदर्थान्नैकः सर्वहरो भवेत् ॥ब्राह्मणं ब्राह्मणो वेद भर्ता वेद स्त्रियं तथा}
{}


\twolineshloka
{अमात्यं नृपतिर्वेद राजा राजानमेव च ॥न शत्रुर्वशमापन्नो मोक्तव्यो वध्यतां गतः}
{}


% Check verse!
न्यग्भूत्वा पर्युपासीत वध्वं हन्याद्बले सति ॥अहताद्धि भयं तस्माज्जायते नचिरादिव
\twolineshloka
{दैवतेषु प्रयत्नेन राजसु ब्राह्मणेषु च}
{नियन्तव्यः सदा क्रोधो वृद्धबालातुरेषु च}


\twolineshloka
{निरर्थं कलहं प्राज्ञो वर्जयेन्मूढसेवितम्}
{कीर्ति च लभते लोके न चानर्थेन युज्यते}


\twolineshloka
{प्रसादो निष्फलो यस्य क्रोधश्चापि निरर्थकः}
{न तं भर्तारमिच्छन्ति षण्ढं पतिमिव स्त्रियः}


\twolineshloka
{न बुद्धिर्धनलाभाय न जाड्यमसमृद्धये}
{लोकपर्यायवृत्तान्तं प्राज्ञो जानाति नेतरः}


\twolineshloka
{विद्याशीलवयोवृद्धान्बुद्धिवृद्धांश्च भारत}
{धनाभिजातवृद्धांश्च नित्यं मूढोऽवमन्यते}


\twolineshloka
{अनार्यवृत्तमप्राज्ञमसूयकमधार्मिकम्}
{अनर्थाः क्षिप्रमायान्ति वाग्दुष्टं क्रोधनं तथा}


\twolineshloka
{अविसंवादनं दानं समयस्याव्यतिक्रमः}
{आवर्तयन्ति भूतानि सम्यक्प्रणिहिता च वाक्}


\twolineshloka
{अविसंवादको दक्षः कृतज्ञो मतिमानृजुः}
{अपि संक्षीणकोशोऽपि लभते परिवारणम्}


\twolineshloka
{धृतिः शमो दमः शौचं कारुण्यं वागनिष्ठुरा}
{मित्राणां चानभिद्रोहः सप्तैताः समिधः श्रियः}


\twolineshloka
{असंविभागी दुष्टात्मा कृतघ्नो निरपत्रपः}
{तादृङ्वराधिपो लोके वर्जनीयो नराधिप}


\twolineshloka
{न च रात्रौ सुखं शेते ससर्प इव वेश्मनि}
{यः कोपयति निर्दोषं सदोषोऽभ्यन्तरं जनम्}


\twolineshloka
{येषु दुष्टेषु दोषः स्याद्योगक्षेमस्य भारत}
{सदा प्रसादनं तेषां देवतानामिवाचरेत्}


\twolineshloka
{येऽर्थाः स्त्रीषु समायुक्ताः प्रमत्तपतितेषु च}
{ये चानार्ये समासक्ताः सर्वे ते संशयं गताः}


\twolineshloka
{यत्र स्त्री यत्र कितवो बालो यत्रानुशासिता}
{मञ्जन्ति तेऽवशा राजन्नद्यामश्मप्लवा इव}


\twolineshloka
{प्रयोजनेषु ये सक्ता न विशेषेषु भारत}
{तानहं पण्डितान्मन्ये विशेषा हि प्रसङ्गिनः}


\twolineshloka
{यं प्रशंसन्ति कितवा यं प्रशंसन्ति चारणाः}
{यं प्रशंसन्ति बन्धक्यो न स जीवति मानवः}


\twolineshloka
{हित्वा तान्परमेष्वासान्पाण्डवानमितौजसः}
{आहितं भारतैश्वर्यं त्वया दुर्योधने महत्}


\twolineshloka
{तं द्रक्ष्यसि परिभ्रष्टं तस्मात्त्वमचिरादिव}
{ऐश्वर्यमदसंमूढं बलिं लोकत्रयादिव}


\chapter{अध्यायः ३९}
\twolineshloka
{धृतराष्ट्र उवाच}
{}


\threelineshloka
{अनीश्वरोऽयं पुरुषो भवाभवेसूत्रप्रोता दारुमयीव योषा}
{धात्रा तु दिष्टस्य वशे कृतोऽयंतस्माद्वद त्वं श्रवणे धृतोऽहम् ॥विदुर उवाच}
{}


\twolineshloka
{अप्राप्तकालं वचनं बृहस्पतिरपि ब्रुवन्}
{लभते बुध्द्यवज्ञानमवमानं च भारत}


\twolineshloka
{प्रियो भवति दानेन प्रियवादेन चापरः}
{मन्त्रमूलबलेनान्यो यः प्रियः प्रिय एव सः}


\twolineshloka
{द्वेष्यो न साधुर्भवति न मेधावी न पण्डितः}
{प्रिये शुभानि कार्याणि द्वेष्ये पापानि चैव ह}


\twolineshloka
{उक्तं मया जातमात्रेऽपि राजन्दुर्योधनं त्यज पुत्रं त्वमेकम्}
{तस्य त्यागात्पुत्रशतस्य वृद्धि-रस्यात्यागात्पुत्रशतस्य नाशः}


\twolineshloka
{न वृद्धिर्बहुमन्तव्या या वृद्धिः क्षयमावहेत्}
{क्षयोऽपि बहुमन्तव्यो यः क्षयो वृद्धिमावहेत्}


\twolineshloka
{न स क्षयो महाराज यः क्षयो वृद्धिमावहेत्}
{क्षयः स त्विह मन्तव्यो यं लब्ध्वा बहु नाशयेत्}


\threelineshloka
{समृद्धा गुणतः केचिद्भवन्ति धनतोऽपरे}
{धनवृद्धान्गुणैर्हीनान्धृतराष्ट्र विवर्जय ॥धृतराष्ट्र उवाच}
{}


\threelineshloka
{सर्वं त्वमायतीयुक्तं भाषसे प्राज्ञसंमतम्}
{न चोत्सहे सुतं त्यक्तुं यतो धर्मस्ततो जयः ॥विदुर उवाच}
{}


\twolineshloka
{अतीव गुणसंपन्नो न जातुः विनयान्वितः}
{सुसूक्ष्ममपि भूतानामुपमर्दमुपेक्षते}


\twolineshloka
{परापवादनिरताः परदुःखोदयेषु च}
{परस्परविरोधे च यतन्ते सततोत्थिताः}


\twolineshloka
{सदोषं दर्शनं येषां संवासे सुमहद्भयम्}
{अर्थादाने महान्दोषः प्रदाने च महद्भयम्}


\twolineshloka
{ये वै भेदनशीलास्तु सकामा निस्त्रपाः शठाः}
{ये पापा इति विख्याताः संवासे पिरिगर्हिताः}


\twolineshloka
{युक्ताश्चान्यैर्महादोषैर्ये नरास्तान्विवर्जयेत्}
{निवर्तमाने सौहार्दे प्रीतिर्नीचे प्रणश्यति}


\twolineshloka
{या चैव फलनिर्वृतिः सौहृदे चैव यत्सुखम्}
{यतते चापवादाय यत्नमारभते क्षये}


\threelineshloka
{अल्पेऽप्यपकृते मोहान्न शान्तिमधिगच्छति}
{तादृशैः सङ्गतं नीचैर्नृशंसैरकृतात्मभिः}
{निशाम्य निपुणं बुद्ध्या विद्वान्दूराद्विवर्जयेत्}


\twolineshloka
{यो ज्ञातिमनुगृह्णाति दरिद्रं दीनमातुरम्}
{स पुत्रपशुभिर्वृद्धिं श्रेयश्चानन्त्यमश्रुते}


% Check verse!
ज्ञातयो वर्धनीयास्तैर्य इच्छन्त्यात्मनः शुभम्
\twolineshloka
{कुलवृद्धिं च राजेन्द्र तस्मात्साधु समाचरं}
{श्रेयसा योक्ष्यते राजन्कुर्वाणो ज्ञातिसत्क्रियां}


\twolineshloka
{विगुणा ह्यपि संरक्ष्या ज्ञातयो भरतर्षभ}
{किंपुनर्गुणवन्तस्ते त्वत्प्रसादाभिकाङ्क्षिणः}


\twolineshloka
{प्रसादं कुरु वीराणां पाण्डवानां विशांपते}
{दीयतां ग्रामकाः केचित्तेषां वृत्त्यर्थमीश्वर}


\twolineshloka
{एवं लोके यशः प्राप्तं भविष्यति नराधिप}
{वृद्धेन हि त्वया कार्यं पुत्राणां तात शासनम्}


\threelineshloka
{मया चापि हितं वाच्यं विद्धि मां त्वद्धितैषिणम्}
{ज्ञातिभिर्विग्रहस्तात न कर्तव्यः शुभार्थिना}
{सुखानि सह भोज्यानि ज्ञातिभिर्भरतर्षभ}


\twolineshloka
{संभोजनं संकथनं संप्रीतिश्च परस्परम्}
{ज्ञातिभिः सह कार्याणि न विरोधः कदाचन}


\twolineshloka
{ज्ञातयस्तारयन्तीह ज्ञातयो मञ्जयन्ति च}
{सुवृत्तास्तारयन्तीह दुर्वृत्ता मञ्जयन्ति च}


\twolineshloka
{सुवृत्तो भव राजेन्द्र पाण्जवान्प्रति मानद}
{अधर्षणीयः शत्रूणां तैर्वृतस्त्वं भविष्यति}


\twolineshloka
{श्रीमन्तं ज्ञातिमासाद्य यो ज्ञातिरवसीदति}
{दिग्धहस्तं मृग इव स एनस्तस्य विन्दति}


\twolineshloka
{पश्चादपि नरश्रेष्ठ तव तापो भविष्यति}
{तान्वा हतान्सुतान्वापि श्रुत्वा तदनुचिन्तय}


\twolineshloka
{येन खट्वां समारूढः परितप्येत कर्मणा}
{आदावेन न तत्कुर्यादध्रुवे जीविते सति}


\twolineshloka
{न कश्चिन्नापनयते पुमानन्यत्र भार्गवात्}
{शेषसंप्रतिपत्तिस्तु बुद्धिमत्स्वेव तिष्ठति}


\twolineshloka
{दुर्योधनेन यद्येतत्पापं तेषु पुरा कृतम्}
{त्वया तत्कुलवृद्धेन प्रत्यानेयं नरेश्वर}


\twolineshloka
{तांस्त्वं पदे प्रतिष्ठाप्य लोके विगतकल्मषः}
{भविष्यसि नरश्रेष्ठ पूजनीयो मनीषिणाम्}


\twolineshloka
{सुव्याहृतानि धीराणां फलतः परिचिन्त्य यः}
{अध्यवस्यति कार्येषु चिरं यशसि तिष्ठति}


\twolineshloka
{असम्यगुपयुक्तं हि ज्ञानं सुकुशलैरपि}
{उपलभ्यं चाविदितं विदितं चाननुष्ठितम्}


% Check verse!
पापोदयफलं विद्वान्यो नारभति वर्धते
\twolineshloka
{यस्तु पूर्वकृतं पापमविमृश्यानुवर्तते}
{अगाधपङ्के दुर्मेधा विषमे विनिपात्यते}


\twolineshloka
{मन्त्रभेदस्य षट् प्राज्ञो द्वाराणीमानि लक्षयेत्}
{अर्थसन्ततिकामश्च रक्षेदेतानि नित्यशः}


\twolineshloka
{मदं स्वप्नपविज्ञानमाकारं चात्मसंभवम्}
{दुष्टामात्येषु विश्रम्भं दूताच्चाकुशलादपि}


\twolineshloka
{द्वाराण्येतानि यो ज्ञात्वा संवृणोति सदा नृप}
{त्रिवर्गाचरणे युक्तः स शत्रूनधितिष्ठति}


\twolineshloka
{नवै श्रुतमविज्ञाय वृद्धाननुपसेव्य वा}
{धर्मार्थौ वेदुतुं शक्यौ बृहस्पतिसमैरपि}


\twolineshloka
{नष्टं समुद्रे पतितं नष्टं वाक्यमशृण्वति}
{अनात्मनि श्रुतं नष्टं नष्टं हुतमनग्निकम्}


\twolineshloka
{मत्या परीक्ष्य मेधावी बुद्ध्या संपाद्य चासकृत्}
{श्रुत्वा दृष्ट्वाथ विज्ञाय प्राज्ञैर्मैत्रीं समाचरेत्}


\twolineshloka
{अकीर्ति विनयो हन्ति हन्त्यनर्थं पराक्रमः}
{हन्ति नित्यं क्षमा क्रोधमाचारो हन्त्यलक्षणम्}


\twolineshloka
{परिच्छदेन क्षेत्रेण वेश्मना परिचर्यया}
{परीक्षेत कुलं राजन्भोजनाच्छादनेन च}


\twolineshloka
{उपस्थितसय कामस्य प्रतिवादो न विद्यते}
{अपि निर्मुक्तदेहस्य कामरक्तस्य किं पुनः}


\twolineshloka
{प्राज्ञोपसेविनं वैद्यं धार्मिकं प्रियदर्शनम्}
{मित्रवन्तं सुवाक्यं च सुहृदं परिपालयेत्}


\twolineshloka
{दुष्कुलीनः कुलीनो वा मर्यादां यो न लङ्घयेत्}
{धर्मापेक्षी मृदुर्ह्रीमान्स कुलीनशताद्वरः}


\twolineshloka
{ययोश्चित्तेन वा चित्तं निभृतं निभृतेन वा}
{समेति प्रज्ञया प्रज्ञा तयोर्मैत्री न जीर्यति}


\twolineshloka
{दुर्बुद्धिमकृतप्रज्ञं छन्नं कूपं तृणैरिव}
{विवर्जयीत मेधावी तस्मिन्मैत्री प्रणश्यति}


\twolineshloka
{अवलिप्तेषु मूर्खेषु रौद्रसाहसिकेषु च}
{तथैवापेतधर्मेषु न मैत्रीमाचरेद्बुधः}


\twolineshloka
{कृतज्ञं धार्मिकं सत्यमक्षुद्रं दृढभक्तिकम्}
{जितेन्द्रियं स्थितं स्थित्यां मित्रमित्यभिवाञ्छति}


\twolineshloka
{इन्द्रियाणामनुत्सर्गो मृत्युनापि विशिष्यते}
{अत्यर्थं पुनरुत्सर्गः सादयेद्दैवतानपि}


\twolineshloka
{मार्दवं सर्वभूतानामनसूया क्षमा धृतिः}
{आयुष्याणि बुधाः प्राहुर्मित्राणां चाविमानना}


\twolineshloka
{अपनीतं सुनीतेन योऽर्थं प्रत्यानिनीपते}
{मतिमास्थाय सुदृढां तदकापुरुषव्रतम्}


\twolineshloka
{आयत्यां प्रतिकारज्ञस्तदात्वे दृढनिश्चयः}
{अतीते कार्यशेषज्ञो नरोऽर्थैर्न प्रहीयते}


\twolineshloka
{कर्मणा मनसा वाचा यदभीक्ष्णं निषेवते}
{तदेवापहरत्येनं तस्मात्कल्याणमाचरेत्}


\twolineshloka
{मङ्गलालम्भनं योगः श्रुतमुत्थानमार्जवम्}
{भूतिमेतानि कुर्वन्ति सतां चाभीक्ष्णदर्शनम्}


\twolineshloka
{अनिर्वेदः श्रियो मूलं लाभस्य च शुभस्य च}
{महान्भवत्यनिर्विष्णः सुखं चानन्त्यमश्नुते}


\twolineshloka
{नातः श्रीमत्तरं किंचिदन्यत्पथ्यतमं मतम्}
{प्रभविष्णोर्यथा तात क्षमा सर्वत्र सर्वदा}


\twolineshloka
{क्षमेदशक्तः सर्वस्य शक्तिमान्धर्मकारणात्}
{अर्थानर्थौ समौ यस्य तस्य नित्यं क्षमा हिता}


\twolineshloka
{यत्सुखं सेवमानोऽपि धर्मार्थाभ्यां न हीयते}
{कामं तदुपसेवेत न मूढव्रतमाचरेत्}


\twolineshloka
{दुःखार्तेषु प्रमत्तेषु नास्तिकेष्वलसेषु च}
{न श्रीर्वसत्यदान्तेषु ये चोत्साहविवर्जिताः}


\twolineshloka
{आर्जवेन नरं युक्तमार्जवात्सव्यपत्रपम्}
{अशक्तं मन्यमानास्तु धर्षयन्ति कुबुद्धयः}


\twolineshloka
{अत्यार्यमतिदातारमतिशूरमतिव्रतम्}
{प्रज्ञाभिमानिनं चैव श्रीर्भयान्नोपसर्पति}


\threelineshloka
{न चातिगुणवत्स्वेषा नात्यन्तं निर्गुणेषु च}
{नेषा गुणान्कामयते नैर्गुण्यान्नानुरज्यते}
{उन्मत्ता गौरिवान्धा श्रीः क्वचिदेवावतिष्ठते}


\twolineshloka
{अग्निहोत्रफला वेदाः शीलवृत्तफलं श्रुतम्}
{रतिपुत्रफला नारी दत्तभुक्तफलं धनम्}


\twolineshloka
{अधर्मोपार्जितैरर्थैर्यः करोत्यौर्ध्वदेहिकम्}
{न स तस्य फलं प्रेत्य भुङ्क्तेऽर्थस्य दुरागमात्}


\twolineshloka
{कान्तारे वनदुर्गेषु कृच्छ्रास्वापत्सु संभ्रमे}
{उद्यतेषु च शस्त्रेषु नास्ति सत्ववतां भयम्}


\twolineshloka
{उत्थानं संयमो दाक्ष्यमप्रमादो धृतिः स्मृतिः}
{समीक्ष्य च समारम्भो विद्धि मूलं भवस्य तु}


\twolineshloka
{तपो बलं तापमानां ब्रह्म ब्रह्मविदां बलम्}
{हिंसा बलमसाधूनां क्षमा गुणवतां बलम्}


\twolineshloka
{अष्टौ तान्यव्रतघ्नानि आपो मूलं फलं पयः}
{हविर्ब्राह्मणकाम्या च गुरोर्वचनमौषधम्}


\twolineshloka
{न तत्परस्य सन्दध्यात्प्रतिकूलं यदात्मनः}
{सङ्ग्रहेणैव धर्मः स्यात्कामादन्यः प्रवर्तते}


\twolineshloka
{अक्रोधेन जयेत्क्रोधमसाधुं नाधुना जयेत्}
{जयेत्कदर्यं दानेन जयेन्मत्येन चानृतम्}


\twolineshloka
{स्त्रीधूर्तकेऽलसे भीरौ चण्डे पुरुषमानिनि}
{चोरे कृतघ्ने विश्वासो न कार्यो न च नास्तिके}


\twolineshloka
{अभिवादनशीलप्य नित्यं वृद्धोपसेविनः}
{चत्वारि संप्रवर्धन्ते कीर्तिरापुर्यशो बलम्}


\twolineshloka
{अतिक्लेशेन येऽथोः स्वधर्मस्यानिक्रमेण वा}
{अरेर्वा प्रणिपातेन मा .. तेषु मनः कृथाः}


\twolineshloka
{अविद्यः पुरुषः शोच्यः शोच्यं मैथुनमप्रजम्}
{निराहाराः प्रजाः शोच्याः शोच्यं राष्ट्रमराजकं}


\twolineshloka
{अध्वा जरा देहवतां पर्वतानां जलं जरा}
{असंभोगो जरा स्त्रीणां वाक्छल्यं मनसो जरा}


\twolineshloka
{अनाम्नायमला वेदा ब्राह्मणस्याव्रतं मलम्}
{मलं पृथिव्या वाह्लीकाः पुरुषस्यानृतं मलम्}


% Check verse!
कौतूहलमला साध्वी विप्रवासमलाः स्त्रियः
\twolineshloka
{सुवर्णस्य मलं रूप्यं रूप्यस्यापि मलं त्रपु}
{ज्ञेयं त्रपुमलं सीमं सीसस्यापि मलं मलम्}


\twolineshloka
{न स्वप्नेन जयेन्निद्रां न कामेन जयेत्स्त्रियः}
{नेन्धनेन जयेदग्निं न पानेन सुरां जयेत्}


\twolineshloka
{यस्य दानजितं मित्रं शत्रवो युधि निर्जिताः}
{अन्नपानजिता दाराः सफलं तस्य जीवितम्}


\twolineshloka
{सहस्रिणोऽपि जीवन्ति जीवन्ति शतिनस्तथा}
{धृतराष्ट्र विमुञ्चेच्छां न कथञ्चिन्न जीव्यते}


\twolineshloka
{यत्पृथिव्यां व्रीहियवं हिरण्यं पशवः स्त्रियः}
{नालमेकस्य तत्सर्वमिति पश्यन्न मुह्यति}


\twolineshloka
{राजन्भूयो ब्रवीमि त्वां पुत्रेषु सममाचर}
{समता यदि ते राजन्स्वेषु पाण्डुसुतेषु वा}


\chapter{अध्यायः ४०}
\twolineshloka
{विदुर उवाच}
{}


\twolineshloka
{योऽभ्यर्चितः मद्भिरसञ्जमानःकरोत्यर्थं शक्तिमहापयित्वा}
{क्षिप्रं यशस्तं समुपैति मन्त-मलं प्रसन्ना हि सुखाय सन्तः}


\twolineshloka
{महान्तमप्यर्थमधर्मयुक्तंयः मन्त्यजत्यनपाकृष्ट एव}
{सुखं सुदुःखान्यवमुच्य शेतेजीर्णां त्वचं सर्प इवावमुच्य}


\twolineshloka
{अनृते च समुत्कर्पो राजगामि च पैशुनम्}
{गुरोश्चालीकनिर्बन्धः समानि ब्रह्महत्यया}


\twolineshloka
{असूयैकपदं मृत्युरतिवादः श्रियो वधः}
{अशुश्रूषा त्वरा श्लाघा विद्यायाः शत्रवस्त्रः}


\twolineshloka
{आलस्यं मदमोहौ च चापलं गोष्ठिरेव च}
{स्तब्धता चाभिमानित्वं तथा त्यागित्वमेव च}


\threelineshloka
{एते वै सप्त दोषाः स्युः सदा विद्यार्थिनां मताः}
{सुखार्थिनः कुतो विद्या नास्ति विद्यार्थिनः सुखं}
{सुखार्थी वा त्यजेद्विद्यां विद्यार्थी वा त्यजेत्सुखम्}


\twolineshloka
{नाग्निस्तृप्यति काष्ठानां नापगानां महोदधिः}
{नान्तकः सर्वभूतानां न पुंसां वामलोचना}


\twolineshloka
{आशा धृतिं हन्ति समृद्धिमन्तकःक्रोधः श्रियं हन्ति यशः कदर्यता}
{अपालनं हन्ति पशूंश्च राज-न्नेकः क्रुद्धो ब्राह्मणो हन्ति राष्ट्रम्}


\twolineshloka
{अजाश्च कांस्यं रजतं च नित्यंमध्वाकर्षः शकुनिः श्रोत्रियश्च}
{वृद्धो ज्ञातिरवसन्नः कुलीनएतानि ते सन्तु गृहे सदैव}


\twolineshloka
{अजोक्षा चन्दनं वीणा आदर्शो मधुसर्पिषी}
{विषमौदुम्बरं शङ्खः स्वर्णनाभोऽथ रोचना}


\twolineshloka
{गृहे स्थापयितव्यानि धन्यानि मनुरब्रवीत्}
{देवब्राह्मणपूजार्थमतिथीनां च भारत}


\twolineshloka
{इदं च त्वां सर्वपरं ब्रवीमिपुण्यं पदं तात महाविशिष्टम्}
{न जातु कामान्न भयान्न लोभा-द्धर्मं जह्याज्जीविस्यापि हेतोः}


\twolineshloka
{नित्यो धर्मः सुखदुःखे त्वनित्येजीवो नित्यो धातुरस्य त्वनित्यः}
{त्यक्त्वाऽनित्यं प्रतितिष्ठस्व नित्येसंतुष्य सन्तोषपरा हि सन्तः}


\threelineshloka
{महाबलान्पश्य महानुभावान्प्रशास्य भूमिं धनधान्यपूर्णाम्}
{राज्यानि हित्वा विपुलांश्च भोगान्}
{गतान्नरेन्द्रान्वशमन्तकस्य}


\twolineshloka
{मृतं पुत्रं दुःखपुष्टं मनुष्याउत्क्षिप्य राजन्स्वगृहान्निर्हरन्ति}
{तं मुक्तकेशाः करुणं रुदन्तिचितामध्ये काष्ठमिव क्षिपन्ति}


\twolineshloka
{अन्यो धनं प्रेतगतस्य भुङ्क्तेवयांसि चाग्निश्च शरीरधातून्}
{द्वाभ्यामयं सह गच्छत्यमुत्रपुण्येन पापेन च वेष्ट्यमानः}


\twolineshloka
{उसृज्य विनिवर्तन्ते ज्ञातयः सुहृदः सुताः}
{अपुष्पानफलान्वृक्षान्यथा तात पतत्रिणः}


\twolineshloka
{अग्नौ प्रास्तं तु पुरुषं कर्मान्वेति स्वयं कृतम्}
{तस्मात्तु पुरुषो यत्नाद्धर्मं सञ्चिनुयाच्छनैः}


\twolineshloka
{अस्माल्लोकादूर्ध्वममुष्य चाधोमहत्तमस्तिष्ठति ह्यन्धकारम्}
{तद्वै महामोहनमिन्द्रियाणांबुद्ध्यस्व मा त्वां प्रलभेत राजन्}


\twolineshloka
{इदं वचः शक्ष्यसि चेद्यथाव-न्निशम्य सर्वं प्रतिपत्तुमेव}
{यशः परं प्राप्स्यसि जीवलोकेभयं नचामुत्र नचेह तेऽस्ति}


\twolineshloka
{आत्मा नदी भारत पुण्यतीर्थासत्योदया धृतिकूला दयोर्मिः}
{तस्यां स्नातः पूयते पुण्यकर्मापुण्यो ह्यात्मा नित्यमलोभ एव}


\twolineshloka
{कामक्रोधग्राहवतीं पञ्चेन्द्रियजलां नदीम्}
{नावं धृतिमयीं कृत्वा जन्मदुर्गाणि संतर}


\twolineshloka
{प्रज्ञावृद्धं धर्मवृद्धं स्वबन्धुंविद्यावृद्धं वयसा चापि वृद्धम्}
{कार्याकार्ये पूजयित्वा प्रसाद्ययः संपृच्छेन्न स मुह्येत्कदाचित्}


\twolineshloka
{धृत्या शिश्रोदरं रक्षेत्पाणिपादं च चक्षुषा}
{चक्षुःश्रोत्रे च मनसा मनो वाचं च कर्मणा}


\twolineshloka
{नित्योदकी नित्ययज्ञोपवीतीनित्यस्वाध्यायी पतितान्नवर्जी}
{सत्यं ब्रुवन्गुरवे कर्म कुर्व-न्न ब्राह्मणश्र्यवते ब्रह्मलोकात्}


\twolineshloka
{अधीत्य वेदान्परिसंस्तीर्य चाग्नी-निष्ट्वा यज्ञैः पालयित्वा प्रजाश्च}
{गोब्रह्मणार्थं शस्त्रपूतान्तरात्माहतः सङ्ग्रामे क्षत्रियः स्वर्गमेति}


\twolineshloka
{वैश्योऽधीत्य ब्राह्मणान्क्षत्रियांश्चधनैः काले संविभज्याश्रितांश्च}
{त्रेतापूतं धूममाघ्राय पुण्यंप्रेत्य स्वर्गे दिव्यसुखानि भुङ्क्ते}


\twolineshloka
{ब्रह्म क्षत्रं वैश्यवर्णं च शूद्रःक्रमेणैतान्न्यायतः पूजयानः}
{तुष्टेष्वेतेष्वव्यथो दग्धपाप-स्त्यक्त्वा देहं स्वर्गसुखानि भुङ्क्ते}


\threelineshloka
{चातुर्वर्ण्यस्यैष धर्मस्तवोक्तोहेतुं चानुब्रुवतो मे निबोध}
{क्षात्राद्धर्माद्धीयते पाण्डुपुत्र-स्तं त्वं राजन्राजधर्मे नियुङ्क्ष ॥धृतराष्ट्र उवाच}
{}


\twolineshloka
{एवमेतद्यथा त्वं मामनुशासनि नित्यदा}
{ममापि च मतिः सौम्य भवत्येवं यथात्थ माम्}


\twolineshloka
{सा तु बुद्धिः कृताप्येवं पाण्डवान्प्रति मे सदा}
{दुर्योधनं समासाद्य पुनर्विपरिवर्तते}


\twolineshloka
{न दिष्टमभ्यतिक्रान्तुं शक्यं भूतेन केनचित्}
{दिष्टमेव ध्रुवं मन्ये पौरुषं तु निरर्थकम्}


\chapter{अध्यायः ४१}
\twolineshloka
{धृतराष्ट्र उवाच}
{}


\threelineshloka
{अनुक्तं यदि ते किञ्चिद्वाचा विदुर विद्यते}
{तन्मे शुश्रूषतो ब्रूहि विचित्राणि हि भाषसे ॥विदुर उवाच}
{}


\twolineshloka
{धृतराष्ट्र कुमारो वै यः पुराणः सनातनः}
{तनत्सुजातः प्रोवाच मृत्युर्नास्तीति भारत}


\threelineshloka
{स ते गुह्यान्प्रकाशांश्च सर्वान्हृदयसंश्रयान्}
{प्रवक्ष्यति महाराज सर्वबुद्धिमतां वरः ॥धृतराष्ट्र उवाच}
{}


\threelineshloka
{किं त्वं न वेद तद्भूयो यन्मे ब्रूयात्सनातनः}
{त्वमेव विदुर ब्रूहि प्रज्ञाशेषोऽस्ति चेत्तव ॥विदुर उवाच}
{}


\twolineshloka
{शूद्रयोनावहं जातो नातोऽन्यद्वक्तुमुत्सहे}
{कुमारस्य तु याबुद्धिर्वेद तां शाश्वतीमहम्}


\threelineshloka
{ब्राह्मीं हि योनिमापन्नः सगुह्यमपि यो वदेत्}
{न तेन गर्ह्यो देवानां तस्मादेतद्ब्रवीमि ते ॥धृतराष्ट्र उवाच}
{}


\threelineshloka
{ब्रवीहि विदुर त्वं मे पुराणं तं सनातनम्}
{कथमेतेन देहेन स्यादिहैव समागमः ॥वैशंपायन उवाच}
{}


\twolineshloka
{चिन्तयामास विदुरस्तभृषिं शंसितव्रतम्}
{स च तच्चिन्तितं ज्ञात्वा दर्शयामास भारत}


\twolineshloka
{स चैनं प्रतिजग्राह विधिदृष्टेन कर्मणा}
{सुखोपविष्टं विश्रान्तमथैनं विदुरोऽब्रवीत्}


\twolineshloka
{भगवन्संशयः कश्चिद्धृतराष्ट्रस्य मानसः}
{यो न शक्यो मया वक्तुं त्वमस्मै वक्तुमर्हसि}


% Check verse!
यं श्रत्वाऽयं मनुष्येन्द्रः सर्वदुःखातिगो भवेत्
\threelineshloka
{लाभालाभौ प्रियद्वेष्यौ यथैनं न जरान्तकौ}
{विषहेरन्भयामर्षौ श्रुत्पिपासे मदोद्भवौ}
{अरतिश्चैव तन्द्री च कामक्रोधौ क्षयोदयौ}


\chapter{अध्यायः ४२}
\twolineshloka
{वैशंपायन उवाच}
{}


\threelineshloka
{ततो राजा धृतराष्ट्रो मनीषीसंपूज्य वाक्यं विदुरेरितं तत्}
{सनत्सुजातं रहिते महात्मापप्रच्छ बुद्धिं परमां बुभूषन् ॥धृतराष्ट्र उवाच}
{}


\threelineshloka
{सनत्सुजात यदिमं शृणोमिन मृत्युरस्तीति तवोपदेशम्}
{देवासुरा ह्याचरन्ब्रह्मचर्य-ममृत्यवे तत्कतरन्नु सत्यम् ॥सनत्सुजात उवाच}
{}


\twolineshloka
{अमृत्युं कर्मणा केचिन्मृत्युर्नास्तीति चापरे}
{शृणु मे ब्रुवतो राजन्यथैतन्मा विशङ्किथाः}


\twolineshloka
{उभे सत्ये क्षत्रियाद्य प्रवृत्तेमोहो मृत्युः संमतोऽयं कवीनाम्}
{प्रमादं वै मृत्युमहं ब्रवीमितथाऽप्रमादममृतत्वं ब्रवीमि}


\twolineshloka
{प्रमादाद्वै असुराः पराभव-न्नप्रमादाद्ब्रह्मभूताः सुराश्च}
{नैव मृत्युर्व्याघ्र इवात्ति जन्तू-न्न ह्यस्य रूपमुपलभ्यते हि}


\twolineshloka
{यमं त्वेके मृत्युमतोऽन्यमाहु-रात्मावासममृतं ब्रह्मचर्यम्}
{पितृलोके राज्यमनुशास्ति देवःशिवः शिवानामशिवोऽशिवानाम्}


\twolineshloka
{अस्यादेशान्निःसरते नराणांक्रोधः प्रमादो लोभरूपश्च मृत्युः}
{अहं गतेनैव चरन्विमार्गा-न्न चात्मनो योगमुपैति कश्चित्}


\twolineshloka
{ते मोहितास्तद्वशे वर्तमानाइतः प्रेतास्तत्र पुनः पतन्ति}
{ततस्तान्देवा अनुविप्लवन्तेअतो मृत्युर्मरणादभ्युपैति}


\twolineshloka
{कर्मोदये कर्मफलानुरागा-स्तत्रानु ते यान्ति न तरन्ति मृत्युम्}
{सदर्थयोगानवगमात्समन्ता-त्प्रवर्तते भोगभोगेन देही}


\twolineshloka
{तद्वै महामोहनमिन्द्रियाणांमिथ्यार्थयोगस्य गतिर्हि नित्या}
{मिथ्यार्थयोगाभिहतान्तरात्मास्मरन्नुपास्ते विषयान्समन्तात्}


\twolineshloka
{अभिध्या वै प्रथमं हन्ति लोकान्कामक्रोधावनुगृह्याशु पश्चात्}
{एते बालान्मृत्यवे प्रापयन्तिधीरास्तु धैर्येण तरन्ति मृत्युम्}


\twolineshloka
{सोऽभिध्यायन्नुत्पतिष्णन्निहन्या-दनादरेणाप्रतिबुध्यमानः}
{नैनं मृत्युर्मृत्युरिवात्ति भूत्वाएवं विद्वान्यो विनिहन्ति कामान्}


\twolineshloka
{कामानुसारी पुरुषः कामाननु विनश्यति}
{कामान्व्युदस्य धुनुते यत्किञ्चित्पुरुषो रजः}


\twolineshloka
{तमोप्रकाशो भूतानां नरकोऽयं प्रदृश्यते}
{मुह्यन्त इव धावन्ति गच्छन्तः श्वभ्रमुन्मुखाः}


\twolineshloka
{अमूढवृत्तेः पुरुषस्येह कुर्या-त्किं वै मृत्युस्तार्ण इवास्य व्याघ्रः}
{अमन्यमानः क्षत्रिय किञ्चिदन्य-न्नाधीयते तार्ण इवास्य सर्पः}


\fourlineindentedshloka
{क्रोधाल्लोभान्मोहभयान्तरात्मास वै मृत्युस्त्वच्छरिरे य एषः}
{एवं मृत्युं जायमानं विदित्वाज्ञाने तिष्ठन्न बिभेतीह मृत्योः}
{विनश्यते विषये तस्य मृत्यु-र्मृत्योर्यथा विषयं प्राप्य मर्त्यः ॥धृतराष्ट्र उवाच}
{}


\threelineshloka
{यानेवाहुरिज्यया साधुलोकान्द्विजातीनां पुण्यतमान्सनातनात्}
{तेषां परार्थं कथयन्तीह वेदाएतद्विद्वान्नैति कथं नु कर्म ॥सनत्सुजात उवाच}
{}


\threelineshloka
{एवं ह्यविद्वानुपयाति तत्रतत्रार्थजातं च वदन्ति वेदाः}
{सवेह आयाति परं परात्माप्रयाति मार्गेण निहत्य मार्गान् ॥धृतराष्ट्र उवाच}
{}


\threelineshloka
{कोऽसौ नियुङ्क्ते तमजं पुराणंस चेदिदं सर्वमनुक्रमेण}
{किं वास्य कार्यमथवा सुखं चतन्मे विद्वन्ब्रूहि सर्वं यथावत् ॥सनत्सुजात उवाच}
{}


\twolineshloka
{दोषो महानत्र विभेदयोगेह्यनादियोगेन भवन्ति नित्याः}
{तथास्य नाधिक्यमपैति किञ्चि-दनादियोगेन भवन्ति पुंसः}


\fourlineindentedshloka
{य एतद्वा भगवान्स नित्योविकारयोगेन करोति विश्वम्}
{तथाच तच्छक्तिरिति स्म मन्यते}
{तथार्थयोगेन भवन्ति वेदाः ॥धृतराष्ट्र उवाच}
{}


\threelineshloka
{यस्माद्धर्मान्नाचरन्तीह केचि-त्तथा धर्मान्केचिदिहाचरन्ति}
{धर्मः पापेन प्रतिहन्यते स्वि-दुताहो धर्मः प्रतिहन्ति पापम् ॥सनत्सुजात उवाच}
{}


\twolineshloka
{तस्मिन्स्थितौ वाप्युभयं हि नित्यंज्ञानेन विद्वान्प्रतिहन्ति सिद्धम्}
{तथान्यथा पुण्यमुपैति देहीतथागतं पापमुपैति सिद्धम्}


\threelineshloka
{गत्वोभयं कर्मणा युज्यते स्थिरंशुभस्य पापस्य स चापि कर्मणा}
{धर्मेण पापं प्रणुदतीह विद्वान्धर्मो बलीयानिति तत्र विद्धिः ॥धृतराष्ट्र उवाच}
{}


\threelineshloka
{यानिहाहुः स्वस्य धर्मस्य लोका-न्द्विजातीनां पुण्यकृतां सनातनान्}
{तेषां क्रमान्कथय ततोऽपि चान्या-न्नैतद्विद्वन्वेत्तुमिच्छामि कर्म ॥सनत्सुजात उवाच}
{}


\twolineshloka
{येषां व्रतेऽथ विस्पर्धा बले बलवतामिव}
{ते ब्राह्मणा इतः प्रेत्य स्वर्गे यान्ति प्रकाशताम्}


\twolineshloka
{येषां धर्मे च विस्पर्धा तेषां जज्ज्ञानसाधनम्}
{ते ब्राह्मणा इतो मुक्ताः स्वर्गं यान्ति त्रिविष्टपम्}


\twolineshloka
{तस्य सम्यक्समाचारमाहुर्वेदविदो जनाः}
{नैनं मन्येत भूयिष्ठं बाह्यमाभ्यन्तरं जनम्}


\twolineshloka
{यत्र मन्येत भूयिष्ठं प्रावृषीव तृणोदकम्}
{अन्नं पानं ब्राह्मणस्य तञ्जीवोन्नानुसंज्वरेत्}


\twolineshloka
{यत्राकथयमानस्य प्रयच्छत्यशिवं भयम्}
{अतिरिक्तमिवाकुर्वन्स श्रेयान्नेतरो जनः}


\threelineshloka
{यो वा कथयमानस्य ह्यात्मानं नानुसंज्वरेत्}
{ब्रह्मस्वं नोपभुञ्जीत तदन्नं संमतं सताम्}
{कुशवल्कलचेलाद्यं ब्रह्मस्वं योगिनो विदुः}


\twolineshloka
{यथा खं वान्तमश्नाति श्वा वै नित्यमभूतये}
{एवं ते वान्तमश्नन्ति स्ववीर्यस्योपसेवनात्}


\twolineshloka
{नित्यमज्ञातचर्या मे इति मन्येत ब्राह्मणः}
{ज्ञातीनां तु वसन्मध्ये नैवं विन्देत किञ्चन}


\twolineshloka
{कोह्येवमन्तरात्मानं ब्राह्मणो हन्तुमर्हति}
{निर्लिङ्गमचलं शुद्धं सर्वद्वन्द्वविवर्जितम्}


\twolineshloka
{योऽन्यथा सन्तमात्मानमन्यथा प्रतिपद्यते}
{किं तेन न कृतं पापं चोरेणात्मापहारिणा}


\twolineshloka
{अश्रान्तः स्यादनादाता संमतो निरुपद्रवः}
{शिष्टो न शिष्टवत्स स्याद्ब्राह्मणो ब्रह्मवित्कविः}


\twolineshloka
{अनाढ्या मानुषे वित्ते आढ्या दैवे तथा क्रतौ}
{ते दुर्धर्षा दुष्प्रकम्प्यास्तान्विद्याद्ब्रह्मणस्तनुम्}


\twolineshloka
{सर्वान्खिष्टकृतो देवान्विद्याद्य इह कश्चन}
{न स मानो ब्राह्मणस्य तस्मिन्प्रयतते स्वयम्}


\twolineshloka
{यमप्रयतमानं तु मानयन्ति समाहिताः}
{न मान्यमानो मन्येत नावमानेऽतिसंज्वरेत्}


\twolineshloka
{लोकः स्वभाववृत्तिर्हि निमेषोन्मषवत्सदा}
{विद्वांसो मानयन्तीह इति मन्येत मानितः}


\twolineshloka
{अधर्मनिपुणा मुढा लोके मायाविशारदाः}
{न मान्यं मानयिष्यन्ति एवं मन्येत मानितः}


\twolineshloka
{न वै मानं च मौनं च सहितौ वसतः सदा}
{अयं हि लोको मानस्य असौ मौनस्य तद्विदुः}


\twolineshloka
{श्रीर्हि मानार्थसंवासा सा चापि परिपन्थिनी}
{ब्राह्मी सुदुर्लभा श्रीर्हि प्रज्ञाहीनेन क्षत्रिय}


\twolineshloka
{द्वाराणि तस्येह वदन्ति सन्तोबहुप्रकाराणि दुराधराणि}
{सत्यार्जवे ह्रीर्दमशौचविद्याषण्मानमोहप्रतिबन्धनानि}


\chapter{अध्यायः ४३}
\twolineshloka
{धृतराष्ट्र उवाच}
{}


\threelineshloka
{कस्यैष मौनः कतरन्नु मौनंप्रब्रूहि विद्वन्निह मौनभावम्}
{मौनेन विद्वानुत याति मौनंकथं मुने मौनमिहाचरन्ति ॥सनत्सुजात उवाच}
{}


\threelineshloka
{यतो न वेदा मनसा सहैन-मनुप्रविशन्ति ततोऽथ मौनम्}
{यत्रोत्थितो वेदशब्दस्तथायंस तन्मयत्वेन विभाति राजन् ॥धृतराष्ट्र उवाच}
{}


\threelineshloka
{ऋचो यजूंषि यो वेद सामवेदं च वेद यः}
{पापानि कुर्वन्पापेन लिप्यते किं न लिप्यते ॥सनत्सुजात उवाच}
{}


\twolineshloka
{नैनं सामान्यृचो वापि न यजूंष्यविचक्षणम्}
{त्रायन्ते कर्मणः पापान्न ते मिथ्या ब्रवीम्यहम्}


\threelineshloka
{न च्छन्दांसि वृजिनात्तारयन्तिमायाविनं मायया सर्वमानम्}
{नीडं शकुन्ता इव जातपक्षा-श्छन्दांस्येनं प्रजहत्यन्तकाले ॥धृतराष्ट्र उवाच}
{}


\threelineshloka
{न चेद्वेदा वेदविदं त्रातुं शक्ता विचक्षण}
{अथ कस्मात्प्रलापोऽयं ब्राह्मणानां सनातनः ॥सनत्सुजात उवाच}
{}


\twolineshloka
{तस्यैव नामादिविशेषरूपै-रिदं जगद्भाति महानुभाव}
{निर्दिश्य सम्यक्प्रवदन्ति वेदा-स्तद्विश्ववैरूप्यमुदाहरन्ति}


\twolineshloka
{तदर्थयुक्तं तप एतदिज्याताभ्यामसौ पुण्यमुपैति विद्वान्}
{पुण्येन पापं विनिहत्य पश्चा-त्संजायते ज्ञानविदीपितात्मा}


\twolineshloka
{ज्ञानेन चात्मानमुपैति विद्वा-नथान्यथा वर्गफलानुकाङ्क्षी}
{अस्मिन्कृतं तत्परिगृह्य सर्व-ममुत्र भुङ्क्ते पुनरेति मार्गम्}


\twolineshloka
{अस्मिँल्लोके तपस्तप्तं फलमन्यत्र भुज्यते}
{ब्राह्मणानामिमे लोका वृद्धे तपसि संयताः}


\threelineshloka
{` ब्राह्मणानां तपः स्वृद्धमन्येषां तावदेव तत्}
{एतत्समृद्धमत्यृद्धं तपो भवति केवलम् ॥'धृतराष्ट्र उवाच}
{}


\threelineshloka
{कथं समृद्धमप्यृद्धं तपो भवति केवलम्}
{सनत्सुजात तद्ब्रूहि यथा विद्याम तद्वयम् ॥सनत्सुजात उवाच}
{}


\twolineshloka
{निष्कल्मषं तपस्त्वेतत्केवलं परिचक्षते}
{एतत्समृद्धमप्यृद्धं तपो भवति केवलम्}


\threelineshloka
{तपोमूलमिदं सर्वं यन्मां पृच्छसि क्षत्रिय}
{तपसा वेदविद्वांसः परं त्वमृतमाप्नुयुः ॥धृतराष्ट्र उवाच}
{}


\threelineshloka
{कल्मषं तपसो ब्रूहि श्रुतं निष्कल्मषं तपः}
{सनत्सुजात येनेदं विद्यां गुह्यं सनातनम् ॥सनत्सुजात उवाच}
{}


\twolineshloka
{क्रोधादयो द्वादश यस्य दोषा-स्तथा नृशंसानि दश त्रि राजन्}
{धर्मादयो द्वादशैते पितॄणांशास्त्रे गुणा ये विदिता द्विजानाम्}


\twolineshloka
{क्रोधः कामो लोभमोहौ विधित्सा-ऽकृपाऽसूये मानशोकौ स्पृहा च}
{ईर्ष्या जुगुप्सा च मनुष्यदोषावर्ज्याः सदा द्वादशैते नराणाम्}


\twolineshloka
{एकैकमेते राजेन्द्र मनुष्यान्पर्युपासते}
{लिप्समानोन्तरं तेषां मृगाणामिव लुब्धकः}


\twolineshloka
{विकत्थनः स्पृहयालुर्मनस्वीबिभ्रत्कोषं चपलो रोषणश्च}
{एतान्पापाः षण्णराः पापधर्मान्प्रकुर्वते नो त्रसन्तः सुदुर्गे}


\twolineshloka
{संभोगसंविद्विषमोऽतिमानीदत्तानुतापी कृपणो बलीयान्}
{वर्गप्रशंसी वनितासु द्वेष्टाएते परे सप्त नृशंसवर्गाः}


\twolineshloka
{धर्मश्च सत्यं च दमस्तपश्चआमात्सर्यं ह्रीस्तितिक्षानसूया}
{यज्ञश्च दानं च धृतिः श्रुतं चव्रतानि वै द्वादश ब्राह्मणस्य}


\twolineshloka
{यस्त्वेतेभ्यः प्रभवेद्द्वादशभ्यःसर्वामपीमां पृथिवीं स शिष्यात्}
{त्रिभिर्द्वाभ्यामेकतो वार्थितो य-स्तस्य स्वमस्तीति स वेदितव्यः}


\twolineshloka
{दमस्त्यागोऽप्रमादश्च एतेष्वमृतमाहितम्}
{तानि सत्यमुखान्याहुर्ब्राह्मणा ये मनीषिणः}


\twolineshloka
{दमोऽष्टादशदोषः स्यात्प्रातिकूल्यं कृते भवेत्}
{अनृतं चाभ्यसूया च कामार्थौ च तथा स्पृहा}


\twolineshloka
{क्रोधः शोकस्तथा तृष्णा लोभः पैशून्यमेव च}
{मत्सरश्च विहिंसा च परितापस्तथाऽरतिः}


\twolineshloka
{अपस्मारश्चातिवादस्तथा संभावनात्मनि}
{एतैर्विमुक्तो दोषैर्यः स दान्तः सद्भिरुच्यते}


\threelineshloka
{मदोऽष्टादशदोषः स्यात्त्यागो भवति षड्विधः}
{विपर्ययाः स्मृता एते मददोषा उदाहृताः}
{दोषा दमस्य ये प्रोक्तास्तान्दोषान्परिवर्जयेत्}


\twolineshloka
{श्रेयांस्तु षड्विधस्त्यागस्तृतीयो दुष्करो भवेत्}
{तेन दुःखं तरत्येव भिन्नं तस्मिञ्जितं कृते}


\twolineshloka
{श्रेयांस्तु षड्विधस्त्यागः श्रियं प्राप्य न हृष्यति}
{इष्टापूर्ते द्वितीयं स्यान्नित्यवैराग्ययोगतः}


% Check verse!
कामत्यागश्च राजेन्द्र स तृतीय इति स्मृतः ॥अप्यवाच्यं वदन्त्येतं स तृतीयो गुणः स्मृतः
\twolineshloka
{त्यक्तैर्द्रव्यैर्यद्भवति नोपयुक्तैश्च कामतः}
{न च द्रव्यैस्तद्भवति नोपयुक्तैश्च कामतः}


\twolineshloka
{न च कर्मस्वसिद्धेषु दुःखं ते न च न ग्लपेत्}
{सर्वैरेव गुणैर्युक्तो द्रव्यवानपि यो भवेत्}


\twolineshloka
{अप्रिये च समुत्पन्ने व्यथां जातु न गच्छति}
{इष्टान्पुत्रांश्च दारांश्च न याचेत कदाचन}


\twolineshloka
{अर्हते याचमानाय प्रदेयं तच्छुभं भवेत्}
{अप्रमादी भवेदेतैः स चाप्यष्टगुणो भवेत्}


\twolineshloka
{सत्यं ध्यानं समाधानं चोद्यं वैराग्यमेव च}
{अस्तेयं ब्रह्मचर्यं च तथाऽसंग्रहमेव च}


\twolineshloka
{एवं दोषा मदस्योक्तास्तान्दोषान्परिवर्जयेत्}
{तथा त्यागोऽप्रमादश्च स चाप्यष्टगुणो मतः}


\threelineshloka
{अष्टौ दोषाः प्रमादस्य तान्दोषान्परिवर्जयेत्}
{इन्द्रियेभ्यश्च पञ्चभ्यो मनसश्चैव भारत}
{अतीतानागतेभ्यश्च मुक्त्युपेतः सुखी भवेत्}


\twolineshloka
{सत्यात्मा भव राजेन्द्र सत्ये लोकाः प्रतिष्ठिताः}
{तांस्तु सत्यमुखानाहुः सत्ये ह्यमृतमाहितम्}


\twolineshloka
{निवृत्तेनैव दोषेण तपो व्रतमिहाचरेत्}
{एतद्धातृकृतं वृत्तं सत्यमेव सतां व्रतम्}


\twolineshloka
{दोषैरेतैर्वियुक्तस्तु गुणैरेतैः समन्वितः}
{एतत्समृद्धमत्यर्थं तपो भवति केवलम्}


\threelineshloka
{यन्मां पृच्छसि राजेन्द्र संक्षेपात्प्रब्रवीमि ते}
{एतत्पापहरं पुण्यं जन्ममृत्युजरापहम् ॥धृतराष्ट्र उवाच}
{}


\twolineshloka
{आख्यानप़ञ्चमैर्वेदैर्भूयिष्ठं कथ्यते जनः}
{तथा चान्ये चतुर्वेदास्त्रिधेदाश्च तथा परे}


\threelineshloka
{द्विवेदाश्चैकवेदाश्चाप्यनृचश्च तथा परे}
{तेषां तु कतरः स स्याद्यमहं वेद वै द्विजम् ॥सनत्सुजात उवाच}
{}


\twolineshloka
{एकस्य वेस्याज्ञानाद्वेदास्ते बहवः कृताः}
{सत्यस्यैकस्य राजेन्द्र सत्ये कश्चिदवस्थितः}


\twolineshloka
{एवं वेदमविज्ञाय प्राज्ञोऽहमिति मन्यते}
{दानमध्ययनं यज्ञो लोभादेतत्प्रवर्तते}


\twolineshloka
{सत्यात्प्रच्यवमानानां सङ्कल्पो वितथो भवेत्}
{ततो यज्ञः प्रतायेत सत्यस्यानवधारणात्}


\twolineshloka
{मनसान्यस्य भवति वाचान्यस्याथ कर्मणा}
{सङ्कल्पसिद्धः पुरुषः सङ्कल्पानधितिष्ठति}


\twolineshloka
{अनैभृत्येन चैतस्य दीक्षितव्रतमाचरेत्}
{नामैतद्धातुनिर्वृत्तं सत्यमेव सतां परम्}


\twolineshloka
{ज्ञानं वै नाम प्रत्यक्षं परोक्षं जायते तपः}
{विद्याद्बहु पठन्तं तु द्विजं वै बहुपाठिनम्}


\twolineshloka
{तस्मात्क्षत्रिय मा मंस्था जल्पितेनैव वै द्विजम्}
{य एव सत्यान्नापैति स ज्ञेयो ब्राह्मणस्त्वया}


% Check verse!
छन्दांसि नाम क्षत्रिय तान्यथर्वापुरा जगौ महर्षिसङ्घ एषःछन्दोविदस्ते य उत नाधीतवेदान वेदवेद्यस्य विदुर्हि तत्त्वम्
\twolineshloka
{छन्दांसि नाम द्विपदां वरिष्ठस्वच्छन्दयोगेन भवन्ति तत्र}
{छन्दोविदस्ते न च तानधीत्यगता न वेदस्य न वेद्यमार्याः}


\twolineshloka
{न वेदानां वेदिता कश्चिदस्तिकश्चित्त्वेतान्बुद्ध्यते वापि राजन्}
{यो वेद वेदान्न स वेद वेद्यंसत्ये स्थितो यस्तु स वेद वेद्यम्}


\twolineshloka
{न वेदानां वेदिता कश्चिदस्तिवेद्येन वेदं न विदुर्न वेद्यम्}
{यो वेद वेदं स च वेद वेद्यंयो वेद वेद्यं न स वेद सत्यम्}


\twolineshloka
{यो वेद वेदान्स च वेद वेद्यंन तं विदुर्वेदविदो न वेदाः}
{तथापि वेदेन विदन्ति वेदंये ब्राह्मणा वेदविदो भवन्ति}


\twolineshloka
{धामांशभागस्य तथा हि वेदायथा च शाखा हि महीरुहस्य}
{संवेदने चैव यथामनन्तितस्मिन्हि सत्ये परमात्मनोऽर्थे}


\twolineshloka
{अभिजानामि ब्राह्मणं व्याख्यातारं विचक्षणम्}
{यश्छिन्नविचिकित्सः स व्याचष्टे सर्वसंशयान्}


\twolineshloka
{नास्य पर्येषणं गच्छेत्प्रचीनं नोत दक्षिणम्}
{नार्वाचीनं कुतस्तिर्यङ्गादिशन्तु कथञ्चन}


\twolineshloka
{तस्य पर्येषणं गच्छेत्प्रत्यर्थिषु कथञ्चन}
{अविचिन्वन्निमं वेदे तपः पश्यति तं प्रभुम्}


\twolineshloka
{तूष्णींभूत उपासीत न चेष्टेन्मनसापि च}
{उपावर्तस्व तद्ब्रह्म अन्तरात्मनि विश्रुतम्}


\twolineshloka
{मौनान्न स मुनिर्भवति नारण्यवसनान्मुनिः}
{स्वलक्षणं तु यो वेद स मुनिः श्रेष्ठ उच्यते}


\twolineshloka
{सर्वार्थानां व्याकरणाद्वैयाकरण उच्यते}
{तन्मूलतो व्याकरणं व्याकरोतीति तत्तथा}


\twolineshloka
{प्रत्यक्षदर्शी लोकानां सर्वदर्शी भवेन्नरः}
{सत्ये वै ब्राह्मणस्तिष्ठंस्तद्विद्वान्सर्वविद्भवेत्}


\twolineshloka
{धर्मादिषु स्थितोऽप्येवं क्षत्रिय ब्रह्म पश्यति}
{वेदानां चानुपूर्व्येण एतद्बुद्ध्या ब्रवीमि ते}


\chapter{अध्यायः ४४}
\twolineshloka
{धृतराष्ट्र उवाच}
{}


\fourlineindentedshloka
{सनत्सुजात यामिमां परां त्वंब्राह्मीं वाचं वदसे विश्वरूपाम्}
{परां हि कामेन सुदुर्लभां कथांप्रब्रूहि मे वाक्यमिदं कुमार}
{सनत्सुजात उवाच}
{}


\threelineshloka
{नैतद्ब्रह्म त्वरमाणेन लभ्यंयन्मां पृच्छन्नतिहृष्यस्यतीव}
{बुद्धौ विलीने मनसि प्रचिन्त्यविद्या हि सा ब्रह्मचर्येण लभ्या ॥धृतराष्ट्र उवाच}
{}


\threelineshloka
{अत्यन्तविद्यामिति यत्सनातनींब्रवीषि त्वं ब्रह्मचर्येण सिद्धाम्}
{अनारभ्यां वसतीह कार्यकालेकथं ब्राह्मण्यममृतत्वं लभेत ॥सनत्सुजात उवाच}
{}


\threelineshloka
{अव्यक्तविद्यामभिधास्ये पराणींबुद्ध्या च तेषां ब्रह्मचर्येण सिद्धाम्}
{यां प्राप्यैनं मर्त्यलोकं त्यजन्तिया वै विद्या गुरुवृद्धेषु नित्या ॥धृतराष्ट्र उवाच}
{}


\threelineshloka
{ब्रह्मचर्येण या विद्या शक्या वेदितुमञ्चसा}
{तत्कथं ब्रह्मचर्यं स्यादेतद्ब्रह्मन्ब्रवीहि मे ॥सनत्सुजात उवाच}
{}


\twolineshloka
{आचार्ययोनिमिह ये प्रविश्यभूत्वा गर्भे ब्रह्मचर्यं चरन्ति}
{इहैव ते शास्त्रकारा भवन्तिप्रहाय देहं परमं यान्ति योगम्}


\twolineshloka
{अस्मिँल्लोके वै जयन्तीह कामा-न्ब्राह्मीं स्थितिं ह्यनुतितिक्षमाणाः}
{त आत्मानं निर्हरन्तीह देहा-न्मुञ्जादिषीकामिव सत्वसंस्थाः}


\twolineshloka
{शरीरमेतौ कुरुतः पिता माता च भारत}
{आचार्यशास्ता या जातिः सा पुण्या साऽऽजराऽमम}


\twolineshloka
{यः प्रावृणोत्यवितथेन वर्णा-नृतं कुर्वन्नमृतं संप्रयच्छम्}
{तं मन्येत पितरं मातरं चतस्मै न द्रुह्येत्कतमस्य जानन्}


\twolineshloka
{गुरुं शिष्यो नित्यमभिवादयीतस्वाध्यायमिच्छेच्छुचिरप्रमत्तः}
{मानं न कुर्यान्नादधीत रोप-मेप प्रथमो ब्रह्मचर्यस्य पादः}


\twolineshloka
{शिष्यवृत्तिक्रमेणैव विद्यामाप्नोति यः शुचिः}
{ब्रह्मचर्यव्रतस्यास्य प्रथमः पाद उच्यते}


\twolineshloka
{आचार्यस्य प्रियं कुर्यात्प्राणैरपि धनैरपि}
{कर्मणा मनसा वाचा द्वितीयः पाद उच्यते}


\twolineshloka
{समा गुरौ यथा वृत्तिर्गरुपत्न्यां तथा चरेत्}
{तत्पुत्रे च तथा कुर्वन्द्वितीयः पाद उच्यते}


\twolineshloka
{आचार्येणात्मकृतं विजानन्ज्ञात्वा चार्थं भावितोऽस्मीत्यनेन}
{यन्मन्यते तं प्रति हृष्टबुद्धिःस वै तृतीयो ब्रह्मचर्यस्य पादः}


\twolineshloka
{नाचार्यस्यानपाकृत्य प्रवासंप्राज्ञः कुर्वीत नैतदहं करोमि}
{इतीव मन्येत न भाषयेतस वै चतुर्थो ब्रह्मचर्यस्य पादः}


\twolineshloka
{कालेन पादं लभते तथार्थंततश्च पादं गुरुयोगतश्च}
{उत्साहयोगेन च पादमृच्छे-च्छास्त्रेण पादं च ततोऽभियाति}


\twolineshloka
{धर्मादयो द्वादश यस्य रूप-मन्यानि चाङ्गानि तथा बलं च}
{आचार्ययोगे फलतीति चाहु-र्ब्रह्मार्थयोगेन च ब्रह्मचर्यम्}


\twolineshloka
{एवं प्रवृत्तो यदुपालभेत वैधनमाचार्याय तदनुप्रयच्छेत्}
{स तां वृत्तिं बहुगुणामेवमेतिगुरोः पुत्रे भवति च वृत्तिरेषा}


\twolineshloka
{एवं वसन्सर्वतो वर्धतीहबहून्पुत्रांल्लभते च प्रतिष्ठाम्}
{वर्षन्ति चास्मै प्रदिशो दिशश्चवसत्यस्मिन्ब्रह्मचर्ये जनाश्च}


\twolineshloka
{एतेन ब्रह्मंचर्येण देवा देवत्वमाप्नुवन्}
{ऋषयश्च महाभागा ब्रह्मलोकं मनीषिणः}


\twolineshloka
{गन्धर्वाणामनेनैव रूपमप्सरसमभूत्}
{एतेन ब्रह्मचर्येण सूर्योऽप्यह्नाय जायते}


\twolineshloka
{आकाङ्क्ष्यार्थस्य संयोगाद्रसभेदार्थिनामिव}
{एवं ह्येते समाज्ञाय तादृग्भावं गता इमे}


\twolineshloka
{य आश्रयेत्पावयेच्चापि राज-न्सर्वं शरीरं तपसा तप्यमानः}
{एतेन वै बाल्यमभ्येति विद्वान्मृत्युं तथा स जयत्यन्तकाले}


\threelineshloka
{अन्तवतः क्षत्रिय ते जयन्तिलोकाञ्जनाः कर्मणा निर्मलेन}
{ब्रह्मैव विद्वांस्तेन चाभ्येति सर्वंनान्यः पन्था अयनाथ विद्यते ॥धृतराष्ट्र उवाच}
{}


\threelineshloka
{आभाति शुक्लमिव लोहितमि-वाथो कृष्णमथाञ्जनं काद्रवं वा}
{सद्ब्रह्मणः पश्यति योऽत्र विद्वा-न्कथं रूपं तदमृतमक्षरं पदम् ॥सनत्सुजात उवाच}
{}


\twolineshloka
{आभाति शुक्लमिव लोहितमि-वाथो कृष्णमायसमर्कवर्णम्}
{न पृथिव्यां तिष्ठति नान्तरिक्षेनैतत्समुद्रे सलिलं बिभर्ति}


\twolineshloka
{न तारकासु न च विद्युदाश्रितंन चाभ्रेषु दृश्यते रूपमस्य}
{न चापि वायौ न च देवतासुनैतच्चन्द्रे दृश्यते नोत सूर्ये}


\twolineshloka
{नैवर्क्षु तन्न यजुष्षु नाप्यथर्वसुन दृश्यते वै विमलेषु सामसु}
{रथन्तरे बार्हद्रथे वापि राज-न्महाव्रते नैव दृश्येद्भ्रुवं तत्}


\twolineshloka
{अपारणीयं तमसः परस्ता-त्तदन्तकोऽप्येति विनाशकाले}
{अणीयो रूपं क्षुरधारया समंमहच्च रूपं तद्वै पर्वतेभ्यः}


\twolineshloka
{सा प्रतिष्ठा तदमृतं लोकास्तद्ब्रह्म तद्यशः}
{भूतानि जझिरे तस्मात्प्रलयं यान्ति तत्र हि}


\twolineshloka
{अनामयं तन्महदुद्यतं यशोवाचो विकारं कवयो वदन्ति}
{यस्मिञ्जगत्सर्वमिदं प्रतिष्ठितंये तद्विदुरमृतास्ते भवन्ति}


\twolineshloka
{` तदेतदह्ना संस्थितं भाति सर्वंतदात्मवित्पश्यति ज्ञानयोगात्}
{तस्मिञ्जगत्सर्वमिदं प्रतिष्ठितंये तद्विदुरमृतास्ते भवन्ति ॥'}


\chapter{अध्यायः ४५}
\twolineshloka
{सनत्सुजात उवाच}
{}


\twolineshloka
{शोकः क्रोधश्च लोभश्च कामो मानः परासुता}
{ईर्ष्या मोहो विवित्सा च कृपाऽसूया जुगुप्सुता}


\threelineshloka
{द्वादशैते महादोषा मनुष्यप्राणनाशनाः}
{एकैकमेते राजेन्द्र मनुष्यान्पर्युपासते}
{यौराविष्टो नरः पापं मूढसंज्ञो व्यवस्यति}


\twolineshloka
{स्पृहयालुरुग्रः परुषो वा वदान्यःक्रोधं बिभ्रन्मनसा वै विकत्थी}
{नृशंसधर्माः षडिमे जना वैप्राप्याप्यर्थं नोत सभाजयन्ते}


\twolineshloka
{संभोगसंविद्विषमोऽतिमानीदत्त्वा विकत्थी कृपणो दुर्बलश्च}
{बहुप्रशंकसी वनिताद्विट् सदैवसप्तैवोक्ताः पापशीला नृशंसाः}


\twolineshloka
{धर्मश्च सत्यं च तपो दमश्चअमात्सर्यं ह्रीस्तितिक्षानसूया}
{दानं श्रुतं चैव धृतिः क्षमा चमहाव्रता द्वादश ब्राह्मणस्य}


\twolineshloka
{यो नैतेभ्यः प्रच्यवेद्द्वादशभ्यःसर्वामपीमां पृथिवीं स शिष्यात्}
{त्रिभिर्द्वाभ्यामेकतो वार्थितो योनास्य स्वमस्तीति च वेदितव्यम्}


\twolineshloka
{दमस्त्यागोऽथाप्रमाद इत्येतेष्वमृतं स्थितम्}
{एतानि ब्रह्ममुख्यानां ब्राह्मणानां मनीषिणाम्}


\twolineshloka
{सद्वाऽसद्वा परीवादे ब्राह्मणस्य न शस्यते}
{नरकप्रतिष्ठास्ते स्युर्य एवं कुरुते जनाः}


\twolineshloka
{मदोऽष्टादशदोषः स स्यात्पुरा योऽप्रकीर्तितः}
{लोकद्वेष्यं प्रतिकूल्यमभ्यसूया मृषावचः}


\twolineshloka
{कामक्रोधौ पारतन्त्र्यं परिवादोऽथ पैशुनम्}
{अर्थहानिर्विवादश्च मात्सर्यं प्राणिपीडनम्}


\twolineshloka
{ईर्ष्या मोदोऽतिवादश्च संज्ञानाशोऽभ्यसूयिता}
{तस्मात्प्राज्ञो न माद्येत सदा ह्येतद्विगर्हितम्}


\threelineshloka
{सौहृदे वै षङ्गुणा वेदितव्यःप्रिये हृष्यन्त्यप्रिये च व्यथन्ते}
{स्यादात्मनः सुचिरं याचते योददात्ययाच्यमपि देयं खलु स्यात्}
{इष्टान्पुत्रान्विभवान्स्वांश्च दारा-नभ्यर्थितश्चार्हति शुद्धभावः}


% Check verse!
त्यक्तद्रव्यः संवसेन्नेह कामा-द्भिङ्क्ते कर्मस्वाशिषं बाधते च
\twolineshloka
{द्रव्यवान्गुणवानेवं त्यागी भवति सात्विकः}
{पञ्चभूतानि पञ्चभ्यो निवर्तयति तादृशः}


\twolineshloka
{एतत्समृद्धमप्यृद्धं तपो भवति केवलम्}
{सत्वात्प्रच्यवमानानां सङ्कल्पेन समाहितम्}


\twolineshloka
{यतो यज्ञः प्रवर्धन्ते मत्यस्यैवावरोधनात्}
{मनसान्यस्य भवति वाच्यन्यस्याथ कर्मणा}


\twolineshloka
{सङ्कल्पसिद्धं पुरुषमसङ्कल्पोऽधितिष्ठति}
{ब्राह्मणस्य विशेषेण किञ्चान्यदपि मे श्रृणु}


\twolineshloka
{अध्यापयेन्महदेतद्यशस्यंवाचो विकराः कवयो वदन्ति}
{अस्मिन्योगे सर्वमिदं प्रतिष्ठितंये तद्विदुरमृतास्ते भवन्ति}


\twolineshloka
{न कर्मणा सुकृतेनैव राजन्सत्यं जयेज्जुहुयाद्वा यजेद्वा}
{नैतेन वालोऽमृत्युमभ्येति राजन्रतिं चासौ न लभत्यन्तकाले}


\twolineshloka
{तृष्णीमेक उपासीत चेष्टेत मनसापि न}
{तथा संस्तुतिनिन्दाभ्यां प्रीतिरोषौ विवर्जयेत्}


\twolineshloka
{अत्रैव तिष्ठन्क्षत्रिय ब्रह्माविशति पश्यति}
{वेदेषु चानुपूर्व्येण एतद्विद्वन्ब्रवीमि ते}


\chapter{अध्यायः ४६}
\twolineshloka
{सनत्सुजात उवाच}
{}


\threelineshloka
{यत्तच्छुक्रं महञ्ज्योतिर्दीप्यमानं महद्यशः}
{तद्वै देवा उपासते तस्मात्सूर्यो विराजते}
{योगिनस्तं प्रपश्यन्ति भगवन्तं सनातनम्}


\threelineshloka
{शुक्राद्ब्रह्म प्रभवति ब्रह्म शुक्रेण वर्धते}
{तच्छुकं ज्योतिषां मध्येऽतप्तं तपति तापनम्}
{योगिनस्तं प्रपश्यन्ति भगवन्तं सनातनम्}


\twolineshloka
{आपोऽथाद्भ्यः सलिलस्य मध्येउभौ देवौ शिश्रियातेऽन्तरिक्षे}
{अतन्द्रितः सवितुर्विवस्वा-नुभौ बिभर्ति पृथिवीं दिवं च ॥योगिनस्तं प्रपश्यन्ति भगवन्तं सनातनम्}


\twolineshloka
{उभौ च देवौ पृथिवीं दिवं चदिशः शुक्रो भुवनं बिभर्ति}
{तस्माद्दिशः सरितश्च स्रवन्तितस्मात्समुद्रा विहिता महान्ताः ॥योगिनस्तं प्रपश्यन्ति भगवन्तं सनातनम्}


\threelineshloka
{चक्रे रथस्य तिष्ठन्तोऽध्रुवस्याव्ययकर्मणः}
{केतुमन्तं वहन्त्यश्वास्तं दिव्यमजरं दिवि}
{योगिनस्तं प्रपश्यन्ति भगवन्तं सनातनम्}


\twolineshloka
{न सादृश्ये तिष्ठति रूपमस्यन चक्षुपा पश्यति कश्चिदेनम्}
{मनीषयाथो मनसा हृदा चय एनं विदुरमृतास्ते भवन्ति ॥योगिनस्तं प्रपश्यन्ति भगवन्तं सनातनम्}


\threelineshloka
{द्वादशपूगां सरितं पिबन्तो देवरक्षिताम्}
{मध्वीक्षन्तश्च ते तस्याः संचरन्तीह घोराम्}
{योगिनस्तं प्रपश्यन्ति भगवन्तं सनातनम्}


\threelineshloka
{तदर्धमासं पिबति संचितं भ्रमरो मधु}
{ईशानः सर्वभूतेषु हविर्भूतमकल्पयत्}
{योगिनस्तं प्रपश्यन्ति भगवन्तं सनातनम्}


\threelineshloka
{हिरण्यपर्णमश्वत्थमभिपद्य ह्यपक्षकाः}
{ते तत्र पक्षिणो भूत्वा प्रपतन्ति यथादिशम्}
{योगिनस्तं प्रपश्यन्ति भगवन्तं सनातनम्}


\threelineshloka
{पूर्णात्पूर्णान्युद्धरन्ति पूर्णात्पूर्णानि चक्रिरे}
{हरन्ति पूर्णात्पूर्णानि पूर्णमेवावशिष्यते}
{योगिनस्तं प्रपश्यन्ति भगवन्तं सनातनम्}


\twolineshloka
{तस्माद्वै वायुरायातस्तस्मिंश्च प्रयतः सदा}
{तस्मादग्निश्च सोमश्च तस्मिंश्च प्राण आततः}


\twolineshloka
{सर्वमेव ततो विद्यात्तत्तद्वक्तुं न शक्नुमः}
{योगिनस्तं प्रपश्यन्ति भगवन्तं सनातनम्}


\threelineshloka
{अपानं गिरपि प्राणः प्राणं गिरति चन्द्रमाः}
{आदित्यो गिरते चन्द्रमादित्यं गिरते परः}
{योगिनस्तं प्रपश्यन्ति भगवन्तं सनातनम्}


\threelineshloka
{एकं पादं नोत्क्षिपति सलिलाद्धंस उच्चरन्}
{तं चेत्सन्ततमूर्ध्वाय न मृत्युर्नामृतं भवेत्}
{योगिनस्तं प्रपश्यन्ति भगवन्तं सनातनम्}


\twolineshloka
{अङ्गुष्ठमात्रः पुरुषोऽन्तरात्मालिङ्गस्य योगेन स याति नित्यम्}
{तमीशमीड्यमनुकल्पमाद्यंपश्यन्ति मूढा न विराजमानम् ॥योगिनस्तं प्रपश्यन्ति भगवन्तं सनातनम्}


\twolineshloka
{असाधना वापि ससाधना वासमानमेतद्दृश्यते मानुषेषु}
{समानमेतदमृतस्येतरस्यमुक्तास्तत्र मध्व उत्सं समापुः ॥योगिनस्तं प्रपश्यन्ति भगवन्तं सनातनम्}


\twolineshloka
{उभौ लोकौ विद्यया व्याप्य यातितदा हुतं चाहुतमग्निहोत्रम्}
{मा ते ब्राह्मी लघुतामादधीतप्रज्ञानं स्यान्नाम धीरा लभन्ते ॥योगिनस्तं प्रपश्यन्ति भगवन्तं सनातनम्}


\threelineshloka
{एवंरूपो महात्मा स पावकं पुरुषो गिरन्}
{यो वै तं पुरुषं वेद तस्येहार्थो न रिष्यते}
{योगिनस्तं प्रपश्यन्ति भगवन्तं सनातनम्}


\threelineshloka
{यः सहस्रं सहस्राणां पक्षान्संतत्य संपतेत्}
{मध्यमे मध्य आगच्छेदपिचेत्स्यान्मनोजवः}
{योगिनस्तं प्रपश्यन्ति भगवन्तं सनातनम्}


\twolineshloka
{न दर्शने तिष्ठति रूपमस्यपश्यन्ति चैनं सुविशुद्धसत्वाः}
{हितो मनीषी मनसा न तप्यतेये प्रव्रजेयुरमृतास्ते भवन्ति ॥योगिनस्तं प्रपश्यन्ति भगवन्तं सनातनम्}


\twolineshloka
{गूहन्ति सर्पा इव गह्वराणिस्वशिक्षया स्वेन वृत्तेन मर्त्याः}
{तेषु प्रमुह्यन्ति जना विमूढायथाध्वानं मोहयन्ते भयाय ॥योगिनस्तं प्रपश्यन्ति भगवन्तं सनातनम्}


\twolineshloka
{नाहं सदाऽसत्कृतः स्यां न मृत्यु-र्न चामृत्युरमृतं मे कुतः स्यात्}
{सत्यानृते सत्यसमानबन्धेसतश्च योनिरतसतश्चैक एव ॥योगिनस्तं प्रपश्यन्ति भगवन्तं सनातनम्}


\twolineshloka
{न साधुना नोत असाधुना वा-ऽसमानमेतद्दृश्यते मानुषेषु}
{समानमेतदमृतस्य विद्या-देवं युक्तो मधु तद्वै परीप्सेत् ॥योगिनस्तं प्रपश्यन्ति भगवन्तं सनातनम्}


\twolineshloka
{नास्यातिवादा हृदयं तापयन्तिनानधीतं नाहुतमग्निहोत्रम्}
{मनो ब्राह्मी लघुतामादधीतप्रज्ञां चास्मै नाम धीरा लभन्ते ॥योगिनस्तं प्रपश्यन्ति भगवन्तं सनातनम्}


\threelineshloka
{एवं यः सर्वभूतेषु आत्मानमनुपश्यति}
{अन्यत्रान्यत्र युक्तेषु किं स शोचेत्ततः परम्}
{}


\twolineshloka
{यथोदपाने महति सर्वतः संप्लुतोदके}
{एवं सर्वेषु वेदेषु आत्मानमनुजानतः}


\twolineshloka
{अङ्गुष्ठमात्रः पुरुषो महात्मान दृश्यतेऽसौ हृदि संनिविष्टः}
{अजश्चरो दिवारात्रमतन्द्रितश्चस तं मत्वा कविरास्ते प्रसन्नः}


\twolineshloka
{अहमेव स्मृतो माता पिता पुत्रोऽस्म्यहं पुनः}
{आत्माहमपि सर्वस्य यच्च नास्ति यदस्ति च}


\twolineshloka
{पितामहोऽस्मि स्थविरः पिता पुत्रश्च भारत}
{ममैव यूयमात्मस्था न मे यूयं न वो ह्यहम्}


\twolineshloka
{आत्मैव स्थानं मम जन्म चात्माओतप्रोतोऽहमजरप्रतिष्ठः}
{अजश्चरो दिवारात्रमतन्द्रितोऽहंमां विज्ञाय कविरास्ते प्रसन्नः}


\twolineshloka
{अणोरणीयान्सुमनाः सर्वभूतेषु जाग्रति}
{पितरं सर्वभूतेषु पुष्करे निहितं विदुः}


\chapter{अध्यायः ४७}
\twolineshloka
{वैशंपायन उवाच}
{}


\twolineshloka
{एवं सनत्सुजातेन विदुरेण च धीमता}
{सार्धं कथयतो राज्ञः सा व्यतीयय सर्वरी}


\twolineshloka
{तस्यां रजन्यां व्युष्टायां राजानः सर्व एव ते}
{सभामाविविशुर्हृष्टाःक सूतस्योपदिदृक्षया}


\twolineshloka
{शुश्रूषमाणाः पार्थानां वाचो धर्मार्थसंहिताः}
{धृतराष्ट्रमुखाः सर्वे ययू राजन्सभां शुभाम्}


\twolineshloka
{सुधावदातां विस्तीर्णां कनकाजिरभूषिताम्}
{चन्द्रप्रभां सुरुचिरां सिक्तां चिन्दनवारिणा}


\twolineshloka
{रुचिरैरासनैः स्तीर्णां काञ्चनैर्दारवैरपि}
{अश्मसारमयैर्दान्तैःक स्वास्तीर्णैः सोत्तरच्छदैः}


\threelineshloka
{भीष्मो द्रोणः कृपः शल्यः कृतवर्मा जयद्रथः}
{अश्वत्थामा विकर्णश्च सोमदत्तश्च बाह्लिकः}
{}


\threelineshloka
{विदुरश्च महाप्राज्ञो युयुत्सुश्च महारथः}
{सर्वे च सहिताः शूराः पार्थिवा भरतर्षभ}
{धृतराष्ट्रं पुरस्कृत्य विविशुस्तां सभां शुभाम्}


\twolineshloka
{दुःशासनश्चित्रसेनः शकुनिश्चापि सौबलः}
{दुर्मुखो दुःसहः कर्ण उलूकोऽथ विविंशतिः}


\twolineshloka
{कुरुराजं पुरस्कृत्य दुर्योधनममर्षणम्}
{विविशुस्तां सभां राजन्सुराः शक्रसदो यथा}


\twolineshloka
{आविशद्भिस्तदा राजञ्शूरैः परिघबाहुभिः}
{शुशुभे सा सभा रजन्सिंहैरिव गिरेर्गुहा}


\twolineshloka
{ते प्रविश्य महेष्वासाः सभां सर्वे महौजसः}
{आसनानि विचित्राणि भेजिरे सूर्यवर्चसः}


\twolineshloka
{आसनस्थेषु सर्वेषु तेषु राजसु भारत}
{द्वाःस्थो निवेदयामास सूतपुत्रमुपस्थितम्}


\twolineshloka
{अयं सरथ आयाति योऽयासीत्पाण्डवान्प्रति}
{दूतो नस्तूर्णमायाति सैन्धवैः साधुवाहिभिः}


\threelineshloka
{..पेयाय स तु क्षिप्रं रथात्प्रस्कन्द्य कुण्डली}
{प्रविवेश सभां पूर्णां महीपालैर्महात्मभिः ॥सञ्जय उवाच}
{}


\twolineshloka
{प्राप्तोऽस्मि पाण्डवान्गत्वा तं विजानीत कौरवाः}
{यथावयः कुरून्सर्वान्प्रतिनन्दन्ति पाण्डवाः}


\twolineshloka
{अभिवादयन्ति वृद्धांश्च वयस्यांश्च वयस्यवत्}
{पुनश्चाभ्यवदन्पार्थाः प्रतिपूज्य यथावयः}


\threelineshloka
{यथाहं धृतराष्ट्रेण शिष्टः पूर्वमितो गतः}
{अब्रवं पाण्डवान्गत्वा नात्र किञ्चन हापितम् ॥` धृतराष्ट्र उवाच}
{}


\threelineshloka
{पृच्छामि त्वां सञ्जय राजमध्येयदब्रवीद्वाक्यमदीनसत्वः}
{जनार्दनस्तात युधां प्रणेतादुरात्मनां जीवितच्छिन्महात्मा ॥सञ्जय उवाच}
{}


\twolineshloka
{आगुल्फेभ्योऽभिसंवीतः प्रयतोऽहं कृताञ्जलिः}
{शुद्धान्तं प्राविशं राजन्नाख्यातो नरसिंहयोः}


\twolineshloka
{न चाभिमन्युर्न यमौ तं देशमभिजग्मतुः}
{यत्र कृष्णौ च कृष्णा च सत्यभामा च भामिनी}


\twolineshloka
{उभौ मध्वासवक्षीबौ वरचन्दनरूषितौ}
{स्त्रग्विणौ वरवस्त्राङ्गौ वराभरणभूषितौ}


\twolineshloka
{नैकरत्नविचित्रं च काञ्चनं च वरासनम्}
{नानास्तरणसंस्तीर्णं यत्रासाते नरर्षभौ}


\threelineshloka
{सत्याङ्कमुपधानं तु कृत्वा शेते जनार्दनः}
{अर्जुनाङ्कगतौ पादौ केशवस्योपलक्षये}
{अर्जुनस्य च कृष्णायाः शुभायाश्चाङ्कगावुभौ}


\twolineshloka
{काञ्चनं पादपीठं तु पार्थो वै प्रादिशन्मुदा}
{दासीभ्यामाहृतं मह्यं स्पृष्ट्वा भूमावुपाविशम्}


\twolineshloka
{ऊर्ध्वरेखाङ्कितौ पादौ पार्थस्य शुभलक्षणौ}
{पादपीठादपहृतौ धारयेतां वरस्त्रियौ}


\twolineshloka
{न नूनं कल्मषं किञ्चिन्मम कर्मसु विद्यते}
{स्त्रीरत्नाभ्यां समेतौ यन्मिथो मामभ्यभाषताम्}


\twolineshloka
{विस्मयो मे महानासीदास्त्रं मे बहुसङ्गतम्}
{हृष्टानि चैव रोमाणि दृष्ट्वा तौ सहितावुभौ}


\twolineshloka
{श्यामौ बृहन्तौ तरुणौ नागाविव समुच्छ्रितौ}
{एकशय्यागतौ दृष्ट्वा भयं मे महदाविशत्}


\threelineshloka
{ततोऽभ्यचिन्तयं तत्र दृष्ट्वा तौ पुरुषर्षभौ}
{सङ्कल्पो धर्मराजस्य नानवाप्योऽस्ति कश्चन}
{निदेशगाविमौ यस्य नरनारायणावुभौ}


\twolineshloka
{सत्कृतश्चान्नपानाभ्यामाच्छन्नो लब्धसत्क्रियः}
{अञ्जलिं मूर्ध्नि सन्धाय सन्देशं चाभ्यचोदयम्}


\twolineshloka
{धनुर्धरोचितेनाथ पाणिनैकं सलक्षणम्}
{पादमानाययत्पार्थः केशवस्य यशस्विनः}


\twolineshloka
{इन्द्रकेतुरिवोत्थाय दिव्याभरणभूषितः}
{इन्द्रवीर्योपमः कृष्णः संविष्टो माऽभ्यभाषत}


\twolineshloka
{स वाचं वदतां श्रेष्ठ आददे वचनक्षमाम्}
{दीपनीं धार्तराष्ट्राणां मृदुपूर्वां सुदारुणाम्}


\twolineshloka
{बहिश्चरस्य प्राणस्य प्रियस्य प्रियकारिणः}
{मतिमान्मतिमास्थाय केशवः सन्दधे वचः}


\threelineshloka
{वचनं वचनज्ञस्य शिक्षाक्षरसमन्वितम्}
{मनःप्रह्लादनं श्रेष्ठं पश्चाद्धृदयतापनम् ॥श्रीभगवानुवाच}
{}


\twolineshloka
{सञ्जयैतद्वचो ब्रूयाः प्राप्य क्षत्रियसंसदम्}
{श्रृण्वतः कुरुवृद्धस्य आचार्यस्य च धीमतः}


\twolineshloka
{अर्थांस्त्यजत पार्थेषु सुखमाप्नुत कामजम्}
{प्रियं प्रियेभ्यश्चरत राजा हि त्वरते जये}


\twolineshloka
{यजध्वं विविधैर्यज्ञैर्दक्षिणाश्च प्रयच्छत}
{पुत्रेर्दारैश्च मोदध्वमागतं वो महद्भयम्}


\twolineshloka
{ऋणं प्रवृद्धमिव मे हृदयान्नापसर्पति}
{गोविन्देति यदाक्रन्दत्कृष्णा मां दूरवासिनम्}


\twolineshloka
{तेजोमयं दुराधर्पं बिभ्रता गाण्डिवं धनुः}
{मद्द्वितीयेन पार्थेन वैरं वः प्रत्युपस्थितम्}


\twolineshloka
{कृष्णस्यैतद्वचः श्रुत्वा महन्मे भयमाविशत्}
{तव पुत्रस्य लोभं च वर्धमानं प्रपश्यतः}


\twolineshloka
{सोमेन्द्रसदृशौ वीरौ तौ मन्दो नावबुध्यते}
{भीष्मद्रोणाश्रयाच्चैव कर्णस्य च विकत्थनात् '}


\chapter{अध्यायः ४८}
\twolineshloka
{धृतराष्ट्र उवाच}
{}


\threelineshloka
{पृच्छामि त्वां सञ्जय राजमध्येकिमब्रवीद्वाक्यमदीनसत्वः}
{धनञ्जयस्तात युधां प्रेणतादुरात्मनां जीवितच्छिन्महात्मा ॥सञ्जय उवाच}
{}


\twolineshloka
{दुर्योधनो वाचमिमां श्रृणोतुयदब्रवीदर्जुनो योत्स्यमानः}
{युधिष्ठिरस्यानुमते महात्माधनञ्जयः श्रृण्वतः केशवस्य}


\twolineshloka
{अवित्रस्तो बाहुवीर्यं विजान-न्नुपह्वरे वासुदेवस्य धीरः}
{अवोचन्मां योत्स्यमानः किरीटीमध्ये ब्रूया धार्तराष्ट्रं कुरूणाम्}


\twolineshloka
{संश्रृण्वतस्तस्य दुर्भाषिणो वैदुरात्मनः सूतपुत्रस्य सूत}
{यो योद्धुमाशंसति मां सदैवमन्दप्रज्ञः कालपक्वोऽतिमूढः}


\twolineshloka
{ये वै राजानः पाण्डवायोधनायसमानीताः श्रृण्वतां चापि तेषाम्}
{यथा समग्रं वचनं मयोक्तंसहामात्यं श्रावयेथा नृपं तत}


\twolineshloka
{यथा नूनं देवराजस्य देवाःशुश्रूषन्ते वज्रहस्तस्य सर्वे}
{तथाऽश्रृण्वन्पाण्डवाः सृञ्जयाश्चकिरीटिना वाचमुक्तां समर्थाम्}


\twolineshloka
{इत्यब्रवीदर्जुनो योत्स्यमानोगाण्डीवधन्वा लोहितपद्मनेत्रः}
{न चेद्राज्यं मुञ्चति धार्तराष्ट्रोयधिष्ठिरस्याजमीढस्य राज्ञः}


\twolineshloka
{अस्ति नूनं कर्म कृतं पुरस्ता-दनिर्विष्टं पापकं धार्तराष्ट्रैः}
{येषां युद्धं भीमसेनार्जुनाभ्यांतथाश्विभ्यां वासुदेवेन चैव}


\twolineshloka
{शैनयेन ध्रुवमात्तायुधेनधृष्टद्युम्नेनाथ शिखण्डिना च}
{युधिष्ठिरेणेन्द्रकल्पेन चैवयोऽपध्यानान्निर्दहेद्गां दिवं च}


\twolineshloka
{तैश्चेद्योद्धुं मन्यते धार्तराष्ट्रोनिर्वृत्तोऽर्थः सकलः पाण्डवानाम्}
{मा तत्कार्षीः पाण्डवस्यार्थहेतो-रुपैहि युद्धं यदि मन्यसे त्वम्}


\twolineshloka
{यां तां वने दुःखशय्यामवात्सी-त्प्रव्राजितः पाण्डवो धर्मचारी}
{आप्नोतु तां दुःखतरामनर्था-मन्त्यां शय्यां धार्तराष्ट्रः पराशुः}


\twolineshloka
{ह्रिया ज्ञानेन तपसा दमेनशौर्येणाथो धर्मगुप्त्या धनेन}
{अन्यायवृत्तिः कुरु पाण्डवे या-नध्यातिष्ठद्धार्तराष्ट्रो दुरात्मा}


\twolineshloka
{मायोपधं प्रतिधानार्जवाभ्यांतपोदमाभ्यां धर्मगुप्त्या बलेन}
{सत्यं ब्रुवन्प्रतिपन्नो नृपो न-स्तितिक्षमाणः क्लिश्यमानोऽतिवेलम्}


\twolineshloka
{यदा ज्येष्ठः पाण्डवः संशितात्माक्रोधं यत्तं वर्षपूगान्सुघोरम्}
{अवस्त्रष्टा कुरुपूद्वृत्तचेता-स्तदा युद्धं धार्तराष्ट्रोऽन्वतप्स्यत्}


\twolineshloka
{कृष्णवर्त्मेव ज्वलितः समिद्धोयथा दहेत्कक्षमग्निर्निदाघे}
{एवं दग्धा धार्तराष्ट्रस्य सेनांयुधिष्ठिरः क्रोधदीप्तोन्ववेक्ष्य}


\twolineshloka
{यदा द्रष्टा भीमसेनं रथस्यंगदाहस्तं क्रोधविषं वमन्तम्}
{अमर्षणं पाण्डवं भीमवेगंतदा युद्धं धार्तराष्ट्रोऽन्वतप्स्यत्}


\twolineshloka
{सेनागगं दंशितं भीमसेनंसुलक्षणं वीरहणं परेषाम्}
{घ्नन्तं चमूमन्तकसन्निकाशंतदा स्मर्ता वचनस्यातिमानी}


\twolineshloka
{यदा द्रष्टा भीमसेनेन नागा-न्निपातितान्गिरिकूटप्रकाशान्}
{किम्भैरिवासृग्वमतो भिन्नकुम्भां-स्तदा युद्धं धार्तराष्ट्रोऽन्वतप्स्यत्}


\twolineshloka
{महासिंहो गा इव संप्रविश्यगदापाणिर्धार्तराष्ट्रानुपेत्य}
{यदा भीमो भीमरूपो निहन्तातदा युद्धं धार्तराष्ट्रोऽन्वतप्स्यत्}


\twolineshloka
{महाभये वीतभयः कृतास्त्रःसमागमे शत्रुबलावमर्दी}
{सकृद्रथेनाप्रतिमान्रथौघा-न्पदातिसङ्घान्गदयाभिनिघ्नन्}


\twolineshloka
{सैन्याननेकांस्तरसा विगृह्ण-न्यदा छेत्ता धार्तराष्ट्रस्य सैन्यम्}
{छिन्दन्वनं परशुनेव शूर-स्तदा युद्धं धार्तराष्ट्रोऽन्वतप्स्यत्}


\twolineshloka
{तृणप्रायं ज्वलनेनेव दग्धंग्रामं यथा धार्तराष्ट्रान्समीक्ष्य}
{पक्वं सस्यं वैद्युतेनेव दग्धंपरासिक्तं विपुलं स्वं बलौघम्}


\twolineshloka
{हतप्रवीरं विमुखं भयार्तंपराङ्मुखं प्रायशोऽनष्टयोधम्}
{शस्त्रार्चिषा भिमसेनेन दग्धंतदा युद्धं धार्तराष्ट्रोऽन्वतप्त्सत्}


\twolineshloka
{उपासङ्गानाचरेद्दक्षिणेनवराङ्गानां नकुलश्चित्रयोधी}
{यदा रथाग्र्यो रथिनः प्रचेतातदा युद्धं धार्तराष्ट्रोऽन्वतप्स्यत्}


\twolineshloka
{सुखोचितो दुःखशय्यां वनेषुदीर्घं कालं नकुलो यामशेत}
{आशीविषः क्रुद्ध इवोद्वमन्विषंतदा युद्धं धार्तराष्ट्रोऽन्वतप्स्यत्}


\twolineshloka
{त्यक्तात्मानः पार्थिवायोधनायसमादिष्टा धर्मराजेन सूत}
{रथैः शुभ्रैः सैन्यमभिद्रवन्तोदृष्ट्वा पश्चात्तप्स्यते धार्तराष्ट्रः}


\twolineshloka
{शिशून्कृतास्त्रानशिशुप्रकाशान्यदा द्रष्टा कौरवः पञ्च शूरान्}
{त्यक्त्वा प्राणान्कौरवानाद्रवन्त-स्तदा युद्धं धार्तराष्ट्रोऽन्वतप्स्यत्}


\twolineshloka
{यथा शक्रो दानवनाशनार्थंसुवर्णतारं रथमुत्तमाश्वैः}
{दान्तैर्युक्तं सहदेवोऽधिरूढःशिरांसि राज्ञां क्षेप्स्यते मार्गणौघैः}


\twolineshloka
{महाभये संप्रवृत्ते रथस्थंविवर्तमानं समरे कृतास्त्रम्}
{सर्वा दिशः सपतन्तं समीक्ष्यतदा युद्धं धार्तराष्ट्रोऽन्वतप्स्यत्}


\twolineshloka
{ह्रीनिषेवो निपुणः सत्यवादीमहाबलः सर्वधर्मोपपन्नः}
{गान्धारिमार्छंस्तुमुले क्षिप्रकारीक्षेप्ता जनान्सहदेवस्तरस्वी}


\twolineshloka
{यदा द्रष्टा द्रौपदेयान्महेषून्शूरान्कृतास्त्रान्रथयुद्धकोविदान्}
{आशीविषान्घोरविषानिवायत-स्तदा युद्धं धार्तराष्ट्रोऽन्वतप्स्यत्}


\twolineshloka
{यदाभिमन्युः परवीरघातीशरैः परान्मेघ इवाभिवर्षन्}
{विगाहिता कृष्णसमः कृतास्त्र-स्तदा युद्धं धार्तराष्ट्रोऽन्वतप्स्यत्}


\twolineshloka
{यदा द्रष्टा बालमबालवीर्यंद्विषच्चमूं मृत्युमिवोत्पतन्तम्}
{सौभद्रमिन्द्रप्रतिमं कृतास्त्रंतदा युद्धं धार्तराष्ट्रोऽन्वतप्स्यत्}


\twolineshloka
{प्रभद्रकाः शीघ्रतरा युवानोविशारदाः सिंहसमानवीर्याः}
{यदा क्षेप्तारो धार्तराष्ट्रान्ससैन्यां-स्तदा युद्धं धार्तराष्ट्रोऽन्वतप्स्यत्}


\twolineshloka
{वृद्धौ विराटद्रुपदौ महारथौपथक्कमूभ्यामभिवर्तमानौ}
{यदा द्रष्टारौ धार्तराष्ट्रान्ससैन्यां-स्तदा युद्धं धार्तराष्ट्रोऽन्वतप्स्यत्}


\twolineshloka
{यदा कृतास्त्रो द्रुपदः प्रचिन्वन्शिरांसि यूनां समरे रथस्थः}
{क्रूद्धः शरैश्छेत्स्यति चापमुक्तै-स्तदा युद्धं धार्तराष्ट्रोऽन्वतप्स्यत्}


\twolineshloka
{यदा विराटः परवीरघातीरथान्तरे शत्रुचमूं प्रवेष्टा}
{मात्स्यैः सार्धमनृशंसरूपै-स्तदा युद्धं धार्तराष्ट्रोऽन्वतप्स्यत्}


\twolineshloka
{ज्येष्ठं मात्स्यमनृसंसार्यरूपंविराटपुत्रं रथिनं पुरस्तात्}
{यदा द्रष्टा दंशतिं पाण्डवार्थेतदा युद्धं धार्तराष्ट्रोऽन्वतप्स्यत्}


\twolineshloka
{रणे हते कौरवाणां प्रवीरेशिखण्डिना शन्तनोर्वै तनूजे}
{न जातु नः शत्रवो धारयेयु-रशंशयं सत्यमेतद्ब्रवीति}


\twolineshloka
{यदा शिखण्डी रथिनः प्रचिन्वन्भीष्मं रथेनाभियाता वरूथी}
{दिव्यैर्हयैरवमृद्गन् रथौघां-स्तदा युद्धं धार्तराष्ट्रोऽन्वतप्स्यत्}


\twolineshloka
{यदा द्रष्टा सृञ्जयानामनीकेधृष्टद्नुम्नं प्रमुखे रोचमानम्}
{अस्त्रं यस्मै गुह्यमुवाच धीमा-न्द्रोणस्तदा तप्स्यति धार्तराष्ट्रः}


\twolineshloka
{यदा स सेनापतिरप्रमेयःपरामृद्गन्निषुभिर्धार्तराष्ट्रान्}
{द्रोणं रणे शत्रुसहोभियातातदा युद्धं धार्ताराष्ट्रोऽन्वतप्स्यत्}


\twolineshloka
{ह्रीमान्मनीपी बलवान्मनस्वीस लक्ष्मीवान्सोमकानां प्रबर्हः}
{न जातु तं शत्रवोऽन्ये सहेर-न्योपां स स्यादग्रणीर्वृष्णिसिंहः}


\twolineshloka
{इदं च ब्रूया मा वृणीष्वेति लोकेयुद्धेऽद्वितीयं सचिवं रथस्थम्}
{शिनेर्नप्तारं प्रवृणीम सात्यकिंमहाबलं वीतभयं कृतास्त्रम्}


\twolineshloka
{महोरस्को दीर्घबाहुः प्रमाथीयुद्धेऽद्वितीयः परमास्त्रवेदी}
{शिनेर्नप्ता तालमात्रायधोऽयंमहारथो वीतभयः कृतास्त्रः}


\twolineshloka
{यदा शिनीनामधिपो मयोक्तःशरैः परान्मेघ इव प्रवर्षन्}
{प्रच्छादयिष्यत्यरिहा योधमुख्यां-स्तदा युद्धं धार्तराष्ट्रोऽन्वतप्स्यत्}


\twolineshloka
{यदा धृतिं कुरुते योत्स्यमानःस दीर्घबाहुर्दृढधन्वा महात्मा}
{सिंहस्येव गन्धमाघ्राय गावःसंचेष्टन्ते शत्रवोऽस्माद्रणाग्रे}


\twolineshloka
{स दीर्घबाहुर्दृढधन्वा महात्माभिन्द्याद्गिरीन्संहरेत्सर्वलोकान्}
{अस्त्रे कृती निपुणः क्षिप्रहस्तोदिवि स्थितः सूर्य इवाभिभाति}


\twolineshloka
{चित्रः सूक्ष्मः सुकृतो यादवस्यअस्त्रे योगो वृष्णिसिंहस्य भूयान्}
{यथाविधं योगमाहुः प्रशस्तंसर्वैर्गुणैः सात्यकिस्तैरुपेतः}


\twolineshloka
{हिरण्मयं श्वेतहयैश्चतुर्भि-र्यदा युक्तं स्यन्दनं माधवस्य}
{द्रष्टा युद्धे सात्यकेर्धार्तराष्ट्र-स्तदा तप्स्यत्यकृतात्मा स मन्दः}


\twolineshloka
{यदा रथं हेममणिप्रकाशंश्वेताश्वयुक्तं वानरकेतुमुग्रम्}
{दृष्ट्वा ममाप्यास्थितं केशवेनतदा तप्स्यत्यकृतात्मा स मन्दः}


\twolineshloka
{यदा मौर्व्यास्तलनिष्पेपमुग्रंमहाशब्दं वज्रनिष्पेपतुल्यम्}
{विद्यूयमानस्य महारणे मयास गाण्डिवस्य श्रोष्यति मन्दबुद्धिः}


\twolineshloka
{तदा मूढो धृतराष्ट्रस्य पुत्र-स्तप्ता युद्धे दुर्मतिर्दुःसहायः}
{दृष्ट्वा सैन्यं बाणवर्षान्धकारेप्रभज्यन्तं गोकुलवद्रणाग्रे}


\twolineshloka
{बलाहकादुच्चरतः सुभीमा-न्विद्युत्स्फुलिङ्गानिव घोररूपान्}
{सहस्रघ्नान्द्विषतां सङ्गरेषुअस्थिच्छिदो मर्मभिदः सुपुङ्खान्}


\twolineshloka
{यदा द्रष्टा ज्यामुखाद्बाणसङ्घान्गाण्डीवमुक्तानापततः शिताग्रान्}
{हयान्गजान्वर्मिणश्चाददानां-स्तदा युद्धं धार्तराष्ट्रोऽन्वतप्स्यत्}


\twolineshloka
{यदा मन्दः परबाणान्विमुक्ता-न्ममेषुभिर्हियमाणान्प्रतीपम्}
{तिर्यग्विध्य च्छिद्यमानान्पृषत्कै-स्तदा युद्धं धार्तराष्ट्रोऽन्वतप्स्यत्}


\twolineshloka
{यदा विपाठा मद्भुजविप्रमुक्ताद्विजाः फलानीव महीरुहाग्रात्}
{प्रचेतार उत्तमाङ्गानि यूनांतदा युद्धं धार्तराष्ट्रोऽन्वतप्स्यत्}


\twolineshloka
{यदा द्रष्टा पततः स्यन्दनेभ्योमहागजेभ्योऽश्वगतान्सुयोधनान्}
{शरैर्हतान्पातितांश्चैव रङ्गेतदा युद्धं धार्तराष्ट्रोऽन्वतप्स्यत्}


\twolineshloka
{असंप्राप्तानस्त्रपथं परस्ययदा द्रष्टा नश्यतो धार्तराष्ट्रान्}
{अकुर्वतः कर्म युद्धे समन्ता-त्तदा युद्धं धार्तराष्ट्रोऽन्वतप्स्यत्}


\twolineshloka
{पदातिसङ्घान्रथसङ्घान्समन्ता-द्व्यात्ताननः काल इवाततेषुः}
{प्रणोत्स्यामि ज्वलितैर्बाणवर्षैःशत्रूंस्तदा तप्स्यति मन्दबुद्धिः}


\twolineshloka
{सर्वा दिशः संपतता रथेनरजोध्वस्तं गाण्डिवेन प्रकृत्तम्}
{यदा द्रष्टा स्वबलं संप्रमूढंतदा पश्चात्तप्स्यति मन्दबुद्धिः}


\twolineshloka
{कान्दिग्भूतं छिन्नगात्रं विसंज्ञंदुर्योधनो द्रक्ष्यति सर्वसैन्यम्}
{हताश्ववीराग्र्यनरेन्द्रनागंपिपासितं श्रान्तपत्रं भयार्तम्}


\twolineshloka
{आर्तस्वरं हन्यमानं हतं चविकीर्णकेशास्थिकपालसङ्घम्}
{प्रजापतेः कर्म यथार्थनिष्ठितंतदा दृष्ट्वा तप्स्यति मन्दबुद्धिः}


\twolineshloka
{यदा रथे गाण्डिवं वासुदेवंदिव्यं शङ्खं पाञ्चजन्यं हयांश्च}
{तूणावक्षय्यौ देवदत्तं च मां चदृष्ट्वा युद्धे धार्तराष्ट्रोऽन्वतप्स्यत्}


\twolineshloka
{उद्वर्तयन्दस्युसङ्घान्समेतान्प्रवर्तयन्युगमन्यद्युगान्ते}
{यदा धक्ष्याम्यग्निवत्कौरवेयां-स्तदा तप्ता धृतराष्ट्रः सपुत्रः}


\twolineshloka
{सभ्राता वै सहसैन्यः सभृत्योभ्रष्टैश्वर्यः क्रोधवशोऽल्पचेताः}
{दर्पस्यान्ते निहतो वेपमानःपश्चान्मन्दस्तप्स्यति धार्तराष्ट्रः}


\twolineshloka
{पूर्वाह्णे मां कृतजप्यं कदाचि-द्विप्रः प्रोवाचोदकान्ते मनोज्ञम्}
{कर्तव्यं ते दुष्करं कर्म पार्थयोद्धव्यं कते शत्रुभिः सव्यसाचिन्}


\twolineshloka
{इन्द्रो वा ते हरिवान्वज्रहस्तःपुरस्ताद्यातु समरेऽरीन्विनिघ्नन्}
{सुग्रीवयुक्तेन रथेन वा तेपश्चात्कृष्णो रक्षतु वासुदेवः}


\twolineshloka
{वज्रे चाहं वज्रहस्तान्महेन्द्रा-दस्मिन्युद्धे वासुदेवं सहायम्}
{स मे लब्धो दस्युवधाय कृष्णोमन्ये चेतद्विहितं दैवतेर्मे}


\twolineshloka
{अयुद्ध्यमानो मनसापि यस्यजयं कृष्णः पुरुषस्याभिनन्देत्}
{एवं सर्वान्स व्यतीयादमित्रान्सेन्द्रान्देवान्मानुषे नास्ति चिन्ता}


\twolineshloka
{स बाहुभ्यां सागरमुत्तितीर्षे-न्महोदधिं सलिलस्याप्रमेयम्}
{तेजस्विनं कृष्णमत्यन्तशूरंयुद्धेन यो वासुदेवं जिगीषेत्}


\twolineshloka
{गिरिं य इच्छेत्तु तलेन भेत्तुंशिलोच्चयं श्वेतमतिप्रमाणम्}
{तस्यैव पाणिः सनखो विशीर्ये-न्न चापि किञ्चित्स गिरेस्तु कुर्यात्}


\twolineshloka
{अग्निं समिद्धं शमयेद्भुजाभ्यांचन्द्रं च सूर्यं च निवारयेत}
{हरेद्देवानाममृतं प्रसह्ययुद्धेन यो वासुदेवं जिगीषेत्}


\twolineshloka
{यो रुक्मिणीमेकरथेन भोज-नुत्साद्य राज्ञः समरे प्रसह्य}
{उवाह भार्यां यशसा ज्वलन्तींयस्यां जज्ञे रौक्मिणेयो महात्मा}


% Check verse!
अयं गान्धारांस्तरसा संप्रमथ्यजित्वा पुत्रान्नग्नजितः समग्रान्बद्धं मुमोच विनदन्तं प्रसह्यसुदर्शनं वै देवतानां ललामम्
\twolineshloka
{अयं कवाटे निजघान पाण्ड्यंतथा कलिङ्गान्दन्तवक्रं ममर्द}
{अनेन दग्धा वर्षपूगान्विनाथावाराणसी नगरी संबभूव}


\twolineshloka
{अयं स्म युद्धे मन्यतेऽन्यैरजेयंतमेकलव्यं नाम निषादराजम्}
{वेगेनैव शैलमभिहत्य जम्भःशेते स कृष्णेन हतः परासुः}


\twolineshloka
{तथोग्रसेनस्य सुतं सुदुष्टंवृष्ण्यन्धकानां मध्यगतं सभास्थम्}
{अपातयद्बलदेवद्वितीयोहत्वा ददौ चोग्रसेनाय राज्यम्}


\twolineshloka
{अयं सौभं योधयामास स्वस्थंविभीषणं मायया साल्वराजम्}
{सौभद्वारि प्रत्यगृह्णाच्छतघ्नींदोर्भ्यां क एनं विषहेत मर्त्यः}


\twolineshloka
{प्राग्ज्योतिषं नाम बभूव दुर्गंपुरं घोरमसुराणामसह्यम्}
{महाबलो नरकस्तत्र भौमोजहारादित्या मणिकुण्डले शुभे}


\twolineshloka
{न तं देवाः सह शक्रेण शेकुःसमागता युधि मृत्योरभीताः}
{दृष्ट्वा च तं विक्रमं केशवस्यबलं तथैवास्त्रमवारणीयम्}


\twolineshloka
{जानन्तोऽस्य प्रकृतिं केशवस्यन्ययोजयन्दस्युवधाय कृष्णम्}
{स तत्कर्म प्रतिशुश्राव दुष्कर-मैश्वर्यवान्सिद्धिषु वासुदेवः}


\twolineshloka
{निर्मोचने षट्सहस्राणि हत्वासंच्छिद्य पाशान्सहसा क्षुरान्तान्}
{मुरं हत्वा विनिहत्यौघरक्षोनिर्मोचनं चापि जगाम वीरः}


\twolineshloka
{तत्रैव तेनास्य बभूव युद्धंमहाबलेनातिबलस्य विष्णोः}
{शेते स कृष्णेन हतः परासु-र्वातेनेव मथितः कर्णिकारः}


\twolineshloka
{आहृत्य कृष्णो मणिकुण्डले तेहत्वा च भौमं नरकं मुर च}
{श्रिया वृतो यशसा चैव विद्वा-न्प्रत्याजगामाप्रतिमप्रभावः}


\twolineshloka
{अस्मै वराण्यददंस्तत्र देवादृष्ट्वा भीमं कर्म कृतं रणे तत्}
{श्रमश्च ते युध्यमानस्य न स्या-दाकाशे चाप्सु च ते क्रमः स्यात्}


\twolineshloka
{शस्त्राणि गात्रे न च ते क्रमेर-न्नित्येव कृष्णश्च ततः कृतार्थः}
{एवं रूपे वासुदेवेऽप्रमेयेमहाबले गुणसंपत्सदैव}


\twolineshloka
{तमसह्यं विष्णुमनन्तवीर्य-माशंसते धार्तराष्ट्रो विजेतुम्}
{सदा ह्येनं तर्कयते दुरात्मातच्चाप्ययं सहतेऽस्मान्समीक्ष्य}


\twolineshloka
{पर्यागतं मम कृष्णस्य चैवयो मन्यते कलहं संप्रसह्य}
{शक्यं हर्तुं पाण्डवानां ममत्वंतद्वेदिता संयुगं तत्र गत्वा}


\twolineshloka
{नमस्कृत्वा शान्तनवाय राज्ञेद्रोणायाथो सहपुत्राय चैव}
{शारद्वतायाप्रतिद्वन्द्विने चयोत्स्याम्यहं राज्यमभीप्समानः}


\twolineshloka
{धर्मेणाप्तं निधनं तस्य मन्येयो योत्स्यते पाण्डवैर्धर्मचारी}
{मिथ्या ग्लहे निर्जिता वै नृशंसैःसंवत्सरान्वै द्वादश राजपुत्राः}


\twolineshloka
{वासः कृच्छ्रो विहितश्चाप्यरण्येदीर्घं कालं चैकमज्ञातवर्षम्}
{ते हि कस्माज्जीवतां पाण्डवानांनन्दिष्यन्ते धार्तराष्ट्राः पदस्थाः}


\twolineshloka
{ते चेदस्मान्युध्यमानाञ्जयेयु-र्देवैर्महेन्द्रप्रमुखैः सहायैः}
{धर्मादधर्मश्चरितो गरीयां-स्ततो ध्रुवं नास्ति कृतं च साधु}


\twolineshloka
{न चेदिदं पौरुषं कर्मबद्धंन चेदस्मान्मन्यतेऽसौ विशिष्टान्}
{आशंसेऽहं वासुदेवद्वितीयोदुर्योधनं सानुबन्धं निहन्तुम्}


\twolineshloka
{न चेदिदं कर्म नरेन्द्र वन्ध्यंन चेद्भवेत्सुकृतं निष्फलं वा}
{इदं च तच्चाभिसमीक्ष्य नूनंपराजयो धार्तराष्ट्रस्य साधुः}


\twolineshloka
{प्रत्यक्षं वः कुरवो यद्ब्रवीमियुद्ध्यमाना धार्तराष्ट्रा न सन्ति}
{अन्यत्र युद्धात्कुरवो यदि स्यु-र्न युद्धे वै शेष इहास्ति कश्चित्}


\twolineshloka
{हत्वा त्वहं धार्तराष्ट्रान्सकर्णा-न्राज्यं कुरूणामवजेता समग्रम्}
{यद्वः कार्यं तत्कुरुध्वं यथास्व-मिष्टान्दरानात्मभोगान्भजध्वम्}


\twolineshloka
{अप्येवं नो ब्राह्मणाः सन्ति वृद्धाबहुश्रुताः शीलवन्तः कुलीनाः}
{सांवत्सरा ज्योतिषि चाभियुक्तानक्षत्रयोगेषु च निश्चयज्ञाः}


\twolineshloka
{उच्चावचं दैवयुक्तं रहस्यंदिव्याः प्रश्ना मृगचक्रा मुहूर्ताः}
{क्षयं महान्तं कुरुसृञ़्जयानांनिवेदयन्ते पाण्डवानां जयं च}


\twolineshloka
{यथा हि नो मन्यतेऽजातशत्रुःसंसिद्धार्थो द्विपतां निग्रहाय}
{जनार्दनश्चाप्यपरोक्षविद्योन संशयं पश्यति वृष्णिसिंहः}


\twolineshloka
{अहं तथैनं खलु भाविरूपंपश्यामि बुद्धया स्वयमप्रमत्तः}
{दृष्टिश्च मे न व्यथते पुराणीसंयुध्यमाना धार्तराष्ट्रा न सन्ति}


\twolineshloka
{अनालब्धं जृम्भति गाण्डिवं धनु-रनाहता कम्पति मे धनुर्ज्या}
{बाणाश्च मे तूणमुखाद्विसृत्यमुहुर्मुहुर्गन्तुमुशन्ति चैव}


\twolineshloka
{खङ्गः कोशान्निःसरति प्रसन्नोहित्वेव जीर्णासुरगस्त्वचं स्वाम्}
{ध्वजे वाचो रौद्ररूपा भवन्तिकदा रथो योक्ष्यते ते किरीटिन्}


\twolineshloka
{गोमायुसङ्घाश्च नदन्ति रात्रौरक्षांस्यथो निष्पतन्त्यन्तरिक्षात्}
{मृगाः श्रृगालाः शितिकण्ठाश्च काकागृध्रा बकाश्चैव तरक्षवश्च}


\twolineshloka
{सुवर्णपत्राश्च पतन्ति पञ्चा-दृष्ट्वा रथं श्वेतहयप्रयुक्तम्}
{अहं ह्येकः पार्थिवान्सर्वयोधान्शरान्वर्पन्मृत्युलोकं नयेयम्}


\twolineshloka
{समादेदानः पृथगस्त्रमार्गा-न्यथाप्रिरिद्धो गहनं निदाघे}
{स्थूणाकर्णं पाशुपतं महास्त्रंब्राह्मं चास्त्रं यच्च शक्रोऽप्यदान्मे}


\twolineshloka
{वधे धृतो वेगवतः प्रमुञ्च-न्नाहं प्रजाः किञ्चिदिहावशिष्ये}
{शान्तिं लप्स्ये परमो ह्येप भावःस्थिरो मम ब्रूहि गावल्गणे तान्}


\twolineshloka
{ये वैजय्याः समरे सूत लब्ध्वादेवानपीन्द्रग्रमुखान्समेतात्}
{तैर्मन्मते कलहं संप्रसह्यस धार्तराष्ट्रः पश्यत मोहमस्य}


\twolineshloka
{वृद्धो भीष्मः शान्तनवः कृपश्चद्रोणः सपुत्रो विदुरश्च धीमान्}
{एते सर्वे यद्वदन्ते तदस्तुआयुष्मन्तः कुरवः सन्तु सर्वे}


\chapter{अध्यायः ४९}
\twolineshloka
{वैशंपायन उवाच}
{}


\twolineshloka
{समवेतेषु सर्वेषु तेषु राजसु भारत}
{दुर्योधनमिदं वाक्यं भीष्मः शान्तनवोऽब्रवीत्}


\twolineshloka
{बृहस्पतिश्चोशना च ब्रह्माणं पर्युपस्थितौ}
{मरुतश्च सहेन्द्रेण वसवश्चाग्रिना सह}


\twolineshloka
{आदित्याश्चैव साध्याश्च ये च सप्तर्पयो दिवि}
{विश्वावसुश्च गन्धर्वः शुभाश्चाप्सरसां गणाः}


\twolineshloka
{नमस्कृत्योपजग्मुस्ते लोकवृद्धं पितामहम्}
{परिवार्य च विश्वेशं पर्यासत दिवौकसः}


\twolineshloka
{तेषां मनश्च तेजश्चाप्याददानाविवौजसा}
{पूर्वदेवौ व्यतिक्रान्तौ नरनारायणावृषी}


\threelineshloka
{बृहस्पतिस्तु पप्रच्छ ब्रह्माणं काविमाविति}
{भवन्तं नोपतिष्ठेते तौ नः शंस पितामह ॥ब्रह्मोवाच}
{}


\twolineshloka
{यावेतौ पृथिवीं द्यां च भासयन्तौ तपस्विनौ}
{ज्वलन्तौ रोचमानौ च व्याप्यातीतौ महाबलौ}


\twolineshloka
{नरनारायणावेतौ लोकाल्लोकं समास्थितौ}
{ऊर्जितौ स्वेन तपसा महासत्वपराक्रमौ}


\fourlineindentedshloka
{एतौ हि कर्मणा लोकं नन्दयामासतुर्ध्रुवम्}
{द्विधाभूतौ महाप्राज्ञौ विद्धि ब्रह्मन्परन्तपौ}
{असुराणां विनाशाय देवगन्धर्वपूजितौ ॥वैशंपायन उवाच}
{}


\twolineshloka
{जगाम शक्रस्तच्छ्रत्वा यत्र तौ तेपतुस्तपः}
{सार्धं देवगणैः सर्वैर्बृहस्पतिपुरोगमैः}


\twolineshloka
{तदा देवासुरे युद्धे भये जाते दिवौकसाम्}
{अयाचत महात्मानौ नरनारायणौ वरम्}


\twolineshloka
{तावब्रूतां वृणीष्वेति तदा भरतसत्तम}
{अथैतावब्रवीच्छकः साह्यं नः क्रियतामिति}


\twolineshloka
{ततस्तौ शक्रमब्रूतां करिष्यावो यदिच्छसि}
{ताभ्यां च सहितः शक्रो विजिग्ये दैत्यदानवान्}


\twolineshloka
{नर इन्द्रस्य संग्रामे हत्वा शत्रून्परन्तपः}
{पौलोमान्कालखञ्जांश्च सहस्राणि शतानि च}


\twolineshloka
{एष भ्रान्ते रथे तिष्ठन्भल्लेनापाहरच्छिरः}
{जम्भस्य ग्रसमानस्य तदा ह्यर्जुन आहवे}


\twolineshloka
{एष पारे समुद्रस्य हिरण्यपुरमारुजत्}
{हत्वा षष्टिं सहस्राणि निवातकवचान्रणे}


\twolineshloka
{एष देवान्सहेन्द्रेण जित्वा परपुरञ्जयः}
{अतर्पयन्महाबाहुरर्जुनो जातवेदसम्}


\twolineshloka
{नारायणस्तथैवात्र भूयसोऽन्याञ्जघान ह}
{एवमेतौ महावीर्यौ तौ पश्यत समागतौ}


\twolineshloka
{वासुदेवार्जुनौ वीरौ समवेतौ महारथौ}
{नरनारायणौ देवौ पूर्वदेवाविति श्रुतिः}


\threelineshloka
{अजेयौ मानुषे लोके सेन्द्रैरपि सुरासुरैः}
{एष नारायणः कृष्णः फाल्गुनश्च नरः स्मृतः}
{नारायणो नरश्चैव सत्त्वमेकं द्विधा कृतम्}


\twolineshloka
{एतौ हि कर्मणा लोकानश्रुवातेऽक्षयान्ध्रुवान्}
{तत्रतत्रैव जायेते युद्धकाले पुनःपुनः}


\twolineshloka
{तस्मात्कर्मैव कर्तव्यमिति होवाच नारदः}
{एतद्धि सर्वमाचष्ट वृष्णिचक्रस्य वेदवित्}


\twolineshloka
{शङ्खचक्रगदाहस्तं यदा द्रक्ष्यसि केशवम्}
{पर्याददानं चास्त्राणि भीमधन्वानमर्जुनम्}


\twolineshloka
{सनातनौ महात्मानौ कृष्णावेकरथे स्थितौ}
{दुर्योधन तदा तात स्मर्तासि वचनं मम}


\twolineshloka
{नोचेदयमभावः स्यात्कुरूणां प्रत्युपस्थितः}
{अर्थाच्च तात धर्माच्च तव बुद्धिरुपप्लुता}


\twolineshloka
{न चेद्ग्रहीष्यसे वाक्यं श्रोतामि सुबहून्हतान्}
{तवैव हि मतं सर्वे कुरवः पर्युपासते}


\twolineshloka
{त्रयाणामेव च मतं तत्त्वमेकोऽनुमन्यसे}
{रामेण चैव शप्तस्य कर्णस्य भरतर्षभ}


\threelineshloka
{दुर्जातेः सुतपुत्रस्य शकुनेः सौबलस्य च}
{तथा क्षुद्रस्य पापस्य भ्रातुर्दुःशासनस्य च ॥कर्ण उवाच}
{}


\twolineshloka
{नैवमायुष्मता वाच्यं यन्मामात्थ पितामह}
{क्षत्रधर्मे स्थितो ह्यस्मि स्वधर्मादनपेयिवान्}


\twolineshloka
{किञ्चान्यन्मयि दुर्वृत्तं येन मां परिगर्हसे}
{न हि मे वृजिनं किञ्चिद्धार्तराष्ट्रा विदुः क्वचित्}


\twolineshloka
{नाचरं वृजिनं किञ्चिद्धार्तराष्ट्रस्य नित्यशः}
{अहं हि पाण्डवान्सर्वान्हनिष्यामि रणे स्थितान्}


\fourlineindentedshloka
{प्राग्विरुद्धैः शमं सद्भिः कथं वा क्रियते पुनः}
{राज्ञो हि धृतराष्ट्रस्य सर्वं कार्यं प्रियं मया}
{तथा दुर्योधनस्यापि स हि राज्ये समाहितः ॥वैशंपायन उवाच}
{}


\twolineshloka
{कर्णस्य तु वचः श्रुत्वा भीष्मः शान्तनवः पुनः}
{धृतराष्ट्रं महाराजमाभाष्येदं वचोऽब्रवीत्}


\twolineshloka
{यदयं कत्थते नित्यं हन्ताहं पाण्डवानिति}
{नायं कलापि संपूर्णा पाण्डवानां महात्मनाम्}


\twolineshloka
{अनयो योयमागन्ता पुत्राणां ते दुरात्मनाम्}
{तदस्य कर्म जानीहि सूतपुत्रस्य दुर्मतेः}


\twolineshloka
{एतमाश्रित्य पुत्रस्ते मन्दबुद्धिः सुयोधनः}
{अवामन्यत तान्वीरान्देवपुत्रानरिन्दमान्}


\twolineshloka
{किञ्चाप्येतेन तत्कर्म कृतपूर्वं सुदुष्करम्}
{तैर्यथा पाण्डवैः सर्वैरेकैकेन कृतं पुरा}


\twolineshloka
{दृष्ट्वा विराटनगरे भ्रातरं निहतं प्रियम्}
{धनञ्जयेन विक्रम्य किमनेन तदा कृतम्}


\twolineshloka
{` सर्वे ह्यस्त्रिविदः शूराः सर्वे प्राप्ता महद्यशः}
{अपि सर्वामरैश्वर्यं त्यजेयुर्न पुनर्जयम् ॥'}


\twolineshloka
{सहितान्हि कुरून्सर्वानभियातो धनञ्जयः}
{प्रमथ्य चाच्छिनद्वासः किमयं प्रोषितस्तदा}


\twolineshloka
{गन्धर्वैर्घोषयात्रायां ह्रियते यत्सुतस्तव}
{क्व तदा सूतपुत्रोऽभूद्य इदानीं वृषायते}


\twolineshloka
{ननु तत्रापि भीमेन पार्थेन च महात्मना}
{यमाभ्यामेव संगम्य गन्धर्वास्ते पराजिताः}


\threelineshloka
{एतान्यस्य मृषोक्तानि बहूनि भरतर्षभ}
{विकत्थनस्य भद्रं ते सदा धर्मार्थलोपिनः ॥वैशंपायन उवाच}
{}


\twolineshloka
{भीष्मस्य तु वचः श्रुत्वा भारद्वाजो महामताः}
{धृतराष्ट्रमुवाचेदं राजमध्येऽभिपूजयन्}


\twolineshloka
{यदाह भरतश्रेष्ठो भीष्मस्तत्क्रियतां नृप}
{न काममवलिप्तानां वचनं कर्तुमर्हसि}


\twolineshloka
{पुरा युद्धात्साधु मन्ये पाण्डवैः सह सङ्गतम्}
{यद्वाक्यमर्जुनेनोक्तं सञ्जयेन निवेदितम्}


\threelineshloka
{सर्वं तदभिजानामि करिष्यति च पाण्डवः}
{न ह्यस्य त्रिषु लोकेषु सदृशोऽस्ति धनुर्धरः ॥वैशंपायन उवाच}
{}


\twolineshloka
{अनादृत्य तु तद्वाक्यमर्थवद्द्रोणभीष्मयोः}
{ततः स सञ्जयं राजा पर्यपृच्छत पाण्डवान्}


\twolineshloka
{तदैव कुरवः सर्वे निराशा जीवितेऽभवन्}
{भीष्मद्रोणौ यदा राजा न सम्यगनुभाषते}


\chapter{अध्यायः ५०}
\twolineshloka
{धृतराष्ट्र उवाच}
{}


\twolineshloka
{किमसौ पाण्डवो राजा धर्मपुत्रोऽभ्यभाषत}
{श्रुत्वेह बहुलाः सेनाः प्रीत्यर्थं नः समागताः}


\twolineshloka
{किमसौ चेष्टते सूत योत्स्यमानो युधिष्ठिरः}
{के वास्य भ्रातृपुत्राणां पश्यन्त्याज्ञेप्सवो सुखम्}


\threelineshloka
{के स्विदेनं वारयन्ति युद्धाच्छाम्येति वा पुनः}
{निकृत्या कोपितं मन्दैर्धर्मज्ञं धर्मचारिणम् ॥सञ्जय उवाच}
{}


\twolineshloka
{राज्ञो मुखमुदीक्षन्ते पाञ्चालाः पाण्डवैः सह}
{युधिष्ठिरस्य भद्रं ते स सर्वाननुशास्ति च}


\twolineshloka
{पृथग्भूताः पाण्डवानां पाञ्चालानां रथव्रजाः}
{आयान्तमभिनन्दन्ति कुन्तीपुत्रं युधिष्ठिरम्}


\twolineshloka
{नभः सूर्यमिवोद्यन्तं कौन्तेयं दीप्ततेजसम्}
{पाञ्चालाः प्रतिनन्दन्ति तेजोराशिमिवोदितम्}


\twolineshloka
{आगोपालाविपालाश्च नन्दमाना युधिष्ठिरम्}
{पाञ्चालाः केकया मत्स्याः प्रतिनन्दति पाण्डवम्}


\threelineshloka
{ब्राह्मण्यो राजपुत्र्यश्च विशां दुहितरश्च याः}
{क्रीडन्त्योभिसमायान्ति पार्थं सन्नद्धमीक्षितुम् ॥धृतराष्ट्र उवाच}
{}


\threelineshloka
{सञ्जयाचक्ष्व येनास्मान्पाडवा अभ्ययुञ्जत}
{धृष्टद्युम्नस्य सैन्येन सोमकानां बलेन च ॥वैशंपायन उवाच}
{}


\twolineshloka
{गवल्गणिस्तु तत्पुष्टः सभायां कुरुसंसदि}
{निःश्वस्य सुभृशं दीर्घं मुहुः सञ्चिन्तयन्निव}


\twolineshloka
{तत्रानिमित्ततो वैवात्सुतं कश्मलमाविशत्}
{तदाचचक्षे विदुरः सभायां राजसंसदि}


\threelineshloka
{सञ्जयोऽयं महाराज मूर्छितः पतितो भुवि}
{वाचं न सृजते कांचिद्धीनप्रज्ञोऽल्पचेतनः ॥धृतराष्ट्र उवाच}
{}


\threelineshloka
{अपश्यत्सञ्जयो नूनं कुन्तीपुत्रान्महारथान्}
{तैरस्य पुरुषव्याघ्रैर्भृशमुद्वेजितं मनः ॥वैशंपायन उवाच}
{}


\threelineshloka
{सञ्जयश्चेतनां लब्ध्वा प्रत्याश्वस्येदभब्रवीत्}
{धृतराष्ट्रं महाराज सभायां कुरुसंसदि ॥सञ्जय उवाच}
{}


\twolineshloka
{दृष्टवानस्मि राजेन्द्र कुन्तीपुत्रान्महारथान्}
{मत्स्यराजगृहावासनिरोधेनावकर्शितान्}


\twolineshloka
{श्रृणु र्यैर्हि महाराज पाण्डवा अभ्ययुञ्जत}
{धृष्टद्युम्नेन वीरेण युद्धे वस्तेऽभ्ययुञ्जत}


\twolineshloka
{यो नैव रोपान्न भयान्न लोभान्नार्थकारणात्}
{न हेतुवादाद्धर्मात्मा सत्यं जाह्यात्कदाचन}


\twolineshloka
{यः प्रमाणं महाराज धर्मे धर्मभृतां वरः}
{अजातशत्रुणा तेन पाण्डवा अभ्ययुञ्जत}


\threelineshloka
{यस्य बाहुबले तुल्यः पृथिव्यां नास्ति कश्चन}
{यो वै सर्वान्महीपालान्वशे चक्रे धनुर्धरः}
{यः काशीनङ्गमगधान्कलिङ्गांश्च युधाजयत्}


\twolineshloka
{तेन वो भीमसेनेन पाण्डवा अभ्ययुञ्जत}
{यस्य वीर्येण सहसा चत्वारो भुवि पाण्डवाः}


\twolineshloka
{निःसृत्य जतुगेहाद्वै हिडिम्बात्पुरुषादकात्}
{यश्चैपामभवद्द्वीपः कुन्तीपुत्रो वृकोदरः}


\twolineshloka
{याज्ञसेनीमथो यत्र सिन्धुराजोपकृष्टवान्}
{तत्रैषामभवद्द्वीपः कुन्तीपुत्रो वृकोदरः}


\twolineshloka
{यश्च तान्सङ्गतान्सर्वान्पाण्डवान्वारणावते}
{दह्यतो मोचयामास तेन वस्तेऽभ्ययुञ्जत}


\twolineshloka
{कृष्णायां चरता प्रीतिं येन क्रोधवशा हताः}
{प्रविश्य विषयं घोरं पर्वतं गन्धमादनम्}


\twolineshloka
{यस्य नागायुतैर्वीर्यं भुजयोः सारमर्पितम्}
{तेन वो भीमसेनेन पाण्डवा अभ्ययुञ्जत}


\twolineshloka
{कृष्णद्वितीयो विक्रम्य तुष्ट्यर्थं जातवेदसः}
{अजयद्यः पुरा वीरो युध्यमानं पुरन्दरम्}


\twolineshloka
{यः स साक्षान्महादेवं गिरिशं शृलपाणिनम्}
{तोपयामास युद्धेन देवदेवमुपामपिम्}


\twolineshloka
{यश्च सर्वान्वशे चक्रे लोकपालान्धनुर्धरः}
{तेन वो विजयेनाजौ पाण्डवा अभ्ययुञ्जत}


\twolineshloka
{यः प्रतीचीं दिशं चक्रे वशे म्लेच्छगणायुताम्}
{स तत्र नकुलो योद्धा चित्रयोधी व्यवस्थितः}


\twolineshloka
{तेन वो दर्शनीयेन वीरेणातिधनुर्भृता}
{माद्रीपुत्रेण कौरव्य पाण्डवा अभ्ययुञ्जत}


\twolineshloka
{यः काशीनङ्गमगधान्कलिङ्गांश्च युधाजयत्}
{तेन वः सहदेवेन पाण्डवा अभ्ययुञ्जत}


\twolineshloka
{यस्य वीर्येण सदृशाश्चत्वारो भुवि मानवाः}
{अश्वत्थामा धृष्टकेतू रुक्मी प्रद्युम्न एव च}


\twolineshloka
{तेन वः सहदेवेन युद्धं राजन्महात्ययमअ}
{यवीयसा नृवीरेण माद्रीनन्दिकरेण च}


\twolineshloka
{तपश्चचार या घोरं काशिकन्या पुरा सती}
{भीष्मस्य वधमिच्छन्ती प्रेत्यापि भरतर्षभ}


\twolineshloka
{पाञ्चालस्य सुता जज्ञे दैवाच्च स पुनः पुमान्}
{स्त्रीपुंसीः पुरुषव्याघ्र यः स वेद गुणागुणान्}


\twolineshloka
{.. कलिङ्गान्समापेदे पाञ्चाल्यो युद्धदुर्मदः}
{शिखण्डिना वः कुरवः कृतास्त्रेणाभ्ययुञ्जत}


\twolineshloka
{यं यक्षः पुरुषं चक्रे भीष्मस्य निधनेच्छया}
{महेष्वासेन रौद्रेण पाण्डवा अभ्ययुञ्जत}


\twolineshloka
{महेष्वासा राजपुत्रा भ्रातरः पञ्च केकयाः}
{आमुक्तवचाः शूरास्तैश्च वस्तेऽभ्ययुञ्जत}


\twolineshloka
{यो दीर्घबाहुः क्षिप्रास्त्रो धृतिमान्सत्यविक्रमः}
{तेन वो वृष्णिवीरेण युयुधानेन सङ्गरः}


\twolineshloka
{य आसीच्छरणं काले पाण्डवानां महात्मनाम्}
{रणे तेन विराटेन भविता वः समागमः}


\twolineshloka
{यः स काशिपती राजा वाराणस्यां महारथः}
{स तेषामभवद्योद्धा तेन वस्तेऽभ्ययुञ्जत}


\twolineshloka
{शिशुभिर्दुर्जयैः सङ्ख्ये द्रौपदेयैर्महात्मभिः}
{आशीविषसमस्पर्शैः पाण्डवा अभ्ययञ्जत}


\twolineshloka
{यः कृष्णासदृशो वीर्ये युधिष्ठिरसमो दमे}
{तेनाभिमन्युना सङ्ख्ये पाण्डवा अभ्ययुञ्जत}


\twolineshloka
{यश्चैवाप्रतिमो वीर्ये धृष्टकेतुर्महायशाः}
{दुःसहः समरे क्रुद्धः शैशुपालिर्महारथः}


\twolineshloka
{तेन वश्चेदिराजेन पाण्डवा अभ्ययुञ्जत}
{अक्षौहिण्या परिवृतः पाण्डवान्योभिसंश्रितः}


\twolineshloka
{यः संश्रयः पाण्डवानां देवानामिव वासवः}
{तेन वो वासुदेवेन पाण्डवा अभ्ययुञ्जत}


\twolineshloka
{यथा चेदिपतेर्भ्राता शरभो भरतर्षभ}
{करकर्षेण सहितस्ताभ्यां वस्तेऽभ्ययुञ्जत}


\twolineshloka
{तारासन्धिः सहदेवो जयत्सेनश्च तावुभौ}
{युद्धे प्रतिरथे वीरौ पाण्डवार्थे व्यवस्थितौ}


\twolineshloka
{द्रुपदश्च महातेजा बलेन महता वृतः}
{त्यक्तात्मा पाण्डवार्थाय सोत्स्यमानो व्यवस्थितः}


\twolineshloka
{एते चान्ये च बहवः प्राच्योदीच्या महीक्षितः}
{शतशो यानुपाश्रिता धर्मराजो व्यवस्थितः}


\chapter{अध्यायः ५१}
\twolineshloka
{धृतराष्ट्र उवाच}
{}


\twolineshloka
{सर्व एते महोत्साहा ये त्वया परिकीर्तिताः}
{एकतस्त्वेव ते सर्वे समेता भीम एकतः}


\twolineshloka
{भीमसेनाद्धि मे भूयो भयं सञ्जायते महत्}
{क्रुद्धादमर्षणात्तात व्याघ्रादिव महारुरोः}


\twolineshloka
{जागर्मि रात्रयः सर्वा दीर्घमुष्णं च निःश्वसन्}
{भीतो वृकोदरात्तात सिंहात्पशुरिवापरः}


\twolineshloka
{न हि तस्य महाबाहोः शक्रप्रतिमतेजसः}
{सैन्येऽस्मिन्प्रतिपश्यामि य एनं विषहेद्युधि}


\twolineshloka
{अमर्षणश्च कौन्तेयो दृढवैरश्च पाण्डवः}
{अनर्महासी सोन्मादस्तिर्यक्प्रेक्षी महास्वनः}


\twolineshloka
{महावेगो महोत्साहो महाबाहुर्महाबलः}
{मन्दानां मम पुत्राणां युद्धेनान्तं करिष्यति}


\twolineshloka
{उरुग्राहगृहीतानां गदां बिभ्रद्वृकोदरः}
{कुरूणामृषभो युद्धे दण्डपाणिरिवान्तकः}


\twolineshloka
{अष्टाश्रिमायसीं घोरां गदां काञ्चनभूषणाम्}
{मनसाहं प्रपश्यामि ब्रह्मदण्डमिवोद्यतम्}


\twolineshloka
{यथा मृगाणां यूथेषु सिंहो जातबलश्चरेत्}
{मामकेषु तथा भीमो बलेषु विचरिष्यति}


\twolineshloka
{सर्वेषां मम पुत्राणां स एकः क्रूरविक्रमः}
{बह्वाशी विप्रतीपश्च बाल्येपि रभसः सदा}


\twolineshloka
{उद्वेपते मे हृदयं ये मे दुर्योधनादयः}
{बाल्येऽपि तेन युद्ध्यन्तो वारणेनेव मर्दिताः}


\twolineshloka
{तस्य वीर्येण संक्लिष्टा नित्यमेव सुता मम}
{स एव हेतुर्भेदस्य भीमो भीमपराक्रमः}


\twolineshloka
{ग्रसमानमनीकानि नरवारणवाजिनाम्}
{पश्यामीवाग्रतो भीमं क्रोधमूर्छितमाहवे}


\twolineshloka
{अस्त्रे द्रोणार्जुनसमं वायुवेगसमं जवे}
{महेश्वरसमं क्रोधे को हन्याद्भीममाहवे}


\threelineshloka
{सञ्जयाचक्ष्व मे शूरं भीमसेनममर्षणम्}
{5-51-15b.......... येऽहंयत्तेन रिपुघातिना}
{..... सर्वे पुत्राः मम मनस्विना}


\twolineshloka
{येन भीमबला यक्षा राक्षसाश्च पुरा हताः}
{कथं तस्य रणे वेगं मानुषः प्रसहिष्यति}


\twolineshloka
{न स जातु वशे तस्थौ मम बाल्येऽपि सञ्जय}
{किं पुनर्मम दुष्पुत्रैः क्लिष्टः संप्रति पाण्डवः}


\twolineshloka
{निष्ठुरो रोषणोऽत्यर्थं भज्येतापि न संनमेत्}
{तिर्यक्प्रेक्षीं संहतभ्रूः कथं शाम्येद्वृकोदरः}


\twolineshloka
{शूरस्तथाऽप्रतिबलो गौरस्ताल इवोन्नतः}
{प्रमाणतो भीमसेनः प्रादेशेनाधिकोऽर्जुनात्}


\twolineshloka
{जवेन वाजिनोऽत्येति बलेनान्त्येति कुञ्जरान्}
{अव्यक्तजल्पी मध्वक्षो मध्यमः पाण्डवो बली}


\twolineshloka
{इति बाल्ये श्रुतः पूर्वं मया व्यासमुखात्पुरा}
{रूपतो वीर्यतश्चैव याथातथ्येन पाण्डवः}


\twolineshloka
{आयसेन स दण्डेन रथन्नागान्नरान्हयान्}
{हनिष्यति रणे क्रुद्धो रौद्रः क्रूरपराक्रमः}


\twolineshloka
{अमर्षी नित्यसंरब्धो भीमः प्रहरतां वरः}
{मया तात प्रतीपानि कुर्वन्पूर्वं विमानितः}


\twolineshloka
{निष्कर्णामायसीं स्थूलां सुपार्श्वां काञ्चनीं गदाम्}
{शतघ्नीं शतनिर्हादां कथं शक्ष्यन्ति मे सुताः}


\twolineshloka
{अपारमप्लवागाधं समुद्रं शरवेधनम्}
{भीमसेनमयं दुर्गं तात मन्दास्तितीर्षवः}


\twolineshloka
{क्रोशतो मे न श्रृण्वन्ति बालाः पण्डितमानिनः}
{विषमं न हि मन्यन्ते प्रापतं मधुदर्शिनः}


\twolineshloka
{संयुगे ये गमिष्यन्ति नररूपेण मृत्युना}
{नियतं चोदिता धात्रा सिंहेनेव महामृगाः}


\twolineshloka
{शैक्यां तात चतुष्किष्कुं षडश्रिममितौजसम्}
{प्रहितां दुःखसंस्पर्शां कथं शक्ष्यन्ति मे सुताः}


\twolineshloka
{गदां भ्रामयतस्तस्य भिन्दतो हस्तिमस्तकान्}
{सृक्किणी लेलिहानस्य बाष्पमुत्सृजतो मुहुः}


\twolineshloka
{उद्दिश्य नागान्पततः कुर्वतो भैरवान्रवान्}
{प्रतीपं पततो मत्तान्कुञ्जरान्प्रति गर्जतः}


\twolineshloka
{विगाह्य रथमार्गेषु वरानुद्दिश्य निघ्नतः}
{अग्नेः प्रज्वलितस्येव अपि मुच्येत मे प्रजा}


\twolineshloka
{वीथीं कुर्वन्महाबाहुर्द्रावयन्मम वाहिनीम्}
{नृत्यन्निव गदापाणिर्युगान्तं दर्शयिष्यति}


\twolineshloka
{प्रभिन्न इव मातङ्गः प्रभञ्जन्पुष्पितान्द्रुमान्}
{प्रवेक्ष्यति रणे सेनां पुत्राणां मे वृकोदरः}


\twolineshloka
{कुर्वन्रथान्विपुरुषान्विसारथिहयध्वजान्}
{आरुजन्पुरुषव्याघ्रो रथिनः सादिनस्तथा}


\twolineshloka
{गङ्गावेग इवानूपांस्तीरजान्विविधान्द्रुमान्}
{प्रभङ्क्ष्यति रणे सेनां पुत्राणां मम सञ्जय}


\twolineshloka
{वधं नूनं गमिष्यन्ति भीमसेनभयार्दिताः}
{मम पुत्राश्च भृत्याश्च राजानश्चैव सञ्जय}


\twolineshloka
{येन राजा महावीर्यः प्रविश्यान्तःपुरं पुरा}
{वासुदेवसहायेन जरासन्धो निपातितः}


\twolineshloka
{कृत्स्नेयं पृथिवी देवी जरासन्धेन धीमता}
{मागधेन्द्रेण बलिना वशे कृत्वा प्रतापिता}


\twolineshloka
{भीष्मप्रतापात्कुरवो नयेनान्धकवृष्णयः}
{यन्न तस्य वशे जग्मुः केवलं दैवमेव तत्}


\twolineshloka
{स गत्वा पाण्डुपुत्रेण तरसा बाहुशालिना}
{अनायुधेन वीरेण निहतः किं ततोऽधिकम्}


\twolineshloka
{दीर्घकालसमासक्तं विशमाशीविषो यथा}
{स मोक्ष्यति रणे तेजः पुत्रेषु मम सञ्जय}


\twolineshloka
{महेन्द्र इव वज्रेण दानवान्देवसत्तमः}
{भीमसेनो गदापाणिः सूदयिष्यति मे सुतान्}


\twolineshloka
{अविषह्यमनावार्यं तीव्रवेगपराक्रमम्}
{पश्यामीवातिताम्राक्षमापतन्तं वृकोदरम्}


\twolineshloka
{अगदस्याप्यधनुषो विरथस्य विवर्मणः}
{बाहुभ्यां युध्यमानस्य कस्तिष्ठेदग्रतः पुमान्}


\twolineshloka
{भीष्मो द्रोणश्च विप्रोऽयं कृपः शारद्वतस्तथा}
{जान्त्येते यथैवहं वीर्यज्ञस्तस्य धीमतः}


\twolineshloka
{आर्यव्रतं तु जानन्तः सङ्गरान्तं विधित्सवः}
{सेनामुखेषु स्थास्यन्ति मामकानां नरर्षभाः}


\twolineshloka
{बलीयः सर्वतो दिष्टं पुरुषस्य विशेषतः}
{पश्यन्नपि जयं तेषां न नियच्छामि यत्सुतान्}


\twolineshloka
{ते पुराणं महेष्वासा मार्गमैन्द्रं समास्थिताः}
{त्यक्ष्यन्ति तुमुले प्राणान्रक्षन्तः पार्थिवं यशः}


\twolineshloka
{यथैषां मामकास्तात तथैषां पाण्डवा अपि}
{पौत्रा भीष्मस्य शिष्याश्च द्रोणस्य च कृपस्य च}


\twolineshloka
{ये त्वस्मदाश्रयं किञ्चिद्दत्तमिष्टं च सञ्जय}
{तस्यापचितिमार्यत्वात्कर्तारः स्थविरास्त्रयः}


\twolineshloka
{आददानस्य शस्त्रं हि क्षत्रधर्मं परीप्सतः}
{निधनं क्षत्रियस्याजौ वरमेवाद्दुरुत्तमम्}


\twolineshloka
{स वै शोचामि सर्वान्वै ये युयुत्सन्ति पाण्डवैः}
{विक्रुष्टं विदुरोणादौ तदेतद्भयमागतम्}


\twolineshloka
{न तु मन्ये विघाताय ज्ञानं दुःखस्य सञ्जय}
{भवत्यतिबलं ह्येतज्ज्ञानस्याप्युपघातकम्}


\twolineshloka
{ऋषयो ह्यपि निर्मुक्ताः पश्यन्तो लोकसङ्ग्रहान्}
{सुखैर्भवन्ति सुखिनस्तथा दुःखेन दुःखिताः}


\twolineshloka
{किं पुनर्मोहमासक्तस्तत्र तत्र सहस्रधा}
{पुत्रेषु राज्यदारेषु पौत्रेष्वपि च बन्धुषु}


\twolineshloka
{संशये तु महत्यस्मिन्किं नु मे क्षममुत्तरम्}
{विनाशं ह्येव पश्यामि कुरूणामनुचिन्ततयन्}


\twolineshloka
{द्यूतप्रमुखमाभाति कुरूणां व्यसनं महत्}
{मन्देनैश्वर्यकामेन लोभात्पापमिदं कृतम्}


\twolineshloka
{मन्ये पर्यायधर्मोऽयं कालस्यात्यन्तगामिनः}
{चक्रे प्रधिरिवासक्तो नास्य शक्यं पलायितुम्}


\twolineshloka
{किं नु कुर्यां कथं कुर्यां क्व नु गच्छामि सञ्जय}
{एते नश्यन्ति कुरवो मन्दाः कालवशं गताः}


\twolineshloka
{अवशोऽहं तदा तात पुत्राणां निहते शते}
{श्रोष्यामि निनदं स्त्रीणां कथं मां मरणं स्पृशेत्}


\twolineshloka
{यथा निदाघे ज्वलनः समद्धोदहेत्कक्षं वायुना चोद्यमानः}
{गदाहस्तः पाण्डवो वै तथैवहन्ता मदीयान्सहितोऽर्जुनेन}


\chapter{अध्यायः ५२}
\twolineshloka
{धृतराष्ट्र उवाच}
{}


\twolineshloka
{यस्य वै नानृता वाचः कदाचिदनुशुश्रुम}
{त्रैलोक्यमपि तस्य स्याद्योद्धा यस्य धनञ्जयः}


\twolineshloka
{तस्यैव च न पश्यामि युधि गाण्डीवधन्वनः}
{अनिशं चिन्तयानोऽपि यः प्रतीयाद्रथेन तम्}


\twolineshloka
{अस्यतः कर्णिनालीकान्मार्गणान्हृदयच्छिदः}
{प्रत्येता न समः कश्चिद्युधि गाण्डीवधन्वनः}


\twolineshloka
{द्रोणाकर्णौ प्रतीयातां यदि वीरौ नरर्षभौ}
{कृतास्त्रौ बलिनां श्रेष्ठौ समरेष्वपराजितौ}


\twolineshloka
{महान्स्यात्संशयो लोके न त्वस्ति विजयो मम}
{घृणी कर्णः प्रमादी च आचार्यः स्थविरो गुरुः}


\twolineshloka
{समर्थो बलवान्पार्थो दृढधन्वा जितक्लमः}
{भवेत्सुतुमुलं युद्धं सर्वशोऽप्यपराजयः}


\twolineshloka
{सर्वे ह्यस्त्रविदः शूराः सर्वे प्राप्ता महद्यशः}
{अपि सर्वामरैश्वर्यं त्यजेयुर्न पुनर्जयम्}


\twolineshloka
{वधे नूनं भवेच्छान्तिस्तयोर्वा फाल्गुनस्य च}
{न तु हन्तार्जुनस्यास्ति जेता चास्य न विद्यते}


\threelineshloka
{मन्युस्तस्य कथं शाम्येन्मन्दान्प्रति य उत्थितः}
{अन्येऽप्यस्त्राणि जानन्ति जीयन्ते च जयन्ति च}
{एकान्तविजयस्त्वेव श्रूयते फाल्गुनस्य ह}


\twolineshloka
{त्रयस्त्रिंशत्समाहूय खाण्डवेऽग्निमतर्पयत्}
{जिगाय च सुरान्सर्वान्नास्य विद्मः पराजयम्}


\twolineshloka
{यस्य यन्ता हृषीकेशः शीलवृत्तसमो युधि}
{ध्रुवस्तस्य जयस्तात यथेन्द्रस्य जयस्तथा}


\twolineshloka
{कृष्णावेकरथे यत्तावधिज्यं गाण्डिवं धनुः}
{युगपत्रीणि तेजांसि समेतान्यनुशुश्रुम}


\threelineshloka
{नैवास्ति नो धनुस्तदृङ्व योद्धा न च सारथिः}
{तच्च मन्दा न जानन्ति दुर्योधनवशानुगाः}
{}


\twolineshloka
{शेषयेदशनिर्दीप्तो विपतन्मूर्ध्नि सञ्जय}
{न तु शेषं शरास्तात कुर्युरस्ताः किरीटिना}


\twolineshloka
{अपि चास्यन्निवाभाति निघ्नन्निव धनञ्जयः}
{उद्धरन्निव कायेभ्यः शिरांसि शरवृष्टिभिः}


\twolineshloka
{अपि बाणमयं तेजः प्रदीप्तमिव सर्वतः}
{गाण्डीवोत्थं दहेताजौ पुत्राणां मम वाहिनी}


\twolineshloka
{अपि सा रथघोषेण भयार्ता सव्यसाचिनः}
{वित्रस्ता बहुधा सेना भारती प्रतिभाति मे}


\twolineshloka
{यथा कक्षं महानग्निः प्रदहेत्सर्वतश्चरन्}
{महार्चिरनिलोद्धूतस्तद्वद्धक्ष्यति मामकान्}


\twolineshloka
{यदोद्वमन्निशितान्बाणसङ्घां-स्तानाततायी समरे किरीटी}
{सृष्टोऽन्तकः सर्वहरो विधात्रायथा भवेत्तद्वदपारणीयः}


\twolineshloka
{यदा हयभीक्ष्णं सुबुहून्प्रकारान्श्रोतास्मि तानावसथे कुरूणाम्}
{तेषां समन्ताच्च तथा रणाग्रेक्षयः किलायं भरतानुपैति}


\chapter{अध्यायः ५३}
\twolineshloka
{धृतराष्ट्र उवाच}
{}


\twolineshloka
{यथैव पाण्डवाः सर्वे पराक्रान्ता जिगीषवः}
{तथौभिसरास्तेषां त्यक्तात्मानो जये धृताः}


\twolineshloka
{......हि पराक्रान्तानाचक्षीथाः परान्मम}
{पाञ्चालान्केकयान्मत्स्यान्मागधान्वत्सभूमिपान्}


\twolineshloka
{यश्च सेन्द्रानिमाँल्लोकानिच्छन्कुर्याद्वशे बली}
{स स्रष्टा जगतः कृष्णः पाण्डवानां जये धृतः}


\twolineshloka
{समस्तामर्जुनाद्विद्यां सात्यकिः क्षिप्रमाप्तवान्}
{शैनेयः समरे स्थाता बीजवत्प्रवपञ्शरान्}


\twolineshloka
{धृष्टद्युम्नश्च पाञ्चाल्यः क्रूरकर्मा महारथः}
{मामकेषु रणं कर्ता बलेषु परमास्त्रवित्}


\twolineshloka
{युधिष्ठिरस्य च क्रोधादर्जुनस्य च विक्रमात्}
{मयाभ्यां भीमसेनाच्च भयं मे तात जायते}


\twolineshloka
{अमानुषं मनुष्येन्द्रैर्जालं विततमन्तरा}
{न मे सैन्यास्तरिष्यन्ति ततः क्रोशामि सञ्जय}


\twolineshloka
{दर्शनीयो मनस्वी च लक्ष्मीवान्ब्रह्मवर्चसी}
{मेधावी सुकृतप्रज्ञो धर्मात्मा पाण्डुनन्दनः}


\twolineshloka
{मित्रामात्यैः सुसंपन्नः संपन्नो युद्धयोजकैः}
{भ्रातृभिः श्वशुरैर्वीरैरुपपन्नो महारथैः}


\twolineshloka
{धृत्या च पुरुषव्याघ्रो नैभृत्येन च पाण्डवः}
{अनुशंसो वदान्यश्च हीमान्सत्यपराक्रमः}


\twolineshloka
{बहुश्रुतः कृतात्मा च वृद्धसेवी जितेन्द्रियः}
{तं सर्वगुणसंपन्नं समिद्धमिव पावकम्}


\twolineshloka
{तपन्तमभि को मन्दः पतिष्यति पतङ्गवत्}
{पाण्डवाग्निमनावार्यं मुमूर्षुर्नष्टचेतनः}


\twolineshloka
{तनुरुच्चः शिखी राजा मिथ्योपचरितो मया}
{मन्दानां मम पुत्राणां युद्धेनान्तं करिष्यति}


\twolineshloka
{तैरयुद्धं साधु मन्ये कुरवस्तन्निबोधत}
{युद्धे विनाशः कृत्स्नस्य कुलस्य भविता ध्रुवम्}


\twolineshloka
{एषा मे परमा बुद्धिर्यया शाम्यति मे मनः}
{यदि त्वयुद्धमिष्टं वो वयं शान्त्यै यतामहे}


\twolineshloka
{न तु नः क्लिश्यमानानामुपेक्षेत युधिष्ठिरः}
{जुगुप्सति ह्यधर्मेण मामेवोद्दिश्य कारणम्}


\chapter{अध्यायः ५४}
\twolineshloka
{सञ्जय उवाच}
{}


\twolineshloka
{एवमेतन्महाराज यथावदसि भारत}
{युद्धे विनाशः क्षत्रस्य गाण्डीवेन प्रदृश्यते}


\twolineshloka
{इदं तु नाभिजानामि तव धीरस्य नित्यशः}
{यत्पुत्रवशमागच्छेस्तत्त्वज्ञः सव्यसाचिनः}


\twolineshloka
{नैष कालो महाराज तव शश्वत्कृतागसः}
{त्वया ह्येवादितः पार्था निकृता भरतर्षभ}


\twolineshloka
{पिता श्रेष्ठः सुहृद्यश्च सम्यक्प्रणिहितात्मवान्}
{आस्थेयं हि हितं तेन न द्रोग्धा गुरुरुच्यते}


\twolineshloka
{इदं जितमिदं लब्धमिति श्रुत्वा पराजितान्}
{द्यूतकाले महाराज स्मरसे स्म कुमारवत्}


\twolineshloka
{परुषाण्युच्यमानांश्च पुरा पार्थानुपेक्षसे}
{कृत्स्नं राज्यं जयन्तीति प्रपातं नानुपश्यसि}


\twolineshloka
{पित्र्यं राज्यं महाराज कुरवस्ते सजाङ्गलाः}
{अथ वीरैर्जितामुर्वीमखिलां प्रत्यपद्यथाः}


\twolineshloka
{बाहुवीर्यार्जिता भूमिस्तव पार्थैर्निवेदिता}
{मयेदं कृतमित्येव मन्यसे राजसत्तम}


\twolineshloka
{ग्रस्तान्गन्धर्वराजेन मज्जतो ह्यप्लवेऽम्भसि}
{आनिनाय पुनः पार्थः पुत्रांस्ते राजसत्तम}


\twolineshloka
{कुमारवच्च स्मयसे द्यूते विनिकृतेषु यत्}
{पाण्डवेषु वने राजन्प्रव्रजत्सु पुनःपुनः}


\twolineshloka
{प्रवर्षतः शरव्रातानर्जुनस्य शितान्बहून्}
{अप्यर्णवा विशुष्येयुः किं पुनर्मासयोनयः}


\threelineshloka
{अस्यतां फाल्गुनः श्रेष्ठो गाण्डीवं धनुषां वरम्}
{केशवः सर्वभूतानामायुधानां सुदर्शनम्}
{वानरो रोचमानश्च केतुः केतुमतां वरः}


\twolineshloka
{एवमेतानि स रथो वहते सह यो रणे}
{क्षपयिष्यति नो राजन्कालचक्रमिवोद्यतम्}


\twolineshloka
{तस्याद्य वसुधा राजन्निखिला भरतर्षभ}
{यस्य भीमार्जिनौ योधौ स राजा राजसत्तम}


\twolineshloka
{तथा भीमहतप्रायां मञ्जन्तीं तव वाहिनीम्}
{दुर्योधनमुखा दृष्ट्वा क्षयं यास्यन्ति कौरवाः}


\twolineshloka
{न भीमार्जुनयोर्भीता लप्स्यन्ते विजयं विभो}
{तव पुत्रा महाराज राजानश्चानुसारिणः}


\twolineshloka
{मत्स्यास्त्वामद्य नार्चन्ति पञ्चालाश्च सकेकयाः}
{साल्वेयाः शूरसेनाश्च सर्वे त्वामवजानते}


\twolineshloka
{पार्थं ह्येते गताः सर्वे वीर्यज्ञास्तस्य धीमतः}
{भक्त्या ह्यस्य विरुध्यन्ते तव पुत्रैः सदैव ते}


\twolineshloka
{अनर्हानेव तु वधे धर्मयुक्तान्विकर्मणा}
{योऽक्लेशयत्पाण्डुपुत्रान्यो विद्वेष्ट्यधुनापि वै}


\twolineshloka
{सर्वोपायैर्नियन्तव्यः सानुगः पापपुरुषः}
{तव पुत्रो महाराज नानुशोचितुमर्हसि}


\twolineshloka
{अपरोहं महाराज साक्षाच्चैनं ब्रवीम्यहम्}
{द्यूतकाले मया चोक्तं विदुरेण च धीमता}


\twolineshloka
{यदिदं ते विलपितं पाण्डवान्प्रति भारत}
{अनीशेनैव राजेन्द्र सर्वमेतन्निरर्थकम्}


\chapter{अध्यायः ५५}
\twolineshloka
{दुर्योधन उवाच}
{}


\twolineshloka
{न भेतव्यं महाराज न शोच्या भवतां वयम्}
{समर्थाः स्म पराञ्जेतुं बलिनः समरे विभो}


\twolineshloka
{वने प्रव्राजितान्पार्थान्यदायान्मधुसूदनः}
{महता बलचक्रेण परराष्ट्रावमर्दिना}


\twolineshloka
{केकया धृष्टकेतुश्च धृष्टद्युम्नश्च पार्षत}
{राजानश्चान्वयुः पार्थान्बहवोऽन्येऽनुयायिनः}


\twolineshloka
{इन्द्रप्रस्थस्य चादूरात्समाजग्मुर्महारथाः}
{व्यगर्हयंश्च संगम्य भवन्तं कुरुभिः सह}


\twolineshloka
{ते युधिष्ठिरमासीनमजिनैः प्रतिवासितम्}
{कृष्णप्रधानाः संहत्य पर्युपासन्त भारत}


\twolineshloka
{प्रत्यादनं च राज्यस्य कार्यमूचुर्नराधिपाः}
{भवतः सानुबन्धस्य समुच्छेदं चिकीर्षवः}


\twolineshloka
{श्रुत्वा चैवं मयोस्तास्तु भीष्मद्रोणकृपास्तदा}
{ज्ञातिक्षयभयाद्राजन्भीतेन भरतर्षभ}


\twolineshloka
{ततः स्थास्यन्ति समये पाण्डवा इति मे मतिः}
{समुच्छेदं हि नः कृत्स्नं वासुदेवश्चिकीर्षति}


\twolineshloka
{ऋते च विदुरात्सर्वे यूयं वध्या मता मम}
{धृतराष्ट्रस्तु धर्मज्ञो न वध्यः कुरुसत्तमः}


\twolineshloka
{समुच्छेदं च कृत्स्नं नः कृत्वा तात जनार्दनः}
{एकराज्यं कुरूणां स्म चिकीर्षति युधिष्ठिरे}


\twolineshloka
{तत्र किं प्राप्तकालं नः प्रणिपातः पलायनम्}
{प्राणान्वा संपरित्यज्य प्रतियुध्यामाहे परान्}


\twolineshloka
{प्रतियुद्धे तु नियतः स्यादस्माकं पराजयः}
{युधिष्ठिरस्य सर्वे हि पार्थिवा वशवर्तिनः}


\twolineshloka
{विरक्तराष्ट्राश्च वयं मित्राणि कुपितानि नः}
{धिक्कृताः पार्थिवैः सर्वैः स्वजनेन च सर्वशः}


\twolineshloka
{प्रणिपाते न दोषोस्ति सन्धिर्नः शाश्वतीः समाः}
{पितरं त्वेव शोचामि प्रज्ञानेत्रं जनाधिपम्}


\threelineshloka
{मत्कृते दुःखमापन्नं क्लेशं प्राप्तमनन्तकम्}
{कृतं हि तव पुत्रैश्च परेषामवरोधनम्}
{मत्प्रयार्थं पुरैवैतद्विदितं ते नरोत्तम}


\twolineshloka
{ते राज्ञो धृतराष्ट्रस्य सामात्यस्य महारथाः}
{वैरं प्रतिकरिष्यन्ति कुलोच्छेदेन पाण्डवाः}


\twolineshloka
{ततो द्रोणोऽब्रवीद्भीष्मः कृपो द्रौणिश्च भारत}
{मत्वा मां महतीं चिन्तामास्थितंव्यथितेन्द्रियं}


\twolineshloka
{अभिद्रुग्धाः परे चेन्नो न भेतव्यं परन्तप}
{असमर्थाः परे जेतुमस्मान्युधि समास्थितान्}


\twolineshloka
{एकैकशः समर्थाः स्मो विजेतुं सर्वपार्थिवान्}
{आगच्छन्तु विनेष्यामो दर्पमेषां शितैः शरैः}


\twolineshloka
{पुरैकेन हि भीष्मेण विजिताः सर्वपार्थिवाः}
{मृते पितर्यतिक्रुद्धो रथेनैकेन भारत}


\twolineshloka
{जघान सुबहूंस्तेषां संरब्धः कुरुसत्तमः}
{ततस्ते शरणं जग्मुर्देवव्रतमिमं भयात्}


\twolineshloka
{स भीष्मः सुसमर्थोऽयमस्माभिः सहितो रणे}
{परान्विजेतुं तस्मात्ते व्येतु भीर्भरतर्षभ}


\twolineshloka
{इत्येषां निश्चयो ह्यासीत्तत्कालेऽमिततेजसाम्}
{पुरा तेषां पृथिवी कृत्स्नासीद्वशवर्तिनी}


\twolineshloka
{अस्मान्पुनरमी नाद्य समर्था जेतुमाहवे}
{छिन्नपक्षाः परे ह्यद्य वीर्यहीनाश्च पाण्डवाः}


\twolineshloka
{अस्मत्संस्था च पृथिवी वर्तते भरतर्षभ}
{एकार्थाः सुखदुःखेषु समानीताश्च पार्थिवाः}


\twolineshloka
{अप्यग्निं प्रविशेयुस्ते समुद्रं वा परन्तप}
{मदर्थं पार्थिवाः सर्वे तद्विद्धि कुरुसत्तम}


\twolineshloka
{उन्मत्तमिव चापि त्वां प्रहसन्तीह दुःखितम्}
{विलपन्तं बहुविधं भीतं परविकत्थने}


\twolineshloka
{एषां ह्येकैकशो राज्ञां समर्थं पाण्डवान्प्रति}
{आत्मानं मन्यते सर्वो व्येतु ते भयमागतम्}


\twolineshloka
{जेतुं समग्रां सेनां मे वासवोऽपि न शक्नुयात्}
{हन्तुमक्षय्यरूपेयं ब्रह्मणोऽपि स्वयंभुवः}


\twolineshloka
{युधिष्ठिरः पुरं हित्वा पञ्चग्रामान्स याचते}
{भीतो हि मामकात्सैन्यात्प्रभावाच्चैव मे विभो}


\twolineshloka
{समर्थं मन्यसे यच्च कुन्तीपुत्रं वृकोदरम्}
{तन्मिथ्या न हि मे कृत्स्नं प्रभावं वेत्सि भारत}


\twolineshloka
{मत्समो हि गदायुद्धे पृथिव्यां नास्ति कश्चन}
{नासीत्कश्चिदतिक्रान्तो भविता न च कश्चन}


\twolineshloka
{युक्तो दुःखोषितश्चाहं विद्यापारगतस्तथा}
{तस्मान्न भीमान्नान्येभ्यो भयं मे विद्यते क्वचित्}


\threelineshloka
{दुर्योधनसमो नास्ति गदायामिति निश्चयः}
{सङ्कर्षणस्य भवने यत्तदैनमुपावसम्}
{}


\twolineshloka
{युद्धे सङ्कर्षणसमो बलेनाभ्यधिको भुवि}
{गदाप्रहारं भीमो मे न जातु विषहेद्युधि}


\twolineshloka
{एकं प्रहारं यं दद्यां भीमाय रुषितो नृप}
{स एवैनं नयोद्धोरः क्षिप्रं वैवस्वतक्षयम्}


\twolineshloka
{इच्छेयं च गदाहस्तं राजन्द्रष्टुं वृकोदरम्}
{सुचिरं प्रार्थितो ह्येष मम नित्यं मनोरथः}


\twolineshloka
{गदया निहतो ह्याजौ मया पार्थो वृकोदरः}
{विशीर्णगावः पृथिवीं परासुः प्रपतिष्यति}


\twolineshloka
{गदाप्रहाराभिहतो हिमवानपि पर्वतः}
{सकृन्मया विदीर्येत गिरिः शतसहस्रधा}


\twolineshloka
{स चाप्येतद्विजानाति वासुदेवार्जुनौ तथा}
{दुर्योधनसमो नास्ति गदायामिति निश्चयः}


\twolineshloka
{तत्ते वृकोदरमयं भयं व्येतु महाहवे}
{व्यपनेष्याम्यहं ह्येनं मा राजन्विमना भव}


\twolineshloka
{तस्मिन्मया हते क्षिप्रमर्जुनं बहवो रथाः}
{तुल्यरूपा विशिष्टाश्च क्षेप्स्यन्ति भरतर्षभ}


\twolineshloka
{भीष्मो द्रोणः कृपो द्रौणिः कर्णो भूरिश्रवास्तथा}
{प्राग्ज्योतिषाधिपः शल्यः सिन्धुराजो जयद्रथः}


\twolineshloka
{एकैक एषां शक्तस्तु हन्तुं भारत पाण्डवान्}
{समेतास्तु क्षणेनैतान्नेष्यन्ति यमसादनम्}


\twolineshloka
{समग्रा पार्थिवी सेना पार्थमेकं धनञ्जयम्}
{कस्मादशक्ता निर्जेतुमिति हेतुर्न विद्यते}


\twolineshloka
{शरव्रातैस्तु भीष्मेण शतशो निचितोऽवशः}
{द्रोणद्रौणिकृपैश्चैव गन्ता पार्थो यमक्षयम्}


\twolineshloka
{पितामहोऽपि गाङ्गेयः शान्तनोरधि भारत}
{ब्रह्मर्पिसदृशो जज्ञे देवैरपि सुदुःसहः}


\twolineshloka
{न हन्ता विद्यते चापि राजन्भीष्मस्य कश्चन}
{पित्रा ह्युक्तः प्रसन्नेन नाकामस्त्वं मरिष्यसि}


\twolineshloka
{ब्रह्मर्षेश्च भरद्वाजाद्द्रोणो द्रोण्यामजायत}
{द्रोणाज्जज्ञे महाराज द्रौणिश्च परमाश्त्रवित्}


\twolineshloka
{कृपश्चाचार्यमुख्योयं महर्षेर्गौतमादपि}
{शरस्तम्बोद्भवः श्रीमानवध्य इति मे मतिः}


\twolineshloka
{अयोनिजास्त्रयो ह्येते पिता माता च मातुलः}
{अश्वत्थाम्नो महारात स च शूरः स्थितो मम}


\twolineshloka
{सर्व एते महाराज देवकल्पा महारथाः}
{शक्रस्यापि व्यथां कुर्युः संयुगे भरतर्षभ}


\twolineshloka
{नैतेषामर्जुनः शक्त एकैकं प्रतिवीक्षितुम्}
{सहितास्तु नरव्याघ्रा हनिष्यन्ति धनञ्जयम्}


\twolineshloka
{भीष्माद्रोणकृपाणां च तुल्यः कर्णो मतो मम}
{अनुज्ञातश्च रामेण मत्समोऽसीति भारत}


\twolineshloka
{कुण्डले रुचिरे चास्तां कर्णस्य सहजे शुभे}
{ते शच्यर्थं महेन्द्रेण याचितः स परन्तपः}


\twolineshloka
{अमोघया महाराज शक्त्या परमभीमया}
{तस्य शक्त्योपगूढस्य कस्माज्जीवेद्धनञ्जयः}


\threelineshloka
{विजयो मे ध्रुवं राजन्फलं पाणाविवाहितम्}
{अभिव्यक्तः परेषां च कृत्स्नो भुवि पराजयः}
{}


\twolineshloka
{अह्ना ह्येकेन भीष्मोयं प्रयुतं हन्ति भारत}
{तत्समाश्च महेष्वासा द्रोणद्रौणिकृपा अपि}


\twolineshloka
{संशप्तकानां वृन्दानि क्षत्रियाणां परन्तप}
{अर्जुनं वयमस्मान्वा निहन्यात्कपिकेतनः}


\twolineshloka
{तांश्चालमिति मन्यन्ते सव्यसाचिवधे धृताः}
{पार्थिवाः स भवांस्तेभ्यो ह्यकस्माद्व्यथते कथम्}


\twolineshloka
{भीमसेने च निहते कोऽन्यो युध्येत भारत}
{परेषां तन्ममाचक्ष्व यदि वेत्थ परन्तप}


\twolineshloka
{पञ्च ते भ्रातरः सर्वे धृष्टद्युम्नोऽथ सात्यकिः}
{परेषां सप्त ये राजन्योधाः सारं बलं मतम्}


\twolineshloka
{अस्माकं तु विशिष्टा ये भीष्मद्रोणकृपादयः}
{द्रौणिर्विकर्तनः कर्णः सोमदत्तोथ बाह्लिकः}


\twolineshloka
{प्राग्ज्योतिषाधिपः शल्य आवन्त्यौ च जयद्रथः}
{दुःशासनो दुर्मखश्च दुःसहश्च विशांपते}


\twolineshloka
{श्रुतायुश्चित्रसेनश्च पुरुमित्रो विविंशतिः}
{शलो भूरिश्रवाश्चैव विकर्णश्च तवात्मजः}


\twolineshloka
{अक्षौहिण्यो हि मे राजन्दशैका च समाहृताः}
{न्यूना परेषां सप्तैव कस्मान्मे स्यात्पराजयः}


\twolineshloka
{बलं त्रिगुणतो हीनं योध्यं प्राह बृहस्पतिः}
{परेभ्यस्त्रिगुणा चेयं मम राजन्ननीकिनी}


\twolineshloka
{गुणहीनं परेषां च बहु पश्यामि भारत}
{गुणोदयं बहुगुणमात्मनश्च विशांपते}


\twolineshloka
{एतत्सर्वं समाज्ञाय बलाग्र्यं मम भारत}
{न्यूनतां पाण्डवानां च न मोहं गन्तुमर्हसि}


\twolineshloka
{इत्युक्त्वा सञ्जयं भूयः पर्यपृच्छत भारत}
{विवित्सुः प्राप्तकालानि ज्ञात्वा परपुरञ्जयः}


\chapter{अध्यायः ५६}
\twolineshloka
{दुर्योधन उवाच}
{}


\threelineshloka
{अक्षौहिणीः सप्त लब्ध्वा राजभिः सह सञ्जय}
{किंस्विदिच्छति कौन्तेयो युद्धप्रेप्सुर्युधिष्ठिरः ॥सञ्जय उवाच}
{}


\twolineshloka
{अतीव मुदितो राजन्युद्धप्रेप्युर्युधिष्ठिरः}
{भीमसेनार्जुनौ चोभौ यमावपि न बिभ्यतः}


\twolineshloka
{रथं तु दिव्यं कौन्तेयः सर्वा विभ्राजयन्दिशः}
{मन्त्रं जिज्ञासमानः सन्बीभत्सुः समयोजयत्}


\threelineshloka
{तमपश्याम सन्नद्धं मेघं विद्युद्युतं यथा}
{समन्तात्समभिध्या हृष्यमाणोऽभ्यभाषत}
{पूर्वरूपमिदं पश्य वयं जेष्याम सञ्जय}


\twolineshloka
{बीभत्सुर्मां यथोवाच तथाऽवैम्यहमप्युत ॥दुर्योधन उवाच}
{}


\threelineshloka
{प्रशंसस्यभिनन्दंस्तान्पार्थानक्षपराजितान्}
{अर्जुनस्य रथे ब्रूहि कथमश्वाः कथं ध्वजाः ॥सञ्जय उवाच}
{}


\twolineshloka
{भौमनः सह शक्रेण बहुचित्रं विशांपते}
{रूपाणि कल्पयामास त्वष्टा धाता सदा विभो}


\twolineshloka
{ध्वजे हि तस्मिन्रूपाणि चक्रुस्ते देवमायया}
{महाधनानि दिव्यानि महान्ति च लघूनि च}


\twolineshloka
{भीमसेनानुरोधायक हनूमान्मारुतात्मजः}
{आत्मप्रतिकृतिं तस्मिन्ध्वज आरोपयिष्यति}


\twolineshloka
{सर्वा दिशो योजनमात्रमन्तरंसतिर्यगूर्ध्वं च रुरोध वै ध्वजः}
{न संसञ्जेत्तरुभिः संवृतोऽपितथा हि माया विहिता भौमनेन}


\twolineshloka
{यथाऽऽकाशे शक्रधनुः प्रकाशतेन चैकवर्णं न च वेद्मि किं नु तत्}
{तथा ध्वजो विहितो भौमनेनबह्वाकारं दृश्यते रूपमस्य}


\threelineshloka
{यथाऽग्निधूमो दिवमेति रुद्ध्वा}
{वर्णान्बिभ्रत्तैजसांश्चित्ररूपान्}
{तथा ध्वजो विहितो भौमनेनन चेद्भारो भविता नोत रोधः}


\threelineshloka
{श्वेतास्तस्मिन्वातवेगाः सदश्वादिव्या युक्ताश्चित्ररथेन दत्ताः}
{भुव्यन्तरिक्षे दिवि वा नरेन्द्रयेषां गतिर्हीयते नात्र सर्वा}
{शतं यत्तत्पूर्यते नित्यकालंहतंहतं दत्तवरं पुरस्तात्}


\twolineshloka
{तथा राज्ञो दन्तवर्णा बृहन्तोरथे युक्ता भान्ति तद्वीर्यतुल्याः}
{ऋक्षप्रख्या भीमसेनस्य वाहारथे वायोस्तुल्यवेग बभूवुः}


\twolineshloka
{कल्माषाङ्गास्तित्तिरिचित्रपृष्ठाभ्रात्रा दत्ताः प्रीयता फाल्गुनेन}
{भ्रातुर्बीरस्य स्वैस्तुरङ्गैर्विशिष्टामुदा युक्ताः सहदेवं वहन्ति}


\threelineshloka
{माद्रीपुत्रं नकुलं त्वाजमीढंमहेन्द्रदत्ता हरयो वाजिमुख्याः}
{समा वायोर्बलवन्तस्तरस्विनोवहन्ति वीरं वृत्रशत्रुं यथेन्द्रम्}
{}


\twolineshloka
{तुल्याश्चैभिर्वयसा विक्रमेणमहाजवाश्रित्ररूपाः सदश्वाः}
{सौभद्रादीन्द्रौपदेयान्कुमारान्वहन्त्यश्वा देवदत्ता बृहन्तः}


\chapter{अध्यायः ५७}
\twolineshloka
{धृतराष्ट्र उवाच}
{}


\threelineshloka
{कांस्तत्र सञ्जयापश्यः प्रीत्यर्थेन समागतान्}
{ये योत्स्यन्ते पाण्डवार्थे पुत्रस्य मम वाहिनीम् ॥सञ्जय उवाच}
{}


\twolineshloka
{मुख्यमन्धकवृष्णीनामपश्यं कृष्णमागतम्}
{चेकितानं च तत्रैव युयुधानं च सात्यकिम्}


\twolineshloka
{पृथगक्षौहिणीभ्यां तु पाण्डवानभिसंश्रितौ}
{महारथौ समाख्यातावुभौ पुरुषमानिनौ}


\threelineshloka
{अक्षौहिण्याऽथ पाञ्चाल्यो दशभिस्तनयैर्वृतः}
{सत्यजित्प्रमुखैर्वीरैर्धृष्टद्युम्नपुरोगमैः}
{}


\twolineshloka
{द्रुपदो वर्धयन्मानं शिखण्डिपरिपालितः}
{उपायात्सर्वसैन्यानां प्रतिच्छाद्य तदा वपुः}


\twolineshloka
{विराटः सह पुत्राभ्यां शङ्खेनैवोत्तरेण च}
{सूर्यदत्तादिभिर्वीरैर्मदिराक्षपुरोगमैः}


\twolineshloka
{सहितः पृथिवीपालो भ्रातृभिस्तनयैस्तथा}
{अक्षौहिण्यैव सैन्यानां वृतः पार्थं समाश्रितः}


\twolineshloka
{जारासन्धिर्मागधश्च धृष्टकेतुश्च चेदिराट्}
{पृथक्पृथगनुप्राप्तौ पृथगक्षौहिणीवृतौ}


\twolineshloka
{केकया भ्रातरः पञ्च सर्वे लोहितकध्वजाः}
{अक्षौहिणीपरिवृताः पाण्डवानभिसंश्रिताः}


\twolineshloka
{एतानेतावतस्तत्र तानपश्यं समागतान्}
{ये पाण्डवार्थे योत्स्यन्ति धार्तराष्ट्रस्य वाहिनीम्}


\twolineshloka
{यो वेद `मानुषं व्यूहं दैवं गान्धर्वमासुरम्}
{स तत्र सेनाप्रमुखे धृषटद्युम्नो महारथः}


% Check verse!
भीष्मः शान्तनवो राजन्भागः क्लृप्तः शिखण्डिनःतं विराटोऽनुसंयाता सार्धं मत्स्यैः प्रहारिभिः
\twolineshloka
{ज्येष्ठस्य पाण्डुपुत्रस्य भागो मद्राधिपो बली}
{तौ तु तत्राब्रुवन्केचिद्विषमौ नो मताविति}


\twolineshloka
{दुर्योधनः सहसुतः सार्धं भ्रातृशतेन च}
{प्राच्याश्च दाक्षिणात्याश्च भीमसेनस्य भागतः}


\twolineshloka
{अर्जुनस्य तु भागेन कर्णे वैकर्तनो मतः}
{अश्वत्थामा विकर्णश्च सैन्धवश्च जयद्रथः}


\twolineshloka
{अशक्याश्चैव ये केचित्पृथिव्यां शूरमानिनः}
{सर्वांस्तानर्जुनः पार्थः कल्पयामास भागतः}


\twolineshloka
{महेष्वांसा राजपुत्रा भ्रातरः पञ्च केकयाः}
{केकयानेव भागेन कृत्वा योत्स्यन्ति संयुगे}


\twolineshloka
{तेषामेव कृतो भागो मालवाः साल्वकास्तथा}
{त्रिगर्तानां च वै मुख्यौ यौ तौ संशप्तकाविति}


\twolineshloka
{दुर्योधनसुताः सर्वे तथा दुःशासनस्य च}
{सौभद्रेण कृतो भागो राजा चैव बृहद्बलः}


\twolineshloka
{द्रौपदेया महेष्वासाः सुवर्णविकृतध्वजाः}
{धृष्टद्युम्नमुखा द्रोणमभियास्यन्ति भारत}


\twolineshloka
{चेकितानः सोमदत्तं द्वैरथे योद्धुमिच्छति}
{भोजं तु कृतवर्माणं युयुधानो युयुत्सति}


\twolineshloka
{सहदेवस्तु माद्रेयः शूरः संक्रन्दनो युधि}
{स्वमंशं कल्पयामास श्यालं ते सुबलात्मजम्}


\twolineshloka
{उलूकं चैव कैतव्यं ये च सारस्वता गणाः}
{नकुलः कल्पयामास भागं माद्रवतीसुतः}


\twolineshloka
{ये चान्ये पार्थिवा राजन्प्रत्युद्यास्यन्ति सङ्गरे}
{समाह्वानेन तांश्चापि पाण्डुपुत्रा अकल्पयन्}


\threelineshloka
{एवमेषामनीकानि प्रविभक्तानि भागशः}
{यत्ते कार्यं सपुत्रस्य क्रियतां तदकालिकम् ॥धृतराष्ट्र उवाच}
{}


\twolineshloka
{न सन्ति सर्वे पुत्रा मे मूढा दुर्द्यूतदेविनः}
{येषां युद्धं बलवता भीमेन रणमूर्धनि}


\twolineshloka
{राजानः पार्थिवाः सर्वे प्रोक्षिताः कालधर्मणा}
{गाण्डीवाग्निं प्रवेक्ष्यन्ति पतङ्गा इव पावकम्}


\twolineshloka
{विद्रुतां वाहिनीं मन्ये कृतवैरैर्महात्मभिः}
{तां रणे केऽनुयास्यन्ति प्रभग्नां पाण्डवैर्युधि}


\twolineshloka
{सर्वे ह्यतिरथाः शूराः कीर्तिमन्तः प्रतापिनः}
{सूर्यपावकयोस्तुल्यास्तेजसा समितिंजयाः}


\twolineshloka
{येषां युधिष्ठिरो नेता गोप्ता च मधुसूदनः}
{योधौ च पाण्डवौ वीरौ सव्यसाचिवृकोदरौ}


\twolineshloka
{नकुलः सहदेवश्च धृष्टद्युम्नश्च पार्षतः}
{सात्यकिर्द्रुपदश्चैव धृष्टकेतुश्च सानुजः}


\twolineshloka
{उत्तमौजाश्च पाञ्चाल्यो युधामन्युश्च दुर्जयः}
{शिखण्डी क्षत्रदेवश्च तथा वैराटिरुत्तरः}


\twolineshloka
{काशयश्चेदयश्चैव मत्स्याः सर्वे च सृञ्जयाः}
{विराटपुत्रो बभ्रुश्च पाञ्चालाश्च प्रभद्रकाः}


\twolineshloka
{येषांमिन्द्रोऽप्यकामानां न हरेत्पृथिवीमिमाम्}
{वीराणां रणधीराणां ये भिन्द्युः पर्वतानपि}


\threelineshloka
{तान्सर्वगुणसंपन्नानमनुष्यप्रतापिनः}
{क्रोशतो मम दुष्पुत्रो योद्धुमिच्छति सञ्जय ॥दुर्योधन उवाच}
{}


\twolineshloka
{उभौ स्व एकजातीयौ तथोभौ भूमिगोचरौ}
{अथ कस्मात्पाण्डवानामेकतो मन्यसे जयम्}


\twolineshloka
{पितामहं च द्रोणं च कृपं कर्णं च दुर्जयम्}
{जयद्रथं सोमदत्तमश्वत्थामानमेव च}


\twolineshloka
{सुतेजसो महेष्वासानिन्द्रोऽपि सहितोऽमरैः}
{अशक्तः समरे जेतुं किं पुनस्तात पाण्डवाः}


\twolineshloka
{सर्वे च पृथिवीपाला मदर्थे तात पाण्डवान्}
{आर्याः शस्त्रभृतः शूराः समर्थाः प्रतिबाधितुं}


\twolineshloka
{न मामकान्पाण्डवास्ते समर्थाः प्रतिवीक्षितुम्}
{पराक्रान्तो ह्यहं पाण्डून्सपुत्रान्योद्धुमाहवे}


\twolineshloka
{मत्प्रियं पार्थिवाः सर्वे ये चिकीर्षन्ति भारत}
{ते तानावारयिष्यन्ति ऐणेयानिव तन्तुना}


\threelineshloka
{महता रथवंशेन शरजालैश्च मामकैः}
{अभिद्रुता भविष्यन्ति पाञ्चालाः पाण्डवैः सह ॥धृतराष्ट्र उवाच}
{}


\twolineshloka
{उन्मत्त इव मे पुत्रो विलपत्येष सञ्जय}
{न हि शक्तो रमे जेतुं धर्मराजं युधिष्ठिरम्}


\twolineshloka
{जानाति हि यथा भीष्मः पाण्डवानां यशस्विनाम्}
{बलवत्तां सपुत्राणां धर्मज्ञानां महात्मनाम्}


\twolineshloka
{यतो नारोचयदयं विग्रहं तैर्महात्मभिः}
{किं तु सञ्जय मे ब्रूहि पुनस्तेषां विचेष्टितम्}


\threelineshloka
{कस्तांस्तरस्विनो भूयः सन्दीपयति पाण्डवान्}
{अर्चिष्मतो महेष्वासान्हविषा पावकानिव ॥सञ्जय उवाच}
{}


\twolineshloka
{धृष्टद्युम्नः सदैवैतान्सन्दीपयति भारत}
{युद्ध्यध्वमिति माभैष्ट युद्धाद्भरतसत्तमाः}


\twolineshloka
{ये केचित्पार्थिवास्तत्र धार्तराष्ट्रेण संवृताः}
{युद्धे समागमिष्यन्ति तुमुले शस्त्रसङ्कुले}


\twolineshloka
{तान्सर्वानाहवे क्रुद्धान्सानुबन्धान्समागतान्}
{अहमेकः समादास्ये तिमिर्यत्स्यानिवोदकात्}


\twolineshloka
{भीष्मं द्रोणं कृपं कर्णं द्रौणिं शल्यं सुयोधनम्}
{एतांश्चापि निरोत्स्यामि वेलेव मकरालयम्}


\twolineshloka
{तथा ब्रुवन्तं धर्मात्मा प्राह राजा युधिष्ठिरः}
{तव धैर्यं च वीर्यं च पाञ्चालः पाण्डवैः सह}


\twolineshloka
{सर्वे समधिरूढाः स्म सङ्ग्रामान्नः समुद्धर}
{जानामि त्वां महाबाहो क्षत्रधर्मे व्यवस्थितम्}


\twolineshloka
{समर्थमेकं पर्याप्तं कौरवाणां विनिग्रहे}
{पुरस्तादुपयातानां कौरवाणां युयुत्सताम्}


\twolineshloka
{भवत्ता यद्विधातव्यं तन्नः श्रेयः परंतप}
{सङ्ग्रामादपयातानां भग्नानां शरणैषिणाम्}


\twolineshloka
{पौरुषं दर्शयञ्शूरो यस्तिष्ठेदग्रतः पुमान्}
{क्रीणीयात्तं सहस्रेण इति नीतिमतां मतम्}


\twolineshloka
{स त्वं शूरश्च वीरश्च विक्रान्तश्च नरर्षभ}
{भयार्तानां परित्राता संयुगेषु न संशयः}


\threelineshloka
{एवं ब्रुवति कौन्तेये धर्मात्मनि युधिष्ठिरे}
{धृष्टद्युम्न उवाचेदं मां वचो गतसाध्वसम्}
{सर्वाञ्जनपदान्सूत योधा दुर्योधनस्य ये}


\twolineshloka
{सबाह्लिकान्कुरून्ब्रूयाः प्रातिपेयाञ्शरद्वतः}
{सूतपुत्रं तथा द्रोणं सहपुत्रं जयद्रथम्}


% Check verse!
दुःशासनं विकर्णं च तथा दुर्योधनं नृपम्भीष्मं च ब्रूहि गत्वा त्वमाशु गच्छ च माचिरम्
\twolineshloka
{युधिष्ठिरः साधुनैवाभ्युपेयोमा वोऽवधीदर्जुनो देवगुप्तः}
{राज्यं दद्ध्वं धर्मराजस्य तूर्णंयाचध्वं वै पाण्डवं लोकवीरम्}


\twolineshloka
{नैतादृशो हि योधोऽस्ति पृथिव्यामिह कश्चन}
{यथाविधः सव्यसाची पाण्डवः सत्यविक्रमः}


\twolineshloka
{देवैर्हि संभृतो दिव्यो रथो गाण्डीवधन्वनः}
{न सजेयोमनुष्येण मा स्म कृद्ध्वं मनो युधि}


\chapter{अध्यायः ५८}
\twolineshloka
{धृतराष्ट्र उवाच}
{}


\twolineshloka
{क्षत्रतेजा ब्रह्मचारी कौमारादपि पाण्डवः}
{तेन संयुगमेष्यन्ति मन्दा विलपतो मम}


\twolineshloka
{दुर्योधन निवर्तस्व युद्धाद्भरतसत्तम}
{न हि युद्धं प्रशंसन्ति सर्वावस्थमरिन्दम}


\twolineshloka
{अलमर्धं पृथिव्यास्ते सहामात्यस्य जीवितुम्}
{प्रयच्छ पाण्डुपुत्राणां यथोचितमरिन्दम}


\twolineshloka
{एतद्धि कुरवः सर्वे मन्यन्ते धर्मसंहितम्}
{यत्त्वं प्रशान्तिं मन्येथाः पाण्डुपुत्रैर्महात्मभिः}


\twolineshloka
{अङ्गेमां समवेक्षस्व पुत्र स्वामेव वाहिनीम्}
{जात एष तवाभावस्त्वं तु मोहान्न बुद्ध्यसे}


\twolineshloka
{न त्वहं युद्धमिच्छामि नैतदिच्छति बाह्लिकः}
{न च भीष्मो न च द्रोणो नाश्वत्थामा न सञ्जयः}


\twolineshloka
{न सोमदत्तो न शलो न कृपो युद्धमिच्छति}
{सत्यव्रतः पुरुमित्रो जयो भूरिश्रवास्तथा}


\twolineshloka
{येषु संप्रतितिष्ठेयुः कुरवः पीडिताः परैः}
{ते युद्धं नाभिनन्दन्ति तत्तुभ्यं तात रोचताम्}


\threelineshloka
{न त्वं करोषि कामेन कर्णः कारयिता तव}
{दुःशासनश्च पापात्मा शकुनिश्चापि सौबलः ॥दुर्योधन उवाच}
{}


\twolineshloka
{नाहं भवति न द्रोणे नाश्वत्थाम्नि न सञ्जये}
{न भीष्मे न काम्भोजे न कृपे न च बाह्लिके}


\twolineshloka
{सत्यव्रते पुरुमित्रे भूरिश्रवसि वा पुनः}
{अन्येषु वा तावकेषु भारं कुत्वा समाह्वयम्}


\twolineshloka
{अहं च तात कर्णश्च रणयज्ञं वितत्य वै}
{युधिष्ठिरं पशुं कृत्वा दीक्षितौ भरतर्षभ}


\twolineshloka
{रथो वेदी स्रुवः खङ्गो गदा स्रुक् कवचोऽजिनम्}
{चातुर्होत्रं च धुर्या मे शरा दर्भा हविर्यशः}


\twolineshloka
{आत्मयज्ञेन नृपते इष्ट्वा वैवस्वतं रणे}
{विजित्य च समेष्यावो हतामित्रौ श्रिया वृतौ}


\twolineshloka
{अहं च तात कर्णश्च भ्राता दुःशासनश्च मे}
{एते वयं हनिष्यामः पाण्डवान्समरे त्रयः}


\twolineshloka
{अहं हि पाण्डवान्हत्वा प्रशास्ता पृथिवीमिमाम्}
{मां वा हत्वा पाण्डुपुत्रा भोक्तारः पृथिवीमिमां}


\twolineshloka
{त्यक्तं मे जीवितं राज्यं धनं सर्वं च पार्थिव}
{न जातु पाण्डवैः सार्धं वसेयमहमच्युत}


\threelineshloka
{यावद्धि सूच्यास्तीक्ष्णाया विध्येदग्रेण मारिष}
{तावदप्यपरित्याज्यं भूमेर्नः पाण्डवान्प्रति ॥धृतराष्ट्र उवाच}
{}


\twolineshloka
{सर्वान्वस्तात शोचामि त्यक्तो दुर्योधनो मया}
{ये मन्दमनुयास्यध्वं यान्तं वैवस्वतक्षयम्}


\twolineshloka
{रुरूणामिव यूथेषु व्याघ्राः प्रहरतां वराः}
{वरान्वरान्हनिष्यन्ति समेता युधि पाण्डवाः}


\twolineshloka
{प्रतीपमिव मे भाति युयुधानेन भारती}
{व्यस्ता सीमन्तिनी ग्रस्ता प्रमृष्टा दीर्घबाहुना}


\twolineshloka
{संपूर्मं पूरयन्भूयो धनं पार्थस्य माधवः}
{शैनेयः समरे स्थाता बीजवत्प्रवपञ्शरान्}


\twolineshloka
{सेनामुखे प्रयुद्धानां भीमसेनो भविष्यति}
{तं सर्वे संश्रयिष्यन्ति प्राकारमकुतोभयम्}


\twolineshloka
{यदा द्रक्ष्यसि भीमेन कुञ्जरान्विनिपातितान्}
{विशीर्णदन्तान्गिर्याभान्भिन्नकुम्भान्सशोणितान्}


\twolineshloka
{तानभिप्रेक्ष्य सङ्ग्रामे विशीर्णानिव पर्वतान्}
{भीतो भीमस्य संस्पर्शात्स्मर्तासि वचनस्य मे}


\twolineshloka
{निर्दग्धं भीमसेनेन सैन्यं रथहयद्विपम्}
{गतिमग्नेरिव प्रेक्ष्य स्मर्तासि वचनस्य मे}


\twolineshloka
{महद्वो भयमागामि न चेच्छाम्यथ पाण्डवैः}
{गदया भीमसेनेन हताः शममुपैष्यथ}


\threelineshloka
{महावनमिव च्छिन्नं यदा द्रक्ष्यसि पातितम्}
{बलं कुरूणां भीमेन तदा स्मर्तासि मे वचः ॥वैशंपायन उवाच}
{}


\twolineshloka
{एतावदुक्त्वा राजा तु सर्वांस्तान्पृथिवीपतीन्}
{अनुभाष्य महाराज पुनः पप्रच्छ सञ्जयम्}


\chapter{अध्यायः ५९}
\twolineshloka
{धृतराष्ट्र उवाच}
{}


\threelineshloka
{ब्रूहि सञ्जय यच्छेषं वासुदेवादनन्तरम्}
{यजर्जुन उवाच त्वां परं कौतूहलं हि मे ॥सञ्जय उवाच}
{}


\twolineshloka
{वासुदेववचः श्रुत्वा कुन्तीपुत्रो धनञ्जयः}
{उवाच काले दुर्धर्षो वासुदेवस्य शृण्वतः}


\twolineshloka
{पितामहं शान्तनवं धृतराष्ट्रं च सञ्जय}
{द्रोणं कृपं च शल्यं च महाराजं च बाह्लिकम्}


\twolineshloka
{द्रौणिं च सौमदत्तिं च शकुनिं चापि सौबलम्}
{दुश्शासनं शलं चैव पुरुमित्रं विविंशतिम्}


\twolineshloka
{विकर्णं चित्रसेनं च जयत्सेनं च पार्थिवम्}
{विन्दानुविन्दावावन्त्यौ दुर्मुखं चापि पौरवम्}


\twolineshloka
{सैन्धवं दुस्सहं चैव भूरिश्रवसमेव च}
{भगदत्तं च राजानं जलसन्धं च पौरवम्}


\twolineshloka
{ये चाप्यन्ये पार्थिवास्तत्र योद्धुंसमागताः कौरवाणां प्रियार्थम्}
{मुमूर्षवः पाण्डवाग्नौ प्रदीप्तेसमानीता धार्तराष्ट्रोण सूत}


% Check verse!
यथान्यायं कुशलं वन्दनं चसमागता मद्वचनेन वाच्याः
\threelineshloka
{इदं ब्रूवाः सञ्जय राजमध्येदुर्योधनं पापकृतां प्रधानम्}
{अमर्षणं दुर्मतिं राजपुत्रंपापात्मानं धार्तराष्ट्रं सुलुब्धम्}
{सर्वं ममैतद्वचनं समग्रंसहामात्यं सञ्जय श्रावयेथाः}


\twolineshloka
{एवं परिष्वज्य धनञ्जयो मांततोऽर्थवद्धर्मवच्चापि वाक्यम्}
{प्रोवाचेदं वासुदेवं समीक्ष्यपार्थो धीमाँल्लोहितान्तायताक्षः}


\twolineshloka
{यथाश्रुतं ते वदतो महात्मनोयदुप्रवीरस्य वचः समाहितम्}
{तथैव वाच्यं भवतापि मद्वचःसमागतेषु क्षितिपेषु सर्वशः}


\twolineshloka
{शराग्निधूमे रथनेमिनादितेधनुस्स्त्रुवेणास्त्रबलापहारिणा}
{यथा न होमः क्रियते महामृधेतथा समेत्य प्रयतध्वमादृताः}


\twolineshloka
{न चेत्प्रयच्छध्वममित्रघातिनोयुधिष्ठिरस्यार्धमभीप्सितं स्वकम्}
{नयामि वः साश्वपदातिकुञ्जरा-न्दिशं पितॄणामशिवां शितैः शरैः}


\twolineshloka
{ततोऽहमामन्त्र्य चतुर्भुजं हरिंधनञ्जयं चैव नमस्य सत्वरम्}
{जवेन संप्राप्त इहामरद्युतेतवान्तिकं प्रापयितुं वचो महत्}


\chapter{अध्यायः ६०}
\twolineshloka
{वैशंपायन उवाच}
{}


\twolineshloka
{सञ्जयस्य वचः श्रुत्वा प्रज्ञाचक्षुर्जनेश्वरः}
{ततः सङ्ख्यातुमारेभे तद्वचो गुणदोषतः}


\twolineshloka
{प्रसङ्ख्याय च सौक्ष्म्येण गुणदोषान्विचक्षणः}
{यथावन्मतितत्त्वेन जयकामः सुतान्प्रति}


\twolineshloka
{बलाबलं विनिश्चित्य याथातथ्येन बुद्धिमान्}
{`यदा तु मेने भूयिष्ठं तद्वचो गुणदोषतः}


\twolineshloka
{पुनरेव कुरूणां च पाण्डवानां च बुद्धिमान्}
{'शक्तिं शङ्ख्यातुमारेभे तदा वै मनुजाधिपः}


\twolineshloka
{देवमानुषयोः शक्त्या तेजसा चैव पाण्डवान्}
{कुरूञ्शक्त्याऽल्पतरया दुर्योधनमथाब्रवीत्}


\twolineshloka
{दुर्योधनेयं चिन्ता मे शश्वन्न व्युपशाम्यति}
{सत्यं ह्येतदहं मन्ये प्रत्यक्षं नानुमानतः}


\twolineshloka
{आत्मजेषु परं स्नेहं सर्वभातिनि कुर्वते}
{प्रियामि चैषां कुर्वन्ति यथाशक्ति हितानि च}


\twolineshloka
{एवमेवोपकर्तॄणां प्रायशो लक्षयामहे}
{इच्छन्ति बहुलं सन्तः प्रतिकर्तुं महत्प्रियम्}


\twolineshloka
{अग्निः साचिव्यर्ता स्यात्खाण्डवे तत्कृतं स्मरन्}
{अर्जुनस्यापि भीमेऽस्मिन्कुरुपाण्डुसमागमे}


\twolineshloka
{जातिगृद्ध्याऽभिपन्नाश्च पाण़्डवानामनेकशः}
{धर्मादयः समेष्यन्ति समाहूता दिवौकसः}


\twolineshloka
{भीष्मद्रोणकृपादीनां भयादशनिसन्निभम्}
{रिरक्षिषन्तः संरभ्यं गमिष्यन्तीति मे मतिः}


\twolineshloka
{ते देवैः सहिताः पार्था न शक्याः प्रतिवीक्षितुम्}
{मानुषेण नरव्याघ्रा वीर्यवन्तोऽस्त्रपारगाः}


\twolineshloka
{दुरासदं यस्य दिव्यं गाण्डीवं धनुरुत्तमम्}
{वारुणौ चाक्षयौ दिव्यौ शरपूर्णौ महेषुधी}


\twolineshloka
{वानरश्च ध्वजे दिव्यो निःसङ्गे धूमवद्गतिः}
{रथश्च चतुरन्तायां यस्य नास्ति समः क्षितौ}


\twolineshloka
{महामेघनिभश्चापि निर्घोषः श्रूयते जनैः}
{महाशनिसमः शब्दःत शात्रवाणां भयंकरः}


\twolineshloka
{यं चातिमानुषं वीर्ये कृत्स्नो लोको व्यवस्यति}
{देवानामपि जेतारं यं विदुः पार्थिवा रणे}


\twolineshloka
{शतानि पञ्च चैवेषून्यो गृह्णन्नैव दृश्यते}
{निमेषान्तरमात्रेण मुञ्चन्दूरं च पातयन्}


\twolineshloka
{यमाह भीष्मो द्रोणश्च कृपो द्रौणिस्तथैव च}
{मद्रराजस्तथा शल्यो मध्यस्था ये च मानवाः}


\twolineshloka
{युद्धायावस्थितं पार्थं पार्थिवैरतिमानुषैः}
{अशक्यं रथशार्दूलं पराजेतुमरिंदमम्}


\twolineshloka
{क्षिपत्येकेन वेगेन पञ्चबाणशतानि यः}
{सदृशं बाहुवीर्येण कार्तवीर्यस्य पाण्डवम्}


\twolineshloka
{तमर्जुनं महेष्वासं महेन्द्रोपेन्द्रविक्रमम्}
{निघ्नन्तमिव पश्याभि विमर्देऽस्मिन्महाहवे}


\twolineshloka
{इत्येवं चिन्तयन्कृत्स्नमहोरात्राणि भारत}
{अनिद्रो निःसुखश्चास्मि कुरूणां शमचिन्तया}


\twolineshloka
{क्षयोदयोऽयं सुमहान्कुरूणां प्रत्युपस्थितः}
{अस्य चेत्कलहन्यान्तः शमादन्यो न विद्यते}


\twolineshloka
{शमो मे रोचते नित्यं पार्थैस्तात न विग्रहः}
{कुरुभ्यो हि सदा मन्ये पाण्डवाञ्शक्तिमत्तरान्}


\chapter{अध्यायः ६१}
\twolineshloka
{वैशंपायन उवाच}
{}


\twolineshloka
{पितुरेतद्वचः श्रुत्वा धार्तराष्ट्रोऽत्यमर्षणः}
{आधाय विपुलं क्रोधं पुनरेवेदमब्रवीत्}


\twolineshloka
{अशक्या देवसचिवाः पार्थाः स्युरिति यद्भवान्}
{मन्यते तद्भयं व्येतु भवतो राजसत्तम}


\twolineshloka
{अकामद्वेषसंयोगाल्लोभद्रोहाच्च भारत}
{उपेक्षया च भावानां देवा देवत्वमाप्नुवन्}


\twolineshloka
{इति द्वैपायेनो व्यासो नारदश्च महातपाः}
{जामदग्न्यश्च रामो नः कथामकथयत्पुरा}


\twolineshloka
{नैव मानुषवद्देवाः प्रवर्तन्ते कदाचन}
{कामात्क्रोधात्तथा लोभाद्द्वेषाच्च भरतर्षभ}


\twolineshloka
{यदा ह्यग्निश्च वायुश्च धर्म इन्द्रोऽश्विनावपि}
{कामयोगात्प्रवर्तेरन्न पार्था दुःखमाप्नुयुः}


\twolineshloka
{तस्मान्न भवता चिन्ता कार्यैषा स्यात्कथंचन}
{दैवेष्वपेक्षका ह्येते शश्वद्भावेषु भारत}


\twolineshloka
{अथ चेत्कामसंयोगाद्द्वेषो लोभश्च लक्ष्यते}
{देवेषु दैवप्रामाण्यान्नैषां तद्विक्रमिष्यति}


\twolineshloka
{मयाऽभिमन्त्रतः शश्वञ्जातवेदाः प्रशाम्यति}
{दिधक्षुः सकलाँल्लोकान्परिक्षिप्य समन्ततः}


\twolineshloka
{यद्वा परमकं तेजो येन युक्ता दिवौकसः}
{ममाप्यनुपमं भूयो देवेभ्यो विद्धि भारत}


\twolineshloka
{विदीर्यमाणां वसुधां गिरीणां शिखराणि च}
{लोकस्य पश्यतो राजन्स्थापयाम्यभिमन्त्रणात्}


\twolineshloka
{चेतनाचेतनस्यास्य जङ्गमस्थावरस्य च}
{विनाशाय समुत्पन्नमहं घोरं महास्वनम्}


\twolineshloka
{अश्मवर्षं च वायुं च शमयामीह नित्यशः}
{जगतः पश्यतोऽभीक्ष्णं भूतानामनुकम्पया}


\twolineshloka
{स्तम्भितास्वप्सु गच्छन्ति मया रथपदातयः}
{देवासुराणां भावानामहमेकः प्रवर्तिता}


\twolineshloka
{अक्षौहिणीभिर्यान्देशान्यामि कार्येण केनचित्}
{तत्रापो मे प्रवर्तन्ते यत्र यत्राभिकामये}


\twolineshloka
{भयानकानि विषये व्यालादीनि न सन्ति मे}
{मन्त्रगुप्तानि भूतानि न हिंसन्ति यंकराः}


\twolineshloka
{निकामवर्षी पर्जन्यो राजन्विषयवासिनाम्}
{धर्मिष्ठाश्च प्रजाः सर्वा ईतयश्च न सन्ति मे}


\twolineshloka
{अश्विनावथ वाय्वग्नी मरुद्भिः सह वृत्रहा}
{धर्मश्चैव मया द्विष्टान्नोत्सहन्तेऽभिरक्षितुम्}


\twolineshloka
{यदि ह्येते समर्थाः स्युर्मद्द्विषस्त्रातुमञ्जसा}
{न स्म त्रयोदश समाः पार्था दुःखभवाप्नुयुः}


\twolineshloka
{नैव देवा न गन्धर्वा नासुरा न च राक्षसाः}
{शक्तास्त्रातुं मया द्विष्टं सत्यमेतद्ब्रवीमि ते}


\twolineshloka
{दयभिध्याम्यहं शश्वच्छुभं वा यदि वाऽशुभम्}
{नैतद्विपन्नपूर्वं मे मित्रेष्वरिषु चोभयोः}


\twolineshloka
{भविष्यतीदमिति वा यद्ब्रवीमि परन्तप}
{नान्यथा भूतपूर्वं च सत्यवागिति मां विदुः}


\twolineshloka
{लोकसाक्षिकमेतन्मे माहात्म्यं दिक्षु विश्रुतम्}
{आश्वासनार्थं भवतः प्रोक्तं न श्लाघया नृप}


\twolineshloka
{न ह्यहं श्लाघनो राजन्भूतपूर्वः कदाचन}
{असदाचरितं ह्येतद्यदात्मानं प्रशंसति}


\twolineshloka
{पाण्डवांश्चैव मत्स्यांश्च पाञ्चालान्केकयैः सह}
{सात्यकिं वासुदेवं च श्रोतासि विजितान्मया}


\twolineshloka
{सरितः सागरं प्राप्य यथा नश्यन्ति सर्वशः}
{तथैव ते विनङ्क्ष्यन्ति मामासाद्य सहान्वयाः}


\twolineshloka
{परा बुद्धिः परं तेजो वीर्यं च परमं मम}
{परा विद्या परो योगो मम तेभ्यो विशिष्यते}


\twolineshloka
{पितामहश्च द्रोणश्च कृपः शल्यः शलस्तथा}
{अस्त्रेषु यत्प्रजानन्ति सर्वं तन्मयि विद्यते}


\twolineshloka
{इत्युक्ते सञ्जयं भूयः पर्यपृच्छत भारत}
{ज्ञात्वा युयुत्सोः कार्याणि प्राप्तकालमरिन्दमः}


\chapter{अध्यायः ६२}
\twolineshloka
{वैशंपायन उवाच}
{}


\twolineshloka
{तथा तु पृच्छन्तमतीव पार्थंवैचित्रवीर्यं तमचिन्तयित्वा}
{उवाच कर्णो धृतराष्ट्रपुत्रंप्रहर्षयन्संसदि कौरवाणाम्}


\twolineshloka
{मिथ्या प्रतिज्ञाय मया यदस्त्रंरामात्कृतं ब्रह्ममयं पुरस्तात्}
{विज्ञाय तेनास्मि तदैवमुक्त-स्ते नान्तकाले प्रतिभास्यतीति}


\twolineshloka
{महापराधे ह्यपि यन्न तेनमहर्षिणाऽहं गुरुणा च शप्तः}
{शक्तः प्रदग्धुं ह्यपि तिग्मतेजाःससागरामप्यवनिं महर्षिः}


% Check verse!
प्रसादितं ह्यस्य मया मनोऽभू-च्छुश्रूषया स्वेन च पौरुषेणतदस्ति चास्त्रं मम सावशेषंतस्मात्समर्थोऽस्मि ममैष भारः
\twolineshloka
{निमेषमात्रात्तमृषेः प्रसाद-मवाप्य पाञ्चालकरूशमत्स्यान्}
{निहत्य पार्थान्सह पुत्रपौत्रै-र्लोकानहं शस्त्रजितान्प्रपत्स्ये}


\twolineshloka
{पितामहस्तिष्ठतु ते समीपेद्रोणश्च सर्वे च नरेन्द्रमुख्याः}
{ऋषिप्रसादेन बलेन गत्वापार्थान्हनिष्यामि ममैष भारः}


\twolineshloka
{एवं ब्रुवन्तं तमुवाच भीष्मःकिं कत्थसे कालपरीतबुद्धे}
{न कर्ण जानासि यथा प्रधानेहते हताः स्युर्धृतराष्ट्रपुत्राः}


\twolineshloka
{यत्खाण्डवं दाहयता कृतं हिकृष्णद्वितीयेन धनञ्जयेन}
{श्रुत्वैव तत्कर्म नियन्तुमात्मायुक्तस्त्वया वै सहबान्धवेन}


\twolineshloka
{यां चापि शक्तिं त्रिदशाधिपस्तेददौ महात्मा भगवान्महेन्द्रः}
{भस्मीकृतां तां समरे विशीर्णांचक्राहतां द्रक्ष्यसि केशवेन}


\twolineshloka
{यस्ते शरः सर्पमुखो विभातिसदाऽग्र्यमाल्यैर्महितः प्रयत्नात्}
{स पाण्डुपुत्राभिहतः शरौघैःसह त्वया यास्यति कर्ण नाशम्}


\threelineshloka
{बाणस्य भौमस्य च कर्ण हन्ताकिरीटिनं रक्षति वासुदेवः}
{यस्त्वादृशानां च वरीयसां चहन्ता रिपूणां तुमुले प्रगाढे ॥कर्ण उवाच}
{}


\twolineshloka
{असंशयं वृष्णिपतिर्यथोक्त-स्तथा च भूयांश्च ततो महात्मा}
{अहं यदुक्तः परुषं तु किंचि-त्पितामहस्तस्य फलं शृणोतु}


\threelineshloka
{न्यस्यामि शस्त्राणि न जातु सङ्ख्येपितामहो द्रक्ष्यति मां सभायाम्}
{त्वयि प्रशान्ते तु मम प्रभावंद्रक्ष्यन्ति सर्वे भुवि भूमिपालाः ॥वैशंपायन उवाच}
{}


\twolineshloka
{इत्येवमुक्त्वा स महाधनुष्मान्हित्वा सभां स्वं भवनं जगाम}
{भीष्मस्तु दुर्योधनमेव राजन्मध्ये कुरूणां प्रहसन्नुवाच}


\twolineshloka
{सत्यप्रतिज्ञः किल सूतपुत्र-स्तथा कस भारं विषहेत कस्मात्}
{व्यूहं प्रतिव्यूह्य शिरांसि भित्त्वालोकक्षयं पश्यत भीमसेनात्}


\twolineshloka
{आवन्त्यकालिङ्गजयद्रथेषुचेदिध्वजे तिष्ठति बाह्लिके च}
{अहं हनिष्यामि सदा परेषांसहस्रशश्चायुतशश्च योधान्}


\threelineshloka
{यदैव रामे भगवत्यनिन्द्येब्रह्मब्रुवाणः कृतवांस्तदस्त्रम्}
{तदैव धर्मश्च तषश्च नष्टंवैकर्तनस्याधमपूरुषस्य ॥वैशंपायन उवाच}
{}


\twolineshloka
{तथोक्तवाक्ये नृपतीन्द्र भीष्मेनिक्षिप्य शस्त्राणि गते च कर्णे}
{वैचित्रवीर्यस्य सुतोऽल्पबुद्धि-र्दुर्योधनः शान्तनवं बभाषे}


\chapter{अध्यायः ६३}
\twolineshloka
{दुर्योधन उवाच}
{}


\twolineshloka
{सदृशानां मनुष्येषु सर्वेषां तुल्यजन्मनाम्}
{कथमेकान्ततस्तेषां पार्थानां मन्यसे जयम्}


\twolineshloka
{वयं च तेऽपि तुल्या वै वीर्येण च पराक्रमैः}
{समेन वयसा चैव प्रातिभेन श्रुतेन च}


\twolineshloka
{अस्त्रेण योधयुग्या च शीघ्रत्वे कौशले तथा}
{सर्वे स्म समजातीयाः सर्वे मानुषयोनयः}


\twolineshloka
{पितामह विजानीषे पार्थेषु विजयं कथम्}
{नाहं भवति न द्रोणे न कृपे न च बाह्लिके}


\twolineshloka
{अन्येषु च नरेन्द्रेषु पराक्रम्य समारभे}
{अहं वैकर्तनः कर्णो भ्राता दुःशासनश्च मे}


\twolineshloka
{पाण्डवान्समरे पञ्च हनिष्यामः शितैः शरैः}
{ततो राजन्महायज्ञैर्विविधैर्भूरिदक्षिणैः}


\threelineshloka
{ब्राह्मणांस्तर्पयिष्यामि गोभिरश्वैर्धनेन च}
{यदा परिकरिष्यन्ति ऐणेयानिव तन्तुना}
{अतरित्रानिव जले बाहुभिर्मामका रणे}


\twolineshloka
{पश्यन्तस्ते परांस्तत्र रथनागसमाकुलान्}
{तदा दर्पं विमोक्ष्यन्ति पाण्डवाः स च केशवः}


\threelineshloka
{`सुखान्यवाप्य सहिताः कृत्वा कर्म सुदुस्तरम्}
{विस्रब्धास्तु भविष्यामः प्राप्ते काले गतज्वराः ॥वैशंपायन उवाच}
{}


\threelineshloka
{अथाब्रवीन्महाराजो धृतराष्ट्रः सुदुर्मनाः}
{विदुरं विदुषां श्रेष्ठं सर्वपार्थिवसंनिधौ ॥धृतराष्ट्र उवाच}
{}


\threelineshloka
{मोहितो मृत्युपाशेन कालस्य वशमागतः}
{तात कर्णेन सहितः पुत्रो दुर्योधनो मम ॥'विदुर उवाच}
{}


\twolineshloka
{इह निःश्रेयसं प्राहुर्वृद्धा निश्चितदर्शिनः}
{ब्राह्मणस्य विशेषेण दमो धर्मः सनातनः}


\twolineshloka
{तस्य दानं क्षमा सिद्धिर्यथावदुपपद्यते}
{दमो दानं तपो ज्ञानमधीतं चानुवर्तते}


\twolineshloka
{दमस्तेजो वर्धयति पवित्रं दम उत्तमम्}
{विपाप्मा वृद्धतेजास्तु पुरुषो विन्दते महत्}


\twolineshloka
{क्रव्याद्भ्य इव भूतानामदान्तेभ्यः सदा भयम्}
{येषां च प्रतिषेधार्थं क्षत्रं सृष्टं स्वयंभुवा}


\twolineshloka
{आश्रमेषु चतुर्ष्वाहुर्दममेवोत्तमं व्रतम्}
{तस्य लिङ्गं प्रवक्ष्यामि येषां समुदयो दमः}


\twolineshloka
{क्षमा धृतिरहिंसा च समता सत्यमार्जवम्}
{इन्द्रियाभिजयो धैर्यं मार्दवं ह्रीरचापलम्}


\twolineshloka
{अकार्पण्यमसंरम्भः सन्तोषं श्रद्दधानता}
{एतानि यस्य राजेन्द्र स दान्तः पुरुषः स्मृतः}


\threelineshloka
{कामो लोभश्च दर्पश्च मन्युर्निद्रा विकत्थनम्}
{मान ईर्ष्यां च शोकश्च नैतद्दान्तो निषेवते}
{अजिह्ममशठं शुद्धमेतद्दान्तस्य लक्षणम्}


\twolineshloka
{अलोलुपस्तथाऽल्पेप्सुः कामानामविचिन्तिता}
{समुद्रकल्पः परुषः स दान्तः परिकीर्तितः}


\twolineshloka
{सुवृत्तः शीलसंपन्नः प्रसन्नात्माऽत्मविद्बुधः}
{प्राप्येह लोके संमानं सुगतिं प्रेत्य गच्छति}


\twolineshloka
{अभयं यस्य भूतेभ्यः सर्वेषामभयं यतः}
{स वै परिणतप्रज्ञः प्रख्यातो मनुजोत्तमः}


\twolineshloka
{सर्वभूतहितो मैत्रस्तस्मान्नोद्विजते जनः}
{समुद्र इव गम्भीरः प्रज्ञातृप्तः प्रशाम्यति}


\twolineshloka
{कर्मणाऽचरितं पूर्वं सद्भिराचरितं च यत्}
{तदेवास्थाय मोदन्ते दान्ताः शमपरायणाः}


\twolineshloka
{नैष्कर्म्यं वा समास्थाय ज्ञानतृप्तो जितेन्द्रियः}
{कालाकाङ्क्षी चरँल्लोके ब्रह्मभूयाय कल्पते}


\twolineshloka
{शकुनीनामिवाकाशे पदं नैवोपलभ्यते}
{एवं प्रज्ञानतृप्तस्य मुनेर्वर्त्म न दृश्यते}


\twolineshloka
{उत्सृज्यैव गृहान्यस्तु मोक्षमेवाभिमन्यते}
{लोकास्तेजोमयास्तस्य कल्पन्ते शाश्वता दिवि}


\chapter{अध्यायः ६४}
\twolineshloka
{विदुर उवाच}
{}


\twolineshloka
{शकुनीनामिहार्थाय पाशं भूमावयोजयत्}
{कश्चिच्छाकुनिकस्तात पूर्वेषामिति शुश्रुम्}


\twolineshloka
{तस्मिंस्तौ शकुनौ बद्धौ युगपत्सहचारिणौ}
{तावुपादाय तं पाशं जग्मतुः खचरावुभौ}


\twolineshloka
{तौ विहायसमाक्रान्तौ दृष्ट्वा शाकुनिकस्तदा}
{अन्वधावदनिर्विण्णो येन येन स्म गच्छतः}


\twolineshloka
{तथा तमनुधावन्तमृगयुं शकुनार्थिनम्}
{आश्रमस्थो मुनिः कश्चिद्ददर्शाथ कृताह्निकः}


\twolineshloka
{तावन्तरिक्षगौ शीघ्रमनुयान्तं महीचरम्}
{श्लोकेनानेन कौरव्य पप्च्छ स मुनिस्तदा}


\threelineshloka
{विचित्रमिदमाश्चर्यं मृगहन्प्रतिभाति मे}
{प्लवमानौ हि खचरौ पदातिरनुधावसि ॥शाकुनिक उवाच}
{}


\threelineshloka
{पाशमेकमुभावेतौ सहितौ हरतो मम}
{यत्र वै विवदिष्येते तत्र मे वशमेष्यतः ॥विदुर उवाच}
{}


\twolineshloka
{तौ विवादमनुप्राप्तौ शकुनौ मृत्युसन्धितौ}
{विगृह्य च स्रुदुर्बुद्धी पृथिव्यां संनिपेततुः}


\twolineshloka
{तौ युध्यमानौ संरब्धौ मृत्युपाशवशानुगौ}
{उपसृत्य प्रमत्तौ तौ जग्राह मृगहा तदा}


\twolineshloka
{एवं ये ज्ञातयोऽर्थेषु मिथो गच्छन्ति विग्रहम्}
{ते मृत्युवशमायान्ति शकुनाविव विग्रहात्}


\twolineshloka
{संभोजनं संकथनं संप्रश्नोऽथ समागमः}
{एतानि ज्ञातिकार्याणि न विरोधः कदाचन}


\twolineshloka
{ये स्म काले सुमनसः सर्वे वृद्धानुपासते}
{सिंहगुप्तमिवारण्यमप्रधृष्या भवन्ति ते}


\twolineshloka
{येऽर्थं सन्ततमासाद्य दीना इव समासते}
{श्रियं ते संप्रयच्छन्ति द्विषद्भ्यो भरतर्षभ}


\twolineshloka
{धूमायन्ते व्यपेतानि ज्वलन्ति सहितानि च}
{धृतराष्ट्रोल्मुकानीव ज्ञातयो भरतर्षभ}


\twolineshloka
{इदमन्यत्प्रवक्ष्यामि यथा दृष्टं गिरौ मया}
{श्रुत्वा तदपि कौरव्य यथा श्रेयस्तथा कुरु}


\twolineshloka
{वयं किरातैः सहिता गच्छामो गिरिमुत्तरम्}
{ब्राह्मणैर्देवकल्पैश्च विद्याजम्भकवार्तिकैः}


\twolineshloka
{कुञ्चभूतं गिरिं सर्वमभितो गन्धमादनम्}
{दीप्यमानौषधिगणं सिद्धगन्धर्वसेवितम्}


\twolineshloka
{तत्रापश्याम वै सर्वे मधु पीतकमाक्षिकम्}
{मरुप्रपाते विषमे निविष्टं कुम्भसंमितम्}


\twolineshloka
{आशीविषै रक्ष्यमाणं कुबेरदयितं भृशम्}
{यत्प्राप्य पुरुषो मर्त्योऽप्यमरत्वं नियच्छति}


\twolineshloka
{अचक्षुर्लभते चक्षुर्वृद्धो भवति वै युवा}
{इति ते कथयन्तिस्म ब्राह्मणा जम्भसाधकाः}


\twolineshloka
{ततः किरातास्तद्दृष्ट्वा प्रार्थयन्तो महीपते}
{विनेशुर्विषमे तस्मिन्ससर्पे गिरिगह्वरे}


\twolineshloka
{तथैव तव पुत्रोऽयं पृथिवीमेक इच्छति}
{मधु पश्यति संमोहात्प्रपातं नानुपश्यति}


\twolineshloka
{दुर्योधनो योद्धुमनाः समरे सव्यसाचिना}
{न च पश्यामि तेजोस्य विक्रमं वा तथाविधम्}


\twolineshloka
{एकेन रथमास्थाय पृथिवी येन निर्जिता}
{भीष्मद्रोणप्रभृतयः सन्त्रस्ताः साधुयायिनः}


\twolineshloka
{विराटनगरे भग्नाः किं तत्र तव दृश्यताम्}
{प्रतीक्षमाणो यो वीरः क्षमते वीक्षितं तव}


\twolineshloka
{द्रुपदो मत्स्यराजश्च सङ्कुद्धश्च धनञ्जयः}
{न शेषयेयुः समरे वायुयुक्ता इवाग्नयः}


\twolineshloka
{अङ्के कुरुष्व राजानं धृतराष्ट्र युधिष्ठिरम्}
{युध्यतोर्हि द्वयोर्युद्धे नैकान्तेन भवेञ्जयः}


\chapter{अध्यायः ६५}
\twolineshloka
{धृतराष्ट्र उवाच}
{}


\twolineshloka
{दुर्योधन विजानीहि यत्त्वां वक्ष्यामि पुत्रक}
{उत्पथं मन्यसे मार्गमनभिज्ञ इवाध्वगः}


\twolineshloka
{पञ्चानां पाण्डुपुत्राणां यत्तेजः प्रजिहीर्षसि}
{पञ्चानामिव भूतानां महतां लोकधारिणाम्}


\twolineshloka
{युधिष्ठिरं हि कौन्तेयं परं धर्ममिहास्थितम्}
{परां गतिमसंप्रेत्य न त्वं जेतुमिहार्हसि}


\twolineshloka
{भीमसेनं च कौन्तेयं यस्य नास्ति समो बले}
{रणान्तकं तर्जयसे महावातमिव द्रुमः}


\twolineshloka
{सर्वशस्त्रभृतां श्रेष्ठं मेरुं सिखरिणामिव}
{युधि गाण्डिवधन्वानं को नु युध्येत बुद्धिमान्}


\twolineshloka
{धृष्टद्युम्नश्च पाञ्चाल्यः कमिवाद्य न शातयेत्}
{शत्रुमध्ये शरान्मुञ्चन्देवराडशनीमिव}


\twolineshloka
{सात्यकिश्चापि दुर्धर्षः संमतोऽन्धकवृष्णिषु}
{ध्वंसयिष्यति ते सेनां पाण्डवेयहिते रतः}


\twolineshloka
{यः पुनः प्रतिमानेन त्रील्लोकानतिरिच्यते}
{तं कृष्णं पुण्डरीकाक्षं को नु युद्ध्येत बुद्धिमान्}


\twolineshloka
{एकतो ह्यस्य दाराश्च ज्ञातयश्च सबान्धवाः}
{आत्मा च पृथिवी चेयमेकतश्च धनञ्जयः}


\twolineshloka
{वासुदेवोऽपि दुर्धर्षो यतात्मा यत्र पाण्डवः}
{अविषह्यं पृथिव्यापि तद्बलं यत्र केशवः}


\twolineshloka
{तिष्ठ तात सतां वाक्ये सुहृदामर्थवादिनाम्}
{वृद्धं शान्तनवं भीष्मं तितिक्षस्व पितामहम्}


\twolineshloka
{मां च ब्रुवाणं शुश्रूष कुरूणामर्थदर्शिनम्}
{द्रोणं कृपं विकर्णं च महाराजं च बाह्लिकम्}


\twolineshloka
{एते ह्यपि यतैवाहं मन्तुमर्हसि तांस्तथा}
{सर्वे धर्मविदो ह्येते तुल्यस्नेहाश्च भारत}


\twolineshloka
{यत्तद्विराटनगरे सह भ्रातृभिरग्रतः}
{उत्सृज्य गाः सुसन्त्रस्तं बलं ते समशीर्यत}


\twolineshloka
{यच्चैव नगरे तस्मिञ्श्रूयते महदद्भुतम्}
{एकस्य च बहूनां च पर्याप्तं तन्निदर्शनम्}


\twolineshloka
{अर्जुनस्तत्तथाकार्षीत्किं पुनः सर्व एव ते}
{स भ्रातॄनभिजानीहि वृत्त्या तं प्रतिपादय}


\chapter{अध्यायः ६६}
\twolineshloka
{वैशंपायान उवाच}
{}


\twolineshloka
{दुर्योधने धार्तराष्ट्रे तद्वचो नाभिनन्दति}
{तूष्णींभूतेषु सर्वेषु समुत्तस्थुर्नरर्षभाः}


\twolineshloka
{उत्थितेषु महाराज पृथिव्यां सर्वराजसु}
{रहिते सञ्जयं राजा परिप्रष्टुं प्रचक्रमे}


\threelineshloka
{आशंसमानो विजयं तेषां पुत्रवशानुगः}
{आत्मनश्च परेषां च पाण्डवानां च निश्चयम् ॥धृतराष्ट्र उवाच}
{}


\twolineshloka
{गावल्गणे ब्रूहि नः सारफल्गुस्वेसनायं यावदिहास्ति किंचित्}
{त्वं पाण्डवानां निपुणं वेत्थ सर्वंकिमेषां ज्यायः किमु तेषां कनीयः}


\threelineshloka
{त्वमेतयोः सारवित्सर्वदर्शीधर्मार्थयोर्निपुणो निश्चयज्ञः}
{स मे पृष्टः सञ्जय ब्रूहि सर्वंयुध्यमानाः कतरेऽस्मिन्न सन्ति ॥सञ्जय उवाच}
{}


\twolineshloka
{न त्वां ब्रूयां रहिते जातु किंचि-दसूया हि त्वां प्रविशेत राजन्}
{आनयस्व पितरं महाव्रतंगान्धारीं च महिषीमाजमीढ}


\threelineshloka
{तो तेऽसूयां विनयेतां नरेन्द्रधर्मज्ञौ तौ निपुणौ निश्चयज्ञौ}
{तयोस्तु त्वां सन्निधौ तद्वदेयंकृत्स्नं मतं केशवपार्थयोर्यत् ॥वैशंपायन उवाच}
{}


\twolineshloka
{इत्युक्तेन च गान्धारी व्यासश्चात्राजगामह}
{आनीतौ विदुरेणेह सभां शीघ्रं प्रवेशितौ}


\threelineshloka
{ततस्तन्मतमाज्ञाय सञ्जयस्यात्मजस्य च}
{अभ्युपेत्य महाप्राज्ञः कृष्णद्वैपायनोऽब्रवीत् ॥व्यास उवाच}
{}


\twolineshloka
{संपृच्छते धृतराष्ट्रय सञ्जयआचक्ष्व सर्वं यावदेषोऽनुयुङ्क्ते}
{सर्वं यावद्वेत्थ तस्मिन्यथाव-द्याथातथ्यं वासुदेवेऽर्जुने च}


\chapter{अध्यायः ६७}
\twolineshloka
{सञ्जय उवाच}
{}


\twolineshloka
{अर्जुनो वासुदेवश्च धन्विनौ परमार्थितौ}
{कामादन्यत्र संभूतौ सर्वभावाय संमितौ}


\twolineshloka
{व्यामान्तरं समास्थय यथामुक्तं मनस्विनः}
{चक्रं तद्वासुदेवस्य मायया वर्तते विभो}


\twolineshloka
{सापह्नवं कौरवेषु पाण्डवानां सुसंमतम्}
{सारासारबलं ज्ञातुं तेजःपुञ्जावभासितम्}


\twolineshloka
{नरकं शम्बरं चैव कंसं चैद्यं च माधवः}
{जितवान्घोरसङ्काशान्क्रीडन्निव महाबलः}


\twolineshloka
{पृथिवीं चान्तरिक्षं च द्यां चैव पुरुषोत्तमः}
{मनसैव विशिष्टात्मा नयत्यात्मवशं वशी}


\twolineshloka
{भूयो भूयो हि यद्राजन्पृच्छसे पाण्डवान्प्रति}
{सारासारबलं ज्ञातुं तत्समासेन मे शृणु}


\twolineshloka
{एकतो वा जगत्कृत्स्नमेकतो वा जनार्दनः}
{सारतो जगतः कृत्स्नादतिरिक्तो जनार्दनः}


\twolineshloka
{भस्म कुर्याञ्जगदिदं मनसैव जनार्दनः}
{न तु कृत्स्नं जगच्छक्तं किञ्चित्कर्तुं जनार्दने}


\twolineshloka
{यतः सत्यं यतो धर्मो यतो ह्रीरार्जवं यतः}
{ततो भवति गोविन्दो यतः कृष्णस्ततो जयः}


\twolineshloka
{पृथिवीं चान्तरिक्षं च दिवं च पुरुषोत्तमः}
{विचेष्टयति भूतात्मा क्रीडन्निव जनार्दनः}


\twolineshloka
{स कृत्वा पाण्डवान्सत्रं लोकं संमोहयन्निव}
{अधर्मनिरतान्मूढान्दग्धुमिच्छति ते सुतान्}


\twolineshloka
{कालचक्रं जगच्चक्रं युगचक्रं च केशवः}
{आत्मयोगेन भगवान्परिवर्तयतेऽनिशम्}


\twolineshloka
{कालस्य च हि मृत्योश्च जङ्गमस्थावरस्य च}
{ईष्टे हि भगवानेकः सत्यमेतद्ब्रवीमि ते}


\twolineshloka
{ईशन्नपि महायोगी सर्वस्य जगतो हरिः}
{कर्माण्यारभते कर्तुं कीनाश इव दुर्बलः}


\twolineshloka
{तेन वञ्चयते लोकान्मायायोगेन केशवः}
{ये तमेव प्रपद्यन्ते न ते मुह्यन्ति मानवाः}


\chapter{अध्यायः ६८}
\twolineshloka
{धृतराष्ट्र उवाच}
{}


\threelineshloka
{कथं त्वं माधवं वेत्थ सर्वलोकमहेश्वरम्}
{कथमेनं न वेदाहं तन्ममाचक्ष्व सञ्जय ॥सञ्जय उवाच}
{}


\twolineshloka
{श्रृणु राजन्न ते विद्या मम विद्या न हीयते}
{विद्याहीनस्तमोध्वस्तो नाभिजानाति केशवम्}


\threelineshloka
{विद्यया तात जानामि त्रियुगं मधुसूदनम्}
{कर्तारमकृतं देवं भूतानां प्रभवाप्ययम् ॥धृतराष्ट्र उवाच}
{}


\threelineshloka
{गावल्गणेऽत्र का भक्तिर्या ते नित्या जनार्दने}
{यया त्वमभिजानासि त्रियुगं मधुसूदनम् ॥सञ्जय उवाच}
{}


\threelineshloka
{मायां न सेवे भद्रं ते न वृथाधर्ममाचरे}
{शुद्धभावं गतो भक्त्या शास्त्रद्वेद्मि जनार्दनम् ॥धृतराष्ट्र उवाच}
{}


\threelineshloka
{दुर्योधन हृषीकेशं प्रपद्यस्व जनार्दनम्}
{आप्तो नः सञ्जयस्तात शरणं गच्छ केशवम् ॥दुर्योधन उवाच}
{}


\threelineshloka
{भगवान्देवकीपुत्रो लोकांश्चेन्निहनिष्यति}
{प्रवदन्नर्जुने सख्यं नाहं गच्छेऽद्य केशवम् ॥धृतराष्ट्र उवाच}
{}


\threelineshloka
{अवाग्गान्धारि पुत्रस्ते गच्छत्येष सुदुर्मतिः}
{ईर्षुर्दुरात्मा मानी च श्रेयसां वचनातिगः ॥गान्धार्युवाच}
{}


\twolineshloka
{ऐश्वर्यकाम दुष्टात्मन्वृद्धानां शासनातिग}
{ऐश्वर्यजीतिते हित्वा पितरं मां च बालिश}


\threelineshloka
{वर्धयन्दुर्हृदां प्रीतिं मां च शोकेन वर्धयन्}
{निहतो भीमसेनेन स्मर्ताति वचनं पितुः ॥व्यास उवाच}
{}


\twolineshloka
{प्रियोऽसि राजन्कृष्णस्य धृतराष्ट्र निबोध मे}
{यस्य ते सञ्जयो दूतो यस्त्वां श्रेयसि योक्ष्यते}


\twolineshloka
{जानात्येष हृषीकेशं पुराणं यच्च वै परम्}
{शुश्रूषमाणमैकाग्र्यं मोक्ष्यते महतो भयात्}


\twolineshloka
{वैचित्रवीर्य पुरुषाः क्रोधहर्षसमावृताः}
{सिता बहुविधैः पाशैर्ये न तुष्टाः स्वकैर्धनैः}


\twolineshloka
{यमस्य वशमायान्ति काममूढाः पुनः पुनः}
{अन्धनेत्रा यथैवान्धा नीयमानाः स्वकर्मभिः}


\threelineshloka
{एष एकायनः पन्था येन यान्ति मनीषिणः}
{तं दृष्ट्वा मृत्युमत्येति महांस्तत्र न सञ्जति ॥धृतराष्ट्र उवाच}
{}


\threelineshloka
{अङ्ग सञ्जय मे शंस पन्थानमकुतोभयम्}
{येन गत्वा हृषीकेशं प्राप्नुयां सिद्धिमुत्तमाम् ॥सञ्जय उवाच}
{}


\twolineshloka
{नाकृतात्मा कृतात्मानं जातु विद्याञ्जनार्दनम्}
{आत्मनस्तु क्रियोपायो नान्यत्रेन्द्रियनिग्रहात्}


\twolineshloka
{इन्द्रियाणामुदीर्णानां कामत्यागोऽप्रमादतः}
{अप्रमादोऽविहिंसा च ज्ञानयोनिरसंशयम्}


\twolineshloka
{इन्द्रियाणां यमे यत्तो भव राजन्नतन्द्रितः}
{बुद्धिश्च ते मा च्युवतु नियच्छैनां यतस्ततः}


\twolineshloka
{एतज्ज्ञानं विदुर्विप्रा ध्रुवमिन्द्रियधारणम्}
{एतज्ज्ञानं च पन्थाश्च येन यान्ति मनीषिणः}


\twolineshloka
{अप्राप्यः केशवो राजन्निन्द्रियैरजितैर्नृभिः}
{आगमाधिगमाद्योगाद्वशी तत्त्वे प्रसीदति}


\chapter{अध्यायः ६९}
\twolineshloka
{धृतराष्ट्र उवाच}
{}


\threelineshloka
{भूयो मे पुण्डरीकाक्षं सञ्जयाचक्ष्व पृच्छतः}
{नामकर्मार्थवित्तात प्राप्नुयां पुरुषोत्तमम् ॥सञ्जय उवाच}
{}


\twolineshloka
{श्रुतं मे वासुदेवस्य नामनिर्वचनं शुभम्}
{यावत्तत्राभिजानेऽहमप्रमेयो हि केशवः}


\twolineshloka
{वसनात्सर्वभूतानां वसुत्वाद्देवयोनितः}
{वासुदेवस्ततो वेद्यो बृहत्त्वाद्विष्णुरुच्यते}


\twolineshloka
{मौनाद्ध्यानाच्च योगाच्च विद्दि भारत माधवम्}
{सर्वतत्त्वलयाच्चैव मधुहा मधुसूदनः}


\twolineshloka
{कृषिर्भूवाचकः शब्दो गश्च निर्वृतिवाचकः}
{विष्णुस्तद्भावयोगाच्च कृष्णो भवति सात्वतः}


\twolineshloka
{पुण्डरीकं परं धाम नित्यमक्षयमव्ययम्}
{तद्भावात्पुण्डरीकाक्षो दस्युत्रासाज्जानार्दनः}


\twolineshloka
{यतः सत्वं न च्यवते यच्च यत्वान्न हीयते}
{सात्वतः सात्वतस्तस्मादार्षभाद्वृषभेक्षणः}


\twolineshloka
{न जायते जनित्राऽयमजस्तस्मादनीकजित्}
{देवानां स्वप्रकाशत्वाद्दमाद्दामोदरो विभुः}


\twolineshloka
{हर्षात्सुखात्सुखैश्वर्याद्धृषीकेशत्वमश्रुते}
{बाहुभ्यां रोदसी बिभ्रन्महाबाहुरिति स्मृतः}


\twolineshloka
{अघो न क्षीयते जातु यस्मात्तस्मादधोक्षजः}
{नराणामयनाच्चापि ततो नारायणः स्मृतः}


\threelineshloka
{पूरणात्सदनाच्चापि ततोऽसौपुरुषोत्तमः}
{असतश्च सतश्चैव सर्वस्य प्रभवाप्ययात्}
{सर्वस्य च सदा ज्ञानात्सर्वमेतं प्रचक्षते}


\twolineshloka
{सत्ये प्रतिष्ठितः कृष्णः सत्यमत्र प्रतिष्ठितम्}
{सत्यात्सत्यं तु गोविन्दस्तस्मात्सत्योपि नामतः}


\twolineshloka
{विष्णुर्विक्रमणाद्देवो जयनाञ्जिष्णुरुच्यते}
{शाश्वतत्वादनन्तश्च गोविन्दो वेदनाद्भवाम्}


% Check verse!
अतत्त्वं कुरुते तत्त्वं तेन मोहयते प्रजाः
\twolineshloka
{एवंविधो धर्मनित्यो भगवान्मधुसूदनः}
{आगन्ता हि महाबाहुरानृशंस्यार्थमच्युतः}


\chapter{अध्यायः ७०}
\twolineshloka
{धृतराष्ट्र उवाच}
{}


\twolineshloka
{चक्षुष्मतां वै स्पृहयामि सञ्जयद्रक्ष्यन्ति ये वासुदेवं समीपे}
{विभ्राजमानं वपुषा परेणप्रकाशयन्तं प्रदिशो दिशश्च}


\twolineshloka
{ईरयन्तं भारतीं भारताना-मभ्यर्चनीयां शङ्करीं सृञ्जयानाम्}
{बुभूषद्भिर्गर्हणीयामनिन्द्यांपरासूनामग्रहणीयरूपाम्}


\twolineshloka
{समुद्यन्तं सात्वतमेकवीरंप्रणेतारमृषभं यादवानाम्}
{निहन्तारं क्षोभणं शात्रवाणांमुञ्चन्तं च द्विषतां वै यशांसि}


\twolineshloka
{द्रष्टारो हि कुरवस्तं समेतामहात्मानं शब्रुहणं वरेण्यम्}
{ब्रुवन्तं वाचमनृशंसरूपांवृष्णिश्रेष्ठं मोहयन्तं मदीयान्}


\twolineshloka
{ऋषिं सनातनतभं विपश्चितंवाचःसमुद्रं कलशं यतीनाम्}
{अरिष्टनेमिं गरुडं सुपर्णंहरिं प्रजानां भुवनस्य धाम}


\twolineshloka
{सहस्रशीर्षं पुरुषं पुराण-मनादिमध्यान्तमनन्तकीर्तिम्}
{शुक्रस्य धातारमजं च नित्यंपरं परेषां शरणं प्रपद्ये}


\twolineshloka
{त्रैलोक्यनिर्माणकरं जनित्रंदेवासुराणामथ नागरक्षसाम्}
{नराधिपानां विदुषां प्रधान-मिन्द्रानुजं तं शरणं प्रपद्ये}


\chapter{अध्यायः ७१}
\twolineshloka
{` जनमेजय उवच}
{}


\twolineshloka
{प्रयाते सञ्जये साधौ कौरवान्प्रति वै तदा}
{किं चक्रुः पाण्डवास्तत्र मम पूर्वपितामहाः}


\threelineshloka
{एतत्सर्वं द्विजश्रेष्ठ विस्तरं मम सत्तम}
{कथयस्व प्रयत्नेन श्रोतुमिच्छामि पण्डित ॥वैशंपायन उवाच}
{}


\twolineshloka
{सञ्जये प्रतियाते तु धर्मराजो युधिष्ठिरः}
{अर्जुनं भीमसेनं च माद्रीपुत्रौ च भारत}


\threelineshloka
{विराटं द्रुपदं चैव केकयानां महारथान्}
{अब्रुवीदुपसङ्गम्य शङ्खचक्रगदाधरम्}
{अभियाचामहे गत्वा प्रयातुं कुरुसंसदम्}


\twolineshloka
{यथा भीष्मेण द्रोणेन बाह्लीकेन च धीमता}
{अन्यैश्च कुरुभिः सार्धं न युध्येमहि संयुगे}


\twolineshloka
{एष नः प्रथमः कल्प एतन्निश्रेय उत्तमम्}
{एवमुक्ताः सुमनसस्तेऽभिजग्मुर्जनार्दनम्}


\twolineshloka
{पाण्डवैः सह राजानो मरुत्वन्त इवामराः}
{तदा च दुःसहाः सर्वे सदस्यास्ते नरर्षभाः}


\twolineshloka
{जनार्दनं समासाद्य कुन्तीपुत्रो युधिष्ठिरः}
{अब्रवीत्परवीरध्नं दाशार्हं पाण्डुनन्दनः ॥'}


\twolineshloka
{अयं स कालः संप्राप्तो मित्राणां मित्रवत्सल}
{न च त्वदन्यं पश्यामि यो न आपत्सु तारयेत्}


\twolineshloka
{त्वां हि माधवमाश्रित्य निर्भया मोघदर्पितम्}
{धार्तराष्ट्रं सहामात्यं स्वयं समनुयुङ्क्ष्महे}


\threelineshloka
{यथा हि सर्वास्वापत्सु पासि वृष्णीनरिन्दम्}
{तथा ते पाण्डवा रक्ष्याः पाह्यस्मान्महतो भयात् ॥श्रीभगवानुवाच}
{}


\threelineshloka
{अयमस्मि महाबाहो ब्रूहि यत्ते विवक्षितम्}
{करिष्यामि हि तत्सर्वं यत्त्वं वक्ष्यसि भारत ॥युधिष्ठिर उवाच}
{}


\threelineshloka
{श्रुतं ते धृतराष्ट्रस्य सपुत्रस्य चिकीर्षितम्}
{एतद्धि सकलं कृष्ण सञ्जयो मां यदब्रवीत् ॥ 5-71-14a`मृदुपूर्वं साममिश्रं सममुग्रं च माधव}
{न कतृं न्यायमास्थाय गर्हिताश्च ततो वयम् ॥'}


\twolineshloka
{तन्मतं धृतराष्ट्रस्य सोऽस्यात्मा विवृतान्तरः}
{यथोक्तं दूत आचष्टे वध्यः स्यादन्यथा ब्रुवन्}


\twolineshloka
{अप्रदानेन राज्यस्य शान्तिमस्मासु मार्गति}
{लुब्धः पापेन मनसा चरन्नसममात्मनः}


\twolineshloka
{यत्तद्द्वादशवर्षाणि वनेषु ह्युषिता वयम्}
{छद्मना शरदं चैकां धृतराष्ट्रस्य शासनात्}


\twolineshloka
{स्थाता नः समये तस्मिन्धृतराष्ट्र इति प्रभो}
{नाहास्म समयं कृष्ण तद्धि नो ब्राह्मणा विदुः}


\twolineshloka
{गृद्धो राजा धृतराष्ट्रः स्वधर्मं नानुपश्यति}
{वश्यत्वात्पुत्रगृद्धित्वान्मन्दस्यान्वेति शासनं}


\twolineshloka
{सुयोधनमते तिष्ठन्राजाऽस्मासु जनार्दन}
{मिथ्याचरति लुब्धः संश्चरन्हि प्रियमात्मनः}


\twolineshloka
{इतो दुःखतरं किं नु यदहं मातरं ततः}
{संविधातुं न शक्नोमि मित्राणां वा जनार्दन}


\twolineshloka
{काशिभिश्चेदिपाञ्चालैर्मत्स्यैश्च मधुसूदन}
{भवता चैव नाथेन पञ्च ग्रामा वृता मया}


\twolineshloka
{अविस्थलं वृकस्थलं माकन्दी वारणावतम्}
{अवसानं च गोविन्द कंचिदेवात्र पञ्चम्}


\twolineshloka
{पञ्च नस्तात दीयन्तां ग्रामा वा नगराणि वा}
{वसेम सहिता येषु मा च नो भरता नशन्}


\twolineshloka
{न च तानपि दुष्टात्मा धार्तराष्ट्रोऽनुमन्यते}
{स्वाम्यमात्मनि मत्वाऽसावतो दुःखतरं नु किं}


\twolineshloka
{कुले जातस्य वृद्धस्य परवित्तेषु गृद्ध्यतः}
{लोभः प्रज्ञानमाहन्ति प्रज्ञा हन्ति हता ह्रियं}


\twolineshloka
{ह्रीर्हता बाधते धर्मं धर्मो हन्ति हतः श्रियम्}
{श्रीर्हता पुरुषं हन्ति पुरुषस्याधं वधः}


\twolineshloka
{अधनाद्धि निवर्तन्ते ज्ञातयः सुहृदो द्विजाः}
{अपुष्पादफलाद्वृक्षाद्यथा कृष्ण पतत्रिणः}


\twolineshloka
{एतच्च मरणं तात उन्मत्तपतितादिव}
{ज्ञातयो विनिवर्तन्ते प्रेतसत्वादिवासवः}


\twolineshloka
{नातः पापीयसीं कांचिदवस्थां शम्बरोऽब्रवीत्}
{यत्र नैवाऽद्य न प्रातर्भोजनं प्रतिदृश्यते}


\twolineshloka
{धनमाहुः परं धर्मं धने सर्वं प्रतिष्ठितम्}
{जीवन्ति धनिनो लोके मृता ये त्वधना नराः}


\twolineshloka
{ये धनादपकर्षन्ति नरं स्वबलमास्थिताः}
{ते धर्ममर्थं कामं च प्रमथ्नन्ति नरं च तम्}


\twolineshloka
{एतामवस्थां प्राप्यैके मरणं वव्रिरे जनाः}
{ग्रामायैके वनायैके नाशायैके प्रवव्रजुः}


\twolineshloka
{उन्मादमेके पुष्यन्ति यान्त्यन्ये द्विषतां वशम्}
{दास्यमेके च गच्छन्ति परेषामर्थहेतुना}


\twolineshloka
{आपदेवास्य मरणात्पुरुषस्य गरीयसी}
{श्रियो विनाशस्तद्ध्यस्य निमित्तं धर्मकामयोः}


\twolineshloka
{यदस्य धर्म्यं मरणं शाश्वतं लोकवर्त्मवत्}
{समन्तात्सर्वभूतानां न तदत्येति कश्चन}


\twolineshloka
{न तथा बाध्यते कृष्ण प्रकृत्या निर्धनो जनः}
{यथा भद्रां श्रियं प्राप्य तया हीनः सुखैधितः}


\twolineshloka
{स तदाऽऽत्मापराधेन संप्राप्तो व्यसनं महत्}
{सेन्द्रान्गर्हयते देवान्नात्मानं च कथंचन}


\twolineshloka
{न चास्य सर्वशास्राणि प्रभवन्ति निबर्हणे}
{सोऽभिक्रुध्याति भृत्यानां सुहृदश्चाभ्यसूयति}


\twolineshloka
{तत्तदा मन्युरेवैति स भूयः संप्रमुह्यति}
{स मोहवशमापन्नः क्रूरं कर्म निषेवते}


\twolineshloka
{पापकर्मतया चैव सङ्करं तेन पुष्यति}
{सङ्करो नरकायैव सा काष्ठा पापकर्मणाम्}


\twolineshloka
{न चेत्प्रबुध्यते कृष्ण नरकायैव गच्छति}
{तस्य प्रबोधः प्रज्ञैव प्रज्ञाचक्षुस्तरिष्यति}


\twolineshloka
{प्रज्ञालाभे हि पुरुषः शास्त्राण्येवान्ववेक्षते}
{शास्त्रनिष्ठः पुनर्धर्मं तस्य ह्रीरङ्गमुत्तमम्}


\twolineshloka
{ह्रीमान्हि पापं प्रद्वेष्टि तस्य श्रीरभिवर्धते}
{श्रीमान्स यावद्भवति तावद्भवति पूरुषः}


\twolineshloka
{धर्मनित्यः प्रशान्तात्मा कार्ययोगवहः सदा}
{नाधर्मे कुरुते बुद्धिं न च पापे प्रवर्तते}


\twolineshloka
{अह्रीको वा विमूढो वा नैव स्त्री न पुनः पुमान्}
{नास्याधिकारो धर्मेऽस्ति यथा शूद्रस्तथैव सः}


\twolineshloka
{ह्रीमानवति देवांश्च पितॄनात्मानमेव च}
{तेनामृतत्वं व्रजति सा काष्ठा पुण्यकर्मणाम्}


\twolineshloka
{तदिदं मयि ते दृष्टं प्रत्यक्षं मधुसूदन}
{यथा राज्यात्परिभ्रष्टो वसामि वसतीरिमाः}


\twolineshloka
{ते वयं न श्रियं हातुमलं न्यायेन केनचित्}
{अत्र नो यतमानानां वधश्चेदपि साधु तत्}


\twolineshloka
{तत्र नः प्रथमः कल्पो यद्वयं ते च माधव}
{प्रशान्ताः समभूताश्च श्रियं तामश्रुवीमहि}


\twolineshloka
{तत्रैषा परमा काष्ठा रौद्रकर्मक्षयोदया}
{यद्वयं कौरवान्हत्वा तानि राष्ट्राण्यवाप्नुमः}


\twolineshloka
{ये पुनः स्युरसंबद्धा अनार्याः कृष्ण शत्रवः}
{तेषामप्यवधः कार्यः किंपुनर्ये स्युरीदृशाः}


\twolineshloka
{ज्ञातयश्चैव भूयिष्ठाः सहाया गुरवश्च नः}
{तेषां वधोऽतिपापीयान्किं नु युद्धेऽस्ति शोभनम्}


\twolineshloka
{पापः क्षत्रियधर्मोऽयं वयं च क्षत्रबन्धवः}
{स नः स्वधर्मोऽधर्मो वा वृत्तिरन्या विगर्हिता}


\twolineshloka
{शूद्रः करोति शुश्रूषां वैश्मा वै पण्यजीविकाः}
{वयं वधेन जीवामः कपालं ब्राह्मणैर्वृतम्}


\twolineshloka
{क्षत्रियः क्षत्रियं हन्ति मत्स्यो मत्स्येन जीवति}
{श्वा श्वानं हन्ति दाशार्ह पश्य धर्मो यथागतः}


\twolineshloka
{युद्धे कृष्ण कलिर्नित्यं प्राणाः सीदन्ति संयुगे}
{बलं तु नीतिमाधाय युध्ये जयपराजयौ}


\twolineshloka
{नात्मच्छन्देन भूतानां जीवितं मरणं तथा}
{नाप्यकाले सुखं प्राप्यं दुःखं वाऽपि यदूत्तम}


\twolineshloka
{एको ह्यपि बहून्हन्ति घ्नन्त्येकं बहवोऽप्युत}
{शूरं कापुरुषो हन्ति अयशस्वी यशस्विनम्}


\twolineshloka
{जयो नैवोभयोर्दृष्टो नोभयोश्च पराजयः}
{तथैवापचयो दृष्टो व्यपयाने क्षयव्ययौ}


\twolineshloka
{सर्वथा वृजिनं युद्धं को ध्नन्न प्रतिहन्यते}
{हतस्य च हृषीकेश समौ जयपराजयौ}


\twolineshloka
{पराजयश्च मरणान्मन्ये नैव विशिष्यते}
{यस्य स्याद्विजयः कृष्ण तस्यप्यपचयो ध्रुवम्}


\twolineshloka
{अन्ततो दयितं ध्नन्ति केचिदप्यपरे जनाः}
{तस्याङ्गबलहीनस्य पुत्रान्भ्रातॄनपश्यतः}


\twolineshloka
{निर्वेदो जीविते कृष्ण सर्वतश्चोपजायते}
{ये ह्येव धीरा ह्रीमन्त आर्याः करुणवेदिनः}


\twolineshloka
{त एव युद्धे हन्यन्ते यवीयान्मुच्यते जनः}
{हत्वाऽप्यनुशयो नित्यं परानपि जनार्दन}


\twolineshloka
{अनुबद्धश्च पापोऽत्र शेषश्चाप्यवशिष्यते}
{शेषो हि बलमासाद्य न शेषमनुशेषयेत्}


\twolineshloka
{सर्वोच्छेदे च यतते वैरस्यान्तविधित्सया}
{जयो वैरं प्रसृजति दुःखमास्ते पराजितः}


\twolineshloka
{सुखं प्रशान्तः स्वपिति हित्वा जयपराजयौ}
{जातवैरश्च पुरुषो दुःखं स्वपिति नित्यदा}


\twolineshloka
{अनिर्वृत्तेन मनसा ससर्प इव वेश्मनि}
{उत्सादयति यः सर्वं यशसा स विमुच्यते}


\twolineshloka
{अकीर्तिं सर्वभूतेषु शाश्वतीं स नियच्छति}
{न हि वैराणि शाम्यन्ति दीर्घकालधृतान्यपि}


\twolineshloka
{आख्यातारश्च विद्यन्ते पुमांश्चेद्विद्यते कुले}
{न चापि वैरं वैरेण केशव व्युपशाम्यति}


\twolineshloka
{हविषाऽग्निर्यथा कृष्ण भूय एवाभिवर्धते}
{अतोऽन्यथा नास्ति शान्तिर्नित्यमन्तरमन्ततः}


\threelineshloka
{अन्तरं लिप्समानानामयं दोषो निरन्तरः}
{पौरुषे यो हि बलवानाधिर्हृदयबाधनः}
{तस्य त्यागेन वा शान्तिर्मरणेनापि वा भवेत्}


\twolineshloka
{अथवा मूलघातेन द्विषतां मधुसूदन}
{फलनिर्वृत्तिरिद्धा स्यात्तन्नृशंसतरं भवेत्}


\twolineshloka
{या तु त्यागेन शान्तिः स्यात्तदृते वध एव सः}
{संशयाच्च समुच्छेदाद्द्विषतामात्मनस्तथा}


\twolineshloka
{न च त्यक्तुं तदिच्छामो न चेच्छामः कुलक्षयम्}
{अत्र या प्रणिपातेन शान्तिः सैव गरीयसी}


\twolineshloka
{सर्वथा यतमानानामयुद्धमभिकाङ्क्षताम्}
{सान्त्वे प्रतिहते युद्धं प्रसिद्धं नापराक्रमः}


\twolineshloka
{प्रतिघातेन सान्त्वस्य दारुणं संप्रवर्तते}
{तच्छुनामिव संपाते पण्डितैरुपलक्षितम्}


\twolineshloka
{लाङ्गूलचालनं क्ष्वेडा प्रतिवाचो विवर्तनम्}
{दन्तदर्शनमारावस्ततो युद्धं प्रवर्तते}


\twolineshloka
{तत्र यो बलवान्कृष्ण जित्वा सोत्ति तदामिषम्}
{एवमेव मनुष्येषु विशेषो नास्ति कश्चन}


\twolineshloka
{सर्वथा त्वेतदुचितं दुर्बलेषु बलीयसाम्}
{अनादरो विरोधश्च प्रणिपाती हि दुर्बलः}


\twolineshloka
{पिता राजा च वृद्धश्च सर्वथा मानमर्हति}
{तस्मान्मान्यश्च पूज्यश्च धृतराष्ट्रो जनार्दन}


\twolineshloka
{पुत्रस्नेहश्च बलवान्धृतराष्ट्रस्य माधव}
{स पुत्रवशमापन्नः प्रणिपातं प्रहास्यति}


\twolineshloka
{तत्र किं मन्यसे कृष्ण प्राप्तकालमनन्तरम्}
{कथमर्थाच्च धर्माच्च न हीयेमहि माधव}


\twolineshloka
{ईदृशेऽत्यर्थकृच्छ्रेऽस्मिन्कमन्यं मधुसूदन}
{उपसंप्रष्टुमर्हामि त्वामृते पुरुषोत्तम}


\threelineshloka
{प्रियश्च प्रियकामश्च गतिज्ञः सर्वकर्मणाम्}
{को हि कृष्णास्ति नस्त्वादृक्सर्वनिश्चयवित्सुहृत् ॥वैशंपायन उवाच}
{}


\twolineshloka
{एवमुक्तः प्रत्युवाच धर्मराजं जनार्दनः}
{उभयोरेव वामर्थे यास्यामि कुरुसंसदम्}


\twolineshloka
{शमं तत्र लभेयं चेद्युष्मदर्थमहापयन्}
{पुण्यं मे सुमहद्राजंश्चरितं स्यान्महाफलम्}


\threelineshloka
{मोचयेयं मृत्युपाशात्संरब्धान्कुरुसृञ्जयान्}
{पाण्डवान्धार्तराष्ट्रांश्च सर्वां च पृथिवीमिमाम् ॥युधिष्ठिर उवाच}
{}


\twolineshloka
{न ममैतन्मतं कृष्ण यत्त्वं यायाः कुरून्प्रति}
{सुयोधनः सूक्तमपि न करिष्यति ते वचः}


\twolineshloka
{समेतं पार्थिवं क्षत्रं दुर्योधनवशानुगम्}
{तेषां मध्यावतरणं तव कृष्ण न रोचये}


\threelineshloka
{न हि नः प्रीणयेद्द्रव्यं न देवत्वं कुतः सुखम्}
{न च सर्वामरैश्वर्यं तव द्रोहेण माधव ॥श्रीभगवानुवाच}
{}


\twolineshloka
{जानाम्येतां महाराज धार्तराष्ट्रस्य पापताम्}
{अवाच्यास्तु भविष्यामः सर्वलोके महीक्षितां}


\threelineshloka
{न चापि मम पर्याप्ताः सहिताः सर्वपार्थिवाः}
{क्रुद्धस्य संयुगे स्थातुं सिंहस्येवेतरे मृगाः}
{}


\twolineshloka
{अथ चेत्ते प्रवर्तेरन्मयि किंचिदसांप्रतम्}
{निर्दहेयं कुरून्सर्वानिति मे धीयते मतिः}


\twolineshloka
{न जातु गमनं पार्थ भवेत्तत्र निरर्थकम्}
{अर्थप्राप्तिः कदाचित्स्यादन्ततो वाप्यवाच्यता}


\threelineshloka
{` एवमुक्तः प्रत्युवाच धर्मराजो जनार्दनम्}
{भातॄणां समवेतानां सकाशे पुरुषोत्तमम् ' ॥युधिष्ठिर उवाच}
{}


\twolineshloka
{यत्तुभ्यं रोचते कृष्ण स्वस्ति प्राप्नुहि कौरवान्}
{कृतार्थं स्वस्तिमन्तं त्वां द्रक्ष्यामि पुनरागतम्}


\twolineshloka
{विष्वक्सेन कुरून्गत्वा भरताञ्शमयन्प्रभो}
{यथा सर्वे सुमनसः सह स्याम सुचेतसः}


\twolineshloka
{भ्राता चासि सखा चासि बीभत्सोर्मम च प्रियः}
{सौहृदेनाविशङ्ख्योऽसि स्वस्ति प्राप्नुहि भूतये}


\twolineshloka
{अस्मान्वेत्थ परान्वेत्थ वेत्थार्थान्वेत्थ भाषितुम्}
{यद्यदस्मद्धितं कृष्ण तत्तद्वाच्यः सुयोधनः}


\twolineshloka
{यद्यद्धर्मेण संयुक्तमुपपद्येद्धितं वचः}
{तत्तत्केशव भाषेथाः सान्त्वं वा यदि वेतरत्}


\chapter{अध्यायः ७२}
\twolineshloka
{श्रीभगवानुवाच}
{}


\twolineshloka
{सञ्जयस्य श्रुतं वाक्यं भवतश्च श्रुतं मया}
{सर्वं जानाम्यभिप्रायं तेषां च भवतश्च यः}


\twolineshloka
{तव धर्माश्रिता बुद्धिस्तेषां वैराश्रया मतिः}
{यदयुद्धेन लभ्येत तत्ते बहुमतं भवेत्}


\twolineshloka
{न चैवं नैष्ठिकं कर्म क्षत्रियस्य विशांपते}
{आहुराश्रमिणः सर्वे न भैक्षं क्षत्रियश्चरेत्}


\twolineshloka
{जयो वधो वा सङ्ग्रामे धात्राऽऽदिष्टः सनातनः}
{स्वधर्मः क्षत्रियस्यैष कार्पण्यं न प्रशत्यते}


\twolineshloka
{न हि कार्पण्यमास्थाय शक्या वृत्तिर्युधिष्ठिर}
{विक्रमस्व महाबाहो जहि शत्रून्परन्तप}


\twolineshloka
{अतिगृद्धाः कृतस्नेहा दीर्घकालं सहोषिताः}
{कृतमित्राः कृतबला धार्तराष्ट्राः परन्तप}


\twolineshloka
{न पर्यायोऽस्ति यत्साम्यं त्वयि कुर्युर्विशांपते}
{बलवत्तां हि मन्यन्ते भीष्मद्रोणकृपादिभिः}


\twolineshloka
{यावच्च मार्दवेनैतान्राजन्नुपचरिष्यसि}
{तावदेते हरिष्यन्ति तव राज्यमरिन्दम}


\twolineshloka
{नानुक्रोशान्न कार्पण्यान्न च धर्मार्थकारणात्}
{अलं कर्तुं धार्तराष्ट्रस्तव काममरिन्दम}


\twolineshloka
{एतदेव निमित्तं ते धार्तराष्ट्रो यथा त्वयि}
{नान्वतप्यत कोपेन तव कृत्वाऽपि दुष्करम्}


\twolineshloka
{पितामहस्य द्रोणस्य विदुरस्य च धीमतः}
{ब्राह्मणानां च साधूनां राज्ञश्च नगरस्य च}


\twolineshloka
{पश्यतां कुरुमुख्यानां सर्वेषामेव तत्त्वतः}
{दानशीलं मृदुं दान्तं धर्मशीलमनुव्रतम्}


\twolineshloka
{यत्त्वामुपधिना राजन्द्यूते वञ्चितवांस्तदा}
{न चापत्रपते तेन नृशंसः स्वेन कर्मणा}


\twolineshloka
{तथाशीलसमाचरे राजन्मा प्रणयं कृथाः}
{वध्यास्ते सर्वलोकस्य किं पुनस्तव भारत}


\twolineshloka
{वाग्भिस्त्वप्रतिरूपाभिरतुदत्त्वां सहानुजम्}
{श्लाघमानः प्रहृष्टः सन्भ्रातृभिः सह भाषते}


\twolineshloka
{एतावत्पाण्डवानां हि नास्ति किंचिदिह स्वकम्}
{नामधेयं च गोत्रं च तदप्येषां न शिष्यते}


\twolineshloka
{कालेन महता चैषां भविष्यति पराभवः}
{प्रकृतिं ते भजिष्यन्ति नष्टप्रकृतयो मयि}


\twolineshloka
{दुःशासनेन पापेन तदा द्यूते प्रवर्तिते}
{अनाथवत्तदा देवी द्रौपदी सुदुरात्मना}


\twolineshloka
{आकृष्य केशे रुदती सभायां राजसंसदि}
{भीष्मद्रोणप्रमुखतो गौरिति व्याहृता मुहुः}


\twolineshloka
{भवता वारिताः सर्वे भ्रातरो भीमविक्रमाः}
{धर्मपाशनिबद्धाश्च न किंचित्प्रतिपेदिरे}


\twolineshloka
{एताश्चान्याश्च परुषा वाचः स समुदीरयन्}
{श्लाघते ज्ञातिमध्ये स्म त्वयि प्रव्रजिते वनम्}


\twolineshloka
{ये तत्रासन्समानीतास्ते दृष्ट्वा त्वामनागसम्}
{अश्रुकण्ठा रुदन्तश्च सभायामासते सदा}


\twolineshloka
{न चैनमभ्यनन्दंस्ते राजानो ब्राह्मणैः सह}
{सर्वे दुर्योधनं तत्र निन्दन्ति स्म सभासदः}


\twolineshloka
{कुलीनस्य च या निन्दा वधो वाऽमित्रकर्शन}
{महागुणो वधो राजन्नु तु निन्दा कुजीविका}


\twolineshloka
{तदैव निहतो राजन्यदैव निरपत्रपः}
{निन्दितश्च महाराज पृथिव्यां सर्वराजभिः}


\twolineshloka
{ईषत्करो वधस्तस्य यस्य चारित्रमीदृशम्}
{प्रस्कुन्देन प्रतिस्तब्धश्छिन्नमूल इव द्रुमः}


\twolineshloka
{वध्यः सर्प इवानार्यः सर्वलोकस्य दुर्मतिः}
{जह्येन त्वममित्रघ्न मा राजन्विचिकित्सिथाः}


\twolineshloka
{सर्वथा त्वत्क्षमं चैतद्रोचते च ममानघ}
{यत्त्वं पितरि भीष्मे च प्रणिपातं समाचरेः}


\twolineshloka
{अहं तु सर्वलोकस्य गत्वा छेत्स्यामि संशयम्}
{येषामस्ति द्विधाभावो राजन्दुर्योधनं प्रति}


\twolineshloka
{मध्ये राज्ञामहं तत्र प्रातिपौरुषिकान्गुणान्}
{तव सङ्कीर्तयिष्यामि ये च तस्य व्यतिक्रमाः}


\twolineshloka
{ब्रुवतस्तत्र मे वाक्यं धर्मार्थसहितं हितम्}
{निशम्य पार्थिवाः सर्वे नानाजनपदेश्वराः}


\twolineshloka
{त्वयि संप्रतिपत्स्यन्ते धर्मात्मा सत्यवागिति}
{तस्मिंश्चाधिगमिष्यन्ति यथा लोभादवर्तत}


\twolineshloka
{गर्हयिष्यामि चैवैनं पौरजानपदेष्वपि}
{वृद्धबालानुपादाय चातुर्वर्ण्ये समागते}


\twolineshloka
{शमं वै याचमानस्त्वं नाधर्मं तत्र लप्स्यसे}
{कुरून्विगर्हयिष्यन्ति धृतराष्ट्रं च पार्थिवाः}


\twolineshloka
{तस्मिंल्लोकपरित्यक्ते किं कार्यमवशिष्यते}
{हते दुर्योधने राजन्यदन्यत्क्रियतामिति}


\twolineshloka
{यात्वा चाहं कुरून्सर्वान्युष्मदर्थमहापयन्}
{यतिष्ये प्रशमं कर्तुं लक्षयिष्ये च चेष्टितम्}


\twolineshloka
{कौरवाणां प्रवृत्तिं च गत्वा युद्धाधिकारिकाम्}
{निशम्य विनिवर्तिष्ये जयाय तव भारत}


\twolineshloka
{सर्वथा युद्धमेवाहमाशंसापि परैः सह}
{निमित्तानि हि सर्वाणि तथा प्रादुर्भवन्ति मे}


\twolineshloka
{मृगाः शकुन्ताश्च वदन्ति घोरंहस्त्यश्वमुख्येषु निशामुखेषु}
{घोराणि रूपाणि तथैव चाग्नि-र्वर्णान्बहून्पुष्यति घोररूपान्}


\twolineshloka
{मनुष्यलोकक्षयकृत्सुघोरोनो चेदनुप्राप्त इहान्तकः स्यात्}
{शस्त्राणि यन्त्रं कवचान्रथांश्चनागान्हयांश्च प्रतिपादयित्वा}


\twolineshloka
{योधाश्च सर्वे कृतनिश्चयास्तेभवन्तु हस्त्यश्वरथेषु यत्ताः}
{साङ््ग्रामिकं ते यदुपार्जनीयंसर्वं समग्रं कुरु तन्नरेन्द्र}


\twolineshloka
{दुर्योधनो न ह्यलमद्य दातुंजीवंस्तवैतन्नृपते कथंचित्}
{यत्ते पुरस्तादभवत्समृद्धंद्यूते हृतं पाण्डवमुख्य राज्यम्}


\chapter{अध्यायः ७३}
\twolineshloka
{भीम उवाच}
{}


\twolineshloka
{यथायथैव शान्तिः स्यात्कुरूणां मधुसूदन}
{तथातथैव भाषेथा मा स्म युद्धेन भीषयेः}


\twolineshloka
{अमर्षी जातसंरम्भः श्रेयोद्वेषी महामनाः}
{नोग्रं दुर्योधनो वाच्यः साम्नैवेनं समाचरेः}


\twolineshloka
{प्रकृत्या पापसत्वश्च तुल्यचेतास्तु दस्युभिः}
{ऐश्वर्यमदमत्तश्च कृतवैरश्च पाण्डवैः}


\twolineshloka
{अदीर्घदर्शी निष्ठूरी क्षेप्ता क्रूरपराक्रमः}
{दीर्घमन्युरनेयश्च पापात्मा निकृतिप्रियः}


\twolineshloka
{म्रियेतापि न भज्येत नैव जह्यात्स्वकं मतम्}
{तादृशेन शमः कृष्ण मन्ये परमदुष्करः}


\twolineshloka
{सुहृदामप्यवाचीनस्त्यक्तधर्मा प्रियानृतः}
{प्रतिहन्त्ये सुहृदां वाचश्चैव मनांसि च}


\twolineshloka
{स मन्युवशमापन्नः स्वभावं दुष्टमास्थितः}
{स्वभावात्पापमभ्येति तृणैश्छन्न इवोरगः}


\twolineshloka
{दुर्योधनो हि यत्सेनः सर्वथा विदितस्तव}
{यच्छीलो यत्स्वभावश्च यद्बलो यत्पराक्रमः}


\twolineshloka
{पुरा प्रसन्नाः कुरवः सहपुत्रास्तथा वयम्}
{इन्द्रज्येष्ठा इवाभूम मोदमानाः सबान्धवाः}


\twolineshloka
{दुर्योधनस्य क्रोधेन भरता मधुसूदन}
{धक्ष्यन्ते शिशिरापाये वनानीव हुताशनैः}


\twolineshloka
{अष्टादशेमे राजानः प्रख्यात मधुसूदन}
{ये समुच्चिच्छिदुर्ज्ञातीन्सुहृदश्च सबान्धवान्}


\twolineshloka
{असुराणां समृद्धानां ज्वलतामिव तेजसा}
{पर्यायकाले धर्मस्य प्राप्ते कलिरजायत}


\twolineshloka
{हैहयानामुदावर्तो नीपानां जनमेजयः}
{बहुलस्तालजङ्घानां कृमीणामुद्धतो वसुः}


\twolineshloka
{अजबिन्दुः सुवीराणां सुराष्ट्राणां रुषर्द्धिकः}
{अर्कजश्च बलीहानां चीनानां धौतमूलकः}


\twolineshloka
{हयग्रीवो विदेहानां वरयुश्च महौजसाम्}
{बाहुः सुन्दरवंशानां दीप्ताक्षाणां पुरूरवाः}


\twolineshloka
{सहजश्चेदिमत्स्यानां प्रवीराणां वृषध्वजः}
{धारणश्चन्द्रवत्सानां मुकुटानां विगाहनः}


\twolineshloka
{शमश्च नन्दिवेगानामित्येते कुलपांसनाः}
{युगान्ते कृष्ण संभूताः कुलेषु पुरुषाधमाः}


\twolineshloka
{अप्ययं नः कुरूणां स्याद्युगान्ते कालसंभृतः}
{दुर्योधनः कुलाङ्गारो जघन्यः पापपूरुषः}


\twolineshloka
{तस्मान्मृदु शनैर्ब्रूया धर्मार्थसहितं हितम्}
{कामानुबद्धं बहुलं नोग्रमुग्रपराक्रमम्}


\twolineshloka
{अपि दुर्योधनं कृष्ण सर्वे वयमधश्चराः}
{नीचैर्भूत्वाऽनुयास्यामो मा स्म नो भरता नशन्}


\twolineshloka
{अप्युदासीनवृत्तिः स्याद्यथा नः कुरुभिः सह}
{वासुदेव तथा कार्यं न कुरूननयः स्पृशेत्}


\twolineshloka
{वाच्यः पितामहो वृद्धो ये च कृष्ण समासदः}
{भ्रातॄणामस्तु सौभ्रात्रं धार्तराष्ट्रः प्रशाम्यताम्}


\twolineshloka
{अहमेतद्ब्रवीम्येवं राजा चैव प्रशंसति}
{अर्जुनो नैव युद्धार्थी भूयसी हि दयाऽर्जुने}


\chapter{अध्यायः ७४}
\twolineshloka
{वैशंपायन उवाच}
{}


\twolineshloka
{एतछ्रुत्वा महाबाहुः केशवः प्रहसन्निव}
{अभूतपूर्वं भीमस्य मार्दवोपहितं वचः}


\twolineshloka
{गिरेरिव लघुत्वं तच्छीतत्वमिव पावके}
{मत्वा रामानुजः शौरिः शार्ङ्गधन्वा वृकोदरम्}


\threelineshloka
{सन्तेजसंस्तदा वाग्भिर्भातरिश्वेव पावकम्}
{उवाच भीममासीनं कृपयाऽभिपरिप्लुतम् ॥श्रीभगवानुवाच}
{}


\twolineshloka
{त्वमन्यदा भीमसेन युद्धमेव प्रशंससि}
{वधामिनन्दिनः क्रूरान्धार्तराष्ट्रन्मिमर्दिषुः}


\twolineshloka
{न च स्वपिषि जागर्षि न्युब्जः शेषे परन्तप}
{घोरामशान्तां रुशतीं सदा वाचं प्रभाषसे}


\twolineshloka
{निःश्वसन्नग्निवत्तेन सन्तप्तः स्वेन मन्युना}
{अप्रशान्तमना भीम सधूम इव पावकः}


\twolineshloka
{एकान्ते निःश्वसञ्शेपे भारार्त इव दुर्बलः}
{अपि त्वां केचिदुन्मत्तं मन्यन्तेऽतद्विदो जनाः}


\twolineshloka
{आरुज्य वृक्षान्निर्मूलान्गजः परिरुजन्निव}
{निघ्नन्पद्भिः क्षितिं भीम निष्टनन्परिधावसि}


\twolineshloka
{नास्मिञ्जने न रमसे रहः क्षिपसि पाण्डव}
{नान्यं निशि दिवा चापि कदाचिदभिनन्दसि}


\twolineshloka
{अकस्मात्स्मयमानश्च रहस्यास्से रुदन्निव}
{जान्वोर्मूर्धानमाधाय चिरमास्ते प्रमीलितः}


\twolineshloka
{भ्रुकिटिं च पुनः कुर्वन्नोष्ठौ च विदशन्निव}
{अभीक्ष्णं दृश्यसे भीम सर्वं तन्मन्युकारितम्}


\twolineshloka
{यथा तपुरस्तात्सविता दृश्यते शुक्रमुच्चरन्}
{यथा च पश्चान्निर्मुक्तो ध्रुवं पर्येति रश्मिवान्}


\twolineshloka
{तथा सत्यं ब्रवीम्येतन्नास्ति तस्य व्यतिक्रमः}
{हन्ताहं गदयाभ्येत्य दुर्योधनममर्पणम्}


\twolineshloka
{इति स्म मध्ये भ्रातृणां सत्येनालभसे वदाम्}
{तस्य ते प्रशमे बुद्धिर्ध्रियतेऽद्य परन्तप}


\twolineshloka
{अहो युद्धाभिकाङ्क्षाणां युद्धकाल उपस्थिते}
{चेतांसि विप्रतीपानि यत्त्वां भीभीम विन्दति}


\twolineshloka
{अहो पार्थ निमित्तानि विपरीतानि पश्यसि}
{स्वप्नान्ते जागरान्ते च तस्मात्प्रशममिच्छसि}


\twolineshloka
{अहो नाशंससे किंचित्पुंस्त्वं क्लीब इवात्मनि}
{कश्मलेनाभिपन्नोऽसि तेन ते विकृतं मनः}


\twolineshloka
{उद्वेपते ते हृदयं मनस्ते प्रतिसीदति}
{ऊरुस्तम्भगृहीतोऽसि तस्मात्प्रशममिच्छसि}


\twolineshloka
{अनित्यं किल मर्त्यस्य पार्थ चित्तं चलाचलम्}
{वातवेगप्रचलिता अष्ठीला शाल्मलेरिव}


\twolineshloka
{तवैषा विकृता बुद्धिर्गवां वागिव मानुषी}
{मनांसि पाण्डुपुत्राणां मञ्जयत्यप्लवानिव}


\twolineshloka
{इदं मे महादाश्चर्यं पर्वतस्येव सर्पणम्}
{यदीदृशं प्रभाषेथा भीमसेनासमं वचः}


\twolineshloka
{स दृष्ट्वा स्वानि कर्माणि कुले जन्म च भारत}
{उत्तिष्ठस्व विषादं मा कृथा वीर स्थिरो भव}


\twolineshloka
{न चैतदनुरूपं ते यत्ते ग्लानिररिन्दम}
{यदोजसा न लभते क्षत्रियो न तदश्रुते}


\chapter{अध्यायः ७५}
\twolineshloka
{वैशंपायन उवाच}
{}


\threelineshloka
{तथोक्तो वासुदेवेन नित्यमन्युरमर्षणः}
{सदश्ववत्समाधावद्बभाषे तदनन्तरम् ॥भीमसेन उवाच}
{}


\twolineshloka
{अन्यथा मां चिकीर्षन्तमन्यथा मन्यसेऽच्युत}
{प्रणीतभावमत्यर्थं युधि सत्यपराक्रमम्}


\twolineshloka
{वेत्सि दाशार्ह सत्यं मे दीर्घकालं सहोषितः}
{उत वा मां न जानासि प्लवन्हृद इवाप्लवे}


\twolineshloka
{तस्मादनभिरुपाभिर्वाग्भिर्मां त्वं समर्च्छसि}
{कथं हि भीमसेनं मां जानन्कश्चन माधव}


\twolineshloka
{ब्रूयादप्रतिरूपाणि यथा मां वक्तमर्हसि}
{तस्मादिदं प्रवक्ष्यामि वचनं वृष्णिनन्दन}


\twolineshloka
{आत्मनः पौरुषं चैव बलं च न समं परैः}
{सर्वथाऽनार्यकर्मैतत्प्रशंसा स्वयमात्मनः}


\twolineshloka
{अतिवादापविद्धस्तु वक्ष्यामि बलमात्मनः}
{पश्येमे रोदसी कृष्ण ययोरासन्निमाः प्रजाः}


\twolineshloka
{अचले चाप्रतिष्ठे चाप्यनन्ते सर्वमातरौ}
{यदीमे सहसा क्रुद्धे समेयातां शिले इव}


\twolineshloka
{अहमेते निगृह्णीयां बाहुभ्यां सचसाचरे}
{पश्यैतदन्तरं बाह्वोर्महापरिघयोरिव}


\twolineshloka
{य एतत्प्राय मुच्येत न तं पश्यामि पुरूषम्}
{हिमवांश्च समुद्रश्च वज्री वा बलमित्स्वयम्}


\twolineshloka
{मयाभिपन्नं त्रायेरन्बलमास्थाय न त्रयः}
{युद्धार्हान्क्षत्रियान्सर्वान्पाण्डवेष्वाततायिनः}


\twolineshloka
{अधः पादतलेनैतानधिष्ठास्यामि भूतले}
{न हि त्वं नाभिजानासि मम विक्रममच्युत}


\twolineshloka
{यथा मया विनिर्जित्य राजानो वशगाः कृताः}
{अथ चेन्मां न जानासि सूर्यस्येवोद्यतः प्रभाम्}


\twolineshloka
{विगाढे युधि संबाधे वेत्स्यसे मां जनार्दन}
{परुषैराक्षिपसि किं व्रणं सूच्येव चानघ}


\twolineshloka
{यथामति ब्रवीम्येतद्विद्धि मामधिकं ततः}
{द्रष्टासि युधि संबाधे प्रवृत्ते वैशसेऽहनि}


\twolineshloka
{मया प्रणुन्नान्मातङ्गात्नथिनः सादनस्तथा}
{तथा नरानभिक्रुद्धं निघ्नन्तं क्षत्रियर्षभान्}


\twolineshloka
{द्रष्टा मां त्वं च लोकश्च विकर्षन्तं वरान्वरान्}
{न मे सीदन्ति गात्राणि न ममोद्वेपते मनः}


\threelineshloka
{सर्वलोकादभिक्रुद्धान्न भयं विद्यते मम}
{किं तु सौहृदमेवैतत्कृपया मधुसूदन}
{सर्वांस्तितिक्षे संक्लेशान्मा स्म नो भरता नशन्}


\chapter{अध्यायः ७६}
\twolineshloka
{श्रीभगवानुवाच}
{}


\twolineshloka
{भावं जिज्ञासमानोऽहं प्रणयादिदमब्रवम्}
{न चाक्षेपान्न पाण्डित्यान्न क्रोधान्न विवक्षया}


\twolineshloka
{वेदाहं तव माहात्म्यसुत ते वेद यद्बलम्}
{उत ते वेद कर्माणि न त्वां परिभवाम्यहम्}


\twolineshloka
{यथा चात्मनि कल्याणं संभावयसि पाण्डव}
{सहस्रगुणमप्येतत्त्वयि संभावयाम्यहम्}


\twolineshloka
{यादृशे च कुले जन्म सर्वराजाभिपूजिते}
{बन्धुभिश्च सुहृद्भिश्च भीम त्वमसि तादृशः}


\twolineshloka
{जिज्ञासन्तो हि धर्मस्य सन्दिग्धस्य वृकोदर}
{पर्यायं नाध्यवस्यन्ति देवमानुषयोर्जनाः}


\twolineshloka
{स एव हेतुर्भूत्वा हि पुरुषस्यार्थसिद्धिषु}
{विनाशेऽपि स एवास्य सन्दिग्धं कर्म पौरुषम्}


\twolineshloka
{अन्यथा परिदृष्टानि कविभिर्दोवदर्शिभिः}
{अन्यथा परिवर्तन्ते वेगा इव नभस्वतः}


\twolineshloka
{सुमन्त्रितं सुनीतं च न्यायतश्चोपपादितम्}
{कृतं मानुष्यकं कर्म दैनेनापि विरुद्ध्यते}


\twolineshloka
{दैवमप्यकृतं कर्म पौरुषेण विहन्यते}
{शीतमुष्णं तथा वर्षं क्षुत्पिपासे च भारत}


\twolineshloka
{यदन्यद्दिष्टभावस्य पुरुषस्य स्वयं कृतम्}
{तस्मादनुपरोधश्च विद्यते तत्र लक्षणम्}


\twolineshloka
{लोकस्य नान्यतो वृत्तिः पाण्डवान्यत्र कर्मणः}
{एवंबुद्धिः प्रवर्तेत फलं स्यादुभयान्वये}


\twolineshloka
{य एवं कृतबुद्धिः स कर्मस्वेव प्रवर्तते}
{नासिद्धो व्यथते तस्य न सिद्धौ हर्षमश्रुते}


\twolineshloka
{तत्रेयमनुमात्रा मे भीमसेन विवक्षिता}
{नैकान्तसिद्धिर्वक्तव्या शत्रुभिः सह संयुगे}


\twolineshloka
{नातिप्रहीणरश्मिः स्यात्तथा भावविपर्यये}
{विषादमर्च्छेद् ग्लानिं वाप्येतमर्थं ब्रवीमिते}


\twolineshloka
{श्वोभूते धृतराष्ट्रस्य समीपं प्राप्य पाण्डव}
{यतिष्ये प्रशमं कर्तुं युष्मदर्थमहापयन्}


\twolineshloka
{शमं चेत्ते करिष्यन्ति ततोऽनन्तं यशो मम}
{भवतां कच कृतः कामस्तेषां च श्रेय उत्तमम्}


\twolineshloka
{ते चेदभिनिवेक्ष्यन्ते नाभ्युपैष्यन्ति मे वचः}
{कुरवो युद्धमेवात्र घोरं कर्म भविष्यति}


\twolineshloka
{अस्मिन्युद्धे भीमसेन त्वयि भारः समाहितः}
{धूरर्जुनेन धार्या स्याद्वेढव्य इतरो जनः}


\twolineshloka
{अहं हि यन्ता बीभत्सोर्भविता संयुगे सति}
{धनञ्जयस्यैष कामो न हि युद्धं न कामये}


\twolineshloka
{तस्मादाशङ्कमानोऽहं वृकोदर मतिं तव}
{गदतः क्लीबया वाचा तेजस्ते समदीदिपम्}


\chapter{अध्यायः ७७}
\twolineshloka
{अर्जुन उवाच}
{}


\twolineshloka
{उक्तं युधिष्ठिरेणैव यावद्वाच्यं जनार्दन}
{तव वाक्यं तु मे श्रुत्वा प्रतिभाति परन्तप}


\twolineshloka
{नैव प्रशममत्र त्वं मन्यसे सुकरं प्रभो}
{लोभाद्वा धृतराष्ट्रस्य दैन्याद्वा समुपस्थितात्}


\twolineshloka
{अफलं मन्यसे वाऽपि पुरुषस्य पराक्रमम्}
{न चान्तरेण कर्माणि पौरुषेण बलोदयः}


\twolineshloka
{तदिदं भाषितं वाक्यं तथाचन तथैव तत्}
{न चैतदेवं द्रष्टव्यमसाध्यमपि किंचन}


\twolineshloka
{किं चैतन्मन्यसे कृच्छ्रमस्माकमवसादकम्}
{कुर्वन्ति तेषां कर्माणि येषां नास्ति फलोदयः}


\twolineshloka
{संपाद्यमानं सम्यक्व स्यात्कर्म सफलं प्रभो}
{स तथा कृष्ण वर्तस्व यथा शर्म भवेत्परैः}


\twolineshloka
{पाण्डवानां कुरूणां च भवान्नः प्रथमः सुहृत्}
{सुराणामसुराणां च यथा वीर प्रजापतिः}


\twolineshloka
{कुरूणां पाण्डवानां च प्रतिपत्स्व निरामयम्}
{अस्मद्धितमनुष्ठानं मन्ये तव न दुष्करम्}


\twolineshloka
{एवं च कार्यतामेति कार्यं तव जनार्दन}
{गमनादेवमेव त्वं करिष्यसि जनार्दन}


\twolineshloka
{चिकीर्षितमथान्यत्ते तस्मिन्वीर दुरात्मानि}
{भविष्यति च तत्सर्वं यथा तव चिकीर्षितम्}


\twolineshloka
{शर्म तैः सह वा नोऽस्तु तव वा यच्चिकीर्षितम्}
{विचार्यमाणो यः कामस्तव कृष्ण स नो गुरुः}


\twolineshloka
{न स नार्हति दुष्टात्मा वधं ससुतबान्धवः}
{येन धर्मसुते दृष्टा न सा श्रीरुपमर्षिता}


\twolineshloka
{यच्चाप्यपश्यतोपायं धर्मिष्ठं मधुसूदन}
{उपायेन नृशंसेन हृता दुर्द्यूतदेविना}


\twolineshloka
{कथं हि पुरुषो जातः क्षत्रियेषु धनुर्धरः}
{समाहूतो निवर्तेत प्राणत्यागेऽप्युपस्थिते}


\twolineshloka
{अधर्मेण जितान्दृष्ट्वा वने प्रवृजितांस्तथा}
{वध्यतां मम वार्ष्णेय निर्गतोऽसौ सुयोधनः}


\twolineshloka
{न चैतदद्भुतं कृष्ण मित्रार्थे यच्चिकीर्षसि}
{क्रिया कथंच मुख्या स्यान्मृदुना चेतरेण वा}


\twolineshloka
{अथवा मन्यसे ज्यायान्वधस्तेषामनन्तरम्}
{तदेव क्रियतामाशु न विचार्यमतस्त्वया}


\twolineshloka
{जानासि हि यथैतेन द्रौपदी पापबुद्धिना}
{परिक्लिष्टा सभामध्ये तच्च तस्योपमर्षितम्}


\twolineshloka
{स नाम सम्यग्वर्तेत पाण्डवेष्विति माधव}
{न मे सञ्जायते बुद्धिर्बीजमुप्तमिवोषरे}


\twolineshloka
{तस्माद्यन्मन्यसे युक्तं पाण्डवानां हितं च यत्}
{तथाऽऽशु कुरु वार्ष्णेय यन्नः कार्यमनन्तरम्}


\chapter{अध्यायः ७८}
\twolineshloka
{श्रीभगवानुवाच}
{}


\threelineshloka
{एवमेतन्महाबाहो यथा वदसि पाण्डव}
{पाण्डवानां कुरूणां च प्रतिपत्स्ये निरामयम्}
{सर्वं त्विदं ममायत्तं बीभत्सो कर्मणोर्द्वयोः}


\twolineshloka
{क्षेत्रं हि रसवच्छुद्धं कर्मणैवोपपादितम्}
{ऋते वर्षान्न कौन्तेय जातु निर्वर्तयेत्फलम्}


\twolineshloka
{तत्र वै पौरुषं ब्रूयुरासेकं यत्र कारितम्}
{तत्र चापि ध्रुवं पश्येच्छोषणं दैवकारितम्}


\twolineshloka
{तदिदं निश्चित्तं बुद्ध्या पूर्वैरपि महात्मभिः}
{दैवे च मानुषे चैव संयुक्तं लोककारणम्}


\twolineshloka
{अहं हि तत्करिष्यामि परं पुरुषकारतः}
{देवं तु न मया शक्यं कर्म कर्तुं कथंचन}


\twolineshloka
{स हि धर्मं च लोकं च त्यक्त्वा चरति दुर्मतिः}
{न हि सन्तप्यते तेन तथारूपेण कर्मणा}


\twolineshloka
{तथापि बुद्धिं पापिष्ठां वर्धयन्त्यस्य मन्त्रिणः}
{शकुनिः सूतपुत्रश्च भ्राता दुःशासनस्तथा}


\twolineshloka
{स हि त्यागेन राज्यस्य न शमं समुपैष्यति}
{अन्तरेण वधं पार्थ सानुबन्धः सुयोधनः}


\twolineshloka
{न चापि प्रणिपातेन त्यक्तुमिच्छति धर्मराट्}
{याच्यमानश्च राज्यं स न प्रजास्यति दुर्मतिः}


\twolineshloka
{न तु मन्ये स तद्वाच्यो यद्युधिष्ठिरशातनम्}
{उक्तं प्रयोजनं यत्तु धर्मराजेन भारत}


\twolineshloka
{तथा पापस्तु तत्सर्वं न करिष्यति कौरवः}
{तस्मिंश्चाक्रियमाणेऽसौ लोके वध्यो भविष्यति}


\twolineshloka
{मम चापि स वध्यो हि जगतश्चापि भारत}
{तेन कौमारके यूयं सर्वे विप्रकृताः सदा}


\twolineshloka
{विप्रलुप्तं च वो राज्यं नृशंसेन दुरात्मना}
{न चोपशाम्यते पापः श्रियं दृष्ट्वा युधिष्ठिरे}


\twolineshloka
{असकृच्चाप्यहं तेन त्वत्कृते पार्थ भेदितः}
{न मया तद्गृहीतं च पापं तस्य चिकीर्षेतम्}


\twolineshloka
{जानासि हि महाबाहो त्वमप्यस्य परं मतम्}
{प्रियं चिकीर्षमाणं च धर्मराजस्य मामपि}


\twolineshloka
{संजानंस्तस्य चात्मानं मम चैव परं मतम्}
{अजानन्निव मां कस्मादर्जुनाद्याभिशङ्कसे}


\twolineshloka
{यच्चापि परमं दिव्यं तच्चाप्यनुगतं त्वया}
{विधानं विहितं पार्थ कथं शर्म भवेत्परैः}


\twolineshloka
{यत्तु वाचा मया शक्यं कर्मणा वाऽपि पाण्डव}
{करिष्ये तदहं पार्थ न त्वाशंसे शमं परैः}


\twolineshloka
{कथं गोहरणे ह्युक्तो नैतच्छर्म तथा हितम्}
{याच्यमानो हि भीष्मेण संवत्सरगतेऽध्वनि}


\twolineshloka
{तदैव ते पराभूता यदा सङ्कल्पितास्त्वया}
{लवशः क्षणशश्चापि न च तुष्टः सुयोधनः}


\twolineshloka
{सर्वथा तु मया कार्यं धर्मराजस्य शासनम्}
{विभाव्यं तस्य भूयश्च कर्म पापं दुरात्मनः}


\chapter{अध्यायः ७९}
\twolineshloka
{नकुल उवाच}
{}


\twolineshloka
{उक्तं बहुविधं वाक्यं धर्मराजेन माधव}
{धर्मज्ञेन वदान्येन श्रुतं चैव हि तत्त्वया}


\twolineshloka
{मतमाज्ञाय राज्ञश्च भीमसेनेन माधव}
{संशमो बाहुवीर्य च ख्यापितं माधवात्मनः}


\twolineshloka
{तथैव फाल्गुनेनापि यदुक्तं तत्त्वया श्रुतम्}
{आत्मनश्च मतं वीर कथितं भवताऽसकृत्}


\twolineshloka
{सर्वमेतदतिक्रम्य श्रुत्वा परमतं भवान्}
{यत्प्राप्तकालं मन्येथास्तत्कुर्याः पुरुषोत्तम}


\twolineshloka
{तस्मिंरतस्मिन्निमित्ते हि मतं भवति केशव}
{प्राप्तकालं मनुष्येण क्षमं कार्यमरिन्दम}


\twolineshloka
{अन्यथा चिन्तितो ह्यर्थः पुनर्भवति सोऽन्यथा}
{अनित्यमतयो लोके नराः पुरुषसत्तम}


\twolineshloka
{अन्यथाबुद्धयो ह्यासन्नस्मासु वनवासिषु}
{अदृश्येष्वन्यथा कृष्ण दृश्येषु पुनरन्यथा}


\twolineshloka
{अस्माकमपि वार्ष्णेय वने विचरतां तदा}
{न तथा प्रणयो राज्ये यथा संप्रति वर्तते}


\twolineshloka
{निवृत्तवनवासान्नः श्रुत्वा वीर समागताः}
{अक्षौहिण्यो हि सप्ते मास्त्वत्प्रसादाज्जनार्दन}


\twolineshloka
{इमान्हि पुरुषव्याघ्रानचिन्त्यबलपौरुषात्}
{आत्तशस्त्रान्रणे दृष्ट्वा न व्यथेदिह कः पुमान्}


\twolineshloka
{स भवान्कुरुमध्ये तं सान्त्वपूर्वं भयोत्तरम्}
{ब्रूयाद्वाक्यं यथा मन्दो न व्यथेत सुयोधनः}


\twolineshloka
{युधिष्ठिरं भीमसेनं बीभत्सुं चापराचितम्}
{सहदेवं च मां चैव त्वां च रामं च केशव}


\twolineshloka
{सात्यकिं च महावीर्यं विराटं च महात्मजम्}
{द्रुपदं च महामात्यं धृष्टद्युम्नं च माधव}


\twolineshloka
{काशिराजं च विक्रान्तं धृष्टकेतुं च चेदिपम्}
{मांसशोणितभृन्मर्त्यः प्रतियुध्येत को युधि}


\twolineshloka
{स भवान्गमनादेव साधयिष्यत्यसंशयम्}
{इष्टमर्थं महाबाहो धर्मराजस्य केवलम्}


\twolineshloka
{विदुरश्चैव भीष्मश्च द्रोणश्च सहबाह्लिकः}
{श्रेयः समर्था विज्ञातुमुच्यमानास्त्वयाऽनघ}


\twolineshloka
{ते चैनमनुनेष्यन्ति धृतराष्ट्रं जनाधिपम्}
{तं च पापसमाचारं सहामात्यं सुयोधनम्}


\twolineshloka
{श्रोता चार्थस्य विदुरस्त्वं च वक्ता जनार्दन}
{कमिवार्थं निवर्तन्तं स्थापयेतां न वर्त्मनि}


\chapter{अध्यायः ८०}
\twolineshloka
{सहदेव उवाच}
{}


\twolineshloka
{यदेतत्कथितं राज्ञा धर्म एष सनातनः}
{यथा च युद्धमेव स्यात्तथा कार्यमरिन्दम}


\twolineshloka
{यदि प्रसममिच्छेयुः कुरवः पाण्डवैः सह}
{तथापि युद्धं दाशार्ह योजयेथाः सहैव तैः}


\twolineshloka
{कथं नु दृष्ट्वा पाञ्चालीं तथा कृष्ण सभागताम्}
{अवधेन प्रशाम्येत मम मन्युः सुयोधने}


\twolineshloka
{यदि भीमार्जुनौ कृष्ण धर्मराजश्च धार्मिकः}
{धर्ममुत्सृज्य तेनाहं योद्धुमिच्छामि संयुगे}


\threelineshloka
{ब्रूहि मद्वचनं कृष्ण सुयोधनमपण्डितम्}
{कृच्छ्रे वने वा वस्तव्यं पुरे वा नागसाह्वये ॥सात्यकिरुवाच}
{}


\twolineshloka
{सत्यमाह महाबाहो सहदेवो महामतिः}
{दुर्योधनवधे शान्तिस्तस्य कोपस्य मे भवेत्}


\twolineshloka
{न जानासि यथा दृष्ट्वा चीराजिनधरान्वने}
{तवापि मन्युरुद्धूतो दुःखितान्प्रेक्ष्य पाण्डवान्}


\threelineshloka
{तस्मान्माद्रीसुतः शूरो यदाह रणकर्कशः}
{वचनं सर्वयोधानां तन्मतं पुरुषोत्तम ॥वैशंपायन उवाच}
{}


\twolineshloka
{एवं वदि वाक्यं तु युयुधाने महामतौ}
{सुभीमः सिंहनादोऽभूद्योधानां तत्र सर्वशः}


\twolineshloka
{सर्वे हि सर्वशो वीरास्तद्वचः प्रत्यपूजयन्}
{साधुसाध्विति शैनेयं हर्षयन्तो युयुत्सवः}


\chapter{अध्यायः ८१}
\twolineshloka
{वैशंपायन उवाच}
{}


\twolineshloka
{राज्ञस्तु वचनं श्रुत्वा धर्मार्थसहितं हितम्}
{कृष्णा दाशार्हमासीनमब्रवीच्छोककर्शिता}


\twolineshloka
{सुता द्रुपदराजस्य स्वसितायतमूर्धजा}
{संपूज्य सहदेवं च सात्यकिं च महारथम्}


\twolineshloka
{भीमसेनं च संशान्तं दृष्ट्वा परमदुर्मनाः}
{अश्रुपूर्णेक्षणा वाक्यमुवाचेदं मनस्विनी}


\twolineshloka
{विदितं ते महाबाहो धर्मज्ञ मधुसूदन}
{यथा निकृतिमास्थाय भ्रंशिताः पाण्डवाः सुखात्}


\twolineshloka
{धृतराष्ट्रस्य पुत्रेण सामात्येन जनार्दन}
{यथा च सञ्जयो राज्ञा मन्त्रं रहसि श्रावितः}


\twolineshloka
{युधिष्ठिरस्य दाशार्ह तच्चापि विदितं तव}
{यथोक्तः सञ्जयश्चैव तच्च सर्वं श्रुतं त्वया}


\twolineshloka
{पञ्च नस्तात दीयन्तां ग्रामा इति महाद्युते}
{अविस्थलं वृकस्थलं माकन्दीं वारणावतम्}


\twolineshloka
{अवसानं महाबाहो कंचिदेकं च पञ्चमम्}
{इति दुर्योधनो वाच्यः सुहृदश्चास्य केशव}


\twolineshloka
{न चापि ह्यकरोद्वाक्यं श्रुत्वा कृष्ण सुयोधनः}
{युधिष्ठिरस्य दाशार्ह श्रीमतः सन्धिमिच्छतः}


\twolineshloka
{अप्रदानेन राज्यस्य यदि कृष्ण सुयोधनः}
{सन्धिमिच्छेन्न कर्तव्यं तत्र गत्वा कथंचन}


\twolineshloka
{शक्ष्यन्ति हि महाबाहो पाण्डवाः सृञ्जयैः सह}
{धार्तराष्ट्रबलं घोरं क्रुद्धं प्रतिसमासितुम्}


\twolineshloka
{न हि साम्ना न दानेन शक्योऽर्थस्तेषु कश्चन}
{तस्मात्तेषु न कर्तव्या कृपा ते मधुसूदन}


\twolineshloka
{साम्ना दानेन वा कृष्ण ये न शाम्यन्ति शत्रवः}
{योक्तव्यस्तेषु दण्डः स्याज्जीवितं परिरक्षता}


\twolineshloka
{तस्मात्तेषु महादण्डः क्षेप्तव्यः क्षिप्रमच्युत}
{त्वया चैव महाबाहो पाण्डवैः सह सृञ्जयैः}


\twolineshloka
{एतत्समर्थं पार्थानां तव चैव यशस्करम्}
{क्रियमाणं भवेत्कृष्ण क्षत्रस्य च सुखावहम्}


\twolineshloka
{क्षत्रियेण हि हन्तव्यः क्षत्रियो लोभमास्थितः}
{अक्षत्रियो वा दाशार्ह स्वधर्ममनुतिष्ठता}


\twolineshloka
{अन्यत्र ब्राह्मणात्तात सर्वपापेष्ववस्थितात्}
{गुरुर्हि सर्ववर्णानां ब्राह्मणः प्रसृताग्रभुक्}


\twolineshloka
{यथाऽवध्ये वध्यमाने भवेद्दोषो जनार्दन}
{स वध्यस्यावधे दृष्ट इति धर्मविदो विदुः}


\twolineshloka
{यथा त्वां न स्पृशेदेष दोषः कृष्ण तथा कुरु}
{पाण्डवैः सह दाशर्हैः सृञ्जयैश्च ससैनिकैः}


\twolineshloka
{पुनरुक्तं च वक्ष्यामि विश्रम्भेण जनार्दन}
{का तु सीमन्तिनी मादृकं पृथिव्यामस्ति केशव}


\twolineshloka
{सुता द्रुपदराजस्य वेदिमध्यात्समुत्थिता}
{धृष्टद्युम्नस्य भगिनी तव कृष्ण प्रिया सखी}


\twolineshloka
{आजमूढकुलं प्राप्त स्नुषा पाण्डोर्महात्मनः}
{महिषी पाण्डुपुत्राणां पञ्चेन्द्रसमवर्चसाम्}


\twolineshloka
{सुता मे पञ्चभिर्वीरैः पञ्च जाता महारथाः}
{अभिमन्युर्यथा कृष्ण तथा ते तव धर्मतः}


\twolineshloka
{साऽहं केशग्रहं प्राप्ता परिक्लिष्टा सभां गता}
{पश्यतां पाण्डुपुत्राणां त्वयि जीवति केशव}


\twolineshloka
{जीवस्तु पाण्डुपुत्रेषु पञ्चालेष्वथ वृष्णिषु}
{दासीभाताऽस्मि पापानां सभामध्ये व्यवस्थिता}


\twolineshloka
{निरमर्षेष्वचेष्टेषु प्रेक्ष्यमाणेषु पाण्डुषु}
{पाहि मामिति गोविन्द मनसा चिन्तितोसि मे}


\twolineshloka
{यत्र मां भगवान्राजा श्वशुरो वाक्यमब्रवीत्}
{वरं वृणीष्व पाञ्चालि वरार्हाऽसि मता मम}


\twolineshloka
{अदासाः पाण्डवाः सन्तु सरथाः सायुधा इति}
{मयोक्ते यत्र निर्मुक्ता वनवासाय केशव}


\twolineshloka
{एवंविधानां दुःखानामभिज्ञोऽसि जनार्दन}
{त्रायस्व पुण्डरीकाक्ष सभर्तृज्ञातिबान्धवान्}


\twolineshloka
{नन्वहं कृष्ण भीष्मस्य धृतराष्ट्रस्य चोभयोः}
{श्नुषा भवामि धर्मेण साऽहं दासीकृता बलात्}


\twolineshloka
{धिक्पार्थस्य धनुष्मत्तां भीमसेनस्य धिग्बलम्}
{यत्र दुर्योधनः कृष्ण मुहूर्तमपि जीवति}


\threelineshloka
{यदि तेऽहमनुग्राह्या यदि तेऽस्ति कृपा मयि}
{धार्तराष्ट्रेषु वै कोपः सर्वः कृष्ण विधीयताम् ॥वैशंपायन उवाच}
{}


\twolineshloka
{इत्युक्त्वा मृदुसंहारं वृजिनाग्रं सुदर्शनम्}
{सुनीलमसितापाङ्गी सर्वगन्धादिवासितम्}


\twolineshloka
{सर्वलक्षणसंपन्नं महाभुजगवर्चसम्}
{केशपक्षं वरारोहा गृह्व वामेन पाणिना}


\twolineshloka
{पद्माक्षी पुण्डरीकाक्षमुपेत्य गजगामिनी}
{अश्रुपूर्णेक्षणा कृष्णा कृष्णं वचनमब्रवीत्}


\twolineshloka
{अयं ते पुण्डरीकाक्ष दुःशासनकरोद्धृतः}
{स्मर्तव्यः सर्वकार्येषु परेषां सन्धिमिच्छताम्}


\twolineshloka
{यदि भीमार्जुनौ कृष्ण कृपणौ सन्धिकामुकौ}
{पिता मे योत्स्यते वृद्धः सह पुत्रैर्महारथैः}


\twolineshloka
{पञ्च चैव महावीर्याः पुत्रा मे मधुसूदन}
{अभिमन्युं पुरस्कृत्य योत्स्यन्ते कुरुभिः सह}


\twolineshloka
{दुःशासनभुजं श्यामं संछिन्नं पांसुकुण्ठितम्}
{यद्यहं तु न पश्यामि का शान्तिर्हृदयस्य मे}


\twolineshloka
{त्रयोदश हि वर्षाणि प्रतीक्षन्त्या गतानि मे}
{विधाय हृदये मन्युं प्रदीप्तमिव पावकम्}


\twolineshloka
{विदीर्यते मे हृदयं भीमवाक्शल्यपीडितम्}
{योऽयमद्य महाबाहुर्धर्ममेवानुपश्यति}


\twolineshloka
{इत्युक्त्वा बाष्परुद्धेन कण्ठेनायतलोचना}
{रुरोद कृष्णा सोत्कम्पं सस्वरं बाष्पगद्गदम्}


\twolineshloka
{स्तनौ पीनायतश्रोणी सहितावभिवर्षती}
{द्रवीभूतमिवात्युष्णं मुञ्चन्ती वारि नेत्रजम्}


\twolineshloka
{तामुवाच महाबाहुः केशवः परिसान्त्वयन्}
{अचिराद्द्रक्ष्यसे कृष्णे रुदतीर्भरतस्त्रियः}


\twolineshloka
{एवं ता भीरु रोस्त्यन्ति निहतज्ञातिबान्धवाः}
{हतमित्रा हतबला येषां क्रुद्धाऽसि भामिनी}


\twolineshloka
{अहं च तत्करिष्यामि भीमार्जुनयमैः सह}
{युधिष्ठिरनियोगेन दैवाच्च विधिनिर्मितात्}


\twolineshloka
{धार्तराष्ट्राः कालपक्वा न चेच्छृण्वन्ति मे वचः}
{शेष्यन्ते निहता भूमौ श्वसृगालादनीकृताः}


\fourlineindentedshloka
{चलेद्धि हिमवाञ्शैलो मेदिनी शतधा भवेत्}
{द्यौः पतेच्च सनक्षत्रा न मे मोघं वचो भवेत् ॥ 5-81-49a`शुष्येत्तोयनिधिः कृष्णे न मे मोघं वचो भवेत्}
{'सत्यं ते प्रतिजानामि कृष्णे बाष्पो निगृह्यताम्}
{हतामित्राञ्श्रिया युक्तानचिदाद्द्रक्ष्यसे पतीन्}


\chapter{अध्यायः ८२}
\twolineshloka
{अर्जुन उवाच}
{}


\twolineshloka
{कुरूणामद्य सर्वेषां भवान्सुहृदनुत्तमः}
{संबन्धी दयितो नित्यमुभयोः पक्षयोरपि}


\twolineshloka
{पाण्डवैर्धार्तराष्ट्राणां प्रतिपाद्यमनामयम्}
{समर्थः प्रशमं चैव कर्तुमर्हसि केशव}


\twolineshloka
{त्वमित्तः पुण्डरीकाक्ष सुयोधनममर्षणम्}
{शान्त्यर्थं भ्रातरं ब्रूया यत्तद्वाच्यममित्रहन्}


\threelineshloka
{त्वया धर्मार्थयुक्तं चेदुक्तं शिवमनामयम्}
{हितं नादास्यते बालो दिष्टस्य वशमेष्यति ॥श्रीभगवानुवाच}
{}


\threelineshloka
{धर्म्यमस्मद्धितं चैव कुरूणां यदनामयम्}
{एष यास्यामि राजानं धृतराष्ट्रमभीप्सया ॥वैशंपायन उवाच}
{}


\twolineshloka
{ततो व्यपेते तमसि सूर्ये विमल उद्गते}
{मैत्रे मुहूर्ते संप्राप्ते मद्वर्चिषि दिवाकरे}


\twolineshloka
{कौमुदे मासि रेवत्यां शरदन्ते हिमागमे}
{स्फीतसस्यसुखे काले कल्यः सत्ववतां वरः}


\twolineshloka
{मङ्गल्याः पुण्यनिर्घोषा वाचः श्रृण्वंश्च सूनृताः}
{ब्राह्मणानां प्रतीतानामृषीणामिव वासवः}


\twolineshloka
{कृत्वा पौर्वाह्णिकं कृत्यं स्नातः शुचिरलङ्कृतः}
{उपतस्थे विवस्वन्तं पावकं च जनार्दनः}


\twolineshloka
{ऋषभं पृष्ठ आलभ्य ब्राह्मणानभिवाद्य च}
{अग्निं प्रदक्षिणं कृत्वा पश्यन्कल्याणमग्रतः}


\twolineshloka
{तत्प्रतिज्ञाय वचनं पाण्डवस्य जनार्दनः}
{शिनेर्नप्तारमासीनमभ्यभाषत सात्यकिम्}


\twolineshloka
{रथ आरोप्यतां शङ्खश्चक्रं च गदया सह}
{उपासङ्गाश्च शक्त्यश्च सर्वप्रहरणानि च}


\twolineshloka
{दुर्योधनश्च दुष्टात्मा कर्णश्च सहसौबलः}
{न च शत्रुरवज्ञेयो दुर्बलोऽपि बलीयसा}


\twolineshloka
{ततस्तन्मतमाज्ञाय केशवस्य पुनःसराः}
{प्रसस्त्नुर्योजयिष्यन्तो रथं चक्रगदाभृतः}


\twolineshloka
{तं दीप्तमिव कालाग्निमाकाशगमिवाशुगम्}
{सूर्यचन्द्रप्रकाशाभ्यां चक्राभ्यां समलङ्कृतम्}


\twolineshloka
{अर्धचन्द्रैश्च चन्द्रैश्च मत्स्यैः समृगपक्षिभिः}
{पुष्पैश्च विविधैश्चित्रं मणिरत्नैश्च सर्वशः}


\twolineshloka
{तरुणादित्यसङ्काशं बृहन्तं चारुदर्शनम्}
{मणिहेमविचित्राङ्गं सुध्वजं सुपताकिनम्}


\twolineshloka
{सूपस्करमनाधृष्यं वैयाघ्रपरिवारणम्}
{यशोघ्नं प्रत्यमित्राणां यदूनां नन्दिवर्धनम्}


\twolineshloka
{वाजिभिः शैब्यसुग्रीवमेधपुष्पबलाहकैः}
{स्नातैः संपादयामासुः संपन्नैः सर्वसंपदा}


\twolineshloka
{महिमानं तु कृष्णस्य भूय एवाभिवर्धयन्}
{सुघोषः पतगेन्द्रेण ध्वजेन युयुजे रथः}


\twolineshloka
{तं मेरुशिखरप्रख्यं मेघदुन्दुभिनिस्वनम्}
{आरुरोह रथं सौरिर्विमानमिव कामगम्}


\twolineshloka
{ततः सात्यकिमारोप्य प्रययौ पुरुषोत्तमः}
{पृथिवीं चान्तरिक्षं च रथघोषेण नादयन्}


\twolineshloka
{व्यपोढाभ्रस्ततः कालः क्षणेन समपद्यत}
{शिवश्चानुववौ वायुः प्रशान्तमभवद्रजः}


\twolineshloka
{प्रदक्षिणानुलोमाश्च मङ्गल्या मृगपक्षिणः}
{प्रयाणे वासुदेवस्य बभूवुरनुयायिनः}


\twolineshloka
{मङ्गल्यार्थप्रदैः शब्दैरन्ववर्तन्त सर्वशः}
{सारसाः शतपत्राश्च हंसाश्च मधुसूदनम्}


\twolineshloka
{मन्त्राहुतिमहाहोमैर्हूयमानश्च पावकः}
{प्रदक्षिणशिखो भूत्वा विधूमः समपद्यत}


\twolineshloka
{वसिष्ठो वामदेवश्च भूरिद्युम्नो गयः क्रथः}
{शुकनारदवाल्मीका मरुत्तः कुशिको भृगुः}


\twolineshloka
{देवब्रह्मर्षयश्चैव कृष्णं यदुसुखावहम्}
{प्रदक्षिणमवर्तन्त सहिता वासवानुजम्}


\twolineshloka
{एवमेतैर्महाभागैर्महर्षिगणसाधुभिः}
{पूजितः प्रययौ कृष्णः कुरूणां सदनं प्रति}


\twolineshloka
{` देवताभ्यो नमस्कृत्य ब्राह्मणान्स्वस्ति वाच्य च}
{प्रययौ पुण्डरीकाक्षः सात्यकेन सहाच्युतः}


\twolineshloka
{तं प्रयान्तमनुप्रायात्कुन्तीपुत्रो युधिष्ठिरः}
{भीमसेनार्जुनौ चोभौ माद्रीपुत्रौ च पाण्डवौ}


\twolineshloka
{चेकितानश्च विक्रान्तो धृष्टकेतुश्च चेदिपः}
{द्रुपदः काशिराजश्च शिखण्डी च महारथः}


\twolineshloka
{धृष्टद्युम्नः सपुत्रश्च विराटः केकयैः सह}
{संसाधनार्थं प्रययुः क्षत्रियाः क्षत्रियर्षभम्}


\twolineshloka
{ततोऽनुव्रज्य गोविन्दं धर्मराजो युधिष्ठिरः}
{राज्ञां सकाशे द्युतिमानुवाचेदं वचस्तदा}


\twolineshloka
{यो वै न कामान्न भयान्न लोभान्नार्थकारणात्}
{अन्यायमनुवर्तेत स्थिरबुद्धिरलोलुपः}


\twolineshloka
{धर्मज्ञो धृतिमान्प्राज्ञः सर्वभूतेषु केशवः}
{ईश्वरः सर्वभूतानां देवदेवः सनातनः}


\twolineshloka
{तं सर्वगुणसंपन्नं श्रीवत्सकृतलक्षणम्}
{संरिष्वज्य कौन्तेयः सन्देष्टुमुपचक्रमे}


\chapter{अध्यायः ८३}
\twolineshloka
{युधिष्ठिर उवाच}
{}


\twolineshloka
{या सा बाल्यात्प्रभृत्यस्मान्पर्यवर्धयताबला}
{उपवासतपःशीला सदा स्वस्त्ययने रता}


\twolineshloka
{देवतातिथिपूजासु गुरुशुश्रूषणे रता}
{वत्सला प्रियपुत्रा च माताऽस्माकं जनार्दन}


\twolineshloka
{सुयोधनभयाद्या नो त्रायतामित्रकर्शन}
{महतो मृत्युसंबाधुद्दुस्तरान्नौरिवार्णवात्}


\twolineshloka
{अस्मत्कृते च सततं यया दुःखानि माधव}
{अनुभूतान्यदुःखार्हां तां स्म पृच्छेरनामयम्}


\twolineshloka
{भृशमाश्वासयेश्चैनां पुत्रशोकपरिप्लुताम्}
{अभिवाद्य स्वजेथास्त्वं पाण्डवान्परिकीर्तयन्}


\twolineshloka
{ऊढात्प्रभृति दुःखानि श्वशुराणामरिन्दम}
{निराकाराऽतदर्हा च पश्यन्ती दुःखमश्रुते}


\twolineshloka
{अपि जातु स कालः स्यात्कृष्ण दुःखविपर्ययः}
{यदहं मातरं क्लिष्टां सुखं दद्यामरिन्दम}


\twolineshloka
{प्रव्रजन्तोऽनुधावन्तीं कृपणां पुत्रगृद्धिनीम्}
{रुदतीमपहायैनामगच्छाम वयं वनम्}


\twolineshloka
{सा नूनं म्रियते दुःखैः सा चेज्जीवति केशव}
{तथा पुत्राधिभिर्गाढमार्तामर्चय सत्कृत}


\twolineshloka
{अभिवाद्याऽथ सा कृष्ण त्वया मद्वचवाद्विभो}
{धृतराष्ट्रश्च कौरव्यो राजानश्च वयोधिकाः}


\twolineshloka
{भीष्मं द्रोणं कृपं चैव महाराजं च वाह्लिकम्}
{द्रौणिं च सोमदत्तं च सर्वांश्च भरतान्पृथक्}


\fourlineindentedshloka
{` यथावयो यथास्थानं प्रपद्यस्व जनार्दन}
{'विदुरं च महाप्राज्ञं कुरूणां मन्त्रधारिणम्}
{अगाधबुद्धिं मर्मज्ञं स्वजेथा मधुसूदन ॥वैशंपायन उवाच}
{}


\twolineshloka
{इत्युक्त्वा केशवं तत्र राजमध्ये युधिष्ठिरः}
{अनुज्ञातो निववृते कृष्णं कृत्वा प्रदक्षिणम्}


\twolineshloka
{व्रजन्नेव तु बीभत्सुः सखायं पुरुषर्षभम्}
{अब्रवीत्परवीरघ्रं दाशार्हमपराजितम्}


\twolineshloka
{यदस्माकं विभो वृत्तं पुरा वै मन्त्रनिश्चये}
{अर्धराज्यस्य गोविद विदितं सर्वराजसु}


\twolineshloka
{तच्चेद्दद्यादखङ्गेन सत्कृत्यानवमत्य च}
{प्रियं मे स्यान्महाबाहो मुच्येरन्महतो भयात्}


\threelineshloka
{अतश्चेदन्यथाकर्ता धार्तराष्ट्रोऽनुपायवित्}
{अन्तं नूनं करिष्यामि क्षत्रियाणां जनार्दन ॥वैशंपायन उवाच}
{}


\twolineshloka
{एवमुक्ते पाण्डवेन समहृष्यद्वृकोदरः}
{मुहुर्मुहुः क्रोधवशात्प्रावेपत च पाण्डवः}


\twolineshloka
{वेपमानश्च कौन्तेयः प्राक्रोशन्महतो रजन्}
{धनञ्जयवचः श्रुत्वा हर्षोत्सिक्तमना भृशम्}


\twolineshloka
{तस्य तं निनदं श्रुत्वा संप्रावेपन्त धन्विनः}
{वाहनानि च सर्वामि शकृन्मूत्रे प्रमुस्रुवुः}


\twolineshloka
{इत्युक्त्वा केशवं तत्र तथा चोह्वा विनिश्चयम्}
{अनुज्ञातो निववृते परिष्वज्य जनार्दनम्}


\twolineshloka
{तेषु राजसु सर्वेषु निवृत्तेषु जनार्दनः}
{तूर्णमभ्यगमद्धृष्टः शैब्यसुग्रीववाहनः}


\twolineshloka
{ते हया वासुदेवस्य दारुकेण प्रचोदिताः}
{पन्थानमाचेमुरिव ग्रसमाना इवाम्बरम्}


\twolineshloka
{अथापश्यन्महाबाहुर्ऋषीनध्वनि केशवः}
{ब्राह्या श्रिया दीप्यमानान्स्थितानुभयतः पथि}


\twolineshloka
{सोऽवतीर्य रथात्तूर्णमभिवाद्य जनार्दनः}
{यथावृत्तानृषीन्सर्वानभ्यभाषत पूजयन्}


\threelineshloka
{कच्चिल्लोकेषु कुशलं कच्चिद्धर्मः स्वनुष्ठितः}
{ब्राह्मणानां त्रयो वर्णाः कच्चित्तिष्ठन्ति शासने}
{` पितृदेवातिथिभ्यश्च कच्चित्पूता स्वनुष्ठिताः ॥'}


\twolineshloka
{तेभ्यः प्रयुज्य तां पूजां प्रोवाच मधुसूदनः}
{भगवन्तः क्व संसिद्धाः कोऽवधिर्भवतामिह}


\twolineshloka
{किं वा कार्यं भगवतामहं किं करवाणि यः}
{केनार्थेनोपप्तंप्राप्ता भगवन्तो महीतलम्}


\twolineshloka
{` एवमुक्ताः केशवेन मुनयः शंसितव्रताः}
{नारदप्रमुखाः सर्वे प्रत्यनन्दन्त केशवम्}


\twolineshloka
{अधश्शिराः सर्पमाली महर्षिः स हि देवलः}
{अर्वावसुः सुजानुश्च मैत्रेयः शुनको बली}


\twolineshloka
{बको दाल्भ्यः स्थूलशिराः कृष्णद्वैपायनस्तथा}
{आपोदधौम्यो धौम्यश्च आणिमाण्डव्यकौशिकौ}


\twolineshloka
{दामोष्णीषस्त्रिषवणः पर्णादो घटजानुकः}
{मौञ्जायनो वायुभक्षः पाराशर्योऽथ शालिकः}


\twolineshloka
{शीलवानशनिर्धाता शून्यपालोऽकृतव्रणः}
{श्वेतकेतुः कहोलश्च रामश्चैव महातपाः ॥'}


\twolineshloka
{तमब्रवीज्जामदग्न्य उपेत्य मधुसूदनम्}
{परिष्वज्य च गोविन्दं सुरासुरपतेः सखा}


\twolineshloka
{देवर्षयः पुण्यकृतो ब्राह्मणाश्च बहुश्रुताः}
{राजर्षयश्च दाशार्ह मानयन्ति तपस्विनः}


\threelineshloka
{दैवासुरस्य द्रुष्टारः पुराणस्य महामते}
{समेत पार्थिवं क्षत्रं दिदृक्षन्तश्च सर्वतः}
{सभासदश्च राजानस्त्वां च सत्यं जनार्दनम्}


% Check verse!
एतन्महत्प्रेक्षणीयं द्रुष्टुमिच्छाम केशव
\twolineshloka
{धर्मार्थसहिता वाचः श्रोतुमिच्छाम माधव}
{त्वयोच्यमानाः कुरुषु राजमध्ये परन्तप}


\twolineshloka
{` सभायां मधुरा वाचः शुश्रूषन्तस्त्वयेरिताः}
{कुरूणां प्रतिवाचश्च श्रोतुमिच्छाम माधव ॥'}


\twolineshloka
{भीष्मद्रोणादयश्चैव विदुरश्च महामतिः}
{त्वं च यादवशार्दूल सभायां वै समेष्यथ}


\threelineshloka
{तव वाक्यानि दिव्यानि तथा तेषां च माधव}
{श्रोतुमिच्छाम गोविन्द सत्यानि च हितानि च}
{आपृष्टोऽसि महाबाहो पुनर्द्रक्ष्यामहे वयम्}


\threelineshloka
{याह्यविघ्नेन वै वीर द्रक्ष्यामस्त्वां सभागतम्}
{आसीनमासने दिव्ये बलतेजःसमाहितम् ॥वैशंपायन उवाच}
{}


\twolineshloka
{प्रयान्तं देवकीपुत्रं परवीररुजो दश}
{महारथा महाबाहुमन्वयुः शस्त्रपाणयः}


\twolineshloka
{पदातीनां सहस्रं च सादिनां च परन्तप}
{भोज्यं च विपुलं राजन्प्रेष्याश्च शतशोपरे}


\chapter{अध्यायः ८४}
\twolineshloka
{जनमेजय उवाच}
{}


\threelineshloka
{कथं प्रयातो दाशार्हो महात्मा मधुसूदनः}
{कानि वा व्रजतस्तस्य निमित्तानि महौजसः ॥वैशंपायन उवाच}
{}


\twolineshloka
{तस्य प्रयाणे यान्यसन्निमित्तानि महात्मनः}
{तानि मे श्रृणु सर्वाणि दैवान्यौत्पातिकानि च}


\twolineshloka
{अनभ्रेऽशनिनिर्घोषः सविद्युत्समजायत}
{अन्वगेव च पर्जन्यः प्रावर्षद्विघने भृशम्}


\twolineshloka
{प्रत्यगूहुर्महानद्यः प्राङ्मुखाः सिन्धुसप्तमाः}
{विपरिता दिशः सर्वा न प्राज्ञायत किंचन}


\twolineshloka
{प्राज्वलन्नग्नयो राजन्पृथिवी समकम्पत}
{उदपानाश्च कुम्भाश्च प्रासिञ्चञ्शतशो जलम्}


\twolineshloka
{तमःसंवृतमप्यासीत्सर्वं जगदिदं तथा}
{न दिशो नादिशो राजन्प्रज्ञायन्तेस्म रेणुना}


\twolineshloka
{प्रादुरासीन्महाञ्छब्दः खेशरीरमदृश्यत}
{सर्वेषु राजन्देशेषु तदद्भुतमिवाभवत्}


\twolineshloka
{प्रामथ्नाद्धास्तिनपुरं वातो दक्षिणपश्चिमः}
{आरुजन्गणशो वृक्षान्परुषोऽशनिनिःस्वनः}


\twolineshloka
{यत्रयत्र च वार्ष्णेयो वर्तते पथि भारत}
{तत्रतत्र सुखो वायुः सर्वं चासीत्प्रदक्षिणम्}


\twolineshloka
{ववर्ष पुष्पवर्षं च कमलानि च भूरिशः}
{समश्च पन्था निर्दुःखो व्यपेतकुशकण्टकः}


\twolineshloka
{संस्तुतो ब्राह्मणैर्गीर्भिस्तत्रतत्र सहस्रशः}
{अर्च्यते मधुपर्कैश्च वसुभिश्च वसुप्रदः}


\twolineshloka
{तं किरन्ति महात्मानं वन्यैः पुष्पैः सुगन्धिभिः}
{स्त्रियः पथि समागम्य सर्वभूतहिते रतम्}


\twolineshloka
{स शालिभवनं रम्यं सर्वसस्यसमाचितम्}
{सुख परमधर्मिष्ठमभ्यगाद्भरतर्षभ}


\twolineshloka
{पश्यन्बहुपशून्ग्रामान्रम्यान्हृदयतोपणान्}
{पुराणि च व्यतिकामन्राष्ट्राणि विविधानि च}


\twolineshloka
{नित्यं हृष्टाः सुमनसो भारतैरभिरक्षिताः}
{नोद्विग्नाः परचक्राणां व्यसनानामकोविदाः}


\twolineshloka
{उपप्लाव्यादथागम्य जनाः पुरनिवासिनः}
{यथ्यतिष्ठन्त सहिता विष्वक्सेनदिदृक्षया}


\twolineshloka
{ते तु सर्वे समायान्तमग्निमिद्धमिव प्रभुम्}
{अर्चयामासुरर्चार्हं देशातिथिमुपस्थितम्}


\twolineshloka
{वृकस्थलं समासाद्य केशवः परवीरहा}
{प्रकीर्णरश्मावादित्ये व्योम्नि वै लोहितायति}


\twolineshloka
{` ततो ह्यनुचरान्सर्वानुवाच मधुसूदनः}
{युधिष्ठिरस्य कार्यार्थमिह वत्स्यामहे वयम्}


\twolineshloka
{तस्य तन्मतमाज्ञाय चक्रुरावसथं नराः}
{क्षणेन चान्नपानानि ररावन्ति समार्जयन् ॥'}


\twolineshloka
{अवतीर्य रथात्तूर्णं कृत्वा शौचं यथाविधि}
{रथमोचनमादिश्य सन्ध्यामुपविवेश ह}


\twolineshloka
{दारुकोऽपि हयान्मुक्त्वा परिचर्य च शास्त्रतः}
{मुमोच सर्वं योक्तादि मुक्त्वा चैतानवासृजत्}


\twolineshloka
{तस्मिन्ग्रामे प्रधानास्तु य आसन्ब्राह्मणा नृप}
{आर्याः कुलीना ह्रीमन्तो ब्राह्मीं वृत्तिमनुष्ठिताः}


\twolineshloka
{तेऽभिगम्य महात्मानं हृषीकेशमरिन्दमम्}
{पूजां चक्रुर्यथान्यायमाशीर्मङ्गलसंयुताम्}


\twolineshloka
{ते पूजयित्वा दाशार्हं सर्वलोकेषु पूजितम्}
{न्यवेदयन्त वेश्मानि गुणवन्ति महात्मने}


\twolineshloka
{तान्प्रभुः कृतमित्युक्ता सत्कृत्य च यथार्हतः}
{अभ्येत्य चैषां वेश्मानि पुनरायात्सहैव तैः}


\twolineshloka
{सुमृष्टं भोजयित्वा च ब्राह्मणांस्तत्र केशवः}
{भुक्ता च सह तैः सर्वैरवसत्तां क्षपां सुखम्}


\chapter{अध्यायः ८५}
\twolineshloka
{वैशंपायन उवाच}
{}


\twolineshloka
{तदा दूतैः समाज्ञाय आयान्तं मधुसूदनम्}
{धृतराष्ट्रोऽब्रवीद्भीष्ममर्चयित्वा महाभुजम्}


\twolineshloka
{द्रोणं च सञ्जयं चैव विदुरं च महामतिम्}
{दुर्योदनं सहामात्यं हृष्टरोमाऽब्रवीदिदम्}


\twolineshloka
{अद्भुतं महादाश्चर्यं श्रूयते कुरुनन्दन}
{स्त्रियो बालाश्च वृद्धाश्च कथयन्ति गृहेगृहे}


\twolineshloka
{सत्कृत्याचक्षते चान्ये तथैवान्ये समागताः}
{पृथग्वादाश्च वर्तन्ते चत्वरेषु सभासु च}


\twolineshloka
{उपायास्यति दाशार्हः पाण्डवार्थे पराक्रमी}
{स नो मान्यश्च पूज्यश्च सर्वथा मधुसूदनः}


\twolineshloka
{तस्मिन्हि यात्रा लोकस्य भूतानामीश्वरोऽहि सः}
{तस्मिन्धृतिश्च वीर्यं च प्रज्ञा चौजश्च माधवे}


\twolineshloka
{स मान्यतां नरश्रेष्ठः स हि धर्मः सनातनः}
{पूजितो हि सुखाय स्यादसुखः स्यादपूजितः}


\twolineshloka
{स चेत्तुप्यति दाशार्ह उपचरैररिन्दमः}
{कृष्णात्सर्वानभिप्रायान्प्राप्स्यामः सर्वराजसु}


\twolineshloka
{तस्य पूजार्थमद्यैव संविधस्त्व परन्तप}
{सभाः पथि विधीयन्तां सर्वकामसमन्विताः}


\threelineshloka
{यथा प्रीतिर्महाबाहो त्वयि जायेत तस्य वै}
{तथा कुरुष्व गान्धारे कथं वा भीष्म मन्यसे ॥वैशंपायन उवाच}
{}


\twolineshloka
{ततो भीष्मादयः सर्वे धृतराष्ट्रं जनाधिपम्}
{ऊचुःक परममित्येवं पूजयन्तोऽस्य तद्वचः}


\twolineshloka
{तेषामनुमतं ज्ञात्वा राजा दुर्योधनस्तदा}
{सभावास्तूनि रम्याणि प्रदेष्टुमुपचक्रमे}


\twolineshloka
{ततो देशेषु देशेषु रमणीयेषु भागशः}
{सर्वरत्नसमाकीर्णाः सभाश्चक्रुरनेकशः}


\twolineshloka
{आसनानि विचित्राणि युतानि विविधैर्गुणैः}
{स्त्रियो गन्धानलङ्कानारान्सूक्ष्माणि वसनानि च}


\twolineshloka
{गुणवन्त्यन्नपानानि भोज्यानि विविधानि च}
{माल्यानि च सुगन्धीनि तानि राजा ददौ ततः}


\twolineshloka
{विशेवतश्च वासार्थं सभां ग्रामे वृकस्थले}
{विदधे कौरवो राजा बहुरत्नां मनोरमाम्}


\twolineshloka
{एतद्विधाय वै सर्वं देवार्हमतिमानुषम्}
{आचख्यौ धृतराष्ट्राय राजा दुर्योधनस्तदा}


\twolineshloka
{ताः सभाः केशवः सर्वा रत्नानि विविधानि च}
{असमीक्ष्यैव दाशार्ह उपायात्कुरुसद्म तत्}


\chapter{अध्यायः ८६}
\twolineshloka
{धृतराष्ट्र उवाच}
{}


\twolineshloka
{उपप्लाव्यादिह क्षत्तरुपायातो जनार्दनः}
{वृकस्थले निवसति स च प्रातरिहैष्यति}


\twolineshloka
{आहुकानामधिपतिः पुरोगः सर्वसात्वताम्}
{महामना महावीर्यो महासत्वो जनार्दनः}


\twolineshloka
{स्फीतस्य वृष्णिराष्ट्रस्य भर्ता गोप्ता च माधवः}
{त्रयाणामपि लोकानां भगवान्प्रपितामहः}


\twolineshloka
{वृष्ण्यन्धकाः सुमनसो यस्य प्रज्ञामुपासते}
{आदित्या वसवो रुद्रा यथा बुद्धिं बृहस्पतेः}


\twolineshloka
{तस्मै पूजां प्रयोक्ष्यामि दाशार्हाय महात्मने}
{प्रत्यक्षं तव धर्मज्ञ तां मे कथयतः श्रुणु}


\twolineshloka
{एकवर्णैः सुक्लृप्ताङ्गैर्बाह्लिजातैर्हयोत्तमैः}
{चतुर्यक्तान्रथांस्तस्मै रौक्मान्दास्यामि षोडश}


\twolineshloka
{नित्यप्रभिन्नान्मातङ्गानीषादन्तान्प्रहारिणः}
{अष्टानुचरमेकैकमष्टौ दास्यामि कौरव}


\twolineshloka
{दासीनामप्रजातानां शुभानां रुक्मवर्चसाम्}
{शतमस्मै प्रदास्यामि दासानामपि तावताम्}


\twolineshloka
{आविकं च सुखस्पर्शं पार्वतीयैरुपाहृतम्}
{तदप्यस्मै प्रदास्यामि सहस्त्राणि दाशाष्ट च}


\twolineshloka
{अजिनानां सहस्राणि चीनदेशोद्भवानि च}
{तान्यप्यस्मै प्रदास्यामि यावदर्हति केशवः}


\twolineshloka
{दिवा रात्रौ च भात्येष सुतेजा विमलो मणिः}
{तमप्यस्मै प्रदास्यामि तमर्हति हि केशवः}


\twolineshloka
{एकेनाभिपतत्यह्ना योजनानि चतुर्दश}
{यानमश्वतरीयुक्तं दास्ये तस्मै तदप्यहम्}


\twolineshloka
{यावन्ति वाहनान्यस्य यावन्तः पुरुषाश्च ते}
{ततोष्टगुणमप्यस्मै भोज्यं दास्याम्यहं सदा}


\twolineshloka
{मम पुत्राश्च पौत्रश्च सर्वे दुर्योधनादृते}
{प्रत्युद्यास्यन्ति दाशार्हं रथैर्मृष्टैः स्वलङ्कृताः}


\twolineshloka
{स्वलङ्कृताश्च कल्याण्यः पादैरेव सहस्रशः}
{वारमुख्या महाभागं प्रत्युद्यास्यन्ति केशवम्}


\twolineshloka
{नगरादपि याः काश्चिद्गमिष्यन्ति जनार्दनम्}
{द्रष्टुं कन्याश्च कल्याण्यस्ताश्च यास्यन्त्यनावृताः}


\twolineshloka
{सस्त्रीपुरुषबालं च नगरं मधुसूदनम्}
{उदीक्षतां महात्मानं भानुमन्तमिव प्रजाः}


\twolineshloka
{महाध्वजपताकाश्च क्रियन्तां सर्वतो दिशः}
{जलावसिक्तो विरजाः पन्थास्तस्येति चान्वशात्}


\twolineshloka
{दुःशासनस्य च गृहं दुर्योधनगृहाद्वरम्}
{तदद्य क्रियतां क्षिप्रं सुसंमृष्टमलङ्कृतम्}


\twolineshloka
{एतद्धि रुचिराकारैः प्रासादैरुपशोभितम्}
{शिवं च रमणीयं च सर्वर्तु सुमहाधनम्}


\twolineshloka
{सर्वमस्मिन्गृहे रत्नं मम दुर्योधनस्य च}
{यद्यदर्हति वार्ष्णेयस्तत्तद्देयमसंशयम्}


\chapter{अध्यायः ८७}
\twolineshloka
{विदुर उवाच}
{}


\twolineshloka
{राजन्बहुमतश्चासि त्रैलोक्यस्यापि सत्तमः}
{संभावितश्च लोकस्य संमतश्चासि भारत}


\twolineshloka
{यत्त्वमेवंगते ब्रूयाः पश्चिमे वयसि स्थितः}
{शास्त्राद्वा सुप्रतर्काद्वा सुस्थिरः स्थविरो ह्यसि}


\twolineshloka
{लेखा शशिनिभाः सूर्ये महोर्मिरिव सागरे}
{धर्मस्त्वयि तथा राजन्निति व्यवसिताः प्रजाः}


\twolineshloka
{सदैव भावितो लोको गुणौघैस्तव पार्थिव}
{गुणानां रक्षणे नित्यं प्रयतस्व सबान्धवः}


\twolineshloka
{आर्जवं प्रतिपद्यस्व मा बाल्याद्ब्रहु नीनशः}
{राजन्पुत्रांश्च पौत्रांश्च सुहृदश्चैव सुप्रियान्}


\twolineshloka
{यत्त्वं दित्ससि कृष्णाय राजन्नतिथये बहु}
{एतदन्यच्च दाशार्हः पृथिवीमपि चार्हति}


\twolineshloka
{न तु त्वं धर्ममुद्दिश्य तस्य वा प्रियकारणात्}
{एतद्दित्ससि कृष्णाय सत्येनात्मानमालभे}


\twolineshloka
{मायैषा सत्यमेवैतच्छद्मैतद्भूरिदक्षिणा}
{जानामि त्वन्मतं राजन्गूढं बाह्येन कर्मणा}


\twolineshloka
{पञ्च पञ्चैव लिप्स्यन्ति ग्रामकान्पाण्डवा नृप}
{न च दित्ससि तेभ्यस्तांस्तच्छमं न करिष्यसि}


\twolineshloka
{अर्थेन तु महाबाहुं वार्ष्णेयं त्वं जिहीर्षसि}
{अनेन चाप्युपायेन पाण्डवेभ्यो विभेत्स्यसि}


\twolineshloka
{न च वित्तेन शक्योऽसौ नोद्यमेन न गर्हया}
{अन्यो धनञ्जयात्कर्तुमेतत्तत्त्वं ब्रवीमि ते}


\twolineshloka
{वेद कृष्णस्य माहात्म्यं वेदास्य दृढभक्तिताम्}
{अत्याज्यमस्य जानामि प्राणैस्तुल्यं धनञ्जयम्}


\twolineshloka
{अन्यत्कुम्भादपांपूर्णादन्यत्पादावसेचनात्}
{अन्यत्कुशलसंप्रश्नान्नैवेक्ष्यति जनार्दनः}


\twolineshloka
{यत्त्वस्य प्रियमातिथ्यं मानार्हस्य महात्मनः}
{तदस्मै क्रियतां राजन्मानार्होऽसौ जनार्दनः}


\twolineshloka
{आशंसमानः कल्याणं कुरूनभ्येति केशवः}
{येनैव राजन्नर्थेन तदेवास्मा उपाकुरु}


\twolineshloka
{शममिच्छति दाशार्हस्तव दुर्योधनस्य च}
{पाण्डवानां च राजेन्द्र तदस्य वचनं कुरु}


\twolineshloka
{पिताऽसि राजन्पुत्रास्ते वृद्धस्त्वं शिशवः परे}
{वर्तस्व पितृवत्तेषु वर्तन्ते ते हि पुत्रवत्}


\chapter{अध्यायः ८८}
\twolineshloka
{दुर्योधन उवाच}
{}


\twolineshloka
{यदाह विदुरः कृष्णे सर्वं तत्सत्यमच्युते}
{अनुरक्तो ह्यसंहार्यः पार्थान्प्रति जनार्दनः}


\twolineshloka
{यत्तत्सत्कारसंयुक्तं देयं वसु जनार्दने}
{अनेकरूपं राजेन्द्र न तद्देयं कदाचन}


\twolineshloka
{देशः कालस्तथाऽयुक्तो न हि नार्हति केशवः}
{मंस्यत्यधोक्षजो राजन्भयादर्चति मामिति}


\twolineshloka
{अवमानश्च यत्र स्यात्क्षत्रियस्य विशांपते}
{न तत्कुर्याद्बुधः कार्यमिति मे निश्चिता मतिः}


\twolineshloka
{स हि पूज्यतमो लोके कृष्णः पृथुललोचनः}
{त्रयाणामपि लोकानां विदितं मम सर्वथा}


\threelineshloka
{न तु तस्मै प्रदेयं स्यात्तथा कार्यगतिः प्रभो}
{विग्रहः समुपारब्धो न हि शाम्यत्यविग्रहात् ॥वैशंपायन उवाच}
{}


\twolineshloka
{तस्य तद्वचनं श्रुत्वा भीष्मः कुरुपितामहः}
{वैचित्रवीर्यं राजानमिदं वचनमब्रवीत्}


\threelineshloka
{सत्कृतोऽसत्कृतो वाऽपि न क्रुध्येत जनार्दनः}
{`नावमंस्यत्यवज्ञातॄनवज्ञातोऽपि केशवः}
{'नालमेनमवज्ञातुं नावज्ञेयो हि केशवः}


\twolineshloka
{यत्तु कार्यं महाबाहो मनसा कार्यतां गतम्}
{सर्वोपायैर्न तच्छक्यं केनचित्कर्तुमन्यथा}


\twolineshloka
{स यद्ब्रूयान्महाबाहुस्तत्कार्यमविशङ्कया}
{वासुदेवेन तीर्थेन क्षिप्रं संशाभ्य पाण्डवैः}


\threelineshloka
{धर्म्यमर्थ्यं च धर्मात्मा ध्रुवं वक्ता जनार्दनः}
{तस्मिन्वाच्याः प्रिया वाचो भवता बान्धवैः सह ॥दुर्योधन उवाच}
{}


\twolineshloka
{न पर्याप्तोस्मि यद्राजञ्श्रियं निष्केवलामहम्}
{तैः सहेमामुपाश्रीयां यावज्जीवं पितामह}


\twolineshloka
{इदं तु सुमहत्कार्यं श्रुणु मे यत्समर्थितम्}
{परायणं पाण्डवानां नियच्छामि जनार्दनम्}


\twolineshloka
{तस्मिन्बद्धे भविष्यन्ति वृष्णयः पृथिवी तथा}
{पाण्डवाश्च विधेया मे स च प्रातरिहैप्यति}


\threelineshloka
{अत्रोपायान्यथा सम्यङ्न बुद्ध्येत जनार्दनः}
{न चापायो भवेत्कश्चित्तद्भवान्प्रब्रवीतु मे ॥वैशंपायन उवाच}
{}


\twolineshloka
{तस्य तद्वचनं श्रुत्वा घोरं कृष्णेऽभिसंहितम्}
{धृतराष्ट्रः सहामात्यो व्यथितो विमनाभवत्}


\twolineshloka
{ततो दुर्योधनमिदं धृतराष्ट्रोऽब्रवीद्वचः}
{मैवं वोचः प्रजापाल नैष धर्मः सनातनः}


\threelineshloka
{दूतश्च हि हृषीकेशः संबन्धी च प्रियश्च नः}
{अपापः कौरवेयेषु स कथं बन्धमर्हति ॥भीष्म उवाच}
{}


\twolineshloka
{परीतस्तव पुत्रोऽयं धृतराष्ट्र सुमन्दधीः}
{वृणोत्यनर्थं नैवार्थं याच्यमानः सुहृज्जनैः}


\twolineshloka
{इममुत्पथि वर्तन्तं पापं पापानुबन्धिनम्}
{वाक्यानि सुहृदां हित्वा त्वमप्यस्यानुवर्तसे}


\twolineshloka
{कृष्णमक्लिष्टकर्माणमासाद्यायं सुदुर्मतिः}
{तव पुत्रः सहामात्यः क्षणेन न भविष्यति}


\twolineshloka
{पापस्यास्य नृशंसस्य त्यक्तधर्मस्य दुर्मतेः}
{नोत्सहेऽनर्थसंयुक्ताः श्रोतुं वाचः कथंचन}


\twolineshloka
{इत्युक्त्वा भरतश्रेष्ठो वृद्धः परममन्युमान्}
{उत्थाय तस्मात्प्रातिष्ठद्भीष्मः सत्यपराक्रमः}


\chapter{अध्यायः ८९}
\twolineshloka
{वैशंपायन उवाच}
{}


\twolineshloka
{प्रातरुत्थाय कृष्णस्तु कृतवान्सर्वमाह्निकम्}
{ब्राह्मणैरभ्यनुज्ञातः प्रययौ नगरं प्रति}


\twolineshloka
{तं प्रयान्तं महाबाहुमनुज्ञाप्य महाबलम्}
{पर्यवर्तन्त ते सर्वे वृकस्थलनिवासिनः}


\twolineshloka
{` प्रददौ पुण्डरीकाक्षो रत्नानि च धनानि च}
{तान्प्रस्थाप्य महाबाहुरुपायात्कुरुसंसदम् ॥'}


\twolineshloka
{धार्तराष्ट्रास्तमायान्तं प्रत्युञ्जग्मुः स्वलङ्कृताः}
{दुर्योधनादृते सर्वे भीष्मद्रोणकृपादयः}


\twolineshloka
{पौराश्च बहुला राजन्हृषीकेशं दिदृक्षवः}
{यानैर्बहुविधैरन्ये पद्भिरेव तथाऽपरे}


\twolineshloka
{स वै पथि समागम्य भीष्मेणाक्लिष्टकर्मणा}
{द्रोणेन धार्तराष्ट्रैश्च तैर्वृतो नगरं ययौ}


\twolineshloka
{कृष्णसंमाननार्थं च नगरं समलङ्कृतम्}
{बभूव राजमार्गश्च बहुरत्नसमाचितः}


\twolineshloka
{न च कश्चिद्गृहे राजंस्तदाऽऽसीद्भरतर्षभ}
{न स्त्री न वृद्धो न शिशुर्वासुदेवदिदृक्षया}


\twolineshloka
{राजमार्गे नरास्तस्मिन्संस्तुवन्त्यवनिं गताः}
{तस्मिन्काले महाराज हृषीकेशप्रवेशने}


\twolineshloka
{आवृतानि वरस्त्रीभिर्गृहाणि सुमहान्त्यपि}
{प्रचलन्तीव भारेण दृश्यन्तेस्म महीतले}


\twolineshloka
{तथा च गतिमन्तस्ते वासुदेवस्य वाजिनः}
{प्रनष्टगतयोऽभूवन्राजमार्गे नरैर्वृते}


\twolineshloka
{स गृहं धृतराष्ट्रस्य प्राविशच्छत्रुकर्शनः}
{पाण्डुरैः पुण्डरीकाक्षः प्रासादैरुपशोभितम्}


\twolineshloka
{तिस्रः कक्ष्या व्यतिक्रम्य केशवो राजवेश्मनः}
{वैचित्रवीर्यं राजानमभ्यगच्छदरिन्दमः}


\twolineshloka
{अभ्यागच्छति दाशार्हे प्रज्ञाचक्षुर्नराधिपः}
{तहैव द्रोणभीष्माभ्यामुदतिष्ठन्महायशाः}


\twolineshloka
{कृपश्च सोमदत्तश्च महाराजश्च बाह्लिकः}
{आसनेभ्योऽचलन्सर्वे पूजयन्तो जनार्दनम्}


\twolineshloka
{ततो राजानमासाद्य धृतराष्ट्रं यशस्विनम्}
{सभीष्मं पूजयामास वार्ष्णेयो वाग्भिरञ्जसा}


\twolineshloka
{तेषु धर्मानुपूर्वी तां प्रयुज्य मधुसूदनः}
{यथावयः समीयाय राजभिः सह माधवः}


\twolineshloka
{अथ द्रोणं सबाह्लीकं सपुत्रं च यशस्विनम्}
{कृपं च सोमदत्तं च समीयाय जनार्दनः}


\twolineshloka
{तत्रासीदूर्जितं मृष्टं काञ्चनं महदासनम्}
{शासनाद्धृतराष्ट्रस्य तत्रोपाविशदच्युतः}


\twolineshloka
{अथ गां मधुपर्कं चाप्युदकं च जनार्दने}
{उपजह्नुर्यथान्यायं धृतराष्ट्रपुरोहिताः}


\twolineshloka
{कृतातिथ्यस्तु गोविन्दः सर्वान्परिहसन्कुरून्}
{आस्ते सांबन्धिकं कुर्वन्कुरुभिः परिवारितः}


\twolineshloka
{सोऽर्चितो धृतराष्ट्रेण पूजितश्च महायशाः}
{राजानं समनुज्ञाप्य निरक्रामदरिन्दमः}


\twolineshloka
{तैः समेत्य यथान्यायं कुरुभिः कुरुसंसदि}
{विदुरावसथं रम्यमुपातिष्ठत माधवः}


\twolineshloka
{विदुरः सर्वकल्याणैरभिगम्य जनार्दनम्}
{अर्चयामास दाशार्हं सर्वकामैरुपस्थितम्}


\twolineshloka
{कृतातिथ्यं तु गोविन्दं विदुरः सर्वधर्मवित्}
{कुशलं पाण्डुपुत्राणामपृच्छन्मधुसूदनम्}


\twolineshloka
{प्रीयमाणस्य सुहृदो विदुरो बुद्धिसत्तमः}
{धर्मार्थनित्यस्य सतो गतरोवस्य धीमतः}


\twolineshloka
{तस्य सर्वं सविस्तारं पाण्डवानां विचेष्टितम्}
{क्षत्तुराचष्ट दाशार्हः सर्वं प्रत्यक्षदर्शिवान्}


\chapter{अध्यायः ९०}
\twolineshloka
{वैशंपायन उवाच}
{}


\twolineshloka
{अथोपगम्य विदुरमपराह्णे जनार्दनः}
{पितृष्वसारं स पृथामभ्यगच्छदरिन्दमः}


\twolineshloka
{सा दृष्ट्वा कृष्णमायान्तं प्रसन्नादित्यवर्चसम्}
{कण्ठे गृहीत्वा प्राक्रोशत्स्मरन्ती तनयान्पृथा}


\twolineshloka
{तेषां सत्ववतां मध्ये गोविन्दं सहचारिणम्}
{चिरस्य दृष्ट्वा वार्ष्णेयं बाष्पमाहारयत्मृथा}


\twolineshloka
{साऽब्रवीत्कृष्णमासीनं कृतातिथ्यं युधां पतिम्}
{बाष्पगद्गदपूर्णेन मुखेन परिशुष्यता}


\threelineshloka
{एते बाल्यान्प्रभृत्येव गुरुशुश्रूषणे रताः}
{परस्परस्य सुहृदः संमताः समचेतसः}
{निकृत्या भ्रंशिता राज्याञ्जनार्हा निर्जनं गताः}


\twolineshloka
{विनीतक्रोधहर्षाश्च ब्रह्मण्याः सत्यवादिनः}
{त्यक्त्वा प्रियमुखे पार्था रुदतीमपहाय माम्}


\twolineshloka
{अहार्षुश्च वनं यान्तः समूलं हृदयं मम}
{अतदर्हा महात्मानः कथं केशव पाण्डवाः}


\twolineshloka
{ऊषुर्महावने तात सिंहव्याघ्रगजाकुले}
{बाला विहीनाः पित्रा ते मया सततलालिताः}


\twolineshloka
{अपश्यन्तश्च पितरौ कथमुषूर्महावने}
{शङ्खदुन्दुभिनिर्घोषैर्भृदङ्गैर्वेणुनिस्वनैः}


\twolineshloka
{पाण्डवाः समबोध्यन्त बाल्यात्प्रभृति केशव}
{ये स्म वारणशब्देन हयानां हेषितेन च}


\twolineshloka
{रथेनेमिनिनादैश्च व्यबोध्यन्त तदा गृहे}
{शङ्खभेरीनिनादेन वेणुवीणानुनादिना}


\twolineshloka
{पुण्याहघोषमिश्रेण पूज्यमाना द्विजातिभिः}
{वस्त्रै रत्नैरलङ्कारैः पूजयन्तो द्विजन्मनः}


\twolineshloka
{गीर्भिर्मङ्गलयुक्ताभिर्ब्राह्मणानां महात्मनाम्}
{अर्चितैरर्चनार्हैश्च स्तुवद्भिरभिनन्दिताः}


\twolineshloka
{प्रासादाग्नेष्वबोध्यन्त राङ्खवाजिनशायिनः}
{क्रूरं च निनदं श्रुत्वा श्वापदानां महावने}


\twolineshloka
{न स्मोपयान्ति निद्रां ते नतदर्हा जनार्दन}
{भेरीमृदङ्गनिनदैः शङ्खवैणवनिस्वनैः}


% Check verse!
स्त्रीणां गीतनिनादैश्च मधुरैर्मधुसूदन ॥वन्दिमागधसूतैश्च स्तुवद्भिर्बोधिताः कथम्
\twolineshloka
{महावनेष्वबोध्यन्त श्वापदानां रुतेन च}
{ह्रीमात्सत्यधृतिर्दान्तो भूतानामनुकम्पिता}


\twolineshloka
{कामद्वेषौ वशे कृत्वा सतां वर्त्मानुवर्तते}
{अम्बरीषस्य मान्धातुर्ययातेर्नहुषस्य च}


\twolineshloka
{भरतस्य दिलीपस्य शिबेरौशीनरस्य च}
{राजर्षीणां पुराणानां धुरं धत्ते दुरुद्वहाम्}


\twolineshloka
{शीलवृत्तोपसंपन्नोः धर्मज्ञः सत्यसङ्गरः}
{राजा सर्वगुणोपेतस्त्रौलोक्यस्यापि यो भवेत्}


\threelineshloka
{अजातशत्रुर्धर्मात्मा शुद्धजाम्बूनदप्रभः}
{श्रेष्ठः कुरुषु सर्वेषु धर्मतः श्रुतवृत्ततः}
{प्रियदर्शो दीर्घभुजः कथं कृष्ण युधिष्ठिरः}


\twolineshloka
{यः स नागायुतप्राणो वातरंहा महाबलः}
{सामर्षः पाण्डवो नित्यं प्रियो भ्रातुः प्रियंकरः}


\twolineshloka
{कीचकस्य तु सज्ञातेर्यो हन्ता मधुसूदन}
{शूरः क्रोधवशानां च हिडिम्बस्य बकस्य च}


\twolineshloka
{पराक्रमे शक्रसमो मातरिश्वसमो बले}
{महेश्वरसमः क्रोधे भीमः प्रहरतां वरः}


\twolineshloka
{क्रोधं बलममर्षं च यो निधाय परंतपः}
{जितात्मा पाण्डवोऽमर्षी भ्रातुस्तिष्ठति शासने}


\twolineshloka
{तेजोराशिं महात्मानं वरिष्ठमभितौजसम्}
{भीमं प्रदर्शनेनापि भीमसेनं जनार्दन}


\twolineshloka
{तं ममाचक्ष्व वार्ष्णेय कथमद्य वृकोदरः}
{आस्ते परिघबाहुः स मध्यमः पाण्डवो बली}


\twolineshloka
{अर्जुनेनार्जुनो यः स कृष्ण बाहुसहस्रिणा}
{द्विबाहुः स्पर्धते नित्यमतीतेनापि केशव}


\twolineshloka
{क्षिपत्येकेन वेगेन पञ्चबाणशतानि यः}
{इष्वस्त्रे सदृशो राज्ञः कार्तवीर्यस्य पाण्डवः}


\twolineshloka
{तेजसाऽऽदित्यसदृशो महर्षिसदृशो दमे}
{क्षमया पृथिवीतुल्यो महेन्द्रसमविक्रमः}


\twolineshloka
{आधिराज्यं महद्दीप्तं प्रथितं मधुसूदन}
{आहृतं येन वीर्येण कुरूणां सर्वराजसु}


\twolineshloka
{यस्य बाहुबलं सर्वे पाण्डवाः पर्युपासते}
{स सर्वरथिनां श्रेष्ठः पाण्डवः सत्यविक्रमः}


\twolineshloka
{यं गत्वाऽभिमुखः सङ्ख्ये न जीवन्तश्चिदाव्रजेत्}
{यो जेता सर्वभूतानामजेयो जिष्णुरच्युत}


\twolineshloka
{योऽपाश्रयः पाण्डवानां देवानामिव वासवः}
{स ते भ्राता सखा चैव कथमद्य धनञ्जयः}


\twolineshloka
{दयावान्सर्वभूतेषु ह्रीनिषेवो महास्रवित्}
{मृदुश्च सुकुमारश्च धार्मिकश्च प्रियश्च मे}


\twolineshloka
{सहदेवो महेष्वासः शूरः समितिशोभनः}
{भ्रातृणां कृष्ण शुश्रूषुर्धर्मार्थकुशलो युवा}


\twolineshloka
{सदैव सहदेवस्य भ्रातरो मधुसूदन}
{वृत्तं कल्याणवृत्तस्य पूजयन्ति महात्मनः}


\twolineshloka
{ज्येष्ठोपचायिनं वीरं सहदेवं युधां पतिम्}
{शुश्रूषुं मम वार्ष्णेय माद्रीपुत्रं प्रचक्ष्व मे}


\twolineshloka
{सुकुमारो युवा शूरो दर्शनीयश्च पाण्डवः}
{भ्रातॄणां चैव सर्वेषां प्रियः प्राणो बहिश्चरः}


\twolineshloka
{चित्रयोधी च नकुलो महेष्वासो महाबलः}
{कच्चित्स कुशली कृष्ण वत्सो मम सुखैधितः}


\twolineshloka
{सुखोचितमदुःखार्हं सुकुमारं महारथम्}
{अपि जातु महाबाहो पश्येयं नकुलं पुनः}


\twolineshloka
{पक्ष्मसंपातजे काले नकुलेन विनाकृता}
{न लभामि धृतिं वीर साऽद्य जीवामि पश्य माम्}


\twolineshloka
{सर्वैः पुत्रैः प्रियतरा द्रौपदी मे जनार्दन}
{कुलीना रूपसंपन्ना सर्वैः समुदिता गुणैः}


\twolineshloka
{पुत्रलोकात्पतिलोकं वृण्वाना सत्यवादिनी}
{प्रियान्पुत्रान्परित्यज्य पाण्डवाननुरुध्यते}


\twolineshloka
{महाभिजनसंपन्ना सर्वकामैः सुपूजिता}
{ईश्वरी सर्वकल्याणी द्रौपदी कथमच्युत}


\twolineshloka
{पतिभिः पञ्चभिः शूरैरग्निकल्पैः प्रहारिभिः}
{उपपन्ना महेष्वासैर्द्रौपदी दुःखभागिनी}


\twolineshloka
{चतुर्दशमिदं वर्षं यन्नापश्यमरिन्दम}
{पुत्रादिभिः परिद्यूनां द्रौपदीं सत्यवादिनीम्}


\twolineshloka
{न नूनं कर्मभिः पुण्यैरश्रुते पुरुषः सुखम्}
{द्रौपदी चेत्तथावृत्ता नाश्रुते सुखमव्ययम्}


\twolineshloka
{न प्रियो मम कृष्णाया बीभत्सुर्न युधिष्ठिरः}
{भीमसेनो यमौ वापि यदपश्यं सभागताम्}


\twolineshloka
{न मे दुःखतरं किंचिद्भूतपूर्वं ततोऽधिकम्}
{स्त्रीधर्मिणीं द्रौपदीं यच्छ्वशुराणां समीपगाम्}


\twolineshloka
{आनायितामनार्येण क्रोधलोभानुवर्तिना}
{सर्वे प्रैक्षन्त कुरव एकवस्त्रां सभागताम्}


\twolineshloka
{तत्रैव धृतराष्ट्रश्च महाराजश्च बाह्लिकः}
{कृपश्च कसोमदत्तश्च निर्विष्णाः कुरवस्तथा}


\twolineshloka
{तस्यां संसदि सर्वेषां क्षत्तारं पूजयाम्यहम्}
{वृत्तेन हि भवत्यार्यो न धनेन न विद्यया}


\threelineshloka
{तस्य कृष्ण महाबुद्धेर्गम्भीरस्य महात्मनः}
{क्षत्तुःक शीलमलङ्कारो लोकान्विष्टभ्य तिष्ठति ॥वैशंपायन उवाच}
{}


\twolineshloka
{सा शोकार्ता च हृष्टा च दृष्ट्वा गोविन्दमागतम्}
{नानाविधानि दुःखानि सर्वाण्येवान्वकीर्तयम्}


\twolineshloka
{पूर्वैराचरितं यत्तत्कुराजभिररिन्दम}
{अक्षद्यूतं मृगवधः कच्चिदेषां सुखावहम्}


\twolineshloka
{तन्मां दहति यत्कृष्णा सभायां कुरुसन्निधौ}
{धार्तराष्ट्रैः परिक्लिष्टा यथा न कुशलं तथा}


\twolineshloka
{निर्वासनं च नगरात्प्रव्रज्या च परन्तप}
{नानाविधानां दुःखानामावासोऽस्मि जनार्दन}


\twolineshloka
{अज्ञातचर्या बालानामवरोधश्च माधव}
{न मे क्लेशतमं तत्स्यात्पुत्रैः सह परन्तप}


\twolineshloka
{दुर्योधनेन निकृता वर्षमद्य चतुर्दशम्}
{दुःखादपि सुखं नः स्याद्यदि पुण्यफलक्षयः}


\threelineshloka
{न मे विशेपो जात्वासीद्धार्तराष्ट्रेषु पाण्डवैः}
{तेन सत्येन कृष्ण त्वां हतामित्रं श्रिया वृतम्}
{अस्माद्विमुक्तं सङ्ग्रामात्पश्येयं पाण्डवैः सह}


\twolineshloka
{नैव शक्याः पराजेतुं सर्वं ह्येषां तथाविधम्}
{पितरं त्वेव गर्हेयं नात्मानं न सुयोधनम्}


\twolineshloka
{येनाहं कुन्तिभोजाय धनं वृत्तैरिवार्पिता}
{बालां मामार्यकस्तुभ्यं क्रीडन्तीं कन्दुहस्तिकां}


\threelineshloka
{अदात्तु कुन्तिभोजाय सखा सख्ये महात्मने}
{साऽहं पित्रा च निकृता श्वशुरैश्च परन्तप}
{अत्यन्तदुःखिता कृष्ण किं जीवितफलं मम}


\threelineshloka
{यन्मां वागब्रवीन्नक्तं सूतके सव्यसाचिनः}
{पुत्रस्ते पृथिवीं जेता यशश्चास्य दिवं स्पृशेत्}
{}


\twolineshloka
{हत्वा कुरून्महाजन्ये राज्यं प्राप्य धनञ्जयः}
{भ्रातृभिः सह कौन्तेयस्त्रीन्मेधानाहरिष्यति}


\twolineshloka
{नाहं तामभ्यसूयामि नमो धर्माय वेधसे}
{कृष्णाय महते नित्यं धर्मो धारयति प्रजाः}


\twolineshloka
{धर्मश्चेदस्ति वार्ष्णेय यथा वागभ्यभाषत}
{त्वं चापि तत्तथा कृष्ण सर्वं संपादयिष्यसि}


\twolineshloka
{न मां माधव वैधव्यं नार्थनाशो न वैरिता}
{तथा शोकाय दहति यथा पुत्रैर्विना भवः}


\threelineshloka
{याऽहं गाण्डीवधन्वानं सर्वशस्त्रभृतां वरम्}
{धनञ्जयं न पश्यामि का शान्तिर्हृदयस्य मे}
{इतश्चतुर्दशं वर्षं यन्नापश्यं युधिष्ठिरम्}


\twolineshloka
{धनञ्जयं च गोविन्द यमौ तं च वृकोदरम्}
{जीवनाशं प्रनष्टानां श्राद्धं कुर्वन्ति मानवाः}


\twolineshloka
{अर्थतस्ते मम मृतास्तेषां चाहं जनार्दना}
{ब्रूया माधव राजानं धर्मात्मानं युधिष्ठिरम्}


\twolineshloka
{भूयांस्ते हीयते धर्मो मा पुत्रक वृथा कृथाः}
{पराश्रया वासुदेव या जीवति घिगस्तु ताम्}


\twolineshloka
{वृत्तेः कार्पण्यलब्धाया अप्रतिष्ठैव ज्यायसी}
{अथो धनञ्जयं ब्रूया नित्योद्युक्तं वृकोदरम्}


\twolineshloka
{यदर्थं क्षत्रिया सूते तस्य कालोऽयमागतः}
{अस्मिंश्चेदागते काले मिथ्या चातिक्रमिष्यति}


\twolineshloka
{लोकसंभाविताः संतः सुनृशंसं करिष्यथ}
{नृशंसेन च वो युक्तांस्त्यजेयं शाश्वतीः समाः}


\twolineshloka
{काले हि समनुप्राप्ते त्यक्तव्यमपि जीवनम्}
{माद्रीपुत्रौ च व्यक्तव्यौ क्षत्रधर्मरतौ सदा}


\twolineshloka
{विक्रमेणार्जितान्भोगान्वृणीतं जीवितादपि}
{विक्रमाधिगता ह्यर्थाः क्षत्रधर्मेण जीवतः}


\twolineshloka
{मनो मनुष्यस्य सदा प्रीणन्ति पुरुषोत्तम}
{गत्वा ब्रूहि महाबाहो सर्वशस्त्रभृतां वरम्}


\twolineshloka
{अर्जुनं पाण्डवं वीरं द्रौपद्याः पदवीं चर}
{विदितौ हि तवात्यन्तं क्रुद्धौ तौ तु यथान्तकौ}


\twolineshloka
{भीमार्जुनौ नयेतां हि देवानपि परां गतिम्}
{तयोश्चैतदवज्ञानं यत्सा कृष्णा सभां गता}


\twolineshloka
{दुःशासतश्च कर्णश्च परुषाण्यभ्यभाषताम्}
{दुर्योधनो भीमसेनमभ्यगच्छन्मनस्विनम्}


\twolineshloka
{पश्यतां कुरुमुख्यानां तस्य द्रक्ष्यति यत्फलम्}
{न हि वैरं समासाद्य प्रशाम्यति वृकोदरः}


\twolineshloka
{सुचिरादपि भीमस्य न हि वैरं प्रशाम्यति}
{यावदन्तं न नयति शात्रवाञ्चत्रुकर्शनः}


\twolineshloka
{न दुःखं राज्यहरणं न च द्यूते पराजयः}
{प्रव्राजनं तु पुत्राणां न मे तद्दुःखकारणम्}


\twolineshloka
{यत्तु सा बृहती श्यामा एकवस्त्रा सभां गता}
{अशृणोत्परुषा वाचः किं नु दुःखतरं ततः}


\twolineshloka
{स्त्रीधर्मिणी वरारोहा क्षत्रधर्मरता सदा}
{नाभ्यगच्छत्तदा नाथं कृष्णा नाथवती सती}


\twolineshloka
{यस्या मम सपुत्रायास्त्वं नाथो मधुसूदन}
{रामश्च बलिनां श्रेष्ठः प्रद्युम्नश्च महारथः}


\threelineshloka
{साऽहमेवंविधं दुःखं सहेयं पुरुषोत्तम}
{भीमे जीवति दुर्धर्षे विजये चापलायिनि ॥वैशंपायन उवाच}
{}


\threelineshloka
{तत आश्वासयामास पुत्राधिभिरभिप्लुताम्}
{पितृष्वसारं शोचन्तीं शौरिः पार्थसखः पृथाम् ॥वासुदेव उवाच}
{}


\twolineshloka
{का नु सीमन्तिनी त्वादृग्लोकेष्वस्ति पितृष्वसः}
{शूरस्य राज्ञो दुहिता आजमीढकुलं गता}


\twolineshloka
{महाकुलीना भवती ह्रदाद्ध्रदमिवागता}
{ईश्वरी सर्वकल्याणी भर्त्रा परमपूजिता}


\twolineshloka
{वीरसूर्वीरपत्नी त्वं सर्वैः समुदिता गुणैः}
{मुखदुःखे महाप्राज्ञे त्वादृशी सोढुमर्हति}


\twolineshloka
{निद्रातन्द्रे क्रोधहर्षौ क्षुत्पिपासे हिमातपौ}
{एतानि पार्था निर्जित्य नित्यं वीरसुखे रताः}


\twolineshloka
{त्यक्तग्राम्यसुखाः पार्था नित्यं वीरसुखप्रियाः}
{न तु स्वल्पेन तुष्येयुर्महोत्साहा महाबलाः}


\twolineshloka
{अन्तं धीरा निषेवन्ते मध्यं ग्राम्यसुखप्रियाः}
{उत्तमांश्च परिक्लेशान्भोगांश्चातीव मानुषान्}


\twolineshloka
{अन्तेषु रेमिरे धीरा न ते मध्येषु रेमिरे}
{अन्तप्राप्तिं सुखं प्राहुर्दुःखमन्तरमन्तयोः}


\twolineshloka
{अभिवादयन्ति भवतीं पाण्डवाः सह कृष्णया}
{आत्मानं ते कुशलिनं निवेद्याहुरनामयम्}


\twolineshloka
{अरोगान्सर्वसिद्धार्थान्क्षिप्रं द्रक्ष्यसि पाण्डवान्}
{ईश्वरान्सर्वलोकस्य हतामित्राञ्श्रिया वृतान्}


\threelineshloka
{एवमाश्वासिता कुन्ती प्रत्युवाच जनार्दनम्}
{पुत्राधिभिरभिध्वस्ता निगृह्याबुद्धिजं तमः ॥कुन्त्युवाच}
{}


\twolineshloka
{यद्यत्तेषां महाबाहो पथ्यं स्यान्मधुसूदन}
{यथायथा त्वं मन्येथाः कुर्याः कृष्ण तथातथा}


\twolineshloka
{अविलोपेन धर्मस्य अनिकृत्या परन्तप}
{प्रभावज्ञाऽस्मि ते कृष्ण सत्यस्याभिजनस्य च}


\twolineshloka
{व्यवस्थायां च मित्रेषु बुद्धिविक्रमयोस्तथा}
{त्वमेव नः कुले धर्मस्त्वं सत्यं त्वं तपो महत्}


\twolineshloka
{त्वं त्राता त्वं महद्ब्रह्म त्वयि सर्वं प्रतिष्ठितम्}
{यथैवात्थ तथैवैतत्त्वयि सत्यं भविष्यति}


\fourlineindentedshloka
{` कुरूणां पाण्डवानां च लोकानां चापराजित}
{सर्वस्यैतस्य वार्ष्णेय गतिस्त्वमसि माधव}
{प्रभावो बुद्धिवीर्यं च तादृशं तव केशव ॥' वैशंपायन उवाच}
{}


\twolineshloka
{तामामन्त्र्य च गोविन्दः कृत्वा चाभिप्रदक्षिणम्}
{प्रातिष्ठत महाबाहुर्दुर्योधनगृहान्प्रति}


\chapter{अध्यायः ९१}
\twolineshloka
{वैशंपायन उवाच}
{}


\twolineshloka
{पृथामामन्त्र्य गोविन्दः कृत्वा चाभिप्रदक्षिणम्}
{दुर्योधनगृहं शौरिरभ्यगच्छदरिन्दमः}


\twolineshloka
{लक्ष्म्या परमया युक्तं पुरन्दरगृहोपमम्}
{विचित्रैरासनैर्युक्तं प्रविवेश जनार्दनः}


\twolineshloka
{तस्य कक्ष्या व्यतिक्रम्य तिस्रो द्वाःस्थैरवारितः}
{ततोऽभ्रघनसङ्काशं गिरिकूटमिवोच्छ्रितम्}


\twolineshloka
{श्रिया ज्वलन्तं प्रासादमारुरोह महायशाः}
{तत्र राजसहस्रैश्च कुरुभिश्चाभिसंवृतम्}


\twolineshloka
{धार्तराष्ट्रं महाबाहुं ददर्शासीनमासने}
{दुःशासनं च कर्णं च शकुनिं चापि सौबलम्}


\twolineshloka
{दुर्योधनसमीपे तानासनस्थान्ददर्श सः}
{अभ्यागच्छति दाशार्हे धार्तराष्ट्रो महायशाः}


\twolineshloka
{उदतिष्ठत्सहामात्यः पूजयन्मधुसूदनम्}
{समेत्य धार्तराष्ट्रेण महामात्येन केशवः}


\twolineshloka
{राजभिस्तत्र वार्ष्णेयः समागच्छद्यथावयः}
{तत्र जाम्बूनदमयं पर्यङ्कं सुपरिष्कृतम्}


\twolineshloka
{विविधास्तरणास्तीर्णमभ्युपाविशदच्युतः}
{तस्मिन्गां मधुपर्कं चाप्युदकं च जनार्दने}


\twolineshloka
{निवेदयामास तदा गृहान्राज्यं च कौरवः}
{5-91-10b` आसनंसर्वतोभद्रं सर्वरत्नविभूषितम्}


\twolineshloka
{कृष्णार्थमेवं संसिद्धं धार्तराष्ट्रस्य शासनात्}
{'तत्र गोविन्दमासीनं प्रसन्नादित्यवर्चसम्}


\twolineshloka
{उपासाञ्चक्रिरे सर्वे कुरवो राजभिः सह}
{ततो दुर्योधनो राजा वार्ष्णेयं जयतां वरम्}


\twolineshloka
{न्यमन्त्रयद्भोजनेन नाभ्यनन्दच्च केशवः}
{ततो दुर्योधनः कृष्णमब्रवीत्कुरुसंसदि}


\twolineshloka
{मृदुपूर्वं शठोदर्कं कर्णमाभाष्य कौरवः}
{कस्मादन्नानि पापानि वासांसि शयनानि च}


\twolineshloka
{त्वदर्थमुपनीतानि नाग्रहीस्त्वं जनार्दन}
{उभयोश्च ददत्साह्यमुभयोश्च हिते रतः}


\fourlineindentedshloka
{संबन्धी दयितश्चासि धृतराष्ट्रस्य माधव}
{त्वं हि गोविन्द धर्मार्थौ वेत्थ तत्त्वेन सर्वशः}
{तत्र कारणमिच्छामि श्रोतुं चक्रगदाधर ॥वैशंपायन उवाच}
{}


\twolineshloka
{स एवमुक्तो गोविन्दः प्रत्युवाच महामनाः}
{उद्यन्मेघस्वनःक काले प्रगृह्य विपुलं भुजम्}


\twolineshloka
{अलघूकृतमग्रस्तमनिरस्तमसंकुलम्}
{राजीवनेत्रो राजानं हेतुमद्वाक्यमुत्तमम्}


\twolineshloka
{कृतार्था भुञ्जते दूताः पूजां गृह्णन्ति चैव ह}
{कृतर्थं मां सहामात्यं समर्चिष्यसि भारत}


\twolineshloka
{एवमुक्तः प्रत्युवाच धार्तराष्ट्रो जनार्दनम्}
{न युक्तं भवताऽस्मासु प्रतिपत्तुमसांप्तम्}


\twolineshloka
{कृतार्थं वाऽकृतार्थं वा त्वां वयं मधुसूदन}
{यतामहे पूजयितुं दाशार्ह न च शक्नुमः}


\twolineshloka
{न च तत्कारणं विद्मो यस्मान्नो मधुसूदन}
{पूजां कृतां प्रीयमाणो नामंस्थाः पुरुषोत्तम}


\threelineshloka
{वैरं नो नास्ति भवता गोविन्द न च विग्रहः}
{स भवान्प्रसमीक्ष्यैतन्नेदृशं वक्तुमर्हति ॥वैशंपायन उवाच}
{}


\twolineshloka
{एवमुक्तः प्रत्युवाच धार्तराष्ट्रं जनार्दनः}
{अभिवीक्ष्य सहामात्यं दाशार्हः प्रहसन्निव}


\twolineshloka
{नाहं कामान्न संरम्भान्न द्वेषान्नार्थकारणात्}
{न हेतुवादाल्लोभाद्वा धर्मं जह्यां कथंचन}


\twolineshloka
{संप्रीतिभोज्यान्यन्नानि आपद्भोज्यानि वा पुनः}
{न च संप्रीयते राजन्न चैवापद्गता वयम्}


\twolineshloka
{`द्विषदन्नं न भोक्तव्यं द्विषन्तं नैव भोजयेत्}
{पाण्डवान्द्विषसे राजन्मम प्राणा हि पाण्डवाः ॥'}


\twolineshloka
{अकस्माद्द्विषसे राजञ्जन्मप्रभृति पाण्डवान्}
{प्रियानुवर्तिनो भ्रातॄन्सर्वैः समुदितान्गुणैः}


\twolineshloka
{अकस्माच्चैव पार्थानां द्वेषणं नोपपद्यते}
{धर्मे स्थिताः पाण्डवेयाः कस्तान्किं वक्तुमर्हति}


\twolineshloka
{यस्तान्द्वेष्टि स मां द्वेष्टि यस्ताननु समामनु}
{ऐकात्म्यं मां गतं विद्धि पाण्डवैर्धर्मचारिभिः}


\twolineshloka
{कामक्रोधानुवर्ती हि यो मोहाद्विरुरुत्सति}
{गुणवन्तं च यो द्वेष्टि तमाहुः पुरुषाधमम्}


\twolineshloka
{यः कल्याणगुणाञ्ज्ञातीन्मोहाल्लोभाद्दिदृक्षते}
{सोजितात्माऽजितक्रोधो न चिरं तिष्ठति श्रिया}


\twolineshloka
{अथ यो गुणसंपन्नान्हृदयस्याप्रियानपि}
{प्रियेण कुरुते वश्यांश्चिरं यशसि तिष्ठति}


\twolineshloka
{सर्वमेतन्न भोक्तव्यमन्नं दुष्टाभिसंहितम्}
{क्षुत्तुरेकस्य भोक्तव्यमिति मे धीयते मतिः}


\twolineshloka
{एवमुक्त्वा महाबाहुर्दुर्योधनममर्षणम्}
{निश्चक्राम ततः शुभ्राद्धार्तराष्ट्रनिवेशनात्}


\twolineshloka
{निर्याय च महाबाहुर्वासुदेवो महामनाः}
{निवेशाय ययौ वेश्य विदुरस्य महात्मनः}


\twolineshloka
{तमभ्यगच्छद्द्रोणश्च कृपो भीष्मोऽथ बाह्लिकः}
{कुरवश्च महाबाहुं विदुरस्य गृहे स्थितम्}


\twolineshloka
{त ऊचुर्माधवं वीरं कुरवो मधुसूदनम्}
{निवेदयामो वार्ष्णेय सरत्नांस्ते गृहान्वयम्}


\threelineshloka
{तानुवाच महातेजाः कौरवान्मधुसूदनः}
{सर्वे भवन्ते गच्छन्तु सर्वा मेऽपचितिः कृता ॥वैशंपायन उवाच}
{}


\twolineshloka
{यातेषु कुरुषु क्षत्ता दाशार्हमपराजितम्}
{अभ्यर्चयामास तदा सर्वकामैः प्रयत्नवान्}


\twolineshloka
{ततः क्षत्राऽन्नपानानि शुचीनि गुणवन्ति च}
{उपाहरदनेकानि केशवाय महात्मने}


\twolineshloka
{तैतर्पयित्वा प्रथमं ब्राह्मणान्मधुसूदनः}
{वेदविद्भ्यो ददौ कृष्णः परमद्रविणान्यपि}


\twolineshloka
{`भुक्तवत्सु द्विजाग्र्येषु निषण्णेषु वरासने}
{शुचिः सुप्रयतो भून्वा विदुरोऽन्नमुपारत्}


\fourlineindentedshloka
{श्रद्धया परया युक्त इदं वचनमब्रवीत्}
{संवृतैस्तुष्य गोविन्द एतन्नः धनम्}
{अन्यथा हि विशेषेण कस्त्वामर्चितुमर्हति ॥वैशंपायन उवाच}
{}


\twolineshloka
{ततोऽनुयायिभिः सार्धं मरुद्भिरिव वासवः}
{विदुरान्नानि बुभुजे शुचीनि गुणवन्ति च}


\twolineshloka
{तं भुक्तवन्तं विविधाः सुशब्दाः सूतमागधाः}
{अभितुष्टुवुरासीनं दाशार्हमपराजितम्}


\chapter{अध्यायः ९२}
\twolineshloka
{वैशंपायन उवाच}
{}


\twolineshloka
{तं भुक्तवन्तमाश्वस्तं निशायां विदुरोऽब्रवीत्}
{नेदं सम्यग्व्यवसितं केशवागमनं तव}


\twolineshloka
{अर्थधर्मातिगो मन्दः संरम्भी च जनार्दन}
{मानघ्नो मानकामश्च वृद्धानां शासनातिगः}


\twolineshloka
{धर्मशास्त्रातिगो मूढो दुरात्मा प्रग्रहं गतः}
{अनेयः श्रेयसां मन्दो धार्तराष्ट्रो जनार्दन}


\twolineshloka
{कामात्मा प्राज्ञमानी च मित्रध्रुक्सर्वशङ्किता}
{अकर्ता चाकृतज्ञश्च त्यक्तधर्मा प्रियानृतः}


\twolineshloka
{मूढश्वाकृतबुद्धिश्च इन्द्रियाणामनीश्वरः}
{कामानुसारी कृत्येषु सर्वेष्वकृतनिश्चयः}


\twolineshloka
{एतैश्चान्यैश्च बहुभिर्दोषैरेव समन्वितः}
{त्वयोच्यमानः श्रेयोऽपि संरम्भान्न ग्रहीष्यति}


\twolineshloka
{भीष्मे द्रोणे कृपे कर्णे द्रोणपुत्रे जयद्रथे}
{भूयसीं वर्तते वृत्तिं न शमे कुरुते मनः}


\twolineshloka
{निश्चितं धार्तराष्ट्राणां सकर्णानां जनार्दन}
{भीष्मद्रोणमुखान्पार्था न शक्ताः प्रतिवीक्षितुम्}


\twolineshloka
{सेनासमुदयं कृत्वा पार्थिवं मधुसूदन}
{कृतार्थं मन्यते बाल आत्मानमविचक्षणाः}


\twolineshloka
{एकः कर्णः पराञ्जेतुं समर्थ इति निश्चितम्}
{धार्तराष्ट्रस्य दुर्बुद्धेः स शमं नोपयास्यति}


\twolineshloka
{संविच्च धार्तराष्ट्राणां सर्वेषामेव केशव}
{शमे प्रयतमानस्य तव सौभ्रात्रकाङ्क्षिणः}


\twolineshloka
{न पाण्डवानामस्माभिः प्रतिदेयं यथोचितम्}
{इति व्यवसितास्तेषु वचनं स्यान्निरर्थकम्}


\twolineshloka
{यत्र सूक्तं दुरुक्तं च समं स्यान्मधुसूदन}
{न तत्र प्रलपेत्प्राज्ञो बधिरेष्विव गायनः}


\twolineshloka
{अविजानत्सु मूढेषु निर्मर्यादेषु माधव}
{तत्त्वं वाक्यं ब्रुवन्निन्द्यश्चण्डालेषु द्विजो यथा}


\twolineshloka
{सोऽयं बलस्थो मूढश्चन करिष्यति ते वचः}
{तस्मिन्निरर्थकं वाक्यमुक्तं संपत्स्यते तव}


\twolineshloka
{तेषां समुपविष्टानां सर्वेषां पापचेतसाम्}
{तव मध्यावतरणं मम कृष्ण न रोचते}


\twolineshloka
{दुर्बुद्धीनामशिष्टानां बहूनां दुष्टचेतसाम्}
{प्रतीपं वचनं मध्ये तव कृष्ण न रोचते}


\twolineshloka
{अनुपासितवृद्धत्वाच्छ्रियो दर्पाच्च मोहितः}
{वयोदर्पादमर्षाच्च न ते श्रेयो ग्रहीष्यति}


\twolineshloka
{बलं बलवदप्यस्य यदि वक्ष्यसि माधव}
{त्वय्यस्य महती शङ्का न करिष्यति ते वचः}


\twolineshloka
{नेदमद्य युधा शक्यमिन्द्रेणापि सहामरैः}
{इति व्यवसिताः सर्वे धार्तराष्ट्रा जनार्दन}


\twolineshloka
{तेष्वेवमुपपन्नेषु कामक्रोधानुवर्तिषु}
{समर्थमपि ते वाक्यमसमर्थं भविष्यति}


\twolineshloka
{मध्ये तिष्ठन्हस्त्यनीकस्य मन्दोरथाश्वयुक्तस्य बलस्य मूढः}
{दुर्योधनो मन्यते बीतभीतिःकृत्स्ना मयेयं पृथिवी जितेति}


\twolineshloka
{आशंसते वै धृतराष्ट्रस्य पुत्रोमहाराज्यमसपत्नं पृथिव्याम्}
{तस्मिञ्शमः केवलो नोपलभ्योबद्धं सन्तं मन्यते लब्धमर्थम्}


\twolineshloka
{पर्यस्तेयं पृथिवी कालपक्वादुर्योधनार्थे पाण्डवान्योद्धुकामाः}
{समागताः सर्वयोधाः पृथिव्यांराजानश्च क्षितिपालैः समेताः}


\twolineshloka
{सर्वे चैते कृतवैराः पुरस्ता-त्त्वया राजानो हृतसाराश्च कृष्ण}
{तवोद्वेगात्संश्रिता धार्तराष्ट्रा-न्सुसंहताः सह कर्णेन वीराः}


\twolineshloka
{त्यक्तात्मानः सह दुर्योधनेनहृष्टा योद्धुं पाण्डवान्सर्वयोधाः}
{` मृत्युर्जयो वेति कृतैकभावाःकामात्मानो मन्युवशाविनीताः ॥'तेषां मध्ये प्रविशेथा यदि त्वंन तन्मतं मम दाशार्हवीर}


\twolineshloka
{तेषां समुपविष्टानां बहूनां दुष्टचेतसाम्}
{कथं मध्यं प्रपद्येथाः शत्रूणां शत्रुकर्शन}


\twolineshloka
{सर्वथा त्वं महाबाहो देवैरपि दुरुत्सहः}
{प्रभावं पौरुषं बुद्धिं जानामि तव शत्रुहन्}


\twolineshloka
{या मे प्रीतिः पाण्डवेषु भूयःक सा त्वयि माधव}
{प्रेम्णा च बहुमानाच्च सौहृदाच्च ब्रवीम्यहम्}


\twolineshloka
{या मे प्रीतिः पुष्कराक्ष त्वद्दर्शनसमुद्भवा}
{सा किमाख्यायते तुभ्यमन्तरात्माऽसि देहिनां}


\chapter{अध्यायः ९३}
\twolineshloka
{` वैशंपायन उवाच}
{}


\threelineshloka
{विदुरस्य वचः श्रुत्वा प्रश्रितं पुरुषोत्तमः}
{इदं होवाच वचनं मधुरं मधुसूदनः ॥'श्रीभगवानुवाच}
{}


\twolineshloka
{यथा व्रूयान्महाप्राज्ञो यथा ब्रूयाद्विचक्षणः}
{सथा वाच्यस्त्वद्विधेन भवता मद्विधः सुहृत्}


\twolineshloka
{धर्मार्थयुक्तं तथ्यं च यथा त्वय्युपपद्यते}
{तथा वचनमुक्तोऽस्ति त्वयैतत्पितृमातृवत्}


\twolineshloka
{सत्यं प्राप्तं च युक्तं वाऽप्येवमेव यथाऽऽत्थ माम्}
{श्रृणुष्वागमने हेतुं विदुरावहितो मम}


\twolineshloka
{दौरात्म्यं धार्तराष्ट्रस्य क्षत्रियाणां च वैरिताम्}
{सर्वमेतदहं जानन्क्षत्तः प्राप्तोऽद्य कौरवान्}


\twolineshloka
{पर्यस्तां पृथिवीं सर्वां साश्वां सरथकुञ्जराम्}
{यो मोचयेन्मृत्युपाशात्प्राप्नुयाद्धर्ममुत्तमम्}


\twolineshloka
{धर्मकार्यं यतञ्शक्त्या नो चेत्प्राप्नोति मानवः}
{प्राप्तो भवति तत्पुण्यमत्र मे नास्ति संशयः}


\twolineshloka
{मनसा चिन्तयन्पापं कर्मणा नातिरोचयन्}
{न प्राप्नोति फलं तस्येत्येवं धर्मविदो विदुः}


\twolineshloka
{सोऽहं यतिष्ये प्रशमं क्षत्तः कर्तुममायया}
{कुरूणां सृञ्जयानां च सङ्ग्राने विनशिष्यताम्}


\twolineshloka
{सेयमापन्महाघोरा कुरुष्वेव समुत्थिता}
{कर्णदुर्योधनकृता सर्वे ह्येते तदन्वयाः}


\twolineshloka
{व्यसने क्लिश्यमानं हि यो मित्रं नाभिपद्यते}
{अनुनीय यथाशक्ति तं नृशंसं विदुर्बुधाः}


\twolineshloka
{आकेशग्रहणान्मित्रमकार्यात्संनिवर्तयन्}
{अवाच्यः कस्यचिद्भवति कृतयत्नो यथाबलम्}


\twolineshloka
{तत्समर्थं शुभं वाक्यं धर्मार्थसहितं हितम्}
{धार्तराष्ट्रः सहामात्यो ग्रहीतुं विदुरार्हति}


\twolineshloka
{हितं हि धार्तराष्ट्राणां पाण्डवानां तथैव च}
{पृथिव्यां क्षत्रियाणां च यतिष्येऽहममायया}


\twolineshloka
{हिते प्रयतमानं मां शङ्केद्दुर्योधनो यदि}
{हृदयस्य च मे प्रीतिरानृण्यं च भविष्यति}


\twolineshloka
{ज्ञातीनां हि मिथो भेदे यन्मित्रं नाभिपद्यते}
{सर्वयत्नेन माध्यस्थ्यं न तन्मित्रं विदुर्बुधाः}


\twolineshloka
{न मां ब्रूयुरधर्मिष्ठा मूढा ह्यसुहृदस्तथा}
{शक्तो नावारयत्कृष्णः संरब्धान्कुरुपाण्डवान्}


\twolineshloka
{उभयोः साधयन्नर्थमहामागत इत्युत}
{तत्र यत्नमहं कृत्वा गच्छेयं नृष्ववाच्यताम्}


\twolineshloka
{मम धर्मार्थयुक्तं हि श्रुत्वा वाक्यमनामयम्}
{न चेदादास्यते बालो दिष्टस्य वशमेष्यति}


\twolineshloka
{अहापयन्पाण्डवार्थं यथाव-च्छमं कुरूणां यदि चाचरेयम्}
{पुण्यं च मे स्याच्चरितं महात्म-न्मुच्येरंश्च कुरवो मृत्युपाशात्}


\twolineshloka
{अपि वाचं भाषमाणस्य काव्यांधर्मरामामर्थवतीमहिंस्राम्}
{अवेक्षेरन्धार्तराष्ट्राः शमार्थंमां च प्राप्तं कुरवः पूजयेयुः}


\threelineshloka
{न चापि मम पर्याप्ताः सहिताः सर्वपार्थिवाः}
{क्रुद्धस्य प्रमुखे स्थातुं सिंहस्येवेतरे मृगाः ॥वैशंपायन उवाच}
{}


\twolineshloka
{इत्येवमुक्त्वा वचनं वृष्णीनामृषभस्तदा}
{शयने सुखसंस्पर्शे शिश्ये यदुसुखावहः}


\chapter{अध्यायः ९४}
\twolineshloka
{वैशंपायन उवाच}
{}


\twolineshloka
{तथा कथयतोरेव तयोर्बुद्धिमतोस्तदा}
{शिवा नक्षत्रसंपन्ना सा व्यतीयाय शर्वरी}


\twolineshloka
{धर्मार्थकामयुक्ताश्च विचित्रार्थपदाक्षराः}
{शृण्वतो विविधा वाचो विदुरस्य महात्मनः}


\twolineshloka
{कथाभिरनुरूपाभी रक्तस्यामिततेजसः}
{अकामस्यैव कृष्णस्य सा व्यतीयाय शर्वरी}


% Check verse!
ततस्तु स्वरसंपन्ना बहवः सूतमागधाः ॥शङ्खदुन्दुमिनिर्घोषैः केशवं प्रत्यबोधयन्
\twolineshloka
{तत उत्थाय दाशार्हऋषभः सर्वसात्वताम्}
{सर्वमावश्यकं चक्रे प्रातः कार्यं जनार्दनः}


\twolineshloka
{कृतोदकानुजप्यः स हुताग्निः समलङ्कृतः}
{ततश्चादित्यमुद्यन्तमुपातिष्ठत माधवः}


\twolineshloka
{अथ दुर्योधनः कृष्मं शकुनिश्चापि सौबलः}
{सन्ध्यां तिष्ठन्तमभ्येत्य दाशार्हमपराजितम्}


\twolineshloka
{आचक्षेतां तु कृष्णस्य धृतराष्ट्रं सभागतम्}
{कुरूश्च भीष्मप्रमुखान्राज्ञः सर्वांश्च पार्थिवान्}


\twolineshloka
{त्वामर्थयन्ते गोविन्द दिवि शक्रमिवामराः}
{तावभ्यनन्दद्गोविन्दः साम्ना परमवल्गुना}


\twolineshloka
{ततो विमल आदित्ये ब्राह्मणेभ्यो जनार्दनः}
{ददौ हिरण्यं वासांसि गाश्चाश्चांश्च परन्तपः}


\twolineshloka
{विसृष्टवन्तं रत्नानि दाशार्हमपराजितम्}
{तिष्ठन्तमुपसङ्गम्य ववन्दे सारथिस्तदा}


\twolineshloka
{` तस्मै रथवरो युक्तः शुशुभे लोकविश्रुतः}
{वाजिभिः शैब्यसुग्रीवमेघपुष्पबलाहकैः}


\twolineshloka
{शैब्यस्तु शुकपत्राभः सुग्रीवः किंशुकप्रभः}
{मेघपुष्पो मेघवर्णः पाण्डरस्तु बलाहकः}


\twolineshloka
{दक्षिणं चावहच्छैब्यः सुग्रीवः सव्यतोऽवहत्}
{पृष्ठवाहौ रथस्यास्तां मेघपुष्पबलाहकौ}


\twolineshloka
{विश्वकर्मकृताऽऽपीडा रत्नजालविभूषिता}
{आश्रिता वै रथे तस्मिन्ध्वजयष्टिरशोभत}


\twolineshloka
{वैनतेयः स्थितस्तस्यां प्रभाकरमिव स्पृशन्}
{तस्य सत्ववतः केतौ भुजगारिरशोभत}


\twolineshloka
{तस्य कीर्तिमतस्तेन भास्वरेण विराजता}
{शुशुभे स्यन्दनश्रेष्ठः पतगेन्द्रेण केतुना}


\twolineshloka
{रश्मिजालैः पताकाभिः सौवर्णेन च केतुना}
{बभूव स रथश्रेष्ठः कालसूर्य इवोदितः}


\twolineshloka
{पक्षिध्वजवितानैश्च रुक्मजालकृताङ्गणैः}
{दण्डमार्गविभागैश्च सुकृतैर्विश्वकर्मणा}


\twolineshloka
{प्रवालमणिशोभैश्च मुक्तावैडूर्यशोभनैः}
{किङ्किणीशतसङ्घैश्च वालजालकृतान्तरैः}


\twolineshloka
{कार्तस्वरमयीभिश्च पद्मिनीभिरलङ्कृतः}
{शुशुभे स्यन्दनश्रेष्ठस्तापनीयैश्च पादपैः}


\twolineshloka
{व्याघ्रसिंहवराहैश्च गोभिश्च मृगपक्षिभिः}
{ताराभिर्भास्करैश्चापि वारणैश्च हिरण्मयैः}


\twolineshloka
{वज्राङ्कुशविमानैश्च कूबरावृत्तसन्धिषु}
{समुच्छ्रितमहानाभिः स्तनयित्नुमहास्वनः ॥'}


\twolineshloka
{ततो रथेन शुभ्रेण महता किङ्किणीकिना}
{हयोत्तमयुजा शीघ्रमुपातिष्ठत दारुकः}


\twolineshloka
{तमुपस्थितमाज्ञाय रथं दिव्यं महामनाः}
{महाभ्रघननिर्घोषं सर्वरत्नविभूषितम्}


\twolineshloka
{अग्निं प्रदक्षिणं कृत्वा ब्राह्मणांश्च जनार्दनः}
{कौस्तुभं मणिमाबध्य श्रिया परमया ज्वलन्}


\twolineshloka
{कुरुभिः संवृतः कृष्णो वृष्णिभिश्चाभिरक्षितः}
{आतिष्ठत रथं शौरिः सर्वयादवनन्दनः}


\twolineshloka
{अन्वारुरोह दाशार्हं विदुरः सर्वधर्मवित्}
{सर्वप्राणभृतां श्रेष्ठं सर्वबुद्धिमतां वरम्}


\twolineshloka
{ततो दुर्योधनः कृष्णं शकुनिश्चापि सौबलः}
{द्वितीयेन रथेनैनमन्वयातां परन्तपम्}


\twolineshloka
{सात्यकिः कृतवर्मा च वृष्णीनां चापरे रथाः}
{पृष्ठतोऽनुययुः कृष्णं गजैरश्वै रथैरपि}


\twolineshloka
{तेषां हेमपरिष्कारैर्युक्ताः परमवाजिभिः}
{गच्छतां घोषिणश्चित्ररथा राजन्विरेजिरे}


\twolineshloka
{संमृष्टसंसिक्तरजः प्रतिपेदे महापथम्}
{राजर्षिचरितं काले कृष्णो धीमाञ्श्रिया ज्वलन्}


\twolineshloka
{ततः प्रयाते दाशार्हे प्रावाद्यन्तैकपुष्कराः}
{शङ्खाश्च दध्मिरे तत्र वाद्यान्यन्यानि यानि च}


\twolineshloka
{प्रवीराः सर्वलोकस्य युवानः सिंहविक्रमाः}
{परिवार्य रथं शौरेरगच्छन्त परन्तपाः}


\twolineshloka
{ततोऽन्ये बहुसाहस्रा विचित्राद्भुतवाससः}
{असिप्रासायुधधराः कृष्णस्यासन्पुरःसराः}


\twolineshloka
{गजाः पञ्चशतास्तत्र रथाश्चासन्सहस्रशः}
{प्रयान्तमन्वयुर्वीरं दाशार्हमपराजितम्}


\twolineshloka
{पुरं कुरूणां संवृत्तं द्रष्टुकामं जनार्दनम्}
{सबालवृद्धं सस्त्रीकं रथ्यागतमरिन्दम}


\twolineshloka
{वेदिकामाश्रिताभिश्च समाक्रान्तान्यनेकशः}
{प्रचलन्तीव भारेण योषिद्भिर्भवनान्युत}


\twolineshloka
{स पूज्यमानः कुरुभिः संश्रृण्वन्मधुराः कथाः}
{यथार्हं प्रतिसत्कुर्वन्प्रेक्षमाणः शनैर्ययौ}


\twolineshloka
{ततः सभां समासाद्य केशवस्यानुयायिनः}
{सशङ्खैर्वेणुनिर्घोषैर्दिशः सर्वा व्यनादयन्}


\twolineshloka
{ततः सा समितिः सर्वा राज्ञाममिततेजसाम्}
{संप्राकम्पत हर्षेण कृष्णागमनकाङ्क्षया}


\twolineshloka
{ततोऽभ्याशंगते कृष्णे समहृष्यन्नराधिपाः}
{श्रुत्वां तं रथनिर्घोषं पर्जन्यनिनदोपमम्}


\twolineshloka
{आसाद्य तु सभाद्वारमृषभः सर्वसात्वताम्}
{अवतीर्य रथाच्छौरिः कैलासशिखरोपमात्}


\twolineshloka
{नवमेघप्रतीकाशां ज्वलन्तीमिव तेजसा}
{महेन्द्रसदनप्रख्यां प्रविवेश सभां ततः}


\twolineshloka
{पाणौ गृहीत्वा विदुरं सात्यकिं च महायशाः}
{ज्येतींष्यादित्यवद्राजन्कुरून्प्राच्छादयच्छ्रिया}


\twolineshloka
{अग्रतो वासुदेवस्य कर्णदुर्योधनावुभौ}
{कृष्णयः कृतवर्मा चाप्यासन्कृष्णस्य पृष्ठतः}


\twolineshloka
{धृतराष्ट्रं पुरस्कृत्य भीष्मद्रोणादयस्ततः}
{आसनेभ्योऽचलन्सर्वे पूजयन्तो जनार्दनम्}


\twolineshloka
{अभ्यागच्छति दाशार्हे प्रज्ञाचक्षुर्नरेश्वर}
{सहैव द्रोणभीष्माभ्यामुदतिष्ठन्महायशाः}


\twolineshloka
{उत्तिष्ठति महाराजे धृतराष्ट्रे जनेश्वरे}
{तानि राजसहस्राणि समुत्तस्थुः समन्ततः}


\twolineshloka
{आसनं सर्वतोभद्रं जाम्बूनदपरिष्कृतम्}
{कृष्णार्थे कल्पितं तत्र धृतराष्ट्रस्य शासनात्}


\twolineshloka
{स्मयमानस्तु राजानं भीष्मद्रोणौ च माधवः}
{अभ्यभाषत धर्मात्मा राज्ञश्चान्यान्यथावयः}


\twolineshloka
{तत्र केशवमानर्चुः सम्यगभ्यागतं सभाम्}
{राजानः पार्थिवाः सर्वे कुरवश्च जनार्दनम्}


\threelineshloka
{तत्र तिष्ठन्स दाशार्हो राजमध्ये परन्तपः}
{अपश्यदन्तरिक्षस्थानृषीन्परपुरञ्जयः}
{}


\twolineshloka
{ततस्तानभिसंप्रेक्ष्य नारदप्रमुखानृषीन् ॥अभ्यभाषत दाशार्हो भीष्मं शान्तनवं शनैः}
{}


\twolineshloka
{पार्थिवीं समितिं द्रष्टुमृषयोऽभ्यागता नृप ॥निमन्त्र्यन्तामासनैश्च सत्कारेण च भूयसा}
{}


\threelineshloka
{नैतेष्वनुपविष्टेषु शक्यं केनचिदासितुम् ॥पूजा प्रयुज्यतामाशु मुनीनां भावितात्मनाम्}
{वैशंपायन उवाच}
{}


\threelineshloka
{ऋषीञ्शान्तनवो दृष्ट्वा सभाद्वारमुपस्थितान्}
{त्वरमाणस्ततो भृत्यानासनानीत्यचोदयत्}
{}


\twolineshloka
{आसनान्यथ मृष्टानि महान्ति विपुलानि च ॥मणिकाञ्चनचित्राणि समाजह्रुस्ततस्ततः}
{}


\twolineshloka
{तेषु तत्रोपविष्टेषु गृहीतार्घ्येषु भारत ॥निषसादासने कृष्णो राजानश्च यथासनम्}
{}


\twolineshloka
{दुःशासनः सात्यकये ददावासनमुत्तमम् ॥विविंशतिर्ददौ पीठं काञ्चनं कृतवर्मणे}
{}


\twolineshloka
{अविदूरे तु कृष्णस्य कर्णदुर्योधनावुभौ ॥एकासने महात्मानौ निषीदतुरमर्षणौ}
{}


\twolineshloka
{गान्धारराजः शकुनिर्गान्धारैरभिरक्षितः ॥निषसादासने राजा सहपुत्रो विशांपते}
{}


\twolineshloka
{विदुरो मणिपीठे तु शुक्लस्पर्ध्याजिनोत्तरे ॥संस्पृशन्नासनं शौरेर्महामतिरुपाविशत्}
{}


\twolineshloka
{चिरस्य दृष्ट्वा दाशार्हं राजानः सर्व एव ते ॥अमृतस्येव नातृप्यन्प्रेक्षमाणा जनार्दनम्}
{}


% Check verse!
अतसीपुष्पसङ्काशः पीतवासा जनार्दनः ॥व्यभ्राजत सभामध्ये हेम्नीवोपहितो मणिः
\twolineshloka
{ततस्तूष्णीं सर्वमासीद्गोविन्दगतमानसम्}
{न तत्र कश्चित्किंचिद्वा व्याजहार पुमान्क्वचित्}


\chapter{अध्यायः ९५}
\twolineshloka
{वैशंपायन उवाच}
{}


\twolineshloka
{तेष्वासीनेषु सर्वेषु तूष्णींभूतेषु राजसु}
{वाक्यमभ्याददे कृष्णः सुदंष्ट्रो दुन्दुभिस्वनः}


\threelineshloka
{जीमूत इव धर्मान्ते सर्वां संश्रावयन्सभाम्}
{धृतराष्ट्रमभिप्रेक्ष्य समभाषत माधवः ॥श्रीभगवानुवाच}
{}


\twolineshloka
{कुरूणां पाण्डवानां च शमः स्यादिति भारत}
{अप्रणशेन वीराणामेतद्याचितुमागतः}


\twolineshloka
{राजन्नान्यत्प्रवक्तव्यं तव नैःश्रेयसं वचः}
{विदितं ह्येव ते सर्वं वेदितव्यमरिन्दम}


\twolineshloka
{इदं ह्यद्य कुलं श्रेष्ठं सर्वराजसु पार्थिव}
{श्रुतवृत्तोपसंपन्नं सर्वैः समुदितं गुणैः}


\twolineshloka
{कृपानुकम्पा कारुण्यमानृशंस्यं च भारत}
{तथार्जवं क्षमा सत्यं कुरुष्वेतद्विशिष्यते}


\twolineshloka
{तस्मिन्नेवंविधे राजन्कुले महति तिष्ठति}
{त्वन्निमित्तं विशेषेण नेह युक्तमसांप्रतम्}


\twolineshloka
{त्वं हि धारयिता श्रेष्ठः कुरूणां कुरुसत्तम}
{मिथ्याप्रचरतां तात बाह्येष्वाभ्यन्तरेषु च}


\twolineshloka
{ते पुत्रास्तव कौरव्य दुर्योधनपुरोगमाः}
{धर्मार्थौ पृष्ठतः कृत्वा प्रचरन्ति नृशंसवत्}


\twolineshloka
{अशिष्टा गतमर्यादा लोभेन हृतचेतसः}
{स्वेषु बन्धुषु मुख्येषु तद्वेत्थ पुरुषर्षभ}


\twolineshloka
{सेयमापन्महाघोरा कुरुष्वेव समुत्थिता}
{उपेक्ष्यमाणा कौरव्य पृथिवीं घातयिष्यति}


\twolineshloka
{शक्या चेयं शमयितुं त्वं चेदिच्छसि भारत}
{न दुष्करो ह्यत्र शमो मतो मे भरतर्षभ}


\twolineshloka
{त्वय्यधीनः शमो राजन्मयि चैव विशांपते}
{पुत्रान्स्थापय कौरव्य स्थापयिष्याम्यहं परान्}


\twolineshloka
{आज्ञा तव हि राजेन्द्र कार्या पुत्रैः सहान्वयैः}
{हितं बलवदप्येषां तिष्ठतां तव शासने}


\twolineshloka
{तव चैव हितं राजन्पाण्डवानामथो हितम्}
{शमे प्रयतमानस्य तव शासनकाङ्क्षिणः}


\twolineshloka
{स्वयं निष्फलमालक्ष्य संविधत्स्व विशांपते}
{सहायभूता भरतास्तवैव स्युर्जनेश्वर}


\twolineshloka
{धर्मार्थयोस्तिष्ठ राजन्पाण्डवैरभिरक्षितः}
{न हि शक्यास्तथाभूता यत्नादपि नराधिप}


\twolineshloka
{न हि त्वां पाण्डवैर्जेतुं रक्ष्यमाणं महात्मभिः}
{इन्द्रोपि देवैः सहितः प्रसहेत कुतो नृपाः}


\twolineshloka
{यत्र भीष्मश्च द्रोणश्च कृपः कर्णो विविंशतिः}
{अश्वत्थामा विकर्णश्च सोमदत्तोऽथ बाह्लिकः}


\twolineshloka
{सैन्धवश्च कलिङ्गश्च काम्भोजश्च सुदक्षिणः}
{युधिष्ठिरो भीमसेनः सव्यसाची यमौ तथा}


\twolineshloka
{सात्यकिश्च महातेजा युयुत्सुश्च महारथः}
{को नु तान्विपरीतात्मा युद्ध्येत भरतर्षभ}


\twolineshloka
{लोकस्येश्वरतां भूयः शत्रुभिश्चाप्यधृष्यताम्}
{प्राप्स्यसि त्वममित्रघ्न सहितः कुरुपाण्डवैः}


\twolineshloka
{तस्य ते पृथिवीपालास्त्वत्समाः पृथिवीपते}
{श्रेयांसश्चैव राजानः सन्धास्यन्ते परन्तप}


\twolineshloka
{स त्वं पुत्रैश्च पौत्रैश्च पितृभिर्भ्रातृभिस्तथा}
{सुहृद्भिः सर्वतो गुप्तः सुखं शक्ष्यसि जीवितुं}


\twolineshloka
{एतानेव पुरोधाय यत्कृत्य च यथा पुरा}
{अखिलां भोक्ष्यसे सर्वां पृथिवीं पृथिवीपते}


\twolineshloka
{एतैर्हि सहितः सर्वैः पाण्डवैः स्वैश्च भारत}
{अन्यान्विजेष्यसे शत्रूनेष स्वार्थस्तवाखिलः}


\twolineshloka
{तैरेवोपार्जितां भूमिं भोक्ष्यसे च परन्तप}
{यदि संपत्स्यसे पुत्रैः सहामात्यैर्नराधिप}


\twolineshloka
{संयुगे वै महाराज दृश्यते सुमहान्क्षयः}
{क्षये चोभयतो राजन्कं धर्ममनुपश्यसि}


\twolineshloka
{पाण्डवैर्निहतैः सङ्ख्ये पुत्रैर्वापि महाबलैः}
{यद्विन्देथाः सुखं राजंस्तद्ब्रूहि भरतर्षभ}


\twolineshloka
{शूराश्च हि कृतास्त्राश्च सर्वे युद्धाभिकाङ्क्षिणः}
{पाण्डवास्तावकाश्चैव तान्रक्ष महतो भयात्}


\twolineshloka
{न पश्येम कुरून्सर्वान्पाण्डवांश्चैव संयुगे}
{क्षीणानुभयतः शूरान्राथिनो रथिभिर्हतान्}


\twolineshloka
{समवेताः पृथिव्यां हि राजानो राजसत्तम}
{अमर्षवशमापन्ना नाशयेयुरिमाः प्रजाः}


\twolineshloka
{त्राहि राजन्निमं लोकं न नश्येयुरिमाः प्रजाः}
{त्वयि प्रकृतिमापन्ने शेषः स्यात्कुरुनन्दन}


% Check verse!
शुक्ला वादन्या ह्रीमन्त आर्याः पुण्याभिजातयः ॥अन्योन्यसचिवा राजंस्तान्पाहि महतो भयात्
\twolineshloka
{शिवेनेमे भूमिपालाः समागम्य परस्परम्}
{सह भुक्त्वा च पीत्वा च प्रतियान्तु यथागृहम्}


\twolineshloka
{सुवाससः स्रग्विणश्च सत्कृता भरतर्षभ}
{अमर्षं च निराकृत्य वैराणि च परन्तप}


\twolineshloka
{हार्दं यत्पाण्डवेष्वासीत्प्राप्तोऽस्मिन्नायुषः क्षये}
{तदेव ते भवत्वद्य सन्धत्स्व भरतर्षभ}


\twolineshloka
{बाला विहीनाः पित्रा ते त्वयैव परिवर्धिताः}
{तान्पालय यथान्यायं पुत्रांश्च भरतर्षभ}


\twolineshloka
{भवतैव हि रक्ष्यास्ते व्यसनेषु विशेषतः}
{मा ते धर्मस्तथैवार्थो नश्येत भरतर्षभ}


\twolineshloka
{आहुस्त्वां पाण्डवा राजन्नभिवाद्य प्रसाद्य च}
{भवतः शासनाद्दुःखसमुभूतं सहानुगैः}


\twolineshloka
{द्वादशेमानि वर्षाणि वने निर्व्युषितानि नः}
{त्रयोदशं तथाऽज्ञातैः सजने परिवत्सरम्}


\twolineshloka
{स्थाता नः समये तस्मिन्पितेति कृतनिश्चयाः}
{नाहास्म समयं तात तच्च नो ब्राह्मणा विदुः}


\threelineshloka
{तस्मिन्नः समये तिष्ठ स्थितानां भरतर्षभ}
{नित्यं संक्लेशिता राजन्स्वराज्यांशं लभेमहि}
{}


\twolineshloka
{त्वं धर्ममर्थं संजानन्सम्यङ्वस्त्रातुमर्हसि ॥गुरुत्वं भवति प्रेक्ष्य बहून्क्लेशांस्तितिक्ष्महे}
{}


\twolineshloka
{स भवान्मातृपितृव्रदस्मासु प्रतिपद्यताम् ॥गुरोर्गरीयसी वृत्तिर्या च शिष्यस्य भारत}
{}


\twolineshloka
{वर्तामहे त्वयि च तां त्वं च वर्तस्व नस्तथा ॥पित्रा स्थापयितव्या हि वयमुत्पथमास्थिताः}
{}


\twolineshloka
{संस्थापय पथिप्वस्मांस्तिष्ठ धर्मे सुवर्त्मनि ॥आहुश्चेमां परिषदं पुत्रास्ते भरतर्षभ}
{}


\twolineshloka
{धर्मज्ञेषु सभासत्सु नेह युक्तमसांप्रतम् ॥यत्र धर्मो ह्यधर्मेण सत्यं यत्रानृतेन च}
{}


\twolineshloka
{हन्यते प्रेक्षमाणानां हतास्तत्र सभासदः ॥विद्धो धर्मो ह्यधर्मेण सभां यत्र प्रपद्यते}
{}


% Check verse!
नचास्य शल्यं कृन्तन्ति विद्धास्तत्र सभासदः ॥धर्म एतानारुजति यथा नद्यनुकूलजान्
\twolineshloka
{ये धर्ममनुपश्यन्तस्तूष्णीं ध्यायन्त आसते}
{ते सत्यमाहुर्धर्म्यं च न्याय्यं च भरतर्षभ}


\twolineshloka
{शक्यं किमन्यद्वक्तुं ते दानादन्यञ्जनेश्वर}
{ब्रुवन्तु ते महीपालाः सभायां ये समासते}


\twolineshloka
{धर्मार्थौ संप्रधार्यैव यदि सत्यं ब्रवीम्यहम्}
{प्रमुञ्चोमान्मृत्युपाशात्क्षत्रियान्पुरुषर्षभ}


\threelineshloka
{प्रशाम्य भरतश्रेष्ठ मा मन्युवशमन्वगाः}
{पित्र्यं तेभ्यः प्रदायांशं पाण्डवेभ्यो यथोचितम्}
{ततः सपुत्रः सिद्धार्थो भुङ्क्ष भोगान्परन्तप}


\twolineshloka
{अजातशत्रुं जानीषे स्थितं धर्मे सतां सदा}
{सपुत्रे त्वयि वृत्तिं च वर्तते यां नराधिप}


\twolineshloka
{दाहितश्च निरस्तश्च त्वामेवोपाश्रितः पुनः}
{इन्द्रप्रस्थं त्वयैवासौ सपुत्रेण विवासितः}


\twolineshloka
{स तत्र विवसन्सर्वान्वशमानीय पार्थिवान्}
{त्वन्मुखानकरोद्राजन्न च त्वामत्यवर्तत}


\twolineshloka
{तस्यैवं वर्तमानस्य सौबलेन जिहीर्षता}
{राष्ट्राणि धनधान्यं च प्रयुक्तः परमोपधिः}


\twolineshloka
{स तामवस्थां संप्राप्य कृष्णां प्रेक्ष्य सभां गताम्}
{क्षत्रधर्मादमेयात्मा नाकम्पत युधिष्ठिरः}


\twolineshloka
{अहं तु तव तेषां च श्रेय इच्छामि भारत}
{धर्मादर्थात्सुखाच्चैव मा राजन्नीनशः प्रजाः}


\twolineshloka
{अनर्थमर्थं मन्वानोऽप्यर्थं चानर्थमात्मनः}
{लोभेऽतिप्रसृतान्पुत्रान्निगृह्णीष्व विशांपते}


\threelineshloka
{स्थिताः शुश्रूषितुं पार्थाः स्थिता योद्धुमरिन्दमाः}
{यत्ते पथ्यतमं राजंस्तस्मिंतिष्ठ परन्तप ॥वैशंपायन उवाच}
{}


\twolineshloka
{तद्वाक्यं पार्थिवाः सर्वे हृदयैः समपूजयन्}
{न तत्र कश्चिद्वक्तुं हि वाचं प्राक्रामदग्रतः}


\chapter{अध्यायः ९६}
\twolineshloka
{वैशंपायन उवाच}
{}


\twolineshloka
{तस्मिन्नभिहिते वाक्ये केशवेन महात्मना}
{स्तिमिता हृष्टरोमाण आसन्सर्वे सभासदः}


\twolineshloka
{कस्स्विदुत्तरमेतेषां वक्तुमुत्सहते पुमान्}
{इति सर्वे मनोभिस्ते चिन्तयन्ति स्म पार्थिवाः}


\twolineshloka
{तथा तेषु च सर्वेषु तूष्णींभूतेषु राजसु}
{जामदग्न्य इदं वाक्यमब्रवीत्कुरुसंसदि}


\twolineshloka
{इमां मे सोपमां वाचं श्रृणु सत्यामशङ्कितः}
{तां श्रुत्वा श्रेय आदत्स्व यदि साध्विति मन्यसे}


\twolineshloka
{राजा दम्भोद्भवो नाम सार्वभौमः पुराऽभवत्}
{अखिलां बुभुजे सर्वां पृथिवीमिति नः श्रुतम्}


\twolineshloka
{स स्म नित्यं निशापाये प्रातरुत्थाय वीर्यवान्}
{ब्राह्मणान्क्षत्रियान्वैश्यान्पृच्छन्नास्ते महारथः}


\twolineshloka
{अस्ति कश्चिद्विशिष्टो वा मद्विधो वा भवेद्युधि}
{शूद्रो वैश्यः क्षत्रियो वा ब्राह्मणो वाऽपि शस्त्रभृत्}


\twolineshloka
{इति ब्रुवन्नन्वचरत्स राजा पृथिवीमिमाम्}
{दर्पेण महता मत्तः कञ्चिदन्यमचिन्तयन्}


\twolineshloka
{तं च वैद्या अकृपणा ब्राह्मणाः सर्वतोऽभयाः}
{प्रत्यषेधन्त राजानं श्लाघमानं पुनः पुनः}


\twolineshloka
{निषिध्यमानोऽप्यसकृत्पृच्छत्येव स वै द्विजान्}
{अतिमानं श्रिया मत्तं तमूचुर्ब्राह्मणास्तदा}


\twolineshloka
{तपस्विनो महात्मानो वेदप्रत्ययदर्शिनः}
{उदीर्यमाणं राजानं क्रोधदीप्ता द्विजातयः}


\twolineshloka
{अनेकजयिनौ शङ्ख्ये यौ वै पुरुषसत्तमौ}
{तयोस्त्वं न समो राजन्भवितासि कदाचन}


\threelineshloka
{एवमुक्तः स राजा तु पुनः पप्रच्छ तान्द्विजान्}
{क्व तौ वीरौ क्वजन्मानौ किंकर्माणौ च कौ च तौ ॥ब्राह्मणा ऊचुः}
{}


\twolineshloka
{नरो नारायणश्चैव तापसाविति नः श्रुतम्}
{आयातौ मानुषे लोके ताभ्यां युद्ध्यस्व पार्थिव}


\twolineshloka
{श्रूयेते तौ महात्मानौ नरनारायणावुभौ}
{तपो घोरमनिर्देश्यं तप्येते गन्धमादने}


\twolineshloka
{स राजा महतीं सेनां योजयित्वा षडङ्गिनीम्}
{अमृष्यमाणः संप्रायाद्यत्र तावपराजितौ}


\twolineshloka
{स गत्वा विषमं घोरं पर्वतं गन्धमादनम्}
{मार्गमाणोऽन्वगच्छत्तौ तापसौ वनमाश्रितौ}


\twolineshloka
{तौ दृष्ट्वा क्षुत्पिपासाभ्यां कृशौ धमनिसन्ततौ}
{शीतवातातपैश्चैव कर्शितौ पुरुषोत्तमौ}


\twolineshloka
{अभिगम्योपसङ्गृह्य पर्यपृच्छदनामयम्}
{तमर्चित्वा मूलफलैरासनेनोदकेन च}


\twolineshloka
{न्यमन्त्रयेतां राजानं किं कार्यं क्रियतामिति}
{ततस्तामानुपूर्वी स पुनरेवान्वकीर्तयत्}


\twolineshloka
{बाहुभ्यां मे जिता भूमिर्निहताः सर्वशत्रवः}
{भवद्भ्यां युद्धमाकाङ्क्षन्नुपयातोऽस्मि पर्वतम्}


\threelineshloka
{आतिथ्यं दीयतामेतत्काङ्क्षितं मे चिरं प्रति}
{नरनारायणावूचतुः}
{अपेतक्रोधलोभोऽयमाश्रमो राजसत्तम}


\threelineshloka
{न ह्यस्मिन्नाश्रमे युद्धं कुतः शस्त्रं कुतोऽनृजुः}
{अन्यत्र युद्धमाकाङ्क्ष बहवः क्षत्रियाः क्षितौ ॥राम उवाच}
{}


\twolineshloka
{उच्यमानस्तथाऽपि स्म भूय एवाभ्यभाषत}
{पुनः पुनः क्षाम्यमाणः सान्त्व्यमानश्च भारत}


\twolineshloka
{दम्भोद्भवो युद्धमिच्छन्नाह्वयत्येव तापसौ}
{ततो नरस्त्विषीकाणां मुष्टिमादाय भारत}


\twolineshloka
{अब्रवीदेहि युद्ध्यस्व युद्धकामुक क्षत्रिय}
{सर्वशस्त्राणि चादत्स्व योजयस्व च वाहिनीम्}


\threelineshloka
{अहं हि ते विनेष्यामि युद्धश्रद्धामितः परम्}
{`यदाह्वयसि दर्पेण ब्राह्मणप्रमुखाञ्जनान् ॥'दम्भोद्भव उवाच}
{}


\threelineshloka
{यद्येतदस्त्रमस्मासु युक्तं तापस मन्यसे}
{एतेनापि त्वया योत्स्ये युद्धार्थी ह्यहमागतः ॥राम उवाच}
{}


\twolineshloka
{इत्युक्त्वा शरवर्षेण सर्वतः समवाकिरत्}
{दम्भोद्भवस्तापसं तं जिवांसुः सहसैनिकः}


\twolineshloka
{तस्य तानस्यतो घोरानिषून्परतनुच्छिदः}
{कदर्थीकृत्य स मुनिरिषीकाभिः समार्पयत्}


\twolineshloka
{ततोऽस्मौ प्रासृजद्धोरमैषीकमपराजितः}
{अस्त्रमप्रतिसन्धेयं तदुद्भुतमिवाभवत्}


\twolineshloka
{तेषामक्षीणि कर्णांश्च नासिकाश्चैव मायया}
{निमित्तवेधी स मुनिरीषीकाभिः समार्पयत्}


\twolineshloka
{स दृष्ट्वा श्वेतमाकाशमिषीकाभिः समाचितम्}
{पादयोर्न्यपतद्राजा स्वस्ति मेस्त्विति चाब्रवीत्}


\twolineshloka
{तमब्रवीन्नरो राजञ्शरण्यः शरणैषिणाम्}
{ब्रह्मण्यो भव धर्मात्मा मा च स्मैवं पुनः कृथाः}


\twolineshloka
{नैतादृक्पुरुषो राजन्क्षत्रधर्ममनुस्मरन्}
{मनसा नृपशार्दूल भवेत्परपुरञ्जयः}


\twolineshloka
{मा च दर्पसमाविष्टः वा तत्ते राजन्समाहितम् ॥कृतप्रज्ञो वीतलोभो निरहङ्कार आत्मवान्}
{}


\twolineshloka
{दान्तः क्षान्तो मृदुः सौम्य प्रजाः पालय पार्थिव ॥मास्म भूयः क्षिपेः कंचिदविदित्वा बलाबलम्}
{}


\twolineshloka
{अनुज्ञातः स्वस्ति गच्च मैवं भूयः समाचरेः}
{कुशलं ब्राह्मणान्पृच्छेरावयोर्वचनाद्भृशम्}


\twolineshloka
{ततो राजा तयोः पादावभिवाद्य महात्मनोः}
{प्रत्याजगाम स्वपुरं धर्मं चैवाचरद्भृशम्}


\twolineshloka
{सुमहच्चापि तत्कर्म यन्नरेण कृतं पुरा}
{ततो गुणैः सुबहुभिः श्रेष्ठो नारायणोऽभवत्}


\twolineshloka
{तस्माद्यावद्धनुःश्रेष्ठे गाण्डीवेऽस्त्रं न युज्यते}
{तावत्त्वं मानमुत्सृज्य गच्छ राजन्धनञ्जयम्}


\twolineshloka
{काकुदीकं शुकं नाकमक्षिसन्तर्जनं तथा}
{सन्तानं नर्तकं घोरमास्यमोदकमष्टमम्}


\twolineshloka
{एतैर्विद्धाः सर्व एव मरणं यान्ति मानवाः}
{उन्मत्ताश्च विचेष्टन्ते नष्टसंज्ञा विचेतसः}


\twolineshloka
{` स्वपन्ति च प्लवन्ते च च्छर्दयन्ति च मानवाः}
{मूत्रयन्ते च सततं रुदन्ति च हसन्ति च ॥'}


\twolineshloka
{कामक्रोधौ लोभमोहौ मदमानौ तथैव च}
{मात्सर्याहङ्कुती चैव क्रमादेत उदाहृताः}


\twolineshloka
{निर्माता सर्वलोकानामीश्वरः सर्वकर्मवित्}
{यस्य नारायणो बन्धुरर्जुनो दुःसहो युधि}


\twolineshloka
{कस्तमुत्सहते जेतुं त्रिषु लोकेषु भारत}
{वीरं कपिध्वजं जिष्णुं यस्य नास्ति समो युधि}


\twolineshloka
{असङ्ख्येया गुणाः पार्थे तद्विशिष्टो जनार्दनः}
{त्वमेव भूयो जानासि कुन्तीपुत्रं धनञ्जयम्}


\twolineshloka
{नरनारायणौ यौ तौ तावेवार्जुनकेशवौ}
{विजानीहि महाराज प्रवीरौ पुरुषोत्तमौ}


\twolineshloka
{यद्येतदेवं जानासि न च मामभिशङ्कसे}
{आर्यां मतिं समास्थाय शाम्य भारत पाण्डवैः}


\twolineshloka
{अथ चेन्मन्यसे श्रेयो न मे भेदो भवेदिति}
{प्रशाम्य भरतश्रेष्ठ मा च युद्धे मनः कृथाः}


\twolineshloka
{भवतां च कुरुश्रेष्ठ कुलं बहुमतं भुवि}
{तत्तथैवास्तु भद्रं ते स्वार्थमेवोपचिन्तय}


\chapter{अध्यायः ९७}
\twolineshloka
{वैशंपायन उवाच}
{}


\threelineshloka
{जामदग्न्यवचः श्रुत्वा कण्वेऽपि भगवानृषिः}
{दुर्योधनमिदं वाक्यमब्रवीत्कुरुसंसदि ॥कण्व उवाच}
{}


\twolineshloka
{अक्षयश्चाव्ययश्चैव ब्रह्मा लोकपितामहः}
{तथैव भगवन्तौ तौ नरनारायणावृषी}


\twolineshloka
{आदित्यानां हि सर्वेषां विष्णुरेकः सनातनः}
{अजथ्यश्चाव्ययश्चैव शाश्वतः प्रभुरीश्वरः}


\twolineshloka
{निमित्तमरणाश्चान्ये चन्द्रसूर्यौ मही जलम्}
{वायुरग्निस्तथाऽऽकाशं ग्रहास्तारागणास्तथा}


\twolineshloka
{ते च क्षयान्ते जगतो हित्वा लोकत्रयं सदा}
{क्षयं गच्छन्ति वै सर्वे सृज्यन्ते च पुनः पुनः}


\twolineshloka
{मुहूर्तमरणास्त्वन्ये मानुषा मूगपक्षिणः}
{तैर्यग्योन्याश्च ये चान्ये जीवलोकचरास्तथा}


\twolineshloka
{भूयिष्ठेन तु राजानः श्रियं भुक्त्वाऽऽयुषः क्षये}
{तरुणाः प्रतिपद्यन्ते भोक्तुं सुकृतदुष्कृते}


\twolineshloka
{स भवान्धर्मपुत्रेण शमं कर्तुमिहार्हसि}
{पाण्डवाः कुरवश्चैव पालयन्तु वसुन्धरान्}


\twolineshloka
{बलवानहमित्येव न मन्तव्यं सुयोधन}
{बलवन्तो बलिभ्यो हि दृश्यन्ते पूरुषर्षभ}


\threelineshloka
{न बलं बलिनां मध्ये बलं भवति कौरव}
{बलवन्तो हि ते सर्वे पाण्डवा देवविक्रमाः}
{}


\twolineshloka
{अत्राप्युदाहरन्तीममितिहासं पुरातनम्}
{मातलेर्दातुकामस्य कन्यां मृगयतो वरम्}


\twolineshloka
{मतस्त्रिलोकराजस्य मातलिर्नाम सारथिः}
{तस्यैकैव कुले मन्या रूपतो लोकविश्रुता}


\twolineshloka
{गुणकेशीति विख्याता नाम्ना सा देवरूपिणी}
{श्रिया च वपुषा चैव स्त्रियोऽन्याःसाऽतिरिच्यते}


\twolineshloka
{तस्याः प्रदानसमयं मातलिः सह भार्यया}
{ज्ञात्वा विममृशे राजंस्तत्परः परिचिन्तयन्}


\twolineshloka
{धिक्खल्वलघुशीलानामुच्छ्रितानां यशस्विनाम्}
{नराणां मृदुसत्वानां कुले कन्याप्ररोहणम्}


\twolineshloka
{मातुः कुलं पितृकुलं यत्र चैव प्रदीयते}
{कुलत्रयं संशयितं कुरुते कन्यका सताम्}


\threelineshloka
{देवमानुषलोकौ द्वौ मानुषेणैव चक्षुषा}
{अपगाह्यैव विचितौ न च मे रोचते वरः ॥कण्व उवाच}
{}


\twolineshloka
{न देवान्नैव दितिजान्न गन्धर्वान्न मानुषान्}
{अरोचयद्वरकृते तथैव बहुलानृषीन्}


\twolineshloka
{भार्यया तु स संमन्त्र्य सह रात्रौ सुधर्मया}
{मातलिर्नागलोकाय चकार गमने मतिम्}


\twolineshloka
{न मे देवमनुष्येषु गुणकेश्याः समो वरः}
{रूपतो दृश्यते कश्चिन्नागेषु भविता ध्रुवम्}


\twolineshloka
{इत्यामन्त्र्य सुधर्मां स कृत्वा चाभिप्रदक्षिणम्}
{कन्यां शिरस्युपाघ्राय प्रविवेश महीतलम्}


\chapter{अध्यायः ९८}
\twolineshloka
{कण्व उवाच}
{}


\twolineshloka
{मातलिस्तु व्रजन्मार्गे नारदेन महर्षिणा}
{वरुणं गच्छता द्रष्टुं समागच्छद्यदृच्छया}


\twolineshloka
{नारदोऽथाब्रवीदेनं क्व भवान्गन्तुमुद्यतः}
{स्येन वा सूत कार्येण शासनाद्वा शतक्रतोः}


\twolineshloka
{मातलिर्नारदेनैवं संपृष्टः पथि गच्छता}
{यथावत्सर्वमाचष्ट स्वकार्यं नारदं प्रति}


\twolineshloka
{तमुवाचाथ स मुनिर्गच्छावः सहिताविति}
{सलिलेशदिदृक्षार्थमहमप्युद्यतो दिवः}


\twolineshloka
{अहं ते सर्वमाख्यास्ये दर्शयन्वसुधातलम्}
{दृष्ट्वा तत्र वरं कंचिद्रोचयिष्याव मातले}


\twolineshloka
{अवगाह्य तु तौ भूमिमुभौ मातलिनारदौ}
{ददृशाते महात्मानौ लोकपालमपां पतिम्}


\twolineshloka
{तत्र देवर्षिसदृशीं पूजां स प्राप नारदः}
{महेन्द्रसदृशीं चैव मातलिः प्रत्यपद्यत}


\twolineshloka
{तावुभौ ग्रीतमनसौ कार्यवन्तौ निवेद्य ह}
{वरुणेनाभ्यनुज्ञातौ नागलोकं विचेरतुः}


\threelineshloka
{नारदः सर्वभूतानामन्तर्भूमिनिवासिनाम्}
{जानंश्चकार व्याख्यानं यन्तुः सर्वमशेषतः ॥नारद उवाच}
{}


\twolineshloka
{दृष्टस्ते वरुणः सूत पुत्रपौत्रसमावृतः}
{पश्योदकपतेः स्थानं सर्वतोभद्रमृद्धिमत्}


\twolineshloka
{एष पुत्रो महाप्रज्ञो वरुणस्येह गोपतेः}
{एष वै शीलवृत्तेन शौचेन च विशिष्यते}


\twolineshloka
{एषोऽस्य पुत्रोऽभिमतः पुष्करः पुष्करेक्षणः}
{रूपवान्दर्शनीयश्च सोमपुत्र्या वृतः पतिः}


\twolineshloka
{ज्योत्स्नाकालीति यामाहुर्द्वितीयां रूपतः श्रियम्}
{अदित्याश्चैव यः पुत्रो ज्येष्ठः श्रेष्ठः कृतः स्मृतः}


\twolineshloka
{भवनं पश्य वारुण्यं यदेतत्सर्वकाञ्चनम्}
{यत्प्राप्य सुरतां प्राप्ताः सुराः सुरपतेः सखे}


\twolineshloka
{एतानि हृतराज्यानां दैतेयानां स्म मातले}
{दीप्यमानानि दृश्यन्ते सर्वप्रहरणान्युत}


\twolineshloka
{अक्षयाणि किलैतानि विवर्तन्ते स्म मातले}
{अनुभावप्रयुक्तानि सुरैरवजितानि ह}


\twolineshloka
{अत्र राक्षसजात्यश्च दैत्यजात्यश्च मातले}
{दिव्यप्रहरणाश्चासन्पूर्वदैवतनिर्मिताः}


\twolineshloka
{अग्निरेष महार्चिष्माञ्जागर्ति वारुणे ह्रदे}
{वैष्णवं चक्रमाविद्धं विधूमेन हविष्मता}


\twolineshloka
{एष गाण्डीमयश्चापो लोकसंहारसंभृतः}
{रक्ष्यते दैवतैर्नित्यं यतस्तद्गाण्डिवं धनुः}


\twolineshloka
{एष कृत्ये समुत्पन्ने तत्तद्धारयते बलम्}
{सहस्रशतसङ्ख्येन प्राणेन सततं ध्रुवः}


\twolineshloka
{अशास्यानपि शास्त्येष रक्षोबन्धुषु राजसु}
{सृष्टः प्रथमतश्चण्डो ब्रह्मणा ब्रह्मवादिना}


\twolineshloka
{एतच्छस्त्रं नरेन्द्राणां महच्चक्रे भासितम्}
{पुत्राः सलिलराजस्य धारयन्ति महोदयम्}


\twolineshloka
{एतत्सलिलराजस्य छत्रं छत्रगृहे स्थितम्}
{सर्वतः सलिलं शीतं जीमूत इव वर्षति}


\twolineshloka
{एतच्छत्रात्परिभ्रष्टं सलिलं सोमनिर्मलम्}
{तमसा मूर्छितं भाति येन नार्छति दर्शनम्}


\twolineshloka
{बहून्यद्भुतरूपाणि द्रष्टव्यानीह मातले}
{तव कार्योपरोधस्तु तस्माद्गच्छाव मा चिरम्}


\chapter{अध्यायः ९९}
\twolineshloka
{नारद उवाच}
{}


\twolineshloka
{एतत्तु नागलोकस्य नाभिस्थाने स्थितं पुरम्}
{पातालमिति विख्यातं दैत्यदानवसेवितम्}


\twolineshloka
{इदमद्भिः समं प्राप्ता ये केचिद्भुवि जङ्गमाः}
{प्रविशन्तो महानादं नदन्ति भयपीडिताः}


\twolineshloka
{अत्रासुरोऽग्निः सततं दीप्यते वारिभोजनः}
{व्यापारेण धृतात्मानं निबद्धं समबुध्यत}


% Check verse!
अत्रामृतं सुरैः पीत्वा निहितं निहतारिभिः ॥अतः सोमस्य हानिश्च वृद्धिश्चैव प्रदृश्यते
\twolineshloka
{अत्रादित्यो हयशिराः काले पर्वणि पर्वणि}
{उत्तिष्ठति सुवर्णाख्यो वाग्भिरापूरयञ्जगत्}


\twolineshloka
{यस्मादलं समस्तास्ताः पतन्ति जलमूर्तयः}
{तस्मात्पातालमित्येव ख्यायते पुरमुत्तमम्}


\twolineshloka
{ऐरावणोऽस्मात्सलिलं गृहीत्वा जगतो हितः}
{मेघेष्वामुञ्चते शीतं यन्महेन्द्रः प्रवर्षति}


\twolineshloka
{ऐरावतो नागराजो वामनः कुमुदोऽञ्जनः}
{प्रसूताः सुप्रतीकस्य वंशे वारणसत्तमाः}


\twolineshloka
{अत्र नानाविधाकारास्तिमयो नैकरूपिणः}
{अप्सु सोमप्रभां पीत्वा वसन्ति जलचारिणः}


\twolineshloka
{अत्र सूर्यांशुभिर्भिन्नाः पातालतलमाश्रिताः}
{मृता हि दिवसे सूत पुनर्जीवन्ति वै निशि}


\twolineshloka
{उदयन्नित्यशश्चात्र चन्द्रमा रश्मिबाहुभिः}
{अमृतं स्पृश्य संस्पर्शात्संजीवयतिदेहिनः}


\twolineshloka
{अत्र तेऽधर्मनिरता बद्धाः कालेन पीडिताः}
{दैतेया निवसन्ति स्म वासवेन हृतश्रियः}


\twolineshloka
{अत्र भूतपतिर्नाम सर्वभूतमहेश्वरः}
{भूतये सर्वभूतानामचरत्तप उत्तमम्}


\twolineshloka
{अत्र गोव्रतिनो विप्राः स्वाध्यायाम्नायकर्शिताः}
{त्यक्तप्राणा जितस्वर्गा निवसन्ति महर्षयः}


\twolineshloka
{यत्रतत्रशयो नित्यं येनकेनचिदाशितः}
{येनकेनचिदाच्छन्नः स गोव्रत इहोच्यते}


\twolineshloka
{पश्य यद्यत्र ते कश्चिद्रोचते गुणतो वरः}
{वरयिष्यामि तं गत्वा यत्नमास्थाय मातले}


\twolineshloka
{अण्डमेतञ्जले न्यस्तं दीप्यमानमिव श्रिया}
{आप्रजानां निसर्गाद्वै नोद्भिद्यति न सर्पति}


\twolineshloka
{नास्य जातिं निसर्गं वा कथ्यमानं श्रृणोमि वै}
{पितरं मातरं चापि नास्य जानाति कश्चन}


\twolineshloka
{अतः किल महानग्निरन्तकाले समुत्थितः}
{धक्ष्यते मातले सर्वं त्रैलोक्यं सचराचरम्}


\twolineshloka
{मातलिस्त्वब्रवीच्छ्रुत्वा नारदस्याथ भाषितम्}
{न मेऽत्र रोचते कश्चिदन्यतो व्रज माचिरम्}


\chapter{अध्यायः १००}
\twolineshloka
{नारद उवाच}
{}


\twolineshloka
{हिरण्यपुरमित्येतत्ख्यातं पुरवरं महत्}
{दैत्यानां दानवानां च मायाशतविचारिणाम्}


\twolineshloka
{अनल्पेन प्रयत्नेन निर्मितं विश्वकर्मणा}
{मयेन मनसा सृष्टं पातालतलमाश्रितम्}


\twolineshloka
{अत्र मायासहस्राणि वविकुर्वाणा महौजसः}
{दानवा निवसन्ति स्म शूरा दत्तवराः पुरा}


\twolineshloka
{नैते शक्रेण नान्येन यमेन वरुणेन वा}
{शक्यन्ते वशमानेतुं तथैव धनदेन च}


\twolineshloka
{असुराः कालखञ्जाश्च तथा वुष्णुपदोद्भवाः}
{नैर्ऋता यातुधानाश्च ब्रह्मपादोद्भवाश्च ये}


\twolineshloka
{दंष्ट्रिणो भीमवेगाश्च वातवेगपराक्रमाः}
{मायावीर्योपसंपन्ना निवसन्त्यत्र मातले}


\twolineshloka
{निवातकवचा नाम दानवा युद्धदुर्मदाः}
{जानासि च यथा शक्रो नैताञ्शक्रोति बाधितुं}


\twolineshloka
{बहुशो मातले त्वं च तव पुत्रश्च गोमुखः}
{निर्भग्नो देवराजश्च सहपुत्रः शचीपतिः}


\twolineshloka
{पश्य वेश्मानि रौक्माणि मातले राजतानि च}
{कर्मणा विधियुक्तेन युक्तान्युपगतानि च}


\twolineshloka
{वैदूर्यमणिचित्राणि प्रवालरुचिराणि च}
{अर्कस्फटिकशुभ्राणि वज्रसारोञ्ज्वलानि च}


\twolineshloka
{पार्थिवानीव चाभास्ति पद्मरागमयानि च}
{शैलानीव च दृश्यन्ते दारवाणीव चाप्युत}


\twolineshloka
{सूर्यरूपाणि चाभान्ति दीप्ताग्निसदृशानि च}
{मणिजालविचित्राणि प्रांशूनि निबिडानि च}


\twolineshloka
{नैतानि शक्यं निर्देष्टु रूपतो द्रव्यतस्तथा}
{गुणतश्चैव सिद्धानि प्रमाणगुणवन्ति च}


\twolineshloka
{आक्रीडान्पश्य दैत्यानां तथैव शयनान्युत}
{रत्नवन्ति महार्हाणि भाजनान्यासनानि च}


\twolineshloka
{जलदाभांस्तथा शैलांस्तोयप्रस्रवणानि च}
{कामपुष्पफलांश्चापि पादपान्कापचारिणः}


\twolineshloka
{मातले कश्चिदत्रापि रुचिरस्ते वरो भवेत्}
{अथवाऽन्यां दिशं भूमेर्गच्छाव यदि मन्यसे}


\twolineshloka
{मातलिस्त्वब्रवीदेनं भाषमाणं तथाविधम्}
{देवर्षे नैव मे कार्यं विप्रियं त्रिदिवौकसाम्}


\twolineshloka
{नित्यानुषक्तवैरा हि भ्रातरो देवदानवाः}
{परपक्षेण संबन्धं रोचयिष्याम्यहं कथम्}


\twolineshloka
{अन्यत्र साधु गच्छाव द्रष्टुं नार्हामि दानवान्}
{जानामि तव चात्मानं हिंसात्मकमनं तथा}


\chapter{अध्यायः १०१}
\twolineshloka
{नारद उवाच}
{}


\twolineshloka
{अयं लोकः सुपर्णानां पक्षिणां पन्नगाशिनाम्}
{विक्रमे गमने भारे नैषामस्ति परिश्रमः}


\twolineshloka
{वैनतेयसुतैः सूत पङ्भिस्ततमिदं कुलम्}
{सुमुखेन सुनाम्ना च सुनेत्रेण सुवर्चसा}


\twolineshloka
{सुरुचा पक्षिराजेन सुबलेन च मातले}
{वर्धितानि प्रसृत्या वै विनाताकुलकर्तृभिः}


\twolineshloka
{पक्षिराजाभिजात्यानां सहस्राणि शतानि च}
{कश्यपस्य ततो वंशे जातैर्भूतिविवर्धनैः}


\twolineshloka
{सर्वे ह्येते श्रिया युक्ताः सर्वे श्रीवत्सलक्षणाः}
{सर्वे श्रियमभीप्सन्तो धारयन्ति बलान्युत}


\twolineshloka
{कर्मणा क्षत्रियाश्चैते निर्घृणा भोगिभोजिनः}
{ज्ञातिसङ्क्षयकर्तृत्वाद्ब्राह्मण्यं न लभन्ति वैः}


\twolineshloka
{नामानि चैषां वक्ष्यामि यथाप्राधान्यतः श्रृणु}
{मातले श्लाघ्यमेतद्धि कुलं विष्णुपरिग्रहम्}


\twolineshloka
{दैवतं विष्णुरेतेषां विष्णुरेव परायणम्}
{हृदि चैषां सदा विष्णुर्विष्णुरेव सदा गतिः}


\twolineshloka
{सुवर्णचूडो नागाशी दारुणश्चण्डतुण्डकः}
{अनिलश्चानलश्चैव विशालाक्षोऽथ कुण्डली}


\twolineshloka
{पङ्कजिद्वज्रविष्कम्भो वैनतेयोऽथ वामनः}
{वातवेगो दिशाचक्षुर्निमेषोऽनिमिषस्तथा}


\twolineshloka
{त्रिरावः सप्तरावश्च वाल्मीकिर्दीपकस्तथा}
{दैत्यद्वीपः सरिद्द्वीपः सारसः पद्मकेतनः}


\twolineshloka
{सुमुखश्चित्रकेतुश्च चित्रबर्हस्तथाऽनघः}
{मेषहृत्कुमुदो दक्षः सर्पान्तः सोमभोजनः}


\twolineshloka
{गुरुभारः कपोतश्च सूर्यनेत्रश्चिरान्तकः}
{विष्णुधर्मा कुमारश्च परिबर्हो हरिस्तथा}


\twolineshloka
{सुस्वरो मधुपर्कश्च हेमवर्णस्तथैव च}
{मालयो मातरिश्वा च निशाकरदिवाकरौ}


\twolineshloka
{एते प्रदेशमात्रेण मयोक्ता गरुडात्मजाः}
{प्राधान्यतस्ते यशसा कीर्तिताः प्राणिनश्चये}


\twolineshloka
{यद्यत्र न रुचिः काचिदेहि गच्छाव मातले}
{तं नयिष्यामि देशं त्वां वरं यत्रोपलप्स्यसे}


\chapter{अध्यायः १०२}
\twolineshloka
{नारद उवाच}
{}


\twolineshloka
{इदं रसातलं नाम सप्तमं पृथिवीतलम्}
{यत्रास्ते सुरभिर्माता गवाममृतसंभवा}


\twolineshloka
{क्षरन्ती सततं क्षीरं पृथिवीसारसंभवम्}
{षण्णं रसानां सारेण रसमेकमनुत्तमम्}


\twolineshloka
{अमृतेनाभितृप्तस्य सारमुद्गिरतः पुरा}
{पितामहस्य वदनादुदतिष्ठदनिन्दिता}


\twolineshloka
{यस्याः क्षीरस्य धाराया निपतन्त्या महीतले}
{ह्रदः कृतः क्षीरनिधिः पवित्रं परमुच्यते}


\twolineshloka
{पुष्पितस्येव फेनेन पर्यन्तमनुवेष्टितम्}
{पिबन्तो निवसन्त्यत्र फेनपा मुनिसत्तमाः}


\twolineshloka
{फेनपा नाम ते ख्याताः फेनाहाराश्च मातले}
{उग्रे तपसि वर्तन्ते येषां बिभ्यति देवताः}


\twolineshloka
{अस्याश्चतस्रो धेन्वोऽन्या दिक्षु सर्वासु मातले}
{निवसन्ति दिशां पाल्यो धारयन्त्योदिशःस्म ताः}


\twolineshloka
{पूर्वां दिशं धारयते सुरूपा नाम सौरभी}
{दक्षिणां हंसिका कनाम धारयत्यपरां दिशम्}


\twolineshloka
{पश्चिमा वारुणी दिक्च धार्यते वै सुभद्रया}
{महानुभावया नित्यं मातले विश्वरूपया}


\twolineshloka
{सर्वकामदुघा नाम धेनुर्धारयते दिशम्}
{उत्तरां मातले धर्म्यां तथैलविलसंश्रिताम्}


\twolineshloka
{आसां त पयसा मिश्रं पयो निर्मथ्य सागरे}
{मन्थानं मन्दरं कृत्वा देवैरसुरसंहितैः}


\twolineshloka
{अद्धृता वारुणी लक्ष्मीरमृतं चापि मातले}
{उच्चैःश्रवाश्चाश्वराजो मणिरत्नं च कौस्तुभम्}


\twolineshloka
{सुधाहारेषु च सुधां स्वधाभोजिषु च स्वधाम्}
{अमृतं चामृताशेषु सुरभी क्षरते पयः}


\twolineshloka
{अत्र गाथा पुरा गीता रसातलनिवासिभिः}
{पौराणी श्रूयते लोके गीयते या मनीषिभिः}


\twolineshloka
{न नागलोके न स्वर्गे न विमाने त्रिविष्टपे}
{परिवासः सुखस्तादृक् रसातलतले यथा}


\chapter{अध्यायः १०३}
\twolineshloka
{नारद उवाच}
{}


\twolineshloka
{इयं भोगवती नाम पुरी वासुकिपालिता}
{यादृशी देवराजस्य पुरी वर्याऽमरावती}


\twolineshloka
{एष शेषः स्थितो नागो येनेयं धार्यते सदा}
{तपसा लोकमुख्येन प्रभावसहिता मही}


\twolineshloka
{श्वेताचलनिभाकारो दिव्याभरणभूषितः}
{सहस्रं धारयन्मूर्ध्नां ज्वालाजिह्वो महाबलः}


\twolineshloka
{इह नानाविधाकारा नानाविधविभूषणाः}
{सुरसायाः सुता नागा निवसन्ति गतव्यथाः}


\twolineshloka
{मणिस्वस्तिकचक्राङ्काः कमण्डलुकलक्षणाः}
{सहस्रसंख्या बलिन सर्वे रौद्राः स्वभावतः}


\twolineshloka
{सहस्रशिरसः केचित्केचित्पञ्चशताननाः}
{शतशीर्षास्तथा केचित्केचित्रिशिरसोऽपि च}


\twolineshloka
{द्विपञ्चशिरसः केचित्केचित्सप्तसुखास्तथा}
{महाभोगा महाकायाः पर्वताभोगभोगिनः}


\twolineshloka
{बहूनीह सहस्राणि प्रयुतान्यर्बुदानि च}
{नागानामेकवंशानां यथाश्रेष्ठं तु मे श्रुणु}


\twolineshloka
{वासुकिस्तक्षकश्चैव कर्कोटकधनञ्जयौ}
{कालीयो नहुषश्चैव कम्बलाश्वतरावुभौ}


\twolineshloka
{बाह्यकुण्डो मणिर्नागस्तथैवापूरणः खगः}
{वामनश्चैलपत्रश्च कुकुरः कुकुणस्तथा}


\twolineshloka
{आर्यको नन्दकश्चैव तथा कलशपोतकौ}
{कैलासकः पिञ्जरको नागश्चैरावतस्तथा}


\twolineshloka
{सुमनोमुखो दधिमुखः शङ्खो नन्दोपनन्दकौ}
{आप्तः कोटरकश्चैव शिखी निष्ठूरिकस्तथा}


\twolineshloka
{तित्तिरिर्हस्तिभद्रश्च कुमुदो माल्यपिण्डकः}
{द्वौ पद्मौ पुण्डरीकश्च पुष्पो मुद्गरपर्णकः}


\twolineshloka
{करवीरः पीठरकः संवृत्तो वृत्त एव च}
{पिण्डारो बिल्वपत्रश्च मूषिकादः शिरीषकः}


\twolineshloka
{दिलीपः शङ्खशीर्षश्च ज्योतिष्कोऽथापराजितः}
{कौरव्यो धृतराष्ट्रश्च कुहुरः कृशकस्तथा}


\twolineshloka
{विरजा धारणश्चैव सुबाहुर्मुखरो जयः}
{बधिरान्धौ विशुण्डिश्च विरसः सुरसस्तथा}


\threelineshloka
{एते चान्ये च बहवः कश्यपस्यात्मजाः स्मृताः}
{मातले पश्य यद्यत्र कश्चित्ते रोचते वरः ॥कण्व उवाच}
{}


\threelineshloka
{मातलिस्त्वेकमव्यग्रः सततं संनिरीक्ष्य वै}
{पप्रच्छ नारदं तत्र प्रीतिमानिव चाभवत् ॥मातलिरुवाच}
{}


\twolineshloka
{स्थितो य एष पुरतः कौरव्यस्यार्यकस्य तु}
{द्युतिमान्दर्शनीयश्च कस्यैष कुलनन्दनः}


\twolineshloka
{कः पिता जननी चास्य कतमस्यैष भोगिनः}
{वंशस्य कस्यैष महान्केतुभूत इव स्थितः}


\threelineshloka
{प्रणिधानेन धैर्येण रूपेण वयसा च मे}
{मनः प्रविष्टो देवर्षे गुणकेश्याः पतिर्वरः ॥कण्व उवाच}
{}


\threelineshloka
{मातलिं प्रीतमनसं दृष्ट्वा सुमुखदर्शनात्}
{निवेदयामास तदा माहात्म्यं जन्म कर्म च ॥नारद उवाच}
{}


\twolineshloka
{ऐरावतकुले जातः सुमुखो नाम नागराट्}
{आर्यकस्य मतः पौत्रो दौहित्रो वामनस्य च}


\twolineshloka
{एतस्य हि पिता नागश्चिकुरो नाम मातले}
{नचिराद्वैतनेयेन पञ्चत्वमुपपादितः}


\twolineshloka
{ततोऽब्रवीत्प्रीतमना मातलिर्नारदं वचः}
{एष मे रुचितस्तात जामाता भुजगोत्तमः}


\twolineshloka
{क्रियतामत्र यत्नो वै प्रीतिमानस्म्यनेन वै}
{अस्मै नागाय वै दातुं प्रियां दुहितरं मुने}


\chapter{अध्यायः १०४}
\twolineshloka
{कण्व उवाच}
{}


\twolineshloka
{मातलेर्वचनं श्रुत्वा नारदो मुनिसत्तमः}
{अब्रवीद्देवराजं तमार्यकं कुरुनन्दन}


\twolineshloka
{सूतोऽयं मातलिर्नाम शक्रस्य दयितः सुहृत्}
{शुचिः शीलगुणोपेतस्तेजस्वी वीर्यवान्बली}


\twolineshloka
{शक्रस्यायं सखा चैव मन्त्री सारथिरेव च}
{अल्पान्तरप्रभावश्च वासवेन रणे रणे}


\twolineshloka
{अयं हरिसहस्रेण युक्तं जैत्रं रथोत्तमम्}
{देवासुरेषु युद्धेषु मनसैव नियच्छति}


\twolineshloka
{अनेन विजितानश्वैर्दोर्भ्यां जयति वासवः}
{अनेन बलभित्पूर्वं प्रहृते प्रहरत्युत}


\twolineshloka
{अस्य कन्या वरारोहा रूपेणासदृशी भुवि}
{सत्यशीलगुणोपेता गुणकेशीति विश्रुता}


\twolineshloka
{तस्यास्य यत्नाच्चरतस्त्रैलोक्यममरद्युते}
{सुमुखो भवतः पौत्रो रोचते दुहितुः पतिः}


\twolineshloka
{यदि ते रोचते सम्यग्भुजगोत्तम मा चिरम्}
{क्रियतामार्यक क्षिप्रं बुद्धिः कन्यापरिग्रहे}


\twolineshloka
{यथा विष्णुकुले लक्ष्मीर्यथा स्वाहा विभावसोः}
{कुले तव तथैवास्तु गुणकेशी सुमध्यमा}


\twolineshloka
{पौत्रस्यार्थे भवांस्तस्माद्गुणकेशीं प्रतीच्छतु}
{सदृशीं प्रतिरूपस्य वासवस्य शचीमिव}


\twolineshloka
{पितृहीनमपि ह्येनं गुणतो वरयामहे}
{बहुमानाच्च भवतस्तथैवैरावतस्य च}


\fourlineindentedshloka
{सुमुखस्य गुणैश्चैव शीलशौचदमादिभिः}
{अभिगम्य स्वयं कन्यामयं दातुं समुद्यतः}
{मातलिस्तस्य संमानं कर्तुमर्हो भवानपि ॥कण्व उवाच}
{}


\threelineshloka
{स तु दीनः प्रहृष्टश्च प्राह नारदमार्यकः}
{व्रियमाणे तथा पौत्रे पुत्रे च निधनं गते ॥आर्यक उवाच}
{}


\threelineshloka
{मन्ये नैतद्बहुमतं महर्षे वचनं तव}
{सखा शक्रस्य संयुक्तः कस्यायं नेप्सितो भवेत्}
{कारणस्य तु दौर्बल्याच्चिन्तयामि महामुने}


\twolineshloka
{अस्य देहकरस्तात मम पुत्रो महाद्युते}
{भक्षितो वैनतेयेन दुःखार्तास्तेन वै वयम्}


\twolineshloka
{पुनरेव च तेनोक्तं वैनतेयेन गच्छता}
{मासेनान्येन सुमुखं भक्षयिष्य इति प्रभो}


\threelineshloka
{ध्रुवं तथा तद्भविता जानीमस्तस्य निश्चयम्}
{तेन हर्षः प्रनष्टो मे सुपर्णवचनेन वै ॥कण्व उवाच}
{}


\twolineshloka
{मातलिस्त्वब्रवीदेनं बुद्धिरत्र कृता मया}
{जामातृभावेन वृतः सुमुखस्तव पुत्रजः}


\twolineshloka
{सोऽयं मया च सहितो नारदेन च पन्नगः}
{त्रिलोकेशं सुरपतिं गत्वा पश्यतु वासवम्}


\twolineshloka
{शेषेणैवास्य कार्येण प्रज्ञास्वाम्यहमायुषः}
{सुपर्णस्य विघाते च प्रयतिष्यामि सत्तम}


\threelineshloka
{सुमुखश्च मया सार्धं देवेशमिगच्छतु}
{कार्यसंसाधनार्थाय स्वस्ति तेऽस्तु भुजङ्गम ॥`कण्व उवाच}
{}


\threelineshloka
{आर्यकेणाभ्यनुज्ञाता गम्यतामिति भारत}
{'ततस्ते सुमुखं गृह्य सर्व एव महौजसः}
{ददृशुः शक्रमासीनं देवराजं महाद्युतिम्}


\threelineshloka
{सङ्गत्या तत्र भगवान्विष्णुरासीच्चतुर्भुजः}
{ततस्तत्सर्वमाचख्यौ नारदो मातलिं प्रति ॥वैशंपायन उवाच}
{}


\twolineshloka
{ततः पुरन्दरं विष्णुरुवाच भुवनेश्वरम्}
{अमृतं दीयतामस्मै क्रियताममरैः समः}


\twolineshloka
{मातलिर्नारदश्चैव सुमुखश्चैव वासव}
{लभन्तां भवतः कामात्काममेतं यथेप्सितम्}


\threelineshloka
{पुरन्दरोऽथ संचिन्त्य वैनतेयपराक्रमम्}
{विष्णुमेवाब्रवीदेनं भवानेव ददात्विति ॥विष्णुरुवाच}
{}


\twolineshloka
{ईशस्त्वं सर्वलोकानां चराणामचराश्च ये}
{त्वया दत्तमदत्तं कः कर्तुमुत्सहते विभो}


\twolineshloka
{प्रादाच्छक्रस्ततस्तस्मै पन्नगायायुरुत्तमम्}
{न त्वेनममृतप्राशं चकार बलवृत्रहा}


\twolineshloka
{लब्ध्वा वरं तु सुमुखः सुमुखः संबभूव ह}
{कृतदारो यथाकामं जगाम च गृहान्प्रति}


\twolineshloka
{नारदस्त्वार्यकश्चैव कृतकार्यौ मुदा युतौ}
{अभिजग्मतुरभ्यर्च्य देवराजं महाद्युतिम्}


\chapter{अध्यायः १०५}
\twolineshloka
{कण्व उवाच}
{}


\twolineshloka
{गरुडस्तत्र शुश्राव यथावृत्तं महाबलः}
{आयुःप्रदानं शक्रेण कृतं नागस्य भारत}


\threelineshloka
{पक्षवातेन महता रुद्ध्वा त्रिभुवनं खगः}
{सुपर्णः परमक्रुद्धो वासवं समुपाद्रवत् ॥गरुड उवाच}
{}


\twolineshloka
{भगवन्किमवज्ञानाद्वृत्तिः प्रतिहता मम}
{कामकारवरं दत्त्वा पुनश्चलितवानसि}


\twolineshloka
{निसर्गात्सर्वभूतानां सर्वभूतेश्वरेण मे}
{आहारो विहितो धात्रा किमर्थं वार्यते त्वया}


\twolineshloka
{वृतश्चैष महानागः स्थापितः समयश्च मे}
{अनेन च मया देव भर्तव्यः प्रसवो महान्}


\twolineshloka
{एतस्मिंस्तु तथाभूते नान्यं हिंसितुमुत्सहे}
{क्रीडसे कामकारेण देवराज यथेच्छकम्}


\threelineshloka
{सोऽहं प्राणान्विमोक्ष्यामि तथा परिजनो मम}
{ये च भृत्या मम गृहे प्रीतिमान्भव वासव ॥`कण्व उवाच}
{}


\fourlineindentedshloka
{श्रुत्वा सुपर्णवचनं सुमुखो दुर्मुखस्तदा}
{त्यक्त्वा रूपं विवर्णस्तु सर्परूपधरोऽभवत्}
{गत्वा विष्णुसमीपं तु पादपीठं समाश्लिषत् ॥इन्द्र उवाच}
{}


\threelineshloka
{न मत्कृतं वैनतेय न मां क्रोद्धुं त्वमर्हसि}
{दत्ताभयः स सुमुखो विष्णुना प्रभविष्णुना}
{श्रत्वा पुरन्दरेणोक्तमुवाच विनतासुतः ॥'}


\twolineshloka
{एतच्चैवाहमर्हामि भूयश्च बलवृत्रहन्}
{त्रैलोक्यस्येश्वरो योऽहं परभृत्यत्वमागतः}


\twolineshloka
{त्वयि तिष्ठति देवेश न विष्णुः कारणं मम}
{त्रैलोक्यराजराज्यं हि त्वयि वासव शाश्वतम्}


\twolineshloka
{ममापि दक्षस्य सुता जननी कश्यपः पिता}
{अहमत्युत्सहे लोकान्समन्ताद्वोढुमोजसा}


\twolineshloka
{असह्यं सर्वभूतानां ममापि विपुलं बलम्}
{मयाऽपि सुमहत्कर्म कृतं दैतेयविग्रहे}


\twolineshloka
{श्रुतश्रीः श्रुतसेनश्च विवस्वान्रोचनामुखः}
{प्रस्रुतः कालकाक्षश्च मयाऽपि दितिजा हताः}


\twolineshloka
{यत्तु ध्वजस्थानगतो यत्नात्परिचराम्यहम्}
{वहामि चैवानुजं ते तेन मामवमन्यसे}


\twolineshloka
{कोऽन्यो भारसहो ह्यस्ति कोऽन्योस्ति बलवत्तरः}
{मया योऽहं विशिष्टः सन्वहामीमं सबान्धवं}


\twolineshloka
{अवज्ञाय तु यत्तेऽहं भोजनाद्व्यपरोपितः}
{तेन मे गौरवं नष्टं त्वत्तश्चास्माच्च वासव}


\twolineshloka
{अदित्यां य इमे जाता बलविक्रमशालिनः}
{त्वमेषां किल सर्वेषां बलेन बलवत्तरः}


\threelineshloka
{सोऽहं पक्षैकदेशेन वहामि त्वां गतक्लमः}
{विमृश त्वं शनैस्तात कोऽन्वत्र बलवानिति ॥कण्व उवाच}
{}


\twolineshloka
{स तस्य वचनं श्रुत्वा खगस्योदर्कदारुणम्}
{अक्षोभ्यं क्षोभयंस्तार्क्ष्यमुवाच रथचक्रभृत्}


\twolineshloka
{गरुत्मन्मन्यसेत्मानं बलवन्तं सुदुर्बलम्}
{अलमस्मत्समक्षं ते स्तोतुमात्मानमण्डज}


\twolineshloka
{त्रैलोक्यमपि मे कृत्स्नमशक्तं देहधारणे}
{अहमेवात्मनात्मानं वहामि त्वां च धारये}


\threelineshloka
{`न त्वं वहसि मां दोर्भ्यां मोघं तव विकत्थनम्}
{'इमं तावन्ममैकं त्वं बाहुं सव्येतरं वह}
{यद्येनं धारयस्येकं सफलं ते विकत्थितम्}


\threelineshloka
{इत्युक्त्वा भगवांस्तस्य स्कन्धे बाहुं समासजत्}
{`आरोपितं समुद्वोढुं भारं तं नाशकद्बलात्}
{'निपपात स भारार्तो विहलो नष्टचेतनः}


\twolineshloka
{यावान्हि भारः कृत्स्नायाः पृथिव्याः पर्वतैः सह}
{एकस्या देहशाखायास्तावद्भारममन्यत}


\twolineshloka
{न त्वेनं पीडयामास बलेन बलवत्तरः}
{ततो हि जीवितं तस्य न व्यनीनशदच्युतः}


\twolineshloka
{व्यात्तास्यः स्रस्तकायश्च विचेता विह्वलः खगः}
{मुमोच पत्राणि तदा गुरुभारप्रपीडितः}


\twolineshloka
{स विष्णुं शिरसा पक्षी प्रणम्य विनतासुतः}
{विचेता विह्वलो दीनः किंचिद्वचनमब्रवीत्}


\twolineshloka
{भगवँल्लोकसारस्य सदृशेन वपुष्मता}
{भुजेन स्वैरमुक्तेन निष्पिष्टोऽस्मि महीतले}


\twolineshloka
{क्षन्तुमर्हसि मे देव विह्वलस्याल्पचेतसः}
{बलदाहविदग्धस्य पक्षिणो ध्वजवासिनः}


\threelineshloka
{न हि ज्ञातं बलं देव मया ते परम विभो}
{तेन मन्याम्यहं वीर्यमात्मनो न समं परैः ॥कण्व उवाच}
{}


\twolineshloka
{ततश्चक्रे स भगवान्प्रसादं वै गरुत्मतः}
{मैवं भूय इति स्नेहात्तदा चैनमुवाच ह}


\twolineshloka
{पादाङ्गुष्ठेन चिक्षेप सुमुखं गरुडोरसि}
{ततः प्रभृति राजेन्द्र सहसर्पेण वर्तते}


\threelineshloka
{एवं विष्णुबलाक्रान्तो गर्वनाशमुपागतः}
{गरुडो बलवान्राजन्वैनतेयो महायशाः ॥कण्व उवाच}
{}


\twolineshloka
{तथा त्वमपि गान्धारे यावत्पाण्डुसुतान्रणे}
{नासादयसि तान्वीरांस्तावज्जीवसि पुत्रक}


\twolineshloka
{भीमः प्रहरतां श्रेष्ठो वायुपुत्रो महाबलः}
{धनञ्जयश्चेन्द्रसुतो न हन्यातां तु कं रणे}


\twolineshloka
{विष्णुर्वायुश्च शक्रश्च धर्मस्तौ चाश्विनावुभौ}
{एते देवास्त्वया केन हेतुना वीक्षितुं क्षमाः}


\twolineshloka
{तदलं ते विरोधेन शमं गच्छ नृपात्मज}
{वासुदेवेन तीर्थेन कुलं रक्षितुमर्हसि}


\threelineshloka
{प्रत्यक्षदर्शी सर्वस्य नारदोऽयं महातपाः}
{माहात्म्यस्य तदा विष्णोः सोऽयं चक्रगदाधरः ॥वैशंपायन उवाच}
{}


\twolineshloka
{दुर्योधनस्तु तच्छ्रुत्वा निश्वसन्भृकुटीमुखः}
{राधेयमभिसंप्रेक्ष्य जहास स्वनवत्तदा}


\twolineshloka
{कदर्थीकृत्य तद्वाक्यमृषेः कण्वस्य दुर्मतिः}
{ऊरुं गजकराकारं ताडयन्निदमब्रवीत्}


\twolineshloka
{यथैवेश्वरसृष्टोऽस्मि यद्भावि या च मे गतिः}
{तथा महर्षे वर्तामि किं प्रलापः करिष्यति}


\twolineshloka
{` ततः कण्वोऽब्रवीत्क्रुद्धो दुर्योधनमपण्डितम्}
{यस्मादूकं ताडयसि ऊरौ मृत्युर्भविष्यति ॥'}


\chapter{अध्यायः १०६}
\twolineshloka
{जनमेजय उवाच}
{}


\twolineshloka
{अनर्थे जातिनिर्बन्धं परार्थे लोभमोहितम्}
{अनार्यकेष्वभिरतं मरणे कृतनिश्चयम्}


\twolineshloka
{ज्ञातीनां दुःखकर्तारं बन्धूनां शोकवर्धनम्}
{सुहृदां क्लेशदातारं द्विषतां हर्षवर्धनम्}


\threelineshloka
{कथं नैनं विमार्गस्थं वारयन्तीह बान्धवाः}
{सौहृदाद्वा सुहृत्स्निग्धो भगवान्वा पितामहः ॥वैशंपायन उवाच}
{}


\threelineshloka
{उक्तं भगवता वाक्यमुक्तं भीष्मेण यत्क्षमम्}
{उक्तं बहुविधं चैव नारदेनापि तच्छृणु ॥नारद उवाच}
{}


\twolineshloka
{दुर्लभो वै सुहृच्छ्रोता दुर्लभश्च हितः सुहृत्}
{तिष्ठते हि सुह्यद्यत्र न बन्धुस्तत्र तिष्ठते}


\twolineshloka
{श्रोतव्यमपि पश्यापि सुहृदां कुरुनन्दन}
{न कर्तव्यश्च निर्बन्धो निर्बन्धो हि सुदारुणः}


\twolineshloka
{अत्राप्युदाहरन्तीममितिहासं पुरातनम्}
{यथा निर्बन्धतः प्राप्तो गालवेन पराजयः}


\twolineshloka
{विश्वामित्रं तपस्यन्तं धर्मो जिज्ञासया पुरा}
{अभ्यगच्छत्स्वयं भूत्वा वसिष्ठो भगवानृषिः}


\twolineshloka
{सप्तर्षीणामन्यतमं वेषमास्थाय भारत}
{बभुक्षुः क्षुधितो राजन्नाश्रमं कौशिकस्य तु}


\twolineshloka
{विश्वामित्रोऽथ संभ्रान्तः श्रपयामास वै चरुम्}
{परमान्नस्य यत्नेन न च तं प्रत्यपालयत्}


\twolineshloka
{अन्नं तेन यदा भुक्तमन्यैर्दत्तं तपस्विभिः}
{अथ गृह्णान्नमत्युष्णं विश्वामित्रोऽप्युपागमत्}


\twolineshloka
{भुक्तं मे तिष्ठ तावत्त्वमित्युक्त्वा भगवान्ययौ}
{विश्वामित्रस्ततो राजन्स्थित एव महाद्युतिः}


\twolineshloka
{भक्तं प्रगृह्य मूर्ध्ना वै बाहुभ्यां संशितव्रतः}
{स्थितः स्थाणुरिवाभ्याशे निश्चेष्टो मारुताशनः}


\twolineshloka
{तस्य शुश्रूषणे यत्नमकरोद्गालवो मुनिः}
{गौरवाद्बहुमानाच्च हार्देन प्रियकाम्यया}


\twolineshloka
{अथ वर्षशते पूर्णे धर्मः पुनरुपागमत्}
{वासिष्ठं वेषमास्थाय कौशिकं भोजनेप्सया}


\twolineshloka
{स दृष्ट्वा शिरसा भक्तं ध्रियमाणं महर्षिणा}
{तिष्ठता वायुभक्षेण विश्वामित्रेण धीमता}


\twolineshloka
{प्रतिगृह्य ततो धर्मस्तथैवोष्णं तथा नवम्}
{भुक्त्वा प्रीतोस्मि विप्रर्षे तमुक्त्वा स मुनिर्गतः}


\twolineshloka
{क्षत्रभावादपगतो ब्राह्मणत्वमुपागतः}
{धर्मस्य वचनात्प्रीतो विश्वामित्रस्तथाऽभवत्}


\twolineshloka
{विश्वामित्रस्तु शिष्यस्य गालवस्य तपस्विनः}
{शुश्रूषया च भक्त्या च प्रीतिमानित्युवाच ह}


\twolineshloka
{अनुज्ञातो मया वत्स यथेष्टं गच्छ गवालव}
{इत्युक्तः प्रत्युवाचेदं गालवो मुनिसत्तमम्}


\twolineshloka
{प्रीतो मधुरया वाचा विश्वामित्रं महाद्युतिम्}
{दक्षिणाः काः प्रयच्छामि भवते गुरुकर्मणि}


\twolineshloka
{दक्षिणाभिरुपेतं हि कर्म सिद्ध्यति मानद}
{दक्षिणानां हि दाता वै अपवर्गेण युज्यते}


\twolineshloka
{स्वर्गे क्रतुफलं तद्धि दक्षिणा शान्तिरुच्यते}
{किमाहरामि गुर्वर्थं ब्रवीतु भगवानिति}


\twolineshloka
{स जानानस्तु भगवान्खिन्नं शुश्रूषणेन वै}
{विश्वामित्रस्तमसकृद्गच्छ गच्छेत्यचोदयत्}


\twolineshloka
{असकृद्गच्छगच्छेति विश्वामित्रेण भाषितः}
{किं ददानीति बहुशो गालवः प्रत्यभाषत}


\twolineshloka
{निर्बन्धतस्तु बहुशो गालवस्य तपस्विनः}
{किंचिदागतसंरम्भो विश्वामित्रोऽब्रवीदिदम्}


\twolineshloka
{एकतःश्यामकर्णानां हयानां चन्द्रवर्चसाम्}
{अष्टौ शतानि मे देहि गच्छ गालव मा चिरम्}


\chapter{अध्यायः १०७}
\twolineshloka
{नारद उवच}
{}


\twolineshloka
{एवमुक्तस्तदा तेन विश्वामित्रेण धीमता}
{नास्ते न शेते नाहारं कुरुते गालवस्तदा}


\threelineshloka
{त्वगस्थिभूतो हरिणाश्चिन्ताशोकपरायणः}
{शोचमानोऽतिमात्रं स दह्यमानश्च मन्युना}
{गालवो दुःखितो दुःखाद्विललाप सुयोधन}


\twolineshloka
{कुतः पुष्टानि मित्राणि कुतोऽर्थाः सञ्चयः कृतः}
{हयानां चन्द्रशुभ्राणां शतान्यष्टौ कुतो मम}


\twolineshloka
{कुतो मे भोजने श्रद्धा सुखश्रद्धा कुतश्च मे}
{श्रद्धा मे जीवितस्यापि छिन्ना किं जीवितेन मे}


\twolineshloka
{अहं पारे समुद्रस्य पृथिव्या वा परं परात्}
{गत्वाऽऽत्मानं विमुञ्चामि किं फलं जीवितेन मे}


\twolineshloka
{अदनस्याकृतार्थस्य त्यक्तस्य विविधैः फलैः}
{ऋणं धारयमाणस्य कुतः सुखमनीहया}


\twolineshloka
{सुहृदां हि धनं भुक्त्वा कृत्वा प्रणयमीप्सितम्}
{प्रतिकर्तुमशक्तस्य जीवितान्मरणं वरम्}


\twolineshloka
{प्रतिश्रुत्य करिष्येति कर्तव्यं तदकुर्वतः}
{मिथ्यावचनदग्धस्य इष्टापूर्तं प्रणश्यति}


\twolineshloka
{न रूपमनृतस्यास्ति नानृतस्यास्ति सन्ततिः}
{नानृतस्याधिपत्यं च कुत एव गतिः शुभा}


\twolineshloka
{कुतः कृतघ्नस्य यशः कुतः स्थानं कुतः सुखम्}
{अश्रद्धेयः कृतघ्नो हि कृतघ्ने नास्ति निष्कृतिः}


\twolineshloka
{न जीवत्यधनः पापः कुतः पापस्य तन्त्रणम्}
{पापो ध्रुवमवाप्नोति विनाशं नाशयन्कृतम्}


\twolineshloka
{सोऽहं पापः कृतघ्नस्य कृपणश्चानृतोऽपि च}
{गुरोर्यः कृतकार्यः संस्तत्करोमि न भाषितम्}


\twolineshloka
{सोऽहं प्राणान्विमोक्ष्यामि कृत्वा यत्नमनुत्तमम्}
{अर्थिता न मया काचित्कृतपूर्वा दिवौकसाम्}


\threelineshloka
{मानयन्ति च मां सर्वे त्रिदशा यज्ञसंस्तरे}
{अहं तु विबुधश्रेष्ठं देवं त्रिभुवनेश्वरम्}
{विष्णुं गच्छाम्यहं कृष्णं गतिं गतिमतां वरम्}


\twolineshloka
{भोगा यस्मात्प्रतिष्ठन्ते व्याप्य सर्वान्सुरासुरान्}
{प्रणतो द्रष्टुमिच्छामि कृष्णं योगिनमव्ययम्}


\twolineshloka
{एवमुक्ते सखा तस्य गरुडो विनतात्मजः}
{दर्शयामास तं प्राह संहृष्टः प्रियकाम्यया}


\twolineshloka
{सहृद्भवान्मम मतः सुहृदां च मतः सुहृत्}
{ईप्सितेनाभिलाषेण योक्तव्यो विभवे सति}


\twolineshloka
{विभवश्चास्ति मे विप्र वासवावरजो द्विज}
{पूर्वमुक्तस्त्वदर्थं च कृतः कामश्च तेन मे}


\twolineshloka
{स भवानेतु गच्छाव नयिष्ये त्वां यथासुखम्}
{देशं पारं पृथिव्या वा गच्छ गालव मा चिरम्}


\chapter{अध्यायः १०८}
\twolineshloka
{सुपर्ण उवाच}
{}


\twolineshloka
{अनुशिष्टोऽस्मि देवेन गालव ज्ञानयोनिना}
{ब्रूहि कामं तु कां यामि द्रष्टुं प्रथममो दिशम्}


\twolineshloka
{पूर्वां वा दक्षिणां वाहमथवा पश्चिमां दिशम्}
{उत्तरां वा द्विजश्रेष्ठ कुतो गच्छामि गालव}


\twolineshloka
{यस्यामुदयते पूर्वं सर्वलोकप्रभावनः}
{सविता यत्र सन्ध्यायां साध्यानां वर्तते तपः}


\twolineshloka
{यस्यां पूर्वं मतिर्जाता यया व्याप्तमिदं जगत्}
{चक्षुषी यत्र धर्मस्य यत्र चैष प्रतिष्ठितः}


% Check verse!
कृतं यतो हुतं हव्यं सर्पते सर्वतोदिशम् ॥एतद््द्वारं द्विजश्रेष्ठ दिवसस्य तथाऽध्वनः
\twolineshloka
{अत्र पूर्वं प्रसूता वै दाक्षायण्यः प्रजाः स्त्रियः}
{यस्यां दिशि प्रवृद्धाश्च कश्यपस्यात्मसंभवाः}


\twolineshloka
{अदोमूला सुराणां श्रीर्यत्र शक्रोऽभ्यषिच्यत}
{सुरराज्येन विप्रर्षे देवैश्चात्र तपश्चितम्}


\twolineshloka
{एतस्मात्कारणाद्ब्रह्मन्पूर्वेत्येषा दिगुच्यते}
{यस्मात्पूर्वतरे काले पूर्वमेवावृतासुरैः}


\twolineshloka
{अत एव च सर्वेषां पूर्वामाशां प्रचक्षते}
{पूर्वं सर्वाणि कार्याणि दैवानि सुखमीप्सता}


\twolineshloka
{अत्र वेदाञ्जगौ पूर्वं भगवाँल्लोकभावनः}
{अत्रैवोक्ता सवित्रासीत्सावित्री ब्रह्मवादिषु}


\twolineshloka
{अत्र दत्तानि सूर्येण यजूंषि द्विजसत्तम}
{अत्र लब्धवरः सोमः सुरैः क्रतुषु पीयते}


\twolineshloka
{अत्र तृप्ता हुतवहाः स्वां योनिमुपभुञ्जते}
{अत्र पातालमाश्रित्य वरुणः श्रियमाप च}


\twolineshloka
{अत्र पूर्वं वसिष्ठस्य पौराणस्य द्विजर्षभ}
{सूतिश्चैव प्रतिष्ठा च निधनं च प्रकाशते}


\twolineshloka
{ओङ्कारस्यात्र जायन्ते सृतयो दशतीर्दश}
{पिबन्ति मुनयो यत्र हविर्धूमं स्म धूमपाः}


\twolineshloka
{प्रोक्षिता यत्र बहवो वराहाद्या मृगा वने}
{शक्रेण यज्ञभागार्थे दैवतेषु प्रकल्पिताः}


\twolineshloka
{अत्राहिताः कृतघ्नाश्च मानुषाश्चासुराश्च ये}
{उदयंस्तान्हि सर्वान्वै क्रोधाद्धन्ति विभावसुः}


\twolineshloka
{एतद्द्वारं त्रिलोकस्य स्वर्गस्य च सुखस्य च}
{एष पूर्वो दिशां भागो विशावोऽत्र यदीच्छसि}


\twolineshloka
{प्रियं कार्यं हि मे तस्य यस्यास्मि वचने स्थितः}
{ब्रूहि गालव यास्यामि शृणु चाप्यपरां दिशम्}


\chapter{अध्यायः १०९}
\twolineshloka
{सुपर्ण उवाच}
{}


\twolineshloka
{इयं विवस्वदा पूर्वं श्रौतेन विधिना किल}
{गुरवे दक्षिणा दत्ता दक्षिणेत्युच्यते च दिक्}


\twolineshloka
{अत्र लोकत्रयस्यास्य पितृपक्षः प्रतिष्ठितः}
{अत्रोष्मपाणां देवानां निवासः श्रूयते द्विज}


\twolineshloka
{अत्र विश्वे सदा देवाः पितृभिः सार्धमासते}
{इज्यमानाः स्म लोकेषु संप्राप्तास्तुल्यभागताम्}


\twolineshloka
{एतद्द्वितीयं देवस्य द्वारमाचक्षते द्विज}
{त्रुटिशो लवशश्चापि गण्यते कालनिश्चयः}


\twolineshloka
{अत्र देवर्षयो नित्यं पितृलोकर्षयस्तथा}
{तथा राजर्षयः सर्वे निवसन्ति गतव्यथाः}


\twolineshloka
{अत्र धर्मश्च सत्यं च कर्म चात्र निगद्यते}
{गतिरेषा द्विजश्रेष्ठ कर्मणामवसायिनाम्}


\twolineshloka
{एषा दिक्सा द्विजश्रेष्ठ यां सर्वः प्रतिपद्यते}
{वृता त्वनवबोधेन सुखं तेन न गम्यते}


\twolineshloka
{नैर्ऋतानां सहस्रामि बहून्यत्र द्विजर्षभ}
{सृष्टानि प्रतिकूलानि द्रष्टव्यान्यकृतात्मभिः}


\twolineshloka
{अत्र मन्दरकुञ्जेषु विप्रर्षिसदनेषु च}
{गायन्ति गाथा गन्धर्वाश्चित्तबुद्धिहरा द्विज}


\twolineshloka
{अत्र सामानि गाथाभिः श्रुत्वा गीतानि रैवतः}
{गतदारो गतामात्यो गतराज्यो वनं गतः}


\twolineshloka
{अत्र सावर्मिना चैव यवक्रीतात्मजेन च}
{मर्यादा स्थापिता ब्रह्मन्यां सूर्यो नातिवर्तते}


\twolineshloka
{अत्र राक्षसराजेन पौलस्त्येन महात्मना}
{रावणेन तपः कृत्वा सुरेभ्योऽमरता वृता}


\twolineshloka
{अत्र वृत्तेन वृत्रोऽपि शक्रशत्रुत्वमेयिवान्}
{अत्र सर्वासवः प्राप्ताः पुनर्गच्छन्ति पञ्चधा}


\twolineshloka
{अत्र दुष्कृतकर्माणो नराः पच्यन्ति गालव}
{अत्र वैतरणी नाम नदी वैतरणैर्वृता}


\twolineshloka
{अत्र गत्वा सुखस्यान्तं दुःखस्यान्तं प्रपद्यते}
{अत्रावृत्तो दिनकरः सुरसं क्षरते पयः}


\twolineshloka
{काष्ठां चासाद्य वासिष्ठीं हिममुत्सृजते पुनः}
{अत्राहं गालव पुरा क्षुधाऽऽर्तः परिचिन्तयन्}


\twolineshloka
{लब्धवान्युध्यमानौ द्वौ बृहन्तौ गजकच्छपौ}
{अत्र चक्रधनुर्नाम सूर्याञ्जातो महानृषिः}


\twolineshloka
{विदुर्यं कपिलं देवं येनार्ताः सगरात्मजाः}
{अत्र सिद्धाः शिवा नाम ब्राह्मणा वेदपारगाः}


\twolineshloka
{अधीत्य सकलान्वेदाँल्लेभिरे मोक्षमक्षयम्}
{अत्र भोगवती नाम पुरी वासुकिपालिता}


\twolineshloka
{तक्षकेण च नागेन तथैवैरावतेन च}
{अत्र निर्याणकालेऽपि तमः संप्राप्यते महत्}


\threelineshloka
{अभेद्यं भास्करेणापि स्वयं वा कृष्णवर्त्मना}
{एष तस्यापि ते मार्गः परिचारस्य गालव}
{ब्रूहि मे यदि गन्तव्यं प्रतीचीं शृणु चापराम्}


\chapter{अध्यायः ११०}
\twolineshloka
{सुपर्ण उवाच}
{}


\twolineshloka
{इयं दिग्दियिता राज्ञो वरुणस्य तु गोपतेः}
{सदा सलिलराजस्य प्रतिष्ठा चादिरेव च}


\twolineshloka
{अत्र पश्चादहः सूर्यो विसर्जयति गाः स्वयम्}
{पश्चिमेत्यभिविख्याता दिगियं द्विजसत्तम}


\twolineshloka
{यादसामत्र राज्येन सलिलस्य च गुप्तये}
{कश्यपो भगवान्देवो वरुणं स्माभ्यषेचयत्}


\twolineshloka
{अत्र पीत्वा समस्तान्वै वरुणस्य रसांस्तु षट्}
{जायते तरुणः सोमः शुक्लस्यादौ तमिस्रहा}


\twolineshloka
{अत्र पञ्चात्कृता दैत्या वायुना संयतास्तदा}
{निश्चसन्तो महावातैरर्दिताः सुषुपुर्द्विज}


\twolineshloka
{अत्र सूर्यं प्रणयिनं प्रतिगृह्णाति पर्वतः}
{अस्तो नाम यतः सन्ध्या पश्चिमा प्रतिसर्यति}


\twolineshloka
{अतो रात्रिश्च निद्रा च निर्गता दिवसक्षये}
{जायते जीवलोकस्य हर्तुमर्धमिवायुषः}


\twolineshloka
{अत्र देवीं दितिं सुप्तामात्मप्रसवधारिणीम्}
{विगर्भामकरोच्छक्रो यत्र जातो मरुद्गणः}


\twolineshloka
{अत्र मूलं हिमवतो मन्दरं याति शाश्वतम्}
{अपि वर्षसहस्रेण न चास्यान्तोऽधिगम्यते}


\twolineshloka
{अत्र काञ्चनशैलस्य काञ्चनाम्बुरुहस्य च}
{उदधेस्तीरमासाद्य सुरभिः क्षरते पयः}


\twolineshloka
{अत्र मध्ये समुद्रस्य कबन्धः प्रतिदृश्यते}
{स्वर्भानोः सूर्यकल्पस्य सोमसूर्यौ जिघांसतः}


\twolineshloka
{सुवर्णशिरसोऽप्यत्र हरिरोम्णः प्रगायतः}
{अदृश्यस्याप्रमेयस्य श्रूयते विपुलो ध्वनिः}


\twolineshloka
{अत्र ध्वजवती नाम कुमारी हरिमेधसः}
{आकाशे तिष्ठतिष्ठेति तस्थौ सूर्यस्य शासनात्}


\twolineshloka
{अत्र वायुस्तथा वह्निरापः खं चापि गालव}
{आह्निकं चैव नैशं च दुःखं स्पर्शं विमुञ्चति}


\twolineshloka
{अतःप्रभृति सूर्यस्य तिर्यगावर्तते गतिः}
{अत्र ज्योतींषि सर्वाणि विशन्त्यादित्यमण्डलम्}


\twolineshloka
{अष्टाविंशतिरात्रं च क्रम्य सह भानुना}
{निष्पतन्ति पुनः सूर्यात्सोमसंयोगयोगतः}


\twolineshloka
{अत्र नित्यं स्रवन्तीनां प्रभवः सागरोदयः}
{अत्र लोकत्रयस्यापस्तिष्ठन्ति वरुणालये}


\twolineshloka
{अत्र पन्नगराजस्याप्यनन्तस्य निवेशनम्}
{अनादिनिधनस्यात्र विष्णोः स्थानमनुत्तमम्}


\twolineshloka
{अत्रानलसखस्यापि पवनस्य निवेशनम्}
{महर्षेः कश्यपस्यात्र मारीचस्य निवेशनम्}


\twolineshloka
{एष ते पश्चिमो मार्गो दिग्द्वारेण प्रकीर्तितः}
{ब्रूहि गालव गच्छावो बुद्धिः का द्विजसत्तम}


\chapter{अध्यायः १११}
\twolineshloka
{सपर्ण उवाच}
{}


\twolineshloka
{यस्मादुत्तार्यते पापाद्यस्मान्निश्रेयसोऽश्रुते}
{अस्मादुत्तारणबलादुत्तरेत्युच्यते द्विज ॥ 1}


\twolineshloka
{उत्तरस्य हिरण्यस्य परिवापस्य गालव}
{मार्गः पश्चिमपूर्वाभ्यां दिग्भ्यां वै मध्यमः स्मृतः}


\twolineshloka
{अस्यां दिशि वरिष्ठायामुत्तरायां द्विजर्षभ}
{नासौम्यो नाविधेयात्मा नाधर्मो वसते जनः}


\twolineshloka
{अत्र नारायणः कृष्णो जिष्णुश्चैव नरोत्तमः}
{बदर्यामाश्रमपदे तथा ब्रह्मा च शाश्वतः}


\twolineshloka
{अत्र वै हिमवत्पृष्ठे नित्यमास्ते महेश्वरः}
{`प्रकृत्या पुरुषः सार्धं युगान्ताग्निसमप्रभः}


\twolineshloka
{न स दृश्यो मुनिगणैस्तथा देवैः सवासवैः}
{गन्धर्वयक्षसिद्धैर्वा नरनारायणादृते}


\twolineshloka
{अत्र विष्णुः सहस्राक्षः सहस्रचरणोऽव्ययः}
{सहस्रशिरसः श्रीमानेकः पश्यति मायया ॥'}


\twolineshloka
{अत्र राज्येन विप्राणां चन्द्रमाश्चाभ्यषिच्यत}
{अत्र गङ्गां महादेवः पतन्तीं गगनाच्च्युताम्}


\twolineshloka
{प्रतिगृह्य ददौ लोके मानुषे ब्रह्मवित्तम}
{अत्र देव्या तपस्तप्तं महेश्वरपरीप्सया}


\twolineshloka
{अत्र कामश्च रोषश्च शैलश्चोमा च संबभुः}
{अत्र राक्षसयक्षाणां गन्धर्वाणां च गालव}


\twolineshloka
{आधिपत्येन कैलासे धनदोऽप्यभिषेचितः}
{अत्र चैत्ररथं रम्यमत्र वैखानसाश्रमः}


\twolineshloka
{अत्र मन्दाकिनी चैव मन्दरश्च द्विजर्षभ}
{अत्र सौगन्धिकवनं नैर्ऋतैरभिरक्ष्यते}


\twolineshloka
{शाद्वलं कदलीस्कन्धमत्र सन्तानका नगाः}
{अत्र संयमनित्यानां सिद्धानां स्वैरचारिणाम्}


\twolineshloka
{विमानान्युनुरूपाणि कामभोग्यानि गालव}
{अत्र ते ऋषयः सप्त देवी चारुन्धती तथा}


\twolineshloka
{अत्र तिष्ठति वै रात्रिन्दिवाप्यत्रावतिष्ठते}
{अत्र यज्ञं समासाद्य ध्रुवं स्थाता पितामहः}


\twolineshloka
{ज्योतींषि चन्द्रसूर्यौ च परिवर्तन्ति नित्यशः}
{अत्र गङ्गामहाद्वारं रक्षन्ति द्विजसत्तम}


\twolineshloka
{धामा नाम महात्मानो मुनयः सत्यवादिनः}
{न तेषां ज्ञायते मूर्तिर्नाकृतिर्न तपश्चितम्}


\threelineshloka
{परिवर्तसहस्रामि कामभोज्याननि गालव}
{यथायथा प्रविशति तस्मात्परतरं नरः}
{}


\twolineshloka
{तथातथा द्विजश्रेष्ठ प्रविलीयति गालव}
{नैतत्केनचिदन्येन गतपूर्वं द्विजर्षभ}


\twolineshloka
{ऋते नारायणं देवं नरं वा जिष्णुमव्ययम्}
{अत्र कैलासमित्युक्तं स्थानमैलविलस्य तत्}


\twolineshloka
{अत्र विद्युत्प्रभा नाम जझिरेऽप्सरसो दश}
{अत्र विष्णुपदं नाम क्रमता विष्णुना कृतम्}


\twolineshloka
{त्रिलोकविक्रमे ब्रह्मन्नुत्तरां दिशमाश्रितम्}
{अत्र राज्ञा मरुतेन यज्ञेनेष्टं द्विजोत्तम}


\twolineshloka
{उशीरबीजे विप्रर्षे यत्र जाम्बूनदं सरः}
{जीमूतस्यात्र विप्रर्षेरुपतस्थे महात्मनः}


\twolineshloka
{साक्षाद्धैमवतः पुण्यो विमलः कनकाकरः}
{ब्राह्मणेषु च यत्कृत्स्नं स्वन्तं कृत्वा धनं महत्}


\twolineshloka
{वव्रे धनं महर्षिः स जैमूतं तद्धनं ततः}
{अत्र नित्यं दिशांपालाः सायंप्रातर्द्विजर्षभ}


\twolineshloka
{कस्य कार्यं किमिति वै परिक्रोशान्ति गालव}
{एवमेषा द्विजश्रेष्ठ गुणैरन्यैर्दिगुत्तरा}


\twolineshloka
{उत्तरेति परिख्याता सर्वकर्मस्तु चोत्तरा}
{एता विस्तरशस्तात तव शङ्गीर्तिता दिशः}


\threelineshloka
{चतस्रः क्रमयोगेन कामाशां गन्तुमिच्छसि}
{उद्यतोऽहं द्विजश्रेष्ठ तव दर्शयितुं दिशः}
{पृथिवीं चाखिलां ब्रह्मंस्तस्मादारोह मां द्विज}


\chapter{अध्यायः ११२}
\twolineshloka
{गालव उवाच}
{}


\twolineshloka
{गरुत्मन्भुजगेन्द्रारे सुपर्ण विनतात्मज}
{नय मां तार्क्ष्य पूर्वेण यत्र धर्मस्य चक्षुषी}


\twolineshloka
{पूर्वमेतां दिशं गच्छ या पूर्वं परिकीर्तिता}
{देवतानां हि सान्निध्यमत्र कीर्तितवानसि}


\fourlineindentedshloka
{अत्र सत्यं च धर्मश्च त्वया सम्यक्प्रकीर्तितः}
{इच्छेयं तु समागन्तुं समस्तैर्दैवतैरहम्}
{भूयश्च तान्सुरान्द्रष्टुमिच्छेयमरुणानुज ॥नारद उवाच}
{}


\threelineshloka
{तमाह विनतासूनुरारोहस्वेति वै द्विजम्}
{आरुरोहाथ स मुनिर्गरुडं गालवस्तदा ॥गालव उवाच}
{}


\twolineshloka
{क्रममाणस्य ते रूपं दृश्यते पन्नगाशन}
{भास्करस्येव पूर्वाह्णे सहस्रांशोर्विवस्वतः}


\twolineshloka
{पक्षवातप्रणुन्नानां वृक्षाणामनुगामिनाम्}
{प्रस्थितानामिव समं पश्यामीह गतिं खग}


\twolineshloka
{ससागरवनामुर्वी सशैलवनकाननाम्}
{आकर्षन्निव चाभासि पक्षवातेन खेचर}


\twolineshloka
{समीननागनक्रं च खमिवारोप्यते जलम्}
{वायुना चैव महता पक्षवातेन चानिशम्}


\twolineshloka
{तुल्यरूपाननान्मत्स्यांस्तथा तिमितिमिङ्गिलान्}
{नागाश्वनरवक्रांश्च पश्याम्युन्मथितानिव}


\twolineshloka
{महार्णवस्य च रवैः श्रोत्रे मे बधिरे कृते}
{न श्रृणोमि न पश्यामि नात्मनो वेद्मि कारणम्}


\twolineshloka
{शनैः स तु भवात्यातु ब्रह्मवध्यामनुस्मरन्}
{न दृश्यते रविस्तात न दिशो न च खं खग}


\twolineshloka
{तम एव तु पश्यामि शरीरं ते न लक्षये}
{मणी व जात्यौ पश्यामि चक्षुषी तेऽहमण्डज}


\twolineshloka
{शरीरं तु न पश्यामि तव चैवात्मनश्च ह}
{पदेपदे तु पश्यामि शरीरादग्निमुत्थितम्}


\twolineshloka
{स मे निर्वाप्य सहसा चक्षुषी शाम्यते पुनः}
{तन्नियच्छ महावेगं गमने विनतात्मज}


\twolineshloka
{न मे प्रयोजनं किंचिद्गमने पन्नगाशन}
{संनिवर्त महाभाग न वेगं विषहामि ते}


\twolineshloka
{गुरवे संश्रुतानीह शतान्यष्टौ हि वाजिनाम्}
{एकतःश्यामकर्णानां शुभ्राणां चन्द्रवर्चसाम्}


\twolineshloka
{तेषां चैवापवर्गाय मार्गं पश्यामि नाण्डज}
{ततोऽयं जीवितत्यागे दृष्टो मार्गोमयाऽऽत्मनः}


\threelineshloka
{नैव मेऽस्ति धनं किंचिन्न धनेनान्वितः सुहृत्}
{न चार्थेनापि महता शक्यमेतद्व्यपोहितुम् ॥नारद उवाच}
{}


\twolineshloka
{एवं बहु च दीनं च ब्रुवाणं गालवं तदा}
{प्रत्युवाच व्रजन्नेव प्रहसन्विनतात्मजः}


\twolineshloka
{नातिप्रज्ञोऽसि विप्रर्षे योत्मानं त्युक्तुमिच्छसि}
{न चापि कृत्रिमः कालः कालो हि परमेश्वरः}


\twolineshloka
{किमहं पूर्वमेवेह भवता नाभिचोदितः}
{उपायोऽत्र महानस्ति येनैतदुपपद्यते}


\twolineshloka
{तदेष ऋषभो नाम पर्वतः सागरान्तिके}
{अत्र विश्रम्य भुक्त्वा च निवर्तिष्याव गालव}


\chapter{अध्यायः ११३}
\twolineshloka
{नारद उवाच}
{}


\twolineshloka
{ऋषभस्य ततः शृङ्गं निपत्य द्विजपक्षिणौ}
{शाण्डिलीं ब्राह्मणीं तत्र ददृशो तपोन्विताम्}


\twolineshloka
{अभिवाद्य सुपर्णस्तु गालवश्चाभिपूज्य ताम्}
{तया च स्वागतेनोक्तौ विष्टरे सन्निषीदतुः}


\twolineshloka
{सिद्धमन्नं तया दत्तं बलिमन्त्रोपबृंहितम्}
{भुक्त्वा तृप्तावुभौ भूमौ सुप्तौ तावनुमोहितौ}


\twolineshloka
{मुहूर्तात्प्रतिबुद्धस्तु सुपर्णो गमनेप्सया}
{` तां दृष्ट्वा चारुसर्वाङ्गी तापसीं ब्रह्मचारिणीम्}


\twolineshloka
{ग्रहीतुं हि मनश्चक्रे रूपात्साक्षादिव श्रियम्}
{'अथ भ्रष्टतनूजाङ्गमात्मानं ददृशे खगः}


\twolineshloka
{मांसपिण्डोपमोऽभूत्स मुखपादान्वितः खगः}
{गालवस्तं तथा दृष्ट्वा विमनाः पर्यपृच्छत}


\twolineshloka
{किमिदं भवता प्राप्तमिहागमनजं फलम्}
{वासोऽयमिह कालं तु कियन्तं नौ भविष्यति}


\twolineshloka
{किं नु ते मनसा ध्यातमशुभं धर्मदूषणम्}
{न ह्ययं भवतः स्वल्पो व्यभिचारो भविष्यति}


\twolineshloka
{सुपर्णोऽथाब्रवीद्विप्रं प्रध्यातं वै मया द्विज}
{इमां सिद्धामितो नेतुं तत्र यत्र प्रजापतिः}


\twolineshloka
{यत्र देवो महादेवो यत्र विष्णुः सनातनः}
{यत्र धर्मश्च यज्ञश्च तत्रेयं निवसेदिति}


\twolineshloka
{सोऽहं भगवतीं याचे प्रणतः प्रियकाम्यया}
{मयैतन्नाम प्रध्यातं मनसा शोचसा किल}


\twolineshloka
{तदेवं बहुमानात्ते मयेहानीप्सितं कृतम्}
{सुकृतं दुष्कृतं वा त्वं माहात्म्यात्क्षन्तुमर्हसि}


\twolineshloka
{सा तौ तदाऽब्रवीत्तुष्टा पतगेन्द्रद्विजर्षभौ}
{न भेतव्यं सुपर्णोऽसि सुपर्ण त्यज संभ्रमम्}


\twolineshloka
{निन्दितास्मि त्वया वत्स न च निन्दां क्षमाम्यहम्}
{लोकेभ्यः सपदि भ्रश्येद्यो मां निन्देत पापकृत्}


\twolineshloka
{हीनयाऽलक्षणैः सर्वैस्तथाऽनिन्दितया मया}
{आचारं प्रतिगृह्णन्त्या सिद्धिः प्राप्तेयमुत्तमा}


\twolineshloka
{आचारः फलते धर्ममाचारः फलते धनम्}
{आचाराच्छ्रियमाप्नोति आचारो हन्त्यलक्षणम्}


\twolineshloka
{तदायुष्मन्खगपते यथेष्टं गम्यतामितः}
{न च ते गर्हणीयाऽहं गर्हितव्याः स्त्रियः क्वचित्}


\twolineshloka
{` यदि त्वमात्मनो ह्यर्थे मां चैवादातुमिच्छसि}
{तदेव नष्टदेहस्तु स्या वै त्वं पन्नगाशन}


\twolineshloka
{तस्यैव हि प्रसादेन देवदेवस्य चिन्तनात्}
{त्वं तु साङ्गस्तु सञ्जातः पुनरेव भविष्यसि ॥'}


\threelineshloka
{भवितासि यथापूर्वं बलवीर्यसमन्वितः}
{नारद उवाच}
{बभूवतुस्ततस्तस्य पक्षौ द्रविणवत्तरौ}


\twolineshloka
{अनुज्ञातस्तु शाण्डिल्या यथागतमुपागमत्}
{नैव चासादयामास तथारूपांस्तुरङ्गमान्}


\twolineshloka
{विश्वामित्रोऽथ तं दृष्ट्वा गालवं चाध्वनि स्थितः}
{उवाच वदतां श्रेष्ठो वैनतेयस्य सन्निधौ}


\twolineshloka
{यस्त्वया स्वयमेवार्थः प्रतिज्ञातो मम द्विज}
{तस्य कालोऽपवर्गस्य यथा वा मन्यते भवान्}


\twolineshloka
{प्रतीक्षिष्याम्यहं कालमेतावन्तं तथा परम्}
{यथासंसिध्यते विप्र स मार्गस्तु निशाम्यताम्}


\twolineshloka
{सुपर्णोऽथाब्रवीद्दीनं गालवं भृशदुःखितम्}
{प्रत्यक्षं खल्विदानीं मे विश्वामित्रो यदुक्तवान्}


\twolineshloka
{तदागच्छ द्विजश्रेष्ठ मन्त्रयिष्याव गालव}
{नादत्त्वा गुरवे शक्यं कृत्स्नमर्थं त्वयाऽऽसितुम्}


\chapter{अध्यायः ११४}
\twolineshloka
{नारद उवाच}
{}


\threelineshloka
{अथाह गालवं दीनं सुपर्णः पततां वरः}
{निर्मितं वह्निना भूमौ वायुना शोधितं तथा}
{यस्माद्धिरण्मयं सर्वं हिरण्यं तेन चोच्यते}


\twolineshloka
{धत्ते धारयते चेदमतस्मात्कारणाद्धनम्}
{तदेतत्रिषु लोकेषु धनं तिष्ठति शाश्वतम्}


\twolineshloka
{नित्यं प्रोष्ठपदाभ्यां च शुक्रे धनपतौ तथा}
{मनुष्येभ्यः समादत्ते शुक्रश्चित्तार्जितं धनम्}


\threelineshloka
{अजैकपादहिर्बुध्न्यौ रक्ष्येते धनदेन च}
{एवं न शक्यते लब्धुमलब्धव्यं द्विजर्षभ}
{ऋते च धनमश्वानां नावाप्तिर्विद्यते तव}


\twolineshloka
{स त्वं याचात्र राजानं कंचिद्राजर्षिवंशजम्}
{अपीड्य राजा पौरान्हि यो नौ कुर्यात्कृतार्थिनौ}


\twolineshloka
{अस्ति सोमान्ववाये मे जातः कश्चिन्नृपः सखा}
{अभिगच्छावहे तं वै तस्यास्ति विभवो भुवि}


\twolineshloka
{ययातिर्नाम राजर्षिर्नोहुषः सत्यविक्रमः}
{स दास्यति मया चोक्तो भवता चार्थितः स्वयम्}


\twolineshloka
{विभवश्चास्य सुमहानासीद्धनपतेरिव}
{एवं गुरुधनं विद्वन्दानेनैव विशोधय}


\twolineshloka
{तथा तौ कथयन्तौ च चिन्तयन्तौ च यत्क्षमम्}
{प्रतिष्ठाने नरपतिं ययातिं प्रत्युपस्थितौ}


\twolineshloka
{प्रतिगृह्य च सत्कारैरर्घ्यपाद्यादिकं वरम्}
{पृष्टश्चागमने हेतुमुवाच विनतासुतः}


\twolineshloka
{अयं मे नाहुष सखा गालवस्तपसो निधिः}
{विश्वामित्रस्य शिष्योऽभूद्वर्षाण्ययुतशो नृप}


\twolineshloka
{सोऽयं तेनाभ्यनुज्ञात उपकारेप्सया द्विजः}
{तमाह भगवान्किं ते ददानि गुरुदक्षिणाम्}


\twolineshloka
{असकृत्तेन चोक्तेन किंचिदागतमन्युना}
{अयमुक्तः प्रयच्छेति जानता विभवं लघु}


\twolineshloka
{एकतःश्यामकर्णानां शुभ्राणां शुद्धजन्मनाम्}
{अष्टौ शतानि मे देहि हयानां चन्द्रवर्चसाम्}


\twolineshloka
{गुर्वर्थो दीयतामेव यदि गालव मन्यसे}
{इत्येवमाह सक्रोधो विश्वामित्रस्तपोधनः}


\twolineshloka
{सोऽयं शोकेन महता तप्यमानो द्विजर्षभः}
{अशक्तः प्रतिकर्तुं तद्भवन्तं शरणं गतः}


\twolineshloka
{प्रतिगृह्य नरव्याघ्र त्वत्तो भिक्षां गतव्यथः}
{कृत्वाऽऽपवर्गं गुरवे चरिष्यति महत्तपः}


\twolineshloka
{तपसः संविभागेन भवन्तमपि योक्ष्यते}
{स्वेन राजर्षितपसा पूर्णं त्वां पूरयिष्यति}


\twolineshloka
{यावन्ति रोमाणि हये भवन्तीह नरेश्वर}
{तावन्तो वाजिनो लोकान्प्राप्नुवन्ति महीपते}


\twolineshloka
{पात्रं प्रतिग्रहस्यायं दातुं पात्रं तथा भवान्}
{शङ्खे क्षीरमिवासक्तं भवत्वेतत्तथोपमम्}


\chapter{अध्यायः ११५}
\twolineshloka
{नारद उवाच}
{}


\twolineshloka
{एवमुक्तः सुपर्णेन तथ्यं वचनमुत्तमम्}
{विमृश्यावहितो राजा निश्चित्य च पुनः पुनः}


\twolineshloka
{यष्टा क्रतुसहस्राणां दाता दानपतिः प्रभुः}
{ययातिः सर्वकाशीश इदं वचनमब्रवीत्}


\twolineshloka
{दृष्ट्वा प्रियसखं तार्क्ष्यं गालवं च द्विजर्षभम्}
{निदर्शनं च तपसो भिक्षां श्लाघ्यां च कीर्तिताम्}


\twolineshloka
{अतीत्य च नृपानन्यानादित्यकुलसंभवान्}
{मत्सकाशमनुप्राप्तावेतां बुद्धिमवेक्ष्य च}


\twolineshloka
{अद्य मे सफलं जन्म तारितं चाद्य मे कलम्}
{अद्यायं तारितो देशो मम तार्क्ष्य त्वयाऽनघ}


\twolineshloka
{वक्तुमिच्छामि तु सखे यथा जानासि मां पुरा}
{न तथा वित्तवानस्मि क्षीणं वित्तं च मे सखे}


\twolineshloka
{न च शक्तोऽस्मि ते कर्तुं मोघमागमनं खग}
{न चाशामस्य विपर्षेर्वितथीकर्तुमुत्सहे}


\twolineshloka
{पुत्रीं दास्यामि यत्कार्यमियं संपादयिष्यति}
{अभिगम्य हताशो हि निवृत्तो दहते कुलम्}


\twolineshloka
{नातः परं वैनतेय किंचित्पापिष्ठमुच्यते}
{प्रथाशानाशनं लोके देहि नास्तीति वा वचः}


\twolineshloka
{हताशो ह्यकृतार्थः सन्हतः संभावितो नरः}
{हिनस्ति तस्य पुत्रांश्च पौत्रांश्चाकुर्वतो हितम्}


\twolineshloka
{तस्माच्चतुर्णां वंशानां स्थापयित्री सुता मम}
{`माधवी नाम तार्क्ष्येयं सर्वधर्मप्रदायिनी ॥'}


\threelineshloka
{इयं सुरसुतप्रख्या सर्वधर्मोपचायिनी}
{सदा देवमनुष्यणामसुराणां च गालव}
{काङ्क्षिता रूपतो बाला सुता मे प्रतिगृह्यताम्}


\twolineshloka
{अस्याः शुल्कं प्रदास्यन्ति नृपा राज्यमपि ध्रवम्}
{किं पुनः श्यामकर्णानां हयानां द्वे चतुःशते}


\fourlineindentedshloka
{स भवाप्रतिगृह्णातु ममैतां माधवीं सुताम्}
{अहं दौहित्रवान्त्स्यां वै वर एष मम प्रभो ॥ 5-115-15a` स तस्यवचनं श्रुत्वा ब्राह्मणः शंसितव्रतः}
{'प्रतिगृह्य च तां कन्यां गालवः सह पक्षिणा}
{पुनर्द्रक्ष्याव इत्युक्त्वा प्रतस्थे सह कन्यया}


\twolineshloka
{उपलब्धमिदं द्वारमश्वानामिति चाण्डजः}
{उक्त्वा गालवमापृच्छ्य जगाम भवनं स्वकम्}


\twolineshloka
{गते पतगराजे तु गालवः सह कन्यया}
{चिन्तयानः क्षमंदाने राजानं शुल्कतोऽगमत्}


\twolineshloka
{सोऽगच्छन्मनसेक्ष्वाकुं हर्यश्वं राजसत्तमम्}
{अयोध्यायां महावीर्यं चतुरङ्गबलान्वितम्}


\twolineshloka
{कोशधान्यबलोपेतं प्रियपौरं द्विजप्रियम्}
{प्रजाभिकामं शाम्यन्तं कुर्वाणं तप उत्तमम्}


\twolineshloka
{तमुपागम्य विप्रः स हर्यश्वं गालवोऽब्रवीत्}
{कन्येयं मम राजेन्द्र प्रसवैः कुलवर्धिनी}


\twolineshloka
{इयं शुल्केन भार्यार्थं हर्यश्व प्रतिगृह्यताम्}
{शुल्कं ते कीर्तियिष्यामि तच्छ्रुत्वा संप्रधार्यताम्}


\chapter{अध्यायः ११६}
\twolineshloka
{नारद उवाच}
{}


\twolineshloka
{हर्यश्वस्त्वब्रवीद्राजा विचिन्त्य बहुधा ततः}
{दीर्घमुष्णं च निश्वस्य प्रजाहेतोर्नृपोत्तमः}


\twolineshloka
{उन्नतेषून्नता षट्सु सूक्ष्मा सूक्ष्मेषु पञ्चसु}
{गम्भीरा त्रिषु चाङ्गेषु इयं रक्ता च पञ्चसु}


\threelineshloka
{`श्रोण्यौ ललाटमूरू च घ्राणं चेति षडुन्नतम्}
{सूक्ष्माण्यङ्गुलिपर्वाणि केशरोमनस्वत्वचः}
{}


\twolineshloka
{स्वरः सत्वं च नाभिश्च त्रिगम्भीरं प्रचक्षते}
{पाणिपादतले रक्ते नेत्रान्तौ च नखानि च}


\twolineshloka
{पञ्चदीर्घं चतुर्ह्रस्वं पञ्चसूक्ष्मं षडुन्नतम्}
{सप्तरक्तं त्रिविस्तीर्णं त्रिगम्भीरं प्रशकस्यते}


\twolineshloka
{पञ्चैव दीर्घा हनुलोचनानिबाहूरुनासाश्च सुखप्रदानि}
{ह्रस्वानि चत्वारि च लिङ्गपृष्ठेग्रीवा च जङ्घे च हितप्रदानि}


\twolineshloka
{सूक्ष्माणि चत्वारि च लिङ्गपृष्ठेग्रीवा च जङ्घे च हितप्रदानि ॥सूक्ष्माणि पञ्च दशनाङ्गुलिपर्वकेशास्त्वक्वैव वै कररुहाश्च न दुःखितानाम्}
{वक्षोऽथ कक्षो नखनासिकास्यु-रंसत्रिकं चेति षडुन्नतानि}


% Check verse!
नेत्रान्तपादकरताल्वधरोष्ठजिह्वारक्ता नखाश्च खलु सर्वसुखावहानि ॥'
\twolineshloka
{बहुदेवासुरालोका बहुगन्धर्वदर्शना}
{बहुलक्षणसंपन्ना बहुकल्याणधारिणी}


\threelineshloka
{समर्थेयं जनयितुं चक्रवर्तिनमात्मजम्}
{ब्रूहि शुल्कं द्विजश्रेष्ठ समीक्ष्य विभवं मम ॥गालव उवाच}
{}


\twolineshloka
{एकतःश्यामकर्णानां शतान्यष्टौ प्रयच्छ मे}
{हयानां चन्द्रशुभ्राणां देशजानां वपुष्मताम्}


\threelineshloka
{ततस्तव भवित्रीयं पुत्राणां जननी शुभा}
{अरणीव हुताशानां योनिरायतलोचना ॥नारद उवाच}
{}


\twolineshloka
{एतच्छ्रुत्वा वचो राजा हर्यश्वः काममोहितः}
{उवाच गालवं दीनो राजर्षिर्ऋषिसत्तमम्}


\twolineshloka
{द्वे मे शते संनिहिते हयानां यद्विधास्तव}
{एष्टव्याः शतशस्त्वन्ये चरन्ति मम वाजिनः}


\twolineshloka
{सोऽहमेकमपत्यं वै जनयिष्यामि गालव}
{अस्यामेतं भवान्कामं संपादयतु मे वरम्}


\twolineshloka
{एतच्छ्रुत्वा तु सा कन्या गालवं वाक्यमब्रवीत्}
{मम दत्तो वरः कश्चित्केनचिद्ब्रह्मवादिना}


\twolineshloka
{प्रसूत्यन्ते प्रसूत्यन्ते कन्यैव त्वं भविष्यसि}
{स त्वं ददस्व मां राज्ञे प्रतिगृह्य हयोत्तमान्}


\twolineshloka
{नृपेभ्यो हि चतुर्भ्यस्ते पूर्णान्यष्टौ शतानि वै}
{भविष्यन्ति तथा पुत्रा मम चत्वार एव च}


\twolineshloka
{क्रियतामुपसंहारो गुर्वर्थं द्विजसत्तम}
{एषा तावन्मम प्रज्ञा यथा वा मन्यसे द्विज}


\twolineshloka
{एवमुक्तस्तु स मुनिः कन्यया गालवस्तदा}
{हर्यश्वं पृथिवीपालमिदं वचनमब्रवीत्}


\threelineshloka
{इयं कन्या नरश्रेष्ठ हर्यश्व प्रतिगृह्यताम्}
{चतुर्भागेन शुल्कस्य जनयस्वैकमात्मजम् ॥नारद उवाच}
{}


\twolineshloka
{प्रतिगृह्य स तां कन्यां गालवं प्रतिनन्द्य च}
{समये देशकाले च लब्धवान्सुतमीप्सितम्}


\twolineshloka
{ततो वसुमना नाम वसुभ्यो वसुमत्तरः}
{वसुप्रख्यो नरपतिः स बभूव वसुप्रदः}


\twolineshloka
{अथ काले पुनर्धीमान्गालवः प्रत्युपस्थितः}
{उपसङ्गम्य चोवाच हर्यश्वं प्रीतमानसम्}


\threelineshloka
{जातो नृप सुतस्तेऽयं बालो भास्करसंनिभः}
{कालो गन्तुं नरश्रेष्ठ शुल्कार्थमपरं नृपम् ॥नारद उवाच}
{}


\twolineshloka
{हर्यश्वः सत्यवचने स्थितः स्थित्वा च पौरुषे}
{दुर्लभत्वाद्धयानां च प्रददौ माधवीं पुनः}


\twolineshloka
{माधवी च पुनर्दीप्तां परित्यज्यनृपश्रियम्}
{कुमारी कामतो भूत्वा गालवं पृष्ठतोऽन्वगात्}


\twolineshloka
{त्वय्येव तावत्तिष्ठन्तु हया इत्युक्तवान्द्विजः}
{प्रययौ कन्यया सार्धं दिवोदासं प्रजेश्वरम्}


\chapter{अध्यायः ११७}
\twolineshloka
{गालव उवाच}
{}


\twolineshloka
{महावीर्यो महीपालः काशीनामीश्वरः प्रभुः}
{दिवोदास इति ख्यातो भैमसेनिर्नराधिपः}


\threelineshloka
{तत्र गच्छावहे भद्रे शनैरागच्छ मा शुचः}
{धार्मिकः संयमे युक्तः सत्ये चैव जनेश्वरः ॥नारद उवाच}
{}


\threelineshloka
{तमुपागम्य स मुनिर्न्यायतस्तेन सत्कृतः}
{गालवः प्रसवस्यार्थे तं नृपं प्रत्यचोदयत् ॥दिवोदाम उवाच}
{}


\twolineshloka
{श्रुतमेतन्मया पूर्वं किमुक्त्वा विस्तरं द्विज}
{काङ्क्षितो हि मयैषोऽर्थः श्रुत्वैव द्विजसत्तम}


\twolineshloka
{एतच्च मे बहुमतं यदुत्सृज्य नराधिपान्}
{मामेवमुपयातोऽसि भावि चैतदसंशयम्}


\threelineshloka
{स एव विभवोऽसमाकमश्वानामपि गालव}
{अहमप्येकमेवास्यां जनयिष्यामि पार्थिवम् ॥नारद उवाच}
{}


\twolineshloka
{तथेत्युक्त्वा द्विजश्रेष्ठः प्रादात्कन्यां महीपतेः}
{विधिपूर्वां च तां राजा कन्यां प्रतिगृहीतवान्}


\twolineshloka
{रेमे स तस्यां राजर्षिः प्रभावत्यां यथा रविः}
{स्वाहायां च यथा वह्निर्यथा शाच्यां च वासवः}


\twolineshloka
{यथा चन्द्रश्च रोहिण्यां यथा धूमोर्णया यमः}
{वरुणश्च यथा गौर्यां यथा चर्ध्यां धनेश्वरः}


\twolineshloka
{यथा नारायणो लक्ष्म्यां जाह्नव्यां च यथोदधिः}
{यथा रुद्रश्च रुद्राण्यां यथा वेद्यां पितामहः}


\twolineshloka
{अदृश्यन्त्यां च वासिष्ठो वसिष्ठश्चाक्षमालया}
{च्यवनश्च सुकन्यायां पुलस्त्यः सन्ध्यया यथा}


\twolineshloka
{अगस्त्यश्चापि वैदर्भ्यां सावित्र्यां सत्यवान्यथा}
{यथा भृगुः पुलोमायामदित्यां कश्यपो यथा}


\twolineshloka
{रेणुकायां यथार्चीको हैमवत्यां च कौशिकः}
{बृहस्पतिश्च तारायां शुक्रश्च शतपर्वणा}


\twolineshloka
{यथा भूम्यां भूमिपतिरुर्वश्यां च पुरूरवाः}
{ऋचीकः सत्यवत्यां च सरस्वस्यां यथा मनुः}


\twolineshloka
{शकुन्तलायां दुष्यन्तो धृत्यां धर्मश्च शाश्वतः}
{दमयन्त्यां नलश्चैव सत्यवत्यां च नारदः}


\twolineshloka
{जरत्कारुर्जरत्कार्वां पुलस्त्यश्च प्रतीच्यया}
{मेनकायां यथोर्णायुस्तुम्बुरुश्चैव रम्भया}


\twolineshloka
{वासुकिः शतशीर्षायां कुमार्यां च धनञ्जयः}
{वैदेह्यां च यथा रामो रुक्मिण्यां च जनार्दनः}


\twolineshloka
{तथा तु रममाणस्य दिवोदासस्य भूपतेः}
{माधवी जनयामास पुत्रमेकं प्रतर्दनम्}


\twolineshloka
{अथाजगाम भगवान्दिवोदासं स गालवः}
{समये समनुप्राप्ते वचनं चेदमब्रवीत्}


\threelineshloka
{निर्यातयतु मे कन्यां भवांस्तिष्ठन्तु वाजिनः}
{यावदन्यत्र गच्छामि शुल्कार्थं पृथिवीपते ॥नारद उवाच}
{}


\twolineshloka
{दिवोदासोऽथ धर्मात्मा समये गालवस्य ताम्}
{कान्यां निर्यातयामास स्थितः सत्ये महीपतिः}


\chapter{अध्यायः ११८}
\twolineshloka
{नारद उवाच}
{}


\twolineshloka
{तथैव तां श्रियं त्यक्त्वा कन्या भूत्वा यशस्विनी}
{माधवी गालवं विप्रमभ्ययात्सत्यसङ्गरा}


\twolineshloka
{गालवो विमृशन्नेव स्वकार्यगतमानसः}
{जगाम भोजनगरं द्रष्टुमौशीनरं नृपम्}


\twolineshloka
{तमुवाचाथ गत्वा स नृपतिं सत्यविक्रमम्}
{इय कन्या सुतौ द्वौ ते जनयिष्यति पार्थिवौ}


\twolineshloka
{अस्यां भवानवाप्तार्थो भविता प्रेत्य चेह च}
{सोमार्कप्रतिसङ्काशौ जनयित्वा सुतौ नृप}


\twolineshloka
{शुल्कं तु सर्वधर्मज्ञ हयानां चन्द्रवर्चसाम्}
{एकतःश्यामकर्णानां देयं मह्यं चतुःशतम्}


\twolineshloka
{गुर्वर्थोऽयं समारम्भो न हयैः कृत्यमस्ति मे}
{यदि शक्यं महाराज क्रियतामविचारितम्}


\twolineshloka
{अनपत्योऽसि राजर्षे पुत्रौ जनय पार्थिव}
{पितॄन्पुत्रप्लवेन त्वमात्मानं चैव तारय}


\twolineshloka
{न पुत्रफलभोक्ता हि राजर्षे पात्यते दिवः}
{न याति नरकं घोरं यत्र गच्छन्त्यनात्मजाः}


\twolineshloka
{एतच्चान्यच्च विविधं श्रुत्वा गालवभाषितम्}
{उशीनरः प्रतिवचो ददौ तस्य नराधिपः}


\twolineshloka
{श्रुतवानस्मि ते वाक्यं यथा वदसि गालव}
{विधिस्तु बलवान्ब्रह्मन्प्रवणं हि मनो मम}


\twolineshloka
{शते द्वे तु ममाश्वानामीदृशानां द्विजोत्तम}
{इतरेषां सहस्राणि सुबहूनि चरन्ति मे}


\twolineshloka
{अहमप्येकमेवास्यां जनयिष्यामि गालव}
{पुत्रवद्भिर्गतं मार्गं गमिष्यामि परैरहम्}


\twolineshloka
{मूल्येनापि समं कुर्यां तवाहं द्विजसत्तम}
{पौरजानपदार्थं तु ममार्थो नात्मभोगतः}


\twolineshloka
{कामतो हि धनं राजा पारक्यं यः प्रयच्छति}
{न स धर्मेण धर्मात्मन्युज्यते यशसा न च}


\twolineshloka
{सोऽहं प्रतिग्रहीष्यामि ददात्वेतां भवान्मम}
{कुमारीं देवगर्भाभामेकपुत्रभवाय मे}


\twolineshloka
{तथा तु बहुधा कन्यामुक्तवन्तं नराधिपम्}
{उशीनरं द्विजश्रेष्ठो गालवः प्रत्यपूजयत्}


\twolineshloka
{उशीनरं प्रतिग्राह्य गालवः प्रययौ वनम्}
{रेमे स तां समासाद्य कृतपुण्य इव श्रियम्}


\twolineshloka
{कन्दरेषु च शैलानां नदीनां निर्झरेषु च}
{उद्यानेषु विचित्रेषु वनेषूपवनेषु च}


\twolineshloka
{हर्म्येषु रमणीयेषु प्रासादशिखरेषु च}
{वातायनविमानेषु तथा गर्भगृहेषु च}


\twolineshloka
{ततोऽस्य समये यज्ञे पुत्रो बालरविप्रभः}
{शिबिर्नाम्नाभिविख्यातो यः स पार्थिवसत्तमः}


\twolineshloka
{उपस्थाय स तं विप्रो गालवः प्रतिगृह्य च}
{कन्यां प्रयातस्तां राजन्दृष्टवान्विनतात्मजम्}


\chapter{अध्यायः ११९}
\twolineshloka
{नारद उवाच}
{}


\twolineshloka
{गालवं वैनतेयोऽथ प्रहसन्निदमब्रवीत्}
{दिष्ट्या कृतार्थं पश्यामि भवन्तमिह वै द्विज}


\twolineshloka
{गालवस्तु वचः श्रुत्वा वैनतेयेन भाषितम्}
{चतुर्भागावशिष्टं तदाचख्यौ कार्यमस्य हि}


\twolineshloka
{सुपर्णस्त्वब्रवीदेनं गालवं वदतां वरः}
{प्रयत्नस्ते न कर्तव्यो नैष संपत्स्यते तव}


\twolineshloka
{पुरा हि कान्यकुब्जे वै गाधेः सत्यवतीं सुताम्}
{भार्यार्थे वरयत्कन्यामृचीकस्तेन भाषितः}


\twolineshloka
{एकतःश्यामकर्णानां हयानां चन्द्रवर्चसाम्}
{भगवन्दीयताकं मह्यं महस्रमिति गालव}


\twolineshloka
{ऋचीकस्तु तथेत्युक्त्वा वरुणस्यालयं गतः}
{अश्वतीर्थे हयाँल्लब्ध्वा दत्तवान्पार्थिवाय वै}


\twolineshloka
{इष्ट्वा ते पुण्डरीकेण दत्ता राज्ञा द्विजातिषु}
{तेभ्यो द्वे द्वे शते क्रीत्वा प्राप्ते तैः पार्थिवैस्तदा}


\twolineshloka
{अपराण्यपि चत्वारि शतानि द्विजसत्तम}
{नीयमानानि संतारे हृतान्यासन्नितस्ततः}


\twolineshloka
{एवं न शक्यमप्राप्यं प्राप्तुं गालव कर्हिचित्}
{इमामश्वशताभ्यां वै द्वाभ्यां तस्मै निवेदय}


\twolineshloka
{विश्वामित्राय धर्मात्मन्षङ्भिरश्वशतैः सह}
{ततोऽसि गतसंमोहः कृतकृत्यो द्विजोत्तम}


\twolineshloka
{गालवस्तं तथेत्युक्त्वा सुपर्णसहितस्ततः}
{आदायाश्वांश्च कन्यां च विश्वामित्रपुपागमत्}


\twolineshloka
{अश्वानां काङ्क्षितार्थानां षडिमानि शतानि वै}
{शतद्वयेन कन्येयं भवता प्रतिगृह्यताम्}


\twolineshloka
{अस्यां राजर्षिभिः पुत्रा जाता वै धार्मिकास्त्रयः}
{चतुर्थं जनयत्वेकं भवानपि नरोत्तमम्}


\twolineshloka
{पूर्णान्येवं शतान्यष्टौ तुरगाणां भवन्तु ते}
{भवतो ह्यनृणो भूत्वा तपः कुर्यां यथासुखम्}


\twolineshloka
{विश्वामित्रस्तु तं दृष्ट्वा गालवं सह पक्षिणा}
{कन्यां च तां वरारोहामिदमित्यब्रवीद्वचः}


\twolineshloka
{किमियं पूर्वमेवेह न दत्ता मम गालव}
{पुत्रा ममैव चत्वारो भवेयुः कुलभावनाः}


\threelineshloka
{प्रतिगृह्णामि ते कन्यामेकपुत्रफलाय वै}
{अश्वाश्चाश्रममासाद्य चरन्तु मम सर्वशः ॥नारद उवाच}
{}


\twolineshloka
{स तया रममाणोऽथ विश्वामित्रो महाद्युतिः}
{आत्मजं जनयामास माधवीपुत्रमष्टकम्}


\twolineshloka
{जातमात्रं सुतं तं च विश्वामित्रो महामुनिः}
{संयोज्यार्थैस्तथा धर्मैरश्वैस्तैः समयोजयत्}


\twolineshloka
{अथाष्टकः पुरं प्रायात्तदा सोमपुरप्रभम्}
{निर्यात्य कन्यां शिष्याय कौशिकोपि वनं ययौ}


\twolineshloka
{गालवोपि सुपर्णेन सह निर्यात्य दक्षिणाम्}
{मनसाऽतिप्रतीतेन कन्यामिदमुवाच ह}


\twolineshloka
{जातो दानपतिः पुत्रस्त्वया शूरस्तथाऽपरः}
{सत्यधर्मरतश्चान्यो यज्वा चापि तथाऽपरः}


\threelineshloka
{तदागच्छ वरारोहे तारितस्ते पिता सुतैः}
{चत्वारश्चैव राजानस्तथा चाहं सुमध्यमे ॥नारद उवाच}
{}


\twolineshloka
{गालवस्त्वभ्यनुज्ञाय सुपर्णं पन्नगाशनम्}
{पितुर्निर्यात्य तां कन्यां प्रययौ वनमेव ह}


\chapter{अध्यायः १२०}
\twolineshloka
{नारद उवाच}
{}


\twolineshloka
{स तु राजा पुनस्तस्याः कर्तुकामः स्वयंवरम्}
{उपगम्याश्रमपदं गङ्गायामुनसङ्गमे}


\twolineshloka
{गृहीतमाल्यदामां तां रथमारोप्य माधवीम्}
{पूरुर्यदुश्च भगिनीमाश्रमे पर्यधावताम्}


\twolineshloka
{नागयक्षमनुष्याणां गन्धर्वमृगपक्षिणाम्}
{शैलद्रुमवनौकानामासीत्तत्र समागमः}


\twolineshloka
{नानापुरुषदेश्यानामीश्वरैश्च समाकुलम्}
{ऋषिभिर्ब्रह्मकल्पैश्च समन्तादावृतं वनम्}


\twolineshloka
{निर्दिश्यमानेषु तु सा वरेषु वरवर्णिनी}
{वरानुत्क्रम्य सर्वास्तान्वरं वृतवती वनम्}


\twolineshloka
{अवतीर्य रथात्कन्या नमस्कृत्य च बन्धुषु}
{उपगम्य वनं पुण्यं तपस्तेपे ययातिजा}


\twolineshloka
{उपवसैश्च विविधैर्दीक्षाभिर्नियमैस्तथा}
{आत्मनो लघुतां कृत्वा बभूव मृगचारिणी}


\twolineshloka
{वैदूर्याङ्कुरकल्पानि मृदूनि हरितानि च}
{चरन्ती श्लक्ष्णशष्पाणि तिक्तानि मधुराणि च}


\twolineshloka
{स्रवन्तीनां च पुण्यानां सुरसानि सुचीनि च}
{पिबन्ती वारिमुख्यानि शीतानि विमलानि च}


\twolineshloka
{वनेषु मृगराजेषु व्याघ्रविप्रोषितेषु च}
{दावाग्निविप्रयुक्तेषु शून्येषु गहनेषु च}


\twolineshloka
{चरन्ती हरिणैः सार्धं मृगीव वनचारिणी}
{चचार विपुलं धर्मं ब्रह्मचर्येण संवृतम्}


\twolineshloka
{ययातिरपि पूर्वेषां राज्ञां वृत्तमनुष्ठितः}
{बहुवर्षसहस्रायुर्युयुजे कालधर्मणा}


\twolineshloka
{पुरुर्यदुश्च द्वौ वंशे वर्धमानौ नरोत्तमौ}
{ताभ्भां प्रतिष्ठितो लोके परलोके च नाहुषः}


\twolineshloka
{महीपते नरपतिर्ययातिः स्वर्गमास्थितः}
{महर्षिकल्पो नृपतिः स्वर्गाग्र्यफलभुग्विभुः}


\twolineshloka
{बहुवर्षसहस्राख्ये काले बहुगुणे गते}
{राजर्षिषु निषण्णेषु महीयःसु महर्धिषु}


\twolineshloka
{अवमेने नरान्सर्वान्देवानृषिगणांस्तथा}
{ययातिर्मूढविज्ञानो विस्मयाविष्टचेतनः}


\twolineshloka
{ततस्तं बुबुधे देवः शक्रो बलनिषूदनः}
{ते च राजर्षयः सर्वे धिग्धिगित्येवमब्रुवन्}


\twolineshloka
{विचारश्च समुत्पन्नो निरीक्ष्य नहुषात्मजम्}
{कोऽन्वयं कस्य वा राज्ञः कथं वा स्वर्गमागतः}


\twolineshloka
{कर्मणा केनसिद्धोऽयं क्व वाऽनेन तपश्चितम्}
{कथं वा ज्ञायते स्वर्गे केन वा ज्ञायतेऽप्युत}


\twolineshloka
{एवं विचास्यन्तस्ते राजानं स्वर्गवासिनः}
{दृष्ट्वा पप्रच्छुरन्योन्यं ययातिं नृपतिं प्रति}


\twolineshloka
{विमानपालाः शतशः स्वर्गद्वाराभिरक्षिणः}
{पृष्टा आसनपालाश्च न जानीमेत्यथाऽब्रुवन्}


\twolineshloka
{सर्वे ते ह्यावृतज्ञाना नाभ्यजानन्त तं नृपम्}
{स मुहूर्तादथ नृपो हतौजाश्चाभवत्तदा}


\chapter{अध्यायः १२१}
\twolineshloka
{नारद उवाच}
{}


\twolineshloka
{अथ प्रचलितः स्थानादासनाच्च परिच्युतः}
{कम्पितेनेव मनसा धर्षितः शेकवाह्निना}


\twolineshloka
{म्लानस्रग्भ्रष्टविज्ञानः प्रभ्रष्टमुकुटाङ्गदः}
{विघूर्णन्स्रस्तसर्वाङ्गः प्रभ्रष्टाभरणाम्बरः}


\twolineshloka
{अदृश्यमानस्तान्पश्यन्नपश्यंश्च पुनः पुनः}
{शून्यः शून्येन मनसा प्रपतिष्यन्महीतलम्}


\twolineshloka
{किं मया मनसा ध्यातमशुभं धर्मदूषणम्}
{येनाहं चलितः स्थानादिति राजा व्यचिन्तयत्}


\twolineshloka
{ते तु तत्रैव राजानः सिद्धाश्चाप्सरसस्तथा}
{अपश्यन्त निरालम्बं तं ययातिं परिच्युतम्}


\twolineshloka
{अथैत्य पुरुषः कश्चित्क्षीणपुण्यनिपातकः}
{ययातिमब्रवीद्राजन्द्रेवराजस्य शासनात्}


\twolineshloka
{अतीव मदमत्तस्त्वं न कंचिन्नावमन्यसे}
{मानेन भ्रष्टः स्वर्गस्ते नार्हस्त्वं पार्थिवात्मज}


\twolineshloka
{न च प्रज्ञायसे गच्छ पतस्वेति तमब्रवीत्}
{पतेयं सत्स्विति वचस्त्रिरुत्त्का नहुषात्मजः}


\twolineshloka
{पतिष्यंन्तयामास गतिं गतिमतां वरः}
{एतस्मिन्नेव काले तु नैमिषे पार्थिवर्षभान्}


\twolineshloka
{चतुरोऽपश्यत नृपस्तेषां मध्ये पपात ह}
{प्रतर्दनो वसुमनाः शिबिरौशीनरोऽष्टकः}


\twolineshloka
{वाजपेयेन यज्ञेन तर्पयन्ति सुरेश्वरम्}
{तेषामध्वरजं धूमं स्वर्गद्वारमुपस्थितम्}


\threelineshloka
{ययातिरुपजिघ्रन्वै निपपात महीं प्रति}
{भूमौ स्वर्गे च संबद्धां नदीं धूममयीमिव}
{गङ्गां गामिव गच्छन्तीमालम्ब्य जगतीपतिः}


\twolineshloka
{श्रीमत्स्ववभृथाग्र््येषु चतुर्षु प्रतिबन्धुषु}
{मध्ये निपतितो राजा लोकपालोपमेषु सः}


\twolineshloka
{चतुर्षु हुतकल्पेषु राजसिंहमहाग्निषु}
{पपात मध्ये राजर्षिर्ययातिः पुण्यसङ्क्षये}


\twolineshloka
{तमाहुः पार्थिवाः सर्वे दीप्यमानमिव श्रिया}
{को भवान्कस्य वा बन्धुर्देशस्य नगरस्य वा}


\threelineshloka
{यक्षो वाऽप्यथवा देवो गन्धर्वो राक्षसोऽपि वा}
{न हि मानुषरूपोसि कोवार्थः काङ्क्ष्यते त्वया ॥ययातिरुवाच}
{}


\threelineshloka
{ययातिरस्मि राजर्षिः क्षीणपुण्यश्च्युतो दिवः}
{पतेयं सत्स्विति ध्यायन्भवत्सु पतितस्ततः ॥राजान ऊचुः}
{}


\threelineshloka
{सत्यमेतद्भवतु ते काङ्क्षितं पुरुषर्षभ}
{सर्वेषां नः क्रतुफलं धर्मश्च प्रतिगृह्यताम् ॥ययातिरुवाच}
{}


\threelineshloka
{नाहं प्रतिग्रहधनो ब्राह्मणः क्षत्रियोऽह्यहम्}
{न च मे प्रवणा बुद्धिः परपुण्यविनाशने ॥नारद उवाच}
{}


\twolineshloka
{एतस्मिन्नेव काले तु मृगचर्याक्रमागताम्}
{माधवीं प्रेक्ष्य राजानस्तेऽभिवाद्येदमब्रुवन्}


\threelineshloka
{किमागमनकृत्यं ते किं कुर्मः शासनं तव}
{आज्ञाप्या हि वयं सर्वे तव पुत्रास्तपोधने ॥नारद उवाच}
{}


\twolineshloka
{तेषां तद्भाषितं श्रुत्वा माधवी परया मुदा}
{पितरं समुपागच्छद्ययातिं सा ववन्द च}


\twolineshloka
{स्पृष्ट्वा मूर्धनि तान्पुत्रांस्तापसी वाक्यमब्रवीत्}
{दौहित्रास्तव राजेन्द्र मम पुत्रा न ते पराः}


\twolineshloka
{इमे त्वां तारयिष्यन्ति दिष्टमेतत्पुरातनम्}
{अहं ते दुहिता राजन्माधवी मृगचारिणी}


\twolineshloka
{मयाऽप्युपचितो धर्मस्ततोऽर्धं प्रतिगृह्यताम्}
{यस्माद्राजन्नराः सर्वे अपत्यफलभागिनः}


\threelineshloka
{तस्मादिच्छन्ति दौहित्रान्यथा त्वं वसुधाधिप}
{नारद उवाच}
{ततस्ते पार्थिवाः सर्वे शिरसा जननीं तदा}


% Check verse!
अभिवाद्य नमस्कृत्य मातामहमथाब्रुवन्

उच्चैरनुपमैः स्निग्धैः स्वरैरापूर्य मेदिनीम्

मातामहं नृपतयस्तारयन्तो दिवश्च्युतम्

` राजान ऊचुः


\threelineshloka
{राजधर्मगुणोपेताः सत्यधर्मगुणान्विताः}
{दौहित्रास्ते वयं राजन्दिवमारोह पार्थिव ॥नारद उवाच}
{'}


\twolineshloka
{अथाकस्मादुपगतो गालवोऽप्याह पार्थिवम्}
{तपसो मेऽष्टभागेन स्वर्गमारोहतां भवान्}


\chapter{अध्यायः १२२}
\twolineshloka
{नारद उवाच}
{}


\threelineshloka
{प्रत्यभिज्ञातमात्रोऽथ सद्भिस्तैर्नरपुङ्गवः}
{समारुरोह नृपतिरस्पृशन्वसुधातलम्}
{ययातिर्दिव्यसंस्थानो बभूव विगतज्वरः}


\twolineshloka
{दिव्यमाल्याम्बरधरो दिव्याभरणभूषितः}
{दिव्यगन्धगुणोपेतो न पृथ्वीमस्पृशत्पदा}


% Check verse!
ततो वसुमनाः पूर्वमुच्चैरुच्चारयन्वचः ॥ख्यातो दानपतिर्लोके व्याजहार नृपं तदा
\twolineshloka
{प्राप्तवानस्मि यल्लोके सर्ववर्णेष्वगर्हया}
{तदप्यथ च दास्यामि तेन संयुज्यतां भवान्}


\twolineshloka
{यत्फलं दानशीलस्य क्षमाशीलस्य यत्फलम्}
{यच्च मे फलमाधाने तेन संयुज्यतां भवान्}


\twolineshloka
{ततः प्रतर्दनोऽप्याह वाक्यं क्षत्रियपुङ्गवः}
{यथा धर्मरतिर्नित्यं नित्यं युद्धपरायणः}


\threelineshloka
{प्राप्तवानस्मि यल्लोके क्षत्रवंशोद्भवं यशः}
{वीरशब्दफलं चैव तेन संयुज्यतां भवान्}
{यथा धर्मे रतिर्नित्यं तेन सत्येन खं व्रज}


\twolineshloka
{शिबिरौशीनरो धीमानुवाच मधुरां गिरम्}
{यथा बालेषु नारीषु वैवाह्येषु तथैव च}


\twolineshloka
{सङ्गरेषु निपातेषु तथा तद्व्यसनेषु च}
{अनृतं नोक्तपूर्वं मे तेन सत्येन खं व्रज}


\twolineshloka
{यथा प्राणांश्च राज्यं च राजन्कामसुखानि च}
{त्यजेयं न पुनः सत्यं तेन सत्येन खं व्रज}


\twolineshloka
{यथा सत्येन मे धर्मो यथा सत्येन पावकः}
{प्रीतः शतक्रतुश्चैव तेन सत्येन खं व्रज}


\twolineshloka
{अष्टकस्त्वथ राजर्षिः कौशिको माधवीसुतः}
{अनेकशतयज्वानं नाहुषं प्राह धर्मवित्}


\twolineshloka
{शतशः पुण्डरीका मे गोसवाश्चरिताः प्रभो}
{क्रतवो वाजपेयाश्च तेषां फलमवाप्नुहि}


\threelineshloka
{न मे रत्नानि न धनं न तथाऽन्ये परिच्छदा}
{क्रतुष्वनुपयुक्तानि तेन सत्येन खं व्रज ॥नारद उवाच}
{}


\twolineshloka
{यथायथा हि जल्पन्ति दौहित्रास्तं नराधिपम्}
{तथतथा वसुमतीं त्यक्त्वा राजा दिवं ययौ}


\twolineshloka
{एवं सर्वे समस्तैस्ते राजानः सुकृतैस्तदा}
{ययातिं स्वर्गतो भ्रष्टं तारयामासुरञ्जसा}


\fourlineindentedshloka
{दौहित्राः स्वेन धर्मेण यज्ञदानकृतेन वै}
{चतुर्षु राजवंशेषु संभूताः कुलवर्धनाः}
{मातामहं महाप्राज्ञं दिवमारोपयन्त ते ॥राजान ऊचुः}
{}


\twolineshloka
{राजधर्मगुणोपेताः सर्वधर्मगुणान्विताः}
{दौहित्रास्ते वयं राजन्दिवमारोह पार्थिव}


\chapter{अध्यायः १२३}
\twolineshloka
{नारद उवाच}
{}


\twolineshloka
{सद्भिरारोपितः स्वर्गं पार्थिवैर्भूरिदक्षिणैः}
{अभ्यनुज्ञाय दौहित्रान्ययातिर्दिवमास्थितः}


\twolineshloka
{अभिवृष्टश्च वर्षेण नानापुष्पसुगन्धिना}
{परिष्वक्तश्च पुण्येन वायुना पुण्यगन्धिना}


\twolineshloka
{अचलं स्थानमासाद्य दौहित्रफलनिर्जितम्}
{कर्मभिः स्वैरुपचितो जज्वाल परया श्रिया}


\twolineshloka
{उपगीतोपनृत्तश्च गन्धर्वाप्सरसां गणैः}
{प्रीत्या प्रतिगृहीतश्च स्वर्गे दुन्दुभिनिःस्वनैः}


\twolineshloka
{अभिष्टुतश्च विविधैर्देवराजर्षिचारणैः}
{अर्चितश्चोत्तमार्घेण दैवतैरभिनन्दितः}


\twolineshloka
{प्राप्तः स्वर्गफलं चैव तमुवाच पितामहः}
{निर्वृतं शान्तमनसं वचोभिस्तर्पयन्निव}


\twolineshloka
{चतुष्पादस्त्वया धर्मश्रितो लोक्येन कर्मणा}
{अक्षयस्तव लोकोऽयं कीर्तिश्चैवाक्षया दिवि}


\twolineshloka
{पुनस्त्वयैव राजर्षे स्वकृतेन विघातितम्}
{आवृतं तमसा चेतः सर्वेषां स्वर्गवासिनाम्}


\twolineshloka
{येन त्वां नाभिजानन्ति ततोऽज्ञातोसि पातितः}
{प्रीत्यैव चासि दौहित्रैस्तारितस्त्वमिहागतः}


\threelineshloka
{स्थानं च प्रतिपन्नोऽसि कर्मणा स्वेन निर्जितम्}
{अचलं शाश्वतं पुण्यमुत्तमं ध्रुवमव्ययम् ॥ययातिरुवाच}
{}


\twolineshloka
{भगवन्संशयो मेऽस्ति कश्चितं छेत्तुमर्हसि}
{न ह्यन्यमहमर्हामि प्रष्टुं लोकपितामह}


\twolineshloka
{बहुवर्षसहस्रान्तं प्रजापालनवर्धितम्}
{अनेकक्रतुदानौघैरार्जितं मे महत्फलम्}


\fourlineindentedshloka
{कथं तदल्पकालेन क्षीणं येनास्मि पातितः}
{भगवन्वेत्थ लोकांश्च शाश्वतान्मम निर्मितान्}
{कथं नु मम तत्सर्वं विप्रनष्टं महाद्युते ॥पितामह उवाच}
{}


\twolineshloka
{बहुवर्षसहस्रान्तं प्रजापालनवर्धितम्}
{अनेकक्रतुदानौघैर्यत्त्वयोपार्जितं फलम्}


\twolineshloka
{तदनेनैव दोषेण क्षीणं येनासि पातितः}
{अभिमानेन राजेन्द्र धिक्कृतः स्वर्गवासिभिः}


\twolineshloka
{नायं मानेन राजर्षे न बलेन न हिंसया}
{न शाठ्येन न मायाभिर्लोको भवति शाश्वतः}


\twolineshloka
{नावमान्यास्त्वया राजन्नधमोत्कृष्टमध्यमाः}
{न हि मानप्रदग्धानां कश्चिदस्ति शमः क्वचित्}


\threelineshloka
{पतनारोहणमिदं कथयिष्यन्ति ये नराः}
{विषमाण्यपि ते प्राप्तास्तरिष्यन्ति न संशयः ॥नारद उवाच}
{}


\twolineshloka
{एष दोषोऽभिमानेन पुरा प्राप्तो ययातिना}
{निर्बध्नताऽतिमात्रं च गालवेन महीपते}


\twolineshloka
{श्रोतव्यं हितकामानां सुहृदां हितमिच्छताम्}
{न कर्तव्यो हि निर्बन्धो निर्बन्धो हि क्षयोदयः}


\twolineshloka
{तस्मात्त्वमपि गान्धारे मानं क्रोधं च वर्जय}
{सन्धत्स्व पाण्डवैर्वीर संरम्भं त्यज पार्थिव}


\twolineshloka
{स भवान्सुहृदां पथ्यं वचो गृह्णातु माऽनृतम्}
{समर्थैर्विग्रहं कृत्वा विषमस्थो भविष्यसि}


\twolineshloka
{ददाति यत्पार्थिव यत्करतियद्वा तपस्तप्यति यञ्जुहोति}
{न तस्य नाशोऽस्ति न चापकर्षोनान्यस्तदश्नाति स एव कर्ता}


\twolineshloka
{इदं महाख्यानमनुत्तमं हितंबहुश्रुतानां गतरोषरागिणाम्}
{समीक्ष्य लोके बहुधा प्रधारितंत्रिवर्गदृष्टिः पृथिवीमुपाश्नुते}


\twolineshloka
{` एतत्पुण्यतमं राजन्ययातेश्चरितं महत्}
{यच्छ्रुत्वा श्रावयित्वा च स्वर्गं यान्तीह मानवाः ॥'}


\chapter{अध्यायः १२४}
\twolineshloka
{धृतराष्ट्र उवाच}
{}


\threelineshloka
{भगवन्नेवमेवैतद्यथा वदसि नारद}
{इच्छामि चाहमप्येवं न त्वीशो भगवन्नहम् ॥वैशंपायन उवाच}
{}


\twolineshloka
{एवमुक्त्वा ततः कृष्णमभ्यभाषत कौरवः}
{स्वर्ग्यं लोक्यं च मामात्थ धर्म्यं न्याय्यं च केशव}


\twolineshloka
{न त्वहं स्ववशस्तात क्रियमाणं न मे प्रियम्}
{`न मंस्यन्ते दुरात्मानः पुत्रा मम जनार्दन ॥'}


\twolineshloka
{अङ्ग दुर्योधनं कृष्ण मन्दं शास्त्रातिगं मम}
{अनुनेतुं महाबाहो यतस्व पुरुषोत्तम}


\threelineshloka
{न श्रृणोति महाबाहो वचनं साधु भाषितम्}
{गान्धार्याश्च हृषीकेश विदुरस्य च धीमतः}
{अन्येषां चैव सुहृदां भीष्मादीनां हितैषिणाम्}


\twolineshloka
{स त्वं पापमतिं क्रूरं पापचित्तमचेतनम्}
{अनुशाधि दुरात्मानं स्वयं दुर्योधनं नृपम्}


\twolineshloka
{सुहृत्कार्यं तु सुमहत्कृतं ते स्याञ्जनार्दन ॥वैशंपायन उवाच}
{}


\twolineshloka
{ततोऽभ्यावृत्य वार्ष्णेयो दुर्योधनममर्षणम्}
{अब्रवीन्मधुरां वाचं सर्वधर्मार्थतत्त्ववित्}


\twolineshloka
{दुर्योधन निबोधेदं मद्वाक्यं कुरुसत्तम}
{शर्मार्थं ते विशेषेण सानुबन्धस्य भारत}


\twolineshloka
{महाप्रज्ञकुले जातः साध्वेतत्कर्तुमर्हसि}
{श्रुतवृत्तोपसंपन्नः सर्वैः समुदितो गुणैः}


\twolineshloka
{दौष्कुलेया दुरात्मानो नृशंसा निरपत्रपाः}
{त एतदीदृशं कुर्युर्यथा त्वं तात मन्यसे}


\threelineshloka
{धर्मार्थयुक्ता लोकेऽस्मिन्प्रवृत्तिर्लक्ष्यते सताम्}
{असतां विपरीता तु लक्ष्यते भरतर्षभ}
{विपरीता त्वियं वृत्तिरसकृल्लक्ष्यते त्वयि}


\twolineshloka
{अधर्मश्चानुबन्धोऽत्र घोरः प्राणहरो महान्}
{अनिष्टश्चानिमित्तश्च न च शक्यश्च भारत}


\twolineshloka
{तमनर्थं परिहरन्नात्मश्रेयः करिष्यसि}
{भ्रातॄणामथ भृत्यानां मित्राणां च परन्तप}


% Check verse!
अधर्म्यादयशस्याच्च कर्मणस्त्वं प्रमोक्ष्यसे
\threelineshloka
{प्राज्ञैः शूरैर्महोत्साहैरात्मवद्भिर्बहुश्रुतैः}
{सन्धत्स्व पुरुषव्याघ्र पाण्डवैर्भरतर्षभ}
{तद्धितं च प्रियं चैव धृतराष्ट्रस्य धीमतः}


\twolineshloka
{पितामहस्य द्रोणस्य विदुरस्य महामतेः}
{कृपस्य सोमदत्तस्य बाह्लीकस्य च धीमतः}


\twolineshloka
{अश्वत्थाम्नो विकर्णस्य सञ्जयस्य विविंशतेः}
{ज्ञातीनां चैव भूयिष्ठं मित्राणां च परन्तप}


\threelineshloka
{शमे शर्म भवेत्तात सर्वस्य जगतस्तथा}
{ह्रीमानसि कुले जातः श्रुतवानानृशंस्यवान्}
{तिष्ठ तात पितुः शास्त्रे मातुश्च भरतर्षभ}


\twolineshloka
{एतच्छ्रेयो हि मन्यन्ते पिता यच्छास्ति भारत}
{उक्तमापद्गतः पूर्वं पितुः स्मरसि शासनम्}


\twolineshloka
{रोचते ते पितुस्तात पाण्डवैः सह सङ्गमः}
{सामात्यस्य कुरुश्रेष्ठ तत्तुभ्यां तात रोचताम्}


\twolineshloka
{श्रुत्वा यः सुहृदां शास्त्रं मर्त्यो न प्रतिपद्यते}
{विपाकान्ते दहत्येनं किंपाकमिव भक्षितम्}


\twolineshloka
{यस्तु निःश्रेयसं वाक्यं मोहान्न प्रतिपद्यते}
{स दीर्घसूत्रो हीनार्थः पश्चात्तापेन युज्यते}


\twolineshloka
{यस्तु न्निःश्रेयसं श्रुत्वा प्राक्तदेवाभिपद्यते}
{आत्मनो मतमुत्सृज्य स लोके सुखमेधते}


\twolineshloka
{योऽर्थकामस्य वचनं प्रातिकूल्यान्न मृष्यते}
{शृणोति प्रतिकूलानि द्विषतां वशमेति सः}


\twolineshloka
{सतां मतमतिक्रम्य योऽसतां वर्तते मते}
{शोचन्ते व्यसने तस्य सुहृदो नचिरादिव}


\twolineshloka
{मुख्यानमात्यानुत्सृज्य यो निहीनान्निषेवते}
{स घोरामापदं प्राप्य नोत्तारमधिगच्छति}


\twolineshloka
{योऽसत्सेवी वृथाचारो न श्रोता सुहृदां सताम्}
{परान्वृणीति स्वान्द्वेष्टि तं गौस्त्यजति भारत}


\twolineshloka
{तत्वं विरुद्धा तैर्वीरैस्येतत्राणमिच्छसि}
{अशिष्टेभ्योऽसमर्थेभ्यो मूढेभ्यो भरतर्षभ}


\twolineshloka
{को हि शक्रसमाञ्ज्ञातीनतिक्रम्य महारथान्}
{अन्येभ्यस्त्राणमाशंसेत्त्वदन्यो भुवि मानवः}


\twolineshloka
{जन्मप्रभृति कौन्तेया नित्यं विनिकृतास्त्वया}
{न च ते जातु कुप्यन्ति धर्मात्मानो हि पाण्डवाः}


\twolineshloka
{मिथ्योपचरितास्तात जन्मप्रभृति बान्धवाः}
{त्वयि सम्यङ्भहाबाहो प्रतिपन्ना यशस्विनः}


\twolineshloka
{त्वयाऽपि प्रतिपत्तव्यं तथैव भरतर्षभ}
{स्वेषु बन्धुषु मुख्येषु मा मन्युवशमन्वगाः}


\twolineshloka
{त्रिवर्गयुक्तः प्राज्ञानामारम्भो भरतर्षभ}
{धर्मार्थावनुरुद्ध्यन्ते त्रिवर्गासंभवे नराः}


\twolineshloka
{पृथक्व विनिविष्टानां धर्मं धीरोऽनुरुध्यते}
{मध्यमोऽर्थं कलिं बालः काममेवानुरुद्ध्यते}


\twolineshloka
{इन्द्रियैः प्राकृतो लोभाद्धर्मं विप्रजहाति यः}
{कामार्थावनुपायेन लिप्समानो विनश्यति}


\twolineshloka
{कामार्थौ लिप्समानस्तु धर्ममेवादितश्चरेत्}
{न हि धर्मादपैत्यर्थः कामो वाऽपि कदाचन}


\twolineshloka
{उपायं धर्ममेवाहुस्त्रिवर्गस्य विशांपते}
{लिप्समानो हि तेनाशु कक्षेऽग्निरिव वर्धते}


\twolineshloka
{स त्वं तातानुपायेन लिप्ससे भरतर्षभ}
{आधिराज्यं महद्दीप्तं प्रथितं सर्वराजसु}


\threelineshloka
{आत्मानं तक्षति ह्येष वनं परशुना यथा}
{यः सम्यग्वर्तमानेषु मिथ्या राजन्प्रवर्तते}
{न तस्य हि मतिं छिन्द्याद्यस्य नेच्छेत्पराभवम्}


\twolineshloka
{अविच्छिन्नमतेरस्य कल्याणे धीयते मतिः}
{आत्मवान्नावमन्येत त्रिषु लोकेषु भारत}


\twolineshloka
{अप्यन्यं प्राकृतं कञ्चित्किमु तान्पाण्डवर्षभान्}
{अमर्षवशमापन्नो न किंचिद्बुद्व्यते जनः}


\twolineshloka
{छिद्यते ह्याततं सर्वं प्रमाणं पश्य भारत}
{श्रेयस्ते दुर्जनात्तात पाण्डवैः सह सङ्गतम्}


\twolineshloka
{तैर्हि संप्रीपमाणस्त्वं सर्वान्कामानवाप्स्यसि}
{पाण्डवैर्निर्मितां भूमिं भुञ्जानो राजसत्तम}


\twolineshloka
{पाण्डवान्पृष्ठतः कृत्वा त्राणमाशससऽन्वतः}
{दुःशासने दुर्विषहे कर्णे चापि ससौबले}


\twolineshloka
{एतेष्वैश्वर्यमाधाय भूतिमिच्छसि भारत}
{न चैते तव पर्याप्ता ज्ञाने धर्मार्थयोस्तथा}


\twolineshloka
{विक्रमे चाप्यपर्याप्तः पाण्डवान्प्रति भारत}
{न हीमे सर्वराजानः पर्याप्ताः सहितास्त्वया}


\twolineshloka
{क्रुद्धस्य भीमसेनस्य प्रेक्षितुं मुखमाहवे}
{इदं सनिहितं तात समग्रं पार्थिवं बलम्}


\twolineshloka
{अयं भीष्मस्तथा द्रोणः कर्णश्चायं तथा कृपः}
{भूरिश्रवाः सौमदत्तिरश्वत्थामा जयद्रथः}


\threelineshloka
{अशक्ताः सर्व एवैते प्रतियोद्धुं धनञ्जयम्}
{अजेयो ह्यर्जुनः सङ्ख्ये सर्वैरपि सुरासुरैः}
{मानुषैरपि गन्धर्वैर्मा युद्धे चेत आधिथाः}


\twolineshloka
{दृश्यतां वा पुमान्कश्चित्समग्रे पार्थिवे बले}
{योऽर्जुनं समरे प्राप्य स्वस्तिमानाव्रजेद्गृहान्}


\twolineshloka
{किं ते जनक्षयेणेह कृतेन भरतर्षभ}
{यस्मिञ्चिते जितं ते स्यात्पुमानेकः स दृश्यतां}


\twolineshloka
{यः सदेवान्सगन्धर्वान्सयक्षासुरपन्नगान्}
{अजयत्खाण्डवप्रस्थे कस्तं युद्धेय मानवः}


\twolineshloka
{तथा विराटनगरे श्रूयते महदद्भुतम्}
{एकस्य च बहूनां च पर्याप्तं तन्निदर्शनम्}


\threelineshloka
{युद्धे येन महादेवः साक्षात्सन्तोषितः शिवः}
{तमजेयमनाधृष्यं विजेतुं जिष्णुमच्युतम्}
{आशंससीह समरे वीरमर्जुनमूर्जितम्}


\twolineshloka
{मद्द्वितीयं पुनः पार्थं कः प्रार्थयितुमर्हति}
{युद्धे प्रतीपमायान्तमपि साक्षात्पुरन्दरः}


\twolineshloka
{बाहुभ्यामुद्वहेद्भूमिं दहेत्क्रुद्ध इमाः प्रजाः}
{पातयेत्रिदिवाद्देवान्योऽर्जुनं समरे जयेत्}


\twolineshloka
{पश्य पुत्रांस्तथा भ्रातॄञ्ज्ञातीन्संबन्धिनस्तथा}
{त्वत्कृते न विनश्येयुरिमे भरतसत्तमाः}


\twolineshloka
{अस्तु शेषं कौरवाणां मा पराभूदिदं कुलम्}
{कुलघ्न इति नोच्येथा नष्टकीर्तिर्नराधिप}


\twolineshloka
{त्वामेव स्थापयिष्यन्ति यौवराज्ये महारथाः}
{महाराज्येऽपि पितरं धृतराष्ट्रं जनेश्वरम्}


\twolineshloka
{मा तात श्रियमायान्तीमवमंस्थाः समुद्यताम्}
{अर्धं प्रदाय पार्थेभ्यो महतीं श्रियमाप्नुहि}


\twolineshloka
{पाण्डवैः संशमं कृत्वा कृत्वा च सुहृदां वचः}
{संप्रीयमाणो मित्रैश्च चिरं भद्राण्यवाप्स्यसि}


\chapter{अध्यायः १२५}
\twolineshloka
{वैशंपायन उवाच}
{}


\twolineshloka
{ततः शान्तनवो भीष्मो दुर्योधनममर्षणम्}
{केशवस्य वचः श्रुत्वा प्रोवाच भरतर्षभ}


\threelineshloka
{कृष्णेन वाक्यमुक्तोऽसि सृहृदां शममिच्छता}
{अन्वपद्यस्व तत्तात मा मन्युवशमन्वगाः}
{}


\twolineshloka
{अकृत्वा वचनं तात केशवस्य महात्मनः}
{श्रेयो न जातु न सुखं न कल्याणमवाप्स्यसि}


\twolineshloka
{धर्म्यमर्थ्यं महाबाहुराह त्वां तात केशवः}
{तदर्थमभिपद्यस्व मा राजन्नीनशः प्रजाः}


\twolineshloka
{ज्वलितां त्वमिमां लक्ष्मीं भारतीं सर्वराजसु}
{जीवतो धृतराष्ट्रस्य दौरात्म्याद्भंशयिष्यसि}


\twolineshloka
{आत्मानं च सहामात्यं सपुत्रभ्रातृबान्धवम्}
{अहमित्यनया बुद्ध्या जीविताद्धंशयिष्यतसि}


\twolineshloka
{अतिक्रामन्केशवस्य तथ्यं वचनमर्थवत्}
{पितुश्च भारतश्रेष्ठ विदुरस्य च धीमतः}


\twolineshloka
{मा कुलघ्नः कुपुरुषो दुर्मतिः कापथं गमः}
{मातरं पितरं चैव मा मञ्जीः शोकसागरे}


\twolineshloka
{अथ द्रोणोऽब्रवीत्तत्र दुर्योधनमिदं वचः}
{अमर्षवशमापन्नं निःश्वसन्तं पुनःपुनः}


\twolineshloka
{धर्मार्थयुक्तं वचनमाह त्वां तात केशवः}
{तथा भीष्मः शान्तनवस्तञ्जुषस्व नराधिप}


\twolineshloka
{प्राज्ञौ मेधाविनौ दान्तावर्थकामौ बहुश्रुतौ}
{आहतुस्त्वां हितं वाक्यं तञ्जुषस्व नराधिप}


\twolineshloka
{अनुतिष्ठ महाप्राज्ञ कृष्णभीष्मौ यदूचतुः}
{माधवं बुद्धिमोहेन माऽवमंस्थाः परन्तप}


\twolineshloka
{ये त्वां प्रोत्साहयन्त्येते नैते कृत्याय कर्हिचित्}
{वैरं परेषां ग्रीवायां प्रतिमोक्ष्यन्ति संयुगे}


\twolineshloka
{मा जीघनः प्रजाः सर्वाः पुत्रान्भ्रातॄंस्तथैव च}
{वासुदेवार्जुनौ यत्र विद्ध्यजेयानलं हि तान्}


\twolineshloka
{एतच्चैव मतं सत्यं सुहृदोः कृष्णभीष्मयोः}
{यदि नादास्यसे तात पश्चात्तप्स्यसि भारत}


\threelineshloka
{यथोक्तं जामदग्न्येन भूयानेष ततोऽर्जुनः}
{कृष्णो हि देवकीपुत्रो देवैरपि सुदुःसहः}
{किं ते सुखप्रियेणेह प्रोक्तेन भरतर्षभ}


\threelineshloka
{एतत्ते सर्वमाख्यातं यथेच्छसि तथा कुरु}
{न हि त्वामुत्सहे वक्तुं भूयो भरतसत्तम ॥वैशंपायन उवाच}
{}


\twolineshloka
{तस्मिन्वाक्यान्तरे वाक्यं क्षत्ताऽपि विदुरोऽब्रवीत्}
{दुर्योधनमभिप्रेक्ष्य धार्तराष्ट्रममर्षणम्}


\twolineshloka
{दुर्योधन न शोचामि त्वामहं भरतर्षभ}
{इमो तु वृद्धौ शोचामि गान्धारीं पितरं च ते}


\twolineshloka
{यावनाथौ चरिष्येते त्वया नाथेन दुर्हृदा}
{हतमित्रौ हतामात्यौ लूनपक्षाविवाण्डजौ}


\twolineshloka
{भिक्षुकौ विचरिष्येते शोचन्तौ पृथिवीमिमाम्}
{कुलघ्नमीदृशं पापं जनयित्वा कुपूरुषम्}


\twolineshloka
{अथ दुर्योधं राजा धृतराष्ट्रोऽभ्यभाषत}
{आसीनं भ्रातृभिः सार्धं राजभिः परिवारितम्}


\twolineshloka
{दुर्योधन निबोधेदं शौरिणोक्तं महात्मना}
{आदत्स्व शिवमत्यन्तं योगक्षेमवदव्ययम्}


\twolineshloka
{अनेन हि सहायेन कृष्णेनाक्लिष्टकर्मणा}
{इष्टान्सर्वानभिप्रायान्प्राप्स्यामः सर्वराजसु}


\twolineshloka
{सुसंहतः केशवेन तात गच्छ युधिष्ठिरम्}
{चर स्वस्त्ययनं कृत्स्नं भरतानामनामयम्}


\twolineshloka
{वासुदेवेन तीर्थेन तात गच्छस्व संशमम्}
{कालप्राप्तमिदं मन्ये मा त्वं दुर्योधनातिगाः}


\twolineshloka
{शमं चेद्याचमानं त्वं प्रत्याख्यास्यसि केशवम्}
{त्वदर्थमभिजल्पन्तं न तवास्त्यपराभवः}


\chapter{अध्यायः १२६}
\twolineshloka
{वैशंपायन उवाच}
{}


\twolineshloka
{धृतराष्ट्रवचः श्रुत्वा भीष्मद्रोणौ समव्यथौ}
{दुर्योधनमिदं वाक्यमूचतुः शासनातिगम्}


\twolineshloka
{यावत्कृष्णावसन्नद्धौ यावत्तिष्ठति गाण्डिवम्}
{यावद्धौम्यो न मेधाग्नौ जुहोतीह द्विषद्बलम्}


\twolineshloka
{यावन्न प्रेक्षते क्रूद्धः सेनां तव युधिष्ठिरः}
{ह्रीनिषेवो महेष्वासस्तावच्छाम्यतु वैशसम्}


\twolineshloka
{यावन्न दृश्यते पार्थः स्वेऽप्यनीके व्यवस्थितः}
{भीमसेनो महेष्वासस्तावच्छाम्यतु वैशसम्}


\twolineshloka
{यावन्न चरते मार्गान्पृतनामभिधर्षयन्}
{भीमसेनो गदापाणिस्तावत्संशाम्य पाण्डवैः}


\threelineshloka
{यावन्न शातयत्याजौ शिरांसि गजयोधिनाम्}
{गदया वीरघातिन्या फलानीव वनस्पतेः}
{कालेन परिपक्वानि तावच्छाम्यतु वैशसम्}


% Check verse!
नकुलः सहदेवश्च धृष्टद्युम्नश्च पार्षतः ॥विराटश्च शिखण्डी च शैशुपालिश्च दंशिताः
\twolineshloka
{यावन्न प्रविशन्त्येते नक्रा इव महार्णवम्}
{कृतास्त्राः क्षिप्रमस्यन्तस्तावच्छाम्यतु वैशसम्}


\twolineshloka
{यावन्न सुकुमारेषु शरीरेषु महीक्षिताम्}
{गार्ध्रपत्राः पतन्त्युग्रास्तावच्छाम्यतु वैशसम्}


\twolineshloka
{चन्दनागुरुदिग्धेषु हारनिष्कधरेषु च}
{नोरस्सु यावद्योधानां महेष्वासैर्महेषवः}


\twolineshloka
{कृतस्त्रैः क्षिप्रमस्यद्भिर्दूरपातिभिरायसाः}
{अभिलक्ष्यैर्निपात्यन्ते तावच्छाम्यतु वैशसम्}


\twolineshloka
{अभिवादयमानं त्वां शिरसा राजकुञ्जरः}
{पाणिभ्यां प्रतिगृह्णातु धर्मराजो युधिष्ठिरः}


\twolineshloka
{ध्वजाङ्कुशपताकाङ्कं दक्षिणं ते सुदक्षिणः}
{स्कन्धे निक्षिपतां बहुं शान्तये भरतर्षभ}


\twolineshloka
{रत्नौषधिसमेतेन रत्नाङ्गुलितलेन च}
{उपविष्टस्य पृष्ठं ते पाणिना परिमार्जतु}


\twolineshloka
{शालस्कन्धो महाबहुस्त्वां स्वजानो वृकोदरः}
{5-126-15bसाम्नाऽभिवदतां चापि शान्तये भरतर्षभ}


\twolineshloka
{अर्जुनेन यमाभ्यां च त्रिभिस्तैरभिवादितः}
{मूर्ध्नि तान्समुपाघ्राय प्रेम्णाऽभिवद पार्थिव}


\twolineshloka
{दृष्ट्वा त्वां पाण्डवैर्वीरैर्भ्रातृभिः सह संगतम्}
{यावदानन्दजाश्रूणि प्रन्मुञ्चन्तु नराधिपाः}


\twolineshloka
{घुष्यतां राजधानीषु सर्वसंपन्महीक्षिताम्}
{पृथिवी भ्रातृभावेन भुज्यतां विज्वरो भव}


\chapter{अध्यायः १२७}
\twolineshloka
{वैशंपायन उवाच}
{}


\twolineshloka
{श्रुत्वा दुर्योधनो वाक्यमप्रियं कुरुसंसदि}
{प्रत्युवाच महाबाहुं वासुदेवं यशस्विनम्}


\twolineshloka
{प्रसमीक्ष्य भवानेतद्वक्तुमर्हति केशव}
{मामेव हि विशेषेण विभाष्य परिगर्हसे}


\twolineshloka
{भक्तिवादेन पार्थानामकस्मान्मधुसूदन}
{भवान्गर्हयति नित्यं किं समीक्ष्य बलाबलम्}


\twolineshloka
{भवान्क्षत्ता च राजा वाऽप्याचार्यो वा पितामहः}
{मामेव परिगर्हन्ते नान्यं कंचन पाण्डवम्}


\twolineshloka
{न चाहं लक्षये कंचिद्व्यभिचारमिहात्मनः}
{अथ सर्वे भवन्तो मां विद्विषन्ति सराजकाः}


\twolineshloka
{न चाहं कंचिदत्यर्थमपराधमरिन्दम}
{विचिन्तयन्प्रपश्यामि सुसूक्ष्ममपि केशव}


\twolineshloka
{प्रियाभ्युपगते द्यूते पाण्डवा मधूसूदन}
{जिताः शकुनिना राज्यं तत्र किं मम दुष्कृतम्}


\twolineshloka
{यत्पुनर्द्रविणं किंचित्तत्राजीयन्त पाण्डवाः}
{तेभ्य एवाभ्यनुज्ञातं तत्तद मधुसूदन}


\twolineshloka
{अपराधो न चास्माकं यत्ते ह्यक्षैः पराजिताः}
{अजेया जयतां श्रेष्ठ पार्थाः प्रव्राजिता वनम्}


\twolineshloka
{केन वाऽप्यपवादेन विरुद्ध्यन्त्यरिभिः सह}
{अशक्ताः पाण्डवाः कृष्ण प्रहृष्टाः प्रत्यमित्रवत्}


\twolineshloka
{किमस्माभिः कृतं तेषां कस्मिन्वा पुनरागसि}
{धार्तराष्ट्राञ्जिघांसन्ति पाण्डवाः सृञ्जयैः सह}


\twolineshloka
{न चापि वयमुग्रेण कर्मणा वचनेन वा}
{प्रभ्रष्टाः प्रणमामेह भयादपि शतक्रतुम्}


\twolineshloka
{न च तं कृष्ण पश्यामि क्षत्रधर्ममनुष्ठितम्}
{उत्सहेत युधा जेतुं यो नः शत्रुनिबर्हण}


\twolineshloka
{न हि भीष्मकृपद्रोणाः सकर्णा मधुसूदन}
{देवैरपि युधा जेतुं शक्याः किमुत पाण्डवैः}


\twolineshloka
{स्वधर्ममनुपश्यन्तो यदि माधव संयुगे}
{अस्त्रेण निधनं काले प्राप्स्यामः स्वर्ग्यमेव तत्}


\twolineshloka
{मुख्यश्चैवैष नो धर्मः क्षत्रियाणां जनार्दन}
{यच्छयीमहि सङ्ग्रामे शरतल्पगता वयम्}


\twolineshloka
{ते वयं वीरशयनं प्राप्स्यामो यदि संयुगे}
{अप्रणम्यैव शत्रूणां न नस्तप्स्यन्ति माधव}


\twolineshloka
{कश्च जातु कुले जातः क्षत्रधर्मेण वर्तयन्}
{भयाद्वृत्तिं समीक्ष्यैवं प्रणमेदिह कर्हिचित्}


\twolineshloka
{उद्यच्छेदेव न नमेदुद्यमो ह्येव पौरुषम्}
{अप्यपर्वणि भज्येत न नमेदिह कर्हिचित्}


\twolineshloka
{इति मातङ्गवचनं परिप्सन्ति हितेप्सवः}
{धर्माय चैव प्रणमेद्ब्राह्मणेभ्यश्च मद्विधः}


\twolineshloka
{अचिन्तयन्कंचिदन्यं यावज्जीवं तथा चरेत्}
{एष धर्मः क्षत्रियाणां मतमेतच्च मे सदा}


\twolineshloka
{राज्यांशश्चाभ्यनुज्ञातो यो मे पित्रा पुराऽभवत्}
{न स लभ्यः पुनर्जातु मयि जीवति केशव}


\threelineshloka
{यावच्च राजा ध्रियते धृतराष्ट्रो जनार्दन}
{न्यस्तशस्त्रा वयं ते वाऽप्युपजीवाम माधव}
{अप्रदेयं पुरा दत्तं राज्यं परवतो मम}


\twolineshloka
{अज्ञानाद्वा भयाद्वाऽपि मयि बाले जनार्दन}
{न तदद्य पुनर्लभ्यं पाण्डवैर्वृष्णिनन्दन}


\threelineshloka
{ध्रियमाणे महाबाहौ मयि संप्रति केशव}
{यावद्धि तीक्ष्णया सूच्या विद्ध्येदग्रेण केशव}
{तावदप्यपरित्याज्यं भूमेर्नः पाण्डवान्प्रति}


\chapter{अध्यायः १२८}
% Check verse!
वैशंपायन उवाच
\twolineshloka
{ततः प्रहस्य दाशार्हः क्रोधपर्याकुलेक्षणः}
{दुर्योधनमिदं वाक्यमब्रवीत्कुरुसंसदि}


\twolineshloka
{लप्स्यसे वीरशयनं काममेतदवाप्स्यसि}
{स्थिरो भव सहामात्यो विमर्दो भविता महान्}


\twolineshloka
{यच्चैतन्मन्यसे मूढ न मे कश्चिद्व्यतिक्रमः}
{पाण्डवेष्विति तत्सर्वं निबोध त्वं नराधिप}


\twolineshloka
{श्रियां संतप्यमानेन पाण्डवानां महात्मनाम्}
{त्वया दुर्मन्त्रितं द्यूतं सौबलेन च भारत}


\twolineshloka
{कथं च ज्ञातयस्तात श्रेयांसः साधुसंमताः}
{तथाऽन्याय्यमुपस्थातुं जिह्मेनाजिह्मचारिणः}


\twolineshloka
{अक्षद्यूतं महाप्रज्ञ सतां मतिविनाशनम्}
{असतां तत्र जायन्ते भेदाश्च व्यसनानि च}


\twolineshloka
{तदिदं व्यसनं घोरं त्वया द्यूतमुखं कृतम्}
{असमीक्ष्य सदाचारान्सार्धं पापानुबन्धनैः}


\twolineshloka
{कश्चान्यो भ्रातृभार्यां वै विप्रकर्तृं तथार्हति}
{आनीय च सभां व्यक्तं यथोक्ता द्रौपदी त्वया}


\twolineshloka
{कुलीना शीलसंपन्ना प्राणेभ्योऽपि गरीयसी}
{महिषी पाण्डुपुत्राणां तथा विनिकृता त्वया}


\twolineshloka
{जानन्ति कुरवः सर्वे यथोक्ताः कुरुसंसदि}
{दुःशासनेन कौन्तेयाः प्रव्रजन्तः परन्तपाः}


\twolineshloka
{सम्यग्वृत्तेष्वलुब्धेषु सततं धर्मचारिषु}
{स्वेषु बन्धुषु कः साधुश्चरेदेवमसांप्रतम्}


\twolineshloka
{नृशंसानामनार्याणां तथा परुषभाषणम्}
{कर्णदुःशासनाभ्यां च त्वया च बहुशः कृतम्}


\twolineshloka
{सह मात्रा प्रदग्धुं तान्बालकान्वारणावते}
{आस्थितः परमो यत्नो न समृद्धश्च तत्तव}


\twolineshloka
{ऊषुश्च सुचिरं कालं प्रच्छन्नाः पाण्डवास्तदा}
{मात्रा सहैकचक्रायां ब्राह्मणस्य निवेशने}


\twolineshloka
{विषेण सर्पबन्धैश्च यतिताः पाण्डवास्त्वया}
{सर्वोपायैर्विनाशाय न समृद्धं च तत्तव}


\twolineshloka
{एवंबुद्धिः पाण्डवेषु मिथ्यावृत्तिः सदा भवान्}
{कथं ते नापराधोऽस्ति पाण्डवेषु महात्मसु}


\twolineshloka
{`एवंवृत्तः कथं राज्ये स्थातुमर्हसि पापकृत्}
{स राज्याच्च सुखाच्चैव हास्यसे कुलपांसन}


\twolineshloka
{यच्चैभ्यो याचमानेभ्यः पित्र्यमंशं न दित्सति}
{तच्च पाप प्रदाताऽसि भ्रष्टैश्वर्यो निपातितः}


\twolineshloka
{कृत्वा बहून्यकार्याणि पाण्डवेषु नृशंसवत्}
{मिथ्यावृत्तिरनार्यः सन्नद्य विप्रतिपद्यसे}


\twolineshloka
{मातापितृभ्यां भीष्मेण द्रोणेन विदुरेण च}
{शाम्येति मुहुरुक्तोसि न च शाम्यसि पार्थिव}


\twolineshloka
{शमे हि सुमहाँल्लाभस्तव पार्थस्य चोभयोः}
{न च रोचयसे राजन्किमन्यद्बुद्धिलाघवात्}


\threelineshloka
{न शर्म प्राप्स्यसे राजन्नुत्क्रम्य सुहृदां वचः}
{अधर्म्यमयशस्यं च क्रियते पार्थिव त्वया ॥वैशंपायन उवाच}
{}


\twolineshloka
{एवं ब्रुवति दाशार्हे दुर्योधनममर्षणम्}
{दुःशासन इदं वाक्यमब्रवीत्करुसंसदि}


\twolineshloka
{न चेत्सन्धास्यसे राजन्स्वेन कामेन पाण्डवैः}
{बद्ध्वा किल त्वां दास्यन्ति कुन्तीपुत्राय कौरवाः}


\twolineshloka
{वैकर्तनं त्वां च मां च त्रीनेतान्मनुजर्षभ}
{पाण्डवेभ्यः प्रदास्यन्ति भीष्मो द्रोणः पिता च ते}


\twolineshloka
{भ्रातुरेतद्वचः श्रुत्वा धार्तराष्ट्रः सुयोधनः}
{क्रुद्धः प्रातिष्ठतोत्थाय महानाग इव श्वसन्}


\twolineshloka
{विदुरं च सोमदत्तं च महाराजं च बाह्लिकम्}
{कृपं च सोमदत्तं च भीष्मं द्रोणं जनार्दनम्}


\twolineshloka
{सर्वानेताननादृत्य दुर्मतिर्निरपत्रपः}
{अशिष्टवदमर्यादो मानी मान्यावमानिता}


\twolineshloka
{तं प्रस्थितमभिप्रेक्ष्य भ्रातरो मनुजर्षभम्}
{अनुजग्मुः सहामात्या राजानश्चापि सर्वशः}


\twolineshloka
{सभायामुत्थितं क्रुद्धं प्रस्थितं भ्रातृभिः सह}
{दुर्योधनमभिप्रेक्ष्य भीष्मः शान्तनवोऽब्रवीत्}


\twolineshloka
{धर्मार्थावभिसन्त्यज्य संरम्भं योऽनुमन्यते}
{हसन्ति व्यसने तस्य दुर्हृदो नचिरादिव}


\twolineshloka
{दुरात्मा राजपुत्रोऽयं धार्तराष्ट्रोऽनुपायवित्}
{मिथ्याभिमानी राज्यस्य क्रोधलोभवशानुगः}


\twolineshloka
{कालपक्वमिदं मन्ये सर्वं क्षत्रं जनार्दन}
{सर्वे ह्यनुसृता मोहात्पार्थिवाः सह मन्त्रिभिः}


\twolineshloka
{भीष्मस्याथ वचः श्रुत्वा दाशार्हः पुष्करेक्षणः}
{भीष्मद्रोणमुखान्सर्वानभ्यभाषत वीर्यवान्}


\twolineshloka
{सर्वेषां कुरुवृद्धानां महानयमतिक्रमः}
{प्रसह्य मन्दमैश्वर्ये न नियच्छन्ति यन्नृपम्}


\twolineshloka
{तत्र कार्यमहं मन्ये कालप्राप्तमरिन्दमाः}
{क्रियमाणे भवेच्छ्रेयस्तत्सर्वं श्रृणुतानघाः}


\twolineshloka
{प्रत्यक्षमेतद्भवतां यद्वक्ष्यामि हितं वचः}
{भवतामानुकूल्येन यदि रोचेत भारताः}


\twolineshloka
{पुत्रो वै भोजराजस्य दुराचारो ह्यनात्मवान्}
{जीवतः पितुरैश्वर्यं हृत्वा मृत्युवशं गतः}


\twolineshloka
{उग्रसेनसुतः कंसः परित्यक्तः स बान्धवैः}
{ज्ञातीनां हितकामेन मया शस्तो महामृधे}


\twolineshloka
{आहुकः पुनरस्माभिर्ज्ञातिभिश्चापि सत्कृतः}
{उग्रसेनः कृतो राजा भोजराजन्यवर्धनः}


\twolineshloka
{संसमेकं परित्यज्य कुलार्थे सर्वयादवाः}
{संभूय सुखमेधन्ते भारतान्धकवृष्णयः}


\twolineshloka
{अपि चाप्यवदद्राजन्परमेष्ठी प्रजापतिः}
{व्यूढे देवासुरे युद्धेऽभ्युद्यतेष्वायुधेषु च}


\twolineshloka
{द्वैधीभूतेषु लोकेषु विनश्यत्सु च भारत}
{अब्रवीत्तु तदा देवो भगवाँल्लोकभावनः}


\twolineshloka
{पराभविष्यन्त्यसुरा दैतेया दानवैः सह}
{आदित्या वसवो रुद्रा भविष्यन्ति दिवौसकः}


\twolineshloka
{देवासुरमनुष्याश्च गन्धर्वोरगराक्षसाः}
{अस्मिन्युद्धे सुसंक्रुद्धा हनिष्यन्ति परस्परम्}


\threelineshloka
{` वर्तमानं जगत्सर्वं मुहूर्तान्न भविष्यति}
{'इति मत्वाऽब्रवीद्धर्मं परमेष्ठी प्रजापतिः}
{वरुणाय प्रयच्छैतान्बद्ध्वा दैतेयदानवान्}


\twolineshloka
{एवमुक्तस्ततो धर्मो नियोगात्परमेष्ठिनः}
{वरुणाय ददौ सर्वान्बद्ध्वा दैतेयदानवान्}


\twolineshloka
{तान्बद्ध्वा धर्मपाशैश्च स्वैश्च पाशैर्जलेश्वरः}
{वरुणः सागरे यत्तो नित्यं रक्षति दानवान्}


\twolineshloka
{तथा दुर्योधनं कर्णं शकुनिं चापि सौबलम्}
{बद्ध्वा दुःशासनं चापि पाण्डवेभ्यः प्रयच्छत}


\twolineshloka
{त्यजेत्कुलार्थे पुरुषं ग्रामस्यार्थे कुलं त्यजेत्}
{ग्रामं जनपदस्यार्थे आत्मार्थे पृथिवीं त्यजेत्}


\twolineshloka
{राजन्दुर्योधनं बद्ध्वा ततः संशाम्य पाण्डवैः}
{त्वत्कृते न विनश्येयुः क्षत्रियाः क्षत्रियर्षभ}


\chapter{अध्यायः १२९}
\twolineshloka
{वैशंपायन उवाच}
{}


\twolineshloka
{कृष्णस्य तु वचः श्रुत्वा धृतराष्ट्रो जनेश्वरः}
{विदुरं सर्वधर्मज्ञं त्वरमाणोऽभ्यभाषत}


\twolineshloka
{गच्छ तात महाप्राज्ञ गान्धारीं दीर्घदर्शिनीम्}
{आनयेह तया सार्धमनुनेष्यामि दुर्मतिम्}


\twolineshloka
{यदि सा न दुरात्मानं शमयेद्दुष्टचेतसम्}
{अपि कृष्णस्य सुहृदस्तिष्ठेम वचने वयम्}


\twolineshloka
{दुर्बुद्धेर्दुःसहायस्य शमार्थं ब्रुवती वचः ॥अपि नो व्यसनं घोरं दुर्योधनकृतं महत्}
{}


\threelineshloka
{शमयेच्चिररात्राय योगक्षेमवदव्ययम्}
{राज्ञस्तु वचनं श्रुत्वा विदुरो दीर्घदर्शिनीम्}
{}


\twolineshloka
{आनयामास गान्धारीं धृतराष्ट्रस्य शासनात् ॥धृतराष्ट्र उवाच}
{}


\twolineshloka
{एष गान्धारि पुत्रस्ते दुरात्मा शासनातिगः}
{ऐश्वर्यलोभादैश्वर्यं जीवितं च प्रहास्यति}


\threelineshloka
{अशिष्टवदमर्यादः पापैः सह दुरात्मवान्}
{सभाया निर्गतो मूढो व्यतिक्रम्य सुहृद्वचः ॥वैशंपायन उवाच}
{}


\twolineshloka
{सा भर्तृवचनं श्रुत्वा राजपुत्री यशस्विनी}
{अन्विच्छन्ती महच्छ्रेयो गान्धारी वाक्यमब्रवीत्}


\threelineshloka
{आनायय सुतं क्षिप्रं राज्यकामुकमातुरम्}
{नहि राज्यमशिष्टेन शक्यं धर्मार्थलोपिना}
{आप्तुमाप्तं तथाऽपीदमविनीतेन सर्वथा}


\twolineshloka
{त्वं ह्येवात्र भृशं गर्ह्यो धृतराष्ट्र सुतप्रियः}
{यो जानन्पापतामस्य तत्प्रज्ञामनुवर्तसे}


\twolineshloka
{स एष काममन्युभ्यां प्रलब्धो लोभमास्थितः}
{अशक्योऽद्य त्वया राजन्विनिवर्तयितुं बलात्}


\twolineshloka
{राष्ट्रप्रदाने मूढस्य बालिशस्य दुरात्मनः}
{दुःसहायस्य लुब्धस्य धृतराष्ट्रोऽश्रुते फलम्}


\twolineshloka
{कथं हि स्वजने भेदमुपेक्षेत महीपतिः}
{भिन्नं हि स्वजनेन त्वां प्रहरिष्यन्ति शत्रवः}


\threelineshloka
{या हि शक्या महाराज साम्ना भेदेन वा पुनः}
{निस्तर्तुमापदः स्वेषु दण्डं कस्तत्र पातयेत् ॥वैशंपायन उवाच}
{}


\twolineshloka
{शासनाद्धृतराष्ट्रस्य दुर्योधनममर्षणम्}
{मातुश्च वचनात्क्षत्ता सभां प्रावेशयत्पुनः}


\twolineshloka
{स मातुर्वचनाकाङ्क्षी प्रविवेश पुनः सभाम्}
{अभिताम्रेक्षणः क्रोधान्निश्वसन्निव पन्नगः}


\twolineshloka
{तं प्रविष्टमभिप्रेक्ष्य पुत्रमुत्पथमास्थितम्}
{विगर्हमाणा गान्धारी शमार्थं वाक्यमब्रवीत्}


\twolineshloka
{दुर्योधन निबोधेदं वचनं मम पुत्रक}
{हितं ते सानुबन्धस्य तथाऽऽयत्यांसुखोदयम्}


\twolineshloka
{दुर्योधन यदाह त्वां पिता भरतसत्तम}
{भीष्मो द्रोणः कृपः क्षत्ता सुहृदां कुरु तद्वचः}


\twolineshloka
{भीष्मस्य तु पितुश्चैव मम चापचितिः कृता}
{भवेद्द्रोणमुखानां च सुहृदां शाम्यता त्वया}


\twolineshloka
{न हि राज्यं महाप्राज्ञ स्वेन कामेन शक्यते}
{अवाप्तुं रक्षितुं वाऽपि भोक्तुं भरतसत्तम}


\twolineshloka
{न ह्यवश्येन्द्रियो राज्यमश्रीयाद्दीर्घमन्तरम्}
{विजितात्मा तु मेधावी स राज्यमभिपालयेत्}


\twolineshloka
{कामक्रोधौ हि पुरुषमर्थेभ्यो व्यपकर्षतः}
{तौ तु शत्रू विनिर्जित्य राजा विजयते महीम्}


\twolineshloka
{लोकेश्वर प्रभुत्वंहि महदेतद्दुरात्मभिः}
{राज्यं नामेप्सितं स्थानं न शक्यमभिरक्षितुम्}


\twolineshloka
{इन्द्रियाणि महत्प्रेप्सुर्नियच्छेदर्थधर्मयोः}
{इन्द्रियैर्नियतैर्बुद्धिर्वर्धतेऽग्निरिवेन्धनैः}


\twolineshloka
{अविधेयानि हीमानि व्यापादयितुमप्यलम्}
{अविधेया इवादान्ता हयाः पथि कुसारथिम्}


\twolineshloka
{अविजित्य य आत्मानममात्यान्विजिगीषते}
{अमित्रान्वाऽजितामात्यः सोऽवशः परिहीयते}


\twolineshloka
{आत्मानमेव प्रथमं द्वेष्यरूपेण योजयेत्}
{ततोऽमात्यानमित्रांश्च न मोघं विजिगीषते}


\twolineshloka
{वश्येन्द्रियं जितामात्यं धृतदण्डं विकारिषु}
{परीक्ष्यकारिणं धीरमत्यर्थं श्रीर्निषेवते}


\twolineshloka
{क्षुद्राक्षेणेव जालेन झषावपिहितावुभौ}
{कामक्रोधौ शरीरस्थौ प्रज्ञानं तौ विलुम्पतः}


\twolineshloka
{याभ्यां हि देवाः स्वर्यातुः स्वर्गस्य पिदधुर्मुखम्}
{बिभ्यतोऽनुपरागस्य कामक्रोधौ स्म वर्धितौ}


\twolineshloka
{कामं क्रोधं च लोभं च दम्भं दर्पं च भूमिपः}
{सम्यग्विजेतुं यो वेद स महीमभिजायते}


\twolineshloka
{सततं निग्रहे युक्त इन्द्रियाणां भवेन्नृपः}
{ईप्सन्नर्थं च धर्मं च द्विषतां च पराभवम्}


\twolineshloka
{कामाभिभूतः क्रोधाद्वा यो मिथ्या प्रतिपद्यते}
{स्वेषु चान्येषु वा तस्य न सहाया भवन्त्युत}


\twolineshloka
{एकीभूतैर्महाप्राज्ञैः शूरैररिनिबर्हणैः}
{पाण्डवैः पृथिवीं तात भोक्ष्यसे सहितः सुखी}


\twolineshloka
{यथा भीष्मः शान्तनवो द्रोणश्चापि महारथः}
{आहतुस्तात तत्सत्यमजेयौ कृष्णपाण्डवौ}


\twolineshloka
{प्रपद्यष्व महाबाहुं कृष्णमक्लिष्टकारिणम्}
{प्रसन्नो हि सुखाय स्यादुभयोरेव केशवः}


\twolineshloka
{सुहृदामर्थकामानां यो न तिष्ठति शासने}
{प्राज्ञानां कृतविद्यानां स नरः शत्रुनन्दनः}


\twolineshloka
{न युद्धे तात कल्याणं न धर्मार्थौ कुतः सुखम्}
{न चापि विजयो नित्यं मा युद्धे चेत आधिथाः}


\twolineshloka
{भीष्मेण हि महाप्राज्ञ पित्र ते बाह्लिकेन च}
{दत्तोंऽशः पाण्डुपुत्राणां भेदाद्भतैररिन्दम}


\twolineshloka
{तस्थ चैतत्प्रदानस्य फलमद्यानुपश्यसि}
{यद्भुङ्क्षे पृथिवीं कृत्स्नां शूरैर्निहतकण्टकाम्}


\twolineshloka
{प्रयच्छ पाण्डुपुत्राणां यथोचितमरिन्दम}
{यदीच्छसि सहामात्यो भोक्तुमर्धं प्रदीयताम्}


\twolineshloka
{अलमर्धं पृथिव्यास्ते महामात्यस्य जीवितुम्}
{सुहृदां वचने तिष्ठन्यशः प्राप्स्यति भारत}


\twolineshloka
{श्रीमद्भिरात्मवद्भिस्तैर्बुद्धिमद्भिर्जितेन्द्रियैः}
{पाण्डवैर्विग्रहस्तात भ्रंशयेन्महतः सुखात्}


\twolineshloka
{निगृह्य सुहृदां मन्युं शाधि राज्यं यथोचितम्}
{स्वमंशं पाण्डुपुत्रेभ्यः प्रदाय भरतर्षभ}


\twolineshloka
{अलमङ्ग निकारोऽयं त्रयोदशसमाः कृतः}
{शमयैनं महाप्राज्ञ कामक्रोधसमेधितम्}


\twolineshloka
{न चैष शक्तः पार्थानां यस्त्वदर्थमभीप्सति}
{सूतपुत्रो दृढक्रोधो भ्राता दुःशासनश्च ते}


\twolineshloka
{भीष्मे द्रोणे कृपे कर्णे भीमसेने धनञ्जये}
{धृष्टद्युम्ने च संक्रुद्धे न स्युः सर्वाः प्रजा ध्रुवम्}


\twolineshloka
{अमर्षवशमापन्नो मा कुरूंस्तात जीघनः}
{एषा हि पृथिवी कृत्स्ना मा गमत्त्वत्कृते वधम्}


\twolineshloka
{यच्च त्वं मन्यसे मूढ भीष्मद्रोणकृपादयः}
{योत्स्यन्ते सर्वशक्त्येति नैतदद्योपपद्यते}


\twolineshloka
{समं हि राज्यं प्रीतिश्च स्थानं हि विदितात्मनाम्}
{पाण्डवेष्वथ युष्मासु धर्मस्त्वभ्यधिकस्ततः}


\twolineshloka
{राजपिण्डभयादेते यदि हास्यन्ति जीवितम्}
{न हि शक्ष्यन्ति राजानं युधिष्ठिरमुदीक्षितुम्}


\twolineshloka
{न लोभादर्थसंपत्तिर्नराणामिह दृश्यते}
{तदलं तात लोभेन प्रशाम्य भरतर्षभ}


\chapter{अध्यायः १३०}
\twolineshloka
{वैशंपायन उवाच}
{}


\twolineshloka
{तत्तु वाक्यमनादृत्य सोऽर्थवन्मातृभाषितम्}
{पुनः प्रतस्थे संरम्भात्सकाशमकृतात्मनाम्}


\twolineshloka
{ततः सभाया निर्गम्य मन्त्रयामास कौरवः}
{सौबलेन मताक्षेण राज्ञा शकुनिना सह}


\threelineshloka
{` दुर्योधनं धार्तराष्ट्रं कर्णं दुःशासनोऽब्रवीति}
{नोचेत्सन्धास्यसे राजन्स्वेन कामेन पाण्डवैः}
{बद्ध्वैव त्वां प्रदास्यन्ति पाण्डुपुत्राय भारत}


\twolineshloka
{वैकर्तनं त्वां च मां च त्रीनेतान्भरतर्षभ}
{पाण्डवेभ्यः प्रदास्यन्ति भीष्मद्रोणौ पिता च ते}


\twolineshloka
{दुःशासनस्य तद्वाक्यं निशम्य भरतर्षभ}
{दुर्योधनो धार्तराष्ट्रो निश्वस्य प्रहसन्निव}


\twolineshloka
{एकान्तमुपसृत्येह मन्त्रं पुनरमन्त्रयत्}
{सौबलेन मताक्षेण राज्ञा शकुनिना सह ॥'}


\twolineshloka
{दुर्योधनस्य कर्णस्य शकुनेः सौबलस्य च}
{दुःशासनचतुर्थानामिदमासीद्विचेष्टितम्}


\twolineshloka
{पुराऽयमस्मान्गृह्णाति क्षिप्रकारी जनार्दनः}
{सहितो धृतराष्ट्रेण राज्ञा शान्तनवेन च}


\twolineshloka
{वयमेव हृषीकेशं निगृह्णीम बलादिव}
{प्रसह्य पुरुषव्याघ्रमिन्द्रो वैरोचनिं यथा}


\twolineshloka
{श्रुत्वा गृहीतं वार्ष्णेयं पाण्डवा हतचेतसः}
{निरुत्साहा भविष्यन्ति भग्नदंष्ट्रा इवोरगाः}


\threelineshloka
{अयं ह्येषां महाबाहुः सर्वेषां शर्म वर्म च}
{अस्मिन्गृहीते वरदे ऋषभे सर्वसात्वताम्}
{निरुद्यमा भविष्यन्ति पाण्डवाः सोमकैः सह}


\threelineshloka
{तस्माद्वयमिहैवैकं केशवं क्षिप्रकारिणम्}
{क्रोशतो धृतराष्ट्रस्य बद्ध्वा योत्स्यामहे रिपून् ॥वैशंपायन उवाच}
{}


\twolineshloka
{तेषां पापमभिप्रायं पापानां दुष्टचेतसाम्}
{इङ्गितज्ञः कविः क्षिप्रमन्वबुद्ध्यत सात्यकिः}


\twolineshloka
{तदर्थमभिनिष्क्रम्य हार्दिक्येन सहास्थितः}
{अब्रवीत्कृतवर्माणं क्षिप्रं योजय वाहिनीम्}


\twolineshloka
{व्यूढानीकः सभाद्वारमुपतिष्ठस्व दंशितः}
{यावदाख्याम्यहं चैतत्कृष्णायाक्लिष्टकारिणे}


\twolineshloka
{स प्रविश्य सभां वीरः सिंहो गिरिगुहामिव}
{आचष्ट तमभिप्रायं केशवाय महात्मने}


\twolineshloka
{धृतराष्ट्रं ततश्चैव विदुरं चान्वभाषत}
{तेषामेतमभिप्रायमाचचक्षे स्मयन्निव}


\twolineshloka
{धर्मादर्थाच्च कामाच्च कर्म साधुविगर्हितम्}
{मन्दाः कर्तुमिहेच्छन्ति न चावाप्यं कथंचन}


\twolineshloka
{पुरा विकुर्वते मूढाः पापात्मानः समागताः}
{धर्षिताः काममन्युभ्यां क्रोधलोभवशानुगाः}


\twolineshloka
{इमं हि पुण्डरीकाक्षं जिघृक्षन्त्यल्पचेतसः}
{पटेनाग्निं प्रज्वलितं यथा बाला यथा जडाः}


\twolineshloka
{सात्यकेस्तद्वचः श्रुत्वा विदुरो दीर्घदर्शिवान्}
{धृतराष्ट्रं महाबाहुमब्रवीत्कुरुसंसदि}


\twolineshloka
{राजन्परीतकालास्ते पुत्राः सर्वे परन्तप}
{असक्यमयशस्यं च कर्तुं कर्म समुद्यताः}


\twolineshloka
{इमं हि पुण्डरीकाक्षमभिभूय प्रसह्य च}
{निग्रहीतुं किलेच्छन्ति सहिता वासवानुजम्}


\twolineshloka
{इमं पुरुषशार्दूलमप्रधृष्यं दुरसदम्}
{आसाद्य नभविष्यन्ति पतङ्गा इव पावकम्}


\twolineshloka
{अयमिच्छन्हि तान्सर्वान्युध्यमानाञ्जनार्दनः}
{सिंहो नागानिव क्रुद्धो गमयेद्यमसादनम्}


\twolineshloka
{न त्वयं निन्दितं कर्म कुर्यात्पापं कथंचन}
{न च धर्मादपक्रामदेच्युतः पुरुषोत्तमः}


\twolineshloka
{` यथा वाराणसी दग्धा साश्वा सरथकुञ्जरा}
{सानुबन्धस्तु कृष्णेन काशीनामृषभो हतः}


\twolineshloka
{तथा नागपुरं दग्ध्वा शङ्खचक्रगदाधरः}
{स्वयं कालेश्वरो भूत्वा नाशयिष्यति कौरवान्}


\twolineshloka
{पारिजातहरं ह्येनमेकं यदुसुखावहम्}
{नाभ्यवर्तत संरब्धो वृत्रहा वसुभिः सह}


\twolineshloka
{प्राप्य निर्मोचने पाशान्षट्सहस्रांस्तरस्विनः}
{हृतास्ते वासुदेवेन ह्युपसङ्क्रम्य कौरवान्}


\twolineshloka
{द्वारमासाद्य सौभस्य विधूय गदया गिरिम्}
{द्युमत्सेनः सहामात्यः कृष्णेन विनिपातितः}


\twolineshloka
{शेषवत्त्वात्कुरूणां तु धर्मापेक्षी तथाऽच्युतः}
{क्षमते पुण्डरीकाक्षः शक्तः सन्पापकर्मणाम्}


\twolineshloka
{एते हि यदि गोविन्दमिच्छन्ति सह राजभिः}
{अद्यैवातिथयः सर्वे भविष्यन्ति यमस्य ते}


\threelineshloka
{यथा वायोस्तृणाग्रणि वशं यान्ति बलीयसः}
{तथा चक्रभृतः सर्वे वशमेष्यन्ति कौरवाः ॥वैशंपायन उवाच}
{'}


\twolineshloka
{विदुरेणैवमुक्ते तु केशवो वाक्यमब्रवीत्}
{धृतराष्ट्रमभिप्रेक्ष्य सुहृदां श्रृण्वतां मिथः}


\twolineshloka
{राजन्नेते यदि क्रुद्धा मां निगृह्णीयुरोजसा}
{एते वा मामहं वैनाननुजानीहि पार्थिव}


\twolineshloka
{एतान्हि सर्वान्संरब्धान्नियन्तुमहमुत्सहे}
{न त्वहं निन्दितं कर्म कुर्यां पापं कथंचन}


\twolineshloka
{पाण्डवार्थे हि लुभ्यन्तः स्वार्थान्हास्यन्ति ते सुताः}
{एते चेदेवमिच्छन्ति कृतकार्यो युधिष्ठिरः}


\twolineshloka
{अद्यैव ह्यहमेनांश्च ये चैनाननु भारत}
{निगृह्य राजन्पार्थेभ्यो दद्यां किं दुष्कृतं भवेत्}


\twolineshloka
{` एष मे निश्चयो राजन्यद्येषोऽस्य विनिश्चयः}
{नर्दन्तु सहिताः शङ्खाः पणवानकनिस्वनैः}


\twolineshloka
{अनायासेन पार्थानां पर्वतां च शिवं महत्}
{निगृह्य राजन्पार्थेभ्यो दद्यां चेत्सुकृतं भवेत् ॥'}


\twolineshloka
{इदं तु न प्रवर्तेयं निन्दितं कर्म भारत}
{सन्निधौ ते महाराज क्रोधजं पापबुद्धिजम्}


\twolineshloka
{एष दुर्योधनो राजन्यथेच्छति तथाऽस्तु तत्}
{अहं तु सर्वांस्तनयाननुजानामि ते नृप}


\twolineshloka
{एतच्छ्रुत्वा तु विदुरं धृतराष्ट्रोऽभ्यभाषत}
{क्षिप्रमानय तं पापं राज्यलुब्धं सुयोधनम्}


\twolineshloka
{सहमित्रं सहामात्यं ससोदर्यं सहानुगम्}
{शक्नुयां यदि पन्थानमवतारयितुं पुनः}


\twolineshloka
{ततो दुर्योधनं क्षत्ता पुनः प्रावेशयत्मभाम्}
{अकामं भ्रातृभिः सार्धं राजभिः परिवारितम्}


\twolineshloka
{अथ दुरोयधनं राजा धृतराष्ट्रोऽभ्यभाषत}
{कर्णदुःशासनाभ्यां च राजभिश्चापि संवृतम्}


\twolineshloka
{नृशंस पापभूयिष्ठ क्षुद्रकर्मसहायवान्}
{पापैः सहायैः संहत्य पापं कर्म चिकीर्षसि}


\twolineshloka
{अशक्यमयशस्यं च सद्भिश्चापि विगर्हितम्}
{यथा त्वादृशको मूढो व्यवस्येत्कुलपांसनः}


\twolineshloka
{त्वमिमं पुण्डरीकाक्षमप्रधृष्यं दुरासदम्}
{पापैः सहायैः संहत्य निग्रहीतुं किलेच्छसि}


\twolineshloka
{यो न शक्यो बलात्कर्तुं देवैरपि सवासवैः}
{तं त्वं प्रार्थयसे मन्द बालश्चन्द्रमसं यथा}


\twolineshloka
{देवैर्मनुष्यैर्गन्धर्वैरसुरैरुरगैश्च यः}
{न सोढुं समरे शक्यस्तं न बुध्यसि केशवम्}


\twolineshloka
{दुर्ग्राह्यः पाणिना वायुर्दुस्पर्शः पाणिना शशी}
{दुर्धरा पृथिवी मूर्ध्ना दुर्ग्राह्यः केशवो बलात्}


\twolineshloka
{इत्युक्ते धृतराष्ट्रेण क्षत्ताऽपि विदुरोऽब्रवीत्}
{दुर्योधनमभिप्रेक्ष्य धार्तराष्ट्रममर्षणम्}


\threelineshloka
{दुर्योधन निबोधेदं वचनं मम सांप्रतम्}
{सौभद्वारे वानरेन्द्रो द्विविदो नाम नामतः}
{शिलावर्षेण महता छादयामास केशवम्}


\twolineshloka
{ग्रहीतुकामो विक्रम्य सर्वयत्नेन माधवम्}
{ग्रहीतुं नाशकच्चैनं तं त्वं प्रार्थयसे बलात्}


\twolineshloka
{प्राग्ज्योतिषगतं शौरिं नरकः सह दानवैः}
{ग्रहीतुं नाशकत्तत्र तं त्वं प्रार्थयसे बलात्}


\twolineshloka
{अनेकयुगवर्षायुर्निहत्य नरकं मृधे}
{नीत्वा कन्यासहस्राणि उपयेमे यथाविधि}


\twolineshloka
{निर्मोचने षट्सहस्राः पाशैर्बद्ध्वा महासुराः}
{ग्रहीतुं नाशकंश्चैनं तं त्वं प्रार्थयसे बलात्}


\twolineshloka
{अनेन हि हता बाल्ये पूतना शिशुना तदा}
{गोवर्धनो धारितश्च गवार्थे भरतर्षभ}


\twolineshloka
{अरिष्टो धेनुकश्चैव चाणूरश्च महाबलः}
{अश्वराजश्च निहतः कंसश्चारिष्टमाचरन्}


\twolineshloka
{जरासन्धश्च वक्रश्च शिशुपालश्च वीर्यवान्}
{बाणाश्च निहतः सङ्ख्ये राजानश्च निषूदिताः}


\twolineshloka
{वरुणो निर्जितो राजा पावकश्चामितौजसा}
{पारिजातं च हरता जितः साक्षाच्छचीपतिः}


\twolineshloka
{एकार्णवे च स्वपता निहतौ मधुकैटभौ}
{जन्मान्तरमुपागम्य हयग्रीवस्तथा हतः}


\twolineshloka
{अयं कर्ता न क्रियते कारणं चापि पौरुषे}
{यद्यदिच्छेदयं शौरिस्तत्तत्कुर्यादयत्नतः}


\twolineshloka
{तं न बुध्यसि गोविन्द घोरविक्रममच्युतम्}
{आशीविषमिव क्रूद्धं तेजोराशिमनिन्दितम्}


\twolineshloka
{प्रधर्षयन्महाबाहुं कृष्णमक्लिष्टकारिणम्}
{पतङ्गोऽग्निमिवासाद्य सामात्यो नभविष्यसि}


\chapter{अध्यायः १३१}
\twolineshloka
{वैशंपायन उवाच}
{}


\twolineshloka
{विदुरेणैवमुक्तस्तु केशवः शत्रुपूगहा}
{दुर्योधनं धार्तराष्ट्रमभ्यभाषत वीर्यवान्}


\twolineshloka
{एकोऽहमिति यन्मोहान्मन्यसे मां सुयोधन}
{परिभूय सुदुर्बुद्धे ग्रहीतुं मां चिकीर्षसि}


\twolineshloka
{इहैव पाण्डवाः सर्वे तथैवान्धकवृष्णयः}
{इहादित्याश्च रुद्राश्च वसवश्च महर्षिभिः}


\threelineshloka
{एवमुक्त्वा जाहासोच्चैः केशवः परवीरहा}
{तस्य संस्मयतः शौरेर्विद्युद्रूपा महात्मनः}
{`युगपच्च विनिष्पेतुः साक्षात्सर्वास्तु देवताः ॥'}


\twolineshloka
{अङ्गुष्ठमात्रास्त्रिदशा बभूवुः पावकार्चिषः}
{तस्य ब्रह्मा ललाटस्थो रुद्रो वक्षसि चाभवत्}


\twolineshloka
{लोकपाला भुजेष्वासन्नग्निरास्यादजायत}
{आदित्याश्चैव साध्याश्च वसवोऽथाश्विनावपि}


\twolineshloka
{मरुतश्च सहेन्द्रेण विश्वेदेवास्तथैव च}
{बभूवुश्चैव यक्षाश्च गन्धर्वोरगराक्षसाः}


\twolineshloka
{प्रादुरास्तां तथा दोर्भ्यां सङ्कर्षणधनञ्जयौ}
{दक्षिणेऽथार्जुनो धन्वी हली रामश्च सव्यतः}


\twolineshloka
{भीमो युधिष्ठिरश्चैव माद्रीपुत्रौ च पृष्ठतः}
{अन्धका वृष्णयश्चैव प्रद्युम्नपरमुखास्ततः}


\twolineshloka
{अग्रे बभूवुः कृष्णस्य समुद्यतमहायुधाः}
{शङ्खचक्रगदाशक्तिशार्ङ्गलाङ्गलनन्दकाः}


\twolineshloka
{अदृश्यन्तोद्यतान्येव सर्वप्रहरणानि च}
{नानाबाहुषु कृष्णस्य दीप्यमानानि सर्वशः}


\twolineshloka
{नेत्राभ्यां नासिकाभ्यां च श्रोत्राभ्यां च समन्ततः}
{प्रादुरासन्महारौद्राः सधूमाः पावकार्चिषः}


\twolineshloka
{रोमकूपेषु च तथा सूर्यस्येव मरीचयः}
{`सहस्रचरणः श्रीमाञ्शतबाहुः सहस्रदृक् ॥'}


\twolineshloka
{तस्य वै नागलोकश्च गुल्फाधो ददृशे तदा}
{चन्द्रसूर्यौ तथा नेत्रे ग्रहः वै सर्वतः स्थिताः}


\twolineshloka
{ऊर्ध्वलोकाश्च सर्वेऽपि कुक्षौ तस्य व्यवस्थिताः}
{सरितः सागराश्चैव स्वेदस्तस्य महात्मनः}


\twolineshloka
{अस्थीनि पर्वताः सर्वे वृक्षा रोमाणि तस्य हि}
{निमेषणं रात्र्यहनी जिह्वायां शारदा तथा ॥ '}


\twolineshloka
{तं दृष्ट्वा घोरमात्मानं केशवस्य महात्मनः}
{न्यमीलयन्त नेत्राणि राजानस्त्रस्तचेतसः}


\twolineshloka
{ऋत द्रोणं च भीष्मं च विदुरं च महामतिम्}
{संजयं धृतराष्ट्रं च ऋषींश्चैव तपोधनान्}


% Check verse!
प्रादात्तेषां स भगवान्दिव्यं चक्षुर्जनार्दनः
\twolineshloka
{`धृतराष्ट्राय प्रददौ भगवान्दिव्यचक्षुषी}
{ददर्श परमं रूपं धृतराष्ट्रोऽम्बिकासुतः}


\threelineshloka
{ततो देवाः सगन्धर्वाः किन्नराश्च महोरगाः}
{ऋषयश्च महाभागा लोकपालैः समन्विताः}
{प्रणम्य शिरसा देवं तुष्टुवुः प्राञ्जलिस्थिताः}


\twolineshloka
{क्रोधं प्रभो संहर संहर स्वंरूपं च यद्दर्शितमात्मसंस्थम्}
{यावत्त्विमे देवगणैः समेतालोकाः समस्ता भुवि नाशमीयुः}


\twolineshloka
{त्वं च कर्ता विकर्ता च त्वमेव परिरक्षसे}
{त्वया व्याप्तमिदं सर्वं जगत्स्थावरजङ्गमम्}


\threelineshloka
{कियन्मात्रा महीपालाः किंवीर्याः किंपराक्रमाः}
{तेषामर्थे महाबाहो दिव्यं रूपं प्रदर्शिवान्}
{एवमुच्चारिता वाचः सह देवार्विभुं तदा ॥'}


\threelineshloka
{तद्दृष्ट्वा महदाश्चर्यं माधवस्य सभातले}
{देवदुन्दुभयो नेदुः पुष्पवर्षं पपात च ॥धृतराष्ट्र उवाच}
{}


\twolineshloka
{त्वमेव पुण्डरीकाक्ष सर्वस्य जगतो हितः}
{तस्मान्मे यादवश्रेष्ठ प्रसादं कर्तुमर्हसि}


\twolineshloka
{भगवन्मम नेत्राभ्यामन्तर्धानं वृणे पुनः}
{भवन्तं दृष्टवानस्मि नान्यं द्रष्टुमिहोत्सहे}


\twolineshloka
{ततोऽब्रवीन्महाबाहुर्धृतराष्ट्रं जनार्दनः}
{अदृश्यमाने नेत्रे द्वे भवेतां कुरुनन्दन}


\twolineshloka
{तत्राद्भुतं महाराज धृतराष्ट्रश्च चक्षुषी}
{लब्धवान्वासुदेवाच्च विश्वरूपदिदृक्षया}


\twolineshloka
{लब्धचक्षुषमासीनं धृतराष्ट्रं नराधिपाः}
{विस्मिता ऋषिभिः सार्धं तुष्टुवुर्मधुसूदनम्}


\twolineshloka
{चचाल च मही कृत्स्ना सागरश्चापि चुक्षुभे}
{विस्मयं परमं जग्मुः पार्थिवा भरतर्षभ}


\twolineshloka
{ततः स पुरुषव्याघ्रः सञ्जहार वपुः स्वकम्}
{तां दिव्यामद्भुतां चित्रामृद्धिमत्तामरिन्दमः}


\twolineshloka
{ततः सात्यकिमादाय पाणौ विदुरमेव च}
{ऋषिभिस्तैरनुज्ञातो निर्ययौ मधुसूदनः}


\twolineshloka
{ऋषयोऽन्तर्हिता जग्मुस्ततस्ते नारदादयः}
{तस्मिन्कोलाहले वृत्ते तदद्भुतमिवाभवत्}


\twolineshloka
{तं प्रस्थितमभिप्रेक्ष्य कौरवाः सह राजभिः}
{अनुजग्मुर्नरव्याघ्रं देवा इव शतक्रतम्}


\twolineshloka
{अचिन्तयन्नमेयात्मा सर्वं तद्राजमण्डलम्}
{निश्चक्राम ततः शौरिः सधूम इव पावकः}


\twolineshloka
{ततो रथेन शुभ्रेण महता किङ्किणीकिना}
{हेमजालविचित्रेण लघुना मेघनादिना}


\threelineshloka
{सूपस्करेण शुभ्रेण वैयाघ्रेण वरूथिना}
{शैब्यसुग्रीवयुक्तेन प्रत्यदृश्यत दारुकः ॥ 5-131-39aतथैवरथमास्थाय कृतवर्मा महारथः}
{वृष्णीनां संमतो वीरो हार्दिक्यः समदृश्यत}


\twolineshloka
{उपस्थितरथं शौरिं प्रयास्यन्तमरिन्दमम्}
{धृतराष्ट्रो महाराजः पुनरेवाभ्यभाषत}


\twolineshloka
{यावद्बलं मे पुत्रेषु पश्यतस्ते जनार्दन}
{प्रत्यक्षं ते न ते किंचित्परोक्षं शत्रुकर्शन}


\twolineshloka
{कुरूणां शममिच्छन्तं यतमानं च केशव}
{विदित्वैतामवस्थां मे नातिशङ्कितुमर्हसि}


\twolineshloka
{न मे पापोऽस्त्यभिप्रायः पाण्डवान्प्रति केशव}
{ज्ञातमेव हितं वाक्यं यन्मयोक्तः सुयोधनः}


\threelineshloka
{जानन्ति कुरवः सर्वे राजानश्चैव पार्थिवाः}
{शमे प्रयतमानं मां सर्वयत्नेन माधव ॥वैशंपायन उवाच}
{}


\twolineshloka
{ततोऽब्रवीन्महाबाहुर्धृतराष्ट्रं जनार्दनः}
{द्रोणं पितामहं भीष्मं क्षत्तारं बाह्लिकं कृपम्}


\twolineshloka
{प्रत्यक्षमेतद्भवतां यद्वृत्तं कुरुसंसदि}
{यथा चाशिष्टवन्मन्दो रोषादद्य समुत्थितः}


\twolineshloka
{वदत्यनीशमात्मानं धृतराष्ट्रो महीपतिः}
{आपृच्छे भवतः सर्वान्गमिष्यामि युधिष्ठिरम्}


\twolineshloka
{आमन्त्र्य प्रस्थितं शौरिं रथस्थं पुरुषर्षभ}
{अनुजग्मुर्महेष्वासाः प्रवीरा भरतर्षभाः}


\twolineshloka
{भीष्मो द्रोणः कृपः क्षत्ता धृतराष्ट्रोऽथ बाह्लिकः}
{अश्वत्थामा विकर्णश्च युयुत्सुश्च महारथः}


\twolineshloka
{ततो रथेन शुभ्रेण महता किङ्किणीकिना}
{कुरूणां पश्यतां द्रष्टुं स्वसारं स पितुर्ययौ}


\chapter{अध्यायः १३२}
\twolineshloka
{वैशंपायन उवाच}
{}


\threelineshloka
{प्रविश्याथ गृहं तस्याश्चरणावभिवाद्य च}
{आचख्यौ तत्समासेन यद्वृत्तं कुरुसंसदि ॥वासुदेव उवाच}
{}


\twolineshloka
{उक्तं बहुविधं वाक्यं ग्रहणीयं सहेतुकम्}
{ऋषिभिश्चैव च मया न चासौ तद्गृहीतवान्}


\threelineshloka
{कालपक्वमिदं सर्वं सुयोधनवशानुगम्}
{` सर्वक्षत्रं क्षणेनैव दह्यते पार्थवह्निना}
{'आपृच्छे भवतीं शीघ्रं प्रयास्ये पाण्डवान्प्रति}


\threelineshloka
{किं वाच्याः पाण्डवेयास्ते भवत्या वचनान्मया}
{तद्ब्रूहि त्वं महाप्रज्ञे शुश्रूषे वचनं तव ॥कुन्त्युवाच}
{}


\twolineshloka
{ब्रूयाः केशव राजानं धर्मात्मानं युधिष्ठिरम्}
{भूयांस्ते हीयते धर्मो मा पुत्रक वृथा कृथाः}


\twolineshloka
{श्रोत्रियस्येव ते राजन्मन्दकस्याविपश्चितः}
{अनुवाकहता बुद्धिर्धर्ममेवैकमीक्षते}


\twolineshloka
{अङ्गावेक्षस्व धर्मं त्वं यथा सृष्टः स्वयंभुवा}
{बाहुभ्यां क्षत्रियाः सृष्टा बाहुवीर्योपजीविनः}


\twolineshloka
{क्रूराय कर्मणे नित्यं प्रजानां परिपालने}
{श्रृणु चात्रोपमामेकां या वृद्धेभ्यः श्रुता मया}


\twolineshloka
{मुचुकुन्दस्य राजर्षेरददत्पृथिवीमिमाम्}
{पुरा वैश्रवणः प्रीतो न चासौ तद्गृहीतवान्}


\twolineshloka
{बाहुवीर्यार्जितं राज्यमश्रीयामिति कामये}
{ततो वैश्रवणः प्रीतो विस्मितः समपद्यत}


\twolineshloka
{मुचुकुन्दस्ततो राजा सोऽन्वशासद्वसुन्धराम्}
{बाहुवीर्यार्जितां सम्यक् क्षत्रधर्ममनुव्रतः}


\twolineshloka
{यं हि धर्मं चरन्तीह प्रजा राज्ञा सुरक्षिताः}
{चतुर्थं तस्य धर्मस्य राजा विन्देन भारत}


\twolineshloka
{राजा चरति चेद्धर्मं देवत्वायैव कल्पते}
{स चेदधर्मं चरति नरकायैव गच्छति}


\twolineshloka
{दण्डनीतिः स्वधर्मेण चातुर्वर्ण्यं नियच्छति}
{प्रयुक्तास्वामिना सम्यगधर्मेभ्यो नियच्छति}


\twolineshloka
{दण्डनीत्यां यदा राजा सम्यक्कार्त्स्न्येन वर्तते}
{तदा कृतयुं नाम कालः श्रेष्ठः प्रवर्तते}


\twolineshloka
{कालो वा कारणं राज्ञो राजा वा कालकारणम्}
{इति ते संशयो मा भूद्राजा कालस्य कारणम्}


\twolineshloka
{राजा कृतयुगस्रष्टा त्रेताया द्वापरस्य च}
{युगस्य च चतुर्थस्य राजा भवति कारणम्}


\twolineshloka
{कृतस्य करणाद्राजा स्वर्गमत्यन्तमश्रुते}
{त्रेतायाः करणाद्राजा स्वर्गं नात्यन्तमश्रुते}


\twolineshloka
{प्रवर्तनाद्द्वापरस्य यथाभागमुपाश्रुते}
{कलेः प्रवर्तनाद्राजा पापमत्यन्तमश्रुते}


\twolineshloka
{ततो वसति दुष्कर्मा नरके शाश्वतीः समाः}
{राजदोषेण हि जगत्स्पृश्यते जगतः स च}


\twolineshloka
{राजधर्मानवेक्षस्व पितृपैतामहोचितान्}
{नैतद्राजर्षिवृत्तं हि यत्र त्वं स्थातुमिच्छसि}


\twolineshloka
{न हि वैक्लब्यसंसृष्ट आनृशंस्ये व्यवस्थितः}
{प्रजापालनसंभूतं फलं किंचन लब्धवान्}


\twolineshloka
{न ह्येतामाशिषं पाण्डुर्न चाहं न पितामहः}
{प्रयुक्तवन्तः पूर्वं ते यया चरसि मेधया}


\twolineshloka
{यज्ञो दानं तपः शौर्यं प्रज्ञा सन्तानमेव च}
{माहात्म्यं बलमोजश्च नित्यमाशंसितं मया}


\twolineshloka
{नित्यं स्वाहा स्वधा नित्यं दद्युर्मानुषदेवताः}
{दीर्घमायुर्धनं पुत्रान्सम्यगाराधिताः शुभाः}


\twolineshloka
{पुत्रेष्वाशासते नित्यं पितरो दैवतानि च}
{दानमध्ययनं यज्ञः प्रजानां परिपालनम्}


\twolineshloka
{एतद्धर्ममधर्मं वा जन्मनैवाभ्यजायथाः}
{ते तु वैद्याः कुले जाता अवृत्त्या तात पीडिताः}


\twolineshloka
{यत्र दानपतिं शूरं क्षुधिताः पृथिवीचराः}
{प्राप्य तुष्टाः प्रतिष्ठन्ते धर्मः कोऽभ्यधिकस्ततः}


\twolineshloka
{दानेनान्यं बलेनान्यं तथा सूनृतयाऽपरम्}
{सर्वतः प्रतिगृह्णीयाद्राज्यं प्राप्येह धार्मिकः}


\twolineshloka
{ब्राह्मणः प्रचरेद्भैक्षं क्षत्रियः परिपालयेत्}
{वैश्यो धनार्जनं कुर्याच्छूद्रः परिचरेच्च तान्}


\twolineshloka
{भैक्षं विप्रतिषिद्धं ते कृषिर्नैवोपपद्यते}
{क्षत्रियोऽसि क्षतात्राता बाहुवीर्योपजीविता}


\twolineshloka
{पित्र्यमंशं महाबाहो निमग्नं पुनरुद्धर}
{साम्ना भेदेन दानेन दण्डेनाथ नयेन वा}


\twolineshloka
{इतो दुःखतरं किं नु यदहं दीनबान्धवा}
{परपिण्डमुदीक्षे वै त्वां सूत्वा मित्रनन्दन}


\twolineshloka
{युध्यस्व राजधर्मेण मा निमञ्जीः पितामहान्}
{मा गमः क्षीणपुण्यस्त्वं सानुजः पापिकां गतिं}


\chapter{अध्यायः १३३}
\twolineshloka
{कुन्त्युवाच}
{}


\twolineshloka
{अत्राप्युदाहरन्तीममितिहासं पुरातनम्}
{विदुलायाश्च संवादं पुत्रस्य च परन्तप}


\twolineshloka
{ततः श्रेयश्च भूयश्च यथावद्वक्तुमर्हसि}
{यशस्विनी मन्युमती कुले जाता विभावरी}


\twolineshloka
{क्षत्रधर्मरता दान्ता विदुला दीर्घदर्शिनी}
{विश्रुता राजसंसत्सु श्रुतवाक्या बहुश्रुता}


\threelineshloka
{विदुला नाम राजन्या जगर्हे पुत्रमौरसम्}
{निर्जितं सिन्धुराजेन शयानं दीनचेतसम् ॥विदुलोवाच}
{}


\twolineshloka
{अनन्दन मया जात द्विषतां हर्षवर्धन}
{न मया त्वं न पित्रा च जातः क्वाभ्यागतोह्यसि}


\twolineshloka
{निर्मन्युश्चाप्यसङ्ख्येयः पुरुषः क्लीबसाधनः}
{यावज्जीवं निराशोऽसि कल्याणाय धुरं वह}


\twolineshloka
{माऽऽत्मानमवमन्यस्व मैनमल्पेन बीभरः}
{मनः कृत्वा सुकल्याणं मा भैस्त्वं प्रतिसंहर}


\twolineshloka
{उत्तिष्ठ हे कापुरुष मा शेष्वैवं पराजितः}
{अमित्रान्नन्दयन्सर्वान्निर्मानो बन्धुशोकदः}


\twolineshloka
{सुपूरा वै कुनदिका सुपूरो मुषिकाञ्जलिः}
{सुसंतोषः कापुरुषः स्वल्पकेनैव तुष्यति}


\twolineshloka
{अप्यहेरारुजन्दंष्ट्रामाश्वेव निधनं व्रज}
{अपि वा संशयं प्राप्य जीवितेऽपि पराक्रमेः}


\twolineshloka
{अप्यरेः श्येनवच्छिद्रं पश्येस्त्वं विपरिक्रमन्}
{विवदन्वाऽथवा तूष्णीं व्योम्नीवापरिशङ्कितः}


\twolineshloka
{त्वमेवं प्रेतवच्छेषे कस्माद्वज्रहतो यथा}
{उत्तिष्ठ हे कापुरुष मा स्वाप्सीः शत्रुनिर्जितः}


\twolineshloka
{माऽस्तं गमस्त्वं कृपणो वि श्रूयस्व स्वकर्मणा}
{मा मध्ये मा जघन्ये त्वं माऽधो भूस्तिष्ठ गर्जितः}


\twolineshloka
{अलातं तिन्दुकस्येव मुहूर्तमपि हि ज्वल}
{मा तुषाग्निरिवानर्चिर्धूमायस्व जिजीविषुः}


\twolineshloka
{मुहूर्तं ज्वलितं श्रेयो न च धूमायितं चिरम्}
{मा ह स्म कस्यचिद्गेहे जनी राज्ञः स्वरीमृदुः}


\twolineshloka
{कृत्वा मानुष्यकं कर्म सृत्वाऽऽजिं यावदुत्तमम्}
{धर्मस्यानृण्यमाप्नोति न चात्मानं विगर्हते}


\twolineshloka
{अलब्ध्वा यदि वा लब्ध्वा नानुशोचति पण्डितः}
{आनन्तर्यं चारभते न प्राणानां धनायते}


\twolineshloka
{उद्भावयस्वं वीर्यं वा तां वा गच्छ ध्रुवां गतिम्}
{धर्मं पुत्राग्रतः कृत्वा किंनिमित्तं हि जीवसि}


\twolineshloka
{इष्टापूर्तं हि ते क्लीब कीर्तिश्च सकला हता}
{विच्छिन्नं भोगमूलं ते किंनिमित्तं हि जीवसि}


\twolineshloka
{शत्रुर्निमञ्जता ग्राह्यो जङ्घायां प्रपतिष्यता}
{विपरिच्छिन्नमूलोऽपि न विषीदेत्कथंचन}


\threelineshloka
{उद्यम्य धुरमुत्कर्षेदाजानेयकतं स्मरन्}
{कुरु सत्वं च मानं च विद्धि पौरुषमात्मनः}
{}


\twolineshloka
{उद्भावय कुलं मग्नं त्वत्कृते स्वयमेव हि}
{यस्य वृत्तं न जल्पन्ति मानवा महदद्भुतम्}


\twolineshloka
{राशिवर्धनमात्रं स नैव स्त्री न पुनः पुमान्}
{दाने तपसि सत्ये च यस्य नोच्चरितं यशः}


\twolineshloka
{विद्यायामर्थलाभे वा मातुरुच्चार एव सः}
{श्रुतेन तपसा वाऽपि श्रिया वा विक्रमेण वा}


\twolineshloka
{जनान्योऽभिभवत्यन्यान्कर्मणा हि स वै पुमान्}
{न त्वेव जाल्मीं कापालीं वृत्तिमेषितुमर्हसि}


\twolineshloka
{नृशंस्यामयशस्यां च दुःखां कापुरुषोचिताम्}
{यमेनमभिनन्देयुरमित्राः पुरुषं कृशम्}


\twolineshloka
{लोकस्य समवज्ञातं निहीनासनवाससम्}
{अहोलाभकरं हीनमल्पजीवनमल्पकम्}


\twolineshloka
{नेदृशं बन्धुमासाद्य बान्धवः सुखमेधते}
{अवृत्त्यैव विपत्स्यामो वयं राष्ट्रात्प्रवासिताः}


\twolineshloka
{सर्वकामरसैर्हीनाः स्थानभ्रष्टा अकिंचनाः}
{अवल्गुकारिण सत्सु कुलवंशस्य नाशनम्}


\twolineshloka
{कलिं पुत्रप्रवादेन सञ्जय त्वामजीजनम्}
{निरमर्षं निरुत्साहं निर्वीर्यमरिनन्दनम्}


\twolineshloka
{मा स्म सीमन्तिनी काचिञ्जनयेत्पुत्रमीदृशम्}
{मा धूमाय ज्वलात्यन्तमाक्रम्य जहि शात्रवान्}


\twolineshloka
{ज्वल मूर्धन्यमित्राणां मुहूर्तमपि वा क्षणम्}
{एतावानेव पुरुषो यदमर्षी यदक्षमी}


\twolineshloka
{क्षमावान्निरमर्षश्च नैव स्त्री न पुनः पुमान्}
{संतोषो वै श्रियं हन्ति तथाऽनुक्रोश एव च}


\twolineshloka
{अनुत्थानभये चोभे निरीहो नाश्रुते महत्}
{एभ्यो निकृतिपापेभ्यः प्रमुञ्चात्मानमात्मना}


\twolineshloka
{आयसं हृदयं कृत्वा मृगयस्व पुनः स्वकम्}
{परं विषहते यस्मात्तस्मात्पुरुष उच्यते}


\twolineshloka
{तमाहुर्व्यर्थनामानं स्त्रीवद्य इह जीवति}
{शूरस्योर्जितसत्वस्य सिंहविक्रान्तचारिणः}


\twolineshloka
{दिष्टभावं गतस्यापि विषये मोदते प्रजा}
{य आत्मनः प्रियमुखे हित्वा मृगयते श्रियम्}


\twolineshloka
{अमात्यानामथो हर्षमादधात्यचिरेण सः ॥पुत्र उवाच}
{}


\fourlineindentedshloka
{किं नु मे मामपश्यन्त्याः पृथिव्या अपि सर्वया}
{किमाभरणकृत्यं ते किं भोगैर्जीवितेन वा}
{मातोवाच}
{}


\threelineshloka
{`पैरर्विहन्यमानस्य जीवितेनापि किं फलम्}
{'निर्मन्युकानां ये लोका द्विषन्तस्तानवाप्नुयुः}
{ये त्वादृतात्मनां लोकाः सुहृदस्तान्व्रजन्तु नः}


\twolineshloka
{भृत्यैर्विहीयमानानां परपिण्डोपजीविनाम्}
{कृपणानामसत्वानां मा वृत्तिमनुवर्तिथाः}


\twolineshloka
{अनु त्वां तात जीवन्तु ब्राह्मणाः सुहृदस्तथा}
{पर्यन्यमिव भूतानि देवा इव शतक्रतुम्}


\twolineshloka
{यमाजीवन्ति पुरुषं सर्वभूतानि संजय}
{पक्वं द्रुममिवासाद्य तस्य जीवितमर्थवत्}


\twolineshloka
{यस्य शूरस्य विक्रान्तैरेधन्ते बान्धवाः सुखम्}
{त्रिदशा इव शक्रस्य साधु तस्येह जीवितम्}


\twolineshloka
{स्वबाहुबलमाश्रित्य योहि जीवति मानवः}
{स लोके लभते कीर्तिं परत्र च शुभां गतिम्}


\chapter{अध्यायः १३४}
\twolineshloka
{विदुलोवाच}
{}


\twolineshloka
{अथैतस्यामवस्थायां पौरुषं हातुमिच्छसि}
{निहीनसेवितं मार्गं गमिष्यस्यचिरादिव}


\twolineshloka
{यो हि तेजो यथाशक्ति न कर्शयति विक्रमात्}
{क्षत्रियो जीविताकाङ्क्षी स्तेन इत्येव तं विदुः}


\twolineshloka
{अर्थवन्त्युपपन्नानि वाक्यानि गुणवन्ति च}
{नैव संप्राप्नुवन्ति त्वां मुमूर्षुमिव भेषजम्}


\twolineshloka
{सन्ति वै सिन्धुराजस्य सन्तुष्टा न तथा जनाः}
{दौर्बल्यादासते मूढा व्यसतौघपतीक्षिणः}


\twolineshloka
{सहायोपचितिं कृत्वा व्यवसाय्य ततस्ततः}
{अनुदुष्येयुरपरे पश्यन्तस्तव पौरुषम्}


\twolineshloka
{तैः कृत्वा सह संघातं गिरिदुर्गालयं चर}
{काले व्यसनमाकाङ्क्षन्नैवायमजरामरः}


\twolineshloka
{सञ्जयो नामतश्च त्वं न च पश्यामि तत्त्वयि}
{अन्वर्थनामा भव मे पुत्र मा व्यर्थनामकः}


\twolineshloka
{सम्यग्दृष्टिर्महाप्राज्ञो बालं त्वां ब्राह्मणोऽब्रवीत्}
{अयं प्राप्य महत्कृच्छ्रं पुनर्वृद्धिं गमिष्यति}


\twolineshloka
{तस्य स्मरन्ती वचनमाशंसे विजयं तव}
{तस्मात्तात ब्रवीमि त्वां वक्ष्यामि च पुनःपुनः}


\twolineshloka
{यस्य ह्यर्थाभिनिर्वृत्तौ भवन्त्याप्यायिताः परे}
{तस्यार्थसिद्धिर्नियता नयेष्वर्थानुसारिणः}


\twolineshloka
{समृद्धिरसमृद्धिर्वा पूर्वेषां मम सञ्जय}
{एवं विद्वान्युद्धमना भव मा प्रत्युपाहरः}


\twolineshloka
{नातः पापीयसीं कांचिदवस्थां शम्बरोऽब्रवीत्}
{यत्र नैवाद्य न प्रातर्भोजनं प्रतिदृश्यते}


\twolineshloka
{पतिपुत्रवधादेतत्परमं दुःखमब्रवीत्}
{दारिद्र्यमिति यत्प्रोक्तं पर्यायमरणं हि तत्}


\twolineshloka
{अहं महाकुले जाता ह्रदाद्ध्रदमिवागता}
{ईश्वरी सर्वकल्याणी भर्त्रा परमपूजिता}


\twolineshloka
{महार्हमाल्याभरणां सुमृष्टाम्बरवाससम्}
{पुरा हृष्टः सुहृद्वर्गो मामपश्यत्सुहृद्गताम्}


\twolineshloka
{यदा मां चैव भार्यां च द्रष्टासि भृशदुर्बलाम्}
{न तदा जीवितेनार्थो भविता तव सञ्जय}


\twolineshloka
{दासकर्मकरान्भृत्यानाचार्यर्त्विक्पुरोहितान्}
{अवृत्त्याऽस्मान्प्रजहतो दृष्ट्वा किं जीवितेन ते}


\twolineshloka
{यदि कृत्यं न पश्यामि तवाद्याहं यथा पुरा}
{श्लाघनीयं यशस्यं च का शान्तिर्हृदयस्य मे}


\twolineshloka
{नेति चेद्ब्राह्मणं ब्रूयां दीर्येत हृदयं मम}
{न ह्यद न च मे भर्ता नेति ब्राह्मणमुक्तवान्}


\twolineshloka
{वयमाश्रयणीयाः स्म न श्रोतारः परस्य च}
{अन्यमासाद्य जीवन्ती परित्यक्ष्यामि जीवितम्}


\twolineshloka
{अपारे भव नः पारमप्लवे भव नः प्लवः}
{कुरुष्व स्थानमस्थाने मृतान्सञ्जीवयस्व नः}


\twolineshloka
{सर्वे ते शत्रवः शक्या न चेञ्जीवितुमर्हसि}
{अथ चेदीदृशीं वृत्तिं क्लीबामभ्युपपद्यसे}


\twolineshloka
{निर्विण्णात्मा हतमना मुञ्चैतां पापजीविकाम्}
{एकशत्रुवधेनैव शूरो गच्छति विश्रुतिम्}


\twolineshloka
{इन्द्रो वृत्रवधेनैव महेन्द्रः समपद्यत}
{माहेन्द्रं च ग्रहं लेभे लोकानां चेश्वरोऽभवत्}


\twolineshloka
{नाम विश्राव्य वै शङ्ख्ये शत्रूनाहूय दंशितान्}
{सेनाग्रं चापि विद्राव्य हत्वा वा पुरुषं वरम्}


\twolineshloka
{यदैव लभते वीरः सुयुद्धेन महद्यशः}
{तदैव प्रव्यथन्तेऽस्य शत्रवो विनमन्ति च}


\twolineshloka
{त्यक्त्वात्मानं रणे दक्षं शूरं कापुरुषा जनाः}
{अवशास्तर्पयन्ति स्म सर्वकामसमृद्धिभिः}


\twolineshloka
{राज्यं चाप्युग्रवि भ्रंशं संशयो जीवितस्य वा}
{न लब्धस्य हि शत्रोर्वै शेषं कुर्वन्ति साधवः}


\twolineshloka
{स्वर्गद्वारोपमं राज्यमथवाप्यमृतोपमम्}
{रुद्धमेकायनं मत्वा पतोल्मुक इवारिषु}


\twolineshloka
{जहि शत्रून्रणे राजन्स्वधर्मनुपालय}
{मा त्वा दृशं सुकृपणं शत्रूणां भयवर्धनम्}


\twolineshloka
{अस्मदीयैश्च शोचिद्भिर्नदद्भिश्च परैर्वृतम्}
{अपि त्वां नानुपश्येयं दीनाद्दीनमिवास्थितम्}


\twolineshloka
{हृप्य सौवीरकन्याभिः श्लाघस्वार्थैर्यथा पुरा}
{मा च सैन्धवकन्यानामवसन्नो वशं गमः}


\twolineshloka
{युवा रूपेण संपन्नो विद्ययाभिजनेन च}
{यत्त्वादृशो विकुर्वीत यशस्वी लोकविश्रुतः}


\twolineshloka
{अधुर्यवच्च वोढव्ये मन्ये मरणमेव तत्}
{यदि त्वामनुपश्यामि परस्य प्रियवादिनम्}


\threelineshloka
{पृष्ठतोऽनुव्रजन्तं वा का शान्तिर्हृदयस्य मे}
{नास्मिञ्जातु कुले जातो गच्छेद्योन्यस्य पृष्ठतः}
{}


\twolineshloka
{न त्वं परस्यानुचरस्तात जीवितुमर्हसि}
{अहं हि क्षत्रहृदयं वेद यत्परिशाश्वतम्}


\twolineshloka
{पूर्वैः पूर्वतरैः प्रोक्तं परैः परतरैरपि}
{शाश्वतं चाव्ययं चैव प्रजापतिविनिर्मितम्}


\twolineshloka
{यो वै कश्चिदिहाजातः क्षत्रियः क्षत्रधर्मवित्}
{भयाद्वृत्तिसमीक्षो वा न नमेदिह कस्यचित्}


\twolineshloka
{उद्यच्छेदेव न नमेदुद्यमो ह्येव पौरुषम्}
{अप्यपर्वणि भज्येत न नमेतेह कस्यचित्}


\twolineshloka
{मातङ्गो मत्त इव च परीयात्स महामनाः}
{ब्राह्मणेभ्यो नमेन्नित्यं धर्मायैव च सञ्जय}


\twolineshloka
{नियच्छन्नितरान्वर्णान्विनिघ्नन्सर्वदुष्कृतः}
{ससहायोऽसहायो वा यावज्जीवं तथा भवेत्}


\chapter{अध्यायः १३५}
\twolineshloka
{पुत्र उवाच}
{}


\twolineshloka
{कृष्णायसस्येव च ते संहत्य हृदयं कृतम्}
{मम मातस्त्वकरुणे वीरप्रज्ञे ह्यमर्षणे}


\twolineshloka
{अहो क्षत्रसमाचारो यत्र मामितरं यथा}
{नियोजयसि युद्धाय परमातेव मां तथा}


\twolineshloka
{ईदृशं वचनं ब्रूयाद्भवती पुत्रमेकजम्}
{किं नु ते मामपश्यन्त्याः पृथिव्या अपि सर्वया}


\threelineshloka
{किमाभरणकृत्येन किं भोगैर्जीवितेन वा}
{मयि वा सङ्गरहते प्रियपुत्रे विशेषतः ॥मातोवाच}
{}


\twolineshloka
{सर्वावस्था हि विदुषां तात धर्मार्थकारणात्}
{तावेवाभिसमीक्ष्याहं सञ्जय त्वामचूजुदम्}


\twolineshloka
{स समीक्ष्य क्रमोपेतो मुख्यः कालोऽयमागतः}
{अस्मिंश्चेदागते काले कार्यं न प्रतिपद्यसे}


\twolineshloka
{असंभावितरूपस्त्वमानृशंस्यं करिष्यसि}
{तं त्वामयशसा स्पृष्टं न ब्रूयां यदि सञ्जय}


\twolineshloka
{खरीवात्सल्यमाहुस्तन्निःसामर्थ्यमहेतुकम्}
{सद्भिर्विगर्हितं मार्गं त्यज मूर्खनिषेवितम्}


\twolineshloka
{अविद्या वै महत्यस्ति यामिमां संश्रिताः प्रजाः}
{तव स्याद्यदि सद्वृत्तं तेन मे त्वं प्रियो भवेः}


\twolineshloka
{धर्मार्थगुणयुक्तेन नेतरेण कथञ्चन}
{दैवमानुषयुक्तेन सद्भिराचरितेन च}


\twolineshloka
{यो ह्येवमविनीतेन रमते पुत्रनप्तृणा}
{अनुत्थानवता चापि दुर्विनीतेन दुर्धिया}


\twolineshloka
{रमते यस्तु पुत्रेण मोघं तस्य प्रजाफलम्}
{अकुर्वन्तो हि कर्माणि कुर्वन्तो निन्दितानि च}


\threelineshloka
{सुखं नैवेह नामुत्र लभन्ते पुरुषाधमाः}
{युद्धाय क्षत्रियः सृष्टः सञ्जयेह जयाय च}
{`क्रूराय कर्मणे नित्यं प्रजानां परिपालने ॥'}


\threelineshloka
{जयन्वा वध्यमानो वा प्राप्नोतीन्द्रसलोकताम्}
{न शक्रभवने पुण्ये दिवि तद्विद्यते सुखम्}
{यदमित्रान्वशे कृत्वा क्षत्रियः सुखमेधते}


\twolineshloka
{मन्युना दह्यमानेह पुरुषेण मनस्विना}
{निकृतेनेह बहुशः शत्रून्प्रति जिगीषया}


\threelineshloka
{आत्मानं वा परित्यज्य शत्रुं वा विनिपात्य च}
{प्राप्यते नेह शान्तिर्हि नित्यमेव तु सञ्जय}
{अतोन्येन प्रकारेण शान्तिरस्य कुतो भवेत्}


\twolineshloka
{इह प्राज्ञो हि पुरुषः स्वल्पमप्रियमिच्छति}
{यस्य स्वल्पं प्रियं लोके ध्रुवं तस्याल्पमप्रियम्}


\threelineshloka
{प्रियाभावाच्च पुरुषो नैव प्राप्नोति शोभनम्}
{ध्रुवं चाभावमभ्येति गत्वा गङ्गेव सागरम् ॥पुत्र उवाच}
{}


\threelineshloka
{नेयं मतिस्त्वया वाच्या मातः पुत्रे विशेषतः}
{कारुण्यमेव पश्य त्वं भूत्वेह जडमूकवत् ॥मातोवाच}
{}


\twolineshloka
{अतो मे भूयसी नन्दिर्यदेवमनुपश्यसि}
{चोद्यं मां चोदयस्येतद्भृशं वै चोदयामि ते}


\threelineshloka
{अथ त्वां पूजयिष्यामि हत्वा वै सर्वसैन्धवान्}
{अहं पश्यामि विजयं कृच्छ्रभाविनमेव ते ॥पुत्र उवाच}
{}


\twolineshloka
{अकोशस्यासहायस्य कुतः सिद्धिर्जयो मम}
{इत्यवस्थां विदित्वैतामात्मनात्मनि दारुणाम्}


\twolineshloka
{राज्याद्भावो निवृत्तो मे त्रिदिवादिव दृष्कृतः}
{ईदृशं भवती कंचिदुपायमनुपश्यति}


\threelineshloka
{तन्मे परिणतप्रज्ञे सम्यक्प्रब्रूहि पृच्छते}
{करिष्यामि हि तत्सर्वं यथावदनुशासनम् ॥मातोवाच}
{}


\threelineshloka
{पुत्र नात्मावमन्तव्यः पूर्वाभिरसमृद्धिभिः}
{अभूत्वा हि भवन्त्यर्था भूत्वा नश्यन्ति चापरे}
{अमर्षेणैव चाप्यर्था नारब्धव्याः सुबालिशैः}


\twolineshloka
{सर्वेषां कर्मणां तात फले नित्यमनित्यता}
{अनित्यमिति जानन्तो न भवन्ति भवन्ति च}


\twolineshloka
{अथ ये नैव कुर्वन्ति नैव जातु भवन्ति ते}
{ऐकगुण्यमनीहायामभावः कर्मणां फलम्}


\twolineshloka
{अथ द्वैगुण्यमीहायां फलं भवति वा नवा}
{यस्य प्रागेव विदिता सर्वार्थानामनित्यता}


\twolineshloka
{नुदेद्वृद्धिसमृद्धी स प्रतिकूले नृपात्मज}
{उत्थातव्यं जागृतव्यं योक्तव्यं भूतिकर्मसु}


\twolineshloka
{भविष्यतीत्येव मनः कृत्वा सततमव्यथैः}
{मङ्गलानि पुरस्कृत्य ब्राह्मणांश्चेश्वरैः सह}


\twolineshloka
{प्राज्ञस्य नृपतेराशु वृद्धिर्भवति पुत्रक}
{अभिवर्तति लक्ष्मीस्तं प्राचीमिव दिवाकरः}


\twolineshloka
{निदर्शनान्युपायांश्च बहून्युद्धर्षणानि च}
{अनुदर्शितरूपोऽसि पश्यामि कुरु पौरुषम्}


\twolineshloka
{पुरुषार्थमभिप्रेतं समाहर्तुमिहार्हसि}
{क्रुद्धान्लुब्धान्परिक्षीणानवलिप्तान्विमानितान्}


\threelineshloka
{स्पर्धिनश्चैव ये केचित्तान्युक्त उपधारय}
{एतेन त्वं प्रकारेण महतो भेत्स्यसे गणान्}
{महावेग इवोद्धूतो मातरिश्वा बलाहकान्}


\twolineshloka
{तेषामग्रप्रदायी स्याः कल्पोत्थायी प्रियंवदः}
{ते त्वां प्रियं करिष्यन्ति पुरोधास्यन्ति च ध्रुवम्}


\twolineshloka
{यदैव शत्रुर्जानीयात्सपत्नं त्यक्तजीवितम्}
{तदैवास्मादुद्विजेत सर्पाद्वेश्मगतादिव}


\twolineshloka
{तं विदित्वा पराक्रान्तं वशे न कुरुते यदि}
{निर्वादैर्निर्वदेदेनमन्ततस्तद्भविष्यति}


\twolineshloka
{निर्वादादास्पदं लब्धा धनवृद्धिर्भविष्यति}
{धनवन्तं हि मित्राणि भजन्ते चाश्रयन्ति च}


\twolineshloka
{स्खलितार्थं पुनस्तानि सन्त्यजन्ति च बान्धवाः}
{नह्यस्मिन्नाश्रयन्ते च जुगुप्सन्ते च तादृशम्}


\twolineshloka
{शत्रुं कृत्वा यः सहायं विश्वासमुपगच्छति}
{स न संभाव्यमेवैतद्यद्राज्यं प्राप्नुयादिति}


\chapter{अध्यायः १३६}
\twolineshloka
{मातोवाच}
{}


\twolineshloka
{नैव राज्ञा दरः कार्यो जातु कस्यांचिदापदि}
{अथ चेदपि दीर्णः स्यान्नैव वर्तेत दीर्णवत्}


\twolineshloka
{दीर्णं हि दृष्ट्वा राजानं सर्वमेवानुदीर्यते}
{राष्ट्रं बलममात्याश्च पृथक्कुर्वन्ति ते मतीः}


\twolineshloka
{शत्रुनेके प्रपद्यन्ते प्रजहत्यपरे पुनः}
{अन्ये तु प्रजिहीर्षन्ति ये पुरस्ताद्विमानिताः}


\twolineshloka
{य एवात्यन्तसुहृदस्त एनं पर्युपासते}
{अशक्तयः स्वस्तिकामा बद्धवत्सा इला इव}


\twolineshloka
{शोचन्तमनुशोचन्ति पतितानिव बान्धवान्}
{अपि ते पूजिताः पूर्वमपि ते सुहृदो मताः}


\twolineshloka
{ये राष्ट्रमभिमन्यन्ते राज्ञो व्यसनमीयुषः}
{मा दीदरस्त्वं सुहृदो मा त्वां दीर्णं प्रहासिषुः}


\twolineshloka
{प्रभावं पौरुषं बुद्धिं जिज्ञासन्त्या मया तव}
{विदधत्या समाश्वासमुक्तं तेजोविवृद्धये}


\twolineshloka
{यदेतत्संविजानासि यदि सम्यग्ब्रवीम्यहम्}
{कृत्वा सौम्यमिवात्मानं जायायोत्तिष्ठ सञ्जय}


\twolineshloka
{अस्ति नः कोशनिचयो महानविदितस्तव}
{तमहं वेद नान्यस्तमुपसंपादयामि ते}


\twolineshloka
{सन्ति नैकशता भूयः सुहृदस्तव सञ्जय}
{सुखदुःखसहा वीर संग्रामादनिवर्तिनः}


\twolineshloka
{तादृशा हि सहाया वै पुरुषस्य बुभूषतः}
{इष्टं जिहीर्षतः किंचित्सचिवाः शत्रुकर्शन}


\threelineshloka
{तस्यास्त्वीदृशकं वाक्यं श्रुत्वापि स्वल्पचेतसः}
{तमस्त्वपागमत्तस्य सुचित्रार्थपदाक्षरम् ॥पुत्र उवाच}
{}


\twolineshloka
{उदके नौरियं धार्या वक्तव्यं प्रवणे मया}
{यस्य मे भवती नेत्री भविष्यद्भूतिदर्शिनी}


\twolineshloka
{अहं हि वचनं त्वत्तः शुश्रूषुरपरापरम्}
{किञ्चित्किञ्चित्प्रतिवदंस्तूष्णीमासं मुहुर्मुहुः}


\threelineshloka
{अतृप्यन्नमृतस्येव कृच्छ्राल्लब्धस्य बान्धवात्}
{उद्यच्छाम्येष शत्रूणां नियमार्थं जायय च ॥कुन्त्युवाच}
{}


\twolineshloka
{सदश्च इव स क्षिप्तः प्रणुन्नो वाक्यसायकैः}
{तच्चकार तथा सर्वं यथावदनुशासनम्}


\twolineshloka
{इदमुद्धर्षणं भीमं तेजोवर्धनमुत्तमम्}
{राजानं श्रावयेन्मन्त्री सीदन्तं शत्रुपीडितम्}


\twolineshloka
{जयो नामेतिहासोऽयं श्रोतव्यो विजिगीषुणा}
{महीं विजयते क्षिप्रं श्रुत्वा शत्रूंश्च मर्दति}


\twolineshloka
{इदं पुंसवनं चैव वीराजननमेव च}
{अभीक्ष्णं गर्भिणी श्रुत्वा ध्रुवं वीरं प्रजायते}


\twolineshloka
{विद्याशूरं तपःशूरं दानशूरं तपस्विनम्}
{ब्राह्मया श्रिया दीप्यमानं साधुवादे च संमतम्}


\twolineshloka
{अर्चिष्मन्तं बलोपेतं महाभागं महारथम्}
{धृतिमन्तमनाधृष्यं जेतारमपराजितम्}


\twolineshloka
{नियन्तारमसाधूनां गोप्तारं धर्मचारिणाम्}
{ईदृशं क्षत्रिया सूते वीरं सत्यपराक्रमम्}


\chapter{अध्यायः १३७}
\twolineshloka
{कुन्त्युवाच}
{}


\twolineshloka
{अर्जुनं केशव ब्रूयास्त्वयि जाते स्म सूतके}
{उपोपविष्टा नारीभिराश्रमे परिवारिता}


\threelineshloka
{अथान्तरिक्षे वागासीद्दिव्यरूपा मनोरमा}
{सहस्राक्षसमः कुन्ति भविष्यत्येष ते सुतः}
{}


\twolineshloka
{एष जेष्यति सङ्ग्रामे कुरून्सर्वान्समागतान्}
{भीमसेनद्वितीयश्च लोकमुद्वर्तयिष्यति}


\twolineshloka
{पुत्रस्ते पृथिवीं जेता यशश्चास्य दिवं स्पृशेत्}
{हत्वा कुरूंश्च सङ्ग्रामे वासुदेवसहायवान्}


\twolineshloka
{पित्र्यमंशं प्रनष्टं च पुनरप्युद्धरिष्यति}
{भ्रातृभिः सहितः श्रीमांस्त्रीन्मेधानाहरिष्यति}


\twolineshloka
{स सत्यसन्धो बीभत्सुः सव्यसाची यथाच्युत}
{तथा त्वमेव जानासि बलवन्तं दुरासदम्}


\twolineshloka
{तथा तदस्तु दाशार्ह यथा वागभ्यभाषत}
{धर्मश्चेदस्ति वार्ष्णेय तथा सत्यं भविष्यति}


\twolineshloka
{त्वं चापि तत्तथा कृष्ण सर्वं संपादयिष्यसि}
{नाहं तदभ्यसूयामि यथा वागभ्यभाषत}


\twolineshloka
{नमो धर्माय महते धर्मो धारयति प्रजाः}
{एतद्धनञ्जयो वाच्यो नित्योद्युक्तो वृकोदरः}


\twolineshloka
{यदर्थं क्षत्रिया सूते तस्य कालोऽयमागतः}
{न हि वैरं समासाद्य सीदन्ति पुरुषर्षभाः}


\twolineshloka
{विदिता ते सदा बुद्धिर्भीमस्य न स शाम्यति}
{यावदन्तं न कुरुते शत्रूणां शत्रुकर्शनः}


\threelineshloka
{तावदेव महापबाहुर्निशासु न सुखं लभेत्}
{सर्वधऱ्मविशेषज्ञां स्नुषां पाण्डोर्महात्मनः}
{व्रूया माधव कल्याणीं कृष्ण कृष्णां यशस्विनीम्}


\twolineshloka
{युक्तमेतन्महाभागे कुले जाते यशस्विनि}
{यन्मे पुत्रेषु सर्वेषु यथावत्त्वमवर्तिथाः}


\twolineshloka
{माद्रीपुत्रौ च वक्तव्यौ क्षत्रधर्मरतावुभौ}
{विक्रमेणार्जितान्भोगान्वृणीतं जीवितादपि}


\twolineshloka
{विक्रमाधिगता ह्यर्थाः क्षत्रधर्मेण जीवतः}
{मनो मनुष्यस्य सदा प्रीणन्ति पुरुषोत्तम}


\twolineshloka
{यच्च वः प्रेक्षमाणानां सर्वधर्मोपचायिनाम्}
{पाञ्चाली परुषाण्युक्तका को नु तत्क्षन्तुमर्हति}


\twolineshloka
{न राज्यहरणं दुःखं द्यूते चापि पराजयः}
{प्रव्राजनं सुतानां वा न मे तद्दुःखकारणम्}


\twolineshloka
{यत्र सा बृहती श्यामा सभायां रुदती तदा}
{अश्रौषीत्परुषा वाचस्तन्मे दुःखतरं महत्}


\twolineshloka
{स्त्रीधर्मिणी वरारोहा क्षत्रधर्मरता सदा}
{नाध्यगच्छत्तदा नाथं कृष्णा नाथवती सती}


\twolineshloka
{तं वै ब्रूहि महाबाहो सर्वशस्त्रभृतां वरम्}
{अर्जुनं पुरुषव्याघ्रं द्रौपद्याः पदवीं चर}


\twolineshloka
{विदितं हि तवात्यन्तं क्रूद्धाविव यमान्तकौ}
{भीमार्जुनौ नयेतां हि देवानपि परां गतिम्}


\twolineshloka
{तयोश्चैतदवज्ञानं यत्सा कृष्णा सभागता}
{दुःशासनश्च यद्भीमं कटुकान्यभ्यभाषत}


\twolineshloka
{पश्यतां कुरुवीराणां तच्च संस्मारयेः पुनः}
{पाण्डवान्कुशलं पृच्छेः सपुत्रान्कृष्णया सह}


\threelineshloka
{मां वै कुशलिनीं ब्रूयास्तेषु भूयो जनार्दन}
{अरिष्टं गच्छ पन्थानं पुत्रान्मे प्रतिपालय ॥वैशंपायन उवाच}
{}


\twolineshloka
{अभिवाद्याथ तां कृष्णः कृत्वा चापि प्रदक्षिणम्}
{निश्चक्राम महाबाहुः सिंहखेलगतिस्ततः}


\twolineshloka
{ततो विसर्जयामास भीष्मादीन्कुरुपुङ्गवान्}
{आरोप्याथ रथे कर्णं प्रायात्सात्यकिना सह}


\twolineshloka
{ततः प्रयाते दाशार्हे कुरवः सङ्गता मिथः}
{जजल्पुर्महदाश्चर्यं केशवे परमाद्भुतम्}


\twolineshloka
{प्रमूढा पृथिवी सर्वा मृत्युपाशवशीकृता}
{दुर्योधनस्य बालिश्यान्नैतदस्तीति चाब्रुवन्}


\twolineshloka
{ततो निर्याय नगरात्प्रययौ पुरुषोत्तमः}
{मन्त्रयामास च तदा कर्णेन सुचिरं सह}


\twolineshloka
{विसर्जयित्वा राधेयं सर्वयादवनन्दनः}
{ततो जवेन महता तूर्णमश्वानचोदयत्}


\twolineshloka
{ते पिबन्त इवाकाशं दारुकेण प्रचोदिताः}
{हया जग्मुर्महावेगा मनोमारुतरंहसः}


\twolineshloka
{ते व्यतीत्य महाध्वानं क्षिप्रं श्येना इवाशुगाः}
{उच्चैर्जग्मुरुपप्लाव्यं शार्ङ्गधन्वानमावहन्}


\chapter{अध्यायः १३८}
\twolineshloka
{वैशंपायन उवाच}
{}


\twolineshloka
{कुन्त्यास्तु वचनं श्रुत्वा भीष्मद्रोणौ महारथौ}
{दुर्योधनमिदं वाक्यमूचतुः शासनातिगम्}


\twolineshloka
{श्रुतं ते पुरुषव्याघ्र कुन्त्याः कृष्णस्य संनिधौ}
{वाक्यमर्थवदत्युग्रमुक्तं धर्म्यमनुत्तमम्}


\twolineshloka
{तत्करिष्यन्ति कौन्तेया वासुदेवस्य संमतम्}
{न हि ते जातु शाम्येरन्नृते राज्येन कौरव}


\twolineshloka
{क्लेशिता हि त्वया पार्था धर्मपाशसितास्तदा}
{सभायां द्रौपदी चैव तैश्च तन्मर्षितं तव}


\twolineshloka
{कृतास्त्रं ह्यर्जुनं प्राप्य भीमं च कृतनिश्चयम्}
{गाण्डीवं चेषुधी चैव रथं च ध्वजमेव च}


\twolineshloka
{नकुलं सहदेवं च बलवीर्यसमन्वितौ}
{सहायं वासुदेवं च न क्षंस्यति युधिष्ठिरः}


\twolineshloka
{प्रत्यक्षं ते महाबाहो यथा पार्थेन धीमता}
{विराटनगरे पूर्वं सर्वे स्म युधि निर्जिताः}


\twolineshloka
{दानवा घोरकर्माणो निवातकवचा युधि}
{रौद्रमस्त्रं समादाय दग्धा वानरकेतुना}


\twolineshloka
{कर्णप्रभृतयश्चेमे त्वं चापि कवची रथी}
{मोक्षितो घोषयात्रायां पर्याप्तं तन्निदर्शनम्}


\twolineshloka
{प्रशाम्य भरतश्रेष्ठ भ्रातृभिः सह पाण्डवैः}
{रक्षेमां पृथिवीं सर्वां मृत्योर्दंष्ट्रान्तरं गताम्}


\twolineshloka
{ज्येष्ठो भ्राता धर्मशीलो वत्सलः श्लक्ष्णवाक्कविः}
{तं गच्छ पुरुषव्याघ्रं व्यपनीयेह किल्बिषम्}


\twolineshloka
{दृष्टश्चेत्त्वं पाण्डवेन व्यपनीतशरासनः}
{प्रशान्तभ्रुकुटिः श्रीमान्कृता शान्तिःकुलस्य नः}


\twolineshloka
{तमभ्येत्य सहामात्यः परिष्वज्य नृपात्मजम्}
{अभिवादय राजानं यथापूर्वमरिन्दम}


\twolineshloka
{अभिवादयमानं त्वां पाणिभ्यां भीमपूर्वजः}
{प्रतिगृह्णातु सौहार्दात्कुन्तीपुत्रो युधिष्ठिरः}


\twolineshloka
{सिंहस्कन्धोरुबाहुस्त्वां वृत्तायतमहाभुजः}
{परिष्वजतु बाहुभ्यां भीमः प्रहरतां वरः}


\twolineshloka
{कम्बुग्रीवो गुडाकेशस्ततस्त्वां पुष्करेक्षणः}
{अभिवादयतां पार्थः कुन्तीपुत्रो धनञ्जयः}


\twolineshloka
{आश्विनेयौ नरव्याघ्रौ रूपेणाप्रतिमौ भुवि}
{तौ च त्वां गुरुवत्प्रोम्णा पूजया प्रत्युदीयताम्}


\twolineshloka
{मुञ्चन्त्वानन्दजाश्रूणि दाशार्हप्रमुखा नृपाः}
{संगच्छ भ्रातृभिः सार्धं मानं संत्यज्य पार्थिव}


\twolineshloka
{प्रशाधि पृथिवीं कृत्स्नां ततस्त्वं भ्रातृभिः सह}
{समालिङ्ग्य च हर्षेण नृपा यान्तु परस्परम्}


\twolineshloka
{अलं युद्धेन राजेन्द्र सुहृदां शृणु वारणम्}
{ध्रुवं विनाशो युद्धे हि क्षत्रियाणां प्रदृश्यते}


\twolineshloka
{ज्योतींषि प्रतिकूलानि दारुणा मृगपक्षिणः}
{उत्पाता विविधा वीर दृश्यन्ते क्षत्रनाशनाः}


\twolineshloka
{विशेषत इहास्माकं निमित्तानि विनाशने}
{उल्काभिर्हि प्रदीप्ताभिर्बाध्यते पृतना तव}


\twolineshloka
{वाहनान्यप्रहृष्टानि रुदन्तीव विशांपते}
{गृध्रास्ते पर्युपासन्ते सैन्यानि च समन्ततः}


\twolineshloka
{नगरं न यथापूर्वं तथा राजनिवेशनम्}
{शिवाश्चाशिवनिर्घोषा दीप्तां सेवन्ति वै दिशं}


\twolineshloka
{कुरु वाक्यं पितुर्मातुरस्माकं च हितैषिणाम्}
{त्वय्यायत्तो महाबाहो शमो व्यायाम एव च}


\twolineshloka
{न चेत्करिष्यसि वचः सुहृदामरिकर्शन}
{तप्स्यसे वाहिनीं दृष्ट्वा पार्थबाणप्रपीडिताम्}


\threelineshloka
{भीमस्य च महानादं नदतः शुष्मिणो रणे}
{श्रुत्वा स्मर्तासि मे वाक्यं गाण्डीवस्य च निःश्वनम्}
{यद्येतदपसव्यं ते वचो मम भविष्यति}


\chapter{अध्यायः १३९}
\twolineshloka
{वैशंपायन उवाच}
{}


\twolineshloka
{एवमुक्तस्तु विमनास्तिर्यग्दृष्टिरधोमुखः}
{संहत्य च भ्रुवोर्मध्यं न किञ्चिद्व्याजहार ह}


\threelineshloka
{तं वै विमनसं दृष्ट्वा संप्रेक्ष्यान्योन्यमन्तिकात्}
{पुनरेवोत्तरं वाक्यमुक्तवन्तौ नरर्षभौ ॥भीष्म उवाच}
{}


\threelineshloka
{शुश्रूषमनसूयं च ब्रह्मण्यं सत्यवादिनम्}
{प्रतियोत्स्यामहे पार्थमतो दुःखतरं नु किम् ॥द्रोण उवाच}
{}


\twolineshloka
{अश्वत्थाम्नि यथा पुत्रे भूयो मम धनञ्जये}
{बहुमानः परो राजन्सन्नतिश्च कपिध्वजे}


\twolineshloka
{तं चेत्पुत्रात्प्रियतमं प्रतियोत्स्ये धनञ्जयम्}
{क्षात्रं धर्ममनुष्ठाय धिगस्तु क्षत्रिजीविकाम्}


\twolineshloka
{यस्य लोके समो नास्ति कश्चिदन्यो धनुर्धरः}
{मत्प्रसादात्स बीभत्सुः श्रेयानन्यैर्धनुर्धरैः}


\twolineshloka
{मित्रध्रुग्दुष्टभावश्च नास्तिकोऽथानृजुः शठः}
{न सत्सु लभते पूजां यज्ञे मूर्ख इवागतः}


\twolineshloka
{वार्यमाणोऽपि पापेभ्यः पापात्मा पापमिच्छति}
{चोद्यमानोऽपि पापेन शुभात्मा शुभमिच्छति}


\twolineshloka
{मिथ्योपचरिता ह्येते वर्तमान ह्यनुप्रिये}
{अहितत्वाय कल्पन्ते दोषा भरतसत्तम}


\twolineshloka
{त्वमुक्तः कुरुवृद्धेन मया च विदुरेण च}
{वासुदेवेन च तथा श्रेयो नैवाभिमन्यसे}


\twolineshloka
{अस्ति मे बलमित्येव सहसा त्वं तितीर्षसि}
{सग्राहनक्रमकरं गङ्गावेगमिवोष्णगे}


\twolineshloka
{वाससैव यथा हि त्वं प्रावृण्वानोऽभिमन्यसे}
{स्रजं त्यक्तामिव प्राप्य लोभाद्यौधिष्ठिरीं श्रियम्}


\twolineshloka
{द्रौपदीसहितं पार्थं सायुधैर्भ्रातृभिर्वृतम्}
{वनस्थमपि राज्यस्थः पाण्डवं को विजेष्यति}


\twolineshloka
{निदेशे यस्य राजानः सर्वे तिष्ठन्ति किङ्कराः}
{तमैलविलमासाद्य धर्मराजो व्यराजत}


\twolineshloka
{कुबेरसदनं प्राप्य ततो रत्नान्यवाप्य च}
{स्फीतमाक्रम्य ते राष्ट्रं राज्यमिच्छन्ति पाण्डवाः}


\twolineshloka
{दत्तं हुतमधीतं च ब्राह्मणास्तर्पिता धनैः}
{आवयोर्गतमायुश्च कृतकृत्यौ च विद्धि नौ}


\twolineshloka
{त्वं तु हित्वा सुखं राज्यं मित्राणि च धनानि च}
{विग्रहं पाण्डवैः कृत्वा महद्व्यसनमाप्स्यसि}


\twolineshloka
{द्रौपदी यस्य चाशास्ते विजयं सत्यवादिनी}
{तपोघोरव्रता देवी कथं जेष्यसि पाण्डवम्}


\twolineshloka
{मन्त्री जनार्दनो यस्य भ्राता यस्य धनञ्जयः}
{सर्वशस्त्रभृतां श्रेष्ठः कथं जेष्यसि पाण्डवम्}


\twolineshloka
{सहाया ब्राह्मणा यस्य धृतिमन्तो जितेन्द्रियाः}
{तमुग्रतपसं वीरं कथं जेष्यसि पाण्डवम्}


\twolineshloka
{पुनरुक्तं च वक्ष्यामि यत्कार्यं भूतिमिच्छता}
{सुहृदा मञ्जमानेषु सुहृत्सु व्यसनार्णवे}


\twolineshloka
{अलं युद्धेन तैर्वीरैः शाम्य त्वं कुरुवृद्धये}
{मा गमः सतुतामात्यःसमित्रश्च यमक्षयम्}


\chapter{अध्यायः १४०}
\twolineshloka
{जनमेजय उवाच}
{}


\twolineshloka
{राजपुत्रैः परिवृतस्तथा भृत्यैश्च सत्तम}
{उपारोप्य रथे कर्णं निर्यातो मधुसूदनः}


\twolineshloka
{किमब्रवीद्रथोपस्थे राधेयं परवीरहा}
{कानि सान्त्वानि गोविन्दः सूतपुत्रे प्रयुक्तवान्}


\threelineshloka
{उद्यन्मेघस्वनः काले यत्कृष्णः कर्णमब्रवीत्}
{मृदु वा यदि वा तीक्ष्णं तन्ममाचक्ष्व सर्वशः ॥वैशंपायन उवाच}
{}


\twolineshloka
{आनुपूर्व्येण वाक्यानि तीक्ष्णानि च मृदूनि च}
{प्रियाणि धर्मयुक्तानि सत्यानि च हितानि च}


\threelineshloka
{हृदयग्रहणीयानि राधेयं मधुसूदनः}
{यान्यब्रवीदमेयात्मा तानि मे श्रृणु भारत ॥वासुदेव उवाच}
{}


\twolineshloka
{उपासितास्ते राधेय ब्राह्मणा वेदपारगाः}
{तत्त्वार्थं परिपृष्टाश्च नियतेनानसूयया}


\twolineshloka
{त्वमेव कर्ण जानासि वेदवादान्सनातनान्}
{त्वमेव धर्मशास्त्रेषु सूक्ष्मेषु परिनिष्ठितः}


\twolineshloka
{कानीनश्च सहोढश्च कन्यायां यश्च जायते}
{वोढारं पितरं तस्य प्राहुः शास्त्रविदो जनाः}


\twolineshloka
{सोऽसि कर्ण तथाजातः पाण्डोः पुत्रोसि धर्मतः}
{निश्चयाद्धर्मशास्त्राणामेहि राजा भविष्यसि}


\twolineshloka
{पितृपक्षे च ते पार्था मातृपक्षे च वृष्णयः}
{द्वौ पक्षावभिजानीहि त्वमेतौ पुरुषर्षभ}


\twolineshloka
{मया सार्धमितो यातमद्य त्वां तात पाण्डवाः}
{अभिजानन्तु कौन्तेयं पूर्वजातं युधिष्ठिरात्}


\twolineshloka
{पादौ तव ग्रहीष्यन्ति भ्रातरः पञ्च पाण्डवाः}
{द्रौपदेयास्तथा पञ्च सौभद्रश्चापराजितः}


\twolineshloka
{राजानो राजपुत्राश्च पाण्डवार्थे समागताः}
{पादौ तव ग्रहीष्यन्ति सर्वे चान्धकवृष्णयः}


\twolineshloka
{हिरण्मयांश्च ते कुम्भान्राजतान्पार्थिवांस्तथा}
{ओषध्यः सर्वबीजानि सर्वरत्नानि वीरुधः}


\twolineshloka
{राजन्या राजकन्याश्चाप्यानयन्त्वाभिषेचनम्}
{षष्ठे त्वां च तथा काले द्रौपद्युपगमिष्यति}


\twolineshloka
{अग्निं जुहोतु वै धौम्यः संशितात्मा द्विजोत्तमः}
{अद्य त्वामभिषिञ्चन्तु चातुर्वैद्या द्विजातयः}


\twolineshloka
{पुरोहितः पाण्डवानां ब्रह्मकर्मण्यवस्थितः}
{तथैव भ्रातरः पञ्च पाण्डवाः पुरुषर्षभाः}


\twolineshloka
{द्रौपदेयास्तथा पञ्च पाञ्चालाश्चेदयस्तथा}
{अहं च त्वाऽभिषेक्ष्यामि राजानं पृथिवीपतिम्}


\threelineshloka
{युवराजोऽस्तु ते राजा धर्मपुत्रो युधिष्ठिरः}
{गृहीत्वा व्यजनं श्वेतं धर्मात्मा संशितव्रतः}
{}


\twolineshloka
{उपान्वारोहतु रथं कुन्तीपुत्रो युधिष्ठिरः}
{छत्रं च ते महाश्वेतं भीमसेनो महाबलः}


\twolineshloka
{अभिषिक्तस्य कौन्तेयो धारयिष्यति मूर्धनि}
{किङ्किणीशतनिर्घोषं वैयाघ्रपरिवारणम्}


\twolineshloka
{रथं श्वेतहयैर्युक्तमर्जुनो वाहयिष्यति}
{अभिमन्युश्च ते नित्यं प्रत्यासन्नो भविष्यति}


\threelineshloka
{नकुलः सहदेवश्च द्रौपदेयाश्च पञ्च ये}
{पञ्चालाश्चानुयास्यन्ति शिखण्डी च महारथः}
{}


\twolineshloka
{अहं च त्वाऽनुयास्यामि सर्वे चान्धकवृष्णयः}
{दाशार्हाः परिवारास्ते दाशार्णाश्च विशांपते}


\twolineshloka
{भुङ्क्ष्व राज्यं महाबाहो भ्रातृभिः सह पाण्डवैः}
{जपैर्होमैश्च संयुक्तो मङ्गलैश्च पृथग्विधैः}


\twolineshloka
{पुरोगमाश्च ते सन्तु द्रविडाः सहकुन्तलैः}
{आन्ध्रास्तालचराश्चैव चूचुपा वेणुपास्तथा}


\twolineshloka
{स्तुवन्तु त्वां च बहुभिः स्तुतिभिः सूतमागधाः}
{विजयं वसुषेणस्य घोषयन्तु च पाण्डवाः}


\twolineshloka
{स त्वं परिवृतः पार्थैर्नक्षत्रैरिव चन्द्रमाः}
{प्रशाधि राज्यं कौन्तेय कुन्तीं च प्रतिनन्दय}


\twolineshloka
{मित्राणि ते प्रहृष्यन्तु व्यथन्तु रिपवस्तथा}
{सौभ्रात्रं चैव तेऽद्यास्तु भ्रातृभिः सह पाण्डवैः}


\chapter{अध्यायः १४१}
\twolineshloka
{कर्ण उवाच}
{}


\twolineshloka
{असंशयं सौहृदान्मे प्रणयाच्चाथ केशव}
{सख्येन चैव वार्ष्णेय श्रेयस्कामतयैव च}


\twolineshloka
{सर्वं चैवाभिजानामि पाण्डोः पुत्रोऽस्मि धर्मतः}
{निश्चयाद्धर्मशास्त्राणां यथा त्वं कृष्ण मन्यसे}


\twolineshloka
{कन्या गर्भं समाधत्त भास्करान्मां जनार्दन}
{आदित्यवचनाच्चैव जातं मां सा व्यसर्जयत्}


\twolineshloka
{सोस्मि कृष्ण तथाजातः पाण्डोः पुत्रोस्मि धर्मतः}
{कुन्त्या त्वहमपाकीर्णो यथा न कुशलं तथा}


\twolineshloka
{सूतो हि मामधिरथो दृष्ट्वैवाभ्यानयद्गृहान्}
{राधायाश्चैव मां प्रादात्सौहार्दान्मधुसूदन}


\twolineshloka
{मत्स्नेहाच्चैव राधायां सद्यः क्षीरमवातरत्}
{सा मे मूत्रं पुरीषं च प्रतिजग्राह माधव}


\twolineshloka
{तस्याः पिण्डव्यपनयं कुर्यादस्मद्विधः कथम्}
{धर्मविद्धर्मशास्त्राणां श्रवणे सततं रतः}


\twolineshloka
{तथा मामभिजानाति सूतश्चाधिरथः सुतम्}
{पितरं चाभिजानामि तमहं सौहृदात्सदा}


\twolineshloka
{स हि मे जातकर्मादि कारयामास माधव}
{शास्त्रदृष्टेन विधिना पुत्रप्रीत्या जनार्दन}


\twolineshloka
{नाम वै वसुषेणेति कारयामास वै द्विजैः}
{भार्याश्चोढा मम प्राप्ते यौवने तत्परिग्रहात्}


\twolineshloka
{तासु पुत्राश्च पौत्राश्च मम जाता जनार्दन}
{तासु मे हृदयं कृष्ण संजातं कामबन्धनम्}


\twolineshloka
{न पृथिव्या सकलया न सुवर्णस्य राशिभिः}
{हर्षाद्भयाद्वा गोविन्द मिथ्या कर्तुं तदुत्सहे}


\twolineshloka
{धृतराष्ट्रकुले कृष्ण दुर्योधनसमाश्रयात्}
{मया त्रयोदशसमा भुक्तं राज्यमकण्टकम्}


\twolineshloka
{इष्टं च बहुभिर्यज्ञैः सहसूतैर्मयाऽसकृत्}
{आवाहाश्च विवाहाश्च सहसूतैर्मया कृताः}


\twolineshloka
{मां च कृष्ण समासाद्य कृतः शस्त्रसमुद्यमः}
{दुर्योधनेन वार्ष्णेय विग्रहश्चापि पाण्डवैः}


\twolineshloka
{तस्माद्रणे द्वैरथे मां प्रत्युद्यातारमच्युत}
{वृतवान्परमं कृष्ण प्रतीपं सव्यसाचिनः}


\twolineshloka
{वधाद्बन्धाद्भयाद्वापि लोभाद्वापि जनार्दन}
{अनृतं नोत्सहे कर्तुं धार्तराष्ट्रस्य धीमतः}


\twolineshloka
{यदि ह्यद्य न गच्छेयं द्वैरथं सव्यसाचिना}
{अकीर्तिः स्याद्धृपीकेश मम पार्थस्य चोभयोः}


\twolineshloka
{असंशयं हितार्थाय ब्रूयास्त्वं मधुसूदन}
{सर्वं च पाण्डवाः कुर्युस्त्वद्वशित्वान्न संशयः}


\twolineshloka
{मन्त्रस्य नियमं कुर्यास्त्वमत्र मधुसूदन}
{एतदत्र हितं मन्ये सर्वं यादवनन्दन}


\twolineshloka
{यदि जानाति मां राजा धर्मात्मा संयतेन्द्रियः}
{कुन्त्याः प्रथमजं पुत्रं न स राज्यं ग्रहीष्यति}


\twolineshloka
{प्राप्य चापि महद्राज्यं तदहं मधुसूदन}
{स्फीतं दुर्योधनायैव संप्रदद्यामरिन्दम}


\twolineshloka
{स एव राजा धर्मात्मा शाश्वतोऽस्तु युधिष्ठिरः}
{नेता यस्य हृषीकेसो योद्धा यस्य धनञ्जयः}


\twolineshloka
{पृथिवी तस्य राष्ट्रं च यस्य भीमो महारथः}
{नकुलः सहदेवश्च द्रौपदेयाश्च माधव}


\twolineshloka
{धृष्टद्युन्मश्च पाञ्चल्यः सात्यकिश्च महारथः}
{उत्तमौजा युधामन्युः सत्यधर्मा च सौमकिः}


\threelineshloka
{चैद्यश्च चेकितानश्च शिखण्डी चापराजितः}
{इन्द्रगोपकवर्णाश्च केकया भ्रातरस्तथा}
{इन्द्रायुधसवर्णश्च कुन्तिभोजो महामनाः}


\twolineshloka
{मातुलो भीमसेनस्य श्येनजिच्च महारथः}
{शङ्खः पुत्रो विराटस्य निधिस्त्वं च जनार्दन}


\twolineshloka
{महानयं कृष्ण कृतः क्षत्रस्य समुदानयः}
{राज्यं प्राप्तमिदं दीप्तं प्रथितं सर्वराजसु}


\twolineshloka
{धार्तराष्ट्रस्य वार्ष्णेय शस्त्रयज्ञो भविष्यति}
{अस्य यज्ञस्य वेत्ता त्वं भविष्यसि जनार्दन}


\twolineshloka
{आध्वर्यवं च ते कृष्ण ऋतावस्मिन्भविष्यति}
{होता चैवात्र बीभत्सुः सन्नद्धः स कपिध्वजः}


\threelineshloka
{गाण्डीवं स्रुक् तथा चाज्यं वीर्यं पुंसां भविष्यति}
{ऐन्द्रं पाशुपतं ब्राह्मं स्थूणाकर्णं च माधव}
{मन्त्रास्तत्र भविष्यन्ति प्रयुक्ताः सव्यसाचिना}


\twolineshloka
{अनुयातश्च पितरमधिको वा पराक्रमे}
{गीतं स्तोत्रं स सौभद्रः सम्यक् तत्र भविष्यति}


\twolineshloka
{उद्गातात्र पुनर्भीमः प्रस्तोता सुमहाबलः}
{विनदन्स नरव्याघ्रो नागानीकान्तकृद्रणे}


\twolineshloka
{स चैव तत्र धर्मात्मा शश्वद्राजा युधिष्ठिरः}
{जपैर्होमैश्च संयुक्तो ब्रह्मत्वं कारयिष्यति}


\twolineshloka
{शङ्खशब्दाः समुरजा भेर्यश्च मधुसूदन}
{उत्कृष्टः सिंहनादश्च सुब्रह्मण्यो भविष्यति}


\threelineshloka
{नकुलः सहदेवश्च माद्रीपुत्रौ यशस्विनौ}
{शामित्रं तौ महावीर्यौ सम्यक् तत्र भविष्यतः}
{`तौ मैत्रावरुणाग्नीध्रौ महावीर्यौ भविष्यतः ॥'}


\twolineshloka
{कल्माषदण्डा गोविन्द विमला रथपङ्क्तयः}
{यूपाः समुपकल्पन्तामस्मिन्यज्ञे जनार्दन}


\twolineshloka
{कर्णिनालीकनाराचा वत्सदन्तोपबृंहणाः}
{तोमराः सोमकलशाः पवित्राणि धनूंषि च}


\twolineshloka
{असयोऽत्र कपालानि पुरोडाशाः शिरांसि च}
{हविस्तु रुधिरं कृष्ण तस्मिन्यज्ञे भविष्यति}


\twolineshloka
{इध्माः परिधयश्चैव शक्तयो विमला गदाः}
{सदस्या द्रोणशिष्याश्च कृपस्य च शरद्वतः}


\twolineshloka
{इषवोऽत्र परिस्तोमा मुक्ता गाण्डीवधन्वना}
{महारथप्रयुक्ताश्च द्रोणद्रौणिप्रचोदिताः}


\twolineshloka
{प्रतिप्रास्थानिकं कर्म सात्यकिस्तु करिष्यति}
{दीक्षितो धार्तराष्ट्रोऽत्र पत्नी चास्य महाचमूः}


\twolineshloka
{घटोत्कचोऽत्र शामित्रं करिष्यति महाबलः}
{अतिरात्रे महाबाहो वितते यज्ञकर्मणि}


\twolineshloka
{दक्षिणा त्वस्य यज्ञस्य धृष्टद्युम्नः प्रतापवान्}
{वैतानिके कर्ममुखे जातो यत्कृष्ण पावकात्}


\twolineshloka
{यदब्रुवमहं कृष्ण कटुकानि स्म पाण्डवान्}
{प्रियार्थं धार्तराष्ट्रस्य तेन तप्ये ह्यकर्मणा}


\twolineshloka
{यदा द्रक्ष्यसि मां कृष्ण निहतं सव्यसाचिना}
{पुनश्चितिस्तदा चास्य यज्ञस्याथ भविष्यति}


\twolineshloka
{दुःशासनस्य रुधिरं यदा पास्यति पाण्डवः}
{आनर्दं नर्दतः सम्यक् तदा सूयं भविष्यति}


\twolineshloka
{यदा द्रोणं च भीष्मं च पाञ्चाल्यौ पातयिष्यतः}
{तदा यज्ञावसानं तद्भविष्यति जनार्दन}


\twolineshloka
{दुर्योधनं यदा हन्ता भीमसेनो महाबलः}
{तदा समाप्स्यते यज्ञो धार्तराष्ट्रस्य माधव}


\twolineshloka
{स्नुषाश्च प्रस्नुषाश्चैव धृतराष्ट्रस्य सङ्गताः}
{हतेश्वरा नष्टपुत्रा हतनाथाश्च केशव}


\twolineshloka
{रुदन्त्यः सहगान्धार्या श्वगृध्रकुरराकुले}
{स यज्ञेऽस्मिन्नवभृथो भविष्यति जनार्दन}


\twolineshloka
{विद्यावृद्धा वयोवृद्धाः क्षत्रियाः क्षत्रियर्षभ}
{वृथा मृत्युं न कुर्वीरंस्त्वत्कृते मधुसूदन}


\twolineshloka
{शस्त्रेण निधनं गच्छेत्समृद्धं क्षत्रमण्डलम्}
{करुक्षेत्रे पुण्यतमे त्रैलोक्यस्यापि केशव}


\twolineshloka
{तदत्र पुण्डरीकाक्ष विधत्स्व यदभीप्सितम्}
{यथा कार्त्स्न्येन वार्ष्णेय क्षत्रं स्वर्गमवाप्नुयात्}


\twolineshloka
{यावत्स्थास्यन्ति गिरयः सरितश्च जनार्दन}
{तावत्कीर्तिभवः शब्दः शाश्वतोयं भविष्यति}


\twolineshloka
{ब्राह्मणाः कथयिष्यन्ति महाभारतमाहवम्}
{समागमेषु वार्ष्णेय क्षत्रियाणां यशोधनम्}


\twolineshloka
{समुपानय कौन्तेयं युद्धाय मम केशव}
{मन्त्रसंवरणं कुर्वन्नित्यमेव परन्तप}


\chapter{अध्यायः १४२}
\twolineshloka
{वैशंपायन उवाच}
{}


\threelineshloka
{कर्णस्य वचनं श्रुत्वा केशवः परवीरहा}
{उवाच प्रहसन्वाक्यं स्मितपूर्वमिदं यथा ॥श्रीभगवानुवाच}
{}


\twolineshloka
{अपि त्वां न लभेत्कर्ण राज्यलम्भोपपादनम्}
{मया दत्तां हि पृथिवीं न प्रशासितुमिच्छसि}


\threelineshloka
{ध्रुवो जयः पाण्डवानामितीदंन संशयः कश्चन विद्यतेऽत्र}
{जयध्वजो दृश्यते पाण्डवस्यसमुच्छ्रितो वानरराज उग्रः}
{}


\twolineshloka
{दिव्या माया विहिता भौमनेनसमुच्छ्रिता इन्द्रकेतुप्रकाशाः}
{दिव्यानि भूतानि जयावहानिदृश्यन्ति चैवात्र भयानकानि}


\twolineshloka
{न सञ्जते शैलवनस्पतिभ्यऊर्ध्वं तिर्यग्योजनमात्ररूपः}
{श्रीमान्ध्वजः कर्ण धनञ्जयस्यसमुच्छ्रितः पावकतुल्यरूपः}


\twolineshloka
{यदा द्रक्ष्यसि सङ्ग्रामे श्वेताश्वं कृष्णसारथिम्}
{ऐन्द्रमस्त्रं विकुर्वाणमुभे चाप्यग्निमारुते}


\twolineshloka
{गाण्डीवस्य च निर्घोषं विस्फूर्जितमिवाशनेः}
{न तदा भविता त्रेता न कृतं द्वापरं न च}


\twolineshloka
{यदा द्रक्ष्यसि सङ्ग्रामे कुन्तीपुत्रं युधिष्ठिरम्}
{जपहोमसमायुक्तं स्वां रक्षन्तं महाचमूम्}


\twolineshloka
{आदित्यमिव दुर्धर्षं तपन्तं शत्रुवाहिनीम्}
{न तदा भविता त्रेता न कृतं द्वापरं न च}


\twolineshloka
{यदा द्रक्ष्यसि सङ्ग्रामे भीमसेनं महाबलम्}
{दुःशासनस्य रुधिरं पीत्वा नृत्यन्तमाहवे}


\twolineshloka
{प्रभिन्नमिव मातङ्गं प्रतिद्विरदघातिनम्}
{न तदा भविता त्रेता न कृतं द्वापरं न च}


\twolineshloka
{यदा द्रक्ष्यसि सङ्ग्रामे द्रोणं शान्तनवं कृपम्}
{सुयोधनंच राजानं सैन्धवं च जयद्रथम्}


\twolineshloka
{युद्धायापततस्तूर्णं वारितान्सव्यसाचिना}
{न तदा भविता त्रेता न कृतं द्वापरं न च}


\twolineshloka
{यदा द्रक्ष्यसि सङ््ग्रामे माद्रीपुत्रौ महाबलौ}
{वाहिनीं धार्तराष्ट्राणां क्षोभयन्तौ गजाविव}


\twolineshloka
{विगाढे शस्त्रसंपाते परवीररथारुजौ}
{न तदा भविता त्रेता न कृतं द्वापरं न च}


\twolineshloka
{ब्रूयाः कर्ण इतो गत्वा द्रोणं शान्तनवं कृपम्}
{सौम्योऽयं वर्तते मासः सुप्रापयवसेन्धनः}


\twolineshloka
{सर्वौषधिवनस्फीतः फलवानल्पमक्षिकः}
{निष्पङ्को रसवत्तोयो नात्युष्णशिशिरः सुखः}


\twolineshloka
{सप्तमाच्चापि दिवसादमावास्या भविष्यति}
{सङ्ग्रामो युज्यतां तस्यां तामाहुः शक्रदेवताम्}


\twolineshloka
{तथा राज्ञो वदेः सर्वान्ये युद्धायाभ्युपागताः}
{यद्वो मनीषितं तद्वै सर्वं संपादयाम्यहम्}


\twolineshloka
{राजानो राजपुत्राश्च दुर्योधनवशानुगाः}
{प्राप्य शस्त्रेण निधनं प्राप्स्यन्ति गतिमुत्तमाम्}


\chapter{अध्यायः १४३}
\twolineshloka
{वैशंपायन उवाच}
{}


\twolineshloka
{केशवस्य तु तद्वाक्यं कर्णः श्रुत्वा हितं शुभम्}
{अब्रवीदभिसंपूज्य कृष्णं तं मधुसूदनम्}


\twolineshloka
{जानन्मां किं महाबाहो संमोहयितुमिच्छसि}
{योयं पृथिव्याः कार्त्स्न्येन विनाशः समुपस्थितः}


\twolineshloka
{निमित्तं तत्र शकुनिरहं दुःशासनस्तथा}
{दुर्योधनश्च नृपतिर्धृतराष्ट्रसुतोऽभवत्}


\twolineshloka
{असंशयमिदं कृष्ण महद्युद्धमुपस्थितम्}
{पाण्डवानां कुरूणां च घोरं रुधिरकर्दमम्}


\twolineshloka
{राजानो राजपुत्राश्च दुर्योधनवशानुगाः}
{रणे शस्त्राग्निना दग्धाः प्राप्स्यन्ति यमसादनम्}


\twolineshloka
{स्वप्ना हि बहवो घोरा दृश्यन्ते मधुसूदन}
{निमित्तानि च घोराणि तथोत्पाताः सुदारुणाः}


\twolineshloka
{पराजयं धार्तराष्ट्रे विजयं च युधिष्ठिरे}
{संसन्त इव वार्ष्णेय विविधा रोमहर्षणाः}


\twolineshloka
{प्राजापत्यं हि नक्षत्रं ग्रहस्तीक्ष्णो महाद्युतिः}
{शनैश्चरः पीडयति पीडयन्प्राणिनोऽधिकम्}


\threelineshloka
{कृत्वा चाङ्गारको वक्रं ज्योष्ठायां मधुसूदन}
{अनुराधां प्रार्थयते मैत्रं संगमयन्निव}
{}


\twolineshloka
{नूनं महद्भयं कृष्ण कुरूणां समुपस्थितम्}
{विशेषेण हि वार्ष्णेय चित्रां पीडयते ग्रहः}


\twolineshloka
{सोमस्य लक्ष्म व्यावृत्तं राहुरर्कमुपैति च}
{दिवश्चोल्काः पतन्त्येताः सनिर्घाताः सकंपनाः}


\twolineshloka
{निष्टनन्ति च मातङ्गा मुञ्चन्त्यश्रूणि वाजिनः}
{पानीयं यवसं चापि नाभिनन्दन्ति माधव}


\twolineshloka
{प्रादुर्भूतेषु चैतेषु भयमाहुरुपास्थितम्}
{निमित्तेषु महाबाहो दारुणं प्राणिनाशनम्}


\twolineshloka
{अल्पे भुक्ते पुरीषं च प्रभूतमिह दृश्यते}
{वाजिनां वारणानां च मनुष्याणां च केशव}


\twolineshloka
{धार्तराष्ट्रस्य सैन्येषु सर्वेषु मधुसूदन}
{पराभवस्य तल्लिङ्गमिति प्राहुर्मनीषिणः}


\twolineshloka
{प्रहृष्टं वाहनं कृष्ण पाण्डवानां प्रचक्षते}
{प्रदक्षिणा मृगाश्चैव तत्तेषां जयलक्षणम्}


\twolineshloka
{अपसव्या मृगाः सर्वे धार्तराष्ट्रस्य केशव}
{वाचश्चाप्यशरीरिण्यस्तकत्पराभवलक्षणम्}


\twolineshloka
{मयूराः पुण्यशकुना हंससारसचातकाः}
{जीवञ्जीवकसङ्घाश्चाप्यनुगच्छन्ति पाण्डवान्}


\twolineshloka
{गृध्राः कङ्का बकाः श्येना यातुधानास्तथा वृकाः}
{मक्षिकाणां च सङ्घाता अनुधावन्ति कौरवान्}


\twolineshloka
{धार्तराष्ट्रस्य सैन्येषु भेरीणां नास्ति निःस्वनः}
{अनाहताः पाण्डवानां नदन्ति पटहाः किल}


\twolineshloka
{उदपानाश्च नर्दन्ति यथा गोवृषभास्तथा}
{धार्तराष्ट्रस्य सैन्येषु तत्पराभवलक्षणम्}


\twolineshloka
{मांसशोणितवर्षं च वृष्टं देवेन माधव}
{तथा गन्धर्वनगरं भानुमत्समुपस्थितम्}


\twolineshloka
{सप्राकारं सपरिखं सवप्रं चारुतोरणम्}
{कृष्णश्च परिघस्तत्र भानुमावृत्य तिष्ठति}


\twolineshloka
{उदयास्तमने संध्ये वेदयन्ती महद्भयम्}
{शिवा च वाशते घोरं तत्पराभवलक्षणम्}


\twolineshloka
{एकपक्षाक्षिचरणाः पक्षिणो मधुसूदन}
{उत्सृजन्ति महद्धोरं तत्पराभवलक्षणम्}


\twolineshloka
{कृष्णग्रीवाश्च शकुना रक्तपादा भयानकाः}
{सन्ध्यामिमुखा यान्ति तत्पराभवलक्षणम्}


\twolineshloka
{ब्राह्मणान्प्रथमं द्वेष्टि गुरूंश्च मधुसूदन}
{भृत्यान्भक्तिमतश्चापि तत्पराभवलक्षणम्}


\threelineshloka
{पूर्वा दिग्लोहिताकारा शस्त्रवर्णा च दक्षिणा}
{आमपात्रप्रतीकाशा पश्चिमा मधुसूदन}
{}


\twolineshloka
{उत्तरा शङ्खवर्णाभा दिशां वर्णा उदाहृताः ॥प्रदीप्ताश्च दिशः सर्वा धार्तराष्ट्रस्य माधव}
{}


\twolineshloka
{महद्भयं वेदयन्ति तस्मिन्नुत्पातदर्शने ॥सहस्रपादं प्रासादं स्वप्नान्ते स्म युधिष्ठिरः}
{}


\twolineshloka
{अधिरोहन्मया दृष्टः सह भ्रातृभिरच्युत ॥श्वेतोष्णीषाश्च दृश्यन्ते सर्वे वै शुक्लवाससः}
{}


\twolineshloka
{आसनानि च शुभ्राणि सर्वेषामुपलक्षये ॥तव चापि मया कृष्ण स्वप्नान्ते रुधिराविला}
{}


\twolineshloka
{आन्त्रेण पृथिवी दृष्टा परिक्षिप्ता जनार्दन ॥अस्थिसञ्चयमारूढश्चामितौजा यधिष्ठिरः}
{}


\twolineshloka
{सुवर्णपात्र्यां संहृष्टो भुक्तवान्घृतपायसम् ॥युधिष्ठिरो मया दृष्टो ग्रसमानो वसुन्धराम्}
{}


\twolineshloka
{उच्चं पर्वतमारूढो भीमकर्मा वृकोदरः}
{गदापाणिर्नरव्याघ्रो ग्रसन्निव महीमिमाम्}


\twolineshloka
{क्षपयिष्यति नः सर्वान्स सुव्यक्तं महारणे}
{विदितं मे हृषीकेश श्रिया परमया ज्वलन्}


\twolineshloka
{पाण्डुरं गजमारूढो गाण्डीवी स धनञ्जयः}
{त्वया सार्धं हृषीकेश श्रिया परमया ज्वलन्}


\twolineshloka
{यूयं सर्वे वधिष्यध्वं तत्र मे नास्ति संशयः}
{पार्थिवान्समरे कृष्ण दुर्योधनपुरोगमान्}


\twolineshloka
{नकुलः सहदेवश्च सात्यकिश्च महारथः}
{शुक्लकेयूरकण्ठत्राः शुक्लमाल्याम्बरावृताः}


\twolineshloka
{अधिरूढा नरव्याघ्रा नरवाहनमुत्तमम्}
{त्रय एते मया दृष्टाः पाण्डुरच्छत्रवाससः}


\twolineshloka
{श्वेतोष्णीषाश्च दृश्यन्ते त्रय एते जनार्दन}
{धार्तराष्ट्रेषु सैन्येषु तान्विजानीहि केशव}


\twolineshloka
{अश्वत्थामा कृपश्चैव कृतवर्मा च सात्वतः}
{रक्तोष्णीषाश्च दृश्यन्ते सर्वे माधव पार्थिवाः}


\twolineshloka
{उष्ट्रप्रयुक्तमारूढौ भीष्मद्रोणौ महारथौ}
{मया सार्धं महाबाहो धार्तराष्ट्रेण वा विभो}


\twolineshloka
{अगस्त्यशास्तां च दिशं प्रयाताः स्म जनार्दन}
{अचिरेणैव कालेन प्राप्स्यामो यमसादनम्}


\threelineshloka
{अहं चान्ये च राजानो यच्च तत्क्षत्रमण्डलम्}
{गाण्डिवाग्निं प्रवेक्ष्याम इति मे नास्ति संशयः ॥कृष्ण उवाच}
{}


\twolineshloka
{अपस्थितविनाशेयं नूनमद्य वसुंधरा}
{यथा हि मे वचः कर्ण नोपैति हृदयं तव}


\threelineshloka
{सर्वेषां तात भूतानां विनाशे प्रत्युपस्थिते}
{अनयो नयसङ्काशो हृदयान्नापसर्पति ॥कर्ण उवाच}
{}


\twolineshloka
{अपि त्वां कृष्ण पश्यामो जीवन्तोऽस्मान्महारणात्}
{समुत्तीर्णा महाबाहो वीरक्षत्रविनाशनात्}


\threelineshloka
{अथवा सङ्गमः कृष्ण स्वर्गे नो भविता ध्रुवम्}
{तत्रेदानीं समेष्यामः पुनः सार्धं त्वयानघ ॥वैशंपायन उवाच}
{}


\twolineshloka
{इत्युक्त्वा माधवं कर्णः परिष्वज्य च पीडितम्}
{विसर्जितः केशवेन रथोपस्थादवातरत्}


\twolineshloka
{ततः स्वरथामस्थाय जाम्बूनदविभूषितम्}
{महात्मा वै निववृते राधेयो दीनमानसः}


\twolineshloka
{ततः शीघ्रतरं प्रायात्केशवः सहसात्यकिः}
{पुनरुच्चारयन्वाणीं याहि याहीति सारथिम्}


\chapter{अध्यायः १४४}
\twolineshloka
{वैशंपायन उवाच}
{}


\twolineshloka
{असिद्धानुनये कृष्णे कुरुभ्यः पाण्डवान्गते}
{अभिगम्य पृथां क्षत्ता शनैः शोचन्निवाब्रवीत्}


\twolineshloka
{जानासि मे जीवपुत्री भावं नित्यमविग्रहे}
{क्रोशतो नच गृह्णीते वचनं मे सुयोधनः}


\twolineshloka
{उपपन्नो ह्यसौ राजा चेदिपाञ्चालकेकयैः}
{भीमार्जुनाभ्यां कृष्णेन युयुधानयमैरपि}


\twolineshloka
{उपप्लाव्ये निविष्टोऽपि धर्ममेव युधिष्ठिरः}
{काङ्क्षते ज्ञातिसौहार्दाद्बलवान्दुर्बलो यथा}


\twolineshloka
{राजा तु धृतराष्ट्रोऽयं वयोवृद्धो न शाम्यति}
{मत्तः पुत्रमदेनैव विधर्मे पथि वर्तते}


\threelineshloka
{जयद्रथस्य कर्णस्य तथा दुःशासनस्य च}
{सौबलस्य च दुर्बुद्ध्या मिथो भेदः प्रपत्स्यते}
{}


\twolineshloka
{अधर्मेम हि धर्मिष्ठं ह्रियते राज्यमीदृशम्}
{येषां तेषामयं धर्मः सानुबन्धो भविष्यति}


\twolineshloka
{क्रियमाणे बलाद्धर्मे कुरुभिः को न संज्वरेत्}
{असाम्ना केशवे याते समुद्योक्ष्यन्ति पाण्डवाः}


\twolineshloka
{ततः कुरूणामनयो भविता वीरनाशनः}
{चिन्तयन्न लभे निद्रामहःसु च निशासु च}


\twolineshloka
{श्रुत्वा तु कुन्ती तद्वाक्यमर्थकामेन भाषितम्}
{सा निःश्वसन्ती दुःखार्ता मनसा विममर्श ह}


\twolineshloka
{धिगस्त्वर्थं यत्कृतेयं सुमहाञ्ज्ञातिसंक्षयः}
{वर्त्स्यते सुहृदां चैव युद्धेऽस्मिन्वै पराभवः}


\twolineshloka
{पाण्डवाश्चेदिपञ्चाला यादवाश्च समागताः}
{भारतैः सह योत्स्यन्ति किं नु दुःखमतःपरम्}


\threelineshloka
{पश्ये दोषं ध्रुवं युद्धे तथाऽयुद्धे पराभवम्}
{अधनस्य मृतं श्रेयो न हि ज्ञातिक्षयो जयः}
{इति मे चिन्तयन्त्या वै हृदि दुःखं प्रवर्तते}


\twolineshloka
{पितामहः शान्तनव आचार्यश्च युधां पतिः}
{कर्णश्च धार्तराष्ट्रार्थं वर्धयन्ति भयं मम}


\threelineshloka
{नाचार्यः कामवाञ्शिष्यैद्रौणोयुद्ध्येत जातुचित्}
{पाण्डवेषु कथं हार्दं कुर्यान्न च पितामहः}
{}


\twolineshloka
{अयं त्वेको वृथादृष्टिर्धार्तराष्ट्रस्य दुर्मतेः}
{मोहानुवर्ती सततं पापो द्वेष्टि च पाण्डवान्}


\twolineshloka
{महत्यनर्थे निर्बन्धी बलवांश्च विशेषतः}
{कर्णः सदा पाण्डवानां तन्मे दहति संप्रति}


\twolineshloka
{आशंसे त्वद्य कर्णस्य मनोऽहं पाण्डवान्प्रति}
{ग्रसादयितुमासाद्य दर्शयन्ती यथातथम्}


\twolineshloka
{तोषितो भगवान्यत्र दुर्वासा मे वरं ददौ}
{आह्वानं मन्त्रसंयुक्तं वसन्त्याः पितृवेश्मनि}


\twolineshloka
{साहमन्तःपुरे राज्ञः कुन्तिभोजपुरस्कृता}
{चिन्तयन्ती बहुविधं हृदयेन विदूयता}


\twolineshloka
{बलाबलं च मन्त्राणां ब्राह्मणस्य च वाग्बलम्}
{स्त्रीभावाद्बालभावाच्च चिन्तयन्ती पुनः पुनः}


\twolineshloka
{धात्र्या विस्रब्धया गुप्ता सखीजनवृता तदा}
{दोषं परिहरन्ती च पितुश्चारित्र्यरक्षिणी}


\twolineshloka
{कथं नु सुकृतं मे स्यान्नापराधवती कथम्}
{भवेयमिति संचिन्त्य ब्राह्मणं तं नमस्य च}


\twolineshloka
{कौतूहलात्तु तं लब्धा बालिश्यादाचरं तदा}
{कन्या सती देवमर्कमासादयमहं ततः}


\twolineshloka
{योऽसौ कानीनगर्भो मे पुत्रवत्परिरक्षितः}
{कस्मान्न कुर्याद्वचनं पथ्यं भ्रातृहितं तथा}


\twolineshloka
{इति कुन्ती विनिश्चित्य कार्यनिश्चयमुत्तमम्}
{कार्यार्थमभिनिश्चित्य ययौ भागीरथीं प्रति}


\twolineshloka
{आत्मजस्य ततस्तस्य घृणिनः सत्यसङ्गिनः}
{गङ्गातीरे पृथापश्यञ्जपस्थानमनुत्तमम्}


\twolineshloka
{प्राङ्मुखस्योर्ध्वबाहोः सा पर्यतिष्ठत पृष्ठतः}
{जप्यावसानं कार्यार्थं प्रतीक्षन्ती तपस्विनी}


\twolineshloka
{अतिष्ठत्सूर्यतापार्ता कर्णस्योत्तरवाससि}
{कौरव्यपत्नी वार्ष्णेयी मद्ममालेव शुष्यती}


\twolineshloka
{आपृष्ठतापाञ्जप्त्वा स परिवृत्त्य यतव्रतः}
{दृष्ट्वा कुन्तीमुपातिष्ठदभिवाद्य कृताञ्जलिः}


\twolineshloka
{यथान्यायं महातेजा मानी धर्मभृतां वरः}
{उत्स्मयन्प्रणतः प्राह कुन्तीं वैकर्तनो वृषः}


\chapter{अध्यायः १४५}
\twolineshloka
{कर्ण उवाच}
{}


\threelineshloka
{राधेयोऽहमाधिरथिः कर्णस्त्वामभिवादये}
{प्राप्ता किमर्थं भवती ब्रूहि किं करवामि ते ॥कुन्त्युवाच}
{}


\twolineshloka
{कौन्तेयस्त्वं न राधेयो न तवाधिरथः पिता}
{नासि सूतकुले जातः कर्ण तद्विद्धि मे वचः}


\twolineshloka
{कानीनस्त्वं मया जातः पूर्वजः कुक्षिणा धृतः}
{कुन्तिराजस्य भवने पार्थस्त्वमसि पुत्रक}


\twolineshloka
{प्रकाशकर्मा तपनो योऽयं देवो विरोचनः}
{अजीजनत्त्वां मय्येष कर्ण शस्त्रभृतां वरम्}


\twolineshloka
{कुण्डली बद्धकवचो देवगर्भः श्रिया वृतः}
{जातस्त्वमसि दुर्धर्ष मया पुत्र पितुर्गृहे}


\twolineshloka
{स त्वं भ्रातॄनसंबुद्ध्य मोहद्यदुपसेवसे}
{धार्तराष्ट्रान्न तद्युक्तं त्वयि पुत्र विशेषतः}


\twolineshloka
{एतद्धर्मफलं पुत्र नराणां धर्मनिश्चये}
{यत्तुष्यन्त्यस्य पितरो माता चाप्येकदर्शिनी}


\twolineshloka
{अर्जुनेनार्जितां पूर्वं हृतां लोभादसाधुभिः}
{आच्छिद्य धार्तराष्ट्रेभ्यो भुङ्क्ष यौधिष्ठिरीं श्रियम्}


\threelineshloka
{अद्य पश्यन्तु कुरवः कर्णार्जुनसमागमम्}
{सौभ्रात्रेण समालक्ष्य संनमन्तामसाधवः}
{}


\twolineshloka
{कर्णार्जुनौ वै भवतां यथा रामजनार्दनौ}
{असाध्यं किं नु लोके स्याद्युवयोः संहितात्मनोः}


\twolineshloka
{कर्ण शोभिष्यसे नूनं पञ्चमिर्भ्रातृभिर्वृतः}
{देवैः परिवृतो ब्रह्मा वेद्यामिव महाध्वरे}


\twolineshloka
{उपपद्यो गुणैः सर्वैर्ज्येष्ठः श्रेष्ठेषु बन्धुषु}
{सूतपुत्रेति मा शब्दः पार्थस्त्वमसि वीर्यवान्}


\chapter{अध्यायः १४६}
\twolineshloka
{वैशंपायन उवाच}
{}


\threelineshloka
{ततः सूर्यान्निश्चरितां कर्णः शुश्राव भारतीम्}
{दुरत्ययां प्रणयिनीं पितृवद्भास्करेरिताम् ॥सूर्य उवाच}
{}


\threelineshloka
{सत्यमाह पृथा वाक्यं कर्ण मातृवचः कुरु}
{श्रेयस्ते स्यान्नरव्याघ्र सर्वमाचरतस्तथा ॥वैशंपायन उवाच}
{}


\threelineshloka
{एवमुक्तस्य मात्रा च स्वयं पित्रा च भानुना}
{चचाल नैव कर्णस्य मतिः सत्यधृतेस्तदा ॥कर्ण उवाच}
{}


\twolineshloka
{न चैतच्छ्रद्दधे वाक्यं क्षत्रिये भाषितं त्वया}
{धर्मद्वारं ममैतत्स्यान्नियोगकरणं तव}


\twolineshloka
{अकरोन्मयि यत्पापं भवती सुमहात्ययम्}
{अपाकीर्णोऽस्मि यन्मातस्तद्यशःकीर्तिनाशनम्}


\twolineshloka
{अहं चेत्क्षत्रियो जातो न प्राप्तः क्षत्रसत्क्रियाम्}
{त्वत्कृते किं नु पापीयः शत्रुः कुर्यान्ममाहितम्}


\twolineshloka
{क्रियाकाले त्वनुक्रोशमकृत्वा त्वमिमं मम}
{हीनसंस्क्रासमयमद्य मां समचूचुदः}


\twolineshloka
{न वै मम हितं पूर्वं मातृवच्चेष्टितं त्वया}
{सा मां संबोधयस्यद्य केवलात्महितैषिणी}


\twolineshloka
{कृष्णेन सहितात्को वै न व्यथेत धनंजयात्}
{कोद्य भीतं न मां विद्यात्पर्थानां समितिं गतम्}


\twolineshloka
{अभ्राता विदितः पूर्वं युद्धकाले प्रकाशितः}
{पाण्डवान्यदि गच्छामि किं मां क्षत्रं वदिष्यति}


\twolineshloka
{सर्वकामैः संविभक्तः पूजितश्च यथासुखम्}
{अहं वै धार्तराष्ट्राणां कुर्यां तदफलं कथम्}


\twolineshloka
{उपनह्य परैर्वैरं ये मां नित्यमुपासते}
{नमस्कुर्वन्ति च सदा वसवो वासवं यथा}


\twolineshloka
{मम प्राणेन ये शत्रूञ्शक्ताः प्रतिसमासितुम्}
{मन्यन्ते ते कथं तेषामहं छिन्द्यां मनोरथम्}


\twolineshloka
{मया प्लवेन संग्रामं तितीर्षन्ति दुरत्ययम्}
{अपारे पारकामा ये त्यजेयं तानहं कथम्}


\twolineshloka
{अयं हि कालः संप्राप्तो धार्तराष्ट्रोपजीविनाम्}
{निर्वेष्टव्यं मया तत्र प्राणानपरिरक्षता}


\twolineshloka
{कृतार्थाः सुभृता ये हि कृत्यकाले ह्युपस्थिते}
{अनवेक्ष्य कृतं पापा विकुर्वन्त्यनवस्थिताः}


\threelineshloka
{राजकिल्बिषिणां तेषां भर्तृपिण्डापहारिणाम्}
{नैवायं न परो लोको विद्यते पापकर्मणाम्}
{}


\twolineshloka
{धृतराष्ट्रस्य पुत्राणामर्थे योत्स्यामि ते सुतैः}
{बलं च शक्तिं चास्थाय न वै त्वय्यनृतं वदे}


\twolineshloka
{आनृशंस्यमथो वृत्तं रक्षन्सत्पुरुषोचितम्}
{अतोऽर्थकरमप्येतन्न करोम्यद्य ते वचः}


\twolineshloka
{न च तेऽयं समारम्भो मयि मोघो भविष्यति}
{वध्यान्विषह्यान्संग्रामे न हनिष्यामि ते सुतान्}


\twolineshloka
{युधिष्ठिरं च भीमं च यमौ चैवार्जुनादृते}
{अर्जुनेन समं युद्धमपि यौधिष्ठिरे बले}


\twolineshloka
{अर्जुनं हि निहत्याजौ संप्राप्तं स्यात्फलं मया}
{यशसा चापि युज्येयं निहतः सव्यसाचिना}


\twolineshloka
{न ते जातु नशिष्यन्ति पुत्राः पञ्च यशस्विनि}
{निरर्जुनाः सकर्णा वा सार्जुना वा हते मयि}


\twolineshloka
{इति कर्णवचः श्रुत्वा कुन्ती दुःखात्प्रवेषती}
{उवाच पुत्रमाश्लिष्य कर्णं धैर्यादकम्पनम्}


\twolineshloka
{एवं वै भाव्यमेतेन क्षयं यास्यन्ति कौरवाः}
{यथा त्वं भाषसे कर्ण दैवं तु बलवत्तरम्}


\twolineshloka
{त्वया चतुर्णां भ्रातॄणामभयं शत्रुकर्शन}
{दत्तं तत्प्रतिजानीहि सङ्गरप्रतिमोचनम्}


\twolineshloka
{अनामयं स्वस्ति चेति पृथाथो कर्णमब्रवीत्}
{तां कर्णोऽभ्यवदत्प्रीतस्ततस्तौ जग्मतुः पृथक्}


\chapter{अध्यायः १४७}
\twolineshloka
{वैशंपायन उवाच}
{}


\twolineshloka
{आगम्य हास्तिनपुरादुपप्लाव्यमरिन्दमः}
{पाण्डवानां यथावृत्तं केशवः सर्वमुक्तवान्}


\twolineshloka
{संभाष्य सुचिरं कालं मन्त्रयित्वा पुनः पुनः}
{स्वमेव भवनं शौरिर्विश्रमार्थं जगाम ह}


\twolineshloka
{विसृज्य सर्वान्नृपतीन्विराटप्रमुखांस्तदा}
{पाण्डवा भ्रातरः पञ्च भानावस्तंगते सति}


\threelineshloka
{सन्ध्यामुपास्य ध्यायन्तस्तमेव गतमानसाः}
{आनाय्य कृष्णं दाशार्हं पुनर्मन्त्रममन्त्रयन् ॥युधिष्ठिर उवाच}
{}


\threelineshloka
{त्वया नागपुरं गत्वा सभायां धृतराष्ट्रजः}
{किमुक्तः पुण्डरीकाक्ष तन्नः शंसितुमर्हसि ॥वासुदेव उवाच}
{}


% Check verse!
मया नागपुरं गत्वा सभायां धृतराष्ट्रजः

तथ्यं पथ्यं हितं चोक्तो न च गृह्णाति दुर्मतिः

युधिष्ठिर उवाच

5-147-7aतस्मिन्नुत्पथमापन्ने कुरुवृद्धः पितामहः

किमुक्तवान्हृषीकेश दुर्योधनममर्षणम्
\twolineshloka
{आचार्यो वा महाभाग भारद्वाजः किमब्रवीत्}
{पितरौ धृतराष्ट्रस्तं गान्धारी वा किमब्रवीत्}


\twolineshloka
{पिता यवीयानस्माकं क्षत्ता धर्मविदां वरः}
{पुत्रशोकाभिसन्तप्तः किमाह धृतराष्ट्रजम्}


\twolineshloka
{किंच सर्वे नृपतयः सभायां ये समासते}
{उक्तवन्तो यथातत्त्वं तद्ब्रूहि त्वं जनार्दन}


\twolineshloka
{उक्तवान्हि भवान्सर्वं वचनं कुरुमुख्ययोः}
{धार्तराष्ट्रस्य तेषां हि वचनं कुरुसंसदि}


\twolineshloka
{कामलोभाभिभूतस्य मन्दस्य प्राज्ञमानिनः}
{अप्रियं हृदये मह्यं तन्न तिष्ठति केशव}


\fourlineindentedshloka
{तेषां वाक्यानि गोविन्द श्रोतुमिच्छाम्यहं विभो}
{यथा च नाभिपद्येत कालस्तात तथा कुरु}
{भवान्हि नो गतिः कृष्ण भवान्नाथो भवान्गुरुः ॥वासुदेव उवाच}
{}


\twolineshloka
{श्रृणु राजन्यथा वाक्यमुक्तो राजा सुयोधनः}
{मध्ये कुरूणां राजेन्द्र सभायां तन्निबोध मे}


\twolineshloka
{मया विश्राविते वाक्ये जहास धृतराष्ट्रजः}
{अथ भीष्मः सुसंक्रुद्ध इदं वचनमब्रवीत्}


\twolineshloka
{दुर्योधन निबोधेदं कुलार्थे यद्ब्रवीमि ते}
{तच्छ्रुत्वा राजशार्दूल स्वकुलस्य हितं कुरु}


\twolineshloka
{मम तात पिता राजञ्शन्तनुर्लोकविश्रुतः}
{तस्याहमेक एवासं पुत्रः पुत्रवतां वरः}


\twolineshloka
{तस्य बुद्धिः समुत्पन्ना द्वितीयः स्यात्कथं सुतः}
{एकपुत्रमपुत्रं वै प्रवदन्ति मनीषिणः}


\twolineshloka
{न चोच्छेदं कुलं यायाद्विस्तीर्येच्च कथं यशः}
{तस्याहमीप्सितं बुद्ध्वा कालीं मातरमावहम्}


\threelineshloka
{प्रतिज्ञां दुष्करां कृत्वा पितुरर्थे कुलस्य च}
{अराजा चोर्ध्वरेताश्च यथा सुविदितं तव}
{प्रतीतो निवसाम्येष प्रतिज्ञामनुपालयन्}


\twolineshloka
{तस्यां जज्ञे महाबाहुः श्रीमान्कुरुकुलोद्वहः}
{विचित्रवीर्यो धर्मात्मा कनीयान्मम पार्थिव}


\twolineshloka
{स्वर्यातेऽहं पितरि तं स्वराज्ये सन्न्यवेशयम्}
{विचित्रवीर्यं राजानं भृत्यो भूत्वा ह्यधश्चरः}


\twolineshloka
{तस्याहं सदृशान्दारान्राजेन्द्र समुपाहरम्}
{जित्वा पार्थिवसङ्घातमपि ते बहुशः श्रुतम्}


\twolineshloka
{ततो रामेण समरे द्वन्द्वयुद्धमुपागमम्}
{स हि रामभयादेभिर्नागरैर्विप्रवासितः}


\fourlineindentedshloka
{दारेष्वप्यतिसक्तश्च यक्ष्माणं समपद्यत}
{यदा त्वराजके राष्ट्रे न ववर्ष सुरेश्वरः}
{तदाभ्यधावन्मामेव प्रजाः क्षुद्भयपीडिताः ॥प्रजा ऊचुः}
{}


\twolineshloka
{उपक्षीणाः प्रजाः सर्वा राजा भव भवाय नः}
{ईतीः प्रणुद भद्रं शन्तनोः कुलवर्धन}


\twolineshloka
{पीड्यन्ते ते प्रजाः सर्वा व्याधिभिर्भृशदारुणैः}
{अल्पावशिष्टा गाङ्गेय ताः परित्रातुमर्हसि}


\threelineshloka
{व्याधीन्प्रणुद्य वीर त्वं प्रजा धर्मेण पालय}
{त्वयि जिवती मा राष्ट्रं विनाशमुपगच्छतु ॥भीष्म उवाच}
{}


\threelineshloka
{प्रजानां क्रोशतीनां वै नैवाक्षुभ्यत मे मनः}
{प्रतिज्ञां रक्षमाणस्य तद्वृत्तं स्मरतस्तथा}
{ततः पौरा महाराज माता काली च मे शुभा}


\twolineshloka
{भृत्याः पुरोहिताचार्या ब्राह्मणाश्च बहुश्रुताः}
{मामूचुर्भृशसंतप्ता भव राजेति सन्ततम्}


\twolineshloka
{प्रतीपरक्षितं राष्ट्रं त्वां प्राप्य विनशिष्यति}
{स त्वमस्मद्धितार्थं वै राजा भव महामते}


\twolineshloka
{इत्युक्तः प्राञ्जलिर्भूत्वा दुःखितो भृशमातुरः}
{तेभ्यो न्यवेदयं तत्र प्रतिज्ञां पितृगोरवात्}


\twolineshloka
{ऊर्ध्वरेता ह्यराजा च कुलस्यार्थे पुनः पुनः}
{विशेषतस्त्वदर्थं च धुरि मा मां नियोजय}


\twolineshloka
{ततोऽहं प्राञ्जलिर्भूत्वा मातरं संप्रसादयम्}
{नाम्ब शन्तनुना जातः कौरवं वंशमुद्वहन्}


\twolineshloka
{प्रतिज्ञां वितथां कुर्यामिति राजन्पुनः पुनः}
{विशेषतस्त्वदर्थं च प्रतिज्ञां कृतवानहम्}


\twolineshloka
{अहं प्रेष्यश्च दासश्च तवाद्य सुतवत्सले}
{एवं तामनुनीयाहं मातरं जनसन्निधौ}


\threelineshloka
{अयाचं भ्रतृदारेषु तदा व्यासं महामुनिम्}
{सह मात्रा महाराज प्रसाद्य तमृषिं महामुनिम्}
{}


\twolineshloka
{अपत्यार्थं महाराज प्रसादं कृतवांश्च सः}
{त्रीन्स पुत्रानजनयत्तदा भरतसत्तम}


\twolineshloka
{अन्धः करणहीनत्वान्न वै राजा पिता तव}
{राजा तु पाण्डुरभवन्महात्मा लोकविश्रुतः}


\twolineshloka
{स राजा तस्य ते पुत्राः पितुर्दायाद्यहारिणः}
{मा तात कलहं कार्षी राज्यस्यार्धं प्रदीयताम्}


\twolineshloka
{मयि जीवति राज्यं कः संप्रशासेत्पुमानिह}
{मावमंस्या वचो मह्यं शममिच्छामि वः सदा}


\twolineshloka
{न विशेषोऽस्ति मे पुत्र त्वयि तेषु च षार्थिव}
{मतमेतत्पितुस्तुभ्यं गान्धार्या विदुरस्य च}


\twolineshloka
{श्रोतव्यं खलु वृद्धानां नाभिशङ्कीर्वचो मम}
{नाशयिष्यसि मा सर्वमात्मानं पृथिवीं तथा}


\chapter{अध्यायः १४८}
\twolineshloka
{वासुदेव उवाच}
{}


\twolineshloka
{भीष्मेणोक्ते ततो द्रोणो दुर्योधनमभाषत}
{मध्ये नृपाणां भद्रं ते वचनं वचनक्षमः}


\twolineshloka
{प्रातीपः शन्तनुस्तात कुलस्यार्थे यथा स्थितः}
{यथा देवव्रतो भीष्मः कुलस्यार्थे स्थितोऽभवत्}


\twolineshloka
{तथा पाण्डुर्नरपतिः सत्यसन्धो जितेन्द्रियः}
{राजा कुरूणां धर्मात्मा सुव्रतः सुसमाहिताः}


\threelineshloka
{ज्येष्ठाय राज्यमददद्धृतराष्ट्राय धीमते}
{यवीयसे तथा क्षत्रे कुरूणां वंशवर्धनः}
{}


% Check verse!
ततः सिंहासने राजन्स्थापयित्वैनमच्युतम् ॥वनं जगाम कौरव्यो भार्याभ्यां सहितो नृपः
\twolineshloka
{नीचैः स्थित्वा तु विदुर उपास्ते स्म विनीतवत्}
{प्रेष्यवत्पुरुषव्याघ्रो वालव्यजनमुत्क्षिपन्}


\twolineshloka
{ततः सर्वाः प्रजास्तात धृतराष्ट्रं जनेश्वरम्}
{अन्वपद्यन्त विधिवद्यथा पाण्डुं जनाधिपम्}


\twolineshloka
{विसृज्य धृतराष्ट्राय राज्यं स विदुराय च}
{चचार पृथिवीं पाण्डुः सर्वां परपुरञ्जयः}


\twolineshloka
{कोशसंवनने दाने भृत्यानां चान्ववेक्षणे}
{भरणे चैव सर्वस्य विदुरः सत्यसङ्गरः}


\twolineshloka
{सन्धिविग्रहसंयुक्तो राज्ञां संवाहनक्रियाः}
{अवैक्षत महातेजा भीष्मः परपुरञ्जयः}


\twolineshloka
{सिंहासनस्थो नृपतिर्धृतराष्ट्रो महाबलः}
{अन्वास्यमानः सततं विदुरेण महात्मना}


\twolineshloka
{कथं तस्य कुले जातः कुलभेदं व्यवस्यसि}
{संभूय भ्रातृभिः सार्धं भुङ्क्ष भोगाञ्जनाधिप}


\twolineshloka
{ब्रवीम्यहं न कार्पण्यान्नार्थहेतोः कथञ्चन}
{भीष्मेण दत्तमिच्छामि न त्वया राजसत्तम}


\twolineshloka
{नाहं त्वत्तोऽभिकाङ्क्षिष्ये वृत्त्युपायं जनाधिप}
{यतो भीष्मस्ततो द्रोणो यद्भीष्मस्त्वाह तत्कुरु}


\twolineshloka
{दीयतां पाण्डुपुत्रेभ्यो राज्यार्धमरिकर्शन}
{सममाचार्यकं तात तव तेषां च मे सदा}


\threelineshloka
{अश्वत्थामा यथा मह्यं तथा श्वेतहयो मम}
{बहुना किं प्रलापेन यतो धर्मस्ततो जयः ॥वासुदेव उवाच}
{}


\fourlineindentedshloka
{एवमुक्ते महाराज द्रोणेनामिततेजसा}
{व्याजहार ततो वाक्यं विदुरः सत्यसङ्गरः}
{पितुर्वदनमन्वीक्ष्य परिवृत्त्य च धर्मवित् ॥विदुर उवाच}
{}


\twolineshloka
{देवव्रत निबोधेदं वचनं मम भाषतः}
{प्रनष्टः कौरवो वंशस्त्वयायं पुनरुद्धृतः}


\twolineshloka
{तन्मे विलपमानस्य वचनं समुपेक्षसे}
{कोयं दुर्योधनो नाम कुलेऽस्मिन्कुलपांसनः}


\threelineshloka
{यस्य लोभाभिभूतस्य मतिं समनुवर्तसे}
{अनार्यस्याकृतज्ञस्य लोभेन हृतचेतसः}
{अतिक्रामति यः शास्त्रं पितुर्धर्मार्थदर्शिनः}


\twolineshloka
{एते नश्यन्ति कुरवो दुर्योधनकृतेन वै}
{यथा ते न प्रणश्येयुर्महाराज तथा कुरु}


\twolineshloka
{मां चैव धृतराष्ट्रं च पूर्वमेव महामते}
{चित्रकार इवालेख्यं कृत्वा स्थापितवानसि}


\twolineshloka
{प्रजापतिः प्रजाः सृष्ट्वा यथा संहरते तथा}
{नोपेक्षश्व महाबाहो पश्यमानः कुलक्षयम्}


\twolineshloka
{अथ तेऽद्य मतिर्नष्टा विनाशे प्रत्युपस्थिते}
{वनं गच्छ मया सार्धं धृतराष्ट्रेण चैव ह}


\twolineshloka
{बद्ध्वा वा निकृतिप्रज्ञं धार्तराष्ट्रं सुदुर्मतिम्}
{शाधीदं राज्यमद्याशु पाण्डवैरभिरक्षितम्}


\threelineshloka
{प्रसीद राजशार्दूल विनाशो दृश्यते महान्}
{पाण्डवानां कुरूणां च राज्ञाममिततेजसाम् ॥वासुदेव उवाच}
{}


\twolineshloka
{विररामैवमुक्त्वा तु विदुरो दीनमानसः}
{प्रध्यायमानः स तदा निःश्वसंश्च पुनः पुनः}


\twolineshloka
{ततोऽस्य राज्ञः सुबलस्य पुत्रीधर्मार्थयुक्तं कुलनाशमीता}
{दुर्योधनं पापमतिं नृशंसंराज्ञां समक्षं सुतमाह कोपात्}


\twolineshloka
{ये पार्थिवा राजसभां प्रविष्टाब्रह्मर्षयो ये च सभासदोऽन्ये}
{श्रृण्वन्तु वक्ष्यामि तवापराधंपापस्य सामात्यपरिच्छदस्य}


\twolineshloka
{राज्यं कुरूणामनुरूपभोज्यंक्रमागतो नः कुलधर्म एषः}
{त्वं पापबुद्धेऽतिनृशंसकर्मन्राज्यं कुरूणामनयाद्विहंसि}


\twolineshloka
{राज्ये स्थितो धृतराष्ट्रो मनीषीतस्यानुजो विदुरो दीर्घदर्शी}
{एतावतिक्रम्य कथं नृपत्वंदुर्योधन प्रार्थयसेऽद्य मोहात्}


\twolineshloka
{राजा च क्षत्ता च महानुभावौभीष्मे स्थिते परवन्तौ भवेताम्}
{अयं तु धर्मज्ञतया महात्मान राज्यकामो नृवरो नदीजः}


\twolineshloka
{राज्य तु पाण्डोरिदमप्रधृष्यंतस्याद्य पुत्राः प्रभवन्ति नान्ये}
{राज्यं तदेतन्निखिलं पाण्डवानांपैतामहं पुत्रपौत्रानुगामि}


\twolineshloka
{यद्वै ब्रूते कुरुमुख्यो महात्मादेवव्रतः सत्यसन्धो मनीषी}
{सर्वं तदस्माभिरहत्य कार्यंराज्यं स्वधर्मान्परिपालयद्भिः}


\twolineshloka
{अनुज्ञया चाथ महाव्रतस्यब्रूयान्नृपोऽयं विदुरस्तथैव}
{कार्यं भवेत्तत्सुहृद्भिर्नियोज्यंधर्मं पुरस्कृत्य सुदीर्घकालम्}


\twolineshloka
{न्यायागतं राज्यमिदं कुरूणांयुधिष्ठिरः शास्तु वै धर्मपुत्रः}
{प्रचोदितो धृतराष्ट्रेण राज्ञापुरस्कृतः शान्तनवेन चैव}


\chapter{अध्यायः १४९}
\twolineshloka
{वासुदेव उवाच}
{}


\twolineshloka
{एवमुक्ते तु गान्धार्या धृतराष्ट्रो जनेश्वरः}
{दुर्योधनमुवाचेदं राजमध्ये जनाधिप}


\twolineshloka
{दुर्योधन निबोधेदं यत्त्वां वक्ष्यामि पुत्रक}
{तथा तत्कुरु भद्रं ते यद्यस्ति पितृगौरवम्}


\twolineshloka
{सोमः प्रजापतिः पूर्वं कुरूणां वंशवर्धनः}
{सोमाद्बभूव षष्ठोऽयं ययातिर्नहुषात्मजः}


\twolineshloka
{तस्य पुत्रा बभूवुर्हि पञ्च राजर्षिसत्तमाः}
{तेषां यदुर्महातेजा ज्येष्ठः समभवत्प्रभुः}


\twolineshloka
{पूरुर्यवीयांश्च ततो योऽस्माकं वंशवर्धनः}
{शर्मिष्ठया संप्रसूतो दुहित्रा वृषपर्वणः}


\twolineshloka
{यदुश्च भरतश्रेष्ठ देवयान्याः सुतोऽभवत्}
{दौहित्रस्तात शुक्रस्य काव्यस्यामिततेजसः}


\twolineshloka
{यादवानां कुलकरो बलवान्वीर्यसंमतः}
{अवमेने स तु क्षत्रं दर्पपूर्णः सुमन्दधीः}


\twolineshloka
{न चातिष्ठत्पितुः शस्त्रि बलदर्पविमोहितः}
{अवमेने च पितरं भ्रातॄंश्चाप्यपराजितः}


\twolineshloka
{पृथिव्यां चतुरन्तायां यदुदेवाभवद्बली}
{वशे कृत्वा स नृपतीन्न्यवसन्नागसाह्वये}


\twolineshloka
{तं पिता परमक्रुद्धो ययातिर्नहुषात्मजः}
{शशाप पुत्रं गान्धारे राज्याच्चापि व्यरोपयत्}


\twolineshloka
{ये चैनमन्ववर्तन्त भ्रातरो बलदर्पिताः}
{शशाप तानपि क्रुद्धो ययातिस्तनयानथ}


\twolineshloka
{यवीयांसं ततः पुरुं पुत्रं स्ववशवर्तिनम्}
{राज्ये निवेशयामास विधेयं नृपसत्तमः}


\twolineshloka
{एवं ज्येष्ठोऽप्यथोत्सिक्तो न राज्यमभिजायते}
{यवीयांसोपि जायन्ते राज्यं वृद्धोपसेवया}


\twolineshloka
{तथैव सर्वधर्मज्ञः पितुर्मम पितामहः}
{प्रतीपः पृथिवीपालस्त्रिषु लोकेषु विश्रुतः}


\twolineshloka
{तस्य पार्थिवसिंहस्य राज्यं धर्मेण शासतः}
{त्रयः प्रजज्ञिरे पुत्रा देवकल्पा यशस्विनः}


\twolineshloka
{देवापिरभवच्छ्रेष्ठो बाह्लीकस्तदनन्तरम्}
{तृतीयः शन्तनुस्तात धृतिमान्मे पितामहः}


\twolineshloka
{देवापिस्तु महातेजास्त्वग्दोषी राजसत्तमः}
{धार्मिकः सत्यवादी च पितुः शुश्रूषणे रतः}


\twolineshloka
{पौरजानपदानां च संमतः साधुसत्कृतः}
{सर्वेषां बालवृद्धानां देवापिर्हृदयङ्गमः}


\twolineshloka
{वदान्यः सत्यसन्धश्च सर्वभूतहिते रतः}
{वर्तमानः पितुः शास्त्रे ब्राह्मणानां तथैव च}


\twolineshloka
{बाह्लीकस्य प्रियो भ्राता शन्तनोश्च महात्मनः}
{सौभ्रात्रं च परं तेषां सहितानां महात्मनाम्}


\twolineshloka
{अथ कालस्य पर्याये वृद्धो नृपतिसत्तमः}
{संभारानभिषेकार्थं कारयामास शास्त्रतः}


\twolineshloka
{कारयामास सर्वाणि मङ्गलार्थानि वै विभुः}
{तं ब्राह्मणाश्च वृद्धाश्च पौरजानपदैः सह}


\threelineshloka
{सर्वे निवारयामासुर्देवापेरभिषेचनम्}
{स तच्छ्रुत्वा तु नृपतिरभिषेकनिवारणम्}
{अश्रुकण्ठोऽभवद्राजा पर्यशोचत चात्मजम्}


\twolineshloka
{एवं वदान्यो धर्मज्ञः सत्यसन्धश्च सोऽभवत्}
{प्रियः प्रजानामपि संस्त्वग्दोषेण प्रदूषितः}


\twolineshloka
{हीनाङ्गं पृथिवीपालं नाभिनन्दन्ति देवताः}
{इति कृत्वा नृपश्रेष्ठं प्रत्यषेधन्द्विजर्षभाः}


\twolineshloka
{ततः प्रव्यथिताङ्गोऽसौ पुत्रशोकसमन्वितः}
{ममार तं मृतं दृष्ट्वा देवापिः संश्रितो वनम्}


\twolineshloka
{बाह्लीको मातुलकुलं त्यक्त्वा राज्यं समाश्रितः}
{पितृभ्रातॄन्परित्यज्य प्राप्तवान्परमर्धिमत्}


\twolineshloka
{बाह्लीकेन त्वनुज्ञातः शन्तनुर्लोकविश्रुतः}
{पितर्युपरते राजन्राजा राज्यमकारयत्}


\twolineshloka
{तथैवाहं मतिमता परिचिन्त्येह पाण्डुना}
{ज्येष्ठः प्रभ्रंशितो राज्याद्धीनाङ्ग इति भारत}


\twolineshloka
{पाण्डुस्तु राज्यं संप्राप्तः कनीयानपि सन्नृपः}
{विनाशे तस्य पुत्राणामिदं राज्यमरिन्दम}


\twolineshloka
{मय्यभागिनि राज्याय कथं त्वं राज्यमिच्छसि}
{अराजपुत्रो ह्यस्वामी परस्वं हर्तुमिच्छसि}


\twolineshloka
{युधिष्ठिरो राजपुत्रो महात्मान्यायागतं राज्यमिदं च तस्य}
{स कौरवस्यास्य कुलस्य भर्ताप्रशासिता चैव महानुभावः}


\twolineshloka
{स सत्यसन्धः स तथाऽप्रमत्तःशास्त्रे स्थितो बन्धुजनस्य साधुः}
{प्रियः प्रजानां सुहृदानुकम्पीजितेन्द्रियः साधुजनस्य भर्ता}


\twolineshloka
{क्षमा तितिक्षा दम आर्जवं चसत्यव्रतत्वं श्रुतमप्रमादः}
{भूतानुकम्पा ह्यनुशासनं चयुधिष्ठिरे राजगुणाः समस्ताः}


\twolineshloka
{अराजपुत्रस्त्वमनार्यवृत्तोलुब्धः सदा बन्धुषु पापबुद्धिः}
{क्रमागतं राज्यमिदं परेषांहर्तुं कथं शक्ष्यसि दुर्विनीत}


\twolineshloka
{प्रयच्छ राज्यार्धमपेतमोहःसवाहनं त्व सपरिच्छदं च}
{ततोऽवशेषं तव जीवितस्यसहानुजस्यैव भवेन्नरेन्द्र}


\chapter{अध्यायः १५०}
\twolineshloka
{भगवानुवाच}
{}


\twolineshloka
{एवमुक्ते तु भीष्मेण द्रोणेन विदुरेण च}
{गान्धार्या धृतराष्ट्रेण न वै मन्दोऽन्वबुद्ध्यत}


\twolineshloka
{अवधूयोत्थितो मन्दः क्रोधसंरक्तलोचनः}
{अन्वद्रवन्त तं पश्चाद्राजानस्त्यक्तजीविताः}


\twolineshloka
{आज्ञापयच्च राज्ञस्तान्पार्थिवान्नष्टचेतसः}
{प्रयात वै कुरुक्षेत्रं पुष्योऽद्येति पुनः पुनः}


\twolineshloka
{ततस्ते पृथिवीपालाः प्रययुः सहसैनिकाः}
{भीष्मं सेनापतिं कृत्वा संहृष्टाः कालचोदिताः}


\twolineshloka
{अक्षौहिण्यो दशैका च कौरवाणां समागताः}
{तासां प्रमुखतो भीष्मस्तालकेतुर्व्यरोचत}


\twolineshloka
{यदत्र युक्तं प्राप्तं च तद्विधत्स्व विशांपते}
{उक्तं भीष्मेण यद्वाक्यं द्रोणेन विदुरेण च}


\twolineshloka
{गान्धार्या धृतराष्ट्रेण समक्षं मम भारत}
{एतत्ते कथितं राजन्यद्वृत्तं कुरुसंसदि}


\twolineshloka
{साम चादौ प्रयुक्तं मे राजन्सौभ्रात्रमिच्छता}
{अभेदायास्य वंशस्य प्रजानां च विवृद्धये}


\twolineshloka
{पुनर्भेदश्च मे युक्तो यदा साम न गृह्यते}
{कर्मानुकीर्तनं चैव देवमानुपसंहितम्}


\twolineshloka
{यदा नाद्रियते वाक्यं सामपूर्वं सुयोधनः}
{तदा मया समानीय भेदिताः सर्वपार्थिवाः}


\twolineshloka
{अद्भुतानि च घोराणि दारुणानि च भारत}
{अमानुषाणि कर्माणि दर्शितानि मया विभो}


\twolineshloka
{निर्भर्त्सयित्वा राज्ञस्तांस्तृणीकृत्य सुयोधनम्}
{राधेयं भीषयित्वा च सौबलं च पुनः पुनः}


\twolineshloka
{द्यूततो धार्तराष्ट्राणां निन्दां कृत्वा तथा पुनः}
{भेदयित्वा नृपास्नर्वान्वाग्भिर्मन्त्रेम चासकृत्}


\twolineshloka
{पुनः सामाभिसंयुक्तं संप्रदानमथाब्रुवम्}
{अभेदात्कुरुवंशस्य कार्ययोगात्तथैव च}


\twolineshloka
{ते शूरा धृतराष्ट्रस्य भीष्मस्य विदुरस्य च}
{तिष्ठेयुः पाण्डवाः सर्वे हित्वा मानमधश्चराः}


\twolineshloka
{प्रयच्छन्तु च ते राज्यमनीशास्ते भवन्तु च}
{यथाह राजा गाङ्गेयो विदुरश्च हितं तव}


\twolineshloka
{सर्वं भवतु ते राज्यं पञ्च ग्रामान्विसर्जय}
{अवश्यं भरणीया हि पितुस्ते राजसत्तम}


\twolineshloka
{एवमुक्तोऽपि दुष्टात्मा नैव भागं व्यमुञ्चत}
{दण्डं चतुर्थं पश्यामि तेषु पापेषु नान्यथा}


\twolineshloka
{निर्याताश्च विनाशाय कुरुक्षेत्रं नराधिपाः}
{एतत्ते कथितं राजन्यद्वृत्तं कुरुसंसदि}


\twolineshloka
{न ते राज्यं प्रयच्छन्ति विना युद्धेन पाण्डव}
{विनाशहेतवः सर्वे प्रत्युपस्थितमृत्यवः}


\chapter{अध्यायः १५१}
\twolineshloka
{वैशंपायन उवाच}
{}


\twolineshloka
{जनार्दनवचः श्रुत्वा धर्मराजो युधिष्ठिरः}
{भ्रातॄनुवाच धर्मात्मा समक्षं केशवस्य ह}


\twolineshloka
{श्रुतं भवद्भिर्यद्वृत्तं सभायां कुरुसंसदि}
{केशवस्यापि यद्वाक्यं तत्सर्वमवधारितम्}


\twolineshloka
{तस्मात्सेनाविभागं मे कुरुध्वं नरसत्तमाः}
{अक्षौहिण्यश्च सप्तैताः समेता विजयाय वै}


\twolineshloka
{तासां ये पतयः सप्त विख्यातास्तान्निबोधत}
{द्रुपदश्च विराटश्च धृष्टद्युम्नशिखण्डिनौ}


\twolineshloka
{सात्यकिश्चेकितानश्च भीमसेनश्च वीर्यवान्}
{एते सेनाप्रणेतारो वीराः सर्वे तनुत्यजः}


\twolineshloka
{सर्वे वेदविदः शूराः सर्वे सुचिरितव्रताः}
{ह्रीमन्तो नीतिमन्तश्च सर्वे युद्धविशारदाः}


\twolineshloka
{इष्वस्त्रकुशलाः सर्वे तथा सर्वास्त्रयोधिनः}
{सप्तानामपि यो नेता सेनानां प्रविभागवित्}


\fourlineindentedshloka
{यः सहेत रणे भीष्णं शरार्चिःपावकोपमम्}
{तं तावत्सहदेवात्र प्रब्रूहि कुरुनन्दन}
{स्वगतं पुरुषव्याघ्र को नः सेनापतिः क्षमः ॥सहदेव उवाच}
{}


\twolineshloka
{संयुक्त एकदुःखश्च वीर्यवांश्च महीपतिः}
{यं समाश्रित्य धर्मज्ञं स्वमंशमनुयुञ्ज्महे}


\threelineshloka
{मत्स्यो विराटो बलवान्कृतास्त्रो युद्धदुर्मदः}
{प्रसहिष्यति सङ््ग्रामे भीष्मं तांश्च महारथान् ॥वैशंपायन उवाच}
{}


\twolineshloka
{तथोक्ते सहदेवेन वाक्ये वाक्यविशारदः}
{नकुलोऽनन्तरं तस्मादिदं वचनमाददे}


\twolineshloka
{वयसा शास्त्रतो धैर्यात्कुलेनाभिजनेन च}
{ह्रीमान्बलान्वितः श्रीमान्सर्वशास्त्रविशारदः}


\twolineshloka
{वेद चास्त्रं भारद्वाजाहुर्धर्षः सत्यसङ्गरः}
{यो नित्यं स्पर्धते द्रोणं भीष्मं चैव महाबलम्}


\twolineshloka
{श्लाघ्यः पार्थिववंशस्य प्रमुखे वाहिनीपतिः}
{पुत्रपौत्रेः परिवृतः शतशाख इव द्रुमः}


\twolineshloka
{यस्तताप तपो घोरं सदारः पृथिवीपतिः}
{रोषाद्द्रोणविनाशाय वीरः समितिशोभनः}


\twolineshloka
{पितेवास्मान्समाधत्ते यः सदा पार्थिवर्षभः}
{श्वशुरो द्रुपदोऽस्माकं सेनाग्रं स प्रकर्षतु}


\twolineshloka
{स द्रोणभीष्मावायातौ सहेदिति मतिर्मम}
{स हि दिव्यास्त्रविद्राजा सखा चाङ्गिरसो नृपः}


\twolineshloka
{माद्रीसुताभ्यामुक्ते तु स्वमते कुरुनन्दनः}
{वासविर्वासवसमः सव्यसाच्यब्रवीद्वचः}


\twolineshloka
{योयं तपःप्रभावेन ऋषिसन्तोषणेन च}
{दिव्यः पुरुष उत्पन्नो ज्वालावर्णो महाभुजः}


\twolineshloka
{धनुष्मान्कवची खङ्गी रथमारुह्य दंशितः}
{दिव्यैर्हयवरैर्युक्तमग्निकुण्डात्समुत्थितः}


\twolineshloka
{गर्जन्निव महामेघो रथघोषेण वीर्यवान्}
{सिंहसंहननो वीरः सिंहतुल्यपराक्रमः}


\twolineshloka
{सिंहोरस्कः सिंहभुजः सिंहवक्षा महाबलः}
{सिंहप्रगर्जनो वीरः संहस्कन्धो महाद्युतिः}


\twolineshloka
{शुभ्रूः सुदंष्ट्रः सुहनुः सुबाहुः सुमुखोऽकृशः}
{सुजत्रुः सुविशालाक्षः सुपादः सुप्रतिष्ठितः}


\twolineshloka
{अभेद्यः सर्वशस्त्राणां प्रभिन्न इव वारणः}
{जज्ञे द्रोणविनाशाय सत्यवादी जितेन्द्रियः}


\twolineshloka
{धृष्टद्युम्नमहं मन्ये सहेद्भीष्मस्य सायकान्}
{वज्राशनिसमस्पर्शान्दीप्तास्यानुरगानिव}


\twolineshloka
{यमदूतसमान्वेगे निपाते पावकोपमान्}
{रामेणाजौ विषिहितान्वज्रनिष्पेषदारुणान्}


\twolineshloka
{पुरुष तं न पश्यामि यः सहेत महाव्रतम्}
{धृष्टद्युम्नमृते राजन्निति मे धीयते मतिः}


\fourlineindentedshloka
{क्षिप्रहस्तश्चित्रयोधी मतः सेनापतिर्मम}
{अभेद्यकवचः श्रीमान्मातङ्ग इव यूथपः ॥ 5-151-29a`अर्जुनेनैवमुक्ते तु भीमो वाक्यं समाददे}
{'वधार्थं यः समुत्पन्नः शिखण्डी द्रुपदात्मजः}
{वदन्ति सिद्धा राजेन्द्र ऋषयश्च समागताः}


\twolineshloka
{यस्य सङ्ग्राममध्ये तु दिव्यमस्त्रं प्रकुर्वतः}
{रूपं द्रक्ष्यन्ति पुरुषा रामस्येव महात्मनः}


\twolineshloka
{न तं युद्धे प्रपश्यामि यो भिन्द्यात्तु शिखण्डिनाम्}
{शस्त्रेण समरे राजन्सन्नद्वं स्यन्दने स्थितम्}


\threelineshloka
{द्वैरथे समरे नान्यो भीष्मं हन्यान्महाव्रतम्}
{शिखण्डिनमृते वीरं स मे सेनापतिर्मतः ॥युधिष्ठिर उवाच}
{}


\twolineshloka
{सर्वस्य जगतस्तात सारासारं बालाबलम्}
{सर्वं जानाति धर्मात्मा मतमेषां च केशवः}


\twolineshloka
{यमाह कृष्णो दाशार्हः सोऽस्तु सेनापतिर्मम}
{कृतास्त्रोप्यकृतास्त्रो वा वृद्धो वा यदि वा युवा}


\twolineshloka
{एष नो विजये मूलमेष तात विपर्यये}
{अत्र प्राणाश्च राज्यं च भावाभावौ सुखासुखे}


\twolineshloka
{एष धाता विधाता च सिद्धिरत्र प्रतिष्ठिता}
{यमाह कृष्णो दाशार्हः सोस्तु नो वाहिनीपतिः}


\twolineshloka
{ब्रवीतु वदतां श्रेष्ठो निशा समभिवर्तते}
{ततः सेनापतिं कृत्वा कृष्णस्य वशवर्तिनः}


\threelineshloka
{रात्रेः शेषे व्यतिक्रान्ते प्रयास्यामो रणाजिरम्}
{अधिवासितशस्त्राश्च कृतकौतुकमङ्गलाः ॥वैशंपायन उवाच}
{}


\twolineshloka
{तस्य तद्वचनं श्रुत्वा धर्मराजस्य धीमतः}
{अब्रवीत्पुण्डरीकाक्षो धनञ्जयमवेक्ष्य ह}


\twolineshloka
{ममाप्येते महाराज भवद्भिर्य उदाहृताः}
{नेतारस्तव सेनाया मता विक्रान्तयोधिनः}


\twolineshloka
{सर्व एव समर्था हि तव शत्रुं प्रबाधितुम्}
{इन्द्रस्यापि भयं ह्येते जनयेयुर्महाहवे}


\twolineshloka
{किं पुनर्धार्तराष्ट्राणां लुब्धानां पापचेतसाम्}
{मयापि हि महाबाहो त्वत्प्रियार्थं महाहवे}


\twolineshloka
{कृतो यत्नो महांस्तत्र शमः स्यादिति भारत}
{धर्मस्य गतमानृण्यं न स्म वाच्या विवक्षताम्}


\twolineshloka
{कृतार्थं मन्यते बाल आत्मनमविचक्षणः}
{धार्तराष्ट्रो बलस्थं च पश्यत्यात्मानमातुरः}


\twolineshloka
{युज्यतां वाहिनी साधु वधसाध्या हि मे मताः}
{न धार्तराष्ट्राः शक्ष्यन्ति स्थातुं दृष्ट्वा धनञ्जयम्}


\twolineshloka
{भीमसेनं च सङ्क्रुद्धं यमौ चापि यमोपमौ}
{युयुधानं द्वितीयं च धृष्टद्युम्नममर्षणम्}


\twolineshloka
{अभिमन्युं द्रौपदेयान्विराटद्रुपदावपि}
{अक्षौहिणीपतींश्चान्यान्नरेन्द्रान्भीमविक्रमान्}


\twolineshloka
{सारवद्बलमस्माकं दुष्प्रधर्षं दुरासदम्}
{धार्तराष्ट्रबलं सङ्ख्ये हनिष्यति न संशयः}


\threelineshloka
{धृष्टद्युम्नमहं मन्ये सेनापतिमरिन्दम}
{वैशंपायन उवाच}
{एवमुक्ते तु कृष्णेन संप्राहृष्यन्नरोत्तमाः}


\twolineshloka
{तेषां प्रहृष्टमनसां नादः समभवन्महान्}
{योग इत्यथ सैन्यानां त्वरतां संप्रधावताम्}


\twolineshloka
{हयवारणशब्दाश्च नेमिघोषाश्च सर्वतः}
{शङ्खदुन्दुभिघोषाश्च तुमुलाः सर्वतोऽभवन्}


\twolineshloka
{तद्रुग्रं सागरनिभं क्षुब्धं बलसमागमम्}
{रथपत्तिगजोदयं महोर्मिभिरिवाकुलम्}


\twolineshloka
{धावतामाहुयानानां तनुत्राणि च बध्नताम्}
{प्रयास्यतां पाण्डवानां ससैन्यानां समन्ततः}


\twolineshloka
{गङ्गेव पूर्णा दुर्धर्षा समदृश्यत वाहिनी}
{अग्रानीके भीमसेनो माद्रीपुत्रौ च दंशितौ}


\twolineshloka
{सौभद्रो द्रौपदेयाश्च धृष्टद्युम्नस्य पार्षतः}
{प्रभद्रकाश्च पञ्चाला भीमसेनमुखा ययुः}


\twolineshloka
{ततः शब्दः समभवत्सुमुद्रस्येव पर्वणि}
{हृष्टानां संप्रयातानां घोषो दिवमिवास्पृशम्}


\twolineshloka
{प्रहृष्टा दंशिता योधाः परानीकविदारणाः}
{तेषां मध्ये ययौ राजा कुन्तीपुत्रो युधिष्ठिरः}


\twolineshloka
{शकटापणवेशाश्च यानयुग्यं च सर्वशः}
{कोशं यन्त्रायुधं चैव ये च वैद्याश्चिकित्सकाः}


\twolineshloka
{फल्गु यच्चं बलं किंचिद्यच्चापि कृशदुर्बलम्}
{तत्सङ्गृह्य ययौ राजा ये चापि परिचारकाः}


\twolineshloka
{उपप्लाव्ये तु पाञ्चाली द्रौपदी सत्यवादिनी}
{सह स्त्रीभिर्निनवृते दासीदाससमावृता}


\twolineshloka
{कृत्वा मूलप्रतीकारं गुल्मैः स्थावरजङ्गमैः}
{स्कन्धावारेण महता प्रययुः पाण्डुनन्दनाः}


\twolineshloka
{ददतो गां हिरण्यं च ब्राह्मणैरभिसंवृताः}
{स्तूयमाना ययू राजन्रथैर्मणिविभूषितैः}


\twolineshloka
{केकया धृष्टकेतुश्च पुत्रः काश्यस्य चाभिभूः}
{श्रेणिमान्वसुदानश्च शिखण्डी चापराजितः}


\twolineshloka
{हृष्टास्तुष्टाः कवचिनः सशस्त्राः समलङ्कृताः}
{राजानमन्वयुः सर्वे परिवार्य युधिष्ठिरम्}


\twolineshloka
{जघनार्धे विराटश्च याज्ञसेनिश्च सौमकिः}
{सुधर्मा कुन्तिभोजश्च धृष्टद्युम्नस्य चात्मजाः}


\twolineshloka
{रथायुतानि चत्वारि हयाः पञ्चगुणास्तथा}
{पत्तिसैन्यं दशगुणं गजानामयुतानि षट्}


\twolineshloka
{अनाधृष्टिश्चेकितानो धृष्टकेतुश्च सात्यकिः}
{परिवार्य ययुः सर्वे वासुदेवधनञ्जयौ}


\twolineshloka
{आसाद्य तु कुरुक्षेत्रं व्यूढानीकाः प्रहारिणः}
{पाण्डवाः समदृश्यन्त नर्दन्तो वृषभा इव}


\twolineshloka
{तेऽवगाह्य कुरुक्षेत्रं शङ्खान्दध्मुररिन्दमाः}
{तथैव दध्मतुः शङ्खं वासुदेवधनञ्जयौ}


\twolineshloka
{पाञ्चजन्यस्य निर्घोषं विस्फूर्जितमिवाशनेः}
{निशम्य सर्वसैन्यानि समहष्यन्त सर्वशः}


\twolineshloka
{शङ्खदुन्दुभिसंहृष्टः सिंहनादस्तरस्विनाम्}
{पृथिवीं चान्तरिक्षं च सागरांश्चान्वनादयत्}


\chapter{अध्यायः १५२}
\twolineshloka
{वैशंपायन उवाच}
{}


\twolineshloka
{ततो देशे समे स्निग्धे प्रभूतयवसेन्धने}
{निवेशयामास तदा सेनां राजा युधिष्ठिरः}


\twolineshloka
{परिहृत्य श्मशानानि देवतायतनानि च}
{आश्रमांश्च महर्षीणां तीर्थान्यायतनानि च}


\twolineshloka
{मधुरानूषरे देशे शुचौ पुण्ये महामतिः}
{निवेशं कारयामास कुन्तीपुत्रो युधिष्ठिरः}


\twolineshloka
{ततश्च पुनरुत्थाय सुखी विश्रान्तवाहनः}
{प्रययौ पृथिवीपालैर्वृतः शतसहस्रशः}


\twolineshloka
{विद्राव्य शतशो गुल्मान्धार्तराष्ट्रस्य सैनिकान्}
{पर्यक्रामत्समन्ताच्च पार्थेन सह केशवः}


\twolineshloka
{शिबिरं मापयामास धृष्टद्युम्नश्च पार्षतः}
{सात्यकिश्च रथोदारो युयुधानश्च वीर्यवान्}


\twolineshloka
{आसाद्य सरितं पुण्यां कुरुक्षेत्रे हिरण्वतीम्}
{सूपतीर्थां शुचिजलां शर्करापङ्कवर्जिताम्}


\twolineshloka
{खानयामास परिखां केशवस्त्रत्र भारत}
{गुप्त्यर्थमपि चादिश्य बलं तत्र न्यवेशयत्}


\twolineshloka
{विधिर्यः शिबिरस्यासीत्पाण्डवानां महात्मनाम्}
{तद्विधानि नरेन्द्राणां कारयामास केशवः}


\twolineshloka
{प्रभूततरकाष्ठानि दुराधर्षतराणि च}
{भक्ष्यभोज्यान्नपानानि शतशोऽथ सहस्रशः}


\twolineshloka
{शिबिराणि महार्हाणि राज्ञां तत्र पृथक्पृथक्}
{विमानानीव राजेन्द्र निविष्टानि महीतले}


\twolineshloka
{तत्रासञ्शिल्पिनः प्राज्ञाः शतशो दत्तवेतनाः}
{सर्वापकरणैर्युक्ता वैद्याः शास्त्रविशारदाः}


\twolineshloka
{ज्याधनुर्वर्मशस्त्राणां तथैव मधुसर्पिषोः}
{ससर्जरकसपांसूनां राशयः पर्वतोपमाः}


\twolineshloka
{बहूदकं सुयवसं तुषाङ्गारसमन्वितम्}
{शिबिरे शिबिरे राजा सञ्चकार युधिष्ठिरः}


\twolineshloka
{महायन्त्राणि नाराचास्तोमराणि परश्वधाः}
{धनूंषि कवचादीनि ऋष्टयस्तूणसंयुताः}


\twolineshloka
{गजाः कण्टकसन्नाहा लोहवर्मोत्तरच्छदाः}
{दृश्यन्ते तत्र गिर्याभाः सहस्रशतयोधिनः}


\twolineshloka
{निविष्टान्पाण्डवांस्तत्र ज्ञात्वा मित्राणि भारत}
{अभिसस्रुर्यथादेशं सबलाः सहवाहनाः}


\twolineshloka
{चरितब्रह्मचर्यास्ते सोमपा भूरिदक्षिणाः}
{जयाय पाण्डुपुत्राणां समाजग्मुर्महीक्षितः}


\chapter{अध्यायः १५३}
\twolineshloka
{युधिष्ठिरं सहानीकमुपायान्तं युयुत्सया}
{सन्निविष्टं कुरुक्षेत्रे वासुदेवेन पालितम्}


\twolineshloka
{विराटद्रुपदाभ्यां च सपुत्राभ्यां समन्वितम्}
{केकयैर्वृष्णिभिश्चैव पार्थिवैः शतशो वृतम्}


\twolineshloka
{महेन्द्रमिव चादित्यैरभिगुप्तं महारथैः}
{श्रुत्वा दुर्योधनो राजा किं कार्यं प्रत्यपद्यत}


\twolineshloka
{एतदिच्छाम्यहं श्रोतुं विस्तरेण महामते}
{संभ्रमे तुमुले तस्मिन्यदासीत्कुरुजाङ्गले}


\twolineshloka
{व्यथयेयुरिमे देवान्सेन्द्रानपि समागमे}
{पाण्डवा वासुदेवश्च विराटद्रुपदौ तथा}


\twolineshloka
{धृष्टद्युम्नश्च पाञ्चाल्यः शिखण्डी च महारथः}
{युधामन्युश्च विक्रान्तो देवैरपि दुरासदः}


\threelineshloka
{एतदिच्छाम्यहं श्रोतुं विस्तरेण तपोधन}
{कुरूणां पाण्डवानां च यद्यदासीद्विचेष्टितम् ॥वैशंपायन उवाच}
{}


\twolineshloka
{प्रतियाते तु दाशार्हे राजा दुर्योधनस्तदा}
{कर्णं दुःशासनं चैव शकुनिं चाब्रवीदिदम्}


\twolineshloka
{अकृतेनैव कार्येण गतः पार्थानधोक्षजः}
{स एनान्मन्युनाविष्टो ध्रुवं धक्ष्यत्यसंशयम्}


\twolineshloka
{इष्टो हि वासुदेवस्य पाण्डवैर्मम विग्रहः}
{भीमसेनार्जुनौ चैव दाशार्हस्य मते स्थितौ}


\twolineshloka
{अजातशत्रुरत्यर्थं भीमसेनवशानुगः}
{निकृतश्च मया पूर्वं सह सर्वैः सहोदरैः}


\twolineshloka
{विराटद्रुपदौ चैव कृतवैरौ मया सह}
{तौ च सेनाप्रणेतारौ वासुदेववशानुगौ}


\twolineshloka
{भविता विग्रहः सोयं तुमुलो लोमहर्षणः}
{तस्मात्साङ्ग्रामिकं सर्वं कारयध्वमतन्द्रिताः}


\twolineshloka
{शिबिराणि कुरुक्षेत्रे क्रियन्तां वसुधाधिपाः}
{स्वपर्याप्तावकाशानि दुरादेयानि शत्रुभिः}


\threelineshloka
{आसन्नजलकोष्ठानि शतशोथ सहस्रशः}
{अच्छेद्याहारमार्गाणि बन्धोच्छ्रयचितानि च}
{}


\twolineshloka
{विविधायुधपूर्णानि पताकाध्वजवन्ति च}
{समाश्च तेषां पन्थानः क्रियन्तां नगराद्बहिः}


\threelineshloka
{प्रयाणं घुष्यतामद्य श्वोभूत इति माचिरम्}
{वैशंपायन उवाच}
{ते तथेति प्रतिज्ञाय श्वोभूते चक्रिरे तथा}


\twolineshloka
{हृष्टरूपा महात्मानो निवासाय महीक्षिताम्}
{ततस्ते पार्थिवाः सर्वे तच्छ्रुत्वा राजशासनम्}


\twolineshloka
{आसनेभ्यो महार्हेभ्य उदतिष्ठन्नमर्षिताः}
{बाहून्परिघसङ्काशान्संस्पृशन्तः शनैः शनैः}


\threelineshloka
{काञ्चनाङ्गददीप्तांश्च चन्दनागरुभूषितान्}
{उष्णीषाणि नियच्छन्तः पुण्डरीकनिभै करैः}
{अन्तरीयोत्तरीयाणि भूषणानि च सर्वशः}


\twolineshloka
{ते रथान्रथिनः श्रेष्ठा हयांश्च हयकोविदाः}
{सज्जयन्ति स्म नागांश्च नागशिक्षास्वनुष्ठिताः}


\twolineshloka
{अथ वर्माणि चित्राणि काञ्चनानि बहूनि च}
{विविधानि च शस्त्राणि चक्रुः सर्वाणि सर्वशः}


\twolineshloka
{पदातयश्च पुरुषाः शस्त्राणि विविधानि च}
{उपाजह्रुः शरीरेषु हेमचित्राण्यनेकशः}


\twolineshloka
{तदुत्सव इवोदग्रं संप्रहृष्टनरावृतम्}
{नगरं धार्तराष्ट्रस्य भारतासीत्समाकुलम्}


\twolineshloka
{जनौघसलिलावर्तो रथनागाश्वमीनवान्}
{शङ्खदुन्दुभिनिर्घोषः कोशसञ्चयरत्नवान्}


\twolineshloka
{चित्राभरणवर्मोर्मिः शस्त्रनिर्मलफेनवान्}
{प्रासादमालाद्रिवृतो रथ्यापणमहाह्रदः}


\twolineshloka
{योधचन्द्रोदयोद्भूतः कुरुराजमहार्णवः}
{व्यदृश्यत तदा राजंश्चन्द्रोदय इवोदधिः}


\chapter{अध्यायः १५४}
\twolineshloka
{वैशंपायन उवाच}
{}


\twolineshloka
{वासुदेवस्य तद्वाक्यमनुस्मृत्य युधिष्ठिरः}
{पुनः पप्रच्छ वार्ष्णेयं कथं मन्दोऽऽब्रवीदिदम्}


\twolineshloka
{अस्मिन्नभ्यागते काले किंच नः क्षममच्युत}
{कथं च वर्तमाना वै स्वधर्मान्न च्यवेमहि}


\twolineshloka
{दुर्योधनस्य कर्णस्य शकुनेः सौबलस्य च}
{वासुदेव मतज्ञोऽसि मम सभ्रातृकस्य च}


\twolineshloka
{विदुरस्यापि तद्वाक्यं श्रुतं भीष्मस्य चोभयोः}
{कुन्त्याश्च विपुलप्रज्ञ प्रज्ञा कार्त्स्न्येन ते श्रुता}


\twolineshloka
{सर्वमेतदतिक्रम्य विचार्य च पुनः पुनः}
{क्षमं यन्नो महाबाहो तद्ब्रवीह्यविचारयन्}


\twolineshloka
{श्रुत्वैतद्धर्मराजस्य धर्मार्थसहितं वचः}
{मेघदुन्दुभिनिर्घोषः कृष्णो वाक्यमथाब्रवीत्}


\twolineshloka
{उक्तवानस्मि यद्वाक्यं धर्मार्थसहितं हितम्}
{न तु तन्निकृतिप्रज्ञे कौरव्ये प्रतितिष्ठति}


\twolineshloka
{न च भीष्मस्य दुर्मेधाः शृणोति विदुरस्य वा}
{मम वा भाषितं किंचित्सर्वमेवातिवर्तते}


\twolineshloka
{नैष कामयते धर्मं नैष कामयते यशः}
{जितं स मन्यते सर्वं दुरात्मा कर्णमाश्रितः}


\twolineshloka
{बन्धमाज्ञापयामास मम चापि सुयोधनः}
{न च तं लब्धवान्कामं दुरात्मा पापनिश्चयः}


\twolineshloka
{न च भीष्मो न च द्रोणो युक्तं तत्राहतुर्वचः}
{सर्वे तमनुवर्तन्ते ऋते विदुरमुच्युत}


\twolineshloka
{शकुनिः सौबलश्चैव कर्णदुःशासनावपि}
{त्वय्ययुक्तान्यभाषन्त मूढा मूढममर्षणम्}


\twolineshloka
{किंच तेन मयोक्तेन यान्यभाषत कौरवः}
{संक्षेपेण दुरात्मासौ न युक्तं त्वयि वर्तते}


\twolineshloka
{पार्थिवेषु न सर्वेषु य इमे तव सैनिकाः}
{यत्पापं यन्न कल्याणं सर्वं तस्मिन्प्रतिष्ठितम्}


\threelineshloka
{न चापि वयमत्यर्थं परित्यागेन कर्हिचित्}
{कौरवैः शममिच्छामस्तत्र युद्धमनन्तरम् ॥वैशंपायन उवाच}
{}


\twolineshloka
{तच्छ्रुत्वा पार्थिवाः सर्वे वासुदेवस्य भाषितम्}
{अब्रुवन्तो मुखं राज्ञः समुदैक्षन्त भारत}


\twolineshloka
{युधिष्ठिरस्त्वभिप्रायमभिलक्ष्य महीक्षिताम्}
{योगमाज्ञापयामास भीमार्जुनयमैः सह}


\twolineshloka
{ततः किलकिलाभूतमनीकं पाण्डवस्य ह}
{आज्ञापिते तदा योगे समहृष्यन्त सैनिकाः}


\twolineshloka
{अवध्यानां वधं पश्यन्धर्मराजो युधिष्ठिरः}
{निश्वसन्भीमसेनं च विजयं चेदमब्रवीत्}


\twolineshloka
{यदर्थं वनवासश्च प्राप्तं दुःखं च यन्मया}
{सोयमस्मानुपैत्येव परोऽनर्थः प्रयत्नतः}


\twolineshloka
{यस्मिन्यत्नः कृतोऽस्माभिः स नो हीनः प्रयत्नतः}
{अकृते तु प्रयत्नेऽस्मानुपावृत्तः कलिर्महान्}


\threelineshloka
{कथं ह्यवध्यैः सङ्ग्रामः कार्यः सह भविष्यति}
{कथं हत्वा गुरून्वृद्धान्विजयो नो भविष्यति ॥वैशंपायन उवाच}
{}


\twolineshloka
{तच्छ्रुत्वा धर्मराजस्य सव्यसाची परन्तपः}
{यदुक्तं वासुदेवेन श्रावयामास तद्वचः}


\twolineshloka
{उक्तवान्देवकीपुत्रः कुन्त्याश्च विदुरस्य च}
{वचनं तत्त्वया राजन्निखिलेनावधारितम्}


\twolineshloka
{न च तौ वक्ष्यतोऽधर्ममिति मे नैष्ठिकी मतिः}
{नापि युक्तं च कौन्तेय निवर्तितुमयुध्यतः}


\twolineshloka
{तच्छ्रुत्वा वासुदेवोऽपि सव्यसाचिवचस्तदा}
{स्मयमानोऽब्रवीद्वाक्यं पार्थमेवमिति ब्रुवन्}


\twolineshloka
{ततस्ते धृतसंकल्पा युद्धाय सह सैनिकाः}
{पाण्डवेया महाराज तां रात्रिं सुखमावसन्}


\chapter{अध्यायः १५५}
\twolineshloka
{वैशंपायन उवाच}
{}


\twolineshloka
{व्युष्टायां वै रजन्यां हि राजा दुर्योधनस्ततः}
{व्यभजत्तान्यनिकानि दश चैकं च भारत}


\twolineshloka
{नरहस्तिरथाश्वानां सारं मध्यं च फल्गु च}
{सर्वेष्वेतेष्वनीकेषु सन्दिदेश नराधिपः}


\threelineshloka
{सानुकर्षाः सतूणीराः सवरूथाः सतोमराः}
{सोपासङ्गाः सशक्तीकाः सनिषङ्गाः सहर्ष्टयः}
{}


\twolineshloka
{सध्वजाः सपताकाश्च सशरासनतोमराः}
{रज्जुभिश्च विचित्राभिः सपाशाः सपरिच्छदाः}


\twolineshloka
{सकचग्रहविक्षेपाः सतैलगुडवालुकाः}
{साशीविषघटाः सर्वे ससर्जरसपांसवः}


% Check verse!
सघण्टफलकाः सर्वे सायोगुडजलोपलाः ॥सशालभिन्दिपालाश्च समधूच्छिष्टमुद्गराः
\twolineshloka
{सकाण्डदण्डकाः सर्वे ससीरविषतोमराः}
{सशूर्पपिटकाः सर्वे सदात्राङ्कुशतोमराः}


\twolineshloka
{सकीलकवचाः सर्वे वाशीवृक्षादनान्विताः}
{व्याघ्रचर्मपरीवारा द्वीपिचर्मावृताश्च ते}


\twolineshloka
{सहर्ष्टयः सश्रृङ्गाश्च सप्रासविविधायुधाः}
{सकुठाराः सकुद्दालाः सतैलक्षौमसर्पिषः}


\twolineshloka
{रुक्मजालप्रतिच्छन्ना नानामणिविभूषिताः}
{चित्रानीकाः सुवपुषो ज्वलिता इव पावकाः}


\twolineshloka
{तथा कवचिनः शूराः शस्त्रेषु कृतनिश्चयाः}
{कुलीना हययोनिज्ञाः सारथ्ये विनिवेशिताः}


\twolineshloka
{बद्धारिष्टा बद्धकक्षा बद्धध्वजपताकिनः}
{बद्धाभरणनिर्यूहा बद्धचर्मासिपट्टिशाः}


\twolineshloka
{चतुर्युजो रथाः सर्वे सर्वे चोत्तमवाजिनः}
{सप्रासऋष्टिकाः सर्वे सर्वे शतशरासनाः}


\twolineshloka
{धुर्ययोर्हकययोरेकस्तथान्यौ पर्ष्णिसारथी}
{तौ चापि रथिनां श्रेष्ठौ रथी च हयवित्तथा}


\twolineshloka
{नगराणीव गुप्तानि दुराधर्षाणि शत्रुभिः}
{आसन्रथसहस्राणि हेममालीनि सर्वशः}


\twolineshloka
{यथा रथास्तथा नागा बद्धकक्षाः स्वलङ्कृताः}
{बभूवुः सप्तपुरुषा रत्नवन्त इवाद्रयः}


\twolineshloka
{द्वावङ्कुशधरौ तत्र द्वावुत्तमधनुर्धरौ}
{द्वौ वरासिधरौ राजन्नेकः शक्तिपिनाकधृत्}


\twolineshloka
{गजैर्मत्तैः समाकीर्णं सर्वमायुधकोशकैः}
{तद्बभूव बलं राजन्कौरव्यस्य महात्मनः}


\twolineshloka
{आमुक्तकवचैर्युक्तैः सपताकैः स्वलङ्कृतैः}
{सादिभिश्चोपपन्नास्तु तथा चायुतशो हयाः}


\twolineshloka
{असङ्ग्राहाः सुसंपन्ना हेमभाण्डपरिच्छदाः}
{अनेकशतसाहस्राः सर्वे सादिवशे स्थिताः}


\twolineshloka
{नानारूपविकाराश्च नानाकवचशस्त्रिणः}
{पदातिनो नरास्तत्र बभूवुर्हेममालिनः}


\twolineshloka
{रथस्यासन्दश गजा गजस्य दश वाजिनः}
{नरा दश हयस्यासन्पादरक्षाः समन्ततः}


\twolineshloka
{रथस्य नागाः पञ्चाशन्नागस्यासञ्शतं हयाः}
{हयस्य पुरुषाः सप्त भिन्नसन्धानकारिणः}


\twolineshloka
{सेना पञ्चशतं नागा रथास्तावन्त एव च}
{दशसेना च पृतना पृतना दश वाहिनी}


\twolineshloka
{सेना च वाहिनी चैव पृतना ध्वजिनी चमूः}
{अक्षौहिणीति पर्यायैर्निरुक्ता च वरूथिनी}


\twolineshloka
{एवं व्यूढान्यनीकानि कौरवेयेण धीमता}
{अक्षौहिण्यो दशैका च सङ्ख्याताः सप्त चैव ह}


\twolineshloka
{अक्षौहिण्यस्तु सप्तैव पाण्डवानामभूद्बलम्}
{अक्षौहिण्यो दशैका च कौरवाणामभूद्बलम्}


\twolineshloka
{नराणां पञ्चपञ्चाशदेषा पत्तिर्विधीयते}
{सेनामुखं च तिस्रस्ता गुल्म इत्यभिशब्दितम्}


\twolineshloka
{त्रयो गुल्मा गणस्त्वासीद्गणास्त्वयुतशोऽभवन्}
{दुर्योधनस्य सेनासु योत्स्यमानाः प्रहारिणः}


\threelineshloka
{तत्र दुर्योधनो राजा शूरान्बुद्धिमतो नरान्}
{प्रसमीक्ष्य महाबाहुश्चक्रे सेनापतींस्तदा}
{}


\twolineshloka
{पृथगक्षौहिणीनां च प्रणेतॄन्नरसत्तमान्}
{विधिवत्पूर्वमानीय पार्थिवानभ्यषेचयत्}


\twolineshloka
{कृपं द्रोणं च शल्यं च सैन्धवं च जयद्रथम्}
{सुदक्षिणं च काम्भोजं कृतवर्माणमेव च}


\twolineshloka
{द्रोणपुत्रं च कर्णं च भूरिश्रवसमेव च}
{शकुनिं सौबलं चैव बाह्लीकं च महाबलम्}


\twolineshloka
{दिवसे दिवसे तेषां प्रतिवेलं च भारत}
{चक्रे स विविधाः पूजाः प्रत्यक्षं च पुनः पुनः}


\twolineshloka
{तथा विनियताः सर्वे ये च तेषां पदानुगाः}
{बभूवुः सैनिका राज्ञां प्रियं राज्ञश्चिकीर्षवः}


\chapter{अध्यायः १५६}
\twolineshloka
{वैशंपायन उवाच}
{}


\twolineshloka
{ततः शान्तनवं भीष्मं प्राञ्जलिर्धृतराष्ट्रजः}
{सह सर्वैर्महीपालैरिदं वचनमब्रवीत्}


\twolineshloka
{ऋते सेनाप्रणेतारं पृतना सुमहत्यपि}
{दीर्यते युद्धमासाद्य पिपीलिकपुटं यथा}


\twolineshloka
{नहि जातु द्वयोर्बुद्धिः समा भवति कर्हिचित्}
{शौर्यं च वलनेतॄणां स्पर्धते च परस्परम्}


\twolineshloka
{श्रूयते च महाप्राज्ञ हैहयानमितौजसः}
{अभ्ययुर्ब्राह्मणाः सर्वे समुच्छ्रितकुशध्वजाः}


\twolineshloka
{तानभ्ययुस्तदा वैश्याः शूद्राश्चैव पितामह}
{एकतस्तु त्रयो वर्णा एकतः क्षत्रियर्षभाः}


\twolineshloka
{ते स्म युद्धे प्रभज्यन्ते त्रयो वर्णाः पुनः पुनः}
{क्षत्रियाश्च जयन्त्येव बहुलं चैकतो बलम्}


\twolineshloka
{ततस्ते क्षत्रियानेव पप्रच्छुर्द्विचसत्तमाः}
{तेभ्यः शशंसुर्धर्मज्ञा याथातथ्यं पितामह}


\twolineshloka
{वयमेकस्य श्रृणुमो महाबुद्धिमतो रणे}
{भवन्तस्तु पृथक्सर्वे स्वबुद्धिवशवर्तिनः}


\twolineshloka
{ततस्ते ब्राह्मणाश्चक्रुरेकं सेनापतिं द्विजम्}
{नये सुकुशलं शूरमजयन्क्षत्रियांस्ततः}


\twolineshloka
{एवं ये कुशलं शूरं हितेप्सितमकल्पषम्}
{सेनापतिं प्रकुर्वन्ति ते जयन्ति रणे रिपून्}


\twolineshloka
{भवानुशनसा तुल्यो हितैषी च सदा मम}
{असंहार्थः स्थितो धर्मे स नः सेनापतिर्भव}


\twolineshloka
{रश्मीवतामिवादित्यो वीरुधामिव चन्द्रमाः}
{कुबेर इव यक्षाणां देवानामिव वासवः}


\twolineshloka
{पर्वतानां यथा मेरुः सुपर्णः पक्षिणां यथा}
{कुमार इव देवानां वसूनामिव हव्यवाट्}


\twolineshloka
{भवता हि वयं गुप्ताः शत्रेणेव दिवौकसः}
{अनाधृष्या भविष्यामस्त्रिदशानामपि ध्रुवम्}


\threelineshloka
{प्रयातु नो भवानग्रे देवानामिव पावकिः}
{वयं त्वामनुयास्यामः सौरभेया इवर्षभम् ॥भीष्म उवाच}
{}


\twolineshloka
{एवमेतन्महाबाहो यथा वदसि भारत}
{यथैव हि भवन्तो मे तथैव मम पाण्डवाः}


\twolineshloka
{अपि चैव मया श्रोयो वाच्यं तेषां नराधिप}
{संयोद्धव्यं तवार्थाय यथा मे समयः कृतः}


\twolineshloka
{न तु पश्यामि योद्धारमात्मनः सदृशं भुवि}
{ऋते तस्मान्नरव्याघ्रात्कुन्तीपुत्राद्धनञ्जयात्}


\twolineshloka
{स हि वेद महाबुद्धिर्दिव्यान्यस्त्राण्यनेकशः}
{न तु मां विवृतो युद्धे जातु युध्येत पाण्डवः}


\twolineshloka
{अहं स च क्षणेनैव निर्मनुष्यमिदं जगत्}
{कुर्यावास्त्रबलेनैव ससुरासुरराक्षसम्}


\twolineshloka
{न त्वेवोत्सादनीया मे पाण्डोः पुत्रा जनाधिप}
{तस्माद्योधान्हनिष्यामि प्रयोगेणायुतं सदा}


\twolineshloka
{एवमेषां फरिष्यामि निधनं कुरुनन्दन}
{न चेत्ते मां हनिष्यन्ति पूर्वमेव समागमे}


\twolineshloka
{सेनापतिस्त्वहं राजन्समयेनापरेण ते}
{भविष्यामि यथाकामं तन्मे श्रोतुमिहार्हसि}


\threelineshloka
{कर्णो वा युध्यतां पूर्वमहं वा पृथिवीपते}
{स्पर्धते हि सदात्यर्थं सूतपुत्रो मया रणे ॥कर्ण उवाच}
{}


\threelineshloka
{नाहं जीवति गाङ्गेये राजन्योत्स्ये कथंचन}
{हते भीष्मे तु योत्स्यामि सह गाण्डीवधन्वना ॥वैशंपायन उवाच}
{}


\twolineshloka
{ततः सेनापतिं चक्रे विधिवद्भूरिदक्षिणम्}
{धृतराष्ट्रात्मजो भीष्मं सोऽभिषिक्तो व्यरोचत}


\twolineshloka
{ततो भेरीश्च शङ्खांश्च शतशोऽथ सहस्रशः}
{वादयामासुव्यग्रा वादका राजशासनात्}


% Check verse!
सिंहनादाश्च विविधा वाहनानां च निःस्वनाः
\twolineshloka
{निर्घार्ताः पृथिवीकम्पा गजबृंहितनिःस्वनाः}
{आसंश्च सर्वयोधानां पातयन्तो मनांस्युत}


\twolineshloka
{वाचश्चाप्यशरीरिण्यो दिवश्चोल्काः प्रपेदिरे}
{शिवाश्च भयवेदिन्यो नेदुर्दीप्ततरा भृशम्}


\twolineshloka
{सैनापत्ये यदा राजा गाङ्गेयमभिषिक्तवान्}
{तदैतान्युग्ररूपाणि बभूवुः शतशो नृप}


\twolineshloka
{ततः सेनापतिं कृत्वा भीष्मं परबलार्दनम्}
{वाचयित्वा द्विजश्रेष्ठान्गोभिर्निष्कैश्च भूरिशः}


\twolineshloka
{वर्धमानो जयाशीर्भिर्निर्ययौ सैनिकैर्वृतः}
{आपगेयं पुरस्कृत्य भ्रातृभिः सहितस्तदा}


% Check verse!
स्कन्धावारेण महता कुरुक्षेत्रं जगाम ह
\twolineshloka
{परिक्रम्य कुरुक्षेत्रं कर्णेन सह कौरवः}
{शिबिरं मापयामास समे देशे जनाधिप}


\twolineshloka
{मधुरानूषरे देशे प्रभूतयवसेन्धने}
{यथैव हास्तिनपुरं तद्वच्छिबिरमाबभौ}


\chapter{अध्यायः १५७}
\twolineshloka
{जनमेजय उवाच}
{}


\twolineshloka
{आपगेयं महात्मानं भीष्मं शस्त्रभृतां वरम्}
{पितामहं भारतानां ध्वजं सर्वमहीक्षिताम्}


\twolineshloka
{बृहस्पतिसं बुद्ध्या क्षमया पृथिवीसमम्}
{समुद्रमिव गाम्भीर्ये हिमवन्तमिव स्थिरम्}


\twolineshloka
{प्रजापतिमिवौदार्ये तेजसा भास्करोपमम्}
{महेन्द्रमिव शत्रूणां ध्वंसनं शरवृष्टिभिः}


\twolineshloka
{रणयज्ञे प्रवितते सुभीमे लोमहर्षणे}
{दीक्षितं चिररात्राय श्रुत्वा तत्र युधिष्ठिरः}


\threelineshloka
{किमब्रवीन्महाबाहुः सर्वशस्त्रभृतां वरः}
{भीमसेनार्जुनौ वापि कृष्णो वा प्रत्यभाषत ॥वैशंपायन उवाच}
{}


\twolineshloka
{आपद्धर्मार्थकुशलो महाबुद्धिर्युधिष्ठिरः}
{सर्वान्भ्रातॄन्समानीय वासुदेवं च शाश्वतम्}


\twolineshloka
{उवाच वदतां श्रेष्ठः सान्त्वपूर्वमिदं वचः}
{पर्याक्रामत सैन्यानि यत्तास्तिष्ठत दंशिताः}


\threelineshloka
{पितामहेन वो युद्धं पूर्वमेव भविष्यति}
{तस्मात्सप्तसु सेनासु प्रणेतॄन्मम पश्यत ॥कृष्ण उवाच}
{}


\twolineshloka
{यथार्हति भवान्वक्तुमस्मिन्काले ह्युपस्थिते}
{तथेदमर्थवद्वाक्यमुक्तं ते भरतर्षभ}


\threelineshloka
{रोचते मे महाबाहो क्रियतां यदनन्तरम्}
{नायकास्तव सेनायां क्रियन्तामिह सप्त वै ॥वैशंपायन उवाच}
{}


\threelineshloka
{ततो द्रुपदमानाय्य विराटं शिनिपुङ्गवम्}
{धृष्टद्युम्नं च पाञ्चाल्यं धृष्टकेतुं च पार्थिव}
{शिखण्डिनं च पाञ्चाल्यं सहदेवं च मागधम्}


\twolineshloka
{एतान्सप्त महाभागान्वीरान्युद्धाभिकाङ्क्षिणः}
{सेनाप्रणेतॄन्विधिवदभ्यषिञ्चद्युधिष्ठिरः}


\twolineshloka
{सर्वसेनापतिं चात्र धृष्टद्युम्नं चकार ह}
{द्रोणान्तहेतोरुत्पन्नो य इद्धाञ्जातवेदसः}


\twolineshloka
{सर्वेषामेव तेषां तु समस्तानां महात्मनाम्}
{सेनापतिपतिं चक्रे गडाकेशं धनञ्जयम्}


\twolineshloka
{अर्जुनस्यापि नेता च संयन्ता चैव वाजिनाम्}
{सङ्कर्णणानुजः श्रीमान्महाबुद्धिर्जनार्दनः}


\twolineshloka
{तद्दृष्ट्वोपस्थितं युद्धं समासन्नं महात्ययम्}
{प्राविशद्भवनं राज्ञः पाण्डवानां हलायुधः}


\twolineshloka
{सहाक्रूरप्रभृतिभिर्ददसाम्बोद्धवादिभिः}
{रौक्मिणेयाहुकसुतैश्चारुदेष्णपुरोगमैः}


\twolineshloka
{वृष्णिमुख्यैरधिगतैर्व्याघ्रैरिव बलोत्कटैः}
{अभिगुप्तो महाबाहुर्मरुद्भिरिव वासवः}


\twolineshloka
{नीलकौशेयवसनः कैलासशिखरोपमः}
{सिंहखेलगतिः श्रीमान्मदरक्तान्तलोचनः}


\twolineshloka
{तं दृष्ट्वा धर्मराजश्च केशवश्च महाद्युतिः}
{उदतिष्ठत्ततः पार्थो भीमकर्मा वृकोदरः}


\twolineshloka
{गाण्डीवधन्वा ये चान्ये राजानस्तत्र केचन}
{पूजयाञ्चक्रिरे ते वै समायान्तं हलायुधम्}


\twolineshloka
{ततस्तं पाण्डवो राजा करे पस्पर्श पाणिना}
{वासुदेवपुरोगास्तं सर्व एवाभ्यवादयन्}


\twolineshloka
{विराटद्रुपदौ वृद्धावभिवाद्य हलायुधः}
{युधिष्ठिरेण सहित उपाविशदरिन्दमः}


\twolineshloka
{ततस्तेषूपविष्टेषु पार्थिवेषु समन्ततः}
{वासुदेवमभिप्रेक्ष्य रौहिणेयोऽभ्यभाषत}


\twolineshloka
{भवितायं महारौद्रो दारुणः पुरुषक्षयः}
{दिष्टमेतद्ध्रुवं मन्ये न शक्यमतिवर्तितुम्}


\twolineshloka
{तस्माद्युद्धात्समुत्तीर्णानपि वः समुहृञ्जानान्}
{अरोगानक्षतैर्देहैर्द्रष्टास्मीति मतिर्मम}


\twolineshloka
{समेतं पार्थिवं क्षत्रं कालपक्वमसंशयम्}
{विमर्दश्च महान्भावी मांसशोणितकदर्दमः}


\twolineshloka
{उक्तो मया वासुदेवः पुनः पुनरुपह्वरे}
{संबन्धिषु समां वृत्तिं वर्तस्व मधुसूदन}


\twolineshloka
{पाण्डवा हि यथास्माकं तथा दुर्योधनो नृपः}
{तस्यापि क्रियतां साह्यं स पर्येति पुनःपुनः}


\twolineshloka
{तच्च मे नाकरोद्वाक्यं त्वदर्थे मधुसूदनः}
{निर्विष्टः सर्वभावेन धनञ्जयमवेक्ष्य ह}


\twolineshloka
{ध्रुवो जयः पाण्डवानामिति मे निश्चिता मतिः}
{तथा ह्यभिनिवेशोऽयं वासुदेवस्य भारत}


\twolineshloka
{न चाहमृत्सहे कृष्णमृते लोकमुदीक्षितुम्}
{ततोऽहमनुवर्तामि केशवस्य चिकीर्षितम्}


\twolineshloka
{उभौ शिष्यौ हि मे वीरौ गदायद्धविशारदौ}
{तुल्यस्नेहोऽस्म्यतो भीते तथा दुर्योधने नृपे}


\threelineshloka
{तस्माद्यास्यामि तीर्थानि सरस्वत्या निषेवितुम्}
{न हि शक्ष्यामि कौरव्यान्नश्यमानानवेक्षितुम् ॥वैशंपायन उवाच}
{}


\twolineshloka
{एवमुक्त्वा महाबाहुरनुज्ञातश्च पाण्डवैः}
{तीर्थयात्रां ययौ रामो निवर्त्य मधुसूदनम्}


\chapter{अध्यायः १५८}
\twolineshloka
{वैशंपायन उवाच}
{}


\twolineshloka
{एतस्मिन्नैव काले तु भीष्मकस्य महात्मनः}
{हिरण्यरोम्णो नृपतेः साक्षादिन्द्रसखस्य वै}


\twolineshloka
{आकूतीनामधिपतिर्भोजस्यातियशस्विनः}
{दाक्षिणात्यपतेः पुत्रो दिक्षु रुक्मीति विश्रुतः}


\twolineshloka
{यः किंपुरुषसिंहस्य गन्धमादनवासिनः}
{कृत्स्नं शिष्यो धनुर्वेदं चतुष्पादमवाप्तवान्}


\twolineshloka
{यो माहेन्द्रं धनुर्लेभे तुल्यं गाण्डीवतेजसा}
{शार्ङ्गेण च महाबाहुः संमितं दिव्यलक्षणम्}


\threelineshloka
{त्रीण्येवैतानि दिव्यानि धनूंषि दिविचारिणाम्}
{वारुणं गाण्डिवं तत्र माहेन्द्रं विजयं धनुः}
{शार्ङ्ग तु वैष्णवं प्राहुर्दिव्यं तेजोमयं धनुः}


\twolineshloka
{धारयामास तत्कृष्णः परसेनाभयावहम्}
{गाण्डीवं पावकाल्लेभे खाण्डवे पाकशासनिः}


\twolineshloka
{द्रुमाद्रुक्मी महातेजा विजयं प्रत्यपद्यत}
{संछिद्य मौरवान्पाशान्निहत्य मुरमोजसा}


\twolineshloka
{निर्जित्य नरकं भौममाहृत्य मणिकुण्डले}
{षोडश स्त्रीसहस्राणि रत्नानि विविधानि च}


\twolineshloka
{प्रतिपेदे हृषीकेशः शार्ङ्गं च धनुरुत्तमम्}
{रुक्मी तु विजयं लब्ध्वा धनुर्मेघनिभस्वनम्}


\twolineshloka
{विभीषयन्निव जगत्पाण्डवानभ्यवर्तत}
{नामृष्यत पुरा योऽसौ स्वबाहुलगर्वितः}


\twolineshloka
{रुक्मिण्या हरणं वीरो वासुदेवेन धीमता}
{कृत्वा प्रतिज्ञां नाहत्वा निवर्तिष्ये जनार्दनम्}


\twolineshloka
{ततोऽन्वधावद्वार्ष्णेयं सर्वशस्त्रभृतां वरः}
{सेनया चतुरङ्गिण्या महत्या दूरपातया}


\twolineshloka
{विचित्रायुधवर्मिण्या गङ्ग्येव प्रवृद्धया}
{स समासाद्य वार्ष्णेयं योगानामीश्वरं प्रभुम्}


\twolineshloka
{व्यंसितो व्रीडितो राजन्नाजगाम स कुण्डिनम्}
{यत्रैव कृष्णेन रणे निर्जितः परवीरहा}


\twolineshloka
{तत्र भोजकटं नाम कृतं नगरमुत्तमम्}
{सैन्येन महता तेन प्रभूतगजवाजिना}


\twolineshloka
{पुरं तद्भुवि विख्यातं नाम्ना भोजकटं नृप}
{स भोजराजः सैन्येन महता परिवारितः}


\twolineshloka
{अक्षौहिण्या महावीर्यः पाण्डवान्क्षिप्रमागमत्}
{ततः स कवची धन्वी तली खङ्गी शरासनी}


\twolineshloka
{रथेनादित्यवर्णेन प्रविवेश महाचमूम्}
{विदितः पाण्डवेयानां वासुदेवप्रियेप्सया}


\twolineshloka
{युधिष्ठिरस्तु तं राजा प्रत्युद्गम्याभ्यपूजयत्}
{स पूजितः पाण्डुपुत्रैर्यथान्यायं सुसंस्तुतः}


\twolineshloka
{प्रतिगृह्य तु तान्सर्वान्विश्रान्तः सहसैनिकः}
{उवाच मध्ये वीराणां कुन्तीपुत्रं धनञ्जयम्}


\twolineshloka
{सहायोस्मि स्थितो युद्धे यदि भीतोसि पाण्डव}
{करिष्यामि रणे साह्यमसह्यं तव शत्रुभिः}


\twolineshloka
{न हि मे विक्रमे तुल्यः पुमानस्तीह कश्चन}
{हनिष्यामिरणे भागं यन्मे दास्यसि पाण्डव}


\twolineshloka
{अपि द्रोणकृपौ वीरौ भीष्मकर्णावथो पुनः}
{अथवा सर्व एवैते तिष्ठन्तु वसुधाधिपाः}


\twolineshloka
{निहत्य समरे शत्रूंस्तव दास्यामि मेदिनीम्}
{इत्युक्तो धर्मराजस्य केशवस्य च संनिधौ}


\threelineshloka
{श्रृण्वतां पार्थिवेन्द्राणामन्येषां चैव सर्वशः}
{वासुदेवमभिप्रेक्ष्य धर्मराजं च पाण्डवम्}
{उवाच धीमान्कौन्तेयः प्रहस्य सखिपूर्वकम्}


\twolineshloka
{युध्यमानस्य मे वीर गन्धर्वैः सुमहाबलैः}
{सहायो घोषयात्रायां कस्तदासीत्सखा मम}


\twolineshloka
{तथा प्रतिभये तस्मिन्देवदानवसंकुले}
{खाण्डवे युध्यमानस्य कः सहायस्तदावभवत्}


\twolineshloka
{निवातकवचैर्युद्धे कालकेयैश्च दानवैः}
{तत्र मे युध्यमानस्य कः सहायस्तदाभवत्}


\twolineshloka
{तथा विराटनगरे कुरुभिः सह सङ्गरे}
{युध्यतो बहुभिस्तत्र कः सहायोऽभवन्मम}


\twolineshloka
{उपजीव्य रणे रुद्रं शक्रं वैश्रवणं यमम्}
{वरुणं पावकं चैव कृपं द्रोणं च माधवम्}


\twolineshloka
{धारयन्गाण्डिवं दिव्यं धनुस्तेजोमयं दृढम्}
{अक्षय्यशरसंयुक्तो दिव्यास्त्रपरिबृंहितः}


\twolineshloka
{कौरवामां कुले जातः पाण्डोः पुत्रो विशेषतः}
{द्रोणं व्यपदिशञ्छिष्यो वासुदेवसहायवान्}


\twolineshloka
{कथमस्मद्विधो ब्रूयाद्भीतोऽस्मीति यशोहरम्}
{वचनं नरशार्दूल वज्रायुधसमस्वनम्}


\twolineshloka
{नास्मि भीतो महाबाहो सहायार्थश्च नास्ति मे}
{यथाकामं यथायोगं गच्छ वात्रैव तिष्ठ वा ॥`वैशंपायन उवाच}


\threelineshloka
{तच्छ्रुत्वा वचनं तस्य विजयस्य च धीमतः}
{'विनिवर्त्य ततो रुक्मी सेनां सागरसंनिभाम्}
{दुर्योधनमुपागच्छत्तथैव भरतर्षभ}


\twolineshloka
{तथैव चाभिगम्यैनमुवाच वसुधाधिपः}
{प्रत्याख्यातश्च तेनापि स तदा शूरमानिना}


\twolineshloka
{द्वावेव तु महाराज तस्माद्युद्धादपेयतुः}
{रौहिणेयश्च वार्ष्णेयो रुक्मी च वसुधाधिप}


\twolineshloka
{गते रामे तीर्थयात्रां भीष्मकस्य सुते तथा}
{उपाविशन्पाण्डवेया मन्त्राय पुनरेव च}


\twolineshloka
{समितिर्धर्मराजस्य सा पार्थिवसमाकुला}
{शुशुभे तारकैश्चित्रा द्यौश्चन्द्रेणेव भारत}


\chapter{अध्यायः १५९}
\twolineshloka
{जनमेजय उवाच}
{}


\threelineshloka
{तथा व्यूढेष्वनीकेषु कुरुक्षेत्रे द्विजर्षभ}
{किमर्कुर्वंश्च कुरवः कालेनाभिप्रचोदिताः ॥वैशंपायन उवाच}
{}


\twolineshloka
{तथा व्यूढेष्वनीकेषु यत्तेषु भरतर्षभ}
{धृतराष्ट्रो महाराज सञ्जयं वाक्यमब्रवीत्}


\twolineshloka
{एहि सञ्जय सर्व मे आचक्ष्वानवशेषतः}
{सेनानिवेशे यद्वृत्तं कुरुपाण्डवसेनयोः}


\twolineshloka
{दिष्टमेव परं मन्ये पौरुषं चाप्यनर्थकम्}
{यदहं बुद्ध्यमानोऽपि युद्धदोषान्क्षयोदयान्}


\twolineshloka
{तथापि निकृतिप्रज्ञं पुत्रं दुर्द्यूतदेविनम्}
{न शक्नोमि नियन्तुं वा कर्तुं वा हितमात्मनः}


\twolineshloka
{भवत्येव हि मे सूत बुद्धिर्दोषानुदर्शिनी}
{दुर्योधनं समासाद्य पुनः सा परिवर्तते}


\threelineshloka
{एवं गते वै यद्भावि तद्भविष्यति सञ्जय}
{क्षत्रधर्मः किल रणे तनुत्यागे हि पूजितः ॥सञ्जय उवाच}
{}


\twolineshloka
{त्वद्युक्तोऽयमनुप्रश्नो महाराज यथेच्छसि}
{न तु दुर्योधने दोषमिममाधातुमर्हसि}


\threelineshloka
{शृणुष्वानवशेषेण वदतो मम पार्थिव}
{य आत्मनो दुश्चरितादशुभं प्राप्नुयान्नरः}
{न स कालं न वा देवानेनसा गन्तुमर्हति}


\twolineshloka
{महाराज मनुष्येषु निन्द्यं यः सर्वमाचरेत्}
{स वध्यः सर्वलोकस्य निन्दितानि समाचरन्}


\twolineshloka
{निकारा मनुजश्रेष्ठ पाण्डवैस्त्वत्प्रतीक्षया}
{अनुभूताः सहामात्यैर्निकृतैरधिदेवने}


\twolineshloka
{हयानां च गजानां च राज्ञां चामिततेजसाम्}
{वैशसं समरे वृत्तं यत्तन्मे शृणु सर्वशः}


\twolineshloka
{स्थिरो भूत्वा महाप्राज्ञ सर्वलोकक्षयोदयम्}
{यथाभूतं महायुद्धे श्रुत्वा मा विमना भव}


\twolineshloka
{न ह्येव कर्ता पुरुषः कर्मणोः शुभपापयोः}
{अस्वतन्त्रो हि पुरुषः कार्यते दारुयन्त्रवत्}


\threelineshloka
{केचिदीश्वरनिर्दिष्टाः केचिदेव यदृच्छया}
{पूर्वकर्मभिरप्यन्ये त्रैधमेतत्प्रदृश्यते}
{तस्मादनर्थमापन्नः स्थिरो भूत्वा निशामय}


\chapter{अध्यायः १६०}
\twolineshloka
{सञ्जय उवाच}
{}


\twolineshloka
{हिरण्वत्यां निविष्टेषु पाण्डवेषु महात्मसु}
{न्यविशन्त महाराज कौरवेया यथाविधि}


\twolineshloka
{तत्र दुर्योधनो राजा निवेस्य बलमोजसा}
{संमानयित्वा नृपतीन्न्यस्य गुल्मांस्तथैव च}


\twolineshloka
{आरक्षस्य विधिं कृत्वा योधानां तत्र भारत}
{कर्णं दुःशासनं चैव शकुनिं चापि सौबलम्}


\twolineshloka
{आनाय्य नृपतिस्तत्र मन्त्रयामास भारत}
{तत्र दुर्योधनो राजा कर्णेन सह भारत}


\twolineshloka
{सौबलेन च राजेन्द्र मन्त्रयित्वा नरर्षभ}
{आहूयोपह्वरे राजन्नुलूकमिदमब्रवीत्}


\twolineshloka
{उलूक गच्छ कैतव्य पाण्डवान्सहसोमकान्}
{गत्वा मम वचो ब्रूहि वासुदेवस्य श्रृण्वतः}


\twolineshloka
{इदं तत्समनुप्राप्तं वर्षपूगाभिचिन्तितम्}
{पाण्डवानां कुरूणां च युद्धं लोकभयङ्करम्}


\twolineshloka
{यदेतत्कत्थनावाक्यं सञ्जयो महदब्रवीत्}
{वासुदेवसहायस्य गर्जतः सानुजस्य ते}


\twolineshloka
{मध्ये करूणां कौन्तेय तस्य कालोयमागतः}
{यथावत्संप्रतिज्ञातं तत्सर्वं क्रियतामिति}


% Check verse!
ज्येष्ठं तथैव कौन्तेयं ब्रूयास्त्यं वचनान्मम
\threelineshloka
{भ्रातृभिः सहितः सर्वैः सोमकैश्च सकेकयैः}
{कथं वा धार्मिको भूत्वा त्वमधर्मे मनः कृथाः}
{य इच्छसि जगत्सर्वं नश्यमानं नृशंसवत्}


% Check verse!
अभयं सर्वभूतेभ्यो पाता त्वमिति मे मतिः
\twolineshloka
{श्रूयते हि पुरा गीतः श्लोकोऽयं भरतर्षभ}
{प्रह्लादेनाथ भद्रं ते हृते राज्ये तु दैवतैः}


\twolineshloka
{यस्य धर्मध्वजो नित्यं सुरा ध्वज इवोच्छ्रितः}
{प्रच्छन्नानि च पापानि बैडालं नाम तद्ब्रतम्}


\twolineshloka
{अत्र ते वर्तयिष्यामि आख्यानमिदमुत्तमम्}
{कथितं नारदेनेह पितुर्मम नराधिप}


\twolineshloka
{मार्जारः किल दुष्टात्मा निश्चेष्टः सर्वकर्मसु}
{ऊर्ध्वबाहुः स्थितो राजन् गङ्गातीर कदाचन}


\twolineshloka
{स वै कृत्वा मनःशुद्धिं प्रत्ययार्थं शरीरिणाम्}
{करोमि धर्ममित्याह सर्वानेव शरीरिणः}


\twolineshloka
{तस्य कालेन महता विस्त्रम्यं जग्मुरण्डजाः}
{समेत्य च प्रशंसन्ति मार्जारं तं विशांपते}


\twolineshloka
{पूज्यमानस्तु तैः सर्वैः पक्षिभिः पक्षिभोजनः}
{आत्मकार्यं कृतं मेने चार्यायाश्च कृतं फलम्}


\twolineshloka
{अथ दीर्घस्य कालस्य तं देशं मूषिका ययुः}
{ददृशुस्तं च ते तत्र धार्मिकं व्रतचारिणम्}


\twolineshloka
{कार्येण महता युक्तं दम्भयुक्तेन भारत}
{तेषां मतिरियं राजन्नासीत्तत्र विनिश्चये}


\twolineshloka
{बहुमित्रा वयं सर्वे तेषां नो मातुलो ह्ययम्}
{रक्षां करोतु सततं वृद्धबालस्य सर्वशः}


\twolineshloka
{उपगम्य तु ते सर्वे बिडालमिदमब्रुवन्}
{भवत्प्रसादादिच्छामश्चर्तुं चैव यथासुखम्}


\twolineshloka
{भवान्नो गतिरव्यग्रा भवान्नः परमः सुहृत्}
{ते वयं सहिताः सर्वे भवन्तं शरणं गताः}


\twolineshloka
{भवान्धर्मपरो नित्यं भवान्धर्मे व्यवस्थितः}
{स नो रक्ष महाप्रज्ञ त्रिदशानिव वज्रभृत्}


\twolineshloka
{एवमुक्तस्तु तैः सर्वैर्मूषिकैः स विशांपते}
{प्रत्युवाच ततः सर्वान्मूषिकान्मूषिकान्तकृत्}


\twolineshloka
{द्वयोर्योगं न पश्यामि तपसो रक्षणस्य च}
{अवश्यं तु मया कार्यं वचनं भवतां हितम्}


\twolineshloka
{युष्माभिरपि कर्तव्यं वचनं मम नित्यशः}
{तपसास्मि परिश्रान्तो दृढं नियमामास्थितः}


\twolineshloka
{न चापि गमने शक्तिं कांचित्पश्यामि चिन्तयन्}
{सोस्मि नेयः सदा ताता नदीकूलमितःपरम्}


\twolineshloka
{तथेति तं प्रतिज्ञाय मूषिका भरतर्षभ}
{वृद्धबालमथो सर्वं मार्जाराय न्यवेदयन्}


\twolineshloka
{ततः स पापो दुष्टात्मा मूषिकानथ भक्षयन्}
{पीवरश्च सुवर्णश्च दृढबन्धश्च जायते}


\twolineshloka
{मूषिकाणां गणश्चात्र भृशं संक्षीयतेऽथ सः}
{मार्जारो वर्धते चापि तेजोबलसमन्वितः}


\twolineshloka
{ततस्ते मूषिकाः सर्वे समेत्यान्योन्यमब्रुवन्}
{मातुलो वर्धते नित्यं वयं क्षीयामहे भृशम्}


\twolineshloka
{ततः प्राज्ञतमः कश्चिड्डिण्डिको नाम मूषिकः}
{अब्रवीद्वचनं राजन्मूषिकाणां महागणम्}


\twolineshloka
{गच्छतां वो नदीतीरं सहितानां विशेषतः}
{पृष्ठतोऽहं गमिष्यामि सहैव मातुलेन तु}


\twolineshloka
{साधु साध्विति ते सर्वे पूजयाञ्चक्रिरे तदा}
{चक्रुश्चैव यथान्यायं डिण्डिकस्य वचोऽर्थवत्}


\twolineshloka
{अविज्ञानात्ततः सोऽथ डिण्डिकं ह्युपभुक्तवान्}
{ततस्ते सहिताः सर्वे मन्त्रयामासुरञ्जसा}


\twolineshloka
{तत्र वृद्धतमः कश्चित्कोलिको नाम मूषिकः}
{अब्रवीद्वचनं राजञ्ज्ञातिमध्ये यथातथम्}


\twolineshloka
{न मातुलो धर्मकामश्छद्ममात्रं कृता शिखा}
{न मूलफलभक्षस्य विष्ठा भवति लोमशा}


\twolineshloka
{अस्य गात्राणि वर्धन्ते गणश्च परिहीयते}
{अद्य सप्ताष्टदिवसान्डिण्डिकोपि न दृश्यते}


\twolineshloka
{एतच्छ्रुत्वा वचः सर्वे मूषिका विप्रदुद्रुवुः}
{बिडालोपि स दुष्टात्मा जगामैव यथागतम्}


\twolineshloka
{तथा त्वमपि दुष्टात्मन्बैडालं व्रतमास्थितः}
{चरसि ज्ञातिषु सदा बिडालोमूषिकेष्विव}


\twolineshloka
{अन्यथा किल ते वाक्यमन्यथा कर्म दृश्यते}
{दम्भनार्थाय लोकस्य वेदाश्चोपशमश्च ते}


\twolineshloka
{त्यक्त्वा छद्म त्विदं राजन्क्षत्रधर्मं समाश्रितः}
{कुरु कार्याणि सर्वाणि धर्मिष्ठोऽसि नरर्षभ}


\twolineshloka
{बाहुवीर्येण पृथिवीं लब्ध्वा भरतसत्तम}
{देहि दानं द्विजातिभ्यः पितृभ्यश्चयथोचितम्}


\twolineshloka
{क्लिष्टाया वर्षपूगांश्च मातुर्मातृहिते स्थितः}
{प्रमार्जाश्रु रणे जित्वा संमानं परमावह}


\twolineshloka
{पञ्चग्रामा वृता यत्नात्रास्माभिरपवर्जिताः}
{युध्यामहे कथं सङ्ख्ये कोपयेम च पाण्डवान्}


\twolineshloka
{त्वत्कृते दुष्टभावस्य संत्यागो विदुरस्य च}
{जातुषे च गृहे दाहं स्मर तं पुरुषो भव}


\twolineshloka
{यच्च कृष्णमवोचस्त्वमायान्तं कुरुसंसदि}
{अयमस्मि स्थितो राजञ्शमायसमराय च}


\twolineshloka
{तस्यायमागतः कालः समरस्य नराधिप}
{एतदर्थं मया सर्वं कृतमेतद्युधिष्ठिर}


\twolineshloka
{किं नु युद्धात्परं लाभं क्षत्रियो बहु मन्यते}
{किंच त्वं क्षत्रियकुले जातः संप्रथितो भुवि}


\twolineshloka
{द्रोणादस्त्राणि संप्राप्य कृपाच्च भरतर्षभ}
{तुल्ययोनौ समबले वासुदेवं समाश्रितः}


\twolineshloka
{ब्रूयास्त्वं वासुदेवं च पाण्डवानां समीपतः}
{आत्मार्थं पाण्डवार्थं च यत्तो मां प्रतियोधय}


\twolineshloka
{सभामध्ये च यद्रूपं मायया कृतवानसि}
{तत्तथैव पुनः कृत्वा सार्जुनो मामभिद्रव}


\twolineshloka
{इन्द्रजालं च मायां वै कुहका वापि भीषणा}
{आत्तशस्त्रस्य सङ्ग्रमे वहन्ति प्रतिगर्जनाः}


\twolineshloka
{वयमप्युत्सहेम द्यां खं च गच्छेम मायया}
{रसातलं विशामोपि ऐन्द्रं वा पुरमेव तु}


\twolineshloka
{दर्शयेम च रूपाणि स्वशरीरे बहून्यपि}
{न तु पर्यायतः सिद्धिर्बुद्धिमाप्नोति मानुषीम्}


\twolineshloka
{मनसैव हि भूतानि धातैव कुरुते वशे}
{यद्ब्रवीषि च वार्ष्णेय धार्तराष्ट्रानहं रणे}


\threelineshloka
{घातयित्वा प्रदास्यामि पार्थेभ्यो राज्यमुत्तमम्}
{आचचक्षे च मे सर्वं सञ्जयस्तव भाषितम्}
{मद्द्वितीयेन पार्थेन वैरं वः सव्यसाचिना}


\twolineshloka
{स सत्यसङ्गरो भूत्वा पाण्डवार्थे पराक्रमी}
{युद्ध्यस्वाद्य रणे यत्तः पश्यामः पुरुषो भव}


\twolineshloka
{यस्तु शत्रुमभिज्ञाय शुद्धं पौरुषमास्थितः}
{करोति द्विषतां शोकं स जीवति सुजीवितम्}


\twolineshloka
{अकस्माच्चैव ते कृष्ण ख्यातं लोके महद्यशः}
{अद्येदानीं विजानीमः सन्ति षण्डाः शश्रृङ्गकाः}


\twolineshloka
{मद्विधो नापि नृपतिस्त्वयि युक्तः कथंचन}
{सन्नाहं संयुगे कर्तुं कंसभृत्ये विशेषतः}


\twolineshloka
{तं च तूबरकं बालं बह्वाशिनमविद्यकम्}
{उलूक मद्वचो ब्रूहि असकृद्भीमसेनकम्}


\twolineshloka
{विराटनगरे पार्थ यस्त्वं सूदो ह्यभूः पुरा}
{बल्लवो नाम विख्यातस्तन्ममैव हि पौरुषम्}


\twolineshloka
{प्रतिज्ञातं सभामध्ये न तन्मिथ्या त्वया पुरा}
{दुःशासनस्य रुधिरं पीयतां यदि शक्यते}


\twolineshloka
{यद्ब्रवीषि च कौन्तेय धार्तराष्ट्रानहं रणे}
{निहनिष्यामि तरसा तस्य कालोऽयमागतः}


\twolineshloka
{त्वं हि भोज्ये पुरस्कार्यो भक्ष्ये पेये च भारत}
{क्व युद्धं क्व च भोक्तव्यं युध्यस्व पुरुषो भव}


\twolineshloka
{शयिष्यसे हतो भूमौ गदामालिङ्ग्य भारत}
{तद्वृथा च सभामध्ये वल्गितं ते वृकोदर}


\twolineshloka
{उलूक नकुलं ब्रूहि वचनान्मम भारत}
{युध्द्यस्वाद्य स्थिरो भूत्वा पश्यामस्तव पौरुषम्}


\twolineshloka
{युधिष्ठिरानुरागं च द्वेषं च मयि भारत}
{कृष्णायाश्च परिक्लेशं स्मरेदानीं यथातथम्}


\twolineshloka
{ब्रूयास्त्वं सहदेवं च राजमध्ये वचो मम}
{युद्ध्येदानीं रणे यत्तः क्लेशान्स्मर च पाण्डव}


\threelineshloka
{विराटद्रुपदौ चोभौ ब्रूयास्त्वं वचनान्मम}
{न दृष्टपूर्वा भर्तारो भृत्यैरपि महागुणैः}
{तथार्थपतिभिर्भृत्या यतः सृष्टाः प्रजास्ततः}


% Check verse!
अश्लाघ्योऽयं नरपतिर्युवयोरिति चागतम्
\twolineshloka
{ते यूयं संहता भूत्वा तद्वधार्थं ममापि च}
{आत्मार्थं पाण्डवार्थं च प्रयुद्ध्यध्वं मया सह}


\twolineshloka
{धृष्टद्युम्नं च पाञ्चाल्यं ब्रूयास्त्वं वचनान्मम}
{एष ते समयः प्राप्तो लब्धव्यश्च त्वयापि सः}


\twolineshloka
{द्रोणमासाद्य समरे ज्ञास्यसे हितमुत्तमम्}
{युद्ध्यस्व समुहृत्पापं कुरु कर्म सुदुष्करम्}


\twolineshloka
{शिखण्डिनमथो ब्रूहि उलूक वचनान्मम}
{स्त्रीति मत्वा महाबाहुर्न हनिष्यति कौरवः}


\twolineshloka
{गाङ्गेयो धन्विनां श्रेष्ठो युद्ध्येदानीं सुनिर्भयः}
{कुरु कर्म रणे यत्तः पश्यामः पौरुषं तव}


\twolineshloka
{एवमुक्त्वा ततो राजा प्रहस्योलूकमब्रवीत्}
{धनञ्जयं पुनर्ब्रूहि वासुदेवस्य श्रृण्वतः}


\twolineshloka
{अस्मान्वा त्वं पराजित्य प्रसाधि पृतिवीमिमाम्}
{अथवा निर्जितोस्माभी रणे वीर शयिष्यसि}


\twolineshloka
{राष्ट्रान्निर्वासनक्लेशं वनवासं च पाण्डव}
{कृष्णायाश्च परिक्लेशं संस्मरन्पुरुषो भव}


\twolineshloka
{यदर्थं क्षत्रिया सूते सर्वं तदिदमागतम्}
{बलं वीर्यं च शौर्यं च परं चाप्यस्त्रलाघवम्}


\threelineshloka
{पौरुषं दर्शयन्युद्धे कोपस्य कुरु निष्कृतिम्}
{परिक्लिष्टस्य दीनस्य दीर्घकालोषितस्य च}
{हृदयं कस्य न स्फोटेदैश्वर्याद्धंशितस्य च}


\twolineshloka
{कुले जातस्य शूरस्य परिवित्तेष्वगृध्यतः}
{आस्थितं राज्यमाक्रम्य कोपं कस्य न दीपयेत्}


\twolineshloka
{यत्तदुक्तं महद्वाक्यं कर्मणा तद्विभाव्यताम्}
{अकर्मणा कत्थितेन सन्तः कुपुरुषं विदुः}


\twolineshloka
{अमित्राणां वशे स्थानं राज्यं च पुनरुद्धर}
{द्वावर्थौ युद्धकामस्य तस्मात्तत्कुरु पौरुषम्}


\twolineshloka
{पराजितोऽसि द्यूतेन कृष्णा चानायिता सभाम्}
{शक्योऽमर्षो मनुष्येण कर्तु पुरुषमानिना}


\twolineshloka
{द्वादशैव तु वर्षाणि वने धिष्ण्याद्विवासितः}
{संवत्सरं विराटस्य दास्यमास्थाय चोषितः}


\twolineshloka
{राष्ट्रान्निर्वासनक्लेशं वनवासं च पाण्डव}
{कृष्णायाश्च परिक्लेशं संस्मरन्पुरुषो भव}


\twolineshloka
{अप्रियाणां च वचनं प्रब्रुवत्सु पुनः पुनः}
{अमर्षं दर्शयस्व त्वममर्षो ह्येव पौरुषम्}


\twolineshloka
{क्रोधो बलं तथा वीर्यं ज्ञानयोगोऽस्त्रलाघवम्}
{इह ते दृश्यतां पार्थ युद्ध्यस्व पुरुषो भव}


\twolineshloka
{लोहाभिसारो निर्वृत्तः कुरुक्षेत्रमकर्दमम्}
{पुष्टास्तेऽश्वा भृता योधाः श्वो युद्ध्यस्व सकेशवः}


\twolineshloka
{असमागम्य भीष्मेण संयुगे किं विकत्थसे}
{आरुरुक्षुर्यथा मन्दः पर्वतं गन्धमादनम्}


\twolineshloka
{एवं कत्थसि कौन्तेय अकत्थन्पुरुषो भव}
{सूतपुत्रं सुदुर्धर्षं शल्यं च बलिनां वरम्}


\twolineshloka
{द्रोणं च बलिनां श्रेष्ठं शचीपतिसमं युधि}
{अजित्वा संयुगे पार्थ राज्यं कथमिहेच्छसि}


\twolineshloka
{ब्राह्मे धनुषि चाचार्यं वेदयोरन्तगं द्वयोः}
{युधि धुर्यमविक्षोभ्यमनीकचरमच्युतम्}


\twolineshloka
{द्रोणं महाद्युतिं पार्थ जेतुमिच्छसि तन्मृषा}
{न हि शुश्रुम वातेन मेरुमुन्मथितं गिरिम्}


\twolineshloka
{अनिलो वा वहेन्मेरुं द्यौर्वापि निपतेन्महीम्}
{युगं वा रिवर्तेत यद्येवं स्याद्यथात्थ माम्}


\twolineshloka
{को ह्यस्ति जीविताकाङ्क्षी पराप्येममरिमर्दनम्}
{पार्थो वा इतरो वापि कोन्यः स्वस्ति गृहान्व्रजेत्}


\twolineshloka
{कथमाभ्यामभिध्यातः संस्पृष्टो दारुणेन वा}
{रणे जीवन्प्रमुच्येत पदा भूमिपुमस्पृशन्}


\twolineshloka
{किं दर्दुरः कूपशयो यथेमांन बुध्यसे राजचमूं समेताम्}
{दुराधर्षां देवचमूप्रकाशांगुप्तां नरेन्द्रैस्त्रिदशैरिव द्याम्}


\twolineshloka
{प्राच्यैः प्रतीच्यैरथ दाक्षिणात्यै-रुदीच्यकाम्भोजशकैः खशैश्च}
{साल्वैः समत्स्यैः कुरुमध्यदेश्यै-र्म्लेच्छैः पुलिन्दैर्द्रविडान्ध्रकाञ्च्यैः}


\twolineshloka
{नानाजनौघं युधि संप्रवृद्धंगाङ्गं यथा वेगमपारणीयम्}
{मां च स्थितं नानाबलस्य मध्येयुयुत्ससे मन्द किमल्पबुद्धे}


\twolineshloka
{अक्षय्याविषुधी चैव अग्निदत्तं च ते रथम्}
{जानीमो हि रणे पार्थ केतुं दिव्यं च भारत}


\twolineshloka
{अकत्थमानो युद्ध्यस्व कत्थसेऽर्जुन किं बहु}
{पर्यायात्सिद्धिरेतस्य नैतत्सिध्याति कत्थनात्}


\twolineshloka
{यदीदं कत्थनाल्लोके सिध्येत्कर्म धनञ्जय}
{सर्वे भवेयुः सिद्धार्थाः कत्थने को हि दुर्गतः}


\twolineshloka
{जनामि ते वासुदेवं सहायंजानामि ते गाण्डिवं तालमात्रम्}
{जानाम्यहं त्वादृशो नास्ति योद्धाजानानस्ते राज्यमेतद्धरामि}


\twolineshloka
{न तु पर्यायधर्मेम सिद्धिं प्राप्नोति मानवः}
{मनसैवानुकूलानि धातैव कुरुते वशे}


\twolineshloka
{त्रयोदशसमा भुक्तं राज्यं विलपतस्तव}
{भूयश्चैव प्रशासिष्ये त्वां निहत्य सबान्धवम्}


\twolineshloka
{क्व तदा गाण्डिवं तेऽभूद्यत्त्वं दासपणैर्जितः}
{क्व तदा भीमसेनस्य बलमासीच्च फाल्गुन}


\twolineshloka
{सगदाद्भीमसेनाद्वा फाल्गुनाद्वा सगाण्डिवात्}
{न वै मोक्षस्तदा यो भूद्विना कृष्णमनिन्दितां}


\twolineshloka
{सा वो दास्ये समापन्नान्मोचयामास पार्षती}
{अमानुष्यं समापन्नान्दासकर्मण्यवस्थितान्}


\twolineshloka
{अवोचं यत्षण्डतिलानहं वस्तथयमेव तत्}
{धृता हि वेमी पार्थेन विराटनगरे तदा}


\twolineshloka
{सूदकर्मणि च श्रान्तं विराटस्य महानसे}
{भीमसेनेन कौन्तेय यत्तु तन्मम पौरुषम्}


\twolineshloka
{एवमेव सदा दण्डं क्षत्रियाः क्षत्रिये दधुः}
{वेणीं कृत्वा षण्डवेषः कन्यां नर्तितवानसि}


\twolineshloka
{न भयाद्वासुदेवस्य न चापि तव फल्गुन}
{राज्यं प्रतिप्रदास्यामि युध्यस्व सहकेशवः}


\twolineshloka
{न माया हीन्द्रजालं वा कुहका वापि भीषणा}
{आत्तशस्त्रस्य सङ्ग्रामे वहन्ति प्रतिगर्जनाः}


\twolineshloka
{वासुदेवसहस्रं वा फाल्गुनानां शतानि वा}
{आसाद्य माममोघेषुं द्रविष्यन्ति दिशो दश}


\twolineshloka
{संयुगं गच्छ भीष्मेण भिन्धि वा शिरसा गिरिम्}
{तरस्व वा महागाधं बाहुभ्यां पुरुषोदधिम्}


\twolineshloka
{शारद्वतमहामीनं विविंशतिमहोरगम्}
{बृहद्बलमहोद्वेलं सौमदत्तितिमिङ्गिलम्}


\twolineshloka
{भीष्मवेगमपर्यन्तं द्रोणग्राहदुरसदम्}
{कर्णशल्यझषावर्तं काम्भोजबडबामुखम्}


\twolineshloka
{दुःशासनौघं शलशल्यमत्स्यंसुषेणचित्रायुधनागनक्रम्}
{जयद्रथाद्रिं पुरुमित्रगाधंदुर्मर्षणोदं शकुनिप्रपातम्}


\twolineshloka
{शश्त्रौघमक्षय्यमभिप्रवृद्धंयदावगाह्य श्रमनष्टचेताः}
{भविष्यसि त्वं हतसर्वबान्धव-स्तदा मनस्ते परितापमेष्यति}


\twolineshloka
{तदा मनस्ते त्रिदिवादिवाशुचे-र्निवर्तिता पार्थ महीप्रशासनात्}
{प्रशाम्य राज्यं हि सुदुर्लभं त्वयाबुभूषितः स्वर्ग इवातपस्विना}


\chapter{अध्यायः १६१}
\twolineshloka
{सञ्जय उवाच}
{}


\twolineshloka
{सेनानिवेशं संप्राप्तः कैतव्यः पाण्डवस्य ह}
{समागतः पाण्डवेयैर्युधिष्ठिरमभाषत}


\threelineshloka
{अभिज्ञो दूतवाक्यानां यथोक्तं ब्रुवतो मम}
{दुर्योधनसमादेशं श्रुत्वा न क्रोद्धुमर्हसि ॥युधिष्ठिर उवाच}
{}


\threelineshloka
{उलूक न भयं तेऽस्ति ब्रूहि त्वं विगतज्वरः}
{यन्मत्तं धार्तराष्ट्रस्य लुब्धस्यादीर्घदर्शिनः ॥सञ्जय उवाच}
{}


\twolineshloka
{ततो द्युतिमतां मध्ये पाण्डवानां महात्मनाम्}
{सृञ्जयानां च मत्स्यानां कृष्णस्य च यशस्विनः}


\threelineshloka
{द्रुपदस्य सपुत्रस्य विराटस्य च संनिधौ}
{भूमिपानां च सर्वेषां मध्ये वाक्यं जगाद ह ॥उलूक उवाच}
{}


\twolineshloka
{इदं त्वामब्रवीद्राजा धार्तराष्ट्रो महामनाः}
{श्रृण्वतां कुरुवीराणां तन्निबोध युधिष्ठिर}


\twolineshloka
{पराजितोऽसि द्यूतेन कृष्णा चानायिता सभाम्}
{शक्योऽमर्षो मनुष्येण कर्तुं पुरुषमानिना}


\twolineshloka
{द्वादशैव तु वर्षाणि वने धिष्ण्याद्विवासितः}
{संवत्सरं विराटस्य दास्यमास्थाय चोषितः}


\twolineshloka
{अमर्षं राज्यहरणं वनवासं च पाण्डव}
{द्रौपद्याश्च परिक्लेशं संस्मरन्पुरुषो भव}


\twolineshloka
{अशक्तेन च यच्छप्तं भीमसेनेन पाण्डव}
{दुःशासनस्य रुधिरं पीयतां यदि शक्यते}


\twolineshloka
{लोहाभिसारो निर्वृत्तः कुरुक्षेत्रमकर्दमम्}
{समः पन्था भृतास्तेश्वाः श्वो युध्यस्व सकेशवः}


\twolineshloka
{असमागम्य भीष्मेण संयुगे किं विकत्थसे}
{आरुरुक्षुर्यथा पङ्गुः पर्वतं गन्धमादनम्}


\twolineshloka
{एवं कत्थसि कौन्तेय अकत्थन्पुरुषो भव}
{सूतपुत्रं सुदुर्घर्षं शल्यं च बलिनां वरम्}


\twolineshloka
{द्रोणं च बलिनां श्रेष्ठं शचीपतिसमं युधि}
{अजित्वा संयुगे पार्थ राज्यं कथमिहेच्छसि}


\twolineshloka
{ब्राह्मे धनुषि चाचार्यं वेदयोरन्तगं द्वयोः}
{युधि धुर्यमविक्षोभ्यमनीकचरमच्युतम्}


\twolineshloka
{द्रोणं महाद्युतिं पार्थ जेतुमिच्छसि तन्मृषा}
{न हि शुश्रुम वातेन मेरुमुन्मथितं गिरिम्}


\twolineshloka
{अनिलो वा वहेन्मेरुं द्यौर्वापि निपतेन्महीम्}
{युगं वा परिवर्तेत यद्येवं स्याद्यथात्थ माम्}


\twolineshloka
{को ह्यस्ति जीविताकाङ्क्षी प्राप्येममरिमर्दनम्}
{गजो वाजी रथो वापि पुनः स्वस्ति गृहान्व्रजेत्}


\twolineshloka
{कथमाभ्यामभिध्यातः संसृष्टो दारुणेन वा}
{रणे जीवन्विमुच्येत पदा भूमिमुपस्पृशन्}


\twolineshloka
{किं दुर्दुरः कूपशयो यथेमांन बुध्यमे राजचमूं समेताम्}
{दुराधरषां देवचमूप्रकाशांगुप्तां नरेन्द्रैस्त्रिदशैरिव द्याम्}


\twolineshloka
{प्राच्यैः प्रतीच्यैरथ दाक्षिणात्यै-रुदीच्यकाम्भोजशकैः खशैश्च}
{साल्वैः समत्स्यैः कुरुमुख्यदेश्यै-र्म्लेच्छैः पुलिन्दैर्द्रविडान्ध्रकाञ्च्यैः}


\twolineshloka
{नानाजनौघं युधि संप्रवृद्धंनाङ्गं यथा वेगमपारकणीयम्}
{मां च स्थितं नागबलस्य मध्येयुयुत्समे मन्द किमल्पबुद्धे}


\twolineshloka
{इत्येवमुक्त्वा राजानं धर्मपुत्रं युधिष्ठिरम्}
{अभ्यवृत्य पुनर्जिष्णुमुलूकः प्रत्यभाषत}


\twolineshloka
{अकत्थमानो युध्यस्व कत्थेसेऽर्जुन किं बहु}
{पर्यायात्सिद्धिरेस्य नैतत्सिध्यति कत्थनात्}


\twolineshloka
{यदीदं कत्थनाल्लोके सिध्येत्कर्म धनञ्जय}
{सर्वे भवेयुः सिद्धार्थाः कत्थने को हि दुर्गतः}


\twolineshloka
{जानामि ते वासुदेवं सहायंजानामि ते गाण्डिवं तालमात्रम्}
{जानाम्येतत्त्वादृशो नास्ति योद्धाजानानस्ते राज्यमेतद्धरामि}


\twolineshloka
{न तु पर्यायधर्मेण राज्यं प्राप्नोति मानुषः}
{मनसैवानुकूलानि विधाता कुरुते वशे}


\twolineshloka
{त्रयोदश समा भुक्तं राज्यं विलपतस्तव}
{भूयश्चैव प्रशासिष्ये निहत्य त्वां सबान्धवम्}


\twolineshloka
{क्व तदा गाण्डिवं तेऽभूद्यत्त्वं दासपणैर्जितः}
{क्व तदा भीमसेनस्य बलमासीच्च फाल्गुना}


\twolineshloka
{सगदाद्भीमसेनाद्वा पार्थाद्वापि सगाण्डिवात्}
{न वै मोक्षस्तदा वोभूद्विना कृष्णामनिन्दिताम्}


\twolineshloka
{सा वो दास्ये समापन्नान्मोचयामास पार्षती}
{अमानुष्यं समापन्नान्दासकर्मण्यवस्थितान्}


\twolineshloka
{अवोचं यत्षण्डतिलानहं वस्तथ्यमेव तत्}
{धृता हि वेणी पार्थेन विराटनगरे तदा}


\twolineshloka
{सूदकर्मणि च श्रान्तं विराटस्य महानसे}
{भीमसेनेन कौन्तेय यच्च तन्मम पौरुषम्}


\twolineshloka
{एवमेतत्सदां दण्डं क्षत्रियाः क्षत्रिये दधुः}
{वेणीं कृत्वा षण्डवेषः कन्यां नर्तितवानसि}


\twolineshloka
{न भयाद्वासुदेवस्य न चापि तव फाल्गुन}
{राज्यं प्रतिप्रदाकस्यामि युद्ध्यस्व सहकेशवः}


\twolineshloka
{न माया हीन्द्रजालं वा कुहका वापि भीषणा}
{आत्तशस्त्रस्य मे युद्धे वहन्ति प्रतिगर्जनाः}


\twolineshloka
{वासुदेवसहस्रं वा फाल्गुनानां शतानि वा}
{आसाद्य माममोघेषुं द्रविष्यन्ति दिशो दशः}


\twolineshloka
{संयुगं गच्छ भूष्मेण भिन्धि वा शिरसा गिरिम्}
{तरेमं वा महागाधं सौमदत्तितिमिङ्गिलम्}


\twolineshloka
{शारद्वतमहामीनं विविंशतिमहोरगम्}
{बृहद्बलमहोद्वेलं सौमदत्तिकतिमिङ्गितम्}


\twolineshloka
{भीष्मवेगमपर्यन्तं द्रोणग्राहदुरासदम्}
{कर्णशल्यझषावर्तं काम्भोजबडबासुखम्}


\twolineshloka
{दुःशासनौघं शलशल्यमत्स्यंसुषेणचित्रायुधनागनक्रम्}
{जयद्रथाद्रिं पुरुमित्रगाधंदुर्मर्षणोदं शकुनिप्रपातम्}


\twolineshloka
{शस्त्रौघमक्षय्यमतिप्रवृद्धंयदावगाह्य श्रमनष्टचेताः}
{भविष्यसि त्वं हतसर्वबान्धव-स्तदा मनस्ते परितापमेष्यति}


\twolineshloka
{तदा मनस्ते त्रिदिवादिवाशुचे-र्निवर्तिता पार्थ महीप्रशासनात्}
{प्रशाम्य राज्यं हि सुदुर्लभं त्वयाबुभूषितः स्वर्ग इवातपस्विना}


\chapter{अध्यायः १६२}
\twolineshloka
{सञ्जय उवाच}
{}


\twolineshloka
{उलूकस्त्वर्जुनं भूयो यथोक्तं वाक्यमब्रवीत्}
{आशीविषमिव क्रुद्धं तुदन्वाक्यशलाकया}


\twolineshloka
{तस्य तद्वचनं श्रुत्वा रुषिताः पाण्डवा भृशम्}
{प्रागेव भृशसंक्रुद्धाः कैतव्येनापि धर्षिताः}


\twolineshloka
{आसनेषूदतिष्ठन्त बाहूंश्चैव प्रचिक्षिपुः}
{आशीविषा इव क्रुद्धा वीक्षाञ्चक्रुः परस्परम्}


\twolineshloka
{अवाक्छिरा भीमसेनः समुदैक्षत कैतवम्}
{नेत्राभ्यां लोहितान्ताभ्यामाशीविष इव श्वसन्}


\twolineshloka
{आर्तं वातात्मजं दृष्ट्वा क्रोधेनाभिहतं भृशम्}
{उत्स्मयन्निव दाशार्हः कैतव्यं प्रत्यभाषत}


\twolineshloka
{प्रयाहि शीघ्रं कैतव्य ब्रूयाश्चैव सुयोधनम्}
{श्रुतं वाक्यं गृहीतोर्थो मतं यत्ते तथास्तु तत्}


\twolineshloka
{एवमुक्त्वा महाबाहुः केशवो राजसत्तम}
{पुनरेव महाप्राज्ञं युधिष्ठिरमुदैक्षत}


\twolineshloka
{सृञ्जयानां च सर्वेषां कृष्णस्य च यशस्विनः}
{द्रुपदस्य सपुत्रस्य विराटस्य च संनिधौ}


\twolineshloka
{भूमिपानां च सर्वेषां मध्ये वाक्यं जगाद ह}
{उलूकोऽप्यर्जुनं भूयो यथोक्तं वाक्यमब्रवीत्}


\twolineshloka
{आशीविषमिव क्रुद्धं तुदन्वाक्यशलाकया}
{कृष्णादींश्चैव तान्सर्वान्यथोक्तं वाक्यमब्रवीत्}


\twolineshloka
{उलूकस्य तु तद्वाक्यं पापं दारुणमीरितम्}
{श्रुत्वा विचुक्षुभे पार्थो ललाटं चाप्यमार्जयत्}


\twolineshloka
{तदवस्थं तदा दृष्ट्वा पार्थं सा समितिर्नृप}
{नामृष्यत महाराज पाण्डवानां महारथाः}


\twolineshloka
{अधिक्षेपेण कृष्णस्य पार्थस्य च महात्मनः}
{श्रुत्वा ते पुरुषव्याघ्राः क्रोधाञ्जज्वलुरच्युत}


\twolineshloka
{धृष्टद्युम्नः शिखण्डी च सात्यकिश्च महारथः}
{केकया भ्रातरः पञ्च राक्षसश्च घटोत्कचः}


\twolineshloka
{द्रौपदेयाभिमन्युश्च धृष्टकेतुश्च पार्थिवः}
{भीमसेनश्च विक्रान्तो यमजौ च महारथौ}


\threelineshloka
{उत्पेतुरासनात्सर्वे क्रोधसंरक्तलोचनाः}
{बाहुन्प्रगृह्य रुचिरान्रक्तचन्दनरूषितान्}
{अङ्गदैः पारिहार्यैश्च केयूरैश्च विभूषितान्}


\twolineshloka
{दन्तान्दन्तेषु निष्पिष्य सृक्किणी परिलेलिहन्}
{तेषामाकारभावज्ञः कुन्तीपुत्रो वृकोदरः}


\twolineshloka
{उदतिष्ठत्स वेगेन क्रोधेन प्रज्वलन्निव}
{उद्वृत्य सहसा नेत्रे दन्तान्कटकटाय्य च}


\twolineshloka
{हस्तं हस्तेन निष्पिष्य उलूकं वाक्यमब्रवीत्}
{अशक्तानामिवास्माकं प्रोत्साहननिमित्तकम्}


\twolineshloka
{श्रुतं ते चवनं मूर्ख यत्त्वां दुर्योधनोऽब्रवीत्}
{तन्मे कथयतो मन्द श्रृणु वाक्यं दुरासदम्}


\twolineshloka
{सर्वक्षत्रस्य मध्ये तं यद्वक्ष्यसि सुयोधनम्}
{श्रृण्वतः सूतपुत्रस्य पितुश्च त्वं दुरात्मनः}


\twolineshloka
{अस्माभिः प्रीतिकामैस्तु भ्रातुर्ज्येष्ठस्य नित्यशः}
{मर्षितं ते दुराचार तत्त्वं न बहु मन्यसे}


\twolineshloka
{प्रेषितश्च हृषीकेशः शमाकाङ्क्षी कुरून्प्रति}
{कुलस्य हितकामेन धर्मराजेन धीमता}


\twolineshloka
{त्वं कालचोदितो नूनं गन्तुकामो यमक्षयम्}
{गच्छस्वाहवमस्माभिस्तच्च श्वो भविता ध्रुवम्}


\twolineshloka
{मयापि च प्रतिज्ञातो वधः सभ्रातृकस्य ते}
{स तथा भविता पाप नात्र कार्या विचारणा}


\twolineshloka
{वेलामतिक्रमेत्सद्यः सागरो वरुणालयः}
{पर्वताश्च विशीर्येयुर्मयोक्तं न मृषा भवेत्}


\threelineshloka
{सहायस्ते यदि यमः कुबेरो रुद्र एव वा}
{यथाप्रतिज्ञं दुर्बुद्धे प्रकरिष्यन्ति पाण्डवाः}
{दुःशासनस्य रुधिरं पाता चास्मि यथेप्सितम्}


\twolineshloka
{यश्चेह प्रतिसंरब्धः क्षत्रियो माभियास्यति}
{अपि भीष्मं पुरुस्कृत्य तं नेष्यामि यमक्षयम्}


\twolineshloka
{यच्चैतदुक्तं वचनं मया क्षत्रस्य संसदि}
{यथैतद्भविता सत्यं तथैवात्मानमालभे}


\twolineshloka
{भीमसेनवचः श्रुत्वा सहदेवोऽप्यमर्षणः}
{क्रोधसंरक्तनयनस्ततो वाक्यमुवाच ह}


\twolineshloka
{शौटीरशूरसदृशमनीकजनसंसदि}
{श्रृणु पाप वचो मह्यं यद्वाच्यो हि पितात्वया}


\twolineshloka
{नास्माकं भविता भेदः कदाचित्कुरुभिः सह}
{धृतराष्ट्रस्य संबन्धो यदि न स्यात्त्वया सह}


\twolineshloka
{त्वं तु लोकविनाशाय धृतराष्ट्रकुलस्य च}
{उत्पन्नो वैरपुरुषः स्वकुलघ्नश्च पापकृत्}


\twolineshloka
{जन्मप्रभृति चास्माकं पिता ते पापपूरुषः}
{अहितानि नृशंसानि नित्यशः कर्तुमिच्छति}


\twolineshloka
{तस्य वैरानुषङ्गस्य गन्तास्म्यन्तं सुदुर्गमम्}
{अहमादौ निहत्य त्वां शकुनेः संप्रपश्यतः}


\twolineshloka
{ततोऽस्मि शकुनिं हन्ता मिषतां सर्वधन्विनाम्}
{भीमस्य वचनं श्रुत्वा सहदेवस्य चोभयोः}


\twolineshloka
{उवाच फाल्गुनो वाक्यं भीमसेनं स्मयन्निव}
{भीमसेन न ते सन्ति येषां वैरं त्वया सह}


\twolineshloka
{मन्दा गृहेषु सुखिनो मृत्युपाशवशं गताः}
{उलूकश्च न ते वाच्यः परुषं पुरुषोत्तम}


\twolineshloka
{दूताः किमपराध्यन्ते यथोक्तस्यानुभाषिणः}
{एवमुक्त्वा महाबाहुर्भीमं भीमपराक्रमम्}


\twolineshloka
{धृष्टद्युम्नसुखान्वीरान्सुहृदः समभाषत}
{श्रुतं वस्तस्य पापस्य धार्तराष्ट्रस्य भाषितम्}


\twolineshloka
{कुत्सनं वासुदेवस्य मम चैव विशेषतः}
{श्रुत्वा भवन्तः संरब्धा अस्माकं हितकाम्यया}


\twolineshloka
{प्रभावाद्वासुदेवस्य भवतां च प्रयत्नतः}
{समग्रं पार्थिवं क्षत्रं सर्वं न गणयाम्यहम्}


\twolineshloka
{भवद्भिः समनुज्ञातो वाक्यमस्य यदुत्तरम्}
{उलूके प्रापयिष्यामि यद्वक्ष्यति सुयोधनम्}


\threelineshloka
{श्वो भूते कत्थितस्यास्य प्रतिवाक्यं चमूमुखे}
{गाण्डीवेनाभिधास्याकमि क्लीबा हि वचनोत्तराः}
{}


\twolineshloka
{ततस्ते पार्थिवाः सर्वे प्रशशंसुर्धनञ्जयम्}
{तेन वाक्योपचारेण विस्मिता राजसत्तमाः}


\twolineshloka
{अनुनीय च तान्सर्वान्यथामान्यं यथावयः}
{धर्मराजं तदा वाक्यं तत्प्राप्यं प्रत्यभाषत}


\twolineshloka
{आत्मानमवमन्वानो न हि स्यात्पार्थिवोत्तमः}
{तत्रोत्तरं प्रवक्ष्यामि तव शुश्रूषणे रतः}


\twolineshloka
{उलूकं भरतश्रेष्ठ सामपूर्वमथोर्जितम्}
{दुर्योधनस्य तद्वाक्यं निशम्य भरतर्षभः}


\twolineshloka
{अतिलोहितनेत्राभ्यामाशीविष इव श्वसन्}
{स्मयमान इव क्रोधात्सृक्किणी परिसंलिहन्}


\twolineshloka
{जनार्दनमभिप्रेक्ष्य भ्रातॄंश्चैवेदमब्रवीत्}
{अभ्यभाषत कैतव्यं प्रगृह्य विपुलं भुजम्}


\twolineshloka
{उलूक गच्छ कैतव्य ब्रूहि तात सुयोधनम्}
{कृतघ्नं वैरपुरुषं दुर्मतिं कुलपांसनम्}


\threelineshloka
{पाण्डवेषु सदा पाप नित्यं जिह्मं प्रवर्तते}
{स्ववीर्याद्यः पराक्रम्य पाप आह्वयते परान्}
{अभीतः पूरयन्वाक्यमेष वैः क्षत्रियः पुमान्}


\twolineshloka
{स पापः क्षत्रियो भूत्वा अस्मानाहूय संयुगे}
{मान्यामान्यान्पुरस्कृत्य युद्धं मा गाः कुलाधम}


\twolineshloka
{आत्मवीर्यं समाश्रिकत्य भृत्यवीर्यं च कौरव}
{आह्वयस्व रणे पार्थान्सर्वथा क्षत्रियो भव}


\twolineshloka
{परवीर्यं समाश्रित्य यः समाह्वयते परान्}
{अशक्तः स्वयमादातुमेतदेव नपुंसकम्}


\threelineshloka
{स त्वं परेषां वीर्येण आत्मानं बहुमन्यसे}
{कथमेवमशक्तस्त्वमस्मान्समभिगर्जसि ॥कृष्ण उवाच}
{}


\twolineshloka
{मद्वचश्चापि भूयस्ते वक्तव्यः स सुयोधनः}
{श्व इदानीं प्रपद्येथाः पुरुषो भव दुर्मते}


\twolineshloka
{मन्यसे यच्च मूढ त्वं न योत्स्यति जनार्दनः}
{सारथ्येन वृतः पार्थैरिति त्वं न बिभेषि च}


\twolineshloka
{जघन्यकालमप्येतन्न भवेत्सर्वपार्थिवान्}
{निर्दहेयमहं क्रोधात्तृणानीव हुताशनः}


\twolineshloka
{युधिष्ठिरनियोगात्तु फाल्गुनस्य महात्मनः}
{करिष्ये युध्यमानस्य सारथ्यं विजितात्मनः}


\twolineshloka
{यद्युत्पतसि लोकांस्त्रीन्यद्याविशसि भूतलम्}
{तत्रतत्रार्जुनरथं प्रभाते द्रक्ष्यसे पुनः}


\twolineshloka
{यच्चापि भीमसेनस्य मन्यसे मोघभाषितम्}
{दुःशासनस्य रुधिरं पीतमद्यावधारय}


\twolineshloka
{न त्वां समीक्षते पार्थो नापि राजा युधिष्ठिरः}
{न भीमसेनो न यमौ प्रतिकूलप्रभाषिणम्}


\chapter{अध्यायः १६३}
\twolineshloka
{सञ्जय उवाच}
{}


\twolineshloka
{दुर्योधनस्य तद्वाक्यं निशम्य भरतर्षभ}
{नेत्राभ्यामतिताम्राभ्यां कैतव्यं समुदैक्षत}


\twolineshloka
{स केशवमभिप्रेक्ष्य गुडाकेशो महायशाः}
{अभ्यभाषत कैतव्यं प्रगृह्य विपुलं भुजम्}


\twolineshloka
{स्ववीर्यं यः समाश्रित्य समाह्वयति वै परान्}
{अभीतो युध्यते शत्रून्स वै पुरुष उच्यते}


\twolineshloka
{परवीर्यं समाश्रित्य यः समाह्वयते परान्}
{क्षत्रबन्धुरशक्तत्वाल्लोके स पुरुषाधमः}


\twolineshloka
{स त्वं परेषां वीर्येण मन्यसे वीर्यमात्मनः}
{स्वयं कापुरुषो मूढ परांश्च क्षेप्तुमिच्छसि}


\twolineshloka
{यस्त्वं वृद्धं सर्वराज्ञां हितबुद्धिं जितेन्द्रियम्}
{मरणाय महाप्रज्ञं दीक्षयित्वा विकत्थसे}


\twolineshloka
{भावस्ते विदितोऽस्माभिर्दुर्बुद्धे कुलपांसन}
{न हनिष्यन्ति गाङ्गेयं पाण्डवा घृणयेति हि}


\threelineshloka
{यस्य वीर्यं समाश्रित्य धार्तराष्ट्र विकत्थसे}
{हन्तास्मि प्रथमं भीष्मं मिषतां सर्वधन्विनाम्}
{}


\twolineshloka
{कैतव्य गत्वा भरतान्समेत्यसुयोधनं धार्तराष्ट्रं वदस्व}
{तथेत्युवाचार्जुनः सव्यसाचीनिशाव्यपाये भविता विमर्दः}


\threelineshloka
{यद्वा ब्रवीद्वाक्यमदीनसत्वोमध्ये कुरून्हर्षयन्सत्यसन्धः}
{अहं हन्ता सृञ्जयानामनीकंसाल्वेयकांश्चेति ममैष भारः}
{कैतव्य गत्वा भरतान्समेत्यसुयोधनं धार्तराष्ट्रं वदस्व}


\twolineshloka
{हन्यामहं द्रोणमृतेऽपि लोकंन ते भयं विद्यते पाण्डवेभ्यः}
{ततो हि ते लब्धमतं च राज्य-मापद्गताः पाण्डवाश्चेति भावः}


\twolineshloka
{स दर्पपूर्णो न समीक्षसे त्व-मनर्थमात्मन्यपि वर्तमानम्}
{तस्मादहं ते प्रथमं समूहेहन्ता समक्ष कुरुवृद्धमेव}


\twolineshloka
{सूर्योदये युक्तसेनः प्रतीक्ष्यध्वजी रथी रक्षत सत्यसन्धम्}
{अहं हि वः पश्यतां द्विपमेनंभीष्मं रथात्पातयिष्यामि बाणैः}


\twolineshloka
{श्वोभूते कत्थनावाक्यं विज्ञास्यति सुयोधनः}
{आचितं शरजालेन मया दृष्ट्वा पितामहम्}


\twolineshloka
{यदुक्तश्च सभामध्ये पुरुषो ह्रस्वदर्शनः}
{क्रुद्धेन भीमसेनेन भ्राता दुःशासनस्तव}


\twolineshloka
{अधर्मज्ञो नित्यवैरी पापबुद्धिर्नृशंसकृत्}
{सत्यां प्रतिज्ञामचिराद्द्रक्ष्यसे तां सुयोधन}


\twolineshloka
{अभिमानस्य दर्पस्य क्रोधपारुष्ययोस्तथा}
{नैष्ठुर्यस्यावलेपस्य आत्मसंभावनस्य च}


\twolineshloka
{नृशंसतायास्तैक्ष्ण्यस्य धर्मविद्वेषणस्य च}
{अधर्मस्यातिवादस्य वृद्धातिक्रमणस्य च}


\twolineshloka
{दर्शनस्य च चक्रस्य कृत्स्नस्यापनयस्य च}
{द्रक्ष्यसि त्वं फलं तीव्रमचिरेण सुयोधन}


\twolineshloka
{वासुदेवद्वितीये हि मयि क्रुद्धे नराधम}
{आशा ते जीविते मूढ राज्ये वा केन हेतुना}


\twolineshloka
{शान्ते भीष्मे तथा द्रोणे सूतपुत्रे च पातिते}
{निराशो जीविते राज्ये पुत्रेषु च भविष्यसि}


\twolineshloka
{भ्रातॄणां निधनं श्रुत्वा पुत्राणां च सुयोधन}
{भीमसेनेन निहतो दुष्कृतानि स्मरिष्यसि}


\twolineshloka
{न द्वितीयां प्रतिज्ञां हि प्रतिजानामि कैतव}
{सत्यं ब्रवीम्यहं ह्येतत्सर्वं सत्यं भविष्यति}


\twolineshloka
{युधिष्ठिरोऽपि कैतव्यमुलूकमिदमब्रवीत्}
{उलूक मद्वचो ब्रूहि गत्वा तात सुयोधनम्}


\twolineshloka
{स्वेन वृत्तेन मे वृत्तं नाधिगन्तुं त्वमर्हसि}
{उभयोरन्तरं वेद सूनृतानृतयोरपि}


\twolineshloka
{न चाहं कामये पापमपि कीटपिपीलकयोः}
{किं पुनर्ज्ञातिषु वधं कामयेयं कथं च न}


\twolineshloka
{एतदर्थं मया तात पञ्च ग्रामा वृताः पुरा}
{कथं तव सुदुर्बुद्धे न प्रेक्ष्ये व्यसनं महत्}


\threelineshloka
{स त्वं कामपरीतात्मा मूढभावाच्च कत्थसे}
{तथैव वासुदेवस्य न गृह्णासि हितं वचः}
{}


\threelineshloka
{किंचेदानीं बहूक्तेन युध्यस्व सह बान्धवैः}
{मम विप्रियकर्तारं कैतव्य सह बान्धवैः}
{}


\twolineshloka
{मम विप्रियकर्तारं कैतव्य ब्रूहि कौरवम् ॥श्रुतं वाक्यं गृहीतोऽर्थो मतं यत्ते तथास्तु तत्}
{}


\twolineshloka
{भीमसेनस्ततो वाक्यं भूय आह नृपात्मजम् ॥उलूक मद्वचो ब्रूहि दुर्मतिं पापपूरुषम्}
{}


\twolineshloka
{शठं नैकृतिकं पापं दुराचारं सुयोधनम् ॥गृध्रोदरे वा वस्तव्यं पुरे वा नागसाह्वये}
{}


\twolineshloka
{प्रतिज्ञातं मया यच्च सभामध्ये नराधम ॥कर्ताहं तद्वचः सत्यं सत्येनैव शपामि ते}
{}


\twolineshloka
{दुःशासनस्य रुधिरं हत्वा पास्याम्यहं मृधे ॥सक्थिनी तव भंक्त्वैव हत्वा हि तव सोदनान्}
{}


\twolineshloka
{सर्वेषां राजपुत्राणामभिमन्युरसंशयम्}
{कर्मणा तोषयिष्यामि भूयश्चैव वचः श्रृणु}


\twolineshloka
{हत्वा सुयोधन त्वां वै सहितं सर्वसोदरैः}
{आक्रमिष्ये पदा मूर्ध्नि धर्मराजस्य पश्यतः}


\twolineshloka
{नकुलस्तु ततो वाक्यमिदमाह महीपते}
{उलूक ब्रूहि कौरव्यं धार्तराष्ट्रं सुयोधनम्}


\threelineshloka
{श्रुतं ते गदतो वाक्यं सर्वमेव यथातथम्}
{तथा कर्तास्मि कौरव्य यथा त्वमनुशासि मां ॥ 5-163-39aसहदेवोऽपिनृपते इदमाह वचोऽर्थवत्}
{सुयोधन मतिर्या ते वृथैषा ते भविष्यति}


\twolineshloka
{शोचिष्यसे महाराज सपुत्रज्ञातिबान्धवः}
{इमं च क्लेशमस्माकं हृष्टो यत्त्वं विकत्थसे}


\threelineshloka
{विराटद्रुपदौ वृद्धावुलूकमिदमूचतुः}
{दासभावं नियच्छेव साधोरिति मतिः सदा}
{तौ च दासावदासौ वा पौरुषं यस्य यादृशम्}


\twolineshloka
{शिखण्डी तु ततो वाक्यमुलूकमिदमब्रवीत्}
{वक्तव्यो भवता राजा पापेष्वभिरतः सदा}


\twolineshloka
{पश्य त्वं मां रणे राजन्कुर्वाणं कर्म दारुणम्}
{यस्य वीर्यं समासाद्य मन्यसे विजयं युधि}


\twolineshloka
{तमहं पातयिष्यामि रथात्तव पितामहम्}
{अहं भीष्मवधात्सृष्टो नूनं धात्रा महात्मना}


\twolineshloka
{सोऽहं भीष्मं हनिष्यामि मिषतां सर्वधन्विनाम्}
{धृष्टद्युम्नोऽपि कैतव्यमुलूकमिदमब्रवीत्}


\twolineshloka
{सुयोधनो मम वचो वक्तव्यो नृपतेः सुतः}
{अहं द्रोणं हनिष्यामि सगणं सहबान्धवम्}


\twolineshloka
{अवश्यं च मया कार्यं पूर्वेषां चरितं महत्}
{कर्ता चाहं तथा कर्म यथा नान्यः करिष्यति}


\twolineshloka
{तमब्रवीद्धर्मराजः कारुण्यार्थं वचो महत्}
{नाहं ज्ञातिवधं राजन्कामयेयं कथंचन}


\twolineshloka
{तवैव दोषाद्दुर्बुद्धे सर्वमेतत्त्वनावृतम्}
{स गच्छ माचिरं तात उलूक यदि मन्यसे}


\twolineshloka
{इह वा तिष्ठ भद्रं ते वयं हि तव बान्धवाः}
{उलूकस्तु ततो राजन्धर्मपुत्रं युधिष्ठिरम्}


\twolineshloka
{आमन्त्र्य प्रययौ तत्र यत्र राजा सुयोधनः}
{उलूकस्तत आगमक्य दुर्योधनममर्षणम्}


\twolineshloka
{अर्जुनस्य समादेशं यथोक्तं सर्वमब्रवीत्}
{वासुदेवस्य भीमस्य धर्मराजस्य पौरुषम्}


\threelineshloka
{नकुलस्य विराटस्य द्रुपदस्य च भारत}
{सहदेवस्य च वचो धृष्टद्युम्नशिखण्डिनोः}
{केशवार्जुनयोर्वाक्यं यथोक्तं सर्वमब्रवीत्}


\twolineshloka
{कैतव्यस्य तु तद्वाक्यं निशम्य भरतर्षभः}
{दुःशासनं च कर्णं च शकुनिं चापि भारत}


\twolineshloka
{आज्ञपयत राज्ञश्च बलं मित्रबलं तथा}
{यथा प्रागुदयात्सर्वे युक्तास्तिष्ठन्त्यनीकिनः}


\twolineshloka
{ततः कर्णसमादिष्टा दूताः सन्त्वरिता रथैः}
{उष्ट्रवामीभिरप्यन्ये सदश्वैश्च महाजवैः}


\twolineshloka
{तूर्णं परिययुः सेनां कृत्स्नां कर्णस्य शासनात्}
{आज्ञापयन्तो राज्ञश्च योगः प्रागुदयादिति}


\chapter{अध्यायः १६४}
\twolineshloka
{संजय उवाच}
{}


\twolineshloka
{उलूकस्य वृत्तः श्रुत्वा कुन्तीपुत्रो युधिष्ठिरः}
{सेनां नियापयामास धृष्टद्युम्नपुरोगमाम्}


\twolineshloka
{पदातिनीं नागवतीं रथिनीमश्वबृन्दिनीम्}
{चतुर्विधबलां भीमामकम्पां पृथिवीमिव}


\twolineshloka
{भीमसेनादिभिर्गुप्तां सार्जुनैश्च महारथैः}
{धृष्टद्युम्नवशां दुर्गां सागरस्तिमितोपमाम्}


\twolineshloka
{तस्यास्त्वग्रे महेष्वासः पाञ्चाल्यो युद्धदुर्मदः}
{द्रोणप्रेप्सुरनीकानि धृष्टद्युम्नो व्यकर्षत}


\twolineshloka
{यथाबलं यथोत्साहं रथिनः समुपादिशत्}
{अर्जुनं सूतपुत्राय भीमं दुर्योधनाय च}


\twolineshloka
{धृष्टकेतुं च शल्याय गौतमायोत्तमौजसम्}
{अश्वात्थाम्ने च नकुलं शैब्यं च कृतवर्मणे}


\twolineshloka
{सैन्धवाय च वार्ष्णेयं युयुधानं समादिशत्}
{शिखण्डिनं च भीष्माय प्रमुखे समकल्पयत्}


\twolineshloka
{सहदेवं शकुनये चेकितानं शलाय वै}
{द्रौपदेयांस्तथा पञ्च त्रिगर्तेभ्यः समादिशत्}


\twolineshloka
{वृषसेनाय सौभद्रं शेषाणां च महीक्षिताम्}
{स समर्थं हि तं मेने पार्थादभ्यधिकं रणे}


\twolineshloka
{एवं विभज्य योधांस्तान्पृथक्व सह चैव ह}
{ज्वालावर्णो महेष्वासो द्रोणमंशमकल्पयत्}


\twolineshloka
{धृष्टद्युम्नो महेष्वासः सेनापतिपतिस्ततः}
{विधिवद्व्यूह्य मेधावी युद्धाय धृतमानसः}


\twolineshloka
{यथोद्दिष्टानि सैन्यानि पाण्डवानामयोजयत्}
{जयाय पाण्डुपुत्राणां यत्तस्तस्थौ रणाजिरे}


\chapter{अध्यायः १६५}
\twolineshloka
{धृतराष्ट्र उवाच}
{}


\twolineshloka
{प्रतिज्ञाते फाल्गुनेन वधे भीष्मस्य संयुगे}
{किमकुर्वत मे मन्दाः पुत्रा दुर्योधनादयः}


\twolineshloka
{हतमेव हि पश्यामि गाङ्गेयं पितरं रणे}
{वासुदेवसहायेन पार्थेन दृढधन्वना}


\twolineshloka
{स चापरिमितप्रज्ञस्तच्छ्रुत्वा पार्थभाषितम्}
{किमुक्तवान्महेष्वासो भीष्मः प्रहरतां वरः}


\threelineshloka
{सैनापत्यं च संप्राप्य कौरवाणां धुरंधरः}
{किमचेष्टत गाङ्गेयो महाबुद्धिपराक्रमः ॥वैशंपायन उवाच}
{}


\threelineshloka
{ततस्तत्सञ्जयस्तस्मै सर्वमेव न्यवेदयत्}
{यथोक्तं कुरुवृद्धेन भीष्मेणामिततेजसा ॥सञ्जय उवाच}
{}


\twolineshloka
{सैनापत्यमनुप्राप्य भीष्मः शान्तनवो नृप}
{दुर्योधनमुवाचेदं वचनं हर्षयन्निव}


\threelineshloka
{नमस्कृत्य कुमाराय सेनान्ये शक्तिपाणये}
{अहं सेनापतिस्तेऽद्य भविष्यामि न संशयः}
{}


\twolineshloka
{सेनाकर्मण्यिज्ञोऽस्मि व्यूहेषु विविधेषु च}
{कर्म कारयितुं चैव भृतानप्यभृतांस्तथा}


\twolineshloka
{यात्रायाने च युद्धे च तथा प्रशमनेषु च}
{भृशं वेद महाराज यथा वेद बृहस्पतिः}


\twolineshloka
{व्यूहानां च समारम्भान्दैवगान्धर्वमानुषान्}
{तैरहं मोहयिष्यामि पाण्डवान्व्येतु ते ज्वरः}


\threelineshloka
{सोऽहं योत्स्यामि तत्त्वेन पालयंस्तव वाहिनीम्}
{यथावच्छास्त्रतो राजन्व्येतु ते मानसो ज्वरः ॥दुर्योधन उवाच}
{}


\twolineshloka
{विद्यते मे न गाङ्गेय भयं देवासुरेष्वपि}
{समस्तेषु महाबाहो सत्यमेतद्ब्रवीमि ते}


\twolineshloka
{किं पुनस्त्वयि दुर्धर्पे सैनापत्ये व्यवस्थिते}
{द्रोणे च पुरुषव्याघ्रे स्थिते युद्धाभिनन्दिनि}


\twolineshloka
{भवद्र्यां पुरुषाग्र्याभ्यां स्थिताभ्यां विजयो मम}
{न दुर्लभं कुरुश्रेष्ठ देवराज्यमपि ध्रुवम्}


\twolineshloka
{रथसङ्ख्यां तु कार्त्स्न्येन परेषामात्मनस्तथा}
{तथैवातिरथानां च वेत्तुमिच्छामि कौरव}


\threelineshloka
{पितामहो हि कुशलः परेषामात्मनस्तथा}
{श्रोतुमिच्छाम्यहं सर्वैः सहैभिर्वसुधाधिपैः ॥भीष्म उवाच}
{}


\twolineshloka
{गान्धारे श्रृणु राजेन्द्र रथसंख्यां स्वके बले}
{ये रथाः पृथिवीपाल तथैवातिरथाश्च ये}


\twolineshloka
{बहूनीह सहस्राणि प्रयुतान्यर्बुदानि च}
{रथानां तव सेनायां यथामुख्यं तु मे श्रुणु}


\twolineshloka
{भवानग्रे रथोदारः सह सर्वैः सहोदरैः}
{दुःशासनप्रभृतिभिर्भ्रातृभिः शतसंमितैः}


\twolineshloka
{सर्वे कृतप्रहरणाश्छेदभेदविशारदाः}
{रथोपस्थे गजस्कन्धे गदाप्रासासिचर्मणि}


\twolineshloka
{संयन्तारः प्रहर्तारः कृतास्त्रा भारसाधनाः}
{इष्वस्त्रे द्रोणशिष्याश्च कृपस्य च शरद्वतः}


\twolineshloka
{एते हनिष्यन्ति रणे पाञ्चालान्युद्धदुर्मदान्}
{कृतकिल्बिषाः पाण्डवेयैर्धार्तराष्ट्रा मनस्विनः}


\twolineshloka
{तथाऽहं भरतश्रेष्ठ सर्वसेनापतिस्तव}
{शत्रून्विध्वंसयिष्यामि कदर्थीकृत्य पाण्डवान्}


\twolineshloka
{न त्वात्मनो गुणान्वक्तुमर्हामि विदितोऽस्मि ते}
{कृतवर्मा त्वयिरथो भोजः शस्त्रभृतां वरः}


\twolineshloka
{अर्थसिद्धिं तव रणे करिष्यति न संशयः}
{शस्त्रविद्भिरनाधृष्यो दूरपाती दृढायुधः}


\twolineshloka
{हनिष्यति चमूं तेषां महेन्द्रो दानवानिव}
{मद्रराजो महेष्वासः शल्यो मेऽतिरथो मतः}


\threelineshloka
{स्पर्धते वासुदेवेन नित्यं यो वै रणेरणे}
{भागिनेयान्निजांस्त्यक्त्वा शल्यस्तेऽतिरथो मतः}
{एष योत्स्यति संग्रामे पाण्डवांश्च महारथान्}


\twolineshloka
{सागरोर्मिसमैर्बाणैः प्लावयन्निव शातवान्}
{भूरिश्रवाः कृतास्रश्च तव चापि हितः सुहृत्}


\twolineshloka
{सौमदत्तिर्महेष्वासो रथयूथपयूथपः}
{वलक्षयममित्राणां सुमहान्तं करिष्यति}


\twolineshloka
{सिन्धुराजो महाराज मतो मे द्विगुणो रथाः}
{योत्स्यते समरे राजन्विक्रान्तो रथसत्तमः}


\twolineshloka
{द्रौपदीहरणे राजन्परिक्लिष्टश्च पाण्डवैः}
{संस्मरंस्तं परिक्लेशं योत्स्यते परवीरहा}


\twolineshloka
{एतेन हि तदा राजंस्तप आस्थाय दारुणम्}
{सुदुर्लभो वरो लब्धः पाण्डवान्योद्ध्रुमाहवे}


\twolineshloka
{स एष रथशार्दूलस्तद्वैरं संस्मरन्रणे}
{योत्स्यते पाण्डवैस्तात प्राणांस्त्यक्त्वा सुदुस्त्यजान्}


\chapter{अध्यायः १६६}
\twolineshloka
{भीष्म उवाच}
{}


\twolineshloka
{सुदक्षिणस्तु काम्भोजो रथ एकगुणो मतः}
{तवार्थसिद्धिमाकाङ्क्षन्योत्स्यते समरे परैः}


\twolineshloka
{एतस्य रथसिंहस्य तवार्थे राजसत्तम}
{पराक्रमं यथेन्द्रस्य द्रक्ष्यन्ति कुरवो युधि}


\twolineshloka
{एतस्य रथवंशे हि तिग्मवेगप्रहारिणः}
{काम्भोजानां महाराज शलभानामिवायतिः}


\twolineshloka
{नीलो माहिष्यतीवासी नीलवर्मा रथस्तव}
{रथवंशेन कदनं शत्रूणां वै करिष्यति}


\twolineshloka
{कृतवैरः पुरा चैव सहदेवेन मारिष}
{योत्स्यते सततं राजंस्तवार्थे कुरुनन्दन}


\twolineshloka
{विन्दानुविन्दावावन्त्यौ संमतौ रथसत्तमौ}
{कृतिनौ समरे तात दृढवीर्यपराक्रमौ}


\twolineshloka
{एतौ तौ पुरुषव्याघ्रौ रिपुसैन्यं प्रधक्ष्यतः}
{गदाप्रासासिनाराचैस्तोमरैश्च करच्यतैः}


\twolineshloka
{युद्धाभिकामौ समरे क्रीडन्ताविव यूथपौ}
{यूथमध्ये महाराज विचरन्तौ कृतान्तवत्}


\twolineshloka
{त्रिगर्ता भ्रातरः पञ्च रथोदारा मता मम}
{कृतवैराश्च पार्थैस्ते विराटनगरे तदा}


\twolineshloka
{मकरा इव राजेन्द्र समुद्धततरङ्गिणीम्}
{गङ्गां विक्षोभयिष्यन्ति पार्थानां युधि वाहिनी}


\twolineshloka
{ते रथाः पञ्च राजेन्द्र येषां सत्यरथो मुखम्}
{एते योत्स्यन्ति सङ्ग्रामे संस्मरन्तः पुराकृतमक्}


\twolineshloka
{व्यलीकं पाण्डवेयेन भीमसेनानुजेन ह}
{दिशो विजयता राजञ्श्वेतवाहेन भारत}


\twolineshloka
{ते हनिष्यन्ति पार्थानां तानासाद्य महारथान्}
{वरान्वरान्महेष्वासान्क्षत्रियाणां धुरंधरान्}


\twolineshloka
{लक्ष्णणस्तव पुत्रश्च तथा दुःशासनस्य च}
{उभौ तौ पुरुषव्याघ्रौ संग्रामेष्वपलायिनौ}


\twolineshloka
{तरुणौ सुकुमारौ च राजपुत्रौ रतस्विनौ}
{युद्धानां च विशेषज्ञौ प्रणेतारौ च सर्वशः}


\twolineshloka
{रथौ तौ कुरुशार्दूल मतौ मे रथसत्तमौ}
{क्षत्रधर्मरतौ वीरौ महत्कर्म करिष्यतः}


\twolineshloka
{दण्डधारो महाराज रथ एको नरर्षभ}
{योत्स्यते तव सङ्ग्रामे स्वेन सैन्येन पालितः}


\twolineshloka
{बृहद्बलस्तथा राजा कौसल्यो रथसत्तमः}
{रथो मम मतस्तात महावेगपराक्रमः}


\twolineshloka
{एष योत्स्यति सङ््ग्रामे स्वान्बन्धून्संप्रहर्षयन्}
{उग्रायुधो महेष्वासो धार्तराष्ट्रहिते रतः}


\twolineshloka
{कृपः शारद्वतो रजन्रथयूथपयूथपः}
{प्रियान्प्राणान्परित्यज्य प्रधक्ष्यति रिपूंस्तव}


\twolineshloka
{गौतमस्य महर्षेर्य आचार्यस्य शरद्वतः}
{कार्तिकेय इवाजयः शरस्तम्बात्सुतोऽभवत्}


\twolineshloka
{एष सेनाः सुबहुला विविधायुधकार्मुकाः}
{अग्निवत्समरे तात चरिष्यति विनिर्दहन्}


\chapter{अध्यायः १६७}
\twolineshloka
{भीष्म उवाच}
{}


\twolineshloka
{शकुनिर्मातुलस्तेऽसौ रथ एको नराधिप}
{प्रयुज्य पाण्डवैर्वैरं योत्स्यते नात्र संशयः}


\twolineshloka
{एतस्य सेना दुर्धर्षा समरे प्रतियायिनः}
{विकृतायुधभूयिष्ठा वायुवेगसमा जवे}


\twolineshloka
{द्रोणपुत्रो महेष्वासः सर्वानेवातिधन्विनः}
{समरे चित्रयोधी च दृढास्त्रश्च महारथः}


\twolineshloka
{एतस्य हि महाराज यथा गाण्डीवधन्वनः}
{शरासनविनिर्मुक्ताः संसक्ता यान्ति सायकाः}


\twolineshloka
{नैष शक्यो मया वीरः संख्यातुं रथसत्तमः}
{निर्दहेदपि लोकांस्त्रीनिच्छन्नेष महारथः}


\twolineshloka
{क्रोधस्तेजश्च तपसा संभृतोश्रमवासिनाम्}
{द्रोणेनानुगृहीतश्च दिव्यैरस्त्रैरुदारधीः}


\twolineshloka
{दोषस्त्वस्य महानेको येनैव भरतर्षभ}
{न मे रथो नातिरथो मतः पार्थिवसत्तम}


\twolineshloka
{जीवितं प्रियमत्यर्थमायुष्कामः सदा द्विजः}
{न ह्यस्य सदृशः कश्चिदुभयोः सेनयोरपि}


\twolineshloka
{हन्यादेकरथेनैव देवानामपि वाहिनीम्}
{वपुष्मांस्तलघोषेण स्फोटयेदपि पर्वतान्}


\twolineshloka
{असंख्येयगुणो वीरः प्रहन्ता दारुणद्युतिः}
{दण्डपाणिरिवासह्यः कालवत्प्रचरिष्यति}


\twolineshloka
{युगान्ताग्निसमः क्रोधात्सिंहग्रीवो महाद्युतिः}
{एष भारत युद्धस्य पृष्ठं संशयमिष्यति}


\twolineshloka
{पिता त्वस्य महातेजा वृद्धोऽपि युवभिर्वरः}
{रणे कर्म महत्कर्ता अत्र मे नास्ति संशयः}


\twolineshloka
{अस्त्रवेगानिलोद्भूतः सेनाकक्षेन्धनोत्थितः}
{पाण्डुपुत्रस्य सैन्यानि प्रधक्ष्यति रणे धृतः}


\twolineshloka
{रथयूथपयूथानां यूथपोऽय नरर्षभः}
{भरद्वाजात्मजः कर्ता कर्म तीव्रं हितं तव}


\twolineshloka
{सर्वमूर्धाभिषिक्तानामाचार्यः स्थविरो गुरुः}
{गच्छेदन्तं सृञ्जयानां प्रियस्त्वस्य धनञ्जयः}


\twolineshloka
{नैष जातु महेष्वासः पार्थमक्लिष्टकारिणम्}
{हन्यादाचार्यकं दीप्तं संस्मृत्य गुणनिर्जितम्}


\twolineshloka
{श्लाघते यं सदा वीर पार्थस्य गुणविस्तरैः}
{पुत्रादभ्यधिकं चैनं भारद्वाजोऽनुपश्यति}


\twolineshloka
{हन्यादेकरथेनैव देवगन्धर्वमानुषान्}
{एकीभूतानपि रणे दिव्यैरस्त्रैः प्रतापवान्}


\twolineshloka
{पौरवो राजशार्दूलस्तव राजन्महारथः}
{मतो मम रथोदारः परवीररथारुजः}


\twolineshloka
{स्वेन सैन्येन महता प्रतपञ्शत्रुवाहिनीम्}
{प्रधक्ष्यति स पाञ्चालान्कक्षमग्निगतिर्यथा}


\twolineshloka
{सत्यश्रवा रथस्त्वेको राजपुत्रो बृहद्बलः}
{तव राजन्रिपुबले कालवत्प्रचरिष्यति}


\twolineshloka
{एतस्य योधा राजेन्द्र विचित्रकवचायुधाः}
{विचरिष्यन्ति सङ््ग्रामे निन्घन्तः शास्त्रवांस्तव}


\twolineshloka
{वृषसेनो रथस्तेऽग्र्यः कर्णपुत्रो महारथः}
{प्रधक्ष्यति रिपूणां ते बलं तु बलिनां वरः}


\twolineshloka
{जघसन्धो महातेजा राजन्रथवरस्तव}
{त्यक्ष्यते समरे प्राणान्माधवः परवीरहा}


\twolineshloka
{एष योत्स्यति सङ्ग्रामे गजस्कन्धविशारदः}
{रथेन वा माहबाहुः क्षपयञ्शत्रुवाहिनीम्}


\twolineshloka
{रथ एष महाराज मतो मे राजसत्तम}
{त्वदर्थे त्यक्ष्यते प्राणान्सहसैन्यो महारणे}


\twolineshloka
{एष विक्रान्तयोधी च चित्रयोधी च संगरे}
{वीतभीश्चापि ते राजञ्शत्रुभिः सह योत्स्यते}


\twolineshloka
{बाह्लीकोऽतिरथश्चैव समरे चानिवर्तनः}
{मम राजन्मतो युद्धे शूरो वैवस्वतोपमः}


\twolineshloka
{न ह्येष समरं प्राप्य निवर्तेत कथञ्चन}
{यथा सततगो राजन्स हि हन्यात्परान्रणे}


\twolineshloka
{सेनापतिर्महाराज सत्यवांस्ते महारथः}
{रणेष्वद्भुतकर्मा च रथी पररथारुजः}


\twolineshloka
{एतस्य समरं दृष्ट्वा न व्यथास्ति कथञ्चन}
{उत्स्मयन्नुत्पतत्येष परान्रथपथे स्थितान्}


\twolineshloka
{एष चारिषु विक्रान्तः कर्म सत्पुरुषोचितम्}
{कर्ता विमर्दे सुमहत्त्वदर्थे पुरुषोत्तमः}


\twolineshloka
{अलम्बुसो राक्षसेन्द्रः क्रूरकर्मा महारथः}
{हनिष्यति परान्राजन्पूर्ववैरमनुस्मरन्}


\twolineshloka
{एष राक्षससैन्यानं सर्वेषां रथसत्तमः}
{मायावी दृढवैरश्च समरे विचरिष्यति}


\twolineshloka
{प्राग्ज्योतिषाधिपो वीरो भगदत्तः प्रतापवान्}
{गजाङ्कुशधरश्रेष्ठो रथे चैव विशारदः}


\twolineshloka
{एतेन युद्धमभवत्पुरा गाण्डीवधन्वनः}
{दिवसान्सुबहून्राजन्नुभयोर्जयगृद्धिनोः}


\twolineshloka
{ततः सखायं गान्धारे मानयन्पाकशासनम्}
{अकरोत्संविदं तेन पाण्डवेन महात्मना}


\twolineshloka
{एष योत्स्यति सङ्ग्रामे गजस्कन्धविशारदः}
{ऐरावतगतो राजा देवानामिव वासवः}


\chapter{अध्यायः १६८}
\twolineshloka
{भीष्म उवाच}
{}


\twolineshloka
{अचलो वृषकश्चैव सहितौ भ्रातरावुभौ}
{रथौ तव दुराधर्षौ शत्रून्विध्वंसयिष्यतः}


\twolineshloka
{बलवन्तौ नरव्याघ्रौ दृढक्रोधौ प्रहारिणौ}
{गान्धारमुख्यौ तरुणौ दर्शनीयौ महाबलौ}


\twolineshloka
{सखा ते दयितो नित्यं य एष रणकर्कशः}
{उत्साहयति राजंस्त्वां विग्रहे पाण्डवैः सह}


\twolineshloka
{परुषः कत्थनो नीचः कर्णो वैकर्तनस्त्रव}
{मन्त्री नेता च बन्धुश्च मानी चात्यन्तमुच्छ्रितः}


\twolineshloka
{एष नैव रथः पूर्णो न चाप्यतिरथो रणे}
{वियुक्तः कवचेनैष सहजेन विचेतनः}


\twolineshloka
{कुण्डलाभ्यां च दिव्याभ्यां वियुक्तः सततं घृणी}
{अभिशापाच्च रामस्य ब्राह्मणस्य च भाषणात्}


\twolineshloka
{करणानां वियोगाच्च तेन मेऽर्थरथो मतः}
{नैष फाल्गुनमासाद्य पुनर्जीवन्विमोक्ष्यते}


\twolineshloka
{ततोऽब्रवीत्पुनर्द्रोणः सर्वशस्त्रभृतां वरः}
{एवमेतद्यथाऽऽत्थ त्वं न मिथ्यास्ति कदाचन}


\threelineshloka
{रणेरणेऽभिमानी च विमुखश्चापि दृश्यते}
{घृणी कर्णः प्रमादी च तेन मेऽर्धरथो मतः ॥संजय उवाच}
{}


\twolineshloka
{एतच्छ्रुत्वा तु राधेयः क्रोधादुल्फाल्य लोचने}
{उवाच भीष्मं राधेयस्तुदन्वाग्भिः प्रतोदवत्}


\twolineshloka
{पितामह यथेष्टं मां वाक्शरैरुपकृन्तसि}
{अनागसं सदा द्वेषादेवमेव पदेपदे}


\twolineshloka
{मर्षयामि च तत्सर्वं दुर्योधनकृतेन वै}
{त्वं तुं मां मान्यसे मन्दं यथा कापुरुषं तथा}


\twolineshloka
{भवानर्धरथो मह्यं मतो वै नात्र संशयः}
{सर्वस्य जगतश्चैव गाङ्गेयो न मृषा वदेत्}


\twolineshloka
{कुरूणामहितो नित्यं न च राजाऽवबुध्यते}
{को हि नाम समानेषु राजसूदारकर्मसु}


\twolineshloka
{तेजोवधमिमं कुर्याद्विभेदयिषुराहवे}
{यथा त्वं गुणविद्वेषादपरागं चिकीर्षसि}


\twolineshloka
{न हायनैर्न पलितैर्न वित्तैर्न च बन्धुभिः}
{महारयत्वं संख्यातुं शक्यं क्षत्रस्य कौरव}


\twolineshloka
{बलज्येष्ठं स्मृतं क्षत्रं मन्त्रज्येष्ठा द्विजातयः}
{धनज्येष्ठाः स्मृता वैश्याः शूद्रास्तु वयसाऽधिकाः}


\twolineshloka
{यथेच्छकं स्वयं ब्रूया रथानतिरथांस्तथा}
{कामद्वेषसमायुक्तो मोहात्प्रकुरुते भवान्}


\twolineshloka
{दुर्योधन महाबाहो साधु सम्यगवेक्ष्यताम्}
{त्यज्यतां दुष्टभावोऽयं भीष्मः किल्बिषकृत्तव}


\twolineshloka
{भिन्ना हि सेना नृपते दुःसन्धेया भवत्युत}
{मौला हि पुरुषव्याघ्र किमु नानासमुत्थिताः}


\twolineshloka
{एषां द्वैधं समुत्पन्नं योधानां युधि भारत}
{तेजोवधो नः क्रियते प्रत्यक्षेण विशेषतः}


\twolineshloka
{रथानां क्वच विज्ञानं क्वच भीष्मोऽल्पचेतनः}
{अहमावारयिष्यामि पाण्डवानामनीकिनीम्}


\twolineshloka
{आसाद्य माममोघेषुं गमिष्यन्ति दिशो दश}
{पाण्डवाः सहपञ्चालाः शार्दूलं वृषभा इव}


\twolineshloka
{क्वच युद्धं विमर्दो वा मन्त्रे सुव्याहृतानि च}
{क्वचभीष्मो गतवयामन्दात्मा कालचोदितः}


\twolineshloka
{एकाकी स्पर्धति नित्यं सर्वेण रमता सह}
{न चान्यं पुरुषं कंचिन्मन्यते मोघदर्शनः}


\twolineshloka
{श्रोतव्यं खलु वृद्धानमिति शास्त्रनिदर्शनम्}
{न त्वेव ह्यतिवृद्धानां पुनर्बालाहि ते मताः}


\twolineshloka
{अहमेको हनिष्यामि पाण्डवानानीकिनीम्}
{सुयुद्धे राज्यशार्दूल यशो भीष्मं गमिष्यति}


\twolineshloka
{कृतः सेनापतिस्त्वेष त्वया भीष्मो नराधिप}
{सेनापतौ यशो गन्ता न तु योधान्कथञ्चन}


\threelineshloka
{नाहं जीवति गाङ्गेये योत्स्ये राजन्कथञ्चन}
{हते भीष्मे तु योद्धाऽस्मि सर्वैरेव महारथैः ॥भीष्म उवाच}
{}


\twolineshloka
{समुद्यतोयं भारो मे सुमहान्सागरोपमः}
{धार्तराष्ट्रस्य संग्रामे वर्षपूगाभिचिन्तितः}


\twolineshloka
{तस्मिन्नभ्यागते काले प्रतप्ते रोमहर्षणे}
{मिथो भेदो न मे कार्यस्तेन जीवसि सूतज}


\twolineshloka
{न ह्यहं त्वद्य विक्रम्य स्थविरोऽपि शिशोस्तव}
{युद्धश्रद्धामहं छिन्द्यां जीवितस्य च सूतज}


\twolineshloka
{जामदग्र्येन रामेण महास्त्राणि विमुञ्चता}
{न मे व्यथा कृता काचित्त्वं तु मे कि करिष्यसि}


\twolineshloka
{कामं नैतत्प्रशंसन्ति सन्तः स्वबलसंस्तवम्}
{वक्ष्यामि तु त्वां सन्तप्तो निहीन कुलपांसन}


\twolineshloka
{समेतं पार्थिवं क्षत्रं काशिराजस्वयंवरे}
{निर्जित्यैकरथेनैव याः कन्यास्तरसाहृताः}


\twolineshloka
{ईदृशानां सहस्राणि विशिष्टानामथो पुनः}
{मयैकेन निरस्तानि ससैन्यानि रणाजिरे}


\twolineshloka
{त्वां प्राप्य वैरपुरुषं कुरूणामनयो महान्}
{उपस्थितो विनाशाय यतस्व पुरुषो भव}


\twolineshloka
{युध्यस्व समरे पार्थं येन विस्पर्धसे सह}
{द्रक्ष्यामि त्वां विनिर्मुक्तमस्माद्युद्धात्सुदुर्मते}


\twolineshloka
{तमुवाच ततो राजा धार्तराष्ट्रः प्रतापवान्}
{मां समीक्षस्व गाङ्गेय कार्यं हि महदुद्यतम्}


\twolineshloka
{चिन्त्यतामिदमेकाग्रं मम निःश्रेयसं परम्}
{उभावपि भवन्तौ मे महत्कर्म करिष्यतः}


\twolineshloka
{भूयश्च श्रोतुमिच्छामि परेषां रथसत्तमान्}
{ये चैवातिथास्तत्र ये चैव रथयूथपाः}


\twolineshloka
{बलाबलममित्राणां श्रोतुमिच्छामि कौरव}
{प्रभातायां रजन्यां वै इदं युद्धं भविष्यति}


\chapter{अध्यायः १६९}
\twolineshloka
{भीष्म उवाच}
{}


\twolineshloka
{एते रथास्तवाख्यातास्तथैवातिरथा नृप}
{ये चाप्यर्धरथा राजन्पाण्डवानामतः श्रृणु}


\twolineshloka
{यदि कौतूहलं तेऽद्य पाण्डवानां बले नृप}
{रथसंख्यां शृणुष्व त्वं सहैभिर्वसुधाधिपैः}


\twolineshloka
{स्वयं राजा रथोदारः पाण्डवः कुन्तनन्दनः}
{अग्निवत्समरे तात चरिष्यति न संशयः}


\twolineshloka
{भीमसेनस्तु राजेन्द्र रथोऽष्टगुणसंमितः}
{न तस्यास्ति समो युद्धे गदया सायकैरपि}


\twolineshloka
{नागायुतबलो मानी तेजसा न स मानुषः}
{माद्रीपुत्रो च रथिनौ द्वावेव पुरुषर्षभौ}


\twolineshloka
{अश्विनाविव रूपेण तेजसा च समन्वितौ}
{एते चमूमुखगताः स्मरन्तः क्लेशमुत्तमम्}


\twolineshloka
{रुद्रवत्प्रचरिष्यन्ति तत्र मे नास्ति संशयः}
{सर्व एव महात्मानः सालस्तम्भा इवोद्गताः}


\twolineshloka
{प्रादेशेनाधिकाः पुंभिरन्यैस्ते च प्रमाणतः}
{सिंहसंहननाः सर्वे पाण्डुपुत्रा महाबलाः}


\twolineshloka
{चरितब्रह्मचर्याश्च सर्वे तात तपस्विनः}
{ह्रीमन्तः पुरुषव्याघ्रा व्याघ्रा इव बलोत्कटाः}


\twolineshloka
{जवे प्रहारे संमर्दे सर्व एवातिमानुषाः}
{सर्वैर्जिता महीपाला दिग्जये भरतर्षभ}


\twolineshloka
{न चैषां पुरुषाः केचिदायुधानि गदाः शरान्}
{विषहन्ति सदा कर्तुमधिज्यान्यपि कौरव}


\twolineshloka
{उद्यन्तुं वा गदा गुर्वीः शरान्वा क्षेप्तुमाहवे}
{जवे लक्ष्यस्य हरणे भोज्ये पांसुविकर्षणे}


\twolineshloka
{बालैरपि भवन्तस्तैः सर्व एव विशेषिताः}
{एतत्सैन्यं समासाद्य सर्व एव बलोत्कटाः}


\twolineshloka
{विध्वंसयिष्यन्ति रणे मा स्म तैः सह संगमः}
{एकैकशस्त संमर्दे हन्युः सर्वान्महीक्षितः}


\twolineshloka
{प्रत्यक्षं तव राजेन्द्र राजसूये यथाभवत्}
{द्रौपद्याश्च परिक्लेशं द्यूते च परुषा गिरः}


\twolineshloka
{ते स्मरन्तश्च सङ्ग्रामे चरिष्यन्ति च रुद्रवत्}
{लोहिताक्षो गुडाकेशो नारायणसहायवान्}


\twolineshloka
{उभोयोः सेनयोर्वीरो रथो नास्तीति तादृशः}
{न हि देवेषु वा पूर्वं मनुष्येषूरगेषु च}


\twolineshloka
{राक्षसेष्वथ यक्षेषु नरेषु कुत एव तु}
{भूतोथ वा भविष्यो वा रथः कश्चिन्मया श्रुतः}


\twolineshloka
{समायुक्तो महाराज रथः पार्थस्य धीमतः}
{वासुदेवश्च संयन्ता योद्धा चैव धनञ्जयः}


\twolineshloka
{गाण्डीवं च धनुर्दिव्यं ते चाश्वा वातरंहसः}
{अभेद्यं कवचं दिव्यमक्षय्यौ च महेषुधी}


\twolineshloka
{अस्त्रग्रामश्च माहेन्द्रो रौद्रः कौबेर एवच}
{याम्यश्च वारुणश्चैव गदाश्चोग्रप्रदर्शनाः}


\twolineshloka
{वज्रादीनि च मुख्यानि नानाप्रहरणानि च}
{दानवानां सहस्राणि हिरण्यपुरवासिनाम्}


\twolineshloka
{हतान्येकरथेनाजौ कस्तस्य सदृशो रथः}
{एष हन्याद्धि संरम्भी बलवान्सत्यविक्रमः}


\twolineshloka
{तव सेनां महाबाहुः स्वां चैव परिपालयन्}
{अहं चैनं प्रत्युदियामाचार्यो वा धनञ्जयम्}


\twolineshloka
{न तृतीयोऽस्ति राजेन्द्र सेनयोरुभयोरपि}
{य एनं शरवर्षाणि वर्षन्तमुदियाद्रथी}


\fourlineindentedshloka
{जीमूत इव घर्मान्ते महावातसमीरितः}
{समायुक्तस्तु कौन्तेयो वासुदेवसहायवान्}
{तरुणश्च कृती चैव जीर्णावावामुभावपि ॥वैशंपायन उवाच}
{}


\twolineshloka
{एतच्छ्रुत्वा तु भीष्मस्य राज्ञां दध्वंसिरे तदा}
{काञ्चनाङ्गदिनः पीना भुजाश्चन्दनरूषिताः}


\twolineshloka
{मनोभिः सह संवेगैः संस्मृत्य च पुरातनम्}
{सामर्थ्यं पाण्डवेयानां यथा प्रत्यक्षदर्शनात्}


\chapter{अध्यायः १७०}
\twolineshloka
{भीष्ण उवाच}
{}


\twolineshloka
{द्रौपदेया महाराज सर्वे पञ्च महारथाः}
{वैराटिरुत्तरश्चैव रथोदारो मतो मम}


\twolineshloka
{अभिमन्युर्महाबाहू रथयूथमयूथपः}
{समः पार्थेन समरे वासुदेवेन चारिहा}


\twolineshloka
{लघ्वस्त्रश्चित्रयोधी च मनस्वी च दृढव्रतः}
{संस्मरन्वै परिक्लेशं स्वपितुर्विक्रमिष्यति}


\twolineshloka
{सात्यकिर्माधवः शूरो रथयूथपयूथपः}
{एष वृष्णिप्रवीराणाममर्षी जितसाध्वसः}


\twolineshloka
{उत्तमौजास्तथा राजन्रथोदारो मतो मम}
{युधामन्युश्च विक्रान्तो रथोदारो मतो मम}


\twolineshloka
{एतेषां बहुसाहस्रा रथा नागा हयास्तथा}
{योत्स्यन्ते ते तनूस्त्यक्त्वा कुन्तीपुत्रप्रियेप्सया}


\twolineshloka
{पाण्डवैः सह राजेन्द्र तव सेनासु भारत}
{अग्निमारुतवद्राजन्नाह्वयन्तः परस्परम्}


\twolineshloka
{अजेयौ समरे वृद्धौ विराटद्रुपदौ तथा}
{महारथौ महावीर्यौ मतौ मे पुरुषर्षभौ}


\twolineshloka
{वयोवृद्धावपि हि तौ क्षत्रधर्मपरायणौ}
{यतिष्येते परं शक्त्या स्थितौ वीरगते पथि}


\twolineshloka
{संबन्धिकेन राजेन्द्र तौ तु वीर्यबलान्वयात्}
{आर्यवृत्तौ महेष्वासौ स्नेहवीर्यसितावुभौ}


\twolineshloka
{कारणं प्राप्य तु नराः सर्व एव महाभुजाः}
{शूरा वा कातरा वापि भवन्ति कुरुपुङ्गव}


\twolineshloka
{एकायनगतावेतौ पार्थिवौ दृढधन्विनौ}
{प्राणांस्त्यक्त्वा परं शक्त्या घट्टितारौ परन्तप}


\twolineshloka
{पृथगक्षौहिणीभ्यां तावुभौ संयति दारुणौ}
{संबन्धिभावं रक्षन्तौ महत्कर्म करिष्यतः}


\twolineshloka
{लोकवीरौ महेष्वासौ त्यक्तात्मानौ च भारत}
{प्रत्ययं परिरक्षन्तौ महत्कर्म करिष्यतः}


\chapter{अध्यायः १७१}
\twolineshloka
{भीष्म उवाच}
{}


\twolineshloka
{पाञ्चालराजस्य सुतो राजन्परपुरंजयः}
{शिखण्डी रथमुख्यो मे मतः पार्थस्य भारत}


\twolineshloka
{एष योत्स्यति संग्रामे नाशयन्पूर्वसंस्थितम्}
{परं यशो विप्रथयंस्तव सेनासु भारत}


\twolineshloka
{एतस्य बहुलाः सेनाः पाञ्चालाश्च प्रभद्रकाः}
{तेनासौ रथवंशेन महत्कर्म करिष्यति}


\twolineshloka
{धृष्टद्युम्नश्च सेनानीः सर्वसेनासु भारत}
{मतो मेऽतिरथो राजन्द्रोणशिष्यो महारथः}


\twolineshloka
{एष योत्स्यति संग्रामे सूदयन्वै परान्रणे}
{भगवानिव संक्रुद्धः पिनाकी युगसंक्षये}


\twolineshloka
{एतस्य तद्रथानीकं कथयन्ति रणप्रियाः}
{बहुत्वात्सागरप्रख्यं देवानामिव संयुगे}


\twolineshloka
{क्षत्रधर्मा तु राजेन्द्र मतो मेऽर्धरथो नृप}
{धृष्टद्युम्नस्य तनयो बाल्यान्नातिकृतश्रमः}


\twolineshloka
{शिशुपालसुतो वीरश्चेदिराजो महारथः}
{धृष्टकेतुर्महेष्वासः संबन्धी पाण्डवस्य ह}


\twolineshloka
{एष चेदिपतिः शूरः सह पुत्रेण भारत}
{महारथानां सुकरं महत्कर्म करिष्यति}


\twolineshloka
{क्षत्रधर्मरतो मह्यं मतः परपुरंजयः}
{क्षत्रदेवस्तु राजेन्द्र पाण्डवेषु रथोत्तमः}


\twolineshloka
{जयन्तश्चामितौजाश्च सत्यजिच्च महारथः}
{महारथा महात्मानः सर्वे पाञ्चालसत्तमाः}


\twolineshloka
{योत्स्यन्ते समरे तात संरब्धा इव कुञ्जराः}
{अजो भोजश्च विक्रान्तौ पाण्डवार्थे महारथौ}


\twolineshloka
{योत्स्येते बलिनौ शूरौ परं शक्त्या यतिष्यतः}
{शीघ्रास्त्राश्चित्रयोद्धारः कृतिनो दृढविक्रमाः}


\twolineshloka
{केकयाः पञ्च राजेन्द्र भ्रातरो दृढविक्रमाः}
{सर्वे चैव रथोदाराः सर्वे लोहितकध्वजाः}


\twolineshloka
{काशिकः सुकुमारश्च नीलो यश्चापरो नृप}
{सूर्यदत्तश्च शङ्खश्च मदिराश्वश्च नामतः}


\twolineshloka
{सर्व एव रथोदाराः सर्वे चाहवलक्षणाः}
{सर्वास्त्रविदुषः सर्वे महात्मानो मता मम}


\twolineshloka
{वार्धक्षेमिर्महाराज मतो मम महारथः}
{चित्रायुघश्च नृपतिर्मतो मे रथसत्तमः}


\threelineshloka
{स हि सङ्ग्रामशोभी च भक्तश्चापि किरीटिनः}
{चेकितानः सत्यधृतिः पाण्डवानां महारथौ}
{द्वाविमौ पुरुषव्याघ्रौ रथोदारौ मतौ मम}


\twolineshloka
{व्याघ्रदत्तश्च राजेन्द्र चन्द्रसेनश्च भारत}
{मतौ मम रथोदारौ पाण्डवानां न संशयः}


\twolineshloka
{सेनाबिन्दुश्च राजेन्द्र क्रोधहन्ता च नामतः}
{यः समो वासुदेवेन भीमसेनेन वा विभो}


\twolineshloka
{स योत्स्यति हि विक्रम्य समरे तव सैनिकैः}
{मां च द्रोणं कृपं चैव यथा संमन्यते भवान्}


\twolineshloka
{तथा स समरश्लाघी मन्तव्यो रथसत्तमः}
{काश्यः परमशीघ्रास्त्रः श्लाघनीयो नरोत्तमः}


\twolineshloka
{रथ एकगुणो मह्यं ज्ञेयः परपुरंजयः}
{अयं च युधि विक्रान्तो मन्तव्योष्टगुणो रथः}


\twolineshloka
{सत्यजित्समरश्लाघी द्रुपदस्यात्मजो युवा}
{गतः सोऽतिरथत्वं हि धृष्टद्युम्नेन संमितः}


\twolineshloka
{पाण्डवानां यशस्कामः परं कर्म करिष्यति}
{अनुरक्ताश्च शूरश्च रथोऽयमपरो महान्}


\twolineshloka
{पाण्ड्यराजो महावीर्यः पाण्डवानां धुरंधरः}
{दृढधन्वा महेष्वासः पाण्डवानां महारथः}


\twolineshloka
{श्रेणिमान्कौरवश्रेष्ठ वसुदानश्च पार्थिवः}
{उभावेतावतिरथौ मतौ परपुरंजयौ}


\chapter{अध्यायः १७२}
\twolineshloka
{भीष्म उवाच}
{}


\twolineshloka
{रोचमानो महाराज पाण्डवानां महारथः}
{योत्स्यतेऽमरवत्संख्ये परसैन्येषु भारत}


\twolineshloka
{पुरुजित्कृन्तिभोजश्च महेष्वासो महाबलः}
{मातुलो भीमसेनस्य स च मेऽतिरथो मतः}


\twolineshloka
{एष वीरो महेष्वासः कृती च निपुणश्च ह}
{चित्रयोधी च शक्तश्च मतो मे रथपुङ्गवः}


\twolineshloka
{स योत्स्यति हि विक्रम्य मघवानिव दानवैः}
{योधा ये चास्य विख्याताः सर्वे युद्धविशारदाः}


\twolineshloka
{भागिनेयकृते वीरः स करिष्यति संगरे}
{सुमहत्कर्म पाण्डूनां स्थितः प्रियहिते रतः}


\twolineshloka
{भैमसेनिर्महाराज हैडिम्बो राक्षसेश्वरः}
{मतो मे बहुमायावी रथयूथपयूथपः}


\twolineshloka
{योत्स्यते समरे तात मायावी समरप्रियः}
{ये चास्य राक्षसा वीराः सचिवा वशवर्तिनः}


\twolineshloka
{एते चान्ये च बहवो नानाजनपदेश्वराः}
{समेताः पाण्डवस्यार्थे वासुदेवपुरोगमाः}


\twolineshloka
{एते प्राधान्यतो राजन्पाण्डवस्य महात्मनः}
{रथाश्चातिरथाश्चैव ये चान्येऽर्धंरथा नृप}


\twolineshloka
{नेष्यन्ति समरे सेनां भीमां यौधिष्ठिरीं नृप}
{महेन्द्रेणेव वीरेण पाल्यमानां किरीटिना}


\twolineshloka
{तैरहं समरे वीर मायाविद्भिर्जयैषिभिः}
{योत्स्यामि जयमाकाङ्क्षन्नथ वा निधनं रणे}


\twolineshloka
{वासुदेवं च पार्थं च चक्रगाण्डीवधारिणौ}
{सन्ध्यागताविवार्केन्दू सभेष्येते रथोत्तमौ}


\twolineshloka
{यं चैव ते रथोदाराः पाण्डुपुत्रस्य सैनिकाः}
{सह सैन्यानहं तांश्च प्रतीयां रणमूर्धनि}


\twolineshloka
{एते रथाश्चातिरथाश्च तुभ्यंयथाप्रधानं नृप कीर्तिता मया}
{तथा परे येऽर्धरथाश्च केचि-त्तथैंव तेषामपि कौरवेन्द्र}


\twolineshloka
{अर्जुनं वासुदेवं च ये चान्ये तत्र पार्थिवाः}
{सर्वांस्तान्वारयिष्यामि यावद्द्रक्ष्यामि भारत}


\twolineshloka
{पाञ्चाल्यं तु महाबाहो नाहं हन्यां शिखण्डिनम्}
{उद्यतेषमथो दृष्ट्वा प्रतियुध्यन्तमाहवे}


\twolineshloka
{लोकस्तं वेद यदहं पितुः प्रियचिकीर्षया}
{प्राप्तं राज्यं परित्यज्य ब्रह्मचर्यव्रते स्थितः}


\twolineshloka
{चित्राङ्गदं कौरवाणामाधिपत्येऽभ्यषेचयम्}
{विचित्रवीर्यं च शिशुं यौवराज्येऽभ्यषेचयम्}


\twolineshloka
{देवव्रतत्वं विज्ञाप्य पृथिवीं सर्वराजसु}
{नैव हन्यां स्त्रियं जातु न स्त्रीपूर्वं कदाचन}


\twolineshloka
{स हि स्त्रीपूर्वको राजञ्शिखण्डी यदि ते श्रुतः}
{कन्या भूत्वा पुमाञ्चातो न योत्स्ये तेन भारत}


\twolineshloka
{सर्वांस्त्वन्यान्हनिष्यामि पार्थिवान्भरतर्षभ}
{यान्समेष्यामि समरे न तु कुन्तीसुतान्नृप}


\chapter{अध्यायः १७३}
\twolineshloka
{दुर्योधनं उवाच}
{}


\twolineshloka
{किमर्थं भरतश्रेष्ठ नैव हन्याः शिखण्डिनम्}
{उद्यतेषुमथो दृष्ट्वा समरेष्वाततायिनम्}


\threelineshloka
{पूर्वमुक्त्वा महाबाहो पाञ्चालान्सह सोमकैः}
{हनिष्यामीति गाङ्गेय तन्मे ब्रूहि पितामह ॥भीष्म उवाच}
{}


\twolineshloka
{श्रृणु दुर्योधन कथां सहैभिर्वसुधाधिपैः}
{यदर्थं युधि संप्रेक्ष्य नाहं हन्यां शिखण्डिनम्}


\twolineshloka
{महाराजो मम पिता शान्तनुर्लोकविश्रुतः}
{दिष्टान्तमाप धर्मात्मा समये भरतर्षभ}


\twolineshloka
{ततोऽहं भरतश्रेष्ठ प्रतिज्ञां परिपालयन्}
{चित्राङ्गदं भ्रातरं वै महाराज्येऽभ्यपेचयम्}


\twolineshloka
{तस्मिंश्च निधनं प्राप्ते सत्यवत्या मते स्थितः}
{विचित्रवीर्यं राजानमभ्यषिञ्चं यथाविधि}


\twolineshloka
{मयाऽभिषिक्तो राजेन्द्र यवीयानपि धर्मतः}
{विचित्रवीर्यो धर्मात्मा मामेव समुदैक्षत}


\twolineshloka
{तस्य दारक्रियां तात चिकीर्षुरहमप्युत}
{अनुरूपादिव कुलादित्येव च मनो दधे}


\threelineshloka
{तथाऽश्रौषं महाबाहो तिस्रः कन्याः स्वयंवरे}
{रूपेणाप्रतिमाः सर्वाः काशिराजसुतास्तदा}
{अम्बां चैवाम्बिकां चैव तथैवाम्बालिकामपि}


\twolineshloka
{राजानश्च समाहूताः पृथिव्यां भरतर्षभ}
{अम्बा ज्येष्ठाभवत्तासामम्बिका त्वथ मध्यमा}


\twolineshloka
{अम्बालिका च राजेन्द्र राजकन्या यवीयसी}
{सोऽहमेकरथेनैव गतः काशिपतेः पुरीम्}


\twolineshloka
{अपश्यं ता महाबाहो तिस्रः कन्याः स्वलङ्कृताः}
{राज्ञश्चैव समाहूतान्पार्थिवान्पृथिवीपते}


\twolineshloka
{ततोऽहं तान्नृपान्सर्वानाहूय समरे स्थितान्}
{रथमारोपयाञ्चक्रे कन्यास्ता भरतर्षभ}


\threelineshloka
{वीर्यशुल्काश्च ता ज्ञात्वा समारोप्य रथं तदा}
{अवोचं पार्थिवान्सर्वानहं तत्र समागतान्}
{भीष्मः शान्तनवः कन्या हरतीति पुनःपुनः}


\twolineshloka
{ते यतध्वं परं शक्त्या सर्वे मोक्षाय पार्थिवाः}
{प्रसह्य हि हराम्येष मिषतां वो नरर्षभाः}


\twolineshloka
{ततस्ते पृथिवीपालाः समुत्पतुरुदायुधाः}
{योगो योग इति क्रुद्धाः सारथीनभ्यचोदयन्}


\twolineshloka
{ते रथैर्गसङ्काशैर्गजैश्च गजयोधिनः}
{पुष्टैश्वाश्वैर्महीपालाः समुत्पेतुरुदायुधाः}


\twolineshloka
{ततस्ते मां महीपालाः सर्व एव विशां पते}
{रथव्रातेन महता सर्वतः पर्यवारयन्}


\twolineshloka
{तानहं शरवर्षेण समन्तात्पर्यवारयम्}
{सर्वान्नृपांश्चाप्यजयं देवराडिव दानवान्}


\twolineshloka
{अपातयं शरैर्दीप्तैः प्रहसन्भरतर्षभ}
{तेषामापततां चित्रान्ध्वजान्हेमपरिष्कृतान्}


\twolineshloka
{एकैकेन हि बाणेन भूमौ पातितवानहम्}
{हयांस्तेषां गजांश्चैव सारथींश्चाप्यहं रणे}


\twolineshloka
{ते निवृत्ताश्च भग्नाश्च दृष्ट्वा तल्लाघवं मम}
{5-173-22b`प्रणिपेतुश्च सर्वे वै प्रशशंसुश्च पार्थिवाः}


\twolineshloka
{तत आदाय ताः कन्या नृपतींश्च विसृज्य तान्}
{'अथाऽहं हास्तिनपुरमायां जित्वा महीक्षितः}


\twolineshloka
{ततोऽहं ताश्च कन्या वै भ्रातुरर्थाय भारत}
{तच्च कर्म महाबाहो सत्यवत्यै न्यवेदयम्}


\chapter{अध्यायः १७४}
\twolineshloka
{भीष्म उवाच}
{}


\twolineshloka
{ततोऽहं भरतश्रेष्ठ मातरं वीरमातरम्}
{अभिगम्योपसंगृह्य दाशेयीमिदमब्रुवम्}


\twolineshloka
{इमाः काशिपतेः कन्या मया निर्जित्य पार्थिवान्}
{विचित्रवीर्यस्य कृते वीर्यशुल्का हृता इति}


\twolineshloka
{ततो मूर्धन्युपाघ्राय पर्यश्रुनयना नृप}
{आह सत्यवती हृष्टा दिष्ट्या पुत्र जितं त्वया}


\twolineshloka
{सत्यवत्यास्त्वनुमते विवाहे समुपस्थिते}
{उवाच वाक्यं सव्रीडा ज्येष्ठा काशिपतेः सुता}


\twolineshloka
{भीष्म त्वमसि धर्मज्ञः सर्वशास्त्रविशारदः}
{श्रुत्वा च वचनं धर्म्यं मह्यं कर्तुमिहार्हसि}


\twolineshloka
{मया साल्वपत्तिः पूर्वं मनसाऽभिवृतो वरः}
{तेन चास्मि वृता पूर्वं रहस्यविदिते पितुः}


\twolineshloka
{कथं मामन्यकामां त्वं राजधर्ममतीत्य वै}
{वासयेथा गृहे भीष्म कौरवः सन्विशेषतः}


\twolineshloka
{एतद्बुद्ध्या विनिश्चित्य मनसा भरतर्षभ}
{यत्क्षमं ते महाबाहो तदिहारब्धुमर्हसि}


\twolineshloka
{स मां प्रतीक्षते व्यक्तं साल्वराजो विशांपते}
{तस्मान्मां त्वं कुरुश्रेष्ठ समनुज्ञातुमर्हसि}


\twolineshloka
{कृपां कुरु महाबाहो मयि धर्मभृतां वर}
{त्वं हि सत्यव्रतो वीर पृथिव्यामिति नः श्रुतम्}


\chapter{अध्यायः १७५}
\twolineshloka
{भीष्म उवाच}
{}


\threelineshloka
{ततोऽहं समनुज्ञाप्य कालीं गन्धवतीं तदा}
{मन्त्रिणश्चर्त्विजश्चैव तथैव च पुरोहितान्}
{समनुज्ञामिषं कन्यामम्बां ज्येष्ठां नराधिप}


\twolineshloka
{अनुज्ञाता ययौ सा तु कन्या साल्वपतेः पुरम्}
{वृद्धैर्द्विजातिभिर्गुप्ता धात्र्या चानुगता तदा}


\twolineshloka
{अतीत्य च तमध्वानमासमाद नराधिपम}
{सा तमासाद्य राजानं साल्वं वचनमब्रवीत्}


\twolineshloka
{आगताहं महाबाहो त्वामुद्दिश्य महामते}
{`अभिनन्दस्व मां राजन्सदा प्रियहिते रताम्}


\threelineshloka
{प्रतिपादय मां राजन्धर्मादींश्चर धर्मतः}
{त्वं हि मे मनसा ध्यातस्त्वया चाप्युपमन्त्रिता ॥भीष्म उवाच}
{'}


\twolineshloka
{तामब्रवीत्साल्वपतिः स्मयन्निव विशांपते}
{त्वयाऽन्यपूर्वया नाहं भार्यार्थी वरवर्णिनि}


\twolineshloka
{गच्छ भद्रे पुनस्तत्र सकाशं भीष्मकस्य वै}
{नाहमिच्छामि भीष्मेण गृहीतां त्वां प्रसह्य वै}


\twolineshloka
{त्वं हि भीष्मेण निर्जित्य नीता प्रीतिमती तदा}
{परामृश्य महायुद्धे निर्जित्य पृथिवीपतीन्}


\twolineshloka
{नाहं त्वय्यन्यपूर्वायां भार्यार्थी वरवर्णिनि}
{कथमस्मद्विधो राजा परपूर्वां प्रवेशयेत्}


\twolineshloka
{नारीं विदितविज्ञानः परेषां धर्ममादिशन्}
{यथेष्टं गम्यतां भद्रे मा त्वां कालोऽत्यगादयम्}


\twolineshloka
{अम्बा तमब्रवीद्राजन्ननङ्गशरपीडिता}
{नैवं वद महीपाल नैतदेवं कथंचन}


\twolineshloka
{नास्मि प्रीतिमती नीता भीष्मेणामित्रकर्शन}
{बलान्नीतास्मि रुदती विद्राव्य पृथिवीपतीन्}


\twolineshloka
{भजस्व मां साल्वपते भक्तां बालामनागसम्}
{भक्तानां हि परित्यागो न धर्मेषु प्रशस्यते}


\twolineshloka
{साहमामन्त्र्य गाङ्गेयं समरेष्वनिवर्तिनम्}
{अनुज्ञाता च तेनैव ततोऽहं भृशमागता}


\twolineshloka
{न स भीष्मो महाबाहुर्मामिच्छति विशांपते}
{भ्रातृहेतोः समारम्भो भीष्मस्येति श्रुतं मया}


\twolineshloka
{भगिन्यौ मम ये नीते अम्बिकाम्बालिके नृप}
{प्रादाद्विचित्रवीर्याय गाङ्गेयो हि यवीयसे}


\twolineshloka
{यथा साल्वपते नान्यं वरं यामि कथंचन}
{त्वामृते पुरुषव्याघ्र तथा मूर्धानमालभे}


\twolineshloka
{न चान्यपूर्वा राजेन्द्र त्वामहं समुपस्थिता}
{सत्यं ब्रवीमि साल्वैतत्सत्येनात्मानमालभे}


\twolineshloka
{भजस्व मां विशालाक्ष स्वयं कन्यामुपस्थिताम्}
{अनन्यपूर्वां राजेन्द्र त्वत्प्रसादाभिकाङ्क्षिणीम्}


\twolineshloka
{तामेवं भाषमाणां तु साल्वः काशिपतेः सुताम्}
{अत्यजद्भरतश्रेष्ठ जीर्णां त्वचमिवोरगः}


\twolineshloka
{एवं बहुविधैर्वाक्यैर्याच्यमानस्तया नृपः}
{नाश्रद्दधत्साल्वपतिः कन्यायां भरतर्षभ}


\twolineshloka
{ततः सा मन्युनाऽऽविष्टा ज्येष्ठा काशिपतेः सुता}
{अब्रवीत्साश्रुनयना बाष्पविप्लुतया गिरा}


\twolineshloka
{त्वया त्यक्ता गमिष्यामि यत्र तत्र विशांपते}
{तत्र मे गतयः सन्तु सन्तः सत्यं यथा ध्रुवम्}


\twolineshloka
{एवं तां भाषमाणां तु कन्यां साल्वपतिस्तदा}
{परितत्याज कौरव्य करुणं परिदेवतीम्}


\twolineshloka
{गच्छ गच्छेति तां साल्वः पुनः पुनरभाषत}
{बिभेमि भीष्मात्सुश्रोणि त्वं च भीष्मपरिग्रहः}


\threelineshloka
{एवमुक्ता तु सा तेन साल्वेनादीर्घदर्शिना}
{निश्चक्राम पुराद्दीना रुदती कुररी यथा ॥भीष्म उवाच}
{}


\twolineshloka
{निष्क्रामन्ती तु नगराच्चिन्तयामास दुःखिता}
{पृथिव्यां नास्ति युवतिर्विषमस्थतरा मया}


\twolineshloka
{बन्धुर्भिर्विप्रहीणास्मि साल्वेन च निराकृता}
{न च शक्यं पुनर्गन्तुं मया वारणसाह्वयम्}


\twolineshloka
{अनुज्ञाता तु भीष्मेण साल्वमुद्दिश्य कारणम्}
{किं नु गर्हाम्यथात्मानमथ भीष्मं दुरासदम्}


\twolineshloka
{अथवा पितरं मूढं यो मेऽकार्षीत्स्वयंवरम्}
{मयाऽयं स्वकृतो दोषो याऽहं भीष्मरथात्तदा}


\twolineshloka
{प्रवृत्ते दारुणे युद्धे साल्वार्थं नापतं पुरा}
{तस्येयं फलनिर्वृत्तिर्यदापन्नाऽस्मि मूढवत्}


\twolineshloka
{धिग्भीष्मं धिक्क मे मन्दं पितरं मूढचेतसम्}
{येनाहं वीर्यशुल्केन पण्यस्त्रीव प्रचोदिता}


\twolineshloka
{धिङ्मां धिक्साल्वराजानं धिग्धातारमथापि वा}
{येषां दुर्नीतभावेन प्राप्तास्म्यापदमुत्तमाम्}


\twolineshloka
{सर्वथा भागधेयानि स्वानि प्राप्नोति मानवः}
{अनयस्यास्य तु मुखं भीष्मः शान्तनवो मम}


\twolineshloka
{सा भीष्मे प्रतिकर्तव्यमहं पश्यामि सांप्रतम्}
{तपसा वा युधा वापि दुःखहेतुः स मे मतः}


\twolineshloka
{को नु भीष्मं युधा जेतुमुत्सहेत महीपतिः}
{एवं सा परिनिश्चित्य जगाम नगराद्बहिः}


\twolineshloka
{आश्रमं पुण्यशीलानां तापसानां महात्मनाम्}
{ततस्तामवसद्रात्रिं तापसैः परिवारिता}


\threelineshloka
{आचख्यौ च यथावृत्तं सर्वमात्मनि भारत}
{विस्तरेण महाबाहो निखिलेन शुचिस्मिता}
{हरणं च विसर्गं च साल्वेन च विसर्जनम्}


\twolineshloka
{ततस्तत्र महानासीद्ब्राह्मणः संशितव्रतः}
{शैखावत्यस्तपोबृद्धः शास्त्रे चारण्यके गुरुः}


\twolineshloka
{आर्ता तामाह स मुनिः शैखावत्यो महातपाः}
{निःश्वसन्तीं सतीं बालां दुःखशोकपरायणाम्}


\twolineshloka
{एवं गते तु किं भद्रे शक्यं कर्तुं तपस्विभिः}
{आश्रमस्थैर्महाभागे तपोयुक्तैर्महात्मभिः}


\twolineshloka
{सा त्वेनमब्रवीद्राजन्क्रियतां मदनुग्रहः}
{प्राव्राज्यमहमिच्छामि तपस्तप्स्यामि दुश्चरम्}


\twolineshloka
{मयैव यानि कर्माणि पूर्वदेहे तु मूढया}
{कृतानि नूनं पापानि तेषामेतत्फलं ध्रुवम्}


\twolineshloka
{नोत्सहे तु पुनर्गन्तुं स्वजनं प्रति तापसाः}
{प्रत्याख्याता निरानन्दा साल्वेन च निराकृता}


\twolineshloka
{उपदिष्टमिहेच्छामि तापस्यं वीतकल्मषाः}
{युष्माभिर्देवसंकाशैः कृपा भवतु वो मयि}


\twolineshloka
{स तामाश्वासयत्कन्यां दृष्टान्तागमहेतुभिः}
{सान्त्वयामास कार्यं च प्रतिजज्ञे द्विजैः सह}


\chapter{अध्यायः १७६}
\twolineshloka
{भीष्म उवाच}
{}


\twolineshloka
{ततस्ते तापसाः सर्वे कार्यवन्तोऽभवंस्तदा}
{तां कन्यां चिन्तयन्तस्ते किंकार्यमिति धर्मिणः}


\twolineshloka
{केचिदाहुः पितुर्वेश्म नीयतामिति तापसाः}
{केचिदस्यदुमालम्भे मतिं चक्रुर्हि तापसाः}


\twolineshloka
{केचित्साल्वपतिं गत्वा नियोज्यमिति मेनिरे}
{नेति केचिद्व्यवस्यन्ति प्रत्याख्याता हि तेन सा}


\twolineshloka
{एवं गते तु किं शक्यं भद्रे कर्तुं मनीषिभिः}
{पुनरूचुश्च तां सर्वे तापसाः संशितव्रताः}


\twolineshloka
{अलं प्रव्रजितेनेह भद्रे श्रृणु हितं वचः}
{इतो गच्छस्व भद्रं ते पितुरेव निवेशनम्}


\twolineshloka
{प्रतिपस्त्यति राजा स पिता ते यदनन्तरम्}
{तत्र वत्स्यसि कल्याणि सुखं सर्वगुणान्विता}


\threelineshloka
{न च तेऽन्या गतिर्न्याय्या भवेद्भद्रे यथा पिता}
{पतिर्वापि गतिर्नार्याः पिता वा वरवर्णिनि}
{गतिः पतिः समस्थाया विषमे च पिता गतिः}


\twolineshloka
{प्रव्रज्या हि सुदुःखेयं सुकुमार्या विशेषतः}
{राजपुत्र्याः प्रकृत्या चक कुमार्यास्तव भामिनि}


\twolineshloka
{भद्रे दोषा हि विद्यन्ते बहवो वरवर्णिनि}
{आश्रमे वै वसन्त्यास्ते न भवेयुः पितुर्गृहे}


% Check verse!
ततस्त्वन्येऽब्रुवन्वाक्यं तापसास्तां तपस्विनीम्
\threelineshloka
{त्वामिहैकाकिनीं दृष्ट्वा निर्जने गहने वने}
{प्रार्थयिष्यन्तिराजानस्तस्मान्मैवं मनः कृथाः ॥अम्बोवाच}
{}


\twolineshloka
{न शक्यं काशिनगरं पुनर्गन्तुं पितुर्गृहान्}
{अवज्ञाता भविष्यामि बान्धवानां न संशयः}


\threelineshloka
{उषितास्मि तथा बाल्ये पितुर्वेश्मनि तापसाः}
{नाहं गमिष्ये भद्रं वस्तत्र यत्र पिता मम}
{तपस्तप्तुमभीप्सामि तापसैः परिरक्षिता}


\threelineshloka
{यथा परेऽपि मे लोके न स्यादेवं महात्ययः}
{दौर्भाग्यं तापसश्रेष्ठास्तस्मात्तप्स्याम्यहं तपः ॥भीष्म उवाच}
{}


\threelineshloka
{इत्येवं तेषु विप्रेषु चिन्तयन्सु यथातथम्}
{राजर्षिस्तद्वनं प्राप्तस्तपस्वी होत्रवाहनः}
{`तां तथा भाविनीं दृष्ट्वा श्रुत्वा चोद्विग्रमानसः'}


\twolineshloka
{ततस्ते तापसाः सर्वे पूजयन्ति स्म तं नृपम्}
{पूजाभिः स्वागताद्याभिरासनेनोदकेन च}


\twolineshloka
{तस्योपविष्टस्य सतो विश्रान्तस्योपश्रृण्वतः}
{पुनरेव कथां चक्रुः कन्यां प्रति वनौकसः}


\twolineshloka
{अम्बायास्तां कथां श्रुत्वा काशिराज्ञश्च भारत}
{राजर्षिः स महातेजा बभूवोद्विग्रमानसः}


\twolineshloka
{तां तथावादिनीं श्रुत्वा दृष्ट्वा च स महातपाः}
{राजर्षिः कृपयाविष्टो महात्मा होत्रवाहनः}


\twolineshloka
{स वेपमान उत्थाय मातुस्तस्याः पिता तदा}
{तां कन्यामङ्कमारोप्य पर्यश्वासयत प्रभो}


\twolineshloka
{स तामपृच्छत्कार्त्स्न्येन व्यसनोत्पत्तिमादितः}
{सा च तस्मै यथावृत्तं विस्तरेण न्यवेदयत्}


\twolineshloka
{ततः स राजर्षिरभूद्दुःखशोकसमन्वितः}
{कार्यं च प्रतिपेदे तन्मनसां सुमहातपाः}


\twolineshloka
{अब्रवीद्वेपमानश्च कन्यामार्तां सुदुःखितः}
{मा गाः पितुर्गृहं भद्रे मातुस्ते जनको ह्यहम्}


\twolineshloka
{दुःखं छिन्द्यामहं ते वै मयि वर्तस्व पुत्रिके}
{पर्याप्तं ते मनो वत्से यदेवं परिशुष्यसि}


\twolineshloka
{गच्छ मद्वचनाद्रामं जामदग्न्यं तपस्विनम्}
{रामस्ते सुमहद्दुःखं शोकं चैवापनेष्यति}


\twolineshloka
{हनिष्यति रणे भीष्मं न करिष्यति चेद्वचः}
{तं गच्छ भार्गवश्रेष्ठं कालाग्निसमतेजसम्}


\twolineshloka
{प्रतिष्ठापयिता स त्वां समे पथि महातपाः}
{ततस्तु सुस्वरं बाष्पसुत्सृजन्ती पुनः पुनः}


\twolineshloka
{अब्रवीत्पितरं मातुः मा तदा होत्रवाहनम्}
{अभिवादयित्वा शिरसा गमिष्ये तव शासनात्}


\fourlineindentedshloka
{अपि नामाद्य पश्येयमार्यं तं लोकविश्रुतम्}
{कथं च तीव्रं दुःकं मे नाशकयिष्यति भार्गवः}
{एतदिच्छाम्यहं ज्ञातुं यथा यास्यमि तत्र वैः ॥होत्रवाहन उवाच}
{}


\twolineshloka
{रामं द्रक्ष्यसि भद्रे त्वं जामदग्न्यं महावने}
{उग्रे तपसि वर्तन्तं सत्यसन्धं महाबलम्}


\twolineshloka
{महेन्द्रं वै गिरिश्रेष्ठं रामो नित्यमुपास्ति ह}
{ऋषयो वेदविद्वांसो गन्धर्वाप्सरसस्तथा}


\twolineshloka
{तत्र गच्छस्व भद्रं ते ब्रूयाश्चैनं वचो मम}
{अभिवाद्य च तं मूंर्ध्ना तपोवृद्धं दृढव्रतम्}


\twolineshloka
{ब्रूयाश्चैनं पुनर्भद्रे यत्ते कार्यं मनीषितम्}
{मयि संकीर्तिते रामः सर्वं तत्ते करिष्यति}


\twolineshloka
{मम रामः सखा वत्से प्रीतियुक्तः सुहृच्च मे}
{जमदग्निसुतो वीरः सर्वशस्त्रभृतां वरः}


\twolineshloka
{एवं ब्रुवति कन्यां तु पार्थिवे होत्रवाहने}
{अकृतव्रणः प्रादुरासीद्रामस्यानुचरः प्रियः}


\twolineshloka
{ततस्ते मनुयः सर्वे समुत्तस्थुः महस्रशः}
{स च राजा वयोवृद्धः सृञ्ययो होत्रवाहनः}


\twolineshloka
{ततो दृष्ट्वा कृतातिथ्यमन्योन्यं ते वनौकसः}
{सहिता भरतश्रेष्ठ निषेदुः परिवार्य तम्}


\twolineshloka
{ततस्ते कथयामासुः कथास्तास्ता मनोरमाः}
{धन्या दिव्याश्च राजेन्द्र प्रीतिहर्षमुदा युताः}


\twolineshloka
{ततः कथान्ते राजर्षिर्महात्मा होत्रवाहनः}
{रामं श्रेष्ठं महर्षीणामपृच्छदकृतव्रणम्}


\threelineshloka
{क्व संप्रति महाबाहो जामदग्न्यः प्रतापवान्}
{अकृतव्रण शक्यो वै द्रुष्टुं वेदविदां वर ॥अकृतव्रण उवाच}
{}


\twolineshloka
{भवन्तमेव सततं रामः कीर्तयति प्रभो}
{सृञ्जयो मे प्रियसखो राजर्षिरिति पार्थिव}


\twolineshloka
{इह रामः प्रभाते श्वो भवितेति मतिर्मम}
{द्रष्टस्येनमिहायान्तं तव दर्शनकाङ्क्षया}


\threelineshloka
{इयं च कन्या राजर्षे किमर्थं वनमागता}
{कस्य चेयं तव च का भवतीच्छामि वेदितुम् ॥होत्रवाहन उवाच}
{}


\twolineshloka
{दौहित्रीयं मम विभो काशिराजसुता प्रिया}
{ज्येष्ठा स्वयंवरे तस्थौ भगिनीभ्यां सहानध}


\twolineshloka
{इयमम्बेति विख्याता ज्येष्ठा काशिपतेः सुता}
{अम्बिकाम्बालिके कन्ये कनीयस्यौ तपोधन}


\twolineshloka
{समेतं पार्थिवं क्षत्रं काशिपुर्यां ततोऽभवत्}
{कन्यानिमित्तं विप्रर्षे तत्रासीदुत्सवो महान्}


\twolineshloka
{तता किल महावीर्यो भीष्मः शान्तनवो नृपान्}
{अधिक्षिप्य महातेजास्तिस्त्रः कन्या जहार ताः}


\twolineshloka
{निर्जित्य पृथिवीपालानथ भीष्मो जगाह्वयम्}
{आजगाम विशुद्धात्मा कन्याभिः सह भारतः}


\twolineshloka
{सत्यवत्यै निवेद्याथ विवाहं समनन्तरम्}
{भ्रातुर्विचित्रवीर्यस्य समाज्ञापयत प्रभुः}


\twolineshloka
{तं तु वैवाहिकं दृष्ट्वा कन्येयं समुपार्जितम्}
{अब्रवीत्तत्र गाङ्गेयं मन्त्रिमध्ये द्विजर्षभ}


\twolineshloka
{मया साल्वपतिर्वीरो मनसाऽभिवृतः पतिः}
{न ममार्हसि धर्मज्ञ दातुं भ्रात्रेऽन्यमानसाम्}


\twolineshloka
{तच्छ्रुत्वा वचनं भीष्मः संमन्त्र्य सह मन्त्रिभिः}
{निश्चित्य विससर्जेमां सत्यवत्या मते स्थितः}


\twolineshloka
{अनुज्ञाता तु भीष्मेण साल्वं सौभपतिं ततः}
{कन्येयं मुदिता तत्र काले वाचनमब्रवीत्}


\twolineshloka
{विसर्जितास्मि भीष्मेण धर्मं मां प्रतिपादय}
{मनसाभिवृत्तः पूर्वं मया त्वं पार्थिवर्षम}


\twolineshloka
{प्रत्याचख्यौ च साल्वोऽस्याश्चारित्रस्याभिशङ्कितः}
{सेयं तपोवनं प्राप्ता तापस्येऽभिरता भृशम्}


\fourlineindentedshloka
{मया च प्रत्यभिज्ञाता वंशस्य परिकीर्तनात्}
{अस्य दुःखस्य चोत्पत्तिं भीष्ममेवेह मन्यते}
{अम्बोवाच}
{}


\twolineshloka
{भगवन्नेवमेवेह यथाह पृथिवीपतिः}
{शरीरकर्ता मातुर्मे सृञ्जयो होत्रवाहनः}


\twolineshloka
{न ह्युत्सहे स्वनगरं प्रतियातुं तपोधन}
{अपमानभयाच्चैव व्रीडया च महामुने}


\twolineshloka
{यत्तु मां भगवान्रामो वक्ष्यति द्विजसत्तम}
{तन्मे कार्यतमं कार्यमिति मे भगवन्मतिः}


\chapter{अध्यायः १७७}
\twolineshloka
{अकृतव्रण उवाच}
{}


\twolineshloka
{दुःख द्वयोरिदं भद्रे कतरस्य चिकीर्षसि}
{प्रतिकर्तव्यमबले तत्त्वं वत्से वदस्व मे}


\twolineshloka
{यदि सौभपतिर्भद्रे नियोक्तव्यो मतस्तव}
{नियोक्ष्यति महात्मा स रामस्त्वद्धितकाम्यया}


\twolineshloka
{अथापगेयं भीष्मं त्वं रामेणेच्छसि धीमता}
{रणे विनिर्जितं द्रष्टुं कुर्यात्तदपि भार्गवः}


\threelineshloka
{सृञ्जयस्य वचः श्रुत्वा तव चैव शुचिस्मिते}
{यदत्र ते भृशं कार्यं तदद्यैव विचिन्त्यताम् ॥अम्बोवाच}
{}


\twolineshloka
{अपनीतास्मि भीष्मेण भगवन्नविजानता}
{नाभिजानाति मे भीष्मो ब्रह्मन्साल्वगतं मनः}


\twolineshloka
{एतद्विचार्य मनसा भवानेतद्विनिश्चयम्}
{विचिनोतु यथान्यायं विधानं क्रियतां तथा}


\twolineshloka
{भीष्मे वा कुरुशार्दूले शाल्वराजेऽथवा पुनः}
{उभयोरेव वा ब्रह्मन्युक्तं यत्तत्समाचर}


\threelineshloka
{निवेदितं मया ह्येतद्दुःखमूलं यथातथम्}
{विधानं तत्र भगवन्कर्तुमर्हसि युक्तितः ॥अकृतव्रण उवाच}
{}


\twolineshloka
{उपपदमिदं भद्रे यदेवं वरवर्णिनि}
{धर्मं प्रतिवचो ब्रूयाः श्रृणु चदं वचो मम}


\twolineshloka
{यदि त्वामापगेयो वै न नयेद्गजसाह्वयम्}
{साल्वस्त्वा शिरसा भीरु गृह्णीयाद्रामचोदितः}


\twolineshloka
{तेन त्वं निर्जिता भद्रे यस्मान्नीतासि भामिनि}
{संशयः साल्वराजस्य तेन त्वयि सुमध्यमे}


\threelineshloka
{भीष्मः पुरुषमानी च जितकाशी तथैव च}
{तस्मात्प्रतिक्रिया युक्ता भीष्मे कारयितुं तव ॥अम्बोवाच}
{}


\twolineshloka
{ममाप्येष सदा ब्रह्मन्हृदि कामोऽभिवर्तते}
{घातयेयं यदि रणे भीष्ममित्येव नित्यदा}


\threelineshloka
{भीष्मं वा साल्वराजं वा यं वा दोषेण गच्छसि}
{प्रशाधि तं महाबाहोयत्कृतेऽहं सुदुःखिता ॥भीष्म उवाच}
{}


\twolineshloka
{एवं कथयतामेव तेषां स दिवसो गतः}
{रात्रिश्च भरतश्रेष्ठ सुखशीतोष्णमारुता}


\twolineshloka
{ततो रामः प्रादुरासीत्प्रज्वलन्निव तेजसा}
{शिष्यैः परिवृतो राजञ्जटाचीरधरो मुनिः}


\twolineshloka
{धनुष्पाणिरदीनात्मा खङ्गं बिभ्रत्परश्वधी}
{विरजा राजशार्दूल सृञ्जयं सोऽभ्ययान्नृपम्}


\twolineshloka
{ततस्तं तापसा दृष्ट्वा स च राजा महातपाः}
{तस्थुः प्राञ्जलयो राजन्सा च कन्या तपस्विनी}


\twolineshloka
{पूजयामासुरव्यग्रा मधुपर्केण भार्गवम्}
{अर्चितश्च यथान्यायं निषसाद सहैव तैः}


\twolineshloka
{ततः पूर्वव्यतीतानि कथयन्तौ स्म तावुभौ}
{आसातां जामदग्न्यश्च सृञ्जयश्चैव भारत}


\twolineshloka
{तथा कथान्ते राजर्षिर्भृगुश्रेष्ठं महाबलम्}
{उवाच मधुरं काले रामं वचनमर्थवत्}


\twolineshloka
{रामेयं मम दौहित्री काशिराजसुता प्रभो}
{अस्याः श्रृणु यथातत्त्वं कार्यं कार्यविशारद}


\twolineshloka
{परमं कथ्यतां चेति तां रामः प्रत्यभाषत}
{ततः साभ्यवदद्रामं ज्वलन्तमिव पावकम्}


\twolineshloka
{ततोऽभिवाद्य चरणौ रामस्य शिरसा शुभौ}
{स्पृष्ट्वा पद्मदलाभाभ्यां पाणिभ्यामग्रतः स्थिता}


\threelineshloka
{रुरोद सा शोकवती बाष्पव्याकुललोचना}
{प्रपेदे शरणं चैव शरण्यं भृगुनन्दनम् ॥राम उवाच}
{}


\threelineshloka
{यथा त्वं सृंजयस्यास्य तथा मे त्वं नृपात्मजे}
{ब्रूहि यत्ते मनोदुःखं करिष्ये वचनं तव ॥अम्बोवाच}
{}


\threelineshloka
{भगवञ्शरणं त्वाद्य प्रपन्नाऽस्मि महाव्रतम्}
{शोकपङ्कार्णवान्मग्नं घोरादुद्धर मां विभो ॥भीष्म उवाच}
{}


\twolineshloka
{तस्याश्च दृष्ट्वा रूपं च वपुश्चाभिनवं पुनः}
{सौकुमार्यं परं चैव रामश्चिन्तापरोऽभवत्}


\twolineshloka
{किमियं वक्ष्यतीत्येवं विममर्श भृगूद्वहः}
{इति दध्यौ चिरं रामः कृपयाभिपरिप्लुतः}


\twolineshloka
{कथ्यतामिति सा भूयो रामेणोक्ता शुचिस्मिता}
{सर्वमेव यथातत्त्वं कथयामास भार्गवे}


\threelineshloka
{तच्छुत्वा जामदग्र्यस्तु राजपुत्र्या वचस्तदा}
{उवाच तां वरारोहां निश्चित्यार्थविनिश्चयम् ॥राम उवाच}
{}


\twolineshloka
{प्रेषयिष्यामि भीष्माय कुरुश्रेष्ठाय भामिनि}
{करिष्यति वचो मह्यं श्रुत्वा च स नराधिपः}


\twolineshloka
{न चेत्करिष्यति वचो मयोक्तं जाह्नवीसुतः}
{धक्ष्याम्यहं रणे भद्रे सामात्यं शस्त्रतेजसा}


\threelineshloka
{अथवा ते मतिस्तत्र राजपुत्रि न वर्तते}
{यावत्साल्वपतिं वीरं योजयाम्यत्र कर्मणि ॥अम्बोवाच}
{}


\twolineshloka
{विसर्जिताऽहं भीष्मेण श्रुत्वैव भृगुनन्दन}
{साल्वराजगतं भावं मम पूर्वं मनीषितम्}


\twolineshloka
{सौभराजमुपेत्याहमवोचं दुर्वचं वचः}
{न च मां प्रत्यगृह्णात्स चारित्र्यपरिशङ्कितः}


\twolineshloka
{एतत्सर्वं विनिश्चित्य स्वबुद्ध्या भृगुनन्दन}
{यदत्रौपयिकं कार्यं तच्चिन्तयितुमर्हसि}


\twolineshloka
{मम तु व्यसनस्यास्य भीष्मो मूलं महाव्रतः}
{येनाहं वशमानीता समत्क्षिप्य बलात्तदा}


\twolineshloka
{भीष्मं जहि महाबाहो यत्कृते दुःखमीदृशम्}
{प्राप्ताहं भृगुशार्दूल चराम्यप्रियमुत्तमम्}


\twolineshloka
{स हि लुब्धश्च नीचश्च जितकाशी च भार्गव}
{तस्मात्प्रतिक्रिया कर्तुं युक्ता तस्मै त्वयाऽनघ}


\twolineshloka
{एष मे ह्रियमाणाया भारतेन तदा विभो}
{अभवद्धृदि संकल्पो घातयेदं महाव्रतम्}


\twolineshloka
{तस्मात्कामं ममाद्येमं राम संपादयानघ}
{जहि भीष्मं महाबाहो यथा वृत्रं पुरन्दरः}


\chapter{अध्यायः १७८}
\twolineshloka
{भीष्म उवाच}
{}


\twolineshloka
{एवमुक्तस्तदा रामो जहि भीष्ममिति प्रभो}
{उवाच रुदतीं कन्यां चोदयन्तीं पुनः पुनः}


\twolineshloka
{काश्ये न कामं गृह्णामि शस्त्रं वै वरवर्णिनि}
{ऋते ब्रह्मविदां हेतोः किमन्यत्करवाणि ते}


\twolineshloka
{वाचा भीष्मश्च साल्वश्च मम राज्ञि वशानुगौ}
{भविष्यतोऽनवद्याङ्गि तत्करिष्यामि मा शुचः}


\threelineshloka
{न तु शस्त्रं ग्रहीष्यामि कथंचिदपि भामिनि}
{ऋते नियोगाद्विप्राणामेष मे समयः कृतः ॥अम्बोवाच}
{}


\threelineshloka
{मम दुःखं भगवता व्यपनेयं यतस्ततः}
{तच्च भीष्मप्रसूतं मे तं जहीश्वर मा चिरम् ॥राम उवाच}
{}


\threelineshloka
{काशिकन्ये पुनर्ब्रूहि भीष्मस्ते चरणावुभौ}
{शिरसा वन्दनार्होऽपि ग्रहीष्यति गिरा मम ॥अम्बोवाच}
{}


\threelineshloka
{जहि भीष्मं रणे राम मम चेदिच्छसि प्रियम्}
{प्रतिश्रुतं च यदपि तत्सत्यं कर्तुमर्हसि ॥भीष्म उवाच}
{}


\twolineshloka
{तयोः संवदतोरेवं राजन्रामाम्बयोस्तदा}
{अकृतव्रणो जामदग्न्यमिदं वचनमब्रवीत्}


\twolineshloka
{शरणागतां महाबाहो कन्यां न त्यक्तुमर्हसि}
{यदि भीष्मो रणे राम समाहूतस्त्वया मृधे}


\threelineshloka
{निर्जितोऽस्मीति वा ब्रूयात्कुर्याद्वा वचनं तव}
{कृतमस्या भवेत्कार्यं कन्याया भृगुनन्दन ॥वाक्यं सत्यं च ते वीर भविष्यति कृतं विभो}
{इयं चापि प्रतिज्ञा ते तदा राम महामुने}


\twolineshloka
{जित्वा वै क्षत्रियान्सर्वान्ब्रह्मणेषु प्रतिश्रुता}
{ब्राह्मणः क्षत्रियो वैश्यः शूद्रश्चैव रणे यदि}


\twolineshloka
{ब्रह्मद्विङ्भविता तं वै हनिष्यामीति भार्गव}
{शरणार्थे प्रपन्नानां भीतानां शरणार्थिनाम्}


\twolineshloka
{न शक्ष्यामि परित्यागं कर्तुं जीवन्कथंचन}
{यश्च कृत्स्नं रणे क्षत्रं विजेष्यति समागतम्}


\fourlineindentedshloka
{दीप्तात्मानमहं तं च हनिष्यामीति भार्गव}
{स एवं विजयी राम भीष्मः कुरुकुलोद्वहः}
{तेन युध्यस्व सङ्ग्रमे समेत्य भृगुनन्दन ॥राम उवाच}
{}


\twolineshloka
{स्मराम्यहं पूर्वकृतां प्रतिज्ञामृषिसत्तम}
{तथैव त्त करिष्यामि यथा साम्नैव लप्स्यते}


\twolineshloka
{कार्यमेतन्महद्ब्रह्मन्काशिकान्यामनोगतम्}
{गमिष्यामि स्वयं कन्यामादाय यत्र सः}


\twolineshloka
{यदि भीष्मो रणश्लाघी न करिष्यति मे वचः}
{हनिष्याम्येनमुद्रिक्तमिति मे निश्चिता मतिः}


\threelineshloka
{न हि बाणा मयोत्सृष्टाः सज्जन्तीह शरीरिणाम्}
{कायेषु विदितं तुभ्यं पुरा क्षत्रियसंगरे ॥भीष्म उवाच}
{}


\twolineshloka
{एवमुक्त्वा ततो रामः सह तैर्ब्रह्मवादिभिः}
{प्रयाणाय मतिं कृत्वा समुत्तस्थौ महातपाः}


\twolineshloka
{ततस्ते तामुषित्वा तु रजनीं तत्र तापसाः}
{हुताग्नयो जप्तजप्याः प्रतस्थुर्मञ्जिघांसया}


\twolineshloka
{अभ्यगच्छत्ततो रामः सह तैर्ब्रह्मवादिभिः}
{कुरुक्षेत्रं महाराज कन्यया सह भारत}


\twolineshloka
{न्यविशन्त ततः सर्वे परिगृह्य सरस्वतीम्}
{तापसास्ते महात्मानो भृगुश्रेष्ठपुरस्कृताः}


\twolineshloka
{ततस्तृतीये दिवसे समे देशे व्यवस्थितः}
{प्रेषयामास मे राजन्प्राप्तोऽस्मीति महाव्रतः}


\twolineshloka
{तमागतमहं श्रुत्वा विषयान्तं महाबलम्}
{अभ्यगच्छं जवेनाशु प्रीत्या तेजोनिधिं प्रभुम्}


\twolineshloka
{गां पुरस्कृत्य राजेन्द्र ब्राह्मणैः परिवारितः}
{ऋत्विग्भिर्देवकल्पैश्च तथैव च पुरोहितैः}


\threelineshloka
{स मामभिगतं दृष्ट्वा जामदग्न्यः प्रतापवान्}
{प्रतिजग्राह तां पूजां वचनं चेदमब्रवीत् ॥राम उवाच}
{}


\twolineshloka
{भीष्म कां बुद्धिमास्थाय काशिराजसुता तदा}
{अकामेन त्वया नीता पुनश्चैव विसर्जिता}


\twolineshloka
{विभ्रंशिता त्वया हीयं धर्मादास्ते यशस्विनी}
{परामृष्टां त्वया हीमां को हि गन्तुमिहार्हति}


\twolineshloka
{प्रत्याख्याता हि साल्वेन त्वया नीतेति भारत}
{तस्मादिमां मन्नियोगात्प्रतिगृह्णीष्व भारत}


\threelineshloka
{स्वधर्मं पुरुषव्याघ्र राजपुत्री लभत्वियम्}
{न युक्तस्त्ववमानोऽयं राज्ञां कर्तुं त्वयाऽनघ ॥भीष्म उवाच}
{}


\twolineshloka
{ततस्तं वै विमनसमुदीक्ष्याहमथाब्रुवम्}
{नाहमेनां पुनर्दद्यां ब्रह्मन्भ्रात्रे कथञ्चन}


\twolineshloka
{साल्वस्याहमिति प्राह पुरा मामेव भार्गव}
{मया चैवाभ्यनुज्ञाता गतेयं नगरं प्रति}


\twolineshloka
{न भयान्नाप्यनुक्रोशान्नार्थलोभान्न काम्यया}
{क्षात्रं धर्ममहं जह्यमिति मे व्रतमाहितम्}


\twolineshloka
{अथ मामब्रवीद्रामः क्रोधपर्याकुलेक्षणः}
{न करिष्यसि चेदेतद्वाक्यं मे नरपुङ्गव}


\twolineshloka
{हनिष्यामि सहामात्यं त्वामद्येति पुनः पुनः}
{संरम्भादब्रवीद्रामः क्रोधपर्याकुलेक्षणः}


\twolineshloka
{तमहं गीर्भिरिष्टाभि पुनः पुनररिन्दम}
{अयाचं भृगुशार्दूलं न चैव प्रशशाम सः}


\twolineshloka
{प्रणम्य तमहं मूर्ध्ना भूयो ब्राह्मणसत्तमम्}
{अब्रुवं कारणं किं तद्यत्त्वं युद्धं मयेच्छसि}


\twolineshloka
{इष्वस्त्रं मम बालस्य भवतैव चतुर्विधम्}
{उपदिष्टं महाबाहो शिष्योऽस्मि तव भार्गव}


\twolineshloka
{ततो मामब्रवीद्रामः क्रोधसंरक्तलोचनः}
{जानीषे मां गुरुं भीष्म गृह्णासीमां न चैव ह}


\twolineshloka
{सुतां काश्यस्य कौरव्य मत्प्रियार्थं महामते}
{न हि मे विद्यते शान्तिरन्यथा कुरुनन्दन}


\twolineshloka
{गृहाणेमां महाबाहो रक्षस्व कुलमात्मनः}
{त्वया विभ्रंशिता हीयं भर्तारं नाधिगच्छति}


\twolineshloka
{तथा ब्रुवन्तं तमहं रामं परपुरंयम्}
{नैतदेवं पुनर्भावि ब्रह्मर्षे किं श्रमेण ते}


\twolineshloka
{गुरुत्वं त्वयि संप्रेक्ष्य जामदग्न्य पुरातनम्}
{प्रसादये त्वां भगवंस्त्यक्तैषा तु पुरा मया}


\twolineshloka
{को जातु परभावां हि नारीं व्यालीमिव स्थिताम्}
{वासयेत गृहे जानन्स्त्रीणां दोषो महात्ययः}


\twolineshloka
{न भयाद्वासवस्यापि धर्मं जह्यां महाव्रत}
{प्रसीद मा वा यद्वा ते कार्यं तत्कुरु मा चिरम्}


\twolineshloka
{अयं चापि विशुद्धात्मन्पुराणे श्रूयते विभो}
{मरुत्तेन महाबुद्धे गीतः श्लोको महात्मना}


\twolineshloka
{गुरोरप्यवलिप्तस्य कार्याकार्यमजानतः}
{उत्पथं प्रतिपन्नस्य परित्यागो विधीयते}


\twolineshloka
{स त्वं गुरुरिति प्रेम्णा मया संमानितो भृशम्}
{गुरुवृत्तिं न जानीषे तस्माद्योत्स्यामि वै त्वया}


\twolineshloka
{गुरुं न हन्यां समरे ब्राह्मणं च विशेषतः}
{विशेषतस्तपोवृद्धमेवं क्षान्तं मया तव}


\twolineshloka
{उद्यतेषुमथो दृष्ट्वा ब्राह्मणं क्षत्रबन्धुवत्}
{यो हन्यात्समरे क्रुद्धं युध्यन्तमपलायिनम्}


\twolineshloka
{ब्रह्महत्या न तस्य स्यादिति धर्मेषु निश्चयः}
{क्षत्रियाणां स्थितो धर्म क्षत्रियोऽस्मि तपोधन}


\twolineshloka
{यो यथा वर्तते यस्मिंस्तस्मिन्नेव प्रवर्तयन्}
{नाधर्मं समवाप्नोति न चाश्रेयश्च विन्दति}


\twolineshloka
{अर्थे वा यदि वा धर्मे समर्थो देशकालवित्}
{अर्थसंशयमापन्नः श्रोयान्निःसंशयो नरः}


\twolineshloka
{यस्मात्संशयितेऽप्यर्थेऽयथान्यायं प्रवर्तसे}
{तस्माद्योत्स्यामि सहितस्त्वया राम महाहवे}


\twolineshloka
{पश्य मे बाहुवीर्यं च विक्रमं चातिमानुषम्}
{एवं गतेऽपि तु मया यच्छक्यं भृगुनन्दन}


\twolineshloka
{तत्करिष्ये कुरुक्षेत्रे योत्स्ये विप्र त्वया सह}
{द्वन्द्वे राम यथेष्टं मे सञ्जीभव महाद्युते}


\twolineshloka
{तत्र त्वं निहतो राम मया शरशतार्दितः}
{प्राप्स्यसे निर्जिताँल्लोकाञ्शस्त्रपूतो महारणे}


\twolineshloka
{स गच्छ विनिवर्तस्व कुरुक्षोत्रं रणप्रिय}
{तत्रैष्यामि महाबाहो युद्धाय त्वांतपोधन}


\twolineshloka
{अपि यत्र त्वया राम कृतं शौचं पुरा पितुः}
{तत्राहमपि हत्वा त्वां शौचं कर्तास्मि भार्गव}


\twolineshloka
{तत्र राम समागच्छ त्वरितं युद्धदुर्मद}
{व्यपनेष्यामि ते दर्पं पौराणं ब्राह्मणब्रुव}


\twolineshloka
{यच्चापि कत्थसे राम बहुशः परिवत्सरे}
{निर्जिताः क्षत्रिया लोके मयैकेनेति तच्छ्रुणु}


\twolineshloka
{न तदा जातवान्भीष्मः क्षत्रियो वापि मद्विधः}
{पञ्चाञ्चातानि तेजांसि तृणेषु ज्वलितं त्वया}


\threelineshloka
{यस्ते युद्धमयं दर्पं कामं च व्यपनाशयेत्}
{सोऽहं जातो महाबाहो भीष्मः परपुरंजयः}
{व्यपनेष्यामि ते दर्पं युद्धे राम न संशयः}


\twolineshloka
{ततो मामब्रवीद्रामः प्रहसन्निव भारत}
{दिष्ट्या भीष्म मया सार्धं योद्धुमिच्छसि संगरे}


\twolineshloka
{अयं गच्छामि कौरव्य कुरुक्षेत्रं त्वया सह}
{भाषितं ते करिष्यामि तत्रागच्छ परंतप}


\twolineshloka
{तत्र त्वां निहतं माता मया शरशताचितम्}
{जाह्नवी पश्यतां भीष्म गृध्रकङ्कबलाशनम्}


\twolineshloka
{कृपणं त्वामभिप्रेक्ष्य सिद्धचारणसेविता}
{मया विनिहतं देवी रोदतामद्य पार्थिव}


\twolineshloka
{अतदर्हा महाभागा भगीरथसुताऽनघा}
{या त्वामजीजनन्मन्दं युद्धकामुकमातुरम्}


\twolineshloka
{एहि गच्छ मया भीष्म युद्धकामुक दुर्मद}
{गृहाण सर्वं कौरव्य रथादि भरतर्षभ}


\twolineshloka
{इति ब्रुवाणं तमहं रामं परपुरंजयमक्}
{प्रणम्यक शिरसा राममेवमस्त्वित्यथाब्रवम्}


\twolineshloka
{एवमुक्त्वा ययौ रामः कुरुक्षेत्रं युयुत्सया}
{प्रविश्य नगरं चाहं सत्यवत्यै न्यवेदयम्}


\twolineshloka
{ततः कृतस्वस्त्ययनो मात्रा च प्रतिनन्दितः}
{द्विजातीन्वाच्य पुण्याहं स्वस्ति चैव महाद्युते}


\twolineshloka
{रथमास्थाय रुचिरं राजतं पाण्डुरैर्हयैः}
{सूपस्करं स्वधिष्ठानं वैयाघ्रपरिवारणम्}


\twolineshloka
{उपपन्नं महाशस्त्रैः सर्वोपकरणान्वितम्}
{तत्कुलीनेन वीरेण हयशास्त्रविदा रणे}


\twolineshloka
{यत्तं सूतेन शिष्टेन बहुशो दृष्टकर्मणा}
{दंशितः पाण्डुरेणाहं कवचेन वपुष्मता}


\twolineshloka
{पाण्डुरं कार्मुकं गृह्य प्रायां भरतसत्तम}
{पाण्डुरेणातपत्रेण ध्रियमाणेन मूर्धनि}


\twolineshloka
{पाण्डुरैश्चापि व्यजनैर्वीज्यमानो नराधिप}
{शुक्लवासाः सितोष्णीषः सर्वशुक्लविभूषणः}


\twolineshloka
{स्तूयमानो जयाशीर्भिर्निष्क्रम्य गजसाह्वयात्}
{कुरुक्षेत्रं रणक्षेत्रमुपायां भरतर्षभ}


\twolineshloka
{ते हयाश्चोदितास्तेन सूतेन परमाहवे}
{अवहन्मां भृशं राजन्मनोमारुतरंहसः}


\twolineshloka
{गत्वाऽहं तत्कुरुक्षेत्रं स च रामः प्रतापवान्}
{युद्धाय सहसा राजन्पराक्रान्तौ परस्परम्}


\twolineshloka
{ततः सदर्शनेऽतिष्ठं रामस्यातितपस्विनः}
{प्रगृह्य शङ्खप्रवरंक ततः प्राधममुत्तमम्}


\twolineshloka
{ततस्तत्र द्विजा राजंस्तापसाश्च वनौकसः}
{अपश्यन्त रणं दिव्यं देवाः सेन्द्रगणास्तदा}


\twolineshloka
{ततो दिव्यानि माल्यानि प्रादुरासंस्ततस्ततः}
{वादित्राणि च दिव्यानि मेघवृन्दानि चैव ह}


\twolineshloka
{ततस्ते तापसाः सर्वे भार्गवस्यानुयायिनः}
{प्रेक्षकाः समपद्यन्त परिवार्य रणाजिरम्}


\twolineshloka
{ततो मामब्रवीद्देवी सर्वभूतहितैषिणी}
{माता स्वरूपिणी राजन्किमिदं ते चिकीर्षितम्}


\twolineshloka
{गत्वाऽहं जामदग्न्यं तु प्रयाचिष्ये कुरूद्वह}
{भूष्मेण सह मा योत्सीः शिष्येणेति पुनः पुनः}


\twolineshloka
{मा मैवं पुत्र निर्बन्धं कुरु विप्रेण पार्थिव}
{जामदग्न्येन समरे योद्धुमित्येव भर्त्सयत्}


\twolineshloka
{किं न वै क्षत्रियहणो हरतुल्यपराक्रमः}
{विदितः पुत्र रामस्ते यतस्तं योद्धुमिच्छसि}


\twolineshloka
{ततोऽहमब्रवं देवीमभिवाद्य कृताञ्जलिः}
{सर्वं तद्भरतश्रेष्ठ यथावृत्तं स्वयंवरे}


\twolineshloka
{यथा च रामो राजेन्द्र मया पूर्वं प्रचोदितः}
{काशिराजसुतायाश्च यथा कर्म पुरातनम्}


\twolineshloka
{ततः सा राममभ्येत्य जननी मे महानदी}
{मदर्थं तमृषिं वीक्ष्य क्षमयामास भार्गवम्}


\threelineshloka
{भीष्मेण सह मा योत्सीः शिष्येणेति वचोऽब्रवीत्}
{स च तामाह याचन्तीं भीष्ममेव निवर्तय}
{न च मे कुरुते काममित्यहं तमुपागमम्}


\twolineshloka
{ततो गङ्गा सुतस्नोहान्मां सा पुनरुपागमत्}
{नास्या अकरवं वाक्यं क्रोधपर्याकुलेक्षणः}


\twolineshloka
{अथादृश्यत धर्मात्मा भृगुश्रेष्ठो महातपाः}
{आह्वयामास च तदा युद्धाय द्विजसत्तमः}


\chapter{अध्यायः १७९}
\twolineshloka
{भीष्म उवाच}
{}


\twolineshloka
{तमहं स्मयन्निव रणे प्रत्यभाषं व्यवस्थितम्}
{भूमिष्ठं नोत्सहे योद्धुं भवन्तं रथमास्थितः}


\twolineshloka
{आरोह स्यन्दनं वीर कवचं च महाभुज}
{बधान समरे राम यदि योद्धुं मयेच्छसि}


\twolineshloka
{ततो मामब्रवीद्रामः स्मयमानो रणाजिरे}
{रथो मे मेदिनी भीष्म वाहा वेदाः सदश्ववत्}


\twolineshloka
{सूतश्च मातरिश्वा वै कवचं वेदमातरः}
{सुसंवीतो रणे ताभिर्योत्स्येऽहं कुरुनन्दन}


\twolineshloka
{एवं ब्रुवाणो गान्धारे रामो मां सत्यविक्रमः}
{शरव्रातेन महता सर्वतः प्रत्यवारयत्}


\twolineshloka
{ततोऽपश्यं जामदग्न्यं रथमध्ये व्यवस्थितम्}
{सर्वायुधवरे श्रीमत्यद्भुतोपमदर्शने}


\twolineshloka
{मनसा विहिते पुण्ये विस्तीर्णे नगरोपभे}
{दिव्याश्वयुजि सन्नद्वे काञ्चनेन विभूषिते}


\twolineshloka
{कवचेन महाबाहो सोमार्ककृतलक्ष्मणा}
{धनुर्धरो बद्धतूणो बद्धगोधाङ्गुलित्रवान्}


\twolineshloka
{सारथ्यं कृतवांस्तत्र युयुत्सोरकुतव्रणः}
{सखा वेदविदत्यन्तं दयितो भार्गवस्य ह}


\twolineshloka
{आह्वयानः स मां युद्धे मनो हर्षयतीव मे}
{पुनः पुनरभिक्रोशन्नभियाहीति भार्गवः}


\twolineshloka
{तमादित्यमिवोद्यन्तमनाधृष्यं महाबलम्}
{क्षत्रियान्तकरं राममेकमेकः समासदूम्}


\twolineshloka
{ततोऽहं बाणपातेषु त्रिषु वाहान्निगृह्य वै}
{अवतीर्य धनुर्न्यस्य पदातिर्ऋषिसत्तमम्}


\twolineshloka
{अभ्यागच्छं तदा राममर्चिष्यन्द्विजसत्तमम्}
{अभिवाद्य चैनं विधिवदब्रुवं वाक्यमुत्तमम्}


\threelineshloka
{योत्स्ये त्वया रणे राम सदृशेनाधिकेन वा}
{गुरुणा धर्मशीलेन जयमाशास्व मे विभो ॥राम उवाच}
{}


\twolineshloka
{एवमेतत्कुरुश्रेष्ठ कर्तव्यं भूतिमिच्छता}
{धर्मो ह्येष महाबाहो विशिष्टैः सह युध्यताम्}


\twolineshloka
{शपेयं त्वां नचेदेवमागच्छेथा विशांपते}
{युध्यस्व त्वं रणे यत्तो धैर्यमालम्ब्य कौरव}


\twolineshloka
{न तु ते जयमाशासे त्वां विजेतुमहं स्थितः}
{गच्छ युध्यस्व धर्मेण प्रीतोऽस्मि चरितेन ते}


\twolineshloka
{ततोऽहं तं नमस्कृत्य रथमारुह्य सत्वरः}
{प्राध्मापयं रणे शङ्खं पुनर्हेमपरिष्कृतम्}


\twolineshloka
{ततो युद्धं समभवन्मम तस्य च भारत}
{दिवसान्सुबहून्राजन्परस्परजिगीषया}


\twolineshloka
{स मे तस्मिन्रणे पूर्वं प्राहरत्कङ्खपत्रिभिः}
{वष्ट्या शतैश्च नवभिः शराणां नतषर्वणाम्}


\twolineshloka
{चत्वारस्तेन मे वाहाः सूतश्चैव विशांपते}
{प्रतिरुद्धास्तथैवाहं समरे दंशितः स्थितः}


\twolineshloka
{नमस्कृत्य च देवेभ्यो ब्राह्मणेभ्यो विशेषतः}
{तमहं स्मयन्निव रणे प्रत्यभाषं व्यवस्थितम्}


\twolineshloka
{आचार्यता मानिता मे निर्मर्यादे ह्यपि त्वयि}
{भूयश्च शृणु मे ब्रह्मन्संपदं धर्मसंग्रहे}


\twolineshloka
{ये ते वेदाः शरीरस्था .... यच्च ते महत्}
{तपश्च ते महत्तप्तं न तेभ्यः प्रहराम्यहम्}


\twolineshloka
{प्रहरे क्षत्रधर्मस्य यं त्वं राम समाश्रितः}
{ब्राह्मणः क्षत्रियत्वं हि याति शस्त्रसमुद्यमात्}


\twolineshloka
{पश्य मे धनुषो वीर्यं पश्य बाह्वोर्बलं मम}
{एष ते कार्मुक वीर छिनद्मि निशितेषुणा}


\twolineshloka
{तस्याहं निशितं भल्लं चिक्षेप भरतर्षभ}
{तेनास्य धनुषः कोटिं छित्वा भूमावपातयम्}


\threelineshloka
{तथैव च पृषत्कानां शतानि नतपर्वणाम्}
{चिक्षेप कङ्कपत्राणां जामदग्न्यरथं प्रति}
{}


\twolineshloka
{काये विषक्तास्तु तदा वायुना समुदीरिताः}
{चेलुः क्षरन्तो रुधिरं नागा इव च ते शराः}


\twolineshloka
{क्षतजोक्षितसर्वाङ्गः क्षरन्स रुधिरं रणे}
{बभौ रामस्तथा राजन्प्रफुल्ल इव किंशुकः}


\twolineshloka
{हेमन्तान्तेऽशोक इव रक्तस्तबकमण्डितः}
{बभौ रामस्तथा राजन्प्रफुल्ल इव किंशुकः}


\twolineshloka
{ततोऽन्यद्धनुरादाय रामः क्रोधसमन्वितः}
{हेमपुङ्खान्सुनिशिताञ्शरांस्तान्हि ववर्ष सः}


\twolineshloka
{ते समासाद्य मां रौद्रा बहुधा मर्मभेदिनः}
{अकम्पयन्महावेगाः सर्पानलविषोपमाः}


\twolineshloka
{तमहं समवष्टभ्य पुनरात्मानमाहवे}
{शतसंख्यैः शरैः क्रुद्धस्तदा राममवाकिरम्}


\twolineshloka
{स तैरग्न्यर्कसङ्काशैः शरैराशीविषोपमैः}
{शितैरभ्यर्दितो रामो मन्दचेता इवाभवत्}


\twolineshloka
{ततोऽहं कृपयाऽऽविष्टो विनिन्द्यात्मानमात्मना}
{धिग्धिगित्यब्रुवं युद्धं क्षत्रधर्मं च भारत}


\twolineshloka
{असकृच्चाब्रुवं राजञ्शोकवेगपरिप्लुतः}
{अहो बत कृतं पापं मयेदं क्षत्रधर्मं च भारत}


\twolineshloka
{गुरुर्द्विजातिर्धर्मात्मा यदेवं पीडितः शरैः}
{ततो न प्राहरं भूयो जामदग्न्याय भारत}


\twolineshloka
{अथावताप्य पृथिवीं पूषा दिवसमंक्षये}
{जगामास्तं सहस्रांशुस्ततो युद्धमुपारमत्}


\chapter{अध्यायः १८०}
\twolineshloka
{भीष्म उवाच}
{}


\twolineshloka
{आत्मनस्तु ततः सूतो हयानां च विशांपते}
{मम चापनयामास शल्यान्कुशलसंमतः}


\twolineshloka
{स्नातापवृत्तैस्तुरगैर्लब्धतोयैरविह्वलैः}
{प्रभाते चोदिते सूर्ये ततो युद्धमवर्तत}


\twolineshloka
{दृष्ट्वा भां तूर्णमायान्तं दंशितं स्यन्दने स्थितम्}
{अकरोद्रथमत्यर्थं रामः सज्जं प्रतापवान्}


\twolineshloka
{ततोऽहं राममायान्तं दृष्ट्वा समरकाङ्क्षिणम्}
{धनुःश्रेष्ठं समित्सृज्य सहसावतरं रथात्}


\twolineshloka
{अभिवाद्य तथैवाहं रथमारुह्य भारत}
{युयुत्सुर्जामदग्न्यस्य प्रमुखे वीतभीः स्थितः}


\twolineshloka
{ततोऽहं शरवर्षेण महता समवाकिरम्}
{स च मां शरवर्षेण वर्षन्तं समवाकिरम्}


\twolineshloka
{संक्रुद्धो जामदग्न्यस्तु पुनरेव सुतेजितान्}
{संप्रैषीन्मे शरान्घोरान्दीप्तास्यानुरगानिव}


\twolineshloka
{ततोऽहं निशितैर्भल्लैः शतशोऽथ सहस्रशः}
{अच्छिदं सहसा राजन्नन्तरिक्षे पुनः पुनः}


\threelineshloka
{ततस्त्वस्त्राणि दिव्यानि जामदग्न्यः प्रतापवान्}
{मयि प्रयोजयामास तान्यहं प्रत्यषेधयम्}
{}


\twolineshloka
{अस्त्रैरेव महाबाहो चिकीर्षन्नधिकां क्रियाम्}
{ततो विदि महान्नादः प्रादुरासीत्समन्ततः}


\twolineshloka
{ततोऽहमस्त्रं वायव्यं जामदग्न्ये प्रयुक्तवान्}
{प्रत्याजघ्ने च तद्रामो गुह्यकास्त्रेण भारत}


\twolineshloka
{ततोऽहमस्त्रमाग्नेयमनुमन्त्र्य प्रयुक्तवान्}
{वारुणेनैव तद्रामो वारयामास मे विभुः}


\twolineshloka
{एवमस्त्राणि दिव्यानि रामस्याहमवारयम्}
{रामश्च मम तेजस्वी दिव्यास्त्रविदरिन्दमः}


\twolineshloka
{ततो मां सव्यतो राजन्रामः कुर्वन्द्विजोत्तमः}
{उरस्यविध्यत्संक्रुद्धो जामदग्न्यः प्रतापवान्}


\twolineshloka
{ततोऽहं भारतश्रेष्ठ संन्यषीदं रथोत्तमे}
{ततो मां कश्मलाविष्टं सूतस्तूर्णमुदावहत्}


\twolineshloka
{ग्लायन्तं भरश्रेष्ठ रामबाणप्रपीडितम्}
{ततो मामपयातं वै भृशं विद्धमचेतसम्}


\twolineshloka
{रामस्यानुचरा हृष्टाः सर्वे दृष्ट्वा विचुक्रुशुः}
{अकृतव्रणप्रभृतयः काशिकन्या च भारत}


\twolineshloka
{ततस्तु लब्धसंज्ञोऽहं ज्ञात्वा सूतमथाब्रुवम्}
{याहि सूत यतो रामः सज्जोऽहं गतवेदनः}


\twolineshloka
{ततो मामवहत्सूतो हयैः परमशोभितैः}
{नृत्यद्भिरिव कौरव्य मारुतप्रतिमैर्गतौ}


\twolineshloka
{ततोऽहं राममासाद्य बाणवर्षैश्च कौरव}
{अवाकिरं सुसंरब्धः संरब्धं च जिगीषया}


\twolineshloka
{तानापतत एवासौ रामो बाणानजिह्मगान्}
{बाणैरेवाच्छिनत्तूर्णमेकैकं त्रिभिराहवे}


\twolineshloka
{ततस्ते सूदिताः सर्वे मम बाणाः सुसंशिताः}
{रामबाणैर्द्विधा च्छिन्नाः शतशोऽथ सहस्रशः}


\twolineshloka
{ततः पुनः शरं दीप्तं सुप्रभं कालसंमितम्}
{असृजं जामदग्न्याय रामायाहं जिघांसया}


\twolineshloka
{तेन त्वभिहतो गाढं बाणवेगवशं गतः}
{मुमोह समरे रामो भूमौ च निपपात ह}


\twolineshloka
{ततो हाहाकृतं कसर्वं रामे भूतलमाश्रिते}
{जगद्भारत संविग्नं यथार्कपतने भवेत्}


\twolineshloka
{तत एनं समुद्विग्नाः सर्व एवाभिदुद्रुवुः}
{तपोधनास्ते सहसा काश्या च कुरुनन्दन}


\twolineshloka
{तत एनं परिष्वज्य शनैराश्वासयंस्तदा}
{पाणिभिर्जलशीतैश्च जयाशीर्भिश्च कौरव}


\twolineshloka
{ततः स विह्वलं वाक्यं राम उत्थाय चाब्रवीत्}
{तिष्ठ भीष्म हतोऽसीति बाणं संधाय कार्मुके}


\twolineshloka
{स मुक्तो न्यपतत्तूर्णं सव्ये पार्श्वे महाहवे}
{येनाहं भृशमुद्विग्रो व्याघूर्णित इव द्रुमः}


\twolineshloka
{हत्वा हयांस्ततो रामः शीघ्रास्त्रेण महाहवे}
{अवाकिरन्मां विस्रब्धो बाणैस्तैर्लोमवाहिभिः}


\twolineshloka
{ततोऽहमपि शीघ्रास्त्रं समरप्रतिवारणम्}
{अवासृजं महाबाहो तेन्तराधिष्ठिताः शराः}


\twolineshloka
{रामस्य मम चैवाशु व्योमावृत्य समन्ततः}
{न स्म सूर्यः प्रतपति शरजालसमावृतः}


\twolineshloka
{मातरिश्वा ततस्तस्मिन्मेघरुद्ध इवाभवत्}
{ततो वायोः प्रकम्पाच्च सूर्यस्य च गभस्तिभिः}


\twolineshloka
{अभिघातप्रभावाच्च पावकः समजायत}
{ते शराः स्वसमुत्थेन प्रदीप्ताश्चित्रभानुना}


\twolineshloka
{भूमौ सर्वे तदा राजन्भस्मभूताः प्रपेदिरे}
{तदा शतसहस्राणि प्रयुतान्यर्बुदानि च}


\twolineshloka
{अयुतान्यथ खर्वाणि निखर्वाणि च कौरव}
{रामः शराणां संक्रुद्धो मयि तूर्णं न्यपातयत्}


\twolineshloka
{ततोऽहं तानपि रणे शरैराशीविषोपमैः}
{संछिद्य भूमौ नृपते पातयेयं नगानिव}


\twolineshloka
{एवं तदभवद्युद्वं तदा भरतसत्तम}
{सन्ध्याकाले व्यतीते तु व्यपायात्स च मे गुरुः}


\chapter{अध्यायः १८१}
\twolineshloka
{भीष्म उवाच}
{}


\twolineshloka
{समागतस्य रामेण पुनरेवातिदारुणम्}
{अन्येद्युस्तुमुलं युद्धं तदा भरतसत्तम}


\twolineshloka
{ततो दिव्यास्त्रविच्छूरो दिव्यान्यस्त्राण्यनेकशःक}
{अयोजयत्स धर्मात्मा दिवसे दिवसे विभुः}


\twolineshloka
{तान्यहं तत्प्रतीघातैरस्त्रैरस्त्राणि भारत}
{व्यधमं तुमुलेयुद्धेप्राणांस्त्यक्त्वासुदुस्त्यजान्}


\twolineshloka
{अस्त्रैरस्त्रेषु बहुधा हतेष्वेव च भारत}
{अक्रुध्यत महातेजास्त्यक्तप्राणः स संयुगे}


\twolineshloka
{ततः शक्तिं प्राहिणोद्धोररूपा-मस्त्रे रुद्धे जामदग्न्यो महात्मा}
{कालोत्सृष्टां प्रज्वलितामिवोल्कांसंदीप्ताग्रां तेजसा व्याप्य लोकम्}


\twolineshloka
{ततोऽहं तामिषुभिर्दीप्यमानांसमायान्तीमन्तकालार्कदीप्ताम्}
{छित्त्वा त्रिधा पातयामास भूमौततो ववौ पवनः पुण्यगन्धिः}


\twolineshloka
{तस्यां छिन्नायां क्रोधदीप्तोऽथ रामःशक्तीर्घोराः प्राहिणोद््द्वादशान्याः}
{तासां रूपं भारत नोत शक्यंतेजस्वित्वाल्लाघवाच्चैव वक्तुम्}


\twolineshloka
{किन्त्वेवाहं विह्वलः संप्रदृश्यदिग्भ्यः सर्वास्ता महोत्का इवाग्नेः}
{नानारूपास्तेजसोग्रेण दीप्तायथाऽऽदित्या द्वादश लोकसंक्षये}


\twolineshloka
{ततो जालं बाणमयं विवृत्तंसंदृश्य भित्त्वा शरजालेन राजन्}
{द्वादशेषून्प्राहिणवं रणेऽहंततः शक्तीरप्यधमं घोररूपाः}


\twolineshloka
{ततो राजञ्जामदग्न्यो महात्माशक्तीर्घोरा व्याक्षिपद्धेमदण्डाः}
{विचित्रिताः काञ्चनपट्टनंद्धायथा महोल्का ज्वलितास्तथा ताः}


\twolineshloka
{ताश्चाप्युग्राश्चर्मणा वारयित्वाखङ्गेनाजौ पातयित्वा नरेन्द्र}
{बाणैर्दिव्यैर्जामदग्न्यस्य सङ्ख्येदिव्यानश्वानभ्यवर्षं ससूतान्}


\twolineshloka
{निर्मिक्तानां पन्नगानां सरूपादृष्ट्वा शक्तीर्हेमचित्रा निकृत्ताः}
{प्रादुश्चक्रे दिव्यमस्त्रं महात्माक्रोधाविष्टो हैहयेशममाथी}


\twolineshloka
{ततः श्रेण्यः शलभानामिवोग्राःसमापेतुर्विशिखानां प्रदीप्ताः}
{समाचिनोच्चापि भृशं शरीरंहयान्सूतं सरथं चैव मह्यम्}


\twolineshloka
{रथः शरैर्मे निचितः सर्वतोऽभू-त्तथा वाहाः सारथिश्चैव राजन्}
{युगं रथेषां च तथैव चक्रेतथैवाक्षः शरकृत्तोऽथ भग्नः}


\twolineshloka
{ततस्तस्मिन्बाणवर्षे व्यतीतेशरौघेण प्रत्यवर्षं गुरुं तम्}
{स विक्षतो मार्गणैर्ब्रह्मराशि-र्देहादसक्तं मुमुचे भूरि रक्तम्}


\twolineshloka
{यथा रामो बाणजालाभितप्त-स्तथैवाहं सुभृशं गाढविद्धः}
{ततो युद्धं व्यरमच्चापराह्णेभानावस्तं प्रतियाते महीध्रम्}


\chapter{अध्यायः १८२}
\twolineshloka
{भीष्म उवाच}
{}


\twolineshloka
{ततः प्रभाते राजेन्द्र सूर्ये विमलतां गते}
{भार्गवस्य मया सार्धं पुनर्युद्धमवर्तत}


\twolineshloka
{ततोऽभ्रान्ते रथे तिष्ठन्रामः प्रहरतां वरः}
{ववर्ष शरजालानि मयि मेध इवाचले}


\twolineshloka
{ततः सूतो मम सुहृच्छरवर्षेण ताडितः}
{अपयातो रथोपस्थान्मनो मम विषादयन्}


\twolineshloka
{ततः सूतो ममात्यर्थं कश्मलं प्राविशन्महत्}
{पृथिव्यां च शराघातान्निपपात मुमोह च}


\twolineshloka
{ततः सूतो जहात्प्राणान्रामबाणप्रपीडितः}
{मुहूर्तादिव राजेन्द्र मां च भीराविशत्तदा}


\twolineshloka
{ततः सूते हते तस्मिन्क्षिपतस्तस्य मे शरान्}
{प्रमत्तमनसो रामः प्राहिणोन्मृत्युसंमितम्}


\twolineshloka
{ततः सूतव्यसनिनं विप्लुतं मां स भार्गवः}
{शरेणाभ्यहनद्गाढं विकृष्य बलवद्धनुः}


\twolineshloka
{स मे भुजान्तरे राजन्निपत्य रुधिराशनः}
{मयैव सह राजेन्द्र जगाम वसुधातलम्}


\twolineshloka
{मत्वा तु निहतं रामस्ततो मां भरतर्षभ}
{मेघवद्विननादोच्चैर्जहृषे च पुनः पुनः}


\twolineshloka
{तथा तु पतिते राजन्मयि रामो मुदा युतः}
{उदक्रोशन्महानादं सह तैरनुयायिभिः}


\threelineshloka
{मम तत्राभवन्ये तु कुरवः पार्श्वतः स्थिताः}
{आगता अपि युद्धं तञ्जनास्तत्र दिदृक्षवः}
{आर्तिं परमिकां जग्मुस्ते तदा पतिते मयि}


\twolineshloka
{ततोऽपश्यं पतितो राजसिंहद्विजानष्टौ सूर्यहुताशनाभान्}
{ते मां समन्तात्परिर्वाय तस्थुःस्वबाहुभिः परिधार्याजिमध्ये}


\twolineshloka
{रक्ष्यमाणश्च तैर्विप्रैर्नाहं भूमिमुपास्पृशम्}
{अन्तरिक्षे धृतो ह्यस्मि तैर्विप्रैर्बान्धवैरिव}


\twolineshloka
{श्वसन्निवान्तरिक्षे च जलबिन्दुभिरुक्षितः}
{ततस्ते ब्राह्मणा राजन्नब्रुवन्परिगृह्य माम्}


\threelineshloka
{मा भैरिति समं सर्वे स्वस्ति तेऽस्त्विति चासकृत्}
{ततस्तेषामहं वाग्भिस्तर्पितः सहसोत्थितः}
{मातरं सरितां श्रेष्ठामपश्यं रथमास्थिताम्}


\twolineshloka
{हयाश्च मे संगृहीतास्तयाऽऽस-न्महानद्या संयति कौरवेन्द्र}
{पादौ जनन्याः प्रतिगृह्य चाहंतथा पितॄणां रथमभ्यरोहम्}


\twolineshloka
{ररक्ष सा मां सरथं हयांश्चोपस्काराणि च}
{तामहं प्राञ्जलिर्भूत्वा पुनरेव व्यसर्जयम्}


\twolineshloka
{ततोऽहं स्वयमुद्यम्य हयांस्तान्वातरंहसः}
{अयुध्यं जामदग्न्येन निवृत्तेऽहनि भारत}


\twolineshloka
{ततोऽहं भरतश्रेष्ठ वेगवन्तं महाबलम्}
{अमुञ्चं समरे बाणं रामाय हृदयच्छिदम्}


\twolineshloka
{ततो जगाम वसुधां मम बाणप्रपीडितः}
{जानुभ्यां धनुरुत्सृज्य रामो मोहवशं गतः}


\twolineshloka
{ततस्तस्मिन्निपतिते रामे भूरिसहस्रदे}
{आवव्रुर्जलदा व्योम क्षरन्तो रुधिरं बहु}


\twolineshloka
{उल्काश्च शतशः पेतुः सनिर्घाताः सकम्पनाः}
{5-182-22bअर्कं च सहसादीप्तं स्वर्भानुरभिसंवृणोत्}


\twolineshloka
{ववुश्च वाताः परुषाश्चलिता च वसुंधरा}
{गृध्रा बलाश्च कङ्काश्च परिपेतुर्गुदा युताः}


\twolineshloka
{दीप्तायां दिशि गोमायुर्दारुणं मुहुरुन्नदत्}
{अनाहता दुन्दुभयो विनेदुर्भृशनिःस्वनाः}


\twolineshloka
{एतदौत्पातिकं सर्वं घोरमासीद्भयंकरम्}
{विसंज्ञकल्पे धरणीं गते रामे महात्मनि}


\twolineshloka
{ततो वै सहसोत्थाय रामो मामभ्यवर्तत}
{पुनर्युद्धाय कौरव्य विह्वलः क्रोधमूर्छितः}


\twolineshloka
{आददानो महाबाहुः कार्मुकं तालसन्निभम्}
{ततो मय्याददानं तं राममेव न्यवारयन्}


\twolineshloka
{महर्षयः कृपायुक्ताः क्रोधाविष्टोऽपि भार्गवः}
{समाहरदमेयात्मा शरं कालानलोपमम्}


% Check verse!
सतो रविर्मन्दमरीचिमण्डलोजगामास्तं पांसुपुञ्जावगूढःनिशा व्यगाहत्सुखशीतमारुताततो युद्धं प्रत्यवहारयाव-
\twolineshloka
{एवं राजन्नवहारो बभूवततः पुनर्विमलेऽभूत्सुघोरम्}
{कल्यंकल्यं विंशतिं वै दिनानितथैव चान्यानि दिनानि त्रीणि}


\chapter{अध्यायः १८३}
\twolineshloka
{भीष्म उवाच}
{}


\twolineshloka
{ततोऽहं निशि राजेन्द्र प्रणम्य शिरसा तदा}
{ब्राह्मणानां पितॄणां च देवतानां च सर्वशः}


\twolineshloka
{नक्तंचराणां भूतानां राजन्यानां विशांपते}
{शयनं प्राप्य रहिते मनसा समचिन्तयम्}


\twolineshloka
{जामदग्न्येन मे युद्धमिदं परमदारुणम्}
{अहानि च बहून्यद्य वर्तते सुमहात्ययम्}


\twolineshloka
{न च रामं महावीर्यं शक्नोमि रणमूर्धनि}
{विजेतुं समरे विप्रं जामदग्न्यं महाबलम्}


\twolineshloka
{यदि शक्यो मया जेतुं जामदग्न्यः प्रतापवान्}
{दैवतानि प्रसन्नानि दर्शयन्तु निशां मम}


\twolineshloka
{ततो निशि च राजेन्द्र प्रसुप्तः शरविक्षतः}
{दक्षिणेनेह पार्श्वेन प्रभातसमये तदा}


\twolineshloka
{ततोऽहं विप्रमुख्यैस्तैर्यैरस्मि पतितो रथात्}
{उत्थापितो धृतश्चैव मा भैरिति च सान्त्वितः}


\twolineshloka
{त एव मां महाराज स्वप्ने दर्शनमेत्य वै}
{परिवार्याब्रुवन्वाक्यं तन्निबोध कुरूद्वह}


\twolineshloka
{उत्तिष्ठ मा भैर्गाङ्गेय न भयं तेऽस्ति किंचन}
{रक्षामहे त्वां कौरव्य स्वशरीरं हि नो भवान्}


\twolineshloka
{न त्वां रामो रणे जेता जामदग्न्यः कथंचन}
{त्वमेव समरे रामं विजेता भरतर्षभ}


\twolineshloka
{इदमस्त्रं सुदयितं प्रत्यभिज्ञास्यते भवान्}
{विदितं हि तवाप्येतत्पूर्वस्मिन्देहधारणे}


\twolineshloka
{प्राजापत्यं विश्वकृतं प्रस्वापं नाम भारत}
{न हीदं वेद रामोऽपि पृथिव्यां वा पुमान्क्वचित्}


\twolineshloka
{तत्स्मरस्व महाबाहो भृशं संयोजयस्व च}
{उपस्थास्यति राजेन्द्र स्वयमेव तवानघ}


\twolineshloka
{येन सर्वान्महावीर्यान्प्रशासिष्यसि कौरव}
{न च रामः क्षयं गन्ता तेनास्त्रेण नराधिप}


\twolineshloka
{एनसा न तु संयोगं प्राप्स्यसे जातु मानद}
{स्वप्स्यते जामदग्न्योऽसौ त्वद्बाणबलपीडितः}


\twolineshloka
{ततो जित्वा त्वमेवैनं पुनरुत्थापयिष्यसि}
{अस्त्रेण दयितेनाजौ भीष्म संबोधनेन वै}


\twolineshloka
{एवं कुरुष्व कौरव्य प्रभाते रथमास्थितः}
{प्रसुप्तं वा मृतं वेति तुल्यं मन्यामहे वयम्}


\twolineshloka
{न च रामेण मर्तव्यं कदाचिदपि पार्थिव}
{ततः समुत्पन्नमिदं प्रस्वापं युज्यतामिति}


\twolineshloka
{इत्युक्त्वान्तर्हिता राजन्सर्व एव द्विजोत्तमाः}
{अष्टौ सदृशरूपास्ते सर्वे भासुरमूर्तयः}


\chapter{अध्यायः १८४}
\twolineshloka
{भीष्म उवाच}
{}


\twolineshloka
{ततो रात्रौ व्यतीतायां प्रतिबुद्धोऽस्मि भारत}
{ततः संचिन्त्य वै स्वप्नमवापं हर्षमुत्तमम्}


\twolineshloka
{ततः समभवद्युद्धं मम तस्य च भारत}
{तुमुलं सर्वभूतानां रोमहर्षणमद्भुतम्}


\twolineshloka
{ततो बाणमयं वर्षं ववर्ष मयि भार्गवः}
{न्यवारयमहं तच्च शरजालेन भारत}


\twolineshloka
{ततः परमसंक्रुद्धः पुनरेव महातपाः}
{ह्यस्तनेन च कोपेन शक्तिं वै प्राहिणोन्मयि}


\twolineshloka
{इन्द्राशनिसमस्पर्शां यमदण्डसमप्रभाम्}
{ज्वलन्तीमग्निवत्सङ्ख्ये लेलिहानां समन्ततः}


\twolineshloka
{ततो भरतशार्दूल धिष्ण्यमाकाशगं यथा}
{सा ममाभ्यवधीत्तूर्णं जत्रुदेशे कुरूद्वह}


\twolineshloka
{अथास्रमस्रवद्धोरं गिरेर्गैरिकधातुवात्}
{रामेण सुमहाबाहो क्षतस्य क्षतजेक्षण}


\twolineshloka
{ततोऽहं जामदग्न्याय भृशं क्रोधसमन्वितः}
{चिक्षेप मृत्युसंकाशं बाणं सर्पविषोपमम्}


\twolineshloka
{स तेनाभिहतो वीरो ललाटे द्विजसत्तमः}
{अशोभत महाराज शश्रृङ्ग इव पर्वतः}


\twolineshloka
{स संरब्धः समावृत्य शरं कालान्तकोपमम्}
{संदधे बलवत्कृष्य घोरं शत्रनिबर्हणम्}


\twolineshloka
{स वक्षसि पापतोग्रः शरो व्याल इव श्वसन्}
{महीं राजंस्ततश्चाहमगमं रुधिराविलः}


\twolineshloka
{संप्राप्य तु पुनः संज्ञां जामदघ्न्याय धीमते}
{प्राहिण्वं विमलां शक्तिं ज्वलन्तीमशनीमिव}


\twolineshloka
{सा तस्य द्विजमुख्यस्य निपपात भुजान्तरे}
{विह्वलश्चाभवद्राजन्वेपथुश्चैनमाविशत्}


\twolineshloka
{तत एनं परिष्वज्य सखा विप्रो महातपाः}
{अकृतव्रणः शुभैर्वाक्यैराश्वासयदनेकधा}


\twolineshloka
{समाश्वस्तस्ततो रामः क्रोधामर्षसमन्वितः}
{प्रादुश्चक्रे तदा ब्राह्मं परमास्त्रं महाव्रतः}


\twolineshloka
{ततस्तत्प्रतिघातार्थं ब्राह्ममेवास्त्रमुत्तमम्}
{मया प्रयुक्तं जज्वाल युगान्तमिव दर्शयत्}


\twolineshloka
{तयोर्ब्रह्मास्त्रयोरासीदन्तरा वै समागमः}
{असंप्राप्यैव रामं च मां च भारतसत्तम}


\twolineshloka
{ततो व्योम्नि प्रादुरभूत्तेज एव हि केवलम्}
{भूतानि चैव सर्वाणि जग्मुरार्तिं विशांपते}


\twolineshloka
{ऋषयश्च सगन्धर्वा देवताश्चैव भारत}
{संतापं परमं जग्मुरस्त्रतेजोभिपीडिताः}


\twolineshloka
{ततश्चचाल पृथिवी सपर्वतवनद्रुमा}
{संतप्तानि च भूतानि विषादंक जग्मुरुत्तमम्}


\twolineshloka
{प्रजज्वाल नभो राजन्धूमायन्ते दिशो दश}
{न स्थातुमन्तरिक्षे च शेकुराकाशगास्तदा}


\twolineshloka
{ततो हाहाकृते लोके सदेवासुरराक्षसे}
{इदमन्तरमित्येवं मोक्तुकामोऽस्मि भारत}


\twolineshloka
{प्रस्वापमस्त्रं त्वरितो वचनाद्ब्रह्मवादिनाम्}
{चिन्तितं च तदस्त्रं मे मनसि प्रत्यभात्तदा}


\chapter{अध्यायः १८५}
\twolineshloka
{भीष्म उवाच}
{}


\twolineshloka
{ततो हलहलाशब्दो दिवि राजन्महानभूत्}
{प्रस्वापं भीष्म मा स्राक्षीरिति कौरवनन्दन}


\twolineshloka
{अयुञ्जमेव चैवाहं तदस्त्रं भृगुनन्दने}
{प्रस्वापं मां प्रयुज्जानं नारदो वाकयमब्रवीत्}


\twolineshloka
{एते वियति करव्य दिवि देवगणाः स्थिताः}
{ते त्वां निवारयन्त्यद्य प्रस्वापं मा प्रयोजय}


\twolineshloka
{रामस्तपस्वी ब्रह्मण्यो ब्राह्मणश्च गुरुश्च ते}
{तस्यवमानं कौरव्य मास्म कार्षीः कथञ्चन}


\twolineshloka
{ततोऽपश्यं दिविष्ठान्वै तानष्टौ ब्रह्मवादिनः}
{ते मां स्मयन्तो राजेन्द्र शनकैरिदमब्रुवन्}


\twolineshloka
{यथाऽऽह भरतश्रेष्ठ नारदस्तत्तथा कुरु}
{एतद्धि परमं श्रेयो लोकानां भरतर्षभ}


\twolineshloka
{ततश्च प्रतिसंहृत्य तदस्त्रं स्वापनं महत्}
{ब्रह्मास्त्रं दीपयाञ्चक्रे तस्मिन्युधि यथाविधि}


\twolineshloka
{ततो रामो हृषितो राजसिंहदृष्ट्वा तदस्त्रं विनिवर्तितं वै}
{जितोऽस्मि भीष्मेण सुमन्दबुद्धि-रित्येव वाक्यं सहसा व्यमुञ्चत्}


\threelineshloka
{ततोऽपश्यत्पितरं जामदग्न्यःपितुस्तथा पितरं चास्य मान्यम्}
{ते तत्र चैनं परिवार्य तस्थु-रूचुश्चैनं सान्त्वपूर्वं तदनीम् ॥पितर ऊचुः}
{}


\twolineshloka
{मा स्मैवं साहसं तात पुनः कार्षीः कथञ्चन}
{भीष्मेण संयुगं गन्तुं क्षत्रियेण विशेषतः}


\twolineshloka
{क्षत्रियस्य तु धर्मोऽयं यद्युद्धं भृगुनन्दन}
{स्वाध्यायो व्रतचर्याऽथ ब्राह्मणानां परं धनम्}


\twolineshloka
{इदं निमित्ते कस्मिंश्चिदस्माभिः प्रागुदाहृतम्}
{शस्त्रधारणमत्युग्रं तच्चाकार्यं कृतं त्वया}


\twolineshloka
{वत्स पर्याप्तमेतावद्भीष्मेण सह संयुगे}
{विमर्दस्ते महाबाहो व्यपयाहि रणादितः}


\twolineshloka
{पर्याप्तमेतद्भद्रं ते तव कार्मुकधारणम्}
{विसर्जयैतद्दुर्धर्ष तपस्तप्यस्व भार्गव}


\twolineshloka
{एष भीष्मः शान्तनवो देवैः सर्वैर्निवारितः}
{निवर्तस्व रणादस्मादिति चैव प्रसादितः}


\twolineshloka
{रामेण सह मायोत्सीर्गुरुणेति पुनः पुनः}
{न हि रामो रणे जेतुं त्वया न्याय्यः कुरूद्वह}


\twolineshloka
{मानं कुरुष्व गाङ्गेय ब्राह्मणस्य रणाजिरे}
{वयं तु गुरवस्तुभ्यं तस्मात्त्वां वारयामहे}


\twolineshloka
{भीष्मो वसूनामन्यतमो दिष्ट्या जीवसि पुत्रक}
{गाङ्गेयः शन्तनोः पुत्रो वसुरेष महायशाः}


\twolineshloka
{कथं शक्यस्त्वया जेतुं निवर्तस्वेह भार्गव}
{अर्जुनः पाण्डवश्रेष्ठः पुरन्दरसुतो बली}


\fourlineindentedshloka
{नरः प्रजापतिर्वीरः पूर्वदेवः सनातनः}
{सव्यसाचीति विख्यातस्त्रिषु लोकेषु वीर्यवान्}
{भीष्ममृत्युर्यथाकालं विहितो वै स्वयंभुवा ॥भीष्म उवाच}
{}


\threelineshloka
{एवमुक्तः स पितृभिः पितॄन्रामोऽब्रवीदिदम्}
{नाहं युधि निवर्तेयमिति मे व्रतमाहितम्}
{न निवर्तितपूर्वश्च कदाचिद्रणमूर्धनि}


\twolineshloka
{निवर्त्यतामापगेयः कामं युद्धात्पितामहाः}
{न त्वहं विनिवर्तिष्ये युद्धादस्मात्कथंचन}


\twolineshloka
{ततस्ते मुनयो राजन्नृचीकप्रमुखास्तदा}
{नारदेनैव सहिताः समागम्येदमब्रुवन्}


\twolineshloka
{निवर्तस्व रणात्तात मानयस्व द्विजोत्तमम्}
{इत्यवोचमहं तांश्च क्षत्रधर्मव्यपेक्षया}


\twolineshloka
{मम व्रतमिदं लोके नाहं युद्धात्कदाचन}
{विमुखो विनिवर्तेयं पृष्ठतोऽभ्याहतः शरैः}


\twolineshloka
{नाहं लोभान्न कार्पण्यान्न भयान्नार्थकारणात्}
{त्यजेयं शाश्वतं धर्ममिति मे निश्चिता मतिः}


\twolineshloka
{ततस्ते मुनयः सर्वे नारदप्रमुखा नृप}
{भागीरथी च मे माता रणमध्यं प्रपेदिरे}


\twolineshloka
{तथैवात्तशरो धन्वी तथैव दृढनिश्चयः}
{स्थितोऽहमाहवे योद्धुं ततस्ते राममब्रुवन्}


\twolineshloka
{समेत्य सहिता भूयः समरे भृगुनन्दनम्}
{नावनीतं हि हृदयं विप्राणां शाम्य भार्गव}


\twolineshloka
{राम राम निवर्तस्व युद्धादस्माद्द्विजोत्तम}
{अवध्यो वै त्वया भीष्मस्त्वं च भीष्मस्य भार्गव}


\twolineshloka
{एवं ब्रुवन्तस्ते सर्वे प्रतिरुद्ध्य रणाजिरम्}
{न्यासयाञ्चक्रिरे शस्त्रं पितरो भृगुनन्दनम्}


\twolineshloka
{ततोऽहं पुनरेवाथ तानष्टौ ब्रह्मवादिनः}
{अद्राक्षं दीप्यमानान्वै ग्रहानष्टाविवोदितान्}


\twolineshloka
{ते मां सप्रणयं वाक्यमब्रुवन्समरे स्थितम्}
{प्रैहि रामं महाबाहो गुरुं लोकहितं कुरु}


\twolineshloka
{दृष्ट्वा निवर्तितं रामं सुहृद्वाक्येन तेन वै}
{लोकानां च हितं कुर्वन्नहमप्याददे वचः}


\twolineshloka
{ततोऽहं राममासाद्य ववन्दे भृशविक्षतः}
{रामश्चाभ्युत्स्मयन्प्रेम्णा मामुवाच महातपाः}


\twolineshloka
{त्वत्समो नास्ति लोकेऽस्मिन्क्षत्रियः पृथिवीचरः}
{गम्यतां भीष्म युद्धेऽस्मिंस्तोषितोऽहं भृशं त्वया}


\twolineshloka
{मम चैव समक्षं तां कन्यामाहूय भार्गवः}
{उक्तवान्दीनया वाचा मध्ये तेषां महात्मनाम्}


\chapter{अध्यायः १८६}
\twolineshloka
{राम उवाच}
{}


\twolineshloka
{प्रत्यक्षमेतल्लोकानां सर्वेषामेव भामिनि}
{यथाशक्त्या मया युद्धं कृतं वै पौरुषं परम्}


\twolineshloka
{न चैवमपि शक्नोमि भीष्मं शस्त्रभृतां वरम्}
{विशेषयितुमत्यर्थमुत्तमास्त्राणि दर्शयन्}


\twolineshloka
{एषा मे परमा शक्तिरेतन्मे परमं बलम्}
{यथेष्टं गम्यतां भद्रे किमन्यद्वा करोमि ते}


\twolineshloka
{भीष्ममेव प्रपद्यस्व न तेऽन्या विद्यते गतिः}
{निर्जितो ह्यस्मि भीष्मेण महास्त्राणि प्रमुञ्चता}


\twolineshloka
{एवमुक्त्वा ततो रामो विनिःश्वस्य महामनाः}
{तूष्मीमासीत्ततः कनक्या प्रोवाच भृगुनन्दनम्}


\twolineshloka
{भगवन्नेवमेवैतद्यथाऽऽह भगवांस्तथा}
{अजेयो युधि भीष्मोऽयमपि देवैरुदारधीः}


\twolineshloka
{यथाशक्ति यथोत्साहं मम कार्यं कृतं त्वया}
{अनिवार्यं रणे वीर्यमस्त्राणि विविधानि च}


\twolineshloka
{ने चैव शक्यते युद्धे विशेषयितुमन्ततः}
{न चाहमेनं यास्यामि पुनर्भीष्मं कथंचन}


\twolineshloka
{गमिष्यामि तु तत्राहं यत्र भीष्मं तपोधन}
{समरे पातयिष्यामि स्वयमेव भृगूद्वह}


\twolineshloka
{एवमुक्त्वा ययौ कन्या रोषव्याकुललोचना}
{तापस्ये धृतसंकल्पा सा मे चिन्तयती वधम्}


\twolineshloka
{ततो महेन्द्रं सहितैरमुनिभिर्भृगुसत्तमः}
{यथाऽऽगतं तथा सोऽगन्मामुपामन्त्र्य भारत}


\twolineshloka
{ततो रथं समारुह्य स्तूयमानो द्विजातिभिः}
{प्रविश्य नगरं मात्रे सत्यवत्यै न्यवेदयम्}


\twolineshloka
{यथावृत्तं महाराज सा च मां प्रत्यनन्दत}
{पुरुषांश्चादिशं प्राज्ञान्कन्यावृत्तान्तकर्मणि}


\twolineshloka
{दिवसे दिवसे ह्यस्या गतिजल्पितचेष्टितम्}
{प्रत्याहरंश्च मे युक्ताः स्थिताः प्रियहिते सदा}


\twolineshloka
{यदैव हि वनं प्रायात्सा कन्या तपसे धृता}
{तदैव व्यथितो दीनो गतचेता इवाभवम्}


\twolineshloka
{न हि मां क्षत्रियः कश्चिद्वीर्येण व्यजयद्युधि}
{ऋते ब्रह्मविदस्तात तपसा संशितव्रतात्}


\twolineshloka
{अपि चैतन्मया राजन्नारदेऽपि निवेदितम्}
{व्यासे चैव तथा कार्यं तौ चोभौ मामवोचताम्}


\twolineshloka
{न विषादस्त्वया कार्यो भीष्म काशिसुतां प्रति}
{दैवं पुरुषकारेण को निवर्तितुमुत्सहेत्}


\twolineshloka
{सा कन्या तु महाराज प्रविश्याश्रममण्डलम्}
{यमुनातीरमाश्रित्य तपस्तेपेऽतिमानुषम्}


\twolineshloka
{निराहारा कृशा रूक्षा जटिला मलपङ्किनी}
{षण्मासान्वायुभक्षा च स्थाणुभूता तपोधना}


\twolineshloka
{यमुनाजलमाश्रित्य संवत्सरमथाऽपरम्}
{उदवासं निराहारा पारयामास भामिनी}


\twolineshloka
{शीर्णपर्णेन चैकेन पारयामास सा परम्}
{संवत्सरं तीव्रकोपा पादाङ्गुष्ठाग्रधिष्ठिता}


\twolineshloka
{एवं द्वादश वर्षाणि तापयामास रोदसी}
{निवर्त्यमानापि च सा ज्ञातिभिर्नैव शक्यते}


\twolineshloka
{ततोऽगमद्वत्सभूमिं सिद्धचारणसेविताम्}
{आश्रमं पुण्यशीलानां तापसानां महात्मनाम्}


\twolineshloka
{तत्र पुण्येषु तीर्थेषु साऽऽप्लुताङ्गी दिवानिशम्}
{व्यचरत्काशिकन्या सा यथाकामविचारिणी}


\twolineshloka
{नन्दाश्रमे महाराज तथोलूकाश्रमे शुभे}
{च्यवनस्याश्रमे चैव ब्रह्मणः स्थान एव च}


\twolineshloka
{त्रयागे देवयजने देवारण्येषु चैव ह}
{भोगवत्यां महाराज कौशिकस्याश्रमे तथा}


\twolineshloka
{माण्डव्यस्याश्रमे राजन्दिलीपस्याश्रमे तथा}
{रामह्रदे च कौरव्य पैलगर्गस्य चाश्रमे}


\twolineshloka
{एतेषु तीर्थेषु तदा काशिकन्या विशांपते}
{आप्लावयत गात्राणि व्रतमास्थाय दुष्करम्}


\twolineshloka
{तामब्रवीच्च कौरव्य मम माता जले स्थिता}
{किमर्थं क्लिश्यसे भद्रे तथ्यमेव वदस्व मे}


% Check verse!
सैनामथाब्रवीद्राजन्कृताञ्जलिरनिन्दिताभीष्मेण समरे रामो निर्जितश्चारुलोचनि
\twolineshloka
{कोऽन्यस्तमुत्सहेञ्जेतुमुद्यतेषुं महीपतिः}
{साहं भीष्मविनाशाय तपस्तप्स्ये सुदारुणम्}


\twolineshloka
{विचरामि महीं देवि यथा हन्यामहं नृपम्}
{एतद्व्रतफलं देवि परमस्मिन्यथा हि मे}


\twolineshloka
{ततोऽब्रवीत्सागरगा जिह्मं चरसि भामिनि}
{नैष कामोऽनवद्याङ्गि शक्यः प्राप्तुं त्वयाऽबले}


\twolineshloka
{यदि भीष्मविनाशाय काश्ये चरसि वै व्रतम्}
{व्रतस्था च शरीरं त्वं यदि नाम विमोक्ष्यसि}


\twolineshloka
{नदी भविष्यसि शुभे कुटिला वार्षिकोदका}
{दुस्तीर्था न तु विज्ञेया वार्षिकी नाष्टमासिकी}


\twolineshloka
{भीमग्राहवती घोरा सर्वभूतभयंकरी}
{एवमुक्त्वा ततो राजन्काशिकन्यां न्यवर्तत}


\threelineshloka
{माता मम महाभागा स्मयमानेव भामिनी}
{कदाचिदष्टमे मासि कदाचिद्दशमे तथा}
{न प्राश्नीतोदकमपि पुनः सा वरवर्णिनी}


\twolineshloka
{सा वत्सभूमिं कौरव्य तीर्थलोभात्ततस्ततः}
{पतिता परिधावन्ती पुनः काशिपतेः सुता}


\twolineshloka
{सा नदी वत्सभूम्यां तु प्रथिताम्बेति भारत}
{वार्षिकी ग्राहबहुला दुस्तीर्था कुटिला तथा}


\twolineshloka
{सा कन्या तपसा तेन देहार्धेन व्यजायत}
{नदी च राजन्वत्सेषु कन्या चैवाभवत्तदा}


\chapter{अध्यायः १८७}
\twolineshloka
{भीष्म उवाच}
{}


\twolineshloka
{ततस्ते तापसाः सर्वे तपसे धृतनिश्चयाम्}
{दृष्ट्वा न्यवर्तयंस्तात किं कार्यमिति चाब्रुवन्}


\twolineshloka
{तानुवाच ततः कन्या तपोवृद्धानृषींस्तदा}
{निराकृतास्मि भीष्मेण भ्रंशिता पतिधर्मतः}


\twolineshloka
{वधार्थं तस्य दीक्षा मे न लोकार्थं तपोधनाः}
{निहत्य भीष्मं गच्छेयं शान्तिमित्येव निश्चयः}


\twolineshloka
{यत्कृते दुःखवसतिमिमां प्राप्तऽस्मि शाश्वतीम्}
{पतिलोकाद्विहीना च नैव स्त्री न पुमानिह}


\twolineshloka
{नाहत्वा युधि गाङ्गेयं निवर्तिष्ये तपोधनाः}
{एष मे हृदि संकल्पो यदिदं कथितं मया}


\twolineshloka
{स्त्रीभावे परिनिर्विण्णा पुंस्त्वार्थे कृतनिश्चया}
{भीष्मे प्रतिचिकीर्षामि नास्मि वार्येति वै पुनः}


\twolineshloka
{तां देवो दर्शयामास शूलपाणिरुमापतिः}
{मध्ये तेषां महार्षीणां स्वेन रूपेण तापसीम्}


\twolineshloka
{छन्द्यमाना वरेणाथ सा वव्रे मत्पराजयम्}
{हनिष्यसीति तां देवः प्रत्युवाच मनस्विनीम्}


\twolineshloka
{ततः सा पुनरेवाथ कन्या रुद्रमुवाच ह}
{उपपद्येत्कथं देव स्त्रिया युधि जयो मम}


\twolineshloka
{स्त्रीभावेन च मे गाढं मनः शान्तमुमापते}
{प्रतिश्रुतश्च भूतेश त्वया भीष्मपराजयः}


\twolineshloka
{यथा स सत्यो भवति तथा कुरु वृषध्वज}
{यथा हन्यां समागम्य भीष्मं शान्तनवं युधि}


\twolineshloka
{तामुवाच महादेवः कन्यां किल वृषध्वजः}
{न मे वागमृतं प्राह सत्यं भद्रे भविष्यसि}


% Check verse!
हनिष्यसि रणे भीष्मं पुरुषत्वं च लप्स्यति ॥स्मरिष्यसि च तत्सर्वं देहमन्यं गता सती
\twolineshloka
{द्रुपदस्य कुले जाता भविष्यसि महारथः}
{शीघ्रास्त्रश्चित्रयोधी च भविष्यसि सुसंमतः}


\twolineshloka
{यथोक्तमेव कल्याणि सर्वमेतद्भविष्यति}
{भविष्यसि पुमान्पश्चात्कस्माच्चित्कालपर्ययात्}


\twolineshloka
{एवमुक्त्वा महादेवः कपर्दी वृषभध्वजः}
{पश्यतामेव विप्राणां तत्रैवान्तरधीयत}


\twolineshloka
{ततः सा पश्यतां तेषां महर्षीणामनिन्दिता}
{समाहृत्य वनात्तस्मात्काष्ठानि वरवर्णिनी}


\twolineshloka
{चितां कृत्वा सुमहतीं प्रदाय च हुताशनम्}
{प्रदीप्तेऽग्नौ महाराज रोषदीप्तेन चेतसा}


\twolineshloka
{उक्त्वा भीष्मवधायेति प्रविवेश हुताशनम्}
{ज्येष्ठा काशिसुता राजन्यमुनामभितो नदीम्}


\chapter{अध्यायः १८८}
\twolineshloka
{दुर्योधन उवाच}
{}


\threelineshloka
{कथं शिखण्डी गाङ्गेय कन्यां भूत्वा पुरा तदा}
{पुरुषोऽभूद्युधिश्रेष्ठ तन्मे ब्रूहि पितामह ॥भीष्म उवाच}
{}


\twolineshloka
{भार्या तु तस्य राजेन्द्र द्रुपदस्य महीपतेः}
{महिषी दयिता ह्यासीदपुत्रा च विशांपते}


\twolineshloka
{एतस्मिन्नेव काले तु द्रुपदो वै महीपतिः}
{अपत्यार्थे महाराज तोषयामास शङ्करम्}


\twolineshloka
{अस्माद्वधार्थं निश्चित्य तपो घोरं समास्थितः}
{ऋते कन्यां महादेव पुत्रो मेस्यादिति ब्रुवन्}


\twolineshloka
{भगवन्पुत्रमिच्छामि भीष्मं प्रतिचिकीर्षया}
{इत्युक्तो देवदेवेन स्त्रीपुमांस्ते भविष्यति}


\twolineshloka
{निवर्तस्व महीपाल नैतञ्जात्वन्यथा भवेत्}
{स तु गत्वा च नगरं भार्यामिदमुवाच ह}


\twolineshloka
{कृतो यत्नो महादेवस्तपसाऽऽराधितो मया}
{कन्या भूत्वा पुमान्भावी इति चोक्तोस्मि शंभुना}


\twolineshloka
{पुनः पुनर्याच्यमानो दिष्टमित्यब्रवीच्छिवः}
{नतदन्यच्च भविता भवितव्यं हि तत्तथा}


\twolineshloka
{ततः सा नियता भूत्वा ऋतुकाले मनस्विनी}
{पत्नी द्रुपदराजस्य द्रुपदं प्रविवेश ह}


\twolineshloka
{लेभे गर्भं यथाकालं विधिदृष्टेन कर्मणा}
{पार्षतस्य महीपाल यथा मां नारदोऽब्रवीत्}


\twolineshloka
{ततो दधार सा देवी गर्भं राजीवलोचना}
{तां स राजा प्रियां भार्यां द्रुपदः कुरुनन्दन}


\twolineshloka
{पुत्रस्नेहान्महाबाहुः मुखं पर्यचरत्तदा}
{सर्वानभिप्रायकृतान्भार्याऽलभत कौरव}


\twolineshloka
{अपुत्रस्य सतो राज्ञो द्रुपदस्य महीपतेः}
{यथाकालं तु सा देवी महिषी द्रुपदस्य ह}


\twolineshloka
{कन्यां प्रवररूपां तु प्राजायत नराधिप}
{अपुत्रस्य तु राज्ञः सा द्रुपदस्य मनस्विनी}


\twolineshloka
{ख्यापयामास राजेन्द्र पुत्रो ह्येष ममेति वै}
{ततः स राजा द्रुपद प्रच्छन्नाया नराधिप}


\twolineshloka
{पुत्रवत्पुत्रकार्याणि सर्वाणि समकारयत्}
{रक्षणं चैव मन्त्रस्य महिषी द्रुपदस्य सा}


\twolineshloka
{चकार सर्वयत्नेन ब्रुवणा पुत्र इत्युत}
{न च तां वेदक नगरे कश्चिदन्यत्र पार्षतात्}


\twolineshloka
{श्रद्दधानो हि तद्वाक्यं देवस्याच्युततेजसः}
{छादयामास तां कन्यां पुमानिति च सोब्रवीत्}


\twolineshloka
{जातकर्माणि सर्वाणि कारयामास पार्थिवः}
{पुंवद्विधानयुक्तानि शिखण्डीति च तां विदुः}


\twolineshloka
{अहमेकस्तु चारेण वचनान्नारदस्य च}
{ज्ञातवान्देववाक्येन अम्बायास्तपसा तथा}


\chapter{अध्यायः १८९}
\twolineshloka
{भीष्म उवाच}
{}


\twolineshloka
{चकार यत्नं द्रुपदः सुतायाः सर्वकर्मसु}
{ततो लेख्यादिषु तथा शिल्पेषु च परंतप}


\twolineshloka
{इष्वस्त्रे चैव राजेन्द्र द्रोणशिष्यो बभूव ह}
{तस्य माता महाराज राजानं वरवर्णिनी}


\fourlineindentedshloka
{चोदयामास भार्यार्थं कन्यायाः पुत्रवत्तदा}
{ततस्तां पार्षतो दृष्ट्वा कन्यां संप्राप्तयौवनाम्}
{स्त्रियं मत्वा ततश्चिन्तां प्रपेदे सह भार्यया ॥द्रुपद उवाच}
{}


\threelineshloka
{कन्या ममेयं संप्राप्ता यौवनं शोकवर्धिनी}
{मया प्रच्छादिता चेयं वचनाच्छूलपाणिनः ॥भार्योवाच}
{}


\twolineshloka
{न तन्मिथ्या महाराज भविष्यति कथंचन}
{त्रैलोक्यकर्ता कस्माद्धि वृथा वक्तुमिहार्हति}


\twolineshloka
{यदि ते रोचते राजन्वक्ष्यामि श्रृणु मे वचः}
{श्रुत्वेदानीं प्रपद्येथाः स्वां मतिं पृषतात्मज}


\twolineshloka
{क्रियतामस्य यत्नेन विधिवद्दारसंग्रहः}
{भविता तद्वचः सत्यमिति मे निश्चिता मतिः}


\twolineshloka
{ततस्तौ निश्चयं कृत्वा तस्मिन्कार्येऽथ दंपती}
{वरयाञ्चक्रतुः कन्यां दशार्णाधिपतेः सुताम्}


\twolineshloka
{ततो राजा द्रुपदो राजसिंहःसर्वान्राज्ञः कुलतः सन्निशाम्य}
{दाशार्णकस्य नृपतेस्तनूजांशिखण्डिने वरयामास दारान्}


\twolineshloka
{हिरण्यवर्मेति नृपो योऽसौ दाशार्णकः स्मृतः}
{स च प्रादान्महीपालः कन्यां तस्मै शिखण्डिने}


\twolineshloka
{स च राजा दशार्णेषु महानासीत्सुदुर्जयः}
{हिरण्यवर्मा दुर्धर्षो महासेनो महामनाः}


\twolineshloka
{कृते विवाहे तु तदा सा कन्या राजसत्तम}
{यौवनं समनुप्राप्ता सा च कन्या शिखण्डिनी}


\threelineshloka
{कृतदारः शिखण्डी च काम्पिल्यं पुनरागमत्}
{न च सा वेद तां कन्यां कंचित्कालं स्त्रियं किल}
{यदा त्वेनामजानात्सा स्त्रियमेव नृपात्मजा}


\twolineshloka
{धात्रीणां च सखीनां च व्रीडयाना न्यवेदयत्}
{कन्यां पाञ्चालराजस्य सुतां तां वै शिखण्डिनीम्}


\twolineshloka
{ततस्ता राजशार्दूल धात्र्यो दाशार्णिकास्तदा}
{जग्मुरार्ति परां प्रेष्याः प्रेषयामासुरेव च}


\twolineshloka
{ततो दशार्णाधिपतेः प्रेष्याः सर्वा न्यवेदयन्}
{विप्रलम्भं यथावृत्तं स च चुक्रोध पार्थिवः}


\twolineshloka
{शिखण्ड्यपि महाराज पुंवद्राजकुले तदा}
{विजहार म्रुदा युक्तः स्त्रीत्वं नैवातिरोचयन्}


\twolineshloka
{ततः कतिपयाहस्य तच्छ्रुत्वा भरतर्षभ}
{हिरण्यवर्मा राजेन्द्र रोषादर्तिं जगाम ह}


\twolineshloka
{ततो दाशार्णको राजा तीव्रकोपसमन्वितः}
{दूतं प्रस्थापयामास द्रुपदस्य निवेशनम्}


\twolineshloka
{ततो द्रुपदमासाद्य दूतः काञ्चनवर्मणः}
{एक एकान्तसुत्सार्य रहो वचनमब्रवीत्}


\twolineshloka
{दाशार्णराजो राजंस्त्वामिदं वचनमब्रवीत्}
{अभिष्गात्प्रकुपितो विप्रलब्धस्त्वयाऽनघ}


\twolineshloka
{अवमन्यसे मां नृपते नूनं दुर्मन्त्रितं तव}
{यन्मे कन्यां स्वकन्यार्थे मोहाद्याचितवानसि}


\twolineshloka
{तस्याद्य विप्रलम्भस्य फलं प्राप्नुहि दुर्मते}
{एष त्वं सजनामात्यमुद्धरामि स्थिरो भव}


\twolineshloka
{अवमत्य च वीर्यं मे कुलं चारित्रमेव च}
{विप्रलम्भस्त्वयापूर्वो मनुष्येषु प्रवर्तितः}


\twolineshloka
{कुरु सर्वाणि कार्याणि भुङ्क्ष्व कभोगाननुत्तमान्}
{अभियास्यामि शीघ्रं त्वां समुद्धर्तुं सबान्धवम्}


\chapter{अध्यायः १९०}
\twolineshloka
{भीष्म उवाच}
{}


\twolineshloka
{एवमुक्तस्य दूतेन द्रुपदस्य तदा नृप}
{चोरस्येव गृहीतस्य न प्रावर्तत भारती}


\twolineshloka
{स यत्नमकरोत्तीव्रं संबन्धिन्युनुमानने}
{दूतैर्मधुरसंभाषैर्न तदस्तीति संदिशन्}


\twolineshloka
{स राजा भूय एवाथ ज्ञात्वा तत्त्वमथागमत्}
{कन्येति पाञ्चालसुतां त्वरमाणो विनिर्ययौ}


\twolineshloka
{ततः संप्रेषकयामास मित्रांणाममितौजसाम्}
{दुहितुर्विप्रलम्भं तं धात्रीणां वचनात्तदा}


\twolineshloka
{ततः समुदयं कृत्वा बलानां राजसत्तमः}
{अभियाने मतिं चक्रे द्रुपदं प्रति भारत}


\twolineshloka
{ततः संमन्त्रयामास मन्त्रिभिः स महीपतिः}
{हिरण्यवर्मा राजेन्द्र पाञ्चाल्यं पार्थिवं प्रति}


\twolineshloka
{तत्र वै निश्चितं तेषामभूद्राज्ञां महात्मनाम्}
{तथ्यं भवति चेदेतत्कन्या राजञ्शिखण्डिनी}


\twolineshloka
{बद्ध्वा पाञ्चालराजानमानयिष्यामहे गृहम्}
{अन्यं राजानमाधाय पाञ्चालेषु नरेश्वरम्}


% Check verse!
घातयिष्याम नृपतिं पाञ्चालं सशिखण्डिनम्
\threelineshloka
{स तदा द्रुतमाज्ञाय पुनर्दूतान्नराधिपः}
{प्रास्थापयत्पार्षतया निहन्मीति स्थिरो भव ॥भीष्म उवाच}
{}


\twolineshloka
{स हि प्रकृत्या वै भीतः किल्बिषी च नराधिपः}
{भयं तीव्रमनुप्राप्तो द्रुपदः पृथिवीपतिः}


\twolineshloka
{विसृज्य दूतान्दाशार्णे द्रुपदः शोकमूर्च्छितः}
{समेत्य भार्यां रहिते वाक्यमाह नराधिपः}


\twolineshloka
{भयेन महताऽऽविष्टो हृदि शोकेन चाहतः}
{पाञ्चालराजो दयितां मातरं वै शिखण्डिनः}


\twolineshloka
{अभियास्यति मां कोपात्संबन्धी सुमहाबलः}
{हिरण्यवर्मा नृपतिः कर्षमाणो वरूथिनीम्}


\twolineshloka
{किमिदानीं करिष्यावो मूढौ कन्यामिमां प्रति}
{शिखण्डी किल पुत्रस्ते कन्येति परिशङ्कितः}


\twolineshloka
{इति संचिन्त्य यत्नेन समित्रः सबलानुगः}
{वञ्चितोऽस्मीति मन्वानो मां किलोद्धर्तुमिच्छति}


\twolineshloka
{किमत्र तथ्यं सुश्रोणि मिथ्या किं ब्रूहि शोभने}
{श्रुत्वा त्वत्तः शुभं वाक्यं संविधास्याम्यहं तथा}


\twolineshloka
{अहं हि संशयप्राप्तो बाला चेयं शिखण्डिनी}
{त्वं च राज्ञि महत्कृच्छ्रं संप्राप्ता वरवर्णिनि}


\twolineshloka
{सा त्वं सर्वविमोक्षाय तत्त्वमाख्याहि पृच्छतः}
{तथा विदध्यां सुश्रोणि कृत्यमाशु शुचिस्मिते}


\twolineshloka
{शिखण्डिनि च मा भैस्त्वं विधास्ये तत्र तत्त्वतः}
{कृपयाहं वरारोहे वञ्चितः पुत्रधर्मतः}


\twolineshloka
{मया दाशार्णको राजा वञ्चितः स महीपतिः}
{तदाचक्ष्व महाभागे विधास्ये तत्र यद्धितम्}


\twolineshloka
{जनाता ह नरेन्द्रेण ख्यापनार्थं परस्य वै}
{प्रकाशं चोदिता देवी प्रत्युवाच महीपतिम्}


\chapter{अध्यायः १९१}
\twolineshloka
{भीष्म उवाच}
{}


\twolineshloka
{ततः शिखण्डिनो माता यथातत्त्वं नराधिप}
{आचचक्षे महाबाहो भर्त्र कन्यां शिखण्डिनीम्}


\twolineshloka
{अपुत्रया मया राजन्सपत्नीनां भयादिदम्}
{कन्या शिखण्डिनी जाता पुरुषो वै निवेदिता}


\twolineshloka
{त्वया चैव नरश्रेष्ठ तन्मे प्रीत्याऽनुमोदितम्}
{पुत्रकर्म कृतं चैव कन्यायाः पार्थिवर्षभ}


\threelineshloka
{भार्या चोढा त्वया राजन्दशार्णाधिपतेः सुता}
{मया च प्रत्यभिहितं देववाक्यार्थदर्शनात्}
{कन्या भूत्वा पुमान्भावीत्येवं चैतदुपेक्षितम्}


\twolineshloka
{एतच्छ्रुत्वा द्रुपदो यज्ञसेनःसर्वं तत्त्वं मन्त्रविद्भ्यो निवेद्य}
{मन्त्रं राजा मन्त्रयामास राजन्यथायुक्तं रक्षणे वै प्रजानाम्}


\twolineshloka
{संबन्धकं चैव समर्थ्य तस्मिन्दाशार्णके वै नृपतौ नरेन्द्र}
{स्वयं कृत्वा विप्रलम्भं यथाव-न्मन्त्र्यैकाग्रो निश्चयं वै जगाम}


\twolineshloka
{स्वभावगुप्तं नगरमापत्काले तु भारत}
{गोपयामास राजेन्द्र सर्वतः समलङ्कृतम्}


\twolineshloka
{आर्तिं च परमां राजा जगाम सह भार्यया}
{दशार्णपतिना सार्धं विरोधे भरतर्षभ}


\twolineshloka
{कथं संबन्धिना सार्धं न मे स्याद्विग्रहो महान्}
{इति संचिन्त्य मनसा देवतामर्चयत्तदा}


\twolineshloka
{तं तु दृष्ट्वा तदा राजन्देवी देवपरं तदा}
{अर्चां प्रयुञ्जानमथो भार्या वचनमब्रवीत्}


\twolineshloka
{देवानां प्रतिपत्तिश्च सत्यं साधुमता सताम्}
{किमु दुःस्वार्णवं प्राप्य तस्मादर्चयतां गुरून्}


\twolineshloka
{दैवतानि च सर्वाणि पूज्यनां भूरिदक्षिणम्}
{अग्नयश्चापि हृयन्तां दाशार्णप्रतिषेधने}


\twolineshloka
{अयुद्धेन निवृत्तिं च मनसा चिन्तय प्रभो}
{देवतानां प्रसादेन सर्वमेतद्भविष्यति}


\twolineshloka
{मन्त्रिभिर्मन्त्रितं सार्धं त्वया पृथुललोचन}
{पुरस्यास्याविनाशाय यच्च राजंस्तथा कुरु}


\twolineshloka
{दैवं हि मानुषोपेतं भृशं सिद्ध्यति पार्थिव}
{परम्परविरोधाद्धि सिद्धिरस्ति न चैतयोः}


\twolineshloka
{तस्माद्विधाय नगरे विधानं सचिवैः सह}
{अर्चयस्व यथाकामं दैवतानि विशांपते}


\twolineshloka
{एवं संभाषमाणौ तु दृष्ट्वा शोकपरायणौ}
{शिखण्डिनी तदा कन्या व्रीडितेव तपस्विनी}


\twolineshloka
{ततः सा चिन्तयामास मन्कृते दुःखितावुभौ}
{इमाविति ततश्चक्रे मतिं प्राणविनाशने}


\twolineshloka
{एवं सा निश्चयं कृत्वा भृशं शोकपरायणा}
{निर्जगाम गृहं त्यक्त्वा गहनं निर्जनं वनम्}


\twolineshloka
{यक्षेणर्द्धिमता राजन्स्थूणाकर्णेन पालितम्}
{तद्भयादेव च जनो विसर्जयति तद्वनम्}


\twolineshloka
{तत्र च स्थूणभवनं सुधामृत्तिकलेपनम्}
{लाजोल्लापिकधूमाढ्यमुच्चप्राकारतोरणम्}


\twolineshloka
{तन्प्रविश्य शिखण्डी सा द्रुपदस्यात्मजा नृप}
{अनश्नाना बहुतिथं शरीरमुदशोपयत्}


\twolineshloka
{दर्शयामास तां यक्षः स्थूणो मार्दवसंयुतः}
{किमर्थोऽयं तवारम्भः करिष्ये ब्रूहि माचिरम्}


\twolineshloka
{अशक्ययिति सा यक्षं पुनः पुनरुवाच ह}
{करिष्यामीति वै क्षिप्रं प्रत्युवाचाथ गुह्यकः}


\twolineshloka
{धनेश्वरस्यानुचरो वरदोऽस्मि नृपात्मजे}
{अदेयमपि दास्यादि ब्रूहि यत्ते विवक्षितम्}


\threelineshloka
{ततः शिखण्डी तत्सर्वमखिलेन न्यवेदयत्}
{तस्मै यक्षप्रधानाय स्थूणाकर्णाय भारत ॥शिखण्ड्युवाच}
{}


\twolineshloka
{आपन्नो मे पिता यक्ष न चिरान्नाशमेष्यति}
{अभियास्यति सक्रोधो दशार्णाधिपतिर्हि तम्}


\twolineshloka
{मन्निमित्तं महोत्साहः सहेमकवचो नृपः}
{तस्माद्रक्षस्व मां यक्ष मातरं पितरं च मे}


\twolineshloka
{प्रतिज्ञातो हि भवता दुःखप्रतिशमो मम}
{भवेयं पुरुषो यक्ष त्वत्प्रसादादनिन्दितः}


\twolineshloka
{यावदेव स राजा वै नोपयाति पुरं मम}
{तावदेव महायक्ष प्रसादं कुरु गुह्यक}


\chapter{अध्यायः १९२}
\twolineshloka
{भीष्म उवाच}
{}


\twolineshloka
{शिखण्डिवाक्यं श्रुत्वाऽथ स यक्षो भरतर्षभ}
{प्रोवाच मनसा चिन्त्य दैवेनोपनिपीडितः}


\twolineshloka
{भवितव्यं तथा तद्धि मम दुःखाय कौरव}
{भद्रे कामं करिष्यामि समयं तु निबोध मे}


\twolineshloka
{`स्वं ते पुंस्त्वं प्रदास्यामि स्त्रीत्वं धारयिताऽस्मिते'किंचित्कालं तु ते दास्ये पुल्लिङ्गं स्वमिदं तव}
{आगन्तव्यं त्वया काले सत्यं चैव वदस्व मे}


\twolineshloka
{प्रभुः संकल्पसिद्धोऽस्मि कामचारी विहंगमः}
{मत्प्रसादात्पुरं चैव त्राहि बन्धूंश्च केवलम्}


\threelineshloka
{स्त्रीलिङ्गं धारयिष्यामि तवेदं पार्थिवात्मजे}
{सत्यं मे प्रतिजानीहि करिष्यामि प्रियं तव ॥शिखण्ड्युवाच}
{}


\twolineshloka
{प्रतिदास्यामि भगवन्पुलिङ्गं तव सुव्रत}
{किंचित्कालान्तरं स्त्रीत्वं धारयस्व निशाचर}


\threelineshloka
{प्रतियाते दशार्णे तु पार्थिवे हेमवर्मणि}
{कन्यैव हि भविष्यामि पुरुषस्त्वं भविष्यसि ॥भीष्म उवाच}
{}


\twolineshloka
{इत्युक्त्वा समयं तत्र चक्राते तावुभौ नृप}
{अन्योन्यस्यानभिद्रोहे तौ संक्रामयतां ततः}


\twolineshloka
{स्त्रीलिङ्गं धारयामास स्थूणो यक्षोऽथ भारत}
{यक्षरूपं च तद्दीप्तं शिखण्डी प्रत्यपद्यत}


\twolineshloka
{ततः शिखण्डी पाञ्चाल्यः पुंस्त्वमासाद्य पार्थिव}
{विवेश नगरं हृष्टः पितरं च समासदत्}


\threelineshloka
{यथावृत्तं तु तत्सर्वमाचख्यौ द्रुपदस्य तत्}
{`मातुश्च रहिते राजन्प्रसादं यक्षजं तदा}
{'द्रुपदस्तस्य तच्छ्रुत्वा हर्षमाहारयत्परम्}


\twolineshloka
{सभार्यस्तच्च सस्मार महेश्वरवचस्तदा}
{ततः संप्रेषयामास दशार्णाधिपतेर्नृपः}


\twolineshloka
{पुरुषोऽयं मम सुतः श्रद्धत्तां मे भवानिति}
{अथ दाशार्णको राजा सहसाभ्यागमत्तदा}


\twolineshloka
{पाञ्चालराजं द्रुपदं दुःखशोकसमन्वितः}
{ततः काम्पिल्यमासाद्य दशार्णाधिपतिस्ततः}


\twolineshloka
{प्रेषयामास सत्कृत्य दूतं ब्रह्मविदां वरम्}
{ब्रूहि मद्वचनाद्दूत पाञ्चाल्यं तं नृपाधमम्}


\twolineshloka
{यन्मे कन्यां स्वकन्यार्थे वृतवानसि दुर्मते}
{फलं तस्यावलेपस्य द्रक्ष्यस्यद्य न संशयः}


\twolineshloka
{एवमुक्तश्च तेनासौ ब्राह्मणो राजसत्तम}
{दूतः प्रयातो नगरं दाशार्णनृपचोदितः}


\twolineshloka
{ततः आसादयामास पुरोधा द्रुपदं पुरे}
{तस्मै पाञ्चालको राजा गामर्घ्यं च सुसत्कृतम्}


\twolineshloka
{प्रापयामास राजेन्द्र सह तेन शिखण्डिना}
{तां पूजां नाभ्यनन्दत्स वाक्यं चेदमुवाच ह}


\twolineshloka
{यदुक्तं तेन वीरेण राज्ञा काञ्चनवर्मणा}
{यत्तेऽहमधमाचार दुहित्राऽस्म्यभिवञ्चितः}


\twolineshloka
{तस्य पापस्य करणात्फलं प्राप्नुहि दुर्मते}
{देहि युद्धं नरपते ममाद्य रणमूर्धनि}


\twolineshloka
{उद्धरिष्यामि ते सद्यः सामात्यसुतबान्धवम्}
{तदुपालम्भसंयुक्तं श्रावितः किल पार्थिवः}


\twolineshloka
{दशार्णपतिना चोक्तो मन्त्रिमध्ये पुरोधसा}
{अभवद्भरतश्रेष्ठ द्रुपदः प्रणयानतः}


\twolineshloka
{यदाह मां भवान्ब्रह्मन्संबन्धिवचनाद्वचः}
{अस्योत्तरं प्रतिवचो दूतो राज्ञे वदिष्यति}


\twolineshloka
{ततः संप्रेषयामास द्रुपदोऽपि महात्मने}
{हिरण्यवर्मणे दूतं ब्राह्मणं वेदपारगम्}


\twolineshloka
{तमागम्य तु राजानं दशार्णाधिपतिं तदा}
{तद्वाक्यमाददे राजन्यदुक्तं द्रुपदेन ह}


\twolineshloka
{आगमः क्रियतां व्यक्तः कुमारोऽयं सुतो मम}
{मिथ्यैतदुक्तं केनापि तदश्रद्धेयमित्युत}


\threelineshloka
{ततः स राजा द्रुपदस्य श्रुत्वा}
{विमर्शयुक्तो युवतीर्वरिष्ठाः}
{संप्रेषयामास सुचारुरूपाःशिखण्डिनं स्त्रीपुमान्वेति वेत्तुम्}


\twolineshloka
{ताः प्रेषितास्तत्त्वभावं विदित्वाप्रीत्या राज्ञे तच्छशंसुर्हि सर्वम्}
{शिखण्डिनं पुरुषं कौरवेन्द्रदाशार्णराजाय महानुभावम्}


\twolineshloka
{ततः कृत्वा तु राजा स आगमं प्रीतिमानथ}
{संबन्धिना समागम्य हृष्टो वासमुवास ह}


\threelineshloka
{शिखण्डिने च मुदितः प्रादाद्वित्तं जनेश्वरः}
{हस्तिनोऽश्वांश्च गाश्चैव दासीर्बहुशतास्तथा}
{5-192031cपूजितश्च प्रतिययौ निर्भर्त्स्य तनयां किल}


\twolineshloka
{विनीतकिल्विषे प्रीते हेमवर्मणि पार्थिवे}
{प्रतियाते दशार्णे तु हृष्टरूपा शिखण्डिनी}


\twolineshloka
{कस्यचित्त्वथ कालस्य कुबेरो नरवाहनः}
{लोकयात्रां प्रकुर्वाणः स्थूणस्यागान्निवेशनम्}


\twolineshloka
{स नद्गृहस्योपरि वतमानआलोकयामास धनाधिगोप्ता}
{स्थूणस्य यक्षस्य विवेश वेश्मस्वलंकृतं माल्यगुणैर्विचित्रैः}


\twolineshloka
{लाचैश्च गन्धैश्च तथा वितानै-रभ्यर्चितं धृनधृपितं च}
{ध्वजैः पताकाभिरलंकृतं चभक्ष्यान्नपेयामिपदत्तमोदम्}


\twolineshloka
{तत्स्थानं तस्य दृष्ट्वा तु सर्वतः समलंकृतम्}
{मणिरत्नसुवर्णानां मालाभिः परिपूरितम्}


\twolineshloka
{नानाकुसूमगन्धाढ्यं सिक्तसंमृष्टशोभितम्}
{अथाब्रवीद्यक्षपतिस्तान्यक्षाननुगांस्तदा}


\twolineshloka
{स्वलंकृतमिदं वेश्म स्थूणस्यामितविक्रमाः}
{नोपसर्पति मां चैव कस्मादद्य स मन्दधीः}


\threelineshloka
{यस्माज्जानन्स मन्दात्मा मामसौ नोपसर्पति}
{तस्मात्तस्मै महादण्डो धार्यः स्यादिति मे मतिः ॥यक्षा ऊचुः}
{}


\twolineshloka
{द्रुपदस्य सुता राजन्राज्ञो जाता शिखण्डिनी}
{तस्या निमित्ते कस्मिंश्चित्प्रादात्पुरुषलक्षणम्}


\twolineshloka
{अग्रहील्लक्षणं स्त्रीणां स्त्रीभूता तिष्ठते गृहे}
{नोपसर्पति तेनासौ सव्रीडः स्त्रीसरूपवान्}


\twolineshloka
{एतस्मात्कारणाद्राजन्स्थूणो न त्वाऽद्य सर्पति}
{श्रुत्वा कुरु यथान्यायं विमानमिह तिष्ठताम्}


\twolineshloka
{आनीयतां स्थूण इति ततो यक्षाधिपोऽब्रवीत्}
{कर्तास्मि निग्रहं तस्य प्रत्युवाच पुनः पुनः}


\twolineshloka
{सोऽभ्यगच्छत यक्षेन्द्रमाहूतः पृथिवीपते}
{स्त्रीमरूपो महाराज तस्थौ व्रीडासमन्वितः}


\twolineshloka
{तं शशापाथ संक्रुद्धो धनदः कुरूनन्दन}
{एवमेव भवत्वद्य स्त्रीत्वं पापस्य गुह्यकाः}


\twolineshloka
{ततोऽब्रवीद्यक्षपतिर्महात्मायस्माददास्त्ववमन्येह यक्षान्}
{शिखण्डिनो लक्षणं पापबुद्धेस्त्रीलक्षणं चाग्रहीः पापकर्मन्}


\twolineshloka
{अप्रवृत्तं सुदुर्बुद्धे यस्मादेतत्त्वया कृतम्}
{तस्मादद्य प्रभृत्येव स्त्री त्वं सा पुरुषस्तथा}


\twolineshloka
{ततः प्रसादयामासुर्यक्षा वैश्रवणं किल}
{स्थूणस्यार्थे कुरुष्वान्तं शापस्येति पुनः पुनः}


\twolineshloka
{ततो महात्मा यक्षेन्द्रः प्रत्युवाचानुगामिनः}
{सर्वान्यक्षगणांस्तात शापस्यान्तचिकीर्षया}


\twolineshloka
{शिखण्डिनि हते यक्षाः स्वं रूपं प्रतिपत्स्यते}
{स्थूणो यक्षो निरुद्वेगो भवत्विति महामनाः}


\twolineshloka
{इत्युक्त्वा भगवान्देवो यक्षराजः सुपूजितः}
{प्रययौ सहितः सर्वैर्निमेषान्तरचारिभिः}


\twolineshloka
{स्थूणस्तु शापं संप्राप्य तत्रैव न्यवसत्तदा}
{समये चागमूत्तूर्णं शिखण्डी तं क्षपाचरम्}


\twolineshloka
{सोऽभिगम्याब्रवीद्वाक्यं प्राप्तोऽस्मि भगवन्निति}
{तमब्रवीत्ततः स्थूणः प्रीतोऽस्मीति पुनः पुनः}


\threelineshloka
{आर्जवेनागतं दृष्ट्वा राजपुत्रं शिखण्डिनम्}
{सर्वमेव यथावृत्तमाचचक्षे शिखण्डिने ॥भीष्म उवाच}
{}


\twolineshloka
{शप्तो वैश्रवणेनाहं त्वत्कृते पार्थिवात्मज}
{गच्छेदानीं यथाकामं चर लोकान्यथासुखम्}


\threelineshloka
{दिष्टमेतत्पुरा मन्ये न शक्यमतिवर्तितुम्}
{गमनं तव चेतो हि पौलस्त्यस्य च दर्शनम् ॥भीष्म उवाच}
{}


\twolineshloka
{एवमुक्तः शिखण्डी तु स्थूणयक्षेण भारत}
{प्रत्याजगाम नगरं हर्वेण महतावृतः}


\twolineshloka
{पूजयामास विविधैर्गन्धमाल्यैर्महाधनैः}
{द्विजातीन्देवताश्चैव चैत्यानथ चतुष्पथान्}


\twolineshloka
{द्रुपदः सह पुत्रेण सिद्धार्थेन शिखण्डिना}
{मुदं च परमां लेभे पाञ्चाल्यः सह बान्धवैः}


\twolineshloka
{शिष्यार्थं प्रददौ चाथ द्रोणाय कुरुपुङ्गव}
{शिखण्डिनं महाराज पुत्रं स्त्रीपूर्विणं तथा}


\twolineshloka
{प्रतिपेदे चतुष्पादं धनुर्वेदं नृपात्मजः}
{शिखण्डी सह युष्माभिर्धृष्टद्युम्नश्च पार्षतः}


\twolineshloka
{मम त्वेतच्चरास्तात यथावत्प्रत्यवेदयन्}
{जडान्धबधिराकारा ये मुक्ता द्रपदे मया}


\twolineshloka
{एवमेष महाराज स्त्री पुमान्द्रुपदात्मजः}
{स संभूतः कुरुश्रेष्ठ शिखण्डी रथसत्तमः}


\twolineshloka
{ज्येष्ठा काशिपतेः कन्या अम्बा नामेति विश्रुता}
{द्रुपदस्य कुले जाता शिखण्डी भरतर्षभ}


\twolineshloka
{नाहमेनं धनुष्पाणिं युयुत्सुं समुपस्थितम्}
{मुहूर्तमपि पश्येयं प्रहरेयं न चाप्युत}


\twolineshloka
{व्रतमेतन्मम सदा पृथिव्यामपि विश्रुतम्}
{स्त्रियां स्त्रीपूर्वके चैव स्त्रीनाम्नि स्त्रीसरूपिणि}


\twolineshloka
{न मुञ्चेयमहं बाणमिति कौरवनन्दन}
{न हन्यामहमेतेन कारणेन शिखण्डिनम्}


\twolineshloka
{एतत्तत्त्वमहं वेद जन्म तात शिखण्डिनः}
{ततो नैनं हनिष्यामि समरेष्वाततायिनम्}


\threelineshloka
{यदि भीष्मः स्त्रियं हन्यात्सन्तः कुर्युर्विगर्हणम्}
{नैनं तस्माद्धनिष्यामि दृष्ट्वापि समरे स्थितम् ॥वैशंपायन उवाच}
{}


\twolineshloka
{एतच्छ्रुत्वा तु करव्यो राजा दुर्योधनस्तदा}
{मुहूर्तमिव स ध्यात्वा भीष्मे युक्तममन्यत}


\chapter{अध्यायः १९३}
\twolineshloka
{संजय उवाच}
{}


\threelineshloka
{प्रभातायां तु शर्वर्यां पुनरेव सुतस्तव}
{मध्ये सर्वस्य सैन्यस्य पितामहमपृच्छत ॥दुर्योधन उवाच}
{}


\twolineshloka
{पाण्डवेयस्य गाङ्गेय यदेतत्सैन्यमुद्यतम्}
{प्रभूतनरनागाश्वं महारथसमाकुलम्}


\twolineshloka
{भीमार्जुनप्रभृतिभिर्महेष्वासैर्महाबलैः}
{लोकपालसमैर्गुप्तं धृष्टद्युम्नपुरोगमैः}


\twolineshloka
{अप्रधृष्यमनावार्यमुद्धूतमिव सागरम्}
{सेनासागरमक्षोभ्यमपि देवार्महाहवे}


\twolineshloka
{केन कालेन गाङ्गेय क्षपयेथा महाद्युते}
{आचार्यो वा महेष्वासः कृपो वाऽऽशु महाबलः}


\twolineshloka
{कर्णो वा समरश्लाघी द्रौणिर्वा द्विजसत्तमः}
{दिव्यास्त्रविदुषः सर्वे भवन्तो हि बले मम}


\threelineshloka
{एतदिच्छाम्यहं ज्ञातुं परं कौतूहलं हि मे}
{हृदि नित्यं महाबाहो वक्तुमर्हसि तन्मम ॥भीष्म उवाच}
{}


\twolineshloka
{अनुरूपं कुरुश्रेष्ठ त्वय्येतत्पृथिवीपते}
{बलाबलममित्राणां तेषां यदिह पृच्छसि}


\twolineshloka
{श्रृणु राजन्मम रणे या शक्तिः परमा भवेत्}
{शस्त्रवीर्ये रणे यच्च भुजयोश्च महाभुज}


\twolineshloka
{आर्जवेनैव युद्धेन योद्धव्य इतरो जनः}
{मायायुद्धेन मायावी इत्येतद्धर्मनिश्चयः}


\twolineshloka
{हन्यामहं महाभाग पाण्डवानामनीकिनीम्}
{दिवसे दिवसे कृत्वा भागान्भागान्निजान्मम}


\twolineshloka
{योधानां दशसाहस्रं कृत्वा भागं महाद्युते}
{सहस्रं रथिनामेकमेष भागो मतो मम}


\twolineshloka
{अनेनाहं विधानेन सन्नद्धः सततोत्थितः}
{क्षपयेयं महत्सैन्यं कालेनानेन भारत}


\threelineshloka
{मुञ्चेयं यदि वास्त्राणि महान्ति समरे स्थितः}
{शतसाहस्रघातीनि हन्यां मासेन भारत ॥संजय उवाच}
{}


\twolineshloka
{श्रुत्वा भीष्मस्य तद्वाक्यं राजा दुर्योधनस्ततः}
{पर्यपृच्छत राजेन्द्र द्रोणमङ्गिरसां वरम्}


\twolineshloka
{आचार्य केन कालेन पाण्डुपुत्रस्य सैनिकान्}
{निहन्या इति तं द्रोणः प्रत्युवाच हसन्निव}


\twolineshloka
{स्थविरोऽस्मि महाबाहो मन्दप्राणविचेष्टितः}
{शस्त्राग्निना निर्दहेयं पाण्डवानामनीकिनीम्}


\twolineshloka
{यथा भीष्मः शान्तनवो मासेनेति मतिर्मम}
{एषा मे परमा शक्तिरेतन्मे परमं बलम्}


\threelineshloka
{द्वाभ्यामेव तु मासाभ्यां कृपः शारद्वतोऽब्रवीत्}
{द्रौणिस्तु दशरात्रेण प्रतिजज्ञे बलक्षयम्}
{कर्णस्तु पञ्चरात्रेण प्रतिजज्ञे महास्त्रवित्}


\twolineshloka
{तच्छ्रुत्वा सूतपुत्रस्य वाक्यं सागरगासुतः}
{जहास सस्वनं हासं वाक्यं चेदमुवाच ह}


\twolineshloka
{न हि यावद्रणे पार्थं बाणशङ्खधनुर्धरम्}
{वासुदेवसमायुक्तं रथेनायान्तमाहवे}


\twolineshloka
{समागच्छसि राधेय तेनैवमभिमन्यसे}
{शक्यमेवं च भूयश्च त्वया वक्तुं यथेष्टतः}


\chapter{अध्यायः १९४}
\twolineshloka
{वैशंपायन उवाच}
{}


\threelineshloka
{एतच्छ्रुत्वा तु कौन्तेयः सर्वान्भ्रातॄनुपह्वरे}
{आहूय भरतश्रेष्ठ इदं वचनमब्रवीत् ॥युधिष्ठिर उवाच}
{}


\twolineshloka
{धार्तराष्ट्रस्य सैन्येषु ये चारपुरुषा मम}
{ते प्रवृत्तिं प्रयच्छन्ति ममेमां व्युषितां निशाम्}


\twolineshloka
{दुर्योधनः किलापृच्छदापगेयं महाव्रतम्}
{केन कालेन पाण्डूनां हन्याः सैन्यमिति प्रभो}


\twolineshloka
{मासेनेति च तेनोक्तो धार्तराष्ट्रः सुदुर्मतिः}
{तावता चापि कालेन द्रोणोपि प्रतिजज्ञिवान्}


\twolineshloka
{गौतमो द्विगुणं कालमुक्तवानिति नः श्रुतम्}
{द्रौणिस्तु दशरात्रेण प्रतिजज्ञे महास्त्रवित्}


\twolineshloka
{तथा दिव्यास्त्रवित्कर्णः संपृष्टः कुरुसंसदि}
{पञ्चभिर्दिवसैर्हन्तुं ससैन्यं प्रतिजज्ञिवान्}


\twolineshloka
{तस्मादहमपीच्छामि श्रोतुमर्जुन ते वचः}
{कालेन कियता शत्रून्क्षपयेरिति फाल्गुन}


\twolineshloka
{एवमुक्तो गुडाकेशः पार्थिवेन धनञ्जयः}
{वासुदेवं समीक्ष्येदं वचनं प्रत्यभाषत}


\twolineshloka
{सर्व एते महात्मानः कृतास्त्राश्चित्रयोधिनः}
{असंशयं महाराज हन्युरेव न संशयः}


\twolineshloka
{अपैतु ते मनस्तापो यथा सत्यं ब्रवीम्यहम्}
{हन्यामेकरथेनैव वासुदेवसहायवान्}


\twolineshloka
{सामरानपि लोकांस्त्रीन्सर्वान्स्थावरजङ्गमान्}
{भूतं भव्यं भविष्यं च निमेषादिति मे मतिः}


\threelineshloka
{`यावदिच्छेद्धरिरयं तावदस्ति न चान्यथा}
{'यत्तद्धोरं पशुपतिः प्रादादस्त्रं महन्मम}
{कैराते द्वन्द्वयुद्धे तु तदिदं मयि वर्तते}


\twolineshloka
{यद्युगान्ते पशुपतिः सर्वभूतानि संहरन्}
{प्रयुङ्क्ते पुरुषव्याघ्र तदिदं मयि वर्तते}


\twolineshloka
{तन्न जानाति गाङ्गेयो न द्रोणो न च गौतमः}
{न च द्रोणसुतो राजन्कुत एव तु सूतजः}


\twolineshloka
{न तु युक्तं रणे हन्तुं दिव्यैरस्त्रैः पृथग्जनम्}
{आर्जवेनैव युद्धेन विजेष्यामो वयं परान्}


\twolineshloka
{तथेमे पुरुषव्याघ्राः सहायास्तव पार्थिव}
{सर्वे दिव्यास्त्रविद्वांसः सर्वे युद्धाभिकाङ्क्षिणः}


\twolineshloka
{वेदान्तावभृथस्नाताः सर्व एतेऽपराजिताः}
{निहन्युः समरे सेनां देवानामपि पाण्डव}


\twolineshloka
{शिखण्डी युयुधानश्च धृष्टद्युम्नश्च पार्षतः}
{भीमसेनो यमौ चोभौ युधामन्यूत्तमौजसौ}


\twolineshloka
{विराटद्रुपदौ चोभौ भीष्मद्रोणसमौ युधि}
{शङ्खश्चैव महाबाहुर्हैडिम्बश्च महाबलः}


\twolineshloka
{पुत्रोऽस्याञ्जनपर्वा तु महाबलपराक्रमः}
{शैनेयश्च महाबाहुः सहायो रणकेविदः}


\twolineshloka
{अभिमन्युश्च बलवान्द्रौपद्याः पञ्च चात्मजाः}
{स्वयं चापि समर्थोसि त्रेलोक्योत्सादनेपि च}


\twolineshloka
{क्रोधाद्यं पुरुषं पश्येस्तथा शक्रसमद्युते}
{स क्षिप्रं नभवेद्व्यक्तमिति त्वां वेद्मि कौरव}


\chapter{अध्यायः १९५}
\twolineshloka
{वैशंपायन उवाच}
{}


\twolineshloka
{ततः प्रभाते विमले धार्तराष्ट्रेण चोदिताः}
{दुर्योधनेन राजानः प्रययुः पाण्डवान्प्रति}


\threelineshloka
{आप्लाव्य शुचयः सर्वे स्रग्विणः शुक्लवाससः}
{गृहीतशस्त्रा ध्वजिनः स्वस्ति वाच्य हुताग्नयः}
{}


\twolineshloka
{सर्वे ब्रह्मविदः शूराः सर्वे सुचरितव्रताः}
{सर्वे कामकृतश्चैव सर्वे चाहवलक्षणाः}


\twolineshloka
{आहवेषु पराँल्लोकाञ्जिगीषन्तो महाबलाः}
{एकाग्रमनसः सर्वे श्रद्दधानाः परस्परम्}


\twolineshloka
{विन्दानुविन्दावावन्त्यौ केकया बाह्लिकैः सह}
{प्रययुः सर्व एवैते भारद्वाजपुरोगमाः}


\twolineshloka
{अश्वत्थामा शान्तनवः सैन्धवोऽथ जयद्रथः}
{दाक्षिणात्याः प्रतीच्याश्च पार्वतीयाश्च ने नृपाः}


\twolineshloka
{गान्धारराजः शकुनिः प्राच्योदीच्याश्च सर्वशः}
{शकाः किराता यवनाः शिबयोऽथ वसातयः}


\twolineshloka
{स्वैः स्वैरनीकैः सहिताः परिवार्य महारथम्}
{एते महारथाः सर्वे द्वितीये निर्ययुर्बले}


\twolineshloka
{कृतवार्मा सहानीकस्त्रिगर्तश्च महारथः}
{दुर्योधनश्च नृपतिर्भ्रातृभिः परिवारितः}


\twolineshloka
{शलो भूरिश्रवाः शल्यः कौसल्योऽथ बृहद्रथः}
{एते पश्चादनुगता धार्तराष्ट्रपुरोगमाः}


\twolineshloka
{ते समेत्य यथान्यायं धार्तराष्ट्रा महाबलाः}
{कुरुक्षेत्रस्य पश्चार्धे व्यवातिष्ठन्त दंशिताः}


\twolineshloka
{दुर्योधनस्तु शिबिरं कारयामास भारत}
{यथैव हास्तिनपुरं द्वितीयं समलंकृतम्}


\twolineshloka
{न विशेषं विजानन्ति पुरस्य शिबिरस्य वा}
{कुशला अपि राजेन्द्र नरा नगरवासिनः}


\twolineshloka
{तादृशान्येव दुर्गाणि राज्ञामपि महीपतिः}
{कारयामास कौरव्यः शतशोऽथ सहस्रशः}


\twolineshloka
{पञ्चयोजनमुत्सृज्य माण्डलं तद्रणाजिरम्}
{सेनानिवेशास्ते राजन्नाविशञ्छतसङ्घशः}


\twolineshloka
{तत्र ते पृथिवीपाला यथोत्साहं यथाबलम्}
{विविशुः शिबिराण्यत्र द्रव्यवन्ति सहस्रशः}


\twolineshloka
{तेषां दुर्योधनो राजा ससैन्यानां महात्मनाम्}
{व्यादिदेश स बाह्यानां भक्ष्यभोज्यमनुत्तमम्}


\twolineshloka
{सनागाश्वमनुण्याणां ये च शिल्पोपजीविनः}
{ये चान्येऽनुगतास्तत्र सूतमागधबन्दिनः}


\twolineshloka
{वणिजो गणिकाश्चारा ये चैव प्रेक्षका जनाः}
{सर्वांस्तन्कौरवो राजा विधिवत्प्रत्यवैक्षत}


\chapter{अध्यायः १९६}
\twolineshloka
{वैशंपायन उवाच}
{}


\twolineshloka
{तथैव राजा कौन्तेयो धर्मपुत्रो युधिष्ठिरः}
{धृष्टद्युम्नमुखान्वीरांश्चोदयामास भारत}


\twolineshloka
{चेदिकाशिकरूशानां नेतारं दृढविक्रमम्}
{सेनापतिममित्रघ्नं धृष्टकेतुमथादिशत्}


\twolineshloka
{विराटं द्रुपदं चैव युयुधानं शिखण्डिनम्}
{पाञ्चाल्यौ च महेष्वासौ युधामन्यूत्तमौजसौ}


\twolineshloka
{ते शूराश्चित्रवर्माणस्तप्तकुण्डलधारिणः}
{आज्यावसिक्ता ज्वलिता धिष्ण्येष्विव हुताशनाः}


\twolineshloka
{अशोभन्त महेष्वासा ग्रहाः प्रज्वलिता इव}
{अथ सैन्यं यथायोगं पूजयित्वा नरर्षभः}


\twolineshloka
{दिदेश तान्यनीकानि प्रयाणाय महीपतिः}
{तेषां युधिष्ठिरो राजा ससैन्यानां महात्मनाम्}


\twolineshloka
{व्यादिदेश सबाह्यानां भक्ष्यभोज्यमनुत्तमम्}
{स गजाश्वमनुष्याणां ये च शिल्पोपजीविनः}


\twolineshloka
{अभिमन्युं बृहन्तं च द्रौपदेयांश्च सर्वशः}
{धृष्टद्युम्नमुखानेतान्प्राहिणोत्पाण्डुनन्दनः}


\twolineshloka
{भीमं च युयुधानं च पाण्डवं च धनंजयम्}
{द्वितीयं प्रेषयामास बलस्कन्धं युधिष्टिरः}


\twolineshloka
{भाण्डं समारोपयतां चरतां संप्रधावताम्}
{हृष्टानां तत्र योधानां शब्दो दिवमिवास्पशत्}


\twolineshloka
{स्वयमेव ततः पश्चाद्विराटद्रपदान्वितः}
{अथापरैर्महीपालैः सह प्रायान्महीपतिः}


\twolineshloka
{भीमधन्वायती सेना धृष्टद्युम्नेन पालिता}
{गङ्गेव पूर्णा स्तिमिता स्यन्दमाना व्यदृश्यत}


\twolineshloka
{ततः पुनरनीकानि न्ययोजयत बुद्धिमान्}
{मोहयन्धृतराष्ट्रस्य पुत्राणां बुद्धिनिश्चयम्}


\twolineshloka
{द्रौपदेयान्महेष्वासानभिमन्युं च पाण्डवः}
{नकुलं सहदेवं च सर्वांश्चैव प्रभद्रकान्}


\twolineshloka
{दश चाश्वसहस्राणि द्विसहस्त्राणि दन्तिनाम्}
{अयुतं च पदातीनां रथाः पञ्चशतं तथा}


\twolineshloka
{भीमसेनस्य दुर्धर्षं प्रथमं प्रादिशद्बलम्}
{मध्यमे च विराटं च जयत्सेनं च पाण्डवः}


\twolineshloka
{महारथौ च पाञ्चाल्यौ युधामन्यूत्तमौजसौ}
{वीर्यवन्तौ महात्मानौ युधामन्यूत्तमौजसौ}


\threelineshloka
{अन्वयातां तदा मध्ये वासुदेवधनञ्जयौ}
{`तौ दृष्ट्वा पृथिवीपाला नष्टमित्येव मेनिरे}
{अन्तरिक्षगताः सर्वे देवाः सेन्द्रपुरोगमाः '}


\threelineshloka
{बभूवुरतिसंरब्धाः कृतप्रहरणा नराः}
{तेषां विंशतिसाहस्रा हयाः शूरैरधिष्ठिताः}
{पञ्च नागसहस्राणि रथवंशाश्च सर्वशः}


\twolineshloka
{पदातयश्च ये शूराः कार्मुकासिगदाधराः}
{सहस्रशोऽन्वयुः पश्चादग्रतश्च सहस्रशः}


\twolineshloka
{युधिष्ठिरो यत्र सैन्ये स्वयमेव बलार्णवे}
{तत्र ते पृथिवीपाला भूयिष्ठं पर्यवस्थिताः}


\twolineshloka
{तत्र नागसहस्राणि हयानामयुतानि च}
{तथा रथसहस्राणि पदातीनां च भारत}


\twolineshloka
{चेकितानः स्वसैन्येन महता पार्थिवर्षभ}
{धृष्टकेतुश्च चेदीनां प्रणेता पार्थिवो ययौ}


\twolineshloka
{सात्यकिश्च महेष्वासो वृष्णीनां प्रवरो रथः}
{वृतः शतसहस्रेण रथानां प्रमुदन्बली}


\twolineshloka
{क्षत्रदेवब्रह्मदेवौ रथस्थौ पुरुषर्षभौ}
{जघनं पालयन्तौ च पृष्ठतोऽनुप्रजग्मतुः}


\threelineshloka
{शकटापणवेशाश्च यानं युग्यं च सर्वशः}
{तत्र नागसहस्राणि हयानामयुतानि च}
{फल्गु सर्वं कलत्रं च यत्किंचित्कृशदुर्बलम्}


\twolineshloka
{कोशसञ्चयवाहांश्च कोष्ठागारं तथैव च}
{गजानीकेन संगृह्य शनैः प्रायाद्युधिष्ठिरः}


\twolineshloka
{तमन्वयात्सत्यधृतिः सौचित्तिर्युद्धदुर्मदः}
{श्रेणिमान्वसुदानश्च पुत्रः काश्यस्य वा विभुः}


\twolineshloka
{रथा विंशतिसाहस्रा ये तेषामनुयायिनः}
{हयानां दश कोट्यश्च महतां किंकिणीकिनाम्}


\twolineshloka
{गजा विंशतिसाहस्रा ईषादन्ताः प्रहारिणः}
{कुलीना भिन्नकरटा मेघा इव विसर्पिणः}


\twolineshloka
{षष्टिर्नागसहस्राणि दशान्यानि च भारत}
{युधिष्ठिरस्य यान्यासन्युधि सेना महात्मनः}


\twolineshloka
{क्षरन्त इव जीमूताः प्रभिन्नकरटामुखाः}
{राजानमन्वयुः पश्चाच्चलन्त इव पर्वताः}


\twolineshloka
{एवं तस्य बलं भीमं कुन्तीपुत्रस्य धीमतः}
{यदाश्रित्याथ युयुधे धार्तराष्ट्रं सुयोधनम्}


\twolineshloka
{ततोऽन्ये शतशः पश्चात्सहस्रायुतशो नराः}
{नर्दन्तः प्रययुस्तेषाणनीकानि सहस्रशः}


\twolineshloka
{तत्र भेरीसहस्राणि शङ्खानामयुतानि च}
{न्यवादयन्त संहृष्टाः सहस्रायुतशो नराः}


