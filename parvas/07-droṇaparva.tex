\part{द्रॊणपर्व}
\chapter{अध्यायः १}
\threelineshloka
{श्रीवेदव्यासाय नमः}
{नारायणं नमस्कृत्य नरं चैव नरोत्तमम्}
{देवीं सरस्वतीं व्यासं ततो जयमुदीरयेत्}


\threelineshloka
{जनमेजय उवाच}
{तमप्रतिमसत्वौजोबलवीर्यपराक्रमम्}
{हतं देवव्रतं श्रुत्वा पाञ्चाल्येन शिखण्डिना}


\twolineshloka
{धृतराष्ट्रस्ततो राजा शोकव्याकूललोचनः}
{किमचेष्टत विप्रर्षे हते पिताx वीर्यवान्}


\twolineshloka
{तस्य पुत्रो हि भगवान्भीष्भद्रोणमुखै रथैः}
{पमजित्य महेष्वासान्पाण्डव राज्यमिच्छति}


\threelineshloka
{तस्मिन्हा तु भगवन्कतौ सर्वधनुष्मताम्}
{यदचेष्टत कौरवस्तन्मे ब्रूहि तपौधन ॥वैशंपायन उवाच}
{}


\twolineshloka
{निहतं पितरं श्रुत्वा धृतराष्ट्रो जनाधिपः}
{लेभे न शान्तिं कौरव्यश्चिन्ताशौकपरायणः}


\twolineshloka
{तत्व चिन्तयतो दुःखमनिशं पार्थिवस्य तत्}
{आजगाम विशुद्धात्मा पुनर्गावल्गणिस्तदा}


\twolineshloka
{`व्यासप्रसादाद्विज्ञाय सर्वं वृत्तान्तमुत्तमम्}
{सैनिकानां च सर्वेषां सेनयोरुभयोस्तदा'}


\twolineshloka
{शिबिरात्सञ्जयं प्राप्तं निशि नागाह्वयं पुरम्}
{आभ्विकेयो माहारज धृतराष्ट्रोऽन्वपृच्छत}


\threelineshloka
{श्रुत्वा भीष्मस्य निधनमप्रहृष्टमना भृशम्}
{पुत्राणां जयमाकाङ्क्षन्विललापातुरो यथा ॥धृतराष्ट्र उवाच}
{}


\twolineshloka
{संशोच्य तु महात्मानं भीष्मं भीमपराक्रमम्}
{किमकार्षुःपरं तात कुरवः कालचोदिताः}


\twolineshloka
{तस्मिन्विनिहते शूरे दुराधर्षे महात्मनि}
{किं नु स्वित्कुरवोऽकार्षुर्निमग्नाः शोकसागरे}


\twolineshloka
{तदुदीर्णं महत्सैन्यं त्रैलोक्यस्यापि सञ्जय}
{भयमुत्पादयेत्तीव्रं पाण्डवानां महात्मनाम्}


\twolineshloka
{को हि दौर्योधने सैन्ये पुमानासीन्महारथः}
{यं प्राप्य समरे वीरा न त्रस्यन्ति महाभये}


\threelineshloka
{देवव्रते तु निहते कुरूणामृषभे तदा}
{किमकार्षुर्नृपतयस्तन्ममाचक्ष्व सञ्जया ॥सञ्जय उवाच}
{}


\twolineshloka
{शृणु राजन्नेकमना वचनं ब्रुवतो मम}
{यत्ते पुत्रास्तदाकार्षुर्हते देवव्रते मृधे}


\twolineshloka
{निहते तु तदा भीष्मे राजन्सत्यपराक्रमे}
{तावकाः पाण्डवेयाश्च प्राध्यायन्त पृथक्पृथक्}


\twolineshloka
{विस्मिताश्च प्रहृष्टाश्च क्षत्रधर्मं निशाम्य ते}
{स्वधर्मं निन्दमानास्ते प्रमिपत्य महात्मने}


\twolineshloka
{शयनं कल्पयामासुर्भीष्मायामितकर्मणे}
{सोपधानं नरव्याघ्र शरैः सन्नतपर्वभिः}


\twolineshloka
{विधाय रक्षां भीष्माय समाभाष्य परस्परम्}
{अनुमान्य च गाङ्गेयं कृत्वा चापि प्रदक्षिणम्}


\twolineshloka
{क्रोधसंरक्तनयनाः समवेत्य परस्परम्}
{पुनर्युद्धाय निर्जग्मुः क्षत्रियाः कालचोदिताः}


\twolineshloka
{ततस्तूर्यनिनादैश्च भेरीणां निनदेन च}
{तावकानामनीकानि परेषां च विनिर्ययुः}


\twolineshloka
{व्यावृत्तेऽर्यम्णि राजेन्द्र पतिते जाह्नवीसुते}
{अमर्पवशमापन्नाः कालोपहतचेतसः}


\twolineshloka
{अनादृत्य वचः पथ्यं गाङ्गेयस्य महात्मनः}
{निर्ययुर्भरतश्रेष्टाः शस्त्राण्यादाय सत्वराः}


\twolineshloka
{मोहात्तव सपुत्रस्य वधाच्छान्तनवस्य च}
{कौरवा म्नृत्युनाऽऽहूताः सहिताः सर्वराजभिः}


\threelineshloka
{अजावय इवागोपा वने श्वापदसंकुले}
{`कर्णकर्णेति चाक्रन्दञ्छेषा भारत पार्थिवाः'}
{भृशमुद्विग्नमनसो हीना देवव्रतेन ते}


\twolineshloka
{पतिते भरतश्रेष्ठे बभूव कुरुवाहिनी}
{द्यौरिवापेतनक्षत्रा हीनं खमिव वायुना}


\twolineshloka
{विपन्नसस्येव मही वाक्चैवासंस्कृता यथा}
{आसुरीव यथा सेना निगृहीते नृपे वलौ}


\twolineshloka
{विधवेव वारारोहा शुष्कतोयेव निम्नगा}
{वृकैरिव वने रुद्धा पृपती हतयूथपा}


\twolineshloka
{शरभाऽऽहतसिंहेव महती गिरिकन्दरा}
{सा सेना भरतश्रेष्ठे पतिते जाह्नवीसुते}


\twolineshloka
{विष्वग्वाताहतिशुब्धा नौरिवासीन्महार्णवे}
{वलिभिः पाण्डवैर्वीरैर्लब्धलक्षैर्भृशार्दिता}


\twolineshloka
{सा तदासीद्भृशं सेना व्याकुलाश्वरथद्विपा}
{विपन्नभूयिष्ठनरा कृपणा ध्वस्तमानसा}


\threelineshloka
{तस्यां त्रस्ता नृपतयः सैनिकाश्च पृथग्विधाः}
{पाताल इव मज्जन्तो हीना देवव्रतेन ते}
{}


\twolineshloka
{कर्णस्य कुरवोऽस्मार्षुः स हि देवव्रतोपमः}
{सर्वशस्त्रभृतां श्रेष्ठं रोचमानमिवातिथिम्}


\twolineshloka
{बन्धुमापद्गतस्येव तमेवोपागमन्मनः}
{चुक्रुशुः कर्मकर्णेति तत्र भारत पार्थिवाः}


\twolineshloka
{राधेयं हितमस्माकं सूतपुत्रं तनुत्यजम्}
{स हि नाबुध्यत तदा दशाहानि महायशाः}


\twolineshloka
{सामात्यबन्धुः कर्णो वै तमानयत माचिरम्}
{भीष्णेण हि महाबाहुः सर्वक्षत्रस्य पश्यतः}


\twolineshloka
{रथेषु गण्यमानेषु बलविक्रमशालिषु}
{शङ्ख्यातोऽर्धरथः कर्णो द्विगुणः सन्नरर्षभः}


\twolineshloka
{रथातिरक्षसङ्ख्यायां योऽग्रणीः शूरसम्भतः}
{सासुरानपि देवेशान्रणे यो योद्धुमृत्सहेत्}


\threelineshloka
{स तु तेनैव कोपेन राजन्गाङ्गेयमुक्तवान्}
{त्वयि जीकते कौरव्य नाहं योत्स्ये कदाचन}
{}


\twolineshloka
{त्वया तु पाण्डवेयेषु निहतेषु महामृधे}
{दुर्योधनमनुज्ञाप्य वनं यास्यामि कौरव}


\twolineshloka
{हते वा त्वयि पार्थैस्तु युधि स्वर्गमुपेयुषि}
{हन्तास्म्येकरथेनैव कृत्स्नान्वान्मन्यसे रथा}


\twolineshloka
{एवमुक्त्वा महाबाहुर्दशाहान्येक एव स}
{नायुध्यत ततः कर्णः पुत्रस्य तव संयते}


\twolineshloka
{भीष्मः समरविक्रान्तः पाण्डवेयस्य भारत}
{जघान समरे योधानसङ्ख्येयपराक्रमः}


\twolineshloka
{तस्मिंस्तु निहते शूरे सत्यसन्धे महौजसि}
{त्वत्सुताः वर्णमस्मार्पुस्तर्तुकामा इव प्लवम्}


\twolineshloka
{तावकास्तव पुत्राश्च सहिताः सर्वराजभिः}
{हा कर्ण इति चाक्रन्दन्कालोऽयमिति अब्रवन्}


% Check verse!
एवं ते स्म हि राधेयं सूतपुत्रं नतxxxxचुक्रुशुः सहिता योधाxxxxxxxxxxx
\twolineshloka
{जामदग्न्याभ्यनुज्ञातमस्त्रे दुर्वारपौरुषम्}
{अगमन्नो मनः कर्णं बन्धुमात्ययिकेष्विव}


\twolineshloka
{स हि शक्तो रणे राजञ्स्त्रातुमस्मान्महाभयात् ॥त्रिदशानिव गोविन्दः सततं सुमहाभयात् ॥वैशंपायन उवाच}
{}


\threelineshloka
{तथा तु सञ्जयं कर्णं कीर्तयन्तं पुनः पुनः}
{आशीविषवदुच्छ्वस्य धृतराष्ट्रोऽब्रवीदिदम् ॥धृतराष्ट्र उवाच}
{}


\twolineshloka
{यत्तद्वैकर्तनं कर्णमगमद्वो मनस्तदा}
{उप्यरक्षत्स राधेयः सूतपुत्रस्तनुत्यजः}


\twolineshloka
{अपि तन्न मृषाकार्षीत्कच्चित्सत्यपराक्रमः}
{सम्भान्तानां तदार्तानां त्रस्तानां त्राणमिच्छतां}


\twolineshloka
{अपि तत्पूरयांचक्रे धनुर्धरवरो युधि}
{यत्तद्विनिहते भीष्मे कौरवाणामपाकृतम्}


\twolineshloka
{तत्खण्डं पूरयन्कर्णः परेषामादधद्भयम्}
{स हि वै पुरुषव्याघ्रो लोके सञ्जय कथ्यते}


\threelineshloka
{आर्तानां बान्धवानां च क्रन्दतां च विशेषतः}
{परित्यज्य रणे प्राणांस्तत्त्राणार्थं च शर्म च}
{कृतवान्मम पुत्राणां जयाशां सफलामपि}


\chapter{अध्यायः २}
\twolineshloka
{सञ्जय उवाच}
{}


\twolineshloka
{हतं भीष्ममथाधिरथिर्विदित्वाभिन्नां नावं वारिधावत्यगाधे}
{सोदर्यवद्व्यसनात्सूतपुत्रःसंतारयिष्यंस्तव पुत्रस्य सेनाम्}


\twolineshloka
{श्रुत्वा तु कर्णः पुरुषेन्द्रमच्युतंनिपातितं शान्तनवं महारथम्}
{अथोपयायात्सहसारिकर्षणोधनुर्धराणां प्रवरस्तदा नृप}


\twolineshloka
{हते तु रथसत्तमे परैर्निमज्जतीं नावमिवार्णवे कुरून्}
{पितेव पुत्रांस्त्वरितोऽभ्ययात्ततः सन्तारयिष्यंस्तव पुत्रस्य सेनाम्}


% Check verse!
सम्मृज्य दिव्यं धनुराततज्यं स रामदत्तं रिपुसङ्घहन्ता

बाणांश्च कालानलवायुकल्पानुल्लालयन्वाक्यमिदं बभाषे

कर्ण उवाच

यस्मिन्धृतिर्बुद्धिपराक्रमौजः सत्यं स्मृतिर्वीरगुणाश्चसर्वे

अस्त्राणि दिव्यान्यथ संनतिह्रीः प्रिया च वागनसूया चभीष्मे
\twolineshloka
{सदा कृतज्ञे द्विजशत्रुघातके सनातनं चन्द्रमसीव लक्ष्मीः}
{स चेत्प्रशान्तः परवीरहन्ता मन्ये हतानेव च सर्ववीरान्}


\threelineshloka
{नेह ध्रुवं किङ्चन जातु विद्यते}
{लोके ह्यस्मिन्कर्मणोऽनित्ययोगात्}
{सूर्योदये को हि विमुक्तसङ्शयो भावं कुर्वीतार्यमहाव्रते हते}


\threelineshloka
{वसुप्रभावे वसुवीर्यसंभवे गते वसूनेव वसुंधराधिपे}
{वसूनि पुत्रांश्च वसुंधरां तथा कुरूंश्च शोचध्वमिमां च वाहिनीम् ॥सञ्जय उवाच}
{}


\twolineshloka
{महाप्रभावे वरदे निपातिते लोकेश्वरे शास्तरि चामितौजसि}
{पराजितेषु भरतेषु दुर्मनाः कर्णो भृशं न्यश्वसदश्रु वर्तयन्}


\twolineshloka
{इदं च राधेयवचो निशम्य सुताश्च राजंस्तव सैनिकाश्च ह}
{परस्परं चुक्रुशुरार्तिजं मुहुस्तदाश्रु नेत्रैर्मुमुचुश्च शब्दवत्}


\twolineshloka
{प्रवर्तमाने तु पुनर्महाहवे विगाह्यमानासु चमूषु पार्थिवैः}
{अथाब्रवीद्धर्षकरं तदा वचो रथर्पभान्सर्वमहारथर्षभः}


\twolineshloka
{जगत्यनित्ये सततं प्रधावति प्रचिन्तयन्नस्थिरमद्य लक्षये}
{भवत्सु तिष्ठत्स्विह पातितो मृधे निरिप्रकाशः कुरुपुङ्गवः कथम्}


\twolineshloka
{निपातिते शान्तनवे महारथे दिवाकरे भूतलमास्थिते यथा}
{न पार्थिवाः सोढुमलं धनंजयं गिरिप्रवोढारमिवानिलं द्रुमाः}


\twolineshloka
{हतप्रधानं त्विदमार्तरूपं परैर्हतोत्साहमनाथमद्य वै}
{मर्या कुख्णां परिपाल्यमाहवे बलं यथा तेन महात्मना तथा}


\twolineshloka
{समाहिते चात्मनि भारमीदृशं जगत्तथाऽनित्यमिदं च लक्षये}
{निपातितं चाहवशौण्डमाहखे कथं नु कुर्यामहमीदृशे भयम्}


\twolineshloka
{अहं तु तान्कुरुवृषभानजिह्मगैः प्रवेशयन्यमसदनं चरन्रणे}
{यशः परं जगति विभाव्य वर्तिता परैर्हतो भुवि शयिताथवा पुनः}


\twolineshloka
{युधिष्ठिरो धृतिमतिसत्यसत्ववान् वृकोदरो गजशततुल्यविक्रमः}
{तथार्जुनस्त्रिदशवरात्मजो युवा न तद्बलं सुजयमिहामरैरपि}


\twolineshloka
{यमौ रणे यत्र यमोपमौ बले ससात्यकिर्यत्र च देवकीसुतः}
{न तद्बलं कापुरुषोऽभ्युपेयिवान्निवर्तते भृत्युमुखादिवासुभृत्}


\twolineshloka
{तपोऽभ्युदीर्णं तपसैव बाध्यते बलं बलेनैव तथा मनस्विभिः}
{मनश्च मे शत्रुनिवारणे ध्रुं स्वरक्षणे चाचलवद्व्यवस्थितम्}


\twolineshloka
{एवं चैषां बाधमानः प्रभावं गत्वैवाहं ताञ्जयाम्यद्य सूत}
{मित्रद्रोहो मर्षणीयो न मेऽयं भग्ने सैन्ये यः समेयात्स मित्रम्}


\twolineshloka
{कर्तास्म्येतत्सत्पुरुषार्यकर्म त्यक्त्वा प्राणाननुयास्यामि भीष्मम्}
{सर्वान्सङ्ख्ये शत्रुशङ्घान्हनिष्ये हतस्तैर्वा वीरलोकं प्रपत्स्ये}


\twolineshloka
{संप्राक्रुष्टे रुदितस्त्रीकुमारे पराहते पौरुषे धार्तराष्ट्रे}
{मया कृत्यमिति जानामि सूत तस्माद्राज्ञस्त्वद्य शत्रून्विजेष्ये}


\twolineshloka
{कुरून्रक्षन्पाण्डुपुत्राञ्जिघांसंस्त्यक्त्वा प्राणान्घोररूपेरणेऽस्मिन्}
{सर्वान्सङ्ख्ये शत्रुसङ्घान्निहत्य दास्याम्यहंधर्मxxxxxxxx}


\twolineshloka
{निबध्यतां मे कवचं विचित्रं हैमं शुभ्रं मणिरत्नावभासि}
{शिरस्त्राणं चार्कसमानभासं धनुः शरांश्चाग्निविपाहिकल्पान्}


\twolineshloka
{उपासङ्गान्पोडश योजयन्तु धनूम्पि दिव्यानि तथाऽऽहरन्तु}
{असींश्च शक्तीश्च गदाश्च गुर्वीः शङ्खं च जाम्बूनदचित्रनालम्}


\twolineshloka
{इमां रौक्मीं नागकक्ष्यां विचित्रां ध्वजं जैत्रं दिव्यमिन्दीवराङ्कम्}
{श्लक्ष्णैर्वस्त्रैर्विप्रमृज्यानयन्तु चित्रां मालां चारुबद्धांसलाजाम्}


\twolineshloka
{अश्वानग्र्यान्पाण्डुराभ्रप्रकाशान्पुष्टान्स्नातान्मन्त्रपूताभिरद्भिः}
{तप्तैर्भाण्डैः काञ्चनैरभ्युपेतान् शीघ्राञ्शीघ्रं सूतपुत्रानयस्व}


\twolineshloka
{रथं चाग्र्यं हेममालावनद्धं रत्नैश्चित्रं सूर्यचन्द्रप्रकाशैः}
{द्रव्यैर्युक्तं सम्प्रहारोपपन्नैर्बाहैर्युक्तं तूर्णमावर्तयस्व}


\twolineshloka
{चित्राणि चापानि च वेगवन्ति ज्याश्चोत्तमाः सन्नहनोपपन्नाः}
{तूणांश्च पूर्णान्महतः शराणामासाद्य गात्रावरणानि चैव}


\twolineshloka
{प्रायात्रिकं चानयताशु सर्वं पूर्णं कान्त्या वीरकांस्यां च हैमम्}
{आनीय मालामवबध्य चाङ्गे प्रवादयन्त्वाशु जयाय भेरीः}


\twolineshloka
{प्रयाहि मूताशु यतः किरीटी वृकोदरो धर्मसुतो यमौ च}
{तान्वा हनिष्यामि समेत्य सङ्ख्ये भीष्मो यथैष्यामि हतो द्विषद्भिः}


\twolineshloka
{यस्मिन्राजा सत्यधृतिर्युधिष्ठिरः समास्थितो भीमसेनार्जुनौ च}
{वासुदेवः सात्यकिः सृंजयाश्च मन्ये बलं तदजय्यं महीपैः}


\twolineshloka
{तं चेन्मृत्युः सर्वहरोऽभिरक्षेत्सदाऽप्रमत्तः समरे किरीटिनम्}
{तथापि हन्तास्मि समेत्य सङ्ख्ये यास्यामि वा भीष्ममुखो यमाय}


\threelineshloka
{न त्वेवाहं न गमिष्यामि तेषां मध्ये शराणां तत्र चाहं ब्रवीमि}
{मित्रद्रुहो दुर्बलभक्तयो ये पापात्मानो न ममैते सहायाः ॥सञ्जय उवाच}
{}


\twolineshloka
{समृद्धिमन्तं रथमुत्तमं दृढं सकूबरं हेमपरिष्कृतं शुभम्}
{पताकिनं वातजवैर्हयोत्तमैर्युक्तं समास्थाय ययौ जयाय}


\twolineshloka
{सम्पूज्यमानः कुरुभिर्महात्मा रथर्षभो देवगणैर्यथेन्द्रः}
{ययौ तदायोधनमुग्रधन्वा यत्रावसानं भरतर्षभस्य}


\twolineshloka
{वरूथिना महता सध्वजेन सुवर्णमुक्तामणिरत्नमालिना}
{सदश्वयुक्तेन रथेन कर्णो मेघस्वनेनार्क इवामितौजाः}


\twolineshloka
{हुताशनाभः स हुताशनप्रभे शुभः शुभे वै स्वरथे धनुर्धरः}
{स्थितो रराजाधिरथिर्महारथः स्वयं विमाने सुरराडिवास्थितः}


\chapter{अध्यायः ३}
\twolineshloka
{सञ्जय उवाच}
{}


\twolineshloka
{शरवल्पे महात्मानं शयानममितौजसम्}
{महाशतसमूंहन समुद्रमिव शोषितम्}


\threelineshloka
{दृष्ट्वा पितामहं भीष्मं सर्वक्षत्रान्तकं गुरुम्}
{दिव्यैरस्त्रैर्महेष्वासं पातितं सव्यसाचिना}
{जयाणा तव पुत्राणां संभग्ना शर्म वर्म च}


\twolineshloka
{अपाxणामिव द्वीपमगाधे गाधमिच्छताम्}
{स्तोतमा यामुनेनेव शरौघेण परिप्लुतम्}


\twolineshloka
{महान्तमिव मैनाकमहार्यं भुवि पातितम्}
{तxxxयुतमिवादित्यं शक्रस्येवामृतं हृतम्}


\twolineshloka
{शतक्रतुभिवाचिन्त्यं पुरा वृत्रेणा निर्जितम्}
{महोनं सर्वसैन्यस्य युधि भीष्मं निपातितम्}


\threelineshloka
{ककुदं सर्वसैन्यानां लक्ष्म सर्वधनुष्मताम्}
{धनन्तयशरैर्व्याप्तं पितरं ते महाव्रतम्}
{}


\twolineshloka
{तं वीरशयने वीरं शयानं पुरुषर्पभम्}
{भीष्ममाधिरथिर्दृष्ट्वा भरतानां पितामहम्}


\twolineshloka
{प्रस्कन्द्य स रथात्तूर्णं शोकमोहपरिप्लुतः}
{`पद्य्भामेव जगामार्तो बाष्पव्याकुललोचनः' ॥अभिवाद्याञ्जलिं बद्ध्वा वन्दमानोऽभ्यभाषत}


\twolineshloka
{कर्णोऽहमस्मि भद्रं ते वद मामभि भारत}
{पुण्यया क्षेमया वाचा चक्षुषा चावलोकय}


\twolineshloka
{न नूनं सुकृतस्येह फलं कश्चित्समश्नुते}
{यत्र धर्मपरो वृद्धः शेते भुवि भवानिह}


\twolineshloka
{कोशसञ्चयने मन्त्रे व्यूहे प्रहरणेषु च}
{नाहमन्यं प्रपश्यामि कुरूणां कुरुपुङ्गव}


\twolineshloka
{बुद्ध्या विशुद्धया युक्तो यः कुरूंस्तारयेद्भयात्}
{योधांस्त्वमप्लुवे हित्वा पितुलोकं गमिष्यसि}


\twolineshloka
{अद्यप्रभृति सङ्क्रुद्धा व्याघ्रा इव मृगक्षयम्}
{पाण्डवा भरतश्रेष्ठ करिष्यन्ति कुरुक्षयम्}


\twolineshloka
{अद्य गाण्डीवधोपेण वीर्यज्ञाः सव्यसाचिनः}
{कुरवः सन्त्रसिष्यन्ति वज्रघोषादिवासुराः}


\twolineshloka
{अद्य गाण्डीवमुक्तानामशनीनामिव xxxः}
{त्रासयिष्यति बाणानां कुरूनन्यांश्च पथिxx}


\twolineshloka
{समिद्धोऽग्निर्यथा वीर महाज्वालो द्रुमान्दxxxत्}
{धार्तराष्ट्राः इधक्ष्यन्ति तथा बाणाः किरीटिxx}


\twolineshloka
{येन येन प्रसरतो वाय्वग्नी सहितौ वने}
{तेन तेन प्रदहतो भूरिगुल्मतृणद्रुमान्}


\twolineshloka
{यादृशोऽग्निः समुद्भूतस्तादृक्पार्थो न संशयः}
{यथा वायुर्नरव्याघ्र तथा कृष्णो संशयः}


\threelineshloka
{xxदतः पाञ्चजन्यस्य रसतो गाण्डिवस्य च}
{श्रुत्वा सर्वाणि सैन्यानि त्रासं यास्यन्ति भारत}
{}


\twolineshloka
{कपिध्वजस्योत्पततो रथस्यामित्रकर्शिनः}
{शब्दं सोढुं न शक्ष्यन्ति त्वामृते वीर पार्थिवाः}


\threelineshloka
{को ह्यर्जुनं योधयितुं त्वदन्यः पार्थिवोऽर्हति}
{यस्य दिव्यानि कर्माणि प्रवदन्ति मनीषिणः}
{}


\twolineshloka
{अमानुषैश्च संग्रामं त्र्यम्बकेण महात्मना}
{तस्माच्चैव वरं प्राप्तो दुष्प्रापश्चाकृतात्मभिः}


\threelineshloka
{कोऽन्यः शक्तो रणे जेतुं पूर्वं यो न जितस्त्वया}
{जितो येन रणे रामो भवता वीर्यशालिना}
{}


\twolineshloka
{क्षत्रियान्तकरो घोरो देवदानवदर्पहा}
{`सोप्यद्याभिहतः शेते शरैः प्रोतः शिखण्डिना}


\twolineshloka
{तमद्याहं पाण्डवं युद्धशौण्डममृष्यमाणो भवता चानुशिष्टः}
{आशीविषं हष्टिहरं सुघोरं शूरं शक्ष्याम्यस्त्रबलान्निहन्तुम्}


\chapter{अध्यायः ४}
\twolineshloka
{सञ्जय उवाच}
{}


\threelineshloka
{तस्य लालप्यमानस्य कुरुवृद्धः पितामहः}
{देशकालोचितं वाक्यमब्रवीत्प्रीतमानसः ॥भीष्म उवाच}
{}


\twolineshloka
{समुद्र इव सिन्धूनां ज्योतिषामिव भास्करः}
{सत्यस्य च यथा सन्तो बीजानामिव चोर्वरा}


\twolineshloka
{पर्जन्य इव भूतानां प्रतिष्ठा सुहृदां भवान्}
{बान्धवास्त्वाऽनुजीवन्तु स्वादुवृक्षमिवाण़्डजाः}


\twolineshloka
{मानहा भव शत्रूणां मित्राणां नन्दिवर्धनः}
{कौरवाणां भव गतिर्यथा विष्णुर्दिवौकसाम्}


\twolineshloka
{स्वबाहुबलवीर्येण धार्तराष्ट्रप्रियैषिणा}
{कर्ण राजपुरं गत्वा काम्भोजा निर्जितास्त्वया}


\twolineshloka
{गिरिव्रजगताश्चापि नग्नजित्प्रमुखा नृपाः}
{अम्बष्ठाश्च विदेहाश्च गान्धाराश्च जितास्त्वया}


\twolineshloka
{हिमवद्दुर्गनिलयाः किराता रणकर्कशाः}
{दुर्योधनस्य वशगास्त्वया कर्ण पुरा कृताः}


\twolineshloka
{उत्कला मेकलाः पौण्ड्राः कलिङ्गान्ध्राश्च संयुगे}
{निषादाश्च त्रिगर्ताश्च बाह्लीकाश्च जितास्त्वया}


\twolineshloka
{तत्रतत्र च संग्रामे दुर्योधनहितैषिणा}
{बहवश्च जिताः कर्ण त्वया वीरा महौजसा}


\twolineshloka
{यथा दुर्योधनस्यातः सज्ञातिकुलबान्धवः}
{तथा त्वमपि सर्वेषां कौरवाणां गतिर्भव}


\twolineshloka
{शिवेशनाभिवदामि त्वां गच्छ युध्यस्व शत्रुभिः}
{अनुशास्य कुरून्सर्वान्धत्स्व दुर्योधने जयम्}


\twolineshloka
{भवान्पौत्रसमोऽस्माकं यथा दुर्योधनस्तथा}
{तवापि धर्मतः सर्वे यथा तस्य वयं तथा}


\twolineshloka
{यौनात्सम्बन्धकाल्लोके विशिष्टं सङ्गतं सताम्}
{सद्भिः सङ्गममिच्छन्ति तस्मात्प्राज्ञाः परैरपि}


\twolineshloka
{स सत्यसङ्गरो भूत्वा ममेदमिति निश्चितः}
{कुरूणां पालय बलं यथा दुर्योधनस्तथा}


\twolineshloka
{`यथा च कुरवो युद्धे योधयन्ति स्म पाण्डवान्}
{तथा कर्ण त्वया कार्यं दुर्योधनहितैषिणा'}


\twolineshloka
{निशम्य वचनं तस्य चरणावभिवाद्य च}
{ययौ वैकर्तनः कर्णस्तूर्णमायोधनं प्रति}


\twolineshloka
{सोऽभिवीक्ष्य नरौघाणां स्थानमप्रतिमं महत्}
{व्यूढप्रहरणोरस्कं सैन्यं तत्समबृंहयत्}


\twolineshloka
{हृषिताः कुरवः सर्वे दुर्योधनपुरोगमाः}
{उपागतं महाबाहुं सर्वानीकपुरःसरम्}


\threelineshloka
{कर्णं दृष्ट्वा महात्मानं युद्धाय समुपस्थितम्}
{क्ष्वेडितास्फोटितरवैः सिंहनादरवैरपि}
{धनुःशब्दैश्च विविधैः कुरवः समपूजयन्}


\chapter{अध्यायः ५}
\twolineshloka
{सञ्जय उवाच}
{}


\twolineshloka
{रथस्थं पुरुषव्याघ्रं दृष्टा कर्णमवस्थितम्}
{हृष्टो दुर्योधनो राजन्निदं वचनमब्रवीत्}


\threelineshloka
{सनाथमिव मन्येऽहं भवता पालितं बलम्}
{अत्र किं नु समर्थं यद्धितं तत्सम्प्रधार्यताम् ॥कर्ण उवाच}
{}


\twolineshloka
{ब्रूहि नः पुरुषव्याघ्र त्वं हि प्राज्ञतमो नृषु}
{यथा चार्थपतिः कृत्यं पश्यते न तथेतरः}


\threelineshloka
{ते स्म सर्वे तव वचः श्रोतुकामा नरेश्वर}
{नान्याय्यं हि भवान्वाक्यं ब्रूयादिति मतिर्मम ॥दुर्योधन उवाच}
{}


\twolineshloka
{भीष्मः सेनाप्रणेताऽऽसीद्वयसा विक्रमेण च}
{श्रुतेन चोपसम्पन्नः सर्वैर्योधगणैस्तथा}


\twolineshloka
{तेनातियशसा कर्ण घ्नता शत्रुगणान्मम}
{सुयुद्धेन दशाहानि पालिताः स्मो महात्मना}


\twolineshloka
{तस्मिन्नसुकरं कर्म कृतवत्यास्थिते दिवम्}
{कं तु सेनाप्रणेतारं मन्यसे तदनन्तरम्}


\threelineshloka
{न विना नायकं सेना मुहूर्तमपि तिष्ठति}
{`आहवेषु विशेषेण भ्रष्टनेत्रेष्विवाञ्जनम्}
{'आहवेष्वाहवश्रेष्ठ नेतृहीनेव नौर्जले}


\twolineshloka
{यथा ह्यकर्णधारा नौ रथश्चासारथिर्यथा}
{द्रवेद्यथेष्टं तद्वत्स्यादृते सेनापतिं बलम्}


\twolineshloka
{अदेशिको यथा सार्थः सर्वः कृच्छ्रं समृच्छति}
{अनायका तथा सेना सर्वान्दोषान्समर्च्छति}


\twolineshloka
{स भावन्वीक्ष्य सर्वेषु मामकेषु महात्मसु}
{पश्य सेनापतिं युक्तमनु शान्तनवादिह}


\fourlineindentedshloka
{यं हि सेनाप्रमेतारं भवान्वक्ष्यति संयुगे}
{तं वयं सहिताः सर्वे करिष्यामो न संशयः}
{कर्ण उवाच}
{}


\twolineshloka
{सर्व एव महात्मान इमे पुरुषसत्तमाः}
{सेनापतित्वमर्हन्ति नात्र कार्या विचारणा}


\twolineshloka
{कुलसंहननज्ञानैर्बलविक्रमबुद्धिभिः}
{युक्ताः श्रुतज्ञा धीमन्त आहवेष्वनिवर्तिनः}


\twolineshloka
{युगपन्न तु ते शक्याः कर्तुं सर्वे पुरःसराः}
{एक एव तु कर्तव्यो यस्मिन्वैशेषिका गुणाः}


\twolineshloka
{अन्योन्यस्पर्धिनां ह्येषां यद्येकं त्वं करिष्यसि}
{शेषा विमनसो व्यक्तं न योत्स्यन्ति हितास्तव}


\twolineshloka
{अयं च सर्वयोधानामाचार्यः स्थिवरो गुरुः}
{युक्तः सेनापतिः कर्तुं द्रोणः शस्त्रभृतां वरः}


\twolineshloka
{को हि तिष्ठति दुर्धर्षे द्रोणे शस्त्रभृतां वरे}
{सेनापतिःस्यादन्योस्माच्छुक्राङ्गिरसदर्शनात्}


\twolineshloka
{न च सोऽप्यस्ति ते योधः सर्वराजसु भारत}
{द्रोणं यः समरे यान्तमनुयास्यति संयुगे}


\twolineshloka
{एष सेनाप्रणेतॄणामेष शस्त्रभृतामपि}
{एष बुद्धिमतां चैव श्रेष्ठो राजन्गुरुस्तव}


\twolineshloka
{एवं दुर्योधनाचार्यमाशु सेनापतिं कुरु}
{जिगीषन्तो सुरान्सङ्ख्ये कार्तिकेयमिवामराः}


\chapter{अध्यायः ६}
\twolineshloka
{सञ्जय उवाच}
{}


\threelineshloka
{कर्णस्य वचनं श्रुत्वा राजा दुर्योधनस्तदा}
{सेनामध्यगतं द्रोणमिदं वचनमब्रवीत् ॥दुर्योधन उवाच}
{}


\twolineshloka
{वर्णश्रैष्ठ्यात्कुलोत्पत्त्या श्रुतेन वयसा धिया}
{वीर्याद्दाक्ष्यादधृष्यत्वादर्थज्ञानान्नयाज्जयात्}


\twolineshloka
{तपसा च कृतज्ञत्वाद्वृद्धः सर्वगुणैरपि}
{युक्तो भवत्सश्रो गोप्ता राज्ञामन्यो न विद्यते}


\twolineshloka
{स भवान्पातु नः सर्वान्देवानिव शतक्रतुः}
{भवन्नेत्राः पराञ्जेतुमिच्छामो द्विजसत्तम}


\twolineshloka
{रुद्राणामिव कापालिर्वसूनामिव पावकः}
{कुबेर इव यक्षाणां मरुतामिव वासवः}


\twolineshloka
{वसिष्ठ इव विप्राणां तेजसामिव भास्करः}
{पितॄणामिव धर्मेन्द्रो यादसामिव चाम्बुराट्}


\twolineshloka
{नक्षत्राणामिव शशी दितिजानामिवोशनाः}
{`सर्वेषामिव लोकानां विश्वस्य च यथा क्षयः'}


\twolineshloka
{विश्वोत्पत्तिस्थितिलये श्रेष्ठो नारायणः प्रभुः}
{एवं सेनाप्रणेतॄणां मम सेनापतिर्भव}


\twolineshloka
{अक्षौहिण्यो दशैका च वशगाः सन्तु तेऽनघ}
{ताभिः शत्रून्प्रतिव्यूह्य जहीन्द्रो दानवानिव}


\twolineshloka
{प्रयातु नो भवानग्रे देवानामिव पावकिः}
{अनुयास्यामहे त्वाऽऽजौ सौरभेया इवर्षभम्}


\twolineshloka
{उग्रधन्वा महेष्वासो दिव्यं विष्फारयन्धनुः}
{दृष्ट्वा भवन्तं सङ्ग्रामे नार्जुनः प्रहरिष्यति}


\threelineshloka
{ध्रुवं युधिष्ठिरं सङ्ख्ये सानुबन्धं सबान्धवम्}
{जेष्यामि पुरुषव्याघ्र भवान्सेनापतिर्यदि ॥सञ्जय उवाच}
{}


\twolineshloka
{एवमुक्ते ततो द्रोणं जयेत्यूचुर्नराधिपाः}
{सिंहनादेन महता हर्षयन्तस्तवात्मजम्}


\threelineshloka
{सैनिकाश्च मुदा युक्ता वर्धयन्ति द्विजोत्तमम्}
{दुर्योधनं पुरस्कृत्य प्रार्थयन्तो महद्यशः}
{दुर्योधनं ततो राजन्द्रोणो वचनमब्रवीत्}


\chapter{अध्यायः ७}
\twolineshloka
{द्रोण उवाच}
{}


\twolineshloka
{वेदं षडङ्गं वेदाहमर्थविद्यां च मानवीम्}
{त्रैय्यम्बकमथेष्वस्त्रं शस्त्राणि विविधानि च}


\twolineshloka
{ये चाप्युक्ता मयि गुणा भवद्भिर्जयकाङ्क्षिभिः}
{चिकीर्षुस्तानहं सर्वान्योधयिष्यामि पाण्डवान्}


\twolineshloka
{पार्षतं तु रणे राजन्न हनिष्ये कथञ्चन}
{स हि सृष्टो वधार्थाय ममैव पुरुषर्षभः}


\threelineshloka
{योधयिष्यामि सैन्यानि नाशयन्सर्वसोमकान्}
{न च मां पाण्डवा युद्धे योधयिष्यन्ति हर्षिताः ॥सञ्जय उवाच}
{}


\twolineshloka
{स एवमभ्यनुज्ञातश्चक्रे सेनापतिं ततः}
{द्रोणं तव सुतो राजन्विधिदृष्टेन कर्मणा}


\threelineshloka
{अथाभिषिषिचुर्द्रोणं दुर्योधनमुखा नृपाः}
{`स्वस्तिवादरवैश्चान्यैः श्लक्ष्णैश्चान्यैर्मनोरमैः'}
{सैनापत्ये यथा स्कन्दं पुरा शक्रमुखाः सुराः}


\twolineshloka
{ततो वादित्रघोषेण शङ्खानां च महास्वनैः}
{प्रादुरासीत्कृते द्रोणे हर्षः सेनापतौ तदा}


\twolineshloka
{ततः पुण्याहघोषेण साशीर्वादस्वनेन च}
{संस्तवैर्गीतशब्दैश्च सूतमागधबन्दिनाम्}


\threelineshloka
{जयशब्दैर्द्विजाग्र्याणां सुभगानर्तितैस्तथा}
{सत्कृत्य विधिना द्रोणं मेनिरे पाण्डवाञ्जितान् ॥सञ्जय उवाच}
{}


\twolineshloka
{सैनापत्यं तु संप्राप्य भारद्वाजो महारथः}
{युयुत्सुर्व्यूह्य सैन्यानि प्रायात्तव सुतैः सह}


\twolineshloka
{सैन्धवश्च कलिङ्गश्च विकर्णश्च तवात्मजः}
{दक्षिणं पार्श्वमास्थाय समतिष्ठन्त दंशिताः}


\twolineshloka
{प्रपक्षः शकुनिस्तेषां प्रवरैर्हयसादिभिः}
{ययौ गान्धारकैः सार्धं विमलप्रासयोधिभिः}


\twolineshloka
{कृपश्च कृतवर्मा च चित्रसेनो विविंशतिः}
{दुःशासनमुखा यत्ताः सव्यं पक्षमपालयन्}


\twolineshloka
{तेषां प्रपक्षाः काम्भोजाः सुदक्षिणपुस्सराः}
{ययुरश्चैर्महावेगैः शकाश्च यवनैः सह}


\twolineshloka
{मद्रास्त्रिगर्ताः साम्बष्ठाः प्रतीच्योदीच्यमालवाः}
{शिबयः शूरसेनाश्च शूद्राश्च मलदैः सह}


\twolineshloka
{सौवीराः कितवाः प्राच्या दाक्षिणात्याश्च सर्वशः}
{नवात्मजं पुरस्कृत्य सूतपुत्रस्य पृष्ठतः}


\twolineshloka
{हर्षयन्तः स्वसैन्यानि ययुस्तव सुतैः सह}
{प्रवरः सर्वयोधानां बलेषु बलमादधम्}


\twolineshloka
{ययौ वैकर्तनः कर्णः प्रमुखे सर्वधन्विनाम्}
{तस्य दीप्तो महाकायः स्वान्यनीकानिहर्षयन्}


\twolineshloka
{हस्तिकक्ष्यो गहाकेतुर्बभौ सूर्यसमद्युतिः}
{न भीष्मव्यसनं कश्चिदृष्ट्वा कर्णममन्यत}


\twolineshloka
{विशोकाश्चाभवन्सर्वे राजानः कुरुभिः सह}
{हृष्टाश्च बहवो योधास्तत्राजल्पन्त वेगतः}


\twolineshloka
{न हि कर्णं रणे दृष्ट्वा युधि स्थास्यन्ति पाण्डवाः}
{कर्णो हि समरे शक्तो जेतुं देवान्सवासवान्}


\twolineshloka
{किमु पाण्डुसुतान्युद्धे हीनवीर्यपराक्रमान्}
{भीष्मेण तु रणे पार्थाः पालिता बाहुशालिना}


\twolineshloka
{तांस्तु कर्णः शरैस्तीक्ष्णैर्नाशयिष्यति संयुगे}
{एवं ब्रुवन्तस्तेऽन्योन्यं हृष्टरूपा विशांपते}


\twolineshloka
{राधेयं पूजयन्तश्च प्रशंसन्तश्च निर्ययुः}
{अस्माकं शकटव्यूहों द्रोणेन विहितोऽभवत्}


\twolineshloka
{परेषां क्रौञ्च एवासीद्व्यूहो राजन्महात्मनाम्}
{प्रीयमाणेन विहितो धर्मराजेन भारत}


\twolineshloka
{व्यूहप्रमुखतस्तेषां तस्थतुः पुरुषर्षभौ}
{वानरध्वजमुच्छ्रित्य विष्वक्सेनधनंजयौ}


\twolineshloka
{ककुदं सर्वसैन्यानां धाम सर्वधनुष्मताम्}
{आदित्यपथगः केतुः पार्थस्यामिततेजसः}


\twolineshloka
{दीपयामास तत्सैन्यं पाण्डवस्य महात्मनः}
{यथा प्रज्वलितः सूर्यो युगान्ते वै वसुंधराम्}


\twolineshloka
{दीप्यन्दृश्येत हि तथा केतुः सर्वत्र धीमतः}
{योधानामर्जुनः श्रेष्ठो गाण्डीवं धनुषां वरम्}


\twolineshloka
{वासुदेवश्च भूतानां चक्राणां च सुदर्शनम्}
{चत्वार्येतानि तेजाञ्सि वहञ्श्वेतहयो रथः}


\twolineshloka
{परेषामग्रतस्तस्थौ कालचक्रमिवोद्यतम्}
{एवं तौ सुमहात्मानौ बलसेनाग्रगावुभौ}


\twolineshloka
{तावकानां मुखे कर्णः परेषां च धनंजयः}
{ततो जयाभिसंरब्धौ परस्परवधैषिणौ}


\twolineshloka
{अवेक्षेतां तदान्योन्यं समरे कर्णपाण्डवौ}
{ततः प्रयाते सहसा भारद्वाजे महारथे}


\twolineshloka
{सिंहनादेन घोरेण वसुधा समकम्पत}
{ततस्तुमुलमाकाशमावृणोत्सदिवाकरम्}


\twolineshloka
{वातोद्भूतं रजस्तीव्रं कौशेयनिकरोपमम्}
{ववर्ष द्यौरनभ्राऽपि मांसास्थिरुधिराण्युत}


\twolineshloka
{गृध्राः श्येना बकाः कङ्का वायसाश्च सहस्रशः}
{उपर्युपरि सेनां ते तदा पर्यपतन्नृप}


\twolineshloka
{गोमायवश्च प्राक्रोशन्भयदान्दारुणान्रवान्}
{अकार्षुरपसव्यं च बहुशः पृतनां तव}


\twolineshloka
{चिखादिषन्तो मांसानि पिपासन्तश्च शोणितम्}
{अपतद्दीप्यमाना च सनिर्घाता सकम्पना}


\twolineshloka
{उल्का ज्वलन्ती सङ्ग्रामे पुच्छेनावृत्य सर्वशः}
{परिवेषो महांश्चापि सविद्युत्स्तनयित्नुमान्}


\threelineshloka
{भास्करस्याभवद्राजन्प्रयाते वाहिनीपतौ}
{एते चान्ये च बहवः प्रादुरासन्सुदारुणाः}
{उत्पाता युधि वीराणां जीवितक्षयकारिणः}


\twolineshloka
{ततः प्रववृते युद्धं परस्परवधैषिणाम्}
{कुरुपाण्डवसैन्यानां शब्देनापूरयज्जगत्}


\twolineshloka
{ते त्वन्योन्यं सुसंरब्धाः पाण्डवाः कौरवैः सह}
{अभ्यघ्नन्निशितैः शस्त्रैर्जयगृद्धाः प्रहारिणः}


\twolineshloka
{स पाण्डवानां महतीं महेष्वासो महाद्युतिः}
{वेगेनाभ्यद्रवत्सेनां किरञ्शरशतैः शितैः}


\twolineshloka
{द्रोणमभ्युद्यतं दृष्ट्वा पाण्डवाः सहसृञ्जयैः}
{प्रत्यगृह्णंस्तदा राजञ्छरवर्षैः पृथक्पृथक्}


\twolineshloka
{विक्षोभ्यमाणआ द्रोणेन भिद्यमाना महाचमूः}
{व्यशीर्यत सपाञ्चाला वातेनेव वलाहकाः}


\twolineshloka
{बहूनीह विकुर्वाणो दिव्यान्यस्त्राणि संयुगे}
{अपीडयत्क्षणेनैव द्रोणः पाण्डवसृञ्जयान्}


\twolineshloka
{ते वध्यमाना द्रोणेन वासवेनेव दानवाः}
{पाञ्चालाः समकम्पन्त धृष्टद्युम्नपुरोगमाः}


\twolineshloka
{ततो दिव्यास्त्रविच्छूरो याज्ञसेनिर्महारथः}
{अभिनच्छरवर्षेण द्रोणानीकमनेकधा}


\twolineshloka
{द्रोणस्य शरवर्षाणि शरवर्षेण पार्षतः}
{सन्निवार्य ततः सर्वान्कुरूनप्यवधीद्बली}


\twolineshloka
{संयम्य तु ततो द्रोणः समवस्थाप्य चाहवे}
{स्वमनीकं महेष्वासः पार्षतं समुपाद्रवत्}


\twolineshloka
{स बाणवर्षं सुमहदसृजत्पार्षतं प्रति}
{मघवान्समभिक्रुद्धः सहसा दानवानिव}


\twolineshloka
{ते कम्प्यमाना द्रोणेन बाणैः पाण्डवसृञ्जयाः}
{पुनःपुनरभज्यन्त सिंहेनैवेतरे मृगाः}


\twolineshloka
{तथा पर्यपतद्रोणः पाण्डवानां बले बली}
{अलातचक्रवद्राजंस्तदद्भुतमिवाभवत्}


\twolineshloka
{खचरनगरकल्पं कल्पितं शास्त्रदृष्ट्या चलदनिलपताकं ह्लादनं वल्गिताश्वम्}
{स्फटिकविमलकेतुं त्रासनं शात्रवाणां रथवरमधिरूढः सञ्जहारारिसेनाम्}


\chapter{अध्यायः ८}
\twolineshloka
{सञ्जय उवाच}
{}


\twolineshloka
{तथा द्रोणमभिघ्नन्तं साश्वसूतरथद्विपान्}
{व्यथिताः पाण्डवा दृष्ट्वा न चैनं पर्यवारयन्}


\twolineshloka
{ततो युधिष्ठिरो राजा धृष्टद्युम्नधनञ्जयौ}
{अब्रवीत्सर्वतो यत्तैः कुम्भयोनिर्निवार्यताम्}


\threelineshloka
{तत्रैनमर्जुनश्चैव धृष्टद्युम्नश्च पार्षतः}
{`ये चान्ये पार्थिवा राजन्पाण्डवस्यानुसैनिकाः}
{प्रत्यगृह्णंस्ततस्तत्र समागच्छन्महारथाः}


\twolineshloka
{केकया भीमसेनश्च सौभद्रोऽथ घटोत्कचः}
{युधिष्ठिरो यमौ मात्स्यो द्रुदश्चात्मजैः सह}


\twolineshloka
{द्रौपदेयाश्च संहृष्टा धृष्टकेतुः ससात्यकिः}
{चेकितानश्च सङ्क्रुद्धो युयुत्सुश्च महारथः}


\twolineshloka
{ये चान्ये पार्थिवा राजन्पाण्डवस्यानुयायिनः}
{कुलवीर्यानुरूपाणि चक्रुः कर्माण्यनेकशः}


\twolineshloka
{संभिद्यमानां तां दृष्ट्वा पाण्डवैर्वाहिनीं रणे}
{व्यावृत्य चक्षुषी कोपाद्भारद्वाजोऽन्ववैक्षत}


\twolineshloka
{स तीव्रं कोपमास्थाय रथे समरदुर्जयः}
{व्यधमत्पाण्डवानीकं महाभ्राणीव मारुतः}


\twolineshloka
{रथानश्वान्नरान्नागानभिधावन्नितस्ततः}
{चचारोन्मत्तवद्द्रोणो वृद्धोऽपि तरुणो यथा}


\twolineshloka
{तस्य शोणितदिग्धाङ्गाः शोणास्ते वातरंहसः}
{आजानेया हया राजन्नविश्रान्ताः सुखं ययुः}


\twolineshloka
{तमन्तकमिव क्रुद्धमापतन्तं यतव्रतम्}
{दृष्ट्वा सम्प्राद्रवन्योधाः पाण्डवस्य ततस्ततः}


% Check verse!
तेषां प्राद्रवतां भीमः पुनरावर्ततामपि ॥पश्यतां तिष्ठतां चासीच्छब्दः परमदारुणः
\twolineshloka
{शूराणां हर्षजननो भीरूणां भयवर्धनः}
{द्यावापृथिव्योर्विवरं पूरयामास सर्वतः}


\twolineshloka
{ततः पुनरपि द्रोणो नाम विश्रावयन्युधि}
{अकरोद्रौद्रमात्मानं किरञ्छरशतैः परान्}


\twolineshloka
{स तथा तेष्वनीकेषु पाण्डुपुत्रस्य मारिष}
{कालवद्व्यचरद्द्रोणो युवेव स्थविरो बली}


\twolineshloka
{उत्कृत्य च शिरांस्युग्रो बाहूनपि सुभूषणान्}
{कृत्वा शून्यान्रथोपस्थानुदक्रोशन्महारवान्}


\twolineshloka
{तस्य हर्षप्रणादेन बाणवर्षेण च प्रभो}
{प्राकम्पन्त रणे योधा गावः शीतार्दिता इव}


\twolineshloka
{द्रोणस्य रथघोषेण मौर्वीनिष्पेषणेन च}
{धनुःशब्देन चाकाशे शब्दः समभवन्महान्}


\threelineshloka
{अथास्य धनुषो बाणा निस्सरन्तः सहस्रशः}
{व्याप्य सर्वा दिशः पेतुर्नागाश्वरथपत्तिषु}
{}


\twolineshloka
{तं कार्मुकमहावेगमस्त्रज्वलितपावकम्}
{द्रोणमासादयाञ्चक्रुः पाञ्चालाः पाण्डवैः सह}


\twolineshloka
{सनागरथपत्त्यश्वान्प्राहिणोद्यमसादनम्}
{अचिरादकरोद्दोणो महीं शोणितकर्दमाम्}


\twolineshloka
{तन्वता परमास्त्राणि शरान्सततमस्यता}
{द्रोणेन विहितं दिक्षु शरजालमदृश्यत}


\twolineshloka
{पदातिषु रथाश्वेषु वारणेषु च सर्वशः}
{तस्य विद्युदिवाभ्रेषु चरन्केतुरदृश्यत}


\twolineshloka
{सकेकयानां प्रवरांश्च पञ्च पाञ्चालराजं च शरैः प्रमथ्य}
{युधिष्ठिरानीकमदीनसत्वो द्रोणोऽभ्ययात्कार्मुकबाणपाणिः}


\twolineshloka
{तं भीमसेनश्च धनञ्जयश्च शिनेश्च नप्ता द्रुपदात्मजश्च}
{शैब्यात्मजः काशिपतिः शिबिश्च दृष्ट्वा नदन्तो व्यकिरञ्छरौघैः}


\twolineshloka
{`तेषां शरा द्रोणशरैर्निकृत्ता भूमावदृश्यन्त विवर्तमानाः}
{श्रेणीकृताः संयति मोघवेगा द्वीपे नदीनामिव काशरोहाः'}


\twolineshloka
{ते द्रोणबाणासनविप्रमुक्ताः पतत्रिणः काञ्चनचित्रपुङ्खाः}
{भित्त्वा शरीराणि गजाश्वयूनां जग्मुर्महीं शोणितदिग्धवाजाः}


\twolineshloka
{सा योधसङ्घैश्च रथैश्च भूमिः शरैर्विभिन्नैर्गजवाजिभिश्च}
{प्रच्छाद्यमाना पतितैर्बभूव समावृता द्यौरिव कालमेघैः}


\twolineshloka
{शैनेयभीमार्जुनवाहिनीशं सौभ्रद्रपाञ्चालसकाशिराजम्}
{अन्यांश्च वीरान्समरे ममर्द द्रोणः सुतानां तव भूतिकामः}


\twolineshloka
{एतानि चान्यानि च कौरवेन्द्र कर्माणि कृत्वा समरे महात्मा}
{प्रताप्य लोकानिव कालसूर्यो द्रोणो गतः स्वर्गमितो हि राजन्}


\twolineshloka
{एवं रुक्मरथः शूरो हत्वा शतसहस्रशः}
{पाण़्डवानां रणे योधान्पार्षतेन निपातितः}


\twolineshloka
{अक्षौहिणीमभ्यधिकां शूराणामनिवर्तिनाम्}
{निहत्य पश्चाद्धृतिमानगच्छत्परमां गतिम्}


\twolineshloka
{पाण्डवैः सहपाञ्चालैरशिवैः क्रूरकर्मभिः}
{हतो रुक्मरथो राजन्कृत्वा कर्म सुदुष्करम्}


\twolineshloka
{ततो निनादो भूतानामाकाशे समजायत}
{सैन्यानां च ततो राजन्नाचार्ये निहते युधि}


\twolineshloka
{द्यां धरां खं दिशो वापि प्रदिशर्श्चानुनादयन्}
{अहो धिगिति भूतानां शब्दः समभावद्भृशम्}


\twolineshloka
{देवताः पितरश्चैव पूर्वे ये चास्य बान्धवाः}
{ददृशुर्निहतं तत्र भारद्वाजं महारथम्}


\twolineshloka
{पाण्डवास्तु जयं लब्ध्वा सिंहनादान्प्रचक्रिरे}
{सिंहनादेन महता समकम्पत मेदिनी}


\twolineshloka
{`विचित्रजाम्बूनदभूषितध्वजं महारथं रुक्मरथं निपातितम्}
{}


\chapter{अध्यायः ९}
\twolineshloka
{धृतराष्ट्र उवाच}
{}


\twolineshloka
{किं कुर्वाणं रणे द्रोणं जघ्रुः पाण्डवसृञ्जयाः}
{तथा निपुणमस्त्रेषु सर्वशस्त्रभृतामपि}


\twolineshloka
{रथः पर्यपतद्वाऽस्य धनुर्वाऽशीर्यतास्यतः}
{प्रमत्तो वाऽभवद्दोणो यथा मृत्युमुपेयिवान्}


\twolineshloka
{कथं नु पार्षतस्तात शत्रुभिर्दुष्प्रधर्षणम्}
{किरन्तमिषुसङ्घातान्रुक्मपुङ्खाननेकशः}


\twolineshloka
{क्षिप्रहस्तं द्विजश्रेष्ठं कृतिनं चित्रयोधिनम्}
{दूरेषुपातिनं दान्तमस्त्रयुद्धेषु पारगम्}


\twolineshloka
{पाञ्चालपुत्रो न्यवधीद्दिव्यास्त्रधरमच्युतम्}
{कुर्वाणं दारुणं कर्म रणे यत्तं महारथम्}


\twolineshloka
{व्यक्तं हि दैवं बलवत्पौरुषादिति मे मतिः}
{यद्द्रोणो निहतः शूरः पार्षतेन महात्मना}


\twolineshloka
{अस्त्रं चतुर्विधं वीरे यस्मिन्नासीत्प्रतिष्ठितम्}
{तमिष्वस्त्रधराचार्यं द्रोणं शंससि मे हतम्}


\twolineshloka
{श्रुत्वा हतं रुक्मरथं वैयाघ्रपरिवारणम्}
{जातरूपपरिष्कारं नाद्य शोकमपानुदे}


\threelineshloka
{न नूनं परदुःखेन म्रियते कोऽपि सञ्जय}
{यत्र द्रोणमहं श्रुत्वा हतं जीवामि मन्दधीः}
{दैवमेव परं मन्ये नन्वनर्थं हि पौरुषम्}


\twolineshloka
{अश्मसारमयं नूर्न हृदयं सुदृढं मम}
{यच्छ्रुत्वा निहतं द्रोणं शतधा न विदीर्यते}


\twolineshloka
{ब्राह्मे दैवे तथेष्वस्त्रे यमुपासन्गुणार्थिनः}
{ब्राह्मणा राजपुत्राश्च स कथं मृत्युना हृतः}


\twolineshloka
{शोषणं सागरस्येव मेरोरिव विसर्पणम्}
{पतनं भास्करस्येव न मृष्ये द्रोणपातनम्}


\twolineshloka
{दुष्टानां प्रतिषेद्धासीद्धार्मिकाणां च रक्षिता}
{योऽहासीत्कृपणस्यार्थे प्राणानपि परन्तपः}


\twolineshloka
{मन्दानां मम पुत्राणां जयाशा यस्य विक्रमे}
{बृहस्पत्युशनस्तुल्यो बुद्ध्या स निहतः कथम्}


\twolineshloka
{`गुणानां सर्वयोधानां स्थितिरासीन्महाद्युतिः}
{यं मृत्युर्वशगस्तिष्ठेत्स कथं मृत्युना हतः'}


\twolineshloka
{ते च शोणा बृहन्तोऽश्वाश्छन्ना जालैर्हिरण्मयैः}
{रथे वातजवा युक्ताः सर्वशस्त्रातिगा रणे}


\twolineshloka
{बलिनो हेषिणो दान्ताः सैन्धवाः साधुवाहिनः}
{दृढाः सङ्ग्राममध्येषु कच्चिदासन्न विह्वलाः}


\twolineshloka
{करिणां बृंहतां युद्धे शङ्खदुन्दुभिनिःस्वनैः}
{ज्याक्षेपशरवर्षाणां शस्त्राणां च सहिष्णवः}


\twolineshloka
{आशंसन्तः पराञ्जेतुं जितश्वासा जितव्यथाः}
{हयाः पराजिताः शीघ्रा भारद्वाजरथोद्वहाः}


\twolineshloka
{ते स्म रुक्मरथे युक्ता नरवीरसमास्थिताः}
{कथं नाभ्यतरंस्तात पाण्डवानामनीकिनीम्}


\twolineshloka
{जातरूपपरिष्करामास्थाय रथमुत्तमम्}
{भारद्वाजः किमकरोद्युधि सत्यपराक्रमः}


\twolineshloka
{विद्यां यस्योपजीवन्ति सर्वलोकधनुर्धराः}
{स सत्यसन्धो बलवान्द्रोणः किमकरोद्युधि}


\twolineshloka
{दिवि शक्रमिव श्रेष्ठं महामात्रं धनुर्भृताम्}
{के नु तं रौद्रकर्माणं युद्धे प्रत्युद्ययू रथाः}


\twolineshloka
{ननु रुक्मरथं दृष्ट्वा प्राद्रवन्ति स्म पाण्डवाः}
{दिव्यमस्त्रं विकुर्वाणं रणे तस्मिन्महाबलम्}


\twolineshloka
{उताहो सर्वसैन्येन धर्मराजः सहानुजः}
{पाञ्चालप्रग्रहो द्रोणं सर्वतः समवारयत्}


\twolineshloka
{नूनमावारयत्पार्थो रथिनोऽन्यानजिह्मगैः}
{ततो द्रोणं समहरत्पार्षतः पापकर्मकृत्}


\twolineshloka
{नह्यहं परिपश्यामि वधे कञ्चन शुष्मिणः}
{धृष्टद्युम्नादृते रौद्रात्पाल्यमानात्किरीटिना}


\twolineshloka
{`उताहो सर्वसैन्येन धर्मराजः सहानुजः}
{उत्सृज्य सर्वसैन्यानि द्रोणं तमभिदुद्रुवे'}


\twolineshloka
{तैर्वृतः सर्वतः क्षुद्रैः पाञ्चालापशदैस्ततः}
{केकयैश्चेदिकारूशैर्मत्स्यैरन्यैश्च भूमिपैः}


\twolineshloka
{व्याकुलीकृतमाचार्यं पिपीलैरुरगं यथा}
{कर्मण्यसुकरे सक्तं जघानेति मतिर्मम}


\twolineshloka
{योऽधीत्य चतुरो वेदान्साङ्गानाख्यानपञ्चमान्}
{ब्राह्मणानां प्रतिष्ठासीत्स्रोतसामिव सागरः}


\twolineshloka
{क्षत्रं च ब्रह्म चैवेह योऽभ्यतिष्ठत्परन्तपः}
{`दृप्तानां प्रतिषेद्धा च चक्षुरासीदचक्षुषाम्}


\twolineshloka
{अमर्षी चावलिप्तेषु धार्मिकेषु च धार्मिकः'}
{स कथं ब्राह्मणो वृद्धः शस्त्रेण वधमाप्तवान्}


\twolineshloka
{अमर्षिणा मर्षितवान्क्लिश्यमानान्सदा मया}
{अनर्हमाणान्कौन्तेयान्कर्मणस्तस्य तत्फलम्}


\twolineshloka
{यस्य कर्मानुजीवन्ति लोके सर्वधनुर्भृतः}
{स सत्यसन्धः सुकृती श्रीकामैर्निहतः कथम्}


\twolineshloka
{दिवि शक्र इव श्रेष्ठो महासत्वो महाबलः}
{स कथं निहतः पार्थैः क्षुद्रमत्स्यैर्यथा तिमिः}


\twolineshloka
{क्षिप्रहस्तश्च बलवान्दृढघन्वारिमर्दनः}
{न यस्य विजयाकाङ्क्षी विषयं प्राप्य जीवति}


\twolineshloka
{यं द्वौ न जहतः शब्दौ जीवमानं कदाचन}
{ब्राह्मश्च वेदकामानां ज्याघोषश्च धनुष्मताम्}


\twolineshloka
{अदीनं पुरुषव्याघ्रं हीमन्तमपराजितम्}
{तमिष्वासवराचार्ये द्रोणं जघ्नुः कथं रथाः}


\twolineshloka
{`नाहं मन्ये हतं द्रोणं स हि लोकमयोत्स्यत}
{को हि शक्तो रणे जेतुमनाधृष्ययशोबलम्}


\twolineshloka
{ब्रह्मकल्पो भवेद्ब्राह्मे क्षात्रे नारायणोपमः}
{ब्रह्मक्षत्रे च यस्यास्तां वशे स्थाणोरिवाखिले}


\twolineshloka
{सर्वान्हि मामकान्वीरान्सहाश्वरथकुञ्जरान्}
{युधिष्ठिरस्य तपसा हतान्मन्यामहे कुरून्'}


\twolineshloka
{कथं सञ्जय दुर्धर्षमनाधृष्ययशोबलम्}
{पश्यतां पुरुषेन्द्राणां समरे पार्षतोऽवधीत्}


\twolineshloka
{के पुरस्तादयुध्यन्त रक्षन्तो द्रोणमन्तिकात्}
{के नु पश्चादवर्तन्त गच्छतो दुर्गमां गतिम्}


\twolineshloka
{केऽरक्षन्दक्षिणं चक्रं सव्यं के च महात्मनः}
{पुरस्तात्के च वीरस्य युध्यमानस्य संयुगे}


\twolineshloka
{के च तस्मिंस्तनूस्त्यक्त्वा प्रतीपं मृत्युमाव्रजन्}
{द्रोणस्य समरे वीराः केऽकुर्वन्त परां धृतिम्}


\twolineshloka
{कच्चिन्नैनं भयान्मन्दाः क्षत्रिया व्यजहन्रणे}
{रक्षितारस्ततः शून्ये कच्चित्तैर्न हतः परैः}


\twolineshloka
{न स पृष्ठमरेस्त्रासाद्रणे शौर्यात्प्रदर्शयेत्}
{परामप्यापदं प्राप्य स कथं निहतः परैः}


\threelineshloka
{एतदार्येण कर्तव्यं कृच्छ्रास्वापत्सु सञ्जय}
{पराक्रमेद्यथा शक्त्या तच्च तस्मिन्प्रतिष्ठितम्}
{`यो यथाशक्ति युद्ध्येत द्विषद्भिःस्वांश्च पालयन्}


\twolineshloka
{कच्चिन्नैनं भयात्क्षुद्राः पार्थेभ्यः प्रददू रणे}
{गोप्तृभिस्तैः समुत्सृष्टः कच्चिन्नैष परैर्हतः}


\twolineshloka
{धृष्टद्युम्नं प्रपश्यामि निघ्नन्तमिव ब्राह्मणम्}
{वार्यमाणं रणे तात द्रौणिनाऽमिततेजसा'}


\twolineshloka
{मुह्यते मे मनस्तात कथा तावन्निवार्यताम्}
{भूयस्तु लब्धसंज्ञस्त्वां परिपृच्छामि सञ्जय}


\chapter{अध्यायः १०}
\twolineshloka
{वैशम्पायन उवाच}
{}


\twolineshloka
{एवं पृष्ट्वा सूतपुत्रं हृच्छोकेनार्दितो भृशम्}
{जये निराशः पुत्राणां धृतराष्ट्रोऽपतत्क्षितौ}


\twolineshloka
{तं विसंज्ञं निपतिन्त सिषिचुः परिचारिकाः}
{जलेनात्यर्थशीतेन जीवन्त्यः पुण्यगन्धिना}


\twolineshloka
{पतितं चैनमालोक्य समस्ता भरतस्त्रियः}
{परिवव्रुर्महाराजमस्पृशंश्चैव पाणिभिः}


\twolineshloka
{उत्थाप्य चैनं शनकै राजानं पृथिवीतलात्}
{आसनं प्रापयामासुर्बाष्पकण्ठ्यो वराननाः}


\twolineshloka
{आसनं प्राप्य राजा तु मूर्च्छयाऽभिपरिप्लुतः}
{निश्चेष्टोऽतिष्ठत तदा वीज्यमानः समन्ततः}


\threelineshloka
{स लब्ध्वा शनकैः संज्ञां वेपमानो महीपतिः}
{पुनर्गावल्गणिं सूतं पर्यपृच्छद्यथातथम् ॥धृतराष्ट्र उवाच}
{}


\twolineshloka
{यः स उद्यन्निवादित्यो ज्योतिषा प्रणुदंस्तमः}
{अजातशत्रुमायान्तं कस्तं द्रोणादवारयत्}


\twolineshloka
{प्रभिन्नमिव मातङ्गं यथा क्रुद्धं तरस्विनम्}
{प्रसन्नवदनं दृष्ट्वा प्रतिद्विरदगामिनम्}


\twolineshloka
{वासितासङ्गमे यद्वदजय्यं प्रतियूथपैः}
{निजघान रणे वीरान्वीरः पुरुषसत्तमः}


\twolineshloka
{यो ह्येको हि महावीर्यो निर्दहेद्धोरचक्षुषा}
{कृत्स्नं दुर्योधनबलं धृतिमान्सत्यसङ्गरः}


\twolineshloka
{चक्षुर्हणं जये सक्तमिष्वासधरमच्युतम्}
{दान्तं बहुमतं लोके के शूराः पर्यवारयन्}


\twolineshloka
{के दुष्प्रधर्षं राजानमिष्वासधरमच्युतम्}
{समासेदुर्नरव्याघ्रं कौन्तेयं तत्र मामकाः}


\twolineshloka
{तरसैवाभिपद्याथ यो वै द्रोणमुपाद्रवत्}
{यः करोति महत्कर्म शत्रूणां वै महाबलः}


\twolineshloka
{महाकायो महोत्साहो नागायुतसमो बले}
{तं भीमसेनमायान्तं के शूराः पर्यवारयन्}


\twolineshloka
{यदायाज्जलदप्रख्यो रथः परमवीर्यवान्}
{पर्जन्य इव बीभत्सुस्तुमुलामशनीं सृजन्}


\twolineshloka
{विसृजञ्छरजालानि वर्षाणि मघवानिव}
{अवस्फूर्जन्दिशः सर्वास्तलनेमिस्वनेन च}


\twolineshloka
{चापविद्युत्प्रभो घोरो रथगुल्मवलाहकः}
{सनेमिघोषस्तनितः शरशब्दातिबन्धुरः}


\twolineshloka
{रोषनिर्जितजीमूतो मनोभिप्रायशीघ्रगः}
{मर्मातिगो बाणधरस्तुमुलः शोणितोदकैः}


\twolineshloka
{सम्प्लावयन्दिशः सर्वा मानवैरास्तरन्महीम्}
{भीमनिःस्वनितो रौद्रो दुर्योधनपुरोगमान्}


\twolineshloka
{युद्धेऽभ्यषिञ्चद्विजयो गार्ध्रपत्रैः शिलाशितैः}
{गाण्डीवं धारयन्धीमान्कीदृशं वो मनस्तदा}


\twolineshloka
{इषुसम्बाधमाकाशं कुर्वन्कपिवरध्वजः}
{यदायात्कथमासीत्तु तदा पार्थं समीक्षताम्}


\twolineshloka
{कच्चिद्गाण्डीवशब्देन न प्रणश्यति वै बलम्}
{यद्वः सभैरवं कुर्वन्नर्जुनो भृशमन्वयात्}


\twolineshloka
{कच्चिन्नापानुदत्प्रामानिषुभिर्वो धनञ्जयः}
{वातो वेगादिवाविध्यन्मेघाञ्शरगणैर्नृपान्}


\twolineshloka
{को हि गाण्डीवधन्वानं रणे सोढुं नरोऽर्हति}
{यमुषश्चुत्य सेनाग्रे जनः सर्वो विदीर्यते}


\twolineshloka
{यत्सेनाः समकम्पन्त यद्वीरानस्पृशद्भयम्}
{के तत्र नाजहुर्द्रोणं के क्षुद्राः प्राद्रवन्भयात्}


\twolineshloka
{के वा तत्र तनूस्त्यक्त्वा प्रतीपं मृत्युमाव्रजन्}
{अमानुषाणां जेतारं युद्धेष्वपि धनञ्जयम्}


\twolineshloka
{न च वेगं सिताश्वस्य विसहिष्यन्ति मामकाः}
{गाण्डीवस्य च निर्घोषं प्रावृड््जलदनिःस्वनम्}


\twolineshloka
{विष्वक्सेनो यस्य यन्ता यस्य योद्धा धनञ्चयः}
{अशक्यः स रथो जेतुं मन्ये देवासुरैरपि}


\twolineshloka
{सुकुमारो युवा शूरो दर्शनीयश्च पाण्डवः}
{मेधावी निपुणो धीमान्युधि सत्यपराक्रमः}


\twolineshloka
{आरावं विपुलं कुर्वन्व्यथयन्सर्वसैनिकान्}
{यदायान्नकुलो द्रोणं के शूराः पर्यवारयन्}


\twolineshloka
{आशीविष इव क्रुद्धः सहदेवो यदाऽभ्ययात्}
{कदनं करिष्यञ्शत्रूणां तेजसा दुर्जयो युधि}


\twolineshloka
{आर्यव्रतममोघेषुं हीमन्तमपराजितम्}
{सहदेवं तमायान्तं के शूराः पर्यवारयन्}


\twolineshloka
{यस्तु सौवीरराजस्य प्रमथ्य महतीं चमूम्}
{आदत्त महिषीं भोजां काम्यां सर्वाङ्गशोभनाम्}


\twolineshloka
{सत्यं धृतिश्च शौर्यं च ब्रह्मचर्यं च केवलम्}
{सर्वाणि युयुधानेऽस्मिन्नित्यानि पुरुषर्षभे}


\twolineshloka
{बलिनं सत्यकर्माणमदीनमपराजितम्}
{वासुदेवसमं युद्धे वासुदेवादनन्तरम्}


\twolineshloka
{धनञ्जयोपदेशेन श्रेष्ठमिष्वस्त्रकर्मणि}
{पार्थेन सममस्त्रेषु कस्तं द्रोणादवारयत्}


\twolineshloka
{वृष्णीनां प्रवरं वीरं शूरं सर्वधनुष्मताम्}
{रामेण सममस्त्रेषु यशसा विक्रमेण च}


\twolineshloka
{सत्यं धृतिर्मतिः शौर्यं ब्राह्मं चास्त्रमनुत्तमम्}
{सात्वते तानि सर्वाणि त्रैलोक्यमिव केशवे}


\twolineshloka
{तमेवङ्गुणसम्पन्नं दुर्वारमपि दैवतैः}
{समासाद्य महेष्वासं के शूराः पर्यवारयन्}


\twolineshloka
{पाञ्चालेषूत्तमं वीरमुत्तमाभिजनप्रियम्}
{नित्यमुत्तमकर्माणमुत्तमौजसमाहवे}


\twolineshloka
{युक्तं धनञ्जयहिते ममानर्थार्थमुत्थितम्}
{यमवैश्रवणादित्यमहेन्द्रवरुणोपमम्}


\twolineshloka
{महारथं समाख्यातं द्रोणायोद्यतमाहवे}
{त्यजन्तं तुमुले प्राणान्के शूराः समवारयन्}


\twolineshloka
{एकोपसृत्य चेदिभ्यः पाण्डवान्यः समाश्रितः}
{धृष्टकेतुं समायान्तं द्रोणं कस्तं न्यवारयत्}


\twolineshloka
{योऽवधीत्केतुमान्वीरो राजपुत्रं दुरासदम्}
{अपरान्तगिरिद्वारे द्रोणात्कस्तं न्यवारयत्}


\twolineshloka
{स्त्रीम्पुसयोर्नरव्याघ्रो यः स वेद गुणागुणान्}
{शिखण्डिनं याज्ञसेनिमम्लानमनसं युधि}


\twolineshloka
{देवव्रतस्य समरे हेतुं मृत्योर्महात्मनः}
{द्रोणायाभिमुखं यान्तं के शूराः पर्यवारयन्}


\twolineshloka
{यस्मिन्नभ्यधिका वीरे गुणाः सर्वे धनञ्जयात्}
{यस्मिन्नस्त्राणि सत्यं च ब्रह्मचर्यं च सर्वदा}


\twolineshloka
{वासुदेवसमं वीर्ये धनञ्जयसमं बले}
{तेजसाऽऽदित्यसदृशं बृहस्पतिसमं मतौ}


\twolineshloka
{अभिमन्युं महात्मानं व्यात्ताननमिवान्तकम्}
{द्रोणायाभिमुखं यान्तं के शूराः समवारयन्}


\twolineshloka
{तरुणस्तरुणप्रज्ञः सौभद्रः परवीरहा}
{यदाभ्यधावद्वै द्रोणं तदाऽऽसीद्वो मनः कथं}


\twolineshloka
{द्रौपदेया नरव्याघ्राः समुद्रमिव सिन्धवः}
{यद्रोणमाद्रवन्सङ्ख्ये के शूरास्तान्न्यवारयन्}


\twolineshloka
{एते द्वादश वर्षाणि क्रीडामुत्सृज्य बालकाः}
{अस्त्रार्थमवसन्भीष्मे बिभ्रतो व्रतमुत्तमम्}


\twolineshloka
{क्षत्रञ्जयः क्षत्रदेवः क्षत्रवर्मा च मानदः}
{धृष्टद्युम्नात्मजा वीराः के तान्द्रोणादवारयन्}


\twolineshloka
{शताद्विशिष्टं यं युद्धे सममन्यन्त वृष्णयः}
{चेकितानं महेष्वासं कस्तं द्रोणादवारयत्}


\twolineshloka
{वार्धक्षेमिः कलिङ्गानां यः कन्यामाहरद्युधि}
{अनाधृष्टिरदीनात्मा कस्तं द्रोणादवारयत्}


\twolineshloka
{भ्रातरः पञ्च कैकेया धार्मिकाः सत्यविक्रमाः}
{इन्द्रगोपकसङ्काशा रक्तवर्मायुधध्वजाः}


\twolineshloka
{मातृष्वसुः सुता वीराः पाण्डवानां जयार्थिनः}
{तान्द्रोणं हन्तुमायातान्के वीराः पर्यवारयन्}


\twolineshloka
{यं योधयन्तो राजानो नाजयन्वारणावते}
{षण्मासानपि संरब्धा जिघांसन्तो युधां पतिं}


\twolineshloka
{धनुष्मतां वरं शूरं सत्यसन्धं महाबलम्}
{द्रोणात्कस्तं नरव्याघ्रं युयुत्सुं पर्यवारयत्}


\twolineshloka
{यः पुत्रं काशिराजस्य वाराणस्यां महारथम्}
{समरे स्त्रीषु गृध्यन्तं भल्लेनापाहरद्रथात्}


\twolineshloka
{धृष्टद्युम्नं महेष्वासं पार्थानां मन्त्रधारिणम्}
{युक्तं दुर्योधनानर्थे सृष्टं द्रोणवधाय च}


\twolineshloka
{निर्दहन्तं रणे योधान्दारयन्तं च सर्वतः}
{द्रोणाभिमुखमायान्तं के शूराः पर्यवारयन्}


\twolineshloka
{उत्सङ्ग इव सम्वृद्धं द्रुपदस्यास्त्रवित्तमम्}
{शैखण्डिनं शस्त्रगुप्तं के द्रोणादवारयन्}


\twolineshloka
{य इमां पृथिवीं कृत्स्नां चर्मवत्समवेष्टयत्}
{महता रथघोषेण मुख्यारिघ्नो महारथः}


\twolineshloka
{दशाश्वमेधानाजहे स्वन्नपानाप्तदक्षिणान्}
{निरर्गलान्सर्वमेधान्पुत्रवत्पालयन्प्रजाः}


\threelineshloka
{विसृष्टा दक्षिणा यस्य गङ्गाजलमवारयत्}
{गङ्गास्रोतसि यावत्यः सिकता अप्यशेषतः}
{तावतीर्गा ददौ वीर उशीनरसुतोऽध्वरे}


\twolineshloka
{न पूर्वे नापरे चक्रुरिदं केचन मानवाः}
{इतीदं चुक्रुशुर्देवाः कृते कर्मणि दुष्करे}


\twolineshloka
{पश्यामस्त्रिषु लोकेषु न तं संस्थास्नुचारिषु}
{जातं चापि जनिष्यन्तं द्वितीयं चापि साम्प्रतम्}


\twolineshloka
{अन्यमौशीनराच्छेब्याद्धुरो वोढारमित्युत}
{गतिं यस्य न यास्यन्ति मानुषा लोकवासिनः}


\twolineshloka
{तस्य नप्तारमायान्तं शैब्यं कः समवारयत्}
{द्रोणायाभिमुखं यत्तं व्यात्ताननमिवान्तकम्}


\twolineshloka
{विराटस्य रथानीकं मत्स्यस्यामित्रघातिनः}
{प्रेप्सन्तं समरे द्रोणं के वीराः पर्यवारयन्}


\twolineshloka
{सद्यो वृकोदराज्जातो महाबलपराक्रमः}
{मायावी राक्षसो वीरो यस्मान्मम महद्भयम्}


\twolineshloka
{पार्थानां जयकामं तं तं पुत्राणां मम कण्टकम्}
{घटोत्कचं महात्मानं कस्तं द्रोणादवारयत्}


\twolineshloka
{एते चान्ये च बहवो येषामजितं युधि}
{त्यक्तारः संयुगे प्राणान्किं तेषामजितं युधि}


\twolineshloka
{येषां च पुरुषव्याघ्रः शार्ङ्गधन्वा व्यपाश्रयः ॥लोकानां गुरुरत्यर्थं लोकनाथः सनातनः}
{}


\twolineshloka
{नारायणोरणे नाथो दिव्यो दिव्यात्मकः प्रभुः ॥यस्य दिव्यानि कर्माणि प्रवदन्ति मनीषिणः}
{}


% Check verse!
तान्यहं कीर्तयिष्यामि भक्त्या स्थैर्यार्थमात्मनः
\chapter{अध्यायः ११}
\twolineshloka
{धृतराष्ट्र उवाच}
{}


\twolineshloka
{शृणु दिव्यानि कर्माणि वासुदेवस्य सञ्जय}
{कृतवान्यानि गोविन्दो यथा नान्याःपुमान्क्वचित्}


\twolineshloka
{संवर्धता गोपकुले बालेनैव महात्मना}
{विख्यापितं बलं बाह्वोस्त्रिषु लोकेषु सञ्जय}


\twolineshloka
{पूतनां शकटं हत्वा केशिनं चैव वाजिनम्}
{ऋषभं धेनुकं चैव अरिष्टं च महाबलम्}


\twolineshloka
{दावान्मुक्त्वा महाबाहुर्धृत्वा गोवर्धनं निगिम्}
{चाणूरं मुष्टिकं चैव रङ्गमध्ये निहत्य च}


\twolineshloka
{प्रलम्बं नरकं जम्भं पीठं चापि महासुरम्}
{मुरुं चान्तकसङ्काशमवधीत्पुष्करेक्षणः}


\twolineshloka
{तथा कंसो महातेजा जरासन्धेन पालितः}
{विक्रमेणैव कृष्णेन सगणः पातितो रणे}


\twolineshloka
{सुनामा रणविक्रान्तः समग्राक्षौहिणीपतिः}
{भोजराजस्य मध्यस्थो भ्राता कंसस्य वीर्यवान्}


\twolineshloka
{बलदेवद्वितीयेन कृष्णेनामित्रघातिना}
{तरस्वी समरे दग्धः ससैन्यः शूरसेनराट्}


\twolineshloka
{दुर्वासा नाम विप्रर्षिस्तथा परमकोपनः}
{आराधितः सदारेण स चास्मै प्रददौ वरान्}


\twolineshloka
{तथा गान्धारराजस्य सुतां वीरः स्वयंवरे}
{निर्जित्य पृथिवीपालानावहत्पुष्करेक्षणः}


\twolineshloka
{अमृष्यमाणा राजानो यस्य जात्या हया इव}
{रथे वैवाहिके युक्ताः प्रतोदेन कृतव्रणाः}


\twolineshloka
{जरासन्धं महाबाहुमुपायेन जनार्दनः}
{परेण घातयामास समग्राक्षौहिणीपतिम्}


\twolineshloka
{चेदिराजं च विक्रान्तं राजसेनापतिं बली}
{अर्घे विवदमानं च जघान पशुवत्तदा}


\twolineshloka
{सौभं दैत्यपुरं खस्थं शाल्वगुप्तं दुरासदम्}
{समुद्रकुक्षौ विक्रम्य पातयामास माधवः}


\twolineshloka
{अङ्गान्वङ्गान्कलिङ्गंश्च मागधान्काशिकोसलान्}
{वात्स्यगार्ग्यकरूषांश्च पौण्ड्रांश्चाप्यजयद्रणे}


\twolineshloka
{आवन्त्यान्दाक्षिणात्यांश्च पार्वतीयान्दशेरकान्}
{काश्मीरकानौरसिकान्पिशाचांश्च समुद्गलान्}


\twolineshloka
{काम्भोजान्वाटधानांश्च चोलान्पाण्ड्यांश्च सञ्जय}
{त्रिगर्तान्मालवांश्चैव दरदांश्च सुदुर्जयान्}


\twolineshloka
{नानादिग्भ्यश्च सम्प्राप्तान्खशांश्चैव शकांस्तथा}
{जितवान्पुण्डरीकाक्षो यवनं च सहानुगम्}


\twolineshloka
{प्रविश्य मकरावासं यादोगणनिषेवितम्}
{जिगाय वरुणं सङ्ख्ये सलिलान्तर्गतं पुरा}


\twolineshloka
{युधि पञ्चजनं हत्वा दैत्यं पातालवासिनम्}
{पाञ्चजन्यं हृषीकेशो दिव्यं शङ्खमवाप्तवान्}


\twolineshloka
{खाण्डवे पार्थसहितस्तोषयित्वा हुताशनम्}
{आग्नेयमस्त्रं दुर्धर्षं चक्रं लेभे महाबलः}


\twolineshloka
{वैनतेयं समारुह्य त्रासयित्वाऽमरावतीम्}
{महेन्द्रभवनाद्वीरः पारिजातमुपानयत्}


\twolineshloka
{तच्च मर्षितवाञ्शक्रो जानंस्तस्य पराक्रमम्}
{राज्ञां चाप्यजितं कंचित्कृष्णेनेह न शुश्रुम}


\twolineshloka
{यच्च तन्महदाश्चर्यं सभायां मम सञ्जय}
{कृतवान्पुण्डरीकाक्षः कस्तदन्य इहार्हति}


\twolineshloka
{यच्च भक्त्या प्रसन्नोऽहमद्राक्षं कृष्णमीश्वरम्}
{तन्मे सुविदितं सर्वं प्रत्यक्षमिव चागमम्}


\twolineshloka
{नान्तो विक्रमयुक्तस्य बुद्ध्या युक्तस्य वा पुनः}
{कर्मणां शक्यते गन्तुं हृषीकेशस्य सञ्जय}


\twolineshloka
{तथा गदश्च साम्बश्च प्रद्युम्नोऽथ विदूरथः}
{अगावहोऽनिरुद्धश्च चारुदेष्णः ससारणः}


\twolineshloka
{उल्मुको निशठश्चैव झिल्ली बभ्रुश्च वीर्यवान्}
{पृथुश्च विपृथुश्चैव शमीङ्कोऽथारिमेजयः}


\twolineshloka
{एतेऽन्ये बलवन्तश्च वृष्णिवीराः प्रहारिणः}
{कथंचित्पाण्जवानीकं श्रयेयुः समरे स्थिताः}


\twolineshloka
{आहूता वृष्णिवीरेण केशवेन महात्मना}
{ततः संशयितं सर्वे भवेदिति मतिर्मम}


\twolineshloka
{नागायुतबलो वीरः कैलासशिखरोपमः}
{वनमाली हली रामस्तत्र यत्र जनार्दनः}


\twolineshloka
{यमाहुः सर्वपितरं वासुदेवं द्विजातयः}
{अपि वा ह्येष पाण्डूनां योत्स्यतेऽर्थाय सञ्जय}


\twolineshloka
{स यदा तात सन्नह्येत्पाण्डवार्थाय सञ्जय}
{न तदा प्रतिसंयोद्धा भविता तत्र कश्चन}


\twolineshloka
{यदि स्म कुरवः सर्वे जयेयुर्नाम पाण्डवान्}
{वार्ष्णेयोऽर्थाय तेषां वै गृह्णीयाच्छस्त्रमुत्तमम्}


\twolineshloka
{ततः सर्वान्नरव्याघ्रो हत्वा नरपतीन्रणे}
{कौरवांश्च महाबाहुःकुन्त्यै दद्यात्स मेदिनीम्}


\twolineshloka
{यस्य यन्ता हृषीकेशो योद्धा यस्य धनञ्जयः}
{रथस्य तस्य कः सङ्ख्ये प्रत्यनीको भवेद्रथः}


\twolineshloka
{न केनचिदुपायेन कुरूणां दृश्यते जयः}
{तस्मान्मे सर्वमाचक्ष्व यथा युद्धमवर्तत}


\twolineshloka
{अर्जुनः केशवस्यात्मा कृष्णोऽप्यात्मा किरीटिनः}
{अर्जुने विजयो नित्यं कृष्णे कीर्तिश्च शाश्वती}


\twolineshloka
{सर्वेष्वपि च लोकेषु बीभत्सुरपराजितः}
{प्राधान्येनैव भूयिष्ठममेयाः केशवे गुणाः}


\threelineshloka
{मोहाद्दुर्योधनः कृष्णं यो न वेत्तीह केशवम्}
{मोहितो दैवयोगेन मृत्युपाशपुरस्कृतः}
{}


\twolineshloka
{न वेद कृष्णं दाशार्हमर्जुनं चैव पाण्डवम्}
{पूर्वदेवौ महात्मानौ नरनारायणावुभौ}


\twolineshloka
{एकात्मानौ द्विधाभूतौ दृश्येते मानवैर्भुवि}
{मनसापि हि दुर्धर्षौ सेनामेतां यशस्विनौ}


\twolineshloka
{नाशयेतामिहेच्छन्तौ मानुषत्वाच्च नेच्छतः}
{युगस्येव विपर्यासो लोकानामिव मोहनम्}


\twolineshloka
{भीष्मस्य च वधस्तात द्रोणस्य च महात्मनः}
{न ह्येव ब्रह्मचर्येण न वेदाध्ययनेन च}


\twolineshloka
{न क्रियाभिर्न चास्त्रेण मृत्योः कश्चिन्निर्वायते}
{लोकसम्भावितौ वीरौ कृतास्त्रौ युद्धदुर्मदौ}


\twolineshloka
{भीष्मद्रोणौ हतौ श्रुत्वा किन्नु जीवामि सञ्जय}
{यां तां श्रियमसूयामः पुरा दृष्ट्वा युधिष्ठिरे}


\twolineshloka
{अद्य तामनुजानीमो भीष्मद्रोणवधेन ह}
{मत्कृते चाप्यनुप्राप्तः कुरूणामेष सङ्क्षयः}


\twolineshloka
{पक्वानां हि वधे सूत वज्रायन्ते तृणान्युत}
{अनन्तमिदमैश्वर्यं लोके प्राप्तो युधिष्ठिरः}


\twolineshloka
{यस्य कोपान्महात्मानौ भीष्मद्रोणौ निपातितौ}
{प्राप्तः प्रकृतितो धर्मो न धर्मो मामकान्प्रति}


\threelineshloka
{क्रूरः सर्वविनाशाय कालोऽसौ नातिवर्तते}
{अन्यथा चिन्तिता ह्यर्था नरैस्तात मनस्विभिः}
{अन्यथैव प्रपद्यन्ते दैवादिति मतिर्मम}


\twolineshloka
{तस्मादपरिहार्येऽर्थे सम्प्राप्ते कृच्छ्र उत्तमे}
{अपारणीये दुश्चिन्त्ये यथाभूतं प्रचक्ष्व मे}


\chapter{अध्यायः १२}
\twolineshloka
{सञ्जय उवाच}
{}


\threelineshloka
{`शुश्रूषस्व स्थिरो भूत्वां शंसतो मम भारत'}
{हन्त ते कथयिष्यामि सर्वं प्रत्यक्षदर्शिवान्}
{यथा सत्यपरो द्रोणः सादिताः पाण्डुसृञ्जयैः}


\twolineshloka
{सेनापतित्वं सम्प्राप्य भारद्वाजो महारथः}
{मध्ये सर्वस्य सैन्यस्य पुत्रं ते वाक्यमब्रवीत्}


\twolineshloka
{यत्कौरवाणामृषभादापगेयादनन्तरम्}
{सैनापत्ये त्वया राजन्नद्य सत्कृतवानहम्}


\threelineshloka
{सदृशं कर्मणस्तस्य फलं प्राप्नुहि भारत}
{करोमि कामं कं तेऽद्य प्रवृणीष्व यमिच्छसि ॥सञ्जय उवाच}
{}


\twolineshloka
{ततो दुर्योधनो राजा कर्णदुःशासनादिभिः}
{सम्मन्त्र्योवाच दुर्धर्षमाचार्यं जयतां वरम्}


\twolineshloka
{ददासि चेद्वरं मह्यं जीवग्राहं युधिष्ठिरम्}
{गृहीत्वा रयिनां श्रेष्ठं मत्समीपमिहानय}


\twolineshloka
{`इच्छेयं वै महात्मानं धर्मात्मानं युधिष्ठिरम्}
{भ्रातृणां पश्यतामेव जीवग्राहेण मे द्विज}


\threelineshloka
{गृहीत्वा रथिनां श्रेष्ठं मत्सकाशमिहानय}
{प्रीणाम्यनेन कार्येण ससुहृज्जनबान्धवः ॥सञ्जय उवाच}
{}


\twolineshloka
{ततः कुरूणामाचार्यः श्रुत्वा पुत्रस्य ते वचः}
{सेनां प्रहर्षयन्सर्वामिदं वचनमब्रवीत्}


\twolineshloka
{धन्यः कुन्तीसुतो राजन्यस्य ग्रहणमिच्छसि}
{न वधं तस्य दुर्धर्ष वरमद्य प्रयाचसे}


\twolineshloka
{किमर्थं च नरव्याघ्र न वधं तस्य काङ्क्षसे}
{नाशंससि क्रियामेतां मत्तो दुर्योधन ध्रुवम्}


\twolineshloka
{आहोस्विद्धर्मराजस्य द्वेष्टा कश्चिन्न विद्यते}
{यदीच्छसि त्वं जीवन्तं कुलं रक्षसि चात्मनः}


\twolineshloka
{अथवा भरतश्रेष्ठ निर्जित्य युधि पाण्डवान्}
{राज्यांशं प्रति दत्त्वा च सौभ्रात्रं कर्तुमिच्छसि}


\threelineshloka
{धन्यः कुन्तीसुतो राजा सुजातं चास्य धीमतः}
{अजातशत्रुता सत्या तस्य यत्स्निह्यते भवान् ॥सञ्जय उवाच}
{}


\twolineshloka
{द्रोणेन चैवमुक्तस्य तव पुत्रस्य भारत}
{सहसा निःसृतो भावो योस्य नित्यं हृदि स्थितः}


\twolineshloka
{नाकारो गूहितुं शक्यो बृहस्पतिसमैरपि}
{तस्मात्तव सुतो राजन्प्रहृष्टो वाक्यमब्रवीत्}


\twolineshloka
{वधे कुन्तीसुतस्याजौ नाचार्य विजयो मम}
{हते युधिष्ठिरे पार्था हन्युःसर्वान्हि नो ध्रुवम्}


\twolineshloka
{नैव शक्या रणे सर्वे निहन्तुममरैरपि}
{`यदि सर्वे हनिष्यन्ते पाण्डवाः ससुता मृधे}


\twolineshloka
{ततः कृत्स्नं वशे कृत्वा निःशेषं नृपमण्डलम्}
{ससागरवनां स्फीतां विजित्य वसुधामिमाम्}


\twolineshloka
{विष्णुर्दास्यति कृष्णायै कुन्त्यै वा पुरुषोत्तमः'}
{यद्येष चैषां शेषःस्यात्स एवास्मान्न शेषयेत्}


\threelineshloka
{सत्यप्रतिज्ञे त्वानीते पुनर्द्यूतेन निर्जिते}
{`तस्मिञ्जीवति चायाते ह्युपायैर्बहुभिः कृतैः'}
{पुनर्यास्यन्त्यरण्याय पाण्डवास्तमनुव्रताः}


\threelineshloka
{सोऽयं मम जयो व्यक्तं दीर्घकालं भविष्यति}
{अतो न वधमिच्छामि धर्मराजस्य कर्हिचित् ॥सञ्जय उवाच}
{}


\threelineshloka
{तस्य जिह्ममभिप्रायं ज्ञात्वा द्रोणोऽर्थतत्त्ववित्}
{तं वरं सान्तरं तस्मै ददौ संचिन्त्य बुद्धिमान् ॥द्रोण उवाच}
{}


\twolineshloka
{न चेद्युधिष्ठिरं वीरः पालयेदर्जुनो युधि}
{मन्यस्व पाण्डवश्रेष्ठमानीतं वशमात्मनः}


\twolineshloka
{न हि शक्यो रणे पार्थः सेन्द्रैर्देवासुरैरपि}
{प्रत्युद्यातुमतस्तात न तस्माद्धर्षयाम्यहम्}


\twolineshloka
{असंशयं स मे शिष्यो मत्पूर्वश्चास्त्रकर्मणि}
{तरुणः सुकृती युक्त एकायनगतश्च सः}


\twolineshloka
{अस्त्राणीन्द्राच्च रुद्राच्च भूयांसि समवाप्तवान्}
{अमर्षितश्च ते राजंस्ततो नामर्षयाम्यहम्}


\twolineshloka
{स चापक्रम्यतां युद्धाद्येनोपायेन शक्यते}
{अपनीते ततः पार्थे धर्मराजो जितस्त्वया}


\threelineshloka
{ग्रहणे हि जयस्तस्य न वधे पुरुषर्षभ}
{`ग्रहणं हि वधात्तस्य धर्मराजस्य मन्यसे'}
{अनेनैवाभ्युपायेन ग्रहणं समुपैष्यसि}


\twolineshloka
{अहं गृहीत्वा राजानं सत्यधर्मपरायणम्}
{आनयिष्यामि ते राजन्वशमद्य न संशयः}


\twolineshloka
{यदि स्यास्यति सङ्ग्रामे मुहूर्तमपि मेऽग्रतः}
{अपनीते नरव्याघ्रे कुन्तीपुत्रे धनञ्जये}


\threelineshloka
{फल्गुनस्य समीपे तु न हि शक्यो युधिष्ठिरः}
{ग्रहीतुं समरे राजन्सेन्द्रैरपि सुरासुरैः ॥सञ्जय उवाच}
{}


\threelineshloka
{`एवमुक्ते तदा तस्मिन्युद्धे देवासुरोपमे'}
{सान्तरं तु प्रतिज्ञाते राज्ञो द्रोणेन निग्रहे}
{गृहीतं तममन्यन्त तव पुत्राः सुबालिशाः}


\threelineshloka
{ततो दुर्योधनेनापि ग्रहणं पाण्डवस्य तत्}
{`स्कन्धावारेषु सर्वेषु यथास्थानेषु मारिष'}
{सैन्यस्थानेषु सर्वेषु सुघोषितमरिन्दम}


\twolineshloka
{पाण्डवेयेषु सापेक्षं द्रोणं जानाति ते सुतः}
{ततः प्रतिज्ञास्थैर्यार्थं स मन्त्रो बहुलीकृतः}


\chapter{अध्यायः १३}
\twolineshloka
{सञ्जय उवाच}
{}


\threelineshloka
{सान्तरे तु प्रतिज्ञाने राज्ञो द्रोणेन निग्रहे}
{ततस्ते सैनिकाः श्रुत्वा तं युधिष्ठिरनिग्रहम्}
{सिंहनादरवांश्चक्रुर्बाहुशब्दांश्च कृत्स्नशः}


\twolineshloka
{तच्च सर्वं यथान्यायं धर्मराजेन भारत}
{आप्तैश्चारैः परिज्ञातं भारद्वाजचिकीर्षितम्}


\twolineshloka
{ततः सर्वान्समानाय्य भ्रातृनन्यांश्च सर्वशः}
{अव्रवीद्धर्मराजस्तु धनञ्जयमिदं वचः}


\twolineshloka
{श्रुतं ते पुरुषव्याघ्र द्रोणस्याद्य चिकीर्षितम्}
{यथा तन्न भवेत्सत्यं तथा नीतिर्विधीयताम्}


\twolineshloka
{सान्तरं हि प्रतिज्ञातं द्रोणेनामित्रघातिना}
{तच्चान्तरममोघेषौ त्वयि तेन समाहितम्}


\threelineshloka
{तत्त्वमद्य महाबाहो युध्यस्व मदनन्तरम्}
{यथा दुर्योधनः कामं नेमं द्रोणादवाप्नुयात् ॥अर्जुन उवाच}
{}


\twolineshloka
{यथा मे न वधः कार्य आचार्यस्य कथञ्चन}
{तथा तव परित्यागो न मे राजंश्चिकीर्षितः}


\twolineshloka
{अप्येवं पाण्डव प्राणानुत्सृजेयमहं युधि}
{प्रतियाताऽहमाचार्यं त्वां न जह्यां कथञ्चन}


\twolineshloka
{त्वां निगृह्याहवे राजन्धार्तराष्ट्रो यमिच्छति}
{न स तं जीवलोकेऽस्मिन्कामं प्राप्ता कथञ्चन}


\twolineshloka
{प्रपतेद्द्यौः सनक्षत्रा पृथिवी शकली भवेत्}
{न त्वां द्रोणो निगृह्णीयाज्जीवमाने मयि ध्रुवम्}


\twolineshloka
{यदि तस्य रणे साह्यं कुरुते वज्रभृत्स्वयम्}
{विष्णुर्वा सहितो देवैर्न त्वां प्राप्स्यत्यसौ मृधे}


\twolineshloka
{मयि जीवति राजेन्द्र न भयं कर्तुमर्हसि}
{द्रोणादस्त्रभृतां श्रेष्ठात्सर्वशस्त्रभृतामपि}


% Check verse!
अन्यच्च ब्रूयां राजेन्द्र प्रतिज्ञां मम निश्चलाम्
\threelineshloka
{न स्मराम्यनृतं तावन्न स्मरामि पराजयम्}
{न स्मरामि प्रतिश्रुत्य विस्मृत्य मनस्वऽकृतम् ॥सञ्जय उवाच}
{}


\twolineshloka
{ततः शङ्खाश्च भेर्यश्च मृदङ्गाश्चानकैः सह}
{प्रावाद्यन्त महाराज पाण्डवानां निवेशने}


% Check verse!
सिंहनादश्च संजज्ञे पाण्डवानां महात्मनाम् ॥धनुर्ज्यातलशब्दश्च गगनस्पृक्सुभैरवः
\twolineshloka
{श्रुत्वा शङ्खस्य निर्घोषं पाण्डवस्य महौजसः}
{त्वदीयेष्वप्यनीकेषु वादित्राण्यभिजघ्निरे}


\twolineshloka
{ततो व्यूढान्यनीकानि तव तेषां च भारत}
{शनैरुपेयुरन्योन्यं योत्स्यमानानि संयुगे}


\twolineshloka
{ततः प्रववृते युद्धं तुमुलं रोमहर्षणम्}
{पाण्डवानां कुरूणां च द्रोणपाञ्चाल्ययोरपि}


\twolineshloka
{यतमानाः प्रयतेन द्रोणानीकविशातने}
{न शेकुः सृञ्जया युद्धे तद्धि द्रोणेन पालितम्}


\twolineshloka
{तथैव तव पुत्रस्य रथोदाराः प्रहारिणः}
{न शेकुः पाण्डवीं सेनां पाल्यमानां किरीटीना}


\twolineshloka
{आस्तां ते स्तिमिते सेने रक्ष्यमाणे परस्परम्}
{सम्प्रसुप्ते यथा नक्तं वनराज्यौ सुपुष्पिते}


\twolineshloka
{ततो रुक्मरथो राजन्नर्केणेव विराजता}
{वरूथिना विनिष्पत्य व्यचरत्पृतनामुखे}


\twolineshloka
{तमुद्यन्तं रथेनैकमाशुकारिणमाहवे}
{अनेकमिव सन्त्रसान्मेनिरे पाण्डुसृञ्जयाः}


\twolineshloka
{तेन मुक्ताः शरा घोरा विचेरुः सर्वतोदिशम्}
{त्रासयन्तो महाराज पाण्डवेयस्य वाहिनीम्}


\twolineshloka
{मध्यन्दिनमनुप्राप्तो गभस्तिशतसंवृतः}
{यथा दृश्येत घर्मोशुस्तथा द्रोणोऽप्यदृश्यत}


\twolineshloka
{न चैनं पाण्डवेयानां कश्चिच्छक्नोति भारत}
{वीक्षितुं समरे क्रुद्धं महेन्द्रमिव दानवाः}


\twolineshloka
{मोहयित्वा ततः सैन्यं भारद्वाजः प्रतापवान्}
{धृष्टद्युम्नबलं तूर्णं व्यधमन्निशितैः शरैः}


\twolineshloka
{स दिशः सर्वतो रुद्ध्वा संवृत्य खमजिह्मगैः}
{पार्षतो यत्र तत्रैनामभिनत्पाण्डुवाहिनीम्}


\chapter{अध्यायः १४}
\twolineshloka
{सञ्जय उवाच}
{}


\twolineshloka
{ततः स पाण्डवानीके जनयन्सुमहद्भयम्}
{व्यचरत्पृतनां द्रोणो दहन्कक्षमिवानलः}


\twolineshloka
{निर्दहन्तमनीकानि साक्षादग्निमिवोत्थितम्}
{दृष्ट्वा रुक्मरथं क्रुद्धं समकम्पन्त सृञ्जयाः}


\twolineshloka
{सततं कृष्यतः सङ्ख्ये धनुषोऽस्याशुकारिणः}
{ज्याघोषः शुश्रुवेऽत्यर्थं विस्फूर्जितमिवाशनेः}


\twolineshloka
{रथिनः सादिनश्चैव नागानश्वान्पदातिनः}
{रौद्रा हस्तवता मुक्ताः सम्मृद्गन्ति स्स सायकाः}


\twolineshloka
{नानद्यमानः पर्जन्यः प्रवृद्धः शुचिसङ्क्षये}
{अश्मवर्षमिवावर्षत्परेषामावहद्भयम्}


\twolineshloka
{विचरन्स दिशः सर्वाः सेनां सङ्क्षोभयन्प्रभुः}
{वर्धयामास सन्त्रासं शात्रवाणाममानुषम्}


\twolineshloka
{तस्य विद्युदिवाभ्रेषु चापं हेमपरिष्कृतम्}
{भ्राजमानं रथे तस्मिन्दृश्यते स्म महाभयम्}


\twolineshloka
{स वीरः सत्यवान्प्राज्ञो धर्मनित्यः सदा पुनः}
{युगान्तकालवद्धोरां रौद्रां प्रावर्तयन्नदीम्}


\twolineshloka
{अमर्षवेगप्रभवां क्रव्यादगणसङ्कुलाम्}
{बलौघैः सर्वतः पूर्णां ध्वजवृक्षापहारिणीम्}


\twolineshloka
{शोणितोदां रथावर्तां हस्त्यश्वकृतरोधसम्}
{कवचोडुपसंयुक्तां मांसपङ्कसमाकुलाम्}


% Check verse!
मेदोमज्जास्थिसिकतामुष्मीषचयफेनिलाम् ॥सङ्ग्रामजलदापूर्णां प्रासमत्स्यसमाकुलाम्
\twolineshloka
{नरनागाश्वकलिलां शरवेगौघवाहिनीम्}
{शरीरदारुसङ्घाटां रथकच्छपसङ्कुलाम्}


\twolineshloka
{उत्तमाङ्गैः पङ्कजिनीं निस्त्रिंशझषसङ्कुलाम्}
{रथनागहूदोपेतां नानाभारणभूषिताम्}


\twolineshloka
{महारथशतावर्तां भूमिरेणूर्मिमालिनीम्}
{महावीर्यवतां सङ्ख्ये सुतरां भीरुदुस्तराम्}


\twolineshloka
{शरीरशतसम्बाधां गृध्रकङ्कनिषेविताम्}
{महारथसहस्राणि नयन्तीं यमसादनम्}


\twolineshloka
{शूलव्यालसमाकीर्णां प्राणिवाजिनिषेविताम्}
{छिन्नक्षत्रमहाहंसां मुकुटाण्डजसेविताम्}


\twolineshloka
{चक्रकूर्माङ्गदानक्रां शरक्षुद्रझषाकुलाम्}
{बकगृध्रसृगालानां घोरसङ्घैर्निषेविताम्}


\twolineshloka
{निहतान्प्राणिनः सङ्ख्ये द्रोणेन बलिना रणे}
{वहन्तीं पितृलोकाय शतशो राजसत्तम}


\twolineshloka
{शऱीरशतसम्बाधां केशशैवलशाद्वलाम्}
{नदीं प्रावर्तयद्राजन्भीरूपणां भयवर्धिनीम्}


\twolineshloka
{तर्जयन्तमनीकानि तानि तानि महारथम्}
{सर्वतोऽभ्यद्रवन्द्रोणं युधिष्ठिरपुरोगमाः}


\twolineshloka
{तानभिद्रवतः शूरांस्तावका दृढविक्रमाः}
{सर्वतः प्रत्यगृह्णन्त तदभूद्रोमहर्षणम्}


\twolineshloka
{शतमायस्तु शकुनिः सहदेवं समाद्रवत्}
{सनियन्तृध्वजरथं विव्याघ निशितैः शरैः}


\threelineshloka
{तस्य माद्रीसुतः केतुं धनुः सूतं हयानपि}
{नातिक्रुद्धः शरैश्छित्त्वा षष्ठ्या विव्याध सौबलम्}
{`भित्त्वा च शरवर्षेण शकुनिं प्रत्यवारयत्'}


\twolineshloka
{गदां गृहीत्वा शकुनिः प्रचस्कन्द रथोत्तमात्}
{स तस्य गदया राजन्रथात्सूतमपातयत्}


\twolineshloka
{ततस्तौ विरथौ राजन्गदाहस्तौ महाबलौ}
{चिक्रीडतू रणे शूरौ सशृङ्गाविव पर्वतौ}


\threelineshloka
{द्रोणः पाञ्चालदायादं विव्याध निशितैः शरैः}
{तयोस्तत्र महाराज बाणवर्षैः प्रकाशितम्}
{खद्योतैरिव चाकाशं प्रदोषे पुरुषर्षभ}


\twolineshloka
{विविंशतिं भीमसेनो विंशत्या निशितैः शरैः}
{विद्ध्वा नाकम्पयद्वीरस्तदद्भुतमिवाभवत्}


\twolineshloka
{विविंशतिस्तु सहसा व्यश्वकेतुशरासनम्}
{भीमं चक्रे महाराज ततः सैन्यान्यपूजयन्}


\twolineshloka
{स तन्न ममृषे वीरः शत्रोर्विक्रममाहवे}
{ततोऽस्य गदया दान्तान्हयात्सर्वानपातयत्}


\twolineshloka
{हताश्वात्स रथाद्राजन्गृह्य चर्म महाबलः}
{अभ्ययाद्भीमसेनं तु मत्तो मत्तमिव द्विपम्}


\twolineshloka
{शल्यस्तु नकुलं वीरः स्वस्रीयं प्रियमात्मनः}
{विव्याध प्रहसन्बाणैर्लालयन्कोपयन्निव}


\twolineshloka
{तस्याश्वानातपत्रं च ध्वजं सूतमथो धनुः}
{निपात्य नकुलः सङ््ख्ये शङ्खं दध्मौ प्रतापवान्}


\twolineshloka
{धृष्टकेतुः कृपेणास्ताञ्छित्त्वा बहुविधाञ्छरान्}
{कृपं विव्याध सप्तत्या ध्वजं चास्य त्रिभिः शरैः}


\twolineshloka
{तं कृपः शरवर्षेण महता समवारयत्}
{विव्याध च रणे विप्रो धृष्टकेतुममर्षणम्}


\twolineshloka
{सात्यकिः कृतवर्माणं नाराचेन स्तनान्तरे}
{विद्ध्वा विव्याध सप्तत्या पुनरन्यैः स्मयन्निव}


\twolineshloka
{तं भोजः सप्तसप्तत्या विद्ध्वाऽऽशु निशितैः शरैः}
{नाकम्पयत शैनेयं शीघ्रो वायुरिवाचलम्}


\twolineshloka
{सेनापतिः सुशर्माणं भृशं मर्मस्वताडयत्}
{स चापि तं तोमरेण जत्रुदेशेऽभ्यताडयत्}


\twolineshloka
{वैकर्तनं तु समरे विराटः प्रत्यवारयत्}
{सह मत्स्यैर्महावीर्यैस्तदद्भुतमिवाभवत्}


\twolineshloka
{तत्पौरुषमभूत्तत्र सूतपुत्रस्य दारुणम्}
{यत्सैन्यं वारयामास शरैः सन्नतपर्वभिः}


\twolineshloka
{द्रुपदस्तु स्वयं राजा भगदत्तेन सङ्गतः}
{तयोर्युद्धं महाराज चित्ररूपमिवाभवत्}


\twolineshloka
{भगदत्तस्तु राजानं द्रुपदं नतपर्वभिः}
{सनियन्तृध्वजरथं विव्याध पुरुषर्षभः}


\twolineshloka
{द्रुपदस्तु ततः क्रुद्धो भगदत्तं महारथम्}
{आजघानोरसि क्षिप्रं शरेणानतपर्वणा}


\twolineshloka
{युद्धं योधवरौ लोके सौमदत्तिशिखण्डिनौ}
{भूतानां त्रासजननं चक्रातेऽस्त्रविशारदौ}


\twolineshloka
{भूरिश्रवा रणे राजन्याज्ञसेनिं महारथम्}
{महता सायकौघेन च्छादयामास वीर्यवान्}


\twolineshloka
{शिखण्डी तु ततः क्रुद्धः सौमदत्तिं विशाम्पते}
{नवत्या सायकानां तु कम्पयामास भारत}


\twolineshloka
{राक्षसौ रौद्रकर्माणौ हेडिम्बालम्बुसावुभौ}
{चक्रातेऽत्युद्भुतं युद्धं परस्परजयैषिणौ}


\twolineshloka
{मायाशतसृजौ दृप्तौ मायाभिरितरेतरम्}
{अन्तर्हितौ चेरतुस्तौ भृशं विस्मयकारिणौ}


\twolineshloka
{चेकितानोऽनुविन्देन युयुधे चातिभैरवम्}
{यथा देवासुरे युद्धे बलशक्रौ महाबलौ}


\twolineshloka
{लक्ष्मणः क्षत्रदेवेन विमर्दमकरोद्भृशम्}
{यथा विष्णुः पुरा राजन्हिरण्याक्षेण संयुगे}


\twolineshloka
{ततः प्रचलिताश्वेन विधिवत्कल्पितेन च}
{रथेनाभ्यपतद्राजन्सौभद्रं पौरवो पौरवो नदन्}


\twolineshloka
{ततोऽभ्ययात्स त्वरितो युद्धाकाङ्क्षी महाबलः}
{तेन चक्रे महद्युद्धमभिमन्युररिन्दमः}


\twolineshloka
{पौरवस्त्वथ सौभद्रं शरव्रातैरवाकिरत्}
{तस्यार्जुनिर्ध्वजं छत्रं धनुश्चोर्व्यामपातयत्}


\twolineshloka
{सौभद्रः पौरवं त्वन्यैर्विद्ध्वा सप्तभिराशुगैः}
{पञ्चभिस्तस्य विव्याध हयान्सूनं च सायकैः}


\twolineshloka
{ततः प्रहर्षयन्सेनां सिंहवद्विनदन्मुहुः}
{समादत्तार्जुनिस्तूर्णं पौरवान्तकरं शरम्}


\twolineshloka
{तं तु सन्धितमाज्ञाय सायकं घोरदर्शनम्}
{द्वाभ्यां शराभ्यां हार्दिक्यश्चिच्छेद सशरं धनुः}


\twolineshloka
{तदुत्सृज्य धनुश्छिन्नं सौभद्रः परवीरहा}
{उद्बबर्ह सितं खङ्गमाददनाः शरावरम्}


\twolineshloka
{स तेनानेकतारेण चर्मणा कृतहस्तवत्}
{भ्रान्तासिना चरन्मार्गान्दर्शयन्वीर्यमात्मनः}


\twolineshloka
{भ्रामितं पुनरुद्धान्तमाधूतं पुनरुत्थितम्}
{चर्मनिस्त्रिंशयो राजन्निर्विशेषमदृश्यत}


\twolineshloka
{स पौरवरथस्येषामाप्लुत्य सहसा नदन्}
{पौरवं रथमास्थाय केशपक्षे परामृशत्}


\twolineshloka
{जघानास्य पदा सूतमसिना पातयद्धृजम्}
{विक्षोभ्याम्भोनिधिं तार्क्ष्यस्तं नागमिव चाक्षिपत्}


\twolineshloka
{तमागलितकेशान्तं ददृशुः सर्वपार्थिवाः}
{उक्षाणमिव सिंहेन पात्यमानमचेतसम्}


\twolineshloka
{तमार्जुनिवशं प्राप्तं कृष्यमाणमनाथवत्}
{पौरवं पातितं दृष्ट्वा नामृष्यत जयद्रूथः}


\twolineshloka
{स बर्हिबर्हावततं किङ्किणीशतजालवत्}
{चर्म चादाय खङ्गं च नदन्पर्यपतद्रथात्}


\threelineshloka
{ततः सैन्धवमालोक्य कार्ष्णिरुत्सृज्य पौरवम्}
{उत्पपात रथात्तूर्णं श्येनवन्निपपात च}
{}


\twolineshloka
{प्रासपट्टिशनिस्त्रिंशाञ्छत्रुभिः सम्प्रचोदितान्}
{चिच्छेद चासिना कार्ष्णिश्चर्मणा संरुरोध च}


\twolineshloka
{स दर्शयित्वा सैन्यानां स्वबाहुबलमात्मनः}
{तमुद्यम्य महाखङ्गं चर्म चाथ पुनर्बली}


\twolineshloka
{वृद्धक्षत्रस्य दायादं पितुरत्यन्तवैरिणम्}
{ससाराभिमुखः शूरः शार्दूल इव कुञ्जरम्}


\twolineshloka
{तौ परस्परमासाद्य खह्गदन्तनखायुधौ}
{हृष्टवत्सम्प्रजहाते व्याघ्रकेसरिणाविव}


\twolineshloka
{सम्पातेष्वभिघातेषु निपातेष्वसिचर्मणोः}
{न तयोरन्तरं कश्चिद्ददर्श नरसिंहयोः}


\twolineshloka
{अवक्षेपोऽसिनिर्हादः शस्त्रान्तरनिदर्शनम्}
{बाह्यान्तरनिपातश्च निर्विशेमदृश्यत}


\twolineshloka
{बाह्यमाभ्यन्तरं चैव चरन्तौ मार्गमुत्तमम्}
{ददृशाते महात्मानौ सपक्षाविव पर्वतौ}


\twolineshloka
{ततो विक्षिपतः खङ्गं सौभद्रस्य यशस्विनः}
{शरावरणपक्षान्ते प्रजहार जयद्रथः}


\twolineshloka
{रुक्मपत्रान्तरे सक्तस्तस्मिंश्चर्मणि भास्वरे}
{सिन्धुराजबलोद्धूतः सोऽभज्यत महानसिः}


\twolineshloka
{भग्नमाज्ञाय निस्त्रिंशमवप्लुत्य पदानि षट्}
{अदृश्यत निमेषेण स्वरथं पुनरास्थितः}


\twolineshloka
{तं कार्ष्णिं समरान्मुक्तमास्थितं रथमुत्तमम्}
{सहिताः सर्वराजानः परिवव्रुः समन्ततः}


\twolineshloka
{ततश्चर्म च खङ्गं च समुत्क्षिप्य महाबलः}
{ननादार्जुनदायादः प्रेक्षमाणो जयद्रथम्}


\twolineshloka
{सिन्धुराजं परित्यज्य सौभद्रः परवीरहा}
{तापयामास तत्सैन्यं भुवनं भास्करो यथा}


\twolineshloka
{तस्य सर्वायसीं शक्तिं शल्यः कनकभूषणाम्}
{चिक्षेप समरे घोरां दीप्तामग्निशिखामिव}


\twolineshloka
{तामवप्लुत्य जग्राह विकोशं चाकरोदसिम्}
{वैनतेयो यथा कार्ष्णिः पतन्तमुरगोत्तमम्}


\twolineshloka
{तस्य लाषवमाज्ञाय सत्वं चामिततेजसः}
{सहिताः सर्वराजानः सिंहनादमथानदन्}


\twolineshloka
{ततस्तामेव शल्यस्य सौभद्रः परवीरहा}
{मुमोच भुजवीर्येण वैदूर्यविकृतां शिताम्}


\twolineshloka
{सा तस्य रथमासाद्य निर्मुक्तभुजगोपमा}
{जघान सूतं शल्यस्य रथाच्चैनमपातयत्}


\twolineshloka
{ततो विराटद्रुपदौ धृष्टकेतुर्युधिष्ठिरः}
{सात्यकिः केकया भीमो धृष्टद्युम्नशिखण्डिनौ}


\twolineshloka
{यमौ च द्रौपदेयाश्च साधुसाध्विति चुक्रुशुः}
{बाणशब्दाश्च विविधाः सिंहनादाश्च पुष्कलाः}


\twolineshloka
{प्रादुरासन्हर्षयन्तः सौभद्रमपलायिनम्}
{तन्नामृष्यन्त पुत्रास्ते शत्रोर्विजयलक्षणम्}


\twolineshloka
{अथैनं सहसा सर्वे समन्तान्निशितैः शरैः}
{अभ्याकिरन्महाराज जलदा इव पर्वतम्}


\twolineshloka
{तेषां च प्रियमन्विच्छन्सूतस्य च पराभवम्}
{आर्तायनिरमित्रघ्नः क्रुद्धः सौभद्रमभ्ययात्}


\chapter{अध्यायः १५}
\chapter{अध्यायः १६}
\twolineshloka
{सञ्जय उवाच}
{}


\twolineshloka
{तद्बलं सुमहद्दीर्णं त्वदीयं प्रेक्ष्य वीर्यवान्}
{दधारैको रणे पाण्डून्वृषसेनोऽस्त्रमायया}


\twolineshloka
{शरा दश दिशो मुक्ता वृषसेनेन संयुगे}
{विचेरुस्ते विनिर्भिद्य नरवाजिरथद्विपान्}


\twolineshloka
{तस्य दीप्ता महाबाणा विविश्चेरुः सहस्रशः}
{भानोरिव महाराज घर्मकाले मरीचयः}


\twolineshloka
{तेनार्दिता महाराज रथिनः सादिनस्तथा}
{निपेतुरुर्व्यां सहसा वातभग्ना इव द्रुमाः}


\twolineshloka
{हयौघांश्च रथौघांश्च गजौघांश्च महारथः}
{अपातयद्रणे राजञ्शतशोऽथ सहस्रशः}


\twolineshloka
{दृष्ट्वा तमेकं समरे विचरन्तमभीतवत्}
{सहिताः सर्वराजानः परिवव्रुः समन्ततः}


\twolineshloka
{नाकुलिस्तु शतानीको वृषसेनं समभ्ययात्}
{विव्याध चैनं दशभिर्नाराचैर्मर्मभेदिभिः}


\twolineshloka
{तस्य कर्णात्मजश्चापं छित्त्वा केतुमपातयत्}
{तं भ्रातरं परीप्सन्तो द्रौपदेयाः समभ्ययुः}


\twolineshloka
{कर्णात्मजं शरव्रातैरदृश्यं चक्रुरञ्जसा}
{तान्नदन्तोऽभ्यधावन्त द्रोणपुत्रमुखा रथाः}


\twolineshloka
{छादयन्तो महाराज द्रौपदेयान्महारथान्}
{शरैर्नानाविधैस्तूर्णं पर्वताञ्जलदा इव}


\twolineshloka
{तान्पाण्डवाः प्रत्यगृह्णंस्त्वरिताः पुत्रगृद्धिनः}
{पाञ्चालाः केकया मत्स्याः सृञ्जयाश्चोद्यतायुधाः}


\twolineshloka
{तद्युद्धमभवद्धोरं सुमहद्रोमहर्षणम्}
{त्वदीयैः पाण्डुपुत्राणां देवानामिव दानवैः}


\twolineshloka
{एवं युयुधिरे वीराः संरब्धाः कुरुपाण्डवाः}
{परस्परमुदीक्षन्तः परस्परकृतागसः}


\twolineshloka
{तेषां ददृशिरे कोपाद्वपूम्ष्यमिततेजसाम्}
{युयुत्सूनामिवाकाशे पतत्त्रिवरभोगिनाम्}


\twolineshloka
{भीमकर्णकृपद्रोणद्रौणिपार्षतसात्यकैः}
{बभासे स रणोद्देशः कालसूर्यैरिवोदितैः}


\twolineshloka
{`प्रजानां संक्षये घोरे यथा सूर्योदयो भवेत्}
{शूराणामुदयस्तद्वदासीत्पुरुषसत्तम'}


\twolineshloka
{तदासीत्तुमुलं युद्धं निघ्नतामितरेतरम्}
{महाबलानां बलिभिर्दानवानां यथा सुरैः}


\twolineshloka
{ततो युधिष्ठिरानीकमुद्धूतार्णवनिःस्वनम्}
{त्वदीयमवधीत्सैन्यं संप्रद्रुतमहारथम्}


\twolineshloka
{तत्प्रभग्नं बलं दृष्ट्वा शत्रुभिर्भृशमर्दितम्}
{अलं द्रुतेन वः शूरा इति द्रोणोऽभ्यभाषत}


\threelineshloka
{`भारद्वाजममर्षश्च विक्रमश्च समाविशत्}
{समुद्धृत्य निषङ्गाच्च धनुर्ज्यामवमृज्य च}
{महाशरधनुष्पाणिर्यन्तारमिदमब्रवीत्}


\twolineshloka
{सारथे याहि यत्रैष पाण्डरेण विराजता}
{ध्रियमाणेन च्छेण राजा तिष्ठति धर्मराट्}


\twolineshloka
{तदेतद्दीर्यते सैन्यं धार्तराष्ट्रमनेकधा}
{एतत्संस्तम्भयिष्यामि प्रतिवार्य युधिष्ठिरम्}


\twolineshloka
{न हि मामभिवर्षन्तं संयुगे तात पाण्डवाः}
{मात्स्यपाञ्चालराजानः सर्वे च सहसोमकाः}


\twolineshloka
{अर्जुनो मत्प्रसादाद्धि महास्त्राणि समाप्तवान्}
{न मामुत्सहते तात न भीमो न च सात्यकिः}


\twolineshloka
{मत्प्रसादाद्धि बीभत्सुः परमेष्वासतां गतः}
{ममैवास्त्रं विजानाति धृष्टद्युम्नोऽपि पार्षतः}


\threelineshloka
{नायं संरक्षितुं कालः प्राणांस्तात जयैषिणा}
{याहि सर्गं पुरस्कृत्य यशसे च जयाय च ॥सञ्जय उवाच}
{}


% Check verse!
एवं सञ्चोदितो यन्ता द्रोणमभ्यवहत्ततः
\twolineshloka
{तदाऽश्वहृदयेनाश्वानभिमन्त्र्याशु हर्षयन्}
{रथेन सवरूथेन भास्वरेण विराजता}


\twolineshloka
{तं करूशाश्च मात्स्याश्च चेदयश्च ससात्वताः}
{पाण्डवाश्च सपाञ्चालाः सहिताः पर्यवारयन्'}


\twolineshloka
{ततः शोणहयः क्रुद्धश्चतुर्दन्त इव द्विपः}
{प्रविश्य पाण्डवानीकं युधिष्ठिरमुपाद्रवत्}


\twolineshloka
{तमाविध्यच्छितैर्बाणैः कङ्कपत्रैर्युधिष्ठिरः}
{तस्य द्रोणो धनुश्छित्वा तं द्रुतं समुपाद्रवत्}


\twolineshloka
{चक्ररक्षः कुमारस्तु पाञ्चालानां यशस्करः}
{दधार द्रोणमायान्तं वेलेव सरितां पतिम्}


\twolineshloka
{द्रोणं निवारितं दृष्ट्वा कुमारेण द्विजर्षभम्}
{सिंहनादरवो ह्यासीत्साधुसाध्विति भाषितम्}


\twolineshloka
{कुमारस्तु ततो द्रोणं सायकेन महाहवे}
{विव्याधोरसि सङ्क्रुद्धः सिंहवच्च नदन्मुहुः}


\twolineshloka
{संवार्य च रणे द्रोणं कुमारस्तु महाबलः}
{शरैरनेकसाहस्रैः कृतहस्तो जितश्रमः}


\twolineshloka
{तं शूरमार्यव्रतिनं मन्त्रास्त्रेषु कृतश्चमम्}
{चक्ररक्षतं परामृद्रात्कुमारं द्विजपुङ्गवः}


\twolineshloka
{स मध्यं प्राप्य सैन्यानां सर्वाः प्रविचरन्दिशः}
{तव सैन्यस्य गोप्तासीद्भारद्वाजो द्विजर्षभः}


\twolineshloka
{शिखण्डिनं द्वादशभिर्विंशत्या चोत्तमौजसम्}
{नकुलं पञ्चभिर्विद्ध्वा सहदेवं च सप्तभिः}


\twolineshloka
{युधिष्ठिरं द्वादशभिर्द्रौपदेयांस्त्रिभिस्त्रिभिः}
{सात्यकिं पञ्चभिर्विद्ध्वा मत्स्यं च दशभिः शरैः}


\twolineshloka
{व्यक्षोभयद्रणे योधान्यथामुख्यमभिद्रवन्}
{अभ्यवर्ततं सम्प्रेप्सुः कुन्तीपुत्रं युधिष्ठिरम्}


\twolineshloka
{युगन्धरस्ततो राजन्भारद्वाजं महारथम्}
{वारयामास सङ्क्रुद्वं वातोद्धतमिवार्णवम्}


\twolineshloka
{युधिष्ठिरं स विद्ध्वा तु शरैः सन्नतपर्वभिः}
{युगन्धरं तु भल्लने रथनीडादपातयत्}


\twolineshloka
{`तं विजित्य महातेजा भारद्वाजो महामनाः}
{अभ्यवर्तत सम्प्रेप्सुः कुन्तीपुत्रं युधिष्ठिरम्'}


\twolineshloka
{ततो विराटद्रुपदौ केकयाः सात्यकिः शिबिः}
{व्याघ्रदत्तश्च पाञ्चल्यः सिंहसेनश्च वीर्यवान्}


\twolineshloka
{एते चान्ये च बहवः परीप्सन्तो युधिष्ठिरम्}
{आवव्रुस्तस्य पन्थानं किरन्तः सायकान्बहून्}


\twolineshloka
{व्याघ्रदत्तस्तु पाञ्चाल्यो द्रोणं विव्याध मार्गणैः}
{पञ्चाशता शितै राजंस्तत उच्चुक्रुशुर्जनाः}


\twolineshloka
{त्वरितं सिंहसेनस्तु द्रोणं विद्ध्वा महारथम्}
{प्राहसत्सहसा हृष्टस्त्रासयन्वै महारथान्}


\twolineshloka
{ततो विष्फार्य नयने धनुर्ज्यामवमृज्य च}
{तलशब्दं महत्कृत्वा द्रोणस्तं समुपाद्रवत्}


\twolineshloka
{ततस्तु सिंहसेनस्य शिरः कायात्सकुण्डलम्}
{व्याघ्रदत्तस्य चाक्रम्य भल्लाभ्यामाहरद्बली}


\twolineshloka
{तान्प्रमृज्य शरव्रातैः पाण्डवानां महारथान्}
{युधिष्ठिररथाभ्याशे तस्थौ मृत्युरिवान्तकः}


\twolineshloka
{ततोऽभवन्महाशब्दो राजन्यौधिष्ठिरे बले}
{हृतो राजेति योधानां समीपस्थे यतव्रते}


\twolineshloka
{अब्रुवन्सैनिकास्तत्र दृष्ट्वा द्रोणस्य विक्रमम्}
{अद्य राजा धार्तराष्ट्रः कृतार्थो वै भविष्यति}


\twolineshloka
{अस्मिन्महूर्ते द्रोणस्तु पाण्डवं गृह्य हर्षितः}
{आगमिष्यति नो नूनं धार्तराष्ट्रस्य संयुगे}


\twolineshloka
{एवं सञ्जल्पतां तेषां तावकानां महारथः}
{आयाज्जवेन कौन्तेयो रथघोषेण नादयन्}


\twolineshloka
{शोणितोदां रथावर्तां कृत्वा विशसने नदीम्}
{शूरास्थिचयसङ्कीर्णां प्रेतकूलापहारिणीम्}


\twolineshloka
{तां शरौघमहाफेनां प्रासमत्स्यसमाकुलाम्}
{नदीमुत्तीर्य वेगेन कुरून्विद्राव्य पाण्डवः}


\twolineshloka
{ततः किरीटि सहसा द्रोणानीकमुपाद्रवत्}
{छादयन्निषुजालेन महता मोहयन्निव}


\twolineshloka
{शीघ्रमभ्यस्यतो बाणान्सन्दधानस्य चानिशम्}
{नान्तरं ददृशे कश्चित्कौन्तेयस्य यशस्विनः}


\twolineshloka
{न दिशो नान्तरिक्षं च न द्यौर्नैव च मेदिनी}
{अदृश्यन्त महाराज बाणभूता इवाभवन्}


\twolineshloka
{नादृश्यत तदा राजंस्तत्र किञ्चन संयुगे}
{बाणान्धकारे महति कृते गाण्डीवधन्वना}


\threelineshloka
{सूर्ये चास्तमनुप्राप्ते तमसा चाभिसंवृते}
{नाज्ञायत तदा शत्रुर्न सुहृन्न च कश्चन}
{}


\twolineshloka
{ततोऽवहारं चक्रुस्ते द्रोणदुर्योधनादयः}
{तान्विदित्वा पुनस्त्रस्ता न युद्धमनसः परान्}


\twolineshloka
{स्वान्यनीकानि बीभत्सुः शनकैरवहारयत्}
{ततोऽभितुष्टुवुः पार्थं प्रहृष्टाः पाण्डुसृञ्जयाः}


\twolineshloka
{पाञ्चालाश्च मनोज्ञाभिर्वाग्भिः सूर्यमिवर्षयः}
{एं स्वशिबिरं प्रायाज्जित्वा शत्रून्धनञ्जयः}


\twolineshloka
{पृष्ठतः सर्वसैन्यानां मुदितो वै सकेशवः ॥मसारगल्वर्कमुवर्णरूपैर्वज्रप्रवालस्फटिकैश्च मुख्यैः}
{}


% Check verse!
चित्रे रथे पाण्डुसुतो बभासे नक्षत्रचित्रे वियतीव चन्द्रः
\chapter{अध्यायः १७}
\twolineshloka
{सञ्जय उवाच}
{}


\twolineshloka
{ते सेने शिबिरं गत्वा न्यविशेतां विशाम्पते}
{यथाभागं यथान्यायं यथागुल्मं च सर्वशः}


\twolineshloka
{कृत्वाऽवहारं सैन्यानां द्रोणः परमदुर्मनाः}
{दुर्योधनमभिप्रेक्ष्य सव्रीडमिदमब्रवीत्}


\twolineshloka
{उक्तमेतन्मया पूर्वं न तिष्ठति धन्जये}
{शक्यो ग्रहीतुं सङ्ग्रामे देवैरपि युधिष्ठिरः}


\twolineshloka
{इति तद्वः प्रयततां कृतं पार्थेन संयुगे}
{मा विशङ्कीर्वचो मह्यमजेयौ कृष्णपाण्डवौ}


\twolineshloka
{अपनीते तु योगेन केनचिच्छ्वेतवाहने}
{तत एष्यति ते राजन्वशमेष युधिष्ठिरः}


\twolineshloka
{कश्चिदाहूय तं सङ्ख्ये देशमन्यं प्रकर्षतु}
{तमजित्वा न कौन्तेयो निवर्तेत कथञ्चन}


\twolineshloka
{एतस्मिन्नन्तरे शून्येधर्मराजमहं नृप}
{ग्रहीष्यामि चमूं भित्त्वा धृष्टद्युम्नस्य पश्यतः}


\twolineshloka
{अर्जुनेन परित्यक्तो न चेत्त्रासात्पलायते}
{मामुपायान्तमालोक्य गृहीतं विद्धि पाण्डवम्}


\twolineshloka
{एवं तेऽहं महाराज धर्मपुत्रं युधिष्ठिरम्}
{समानेष्यामि सगणं वशमद्य न संशयः}


\threelineshloka
{यदि तिष्ठति सङ्ग्रामे मुहूर्तमपि पाण्डवः}
{अथापयाति सङ्ग्रामाद्विजयात्तद्विशिष्यते ॥सञ्जय उवाच}
{}


\twolineshloka
{द्रोणस्य तद्वचः श्रुत्वा त्रिगर्ताधिपतिस्तदा}
{भ्रातृभिः सहितो राजन्निदं वचनमब्रवीत्}


\twolineshloka
{वयं विनिकृता राजन्सदा गाण्डीवधन्वना}
{अनागःस्वपि चागस्तत्कृतमस्मासु तेन वै}


\twolineshloka
{ते वयं स्मरमाणास्तान्विनिकारान्पृथग्विधान्}
{क्रोधाग्निना दह्यमाना ह्यशेमहि सदा निशि}


\twolineshloka
{स नो दिष्ट्याऽस्त्रसम्पन्नश्चक्षुर्विषयमागतः}
{कर्तारः स्मरयं कर्म यच्चिकीर्षाम हृद्गतम्}


\twolineshloka
{भवतश्च प्रियं यत्स्यादस्माकं च यशस्करम्}
{वयमेनं हनिष्यामो निकृष्यायोधनाद्बहिः}


\twolineshloka
{अद्यास्त्वनर्जुना भूमिरत्रिगर्ताऽवा पुनः}
{सत्यं ते प्रतिजानीमो नैतन्मिथ्या भविष्यति}


\twolineshloka
{एवं सत्यरथश्चोक्त्वा सत्यवर्मा च भारत}
{सत्यव्रतश्च सत्येषुः सत्यकर्मा तथैव च}


\twolineshloka
{सहिता भ्रातरः पञ्च रथानामयुतेन च}
{न्यवर्तन्त महाराज कृत्वा शपथमाहवे}


\twolineshloka
{मालवास्तुण्डिकेराश्च रथानामयुतैस्त्रिभिः}
{सुशर्मा च नरव्याघ्रस्त्रिगर्तः प्रस्थलाधिपः}


\twolineshloka
{मावेल्लकैर्ललित्थैश्च सहितो मद्रकैरपि}
{रथानामयुतेनैव सोऽगमद्धातृभिः सह}


\twolineshloka
{नानाजनपदेभ्यश्च रथानामयुतं पुनः}
{समुत्थितं विशिष्टानां शपथार्थमुपागमत्}


\twolineshloka
{ततो ज्वलनमानर्च्य हुत्वा सर्वे पृथक्पृथक्}
{जगृहुः कुशचीराणि चित्राणि कवचानि च}


\twolineshloka
{ते च बद्धतनुत्राणा घृताक्ताः कुशचीरिणः}
{मौर्वीमेखलिनो वीराः सहस्रशतदक्षिणाः}


\twolineshloka
{यज्वानः पुत्रिणो लोक्याः कृतकृत्यास्तनुत्यजः}
{योक्ष्यमाणास्तदाऽऽत्मानं यशसा विजयेन च}


\twolineshloka
{ब्रह्मचर्युश्रुतिमुखैः क्रतुभिश्चाप्तदक्षिणैः}
{प्राप्याँल्लोकान्सुयुद्धेन क्षिप्रमेव यियासवः}


\twolineshloka
{ब्राह्ममांस्तर्पयित्वा च निष्कान्दत्वा पृथक्पृथक्}
{गाश्च वासांसि च पुनः समाभाष्य परस्परम्}


\twolineshloka
{`द्विजमुख्यैः समुदितैः कृतस्वस्त्ययनाशिषः}
{मुदिताश्च प्रहृष्टाश्च जलं संस्पृश्य निर्मलम्'}


\twolineshloka
{प्रज्वाल्य कृष्णवर्त्मानमुपागम्य रणव्रतम्}
{तस्मिन्नग्नौ तदा चक्रुः प्रतिज्ञां दृढनिश्चयाः}


\threelineshloka
{शृण्वतां सर्वभूतानामुच्चैर्वाचो बभाषिरे}
{सर्वे धनञ्जयवधे प्रतिज्ञां चापि चक्रिरे}
{}


\twolineshloka
{ये वै लोकाश्चाव्रतिनां ये चैव ब्रह्मघातिनाम्}
{मद्यपस्य च ये लोका गुरुदाररतस्य च}


\twolineshloka
{ब्रह्मस्वहारिणश्चैव राजपिण्डापहारिणः}
{शरणागतं च त्यजतो याचमानं तथा घ्नतः}


\twolineshloka
{अगारदाहिनां चैव ये च गां निघ्नतामपि}
{अपकारिणां च ये लोका ये च ब्रह्मद्विषामपि}


\twolineshloka
{स्वभार्यामृतुकालेषु मोहाद्वै नाभिगच्छताम्}
{श्राद्धमैथुनिकानां च ये चाप्यात्मापहारिणाम्}


\twolineshloka
{न्यासापहारिणां ये च श्रुतं नाशयतां च ये}
{क्लीबने युध्यमानानां ये च नीचानुसारिणाम्}


\twolineshloka
{नास्तिकानां च ये लोका येऽग्निमातृपितृत्यजाम्}
{तानाप्नुयामहे लोकान्ये च पापकृतामपि}


\twolineshloka
{यद्यहत्वा वयं युद्धे निवर्तेम धनञ्जयम्}
{तेन चाभ्यर्दितास्त्रासाद्भवेमहि पराङ्मुखाः}


\twolineshloka
{यदि त्वसुकरं लोके कर्म कुर्याम संयुगे}
{इष्टाँल्लोकान्प्राप्नुयामो वयमद्य न संशयः}


\twolineshloka
{एवमुक्त्वा तदा राजंस्तेऽभ्यवर्तन्त संयुगे}
{आह्वयन्तोऽर्जुनं वीराः पितृजुष्टां दिशं प्रति}


\twolineshloka
{आहूतस्तैर्नरव्याघ्रैः पार्थः परपुरञ्चयः}
{धर्मराजमिदं वाक्यमपदान्तरमब्रवीत्}


\twolineshloka
{आहूतो न निवर्तेयमिति मे व्रतमाहितम्}
{संशप्तकाश्च मां राजन्नाह्वयन्ति महामृधे}


\twolineshloka
{एष च भ्रातृभिः सार्धं सुशर्माऽऽह्वयते रणे}
{वधाय सगणस्यास्य मामनुज्ञातुमर्हसि}


\threelineshloka
{नैतच्छक्नोमि संसोढुमाह्वानं पुरुषर्षभ}
{सत्यं ते प्रतिजानामि हतान्विद्धि परान्युधि ॥युधिष्ठिर उवाच}
{}


\twolineshloka
{श्रुतं ते तत्त्वतस्तात यद्द्रोणस्य चिकीर्षितम्}
{यथा तदनृतं तस्य भवेतत्त्वं समाचर}


\threelineshloka
{द्रोणो हि बलवाञ्शूरः कृतास्त्रश्च जितश्रमः}
{प्रतिज्ञातं च तेनैतद्ग्रहणं मे महारथ ॥अर्जुन उवाच}
{}


\twolineshloka
{अयं वै सत्यजिद्राजन्नद्य त्वां रक्षिता युधि}
{ध्रियमाणे च पाञ्चाल्ये नाचार्यः काममाप्स्यति}


\threelineshloka
{हते तु पुरुषव्याघ्रे रणे सत्यजिति प्रभो}
{सर्वैरपि समेतैर्वा न स्थातव्यं कथंचन ॥सञ्जय उवाच}
{}


\twolineshloka
{अनुज्ञातस्ततो राज्ञा परिष्वक्तश्च फल्गुनः}
{प्रेम्णा दृष्टश्च बहुधा ह्याशिषश्चास्य योजिताः}


\twolineshloka
{विहायैनं ततः पार्थस्त्रिगर्तान्प्रत्ययाद्बली}
{क्षुधितः क्षुद्विघातार्थं सिंहो मृगगणानिव}


\twolineshloka
{ततो दौर्योधनं सैन्यं मुदा परमया युतम्}
{ऋतेऽर्जुनं भृशं क्रुद्धं धर्मराजस्य निग्रहे}


\twolineshloka
{ततोऽन्योन्येन ते सैन्ये समाजग्मतुरोजसा}
{गङ्गासरय्वौ वेगेन प्रावृषीवोल्बणोदके}


\chapter{अध्यायः १८}
\twolineshloka
{सञ्जय उवाच}
{}


\twolineshloka
{ततः संशप्तका राजन्समे देशे व्यवस्थिताः}
{व्यूढानीकै रथैरेव चन्द्रार्धाख्यैर्मुदा युताः}


\threelineshloka
{ते किरीटिनमायान्तं दृष्ट्वा हर्षेण मारिष}
{`अतीव सम्प्रहृष्टास्ते ह्युपलक्ष्य धनञ्जयम्'}
{उदक्रोशन्नरव्याघ्राः शब्देन महता तदा}


\twolineshloka
{स दिशः प्रदिशः सर्वा नभश्चैव समावृणोत्}
{आवृतत्वाच लोकस्य नासीत्तत्र प्रतिस्वनः}


\twolineshloka
{सोऽतीव संप्रहृष्टांस्तानुपलभ्य धनञ्जयः}
{किञ्चिदभ्युत्स्मयन्कृष्णमिदं वचनमब्रवीत्}


\twolineshloka
{देवकीपुत्र पश्येमान्मुमूर्षूनद्य संयुगे}
{भ्रातॄंस्त्रैगर्तकानेतान्रोदितव्ये प्रहर्षितान्}


\fourlineindentedshloka
{अथवा हर्षकालोऽयं त्रैगर्तानामसंशयम्}
{कुनरैर्दुरवापान्हि लोकान्प्राप्स्यन्ति संयुगे}
{`मया हता हिं सङ्ग्रामे लोकान्प्राप्स्यन्ति पुष्कलान्'एवमुक्त्वा महाबाहुर्हृषीकेशं ततोऽर्जुनः}
{आससाद रणे व्यूढां त्रिगर्तानामनीकिनीम्}


\twolineshloka
{स देवदत्तमादाय शङ्खं हेमपरिष्कृतम्}
{दध्मौ वेगेन महता घोषेणापूरयन्दिशः}


\twolineshloka
{तेन शब्देन वित्रस्ता संशप्तकवरूथिनी}
{विचेष्टाऽवस्थिता सङ्ख्ये ह्यश्मसारमयी यथा}


\twolineshloka
{`सा सेना भरतश्रेष्ठ निश्चेष्टा शुशुभे तदा}
{चित्रे पटे यथा न्यस्ता कुशलैः शिल्पिभिर्नरैः}


\threelineshloka
{स्वनेन तेन सैन्यानां दिवमावृण्वता तदा}
{सस्वना पृथिवी सर्वा तथैव च महोदधिः}
{स्वनेन तेन सैन्यानां कर्णास्तु बधिरीकृताः'}


\threelineshloka
{वाहास्तेषां विवृत्ताक्षाः स्तब्धकर्णशिरोधराः}
{विष्टब्धचरणा मूत्रं रुधिरं च प्रसुस्रुवुः}
{`ततो व्युपारमच्छब्दः प्रहृष्टास्ते ततोऽभवन्'}


\twolineshloka
{उपलभ्य ततः संज्ञामवस्थाप्य च वाहिनीम्}
{युगपत्पाण्डुपुत्राय चिक्षिपुः कङ्कपत्रिणः}


\twolineshloka
{तानर्जुनः सहस्राणि दशपञ्चभिराशुगैः}
{अनागतानेव शरैश्चिच्छेदाशु पराक्रमी}


\twolineshloka
{ततो़र्जुनं शितैर्बाणैर्दशभिर्दशभिः पुनः}
{प्राविध्यन्त ततः पार्थस्तानविध्यत्त्रिभिस्त्रिभिः}


\twolineshloka
{एकैकस्तु ततः पार्थं राजन्विव्याध पञ्चभिः}
{स च तान्प्रतिविव्याध द्वाभ्यां द्वाभ्यां पराक्रमी}


\twolineshloka
{भूय एव तु सङ्क्रुद्धास्त्वर्जुनं सहकेशवम्}
{आपूरयञ्शरैस्तीक्ष्णैस्तटाकमिव वृष्टिभिः}


\twolineshloka
{ततः शरसहस्राणि प्रापतन्नर्जुनं प्रति}
{भ्रमराणामिव व्राताः फुल्लं द्रुमगणं वने}


\twolineshloka
{ततः सुबाहुस्त्रिंशद्भिरद्रिसारमयैः शरैः}
{अविध्यदिषुभिर्गाढं किरीटे सव्यसाचिनम्}


\twolineshloka
{तैः किरीटि किरीटस्थैर्हेमपुङ्खैरजिह्मगैः}
{शातकुम्भमयापीडो बभौ यूप इवोच्छ्रितः}


\twolineshloka
{हस्तावापं सुबाहोस्तु भल्लेन युधि पाण्डवः}
{चिच्छेद तं चैव पुनः शरवर्षैरवाकिरत्}


\twolineshloka
{ततः सुशर्मा दशभिः सुरथस्तु किरीटिनम्}
{सुधर्मा सुधनुश्चैव सुबाहुश्च समार्पयत्}


\twolineshloka
{तांस्तु सर्वान्पृथग्बाणैर्वानरप्रवरध्वजः}
{प्रत्यविध्यद्ध्वजांश्चैषां भल्लैश्चिच्छेद सायकान्}


\twolineshloka
{सुधन्वनो धनुश्छित्त्वा हयांश्चास्यावधीच्छरैः}
{अथास्य सशिरस्त्राणं शिरः कायादपातयत्}


\twolineshloka
{`जहार पार्थः सैन्येषु सहस्रे द्वे च योधिनाम्'}
{तस्मिन्निपतिते वीरे त्रस्तास्तस्य पदानुगाः}


\twolineshloka
{`भूयिष्ठं प्रतिरुद्धा ये हतैर्योधैर्यशस्विनः'}
{व्यद्रवन्त भयाद्भीता यत्र दौर्योधनं बलम्}


\twolineshloka
{ततो जघानं सङ्क्रुद्धो वासविस्तां महाचमूम्}
{शरजालैरविच्छिन्नैस्तमः सूर्य इवांशुभिः}


\twolineshloka
{ततो भग्ने बले तस्मिन्विप्रलीने समन्ततः}
{सव्यसाचिनि सङ्क्रुद्वे त्रैगर्तान्भयमाविशत्}


\twolineshloka
{ते वध्यमानाः पार्थेन शरैः सन्नतपर्वभिः}
{अमुह्यंस्तत्रतत्रैव त्रस्ता मृगगणा इव}


\twolineshloka
{ततस्त्रिगर्तराट् क्रुद्धस्तानुवाच महारथान्}
{अलं द्रुतेन वः शूरा न भयं कर्तुमर्हथ}


\threelineshloka
{शप्त्वाऽथ शपथान्घोरान्सर्वसैन्यस्य पश्यतः}
{गत्वा दौर्योधनं सैन्यं किं वै वक्ष्यथ मुख्यशः}
{}


\twolineshloka
{नावहास्याः कथं लोके कर्मणाऽनेन संयुगे}
{भवेम सहिताः सर्वे निवर्तध्वं यथाबलम्}


\twolineshloka
{एवमुक्तास्तु ते राजन्नुदक्रोशन्मुहुर्मुहुः}
{शङ्खांश्च दध्मिरे वीरा हर्षयन्तः परस्परम्}


\twolineshloka
{ततस्ते सन्न्यवर्तन्तं संशप्तकगणाः पुनः}
{नारायणाश्च गोपाला मृत्युं कृत्वाऽनिवर्तनम्}


\chapter{अध्यायः १९}
\twolineshloka
{सञ्च उवाच}
{}


\twolineshloka
{दृष्टा तु सन्निवृत्तांस्तान्संशप्तकगणान्पुनः}
{वासुदेवं महात्मानमर्जुनः समभाषत}


\twolineshloka
{चोदयाश्वान्हृषीकेश संशप्तकगणान्प्रति}
{नैते हास्यन्ति सङ्ग्रामं जीवन्त इति मे मतिः}


\threelineshloka
{पश्य मेऽस्त्रबलं घोरं बाह्वोरिष्वसनस्य च}
{अद्यैतान्पातयिष्यामि क्रुद्धो रुद्रः पशूनिव ॥सञ्जय उवाच}
{}


\twolineshloka
{ततः कृष्णः स्मितं कृत्वा प्रतिनन्द्य शिवेन तम्}
{प्रावेशयत दुर्धर्षो यत्रयत्रैच्छदर्जुनः}


\twolineshloka
{स रथो भ्राजतेऽत्यर्थमुह्यमानो रणे तदा}
{उह्यमानमिवाकाशे विमानं पाण्डुरैर्हयैः}


\twolineshloka
{मण्डलानि ततश्चक्रे गतप्रत्यागतानि चे}
{यथा शक्ररथो राजन्युद्धे देवासुरे पुरा}


\twolineshloka
{अथ नारायणाः क्रुद्धा विविधायुधपाणयः}
{छादयन्तः शरव्रातैः परिवव्रुर्धनञ्जयम्}


\twolineshloka
{अदृश्यं च मुहूर्तेन चक्रुस्ते भरतर्षभ}
{कृष्णेन सहितं युद्धे कुन्तीपुत्रं धनञ्जयम्}


\twolineshloka
{क्रुद्धस्तु फल्गुनः सङ्ख्ये द्विगुणीकृतविक्रमः}
{गाण्डीवं धनुरामृज्य तूर्णं जग्राह संयुगे}


\twolineshloka
{बद्ध्वा च भ्रुकुटिं वक्त्रे क्रोधस्य प्रतिलक्षणम्}
{देवदत्तं महाशङ्खं पूरयामास पाण्डवः}


\twolineshloka
{अथास्त्रमरिसङ्घघ्नं त्वाष्ट्रमभ्यस्यदर्जुनः}
{ततो रूपसहस्राणि प्रादुरासन्पृथक्पृथक्}


\twolineshloka
{अर्जुनप्रतिरूपैस्तैर्नानारूपैर्विमोहिताः}
{अन्योन्येनार्जुनं मत्वा स्वमात्मानं च जघ्निरे}


\twolineshloka
{अयमर्जुनोऽयं गोविन्द इमौ पाण्डवयादवौ}
{इति ब्रुवाणाः संमूढा जघ्नुरन्योन्यमाहवे}


\twolineshloka
{`अन्योन्यं समरे जघ्नुस्तावका भरतर्षभ'}
{मोहिताः परमास्त्रेण क्षयं जग्मुः परस्परम्}


\twolineshloka
{`रुधिरोक्षितगात्रास्ते प्रेक्षमाणः परस्परम्'}
{अशोभन्त रणे योधाः पुष्पिता इव किंशुकाः}


\twolineshloka
{`रुधिरोत्पीडनास्ते तु रुधिरेण समुक्षिताः}
{चन्दनस्य रसेनेव व्यभ्राजन्त रणाजिरे}


\twolineshloka
{ततः प्रसह्य बीभत्सुर्व्याक्षिपद्गाण्डिवं धनुः}
{न्यहनत्ताञ्छरैस्तीक्ष्णैस्तमः सूर्य इवांशुभिः}


\twolineshloka
{हतावशिष्टास्ते भूयः परिवार्य धनञ्जयम्}
{साश्वध्वजरथं चक्रुरदृश्यं शरवृष्टिभिः'}


\twolineshloka
{ततः शरसहस्राणि तैर्विमुक्तानि भस्मसात्}
{कृत्वा तदस्त्रं तान्वीराननयद्यमसादनम्}


\twolineshloka
{अथ प्रहस्य बीभत्सुर्ललित्थान्मालवानपि}
{मावेल्लकांस्त्रिगर्तांश्च यौधेयांश्चार्दयच्छरैः}


\twolineshloka
{ते हन्यमाना वीरेण क्षत्रियाः कालचोदिताः}
{व्यसृजञ्छरजालानि पार्थे नानाविधानि च}


\twolineshloka
{न ध्वजो नार्जुनस्तत्र न रथो न च केशवः}
{प्रत्यदृश्यत घोरेण शरवर्षेण संवृतः}


\twolineshloka
{ततस्ते लब्धलक्षत्वादन्योन्यमभिचुक्रुशुः}
{हतौ कृष्णाविति प्रीत्या वासांस्यादुधुवुस्तदा}


\twolineshloka
{भेरीमृदङ्गशङ्खांश्च दध्मुर्वीराः सहस्रशः}
{सिंहनादरवांश्चोग्रांश्चक्रिरे तत्र मारिष}


\twolineshloka
{ततः प्रसिष्विदे कृष्णः खिन्नश्चार्जुनमब्रवीत्}
{क्वासि पार्थ न पश्यामि कच्चिज्जीवसि शत्रुहन्}


\twolineshloka
{तस्य तद्भाषितं श्रुत्वा त्वरमाणो धनञ्जयः}
{वायव्यास्त्रेण तैरस्तां शरवृष्टिमपाहरत्}


\twolineshloka
{ततः संशप्तकव्रातान्साश्वद्विपरथायुधान्}
{उवाह भगवान्वायुः शुष्कपर्मचयानिव}


\twolineshloka
{उह्यमानास्तु ते राजन्बह्वशोभन्त वायुना}
{प्रडीनाः पक्षिणः काले वृक्षेभ्य इव मारिष}


\twolineshloka
{तांस्तथा व्याकुलीकृत्य त्वरमाणो धनञ्जयः}
{जघान निशितैर्बाणैः सहस्राणि शतानि च}


\twolineshloka
{शिरांसि भल्लैरहरद्बाहूनपि च सायुधान्}
{हस्तिहस्तोपमांश्चोरूञ्शरैरुर्व्यामपातयत्}


\twolineshloka
{पृष्ठच्छिन्नान्विचरणान्विमस्तकदृगङ्गुलीन्}
{नानाङ्गावयवैर्हीनांश्चकारारीन्धनञ्जयः}


\twolineshloka
{गन्धर्वनगराकारान्विधिवत्कल्पितान्रथान्}
{शरैर्विशकलीकुर्वंश्चक्रे व्यश्वरथद्विपान्}


\twolineshloka
{मुण्डतालवनानीव तत्रतत्र चकाशिरे}
{छिन्ना रथध्वजव्राताः केचित्तत्र क्वचित्क्वचित्}


\twolineshloka
{सोत्तरायुधिनो नागाः सपताकाङ्कुशध्वजाः}
{पेतुः शक्राशनिहता द्रुमवन्त इवाचलाः}


\twolineshloka
{चामरापीडकवचाः स्रस्तान्त्रनयनास्तथा}
{सारोहास्तुरगाः पेतुः पार्तबाणहताः क्षितौ}


\twolineshloka
{विप्रविद्धासिनाराचाश्छिन्नवर्मर्ष्टिशक्तयः}
{पत्तयश्चिन्नवर्माणः कृपणाः शेरते हताः}


\twolineshloka
{तैर्हतैर्हन्यमानैश्च पतद्भिः पतितैरपि}
{भ्रमद्भिर्निष्टनद्भिश्च क्रूरमायोधनं बभौ}


\twolineshloka
{रजश्च चाप्यभवद्दुर्गा कबन्धशतसङ्कुला ॥तद्बभौ रौद्रबीभत्सं बीभत्सोर्यानमाहवे}
{}


\twolineshloka
{आक्रीडमिव रुद्रस्य घ्नतः कालात्यये पशून् ॥कते वध्यमानाः पार्थेन व्याकुलाश्वरथद्विपाः}
{}


\twolineshloka
{तमेवाभिमुखाः क्षीणाः शक्रस्यातिथितां गताः ॥सा भूमीर्भरतश्रेष्ठ निहतैस्तैर्महारथैः}
{}


\twolineshloka
{आस्तीर्णा सम्बभौ सर्वा प्रेतीभूतैः समन्ततः ॥एतस्मिन्नन्तरे चैव प्रमत्ते सव्यसाचिनि}
{}


\twolineshloka
{व्यूढानीकस्ततो द्रोणो युधिष्ठिरमुपाद्रवत् ॥तं प्रत्यगृह्णंस्त्वरिता व्यूढानीकाः प्रहारिणः}
{}


% Check verse!
युधिष्ठिरं परीप्सन्तस्तदासीत्तुमुलं महत्
\chapter{अध्यायः २०}
\twolineshloka
{सञ्जय उवाच}
{}


\twolineshloka
{परिणाम्य निशां तां तु भारद्वाजो महारथः}
{उक्त्वा सुबहु राजेन्द्र वचनं वै सुयोधनम्}


\twolineshloka
{विधाय गोप्तॄन्पार्थस्य संशप्तकगणैः सह}
{निष्क्रान्ते च तदा पार्थे संशप्तकवधं प्रति}


\twolineshloka
{व्यूढानीकस्ततो द्रोणः पाण्डवानां महाचमूम्}
{अभ्ययाद्भरतश्रेष्ठ धर्मराजजिघृक्षया}


\twolineshloka
{व्यूढां दृष्ट्वा सुपर्णेन भारद्वाजस्य तां चमूम्}
{व्यूहेन मण्डलार्धेन प्रत्यव्यूहद्युधिष्ठिरः}


\twolineshloka
{मुखं त्वासीत्सुपर्णस्य भारद्वाजो महारथः}
{शिरो दुर्योधनो राजा सोदर्यैः सानुगैर्वृतः}


% Check verse!
चक्षुषी कृतवर्माऽऽसीद्गौतमश्चास्यतां वरः
\twolineshloka
{भूतशर्मा क्षेमशर्मा करकाक्षश्च वीर्यवान्}
{कलिङ्गाः सिंहलाः प्राच्याः शूरा भीरा दशेरकाः}


\threelineshloka
{शका यवनकाम्भोजास्तथा हंसपथाश्च ये}
{ग्रीवायां शूरसेनाश्च दरन्दा मद्रकेकयाः}
{गजाश्वरथपत्त्योघास्तस्थुः परमदंशिताः}


\twolineshloka
{भूरिश्रवास्तथा शल्यः सोमदत्तश्च बाह्लिकः}
{अक्षौहिण्या वृता वीरा दक्षिणं पार्श्वमास्थिताः}


\twolineshloka
{विन्दानुविन्दावावन्त्यौ काम्भोजश्च सुदक्षिणः}
{वामं पार्श्वं समाश्रित्य द्रोणपुत्राग्रतः स्थिताः}


\twolineshloka
{पृष्ठे कलिङ्गाः साम्बष्ठा मागधाः पौण्ड्रमद्रकाः}
{गान्धाराः शकुनाः प्राच्याः पार्वतीया वसातयः}


\twolineshloka
{पुच्छे वैकर्तनः कर्णः सपुत्रज्ञातिबान्धवः}
{महत्या सेनया तस्थौ नानाजनपदोत्थया}


\twolineshloka
{जयद्रथो भीमरथः सम्पातिर्ऋषभो जयः}
{भूमिञ्जयो वृषक्राथो नैषधश्च महाबलः}


\twolineshloka
{वृता बलेन महता ब्रह्मलोकपरिष्कृताः}
{व्यूहस्योरसि ते राजन्स्थिता युद्धविशारदाः}


\twolineshloka
{द्रोणेन विहितो व्यूहः पदात्यश्वरथद्विपैः}
{वातोद्धूतार्णवाकारः प्रनृत्त इव लक्ष्यते}


\twolineshloka
{तस्य पक्षप्रपक्षेभ्यो निष्पतन्ति युयुत्सवः}
{सविद्युत्स्तनिता मेघाः सर्वदिग्भ्यः इवोष्णगे}


\twolineshloka
{तस्य प्राग्ज्योतिषो मध्ये विधिवत्कल्पितं गजम्}
{आस्थितः शुशुभे राजन्नंशुमानुदये यथा}


\twolineshloka
{माल्यदामवता राजञ्श्वेतच्छत्रेण धार्यता}
{कृत्तिकायोगयुक्तेन पौर्णमास्यामिवेन्दुना}


\twolineshloka
{नीलाञ्जनचयप्रख्यो मदान्धो द्विरदो बभौ}
{अतिवृष्टो महामेघैर्यथा स्यात्पर्वतो महान्}


\twolineshloka
{नानानृपतिभिर्वीरै र्विविधायुधभूषणैः}
{समन्वितः पार्वतीयैः शक्रो देवगणैरिव}


\twolineshloka
{ततो युधिष्ठिरः प्रेक्ष्य व्यूहं तमतिमानुषम्}
{अजय्यमरिभिः सङ्ख्ये पार्षतं वाक्यमब्रवीत्}


\threelineshloka
{ब्राह्मणस्य वशं नाहमियामद्य यथा प्रभो}
{पारावतसवर्णाश्व तथा नीतिर्विधीयताम् ॥धृष्टद्युम्न उवाच}
{}


\twolineshloka
{द्रोणस्य यतमानस्य वशं नैष्यसि सुव्रत}
{अहमावारयिष्यामि द्रोणमद्य सहानुगम्}


\threelineshloka
{मयि जीवति कौरव्य नोद्वेगं कर्तुमर्हसि}
{न हि शक्तो रणे द्रोणो विजेतुं मां कथञ्चन ॥सञ्जय उवाच}
{}


\twolineshloka
{एवमुक्त्वा किरन्बाणान्द्रुपदस्य सुतो बली}
{पारावतसवर्णाश्वः स्वयं द्रोणमुपाद्रवत्}


\twolineshloka
{अनिष्टदर्शनं दृष्ट्वा धृष्टद्युम्नमवस्थितम्}
{क्षणेनैवाभवद्द्रोणो नातिहृष्टमना इव}


% Check verse!
`स हि जातो महाराज द्रोणस्य निधनं प्रति
\twolineshloka
{मर्त्यधर्मतया तस्माद्भारद्वाजो व्यमुह्यत}
{नाशकत्तत्र तत्किञ्चिदनीकं प्रतिवीक्षितुम्}


\threelineshloka
{ततः किरन्निषूंस्तीक्ष्णान्द्रुपदस्य वरूथिनीम्}
{भराद्वाजो ययौ तूर्णं पार्षतं वर्जयन्युधि}
{द्रुपदस्य महत्सैन्यं दारयामास ब्राह्मणः'}


\twolineshloka
{तं तु सम्प्रेक्ष्य पुत्रस्ते दुर्मुखः शत्रुकर्षणः}
{प्रियं चिकीर्षुर्द्रोणस्य धृष्टद्युम्नमवारयत्}


\twolineshloka
{स सम्प्रहारस्तुमुलः सुघोरः समपद्यत}
{पार्षतस्य च शूरस्य दुर्मुखस्य च भारत}


\twolineshloka
{पार्षतः शरजालेन क्षिप्रं प्रच्छाद्य दुर्मुखम्}
{भारद्वाजं शरौघेण महता समवारयत्}


\twolineshloka
{द्रोणावारितं दृष्ट्वा भृशायस्तस्तवात्मजः}
{नानालिङ्गैः शरव्रातैः पार्षतं सममोहयत्}


\twolineshloka
{तयोर्विषक्तयोः सङ्ख्ये पाञ्चाल्यकुरुमुख्ययोः}
{द्रोणो यौधिष्ठिरं सैन्यं बहुधा व्यधमच्छरैः}


\twolineshloka
{अनिलेन यथाऽभ्राणि विच्छिन्नानि समन्ततः}
{तथा पार्थस्य सैन्यानि विच्छिन्नानि क्वचित्क्वचित्}


\twolineshloka
{मुहूर्तमिव तद्युद्धमासीन्मधुरदर्शनम्}
{तत उन्मत्तवद्राजन्निर्मर्यादमवर्तत}


\threelineshloka
{`नैव खं न दिशो भूमिर्बभासे न च भास्करः'}
{नैव स्वे न परे राजन्नाज्ञायन्त परस्परम्}
{अनुमानेन संज्ञाभिर्युद्धं तत्समवर्तत}


\twolineshloka
{चूडामणिषु निष्केषु भूषणेष्वपि वर्मसु}
{तेषामादित्यवर्णाभा रश्मयः प्रचकाशिरे}


\twolineshloka
{तत्प्रकीर्णपताकानां रथवारणवाजिनाम्}
{बलाकाशबलाभ्रामं ददृशे रूपमाहवे}


\twolineshloka
{नरानेव नरा जघ्नरुदग्राश्च हया हयान्}
{रथांश्च रथिनो जघ्नुर्वारणा वरवारणान्}


\twolineshloka
{समुच्छ्रितपताकानां गजानां परमद्विपैः}
{क्षणेन तुमुलो घोरः सङ्ग्रामः समपद्यत}


\twolineshloka
{तेषां संसक्तगात्राणां कर्षतामितरेतरम्}
{दन्तसङ्घातसङ्घर्षात्सधूमोऽग्निरजायत}


\twolineshloka
{विप्रकीर्णपताकास्ते विषाणजनिताग्नयः}
{बभूवुः खं समासाद्य सविद्युत इवाम्बुदाः}


\twolineshloka
{विक्षिपद्भिर्नदद्भिश्च निपतद्भिश्च वारणैः}
{सम्बभूव मही कीर्णा मेघैर्द्यौरिव वार्षिकी}


\twolineshloka
{तेषामाहन्यमानानां बाणतोमरऋष्टिभिः}
{वारणानां रवो जज्ञे मेघानामिव सम्पुवे}


\twolineshloka
{तोमराभिहताः केचिद्बाणैश्च परमद्विपाः}
{वित्रेसुः सर्वनागानां शब्दमेवापरेऽसृजन्}


\twolineshloka
{विषाणाभिहताश्चापि केचित्तत्र गजा गजैः}
{चक्रुरार्तस्वनं घोरमुत्पातजलदा इव}


\twolineshloka
{प्रतीपाः क्रियमाणाश्च वारणा वरवारणैः}
{उन्मथ्य पुनराजग्मुः प्रेरिताः परमाङ्कुशैः}


\twolineshloka
{महामात्रैर्महामात्रास्ताडिताः शरतोमरैः}
{गजेभ्यः पृथिवीं जग्मुर्मुक्तप्रहरणाङ्कुशाः}


\twolineshloka
{निर्मनुष्याश्च मातङ्गा विनदन्तस्ततस्ततः}
{छिन्नाभ्राणीव सम्पेतुः सम्प्रविश्य परस्परम्}


\twolineshloka
{हतान्परिवहन्तश्च पतितान्पतितायुधान्}
{दिशो जग्मुर्महानागाः केचिदेकचरा इव}


\twolineshloka
{ताडितास्ताड्यमानाश्च तोमरर्ष्टिपरश्वथैः}
{पेतुरार्तस्वनं कृत्वा तदा विशसने गजाः}


\twolineshloka
{तेषां शैलोपमैः कायैर्निपतद्भिः समन्ततः}
{आहता सहसा भूमिश्चकम्पे च ननाद च}


\twolineshloka
{सादितैः सगजारोहैः सपताकैः समन्ततः}
{मातङ्गैः शुशुभे भूमिर्विकीर्णैरिव पर्वतैः}


\twolineshloka
{गजस्थाश्च महामात्रा निर्भिन्नहृदया रणे}
{रथिभिः पातिता भल्लैर्विकीर्णाङ्कुशतोमराः}


\twolineshloka
{क्रौञ्चवद्विनदन्तोऽन्ये नाराचाभिहता गजाः}
{परान्स्वांश्चापि मृद्गन्तः परिपेतुर्दिशो दश}


\twolineshloka
{गजाश्वरथयोधानां शरीरौघसमावृता}
{बभूव पृथिवी राजन्मांसशोणितकर्दमा}


\twolineshloka
{प्रमथ्य च विषाणाग्रैः समुत्क्षिप्ताश्च वारणैः}
{सचक्राश्च विचक्राश्च रथैरेव महारथाः}


\twolineshloka
{रथाश्च रथिभिर्हीना निर्मनुष्याश्च वाजिनः}
{हतारोहाश्च मातङ्गा दिशो जग्मुर्भयातुराः}


\twolineshloka
{जघानात्र पिता पुत्रं पुत्रश्च पितरं तथा}
{इत्यासीत्तुमुलं युद्धं न प्राज्ञायत किञ्चन}


\twolineshloka
{आगुल्फेभ्योऽवसीदन्ते नरा लोहितकर्दमैः}
{दीप्यमानैः परिक्षिप्ता दावैरिव महाद्रुमाः}


\twolineshloka
{शोणितैः सिच्यमानानि वस्त्राणि कवचानि च}
{छत्राणि च पताकाश्च सर्वं रक्तमदृश्यत}


\twolineshloka
{हयौघाश्च रथौघाश्च नरौघाश्च निपातिताः}
{संक्षुण्णाः पुनरावृत्य बहुधा रथनेमिभिः}


\twolineshloka
{सगजौघमहावेगः परासुनरशैवलः}
{रथौघतुमुलावर्तः प्रबभौ सैन्यसागरः}


\twolineshloka
{तं वाहनमहानौभिर्योधा जयधनैषिणः}
{अवगाह्याथ मज्जन्तो नैव मोहं प्रचक्रिरे}


\twolineshloka
{शरवर्षाभिवृष्टेषु योधेष्वञ्चितलक्ष्मसु}
{न तेष्वचित्ततां लेभे कश्चिदाहतलक्षणः}


\twolineshloka
{वर्तमाने तथा युद्धे घोररूपे भयङ्करे}
{मोहयित्वा परान्द्रोणो युधिष्ठिरमुपाद्रवत्}


\chapter{अध्यायः २१}
\twolineshloka
{सञ्जय उवाच}
{}


\twolineshloka
{ततो युधिष्ठिरो द्रोणं दृष्ट्वान्तिकमुपागतम्}
{महता शरवर्षेण प्रत्यगृह्णादभीतवत्}


\twolineshloka
{ततो हलहलाशब्द आसीद्यौधिष्ठिरे बले}
{जिघृक्षति महासिंहे गजानामिव यूथपम्}


\twolineshloka
{दृष्ट्वा द्रोणं ततः शूरः सत्यजित्सत्यविक्रमः}
{युधिष्ठिरमभिप्रेप्सुराचार्यं समुपाद्रवत्}


\twolineshloka
{तत आचार्यपाञ्चाल्यौ युयुधाते महाबलौ}
{विक्षोभयन्तौ तत्सैन्यमिन्द्रवैरोचनाविव}


\twolineshloka
{ततो द्रोणं महेष्वासः सत्यजित्सत्यविक्रमः}
{अविध्यन्निशिताग्रेण परमास्त्रं विदर्शयन्}


\twolineshloka
{तथास्य सारथेः पञ्च शरान्सर्पविषोपमान्}
{अमुञ्चदन्तकप्रख्यान्संमुमोहास्य सारथिः}


\twolineshloka
{अथास्य सहसाऽविध्यद्धयान्दशभिराशुगैः}
{दशभिर्दशभिः क्रुद्ध उभौ च पार्ष्णिसारथी}


\twolineshloka
{मण्डलं तु समावृत्य विचरन्पृतनामुखे}
{ध्वजं चिच्छेद च क्रुद्धो द्रोणस्यामित्रकर्षणः}


\twolineshloka
{द्रोणस्तु तत्समालोक्य चरितं तस्य संयुगे}
{मनसा चिन्तयामास प्राप्तकालमरिन्दमः}


\twolineshloka
{ततः सत्यजितं तीक्ष्णैर्दशभिर्मर्मभेदभिः}
{अविध्यच्छीघ्रमाचार्यश्छित्त्वाऽस्य सशरं धनुः}


\twolineshloka
{स शीघ्रतरमादाय धनुरन्यत्प्रतापवान्}
{द्रोणमब्यहनद्राजंस्त्रिंशता कङ्कपत्रिभिः}


\twolineshloka
{दृष्ट्वा सत्यजिता द्रोणं ग्रस्यमानमिवाहवे}
{वृकः शरशतैस्तीक्ष्णैः पाञ्चाल्यो द्रोणमार्दयत्}


\twolineshloka
{सञ्छाद्यमानं समरे द्रोणं दृष्ट्वा महारथम्}
{चुक्रुशुः पाण्डवा राजन्वस्त्राणि दुधुवुश्च ह}


\twolineshloka
{वृकस्तु परमक्रुद्धो द्रोणं षष्ट्या स्तनान्तरे}
{विव्याध बलवान्राजंस्तदद्भुतमिवाभवत्}


\twolineshloka
{द्रोणस्तु शरवर्षेण च्छाद्यमानो महारथः}
{वेगं चक्रे महावेगः क्रोधादुद्वृत्य चक्षुषी}


\twolineshloka
{ततः सत्यजितश्चापं छित्त्वा द्रोणो वृकस्य च}
{षड्भिः ससूतं सहयं शरैर्द्रोणोऽवधीद्वृकम्}


\twolineshloka
{अथान्यद्धनुरादाय सत्यजिद्वेगवत्तरम्}
{साश्वं ससूतं विशिखैर्द्रोणं विव्याध सध्वजम्}


\twolineshloka
{स तन्न ममृषे द्रोणः पाञ्चाल्येनार्दितो मृधे}
{ततस्तस्य विनाशाय सत्वरं व्यसृजच्छरान्}


\twolineshloka
{हयान्ध्वजं धनुर्मुष्टिमुभौ च पार्ष्णिसारथी}
{अवाकिरत्ततो द्रोणः शरवर्षैः सहस्रशः}


\twolineshloka
{तथा सञ्छिद्यमानेषु कार्मुकेषु पुनःपुनः}
{पाञ्चाल्यः परमास्त्रज्ञः शोणाश्वं समयोधयत्}


\twolineshloka
{स सत्यजितमालोक्य तथोदीर्णं महाहवे}
{अर्धचन्द्रेण चिच्छेद शिरस्तस्य महात्मनः}


\twolineshloka
{तस्मिन्हते महामात्रे पाञ्चालानां महारथे}
{अपायाज्जवनैरश्वैर्द्रोणात्त्रस्तो युधिष्ठिरः}


\twolineshloka
{पाञ्चालाः केकया मात्स्याश्चेदिकारूशकोसलाः}
{युधिष्ठिरमभीप्सन्तो दृष्ट्वा द्रोणमुपाद्रवन्}


\twolineshloka
{ततो युधिष्ठिरं प्रेप्सुराचार्यः शत्रुपूगहा}
{व्यधमत्तान्यनीकानि तूलराशिमिवानलः}


\twolineshloka
{निर्दहन्तमनीकानि तानि तानि पुनःपुनः}
{द्रोणं मत्स्यादवरजः शतानीकोऽभ्यवर्तत}


\twolineshloka
{सूर्यरश्मिप्रतीकाशैः कर्मारपरिमार्जितैः}
{षड्भिः ससूतं सहयं द्रोणं विद्ध्वाऽनदद्भृशम्}


\twolineshloka
{क्रूराय कर्मणे युक्तश्चिकीर्षुः कर्म दुष्करम्}
{अवाकिरच्छरशतैर्भारद्वाजं महारथम्}


\twolineshloka
{तस्य नानदतो द्रोणः शिरः कायात्सकुण्डलम्}
{क्षुरेणापाहरत्तुर्णं ततो मत्स्याः प्रदुद्रुवुः}


\twolineshloka
{मत्स्याञ्जित्वाऽजयच्चेदीन्करूशान्केकयानपि}
{पाञ्चालान्सृञ्जयान्पाण्डून्भारद्वाजः पुनःपुनः}


\twolineshloka
{तं दहन्तमनीकानि क्रुद्धमग्निं यथा वनम्}
{दृष्ट्वा रुक्मरथं वीरं समकम्पन्त सृञ्जयाः}


\twolineshloka
{उभाभ्यां सन्दधानस्य धनुरस्याशुकारिणः}
{ज्याघोषो निघ्नतोऽमित्रान्दिक्षुसर्वासु शुश्रुवे}


\twolineshloka
{नागानश्वान्पदातींश्च रथिनो गजसादिनः}
{रौद्रा हस्तवता मुक्ताः प्रमथ्नन्ति स्म सायकाः}


\twolineshloka
{नानद्यमानः पर्जन्यो मिश्रवातो हिमात्यये}
{अश्मवर्षमिवावर्षन्परेषां भयमादधत्}


\twolineshloka
{सर्वा दिशः समचरत्सैन्यं विक्षोभयन्निव}
{बली शूरो महेष्वासो मित्राणामभयङ्करः}


\twolineshloka
{तस्य विद्युदिवाभ्रेषु चापं हेमपरिष्कृतम्}
{दिक्षु सर्वासु पश्यामो द्रोणस्यामिततेजसः}


\twolineshloka
{शोभमानां ध्वजे चास्य वेदीमद्राक्ष्य भारत}
{हिमवच्छिखराकारं चरतः संयुगे भृशम्}


\twolineshloka
{द्रोणस्तु पाण्डवानीके चकार कदनं महत्}
{यथा दैत्यगणे विष्णुः सुरासुरनमस्कृतः}


\twolineshloka
{स शूरः सत्यवाक्प्राज्ञो बलवान्सत्यविक्रमः}
{महानुभावः कल्पान्ते रौद्रां भीरुविभीषणाम्}


% Check verse!
कवचोर्मिध्वजावर्तां मर्त्यकूलापहारिणीम् ॥गजवाजिमहाग्राहामसिमीनां दुरासदाम्
\twolineshloka
{वीरास्यिशर्करां रौद्रां भेरीमुरजकच्छपाम्}
{चर्मवर्मप्लवां घोरां केशशैवलशाद्वलाम्}


\twolineshloka
{शरौघिणीं धनुःस्रोतां बाहुपन्नगसङ्कुलाम्}
{रणभूमिवहां तीव्रां कुरुसृञ्जयवाहिनीम्}


\twolineshloka
{मनुष्यशीर्षपाषाणां शक्तिमीनां गदोडुपाम्}
{उष्णीषफेनवसनां विकीर्णान्त्रसरीसृपाम्}


\twolineshloka
{वीरापहारिणीमुग्रां मांसशोमितकर्दमाम्}
{हस्तिग्राहां केतुवृक्षां क्षत्रियाणां निमज्जनीम्}


\twolineshloka
{क्रूरां शरीरसङ्घाटां सादिनक्रां दुरत्ययाम्}
{द्रोणः प्रावर्तयत्तत्र नदीमन्तकगामिनीम्}


\twolineshloka
{क्रव्यादगणसञ्जुष्टां श्वशृगालगणायुताम्}
{निषेवितां महारौद्रैः पिशिताशैः समन्ततः}


\twolineshloka
{तं दहन्तमनीकानि रथोदारं कृतान्तवत्}
{सर्वतोऽभ्यद्रवन्द्रोणं कुन्तीपुत्रपुरोगमाः}


\twolineshloka
{ते द्रोणं सहिताः शूराः सर्वतः प्रत्यवारयन्}
{गभस्तिभिरिवादित्यं तपन्तं भुवनं यथा}


\twolineshloka
{तं तु शूरं महेष्वासं तावकाऽभ्युद्यतायुधाः}
{राजानो राजपुत्राश्च समन्तात्पर्यवारयन्}


\twolineshloka
{शिखण्डी तु ततो द्रोणं पञ्चभिर्नतपर्वभिः}
{क्षत्रवर्मा च विंशत्या वसुदानश्च पञ्चभिः}


\twolineshloka
{उत्तमौजास्त्रिभिर्बाणैः क्षत्रदेवश्च सप्तभिः}
{सात्यकिश्च शतेनाजौ युधामन्युस्तथाऽष्टभिः}


\twolineshloka
{युधिष्ठिरो द्वादशभिर्द्रोणं विव्याध सायकैः}
{धृष्टद्युम्नश्च दशभिश्चेकितानस्त्रिभिः शरैः}


\twolineshloka
{ततो द्रोणः सत्यसन्धः प्रभिन्न इव कुञ्जरः}
{अभ्यतीत्य स्थानीकं दृढसेनमपातयत्}


\twolineshloka
{ततो राजानमासाद्य प्रहरन्तमभीतवत्}
{अविध्यन्नवभिः क्षेमं स हतःप्रापतद्रथात्}


\twolineshloka
{स मध्यं प्राप्य सैन्यानां सर्वाः प्रविचरन्दिशः}
{त्राता ह्यभवदन्येषां न त्रातव्यः कथंचन}


\twolineshloka
{शिखण्डिनं द्वादशभिर्विंशत्या चोत्तमौजसम्}
{वसुदानं च भल्लेन प्रैषयद्यमसादनम्}


\twolineshloka
{अशीत्या क्षत्रवर्माणं षड्विंशत्या सुदक्षिणम्}
{क्षत्रदेवं तु भल्लेन रथनीडाडपातयत्}


\twolineshloka
{युधामन्युं चतुःषष्ट्या त्रिंशता चैव सात्यकिम्}
{विद्ध्वा रुक्मरथस्तूर्णं युधिष्ठिरमुपाद्रवत्}


\twolineshloka
{ततो युधिष्ठिरः क्षिप्रं गुरुतो राजसत्तमः}
{अपायाञ्जवनैरश्वैः पाञ्चाल्यो द्रोणमभ्ययात्}


\threelineshloka
{तं द्रोणः सधनुष्कं तु साश्वयन्तारमाक्षिणोत्}
{स हतः प्रापतद्भूमौ रथाज्ज्योतिरिवाम्बरात्}
{`हत्वा तं विबभौ द्रोणः कुरुभिः परिवारितः}


\twolineshloka
{वार्धक्षेमिस्तु वार्ष्णेयो द्रोणं विद्ध्वा शरोत्तमैः}
{नवभिश्च महीपालः पुनर्विव्याध पञ्चभिः}


\twolineshloka
{स सात्यकिश्चतुष्षष्ट्या राजन्विव्याध सायकैः}
{सुचित्रो दशभिर्बाणैर्द्रोणं विद्व्वाऽनदद्बली}


\twolineshloka
{तं द्रोणः समरे राजञ्शरवर्षैरवाकिरत्}
{अपातयत्तयो द्रोणः सुचित्रं सहसारथिम्}


\twolineshloka
{साश्वं च समरे राजन्हतो वै प्रापतत्क्षितौ}
{पात्यमानो महाराज बभौ ज्योतिरिवाम्बरात्}


\twolineshloka
{तस्मिन्हते राजपुत्रे पाञ्चालानां यशस्करे}
{हत द्रोणं हत द्रोणमित्यासीन्निःस्वनो महान्}


\twolineshloka
{तांस्तथा भृशसंरब्धान्पाञ्चालान्मत्स्यकेकयान्}
{सृञ्जयान्पाण्डवांश्चैव द्रोणो व्यक्षोभयद्बली}


\twolineshloka
{सात्यकिं चेकितानं च धृष्टद्युम्नशिखण़्डिनौ}
{वार्धक्षेमिं चैत्रसेनिं सेनाबिन्दुं सुवर्चसम्}


\twolineshloka
{एतांश्चान्यांश्च सुबहून्नानाजनपदेश्वरान्}
{सर्वान्द्रोणोऽजयद्युद्धे कुरुभिः परिवारितः}


\twolineshloka
{तावकाश्च महाराज जयं लब्ध्वा महाहवे}
{पाण्डवेयान्रणे जघ्नुर्द्रवमाणान्समन्ततः}


\twolineshloka
{ते दानवा इवेन्द्रेण वध्यमाना महात्मना}
{पाञ्चालाः केकया मात्स्याः समकम्पन्त भारत}


\chapter{अध्यायः २२}
\twolineshloka
{धृतराष्ट्र उवाच}
{}


\twolineshloka
{भारद्वाजेन भग्नेषु पाण्डवेषु महामृधे}
{पाञ्चालेषु च सर्वेषु कश्चिदन्योऽभ्यवर्तत}


\twolineshloka
{आर्यां युद्धे मतिं कृत्वा क्षत्रियाणां यशस्करीम्}
{असेवितां कापुरुषैः सेवितां पुरुषर्षभैः}


\twolineshloka
{स हि वीरोन्नतः शूरो यो भग्नेषु निवर्तते}
{अहो नासीत्पुमान्कश्चिद्दृष्ट्वा द्रोणं व्यवस्थितम्}


\twolineshloka
{जृम्भमाणमिव व्याघ्रं प्रभिन्नमिव कुञ्जरम्}
{त्यजन्तमाहवे प्राणान्सन्नद्धं चित्रयोधिनम्}


\twolineshloka
{महेष्वासं नरव्याघ्रं द्विषतां भयवर्धनम्}
{कृतज्ञं सत्यनिरतं दुर्योधनहितैषिणम्}


\threelineshloka
{भारद्वाजं तथाऽनीके दृष्ट्वा शूरमवस्थितम्}
{के शूराः सन्न्यवर्तन्त तन्ममाचक्ष्व सञ्जय ॥सञ्चय उवाच}
{}


\twolineshloka
{तान्दृष्ट्वा चलितान्सङ्ख्ये प्रणुन्नान्द्रोणसायकैः}
{पाञ्चालान्पाण्वान्मात्स्यान्सृञ्जयांश्चेदिकेकयान्}


\twolineshloka
{द्रोणचापविमुक्तेन शरौघेणाशु हारिणा}
{सिन्धोरिव महौधेन हियमाणान्यथा प्लवान्}


\twolineshloka
{कौरवाः सिंहनादेन नानावाद्यस्वनेन च}
{रथद्विपनरांश्चैव सर्वतः समवारयन्}


\threelineshloka
{तान्पश्यन्सैन्यमध्यस्थो राजा स्वजनसम्वृतः}
{दुर्योधनोऽब्रवीत्कर्णं प्रहृष्टः प्रहसन्निव ॥दुर्योधन उवाच}
{}


\twolineshloka
{पश्य राधेय पाञ्चालान्प्रणुन्नान्द्रोणसायकैः}
{सिंहेनेव मृगान्वन्यांस्त्रासितान्दृढधन्वना}


\twolineshloka
{नैते जातु पुनर्युद्धमीहेयुरिति मे मतिः}
{यथा तु भग्ना द्रोणेन वातेनेव महाद्रुमाः}


\twolineshloka
{अर्द्यमानाः शरैरेते रुक्मपुङ्खैर्महात्मना}
{पथा नैकेन गच्छन्ति घूर्णमानास्ततस्ततः}


\twolineshloka
{सन्निरुद्धाश्च कौरव्यैर्द्रोणेन च महात्मना}
{एतेऽन्ये मण्डलीभूताः पावकेनेव कुञ्जराः}


\twolineshloka
{भ्रमरैरिव चाविष्टा द्रोणस्य निशितैः शरैः}
{अन्योन्यं समलीयन्त पलायनपरायणाः}


\twolineshloka
{एष भीमो महाक्रोधी हीनः पाण्डवसृञ्जयैः}
{मदीयैरावृतो योधैः कर्ण नन्दयतीव माम्}


\threelineshloka
{व्यक्तं द्रोणमयं लोकमद्य पश्यति दुर्मतिः}
{निराशो जीवितान्नूनमद्य राज्याच्च पाण्डवः ॥कर्ण उवाच}
{}


\twolineshloka
{नैष जातु महाबाहुर्जीवन्नाहवमुत्सृजेत्}
{न चेमान्पुरुषव्याघ्र सिंहनादान्सहिष्यति}


\twolineshloka
{न चापि पाण्डवा युद्धे भज्येरन्निति मे मतिः}
{शूराश्च बलवन्तश्च कृतास्त्रा युद्धदुर्मदाः}


\twolineshloka
{विषाग्निद्यूतसङ्क्लेशान्वनवासं च पाण्डवाः}
{स्मरमाणा न हास्यन्ति सङ्ग्राममिति मे मतिः}


\twolineshloka
{निवृत्तो हि महाबाहुरमितौजा वृकोदरः}
{वरान्वरान्हि कौन्तेयो रथोदारान्हनिष्यति}


\twolineshloka
{असिना धनुषा शक्या हयैर्नागैर्नरै स्थैः}
{आयसेन च दण्डेन व्रातान्व्रातान्हनिष्यति}


\twolineshloka
{तमेनमनुवर्तन्ते सात्यकिप्रमुखा रथाः}
{पाञ्चालाः केकया मास्त्याः पाश्डवाश्च विशेषतः}


\twolineshloka
{शूराश्च बलवन्तश्च विक्रान्ताश्च महारथाः}
{विनिघ्नन्तश्च भीमेन संरब्धेनाभिचोदिताः}


\twolineshloka
{ते द्रोणमभिवर्तन्ते सर्वतः कुरुपुङ्गवाः}
{वृकोदरं परीप्सन्तः सूर्यमभ्रगणा इव}


\twolineshloka
{`समरेषु तु निर्दिष्टाः पाण्डवाः कृष्णबान्धवाः}
{पाञ्चालाः केकया मात्स्याः पाण्डवेयाश्च सर्वशः}


\twolineshloka
{शूराश्च बलवन्तश्च विक्रान्ताश्च महारथाः}
{हीमन्तः शत्रुमरणे निपुणाः पुण्यलक्षणाः}


\twolineshloka
{बहवः पार्थिवा राजंस्तेषां वशगता रणे}
{मावमंस्थाः पाण्डवांस्त्वं नारायणपुरोगमान्}


\twolineshloka
{एकायनगता ह्येते पीडयेयुर्यतव्रतम्}
{अरक्षमाणं शलभा यथा दीपं मुमूर्षवः}


\twolineshloka
{असंशयं कृतास्त्राश्च पर्याप्ताश्चापि वारणे}
{अतिभारमहं मन्ये भारद्वाजे समाहितम्}


\threelineshloka
{ते शीघ्रमनुगच्छामो यत्र द्रोणो व्यवस्थितः}
{कोका इव महानागं मा वै हन्युर्यतव्रतम् ॥सञ्जय उवाच}
{}


\twolineshloka
{राधेयस्य वचः श्रुत्वा राजा दुर्योधनस्ततः}
{भ्रातृभिः सहितो राजन्प्रायाद्द्रोणरथं प्रति}


\twolineshloka
{तत्रारावो महानासीदेकं द्रोणं जिघांसताम्}
{पाण्डवानां निवृत्तानां नानावर्णैर्हयोत्तमैः}


\chapter{अध्यायः २३}
\twolineshloka
{धृतराष्ट्र उवाच}
{}


\threelineshloka
{सर्वेषामेव मे ब्रूहि रथचिह्वानि सञ्जय}
{ये द्रोणमभ्यवर्तन्त क्रुद्धा भीमपुरोगमाः}
{`तान्यहं श्रोतुमिच्छामि विस्तरेण पृथक्पृथक्}


\threelineshloka
{दूयते मे मनस्तात द्रोणं प्रति परन्तपम्}
{श्रुत्वा भीष्मस्य निधनं तद्वदेतद्भविष्यति' ॥सञ्जय उवाच}
{}


\twolineshloka
{ऋक्षवर्णैर्हयैर्दृष्ट्वा व्यायच्छन्तं वृकोदरम्}
{रजताश्वस्ततः शूरः शैनेयः सन्न्यवर्तत}


\twolineshloka
{सारङ्गाश्वो युधामन्युः स्वयं प्रत्वरयन्हयान्}
{पर्यवर्तत दुर्धर्षः क्रुद्धो द्रोणरथं प्रति}


\twolineshloka
{पारावतसवर्णैस्तु हेमभाण्डैर्महाजवैः}
{पाञ्चालराजस्य सुतो धृष्टद्युम्नो न्यवर्तत}


\twolineshloka
{पितरं तु परिप्रेप्सुः क्षत्रधर्मा यतव्रतः}
{सिद्धिं चास्य परां काङ्क्षञ्शोणाश्वः सन्न्यवर्तत}


\twolineshloka
{पद्मपत्रनिभांश्चाश्वान्मल्लिकाक्षान्स्वलङ्कृतान्}
{शैखण़्डिः क्षत्रदेवस्तु स्वयं प्रत्वरयन्ययौ}


\twolineshloka
{दर्शनीयास्तु काम्भोजाः शुकपत्रपरिच्छदाः}
{वहन्तो नकुलं शीघ्रं तावकानाभिदुद्रुवुः}


\twolineshloka
{कृष्णास्तु मेघसङ्काशा अवहन्नुत्तमौजसम्}
{दुर्धर्षायाभिसन्धाय क्रुद्धं युद्धाय भारत}


\twolineshloka
{तथा तित्तिरिकल्माषा हया वातसमा जवे}
{अवहंस्तुमुले युद्धे सहदेवमुदायुधम्}


\twolineshloka
{दन्तवर्णास्तु राजानं कालवाला युधिष्ठिरम्}
{भीमवेगा नरव्याघ्रमवहन्वातरंहसः}


\twolineshloka
{हेमोत्तमप्रतिच्छन्नैर्हयैर्वातसमैर्जवे}
{अभ्यवर्तन्त सैन्यानि सर्वाण्येव युधिष्ठिरम्}


\twolineshloka
{राज्ञस्त्वनन्तरो राजा पाञ्चाल्यो द्रुपदोऽभवत्}
{जातरूपमयच्छत्रः सर्वैस्तैरभिरक्षितः}


\twolineshloka
{ललामैर्हरिभिर्युक्तः सर्वशब्दक्षमैर्युधि}
{राज्ञं मध्ये महेष्वासः शान्तभीरभ्यवर्तत}


\threelineshloka
{तं विराटोऽन्वयाच्छीघ्रं सह सर्वैर्महारथैः}
{केकयाश्च सिखण्डी च धृष्टकेतुस्तथैव च}
{स्वैः स्वैः सैन्यैः परिवृता मात्स्यराजानमन्वयुः}


\twolineshloka
{तं तु पाटलिपुष्पाणां समवर्णा हयोत्तमाः}
{वहमाना व्यराजन्त मास्त्यस्यामित्रघातिनः}


\twolineshloka
{हरिद्रासमवर्णास्तु जवना हेममालिनः}
{पुत्रं विराटराजस्य सत्वरं समुदावहन्}


\twolineshloka
{इन्द्रगोपकवर्णैश्च भ्रातरः पञ्च केकयाः}
{जातरूपसमाभासाः सर्वे लोहितकध्वजाः}


\twolineshloka
{ते हेममालिनः शूराः सर्वे युद्धविशारदाः}
{वर्षन्त इव जीमूताः प्रत्यदृश्यन्त दंशिताः}


\twolineshloka
{आमपात्रनिकाशास्तु पाञ्चाल्यममितौजसम्}
{दत्तास्तुम्बुरुणा दिव्याः शिखण्डिनमुदावहन्}


\twolineshloka
{तथा द्वादशसाहस्राः पाञ्चालानां महारथाः}
{तेषां तु षट््सहस्राणि ये शिखण्डिनमन्वयुः}


\twolineshloka
{पुत्रं तु शिशुपालस्य नरसिंहस्य मारिष}
{आक्रीडन्तो वहन्ति स्म सारङ्गशबला हयाः}


\twolineshloka
{धृष्टकेतुस्तु चेदीनामृषभोऽतिबलोदितः}
{काम्भोजैः शबलैरश्वैरभ्यवर्तत दुर्जयः}


\twolineshloka
{बृहत्क्षत्रं तु कैकेयं सुकुमारं हयोत्तमाः}
{पलालधूमसङ्काशाः सैन्धवाः शीघ्रमावहन्}


\twolineshloka
{मल्लिकाक्षाः पद्मवर्णा बाह्लिजाताः स्वलङ्कृताः}
{शूरं शिखण्डिनः पुत्रमृक्षदेवमुदावहन्}


\twolineshloka
{रुक्मभाण्डप्रतिच्छन्नाः कौशेयसदृशा हयाः}
{क्षमावन्तोऽवहन्सङ्ख्ये सेनाबिन्दुमरिन्दमम्}


\twolineshloka
{युवानमवहन्युद्धे क्रौञ्चवर्णा हयोत्तमाः}
{काश्यस्याभिभुवः पुत्रं सुकुमारं महारथम्}


\twolineshloka
{श्वेतास्तु प्रतिविन्ध्यन्तं कृष्णग्रीवा मनोजवाः}
{यन्तुः प्रेष्यकरा राजन्राजपुत्रमुदावहन्}


\twolineshloka
{सुतसोमं तु योधाग्र्यं भीमपुत्रं महाबलम्}
{माषपुष्पसवर्णास्तमवहन्वाजिनो रणे}


\twolineshloka
{सहस्रसोमप्रतिमो बभूव पुरे कुरूणामुदयेन्दुनाम्नि}
{तस्मिञ्जातः सोमसंक्रन्दमध्ये यस्मात्तस्मात्सुतसोमोऽभवत्सः}


\twolineshloka
{नाकुलिं तु शतानीकं शालपुष्पनिभा हयाः}
{आदित्यतरुणप्रख्याः श्लाघनीयमुदावहन्}


\twolineshloka
{काञ्चनापिहितैर्योक्त्रैर्मयूरग्रीवसन्निभाः}
{द्रौपदेयं नरव्याघ्रं श्रुतकर्माणमाहवे}


\twolineshloka
{श्रुतकीर्तिं श्रुतनिधिं द्रौपदेयं हयोत्तमाः}
{ऊहुः पार्थसमं युद्धे चाषपत्रनिभा हयाः}


\twolineshloka
{यमाहुरध्यर्धगुणं कृष्णात्पार्थाच्च संयुगे}
{अभिमन्युं पिशङ्गास्तं कुमारमवहन्रणे}


\threelineshloka
{एकस्तु धार्तराष्ट्रेभ्यः पाण्डवान्यः समाश्रितः}
{तं बृहन्तो महाकाया युयुत्सुमवहन्रणे}
{`ये तु पुष्करसारस्य तुल्यवर्णा हयोत्तमाः'}


\twolineshloka
{पलालकाण्डवर्णास्तु वार्धक्षेमिं तरस्विनम्}
{ऊहुः सुतुमुले युद्धे हया हृष्टाः स्वलङ्कृताः}


\twolineshloka
{कुमारं शितिपादास्तु रुक्मचित्रैरुरश्छदैः}
{सौचित्तिमवहद्युद्धे यन्तुः प्रेष्यकरा हयाः}


\twolineshloka
{रुक्मपीठावकीर्णास्तु कौशेयसदृशा हयाः}
{सुवर्णमालिनः क्षान्ताः श्रेणिमन्तमुदावहन्}


\twolineshloka
{रुक्ममालाधराः शूरा हेमपृष्ठाः स्वलङ्कृताः}
{काशिराजं नरश्रेष्ठं श्लाघनीयमुदावहन्}


\twolineshloka
{अस्त्राणां च धनुर्वेदे ब्राह्मे वेदे च पारगम्}
{तं सत्यधृतिमायान्तमरूणाः समुपावहन्}


\twolineshloka
{यः स पाञ्चालसेनानीद्रोणमंशमकल्पयत्}
{पारावतसवर्णास्तं धृष्टद्युम्नमुदावहन्}


\twolineshloka
{तमन्वयात्सत्यधृतिः सौचित्तिर्युद्धदुर्मदः}
{श्रेणिमान्वसुदानश्च पुत्रः काश्यस्य चाभिभूः}


\twolineshloka
{युक्तैः परमकाम्भोजैर्जवनैर्हेममालिभिः}
{भीषयन्तो द्विषत्सैन्यं यमवैश्रवणोपमाः}


\twolineshloka
{प्रभद्रकास्तु काम्भोजाः षट््सहस्राण्युदायुधाः}
{नानावर्णैर्हयैः श्रेष्ठैर्हेमवर्णरथध्वजाः}


\twolineshloka
{शरव्रातैर्विधुन्वन्तः शत्रून्विततकार्मुकाः}
{समानमृत्यवो भूत्वा धृष्टद्युम्नं समन्वयुः}


\twolineshloka
{बभ्रुकौशेयवर्णास्तु सुवर्णवरमालिनः}
{ऊहुरम्लानमनसश्चेकितानं हयोत्तमाः}


\twolineshloka
{इन्द्रायुधसवर्णैस्तु कुन्तिभोजो हयोत्तमैः}
{आयात्सदश्वैः पुरुजिन्मातुलः सव्यसाचिनः}


\twolineshloka
{अन्तरिक्षसवर्णास्तु तारकाचित्रिता इव}
{राजानं रोचमानं ते हयाः सङ्ख्ये समावहन्}


\twolineshloka
{कर्बुराः शितिपादास्तु स्वर्णजालपरिच्छदाः}
{जारासन्धिं हयाः श्रेष्ठाः सहदेवमुदावहन्}


\twolineshloka
{ये तु पुष्करनालस्य समवर्णा हयोत्तमाः}
{जवे श्येनसमाश्चित्राः सुदामानमुदावहन्}


\twolineshloka
{शशलोहितवर्णास्तु पाण्डुरोद्गतराजयः}
{पाञ्चाल्यं गोपतेः पुत्रं सिंहसेनमुदावहन्}


\twolineshloka
{पाञ्चालानां नरव्याघ्रो यः ख्यातो जनमेजयः}
{तस्य सर्षपपुष्पाणां तुल्यवर्णा हयोत्तमाः}


\twolineshloka
{माषवर्णाश्च जवना बृहन्तो हेममालिनः}
{दधिपृष्ठाश्चित्रमुखाः पाञ्चाल्यमवहन्द्रुतम्}


\twolineshloka
{शूराश्च भद्रकाश्चैव शरकाण्डनिभा हयाः}
{पद्मकिञ्जल्कवर्णाभा दण्डधारमुदावहन्}


\twolineshloka
{रासभारुणवर्णाभाः पृष्ठतो मूषिकप्रभाः}
{वल्गन्त इव संयत्ता व्याघ्रदत्तमुदावहन्}


\twolineshloka
{हरयः कालकाश्चित्राश्चित्रमाल्यविभूषिताः}
{सुधन्वानं नरव्याघ्रं पाञ्चाल्यं समुदावहन्}


\twolineshloka
{इन्द्राशनिसमस्पर्शा इन्द्रगोपकसन्निभाः}
{काये चित्रान्तराश्चित्राश्चित्रायुधमुदावहन्}


\twolineshloka
{बिभ्रतो हेपमालास्तु चक्रवाकोदरा हयाः}
{कोसलाधिपतेः पुत्रं सुक्षत्रं वाजिनोऽवहन्}


\twolineshloka
{शबलास्तु बृहन्तोऽश्वा दान्ता जाम्बूनदस्रजः}
{युद्धे सत्यधृतिं क्षेमिमवहन्प्रांशवः शुभाः}


\twolineshloka
{एकवर्णेन सर्वेण ध्वजेन कवचेन च}
{अश्वैश्च धनुषा चैव शुक्लैः शुक्लो न्यवर्तत}


\twolineshloka
{समुद्रसेनपुत्रं तु सामुद्रा रुद्रतेजसम्}
{अश्वाः शशाङ्कसदृशाश्चन्द्रसेनमुदावहन्}


\twolineshloka
{नीलोत्पलसवर्णास्तु तपनीयविभूषिताः}
{शैब्यं चित्ररथं सङ्ख्ये चित्रमाल्याऽवहन्हयाः}


\twolineshloka
{कलायपुष्पवर्णास्तु श्वेतलोहितराजयः}
{रथसेनं हयश्रेष्ठाः समूहुर्युद्धदुर्मदम्}


\twolineshloka
{यं तु सर्वमनुष्येभ्यः प्राहुः शूरतरं नृपम्}
{तं पटच्चरहन्तारं शुकवर्णाऽवहन्हयाः}


\twolineshloka
{चित्रायुधं चित्रमाल्यं चित्रवर्मायुधध्वजम्}
{ऊहुः किंशुकपुष्पाणां समवर्णा हयोत्तमाः}


\twolineshloka
{एकवर्णेन सर्वेण ध्वजेन कवचेन च}
{धनुषा रथवाहैश्च नीलैर्नीलोऽभ्यवर्तत}


\twolineshloka
{नानारूपै रत्नचिह्नैर्वरूथरथकार्मुकैः}
{वाजिध्वजपताकाभिश्चित्रैश्चित्रोऽभ्यवर्तत}


\twolineshloka
{ये तु पुष्करवर्णस्य तुल्यवर्णा हयोत्तमाः}
{ते रोचमानस्य सुतं हेमवर्णमुदावहन्}


\twolineshloka
{योधाश्च भद्रकाराश्च शरदण्डानुदण्डयः}
{श्वेताण्डाः कुक्कुटाण्डाभा दण्डकेतुं हयाऽवहन्}


\twolineshloka
{केशवे न हते सङ्ख्ये पितर्यथ नराधिपे}
{भिन्ने कपाटे पाण़्डानां विद्रुतेषु च बन्धुषु}


\twolineshloka
{भीष्मादवाप्य चास्त्राणि द्रोणाद्रामात्कृपात्तथा}
{अस्त्रैः समत्वं सम्प्राप्य रुक्मिकर्णार्जुनाच्युतैः}


\twolineshloka
{इयेष द्वारकां हन्तुं कृत्स्नां जेतुं च मेदिनीम्}
{निवारितस्ततः प्राज्ञैः सुहृद्भिर्हितकाम्यया}


\twolineshloka
{वैरानुबन्धमुत्सज्य स्वराज्यमनुशास्ति यः}
{स सागरध्वजः पाण्ड्यश्चन्द्ररश्मिनिभैर्हयैः}


\twolineshloka
{वैडूर्यजालसञ्छन्नैर्वीर्यद्रविणमाश्रितः}
{दिव्यं विस्फारयंश्चापं द्रोणमभ्यद्रवद्बली}


\twolineshloka
{आरकूटकवर्णाश्च हयाः पाण्ड्यानुयायिनाम्}
{अवहन्रथमुख्यानामयुतानि चतुर्दश}


\twolineshloka
{नानावर्णेन रूपेण नानाकृतिमुखा हयाः}
{रथचक्रध्वजं वीरं घटोत्कचमुदावहन्}


\twolineshloka
{भारतानां समेतानामुत्सृज्यैको मतानि यः}
{गतो युधिष्ठिरं भक्त्या त्यक्त्वा सर्वमभीप्सितम्}


\twolineshloka
{लोहिताक्षं महाबाहुं युयुत्सुं मकरध्वजम्}
{महासत्वा महाकायाः सौवर्णस्यन्दने स्थितम्}


\twolineshloka
{सुवर्णवर्णा धर्मज्ञमनीकस्थं युधिष्ठिरम्}
{राजश्रेष्ठं हयश्रेष्ठाः सर्वतः पृष्ठतोऽन्वयुः}


\twolineshloka
{वर्णैरुच्चावचैरन्यैः सदश्वानां प्रभद्रकाः}
{सन्न्यवर्तन्त युद्धाय बहवो देवरूपिणः}


\twolineshloka
{ते यत्ता भीमसेनेन सहिताः काञ्चनध्वजाः}
{प्रत्यदृश्यन्त राजेन्द्र सेन्द्रा इव दिवौकसः}


\twolineshloka
{अत्यरोचत तान्सर्वान्धृष्ठद्युम्नः समागतान्}
{सर्वाण्यति च सैन्यानि भारद्वाजो व्यरोचत}


\twolineshloka
{अतीव शुशुभे तस्य ध्वजः कृष्णाजिनोत्तरः}
{कमण्डलुर्महाराज जातरूपमयः शुभः}


\twolineshloka
{ध्वजं तु भीमसेनस्य वै2डूर्यमणिलोचनम्}
{भ्राजमानं महासिंहं राजन्तं दृष्टवानहम्}


\twolineshloka
{ध्वजं तु कुरुराजस्य पाण्डवस्य महौजसः}
{दृष्टवानस्मि सौवर्णं सोमं ग्रहगणान्वितम्}


\twolineshloka
{मृदङ्गौ चात्र विपुलौ दिव्यौ नन्दोपनन्दकौ}
{यन्त्रेणाहन्यमानौ च सुस्वनौ हर्षवर्धनौ}


\twolineshloka
{शरभं पृष्ठसौवर्णं नकुलस्य महाध्वजम्}
{अपश्याम रथेत्युग्रं भीषयाणमवस्थितम्}


\twolineshloka
{हंसस्तु राजतः श्रीमान्ध्वजे घण्टापताकवान्}
{सहदेवस्य दुर्धर्षो द्विषतां शोकवर्धनः}


\twolineshloka
{पञ्चानां द्रौपदेयानां प्रतिमाध्वजभूषणम्}
{धर्ममारुतशक्राणामश्विनोश्च महात्मनोः}


\twolineshloka
{अभिमन्योः कुमारस्य शार्ङ्गपक्षी हिरण्मयः}
{रथे ध्वजवरो राजंस्तप्तचामीकरोज्ज्वलः}


\twolineshloka
{घटोत्कचस्य राजेन्द्र ध्वजे गृध्रो व्यरोचत}
{अश्वाश्च कामगास्तस्य रावणस्य पुरा यथा}


\twolineshloka
{माहेन्द्रं च धनुर्दिव्यं धर्मराजे युधिष्ठिरे}
{वायव्यं भीमसेनस्य धनुर्दिव्यमभून्नृप}


\twolineshloka
{त्रैलोक्यरक्षणार्थाय ब्रह्मणा सृष्टमायुधम्}
{तद्दिव्यमजरं चैव फाल्गुनार्थाय वै धनुः}


\twolineshloka
{वैष्णवं नकुलायाथ सहदेवाय चाश्चिजम्}
{घटोत्कचाय पौलस्त्यं धनुर्दिव्यं भयामकम्}


\twolineshloka
{रौद्रमाग्नेयकौबेरं याम्यं गिरिशमेव च}
{पञ्चानां द्रौपदेयानां धनूरत्नानि भारत}


\twolineshloka
{रौद्रं धनुर्वरं श्रेष्ठं लेभे यद्रोहिणीसुतः}
{तत्तुष्टः प्रददौ रामः सौभद्राय महात्मने}


\twolineshloka
{एते चान्ये च बहवो ध्वजा हेमविभूषिताः}
{तत्रादृश्यन्त शूराणां द्विषतां शोकवर्धनाः}


\twolineshloka
{तदभूद्ध्वजसम्बाधमकापुरुषसेवितम्}
{द्रोणानीकं महाराज पटे चित्रमिवार्पितम्}


\twolineshloka
{शुश्रुवुनार्मगोत्राणि वीराणां संयुगे तदा}
{द्रोणमाद्रवतां राजन्स्वयंवर इवाहवे}


\chapter{अध्यायः २४}
\twolineshloka
{धृतराष्ट्र उवाच}
{}


\twolineshloka
{व्यथयेयुरिमे सेनां देवानामपि सञ्जय}
{आहवे ये न्यवर्तन्त वृकोदरमुखा नृपाः}


\twolineshloka
{सम्प्रयुक्तः किलैवायं दिष्टैर्भवति पूरुषः}
{तस्मिन्नेव च सर्वार्थाः प्रदृश्यन्ते पृथग्विधाः}


\twolineshloka
{दीर्घं विप्रोषितः कालमरण्ये जटिलोऽजिनी}
{अज्ञातश्चैव लोकस्य विजहार युधिष्ठिरः}


\twolineshloka
{स एव महतीं सेनां समावर्तयदाहवे}
{किमन्यद्दैवसंयोगान्मम पुत्राभवाय च}


\twolineshloka
{युक्त एव हि भाग्येन ध्रुवमुत्पद्यते नरः}
{स तथाऽऽकृष्यते तेन न यथा स्वयमिच्छति}


\twolineshloka
{द्यूतव्यसनमासाद्य क्लेशितो हि युधिष्ठिरः}
{स पुनर्भागधेयेन सहायानुपलब्धवान्}


\twolineshloka
{अर्थे मे केकया लब्धाः काशिका कोसलाश्चये}
{चेदीनां चार्धमपरे मामे व समुपाश्रिताः}


\twolineshloka
{पृथिवी भूयसी तात मम पार्थस्य नो तथा}
{इति मामब्रवीत्सूत मन्दो दुर्योधनः पुरा}


\twolineshloka
{तस्य सेनासमूहस्य मध्ये द्रोणः सुरक्षितः}
{निहतः पार्षतेनाजौ किमन्यद्भागधेयतः}


\twolineshloka
{मध्ये राज्ञां महाबाहुं सदा युद्धाभिनन्दिनम्}
{सर्वास्त्रपारगं द्रोणं कथं मृत्युरुपेयिवान्}


\twolineshloka
{समनुप्राप्तकृच्छ्रोऽहं मोहं परममागतः}
{भीष्मद्रोणौ हतौ श्रुत्वा नाहं जीवितुमुत्सहे}


\twolineshloka
{यन्मां क्षत्ताऽब्रवीत्तात प्रपश्यन्पुत्रगृद्धिनम्}
{दुर्योधनेन तत्सर्वं प्राप्तं सूत मया सह}


\twolineshloka
{नृशंसं तु परं तात त्यक्त्वा दुर्योधनं यदि}
{पुत्रशेषं चिकीर्षेयं कृत्स्नं न मरणं व्रजेत्}


\twolineshloka
{यो हि धर्मं परित्यज्य भवत्यर्थपरो नरः}
{सोऽस्माच्च हीयते लोकात्क्षुद्रभावं च गच्छति}


\twolineshloka
{अद्य चाप्यस्य राष्ट्रस्य कृतोच्छेदस्य सञ्जय}
{अवशेषं न पश्यामि ककुदे मृदिते सति}


\twolineshloka
{कथं स्यादवशेषो हि धुर्ययोरभ्यतीतयोः}
{यौ नित्यमुपजीवामः क्षमिणौ पुरुषर्षभौ}


\twolineshloka
{व्यक्तमेव च मे शंस यथा युद्धमवर्तत}
{केऽयुध्यन्के व्यपाकुर्वन्के क्षुद्राः प्राद्रवन्भयात्}


\twolineshloka
{धनुञ्जयं च मे शंस यद्यच्चक्रे रथर्षभः}
{तस्माद्भयं नो भूयिष्ठं भ्रातृव्याच्च वृकोदरात्}


\twolineshloka
{यथाऽसीच्च निवृत्तेषु पाण्डवेयेषु सञ्जय}
{मम सैन्यावशेषस्य सन्निपातः सुदारुणः}


\twolineshloka
{कथं च वो मनस्तात निवृत्तेष्वभवत्तदा}
{मामकानां च ये शूराः के कांस्तत्र न्यवारयन्}


\chapter{अध्यायः २५}
\twolineshloka
{सञ्जय उवाच}
{}


\twolineshloka
{महद्भैरवमासीन्नः सन्निवृत्तेषु पाण्डुषु}
{दृष्ट्वा द्रोणं छाद्यमानं तैर्भास्करमिवाम्बुदैः}


\twolineshloka
{तैश्चोद्भूतं रजस्तीव्रमवचक्रे चमूं तव}
{ततो हतममंस्याम द्रोणं दृष्टिपथे हते}


\twolineshloka
{तांस्तु शूरान्महेष्वासान्क्रूरं कर्म चिकीर्षतः}
{दृष्ट्वा दुर्योधनस्तूर्णं स्वसैन्यं समचूचुदत्}


\twolineshloka
{यथाशक्ति यथोत्साहं यथासत्वं नराधिपाः}
{वारयध्वं यथायोगं पाण्डवानामनीकिनीम्}


\twolineshloka
{ततो दुर्मर्षणो भीममभ्यगच्छत्सुतस्तव}
{आराद्दृष्ट्वा किरन्बाणैरिच्छन्दोणस्य जीवितम्}


\twolineshloka
{तं बाणेरवतस्तार क्रुद्धो मृत्युरिवाहवे}
{तं च भीमोऽतुदद्बाणैस्तदाऽऽसीत्तुमुलं महत्}


\twolineshloka
{त ईश्वरसमादिष्टाः प्राज्ञाः शूराः प्रहारिणः}
{राज्यं मृत्युभयं त्यक्त्वा प्रत्यतिष्ठन्परान्युधि}


\twolineshloka
{कृतवर्मा शिनेः पौत्रं द्रोणं प्रेप्सुं विशाम्पते}
{पर्यवारयदायान्तं शूरं समरशोभिनम्}


\twolineshloka
{तं शैनेयः शरव्रातैः क्रुद्धः क्रुद्धमवारयत्}
{कृतवर्मा च शैनेयं मत्तो मत्तमिव द्विपम्}


\twolineshloka
{सैन्धवः क्षत्रवर्माणमायान्तं निशितैः शरैः}
{उग्रधन्वा महेष्वासं यत्ततो द्रोणादवारयत्}


\twolineshloka
{क्षत्रवर्मा सिन्धुपतेश्छित्त्वा केततनकार्मुके}
{नाराचैर्दशभिः क्रुद्धः सर्वमर्मस्वताडयत्}


\twolineshloka
{अथान्यद्धनुरादाय सैन्धवः कृतहस्तवत्}
{विव्याध क्षत्रवर्माणं रणे सर्वायसैः शरैः}


\twolineshloka
{युयुत्सुं पाण्डवार्थाय यतमानं महारथम्}
{सुबाहुर्भारतं शूरं यत्तो द्रोणादवारयत्}


\twolineshloka
{सुबाहोः सधनुर्बाणावस्यतः परिघोपमौ}
{युयुत्सुः शितपीताभ्यां क्षुराभ्यामच्छिनद्भुजौ}


\twolineshloka
{राजानं पाण्डवश्रेष्ठं धर्मात्मानं युधिष्ठिरम्}
{वेलेव सागरं क्षुब्धं मद्रराट् समवारयत्}


\twolineshloka
{तं धर्मराजो बहुभिर्मर्मभिद्भिरवाकिरत्}
{मद्रेशस्तं चतुःषष्ट्या शरैर्विद्ध्वाऽनदद्भृशम्}


\twolineshloka
{तस्य नानदतः केतुमुच्चकर्त च कार्मुकम्}
{क्षुराभ्यां पाण्डवो ज्येष्ठस्तत उच्चुक्रुशुर्जनाः}


\twolineshloka
{तथैव राजा बाह्लीको राजानं द्रुपदं शरैः}
{आद्रवन्तं सहानीकः सहानीकं न्यवारयत्}


\twolineshloka
{तद्युद्धमभवद्धोरं वृद्धयोः सहसेनयोः}
{यथा महायूथपयोर्द्विपयोः सम्प्रभिन्नयोः}


\twolineshloka
{विन्दानुविन्दावावन्त्यौ विराटं मत्स्यमार्च्छताम्}
{सहसैन्यौ सहानीकं यथेन्द्राग्नी पुरा बलिम्}


\twolineshloka
{तदुत्पिञ्जलकं युद्धमासीद्देवासुरोपमम्}
{मत्स्यानां केकयैः सार्धमभीताश्वरथद्विपम्}


\twolineshloka
{नाकुलिं तु शतानीकं भूतकर्मा सभापतिः}
{अस्यन्तमिषुजालानि यान्तं द्रोणादवारयत्}


\twolineshloka
{ततो नकुलदायादस्त्रिभिर्भल्लैः सुसंशितैः}
{चक्रे विबाहुशिरसं भूतकर्माणमाहवे}


\twolineshloka
{सुतसोमं तु विक्रान्तमायान्तं तं शरौघिणम्}
{द्रोणायाभिमुखं वीरं साल्वो बाणैरवारयत्}


\twolineshloka
{स तु भीमरथः साल्वमाशुगैरायसैः शितैः}
{षड्भिः साश्वनियन्तारमनयद्यमसादनम्}


\twolineshloka
{श्रुतकीर्तिं समायान्तं मयूरसदृशैर्हयैः}
{चित्रसेनो महाराज तव पौत्रं न्यवारयत्}


\twolineshloka
{तमार्जुनिः शरैश्चक्रे पितृव्यं जर्झरच्छविम्}
{शरैश्च द्रौपदीपुत्रं चित्रसेनः समावृणोत्}


\twolineshloka
{किरन्तं शरजालानि प्रभिन्नमिव कुञ्जरम्}
{श्रुतकर्माणमायान्तं दौश्शासनिरवारयत्}


\twolineshloka
{तौ पौत्रौ तव दुर्धर्षौ परस्परवधैषिणौ}
{पितॄणामर्थसिद्ध्यर्थं चक्रतुर्युद्धमुत्तमम्}


\twolineshloka
{तिष्ठन्तमग्रे तं दृष्ट्वा प्रतिविन्ध्यं महाहवे}
{द्रौणिर्मानं पितुः कुर्वन्मार्गणैः समवारयत्}


\twolineshloka
{तं क्रुद्धं प्रतिविव्याध प्रतिविन्ध्यः शितैः शरैः}
{सिंहलाङ्गूललक्ष्माणं पितुरर्थे व्यवस्थितम्}


\twolineshloka
{प्रवपन्निव बीजानि बीजकाले कृषीवलः}
{द्रौणायनिर्द्रौपदेयं शरवर्षैरवाकिरत्}


\twolineshloka
{यस्तु शूरतमो राजन्नुभयोः सेनयोर्मतः}
{तं पटच्चरहन्तारं लक्ष्मणः समवारयत्}


\twolineshloka
{स लक्ष्मणस्येष्वसनं छित्त्वा लक्ष्म च भारत}
{लक्ष्मणे शरजालानि विसृजन्बह्वशोभत}


\twolineshloka
{ततोऽभिमन्युः कर्माणि कुर्वन्तं चित्रयोधिनम्}
{आर्जुनिः कृतिनं शूरं लक्ष्मणं समयोधयत्}


\twolineshloka
{स सम्प्रहारस्तुमुलस्तयोरासीन्महात्मनोः}
{श्रोतॄणामीक्षितॄणां च भृशं प्रीतिविवर्धनः}


\twolineshloka
{विकर्णस्तु महाप्राज्ञो याज्ञसेनिं शिखण्डिनम्}
{पर्यवारयदायान्तं युवानं समरे युवा}


\twolineshloka
{ततस्तमिषुजालेन याज्ञसेनिः समावृणोत्}
{विधूय तद्बाणजालं बभौ तव सुतो बली}


\twolineshloka
{अङ्गदोऽभिमुखं वीरमुत्तमौजसमाहवे}
{द्रोणायाभिमुखं यान्तं शरौघेण न्यवारयत्}


\twolineshloka
{स सम्प्रहारस्तुमुलस्तयोः पुरुषसिंहयोः}
{सैनिकानां च सर्वेषां तयोश्च प्रीतिवर्धनः}


\twolineshloka
{दुर्मुखस्तु महेष्वासो वीरं पुरुजितं बली}
{द्रोणायाभिमुखं यान्तं वत्सदन्तैरवारयत्}


\twolineshloka
{स दुर्मुखं भ्रुवोर्मध्ये नाराचेनाभ्यताडयत्}
{तस्य तद्धि बभौ वक्त्रं सनालमिव पङ्कजम्}


\twolineshloka
{कर्णस्तु केकयान्भ्रातॄन्पञ्च लोहितकध्वजान्}
{द्रोणायाभिमुखं याताञ्शरवर्षैरवारयत्}


\twolineshloka
{ते चैनं भृशसन्तप्ताः शरवर्षैरवाकिरन्}
{स च तांश्छादयामास शरजालैः पुनः पुनः}


\twolineshloka
{नैव कर्णो न ते पञ्च ददृशर्बाणसंवृताः}
{साश्वसूतध्वजरथाः परस्परशराचिताः}


\twolineshloka
{पुत्रास्ते दुर्जयश्चैव जयश्च विजयश्च ह}
{नीलकाश्यजयत्सेनांस्त्रयस्त्रीन्प्रत्यवारयन्}


\twolineshloka
{तद्युद्धमभवद्धोरमीक्षितृप्रीतिवर्धनम्}
{सिंहव्याघ्रतरक्षूणां यथर्क्षमहिषर्षभैः}


\twolineshloka
{क्षेमधूर्तिबृहन्तौ तु भ्रातरौ सात्वतं युधि}
{द्रोणायाभिमुखं यान्तं शरैस्तीक्ष्णैस्ततक्षतुः}


\threelineshloka
{तयोस्तस्य च तद्युद्धमत्यद्भुतमिवाभवत्}
{सिंहस्य द्विपमुख्याभ्यां प्रभिन्नाभ्यां यथा वने}
{}


\twolineshloka
{राजानं तु तथाम्बष्ठमेकं युद्धाभिनन्दिनम्}
{चेदिराजः शरानस्यन्क्रुद्धो द्रोणादवारयत्}


\twolineshloka
{ततोऽम्बष्ठोऽस्थिभेदिन्या निरभिद्यच्छलाकया}
{स त्यक्त्वा सशरं चापं रथाद्भूमिमुपागमत्}


\twolineshloka
{वार्धक्षेमिं तु वार्ष्णेयं कृपः शारद्वतः शरैः}
{अक्षुद्रः क्षुद्रकैर्बाणैः क्रुद्धरूपमवारयत्}


\twolineshloka
{युध्यन्तौ कृपवार्ष्णेयौ येऽपश्यंश्चित्रयोधिनौ}
{ते युद्धासक्तमनसो नान्यं बुबुधिरे क्रियाम्}


\twolineshloka
{सौमदत्तिस्तु राजानं मणिमन्तमतन्द्रितम्}
{पर्यवारयदायान्तं यशो द्रोणस्य वर्घयन्}


\twolineshloka
{स सौमदत्तेस्त्वरितश्चित्रेष्वसनकेतने}
{पुनः पताकां सूतं चच्छत्रं चापातयद्रथात्}


\twolineshloka
{अथाप्लुत्य रथात्तूर्णं यूपकेतुरमित्रहा}
{साश्वसूतध्वजरथं तं चकर्त वरासिना}


\twolineshloka
{रथं च स्वं समास्थाय धनुरादाय चापरम्}
{स्वयं यच्छन्हयान्राजन्व्यधमत्पाण्डवीं चमूम्}


\twolineshloka
{पाण्ड्यमिन्द्रमिवायान्तमसुरान्प्रति दुर्जयम्}
{समर्थः सायकौघेन वृषसेनो न्यवारयत्}


\twolineshloka
{`अथाजगाम हैडिम्बो महामायो महाबलः}
{पितॄणां हितमन्विच्छन्दहन्कौरववाहिनीम्'}


\twolineshloka
{गदापरिघनिस्त्रिंशपट्टसायोघनोपलैः}
{कडङ्गरैर्भुशुण्डीभिः प्रासैस्तोमरसायकैः}


\twolineshloka
{मुसलैर्मुद्गरैश्चक्रैर्न्भिण्डिपालपरश्वथैः}
{पांसुवाताग्निसलिलैर्भस्मलोष्ठतृणद्रुमैः}


\twolineshloka
{आतुदन्प्ररुजन्भञ्जन्निघ्नन्विद्रावयन्क्षिपन्}
{सेनां विभीषयन्नायाद्दोणप्रेप्सुर्घटोत्कचः}


\twolineshloka
{तं तु नानाप्रहरणैर्नानायुद्धविशेषणैः}
{राक्षसं राक्षसः क्रुद्धः समाजघ्ने ह्यलम्बुसः}


\twolineshloka
{तयोस्तदभवद्युद्धं रक्षोग्रामणिमुख्ययोः}
{तादृग्यादृक्पुरावृत्तं शम्बरामरराजयोः}


\twolineshloka
{एवं द्वन्द्वशतान्यासन्रथवारणवाजिनाम्}
{पदातीनां च भद्रं ते तव तेषां च सङ्कुले}


\twolineshloka
{नैतादृशो दृष्टपूर्वः सङ्ग्रामो नैव च श्रुतः}
{द्रोणस्याभावभावे तु प्रसक्तानां परस्परम्}


\twolineshloka
{इदं घोरमिदं चित्रमिदं रौद्रमिति प्रभो}
{तत्र युद्धान्यदृश्यन्त प्रततानि बहूनि च}


\chapter{अध्यायः २६}
\twolineshloka
{धृतराष्ट्र उवाच}
{}


\twolineshloka
{तेष्वेवं सन्निवृत्तेषु प्रत्युद्यातेषु भागशः}
{कथं युयुधिरे पार्था मामकाश्च तरस्विनः}


\threelineshloka
{किमर्जुनश्चाप्यकरोत्संशप्तकबलं प्रति}
{संशप्तका वा पार्थस्य किमकुर्वत सञ्जय ॥सञ्जय उवाच}
{}


\twolineshloka
{तथा तेषु निवृत्तेषु प्रत्युद्यातेषु भागशः}
{स्वयमभ्यद्रवद्भीमं नागानीकेन ते सुतः}


\twolineshloka
{स नाग इव नागेन गोवृषेणेव गोवृषः}
{समाहूतः स्वयं राज्ञा नागानीकमुपाद्रवत्}


\twolineshloka
{स युद्धकुशलः पार्थो बाहुवीर्येण चान्वितः}
{अभिनत्कुञ्जरानीकमचिरेणैव मारिष}


\twolineshloka
{ते गजा गिरिसङ्काशाः क्षरन्तः सर्वतो मदम्}
{भीमसेनस्य नाराचैर्विमुखा विमदीकृताः}


\twolineshloka
{विधमेदभ्रजालानि यथा वायुः समुद्धतः}
{व्यधमत्तान्यनीकानि तथैव पवनात्मजः}


\twolineshloka
{स तेषु विसृजन्बाणान्भीमो नागेष्वशोभत}
{भुवनेष्विव सर्वेषु गभस्तीनुदितो रविः}


\twolineshloka
{ते भीमबाणाभिहताः संस्यूता विबभुर्गजाः}
{गभस्तिभिरिवार्कस्य व्योम्नि नानाबलाहकाः}


\twolineshloka
{तथा गजानां कदनं कुर्वाणमनिलात्मजम्}
{क्रुद्धो दुर्योधनोऽभ्येत्य प्रत्यविध्यच्छितैः शरैः}


\twolineshloka
{ततः क्षणेन क्षितिपं क्षतजप्रतिमेक्षणः}
{क्षयं निनीषुर्निशितैर्भीमो विव्याध पत्रिभिः}


\twolineshloka
{स शराचितसर्वाङ्गः क्रुद्धो विव्याध पाण्डवम्}
{नाराचैरर्करश्म्याभैर्भीमसेनं स्मयन्निव}


\twolineshloka
{तस्य नागं मणिमयं रत्नचित्रध्वजे स्थितम्}
{भल्लाभ्यां कार्मुकं चैव क्षिप्रं चिच्छेद पाण्डवः}


\twolineshloka
{दुर्योधनं पीड्यमानं दृष्ट्वा भीमेन मारिष}
{चुक्षोभयिषुरभ्यागादङ्गो मातङ्गमास्थितः}


\twolineshloka
{तमापतन्तं नागेन्द्रमम्बुदप्रतिमस्वनम्}
{कुम्भान्तरे भीमसेनो नाराचैरार्दयद्भृशम्}


\twolineshloka
{तस्य कायं विनिर्भिद्य न्यमज्जद्धरणीतले}
{ततः पपात द्विरदो वज्राहत इवाचलः}


\twolineshloka
{तस्वावर्जितनागस्य म्लेच्छस्याधः पतिष्यतः}
{शिरश्चिच्छेद भल्लेन क्षिप्रकारी वृकोदरः}


\twolineshloka
{तस्मिन्निपतिते वीरे सम्प्राद्रवत सा चमूः}
{सम्भ्रान्ताश्वद्विपरथा पदातीनवमृद्गती}


\twolineshloka
{तेष्वनीकेषु भग्नेषु विद्रवत्सु समन्ततः}
{प्राग्ज्योतिषस्ततो भीमं कुञ्जरेण समाद्रवत्}


\twolineshloka
{येन नागेन मघवानजयद्दैत्यदानवान्}
{तदन्वयेन नागेन भीमसेनमुपाद्रवत्}


\twolineshloka
{स नागप्रवरो भीमं सहसा समुपाद्रवत्}
{श्रवणाभ्यामथो द्वाभ्यां संहतेन करेण च}


\twolineshloka
{व्यावृत्तनयनः क्रुद्वः प्रमथन्निव पाण्डवम्}
{वृकोदररथं साश्वमविशेषमचूर्णयत्}


\twolineshloka
{पद्भ्यां भीमोऽप्यथो धावंस्तस्य गात्रेष्वलीयत}
{जानन्नञ्जलिकावेधं नापाक्रामत पाण्डवः}


\twolineshloka
{गात्राभ्यन्तरगो भूत्वा करेणाताडयन्मुहुः}
{लालयामास तं नागं वधाकाङ्क्षिणमव्ययम्}


\twolineshloka
{कुलालचक्रवन्नागस्तदा तूर्णमथाभ्रमत्}
{नागायुतबलः श्रीमान्कालयानो वृकोदरम्}


\twolineshloka
{भीमोऽपि निष्क्रम्य ततः सुप्रतीकाग्रतोऽभवत्}
{भीमं करेणावनम्य जानुभ्यामभ्यताडयत्}


\twolineshloka
{ग्रीवायां वेष्टयित्वैनं स गजो हन्तुमैहत}
{करवेष्टं भीमसेनो भ्रमं दत्त्वा व्यमोचयत्}


\threelineshloka
{पुनर्गात्राणि नागस्य प्रविवेश वृकोदरः}
{यावत्प्रतिगजायातं स्वबलं प्रत्यवैक्षत}
{भीमोपि नागगात्रेभ्यो विनिःसृत्यापयाज्जवात्}


\twolineshloka
{ततः सर्वस्य सैन्यस्य नादः समभवन्महान्}
{अहो धिङ्गिहतो भीमः कुज्जरेणेति मारिष}


\twolineshloka
{तेन नागेन सन्त्रस्ता पाण्डवानामनीकिनी}
{सहसाऽभ्यद्रवद्राजन्यत्र तस्यौ वृकोदरः}


\twolineshloka
{ततो युधिष्ठिरो राजा हतं मत्वा वृकोदरम्}
{भगदत्तं सपाञ्चाल्यः सर्वतः समवारयत्}


\twolineshloka
{तं रथं रथिनां श्रेष्ठाः परिवार्य परन्तपाः}
{अवाकिरञ्शरैस्तीक्ष्णैः शतशोऽथ सहस्रशः}


\twolineshloka
{स विघातं पृषत्कानामङ्कुशेन समाहरन्}
{क्षणेन पाण्डुपाञ्जालान्व्यधमत्पर्वतेश्वरः}


\twolineshloka
{तदद्भुतमपश्याम भगदत्तस्य संयुगे}
{तथा वृद्धस्य चरितं कुञ्जरेण विशाम्पते}


\twolineshloka
{ततो राजा दशार्णानां प्राग्ज्योतिषमुपाद्रवत्}
{तिर्यग्यातेन नागेन समदेनाशुगामिना}


\twolineshloka
{तयोर्युद्धं समभवन्नागयोर्भीमरूपयोः}
{सपक्षयोः पर्वतयोर्यथा सद्रुमयोः पुरा}


\twolineshloka
{प्राग्ज्योतिषपतेर्नागः सन्निवृत्त्यापसृत्य च}
{पार्श्वे दशार्णाधिपतेर्भित्त्वा नागमपातयत्}


\twolineshloka
{तोमरैः सूर्यरश्म्याभैर्भगदत्तोऽथ सप्तभिः}
{जघान द्विरदस्थं तं शत्रुं प्रचलितासनम्}


\twolineshloka
{व्यवच्छिद्य तु राजानं भगदत्तं युधिष्ठिरः}
{रथानीकेन महता सर्वतः पर्यवारयत्}


\twolineshloka
{स कुञ्जरस्थो रथिभिः शुशुभे सर्वतो वृतः}
{पर्वते वनमध्यस्थो ज्वलन्निव हुताशनः}


\twolineshloka
{मण्डलं सर्वतः श्लिष्टं रथिनामुग्रधन्विनाम्}
{किरतां शरवर्षाणि स नागः पर्यवर्तत}


\twolineshloka
{ततः प्राग्ज्योतिषो राजा परिगृह्य महागजम्}
{प्रेषयामास सहसा युयुधानरथं प्रति}


\twolineshloka
{`आपतन्तं च सम्प्रेक्ष्य नागं सात्वतपुङ्गवः}
{अविध्यत्पञ्चभिर्बाणैः शितैराशीविषोपमैः'}


\twolineshloka
{शिनेः पौत्रस्य तु रथं परिगृह्य महाद्विपः}
{अभिचिक्षेप वेगेन युयुधानस्त्वपाक्रमत्}


\twolineshloka
{बृहतः सैन्धवानश्वान्समुत्थाप्यथ सारथिः}
{तस्थौ सात्यकिमासाद्य सम्प्लुतस्तं रथं प्रति}


\twolineshloka
{स तु लब्ध्वान्तरं नागस्त्वरितो रथमण्डलात्}
{निष्पतन्सततं सर्वान्परिचिक्षेप पार्थिवान्}


\twolineshloka
{ते त्वाशुगतिना तेन त्रास्यमाना नरर्षभाः}
{तमेकं द्विरदं सङ्ख्ये मेनिरे शतशो द्विपान्}


\twolineshloka
{ते गजस्थेन काल्यन्ते भगदत्तेन पार्थिवाः}
{ऐरावतस्थेन यथा देवराजेन दानवाः}


\twolineshloka
{तेषां प्रद्रवतां भीमः पाञ्चालानामितस्ततः}
{गजवाजिकृतः शब्दः सुमहान्समजायत}


\twolineshloka
{भगदत्तेन समरे काल्यमानेषु पाण्डुषु}
{प्राग्ज्योतिषमभिक्रुद्धः पुनर्भीमः समभ्ययात्}


\twolineshloka
{तस्याभिद्रवतो वाहान्हस्तमुक्तेन वारिणा}
{सिक्त्वा व्यत्रासयन्नागस्ते पार्थसहरंस्ततः}


\twolineshloka
{ततस्तमभ्ययात्तूर्णं रुचिपर्वा कृतीसुतः}
{समघ्नञ्छरवर्षेण रथस्योऽन्तकसन्निभः}


\twolineshloka
{ततः स रुचिपर्वाणं शरेणानतपर्वणा}
{सुपर्वा पर्वतपतिर्निन्ये वैवस्वतक्षयम्}


\twolineshloka
{तस्मिन्निपतिते वीरे सौभद्रो द्रौपदीसुतः}
{चेकितानो धृष्टकेतुर्ययुत्सुश्चार्दयन्द्विपम्}


\twolineshloka
{त एनं शरधाराभिर्धाराभिरिव तोयदाः}
{सिषिचुर्भैरवान्नादान्विनदन्तो जिघांसवः}


\twolineshloka
{ततः पार्ष्ण्यङ्कुशाङ्गुष्ठैः कृतिना चोदितो द्विपः}
{प्रसारितकरः प्रायात्स्तब्धकर्णेक्षणो द्रुतम्}


\twolineshloka
{सोऽधिष्ठाय पदा वाहान्युयुत्सोः सूतमारुजत्}
{युयुत्सुस्तु रथाद्राजन्नपाक्रामत्त्वरान्वितः}


\twolineshloka
{ततः पाण्डवयोधास्ते नागराजं शरैर्द्रुतम्}
{सिषिचुर्भैरवान्नादान्विनदन्तो जिघांसवः}


\threelineshloka
{पुत्रस्तु तव सम्भ्रान्तः सौभद्रस्याप्लुतो रथम्}
{स कुञ्जरस्थो विसृजन्निषूनरिषु पार्थिवः}
{बभौ रश्मीनिवादित्यो भुवनेषु समुत्सृजन्}


\twolineshloka
{तमार्जुनिर्द्वादशभिर्युयुत्सुर्दशभिः शरैः}
{त्रिभिस्त्रिभिर्द्रौपदेया धृष्टकेतुश्च विव्यधुः}


\twolineshloka
{`चेकितानः चतुःषष्ट्या सोत्तरायुधिकं पुनः}
{प्रत्यविध्यत्ततः सर्वान्भगदत्तस्त्रिभिस्त्रिभिः'}


\twolineshloka
{सोऽतियत्नार्पितैर्बाणैराचितो द्विरदो बभौ}
{संस्यूत इव सूर्यस्य रश्मिभिर्जलदो महान्}


\twolineshloka
{नियन्तुः शिल्पयत्नाभ्यां प्रेरितोऽरिशरार्दितः}
{परिचिक्षेप तान्नागः स रिपून्सव्यदक्षिणम्}


\twolineshloka
{गोपाल इव दण्डेन यथा पशुगणान्वने}
{सङ्कालयति तां सेनां भगदत्तस्तथा मुहुः}


\twolineshloka
{क्षिप्रं श्येनाभिप्रन्नानां वायसानामिव स्वनः}
{बभूव पाण्डवेयानां भृशं विद्रवतां स्वनः}


\twolineshloka
{स नागराजः प्रवराङ्कुशाहतःपुरा सपक्षोऽद्रिवरो यथा नृप}
{भयं तदा रिपुषु समादधद्भृशंवणिग्जनानां क्षुभितो यथाऽर्णवः}


\twolineshloka
{ततो ध्वनिर्द्विरदरथाश्वपार्थिवैर्भयाद्रवद्भिर्जनितोऽतिभैरवः}
{क्षितिं वियद् द्यां विदिशो दिशस्तथासमावृणोत्पार्थिवसंयुगे ततः}


\twolineshloka
{स तेन नागप्रवरेण पार्थिवोभृशं जगाहे द्विषतामनीकिनीम्}
{पुरा सुगुप्तां विबुधैरिवाहवेविरोचनो देववरूतिनीमिव}


\twolineshloka
{भृशं ववौ ज्वलनसखो वियद्रजःसमावृणोन्मुहुरपि चैव सैनिकान्}
{तमेकनागं गणशो यथा गजान्समन्ततो द्रुतमथ मेनिरे जनाः}


\chapter{अध्यायः २७}
\twolineshloka
{सञ्जय उवाच}
{}


\twolineshloka
{यन्मां पार्थस्य सङ्ग्रामे कर्माणि परिपृच्छसि}
{तृच्छृणुष्व महाबाहो पार्थो यदकरोद्रणे}


\twolineshloka
{रजो दृष्ट्वा समुद्भूतं श्रुत्वा च गजनिःस्वनम्}
{भगदत्ते विकुर्वाणे कौन्तेयः कृष्णमब्रवीत्}


\twolineshloka
{यथा प्राग्ज्योतिषो राजा गजेन मधुसूदन}
{त्वरमाणोऽभिनिष्क्रान्तो ध्रुवं तस्यैष निःस्वनः}


\twolineshloka
{इन्द्रादनवरः सङ्ख्ये गजयानविशारदः}
{प्रथमो गजघोधानां पृथिव्यामिति मे मतिः}


\twolineshloka
{स चापि द्विरदश्रेष्ठः सदाऽप्रतिगजो युधि}
{सर्वशस्त्रातिगः सङ्ख्ये कृतकर्मा जितक्लमः}


\twolineshloka
{सहः सस्त्रनिपातानामग्निस्पर्शस्य चानघ}
{स पाण़्डवबलं सर्वमद्यैको नाशयिष्यति}


\threelineshloka
{`सिंहनादं महत्कृत्वा धनुर्बाणरवैः सह}
{विद्राव्यमाणं सम्पश्य हतभूयिष्ठनायकम्}
{दृष्ट्वा विनद्य सहसा मम सेनां प्रमृद्गति'}


\twolineshloka
{न चावाभ्यामृतेऽन्योऽस्ति शक्तस्तं प्रतिबाधितुम्}
{त्वरमाणस्ततो याहि यतः प्राग्ज्योतिषाधिपः}


\twolineshloka
{दृप्तं सङ्ख्ये द्विपबलाद्वयसा चापि विस्मितम्}
{अद्यैनं प्रेषयिष्यामि बलहन्तुः प्रियातिथिम्}


\twolineshloka
{वचनादथ कृष्णस्तु प्रययौ सव्यसाचिनः}
{दीर्यते भगदत्तेन यत्र पाण्डववाहिनी}


\twolineshloka
{तं प्रयान्तं ततः पश्चादाह्वयन्तो महारथाः}
{संशप्तकाः समारोहन्सहस्राणि चतुर्दश}


\twolineshloka
{दशैव तु सहस्राणि त्रिगर्तानां महारथाः}
{चत्वारि च सहस्राणि वासुदेवस्य चानुगाः}


\twolineshloka
{दीर्यमाणां चमूं दृष्ट्वा भगदत्तेन मानिष}
{आहूयमानस्य च तैरभवद्धृदयं द्विधा}


\twolineshloka
{किन्नु श्रेयस्करं कर्म भवेदद्येत्यचिन्तयत्}
{इह वा विनिवर्तेयं गच्छेयं वा युधिष्ठिरम्}


\twolineshloka
{तस्य बुद्ध्या विचार्यैवमर्जुनस्य कुरूद्वह}
{अभवद्भूयसी बुद्धिः संशप्तकवधे स्थिरा}


\twolineshloka
{स सन्निवृत्तः सहसा कपिप्रवरकेतनः}
{एको रथसहस्राणि निहन्तुं वासवी रणे}


\twolineshloka
{सा हि दुर्योधनस्यासीन्मतिः कर्णस्य चोभयोः}
{अर्जुनस्य वधोपाये या वै द्वैधमकल्पयत्}


\twolineshloka
{स तु दोलायमानोऽबूद्द्वैधीभावेन पाण्डवः}
{वधेन तु नराग्र्याणामकरोत्तां मृषा तदा}


\twolineshloka
{ततः शतसहस्राणि शराणां नतपर्वणाम्}
{असृजन्नर्जुने राजन्संशप्तकमहारथाः}


\twolineshloka
{नैव कुन्तीसुतः पार्थो नैव कृष्णो जनार्दनः}
{न हया न रथो राजन्दृश्यन्ते स्म शरैश्चिताः}


\twolineshloka
{तदा मोहमनुप्राप्तः सिष्विदे हि जनार्दनः}
{ततस्तान्प्रायशः पार्थो ब्रह्मास्त्रेण निजघ्निवान्}


\twolineshloka
{शतशः पाणयश्छिन्नाः सेषुज्यातलकार्मुकाः}
{केतवो वाजिनः सूता रथिनश्चापतन्क्षितौ}


\twolineshloka
{द्रुमाचलाग्राम्बुधरैः समकायाः सुकल्पिताः}
{हतारोहाः क्षितौ पेतुर्द्विपाः पार्थशराहताः}


\twolineshloka
{विप्रविद्धकुथा नागाश्छिन्नभाण्डाः परासवः}
{सारोहास्तु रणे पेतुर्मथिता मार्गणैर्भृशम्}


\twolineshloka
{सर्ष्टिप्रासासिनखराः समुद्गरपरश्वथाः}
{विच्छिन्ना बाहवः पेतुर्नृणां भल्लैः किरीटिना}


\twolineshloka
{बालादित्याम्बुजेन्दूनां तुल्यरूपाणि मारिष}
{संच्छिन्नान्यर्जुनशरैः शिरांस्युर्वी प्रपेदिरे}


\twolineshloka
{जज्वालालङ्कृता सेना पत्रिभिः प्राणिभोजनैः}
{नानारूपैस्तदाऽमित्रान्क्रुद्धेनिघ्नति फल्गुने}


\twolineshloka
{क्षोभयन्तं तदा सेनां द्विरदं नलिनीमिव}
{धनञ्जयं भूतगणाः साधुसाध्वित्यपूजयन्}


\twolineshloka
{दृष्ट्वा तत्कर्म पार्थस्य वासवस्येव माधवः}
{विस्मयं परमं गत्वा प्राञ्जलिस्तमुवाच ह}


\twolineshloka
{कर्मैतत्पार्थ शक्रेण यमेन धनदेन च}
{दुष्करं समरे यत्ते कृतमद्येति मे मतिः}


\twolineshloka
{युगपच्चैव सङ्ग्रामे शतशोऽथ सहस्रशः}
{पतिता एव मे दृष्टाः संशप्तकमहारथाः}


\twolineshloka
{संशप्तकांस्ततो हत्वा भूयिष्ठा ये व्यवस्थिताः}
{भगदत्ताय याहीति कृष्णं पार्थोऽभ्यनोदयत्}


\chapter{अध्यायः २८}
\twolineshloka
{सञ्जय उवाच}
{}


\twolineshloka
{यियासतस्ततः कृष्णः पार्थस्याश्वान्मनोजवान्}
{सम्प्रैषीद्धेमसञ्छन्नान्द्रोणानीकाय सन्त्वरन्}


\twolineshloka
{तं प्रयान्तं कुरुश्रेष्ठं स्वांस्त्रातुं द्रोणतापितान्}
{सुशर्मा भ्रातृभिः सार्धं युद्धार्थी पृष्ठतोऽन्वयात्}


\twolineshloka
{ततः श्वेतहयः कृष्णमब्रवीदजितञ्जयः}
{एष मां भ्रातृभिः सार्धं सुशर्माऽऽह्वयतेऽच्युत}


\twolineshloka
{दीर्यते चोत्तरेणैव तत्सैन्यं मधुसूदन}
{द्वैधीभूतं मनो मेऽद्य कृतं संशप्तकैरिदम्}


\twolineshloka
{किन्नु संशप्तकान्हन्मि स्वान्रक्षाम्यहितार्दितान्}
{इति मे त्वं मतं वेत्सि तत्र किं सुकृतं भवेत्}


\twolineshloka
{एवमुक्तस्तु दाशार्हः स्यन्दनं प्रत्यवर्तयत्}
{येन त्रिगर्ताधिपतिः पाण्डवं समुपाह्वयत्}


\twolineshloka
{ततोऽर्जुनः सुशर्माणं विद्ध्वा सप्तभिराशुगैः}
{ध्वजं धनुश्चास्य तथा क्षुराभ्यां समकृन्तत}


\twolineshloka
{त्रिगर्ताधिपतेश्चापि भ्रातरं षङ्भिराशुगैः}
{साश्वं ससूतं त्वरितः पार्थः प्रैषीद्यमक्षयम्}


\twolineshloka
{ततो भुजगसङ्काशां सुशर्मा शक्तिमायसीम्}
{चिक्षेपार्जुनमादिश्य वासुदेवाय तोमरम्}


\twolineshloka
{शक्तिं त्रिभिः शरैश्छित्त्वा तोमरं त्रिभिरर्जुनः}
{सुशर्माणं शरव्रातैर्मोहयित्वा न्यवर्तयत्}


\twolineshloka
{तं वासवमिवायान्तं भूरिवर्षं शरौघिणम्}
{राजंस्तावकसैन्यानां नोग्रं कश्चिदवारयत्}


\twolineshloka
{ततो धनञ्जयो बाणैः सर्वानेव महारथान्}
{आयाद्विनिघ्नन्कौरव्यान्दहन्कक्षमिवानलः}


\twolineshloka
{तस्य वेगमसह्यं तं कुन्तीपुत्रस्य धीमतः}
{नाशक्नुवंस्ते संसोढुं स्पर्शमग्नेरिव प्रजाः}


\twolineshloka
{संवेष्टयन्ननीकानि शरवर्षेण पाण्डवः}
{सुपर्ण इव नागेन्द्रं प्रायात्प्राग्ज्योतिषं प्रति}


\twolineshloka
{यत्तदानामयज्जिष्णुर्भरतानामपापिनाम्}
{धनुः क्षेमकरं सङ्ख्ये द्विषतामश्रुवर्धनम्}


\twolineshloka
{तदेव तव पुत्रस्य राजन्दुर्द्यूतदेविनः}
{कृते क्षत्रविनाशाय धनुरायच्छदर्जुनः}


\twolineshloka
{तथा विक्षोभ्यमाणा सा पार्थेन तव वाहिनी}
{व्यशीर्यत महाराज नौरिवासाद्य पर्वतम्}


\twolineshloka
{ततो दशसहस्राणि न्यवर्तन्त धनुष्मताम्}
{मतिं कृत्वा रणे क्रूरां वीरा जयपराजये}


\twolineshloka
{व्यपेतहृदयत्रासा आवव्रुस्तं महारथाः}
{आर्च्छत्पार्थो गुरुं भारं सर्वभारसहो युधि}


\twolineshloka
{यथा नलवनं क्रुद्धः प्रभिन्नः षष्टिहायनः}
{मृद्गीयात्तद्वदायस्तः पार्थोऽमृद्गाच्चमूं तव}


\twolineshloka
{तस्मिन्प्रमथिते सैन्ये भगदत्तो नराधिपः}
{तेन नागेन सहसा धनञ्जयमुपाद्रवत्}


\twolineshloka
{तं रथेन नरव्याघ्रः प्रत्यगृह्णाद्धनञ्जयः}
{स सन्निपातस्तुमुलो बभूव नरनागयोः}


\twolineshloka
{कल्पिताभ्यां यथाशास्त्रं रथेन च गजेन च}
{सङ्ग्रामे चेरतुर्वीरौ भगदत्तधनञ्जयौ}


\twolineshloka
{ततो जीमूतसङ्काशान्नागादिन्द्र इव प्रभुः}
{अभ्यवर्षच्छरौघेण भगदत्तो धनञ्जयम्}


\twolineshloka
{स चापि शरवर्षं तं शरवर्षेण वासिवः}
{अप्राप्तमेव चिच्छेद भगदत्तस्य वीर्यवान्}


\twolineshloka
{ततः प्राग्ज्योतिषो राजा शरवर्षे निवारिते}
{संरक्तनयनो रोषात्पार्थं पुनरवाकिरत्}


\twolineshloka
{स तथा शरवर्षेण पार्थं समभिहत्य वै}
{चोदयामास तं नागं वधायाच्युतपार्थयोः}


\twolineshloka
{तमापतन्तं द्विरदं दृष्ट्वा क्रुद्धमिवान्तकम्}
{चक्रेऽपसव्यं त्वरितः स्यन्दनेन जनार्दनः}


\twolineshloka
{सम्प्राप्तमपि नेयेष परावृत्तं महाद्विपम्}
{सारोहं मृत्युसात्कर्तुं स्मरन्धर्मं धनञ्जयः}


\twolineshloka
{स तु नागो द्विपरथान्हयांश्चामृद्य मारिष}
{प्राहिणोन्मृत्युलोकाय ततः क्रुद्धो धनञ्जयः}


\chapter{अध्यायः २९}
\twolineshloka
{धृतराष्ट्र उवाच}
{}


\threelineshloka
{तथा क्रुद्धः किमकरोद्भगदत्तस्य पाण्डवः}
{प्राग्ज्योतिषो वा पार्थस्य तन्मे शंस यथातथम् ॥सञ्जय उवाच}
{}


\twolineshloka
{प्राग्ज्योतिषेण संसक्तावुभौ दाशार्हपाण्डवौ}
{मृत्युदंष्ट्रान्तिकं प्राप्तौ सर्वभूतानि मेनिरे}


\twolineshloka
{तथा तु शरवर्षाणि पातयत्यनिशं प्रभो}
{गजस्कन्धान्महाराज कृष्णयोः स्यन्दनस्थयोः}


\twolineshloka
{अथ कार्ष्णायसैर्बाणैः पूर्णकार्मुकनिःसृतैः}
{अविध्यद्देवकीपुत्रं हेमपुङ्खैः शिलाशितैः}


\twolineshloka
{अग्निस्पर्शसमास्तीक्ष्णा भगदत्तेन चोदिताः}
{निर्भिद्य देवकीपुत्रं क्षितिं जग्मुः सुवाससः}


\twolineshloka
{तस्य पार्थो धनुश्छित्त्वा परिवारं निहत्य च}
{लालयन्निव राजानं भगदत्तमयोधयत्}


\twolineshloka
{सोऽर्करश्मिनिभांस्तीक्ष्णांस्तोमरान्वै चतुर्दश}
{अप्रेषयत्सव्यसाची द्विधैकैकमथाच्छिनत्}


\twolineshloka
{ततो नागस्य तद्वर्म व्यधमत्पाकशासनिः}
{शरजालेन महता तद्व्यशीर्यत भूतले}


\twolineshloka
{शीर्णवर्मा स तु गजः शरैः सुभृशमर्दितः}
{बभौ धारानिपाताक्तो व्यभ्रः पर्वतराडिव}


\twolineshloka
{ततः प्राग्ज्योतिषः शक्तिं हेमदण्डामयस्मयीम्}
{व्यसृजद्वासुदेवाय द्विधा तामर्जुनोऽच्छिनत्}


\twolineshloka
{ततश्छत्रं ध्वजं चैव च्छित्त्वा राज्ञोऽर्जुनः शरैः}
{विव्याध दशभिस्तूर्णमुत्स्मयन्पर्वतेश्वरम्}


\twolineshloka
{सोऽतिविद्धोऽर्जुनशरैः सुपुङ्खैः कङ्कपत्रिभिः}
{भगदत्तस्ततः क्रुद्धः पाण्डवस्य जनाधिपः}


\twolineshloka
{व्यसृजत्तोमरान्मूर्ध्नि श्वेताश्वस्योन्नाद च}
{तैरर्जुनस्य समरे किरीटं परिवर्तितम्}


\twolineshloka
{परिवृत्तं किरीटं तद्यमयन्नेव पाण्डवः}
{सुदृष्टः क्रियतां लोक इति राजानमब्रवीत्}


\twolineshloka
{एवमुक्तस्तु सङ्क्रुद्धः शरवर्षेण पाण्डवम्}
{अभ्यवर्षत्स गोविन्दं धनुरादाय भास्वरम्}


\twolineshloka
{तस्य पार्थो धनुश्छित्त्वा तूणीरान्सन्निकृत्य च}
{त्वरमाणो द्विसप्तत्या सर्वमर्मस्वताडयत्}


\twolineshloka
{विद्धस्ततोऽतिव्यथितो वैष्णवास्त्रमुदीरयन्}
{अभिमन्त्र्याङ्कुशं क्रुद्धो व्यसृजत्पाण्डवोरसि}


\twolineshloka
{विसृष्टं भगदत्तेन तदस्त्रं सर्वघाति वै}
{उरसा प्रतिजग्राह पार्थं सञ्छाद्य केशवः}


\twolineshloka
{वैजयन्त्यभवन्माला तदस्त्रं केशवोरसि}
{पद्मकोशविचित्रा सा सर्वर्तुकुसुमोत्करा}


\twolineshloka
{ज्वलनार्केन्दुवर्णाभा पावकोज्ज्वलपल्लवा}
{तया पद्मपलाशिन्या वातकम्पितपत्रया}


\twolineshloka
{शुशुभेऽभ्यधिकं शौरिरतसीपुष्पसन्निभः}
{ततोऽर्जुनः क्लान्तमनाः केशवं प्रत्यभाषत}


\twolineshloka
{अयुध्यमानस्तुरगान्संयन्तास्मि जनार्दन}
{इत्युक्त्वा पुण्डरीकाक्ष प्रतिज्ञां स्वां न रक्षसि}


\twolineshloka
{यद्यहं व्यसनी वा स्यामशक्तो वा निवारणे}
{ततस्त्वयैवं कार्यं स्यान्न तु कार्यं मयि स्थिते}


\twolineshloka
{सबाणः सधनुश्चाहं ससुहासुरमानुषान्}
{शक्तो लोकानिमाञ्जेतुं तच्चापि विदितं तव}


\twolineshloka
{ततोऽर्जुनं वासुदेवः प्रत्युवाचार्यवद्वचः}
{शृणु गुह्यमिदं पार्थ पुरावृत्तं यथाऽनघ}


\twolineshloka
{`येयं भूतधरा देवी सर्वभूतधरा धरा}
{सकामा लोककर्तारं नारायणमुपस्थिता}


\twolineshloka
{स सङ्गत्य तया सार्धं प्रीतस्तस्या वरं ददौ}
{सा वव्रे विष्णुसदृशं पुत्रमस्त्रं च वैष्णवम्}


\twolineshloka
{बभूव च सुतस्तस्या नरको नाम विश्रुतः}
{अस्त्रं च वैष्णवं तस्मै ददौ नारायणः स्वयम्}


\twolineshloka
{तदेवं नरकस्यासीदस्त्रं सर्वाहितान्तकम्}
{तस्मात्प्राग्ज्योतिषं प्राप्तं सर्वशस्त्रविघातनम्}


\threelineshloka
{नास्यावध्योऽस्ति लोकेऽस्मिन्मदन्यः कश्चिदर्जुन}
{तस्मान्मया कृतं ह्येतन्मा भूत्ते बुद्धिरन्यथा ॥सञ्जय उवाच}
{}


\twolineshloka
{अनुनीय तु दाशार्हः पाण्डवं त्वरितोऽब्रवीत्}
{जहि प्राग्ज्योतिषं क्षिप्रं वाक्यशेषं च मे शृणु'}


\twolineshloka
{चतुर्मूर्तिरहं शश्वल्लोकत्राणार्थमुद्यतः}
{आत्मानं प्रविभज्येह लोकानां हितमादधे}


\twolineshloka
{एका मूर्तिस्तपश्चर्यां कुरुते मे भुवि स्थिता}
{अपरा पश्यति जगत्कुर्वाणं साध्वसाधुनी}


\twolineshloka
{अपरा कुरुते कर्म मानुषं लोकमाश्रिता}
{शेते चतुर्थीं त्वपरा निद्रां वर्षसहस्रिकीम्}


\twolineshloka
{याऽसौ वर्षसहस्रान्ते मूर्तिरुत्तिष्ठते मम}
{वरार्हेभ्यो वराञ्श्रेष्ठांस्तस्मिन्काले ददाति सा}


\twolineshloka
{तं तु कालमनुप्राप्तं विदित्वा पृथिवी तदा}
{अयाचत वरं यन्मां नरकार्थाय तच्छृणु}


\twolineshloka
{देवानां दानवानां च अवध्यस्तनयोऽस्तु मे}
{उपेतो वैष्णवास्त्रेण तन्मे त्वं दातुमर्हसि}


\twolineshloka
{एवं वरमहं श्रुत्वा जगत्यास्तनये तदा}
{अमोघमस्त्रं प्रायच्छं वैष्णवं परमं पुरा}


\twolineshloka
{अवोचं चैतदस्त्रं वै ह्यमोघं भवतु क्षमे}
{नरकस्याभिरक्षार्थं नैनं कश्चिद्वधिष्यति}


\twolineshloka
{अनेनास्त्रेण ते गुप्तः सुतः परबलार्दनः}
{भविष्यति दुराधर्षः सर्वलोकेषु सर्वदा}


\twolineshloka
{तथेत्युक्त्वा गता देवी कृतकामा मनस्विनी}
{स चाप्यासीद्दुराधर्षो नरकः शत्रुतापनः}


\twolineshloka
{तस्मात्प्राग्ज्योतिषं प्राप्तं तदस्त्रं पार्थ मामकम्}
{नास्यावध्योऽस्ति लोकेषु सेन्द्ररुद्रेषु मारिष}


\twolineshloka
{तन्मया त्वत्कृते चैतदन्यथैवोपनिर्मितम्}
{वियुक्तं परमास्त्रेण जहि पार्थ महासुरम्}


\fourlineindentedshloka
{`अवध्योऽयं महास्त्रेण भगदत्तः सुरासुरैः'}
{वैरिणं जहि दुर्धर्षं भगदत्तं सुरद्विषम्}
{यथाऽहं जघ्निवान्पूर्वं हितार्थं नरकं तथा ॥सञ्जय उवाच}
{}


\twolineshloka
{एवमुक्तस्तदा पार्थः केशवेन महात्मना}
{भगदत्तं शितैर्बाणैः सहसा समवाकिरत्}


\twolineshloka
{ततः पार्थो महाबाहुरसम्भ्रान्तो महामनाः}
{कुम्भयोरन्तरे नागं नाराचेन समार्पयत्}


\twolineshloka
{स समासाद्य तं नागं बाणो वज्र इवाचलम्}
{अभ्यगात्सह पुङ्खेन वल्मीकमिव पन्नगः}


\twolineshloka
{स करी भगदत्तेन प्रेर्यमाणो मुहुर्मुहुः}
{न करोति वचस्तस्य दरिद्रस्येव योषिता}


\twolineshloka
{स तु विष्टभ्य गात्राणि दन्ताभ्यामवनिं ययौ}
{नदन्नार्तस्वनं प्राणानुत्ससर्ज महाद्विपः}


\twolineshloka
{ततो गाण्डीवधन्वानमभ्यभाषत केशवः}
{अयं महत्तरः पार्थ पलितेन समावृतः}


\twolineshloka
{वलीसञ्छन्ननयनः शूरः परमद्रुर्जयः}
{अक्ष्णोरुन्मीलनार्थाय बद्धपट्टो ह्यसौ नृपः}


\twolineshloka
{देववाक्यात्प्रचिच्छेद शरेण भृशमर्जुनः}
{छिन्नमात्रेंऽशुके तस्मिन्रुद्धनेत्रो बभूव सः}


\twolineshloka
{तमोमयं जगन्मेने भगदत्तः प्रतापवान्}
{ततश्चन्द्रार्धबिम्बेन बाणेन नतपर्वणा}


\twolineshloka
{बिभेद हृदयं राज्ञो भगदत्तस्य पाण्डवः}
{स भिन्नहृदयो राजा भगदत्तः किरीटिना}


\threelineshloka
{शरासनं शरांश्चैव गतासुः प्रमुमोच ह}
{शिरसस्तस्य विभ्रष्टं पपात च वरांशुकम्}
{नालताडनविभ्रष्टं पलाशं नलिनादिव}


\twolineshloka
{स हेममाली तपनीयभाण्डात्पपात नागाद्गिरिसन्निकाशात्}
{सुपुष्पितो मारुतवेगभग्नोमहीधराग्रादिव कर्णिकारः}


\twolineshloka
{निहत्य तं नरपतिमिन्द्रविक्रमंसस्वायमिन्द्रस्य तदैन्द्रिराहवे}
{ततोऽपरांस्तव जयकाङ्क्षिणो नरान्बभञ्ज वायुर्बलवान्द्रुमानिव}


\chapter{अध्यायः ३०}
\twolineshloka
{सञ्जय उवाच}
{}


\twolineshloka
{प्रियमिन्द्रस्य सततं सखायममितौजसम्}
{हत्वा प्राग्ज्योतिषं पार्थः प्रदक्षिणमवर्तत}


\twolineshloka
{ततो गान्धारराजस्य सुतौ परपुरञ्जयौ}
{अर्देतामर्जुनं सङ्ख्ये भ्रातरौ वृषकाचलौ}


\twolineshloka
{तौ समेत्यार्जुनं वीरौ पुरः पश्चाच्च धन्वितौ}
{अविध्येतां महावेगैर्निशितैराशुगैर्भृशम्}


\twolineshloka
{वृषकस्य हयान्सूतं धनुश्छत्रं रथं ध्वजम्}
{तिलशो व्यधमत्पार्थः सौबलस्य शितैः शरैः}


\twolineshloka
{ततोऽर्जुनः शरव्रातैर्जानाप्रहरणैरपि}
{गान्धारानाकुलांश्चक्रे सौबलप्रमुखान्पुनः}


\twolineshloka
{ततः पञ्चशतान्वीरान्गान्धारानुद्यतायुधान्}
{प्राहिणोन्मृत्युलोकाय क्रुद्धो बाणैर्धनञ्जयः}


\twolineshloka
{हताश्वात्तु रथात्तूर्णमवतीर्य महाभुजः}
{आरुरोह रथं भ्रातुरन्यश्च धनुराददे}


\twolineshloka
{तावेकरथमारूढौ भ्रातरौ वृषकाचलौ}
{शरवर्षेण बीभत्सुमविध्येतां मुहुर्मुहुः}


\twolineshloka
{स्यालौ तव महात्मानौ राजानौ वृषकाचलौ}
{भृशं विजघ्नतुः पार्थमिन्द्रं वृत्रबलाविव}


\twolineshloka
{लब्धलक्षौ तु गान्धारावहतां पाण्डवं पुनः}
{नैदाघवार्षिकौ मासौ लोकं घर्मांशुभिर्यथा}


\twolineshloka
{तौ रथस्थौ नरव्याघ्रौ राजानौ वृषकाचलौ}
{संश्चिष्टाङ्गौ स्थितौ राजञ्जघानैकेषुणाऽर्जुनः}


\twolineshloka
{तौ रथात्सिंहसंकाशौ लोहिताक्षौ महाभुजौ}
{राजन्सम्पेततुर्वीरौ सोदर्यावेकलक्षणौ}


\twolineshloka
{तयोर्भूमिं गतौ देहौ रथाद्बन्धुजनप्रियौ}
{यशो दशः दिशः पुण्यं गमयित्वा व्यवस्थितौ}


\twolineshloka
{दृष्ट्वा विनिहतौ सङ्ख्ये मातुलावपलायिनौ}
{भृशं मुमुचुरश्रूणि पुत्रास्तव विशांपते}


\twolineshloka
{निहतौ भ्रातरौ दृष्ट्वा मायाशतविशारदः}
{कृष्णौ सम्मोहन्मायां विदधे शकुनिस्ततः}


\twolineshloka
{लगुडायोगुडाश्मानः शतघ्न्यश्च सशक्तयः}
{गदापरिघनिस्त्रिंशशूलमुद्गरपट्टसाः}


\twolineshloka
{सकम्पनर्ष्टिनखरा मुसलानि परश्वथाः}
{क्षुराः क्षुरप्रनालीका वत्सदन्तास्थिसन्धयः}


\twolineshloka
{चक्राणि विशिखाः प्रासा विविधान्यायुधानि च}
{प्रपेतुः शतशो दिग्भ्यः प्रदिग्भ्यश्चार्जुनं प्रति}


\twolineshloka
{खरोष्ट्रमहिषाः सिंहा व्याघ्राः सृमरचित्रकाः}
{ऋक्षाः सालावृका गृध्राः कपयश्च सरीसृपाः}


\twolineshloka
{विविधानि च रक्षांसि क्षुधितान्यर्जुनं प्रति}
{सङ्क्रुद्धान्यभ्यधावन्त विविधानि वयांसि च}


\twolineshloka
{ततो दिव्यास्त्रविच्छूरः कुन्तीपुत्रो धनञ्जयः}
{विसृजन्निषुजालानि सहसा तान्यताडयत्}


\twolineshloka
{ते हन्यमानाः शूरेण प्रवरैः सायकैर्दृढैः}
{विरुवन्तो महारावान्विनेशुः सर्वतो हताः}


\twolineshloka
{ततस्तमः प्रादुरभूदर्जुनस्य रथं प्रति}
{तस्माच्च तमसो वचाः क्रूराः पार्थमभर्त्सयन्}


\twolineshloka
{तत्तमो भैरवं घोरं भयकर्तृ महाहवे}
{उत्तमास्त्रेण महता ज्यौतिषेणार्जुनोऽवधीत्}


\threelineshloka
{हते तस्मिञ्जलौघास्तु प्रादुरासन्भयानकाः}
{अम्भसस्तस्य नाशार्थमादित्यास्त्रमथार्जुनः}
{प्रायुङ्क्ताम्भस्ततस्तेन प्रायशोऽस्त्रेण शोषितम्}


\twolineshloka
{एवं बहुविधा मायाः सौबलस्य कृताः कृताः}
{जघानास्त्रबलेनाशु प्रहसन्नर्जुनोऽब्रवीत्}


\twolineshloka
{`दुर्द्यूतदेविन्गान्धारे नाक्षान्क्षिपति गाण्डिवम्}
{ज्वलितान्निशितांस्तीक्ष्णाञ्शरान्क्षिपतिगाण्डिवं'}


\twolineshloka
{तदा हतासु मायासु त्रस्तोऽर्जुनशराहतः}
{अपायाञ्जवनैरश्वैः शकुनिः प्राकृतो यथा}


\twolineshloka
{ततोऽर्जुनोऽस्त्रविच्छैघ्र्यं दर्शयन्नात्मनोऽरिषु}
{अभ्यवर्षच्छरौधेण कौरवाणामनीकिनीम्}


\twolineshloka
{सा हन्यमाना पार्थेन तव पुत्रस्य वाहिनी}
{द्वैधीभूता महाराज गङ्गेवासाद्य पर्वतम्}


\twolineshloka
{द्रोणमेवान्वपद्यन्त केचित्तत्र नरर्षभाः}
{केचिद्दुर्योधनं राजन्नर्द्यमानाः किरीटिना}


\twolineshloka
{नापश्याम ततस्त्वेनं सैन्ये वै रजसावृते}
{गाण्डीवस्य च निर्घोषः श्रुतो दक्षिणतो मया}


\twolineshloka
{शङ्खदुन्दुभिनिर्घोषं वादित्राणां च निःस्वनम्}
{गाण्डीवस्य तु निर्घोषो व्यतिक्रम्यास्पृशद्दिवम्}


\twolineshloka
{ततः पुनर्दक्षिणतः सङ्ग्रामश्चित्रयोधिनाम्}
{सुयुद्धं चार्जुनस्यासीदहं द्रोणमन्वियाम्}


\threelineshloka
{यौधिष्ठिराभ्यनीकानि प्रहरन्ति ततस्ततः}
{नानाविधान्यनीकानि पुत्राणां तव भारत}
{अर्जुनो व्यधमत्काले दिवीवाभ्राणि मारुतः}


\twolineshloka
{तं वासवमिवायान्तं भूरिवर्षं शरौघिणम्}
{महेष्वासा नरव्याघ्रा नोग्रं केचिदवारयन्}


\twolineshloka
{ते हन्यमानाः पार्थेन त्वदीया व्यथिता भृशम्}
{स्वानेव बहवो जघ्नुर्विद्रवन्तस्ततस्ततः}


\twolineshloka
{तेऽर्जुनेन शरा मुक्ताः कङ्कपत्रास्तनुच्छिदः}
{शलभा इव सम्पेतुः संवृण्वाना दिशो दश}


\twolineshloka
{तुरगं रथिनं नागं पदातिमपि मारिष}
{विनिर्भिद्य क्षितिं जग्मुर्वल्मीकमिव पन्नगाः}


\twolineshloka
{न च द्वितीयं व्यसृजत्कुञ्जराश्वनरेषु सः}
{पृथगेकशरारुग्णा निपेतुस्ते गतासवः}


\twolineshloka
{हतैर्मनुष्यौर्द्विरदैश्च सर्वतःशराभिसृष्टैश्च हयैर्निपातितैः}
{तदाऽश्वगोमायुबलाभिनादितंविचित्रमायोधशिरो बभूव तत्}


\twolineshloka
{पिता सुतं त्यजति सुहृद्वरं सुहृत्तथैव पुत्रः पितरं शरातुरः}
{स्वरक्षणे कृतमतयस्तदा जनासत्यजन्ति वाहानपि पार्थपीडिताः}


\chapter{अध्यायः ३१}
\twolineshloka
{धृतराष्ट्र उवाच}
{}


\twolineshloka
{तेष्वनीकेषु भग्नेषु पाण्डुपुत्रेण सञ्जय}
{5-31-1bचलितानांद्रुतानां च कथमासीन्मनांसि वः}


\threelineshloka
{आनीकानां प्रभग्नानामवस्थानमपश्यताम्}
{दुष्करं प्रतिसन्धानं तन्ममाचक्ष्व सञ्जय ॥सञ्जय उवाच}
{}


\twolineshloka
{तथापि तव पुत्रस्य प्रियकामा विशाम्पते}
{यशः प्रवीरा लोकेषु रक्षन्तो द्रोणमन्वयुः}


\twolineshloka
{समुद्यतेषु चास्त्रेषु सम्प्राप्ते च युधिष्ठिरे}
{अकुर्वन्नार्यकर्माणि भैरवे सत्यभीतवत्}


\twolineshloka
{अन्तरं भीमसेनस्य प्रापतन्नमितौजसः}
{सात्यकेश्चैव वीरस्य धृष्टद्युम्नस्य वा विभो}


\twolineshloka
{द्रोणं द्रोणमिति क्रूराः पाञ्चालाः समचोदयन्}
{मा द्रोणमिति पुत्रास्ते कुरून्सर्वानचोदयन्}


\twolineshloka
{द्रोणं द्रोणमिति ह्येके मा द्रोणमिति चापरे}
{कुरूणां पाण्डवानां च द्रोणद्यूतमवतेत}


\twolineshloka
{यंयं प्रमथते द्रोणः पाञ्चालानां रथव्रजम्}
{तत्रतत्र तु पाञ्चाल्यो धृष्टद्युम्नोऽभ्यवर्तत}


\twolineshloka
{तेषां भाग्यविपर्यासैः सङ्ग्रामे भैरवे सति}
{वीराः समासदन्वीरान्कुर्वन्तो भैरवं रवम्}


\twolineshloka
{अकम्पनीयाः शत्रूणां बभूवुस्तत्र पाण्डवाः}
{अकम्पयन्ननीकानि स्मरन्तः क्लेशमात्मनः}


\twolineshloka
{ते त्वमर्षवशं प्राप्ता हीमन्तः सत्वचोदिताः}
{त्यक्त्वा प्राणान्न्यवर्तन्त घ्नन्तो द्रोणं महाहवे}


\twolineshloka
{अयासमिव सम्पातः शिलानामिव चाभवत्}
{दीव्यतां तुमुले युद्धे प्राणैरमिततेजसाम्}


\twolineshloka
{नानुस्मरन्ति सङ्ग्राममपि वृद्धास्तथाविधम्}
{ये पूर्वदेवैर्देवानामपश्यन्युद्धमद्भुतम्}


\twolineshloka
{प्राकम्पतेव पृथिवी तस्मिन्वीरावसादने}
{निवर्तता बलौघेन महता भारपीडिता}


\twolineshloka
{घूर्णतोऽपि बलौघस्य दिवं स्तब्ध्वेव निःस्वनः}
{अजातशत्रोस्तत्सैन्यमाविवेश सुभैरवः}


\twolineshloka
{समासाद्य तु पाण्डूनामनीकानि सहस्रशः}
{द्रोणेन चरता सङ्ख्ये प्रभग्नानि शितैः शरैः}


\twolineshloka
{तेषु प्रमथ्यमानेषु द्रोणेनाद्भुतकर्मणा}
{पर्यवरायदासाद्य द्रोणं सेनापतिः स्वयम्}


\twolineshloka
{तदद्भुतमभूद्युद्धं द्रोणपाञ्चालयोस्तथा}
{नैव तस्योपमा काचिदिति मे निश्चिता मतिः}


\twolineshloka
{ततो नीलोऽनलप्रख्यो ददाह कुरुवाहिनीम्}
{शरस्कुलिङ्गश्चापार्चिर्दिहन्कक्षमिवानलः}


\twolineshloka
{तं दहन्तमनीकानि द्रोणपुत्रः प्रतापवान्}
{पूर्वाभिभाषी सुश्लक्ष्णं स्मयमानोऽभ्यभाषत}


\twolineshloka
{नील किं बहुभिर्दग्धैस्तव योधैः सरार्चिषा}
{मयैकेन हि युध्यस्व क्रुद्धः प्रहर चाशु माम्}


\twolineshloka
{तं पद्मनिकराकारं पद्मपत्रनिभेक्षणम्}
{व्याकोशपद्माभमुखो नीलो विव्याध सायकैः}


\twolineshloka
{तेनापि विद्धः सहसा द्रौणिर्भल्लैः शितैस्त्रिभिः}
{धनुर्ध्वजं च च्छत्रं च द्विषतः सन्न्यकृन्तत}


\twolineshloka
{स प्लुतः स्यन्दनात्तस्मान्नीलश्चर्मवरासिभृत्}
{द्रौणायनेः शिरः कायाद्धर्तुमैच्छत्पतत्त्रिवत्}


\twolineshloka
{तस्योद्यतासेः सुनसं शिरः कायात्सकुण्डलम्}
{भल्लेनापाहरद्द्रौणिः स्मयमान इवानघ}


\twolineshloka
{सम्पूर्णचन्द्राभमुखः पद्मपत्रनिभेक्षणः}
{प्रांशुरुत्पलपत्राभो निहतो न्यपतद्भुवि}


\twolineshloka
{ततः प्रविव्यथे सेना पाण्डवी भृशमाकुला}
{आचार्यपुत्रेण हते नीले ज्वलिततेजसि}


\twolineshloka
{अचिन्तयंश्च ते सर्वे पाण्डवानां महारथाः}
{कथं नो वासविस्त्रायाच्छत्रुभ्य इति मारिष}


\twolineshloka
{दक्षिणेन तु सेनायाः कुरुते कदनं बली}
{संशप्तकावशेषस्य नारायणबलस्य च}


\chapter{अध्यायः ३२}
\twolineshloka
{सञ्जय उवाच}
{}


\twolineshloka
{प्रतिघातं तु सैन्यस्य नामृष्यत वृकोदरः}
{सोऽभ्याहनद्गुरुं षष्ट्या कर्णं च दशभिः शरैः}


\twolineshloka
{तस्य द्रोणः शितैर्बाणैस्तीक्ष्णधारैरजिह्मगैः}
{जीवितान्तमभिप्रेप्सुर्मर्मण्याशु जघान ह}


\twolineshloka
{आनन्तर्यमभिप्रेप्सुः षड्विंशत्या शमार्पयत्}
{कर्णो द्वादशभिर्बाणैरश्वत्थामा च सप्तभिः}


\twolineshloka
{षङ्भिर्दुर्योधनो राजा तत एनमथाकिरत्}
{भीमसेनोऽपि तान्सर्वान्प्रत्यविध्यन्महाबलः}


\twolineshloka
{द्रोणं पञ्चाशतेषूणां कर्णं च दशभिः शरैः}
{दर्योधनं द्वादशभिर्दौणिमष्टाभिराशुगैः}


\twolineshloka
{आरावं तुमुलं कुर्वन्नभ्यवर्तत तान्रणे}
{तस्मिन्सन्त्यति प्रामान्मृत्युसाधारणीकृते}


\twolineshloka
{अजातशत्रुस्तान्योधान्भीमं त्रातेत्यचोदयत्}
{ते ययुर्भीमसेनस्य समीपममितौजसः}


\twolineshloka
{युयुधानप्रभृतयो माद्रीपुत्रौ च पाण्डवौ}
{ते समेत्य सुसंरब्धाः सहिताः पुरुषर्षभाः}


\twolineshloka
{महेष्वासवरैर्गुप्ता द्रोणानीकं बिभित्सवः}
{समापेतुर्महावीर्या भीमप्रभृतयो रथाः}


\twolineshloka
{तान्प्रत्यगृह्णादव्यग्रो द्रोणोऽपि रथिनां वरः}
{महारथानतिबलान्वीरान्समरयोधिनः}


\twolineshloka
{बाह्यं मृत्युभयं कृत्वा तावकान्पाण्डवा ययुः}
{सादिनः सादिनोऽब्यघ्नंस्तथैव रथिनो रथान्}


\twolineshloka
{असिशक्त्यृष्टिसङ्घातैर्युद्धमासीत्परश्वथैः}
{प्रकृष्टमसियुद्धं च बभूव कटुकोदयम्}


\twolineshloka
{कुञ्जराणां च सम्पाते युद्धमासीत्सुदारुणम्}
{अपतत्कुञ्जरादन्यो हयादन्यस्त्ववाक्शिराः}


\twolineshloka
{नरो बाणविनिर्भिन्नो रथादन्यश्च मारिष}
{तत्रान्यस्य च सम्पर्दे पतितस्य विवर्मणः}


\twolineshloka
{शिरः प्रध्वंसयामास वक्षस्याक्रम्य कुञ्जरः}
{अपरांश्चापरेऽमृद्रन्वारणाः पतितान्नरान्}


\twolineshloka
{विषाणैश्चावनिं गत्वा व्यभिन्दन्रथिनो बहून्}
{नरान्त्रैः केचिदपरे विषाणालग्नसंश्रयैः}


\twolineshloka
{बभ्रमुः समरे नागा मृद्रन्तः शतशो नरान्}
{कार्ष्णायसतनुत्राणान्नराश्वरथकुञ्जरान्}


\twolineshloka
{पतितान्पोथयाञ्चक्रुर्द्विपाः स्थूलनलानिव}
{गृध्रपत्राधिवासांसि शयनानि नारधिपाः}


\twolineshloka
{हीमन्तः कालसम्पर्कात्सुदुःस्वान्यनुशेरते}
{हन्ति स्मात्र पिता पुत्रं रथेनाभ्येत्य संयुगे}


\twolineshloka
{पुत्रश्च पितरं मोहान्निर्मर्यादमवर्तत}
{रथो भग्नो ध्वजश्छिन्नश्छत्रमुर्व्यों निपातितम्}


\twolineshloka
{युगार्धं च्छिन्नमादाय प्रदुद्राव तथा हयः}
{सासिर्बाहुर्निपतितः शिरश्छिन्नं सकुण्डलम्}


\twolineshloka
{गजेनाक्षिप्य बलिना रथः सञ्चूर्णितः क्षितौ}
{रथिना ताडितो नागो नाराचेनापतत्क्षितौ}


\twolineshloka
{सारोहश्चापतद्वाजी गेजेनाभ्याहतो भृशम्}
{निर्मर्यादं महद्युद्धमवर्तत सुदारुणम्}


\threelineshloka
{हा तात हा पुत्र सखे क्वासि तिष्ठ क्व धावसि}
{प्रहराहर जह्मेनं स्मितश्वेडितगर्जितैः}
{इत्येवमुच्चैरत्यर्थं श्रूयन्ते विविधा गिरः}


\twolineshloka
{नरस्याश्वस्य नाग्स्य समसञ्जत शोणितम्}
{उपाशाम्यद्रजो भौमं भीरून्कश्मलमाविशत्}


\twolineshloka
{चक्रेण चक्रमासाद्य वीरो वीरस्य संयुगे}
{अतीतेषुपथे काले जहार गदया सिरः}


\twolineshloka
{असीत्केशपरामर्शो मुष्टियुद्धं च दारुणम्}
{नखैर्दन्तैश्च सूराणामद्वीपे द्वीपमिच्छताम्}


\twolineshloka
{तत्राच्छिद्यत शूरस्य सखङ्गो बाहुरुद्यतः}
{सधुनश्चापरस्यापि सशरः साङ्कुशस्तथा}


\twolineshloka
{आक्रोशदन्यमन्योऽत्र तथाऽन्यो विमुखोऽद्रवत्}
{अन्यः प्राप्तस्य चान्यस्य शिरः कायादपाहरत्}


\twolineshloka
{सशब्दमद्रवच्चान्यः शब्दादन्योऽत्रसद्भृशम्}
{स्वानन्योऽथ परानन्यो जघान निशितैः शरैः}


\twolineshloka
{गिरिशृङ्गोपमश्चात्र नाराचेन निपातितः}
{मातङ्गो न्यपतद्भूमौ नदीरोध इवोष्णगे}


\twolineshloka
{तथैव रथिनं नागः क्षरन्गिरिरिवारुजन्}
{अभ्यतिष्ठत्पदा भूमौ सहाश्वं सहसारथिम्}


\twolineshloka
{शूरान्प्रहरतो दृष्ट्वा कृताश्स्त्रान्रुधिरोक्षितान्}
{बहूनप्याविशन्मोहो भीरून्हृदयदुर्बलान्}


\twolineshloka
{सर्वमाविग्नमभवन्न प्राज्ञायत किञ्चन}
{सैन्येन रजसा ध्वस्तं निर्मर्यादमवर्तत}


\twolineshloka
{ततः सेनापतिः शीघ्रमयं क्राल इति ब्रुवन्}
{नित्याभित्वरितानेव त्वरयामास पाण्डवान्}


\twolineshloka
{कुर्वन्तः शासनं तस्य पाण्डवा बाहुशालिनः}
{सरो हंसा इवापेतुर्घ्नन्तो द्रोणरथं प्रति}


\twolineshloka
{गृह्णीताद्रवतान्योन्यं विभीता विनिकृन्तत}
{इत्यासीत्तुमुलः शब्दो दुर्धर्षस्य रथं प्रति}


\twolineshloka
{ततो द्रोणः कृपः कर्णो द्रौणी राजा जयद्रथः}
{विन्दानुविन्दावावन्त्यौ शल्यश्चैतान्न्यवारयन्}


\twolineshloka
{ते त्वार्यधर्मसंरब्धा दुर्निवारा दुरासदाः}
{शरार्ता न जहुर्द्रोणं पाञ्चालाः पाण्डवै सह}


\twolineshloka
{ततो द्रोणोऽतिसंक्रुद्धो विसृजञ्शतशः शरान्}
{चेदिपाञ्चालपाण्डूनामकरोत्कदनं महत्}


\twolineshloka
{तस्य ज्यातलनिर्घोषः शुश्रुवे दिक्षं मारिष}
{वज्रसंह्रादसङ्काशस्त्रासयन्मानवान्बहून्}


\twolineshloka
{एतस्मिन्नन्तरे जिष्णुर्जित्वा संशप्तकान्बहून्}
{अभ्ययात्तत्र यत्रासौ द्रोणः पाण्डून्प्रमर्दति}


\twolineshloka
{ताञ्शरौघान्महावर्ताञ्शोणितोदान्महाहदान्}
{तीर्णः संशप्तकान्हत्वा प्रत्यदृश्यत फल्गुनः}


\twolineshloka
{तस्य कीर्तिमतो लक्ष्म सूर्यप्रतिमतेजसः}
{दीप्यमानमपश्याम तेजसा वानरध्वजम्}


\twolineshloka
{संशप्तकसमुद्रं तमुच्छोष्यास्त्रगभस्तिभिः}
{स पाण्डवयुगान्तार्कः कुरूनप्यभ्यतीतपत्}


\twolineshloka
{प्रददाह कुरून्सर्वानर्जुनः शस्त्रतेजसा}
{युगान्ते सर्वभूतानि धूमकेतुरिवोत्थितः}


\twolineshloka
{तेन बाणसहस्रौधैर्गजाश्वरथयोधिनः}
{ताड्यमानाः क्षितिं जग्मुर्मुक्तकेशाःशरार्दिताः}


\twolineshloka
{केचिदार्तस्वनं चक्रुर्विनेशुरपरे पुनः}
{पार्थबाणहताः केचिन्निपेतुर्विगतासवः}


\twolineshloka
{तेषामुत्पतितान्कांश्चित्पतितांश्च पराङ्मुखान्}
{न जघानार्जुनो योधान्योधव्रतमनुस्मरन्}


\twolineshloka
{ते विकीर्णरथाश्चित्राः प्रायशश्च पराङ्मुखः}
{कुरवः कर्णकर्णेति हाहेति च विचुक्रुशुः}


\twolineshloka
{तमाधिरथिराक्रन्दं विज्ञाय शरणैषिणाम्}
{मा भैष्टेति प्रतिश्रुत्य ययावभिमुखोऽर्जुनम्}


\twolineshloka
{स भारतरथश्रेष्ठः सर्वभारतहर्षणः}
{प्रादुश्चक्रे तदाग्नेयमस्त्रमस्त्रविदां वारः}


\threelineshloka
{तस्य दीप्तशरौघस्य दीप्तचापधरस्य च}
{`वारुणेन तदस्त्रेण विधूयाथ धनञ्जयः'}
{शरौघाञ्शरजालेन विदुधाव धनञ्जयः}


\twolineshloka
{तथैवाधिरथिस्तस्य बाणाज्ज्वलिततेजसः}
{अस्त्रमस्त्रेण संवार्य प्राणदद्विसृजञ्शरान्}


\twolineshloka
{धृष्टद्युम्नश्च भीमश्च सात्यकिश्च महारथः}
{विव्यधुः कर्णमासाद्य त्रिभिस्त्रिभिरजिह्मगैः}


\twolineshloka
{अर्जुनास्त्रं तु राधेयः संवार्य शरवृष्टिभिः}
{तेषां त्रयाणां चापानि चिच्छेद विशिखैस्त्रिभिः}


\twolineshloka
{ते निकृत्तायुघाः शूरा निर्विषा भुजगा इव}
{रथशक्तीः समुत्क्षिप्य भृशं सिंहा इवानदन्}


\twolineshloka
{ता भुजाग्रैर्महावेगा निसृष्टा भुजगोपमाः}
{दीप्यमाना महाशक्त्यो जग्मुराधिरथिं प्रति}


\twolineshloka
{ता निकृत्य शरव्रातैस्त्रिभिस्त्रिभिरजिह्मगैः}
{ननाद बलवान्कर्णः पार्थाय विसृजञ्शरान्}


\twolineshloka
{अर्जुनश्चापि राधेयं विद्ध्वा सप्तभिराशुगैः}
{कर्णादवरजं बाणैर्जघान निशितैः शरैः}


\twolineshloka
{ततः शत्रुंजयं हत्वा पार्थः षङ्भिरजिह्मगैः}
{जहार सद्यो भल्लेन विपाटस्य शिरो रथात्}


\twolineshloka
{पश्यतां धार्तराष्ट्राणामेकेनैव किरीटिना}
{प्रमुखे सूतपुत्रस्य सोदर्या निहतास्त्रयः}


\twolineshloka
{ततो भीमः समुत्पत्य स्वरथाद्वैनतेयवत्}
{वरासिना कर्णपक्षाञ्जधान दश पञ्च च}


\twolineshloka
{पुनस्तु रथमास्थाय धनुरादाय चापरम्}
{विव्याध दशभिः कर्णं सूतमश्वांश्च पञ्चभिः}


\twolineshloka
{धृष्टद्युम्नोऽप्यसिवरं चर्म चादाय भास्वरम्}
{जघान चन्द्रवर्माणं बृहत्क्षत्रं च नैषधम्}


\twolineshloka
{ततः स्वरथमास्थाय पाञ्चाल्योऽन्यच्च कार्मुकम्}
{आदाय कर्णं विव्याध त्रिसप्तत्या नदन्रणे}


\twolineshloka
{शैनेयोऽप्यन्यदादाय धनुरिन्दुसमद्युतिः}
{सूतपुत्रं चतुःषष्ट्या विद्ध्वा सिंह इवानदत्}


\twolineshloka
{भल्लाभ्यां साधु मुक्ताभ्यां छित्त्वा कर्णस्य कार्मुकम्}
{पुनः कर्णं त्रिभिर्बाणैर्बाह्वोरुरसि चार्पयत्}


\twolineshloka
{ततो दुर्योधनो द्रोणो राजा चैव जयद्रथः}
{निम़ज्जमानं राधेयमुज्जह्रुः सात्यकार्णवात्}


\twolineshloka
{पत्त्यश्वरथमातङ्गास्त्वदीयाः शतशोऽपरे}
{कर्णमेवाभ्यधावन्त त्रास्यमानाः प्रहारिणः}


\twolineshloka
{धृष्टद्युम्नश्च भीमश्च सौभद्रोऽर्जुन एव च}
{नकुलः सहदेवश्च सात्यकिं जुगुपू रणे}


\threelineshloka
{एवमेष महारौद्रः क्षयार्थं सर्वधन्विनाम्}
{तावकानां परेषां च त्यक्त्वा प्राणानभूद्रणः ॥पदातिरथनागाश्वा गजाश्वरथपत्तिभिः}
{रथिनो नागपत्त्यश्वैरथ पत्तीरथ द्विपैः}


\threelineshloka
{अश्वैरश्वा गजैर्नागा रथिनो रथिभिः सह}
{संयुक्ताः समदृश्यन्त पत्तयश्चापि पत्तिभिः ॥एवं सुकलिलं युद्धमासीत्क्रव्यादहर्षणम्}
{महद्भिस्तैरभीतानां यमराष्ट्रविवर्धनम्}


\threelineshloka
{ततो हता नररथवाजिकुञ्जरैरनेकशो द्विपरथपत्तिवाजिनः}
{गजैर्गजा रथिभिरुदायुधा रथा}
{हयैर्हयाः पत्तिगणैश्च पत्तयः}


\twolineshloka
{रथैर्द्विपा द्विरदवरैर्महाहयाहयैर्नरा वररथिभिश्च वाजिनः}
{निरस्तजिह्वादशनेक्षणाः क्षितौक्षयं गताः प्रमथितवर्मभूषणाः}


\twolineshloka
{तथा परैर्बहुकरणैर्वरायुधै-र्हता गताः प्रतिभयदर्शनाः क्षितिम्}
{विपोथिता हयगजपादताडिताभृशाकुला रथमुखनेमिभिः क्षताः}


\twolineshloka
{प्रमोदने श्वापदपक्षिरक्षसांजनक्षये वर्तति तत्र दारुणे}
{महाबलास्ते कुपिताः परस्परंनिषूदयन्तः प्रविचेरुरोजसा}


\twolineshloka
{ततो बले भृशलुलिते परस्परंनिरीक्षमाणे रुधिरौघसम्प्लुते}
{दिवाकरेऽस्तंगिरिमास्थिते शनैरुभे प्रयाते शिबिराय भारत}


\chapter{अध्यायः ३३}
\twolineshloka
{सञ्जय उवाच}
{}


\twolineshloka
{पूर्वमस्मासु भग्नेषु फल्गुगेनामितौजसा}
{द्रोणे च मोघसङ्कल्पे रक्षिते च युधिष्ठिरे}


\twolineshloka
{सर्वे विध्वस्तकवचास्तावका युधि निर्जिताः}
{रजस्वला भृशोद्विग्ना वीक्षमाणा दिशो दश}


\twolineshloka
{अवहारं ततः कृत्वा भारद्वाजस्य सम्मते}
{लब्धलक्षैः शरैर्भिन्ना भृशावहसिता रणे}


\twolineshloka
{श्लाघमानेषु भूतेषु फल्गुनस्यामितान्गुणान्}
{केशवस्य च सौहार्दे कीर्त्यमानेऽर्जुनं प्रति}


\twolineshloka
{अभिशस्ता इवाभूवन्ध्यानमूकत्वमास्थिताः}
{ततः प्रभातसमये द्रोणं द्रुयोधनोऽब्रवीत्}


\twolineshloka
{प्रणयादभिमानाच्च द्विषद्वृद्ध्या च दुर्मनाः}
{शृण्वतां सर्वयोधानां संरब्धो वाक्यकोविदः}


\twolineshloka
{नूनं वयं वध्यपक्षे भवतो द्विजसत्तम}
{तथाहि नाग्रहीः प्राप्तं समीपेऽद्य युधिष्ठिरम्}


\twolineshloka
{इच्छतस्ते न मुच्येत चक्षुःप्राप्तो रणे रिपुः}
{जिघृक्षतो रक्ष्यमाणः सामरैरपि पाण्डवैः}


\threelineshloka
{वरं दत्त्वा मम प्रीतः पश्चाद्विकृतवानसि}
{आशाभङ्गं न कुर्वन्ति भक्तस्यार्याः कथञ्चन ॥सञ्जय उवाच}
{}


\twolineshloka
{ततोऽप्रीतस्तथोक्तः सन्भारद्वाजोऽब्रवीन्नृपम्}
{नार्हसे मां तथा ज्ञातुं घटमानं तव प्रिये}


\twolineshloka
{ससुरासुरगन्धर्वाः सयक्षोरगराक्षसाः}
{नालं लोका रणे जेतुं पाल्यमानं किरीटिना}


\twolineshloka
{विश्वसृग्यत्र गोविन्दः पृतनानीस्तथाऽर्जुनः}
{तत्र कस्य बलं क्रामेदन्यत्र त्र्यम्बकात्प्रभोः}


\twolineshloka
{सत्यं तात ब्रवीम्यद्य नैतज्जात्वन्यथा भवेत्}
{अद्यैकं प्रवरं कञ्चित्पातयिष्ये महारथम्}


\twolineshloka
{चक्रब्यूहं विधास्यामि योऽभेद्यस्त्रिदशैरपि}
{योगेन केनचिद्राजन्नर्जुनस्त्वपनीयताम्}


\twolineshloka
{न ह्यज्ञातमसाध्यं वा तस्य सङ्ख्येऽस्ति किञ्चन}
{तेन ह्युपात्तं सकलं सर्वज्ञानमितस्ततः}


\twolineshloka
{द्रोणेन व्याहृते त्वेवं संशप्तकगणाः पुनः}
{आह्वयन्नर्जुनं सङ्ख्ये दक्षिणामभितो दिशम्}


\threelineshloka
{ततोऽर्जुनस्याथ परैः सार्धं समभवद्रणः}
{तादृशो यादृशो नान्यः श्रुतो दृष्टोपि वा क्वचित्}
{}


\twolineshloka
{तत्र द्रोणेन विहितो व्यूहो राजन्व्यरोचत}
{चरन्मध्यन्दिने सूर्यः प्रतपन्निव दुर्दृशः}


\twolineshloka
{तं चाभिमन्युर्वचनात्पितुर्ज्येष्ठस्य भारत}
{बिभेद दुर्भिदं सङ्ख्ये चक्रव्यूहमनेकधा}


\twolineshloka
{स कृत्वा दुष्करं कर्म हत्वा वीरान्सहस्रशः}
{राजपुत्रशतं हत्वा कौसल्यं च बृहद्बलम्}


\threelineshloka
{महारथं शल्यपुत्रं पौत्रं ते लक्ष्मणं तथा}
{षट््सु वीरेषु संसक्तो दौःशासनिवशं गतः}
{सौभद्रः पृथिवीपाल जहौ प्राणान्परन्तप}


\threelineshloka
{वयं परमसंहृष्टाः पाण्डवाः शोककर्शिताः}
{सौभद्रे निहते राजन्नवहारमकुर्महि ॥`जनमेजय उवाच}
{}


\twolineshloka
{समरोद्युक्तकर्माणः कर्मभिर्व्यञ्जितश्रमाः}
{सकृष्णाः पाण्डवाः पञ्च देवैरपि दुरासदाः}


\twolineshloka
{शश्वत्कर्मान्वयैर्बुद्ध्या प्रकृत्या यशसा श्रिया}
{न भूतो भविता वापि कृष्णस्य सदृशः पुमान्}


\twolineshloka
{सत्यधर्मतपोदानैर्द्विजपूजादिभिर्गुणैः}
{सदेहस्त्रिदिवं प्राप्तो राजा किल युधिष्ठिरः}


\twolineshloka
{युगान्ते चान्तकश्चैव जामदग्न्यश्च कीर्तिमान्}
{रणस्थो भीमसेनश्च कथ्यन्ते सदृशास्त्रयः}


\twolineshloka
{प्रतिज्ञाकर्मदक्षस्य रणे गाण्डीवधन्वनः}
{उपमां नाधिगच्छामि पार्थस्य सदृशीं क्षितौ}


\twolineshloka
{गुरुवात्सल्यमत्यन्तं नैभृत्यं विनयो दमः}
{नकुले त्वाभिरूप्यं च शौर्यं च नियतानि षट्}


\twolineshloka
{श्रुतमाधुर्यगाम्भीर्यसत्त्वरूपपराक्रमैः}
{सदृशो देवयोर्वीरः सहदेवः किलाश्विनोः}


\twolineshloka
{ये च कृष्णे गुणाः स्फीताः पाण्डवेषु च ये गुणाः}
{अभिमन्यौ किलैकस्था दृश्यन्ते ते गुणाः शुभाः}


\twolineshloka
{युधिष्ठिरस्य शौर्येण कृष्णस्य चरितेन च}
{कर्मणा भीमसेनस्य सदृशं भीमकर्मणः}


\twolineshloka
{धनञ्जयस्य रूपेण विक्रमेण शुभेन च}
{विनये सहदेवस्य सदृशं नकुलस्य च}


\threelineshloka
{अभिमन्युं द्विजश्रेष्ठ सौभद्रमपराजितम्}
{श्रोतुमिच्छामि कार्त्स्न्येन कथमायोधने हतः ॥वैशंपायन उवाच}
{}


\threelineshloka
{अभिमन्युं हतं श्रुत्वा धृतराष्ट्रो जनाधिपः}
{विस्तरेण महाराज पर्यपृच्छत्स सञ्जयम्' ॥धृतराष्ट्र उवाच}
{}


\twolineshloka
{पुत्रं पुरुषसिंहस्य सञ्जयाप्राप्तयौवनम्}
{रणे विनिहतं श्रुत्वा भृशं मे दीर्यते मनः}


\twolineshloka
{दारुणः क्षत्रधर्मोऽयं विहितो धर्मकर्तृभिः}
{यत्र राज्येप्सवः शूरा बाले शस्त्रमपातयन्}


\twolineshloka
{बालमत्यन्तसुखिनं विचरन्तमभीतवत्}
{कृतास्त्रा बहवो जघ्नुर्ब्रूहि गावल्गणे कथम्}


\threelineshloka
{बिबित्सता रथानीकं सौभद्रेणामितौजसा}
{विक्रीडितं यथा सङ्ख्ये तन्ममाचक्ष्व सञ्जय ॥सञ्जय उवाच}
{}


\twolineshloka
{यन्मां पृच्छसि राजेन्द्र सौभद्रस्य निपातनम्}
{तत्ते कार्त्स्न्येन वक्ष्यामि शृणु राजन्समाहितः}


\twolineshloka
{विक्रीडितं कुमारेण यथानीकं बिभित्सता}
{आरुग्णाश्च यथा वीरा दुःसाध्याश्चापि विप्लुवे}


\twolineshloka
{दावाग्न्यभिपरीतानां भूरिगुल्मतृणद्रुमे}
{वनौकसामिवारण्ये त्वदीयानामभूद्भयम्}


\chapter{अध्यायः ३४}
\twolineshloka
{सञ्जय उवाच}
{}


\twolineshloka
{स्थिरो भव महाराज शोकं धारय दुर्धरम्}
{महान्तं बन्धुनाशं ते कथयिष्यामि तच्छृशु}


\twolineshloka
{पद्मव्यूहो महाराज आचार्येणाभिकल्पितः}
{तत्र पद्मोपमाः सर्वे राजानो विनिवेशिताः}


\twolineshloka
{कुमारा राजलोकस्य निक्षिप्ताः केसरोपामाः}
{कर्णिकास्थो महाराज तस्य दुर्योधनोऽभवत्}


\twolineshloka
{कृताभिसमयाः सर्वे सुवर्णविकृतध्वजाः}
{रक्ताम्बरधराः सर्वे सर्वे रक्तविभूषणाः}


\twolineshloka
{सर्वे रक्तपताकाश्च सर्वे वै हेममालिनः}
{चन्दनागुरुदिग्धाङ्गाः स्रग्विणः सूक्ष्मवाससः}


\twolineshloka
{सहिताः पर्यधावन्त कार्ष्णि प्रतियुयुत्सवः}
{तेषां दशसहस्राणि बभूवुर्दृढधन्विनाम्}


\twolineshloka
{पौत्रं तव पुरस्कृत्य लक्ष्मणं प्रियदर्शनम्}
{अन्योन्यसमदुःखास्ते अन्योन्यसमसाहसाः}


\twolineshloka
{अन्योन्यं स्पर्धमानाश्च अन्योन्यस्य हिते रताः}
{दुर्योधनस्तु राजेन्द्र सैन्यमध्ये व्यवस्थितः}


\twolineshloka
{कर्णदुःशासनकृपैर्वृतो राजा महारथैः}
{देवराजोपमः श्रीमाञ्श्वेतच्छत्राभिसंवृतः}


\twolineshloka
{चामरव्यजनाक्षेपैरुदयन्निव भास्करः}
{प्रमुखे तस्य सैन्यस्य द्रोणोऽवस्थित नायकः}


\twolineshloka
{सिन्धुराजस्तथाऽतिष्ठच्छ्रीमान्मेरुरिवाचलः}
{सिन्धुराजस्य पार्श्वस्था अश्वत्थामपुरोगमाः}


\twolineshloka
{सुतास्तव महाराज त्रिंशत्त्रिदशसन्निभाः}
{गान्धारराजः कितवः शल्यो भूरिश्रवास्तथा}


\threelineshloka
{पार्श्वतः सिन्धुराजस्य व्यराजन्त महारथाः}
{ततः प्रववृते युद्धं तुमुलं लोमहर्षणम्}
{तावकानां परेषां च मृत्युं कृत्वा निवर्तनम्}


\chapter{अध्यायः ३५}
\twolineshloka
{सञ्जय उवाच}
{}


\twolineshloka
{तदनीकमनाधृष्टं भारद्वाजेन रक्षितम्}
{पार्थाः समभ्यवर्तन्त भीमसेनपुरोगमाः}


\twolineshloka
{सात्यकिश्चेकितानश्च धृष्टद्युम्नश्च पार्षतः}
{कुन्तिभोजश्च विक्रान्तो द्रुपदश्च महारथः}


\twolineshloka
{आर्जुनिः क्षत्रधर्मा च बृहत्क्षत्रश्च वीर्यवान्}
{चेदिपो धृष्टकेतुश्च माद्रीपुत्रौ घटोत्कचः}


\twolineshloka
{युधामन्युश्च विक्रान्तः शिखण्डी चापराजितः}
{उतक्तमौजाश्च दुर्धर्षो विराटश्च महारथः}


\twolineshloka
{द्रौपदेयाश्च संरब्धाः शैशुपालिश्च वीर्यवान्}
{केकयाश्च महावीर्याः सृञ्जयाश्च सहस्रशः}


\twolineshloka
{एते चान्ये च सगणाः कृतास्त्रा युद्धदुर्मदाः}
{समभ्यधावन्सहसा भारद्वाजं युयुत्सवः}


\twolineshloka
{समीपे वर्तमानांस्तान्भारद्वाजोऽतिवीर्यवान्}
{असम्भ्रान्तः शरौघेण महता समवारयत्}


\twolineshloka
{महौघः सलिलस्येव गिरिमासाद्य दुर्भिदम्}
{द्रोणं ते नाभ्यवर्तन्त वेलामिव जलाशयाः}


\twolineshloka
{पीड्यमानाः शरै राजन्द्रोणचापविनिःसृतैः}
{न शेकुः प्रमुखे स्थातुं भारद्वाजस्य पाण्डवाः}


\twolineshloka
{तदद्भुतमपश्याम द्रोणस्य भुजयोर्बलम्}
{यदेनं नाभ्यवर्तन्त पाञ्चालाः सृञ्जयैः सह}


\twolineshloka
{तमायान्तमभिक्रुद्धं द्रोणं दृष्ट्वा युधिष्ठिरः}
{बहुधा चिन्तयामास द्रोणस्य प्रतिवारणम्}


\twolineshloka
{अशक्यं तु तमन्येन द्रोणं मत्त्वा युधिष्ठिरः}
{अविषह्यं गुर भारं सौभद्रे समवासृजत्}


\threelineshloka
{वासुदेवादनवरं फल्गुनाच्चामितौजसम्}
{अब्रवीत्परवीरघ्नमभिमन्युमिदं वचः ॥युधिष्ठिर उवाच}
{}


\twolineshloka
{एत्य नो नार्जुनो गर्हेद्यथा तात तथा कुरु}
{द्रोणानीकस्य न वयं विद्मो भेदं कथञ्चन}


\twolineshloka
{त्वं वाऽर्जुनो वा कृष्णो वा भिन्द्यात्प्रद्युम्न एव वा}
{द्रोणानीकं महाबाहो पञ्चमो नोपपद्यते}


\twolineshloka
{अभिमन्यो वरं तात याचे त्वां दातुमर्हसि}
{पितॄणां मातुलानां च सैन्यानां चैव सर्वशः}


\threelineshloka
{धनञ्जयो हि नस्तात गर्हयेदेत्य संयुगात्}
{क्षिप्रमस्त्रं समादाय द्रोणानीकं विशातय ॥अभिमन्युरुवाच}
{}


\twolineshloka
{द्रोणस्य दृढमत्युग्रमनीकप्रवरं युधि}
{पितॄणां जयमाकाङ्क्षन्नवगाहेऽविलम्बितम्}


\threelineshloka
{उपदिष्टो हि मे पित्रा योगोऽनीकविशातने}
{नोत्सहे हि विनिर्गन्तुमहं कस्याञ्चिदापदि ॥युधिष्ठिर उवाच}
{}


\twolineshloka
{भिन्ध्यनीकं युधांश्चेष्ठ द्वारं सञ्जनयस्व नः}
{वयं त्वाऽनुगमिष्यामो येन त्वं तात यास्यसि}


\threelineshloka
{धनञ्जयसमं युद्धे त्वां वयं तात संयुते}
{प्रणिधायानुयास्यामो रक्षन्तः सर्वतोमुखाः ॥भीम उवाच}
{}


\twolineshloka
{अहं त्वाऽनुगमिष्यामि धृष्टद्युम्नोऽथ सात्यकिः}
{पाञ्चलाः केकया मात्स्यास्तथा सर्वे प्रभद्रकाः}


\threelineshloka
{सकृद्भिन्नं त्वया व्यूहं तत्र तत्र पुनः पुनः}
{वयं प्रध्वंसयिष्यामो निघ्नामाना वरान्वरान् ॥अभिमन्युरुवाच}
{}


\twolineshloka
{अहमेतत्प्रवेक्ष्यामि द्रोणानीकं दुरासदम्}
{पतङ्ग इव सङ्क्रुद्धो ज्वलितं जातवेदसम्}


\twolineshloka
{तत्कर्माद्य करिष्यामि हितं यद्वंशयोर्द्वयोः}
{मातुलस्य च यत्प्रीतिं करिष्यति पितुश्च मे}


\twolineshloka
{शिशुनैकेन सङ्ग्रमे काल्यमानानि सङ्घशः}
{द्रक्ष्यन्ति सर्वभूतानि द्विषत्सैन्यानि वै मया}


\twolineshloka
{नाहं पार्थेन जातः स्यां न च जातः सुभद्रया}
{यदि मे संयुगे कश्चिज्जीवितो नाद्य मुच्यते}


\threelineshloka
{यदि चैकरथेनाहं समग्रं क्षत्रमण्डलम्}
{न करोम्यष्टधा युद्धे न भवाम्यर्जुनात्मजः ॥युधिष्ठिर उवाच}
{}


\twolineshloka
{एवं ते भाषमाणस्य बलं सौभद्र वर्धताम्}
{यत्समुत्सहसे भेत्तुं द्रोणानीकं दुरासदम्}


\threelineshloka
{रक्षितं पुरुषव्याघ्रैर्महेष्वासैर्महाबलै}
{साध्यरुद्रमरुत्तुल्यैर्वस्वग्न्यादित्यविक्रमैः ॥सञ्जय उवाच}
{}


% Check verse!
तस्य तद्वचनं श्रुत्वा स यन्तारमचोदयत्
% Check verse!
सुमित्राश्वान्रणे क्षिप्रं द्रोणानीकाय चोदय
\chapter{अध्यायः ३६}
\twolineshloka
{सञ्जय उवाच}
{}


\twolineshloka
{सौभद्रस्तद्वचः श्रुत्वा धर्मराजस्य धीमतः}
{अचोदयत यन्तारं द्रोणानीकाय भारत}


\twolineshloka
{तेन सञ्चोद्यमानस्तु याहियाहीति सारथिः}
{प्रत्युवाच ततो राजन्नभिमन्युमिदं वचः}


\twolineshloka
{अतिभारोऽयमायुष्मन्नाहितस्त्वयि पाण्डवैः}
{सम्प्रधार्य क्षणं बुद्ध्या ततस्त्वं योद्धुमर्हसि}


\twolineshloka
{आचार्यो हि कृती द्रोणः परमास्त्रे कृतश्रमः}
{त्वं तु बालः स बलवान्सङ्ग्रामाणामकोविदः}


\twolineshloka
{ततोऽभिमन्युः प्रहसन्सारथिं वाक्यमब्रवीत्}
{सारथे कोन्वयं द्रोणः समग्रं क्षत्रमेव वा}


\twolineshloka
{`व्यवसायो हि मे योद्धुं रणोत्सवसमुद्यतः'}
{ऐरावतगतं शक्रं सहामरगणैरहम्}


\twolineshloka
{अथवा रुद्रमीशानं सर्वभूतगणार्चितम्}
{योधयेयं रणमुखे न मे क्षत्रेऽद्य विस्मयः}


\twolineshloka
{`यच्चैतत्पश्यसे सूत सयोधाश्वरथद्विपम्'}
{तन्ममाद्य द्विषत्सैन्यं कलामर्हति षोडशीम्}


\threelineshloka
{अपि विश्वजितं विष्णुं मातुलं प्राप्य सूतज}
{पितरं चार्जुनं युद्धे न भीर्मामुपयास्यति ॥सञ्जय उवाच}
{}


\twolineshloka
{एवमप्युच्यमानः स सारथिस्तं पुनः पुनः}
{वीर ते तेन मा युद्धमिति सौभद्रमब्रवीत्'}


\twolineshloka
{अभिमन्युश्च तां वाचं कदर्थीकृत्य सारथेः}
{याहीत्येवाब्रवीदेनं द्रोणानीकाय माचिरम्}


\twolineshloka
{ततः सन्नोदयामास हयानाशु त्रिहायनान्}
{नातिहृष्टमनाः सूतो हेमभाण्डपरिच्छदान्}


\twolineshloka
{ते प्रेषिताः सुमित्रेण द्रोणानीकाय वाजिनः}
{द्रोणमभ्यद्रवन्राजन्महावेगा महाबलाः}


\twolineshloka
{तमुदीक्ष्य तथायान्तं सर्वे द्रोणपुरोगमाः}
{अभ्यवर्तन्त कौरव्याः पाण्डवाश्च तमन्वयुः}


\twolineshloka
{सकर्णिकारप्रवरोच्छ्रितध्वजःसुवर्णवर्माऽऽर्जुनिरर्जुनाद्वरः}
{युयुत्सया द्रोणमुखान्महारथान्समासदत्सिंहशिशुर्यथा द्विपान्}


\twolineshloka
{ते विंशतिपदे यत्ताः सम्प्रहारं प्रचक्रिरे}
{आसीद्गाङ्ग इवावर्तो मुहूर्तमुदधाविव}


\twolineshloka
{शूराणां युध्यमानानां निघ्नतामितरेतरम्}
{सङ्ग्रामस्तुमुलो राजन्प्रावर्तत सुदारुणः}


\twolineshloka
{प्रवर्तमाने सङ्ग्रामे तस्मिन्नतिभयङ्करे}
{द्रोणस्य मिषतो व्यूहं भित्त्वा व्यचरदार्जुनिः}


\twolineshloka
{`तदभेद्यमनाधृष्यं द्रोणानीकं सुदुर्जयम्}
{भित्त्वार्जुनिरसम्भ्रान्तो विवेशाचिन्त्यविक्रमः'}


\twolineshloka
{तं प्रविष्टं विनिघ्नन्तं शत्रुसङ्घान्महाबलम्}
{हस्त्यश्वरथपत्त्यौघाः परिवव्रुरुदायुधाः}


\twolineshloka
{नानावादित्रनिनदैः क्ष्वेडितोत्क्रुष्टगर्जितैः}
{हुङ्कारैः सिंहनादैश्च तिष्ठ तिष्ठेति निःस्वनैः}


\twolineshloka
{घोरैर्हलहलाशब्दैर्मागास्तिष्ठैहि मामिति}
{असावहममित्रेति प्रवदन्तो मुहुर्मुहुः}


\twolineshloka
{बृंहितैः शिञ्जितैर्हासैः करनेमिस्वनैरपि}
{सन्नादयन्तो वसुधामभिदुद्रुवुरार्जुनिम्}


\twolineshloka
{तेषामापततां वीरः शीघ्रयोधी महाबलः}
{क्षिप्रास्त्रो न्यवधीद्राजन्मर्मज्ञो मर्मभेदिभिः}


\twolineshloka
{ते हन्यमाना विवशा नानालिङ्गैः शितैः शरैः}
{अभिपेतुः सुबहुशः शलभा इव पावकम्}


\twolineshloka
{ततस्तेषां शरीरैश्च शरीरावयवैश्च सः}
{सन्तंस्तार क्षितिं क्षिप्रं कुशैर्वेदिमिवाध्वरे}


\twolineshloka
{बद्धगोधाङ्गुलित्राणान्सशरासनसायकान्}
{सासिचर्माङ्कुशाभीषून्सतोमरपरश्वथान्}


\twolineshloka
{सगदायोगुडप्रासान्सर्ष्टितोमरपट्टसान्}
{सभिण्डिपालपरिघान्यशक्तिवरकम्पनान्}


\twolineshloka
{सप्रतोदमहाशङ्खान्सकुन्तान्सकचग्रहान्}
{समुद्गरक्षेपणीयान्सपाशपरिघोपलान्}


\twolineshloka
{सकेयूराङ्गदान्बाहून्हृद्यगन्धानुलेपनान्}
{स़ञ्चिच्छेदार्जुनिर्वृत्तांस्त्वदीयानां सहस्रशः}


\twolineshloka
{तैः स्फुरद्भिर्महाराज शुशुभे भूः सुलोहितैः}
{पञ्चास्यैः पन्नगैश्छिन्नैर्गरुडेनेव मारिष}


\twolineshloka
{सुनासाननकेशान्तैरव्रणैश्चारुकुण्डलैः}
{सन्दष्टौष्ठपुटैः क्रोधात्क्षरद्भिः शोणितं बहु}


\twolineshloka
{सचारुमुकुटोष्णीषैर्मणिरत्नविभूषितैः}
{विनालनलिनाकारैर्दिवाकरशशिप्रभैः}


\twolineshloka
{हितप्रियंवदैः काले बहुभिः पुण्यगन्धिभिः}
{द्विषच्छिरोभिः पृथिवीं स वै तस्तार फाल्गुनिः}


\twolineshloka
{गन्धर्वनगराकारान्विधिवत्कल्पितान्रथान्}
{वीषामुखान्वित्रिवेणून्न्यस्तदण्डकबन्धुरान्}


\twolineshloka
{विजङ्घाकूबरांस्तत्र विनेमींश्र व्यरानपि}
{विचक्रोपस्करोपस्थान्भग्नोपकरणानपि}


\twolineshloka
{`समास्थितान्योधवरैर्दान्ताश्वान्साधुसारथीन्}
{विपताकाध्वजच्छत्रान्वितूणीरायुधानपि}


\twolineshloka
{प्रपातितोपस्तरणान्हतयोधान्सहस्रशः}
{शरैर्विशकलीकुर्वन्दिक्षु सर्वास्वदृश्यत}


\twolineshloka
{पुनर्द्विपान्द्विपारोहान्वैजयन्त्यङ्कुशध्वजान्}
{तूणान्वर्माण्यथो कक्ष्या ग्रैवेयांश्च सकम्बलान्}


\twolineshloka
{घण्टाः शुण्डाविषाणाग्राञ्छत्रमालाः पदानुगान्}
{शरैर्निशितधाराग्रैः शात्रवाणामशातयत्}


\twolineshloka
{वनायुजान्पार्वतीयान्काम्भोजानथ बाह्लिकान्}
{स्थिरवालधिकर्णाक्षाञ्जवनान्साधुवाहिनः}


\twolineshloka
{आरूढाञ्शिक्षितैर्योधैः शक्त्यृष्टिप्रासयोधिभिः}
{विध्वस्तचामरमुखान्विप्रविद्धप्रकीर्णकान्}


\twolineshloka
{निरस्तजिह्वानयनान्निष्कीर्णान्त्रयकृद्धनान्}
{हतारोहांश्छिन्नघण्टान्क्रव्यादगणमोदकान्}


\threelineshloka
{निकृत्तचर्मकवचाञ्शकृन्मूत्रासृगाप्लुतान्}
{निपातयन्नश्ववरांस्तावकान्स व्यरोचत}
{एको विष्णुरिवाचिन्त्यं कृत्वा कर्म सुदुष्करम्}


\twolineshloka
{तथा निर्माथितं तेन चतुरङ्गबलं महत्}
{यथाऽसुरबलं घोरं त्र्यम्बकेन महौजसा}


\twolineshloka
{कृत्वा कर्म रणेऽसह्यं परैरार्जुनिराहवे}
{अभिनच्च पदात्योघांस्त्वदीयानेव सर्वशः}


\twolineshloka
{एवमेकेन तां सेनां सौभद्रेण सितैः शरैः}
{भृशं विप्रहतां दृष्ट्वा स्कन्देनेवासुरीं चमूम्}


\twolineshloka
{त्वदीयास्तव पुत्राश्च वीक्षमाणा दिशो दश}
{संशुष्कास्याश्चलन्नेत्राः प्रस्विन्ना रोमहर्षिणः}


\twolineshloka
{पलायनकृतोत्साहा निरुत्साहा द्विषज्जये}
{गोत्रनामभिरन्योन्यं क्रन्दन्तो जीवितैषिणः}


\twolineshloka
{हतान्पुत्रान्पितॄन्भ्रातॄन्बन्धून्सम्बन्धिनस्तथा}
{प्रातिष्ठन्त समुत्सृज्य त्वरयन्तो हयद्विपान्}


\chapter{अध्यायः ३७}
\twolineshloka
{सञ्जय उवाच}
{}


\twolineshloka
{तां प्रभाग्नां चमूं दृष्ट्वा सौभद्रेणामितौजसा}
{दुर्योधनो भृशं क्रुद्धः स्वयं सौभद्रमभ्ययात्}


\twolineshloka
{ततो राजानमावृत्तं सौभद्रं प्रति संयुगे}
{दृष्ट्वा द्रोणोऽब्रवीद्योधान्परीप्सध्वं नराधिपम्}


\fourlineindentedshloka
{पुराऽभिमन्युर्लक्षं नः पश्यतां हन्ति वीर्यवान्}
{तमाद्रवत माभैष्ट क्षिप्रं रक्षत कौरवम्}
{सञ्जय उवाच}
{}


\threelineshloka
{`एवमुक्तास्ततस्तेन भारद्वाजेन धीमता'}
{कृतज्ञानबलोत्सेधाः सुहृदो जितकाशिनः}
{त्रास्यमाना भयाद्वीरं परिवव्रुस्तवात्मजम्}


\twolineshloka
{द्रोणो द्रौणिः कृपः कर्णः कृतवर्मा च सौबलः}
{बृहद्बलो मद्रराजो भूरिर्भूरिश्रवाः शलः}


\twolineshloka
{पौरवो वृषसेनश्च विसृजन्तः शिताञ्शरान्}
{सौभद्रं शरवर्षेण महता समवाकिरन्}


\twolineshloka
{सम्मोहयित्वा तमथ दुर्योधनममोचयन्}
{आस्याद्ग्रासमिवाक्षिप्तं ममृषे नार्जुनात्मजः}


\twolineshloka
{ताञ्शरौघेण महता साश्वसूतान्महारथान्}
{विमुखीकृत्य सौभद्रः सिंहनादमथानदत्}


\twolineshloka
{तस्य नादं ततः श्रुत्वा सिंहस्येवामिषैषिणः}
{नामृष्यन्त सुसंरब्धाः पुनर्द्रोणमुखा रथाः}


\twolineshloka
{त एनं कोष्ठकीकृत्य रथवंशेन मारिष}
{व्यसृजन्निषुजालानि नानालिङ्गानि सङ्घशः}


\twolineshloka
{तान्यन्तरिक्षे चिच्छेद पौत्रस्ते निशितैः शरैः}
{तांश्चैव प्रतिविव्याध तदद्भुतमिवाभवत्}


\threelineshloka
{`ततो रथपदात्योघाः कुञ्जराः सादिनश्च ह'}
{कोपितास्तेन शूरेण शरैराशीविषोपमैः}
{परिवब्रुर्जिघांसन्तः सौभद्रमपराजितम्}


\twolineshloka
{समुद्रमिव पर्यस्तं त्वदीयं तं बलार्णवम्}
{दधारैकोऽर्जुनिर्बाणैर्वेलेवोद्वृत्तमर्णवम्}


\twolineshloka
{शूराणां युध्यमानानां निघ्नतामितरेतरम्}
{अभिमन्योः परेषां च नासीत्कश्चित्पराङ्मुखः}


\twolineshloka
{तस्मिंस्तु घोरे सङ्ग्रामे वर्तमाने भयङ्करे}
{दुःसहो नवभिर्बाणैरभिमन्युमविध्यत}


\twolineshloka
{दुःशानो द्वादशभिः कृपः शारद्वतस्त्रिभिः}
{द्रोणस्तु सप्तदशभिः शरैराशीविषोपमैः}


\twolineshloka
{विविंशतिस्तु सप्तत्या कृतवर्मा च सप्तभिः}
{बृहद्बलस्तथाऽष्टाभिरश्वत्थामा च सप्तभिः}


\twolineshloka
{भूरिश्रवास्त्रिभिर्बाणैर्मद्रेशः षङ्भिराशुगैः}
{द्वाभ्यां शराभ्यां शकुनिस्त्रिभिर्दुर्योधनो नृपः}


\twolineshloka
{स तु तान्प्रतिविव्याध त्रिभिस्त्रिभिरजिह्मगैः}
{नृत्यन्निव महाराज चापहस्तः प्रतापवान्}


\twolineshloka
{ततोऽभिमन्युः सङ्क्रुद्धस्त्रास्यमानस्तवात्मजैः}
{विदर्शयन्वै सुमहच्छिक्षौरसकृतं बलम्}


\twolineshloka
{गरुडानिलरंहोभिर्यन्तुर्वाक्यकरैर्हयैः}
{अभ्यद्रवत तं कार्ष्णिमश्मकेन्द्रः कृतत्वरः}


\twolineshloka
{अश्वाश्चाश्मकमादाय त्वरमाणाऽभ्यहारयन्}
{विव्याध दशभिर्बाणैस्तिष्ठतिष्ठेति चाब्रवीत्}


\twolineshloka
{तस्याभिमन्युर्दशभिर्हयान्सूतं ध्वजं शरैः}
{बाहू धनुः शिरश्चोर्व्यां स्मयमानोऽभ्यपातयत्}


\twolineshloka
{ततस्तस्मिन्हते वीरे सौभद्रेणाश्मकेश्वरे}
{सञ्चचाल बलं सर्वं पलायनपरायणम्}


\threelineshloka
{ततः कर्णः कृपो द्रोणो द्रौणिर्गान्धारराट् शलः}
{शल्यो भूरिश्रवाः क्राथः सोमदत्तो विविंशतिः}
{वृषसेनः सुषेणश्च कुण्डभेदी प्रतर्दनः}


\twolineshloka
{बृन्दारको ललित्थश्च प्रबाहुर्दीर्घलोचनः}
{दुर्योधनश्च सङ्क्रुद्धः शरवर्षैरवाकिरन्}


\twolineshloka
{सोऽतिविद्धो महेष्वासैरभिमन्युरजिह्मगैः}
{शरमादत्त कर्णाय वर्मकायावभेदिनम्}


\twolineshloka
{तस्य भित्त्वा तनुत्राणं देहं निर्भिद्य चाशुगः}
{प्राविशद्धरणीं वेगाद्वल्मी कमिव पन्नगः}


\twolineshloka
{स तेनातिप्रहारेण व्यथितो विह्वलन्निव}
{सञ्चचाल रणे कर्णः क्षितिकम्पे यथाऽचलः}


\twolineshloka
{तथान्यैर्निशितैर्बाणैः सुषेणं दीर्घलोचनम्}
{कुण्डभेदिं च सङ्क्रुद्धस्त्रिभिस्त्रीनवधीद्बली}


\twolineshloka
{कर्णस्तं पञ्चविंशत्या नाराचानां समार्पयत्}
{अश्वत्थामा च विंशत्या कृतवर्मा च सप्तभिः}


\twolineshloka
{स शराचितसर्वाङ्गः क्रुद्धः शक्रात्मजात्मजः}
{विचरन्ददृशे सैन्ये पाशहस्त इवान्तकः}


\twolineshloka
{शल्यं च शरवर्षेण समीपस्थामवाकिरत्}
{उदक्रोशन्महाबाहुस्तव सैन्यानि भीषयन्}


\twolineshloka
{ततः स विद्धोऽस्त्रविदा मर्मभिद्भिरजिह्मगैः}
{शल्यो राजन्रथोपस्थे निषसाद मुमोह च}


\twolineshloka
{तं हि दृष्ट्वा तथा विद्धं सौभद्रेण यशास्विना}
{सम्प्राद्रवच्चमूः सर्वा भारद्वाजस्य पश्यतः}


\twolineshloka
{सम्प्रेक्ष्य तं महाबाहुं रुक्मपुङ्खैः समावृतम्}
{त्वदीयाः प्रपलायन्ते मृगाः सिंहार्दिता इव}


\twolineshloka
{`सौभद्रशरनिर्भग्नाः समरेऽमरविक्रमाः}
{एवं शल्यो विमृदितस्तव पौत्रेण भारत'}


% Check verse!
स तु रणयशसाऽभिपूज्यमानःपितृसुरचारणसिद्धयक्षसङ्गैःअवनितलगतैश्च भूतसङ्घै--रतिविबभौ हुतभुग्यथाऽऽज्यसिक्तः
\chapter{अध्यायः ३८}
\twolineshloka
{धृतराष्ट्र उवाच}
{}


\threelineshloka
{तथा प्रमथमानं तं महेष्वासानजिह्मगैः}
{आर्जुनिं मामकाः सङ्ख्ये के त्वेनं समवारयन् ॥सञ्जय उवाच}
{}


\twolineshloka
{शृणु राजन्कुमारस्य रणे विक्रीडितं महत्}
{बिभित्सतो रथानीकं भारद्वाजेन रक्षितम्}


\twolineshloka
{मद्रेशं सादितं दृष्ट्वा सौभद्रेणाशुगै रणे}
{शल्यादवरजः क्रुद्ध क्रिन्बाणान्समभ्ययात्}


\twolineshloka
{स विद्ध्वा दशभिर्बाणैः साश्वयन्तारमार्जुनिम्}
{उदक्रोशन्महाशब्दं तिष्ठतिष्ठेति चाब्रवीत्}


\twolineshloka
{तस्यार्जुनिः शिरोग्रीवं पाणिपादं धनुर्हयान्}
{छत्रं ध्वजं नियन्तारं त्रिवेणुं चाप्युपस्करम्}


\twolineshloka
{चक्रं युगं च तूणीरं ह्यनुकर्षं सायकैः}
{पताकां चक्रगोप्तारौ सर्वोपकरणानि च}


\twolineshloka
{लघुहस्तः प्रचिच्छेद ददृशे तं न कश्चन}
{स पपात क्षितौ क्षीणः प्रविद्धाभरणाम्बरः}


\twolineshloka
{वायुनेव महाचैत्यः समूलं चित्रवेदिकः}
{अनुगास्तस्य वित्रस्ताः प्राद्रवन्सर्वतो दिशः}


\twolineshloka
{आर्जुनेः कर्म तद्दृष्ट्वा सम्प्रणेदुः समन्ततः}
{नादेन सर्वभूतानि साधुसाध्विति भारत}


\twolineshloka
{शल्यभ्रातर्यथारुग्णे बहुशस्तस्य सैनिकाः}
{कुलाधिवासनामानि श्रावयन्तोऽर्जुनात्मजम्}


\twolineshloka
{अभ्यधावन्त सङ्क्रुद्धा विविधायुधपाणयः}
{रथैरश्वैर्गजैश्चान्ये पद्भिश्चान्ये बलोत्कटाः}


\twolineshloka
{बाणशब्देन महता रथनेमिस्वनेन च}
{हुंकारैः क्ष्वेडितोत्कृष्टैः सिंहनादैः सगर्जितैः}


\twolineshloka
{ज्यातलत्रस्वनैरन्ये गर्जन्तोऽर्जुननन्दनम्}
{ब्रुवन्तश्च न नो जीवन्मोक्ष्यसे जीवितादिति}


\twolineshloka
{तांस्तथा ब्रुवतो दृष्ट्वा सौभद्रः प्रहसन्निव}
{यो योस्मै प्राहरत्पूर्वं तं तं विव्याध पत्रिभिः}


\twolineshloka
{सन्दर्शयिष्यन्नस्त्राणि विचित्राणि लघूनि च}
{आर्जुनिः समरे शूरो मृदुपूर्वमयुध्यत}


\twolineshloka
{वासुदेवादुपात्तं यदस्त्रं यच्च धनञ्जयात्}
{अदर्शयत तत्कार्ष्णिः कृष्णाभ्यामविशेषवत्}


\twolineshloka
{दूरमस्य गुरुं भारं साध्वसं च पुनः पुनः}
{सन्दधद्विसृजंश्चेषून्निर्विशेषमदृश्यत}


\twolineshloka
{चापमण्डलमेवास्य विस्फुरद्दिक्ष्वदृश्यत}
{सुदीप्तस्य शरत्काले सवितुर्मण्डलं यथा}


\twolineshloka
{ज्याशब्दः शुश्रुवे तस्य तलशब्दश्च दारुणः}
{महाशनिमुचः काले पयोदस्येव निःस्वनः}


\twolineshloka
{हीमानमर्षी सौभद्रो मानकृत्प्रियदर्शनः}
{सम्मिमानयिषुर्वीरानिष्वस्त्रैश्चाप्ययुध्यत}


\twolineshloka
{मृदुर्भूत्वा महाराज दारुणः समपद्यत}
{वर्षाभ्यतीतो भगवाञ्शरदीव दिवाकरः}


\twolineshloka
{शरान्विचित्रान्सुबहून्रुक्मपुङ्खाञ्शिलाशितान्}
{मुमोच शतशः क्रुद्धो गभस्तीनिव भास्करः}


\twolineshloka
{क्षुरप्रैर्वत्सदन्तैश्च विपाठेश्च महायशाः}
{नाराचैरर्धचन्द्राभैर्भल्लैरञ्जलिकैरपि}


\twolineshloka
{अवाकिरद्रथानीकं भारद्वाजस्य पश्यतः}
{ततस्तत्सैन्यमभवद्विमुखं शरपीडितम्}


\chapter{अध्यायः ३९}
\twolineshloka
{धृतराष्ट्र उवाच}
{}


\twolineshloka
{द्वैधीभवति मे चित्तं शुचा तुष्ट्या च सञ्जय}
{मम पुत्रस्य यत्नैन्यं सौभद्रः समवारयत्}


\threelineshloka
{विस्तरेणैव मे शंस सर्वं गावल्गणे पुनः}
{विक्रीडितं कुमारस्य स्कन्दस्येवासुरैः सह ॥सञ्जय उवाच}
{}


\twolineshloka
{हन्त ते सम्प्रवक्ष्यामि विमर्दमतिदारुणम्}
{एकस्य च बहूनां च यथासीत्तुमुलो रणः}


\twolineshloka
{अभिमन्युः कृतोत्साहः कृतोत्साहानरिन्दमान्}
{रथस्थो रथिनः सर्वांस्तावकानभ्यवर्षयत्}


% Check verse!
द्रोणं कर्णं कृपं शल्यं द्रौणिं भोजं बृहद्बलम् ॥दुर्योधनं सौमदत्तिं शकुनिं च महाबलम्
\twolineshloka
{नानानृपान्नृपसुतान्सैन्यानि विविधानि च}
{अलातचक्रवत्सर्वांश्चरन्बाणैः समार्पयत्}


\twolineshloka
{निघ्नन्नमित्रान्सौभद्रः परमास्त्रैः प्रतापवान्}
{अदर्शयत तेजस्वी दिक्षु सर्वासु भारत}


\twolineshloka
{तद्दृष्ट्वा चरितं तस्य सौभद्रस्यामितौजसः}
{समकम्पन्त सैन्यानि त्वदीयानि सहस्रशः}


\twolineshloka
{अथाब्रवीन्महाप्राज्ञो भारद्वाजः प्रतापवान्}
{हर्षेणोत्फुल्लनयनः कृपमाभाष्य सत्वरम्}


\threelineshloka
{घटयन्निव मर्माणि पुत्रस्य तव भारत}
{अभिमन्युं रणे दृष्ट्वा तदा रणविशारदम् ॥द्रोण उवाच}
{}


\twolineshloka
{एष गच्छति सौभद्रः पार्थानां प्रथितो युवा}
{नन्दयन्सुहृदः सर्वान्राजानं च युधिष्ठिरम्}


\twolineshloka
{नकुलं सहदेवं च भीमसेनं च पाण्डवम्}
{बन्धून्सम्बन्धिनश्चान्यान्मध्यस्थान्सुहृदस्तथा}


\threelineshloka
{नास्य युद्धे समं मन्ये कञ्चिदन्यं धनुर्धरम्}
{इच्छन्हन्यादिमां सेनां किमर्थमपि नेच्छति ॥सञ्जय उवाच}
{}


\twolineshloka
{द्रोणस्य प्रीतिसंयुक्तं श्रुत्वा वाक्यं तवात्मजः}
{आर्जुनिं प्रति सङ्क्रुद्धो द्रोणं दृष्ट्वा स्मयन्निव}


\twolineshloka
{अथ दुर्योधनः कर्णमब्रवीद्बाह्लिकं नृपः}
{दुःशासनं मद्रराजं तांस्तथाऽन्यान्महारथान्}


\twolineshloka
{सर्वभूर्धाभिषिक्तानामाचार्यो ब्रह्मवित्तमः}
{अर्जुनस्य सुतं मूढं नायं हन्तुमिहेच्छति}


\twolineshloka
{न ह्यस्य समरे युद्ध्येदन्तकोऽप्याततायिनः}
{किमङ्ग पुनरेवान्यो मर्त्यः सत्यं ब्रवीमि वः}


\twolineshloka
{अर्जुनस्य सुतं त्वेष शिष्यत्वादभिरक्षति}
{शिष्याः पुत्राश्च दयितास्तदपत्यं च धर्मिणाम्}


\twolineshloka
{संरक्ष्यमाणो द्रोणेन मन्यते वीर्यमात्मनः}
{आत्मसम्भावितो मूढस्तं प्रमथ्नीत माचिरम्}


\twolineshloka
{एवमुक्तास्तु ते राज्ञा सात्वतीपुत्रमभ्ययुः}
{संरब्धास्ते जिघांसन्तो भारद्वाजस्य पश्यतः}


\twolineshloka
{दुःशासनस्तु तच्छ्रुत्वा दुर्योधनवचस्तदा}
{अब्रवीत्कुरुशार्दूल दुर्योधनमिदं वचः}


\twolineshloka
{अहमेनं हनिष्यामि महाराज ब्रवीमि ते}
{मिषतां पाण्डु पुत्राणां पाञ्चालानां च पश्यताम्}


\twolineshloka
{ग्रसिष्याम्यद्य सौभद्रं यथा राहुर्दिवाकरम्}
{उत्क्रुश्य चाब्रवीद्वाक्यं कुरुराजमिदं पुनः}


\twolineshloka
{श्रुत्वा कृष्णौ मया ग्रस्तं सौभद्रमतिमानिनौ}
{गमिष्यतः प्रेतलोकं जीवलोकान्न संशयः}


\twolineshloka
{तौ च श्रुत्वा मृतौ व्यक्तं पाण्डोः क्षेत्रोद्भवाः सुताः}
{एकाह्वा ससुहृद्वर्गाः क्लैब्याद्धास्यन्ति जीवितम्}


\threelineshloka
{तस्मादस्मिन्हते शत्रौ हताः सर्वेऽहितास्तव}
{शिवेन मां ध्याहि राजन्नेष हन्मि रिपूंस्तव ॥सञ्जय उवाच}
{}


\twolineshloka
{एवमुक्त्वाऽनदद्राजन्पुत्रो दुःशासनस्तव}
{सौभद्रमभ्ययात्क्रुद्धः शरवर्षैरवाकिरन्}


\twolineshloka
{तमतिक्रुद्धमायान्तं तव पुत्रमरिंदमः}
{अभिमन्युः शरैस्तीक्ष्णैः षड्विंशत्या समार्पयत्}


\twolineshloka
{दुःशासनस्तु सङ्क्रुद्धः प्रभिन्न इव कुञ्जरः}
{अयोधयत सौभद्रमभिमन्युश्च तं रणे}


\twolineshloka
{तौ मण्डलानि चित्राणि रथाभ्यां सव्यदक्षिणम्}
{चरमाणावयुध्येतां रथशिक्षाविशारदौ}


\twolineshloka
{अथ पणवमृदङ्गदुन्दुभीनांक्रकचमहानकभेरिझर्झराणम्}
{निनदमतिभृशं नराः प्रचक्रु--र्लवणजलोद्भवसिंहनादमिश्रम्}


\chapter{अध्यायः ४०}
\twolineshloka
{सञ्जय उवाच}
{}


\twolineshloka
{`ततः समभवद्युद्धं तयोः पुरुषसिंहयोः}
{तस्मिन्काले महाबाहुः सौभद्रः परवीरहा}


\twolineshloka
{सशरं कार्मुकं छित्त्वा लाघवेन व्यपातयत्}
{दुःशासनं शरैस्तीक्ष्णैः संततक्ष समन्ततः'}


\twolineshloka
{शरविक्षतगात्रं तु प्रत्यमित्रमवस्थितम्}
{अभिमन्युः स्मयन्धीमान्दुःशासनमथाब्रवीत्}


\twolineshloka
{दिष्ट्या पश्यामि सङ्ग्रमे मानिनं शूरमागतम्}
{निष्ठुरं त्यक्तधर्माणमाक्रोशनपरायणम्}


\twolineshloka
{यत्सभायां त्वया राज्ञो धृतराष्ट्रस्य शृण्वतः}
{कोपितः परुषैर्वाक्यैर्धर्मराजो युधिष्ठिरः}


\twolineshloka
{जयोन्मत्तेन भीमश्च बह्वबद्धं प्रभाषितः}
{अक्षकूटं समाश्रित्य सौबलस्यात्मनो बलम्}


\twolineshloka
{तत्त्वयेदमनुप्राप्तं तस्य कोपान्महात्मनः}
{परवित्तापहारस्य कोधस्याप्रशमस्य च}


\twolineshloka
{लोभस्य ज्ञाननाशस्य द्रोहस्यात्याहितस्य च}
{पितॄणां मम राज्यस्य हरणस्योग्रधन्विनाम्}


\twolineshloka
{तत्त्वयेदमनुप्राप्तं प्रकोपाद्वै महात्मनाम्}
{स तस्योग्रमधर्मस्य फलं प्राप्नुहि दुर्मते}


\twolineshloka
{शासिताऽस्म्यद्य ते बाणैः सर्वसैन्यस्य पश्यतः}
{अद्याहमनृणस्तस्य कोपस्य भविता रणे}


\fourlineindentedshloka
{अमर्षितायाः कृष्णायाः काङ्क्षितस्य च मे पितुः}
{अद्य कौरव्य भीमस्य भवितास्म्यनृणो युधि}
{न हि मे मोक्ष्यसे जीवन्यदि नोत्सृजसे रणं ॥सञ्जय उवाच}
{}


\twolineshloka
{एवमुक्त्वा महाबाहुर्बाणं दुःशासनान्तकम्}
{संदधे परवीरघ्नः कालाग्न्यनिलवर्चसम्}


\threelineshloka
{तस्योरस्तूर्णमासाद्य जत्रुदेशे विभिद्य तम्}
{जगाम सह पुङ्खेन वल्मीकमिव पन्नगः}
{अथैनं पञ्चविंशत्या पुनरेव समार्पयत्}


\fourlineindentedshloka
{शरैरग्निसमस्पर्शैराकर्णसमचोदितैः}
{स गाढविद्धो व्यथितो रथोपस्थ उपाविशत्}
{दुःशासनो महाराज कश्मलं चाविशन्महत् ॥सारथिस्त्वरमाणस्तु दुःशासनमचेतनम्}
{रणमध्यादपोवाह सौभद्रशरपीडितम्}


\twolineshloka
{पाण्डवा द्रौपदेयाश्च विराटश्च समीक्ष्य तम्}
{पाञ्चालाः केकयाश्चैव सिंहनादमथानदन्}


\twolineshloka
{वादित्राणि च सर्वाणि नानालिङ्गानि सर्वशः}
{प्रावादयन्त संहृष्टाः पाण्डूनां तत्र सैनिकाः}


\twolineshloka
{अपश्यन्स्मयमानाश्च सौभद्रस्य विचेष्टितम्}
{अत्यन्तवैरिणं दृप्तं दृष्ट्वा शत्रुं पराजितम्}


\twolineshloka
{धर्ममारुतशक्राणामश्विनोः प्रतिमास्तथा}
{धारयन्तो ध्वजाग्रेषु द्रौपदेया महारथाः}


\twolineshloka
{सात्यकिश्चेकितानश्च धृष्टद्युम्नशिखण्डिनौ}
{केकया धृष्टकेतुश्च मात्स्याः पाञ्चालसृञ्जयाः}


\twolineshloka
{पाण्डवाश्च मुदा युक्ता युधिष्ठिरपुरोगमाः}
{अभ्यद्रवन्त त्वरिता द्रोणानीकं बिभित्सवः}


\twolineshloka
{ततोऽभवन्महायुद्धं त्वदीयानां परैः सह}
{जयमाकाङ्क्षमाणानां शूराणामनिवर्तिनाम्}


\twolineshloka
{तथा तु वर्तमाने वै सङ्ग्रामेऽतिभयंकरे}
{दुर्योधनो महाराज राधेयमिदमब्रवीत्}


\threelineshloka
{पश्य दुःशासनं वीरमभिमन्युवशंगतम्}
{प्रतपन्तमिवादित्यं निघ्नन्तं शात्रवान्रणे}
{}


\threelineshloka
{अथ चैते सुसंरब्धाः सिंहा इव बलोत्कटाः}
{सौभद्रमुद्यतास्त्रातुमभ्यधावन्त पाण्डवाः ॥सञ्जय उवाच}
{}


\twolineshloka
{ततः कर्णः शरैस्तीक्ष्णैरभिमन्युं दुरासदम्}
{अभ्यवर्षत सङ्क्रुद्धः पुत्रस्य हितकृत्तव}


\twolineshloka
{तस्य चानुचरांस्तीक्ष्णैर्विव्याघ परमेषुभिः}
{अवज्ञापूर्वकं शूरः सौभद्रस्य परणाजिरे}


\twolineshloka
{अभिमन्युस्तु राधेयं त्रिसप्तत्या शिलीमुखैः}
{अविध्यत्त्वरितो राजन्द्रोणं प्रेप्सुर्महामनाः}


\twolineshloka
{तं तथा नाशकत्कश्चिद्दोणाद्वारयितुं रथी}
{आरुजन्तं रथव्रातान्वज्रहस्तात्मजात्मजम्}


\twolineshloka
{ततः कर्णो जयप्रेप्सुर्मानी सर्वधनुष्मताम्}
{सौभद्रं शतशोऽविध्यदुत्तमास्त्राणि दर्शयन्}


\twolineshloka
{सोऽस्त्रैरस्त्रविदां श्रेष्ठो रामशिष्यः प्रतापवान्}
{समरे शत्रुदुर्धर्षमभिमन्युमपीडयत्}


\twolineshloka
{स तथा पीड्यमानस्तु राधेयेनास्त्रवृष्टिभिः}
{समरेऽमरसङ्काशः सौभद्रो न व्यशीर्यत}


\twolineshloka
{ततः शिलाशितैस्तीक्ष्णैर्भल्लैरानतपर्वभिः}
{छित्त्वा धनूंषि शूराणामार्जुनिः कर्णमार्दयत्}


\twolineshloka
{धनुर्मण्डलनिर्मुक्तैः शरैराशीविषोपमैः}
{सच्छत्रध्वजयन्तारं साश्वमाशु स्मयन्निव}


\twolineshloka
{कर्णोऽपि चास्य चिक्षेप बाणान्सन्नतपर्वणः}
{असम्ब्रान्तश्च तान्सर्वानगृह्णात्फल्गुनात्मजः}


\twolineshloka
{ततो मुहूर्तात्कर्णस्य बाणेनैकेन वीर्यवान्}
{सध्वजं कार्मुकं वीरश्छित्त्वा भूमावपातयत्}


\twolineshloka
{ततः कृच्छ्रगतं कर्णं दृष्ट्वा कर्णादनन्तरः}
{सौभद्रमभ्ययात्तूर्णं दृढमुद्यम्य कार्मुकम्}


\twolineshloka
{तत उच्चुक्रुशुः पार्थास्तेषां चानुचरा जनाः}
{वादित्राणि च सञ्जघ्नः सौभद्रं चापि तुष्टुवुः}


\chapter{अध्यायः ४१}
\twolineshloka
{सञ्जय उवाच}
{}


\twolineshloka
{सोऽभ्यगच्छद्धनुष्पाणिर्ज्यां विकर्षन्पुनः पुनः}
{सच्छत्रध्वजयन्तारं साश्वमाशु स्मयन्निव}


\twolineshloka
{सोऽपिध्यद्दशभिर्बाणैरभिमन्युं दुरासदम्}
{सच्छत्रध्वजयन्तारं साश्वमाशु स्मयन्निव}


\twolineshloka
{पितृपैतामहं कर्म कुर्वाणमतिमानुषम्}
{दृष्ट्वार्दितं शरैः कार्ष्मिं त्वदीया हृषिताऽभवन्}


\twolineshloka
{तस्याभिमन्युरायम्य स्मयन्नेकेन पत्रिणा}
{शिरः प्रच्यावयामास तद्रथात्प्रापतद्भुवि}


\threelineshloka
{कर्णिकारमिवाधूतं वातेनापतितं नगात्}
{`कर्णानुजं च सम्प्रेक्ष्य तावका व्यथिताऽभवन्'}
{भ्रातरं निहतं दृष्ट्वा कर्णश्चासीत्पराङ्मुखः}


\twolineshloka
{विमुखीकृत्य कर्णं तु सौभद्रः कङ्कपत्रिभिः}
{अन्यानपि महेष्वासांस्तूर्णमेवाभिदुद्रुवे}


\twolineshloka
{ततस्तद्विततं सेन्यं हस्त्यश्वरथपत्तिमत्}
{क्रद्धोऽभिमन्युरभिनत्तिग्मतेजा महारथः}


\twolineshloka
{कर्णस्तु बहुभिर्बाणैरर्द्यमानोऽभिमन्युना}
{अपायाज्जवनैरश्वैस्ततोऽनीकमभिद्रवत्}


\twolineshloka
{`तिंष्ठ कर्ण महेष्वास कृप दुर्योधनेति च}
{द्रोणस्य क्रोशतो राजंस्तदनीकमभज्यत'}


\twolineshloka
{शलभैरिव चाकाशं धाराभिरिव पर्वतः}
{अभिमन्यो शरैश्छन्नं न प्राज्ञायत किञ्चन}


\twolineshloka
{तावकानां तु योधानां वध्यतां निशितैः शरैः}
{अन्यत्र सैन्धवाद्राजन्न स्म कश्चिदतिष्ठत}


% Check verse!
सौभद्रस्तु ततः शङ्खं प्रध्माय पुरुषर्षभः ॥शीघ्रमभ्यपतत्सेनां भारतीं भरतर्षभ
\twolineshloka
{स कक्षेऽग्निरिवोत्सृष्टो निर्दहंस्तरसा रिपून्}
{मध्ये भारतसैन्यानामार्जुनिः पर्यवर्तत}


\twolineshloka
{रथनागाश्वमनुजानर्दयन्निशितैः शरैः}
{सम्प्रविश्याकरोद्भूमिं कबन्धगणसङ्कुलाम्}


\twolineshloka
{सौभद्रचापप्रभवैर्निकृत्ताः परमेषुभिः}
{स्वानेवाभिमुखान्ध्नन्तः प्राद्रवञ्जीवितार्थिनः}


\twolineshloka
{ते घोरा रौद्रकर्माणो विपाठा बहवः शिताः}
{निघ्न्तो रथनागाश्वाञ्जग्मुराशु वसुन्धराम्}


\twolineshloka
{सायुधाः साङ्गुलित्राणाः सगदाः साङ्गदा रणे}
{दृश्यन्ते बाहवश्चिन्ना हेमाभारणभूषिताः}


\twolineshloka
{शराश्चापानि खङ्गाश्च शरीराणि शिरांसि च}
{सकुण्डलानि स्रग्वीणि भूमावासन्सहस्रशः}


\twolineshloka
{सोपस्करैरधिष्ठानैरीषादण्डकबन्धुरैः}
{अक्षैर्विमथितैश्चक्रैर्बहुधा पतितैर्युगैः}


\twolineshloka
{शक्तिचापासिभिश्चैव पतितैश्च महाध्वजैः}
{चर्मचापशरैश्चैव व्यवकीर्णैः समन्तततः}


\twolineshloka
{निहतैः क्षत्रियैरश्वैर्वारणैश्च विशाम्पते}
{अगम्यरूपा पृथिवी क्षणेनासीत्सुदारुणा}


\twolineshloka
{वध्यतां राजपुत्राणां क्रन्दतामितरेतरम्}
{प्रादुरासीन्महाशब्दो भीरूणां भयवर्धनः}


% Check verse!
स शब्दो भरतश्चेष्ठ दिशः सर्वा व्यनादयत् ॥सौभद्रश्चाद्रवत्सेनां घ्नन्वराश्वरथद्विपान्
\twolineshloka
{कक्षमग्निरिवोत्सृष्टो निर्दहंस्तरसा रिपून्}
{मध्ये भारतसैन्यानामार्जुनिः प्रत्यदृश्यत}


\twolineshloka
{विचरन्तं दिशः सर्वाः प्रदिशश्चापि भारत}
{तं तदा नानुपश्यामः सैन्ये च रजसाऽऽवृते}


\twolineshloka
{आददानं गजाश्वानां णां चायुंषि भारत}
{क्षमेन भूयः पश्यामः सूर्यं मध्यंदिने यथा}


\twolineshloka
{अभिमन्युं महाराज प्रतपन्तं द्विषद्गणान्}
{स वासवसमः सङ्ख्ये वासवस्यात्मजात्मजः}


\twolineshloka
{अभिमन्युर्महाराज सैन्यमध्ये व्यरोचत}
{`यथा पुरा वह्निसुतः सुरसैन्येषु वीर्यवान्'}


\chapter{अध्यायः ४२}
\twolineshloka
{धृतराष्ट्र उवाच}
{}


\twolineshloka
{बालमत्यन्तसुखिनं स्वबाहुबलदर्पितम्}
{युद्धेष्वकुशलं वीरं कुलपुत्रं तनुत्यजम्}


\threelineshloka
{गाहमानमनीकानि सदश्वैश्च त्रिहायनैः}
{अपि यौधिष्ठिरात्सैन्यात्कश्चिदन्वपतद्बली ॥सञ्जय उवाच}
{}


\twolineshloka
{युधिष्ठिरो भीमसेनः शिखण्डी सात्यकिर्यमौ}
{धृष्टद्युम्नो विराटश्च द्रुपदश्च सङ्केकयः}


\twolineshloka
{धृष्टकेतुश्च संरब्धो मात्स्याश्चाभ्यपतन्रणे}
{तेनैव तु पथा यान्तः पितरो मातुलैः सह}


\twolineshloka
{अभ्यद्रवन्परीप्सन्तो व्यूढानीकाः प्रहारिणः}
{तान्दृष्ट्वा द्रवतः शूरांस्त्वदीया विमुखाऽभवन्}


\twolineshloka
{ततस्तद्विमुखं दृष्ट्वा तव सूनोर्महद्बलम्}
{जामाता तव तेजस्वी संस्तम्भयिषुराद्रवत्}


\twolineshloka
{सैन्धवस्य महाराज पुत्रो राजा जयद्रथः}
{स पुत्रगृद्धिनः पार्थान्सहसैन्यानवारयत्}


\threelineshloka
{उग्रधन्वा महेष्वासो दिव्यमस्त्रमुदीरयन्}
{वार्धक्षत्रिरुपासेधत्प्रवणादिव कुञ्जरः ॥धृतराष्ट्र उवाच}
{}


\twolineshloka
{कतिभारमहं मन्ये सैन्धवे सञ्जयाहितम्}
{यदेकः पाण्डवान्क्रुद्धान्पुप्रेप्सूनवारयत्}


\twolineshloka
{अत्यद्भुतमहं मन्ये बलं शौर्यं च सैन्धवे}
{तस्य प्रब्रूहि मे वीर्यं कर्म चाग्र्यं महात्मनः}


\twolineshloka
{किं जप्तं हुतमिष्टं वा किं सुतप्तमथो तपः}
{`दमो वा ब्रह्मचर्यं वा सूत यच्चास्य सत्तम}


\twolineshloka
{देवं कतममाराध्य विष्णुमीशानमब्जजम्}
{सिन्धुराट् तनये सक्तान्क्रुद्धान्पार्थानवारयत्}


\threelineshloka
{नैवं कृतं महत्कर्म भीष्मेणाज्ञासिषं तथा'}
{सिन्धुराजो हि येनैकः पाण्डवान्समवारयत् ॥सञ्जय उवाच}
{}


\twolineshloka
{द्वीपदीहरणे यत्तद्भीमसेनेन निर्जितः}
{मानात्स तप्तवान्राजा वरार्थी सुमहत्तपः}


\twolineshloka
{इन्द्रियाणीन्द्रियार्थेभ्यः प्रियेभ्यः सन्निवर्त्य सः}
{क्षुत्पिपासातकपसहः कृशो धमनिसन्ततः}


\twolineshloka
{देवमाराधयच्छर्वं गृणन्ब्रह्म सनातनम्}
{भक्तानुकम्पी भगवांस्तस्य चक्रे ततो दयाम्}


\twolineshloka
{स्वप्नान्तेऽप्यथ चैवाह हरः सिन्धुपतेः सुतम्}
{वरं वृणीष्व प्रीतोऽस्मि जयद्रथ किमिच्छसि}


\twolineshloka
{एवमुक्तस्तु शर्वेण सिन्धुराजो जयद्रथः}
{उवाच प्रणतो रुद्रं प्राञ्जलिर्नियतात्मवान्}


\twolineshloka
{पाण्डवेयानहं सङ्ख्ये भीमवीर्यपराक्रमान्}
{वारयेयं रथेनैकः समस्तानिति भारत}


\threelineshloka
{एवमुक्तस्तु देवेशो जयद्रथमथाब्रवीत्}
{ददामि ते वरं सौम्य विना पार्थं धनञ्जयम्}
{वारयिष्यसि सङ्ग्रामे चतुरः पाण्डुनन्दनान्}


\twolineshloka
{`एकाहमिति राजेन्द्र तत्रैवान्तरधीयत'}
{एवमस्त्विति देवेशमुक्त्वाबुद्ध्यत पार्थिवः}


\twolineshloka
{स तेन वरदानेन दिव्येनास्त्बलेन च}
{एकः संवारयामास पाण्डवानामनीकिनीम्}


\twolineshloka
{तस्य ज्यातलघोषेण क्षत्रियान्भयमाविशत्}
{परांस्तु तव सैन्यस्य हर्षः परमकोऽभवत्}


\twolineshloka
{दृष्ट्वा तु क्षत्रिया भारं सैन्धवे सर्वमाहितम्}
{उत्क्रुश्याभ्यद्रवन्राजन्येन यौधिष्ठिरं बलम्}


\chapter{अध्यायः ४३}
\twolineshloka
{सञ्जय उवाच}
{}


\twolineshloka
{यन्मां पृच्छसि राजेन्द्र सिन्धुराजस्य विक्रमम्}
{शृणु तत्सर्वमाख्यास्ये यथा पाण्डूनयोधयत्}


\twolineshloka
{तमूहुर्वाजिनो वश्याः सैन्धवाः साधुवाहिनः}
{विकुर्वाणा बृहन्तोऽश्वाः श्वसनोपरंहसः}


\twolineshloka
{गन्धर्वनगराकारं विधिवत्कल्पितं रथम्}
{तस्याभ्यशोभयत्केतुर्वाराहो राजतो महान्}


\twolineshloka
{श्वेतश्चत्रपताकाभिश्चामरव्यजनेन च}
{स बभौ राजलिङ्गैस्तैस्तारापतिरिवाम्बरे}


\twolineshloka
{मुक्तावज्रमणिस्वर्णैर्भूषितं तदयस्मयम्}
{वरूथं विबभौ तस्य ज्योतिर्भिः खमिवावृतम्}


\twolineshloka
{स विष्फार्य महच्चापं किरन्निषुगणान्बहून}
{तत्कण्डं पूरयामास यदार्जुनिरदारयत्}


\twolineshloka
{स सात्यकिं त्रिभिर्बाणैरष्टभिश्च वृकोदरम्}
{धृष्टद्युम्नं तथा षष्ट्या विराटं दशभिः शरैः}


\twolineshloka
{द्रुपदं पञ्चभिस्तीक्ष्णैः सप्तभिश्च शिखण्डिनम्}
{केकयान्पञ्चविंशत्या द्रौपदेयांस्त्रिभिस्त्रिभिः}


\twolineshloka
{युधिष्ठिरं तु सप्तत्या ततः शेषानपानुदत्}
{इषुजालेन महता तदद्भुतमिवाभवत्}


\twolineshloka
{अथास्य शितपीतेन भल्लेनादिश्य कार्मुकम्}
{चिच्छेद प्रहसन्राजा धर्मपुत्रः प्रतापवान्}


\twolineshloka
{अक्ष्णोर्निमेषमात्रेण सोऽन्यदादाय कार्मुकम्}
{विव्याध दशभिः पार्थं तांश्चैवान्यांस्त्रिभिस्त्रिभिः}


\twolineshloka
{तत्तस्य लाघवं ज्ञात्वा भीमो भल्लैस्त्रिभिस्त्रिभिः}
{धनुर्ध्वजं च च्छत्रं च क्षितौ क्षिप्रमपातयत्}


\twolineshloka
{सोऽन्यदादाय बलवान्सज्जं कृत्वा च कार्मुकम्}
{भीमस्यापातयत्केतुं धनुरश्वांश्च मारिष}


\twolineshloka
{स हताश्वादवप्लुत्य च्छिन्नधन्वा रथोत्तमात्}
{सोत्यकेराप्लुतो यानं गिर्यग्रमिव केसरी}


\twolineshloka
{ततस्त्वदीयाः संहृष्टाः साधुसाध्विति वादिनः}
{सिन्धुराजस्य तत्कर्म प्रेक्ष्याश्रद्धेयमद्भुतम्}


\twolineshloka
{सङ्क्रुद्धान्पाण्डवानेको यद्दधारास्त्रतेजसा}
{तत्तस्य कर्म भूतानि सर्वाण्येवाभ्यपूजयन्}


\twolineshloka
{सौभद्रेण हतैः पूर्वं सोत्तरायोधिभिर्द्विपैः}
{पाण्डूनां दर्शितः पन्थाः सैन्धवेन निवारितः}


\twolineshloka
{यतमानास्तु ते वीरा मात्स्यपाञ्चालकेकयाः}
{पाण्डवाश्चान्वपद्यन्त प्रतिशेकुर्न सैन्धवम्}


\twolineshloka
{यो यो हि यतते भेत्तुं द्रोणानीकं तवाहितः}
{तत्तमेव वरं प्राप्य सैन्धवः प्रत्यवारयत्}


\chapter{अध्यायः ४४}
\twolineshloka
{सञ्जय उवाच}
{}


\twolineshloka
{सैन्धवेन निरुद्धेषु जयगृद्धिषु पाण़्डुषु}
{सुघोरमभवद्युद्धं त्वदीयानां परैः सह}


\twolineshloka
{प्रविश्याथार्जुनिः सेनां सत्यसन्धो दुरासदः}
{व्यक्षोभयत तेजस्वी मकरः सागरं यथा}


\twolineshloka
{तं तथा शरवर्षेण छादयन्तमरिंदमम्}
{यथा प्रधानाः सौभद्रमभ्ययुः कुरुसत्तमाः}


\twolineshloka
{तेषां तस्य च सम्मर्दो दारुणः समपद्यत}
{सृजतां शरवर्षाणि प्रसक्तममितौजसाम्}


\twolineshloka
{रथव्रजेन संरुद्धस्तैरमित्रैस्तथाऽऽर्जुनिः}
{वृषसेनस्य यन्तारं हत्वा चिच्छेद कार्मुकम्}


\twolineshloka
{तं च विव्याध बलवांस्तस्य चाश्वानजिह्मगैः}
{वातायमानैस्तैरश्वैर्विसंज्ञोऽपहृतो रणात्}


\twolineshloka
{तेनान्तरेणाभिमन्योर्यन्तापासारयद्रथम्}
{रथव्रजास्ततो हृष्टाः साधुसाध्विति चुक्रुशुः}


\twolineshloka
{तं सिंहमिव सङ्क्रुद्धं प्रमथ्नन्तं शरैररीन्}
{आरादायान्तमभ्येत्य वसातीयोऽभ्ययाद्द्रुतम्}


\twolineshloka
{सोऽभिमन्युं शरैः षष्ट्या रुक्मपुङ्कैरवाकिरत्}
{अब्रवीच्च न मे जीवञ्जीवतो युधि मोक्ष्यसे}


\twolineshloka
{तमयस्मयवर्माणमिषुणा दूरपातिना}
{विव्याध हृदि सौभद्रः स पपात व्यसुः क्षितौ}


\twolineshloka
{वसातीयं हतं दृष्ट्वा क्रुद्धाः क्षत्रियपुङ्गवाः}
{परिवव्रुस्तदा राजंस्तव पौत्रं जिघांसवः}


\twolineshloka
{विष्फारयन्तश्चापानि नानारूपाण्यनेकशः}
{तद्युद्धमभवद्रौद्रं सौभद्रस्यारिभिः सह}


\twolineshloka
{तेषां शरान्सेष्वसनाञ्शरीराणि शिरांसि च}
{सकुण्डलानि स्रग्वीणि क्रुद्धश्चिच्छेद फाल्गुनिः}


\twolineshloka
{सखङ्गाः साङ्गुलित्राणाः सपट्टसपरश्वथाः}
{अदृश्यन्त भुजाश्छिन्ना हेमाभरणभूषिताः}


\twolineshloka
{स्रग्भिराभरणैर्वस्त्रैः पतितैर्विविधैर्ध्वजैः}
{वर्मभिश्चर्मभिर्हारैर्मुकुटैश्छत्रचामरैः}


\twolineshloka
{उपस्करैरधिष्ठानैरीषादण्डकबन्धुरैः}
{अक्षैर्विमथितैश्चक्रैर्भग्नैश्च बहुधा युगैः}


\twolineshloka
{अनुकर्षैः पताकाभिस्तथा सारथिवाजिभिः}
{रथैश्च भग्नैर्नागैश्च हतैः कीर्णाभवन्मही}


\twolineshloka
{निहतैः क्षत्रियैः शूरैर्नानाजनपदेश्वरैः}
{जयगृद्धैर्वृता भूमिर्दारुणा समपद्यत}


\twolineshloka
{दिशो विचरतस्तस्य सर्वाश्च प्रदिशस्तथा}
{रणेऽभिमन्योः क्रुद्धस्य रूपमन्तरधीयत}


\twolineshloka
{काञ्चनं यद्यदस्यासीद्वर्म चाभरणानि च}
{धनुषश्च शराणां च तदपश्याम केवलम्}


\twolineshloka
{तं तदा नाशकत्कश्चिच्चक्षुर्भ्यामभिवीक्षितुम्}
{आददानं शरैर्योधान्मध्ये सूर्यमिव स्थितम्}


\chapter{अध्यायः ४५}
\twolineshloka
{सञ्जय उवाच}
{}


\twolineshloka
{आददानस्तु शूराणामायुंष्यभवदार्जुनिः}
{अन्तकः सर्वभूतानां प्राणान्काल इवागते}


\twolineshloka
{स शक्र इव विक्रान्तः शक्रसूनोः सुतो बली}
{अभिमन्युस्तदानीकं लोडयन्समदृश्यत}


\twolineshloka
{प्रविश्यैव तु राजेन्द्र क्षत्रियेन्द्रान्तकोपमः}
{सत्यश्रवसमादत्त व्याघ्रो मृगमिवोल्बणः}


\twolineshloka
{सत्यश्रवसि चाक्षिप्ते त्वरमाणा महारथाः}
{प्रगृह्य विपुलं शस्त्रमभिमन्युमुपाद्रवन्}


\twolineshloka
{अहंपूर्वमहंपूर्वमिति क्षत्रियपुङ्गवाः}
{स्पर्धमानाः समाजग्मुर्जिघांसन्तोऽर्जुनात्मजम्}


\twolineshloka
{क्षत्रियाणामनीकानि प्रद्रुतान्यभिधावताम्}
{जग्राह तिमिरासाद्य क्षुद्रमत्स्यानिवार्णिवे}


\twolineshloka
{ये केचन गतास्तस्य समीपमपलायिनः}
{न ते प्रतिन्यवर्तन्त समुद्रादिव सिन्धवः}


\twolineshloka
{महाग्रहगृहीतेव वातवेगभयार्दिता}
{समकम्पत सा सेना विभ्रष्टा नौरिवार्णवे}


\twolineshloka
{अथ रुक्मरथो नाम मद्रेश्वरसुतो बली}
{त्रस्तामाश्वासयन्सेनामत्रस्तो वाक्यमब्रवीत्}


\twolineshloka
{अलं त्रासेन वः शूरा नैष कश्चिन्मयि स्थिते}
{अहमेनं ग्रीहष्यामि जीवग्राहं न संशयः}


\twolineshloka
{एवमुक्त्वा तु सौभद्रमभिदुद्राव वीर्यवान्}
{सुकल्पितेनोह्यमानः स्यन्दनेन विराजता}


\twolineshloka
{सोऽभिमन्युं त्रिभिर्बाणैर्विद्ध्वा वक्षस्यथानदत्}
{त्रिभिश्च दक्षिणे बाहौ सव्ये च निशितैस्त्रिभिः}


\twolineshloka
{स तस्येष्वसनं छित्त्वा फाल्गुनिः सव्यदक्षिणौ}
{भुजौ शिरश्च स्वक्षिभ्रु क्षितौ क्षिप्रमपातयत्}


\twolineshloka
{दृष्ट्वा रुक्मरथं रुग्णं पुत्रं शल्यस्य मानिनम्}
{जीवग्राहं जिघृक्षन्तं सौभद्रेण यशस्विना}


\twolineshloka
{सङ्ग्रामदुर्मदा राजन्राजपुत्राः प्रहारिणः}
{वयस्याः शल्यपुत्रस्य सुवर्णविकृतध्वजाः}


\twolineshloka
{तालमात्राणि चापानि विकर्षन्तो महाबलाः}
{आर्जुनिं शरवर्षेण समन्तात्पर्यवारयन्}


\twolineshloka
{शूरैः निक्षाबलोपेतैस्तरुणैरत्यमर्षणैः}
{दृष्ट्वैकं समरे शूरं सौभद्रमपराजितम्}


\twolineshloka
{छाद्यमानं शरव्रातैर्हृष्टो दुर्योधनोऽभवत्}
{वैवस्वतस्य भवनं गतं ह्येममन्यत}


\twolineshloka
{सुवर्णपुङ्खैरिषुभिर्नानालिङ्गैः सुतेजनैः}
{अदृश्यमार्जुनिं चक्रुर्निमेषात्ते नृपात्मजाः}


\twolineshloka
{ससूताश्वध्वजं तस्य स्यन्दनं तं च मारिष}
{आचितं समपश्याम श्वाविधं शललैरिव}


\twolineshloka
{स गाढविद्धः क्रुद्धश्च तोत्रैर्गज इवार्दितः}
{गान्धर्वमस्त्रमायच्छद्रथमायां च भारत}


\twolineshloka
{अर्जुनो तपस्तप्त्वा गन्धर्वेभ्यो यदाहृतम्}
{तुम्बुरुप्रमुखेभ्यो वै तेनामोहयताहितान्}


\twolineshloka
{एकधा शतधा राजन्दृश्यते स्म सहस्रधा}
{अलातचक्रवत्सङ्ख्ये क्षिप्रमस्त्राणि दर्शयन्}


\twolineshloka
{रथचर्यास्त्रमायाभिर्मोहयित्वा परन्तपः}
{बिभेद शतधा राजञ्शरीराणि महीक्षिताम्}


\twolineshloka
{प्राणाः प्राणभृतां सङ्ख्ये प्रेषिता निशितैः शरैः}
{राजन्प्रापुरसुं लोकं शरीराण्यवनिं ययुः}


\twolineshloka
{धनूंष्यश्वान्नियन्तॄंश्च ध्वजान्बाहूंश्च साङ्गदान्}
{सिरांसि च शितैर्बाणैस्तेषां चिच्छेद फाल्गुनिः}


\twolineshloka
{चूतारामो यथा भग्नः पञ्चवर्षः फलोपगः}
{राजपुत्रशतं तद्वत्सौभद्रेण निपातितम्}


\twolineshloka
{क्रुद्धाशीविषसङ्काशान्सुकुमारान्सुखोचितान्}
{एकेन निहतान्दृष्ट्वा भीतो दुर्योधनोऽभवत्}


\twolineshloka
{रथिनः कुञ्जरानश्वान्पदातींश्चावमर्दितान्}
{दृष्ट्वा दुर्योधनः क्षिप्रमुपायात्तममर्षितः}


\twolineshloka
{तयोः क्षणमिवापूर्वः सङ्ग्रामः समपद्यत}
{अथाभवत्ते विमुखः पुत्रः शरशताहतः}


\chapter{अध्यायः ४६}
\twolineshloka
{धृतराष्ट्र उवाच}
{}


\twolineshloka
{यथा वदसि मे सूत एकस्य बहुभिः सह}
{सङ्ग्रामं तुमुलं घोरं जयं चैव महात्मनः}


\twolineshloka
{अश्रद्धेयमिवाश्चर्यं सौभद्रस्याथ विक्रमम्}
{किन्तु नात्यद्भुतं तेषां येषां धर्मो व्यपाश्रयः}


\threelineshloka
{दुर्योधने च विमुखे राजपुत्रशते हते}
{सौभद्रे प्रतिपत्तिं कां प्रत्यपद्यन्त मामकाः ॥सञ्जय उवाच}
{}


\twolineshloka
{संशुष्कास्याश्चलन्नेत्राः प्रस्विन्ना रोमहर्षणाः}
{पालयनकृतोत्साहा निरुत्साहा द्विष़ज्जये}


\twolineshloka
{हतान्भ्रातॄन्पितॄन्पुत्रान्सुहृत्सम्बन्धिबान्धवान्}
{उत्सृज्योत्सृज्य सञ्जग्मुस्त्वरयन्तो हयद्विपान्}


\twolineshloka
{तान्प्रभग्नांस्तथा दृष्ट्वा द्रोणो द्रौणिर्बृहद्बलः}
{कृपो दुर्योधनः कर्णः कृतवर्माथ सौबलः}


\threelineshloka
{अभ्यधावन्सुसङ्क्रुद्धाः सौभद्रमपराजितम्}
{ते तु पौत्रेण ते राजन्प्रायशो विमुखीकृताः}
{`सौभद्रेण महाराज शक्रप्रतिमतेजसा'}


\twolineshloka
{एकस्तु सुखसंवृद्धो बाल्याद्दर्पाच्च निर्भयः}
{इष्वस्त्रविन्महातेजा लक्ष्मणोऽर्जुनिमभ्ययात्}


\twolineshloka
{तमन्वगेवास्य पिता पुत्रगृद्धी न्यवर्तत}
{अनुदुर्योधनं चान्ये न्यवर्तन्त महारथाः}


\twolineshloka
{तं तेऽभिषिषिचुर्बाणैर्मेघा गिरिमिवाम्बुभिः}
{स तु तान्प्रममाथैको विष्वग्वातो यथाम्बुदान्}


\twolineshloka
{पौत्रं तव च दुर्धर्षं लक्ष्मणं प्रियदर्शनम्}
{पितुः समीपे तिष्ठन्तं शूरमुद्यतकार्मुकम्}


\threelineshloka
{अत्यन्तसुखसंवृद्धं धनेश्वरसुतोपमम्}
{आससाद रणे कार्ष्णिर्मत्तो मत्तमिव द्विपम्}
{`सिंहशाबो वने शश्वत्पुण्डरीकशिशुं यथा'}


\twolineshloka
{लक्ष्मणेन तु सङ्गम्य सौभद्रः परवीरहा}
{शरैः सुनिशितैस्तीक्ष्णैर्बाह्वोरुरसि चार्पयत्}


\twolineshloka
{सङ्क्रुद्धौ वै महाराज दण्डाहत इवोरगः}
{पौत्रस्तव महाराज तव पौत्रमभाषत}


\twolineshloka
{सुदृष्टः क्रियतां लोको ह्यमुं लोकं गमिष्यसि}
{पश्यतां बान्धवानां त्वां नयामि यमसादनम्}


\twolineshloka
{एवमुक्त्वा ततो भल्लं सौभद्रः परवीरहा}
{उद्बबर्ह महाबाहुर्निर्मुक्तोरगसन्निभम्}


\twolineshloka
{स तस्य भुजनिर्मुक्तो लक्ष्मणस्य सुदर्शनम्}
{सुनसं सुभ्रु केशान्तं शिरोऽहार्षीत्सकुण्डलम्}


\twolineshloka
{`पौत्रस्तु तव दुर्धर्षं लक्ष्मणं प्रियदर्शनम्}
{पितुः समीपे तिष्ठन्तं प्राहिणोद्यमसादनम्}


\twolineshloka
{अत्यन्तसुखसंवृद्धं धनेश्वरसुतोपमम्'}
{लक्ष्मणां निहतं दृष्ट्वा हाहेत्युच्चुक्रुशुर्जनाः}


\twolineshloka
{ततो दुर्योधनः क्रुद्धः प्रिये पुत्रे निपातिते}
{हतैनमिति चुक्रोश क्षत्रियान्क्षत्रियर्षभः}


\twolineshloka
{ततो द्रोणः कृपः कर्णो द्रोणपुत्रो बृहद्बलः}
{कृतवर्मा च हार्दिक्यः षड्रथाः पर्यवारयन्}


\twolineshloka
{तांस्तु विद्धा शितैर्बाणैर्विमुखीकृत्य चार्जुनिः}
{वेगेनाभ्यपतत्क्रुद्धः सैन्धवस्य महद्बलम्}


\threelineshloka
{आवव्रुस्तस्य पन्थानं गजानीकेन दंशिताः}
{कलिङ्गाश्च निषादाश्च क्राथपुत्रश्च वीर्यवान्}
{तत्प्रसक्तमिवात्यर्थं युद्धमासीद्विशाम्पते}


\twolineshloka
{ततस्तत्कुञ्जरानीकं व्यधमद्धृष्टमार्जुनिः}
{यथा वायुर्नित्यगतिर्जलदाञ्शतशोऽम्बरे}


\threelineshloka
{ततः क्रुद्धाः शरव्रातै राजानः समवारयन्}
{अथेतरे सन्निवृत्ताः पुनर्द्रोणमुखा रथाः}
{परमास्त्राणि धुन्वानाः सौभद्रमभिदुद्रुवुः}


\twolineshloka
{तान्निवार्यार्जुनिर्बाणैः क्राथपुत्रमयोधयत्}
{शरौघेणाप्रमेयेण त्वरमाणो जिघौसया}


\twolineshloka
{सधनुर्बाणकेयूरौ बाहू समुकुटं शिरः}
{सच्छत्रध्वजयन्तारं रथं चाश्वान्न्यपातयत्}


\twolineshloka
{कुलशीलश्रुतिबलैः कीर्त्या चास्त्रबलेन च}
{युक्ते तस्मिन्हते वीराः प्रायशो विमुखाऽभवन्}


\chapter{अध्यायः ४७}
\twolineshloka
{धृतराष्ट्र उवाच}
{}


\twolineshloka
{तथा प्रविष्टं तरुणं सौभद्रमपराजितम्}
{कुलानुरूपं कुर्वाणं सङ्ग्रामेष्वपराजितम्}


\threelineshloka
{आजानेयैः सुबलिभिर्यान्तमश्वैस्त्रिहायनैः}
{प्लवमानमिवाकाशे के शूराः समवारयन् ॥सञ्जय उवाच}
{}


\twolineshloka
{अभिमन्युः प्रविश्यैतांस्तावकान्निशितैः शरैः}
{अकरोत्पार्थिवान्सर्वान्विमुखान्पाण्डुनन्दनः}


\twolineshloka
{तं तु द्रोणः कृपः कर्णो द्रौणिश्च सबृहद्बलः}
{कृतवर्मा च हार्दिक्यः षड्रथाः पर्यवारयन्}


\twolineshloka
{दृष्ट्वा तु सैन्धवे भारमतिमात्रं समाहितम्}
{सैन्यं तव महाराज युधिष्ठिरमुपाद्रवत्}


\twolineshloka
{सौभद्रमितरे वीरमभ्यवर्षञ्शराम्बुभिः}
{तालमात्राणि चापानि विकर्षन्तो महाबलाः}


\twolineshloka
{तांस्तु सर्वान्महेष्वासान्सर्वविद्यासु निष्ठितान्}
{व्यदृष्टम्भयद्रणे बाणैः सौभद्रः परवीरहा}


\twolineshloka
{द्रोणं पञ्चाशताविध्यद्विंशत्या च बृहद्बलम्}
{अशीत्या कृतवर्माणं कृपं षष्ट्या शिलीमुखैः}


\twolineshloka
{रुक्मपुङ्खैर्महावेगैराकर्णसमचोदितैः}
{अविध्यद्दशभिर्बाणैरश्वत्थामानमार्जुनिः}


\twolineshloka
{कर्णं च कर्णिना कर्णे पीतेन च शितेन च}
{फाल्गुनिर्द्विषतां मध्ये विव्याध परमेषुणा}


\twolineshloka
{पातयित्वा कृपस्याश्वांस्तथोभौ पार्ष्णिसारथी}
{अथैनं दशभिर्बाणैः प्रत्यविध्यत्स्तनान्तरे}


\twolineshloka
{ततो बृन्दारकं वीरं कुरूणां कीर्तिवर्धनम्}
{पुत्राणां तव वीराणां पश्यतामवधीद्बली}


\twolineshloka
{तं द्रौणिः पञ्चविंशत्या क्षुद्रकाणां समार्पयत्}
{वरंवरममित्राणामारुजन्तमभीतवत्}


\twolineshloka
{स तु बाणैः शितैस्तूर्णं प्रत्यविध्यत मारिष}
{पश्यतां धार्तराष्ट्राणामश्वत्थामानमार्जुनिः}


\twolineshloka
{षष्ट्या शराणां तं द्रौणिस्तिग्मधारैः सुतेजनैः}
{उग्रैर्नाकम्पयद्विद्ध्वा मैनाकमिव पर्वतम्}


\twolineshloka
{स तु द्रौणिं त्रिसप्तत्या हेमपुङ्घैरजिह्मगैः}
{प्रत्यविध्यन्महातेजा बलवानपकारिणम्}


\twolineshloka
{तस्मिन्द्रोणो बाणशतं पुत्रगृद्धी न्यपातयत्}
{अश्वत्थामा तथाऽष्टौ च परीप्सन्पितरं रणे}


\twolineshloka
{कर्णो द्वाविंशतिं भल्लान्कृतवर्मा च विंशतिम्}
{बृहद्बलस्तु पञ्चाशत्कृपः शारद्वतो दश}


\twolineshloka
{तांस्तु प्रत्यवधीत्सर्वान्दशभिर्दशभिः शरैः}
{तैरर्द्यमानः सौभद्रः सर्वतो निशितैः शरैः}


\twolineshloka
{तं कोसलानामधिपः कर्णिनाऽताडयद्धृदि}
{स तस्याश्वान्ध्वजं चापं सूतं चापातयत्क्षितौ}


\twolineshloka
{अथ कोसलराजस्तु विरथः खङ्ग्रचर्मधृत्}
{इयेष फाल्गुनेः कायाच्छिरो हर्तुं सकुण्डलम्}


\twolineshloka
{स कोसलानामधिपं राजपुत्रं बृहद्बलम्}
{हृदि विव्याध वाणेन स भिन्नहृदयोऽपतत्}


\twolineshloka
{बभञ्ज च सहस्राणि दश राज्ञां महात्मनाम्}
{सृजतामशिवा वाचः खङ्गकार्मुकधारिणाम्}


\twolineshloka
{तथा बृहद्बलं हत्वा सौभद्रो व्यचरद्रणे}
{व्यष्टम्भयन्महेष्वासो योधांस्तव शराम्बुभिः}


\chapter{अध्यायः ४८}
\twolineshloka
{सञ्जय उवाच}
{}


\twolineshloka
{स कर्णं कर्णिना कर्णे पुनर्विव्याध फाल्गुनिः}
{शरैः पञ्चाशता चैनमविध्यत्कोपयन्भृशम्}


\twolineshloka
{प्रतिविव्याध राधेयस्तावद्भिरथ तं पुनः}
{शरैराचितसर्वाङ्गो बह्वशोभत भारत}


\threelineshloka
{कर्णं चाप्यकरोत्क्रुद्धो रुधिरोत्पीडवाहिनम्}
{कर्णोऽपि विबभौ शूरः शरैश्छिन्नोऽसृगातप्लुः}
{`सन्धानुगतपर्यन्तः शरदीव दिवाकरः'}


\twolineshloka
{तावुभौ शरचित्राङ्गौ रुधिरेण समुक्षितौ}
{बभूवतुर्महात्मानौ पुष्पिताविव किंशुकौ}


\twolineshloka
{अथ कर्णस्य सचिवान्षट् शूरांश्चित्रयोधिनः}
{साश्वसूतध्वजरथान्सौभद्रो निजघान ह}


\twolineshloka
{तथेतरान्महेष्वासान्दशभिर्दशभिः शरैः}
{प्रत्यविध्यदसम्भ्रान्तस्तदद्भुतमिवाभवत्}


\twolineshloka
{मागधस्य तथा पुत्रं हत्वा षड्भिरजिह्मगैः}
{साश्वं ससूतं तरुणमश्वकेतुमपातयत्}


\twolineshloka
{मार्तिकावतकं भोजं ततः कुञ्जरकेतनम्}
{क्षुरप्रेण समुन्मथ्य ननाद विसृजन्शरान्}


\twolineshloka
{तस्य दौःशासनिर्विद्ध्वा चतुर्भिश्चतुरो हयान्}
{सूतमेकेन विव्याध दशभिश्चार्जुनात्मजम्}


\twolineshloka
{ततो दौःशासनिं कार्ष्णिर्विद्ध्वा सप्तभिराशुगैः}
{संरम्भाद्रक्तनयनो वाक्यमुच्चैरथाब्रवीत्}


\twolineshloka
{पिता तवाहवं त्यक्त्वा गतः कापुरुषो यथा}
{दिष्ट्या त्वमपि जानीषे योद्धुं न त्वद्य मोक्ष्यसे}


\twolineshloka
{एतावदुक्त्वा वचनं कर्मारपरिमार्जितम्}
{नाराचं विससर्जास्मै तं द्रौणिस्त्रिभिराच्छिनत्}


\twolineshloka
{तस्यार्जुनिर्ध्वजं छित्त्वा शल्यं त्रिभिरताडयत्}
{तं शल्यो नवभिर्बाणैर्गार्ध्रपत्रैरताडयत्}


\twolineshloka
{हृद्यसम्भ्रान्तवद्राजंस्तदद्भुतमिवाभवत्}
{तस्यार्जुनिर्ध्वजं छित्त्वा हत्वोभौ पार्ष्णिसारथी}


\twolineshloka
{तं विव्याधायसैः षड्भिः सोपाक्रामद्रथान्तरम्}
{शत्रुञ्जयं चन्द्रकेतुं मेघवेगं सुवर्चसम्}


\twolineshloka
{सूर्यभासं च पञ्चैतान्हत्वा विव्याध सौबलम्}
{तं सौबलस्त्रिभिर्विद्ध्वा दुर्योधनमथाब्रवीत्}


\twolineshloka
{सर्व एनं विमथ्नीमः पुरैकैकं हिनस्ति नः}
{अथाब्रवीत्पुनर्द्रोणं कर्णो वैकर्तनो रणे}


\twolineshloka
{पुरा सर्वान्प्रमथ्नाति ब्रूह्यस्य वधम्माशु नः}
{ततो द्रोणो महेष्वासः सर्वांस्तान्प्रत्यभाषत}


\twolineshloka
{अस्ति वाऽस्यान्तरं किञ्चित्कुमारस्याथ पश्यत}
{अन्वस्य पितरं चाद्य चरतः सर्वतो दिशम्}


\twolineshloka
{शीघ्रतां नरसिंहस्य पाण्डवेयस्य पश्यत}
{धनुर्मण्डलमेवास्य रथमार्गेषु दृश्यते}


\twolineshloka
{संदधानस्य विशिखाञ्शीघ्रं चैव विमुञ्चतः}
{आरुजन्नपि मे प्राणान्मोहयन्नपि सायकैः}


\twolineshloka
{प्रहर्षयति मां भूयः सौभद्रः परवीरहा}
{अतिमानं दधात्येष सौभद्रो विचरन्रणे}


\fourlineindentedshloka
{अन्तरं यस्य संरब्धा न पश्यन्ति महारथाः}
{अस्यतो लघुहस्तस्य दिशः सर्वा महेषुभिः}
{न विशेषं प्रपश्यामि रणे गाण्डीवधन्वनः ॥सञ्जय उवाच}
{}


\twolineshloka
{अथ कर्णः पुनर्द्रोणमाहार्जुनिशराहतः}
{स्थातव्यमिति तिष्ठामि पीड्यमानोऽभिमन्युना}


\twolineshloka
{तेजस्विनः कुमारस्य शराः परमदारुणाः}
{क्षिण्वन्ति हृदयं मेऽद्य घोराः पावकतेजसः}


\twolineshloka
{तमाचार्योऽब्रवीत्कर्णं शनकैः प्रहसन्निव}
{अभेद्यमस्य कवचं युवा चाशुपराक्रमः}


\twolineshloka
{उपदिष्टा मया चास्य पितुः कवचधारणा}
{तामेष निखिलां वेत्ति ध्रुवं परपुरञ्जयः}


\twolineshloka
{शक्यं त्वस्य धनुश्छेत्तुं ज्यां च बाणैः समाहितैः}
{अभीषूंश्च हयांश्चैव तथोभौ पार्ष्णिसारथी}


\twolineshloka
{एतत्कुरु महेष्वास राधेय यदि शक्यते}
{अथैनं विमुखीकृत्य पश्चात्प्रहरणं कुरु}


\threelineshloka
{सधनुष्को न शक्योऽयमपि जेतुं सुरासुरैः}
{विरथं विधनुष्कं च कुरुष्वैनं यदीच्छसि ॥`पुनः स्थितेन केनापि दुःशको जेतुमार्जुनिः' ॥सञ्जय उवाच}
{}


\twolineshloka
{तदाचार्यवचः श्रुत्वा कर्णो वैकर्तनस्त्वरन्}
{अस्यतो लघुहस्तस्य पृष्ठतो धनुराच्छिनत्}


\twolineshloka
{अश्वानस्यावधीद्दोणो गौतमः पार्ष्णिसारथी}
{शेषास्तु च्छिन्नधन्वानं शरवर्षैरवाकिरन्}


\twolineshloka
{त्वरमाणास्त्वराकाले विरथं षण्महारथाः}
{शरवर्षैरकरुणा बालमेकमवाकिरन्}


\twolineshloka
{स च्छिन्नधन्वा विरथः स्वधर्ममनुपालयन्}
{खङ्गचर्मधरः श्रीमानुत्पपात विहायसा}


\twolineshloka
{मार्गैः सकौशिकाद्यैश्च लाघवेन बलेन च}
{आर्जुनिर्व्यचरद्य्वोम्नि भृशं वै पक्षिराडिव}


\twolineshloka
{मय्येव निपतत्येष सासिरित्यूर्ध्वदृष्टयः}
{विव्यधुस्तं महेष्वासं समरे छिद्रदर्सिनः}


% Check verse!
क्षुरप्रेण महातेजास्त्वरमाणः सपत्नजित्
% Check verse!
राधेयो निशितैर्बाणैर्व्यधमच्चर्म चोत्तमम्
\twolineshloka
{व्यसिचर्मेषुपूर्णाङ्गः सोऽन्तरिक्षात्पुनः क्षितिम्}
{आस्थितश्चक्रमुद्यम्य द्रोणं क्रुद्धोऽभ्यधावत}


\twolineshloka
{स चक्ररेणूकज्ज्वलशोभिताङ्गोबभावतीवोज्ज्वलचक्रपाणिः}
{रणेऽभिमन्युः क्षणमास रौद्रःस वासुदेवानुकृतिं प्रकुर्वन्}


\twolineshloka
{स्रुतरुधिरकृतैकरागवस्त्रोभ्रुकुटिपुटाकुलितोऽतिसिंहनादः}
{प्रभुरमितबलो रणेऽभिमन्यु--र्नृपवरमध्यगतो भृशं व्यराजत्}


\chapter{अध्यायः ४९}
\twolineshloka
{सञ्जय उवाच}
{}


\twolineshloka
{विष्णोः स्वसुर्नन्दकरः स विष्ण्वायुधभूषणः}
{रराजातिरथः सङ्ख्ये जनार्दन इवापरः}


\twolineshloka
{मारुतोद्धूतकेशान्तमुद्यतारिवरायुधम्}
{वपुः समीक्ष्य पृथ्वीशा दुःसमीक्ष्यं सुरैरपि}


\twolineshloka
{`यदि पाणितलादेतच्चक्रं मुञ्चेत फाल्गुनिः}
{वरदानान्मातुलस्य विष्णोश्चक्रमिवापतेत्'}


\twolineshloka
{तच्चक्रं भृशमुद्विग्नाः संचिच्छिदुरनेकधा}
{महारथस्ततः कार्ष्मिः स जग्राह महागदाम्}


\twolineshloka
{विधनुःस्यन्दनासिस्तैर्विचक्रश्चारिभिः कृतः}
{अभिमन्युर्गदापाणिरश्वत्थामानमार्दयत्}


\twolineshloka
{स गदामुद्यतां दृष्ट्वा ज्वलन्तीमशनीमिव}
{अपाक्रामद्रथोपस्थाद्विक्रमांस्त्रीन्नरर्षभः}


\twolineshloka
{तस्याश्वान्गदया हत्वा तथोभौ पार्ष्णिसारथी}
{शराचिताङ्गः सौभद्रः श्वाविद्वत्समदृश्यत}


\twolineshloka
{ततः सुबलदायादं कालिकेयमपोथयत्}
{जघान चास्यानुचरान्गान्धारान्सप्तसप्ततिम्}


\threelineshloka
{पुनश्चैव वसातीयाञ्जघान रथिनो दश}
{केकयानां रथान्सप्त हत्वा च दश कुञ्जरान्}
{दौःशासनिरथं साश्वं गदया समपोथयत्}


\twolineshloka
{ततो दौःशसनिः क्रुद्धो गदामुद्यम्य मारिष}
{अभिदुद्राव सौभद्रं तिष्ठतिष्ठेति चाब्रवीत्}


\twolineshloka
{तावुद्यतगदौ वीरावन्योन्यवधकाङ्क्षिणौ}
{भ्रातृव्यौ सम्प्रजहाते पुरेव त्र्यम्बकान्धकौ}


\twolineshloka
{तावन्योन्यं गदाग्राभ्यामाहत्य पतितौ क्षितौ}
{इन्द्रध्वजाविवोत्सृष्टौ रणमध्ये परंतपौ}


\twolineshloka
{दौःशासनिरथोत्थाय कुरूणां कीर्तिवर्धनः}
{उत्तिष्ठमानं सौभद्रं गदया मूर्ध्न्यताडयत्}


\twolineshloka
{गदावेगेन महता व्यायामेन च मोहितः}
{विचेता न्यपतद्भूमौ सौभद्रः परवीरहा}


\twolineshloka
{एवं विनिहतो राजन्नेको बहुभिराहवे}
{क्षोभयित्वा चमूं सर्वां नलिनीमिव कुञ्जरः}


\twolineshloka
{अशोभत हतो वीरो व्याधैर्वनगजो यथा}
{तं तथा पतितं शूरं तावकाः पर्यवारयन्}


\twolineshloka
{दावं दग्ध्वा यथा शान्तं पावकं शिशिरात्यये}
{विमृद्व्य नगशृङ्गाणि सन्निवृत्तमिवानिलम्}


\twolineshloka
{अस्तिं गतमिवादित्यं तप्त्वा भारत वाहिनीम्}
{उपप्लुतं यथा सोमं संशुष्कमिव सागस्म्}


\threelineshloka
{पूर्णचन्द्राभवदनं काकपक्षवृतालिकम्}
{तं भूमौ पतितं दृष्ट्वा तावकास्ते महारथाः}
{मुदा परमया युक्ताश्चुक्रुशुः सिंहवन्मुहुः}


\twolineshloka
{आसीत्परमको हर्षस्तावकानां विशाम्पते}
{इतरेषां तु वीराणां नेत्रेभ्यः प्रापतज्जलम्}


\twolineshloka
{अन्तरिक्षे च भूतानि प्राक्रोशन्त विशाम्पते}
{दृष्ट्वा निपतितं वीरं च्युतं चन्द्रमिवाम्बरात्}


\twolineshloka
{द्रोणकर्णमुखैः षड्भिर्धार्तराष्ट्रैर्महारथैः}
{एकोऽयं निहतः शेते नैष धर्मो मतो हि नः}


\twolineshloka
{तस्मिन्विनिहते वीरे बह्वशोभत मेदिनी}
{द्यौर्यथा पूर्णचन्द्रेण नक्षत्रगणमालिनी}


\twolineshloka
{रुक्मपुङ्खैश्च सम्पूर्णा रुधिरौघपरिप्लुता}
{उत्तमाङ्गैश्च शूराणां भ्राजमानैः सकुण्डलैः}


\twolineshloka
{विचित्रैश्च परिस्तोमैः पताकाभिश्च संवृता}
{चामरैश्च कुथाभिश्च प्रविद्धैश्चाम्बरोत्तमैः}


\twolineshloka
{रथाश्वनरनागानामलङ्कारैश्च सुप्रभैः}
{खङ्गैः सुनिशितैः पीतैर्निर्मुक्तैर्भुजगैरिव}


\twolineshloka
{चापैश्च विविधैश्छिन्नैः शक्त्यृष्टिप्रासकम्पनैः}
{विविधैश्चायुधैश्चान्यैः संवृता भूरशोभत}


\twolineshloka
{`निष्टनद्भिरतीवान्यैरुद्वहद्रुधिरस्रवैः}
{नरैः पतद्भिः पतितैरवनी स्वधिकं बभौ}


\twolineshloka
{वाजिभिश्चापि निर्जीवैः श्वसद्भिः शोणितोक्षितैः}
{सारोहैर्विषमा भूमिः सौभद्रेण निपातितैः}


\twolineshloka
{साङ्कुशैः समहामात्रैः सर्वमायुधकेतुभिः}
{पर्वतैरिव विध्वस्तैर्विशिखैर्मथितैर्गजैः}


\twolineshloka
{पृथिव्यानुकीर्णैश्च अश्वसारथियोधिभिः}
{हदैरिव प्रक्षुभितैर्हृतनागै रथोत्तमैः}


\twolineshloka
{पदातिसङ्घैश्च हतैर्विविधायुधभूषणैः}
{भीरूणां त्रासजननी घोररूपाऽभवन्मही}


\twolineshloka
{तं दृष्ट्वा पतितं भूमौ चन्द्रार्कसदृशद्युतिम्}
{तावकानां परा प्रीतिः पाण्डूनां चाभवद्व्यथा}


\twolineshloka
{अभिमन्यौ हते राजञ्शिशुकेऽप्राप्तयौवने}
{सम्प्राद्रवच्चमूः सर्वा धर्मराजस्य पश्यतः}


\twolineshloka
{दीर्यमाणं बलं दृष्ट्वा सौभद्रे विनिपातिते}
{अजातशत्रुस्तान्वीरानिदं वचनमब्रवीत्}


\twolineshloka
{स्वर्गमेष गतः शूरो यो हतो न पराङ्मुखः}
{संस्तम्भयत माभैष्ट विजेष्यामो रणे रिपून्}


\twolineshloka
{इत्येवं स महातेजा दुःखितेभ्यो महाद्युतिः}
{धर्मराजो युधां श्रेष्ठो ब्रुवन्दुःखमपानुदत्}


\twolineshloka
{युद्धे ह्याशीविषाकारान्राजपुत्रान्रणे रिपून्}
{पूर्वं निहत्य सङ्ग्रामे पश्चादार्जुनिरभ्ययात्}


\twolineshloka
{हत्वा दशसहस्राणि कौसल्यं च महारथम्}
{कृष्णार्जुनसमः कार्ष्णिः शक्रलोकं गतो ध्रुवम्}


\threelineshloka
{रथाश्वनरमातङ्गान्विनिहत्य सहस्रशः}
{अवितृप्तः स सङ्ग्रामादशोच्यः पुण्यकर्मकृत्}
{}


% Check verse!
गतः पुण्यकृतां लोकाञ्शाश्वतान्पुण्यनिर्जितान्
\chapter{अध्यायः ५०}
\twolineshloka
{सञ्जय उवाच}
{}


\twolineshloka
{वयं तु प्रवरं हत्वा तेषां तैः शरपीडिताः}
{निवेशायाभ्युपायामः सायाह्ने रुधिरोक्षिताः}


\twolineshloka
{निरीक्षमाणास्तु वयं परे चायोधनं शनैः}
{अपयाता महाराज ग्लानिं प्राप्ता विचेतसः}


\twolineshloka
{ततो निशाया दिवसस्य चाशिवःशिवारुतैः सन्धिरवर्तताद्भुतः}
{कुशेशयापीडनिभे दिवाकरेविलम्बमानेऽस्तमुपेत्य पर्वतम्}


\twolineshloka
{वरासिशक्त्यृष्टिवरूथचर्मणांविभूषणानां च समाक्षिपन्प्रभाः}
{दिवं च भूमिं च समानयन्निव प्रियां तनुं भानुरुपैति पावकम्}


\twolineshloka
{महाभ्रकूटाचलशृङ्गसन्निभै--र्गजैरनेकैरिव वज्रपातितैः}
{सवैजयन्त्यङ्कुशवर्मयन्तृभि--र्निपातितैर्नष्टगतिश्चिता क्षितिः}


\twolineshloka
{हतेश्वरैश्चूर्णितचक्रकूबरै--र्हताश्वसूतैर्विपताककेतुभिः}
{महारथैर्भूः शुशुभे विचूर्णितैःपुरैरिवामित्रहतैर्नराधिप}


\twolineshloka
{रथाश्वबृन्दैः सहसादिभिर्हतैःप्रविद्धभाण्डाभरणैः पृथग्विधैः}
{निरस्तजिह्वादशनान्त्रलोचनै--र्धरा बभौ घोरविरूपदर्शना}


\twolineshloka
{प्रविद्धवर्माभणाम्बरायुधाविपन्नहस्त्यश्वरथानुगा नराः}
{महार्हशय्यास्तरणोचितास्तदाक्षितावनाथा इव शेरते हताः}


\twolineshloka
{अतीव हृष्टाः श्वशृगालवायसाबकाः सुपर्णाश्च वृकास्तरक्षवः}
{वयांस्यसृक्पान्यथ रक्षसां गणाःपिशाचसङ्घाश्च सुदारुणा रणे}


\twolineshloka
{त्वचो विनिर्भिद्य पिबन्वसामसृक्तथैव मज्जाः पिशितानि चाश्नुवन्}
{वपां विलुम्पन्ति हसन्ति गान्ति चप्रकर्षमाणाः कुणपान्यनेकशः}


\twolineshloka
{शरीरसङ्घाटवहा ह्यसृग्जलारथोडुपा कुञ्जरशैलसङ्कटा}
{मनुष्यशीर्षोपलमांसकर्दमाप्रविद्धनानाविधशस्त्रमालिनी}


\twolineshloka
{भयावहा वैतरणीव दुस्तराप्रवर्तिता योधवरैस्तदा नदी}
{उवाह मध्येन रणाजिरे भृशंभयावहा दीनमृतप्रवाहिनी}


\twolineshloka
{पिबन्ति च स्नान्ति च यत्र दुर्दृशाःपिशाचसङ्घास्तु नदन्ति भैरवाः}
{सुनन्दिताः प्रणभृतां क्षयंकराःसमानभक्षाः श्वसृगालपक्षिणः}


\twolineshloka
{तथा तदायोधनमुग्रदर्शनंनिशामुखे पितृपतिराष्ट्रवर्धनम्}
{निरीक्षमाणः शनकैर्जहुर्नराःसमुत्थिता नृत्तकबन्धसङ्कुलम्}


\twolineshloka
{अपेतविध्वस्तमहार्हभूषणंनिपातितं शक्रसमं महाबलम्}
{रणेऽभिमन्युं ददृशुस्तदा जनाव्यपोढहव्यं सदसीव पावकम्}


\chapter{अध्यायः ५१}
\twolineshloka
{सञ्जय उवाच}
{}


\twolineshloka
{हते तस्मिन्महावीर्ये सौभद्रे रथयूथपे}
{विमुक्तरथसन्नाहाः सर्वे निक्षिप्तकार्मुकाः}


\twolineshloka
{उपोपविष्टा राजानं परिवार्य युधिष्ठिरम्}
{तदेव युद्धं ध्यायन्तः सौभद्रगतमानसाः}


\threelineshloka
{ततो युधिष्ठिरो राजा विललाप सुदुःखितः}
{ततो युधिष्ठिरो राजा विललाप सुदुःखितः}
{अभिमन्यौ हते वीरे भ्रातुः पुत्रे महारथे}


\threelineshloka
{युधिष्ठिर उवाच}
{एनं जित्वा कृपं शल्यं राजानं च सुयोधनम्}
{द्रोँणं द्रौणिं महेष्वासं तथैवान्यान्महारथान्}


\twolineshloka
{द्रोणानीकमसम्बाधं मम प्रियचिकीर्षया}
{`हत्वा शत्रुगणान्वीरानेष शेते महारथः}


\twolineshloka
{कृतास्त्रयु द्धकुशलान्महेष्वासान्महाबलान्}
{कुलशः पूणैर्युक्ताञ्छूरान्विख्यातपौरुषान्}


\twolineshloka
{द्रोणेन र्प्राहेतं व्यूहमभेद्यममरैरपि}
{अदृष्टपूर्वमस्माभिः पद्मं चक्रायुधप्रियः'}


\twolineshloka
{भित्त्वा मध्यं प्रविष्टोऽसौ गोमध्यमिव केसरी}
{`विक्रीडितं रणे तेन निघ्नता वै परान्वरान्'}


\twolineshloka
{यस्य शूरा महेष्वासाः प्रत्यनीकगता रणे}
{प्रभग्ना विनिवर्तन्ते कृतास्त्रा युद्धदुर्मदाः}


\twolineshloka
{अत्यन्तशत्रुरस्माकं येन दुःशासनः शरैः}
{क्षिप्रं ह्यभिमुखः सङ्ख्ये विसंज्ञो विमुखीकृतः}


\twolineshloka
{स तीर्त्वा दुस्तरं वीरो द्रोणानीकमहार्णवम्}
{प्राप्य दौःशासनिं कार्ष्णिः प्राप्तो वैवस्वतक्षयम्}


\twolineshloka
{कथं द्रक्ष्यामि कौन्तेयं सौभद्रे निहतेऽर्जुनम्}
{सुभद्रां वा महाभागां प्रियं पुत्रमपश्यतीम्}


\threelineshloka
{`हतवत्सां यथा धेनुं तद्दृर्शनकृतोन्मुखीम्'}
{किंस्विद्वक्ष्याम्यपेतार्थमक्लिष्टमसमञ्जसम्}
{तावुभौ प्रतिवक्ष्यामि हृषीकेशधनञ्जयौ}


\twolineshloka
{अहमेव सुभद्रायाः केशवार्जुनयोरपि}
{प्रियकामो जयाकाङ्क्षी कृतवानिदमप्रियम्}


\twolineshloka
{न लुब्धो बुध्यते दोषाँल्लोभान्मोहात्प्रवर्तते}
{मधुलिप्सुर्हि नापश्यं प्रपातमहमीदृशम्}


\twolineshloka
{यो हि भोज्ये पुरस्कार्यो यानेषु शयनेषु च}
{भूषणेषु च सोऽस्माभिर्बालो युधि पुरस्कृतः}


\twolineshloka
{कथं हि बालस्तरुणो युद्धानामविशारदः}
{सदश्व इव सम्बाधे विषमे क्षेममर्हति}


\twolineshloka
{नो चेद्धि वयमप्येनं महीमनुशयीमहि}
{बीभत्सोः कोपदीप्तस्य दग्धाः कृपणचक्षुषा}


\twolineshloka
{अलुब्धो मितमान्हीमान्क्षमावान्रूपवान्बली}
{वपुष्मान्मानकृद्वीरः प्रियः सत्यपराक्रमः}


\twolineshloka
{यस्य श्लाघन्ति विबुधाः कर्माण्यूर्जितकर्मणः}
{निवातकवचाञ्जघ्ने कालकेयांश्च वीर्यवान्}


\twolineshloka
{महेन्द्रशत्रवो येन हिरण्यपुरवासिनः}
{अक्ष्णोर्निमेषमात्रेण पौलोभाः सगणा हताः}


\twolineshloka
{परेभ्योऽप्यभयार्थिभ्यो यो ददात्यभयं विभुः}
{तस्यास्माभिर्न शकितस्त्रातुमप्यात्मजो बली}


\twolineshloka
{भयं तु सुमहत्प्राप्तं धार्तराष्ट्रान्महाबलान्}
{पार्थः पुत्रवधात्क्रुद्धः कौरवाञ्शोषयिष्यति}


\twolineshloka
{क्षुद्रः क्षुद्रसहायश्च स्वपक्षक्षयकारकः}
{व्यक्तं दुर्योधनो दृष्ट्वा शोचन्हास्यति जीवितं}


\twolineshloka
{न मे जयः प्रीतिकरो न राज्यंन चामरत्वं न सुरैः सलोकता}
{इमं समीक्ष्याप्रतिवीर्यपौरुषंनिपातितं देववरात्मजात्मजम्}


\chapter{अध्यायः ५२}
\twolineshloka
{सञ्जय उवाच}
{}


\twolineshloka
{एवं विलपमाने तु कुन्तीपुत्रे युधिष्ठिरे}
{कृष्णद्वैपायनस्तत्र आजगाम महानृषिः}


\twolineshloka
{अथ दृष्ट्वा महात्मानं पाण्डुपुत्रो युधिष्ठिरः}
{युक्तासनो दीनमनाः पूजां चक्रे महात्मनः}


\twolineshloka
{अर्चयित्वा यथान्यायमुपविष्टं युधिष्ठिरः}
{अब्रवीच्छोकसन्तप्तो भ्रातुः पुत्रवधेन च}


\twolineshloka
{अधर्मयुक्तैर्बहुभिः परिवार्य महारथैः}
{युध्यमानो महेष्वासैः सौभद्रो निहतो रणे}


\twolineshloka
{बालश्च बालबुद्धिश्च वीरश्च परवीरहा}
{अनुपायेन सङ्ग्रामे युध्यमानो विनाशितः}


\twolineshloka
{मया प्रोक्तः स सङ्ग्रामे द्वारं सञ्जनयस्व नः}
{प्रविष्टेऽभ्यन्तरे तस्मिन्सैन्धवेन निवारिताः}


\twolineshloka
{ननु नाम समं युद्धमेष्टव्यं युद्धजीविभिः}
{इदं चैवासमं युद्धमीदृशं यत्कृतं परैः}


\threelineshloka
{तेनास्मि भृशसन्तप्तः शोकबाष्पसमाकुलः}
{शमं नैवाधिगच्छामि चिन्तयानः पुनः पुनः ॥सञ्जय उवाच}
{}


\threelineshloka
{तं तथा विलपन्तं वै शोकव्याकुलमानसम्}
{उवाच सान्त्वयन्व्यासो युधिष्ठिरमिदं वचः ॥व्यास उवाच}
{}


\twolineshloka
{युधिष्ठिर महाप्राज्ञ सर्वशास्त्रविशारद}
{व्यसनेषु न मुह्यन्ति त्वादृशा भरतर्षभ}


\twolineshloka
{स्वर्गमेष गतः शूरः शत्रून्हत्वा बहून्रणे}
{अबालसदृशं कर्म कृत्वा वै पुरुषोत्तमः}


\threelineshloka
{अनतिक्रमणीयो वै विधिरेष युधिष्ठिर}
{देवदानवगन्धर्वान्मूत्युर्हरति भारत ॥युधिष्ठिर उवाच}
{}


\twolineshloka
{इमे वै पृथिवीपालाः शेरते पृथिवीतले}
{निहताः पृतनामध्ये मृतसंज्ञा महाबलाः}


\twolineshloka
{नागायुतबलाश्चान्ये वायुवेगबलास्तथा}
{त एते निहताः सङ्ख्ये तुल्यरूपा नरैर्नराः}


\twolineshloka
{नैषां पश्यामि हन्तारं प्राणिनां संयुगे क्वचित्}
{विक्रमेणोपसम्पन्नास्तपोबलसमन्विताः}


\twolineshloka
{जेतव्यमिति चान्योन्यं येषां नित्यं हृदि स्थितम्}
{अथ चेमे हताः प्राज्ञाः शेरते विगतायुषः}


\twolineshloka
{मृता इति च शब्दोयं वर्तते च ततोऽर्थवत्}
{इमे मृता महीपालाः प्रायशो भीमविक्रमाः}


\twolineshloka
{निश्चेष्टा निरभीमानाः शूराः शत्रुवशंगताः}
{राजपुत्राश्च संरब्धा वैश्वानरमुखं गताः}


\fourlineindentedshloka
{अत्र मे संशयः प्राप्तः कुतः संज्ञा मृता इति}
{कस्य मृत्युः कुतो मृत्युः केन मृत्युरिमाः प्रजाः}
{हरत्यमरसङ्काश तन्मे ब्रूहि पितामह ॥सञ्जय उवाच}
{}


\threelineshloka
{तं तथा परिपृच्छन्तं कुन्तीपुत्रं युधिष्ठिरम्}
{आश्वासनमिदं वाक्यमुवाच भगवानृषिः ॥व्यास उवाच}
{}


\twolineshloka
{अत्राप्युदाहरन्तीममितिहासं पुरातनम्}
{अकम्पनस्य कथितं नारदेन पुरा नृप}


\twolineshloka
{स चापि राजा राजेन्द्र पुत्रव्यसनमुत्तमम्}
{अप्रसह्यतमं लोके प्राप्तवानिति नः श्रुतम्}


\twolineshloka
{तदहं सम्प्रवक्ष्यामि मृत्योः प्रभवमुत्तमम्}
{ततस्त्वं मोक्ष्यसे दुःखात्स्नेहबन्धनसंश्रयात्}


\twolineshloka
{समस्तपापराशिघ्नं शृणु कीर्तयतो मम}
{धन्यमाख्यानमायुष्यं शोकघ्नं पुष्टिवर्धनम्}


\twolineshloka
{पवित्रमरिसङ्घघ्नं पङ्गलानां च मङ्गलम्}
{यथैव वेदाध्ययनमुपाख्यानमिदं तथा}


\twolineshloka
{श्रवणीयं महाराज प्रातर्नित्यं नृपोत्तमैः}
{पुत्रानायुष्मतो राज्यमीहमानैः श्रियं तथा}


\threelineshloka
{पुरा कृतयुगे तात आसीद्राजा ह्यकम्पनः}
{स शत्रुवशमापन्नो मध्ये सङ्ग्राममूर्धनि}
{`योधयामास बलवान्बद्धश्चासीदकम्पनः'}


\twolineshloka
{तस्य पुत्रो हरिर्नाम नारायणसमो बले}
{श्रीमान्कृतास्त्रो मेधावी युधि शक्रोपमो बली}


\twolineshloka
{`स दृष्ट्वा पितरं युद्धे तदवस्थं महाद्युतिः}
{अचिन्तयित्वा मरणं शत्रुमध्यं ततोऽविशत्}


\twolineshloka
{स शत्रुभिः परिवृतो बहुधा रणमूर्धनि}
{योधयामास तान्सर्वान्सर्वास्त्रकुशलो बली}


\twolineshloka
{वर्षन्बाणसहस्राणि शत्रुष्वमितविक्रमः}
{पदातिरथनागाश्वान्प्रममाथ महाबलः}


\twolineshloka
{स शरैराचितश्चक्रैर्गदाभिर्मुसलैरपि}
{मृद्गन्रथसहस्राणि पदातीन्वाजिवारणान्}


\twolineshloka
{परानीकं विभिद्याजौ मोक्षयित्वा च तं नृपम्}
{पुनरेवाकरोद्युद्धं शत्रुमध्यगतो बली}


\twolineshloka
{ततस्ते रथिनः सर्वे समेत्य पुनराहवे}
{जघ्नुस्तं परिवार्यैकं तोमरैरिव कुञ्जरम्'}


\twolineshloka
{स कर्म दुष्करं कृत्वा सङ्ग्रामे शत्रुतापनः}
{शत्रुभिर्निहतः सङ्ख्ये पृतानायां युधिष्ठिर}


\twolineshloka
{द्विषद्भिर्निहतं दृष्ट्वा प्रियं पुत्रमकम्पनः}
{अमर्षजनितक्रोध आहवात्सहसाऽऽगतः}


\twolineshloka
{स राजा प्रेतकृत्यानि तस्य कृत्वा शुचान्वितः}
{शोचन्नहनि रात्रौ च नालभत्सुखमात्मनः}


\twolineshloka
{तस्य शोकं विदित्वा तु पुत्रव्यसनसम्भवम्}
{आजगामाथ देवर्षिर्नारदोऽस्य समीपतः}


\twolineshloka
{स तु राजा महाभागो दृष्ट्वा देवर्षिसत्तमम्}
{पूजयित्वा यथान्यायं कथामकथयत्तदा}


\threelineshloka
{तस्य सर्वं समाचष्ट यथावृत्तं युधिष्ठिर}
{शत्रुभिर्विजयं सङ्ख्ये पुत्रस्य च वधं तथा ॥अकम्पन उवाच}
{}


\twolineshloka
{मम पुत्रो महावीर्य इन्द्रविष्णुसमद्युतिः}
{शत्रुभिर्बहुभिः सङ्ख्ये पराक्रम्य हतो बली}


\fourlineindentedshloka
{शृणु मे प्राणिनां चापि मृत्युः प्रभवते किल}
{क एष मृत्युर्भगवन्किंवीर्यबलपौरुषः}
{एतदिच्छामि तत्त्वेन श्रोतुं मतिमतां वर ॥व्यास उवाच}
{}


\threelineshloka
{तस्य तद्वचनं श्रुत्वा नारदो वरदः प्रभुः}
{आख्यानमिदमाचष्ट पुत्रशोकापहं महत् ॥नारद उवाच}
{}


\twolineshloka
{शृणु राजन्महाबाहो आख्यानं बहुविस्तरम्}
{यथावृत्तं श्रुतं चैव मयाऽपि वसुधाधिप}


\twolineshloka
{प्रजाः सृष्टा तदा ब्रह्मा आदिसर्गे पितामहः}
{असंहृतं महातेजा दृष्ट्वा जगदिदं प्रभुः}


\twolineshloka
{तस्य चिन्ता समुत्पन्ना संहारं प्रति पार्थिव}
{चिन्तयन्न ह्यसौ देव संहारं वसुधाधिप}


\twolineshloka
{तस्य रोषान्महाराज मुखेभ्योऽग्निरजायत}
{तेन सर्वा दिशो व्याप्ताः सान्तर्देशा दिधक्षता}


\twolineshloka
{ततो दिवं भुवं चैव ज्वालामालासमाकुलम्}
{चारचरं जगत्सर्वं ददाह भगवान्प्रभुः}


\threelineshloka
{ततो हतानि भूतानि चराणि स्थावराणि च}
{महता क्रोधवेगेन त्रासयन्निव वीर्यवान्}
{}


\twolineshloka
{ततो रुद्रो जटी स्याणुर्निशाचरपतिर्हरः}
{जगाम शरणं देवं ब्रह्माणं परमेष्ठिनम्}


\twolineshloka
{तस्मिन्नापतिते स्थाणौ प्रजानां हितकाम्यया}
{अब्रवीत्परमो देवो ज्वलन्निव महामुनिः}


\twolineshloka
{किं कर्म कामं कामार्ह कामाज्जातोऽसि पुत्रक}
{करिष्यामि प्रियं सर्वं ब्रूहि स्थाणो यदिच्छसि}


\chapter{अध्यायः ५३}
\twolineshloka
{स्थाणुरुवाच}
{}


\twolineshloka
{प्रजासर्गनिमित्तं हि कृतो यत्नस्त्वया विभो}
{त्वया सृष्टाश्च वृद्धाश्च भूतग्रामाः पृथग्विधाः}


\threelineshloka
{तास्तवेह पुनः क्रोधात्प्रजा नश्यन्ति सर्वशः}
{ता दृष्ट्वा मम कारुण्यं प्रसीद भगवन्प्रभो ॥ब्रह्मोवाच}
{}


\twolineshloka
{संहर्तुं न च मे काम एतदेवं भवेदिति}
{पृथिव्या हितकामं तु ततो मां मन्युराविशत्}


\twolineshloka
{इयं सन्ना तदा देवी भारार्ता समचूचुदत्}
{संहारार्थं महादेव भारेणाभिहता सती}


\threelineshloka
{ततोऽहं नाधिगच्छामि बुद्ध्या बहु विचारयन्}
{संहारमप्रमेयस्य ततो मां मन्युराविशत् ॥रुद्र उवाच}
{}


\twolineshloka
{संहारार्थं प्रसीदस्व मारुषो वसुधाधिप}
{मा प्रजाः स्थावराश्चैव जङ्गमाश्च व्यनीनशः}


\twolineshloka
{तव प्रसादाद्भगवन्निदं वर्तेत्त्रिधा जगत्}
{अनागतमतीतं च यच्च सम्प्रति वर्तते}


\twolineshloka
{भगवन्क्रोधसन्दीप्तः क्रोधादग्निमवासृजत्}
{स दहत्यश्मकूटानि द्रुमांश्च सरितस्तथा}


\twolineshloka
{पल्वलानि च सर्वाणि सर्वै चैव तृणोलपाः}
{स्थावरं जङ्गमं चैव निःशेषं कुरुते जगत्}


\twolineshloka
{तदेतद्भस्मसाद्भूतं जगत्स्थावरजङ्गमम्}
{प्रसीद भगवन्स त्वं रोषो न स्याद्वरो मम}


\twolineshloka
{सर्वे हि सृष्टा नश्यन्ति तव देव कथंचन}
{तस्मान्निवर्ततां तेजस्त्वय्येवेदं प्रलीयताम्}


\twolineshloka
{उपायमन्यं सम्पश्य प्रजानां हितकाम्यया}
{यथेमे प्राणिनः सर्वे निर्वर्तेरंस्तथा कुरु}


\threelineshloka
{अभावं नेह गच्छेयुरुत्सन्नजननाः प्रजाः}
{भवता हि नियुक्तोऽहं प्रानां पालने विभा}
{दया ते न समुत्पन्ना प्रजासु विबुधेश्वर}


\threelineshloka
{मा विनश्येज्जगन्नाथ जगत्स्थावरजङ्गमम्}
{प्रसादाभिमुखं देवं तस्मादेवं ब्रवीम्यहम् ॥नारद उवाच}
{}


\twolineshloka
{श्रुत्वा हि वचनं देवः प्रजानां हितकारणे}
{तेजः सन्धारयामास पुनरेवान्तरात्मनि}


\twolineshloka
{ततोऽग्निमुपसंहृत्य भगवाँल्लोकसत्कृतः}
{प्रवृत्तं च निवृत्तं च कल्पयामास वै प्रभुः}


\twolineshloka
{उपसंहरतस्तस्य तमग्निं रोषजं तथा}
{प्रादुर्बभूव विश्वेभ्यो गोभ्योनारी महात्मनः}


\twolineshloka
{कृष्णरक्ता तथा पिङ्गा रक्तजिह्वास्यलोचना}
{कुण्डलाभ्यां च राजेन्द्र तप्ताभ्यां तप्तभूषणा}


\twolineshloka
{सा निःसृत्य तथा खेभ्यो दक्षिणां दिशमाश्रिता}
{स्मयमाना च साऽवेक्ष्य देवौ विश्वेश्वरावुभौ}


\threelineshloka
{`तां तु तत्र तदा देवीं ब्रह्मा लोकपितामहः}
{उक्तवान्मधुरं वाक्यं सान्त्वयित्वा पुनः पुनः'}
{मृत्यो इति महीपाल जहि चेमाः प्रजा इति}


\twolineshloka
{त्वं हि संहारबुद्ध्याऽथ प्रादुर्भूता रुषो मम}
{तस्मात्संहर सर्वास्त्वं प्रजाः सजडपण्डिताः}


\threelineshloka
{अविशेषेण चैव त्वं प्रजाः संहर भामिनि}
{मम त्वं हि नियोगेन ततः श्रेयो ह्यवाप्स्यासि ॥नारद उवाच}
{}


\twolineshloka
{एवमुक्ता तु सा तेन मृत्युः कमललोचना}
{प्रारुदद्भृशसंविग्ना प्रापतन्नश्रुबिन्दवः}


\twolineshloka
{पाणिभ्यां प्रतिजग्राह तान्यश्रूणि पितामहः}
{सर्वभूतहितार्थाय तां चाप्यनुनयत्तदा}


\chapter{अध्यायः ५४}
\twolineshloka
{नारद उवाच}
{}


\threelineshloka
{विनीय दुःकमबला आत्मन्येव प्रजापतिम्}
{उवाच प्राञ्जलिर्भूत्वा लतेवावर्जिता पुनः ॥मृत्युरुवाच}
{}


\twolineshloka
{त्वया सृष्टा कथं नारी ईदृशी वदतां वर}
{क्रूरं कर्माहितं कुर्यां मूढेव परिजानती}


\twolineshloka
{बिभेम्यहमधर्माद्धि प्रसीद भगवन्प्रभो}
{प्रियान्पुत्रान्वयस्यांश्च भातृन्मातॄः पितॄन्पतीन्}


\twolineshloka
{अपध्यास्यन्ति ये देव मृतांस्तेभ्यो बिभेम्यहम्}
{कृपणानां हि रुदतां ये पतन्त्यश्रुबिन्दवः}


\twolineshloka
{तेभ्योऽहं भगवन्भीता शरणं त्वाऽहमागता}
{यमस्य भवनं देव गच्छेयं न सुरोत्तम}


\twolineshloka
{कायेन विनयोपेता मूर्ध्नोदग्रनखेन च}
{एतदिच्छाम्यहं कामं त्वत्तो लोकपितामह}


\twolineshloka
{इच्छेयं त्वत्प्रसादाद्धि तपस्तप्तुं प्रजेश्वर}
{प्रदिशेमं वरं देव त्वं मह्यं भगवन्प्रभो}


\twolineshloka
{त्वया ह्युक्ता गमिष्यामि धेनुकाश्रममुत्तमम्}
{तत्र तप्स्ये तपस्तीत्रं तवैवाराधने रता}


\threelineshloka
{न हि शक्ष्यामि देवेश प्राणान्प्राणभृतां प्रियान्}
{हर्तुं विलपमानानामधर्मादभिरक्ष माम् ॥ब्रह्मोवाच}
{}


\twolineshloka
{मृत्यो सङ्कल्पिताऽसि त्वं प्रजासंहारहेतुना}
{गच्छ संहर सर्वास्त्वं प्रजा मा ते विचारणा}


\threelineshloka
{भविता त्वेतदेवं हि नैतज्जात्वन्यथा भवेत्}
{भव त्वनिन्दिता लोके कुरुष्व वचनं मम ॥नारद उवाच}
{}


\twolineshloka
{एवमुक्ता भगवता प्राञ्जलिश्चाप्यवाङ्मुखी}
{संहारे नाकरोद्बुद्धिं प्रजानां हितकाम्यया}


\twolineshloka
{तूष्णीमासीत्तदा देवः प्रजानामीश्वरेश्वरः}
{प्रसादं चागमत्क्षिप्रमात्मनैव प्रजापतिः}


\twolineshloka
{स्मयमानश्च देवेशो लोकान्सर्वानवेक्ष्य च}
{लोकास्त्वासन्यथापूर्वं दृष्टास्तेनापमन्युना}


\twolineshloka
{निवृत्तरोषे तस्मिंस्तु भगवत्यपराजिते}
{सा कन्याऽपि जगामाऽथ समीपात्तस्य धीमतः}


\twolineshloka
{अपमृत्याप्रतिश्रुत्य प्रजासंहरणं तदा}
{त्वरमाणा च राजेन्द्र मृत्युर्धेनुकमभ्यगात्}


\twolineshloka
{सा तत्र परमं तीव्रं चचार व्रतमुत्तमम्}
{सा तद ह्येकपादेन तस्थौ पद्मानि षोडश}


\twolineshloka
{पञ्च चाब्दानि कारुण्यात्प्रजानां तु हितैषिणी}
{इन्द्रियाणीन्द्रियार्थेभ्यः प्रियेभ्यः सन्निवर्त्य सा}


\twolineshloka
{ततस्त्वेकेन पादेन पुनरन्यानि सप्त वै}
{तस्थौ पद्मानि षट् चैव सप्त चैकं च पार्थिवा}


\twolineshloka
{ततः पद्मायुतं तात मृगैः सह चचार सा}
{पुनर्गत्वा ततो नन्दां पुण्यां शीतामलोदकाम्}


\twolineshloka
{अप्सु वर्षसहस्राणि सप्त चैकं च साऽनयत्}
{धारयित्वा तु नियमं नन्दायां वीतकल्मषा}


\twolineshloka
{सा पूर्वं कौशिकीं पुण्यां जगाम नियमैधिता}
{तत्र वायुजलाहारा चचार नियमं पुनः}


\twolineshloka
{सप्तगङ्गासु सा पुण्या कन्या वेतसकेषु च}
{तपोविशेषैर्बहुभिः कर्षयद्देहमात्मनः}


\twolineshloka
{ततो गत्वा तु सा गङ्गां महामेरुं च केवलम्}
{तस्थौ चाश्मेव निश्चेष्टा प्राणायामपरायणा}


\twolineshloka
{पुनर्हिमवतो मूर्ध्नि यत्र देवाः पुराऽयजन्}
{तत्राङ्गुष्ठेन सा तस्थौ निखर्वंल परमा शुभा}


\twolineshloka
{पुष्करेष्वथ गोकर्णे नैमिषे मलये तथा}
{अकर्शयत्स्वकं देहं नियमैर्मानसप्रियैः}


\twolineshloka
{अनन्यदेवता नित्यं दृढभक्ता पितामहे}
{तस्थौ पितामहं चैव तोषयामास धर्मतः}


\twolineshloka
{ततस्तामब्रवीत्प्रीतो लोकानां प्रभवोऽव्ययः}
{सौम्येन मनसा राजन्प्रीतः प्रीतमनास्तदा}


\twolineshloka
{मृत्यो किमिदमत्यन्तं तपांसि चरसीति ह}
{ततोऽब्रवीत्पुनर्मृत्युर्भगवन्तं पितामहम्}


\twolineshloka
{नाऽहं हन्यां प्रजा देव स्वस्थाश्चाक्रोशतीस्तथा}
{एतदिच्छामि सर्वेश त्वत्तो वरमहं प्रभो}


\twolineshloka
{अधर्मभयभीताऽस्मि ततोऽहं तप आस्थिता}
{भीतायास्तु महाभाग प्रयच्छाभयमव्यय}


\twolineshloka
{आर्ता चानागसी नारी याचामि भव मे गतिः}
{तामब्रवीत्ततो देवो भूतभव्यभविष्यवित्}


\twolineshloka
{अधर्मो नास्ति ते मृत्यो संहरन्त्या इमाः प्रजाः}
{मया चोक्तं मृषा भद्रे भविता न कथञ्चन}


\twolineshloka
{तस्मात्संहर कल्याणि प्रजाः सर्वाश्चतुर्विधाः}
{धर्मः सनातनश्च त्वां सर्वथा पावयिष्यति}


\twolineshloka
{लोकपालो यमश्चैव सहाया व्याधयश्च ते}
{अहं च विबुधाश्चैव पुनर्दास्याम ते वरम्}


\fourlineindentedshloka
{यथा त्वमेनसा मुक्ता विरजाः ख्यातिमेष्यसि}
{नारद उवाच}
{सैवमुक्ता महाराज कृताञ्जलिरिदं विभुम्}
{}


\twolineshloka
{पुनरेवाब्रवीद्वाक्यं प्रसाद्य शिरसा तदा}
{यद्येवमेतत्कर्तव्यं धर्मतो नास्त्यतो भयम्}


\twolineshloka
{तवाज्ञा मूर्ध्नि मे न्यस्ता यत्ते वक्ष्यामि तच्छृणु}
{लोभः क्रोधोऽभ्यसूयेर्ष्या द्रोहो मोहश्च देहिनां}


\fourlineindentedshloka
{अहूश्चान्योन्यपरुषा देहं भिन्द्युः पृथिग्वधाः}
{ब्रह्मोवाच}
{तथा भविष्यते मृत्यो साधु संहर भोः प्रजाः}
{अधर्मस्ते न भविता नापध्यास्याम्यहं शुभे}


\twolineshloka
{यान्यश्रुबिन्दूनि करे ममासं--स्ते व्याधयः प्राणिनामात्मजाताः}
{ते मारयिष्यन्ति नरान्गतासू--न्नाधर्मस्ते भविता मा स्म भैषीः}


\twolineshloka
{नाधर्मस्ते भविता प्राणिनां वैत्वं वै धर्मस्त्वं हि धर्मस्य चेशा}
{धर्म्या भूत्वा धर्मनित्या धरित्रीतस्मात्प्राणान्सर्वथेमान्नियच्छ}


\threelineshloka
{`प्राप्ता तथा नो भविता कदाचि--देवं मयाद्यैव नियोज्यसे त्वम्'}
{सर्वेषां वै प्राणिनां कामरोषौसन्त्यज्य त्वं संहरस्वेह जीवान्}
{येषां धर्मास्ते भविष्यन्त्यनन्तामिथ्यावृत्तं मारयिष्यत्यधर्मः}


\threelineshloka
{तेनात्मानं पावयस्वात्मना त्वंपापेऽऽत्मानं मज्जयिष्यन्त्यसत्यात्}
{तस्मात्कामं रोषमप्यागतं त्वंसन्त्यज्यान्तः संहरस्वेति जीवान् ॥नारद उवाच}
{}


\twolineshloka
{सा वै भीमा मृत्युसंज्ञोपदेशा--च्छापाद्भीता बाढमित्यब्रवीत्तम्}
{सा च प्राणं प्राणिनामन्तकालेकामक्रोधौ त्यज्य हरत्यसक्ता}


\twolineshloka
{मृत्युस्त्वेषां व्याधयस्तत्प्रसूताव्याधी रोगो रुज्यते येन जन्तुः}
{सर्वेषां च प्राणिनां प्रायणान्तेतस्माच्छोकं माकृथा निष्फलं त्वम्}


\twolineshloka
{सर्वे देवाः प्राणिभिः प्रायणान्तेगत्वा वृत्ताः सन्निवृत्तास्तथैव}
{एवं सर्वे प्राणिनस्तत्र गत्वावृत्ता देवा मर्त्यवद्राजसिंह}


\twolineshloka
{वायुर्भीमो भीमनादो महौजाभेत्ता देहान्प्राणिनां सर्वगोऽसौ}
{नो वा वृत्तिं कदाचि--त्प्राप्नोत्युग्रोऽनन्ततेजोविशिष्टः}


\twolineshloka
{सर्वे देवा मर्त्यसंज्ञाविशिष्टा--स्तस्मात्पुत्रं माशुचो राजसिंह}
{स्वर्गं प्राप्तो मोदते ते तनूजोनित्यं रम्यान्वीरलोकानवाप्य}


\twolineshloka
{त्यक्त्वा दुःखं सङ्गतः पुण्यकृद्भि--रेषा मृत्युर्देवदिष्टा प्रजानाम्}
{प्राप्ते काले संहरन्ती यथाव--त्स्वयं कृता प्राणहरा प्रजानाम्}


\fourlineindentedshloka
{आत्मानं वै प्राणिनो घ्नन्ति सर्वेनैतान्मृत्युर्दण्डपाणिर्हिनस्ति}
{तस्मान्मृतान्नानुशोचन्ति धीरामृत्युं ज्ञात्वा निश्चयं ब्रह्मसृष्टम्}
{इत्थं सृष्टिं देवक्लृप्तां विदित्वापुत्रान्नष्टाच्छोकमाशु त्यजस्व ॥द्वैपायन उवाच}
{}


\twolineshloka
{एतच्छ्रुत्वार्थवद्वाक्यं नारदेन प्रकाशितम्}
{उवाचाकम्पनो राजा सखायं नारदं तथा}


\twolineshloka
{व्यपेतशोकः प्रीतोऽस्मि भगवन्नृषिसत्तम}
{श्रुत्वेतिहासं त्वत्तस्तु कृतार्थोऽस्म्यभिवादये}


\twolineshloka
{तथोक्तो नारदस्तेन राज्ञा देवर्षिसत्तमः}
{जगाम नन्दनं शीघ्रमशोकवनमात्मनः}


\twolineshloka
{पुण्यं यशस्यं स्वर्ग्यं च धन्यमायुष्यमेव च}
{अस्येतिहासस्य सदा श्रवणं श्रावणं तथा}


\threelineshloka
{एतदर्थपदं श्रुत्वा तदा राजा युधिष्ठिर}
{क्षत्रधर्मं च विज्ञाय शूराणां च परां गतिम्}
{सम्प्राप्तोऽसौ महावीर्यः स्वर्गलोकं महारथः}


\twolineshloka
{अभिमन्युः परान्हत्वा प्रमुखे सर्वधन्विनाम्}
{युध्यमानो महेष्वासो हतः सोऽभिमुखो रणे}


\twolineshloka
{असिना गदया शक्त्या धनुषा च महारथः}
{विरजाः सोमसूनुः स पुनस्तत्र प्रलीयते}


\twolineshloka
{तस्मात्परां धृतिं कृत्वा भ्रातृभिः सह पाण्डव}
{अप्रमत्तः सुसन्नद्धः शीघ्रं योद्धुमुपाक्रम}


\chapter{अध्यायः ५५}
\twolineshloka
{सञ्जय उवाच}
{}


\threelineshloka
{श्रुत्वा मृत्युसमुत्पत्तिं कर्माण्यनुपमानि च}
{धर्मराजः पुनर्वाक्यं प्रसाद्यैनमथाब्रवीत् ॥युधिष्ठिर उवाच}
{}


\twolineshloka
{गुरवः पुण्यकर्माणः शक्रप्रतिमविक्रमाः}
{पूर्वं राजर्षयो ब्रह्मन्कियन्तो मृत्युना हताः}


\twolineshloka
{भूय एव तु मां तथ्यैर्वचोभिरभिबृंहय}
{राजर्षीणां पुराणानां समाश्वासय कर्मभिः}


\threelineshloka
{कियन्त्यो दक्षिणा दत्ताः काश्च दत्ता महात्मभिः}
{राजर्षिभिः पुण्यकृद्भिस्तद्भवान्प्रब्रवीतु मे ॥`व्यास उवाच}
{}


\twolineshloka
{अन्येपि यज्वनां लोका अन्ये चापि तपस्विनाम्}
{क्षमावतां च त्रींल्लोकाञ्शूरा गच्छन्ति भारत}


\twolineshloka
{क्षत्रियस्य तुं सङ्ग्रामे शत्रून्हत्वा हतस्य वा}
{फलमत्यन्तमित्याहुर्धर्मशास्त्रविदो जनाः}


\twolineshloka
{वेदविद्याव्रतस्नाता यज्वानः पुत्रिणश्च ये}
{तेभ्यः परार्थ्या यज्वानो यज्वभ्यश्च तनुत्यजः}


\twolineshloka
{स वीरो यज्वनो नित्यं क्षत्रियानार्जुनिर्गतः}
{लोकान्पुण्यतमानिष्टानिति विद्धि विशाम्पते}


\threelineshloka
{सर्वेषां नृपसिंहानां शृणु यज्ञान्नृपोत्तम}
{तानतिक्रम्य गच्छन्ति स्वर्गकामास्तनुत्यजः ॥युधिष्ठिर उवाच}
{}


\threelineshloka
{सर्वेषां यज्वनां यज्ञं श्रोतुमिच्छामि सुव्रत}
{दक्षिणाश्चानुरूपेभ्यो दत्ता राजर्षिसत्तमैः' ॥व्यास उवाच}
{}


\twolineshloka
{शैब्यस्य नृपतेः पुनः सृञ्जयो नाम नामतः}
{सखायौ तस्य चेवोभौ ऋषी पर्वतनारदौ}


\twolineshloka
{तौ कदाचिद्गृहं तस्य प्रविष्टौ तद्दिदृक्षया}
{विधिवच्चार्चितौ तेन प्रीतौ तत्रोषतुः सुखम्}


\twolineshloka
{तं कदाचित्सुखासीनं ताभ्यां सह शुचिस्मिता}
{दुहिताऽभ्यागमत्कन्या सृञ्जयं वरवर्णिनी}


\twolineshloka
{तयाऽभिवादितः कन्यामभ्यनन्दद्यथाविधि}
{तत्सलिङ्गाभिराशीर्भिरिष्टाभिरभितः स्थिताम्}


\twolineshloka
{तां निरीक्ष्याब्रवीद्वाक्यं पर्वतः प्रहसन्निव}
{कस्येयं चञ्चलापाङ्गी सर्वलक्षणसम्मता}


\threelineshloka
{उषा स्विद्भाः स्विदर्कस्य ज्वलनस्य शिखापि वा}
{श्रीर्द्द्रिः कीर्तिंर्धृतिः षुष्टिः सिद्धिश्चन्द्रमसः प्रभा ॥सञ्जय उवाच}
{}


\twolineshloka
{एवं ब्रुवाणं देवर्षिं नृपतिः सृञ्जयोऽब्रवीत्}
{ममेयं भगवन्कन्या वर्या वरमभीप्सति}


\twolineshloka
{नारदस्त्वब्रवीदेनं देहि मह्यमिमां नृप}
{भार्यार्थं सुमहच्छ्रेयः प्राप्तुं चेदिच्छसे नृप}


\fourlineindentedshloka
{ददानीत्येवं संहृष्टः सृञ्जयः प्राह नारदम्}
{व्यास उवाच}
{`एवमुक्ते नृपतिना क्रोधपर्याकुलेक्षणः'}
{पर्वतस्तु सुसङ्क्रुद्धो नारदं वाक्यमब्रवीत्}


\twolineshloka
{पूर्वं ममैव मनसा विद्धि भार्यां वृतामृषे}
{यस्मादपाचरस्तस्मात्स्वर्गं न गच्छसि}


\twolineshloka
{एवमुक्तो नारदस्तं प्रत्युवाचोत्तरं वचः}
{मनोवाग्बुद्धिसम्भाषा सत्यं तोयमथाग्नयः}


\twolineshloka
{पाणिग्रहममन्त्राश्च प्रथितं दारलक्षणम्}
{न त्वेषां निश्चिता निष्ठा त्वया सा मनसा स्मृता}


\threelineshloka
{`एवं विद्वांस्तु मां यस्मादनुव्याहृतवानसि'}
{तस्मात्त्वमपि न स्वर्गं गमिष्यसि मया विना}
{अन्योन्यमेवं शप्त्वा वै तस्थतुस्तत्र तौ तदा}


\twolineshloka
{अथ सोऽपि नृपो विप्रान्पानाच्छादनभोजनैः}
{पुत्रकामः परं शक्त्या यत्नाच्चोपाचरच्छुचिः}


\twolineshloka
{तस्य प्रसन्ना विप्रेन्द्राः कदाचित्पुत्रदर्शिनः}
{तपःस्वाध्यायनिरता वेदवेदाङ्गपारगाः}


\twolineshloka
{सहिता नारदं प्राहुर्देह्यस्मै पुत्रमीप्सितम्}
{तथेत्युक्त्वा द्विजैरुक्तः सृञ्जयं नारदोऽब्रवीत्}


\twolineshloka
{तुभ्यं प्रसन्ना राजर्षे पुत्रमीप्सन्ति ब्राह्मणाः}
{वरं वृणीष्व भद्रं ते यादृशं पुत्रमीप्सितम्}


\twolineshloka
{तथोक्तः प्राञ्जली राजा पुत्रं वव्रे गुणान्वितम्}
{यशस्विनं कीर्तिमन्तं तेजस्विनमरिन्दमम्}


\threelineshloka
{यस्य मूत्रं पुरीषं च क्लेदः स्वेदश्च काञ्चनम्}
{`सर्वं भवेत्प्रसादाद्वै तादृशं तनयं वृणे ॥व्यास उवाच}
{}


\twolineshloka
{तथा भविष्यतीत्युक्ते जज्ञे तस्योप्सितः सुतः}
{काञ्चनस्याकरः श्रीमान्प्रसादाद्वै सुकाङ्क्षितः}


\threelineshloka
{रुदितस्य च नेत्राभ्यामपतत्तस्य नेत्रजम्}
{मूत्रं पुरीषं स्वेदश्च सर्वं भवति काञ्चनम्'}
{सुवर्णष्ठीविरित्येव तस्य नामाभवत्कृतम्}


\twolineshloka
{तस्मिन्वरप्रदानेन वर्धयत्यमितं धनम्}
{कारयामास नृपतिः सौवर्णं सर्वमीप्सितम्}


\twolineshloka
{गृहप्राकारदुर्गाणि ब्राह्मणावसथान्यपि}
{शय्यासनानि यानानि स्थालीपिठरभाजनम्}


\twolineshloka
{तस्य राज्ञोऽपि यद्वेश्म बाह्याश्चोपस्कराश्च ये}
{सर्वं तत्काञ्चनमयं कालेन परिवर्धितम्}


\twolineshloka
{अथ दस्युगणाः श्रुत्वा दृष्ट्वा चैनं तथाविधम्}
{सम्भूय तस्य नृपतेः समारब्धाश्चिकीर्षितुम्}


\twolineshloka
{केचित्तत्राब्रुवन्राज्ञः पुत्रं गृह्णीम वै स्वयम्}
{सोऽस्याकरः काञ्चनस्य तस्य यत्नं चरामहे}


\twolineshloka
{ततस्ते दस्यवो लुब्धाः प्रविश्य नृपतेर्गृहम्}
{राजपुत्रं तथाऽऽजह्रुः सुवर्णष्ठीविनं बलात्}


\twolineshloka
{गृह्यैनमनुपायज्ञा नीत्वाऽरण्यमचेतसः}
{हत्वा विशस्य चापश्यँ ल्लुब्धा वसु न किञ्चन}


\twolineshloka
{तस्य प्राणैर्विमुक्तस्य नष्टं तत्काञ्चनं वरम्}
{दस्यवश्च तदाऽन्योन्यं जघ्नुर्मूर्खा विचेतसः}


\twolineshloka
{हत्वा परस्परं नष्टाः कुमारं चाद्भुतं भुवि}
{असम्भाव्यं गता घोरं नरकं दुष्टकारिणः}


\twolineshloka
{तं दृष्ट्वा निहतं पुत्रं वरदत्तं महातपाः}
{विललाप सुदुःखार्तो बहुधा करुणं नृपः}


\twolineshloka
{विलपन्तं निशम्याथ पुत्रशोकहतं नृपम्}
{प्रत्यदृश्यत देवर्षिर्नारदस्तस्य सन्निधौ}


\threelineshloka
{उवाच चैनं दुःखार्तं विलपन्तमचेतसम्}
{सृञ्जयं नारदो यद्यत्तन्निबोध युधिष्ठिर ॥नारद उवाच}
{}


\twolineshloka
{`त्यज शोकं महाराज वैक्लब्यं त्यज बुद्धिमन्}
{न मृतः शोचतो जीवेन्मुह्यतो वा नराधिप}


\threelineshloka
{त्यज शोकं नृपश्रेष्ठ न शोचन्ति भवद्विधाः}
{वीरो भवान्महाराज ज्ञानवृद्धोऽपि मे मतः'}
{कामानामवितृप्तस्त्वं सृञ्जयेह मरिष्यसि}


\twolineshloka
{यस्य चैते वयं गेहे उषिता ब्रह्मवादिनः}
{आविक्षितं मरुत्तं च मृतं सृञ्जय शुश्रुम}


\threelineshloka
{संवर्तो याजयामास स्पर्धया वै बृहस्पतेः}
{यस्मै राजर्षये प्रादाद्धनं स भगवान्प्रभुः}
{हैमं हिमवतः पादं यियक्षोर्विविधैः सवैः}


\threelineshloka
{यस्य सेन्द्रामरगणा बृहस्पतिपुरोगमाः}
{देवा विश्वसृजः सर्वे यजनान्ते समासते}
{यज्ञवाटस्य सौवर्णाः सर्वे चासन्परिच्छदाः}


\threelineshloka
{यस्य सर्वं तदा ह्यन्नं मनोभिप्रायगं शुचि}
{कामतो बुभुजुर्विप्राः सर्वे चान्नार्थिनो द्विजाः}
{पयो दधि घृतं क्षौद्रं भक्ष्यं भोज्यं च शोभनम्}


\twolineshloka
{यस्य यज्ञेषु सर्वेषु वासांस्याभरणानि च}
{ईप्सितान्युपतिष्ठन्ते प्रहृष्टान्वेदपारगान्}


\twolineshloka
{मरुतः परिवेष्टारो मरुत्तस्याभवन्गृहे}
{आविक्षितस्य राजर्षेर्विश्वेदेवाः सभासदः}


\twolineshloka
{यस्य वीर्यवतो राज्ञः सुवृष्ट्या सस्यसम्पदः}
{हविर्भिस्तर्पिता येन सम्यक्क्लृप्तैर्दिवौकसः}


\twolineshloka
{ऋषीणां च पितॄणां च देवानां सुखजीविनम्}
{ब्रह्मचर्यश्रुतिमुखैः सर्वैर्दानैश्च सर्वदा}


\twolineshloka
{शयनासनपानानि स्वर्णराशीश्च दुस्त्यजाः}
{तत्सर्वममितं वित्तं दत्तं विप्रेभ्य इच्छया}


\twolineshloka
{सोनुध्यातस्तु शक्रेण प्रजाः कृत्वा निरामयाः}
{श्रद्दधानो जिताँल्लोकान्गतः पुण्यदुहोऽक्षयान्}


\twolineshloka
{सप्रजः सनृपामात्यः सदारापत्यबान्धवः}
{यौवनेन सहस्राब्दं मरुत्तो राज्यमन्वशात्}


\threelineshloka
{स चेन्ममार सृञ्जय चतुर्भद्रतरस्त्वया}
{पुत्रात्पुण्यतरस्तुभ्यं मा पुत्रमनुतप्यथाः}
{अयज्वानमदक्षिण्यमभि श्वैत्येत्युदाहरत्}


\chapter{अध्यायः ५६}
\twolineshloka
{नारद उवाच}
{}


\twolineshloka
{सुहोत्रं नाम राजानं मृतं सृञ्जय शुश्रुम}
{एकवीरमशक्यं तममरैरभिवीक्षितुम्}


\twolineshloka
{यः प्राप्य राज्यं धर्मेण ऋत्विङ्मन्त्रिपुरोहितान्}
{सम्मान्य चात्मनः श्रेयः पृष्ट्वा तेषां मते स्थितः}


\twolineshloka
{प्रजानां पालनं धर्मो दानमिज्या द्विषज्जयः}
{एतत्सुहोत्रो विज्ञाय धर्म्यमैच्छद्धनागमम्}


\twolineshloka
{धर्मेणाराधयन्देवान्बाणैः शत्रूञ्जयंस्तथा}
{सर्वाण्यपि च भूतानि स्वगुणैरन्वरञ्जयत्}


\twolineshloka
{यो भुक्त्वेमां वसुमतीं म्लेच्छाटविकवर्जिताम्}
{यस्मै ववर्ष पर्जन्यो हिरण्यं परिवत्सरान्}


\twolineshloka
{हैरण्यास्तत्र वाहिन्यः स्वैरिण्यो व्यवहन्पुरा}
{ग्राहान्कर्कटकांश्चैव मत्स्यांश्च विविधान्बहून्}


\twolineshloka
{कामान्वर्षति पर्जन्यो रूप्याणि विविधानि च}
{सौवर्णान्यप्रमेयाणि वाप्यश्च क्रोशसम्मिताः}


\twolineshloka
{सहस्रं वामनान्कुब्जान्नक्रान्मकरकच्छपान्}
{सौवर्णान्विहितान्दृष्ट्वा ततोऽस्मयत वै तदा}


\twolineshloka
{तत्सुवर्णमपर्यन्तं राजर्षिः कुरुजाङ्गले}
{ईजानो वितते यज्ञे ब्राह्मणेभ्यो ह्यमन्यत}


\twolineshloka
{सोऽश्वमेधसहस्रेण राजसूयशतेन च}
{पुण्यैः क्षत्रिययज्ञैश्च प्रभूतवरदक्षिणैः}


\twolineshloka
{काम्यनैमित्तिकाजस्रैरिष्ट्वेष्टां गतिमाप्तवान्}
{स चेन्ममार सृञ्जय चतुर्भद्रतरस्त्वया}


\twolineshloka
{पुत्रात्पुण्यतरस्तुभ्यं मा पुत्रमनुतप्यथाः}
{अयज्वानमदक्षिण्यमभि श्वैत्येत्युदाहरत्}


\chapter{अध्यायः ५७}
\twolineshloka
{नारद उवाच}
{}


\twolineshloka
{अङ्गं च पौरवं वीरं मृतं सृञ्जय शुश्रुम}
{सहस्रं यः सहस्राणां श्वेतानश्वानवासृजत्}


\twolineshloka
{तस्याश्वमेधे राजर्षेर्देशाद्देशात्समीयुषाम्}
{शिक्षाक्षरविधिज्ञानां नासीत्सङ्ख्या विपश्चिताम्}


\twolineshloka
{वेदविद्याव्रतस्नाता वदान्याः प्रियदर्शनाः}
{सुभिक्षाच्छादनगृहाः सुशय्यासनभोजनाः}


\twolineshloka
{नटनर्तकगन्धर्वैः पूर्णकैर्वर्धमानकैः}
{नित्योद्योगैश्च क्रीडद्भिस्तत्र स्म परिहर्षिताः}


\twolineshloka
{यज्ञे यज्ञे यथाकालं दक्षिणाः सोऽत्यकालयत्}
{द्विपान्दशसहस्राख्यानदात्काञ्चनसंवृतान्}


\twolineshloka
{यः सहस्रं सहस्राणि कन्या हेमविभूषिताः ॥धूर्युजाश्वगजारूढाः सगृहक्षेत्रगोशताः}
{}


\twolineshloka
{शतं शतसहस्राणि स्वर्णमालानथर्षभान् ॥गवां सहस्रानुचरान्दक्षिणामत्यकालयत्}
{}


\twolineshloka
{हेमशृङ्ग्यो रौप्यखुराः सवत्साः कांस्यदोहनाः ॥दासीदासखरोष्ट्रांश्च प्रादादाजाविकं बहु}
{}


\twolineshloka
{रत्नानां विविधानां च विविधांश्चान्नपर्वतान् ॥तस्मिन्संवितते यज्ञे दक्षिणामत्यकालयत्}
{}


\twolineshloka
{तत्रास्य गाथा गायन्ति ये पुराणविदो जनाः ॥`अङ्गस्य यजमानस्य अपि विष्णुपदे वयम्}
{}


\twolineshloka
{अमाद्यदिन्द्रुः सोमेन दक्षिणाभिर्द्विजातयः}
{देवमानुषगन्धर्वांश्चात्यरोचन्त दक्षिणाः'}


\twolineshloka
{अङ्गस्य यजमानस्य स्वधर्माधिगताः शुभाः}
{गुणोत्तरास्तु क्रतवस्तस्यासन्सार्वकामिकाः}


\threelineshloka
{स चेन्ममार सृञ्जय चतुर्भद्रतस्त्वया}
{पुत्रात्पुण्यतरस्तुभ्यं मा पुत्रमनुतप्यथाः}
{अयज्वानमदक्षिण्यमभि श्वैत्येत्युदाहरत्}


\chapter{अध्यायः ५८}
\twolineshloka
{नारद उवाच}
{}


\twolineshloka
{शिबिमौशीनरं चापि मृतं सृञ्जय शुश्रुम}
{य इमां पृथिवीं सर्वां चर्मवत्पर्यवेष्टयत्}


\twolineshloka
{साद्रिद्वीपार्णववनां रथघोषेण नादयन्}
{स शिबिर्यष्टुमन्विच्छन्मुख्यः सर्वसपत्नजित्}


\twolineshloka
{तेन यज्ञैर्बहुविधैरिष्टं पर्याप्तदक्षिणैः}
{अविहिंस्य जनानन्यानवाप वसु पुष्कलम्}


\twolineshloka
{सर्वमूर्धाभिषिक्तानां सम्मतः सोऽभवद्युधि}
{अजयच्चाश्वमेधैर्यो विजित्य पृथिवीमिमाम्}


\twolineshloka
{निरर्गलैर्बहुफलैर्निष्ककोटिसहस्रदः}
{`बिभर्ति दक्षिणा यस्य गङ््गायाः स्रोत आवृणोत्'}


\twolineshloka
{हस्त्यश्वपशुभिर्धान्यैर्मृगैर्गोजाविभिस्तथा}
{विविधैः पृथिवीं पुण्यां शिबिर्ब्राह्मणसात्करोत्}


\twolineshloka
{यावत्यो वर्षतो धारा यावत्यो दिवि तारकाः}
{यावत्यः सिकता गाङ्ग्यो यावन्मेरोर्महोपलाः}


\twolineshloka
{उदन्वति च यावन्ति रत्नानि प्राणिनोऽपि च}
{तावतीरददद्गा वै शिबिरौशीनरोऽध्वरे}


\twolineshloka
{नोद्यन्तारं धुरस्तस्य कञ्चिदन्यं प्रजापतिः}
{भूतं भव्यं भवन्तं वा नाध्यगच्छन्नरोत्तमम्}


\twolineshloka
{तस्यासन्विविधा यज्ञाः सर्वकामैः समन्विताः}
{हेमयूपासनगृहा हेमप्राकारतोरणाः}


\twolineshloka
{शुचिस्वाद्वन्नपानं च ब्राह्मणाः प्रयुतायुताः}
{नानाभक्ष्यैः प्रियकथाः पयोदधिमहाहदाः}


\twolineshloka
{तस्यासन्यज्ञवाटेषु नद्यः शुभ्रान्नपर्वताः}
{पिबत स्नात खादध्वमिति यत्रोच्यते जनैः}


\twolineshloka
{यस्मै प्रादाद्वरं रुद्रस्तुष्टः पुण्येन कर्मणा}
{अक्षयं ददतो वित्तं श्रद्धा कीर्तिस्तथा क्रियाः}


\twolineshloka
{यथोक्तमेव भूतानां प्रियत्वं स्वर्गमुत्तमम्}
{एताँल्लब्ध्वा वरानिष्टाञ्शिबिः काले दिवं गतः}


\threelineshloka
{स चेन्ममार सृञ्जय चतुर्भद्रतरस्त्वया}
{पुत्रात्पुण्यतरस्तुभ्यं मा पुत्रमनुतप्यथाः}
{अयज्वानमदक्षिण्यमभि श्वैत्येत्युदाहरत्}


\chapter{अध्यायः ५९}
\twolineshloka
{नारद उवाच}
{}


\threelineshloka
{रामं दाशरथिं चैव मृतं सृञ्जय शुश्रुम}
{यं प्रजा अन्वमोदन्त पिता पुत्रमिवौरसम्}
{असङ्ख्येया गुणा यस्मिन्नासन्नमिततेजसि}


\twolineshloka
{यश्चतुर्दशवर्षाणि निदेशात्पितुरच्युतः}
{वने वनितया सार्धमवसल्लक्ष्मणाग्रजः}


\twolineshloka
{जघान च जनस्थाने राक्षसान्मनुजर्षभः}
{तपस्विनां रक्षणार्थं सहस्राणि चतुर्दश}


\twolineshloka
{तत्रैव वसतस्तस्य रावणो नाम राक्षसः}
{जहार भार्यां वैदेहीं सम्मोह्यैनं सहानुजम्}


\twolineshloka
{`रामो हृतां राक्षसेन भार्यां श्रुत्वा जटायुषः}
{आतुरः शोकसन्तप्तो रामोऽगच्छद्धरीश्वरम्}


\twolineshloka
{तेन रामः सुसङ्गम्य वानरैश्च महाबलैः}
{आजगामोदधेः पारं सेतुं कृत्वा महार्णवे}


\twolineshloka
{तत्र हत्वा तु पौलस्त्यान्ससुहृद्गणबान्धवान्}
{मायाविनं महाघोरं रावणं लोककण्टकम्'}


\twolineshloka
{सुरासुरैरवध्यं तं देवब्राह्मणकण्टकम्}
{जघान स महाबाहुः पौलस्त्यं सगणं रणे}


\twolineshloka
{`हत्वा तत्र रिपुं सङ्ख्ये भार्यया सह सङ्गतः}
{स च लङ्केश्वरं चक्रे धर्मात्मानं विभीषणम्}


\twolineshloka
{भार्यया सह संयुक्तस्ततो वानरसेनया}
{अयोध्यामागतो वीरः पुष्पकेण विराजता}


\twolineshloka
{तत्र राजन्प्रविष्टः सन्नयोध्यायां महायशाः}
{मातॄर्वयस्यान्सचिवानृत्विजः सपुरोहितान्}


\twolineshloka
{शुश्रूषमाणः सततं मन्त्रिभिश्चाभिषेचितः}
{विसृज्य हरिराजानं हनुमन्तं सहाङ्गदम्}


\twolineshloka
{भ्रातरं भरतं वीरं शत्रुघ्नं चैव लक्ष्मणम्}
{पूजयन्परया प्रीत्या वैदेह्या चाभिपूजितः}


\twolineshloka
{दशवर्षसहस्राणि दशवर्षशतानि च}
{चतुःसगारपर्यन्तां पृथिवीमन्वशासत}


\twolineshloka
{अश्वमेधशतैरीजं क्रतुभिर्भूरिदक्षिणैः}
{यश्च विप्रप्रसादेन सर्वकामानवाप्य च'}


\twolineshloka
{सम्प्राप्य विधिवद्राज्यं सर्वभूतानुकम्पनः}
{`सर्वद्वीपानवष्टभ्य प्रजा धर्मेण पालयन्'}


\threelineshloka
{स निर्गलं मुख्यतममश्वमेधशतं प्रभुः}
{आजहार सुरेशस्य हविषा मुदमाहरन्}
{अन्यैश्च विविधैर्यज्ञैरीजे बहुगुणैर्नृपः}


\twolineshloka
{क्षुत्पिपासेऽजयद्रामः सर्वरोगांश्च देहिनाम्}
{सततं गुणसम्पन्नो दीप्यमानः स्वतेजसा}


\threelineshloka
{अतिसर्वाणि भूतानि रामो दाशरथिर्बभौ}
{ऋषीणां देवतानां च मानुषाणां च सर्वशः}
{पृथिव्यां सह वासोऽभूद्रामे राज्यं प्रशासति}


\twolineshloka
{नाहीयत तदा प्रामः प्राणिनां न तदा व्यथा}
{प्राणापानौ समावास्तां रामे राज्यं प्रशासति}


\twolineshloka
{पर्यदीप्यन्त तेजांसि तदाऽनर्थाश्च नाभवन्}
{दीर्घायुषः प्रजाः सर्वा युवा न म्रियते तदा}


\twolineshloka
{वेदैश्चतुर्भिः सुप्रीताः प्राप्नुवन्ति दिवौकसः}
{हव्यं कव्यं च विविधं निष्पूर्तं हुतमेव च}


\twolineshloka
{अदंशमशका देशा नष्टव्यालसरीसृपाः}
{नाप्यु प्राणभृतां मृत्युर्नाकाले ज्वलनोऽदहत्}


\twolineshloka
{अधर्मरुचयो लुब्धा मूर्खा वा नाभवंस्तदा}
{शिष्टेष्टप्राज्ञकर्माणः सर्वे वर्णास्तदाऽभवन्}


\twolineshloka
{स्वधां पूजां च रक्षोभिर्जनस्थाने प्रणाशिताम्}
{प्रादान्निहत्य रक्षांसि पितृदेवेभ्य ईश्वरः}


\twolineshloka
{सहस्रपुत्राः पुरुषा दशवर्षशतायुषः}
{न च ज्येष्ठाः कनिष्ठेभ्यस्तदा श्राद्धानि कुर्वते}


\twolineshloka
{श्यामो युवा लोहिताक्षो मत्तमातङ्गविक्रमः}
{आजानुबाहुः सुभुजः सिंहस्कन्धो महाबलः}


\twolineshloka
{दशवर्षसहस्राणि दशवर्षशतानि च}
{सर्वभूतमनःकान्तो रामो राज्यमकारयत्}


\twolineshloka
{रामो रामो राम इति प्रजानामभवत्कथा}
{रामभूतं जगदभूद्रामे राज्यं प्रशासति}


\twolineshloka
{चतुर्विधाः प्रजा रामः स्वर्गं नीत्वा दिवं गतः}
{आत्मेच्छया प्रतिष्ठाप्य राजवंशमिहाष्टधा}


\threelineshloka
{स चेन्ममार सृञ्जय चतुर्भद्रतरस्त्वया}
{पुत्रात्पुण्यतरस्तुभ्यं मा पुत्रमनुतप्यथाः}
{अयज्वानमदक्षिण्यमभि श्वैत्येत्युदाहरत्}


\chapter{अध्यायः ६०}
\twolineshloka
{नारद उवाच}
{}


\twolineshloka
{भगीरथं च राजानं मृतं सृञ्जय शुश्रुम}
{`परित्राणाय पूर्वेषां येन गङ्गाऽवतारिता}


\twolineshloka
{यस्येन्द्रो बाहुवीर्येण प्रीतो राज्ञो महात्मनः}
{सोऽश्वमेधशतैरीजे समाप्तवरदक्षिणैः}


\twolineshloka
{हविर्मन्त्रान्नसम्पन्नैर्देवानामादधौ मुदम्'}
{येन भागीरथी गङ्गा चयनैः काञ्चनैश्चिता}


\twolineshloka
{यः सहस्रं सहस्राणां कन्या हेमविभूषिताः}
{राज्ञश्च राजपुत्रांश्च ब्राह्मणेभ्यो ह्यमन्यत}


\twolineshloka
{सर्वा रथगताः कन्या रथाः सर्वे चतुर्युजः}
{रथेरथे शतं नागाः सर्वे वै हेममालिनः}


\twolineshloka
{सहस्रमश्वाश्चैकैकं गजानां पृष्ठतोऽन्वयुः}
{अश्वेअश्वे शतं गावो गवां पश्चादजाविकम्}


\twolineshloka
{तेनाक्रान्ता जनौघेन दक्षिणा भूयसीर्ददत्}
{उपह्वरेऽतिव्यथिता तस्याङ्के निषसाद ह}


\twolineshloka
{तथा मागीरथी गङ्गा ह्यूर्ध्वगा ह्यभवत्पुरा}
{दुहितृत्वं गता राज्ञः पुत्रत्वमगमत्तदा}


\twolineshloka
{तां तु गाथा जगुः प्रीता गन्धर्वाः सूर्यवर्चसः}
{पितृदेवमनुष्याणां शृण्वतां वल्गुवादिनः}


\twolineshloka
{भगीरथं यजमानमैक्ष्वाकुं भूरिदक्षिणम्}
{गङ्गा समुद्रगा देवी वव्रे पितरमीश्वरम्}


\twolineshloka
{तस्य सेन्द्रैः सुरगणैर्देवैर्यज्ञः स्वलङ्कृतः}
{सम्यक्परिगृहीतश्च शान्तविघ्नो निरामयः}


\twolineshloka
{यो य इच्छेत विप्रो वै यत्रयत्रात्मनः प्रियम्}
{भगीरथस्तदा प्रीतस्तत्रतत्राददद्वशी}


\twolineshloka
{नादेयं ब्राह्मणस्यासीद्यस्य यत्स्यात्प्रियं धनम्}
{सोऽपि विप्रप्रसादेन ब्रह्मलोकं गतो नृपः}


\twolineshloka
{येन यातौ मखमुखौ दिशाशाविह पादपाः}
{तेनावस्थातुमिच्छन्ति तं गत्वा राजमीश्वरम्}


\threelineshloka
{स चेन्ममार सृञ्जय चतुर्भद्रतरस्त्वया}
{पुत्रात्पुण्यतरस्तुभ्यं मा पुत्रमनुतप्यथाः}
{अयज्वानमदक्षिण्यमभि श्वैत्येत्युदाहरत्}


\chapter{अध्यायः ६१}
\twolineshloka
{नारद उवाच}
{}


\twolineshloka
{दिलीपं चापि राजेन्द्र मृतां सृञ्जय शुश्रुम}
{यस्य यज्ञशतेष्वासन्प्रयुतायुतशो द्विजाः}


\threelineshloka
{तन्त्रज्ञानार्थसम्पन्ना यज्वानः पुत्रपौत्रिणः}
{`स्वच्छन्दाः पुण्यगन्धाश्च सुभक्षा हेममालिनः}
{यस्य कर्माणि भूरिणि कथयन्ति मनीषिणः'}


\twolineshloka
{य इमां वसुसम्पूर्णां वसुधां वसुधाधिपः}
{ईजानो वितते यज्ञे ब्राह्मणेभ्यो ह्यमन्यत}


\threelineshloka
{`राजानो वितते यज्ञे उपहारानुपाहरन्'}
{दिलीपस्य तु यज्ञेषु कृतः पन्था हिरण्मयः}
{तं धर्म इव कुर्वाणाः सेन्द्रा देवाः समागमन्}


\twolineshloka
{सहस्रं यत्र मातङ्गा अगच्छन्पर्वतोपमाः}
{सौवर्णं चाभवत्सर्वं सदः परमभास्वरम्}


\twolineshloka
{रसानां चाभवन्कुल्या भक्ष्याणां चापि पर्वताः}
{सहस्रव्यामा नृपते यूपाश्चासन्हिरण्मयाः}


\twolineshloka
{चषालप्रचषालौ च यस्य यूपे हिरण्मयौ}
{नृत्यन्त्यप्सरसो यत्र षट््सहस्राणि सप्त च}


\twolineshloka
{यत्र वीणां वादयति प्रीत्या विश्वावसुः स्वयम्}
{सर्वभूतान्यमन्यन्त मम वादयतीति तम्}


% Check verse!
रागखाण्डवभोज्यैश्च मत्ताः पतिषु शेरते
\twolineshloka
{तदेतदद्भुतं मन्ये अन्यैर्न सदृशं नृपैः}
{यदप्सु युध्यमानस्य चक्रे न परिपेततुः}


\twolineshloka
{राजानं दृढधन्वानं दिलीपं सत्यवादिनम्}
{येऽपश्यन्भूरिदक्षिण्यं तेऽपि स्वर्गजितो नराः}


\twolineshloka
{पञ्च शब्दा न जीर्यन्ति दिलीपस्य निवेशने}
{स्वाध्यायघोषो ज्याघोषः पिबताश्नीत खादत}


\threelineshloka
{स चेन्ममार सृञ्जय चतुर्भद्रतरस्त्वया}
{पुत्रात्पुण्यतरस्तुभ्यं मा पुत्रमनुतप्यथाः}
{अयज्वानमदक्षिण्यमभि श्वैत्येत्युदाहरत्}


\chapter{अध्यायः ६२}
\twolineshloka
{नारद उवाच}
{}


\twolineshloka
{मान्धतारं यौवनाश्वं मृतं सृञ्जय शुश्रुम}
{यं देवावश्विनौ गर्भे पितुः पार्श्वे चकर्षतुः}


\twolineshloka
{मृगयां विचरन्राजा तृषितः क्लान्तवाहनः}
{धूमं दृष्ट्वाऽगमत्सत्रं पृषदाज्यमवाप सः}


\twolineshloka
{तं दृष्ट्वा युवनाश्वस्य जठरे रविसन्निभम्}
{गर्भं निरूहतुर्देवावश्विनौ भिषजां वरौ}


\twolineshloka
{तं दृष्ट्वा पितुरुत्सङ्गे शयानं देववर्चसम्}
{अन्योन्यमब्रुवन्देवाः कमयं धास्यतीति वै}


\twolineshloka
{मामेवायं धयत्वङ्ग इति ह स्माह वासवः}
{ततोङ्गुलिभ्यो हीन्द्रस्य प्रादुरासीत्पयोऽमृतम्}


\twolineshloka
{मां धास्यतीति कारण्याद्यदिन्द्रो ह्यन्वकम्पयत्}
{तस्मात्तु मान्धातेत्येवं नाम तस्याद्भुतं कृतम्}


\twolineshloka
{ततस्तु धारां पयसो घृतस्य च महात्मनः}
{तस्यास्ये यौवनाश्वस्य पाणिरिन्द्रस्य चास्रवत्}


\twolineshloka
{अपिबत्पाणिमिन्द्रस्य स चाप्यह्गाऽभ्यवर्धत}
{सोऽभवद्द्वादशसमो द्वादशाहेन वीर्यवान्}


\twolineshloka
{इमां च पृथिवीं कृत्स्नामेकाह्ना स व्यजीजयत्}
{धर्मात्मा धृतिमान्वीरः सत्यसन्धो जितेन्द्रियः}


\twolineshloka
{जनमेजयं सुधन्वानं गयं पूरुं बृहद्रथम्}
{असितं च नृगं चैव मान्धाता मानवोऽजयत्}


\twolineshloka
{उदेति च यतः सूर्यो यत्र च प्रतितिष्ठति}
{तत्सर्वं यौवनाश्वस्य मान्धातुः क्षेत्रमुच्यते}


\twolineshloka
{सोऽश्वमेधशतैरिष्ट्वा राजसूयशतेन च}
{अददद्रोहितान्मत्स्यान्ब्राह्मणेभ्यो विशाम्पते}


\twolineshloka
{हैरण्यान्योजनोत्सेधानायताञ्शतयोजनम्}
{बहुप्रकारान्सुस्वादून्भक्ष्यभोज्यान्नपर्वतान्}


% Check verse!
अतिरिक्तं ब्राह्मणेभ्यो भुञ्जानो हीयते जनः
\twolineshloka
{भक्ष्यान्नपाननिचयाः शुशुभुस्त्वन्नपर्वताः}
{घृतहदाः सूपपङ्का दधिफेनां गुडोदकाः}


\twolineshloka
{रुरुधुः पर्वतान्नद्यो मधुक्षीरवहाः शुभाः}
{देवासुरा नरा यक्षा गन्धर्वोरगपक्षिणः}


\twolineshloka
{विप्रास्तत्रागताश्चासन्वेदवेदाङ्गपारगाः}
{ब्राह्मणा ऋषयश्चापि नासंस्तत्राविपश्चितः}


\twolineshloka
{समुद्रान्तां वसुमतीं वसुपूर्णां तु सर्वतः}
{स तां ब्राह्मणसात्कृत्वा जगाम स्वान्गृहान्प्रति}


\twolineshloka
{`राजाऽपि विविधैरिष्ट्वा यज्ञैर्विविधदक्षिणैः'}
{गतः पुण्यकृतां लोकान्व्याप्य स्वयशसा दिशः}


\threelineshloka
{स चेन्ममार सृञ्जय चतुर्भद्रतरस्त्वया}
{पुत्रात्पुण्यतरस्तुभ्यं मा पुत्रमनुतप्यथाः}
{अयज्वानमदक्षिण्यमभि श्वैत्येत्युदाहरत्}


\chapter{अध्यायः ६३}
\twolineshloka
{नारद उवाच}
{}


\twolineshloka
{ययातिं नाहुषं चैव मृतं सृञ्जय शुश्रुम}
{`य इमां पृथिवीं जित्वा ससमुद्रां सपर्वताम्}


\twolineshloka
{शम्याप्रासेन निर्माय वेदीः सन्नतदक्षिणाः}
{ईजानः क्रतुभिः पुण्यैः पर्यगच्छत्प्रदक्षिणम्'}


\twolineshloka
{राजसूयशतैरिष्ट्वा सोऽश्वमेधशतेन च}
{पुण्डरीकसहस्रेण वाजपेयशतैस्तथा}


\twolineshloka
{अतिरात्रसहस्रेण चातुर्मास्यैश्च कामतः}
{अग्निष्टोमैश्च विविधैः सत्रैश्च प्राज्यदक्षिणैः}


\twolineshloka
{अब्राह्मणानां यद्वित्तं पृथिव्यामस्ति किञ्चन}
{तत्सर्वं परिसङ्ख्याय ततो ब्राह्मणसात्करोत्}


\twolineshloka
{सरस्वती पुण्यतमा नदीनांतथा समुद्राः सरितः साद्रयश्च}
{ईजानाय पुण्यतमाय राज्ञेघृतं पयो दुदुहुर्नाहुषाय}


\twolineshloka
{व्यूढे देवासुरे युद्धे कृत्वा देवसहायताम्}
{चतुर्धा व्यभजत्सर्वां चतुर्भ्यः पृथिवीमिमाम्}


\twolineshloka
{यज्ञैर्नानाविधैरिष्ट्वा प्रजामुत्पाद्य चोत्तमाम्}
{देवयान्यां चौशनस्यां शर्मिष्ठायां च धर्मतः}


\twolineshloka
{देवारण्येषु सर्वेषु विजहारामरोपमः}
{आत्मनः कामचारेण द्वितीय इव वारावः}


\twolineshloka
{यदा नाभ्यगमच्छान्तिं कामानां सर्ववेदवित्}
{ततो गाथामिमां गीत्वा सदारः प्राविशद्वनम्}


\twolineshloka
{यत्पृथिव्यां व्रीहियवं हिरण्यं पशवः स्त्रियः}
{नालमेकस्य तत्सर्वमिति मत्वा शमं व्रजेत्}


\twolineshloka
{एवं कामान्परित्यज्य ययातिर्धृतिमेत्य च}
{पूरुं राज्ये प्रतिष्ठाप्य प्रयातो वनमीश्वरः}


\threelineshloka
{स चेन्ममार सृञ्जय चतुर्भद्रतरस्त्वया}
{पुत्रात्पुण्यतरस्तुभ्यं मा पुत्रमनुतप्यथाः}
{अयज्वानमदक्षिण्यमभि श्वैत्येत्युदाहरत्}


\chapter{अध्यायः ६४}
\twolineshloka
{नारद उवाच}
{}


\twolineshloka
{नाभागमम्बरीषं च मृतं सृञ्जय शुश्रुम}
{यः सहस्रं सहस्राणां राज्ञां चैकस्त्वयोधयत्}


\twolineshloka
{जिगीषमाणाः सङ्ग्रामे समन्ताद्वैरिणोऽभ्ययुः}
{अस्त्रयुद्धविदो घोराः सृजन्तश्चाशिवा गिरः}


\twolineshloka
{बललाघवशिक्षाभिस्तेषां सोऽस्त्रबलेन च}
{छत्रायुधध्वजरथांश्छित्त्वा प्रासान्गतव्यथः}


\twolineshloka
{त एनं मुक्तसन्नाहाः प्रार्थयञ्जीवितैषिणः}
{शरण्यमीयुः शरणं तव स्म इति वादिनः}


\twolineshloka
{स तु तान्वशगान्कृत्वा जित्वा चेमां वसुन्धराम्}
{ईजे यज्ञशतैरिष्टैर्यथा शक्रस्तथाऽनघः}


\twolineshloka
{बुभुजुः सर्वसम्पन्नमन्नमन्ये जनाः सदा}
{तस्मिन्यज्ञे तु विप्रेन्द्राः सन्तृप्ताः परमार्चिताः}


\twolineshloka
{मोदकान्पूरिकापूपान्स्वादुपूर्णाश्च शष्कुलीः}
{करम्भान्पृथुमृद्वीका अन्नानि सुकृतानि च}


\twolineshloka
{सूपान्मैरेयकापूपान्रागषाडवपानकान्}
{मृष्टान्नानि सुयुक्तानि मृदूनि सुरभीणि च}


\twolineshloka
{घृतं मधु पयस्तोयं दधीनि रसवन्ति च}
{फलं मूलं सुस्वादु द्विजास्तत्रोपभुञ्जते}


\twolineshloka
{मदनीयानि पापानि विदित्वा चात्मनः सुखम्}
{अपिबन्त यथाकामं पानपा गीतवादितैः}


\twolineshloka
{तत्र स्म गाथा गायन्ति क्षीबा हृष्टाः पठन्ति च}
{नाभागस्तुतिसंयुक्ता ननृतुश्च सहस्रशः}


\twolineshloka
{तेषु यज्ञेष्वम्बरीषो दक्षिणामत्यकालयत्}
{राज्ञां शतसहस्राणि दशप्रयुतयाजिनाम्}


\twolineshloka
{हिरण्यकवचान्सर्वाञ्श्वेतच्छत्रप्रकीर्णकान्}
{हिरण्यस्यन्दनारूढान्सानुयात्रपरिच्छादान्}


\twolineshloka
{ईजानो वितते यज्ञे दक्षिणामत्यकालयत्}
{मूर्धाभिषिक्तांश्च नृपान्राजपुत्रशतानि च}


\twolineshloka
{स दण्डकोशनिचयान्ब्राह्मणेभ्यो ह्यमन्यत}
{नैवं पूर्वे जनाश्चक्रुर्न करिष्यन्ति चापरे}


\twolineshloka
{यदम्बरीषो नृपतिः करोत्यमितदक्षिणः}
{इत्येवमनुमोदन्ते प्रीता यस्य महर्षयः}


\threelineshloka
{स चेन्ममार सृञ्जय चतुर्भद्रतरस्त्वया}
{पुत्रात्पुण्यतरस्तुभ्यं मा पुत्रमनुतप्यथाः}
{अयज्वानमदक्षिण्यमभि श्वैत्येत्युदाहरत्}


\chapter{अध्यायः ६५}
\twolineshloka
{नारद उवाच}
{}


\twolineshloka
{शशिबिन्दुं च राजानं मृतं सृञ्जय शुश्रुम}
{य ईजे विविधैर्यज्ञैः श्रीमान्सत्यपराक्रमः}


\twolineshloka
{तस्य भार्यासहस्राणां शतमासीन्महात्मनः}
{एकैकस्यां च भार्यायां सहस्रं तनयाभवन्}


\twolineshloka
{ते कुमाराः पराक्रान्ताः सर्वे नियुतयाजिनः}
{राजानः क्रतुभिर्मुख्यैरीजाना वेदपारगाः}


\twolineshloka
{हिरण्यकवचाः सर्वे सर्वे चोत्तमधन्विनः}
{सर्वेऽश्वमेधैरीजानाः कुमाराः शाशिबिन्दवाः}


\twolineshloka
{तानश्वमेधे राजेन्द्रो ब्राह्मणेभ्योददत्पिता}
{शतंशतं रथगजा एकैकं पृष्ठतोऽन्वयुः}


\twolineshloka
{राजपुत्रं तदा कन्यास्तपनीयस्वलङ्कृताः}
{कन्याङ्कन्यां शतं नागा नागेनागे शतं रथाः}


\twolineshloka
{रथेरथे शतं चाश्वा बलिनो हेममालिनः}
{अश्वेअश्वे गोसहस्रं गवां पञ्चाशदाविकाः}


\twolineshloka
{एतद्धनमपर्याप्तमश्वमेधे महामस्वे}
{शशिबिन्दुर्महाभागो ब्राह्मणेभ्यो ह्यमन्यत}


\twolineshloka
{वार्क्षाश्च यूपा यावन्त अश्वमेधे महामखे}
{ते तथैव पुनश्चान्ये तावन्तः काञ्चनाभवन्}


\twolineshloka
{भक्ष्यान्नपाननिचयाः पर्वताः क्रोशमुच्छ्रिताः}
{तस्याश्वमेधे निर्वृत्ते राज्ञः शिष्टास्त्रयोदश}


\twolineshloka
{तुष्टपुष्टजनाकीर्णां शान्तविघ्नामनामयाम्}
{शशिबिन्दुरिमां भूमिं चिरं भुक्त्वा दिवं गतः}


\threelineshloka
{स चेन्ममार सृञ्जय चतुर्भद्रतरस्त्वया}
{पुत्रात्पुण्यतरस्तुभ्यं मा पुत्रमनुतप्यथाः}
{अयज्वानमदक्षिण्यमभि श्वैत्येत्युदाहरत्}


\chapter{अध्यायः ६६}
\twolineshloka
{नारद उवाच}
{}


\twolineshloka
{गयं चाधूर्तरजसं मृतं सृञ्जय शुश्रुम}
{ये वै वर्षशत राजा हुतशिष्टाशनोऽभवत्}


\twolineshloka
{तस्मै ह्यग्निर्वरं प्रादात्ततो वव्रे वरं गयः}
{तपसा ब्रह्मचर्येण व्रतेन नियमेन च}


\twolineshloka
{गुरूणां च प्रसादेन वेदानिच्छामि वेदितुम्}
{स्वधर्मेणाविहिंस्यान्यान्धनमिच्छामि चाक्षयम्}


\twolineshloka
{विप्रेषु ददतश्चैव श्रद्धा भवतु नित्यशः}
{कन्यासु च सवर्णासु सुप्रजस्त्वं च मे भवेत्}


\twolineshloka
{अन्नमोजो महो मह्यं धर्मे मे रमतां मनः}
{अविघ्नं चास्तु मे नित्यं धर्मकार्येषु पावक}


\twolineshloka
{तथा भिष्यतीत्युक्त्वा तत्रैवान्तरधीयत}
{गयो ह्यवाप्य तत्सर्वं धर्मेणाराधयञ्जगत्}


\twolineshloka
{स दर्शपूर्णमासाभ्यां कालेष्वाग्रयणेन च}
{चातुर्मास्यैश्च विविधैर्यज्ञैश्चावाप्तदक्षिणैः}


\twolineshloka
{अयजच्छ्रद्धया राजा परिसंवत्सराञ्शतम्}
{गवां शतसहस्राणि शतमश्वशतानि च}


\twolineshloka
{शतं निष्कसहस्राणि गवां चाप्ययुतानि षट्}
{उत्थायोत्थाय सम्प्रादात्परिसंवत्सराञ्शतम्}


\twolineshloka
{नक्षत्रेषु च सर्वेषु ददन्नक्षत्रदक्षिणाः}
{ईजे च विविधैर्यज्ञैर्यथा सोमोऽङ्गिरा यथा}


\twolineshloka
{सौवर्णां पृथिवीं कृत्वा य इमां प्रमिशर्कराम्}
{विप्रेभ्यः प्राददद्राजा सोऽश्वमेधे महामखे}


\twolineshloka
{जाम्बूनदमया यूपाः सर्वे रत्नपरिच्छदाः}
{गयस्यासन्समृद्धास्तु सर्वभूतमनोहराः}


\twolineshloka
{सर्वकामसमृद्धं च प्रादादन्नं गयस्तदा}
{ब्राह्ममेभ्यः प्रहृष्टेभ्यः सर्वभूतेभ्य एव च}


\twolineshloka
{ससमुद्रवनद्वीपनदीनदवनेषु च}
{नगरेषु च राष्ट्रेषु दिवि व्योम्नि च येऽवसन्}


\twolineshloka
{भूतग्रामाश्च विविधाः सन्तृप्ता यज्ञसम्पदा}
{गयस्य सदृशो यज्ञो नास्त्यन्य इति तेऽब्रुवन्}


\twolineshloka
{षट््त्रिंशद्योजनायामा त्रिंशद्योजनमायता}
{पश्चात्पुरश्चतुर्विंशद्वेदी ह्यासीद्धिरण्मयी}


\twolineshloka
{गयस्य यजमानस्य मुक्तावज्रमणिस्तृता}
{प्रादात्स ब्राह्मणेभ्योऽथ वासांस्याभरणानि च}


\twolineshloka
{यथोक्ता दक्षिणाश्चान्या विप्रेभ्यो भूरिदक्षिणः}
{यत्र भोजनशिष्टस्य पर्वताः पञ्चविंशतिः}


\twolineshloka
{कुल्याश्च क्षीरवाहिन्यो रसानां चापभवंस्तदा}
{वस्त्राभरणगन्धानां शारयश्च पृथग्विधाः}


\twolineshloka
{यस्य प्रभावाच्च गयस्त्रिषु लोकेषु विश्रुतः}
{वटश्चाक्षय्यकरणः पुण्यं ब्रह्मसरश्च तत्}


\threelineshloka
{स चेन्ममार स़ञ्जय चतुर्भद्रतरस्त्वया}
{पुत्रात्पुण्यतरस्तुभ्यं मा पुत्रमनुतप्यथाः}
{अयज्वानमदक्षिण्यमभि श्वैत्येत्युदाहरत्}


\chapter{अध्यायः ६७}
\twolineshloka
{नारद उवाच}
{}


\twolineshloka
{साङ्कृतिं रन्तिदेवं च मृतं सृञ्जय शुश्रुम}
{यस्य द्विशतसाहस्रा आसन्सूदा महात्मनः}


\twolineshloka
{गृहानभ्यागतान्विप्रानतिथीन्परिवेषकाः}
{पक्वंपक्वं दिवारात्रं वरान्नममृतोपमम्}


\twolineshloka
{न्यायेनाधिगतं वित्तं ब्राह्मणेभ्यो ह्यमन्यत}
{वेदानधीत्य धर्मेण यश्चक्रे द्विषतो वशे}


\twolineshloka
{उपस्थिताश्च पशवः स्वयं यं शंसितव्रतम्}
{बहवः स्वर्गमिच्छन्तो विधिवत्सत्रयाजिनम्}


\twolineshloka
{नदी महानसाद्यस्य प्रवृतक्ता चर्मराशितः}
{तस्माच्चर्मण्वती नाम ख्याता पुण्या सरिद्वरा}


\twolineshloka
{ब्राह्मणेभ्यो ददन्निष्कान्सौवर्णान्स प्रभावतः}
{तुभ्यं निष्कं तुभ्यं निष्कमिति ह स्म प्रभाषते}


\twolineshloka
{तुभ्यन्तुभ्यमिति प्रादान्निष्कान्निष्कान्सहस्रशः}
{तत पुनः समाश्वास्य निष्कानेव प्रयच्छति}


\twolineshloka
{अल्पं दत्तं मयाऽद्येति निष्ककोटिं सहस्रशः}
{एकाह्ना ताम्यति ततः क्वाद्य विप्रा इति ब्रुवन्}


\twolineshloka
{द्विजपाणिवियोगेन दुःखं मे शाश्वतं महत्}
{भविष्यति न सन्देह एवं राजाऽददद्वसु}


\twolineshloka
{सहस्रशश्च सौवर्णान्वृषभान्गोशतानुगान्}
{साष्टं शतं सुवर्णानां निष्कमाहुर्धनं तथा}


\twolineshloka
{अध्यर्धमासमददद्ब्राह्मणेभ्यः शतं समाः}
{अग्निहोत्रोपकरणं यज्ञोपकरणं च यत्}


\twolineshloka
{ऋषिभ्यः करकान्कुम्भान्स्थालीः पिठरमेव च}
{शयनासनायानानि प्रासादांश्च गृहाणि च}


\twolineshloka
{वृक्षांश्च विविधान्दद्यादन्नानि च धनानि च}
{सर्वं सौवर्णमेवासीद्रन्तिदेवस्य धीमतः}


\twolineshloka
{तत्रास्य गाथा गायन्ति ये पुराणविदो जनाः}
{रन्तिदेवस्य तां दृष्ट्वा समृद्धिमतिमानुषीम्}


\twolineshloka
{नैतादृशं दृष्टपूर्वं कुबेरसदनेष्वपि}
{धनं च पूर्यमाणं नः किं पुनर्मनुजेष्विति}


\twolineshloka
{व्यक्तं वस्वोकसारेयमित्युचुस्तत्र विस्मिताः}
{साङ्कृते रन्तिदेवस्य यां रात्रिमतिथिर्वसेत्}


\twolineshloka
{आलभ्यन्त तदा गावः सहस्राण्येकविंशतिः}
{तत्र स्म सूदाः क्रोशन्ति सुमृष्टमणिकुण्डलाः}


\twolineshloka
{सूपं भूयिष्ठमश्नीध्वं नाद्य मांसं यथा पुरा}
{रन्तिदेवस्य यत्किञ्चित्सौवर्णमभवत्तदा}


\twolineshloka
{तत्सर्वं वितते यज्ञे ब्राह्मणेभ्यो ह्यमन्यत}
{प्रत्यक्षं तस्य हव्यानि प्रतिगृह्णन्ति देवताः}


\twolineshloka
{कव्यानि पितरः काले सर्वकामान्द्विजोत्तमाः}
{स चेन्ममार सृञ्जय चतुर्भद्रतरस्त्वया}


\twolineshloka
{पुत्रात्पुण्डतरस्तुभ्यं मा पुत्रमनुतप्यथाः}
{अयज्वानमदक्षिण्यमभि श्वैत्येत्युदाहरत्}


\chapter{अध्यायः ६८}
\twolineshloka
{नारद उवाच}
{}


\twolineshloka
{दौष्यन्ति भरतं चापि मृतं सृञ्जय शुश्रुम}
{कर्माण्यसुकराण्यन्यैः कृतवान्यः शिशुर्वने}


\twolineshloka
{हिमावदातान्यः सिंहान्नखदंष्ट्रायुधान्बली}
{निर्वीर्यांस्तरसा कृत्वा विचकर्ष बबन्ध च}


\twolineshloka
{क्रूरांश्चोग्रतरान्व्याघ्रान्दमित्वा चाकरोद्वशे}
{मनःशिला इव शिलाः संयुक्ता जतुराशिभिः}


\twolineshloka
{त्रायलादींश्चातिबलवान्सुप्रतीकान्गजानपि}
{दंष्ट्रासु गृह्य विमुखाञ्शुष्कास्यानकरोद्वशे}


\twolineshloka
{महिषानप्यतिबलो बलिनो विचकर्ष ह}
{सिंहानां च सुदृप्तानां शतान्याकर्षयद्बलात्}


\twolineshloka
{बलिनः सृमरान्खङ्गान्नानासत्त्वानि चाप्युत}
{कृच्छ्रप्राणान्वने बद्ध्वा दमयित्वाप्यवासृजत्}


\twolineshloka
{तं सर्वदमनेत्याहुर्द्विजास्तेनास्य कर्मणा}
{तं प्रत्यषेधज्जननी मा सत्वानि व्यनीनशः}


\twolineshloka
{सोऽश्वमेधशतेनेजे यमुनामनु वीर्यवान्}
{त्रिशतैश्च सरस्वत्यां गङ्गामनु चतुःशतैः}


\twolineshloka
{सौऽश्वमेधसहस्रेण राजसूयशतेन च}
{पुनरीजे महायज्ञैः समाप्तवरदक्षिणैः}


\twolineshloka
{अग्निष्टोमातिरात्राणामुक्थ्यविश्वजितां च सः}
{वाजपेयसहस्रामां सहस्रैश्च सुसंवृतैः}


\threelineshloka
{इष्ट्वा शाकुन्तलो राजा तर्पयित्वा द्विजान्धनैः}
{सहस्रं यत्र पद्मानां कण्वाय भरतो ददौ}
{जाम्बूनदस्य शुद्धस्य कनकस्य महायशाः}


\twolineshloka
{यस्य यूपाः सतव्यामाः परिणाहेन काञ्चनाः}
{समागम्य द्विजैः सार्धे सेन्द्रैर्देवैः समुच्छ्रिताः}


\twolineshloka
{अलङ्कृतान्राजमानान्सर्वरत्नैर्मनोहरैः}
{हैरण्यान्द्विरदानश्वान्रथानुष्ट्रानजाविकान्}


\twolineshloka
{दासी दासं धनं धान्यं गाः सवत्साः पयस्विनीः}
{ग्रामान्गृहांश्च क्षेत्राणि विविधांश्च परिच्छदान्}


\twolineshloka
{कोटीशतायुतांश्चैव ब्राह्मणेभ्यो ह्यमन्यत}
{चक्रवर्ती ह्यदीनात्मा जितारिर्ह्यजितः परैः}


\threelineshloka
{स चेन्ममार सृञ्जय चतुर्भद्रतरस्त्वया}
{पुत्रात्पुण्यतरस्तुभ्यं मा पुत्रमनुतप्यथाः}
{अयज्वानमदक्षिण्यमभि श्वैत्येत्युदाहरत्}


\chapter{अध्यायः ६९}
\twolineshloka
{नारद उवाच}
{}


\twolineshloka
{पुथुं वैन्यं च राजानं मृतं सृञ्जय शुश्रुम}
{यमब्यषिञ्चत्सम्मन्त्र्य राजसूये महर्षयः}


\twolineshloka
{अयं नः प्रथयिष्येत सर्वानित्यभवत्पृथुः}
{क्षतान्नस्त्रास्यते सर्वानत्येवं क्षत्रियोऽभवत्}


\twolineshloka
{पृथुं वैन्यं प्रजा दृष्ट्वा रक्ताः स्मेति यदब्रुवन्}
{ततो राजेति नामास्य अनुरागादजायत}


\twolineshloka
{अकृष्टपच्या पृथिवी आसीद्वैन्यस्य कामधुक्}
{सर्वाः समादुघा गावः पुटकेपुटके मधु}


\twolineshloka
{आसन्हिरण्मया वृक्षाः सुखस्पर्शाः सुखावहाः}
{तेषां चीरxx संवीताः प्रजास्तेष्वेव शेरते}


\twolineshloka
{फलान्यमृतकल्पानि स्वादूनि च मधूनि च}
{तेषामासीत्तदाऽऽहारो निराहाराश्च नाभवन्}


\twolineshloka
{अरोगाः सर्वसिद्धार्था मनुष्या ह्यकुतोभयाः}
{न्यवसन्त यथाकामं वृक्षेषु च गुहासु च}


\twolineshloka
{प्रविभागो न राष्ट्राणां पुराणां चाभवत्तदा}
{यथासुखं यथाकामं तथैता मुदिताः प्रजाः}


\twolineshloka
{तस्य संस्तम्भिता ह्यापः समुद्रमभियास्यतः}
{पर्वताश्च ददुर्मार्गं ध्वजभङ्गश्च नाभवत्}


\threelineshloka
{तं वनस्पतयः शैला देवासुरनरोरगाः}
{सप्तर्षयः पुण्यजना गन्धर्वाप्सरसोऽपि च}
{पितरश्च सुखासीनमभिगम्येदमब्रुवन्}


\threelineshloka
{सम्राडसि क्षत्रियोऽसि राजा गोप्ता पिताऽसि नः}
{देह्यस्मभ्यं महाराज प्रभुः सन्नीप्सितान्वरान्}
{यैर्वयं शाश्वतं तृप्ता वर्तयिष्यामहे सुखम्}


\twolineshloka
{`विज्ञापितः प्रजाभिस्तु प्रजानां हितकाम्यया}
{धनुर्गृह्य पृषत्कांश्च वसुधामाद्रवद्बली}


\twolineshloka
{ततो वैन्यभयाद्राजन्गौर्भूत्वा प्राद्रवन्मही}
{तां पृथुर्धनुरादाय द्रवन्तीमन्वसारयत्}


\twolineshloka
{सा लोकान्ब्रह्मलोकादीन्गत्वा वैन्यभयार्दिता}
{सा ददर्शाग्रतो वैन्यं कार्मुकोद्यतपाणिकम्}


\twolineshloka
{ज्वलद्भिर्विशिखैर्बाणैर्दीप्ततेजःसमद्युतिम्}
{महायोगं महात्मानं दुर्धर्षममरैरपि}


\twolineshloka
{अलभन्ती परित्राणं वैन्यमेवाभ्यपद्यत}
{कृताञ्जलिपुटा राजन्पूज्यं लोकैस्त्रिभिस्तदा}


\threelineshloka
{उवाच चैनं नाधर्म्यं स्त्रीवधं कर्तुमर्हसि}
{कथं धारयिता चासि प्रजा राजन्मया विना ॥पृथुरुवाच}
{}


\twolineshloka
{एकस्यार्थाय यो हन्यादात्मनो वा परस्य वा}
{एकं प्राणान्बहून्वापि प्राणिनां नास्ति पातकं}


\twolineshloka
{यस्मिंस्तु निहते भद्रे बहवः सुखमेधते}
{तस्मिन्हते त्वधं नास्ति पातकं नोपभुज्यते}


\twolineshloka
{सोऽहं पालनिमित्तं त्वां बधिष्यामि वसुंधरे}
{यदि चेद्वचनादद्य न करिष्यसि मे प्रियम्}


\twolineshloka
{त्वां निहत्य तु बाणेन मच्छासनपराङ्मुखीम्}
{आत्मानं प्रथयित्वाऽहं प्रजा धारयिता स्वयं}


\twolineshloka
{सुखं वचनमास्थाय मम धर्मभृतां वरे}
{स़ञ्जीवय प्रजा नित्यं शक्ता ह्यसि वसुन्धरे}


\threelineshloka
{दुहितृत्वं च मे गच्छ एवमेतन्महाशरम्}
{नियच्छेयं त्वदर्थाय उद्यतं घोरदर्शनम् ॥भूमिरुवाच}
{}


\twolineshloka
{सर्वमेतन्महाराज विधास्यासि परन्तप}
{वत्सं त्वं पश्य राजन्वै क्षरेयं येन वत्सला}


\threelineshloka
{समां तु कुरु सर्वज्ञ मां तु धर्मभृतां वर}
{यथा विष्यन्दमानं वै क्षीरं सर्वत्र भावये ॥नारद उवाच}
{}


\twolineshloka
{तत उत्सारयामास शिलाजालानि सर्वशः}
{पृथुर्वैन्यस्तदा राजा तेन शैला विवर्धिताः}


\twolineshloka
{न हि पूर्वनिसर्गे वै विषमे वसुधातले}
{प्रविभागः पुराणां वा ग्रामाणां वा महीपते}


\twolineshloka
{न सस्यानि न गोरक्ष्यं न कृषिर्न वणिक्पथः}
{वैन्यात्प्रभृति राजेन्दर सर्वस्यैतस्य सम्भवः}


\threelineshloka
{यत्रयत्र च साम्यं तु भूमावासीत्किलानघ}
{तत्रतत्र प्रजास्तात निवासमभिरोचयन्}
{कृच्छ्रेण च महाराज इत्येवमनुशुश्रुम}


\threelineshloka
{तथेत्युक्त्वा पुनर्वैन्यो गृहीत्वाऽऽजगवं धनुः}
{शरांश्चाप्रतिमान्घोरांश्चिन्तयित्वाऽब्रवीन्महीम् ॥पृथुरुवाच}
{}


\twolineshloka
{एह्येहि वसुधे क्षिप्रं क्षरैभ्यः काङ्क्षितं पयः}
{मच्छसनातिगां वै त्वां प्रमथिष्याम्यहं शरैः}


\fourlineindentedshloka
{नारद उवाच}
{तथोक्ता साऽत्मनः श्रेयश्चिन्तयित्वाऽब्रवीत्पृथुम्}
{भूमिरुवाच}
{वत्सं पात्राणि दोग्धॄंश्च क्षीराणि च समादिश}


\threelineshloka
{ततो दास्याम्यहं भद्र सर्वं यस्य यथेप्सितम्}
{दुहितृत्वे च मां वीर सङ्कल्पयितुमर्हसि ॥नारद उवाच}
{'}


\twolineshloka
{तथेत्युक्त्वा पृथुः सर्वं विधानमकरोद्वशी}
{ततो भूतनिकायास्तां वसुधां दुदुहुस्तदा}


\twolineshloka
{तां वनस्पतयः पूर्वं समुत्तस्थुर्दुधुक्षवः}
{साऽतिष्ठद्वत्सला वत्सं दोग्धृपात्राणि चेच्छती}


\twolineshloka
{वत्सोऽभूत्पुष्पितः सालः प्लक्षो दोग्धाऽभवत्तदा}
{छिन्नप्ररोहणं दुग्धं पात्रमौदुम्बरं शुभम्}


\twolineshloka
{उदयः पर्वतो वत्सो मेरुर्दोग्धा महागिरिः}
{रत्नान्योषधयो दुग्धं पात्रमस्तमयं तथा}


\twolineshloka
{`देवानां वत्स इन्द्रोऽभूत्पात्रं दारुमयं तथा'}
{दोग्धा च सविता देवो दुग्धमूर्जस्करं प्रियम्}


\twolineshloka
{असुरा दुदुहुर्मायामयःपात्रे तु तां तदा}
{दोग्धा द्विमूर्धा तत्रासीद्वत्सश्चासीद्विरोचनः}


\twolineshloka
{कृषिं च सस्यं च नरा दुदुहुः पृथिवीतले}
{स्वायंभुवो मनुर्वत्सस्तेषां दोग्धाऽभवत्पृथुः}


\twolineshloka
{अलाबुपात्रे च तथा विषं दुग्धा वसुन्धरा}
{धृतराष्ट्रोऽभवद्दोग्धा तेषां वत्सस्तु तक्षकः}


\twolineshloka
{सप्तर्षिभिर्ब्रह्म दुग्धा तथा चाक्लिष्टकर्मभिः}
{दोग्धा बृहस्पतिः पात्रं छन्दो वत्सश्च सोमराट्}


\twolineshloka
{अन्तर्धानं चामपात्रे दुग्धा पुण्यजनैर्विराट्}
{दोग्धा वैश्रवणस्तेषां वत्सश्चासीद्वृषध्वजः}


\twolineshloka
{पुण्यगन्धान्पद्मपात्रे गन्धर्वाप्सरसोऽदुहन्}
{वत्सश्चित्ररथस्तेषां दोग्धा विश्वरुचिः प्रभुः}


\twolineshloka
{स्वधां रजतपात्रेषु दुदुहुः पितरश्च ताम्}
{वत्सो वैवस्वतस्तेषां यमो दोग्धान्तकस्तदा}


\twolineshloka
{एवं निकायैस्तैर्दुग्धा पयोऽभीष्टं हि सा विराट्}
{यैर्वर्तयन्ति ते ह्यद्य पात्रैर्वत्सैश्च नित्यशः}


\twolineshloka
{यज्ञैश्च विविधैरिष्ट्वा पृथुर्वैन्यः प्रतापवान्}
{सन्तर्पयित्वा भूतानि सर्वैः कामैर्मनःप्रियैः}


\twolineshloka
{हैरण्यानकरोद्राजा ये केचित्पार्थिवा भुवि}
{तान्ब्राह्मणेभ्यः प्रायच्छदश्वमेधे महामखे}


\twolineshloka
{स षष्टिं गोसहस्राणि षष्टिं नागशतानि च}
{सौवर्णानकरोद्राजा ब्राह्मणेभ्यश्च तान्ददौ}


\twolineshloka
{इमां च पृथिवीं सर्वां मणिरत्नविभूषिताम्}
{सौवर्णीमकरोद्राजा ब्राह्मणेभ्यश्च तां ददौ}


\threelineshloka
{स चेन्ममार सृञ्जय चतुर्भद्रतरस्त्वया}
{पुत्रात्पुण्यतरस्तुभ्यं मा पुत्रमनुतप्यथाः}
{अयज्वानमदक्षिण्यमभि श्वैत्येत्युदाहरत्}


\chapter{अध्यायः ७०}
\twolineshloka
{नारद उवाच}
{}


\twolineshloka
{रामो महातपाः शूरो वीरो लोकनमस्कृतः}
{जामदग्न्योऽप्यतियशा अवितृप्तो मरिष्यति}


\twolineshloka
{यश्चास्त्रमनुपर्येति भूमिं कुर्वन्विपांसुलाम्}
{नचासीद्विक्रिया यस्य प्राप्य श्रियमनुत्तमाम्}


\twolineshloka
{`जामदग्न्यो न ते राजन्कच्चिच्छ्रोत्रमुपागतः}
{येनैकेन पुरा राजन्क्रुद्धेन हतबन्धुना}


\twolineshloka
{भृगुभिस्ताम्भमानेन त्राहि रामेति विस्वरम्}
{त्रिःसप्तकृत्वो भूमिर्यत्कृता निःक्षत्रिया पुरा'}


\twolineshloka
{यः क्षत्रियैः परामृष्टे वत्से पितरि चाब्रुवन्}
{ततोऽवधीत्कार्तवीर्यमजितं समरे परैः}


\twolineshloka
{क्षत्रियाणां चतुःषष्टिमयुतानि सहस्रशः}
{तदा मृत्योः समेतानि एकेन धनुषाऽजयत्}


\twolineshloka
{ब्रह्मद्विषां चाथ तस्मिन्हस्राणि चतुर्दश}
{पुनरन्यान्निजग्राह दन्तकूरे जघान ह}


\twolineshloka
{सहस्रं मुसलेनाहन्सहस्रमसिनाऽवधीत्}
{उद्बन्धनात्सहस्रं च सहस्रमुदके कृतम्}


\twolineshloka
{दन्तान्भङ्क्त्वा सहस्रस्य भिन्नकर्णांस्तथाऽकरोत्}
{ततः सप्तसहस्राणां कुटुधृपमपाययत्}


\threelineshloka
{शिष्टान्वध्वा च हत्वा वै तेषां मृर्ध्नि विभिद्य च}
{गुणावतीमुत्तरेण खाण्डवाद्दक्षिणेन च}
{गिर्यन्ते शतसाहस्रा हैहयाः समरे हताः}


\twolineshloka
{सरथाश्वगजा वीरा निहतास्तत्र शेरते}
{पितुर्वधामर्षितेन जामदग्न्येन धीमता}


\twolineshloka
{निजघ्ने दशसाहस्रान्रामः परशुना तदा}
{न ह्यमृष्यत ता वाचो यास्तैर्भृशमुदीरिताः}


\twolineshloka
{भृगौ रामाभिधावेति यदाक्रन्दन्द्विजोत्तमाः}
{ततः काश्मीरदरदान्कुन्तिक्षुद्रकमालवान्}


\twolineshloka
{अङ्गवङ्गकलिङ्गांश्च विदेहांस्ताम्रलिप्तकान्}
{रक्षोवाहान्वीतिहोत्रांस्त्रिगर्तान्मार्तिकावतान्}


\twolineshloka
{शिबीनन्यांश्च राजन्यान्देशान्देशान्सहस्रशः}
{निजघान शितैर्बाणैर्जामदग्न्यः प्रतापवान्}


\twolineshloka
{कोटीशतसहस्राणि क्षत्रियाणां सहस्रशः}
{इन्द्रगोपकवर्णस्य बन्धुजीवनिभस्य च}


\twolineshloka
{रुधिरस्य परीवाहैः पूरयित्वा सरांसि च}
{सर्वानष्टादश द्वीपान्वशमानीय भार्गवः}


\twolineshloka
{ईजे क्रतुशतैः पुण्यैः समाप्तवरदक्षिणैः}
{वेदीमष्टनलोत्सेधां सौवर्णां विधिनिर्मिताम्}


\twolineshloka
{सर्वरत्नशतैः पूर्णां पताकाशतमालिनीम्}
{ग्राम्यारण्यैः पशुगणैः सम्पूर्णां च महीमिमाम्}


\twolineshloka
{रामस्य जामदग्न्यस्य प्रतिजग्राह कश्यपः}
{ततः शतसहस्राणि द्विपेन्द्रान्हेमभूषणान्}


\twolineshloka
{निर्दस्युं पृथिवीं कृत्वा शिष्टेष्टजनसङ्कुलाम्}
{कश्यपाय ददौ रामो हयमेधे महामखे}


\twolineshloka
{त्रिःसप्तकृत्वः पृथिवीं कृत्वा निःक्षत्रियां प्रभुः}
{इष्ट्वा क्रतुशतैर्वीरो ब्राह्मणेभ्यो ह्यमन्यत}


\twolineshloka
{सप्तद्वीपां वसुमतीं मारीचोऽगृह्णत द्विजः}
{रामं प्रोवाच निर्गच्छ वसुधातो ममाज्ञया}


\twolineshloka
{स कश्यपस्य वचनात्प्रोत्सार्य सरिताम्पतिम्}
{इषुपाते युधां श्रेष्ठः कुर्वन्ब्राह्मणशासनम्}


\twolineshloka
{अध्यावसद्गिरिश्रेष्ठं महेन्द्रं पर्वतोत्तमम्}
{एवं गुणशतैर्युक्तो भृगूणां कीर्तिवर्धनः}


\twolineshloka
{जामदग्न्यो ह्यतियशा मरिष्यति महाद्युतिः}
{त्वया चतुर्भद्रतरः पुण्यात्पुण्यतरस्तव}


\threelineshloka
{अयज्वानमदक्षिण्यं मा पुत्रमनुतप्यथाः}
{एते चतुर्भद्रतरास्त्वया भद्रशताधिकाः}
{मृता नरवरश्रेष्ठ मरिष्यन्ति च सृञ्जय}


\chapter{अध्यायः ७१}
\twolineshloka
{व्यास उवाच}
{}


\twolineshloka
{पुण्यमाख्यानमायुष्यं श्रुत्वा षोडशराजकम्}
{अव्याहरन्नरपतिस्तूष्णीमासीत्स सृञ्जयः}


\twolineshloka
{तमब्रवीत्तदा दीनं नारदो भगवानृषिः}
{कच्चिन्मया व्याहृतं यद्धृदये तत्स्थितं तव}


\twolineshloka
{आहोस्विदन्ततो नष्टं श्राद्धं शूद्रीपताविव}
{स एवमुक्तः प्रत्याह प्राञ्जलिः सृञ्जयस्तदा}


\twolineshloka
{पुत्रशोकापहं श्रुत्वा धन्यमाख्यानमुत्तमम्}
{राजर्षीणां पुराणानां यज्वनां दक्षिणावताम्}


\threelineshloka
{विस्मयेन हृते शोके तमसीवार्कतेजसा}
{विपाप्माऽस्म्यव्यथोपेतो ब्रूहि किं करवाण्यहम् ॥नारद उवाच}
{}


\threelineshloka
{दिष्ट्यापहतशोकस्त्वं वृणीष्वेह यदिच्छसि}
{तत्ते सम्पत्स्यते सर्वं न मृषावादिनो वयम् ॥सृञ्जय उवाच}
{}


\threelineshloka
{पावितोऽहमनेनैव प्रसन्नो यद्भवान्मम्}
{प्रसन्नो यस्य भगवान्न तस्यास्तीह दुर्लभम् ॥नारद उवाच}
{}


\threelineshloka
{पुनर्ददामि ते पुत्रं दस्युभिर्निहतं वृथा}
{उद्धृत्य नरकात्कष्टात्पशुवत्प्रोक्षितं यथा ॥व्यास उवाच}
{}


\twolineshloka
{प्रादुरासीत्ततः पुत्रः सृञ्जयस्याद्भुतप्रभः}
{प्रसन्नेनर्षिणा दत्तः कुबेरतनयोपमः}


\twolineshloka
{ततः सङ्गम्य पुत्रेण प्रीतिमानभवन्नृपः}
{ईजे च क्रतुभिः पुण्यैः समाप्तवरदक्षिणैः}


\twolineshloka
{अकृतार्थश्च भीतश्च न च सन्नाहकोविदः}
{अयज्वाचानपत्यश्च ततोऽसौ जीवितः पुनः}


\twolineshloka
{शूरो वीरः कृतार्थश्च प्रमथ्यारीन्सहस्रशः}
{अभिमन्युर्गतः स्वर्गं सङ्ग्रामेऽभिमुखाहतः}


\twolineshloka
{ब्रह्मचर्येण याँल्लोकान्प्रजया च श्रुतेन च}
{इष्टैश्च क्रतुभिर्यान्ति तांस्ते पुत्रोऽक्षयान्गतः}


\twolineshloka
{विद्वांसः कर्मभिः पुण्यैः स्वर्गमीहन्ति नित्यशः}
{न तु स्वर्गादयं लोकः काम्यते स्वर्गवासिभिः}


\twolineshloka
{तस्मात्स्वर्गगतं पुत्रमर्जुनस्य हतं रणे}
{न चेहानयितुं शक्यं किञ्चिदप्राप्यमीहितम्}


\twolineshloka
{यां योगिनो ध्यानविविक्तदर्शनाःप्रयान्ति यां चोत्तमयज्विनो जनाः}
{तपोभिरिद्धैरनुयान्ति यां तथातामक्षयां ते तनयो गतो गतिम्}


\twolineshloka
{अन्तात्पुनर्भावगतो विराजतेराजेव वीरो ह्यमृतात्मरश्मिभिः}
{तामैन्दवीमात्मतनुं द्विजोचितांगतोऽभिमन्युर्न स शोकमर्हति}


\twolineshloka
{एवं ज्ञात्वा स्थिरो भूत्वा मा शुचो धैर्यमाप्नुहि}
{जीवन्हि पुरुषः शोच्यो न तु स्वर्गगतोऽनघ}


\twolineshloka
{शोचतो हि महाराज अघमेवाभिवर्धते}
{तस्माच्छोकं परित्यज्य श्रेयसे प्रयतेद्बुधः}


\twolineshloka
{प्रहर्षं प्रीतिमानन्दं प्रियमुत्सिक्तचित्तताम्}
{एतदाहुर्बुधाः शौचमशौचं शोक उच्यते}


\twolineshloka
{एवं विद्वन्समुत्तिष्ठ प्रयतो भव मा शुचः}
{श्रुतस्ते सम्भवो मृत्योस्तपांस्यनुपमानि च}


\twolineshloka
{सर्वभूतसमत्वं च ब्रह्मणा चापि चोदितम्}
{सृञ्जयस्य तु पुत्रोऽसौ मृतः सञ्जीवितः श्रुतः}


\twolineshloka
{एवं विद्वन्महाराज मा शुचः साधयाम्यहम्}
{एतावदुक्त्वा भगवांस्तत्रैवान्तरधीयत}


\twolineshloka
{वागीशाने भगवति व्यासे व्यभ्रनभःप्रभे}
{गते मतिमतां श्रेष्ठे समाश्वास्य युधिष्ठिरम्}


\twolineshloka
{पूर्वेषां पार्थिवेन्द्राणआं महेन्द्रप्रतिमौजसाम्}
{न्यायाधिगतवित्तानां तां श्रुत्वा यज्ञसम्पदम्}


\twolineshloka
{सम्पूज्य मनसा विद्वान्विशोकोऽभूद्युधिष्ठिरः}
{पुनश्चाचिन्तयद्दीनः किंस्विद्वक्ष्ये धनञ्जयम्}


\chapter{अध्यायः ७२}
\twolineshloka
{धृतराष्ट्र उवाच}
{}


\twolineshloka
{अथ संशप्तकैः सार्धं युध्यमाने धनञ्जये}
{अभिमन्यौ हते चापि बाले बलवतां वरे}


\twolineshloka
{महर्षिसत्तमे याते व्यासे स तु युधिष्ठिरः}
{पाण्डवाः किमथाऽकार्षुः शोकोपहतचेतसः}


\fourlineindentedshloka
{कथं संशप्तकेभ्यो वा निवृत्तो वानरध्वजः}
{केन वा कथितस्तस्य प्रशान्तः सुतपावकः}
{एतन्मे शंस तत्त्वेन सर्वमेवेह सञ्जय ॥सञ्जय उवाच}
{}


\twolineshloka
{शृणु राजन्यथा तेभ्यो निवृत्तः कृष्णसारथिः}
{ततः सर्वाणि सैन्यानि दहन्कृष्णगतिर्यथा}


\twolineshloka
{सम्प्रयातेऽस्तमादित्ये संध्याकाल उपस्थिते}
{अयातस्यात्मशिबिरं निमित्तैरघशंसिभिः}


\threelineshloka
{यच्चासीन्मानसं तस्य यच्च कृष्णेन भाषितम्}
{यथा च कथितस्तस्य निहतः सुतपावकः}
{विस्तरेणैव मे सर्वं ब्रुवतः शृणु मारिष'}


\twolineshloka
{तस्मिन्नहनि निर्वृत्ते घोरे प्राणभृतां क्षये}
{आदित्येऽस्तङ्गते श्रीमान्सन्ध्याकाल उपस्थिते}


\twolineshloka
{व्यपयातेषु वासाय सर्वेषु भरतर्षभ}
{हत्वा संशप्तकव्रातान्दिव्यैरस्त्रैः कपिध्वजः}


\twolineshloka
{प्रायात्स्वशिबिरं जिष्णुर्जैत्रमास्थाय तं रथम्}
{गच्छन्नेव च गोविन्दं साश्रुकण्ठोऽभ्यभाषत}


\twolineshloka
{किं तु मे हृदयं त्रस्तं वाक्व सज्जति केशव}
{स्यन्दने नावतिष्ठामि गात्रैः सीदामि चाच्युत}


\twolineshloka
{अनिष्टं चैव मे श्लिष्टं हृदयान्नापसर्पति}
{भुविये दिक्षु चात्युग्रा उत्पातास्त्रासयन्ति माम्}


\threelineshloka
{बहुप्रकारा दृश्यन्ते सर्व एवाघशंसिनः}
{अपि स्वस्ति भवेद्राज्ञः सामात्यस्य गुरोर्मम ॥वासुदेव उवाच}
{}


\threelineshloka
{व्यक्तं शिवं सहभ्रातुर्धर्मराजस्य पाण्डव}
{मा शुचःकिञ्चिदेवान्यत्तत्रानिष्टं भविष्यति ॥सञ्जय उवाच}
{}


\twolineshloka
{ततः सन्ध्यामुपास्यैव वीरौ वीरावसादने}
{कथयन्तौ रणे वृत्तं प्रयातौ रथमास्थितौ}


\twolineshloka
{ततः स्वशिबिरं प्राप्तौ हतामित्रौ हतद्विषौ}
{वासुदेवोऽर्जुनश्चैव कृत्वा कर्म सुदुष्करम्}


\twolineshloka
{ध्वस्ताकारमिवालेख्यं संवीक्ष्य शिबिरं स्वकम्}
{बीभत्सुरब्रवीत्कृष्णमस्वस्थहृदयस्ततः}


\twolineshloka
{नदन्ति नाद्य तूर्याणि मङ्गल्यानि जनार्दन}
{मिश्रा दुन्दुभिनिर्धोषैः शङ्खाश्चाडम्बरैः सह}


\twolineshloka
{वीणा नैवाद्य वाद्यन्ते शम्यातालस्वनैः सह}
{मङ्गल्यानि च गीतानि न गायन्ति पठन्ति च}


\twolineshloka
{स्तुतियुक्तानि रम्याणि ममानीकेषु बन्दिनः}
{योधाश्चापि हि मां दृष्ट्वा निवर्तन्ते ह्यधोमुखाः}


\twolineshloka
{कर्माणि च यथापूर्वं कृत्वा नाभिवदन्ति माम्}
{अपि स्वस्ति भवेदद्य भ्रातृभ्यो मम माधव}


\twolineshloka
{न हि शुद्ध्यति मे भावो दृष्ट्वा स्वजनमाकुलम्}
{अपि पाञ्चालराजस्य विराटस्य च मानद}


\fourlineindentedshloka
{सर्वेषां चैव योधानां सामग्र्यं स्यान्ममाच्युत}
{न च मामद्य सौभद्रः प्रहृष्टो भ्रातृभिः सह}
{रणादायान्तमुचितं प्रत्युद्याति हसन्निव ॥सञ्जय उवाच}
{}


\twolineshloka
{एवं सङ्कथयन्तौ तौ प्रविष्टौ शिबिरं स्वकम्}
{ददृशाते भृशास्वस्थान्पाण्डवान्नष्टचेतसः}


\twolineshloka
{दृष्ट्वा भ्रातृंश्च पुत्रांश्च विमना वानरध्वजः}
{अपश्यंश्चैव सौभद्रमिदं वचनमब्रवीत्}


\twolineshloka
{मुखवर्णोऽप्रसन्नो वः सर्वेषामेव लक्ष्यते}
{न चाभिमन्युं पश्यामि न च मां प्रतिनन्दथ}


\twolineshloka
{मया श्रुतश्च द्रोणेन पद्मव्यूहो विनिर्मितः}
{न च वस्तस्य भेत्ताऽस्ति विना सौभद्रमर्भकम्}


\twolineshloka
{न चोपदिष्टस्तस्यासीन्मयानीकाद्विनिर्गमः}
{कच्चिन्न बालो युष्माभिः परानीकं प्रवेशितः}


\twolineshloka
{भित्त्वाऽनीकं महेष्वासः परेषां बहुभिर्युधि}
{कच्चिन्न निहतः सङ्ख्ये सौभद्रः परवीरहा}


\twolineshloka
{लोहिताक्षं महाबाहुं जातं सिंहमिवाद्रिषु}
{उपेन्द्रसदृशं ब्रूत कथमायोधने हतः}


\twolineshloka
{सुकुमारं महेष्वासं वासवस्यात्मजात्मजम्}
{सदा मम प्रियं ब्रूत कथमायोधने हतः}


\twolineshloka
{सुभद्रायाः प्रियं पुत्रं द्रौपद्याः केशवस्य च}
{अम्बायाश्च प्रियं नित्यं कोवधीत्कालमोहितः}


\twolineshloka
{सदृशो वृष्णिवीरस्य केशवस्य महात्मनः}
{विक्रमश्रुतमाहात्म्यैः कथमायोधने हतः}


\twolineshloka
{वार्ष्णेयीदयितं शूरं मया सततलालितम्}
{यदि पुत्रं न पश्यामि यास्यामि यमसादनम्}


\twolineshloka
{मृदुकुञ्चितकेशान्तं बालं बालमृगेक्षणम्}
{मत्तद्विरदविक्रान्तं सिंहपोतमिवोद्गतम्}


\twolineshloka
{स्मिताभिभाषिणं दान्तं गुरुवाक्यकरं सदा}
{बाल्येऽप्यतुलकर्माणं प्रियवाक्यममत्सरम्}


\twolineshloka
{महोत्साहं महाबाहुं दीर्घराजीवलोचनम्}
{भक्तानुकम्पिनं दान्तं न च नीचानुसारिणम्}


\twolineshloka
{कृतज्ञं ज्ञानसम्पन्नं कृतास्त्रभनिवर्तिनम्}
{युद्धाभिनन्दिनं नित्यं द्विषतां भयवर्धनम्}


\threelineshloka
{स्वेषां प्रियहिते युक्तं पितॄणां जयगृद्धिनम्}
{न च पूर्वं प्रहर्तारं सम्प्रमे नष्टसम्भ्रमम्}
{यदि पुत्रं न पश्यामि यास्यामि यमसादनम्}


\twolineshloka
{रथेषु गण्यमानेषु गणितं तं महारथम्}
{मयाऽध्यर्धगुणं सङ्ख्ये तरुणं बाहुशालिनम्}


\twolineshloka
{प्रद्युम्नस्य प्रियं नित्यं केशवस्य ममैव च}
{यदि पुत्रं न पश्यामि यास्यामि यमसादनम्}


\twolineshloka
{सुनसं सुललाटान्तं स्वक्षिभ्रूदशनच्छदम्}
{अपश्यतस्तद्वदनं का शान्तिर्हृदयस्य मे}


\twolineshloka
{तन्त्रीस्वनसुखं रम्यं पुंस्कोकिलसमध्वनिम्}
{अशृण्वतः स्वनं तस्य का शान्तिर्हृदयस्य मे}


\twolineshloka
{रूपं चाप्रतिमं तस्य त्रिदशैश्चापि दुर्लभम्}
{अपश्यतो हि वीरस्य का शान्तिर्हृदयस्य मे}


\twolineshloka
{अभिवादनदक्षं तं पितॄणां वचने रतम्}
{नाद्याहं यदि पश्यामि का शान्तिर्हृदयस्य मे}


\twolineshloka
{सुकुमारः सदा वीरो महार्हशयनोचितः}
{भूमावनाथवच्छेते नूनं नाथवतां वरः}


\twolineshloka
{शयानं समुपासन्ति यं पुरा परमस्त्रियः}
{तमद्य विप्रविद्धाङ्गमुपासन्त्यशिवाः शिवाः}


\twolineshloka
{यः पुरा बोध्यते सुप्तः सूतमागधबन्दिभिः}
{बोधयन्त्यद्य तं नूनं श्वापदा विकृतैः स्वनैः}


\twolineshloka
{छत्रच्छायासमुचितं तस्य तद्वदनं शुभम्}
{नूनमद्य रजोध्वस्तं रणरेणुः करिष्यति}


\twolineshloka
{हा पुत्रकावितृप्तस्य सततं पुत्रदर्शने}
{भाग्यहीनस्य कालेन यथा मे नीयसे बलात्}


\twolineshloka
{सा च संयमनी नूनं सदा सुकृतिनां गतिः}
{स्वभाभिर्मोहिता रम्या त्वयाऽत्यर्थं विराजते}


\threelineshloka
{नूनं वैवस्वतश्च त्वां वरुणश्च प्रियातिथिम्}
{शतक्रतुर्धनेशश्च प्राप्तमर्चन्त्यभीरुकम् ॥सञ्जय उवाच}
{}


\twolineshloka
{एवं विलप्य बहुधा भिन्नपोतो वगिग्यथा}
{दुःखेन महताऽविष्टो युधिष्ठिरमपृच्छत}


\twolineshloka
{`कथं त्वयि च भीमे च धृष्टद्युम्ने च जीवति}
{सात्यके शक्रविक्रान्ते सौभद्रो निहतः परैः'}


\twolineshloka
{कच्चित्स कदनं कृत्वा परेषां कुरुनन्दन}
{स्वर्गतोऽभिमुखः सङ्ख्ये युध्यमानो नरर्षभैः}


\twolineshloka
{स नूनं बहुभिर्यत्तैर्युध्यमानो नरर्षभैः}
{असहायः सहायार्थी मामनुध्यातवान्ध्रुवम्}


\twolineshloka
{पीड्यमानः शरैर्बालस्तात साध्वभिधाव माम्}
{इति विप्रलपन्मन्ये नृशंसैर्भुवि पातितः}


\twolineshloka
{अथवा मत्प्रसूतः स स्वस्रीयो माधवस्य च}
{सुभद्रायां च सम्भूतो न चैवं वक्तुमर्हति}


\twolineshloka
{वज्रसारमयं नूनं हृदयं सुदृढं मम}
{अपश्यतो दीर्घबाहुं रक्ताक्षं यन्न दीर्यते}


\twolineshloka
{कथं बाले महेष्वासा नृशंसा मर्मभेदिनः}
{स्वस्रीये वासुदेवस्य मम पुत्रेऽक्षिपञ्शरान्}


\twolineshloka
{यो मां नित्यमदीनात्मा प्रत्युद्गम्याभिनन्दति}
{उपयान्तं रिपून्हत्वा सोऽद्य मां किं न पश्यति}


\twolineshloka
{नूनं स पातितः शेते धरण्यां रुधिरोक्षितः}
{शोभयन्मेदिनीं गात्रैरादित्य इव पातितः}


\twolineshloka
{सुभद्रामनुशोचामि या पुत्रमपलायिनम्}
{रणे विनिहतं श्रुत्वा शोकार्ता वै विनङ्क्षति}


\twolineshloka
{सुभद्रा वक्ष्यते किं मामबिमन्युमपश्यती}
{द्रौपदीं चैव दुःखार्तां किं वा वक्ष्यामि तामहम्}


\twolineshloka
{वज्रसारमयं नूनं हृदयं यन्न यास्यति}
{सहस्रधा वधूं दृष्ट्वा रुदतीं शोककर्शिताम्}


\twolineshloka
{हृष्टानां धार्तराष्ट्राणां सिंहनादो मया श्रुतः}
{युयुत्सुश्चापि कृष्णेन श्रुतो वीरानुपालभन्}


\twolineshloka
{अशक्नुवन्तः पार्थस्य बालं हत्वा महाबलम्}
{किं न लज्जन्त्यधर्मज्ञाः पार्थिवा दृश्यतां बलम्}


\twolineshloka
{किं तयोर्विप्रियं कृत्वा केशवार्जुनयोर्मृधे}
{सिंहवन्नदथ प्रीताः शोककाल उपस्थिते}


\twolineshloka
{आगमिष्यति वः क्षिप्रं फलं पापस्य कर्मणः}
{अधर्मो हि कृतस्तीव्रः कथं स्यादफलश्चिरम्}


\twolineshloka
{इति तान्परिभाषन्वै वैश्यापुत्रो महामतिः}
{अपायाच्छस्त्रमुत्सृज्य कोपदुःखसमन्वितः}


\threelineshloka
{किमर्थमेतन्नाख्यातं त्वया कृष्ण रणे मम}
{अधक्ष्यं तानहं क्रूरांस्तदा सर्वान्महारथान् ॥सञ्जय उवाच}
{}


\threelineshloka
{पुत्रशोकार्दितं पार्थं ध्यायन्तं साश्रुलोचनम्}
{निगृह्य वासुदेवस्तं पुत्राधिभिरभिप्लुतम्}
{मैवमित्यब्रवीत्कृष्णस्तीव्रशोकसमन्वितम्}


\twolineshloka
{सर्वेषामेष वै पन्थाः शूराणामनिवर्तिनाम्}
{क्षत्रियाणां विशेषेण येषां नः शस्त्रजीविका}


\twolineshloka
{एषा वै युध्यमानानां शूराणामनिवर्तिनाम्}
{विहिता सर्वशास्त्रज्ञैर्गतिर्मतिमतां वर}


\twolineshloka
{ध्रुवं हि युद्धे मरणं शूरामामनिवर्तिनाम्}
{गतः पुण्यकृतां लोकानभिमन्युर्न संशयः}


\twolineshloka
{एतच्च सर्ववीराणां काङ्क्षितं भरतर्षभ}
{सङ्ग्रामेऽभिमुखा मृत्युं प्राप्नुयामेति मानद}


\twolineshloka
{स च वीरान्रणे हत्वा राजपुत्रान्महाबलान्}
{वीरैराकाङ्क्षितं मृत्युं सम्प्राप्तोऽभिमुखं रणे}


\twolineshloka
{मा शुचः पुरुषव्याघ्र पूर्वैरेष सनातनः}
{धर्मकृद्भिः कृतो धर्मः क्षत्रियाणां रणे क्षयः}


\twolineshloka
{इमे ते भ्रातरः सर्वे दीना भरतसत्तम}
{त्वयि शोकसमाविष्टे नृपाश्च सुहृदस्तव}


\threelineshloka
{एतांश्च वचसा साम्ना समाश्वासय मानद}
{विदितं वेदितव्यं ते न शोकं कर्तुमर्हसि ॥सञ्जय उवाच}
{}


\twolineshloka
{एवमाश्वासितः पार्थः कृष्णेनाद्भुतकर्मणा}
{ततोऽब्रवीत्तदा भ्रातॄन्सर्वान्पार्थः सगद्गदम्}


\twolineshloka
{स दीर्घबाहुः पृथ्वंसो दीर्घराजीवलोचनः}
{अभिमन्युर्यथा वृत्तः श्रोतुमिच्छाम्यहं तथा}


\twolineshloka
{सनागस्यन्दनहयान्सङ्ग्रामे निहतान्मया}
{क्षिप्रं द्रक्ष्यन्ति ते नूनं मम पुत्रं निहत्य वै}


\twolineshloka
{कथं च वः कृतास्त्राणां सर्वेषां शस्त्रपाणिनाम्}
{सौभद्रो निधनं गच्छेद्वज्रिणापि समागतः}


\twolineshloka
{यद्येवमहमज्ञास्यमशक्तान्रक्षणे मम}
{सूनोः पाण्डवपाञ्चालानगोप्स्यं तं महारणे}


\twolineshloka
{कथं च वो रथस्यानां शरवर्षाणि मुञ्चताम्}
{नीतोऽभिमन्युर्निधनं कदर्थीकृत्य वः परैः}


\twolineshloka
{अहो वः पौरुषं नास्ति न च वोऽस्ति पराक्रमः}
{यत्राभिमन्युः समरे पश्यतां वो निपातितः}


\twolineshloka
{आत्मानमेव गर्हेयं यदहं वै सुदुर्बलान्}
{युष्मानाज्ञाय निर्यातो भीरूनकृतनिश्चयान्}


\threelineshloka
{आहोस्विद्भूषणार्थाय वर्मशस्त्रायुधानि वः}
{वाचस्तु वक्तुं संसत्सु मम पुत्रमरक्षताम् ॥सञ्जय उवाच}
{}


\twolineshloka
{एवमुक्त्वा ततो वाक्यमतिष्ठद्वै वरासिमान्}
{न स्माशक्यत बीभत्सुः केनचित्प्रसमीक्षितुम्}


\threelineshloka
{तमन्तकमिव क्रुद्धं निःश्वसन्तं मुहुर्मुहुः}
{`महेन्द्रमिव तिष्ठन्तं वज्रोद्यतमहाभुजम्'}
{पुत्रशोकाभिसन्तप्तमश्रुपूर्णमुखं तदा}


\twolineshloka
{न भाषितुं शक्नुवन्ति द्रष्टुं वा सुहृदोऽर्जुनम्}
{अन्यत्र वासुदेवाद्वा ज्येष्ठाद्वा पाण्डुनन्दनात्}


\twolineshloka
{सर्वास्ववस्थासु हितावर्जुनस्य मनोनुगौ}
{बहुमानात्प्रियत्वाच्च तावेनं वक्तमर्हतः}


\twolineshloka
{ततस्तं पुत्रशोकेन भृशं पीडितमानसम्}
{राजीवलोचनं क्रुद्धं राजा वचनमब्रवीत्}


\chapter{अध्यायः ७३}
\twolineshloka
{युधिष्ठिर उवाच}
{}


\twolineshloka
{त्वयि याते महाबाहो संशप्तकबलं प्रति}
{प्रयत्नमकरोत्तीव्रमाचार्यो प्रहणे मम}


\twolineshloka
{व्यूढानीकं ततो द्रोणं यतमानं महामृधे}
{प्रतिव्यूह्य सहानीकाः प्रत्यरुध्म वयं बलात्}


\twolineshloka
{स वार्यमाणो रथिभिर्मयि चापि सुरक्षिते}
{अस्मानभिजगामाशु पीडयन्निशितैः शरैः}


\twolineshloka
{ते पीड्यमाना द्रोणेन द्रोणानीकं न शक्नुमः}
{प्रतिवीक्षितुमप्याजौ भेत्तुं तत्कुत एव तु}


\twolineshloka
{ततस्तमप्रतिरथमहं सौभद्रमब्रवम्}
{द्रोणानीकमिदं भिन्द्धि द्वारं सञ्जनयस्व नः}


\twolineshloka
{स तथा चोदितोऽस्माभिः सदश्व इव वीर्यवान्}
{असह्यमपि तं भारं वोढुमेवोपचक्रमे}


\twolineshloka
{स तवास्त्रोपदेशेन वीर्येण च समन्वितः}
{प्राविशत्तद्बलं बालः सुपर्ण इव सागरम्}


\twolineshloka
{तेऽनुयाता वयं वीरं सात्वतीपुत्रमाहवे}
{प्रवेष्टुकामास्तेनैव येन स प्राविशच्चमूम्}


\twolineshloka
{ततः सैन्धवको राजा क्षुद्रस्तात जयद्रथः}
{वरदानेन रुद्रस्य सर्वान्नः समवारयत्}


\twolineshloka
{ततो द्रोणः कृपः कर्णो द्रौणिः कौसल्य एव च}
{कृतवर्मा च सौभद्रं षड्रथाः पर्यवारयन्}


\twolineshloka
{परिवार्य तु तैः सर्वैर्युधि बालो महारथैः}
{यतमानः परं शक्त्या बहुभिर्विरथीकृतः}


\threelineshloka
{ततो दौःशासनिः क्षिप्रं तथा तैर्विरथीकृतम्}
{`संशयं परमं प्राप्तं पदातिनमवस्थितम्}
{गदाहस्तोऽभ्ययात्तूर्णं जिघांसुरपराजितम्}


\twolineshloka
{गदिनं त्वथ तं दृष्ट्वा वासवस्यात्मजात्मजः}
{स जग्राह गदां वीरो गदायुद्धविशारदः}


\twolineshloka
{गदामण्डलमार्गस्थौ सर्वक्षत्रस्य पश्यतः}
{तौ सम्प्रजग्मतुर्वीरावन्योन्यस्यान्तरैषणौ}


\twolineshloka
{तावन्योन्यं गदाग्राभ्यां ताडितौ युद्धदुर्मदौ}
{इन्द्रध्वजाविवोत्सृष्टौ गतसत्वौ महींगतौ}


\twolineshloka
{दौःशासनिरथोत्थाय कुरूणां कीर्तिवर्धनः}
{उत्तिष्ठमानं सौभद्रं गदया मूर्ध्न्यताडयत्}


\twolineshloka
{गदावेगेन महता व्यायामेन च मोहितः}
{विमना न्यपतद्भूमौ सौभद्रः परवीरहा}


\twolineshloka
{स तु हत्वा सहस्राणि द्विपाश्वरथपत्तिनाम्'}
{राजपुत्रशतं चाग्र्यं वीरांश्चालक्षितान्बहून्}


\twolineshloka
{बृहद्बलं च राजानं स्वर्गेण समयोजयत्}
{`गतः सुकृतिनां लोकान्ये च स्वर्गजितां शुभाः}


\twolineshloka
{अदीनस्त्रासयञ्छत्रून्नन्दयित्वा च बान्धवान्}
{असकृन्नाम विश्राव्य पितॄणां मातुलस्य च'}


\twolineshloka
{वीरो दिष्टान्तमापन्नः शोचयन्वान्धवान्बहून्}
{`ततः स्म शोकसन्तप्ता भवताऽद्य समेयुषः'}


\threelineshloka
{एतावदेव निर्वृत्तमस्माकं शोकवर्धनम्}
{स चैवं पुरुषव्याघ्रः स्वर्गलोकमवाप्तवान् ॥सञ्जय उवाच}
{}


\twolineshloka
{ततोऽर्जुनो वचः श्रुत्वा धर्मराजेन भाषितम्}
{हा पुत्र इति निःश्वस्य व्यथितो न्यपतद्भुवि}


\twolineshloka
{विषण्णवदनाः सर्वे परिवार्य धनञ्जयम्}
{नेत्रैरनिमिषैर्दीनाः प्रत्यवैक्षन्परस्परम्}


\twolineshloka
{प्रतिलभ्य ततः संज्ञां वासविः क्रोधमूर्च्छितः}
{कम्पमानो ज्वरेणेव निःश्वसंश्च मुहुर्मुहुः}


\fourlineindentedshloka
{पाणिं पाणौ विनिष्पिष्य दन्तान्कटकटाय्य च}
{`त्रिशिखां भ्रुकुटीं कृत्वा क्रोधसंरक्तलोचनः'}
{उन्मत्त इव विप्रेक्षन्निदं वचनमब्रवीत् ॥अर्जुन उवाच}
{}


\twolineshloka
{सत्यं वः प्रतिजानामि श्वोऽस्मि हन्ताऽजयद्रथम्}
{न चेद्वधभयाद्भीतो धार्तराष्ट्रान्प्रहास्यति}


\twolineshloka
{न चास्माञ्शरणं गच्छेत्कृष्णं वा पुरुषोत्तमम्}
{भवन्तं वा महाराज श्वोऽस्मि हन्ता जयद्रथम्}


\twolineshloka
{धार्तराष्ट्रप्रियकरं मयि विस्मृतसौहृदम्}
{पापं बालवधे हेतुं श्वोऽस्मि हन्ता जयद्रथम्}


\twolineshloka
{रक्षमाणाश्च तं सङ्ख्ये ये मां योत्स्यन्ति केचन}
{अपि द्रोणकृपौ राजञ्छादयिष्यामि ताञ्छरैः}


\twolineshloka
{यद्येतदेवं सङ्ग्रामे न कुर्यां पुरुषर्षभाः}
{मा स्म पुण्यकृतां लोकान्प्राप्नुयां शूरसम्मतान्}


\twolineshloka
{ये लोका मातृहन्तॄणां ये चापि पितृघातिनाम्}
{गुरुदारगतानां ये पिशुनानां च ये सदा}


\twolineshloka
{साधूनसूयतां ये च ये चापि परिवादिनाम्}
{ये च निक्षेपहन्तॄणां ये च विश्वासघातिनाम्}


\twolineshloka
{भुक्तपूर्वां स्त्रियं ये च निन्दतामघशंसिनाम्}
{ब्रह्मघ्नानां च ये लोका ये च गोघातिनामपि}


\threelineshloka
{पायसं वा यवान्नं वा शाकं कृसरमेव वा}
{संयावापूपमांसानि ये च लोका वृथाश्नताम्}
{तानन्हायाधिगच्छेयं न चेद्धन्यां जयद्रथम्}


\twolineshloka
{वेदाध्यायिनमत्यर्थं संशितं वा द्विजोत्तमम्}
{अवमन्यमानो यान्याति वृद्धान्साधून्गुरूंस्तथा}


\threelineshloka
{स्पृशतो ब्राह्मणं गां च पादेनाग्निं च या भवेत्}
{याऽप्सु श्लेष्म पुरीषं च मूत्रं वा मुञ्चतां गतिः}
{तां गच्छेयं गतिं कष्टां न चेद्धन्यां जयद्रथम्}


\twolineshloka
{नग्नस्य स्नायमानस्य या च वन्ध्याऽतिथेर्गतिः}
{उत्कोचिनां मृषोक्तीनां वञ्चकानां च या गतिः}


\twolineshloka
{आत्मापहारिणां या च या च मिथ्याभिशंसिनाम्}
{भृत्यैः सन्दृश्यमानानां पुत्रदाराश्रितैस्तथा}


\twolineshloka
{असंविभज्य क्षुद्राणां या गतिर्मिष्टमश्नताम्}
{तां गच्छेयं गतिं घोरां न चेद्धन्यां जयद्रथम्}


\twolineshloka
{संश्रितं चापि यस्त्यक्त्वा साधुं तद्वचने रतम्}
{न बिभर्ति नृसंसात्मा निन्दते चोपकारिणम्}


\twolineshloka
{अर्हते प्रातिवेश्याय श्राद्धं यो न ददाति च}
{अनर्हेभ्यश्च यो दद्याद्वृषलीपतये तथा}


\twolineshloka
{मद्यपो भिन्नमर्यादः कृतघ्नो भर्तृनिन्दकः}
{तेषां गतिमियां क्षिप्रं न चेद्धान्यां जयद्रथम्}


\twolineshloka
{भुञ्जानानां तु सव्येन उत्सङ्गे चापि खादताम्}
{पालाशमासनं चैव तिन्दुकैर्दन्तधावनम्}


\twolineshloka
{ये चावर्जयतां लोकाः स्वपतां च तथोषसि}
{शीतभीताश्च ये विप्रा रणभीताश्च क्षत्रियाः}


\twolineshloka
{एककूपोदकग्रामे वेदध्वनिविवर्जिते}
{षण्मासं तत्र वसतां तथा शास्त्रं विनिन्दताम्}


\twolineshloka
{दिवामैथुनिनां चापि दिवसेषु च शेरते}
{अगारदाहिनां चैव गरदानां च ये मताः}


\twolineshloka
{अग्न्यातिथ्यविहीनाश्च गोपानेषु च विघ्नदाः}
{रजस्वलां सेवयन्तः कन्यां शुल्केन दायिनः}


\twolineshloka
{या च वै बहुयाजिनां ब्राह्मणानां श्ववृत्तिनाम्}
{आस्यमैथुनिकानां च ये दिवामैथुने रताः}


\twolineshloka
{ब्राह्मणस्य प्रतिश्रुत्य यो वै लोभाद्ददाति न}
{तेषां गतिं गमिष्यामि श्वो न हन्यां जयद्रथम्}


\twolineshloka
{धर्मादपेता ये चान्ये मया नात्रानुकीर्तिताः}
{ये चानुकीर्तितास्तेषां गतिं क्षिप्रमवाप्नुयाम्}


\twolineshloka
{यदि व्युष्टामिमां रात्रिं श्वो न हन्यां जयद्रथम्}
{इमां चाप्यपरां भूयः प्रतिज्ञां मे निबोधत}


\twolineshloka
{यद्यस्मिन्न हत्ते पापे सूर्योऽस्तमुपयास्यति}
{इहैव सम्प्रवेष्टाऽहं ज्वलितं जातवेदसम्}


\twolineshloka
{असुरसुरमनुष्याः पक्षिणो वोरगा वापितृरजनिचरा वा ब्रह्मदेवर्षयो वा}
{चरमचरमपीदं यत्परं चापि तस्मा--त्तदपि मम रिपुं तं रक्षितुं नैव शक्ताः}


\threelineshloka
{यदि विशति रसातलं तदग्र्यंवियदपि देवपुरं दितेः पुरं वा}
{तदपि शरशतैरहं प्रभातेभृशमभिमन्युरिपोः शिरोऽभिहर्ता ॥सञ्जय उवाच}
{}


\twolineshloka
{एवमुक्त्वा विचिक्षेप गाण्डीवं सव्यदक्षिणम्}
{तस्य सर्वमतिक्रम्य धनुःशब्दोऽस्पृशद्दिवम्}


\twolineshloka
{अर्जुनेन प्रतिज्ञाते पाञ्चजन्यं जनार्दनः}
{प्रदध्मौ तत्र सङ्क्रुद्धो देवदत्तं च फल्गुनः}


\twolineshloka
{स पाञ्चजन्योऽच्युतवक्त्रवायुनाभृशं सुपूर्णोदरनिःसृतध्वनिः}
{जगत्सपातालवियद्दिगीश्वरंप्रकम्पयामास युगात्यये यथा}


\twolineshloka
{ततो वादित्रघोषाश्च प्रादुरासन्सहस्रशः}
{सिंहनादश्च पाण्डूनां प्रतिज्ञाते महात्मना}


\twolineshloka
{`प्रानृत्यदिव गाण्डीवं शरास्तूणीगता मुदा}
{निराक्रामन्निव तदा स्वयमेव मृधैषिणः}


\twolineshloka
{भीमसेनः सुसंहृष्टः प्रत्यभाषत भारत}
{धनञ्जयमभिप्रेक्ष्य हर्षगद्गदया गिरा}


\twolineshloka
{प्रतिज्ञोद्भवशब्देन कृष्णशङ्खस्वनेन च}
{निहतो धार्तराष्ट्रोऽयं सानुबन्धः सुयोधनः}


\threelineshloka
{अथ मृदिततमाग्र्यदाममाल्यंतव सुतशोकमयं च रोषजातम्}
{व्यपनुदति महाप्रभावसेन--न्नरवर वाक्यमिदं महार्थमिष्टम् ॥सञ्जय उवाच}
{}


\twolineshloka
{अथ शङ्ख्यैश्च भेरीभिः पणवैः सैनिकास्तथा}
{ससूतमागधा जिष्णुं स्तुतिभिः समपूजयन्}


\twolineshloka
{तदा भीमं बलं सर्वे तेन नादेन मोहितम्}
{तावकं तन्महाराज विषण्णं समपद्यत'}


\chapter{अध्यायः ७४}
\twolineshloka
{सञ्जय उवाच}
{}


\twolineshloka
{श्रुत्वा तु तं महाशब्दं पाण्डूनां पुत्रगृद्धिनाम्}
{चारैः प्रवेदितस्त्रस्तः समुत्थाय जयद्रथः}


\threelineshloka
{शोकसम्मूढहृदयो दुःखेनाभ्याहतो भृशम्}
{मज्जमान इवागाधे विपुले शोकसागरे}
{जगाम समितिं राज्ञां सैन्धवो विमृशन्बहु}


\twolineshloka
{स तेषां नरदेवानां सकाशे पर्यदेवयत्}
{अभिमन्युपितुर्भीतः सव्रीडो वाक्यमब्रवीत्}


\twolineshloka
{योऽसौ पाण्डोः किल क्षेत्रे जातः शक्रेण कामिना}
{स निनीषति दुर्बुद्धिर्मां किलैकं यमक्षयम्}


\twolineshloka
{स्वस्ति वोऽस्तु गमिष्यामि स्वगृहं जीवितेप्सया}
{}


\twolineshloka
{अथवास्त्रप्रतिबलास्त्रात मां क्षत्रियर्षभाः}
{पार्थेन प्रार्थितं वीरास्ते सन्दत्त ममाभयम्}


\twolineshloka
{द्रोणदुर्योधनकृपाः कर्णमद्रेशबाह्लिकाः}
{दुःशासनादयः शक्तास्त्रातुमप्यन्तकार्दितम्}


\twolineshloka
{किमङ्ग पुनरेकेन फल्गुनेन जिघांसता}
{न त्रास्यन्ति भवन्तो मां समस्ताः पतयः क्षितेः}


\twolineshloka
{प्रतिज्ञां पाण्डवेयानां श्रुत्वा मम महद्भयम्}
{सीदन्ति मम गात्राणि मुमूर्षोरिव पार्थिवाः}


\twolineshloka
{वधो नूनं प्रतिज्ञातो मम गाण्डीवधन्वना}
{तथाहि हृष्टाः क्रोशन्ति शोककालेऽपि पाण्डवाः}


\twolineshloka
{तन्न देवा न गन्धर्वा नासुरोरगराक्षसाः}
{उत्सहन्तेऽन्यथा कर्तुं कुत एव नराधिपाः}


\threelineshloka
{तस्मान्मामनुजानीत भद्रं वोऽस्तु नरर्षभाः}
{अदर्शनं गमिष्यामि न मां द्रक्ष्यन्ति पाण्डवाः ॥सञ्जय उवाच}
{}


\twolineshloka
{एवं विलपमानं तं भयाद्व्याकुलचेतसम्}
{आत्मकार्यगरीयस्त्वाद्राजा दुर्योधनोऽब्रवीत्}


\twolineshloka
{न भेतव्यं नरव्याघ्र को हि त्वां पुरुषर्षभ}
{मध्ये क्षत्रियवीराणां तिष्ठन्तं प्रार्थयेद्युधि}


\twolineshloka
{`एतेषां नरदेवानां मत्तमातङ्गगामिनाम्}
{सङ्घातमुपयातानामपि विभ्येत्पुरन्दरः'}


\twolineshloka
{अहं वैकर्तनः कर्णश्चित्रसेनो विविंशतिः}
{भूरिश्रवाः शलः शल्यो वृषसेनो दुरासदः}


\twolineshloka
{पुरुमित्रो जयो भोजः काम्भोजश्च सुदक्षिणः}
{सत्यव्रतो महाबाहुर्विकर्णो दुर्मुखश्च ह}


\twolineshloka
{दुःशासनः सुबाहुश्च कालिङ्गश्चाप्युदायुधः}
{विन्दानुविन्दावावन्त्यौ द्रोणो द्रौणिश्च सौबलः}


\threelineshloka
{`मायावी बलवाञ्छूरो राक्षसश्चाप्यलम्बुसः'}
{एते चान्ये च बहवो नानाजनपदेश्वराः}
{ससैन्यास्त्वाभिगोप्स्यन्ति व्येतु ते मानसो ज्वरः}


\twolineshloka
{त्वं चापि रथिनां श्रेष्ठः स्वयं शूरोऽमितद्युते}
{स कथं पाण्डवेयेभ्यो भयं पश्यसि वैन्धव}


\threelineshloka
{अक्षौहिण्यो महावीर्य मदीयास्तव रक्षणे}
{यत्ता योत्स्यन्ति माभैस्त्वं सैन्धव व्येतु ते भयम् ॥सञ्जय उवाच}
{}


\twolineshloka
{एवमाश्वासितो राजन्पुत्रेण तव सैन्धवः}
{दुर्योधनेन सहितो द्रोणं रात्रावुपागमत्}


\twolineshloka
{उपसङ्गृह्य चरणौ द्रोणाय स विशाम्पतिः}
{उपोपविश्य प्रणतः पर्यपृच्छदिदं वचः}


\twolineshloka
{निमित्ते दूरपातित्वे लघुत्वे दृढवेधने}
{मम ब्रवीतु भगवान्विशेषं फल्गुनस्य च}


\threelineshloka
{विद्याविशेषमिच्छामि ज्ञातुमाचार्य तत्त्वतः}
{अर्जुनस्यात्मनश्चैव याथातथ्यं प्रचक्ष्व मे ॥द्रोण उवाच}
{}


\twolineshloka
{सममाचार्यकं तात तव चैवार्जुनस्य च}
{योगाद्दुःखोषितत्वाच्च तस्मात्त्वत्तोऽधिकोऽर्जुनः}


\twolineshloka
{न तु ते युधि सन्त्रासः कार्यः पार्थात्कथञ्चन}
{अहं हि रक्षिता तात भयात्त्वां नात्र संशयः}


\twolineshloka
{न हि मद्बाहुगुप्तस्य प्रभवन्त्यमरा अपि}
{व्यूहयिष्यामि तं व्यूहं यं पार्थो न तरिष्यति}


\twolineshloka
{तस्माद्युध्यस्व माभैस्त्वं स्वधर्ममनुपालय}
{पितृपैतामहं मार्गमनुयाहि महारथ}


\twolineshloka
{अधीता विधिवद्वेदा अग्नयः सुहुतास्त्वया}
{इष्टं च बहुभिर्यज्ञैर्न ते मृत्युर्भयङ्करः}


\twolineshloka
{दुर्लभं मानुषैर्मन्दैर्महाभाग्यमवाप्य तु}
{भुजवीर्यार्जिताँल्लोकान्दिव्यान्प्राप्स्यस्यनुत्तमान्}


\twolineshloka
{कुरवः पाण्डवाश्चैव वृष्णयोऽन्ये च मानवाः}
{अहं च सहपुत्रेण अध्रुवा इति चिन्त्यताम्}


\twolineshloka
{पर्यायेण वयं सर्वे कालेन बलिना हताः}
{परलोकं गमिष्यामः स्वैःस्वैः कर्मभिरन्विताः}


\twolineshloka
{तपस्तप्त्वा तु याँल्लोकान्प्राप्नुवन्ति तपस्विनः}
{क्षत्रधर्माश्रिताः वीराः क्षत्रियाः प्राप्नुवन्ति तान्}


\twolineshloka
{एवमाश्वासितो राजा भारद्वाजेन सैन्धवः}
{अपानुदद्भयं पार्थाद्युद्धाय च मनो दधे}


\twolineshloka
{ततः प्रहर्षः सैन्यानां तस्य चासीद्विशाम्पते}
{वादित्राणां ध्वनिश्चोग्रः सिंहनादरवैः सह}


\chapter{अध्यायः ७५}
\twolineshloka
{सञ्जय उवाच}
{}


\twolineshloka
{प्रतिज्ञाते तु पार्थेन सिन्धुराजवधे तदा}
{वासुदेवो महाबाहुर्धनञ्जयमभाषत}


\twolineshloka
{भ्रातॄणां मतमज्ञाय त्वया वाचा प्रतिश्रुतम्}
{सैन्धवं चास्मि हन्तेति तत्साहसमिदं कृतम्}


\twolineshloka
{असम्मन्त्र्य मया सार्धमतिभारोऽयमुद्यतः}
{कथं तु सर्वलोकस्य नावहास्या भवेमहि}


\twolineshloka
{धार्तराष्ट्रस्य शिबिरे ये तु प्रणिहिताश्चराः}
{त इमे शीघ्रमागम्य प्रवृत्तिं वेदयन्ति नः}


\twolineshloka
{त्वया वै सम्प्रतिज्ञाते सिन्धुराजवधे प्रभो}
{सिंहनादः सवादित्रः सुमहानिह तैः श्रुतः}


\twolineshloka
{तेन शब्देन वित्रस्ता धार्तराष्ट्राः ससैन्धवाः}
{नाकस्मात्सिंहनादोऽयमिति मत्वा व्यवस्थिताः}


\twolineshloka
{सुमहाञ्शब्दसम्पातः कौरवाणां महाभुज}
{आसीन्नागाश्वपत्तीनां रथघोषश्च भैरवः}


\twolineshloka
{अभिमन्योर्वधं श्रुत्वा ध्रुवमार्तो धनञ्जयः}
{रात्रौ निर्यास्यति क्रोधातिति मत्त्वा व्यवस्थिताः}


\twolineshloka
{तैश्चारेभ्य इयं कृत्स्ना श्रुता सत्यवतस्तव}
{प्रतिज्ञा सिन्धुराजस्य वधे राजीवलोचन}


\twolineshloka
{ततो विमनसः सर्वे त्रस्ताः क्षुद्रमृगा इव}
{आसन्सुयोधनामात्याः स च राजा जयद्रथः}


\twolineshloka
{अथोत्थाय सहामात्यैर्दीनः स्वशिविरात्किल}
{अयात्सौवीरसिन्धूनामीश्वरो राजसंसदम्}


\twolineshloka
{स मन्त्रकाले संमन्त्र्य सर्वां नैश्रेयसीं क्रियाम्}
{सुयोधनमिदं वाक्यमब्रवीद्राजसंसदि}


\twolineshloka
{मामसौ पुत्रहन्तेति श्वोऽभियाता धनञ्जयः}
{प्रतिज्ञातो हि सेनाया मध्ये तेन वधो मम}


\twolineshloka
{तां न देवा न गन्धर्वा नासुरोरगराक्षसाः}
{उत्सहन्तेऽन्यथा कर्तुं प्रतिज्ञां सव्यसाचिनः}


\twolineshloka
{ते मां रक्षत सङ्ग्रामे मा वो मूर्ध्नि धनञ्जयः}
{पदं कृत्वाऽऽप्नुयाल्लक्ष्यं युक्तं प्रतिविधीयताम्}


\twolineshloka
{अथ रक्षा न मे सम्यक्क्रियते कुरुनन्दन}
{अनुजानीहि मां राजन्गमिष्यामि गृहान्प्रति}


\twolineshloka
{एवमुक्तस्त्ववाक्शीर्षो विमनाः ससुयोधनः}
{श्रुत्वा तं समयं तस्य ध्यानमेवान्वपद्यत}


\twolineshloka
{तमार्तमभिसम्प्रेक्ष्य राजा किल स सैन्धवः}
{मृदु चात्महितं चैव सापेक्षमिदमुक्तवान्}


\twolineshloka
{नेह पश्यामि भवतां तथावीर्यं धनुर्धरम्}
{योऽर्जुनस्यास्त्रमस्त्रेण प्रतिहन्यान्महाहवे}


\twolineshloka
{वासुदेवसहायस्य गाण्डीवं धुन्वतो धनुः}
{कोऽर्जुनस्याग्रतस्तिष्ठेत्साक्षादपि शतक्रतुः}


\twolineshloka
{महेश्वरोऽपि पार्थेन श्रूयते योधिनः पुरा}
{पदातिना महावीर्यो गिरौ हिमवति प्रभुः}


\twolineshloka
{दानवानां सहस्राणि हिरण्यपुरवासिनाम्}
{जघानैकरथेनैव देवराजप्रचोदितः}


\twolineshloka
{समायुक्तो हि कौन्तेयो बासुदेवेन धीमता}
{सामरानपि लोकांस्त्रीन्हन्यादिति मतिर्मम}


\threelineshloka
{सोऽहमिच्छाम्यनुज्ञातुं रक्षितुं वा महात्मना}
{द्रोणेन सहपुत्रेण वीरेण यदि मन्यसे ॥श्रीभगवानुवाच}
{}


\twolineshloka
{स राज्ञा स्वयमाचार्यो भृशमत्रार्थितोऽर्जुन}
{संविधानं च विहितं रथाश्च किल सज्जिताः}


\twolineshloka
{कर्णो भूरिश्रवा द्रौणिर्वृषसेनश्च दुर्जयः}
{कृपश्च मद्रराजश्च षडेतेऽस्य पुरोगमाः}


\twolineshloka
{शकटः पद्मकश्चार्धो व्यूहो द्रोणेन निर्मितः}
{पद्मकर्णिकमध्यस्थः सूची पार्श्वे जयद्रथः}


\twolineshloka
{स्थास्यते रक्षिते वीरैः सिन्धुराट् स सुदुर्मदः}
{धनुष्यस्त्रे च वीर्ये च प्राणे चैव तथौरसे}


\twolineshloka
{अविषह्यतमा ह्येते निश्चिताः पार्थ षड्रथाः}
{एतानजित्वा सगणान्नैव प्राप्यो जयद्रथः}


\twolineshloka
{तेषामेकैकशो वीर्यं षण्णां त्वमनुचिन्तय}
{सहिता हि नरव्याघ्र न शक्या जेतुमञ्जसा}


\twolineshloka
{भूयस्तु मन्त्रयिष्यामि नीतिमात्महिताय वै}
{मन्त्रज्ञैः सचिवैः सार्धं सुहृद्भिः कार्यसिद्धये}


\chapter{अध्यायः ७६}
\twolineshloka
{अर्जुन उवाच}
{}


\twolineshloka
{षड्रथान्धार्तराष्ट्रस्य मन्यसे यान्बलाधिकान्}
{तेषां वीर्यं ममार्धेन न तुल्यमिति मे मतिः}


\twolineshloka
{अस्त्रमस्त्रेण सर्वेषामेतेषां मधुसूदन}
{मया द्रक्ष्यसि निर्भिन्नं जयद्रथवधैषिणा}


\twolineshloka
{द्रोणस्य मिषतश्चाहं सगणस्य विपश्चितः}
{मूर्धानं सिन्धुराजस्य पातयिष्यामि भूतले}


\twolineshloka
{यदि साध्याश्च रुद्राश्च वसवश्च सहाश्विनः}
{मरुतश्च सहेन्द्रेण विश्वेदेवाः सहेश्वराः}


\twolineshloka
{पितरः सहगन्धर्वाः सुपर्णाः सागराद्रयः}
{द्यौर्वियत्पृथिवी चेयं दिशश्च सदिगीश्वराः}


\twolineshloka
{ग्राम्यारण्यानि भूतानि स्थावराणि चराणि च}
{त्रातारः सिन्धुराजस्य भवन्ति मधुसूदन}


\twolineshloka
{तथापि बाणैर्निहतं श्वो द्रष्टासि रणे मया}
{सत्येन च शपे कृष्ण तथैवायुधमालभे}


\twolineshloka
{यस्य गोप्ता महेष्वासो द्रोणः पापस्य दुर्मतेः}
{तमेव प्रथमं द्रोणमभियास्यामि केशव}


\twolineshloka
{तस्मिन्युद्धमिदं बद्धं मन्यते स सुयोधनः}
{तस्मात्तस्यैव सेनाग्रं भित्त्वा यास्यामि सैन्धवम्}


\twolineshloka
{द्रष्टासि श्वो महेष्वासान्नाराचैस्तिग्मतेजितैः}
{शृङ्गाणीव गिरेर्वज्रैर्दार्यमाणान्मया युधि}


\twolineshloka
{नरनागाश्वदेहेभ्यो विस्रविष्यति शोणितम्}
{पतद्ध्यः पतितेभ्यश्च विभिन्नेभ्यः शितैः शरैः}


\twolineshloka
{गाण्डीवप्रेषिता बाणा मनोऽनिलसमा जवे}
{नृनागाश्वान्विदेहासून्कर्तारश्च सहस्रशः}


\twolineshloka
{शक्राद्भीष्मात्कृपाद्द्रोणाद्देवाद्रुद्राच्च यन्मया}
{उपात्तमस्त्रं घोरं तद्दष्टारोऽत्र नरा युधि}


\twolineshloka
{ब्राह्मेणास्त्रेण चास्त्राणि हन्यमानानि संयुगे}
{मया द्रष्टाऽसि सर्वेषां सैन्धवस्याभिरक्षिणाम्}


\twolineshloka
{शरवेगसमुत्कृत्तै राज्ञां केशव मूर्धभिः}
{आस्तीर्यमाणां पृथिवीं द्रष्टाऽसि श्वो मया युधि}


\twolineshloka
{क्रव्यादांस्तर्पयिष्यामि द्रावयिष्यामि शात्रवान्}
{सुहृदो नन्दयिष्यामि प्रमथिष्यामि सैन्धवम्}


\twolineshloka
{बह्वागस्कृत्कुसंबन्धी पापदेशसमुद्भवः}
{मया सैन्धवको राजा हतः स्वाञ्शोचयिष्यति}


\twolineshloka
{सर्वे क्षीरान्नभोक्तारः पापाचारा रणाजिरे}
{मया सराजका बाणैर्हता नश्यन्ति सैन्धवाः}


\twolineshloka
{तथा प्रभाते कर्ताऽस्मि यथा कृष्ण सुयोधनः}
{नान्यं धनुर्धरं लोके मंस्यते मत्समं युधि}


\twolineshloka
{`गाण्डीवं च धनुर्दिव्यं योद्धारं च धनञ्जम्}
{यन्तारं च हृषीकेशं कोऽतिवर्तेत संयुगे'}


\twolineshloka
{तव प्रसादाद्भगवन्किंनावाप्तं रणे मम}
{अविषह्यं हृषीकेश किं जानन्मां विगर्हसे}


\twolineshloka
{यथा लक्ष्म स्थिरं चन्द्रे समुद्रे च यथा जलम्}
{एवमेतां प्रतिज्ञां मे सत्यां विद्धि जनार्दन}


\twolineshloka
{मावमंस्था ममास्त्राणि मावमंस्थाश्च गाण्डिवम्}
{मावमंस्था बलं बाह्वोर्मावमंस्था धनञ्जयम्}


\twolineshloka
{गाण्डीवं च धनुर्दिव्यं योद्धा चाहं नरर्षभ}
{त्वं च यन्ता हृषीकेश किन्नु स्यादजितं मया}


\twolineshloka
{यथाभियाय सङ्ग्रमं न जितो वै जयामि च}
{तेन सत्येन सङ्ग्रामे हतं विद्धि जयद्रथम्}


\twolineshloka
{`त्वं च माधव सर्वं तत्तथा प्रतिविधास्यसि}
{यथा रिपूणां मिषतां प्रमथिष्यामि सैन्धवम्'}


\threelineshloka
{ध्रुवं वै ब्राह्मणे सत्यं ध्रुवा साधुषु सन्नतिः}
{श्रीर्ध्रुवाऽपि च दक्षेषु ध्रुवो नारायणे जयः ॥सञ्जय उवाच}
{}


\twolineshloka
{एवमुक्त्वा हृषीकेशं स्वयमात्मानमात्मना}
{सन्दिदेशार्जुनो नर्दन्वासविः केशवं प्रभुम्}


\twolineshloka
{यथा प्रभातां रजनीं कल्पितः स्याद्रथोत्तमः}
{तथा कार्यं त्वया कृष्ण प्रतिज्ञा स्याद्यथा ध्रुवा}


\chapter{अध्यायः ७७}
\twolineshloka
{सञ्जय उवाच}
{}


\twolineshloka
{तां निशां दुःखशोकार्तौ निःश्वसन्ताविवोरगौ}
{निद्रां नैवोपलेभाते वासुदेवधनञ्जयौ}


\twolineshloka
{नरनारायणौ क्रुद्धौ ज्ञात्वा देवाः सवासवाः}
{व्यथिताश्चिन्तयामासुः किंस्विदेतद्भविष्यति}


\twolineshloka
{ववुश्च दारुणा वाता रूक्षा घोराभिशंसिनः}
{सकबन्धस्तथाऽऽदित्ये परिघः समदृश्यत}


\twolineshloka
{शुष्काशन्यश्च निष्पेतुः सनिर्घाताः सविद्युतः}
{चचाल चापि पृथिवी सशैलवनकानना}


\twolineshloka
{चुक्षुभुश्च महाराज सागारा मकरालयाः}
{प्रतिस्रोतःप्रवृत्ताश्च तथा गन्तुं समुद्रगाः}


\twolineshloka
{रथाश्वनरनागानां प्रवृत्तमधरोत्तरम्}
{क्रव्यादानां प्रमोदार्थं यमराष्ट्रविवृद्धये}


\twolineshloka
{वाहनानि शकृन्मूत्रे मुमुचू रुरुदुश्च ह}
{तान्दृष्ट्वा दारुणान्सर्वानुत्पाताँल्लोमहर्षणान्}


\twolineshloka
{सर्वे ते व्यथिताः सैन्यास्त्वदीया भरतर्षभा}
{श्रुत्वा महाबलस्योग्रां प्रतिज्ञां सव्यसाचिनः}


\twolineshloka
{अथ कृष्णं महाबाहुरब्रवीत्पाकशासनिः}
{आश्वासय सुभद्रां त्वं भगिनीं स्नुषया सह}


\twolineshloka
{स्नुषां चास्या वयस्याश्च विशोकाः कुरु माधव}
{साम्ना सत्येन युक्तेन वचसाऽऽश्वासय प्रभो}


\threelineshloka
{ततोऽर्जुनगृहं गत्वा वासुदेवः सुदुर्मनाः}
{भगिनीं पुत्रशोकार्तामाश्वासयत दुःखिताम् ॥वासुदेव उवाच}
{}


\twolineshloka
{मा शोकं कुरु वार्ष्णेयि कुमारं प्रति सस्नुषा}
{सर्वेषां प्राणिनां भीरु निष्ठैषा कालनिर्मिता}


\twolineshloka
{कुले जातस्य वीरस्य क्षत्रियस्य विशेषतः}
{सदृशं मरणं ह्येतत्तव पुत्रस्य मा शुचः}


\twolineshloka
{दिष्ट्या महारथो धीरः पितुस्तुल्यपराक्रमः}
{क्षात्रेण विधिना प्राप्तो वीराभिलषितां गतिम्}


\twolineshloka
{जित्वा सुबहुशः शत्रून्प्रेषयित्वा च मृत्यवे}
{गतः पुण्यकृतां लोकान्सर्वकामदुहोऽक्षयान्}


\twolineshloka
{तपसा ब्रह्मचर्येण श्रुतेन प्रज्ञयापि च}
{सन्तो यां गतिमिच्छन्ति तां प्राप्तस्तव पुत्रकः}


\twolineshloka
{वीरसूर्वीरपत्नी त्वं वीरजा वीरबान्धवा}
{मा शुचस्तनयं भद्रे गतःस परमां गतिम्}


\twolineshloka
{`स्वाध्याययुक्तं ब्राह्ममी याचितारं`स्वाध्याययुक्तं ब्राह्मणी याचितारंगौर्बोढारं क्षिप्रगन्तारमश्वा}
{दासं शूद्रा कर्मकारं तु वैश्याशूरं सूते त्वद्विधा राजपुत्री'}


\twolineshloka
{प्राप्स्यते चाप्यसौ पापः सैन्धवो बालघातकः}
{अधर्मस्यास्य तु फलं ससुहृद्गणबान्धवः}


\twolineshloka
{व्युष्टायां तु वरारोहे रजन्यां पापकर्मकृत्}
{न हि मोक्ष्यति पार्थात्स प्रविष्टोऽप्यमरावतीम्}


\twolineshloka
{श्वः शिरः पादमूलं ते सैन्धवस्याहृतं ध्रुवम्}
{पद्ध्यां प्रमथितासि त्वं विशोका भव मा रुदः}


\twolineshloka
{क्षत्रधर्मं पुरस्कृत्य गतः शूरः सतां गतिम्}
{वयं च प्राप्नुयामेह ये चान्ये शस्त्रजीविनः}


\twolineshloka
{व्यूढोरस्को महाबाहुरनिवर्ती रिपुव्रजात्}
{गतस्तव वरारोहे पुत्रः स्वर्गं ज्वरं जहि}


\twolineshloka
{अनुयातश्च पितरं मातृपक्षं च वीर्यवान्}
{सहस्रशो रिपून्हत्वा हतः शूरो महारथः}


\twolineshloka
{आश्वासय स्नुषां राज्ञि मा शुचः क्षत्रिये भृशम्}
{श्वः प्रियं सुमहच्छ्रुत्वा विशोका भव नन्दिनि}


\twolineshloka
{यत्पार्थेन प्रतिज्ञातं तत्तथा न तदन्यथा}
{चिकीर्षितं हि ते भर्तुर्न भवेज्जातु निष्फलम्}


\twolineshloka
{यदि च मनुजपन्नगाः पिशाचारजनिचचाराः पतगाः सुरासुराश्च}
{रणगतमभियान्ति सिन्धुराजंन स भविता सह तैरपि प्रभाते}


\chapter{अध्यायः ७८}
\twolineshloka
{सञ्जय उवाच}
{}


\twolineshloka
{एतच्छ्रुत्वा वचस्तस्य केशवस्य महात्मनः}
{सुभद्रा पुत्रशोकार्ता विललाप सुदुःखिता}


\twolineshloka
{हा पुत्रा मम मन्दायाः कथमेत्यासि संयुगम्}
{निधनं प्राप्तवांस्तात पितुस्तुल्यपराक्रमः}


\twolineshloka
{कथमिन्दीवरश्यामं सुदंष्ट्रं चारुलोचनम्}
{मुखं ते दृश्यते वत्स गुण्ठितं रणरेणुना}


\twolineshloka
{नूनं शूरं निपतितं त्वां पश्यन्त्यनिवर्तिनम्}
{सुशिरोग्रीवबाह्वंसं व्यूढोरस्कं नतोदरम्}


\twolineshloka
{चारूपचितसर्वाङ्गं स्वक्षं शस्त्रक्षताचितम्}
{भूतानि त्वां निरीक्षन्ते भूमौ चन्द्रमिवोदितम्}


\twolineshloka
{शयनीयं पुराऽध्युष्य स्पर्ध्यास्तरणसंवृतम्}
{भूमावद्य कथं शेषे विप्रविद्धः सुखोचितः}


\twolineshloka
{योऽन्वास्यत पुरा वीरो वरस्त्रीभिर्महाभुजः}
{कथमन्वास्यते सोऽद्य शिवाभिः पतितो मृधे}


\twolineshloka
{योऽस्तूयत पुरा हृष्टैः सूतमागधबन्दिभिः}
{सोऽद्य क्रव्याद्गणैर्घोरैर्विनदद्भिरुपास्येत}


\twolineshloka
{पाण्डवेषु च नाथेषु वृष्णिवीरेषु वा विभो}
{पाञ्चालेषु च वीरेषु हतः केनास्यनाथवत्}


\threelineshloka
{`यत्र त्वं केशवे नाथे सत्यनाथो यथा हतः'}
{अतृप्तदर्शना पुत्र दर्शनस्य तवानघ}
{मन्दभाग्या गमिष्यामि व्यक्तमद्य यमक्षयम्}


\twolineshloka
{विशालाक्षं सुकेशान्तं चारुवाक्यं सुगन्धि च}
{तव पुत्र कदा भूयो मुखं द्रक्ष्यामि निर्व्रणम्}


\twolineshloka
{धिग्बलं भीमसेनस्य धिक्पार्थस्य धनुष्मताम्}
{धिग्वीर्यं वृष्णिवीराणां चाञ्चालानां च धिग्बलं}


\twolineshloka
{धिक्केकयांस्तथा चेदीन्मत्स्यांश्चैवाथ सृञ्जयान्}
{ये त्वां रणगतं वीरं न शेकुरभिरक्षितुम्}


\twolineshloka
{अद्य पश्यामि पृथिवीं शून्यामिव हतत्विषम्}
{अभिमन्युमपश्यन्ती शोकव्याकुललोचना}


\twolineshloka
{स्वस्रीयं वासुदेवस्य पुत्रं गाण्डवीधन्वनः}
{कथं त्वाऽतिरथं वीरं द्रक्ष्याम्यद्य निपातितम्}


\twolineshloka
{एह्येहि तृषितो वत्स स्तनौ पूर्णौ पिबाशु मे}
{अङ्कमारुह्य मन्दाया ह्यतृप्तायाश्च दर्शने}


\twolineshloka
{हा वीर दृष्टो नष्टश्च धनं स्वप्न इवासि मे}
{अहो ह्यनित्यं मानुष्यं जलबुद्बुदचञ्चलम्}


\twolineshloka
{इमां ते तरुणीं भार्यां तवाधिभिरभिप्लुताम्}
{`उत्तरामुत्तमां जात्या सुशीलां प्रिय भाषिणीम्}


\twolineshloka
{शनकैः परिरभ्यैनां स्नुषां मम यशस्विनीम्}
{सुकुमारीं विशालाक्षीं पूर्णचन्द्रनिभाननाम्}


\twolineshloka
{बालपल्लवतन्वङ्गीं मत्तमातङ्गगामिनीम्}
{बिम्बाधरोष्ठीमबलामभिमन्यो प्रहर्षय}


\twolineshloka
{त्वया विना कथं पुत्र जीर्णां पतितमानसाम्'}
{इमां सन्धारयिष्यामि वृषभादिव धेनुकाम्}


\twolineshloka
{अहो ह्यकाले प्रस्थानं कृतवानसि पुत्रक}
{विहाय फलकाले मां सुगृद्धां तव दर्शने}


\twolineshloka
{नूनं गतिः कृतान्तस्य प्राज्ञैरपि सुदुर्विदा}
{यत्र त्वं केशवे नाथे सङ्ग्रामेऽनाथवद्धतः}


\twolineshloka
{यज्वनां दानशीलानां ब्राह्मणानां कृतात्मनाम्}
{चरितब्रह्मचर्याणां पुण्यतीर्थावगाहिनाम्}


\twolineshloka
{कृतज्ञानां वदान्यानां गुरुशुश्रूषिणामपि}
{सहस्रदक्षिणानां च या गतिस्तामवाप्नुहि}


\twolineshloka
{या गतिर्युध्यमानानां शूराणामनिवर्तिनाम्}
{हत्वारीन्निहतानां च सङ्ग्रामे तां गति व्रज}


\twolineshloka
{गोसहस्रप्रदातॄणां क्रतुदानां च या गतिः}
{नैवेशिकं चाभिमतं ददतां या गतिः शुभा}


\twolineshloka
{ब्राह्मणेभ्यः शरण्येभ्यो निधिं निदधतां च या}
{या चापि न्यस्तदण्डानां तां गतिं व्रज पुत्रक}


\twolineshloka
{ब्रह्मचर्येण यां यान्ति मुनयः संशितव्रताः}
{एकपत्न्यश्च यां यान्ति तां गतिं व्रज पुत्रक}


\twolineshloka
{राज्ञा सुचरितैर्या च गतिर्भवति शाश्वती}
{चरमाश्रमिणां पुण्यैः सेवितानां पुरः स्थितैः}


\twolineshloka
{दीनानुकम्पिनां या च सततं संविभागिनाम्}
{पैशुन्याच्च निवृत्तानां तां गतिं व्रज पुत्रक}


\twolineshloka
{व्रतिनां धर्मशीलानां गुरुशुश्रूषिणामपि}
{अमोघातिथिनां या च तां गतिं व्रज पुत्रक}


\twolineshloka
{कृच्छ्रेषु या धारयतामात्मानं व्यसनेषु च}
{गतिः शोकाग्निदग्धानां तां गतिं व्रज पुत्रक}


\twolineshloka
{मातापित्रोश्च शुश्रूषां कल्पयन्तीह ये सदा}
{स्वदारनिरतानां च या गतिस्तामवाप्नुहि}


\twolineshloka
{ऋतुकाले स्वकां भार्यां गच्छतां या मनीषिणाम्}
{परस्त्रीभ्यो निवृत्तानां तां गतिं व्रज पुत्रक}


\twolineshloka
{साम्ना ये सर्वभूतानि पश्यन्ति गतमत्सराः}
{नारुन्तुदानां क्षमिणां या गतिस्तामवाप्नुहि}


\twolineshloka
{मधुमांसान्निवृत्तानां मदाद्दृम्भात्तथाऽनृतात्}
{परोपतापादन्यायात्तां गतिं व्रज पुत्रक}


\threelineshloka
{हीमन्तः सर्वशास्त्रज्ञा ज्ञानतृप्ता जितेन्द्रियाः}
{यां गतिं साधवो यान्ति तां गतिं व्रज पुत्रक ॥सञ्जय उवाच}
{}


\twolineshloka
{एवं विलपतीं दीनां सुभद्रां शोककर्शिताम्}
{अन्वपद्यत पाञ्चाली वैराटिसहितां तदा}


\twolineshloka
{सा प्रकामं रुदित्वा च विलप्य च सुदुःखिता}
{उन्मत्तवत्तदा राजन्विसंज्ञा पतिता क्षितौ}


\twolineshloka
{सोपचारस्तु कृष्णस्तां दुःखितां भृशदुःखितः}
{सिक्त्वाम्भसा समाश्वास्य तत्तदुक्त्वा हितं वचः}


\twolineshloka
{विसंज्ञकल्पां रुदतीमपविद्धां प्रवेपतीम्}
{भगिनीं पुण्डरीकाक्ष इदं वचनमब्रवीत्}


\twolineshloka
{सुभद्रे मा शुचः पुत्रं पाञ्चाल्याश्वासयोत्तराम्}
{गतोऽभिमन्युः प्रथितां गतिं क्षत्रियपुङ्गवः}


\twolineshloka
{ये चान्येऽपि कुले सन्ति पुरुषा नो वरानने}
{सर्वे न ते गतिं यान्ति याऽभिमन्योर्यशस्विनः}


\threelineshloka
{कुर्याम तद्वयं कर्म कुर्युर्युत्सुहृदश्च नः}
{कृतवान्यादृगद्यैकस्तव पुत्रो महारथः ॥सञ्जय उवाच}
{}


\twolineshloka
{एवमाश्वास्य भगिनीं द्रौपदीमपि चोत्तराम्}
{पार्थस्यैव महाबाहुः पार्श्वमागादरिन्दमः}


\twolineshloka
{ततोऽभ्यनुज्ञाय नृपान्कृष्णो बन्धूंस्तथाऽर्जुनम्}
{विवेशान्तः पुरे राजंस्ते च जग्मुः स्वमालयम्}


\chapter{अध्यायः ७९}
\twolineshloka
{सञ्जय उवाच}
{}


\threelineshloka
{ततोऽर्जुनस्य भवनं प्रविश्याप्रतिमं विभुः}
{स्पृष्ट्वाम्भः पुण्डरीकाक्षः स्थण्डिले शुभलक्षणे}
{सन्तस्तार शुभां शय्यां दर्भैर्वैदूर्यसन्निभैः}


\twolineshloka
{ततो माल्येन विधिवल्लाजैर्गन्धैः सुमङ्गलैः}
{अलञ्चकार तां शय्यां परिवार्यायुधोत्तमैः}


\twolineshloka
{ततः स्पृष्टोदकं पार्थं विनीतपरिचारकम्}
{नैत्यकं दर्शयाञ्चक्रे नैशं त्रैयम्बकं बलिम्}


\twolineshloka
{ततः प्रीतमनाः पार्थो गन्धमाल्यैश्च माधवम्}
{अलङ्कृत्योपहारं तं नैशं तस्मै न्यवेदयत्}


% Check verse!
साधुसाध्विति गोविन्दः फल्गुनं प्रत्यभाषत
\twolineshloka
{सुप्यतां पार्थ भद्रं ते कल्याणाय व्रजाम्यहम्}
{स्थापयित्वा ततो द्वास्थान्गोप्तॄंश्चात्तायुधान्नरान्}


\twolineshloka
{दारुकानुगतः श्रीमान्विवेश शिबिरं स्वकम्}
{शिश्ये च शयने शुभ्रे बहुकृत्य विचिन्तयन्}


\twolineshloka
{पार्थाय सर्वं भगवाञ्शोकदुःखापहं विधिम्}
{व्यदधात्पुण्डरीकाक्षस्तेजोद्युतिविवर्धनम्}


\twolineshloka
{योगमास्थाय युक्तात्मा सर्वेषामीश्वरेश्वरः}
{श्रेयस्कामः पृथुयशा विष्णुर्जिष्णुप्रियङ्करः}


\twolineshloka
{न पाण्डवानां शिबिरे कश्चित्सुष्वाप तां निशाम्}
{प्रजागरः सर्वजनं ह्याविवेश विशाम्पते}


\twolineshloka
{पुत्रशोकाभितप्तेन प्रतिज्ञातो महात्मना}
{श्वः सिन्धुराजस्य वधः कार्यो गाण्डीवधन्वना}


\twolineshloka
{तत्कथं नु महाबाहुर्वासविः परवीरहा}
{प्रतिज्ञां सफलां कुर्यादिति ते समचिन्तयन्}


\twolineshloka
{कष्टं हीदं व्यवसितं पाण्डवेन महात्मना}
{पुत्रशोकाभिभूतेन प्रतिज्ञा महती कृता}


\twolineshloka
{भ्रातरश्चापि विक्रान्ता बहुलानि बलानि च}
{धृतराष्ट्रस्य पुत्रेण सर्वतः सन्निवेशिताः}


\twolineshloka
{स हत्वा सैन्धवं सङ्ख्ये पुनरेतु जयी सुखी}
{`हत्वा रिपुगणं सर्वं पारयित्वा महाव्रतम्}


\twolineshloka
{यद्यस्ति सुकृतं किञ्चिदस्माकं हन्तु सैन्धवम्}
{जित्वा सर्वान्रिपून्पार्थस्त्रातु नोऽस्मान्महाभयात्}


\twolineshloka
{एवमाशंसमानास्ते केचित्तस्थुरुपश्रुतिम्}
{श्रुत्वा चेष्टं सुमनसो व्यक्तमाशंसिरे जयम्}


\twolineshloka
{भविता नु कथं कृत्यमिदमित्यब्रुवञ्जनाः'}
{श्वोऽहत्वा सिन्धुराजं वै धूमकेतुं प्रवेक्ष्यति}


\twolineshloka
{न शक्यमनृतां कर्तुं प्रतिज्ञां विजयेन हि}
{`महद्धि साहसं पार्थः कृतवाच्छोकमोहितः'}


\twolineshloka
{धर्मपुत्रः कथं राजा भविष्यति मृतेऽर्जुने}
{तस्मिन्हि विजयः कृत्स्नः पाण्डवेन समाहितः}


\threelineshloka
{यदि नोऽस्ति कृतं किञ्चिद्यदि दत्तं हुतं यदि}
{फलेन तस्य सर्वस्य सव्यसाची जयत्वरीन् ॥सञ्जय उवाच}
{}


\twolineshloka
{एवं कथयतां तेषां जयमाशंसतां प्रभो}
{कृच्छ्रेण महता राजन्रजनी व्यत्यवर्तत}


\twolineshloka
{तस्या रजन्या मध्ये तु प्रतिबुद्धो जनार्दनः}
{स्मृत्वा प्रतिज्ञां पार्थस्य दारुकं प्रत्यभाषत}


\twolineshloka
{अर्जुनेन प्रतिज्ञातमार्तेन हतबन्धुना}
{जयद्रथं वधिष्यामि श्वोभूत इति दारुक}


\twolineshloka
{तत्तु दुर्योधनः श्रुत्वा मन्त्रिभिर्मन्त्रयिष्यति}
{यथा जयद्रथं पार्थो न हन्यादिति संयुगे}


\twolineshloka
{अक्षौहिण्यो हि ताः सर्वा रक्षिष्यन्ति जयद्रथम्}
{द्रोणश्च सहपुत्रेण सर्वास्त्रविधिपारगः}


\twolineshloka
{एको वीरः सहस्राक्षो दैत्यदानवदर्पहा}
{सोऽपि तं नोत्सहेताजौ हन्तुं द्रोणेन रक्षितम्}


\twolineshloka
{सोऽहं श्वस्तत्करिष्यामि यथा कुन्तीसुतोऽर्जुनः}
{अप्राप्तेऽस्तं दिनकरे हनिष्यति जयद्रथम्}


\twolineshloka
{न हि दारा न मित्राणि ज्ञातयो न च बान्धवाः}
{कश्चिदन्यः प्रियतरः कुन्तीपुत्रान्ममार्जुनात्}


\twolineshloka
{अनर्जुनमिमं लोकं मुहूर्तमपि दारुक}
{उदीक्षितुं न शक्तोऽहं भविता न च तत्तथा}


\twolineshloka
{अहं विजित्य तान्सर्वान्सहसा सहयद्विपान्}
{अर्जुनार्थे हनिष्यामि सकर्णान्ससुयोधनान्}


\twolineshloka
{श्वो निरीक्षन्तु मे वीर्यं त्रयो लोका महाहवे}
{धनञ्जयार्थे समरे पराक्रान्तस्य दारुक}


\twolineshloka
{श्वो नरेन्द्रसहस्राणि राजपुत्रशतानि च}
{साश्वद्विपरथान्याजौ विद्रविष्यामि दारुक}


\twolineshloka
{श्वस्तां चक्रप्रमथितां द्रक्ष्यसे नृपवाहिनीम्}
{मया क्रुद्धेन समरे पाण्डवार्थे निपातिताम्}


\twolineshloka
{श्वः सदेवाः सगन्धर्वाः पिशाचोरगराक्षसाः}
{ज्ञास्यन्ति लोकाः सर्वे मां सुहृदं सव्यसाचिनः}


\twolineshloka
{यस्तं द्वेष्टि स मां द्वेष्टि यस्तं चानु स मामनु}
{इति सङ्कल्प्यतां बुद्ध्या शरीरार्धं ममार्जुनः}


\twolineshloka
{यथा त्वं मे प्रभातायामस्यां निशि रथोत्तमम्}
{कल्पयित्वा यथाशास्त्रमादाय व्रज संयतः}


\twolineshloka
{गदां कौमोदकीं दिव्यां शक्तिं चक्रं धनुः शरान्}
{आरोप्य वै रथे सूत सर्वोपकरणानि च}


\twolineshloka
{स्थानं च कल्पयित्वाऽथ रथोपस्थे ध्वजस्य मे}
{वैनतेयस्य वीरस्य समरे रथशोभिनः}


% Check verse!
छत्रं जाम्बूनदैर्जालैरर्कज्वलनसप्रभैः ॥विश्वकर्मकृतैर्दिव्यैरश्वानपि विभूषय
\twolineshloka
{बलाहकं मेघपुष्पं शैब्यं सुग्रीवमेव च}
{युक्त्वा वाजिवरांस्तत्र कवची तिष्ठ दारुक}


\twolineshloka
{पाञ्चजन्यस्य निर्घोषं पर्जन्यनिनदोपमम्}
{श्रुत्वा च भैरवं नादमुपेयास्त्वं जवेन माम्}


\twolineshloka
{एकाह्नाऽहममर्षं च सर्वदुःखानि चैव ह}
{भ्रातुः पैतृष्वसेयस्य व्यपनेष्यामि दारुक}


\twolineshloka
{सर्वोपायैर्यतिष्यामि यथा बीभत्सुराहवे}
{पश्यतां धार्तराष्ट्राणां हनिष्यति जयद्रथम्}


\threelineshloka
{यस्ययस्य च बीभत्सुर्वधे यत्नं करिष्यति}
{आशंसेऽहं रणे तन्तं तत्रतत्र हनिष्यति ॥दारुक उवाच}
{}


\twolineshloka
{जय एव ध्रुवस्तस्य कुत एव पराजयः}
{यस्य त्वं पुरुषव्याघ्र सारथ्यमुपजग्मिवान्}


\twolineshloka
{एवं चैतत्करिष्यामि यथा मामनुभाषसे}
{सुप्रभातामिमां रात्रिं जयाय विजयस्य हि}


\chapter{अध्यायः ८०}
\twolineshloka
{सञ्जय उवाच}
{}


\twolineshloka
{धार्तराष्ट्रस्य तं मन्त्रं स्मरन्नेव धनञ्जयः}
{प्रतिज्ञामात्मनो रक्षन्मुमोहाचिन्त्यविक्रमः}


\twolineshloka
{तं तु शोकेन सन्तप्तं स्वप्ने कपिवरध्वजम्}
{आससाद महातेजा ध्यायन्तं गरुडध्वजः}


\twolineshloka
{प्रत्युत्थानं च कृष्णस्य सर्वावस्थासु फल्गुनः}
{न लोपयति धर्मात्मा भक्त्या प्रेम्णा च सर्वदा}


\twolineshloka
{प्रत्युत्थाय च गोविन्दं स्वं तस्मै चासनं ददौ}
{न चासने स्वयं बुद्धिं बीभत्सुर्व्यदधात्तदा}


\twolineshloka
{ततः कृष्णो महातेजा जानँल्लोकस्य निश्चयम्}
{कुन्तीपुत्रमिदं वाक्यमासीनः स्थितमब्रवीत्}


\twolineshloka
{मा विषादे मनः पार्थ कृथाः कालो हि दुर्जयः}
{कालः सर्वाणि भूतानि नियच्छति भवाभवे}


\twolineshloka
{किमर्थं च विषादस्ते तद्ब्रूहि द्विपदां वर}
{न शोच्यं विदुषां श्रेष्ठ शोकः कार्यविनाशनः}


\twolineshloka
{यत्तु कार्यं भवेत्कार्यं कर्मणा तत्समाचर}
{हीनचेष्टस्य यः शोकः स हि शत्रुर्धनञ्जय}


\threelineshloka
{शोचन्नन्दयते शत्रून्कर्शयत्यपि बान्धवान्}
{क्षीयते च नरस्तस्मान्न त्वं शोचितुमर्हसि ॥सञ्जय उवाच}
{}


\twolineshloka
{इत्युक्तो वासुदेवेन बीभत्सुरपराजितः}
{आबभाषे तदा विद्वानिदं वचनमर्थवत्}


\twolineshloka
{मया प्रतिज्ञा महती जयद्रथवधे कृता}
{श्वोऽस्मि हन्ता दुरात्मानं पुत्रघ्नमिति केशव}


\twolineshloka
{मत्प्रतिज्ञाविघातार्थं धार्तराष्ट्रैः किलाच्युत}
{पृष्ठतः सैन्धवः कार्यः सर्वैर्गुप्तो महारथैः}


\twolineshloka
{दश चैका च ताः कृष्ण अक्षौहिण्यः सुदुर्जयाः}
{हतावशेषास्तत्रेमा हन्त माधव सङ्ख्यया}


\twolineshloka
{ताभिः परिवृतः सङ्ख्ये सर्वैश्चैव महारथैः}
{कथं शक्येत सन्द्रष्टुं दुरात्मा कृष्ण सैन्धवः}


\twolineshloka
{प्रतिज्ञापारणं चापि न भविष्यति केशव}
{प्रतिज्ञायां च हीनायां कथं जीवति मद्विधः}


\threelineshloka
{दुःखापायस्य मे वीर विकाङ्क्षा परिवर्तते}
{द्रुतं च याति सविता तत एतद्ब्रवीम्यहम् ॥सञ्जय उवाच}
{}


\twolineshloka
{शोकस्थानं तु तच्छ्रुत्वा पार्थस्य द्विजकेतनः}
{संस्पृश्याम्भस्ततः कृष्णः प्राङ्मुखः समवस्थितः}


\threelineshloka
{इदं वाक्यं महातेजा बभाषे पुष्करेक्षणः}
{हितार्थं पाण्डुपुत्रस्य सैन्धवस्य वधे कृती ॥श्रीभगवानुवाच}
{}


\twolineshloka
{पार्थ पाशुपतं नाम परमास्त्रं सनातनम्}
{येन सर्वान्मृधे देत्याञ्जघ्ने देवो महेश्वरः}


\twolineshloka
{यदि तद्विदितं तेऽद्य श्वो हन्ताऽसि जयद्रथम्}
{अथ ज्ञातं प्रपद्यस्य मनसा वृषभध्वजम्}


\threelineshloka
{तं देवं मनसा ध्यात्वा जोषमास्स्व धनञ्जय}
{ततस्तस्य प्रसादात्त्वं भक्त्या प्राप्स्यसि तन्महत् ॥सञ्जय उवाच}
{}


\twolineshloka
{ततः कृष्णवचः श्रुत्वा संस्पृश्याम्भो धनञ्जयः}
{भूमावासीन एकाग्रो जगाम मनसा भवम्}


\twolineshloka
{ततः प्रणिहितो ब्राह्मे मुहूर्ते शुभलक्षणे}
{आत्मानमर्जुनोऽपश्यद्गने सहकेशवम्}


\twolineshloka
{पुण्यं हिमवतः पादं मणिमन्तं च पर्वतम्}
{ज्योतिर्भिश्च समाकीर्णं सिद्धचारणसेवितम्}


\twolineshloka
{वायुवेगगतिः पार्थः खं भेजे सहकेशवः}
{केशवेन गृहीतः स दक्षिणे विभुना भुजे}


\twolineshloka
{प्रेक्षमाणो बहून्भावाञ्जगामाद्भुतदर्शनान्}
{उदीच्यां दिशि धर्मात्मा सोपश्यच्छ्वेतपर्वतम्}


\twolineshloka
{कुबेरस्य विराहे च नलिनीं पद्मभूषिताम्}
{सरिच्छ्रेष्ठां च तां गङ्गां वीक्षमाणो बहूदकाम्}


\twolineshloka
{सदा पुष्पफलैर्वृक्षैरुपेतां स्फटिकोपलाम्}
{सिंहव्याघ्रसमाकीर्णां नानामृगसमाकुलाम्}


\twolineshloka
{पुण्याश्रमवतीं रम्यां मनोज्ञाण्डजसेविताम्}
{मन्दरस्य प्रदेशांश्च किन्नरोद्गीतनादितान्}


\threelineshloka
{हेमरूप्यमयैः शृङ्गैर्नानौषधिविदीपितान्}
{तथा मन्दारवृक्षैश्च पुष्पितैरुपशोभितान्}
{}


\twolineshloka
{स्निग्धाञ्जनचयाकारं सम्प्राप्तः कालपर्वतम्}
{ब्रह्मतुङ्गं नदीश्चान्यास्तथा जनपदानपि}


\twolineshloka
{स तुङ्गं शतशृङ्गं च शर्यातिवनमेव च}
{पुण्यमश्वशिरस्थानं स्थानमाथर्वणस्य च}


\twolineshloka
{वृषदंशं च शैलेन्द्रं महामन्दरमेव च}
{अप्सरोभिः समाकीर्णं किन्नरैश्चोपशोभितम्}


\twolineshloka
{तस्मिञ्शैले व्रजन्पार्थः सकृष्णः समवैक्षत}
{शुभैः प्रस्रवणैर्जुष्टां हेमधातुविभूषिताम्}


\twolineshloka
{चन्द्ररश्मिप्रकाशाङ्गीं पृथिवीं पुरमालिनीम्}
{समुद्रांश्चाद्भुताकारानपश्यद्बहुलाकरान्}


\twolineshloka
{वियद्द्यां पृथिवीं चैव तथा विष्णुपदं व्रजन्}
{विस्मितः सह कृष्णेन क्षिप्तो बाण इवाभ्यगात्}


\twolineshloka
{ग्रहनक्षत्रसोमानां सूर्याग्न्योश्च समत्विषम्}
{अपश्यत तदा पार्थो ज्वलन्तमिव पर्वतम्}


\threelineshloka
{समासाद्य तु तं शैलं शैलाग्रे समवस्थितम्}
{तपोनित्यं महात्मानमपश्यद्वॄषभध्वजम्}
{}


\twolineshloka
{सहस्रमिव सूर्याणां दीप्यमानं स्वतेजसा}
{शूलिनं जटिलं गौरं वल्कलाजिनवाससम्}


\twolineshloka
{नयनानां सहस्रैश्च विचित्राङ्गं महौजसम्}
{पार्वत्या सहितं देवं भूतसङ्घैश्च भास्वरैः}


\twolineshloka
{गीतवादित्रसन्नादैर्हास्यलास्यसमन्वितम्}
{वग्लितास्फोटितोत्क्रुष्टैः पुण्यैर्गन्धैश्च सेवितम्}


\twolineshloka
{स्तूयमानं स्तवैर्दिव्यैर्ऋषिभिर्ब्रह्मवादिभिः}
{गोप्तारं सर्वभूतानामीशानं वरदं शिवम्}


\twolineshloka
{वासुदेवस्तु तं दृष्ट्वा जगाम शिरसा क्षितिम्}
{पार्थेन सह धर्मात्मा गृणन्ब्रह्म सनातनम्}


\twolineshloka
{लोकादिं विश्वकर्माणमजमीशानमव्ययम्}
{मनसः परमं योनिं खं वायुं ज्योतिषां निधिम्}


\twolineshloka
{स्रष्टारं वारिधाराणां भुवश्च प्रकृतिं पराम्}
{देवदानवयक्षाणां मानवानां च शासकम्}


\twolineshloka
{योगानां च परं धाम दृष्टं ब्रह्मविदां निधिम्}
{चारचरस्य स्रष्टारं प्रतिहर्तारमेव च}


\twolineshloka
{कालकोपं महात्मानं शक्रसूर्यगुणोदयम्}
{ववन्दतुस्तदा कृष्णौ वाङ्मनोबुद्धिकर्मभिः}


\twolineshloka
{यं प्रपद्यन्ति विद्वांसः सूक्ष्माध्यात्मपदैषिणः}
{तमजं कारमात्मानं जग्मतुः शरणं भवम्}


\twolineshloka
{अर्जुनश्चापि तं देवं भूयो भूयोऽप्यवन्दत}
{ज्ञात्वा तं सर्वभूतादिं भूतभव्यभवोद्भवम्}


\twolineshloka
{`शरण्यं शरणं देवमीशानं परमेश्वरम्}
{जगाम जगतां नाथमर्जुनः सजनार्दनः'}


\twolineshloka
{ततस्तावागतौ दृष्ट्वा नरनारायणावुभौ}
{सुप्रसन्नमनाः शर्वः प्रोवाच प्रहसन्निव}


\twolineshloka
{स्वागतं वां नरश्चेष्ठावुत्तिष्ठेतां गतक्लमौ}
{किञ्च वामीप्सितं वीरौ मनसः क्षिप्रमुच्यताम्}


\threelineshloka
{येन कार्येण सम्प्राप्तौ युवां तत्साधयामि किम्}
{व्रियतामात्मनः श्रेयस्तत्सर्वं प्रददानि वाम् ॥सञ्जय उवाच}
{}


\fourlineindentedshloka
{ततस्तद्वचनं श्रुत्वा प्रत्युत्थाय कृताञ्जली}
{वासुदेवार्जुनौ शर्वं तुष्टुवाते महामती}
{भक्त्यास्तवेन दिव्येन महात्मानावनिन्दितौ ॥कृष्णार्जुनावूचतुः}
{}


\twolineshloka
{नमो भवाय शर्वाय रुद्राय वरदाय च}
{पशूनां पतये नित्यमुग्राय च कपर्दिने}


\twolineshloka
{महादेवाय भीमाय त्र्यम्बकाय च शान्तये}
{ईशानाय मखघ्नाय नमोऽस्त्वन्धकघातिने}


\twolineshloka
{कुमारगुरवे तुभ्यं नीलग्रीवाय वेधसे}
{पिनाकिने हविष्याय सत्याय विभवे सदा}


\twolineshloka
{विलोहिताय ध्रूम्राय व्याधायानपराजिते}
{नित्यं नीलशिखण्डाय शूलिने दिव्यचक्षुषे}


\twolineshloka
{होत्रे पोत्रे त्रिनेत्राय व्याधाय वसुरेतसे}
{अचिन्त्यायाम्बिकाभर्त्रे सर्वदेवस्तुताय च}


\twolineshloka
{वृषध्वजाय मुण्डाय जटिने ब्रह्मचारिणे}
{तप्यमानाय सलिले ब्रह्मण्यायाजिताय च}


\twolineshloka
{विश्वात्मने विश्वसृजे विश्वमावृत्य तिष्ठते}
{नमोनमस्ते सेव्याय भूतानां प्रभवे सदा}


\twolineshloka
{ब्रह्मवक्त्राय सर्वाय शंकराय शिवाय च}
{नमोस्तु वाचस्पतये प्रजानां पतये नमः}


\threelineshloka
{`अभिगम्याय काम्याय स्तुत्यायार्याय सर्वदा}
{नमोऽस्तु देवदेवाय महाभूतधराय च}
{नमो विश्वस्य पतये पत्तीनां पतये नमः'}


\threelineshloka
{नमो विश्वस्य पतये महतां पतये नमः}
{नमः सहस्रशिरसे सहस्रभुजमृत्यवे ॥सहस्रनेत्रपादाय नमोऽसङ्ख्येयकर्मणे}
{}


\threelineshloka
{नमो हिरण्यवर्णाय हिरण्यकवचाय च}
{भक्तानुकम्पिने नित्यं सिध्यतां नो वरः प्रभो ॥सञ्जय उवाच}
{}


\twolineshloka
{एवं स्तुत्वा महादेवं वासुदेवः सहार्जुनः}
{प्रसादयामास भवं तदा ह्यस्त्रोपलब्धये}


\chapter{अध्यायः ८१}
\twolineshloka
{सञ्जय उवाच}
{}


\twolineshloka
{ततः पार्थः प्रसन्नात्मा प्राञ्जलिर्वृषभध्वजम्}
{ददर्शोत्फुल्लनयनः समस्तं तेजसां निधिम्}


\twolineshloka
{तं चोपहारं सुकृतं नैशं नैत्यकमात्मना}
{ददर्श त्र्यम्बकाभ्याशे वासुदेवनिवेदितम्}


\twolineshloka
{ततोऽभिपूज्य मनसा कृष्णं शर्वं च पाण्डवः}
{इच्छाम्यहं दिव्यमस्त्रमित्यभाषत शङ्करम्}


\twolineshloka
{ततः पार्थस्य विज्ञाय वरार्थे वचनं तदा}
{वासुदेवार्जुनौ देवः स्मयमानोऽभ्यभाषत}


\twolineshloka
{स्वागतं वां नरश्रेष्ठौ विज्ञातं मनसेप्सितम्}
{येन कामेन सम्प्राप्तौ भवद्ध्यां तं ददाम्यहम्}


\twolineshloka
{सरोऽमृतमयं दिव्यमभ्याशे शत्रुसूदनौ}
{तत्र मे तद्धनुर्दिव्यं शरश्च निहितः पुरा}


\twolineshloka
{येन देवारयः सर्वे मया युधि निपातिताः}
{तत आनीयतां कृष्णौ सशरं धनुरुत्तमम्}


\twolineshloka
{तथेत्युक्त्वा तु तौ वीरौ सर्वपारिषदैः सह}
{प्रस्थितौ तत्सरो दिव्यं दिव्यैश्वर्यशतैर्युतम्}


\twolineshloka
{निर्दिष्टं यद्वृषाङ्केण पुण्यं सर्वार्थसाधकम्}
{तौ जग्मतुरसम्भ्रान्तौ नरनारायणावृषी}


\twolineshloka
{ततस्तौ तत्सरो गत्वा सूर्यमण्डलसन्निभम्}
{नागमन्तर्जले घोरं ददृशातेऽर्जुनाच्युतौ}


\twolineshloka
{द्वितीयं चापरं नागं सहस्रशिरसं वरम्}
{वमन्तं विपुला ज्वाला ददृशातेऽग्निवर्चसम्}


\twolineshloka
{ततः कृष्णश्च पार्थश्च संस्पृश्याम्भः कृताञ्जली}
{तौ नागावुपतस्थाते नमस्यन्तौ वृषध्वजम्}


\twolineshloka
{गृणन्तौ वेदविद्वांसौ तद्ब्रह्म शतरुद्रियम्}
{अप्रमेयप्रमाणं तौ गत्वा सर्वात्मना भवम्}


\twolineshloka
{ततस्तौ रुद्रमाहात्म्याद्धित्वा रूपं महोरगौ}
{धनुर्बाणश्च शत्रुघ्नं तद्द्वन्द्वं समपद्यत}


\twolineshloka
{तौ तज्जगृहतुः प्रीतौ धनुर्बाणं च सुप्रभम्}
{आजहूतुर्महात्मानौ ददतुश्च महात्मने}


\threelineshloka
{ततः पाश्वाद्वॄषाङ्कस्य ब्रह्मचारी न्यवर्तत}
{पिङ्गाक्षस्तपसः क्षेत्रं बलवान्नीललोहितः}
{}


\twolineshloka
{स तद्गृह्य धनुःश्रेष्ठं तस्थौ स्थानं समाहितः}
{विचकर्षाथ विधिवत्सशरं धनुरुत्तमम्}


\twolineshloka
{तस्य मौर्वी च मुष्टिं च स्थानं चालक्ष्य पाण्डवः}
{श्रुत्वा मन्त्रं भवप्रोक्तं जग्राहाचिन्त्यविक्रमः}


\twolineshloka
{स सरस्येव तं बाणं मुमोचातिबलः प्रभुः}
{चिक्षेप च पुनर्वीरस्तस्मिन्सरसि तद्धनुः}


\threelineshloka
{ततः प्रीतं भवं ज्ञात्वा स्मृतिमानर्जुनस्तदा}
{वरमारण्यके दत्तं दर्शनं शङ्करस्य च}
{मनसा चिन्तयामास तन्मे सम्पद्यतां भव}


\twolineshloka
{तस्य तन्मतमाज्ञाय प्रीतः प्रादाद्वरं भवः}
{तच्च पाशुपतं घोरं प्रतिज्ञायाश्च पारणम्}


\twolineshloka
{ततः पाशुपतं दिव्यमवाप्य पुनरीश्वरात्}
{संहृष्टरोमा दुर्धर्षः कृतं कार्यममन्यत}


\twolineshloka
{`ततोऽर्जुनहृषीकैशौ पुनः पुनररिंदमौ'}
{ववन्दतुश्च संहृष्टौ शिरोभ्यां तं महेश्वरम्}


\twolineshloka
{अनुज्ञातौ क्षणे तस्मिन्भवेनार्जुनकेशवौ}
{प्राप्तौ स्वशिबिरं वीरौ मुदा परमया युतौ}


\twolineshloka
{तथा भवेनानुमतौ महासुरनिघातिना}
{इद्रावीष्णू यथा प्रीतौ जम्भस्य वधकाङ्क्षिणौ}


\chapter{अध्यायः ८२}
\twolineshloka
{सञ्जय उवाच}
{}


\twolineshloka
{तयोः संवदतोरेव कृष्णदारुकयोस्तथा}
{साऽत्यगाद्रजनी राजन्नथ राजा स्म बोध्यते}


\twolineshloka
{पठन्ति पाणिध्वनिका मागधाः स्तवगायकाः}
{वैतालिकाश्च सूताश्च तुष्टुवुः पुरुषर्षभम्}


\twolineshloka
{नर्तकाश्चाप्यनृत्यन्त जगुर्गीतानि गायकाः}
{कुरुवंशस्तवार्थानि मधुरं रक्तकण्ठिनः}


\twolineshloka
{मृदङ्गा झर्झरा भेर्यः पणवानकगोमुखाः}
{आडम्बराश्च शङ्खाश्च दुन्दुभ्यश्च महास्वनाः}


\twolineshloka
{एवमेतानि सर्वाणि तथान्यान्यपि भारत}
{वादयन्ति स्म संहृष्टाः कुशलाः साधुशिक्षिताः}


\twolineshloka
{स मेघसमनिर्घोषो महाञ्शब्दोऽस्पृशद्दिवम्}
{पार्थिवप्रवरं सुप्तं युधिष्ठिरमबोधयत्}


\twolineshloka
{प्रतिबुद्धः सुखं सुप्तो महार्हे शयनोत्तमे}
{उत्थायावश्यकार्यार्थं ययौ स्नानगृहं नृपः}


\twolineshloka
{ततः शुक्लाम्बराः स्नातास्तरुणाः शतमष्ट च}
{स्नापकाः काञ्चनैः कुम्भैः पूर्णैः समुपतस्थिरे}


\twolineshloka
{भद्रासने सूपविष्टः परिधायाम्बरं लघु}
{सस्नौ चन्दनसंयुक्तैः पानीयैरभिमन्त्रितैः}


\twolineshloka
{उत्सादितः कषायेण बलवद्भिः सुशिक्षितैः}
{आप्लुतः साधिवासेन जलेन ससुगन्धिना}


\twolineshloka
{राजहंसनिभं प्राप्य उष्णीषं शिथिलार्पितम्}
{जलक्षयनिमित्तं वै वेष्टयामास मूर्धनि}


\twolineshloka
{हरिणा चन्दनेनाङ्गमुपलिप्य महाभुजः}
{स्रग्वी चाक्लिष्टवसनः प्राङ्मुखः प्राञ्जलिः स्थितः}


\threelineshloka
{`कृत्वेन्द्रियाणामैकाग्र्यं मनसश्च महामनाः'}
{जजाप जप्यं कौन्तेयः सतां मार्गमनुष्ठितः}
{तत्राग्निशरणं दीप्तं प्रविवेश विनीतवत्}


\twolineshloka
{समिद्भिः स पवित्राभिरग्निमाहुतिभिस्तथा}
{मन्त्रपूताभिरर्चित्वा निश्चक्राम गृहात्ततः}


\twolineshloka
{द्वितीयां पुरुषव्याघ्रः कक्ष्यां निर्गम्य पार्थिवः}
{ततो वेदविदो वृद्धानपश्यद्ब्राह्मणर्षभान्}


\twolineshloka
{दान्तान्वेदव्रतस्नातान्स्नातानवभृथेषु च}
{सहस्रानुचरान्सौरान्सहस्रं चाष्ट चापरान्}


\twolineshloka
{अक्षतैः सुमनोभिश्च वाचयित्वा महाभुजः}
{तान्द्विजान्मधुसर्पिर्भ्यां फलैः श्रेष्ठैः सुमङ्गलैः}


\twolineshloka
{प्रादात्काञ्चनमेकैकं निष्कं विप्राय पाण्डवः}
{अलङ्कृतं चाश्वशतं वासांसीष्टाश्च दक्षिणाः}


\twolineshloka
{तथा गाः कपिला दोग्ध्रीः सवत्साः पाण्डुनन्दनः}
{हेमशृङ्गा रौप्यखुरा दत्त्वा तेभ्यः प्रदक्षिणम्}


\twolineshloka
{स्वस्तिकान्वर्धमानांश्च नन्द्यावर्तांश्च काञ्चनान्}
{माल्यं च जलकुम्भांश्च ज्वलितं च हुताशनम्}


\twolineshloka
{पूर्णान्यक्षतपात्राणि रुचकं रोचनास्तथा}
{स्वलङ्कृताः शुभाः कन्या दधिसर्पिर्मधूदकम्}


\twolineshloka
{मङ्गल्यान्पक्षिणश्चैव यच्चान्यदपि पूजितम्}
{दृष्ट्वा स्पृष्ट्वा च कौन्तेयो बाह्यां कक्ष्यां ततोऽगमत्}


\twolineshloka
{ततस्तस्यां नहाबाहोस्तिष्ठतः परिचारकाः}
{सौवर्णं सर्वतोभद्रं मुक्तावैदूर्यमण्डितम्}


\twolineshloka
{परार्ध्यास्तरणास्तीर्णं सोत्तरच्छदमृद्धिमत्}
{विश्वकर्मकृतं दिव्यमुपजह्रुर्वरासनम्}


\twolineshloka
{तत्र तस्योपविष्टस्य भूषणानि महात्मनः}
{उपाजह्रुर्महार्हाणि प्रेप्याः शुभ्राणि सर्वशः}


\twolineshloka
{युक्ताभरणवेषस्य कौन्तेयस्य महात्मनः}
{रूपमासीन्महाराज द्विषतां शोकवर्धनम्}


\twolineshloka
{चामरैश्चन्द्ररश्म्याभैर्हेमदण्डैः सुशोभनैः}
{दोधूयमानैः शुशुभे विद्युद्भिरिव तोयदः}


\twolineshloka
{संस्तूयमानः सूत्रैश्च वन्द्यमानश्च वन्दिभिः}
{उपगीयमानो गन्धर्वैरास्ते स्म कुरुनन्दनः}


\twolineshloka
{ततो मुहूर्तमासीच्च वन्दिनां निनदो महान्}
{रथानां नेमिघोषश्च खुरघोषश्च वाजिनाम्}


\twolineshloka
{हादेन गजघण्टानां शङ्खानां निनदेन च}
{नराणां पदशब्दैश्च कम्पतीव स्म मेदिनी}


\twolineshloka
{ततः शुद्धान्तमासाद्य जानुभ्यां भूतले स्थितः}
{शिरसा वन्दनीयं तमभिवाद्य जनेश्वरम्}


\twolineshloka
{कुण्डली बद्धनिस्त्रिंशः सन्नद्धकवचो युवा}
{अभिप्रणम्य शिरसा द्वास्थो धर्मात्मजाय वै}


\twolineshloka
{न्यवेदयद्धृषीकेशमुपयान्तं महात्मने}
{सोऽब्रवीत्पुरुषव्याघ्रः स्वागतेनैव माधवः}


\threelineshloka
{`प्रवेश्यतां समीपं मे किमर्थं प्रविलम्बसे'}
{आसनं च मधुघ्नाय दीयतां परमार्चितम्}
{ततः प्रवेश्य वार्ष्णेयमुपवेश्य वरासने}


% Check verse!
सत्कृत्य सत्कृतस्तेन पर्यपृच्छद्युधिष्ठिरः
\chapter{अध्यायः ८३}
\twolineshloka
{युधिष्ठिर उवाच}
{}


\threelineshloka
{सुखेन रजनी व्युष्टा कच्चित्ते मधुसूदन}
{कच्चिज्ज्ञानानि सर्वाणि प्रसन्नानि तवाच्युत ॥सञ्जाय उवाच}
{}


\threelineshloka
{वासुदेवोऽपि तद्युक्तं प्रत्युवाच युधिष्ठिरम्}
{दर्शनादेव ते सौम्य न किञ्चिदशुभं मम ॥सञ्जय उवाच}
{}


\twolineshloka
{ततश्च प्रकृतीः क्षत्ता न्यवेदयदुपस्थिताः}
{अनुज्ञातश्च राज्ञा स प्रावेशयत तं जनम्}


\twolineshloka
{विराटं भीमसेनं च धृष्टद्युम्नं च सात्यकिम्}
{चेदिपं धृष्टकेतुं च द्रुपदं च महारथम्}


\twolineshloka
{शिखण़्डिनं यमौ चैव चेकितानं सकेकयम्}
{युयुप्सुं चैव कौरव्यं पाञ्चाल्यं चोत्तमौजसम्}


\twolineshloka
{युधामन्यु सुबाहुं च द्रौपदेयांश्च सर्वशः}
{एतांश्च सुहृदश्चान्यान्दर्शयामास पाण्डवम्}


\twolineshloka
{अनुज्ञाताश्च पार्थेन स्वासीना आसनेषु ते}
{सर्वेष्वथ परार्ध्येषु यथार्हं वन्द्य पाण्डवम्}


\twolineshloka
{एकस्मिन्नासने वीरावुपविष्टौ महाबलौ}
{कृष्णश्च युयुधानश्च वेद्यामिव हुताशनौ}


\twolineshloka
{ततो युधिष्ठिरस्तेषां शृण्वतां मधुसूदनम्}
{अब्रवीत्पुण्डरीकाक्षमाभाष्य मधुरं वचः}


\twolineshloka
{एकं त्वां वयमाश्रित्य सहस्राक्षमिवामराः}
{प्रार्थयामो जयं युद्धे शाश्वतानि सुखानि च}


\twolineshloka
{त्वं हि राज्यविनाशं च द्विषद्भिश्च निराक्रियाम्}
{क्लेशांश्च विविधान्कृष्ण सर्वांस्तानपि वेत्थ नः}


\twolineshloka
{त्वयि सर्वेश सर्वेषामस्माकं भक्तवत्सल}
{सुखमायत्तमत्यर्थं यात्रा च मधुसूदन}


\twolineshloka
{स तथा कुरु वार्ष्णेय यथा त्वयि मनो मम}
{अर्जुनस्य यथा सत्या प्रतिज्ञा स्याच्चिकीर्षिता}


\twolineshloka
{स भवांस्तारयत्वस्माद्दुःखामर्षमहार्णवात्}
{परां तितीर्षतामद्य प्लुवो नो भव माधव}


\twolineshloka
{नहि तत्कुरुते योधः कार्तवीर्यसमोऽपि यः}
{युधि यत्कुरुषे कृष्ण सारथ्यं त्वं समास्थितः}


\twolineshloka
{यथैव सर्वास्वापत्सु पासि वृष्णीञ्जनार्दन}
{तथैवास्मान्महाबाहो वृजिनात्त्रातुमर्हसि}


\twolineshloka
{त्वमगाधेऽप्लवे मग्नान्पाण्डवान्कुरुसागरे}
{समुद्धर प्लुवो भूत्वा शङ्खचक्रगदाधर}


\twolineshloka
{नमस्ते देवदेवेश सनातन विशातन}
{विष्णो जिष्णो हरे कृष्ण वैकुण्ठ पुरुषोत्तम}


\twolineshloka
{नारदस्त्वां समाचख्यौ पुराणमृषिसत्तमम्}
{वरदं शार्ङ्गिणं श्रेष्ठं तत्सत्यं कुरु माधव}


\threelineshloka
{इत्युक्तः पुण्डरीकाक्षो धर्मराजेन संसदि}
{तोयमेघस्वनो वाग्मी प्रत्युवाच युधिष्ठिरम् ॥वासुदेव उवाच}
{}


\twolineshloka
{सामरेष्वपि लोकेषु सर्वेषु न तथाविधः}
{शरासनधरः कश्चिद्यथा पार्थो धनञ्जयः}


\twolineshloka
{वीर्यवानस्त्रसम्पन्नः पराक्रान्तो महाबलः}
{युद्धशौण्डः सदामर्षी तेजसा परमो नृणाम्}


\twolineshloka
{स युवा वृषभस्कन्धो दीर्घबाहुर्महाबलः}
{सिंहर्षभगतिः श्रीमान्द्विषतस्ते हनिष्यति}


\twolineshloka
{अहं च तत्करिष्यामि तथा कुन्तीसुतोऽर्जुनः}
{धार्तराष्ट्रस्य सैन्यानि धक्ष्यत्यग्निरिवेन्धनम्}


\twolineshloka
{अद्य तं पापकर्माणं क्षुद्रं सौभद्रघातिनम्}
{अपुनर्दर्शनं मार्गमिषुभिः क्षेप्स्यतेऽर्जुनः}


\twolineshloka
{तस्याद्य गृध्राः श्येनाश्च चण्डगोमायवस्तथा}
{भक्षयिष्यन्ति मांसानि ये चान्ये पुरुषादकाः}


\twolineshloka
{यद्यस्य देवा गोप्तारः सेन्द्राः सर्वे तथाऽप्यसौ}
{राजधानीं यमस्याद्य हतः प्राप्स्यति दुर्मतिः}


\twolineshloka
{निहत्य सैन्धवं जिष्णुरद्य त्वामुपयास्यति}
{विशोको विज्वरो राजन्भव शान्ति पुरस्कृतः}


\chapter{अध्यायः ८४}
\twolineshloka
{सञ्जय उवाच}
{}


\twolineshloka
{तथा तु वदतां तेषां प्रादुरासीद्धनञ्जयः}
{दिदृक्षुर्भरतश्रेष्ठं राजानं ससुहृद्गणम्}


\twolineshloka
{तं प्रविष्टं शुभां कक्ष्यामभिवन्द्याग्रतः स्यितम्}
{तमुत्थायार्जुनं प्रेम्णा सस्वजे पाण्डवर्षभः}


\twolineshloka
{मूर्ध्नि चैनमुपाघ्राय परिष्वज्य च बाहुना}
{आशिषः परमाः प्रोच्य स्मयमानोऽभ्यभाषत}


\threelineshloka
{व्यक्तमर्जुन सङ्ग्रामे ध्रुवस्ते विजयो महान्}
{तादृग्रूपा च ते च्छाया प्रसन्नश्च जनार्दनः ॥सञ्जय उवाच}
{}


\twolineshloka
{तमब्रवीत्ततो जिष्णुर्महदाश्चर्यमुत्तमम्}
{दृष्टवानस्मि भद्रं ते केशवस्य प्रसादजम्}


\twolineshloka
{ततस्तत्कथयामास यथादृष्टं धनञ्जयः}
{आश्वासनार्थं सुहृदां त्र्यम्बकेण समागमम्}


\twolineshloka
{ततः शिरोभिरवनिं स्पृष्ट्वा सर्वे च विस्मिताः}
{नमस्कृत्य वृषाङ्काय साधुसाध्वित्यथाब्रुवन्}


\twolineshloka
{अनुज्ञातास्ततः सर्वे सुहृदो धर्मसूनुना}
{त्वरमाणाः सुसन्नद्धा हृष्टा युद्धाय निर्ययुः}


\twolineshloka
{अभिवाद्य तु राजानं युयुधानाच्युतार्जुनाः}
{हृष्टा विनिर्ययुस्ते वै युधिष्ठिरनिवेशनात्}


\twolineshloka
{रथेनैकेन दुर्धर्षौ युयुधानजनार्दनौ}
{जग्मतुः सहितौ वीरावर्जुनस्य निवेशनम्}


\twolineshloka
{तत्र गत्वा हृषीकेशः कल्पयामास शास्त्रवित्}
{रथं रथवरस्याजौ वानरर्षभलक्षणम्}


\twolineshloka
{स मेघसमनिर्घोषस्तप्तकाञ्चनसप्रभः}
{बभौ रथवरः क्लृप्तः शिशुर्दिवसकृद्यथा}


\twolineshloka
{ततः पुरुषशार्दूलः सज्जं सर्वं न्यवेदयत्}
{रथं पुरुषसिंहस्य सध्वजं सुपताकिनम्}


\twolineshloka
{स तु लोकवरः पुंसां किरीटी हेमवर्मभृत्}
{चापबाणधरो वाहं प्रदक्षिणमवर्तत}


\twolineshloka
{तपोविद्यावयोवृद्धैः क्रियावद्भिर्जितेन्द्रियैः}
{स्तूयमानो जयाशीर्भिरारुरोह महारथम्}


\twolineshloka
{जैत्रैः साङ्ग्रामिकैर्मन्त्रैः पूर्वमेव रथोत्तमम्}
{अभिमन्त्रितमर्चिष्मानुदयं भास्करो यथा}


\threelineshloka
{स रथे रथिनां श्रेष्ठः काञ्चने काञ्चनावृतः}
{विबभौ विमलोऽर्चिष्मान्मेराविव दिवाकरः}
{}


\twolineshloka
{अन्वारुरुहतुः पार्थं युयुधानजनार्दनौ}
{शर्यातेर्यज्ञमायान्तं यथेन्द्रं देवमश्विनौ}


\twolineshloka
{अथ जग्राह गोविन्दो रश्मीन्रश्मिविदांवरः}
{मातलिर्वासवस्येव वृत्रं हन्तुं प्रयास्यतः}


\twolineshloka
{स ताभ्यां सहितः पार्थो रथप्रवरमास्थितः}
{सहितो बुधशुक्राभ्यां तमो निघ्नन्यथा शशी}


\twolineshloka
{सैन्धवस्य वधप्रेप्सुः प्रयातः शत्रुपूगहा}
{सहाम्बुपतिमित्राभ्यां यथेन्द्रस्तारकामये}


\twolineshloka
{ततो वादित्रनिर्घोषैर्माङ्गल्यैश्च स्तवैः शुभैः}
{प्रयान्तमर्जुनं वीरं मागधाश्चैव तुष्टुवुः}


\twolineshloka
{सजयाशीः सपुण्याहः सूतमागधनिःस्वनः}
{युक्तो वादित्रघोषेण तेषां रतिकरोऽभवत्}


\twolineshloka
{तमनुप्रवणो वायुः पुण्यगन्धवहः शुभः}
{ववौ संहर्षयन्पार्थं द्विषतश्चापि शोषयन्}


\twolineshloka
{ततस्तस्मिन्क्षणे राजन्विविधानि शुभानि च}
{प्रादुरासन्निमित्तानि विजयाय बहूनि च}


\threelineshloka
{पाण्डवानां त्वदीयानां विपरीतानि मारिष}
{दृष्ट्वाऽर्जुनो निमित्तानि विजयाय प्रदक्षिणम्}
{युयुधानं महेष्वासमिदं वचनमब्रवीत्}


\twolineshloka
{युयुधानाद्य युद्धे मे दृश्यते विजयो ध्रुवः}
{यथाहीमानि लिङ्गानि दृश्यन्ते शिनिपुङ्गव}


\twolineshloka
{सोऽहं तत्र गमिष्यामि यत्र सैन्धवको नृपः}
{यियासुर्यमलोकाय मम वीर्यं प्रतीक्षते}


\twolineshloka
{यथा परमकं कृत्यं सैन्धवस्य वधो मम}
{तथैव सुमहत्कृत्यं धर्मराजस्य रक्षणम्}


\twolineshloka
{स त्वमद्य महाबाहो राजानं परिपालय}
{यथैव हि मया गुप्तस्त्वया गुप्तो भवेत्तथा}


\twolineshloka
{न पश्यामि च तं लोके यस्त्वां युद्धे पराजयेत्}
{वासुदेवसमं युद्धे स्वयमप्यमरेश्वरः}


\twolineshloka
{त्वयि चाहं पराश्वस्तः प्रद्युम्ने वा महारथे}
{शक्नुयां सैन्धवं हन्तुमनपेक्षो नरर्षभ}


\twolineshloka
{मय्यपेक्षा न कर्तव्या गोप्ताऽयं मम केशवः}
{राज्ञ एव परा गुप्तिः कार्या सर्वात्मना त्वया}


\twolineshloka
{न हि यत्र महाबाहुर्वासुदेवो व्यवस्थितः}
{किञ्चिद्व्यापद्यते तत्र यत्राहमपि च ध्रुवम्}


\twolineshloka
{एवमुक्तस्तु पार्थेन सात्यकिः परवीरहा}
{तथेत्युक्त्वाऽगमत्तत्र यत्र राजा युधिष्ठिरः}


\chapter{अध्यायः ८५}
\twolineshloka
{धृतराष्ट्र उवाच}
{}


\twolineshloka
{श्वोभूते किमकार्षुस्ते दुःखशोकसमन्विताः}
{अभिमन्यौ हते तत्र कैर्वाऽयुध्यन्त पाण्डवाः}


\twolineshloka
{जानन्तस्तस्य कर्माणि कुरवः सव्यसाचिनः}
{कथं तत्किल्बिषं कृत्वा निर्भया ब्रूहि मामकाः}


\twolineshloka
{पुत्रशोकाभिसन्तप्तं क्रुद्धं मृत्युमिवान्तकम्}
{आयान्तं पुरुषव्याघ्रं कथं ददृशुराहवे}


\twolineshloka
{कपिराजध्वनं सङ्ख्ये विधुन्वानं महद्धनुः}
{दृष्ट्वा पुत्रपरिद्यूनं किमकुर्वत मामकाः}


\twolineshloka
{किन्तु सञ्जय सङ्ग्रमे वृत्तं दुर्योधनं प्रति}
{परिदेवो महानद्य श्रूयते हि गृहेगृहे}


\twolineshloka
{भूयाञ्शब्दानतीत्यान्याञ्शब्दः सैन्धववेश्मनि}
{पौराणिकानां तूर्याणां शङ्खदुन्दुभिघोषवान्}


\twolineshloka
{सूतमागधबन्दीनां नर्तकानां च निःस्वनः}
{सोऽद्य न श्रूयते शब्दः सूत पुत्र यथा पुरा}


\twolineshloka
{शब्दो नानाविधोऽभीष्टमभवद्यत्र मे श्रुतः}
{दीनानामद्य तं शब्दं न शृणोमि समीरितम्}


\twolineshloka
{निवेशने सत्यधृतेः सोमदत्तस्य सञ्जय}
{आसीनोऽहं पुरा तात शब्दमश्रौषमुत्तमम्}


\twolineshloka
{तदद्य पुण्यहीनोऽहमार्तस्वरनिनादितम्}
{निवेशनं गतोत्साहं विकृतं तात लक्षये}


\twolineshloka
{विविंशतेर्दुर्मुखस्य चित्रसेनविकर्णयोः}
{अन्येषां च सुतानां मे न तथा श्रूयते ध्वनिः}


\twolineshloka
{ब्राह्मणआः क्षत्रिया वैश्या यं शिष्याः पर्युपासते}
{द्रोणपुत्रं महेष्वासं पुत्राणां मे परायणम्}


\twolineshloka
{वितण्डालापसंलापैर्द्रुतवादित्रवादितैः}
{गीतैश्च विविधैरिष्टै रमते यो दिवानिशम्}


\twolineshloka
{उपास्यमानो बहुभिः कुरुपाण्डवसात्वतैः}
{श्लक्ष्णस्तस्य गृहे शब्दो नाद्य द्रौणेर्यथा पुरा}


\twolineshloka
{द्रोणपुत्रं महेष्वासं गायना नर्तकाश्च ये}
{अत्यर्थमुपतिष्ठन्ति तेषां न श्रूयते ध्वनिः}


\twolineshloka
{विन्दानुविन्दयोः सायं शिबिरे यो महाध्वनिः}
{श्रूयते सोऽद्य न तथा केकयानां च वेश्मसु}


\twolineshloka
{नित्यं प्रमुदितानां च तालगीतस्वनो महान्}
{नृत्यतां श्रूयते तात गणानां सोऽद्य न स्वनः}


\twolineshloka
{सप्ततन्तून्वितन्वाना याजका यमुपासते}
{सौमदत्तिं श्रुतनिधिं तेषां न श्रूयते ध्वनिः}


\twolineshloka
{ज्याघोषो ब्रह्मघोषश्च तोमरासिरथध्वनिः}
{द्रोणस्यासीदविरतो गृहे तं न शृणोम्यहम्}


\twolineshloka
{नानादेशसमुत्थानां गीतानां योऽभवत्स्वनः}
{वादित्रनादितानां च सोऽद्य न श्रूयते महान्}


\twolineshloka
{यदाप्रभृत्युपप्लुव्याच्छान्तिमिच्छञ्जनार्दनः}
{आगतः सर्वभूतानामनुकम्पार्थमच्युतः}


\twolineshloka
{ततोऽहमब्रुवं सूत मन्दं दुर्योधनं तदा}
{वासुदेवेन तीर्थेन पुत्र संशाम्य पाण्डवैः}


\twolineshloka
{कालप्राप्तमहं मन्ये मा त्वं दुर्योधनातिगाः}
{`संशाम्य भ्रातृभिस्तैस्तु क्षत्रियानभिपालय'}


\twolineshloka
{शमं चेद्याचमानं त्वं प्रत्याख्यास्यसि केशवम्}
{हितार्थमभिजल्पन्तं न तवास्ति रणे जयः}


\twolineshloka
{प्रत्याचष्ट स दाशार्हमृषभं सर्वधन्विनाम्}
{अनुनेयानि जल्पन्तमनयान्नान्वपद्यत}


\twolineshloka
{`कर्णदुःशासनमतः सौबलस्य च दुर्मतेः}
{प्रत्याख्यातो महाबाहुःकुलान्तकरणेन वै'}


\twolineshloka
{ततो दुःशासनस्यैव कर्णस्य च मतं द्वयोः}
{अन्ववर्तत मां हित्वा कृष्टः कालेन दुर्मतिः}


\twolineshloka
{न ह्यहं युद्धमिच्छामि विदुरो द्रोण एव वा}
{बाह्लीकः सोमदत्तो वा भीष्मो द्रौणायनिस्तथा}


\twolineshloka
{शलो भूरिश्रवाश्चैव पुरुमित्रो विविंशतिः}
{जयः कृपो वा धर्मात्मा ये चान्ये मम बान्धवाः}


\twolineshloka
{एतेषां मतमास्थाय यदि वत्स्यति पुत्रकः}
{एतेभ्यश्च मदूर्ध्वं च अभोक्ष्यद्वसुधामिमाम्}


\threelineshloka
{श्लक्ष्णा मधुरसम्भाषा ज्ञातिमध्ये प्रियंवदाः}
{कुलीनाः सम्मताः प्राज्ञाः सुखं जीवन्ति मानवाः}
{}


\twolineshloka
{धर्मापेक्षी नरो नित्यं सर्वत्र लभते सुखम्}
{प्रेत्यभावे च कल्याणं प्रसादं प्रतिपद्यते}


\twolineshloka
{अर्हास्ते पृथिवीं भोक्तुं समर्थाः साधनेऽपि च}
{तेषामपि समुद्रान्ता पितृपैतामही मही}


\twolineshloka
{`न च त्वाऽभिभविष्यन्ति हित्वा धर्मं पृथात्मजाः}
{नियुज्यमानाः स्थास्यन्ति पाण्डवा धर्मवर्त्मनि}


\twolineshloka
{सन्ति मे ज्ञातयस्तात येषां श्रोष्यन्ति पाण्डवाः}
{शल्यस्य सोमदत्तस्य भीष्मस्य च महात्मनः}


\threelineshloka
{द्रोणस्याथ विकर्णस्य बाह्लीकस्य कृपस्य च}
{अन्येषां चैव वृद्धानां भरतानां महात्मनाम्}
{त्वदर्थमब्रवं तात न जहुर्वचनानि ते}


\twolineshloka
{कं वा त्वं मन्यसे तेषां यस्त्वां ब्रूयादतोऽन्यथा}
{कृष्णो न धर्मं सञ्जह्यात्सर्वे ते हि तदन्वयाः}


\twolineshloka
{मयाऽपि चोक्तास्ते वीरा वचनं धर्मसंहितम्}
{नान्यथा प्रकरिष्यन्ति धर्मात्मानो हि पाण्डवाः}


\twolineshloka
{इत्यहं विलपन्सूत बहुशः पुत्रमुक्तवान्}
{न च मे श्रुतवान्मूढो मन्ये कालस्य पर्ययम्}


\twolineshloka
{वृकोदरार्जुनौ यत्र वृष्णिवीरश्च सात्यकिः}
{उत्तमौजाश्च पाञ्चाल्यो युधामन्युश्च दुर्जयः}


\twolineshloka
{धृष्टद्युम्नश्च दुर्धर्षः शिखण्डी चापराजितः}
{अश्मकाः केकयाश्चैव क्षत्रधर्मा च सौमकिः}


\twolineshloka
{चैद्यश्च चेकितानश्च पुत्रः काश्यस्य चाभिभूः}
{द्रौपदेया विराटश्च द्रुपदश्च महारथः}


\twolineshloka
{यमौ च पुरुषव्याघ्रौ मन्त्री च मधुसूदनः}
{क एताञ्जातु युध्येत लोकेऽस्मिन्वै जिजीविषुः}


\threelineshloka
{दिव्यमस्त्रं विकुर्वाणान्प्रसहेद्वा परान्मम}
{अन्यो दुर्योधनात्कार्णाच्छकुनेश्चापि सौबलात्}
{दुःशासनचतुर्थानां नान्यं पश्यामि पञ्चमम्}


\twolineshloka
{येषां वचनकृत्स स्याद्विष्वक्सेनो रथे स्थितः}
{सन्नद्धश्चार्जुनो योद्धा तेषां नास्ति पराजयः}


\twolineshloka
{तथा मम विलापानां नायं दुर्योधनः स्मरेत्}
{हतौ हि पुरुषव्याघ्रौ भीष्मद्रोणौ त्वमात्थ वै}


\twolineshloka
{तेषां विदुरवाक्यानामुक्तानां दीर्घदर्शनात्}
{दृष्ट्वेमां फलनिर्वृत्तिं मन्ये शोचन्ति पुत्रकाः}


\twolineshloka
{सेनां दृष्ट्वाऽभिभूतां मे शैनेयेनार्जुनेन च}
{शून्यान्दृष्ट्वा रथोपस्थान्मन्ये शोचन्ति पुत्रकाः}


\twolineshloka
{हिमात्यये यथा कक्षं शुष्कं वातेरितो महान्}
{अग्निर्दहेत्तथा सेनां मामिकां स धनञ्जयः}


\twolineshloka
{आचक्ष्व मम तत्सर्वं कुशलो ह्यसि सञ्जय}
{यदोपयाताः सायाह्ने कृत्वा पार्थस्य किल्बिषं}


% Check verse!
अभिमन्यौ हते तात कथमासीन्मनो हि वः
\twolineshloka
{न जातु तस्य कर्माणि युधि गाण्डीवधन्वनः}
{अपकृत्य महत्तात सोढुं शक्ष्यन्ति मामकाः}


\twolineshloka
{किन्नुः दुर्योधनः कृत्यं कर्णः कृत्यं किमब्रवीत्}
{दुःशासनः सौबलश्च तेषामेवं गते सति}


\twolineshloka
{सर्वेषां समवेतानां पुत्राणां मम सञ्जय}
{यद्वृत्तं तात सङ्ग्रामे मन्दस्यापनयैर्भृशम्}


\threelineshloka
{लोभानुगस्य दुर्बुद्धेः क्रोधेन विकृतात्मनः}
{राज्यकामस्य मूढस्य रागोपहतचेतसः}
{दुर्नीतं वा सुनीतं वा तन्ममाचक्ष्व सञ्जय}


\chapter{अध्यायः ८६}
\twolineshloka
{सञ्जय उवाच}
{}


\twolineshloka
{हन्त ते सम्प्रवक्ष्यामि सर्वं प्रत्यक्षदर्शिवान्}
{शुश्रूषस्व स्थिरो भूत्वा तव ह्यपनयो महान्}


\twolineshloka
{गतोदके सेतुबन्धो यादृक्तादृगयं तव}
{विलापो निष्फलो राजन्मा शुचो भरतर्षभ}


\twolineshloka
{अनतिक्रमणीयोऽयं कृतान्तस्याद्भुतो विधिः}
{मा शुचो भरतश्रेष्ठ दिष्टमेतत्पुरातनम्}


\twolineshloka
{यदि त्वं हि पुरा द्यूतात्कुन्तीपुत्रं युधिष्ठिरम्}
{निवर्तयेथाः पुत्रांश्च न त्वां व्यसनमाव्रजेत्}


\twolineshloka
{युद्धकाले पुनः प्राप्ते तदैव भवता यदि}
{निवर्तिताः स्युः संरब्धा न त्वां व्यसनमाव्रजेत्}


\twolineshloka
{दुर्योधनं चाविधेयं बध्नीयास्त्वं पुरा यदि}
{अवेक्ष्य कौरवान्राजन्प्राप्स्यसे त्वं महद्यशः}


\twolineshloka
{तत्ते बुद्धिव्यतीचारमुपलप्स्यन्ति तज्जनाः}
{पाञ्चाला वृष्णयः सर्वे ये चान्येऽपि नराधिपाः}


\twolineshloka
{स कृत्वा पितृकर्म त्वं पुत्रं संस्थाप्य सत्पथे}
{वर्तेथा यदि धर्मेण न त्वां व्यसनमाव्रजेत्}


\twolineshloka
{त्वं तु प्राज्ञतमो लोके हित्वा धर्मं सनातनम्}
{दुर्योधनस्य कर्णस्य शकुनेश्चान्वगा मतम्}


\twolineshloka
{तत्ते विलपितं सर्वं मया राजन्निशामितम्}
{अर्थे निविशमानस्य विषमिश्रं यथा मधु}


\twolineshloka
{न तथा मन्यते कृष्णो न च राजा युधिष्ठिरः}
{न भीष्मो नैव च द्रोणो यथा त्वं मन्यसे नृप}


\twolineshloka
{व्यजानीत यदा तु त्वां लुब्धं धर्मप्रवादिनम्}
{तदाप्रभृति कृष्मस्त्वां न तथा बहुमन्यते}


\twolineshloka
{परुषाण्युच्यमानांश्च यथा पार्थानुपेक्षसे}
{तस्यानुबन्धः प्राप्तस्त्वां पुत्रामां राज्यकामुक}


\twolineshloka
{पितृपैतामहं राज्यमपावृत्तं त्वयाऽनघ}
{अथ पार्थैर्जितां कृत्स्नां पृथिवीं प्रत्यपद्यथाः}


\twolineshloka
{पाण्डुना निर्जितं राज्यं कौरवाणां यशस्तथा}
{एतच्चाप्यधिकं भूयः पाण्डवैर्धर्मचाखिभिः}


\twolineshloka
{तेषां तत्तादृशं कर्म त्वामासाद्य सुनिष्फलम्}
{यत्पित्र्याद्धंशिता राज्यात्त्वयेहामिषगृद्धिना}


\twolineshloka
{यत्पुनर्युद्धकाले त्वं पुत्रान्गर्हयसे नृप}
{बहुधा व्याहरन्दोषान्न तदद्योपपद्यते}


\twolineshloka
{न हि रक्षन्ति राजानो युध्यन्तो जीवितं रणे}
{चमूं विगाह्य पार्थानां युध्यन्ते क्षत्रियर्षभाः}


\twolineshloka
{यां तु कृष्णार्जुनौ सेनां यां सात्यकिवृकोदरौ}
{रक्षेरन्को नु तां युध्येच्चमूमन्यत्र कौरवात्}


\twolineshloka
{येषां योद्धा गुडाकेशो येषां मन्त्री जनार्दनः}
{येषां च सात्यकिर्योद्धा येषां योद्धा वृकोदरः}


\twolineshloka
{को हि तान्विषहेद्योद्धुं मर्त्यधर्मा धनुर्धरः}
{अन्यत्र कौरवेयेभ्यो ये वा तेषां पदानुगाः}


\twolineshloka
{यावत्तु शक्यते कर्तुमन्तरज्ञैर्जनाधिपैः}
{क्षत्रधर्मरतैः शूरैस्तावत्कुर्वन्ति कौरवाः}


\twolineshloka
{यथा तु पुरुषव्याघ्रैर्युद्धं परमसङ्कटम्}
{कुरूणां पाण्डवैः सार्धं तत्सर्वं शृणु तत्त्वतः}


\chapter{अध्यायः ८७}
\twolineshloka
{सञ्जय उवाच}
{}


\twolineshloka
{तस्यां निशायां व्युष्टायां द्रोणः शस्त्रभृतां वरः}
{स्वान्यनीकानि सर्वाणि प्राक्रामद्व्यूहितुं ततः}


\twolineshloka
{शूराणां गर्जतां राजन्सङ्क्रुद्धानाममर्षिणाम्}
{श्रूयन्ते स्म गिरश्चित्राः परस्परवधैषिणाम्}


\twolineshloka
{विष्फार्य च धनूंष्यन्ये ज्याः परे परिमृज्य च}
{विनिःश्वसन्तः प्राक्रोशन्क्वेदानीं स धनञ्जयः}


\twolineshloka
{विकोशान्सुत्सरूनन्ये कृतधारान्समाहितान्}
{पीतानाकाशसङ्काशानसीन्केचिच्च चिक्षिपुः}


\twolineshloka
{चरन्तस्त्वसिमार्गांश्च धनुर्मार्गांश्च शिक्षिताः}
{सङ्ग्राममनसः शूरा दृश्यन्ते स्म सहस्रशः}


\threelineshloka
{सघण्टाश्चन्दनादिग्धाः स्वर्णवज्रविभूषिताः}
{समुत्क्षिप्य गदाश्चान्ये पर्यपृच्छन्त पाण्डवम्}
{}


\twolineshloka
{अन्ये बलमदोन्मत्ताः परिघैर्बाहुशालिनः}
{चक्रुः सम्बाधमाकाशमुच्छ्रितेन्द्रध्वजोपमैः}


\twolineshloka
{नानाप्रहरणैश्चान्ये विचित्रस्रगलङ्कृताः}
{सङ्ग्राममनसः शूरास्तत्रतत्र व्यवस्थिताः}


\twolineshloka
{क्वार्जुनः क्व स गोविन्दः क्व च मानी वृकोदरः}
{क्व च ते सुहृदस्तेषामाह्वयन्ते रणे तदा}


\twolineshloka
{ततः शङ्खमुपाध्माय त्वरयन्वाजिनः स्वयम्}
{इतस्ततस्तान्रचयन्द्रोणश्चरति वेगितः}


\twolineshloka
{तेष्वनीकेषु सर्वेषु स्थितेष्वाहवनन्दिषु}
{भारद्वाजो महाराज जयद्रथमथाब्रवीत्}


\twolineshloka
{त्वं चैव सौमदत्तिश्च कर्णश्चैव महारथः}
{अश्वत्थामा च शल्यश्च वृषसेनः कृपस्तथा}


\twolineshloka
{शतं चाश्वसहस्राणां रथानामयुतानि षट्}
{द्विरदानां प्रभिन्नानां सहस्राणि चतुर्दश}


\twolineshloka
{पदातीनां सहस्राणि दंशितान्येकविंशतिः}
{गव्यूतिषु त्रिमात्रासु मामनासाद्य तिष्ठत}


\threelineshloka
{तत्रस्थं त्वां न संसोढुं शक्ता देवाः सवासवाः}
{किं पुनः पाण्डवाः सर्वे समाश्वसिहि सैन्धव ॥सञ्जय उवाच}
{}


\threelineshloka
{एवमुक्तः समाश्वस्तः सिन्धुराजो जयद्रथः}
{सम्प्रायात्सहगान्धारैर्वृतस्तैश्च महारथैः}
{वर्मिभिः सादिभिर्यत्तैः प्रासपाणिहयस्थितैः}


\twolineshloka
{चामरापीडिनः सर्वे जाम्बूनदविभूषिताः}
{जयद्रथस्य राजेन्द्र हयाः साधुप्रवाहिनः}


% Check verse!
ते चैकसप्तसाहस्रास्त्रिसाहस्राश्च सैन्धवाः
\twolineshloka
{मत्तानां सुविरूढानां हस्त्यारोहैर्विशारदैः}
{हस्तिनां भीमरूपाणां वर्मिणां रौद्रकर्मिणाम्}


\twolineshloka
{अध्यर्धेन सहस्रेण पुत्रो दुर्मर्षणस्तव}
{अग्रतः सर्वसैन्यानां योत्स्यमानो व्यवस्थितः}


\twolineshloka
{ततो दुःशासनश्चैव विकर्णश्च तवात्मजौ}
{सिन्धुराजार्थसिद्ध्यर्थमग्रानीके व्यवस्थितौ}


\twolineshloka
{दीर्घो द्वादशगव्यूतिः पश्चार्धे पञ्चविस्तृतः}
{व्यूहस्तु चक्रशकटो भारद्वाजेन निर्मितः}


\twolineshloka
{नानानृपतिभिर्वीरैस्तत्रतत्र व्यवस्थितैः}
{रथाश्वगजपत्त्योघैर्द्रोणेन विहितः स्वयम्}


\twolineshloka
{पश्चार्धे तस्य पद्मस्तु गर्भव्यूहः पुनः कृतः}
{सूचीपद्मस्य गर्भस्थो गूढो व्यूहः कृतः पुनः}


\twolineshloka
{एवमेतं महाव्यूहं व्यूह्य द्रोणो व्यवस्थितः}
{सूचीमुखे महेष्वासः कृतवर्मा व्यवस्थितः}


\twolineshloka
{अनन्तरं च काम्भोजो जलसन्धश्च मारिष}
{दुर्योधनश्च कर्णश्च तदनन्तरमेव च}


\twolineshloka
{ततः शतसहस्राणि योदानानमनिवर्तिनाम्}
{व्यवस्थितानि सर्वाणि शकटे मुखरक्षिणाम्}


\twolineshloka
{तेषां च पृष्ठतो राजा बलेन महता वृतः}
{जयद्रथस्ततो राजा सूचीपार्श्वे व्यवस्थितः}


\twolineshloka
{शकटस्य तु राजेन्द्र भारद्वाजो मुखे स्थितः}
{नात्र तस्याभवद्भोदो जुगोपैनं ततः स्वयम्}


\threelineshloka
{श्वेतवर्माम्बरोष्णीषो व्यूढोरस्को महाभुजः}
{धनुर्विस्फारयन्द्रोणस्तस्थौ क्रुद्ध इवान्तकः}
{`तस्य सैन्यस्य सर्वस्य नेता गोप्ता च वीर्यवान्'}


\twolineshloka
{ताकिनं शोणहयं वेदीकृष्णाजिनध्वजम्}
{द्रोणस्य रथमालोक्य प्रहृष्टाः कुरवोऽभवन्}


\twolineshloka
{सिद्धचारणसङ्घानां विस्मयः सुमहानभूत्}
{द्रोणेन विहितं दृष्ट्वा व्यूहं क्षुब्धार्णवोपमम्}


\twolineshloka
{सशैलसागरवनां नानाजनपदाकुलाम्}
{ग्रसेद्व्यूहः क्षितिं सर्वामिति भूतानि मेनिरे}


\twolineshloka
{बहुरथमनुजाश्वपत्तिनागंप्रतिभयनिःस्वनमद्भुतानुरूपम्}
{अहितहृदयभेदनं महद्वैशकटमवेक्ष्य कृतं ननन्द राजा}


\chapter{अध्यायः ८८}
\twolineshloka
{सञ्जय उवाच}
{}


\twolineshloka
{व्यूढेषु तव सैन्येषु समुत्क्रुष्टेषु मारिष}
{ताड्यमानासु भेरीषु मृदङ्गेष्वानकेषु च}


\twolineshloka
{अनीकानां च संह्रादे वादित्राणां च निःस्वने}
{प्रध्मापितेषु शङ्खेषु सन्नादे रोमहर्षणे}


\twolineshloka
{प्रगृहीतेषु शस्त्रेषु भारतेषु युयुत्सुषु}
{रौद्रे मुहूर्ते सम्प्राप्ते सव्यसाची व्यदृश्यत}


\twolineshloka
{बलानां वायसानां च पुरस्तात्सव्यसाचिनः}
{पिशितासृग्भुजां सङ्घाः प्रलीयन्ते सहस्रशः}


% Check verse!
मृगाश्च घोरसन्नादः शिवाश्चाशिवदर्शनाः ॥दक्षिणेन प्रयातानामस्माकं नेदुरध्वनि
\twolineshloka
{`लोकक्षये महाराज यादृशास्तादृशा हि ते}
{अशिवा धार्तराष्ट्राणां शिवाः पार्थस्य संयुगे'}


\twolineshloka
{सनिर्घाता ज्वलन्त्यश्च पेतुरुल्काः सहस्रशः}
{चचाल च मही कृत्स्ना भये घोरे समुत्थिते}


\twolineshloka
{विष्वग्वाताः सनीहारा रूक्षाः शर्करकर्षिणः}
{ववुरायाति कौन्तेये सङ्ग्रामे समुपस्थिते}


\twolineshloka
{नाकुलिश्च शतानीको धृष्टद्युम्नश्च पार्षतः}
{पाण्डवानामनीकानि प्राज्ञौ तौ व्यूहतुस्तदा}


\twolineshloka
{ततो रथसहस्रेण द्विरदानां शतेन च}
{त्रिभिरश्वसहस्रैश्च पदातीनां शतैः शतैः}


\twolineshloka
{अध्यर्धमात्रे धनुषां सहस्रे तनयस्तव}
{अग्रतः सर्वसैन्यानां स्थित्वा दुर्मर्षणोऽब्रवीत्}


\twolineshloka
{अद्य गाण्डीवधन्वानं तपन्तं युद्धदुर्मदम्}
{अहमावारयिष्यामि वेलेव मकरालयम्}


\twolineshloka
{अद्य पश्यन्तु सङ्ग्रामे धनञ्जयममर्षणम्}
{विषक्तं मयि दुर्धर्षं भिन्नं कुम्भमिवाश्मनि}


\twolineshloka
{तिष्ठध्वं रथिनो यूयं सङ्ग्राममभिकाङ्क्षिणः}
{युध्यामि संहतानेतान्यशो मानं च वर्धयन्}


\twolineshloka
{एवं ब्रुवन्महाराज महात्मा स महीपतिः}
{महेष्वासैर्वृतो राजन्महेष्वासो व्यवस्थितः}


\twolineshloka
{ततोऽर्जुनो महाराज महात्मा महतां पतिः}
{महारथसमाख्यातो महेष्वासो व्यवस्थितः}


\twolineshloka
{कक्षाग्निरिव दुःस्पर्शः सवज्र इव वासवः}
{दण्डपाणिरिवासह्यो मृत्युः कालेन चोदितः}


\twolineshloka
{शूलपाणिरिवाक्षोभ्यो वरुणः पाशवानिव}
{युगान्ताग्निरिवार्चिष्मान्दिधक्षुः स्थास्नुजङ्गमम्}


\twolineshloka
{क्रोधामर्षबलोद्धूतो निवातकवचान्तकः}
{युद्धे जेता स्थितः सत्ये पारयिष्यन्महाव्रतम्}


\twolineshloka
{आमुक्तकवचः खङ्गी जाम्बूनदकिरीटभृत्}
{शुभ्रमाल्याम्बरधरः स्वङ्गदश्चारुकुण़्डलः}


\twolineshloka
{रथप्रवरमास्थाय नरो नारायणानुगः}
{विधुन्वन्गाण्डिवं सङ्ख्ये बभौ सूर्य इवोदितः}


\twolineshloka
{अग्रानीकस्य सोऽध्यर्ध इषुपाते धनञ्जयः}
{व्यवस्थाप्य रथं राजञ्शङ्खं दध्मौ प्रतापवान्}


\twolineshloka
{अथ कृष्णोऽप्यसम्भ्रान्तः पार्थेन सह मारिष}
{प्राध्मापयत्पाञ्चजन्यं शङ्खप्रवरमोजसा}


\twolineshloka
{तयोः शङ्खप्रणादेन तव सैन्ये विशाम्पते}
{आसन्संहृष्टरोमाणः कम्पिता गतचेतसः}


\twolineshloka
{यथा त्रस्यन्ति भूतानि सर्वाण्यशनिनिःस्वनात्}
{तथा शङ्खप्रणादेन वित्रेसुस्तव सैनिकाः}


\twolineshloka
{प्रसुस्रुवुः शकृन्मूत्रं वाहनानि च सर्वशः}
{एवं सवाहनं सर्वमाविग्नमभवद्बलम्}


\twolineshloka
{सीदन्ति स्म नरा राजञ्शङ्खशब्देन मारिष}
{विसंज्ञाश्चाभवन्केचित्केचिद्राजन्वितत्रसुः}


\twolineshloka
{ततः कपिर्महानादं सहभूतैर्ध्वजालयैः}
{अकरोद्व्यादितास्यश्च भीषयंस्तव सैनिकान्}


\twolineshloka
{ततः शङ्खाश्च भेर्यश्च मृदङ्गाश्चानकैः सह}
{पुनरेवाभ्यहन्यन्त तव सैन्यप्रहर्षणाः}


\twolineshloka
{नानावादित्रसंहादैः क्ष्वेडितास्फोटिताकुलैः}
{सिंहनादैः समुत्क्रुष्टैः समाधूतैर्महारथैः}


\twolineshloka
{तस्मिंस्तु तुमुले शब्दे भीरूणां भयवर्धने}
{अतीव हृष्टो दाशार्हमब्रवीत्पाकशासनिः}


\chapter{अध्यायः ८९}
% Check verse!
अर्जुन उवाच
\threelineshloka
{चोदयाश्वान्हृषीकेश यत्र दुर्मर्षणः स्थितः}
{एतद्भित्त्वा गजानीकं प्रवेक्ष्याम्यरिवाहिनीम् ॥सञ्जय उवाच}
{}


\twolineshloka
{एवमुक्तो महाबाहुः केशवः सव्यसाचिना}
{अचोदयद्धयांस्तत्र यत्र दुर्मर्षणः स्थितः}


\twolineshloka
{स सम्प्रहारस्तुमुलः सम्प्रवृत्तः सुदारुणः}
{एकस्य च बहूनां च रथनागनरक्षयः}


\twolineshloka
{ततः सायकवर्षेण पर्जन्य इव वृष्टिमान्}
{परानवाकिरत्पार्थः पर्वतानिव नीरदः}


\twolineshloka
{ते चापि रथिनः सर्वे त्वरिताः कृतहस्तवत्}
{अवाकिरन्बाणजालैस्तत्र कृष्णधनञ्जयौ}


\twolineshloka
{ततः क्रुद्धो महाबाहुर्वार्यमाणः परैर्युधि}
{शिरांसि रथिनां पार्थः कायेभ्योऽपाहरच्छरैः}


\twolineshloka
{उद्धान्तनयनैर्वक्रैः सन्दष्टौष्ठपुटैः शुभैः}
{सकुण्डलशिरस्त्राणैर्वसुधा समकीर्यत}


\twolineshloka
{पुण्डरीकवनानीव विध्वस्तानि समन्ततः}
{विनिकीर्णानि योधानां वदनानि चकाशिरे}


\twolineshloka
{सुवर्णचित्राभरणाः संसिक्ता रुधिरेण च}
{संसक्ता इव दृश्यन्ते मेघसङ्घाः सविद्युतः}


\twolineshloka
{शिरसां पततां राजञ्शब्दोऽभूद्वसुधातले}
{कालेन परिपक्वानां तालानां पततामिव}


\threelineshloka
{कबन्ध उत्थितः कश्चिद्विष्फार्य सशरं धनुः}
{किञ्चित्खङ्गं विनिष्कृष्य भुजेनोद्यम्य तिष्ठति}
{`गृहीत्वाऽन्यस्य केशेषु शिरो नृत्यति चापरः'}


\twolineshloka
{पतितानि न जानन्ति शिरांसि पुरुषर्षभाः}
{अमृष्यमाणाः सङ्ग्रामे कौन्तेयं जयगृद्धिनः}


\twolineshloka
{हयानामुत्तमाङ्गैश्च हस्तिहस्तैश्च मेदिनी}
{बाहुभिश्च शिरोभिश्च वीराणां समकीर्यत}


\threelineshloka
{अयं पार्थः कुतः पार्थः पार्थोऽयमिति सर्वशः}
{`तिष्ठ पार्थेहि मां पार्थ क्व यासीति च जल्पताम्'}
{तव सैन्येषु योधानां पार्थभूतमिवाभवत्}


\twolineshloka
{अन्योन्यमपि चाजघ्नुरात्मानमपि चापरे}
{पार्थभूतममन्यन्त जनाः कालेन मोहिताः}


\twolineshloka
{निष्टनन्तः सरुधिरा विसंज्ञा गाढवेदनाः}
{शयाना बहवो वीराः कीर्तयन्तः स्वबान्धवान्}


\twolineshloka
{सभिण्डिपालाः सप्रासाः सशक्त्यृष्टिपरश्वथाः}
{सनिर्व्यूहाः सनिस्त्रिंशाः सशरासनतोमराः}


\twolineshloka
{सबाणवर्माभरणाः सगदाः साङ्गदा रणे}
{महाभुजगसङ्काशा बाहवः परिघोपमाः}


\twolineshloka
{उद्वेष्टन्ति विचेष्टन्ति सञ्चेष्टन्ति च सर्वशः}
{वेगं कुर्वन्ति संरब्धा निकृत्ताः परमेषुभिः}


\twolineshloka
{यो यः स्म समरे पार्थं प्रतिसञ्चरते नरः}
{तस्यतस्यान्तको बाणः शरीरमुपसर्पति}


\twolineshloka
{नृत्यतो रथमार्गेषु धनुर्व्यायच्छतस्तथा}
{न कश्चित्तत्र पार्थश्च ददृशेऽन्तरमण्वपि}


\twolineshloka
{यत्तस्य घटमानस्य क्षिप्रं विक्षिपतः शरान्}
{लाघवात्पाण्डुपुत्रस्य व्यस्मयन्त परे जनाः}


\twolineshloka
{हस्तिनं हस्तियन्तारमश्वमाश्विकमेव च}
{अभिनत्फल्गुनो बाणै रथिनं च ससारथिम्}


\twolineshloka
{आवर्तमानमावृत्तं युध्यमानं च पाण्डवः}
{प्रमुखे तिष्ठमानं च न कञ्चिन्न निहन्ति सः}


\twolineshloka
{यथोदयन्वै गगने सूर्यो हन्ति महत्तमः}
{तथाऽर्जुनो गजानीकमवधीत्कङ्कपत्रिभिः}


\twolineshloka
{हस्तिभिः पतितैर्भिन्नैस्तव सैन्यमदृश्यत}
{अन्तकाले यथा भूमिर्व्यवकीर्णा महीधरैः}


\twolineshloka
{यथा मध्यंदिने सूर्यो दुष्प्रेक्ष्यः प्राणिभिः सदा}
{तथा धनञ्जयः क्रुद्धो दुष्प्रेक्ष्यो युधि शत्रुभिः}


\twolineshloka
{तत्तथा तव पुत्रस्य सैन्यं युधि परन्तप}
{प्रभग्नं द्रुतमाविग्नमतीव शरपीडितम्}


\twolineshloka
{मारुतेनेव महता मेघानीकं व्यदीर्यत}
{प्रकाल्यमानं तत्सैन्यं नाशकत्प्रतिवीक्षितुम्}


\twolineshloka
{प्रतोदैश्चापकोटीभिर्हुंकारैः साधुवाहितैः}
{कशापार्ष्ण्यभिघातैश्च वाग्भिरुग्राभिरेव च}


\twolineshloka
{चोदयन्तो हयांस्तूर्णं पलायन्ते स्म तावकाः}
{सादिनो रथिनश्चैव पत्तयश्चार्जुनार्दिताः}


\twolineshloka
{पार्ष्ण्यङ्गुष्ठाङ्कुशैर्नागं चोदयन्तस्तथाऽपरे}
{शरैः सम्मोहिताश्चान्ये तमेवाभिमुखा ययुः}


% Check verse!
तव योघ हतोत्साहा विभ्रान्तमनसस्तदा
\chapter{अध्यायः ९०}
\twolineshloka
{धृतराष्ट्र उवाच}
{}


\twolineshloka
{तस्मिन्प्रभग्ने सैन्याग्रे वध्यमाने किरीटिना}
{केतु तत्र रणे वीराः प्रत्युदीयुर्धनञ्जयम्}


\twolineshloka
{आहोस्विच्छकटव्यूहं प्रविष्टा मोघनिश्चयाः ॥द्रोणमाश्रित्य तिष्ठन्ति प्राकारमकुतोभयाः ॥सञ्जय उवाच}
{}


\twolineshloka
{तथाऽर्जुनेन सम्भग्ने तस्मिंस्तव बलेऽनघ}
{हतवीरे हतोत्साहे पलायनकृतक्षणे}


\twolineshloka
{पाकशासनिनाऽभीक्ष्णं वध्यमाने शरोत्तमैः}
{न तत्र कश्चित्सङ्ग्रामे शशाकार्जुनमीक्षितुम्}


\twolineshloka
{ततस्तव सुतो राजन्दृष्ट्वा सैन्यं तथागतम्}
{दुःशासनो भृशं क्रुद्धो युद्धायार्जुनमभ्यगात्}


\twolineshloka
{सकाञ्चनविचित्रेण कवचेन समावृतः}
{जाम्बूनदशिरस्त्राणः शूरस्तीव्रपराक्रमः}


\twolineshloka
{नागानीकेन महता ग्रसन्निव महीमिमाम्}
{दुःशासनो महाराज सव्यसाचिनमावृणोत्}


\twolineshloka
{हादेन गजघण्टानां शङ्खानां निनदेन च}
{ज्याक्षेपनिनदैश्चैव विरावेण च दन्तिनाम्}


\threelineshloka
{भूर्दिशश्चान्तरिक्षं च शब्देनासीत्समावृतम्}
{`युवराजो बलश्लाघी पिङ्गलः प्रियदर्शनः'}
{स मुहूर्तं प्रतिभयो दारुणः समपद्यत}


\twolineshloka
{तान्दृष्ट्वा पततस्तूर्णमङ्कुशैरभिचोदितान्}
{व्यालम्बहस्तान्संरब्धान्सपक्षानिव पर्वतान्}


\threelineshloka
{सिंहनादेन महता नरसिंहो धनञ्जयः}
{`तदाचलघनप्रख्यं पताकाशतसङ्कुलम्'}
{गजानीकममित्राणामभितो व्यधमच्छरैः}


\twolineshloka
{महोर्मिणमिवोद्धूतं श्वसनेन महार्णवम्}
{किरीटी तद्गजानीकं प्राविशन्मकरो यथा}


\twolineshloka
{खमाश्रित इवादित्यः प्रतपन्स युगक्षये}
{ददृशे दिक्षु सर्वासु पार्थः परपुरञ्जनः}


\twolineshloka
{खुरशब्देन चाश्वानां नेमिघोषेण तेन च}
{तेन चोत्क्रुष्टशब्देन ज्यानिनादेन तेन च}


\threelineshloka
{नानावादित्रशब्देन पाञ्चजन्यस्वनेन च}
{देवदत्तस्य घोषेण गाण्डीवनिनदेन च}
{मन्दवेगा नरा नागा बभूवुस्ते विचेतसः}


\twolineshloka
{शरैराशीविषस्पर्शैर्निर्भिन्नाः सव्यसाचिना}
{ते गजाः समसीदन्त मग्नाः पङ्कार्णवेष्विव}


\threelineshloka
{युगपच्च समाविष्टैः शरैर्गाण्डीवधन्वनः}
{अनेकशतसाहस्रैर्द्रुमा मधुकरैरिव}
{`पुङ्खावशिष्टैर्बहुभिः शोणितोत्पीडवाहिभिः'}


\twolineshloka
{आरावं परमं कृत्वा वध्यमानाः किरीटिना}
{निपेतुरनिशं भूमौ छिन्नपक्षा इवाद्रयः}


\twolineshloka
{अपरे दन्तवेष्टेषु कुम्भेषु च कटेषु च}
{शरैः समर्पिता नागाः क्रौञ्चवद्व्यनदन्मुहुः}


\twolineshloka
{गजस्कन्धगतानां च पुरुषाणां किरीटिना}
{छिद्यन्ते चोत्तमाङ्गानि भुल्लैः सन्नतपर्वभिः}


\twolineshloka
{सकुण्डलानां पततां शिरसां धरणीतले}
{पद्मानामिव सङ्घातैः पार्थश्चक्रे निवेदनम्}


\twolineshloka
{यन्त्रबद्धा विकवचा व्रणार्ता रुधिरोक्षिताः}
{भ्रमत्सु युधि नागेषु मनुष्या विललम्बिरे}


\twolineshloka
{केचिदेकेन बाणेन सुयुक्तेन सुपत्रिणा}
{द्वौ त्रयश्च विनिर्भिन्ना निपेतुर्धरणीतले}


\twolineshloka
{अपरे मदसंरब्धा मातङ्गाः पर्वतोपमाः}
{सारोहा न्यपतन्भूमौ द्रुमवन्त इवाचलाः}


\twolineshloka
{मौर्वी ध्वजं धनुश्चैव युगमीषां तथैव च}
{रथिनां कुट्टयामास भल्लैः सन्नतपर्वभिः}


\twolineshloka
{न सन्दधन्न चाकर्षन्न विमुञ्चन्न चोद्वहन्}
{मण्डलेनैव धनुषा नृत्यन्पार्थः स्म दृश्यते}


\twolineshloka
{अतिविद्धाश्च नाराचैर्वमन्तो रुधिरं मुखैः}
{मुहूर्तान्न्यपतन्नन्ये वारणा वसुधातले}


\twolineshloka
{उत्थितान्यगणेयानि कबन्धानि समन्ततः}
{अदृश्यन्त महाराज तस्मिन्परमसङ्कुले}


\twolineshloka
{सचापाः साङ्गुलित्राणाः सखङ्गाः साङ्गदा रणे}
{अदृश्यन्त भुजाश्छिन्ना हेमाभरणभूषिताः}


\twolineshloka
{सूपस्करैरधिष्ठानैरीषादण्डकबन्धुरैः}
{चक्रैर्विमथितैरक्षैर्भग्नैश्च बहुधा युगैः}


\twolineshloka
{चर्मचापधरैश्चैव व्यवकीर्णैस्ततस्ततः}
{स्रग्भिराभरणैर्वस्त्रैः पतितैश्च महाध्वजैः}


\twolineshloka
{निहतैर्वारणैरश्वैः क्षत्रियैश्च निपातितैः}
{अदृश्यत मही तत्र दारुणप्रतिदर्शना}


\twolineshloka
{एवं दुःशासनबलं वध्यमानं किरीटिना}
{सम्प्राद्रवन्महाराज व्यथितं सहनायकम्}


\twolineshloka
{`एवं बले द्रुते याते राजपुत्रं महारथम्}
{विव्याध दशभिर्बाणैस्तिष्ठतिष्ठेति चाब्रवीत्}


\twolineshloka
{जीवितेन कथं गन्ता दुरुक्तं यावदद्य ते}
{तद्वाक्यसदृशं कर्म कुरु त्वं यदि मन्यसे}


\twolineshloka
{एवमुक्त्वा ततो राजा पार्थः पार्थिवमर्दनः}
{भृशं क्रुद्धो महाराज अविध्यत्तनयं तव'}


\twolineshloka
{ततो दुःशासनस्त्रस्तः सहानीकः शरार्दितः}
{द्रोणं त्रातारमाकाङ्क्षञ्शकटव्यूहमभ्यगात्}


\chapter{अध्यायः ९१}
\twolineshloka
{सञ्जय उवाच}
{}


\twolineshloka
{दुःशसनबलं हत्वा सव्यसाची महारथः}
{सिन्धुराजं परीप्सन्वै द्रोणानीकमुपाद्रवत्}


\twolineshloka
{स तु द्रोणं समासाद्य व्यूहस्य प्रमुखे स्थितम्}
{कृताञ्जलिरिदं वाक्यं कृष्णस्यानुमतेऽब्रवीत्}


\twolineshloka
{शिवेन ध्याहि मां ब्रह्मन्स्वस्ति चैव वदस्व मे}
{भवत्प्रसादादिच्छामि प्रवेष्टं दुर्भिदां चमूम्}


\twolineshloka
{भवान्पितृसमो मह्यं धर्मराजसमोऽपि च}
{धौम्यकृष्णसमश्चैव सत्यमेतद्ब्रवीमि ते}


\twolineshloka
{अश्वत्थामा यथा तात रक्षणीयस्त्वयाऽनघ}
{तथाऽहमपि ते रक्ष्यः सदैव द्विजसत्तम}


\threelineshloka
{तव प्रसादादिच्छेयं सिन्धुराजानमाहवे}
{निहन्तुं द्विपदां श्रेष्ठ प्रतिज्ञां रक्ष मे प्रभो ॥सञ्जय उवाच}
{}


\twolineshloka
{एवमुक्तस्तदाऽऽचार्यः प्रत्युवाच स्मयन्निव}
{मामजित्वा न बीभत्सो शक्यो जेतुं जयद्रथः}


\twolineshloka
{एतावदुक्त्वा तं द्रोणः शरव्रातैरवाकिरत्}
{सरथाश्वध्वजं तीक्ष्णैः प्रहसन्वै ससारथिम्}


\twolineshloka
{ततोऽर्जुनः शरव्रातान्द्रोणस्यावार्य सायकैः}
{द्रोणमभ्यर्दयद्बाणैर्घोररूपैर्हत्तरैः}


\twolineshloka
{विव्याध च रणे द्रोणमनुमान्य विशाम्पते}
{क्षत्रधर्मं समास्थाय नवभिः सायकैः पुनः}


\twolineshloka
{तस्येषूनिषुभिश्छित्त्वा द्रोणो विव्याध तावुभौ}
{विषाग्निज्वलितप्रख्यैरिषुभिः कृष्णपाण्डवौ}


\twolineshloka
{इयेप पाण्डवस्तस्य बाणैश्छेत्तुं शरासनम्}
{तस्य चिन्तयतस्त्वेव फल्गुनस्य महात्मनः}


\twolineshloka
{द्रोणः शरैरसम्भ्रान्तो ज्यां चिच्छेदाशु वीर्यवान्}
{विव्याध च हयानस्य ध्वजं सारथिमेव च}


\twolineshloka
{अर्जुनं च शरैर्वीरः स्मयमानोऽभ्यवाकिरत्}
{एतस्मिन्नन्तरे पार्थः सज्यं कृत्वा महद्धनुः}


\twolineshloka
{विशेषयिष्यन्नाचार्यं सर्वास्त्रविदुषां वरः}
{मुमोच षट््शतान्बाणान्गृहीत्वैकमिव द्रुतम्}


\twolineshloka
{पुनः सप्तशतानन्यान्सहस्रं चानिवर्तिनः}
{चिक्षेपायुतशश्चान्यांस्तेऽघ्नन्द्रोणस्य तां चमूम्}


\twolineshloka
{तैः सम्यगस्तैर्बलिना कृतिना चित्रयोधिना}
{मनुष्यवाजिमातङ्गा विद्धाः पेतुर्गतासवः}


\twolineshloka
{विसूताश्वध्वजाः पेतुः सञ्छिन्नायुधजीविताः}
{रथिनो रथमुख्येभ्यः सहसा शरपीडिताः}


\twolineshloka
{चूर्णिताक्षिप्तदग्धानां वज्रानिलहुताशनैः}
{तुल्यरूपा गजाः पेतुर्गिर्यग्राम्बुदवेश्मनाम्}


\twolineshloka
{पेतुरश्वसहस्राणि प्रहतान्यर्जुनेषुभिः}
{हंसा हिमवतः पृष्ठे वारिविप्रहता इव}


\twolineshloka
{रथाश्वद्विपपत्त्योघाः सलिलौघा इवाद्भुताः}
{युगान्तादित्यरश्म्याभैः पाण्डवास्त्रशरैर्हताः}


\twolineshloka
{तं पाण्डवादित्यशरांशुजालंकुरुप्रवीरान्युधि निष्टपन्तम्}
{स द्रोणमेघः शरवृष्टिवेगैःप्राच्छादयन्मेघ इवार्करश्मीन्}


\twolineshloka
{अथात्यर्थं विसृष्टेन द्विषतामसुभोजिना}
{आजघ्ने वक्षसि द्रोणो नाराचेन धनञ्जयम्}


\twolineshloka
{स विह्वलितसर्वाङ्गः क्षितिकम्पे यथाऽचलः}
{धैर्यमालम्ब्य बीभत्सुर्द्रोणं विव्याध पत्रिभिः}


\twolineshloka
{द्रोणस्तु पञ्चभिर्बाणैर्वासुदेवमताडयत्}
{अर्जुनं च त्रिसप्तत्या ध्वजं चास्य त्रिभिः शरैः}


\twolineshloka
{विशेषयिष्यञ्शिष्यं च द्रोणो राजन्पराक्रमी}
{अदृश्यमर्जुनं चक्रे निमेषाच्छरवृष्टिभिः}


\twolineshloka
{प्रसक्तान्पततोऽद्राक्ष्म भारद्वाजस्य सायकान्}
{मण्डलीकृतमेवास्य धनुश्चादृश्यताद्भुतम्}


\twolineshloka
{तेऽभ्ययुः समरे राजन्वासुदेवधनञ्जयौ}
{द्रोणसृष्टाः सुबहवः कङ्कपत्रपरिच्छदाः}


\twolineshloka
{तत्राद्भुतमपश्याम शिलानामिव सर्पणम्}
{यद्द्रोणं तरसा पार्थो वृद्धं बालोऽपि नातरत्}


\twolineshloka
{चिन्तयामास वार्ष्णेयो दृष्ट्वा द्रोणस्य विक्रमम्}
{नातिवर्तिष्यते ह्योनं वेलामिव महार्णवः}


\threelineshloka
{ततः पार्थं समुद्विग्नं लक्ष्य चिन्तयतेऽच्युतः}
{द्रोणस्य चापि विक्रान्तं दृष्ट्वा मधुनिषूदनः}
{इत्यब्रवीद्वासुदेवो धनञ्जयमुदारधीः}


\threelineshloka
{पार्थपार्थ महाबाहो न नः कालात्ययो भवेत्}
{द्रोणमुत्सृज्य गच्छामो ब्राह्मणोऽसौ गतक्लमः ॥सञ्जय उवाच}
{}


% Check verse!
पार्थश्चाप्यब्रवीत्कृष्णं यथेष्टमिति केशवम्
\twolineshloka
{ततः प्रदक्षिणं कृत्वा द्रोणं प्रायान्महाभुजम्}
{परिवृत्तश्च बीभत्सुरगच्छद्विसृजञ्शरान्}


\threelineshloka
{ततोऽब्रवीत्स्वयं द्रोणः क्वेदं पाण्डव गम्यते}
{ननु नाम रणे शत्रुमजित्वा न निवर्तसे ॥अर्जुन उवाच}
{}


\threelineshloka
{गुरुर्भवान्न मे शत्रुः शिष्यः पुत्रसमोऽस्मि ते}
{न चास्ति स पुमाँल्लोके यस्त्वां युधि पराजयेत् ॥सञ्जय उवाच}
{}


\twolineshloka
{एवं ब्रुवाणो बीभत्सुर्जयद्रथवधोत्सुकः}
{त्वरायुक्तो महाबाहुस्त्वत्सैन्यं समुपाद्रवत्}


\twolineshloka
{तं चक्ररक्षौ पाञ्चाल्यौ युधामन्यूत्तमौजसौ}
{अन्वयातां महात्मानौ विशन्तं तावकं बलम्}


\twolineshloka
{ततो जयो महाराज कृतवर्मा च सात्वतः}
{काम्भोजश्च श्रुतायुश्च धनञ्जयमवारयन्}


\twolineshloka
{तेषां दशसहस्राणि रथानामनुयायिनाम्}
{अभीषाहाः शूरसेनाः शिबयोऽथ वसातयः}


\twolineshloka
{मावेल्लका ललित्थाश्च केकया मद्रकास्तथा}
{नारायणाश्च गोपालाः काम्भोजानां च ये गणाः}


\twolineshloka
{कर्णेन विजिताः पूर्वं सङ्ग्रामे शूरसम्मताः}
{भारद्वाजं पुरस्कृत्य हृष्टात्मानोऽर्जुनं प्रति}


\twolineshloka
{पुत्रशोकाभिसन्तप्तं क्रुद्धं मृत्युमिवान्तकम्}
{त्यजन्तं तुमुले प्राणान्सन्नद्धं चित्रयोधिनम्}


\twolineshloka
{गाहमानमनीकानि मातङ्गमिव यूथपम्}
{महेष्वासं पराक्रान्तं नरव्याघ्रमवारयन्}


\twolineshloka
{ततः प्रववृते युद्धं तुमुलं रोमहर्षणम्}
{अन्योन्यं वै प्रार्थयतां योधानामर्जुनस्य च}


\twolineshloka
{जयद्रथवधप्रेप्सुमायान्तं पुरुषर्षभम्}
{न्यवारयन्त सहिताः क्रिया व्याधिमिवोत्थितं}


\chapter{अध्यायः ९२}
\twolineshloka
{सञ्जय उवाच}
{}


\twolineshloka
{सन्निरुद्धस्तु तैः पार्थो महाबलपराक्रमः}
{द्रुतं समनुयातश्च द्रोणेन रथिनां वरः}


\twolineshloka
{किरन्निषुगणांस्तीक्ष्णान्स रश्मीनिव भास्करः}
{तापयामास तत्सैन्यं देहं व्याधिगणो यथा}


\twolineshloka
{अश्वो विद्धो रथश्छिन्नः सारोहः पातितो गजः}
{छत्राणि चापविद्धानि रथाश्चक्रैर्विना कृताः}


\twolineshloka
{विद्रुतानि च सैन्यानि शरार्तानि समन्ततः}
{इत्यासीत्तुमुलं युद्धं न प्राज्ञायत किञ्चन}


\twolineshloka
{तेषां संयच्छतां सङ्ख्ये परस्परमजिह्मगैः}
{अर्जुनो ध्वजिनीं राजन्नभीक्ष्णं समकम्पयत्}


\twolineshloka
{सत्यां चिकीर्षमाणस्तु प्रतिज्ञां सत्यसङ्गरः}
{अभ्यद्रवद्रथश्रेष्ठं शोणाश्वं श्वेतवाहनः}


\twolineshloka
{तं द्रोणः पञ्चविंशत्या मर्मभिद्भिरजिह्मगैः}
{अन्तेवासिनमाचार्यो महेष्वासं समार्पयत्}


\twolineshloka
{तं तूर्णमिव बीभत्सुः सर्वशस्त्रभृतां वरः}
{अभ्यधावदिषूनस्यन्निषुवेगविघातकान्}


\twolineshloka
{तस्याशु क्षिपतो भल्लान्भल्लैः सन्नतपर्वभिः}
{प्रत्यविध्यदमेयात्मा ब्रह्मास्त्रं समुदीरयन्}


\twolineshloka
{तदद्भुतमपश्याम द्रोणस्याचार्यकं युधि}
{यतमानो युवा नैनं प्रत्यविध्यद्यदर्जुनः}


\twolineshloka
{क्षरन्निव महामेघो वारिधाराः सहस्रशः}
{द्रोणमेघः पार्थशैलं ववर्ष शरवृष्टिभिः}


\twolineshloka
{अर्जुनः शरवर्षं तच्छरवर्षेण वीर्यवान्}
{अवारयदसम्भ्रान्तो न त्वाचार्यमपीडयत्}


\twolineshloka
{द्रोणस्तु पञ्चविंशत्या श्वेतवाहनमार्दयत्}
{वासुदेवं च सप्तत्या बाह्वोरुरसि चाशुगैः}


\twolineshloka
{पार्थस्तु प्रहसन्धीमानाचार्यं स शरौघिणम्}
{विसृजन्तं शितान्बाणानवारयत तं युधि}


\twolineshloka
{अथ तौ वध्यमानौ तु द्रोणेन रथसत्तमौ}
{अवर्जयेतां दुर्धर्षं युगान्ताग्निमिवोत्थितम्}


\twolineshloka
{वर्जयन्निशितान्बाणान्द्रोणचापविनिः सृतान्}
{किरीटमाली कौन्तेयो भोजानीकमथाविशत्}


\twolineshloka
{सोऽन्तरा कृतवर्माणं काम्भोजं च सुदक्षिणम्}
{अभ्ययाद्वर्जयन्द्रोणं मैनाकमिव पर्वतम्}


\twolineshloka
{ततो भोजो नरव्याघ्रो दुर्धर्षं कुरुसत्तमम्}
{अविध्यत्तूर्णमव्यग्रो दशभिः कङ्कपत्रिभिः}


\twolineshloka
{तमर्जुनः शतेनाजौ राजन्विव्याध पत्रिणाम्}
{पुनश्चान्यैस्त्रिभिर्बाणैर्मोहयन्निव सात्वतम्}


\twolineshloka
{भोजस्तु प्रहसन्पार्थं वासुदेवं च माधवम्}
{एकैकं पञ्चविंशत्या सायकानां समार्पयत्}


\twolineshloka
{तस्यार्जुनो धनुश्छित्त्वा वियाधैनं त्रिसप्तभिः}
{शरैरग्निशिखाकारैः क्रुद्धाशीविषसन्निभैः}


\twolineshloka
{अथान्यद्धनुरादाय कृतवर्मा महारथः}
{पञ्चभिः सायकैस्तूर्णं विव्याधोरसि भारत}


\twolineshloka
{पुनश्च निशितैर्बाणैः पार्थं विव्याध पञ्चभिः}
{तं पार्थो नवभिर्बाणैराजघान स्तनान्तरे}


\twolineshloka
{दृष्ट्वा विषक्तं कौन्तेयं कृतवर्मरथं प्रति}
{चिन्तयामास वार्ष्णेयो न नः कालात्ययो भवेत्}


\twolineshloka
{ततः कृष्णोऽब्रवीत्पार्थं कृतवर्मणि मा दयाम्}
{कुरु सम्बन्धकं हित्वा प्रमथ्यैनं विशातय}


\twolineshloka
{ततः स कृतवर्माणं महोयित्वाऽर्जुनः शरैः}
{अभ्यगाज्जवनैरश्वैः काम्भोजानामनीकिनीम्}


\twolineshloka
{अमर्षितस्तु हार्दिक्यः प्रविष्टे श्वेतवाहने}
{विधुन्वन्सशरं चापं पाञ्चाल्याभ्यां समागतः}


\twolineshloka
{चक्ररक्षौ तु पाञ्चाल्यावर्जुनस्य पदानुगौ}
{पर्यवारयदायान्तौ कृतवर्मा रथेषुभिः}


\twolineshloka
{तावविध्यत्ततो भोजः कृतवर्मा शितैः शरैः}
{त्रिभिरेव युधामन्युं चतुर्भिश्चोत्तमौजसम्}


\threelineshloka
{तावप्येनं विविधतुर्दशभिर्दशभिः शरैः}
{त्रिभिरेव युधामन्युरुत्तगौजास्त्रिभिस्तथा}
{सञ्चिच्छिदतुरप्यस्य ध्वजं कार्मुकमेव च}


\twolineshloka
{अथान्यद्धनुरादाय हार्दिक्यः क्रोधमूर्च्छितः}
{कृत्वा विधनुषौ वीरौ शरवर्षैरवाकिरत्}


\twolineshloka
{तावन्ये धनुषी सज्ये कृत्वा भोजं विजघ्नतुः}
{तेनान्तरेषु बीभत्सुर्विवेशामित्रवाहिनीम्}


\twolineshloka
{न लेभाते तु तौ द्वारं वारितौ कृतवर्मणा}
{धार्तराष्ट्रेष्वनीकेषु यतमानौ नरर्षभौ}


\twolineshloka
{अनीकान्यर्दयन्युद्धे त्वरितः श्वेतवाहनः}
{नावधीत्कृतवर्माणं प्राप्तमप्यरिसूदनः}


\twolineshloka
{तं दृष्ट्वा तु तथाऽऽयान्तंशूरो राजा श्रुतायुधः}
{अभ्यद्रवत्सुसङ्क्रुद्धो विधुन्वानो महद्धनुः}


\twolineshloka
{स पार्थं त्रिभिरानर्च्छत्सप्तत्या च जनार्दनम्}
{क्षुरप्रेण सुतीक्ष्णेन पार्थकेतुमताडयत्}


\twolineshloka
{ततोऽर्जुनो नवत्या तु शराणां नतपर्वणाम्}
{आजघान भृशं क्रुद्धस्तोत्रैरिव महाद्विपम्}


\twolineshloka
{स तन्न ममृषे राजन्पाण्डवेयस्य विक्रमम्}
{अथैनं सप्तसप्तत्या नाराचानां समार्पयत्}


\twolineshloka
{तस्यार्जुनो धनुश्छित्त्वा शरावापं निकृत्य च}
{आजघानोरसि क्रुद्धः सप्तभिर्नतपर्वभिः}


\twolineshloka
{अथान्यद्धनुरादाय स राजा क्रोधमूर्च्छितः}
{वासविं नवभिर्बाणैर्बाह्वोरुरसि चार्पयत्}


\twolineshloka
{ततोऽर्जुनः स्मयन्नेव श्रुतायुधमरिन्दमः}
{शरैरनेकसाहस्रैः पीडयामास भारत}


\twolineshloka
{अश्वांश्चास्यावधीत्तूर्णं सारथिं च महारथः}
{विव्याध चैनं सप्तत्या नाराचानां महाबलः}


\twolineshloka
{हताश्वं रथमुत्सृज्य स तु राजा श्रुतायुधः}
{अभ्यद्रुवद्रुणे पार्थं गदामुद्यम्य वीर्यवान्}


\twolineshloka
{वरुणस्यत्मजो वीरः स तु राजा श्रुतायुधः}
{पर्णाशाजननी यस्य शीततोया महानदी}


\twolineshloka
{तस्य माताऽब्रवीद्राजन्वरुणं पुत्रकारणात्}
{अवध्योऽयं भवेल्लोके शत्रूणां तनयो मम}


\twolineshloka
{वरुणस्त्वब्रवीत्प्रीतो ददाम्यस्मै वरं हितम्}
{दिव्यमस्त्रं सुतस्तेऽयं येनावध्यो भविष्यति}


\twolineshloka
{नास्ति चाप्यमरत्वं वै मनुष्यस्य कथञ्चन}
{सर्वेणावश्यमर्तव्यं जातेन सरितां वरे}


\twolineshloka
{दुर्धर्षस्त्वेष शत्रूणां रणेषु भविता सदा}
{अस्त्रस्यास्य प्रभावाद्वै व्येतु ते मानसो ज्वरः}


\twolineshloka
{इत्युक्त्वा वरुणः प्रादाद्गदां मन्त्रपुरस्कृताम्}
{यामासाद्य दुराधर्षः सर्वलोके श्रुतायुधः}


\twolineshloka
{उवाच चैनं भगवान्पुनरेव जलेश्वरः}
{अयुध्यति न मोक्तव्या सा त्वय्येव पतेदिति}


\twolineshloka
{हन्यादेषा प्रतीपं हि प्रयोक्तारमपि प्रभो}
{न चाकरोत्स तद्वाक्यं प्राप्ते काले श्रुतायुधः}


\threelineshloka
{स तया वीरघातिन्या जनार्दनमताडयत्}
{प्रतिजग्राह तां कृष्णः पीनेनांसेन वीर्यवान्}
{नाकम्पयत शौरिं सा विन्ध्यं गिरिमिवानिलः}


\twolineshloka
{`ततोऽर्जुनः क्षुरप्राभ्यां भुजौ परिघसन्निभौ}
{चिच्छेद पाण़्डवः शीघ्रं जलेश्वरसुतस्य वै}


\twolineshloka
{स ज्वलन्ती महोल्केव समासाद्य जनार्दनम्'}
{प्रत्यागता महावेगा कृत्येव दुरधिष्ठिता}


\threelineshloka
{जघान चास्थितं वीरं श्रुतायुधममर्षणम्}
{`स पपात हतो भूमौ विशिरा विभुजो बली}
{सम्भग्न इव वातेन बहुशाखो वनस्पतिः}


\twolineshloka
{सा विस्फुरन्ती ज्वलिता वज्रवेगसमा गदा'}
{हत्वा श्रुतायुधं वीरं धरणीमन्वपद्यत}


\threelineshloka
{गदां निवर्तितां दृष्ट्वा निहतं च श्रुतायुधम्}
{हाहाकारो महांस्तत्र सैन्यानां समजायत}
{स्वेनास्त्रेण हतं दृष्ट्वा श्रुतायुधमरिन्दमम्}


\twolineshloka
{अयुध्यमानाय ततः केशवाय नराधिप}
{क्षिप्ता श्रुतायुधेनाथ तस्मात्तमवधीद्गदा}


\twolineshloka
{यथोक्तं वरुणेनाजौ तथा स निधनं गतः}
{व्यसुश्चाप्यपतद्भूमौ प्रेक्षतां सर्वधन्विनाम्}


\twolineshloka
{पतमानस्तु स बभौ पर्णाशायाः प्रियः सुतः}
{सम्भग्न इव वातेन बहुशाखो वनस्पतिः}


\twolineshloka
{ततः सर्वाणि सैन्यानि सेनामुख्याश्च सर्वशः}
{प्राद्रवन्त हतं दृष्ट्वा श्रुतायुधमरिन्दमम्}


\twolineshloka
{ततः काम्भोजराजस्य पुत्रः शूरः सुदक्षिणः}
{अभ्ययाज्जवनैरश्वैः फल्गुनं शत्रुसूदनम्}


\twolineshloka
{तस्य पार्थः शरान्सप्त प्रेषयामास भारत}
{ते तं शूरं विनिर्भिद्य प्राविशन्धरणीतलम्}


\twolineshloka
{सोऽतिविद्धः शरैस्तीक्ष्णैर्गाण्डीवप्रेषितैर्मृधे}
{अर्जुनं प्रतिविव्याध दशभिः कङ्कपत्रिभिः}


\twolineshloka
{वासुदेवं त्रिभिर्विद्ध्वा पुनः पार्थं च पञ्चभिः}
{तस्य पार्थो धनुश्छित्त्वा केतुं चिच्छेद मारिष}


\twolineshloka
{भल्लाभ्यां भृशतीक्ष्णाभ्यां तं च विव्याध पाण्डवः}
{स तु पार्थं त्रिभिर्विद्ध्वा सिंहनादमथानदत्}


\twolineshloka
{सर्वपारशवीं चैव शक्तिं शूरः सुदक्षिणः}
{सघण्टां प्राहिणोद्धोरां क्रुद्धो गाण्डीवधन्वने}


\twolineshloka
{सा ज्वलन्ती महोल्केव तमासाद्य महारथम्}
{सविस्फुलिङ्गा निर्भिद्य निपपात महीतले}


\twolineshloka
{शक्या त्वभिहतो गाढं मूर्च्छयाऽभिपरिप्लुतः}
{समाश्वास्य महातेजाः सृक्विणी परिलेलिहन्}


\twolineshloka
{तं चतुर्दशभिः पार्थो नाराचैः कङ्कपत्रिभिः}
{साश्वध्वजधनुःसूतं विव्याधाचिन्त्यविक्रमः}


\threelineshloka
{रथं चान्यैः सुबहुभिश्चक्रे विशकलं शरैः}
{सुदक्षिणं तं काम्भोजं मोघसङ्कल्पविक्रमम्}
{बिभेद हृदि बाणेन पृथुधारेण पाण्डवः}


\twolineshloka
{स भिन्नवर्मा स्रस्ताङ्गः प्रभ्रष्टमुकुटाङ्गदः}
{पपाताभिमुखः शूरो यन्त्रमुक्त इव ध्वजः}


\threelineshloka
{गिरेः शिखरजः श्रीमान्सुशाखः सुप्रतिष्ठितः}
{निर्भग्न इव वातेन कर्णिकारो हिमात्यये}
{विशीर्णः पतितो राजा प्रसार्य विपुलौ भुजौ}


\twolineshloka
{शेते स्म निहतो भूमौ काम्भोजास्तरणोचितः}
{महार्हाभरणोपेतः सानुमानिव पर्वतः}


\twolineshloka
{सुदर्शनीयस्ताम्राक्षः कर्णिना स सुदक्षिणः}
{पुत्रः काम्भोजराजस्य पार्थेन विनिपातितः}


\twolineshloka
{धारयन्नग्निसङ्काशां शिरसा काञ्चनीं स्रजम्}
{अशोभत महाबाहुर्व्यसुर्भूमौ निपातितः}


\twolineshloka
{ततः सर्वाणि सैन्यानि व्यद्रवन्त सुतस्य ते}
{हतं श्रुतायुधं दृष्ट्वा काम्भोजं च सुदक्षिणम्}


\chapter{अध्यायः ९३}
\twolineshloka
{सञ्जय उवाच}
{}


\twolineshloka
{हते सुदक्षिणे राजन्वीरे चैव श्रुतायुधे}
{जवेनाभ्यद्रवन्पार्थं कुपिताः सैनिकास्तव}


\twolineshloka
{अभीषाहाः शूरसेनाः शिबयोऽथ वसातयः}
{अभ्यवर्षंस्ततो राजञ्शरवर्षैर्धनञ्जयम्}


\twolineshloka
{तेषां षष्टिशतानन्यान्प्रामथ्नात्पाण्डवः शरैः}
{ते स्म भीताः पलायन्ते व्याघ्रात्क्षुद्रमृगा इव}


\twolineshloka
{ते निवृत्ताः पुनः पार्थं सर्वतः पर्यवारयन्}
{रणे सपत्नान्निघ्नन्तं जिगीषन्तं परान्युधि}


\twolineshloka
{तेषामापततां तूर्णं गाण्डीवप्रेषितैः शरैः}
{शिरांसि पातयामास बाहूंश्चापि धनञ्जयः}


\twolineshloka
{शिरोभिः पातितैस्तत्र भूमिरासीन्निरन्तरा}
{अभ्रच्छायेव चैवासीद्ध्वाङ्क्षगृध्रबलैर्युधि}


\twolineshloka
{तेषु तूत्साद्यमानेषु क्रोधामर्षसमन्वितौ}
{स्रुतायुश्चाश्रुतायुश्च धनञ्जयमयुध्यताम्}


\twolineshloka
{बलिनौ स्पर्धिनौ वीरौ कुलजौ बाहुशालिनौ}
{तावेनं शरवर्षाणि सव्यदक्षिणमस्यताम्}


\twolineshloka
{त्वरायुक्तौ महराज प्रार्थयानौ महद्यशः}
{अर्जुनस्य वधप्रेप्सू पुत्रार्थे तव धन्विनौ}


\twolineshloka
{तावर्जुनं सहस्रेण पत्रिणां नतपर्वणाम्}
{पूरयामासतुः क्रुद्धौ तटाकं जलदौ यथा}


\twolineshloka
{श्रुतायुश्च ततः क्रुद्धस्तोमरेण धनञ्जयम्}
{आजघान रथश्रेष्ठः पीतेन निशितेन च}


\twolineshloka
{सोऽतिविद्धो बलवता शत्रुणा शत्रुकर्शनः}
{जगाम परमं मोहं मोहयन्केशवं रणे}


\twolineshloka
{एतस्मिन्नेव काले तु सोऽश्रुतायुर्महारथः}
{शूलेन भृशतीक्ष्णेन ताडयामास पाण्डवम्}


\twolineshloka
{क्षते क्षारं स हि ददौ पाण्डवस्य महात्मनः}
{पार्थोऽपि भृशसंविद्धो ध्वजयष्टिं समाश्रितः}


\twolineshloka
{ततः सर्वस्य सैन्यस्य तावकस्य विशाम्पते}
{सिंहनादो महानासीद्धतं मत्वा धनञ्जयम्}


\twolineshloka
{कृष्णश्च भृशसन्तप्तो दृष्ट्वा पार्थं विचेतनम्}
{आश्वासयत्सुहृद्याभिर्वाग्भिस्तत्र धनञ्जयम्}


\twolineshloka
{ततस्तौ रथिनां श्रेष्ठौ लब्धलक्षौ धनञ्जयम्}
{वासुदेवं च वार्ष्णेयं शरवर्षैः समन्ततः}


\twolineshloka
{सचक्रकूबररथं साश्वध्वजपताकिनम्}
{अदृश्यं चक्रतुर्युद्धे तदद्भुतमिवाभवत्}


\twolineshloka
{प्रत्याश्वस्तस्तु बीभत्सुः शनकैरिव भारत}
{प्रेतराजपुरं प्राप्य पुनः प्रत्यागतो यथा}


\twolineshloka
{सञ्छन्नं शरजालेन रथं दृष्ट्वा सकेशवम्}
{शत्रू चाभिमुखौ दृष्ट्वा दीप्यमानाविवानलौ}


\twolineshloka
{प्रादुश्चक्रे ततः पार्थः शाक्रमस्त्रं महारथः}
{तस्मादासन्सहस्राणि शराणां नतपर्वणाम्}


\twolineshloka
{ते जघ्नुस्तौ महेष्वासौ ताभ्यां मुक्तांश्च सायकान्}
{विचेरुराकाशगताः पार्थचापविनिःसृताः}


\twolineshloka
{प्रतिहत्य सरांस्तूर्णं शरवेगेन पाण्डवः}
{प्रतस्थे तत्रतत्रैव योधयन्वै महारथान्}


\twolineshloka
{तौ च फल्गुनबाणौघैर्विबाहुशिरसौ कृतौ}
{वसुधामन्वपद्येतां वातनुन्नाविव द्रुमौ}


\twolineshloka
{श्रुतायुषश्च निधनं वधश्चेवाश्रुतायुषः}
{लोकविस्मापनमभूत्समुद्रस्येव शोषणम्}


\twolineshloka
{तयोः पदानुगान्हत्वा पुनः पञ्चशतं रथान्}
{प्रत्यगाद्भारतीं सेनां निघ्नन्पार्थो वरान्वरान्}


\twolineshloka
{श्रुतायुषं च निहतं प्रेक्ष्य चैवाश्रुतायुषम्}
{नियतायुश्च सङ्क्रुद्धो दीर्घायुश्चैव भारत}


\twolineshloka
{पुत्रौ तयोर्नरश्रेष्ठौ कौन्तेयं प्रति जग्मतुः}
{किरन्तौ विविधान्बाणान्पितृव्यसनकर्शितौ}


\twolineshloka
{तावर्जुनो मुहूर्तेन शरैः सन्नतपर्वभिः}
{प्रैषयत्परमक्रुद्धो यमस्य सदनं प्रति}


\twolineshloka
{लोडयन्तमनीकानि द्विपं पद्मसरो यथा}
{नाशक्नुवन्वारयितुं पार्थं क्षत्रियपुङ्गवाः}


\twolineshloka
{अङ्गास्तु गजसङ्घैश्च पाण्डवं पर्यवारयन्}
{क्रुद्धाः सहस्रशो राजञ्शिक्षिता हस्तिसादिनः}


\twolineshloka
{दुर्योधनसमादिष्टाः कुञ्जरैः पर्वतोपमैः}
{प्राच्याश्च दाक्षिणात्याश्च कलिङ्गप्रमुखा नृपाः}


\twolineshloka
{तेषामापततां शीघ्रं गाण्डीवप्रेषितैः शरैः}
{निचकर्त शिरांस्युग्रो बाहूनपि सुभूषणान्}


\twolineshloka
{तैः शिरोभिर्मही कीर्णा बाहुभिश्च सहाङ्गदैः}
{बभौ कनकपाषाणा भुजगैरिव संवृता}


\twolineshloka
{बाहवो विशिखैश्छिन्नाः शिरांस्युन्मथितानि च}
{पतमानान्यदृश्यन्त द्रुमेभ्य इव पक्षिणः}


\twolineshloka
{शरैः सहस्रशो विद्धा द्विपाः प्रसृतशोणिताः}
{अदृश्यन्ताद्रयः काले गैरिकाम्बुस्रवा इव}


\twolineshloka
{निहताः शेरते स्मान्ये बीभत्सोर्निशितैः शरैः}
{गजपृष्ठगता म्लेच्छा नानाविकृतदर्शनाः}


\twolineshloka
{नानावेषधरा राजन्नानाशस्त्रौघसंवृताः}
{रुधिरेणानुलिप्ताङ्गा भान्ति चित्रैः शरैर्हताः}


\twolineshloka
{शोणितं निर्वमन्ति स्म द्विपाः पार्थशराहताः}
{सहस्रशश्छिन्नगात्राः सारोहाः सपदानुगाः}


\twolineshloka
{चुक्रुशुश्च निपेतुश्च बभ्रमुश्चापरे दिशः}
{भृशं त्रस्ताश्च बहवः स्वानेव ममृदुर्गजाः}


\twolineshloka
{सोत्तरायुधिनश्चैव द्विपास्तीक्ष्णविषोपमाः}
{विदन्त्यसुरमायां ये सुघोरा घोरचक्षुषः}


\twolineshloka
{यवनाः पारदाश्चैव शकाश्च सह बाह्लिकैः}
{काकवर्णा दुराचाराः स्त्रीलोलाः कलहप्रियाः}


\twolineshloka
{द्राविडास्तत्र युध्यन्ते मत्तमातङ्गविक्रमाः}
{गोयोनिप्रभवा म्लेच्छाः कालकल्पाः प्रहारिणः}


\twolineshloka
{दार्वातिसारा दरदाः पुण्ड्राश्चैव सहस्रशः}
{ते न शक्याः स्म सङ्ख्यातुं व्राताः शतसहस्रशः}


\twolineshloka
{अभ्यवर्षन्त ते सर्वे पाण्डवं निशितैः शरैः}
{अवाकिरंश्च ते म्लेच्छा नानायुद्धविशारदाः}


\twolineshloka
{तेषामपि ससर्जाशु शरवृष्टिं धनञ्जयः}
{सृष्टिस्तथाविधा ह्यासीच्छलभानामिवायतिः}


\threelineshloka
{अभ्रच्छायामिव शरैः सैन्ये कृत्वा धनञ्जयः}
{मुण्डार्धमुण्डाञ्जटिलानशुचीञ्जटिलाननान्}
{म्लेच्छानशातयत्सर्वान्समेतानस्त्रतेजसा}


\twolineshloka
{शरैश्च शतशो विद्धास्ते सङ्घा गिरिचारिणः}
{प्राद्रवन्त रणे भीता गिरिगह्वरवासिनः}


\twolineshloka
{गजाश्वसादिम्लेच्छानां पतितानां शितैः शरैः}
{बकाः कङ्का वृका भूमावपिबन्रुधिरं मुदा}


\threelineshloka
{पत्त्यश्वरथनागैश्च प्रच्छन्नकृतसङ्क्रमाम्}
{शरवर्षप्लवां घोरां केशशैवलशाद्वलाम्}
{प्रावर्तयन्नदीमुग्रां शोणितौघतरङ्गिणीम्}


\twolineshloka
{छिन्नाङ्गुलीक्षुद्रमत्स्यां युगान्ते कालसन्निभाम्}
{प्राकरोद्गजसम्बाधां नदीमुत्तरशोणिताम्}


\threelineshloka
{देहेभ्यो राजपुत्राणां नागाश्वरथसादिनाम्}
{यथा स्थलं च निम्नं च न स्याद्वर्षति वासवे}
{तथाऽऽसीत्पृथिवी सर्वा शोणितेन परिप्लुता}


\twolineshloka
{षट््सहस्रान्हयान्वीरान्पुनर्दशशतान्वरान्}
{प्राहिणोन्मृत्युलोकाय क्षत्रियान्क्षत्रियर्षभः}


\twolineshloka
{शरैः सहस्रशो विद्धा विधिवत्कल्पिता द्विपाः}
{शेरते भूमिमासाद्य शैला वज्रहता इव}


\twolineshloka
{स वाजिरथमातङ्गान्निघ्नन्व्यचरदर्जुनः}
{प्रभिन्न इव मातङ्गो मृद्गन्नलवनं यथा}


\twolineshloka
{भुरिद्रुमलतागुल्मं शुष्केन्धनतृणोलपम्}
{निर्दहेदनलोऽरण्यं यथा वायुसमीरितः}


\twolineshloka
{सेनारण्यं तव तथा कृष्णानिलसमीरितः}
{शरार्चिरदहत्क्रुद्धः पाण्डवोऽग्निर्धनञ्जयः}


\threelineshloka
{शून्यान्कुर्वन्रथोपस्थान्मानवैः संस्तरन्महीम्}
{प्रानृत्यदिव सम्बाधे चापहस्तो धनञ्जयः}
{वज्रकल्पैः शरैर्भूमिं कुर्वन्नुत्तरशोणिताम्}


\twolineshloka
{`ततः प्रावर्तत नदी शोणितौघतरङ्गिणी}
{नराश्वद्विपकायेभ्यः पर्वतेभ्य इवापगा}


\twolineshloka
{अस्थिशर्करसम्बाधा ध्वजवृक्षा रथहदा}
{सञ्छिन्नशीर्षपाणाणा हस्तिहस्तमहाग्रहा}


\twolineshloka
{मांसमज्जास्थिपङ्काढ्यां हताश्वमकराकुला}
{उष्णीषफेनसञ्छन्ना शरघोरझषाकुला}


\twolineshloka
{रुद्रस्याक्रीडसदृशीं भूमिं कुर्वन्विभीषणाम्'}
{प्राविशद्भारतीं सेनां सङ्क्रुद्धो वै धनञ्जयः}


% Check verse!
तं श्रुतायुस्तथाम्बष्ठो व्रजमानं न्यवारयत्
\threelineshloka
{तस्यार्जुनः शरैस्तीक्ष्णैः कङ्कपत्रपरिच्छदैः}
{न्यपातयद्धयाञ्शीघ्रं यतमानस्य मारिष}
{धनुश्चास्यापरैश्छित्त्वा शरैः पार्थो विचक्रमे}


\twolineshloka
{अम्बष्ठस्तु गदां गृह्य क्रोधपर्याकुलेक्षणः}
{आससाद रणे पार्थं केशवं च महारथम्}


\twolineshloka
{ततः सम्प्रहरन्वीरो गदामुद्यम्य भारत}
{रथमावार्य गदया केशवं समताडयत्}


\twolineshloka
{गदया ताडितं दृष्ट्वा केशवं परवीरहा}
{अर्जुनोऽथ भृशं क्रुद्धः सोऽम्बष्ठं प्रति भारत}


\twolineshloka
{ततः शरैर्हेमपुङ्खैः सगदं रथिनां वरम्}
{छादयामास समरे मेघः सूर्यमिवोदितम्}


\twolineshloka
{अथापरैः शरैश्चापि गदां तस्य महात्मनः}
{अचूर्णयत्तदा पार्थस्तिदद्भुतमिवाभवत्}


\twolineshloka
{अथ तां पतितां दृष्ट्वा गृह्यान्यां च महागदाम्}
{अर्जुनं वासुदेवं च पुनः पुनरताडयत्}


\twolineshloka
{तस्यार्जुनः क्षुरप्राभ्यां सगदावुद्यतौ भुजौ}
{चिच्छेदेन्द्रध्वजाकारौ शिरश्चान्येन पत्रिणा}


\twolineshloka
{स पपात हतो राजन्वसुधामनुनादयन्}
{इन्द्रध्वज इवोत्सृष्टो यन्त्रनिर्मुक्तबन्धनः}


% Check verse!
अम्बष्ठे तु तदा भग्ने तव सैन्यमभज्यत
\twolineshloka
{रथानीकावगाढश्च वारणाश्वशतैर्वृतः}
{अदृश्यत पदा पार्थो घनैः सूर्य इवावृतः}


\chapter{अध्यायः ९४}
\twolineshloka
{सञ्जय उवाच}
{}


\twolineshloka
{ततः प्रविष्टे कौन्तेये सिन्धुराजजिघांसया}
{द्रोणानीकं विनिर्भिद्य भोजानीकं च दुस्तरम्}


\twolineshloka
{काम्भोजस्य च दायादे हते राजन्सुदक्षिणे}
{श्रुतायुषि च विक्रान्ते निहते सव्यसाचिना}


\twolineshloka
{विप्रद्रुतेष्वनीकेषु विध्वस्तेषु समन्ततः}
{प्रभग्नं स्वबलं दृष्ट्वा पुत्रस्ते द्रोणमभ्ययात्}


\twolineshloka
{त्वरन्नेकरथेनैव समेत्य द्रोणमब्रवीत्}
{गतः स पुरुषव्याघ्रः प्रमथ्यैतां महाचमूम्}


\twolineshloka
{अथ बुद्ध्या समीक्षस्व किन्नु कार्यमनन्तरम्}
{अर्जुनस्य विघाताय दारुणेऽस्मिञ्जनक्षये}


\twolineshloka
{यथा स पुरुषव्याघ्रो न हन्येत जयद्रथः}
{तथा विधत्स्व भद्रं ते त्वं हि नः परमा गतिः}


\twolineshloka
{असौ धनञ्जयाग्निर्हि कोपमारुतचोदितः}
{सेनाकक्षं दहति मे वह्निः कक्षमिवोत्थितः}


\twolineshloka
{अतिक्रान्ते हि कौन्तेये भित्त्वा सैन्यं परन्तप}
{जयद्रथस्य गोप्तारः संशयं परमं गताः}


\twolineshloka
{स्थिरा बुद्धिर्नरेन्द्राणमासीद्ब्रह्मविदां वर}
{नातिक्रमिष्यति द्रोणं जातु जीवन्धनञ्जयः}


\twolineshloka
{योऽसौ पाथो व्यतिक्रान्तो मिषतस्ते महाद्युते}
{सर्वं ह्यद्यातुरं मन्ये नेदमस्ति बलं मम}


\twolineshloka
{जानामि त्वां महाभाग पाण्डवानां हिते रतम्}
{ततो मुह्यामि च ब्रह्मन्कार्यवत्तां विचिन्तयन्}


\twolineshloka
{यथाशक्ति च ते ब्रह्मन्वर्धयन्वृत्तिमुत्तमाम्}
{प्रीणामि च यथाशक्ति तच्च त्वं नावबुध्यसे}


\twolineshloka
{अस्मान्न त्वं सदा भक्तानिच्छस्यमितविक्रम}
{पाण्डवान्सततं प्रीणास्यास्माकं विप्रिये रतान्}


\twolineshloka
{अस्मानेवोपजीवंस्त्वमस्माकं विप्रिये रतः}
{न ह्यहं त्वां विजानामि मधुदिग्धमिव क्षुरम्}


\threelineshloka
{`नह्यहं त्वां विजानामि सर्वं मण्डूकराविणम्'}
{नादास्यच्चेद्वरं मह्यं भवान्पाण्डवनिग्रहे}
{नावारयिष्यं गच्छन्तमहं सिन्धुपतिं गृहान्}


\twolineshloka
{मया त्वाशंसमानेन त्वत्तस्त्राणमबुद्धिना}
{आश्वासितः सिन्धुपतिर्मोहाद्दत्तश्च मृत्यवे}


\twolineshloka
{यमदंष्ट्रान्तरं प्राप्तो मुच्येतापि हि मानवः}
{नार्जुनस्य वशं प्राप्तो मुच्येताजौ जयद्रथः}


\threelineshloka
{स तथा कुरु शोणश्व यथा मुच्येत सैन्धवः}
{मम चार्तप्रलापानां माक्रुधः पाहि सैन्धवम् ॥द्रोण उवाच}
{}


\twolineshloka
{नाभ्यसूयामि ते वाक्यमश्वत्थाम्नाऽसि मे समः}
{सत्यं तु ते प्रवक्ष्यामि तज्जुषस्व विशाम्पते}


\twolineshloka
{सारथिः प्रवरः कृष्णः शीघ्राश्चास्य हयोत्तमाः}
{अल्पं च विवरं कृत्वा पूर्णं याति धनञ्जयः}


\twolineshloka
{किं न पश्यसि बाणौघान्क्रोशमात्रे किरीटिनः}
{पश्चाद्रथस्य पतितान्क्षिप्ताञ्शीघ्रं हि गच्छतः}


\twolineshloka
{न ताबं शीघ्रयानेऽद्य समर्थो वयसान्वितः}
{सेनामुखे च पार्थानामेतद्बलमुपस्थितम्}


\twolineshloka
{युधिष्ठिरश्च मे ग्राह्यो मिषतां सर्वधन्विनाम्}
{एवं मया प्रतिज्ञातं क्षत्रमध्ये महाभुज}


\twolineshloka
{धनञ्जयेन चोत्सृष्टो वर्तते प्रमुखे नृप}
{तस्माद्व्यूहमुखं हित्वा नाहं यास्यामि फल्गुनम्}


\twolineshloka
{तुल्याभिजनकर्माणं शत्रुमेकं सहायवान्}
{गत्वा योधय मा भैस्त्वं त्वं ह्यस्य जगतः पतिः}


\threelineshloka
{राजा शूरः कृती दक्षो वैरमुत्पाद्य पाण्डवैः}
{वीरः स्वयं प्रयाह्याशु यत्र पार्थो धनञ्जयः ॥दुर्योधन उवाच}
{}


\twolineshloka
{कथं त्वामप्यतिक्रान्तः सर्वशस्त्रभृतां वरम्}
{धनञ्जयो मया शक्य आचार्य प्रतिबाधितुम्}


\twolineshloka
{अपि शक्यो रणे जेतुं वज्रहस्तः पुरन्दरः}
{नार्जुनः समरे शक्यो जेतुं परपुरञ्जयः}


\twolineshloka
{येन भोजश्च हार्दिक्यो भवांश्च त्रिदशोपमः}
{अस्त्रप्रतापेन जितौ श्रुतायुश्च निबर्हितः}


\twolineshloka
{सुदक्षिणश्च निहतः स च राजा श्रुतायुधः}
{श्रुतायुश्चाश्रुतायुश्च भ्लेच्छाश्चायुतशो हताः}


\threelineshloka
{तं कथं पाण्डवं युद्धे दहन्तमिव पावकम्}
{प्रतियोत्स्यामि दुर्धर्षं तन्ममाचक्ष्व सत्तम्}
{}


\threelineshloka
{क्षमं चेन्मन्यसे युद्धं मम तेनाद्य शाधि माम्}
{परवानस्मि भवतः प्रेष्यवद्रक्ष मे यशः ॥द्रोण उवाच}
{}


\twolineshloka
{सत्यं वदसि कौरव्य दुराधर्षो धनञ्जयः}
{अहं तु तत्करिष्यामि यथैनं प्रसहिष्यसि}


\twolineshloka
{अद्भुतं चाद्य पश्यन्तु लोके सर्वधनुर्धराः}
{विषक्तं त्वयि कौन्तेयं वासुदेवस्य पश्यतः}


\twolineshloka
{एष ते कवचं राजंस्तथा बध्नामि काञ्चनम्}
{यथा न बाणा नास्त्राणि प्रहरिष्यन्ति ते रणे}


\twolineshloka
{यदि त्वां सासुरसुराः सयक्षोरगराक्षसाः}
{योधयन्ति त्रयो लोकाः सनरा नास्ति ते भयम्}


\twolineshloka
{न कृष्णो न च कौन्तेयो न चान्यः शस्त्रभृद्रणे}
{शरानर्पयितुं कश्चित्कवचे तव शक्ष्यति}


\threelineshloka
{स त्वं कवचमास्थाय क्रुद्धमद्य रणेऽर्जुनम्}
{त्वरमाणः स्वयं याहि न त्वाऽसौ विषहिष्यति ॥सञ्जय उवाच}
{}


\twolineshloka
{एवमुक्त्वा त्वरन्द्रोणः स्पृष्ट्वाम्भो वर्म भास्वरम्}
{आबबन्धाद्भुततमं जपन्मन्त्रं यथाविधि}


\threelineshloka
{रणे तस्मिन्सुमहति विजयस्य सुतस्य ते}
{विसिष्मापयिषुर्लोकान्विद्यया ब्रह्मवित्तमः ॥द्रोण उवाच}
{}


\twolineshloka
{करोतु स्वस्ति ते ब्रह्मा स्वस्ति कुर्वन्तु ब्राह्मणाः}
{सरीसृपाश्च ये श्रेष्ठास्तेभ्यस्ते स्वस्ति भारत}


\twolineshloka
{ययातिर्नहुषश्चैव धुन्धुमारो भगीरथः}
{तुभ्यं राजर्षयः सर्वे स्वस्ति कुर्वन्तु ते सदा}


\twolineshloka
{स्वस्ति तेऽस्त्वेकपादेभ्यो बहुपादेभ्य एव च}
{स्वस्त्यस्त्वपादकेभ्यश्च नित्यं तव महारणे}


\twolineshloka
{स्वाहा स्वधा शची चैव स्वस्ति कुर्वन्तु ते सदा}
{लक्ष्मीररुन्धती चैव कुरुतां स्वस्ति तेऽनघ}


\twolineshloka
{अमितो देवलश्चैव विश्वामित्रस्तथाङ्गिराः}
{वसिष्ठः कश्यपश्चैव स्वस्ति कुर्वन्तु ते नृप}


\twolineshloka
{धाता विधाता लोकेशो दिशश्च सदिगीश्वराः}
{स्वस्ति तेऽद्य प्रयच्छन्तु कार्तिकेयश्च षण्मुखः}


\twolineshloka
{`येन देवाः सकृद्भग्नाः सङ्ग्रमे तारकामये}
{धृताः स च हतः शूरो ह्यवध्यो देवतागणैः}


\twolineshloka
{विवस्वान्कुरुतां स्वस्ति तथा वैवस्वतो यमः}
{पार्षिदा वशगा यस्य स्वस्ति तुभ्यं प्रयच्छतु'}


\threelineshloka
{दिग्गजाश्चैव चत्वारः क्षितिश्च गगनं ग्रहाः}
{`दिशश्च विदिशश्चैव सिद्धा लोकहिते रताः}
{स्वस्ति कुर्वन्तु ते नित्यं मन्त्रेणानेन रस्तुताः'}


\twolineshloka
{अधस्ताद्धरणीं योऽसौ सदा धारयते नृप}
{शेषश्च पन्नगश्रेष्ठः स्वस्ति तुभ्यं प्रयच्छतु}


\twolineshloka
{गान्धारे युधि विक्रम्य निर्जिते पाकशासनेः}
{पुरा वृत्रेण दैत्येन भिन्नदेहाः सहस्रशः}


\threelineshloka
{हृततेजोबलाः स्रवे तदा सेन्द्रा दिवौकसः}
{ब्रह्माणं शरणं जग्मुर्वृत्राद्भीतां महासुरात् ॥देवा ऊचुः}
{}


\twolineshloka
{प्रमार्दितानां वृत्रेण देवानां देवसत्तम}
{गतिर्भव सुरश्रेष्ठ त्राहि नो महतो भयात्}


\twolineshloka
{अथ पार्श्वे स्थितं विष्णुं शक्रादींश्च सुरोत्तमान्}
{प्राह पथ्यमिदं वाक्यं विषण्णान्सुरसत्तमान्}


\twolineshloka
{रक्ष्या मे सततं देवाः सहेन्द्राः सद्विजातयः}
{त्वष्टुः सुदुर्धरं तेजो येन वृत्रो विनिर्मितः}


\twolineshloka
{त्वष्ट्रा पुरा तपस्तप्त्वा वर्षायुतशतं तदा}
{वृत्रो विनिर्मितो देवाः प्राप्यानुज्ञां महेश्वरात्}


\twolineshloka
{शंकरस्य प्रसादाद्वै हन्याद्देवरिपुर्बली}
{नागत्वा शंकरस्थानं भगवान्दृश्यते हरः}


\twolineshloka
{दृष्ट्वा जेष्यथ वृत्रं तं क्षिप्रं गच्छत मन्दरम्}
{यत्रास्ते तपसां योनिर्दक्षयज्ञविनाशनः}


\twolineshloka
{पिनाकी सर्वभूतेशो भगनेत्रनिपातनः}
{ते गत्वा सहिता देवा ब्रह्मणा सह मन्दरम्}


\twolineshloka
{अपश्यंस्तेजसां राशिं सूर्यकोटिसमप्रभम्}
{सोऽब्रवीत्स्वागतं देवा ब्रूत किं करवाण्यहम्}


\twolineshloka
{अमोघं दर्शनं मह्यं कामप्राप्तिरतोऽस्तु वः}
{एवमुक्तास्तु ते सर्वे प्रत्युचूस्तं दिवौकसः}


\threelineshloka
{हृतौजसां नो वृत्रेण गतिर्भव दिवौकसाम्}
{मूर्तीरीक्षस्व नो देव प्रहारैर्जर्जरीकृताः}
{शरणं त्वां प्रपन्नाः स्म गतिर्भव महेश्वर}


\threelineshloka
{शर्वे उवाच}
{विदितं वो यथा देवाः कृत्येयं सुमहाबला}
{त्वष्टुस्तेजोभवा घोरा दुर्निवार्याऽकृतात्मभिः}


\twolineshloka
{अवश्यं तु मया कार्यं साह्यं सर्वदिवौकसाम्}
{ममेदं गात्रजं शक्र कवचं गृह्य भास्वरम्}


\twolineshloka
{बधानानेन मन्त्रेण मानसेन सुरेश्वर}
{वधायासुरमुख्यस्य वृत्रस्य सुरघातिनः}


\threelineshloka
{द्रोण उवाच}
{इत्युक्त्वा वरदः प्रादाद्वर्म तन्मन्त्रमेव च}
{स तेन वर्मणा गुप्तः प्रायाद्वृत्रचमूं प्रति}


\twolineshloka
{नानाविधैश्च शस्त्रौघैः पात्यमानैर्महारणे}
{न सन्धिः शक्यते भेत्तुं वर्मबन्धस्य तस्य तु}


\twolineshloka
{स तेन वर्मणा गुप्तो वृत्रं देवरिपुं तदा}
{जघान समरेऽभीतः शक्रो देवाग्रणीस्तदा}


\threelineshloka
{तं च मन्त्रमयं बन्धं वर्म चाङ्गिरसे ददौ}
{अङ्गिराः प्राह पुत्रस्य मन्त्रज्ञस्य बृहस्पतेः}
{बृहस्पतिरथोवाच अग्निवेश्याय धीमते}


\threelineshloka
{अग्निवेश्यो मम प्रादात्तेन बध्नामि वर्म ते}
{तवाह्य देहरक्षार्थं मन्त्रेण नृपसत्तम ॥सञ्जय उवाच}
{}


\twolineshloka
{एवमुक्त्वा ततो द्रोणस्तव पुत्रं महाद्युतिम्}
{पुनरे वचः प्राह शनैराचार्यपुङ्गवः}


\twolineshloka
{ब्रह्मसूत्रेण बध्नामि कथचं तव भारत}
{हिरण्यगर्भेण यथा बद्धं विष्णोः पुरा रणे}


\threelineshloka
{यथा च ब्रह्मणा बद्धं सङ्ग्रामे तारकामये}
{शक्रस्य कवचं दिव्यं तथा बध्नाम्यहं तव ॥सञ्जय उवाच}
{}


\twolineshloka
{बद्धा तु कवचं तस्य मन्त्रेण विधिपूर्वकम्}
{प्रेषयामास राजानं युद्धाय महते द्विजः}


\twolineshloka
{स सन्नद्धो महाबाहुराचार्येण महात्मना}
{`प्रस्थितः सहसा राजन्यत्र यातो धनञ्जयः'}


\twolineshloka
{रथानां च सहस्रेण त्रिगर्तानां प्रहारिणाम्}
{तथा दन्तिसहस्रेण मत्तानां वीर्यशालिनाम्}


\twolineshloka
{अश्वानां नियुतेनैव तथाऽन्यैश्च महारथैः}
{वृतः प्रायान्महाबाहुरर्जुनस्य रथं प्रति}


% Check verse!
नानावादित्रघोषेण यथा वैरोचनिस्तथा
\twolineshloka
{ततः शब्दो महानासीत्सैन्यानां तव भारत}
{अगाधं प्रस्थितं दृष्ट्वा समुद्रमिव कौरवम्}


\chapter{अध्यायः ९५}
\twolineshloka
{सञ्जय उवाच}
{}


\twolineshloka
{प्रविष्टयोर्महाराज पार्थवार्ष्णेययो रणे}
{दुर्योधने प्रयाते च पृष्ठतः पुरुषर्षभे}


\twolineshloka
{जवेनाभ्यद्रवन्द्रोणं महता निःस्वनेन च}
{पाण्डवाः सोमकैः सार्धं ततो युद्धमवर्तत}


\twolineshloka
{तद्युद्धमभवत्तीव्रं तुमुलं रोमहर्षणम्}
{कुरूणां पाण्डवानां च व्यूहस्य पुरतोऽद्भुतम्}


\twolineshloka
{राजन्कदाचिन्नास्माभिर्दृष्टं तादृङ्क च श्रुतम्}
{यादृङ्मध्यगते सूर्ये युद्धमासीद्विशाम्पते}


\twolineshloka
{धृष्टद्युम्नमुखाः पार्था व्यूढानीकाः प्रहारिणः}
{द्रोणस्य सैन्यं ते सर्वे शरवर्षैरवाकिरन्}


\twolineshloka
{वयं द्रोणं पुरस्कृत्य सर्वशस्त्रभृतां वरम्}
{पार्षतप्रमुखान्पार्थानभ्यवर्षाम सायकैः}


\twolineshloka
{महामेघाविवोदीर्णौ मिश्रवातौ हिमात्यये}
{सेनाग्रे प्रचकाशेते रुचिरे रथभूषिते}


\twolineshloka
{समेत्य तु महासेने चक्रतुर्वेगमुत्तमम्}
{जाह्नवीयमुने नद्यौ प्रावृषीवोल्बणोदके}


\twolineshloka
{नानाशस्त्रपुरोवातो द्विपाश्वरथसंवृतः}
{गदाविद्युन्महारौद्रः सङ्ग्रामजलदो महान्}


\twolineshloka
{भारद्वाजानिलोद्भूतः शरधारासहस्रवान्}
{अभ्यवर्षन्महारौद्रं पाण्डुसेनाग्निमुद्धतम्}


\twolineshloka
{समुद्रमिव घर्मान्ते विशन्घोरो महानिलः}
{व्यक्षोभयदनीकानि पाण्डवानां द्विजोत्तमः}


\twolineshloka
{तेऽपि सर्वप्रयत्नेन द्रोणमेव समाद्रवन्}
{बिभित्सन्तो महासेतुं वार्योघाः प्रबला इव}


\twolineshloka
{वारयामास तान्द्रोणो जलौघमचलो यथा}
{पाण्डवान्समरे क्रुद्धान्पाञ्चालांश्च सकेकयान्}


\twolineshloka
{अथापरे च राजानः परिवृत्य समन्ततः}
{महाबला रणे शूराः पाञ्चालानन्ववारयन्}


\twolineshloka
{ततो रणे नरव्याघ्रः पार्षतः पाण्डवैः सह}
{सञ्जघानासकृद्द्रोणं बिभित्सुररिवाहिनीम्}


\twolineshloka
{यथैव शरवर्षाणि द्रोणो वर्षति पार्षते}
{तथैव शरवर्षाणि धृष्टद्युम्नोऽप्यवर्षत}


\twolineshloka
{सनिस्त्रिंशपुरोवातः शक्तिप्रासर्ष्टिसंवृतः}
{ज्याविद्युच्चापसंहादो धृष्टद्युम्नबलाहकः}


\twolineshloka
{शरधाराश्मवर्षाणि व्यसृजत्सर्वतो दिशम्}
{निघ्नन्रथवराश्वौघान्प्लावयामास वाहिनीम्}


\twolineshloka
{यंयमार्च्छच्छरैर्द्रोणः पाण्डवानां रथव्रजम्}
{ततस्ततः शरैर्द्रोणमपाकर्षत पार्षतः}


\twolineshloka
{तथा तु यतमानस्य द्रोणस्य युधि भारत}
{धृष्टद्युम्नं समासाद्य त्रिधा सैन्यमभिद्यत}


\twolineshloka
{भोजमेकेऽभ्यवर्तन्त जलसन्धं तथाऽपरे}
{पाण्डवैर्हन्यमानाश्च द्रोणमेवापरे ययुः}


\twolineshloka
{सङ्घट्टयति सैन्यानि द्रोणस्तु रथिनां वरः}
{व्यधमच्चापि तान्यस्य धृष्टद्युम्नो महारथः}


\twolineshloka
{धार्तराष्ट्रास्तथा भूता वध्यन्ते पाण्डुसृञ्जयैः}
{अगोपाः पशवोऽरण्ये बहुभिः श्वापदैरिव}


\twolineshloka
{कालः स्म ग्रसते योधान्धृष्टद्युम्नेन मोहितान्}
{सङ्ग्रामे तुमुले तस्मिन्निति सम्मेनिरे जनाः}


\twolineshloka
{कुनृपस्य यथा राष्ट्रं दुर्भिक्षव्याधितस्करैः}
{द्राव्यते तद्वदापन्ना पाण्डवैस्तव वाहिनी}


\twolineshloka
{अर्करश्मिविमिश्रेषु शस्त्रेषु कवचेषु च}
{चक्षूंषि प्रत्यहन्यन्त सैन्येन रजसा तथा}


\twolineshloka
{त्रिधा भूतेषु सैन्येषु वध्यमानेषु पाण्डवैः}
{अमर्षितस्ततो द्रोणः पाञ्चालान्व्यधमच्छरैः}


\twolineshloka
{मृद्गतस्तान्यनीकानि निघ्नतश्चापि सायकैः}
{बभूव रूपं द्रोणस्य कालाग्नेरिव दीप्यतः}


\twolineshloka
{रथं नागं हयं चापि पत्तिनश्च विशाम्पते}
{एकैकेनेषुणा सङ्ख्ये निर्बिभेद महारथः}


\twolineshloka
{पाण़्डवानां तु सैन्येषु नास्ति कश्चित्स भारत}
{दधार यो रणे बाणान्द्रोणचापच्युतान्प्रभो}


\twolineshloka
{तत्पच्यमानमर्केण द्रोणसायकतापितम्}
{बभ्राम पार्षतं सैन्यं तत्रतत्रैव भारत}


\twolineshloka
{`हन्यमानं तु तत्सैन्यं भारद्वाजेन सर्वतः}
{शरैरग्निशिखाकारैर्दह्यते भरतर्षभ'}


\twolineshloka
{तथैव पार्षतेनापि काल्यमानं बलं तव}
{अभवत्सर्वतो दीप्तं शुष्कं वनमिवाग्निना}


\twolineshloka
{बाध्यमानेषु सैन्येषु द्रोणपार्षतसायकैः}
{त्यक्त्वा प्राणान्परं शक्त्या युध्यन्ते स्वर्गलिप्सवः}


\twolineshloka
{तावकानां परेषां च युध्यतां भरतर्षभ}
{नासीत्कश्चिन्महाराज योऽत्याक्षीत्संयुगं भयात्}


\twolineshloka
{भीमसेनं तु कौन्तेयं सोदर्याः पर्यवारयन्}
{विविंशतिश्चित्रसेनो विकर्णश्च महारथः}


\twolineshloka
{विन्दानुविन्दावावन्त्यौ क्षेमधूर्तिश्च वीर्यवान्}
{त्रयाणां तव पुत्राणां त्रय एवानुयायिनः}


\twolineshloka
{बाह्लीकराजस्तेजस्वी कुलपुत्रो महारथः}
{सहसेनः सहामात्यो द्रौपदेयानवारयत्}


\twolineshloka
{शैब्यो गोवासनो राजा योधैर्दशशतावरैः}
{काश्यस्याभिभुवः पुत्रं पराक्रान्तमवारयत्}


\threelineshloka
{अजातशत्रुं कौन्तेयं ज्वलन्तमिव पावकम्}
{मद्राणामीश्वरः शल्यो राजा राजानमावृणोत्}
{}


\twolineshloka
{दुःशासनस्त्ववस्थाप्य स्वमनीकममर्षणः}
{सात्यकिं प्रययौ क्रुद्धः शूरो रथवरं युधि}


\twolineshloka
{स्वकेनाहमनीकेन सन्नद्धः कवचावृतः}
{चतुःशतैर्महेष्वासैश्चेकितानमवारयम्}


\twolineshloka
{शकुनिस्तु सहानीको माद्रीपुत्राववारयत्}
{गान्धारकैः सप्तशतैश्चापशक्त्यसिपाणिभिः}


\twolineshloka
{विन्दानुविन्दावावन्त्यौ विराटं मत्स्यमार्च्छताम्}
{प्राणांस्तक्त्वा महेष्वासौ मित्रार्थेऽभ्युद्यतायुधौ}


\twolineshloka
{शिखण्डिनं याज्ञसेनिं रुन्धानमपराजितम्}
{बाह्लीकः प्रतिसंयत्तः पराक्रान्तमवारयत्}


\twolineshloka
{धृष्टद्युम्नं तु पाञ्चाल्यं क्रूरैः सार्धं प्रभद्रकैः}
{आवन्त्यः सहसौवीरैः क्रुद्धरूपमवारयत्}


\twolineshloka
{घटोत्कचं तथा शूरं राक्षसं क्रूरकर्मिणम्}
{अलामुधोऽद्रवत्तूर्णं क्रुद्धमायान्तमाहवे}


\twolineshloka
{अलम्बुसं राक्षसेन्द्रं कुन्तिभोजो महारथः}
{सैन्येन महता युक्तः क्रुद्धरूपमवारयत्}


\twolineshloka
{सैन्धवः पृष्ठतस्त्वासीत्सर्वसैन्यस्य भारत}
{रक्षितः परमेष्वासैः कृपप्रभृतिभी रथैः}


\twolineshloka
{तस्यास्तां चक्ररक्षौ द्वौ सैन्धवस्य बृहत्तमौ}
{द्रौणिर्दक्षिणतो राजन्सूतपुत्रश्च वामतः}


\twolineshloka
{पृष्ठगोपास्तु तस्यासन्सौमदत्तिपुरोगमाः}
{कृपश्च वृषसेनश्च शलः शल्यश्च दुर्जयः}


\twolineshloka
{नीतिमन्तो महेष्वासाः सर्वे युद्धविशारदाः}
{सैन्धवस्य विधायैवं रक्षां युयुधिरे ततः}


\chapter{अध्यायः ९६}
\twolineshloka
{सञ्जय उवाच}
{}


\twolineshloka
{राजन्सङ्ग्राममाश्चर्यं शृणु कीर्तयतो मम}
{कुरूणां पाण्डवानां च यथा युद्धमवर्तत}


\twolineshloka
{भारद्वाजं समासाद्य व्यूहस्य प्रमुखे स्थितम्}
{अयोधयन्रणे पार्था द्रोणानीकं बिभित्सवः}


\twolineshloka
{रक्षमाणः स्वथं व्यूहं द्रोणोऽपि सह सैनिकैः}
{अयोधयद्रणे पार्थान्प्रार्थयानो महद्यशः}


\twolineshloka
{विन्दानुविन्दावावन्त्यौ विराटं दशभिः शरैः}
{अयोधयद्रणे पार्थान्प्रार्थयानो महद्यशः}


\twolineshloka
{विन्दानुविन्दावावन्त्यौ विराटं दशभिः शरैः}
{पराक्रान्तौ पराक्रम्य योधयामास सानुगौ}


\twolineshloka
{तेषां युद्धं समभवद्दारुणं शोणितोदकम्}
{सिंहस्य द्विपमुख्याभ्यां प्रभिन्नाभ्यां यथा वने}


\twolineshloka
{बाह्लीकं रभसं युद्धे याज्ञसेनिर्महाबलः}
{आजघ्ने विशिखैस्तीक्ष्णैर्घोरैर्मर्मास्थिभेदिभिः}


\twolineshloka
{बाह्लीको याज्ञसेनिं तु हेमपुङ्खैः शिलाशितैः}
{आजघान भृशं क्रुद्धो नवभिर्नतपर्वभिः}


\twolineshloka
{तद्युद्धमभवद्धोरं शरशक्तिसमाकुलम्}
{भीरूणां त्रासजननं शूराणां हर्षवर्धनम्}


\twolineshloka
{ताभ्यां तत्र शरैर्मुक्तैरन्तरिक्षं दिशस्तथा}
{अभवत्संवृतं सर्वं न प्राज्ञायत किञ्चन}


\twolineshloka
{शैब्यो गोवासनो युद्धे काश्यपुत्रं महारथम्}
{ससैन्यो योधयामास गजः प्रतिगजं यथा}


\twolineshloka
{बाह्लीकराजः सङ्क्रुद्धो द्रौपदेयान्महारथान्}
{मनः पञ्चेन्द्रियाणीव शुशुभे योधयन्रणे}


\twolineshloka
{अयोधयंस्ते सुभृशं तं शरौघैः समन्ततः}
{इन्द्रियाणि यथा देहं शश्वद्देहवतां वर}


\twolineshloka
{वार्ष्णेयं सात्यकिं युद्धे पुत्रो दुःशासनस्तव}
{आजघ्ने सायकैस्तीक्ष्णैर्नवभिर्नतपर्वभिः}


\twolineshloka
{सोऽतिविद्धो बलवता महेष्वासेन धन्विना}
{ईषन्मूर्च्छां जगामाशु सात्यकिः सत्यविक्रमः}


\twolineshloka
{समाश्वस्तस्तु वार्ष्णेयस्तव पुत्रं महारथम्}
{विव्याध दशभिस्तूर्णं सायकैः कङ्कपत्रिभिः}


\twolineshloka
{तावन्योन्यं दृढं विद्धावन्योन्यशरपीडितौ}
{रजेतुः समरे राजन्पुष्पिताविव किंशुकौ}


\twolineshloka
{अलम्बुसस्तु सङ्क्रुद्धः कृन्तिभोजशरार्दितः}
{अशोभत भृशं लक्ष्म्या पुष्पाढ्य इव किंशुकः}


\twolineshloka
{कुन्तिभोजं ततो रक्षो विद्ध्वा बहुभिरायसैः}
{अनदद्भैरवं नादं वाहिन्याः प्रमुखे तव}


\twolineshloka
{ततस्तौ समरे शूरौ योधयन्तौ परस्परम्}
{ददृशुः सर्वसैन्यानि शक्रजम्भौ यथा पुरा}


\twolineshloka
{शकुनिं रभसं युद्धे कृतवैरं च भारत}
{माद्रीपुत्रौ च संरब्धौ शरैश्चार्दयतां भृशम्}


\twolineshloka
{तुमुलः स महान्राजन्प्रावर्तत जनक्षयः}
{त्वया सञ्जनितोऽत्यर्थं कर्णेन च विवर्धितः}


\twolineshloka
{रक्षितस्तव पुत्रैश्च क्रोधमूलो हुताशनः}
{य इमां पृथिवीं राजन्दग्धुं सर्वां समुद्यतः}


\twolineshloka
{शकुनिः पाण्डुपुत्राभ्यां कृतः स विमुखः शरैः}
{न स्म जानाति कर्तव्यं युद्धे किञ्चित्पराक्रमम्}


\twolineshloka
{विमुखं चैनमालोक्य माद्रीपुत्रौ महारथौ}
{ववर्षतुः पुनर्बाणैर्यथा मेघौ महागिरिम्}


\twolineshloka
{स वध्यमानो बहुभिः शरैः सन्नतपर्वभिः}
{सम्प्रायाज्जवनैरश्वैर्द्रोणानीकाय सौबलः}


\twolineshloka
{घटोत्कचस्तथा शूरं राक्षसं तमलामुधम्}
{अभ्ययाद्रभसं युद्धे वेगमास्थाय मध्यमम्}


\twolineshloka
{तयोर्युद्धं महाराज चित्ररूपमिवाभवत्}
{यादृशं हि पुरावृत्तं रामरावणयोर्मृधे}


\twolineshloka
{ततो युधिष्ठिरो राजा मद्रराजानमाहवे}
{विद्ध्वा पञ्चाशता बाणैः पुनर्विव्याध सप्तभिः}


\twolineshloka
{ततः प्रववृते युद्धं तयोरत्यद्भुतं नृप}
{यथा पूर्वं महद्युद्धं शम्बरामरराजयोः}


\twolineshloka
{विविंशतिश्चित्रसेनो विकर्णश्च तवात्मजः}
{अयोधयन्भीमसेनं महत्या सेनया वृताः}


\chapter{अध्यायः ९७}
\twolineshloka
{सञ्जय उवाच}
{}


\twolineshloka
{तथा तस्मिन्प्रवृत्ते तु सङ्ग्रमे रोमहर्षणे}
{कौरवेयांस्त्रिधाभूतान्पाण्डवाः समुपाद्रवन्}


\twolineshloka
{जलसन्धं महाबाहुं भीमसेनोऽभ्यवर्तत}
{युधिष्ठिरः सहानीकः कृतवर्माणमाहवे}


\twolineshloka
{किरंस्तु शरवर्षाणि रोचमान इवांशुमान्}
{धृष्टद्युम्नो महाराज द्रोणमभ्यद्रवद्रणे}


\twolineshloka
{ततः प्रववृते युद्धं त्वरतां सर्वधन्विनाम्}
{कुरूणां पाण्डवानां च सङ्क्रुद्धानां परस्परम्}


\twolineshloka
{सङ्क्षये तु तथाभूते वर्तमाने महाभये}
{द्वन्दवीभूतेषु सैन्येषु युध्यमानेष्वभीतवत्}


\twolineshloka
{द्रोणः पाञ्चालपुत्रेण बली बलवता सह}
{यदक्षिपत्पृषत्कौघांस्तदद्भुतमिवाभवत्}


\twolineshloka
{पुण्डरीकवनानीव विध्वस्तानि समन्ततः}
{चक्राते द्रोणपाञ्चाल्यौ नृणां शीर्षाण्यनेकशः}


\twolineshloka
{विनिकीर्णानि वीराणामनीकेषु समन्ततः}
{वस्त्राभरणशस्त्राणि ध्वजवर्मायुधानि च}


\threelineshloka
{तपनीयतनुत्राणाः संसिक्ता रुधिरेण च}
{कोविदारा इवाभान्ति पुष्पितास्तत्र भारत}
{तावकाः समरे योधाः पाण्डवेयाश्च विक्षताः}


\twolineshloka
{कुञ्जराश्वनरान्ये पातयन्ति स्म पत्रिभिः}
{तालमात्राणि चापानि विकर्षन्तो महारथाः}


\twolineshloka
{असिचर्माणि चापानि शिरांशि कवचानि च}
{विप्रकीर्यन्त शूराणां सम्प्रहारे महात्मनाम्}


\twolineshloka
{उत्थितान्यगणेयानि कबन्धानि समन्ततः}
{अदृश्यन्त महाराज तस्मिन्परमसङ्कुले}


\twolineshloka
{गृध्राः कङ्का बकाः श्येना वायसा जम्बुकास्तथा}
{बहुशः पिशिताशाश्च तत्रादृश्यन्त मारिष}


\twolineshloka
{भक्षयन्तश्च मांसानि पिबन्तश्चापि शोणितम्}
{विलुम्पमानाः केशांश्च मज्जाश्च बहुधा नृप}


\twolineshloka
{आकर्षन्तः शरीराणि शरीरावयवांस्तथा}
{नराश्वगजसङ्घानां शिरांशि च ततस्ततः}


\twolineshloka
{कृतास्त्रा रणशिक्षाभिर्दीक्षिता रणशालिनः}
{रणे जयं प्रार्थयाना भृशं युयुधिरे तदा}


\twolineshloka
{असिमार्गान्बहुविधान्विचेरुः सैनिका रणे}
{ऋष्टिभिः शक्तिभिः प्रासैः शूलतोमरपट्टसैः}


\twolineshloka
{गदाभिः परिघैश्चान्यैरायुधैश्च भुजैरपि}
{अन्योन्यं जघ्निरे क्रुद्धा युद्धरङ्गता नराः}


\twolineshloka
{रथिनो रथिभिः सार्धमश्वारोहाश्च सादिभिः}
{मातङ्गा वरमातङ्गैः पदाताश्च पदातिभिः}


\twolineshloka
{क्षीबा इवान्ये चोन्मत्ता रङ्गेष्विव च वारणाः}
{उच्चुक्रुशुरथान्योन्यं जघ्नुरन्योन्यमेव च}


\twolineshloka
{वर्तमाने तथा युद्धे निर्मर्यादे विशाम्पते}
{धृष्टद्युम्नो हयानश्चैर्द्रोणस्य व्यत्यमिश्रयत्}


\twolineshloka
{ते हयाः साध्वशोभन्त मिश्रिता वातरंहसः}
{पारावतसवर्णाश्च रक्तशोणाश्च संयुगे}


\twolineshloka
{पारावतसवर्णास्ते रक्तशोणविमिश्रिताः}
{हयाः शुशुभिरे राजन्मेघा इव सविद्युतः}


\twolineshloka
{धृष्टद्युम्नस्तु सम्प्रेक्ष्य द्रोणमभ्याशमागतम्}
{असिचर्माददे वीरो धनुरुत्सृज्य भारत}


\twolineshloka
{चिकीर्षुर्दुष्करं कर्म पार्षतः परवीरहा}
{ईषायाः समतिक्रम्य द्रोणस्य रथमाविशत्}


\twolineshloka
{अतिष्ठद्युगमध्ये स युगसन्नहनेषु च}
{जघनार्धेषु चाश्वानां तत्सैन्यान्यभ्यपूजयन्}


\twolineshloka
{खङ्गेन चरतस्तस्य शोणाश्वानधितिष्ठतः}
{न ददार्शान्तरं द्रोणस्तदद्भुतमिवाभवत्}


\twolineshloka
{यथा श्येनस्य पतनं वनेष्वामिषगृद्धिनः}
{तथैवासीदभीसारस्तस्य द्रोणं जिघांसतः}


\twolineshloka
{ततः शरशतेनास्य शतचन्द्रं समाक्षिपत्}
{द्रोणो द्रुपद पुत्रस्य खङ्गं च दशभिः शरैः}


\twolineshloka
{हयांश्चैव चतुःषष्ट्या शराणां जघ्निवान्बली}
{ध्वजं छत्रं च भल्लाभ्यां तथा तौ पार्ष्णिसारथी}


\twolineshloka
{अथास्मै त्वरितो बाणमपरं जीवितान्तकम्}
{धृष्टद्युम्नाय चिक्षेप वज्रं वज्रधरो यथा}


\twolineshloka
{तं चतुर्दशभिस्तीक्ष्णैर्बाणैश्चिच्छेद सात्यकिः}
{ग्रस्तमाचार्यमुख्येन धृष्टद्युम्नं व्यमोचयत्}


\twolineshloka
{सिंहेनेव मृगं ग्रस्तं नरसिंहेन मारिष}
{द्रोणाद्वै मोचयामास पाञ्चाल्यं शिनिपुङ्गवः}


\twolineshloka
{सात्यकिं प्रेक्ष्य गोप्तारं पाञ्चाल्यं च महाहवे}
{शराणां त्वरितो द्रोणः षड्विंशत्या समार्पयत्}


\twolineshloka
{ततो द्रोणं शिनेः पौत्रो ग्रसन्तमपि सृञ्जयान्}
{प्रत्यविध्यच्छितैर्बाणैः षड्विंशत्या स्तनान्तरे}


\twolineshloka
{ततः सर्वे रथास्तूर्णं पाञ्चाल्या जयगृद्धिनः}
{सात्वताभिसृते द्रोणे धृष्टद्युम्नमवाक्षिपन्}


\chapter{अध्यायः ९८}
\twolineshloka
{धृतराष्ट्र उवाच}
{}


\twolineshloka
{बाणे तस्मिन्निकृत्ते तु धृष्टद्युम्ने च मोक्षिते}
{तेन वृष्णिप्रवीरेण युयुधानेन सञ्जय}


\threelineshloka
{अमर्षितो महेष्वासः सर्वशस्त्रभृतां वरः}
{नरव्याघ्रं शिनेः पौत्रं द्रोणः किमकरोद्युधि ॥सञ्जय उवाच}
{}


\twolineshloka
{सम्प्रस्रुतक्रोधविषो व्यादितास्यशरासनः}
{तीक्ष्णधारेषुदशनः सितनाराचदंष्ट्रवान्}


\twolineshloka
{संरम्भामर्षताम्राक्षो महोरग इव श्वसन्}
{नरवीरः प्रमुदितः शोणैरश्वैर्महाजवैः}


\twolineshloka
{उत्पतद्भिरिवाकाशे क्रामद्भिरिव पर्वतम्}
{रुक्मपुङ्खाञ्शरानस्यन्युयुधानमुपाद्रवत्}


\twolineshloka
{शरपातमहावर्षं रथघोषबलाहकम्}
{कार्मुकैरावृतं राजन्नाराचबहुविद्युतम्}


\twolineshloka
{शक्तिखङ्गाशनिधरं क्रोधवेगसमुत्थितम्}
{द्रोणमेघमनावार्यं हयमारुतचोदितम्}


\twolineshloka
{दृष्ट्वैवाभिपतन्तं तं शूरः परपुरञ्जयः}
{उवाच सूतं शैनेयः प्रहसन्युद्धदुर्मदः}


\twolineshloka
{एनं वै ब्राह्मणं शूरं स्वकर्मण्यनवस्थितम्}
{आश्रयं धार्तराष्ट्रस्य धर्मराजभयावहम्}


\threelineshloka
{शीघ्रं प्रजवितैरश्वैः प्रत्युद्याहि प्रहृष्टवत्}
{आचार्यं राजपुत्राणां सततं शूरमानिनम् ॥सञ्जय उवाच}
{}


\threelineshloka
{`एवमुक्तस्ततः सूतः सात्यकस्यावहद्रथम्'}
{ततो रजतसङ्काशा माधवस्य हयोत्तमाः}
{द्रोणस्याभिमुखाः शीघ्रमगच्छन्वातरंहसः}


\twolineshloka
{ततस्तौ द्रोणशैनेयौ युयुधाते परन्तपौ}
{शरैरनेकसाहस्रैस्ताडयन्तौ परस्परम्}


\threelineshloka
{इषुजालावृतं व्योम चक्रतुः पुरुषर्षभौ}
{पूरयामासतुर्वीरावुभौ दश दिशः शरैः}
{मेघाविवातपापाये धाराभिरितरेतरम्}


\twolineshloka
{न स्म सूर्यस्तदा भाति न ववौ च समीरणः}
{हषुजालावृतं घोरमन्धकारं समन्ततः}


\twolineshloka
{अनाधृष्यमिवान्येषां शूराणामभवत्तदा}
{अन्धकारीकृते लोके द्रोणशैनेययोः शरैः}


\twolineshloka
{तयोः शीघ्रास्त्रविदुषोर्द्रोणसात्वतयोस्तदा}
{नान्तरं शरवृष्टीनां ददृशे नरसिंहयोः}


\twolineshloka
{इषूणां सन्निपातेन शब्दो धाराभिघातजः}
{शुश्रुवे शक्रमुक्तानामशनीनामिव स्वनः}


\twolineshloka
{नाराचैर्व्यतिविद्धानां शराणां रूपमाबभौ}
{आशीविषविदष्टानां सर्पाणामिव भारत}


\twolineshloka
{तयोर्ज्यातलनिर्घोषः शुश्रुवे युद्धशौण्डयोः}
{अजस्रं शैलशृङ्गाणां वज्रेणाहन्यतामिव}


\twolineshloka
{उभयोस्तौ रथौ राजंस्ते चाश्वास्तौ च सारथी}
{रुक्मपुङ्खैः शरैछन्नाश्चित्ररूपा बभुस्तदा}


\twolineshloka
{निर्मलानामजिह्मानां नाराचानां विशाम्पते}
{निर्मुक्ताशीविपाभानां सम्पातोऽभूत्सुदारुणः}


\twolineshloka
{उभयोः पतिते छत्रे तथैव पतितौ ध्वजौ}
{`निकृन्ततोः शरैस्तीक्ष्णैर्द्रोणसात्यकयो रणे'}


\threelineshloka
{उभौ रुधिरसिक्ताङ्गावुभौ च विजयैषिणौ}
{स्रवद्भिः शोणितं गात्रैः प्रस्रुताविव वारणौ}
{अन्योन्यमभ्यविध्येतां जीवितान्तकरैः शरैः}


\twolineshloka
{गर्जितोत्कृष्टसन्नादाः शङ्खदुन्दुभिनिःस्वनाः}
{उपारमन्महाराज व्याजहार न कश्चन}


\twolineshloka
{तुष्णींभूतान्यनौकानि योधा युद्धादुपारमन्}
{ददर्श द्वैरथं ताभ्यां जातकौतूहलो जनः}


\twolineshloka
{रथिनो हस्तियन्तारो हयारोहाः पदातयः}
{अवैक्षन्ताचलैर्नेत्रैः परिवार्य नरर्षभौ}


\twolineshloka
{हस्त्यनीकान्यतिष्ठन्त तथानीकानि वाजिनाम्}
{तथैव रथवाहिन्यः प्रतिव्यूह्य व्यवस्थिताः}


\twolineshloka
{मुक्ताविद्रुमचित्रैश्च मणिकाञ्चनभूषितैः}
{ध्वजैराभरणैश्चित्रैः कवचैश्च हिरण्मयैः}


\twolineshloka
{वैजयन्तीपताकाभिः परिस्तोमाङ्गकम्बलैः}
{विमलैर्निशितैः शस्त्रैर्हयानां च प्रकीर्णकैः}


\twolineshloka
{जातरूपमयीभिश्च राजतीभिश्च मूर्धसु}
{गजानां कुम्भमालाभिर्दन्तवेष्टैश्च भारत}


\twolineshloka
{सबलाकाः सखद्योताः सैरावतशतहदाः}
{अदृश्यन्तोष्णपर्याये मेघानामिव वागुराः}


\twolineshloka
{अपश्यन्नस्मदीयाश्च ते च यौधिष्ठिराः स्थिताः}
{तद्युद्धं युयुधानस्य द्रोणस्य च महात्मनः}


\twolineshloka
{विमानाग्रगता देवा ब्रह्मसोमपुरोगमाः}
{सिद्धचारणसङ्खाश्च विद्याधरमहोरगाः}


\threelineshloka
{`गन्धर्वा दानवा यक्षा राक्षसाप्सरसः खगाः'}
{गतप्रत्यागताक्षेपैश्चितैरस्त्रविघातिभिः}
{विविधैर्विस्मयं जग्मुस्तयोः पुरुषसिंहयोः}


\twolineshloka
{हस्तलाघवमस्त्रेषु दर्शयन्तौ महाबलौ}
{अन्योन्यमभिविध्येतां शरैस्तौ द्रोणसात्यकी}


\twolineshloka
{ततो द्रोणस्य दाशार्हः शरांश्चिच्छेद संयुगे}
{पत्रिभिः सुदृढैराशु धनुश्चैव महाद्युतेः}


\twolineshloka
{निमेषान्तरमात्रेण भारद्वाजोऽपरं धनुः}
{सज्यं चकार तदपि चिच्छेदास्य स सात्यकिः}


\twolineshloka
{ततस्त्वरन्पुनर्द्रोणो धनुर्हस्तो व्यतिष्ठत}
{सज्यं सज्यं धनुश्चास्य चिच्छेद निशिथैः शरैः}


\twolineshloka
{एवमेकशतं छिन्नं धनुषां दृढधन्विना}
{न चान्तरं तयोर्दृष्टं सन्धाने च्छेदनेऽपि च}


\twolineshloka
{ततोऽस्य संयुगे द्रोणो दृष्ट्वा कर्मातिमानुषम्}
{युयुधानस्य राजेन्द्र मनसैतदचिन्तयत्}


\twolineshloka
{एतदस्त्रबलं रामे कार्तवीर्ये धनञ्जये}
{भीष्मे च पुरुषव्याघ्रे यदिदं सात्वतां वरे}


\twolineshloka
{तं चास्य मनसा द्रोणः पूजयामास विक्रमम्}
{लाघवं वासवस्येव सम्प्रेक्ष्य द्विजसत्तमः}


\twolineshloka
{तुतोषास्त्रविदां श्रेष्ठस्तथा देवाः सवासवाः}
{न तामालक्षयामासुर्लघुतां शीघ्रचारिणः}


\twolineshloka
{देवाश्च युयुधानस्य गन्धर्वाश्च विशाम्पते}
{सिद्धचारणसङ्घाश्च विदुर्द्रोणस्य कर्म तत्}


\twolineshloka
{ततोऽन्यद्धनुरादाय द्रोणः क्षत्रियमर्दनः}
{अस्त्रैरस्त्रविदां श्रेष्ठो योधयामास भारत}


\twolineshloka
{तस्यास्त्राण्यस्त्रमायाभिः प्रतिहत्य स सात्यकिः}
{जघान निशितैर्बाणैस्तदद्भुतमिवाभवत्}


\twolineshloka
{तस्यातिमानुषं कर्म दृष्ट्वाऽन्यैरसमं रणे}
{युक्तं योगेन योगज्ञास्तावकाः समपूजयन्}


\twolineshloka
{यदस्त्रमस्यति द्रोणस्तदेवास्यति सात्यकिः}
{तमाचार्योऽथ सम्भ्रान्तोऽयोधयच्छत्रुतापनः}


\twolineshloka
{ततः क्रुद्धो महाराज धनुर्वेदस्य पारगः}
{वधाय युयुधानस्य दिव्यमस्त्रमुदैरयत्}


\twolineshloka
{तदाग्नेयं महाघोरं रिपुघ्नमुपलक्ष्य सः}
{दिव्यमस्त्रं महेष्वासो वारुणं समुदैरयत्}


\twolineshloka
{हाहाकारो महानासीद्दृष्ट्वा दिव्यास्त्रधारिणौ}
{न विचेरुस्तदाऽऽकाशे भूतान्याकाशगान्यपि}


\twolineshloka
{अस्त्रे ते वारुणाग्नेये यदा बाणैः समाहिते}
{न यावदभ्यपद्येतां व्यावर्तदथ भास्करः}


\twolineshloka
{ततो युधिष्ठिरो राजा भीमसेनश्च पाण्डवः}
{नकुलः सहदेवश्च पर्यरक्षन्त सात्यकिम्}


\twolineshloka
{धृष्टद्युम्नमुखैः सार्धं विराटश्च सकेकयः}
{मत्स्याः साल्वेयसेनाश्च द्रोणमाजग्मुरञ्जसा}


\twolineshloka
{दुःशासनं पुरस्कृत्य राजपुत्राः सहस्रशः}
{द्रोणमभ्युपपद्यन्त सपत्नैः परिवारितम्}


\twolineshloka
{रजसा संवृते लोके शरजालसमावृते ॥सर्वमाविग्नमभवन्न प्राज्ञायत किञ्चन}
{}


\twolineshloka
{सैन्येन रजसा ध्वस्ते निर्मर्यादमवर्तत ॥`तेनान्तरेण पार्थस्तु रणे जित्वा महारिपून्}
{}


% Check verse!
अतिक्रान्तस्तदा युद्धं कृत्वा पर्यवतस्थिवान्'
\chapter{अध्यायः ९९}
\twolineshloka
{सञ्जय उवाच}
{}


\threelineshloka
{`वर्तमाने तदा युद्धे द्रोणस्य सह पाण्डुभिः'}
{परिवर्तमान आदित्ये तत्र सूर्ये च रश्मिभिः}
{रजसा कीर्यमाणाश्च मन्दीभूताश्च कौरवाः}


\twolineshloka
{तिष्ठतां युध्यमानानां पुनरावर्ततामपि}
{भज्यतां जयतां चैव जगाम तदहः शनैः}


\twolineshloka
{तथा तेषु विषक्तेषु सैन्येषु जयगृद्धिषु}
{अर्जुनो वासुदेवश्च सैन्धवायैव जग्मतुः}


\twolineshloka
{रथमार्गप्रमाणं तु कौन्तेयो निशितैः शरैः}
{चकार यं च पन्थानं ययौ तेन जनार्दनः}


\twolineshloka
{यत्रयत्र रथो याति पाण्डवस्य महात्मनः}
{तत्रतत्रैव दीर्यन्ते सेनास्तव विशाम्पते}


\twolineshloka
{रथशिक्षां तु दाशार्हो दर्शयामास वीर्यवान्}
{उत्तमाधममध्यानि मण्डलानि विदर्शयन्}


\twolineshloka
{ते तु नामाङ्किता बाणाः कालज्वलनसन्निभाः}
{स्नायुनद्धाः सुपर्वाणः पृथवो दीर्घगामिनः}


\twolineshloka
{वैणवाश्चायसाश्चोग्रा ग्रसन्तो विविधानरीन्}
{रुधिरं पतगैः सार्धं प्राणिनां पपुराहवे}


\twolineshloka
{रथस्थितोऽग्रतः क्रोशं यानस्यत्यर्जुनः शरान्}
{रथे क्रोशमतिक्रान्ते तस्य ते घ्नन्ति शात्रवान्}


\twolineshloka
{तार्क्ष्यमारुतरंहोभिर्वाजिभिः साधुवाहिभिः}
{यथाऽगच्छद्धृषीकेशः कृत्स्नं विस्मापयञ्जगत्}


\twolineshloka
{न तथा गच्छति रथस्तपनस्य विशाम्पते}
{नेन्द्रस्य न तु रुद्रस्य नापि वैश्रवणस्य च}


\twolineshloka
{नान्यस्य समरे राजन्गतपूर्वस्तथा रथः}
{यथा ययावर्जुनस्य मनोभिप्रायशीघ्रगः}


\twolineshloka
{प्रविश्य तु रणे राजन्केशवः परवीरहा}
{सेनामध्ये हयांस्तूर्णं चोदयामास भारत}


\twolineshloka
{ततस्तस्य रथौघस्य मध्यं प्राप्य हयोत्तमाः}
{कृच्छ्रेणरथमूहुस्तं क्षुत्पिपासासमन्विताः}


\twolineshloka
{क्षताश्च बहुभिः शस्त्रैर्युद्धशौण्डैरनेकशः}
{मण्डलानि विचित्राणि विचेरुस्ते मुहुर्मुहुः}


\twolineshloka
{हतानां वाजिनागानां रथानां च नरैः सह}
{उपरिष्टादतिक्रान्ताः शैलाभानां सहस्रशः}


\twolineshloka
{`श्रमेण महता युक्तास्ते हया वातरंहसः}
{मन्दवेगगता राजन्संवृत्तास्तत्र संयुगे'}


\twolineshloka
{एतस्मिन्नन्तरे वीरावावन्त्यौ भ्रातरौ नृप}
{सहसेनौ समार्च्छेतां पाण्डवं क्लान्तवाहनम्}


\twolineshloka
{तावर्जुनं चतुःषष्ट्या सप्तत्या च जनार्दनम्}
{शराणां च शतैरश्वानविध्येतां मुदान्वितौ}


\twolineshloka
{तावर्जुनो महाराज नवभिर्नतपर्वभिः}
{आजघान रणे क्रुद्धो मर्मज्ञो मर्मभेदिभिः}


\twolineshloka
{ततस्तौ तु शरौघेण बीभत्सुं सहकेशवम्}
{आच्छादयेतां संरब्धौ सिंहनादं च चक्रतुः}


\threelineshloka
{तयोस्तु धनुषी मुष्टौ भल्लाभ्यां श्वेतवाहनः}
{चिच्छेद समरे तूर्णं भूय एव धनञ्जयः}
{अथान्यैर्विशिखैस्तूर्णं ध्वजौ च कनकोज्ज्वलौ}


\twolineshloka
{अथान्ये धनुषी राजन्प्रगृह्य समरे तदा}
{पाण्डवं भृशसङ्क्रुद्धावर्दयामासतुः शरैः}


\twolineshloka
{तयोस्तु भृशसङ्क्रुद्धः शराभ्यां पाण्डुनन्दनः}
{धनुषी चिच्छिदे तूर्णं भूय एव धनञ्जयः}


\twolineshloka
{तथान्यैर्विशिखैस्तूर्णं रुक्मपुङ्खैः शिलाशितैः}
{जघानाश्वांस्तथा सूतौ पार्ष्णी च सपदानुगौ}


\twolineshloka
{ज्येष्ठस्य च शिरः कायात्क्षुरप्रेण न्यकृन्तत}
{स पपात हतः पृथ्व्यां वातरुग्ण इव द्रुमः}


\twolineshloka
{विन्दं तु निहतं दृष्ट्वा ह्यनुविन्दः प्रतापवान्}
{हताश्वं रथमुत्सृज्य गदां गृह्य महाबलः}


\twolineshloka
{अभ्यवर्तत सङ्ग्रामे भ्रातुर्वधमनुस्मरन्}
{गदया रथिनां श्रेष्ठो नृत्यन्निव महारथः}


\twolineshloka
{अनुविन्दस्तु गदया ललाटे मधुसूदनम्}
{स्पष्ट्वा नाकम्पयत्क्रुद्धो मैनाकमिव पर्वतम्}


\twolineshloka
{तस्यार्जुनः शरैः षड्भिर्ग्रीवां पादौ भुजौ शिरः}
{निचकर्त स सञ्छन्नः पपाताद्रिचयो यथा}


\twolineshloka
{ततस्तौ निहतौ दृष्ट्वा तयो राजन्पदानुगाः}
{अभ्यद्रुवन्त सङ्क्रुद्धाः किरन्तः शतशः शरान्}


\twolineshloka
{तानर्जुनः शरैस्तूर्णं निहत्य भरतर्षभ}
{व्यरोचत यथा वह्निर्दावं दग्ध्वा हिमात्यये}


\twolineshloka
{तयोः सेनामतिक्रम्य कृच्छ्रादिव धनञ्जयः}
{विबभौ जलदं हित्वा दिवाकर इवोदितः}


\twolineshloka
{तं दृष्ट्वा कुरवस्त्रस्ताः प्रहृष्टाश्चाभवन्पुनः}
{अभ्यवर्तन्त पार्थं च समन्ताद्भरतर्षभ}


\twolineshloka
{श्रान्तं चैनं समालक्ष्य ज्ञात्वा दूरे च सैन्धवम्}
{सिंहनादेन महता सर्वतः पर्यवारयन्}


\twolineshloka
{तांस्तु दृष्ट्वा सुसंरब्धानुत्स्मयन्पुरुषर्भः}
{शनकैरिव दाशार्हमर्जुनो वाक्यमब्रवीत्}


\twolineshloka
{शरार्दिताश्च ग्लानाश्च हया दूरे च सैन्धवः}
{किमिहानन्तरं कार्यं साधिष्ठं तव रोचते}


\twolineshloka
{ब्रूहि कृष्ण यथातत्त्वं त्वं हि प्राज्ञतमः सदा}
{भवन्नेत्रा रणे शत्रून्विजेष्यन्तीह पाण्डवाः}


\threelineshloka
{मम त्वनन्तरं कृत्यं यद्वै तत्त्वं निबोध मे}
{हयान्विमुच्य हि सुखं विशल्यान्कुरु माधव ॥सञ्जय उवाच}
{}


\threelineshloka
{एवमुक्तस्तु पार्थेन केशवः प्रत्युवाच तम्}
{ममाप्येतन्मतं पार्थ यदितं ते प्रभाषितम् ॥अर्जुन उवाच}
{}


\threelineshloka
{अहमावारयिष्यामि सर्वसैन्यानि केशव}
{त्वमप्यत्र यथान्यायं कुरु कार्यमनन्तरम् ॥सञ्जय उवाच}
{}


\twolineshloka
{सोऽवतीर्य रथोपस्थादसम्भ्रान्तो धनञ्जयः}
{गाण्डीवं धनुरादाय तस्थौ गिरिविवाचलः)}


\twolineshloka
{तमभ्यधावन्क्रोशन्तः क्षत्रिया जयकाङ्क्षिणः}
{इदं छिद्रमिति ज्ञात्वा धरणीस्थं धनञ्जयम्}


\twolineshloka
{तमेकं रथवंशेन महता पर्यवारयन्}
{विकर्षन्तश्च चापानि विसृजन्तश्च सायकान्}


\twolineshloka
{शस्त्राणि च विचित्राणि क्रुद्धास्तत्र व्यदर्शयन्}
{छादयन्तः शरैः पार्थं मेघा इव दिवाकरम्}


\twolineshloka
{अभ्यद्रवन्त वेगेन क्षत्रियाः क्षत्रियर्षभम्}
{नरसिंहं रथोदाराः सिंहं मत्ता इव द्विपाः}


\twolineshloka
{तत्र पार्थस्य भुजयोर्महद्बलमदृश्यत}
{यत्क्रुद्धो बहुलाः सेनाः सर्वतः समवारयत्}


\twolineshloka
{अस्त्रैरस्त्राणि संवार्य द्विषतां सर्वतो विभुः}
{इषुभिर्बुभिस्तूर्णं सर्वानेव समावृणोत्}


\twolineshloka
{तत्रान्तरिक्षे बाणानां प्रगाढानां विशाम्पते}
{सङ्घर्षेण महार्चिष्मान्पावकः समजायत}


\twolineshloka
{तत्रतत्र महेष्वासैः श्वसद्भिः शोणितोक्षितैः}
{हयैर्नागैश्च सम्भिन्नैर्नदद्भिश्चारिकर्शन}


\twolineshloka
{संरब्धैश्चारिभिर्वीरैः प्रार्थयद्भिर्जयं मृधे}
{एकस्थैर्बहुभिः क्रुद्धैरूष्मेव समजायत}


\twolineshloka
{शरोर्मिणं ध्वजावर्तं नागनक्रं दुरत्ययम्}
{पदातिमत्स्यकलिलं शङ्खदुन्दुभिनिःस्वनम्}


\twolineshloka
{असङ्ख्येयमपारं च रजो नीरमतीव च}
{उष्णीषकमठं छत्रपताकाफेनमालिनम्}


\threelineshloka
{रथसागरमक्षोभ्यं मातङ्गाङ्गशिलाचितम्}
{वेलाभूतस्तदा पार्थः पत्रिभिः समवारयत् ॥[धृतराष्ट्र उवाच}
{}


\threelineshloka
{अर्जुने धरणीं प्राप्ते हयहस्ते च केशवे}
{एतदन्तरमासाद्य कथं पार्थो न घातितः ॥सञ्जय उवाच}
{}


\twolineshloka
{सद्यः पार्थिव पार्थेन निरुद्धाः सर्वपार्थिवाः}
{रथस्था धरणीस्थेन वाक्यमच्छान्दसं यथा}


\twolineshloka
{स पार्थः पार्थिवान्सर्वान्भूमिस्थोपि रथस्थितान्}
{एको निवारयामास लोभः सर्वगुणानिव ॥]}


\twolineshloka
{ततो जनार्दनः सङ्ख्ये प्रियं पुरुषसत्तमम्}
{असम्भ्रान्तो महाबाहुरर्जुनं वाक्यमब्रवीत्}


\twolineshloka
{उदपानमिहाश्वानां नालमस्ति रणेऽर्जुन}
{परीप्सन्ते जलं चेमे पेयं न त्ववगाहनम्}


\twolineshloka
{इदमस्तीत्यसम्भ्रान्तो ब्रुवन्नस्त्रेण मेदिनीम्}
{अभिहत्यार्जुनश्चक्रे वाजिपानं सरः शुभम्}


\twolineshloka
{[हंसकारण्डवाकीर्णं चक्रवाकोपशोभितम्}
{सुविस्तीर्णं प्रसन्नाम्भः प्रफुल्लवरपङ्कजम्}


\twolineshloka
{कूर्ममत्स्यगणाकीर्णमगाधमृषिसेवितम्}
{आगच्छन्नारदमुनिर्दर्शनार्थं कृतं क्षणात् ॥]}


\twolineshloka
{शरवंशं शरस्थूणं शराच्छादनमद्भुतम्}
{शरवेश्माकरोत्पार्थस्त्वष्टेवाद्भुतकर्मकृत्}


\twolineshloka
{ततः प्रहस्य गोविन्दः साधुसाध्वित्यथाब्रवीत्}
{शरवेश्मनि पार्थेन कृते तस्मिन्महात्मना}


\chapter{अध्यायः १००}
\twolineshloka
{सञ्जय उवाच}
{}


\twolineshloka
{सलिले जनिते तस्मिन्कौन्तेयेन महात्मना}
{निस्तारिते द्विषत्सैन्ये कृते च शरवेश्मनि}


\twolineshloka
{वासुदेवो रथात्तूर्णमवतीर्य महाद्युतिः}
{मोचयामास तुरगान्विनुन्नान्कङ्कपत्रिभिः}


\twolineshloka
{अदृष्टपूर्वं तद्दृष्ट्वा साधुवादो महानभूत्}
{सिद्धचारणसङ्घानां सैनिकानां च सर्वशः}


\twolineshloka
{पदातिनं तु कौन्तेयं युध्यमानं महारथाः}
{नाशक्नुवन्वारयितुं तदद्भुतमिवाभवत्}


\twolineshloka
{आपतत्सु रथौघेषु प्रभूतगजवाजिषु}
{नासम्भ्रमत्तदा पार्थस्तदस्य पुरुषानति}


\twolineshloka
{व्यसृजन्त शरौघांस्ते पाण्डवं प्रति पार्थिवाः}
{न चाव्यथत धर्मात्मा वासविः परवीरहा}


\twolineshloka
{शतानि शरजालानि गदाः प्रासांश्च वीर्यवान्}
{आगतानग्रसत्पार्थः सरितः सागरो यथा}


\twolineshloka
{अस्त्रवेगेन महता पार्थो बाहुबलेन च}
{सर्वेषां पार्थिवेन्द्राणामग्रसत्ताञ्शरोत्तमान्}


\threelineshloka
{तत्तु पार्थस्य विक्रान्तं वासुदेवस्य चोभयोः}
{अपूजयन्महाराज कौरवा महदद्भुतम् ॥कौरवा ऊचुः}
{}


\twolineshloka
{किमद्भुततमं लोके भविताऽप्यथवा ह्यभूत्}
{यदश्वान्पार्थ गोविन्दौ पाययामासतू रणे}


\threelineshloka
{भयं विपुलमस्मासु तावधत्तां नरोत्तमौ}
{तेजो विदधतुश्चोग्रं यत्सैन्यं तावकं विभो}
{`दधारैको रणे पार्थो वेलेवोद्वृत्तमर्णवम्}


\twolineshloka
{पार्थस्य शरसञ्छन्ने तस्मिन्सद्मनि भारत}
{आकाशमिव सम्प्राप्य विचेरुस्तत्र पक्षिणः}


\twolineshloka
{न चैनं युधि तिष्ठन्तं सैन्येषु तव मारिष}
{अभ्यद्रवत्सुसङ्क्रुद्धः पुमान्कश्चित्तु तावकः}


\twolineshloka
{सर्वे विमनसोऽभूवंस्तव योधा विशाम्पते}
{सम्प्रेक्ष्य तत्र गोविन्दं पाण्डवं च धनञ्जयम्'}


\twolineshloka
{अथ स्मयन्हृषीकेशः स्त्रीमध्य इव भारत}
{अर्जुनेन कृते सङ्ख्ये शरगर्भगृहे तदा}


\twolineshloka
{उपावर्तयदव्यग्रस्तानश्वान्पुष्करेक्षणः}
{मिषतां सर्वसैन्यानां त्वदीयानां विशाम्पते}


\twolineshloka
{`उपावृत्त्योत्थितानश्वान्पाणिभ्यां पुष्करेक्षणः}
{सम्मार्जयन्रणे राजन्पश्यतां सर्वयोधिनाम्'}


\twolineshloka
{तेषां श्रमं च ग्लानिं च वमथुं वेपथुं व्रणान्}
{सर्वं व्यपानुदत्कृष्णः कुशलो ह्यश्वकर्मणि}


\twolineshloka
{शल्यानुद्धृत्य पाणिभ्यां परिमृज्य च तान्हयान्}
{उपावर्त्य यथान्यायं पाययामास वारि सः}


\twolineshloka
{स ताँल्लब्धोदकान्स्नाताञ्जग्धान्नान्विगतक्लमान्}
{योजयामास संहृष्टः पुनरेव रथोत्तमे}


\twolineshloka
{स तं रथवरं शौरिः सर्वशस्त्रभृतां वरः}
{समास्थाय महातेजाः सार्जुनः प्रययौ द्रुतम्}


\twolineshloka
{`योजयित्वा हयांस्तस्य विधिदृष्टेन कर्मणा}
{रणे चचार गोविन्दस्तृणीकृत्य महारथान्'}


\twolineshloka
{रथं रथवरस्याजौ युक्तं लब्धोदकैर्हयैः}
{दृष्ट्वा कुरुबलश्रेष्ठाः पुनर्विमनसोऽभवन्}


\twolineshloka
{विनिःस्वसन्तस्ते राजन्भग्नदंष्ट्रा इवोरगाः}
{घिगहोऽतिगतः पार्थः कृष्णश्चेत्यब्रुवन्पृथक्}


\twolineshloka
{तत्सैन्यं सर्वतो दृष्ट्वा रोमहर्षणमद्भुतम्}
{त्वरध्वमिति चाक्रन्दन्नैतदस्तीति चावुवन्}


\twolineshloka
{सर्वक्षत्रस्य मिषतो रथेनैकेन दंशितौ}
{बालः क्रीडनकेनेव कदर्थीकृत्य नो बलम्}


\twolineshloka
{क्रोशतां यतमानानामसंसक्तौ परन्तपौ}
{दर्शयित्वाऽत्मनो वीर्यं प्रयातौ सर्वराजसु}


\twolineshloka
{`यथा देवासुरे युद्धे तृणीकृत्य च दानवान्}
{इन्द्राविष्णू पुरा राजञ्जम्भस्य वधकाङ्क्षिणौ'}


\twolineshloka
{तौ प्रयातौ पुनर्दृष्ट्वा तदाऽन्ये सैनिकाब्रुवन्}
{त्वरध्वं कुरवः सर्वे वधे कृष्णकिरीटिनोः}


\twolineshloka
{रथं युङ्क्त्वा हि दाशार्हो मिषतां सर्वधन्विनाम्}
{जयद्रथाय यात्येष कदर्थीकृत्य नो रणे}


\twolineshloka
{तत्र केचिन्मिथो राजन्समभाषन्त भूमिपाः}
{अदृष्टपूर्वं सङ्ग्रामे तदृष्ट्वा महदद्भुतम्}


\twolineshloka
{सर्वसैन्यानि राजा च धृतराष्ट्रोऽत्ययं गतः}
{दुर्योधनापराघेन क्षत्रं कृत्स्ना च मेदिनी}


\twolineshloka
{विलयं समनुप्राप्ता तच्च राजा न बुध्यते}
{इत्येवं क्षत्रियास्तत्र ब्रुवन्त्यन्ये च भारत}


\twolineshloka
{सिन्धुराजस्य यत्कृत्यं गतस्य यमसादनम्}
{तत्करोतु वृथादृष्टिर्धार्तराष्ट्रोऽनुपायवित्}


\twolineshloka
{ततः शीघ्रतरं प्रायात्पाण्डवः सैन्धवं प्रति}
{विवर्तमाने तिग्मांशौ हृष्टैः पीतोदकैर्हयैः}


\twolineshloka
{तं प्रयान्तं महाबाहुं सर्वशस्त्रभृतां वरम्}
{नाशक्नुवन्वारयितुं योधाः क्रुद्धमिवान्तकम्}


\twolineshloka
{विद्राव्य तु ततः सैन्यं पाण्डवः शत्रुतापनः}
{यथा मृगगणान्सिंहः सैन्धवार्थे व्यलोडयत्}


\twolineshloka
{गाहमानस्त्वनीकानि तूर्णमश्वानचोदयत्}
{बलाकाभं तु दाशार्हः पाञ्चजन्यं व्यनादयत्}


\twolineshloka
{कौन्तेयेनाग्रतः सृष्टा न्यपतन्पृष्ठतः शराः}
{तूर्णात्तूर्णतरं ह्यश्वाः प्रावहन्वातरंहसः}


\twolineshloka
{ततो नृपतयः क्रुद्धाः परिवव्रुर्धनञ्जयम्}
{क्षत्रिया बहवश्चान्ये जयद्रथवधैषिणम्}


\twolineshloka
{सैन्येषु विप्रयातेषु धिष्ठितं पुरुषर्षभम्}
{दुर्योधनोऽत्वयात्पार्थं त्वरमाणो महाहवे}


\twolineshloka
{वातोद्धूतपताकं तं रथं जलदनिःस्वनम्}
{घोरं कपिध्वजं दृष्ट्वा विषण्णा रथिनोऽभवन्}


\twolineshloka
{दिवाकरेऽथ रजसा सर्वतः संवृते भृशम्}
{शरार्ताश्चरणे योधाः शेकुः कृष्णौ न वीक्षितुम्}


\chapter{अध्यायः १०१}
\twolineshloka
{सञ्जय उवाच}
{}


\twolineshloka
{स्रंसन्त इव मज्जानस्तावकानां भयान्नृप}
{तौ दृष्ट्वा समतिक्रान्तौ वासुदेवधनञ्जयौ}


\twolineshloka
{सर्वे तु प्रतिसंरब्धा हीमन्तः सत्वचोदिताः}
{स्थिरीभूता महात्मानः प्रत्यगच्छन्धनञ्जयम्}


\twolineshloka
{ये गताः पाण्डवं युद्धे रोषामर्षसमन्विताः}
{तेऽद्यापि न निवर्तन्ते सिन्धवः सागरादिव}


\twolineshloka
{असन्तस्तु न्यवर्तन्त वेदेभ्य इव नास्तिकाः}
{नरकं भजमानास्ते प्रत्यपद्यन्त किल्बिषम्}


\twolineshloka
{तावतीत्य रथानीकं विमुक्तौ पुरुषर्षभौ}
{ददृशाते यथा राहोरास्यान्मुक्तौ प्रभाकरौ}


\twolineshloka
{मत्स्याविव महाजालं विदार्य विगतक्लमौ}
{तथा कृष्णावदृश्येतां सेनाजालं विदार्य तत्}


\twolineshloka
{विमुक्तौ शस्त्रसम्बाधाद्रोणानीकात्सुदुर्भिदात्}
{अदृश्येतां महात्मानौ कालसूर्याविवोदितौ}


\twolineshloka
{अस्त्रसम्बाधनिर्मुक्तौ विमुक्तौ शस्त्रसङ्कटात्}
{अदृश्येतां महात्मानौ शत्रुसम्बाधकारिणौ}


\twolineshloka
{विमुक्तौ ज्वलनस्पर्शान्मकरास्याज्झषाविव}
{अक्षोभयेतां सेनां तौ समुद्रं मकराविव}


\twolineshloka
{तावकास्तव पुत्राश्च द्रोणानीकस्थयोस्तयोः}
{नैतौ तरिष्यतो द्रोणमिति चक्रुस्तदा मतिम्}


\twolineshloka
{तौ तु दृष्ट्वा व्यतिक्रान्तौ द्रोणानीकं महाद्युती}
{नाशशंसुर्महाराज सिन्धुराजस्य जीवितम्}


\twolineshloka
{आशा बलवती राजन्सिन्धुराजस्य जीविते}
{द्रोणहार्दिक्ययोः कृष्णौ न मोक्ष्येते इति प्रभो}


\twolineshloka
{तामाशां विफलीकृत्य सन्तीर्णौ तौ परन्तपौ}
{द्रोणानीकं महाराज भोजानीकं च दुस्तरम्}


\twolineshloka
{अथ दृष्ट्वा व्यतिक्रान्तौ ज्वलिताविव पावकौ}
{निराशाः सिन्धुराजस्य जीवितं न शशंसिरे}


\twolineshloka
{मिथश्च समभाषेतामभीतौ भयवर्धनौ}
{जयद्रथवधे वाचस्तास्ताः कृष्णधनञ्जयौ}


\twolineshloka
{असौ मध्ये कृतः षङ्भिर्धार्तराष्ट्रैर्महारथैः}
{चक्षुर्विषयसम्प्राप्तो न मे मोक्ष्यति सैन्धवः}


\twolineshloka
{यद्यस्य समरे गोप्ता शक्रो देवगणैः सह}
{तथाप्येनं निहंस्याव इति कृष्णावभाषताम्}


\twolineshloka
{इति कृष्णौ महाबाहू मिथः कथयतां तदा}
{सिन्धुराजमवेक्षन्तौ त्वत्पुत्रा बहु चुक्रुशुः}


\twolineshloka
{अतीत्य मरुधन्वानं प्रयान्तौ तृषितौ गजौ}
{पीत्वा वारि समाश्वस्तौ तथैवास्तामरिन्दमौ}


\twolineshloka
{व्याघ्रसिंहगजाकीर्णानतिक्रम्य च पर्वतान्}
{वणिजाविव दृश्येतां हीनमृत्यू जरातिगौ}


\twolineshloka
{तथाहि मुखवर्णोऽयमनयोरिति मेनिरे}
{तावका वीक्ष्य मुक्तौ तौ विक्रोशन्ति स्म सर्वशः}


\twolineshloka
{द्रोणादाशीविषाकाराज्ज्वलितादिव पावकात्}
{अन्येभ्यः पार्थिवेभ्यश्च भास्वन्ताविव भास्करौ}


\twolineshloka
{विमुक्तौ सागरप्रख्याद्रोणानीकादर्रिदमौ}
{अदृश्येतां मुदा युक्तौ समुत्तीर्यार्णवं यथा}


\twolineshloka
{अस्त्रौघान्महतो मुक्तौ द्रोणहार्दिक्यरक्षितात्}
{रोचमानावदृश्येतामिन्द्राग्न्योः सदृशौ रणे}


\twolineshloka
{उद्भिन्नरुधिरौ कृष्णौ भारद्वाजस्य सायकैः}
{शितैश्चितौ व्यरोचेतां कर्णिकारैरिवाचलौ}


\twolineshloka
{द्रोणग्राहहूदान्मुक्तौ शक्त्याशीविषसङ्कटात्}
{अयःशरोग्रमकरात्क्षत्रियप्रवराम्भसः}


\twolineshloka
{ज्याघोषतलनिर्हादाद्गदानिस्त्रिंशविद्युतः}
{द्रोणास्त्रमेघान्निर्मुक्तौ सूर्येन्दू तिमिरादिव}


\twolineshloka
{बाहुभ्यामिव सन्तीर्णौ सिन्धुषष्ठाः समुद्रगाः}
{तपान्ते सरितः पूर्णा महाग्राहसमाकुलाः}


\twolineshloka
{`महाशैलोच्चयां घोरां लङ्घयित्वा महानदीम्}
{निस्तीर्णावध्वगौ यद्वत्तद्वत्तौ तारितौ रणे'}


\twolineshloka
{इति कृष्णौ महेष्वासौ प्रशस्तौ लोकविश्रुतौ}
{सर्वभूतान्यमन्यन्त द्रोणास्त्रबलवारणात्}


\twolineshloka
{जयद्रथं समीपस्थमवेक्षन्तौ जिघांसया}
{रुरुं वनान्ते लिप्सन्तौ व्याघ्रवत्तावतिष्ठताम्}


\twolineshloka
{यथा हि मुखवर्णोऽयमनयोरिति मेनिरे}
{तव योधा महाराज हतमेव जयद्रथम्}


\twolineshloka
{लोहिताक्षौ महाबाहू संयुक्तौ कृष्णपाण्डवौ}
{सिन्धुराजमभिप्रेक्ष्य हृष्टौ व्यनदतां मुहुः}


\twolineshloka
{शौरेरभीषुहस्तस्य पार्थस्य च धनुष्मतः}
{तयोरासीत्प्रभा राजन्सूर्यपावकयोरिव}


\twolineshloka
{हर्ष एव तयोरासीद्द्रोणानीकप्रमुक्तयोः}
{समीपे सैन्धवं दृष्ट्वा श्येनयोरामिषं यथा}


\twolineshloka
{तौ तु सैन्धवमालोक्य वर्तमानमिवान्तिके}
{सहसा पेततुःक्रुद्धौ क्षिप्रं श्येनाविवामिपम्}


\twolineshloka
{तौ दृष्ट्वा तु व्यतिक्रान्तौ हृषीकेशधनञ्जयौ}
{सिन्धुराजस्य रक्षार्थं पराक्रान्तः सुतस्तव}


\twolineshloka
{द्रोणेनाबद्धकवचो राजा दुर्योधनस्ततः}
{ययावेकरथेनाजौ हयसंस्करावित्प्रभो}


\twolineshloka
{कृष्णपार्थौ महेष्वासौ व्यतिक्रम्याथ ते सुतः}
{अग्रतः पुण्डरीकाक्षं प्रतीयाय नराधिप}


\twolineshloka
{ततः सर्वेषु सैन्येषु वादित्राणि प्रहृष्टवत्}
{प्रावाद्यन्त व्यतिक्रान्ते तव पुत्रे धनञ्जयम्}


\twolineshloka
{सिंहनादरवाश्चासञ्शङ्खशब्दविमिश्रिताः}
{दृष्ट्वा दुर्योधनं तत्र कृष्णयोः प्रमुखे स्थितम्}


\twolineshloka
{ये च ते सिन्धुराजस्य गोप्तारः पावकोपमाः}
{ते प्राहृष्यन्त समरे दृष्ट्वा पुत्रं तव प्रभो}


\twolineshloka
{दृष्ट्वा दुर्योधनं कृष्णो व्यतिक्रान्तं सहानुगम्}
{अब्रवीदर्जुनं राजन्प्राप्तकालमिदं वचः}


\chapter{अध्यायः १०२}
\twolineshloka
{वासुदेव उवाच}
{}


\twolineshloka
{दुर्योधनमतिक्रान्तमेतं पश्य धनञ्जय}
{आपद्गतमिमं मन्ये नास्त्यस्य सदृशो रथः}


\twolineshloka
{दूरपाती महेष्वासः कृतास्त्रो युद्धदुर्मदः}
{दृढहस्तश्चित्रयोधी धार्तराष्ट्रो महाबलः}


\twolineshloka
{अत्यन्तसुखसंवृद्धो मानितश्च महारथैः}
{कृती च सततं पार्थ नित्यं द्वेष्टि च पाण्डवान्}


\twolineshloka
{तेन युद्धमहं मन्ये प्राप्तकालं तवानघ}
{अत्र नो द्यूतमायत्तं विजयायेतराय वा}


% Check verse!
अत्र क्रोधविषं पार्थ विमुञ्च चिरसम्भृतम् ॥एष मूलमनर्थानां पाण़्वानां महात्मनाम्
\twolineshloka
{सोऽयं प्राप्तस्तवाक्षेपं पश्य साफल्यमात्मनः}
{कथं हि राजा राज्यार्थी त्वया गच्छेत संयुगम्}


\twolineshloka
{दिष्ट्या त्विदानीं सम्प्राप्त एष ते बाणगोचरम्}
{यथाऽयं जीवितं जह्यात्तथा कुरु धनञ्जय}


\twolineshloka
{ऐश्वर्यमदसम्मूढो नैष दुःखमुपेयिवान्}
{न च ते संयुगे वीर्यं जानाति पुरुषर्षभ}


\twolineshloka
{त्वां हि लोकास्त्रयः पार्थ ससुरासुरमानवाः}
{नोत्सहन्ते रणे जेतुं किमुतैकः सुयोधनः}


\twolineshloka
{स दिष्ट्या समनुप्राप्तस्तव पार्थ रथान्तिकम्}
{जह्येनं त्वं महाबाहो यथा वृत्रं पुरन्दरः}


\twolineshloka
{एष ह्यनर्थे सततं पराक्रान्तस्तवानघ}
{निकृत्या धर्मराजं च द्यूते वञ्चितवानयम्}


\twolineshloka
{बहूनि सुनृशंसानि कृतान्येतेन मानद}
{युष्मासु पापमतिना अपापेष्वपि नित्यदा}


\twolineshloka
{तमनार्यं सदा क्रुद्धं पुरुषं कामरूपिणम्}
{आर्यां युद्धे मतिं कृत्वा जहि पार्थाविचारयन्}


\twolineshloka
{निकृत्या राज्यहरणं वनवासं च पाण्डव}
{परिक्लेशं च कृष्णाया हृदि कृत्वा पराक्रम}


\twolineshloka
{दिष्ट्यैष तव बाणानां गोचरे परिवर्तते}
{प्रतिघाताय कार्यस्य दिष्ट्या च यततेऽग्रतः}


\twolineshloka
{दिष्ट्या जानाति सङ्ग्रामे योद्धव्यं हि त्वया सह}
{दिष्ट्या च सफलाः पार्थ सर्वे कामा ह्यकामिताः}


\twolineshloka
{तस्माज्जहि रणे पार्थ धार्तराष्ट्रं कुलाधमम्}
{यथेन्द्रेण हतः पूर्वं जम्भो देवासुरे मृधे}


\threelineshloka
{अस्मिन्हते त्वया सैन्यमनाथं भिद्यतामिदम्}
{वैरस्य दिष्ट्या सहजं मूलं छिन्धि दुरात्मनाम् ॥सञ्जय उवाच}
{}


\twolineshloka
{तं तथेत्यब्रवीत्पार्थः कृत्यरूपमिदं मम}
{सर्वमन्यदनादृत्य गच्छ यत्र सुयोधनः}


\twolineshloka
{येनैतद्दीर्घकालं नो भुक्तं राज्यमकण्टकम्}
{अप्यस्य युधि विक्रम्य च्छिन्द्यां मूर्धानमाहवे}


\twolineshloka
{अपि तस्य ह्यनर्हायाः परिक्लेशस्य माधव}
{कृष्णायाः शक्नुयां गन्तुं पदं केशप्रधर्षणे}


\twolineshloka
{`अप्यहं तानि दुःखानि पूर्ववृत्तानि माधव}
{दुर्योधनं रणे हत्वा प्रतिमोक्ष्ये कथञ्चन'}


\twolineshloka
{इत्येवंवादिनौ कृष्णौ हृष्टौ श्वेतान्हयोत्तमान्}
{प्रेषयामासतुः सङ्ख्ये प्रेप्सन्तौ तं नराधिपम्}


\twolineshloka
{तयोः समीपं सम्प्राप्य पुत्रस्ते भरतर्षभ}
{न चकार भयं प्राप्ते भये महति मारिष}


\twolineshloka
{तदस्य क्षत्रियाः कर्म सर्व एवाभ्यपूजयन्}
{यदर्जुनहृषीकेशौ प्रत्युद्यातौ न्यवारयत्}


\twolineshloka
{ततः सर्वस्य सैन्यस्य तावकस्य विशाम्पते}
{महानादो ह्यभूत्तत्र दृष्ट्वा राजानमाहवे}


\twolineshloka
{तस्मिञ्जनसमुन्नादे प्रवृत्ते भैरवे सति}
{कदर्थीकृत्य ते पुत्रः प्रत्यमित्रमवारयत्}


\twolineshloka
{आवारितस्तु कौन्तेयस्तव पुत्रेण धन्विना}
{संरम्भमगमद्भूयः स च तस्मिन्परन्तपः}


\twolineshloka
{तौ दृष्ट्वा प्रतिसंरब्धौ दुर्योधनधनञ्जयौ}
{अभ्यवैक्षन्त राजानो भीमरूपाः समन्ततः}


\twolineshloka
{दृष्ट्वा तु पार्थं संरब्धं वासुदेवं च मारिष}
{प्रहसन्नेव पुत्रस्ते योद्धुकामः समाह्वयत्}


\twolineshloka
{ततः प्रहृष्टो दाशार्हः पाण्डवश्च धनञ्जयः}
{व्यक्रोशेतां महानादं दध्मतुश्चाम्बुजोत्तमौ}


\twolineshloka
{तौ हृष्टरूपौ सम्प्रेक्ष्य कौरवेयास्तु सर्वशः}
{निराशाः समपद्यन्त पुत्रस्य तव जीविते}


\twolineshloka
{शोकमापुः परं चैव कुरवः सर्व एव ते}
{अमन्यन्त च पुत्रं ते वैश्वानरमुखे हुतम्}


\twolineshloka
{तथा तु दृष्ट्वा योधास्ते प्रहृष्टौ कृष्णपाण्डवौ}
{हतो राजा हतो राजेत्यूचिरे च भयार्दिताः}


\twolineshloka
{जनस्य सन्निनादं तु श्रुत्वा दुर्योधनोऽब्रवीत्}
{व्येतु वो भीरहं कृष्णौ प्रेषयिष्यामि मृत्यवे}


\twolineshloka
{इत्युक्त्वा सैनिकान्सर्वाञ्जयापेक्षी नराधिपः}
{पार्थमाभाष्य संरम्भादिदं वचनमब्रवीत्}


\twolineshloka
{पार्थ यच्छिक्षितं तेऽस्त्रं दिव्यं पार्थिवमेव च}
{तद्दर्शय मयि क्षिप्रं यदि जातोऽसि पाण़्डुना}


\twolineshloka
{यद्बलं तव वीर्यं च केशवस्य तथैव च}
{तत्कुरुष्व मयि क्षिप्रं पश्यामस्तव पौरुषम्}


\twolineshloka
{अस्मत्परोक्षं कर्माणि कृतानि प्रवदन्ति ते}
{स्वामिसत्कारयुक्तानि यानि तानीह दर्शय}


\chapter{अध्यायः १०३}
\twolineshloka
{सञ्जय उवाच}
{}


\twolineshloka
{एवमुक्त्वाऽर्जुनं राजा त्रिभिर्मर्मातिगैः शरैः}
{अभ्यविध्यन्महावेगैश्चतुर्भिश्चतुरो हयान्}


\twolineshloka
{वासुदेवं च दशभिः प्रत्यविध्यत्स्तनान्तरे}
{प्रतोदं चास्य भल्लेन च्छित्त्वा भूमावपातयत्}


\twolineshloka
{तं चतुर्दशभिः पार्थश्चित्रपुङ्खैः शिलाशितैः}
{अविध्यत्तूर्णमव्यग्रस्ते चाभ्रश्यन्त वर्मणि}


\twolineshloka
{तेषां नैष्फल्यमालोक्य पुनर्नव च पञ्च च}
{प्राहिणोन्निशितान्बाणांस्ते चाभ्रश्यन्त वर्मणः}


\twolineshloka
{अष्टाविंशांस्तु तान्बाणानस्तान्विप्रेक्ष्य निष्फलान्}
{अब्रवीत्परवीरघ्नः कृष्णोऽर्जुनमिदं वचः}


\twolineshloka
{अदृष्टपूर्वं पश्यामि शिलानामिव सर्पणम्}
{त्वया सम्प्रेषिताः पार्थ नार्थं कुर्वन्ति पत्रिणः}


\twolineshloka
{कच्चिद्गाण्डीवजः प्राणस्तथैव भरतर्षभ}
{मुष्टिश्च ते यथा पूर्वं भुजयोश्च बलं तव}


\twolineshloka
{न वा कच्चिदयं कालः प्राप्तः स्यादद्य पश्चिमः}
{तव चैवास्य शत्रोश्च तन्ममाचक्ष्व पृच्छतः}


\twolineshloka
{विस्मयो मे महान्पार्थ तव दृष्ट्वा शरानिमान्}
{व्यर्थान्निपतितान्सङ्खे दुर्योधनरथं प्रति}


\threelineshloka
{वज्राशनिसमा घोराः परकायावभेदिनः}
{शराः कुर्वन्ति ते नार्थं पार्थ काऽद्य विडम्बना ॥अर्जुन उवाच}
{}


\twolineshloka
{द्रोणेन नियतं कृष्ण धार्तराष्ट्रे निदर्शितम्}
{अन्ते विहितमस्त्राणामेतत्कवचधारणम्}


\twolineshloka
{अस्मिन्नन्तर्हितं कृष्ण त्रैलोक्यमपि वर्मणि}
{एको द्रोणो हि वैदैतदहं तस्माच्च सत्तमात्}


\twolineshloka
{न शक्यमेतत्कवचं बाणैर्भेत्तुं कथञ्चन}
{अपि वज्रेण गोविन्द स्वयं मघवता युधि}


\twolineshloka
{जानंस्त्वमपि वै कृष्ण मां विमोहयसे कथम्}
{यद्वृत्तं त्रिषु लोकेषु यच्च केशव वर्तते}


\twolineshloka
{यथा भविष्यद्यच्चैव तत्सर्वं विदितं तव}
{न त्विदं वेद वै कश्चिद्यथा त्वं मधुसूदन}


\twolineshloka
{एष दुर्योधनः कृष्ण द्रोणेन विहितामिमाम्}
{तिष्ठत्यभीतवत्सङ्ख्ये बिभ्रत्कवचधारणाम्}


\twolineshloka
{यत्त्वत्र विहितं कार्यं नैष तद्वेत्ति माधव}
{स्त्रीवदेष बिभर्त्येतां युक्तां कवचधारणाम्}


\twolineshloka
{पश्य बाह्वोश्च मे वीर्यं धनुषश्च जनार्दन}
{पराजयिष्ये कौरव्यं कवचेनापि रक्षितम्}


\twolineshloka
{इदमङ्गिरसे प्रादाद्देवेशो वर्म भास्वरम्}
{तस्माद्वृहस्पतिः प्राप ततः प्राप पुरन्दरः}


\fourlineindentedshloka
{पुनर्ददौ सुरपतिर्मह्यं वर्म ससङ्ग्रहम्}
{दैवं यद्यस्य वर्मैतद्ब्रह्मणा वा स्वयं कृतम्}
{नैनं गोप्स्यति दुर्बुद्धिमद्य बाणहतं मया ॥सञ्जय उवाच}
{}


\twolineshloka
{एवमुक्त्वाऽर्जुनो बाणानभिमन्त्र्य व्यकर्षयत्}
{मानवास्त्रेण मानार्हस्तीक्ष्णावरणभेदिना}


\twolineshloka
{विकृष्यमाणांस्तेनैव धनुर्मध्यगताञ्छरान्}
{तानस्यास्त्रेण चिच्छेद द्रौणिः सर्वास्त्रघातिना}


\twolineshloka
{तान्निकृत्तानिषून्दृष्ट्वा दूरतो ब्रह्मवादिना}
{न्यवेदयत्केशवाय विस्मितः श्वेतवाहनः}


\twolineshloka
{नैतदस्त्रं मया शक्यं द्विः प्रयोक्तुं जनार्दन}
{अस्त्रं मामेव हन्याद्धि हन्याच्चापि बलं मम}


\twolineshloka
{ततो दुर्योधनः कृष्णौ नवभिर्नवभिः शरैः}
{अविध्यत रणे राजञ्शरैराशीविषोपमैः}


\threelineshloka
{भूय एवाभ्यवर्षच्च समरे कृष्णपाण्डवौ}
{शरवर्षेण महता ततोऽहृष्यन्त तावकाः}
{चक्रुर्वादित्रनिनदान्सिंहनादरवांस्तथा}


\twolineshloka
{ततः क्रुद्धो रणे पार्थः सृक्विणी परिसंलिहन्}
{नापश्यच्च ततोऽस्याङ्गं यन्न स्याद्वर्मरक्षितम्}


\twolineshloka
{ततोऽस्य निशितैर्बाणैः सुमुक्तैरन्तकोपमैः}
{हयांश्चकार निर्देहानुभौ च पार्ष्णिसारथी}


\twolineshloka
{धनुरस्याच्छिनत्तूर्णं हस्तावापं च वीर्यवान्}
{रथं च शकलीकर्तुं सव्यसाची प्रचक्रमे}


\twolineshloka
{दुर्योधनं च बाणाभ्यां तीक्ष्णाभ्यां विरथीकृतम्}
{आविद्ध्यद्धस्ततलयोरुभयोरर्जुनस्तदा}


\twolineshloka
{[प्रयत्नज्ञो हि कौन्तयो नखमांसान्तरेषुभिः}
{स वेदनाभिराविग्नः पलायनपरायणः ॥]}


\twolineshloka
{तं कृच्छ्रामापदं प्राप्तं दृष्ट्वा परमधन्विनः}
{समापेतुः परीप्सन्तो धनञ्जयशरार्दितम्}


\twolineshloka
{तं रथैर्बहुसाहस्रैः कल्पितैः कुञ्जरैर्हयैः}
{पदात्योघैश्च संरब्धैः परिवव्रुर्धनञ्जयम्}


\twolineshloka
{अथ नार्जुनगोविन्दौ न रथो वा व्यदृश्यत}
{अस्त्रवर्षेण महता शरौघैश्चापि संवृतौ}


\twolineshloka
{ततोऽर्जुनोऽस्त्रवीर्येण निजघ्ने तां वरूथिनीम्}
{तत्र व्यङ्गी कृताः पेतुः शतशोऽथ रथद्विपाः}


\twolineshloka
{ते हता हन्यमानाश्च न्यगृह्णंस्तं रथोत्तमम्}
{स रथः स्तम्भितस्तस्थौ क्रोशमात्रे समन्ततः}


\twolineshloka
{ततोऽर्जुनं वृष्णिवीरस्त्वरितो वाक्यमब्रवीत्}
{धनुर्विष्फारयात्यर्थमहं ध्मास्यामि चाम्बुजम्}


\twolineshloka
{ततो विष्फार्य बलवद्गाण्डीवं जघ्निवान्रिपून्}
{महता शरवर्षेण तलशब्देन चार्जुनः}


\twolineshloka
{पाञ्चजन्यं च बलवान्दध्मौ तारेण केशवः}
{रजसा ध्वस्तपक्ष्मान्तः प्रस्विन्नवदनो भृशम्}


\twolineshloka
{`तेनाच्युतोष्ठपुटपूरितमारुतेनशङ्खान्तरोदरविवृद्धविनिःसृतेन}
{नादेन सासुरवियत्सुरलोकपाल--मुद्विग्नमीश्वर जगत्स्फुटतीव सर्वम्'}


\twolineshloka
{तस्य शङ्खस्य नादेन धनुषो निःस्वनेन च}
{निःसत्वाश्च ससत्वाश्च क्षितौ पेतुस्तदा जनाः}


\twolineshloka
{तैर्विमुक्तो रथो रेजे वाय्वीरित इवाम्बुदः}
{जयद्रथस्य गोप्तारस्ततः क्षुब्धाः सहानुगाः}


\twolineshloka
{ते दृष्ट्वा सहसा पार्थं गोप्तारः सैन्धवस्य तु}
{चक्रुर्नादान्महेष्वासाः कम्पयन्तो वसुन्धराम्}


\twolineshloka
{बाणशब्दरवांश्चोग्रान्विमिश्राञ्शङ्खनिःस्वनैः}
{प्रादुश्चक्रुर्महात्मानः सिंहनादरवानपि}


\twolineshloka
{तं श्रुत्वा निनदं घोरं तावकानां समुत्थितम्}
{प्रदध्मतुः शङ्खवरौ वासुदेवधनञ्जयौ}


\twolineshloka
{तेन शब्देन महता पूरितेयं वसुन्धरा}
{सशैला सार्णवद्वीपा सपाताला विशाम्पते}


\twolineshloka
{स शब्दो भरतश्रेष्ठ व्याप्य सर्वा दिशो दश}
{प्रतिसस्वान तत्रैव कुरुपाण्डवयोर्बले}


\twolineshloka
{तावका रथिनस्तत्र दृष्ट्वा कृष्णधनञ्जयौ}
{सम्भ्रमं परमं प्राप्तास्त्वरमाणा महारथाः}


\twolineshloka
{अथ कृष्णौ महाभागौ तावका वीक्ष्य दंशितौ}
{अभ्यद्रवन्त सङ्क्रुद्धास्तदद्भुतमिवाभवत्}


\chapter{अध्यायः १०४}
\twolineshloka
{सञ्जय उवाच}
{}


\twolineshloka
{तावकास्त्वभिसम्प्रेक्ष्य वृष्ण्यन्धकरूद्वहौ}
{सत्वरास्तौ जिघांसन्तस्तथैव विजयः परान्}


\twolineshloka
{सुवर्णचित्रैर्वैयाघ्रैः स्वनवद्भिर्महारथैः}
{दीपयन्तो दिशः सर्वा ज्वलद्भिरिव पावकैः}


\twolineshloka
{रुक्मपुङ्खैश्च दुष्प्रेक्ष्यैः कार्मुकैः पृथिवीपते}
{कूजद्भिरतुलान्नादन्रोषितैरुरगैरिव}


\twolineshloka
{भूरिश्रवाः शलः कर्णो वृषसेनो जयद्रथः}
{कृपश्च मद्रराजश्च द्रोणिश्च रथिनां वरः}


\twolineshloka
{ते पिबन्त इवाकाशमश्वैरष्टौ महारथाः}
{व्यराजयन्दश दिशो वैयार्घ्रैर्हेमचन्द्रकैः}


\twolineshloka
{ते दंशिताः सुसंरब्धा रथैर्मेघौघनिःस्वनैः}
{समावृण्वन्दश दिशः पार्थस्य निशितैः शरैः}


\twolineshloka
{कौलूतका हयाश्चित्रा वहन्तस्तान्महारथान्}
{व्यशोभन्त तदा शीघ्रा दीपयन्तो दिशो दश}


\twolineshloka
{आजानेयैर्महावेगैर्नानादेशसमुत्थितैः}
{पार्वतीयैर्नदीजैश्च सैन्धवैश्च हयोत्तमैः}


\twolineshloka
{कुरुयोधवरा राजंस्तव पुत्रं परीप्सवः}
{धनञ्जयरथं शीघ्रं सर्वतः समुपाद्रवन्}


\twolineshloka
{ते प्रगृह्य महाशङ्खान्दध्मुः पुरुषसत्तमाः}
{पूरयन्तो दिवं राजन्पृथिवीं च ससागराम्}


\twolineshloka
{तथैव दध्मतुः शङ्खौ वासुदेवधनञ्जयौ}
{प्रवरौ सर्वदेवानां सर्वशङ्खवरौ भुवि}


\twolineshloka
{देवदत्तं च कौन्तेयः पाञ्चजन्यं च केशवः}
{शब्दस्तु देवदत्तस्य धनञ्जयसमीरितः}


\twolineshloka
{पृथिवीं चान्तरिक्षं च दिशश्चैव समावृणोत्}
{तथैव पाञ्चजन्योऽपि वासुदेवसमीरितः}


\twolineshloka
{सर्वशब्दानतिक्रम्य पूरयामास रोदसी}
{तस्मिंस्तथा वर्तमाने दारुणे नादसङ्कुले}


\twolineshloka
{भीरूणां त्रासजनने शूराणां हर्षवर्धने}
{प्रवादितासु भेरीषु झर्झरेष्वानकेषु च}


\twolineshloka
{मृदङ्गेष्वपि राजेन्द्र वाद्यमानेष्वेनकशः}
{महारथसमाख्याता दुर्योधनहितैषिणः}


\twolineshloka
{अमृष्यमाणास्तं शब्दं क्रुद्धाः परमधन्विनः}
{कर्णादयो महेष्वासास्त्वत्सैन्यपरिरक्षिणः}


\twolineshloka
{अमर्षिता महाशङ्खान्दध्मुर्वीरा महारथाः}
{कृते प्रतिकरिष्यन्तः केशवस्यार्जुनस्य च}


\twolineshloka
{बभूव तव तत्सैन्यं शङ्खशब्दे समीरिते}
{उद्विग्नरथनागाश्वमस्वस्थमिव चाभिभों}


\twolineshloka
{तत्प्रहृष्टजनाबाधं भेरीशङ्खनिनादितम्}
{बभूव भृशमुद्विग्नं निर्घातैरिव संवृतम्}


\twolineshloka
{स शब्दः सुमहान्राजन्दिशः सर्वा व्यनादयत्}
{त्रासयामास तत्सैन्यं युगान्त इव सम्भृतः}


\twolineshloka
{ततो दुर्योधनोऽष्टौ च राजानस्ते महारथाः}
{जयद्रथस्य रक्षार्थं पाण्डवं पर्यवारयन्}


\twolineshloka
{ततो द्रौणिस्त्रिसप्तत्या वासुदेवमताडयत्}
{अर्जुनं च त्रिभिर्भल्लैर्ध्वजमश्वांश्च पञ्चभिः}


\twolineshloka
{तमर्जुनः पृषत्कानां शतैः षङ्भिरताडयत्}
{अत्यर्थमिव सङ्क्रुद्धः प्रतिविद्धे जनार्दने}


\twolineshloka
{कर्णं च दशभिर्विद्ध्वा वृषसेनं त्रिभिस्तथा}
{शल्यस्य सशरं चापं मुष्टौ चिच्छेद वीर्यवान्}


\twolineshloka
{गृहीत्वा धनुरन्यत्तु शल्यो विव्याध पाण्डवम्}
{भूरिश्रवास्त्रिभिर्बाणैर्हेमपुङ्खैः शिलाशितैः}


\threelineshloka
{कर्णो द्वात्रिंशता चैव वृषसेनश्च सप्तभिः}
{जयद्रथस्त्रिसप्तत्या कृपश्च दशभिः शरैः}
{मद्राजश्च दशभिर्विव्यधुः फल्गुनं रणे}


\twolineshloka
{ततः शराणां षष्ट्या तु द्रौणिः पार्थमवाकिरत्}
{वासुदेवं च विंशत्या पुनः पार्थं च पञ्चभिः}


\twolineshloka
{प्रहसंस्तु नरव्याघ्रः श्वेताश्वः कृष्णसारथिः}
{प्रत्यविध्यत्स तान्सर्वान्दर्शयन्पाणिलाघवम्}


\twolineshloka
{कर्णं द्वादशभिर्विद्व्वा वृषसेनं त्रिभिः शरैः}
{शल्यस्य सशरं चापं मुष्टिदेशे व्यकृन्तत}


\twolineshloka
{सौमदत्तिं त्रिभिर्विद्व्वा शल्यं च दशभिः शरैः}
{शितैरग्निशिखाकारैर्द्रौणिं विव्याध चाष्टभिः}


\twolineshloka
{गौतमं पञ्चविंशत्या सैन्धवं च शतेन ह}
{पुनर्द्रौणिं च सप्तत्या शराणां सोऽभ्यताडयत्}


\twolineshloka
{भूरिश्रवास्तु सङ्क्रुद्धः प्रतोदं चिच्छिदे हरेः}
{अर्जुनं च त्रिसप्तत्या बाणानामाजघान ह}


\twolineshloka
{ततः शरशतैस्तीक्ष्णैस्तानरीन्श्वेतवाहनः}
{प्रत्यषेधद्द्रुतं क्रुद्धो महावातो घनानिव}


\chapter{अध्यायः १०५}
\twolineshloka
{धृतराष्ट्र उवाच}
{}


\threelineshloka
{ध्वजान्बहुविधाकारान्भ्राजमानानतिश्रिया}
{पार्थानां मामकानां च तान्ममाचक्ष्व सञ्जय ॥सञ्जय उवाच}
{}


\twolineshloka
{ध्वजान्बहुविधाकाराञ्शृणु तेषां महात्मनाम्}
{रूपतो वर्णतश्चैव नामतश्च निबोध मे}


\twolineshloka
{तेषां तु रथमुख्यानां रथेषु विविधा ध्वजाः}
{प्रत्यदृश्यन्त राजेन्द्र ज्वलिता इव पावकाः}


\twolineshloka
{काञ्चनाः काञ्चनापीडाः काञ्चनस्रगलङ्कृताः}
{काञ्चनानीव शृङ्गाणि काञ्चनस्य महागिरेः}


\threelineshloka
{अनेकवर्णा विविधा ध्वजाः परमशोभनाः}
{ते ध्वजाः संवृतास्तेषां पताकाभिः समन्ततः}
{नानावर्णविराणाभिः शुशुभुः सर्वतो वृताः}


\twolineshloka
{पताकाश्च ततस्तास्तु श्वसनेन समीरिताः}
{नृत्यमाना व्यदृश्यन्त रङ्गमध्ये विलासिकाः}


\twolineshloka
{इन्द्रायुधसवर्णाभाः पताका भरतर्षभ}
{दोधूयमाना रथिनां शोभयन्ति महारथान्}


\twolineshloka
{सिंहलाङ्गूल उग्रास्यो ध्वजो वानरलक्षणः}
{धनञ्जयस्य सङ्ग्रामे प्रत्यदृश्यत भारत}


\twolineshloka
{स वानरध्वजो राजन्पताकाभिरलङ्कृतः}
{त्रासयामास तत्सैन्यं ध्वजो गाण्डीवधन्वनः}


\twolineshloka
{तथैव सिंहलाङ्गूलं द्रोणपुत्रस्य भारत}
{ध्वजाघ्रं समपश्याम बालसूर्यसमप्रभम्}


\twolineshloka
{काञ्चनं पवनोद्धूतं शक्रध्वजसमप्रभम्}
{नन्दनं कौरवेन्द्राणां द्रौणेर्लक्ष्म समुच्छ्रितम्}


\twolineshloka
{हस्तिकक्ष्या पुनर्हैमी बभूवाधिरथेर्ध्वजः}
{आहवे खं महाराज ददृशे पूरयन्निव}


\twolineshloka
{पताकी काञ्चनः स्रग्वी ध्वजः कर्णस्य संयुगे}
{नृत्यतीव रथोपस्थे श्वसनेन समीरितः}


\twolineshloka
{आचार्यस्य तु पाण्डूनां ब्राह्मणस्य तपस्विनः}
{गोवृषो गौतमस्यासीत्कृपस्य सुपरिष्कृतः}


\twolineshloka
{स तेन भ्राजते राजन्गोवृषेण महारथः}
{त्रिपुरघ्नरथो यद्वद्गोवृषेण विराजता}


\twolineshloka
{मयूरो वृषसेनस्य काञ्चनो मणिरत्नवान्}
{व्याहरिष्यन्निवातिष्ठत्सेनाग्रमुपशोभयन्}


\twolineshloka
{तेन तस्य रथो भाति मयूरेण महात्मनः}
{यथा स्कन्दस्य राजेन्द्र मयूरेण विराजता}


\twolineshloka
{मद्रराजस्य शल्यस्य ध्वजाग्रेऽग्निशिखामिव}
{सौवर्णीं प्रतिपश्याम सीतामप्रतिमां शुभाम्}


\twolineshloka
{सा सीता भ्राजते तस्य रथमास्थाय मारिष}
{सर्वबीजविरूढेव यथा सीता श्रिया वृता}


\twolineshloka
{वराहः सिन्धुराजस्य राजतोऽभिविराजते}
{ध्वजाग्रे लोहिताङ्गाभो हेमजालपरिष्कृतः}


\twolineshloka
{शुशुभे केतुना तेन राजतेन जयद्रथः}
{यथा देवासुरे युद्धे पुरा पूषा स्म शोभते}


\twolineshloka
{सौमदत्तेः पुनर्यूपो यज्ञशीलस्य धीमतः}
{ध्वजः सूर्य इवाभाति सोमश्चात्र प्रदृश्यते}


\twolineshloka
{स यूपः काञ्चनो राजन्सौमदत्तेर्विराजते}
{राजसूये मखश्रेष्ठे यथा यूपः समुच्छ्रितः}


\twolineshloka
{शलस्य तु महाराज राजतो द्विरदो महान्}
{केतुः काञ्चनचित्राङ्गैर्मयूरैरुपशोभितः}


\twolineshloka
{स केतुः शोभयामास सैन्यं ते भरतर्षभ}
{यथा श्वेतो महानागो देवराजचमूं तथा}


\twolineshloka
{नागो मणिमयो राज्ञो ध्वजः कनकसंवृतः}
{किङ्किणीशतसंहादो भ्राजंश्चित्रो रथोत्तमे}


\twolineshloka
{व्यभ्राजत भृशं राजन्पुत्रस्तव विशाम्पते}
{ध्वजेन महता सङ्ख्ये कुरूणामृषभस्तदा}


\twolineshloka
{नवैते तव वाहिन्यामुच्छ्रिताः परमध्वजाः}
{व्यदीपयंस्ते पृतनां युगान्तादित्यसन्निभाः}


\twolineshloka
{दशमस्त्वर्जुनस्यासीदेक एव महाकपिः}
{अदीप्यतार्जुनो येन हिमवानिव शम्भुना}


\twolineshloka
{ततश्चित्राणि शुभ्राणि सुमहान्ति महारथाः}
{कार्मुकाण्याददुस्तूर्णमर्जुनार्थे परन्तपाः}


\twolineshloka
{तथैव धनुरायच्छत्पार्थः शत्रुविनाशनः}
{गाण्डीवं दिव्यकर्मा तद्राजन्दुर्मन्त्रिते तव}


\twolineshloka
{तवापराधाद्राजानो निहता बहुशो युधि}
{नानादिग्भ्यः समाहूताः सहयाः सरथद्विपाः}


\twolineshloka
{तेषामासीद्व्यातिक्षेपो गजर्तामितरेतरम्}
{दुर्योधनमुखानां च पाण्डूनामृषभस्य च}


\twolineshloka
{तत्राद्भुतं परं चक्रे कौन्तेयः कृष्णसारथिः}
{यदेको बहुभिः सार्घं समागच्छदभीतवत्}


\twolineshloka
{अशोभत महाबाहुर्गाण्डीवं विक्षिपन्धनुः}
{जिगीषुस्तान्नरव्याघ्रो जिघांसुश्च जयद्रथम्}


\twolineshloka
{तत्रार्जुनो नरव्याघ्रः शरैर्मुक्तैः सहस्रशः}
{अदृश्यांस्तावकान्योधान्प्रचक्रे शत्रुतापनः}


\twolineshloka
{ततस्तेऽपि नरव्याघ्राः पार्थं सर्वे महारथाः}
{अदृश्यं समरे चक्रुः सायकौघैः समन्ततः}


\twolineshloka
{संवृते नरसिंहैस्तु कूरूणामृषभेऽर्जुने}
{महानासीत्समुद्भूतस्तस्य सैन्यस्य निःस्वनः}


\chapter{अध्यायः १०६}
\twolineshloka
{धृतराष्ट्र उवाच}
{}


\threelineshloka
{अर्जुने सैन्धवं प्राप्ते भारद्वाजेन संवृताः}
{पाञ्चालाः कुरुभिः सार्धं किमकुर्वत स़ञ्जय ॥सञ्जय उवाच}
{}


\twolineshloka
{अपराह्णे महाराज सङ्ग्रमे रोमहर्षणे}
{पाञ्चालानां कुरूणां च द्रोणद्यूतमवर्तत}


\twolineshloka
{पाञ्चाला हि जिघांसन्तो द्रोणं संहृष्टचेतसः}
{अभ्यमुञ्चन्त गर्जन्तः शरवर्षाणि मारिष}


\twolineshloka
{ततस्तुल तुमुलस्तेषां सङ्ग्रामोऽवर्तताद्भुतः}
{पाञ्चालानां कुरूणां च घोरो देवासुरोपमः}


\twolineshloka
{सर्वे द्रोणरथं प्राप्य पाञ्चालाः पाण्डवैः सह}
{तदनीकं बिभित्सन्तो महास्त्राणि व्यदर्शयन्}


\twolineshloka
{द्रोणस्य रथपर्यन्तं रथिनो रथमास्थिताः}
{कम्पयन्तोऽभ्यवर्तन्त वेगमास्थाय मध्यमम्}


\twolineshloka
{तमभ्ययाद्बृहत्क्षत्रः केकयानां महारथः}
{विमुञ्चन्निशितान्बाणान्महेन्द्राशनिसन्निभान्}


\twolineshloka
{तं तु प्रत्युद्ययौ शीघ्रं क्षेमधूर्तिर्महायशाः}
{विमुञ्चन्निशितान्बाणाञ्शतशोऽथ सहस्रशः}


\twolineshloka
{धृष्टकेतुश्च चेदीनामृषभोऽतिबलोदितः}
{त्वरितोऽभ्यद्रवद्द्रोणं महेन्द्रमिव शम्बरः}


\twolineshloka
{तमापतन्तं सहसा व्यादितास्यमिवान्तकम्}
{वीरधन्वा महेष्वासस्त्वरमाणः समभ्ययात्}


\twolineshloka
{युधिष्ठिरं महाराजं जिगीषुं समवस्थितम्}
{सहानीकं ततो द्रोणो न्यवारयत वीर्यवान्}


\twolineshloka
{नकुलं कुशलं युद्धे पराक्रन्तं पराक्रमी}
{अभ्यगच्छत्समायान्तं विकर्णस्ते सुतः प्रभो}


\twolineshloka
{सहदेवं तथाऽऽयान्तं दुर्मुखः शत्रुकर्शनः}
{शैररनेकसाहस्रैः समवाकिरदाशुगैः}


\twolineshloka
{सात्यकिं तु नरव्याघ्रं व्याघ्रदत्तस्त्ववारयत्}
{शरैः सुनिशितैस्तीक्ष्णैः कम्पयन्वै मुहुर्मुहुः}


\twolineshloka
{द्रौपदेयान्नरव्याघ्रान्मुञ्चतः सायकोत्तमान्}
{संरब्धान्रथिनः श्रेष्ठान्सौमदत्तिरवारयत्}


\twolineshloka
{भीमसेनं तदा क्रुद्धं भीमरूपो भयानकः}
{प्रत्यवारयदायान्तमार्श्यशृङ्गिर्महाबलः}


\twolineshloka
{तयोः समभवद्युद्धं नरराक्षसयोर्मृधे}
{यादृगेव पुरावृत्तं रामरावणयोर्नृप}


\twolineshloka
{ततो युधिष्ठिरो द्रोणं नवत्या नतपर्वणाम्}
{आजघ्ने भरतश्रेष्ठः सर्वमर्मसु भारत}


\twolineshloka
{तं द्रोणः पञ्चविंशत्या निजघान स्तनान्तरे}
{रोषितो भरतश्रेष्ठ कौन्तेयेन यशस्विना}


\twolineshloka
{अदृश्यं समरे चक्रे राजानं सायकोत्तमैः}
{साश्वसूतध्वजं द्रोणः पश्यतां सर्वधन्विनाम्}


\twolineshloka
{ताञ्शरान्द्रोणमुक्तांस्तु शरवर्षेण पाण्डवः}
{अवारयत धर्मात्मा दर्शयन्पाणिलाघवम्}


\twolineshloka
{ततो द्रोणो भृशं क्रुद्धो दर्मराजस्य संयुगे}
{चिच्छेद समरे धन्वी धनुस्तस्य महात्मनः}


\twolineshloka
{अथैनं छिन्नधन्वानं त्वरमाणो महारथः}
{शरैरनेकसाहस्रैः पूरयामास सर्वतः}


\twolineshloka
{अदृश्यं वीक्ष्य राजानं भारद्वाजस्य सायकैः}
{सर्वभूतान्यमन्यन्त हतमेव युधिष्ठिरम्}


\twolineshloka
{केचिच्चैनममन्यन्त तथैव विमुखीकृतम्}
{हृतो राजेति राजेन्द्र ब्राह्मणेन महात्मना}


\threelineshloka
{स कृच्छ्रं परमं प्राप्तो धर्मराजो युधिष्ठिरः}
{त्यक्त्वा तत्कार्मुकं छिन्नं भारद्वाजेन संयुगे}
{आददेऽन्यद्धनुर्दिव्यं भारघ्नं वेगवत्तरम्}


\twolineshloka
{ततस्तान्सायकांस्तत्र दोणनुन्नान्सहस्रशः}
{चिच्छेद समरे वीरस्तदद्भुतमिवाभवत्}


\threelineshloka
{छित्त्वा तु ताञ्शरान्राजक्रोधसंरक्तलोचनः}
{शक्तं जग्राह समरे गिरीणामपि दारिणीम्}
{स्वर्णदण्डां महाघोरामष्टघण्टां भयावहाम्}


\twolineshloka
{समुत्क्षिप्य च तां हृष्टो ननाद बलवद्बली}
{नादेन सर्वभूतानि त्रासयन्निव भारत}


\twolineshloka
{शक्तिं समुद्यतां दृष्ट्वा धर्मराजेन संयुगे}
{स्वस्ति द्रोणाय सहसा सर्वभूतान्यथाब्रुवन्}


\threelineshloka
{सा राजभुजनिर्मुक्ता निर्मुक्तोरगसन्निभा}
{प्रज्वालयन्ती गगनं दिशः सप्रदिशस्तथा}
{द्रोणान्तिकमनुप्राप्ता दीप्तास्या पन्नगी यथा}


\twolineshloka
{तामापतन्तीं सहसा दृष्ट्वा द्रोणो विशाम्पते}
{प्रादुश्चक्रै ततो ब्राह्ममस्त्रमस्त्रविदां वरः}


\twolineshloka
{तदस्त्रं भस्मसात्कृत्वा तां शक्तिं घोरदर्शनाम्}
{जगाम स्यन्दनं तूर्णं पाण्डवस्य यशस्विनः}


\twolineshloka
{ततो युधिष्ठिरो राजा द्रोणास्त्रं तत्समुद्यतम्}
{अशामयन्महाप्राज्ञो ब्रह्मास्त्रेणैव मारिष}


\twolineshloka
{विद्व्वा तं च रणे द्रोणं पञ्चभिर्नतपर्वभिः}
{क्षुरप्रेण सुतीक्ष्णेन चिच्छेदास्य महद्धनुः}


\twolineshloka
{तदपास्य धनुश्छिन्नं द्रोणः क्षत्रियमर्दनः}
{गदां चिक्षेप सहसा धर्मपुत्राय मारिष}


\twolineshloka
{तामापतन्तीं सहसा गदां दृष्ट्वा युधिष्ठिरः}
{गदामेवाग्रहीत्क्रुद्धश्चिक्षेप च परन्तप}


\twolineshloka
{ते गदे सहसा मुक्ते समासाद्य परस्परम्}
{सङ्घर्षात्पावकं मुक्त्वा समेयातां महीतले}


\twolineshloka
{ततो द्रोणो भृशं क्रुद्धो धर्मराजस्य मारिष}
{चतुर्भिर्निशितैस्तीक्ष्णैर्हयाञ्जघ्ने शरोत्तमैः}


\twolineshloka
{चिच्छेदैकेन भल्लेन धनुश्चेन्द्रध्वजोपमम्}
{केतुमेकेन चिच्छेद पाण्डवं चार्दयत्त्रिभिः}


\twolineshloka
{हताश्वात्तुल रथात्तूर्णमवप्लुत्य युधिष्ठिरः}
{तस्थावूर्ध्वभुजो राजा व्यायुधो भरतर्षभ}


\twolineshloka
{विरथं तं समालोक्य व्यायुधं च विशेषतः}
{द्रोणो व्यमोहयच्छत्रून्सर्वसैन्यानि चाभिभो}


\twolineshloka
{मुञ्चंश्चेषुगणांस्तीक्ष्णाँल्लघुहस्तो दृढव्रतः}
{अभिदुद्राव राजानं सिंहो मृगमिवोल्बणः}


\twolineshloka
{तमभिद्रुतमालोक्य द्रोणेनामित्रघातिना}
{हाहेति सुमहाञ्शब्दः पाण्डूनां समजायत}


\twolineshloka
{हृतो राजा हृतो राजा भारद्वाजेन मारिष}
{इत्यांसीत्सुमहाञ्शब्दः पाण्डूसैन्यस्य सर्वतः}


\twolineshloka
{ततस्त्वरितमारुह्य सहदेवरथं नृपः}
{अपायाज्जवनैरश्वैः कुन्तीपुत्रो युधिष्ठिरः}


\chapter{अध्यायः १०७}
\twolineshloka
{सञ्जय उवाच}
{}


\twolineshloka
{बृहत्क्षत्रमथायान्तं केकयं दृढविक्रमम्}
{क्षेमधूर्तिर्महाराज विव्याधोरसि मार्गणैः}


\twolineshloka
{बृहत्क्षत्रस्तु तं राजा नवत्या नतपर्वणाम्}
{आजघ्ने त्वरितो राजन्द्रोणानीकबिभित्सया}


\twolineshloka
{क्षेमधूर्तिस्तु सङ््क्रुद्धः केकयस्य महात्मनः}
{धनुश्चिच्छेद भल्लेन पीतेन निशितेन ह}


\twolineshloka
{अथैनं छिन्नधन्वानं शरेणानतपर्वणा}
{विव्याध समरे तूर्णं प्रवरं सर्वधन्विनाम्}


\twolineshloka
{अथान्यद्धनुरादाय बृहत्क्षत्रो हसन्निव}
{व्यश्वसूतरथं चक्रे क्षमदूर्तिं महारथम्}


\twolineshloka
{ततोऽपरेण भल्लेन पीतेन निशितेन च}
{जहार नृपतेः कायाच्छिरोऽज्वलितकुण़्डलम्}


\twolineshloka
{तच्छिन्नं सहसा तस्य शिरः कुञ्चितमूर्धजम्}
{सकिरीटं महीं प्राप्य बभौ ज्योतिरिवाम्बरात्}


\twolineshloka
{तं निहत्य रणे हृष्टो बृहत्क्षत्रो महारथः}
{सहसाऽभ्यपतत्सैन्यं तावकं पार्थकारणात्}


\twolineshloka
{धृष्टकेतुं तथाऽऽयान्तं द्रोणहेतोः पराक्रमी}
{वीरधन्वा महेष्वासो वारयामास भारत}


\twolineshloka
{तौ परस्परमासाद्य शरदंष्ट्रौ तरस्विनौ}
{शरैरनेकसाहस्रेरन्योन्यमभिजघ्नतुः}


\twolineshloka
{तावुभौ नरशार्दूलौ युयुधाते परस्परम्}
{महावने तीव्रमदौ वारणाविव यूथपौ}


\twolineshloka
{गिरिगह्वरमासाद्य शार्दूलाविव रोषितौ}
{युयुधाते महावीर्यौ परस्परजिघांसया}


\twolineshloka
{तद्युद्धमासीत्तुमुलं प्रेक्षणीयं विशाम्पते}
{सिद्धचारणसङ्घानां विस्मयाद्भुतदर्शनम्}


\twolineshloka
{वीरधन्वा ततः क्रुद्धो धृष्टकेतोः शरासनम्}
{द्विधा चिच्छेद भल्लेन प्रहसन्निव भारत}


\twolineshloka
{तदुत्सृज्य धनुच्छिन्नं चेदिराजो महारथः}
{शक्तिं जग्राह विपुलां हेमदण्डामयस्मयीम्}


\twolineshloka
{तां तु शक्तिं महावीर्यां दोर्भ्यामायस्य भारत}
{चिक्षेप सहसा यत्तो वीरधन्वरथं प्रति}


\twolineshloka
{तया तु वीरघातिन्या शक्त्या त्वभिहतो भृशम्}
{निर्भिन्नहृदयस्तूर्णं निपपात रथान्महीम्}


\twolineshloka
{तस्मिन्विनिहते वीरे त्रैगर्तानां महारथे}
{बलं तेऽभज्यत विभो पाण्डवेयैः समन्ततः}


\twolineshloka
{सहदेवे ततः षष्टिं सायकान्दुर्मुखोऽक्षिपत्}
{ननाद च महानादं तर्जयन्पाण्डवं रणे}


\twolineshloka
{माद्रेयस्तु ततः क्रुद्धो दुर्मुखं च शितैः शरैः}
{भ्राता भ्रातरमायत्तो विव्याध प्रहसन्निव}


\twolineshloka
{तं रणे रभसं दृष्ट्वा सहदेवं महाबलम्}
{दुर्मुखो नवभिर्बाणैस्ताडयामास भारत}


\twolineshloka
{दुर्मुखस्य तु भल्लेन च्छित्त्वा केतुं महाबलः}
{जघान चतुरो वाहांश्चतुर्भिर्निशितैः शरैः}


\twolineshloka
{अथापरेण भल्लेन पीतेन निशितेन ह}
{चिच्छेद सारथेः कायाच्छिरो ज्वलितकुण्डलम्}


\twolineshloka
{क्षुरप्रेण च तीक्ष्णेन कौरव्यस्य महद्धनुः}
{सहदेवो रणे छित्त्वा तं च विव्याध पञ्चभिः}


\twolineshloka
{हताश्वं तु रथं त्यक्त्वा दुर्मुखो विमनास्तदा}
{आरुरोह रथं राजन्निरमित्रस्य भारत}


\twolineshloka
{सहदेवस्ततः क्रुद्धो निरमित्रं महाहवे}
{जघान पृथुधारेण भल्लेन परवीरहा}


\twolineshloka
{स पपात रथोपस्थान्निरमित्रो जनेश्वरः}
{त्रिगर्तराजस्य सुतो व्यथयंस्तव वाहिनीम्}


\twolineshloka
{तं तु हत्वा महाबाहुः सहदेवो व्यरोचत}
{यथा दाशरथी रामः खरं हत्वा महाबलम्}


\twolineshloka
{हाहाकारो महानासीत्त्रिगर्तानां जनेश्वर}
{राजपुत्रं हतं दृष्ट्वा निरमित्रं महारथम्}


\twolineshloka
{नकुलस्ते सुतं राजन्विकर्णं पृथुलोचनम्}
{मुहूर्ताज्जितवाँल्लोके तदद्भुतमिवाभवत्}


\twolineshloka
{सात्यकिं व्याघ्रदत्तस्तु शरैः सन्नतपर्वभिः}
{चक्रेऽदृश्यं साश्वसूतं सध्वजं पृतनान्तरे}


\twolineshloka
{तान्निवार्य शराञ्शूरः शैनेयः कृतहस्तवत्}
{साश्वसूतध्वजं बाणैर्व्याघ्रदत्तमपातयत्}


\twolineshloka
{कुमारे निहते तस्मिन्मागधस्य सुते प्रभो}
{मागधाः सर्वतो यत्ता युयुधानमुपाद्रवन्}


\twolineshloka
{विसृजन्तः शरांश्चैव तोमरांश्च सहस्रशः}
{मिण्डिपालांस्तथा प्रासान्मुद्रान्मुसलानपि ॥अयोधयन्रणे शूराः सात्वतं युद्धदुर्मदम्}


\twolineshloka
{तांस्तु सर्वान्स बलवान्सात्यकिर्युद्धदुर्मदः}
{नातिकृच्छ्राद्धसन्नेव विजिग्ये पुरुषर्षभः}


\twolineshloka
{मागधान्द्रवतो दृष्ट्वा हतशेषान्समन्ततः}
{बलं तेऽभज्यत विभो युयुधानशरार्दितम्}


\twolineshloka
{नाशयित्वा रणे सैन्यं त्वदीयं तु समन्ततः}
{विधुन्वानो धनुःश्रेष्ठं व्यभ्राजत महायशाः}


\twolineshloka
{भज्यमानं बलं राजन्सात्वतेन महात्मना}
{नाभ्यवर्तत युद्धाय त्रासितं दीर्घबाहुना}


\twolineshloka
{ततो द्रोणो भृशं क्रुद्धः सहसोद्वृत्य चक्षुषी}
{सात्यकिं सत्यकर्माणं स्वयमेवाभिदुद्रुवे}


\chapter{अध्यायः १०८}
\twolineshloka
{सञ्जय उवाच}
{}


\twolineshloka
{द्रौपदेयान्महेष्वासान्सौमदत्तिर्महायशाः}
{एकैकं पञ्चभिर्विद्व्वा पुनर्विव्याध सप्तभिः}


\twolineshloka
{ते पीडिता भृशं तेन रौद्रेण सहसा विभो}
{प्रमूढा नैव विविदुर्मृधे कृत्यं स्म किंचन}


\twolineshloka
{नाकुलिश्च शतानीकः सौमदत्तिं नरर्षभम्}
{द्वाभ्यां विद्व्वाऽनदद्वृष्टः शराभ्यां शत्रुकर्शनः}


\twolineshloka
{तथेतरे रणे यत्तास्त्रिभिस्त्रिभिरजिह्मगैः}
{विव्यधुः समरे तूर्णं सौमदत्तिममर्षणम्}


\twolineshloka
{स तान्प्रति महाराज पञ्च चिक्षेप सायकान्}
{एकैकं हृदि चाजघ्ने एकैकेन महायशाः}


\twolineshloka
{ततस्ते भ्रातरः पञ्च शरैर्विद्धा महात्मना}
{परिवार्य रणे वीरं विव्यधुः सायकैर्भृशम्}


\twolineshloka
{अर्जुनिस्तु हयांस्तस्य चतुर्भिर्निशितैः शरैः}
{प्रेषयामास सङ्क्रुद्धो यमस्य सदनं प्रति}


\twolineshloka
{भैमसेनिर्धनुश्छित्त्वा सौमदत्तेर्महात्मनः}
{ननाद बलवन्नादं विव्याध च शितैः शरैः}


\twolineshloka
{यौधिष्ठिरिर्ध्वजं तस्य च्छित्त्वा भूमावपातयत्}
{नाकुलिश्चाथ यन्तारं रथनीडादपाहरत्}


\twolineshloka
{साहदेविस्तु तं ज्ञात्वा भ्रातृभिर्विमुखीकृतम्}
{क्षुरप्रेण शिरो राजन्निचकर्त महात्मनः}


\twolineshloka
{तच्छिरो न्यपतद्भूतौ तपनीयविभूषितम्}
{अभासयद्रुणोद्देशं बालसूर्यसमप्रभम्}


\twolineshloka
{सौमदत्तेः शिरो दृष्ट्वा निपतत्तन्महात्मनः}
{वित्रस्तास्तावका राजन्प्रदुद्रुवुरनेकधा}


\twolineshloka
{अलम्बुसस्तु समरे भीमसेनं महाबलम्}
{योधयामस सङ्क्रुद्धो लक्ष्मणं रावणिर्यथा}


\twolineshloka
{सम्प्रयुद्धौ रमे दृष्ट्वा तावुभौ नरराक्षसौ}
{विस्मयः सर्वभूतानां भयं चासीत्सुदारुणम्}


\twolineshloka
{आर्श्यशृङ्गिं ततो भीमो नवभिर्निशितैः शरैः}
{विव्याध प्रहसन्राजन्राक्षसेन्द्रममर्षणम्}


\twolineshloka
{तद्रक्षः समरे विद्धं कृत्वा नादं भयावहम्}
{अभ्यद्रवत्ततो भीमं ये च तस्य पदानुगाः}


\threelineshloka
{स भीमं पञ्छभिर्विद्व्वा शरैः सन्नतपर्वभिः}
{भैमान्परिजघानाशु रथांस्त्रिशतमाहवे}
{पुनश्चतुःशतान्हत्वा भीमं विव्याध पत्रिणा}


\twolineshloka
{सोऽतिविद्धस्तथा भीमो राक्षसेन महाबलः}
{निषसाद रथोपस्थे मूर्च्छयाऽभिपरिप्लुतः}


\threelineshloka
{प्रतिलभ्य ततः संज्ञां मारुतिः क्रोधमूर्च्छितः}
{विकृष्य कार्मुकं घोरं भारसाधनमुत्तमम्}
{अलम्बुसं शरैस्तीक्ष्णैरर्दयामास सर्वतः}


\twolineshloka
{स विद्धो बहुभिर्बाणैर्नीलाञ्जनचयोपमः}
{शुशुभे सर्वतो राजन्प्रफुल्ल इव किंशुकः}


\threelineshloka
{स वध्यमानः समरे भीमचापच्युतैः शरैः}
{स्मरन्भ्रातृवधं चैव पाण्डवेन महात्मना}
{घोरं रूपमथो कृत्वा भीमसेनमभाषत}


\twolineshloka
{तिष्ठेदानीं रणे पार्थ पश्य मेऽद्य पराक्रमम् ॥बाको नाम सुदुर्बुद्धे राक्षसप्रवरो बली}
{}


\twolineshloka
{परोक्षं मम तद्वृत्तं यद्धाता मे हतस्त्वया ॥सञ्जय उवाच}
{}


\twolineshloka
{एवमुक्त्वा ततो भीममन्तर्धानं गतस्तदा}
{महता शरवर्षेण भृशं तं समवाकिरत्}


\twolineshloka
{भीमस्तु समरे राजन्नदृश्ये राक्षसे तदा}
{आकाशं पूरयामास शरैः सन्नतपर्वभिः}


\twolineshloka
{स वध्यमानो भीमेन निमेषाद्रथमास्थितः}
{जगाम धरणीं चैव क्षुद्रः खं सहसाऽगमत्}


\twolineshloka
{उच्चावचानि रूपाणि चकार सुबहूनि च}
{अणुर्बृहत्पुनः स्थूलो नादान्मुञ्चन्निवाम्बुदः}


\twolineshloka
{[निपेतुर्गगनाचत्चैव शरधाराः सहस्रशः}
{शक्तयः कणपाः प्रासाः शूलपट्टसतोमराः}


\twolineshloka
{शतघ्न्यः परिघाश्चैव भिण़्डिपालाः परश्वथाः}
{शिलाः खङ्घा गुडाश्चैव ऋष्टीर्वज्राणि चैव ह}


\twolineshloka
{सा राक्षसविसृष्टा तु शस्त्रवृष्टिः सुदारुणा}
{जघान पाण्डुपुत्रस्य सैनिकान्रणमूर्धनि ॥]}


\twolineshloka
{उच्चावचास्तथा वाचो व्याजहार समन्ततः}
{तेन पाण्डवसैन्यानां सूदिता युधि वारणाः}


\twolineshloka
{हयाश्च बहवो राजन्पत्तयश्च तथा पुनः}
{रथेभ्यो रथिनः पेतुस्तस्य नुन्नाः स्म सायकैः}


\twolineshloka
{शोणितोदां रथावर्तां हस्तिग्राहसमाकुलाम्}
{छत्रहंसां कर्दमिनीं बाहुपन्नगसङ्कुलाम्}


\twolineshloka
{नदीं प्रवर्तयामास रक्षोगणसमाकुलाम्}
{वहन्तीं बहुधा राजंश्चेदिपाञ्चालसृञ्जयान्}


\twolineshloka
{तं तथा समरे राजन्विचरन्तमभीतवत्}
{पाण्डवा भृशसंविग्नाः प्रापश्यंस्तस्य विक्रमम्}


\twolineshloka
{तावकानां तु सैन्यानां प्रहर्षः समजायत}
{वादित्रनिनदश्चोग्रः सुमहान्रोमहर्षणः}


\twolineshloka
{तं श्रुत्वा निनदं घोरं तव सैन्यस्य पाण्डवः}
{नामृष्यत यथा नागस्तलशब्दं समीरितम्}


\twolineshloka
{ततः क्रोधाभिताम्राक्षो निर्दहन्निव पावकः}
{सन्दधे त्वाष्ट्रमस्त्रं स स्वयं त्वष्टेव मारुतिः}


\twolineshloka
{ततः शरसहस्राणि प्रादुरासन्समन्ततः}
{तैः शरैस्तव सैन्यस्य विद्रवः सुमहानभूत्}


\twolineshloka
{तदस्त्रं प्रेरितं तेन भीमसेनेन संयुगे}
{राक्षसस्य महामायां हत्वा राक्षसमार्दयत्}


\twolineshloka
{स वध्यमानो बहुधा भीमसेनेन राक्षसः}
{सन्त्यज्य समरे भीमं द्रोणानीकमुपाद्रवत्}


\twolineshloka
{तस्मिंस्तु निर्जिते राजन्राक्षसेन्द्रे महात्मना}
{अनादयन्सिंहनादैः पाण्डवाः सर्वतो दिशम्}


\twolineshloka
{अपूजयन्मारुतिं च संहृष्टास्ते महाबलम्}
{प्रहादमिव दैतेया यथा शक्रं मरुद्गणाः}


\chapter{अध्यायः १०९}
\twolineshloka
{सञ्जय उवाच}
{}


\twolineshloka
{`किरन्तं शरवर्षाणि रोषाद्दोणं महाहवे}
{वित्रासयन्तं तां सेनां कौन्तेयानां महीपते}


\twolineshloka
{दृष्ट्वा ततो महेष्वासो निघ्नन्तं च रथान्भृशम्}
{घटोत्कचो महाबाहू रणायाभिजगाम ह}


\twolineshloka
{पिशाचवदनैर्युक्तं रथं काञ्चनभूषितम्}
{समास्थाय महाराज नानाप्रहरणैर्युतः}


\twolineshloka
{दंशितस्तपनीयेन कवचेन सुवर्चसा}
{भूषणैराचिताङ्गश्च नदन्निव च तोयदः}


\twolineshloka
{हैडिम्बेयः सुसङ्क्रुद्धो द्रोणमभ्यद्रवद्बली}
{तमभ्यधावदायान्तं क्रद्धरूपलम्बुसः}


\twolineshloka
{ऋक्षचर्मपरिक्षिप्तं रथमास्थाय दंशितः}
{रक्तोष्ठः सधनुष्पाणिः प्रांशुः कल्प इव स्थितः}


\twolineshloka
{क्षिपञ्छतघ्नीर्विपुला मुसलोपमतोमरान्}
{मुसण्ठीर्बहुलाश्चैव त्रिशूलानपि पट्टसान्}


\twolineshloka
{कर्पराञ्छतधारांश्च पिनाकान्विविधांस्तथा}
{चक्राणि च क्षुरप्राणि क्षेपणीश्च कटङ्कटान्}


\twolineshloka
{नारायान्विविधानस्यन्सकङ्कोलूकवायसान्}
{चिक्षेप धनुरादाय निनदन्भैरवान्रवान्}


\twolineshloka
{तं रौद्रं क्रूरमायान्तं दृष्ट्वा कालमिवागतम्}
{प्राद्रवद्भयसंविग्ना राजन्पाण्डववाहिनी}


\twolineshloka
{सात्यकिस्तु रथव्याघ्रो दृष्ट्वा तं राक्षसं युधि}
{अभ्ययादमरप्रख्यो भ्रामयित्वा महद्धनुः}


\twolineshloka
{अभ्यद्रवच्च तद्रक्षस्तिष्ठतिष्ठेति चाब्रवीत्}
{अलम्बुसं राक्षसेन्दरं सोऽस्त्रवर्षैरवाकिरत्}


\twolineshloka
{ततः पाण्डवसैन्यानि विद्रुतान्यथ भारत}
{निरीक्ष्याभ्यद्रवत्तूर्णं त्वरमाणो घटोत्कचः}


\twolineshloka
{चिक्षेप च गदाशक्तीस्तोमरानथ पट्टसान्}
{हेमचित्रत्सरूनुग्रान्खङ्गानाकाशसप्रभान्}


\twolineshloka
{अन्योन्यमरादालोक्य राक्षसौ तौ महाबलौ}
{भैरवं नदतुर्नादान्सतोयाविव तोयदौ}


\twolineshloka
{ततः प्रववृते युद्धं घोरं राक्षससिंहयोः}
{यादृगेव पुरा वृत्तं रामरावणयोर्मृधे}


\twolineshloka
{स शक्तीश्च पिनाकांश्च वज्रान्खङ्गान्परश्वथान्}
{अन्योन्यमभिसङ्क्रुद्धौ तदा व्यसृजतामुभौ}


\twolineshloka
{आकृष्यमाणे धनुषि तयोर्बाहुबलेन च}
{यन्त्रेणेव तदा राजन्भृशं नादान्प्रचक्रतुः}


\twolineshloka
{अलम्बुसस्ततश्चक्रं कृतान्तज्वलनप्रभम्}
{घटोत्कचाय चिक्षेप यत्नमास्थाय वीर्यवान्}


\twolineshloka
{तद्भैमसेनिः सम्प्रेक्ष्य चक्रं वेगवदन्तरे}
{गदया ताडयामास तद्दीर्णं शतधाऽभवत्}


\twolineshloka
{ततोऽग्निचूर्णैः सहसा चक्रघातविनिःसृतैः}
{दंशकैरिव सा सेना पतद्भिर्भृशसंकुला}


\threelineshloka
{ततः प्रतिहते चक्रे स वीरो रोषसंकुलः}
{प्राहिणोत्तरसा शूलं शक्तीर्दशशतं तदा}
{}


\twolineshloka
{ज्वलन्तीश्च किरन्तीश्च ज्वालामालाः समन्ततः}
{युगान्तोल्कानिभास्तीक्ष्णा हेमदण्डा महास्वनाः}


\twolineshloka
{ताश्चापतन्तीः सम्प्रेक्ष्य राक्षसस्य घटोत्कचः}
{अर्धचन्द्रैः प्रचिच्छेद नाराचैः कङ्कपत्रिभिः}


\twolineshloka
{ततो रोषपरीताङ्गः प्रमुमोच स राक्षसः}
{शरवर्षं महाघोरं घटोत्कचरथं प्रति'}


\twolineshloka
{तयोः प्रतिभयं युद्धमासीद्राक्षससिंहयोः}
{कुर्वतोर्विविधा मायाः शक्रशम्बरयोरिव}


% Check verse!
अलम्बुसो भृशं क्रुद्धो घटोत्कचमताडयत्
\twolineshloka
{तयोर्युद्धं समभवद्रक्षोग्रामणिमुख्ययोः}
{यादृगेव पुरा वृत्तं रामरावणयोः प्रभो}


\twolineshloka
{घटोत्कचस्तु विंशत्या नाराचानां स्तनान्तरे}
{अलम्बुसमथो विद्धा सिंहवद्व्यनदन्मुहुः}


\twolineshloka
{तथैवालम्बुसो राजन्हैडिम्बिं युद्धदुर्मदम्}
{विद्व्वा विद्वृऽनदद्वृष्टः पूरयन्खं समन्ततः}


\twolineshloka
{तथा तौ भृशसङ्क्रुद्धौ राक्षसेन्द्रौ महाबलौ}
{निर्विशेषमयुध्येतां मायाभिरितरेतरम्}


\twolineshloka
{मायाशतसृजौ नित्यं मोहयन्तौ परस्परम्}
{मायायुद्धेषु कुशलौ मायायुद्धमयुध्यताम्}


\twolineshloka
{यांयां घटोत्कचो युद्धे मायां दर्शयते नृप}
{तां तामलम्बुसो राजन्माययैव निजघ्निवान्}


\twolineshloka
{तं तथा युध्यमानं तु मायायुद्धविशारदम्}
{अलम्बुसं राक्षसेन्द्रं दृष्ट्वाऽक्रुध्यन्त पाण्डवाः}


\twolineshloka
{त एनं भृशसंविग्नाः सर्वतः प्रवरा रथैः}
{अभ्यद्रवन्त सङ्कुद्धा भीमसेनादयो नृप}


\twolineshloka
{त एनं कोष्ठकीकृत्य रथवंशेन मारिष}
{सर्वतो व्यकिरन्बाणैरुल्काभिरिव कुञ्जरम्}


\twolineshloka
{स तेषामस्त्रवेगं तं प्रतिहत्यास्त्रमायया}
{तस्माद्रथव्रजान्मुक्तो वनदाहादिव द्विपः}


\twolineshloka
{स विष्फार्य अनुर्धोरमिन्द्राशनिसमस्वनम्}
{मारुतिं पञ्चविंशत्या भैमसेनिं च पञ्चभिः}


\threelineshloka
{युधिष्ठिरं त्रिभिर्विद्धा सहदेवं च सप्तभिः}
{नकुलं च त्रिसप्तत्या द्रोपदेयांश्च मारिष}
{पञ्चभिःपञ्चभिर्विद्ध्वा घोरं नादं ननाद ह}


\threelineshloka
{तं भीमसेनो नवभिः सहदेवस्तु पञ्चभिः}
{युधिष्ठिरः शतेनैव राक्षसं प्रत्यविध्यत}
{नकुलस्तु चतुःषष्ट्या द्रौपदेयास्त्रिभिस्त्रिभिः}


\twolineshloka
{हैडिम्बो राक्षसं विद्ध्वा युद्धे पञ्चाशता शरैः}
{पुनर्विव्याध सप्तत्या ननाद च महाबलः}


\twolineshloka
{तस्य नादेन महता कम्पितेयं वसुन्धरा}
{सपर्वतवना राजन्सपादपजलाशया}


\twolineshloka
{सोऽतिविद्धो महेष्वासैः सर्वतस्तैर्महारथैः}
{प्रतिविव्याध तान्सर्वान्पञ्चभिः पञ्चभिः शरैः}


\twolineshloka
{तं क्रुद्धं राक्षसं युद्धे प्रतिक्रुद्धस्तु राक्षसः}
{हैडिम्बो भरतश्रेष्ठ शरैर्विव्याध सप्तभिः}


\twolineshloka
{सोऽतिविद्धो बलवता राक्षसेन्द्रो महाबलः}
{व्यसृजत्सायकांस्तूर्णं रुक्मपुङ्खाञ्शिलाशितान्}


\twolineshloka
{ते शऱा नतपर्वाणो भैमसेनिविनिर्भिदः}
{रुषिताः पन्नगा यद्वद्योधमुख्यानुपागमन्}


\twolineshloka
{ततस्ते पाण्डवा राजन्समन्तान्निशिताञ्शरान्}
{प्रेषयामासुरुद्विग्ना हैडिम्बश्च घटोत्कचः}


\twolineshloka
{स विध्यमानः समरे पाण्डवैर्जितकाशिभिः}
{`*नाभ्यपद्यत कर्तव्यमार्श्यशृङ्गिर्महाबलः}


\twolineshloka
{घटोत्कचं महाराज शरवर्षैरवाकिरत्}
{दग्धाद्रिकूटसदृशं तमञ्जनचयोपरमम्}


\twolineshloka
{घटोत्कचोऽप्यसम्भ्रान्तः शरवर्षं महत्तरम्}
{अलम्बुसवधप्रेप्सुर्मुमोचाग्निरिव ज्वलन्}


\twolineshloka
{अलम्बुसवधाच्चोग्राद्धटोत्कचवधादपि}
{शराः प्रादुर्भवन्ति स्म द्विरेफा इव शाखिनः}


\twolineshloka
{अभ्रच्छायेव रचिता बाणैस्तत्र नरेश्वर}
{न स्म विज्ञायते किञ्चिदन्धकारे कृते शरैः}


\twolineshloka
{तत आकर्णमुक्तेन भल्लेन च घटोत्कचः}
{अलम्बुसस्य चिच्छेद शिरो यन्तुर्महाबलः}


\twolineshloka
{ततोऽपरैर्वेगवद्भिः क्षुरैस्तस्य घटोत्कचः}
{अक्षमीषां युगं चैव चिच्छेद युधि ताडयन्}


\twolineshloka
{अवस्कन्द्य रथात्तूर्णं कैर्मीरः क्रोधमूर्च्छितः}
{तस्मिन्मायामयं घोरमस्त्रवर्षं ववर्ष ह}


\twolineshloka
{घटोत्कचोऽप्याशु रथात्प्रस्कन्द्य स तमेव च}
{मायास्त्रेणैव मायास्त्रं व्यधमत्समरे रिपोः}


\twolineshloka
{हैडिम्बेनार्द्यमानस्तु युधि सोऽलम्बुसोऽद्रवत्}
{अन्तर्हितो महाराज घटोत्कचमयोधयत्}


\twolineshloka
{अन्तर्धानगतं दृष्ट्वा तत्रतत्र घटोत्कचः}
{गदया ताडयामास वेगवत्या महाबलः}


\twolineshloka
{उत्पपात ततो व्योम्नि प्रहारपरिपीडितः}
{अलम्बुसो राक्षसेन्द्रः सहसा पक्षिराडिव}


\twolineshloka
{घटोत्कचोऽप्यसम्भ्रान्तः खङ्गपाणिरथोत्पतत्}
{ततो वेगेन महता विवर्षिषुरिवाम्बुदः}


\twolineshloka
{तमापतन्तं सम्प्रेक्ष्य कैर्मीरी राक्षसोत्तमः}
{अभिदुद्राव वेगेन सिंहः सिंहमिव स्थितम्}


\twolineshloka
{दक्षिणेनासिमुद्यम्य वक्षः प्रच्छाद्य वर्मणा}
{अभिदुद्राव वेगेन वेगवन्तं घटोत्कचः}


\twolineshloka
{तावुभौ वेगसंरब्धावलम्बुसघटोत्कचौ}
{अन्योन्यस्य तथैवोरू समाजघ्नतुरञ्जसा}


\twolineshloka
{अन्योन्यस्याभिघातेन तयो राक्षससिंहयोः}
{शैलेनाभिहतस्येव शैलस्याभून्महास्वनः}


\twolineshloka
{ततोऽपसृत्य सहसा पुनरापेततुर्भृशम्}
{चरन्तावसिमार्गांस्तान्विविधान्राक्षसोत्तमौ}


\twolineshloka
{तयोर्गात्रेषु पतितावसी भिन्नौ निपेततुः}
{वेगोत्सृष्टे मघवता वज्रे शैलतटेष्विव}


\twolineshloka
{ततः सैन्यानि ददृशुस्तद्युद्धमतिदारुणम्}
{युद्धं तयो राक्षसयोरामिषे श्येनयोरिव}


\twolineshloka
{ततो लोहितरक्ताक्षावुभौ तौ राक्षसोत्तमौ}
{तथैक्ष्येतां तु शार्दूलौ सन्ध्यारक्ताविवाम्बुदौ}


\twolineshloka
{चक्राते श्येनवच्चैव मण्डलानि सहस्रशः}
{उभौ निस्त्रिंशहस्तौ तौ सपक्षाविव पक्षिणौ}


\twolineshloka
{भ्रामयित्वा तु तं खङ्गं पाण्डोः किर्मीरनन्दनः}
{चिक्षेपास्य शिरो हर्तुं स च तस्य घटोत्कचः}


\twolineshloka
{तावसी युगपद्दीप्तौ समेत्य विपुलौ भुवि}
{पतितौ तौ तु बाहुभ्यां राक्षसौ समसज्जताम्}


\twolineshloka
{शीर्षाघातांसधातैश्च परस्परमथाहतौ}
{पुनर्विमिश्रितौ वीरौ व्यायुध्येते मुहुर्मुहुः}


\twolineshloka
{भैमसेनिरथोत्क्षिप्य समाविध्य पुनः पुनः}
{निष्पिपेष क्षितौ क्षिप्रं पूर्णकुम्भमिवाश्मनि}


\twolineshloka
{बललाघवसम्पन्नः सम्पन्नो विक्रमेण च}
{भैमसेनिरथ क्रुद्धः सर्वसैन्यस्य पश्यतः}


\twolineshloka
{असृक्क्षरितसर्वाङ्गश्चूर्णितास्थिविभूषणः}
{घटोत्कचेन निष्पिष्टो हतः सालकटङ्कटः}


\twolineshloka
{पाण्डवानां ततः सेना तं दृष्ट्वा विनिपातितम्}
{ननाद सुमहानादं हर्षवेगसमाप्लुता}


\twolineshloka
{ततस्तु निपपाताशु गतासुर्भुवि राक्षसः}
{शिखरं पर्वतस्येव वज्रवेगेन पातितम्}


\twolineshloka
{पतता तेन महता रथिनां दन्तिनां दश}
{तव सैन्ये महाराज निहताः सुबृहत्तया}


\twolineshloka
{ततो घटोत्कचो हत्वा तद्रक्षो वृत्रसन्निभम्}
{पुनः स रथमास्थाय विजिगीषुर्ननादह'}


\twolineshloka
{ततः सुमनसः पार्था हते तस्मिन्निशाचरे}
{चुक्रुशुः सिंहनादांश्च वासांस्यादुधुवुश्च ह}


\threelineshloka
{तावकाश्च हतं दृष्ट्वा राक्षसेन्द्रं महाबलम्}
{अलम्बुसं तथा शूरा विशीर्णमिव पर्वतम्}
{हाहाकारमकार्षुश्च सैन्यानि भरतर्षभ}


\twolineshloka
{जनाश्च तद्ददृशिरे रक्षः कौतूहलान्विताः}
{यदृच्छया निपतितं भूमावङ्गारकं यथा}


\twolineshloka
{घटोत्कचस्तु तद्धत्वा रक्षो बलवतां वरम्}
{मुमोच बलवन्नादं बलं हत्वेव वासवः}


\twolineshloka
{स पूज्यमानः पितृभिः सबान्धवै--र्घटोत्कचः कर्मणि दुष्करे कृते}
{रिपुं निहत्याभिननन्द वै तदाह्यलम्बुसं पक्वमलम्बुषं यथा}


\twolineshloka
{ततो निनादः सुमहान्समुत्थितःसशङ्खनानाविधबाणघोषवान्}
{निशम्य तं प्रत्यनदंस्तु पाण्डवा--स्ततो ध्वनिर्भुवनमथाऽस्पृशद्भृशम्}


\twolineshloka
{`ततोऽभिगम्य राजानं धर्मपुत्रं युधिष्ठिरम्}
{स्वकर्मावेदयन्मूर्ध्ना प्राञ्जलिर्निपपात ह}


\twolineshloka
{मूर्ध्न्युपाघ्राय तं ज्येष्ठः परिष्वज्य च पाण्डवः}
{प्रितोऽस्मीत्यब्रवीद्राजन्हर्षादुत्फुल्ललोचनः}


\twolineshloka
{घटोत्कचेन निष्पिष्टे हते सालकटङ्कटे}
{बभूवुर्मुदिताः सर्वे हते तस्मिन्निशाचरे'}


\chapter{अध्यायः ११०}
\twolineshloka
{धृतराष्ट्र उवाच}
{}


\threelineshloka
{भारद्वाजं कथं युद्धे युयुधानो न्यवारयत्}
{सञ्जयाचक्ष्व तत्त्वेन परं कौतूहलं हि मे ॥सञ्जय उवाच}
{}


\twolineshloka
{शृणु राजन्महाप्राज्ञ सङ्ग्रामं रोमहर्षणम्}
{द्रोणस्य पाण्डवैः सार्धं युयुधानपुरोगमैः}


\twolineshloka
{वध्यमानं बलं दृष्ट्वा युयुधानेन मारिष}
{अभ्यद्रवत्स्वयं द्रोणः सात्यकिं सत्यविक्रमम्}


\twolineshloka
{तमापतन्तं सहसा भारद्वाजं महारथम्}
{सात्यकिः पञ्चविंशत्या क्षुद्रकाणां समार्पयत्}


\twolineshloka
{द्रोणोऽपि युधि विक्रान्तो युयुधानं समाहितः}
{अविध्यत्पञ्चभिस्तूर्णं हेमपुङ्खैः शरैः शितैः}


\twolineshloka
{ते वर्म भित्त्वा सुदृढं द्विषत्पिशितभोजनाः}
{अभ्ययुर्धरणीं राजन्वल्मीकमिव पन्नगाः}


\twolineshloka
{दीर्घबाहुरभिक्रुद्धस्तोत्रार्दित इव द्विपः}
{द्रोणं पञ्चाशताऽविध्यन्नाराचैरग्निसन्निभैः}


\twolineshloka
{भारद्वाजो रणे विद्धो युयुधानेन सत्वरम्}
{सात्यकिं बहुभिर्बाणैर्यतमानमविध्यत}


\twolineshloka
{ततः क्रुद्धो महेष्वासो भूय एव महाबलः}
{सात्वतं पीडयामास व्रातेन नतपर्वणाम्}


\twolineshloka
{स वध्यमानः समरे भारद्वाजेन सात्यकिः}
{नान्वपद्यत कर्तव्यं किञ्चिदेव विशाम्पते}


\twolineshloka
{विषण्णवदनश्चापि युयुधानोऽभवन्नृप}
{भारद्वाजं रणे दृष्ट्वा विसृजन्तं शिताञ्शरान्}


\twolineshloka
{तं तु सम्प्रेक्ष्य ते पुत्राः सैनिकाश्च विशाम्पते}
{प्रहृष्टमनसो भूत्वा सिंहवद्व्यनदन्मुहुः}


\twolineshloka
{तं श्रुत्वा निनदं घोरं पीड्यमानं च माधवम्}
{युधिष्ठिरोऽब्रवीद्राजा सर्वसैन्यानि भारत}


\threelineshloka
{एष वृष्णिवरो वीरः सात्यकिः सत्यविक्रमः}
{ग्रस्यते युधि वीरेण भानुमानिव राहुणा}
{अभिद्रवत गच्छध्वं सात्यकिर्यत्र युध्यते}


\twolineshloka
{अभिद्रवत गच्छद्वं सात्यकिर्यत्र युध्यते ॥धृष्टद्युम्नं च पाञ्चाल्यमिदमाह जनाधिपः}
{`द्रोणं वारय सुक्षित्रं सात्यकिं मावधीद्द्विजः'}


\twolineshloka
{अभिद्रव द्रुतं द्रोणं किमु तिष्ठसि पार्पत}
{न पश्यसि भयं द्रोणाद्धोरं नः समुपस्थितम्}


\twolineshloka
{असौ द्रोणो महेष्वासो युयुधानेन संयुगे}
{क्रीडते सूत्रबद्धेन पक्षिणा बालको यथा}


\twolineshloka
{तत्रैव सर्वे गच्छन्तु भीमसेनपुरोगमाः}
{त्वयैव सहिताः सर्वे युयुधानरथं प्रति}


\threelineshloka
{पृष्ठतोऽनुगमिष्यामि त्वामहं सहसैनिकः}
{सात्यकिं मोक्षयस्वाद्य यमदंष्ट्रान्तरं गतम् ॥सञ्जय उवाच}
{}


\twolineshloka
{एवमुक्त्वा ततो राजा सर्वसैन्येन भारत}
{अभ्यद्रवद्रणे द्रोणं युयुधानस्य कारणात्}


\twolineshloka
{तत्रारावो महानासीद्द्रोणमेकं युयुत्सताम्}
{पाण्डवानां च भद्रं ते सृञ्जयानां च सर्वशः}


\twolineshloka
{ते समेत्य नरव्याघ्रा भारद्वाजं महारथम्}
{अभ्यवर्षञ्शरैस्तीक्ष्णैः कङ्कबर्हिणवाजितैः}


\twolineshloka
{स्मयन्नेव तु तान्वीरान्द्रोणः प्रत्यग्रहीत्स्वयम्}
{अतिथीनागतान्यद्वत्सलिलेनासनेन च}


\twolineshloka
{तर्पितास्ते शरैस्तस्य भारद्वाजस्य धन्विनः}
{आतिथेयं गृहं प्राप्य तर्प्यन्तेऽतिथयो यथा}


\twolineshloka
{भारद्वाजं च ते सर्वे न शेकुः प्रतिवीक्षितुम्}
{मध्यन्दिनमनुप्राप्तं सहस्रांशुमिव प्रभो}


\twolineshloka
{तांस्तु सर्वान्महेष्वासान्द्रोणः शस्त्रभृतां वरः}
{अतापयच्छरव्रातैर्गभस्तिभिरिवांशुमान्}


\twolineshloka
{वध्यमाना महाराज पाण्डवाः सृञ्जयास्तथा}
{त्रातारं नाध्यगच्छन्त पङ्कमग्ना इव द्विपाः}


\twolineshloka
{द्रोणस्य च व्यदृश्यन्त विसर्पन्तो महाशराः}
{गभस्तय इवार्कस्य प्रतपन्तः समन्ततः}


\twolineshloka
{तस्मिन्द्रोणेन निहताः पाञ्चालाः पञ्चविंशतिः}
{महारथाः समाख्याता धृष्टद्युम्नस्य सम्मताः}


\twolineshloka
{पाण्डूनां सर्वसैन्येषु पाञ्चालानां तथैव च}
{द्रोणं स्म ददृशुः शूरं विनिघ्नन्तं वरान्वरान्}


\twolineshloka
{केकयानां शतं हत्वा विद्राव्य च समन्ततः}
{द्रोणस्तस्थौ महाराज व्यादितास्य इवान्तकः}


\twolineshloka
{पाञ्चालान्सृञ्जयान्मत्स्यान्केकयांश्च नराधिप}
{द्रोणोऽजयन्महाबाहुः शतशोऽथ सहस्रशः}


\twolineshloka
{तेषां समभवच्छब्दो विद्धानां द्रोणसायकैः}
{वनौकसामिवारण्ये व्याप्तानां धृमकेतुना}


\twolineshloka
{तत्र देवाः सगन्धर्वाः पितरश्चाब्रुवन्नृप}
{एते द्रवन्ति पाञ्चालाः पाण्डवाश्च ससैनिकाः}


\twolineshloka
{तं तथा समरे द्रोणं निघ्नन्तं सौमकान्रणे}
{न चाप्यभिययुः केचिदपरे नैव विव्यधुः}


\twolineshloka
{वर्तमाने तथा रौद्रे तस्मिन्वीरवरक्षये}
{अशृणोत्सहसा पार्थः पाञ्चजन्यस्य निःस्वनम्}


\twolineshloka
{पूरिते वासुदेवेन शङ्खराजे महात्मना}
{युध्यमानेषु वीरेषु सैन्धवस्याभिरक्षिषु}


\twolineshloka
{नदत्सु धार्तराष्ट्रेषु विजयस्य रथं प्रति}
{गाण्डीवस्य च निर्घोषे विप्रनष्टे समन्ततः}


\twolineshloka
{कश्मलाभिहतो राजा चिन्तयामास पाण्डवः}
{न नूनं स्वस्ति पार्थाय यथा नदति शङ्खराट्}


\twolineshloka
{कौरवाश्च यथा हृष्टा विनदन्ति मुहुर्मुहुः}
{`व्यक्तमद्य विनश्यन्ति सर्वलोकमहारथाः'}


\twolineshloka
{एवं सञ्चिन्तयित्वा तु व्याकुलेनान्तरात्मना}
{अजताशत्रुः कौन्तेयः सात्वतं प्रत्यभाषत}


\threelineshloka
{बाष्पगद्गदया वाचा मुह्यमानो मुहुर्मुहुः}
{कृत्यस्यानन्तरापेक्षी शैनेयं शिनिपुङ्गवम् ॥युधिष्ठिर उवाच}
{}


\twolineshloka
{यः स धर्मः पुरा दृष्टः सद्भिः शैनेय शाश्वतः}
{साम्पराये सुहृत्कृत्ये तस्य कालोऽयमागतः}


\twolineshloka
{सर्वेष्वपि च योधेषु चिन्तयञ्शिनिपुङ्गव}
{त्वत्तः सुहृत्तमं कञ्चिन्नाभिजानामि सात्यके}


\twolineshloka
{यो हि प्रीतमना नित्यं यश्च नित्यमनुव्रतः}
{स कार्ये साम्पराये तु नियोज्य इति मे मतिः}


\twolineshloka
{यथा च केशवो नित्यं पाण्डवानानं परायणम्}
{तथा त्वमपि वार्ष्णेय कृष्णतुल्यपराक्रमः}


\twolineshloka
{सोऽहं भारं समाधास्ये त्वयि तं वोढुमर्हसि}
{अभिप्रायं च मे नित्यं न वृथा कर्तुमर्हसि}


\twolineshloka
{स त्वं भ्रातुर्वयस्यस्य गुरोरपि च संयुगे}
{कुरु कृच्छ्रे सहायार्थमर्जुनस्य नरर्षभ}


\twolineshloka
{त्वं हि सत्यव्रतः शूरो मित्राणामभयङ्करः}
{लोके विख्यायसे वीर कर्मभिः सत्यवागिति}


\twolineshloka
{यो हि शैनेय मित्रार्थे युध्यमानस्त्यजेत्तनुम्}
{पृथिवीं च द्विजातिभ्यो यो दद्यात्स समो भवेत्}


\twolineshloka
{श्रुताश्च बहवोऽस्माभी राजानो ये दिवं गताः}
{दत्त्वेमां पृथिवीं कृत्स्नां ब्राह्मणेभ्यो यथाविधि}


\twolineshloka
{`दीयमाना च बहुभिर्दास्यते च मुहुर्मही}
{न हि कश्चिद्रणे प्राणान्मित्रार्थे त्यक्तवानिह'}


\twolineshloka
{एवं त्वामपि धर्मात्मन्प्रयाचेऽहं कृताञ्जलिः}
{पृथिवीदानतुल्यं स्यादधिकं वा फलं विभो}


\twolineshloka
{एक एव सदा कृष्णो मित्राणामभयङ्करः}
{रणे सन्तजति प्राणान्द्वितीयस्त्वं च सात्यके}


\twolineshloka
{विक्रान्तस्य च वीरस्य युद्धे प्रार्थयतो यशः}
{शूर एव सहायः स्यान्नेतरः प्राकृतो जनः}


\twolineshloka
{ईदृशे तु परामर्दे वर्तमानस्य माधव}
{त्वदन्यो हि रणे गोप्ता विजयस्य न विद्यते}


\twolineshloka
{श्लाघन्नेव हि कर्माणि शतशस्तव पाण्डवः}
{मम सञ्चनयन्हर्षं पुनः पुनरकीर्तयत्}


\twolineshloka
{लघुहस्तश्चित्रयोधी तथा लघुपराक्रमः}
{प्राज्ञः सर्वास्त्रविच्छूरो मुह्यते न च संयुगे}


\twolineshloka
{महास्कन्धो महोरस्को महाबाहुर्महाहनुः}
{महाबलो महावीर्यः स महात्मा महारथः}


\twolineshloka
{शिष्यो मम सखा चैव प्रियोऽस्याहं प्रियश्च मे}
{युयुधानः सहायो मे प्रमथिष्यति कौरवान्}


\twolineshloka
{अस्मदर्थं च राजेन्द्र सन्नह्येद्यदि केशवः}
{रामो वाप्यनिरुद्धो वा प्रद्युम्नो वा महारथः}


\twolineshloka
{गदो वा मारणो वापि साम्वो वा सह वृष्णिभिः}
{सहायार्थं महाराज सङ्ग्रामोत्तममूर्धनि}


\twolineshloka
{तथाप्यहं नरव्याघ्रं शैनेयं सत्यविक्रमम्}
{साहाय्ये विनियोक्ष्यामि नास्ति मेऽन्योहि तत्समः}


\twolineshloka
{इति द्वैतवने तात मामुवाच धनञ्जयः}
{परोक्षे त्वद्गुणांस्तथ्यान्कथयन्नार्यसंसदि}


\twolineshloka
{तस्य त्वमेवं सङ्कल्पं न वृथा कर्तुमर्हसि}
{धनञ्जयस्य वार्ष्णेय मम भीमस्य चोभयोः}


\twolineshloka
{यच्चापि तीर्थानि चरन्नगच्छं द्वारकां प्रति}
{तत्राहमपि ते भक्तिमर्जुनं प्रति दृष्टवान्}


\twolineshloka
{न तन्सौहृदनन्येपु मया शैनेय लक्षितम्}
{यथा त्वमस्मान्भजसे वर्तमानानुपप्लवे}


\twolineshloka
{सोऽभिजात्या च भक्त्या च सख्यस्याचार्यकस्य च}
{सौहृदस्य च वीर्यस्य कुलीनत्वस्य माधव}


\twolineshloka
{सत्यस्य च महावाहो अनुकम्पार्थमेव च}
{अनुरूपं महेप्वास कर्म त्वं कर्तुमर्हसि}


\twolineshloka
{सुयोधनो हि सहसा गतो द्रोणेन दंशितः}
{पूर्वमेवानुयातास्ते कौरवाणां महारथाः}


\twolineshloka
{सुमहान्निनदश्चैव श्रूयते विजयं प्रति}
{स शैनेय जवेनाशु गन्तुमर्हसि मानद}


\twolineshloka
{भीमसेनो वयं चैव संयत्ताः सहसैनिकाः}
{द्रोणमावारयिष्यामो यदि त्वां प्रतियोत्स्यते}


\twolineshloka
{पश्य शैनेय सैन्यानि द्रवमाणानि संयुगे}
{महान्तं चरणे शब्दं दीर्यमाणां च भारतीम्}


\twolineshloka
{महामारुतवेगेन समुद्रमिव पर्वसु}
{धार्तराष्टबलं तात विक्षिप्तं सव्यसाचिना}


\twolineshloka
{रथैर्विपरिधावद्भिर्मनुष्यैश्च हयैश्च ह}
{सैन्यं रजःसमुद्भूतमेतत्सम्परिवर्तते}


\twolineshloka
{संवृतः सिन्धुसौवीरैर्नखरप्रासयोधिभिः}
{अत्यन्तोपचितैः शूरैः फल्गुनः परवीरहा}


\twolineshloka
{नैतद्बलमसंवार्यं शक्यो जेतुं जयद्रथः}
{एते हि सैन्धवस्यार्थे सर्वे सन्त्यक्तजीविताः}


\twolineshloka
{शरशक्तिध्वजवरं हयनागसमाकुलम्}
{पश्यैतद्धार्तराष्ट्राणामनीकं सुदुरासदम्}


\twolineshloka
{शृणु दुन्दुभिनिर्घोषं शङ्खशब्दांश्च पुष्कलान्}
{सिंहनादरवांश्चैव रथनेमिस्वनांस्तथा}


\twolineshloka
{नागानां शृणु शब्दं च पत्तीनां च सहस्रशः}
{सादिनां द्रवतां चैव शृणु कम्पयतां महीम्}


\twolineshloka
{पुरस्तात्सैन्धवानीकं द्रोणानीकं च पृष्ठतः}
{बहुत्वाद्धि नरव्याघ्र देवेन्द्रमपि पीडयेत्}


\twolineshloka
{अपर्यन्ते बले मग्नो जह्यादपि च जीवितम्}
{तस्मिंश्च निहते युद्धे कथं जीवेत मादृशः}


\twolineshloka
{सर्वथा समनुप्राप्तः सुकृच्छ्रं त्वयि जीवति}
{श्यामो युवा गुडाकेशो दर्शनीयश्च पाण्डवः}


\twolineshloka
{लघ्वस्त्रश्चित्रयोधी च प्रविष्टस्यात भारतीम्}
{सूर्योदये महाबाहुर्दिवसश्चातिवर्तते}


\twolineshloka
{तन्न जानामि वार्ष्णेय यदि जीवति वा न वा}
{कुरूणां चापि तत्सैन्यं सागरप्रतिमं महत्}


\twolineshloka
{एक एव च बीभत्सुः प्रविष्टस्तात भारतीम्}
{अविपह्यां महाबाहुः सुरैरपि महाहवे}


\twolineshloka
{न हि मे वर्तते बुद्धिरद्य युद्धे कथञ्चन}
{द्रोणोऽपि रभसो युद्धे मम पीडयते बलम्}


\threelineshloka
{प्रत्यक्षं ते महाबाहो यथाऽसौ चरति द्विजः}
{युगपच्च समेतानां कार्याणां त्वं विचक्षणः}
{महार्थं लघुसंयुक्तं कर्तुमर्हसि मानद}


\twolineshloka
{तस्य मे सर्वकार्येषु कार्यमेतन्मतं महत्}
{अर्जुनस्य परित्राणां कर्तव्यमिति संयुगे}


\twolineshloka
{नाहं शोचामि दाशार्हं गोप्तारं जगतः पतिम्}
{स हि शक्तो रणे तात त्रीँल्लोकानपि सङ्गतान्}


\twolineshloka
{विजेतुं पुरुषव्याघ्रः सत्यमेतद्ब्रवीमि ते}
{किम्पुनर्धार्तराष्ट्रस्य बलमेतत्सुदुर्बलम्}


\twolineshloka
{अर्जुनस्त्वेष वार्ष्णेय पीडितो बहुभिर्युधि}
{प्रजह्यात्समरे प्राणांस्तस्माद्विन्दामि कश्मलम्}


\twolineshloka
{तस्य त्वं पदवीं गच्छ गच्छेयुस्त्वादृशा यथा}
{तादृशस्येदृशे काले मादृशेनाभिनोदितः}


\twolineshloka
{रणे वृष्णिप्रवीराणां द्वावेवातिरथौ स्मृतौ}
{प्रद्युम्नश्च महाबाहुस्त्वं च सात्वत विश्रुतः}


\twolineshloka
{अस्त्रे नारायणसमः सङ्कर्षणसमो बले}
{वीरतायां नरव्याघ्र धनञ्जयसमो ह्यसि}


\twolineshloka
{भीष्मद्रोणावतिक्रम्य सर्वयुद्धविशारदम्}
{त्वामेव पुरुषव्याघ्रं लोके सन्तः प्रचक्षते}


\twolineshloka
{`सदेवासुरगन्धर्वान्सकिन्नरमहोरगान्}
{योधयेच्च जगत्सर्वं विजयेत रिपून्बहून्}


\threelineshloka
{इति ब्रुवन्ति लोकेषु जनास्तव गुणांस्तथा}
{समागमेषु सर्वेषु पृथगेव च सर्वदा'}
{नाशक्यं विद्यते लोके सात्यकेरिति माधव}


% Check verse!
तत्त्वां यदभिवक्ष्यामि तत्कुरुष्व महाबल
\twolineshloka
{सम्भावना हि लोकस्य मम पार्थस्य चोभयोः}
{नान्यथा तां महाबाहो सम्प्रकर्तुमिहार्हसि}


\twolineshloka
{परित्यज्य प्रियान्प्राणान्रणे चर विभीतवत्}
{न हि शैनेय दाशार्हा रणे रक्षन्ति जीवितम्}


\twolineshloka
{अयुद्धमनवस्थानं सङ्ग्रामे च पलायनम्}
{भीरूणामसतां मार्गे नैष दाशार्हसेवितः}


\twolineshloka
{तवार्जुनो गुरुस्तात धर्मात्मा शिनिपुङ्गच}
{वासुदेवो गुरुश्चापि तव पार्थस्य धीमतः}


\twolineshloka
{कारणद्वयमेतद्धि जानंस्त्वामहमब्रुवम्}
{मावमंस्था वचो मह्यं गुरुस्तव गुरोर्ह्यहम्}


\twolineshloka
{वासुदेवमतं चैव मम चैवार्जुनस्य च}
{सत्यमेतन्मयोक्तं ते याहि यत्र धनञ्जयः}


\twolineshloka
{एतद्वचनमाज्ञाय मम सत्यपराक्रम}
{प्रविशैतद्बलं तात धार्तराष्ट्रस्य दुर्मतेः}


\twolineshloka
{प्रविश्य च यथान्यायं सङ्गम्य च महारथैः}
{यथार्हमात्मनः कर्म रणे सात्वत दर्शय}


\chapter{अध्यायः १११}
\twolineshloka
{सञ्जय उवाच}
{}


\twolineshloka
{प्रीतियुक्तं च हृद्यं च माननीयं च सर्वशः}
{कालयुक्तं च चित्रं च स्वधिया चाभिभाषितम्}


\twolineshloka
{धर्मराजस्य तद्वाक्यं निश्यम्य शिनिपुङ्गवः}
{सात्यकिर्भरतश्रेष्ठ प्रत्युवाच युधिष्ठिरम्}


\twolineshloka
{श्रुतं ते गदतो वाक्यं सर्वमेतन्मयाऽच्युत}
{न्याययुक्तं च सत्यं च फल्गुनार्थे यशस्करम्}


\twolineshloka
{एवंविधे तथा काले मादृशं प्रेक्ष्य सत्तम}
{वक्तुमर्हसि राजेन्द्र यथा पार्थं तथैव माम्}


\twolineshloka
{न मे धनञ्जयस्यार्थे प्राणा रक्ष्याः कथञ्चन}
{त्वत्प्रयुक्तः पुनरहं किं नकुर्यां महाहवे}


\twolineshloka
{लोकत्रयं योधयेयं सदेवासुरमानुषम्}
{त्वत्प्रयुक्तो नरेन्द्रेन्द्र किम्पुनस्तान्सुदुर्बलान्}


\twolineshloka
{सुयोधनबलं त्वद्य योधयिष्ये समन्ततः}
{विजेष्ये च रणे राजन्सत्यमेतद्ब्रवीमि ते}


\twolineshloka
{कुशल्यहं कुंशलिनं समासाद्य धनञ्जयम्}
{हते जयद्रथे राजन्पुनरेष्यामि तेऽन्तिकम्}


\twolineshloka
{अवश्यं तु मया सर्वं विज्ञाप्यस्त्वं नराधिप}
{वासुदेवस्य यद्वाक्यं फल्गुनस्य च धीमतः}


\twolineshloka
{दृढं त्वभिहितश्चाहमर्जुनेन पुनःपुनः}
{मध्ये सर्वस्य सैन्यस्य वासुदेवस्य शृण्वतः}


\twolineshloka
{अद्य माधव राजानमप्रमत्तोऽनुपालय}
{आर्यां युद्धे मतिं कृत्वा यावद्धन्मि जयद्रथम्}


\twolineshloka
{त्वयि चाहं महाबाहो प्रद्युम्ने वा महारथे}
{नृपं निक्षिप्य गच्छेयं निरपेक्षो जयद्रथम्}


\twolineshloka
{जानीषे हि रणे द्रोणं कुरुषु श्रेष्ठसम्मतम्}
{प्रतिज्ञा चापि ते नित्यं श्रुता द्रोणस्य माधव}


\twolineshloka
{ग्रहमे धर्मराजस्य भारद्वाजोऽपि गृध्यति}
{शक्तश्चापि रणे द्रोणो निग्रहीतुं युधिष्ठिरम्}


\twolineshloka
{एवं त्वयि समाघाय धर्मराजं नरोत्तमम्}
{अहमद्य गमिष्यामि सैन्धवस्य वधाय हि}


\twolineshloka
{`स त्वमद्य रणे यत्तो रक्ष माधव पाण्डवम्}
{रक्षेणे धर्मराजस्य ध्रुवो हि विजयो मम'}


\twolineshloka
{जयद्रथं च हत्वाऽहं द्रुतमेष्यामि माधव}
{धर्मराजं न चेद्द्रोणो निगृह्णीयाद्रणे बलात्}


\twolineshloka
{निगृहीते नरश्रेष्ठे भारद्वाजेन माधव}
{सैन्धवस्य वधो न स्यान्ममाप्रीतिस्तथा भवेत्}


\twolineshloka
{एवङ्गते नरश्रेष्ठे पाण्डवे सत्यवादिति}
{अस्माकं गमनं व्यक्तं वनं प्रति भवेत्पुनः}


\twolineshloka
{सोऽयं मम जयो व्यक्तं व्यर्थ एव भविष्यति}
{यदि द्रोणो रणे क्रुद्धो निगृह्णीयाद्युधिष्ठिरम्}


\twolineshloka
{स त्वमद्य महाबाहो प्रियार्थं मम माधव}
{जयार्थं च यशोर्थं च रक्ष राजानमाहवे}


\twolineshloka
{स भवान्मयि निक्षेपो निक्षिप्तः सव्यसाचिना}
{भारद्वाजाद्भयं नित्यं मन्यमानेन वै प्रभो}


\twolineshloka
{तस्यापि च महाबाहो नित्यं पश्यामि संयुगे}
{नान्यं हि प्रतियोद्धारं रौक्मिणेयादृते प्रभो}


\threelineshloka
{मां चापि मन्यते युद्धे भारद्वाजस्य धीमतः}
{सोऽहं संभावनानं चैतामाचार्यवचनं च तत्}
{पृष्ठतो नोत्सहे कर्तुं त्वां वा त्यक्तुं महीपते}


\twolineshloka
{आचार्यो लघुहस्तत्वादभेद्यकवचावृतः}
{उपलभ्य रणे क्रीडेद्यथा शकुनिना शिशुः}


\twolineshloka
{यदि कार्ष्णिर्धनुष्पाणिरिह स्यान्मकरध्वजः}
{तस्मै त्वां विसृजेयं वै स त्वां रक्षेद्यथाऽर्जुनः}


\twolineshloka
{कुरु त्वमात्मनो गुप्तिं कस्ते गोप्ता गते मयि}
{यः प्रतीयाद्रमे द्रोणं यावद्गच्छामि पाण्डवम्}


\twolineshloka
{मा च ते भयमद्यास्तु राजन्नर्जुनसम्भवम्}
{न स जातु महाबाहुर्भारमुद्यम्य सीदति}


\twolineshloka
{ये च सौवीरका योधास्तथा सैन्धवपौरवाः}
{उदीच्या दाक्षिणात्याश्च ये चान्येऽपि महारथाः}


\twolineshloka
{ये च कर्णमुखा राजन्रथोदाराः प्रकीर्तितः}
{एतेऽर्जुनस्य क्रुद्धस्य कलां नार्हन्ति षोडशीम्}


\twolineshloka
{उद्युक्ताः पृथिवीपाल ससुरासुरमानुषाः}
{सराक्षसगणा राजन्सकिन्नरमहोरगाः}


\twolineshloka
{जङ्गमस्थावरैः सार्धं नालं पार्थस्य संयुगे}
{एवं ज्ञात्वा महाराज व्येतु ते भीर्धनञ्जये}


\threelineshloka
{एवं वीरौ महेष्वासौ कृष्णौ सत्यपराक्रमौ}
{न तत्र कर्मणो व्यापत्कथंचिदपि विद्यते}
{}


\twolineshloka
{दैवं कृतास्त्रतां योगममर्षं शौर्यमाहवे}
{कृतज्ञतां दयां चैव भ्रातुस्त्वमनुचिन्तय}


\twolineshloka
{मयि चापि सहाये ते गच्छमानेऽर्जुनं प्रति}
{द्रोणे चित्रास्त्रतां सह्ख्ये राजंस्त्वमनुचिन्तय}


\twolineshloka
{आचार्यो हि भृशं राजन्निग्रहे तव गृध्यति}
{प्रतिज्ञामात्मनो रक्षन्सत्यां कर्तुं च भारत}


\twolineshloka
{कुरुष्वाद्यात्मनो गुप्तिं कस्ते गोप्ता गते मयि}
{यस्याहं प्रत्ययात्पार्थं गच्छेयं फल्गुनं प्रति}


\twolineshloka
{न ह्यहं त्वां महाराज अनिक्षिप्य महाहवे}
{क्वचिद्यास्यामि कौरव्य सत्यमेतद्ब्रवीमि ते}


\twolineshloka
{`न पश्यामि रथं कञ्चिद्यस्ते गोप्ता भवेदिह}
{शक्तं यं मन्यसे राजन्गोप्तारं प्रति पाण्डव'}


\threelineshloka
{एतद्विचार्य बहुशो बुद्ध्या बुद्धिमतां वर}
{दृष्ट्वा श्रेयः परं बुद्ध्या ततो राजन्प्रशाधि माम् ॥युधिष्ठिर उवाच}
{}


\twolineshloka
{एवमेतन्महाबाहो यथा वदसि माधव}
{न तु मे शुद्ध्यते भावः श्वेताश्वं प्रति मारिष}


\twolineshloka
{करिष्ये परमं यत्नमात्मनो रक्षणे ह्यहम्}
{गच्छ त्वं समनुज्ञातो यत्र यातो धनञ्जयः}


\twolineshloka
{आत्मसंरक्षणं सङ्ख्ये गमनं चार्जुनं प्रति}
{विचार्यैतत्स्वयं बुद्ध्या गमनं तत्र रोचये}


\twolineshloka
{स त्वमातिष्ठ यानाय यत्र यातो धनञ्जयः}
{ममापि रक्षणं भीमः करिष्यति महाबलः}


\twolineshloka
{पार्षतश्च ससोदर्यः पार्थिवाश्च मबाहलाः}
{द्रौपदेयाश्च मां तात रक्षिष्यन्ति न संशयः}


\twolineshloka
{केकया भ्रातरः पञ्च राक्षसश्च घटोत्कचः}
{विराटो द्रुपदश्चैव शिखण्डी च महारथः}


\threelineshloka
{धृष्टकेतुश्च बलवान्कुन्तिभोजश्च मातुलः}
{नकुलः सहदेवश्च पाञ्चालाः सृञ्जयास्तथा}
{एते समाहितास्तात रक्षिष्यन्ति न संशयः}


\twolineshloka
{न द्रोणः सह सैन्येन कृतवर्मा च संयुगे}
{समासादयितुं शक्तो न च मां धर्षयिष्यति}


\twolineshloka
{धृष्टद्युम्नश्च समरे द्रोणं क्रुद्धं परन्तपः}
{वारयिष्यति विक्रम्य वेलेव मकरालयम्}


\twolineshloka
{यत्र स्थास्यति सङ्ग्रामे पार्षतः परवीरहा}
{न द्रोणस्य बलं तात क्रमेत्तत्र कथञ्चन}


\twolineshloka
{एष द्रोणविनाशाय समुत्पन्नो हुताशनात्}
{कवची स शरी खङ्गी धन्वी च वरभूषणः}


\twolineshloka
{`एष द्रोणं रणे क्रुद्धं वारयेत स वै प्रभो}
{पाञ्चालः सहितैः सर्वैः पाण्डवानां च धन्विभिः'}


\twolineshloka
{विक्रुधं गच्छ शैनेय मा कार्षीर्मयि सम्भ्रमम्}
{धृष्टद्युम्नो रणे क्रुद्धं द्रोणमावारयिष्यति}


\chapter{अध्यायः ११२}
\twolineshloka
{सञ्जय उवाच}
{}


\twolineshloka
{धर्मराजस्य तद्वाक्यं निशम्य शिनिपुङ्गवः}
{स पार्थाद्भयमाशंसन्परित्यागान्महीपतेः}


\twolineshloka
{अपवादं ह्यात्मनश्च लोकाद्रक्षन्विशेषतः}
{न मां भीतमिति ब्रूयात्प्रयान्तं फल्गुनं प्रति}


\twolineshloka
{निश्चित्य बहुधैवं स सात्यकिर्युद्धदुर्मदः}
{नातिव्यक्तमिवाभाष्य धर्मराजं महायशाः}


\twolineshloka
{शोकगद्गदया वाचा शोकोपहतचेतनः}
{यथेदानीमिति ध्यात्वा धर्मराजमथाब्रवीत्}


\twolineshloka
{कृतां चेन्मन्यसे रक्षां स्वस्ति तेऽस्तु विशाम्पते}
{अनुयास्यामि बीभत्सुं करिष्ये वचनं तव}


\twolineshloka
{न हि मे पाण्डवात्कश्चित्त्रिषु लोकेषु विद्यते}
{यो मे प्रियतरो राजन्सत्यमेतद्ब्रवीमि ते}


\twolineshloka
{तस्याहं पदवीं यास्ये सन्देशात्तव मानद}
{त्वत्कृते न च मे किञ्चिदकर्तव्यं कथञ्चन}


\twolineshloka
{यथा हि मे गुरोर्वाक्यं विशिष्टं द्विपदां वर}
{तथा तवापि वचनं विशिष्टतरमेव मे}


\twolineshloka
{प्रिये हि तव वर्तेते भ्रातरौ कृष्णपाण्डवौ}
{तयोः प्रिये स्थितं चैव विद्धि मां राजपुङ्गव}


\twolineshloka
{तवाज्ञां शिरसा गृह्य पाण्डवार्थमहं प्रभो}
{भित्त्वेदं दुर्भिदं सैन्यं प्रयास्ये नरपुङ्गव}


\twolineshloka
{द्रणानीकं विशाम्येष क्रुद्धो झष इवार्णवम्}
{तत्र यास्यामि यत्रासौ राजन्राजा जयद्रथः}


\twolineshloka
{यत्र सेनान्तमाश्रित्य भीतस्तिष्ठति पाण्डवात्}
{गुप्तो रथवरश्रेष्ठैर्द्रौणिकर्णकृपादिभिः}


\twolineshloka
{इतिस्त्रियोजनं मन्ये तमध्वानं विशाम्पते}
{यत्र तिष्ठति पार्थोऽसौ जयद्रथवधोद्यतः}


\twolineshloka
{त्रियोजनगतस्यापि तस्य यास्याभ्यहं पदम्}
{आसैन्धववधाद्राजन्सुदृढेनान्तरात्मना}


\twolineshloka
{अनादिष्टस्तु गुरुणा को नु युध्येत मानवः}
{आदिष्टस्तु यथा राजन्को न युध्येत मादृशः}


\twolineshloka
{अभिजानामि तं देशमितः पूर्णं त्रियोजनम्}
{यत्र तिष्ठति राजाऽसौ सैन्धवो बालघातकः}


\threelineshloka
{सागरप्रतिमं सैन्यं गर्जन्तमिव सागरम्}
{हलशक्तिगदाप्रासचर्मखङ्गर्ष्टितोमरम्}
{इष्वस्त्रवरसम्बाधं क्षोभयित्वा व्रजाम्यहम्}


\twolineshloka
{यदेतत्कुञ्जरानीकं सहस्रमनुपश्यसि}
{कुलमाञ्जनकं नाम यत्रैते वीर्यशालिनः}


\twolineshloka
{आस्थिता बहुभिर्म्लेच्छैर्युद्धशौण्डैः प्रहारिभिः}
{नागा नगनिभा राजन्क्षरन्त इव तोयदाः}


\twolineshloka
{नैते जातु निवर्तेरन्प्रेषिता हस्तिसादिभिः}
{अन्यत्र हि वधादेषां नास्ति राजन्पराजयः}


\twolineshloka
{अथ यान्रथिनो राजन्सहस्रमनुपश्यसि}
{एते रुक्मरथा नाम राजपुत्रा महारथाः}


\twolineshloka
{रथेष्वस्त्रेषु निपुणा नागेषु च विशाम्पते}
{धनुर्वेदे गताः पारं मुष्टियुद्धे च कोविदाः}


\twolineshloka
{गदायुद्धविशेषज्ञा नियुद्धकुशलास्तथा}
{खङ्गप्रहरणे युक्ताः सम्पाते चासिचर्मणोः}


\twolineshloka
{शूराश्च कृतविद्याश्च स्पर्धन्ते च परस्परम्}
{नित्यं हि समरे राजन्विजिगीषन्ति मानवान्}


\twolineshloka
{कर्णेन विहिता राजन्दुःशासनमनुव्रताः}
{एतांस्तु वासुदेवोऽपि रथोदारान्प्रशंसति}


\twolineshloka
{सततं प्रियकामाश्च कर्णस्यैते वशे स्थिताः}
{तस्यैव वचनाद्राजन्निवृत्ताः श्वेतवाहनात्}


\twolineshloka
{ते न क्लान्ता न च श्रान्ता दृढावरणकार्मुकाः}
{मदर्थेऽधिष्ठिता नूनं धार्तराष्ट्रस्य शासनात्}


\twolineshloka
{एतान्प्रमथ्य सङ्ग्रामे प्रियार्थं तव कौरव}
{प्रयास्यामि ततः पश्चात्पदवीं सव्यसाचिनः}


\twolineshloka
{यांस्त्वेतानपरान्राजन्नागान्सप्तशतानिमान्}
{प्रेक्षसे वर्मसञ्छन्नान्किरातैः समधिष्ठितान्}


\twolineshloka
{किरातराजो यान्प्रादाद्गृहीतः सव्यसाचिना}
{स्वलङ्कृतांस्तदा प्रेष्यानिच्छञ्जीवितमात्मनः}


\twolineshloka
{आसन्नेते पुरा राजंस्तव कर्मकरा दृढम्}
{त्वामेवाद्य युयुत्सन्ते पश्य कालस्य पर्ययम्}


\twolineshloka
{एषामेते महामात्राः किराता युद्धदुर्मदाः}
{हस्तिशिक्षाविदश्चैव सर्वे चैवाग्नियोनयः}


\twolineshloka
{एते विनिर्जिताः सङ्ख्ये सङ्ग्रामे सव्यसाचिना}
{मदर्थमद्य संयत्ता दुर्योधनवशानुगाः}


\twolineshloka
{एतान्हत्वा शरै राजन्किरातान्युद्धदुर्मदान्}
{सैन्धवस्य वधे यत्तमनुयास्यामि पाण्डवम्}


\twolineshloka
{ये त्वेते सुमहानाग अञ्जनस्य कुलोद्भवाः}
{कर्कशाश्च विनीताश्च प्रभिन्नकरटामुखाः}


\twolineshloka
{जाम्बूनदमयैः सर्वैर्वर्मभिः सुविभूषिताः}
{लब्धलक्षा रणे राजन्नैरावणसमा युधि}


\twolineshloka
{आस्थिता दस्युभिस्तीक्ष्णैः शूरैरुत्तमपार्वतैः}
{कर्कशैः प्रवरैर्योधैः काष्णार्यसतनुच्छदैः}


\twolineshloka
{सन्ति गोयोनयश्चात्र सन्ति वानरयोनयः}
{अनेकयोनयश्चान्ये तथा मानुषयोनयः}


\twolineshloka
{अनीकमसतामेतद्भूम्रवर्णमुदीर्यते}
{म्लेच्छानां पापकर्तॄणां हिमदुर्गनिवासिनाम्}


\twolineshloka
{एतद्दुर्योधनो लब्ध्वा समग्रं राजमण्डलम्}
{कृपं च सौमदत्तिं च द्रोणं च रथिनां वरम्}


\twolineshloka
{सिन्धुराजं तथा कर्णमवमन्यत पाण्डवान्}
{कृतार्थमथ चात्मानं मन्यते कालचोदितः}


\twolineshloka
{ते तु सर्वेऽद्य सम्प्राप्ता मम नाराचगोचरम्}
{नभविष्यन्ति कौन्तेय यद्यपि स्युर्मनोजवाः}


\twolineshloka
{तेन संभाविता नित्यं परवीर्योपजीविना}
{विनाशमुपयास्यन्ति मच्छरौघनिपीडिताः}


\twolineshloka
{ये त्वेते रथिनो राजन्दृश्यन्ते काञ्चनध्वजाः}
{एते दुर्वारणा नाम काम्भोजा यदि ते श्रुताः}


\twolineshloka
{शूराश्च कृतविद्याश्च धनुर्वेदे च निष्ठिताः}
{संहताश्च भृशं ह्येते अन्योन्यस्य हितैषिणः}


\twolineshloka
{अक्षौहिण्यश्च संरब्धा धार्तराष्ट्रस्य भारत}
{यत्ता मदर्थे तिष्ठन्ति कुरुवीराभिरक्षिताः}


\twolineshloka
{अप्रमत्ता महाराज मामेव प्रत्युपस्थिताः}
{तानहं प्रमथिष्यामि तृणानीव हुताशनः}


\twolineshloka
{तस्मात्सर्वानुपासङ्गान्सर्वोपकरणानि च}
{रथे कुर्वन्तु मे राजन्यथावद्रथकल्पकाः}


\twolineshloka
{अस्मिंस्तु किल सम्मर्दे ग्राह्यं विविधमायुधम्}
{यथोपदिष्टमाचार्यैः कार्यः पञ्चगुणो रथः}


\twolineshloka
{काम्भोजैर्हि समेष्यामि तीक्ष्णैराशीविषोपमैः}
{नानाशस्त्रसमावायैर्विविधायुधयोधिभिः}


\twolineshloka
{किरातैश्च समेष्यामि विषकल्पैः प्रहारिभिः}
{लालितैः सततं राज्ञा दुर्योधनहितैषिभिः}


\twolineshloka
{शकैश्चापि समेष्यामि शक्रतुल्यपराक्रमैः}
{अग्निकल्पैर्युराधर्षैः प्रदीप्तैरिव पावकैः}


\twolineshloka
{तथाऽन्यैर्विविधैर्योधैः कालकल्पैर्दुरासदैः}
{समेष्यामि रणे राजन्बहुभिर्युद्धदुर्मदैः}


\twolineshloka
{`त्रियोजनगतस्यापि पदवीं सव्यसाचिनः}
{यास्यामि रथिनां श्रेष्ठं प्रवरं च धनुष्मताम्}


\twolineshloka
{सूर्योदयगतस्याहं पाण्डवस्य गतिं चरन्}
{अपराह्णगते सूर्ये गमिष्यामि न संशयः'}


\threelineshloka
{तस्माद्वै वाजिनो मुख्यान्विश्रुताञ्शुभलक्षणैः}
{उपावृत्तांश्च पीतांश्च पुनर्युञ्जन्तु मे रथे ॥सञ्जय उवाच}
{}


\twolineshloka
{तस्य सर्वानुपासङ्गान्सर्वोपकरणानि च}
{रथे चास्थापयद्राजा शस्त्राणि विविधानि च}


\twolineshloka
{ततस्तान्सर्वतोमुक्तान्सदश्वांश्चतुरो जनाः}
{रसवत्पाययामासुः पानं मदसमीरणम्}


\twolineshloka
{पीतोपवृत्तान्स्नातांश्च जग्धान्नान्समलङ्कृतान्}
{विनीतशल्यांस्तुरगांश्चतुरो हेममालिनः}


\twolineshloka
{तान्युक्तान्रुक्मवर्णाभान्विनीताञ्शीघ्रगामिनः}
{संहृष्टमनसो व्यग्रान्विधिवत्कल्पितान्रथे}


\twolineshloka
{महाध्वजेन सिंहेन हेमकेसरमालिना}
{संवृते केतकैर्हैर्मैर्मणिविद्रुमचित्रितैः}


\threelineshloka
{पाण्डुराभ्रप्रकाशाभिः पताकाभिरलङ्कृते}
{हेमदण्डोच्छ्रितच्छत्रे बहुशस्त्रपरिच्छदे}
{योजयामास विधिवद्धेमभाण्डविभूषितान्}


\twolineshloka
{दारुकस्यानुजो भ्राता सूतस्तस्य प्रियः सखा}
{न्यवेदयद्रथं युक्तं वासवस्येव मातलिः}


\twolineshloka
{ततः स्नातः शुचिर्भूत्वा कृतकौतुकमङ्गलः}
{स्नातकानां सहस्रस्य स्वर्णनिष्कानथो ददौ}


\twolineshloka
{आशीर्वादैः परिष्वक्तः सात्यकिः श्रीमतां वरः}
{ततः समधुपर्काम्भः पीत्वा कैरातकं मधु}


\twolineshloka
{लोहिताक्षो बभौ तत्र मदविह्वललोचनः}
{आलभ्य वीरकांस्यं च हर्षेण महताऽन्वितः}


\twolineshloka
{द्विगुणीकृततेजोभिः प्रज्वलन्निव पावकः}
{उत्सङ्गे धनुरादाय सशरं रथिनां वरः}


\twolineshloka
{`आशीर्वादैः परिष्वक्तः सात्यकिः श्रीमतां वरः'}
{कृतस्वस्त्ययनो विप्रैः कवची समलङ्कृतः}


\twolineshloka
{लाजैर्गन्धैस्तथा माल्यैः कन्याभिश्चाभिनन्दितः}
{युधिष्ठिरस्य चरणावभिवाद्य कृताञ्जलिः}


\threelineshloka
{तेन मूर्धन्युपाघ्रातः कृतस्वस्ययनोऽपि च}
{आशिषो विविधाः श्रुत्वा धर्मराजमुखेरिताः}
{हर्षेण महता युक्त आरुरोह महारथम्}


\threelineshloka
{ततस्ते वाजिनो हृष्टाः सुपुष्टा वातरंहसः}
{अजय्या जैत्रमूहुस्तं विकुर्वाणाः स्म सैन्धवाः}
{यथा शक्ररथं राजन्नूहुस्ते हरयः पुरा}


\twolineshloka
{तथैव भीमसेनोऽपि धर्मराजेन पूजितः}
{प्रायात्सात्यकिना सार्धमभिवाद्य युधिष्ठिरम्}


\twolineshloka
{तौ दृष्ट्वा प्रविविक्षन्तौ तव सेनामरिंदमौ}
{संयत्तास्तावकाः सर्वे तस्थुर्द्रोणपुरोगमाः}


\twolineshloka
{सन्नद्धमनुगच्छन्तं दृष्ट्वा भीमं स सात्यकिः}
{अभिनन्द्याब्रवीद्वीरस्तदा हर्षकरं वचः}


\twolineshloka
{त्वं भीम रक्ष राजानमेतत्कार्यतमं हि ते}
{अहं भित्त्वा प्रवेक्ष्यामि कालपक्वमिदं बलम्}


\fourlineindentedshloka
{आयत्यां च तदात्वे च श्रेयो राज्ञोऽभिरक्षणम्}
{जानीषे मम वीर्यं त्वं तव चाहमरिन्दम}
{तस्माद्भीम निवर्तस्व मम चेदिच्छसि प्रियम् ॥सञ्जय उवाच}
{}


\twolineshloka
{तथोक्तः सात्यकिं प्राह व्रज त्वं कार्यसिद्धये}
{अहं राज्ञः करिष्यामि रक्षां पुरुषसत्तम}


\twolineshloka
{एवमुक्तः प्रत्युवाच भीमसेनं स माधवः}
{गच्छ गच्छ ध्रुवं पार्थ ध्रुवो हि विजयो मम}


\twolineshloka
{यन्मे स्निग्धोऽनुरक्तश्च त्वमद्य वशमास्थितः}
{निमित्तानि च धन्यानि यथा भीम वदन्ति मां}


\fourlineindentedshloka
{निहते सैन्धवे पापे पाण्डवेन महात्मना}
{परिष्वजिष्ये राजानं धर्मात्मानं युधिष्ठिरम्}
{`अप्रमादश्च ते कार्यो द्रोणं प्रति महारथम्' ॥सञ्चय उवाच}
{}


\twolineshloka
{एतावदुक्त्वा भीमं तु विसृज्य च महायशाः}
{सम्प्रैक्षत्तावकं सैन्यं व्याघ्रो मृगगणानिव}


\twolineshloka
{तं दृष्ट्वा प्रविविक्षन्तं सैन्यं तव जनाधिप}
{भूय एवाभवन्मूढं सुभृशं चाप्यकम्पत}


\twolineshloka
{ततः प्रयातः सहसा तव सैन्यं स सात्यकिः}
{दिदृक्षुरर्जुनं राजन्धर्मराजस्य शासनात्}


\chapter{अध्यायः ११३}
\twolineshloka
{सञ्जय उवाच}
{}


\threelineshloka
{प्रयाते तव सैन्यं तु युयुधाने युयुत्सया}
{धर्मराजो महाराज स्वेनानीकेन संवृतः}
{प्रायाद्द्रोणरथं प्रेप्सुर्युयुधानस्य पृष्ठतः}


\twolineshloka
{ततः पाञ्चालराजस्य पुत्रः समरदुर्मदः}
{`प्रयाते माधवे राजन्निदं वचनमब्रवीत्'}


\twolineshloka
{प्राक्रोशत्पाण्डवानीके वसुदानश्च पार्थिवः}
{आगच्छत प्रहरत द्रुतं विपरिधावत}


\threelineshloka
{यथा सुखेन गच्छेत सात्यकिर्युद्धदुर्मदः}
{महारथा हि बहवो यतिष्यन्तेऽस्य निर्जये ॥सञ्जय उवाच}
{}


\twolineshloka
{`सेनापतिवचः श्रुत्वा पाण्डवेयाः समन्ततः}
{अभ्युद्ययुर्महाराज तव सैन्यं समन्ततः}


\twolineshloka
{जहि प्रहर गृह्णीहि विध्य विद्रव चाद्रव'}
{इति ब्रुवन्तौ वेगेन निपेतुस्ते महारथाः}


\twolineshloka
{वयं प्रतिजिगीषन्तस्तत्र तान्समभिद्रुताः}
{`बाणशङ्खरवान्कृत्वा विमिश्रान्वाद्यनिस्वनैः}


\twolineshloka
{युयुधानरथं दृष्ट्वा तावका अभिदुद्रुवुः'}
{ततः शब्दो महानासीद्युयुधानरथं प्रति}


\twolineshloka
{आकीर्यमाणा धावन्ती तव पुत्रस्य वाहिनी}
{सात्वतेन महाराज शतधाऽभिव्यशीर्यत}


\twolineshloka
{तस्यां विदीर्यमाणायां शिनेः पुत्रो महारथः}
{सप्तवीरान्महेष्वासानग्रानीकेष्वपोथयत्}


\twolineshloka
{अथान्यानपि राजेन्द्र नानाजनपदेश्वरान्}
{शरैरनलसङ्काशैर्निन्ये वीरान्यमक्षयम्}


\threelineshloka
{शतमेकेन विव्याध शतेनैकं च पत्रिणाम्}
{द्विपारोहान्द्विपांश्चैव हयारोहान्हयांस्तथा}
{रथिनः साश्वसूतांश्च जघानेशः पशूनिव}


\twolineshloka
{तं तथाऽद्भुतकर्माणं शरसम्पातवर्षिणम्}
{न केचनाभ्यधावन्वै सात्यकिं तव सैनिकाः}


\twolineshloka
{ते भीता मृद्यमानाश्च प्रमृष्टा दीर्घबाहुना}
{आयोधनं जहुर्वीरा दृष्टा तमतिमानिनम्}


% Check verse!
तमेकं बहुधा पश्यन्मोहितास्तस्य तेजसा
\twolineshloka
{रथैर्विमथितैश्चैव भग्ननीडैश्च मारिष}
{चक्रैर्विमथितैश्छत्रैर्ध्वजैश्च विनिपातितैः}


\twolineshloka
{अनुकर्षैः पताकाभिः शिरस्त्राणैः सकाञ्चनैः}
{बाहुभिश्चन्दनादिग्धैः साङ्गदैश्च विशाम्पते}


\twolineshloka
{हस्तिहस्तोपमैश्चापि भुजङ्गाभोगसन्निभैः}
{ऊरुभिः पृथिवी च्छन्ना मनुजानां नराधिप}


\twolineshloka
{शशाङ्कुसन्निभैश्चैव वदनैश्चारुकुण्डलैः}
{पतितैर्ऋषभाक्षाणां बभौ भारत मेदिनी}


\twolineshloka
{गजैश्च बहुधा छिन्नैः शयानैः पर्वतोपमैः}
{रराजातिभृशं भूमिर्विकीर्णैरिव पर्वतैः}


\threelineshloka
{तपनीयमयैर्योक्त्रैर्मुक्ताजालविभूषितैः}
{उरश्छदैर्विचित्रैश्च व्यशोभन्त तुरङ्गमाः}
{गतसत्वा महीं प्राप्य प्रमृष्टा दीर्घबाहुना}


\twolineshloka
{नानाविधानि सैन्यानि तव हत्वा तु सात्वतः}
{प्रविष्टस्तावकं सैन्यं द्रावयित्वा चमूं भृशम्}


\twolineshloka
{ततस्तेनैव मार्गेण येन यातो धनञ्जयः}
{इयेष सात्यकिर्गन्तुं ततो द्रोणेन वारितः}


\twolineshloka
{भारद्वाजं समासाद्य युयुधानश्च सात्यकिः}
{न न्यवर्तत सङ्क्रुद्धो वेलामिव जलाशयः}


\twolineshloka
{निवार्य तु रणे द्रोणो युयुधानं महारथम्}
{विव्याध निशितैर्बाणैः पञ्चभिर्मर्मभेदिभिः}


\twolineshloka
{सात्यकिस्तु रणे द्रोणं राजन्विव्याध सप्तभिः}
{हेमपुङ्घैः शिलाधौतैः कङ्कबर्हिणवाजितैः}


\twolineshloka
{षं षड्भिः सायकैर्द्रोणः साश्वयन्तारमार्दयत्}
{स तं न ममृषे द्रोणं युयुधानो महारथः}


\twolineshloka
{सिंहनादं ततः कृत्वा द्रोणं विव्याध सात्यकिः}
{दशभिः सायकैश्चान्यैः षड्भिरष्टाभिरेव च}


\threelineshloka
{युयुधानः पुनर्द्रोणं विव्याध दशभिः शरैः}
{एकेन सारथिं चास्य चतुर्भिश्चतुरो हयान्}
{ध्वजमेकेन बाणेन विव्याध युधि मारिष}


\twolineshloka
{तं द्रोणः साश्वयन्तारं सरथध्वजमाशुगैः}
{त्वरन्प्राच्छादयद्बाणैः शलभानामिव व्रजैः}


\twolineshloka
{तथैव युयुधानोऽपि द्रोणं बहुभिराशुगैः}
{`प्राच्छादयदसम्भ्रान्तस्त्वरमाणो महारथः}


\twolineshloka
{द्रोणस्तु परमक्रुद्धः सात्यकिं परवीरहा}
{अवाक्रिच्छितैर्बाणैर्वासुदेवपराक्रमम्}


\twolineshloka
{तं तथा शरजालेन प्रच्छाद्य महता पुनः}
{द्रोणः प्रहस्य शैनेयमिदं वचनमब्रवीत्'}


\twolineshloka
{तवाचार्यो रणं हित्वा गतः कापुरुषो यथा}
{युध्यमानं च मां हित्वा प्रदक्षिणमवर्तत}


\threelineshloka
{त्वं हि मे युध्यतो नाद्य जीवन्यास्यसि माधव}
{यदि मां त्वं रणे हित्वा न यास्याचार्यवद्द्रुतम् ॥सात्यकिरुवाच}
{}


\twolineshloka
{धनञ्जयस्य पदवीं धर्मराजस्य शासनात्}
{गच्छामि स्वस्ति ते ब्रह्मन्न मे कालात्ययो भवेत्}


\threelineshloka
{आचार्यानुगतो मार्गः शिष्यैरन्वास्यते सदा}
{तस्मादेव व्रजाम्याशु यथा मे स गुरुर्गतः ॥सञ्जय उवाच}
{}


\twolineshloka
{एतावदुक्त्वा शैनेय आचार्यं परिवर्जयन्}
{प्रयातः सहसा राजन्सारथिं चेदमब्रवीत्}


\twolineshloka
{द्रोणः करिष्यते यत्नं सर्वथा मम वारणे}
{यत्तो याहि रणे सूत शृणु चेदं वचः परम्}


\twolineshloka
{एतदालोक्यते सैन्यमावन्त्यानां महाप्रभम्}
{अस्यानन्तरतस्त्वेतद्दाक्षिणात्यं महद्बलम्}


\twolineshloka
{तदनन्तरमेतच्च बाह्लिकानां महद्बलम्}
{बाह्लिकाभ्याशतो युक्तं कर्णस्य च महद्बलम्}


\twolineshloka
{अन्योन्येन हि सैन्यानि भिन्नान्येतानि सारथे}
{अन्योन्यं समुपाश्रित्य न त्यक्ष्यन्ति रणाजिरम्}


\twolineshloka
{एतदन्तरमासाद्य चोदयाश्वान्प्रहृष्टवत्}
{मध्यमं जवमास्थाय वह मामत्र सारथे}


\twolineshloka
{बाह्लिका यत्र दृश्यन्ते नानाप्रहरणोद्यताः}
{दाक्षिणात्याश्च बहवः सूतपुत्रपुरोगमाः}


\twolineshloka
{हस्त्यश्वरथसम्बाधं यच्चानीकं विलोक्यते}
{नानादेशसमुत्थैश्च पदातिभिरधिष्ठितम्}


\twolineshloka
{`*ततः शक्यो महाव्यूहो भेत्तुं स सहसा रणे}
{तं देशं त्वरिता यामो मृद्ग्रन्तो युधि शात्रवान्}


% Check verse!
तत्रैते सम्प्रहृष्टत्वान्नास्मान्प्रति युयुत्सवः
\twolineshloka
{अयुध्यमानो बहुभिरेकं प्राप्य सुदुर्बलम्}
{बलं प्रमथ्य गच्छामि मिषतां सर्वधन्विनाम्}


\fourlineindentedshloka
{मध्यतो याहि यत्रोग्रं कर्णस्य सुमहद्बलम्}
{यत्रैते परमक्रुद्धा दाक्षिणात्या महारथाः}
{एतान्विजित्य सङ्ग्रामे ततो यामो धनञ्जयम् ॥सञ्जय उवाच}
{}


\twolineshloka
{युयुधानवचः श्रुत्वा युयुधानस्य सारथिः}
{यथोक्तमगमद्राजन्वर्जयन्द्रोणमाहवे'}


\twolineshloka
{तं द्रोणोऽनुययौ क्रुद्धो विकिरन्विशिखान्बहून्}
{युयुधानं महाभागं गच्छन्तमनिवर्तिनम्}


\twolineshloka
{स च सैन्यं महद्भित्त्वा दाक्षिणात्यबलं च तत्}
{कर्णस्य सैन्यं सुमहदभिहत्य शितैः शरैः}


\twolineshloka
{प्राविशद्भारतीं सेनामपर्यन्तां स सात्यकिः}
{सात्यको हि ततः सैन्यं द्रावयन्स समन्ततः}


\twolineshloka
{प्रविष्टे युयुधाने तु सैनिकेषु द्रुतेषु च}
{अमर्षी कृतवर्मा तु सात्यकिं पर्यवारयत्}


\twolineshloka
{तमापतन्तं विशिखैः षड्भिराहत्य सात्यकिः}
{चतुर्भिश्चतुरोऽस्याश्वानाजघानाशु वीर्यवान्}


\twolineshloka
{ततः पुनः षोडशभिर्नतपर्वभिराशुगैः}
{सात्यकिः कृतवर्माणं प्रत्यविध्यत्स्तनान्तरे}


\twolineshloka
{स ताड्यमानो विशिखैर्बहुभिस्तिग्मतेजनैः}
{सात्वतेन महाराज कृतवर्मा न चक्षमे}


\twolineshloka
{स वत्सन्दतं सन्धाय तिग्मांश्वनलसप्रभम्}
{आकृष्य राजन्नाकर्णाद्विव्याधोरसि सात्यकिम्}


\twolineshloka
{स तस्य देहावरणं भित्त्वा देहं च सायकः}
{सपुङ्खपत्रः पृथिवीं विवेश रुधिरोक्षितः}


\twolineshloka
{अथास्य बहुभिर्बाणैरच्छिनत्परमास्त्रवित्}
{स मार्गणगणं राजन्कृतवर्मा शरासनम्}


\twolineshloka
{विव्याध च रणे राजन्सात्यकिं सत्यविक्रमम्}
{दशभिर्विशिखैस्तीक्ष्णैरभिक्रुद्धः स्तनान्तरे}


\twolineshloka
{ततः प्रशीर्णे धनुषि शक्त्या शक्तिमतां वरः}
{जघान दक्षिणं बहुं सात्यकिः कृतवर्मणः}


\twolineshloka
{ततोऽन्यत्सुदृढं चापं पूर्णमायम्य सात्यकिः}
{व्यसृजद्विशिखांस्तूर्णं शतशोऽथ सहस्रशः}


\threelineshloka
{सरथं कृतवर्माणं समन्तात्पर्यवारयत्}
{छादयित्वा रणे राजन्हार्दिक्यं स तु सात्यकिः}
{अथास्य भल्लेन शिरः सारथेः समकृन्तत}


\twolineshloka
{स पपात हतः सूतो हार्दिक्यस्य महारथात्}
{ततस्ते यन्तृरहिताः प्राद्रवंस्तुरगा भृशम्}


\twolineshloka
{अथ भोजस्तु सम्भ्रान्तो निगृह्य तुरगामन्स्वयम्}
{तस्थौ वीरो धनुष्पाणिस्तत्सैन्यान्यभ्यपूजयन्}


\twolineshloka
{स मुहूर्तमिवाश्वास्य सदश्वान्समनोदयत्}
{व्यपेत भीरमित्राणामावहत्सुमहद्भयम्}


% Check verse!
सात्यकिश्चात्यगात्तस्मात्स तु भीममुपाद्रवत्
\twolineshloka
{युयुधानोऽपि राजेन्द्र भोजानीकाद्विनिःसृतः}
{प्रययौ त्वरितस्तूर्णं काम्भोजानां महाचमूम्}


\twolineshloka
{स तत्र बहुभिः शूरैः सन्निरुद्धो महारथैः}
{न चचाल तदा राजन्सात्यकिः सत्यविक्रमः}


\twolineshloka
{सन्धाय च चमूं द्रोणो भोजे भारं निवेश्य च}
{अभ्यधावद्रणे यत्तो युयुधानं युयुत्सया}


\threelineshloka
{तथा तमनुधावन्तं युयुधानस्य पृष्ठतः}
{न्यवारयन्त संहृष्टाः पाण्डुसैन्ये ब्रृहत्तमाः}
{समासाद्य तु हार्दिक्यं रथानां प्रवरं रथम्}


\twolineshloka
{पाञ्चाला विगतोत्साहा भीमसेनपुरोगमाः}
{विक्रम्य वारिता राजन्वीरेण कृतवर्मणा}


\twolineshloka
{यतमानांश्च तान्सर्वानीषद्विगतचेतसः}
{अभितस्ताञ्शरौघेण क्लान्तवाहानकारयत्}


\threelineshloka
{निगृहीतास्तु भोजेन भोजानीकेप्सवो रणे}
{अतिष्ठन्नार्यवद्वीराः प्रार्थयन्तो महद्यशः}
{`हार्दिक्यं समरे यत्ता न शेकुः प्रतिवीक्षितुम्'}


\chapter{अध्यायः ११४}
\twolineshloka
{धृतराष्ट्र उवाच}
{}


\twolineshloka
{एवं बहुगुणं सैन्यमेवं प्रविचितं बलम्}
{व्यूढमेवं यथान्यायमेवं बहु च सञ्जय}


\twolineshloka
{नित्यं पूजितमस्माभिरभिकामं च नः सदा}
{प्रौढमत्यद्भुताकारं पुरस्ताद्दृष्टविक्रमम्}


\twolineshloka
{नातिवृद्वमबालं च नाकृशं नातिपीवरम्}
{लघुवृत्तायतप्रायं सारगात्रमनामयम्}


\twolineshloka
{आत्तसन्नाहसञ्छन्नं बहुशस्त्रपरिच्छदम्}
{शस्त्रग्रहणविद्यासु बह्वीषु परिनिष्ठितम्}


\twolineshloka
{आरोहे पर्यवस्कन्दे सरणे सान्तरप्लुते}
{सम्यक्प्रहरणे याने व्यपयाने च कोविदम्}


\threelineshloka
{नागेष्वश्वेषु बहुशो रथेषु च परीक्षितम्}
{`चर्मनिस्त्रिंशयुद्धे च नियुद्धे च विशारदम्'}
{परीक्ष्य च यथान्यायं वेतनेनोपपादितम्}


\twolineshloka
{न गोष्ठ्या नोपकारेण न सम्बन्धनिमित्ततः}
{नानाहूतं नाप्यभृतं मम सैन्यं बभूव ह}


\twolineshloka
{कुलीनार्यजनोपेतं तुष्टपुष्टमनुद्धतम्}
{कृतमानोपचारं च यशस्वि च मनस्वि च}


\twolineshloka
{सचिवैश्चापरैर्मुख्यैर्बहुभिः पुण्यकर्मभिः}
{लोकपालोपमैस्तात पालितं नरसत्तमैः}


\twolineshloka
{बहुभिः पार्थिवैर्गुप्तमस्मत्प्रियचिकीर्षुभिः}
{अस्मानभिसृतैः कामात्सबलैः सपदानुगैः}


\twolineshloka
{महोदधिमिवापूर्णमापगाभिः समन्ततः}
{अपक्षैः पक्षिसङ्काशै रथैरश्वैश्च संवृतम्}


\twolineshloka
{प्रभिन्नकरटैश्चैव द्विरदैरावृतं महत्}
{यदहन्यत मे सैन्यं किमन्यद्भागधेयतः}


\twolineshloka
{योधाक्षय्यजलं भीमं वाहनोर्मितरङ्गिणम्}
{क्षेपण्यसिगदाशक्तिशरप्रासझषाकुलम्}


\twolineshloka
{ध्वजभूषणसम्बाधरत्नोपलसुसञ्चितम्}
{वाहनैरभिधावद्भिर्वायुवेगविकम्पितम्}


\twolineshloka
{द्रोणगम्भीरपातालं कृतवर्ममहाहदम्}
{जलसन्धमहाग्राहं कर्मचन्द्रोदयोद्धतम्}


\twolineshloka
{गते सैन्यार्णवं भित्त्वा तरसा पाण्डवर्षभे}
{सञ्जयैकरथेनैव युयुधाने च मामकम्}


\twolineshloka
{तत्र शेषं न पश्यामि प्रविष्टे सव्यसाचिनि}
{सात्वते च रथोदारे मम सैन्यस्य सञ्जय}


\twolineshloka
{तौ तत्र समतिक्रान्तौ दृष्ट्वाऽतीव तरस्विनौ}
{सिन्धुराजं तु सम्प्रेक्ष्य गाण्डीवस्य च गोचरम्}


\twolineshloka
{किं नु वा कुरवः कृत्यं विदधुः कालचोदिताः}
{दारुणैकायनेऽकाले कथं वा प्रतिपेदिरे}


\twolineshloka
{ग्रस्तान्हि कौरवान्मन्ये मृत्युना तात सङ्गतान्}
{विक्रमोऽपि रणे तेषां न तथा दृश्यते हि वै}


\twolineshloka
{अक्षतौ संयुगे तत्र प्रविष्टौ कृष्णपाण्डवौ}
{न च वारयिता कश्चित्तयोरस्तीह सञ्जय}


\twolineshloka
{भृताश्च बहवो योधाः परीक्ष्यैव महारथाः}
{वेतनेन यथायोगं प्रियवादेन चापरे}


\twolineshloka
{असत्कारभृतस्तात मम सैन्ये न विद्यते}
{कर्मणा ह्यनुरूपेण लभ्यते भक्तवेतनम्}


\twolineshloka
{न च योधोऽभवत्कश्चिन्मम सैन्ये तु सञ्जय}
{अल्पदानभृतस्तात तथा चाभृतको नरः}


\twolineshloka
{पूजितो हि यथाशक्त्या दानमानासनैर्मया}
{तथा पुत्रैश्च मे तात ज्ञातिभिश्च सबान्धवैः}


\twolineshloka
{ते च प्राप्यैव सङ्ग्रामे निर्जिताः सव्यसाचिना}
{शैनेयेन परामृष्टाः किमन्यद्भागधेयतः}


\twolineshloka
{रक्ष्यते यश्च सङ्ग्रामे ये च सञ्जय रक्षिणः}
{एकः साधारणः पन्थार रक्ष्यस्य सह रक्षिभिः}


\twolineshloka
{अर्जुनं समरे दृष्ट्वा सैन्धवस्याग्रतः स्थितम्}
{पुत्रो मम भृशं मूढः किं कार्यं प्रत्यपद्यत}


\twolineshloka
{सात्यकिं च रणे दृष्ट्वा प्रविशन्तमभीतवत्}
{किं नु दुर्योधनः कृत्यं प्राप्तकालममन्यत}


\twolineshloka
{सर्वशस्त्रातिगौ सेनां प्रविष्टौ रथिसत्तमौ}
{दृष्ट्वा कां वै धृतिं युद्धे प्रत्यपद्यन्त मामकाः}


\twolineshloka
{दृष्ट्वा कृष्णं तु दाशार्हमर्जुनार्थे व्यवस्थितम्}
{शिनीनामृषभं चैव मन्ये शोचन्ति पुत्रकाः}


\twolineshloka
{दृष्टा सेनां व्यतिक्रान्तां सात्वतेनार्जुनेन च}
{पलायमानांश्च कुरून्मन्ये शोचन्ति पुत्रकाः}


\twolineshloka
{विद्रुतान्रथिनो दृष्ट्वा निरुत्साहान्द्विषज्जये}
{पलायनकृतोत्साहान्मन्ये शोचन्ति पुत्रकाः}


\twolineshloka
{शून्यान्कृतान्रथोपस्थान्सात्वतेनार्जुनेन च}
{हतांश्च योधान्संदृश्य मन्ये शोचन्ति पुत्रकाः}


\twolineshloka
{व्यश्वनागरथान्दृष्ट्वा तत्र वीरान्सहृस्रशः}
{धावमानान्रणे व्यग्रामन्यन्ये शोचन्ति पुत्रकाः}


\twolineshloka
{महानागान्विद्रवतो दृष्ट्वाऽर्जुनशराहतान्}
{पतितान्पततश्चान्यान्मन्ये शोचन्ति पुत्रकाः}


\twolineshloka
{विहीनांश्च कृतानश्वान्विरथांश्च कृतान्नरान्}
{तत्र सात्यकिपार्थाभ्यां मन्ये शोचन्ति पुत्रकाः}


\twolineshloka
{हयौघान्निहतान्दृष्ट्वा द्रवमाणांस्ततस्ततः}
{रणे माधवपार्थाभ्यां मन्ये शोचन्ति पुत्रकाः}


\twolineshloka
{पत्तिसङ्घान्रणे दृष्ट्वा धावमानांश्च सर्वशः}
{निराशा विजये सर्वे मन्ये शोचन्ति पुत्रकाः}


\twolineshloka
{द्रोणस्य समतिक्रान्तावनीकमपराजितौ}
{क्षणेन दृष्ट्वा तौ वीरौ मन्ये शोचन्ति पुत्रकाः}


\twolineshloka
{सम्मूढोऽस्मि भृशं तात श्रुत्वा कृष्णधनञ्जयौ}
{प्रविष्टौ मामकं सैन्यं सात्वतेन सहाच्युतौ}


\twolineshloka
{तस्मिन्प्रविष्टे पृतनां शिनीनां प्रवरे रथे}
{भोजानीकं व्यतिक्रान्ते किमकुर्वत कौरवाः}


\twolineshloka
{तथा द्रोणेन समरे निगृहीतेषु पाण्डुषु}
{कथं युद्धमभूत्तत्र तन्ममाचक्ष्व सञ्जय}


\twolineshloka
{द्रोणो हि बलवान्श्रेष्ठः कृतास्त्रो युद्धदुर्मदः}
{पाञ्चालास्तं महेष्वासं प्रत्ययुध्यन्कथं रणे}


\twolineshloka
{बद्धवैरास्ततो द्रोणे धनञ्जयजयैषिणः}
{भारद्वाजसुतस्तेषु दृढवैरो महारथः}


\threelineshloka
{अर्जुनश्चापि यच्चक्रे सिन्धुराजवधं प्रति}
{तन्मे सर्वं समाचक्ष्व कुशलो ह्यसि सञ्जय ॥सञ्जय उवाच}
{}


\twolineshloka
{आत्मापराधात्सम्भूतं व्यसनं भरतर्षभ}
{प्राप्य प्राकृतवद्वीर न त्वं शोचितुमर्हसि}


\twolineshloka
{पुरा यदुच्यसे प्राज्ञैः सुहृद्भिर्विदुरादिभिः}
{मा हार्षीः पाण्डवान्राजन्निति तन्न त्वया श्रुतम्}


\twolineshloka
{सुहृदां हितकामानां वाक्यं यो न शृणोति ह}
{स महद्व्यसनं प्राप्य शोचते वै यथा भवान्}


\twolineshloka
{याचितोऽसि पुरा राजन्दाशार्हेण शमं प्रति}
{न च तं लब्धवान्कामं त्वत्तः कृष्णो महायशाः}


\threelineshloka
{तव निर्गुणतां ज्ञात्वा पक्षपातं सुतेषु च}
{द्वैधीभावं तथा धर्मे पाण्डवेषु च मत्सरम् ॥ 5-114-52 तवजिह्ममभिप्रायं विदित्वा पाण्डवान्प्रति}
{आर्तप्रलापांश्च बहून्मनुजाधिप सत्तम}


\twolineshloka
{सर्वलोकस्य तत्त्वज्ञः सर्वलोकेश्वरः प्रभुः}
{वासुदेवस्ततो युद्धं कुरूणामकरोन्महत्}


\twolineshloka
{आत्मापराधात्सुमहान्प्राप्तस्ते विपुलः क्षयः}
{नैनं दुर्योधने दोषं कर्तुमर्हसि मानद}


\twolineshloka
{न हि तै सुकृतं किञ्चिदादौ मध्ये च भारत}
{दृश्यते पृष्ठतश्चैव त्वन्मूलो हि पराजयः}


\twolineshloka
{तस्मादद्य स्थिरो भूत्वा ज्ञात्वा लोकस्य निर्णयम्}
{शृणु युद्धं यथावृत्तं घोरं देवासुरोपमम्}


\twolineshloka
{प्रविष्टे तव सैन्यं तु शैनेये सत्यविक्रमे}
{भीमसेनमुखाः पार्थाः प्रतीयुर्वाहिनीं तव}


\twolineshloka
{आगच्छतस्तान्सहसा क्रुद्धरूपान्सहानुगान्}
{दधारैको रणे पाण्डून्कृतवर्मा महारथः}


\twolineshloka
{यथोद्वृत्तं वारयते वेला वै सलिलार्णवम्}
{पाण्डुसैन्यं तथा सङ्ख्ये हार्दिक्यः समवारयत्}


\twolineshloka
{तत्राद्भुतमपश्याम हार्दिक्यस्य पराक्रमम्}
{यदेनं सहिताः पार्था नातिचक्रमुराहवे}


\twolineshloka
{ततो भीमस्त्रिभिर्विद्ध्वा कृतवर्माणमाशुगैः}
{शङ्खं दध्मौ महाबाहुर्हर्षुयन्सर्वपाण्डवान्}


\twolineshloka
{सहदेवस्तु विंशत्या धर्मराजश्च पञ्चभिः}
{शतेन नकुलश्चापि हार्दिक्यं समविध्यत}


\twolineshloka
{द्रौपदेयास्त्रिसप्तत्या सप्तभिश्च घटोत्कचः}
{धृष्टद्युम्नस्त्रिभिश्चापि कृतवर्माणमार्दयत्}


\twolineshloka
{विराटो द्रुपदश्चैव याज्ञसेनिश्च पञ्चभिः}
{शिखण्डी चैव हार्दिक्यं विद्ध्वा पञ्चभिराशुगैः}


\twolineshloka
{पुनर्विव्याध विंशत्या सायकानां हसन्निव}
{कृतवर्मा ततो राजन्सर्वतस्तान्महारथान्}


\twolineshloka
{एकैकं पञ्चभिर्विद्ध्वा भीमं विव्याध सप्तभिः}
{धनुर्ध्वजं चास्य तदा रथाद्भूमावपातयत्}


\twolineshloka
{अथैनं छिन्नधन्वानं त्वरमाणो महारथः}
{आजघानोरसि क्रुद्धः सप्तत्या निशितैः शरैः}


\twolineshloka
{स गाढविद्धो बलवान्हार्दिक्यस्य शरोत्तमैः}
{चचाल रथमध्यस्थः क्षितिकम्पे यथाऽचलः}


\twolineshloka
{भीमसेनं तथा दृष्ट्वा धर्मराजपुरोगमाः}
{विसृजन्तः शरान्राजन्कृतवर्माणमार्दयन्}


\twolineshloka
{तं तथा कोष्ठकीकृत्य रथवंशेन मारिष}
{विव्यधुः सायकैर्हृष्टा रक्षार्थं मारुतेर्मृधे}


\twolineshloka
{प्रतिलभ्य ततः संज्ञां भीमसेनो महाबलः}
{शक्तिं जग्राह समरे हेमदण्डामयस्मायीम्}


\twolineshloka
{चिक्षेप च रथात्तूर्णं कृतवर्मरथं प्रति}
{सा भीमभुजनिर्मुक्ता निर्मुक्तोरगसन्निभा}


\twolineshloka
{कृतवर्माणमभितः प्रजज्वाल सुदारुणा}
{तामापतन्तीं सहसा युगान्ताग्निसमप्रभाम्}


\twolineshloka
{द्वाभ्यां शराभ्यां हार्दिक्यो निजघान द्विधा तदा}
{सा छिन्ना पतिता भूमौ शक्तिः कनकभूषणा}


\twolineshloka
{द्योतयन्ती दिशो राजन्महोल्केव नभश्च्युता}
{शक्तिं विनिहतां दृष्ट्वा भीमश्चुक्रोध वै भृशम्}


\twolineshloka
{ततोऽन्यद्धनुरादाय वेगवत्सुमहास्वनम्}
{भीमसेनो रणे क्रुद्धो हार्दिक्यं समवारयत्}


\twolineshloka
{अथैनं पञ्चभिर्बाणैराजघान स्तनान्तरे}
{भीमो भीमबलो राजंस्तव दुर्मन्त्रितेन च}


\twolineshloka
{भोजस्तु क्षतसर्वाङ्गो भीमसेनेन मारिष}
{रक्ताशोक इवोत्फुल्लो व्यभ्राजत रणाजिरे}


\twolineshloka
{ततः क्रुद्धस्त्रिभिर्बाणैर्भीमसेनं हसन्निव}
{अभिहत्य दृढं युद्धे तान्सर्वान्प्रत्यविध्यत}


\twolineshloka
{त्रिभिस्त्रिभिर्महेष्वासो यतमानान्महारथान्}
{तेऽपि तं प्रत्यविध्यन्त सप्तभिः सप्तभिः शरैः}


\twolineshloka
{शिखण्डिनस्ततः क्रुद्धः क्षुरप्रेण महारथः}
{धनुश्चिच्छेद सामरे प्रहसन्निव सात्वतः}


\twolineshloka
{शिखण्डी तु ततः क्रुद्धश्छिन्ने धनुषि सत्वरः}
{असिं जग्राह समरे शतचन्द्रं च भास्वरम्}


\twolineshloka
{भ्रामयित्वा महच्चर्म चामीकरविभूषितम्}
{तमसिं प्रेषयामात कृतवर्मरथं प्रति}


\twolineshloka
{स तस्य सशरं चापं छित्त्वा राजन्महानसिः}
{अभ्यगाद्धरणीं राजंश्च्युतं ज्योतिरिवाम्बरात्}


\twolineshloka
{एतस्मिन्नेव काले तु त्वरमाणं महारथाः}
{विव्यधुः सायकैर्गाढं कृतवर्माणमाहवे}


\twolineshloka
{अथान्यद्धनुरादाय त्यक्त्वा तच्च महद्धनुः}
{विशीर्णं भरतश्रेष्ठ हार्दिक्यः परवीरहा}


\twolineshloka
{विव्याध पाण्डवान्युद्धे त्रिभिस्त्रिभिरजिह्मगैः}
{शिखण्डिनं च विव्याध त्रिभिः पञ्चभिरेव च}


\twolineshloka
{धनुरन्यत्समादाय शिखण्डी तु महायशाः}
{अवारयत्कूर्मनखैराशुगैर्हृदिकात्मजम्}


\twolineshloka
{ततः क्रुद्धो रणे राजन्हृदिकस्यात्मसम्भवः}
{अभिदुद्राव वेगेन याज्ञसेनिं महारथम्}


\twolineshloka
{भीष्मस्य समरे राजन्मृत्योर्हेतुं महात्मनः}
{विदर्शयन्बलं शूरः शार्दूल इव कुञ्जरम्}


\twolineshloka
{तौ दिशागजसङ्काशौ ज्वलिताविव पावकौ}
{समापेततुरन्योन्यं शरसङ्घैररिन्दमौ}


\twolineshloka
{विधुन्वानौ धनुःश्रेष्ठे सन्दधानौ च सायकान्}
{विसृजन्तौ च शतशो गभस्तीनिव भास्करौ}


\twolineshloka
{तापयन्तौ शरैस्तीक्ष्णैरन्योन्यं तौ महारथौ}
{युगान्तप्रतिमौ वीरौ रेजतुर्भास्कराविव}


\twolineshloka
{कृतवर्मा च समरे याज्ञसेनिं महारथम्}
{विद्ध्वेषुभिस्त्रिसप्तत्या पुनर्विव्याध सप्तभिः}


\twolineshloka
{स गाढविद्धो व्यथितो रथोपस्थ उपाविशत्}
{विसृज्य सशरं चापं मूर्च्छयाऽभिपरिप्लुतः}


\twolineshloka
{तं विषण्णं रणे दृष्ट्वा तावकाः पुरुषर्षभ}
{हार्दिक्यं पूजयामासुर्वासांस्यादुधुवुश्च ह}


\twolineshloka
{शिखण्डिनं तथा ज्ञात्वा हार्दिक्यशरपीडितम्}
{अपोवाह रणाद्यन्ता त्वरमाणो महारथम्}


\twolineshloka
{सादितं तु रथोपस्थे दृष्ट्वा पार्थाः शिखण्डिनम्}
{परिवव्रू रथैस्तूर्णं कृतवर्माणमाहवे}


\twolineshloka
{तत्राद्भुतं परं चक्रे कृतवर्मा महारथः}
{यदेकः समरे पार्थान्वारयामास सानुगान्}


\twolineshloka
{पार्थाञ्जित्वाऽजयच्चेदीन्पाञ्चालान्सृञ्जयानपि}
{केकयांश्च महावीर्यान्कृतवर्मा महारथः}


\twolineshloka
{ते वध्यमानाः समरे हार्दिक्येन स्म पाण्डवाः}
{इतश्चेतश्च धावन्तो नैव चक्रुर्धृतिं रणे}


\twolineshloka
{जित्वा पाण्डुसुतान्युद्धे भीमसेनपुरोगमान्}
{हार्दिक्यः समरेऽतिष्ठद्विधूम इव पावकः}


\twolineshloka
{ते द्राव्यमाणाः समरे हार्दिक्येन महारथाः}
{विमुखाः समपद्यन्त शरवृष्टिभिरार्दिताः}


\chapter{अध्यायः ११५}
\twolineshloka
{सञ्जय उवाच}
{}


\twolineshloka
{शृणुष्वैकमना राजन्यन्मां त्वं परिपृच्छसि}
{द्राव्यमाणे बले तस्मिन्हार्दिक्येन महात्मना}


\twolineshloka
{लज्जयावनते चापि परिहृष्टे च तावके}
{द्वीपो य आसीत्पाण्डूनामगाधे गाधमिच्छताम्}


\twolineshloka
{श्रुत्वा स निनदं भीमं तावकानां महाहवे}
{शैनेयस्त्वरितो राजन्कृतवर्माणमभ्ययात्}


\twolineshloka
{उवाच सारथिं तत्र क्रोधामर्षसमन्वितः}
{हार्दिक्याभिमुखं सूत कुरु मे रथमुत्तमम्}


\twolineshloka
{कुरुते कदनं पश्य पाण्डुसैन्ये ह्यमर्षितः}
{एनं जित्वा पुनः सूत यास्यामि विजयं प्रति}


\twolineshloka
{एवमुक्ते तु वचने सूतस्तस्य महामते}
{निमेषान्तरमात्रेण कृतवर्माणमभ्ययात्}


\twolineshloka
{कृतवर्मा तु हार्दिक्यः शैनेयं निशितैः शरैः}
{अवाकिरत्सुसंक्रुद्धस्ततोऽक्रुद्ध्यत्स सात्यकिः}


\twolineshloka
{अथाशु निशितं भल्लुं शैनेयः कृतवर्मणः}
{प्रेषयामास समरे शरांश्च चतुरोऽपरान्}


\twolineshloka
{ते तस्य जघ्निरे वाहान्भल्लेनास्याच्छिनद्धनुः}
{पृष्ठरक्षं तथा सूतमविध्यन्निशितैः शरैः}


\twolineshloka
{ततस्तं विरथं कृत्वा सात्यकिः सत्यविक्रमः}
{सेनामस्यार्दयामास शरैः सन्नतपर्वभिः}


\threelineshloka
{अभज्यताथ पृतना शैनेयशरपीडिता}
{`पलायनकृतोत्साहा भ्रमन्ती तत्रतत्र ह'}
{ततः प्रायात्स त्वरितः सात्यकिः सत्यविक्रमः}


\twolineshloka
{शृणु राजन्यदकरोत्तव सैन्येषु वीर्यवान्}
{अतीत्य स महाराज द्रोणानीकमहार्णवम्}


\twolineshloka
{पराजित्य तु संहृष्टः कृतवर्माणमाहवे}
{यन्तारमब्रवीच्छूरः शनैर्याहीत्यसम्भ्रमम्}


\twolineshloka
{दृष्ट्वा तु तव तत्सैन्यं रथाश्वद्वि पसङ्कुलम्}
{पदातिजनसम्पूर्णमब्रवीत्सारथिं पुनः}


\twolineshloka
{यदेतन्मेघसङ्काशं द्रोणानीकस्य सव्यतः}
{सुमहत्कुञ्जरानीकं यस्य रुक्मरथो मुखम्}


\twolineshloka
{एते हि बहवः सूत दुर्निवार्याश्च संयुगे}
{दुर्योधनसमादिष्टा मदर्थे त्यक्तजीविताः}


\twolineshloka
{राजपुत्रा महेष्वासाः सर्वे विक्रान्तयोधिनः}
{`न चाजित्वा रणे ह्येताञ्शक्यः प्राप्तुं जयद्रथः}


\twolineshloka
{नापि सूत मया पार्थः शक्यः प्राप्तुं कथञ्चन}
{एते तिष्ठन्ति सहिताः सर्वविद्यासु निष्ठिताः'}


\twolineshloka
{त्रिगर्तानां रथोदाराः सुवर्णविकृतध्वजाः}
{मामेवाभिमुखा वीरा योत्स्यमाना व्यवस्थिताः}


\threelineshloka
{अत्र मां प्रापय क्षिप्रमश्वांश्चोदय सारथे}
{त्रिगर्तैः सह योत्स्यामि भारद्वाजस्य पश्यतः ॥सञ्जय उवाच}
{}


\twolineshloka
{ततः प्रायाच्छनैः सूतः सात्वतस्य मते स्थितः}
{रथेनादित्यवर्णेन भास्वरेण पताकिना}


\twolineshloka
{तमूहुः सारथेर्वश्या वल्गमाना हयोत्तमाः}
{वायुवेगसमाः सङ्ख्ये कुन्देन्दुरजतप्रभाः}


\threelineshloka
{आपतन्तं रणे तं तु शङ्खवर्णैर्हयोत्तमैः}
{परिववुस्ततः शूरा गजानीकेन सर्वतः}
{किरन्तो विविधांस्तीक्ष्णान्सायकाँल्लघुवेधिनः}


\twolineshloka
{सात्वतोऽपि शितैर्बाणैर्गजानीकमयोधयत्}
{पर्वतानिव वर्षेण तपान्ते जलदो महान्}


\twolineshloka
{वज्राशनिसमस्पर्शैर्वध्यमानाः शरैर्गजाः}
{प्राद्रवन्रणमुत्सृज्य शिनिवीरसमीरितैः}


\twolineshloka
{शीर्णदन्ता विरुधिरा भिन्नमस्तकपिण्डिकाः}
{विशीर्णकर्णास्यकरा विनियन्तृपताकिनः}


\twolineshloka
{सम्भिन्नवर्मघण्टाश्च विनिकृत्तमहाध्वजाः}
{हतारोहा दिशो राजन्भेजिरे भ्रष्टकम्बलाः}


\twolineshloka
{रुवन्तो विविधान्नादाञ्जलदोपमनिःस्वनाः}
{नारचैर्वत्सदन्तैश्च भल्लैरञ्जलिकैस्तथा}


\threelineshloka
{क्षुरप्रैरर्धचन्द्रैश्च सात्वतेन विदारिताः}
{क्षरन्तोऽसृक्तथा मूत्रं पुरीषं च प्रदुद्रुवुः}
{बभ्रुमुश्चस्खलुश्चान्ये पेतुर्मम्लुस्तथाऽपरे}


\twolineshloka
{एवं तत्कुञ्जरानीकं युयुधानेन पीडितम्}
{शरैरग्न्यर्कसङ्काशैः प्रदुद्राव समन्ततः}


\twolineshloka
{तस्मिन्हते गजानीके जलसन्धो महाबलः}
{यत्तः सम्प्रापयन्नागं रजताश्वरथं प्रति}


\twolineshloka
{रुक्मवर्मधरः शूरस्तपनीयाङ्गदः शुचिः}
{कुण्डली मकुटी खङ्गी रक्तचन्दनरूषितः}


\twolineshloka
{शिरसा धारयन्दीप्तां तपनीयमयीं स्रजम्}
{उरसा धारयन्निष्कं कण्ठसूत्रं च भास्वरम्}


\twolineshloka
{चापं च रुक्मविकृतं विधुन्वन्गजमूर्धनि}
{अशोभत महाराज सविद्युदिव तोयदः}


\twolineshloka
{तमापतन्तं सहसा मागधस्य गजोत्तमम्}
{सात्यकिर्वारयामास वेलेव मकरालयम्}


\twolineshloka
{नागं निवारितं दृष्ट्वा शैनेयस्य शरोत्तमैः}
{अक्रुध्यत रणे राजञ्जलसन्धो महाबलः}


\twolineshloka
{ततः क्रुद्धो महाराज मार्गणैर्भारसाधनैः}
{अविध्यत शिनेः पौत्रं जलसन्धो महोरसि}


\twolineshloka
{ततोऽपरेण भल्लेन पीतेन निशितेन च}
{अस्यतो वृष्णिवीरस्य निचकर्त शरासनम्}


\twolineshloka
{सात्यकिं छिन्नधन्वानां प्रहसन्निव भारत}
{अविध्यन्मागधो वीरः पञ्चभिर्निशितैः शरैः}


\twolineshloka
{स विद्धो बहुभिर्बाणैर्जलसन्धेन वीर्यवान्}
{नाकम्पत महाबाहुस्तदद्भुतमिवाभवत्}


\twolineshloka
{अचिन्तयन्वै स शरान्नात्यर्थं सम्भ्रमाद्बली}
{धनुरन्यत्समादाय तिष्ठतिष्ठेत्युवाच ह}


\twolineshloka
{एतावदुक्त्वा शैनेयो जलसन्धं महोरसि}
{विव्याध षष्ट्या सुभृशं शराणां प्रहसन्निव}


\twolineshloka
{क्षुरप्रेण सुतीक्ष्णेन मुष्टिदेशे महद्धनुः}
{जलसन्धस्य चिच्छेद विव्याध च त्रिभिः शरैः}


\twolineshloka
{जलसन्धस्तु तत्त्यक्त्वा सशरं वै शरासनम्}
{तोमरं व्यसृजत्तूर्णं सात्यकिं प्रति मारिष}


\twolineshloka
{स निर्भिद्य भुजं सव्यं माधवस्य महारणे}
{अभ्यगाद्धरणीं घोरः श्वसन्निव महोरगः}


\twolineshloka
{निर्भिन्ने तु भुजे सव्ये सात्यकिः सत्यविक्रमः}
{त्रिंशद्भिर्विशिखैस्तीक्ष्णैर्जलसन्धमताडयत्}


\threelineshloka
{प्रगृह्य तु ततः खङ्गं जलसन्धो महाबलः}
{आर्षभं चर्म च महच्छतचन्द्रकसङ्कुलम्}
{आविध्य च ततः खङ्गं सात्वतायोत्ससर्ज ह}


\twolineshloka
{शैनेयस्य धनुश्छित्त्वा स खङ्गो न्यपतन्महीम्}
{अलातचक्रवच्चैव व्यरोचत महीं गतः}


\twolineshloka
{अथान्यद्धनुरादाय सर्वकायावदारणम्}
{`जलसन्धमभिप्रेक्ष्य उत्स्मयित्वा च माधवः'}


\twolineshloka
{शालस्कन्धप्रतीकाशमिन्द्राशनिसमस्वनम्}
{विष्फार्य विव्यधे क्रुद्धो जलसन्धं शरेण ह}


\twolineshloka
{ततः साभरणौ बाहू क्षुराभ्यां माधवोत्तमः}
{साङ्गदौ जलसन्धस्य चिच्छेद प्रहसन्निव}


\twolineshloka
{तौ बाहू परिघप्रख्यौ पेततुर्गजसत्तमात्}
{वसुन्धराधराद्धष्टौ पञ्चशीर्षाविवोरगौ}


\twolineshloka
{ततः सुदंष्ट्रं सुहनु चारुकुण्डलमण्डितम्}
{क्षुरेणास्य तृतीयेन शिरश्चिच्छेद सात्यकिः}


\twolineshloka
{तत्पातितशिरोबाहुकबन्धं भीमदर्शनम्}
{द्विरदं जलसन्धस्य रुधिरेणाभ्यषिञ्चत}


\twolineshloka
{जलसन्धं निहत्याजौ त्वरमाणस्तु सात्वतः}
{विमानं पातयामास गजस्कन्धाद्विशांपते}


\twolineshloka
{रुधिरेणावसिक्ताङ्गौ जलसन्धस्य कुञ्जरः}
{विलम्बमानमवहत्संश्लिष्टं परमासनम्}


\twolineshloka
{शरार्दितः सात्वतेन मर्दमानः स्ववाहिनीम्}
{घोरमार्तस्वरं कृत्वा विदुद्राव महागजः}


\twolineshloka
{हाहाकारो महानासीत्तव सैन्यस्य मारिष}
{जलसन्धं हतं दृष्ट्वा वृष्णीनामृपभेण तु}


\twolineshloka
{विमुखाश्चाभ्यधावन्त तव योधाः समन्ततः}
{पलायनकृतोत्साहा निरुत्साहा द्विषञ्जये}


\twolineshloka
{एतस्मिन्नन्तरे राजन्द्रोणः शस्त्रभृतां वरः}
{`सन्न्यस्य भारं सुमहत्कृतवर्मणि भारत}


\twolineshloka
{अभ्ययाज्जवनैरश्वैर्युयुधानं महारथम्}
{स हि पार्थान्रणे यत्तान्दधारैको महाबलः'}


\twolineshloka
{तमुदीर्णं तथा दृष्ट्वा शैनेयं कुरुपुङ्गवाः}
{द्रोणेनैव सह क्रुद्धा सात्यकिं समुपाद्रवन्}


\twolineshloka
{ततः प्रववृते युद्धं कुरूणां सात्वतस्य च}
{द्रोणस्य च रणे राजन्घोरं देवासुरोपमम्}


\chapter{अध्यायः ११६}
\twolineshloka
{सञ्जय उवाच}
{}


\twolineshloka
{`सात्वतस्य रणे कर्म दृष्ट्वा योद्धुं महारथाः}
{अमृष्यमाणाः संयत्ताः शैनेयं समुपाद्रवन्'}


\twolineshloka
{ते किरन्तः शरव्रातान्सर्वे यत्ताः प्रहारिणः}
{त्वरमाणा महाराज युयुधानमयोधयन्}


\twolineshloka
{तं द्रोणः सप्तसप्त्या जघान निशितैः शरैः}
{दुर्मर्षणो द्वादशभिर्दुःसहो दशभिः शरैः}


\twolineshloka
{विकर्णश्चापि निशितैस्त्रिंशद्भिः कङ्कपत्रिभिः}
{विव्याध मार्गणैस्तूर्णं वामपार्श्वे स्तनान्तरे}


\twolineshloka
{दुर्मुखो दशभिर्बाणैस्तथा दुःशासनोऽष्टभिः}
{चित्रसेनश्च शैनेयं द्वाभ्यां विव्याध मारिष}


\twolineshloka
{दुर्योधनश्च महता शरवर्षेण माधवम्}
{अपीडयद्रमे राजञ्शूराश्चान्ये महारथाः}


\twolineshloka
{सर्वतः प्रतिविद्धस्तु तव पुत्रैर्महारथैः}
{`स तुद्यमानश्च तदा सर्वतः कुरुपुङ्गवैः'}


\twolineshloka
{तान्प्रत्यविध्यद्वार्ष्णेयः पृथक्पृथगजिह्मगैः}
{`सात्यकिः पुण्डरीकाक्षो ह्यस्त्रेषु परिनिष्ठितः'}


\twolineshloka
{भारद्वाजं त्रिभिर्बाणैर्दुःसहं नवभिः शरैः}
{विकर्णं पञ्चविंशत्या चित्रसेनं च सप्तभिः}


\twolineshloka
{दुर्मर्षणं द्वादशभिरष्टाभिश्च विविंशतिम्}
{सत्यव्रतं च नवभिर्विजयं दशभिः शरैः}


\threelineshloka
{`दुर्योधनं त्रिभिर्विद्ध्वा सुपुङ्खैस्तिग्मतेजनैः'}
{ततो रुक्माङ्गदं चापं विधुन्वानो महारथः}
{अभ्ययात्सात्यकिस्तूर्णं पुत्रं तव महारथम्}


\twolineshloka
{राजानं सर्वलोकस्य सर्वलोकमहारथम्}
{शरैरभ्यहनद्गाढं ततो युद्धमभूत्तयोः}


\twolineshloka
{विमुञ्चतौ शरांस्तीक्ष्णान्सन्दधानौ च सायकान्}
{अदृश्यं समरेऽन्योन्यं चक्रतुस्तौ महारथौ}


\twolineshloka
{सात्यकिः कुरुराजेन निर्विद्धो बह्वशोभत}
{अस्रवद्रुधिरं भूरि स्वरसं चन्दनो यथा}


\twolineshloka
{सात्वतेन च बाणौधैर्निर्विद्धस्तनयस्तव}
{शातकुम्भमयापीडो बभौ यूप इवोच्छ्रितः}


\twolineshloka
{माधवस्तु रमे राजन्कुरुराजस्य धन्विनः}
{धनुश्चिच्छेद समरे क्षुरप्रेण हसन्निव}


\threelineshloka
{अथैनं छिन्नधन्वानं शरैर्बहुभिराचिनोत्}
{निर्भिन्नश्च शरैस्तेन द्विषता क्षिप्रकारिणा}
{नामृष्यत रणे राजा शत्रोर्विजयलक्षणम्}


\twolineshloka
{अथान्यद्धनुरादाय हेमपृष्ठं दुरासदम्}
{विव्याध सात्यकिं तूर्णं सायकानां शतेन ह}


\twolineshloka
{सोऽतिविद्धो बलवता तव पुत्रेण धन्विना}
{अमर्षवशमापन्नस्तव पुत्रमपीडयत्}


\twolineshloka
{पीडितं नृपतिं दृष्ट्वा तव पुत्रं महारथाः}
{सात्यकिं शरवर्षेण छादयामासुरोजसा}


\threelineshloka
{स च्छाद्यमानो बहुभिस्तव पुत्रैर्महारथैः}
{एकैकं पञ्चभिर्विद्ध्वा पुनर्विव्याध सप्तभिः}
{दुर्योधनं व त्वरितो विव्याधाष्टभिराशुगैः}


\twolineshloka
{प्रहसंश्चास्य चिच्छेद कार्मुकं रिपुभीषणम्}
{नागं मणिमयं चैव शरैर्ध्वजमपातयत्}


\twolineshloka
{हत्वा तु चतुरो वाहांश्चतुर्भिर्निशितैः शरैः}
{सारथिं पातयामास क्षुरप्रेण महायशाः}


\twolineshloka
{एतस्मिन्नन्तरे चैव कुरुराजं महारथम्}
{अवारिकच्छरैर्हृष्टो बहुभिर्मर्मभेदिभिः}


\threelineshloka
{स विध्यमानः समरे शैनेयस्य शरोत्तमैः}
{प्राद्रवत्सहसा राजन्पुत्रो दुर्योधनस्तव}
{आप्लुतश्च ततो यानं चित्रसेनस्य धन्विनः}


\twolineshloka
{हाहाभूतं जगच्चासीद्दृष्ट्वा राजानमाहवे}
{ग्रस्यमानं सात्यकिना खे मोममिव राहुणा}


\twolineshloka
{तं तु शब्दमथ श्रुत्वा कृतवार्मा महारथः}
{अभ्ययात्सहसा तत्र यत्रास्ते माधवः प्रभुः}


\twolineshloka
{विधुन्वानो धनुः श्रेष्ठं चोदयंश्चैव वाजिनः}
{भर्त्सयन्सारथिं चाग्रे याहियाहीति सत्वरम्}


\twolineshloka
{तमापतन्तं सम्प्रेक्ष्य व्यादितास्यमिवान्तकम्}
{युयुधानो महाराज यन्तारमिदमब्रवीत्}


\twolineshloka
{कृतवर्मा रथेनैष द्रुतमापतते शरी}
{प्रत्युद्याहि रथेनैनं प्रवरं सर्वधन्विनाम्}


\twolineshloka
{`प्रयाहि सत्वरं सूत रथेन रथसत्तमम्}
{निहनिष्यामि तं सङ्ख्ये वृष्णिवीरमरिन्दमम्'}


\twolineshloka
{ततः प्रजविताश्वेन रथेन रथिनांवरः}
{आससाद रणे भोजं प्रतिमानं धनुष्मताम्}


\twolineshloka
{ततो गभस्तिमत्प्रख्यौ ज्वलिताविव पावकौ}
{समेयातां नरव्याघ्रौ व्याघ्राविव तरस्विनौ}


\twolineshloka
{कृतवर्मा तु शैनेयं षड्विंशत्या समार्पयत्}
{निशितैः सायकैस्तीक्ष्णैर्यन्तारं चास्य पञ्चभिः}


\twolineshloka
{चतुरश्चतुरो वाहांश्चतुर्भिः परमेषुभिः}
{अविध्यत्साधुदान्तान्वै सैन्धवान्सात्वतस्य हि}


\twolineshloka
{रुक्मध्वजो रुक्मपृष्ठं महद्विष्फार्य कार्मुकम्}
{रुक्माङ्गदी रुक्मवर्मा रुक्मपुङ्खैरवारयत्}


\twolineshloka
{ततोऽशीतिं शिनेः पौत्रः सायकान्कृतवर्मणे}
{प्राहिणोत्त्वरया युक्तो द्रष्टुकामो धनञ्जयम्}


\twolineshloka
{सोऽतिविद्धो बलवता शत्रुणा शत्रुतापनः}
{समकम्पतः दुर्धर्षः क्षितिकम्पे यथाऽचलः}


\twolineshloka
{त्रिषष्ट्या चतुरोऽस्याश्वान्सप्तभिः सारथिं तथा}
{विव्याध निशितैस्तूर्णं सात्यकिः सत्यविक्रमः}


\threelineshloka
{`हताश्वसूते तु रथे स्थिताय शिनिपुङ्गवः'}
{सुवर्णपुङ्खं विशिखं समादाय च सात्यकिः}
{व्यसृजत्तं महाज्वालं सङ्क्रुद्धमिव पन्नगम्}


\threelineshloka
{सोऽविध्यत्कृतवर्माणं यमदण्डोपमः शरः}
{जाम्बूनदविचित्रं च वर्म निर्भिद्य भानुमत्}
{अभ्यगाद्धरणीमुग्रो रुधिरेण समुक्षितः}


\twolineshloka
{सञ्जातरुधिरश्चाजौ सात्वतेषुभिरर्दितः}
{सशरं धनुरुत्सृज्य न्यपतत्स्यन्दनोत्तमात्}


\twolineshloka
{स सिंहदंष्ट्रो जानुभ्यां पतितोऽमितविक्रमः}
{शरार्दितः सात्यकिना रथोपस्थे नरर्षभः}


\twolineshloka
{सहस्रबाहुसदृशमक्षोभ्यमिव सागरम्}
{निवार्य कृतवर्माणं सात्यकिः प्रययौ ततः}


\twolineshloka
{`न हि तस्य शरोऽपार्थः कथञ्चिदपि पार्थिव}
{दिदृक्षुः स हि वेगेन प्रायाद्यत्र धनञ्जयः}


\twolineshloka
{स शक्त्या क्षत्रियैस्तत्र निरुद्धो बलवत्तरः}
{युयुधे सात्वतो राजन्दिदृक्षुः पाण्डुनन्दनम्'}


\twolineshloka
{खङ्गशक्तिधनुःकीर्णां गजाश्वरथसङ्कुलाम्}
{प्रवर्तितोग्ररुधिरां शतशः क्षत्रियर्षभैः}


\twolineshloka
{प्रेक्षतां सर्वसैन्यानां मध्येन शिनिपुङ्गवः}
{अभ्यगाद्वाहिनीं हित्वा वृत्रहेवासुरीं चमूम्}


\twolineshloka
{समाश्वस्य च हार्दिक्यो गृह्य चान्यन्महद्धनुः}
{तस्थौ स तत्र बलवान्वारयन्युधि पाण्डवान्}


\chapter{अध्यायः ११७}
\twolineshloka
{सञ्जय उवाच}
{}


\twolineshloka
{काल्यमानेषु सैन्येषु शैनेयेन ततस्ततः}
{भारद्वाजः शरव्रातैर्मर्मभिद्भिरवाकिरत्}


\twolineshloka
{स सम्प्रहारस्तुमुलो द्रोणसात्वतयोरभूत्}
{पश्यतां सर्वसैन्यानां बलिवासवयोरिव}


\twolineshloka
{ततो द्रोणः शिनेः पौत्रं चित्रैः सर्वायसैः शरैः}
{त्रिभिराशीविपाकारैर्ललाटे समविध्यत}


\twolineshloka
{तैर्ललाटार्पितैर्बाणैर्युयुधानस्त्वजिह्मगैः}
{व्यरोचत महाराज त्रिशृङ्ग इव पर्वतः}


\twolineshloka
{ततोऽस्य बाणानपरानिन्द्राशनिसमस्वनान्}
{भारद्वाजोऽन्तरप्रेक्षी प्रेषयामास संयुगे}


\twolineshloka
{तान्द्रोणचापनिर्मुक्तान्दाशार्हः पततः शरान्}
{द्वाभ्यांद्वाभ्यां सुपुङ्खाभ्यां चिच्छेद परमास्त्रवित्}


\twolineshloka
{तामस्य लघुतां द्रोणः समवेक्ष्य विशाम्पते}
{प्रहस्य सहसाऽविध्यत्त्रिंशतां शिनिपुङ्गवम्}


\twolineshloka
{पुनः पञ्चशतेषूणां शितेन च समार्पयत्}
{लघुतां युयुधानस्य लाघवेन विशेषयन्}


\twolineshloka
{समुत्पतन्ति वल्मीकाद्यथा क्रुद्धा महोरगाः}
{तथा द्रोणरथाद्राजन्नापतन्ति तनुच्छिदः}


\twolineshloka
{तथैव युयुधानेन सृष्टाः सतसहस्रशः}
{अवाकिरन्द्रोणरथं शरा रुधिरभोजनाः}


\twolineshloka
{लाघवाद्दिजमुख्यस्य सात्वतस्य च मारिष}
{विशेषं नाध्यगच्छाम समावास्तां नरर्षभौ}


\twolineshloka
{सात्यकिस्तु ततो द्रोणं नवभिर्नतपर्वभिः}
{आजघान भृशं क्रुद्धो ध्वजं च निशितैः शरैः}


\twolineshloka
{सारथिं च शतेनैव भारद्वाजस्य पश्यतः}
{लाघवं युयुधानस्य दृष्ट्वा द्रोणो महारथः}


\twolineshloka
{सप्तत्या सारथिं विद्व्वा तुरङ्गांश्च त्रिभिस्त्रिभिः}
{ध्वजमेकेन चिच्छेद माधवस्य रथे स्थितम्}


\twolineshloka
{अथापरेण भल्लेन हेमपुङ्खेन पत्रिणा}
{धनुश्चिच्छेद समरे माधवस्य महात्मनः}


\twolineshloka
{सात्यकिस्तु ततः क्रुद्धो धनुस्त्यक्त्वा महारथः}
{गदां जग्राह महतीं भारद्वाजाय चाक्षिपत्}


\twolineshloka
{तामापतन्तीं सहसा पट्टबद्धामयस्मयीम्}
{न्यवारयच्छरैर्द्रोणो बहुभिर्बहुरूपिभिः}


\twolineshloka
{अथान्यद्धनुरादाय सात्यकिः सत्यविक्रमः}
{विव्याध बहुभिर्वीरं भारद्वाजं शिलाशितैः}


\twolineshloka
{स विद््वा समरे द्रोणं सिंहनादममुञ्चत}
{तं वै न मृमषे द्रोणः सिंहनादं महात्मनः}


\twolineshloka
{ततः शक्तिं रणे द्रोणो रुक्मदण्डामयस्मयीम्}
{तरसा प्रेषयामास माधवस्य रथं प्रति}


\twolineshloka
{अनासाद्य तु शैनेयं सा शक्तिः कालसन्निभा}
{भित्त्वा रथं जगामोग्ना धरणीं दारुणस्वना}


\twolineshloka
{ततो द्रोणं शिनेः पौत्रो राजन्विव्याध पत्रिणा}
{दक्षिणं भुजमासाद्य पीडयन्भरतर्षभ}


\twolineshloka
{द्रोणोऽपि समरे राजन्माधवस्य महद्धनुः}
{अर्धचन्द्रेण चिच्छेद रथशक्त्या च सारथिम्}


\twolineshloka
{मुमोह सारथिस्तस्य रथशक्त्या समाहतः}
{स रथोपस्थमासाद्य मुहूर्तं सन्न्यषीदत}


\twolineshloka
{चकार सात्यकी राजंस्तत्र कर्मातिमानुषम्}
{अयोधयच्च यद्द्रोणं रश्मीञ्जग्राह च स्वयम्}


\twolineshloka
{ततः शरशतेनैव युयुधानो महारथः}
{अविध्यद्ब्राह्मणं सङ्ख्ये हृष्टरूपो विशाम्पते}


\twolineshloka
{तस्य द्रोणः शरान्पञ्च प्रेषयामास भारत}
{ते घोराः कवचं भित्त्वा पपुः शोणितमाहवे}


\twolineshloka
{निर्विद्धस्तु शरैर्घोरैरक्रुद्ध्यत्सात्यकिर्भृशम्}
{सायकान्व्यसृजच्चापि वीरो रुक्मरथं प्रति}


\twolineshloka
{ततो द्रोणस्य यन्तारं निपात्यैकेषुणा भुवि}
{अश्वान्व्यद्रावयद्बाणैर्हतसूतांस्ततस्ततः}


\twolineshloka
{स रथः प्रद्रुतः सङ्ख्ये मण्डलानि सहस्रशः}
{चकार राजतो राजन्भ्राजमान इवांशुमान्}


\twolineshloka
{अभिद्रवत गृह्णीत हयान्द्रोणस्य धावत}
{इति स्म चुक्रुशुः सर्वे राजपुत्राः सराजकाः}


\twolineshloka
{ते सात्यकिमपास्याशु राजन्युधि महारथाः}
{यतो द्रोणस्ततः सर्वे सहसा समुपाद्रवन्}


\twolineshloka
{तान्दृष्ट्वा प्रद्रुतान्सङ्ख्ये सात्वतेन शरार्दितान्}
{प्रभग्नं पुनरेवासीत्तव सैन्यं समाकुलम्}


\twolineshloka
{व्यूहस्यैव पुनर्द्वारं गत्वा द्रोणो व्यवस्थितः}
{वातायमानैस्तैरश्वैर्नीतो वृष्णिशरार्दितैः}


\twolineshloka
{पाण्डुपाञ्चालसम्भिन्नं व्यूहमालोक्य वीर्यवान्}
{शैनेये नाकरोद्यत्नं व्यूहमेवाभ्यरक्षत}


\twolineshloka
{निवार्य पाण़्डुपाञ्चालान्द्रोणाग्निः प्रदहन्निव}
{तस्थौ क्रोधेध्मसन्दीप्तः कालसूर्य इवोद्यतः}


\chapter{अध्यायः ११८}
\twolineshloka
{सञ्जय उवाच}
{}


\twolineshloka
{द्रोणं स जित्वा पुरुषप्रवीर--स्तथैव हार्दिक्यमुखांस्त्वदीयान्}
{प्रहस्य सूतं वचनं बभाषेशिनिप्रवीरः कुरुपुङ्गवाग्र्य}


\threelineshloka
{निमित्तमात्रं वयमद्य सूतदग्धा रथाः केशवफल्गुनाभ्याम्}
{हतान्निहन्मेह नरर्षभेणवयं सुरेशात्मसमुद्भवेत ॥सञ्जय उवाच}
{}


\twolineshloka
{तमेवमुक्त्वा शिनिपुङ्गवस्तदामहामृधे सोऽग्र्यधनुर्धरोऽरिहा}
{किरन्समन्तात्सहसा शरान्बलीसमापतच्छयेन इवामिषाशया}


\twolineshloka
{तं यान्तमश्वैः शशिशङ्खवर्णै--र्विगाह्य सैन्यं पुरुषप्रवीरम्}
{नाशक्नुवन्वारयितुं समन्ता--दादित्यरश्मिप्रतिमं रथाग्र्यम्}


\twolineshloka
{असह्यविक्रान्तमदीनसत्वंसर्वे गणा भारत ये त्वदीयाः}
{सहस्रनेत्रप्रतिमप्रभावंदिवीव सूर्यं जलदव्यपाये}


\twolineshloka
{अमर्षपूर्णस्त्वतिचित्रयोधीशरासनी काञ्चनवर्मधारी}
{सुदर्शनः सात्यकिमापतन्तंन्यवारयद्राजवरः प्रसह्य}


\twolineshloka
{तयोरभूद्भारतसम्प्रहारःसुदारुणस्तं समतिप्रशंसन्}
{योधास्त्वदीयाश्च हि सोमकाश्चवृत्रेन्द्रयोर्युद्धमिवामरौघाः}


\twolineshloka
{शरैः सुतीक्ष्णैः शतशोऽभ्यविध्य--त्सुदर्शनः सात्वतमुख्यमाजौ}
{अनागतानेव तु तान्पृषत्कां--श्चिच्छेद राजञ्शिनिपुङ्गवोऽपि}


\twolineshloka
{तथैव शक्रप्रतिमोऽपि सात्यकिःसुदर्शने यान्क्षिपति स्म सायकान्}
{द्विधा त्रिधा तानकरोत्सुदर्शनःशरोत्तमैः स्यन्दनवर्यमास्थितः}


\twolineshloka
{तान्वीक्ष्य बाणान्निहतांस्तदानींसुदर्शनः सात्यकिबाणवेगैः}
{क्रोधाद्दिधक्षन्निव तिग्मतेजाःशरानमुञ्चत्तपनीयचित्रान्}


\twolineshloka
{पुनः स बाणैस्त्रिभिरग्निकल्पै--राकर्णपूर्णैर्निशितैः सुपुङ्खैः}
{विव्याध देहावरणं विभिद्यते सात्यकेराविविशुः शरीरम्}


\twolineshloka
{तथैव तस्यावनिपालपुत्रःसन्धाय बाणैरपरैर्ज्वलद्भिः}
{आजघ्निवांस्तान्रजतप्रकाशां--श्चतुर्भिरश्वांश्चतुरः प्रसह्य}


\twolineshloka
{तथा तु तेमाभिहतस्तरस्वीनप्ता शिनेरिन्द्रसमानवीर्यः}
{सुदर्शनस्येषुगणैः सुतीक्ष्णै--र्हयान्निहत्याशु ननाद नादम्}


\twolineshloka
{अथास्य सूतस्य शिरो निकृत्यभल्लेन शक्राशनिसन्निभेन}
{सुदर्शनस्यापि शिनिप्रवीरःक्षुरेण कालानलसन्निभेन}


\twolineshloka
{सकुण्डलं पूर्णशशिप्रकाशंभ्राजिष्णु वक्त्रं विचकर्त देहात्}
{यथा पुरा वज्रधरः प्रसह्यबलस्य सङ्ख्येऽतिबलस्य राजन्}


\twolineshloka
{निहत्य तं पार्थिवपुत्रपौत्रंरणे यदूनामृषभस्तरस्वी}
{मुदा समेतः परया महात्मारराज राजन्सुरराजकल्पः}


\twolineshloka
{ततो ययावर्जुनमेव वीरोनिवार्य सैन्यं तव मार्गणौघैः}
{सदश्वयुक्तेन रथेन राजँ--ल्लोकं विसिस्मापयिषुर्नृवीरः}


\twolineshloka
{तत्तस्य विस्मापयनीयमग्र्य--मपूजयन्योधवराः समेताः}
{प्रवर्तमानानिषुगोचरेऽरी--न्ददाह बाणैर्हुतभुग्यथैव}


\chapter{अध्यायः ११९}
\twolineshloka
{`सञ्जय उवाच}
{}


\twolineshloka
{निर्जित्य कृतवर्माणं भारद्वाजं च संयुगे}
{स सात्यकिः सत्यधृतिर्महात्मा शिनिपुङ्गवः}


\twolineshloka
{*दुर्योधनं च निर्जित्य शूरं चैव सुयोधनम्}
{जलसन्धं निहत्याजौ शूरसेनं च पार्थिवम्}


\twolineshloka
{म्लेच्छांश्च बहुधा राजन्काश्यपुत्रं च संयुगे}
{निषादांस्तङ्कणांश्चैव कलिङ्गान्मगधानपि}


\twolineshloka
{केकयाञ्छूरसेनांश्च तथा पर्वतवासिनः}
{काम्भोजान्यवनांश्चैव वसातींश्च शिबीनपि}


\twolineshloka
{कोसलान्मगधांश्चैव यातुधानान्सतित्तिरान्}
{एतांश्चान्यान्रणे हत्वाऽगच्छद्युद्धे स सात्यकिः}


\twolineshloka
{रुधिरौघनदीं घोरां केशशैवालशाद्वलाम्}
{शक्तिग्राहसमाकीर्णां छत्रहंसोपशोभिताम्}


\twolineshloka
{दुस्तरां भीरुभिर्नित्यं शूरलोकप्रवाहिनीम्}
{प्रवर्त्य हृषितो राजन्पुनर्यन्तारमब्रवीत्}


\twolineshloka
{ध्वजनागाश्वकलिलं शरशक्त्यूर्मिमालिनम्'}
{खङ्कमत्स्यं गदाग्राहं शूरायुधमहास्वनम्}


\threelineshloka
{प्राणापहारिणं रौद्रं वादित्रोद्धुष्टनादितम्}
{शूराणां सुखसंस्पर्शमस्पृश्यमथ भीरुणाम्}
{तीर्णाः स्म दुस्तरं तात द्रोणानीकमहार्णवम्}


\twolineshloka
{जलसन्धबलेनाजौ पुरुषादैरिवावृतम्}
{अतोऽन्यत्पृतनाशेषं मन्ये कुनदिकामिव}


\twolineshloka
{तर्तव्यामल्पसलिलां चोदयाश्वानसम्भ्रमम्}
{हस्तप्राप्तमहं मन्ये साम्प्रतं सव्यसाचिनम्}


\twolineshloka
{निर्जित्य दुर्धरं द्रोणं सपदानुगमाहवे}
{हार्दिक्यं योधवर्यं च मन्ये प्राप्तं धनञ्जयम्}


\twolineshloka
{न हि मे जायते त्रासो दृष्ट्वा सैन्यान्यनेकशः}
{वह्नेरिव प्रदीप्तस्य वने शुष्कतृणालयम्}


\twolineshloka
{पश्य पाण्डवमुख्येन यातां भूमिं किरीटिना}
{पत्त्यश्वरथनागौघैः पतितैर्विषमीकृताम्}


\threelineshloka
{द्रवते तद्यथा सैन्यं तेन भग्नं महात्मना}
{रथैर्विपरिधावद्भिर्गजैरश्वैश्च सारथे}
{कौशेयारुणसंकाशमेतदुद्वूयते रजः}


% Check verse!
अभ्याशस्थमहं मन्ये श्वेताश्वं कृष्णसारथिम्
\twolineshloka
{श्रूयते ह्येष निर्घोषो जलदस्येव गर्जतः}
{विष्फार्यमाणस्य रणे गाण्डीवस्यामितौजसः}


\twolineshloka
{अभ्याशस्थमहं मन्ये सैन्धवस्य किरीटिनम्}
{तादृशः श्रूयते शब्दः सैन्यानां सागरोपमः}


\twolineshloka
{यादृशानि निमित्तानि मम प्रादुर्भवन्ति वै}
{अनस्तंगत आदित्ये हन्ता सैन्धवमर्जुनः}


% Check verse!
शनैर्विस्रम्भयन्नश्वान्याहि यत्रारिवाहिनी
\twolineshloka
{यत्रैते सतलत्रामाः सुयोधनपुरोगमाः}
{दंशिताः क्रूरकर्माणः काम्भोजा युद्धदुर्मदाः}


\twolineshloka
{शरबाणासनधरा यवनाश्च प्रहारिणः}
{शकाः किराता दरदा बर्बरास्ताम्रलिप्तकाः}


\twolineshloka
{अन्ये च बहवो म्लेच्छा विविधायुधपाणयः}
{मामेवाभिमुखाः सर्वे तिष्ठन्ति समरार्थिनः}


\threelineshloka
{एतान्सरथनागाश्वान्निहत्याजौ सपत्तिनः}
{इदं दुर्गं महाघोरं तीर्णमेवोपधारय ॥सूत उवाच}
{}


\twolineshloka
{न सम्भ्रमो मे वार्ष्णेय विद्यते सत्यविक्रम}
{यद्यपि स्यात्तव क्रुद्धो जामदग्न्योऽग्रतः स्थितः}


\twolineshloka
{द्रोणो वा रथिनां श्रेष्ठः कृपो मद्रेश्वरोऽपि वा}
{तथापि सम्भ्रामो न स्यात्त्वामाश्रित्य महाभुज}


\twolineshloka
{त्वया सुबहवो युद्धे निर्जिताः शत्रुसूदन}
{दंशिताःक्रूरकर्माणः काम्भोजा युद्धदुर्मदाः}


\threelineshloka
{शरवाणासनधरा यवनाश्च प्रहारिणः}
{शकाः किराता दरदा बर्बरास्ताम्रलिप्तकाः}
{अन्ये च बहवो म्लेच्छा विविधायुधपाणयः}


\twolineshloka
{न च मे सम्भ्रमः कश्चिद्भूतपूर्वः कथञ्चन}
{किमुतैतत्समासाद्य धीरं संयुगगोष्पदम्}


\twolineshloka
{आयुष्मन्कतरेण त्वां प्रापयामि धनञ्जयम्}
{केभ्यः क्रुद्धोऽसि वार्ष्णेय कान्स्विन्मृत्युरुपस्थितः}


\twolineshloka
{केषां सङ्घमनीकस्य हन्तुमुत्सहते मनः}
{केत्वां युधि पराक्रान्तं कालान्तकयमोपमम्}


\threelineshloka
{दृष्ट्वा विक्रमसम्पन्नं विद्रविष्यन्ति संयुगे}
{केषां वैवस्वतो राजा स्मरतेऽद्य महाभुज ॥सात्यकिरुवाच}
{}


\twolineshloka
{मुण्डानेतान्हनिष्यामि दानवानिव वासवः}
{प्रतिज्ञां पारयिष्यामि काम्भोजानेव मां वह}


% Check verse!
xxxxxकदनं कृत्वा क्षिप्रं यास्यामि पाण्डवम्
\twolineshloka
{xxxxxयन्ति मे वीर्यं कौरवाः ससुयोधनाः}
{xxxxxxनाके हते सूत सर्वसैन्येषु चासकृत्}


\twolineshloka
{अद्य कौरवसैन्यस्य दीर्घमाणस्य संयुगे}
{श्रुत्वा विरावं बहुधा सन्तप्स्यति सुयोधनः}


\twolineshloka
{अद्य पाण्डवमुख्यस्य श्वेताश्वस्य महात्मनः}
{आचार्यककृतं मार्गं दर्शयिष्यामि संयुगे}


\twolineshloka
{अद्य मद्बाणनिहतान्योधमुख्यान्सहस्रशः}
{दृष्ट्वा दुर्योधनो राजा पश्चात्तापं गमिष्यति}


\twolineshloka
{अद्य मे क्षिप्रहस्तस्य क्षिपतः सायकोत्तमान्}
{अलातचक्रप्रतिमं धनुर्द्रक्ष्यन्ति कौरवाः}


\twolineshloka
{मत्सायकचिताङ्गानां रुधिरं स्रवतां मुहुः}
{सैनिकानां वधं दृष्ट्वा सन्तप्स्यति सुयोधनः}


\twolineshloka
{अद्य मे क्रुद्धरूपस्य निघ्नतश्च वरान्वरान्}
{द्विफल्गुनामिमं लोकं मंस्यतेऽद्य सुयोधनः}


\twolineshloka
{अद्य राजसहस्राणि निहतानि मया रणे}
{दृष्ट्वा दुर्योधनो राजा सन्तप्स्यति महामृधे}


\twolineshloka
{अद्य स्नेहं च भक्तिं च पाण्डवेषु महात्मसु}
{हत्वा राजसहस्राणि दर्शयिष्यामि राजसु}


\twolineshloka
{बलं वीर्यं कृतज्ञत्वं मम ज्ञास्यन्ति कौरवाः ॥सञ्जय उवाच}
{}


\twolineshloka
{एवमुक्तस्तदा सूतः शिक्षितान्साधुवाहिनः}
{शशाङ्कसन्निकाशान्वै वाजिनो व्यनुदद्भृशम्}


\twolineshloka
{ते पिबन्त इवाकाशं युयुधानं हयोत्तमाः}
{प्रापयन्यवनाञ्शनीघ्रं मनः पवनरंहसः}


\twolineshloka
{सात्यकिं ते समासाद्य पृतनास्वनिवर्तिनम्}
{बहवो लघुहस्ताश्च शरवर्षैरवाकिरन्}


\twolineshloka
{तेषामिषूनथास्त्राणि वेगवान्नतपर्वभिः}
{अच्छिनत्सात्यकी राजन्नैनं ते प्राप्नुवञ्शराः}


\twolineshloka
{रुक्मपुङ्खैः सुनिशितैर्गार्ध्रपत्रैरजिह्मगैः}
{उच्चकर्त शिरांस्युग्रो यवनानां भुजानपि}


\twolineshloka
{शैक्यायसानि वर्माणि कांस्यानि च समन्ततः}
{भित्त्वा देहांस्तथा तेषां शरा जग्मुर्महीतलम्}


\twolineshloka
{ते हन्यमाना वीरेण म्लेच्छाः सात्यकिना रणे}
{शतशोऽभ्यपतंस्तत्र व्यसवो वसुधातले}


\twolineshloka
{सुपूर्णायतमुक्तैस्तानव्यवच्छिन्नपिण्डितैः}
{पञ्च षट् सप्त चाष्टौ च भिभेद यवनाञ्शरैः}


\twolineshloka
{काम्भोजानां सहस्रैश्च शकानां च विशाम्पते}
{शबराणां किरातानां बर्बराणां तथैव च}


\twolineshloka
{अगम्यरूपां पृथिवीं मांसशोणितकर्दमाम्}
{कृतवांस्तत्र शैनेयः क्षपयंस्तावकं बलम्}


\twolineshloka
{दस्यूनां सशिरस्त्राणैः शिरोभिर्लूनमूर्धदौः}
{दीर्घकूर्चैर्मही कीर्णा विबर्हैरण्डजैरिव}


\twolineshloka
{रुधिरोक्षितसर्वाङ्गैस्तैस्तदायोधनं बभौ}
{कबन्धैः संवृतं सर्वं ताम्राभ्रैः खमिवावृतम्}


\twolineshloka
{वज्राशनिसमस्पर्शैः सुपर्वभिरजिह्मगैः}
{ते सात्वतेन निहताः समावव्रुर्वसुन्धराम्}


\twolineshloka
{`तेषामस्त्राणि बाणांश्च शैनेयो नतपर्वभिः}
{निचकर्त महाराज यवनानां शिरोधराः}


\twolineshloka
{ते शरा नतपर्वाणो युयुधानप्रचोदिताः}
{हित्वा देहांस्तथा तेषां पतन्ति स्म महीतले}


\twolineshloka
{ते हन्यमाना वीरास्तु वृष्णिवीरेण संयुगे}
{शस्त्राहताः पतन्त्युर्व्यां काम्भोजाः सपदानुगाः}


\twolineshloka
{काम्भोजानां भुजैश्छिन्नैर्यवनानां च भारत}
{तत्रतत्र मही भाति पञ्चास्यैरिव पन्नगैः'}


\twolineshloka
{अल्पावशिष्टाः सम्भग्नाः कृच्छ्रप्राणा विचेतसः}
{जिताः सङ्ख्ये महाराज युयुधानेन दंशिताः}


\twolineshloka
{पार्ष्णिभिश्च कशाभिश्च ताडयन्तस्तुरङ्गमान्}
{जवमुत्तममास्थाय सर्वतः प्राद्रवन्भयात्}


\twolineshloka
{काम्भोजसैन्यं विद्राव्य दुर्जयं युधि भारत}
{यवनानां च तत्सैन्यं शकानां च महद्बलम्}


\twolineshloka
{ततः स पुरुषव्याघ्रः सात्यकिः सत्यविक्रमः}
{प्रहृष्टस्तावकाञ्जित्वा सूतं याहीत्यचोदयत्}


\twolineshloka
{तत्तस्य समरे कर्म दृष्ट्वाऽन्यैरकृतं पुरा}
{चारणाः सहगन्धर्वाः पूजयाञ्चक्रिरे भृशम्}


\twolineshloka
{तं यान्तं पृष्ठगोप्तारमर्जुनस्य विशाम्पते}
{चारणाः प्रेक्ष्य संहृष्टास्त्वदीयाश्चाभ्यपूजयन्}


\chapter{अध्यायः १२०}
\twolineshloka
{सञ्जय उवाच}
{}


\twolineshloka
{जित्वा यवनकाम्भोजान्युयुधानस्ततोऽर्जुनम्}
{जगाम तव सैन्यस्य मध्येन रथिनां वरः}


\twolineshloka
{शरदंष्ट्रो नरव्याघ्रो विचित्रकवचच्छविः}
{मृगं व्याघ्र इवाजिघ्रंस्तव सैन्यमभीषयत्}


\twolineshloka
{स रथेन चरन्मार्गान्धनुरुद्धामयद्भृशम्}
{रुक्मपृष्ठं महावेगं रुक्मचन्द्रकसङ्कुलम्}


\twolineshloka
{रुक्माङ्गदशिरस्त्राणो रुक्मवर्मसमावृतः}
{रुक्मध्वजधनुः शूरो मेरुशृङ्गमिवाबभौ}


\twolineshloka
{सधनुर्मण्डलः सङ्ख्ये तेजोभासुररश्मिवान्}
{शरदीवोदितः सूर्यो नृसूर्यो विरराज ह}


\twolineshloka
{वृषभस्कन्धविक्रान्तो वृषभाक्षो नरर्षभः}
{तावकानां बभौ मध्ये गवां मध्ये यथा वृषः}


\twolineshloka
{मत्तद्विरदसङ्काशं मत्तद्विरदगामिनम्}
{प्रभिन्नमिव मातङ्गं यूथमध्ये व्यवस्थितम्}


\twolineshloka
{व्याधा इव जिघांसन्तस्त्वदीयाः समुपाद्रवन्}
{`सात्यकिं समरे राजन्परिवव्रुस्तवात्मजाः'}


\twolineshloka
{द्रोणानीकमतिक्रान्तं भोजानीकं च दुस्तरम्}
{जलसन्धार्णवं तीर्त्वा काम्भोजानां च वाहिनीम्}


\twolineshloka
{हार्दिक्यमकरान्मुक्तं तीर्णं वै सैन्यसागरम्}
{परिवव्रुः सुसङ्क्रुद्वास्त्वदीयाः सात्यकिं रथाः}


\twolineshloka
{दुर्योधनश्चित्रसेनो दुःशासनविविंशती}
{शकुनिर्दुःसहश्चैव युवा दुर्धर्षणः क्रथः}


\twolineshloka
{अन्ये च बहवः शूराः शस्त्रवन्तो दुरासदाः}
{पृष्ठतः सात्यकिं यान्तमन्वधावन्नमर्षिणः}


\twolineshloka
{अथ शब्दो महानासीत्तव सैन्यस्य मारिष}
{मारुतोद्धूतवेगस्य सागरस्येव पर्वणि}


\twolineshloka
{तानभिद्रवतः सर्वान्समीक्ष्य शिनिपुङ्गवः}
{शनैर्याहीति यन्तारमब्रवीत्प्रहसन्निव}


\threelineshloka
{इदमेति समुद्धूतं धार्तराष्ट्रबलं महत्}
{मामेवाभिमुखं तूर्णं गजाश्वरथपत्तिमत्}
{नादयद्वै दिशः सर्वा रथघोषेण सारथे}


\threelineshloka
{पृथिवीं चान्तरिक्षं च कम्पयन्सागरानपि}
{एतद्बलार्णवं सूत वारयिष्ये महारणे}
{पौर्णमास्यामिवोद्धूतं वेलेव मकरालयम्}


\twolineshloka
{पश्य मे सूत विक्रान्तमिन्द्रस्येव महामृधे}
{एष सैन्यानि शत्रूणां विधमामि शितैः शरैः}


\twolineshloka
{निहतानाहवे पश्य पदात्यश्वरथद्विपान्}
{मच्छरैरग्निसङ्काशैर्विद्धदेहान्सहस्रशः}


\threelineshloka
{इत्येवं ब्रुवतस्तस्य सात्यकेरमितौजसः}
{समीपे सैनिकास्ते तु शीघ्रमीयुर्युयुत्सवः}
{जह्याद्रवस्व तिष्ठेति पश्यपश्येति वादिनः}


\threelineshloka
{तानेवं ब्रुवतो वीरान्सात्यकिर्निशितैः शरैः}
{जघानं त्रिशतानश्वान्कुञ्जरांश्च चतुःशतान्}
{`लध्वस्त्रश्चित्रयोधी च प्रहरञ्शिनिपुङ्गवः'}


\twolineshloka
{स सम्प्रहारस्तुमुलस्तस्य तेषां च धन्विनाम्}
{देवासुररणप्रख्यः प्रावर्तत जनक्षयः}


\twolineshloka
{मेघजालनिभं सैन्यं तव पुत्रस्य मारिष}
{प्रत्यगृह्णाच्छिनेः पौत्रः शरैराशीविषोपमैः}


\twolineshloka
{प्रच्छाद्यमानः समरे शरजालैः स वीर्यवान्}
{असम्भ्रमन्महाराज तावकानवधीद्बहून्}


\twolineshloka
{आश्चर्यं तत्र राजेन्द्र सुमहद्दृष्टवानहम्}
{न मोघः सायकः कश्चित्सात्यकेरभवत्प्रभो}


\twolineshloka
{रथनागाश्वकलिलः पदात्यूर्मिसमाकुलः}
{शैनेयवेलामासाद्य स्थितः सैन्यमहार्णवः}


\threelineshloka
{सम्भ्रान्तनरनागाश्वमावर्तत मुहुर्मुहुः}
{तत्सैन्यमिषुभिस्तेन वध्यमानं समन्ततः}
{बभ्राम तत्रतत्रैव गावः शीतार्दिता इव}


\twolineshloka
{पदातिनं रथं नागं सादिनं तुरगं तथा}
{अविद्वं तत्र नाद्राक्षं युयुधानस्य सायकैः}


\twolineshloka
{न तादृक्कदनं राजन्कृतवांस्तत्र फल्गुनः}
{यादृक्क्षयमनीकानामकरोत्सात्यकिर्नृप}


\twolineshloka
{अत्यर्जुनं शिनेः पौत्रो युध्यते पुरुषर्षभः}
{वीतभीर्लाघवोपेतः कृतित्वं सम्प्रदर्शयन्}


\twolineshloka
{ततो दुर्योधनो राजा सात्वतस्य त्रिभिः शरैः}
{विव्याध सूतं निशितैश्चतुर्भिश्चतुरो हयान्}


\twolineshloka
{सात्यकिं च त्रिभिर्विद्व्वा पुनरष्टाभिरेव च}
{दुःशासनः षोडशभिर्विव्याध शिनिपुङ्गवम्}


\twolineshloka
{शकुनिः पञ्चविंशत्या चित्रसेनश्च पञ्चभिः}
{दुःसहः पञ्चदशभिर्विव्याधोरसि सात्यकिम्}


\twolineshloka
{उत्स्मयन्वृष्णिशार्दूलस्तथा बाणैः समाहतः}
{तानविध्यन्महाराज सर्वानेव त्रिभिस्त्रिभिः}


\twolineshloka
{गाढविद्धानरीन्कृत्वा मार्गणैः सोऽतितेजनैः}
{शैनेयः श्येनवत्सङ्ख्ये व्यचरल्लघुविक्रमः}


\twolineshloka
{सौबलस्य धनुश्छित्त्वा हस्तावापं निकृत्य च}
{दुर्योधनं त्रिभिर्बाणैरभ्यविध्यत्स्तनान्तरे}


\twolineshloka
{चित्रसेनं शतेनैव दशभिर्दुःसहं तथा}
{दुःशासनं तु विंशत्या विव्याध शिनिपुङ्गवः}


\twolineshloka
{अथान्यद्धनुरादाय श्यालस्तव विशाम्पते}
{अष्टाभिः सात्यकिं विद्व्वा पुनर्विव्याध पञ्चभिः}


\twolineshloka
{दुःशासनश्च दशभिर्दुःसहश्च त्रिभिः शरैः}
{दुर्मुखश्च द्वादशभी राजन्विव्याध सात्यकिम्}


\twolineshloka
{दुर्योधनस्त्रिसप्तत्या किद्व्वा भारत माधवम्}
{ततोस्य निशितैर्बाणैस्त्रिभिर्विव्याध सारथिम्}


\twolineshloka
{तान्सर्वान्सहिताञ्शूरान्यतमानान्महारथान्}
{पञ्चभिः पञ्चभिर्बाणैः पुनर्विव्याध सात्यकिः}


\twolineshloka
{ततः स रथिनां श्रेष्ठस्तव पुत्रस्य सारथिम्}
{आजघानाशु भल्लेन स हतो न्यपतद्भुवि}


\twolineshloka
{पतिते सारथौ तस्मिंस्तव पुत्ररथः प्रभो}
{वातायमानैस्तैरश्वैरपानीयत सङ्गरात्}


\twolineshloka
{ततस्तव सुतो राजन्सैनिकाश्च विशाम्पते}
{राज्ञो रथमभिप्रेक्ष्य विद्रुताः शतशोऽभवन्}


\twolineshloka
{विद्रुतं तत्र तत्सैन्यं दृष्ट्वा भारत सात्यकिः}
{अवाकिरच्छरैस्तीक्ष्णै रुक्मपुङ्खैः शिलाशितैः}


\twolineshloka
{विद्राव्य सर्वसैन्यानि तावकानि सहस्रशः}
{प्रययौ सात्यकी राजञ्श्वेताश्वस्य रथं प्रति}


\twolineshloka
{`तं प्रयान्तं महाबाहुं तावकाः प्रेक्ष्य मारिष}
{दृष्टं चादृष्टवत्कृत्वा क्रियामन्यामयोजयन्'}


\twolineshloka
{तं शरानाददानं च रक्षमाणं च सारथिम्}
{आत्मानं पालयानं च तावकाः समपूजयन्}


\chapter{अध्यायः १२१}
\twolineshloka
{धृतराष्ट्र उवाच}
{}


\twolineshloka
{सम्प्रद्य महत्सैन्यं यान्तं शैनेयमर्जुनम्}
{निर्हीका मम ते पुत्राः किमकुर्वत सञ्जय}


\twolineshloka
{कथं वैषां तदा युद्धे धृतिरासीन्मुमूर्षताम्}
{शैनेयचरितं दृष्ट्वा यादृशं सव्यसाचिनः}


\twolineshloka
{किं नु वक्ष्यन्ति ते क्षात्रं सैन्यमध्ये पराजिताः}
{कथं नु सात्यकिर्युद्धे व्यतिक्रान्तो महायशाः}


\twolineshloka
{कथं च मम पुत्राणां जीवतां तत्र सञ्जय}
{शैनेयोऽभिययौ युद्धे तन्ममाचक्ष्व सञ्जय}


\twolineshloka
{अत्यद्भुतमिदं तात त्वत्सकाशाच्छृणोम्यहम्}
{एकस्य बहुभिः सार्धं शत्रुभिस्तैर्महारथैः}


\twolineshloka
{विपरीतमहं मन्ये मन्दभाग्यं सुतं प्रति}
{यत्रावध्यन्त समरे सात्वतेन महारथाः}


\twolineshloka
{एकस्य हि न पर्याप्तं यत्सैन्यं तस्य सञ्जय}
{क्रुद्धस्य युयुधानस्य सर्वे तिष्ठन्तु पाण़्डवाः}


\twolineshloka
{निर्जित्य समरे द्रोणं कृतिनं चित्रयोधिनम्}
{यथा पशुगणान्सिंहस्तद्वद्धन्ता सुतान्मम}


\twolineshloka
{कृतवर्मादिभिः शूरैर्यत्तैर्बहुभिराहवे}
{युयुधानो न शकितो हन्तुं यत्पुरुषर्षभः}


\threelineshloka
{नैतदीदृशकं युद्धं कृतवांस्तत्र फल्गुनः}
{यादृशं कृतवान्युद्धं शिनेर्नप्ता महायशाः ॥स़ञ्जय उवाच}
{}


\twolineshloka
{तव दुर्मन्त्रिते राजन्दुर्योधनकृतेन च}
{शृणुष्वावहितो भूत्वा यत्ते वक्ष्यामि भारत}


\twolineshloka
{ते पुनः सन्न्यवर्तन्त कृत्वा संशपथान्मिथः}
{परां युद्धे मतिं क्रूरां तव पुत्रस्य शासनात्}


\twolineshloka
{त्रीणि सादिसहस्राणि दुर्योधनपुरोगमाः}
{शककाम्भोजबालह्लीका यवनाः पारदास्तथा}


\threelineshloka
{कुलिन्दास्तङ्कणाम्बष्ठाः पैशाचाश्च सबर्बराः}
{पार्वतीयाश्च राजेन्द्र क्रुद्धाः पाषाणपाणयः}
{अभ्यद्रवन्त शैनेयं शलभाः पावकं यथा}


\twolineshloka
{युक्ताश्च पार्वतीयानां रथाः पाषाणयोधिनाम्}
{शूराः पञ्चशतं राजञ्शैनेयं समुपाद्रवन्}


\twolineshloka
{ततो रथसहस्रेण महारथशतेन च}
{द्विरदानां सहस्रेण द्विसाहस्रैश्च वाजिभिः}


\twolineshloka
{शरवर्षाणि मुञ्चन्तो विविधानि महारथाः}
{अभ्यद्रवन्त शैनेयमसङ्ख्येयाश्च पत्तयः}


\twolineshloka
{तांश्च सञ्चोदयन्सर्वान्घ्नतैनमिति भारत}
{दुःशासनो महाराज सात्यकिं पर्यवारयत्}


\twolineshloka
{तत्राद्भुतमपश्याम शैनेयचरितं महत्}
{यदेको बहुभिः सार्धमसम्भ्रान्तमयुध्यत}


\twolineshloka
{अवधीच्च रथानीकं द्विरदानां च तद्बलम्}
{सादिनश्चैव तान्सर्वान्दस्यूनपि च सर्वशः}


\twolineshloka
{तत्र चक्रैर्विमथितैर्भग्नैश्च परमायुधैः}
{अक्षैश्च बहुधा भग्नैरीषादण्डकबन्धुरैः}


\twolineshloka
{कुञ्जरैर्मथितैश्चापि ध्वजैश्च विनिपातितैः}
{वर्मभिश्च तथाऽनीकैर्व्यवकीर्णा वसुन्धरा}


\twolineshloka
{स्रग्भिराभरणैर्वस्त्रैरनुकर्षैश्च मारिष}
{सञ्छन्ना वसुधा तत्र द्यौर्ग्रहैरिव भारत}


\twolineshloka
{गिरिरूपधरैश्चापि पतितैश्च महागजैः}
{`म्लेच्छस्थितैर्महाराज भिन्नाञ्जनचयोपमैः}


\twolineshloka
{मृगैर्मन्द्रैश्च भद्रैश्च मृगमन्द्रैस्तथाऽपरैः}
{भद्रैर्मन्द्रमृगैश्चैव मृगभद्रैस्तथैव च}


\twolineshloka
{भद्रमन्द्रैश्चैव तथा तथा भद्रमृगैरपि}
{तत्रतत्र धरा कीर्णा शयानैः पर्वतोपमैः'}


\twolineshloka
{अञ्जनस्य कुले जाता वामनस्य च भारत}
{सुप्रतीककुले जाता महापद्मकुले तथा}


\twolineshloka
{ऐरावतकुले चैव तथाऽन्येषु कुलेषु च}
{जाता दन्तिवरा राजञ्शेरते बहवो हताः}


\twolineshloka
{वनायुजान्पार्वतीयान्काम्भोजान्बाह्लिकानपि}
{तथा हयवरान्राजन्निजघ्ने तत्र सात्यकिः}


\twolineshloka
{नानादेशसमुत्थांश्च नानाजातींश्च दन्तिनः}
{निजघ्ने तत्र शैनेयः शतशोऽथ सहस्रशः}


\twolineshloka
{तेषु प्रकाल्यमानेषु दस्यून्दुःशासनोऽब्रवीत्}
{निवर्तध्वमधर्मज्ञा युध्यध्वं किं सृतेन वः}


\twolineshloka
{तांश्चातिभग्नासन्सम्प्रेक्ष्य पुत्रो दुःशासनस्तव}
{पाषाणयोधिनः शूरान्पार्वतीयानचोदयत्}


\twolineshloka
{अश्मयुद्धेषु कुशला नैतञ्जानापि सात्यकिः}
{अश्मयुद्धमजानन्तं हतैनं युद्धकामुकम्}


\twolineshloka
{तथैव कुरवः सर्वे नाश्मयुद्धविशारदाः}
{अभिद्रवत माभैष्ट न वः प्राप्स्यति सात्यकिः}


\twolineshloka
{ते पार्वतीया राजानः सर्वे पाषाणयोधिनः}
{अभ्यद्रवन्त शैनेयं राजानमिव मन्त्रिणः}


\twolineshloka
{ततो गजशिरःप्रख्यैरुपलैः शैलवासिनः}
{उद्यतैर्युयुधानस्य पुरतस्तस्थुराहवे}


\twolineshloka
{क्षेरणीयैस्तथाप्यन्ये सात्वतस्य वधैषिणः}
{चोदितास्तव पुत्रेण सर्वतो रुरुधुर्दिशः}


\twolineshloka
{तेषामापततामेव शिलायुद्धं चिकीर्षताम्}
{सात्यकिः प्रतिसन्धाय निशितान्प्राहिणोच्छरान्}


\twolineshloka
{तामश्मवृष्टिं तुमुलां पार्वतीयैः समीरिताम्}
{चिच्छेदोरगसंकाशैर्नाराचैः शिनिपुङ्गवः}


\twolineshloka
{तैरश्मचूर्णैर्दीप्यद्भिः खद्योतानामिव व्रजैः}
{प्रायः सैन्यान्यहन्यन्त हाहाभूतानि मारिष}


\twolineshloka
{ततः पञ्चशतं शूराः समुद्यतमहाशिलाः}
{निकृत्तबाहवो राजन्निपेतुर्धरणीतले}


\twolineshloka
{सात्वतस्य च भल्लेन निष्पिष्टैस्तैस्तथाऽद्रिभिः}
{न्यपतन्निहता म्लेच्छास्तत्रतत्र गतासवः}


\twolineshloka
{ते हन्यमानाः समरे सात्वतेन महात्मना}
{अश्मवृष्टिं महाघोरां पातयन्ति स्म सात्वते}


\twolineshloka
{पाषाणयोधिनः शूरान्यतमानानवस्थितान्}
{न्यवधीद्बहुसाहस्रांस्तदद्भुतमिवाभवत्}


\twolineshloka
{ततः पुनर्बस्तमुखैरश्मवृष्टीः समन्ततः}
{अधोबस्तैः स्थूलबस्तैर्दरदैः खसतङ्कणैः}


\twolineshloka
{लम्बकैश्च कुणिन्दैश्च क्षिप्ताः क्षिप्ताश्च सात्यकिः}
{नाराचैः प्रतिचिच्छेद प्रतिपत्तिविशारदः}


\twolineshloka
{अद्रीणां भिद्यमानानामन्तरिक्षे शितैः शरैः}
{शब्देन प्राद्रवन्सङ्ख्ये रथाश्वगजपत्तयः}


\twolineshloka
{अश्मचूर्णैरवाकीर्णा मनुष्यगजवाजिनः}
{नाशक्नुवन्नवस्थातुं भ्रमरैरिव दंशिताः}


\threelineshloka
{हतशिष्टाः सरुधिरा भिन्नमस्तकपिण़्डिकाः}
{`विभिन्नशिरसो राजन्दन्तैश्छिन्नैश्च दन्तिनः}
{निर्धूतैश्च करैर्नागा व्यङ्गाश्च शतशः कृताः}


\twolineshloka
{हत्वा पञ्चशतं योधांस्तत्क्षणेनैव मारिष}
{व्यचरत्पृतनामध्ये शैनेयः कृतहस्तवत्}


% Check verse!
कुञ्जराः सन्न्यवर्तन्त युयुधानेन मोहिताः
\twolineshloka
{अश्मनां भिद्यमानानां सायकैः श्रूयते ध्वनिः}
{पद्मपत्रेषुधाराणां पतन्तीनामिव ध्वनिः'}


\twolineshloka
{ततः शब्दः समभवत्तव सैन्यस्य मारिष}
{माधवेनार्द्यमानस्य सागरस्येव पर्वणि}


\twolineshloka
{तं शब्दं तुमुलं श्रुत्वा द्रोणो यन्तारमब्रवीत्}
{एष सूत रणे क्रुद्धः सात्वतानां महारथः}


\twolineshloka
{दारयन्बहुधा सैन्यं रणे चरति कालवत्}
{यत्रैष शब्दस्तुमुलस्तत्र सूत रथं नय}


\twolineshloka
{पाषाणयोधिभिर्नूनं युयुधानः समागतः}
{तथाहि रथिनः सर्वे हियन्ते विद्रुतैर्हयैः}


\threelineshloka
{विशस्त्रकवचा रुग्णास्तत्रतत्र पतन्ति च}
{न शक्नुवन्ति यन्तारः संयन्तुं तुमुले हयान् ॥सञ्जय उवाच}
{}


\twolineshloka
{इत्येतद्वचनं श्रुत्वा भारद्वाजस्य सारथिः}
{प्रत्युवाच ततो द्रोणं सर्वशस्त्रभृतां वरम्}


\twolineshloka
{सैन्यं द्रवति चायुष्मन्कौरवेयं समन्ततः}
{पश्य योधान्रणे भग्नान्धावतो वै ततस्ततः}


\twolineshloka
{इमे च संहताः शूराः पाञ्चालाः पाण्डवैः सह}
{त्वामेव हि जिघांसन्त आद्रवन्ति समन्ततः}


\threelineshloka
{अत्र कार्यं समाधत्स्व प्राप्तकालमरिन्दम}
{स्वाने वा गमने वापि दूरं यातश्च सात्यकिः ॥सञ्चय उवाच}
{}


\twolineshloka
{तथैवं वदतस्तस्य भारद्वाजस्य सारथेः}
{प्रत्यदृश्यत शैनेयो निघ्नन्बहुविधान्रथान्}


\twolineshloka
{ते वध्यमानाः समरे युयुधानेन तावकाः}
{युयुधानरथं त्यक्त्वा द्रोणानीकाय दुद्रुवुः}


\twolineshloka
{यैस्तु दुःशासनः सार्धं रथैः पूर्वं न्यवर्तत}
{ते भीतास्त्वभ्यधावन्त सर्वे द्रोणरथं प्रति}


\chapter{अध्यायः १२२}
\twolineshloka
{सञ्जय उवाच}
{}


\twolineshloka
{दुःशासनरथं दृष्ट्वा समीपे पर्यवस्थितम्}
{भारद्वाजस्ततो वाक्यं दुःशासनमथाब्रवीत्}


\twolineshloka
{दुःशासन रथाः सर्वे कस्माच्चैते प्रविद्रुताः}
{कच्चित्क्षेमं तु नृपतेः कच्चिज्जीवति सैन्धवः}


\twolineshloka
{राजपुत्रो भवानत्र राजभ्राता महारथः}
{किमर्थं द्रवते युद्धे यौवराज्यमवाप्य च}


\twolineshloka
{दासी जिताऽसि द्यूते त्वं यथा कामचरी भव}
{वाससां वाहिका राज्ञो भ्रातुर्ज्येष्ठस्य मे भव}


\twolineshloka
{न सन्ति पतयः सर्वे तेऽद्य षण्ढतिलैः समाः}
{दुःशासनैवं कस्मात्त्वं पूर्वमुक्त्वा पलायसे}


\twolineshloka
{स्वयं वैरं महत्कृत्वा पाञ्चालैः पाण्डवैः सह}
{एकं सात्यकिमासाद्य कथं भीतोऽसि संयुगे}


\twolineshloka
{न जानीषे पुरा त्वं तु गृह्णन्नक्षान्दुरोदरे}
{शरा ह्येते भविष्यन्ति दारुणाशीविषोपमाः}


\twolineshloka
{अप्रियाणां हि वचनं पाण्डवस्य विशेषतः}
{द्रौपद्याश्च परिक्लेशस्त्वन्मूलो ह्यभवत्पुरा}


\twolineshloka
{क्व ते मानश्च दर्पश्च क्व ते वीर्यं क्व गर्जितम्}
{आशीविषसमान्पार्थान्कोपयित्वा क्व यास्यसि}


\twolineshloka
{शोच्येयं भारती सेना राज्यं चैव सुयोधनः}
{यस्य त्वं कर्कशो भ्राता पलायनपरायणः}


\twolineshloka
{ननु नाम त्वया वीर दार्यमाणा भयार्दिता}
{स्वबाहुबलमास्थाय रक्षितव्या ह्यनीकिनी}


\threelineshloka
{स त्वमद्य रणं हित्वा भीतो हर्षयसे परान्}
{विद्रुते त्वयि सैन्यस्य नायके शत्रुमूदन}
{कोऽन्यः स्थास्यति सङ्ग्रामे भीतो भीते व्यपाश्रये}


\twolineshloka
{एकेन सात्वतेनाद्य युध्यमानस्य तेन वै}
{पलायने तव मतिः सङ्ग्रामाद्धि प्रवर्तते}


\twolineshloka
{यदा गाण्डीवधन्वानं भीमसेनं च कौरव}
{यमौ च द्रक्ष्यसि रणे तदा वै किं करिष्यसि}


\twolineshloka
{युधि फल्गुनबाणानां मूर्याग्निसमवर्चसाम्}
{न तुल्याः सात्यकिशरा येभ्यो भीतः पलायसे}


\twolineshloka
{त्वरितो वीर गच्छ त्वं गान्धार्युदरमाविश}
{पृथिव्यां धावमानस्य नान्यत्पश्यामि जीवनम्}


\twolineshloka
{यदि तावत्कृता बुद्धिः पलायनपरायणा}
{पृथिवी धर्मराजाय शमेनैव प्रदीयताम्}


\twolineshloka
{यावत्फल्गुननाराचा निर्मुक्तोरगसन्निभाः}
{नाविशन्ति शरीरं ते तावत्संशाम्य पाण्डवैः}


\twolineshloka
{यावत्ते पृथिवीं पार्था हत्वा भ्रातृशतं रणे}
{नाक्षिपन्ति महात्मानस्तावत्संशाम्य पाण्डवैः}


\twolineshloka
{यावन्न क्रुध्यते राजा धर्मपुत्रो युधिष्ठिरः}
{कृष्णश्च समरश्लाघी तावत्संशाम्य पाण्डवैः}


\twolineshloka
{यावद्भीमो महाबाहुर्विगाह्य महतीं चमूम्}
{सोदरांस्ते न मृद्राति तावत्संशाम्य पाण्डवैः}


\twolineshloka
{पूर्वमुक्तश्च ते भ्राता भीष्मेणासौ सुयोधनः}
{अजेयाः पाण्डवाः सङ्ख्ये सौम्य संशाम्य तैः सह}


\twolineshloka
{`मया शमवता चोक्तो रक्ष शेषं सुयोधन}
{संशाम्य पार्थैस्त्वं रक्ष वीर सर्वान्महीक्षितः'}


\twolineshloka
{न च तत्कृतवान्मन्दस्तव भ्रातां सुयोधनः}
{स युद्धे धृतिमास्थाय यत्तो युध्यस्व पाण्डवैः}


\twolineshloka
{तवापि शोणितं भीमः पास्यतीति मया श्रुतम्}
{तच्चाप्यवितथं तस्य तत्तथैव भविष्यति}


\twolineshloka
{किं भीमस्य न जानासि विक्रमं त्वं सुबालिश}
{यत्त्वया वैरमारब्धं संयुगे प्रपलायिना}


\threelineshloka
{गच्छ तूर्णं रथेनैव यत्र तिष्ठति सात्यकिः}
{त्वया हीनं बलं ह्येतद्विद्रविष्यति भारत ॥आत्मार्थं योधय रणे सात्यकिं सत्यविक्रमम् ॥सञ्जय उवाच}
{}


\twolineshloka
{एवमुक्तस्तव सुतो नाब्रवीत्किञ्चिदप्यसौ}
{श्रुतं चाश्रुतवत्कृत्वा प्रायाद्येन स सात्यकिः}


\twolineshloka
{सैन्येन महता युक्तो म्लेच्छानामनिवर्तिनाम्}
{आसाद्य च रणे यत्तो युयुधानमयोधयत्}


\twolineshloka
{द्रोणोऽपि रथिनां श्रेष्ठः पाञ्चालान्पाण्डवांस्तथा}
{अभ्यद्रवत सङ्क्रुद्धौ जवमास्थाय मध्यमम्}


\twolineshloka
{प्रविश्य च रणे द्रोणः पाण्डवानां वरूथिनीम्}
{द्रावयामास योधान्वै शतशोऽथ सहस्रशः}


\twolineshloka
{ततो द्रोणो महाराज नाम विश्राव्य संयुगे}
{पाण्डुपाञ्चालमत्स्यानां प्रचक्रे कदनं महत्}


\twolineshloka
{तं जयन्तमनीकानि भारद्वाजं ततस्ततः}
{पाञ्चालपुत्रो द्युतिमान्वीरकेतुः समभ्ययात्}


\twolineshloka
{स द्रोणं पञ्चभिर्विद्व्वा शरैः सन्नतपर्वभिः}
{ध्वजमेकेन विव्याध सारथिं चास्य सप्तभिः}


\twolineshloka
{तत्राद्भुतं महाराज दृष्ट्वानस्मि संयुगे}
{यद्द्रोणो रभसं युद्धे पाञ्चाल्यं नात्यवर्तत}


\twolineshloka
{सन्निरुद्धं रमे द्रोणं पाञ्चला वीक्ष्य मारिष}
{आवव्रुः सर्वतो राजन्धर्मपुत्रजयैषिणः}


\twolineshloka
{ते शरैरग्निसंकाशैस्तोमरैश्च महाधनैः}
{शस्त्रैश्च विविधै राजन्द्रोणमेकमवाकिरन्}


\twolineshloka
{निवार्य तान्बाणगणान्द्रोणो राजन्समन्ततः}
{महाजलधरान्व्योम्नि मातरिश्वेव चाबभौ}


\twolineshloka
{ततः शरं महाघोरं मूर्यपावकसन्निभम्}
{सन्दधे परवीरघ्नो वीरकेतो रथं प्रति}


\twolineshloka
{स भित्त्वा तु शरो राजन्पाञ्चालकुलनन्दनम्}
{अभ्यगाद्धरणीं तूर्णं लोहितार्द्रो ज्वलन्निव}


\twolineshloka
{ततोऽपतद्रथात्तूर्णं पाञ्चालकुलनन्दनः}
{पर्वताग्रादिव महांश्चम्पको वायुपीडितः}


\twolineshloka
{तस्मिन्हते महेष्वासे राजपुत्रे महाबले}
{पाञ्चालास्त्वरिता द्रोणं समन्तात्पर्यवारयन्}


\twolineshloka
{चित्रकेतुः सुधन्वा च चित्रवर्मा च भारत}
{तथा चित्ररथश्चैव भ्रातृव्यसनकर्शिताः}


\twolineshloka
{अभ्यद्रवन्त सहिता भारद्वाजं युयुत्सवः}
{मुञ्चन्तः शरवर्षाणि तपान्ते जलदा इव ॥`अदृश्यं समरे चक्रुर्भारद्वाजं सुधन्विनः'}


\twolineshloka
{स वध्यमानो बहुधा राजपुत्रैर्महारथैः}
{क्रोधमाहारयत्तेषामभावाय द्विजर्षभः}


\threelineshloka
{ततः शरमयं जालं द्रोणस्तेषामवासृजत्}
{ते हन्यमाना द्रोणस्य शरैराकर्णचोदितैः}
{कर्तव्यं नाभ्यजानन्वै कुमारा राजसत्तम}


\twolineshloka
{तान्विमूढान्रणे द्रोणः प्रहसन्निव भारत}
{व्यश्वसूतरथांश्चक्रे कुमारान्कुपितो रणे}


\twolineshloka
{अथापरैः सुनिशितैर्भल्लैस्तेषां महायशाः}
{पुष्पाणीव विवान्वायुः सोत्तमाङ्गान्यपातयत्}


\twolineshloka
{ते रथेभ्यो हताः पेतुः क्षितौ राजन्सुवर्चसः}
{देवासुरे पुरा युद्धे यथा दैतेयदानवाः}


\twolineshloka
{तान्निहत्य रणे राजन्भारद्वाजः प्रतापवान्}
{कार्मुकं भ्रामयामास हेमपृष्ठं दुरासदम्}


% Check verse!
`तदस्य भ्राजते राजन्मेघमध्ये तडिद्यथा'
\threelineshloka
{पाञ्चालान्निहतान्दृष्ट्वा देवकल्पान्महारथान्}
{धृष्टद्युम्नो भृशोद्विग्नो नेत्राभ्यां पातयञ्जलम्}
{अभ्यवर्तत सङ्गारमे क्रुद्धो द्रोणरथं प्रति}


\twolineshloka
{ततो हाहेति सहसा नादः समभवन्नृप}
{पाञ्चाल्येन रणे दृष्ट्वा द्रोणमावारितं शरैः}


\twolineshloka
{से च्छाद्यमानो बहुधा पार्षतेन महात्मना}
{न विव्यथे ततो द्रोणः स्मयन्नेवान्वयुध्यत}


\twolineshloka
{ततो द्रोणं महाराज पाञ्चाल्यः क्रोधमूर्च्छितः}
{आजघानोरसि क्रुद्धो नवत्या नतपर्वणाम्}


\twolineshloka
{स गाढविद्धो बलिना भारद्वाजो महायशाः}
{निषसाद रथोपस्थे कश्मलं च जगाम ह}


\twolineshloka
{तं वै तथागतं दृष्ट्वा धृष्टद्युम्नः पराक्रमी}
{चापमुत्सृज्य शीघ्रं तु असिं जग्राह वीर्यवान्}


\threelineshloka
{अवप्लुत्य रथाच्चापि त्वरितः स महारथः}
{आरुरोह रथं तूर्णं भारद्वाजस्य मारिष}
{हर्तुमिच्छञ्शिरः कायात्क्रोधसंरक्रलोचनः}


\twolineshloka
{प्रत्याश्वस्तस्ततो द्रोणो धनुर्गृह्य महारवम्}
{आसन्नमागतं दृष्ट्वा धृष्टद्युम्नं जिघांसया}


\twolineshloka
{शरैर्वैतस्तिकै राजन्विव्याधासन्नवेधिभिः}
{योधयामास समरे धृष्टद्युम्नं महारथम्}


\twolineshloka
{ते हि वैतस्तिका नाम शरा आमन्नयोधिनः}
{द्रोणस्य विहिता राजन्यैर्धृष्टद्युम्नमाक्षिणोत्}


\twolineshloka
{स वध्यमानो बहुभिः सायकैस्तैर्महाबलः}
{अवप्लुत्य रथात्तूर्णं भग्नवेगः पराक्रमी}


\twolineshloka
{आरुह्य स्वरथं वीरः प्रगृह्य च महद्धनुः}
{विव्याध समरे द्रोणं धृष्टद्युम्नो महारथः}


% Check verse!
द्रोणश्चापि महाराज शरैर्विव्याध पार्षतम्
\twolineshloka
{तदद्भुतमभूद्युद्धं द्रोणपाञ्चालयोस्तदा}
{त्रैलोक्यकाङ्क्षिणोरासीच्छक्रप्रह्लादयोरिव}


\twolineshloka
{मण्डलानि विचित्राणि यमकानीतराणि च}
{चरन्तौ युद्धमार्गज्ञौ ततक्षतुरथेषुभिः}


\threelineshloka
{मोहयन्तौ मनांस्याजौ योधानां द्रोणपार्षतौ}
{सृजन्तौ शरवर्षाणि वर्षास्विव बलाहकौ}
{छादयन्तौ महात्मानौ शरैर्व्योम दिशो महीम्}


\twolineshloka
{तदद्भुतं तयोर्युद्धं भूतसङ्घा ह्यपूजयन्}
{क्षत्रियाश्च महाराज ये चान्ये तव सैनिकाः}


\twolineshloka
{अवश्यं समरे द्रोणो दृष्टद्युम्नेन सङ्गतः}
{वशमेष्यति नो राजन्पाञ्चाला इति चुक्रुशुः}


\twolineshloka
{द्रोणस्तु त्वरितो युद्धे धृष्टद्युम्नस्य सारथेः}
{शिरः प्रच्यावयामास फलं पक्वं तरोरिव}


% Check verse!
ततस्तु प्रद्रुता वाहा राजंस्तस्य महात्मनः
\twolineshloka
{तेषु प्रद्रवमाणेषु पाञ्चालान्सृञ्जयांस्तथा}
{अयोधयद्रणे द्रोणस्तत्रतत्र पराक्रमी}


\threelineshloka
{विजित्य पाण्डुपाञ्चालान्भारद्वाजः प्रतापवान्}
{स्वं व्यूहं पुनरास्थाय स्थितोऽभवदरिन्दमः}
{न चैनं पाण़्डवा युद्धे जेतुमुत्सेहिरे प्रभो}


\chapter{अध्यायः १२३}
\twolineshloka
{सञ्जय उवाच}
{}


\twolineshloka
{ततो दुःशासनो राजञ्शैनेयं समुपाद्रवत्}
{किरञ्शरसहस्राणि पर्जन्य इव वृष्टिमान्}


\twolineshloka
{स विद्व्वा सात्यकिं षष्ट्या तथा षोडशभिः शरैः}
{नाकम्पयत्स्थितं युद्धे मैनाकमिव पर्वतम्}


\twolineshloka
{तं तु दुःशासनः शूरः सायकैरावृणोद्भृशम्}
{रथव्रातेन महता नानादेशोद्भवेन च}


\twolineshloka
{सर्वतो भरतश्रेष्ठ विसृजन्सायकान्बहून्}
{पर्जन्य इव घोषेण नादयन्वै दिशो दश}


\twolineshloka
{तमापतन्तमालोक्य सात्यकिः कौरवं रणे}
{अभिद्रुत्य महाबाहुश्छादयामास सायकैः}


\twolineshloka
{ते छाद्यमाना बाणौघैर्दुःशासनपुरोगमाः}
{प्राद्रवन्समरे भीतास्तव सैन्यस्य पश्यतः}


\twolineshloka
{तेषु द्रवत्सु राजेन्द्र पुत्रो दुःशानस्तव}
{तस्थौ व्यपेतभी राजन्सात्यकिं चार्दयच्छरैः}


\twolineshloka
{चतुर्भिर्वाजिनस्तस्य सारथिं च त्रिभिः शरैः}
{सात्यकिं च शतेनाजौ विद्ध्वा नादं मुमोच सः}


\twolineshloka
{ततः क्रुद्धो महाराज माधवस्तस्य संयुगे}
{रथं सूतं ध्वजं तं च चक्रेऽदृश्यमजिह्मगैः}


\twolineshloka
{स तु दुःशासनं शूरं सायकैरावृणोद्भृशम्}
{मशकं समनुप्राप्तमूर्णनाभिरिवोर्णया}


\twolineshloka
{त्वरन्समावृणोद्बाणैर्दुःशासनममित्रजित् ॥दृष्ट्वा दुःशासनं राजा तथा शरशताचितम्}
{}


% Check verse!
त्रिगर्तांश्चोदयामास युयुधानरथं प्रति
\twolineshloka
{तेऽगच्छन्युयुधानस्य समीपं क्रुरकर्मणः}
{त्रिगर्तानां त्रिसाहस्रा रथा युद्धविशारदाः}


\twolineshloka
{ते तु तं रथवंशेन महता पर्यवारयन्}
{स्थिरां कृत्वा मतिं युद्धे भूत्वा संशप्तका मिथः}


\twolineshloka
{तेषां प्रयतमानानां शरवर्षाणि मुञ्चताम्}
{योधान्पञ्चशतान्मुख्यानग्र्यानीके व्यपोथयत्}


\twolineshloka
{तेऽपतन्निहतास्तूर्णं शिनिप्रवरसायकैः}
{महामारुतवेगेन भग्ना इव नगाद्द्रुमाः}


\twolineshloka
{नागैश्च बहुधा च्छिन्नैर्ध्वजैश्चैव विशाम्पते}
{हयैश्च कनकापीडैः पतितैस्तत्र मेदिनी}


\twolineshloka
{शैनेयशरसङ्कृत्तैः शोणितौघपरिप्लुतैः}
{अशोभत महाराज किंशुकैरिव पुष्पितैः}


\twolineshloka
{ते वध्यमानाः समरे युयुधानेन तावकाः}
{त्रातारं नाध्यगच्छन्त पङ्कमग्ना इव द्विपाः}


\twolineshloka
{ततस्ते पर्यवर्तन्त सर्वे द्रोणरथं प्रति}
{भयात्पतगराजस्य गर्तानीव महोरगाः}


\twolineshloka
{हत्वा पञ्चशतान्योधाञ्छरैराशीविषोपमैः}
{प्रायात्स शनकैर्वीरो धनञ्जयरथं प्रति}


\twolineshloka
{तं प्रयान्तं नरश्रेष्ठं पुत्रो दुःशासनस्तव}
{विव्याध नवभिस्तूर्णं शरैः सन्नतपर्वभिः}


\twolineshloka
{स तु तं प्रतिविव्याध पञ्चभिर्निशितैः शरैः}
{रुक्मपुङ्खैर्महेष्वासो गार्ध्रपत्रैरजिह्मगैः}


\twolineshloka
{सात्यकिं तु महाराज प्रहसन्निव भारत}
{दुःशासनस्त्रिभिर्विद्ध्वा पुनर्विव्याध पञ्चभिः}


\twolineshloka
{शैनेयस्तव पुत्रं तु हत्वा पञ्चभिराशुगैः}
{धनुश्चास्य रणे छित्त्वा विस्मयन्नर्जुनं ययौ}


\twolineshloka
{ततो दुःशासनः क्रुद्धो वृष्णिवीराय गच्छते}
{सर्वपारशवीं शक्तिं विससर्ज जिघांसया}


\twolineshloka
{तां तु शक्तिं तदा घोरां तव पुत्रस्य सात्यकिः}
{चिच्छेद शतधा राजन्निशितैः कङ्कपत्रिभिः}


\twolineshloka
{अथान्यद्धनुरादाय पुत्रस्तव जनेश्वर}
{सात्यकिं च शरैर्विद्ध्वा सिंहनादं ननाद च}


\twolineshloka
{सात्यकिस्तु रणे क्रुद्धो मोहयित्वा सुतं तव}
{शरैरग्निशिखाकारैराजघान स्तनान्तरे}


\twolineshloka
{त्रिभिरेव महाभागः शरैः सन्नतपर्वभिः}
{सर्वायसैस्तीक्ष्णवक्त्रैः पुनर्विव्याध चाष्टभिः}


\twolineshloka
{दुःशासनस्तु विंशत्या सात्यकिं प्रत्यविध्यत}
{सात्वतोऽपि महाराज तं विव्याध स्तनान्तरे}


\twolineshloka
{त्रिभिरेव महाभागः शरैः सन्नतपर्वभिः}
{ततोऽस्य वाहान्निशितैः शरैर्जघ्ने महारथः}


\twolineshloka
{सारथिं च सुराङ्क्रुद्धः शरैः सन्नतपर्वभिः}
{धनुरेकेन भल्लेन हस्तावापं च पञ्चभिः}


\twolineshloka
{ध्वजं च रथशक्तिं च भल्लाभ्यां परमस्त्रिवित्}
{चिच्छेद विशिखैस्तीक्ष्णैस्तथोभौ पार्ष्णिसारथी}


\twolineshloka
{स च्छिन्नधन्वा विरथो हताश्वो हतसारथिः}
{त्रिगर्तसेनापतिना स्वरथेनापवाहितः}


\twolineshloka
{तमभिद्रुत्य शैनेयो मुहूर्तमिव भारत}
{न जघान महाबाहुर्भीमसेनवचः स्मरन्}


\twolineshloka
{भीमसेनेन तु वधः सुतानां तव भारत}
{प्रतिज्ञातः सभामध्ये सर्वेषामेव संयुगे}


\twolineshloka
{ततो दुःशासनं जित्वा सात्यकिः संयुगे प्रभो}
{जगाम त्वरितो राजन्येन यातो धनञ्जयः}


\chapter{अध्यायः १२४}
\twolineshloka
{धृतराष्ट्र उवाच}
{}


\twolineshloka
{किं तस्यां मम सेनायां नासन्केचिन्महारथाः}
{ये तथा सात्यकिं यान्तं नैवाघ्नन्नाप्यवारयन्}


\twolineshloka
{एको हि समरे कर्म कृतवान्सत्यविक्रमः}
{शक्रतुल्यबलो युद्धे महेन्द्रो दानवेष्विव}


\twolineshloka
{अथवा शून्यमासीत्तद्येन यातः स सात्यकिः}
{हतभूयिष्ठमथवा येन यातः स सात्यकिः}


\twolineshloka
{यत्कृतं वृष्णिवीरेण कर्म शंससि मे रणे}
{नैतदुत्सहते कर्तुं कर्म शक्रोऽपि सञ्जय}


\twolineshloka
{अश्रद्धेयमचिन्त्यं च कर्म तस्य महात्मनः}
{वृष्ण्यन्धकप्रवीरस्य श्रुत्वा मे व्यथितं मनः}


\twolineshloka
{न सन्ति तस्मात्पुत्रा मे यथा सञ्जय भाषसे}
{एको वै बहुलाः सेनाः प्रामृद्गात्सत्यविक्रमः}


\threelineshloka
{कथं च युध्यमानानामपक्रान्ती महात्मनाम्}
{एको बहूनां शैनेयस्तन्ममाचक्ष्व सञ्जय ॥सञ्जय उवाच}
{}


\twolineshloka
{राजन्सेनासमुद्योगो रथनागाश्वपत्तिमान्}
{अतुल्यस्तव सेनायां युगान्तसदृशोऽभवत्}


\twolineshloka
{आहूतेषु समूहेषु तव सैन्यस्य मानद}
{नास्ति लोके समः कश्चित्समूह इति मे मतिः}


\twolineshloka
{तत्रव देवास्त्वभाषन्त चारणाश्च समागताः}
{एतदन्ताः समूहा वै भविष्यन्ति महीतले}


\twolineshloka
{न च वै तादृशो व्यूह आसीत्कश्चिद्विशाम्पते}
{यादृग्जयद्रथवधे द्रोणेन विहितोऽभवत्}


\twolineshloka
{`उद्धृता पृथिवी नूनं युद्धहेतोः समागतैः}
{इति तत्र जनाश्चाहुर्दृष्टा तां जनसंसदम्'}


\twolineshloka
{चण्डवातविभिन्नानां समुद्राणामिव स्वनः}
{रणेऽभवद्बलौघानामन्योन्यमभिधावताम्}


\twolineshloka
{पार्थिवानां समेतानां बहून्यासन्नरोत्तम}
{त्वद्बले प्राण्डवानां च सहस्राणि शतानि च}


\twolineshloka
{संरब्धानां प्रवीराणां समरे दृढकर्मणाम्}
{तत्रासीत्सुमहाशब्दस्तुमुलो रोमहर्षणः}


\twolineshloka
{`पाण़्डवानां कुरूणां च गर्जतामितरेतरम्}
{क्ष्वेलाः किलकिलाशब्दास्तत्रासन्वै सहस्रशः}


\twolineshloka
{भेरीशब्दाश्च तुमुला बाणशब्दाश्च भारत}
{अन्योन्यं निघ्नतां चैव नराणां शुश्रुवे स्वनः'}


\twolineshloka
{अथाक्रन्दन्भीसमेनो धृष्टद्युम्नश्च मारिष}
{नकुलः सहदेवश्च धर्मराजश्च पाण्डवः}


\twolineshloka
{आगच्छत प्रहरत द्रुतं विपरिधावत}
{प्रविष्टावरिसेनां हि वीरौ माधवपाण्डवौ}


\twolineshloka
{यथा सुखेन गच्छेतां जयद्रथवधं प्रति}
{तथा प्रकुरुत क्षिप्रमिति सैन्यान्यचोदयन्}


% Check verse!
तयोरभावे कुरवः कृतार्थाः स्युर्वयं जिताः
\threelineshloka
{`यत्र यातौ महात्मानौ तूर्णं परपुरञ्जयौ'}
{ते यूयं सहिता भूत्वा तूर्णमेव बलार्णवम्}
{क्षोभयध्वं महावेगाः पवनः सागरं यथा}


\twolineshloka
{भीमसेनेन ते राजन्पाञ्चाल्येन च नोदिताः}
{आजघ्नुः कौरवान्सङ्ख्ये त्यक्त्वाऽसूनात्मनः प्रियान्}


\twolineshloka
{इच्छन्तो निधनं युद्धे शस्त्रैरुत्तमतेजसः}
{स्वर्गेप्सवो मित्रकार्ये नाभ्यनन्दन्त जीवितम्}


\twolineshloka
{तथैव तावका राजन्प्रार्थयन्तो महद्यशः}
{आर्यां युद्धे मतिं कृत्वा युद्धायैवावतस्थिरे}


\twolineshloka
{तस्मिन्सुतुमुले युद्धे वर्तमाने भयावहे}
{जित्वा सर्वाणि सैन्यानि प्रायात्सात्यकिरर्जुनं}


\threelineshloka
{कवचानां प्रभास्तत्र सूर्यरश्मिविराजिताः}
{दृष्टीः सङ्ख्ये सैनिकानां प्रतिजघ्नुः समन्ततः}
{`ध्वजशस्त्रप्रतिहता लोकान्समवदीपयन्}


\twolineshloka
{चण्डवातोद्धतान्मेघान्विकिरन्रश्मिमानिव}
{तथा तव महत्सैन्यं तद्व्यरोचत तापयन्}


\twolineshloka
{सम्प्रहृष्टः स सहसा तव सैन्यार्णवं प्रति}
{लोलयन्सर्वतो गत्वा समुद्रं मकरो यथा}


\twolineshloka
{तत्र राजन्महानासीत्सङ्ग्रामो भीतिवर्धनः}
{पाण्डवस्य महाबाहो तावकानां च दारुणः}


\twolineshloka
{रुद्रस्याक्रीडसङ्काशः संहारः सर्वदेहिनाम्}
{ततः शब्दो महानासीत्प्रायाद्यत्र धनञ्चयः}


\twolineshloka
{तत्र स्म कदनं घोरं वर्तते पाण्डुपूर्वज}
{अर्जुनस्य महाबाहोस्तावकानां च धन्विनाम्}


\twolineshloka
{मध्ये भारतसैन्यस्य माधवस्य महारणे}
{द्रोणस्यापि परैः सार्धं व्यूहद्वारे सुदारुणम्}


\twolineshloka
{एवमेष क्षयो वृत्तः पृथिव्यां पृथिवीपते}
{क्रुद्धेऽर्जुने तथा द्रोणे सात्वते च महारथे'}


\twolineshloka
{तथा प्रयतमानानां पाण्डवानां महात्मनाम्}
{दुर्योधनो महाराजन्व्यगाहत महद्बलम्}


\threelineshloka
{स सन्निपातस्तुमुलस्तेषां तस्य च भारत}
{अभवत्सर्वभूतानामभावकरणो महान् ॥धृतराष्ट्र उवाच}
{}


\twolineshloka
{तथा यातेषु सैन्येषु तथा कृच्छ्रगतः स्वयम्}
{कच्चिद्दुर्योधनः सूत नाकार्षीत्पृष्ठतो रणम्}


\twolineshloka
{एकस्य च बहूनां च सन्निपातो महाहवे}
{विशेषतो नरपतेर्विषमः प्रतिभाति मे}


\threelineshloka
{सोऽत्यन्तसुखसंवृद्धो लक्ष्म्या लोकस्य चेश्वरः}
{एको बहून्समासाद्य कच्चिन्नासीत्पराङ्मुखः ॥सञ्जय उवाच}
{}


\twolineshloka
{राजन्सङ्ग्राममाश्चर्यं तव पुत्रस्य भारत}
{एकस्य बहुभिः सार्धं शृणुष्व गदतो मम}


\twolineshloka
{दुर्योधनेन समरे पृतना पाण्डवी रणे}
{नलिनी द्वरिदेनेव समन्तात्प्रतिलोडिता}


\twolineshloka
{ततस्तां प्रहतां सेनां दृष्ट्वा पुत्रेण ते नृप}
{भीमसेनपुरोगास्तं पाञ्चालाः समुपाद्रुवन्}


\twolineshloka
{स भीमसेनं दशभिः शरैर्विव्याध पाण्डवम्}
{त्रिभिस्त्रिभिर्यमौ वीरौ धर्मराजं च सप्तभिः}


\twolineshloka
{विराटद्रुपदौ षद्भिः शतेन च शिखण्डिनम्}
{धृष्टद्युम्नं च विंशत्या द्रौपदेयांस्त्रिभिस्त्रिभिः}


\twolineshloka
{शतशश्चापरान्योधान्सद्विपांश्च रथान्रणे}
{शरैरवचकर्तोग्रैः क्रुद्धोऽन्तक इव प्रजाः}


\twolineshloka
{न सन्दधद्विमुञ्चन्वा मण्डलीकृतकार्मुकः}
{अदृश्यत रिपून्निघ्नञ्छिक्षयाऽस्तबलेन च}


\twolineshloka
{तस्य तान्निघ्नतः शत्रून्हेमपृष्ठं महद्धनुः}
{अजस्रं मण्डलीभूतं ददृशुः समरे जनाः}


\twolineshloka
{ततो युधिष्ठिरो राजा भल्लाभ्यामच्छिनद्धनुः}
{तव पुत्रस्य कौरव्य यतमानस्य संयुगे}


\twolineshloka
{विव्याध चैनं दशभिः सम्यगस्तैः शरोत्तमैः}
{वर्म चाशु समासाद्य ते भित्त्वा क्षितिमाविशन्}


\twolineshloka
{ततः प्रमुदिताः पार्थाः परिवव्रुर्युधिष्ठिरम्}
{यथा वृत्रवधे देवाः पुरा शक्रं महर्षयः}


\twolineshloka
{ततोऽन्यद्धनुरादाय तव पुत्रः प्रतापवान्}
{तिष्ठेतिष्ठेति राजानं ब्रुवन्पाण्डवमभ्ययात्}


\twolineshloka
{तमायान्तमभिप्रेक्ष्य तव पुत्रं महामृधे}
{प्रत्युद्ययुः समुदिताः पाञ्चाला जयगृद्धिनः}


\twolineshloka
{तान्द्रोणः प्रतिजग्राह परीप्सन्युधि पाण्डवम्}
{चण्डवातोद्धुतान्मेघान्गिरिरम्बुमुचो यथा}


\twolineshloka
{तत्र राजन्महानासीत्सङ्ग्रामो रोमहर्षणः}
{पाण्डवानां महाबाहो तावकानां च संयुगे}


% Check verse!
रुद्रस्याक्रीडसदृशः संहारः सर्वदेहिनाम्
\twolineshloka
{ततः शब्दो महानासीद्यातो येन धनञ्जयः}
{अतीव सर्वशब्देभ्यो रोमहर्षकरः प्रभो}


\threelineshloka
{अर्जुनस्य महाबाहो तावकानां च धन्विनाम्}
{मध्ये भारतसैन्यस्य माधवस्य महारणे}
{द्रोणस्यापि परैः सार्धं व्यूहद्वारे महारणे}


\twolineshloka
{एवमेष क्षयो वृत्तः पृथिव्यां पृथिवीपते}
{क्रुद्धेऽर्जुने तथा द्रोणे सात्वते च महारथे}


\chapter{अध्यायः १२५}
\twolineshloka
{सञ्जय उवाच}
{}


\twolineshloka
{अपराह्णे महाराज सङ्ग्रामः सुमहानभूत्}
{पर्जन्यसमनिर्घोषः पुनर्द्रोणस्य सोमकैः}


\twolineshloka
{शोणाश्वं रथमास्थाय नरवीरः समाहितः}
{समरेऽभ्यद्रवत्पाण्डूञ्जवमास्थाय मध्यमम्}


\threelineshloka
{तव प्रियहिते युक्तो महेष्वासो महाबलः}
{चित्रपुङ्खैः शितैर्बाणैः कलशोत्तमसम्भवः}
{`जघान सोमकान्राजन्सृञ्जयान्केकयानपि'}


\twolineshloka
{वरान्वरान्हि योधानां विचिन्वन्निव भारत}
{आक्रीडत रमे राजन्भारद्वाजः प्रतापवान्}


\twolineshloka
{तमभ्ययाद्बृहत्क्षत्रः केकयानां महारथः}
{भातॄणां नृप पञ्चानां श्रेष्ठः समरकर्कशः}


\twolineshloka
{विमुञ्चन्विशिखांस्तीक्ष्णानाचार्यं भृशमार्दयत्}
{महामेघो यथा वर्षं विमुञ्चन्गन्धमादने}


\twolineshloka
{तस्य द्रोणो महाराज स्वर्णपुङ्खाञ्शिलाशितान्}
{प्रेषयामास सङ्क्रुद्ध- सायकान्दश पञ्च च}


\twolineshloka
{तांस्तु द्रोणविनिर्मुक्तान्क्रुंद्धाशीविषसन्निभान्}
{एकैकं पञ्चभिर्बाणैर्युधि चिच्छेद हृष्टवत्}


\twolineshloka
{तदस्य लाघवं दृष्ट्वा प्रहस्य द्विजपुङ्गवः}
{प्रेषयामास विशिखानष्टौ सन्नतपर्वणः}


\twolineshloka
{तान्दृष्ट्वा पततस्तूर्णं द्रोणचापच्युताञ्शरान्}
{अवारयच्छरैरेव तावद्भिर्निशितैर्मृधे}


\twolineshloka
{ततोऽभवन्महाराज तव सैन्यस्य विस्मयः}
{बृहत्क्षत्रेण तत्कर्म कृतं दृष्ट्वा सुदुष्करम्}


\twolineshloka
{ततो द्रोणो महाराज बृहत्क्षत्रं विशेषयन्}
{प्रादुश्चक्रे रणे दिव्यं ब्राह्ममस्त्रं सुदुर्जयम्}


\twolineshloka
{कैकेयोऽस्त्रं समालोक्य मुक्तं द्रोणेन संयुगे}
{ब्रह्मास्त्रेणैव राजेन्द्र ब्राह्ममस्त्रमशातयत्}


\twolineshloka
{ततोऽस्त्रे निहते ब्राह्मे बृहत्क्षत्रस्तु भारत}
{विव्याध ब्राह्मणं षष्ट्या स्वर्णपुङ्खैः शिलाशितैः}


\twolineshloka
{तं द्रोणो द्विपदां श्रेष्ठो नाराचेन समार्पयत्}
{स तस्य कवचं भित्त्वा प्राविशद्धरणीतलम्}


\twolineshloka
{कृष्णसर्पो यथा मुक्तो वल्मीकं नृपसत्तम}
{तथाऽत्यगान्महीं बाणो भित्त्वा कैकेयमाहवे}


\twolineshloka
{सोऽतिविद्धो महाराज कैकेयो द्रोणसायकैः}
{क्रोधेन महताऽऽविष्टो व्यावृत्य नयने शुभे}


\twolineshloka
{द्रोणं विव्याध सप्तत्या स्वर्णपुङ्खैः शिलाशितैः}
{सारथिं चास्य बाणेन भृशं मर्मस्वताडयत्}


\twolineshloka
{द्रोणस्तु बहुभिर्विद्धो बृहत्क्षत्रेण मारिष}
{असृजद्विशिखांस्तीक्ष्णान्कैकेयस्य रथं प्रति}


\twolineshloka
{व्याकुलीकृत्य तं द्रोणो बृहत्क्षत्रं महारथम्}
{अश्वांश्चतुर्भिरन्यवधीच्चतुरोऽस्य पतत्त्रिभिः}


\twolineshloka
{सूतं चैकेन बाणेन रथनीडादपातयत्}
{द्वाभ्यां ध्वजं च च्छत्रं च च्छित्त्वा भूमावपातयत्}


\twolineshloka
{ततः साधुविसृष्टेन नाराचेन द्विजर्षभः}
{हृद्यविध्यद्बृहत्क्षत्रं स च्छिन्नहृदयोऽपतत्}


\twolineshloka
{बृहत्क्षत्रे हते राजन्केकयानां महारथे}
{शैशुपालिरभिक्रुद्धो यन्तारमिदमब्रवीत्}


\twolineshloka
{सारथे याहि यत्रैष द्रोणस्तिष्ठति दंशितः}
{विनिघ्नन्केकयान्सर्वान्पाञ्चालानां च वाहिनीं}


\twolineshloka
{तस्य तद्वचनं श्रुत्वा सारथी रथिनां वरम्}
{द्रोणाय प्रापयामास काम्भोजैर्जवनैर्हयैः}


\twolineshloka
{धृष्टकेतुश्च चेदीनामृषभोऽतिबलोदितः}
{वधायाभ्यद्रवद्द्रोणं पतङ्ग इव पावकम्}


\twolineshloka
{सोऽविध्यत तदा द्रोणं षष्ट्या साश्वरथध्वजम्}
{पुनश्चान्यैः सऱैस्तीक्ष्णैः सुप्तं व्याघ्रं तुदन्निव}


% Check verse!
तस्य द्रोणो धनुर्मध्ये क्षुरप्रेण शितेन च ॥चकर्त गार्ध्रपत्रेण यतमानस्य शुष्मिणः
\twolineshloka
{अथान्यद्धनुरादाय शैशुपालिर्महारथः}
{विव्याध सायकैर्द्रोणं कङ्कबर्हिणवाजितैः}


\twolineshloka
{तस्य द्रोणो हयान्हत्वा चतुर्भिश्चतुरः शरैः}
{सारथेश्च शिरः कायाच्चकर्त प्रहसन्निव}


% Check verse!
अथैनं पञ्चविंशत्या सायकानां समार्पयत्
\twolineshloka
{अवप्लुत्य रथाच्चैद्यो गदामादाय सत्वरः}
{भारद्वाजाय चिक्षेप रुपितामिव पन्नगीम्}


\threelineshloka
{तामापतन्तीमालोक्य कालरात्रिमिवोद्यताम्}
{अश्मसारमयीं गुर्वीं तपनीयविभूषिताम्}
{शरैरनेकसाहस्रैर्भाद्वाजोऽच्छिनच्छितैः}


\twolineshloka
{सा छिन्ना बहुभिर्बाणैभारद्वाजेन मारिष}
{गदा पपात कौरव्य नादयन्ती धरातलम्}


\twolineshloka
{गदां विनिहतां दृष्ट्वा धृष्टकेतुरमर्षणः}
{तोमरं व्यसृजद्वीरः शक्तिं च कन कोज्ज्वलाम्}


\twolineshloka
{तोमरं पञ्चभिर्भित्त्वा शक्तिं चिच्छेद पञ्चभिः}
{तौ जग्मतुर्महीं छिन्नौ सर्पाविव गरुत्मता}


\twolineshloka
{ततोऽस्य विशिखं तीक्ष्णं वधाय वधकाङ्क्षिणः}
{प्रेषयामास समरे भारद्वाजः प्रतापवान्}


\twolineshloka
{स तस्य कवचं भित्त्वा हृदयं चामितौजसः}
{अभ्यगाद्धरणीं बाणो हंसः पद्मवनं यथा}


\twolineshloka
{पतङ्गं हि ग्रसेच्चाषो यथा क्षुद्रं बुभुक्षितः}
{तथा द्रोणोऽग्रसच्छूरो धृष्टकेतुं महाहवे}


\twolineshloka
{निहते चेदिराजे तु तत्खण्डं पित्र्यमाविशत्}
{अमर्षवशमापन्नः पुत्रोऽस्य परमास्त्रवित्}


\twolineshloka
{तमपि प्रहसन्द्रोणः शरैर्नित्ये यमक्षयम्}
{महाव्याघ्रो महारण्ये मृगशावं यथा बली}


\twolineshloka
{तेषु प्रक्षीयमाणेषु पाण्डवेयेषु भारत}
{जरासन्धसुतो वीरः स्वयं द्रोणमुपाद्रवत्}


\twolineshloka
{स तु द्रोणं महाबाहुः शरधाराभिराहवे}
{अदृश्यमकरोत्तूर्णं जलदो भास्करं यथा}


\twolineshloka
{तस्य तल्लाघवं दृष्ट्वा द्रोणः क्षत्रियमर्दनः}
{व्यसृजत्सायकांस्तूर्णं शतशोऽथ सहस्रशः}


\twolineshloka
{छादयित्वा रमे द्रोणं रथस्थं रथिनां वरम्}
{जारासन्धिं जघानाशु मिषतां सर्वधन्विनाम्}


\twolineshloka
{योयो विधीयते खण्डस्तंतं द्रोणोऽन्तकोपमः}
{आदत्त सर्वभूतानि प्राप्ते काले यथान्तकः}


\twolineshloka
{ततो द्रोणो महाराज नाम विश्राव्य संयुगे}
{शरैरनेकसाहस्रैः पाण्डवेयान्समावृणोत्}


\twolineshloka
{ते तु नामाङ्किता बाणा द्रोणेनास्ताः शिलाशिताः}
{नरान्नागान्हयांश्चैव निजघ्नुः शतशो मृधे}


\twolineshloka
{ते वध्यमाना द्रोणेन शक्रेणेव महासुराः}
{समकम्पन्त पाञ्चाला गावः शीतार्दिता इव}


\twolineshloka
{ततो निष्टानको घोरः पाण्डवानामजायत}
{द्रोणेन वध्यमानेषु सैन्येषु भरतर्षभ}


\twolineshloka
{प्रताप्यमानाः सूर्येण हन्यमानाश्च सायकैः}
{अन्वपद्यन्त पाञ्चालास्तदा सन्त्रस्तचेतसः}


\twolineshloka
{मोहिता बाणजालेन भारद्वाजेन संयुगे}
{ऊरुग्राहगृहीतानां पाञ्चालानां महारथाः}


\twolineshloka
{चेदयश्च महाराज सृञ्जयाः काशिकोसलाः}
{अभ्यद्रवन्त संहृष्टा भारद्वाजं युयुत्सया}


\twolineshloka
{ब्रुवन्तश्च रणेऽन्योन्यं चेदिपाञ्चालसृञ्जयाः}
{हत द्रोणं हत द्रोणमिति ते द्रोणमभ्ययुः}


\twolineshloka
{यतन्तः पुरुपव्याघ्राः सर्वशक्त्या महाद्युतिम्}
{निनीपवो रणे द्रोणं यमस्य सदनं प्रति}


\twolineshloka
{यतमानांस्तु तान्वीरान्भारद्वाजः शिलीमुखैः}
{यमाय प्रेपयामास चेदिमुख्यान्विशेपतः}


\twolineshloka
{तेषु प्रक्षीयमाणेषु चेदिमुख्येषु सर्वशः}
{पाञ्चलाः समकम्पन्त द्रोणसायकपीडिताः}


\threelineshloka
{प्राक्रोशन्भीमसेनं ते धृष्टद्युम्नं च भारत}
{दृष्ट्वा द्रोणस्य कर्माणि तथारूपाणि मारिष ॥भीम उवाच}
{}


\twolineshloka
{ब्राह्मणेन तपो नूनं चरितं दुश्चरं महत्}
{तथा हि युधि सङ्क्रुद्धो दहति क्षत्रियर्पभान्}


\twolineshloka
{धर्मो युद्धं क्षत्रियस्य ब्राह्मणस्य परं तपः}
{तपस्वी कृतविद्यश्च प्रेक्षितेनापि निर्दहेत्}


\twolineshloka
{द्रोणाग्निमस्त्रसंस्पर्शं प्रविष्टाः क्षत्रियर्पभाः}
{बहवो दुस्तरं घोरं यत्रादह्यन्त भारत}


\threelineshloka
{यथाबलं यथोत्साहं यथासत्वं महाद्युतिः}
{मोहयन्सर्वभूतानि द्रोणो हन्ति बलानि नः ॥सञ्जय उवाच}
{}


\twolineshloka
{तस्य तद्वचनं श्रुत्वा क्षत्रधर्मा व्यवस्थितः}
{अर्धचन्द्रेण चिच्छेद द्रोणस्य सशरं धनुः}


% Check verse!
स संरब्धतरो भूत्वा द्रोणः क्षत्रियमर्दनः
\twolineshloka
{अन्यत्कार्मुकमादाय भास्वरं वेगवत्तरम्}
{तत्राधाय शरं तीक्ष्णं परानीकविशातनम्}


\twolineshloka
{आकर्णपूर्णमाचार्यो बलवानभ्यवासृजत्}
{स हत्वा क्षत्रधर्माणं जगाम धरणीतलम्}


\twolineshloka
{स भिन्नहृदयो वाहान्न्यपतन्मेदिनीतले}
{ततः सैन्यान्यकम्पन्त धृष्टद्युम्नसुते हते}


\twolineshloka
{अथ द्रोणं समारोहच्चेकितानो महाबलः}
{स द्रोणं दशभिर्विद्ध्वा प्रत्यविद्ध्यत्स्तनान्तरे}


\twolineshloka
{चतुर्भिः सारथिं चास्य चतुर्भिश्चतुरो हयान्}
{तमाचार्यस्त्रिभिर्बाणैर्बाह्वोरुरसि चार्दयत्}


\twolineshloka
{ध्वजं सप्तभिरुन्मथ्य यन्तारमवधीत्त्रिभिः}
{तस्य सूते हते तेऽश्वा रथमादाय विद्रुताः}


\twolineshloka
{समरे शरसंवीता भारद्वाजेन मारिष}
{चेकितानरथं दृष्ट्वा हताश्वं हतसारथिम्}


\twolineshloka
{तान्समेतान्रणे शूरांश्चेदिपाञ्चालसृञ्जयान्}
{समन्ताद्द्रावयन्द्रोणो बह्वशोभत मारिष}


\twolineshloka
{आकर्णपलितः श्यामो वयसाऽशीतिपञ्चकः}
{रणे पर्यचरद्द्रोणो वृद्धः षोडशवर्षवत्}


\twolineshloka
{अथ द्रोणं महाराज विचरन्तमभीतवत्}
{वज्रहस्तममन्यन्त शत्रवः शत्रुसूदनम्}


\twolineshloka
{ततोऽब्रवीन्महाबाहुर्द्रुपदो बुद्धिमान्नृप}
{लुब्धोऽयं क्षत्रियान्हन्ति व्याघ्रः क्षुद्रमृगानिव}


\twolineshloka
{कृच्छ्रान्दुर्योधनो लोकान्पापः प्राप्स्यति दुर्मतिः}
{यस्य लोभाद्विनिहताः समरे क्षत्रियर्षभाः}


\threelineshloka
{शतशः शेरते भूमौ निकृत्ता गोवृषा इव}
{रुधिरेण परीताङ्गाः श्वशृगालादनीकृताः ॥सञ्जय उवाच}
{}


\twolineshloka
{एवमुक्त्वा महाराज द्रुपदोऽक्षौहिणीपतिः}
{पुरस्कृत्य रणे पार्थान्द्रोणमभ्यद्रवद्द्रुतम्}


\chapter{अध्यायः १२६}
\twolineshloka
{सञ्जय उवाच}
{}


\twolineshloka
{व्यूहेष्वालोड्यमानेषु पाण्डवानां ततस्ततः}
{सुदूरमन्वयुः पार्थाः पाञ्चालाः सह सोमकैः}


\twolineshloka
{वर्तमाने तथा रौद्रे सङ्ग्रामे रोमहर्षणे}
{सङ्क्षये जगतस्तीव्रे युगान्त इव भारत}


\twolineshloka
{द्रोणे युधि पराक्रान्ते नर्दमाने मुहुर्मुहुः}
{पाञ्चालेषु च क्षीणेषु वध्यमानेषु पाण्डुषु}


\twolineshloka
{नापश्यच्छरणं कञ्चिद्वर्मराजो युधिष्ठिरः}
{चिन्तयामास राजेन्द्र कथमेतद्भविष्यति}


\twolineshloka
{ततो विक्ष्य दिशः सर्वाः सव्यसाचिदिदृक्षया}
{युधिष्ठिरो ददर्शाथ नैव पार्थं न माधवम्}


\twolineshloka
{सोऽपश्यन्नरशार्दूलं वानरर्षभलक्षणम्}
{गाण़्डीवस्य च निर्घोषमशृण्वन्व्यथितेन्द्रियः}


\threelineshloka
{अपश्यन्सात्यकिं चापि वृष्णीनां प्रवरं रथम्}
{चिन्तयाभिपरीताङ्गो धर्मराजो युधिष्ठिरः}
{नाध्यगच्छत्तदा शान्ति तावपश्यन्नरोत्तमौ}


\twolineshloka
{लोकोपक्रोशभीरुत्वाद्वर्मराजो महामनाः}
{अचिन्तयन्महाबाहुः शैनेयस्य रथं प्रति}


\twolineshloka
{पदवीं प्रेषितश्चैव फल्गुनस्य मया रणे}
{शैनेयः सात्यकिः सत्यो मित्राणामभयङ्करः}


\twolineshloka
{तदिदं ह्येकमेवासीद्द्द्विधा जातं ममाऽद्य वै}
{सात्यकिश्च हि विज्ञेयः पाण्डवश्च धनञ्जयः}


\twolineshloka
{सात्यकिं प्रेषयित्वा तु पाण्डवस्य पदानुगम्}
{सात्वतस्यापि कं युद्धे प्रेषयिष्ये पदानुगम्}


\twolineshloka
{परिष्यामि प्रयत्नेन भ्रातुरन्वेषणं यदि}
{युयुधानमनन्विष्य लोको मां गर्हयिष्यति}


\twolineshloka
{भ्रातुरन्वेषणं कृत्वा धर्मपुत्रो युधिष्ठिरः}
{परित्यजति वार्ष्णेयं सात्यकिं सत्यविक्रमम्}


\twolineshloka
{लोकापवादभीरुत्वात्सोऽहं पार्थं वृकोदरम्}
{पदवीं प्रेषयिष्यामि माधवस्य महात्मनः}


\twolineshloka
{यथैव च मम प्रीतिरर्जुने शत्रुमूदने}
{तथैव वृष्णिवीरेऽपि सात्वते युद्धदुर्मदे}


\threelineshloka
{अतिभारे नियुक्तश्च मया शैनेयनन्दनः}
{स तु मित्रोपरोधेन गौरवात्तु महाबलः}
{प्रविष्टो भारतीं सेनां मकरः सागरं यथा}


\twolineshloka
{असौ हि श्रूयते शब्दः शूराणामनिवर्तिनाम्}
{मिथः संयुध्यमानानं वृष्णिवीरेण धीमता}


\threelineshloka
{प्राप्तकालं सुबलवन्निश्चितं बहुधा हि मे}
{तत्रैव पाण्डवेयस्य भीमसेनस्य धन्विनः}
{गमनं रोचते मह्यं यत्र यातौ महारथौ}


% Check verse!
न चाप्यसह्यं भीमस्य विद्यते भुवि किञ्चिन
\twolineshloka
{शक्तो ह्येष रणे यत्तः पृथिव्यां सर्वधन्विनाम्}
{स्वबाहुबलमास्थाय प्रतिव्यूहितुमञ्जसा}


\twolineshloka
{यस्य बाहुबलं सर्वे समाश्रित्य महात्मनः}
{वनवासान्निवृत्ताः स्म न च युद्धेषु निर्जिताः}


\twolineshloka
{इतो गते भीमसेने सात्वतं प्रति पाण़्डवे}
{सनाथौ भवितारौ हि युधि सात्वतफल्गुनौ}


\twolineshloka
{कामं त्वशोचनीयौ तौ रणे सात्वतफल्गुनौ}
{रक्षितौ वासुदेवेन स्वयं शस्त्रविशारदौ}


\twolineshloka
{अवश्यं तु मया कार्यमात्मनः शोकनाशनम्}
{तस्माद्भीमं नियोक्ष्यामि सात्वतस्य पदानुगम्}


\twolineshloka
{ततः प्रतिकृतं मन्ये विधानं सात्यकिं प्रति ॥एवं निश्चित्य मनसा धर्मपुत्रो युधिष्ठिरः}
{}


% Check verse!
यन्तारमब्रवीद्राजा भीमं प्रति नयस्व माम्
\twolineshloka
{धर्मराजवचः श्रुत्वा सारथिर्हयकोविदः}
{रथं हेमपरिष्कारं भीमान्तिकमुपानयत्}


\twolineshloka
{भीमसेनमनुज्ञाप्य प्राप्तकालमचिन्तयत्}
{कश्मलं प्राविशद्राजा बहु तत्र समादिशत्}


\threelineshloka
{स कश्मलसमाविष्टो भीममाहूय पार्थिवः}
{अब्रवीद्वचनं राजन्कुन्तीपुत्रो युधिष्ठिरः}
{}


\twolineshloka
{अब्रवीद्वचनं राजन्कुन्तीपुत्रो युधिष्ठिरः ॥यः सदेवान्सगन्धर्वान्दैत्यांश्चैकरथोऽजयत्}
{}


\threelineshloka
{तस्य लक्ष्म न पश्यामि भीमसेनानुजस्य ते ॥सञ्जय उवाच}
{ततोऽब्रवीद्धर्मराजं भीमसेनस्तथागतम्}
{नैवाद्राक्षं नचाश्रौषं तव कश्मलमीदृशम्}


\twolineshloka
{पुरातिदुःखदीर्णानां भवान्गतिरभूद्धि नः}
{उत्तिष्ठोत्तिष्ठ राजेन्द्र शाधि किं करवाणि ते}


\threelineshloka
{न ह्यसाध्यमकार्यं वा विद्यते मम मानद}
{आज्ञापय कुरुश्रेष्ठ मा च शोके मनः कृथाः ॥सञ्जय उवाच}
{}


\twolineshloka
{तमब्रवीदश्रुपूर्णः कृष्णसर्प इव श्वसन्}
{भीमसेनमिदं वाक्यं प्रम्लानवदनो नृपः}


\twolineshloka
{श्रूयते पाञ्चजन्यस्य यथा शङ्खस्य निस्वनः}
{पूरितो वासुदेवेन संरब्धेन यशस्विना}


\twolineshloka
{नूनमद्य हतः शेते तव भ्राता धनञ्जयः}
{तस्मिन्विनिहते नूनं युध्यतेऽसौ जनार्दनः}


\twolineshloka
{यस्य सत्ववतो वीर्यं ह्युपजीवन्ति पाण्डवाः}
{यं भयेष्वभिगच्छन्ति सहस्राक्षमिवामराः}


\twolineshloka
{स शूरः सैन्धवप्रेप्सुरन्वयाद्भारतीं चमूम्}
{तस्य वै गमनं विद्मो भीम नावर्तनं पुनः}


\twolineshloka
{श्यामो युवा गुडाकेशो दर्शनीयो महारथः}
{व्यूढोरस्को महाबाहुर्मत्तद्विरदविक्रमः}


\twolineshloka
{चकोरनेत्रस्ताम्रास्यो द्विषतां भयवर्धनः}
{`मम प्रियहितार्थं च शक्रलोकादिहागतः}


\twolineshloka
{वृद्धोपसेवी धृतिमान्कृतज्ञः सत्यसङ्गरः}
{प्रविष्टो महतीं सेनामपर्यन्तां धनञ्जयः}


\twolineshloka
{प्रविष्टे च चमूं घोरामर्जुने शत्रुनाशने}
{प्रेषितः सात्वतो वीरः फल्गुनस्य पदानुगः}


\twolineshloka
{तस्याभिगमनं जाने भीम नावर्तनं पुनः'}
{तदिदं मम भद्रं ते शोकस्थानमरिन्दम}


\twolineshloka
{अर्जुनार्थे महाबाहो सात्वतस्य च कारणात्}
{वर्धते हविषेवाग्निरिध्यमानः पुनः पुनः}


\twolineshloka
{तस्य लक्ष्म न पश्यामि तेन विन्दामि कश्मलम्}
{तं विद्धि पुरुषव्याघ्रं सात्वतं च महारथम्}


\twolineshloka
{स तं महारथं पश्चादनुयातस्तवानुजम्}
{तमपश्यन्महाबाहुमदं विन्दामि कश्मलम्}


\twolineshloka
{`अथैनं पुनराचक्ष्व लोहिताक्षं सकेशवम्}
{दृष्ट्वा कुशलिनं पार्थं सैन्धवान्ते ससात्यकिम्}


% Check verse!
संविदं मम कुर्यास्त्वं सिंहनादेन पाण्डव
\twolineshloka
{नूनं विनिहतः शूरः सव्यासाची परन्तप'}
{पार्थे तस्मिन्हते चैव युध्यते गरुडध्वजः}


\twolineshloka
{सहयो नास्य वै कश्चित्तेन विन्दामि कश्मलम्}
{तस्मात्कृष्णो रणे नूनं युध्यते युद्धकोविदः}


\twolineshloka
{न हि मे शुध्यते भावस्तयोरेव परन्तप}
{स तत्र गच्छ कौन्तेय यत्र यातो धनञ्जयः}


\twolineshloka
{सात्यकिश्च महावीर्यः कर्तव्यं यदि मन्यसे}
{वचनं मम धर्मज्ञ भ्राता ज्येष्ठो भवामि ते}


% Check verse!
न मेऽर्जुनस्तथा ज्ञेयो ज्ञातव्यः सात्यकिर्यथा
\twolineshloka
{चिकीर्षुर्मत्प्रियं पार्थ स यातः सव्यसाचिनः}
{पदवीं दुर्गमां घोरामगम्यामकृतात्मभिः}


\twolineshloka
{दृष्ट्वा कुशलिनौ कृष्णौ सात्वतं चैव सात्यकिम्}
{संविदं चैव कुर्यास्त्वं सिंहनादेन पाण़्डव}


\chapter{अध्यायः १२७}
\twolineshloka
{भीमसेन उवाच}
{}


\twolineshloka
{ब्रह्मेशानेन्द्रवरुणानवहद्यः पुरा रथः}
{तमास्थाय गतौ कृष्णौ न तयोर्विद्यते भयम्}


\threelineshloka
{तवाज्ञां शिरसा बिभ्रदेष गच्छामि मा शुचः}
{समेत्य तान्नरव्याघ्रास्तव दास्यामि संविदम् ॥सञ्जय उवाच}
{}


\threelineshloka
{एतावदुक्त्वा प्रययौ परिदाय युधिष्ठिरम्}
{धृष्टद्युम्नाय बलवान्सुहृद्भ्यश्च पुनःपुनः}
{धृष्टद्युम्नं चेदमाह भीमसेनो महाबलः}


\twolineshloka
{विदितं ते महाबाहो यथा द्रोणो महारथः}
{ग्रहणे धर्मराजस्य सर्वोपायेन वर्तते}


\twolineshloka
{न च मे गमने कृत्यं तादृक्पार्षत विद्यते}
{यादृशं रक्षणे राज्ञः कार्यमात्ययिकं हि नः}


\twolineshloka
{एवमुक्तोऽस्मि पार्थेन प्रतिवक्तुं न चोत्सहे}
{प्रयास्ये तत्र यत्रासौ मुमूर्षुः सैन्धवः स्थितः}


\twolineshloka
{धर्मराजस्य वचने स्थातव्यमविशङ्कया}
{यास्यामि पदवीं भ्रातुः सात्वतस्य च धीमतः}


\twolineshloka
{सोऽद्य यत्तो रणे रार्थं परिरक्ष युधिष्ठिरम्}
{एतद्धि सर्वकार्याणां परमं कृत्समाहवे}


\twolineshloka
{तमब्रवीन्महाराज धृष्टद्युम्नो वृकोदरम्}
{ईप्सितं ते करिष्यामि गच्छ पार्थाविचारयन्}


\threelineshloka
{नाहत्वा समरे द्रोणो धृष्टद्युम्नं कथञ्चन}
{निग्रहं धर्मराजस्य प्रकरिष्यति संयुगे ॥सञ्जय उवाच}
{}


\twolineshloka
{ततो निक्षिप्य राजानं धृष्टद्युम्ने च पाण्डवम्}
{अभिवाद्य गुरुं ज्येष्ठं प्रययौ येन फल्गुनः}


\twolineshloka
{परिष्वक्तश्च कौन्तेयो धर्मराजेन मारुतिः}
{आघ्रातश्च तथा मूर्ध्नि श्रावितश्चाशिषः शुभाः}


\twolineshloka
{[कृत्वा प्रदक्षिणान्विप्रानर्चितांस्तुष्टमानसान्}
{आलभ्य मङ्गलान्यष्टौ पीत्वा कैरातकं मधु}


\twolineshloka
{द्विगुणद्रविणो वीरो मदरक्तान्तलोचनः}
{विप्रैः कृतस्वस्त्ययनो विजयोत्पादसूचितः}


\twolineshloka
{पश्यन्नेवात्मनो बुद्धिं विजयानन्दकारिणीम्}
{अनुलोमानिलैश्चाशु प्रदर्शितजयोदयः}


\threelineshloka
{भीमसेनो महाबाहुः कवची शुभकुण्डली}
{सगदः सतलत्राणः सशरी रथिनां वरः}
{रथमारुह्य निर्युक्तं सर्वोपकरणान्वितम्'}


\twolineshloka
{तस्य कार्ष्णायसं वर्म हेमचित्रं महर्द्धिमत्}
{विबभौ सर्वतः श्लिष्टं सविद्युदिव तोयदः}


\twolineshloka
{पीतरक्तासितसितैर्वासौभिश्च सुवेष्टितः}
{कण्ठसूत्रेण विबभौ सेन्द्रायुध इवाम्बुदः}


\twolineshloka
{प्रयाते भीमसेने तु तव सैन्यं युयुत्सया}
{पाञ्चजन्यरवो घोरः पुनरासीद्विशाम्पते}


\twolineshloka
{तं श्रुत्वा निनदं घोरं त्रैलोक्यत्रासनं महत्}
{पुनर्भीमं महाबाहुं धर्मपुत्रोऽभ्यभाषत}


\twolineshloka
{एष वृष्णिप्रवीरेण ध्मातः सलिलजो भृशम्}
{पृथिवीं चान्तरिक्षं च विनादयति शङ्खराट्}


\twolineshloka
{नूनं व्यसनमापन्ने सुमहत्सव्यसाचिनि}
{कुरुभिर्युध्यते सार्धं सर्वैश्चक्रगदाधरः}


\threelineshloka
{नूनमार्या महत्कुन्ती पापं व्यसनमीदृशम्}
{द्रौपदी च सुभद्रा च पश्यन्ति सह बन्धुभिः}
{तद्भीम त्वरया युक्तो याहि यत्र धनञ्जयः}


\twolineshloka
{मुह्यतीव हि मे चित्तं धनञ्जयदिदृक्षया}
{दिशश्च प्रदिशः पार्थ सात्वतस्य च कारणात्}


\twolineshloka
{गच्छगच्छेति गुरुणा सोऽनुज्ञातो वृकोदरः}
{ततः पाण्डुसुतो राजन्भीमसेनः प्रतापवान्}


\twolineshloka
{बद्धगोधाङ्गुलित्राणः प्रगृहीतशरासनः}
{ज्येष्ठेन प्रहितो भ्रात्रा भ्राता भ्रातुः प्रियङ्करः}


\twolineshloka
{आहत्य दुन्दुभिं भीमः शङ्खं प्रध्माप्य चासकृत्}
{विनद्य सिंहनादेन ज्यां विकर्षन्पुनःपुनः}


\twolineshloka
{तेन शब्देन वीराणां पातयित्वा मनांस्युत}
{दर्शयन्घोरमात्मानममित्रान्सहसाऽभ्ययात्}


\twolineshloka
{तमूहुर्जवना दान्ता विरुवन्तो हयोत्तमाः}
{विशोकेन सुसंयत्ता मनोमारुतरंहसः}


\twolineshloka
{आरुजन्विरुजन्पार्थो ज्यां विकर्षंश्च पाणिना}
{सम्प्रकर्षन्विकर्षंश्च सेनाग्रं समलोडयत्}


\twolineshloka
{तं प्रयान्तं महाबाहुं पाञ्चालाः सहसोमकाः}
{पृष्ठतोऽनुययुः शूरा मघवन्तमिवामराः}


\twolineshloka
{तं समेत्य महाराज तावकाः पर्यवारयन्}
{दुःशासनश्चित्रसेनः कुण्डभेदी विविंशतिः}


\twolineshloka
{दुर्मुखो दुःसहश्चैव विकर्णश्च शलस्तथा}
{विन्दानुविन्दौ सुमुखो दीर्घबाहुः सुदर्शनः}


\twolineshloka
{वृन्दारकः सुहस्तश्च सुषेणो दीर्घलोचनः}
{अभयो रौद्रकर्णा च सुवर्मा दुर्विमोचनः}


\twolineshloka
{शोभन्तो रथिनां श्रेष्ठाः सहसैन्यपदानुगाः}
{संयत्ताः समरे वीरा भीमसेनमुपाद्रवन्}


\threelineshloka
{तैः समन्ताद्वृतः शूरैः समरेषु महारथः}
{तान्समीक्ष्य तु कौन्तेयो भीमसेनः पराक्रमी}
{अभ्यवर्तत वेगेन सिंहः क्षुद्रमृगानिव}


\twolineshloka
{ते महास्त्राणि दिव्यानि तत्र वीरा अदर्शयन्}
{छादयन्तः शरैर्भीमं मेघाः सूर्यमिवोदितम्}


\twolineshloka
{ततः क्रुद्धो महाराज भीमसेनः पराक्रमी}
{अग्रतः स्यन्दनानीकं शरवर्षैरवाकिरत्}


\twolineshloka
{ते वध्यमानाः समरे तव पुत्रा महारथाः}
{भीमं भीमबला युद्धे योधयन्ति जयैषिणः}


\twolineshloka
{ततो दुःशासनः क्रुद्धो रथशक्तिं समाक्षिपत्}
{सर्वपारसवीं तीक्ष्णां जिघांसुः पाण्डुनन्दनम्}


\twolineshloka
{आपतन्तीं महाशक्तिं तव पुत्रप्रणोदिताम्}
{द्विधा चिच्छेद तां भीमस्तदद्भुतमिवाभवत्}


\twolineshloka
{अथान्यैर्विशिखैस्तीक्ष्णैः सङ्क्रुद्धः कुण्डभेदिनम्}
{सुषेणं दीर्घनेत्रं च त्रिभिस्त्रीनवधीद्बली}


\twolineshloka
{ततो वृन्दारकं वीरं कुरूणां कीर्तिवर्धनम्}
{पुत्राणां तव वीराणां युध्यतामवधीत्पुनः}


\twolineshloka
{अभयं रौद्रकर्माणं दुर्विमोचनमेव च}
{त्रिभिस्त्रीनवधीद्भीमः पुनरेव सुतांस्तव}


\twolineshloka
{वध्यमाना महाराज पुत्रास्तव बलीयसा}
{भीमं प्रहरतां श्रेष्ठं समन्तात्पर्यवारयन्}


\twolineshloka
{ते शरैर्भीमकर्माणं ववर्षुः पाण्डवं युधि}
{मेघा इवातपापाये धाराभिर्धरणीधरम्}


\twolineshloka
{स तद्बाणमयं वर्षमश्मवर्षमिवाचलः}
{प्रतीच्छन्पाण्डुदायादो न प्राव्यथत शत्रुहा}


\twolineshloka
{विन्दानुविन्दौ सहितौ सुवर्माणं च ते सुतम्}
{प्रहसन्नेव कौन्तेयः शरैर्निन्यं यमक्षयम्}


\twolineshloka
{ततः सुदर्शनं वीरं पुत्रं ते भरतर्षभ}
{विव्याध समरे तूर्णं स पपात ममार च}


\twolineshloka
{सोऽचिरेणैव कालेन तद्रथानीकमाशुगैः}
{दिशः सर्वाः समालोक्य व्यधमत्पाण्डुनन्दनः}


\threelineshloka
{ततो वै रथघोषेण गर्जितेन मृगा इव}
{भज्यमानाश्च समरे तव पुत्रा विशाम्पते}
{प्राद्रवन्सहसा सर्वे भीमसेनभयर्दिताः}


\twolineshloka
{अनुयायाच्च कौन्तेयः पुत्राणां ते महद्बलम्}
{विव्याध समरे राजन्कौरवेयान्समन्ततः}


\threelineshloka
{वध्यमाना महाराज भीमसेनेन तावकाः}
{त्यक्त्वा भीमं रणाज्जग्मुश्चोदयन्तो हयोत्तमान्}
{}


\twolineshloka
{तांस्तु निर्जित्य समरे भीमसेनो महाबलः}
{सिंहनादरवं चक्रे बाहुशब्दं च पाण्डवः}


\twolineshloka
{*महान्तं तलशब्दं च कृत्वा द्रोणान्तिकं ययौ}
{तमावारयदाचार्यो वेलेवोद्वृत्तमर्णवम्}


\twolineshloka
{तस्य द्रोणो रथं राजञ्छादयामास संयुगे}
{साश्वसूतध्वजं तूर्णं तदद्भुतमिवाभवत्}


\twolineshloka
{ललाटेऽताडयच्चैनं नाराचेन स्मयन्निव}
{ऊर्ध्वरश्मिरिवादित्यो विबभौ तत्र पाण्डवः}


\twolineshloka
{स मन्यमानस्त्वाचार्यो ममायं फल्गुनो यथा}
{भीमः करिष्यते पूजामित्युवाच वृकोदरम्}


\twolineshloka
{भीमसेन न ते शक्यं प्रवेष्टुमरिवाहिनीम्}
{मामनिर्जित्य समरे शत्रुमद्य महाबल}


\threelineshloka
{यदि ते सानुजः कृष्णः प्रविष्टोऽनुमते मम}
{अनीकं न तु शक्यं मे प्रवेष्टुमिह वै त्वया ॥सञ्जय उवाच}
{}


\twolineshloka
{अथ भीमस्तु तच्छ्रुत्वा गुरोर्वाक्यमशेषतः}
{क्रुद्धः प्रोवाच वै द्रोणं रक्तताम्रेक्षणः श्वसन्}


\twolineshloka
{तवार्जुनो नानुमते ब्रह्बन्धो रमाजिरम्}
{प्रविष्टः स हि दुर्धर्षः शक्रस्यापि विशेद्बलम्}


\twolineshloka
{येन त्वं परमां पूजां कुर्वता मानितो ह्यसि}
{नार्जुनोऽहं घृणी विप्र भीमसेनोऽस्मि ते रिपुः}


\twolineshloka
{पिता नस्त्वं गुरुर्बन्धुस्तदा पुत्रा हि र्ते वयम्}
{इति मन्यामहे सर्वे भवन्तं प्रणताः स्थिताः}


\fourlineindentedshloka
{तदद्य विपरीतं ते वदतोऽस्मासु दृश्यते}
{यदि शत्रुं त्वमात्मानं मन्यसे तत्तथाऽस्त्विह}
{एष ते सदृशं कर्म शत्रोर्भीमः करिष्यति ॥सञ्जय उवाच}
{}


\twolineshloka
{अथोद्धाम्य गदां वीरः कालदण्डमिवान्तकः}
{द्रोणायावासृजद्राजन्स रथादवपुप्लुवे}


\twolineshloka
{साश्वसूतध्वजं यानं द्रोणस्यापोथयत्तदा}
{तदद्भुतमपश्याम पाण्डवेयस्य विक्रमम्}


\twolineshloka
{द्रोणं तु विरथं कृत्वा भीमसेनो महाबलः}
{अभ्यवर्तत सैन्यानि तावकानि समन्ततः}


\twolineshloka
{स मृद्रंस्तरसा योधान्वायुर्वृक्षानिवौजसा}
{विचचार रणे राजन्वायुतुल्यपराक्रमः}


\twolineshloka
{निर्जितस्तु तदा तेन पाण्डवेन महात्मना}
{अन्यं तु रथमातिष्ठद्द्रोणः प्रहरतां वरः}


\chapter{अध्यायः १२८}
\twolineshloka
{सञ्जय उवाच}
{}


\twolineshloka
{समुत्तीर्णं रथानीकं पाण्डवं विहसन्रणे}
{विवारयिषुराचार्यः शरवर्षैरवाकिरत्}


\twolineshloka
{पिबन्निव शरौघांस्तान्द्रोणचापपरिच्युतान्}
{सोऽभ्यद्रवत सोदर्यान्मोहयन्बलमायया}


\twolineshloka
{तं मृधे वेगमास्थाय परं परमधन्विनः}
{चोदितास्तव पुत्रैश्च सर्वतः पर्यवारयन्}


\threelineshloka
{स तैस्तु संवृतो भीमः क्रोधेन प्रदहन्निव}
{उद्यच्छन्स गदां तेभ्यः सुघोरां सिंहवन्नदन्}
{अवासृजच्च वेगेन शत्रुपक्षविनाशिनीम्}


\twolineshloka
{इन्द्राशनिरिवेन्द्रेण प्रविद्धा संहतात्मना}
{प्रामथ्नात्सा महाराज सैनिकांस्तव संयुगे}


\twolineshloka
{घोषेण महता राजन्पूरयन्तीव मेदिनीम्}
{ज्वलन्ती तेजसा भीमा त्रासयामास ते सुतान्}


\twolineshloka
{तां पतन्तीं महावेगां दृष्ट्वा तेजोभिसंवृताम्}
{प्राद्रवंस्तावकाः सर्वे नदन्तो बैरवान्रवान्}


\twolineshloka
{सिंहनादमसह्यं हि श्रुत्वा भीमस्य संयुगे}
{प्रापतन्मुनुजास्त्रस्ता रथेभ्यो रथिनस्तदा}


\twolineshloka
{ते हन्यमाना भीमेन गदाहस्तेन तावकाः}
{प्राद्रवन्त रणे भीता व्याघ्रघ्राता मृगा इव}


\twolineshloka
{स तान्विद्राव्य कौन्तेयः सङ्ख्येऽमित्रान्दुरासदान्}
{सुपर्ण इव वेगेन पक्षिराडत्यगाच्चमूम्}


\twolineshloka
{तथा तु विप्रकुर्वाणं रथयूथपयूथपम्}
{भारद्वाजो महाराज भीमसेनं समभ्ययात्}


\twolineshloka
{भीमं तु समरे द्रोणो वारयित्वा शरोर्मिभिः}
{अकरोत्सहसा नादं षाण्डूनां भयमादधत्}


\twolineshloka
{तद्युद्धमासीत्सुमहद्धोरं देवासुरोपमम्}
{द्रोणस्य च महाराज भीमस्य च महात्मनः}


\twolineshloka
{यदा तु विशिखैस्तीक्ष्णैर्द्रोणचापविनिःसृतैः}
{वध्यन्ते समरे वीराः शतशोऽथ सहस्रशः}


\twolineshloka
{ततो रथादवप्लुत्य वेगमास्थाय पाण्डवः}
{निमील्य नयने राजन्पदातिर्द्रोणमभ्ययात्}


\twolineshloka
{अंसे शिरो भीमसेनः करौ कृत्वोरसि स्थिरौ}
{वेगमास्थाय बलवान्मनोनिलगरुत्मताम्}


\twolineshloka
{यथा हि गोवृषो वर्षं प्रतिगृह्णाति लीलया}
{तथा भीमो नरव्याघ्रः शरवर्षं समग्रहीत्}


\twolineshloka
{स वध्यमानः समरे रथं द्रोणस्य मारिष}
{ईषायां पाणिना गृह्य प्रचिक्षेप महाबलः}


\threelineshloka
{द्रोणस्तु सत्वरो राजन्क्षिप्तो भीमेन संयुगे}
{`ददृशे तावकैर्योधैर्विस्मयोत्फुल्ललोचनैः}
{परिहृत्य रथं द्रोणश्चूर्णितं तं महीतले'}


\twolineshloka
{रथमन्यं समारुह्य व्यूहद्वारं ययौ पुनः}
{तमायान्तं तथा दृष्ट्वा भग्नोत्साहं गुरुं तदा}


\twolineshloka
{गत्वा वेगात्पुनर्भीमो धुरं गृह्य रथस्य तु}
{तमप्यतिरथं भीमश्चिक्षेप भृशरोषितः}


% Check verse!
एवमष्टौ रथाः क्षिप्ता भीमसेनेन लीलया
\twolineshloka
{व्यदृश्यत निमेषेण पुनः स्वरथमास्थितः}
{दृश्यते तावकैर्योधैर्विस्मयोत्फुल्ललोचनैः}


\twolineshloka
{तस्मिन्क्षणे तस्य यन्ता तूर्णमश्वानचोदयत्}
{भीमसेनस्य कौरव्य तदद्भुतमिवाभवत्}


\twolineshloka
{ततः स्वरथमास्थाय भीमसेनो महाबलः}
{अभ्यद्रवत वेगेन तव पुत्रस्य वाहिनीम्}


\twolineshloka
{स मृद्गन्क्षत्रियानाजौ वातो वृक्षानिवोद्धतः}
{अगच्छद्दारयन्सेनां सिन्धुवेगो नगानिव}


\twolineshloka
{भोजानीकं समासाद्य हार्दिक्येनाभिरक्षितम्}
{प्रमथ्य तरसा वीरस्तदप्यतिबलोऽभ्ययात्}


\twolineshloka
{सन्त्रासयन्ननीकानि तलशब्देन पाण्डवः}
{अजयत्सर्वसैन्यानि शार्दूल इव गोवृषान्}


\twolineshloka
{भोजानीकमतिक्रम्य दरदानां च वाहिनीम्}
{तथा म्लेच्छगणानन्यान्बहून्युद्धविशारदान्}


\twolineshloka
{सात्यकिं चैव सम्प्रेक्ष्य युध्यमानं महारथम्}
{रथेन यत्तः कौन्तेयो वेगेन प्रययौ तदा}


\twolineshloka
{भीमसेनो महाराज द्रष्टुकामो धनञ्जयम्}
{अतीत्य समरे योधांस्तावकान्पाण्डुनन्दनः}


\twolineshloka
{सोऽपश्यदर्जुनं तत्र युध्यमानं महारथम्}
{सैन्धवस्य वधार्थं हि पराक्रान्तं पराक्रमी}


\twolineshloka
{तं दृष्ट्वा पुरुषव्याघ्रश्चुक्रोश महतो रवान्}
{प्रावृट््काले महाराज नर्दन्निव बलाहकः}


\twolineshloka
{तं तस्य निनदं घोरं पार्थः शुश्राव नर्दतः}
{वासुदेवश्च कौरव्य भीमसेनस्य संयुगे}


\twolineshloka
{तौ श्रुत्वा युगपद्वीरौ निनदं तस्य शुष्मिणः}
{पुनः पुनः प्राणदतां दिदृक्षन्तौ वृकोदरम्}


\twolineshloka
{ततः पार्थो महानादं मुञ्चन्वै माधवश्च ह}
{अभ्ययातां महाराज नर्दन्तौ गोवृषाविव}


\twolineshloka
{भीमसेनरवं श्रुत्वा फल्गुनस्य च धन्विनः}
{अप्रीयत महाराज धर्मपुत्रो युधिष्ठिरः}


\twolineshloka
{विशोकश्चाभवद्राजा श्रुत्वा तं निनदं तयोः}
{धनञ्जयस्य समरे जयमाशास्तवान्विभिः}


\twolineshloka
{तथा तु नर्दमाने वै भीमसेने मदोत्कटे}
{स्मितं कृत्वा महाबाहुर्धर्मपुत्रो युधिष्ठिरः}


\twolineshloka
{हृद्गतं मनसा प्राह ध्यात्वा धर्मभृतां वरः}
{दत्ता भीम त्वया संवित्कृतं गुरुवचस्तथा}


\twolineshloka
{न हि तेषां जयो युद्धे येषां द्वेष्टाऽसि पाण्डव}
{दिष्ट्या जीवति सङ्ग्रमे सव्यसाची धनञ्जयः}


\twolineshloka
{दिष्ट्या च कुशली वीरः सात्यकिः सत्यविक्रमः}
{दिष्ट्या शृणोमि गर्जन्तौ वासुदेवधनञ्जयौ}


\twolineshloka
{येन शक्रं रणे जित्वा तर्पितो हव्यवाहनः}
{स हन्ता द्विषतां सङ्ख्ये दिष्ट्या जीवति फल्गुनः}


\twolineshloka
{यस्य बाहुबलं सर्वे वयमाश्रित्य जीविताः}
{स हन्ता रि पुसैन्यानां दिष्ट्या जीवति फल्गुनः}


\twolineshloka
{निवातकवचा येन देवैरपि सुदुर्जयाः}
{निर्जिता धनुषैकेन दिष्ट्या पार्थः स जीवति}


\twolineshloka
{कौरवान्सहितान्सर्वान्गोग्रहार्थे समागतान्}
{योऽजयन्मत्स्यनगरे दिष्ट्या पार्थः स जीवति}


\twolineshloka
{कालकेयसहस्राणि चतुर्दश महारणे}
{योऽवधीद्भुजवीर्येण दिष्ट्या पार्थः स जीवति}


\twolineshloka
{गन्धर्वराजं बलिनं दुर्योधनकृते च वै}
{जितवान्योऽस्त्रवीर्येण दिष्ट्या पार्थः स जीवति}


\twolineshloka
{किरीटमाली बलवाञ्छेताश्वः कृष्णसारथिः}
{मम प्रियश्च सततं दिष्ट्या पार्थः स जीवति}


\threelineshloka
{पुत्रशोकाभिसन्तप्तश्चिकीर्षन्कर्म दुष्करम्}
{जयद्रथवधान्वेषी प्रतिज्ञां कृतवान्हि यः}
{कच्चित्स सैन्धवं सङ्ख्ये हनिष्यति धनञ्जयः}


\twolineshloka
{कच्चित्तीर्णप्रतिज्ञं हि वासुदेवेन रक्षितम्}
{अनस्तमित आदित्ये समेष्याम्यहमर्जुनम्}


\twolineshloka
{कच्चित्सैन्धवको राजा दुर्योधनहिते रतः}
{नन्दयिष्यत्यमित्रान्हि फल्गुनेन निपातितः}


\twolineshloka
{कच्चिद्दुर्योधनो राजा फल्गुनेन निपातितम्}
{दृष्ट्वा सैन्धवकं सङ्ख्ये शममस्मासु धास्यति}


\twolineshloka
{दृष्ट्वा विनिहतान्भ्रातॄन्भीमसेनेन संयुगे}
{कच्चिद्दुर्योधनो मन्दः शममस्मासु यास्यति}


\twolineshloka
{दृष्ट्वा चान्यान्महायोधान्पातितान्धरणीतले}
{कच्चिद्दुर्योधनो मन्दः पश्चात्तापं गमिष्यति}


\twolineshloka
{शेषस्य रक्षणार्थं च सन्धास्यति सुयोधनः ॥एवं बहुविधं तस्य राज्ञश्चिन्तयतस्तदा}
{}


% Check verse!
कृपयाऽभिपरीतस्य घोरं युद्धमवर्तत
\chapter{अध्यायः १२९}
\twolineshloka
{धृतराष्ट्र उवाच}
{}


\twolineshloka
{निनदन्तं तथा तं तु भीमसेनं महाबलम्}
{मेघस्तनितनिर्घोषं के वीराः पर्यवारयन्}


\twolineshloka
{न हि पश्याम्यहं तं वै त्रिषु लोकेषु कञ्चन}
{क्रुद्धस्य भीमसेनस्य यस्तिष्ठेदग्रतो रणे}


\twolineshloka
{गदां युयुत्समानस्य कालस्येवेह सञ्चय}
{न हि पश्याम्यहं युद्धे यस्तिष्ठेदग्रतः पुमान्}


\twolineshloka
{रथं रथेन यो हन्यात्कुञ्जरं कुञ्चरेण च}
{कस्तस्य समरे स्थाता साक्षादपि पुरन्दरः}


\twolineshloka
{क्रुद्धस्य भीमसेनस्य मम पुत्राञ्जिघांसतः}
{दुर्योधनहिते युक्ताः समतिष्ठन्त केऽग्रतः}


\twolineshloka
{भीमसेनदवाग्नेस्तु मम पुत्रांस्तृणोलपम्}
{प्रधक्ष्यतो रणमुखे केऽतिष्ठन्नग्रतो नराः}


\twolineshloka
{काल्यमानांस्तु पुत्रान्मे दृष्ट्वा भीमस्य संयुगे}
{कालेनेव प्रजाः सर्वाः के भीमं पर्यवारयन्}


\twolineshloka
{न मेऽर्जुनाद्भयं तादृक्कृष्णान्नापि च सात्वतात्}
{हुतभुग्जन्मनो नैव यादृग्भीमाद्भयं मम}


\threelineshloka
{भीमवह्नेः प्रदीप्तस्य मम पुत्रान्दिधक्षतः}
{के शूराः पर्यवर्तन्त तन्ममाचक्ष्व सञ्जय ॥सञ्चय उवाच}
{}


\twolineshloka
{तथा तु नर्दमानं तं भीमसेनं महाबलम्}
{तुमुलेनैव शब्देन कर्णोऽप्यभ्यद्रवद्बली}


\twolineshloka
{व्याक्षिपन्सुमहच्चापमतिमात्रममर्षणः}
{कर्णः सुयुद्धमाकाङ्क्षन्दर्शयिष्यन्बलं मृधे}


\twolineshloka
{रुरोध मार्गं भीमस्य वातस्येव महीरुहः}
{भीमोऽपि दृष्ट्वा सावेगं पुरो वैकर्तनं स्थितम्}


\twolineshloka
{चुकोप बलवद्वीरश्चिक्षेपास्य शिलाशितान्}
{तान्प्रत्यगृह्णात्कर्णोऽपि प्रतीपं प्रापयच्छरान्}


\twolineshloka
{ततस्तु सर्वयोधानां यतनां प्रेक्षतां तदा}
{प्रावेपन्निव गात्राणि कर्णभीमसमागमे}


\twolineshloka
{रथिनां सादिनां चैव तयोः श्रुत्वा तलस्वनम्}
{भीमसेनस्य निनदं श्रुत्वा घोरं रणाजिरे}


\twolineshloka
{खं च भूमिं च संरुद्धां मेनिरे क्षत्रियर्षभाः}
{पुनर्घोरेण नादेन पाण्डवस्य महात्मनः}


\twolineshloka
{समरे सर्वयोधानां धनूंष्यभ्यपतन्क्षितौ}
{शस्त्राणि न्यपतन्दोर्भ्यः केषाञ्चिच्चासवोऽद्रवन्}


\twolineshloka
{वित्रस्तानि च सर्वाणि शकृन्मूत्रं प्रसुस्रुवुः}
{वाहनानि च सर्वाणि बभूवुर्विमनांसि च}


\twolineshloka
{प्रादुरासन्निमित्तानि घोराणि सुबहून्युत}
{गृध्रकङ्कबलैश्चासीदन्तरिक्षं समावृतम्}


\twolineshloka
{तस्मिन्सुतुमुले राजन्कर्णभीमसमागमे}
{ततः कर्णस्तु विंशत्या शराणां भीममार्दयत्}


\twolineshloka
{विव्याध चास्य त्वरितः सूतं पञ्चभिराशुगैः}
{प्रहस्य भीमसेनोऽपि कर्णं प्रत्याद्रवद्रणे}


\twolineshloka
{सायकानां चतुः षष्ट्या क्षिप्रकारी महायशाः}
{तस्य कर्णो महेष्वासः सायकांश्चतुरोऽक्षिपत्}


\twolineshloka
{असम्प्राप्तांश्च तान्भीमः सायकैर्नतपर्वभिः}
{चिच्छेद बहुधा राजन्दर्शयन्पाणिलाघवम्}


\twolineshloka
{तं कर्णश्चादयामास शरव्रातैरनेकशः}
{सञ्छाद्यमानः कर्णेन बहुधा पाण़्डुनन्दनः}


\twolineshloka
{चिच्छेद चापं कर्णस्य मुष्टिदेशे महारथः}
{विव्याध चैनं बहुभिः सायकैर्नतपर्वभिः}


\twolineshloka
{अथान्यद्धनुरादाय सञ्यं कृत्वा च सूतजः}
{विव्याध समरे भीमं भीमकर्मा महारथः}


\twolineshloka
{तस्य बीमो भृशं क्रुद्धस्त्रीञ्शरान्नतपर्वणः}
{निचखानोरसि क्रुद्धः सूतपुत्रस्य वेगतः}


\twolineshloka
{तैः कर्णोऽराजत शरैरुरोमध्यगतैः सदा}
{महीधर इवोदग्रस्त्रिशृङ्गो भरतर्षभ}


\twolineshloka
{सुस्राव चास्य रुधिरं विद्धस्य परमेषुभिः}
{धातुप्रस्यन्दिनः शैलाद्यथा गैरिकधातवः}


\twolineshloka
{किञ्चिद्विचलितः कर्णः सुप्रहाराभिपीडितः}
{आकर्णपूर्णमाकृष्य भीमं विव्याध सायकैः}


\threelineshloka
{चिक्षेप च पुनर्बाणाञ्शतशोऽथ सहस्रशः}
{स शरैरर्दितस्तेन कर्णेन दृढधन्विना}
{धनुर्ज्यामच्छिनत्तूर्णं भीमस्तस्य क्षुरेण ह}


\twolineshloka
{सारथिं चास्य भल्लेन रथनीडादपातयत्}
{वाहांश्च चतुरस्तस्य व्यसूंश्चक्रे महारथः}


\twolineshloka
{हताश्वात्तु रथात्कर्णः समाप्लुत्य विशाम्पते}
{स्यन्दनं वृषसेनस्य तूर्णमापुप्लुवे भयात्}


\twolineshloka
{निर्जित्य तु रणे कर्णं भीमसेनः प्रतापवान्}
{ननाद बलवान्नादं पर्जन्यनिनदोपमम्}


\twolineshloka
{तस्य तं निनदं श्रुत्वा प्रहृष्टोऽभूद्युधिष्ठिरः}
{कर्णं पराजितं मत्वा भीमसेनेन संयुगे}


\twolineshloka
{समन्ताच्छङ्खनिनदं पाण्डुसेनाऽकरोत्तदा}
{शत्रुसेनाध्वनिं श्रुत्वा तावका ह्यनदन्भृशम्}


\twolineshloka
{स शङ्खबाणनिनदैर्हर्षाद्राजा स्ववाहिनीम्}
{चक्रे युधिष्ठिरः सङ्ख्ये हर्षनादैश्च सङ्कुलाम् ॥गाण्डीवं व्याक्षिपत्पार्थः कृष्णोऽप्यब्जमवादयत्}


\twolineshloka
{तमन्तर्धाय निनदं भीमस्य नदतो ध्वनिः}
{अश्रूयत तदा राजन्सर्वसैन्येषु दारुणः}


\twolineshloka
{ततो व्यायच्छतामस्त्रैः पृथक्पृथगजिह्मगैः}
{मृदुपूर्वं तु राधेयो दृढपूर्वं तु पाण्डवः}


\twolineshloka
{`भीमोऽपि च महाराज वैकर्तनमुपाद्रवत्}
{आसुरे तु महासैन्ये तारकं पावकिर्यथा}


\twolineshloka
{तयोरेवं महद्युद्धमभवद्भीमकर्णयोः}
{तं भीमसेनो महता शरवर्षेण वारयन्}


\threelineshloka
{विव्याध सारथिं चास्य हयांश्च चतुरः शरैः}
{ध्वजं चास्य पताकां च भल्लैः सन्नतपर्वभिः}
{रथं च चक्ररक्षौ च भीमश्चिच्छेद मारिष}


\twolineshloka
{कर्णोऽपि रथिनां श्रेष्ठो भीमसेनेन कम्पितः}
{खङ्गचर्मधरो राजन्भीममभ्यद्रवद्बली}


% Check verse!
भीमश्चिच्छेद खङ्गं च चर्मणा सह मारिष
\fourlineindentedshloka
{दृष्ट्वा कर्णं च पार्थेन बाधितं बहुभिः शरैः}
{दुर्योधनो महाराज दुश्शलं प्रत्यभाषत}
{कर्णं कृच्छ्रगतं पश्य शीघ्रं यानं प्रयच्छह ॥सञ्जय उवाच}
{}


\twolineshloka
{एवमुक्तस्ततो राज्ञा दुश्शलः समुपाद्रवत्}
{दुश्शलस्य रथं कर्णश्चारुरोह महारथः}


\twolineshloka
{तौ पार्थः सहसा गत्वा विव्याध दशभिः शरैः}
{पुनश्च कर्णं विद्ध्वाऽपि दुश्शलस्य शिरोऽहरत्}


\twolineshloka
{दुश्शलं निहतं दृष्ट्वा भीमसेनेन मारिष}
{तस्यैव धनुरादाय कर्णो विव्याध पाण्डवम्}


\twolineshloka
{अन्योन्यं समरे वीरौ युयुधाते महाबलौ}
{शत्रुघ्नौ शत्रुमध्ये तु बलवज्रभृताविव}


\twolineshloka
{भीमो विद्धा हयांश्चैव सारथिं च पुनः पुनः}
{कर्णमभ्यद्रवत्पार्थः प्रहसंश्च महाबलः}


\twolineshloka
{ततो व्यायच्छमानस्य भीमसेनस्य संयुगे}
{तत्सैन्यं कलिलीभूतं न प्राज्ञायत किञ्चन}


\chapter{अध्यायः १३०}
\twolineshloka
{सञ्जय उवाच}
{}


\threelineshloka
{तस्मिन्विलुलिते सैन्ये सैन्धवायार्जुने गते}
{सात्वते भीमसेने च पुत्रस्ते द्रोणमभ्ययात्}
{त्वरन्नेकरथेनैव बहु कृत्यं विचिन्तयन्}


\twolineshloka
{स रथस्तव पुत्रस्य त्वरया परया युतः}
{तूर्णमभ्यद्रवद्द्रोणं मनोमारुतवेगवान्}


\threelineshloka
{`इह दृष्ट इतो नष्टः सरथः प्राद्रवन्नृपः}
{मुहूर्तादिव पुत्रस्ते द्रोणमासाद्य मारिष'}
{ससम्भ्रमामिदं वाक्यमब्रवीत्कुरनन्दनः}


\twolineshloka
{अर्जुनो भीमसेनश्च सात्यकिश्चापराजितः}
{विजित्य सर्वसैन्यानि सुमहान्ति महारथाः}


\twolineshloka
{सम्प्राप्ताः सिन्धुराजस्य समीपमनिवारिताः}
{व्यायच्छन्ति च तत्रापि सर्व एवापराजिताः}


\twolineshloka
{यदि तावद्रणे पार्थो व्यतिक्रान्तो महारथः}
{कथं सात्यकिभीमाभ्यां व्यतिक्रान्तोऽसि मानद}


\threelineshloka
{आश्चर्यभूतो लोकेऽस्मिन्समुद्रस्येव शोषणम्}
{निर्जयस्तव विप्राग्र्य सात्वतेनार्जुनेन च}
{तथैव भीमसेनेन लोकः संवदते भृशम्}


\twolineshloka
{कथं द्रोणो जितः सङ्ख्ये धनुर्वेदस्य पारगः}
{इत्येवं ब्रुवते योधा अश्रद्धेयमिदं तव}


\threelineshloka
{नाश एव तु मे नूनं मन्दभाग्यस्य संयुगे}
{यत्र त्वां पुरुषव्याघ्रं व्यतिक्रान्तास्त्रयो रथाः ॥द्रोण उवाच}
{}


\threelineshloka
{एवं गते तु कृत्येऽस्मिन्ब्रूहि यत्ते विवक्षितम्}
{यद्गतं गतमेवेदं शेषं चिन्तय मानद ॥दुर्योधन उवाच}
{}


\threelineshloka
{यत्कृत्यं सिन्धुराजस्य प्राप्तकालमनन्तरम्}
{तत्संविधीयतां क्षिप्रं साधु सञ्चिन्त्य नो द्विज ॥द्रोण उवाच}
{}


\twolineshloka
{चिन्त्यं बहुविधं तात यत्कृत्यं तच्छृणुष्व मे}
{त्रयो हि समतिक्रान्ताः पाण्डवानां महारथाः}


\twolineshloka
{यावत्तेषां भयं पश्चात्तावदेषां पुरःसरम्}
{तद्गरीयस्तरं मन्ये यत्र कृष्णधनञ्जयौ}


\twolineshloka
{सा पुरस्ताच्च पश्चाच्च गृहीता भारती चमूः}
{तत्र कृत्यमहं मन्ये सैन्धवस्याभिरक्षणम्}


\twolineshloka
{स नो रक्ष्यतमस्तात क्रुद्धाद्भीतो धनञ्जयात्}
{गतौ च सैन्धवं भीमौ युयुधानवृकोदरौ}


\threelineshloka
{सम्प्राप्तं तदिदं द्यूतं यत्तच्छकुनिबुद्धिजम्}
{न सभायां जयो वृत्तो नापि तत्र पराजयः}
{इह नो ग्लहमानानामद्य तावज्जयाजयौ}


\twolineshloka
{यान्स्म तान्ग्लहते घोराञ्छकुनिः कुरुसंसदि}
{अक्षान्स मन्यमानः प्राक्शरास्ते हि दुरासदाः}


\threelineshloka
{यत्र ते बहवस्तात कौरवेया व्यवस्थिताः}
{सेन्नां दुरोदरं विद्धि शरानक्षान्विशाम्पते}
{ग्लहं च सैन्धवं राजंस्तत्र द्यूतस्य निश्चयः}


\twolineshloka
{सैन्धवे तु महद्द्यूंत समासक्तं परैः सह}
{अत्र ते ध्रुवमायत्तो जयो वाऽजय एव वा}


\threelineshloka
{अत्र सर्वे महाराज त्यक्त्वा जीवितमात्मनः}
{सैन्धवस्य रणे रक्षां विधिवत्कर्तुमर्हथ}
{तत्र नो ग्लहमानानां ध्रुवौ जयपराजयौ}


\twolineshloka
{यत्र ते परमेष्वासा यत्ता रक्षन्ति सैन्धवम्}
{तत्र गच्छ स्वयं शीघ्रं तांश्च रक्षस्व रक्षिणः}


\threelineshloka
{इहैव त्वहमासिष्ये प्रेषयिष्यामि चापरान्}
{निरोत्स्यामि च पाञ्चालान्सहितान्पाण्डुसृञ्जयैः ॥सञ्जय उवाच}
{}


\twolineshloka
{ततो दुर्योधनोऽगच्छत्तूर्णमाचार्यशासनात्}
{उद्यम्यात्मानमुग्राय कर्मणे स पदानुगः}


\twolineshloka
{चक्ररक्षौ तु पाञ्चाल्यौ युधामन्यूत्तमौजसौ}
{बाह्येन सेनामभ्येत्य जग्मतुः सव्यसाचिनम्}


\twolineshloka
{यौ तु पूर्वं महाराज वारितौ कृतवर्मणा}
{प्रविष्टे त्वर्जुने राजंस्तव सैन्यं युयुत्सया}


\twolineshloka
{पार्श्वे भित्त्वा चमूं वीरौ प्रविष्टौ तव वाहिनीम्}
{पार्श्वेन सैन्यमायान्तौ कुरुराजो ददर्श ह}


\twolineshloka
{ताभ्यां दुर्योधनः सार्धमकरोत्सङ्ख्यमुत्तमम्}
{त्वरितस्त्वरमाणाभ्यां भ्रातृभ्यां भारतो बली}


\twolineshloka
{तावेनमभ्यद्रवतामुभावुद्यतकार्मुकौ}
{महारथसमाख्यातौ क्षत्रियप्रवरौ युधि}


\twolineshloka
{तमविध्यद्युधामन्युस्त्रिंशता कङ्कपत्रिभिः}
{विंशत्या सारथिं चास्य चतुर्भिश्चतुरो हयान्}


\twolineshloka
{दुर्योधनो युधामन्योर्ध्वजमेकेषुणाऽच्छिनत्}
{एकेन कार्मुकं चास्य चकर्त तनयस्तव}


\twolineshloka
{सारथिं चास्य भल्लेन रथनीडादपाहरत्}
{ततोऽविध्यच्छरैस्तीक्ष्णैश्चतुर्भिश्चतुरो हयान्}


\twolineshloka
{युधामन्युश्च सङ्क्रुद्धः सरांस्त्रिंशतमाहवे}
{व्यसृजत्तव पुत्रस्य त्वरमाणः स्तनान्तरे}


\twolineshloka
{तथोत्तमौजाः सङ्क्रुद्धः शरैर्हेमविभूषितैः}
{अविध्यत्सारथिं चास्य प्राहिणोद्यमसादनम्}


\twolineshloka
{दुर्योधनोऽपि राजेन्द्र पाञ्चाल्यस्यौत्तमौजसः}
{जघान चतुरोऽस्याश्चानुभौ तौ पार्ष्णिसारथी}


\twolineshloka
{उत्तमौजा हताश्वस्तु हतसूतश्च संयुगे}
{आरुरोह रथं भ्रातुर्युधामन्योरभित्वरन्}


\twolineshloka
{स रथं प्राप्य तं भ्रातुर्दुर्योधनहयाञ्शरैः}
{बहुभिस्ताडयामास ते हताः प्रापतन्भुवि}


\threelineshloka
{हयेषु पतितेष्वस्य चिच्छेद परमेषुणा}
{युधामन्युर्धनुः शीघ्रं शरावापं च संयुगे}
{`अविध्यत्सारथिं चास्य प्राहिणोद्यमसादनम्'}


\twolineshloka
{हताश्वसूतात्सरथादवतीर्य नराधिपः}
{गदामादाय ते पुत्रः पाञ्चल्यावभ्यधावत}


\twolineshloka
{तमापतन्तं सम्प्रेक्ष्य क्रुद्धं कुरुपतिं तदा}
{अवप्लुतौ रथोपस्थाद्युधामन्यूत्तमौजसौ}


\twolineshloka
{ततः स हेमचित्रं तं गदया स्यन्दनं गती}
{सङ्क्रुद्धः पोथयामास साश्वसूतध्वजं नृप}


\twolineshloka
{`एतदीदृशकं पुत्रस्तवाकार्षीज्जनाधिप}
{गदया गदिनां श्रेष्ठः सर्वलोकमहारथः'}


\twolineshloka
{भङ्क्त्वा रथं स पुत्रस्ते हताश्वो हतसारथिः}
{मद्रराजरथं तूर्णमारुरोह परन्तपः}


\twolineshloka
{पाञ्चालानां ततो मुख्यौ राजपुत्रौ महारथौ}
{रथावन्यौ समारुह्य बीभत्सुमभिजग्मुतः}


\chapter{अध्यायः १३१}
\twolineshloka
{सञ्जय उवाच}
{}


\twolineshloka
{वर्तमाने महाराज सङ्ग्रामे रोमहर्षणे}
{व्याकुलेषु च सर्वेषु पीड्यमानेषु सर्वशः}


\twolineshloka
{राधेयो भीममानर्च्छ युद्धाय भरतर्षभ}
{यथा नागो वने नागं मत्तो मत्तमभिद्रवेत्}


\threelineshloka
{`ततो व्यायच्छतामस्त्रैः पृथक्पृथगरिन्दमौ}
{मृदुपूर्वं च राधेयो दृढपूर्वं च पाण्डवः' ॥धृतराष्ट्र उवाच}
{}


\twolineshloka
{यौ तौ कर्णश्च भीमश्च सम्प्रयुद्धौ महाबलौ}
{अर्जुनस्य रथोपान्ते कीदृशः सोऽभवद्रणः}


\twolineshloka
{पूर्वं हि निर्जितः कर्णो भीमसेनेन संयुगे}
{कथं भूयः स राधेयो भीममागान्महारथः}


\twolineshloka
{भीमो वा सूततनयं प्रत्युद्यातः कथं रणे}
{महारथसमाख्यातं पृथिव्यां प्रवरं रथम्}


\twolineshloka
{भीष्मद्रोणावतिक्रम्य धर्मराजो युधिष्ठिरः}
{नान्यतो भयमादत्त विना कर्णान्महारथात्}


\threelineshloka
{भयाद्यस्य महाबाहो न शेते बहुलाः समाः}
{चिन्तयन्नित्यशो वीर्यं राधेयस्य महात्मनः}
{तं कथं सूतपुत्रं तु भीमोऽयोधयताहवे}


\twolineshloka
{ब्रह्मण्यं वीर्यसम्पन्नं समरेष्वनिवर्तिनम्}
{कथं कर्णं युधां श्रेष्ठं योधयामास पाण्डवः}


\twolineshloka
{यौ तौ समीयतुर्वीरौ वैकर्तनवृकोदरौ}
{कथं तावत्र युध्येतां महाबलपराक्रमौ}


\twolineshloka
{भ्रातृत्वं दर्शितं पूर्वं घृणी चापि स सूतजः}
{कथं भीमेन युयुधे कुन्त्या वाक्यमनुस्मरन्}


\twolineshloka
{भीमो वा सूतपुत्रेण स्मरन्वैरं पुरा कृतम्}
{अयुध्यत कथं शूरः कर्णेन सह संयुगे}


\twolineshloka
{आशास्ते च सदा सूतपुत्रे दुर्योधनो मम}
{कर्णो जेष्यति सङ्ग्रामे समस्तान्पाण्डवानिति}


\twolineshloka
{जयाशा यत्र पुत्रस्य मम मन्दस्य संयुगे}
{स कथं भीमकर्माणं भीमसेनमयोधयत्}


\twolineshloka
{यं समासाद्य पुत्रैर्मे कृतं वैरं महारथैः}
{तं सूततनयं तात कथं भीमो ह्ययोधयत्}


\twolineshloka
{अनेकान्विप्रकारांश्च सूतपुत्रसमुद्भवान्}
{स्मरमाणः कथं भीमो युयुधे सूतसूनुना}


\twolineshloka
{योऽजयत्पृथिवीं सर्वां रथेनैकेन वीर्यवान्}
{तं सूततनयं युद्धे कथं भीमो ह्ययोधयत्}


\twolineshloka
{यो जातः कुण्डलाभ्यां च कवचेन सहैव च}
{तं सूतपुत्रं समरे भीमः कथमयोधयत्}


\twolineshloka
{`अस्त्रहेतोः पुरा तात भार्गवं समपूजयत्}
{तस्य प्रसादाद्ब्रह्मास्त्रं लब्धवांश्च भृगूत्तमात्'}


\threelineshloka
{यथा तयोर्युद्धमभूद्यश्चासीद्विजयी तयोः}
{तन्ममाचक्ष्व तत्त्वेन कुशलो ह्यसि सञ्जय ॥सञ्चय उवाच}
{}


\twolineshloka
{भीमसेनस्तु राधेयमुत्सृज्य रथिनां वरम्}
{इयेष गन्तुं यत्रास्तां वीरौ कृष्णधनञ्चयौ}


\twolineshloka
{तं प्रयान्तमभिद्रुत्य राधेयः कङ्कपत्रिभिः}
{अभ्यवर्षन्महाराज मेघो वृष्ट्येव पर्वतम्}


\twolineshloka
{फुल्लता पङ्कजेनेव वक्त्रेण विहसन्बली}
{आजुहाव रणे यान्तं भीममाधिरथिस्तदा}


\threelineshloka
{कर्ण उवाच}
{भीमाहितैस्तव रणः स्वप्नेऽपि न विभावितः}
{}


\twolineshloka
{तद्दर्शयसि कस्मान्मे पृष्ठं पार्थदिदृक्षया ॥कुन्त्याः पुत्रस्य सदृशं नेदं पाण्डवनन्दन}
{}


\threelineshloka
{तेन मामभितः स्थित्वा शरवर्षैरवाकिर ॥सञ्जय उवाच}
{भीमसेनस्तदाह्वानं कर्णान्नामर्षयद्युधि}
{अर्धमण्डलमावृत्य सूतपुत्रमयोधयत्}


\twolineshloka
{अवक्रगामिभिर्बाणैरभ्यवर्षन्महायशाः}
{दंशितं द्वैरथे यत्तं सर्वशस्त्रविशारदम्}


% Check verse!
विधित्सुः कलहस्यान्तं जिघांसुः कर्णमक्षिणोत्
\threelineshloka
{हत्वा तस्यानुगांस्तं च हन्तुकामो महाबलः}
{तस्मै व्यसृजदुग्राणि विविधानि परन्तपः}
{अमर्षात्पाण्डवः क्रुद्धः शरवर्षाणि मारिष}


\twolineshloka
{तस्य तानीषुवर्षाणि मत्तद्विरदगामिनः}
{सूतपुत्रोऽस्त्रमायाभिरग्रसत्परमास्त्रवित्}


\twolineshloka
{स यथावन्महाबाहुर्विद्यया वै सुपूजितः}
{आचार्यवन्महेष्वासः कर्णः पर्यचरद्बली}


\twolineshloka
{युध्यमानं तु संरम्भाद्भीमसेनं हसन्निव}
{अभ्यपद्यत कौन्तेयं कर्णो राजन्वृकोदरम्}


\twolineshloka
{तन्नामृष्यत कौन्तेयः कर्णस्य स्मितमाहवे}
{युध्यमानेषु वीरेषु पश्यत्सु च समन्ततः}


\twolineshloka
{तं भीमसेनः सङ्ग्राप्तं वत्सदन्तैः स्तनान्तरे}
{विव्याध बलवान्क्रुद्धस्तोत्रैरिव महाद्विपम्}


\twolineshloka
{पुनश्च सूतपुत्रं तु स्वर्णपुङ्खैः शिलाशितैः}
{सुमुक्तैश्चित्रवर्माणं निर्बिभेद त्रिसप्तभिः}


\twolineshloka
{कर्णो जाम्बूनदैर्जालैः सञ्छन्नान्वातरंहसः}
{हयान्विव्याध भीमस्य पञ्चभिः पञ्चभिः शरैः}


\twolineshloka
{ततो बाणमयं जलं भीमसेनरथं प्रति}
{कर्णेन विहितं राजन्निमेषार्धाददृश्यत}


\twolineshloka
{सरथः सध्वजस्तत्र ससूतः पाण्डवस्तदा}
{प्राच्छाद्यत महाराज कर्णचापच्युतैः शरैः}


\twolineshloka
{तस्य कर्णश्चतुःषष्ट्या व्यधमत्कवचं दृढम्}
{क्रुद्धश्चाप्यहनत्पार्थं नाराचैर्मर्मभेदिभिः}


\twolineshloka
{ततोऽचिन्त्य महाबाहुः कर्णकार्मुकनिःसृतान्}
{समाश्लिष्यदसम्भ्रान्तः सूतपुत्रं वृकोदरः}


\twolineshloka
{स कर्णचापप्रभवानिषूनाशीविषोपमान्}
{बिभ्रद्भीमो महाराज न जगाम व्यथां रणे}


\twolineshloka
{ततो द्वात्रिंशता भल्लैर्निशितैस्तिग्मतेजनैः}
{विव्याध समरे कर्णं भीमसेनः प्रतापवान्}


\twolineshloka
{अयत्नेनैव तं कर्णः शरैर्भृशमवाकिरत्}
{भीमसेनं महाबाहुं सैन्धवस्य वधैषिणम्}


\twolineshloka
{मृदुपूर्वं तु राधेयो भीममाजावयोधयत्}
{क्रोधपूर्वं तथा भीमः पूर्ववैरमनुस्मरन्}


\twolineshloka
{तं भीमसेनो नामृष्यदवमानममर्षणः}
{स तस्मै व्यसृजत्तूर्णं शरवर्षममित्रहा}


\twolineshloka
{ते शराः प्रेषितास्तेन भीमसेनेन संयुगे}
{निपेतुः सर्वतो वीरे कूजन्त इव पक्षिणः}


\twolineshloka
{हेमपुङ्खा महावेगा भीमसेनधनुश्च्युताः}
{प्राच्छादयंस्ते राधेयं शलभा इव पावकम्}


\twolineshloka
{कर्णस्तु रथिनां श्रेष्ठश्छाद्यमानः समन्ततः}
{राजन्व्यसृजदुग्राणि शरवर्षाणि भारत}


\twolineshloka
{तस्य तानशनिप्रख्यानिषून्समरशोभिनः}
{चिच्छेद बहुभिर्भल्लैरसम्प्राप्तान्वृकोदरः}


\twolineshloka
{पुनश्च शरवर्षेण च्छादयामास भारत}
{कर्णो वैकर्तनो युद्धे भीमसेनमरिन्दमः}


\twolineshloka
{तत्र भारत भीमं तु दृष्टवन्तः स्म सायकैः}
{समाचिततनुं सङ््ख्ये श्वाविधं शललैरिव}


\twolineshloka
{हेमपुङ्खाञ्छिलाधौतान्कर्णचापच्युताञ्छरान्}
{दधार समरे वीरः स्वरश्मीनिव रश्मिमान्'}


\twolineshloka
{रुधिरोक्षितसर्वाङ्गो भीमसेनो व्यराजत}
{समृद्धकुसुमापीडो वसन्तेऽशोकवृक्षवत्}


\twolineshloka
{तत्तु भीमो महाबाहोः कर्णस्य चरितं रणे}
{नामृष्यत महाबाहुः क्रोधादुद्वृत्तलोचनः}


\twolineshloka
{स कर्णं पञ्चविंशत्या नाराचानां समार्पयत्}
{महीधरमिव श्वेतं गूढपादैर्विषोल्बणैः}


\twolineshloka
{पुनरेव च विव्याध षङ्भिरष्टाभिरेव च}
{मर्मस्वमरविक्रान्तः सूतपुत्रं तनुत्यजम्}


\twolineshloka
{पुनरन्येन बाणेन भीमसेनः प्रतापवान्}
{चिच्छेद कार्मुकं तूर्णं कर्णस्य प्रहसन्निव}


\twolineshloka
{जघान चतुरश्चाश्वान्सूतं च त्वरितः शरैः}
{नाराचैरर्करश्म्याभैः कर्णं विव्याध चोरसि}


\twolineshloka
{ते जग्मुर्धरणीमाशु कर्णं निर्भिद्य पत्रिणः}
{यथा जलधरं भित्त्वा दिवाकरमरीचयः}


\threelineshloka
{स वैक्लव्यं महत्प्राप्य च्छिन्नधन्वा शराहतः}
{तथा पुरुषमानी स लज्जामुत्सृज्य भारत}
{भीमसेनभयात्कर्णः प्रत्यपायाद्रथान्तरम्}


\chapter{अध्यायः १३२}
\twolineshloka
{धृतराष्ट्र उवाच}
{}


\twolineshloka
{[स्वयं शिष्यो महेशस्य भृगूत्तमधनुर्धरः}
{शिष्यत्वं प्राप्तवान्कर्णस्तस्य तुल्योऽस्त्रविद्यया}


\twolineshloka
{तद्विशिष्टोऽपि वा कर्णः शिष्यः शिष्यगुणैर्युतः}
{कुन्तीपुत्रेण भीमेन निर्जितः स तु लीलया ॥]}


\twolineshloka
{यस्मिञ्जयाशा महती पुत्राणां मम सञ्जय}
{तं भीमाद्विमुखं दृष्ट्वा किन्नु दुर्योधनोऽब्रवीत्}


% Check verse!
कथं च युयुधे भीमो वीर्यश्लाघी महाबलः
\threelineshloka
{कर्णो वा समरे तात किमकार्षीदतः परम्}
{भीमसेनं रणे दृष्ट्वा ज्वलन्तमिव पावकम् ॥सञ्जय उवाच}
{}


\twolineshloka
{रथमन्यं समास्थाय विधिवत्कल्पितं पुनः}
{अभ्ययात्पाण्डवं कर्णो वातोद्धूत इवार्णवः}


\twolineshloka
{क्रुद्धमाधिरथिं दृष्ट्वा पुत्रास्तव विशाम्पते}
{भीमसेनममन्यन्त वैश्वानरमुखे हुतम्}


\twolineshloka
{चापशब्दं ततः कृत्वा तलशब्दं च भैरवम्}
{अभ्यद्रवत राधेयो भीमसेनरथं प्रति}


\twolineshloka
{पुनरेव तयो राजन्घोर आसीत्समागमः}
{वैकर्तनस्य शूरस्य भीमस्य च महात्मनः}


\twolineshloka
{संरब्धौ हि महाबाहू परस्परवधैषिणौ}
{अन्योन्यभीक्षाञ्चक्राते दहन्ताविव लोचनैः}


\twolineshloka
{क्रोधरक्रेक्षणौ तीव्रौ निःश्वसन्ताविवोरगौ}
{शूरावन्योन्यमासाद्य ततक्षतुररिन्दमौ}


\twolineshloka
{व्याघ्राविव सुसंरब्धौ श्येनाविव च शीघ्रगौ}
{शरभाविव सङ्क्रुद्धौ युयुधाते परस्परम्}


\twolineshloka
{ततो भीमः स्मरन्क्लेशानक्षद्यूते वनेऽपि च}
{विराटनगरे चैव दुःखं प्राप्तमरिन्दमः}


\twolineshloka
{राष्ट्राणां स्फीतरत्नानां हरणं च तवात्मजैः}
{सततं च परिक्लेशान्सपुत्रेण त्वया कृतान्}


\twolineshloka
{दग्धुमैच्छश्च यः कुन्तीं सपुत्रां त्वमनागसम्}
{कृष्णायाश्च परिक्लेशं सभामध्ये दुरात्मभिः}


\twolineshloka
{केशपक्षग्रहं चैव दुःशासनकृतं तथा}
{परुषाणि च वाक्यानि कर्णोनोक्तानि भारत}


\twolineshloka
{पतिमन्यं परीप्सस्व न सन्ति पतयस्तव}
{पतिता नरके पार्थाः सर्वे षण्डतिलोपमाः}


\twolineshloka
{समक्षं तव कौरव्य यदूचुः कौरवास्तदा}
{दासीभावेन कृष्णां च भोक्तुकामाः सुतास्तव}


\twolineshloka
{यच्चापि तान्प्रव्रजतः कृष्णाजिननिवासिनः}
{परुषाण्युक्तवान्कर्णः सभायां सन्निधौ तव}


\twolineshloka
{तृणीकृत्य च यत्पार्थांस्तव पुत्रो ववल्ग ह}
{विषमस्थान्समस्थो हि संरब्धो गतचेतनः}


\twolineshloka
{बाल्यात्प्रभृति चारिघ्नः स्वानि दुःखानि चिन्तयन्}
{निरविद्यत धर्मात्मा जीवितेन वृकोदरः}


\twolineshloka
{ततो विष्फार्य सुमहद्धेमपृष्ठं दुरासदम्}
{चापं भरतशार्दूलस्त्यक्तात्मा कर्णमभ्ययात्}


\twolineshloka
{स सायकयैर्जालैर्भीमः कर्णरथं प्रति}
{भानुमद्भिः शिलाधौतैर्भानोः प्राच्छादयत्प्रभां}


\twolineshloka
{ततः प्रहस्याधिरथिस्तूर्णमस्यञ्शिताञ्शरान्}
{व्यधमद्भीमसेनस्य शरजालानि पत्रिभिः}


\twolineshloka
{महारथो महाबाहुर्महाबाणैर्महाबलः}
{विव्याधाधिरथिर्भीमं नवभिर्निशितैस्तदा}


\twolineshloka
{स तोत्रैरिव मातङ्गो वार्यमाणः पतत्त्रिभः}
{अभ्यधावदसम्भ्रान्तः सूतपुत्रं वृकोदरः}


\twolineshloka
{तमापतन्तं वेगेन रभसं पाण्डवर्षभम्}
{कर्णः प्रत्युद्ययौ युद्धे मत्तो मत्तमिव द्विपम्}


\twolineshloka
{ततः प्रध्याप्य जलजं भेरीशतसमस्वनम्}
{अक्षुभ्यत बलं हर्षादुद्धूत इव सागरः}


\twolineshloka
{तदुद्धूतं बलं दृष्ट्वा नागाश्वरथपत्तिमत्}
{भीमः कर्णं समासाद्य च्छादयामास सायकैः}


\twolineshloka
{अश्वानृक्षसवर्णांश्च हंसवर्णैर्हयोत्तमैः}
{व्यामिश्रयद्रणे कर्णः पाण्डवं छादयञ्छरैः}


\twolineshloka
{ऋक्षवर्णान्हयान्कर्कैर्मिश्रान्मारुतरंहसः}
{निरीक्ष्य तव पुत्राणां हाहाकृतमभूद्बलम्}


\twolineshloka
{ते हया बह्वशोभन्त मिश्रिता वातरंहसः}
{सितासिता महाराज यथा व्योम्नि वलाहकाः}


\twolineshloka
{संरब्धौ क्रोधताम्राक्षौ प्रेक्ष्य कर्णवृकोदरौ}
{सन्त्रस्ताः समकम्पन्त त्वदीयानां महारथाः}


\twolineshloka
{यमराष्ट्रोपमं घोरमासीदायोधनं तयोः}
{दुर्दर्शं भरतश्रेष्ठ प्रेतराजपुत्रं यथा}


\twolineshloka
{समाजमिव तच्चित्रं प्रेक्षमाणा महारथाः}
{नालक्षयञ्जयं व्यक्तमेकस्यैव महारणे}


\twolineshloka
{तयोः प्रैक्षन्त सम्मर्दं सन्निकृष्टं महास्त्रयोः}
{तव दुर्मन्त्रिते राजन्सपुत्रस्य विशाम्पते}


\twolineshloka
{छादयन्तौ हि शत्रुघ्नावन्योन्यं सायकैः शितैः}
{शरजालावृतं व्योम चक्रातेऽद्भुतविक्रमौ}


\twolineshloka
{तावन्योन्यं जिघांसन्तौ शरैस्तीक्ष्णैर्महारथौ}
{प्रेक्षणीयतरावास्तां वृष्टिमन्ताविवाम्बुदौ}


\twolineshloka
{सुवर्णविकृतान्वाणान्विमुञ्चन्तावरिन्दमौ}
{भास्वरं व्योम चक्राते महोल्काभिरिव प्रभो}


\twolineshloka
{ताभ्यां मुक्ताः शरा राजन्गार्ध्रपत्राश्चकाशिरे}
{श्रेण्यः शरदि मत्तानां सारसनामिवाम्बरे}


\twolineshloka
{संसक्तं सूतपुत्रेण दृष्ट्वा भीममरिन्दमम्}
{अतिभारममन्येतां भीमे कृष्णधनञ्जयौ}


\twolineshloka
{तत्राधिरथिभीमाभ्यां शरैर्मुक्तैर्दृढं हताः}
{इषुपातमतिक्रम्य पेतुरश्वनरद्विपाः}


\twolineshloka
{पतद्भिः पतितैश्चान्यैर्गतासुभिरनेकशः}
{कृतो राजन्महाराज पुत्राणां ते जनक्षयः}


\twolineshloka
{मनुप्याश्वगजानां च शरीरैर्गतजीवितैः}
{क्षणेन भूमिः सञ्जज्ञे संवृता भरतर्षभ}


\chapter{अध्यायः १३३}
\twolineshloka
{धृतराष्ट्र उवाच}
{}


\twolineshloka
{अत्यद्भुतमहं मन्ये भीमसेनस्य विक्रमम्}
{यत्कर्णं योधयामास समरे लघुविक्रमः}


\twolineshloka
{त्रिदशानपि वा युक्तान्सर्वशस्त्रधरान्युधि}
{वारयेद्यो रणे कर्णः सयक्षासुरमानुषान्}


\twolineshloka
{स कथं पाण्डवं युद्धे भ्राजमानमिव श्रिया}
{नातरत्संयुगे तात तन्ममाचक्ष्व सञ्जय}


\twolineshloka
{कथं च युद्धं सम्भूतं तयोः प्राणदुरोदरे}
{अत्र मन्ये समायत्तो जयो वाऽजय एव च}


\twolineshloka
{कर्णं प्राप्य रणे सूत मम पुत्रः सुयोधनः}
{जेतुमुत्सहते पार्थान्सगोविन्दान्ससात्वतान्}


\twolineshloka
{श्रुत्वा तु निर्जितं कर्णमसकृद्भीमकर्मणा}
{भीमसेनेन समरे मोह आविशतीव माम्}


\twolineshloka
{विनष्टान्कौरवान्मन्ये मम पुत्रस्य दुर्नयैः}
{नहि कर्णो महेष्वासान्पार्थाञ्जेष्यति सञ्जय}


\twolineshloka
{कृतवान्यानि युद्धानि कर्णः पाण्डुसुतैः सह}
{सर्वत्र पाण्डवाः कर्णमजयंस्ते रणाजिरे}


\twolineshloka
{अजेयाः पाण्डवास्तात देवैरपि सवासवैः}
{न च तद्बुध्यते मन्दः पुत्रो दुर्योधनो मम}


\twolineshloka
{धनं धनेश्वरस्येव हृत्वा पार्थस्य मे सुतः}
{मधुप्रेप्सुरिवाबुद्धिः प्रपातं नावबुध्यते}


\twolineshloka
{निकृत्या निकृतिप्रज्ञो राज्यं हृत्वा महात्मनाम्}
{जितमित्येव मन्वानः पाण्डवानवमन्यते}


\twolineshloka
{पुत्रस्नेहाभिभूतेन मया चाप्यकृतात्मना}
{धर्मे स्थिता महात्मानो निकृताः पाण्डुनन्दनाः}


\twolineshloka
{शमकामः ससोदर्यो दीर्घप्रेक्षी युधिष्ठिरः}
{अशक्त इति मत्वा तु मम पुत्रैर्निराकृतः}


\twolineshloka
{तानि दुःखान्यनेकानि विप्रकारांश्च सर्वशः}
{हृदि कृत्वा महाबाहुर्भीमोऽयुध्यत सूतजम्}


\threelineshloka
{तस्मान्मे सञ्जय ब्रूहि कर्णभीमौ यथा रणे}
{अयुध्येतां युधांश्रेष्ठौ परस्परवधैषिणौ ॥सञ्जय उवाच}
{}


\twolineshloka
{शृणु राजन्यथावृत्तं सङ्ग्रामं कर्णभीमयोः}
{परस्परवधप्रेप्स्वोर्वनकुञ्जरयोरिव}


\twolineshloka
{राजन्वैकर्तनो भीमं क्रुद्धः क्रुद्धमरिन्दमम्}
{पराक्रान्तं पराक्रम्य विव्याध त्रिंशता शरैः}


\twolineshloka
{महावेगैः प्रसन्नाग्रैः शातकुम्भपरिष्कृतैः}
{नातरद्भरतश्रेष्ठ भीमं वैकर्तनः शरैः}


\twolineshloka
{तस्यास्यतो धनुर्भीमश्चकर्त विशिखैस्त्रिभिः}
{रथनीडाच्च यन्तारं भल्लेनापातयत्क्षितौ}


\twolineshloka
{स काङ्क्षन्भीमसेनस्य वधं वैकर्तनो भृशम्}
{शक्तिं कनकवैदूर्यचित्रदण्डां परामृशत्}


\threelineshloka
{प्रगृह्य च महाशक्तिं कालरात्रिमिवापराम्}
{समुत्क्षिप्य च राधेयः सन्धाय च महाबलः}
{चिक्षेप भीमसेनाय जीवितान्तकरीमिव}


\twolineshloka
{शक्तिं विसृज्य राधेयः पुरन्दर इवाशनिम्}
{ननाद सुमहानादं बलवान्सूतनन्दनः}


% Check verse!
तं च नादं ततः श्रुत्वा पुत्रास्ते हर्षिताऽभवन्
% Check verse!
शक्तिं वियति चिच्छेद भीमः सप्तभिराशुगैः
\twolineshloka
{छित्त्वा शक्तिं ततो भीमो निर्मुक्तोरगसन्निभाम्}
{मार्गमाणामिव प्राणान्सूतपुत्रस्य मारिष}


\twolineshloka
{प्राहिणोत्कृतसंरम्भः शरान्बर्हिणवाससः}
{स्वर्णपुङ्खाञ्शिलाधौतान्यमदण्डोपमान्मृधे}


\twolineshloka
{कर्णोऽप्यन्यद्धनुर्गृह्य हेमपृष्ठं दुरासदम्}
{विकृष्य तन्महच्चापं व्यसृजत्सायकांस्तदा}


\threelineshloka
{तान्पाण्डुपुत्रश्चिच्छेद नवभिर्नतपर्वभिः}
{वसुषेणेन निर्मुक्तान्नव राजन्महाशरान्}
{छित्त्वा भीमो महाराज नादं सिंह इवानदत्}


\twolineshloka
{तौ वृषाविव नर्दन्तौ बलिनौ वासितान्तरे}
{शार्दूलाविव चान्योन्यमामिषार्थेऽभ्यगर्जताम्}


\twolineshloka
{अन्योन्यं प्रजिहीर्षन्तावन्योन्यस्यान्तरैषिणौ}
{अन्योन्यमभिवीक्षन्तौ गोष्ठेष्विव महर्षभौ}


\twolineshloka
{महागजाविवसाद्य विषाणाग्रैः परस्परम्}
{शरैः पूर्णायतोत्सृष्टैरन्योन्यमभिजघ्नतुः}


\twolineshloka
{निर्दहन्तौ महाराज शस्त्रवृष्ट्या परस्परम्}
{अन्योन्यमभिवीक्षन्तौ कोपाद्विवृतलोचनौ}


\twolineshloka
{प्रहसन्तौ तथाऽन्योन्यं भर्त्सयन्तौ मुहुर्मुहुः}
{शङ्खशब्दं च कुर्वाणौ युयुधाते परस्परम्}


\threelineshloka
{तस्य भीमः पुनश्चापं मुष्टौ चिच्छेद मारिष}
{शङ्खवर्णांश्च तानश्वान्बाणैर्निन्ये यमक्षयम्}
{सारथिं च तथाप्यस्य रथनीडादपातयत्}


\threelineshloka
{ततो वैकर्तनः कर्णश्चिन्तां प्राप दुरत्ययाम्}
{स च्छाद्यमानः समरे हताश्वो हतसारथिः}
{मोहितः शरजालेन कर्तव्यं नाभ्यपद्यत}


\twolineshloka
{तथा कृच्छ्रगतं दृष्ट्वा कर्णं दुर्योधनो नृपः}
{वेपमान इव क्रोधाद्व्यादिदेशाथ दुर्जयम्}


\twolineshloka
{गच्छ दुर्जय राधेयं पुरा ग्रसति पाण्डवः}
{जहि तूबरकं क्षिप्रं कर्णस्य बलमादधत्}


\twolineshloka
{एवमुक्तस्तथेत्युक्त्वा तव पुत्रं तवात्मजः}
{अभ्यद्रवद्भीमसेनं व्यासक्तं विकिरञ्छरैः}


\twolineshloka
{स भीमं नवभिर्बाणैरश्वानष्टभिरार्दयत्}
{षङ्भिः सूतं त्रिभिः केतं पुनस्तं चापि सप्तभिः}


\twolineshloka
{भीमसेनोऽपि सङ्क्रुद्धः साश्वयन्तारमाशुगैः}
{दुर्जयं भिन्नमर्माणमनयद्यमसादनम्}


\twolineshloka
{बाढं हतं क्षितौ क्षुण्णं वेष्टमानं यथोरगम्}
{रुदन्नार्तस्तव सुतं कर्णश्चक्रे प्रदक्षिणम्}


\twolineshloka
{स तु तं विरथं कृत्वा स्मयन्नत्यन्तवैरिणम्}
{समाचिनोद्बाणगणैः शतघ्नीभिश्च शङ्कुभिः}


\twolineshloka
{तथाप्यतिरथः कर्णो भिद्यमानोऽस्य सायकैः}
{न जहौ समरे भीमं क्रुद्धरूपं परन्तपः}


\chapter{अध्यायः १३४}
\twolineshloka
{सञ्जय उवाच}
{}


\twolineshloka
{सर्वथा विरथः कर्णः पुनर्भीमेन निर्जितः}
{रथमन्यं समास्थाय पुनर्विव्याध पाण्डवम्}


\twolineshloka
{महागजाविवासाद्य विषाणाग्रैः परस्परम्}
{शरैः पूर्णायतोत्सृष्टैरन्योन्यमभिजघ्नतुः}


\twolineshloka
{अथ कर्णः शरव्रातैर्भीमसेनं समार्पयत्}
{ननाद च महानादं पुनर्विव्याध चोरसि}


\twolineshloka
{तं भीमो दशभिर्बाणैः प्रत्यविध्यदजिह्मगैः}
{पुनर्विव्याध सप्तत्या शराणां नतपर्वणाम्}


\twolineshloka
{कर्णस्तु नवभिर्भीमं भित्त्वा राजंस्तनान्तरे}
{ध्वजमेकेन विव्याध विशोकं च त्रिभिः शरैः}


\twolineshloka
{सायकानां ततः पार्थस्त्रिषष्ट्या प्रत्यविध्यत}
{तोत्रैरिव महानागं कशाभिरिव वाजिनम्}


\twolineshloka
{सोऽतिविद्धो महाराज पाण्डवेन यशस्विना}
{सृक्विणी लेलिहन्वीरः क्रोधरक्तान्तलोचनः}


\twolineshloka
{ततः शरं महाराज सर्वकायावदारणम्}
{प्राहिणोद्भीमसेनाय बलायेन्द्र इवाशनिम्}


\twolineshloka
{स निर्भिद्य रणे पार्थं सूतपुत्रधनुश्च्युतः}
{अगच्छद्दारयन्भूमिं चित्रपुङ्खः शिलीमुखः}


\threelineshloka
{ततो भीमो महाबाहुः क्रोधसंरक्तलोचनः}
{वज्रकल्पां चतुष्किष्कुं गुर्वी रुक्माङ्गदां गदाम्}
{प्राहिणोत्मूतपुत्राय षडस्रामविचारयन्}


\twolineshloka
{तया जघानाधिरथेः सदश्वान्साधुवाहिनः}
{गदया भारतः क्रुद्धो वज्रेणेन्द्र इवासुरान्}


\twolineshloka
{ततो भीमो महाबाहुः क्षुराभ्यां भरतर्षभ}
{ध्वजमाघिरथेश्छित्त्वा सूतमभ्यहनच्छरैः}


\twolineshloka
{हताश्वसूतमुत्सृज्य स रथं पतितध्वजम्}
{विष्फारयन्धनुः कर्णस्तस्थौ भारत दुर्मनाः}


\twolineshloka
{तत्राद्भुतमपश्याम राधेयस्य पराक्रमम्}
{विरथो रथिनां श्रेष्ठो वारयामास यद्रिपुम्}


\twolineshloka
{विरथं तं नरश्रेष्ठं दृष्ट्वाऽऽधिरथिमाहवे}
{दुर्योधनस्ततो राजन्नभ्यभाषत दुर्मुखम्}


\twolineshloka
{एष दुर्मुख राधेयो भीमेन विरथीकृतः}
{तं रथेन नरश्रेष्ठ सम्पादय महारथम्}


\twolineshloka
{ततो दुर्योधनवचः श्रुत्वा भारत दुर्मुखः}
{त्वरमाणोऽभ्ययात्कर्णं भीमं चावारयच्छरैः}


\twolineshloka
{दुर्मुखं प्रेक्ष्य सङ्ग्रामे सूतपुत्रपदानुगम्}
{वायुपुत्रः प्रहृष्टोऽभूत्सृक्विणी परिसंलिहन्}


\twolineshloka
{ततः कर्णं महाराज वारयित्वा शिलीमुखैः}
{दुर्मुखाय रथं तूर्णं प्रेषयामास पाण्डवः}


\twolineshloka
{तस्मिन्क्षणे महाराज नवभिर्नतपर्वभिः}
{सुमुखैर्दुर्मुखं भीमः शरैर्निन्ये यमक्षयम्}


\twolineshloka
{ततस्तमेवाधिरथिः स्यन्दनं दुर्मुखे हते}
{आस्थितः प्रबभौ राजन्दीप्यमान इवांशुमान्}


\twolineshloka
{शयानं भिन्नमर्माणं दुर्मुखं शोणितोक्षितम्}
{दृष्ट्वा कर्णोऽश्रुपूर्णाक्षो मुहूर्तं नाभ्यवर्तत}


\twolineshloka
{तं गतासुमतिक्रम्य कृत्वा कर्णः प्रदक्षिणम्}
{दीर्घमुष्णं श्वसन्वीरो न किञ्चित्प्रत्यपद्यत}


\twolineshloka
{तस्मिंस्तु विवरे राजन्नाराचान्गार्धवाससः}
{प्राहिणोत्सूतपुत्राय भीमसेनश्चतुर्दश}


\twolineshloka
{ते तस्य कवचं भित्त्वा स्वर्णचित्रं महोजसः}
{हेमपुङ्खा महाराज व्यशोभन्त दिशो दश}


\twolineshloka
{अपिबन्सूत पुत्रस्य शोणितं रक्तभोजनाः}
{क्रुद्धा इव मनुष्येन्द्र भुजङ्गाः कालचोदिताः}


\twolineshloka
{प्रसर्पमाणा मेदिन्यां ते व्यरोचन्त मार्गणाः}
{अर्धप्रविष्टाः संरब्धा बिलानीव महोरगाः}


\twolineshloka
{तं प्रत्यविध्यद्राधेयो जाम्बूनदविभूषितैः}
{चतुर्दशभिरत्युग्रैर्नाराचैरविचारयन्}


\twolineshloka
{ते भीमसेनस्य भुजं सव्यं निर्भिद्य पत्रिणः}
{प्राविशन्मेदिनीं भीमाः क्रौञ्चं पत्ररथा इव}


\twolineshloka
{ते व्यरोचन्त नाराचाः प्रविशन्तो वसुन्धराम्}
{गच्छत्यस्तं दिनकरे दीप्यमाना इवांशवः}


\twolineshloka
{स निर्भिन्नो रणे भीमो नाराचैर्मर्मभेदिभिः}
{सुस्राव रुधिरं भूरि पर्वतः सलिलं यथा}


\twolineshloka
{स भीमस्त्रिभिरायस्तः सूतपुत्रं पतत्रिभिः}
{सुपर्णवेगैर्विव्याध सारथिं चास्य सप्तभिः}


\twolineshloka
{स विह्वलो महाराज कर्णो भीमशराहतः}
{प्राद्रवज्जवनैरश्वै रणं हित्वा महाभयात्}


\twolineshloka
{भीमसेनस्तु विष्फार्य चापं हेमपरिष्कृतम्}
{आहवेऽतिरथोऽतिष्ठज्ज्वलन्निव हुताशनः}


\chapter{अध्यायः १३५}
\twolineshloka
{धृतराष्ट्र उवाच}
{}


\twolineshloka
{दैवमेव परं मन्ये धिक्पौरुषमनर्थकम्}
{यत्राधिरथिरायत्तौ नातरत्पाण्डवं रणे}


\threelineshloka
{कर्णः पार्थान्सगोविन्दाञ्जेतुमुत्सहते रणे}
{न च कर्णसमं लोके पश्यामि कञ्चन}
{इति दुर्योधनस्याहमश्रौषं जल्पतो मुहुः}


\twolineshloka
{कर्णो हि बलवाञ्छूरो दृढधन्वा जितक्लमः}
{इति मामब्रवीत्सूत मन्दो दुर्योधनः पुरा}


\twolineshloka
{वसुषेणसहायं मां नालं देवाऽपि संयुगे}
{किं नु पाण्डुसुता राजन्गतसत्वा विचेतसः}


\twolineshloka
{तत्र तं निर्जितं दृष्ट्वा भुजङ्गमिव निर्विषम्}
{युद्धात्कर्णमपक्रान्तं किंस्विद्दुर्योधनोऽब्रवीत्}


\twolineshloka
{अहो दुर्मुखमेवैकं युद्धानामविशारदम्}
{प्रावेशयद्धुतवहं पतङ्गमिव मोहितः}


\twolineshloka
{अश्वत्थामा मद्रराजः कृपः कर्णश्च सङ्गताः}
{न शक्त्याः प्रमुखे स्थातुं नूनं भीमस्य सञ्जय}


\twolineshloka
{तेऽपि चास्य महाघोरं बलं नागायुतोपमम्}
{जानन्तो व्यवसायं च क्रूरं मारुततेजसः}


\twolineshloka
{किमर्थं क्रूरकर्माणं यमकालान्तकोपमम्}
{बलसंरम्भवीर्यज्ञाः कोपयिष्यन्ति संयुगे}


\twolineshloka
{कर्णस्त्वेको महाबाहुः स्वबाहुबलदर्पितम्}
{भीमसेनमनादृत्य रणेऽयुध्यत सूतजः}


\twolineshloka
{योऽजयत्समरे कर्णं पुरन्दर इवासुरम्}
{न स पाण्डुसुतो जेतुं शक्यः केनचिदाहवे}


\twolineshloka
{द्रोणं यः सम्प्रमथ्यैकः प्रविष्टो मम वाहिनीम्}
{भीमो धनञ्जयान्वेषी कस्तमार्च्छेज्जिजीविषु}


\twolineshloka
{को हि सञ्जय भीमस्य स्थातुमुत्सहतेऽग्रतः}
{उद्यताशनिहस्तस्य महेन्द्रस्येव दानवः}


\twolineshloka
{प्रेतराजपुरं प्राप्य निवर्तेतापि मानवः}
{न भीमसेनं सम्प्राप्य निवर्तेत कदाचन}


\twolineshloka
{पतङ्गा इव वह्निं ते प्राविशन्नल्पचेतसः}
{ये भीमसेनं सङ्क्रुद्धमन्वधावन्विमोहिताः}


\twolineshloka
{यत्तत्सभायां भीमेन मम पुत्रवधाश्रयम्}
{उक्तं संरम्भिणोग्रेण कुरूणां शृण्वतां तदा}


\twolineshloka
{तन्नूनमभिसञ्चिन्त्य दृष्ट्वा कर्णं च निर्जितम्}
{दुःशासनः सह भ्रात्रा भयाद्भीमादुपारमत्}


\twolineshloka
{यश्च सञ्जय दुर्बुद्धिरब्रवीत्समितौ मुहुः}
{कर्णो दुःशासनोऽहं च जेष्यामो युधि पाण्डवान्}


\twolineshloka
{स नूनं विरथं दृष्ट्वा कर्णं भीमेन निर्जितम्}
{प्रत्याख्यानाच्च कृष्णस्य भृशं तप्यति पुत्रकः}


\twolineshloka
{दृष्ट्वा भ्रातॄन्हतान्सङ्ख्ये भीमसेनेन दंशितान्}
{आत्मापराधात्सम्मूढो नूनं तप्यति पुत्रकः}


\twolineshloka
{को हि जीवितमन्विच्छन्प्रतीपं पाण्डवं व्रजेत्}
{भीमं भीमायुधं क्रुद्धं साक्षात्कालमिव स्थितम्}


\twolineshloka
{बडबामुखमध्यस्थो मुच्येतापि हि मानवः}
{न भीममुखसम्प्राप्तो मुच्येदिति मतिर्मम}


\twolineshloka
{न पार्था न च पञ्चाला न च केशवसात्यकी}
{जानते युधि संरब्धा जीवितं परिरक्षितुम्}


\twolineshloka
{अहो मम सुतानां हि विपन्नं सूत जीवितम् ॥सञ्जय उवाच}
{}


\twolineshloka
{यस्त्वं शोचसि कौरव्य वर्तमाने महाभये}
{त्वमस्य जगतो मूलं विनाशस्य न संशयः}


\twolineshloka
{स्वयं वैरं महत्कृत्वा पुत्राणां वचने स्थितः}
{उच्यमानो न गृह्णीषे मर्त्यः पथ्यमिवौषधम्}


\twolineshloka
{स्वयं पीत्वा महाराज कालकूटं सुदुर्जरम्}
{तस्येदानीं फलं कृत्स्नमवाप्नुहि नरोत्तम}


\twolineshloka
{यत्तु कुत्सयसे योधान्युध्यमानान्महाबलान्}
{तत्र ते वर्तयिष्यामि यथा युद्धमवर्तत}


\twolineshloka
{दृष्ट्वा कर्णं तु पुत्रास्ते भीमसेनपराजितम्}
{नामृष्यन्त महेष्वासाः सोदर्याः पञ्च भारत}


\twolineshloka
{दुर्मर्षणो दुःसहश्च दुर्मदो दुर्धरो जयः}
{पाण्डवं चित्रसन्नाहास्तं प्रतीपमुपाद्रवन्}


\twolineshloka
{ते समन्तान्महाबाहुं परिवार्य वृकोदरम्}
{दिशः शरैः समावृण्वञ्शलभानामिव व्रजैः}


\twolineshloka
{आगच्छतस्तान्सहसा कुमारान्देवरूपिणः}
{प्रतिजग्राह समरे भीमसेनो हसन्निव}


\twolineshloka
{तव दृष्ट्वा तु तनयान्दुर्मर्षणपुरोगमान्}
{अभ्यवर्तत राधेयो भीमसेनं महाबलम्}


\twolineshloka
{विसृजन्विशिखांस्तीक्ष्णान्स्वर्णपुङ्खांञ्छिलाशितान्}
{तं तु भीमोऽभ्ययात्तूर्णं वार्यमाणः सुतैस्तव}


\twolineshloka
{कुरवस्तु ततः कर्णं परिवार्य समन्ततः}
{अवाकिरन्भीमसेनं शरैः सन्नतपर्वभिः}


\twolineshloka
{तान्बाणैः पञ्चविंशत्या साश्वान्राजन्नरर्षभान्}
{ससूतान्भीमधनुषो भीमो निन्ये यमक्षयम्}


\twolineshloka
{प्रापतन्स्यन्दनेभ्यस्ते सार्धं सूतैर्गतासवः}
{चित्रपुष्पधरा भग्ना वातेनेव महाद्रुमाः}


\twolineshloka
{तत्राद्भुतमपश्याम भीमसेनस्य विक्रमम्}
{संवार्याधिरथिं बाणैर्यज्जघान तवात्मजान्}


\twolineshloka
{स वार्यमाणो भीमेन शितैर्बाणैः समन्ततः}
{सूतपुत्रो महाराज भीमसेनमवैक्षत}


\twolineshloka
{तं भीससेनः संरम्भात्क्रोधसंरक्तलोचनः}
{विष्फार्य सुमहच्चापं मुहुः कर्णमवैक्षत}


\chapter{अध्यायः १३६}
\twolineshloka
{सञ्जय उवाच}
{}


\twolineshloka
{तवात्मजांस्तु पतितान्दृष्ट्वा कर्णः प्रतापवान्}
{क्रोधेन महताऽऽविष्टो निर्विण्णोऽभूत्स जीवितो}


\twolineshloka
{आगस्कृतमिवात्मानं मेने चाधिरथिस्तदा}
{यत्प्रत्यक्षं तव सुता भीमेन निहता रणे}


\twolineshloka
{भीमसेनस्ततः क्रुद्धः कर्णस्य निशिताञ्शरान्}
{निचखान् स सम्भ्रान्तः पूर्ववैरमनुस्मरन्}


\twolineshloka
{स भीमं पञ्चभिर्विद्ध्वा राधेयः प्रहसन्निव}
{पुनर्विव्याध सप्तत्या स्वर्णपुङ्खैः शिलाशितैः}


\twolineshloka
{अविचिन्त्याथ तान्बाणान्कर्णेनास्तान्वृकोदरः}
{रणे विव्याध राधेयं शरेणानतपर्वणा}


\twolineshloka
{पुनश्च विशिखैस्तीक्ष्णैर्विद्ध्वा मर्मसु पञ्चभिः}
{धनुश्चिच्छेद भल्लेन सूत पुत्रस्य मारिष}


\twolineshloka
{अथान्यद्धनुरादाय कर्णो भारत दुर्मनाः}
{इषुभिश्छादयामास मीमसेनं परन्तपः}


\twolineshloka
{तस्य भीमो हयान्हत्वा विनिहत्य च सारथिम्}
{प्रजहास महाहासं कृतप्रतिकृतं महत्}


\twolineshloka
{इषुभिः कार्मुकं चास्य चकर्त पुरुषर्षभः}
{तत्पपात महाराज स्वर्णपृष्ठं महास्वनम्}


\twolineshloka
{अवारोहद्रथात्तस्मादथ कर्णो महारथः}
{गदां गृहीत्वा समरे भीमाय प्राहिणोद्रुषा}


\twolineshloka
{तामापतन्तीमालक्ष्य भीमसेनो महागदाम्}
{शरैरवारयद्राजन्सर्वसैन्यस्य पश्यतः}


\twolineshloka
{ततो बाणसहस्राणि प्रेषयामास पाण्डवः}
{सूतपुत्रवधाकाङ्क्षी त्वरमाणः पराक्रमी}


\twolineshloka
{तानिषूनिषुभिः कर्णो वारयित्वा महामृधे}
{कवचं भीमसेनस्य पाटयामास सायकैः}


\twolineshloka
{अथैनं पञ्चविंशत्या नाराचानां समार्पयत्}
{पश्यतां सर्वसैन्यानां तदद्भुतमिवाभवत्}


\twolineshloka
{ततो भीमो महाबाहुर्नवभिर्नतपर्वभिः}
{प्रेषयामास सङ्क्रुद्धः सूतपुत्रस्य मारिष}


\twolineshloka
{ते तस्य कवचं भित्त्वा तथा बाहुं च दक्षिणम्}
{अभ्ययुर्धरणीं तीक्ष्णा वल्मीकमिव पन्नगाः}


\twolineshloka
{स च्छाद्यमानो बाणौधैर्भीमसेनधनुश्च्युतैः}
{पुनरेवाभवत्कर्णो भीमसेनात्पराङ्मुखः}


\twolineshloka
{तं पराङ्मुखमालोक्य पदातिं सूतनन्दनम्}
{कौन्तेयशरसञ्छन्नं राजा दुर्योधनोऽब्रवीत्}


\twolineshloka
{त्वरध्वं सर्वतो यत्ता राधेयस्य रथं प्रति}
{ततस्तव सुता राजञ्श्रुत्वा भ्रातुर्वचो द्रुतम्}


\twolineshloka
{अभ्ययुः पाण्डवं युद्धे विसृजन्तः शिलीमुखान्}
{चित्रोपचित्रश्चित्राक्षश्चारुचित्रः शरासनः}


\twolineshloka
{चित्रायुधश्चित्रवर्मा समरे चित्रयोधिनः}
{तानापतत एवाशु भीमसेनो महारथः}


\twolineshloka
{एकैकेन शरेणाजौ पातयामास ते सुतान्}
{ते हता न्यपतन्भूमौ वातरूग्णा इव द्रुमाः}


\twolineshloka
{दृष्ट्वा विनिहतान्पुत्रांस्तव राजन्महारथान्}
{अश्रुपूर्णमुखः कर्णः क्षत्तुः सस्मार तद्वचः}


\twolineshloka
{रथं चान्यं समास्थाय विधिवत्कल्पितं पुनः}
{अभ्ययात्पाण्डवं युद्धे त्वरमाणः पराक्रमी}


\twolineshloka
{तावन्योन्यं शरैर्भित्त्वा स्वर्णपुङ्खैः शिलाशितैः}
{व्यभ्राजेतां यथा मेघौ संस्यूतौ सूर्यरश्मिभिः}


\twolineshloka
{षट््त्रिंशद्भिस्ततो भल्लैर्निशितैस्तिग्मतेजनैः}
{व्यधमत्कवनं क्रुद्धः सूतपुत्रस्य पाण्डवः}


\twolineshloka
{सूतपुत्रोऽपि कौन्तेयं शरैः सन्नतपर्वभिः}
{पञ्चाशता महाबाहुर्विव्याध भरतर्षभ}


\twolineshloka
{रक्तचन्दनदिग्धाङ्गौ शरैः कृतमहाव्रणौ}
{शोणिताक्तौ व्यराजेतां कालसूर्याविवोदितौ}


\twolineshloka
{तौ शोणितोक्षितैर्गात्रैः शरैश्छिन्नतनुच्छदौ}
{विवृताङ्गौ व्यराजेतां निर्मुक्ताविव पन्नगौ}


\twolineshloka
{`क्रोधाग्नितेजसा दिप्तौ संरम्भाद्रक्तलोचनौ}
{व्यभ्राजेतां महाराज विधूमाविव पावकौ'}


\twolineshloka
{व्याघ्राविव नरव्याघ्रौ दंष्ट्राभिरितरेतरम्}
{शरधारासृजौ वीरौ मेघाविव ववर्षतुः}


\twolineshloka
{वारणाविव चान्योन्यं विषाणाभ्यामरिन्दमौ}
{निर्भिन्दन्तौ स्वगात्राणि सायकैश्चारु रेजतुः}


\twolineshloka
{नादयन्तौ प्रहर्षन्तौ विक्रीडन्तौ परस्परम्}
{मण्डलानि विकुर्वाणौ रथाभ्यां रथिषूत्तमौ}


\twolineshloka
{वृषाविवाथ नर्दन्तौ बलिनौ वासितान्तरे}
{सिंहाविव पराक्रान्तौ नरसिंहौ महाबलौ}


\twolineshloka
{परस्परं वीक्षमाणौ क्रोधसंरक्तलोचनौ}
{युयुधाते महावीर्यौ शक्रवैरोचनी यथा}


\twolineshloka
{ततो भीमो महाबाहुर्बाहुभ्यां विक्षिपन्धनुः}
{व्यराजत रणे राजन्सविद्युदिव तोयदः}


\twolineshloka
{सनेमिघोषस्तनितश्चापविद्युच्छराम्बुभिः}
{भीमसेनमहामेघः कर्णपर्वतमावृणोत्}


\twolineshloka
{ततः शरसहस्रेण सम्यगस्तेन भारत}
{पाण्डवो व्यकिरत्कर्णं भीमो भीमपराक्रमः}


\twolineshloka
{तत्रापश्यंस्तव सुता भीमसेनस्य विक्रमम्}
{सुपुङ्खैः कङ्कवासोभिर्यत्कर्णं छादयञ्शरैः}


\twolineshloka
{स नन्दयन्रणे पार्थं केशवं च यशस्विनम्}
{सात्यकिं चक्ररक्षौ च भीमः कर्णमयोधयत्}


\twolineshloka
{विक्रमं भुजयोर्वीर्यं धैर्यं च विदितात्मनः}
{पुत्रास्तव महाराज दृष्ट्वा विमनसोऽभवन्}


\chapter{अध्यायः १३७}
% Check verse!

% Check verse!

% Check verse!

% Check verse!

% Check verse!

% Check verse!

% Check verse!

% Check verse!

% Check verse!

% Check verse!

% Check verse!

% Check verse!

% Check verse!

% Check verse!

% Check verse!

% Check verse!

% Check verse!

% Check verse!

% Check verse!

% Check verse!

% Check verse!

% Check verse!

% Check verse!

% Check verse!

% Check verse!

% Check verse!

% Check verse!

% Check verse!

% Check verse!

% Check verse!

% Check verse!

% Check verse!

% Check verse!

% Check verse!

% Check verse!

% Check verse!

% Check verse!

% Check verse!

% Check verse!

% Check verse!

% Check verse!

% Check verse!

% Check verse!

% Check verse!

% Check verse!

% Check verse!

% Check verse!

% Check verse!

% Check verse!

% Check verse!

% Check verse!

% Check verse!

% Check verse!

% Check verse!

\chapter{अध्यायः १३८}
% Check verse!

% Check verse!

% Check verse!

% Check verse!

% Check verse!

% Check verse!

% Check verse!

% Check verse!

% Check verse!

% Check verse!

% Check verse!

% Check verse!

% Check verse!

% Check verse!

% Check verse!

% Check verse!

% Check verse!

% Check verse!

% Check verse!

% Check verse!

% Check verse!

% Check verse!

% Check verse!

% Check verse!

% Check verse!

% Check verse!

% Check verse!

% Check verse!

% Check verse!

% Check verse!

\chapter{अध्यायः १३९}
\twolineshloka
{सञ्जय उवाच}
{}


\threelineshloka
{ततः कर्णो महाराज भीमं विद्धा त्रिभिः शूरैः}
{`पुनश्च षड््भिस्तीक्ष्णाग्रैरविध्यत्कङ्कपत्रिभिः'}
{मुमोच शरवर्षाणि विचित्राणि बहूनि च}


\twolineshloka
{वध्यमानो महाबाहुः सूतपुत्रेण पाण्डवः}
{न विव्यथे भीमसेनो भिद्यमान इवाचलः}


\twolineshloka
{स कर्णं कर्णिना कर्णे पीतेन निशितेन च}
{विव्याध सुभृशं सङ्ख्ये तैलधौतेन मारिष}


\twolineshloka
{स कुण्डलं महच्चारु कर्णस्यापातयद्भुवि}
{तपनीयं महाराज दीप्तं ज्योतिरिवाम्बरात्}


\twolineshloka
{अथापरेण भल्लेन सूतपुत्रं स्तनान्तरे}
{आजघान भृशं क्रुद्धो हसन्निव वृकोदरः}


\twolineshloka
{पुनरस्य त्वरन्भीपो नाराचान्दश भारत}
{रणे प्रैषीन्महाबाहुर्निर्मुक्ताशीविषोपमान्}


\twolineshloka
{ते ललाटं विनिर्भिद्य सूतपुत्रस्य भारत}
{विविशुश्चोदितास्तेन वल्मीकमिव पन्नगाः}


\twolineshloka
{ललाटस्थैस्ततो बाणैः सूतपुत्रो व्यरोचत}
{नीलोत्पलमयीं मालां धारयन्वै यथा पुरा}


\twolineshloka
{सोऽतिविद्धो भृशं कर्णः पाण्डवेन तरस्विना}
{रथकूबरमालम्ब्य न्यमीलयत लोचने}


\twolineshloka
{स मुहूर्तात्पुनः संज्ञां लेभे कर्णः परन्तपः}
{रुधिरोक्षितसर्वाङ्गः क्रोधमाहारयत्परम्}


\twolineshloka
{ततः क्रुद्धो रणे कर्णः पीडितो दृढधन्वना}
{वेगं चक्रे महावेगो भीमसेनरथं प्रति}


\twolineshloka
{तस्मै कर्णः शतं राजन्निषूणां गार्ध्रवाससाम्}
{अमर्षी बलवान्क्रुद्धः प्रेषयामास भारत}


\twolineshloka
{ततः प्रासृजदुग्राणि शरवर्षाणि पाण्डवः}
{समरे तमनादृत्य तस्य वीर्यमचिन्तयन्}


\twolineshloka
{कर्णस्ततो महाराज पाण्डवं नवभिः शरैः}
{आजघानोऽसि क्रुद्धः क्रुद्धरूपं परन्तप}


\threelineshloka
{तावुभौ नरशार्दूलौ शार्दूलाविव दंष्ट्रिणौ}
{`शरदंष्ट्रौ समासाद्य ततक्षतुररिन्दमौ'}
{जिमूताविव चान्योन्यं प्रववर्षतुराहवे}


\twolineshloka
{तलशब्दरवैश्चैव त्रासयेतां परस्परम्}
{शरजालैश्च विविधैस्त्रासयामासतुर्मृधे}


\twolineshloka
{अन्योन्यं समरे क्रुद्धौ कृतप्रतिकृतैषिणौ}
{ततो भीमो महाबाहुः सूतपुत्रस्य भारत}


\twolineshloka
{क्षुरप्रेण धनुश्छित्त्वा ननाद परवीरहा}
{तदपास्य धनुश्छिन्नं सूतपुत्रो महारथः}


\twolineshloka
{अन्यत्कार्मुकमादत्त भारघ्नं वेगवत्तरम्}
{तदप्यथ निमेषार्धाच्चिच्छेदास्य वृकोदरः}


\twolineshloka
{तृतीयं च चतुर्थं च पञ्चमं षष्ठमेव हि}
{सप्तमं चाष्टमं चैव नवमं दशमं तथा}


\twolineshloka
{एकादशं द्वादशं च त्रयोधशमथापि च}
{चतुर्दशं पञ्चदशं षोडशं च वृकोदरः}


\threelineshloka
{तथा सप्तदशं वेगादष्टादशमथापि वा}
{बहूनि भीमश्चिच्छेद कर्णस्यैवं धनूंषि हि}
{निमेषार्धात्ततः कर्णो धनुर्हस्तो व्यतिष्ठत}


\twolineshloka
{दृष्ट्वा स कुरुसौवीरसिन्धुवीरबलक्षयम्}
{स वर्मध्वजशस्त्रैश्च पतितैः संवृतां महीम्}


\twolineshloka
{`रथैर्विमथितैर्भल्लैरश्वैश्चान्यैः प्रवल्गितैः}
{भ्रष्टश्रीकैर्नरवरैः पांसुकुण्ठितमूर्धजैः'}


\twolineshloka
{हस्त्यश्वरथदेहांश्च गतासून्प्रेक्ष्य सर्वशः}
{सूतपुत्रस्य संरम्भाद्दीप्तं वपुरजायत}


\twolineshloka
{स विष्फार्य महच्चापं कार्तस्वरविभूषितम्}
{भीमं प्रैक्षत राधेयो घोरं घोरेण चक्षुषा}


\twolineshloka
{ततः क्रुद्धः शरानस्यन्सूतपुत्रो व्यरो चत}
{मध्यन्दिनगतोऽर्चिष्माञ्शरदीव दिवाकरः}


\twolineshloka
{मरीचिविकचस्येव राजन्भानुमतो वपुः}
{आसीदाधिरथेर्घोरं वपुः शरशताचितम्}


\twolineshloka
{कराभ्यामाददानस्य सन्दधानस्य चाशुगान्}
{कर्षतो मुञ्चतो बाणान्नान्तरं ददृशे रणे}


\twolineshloka
{अग्निचक्रोपमं घोरं मण्डलीकृतमायुधम्}
{कर्णस्यासीन्महीपाल सव्यं दक्षिणमस्यतः}


\twolineshloka
{स्वर्णपुङ्खाः सुनिशिताः कर्मचापच्युताः शराः}
{प्राच्छादयन्महाराज दिशः सूर्य इवांशुभिः}


\twolineshloka
{ततः कनकपुङ्खानां शराणां नतपर्वणाम्}
{धनुश्च्युतानां वियति ददृशे बहुधा व्रजः}


\twolineshloka
{बाणासनादाधिरथेः प्रभवन्ति स्म सायकाः}
{श्रेणीकृता व्यरोचन्त राजन्क्रौञ्चा इवाम्बरे}


\twolineshloka
{गार्ध्रपत्राञ्शिलाधौतान्कार्तस्वरविभूषितान्}
{महावेगान्प्रदीप्ताग्रान्मुमोचाधिरथिः शरान्}


\twolineshloka
{ते तु चापबलोद्धूताः शातकुम्भविभूषिताः}
{अजस्रमपतन्बाणा भीमसेनरथं प्रति}


\twolineshloka
{ते व्योम्नि रुक्मविकृता व्यकाशन्त सहस्रशः}
{शलभानामिव व्राताः शराः कर्णसमीरिताः}


\twolineshloka
{चापादाधिरथेर्मुक्ताः प्रपतन्तश्चकाशिरे}
{एको दीर्घ इवात्यर्थमाकाशे संस्थितः शरः}


\twolineshloka
{पर्वतं वारिधाराभिच्छादयन्निव तोयदः}
{कर्णः प्राच्छादयत्क्रुद्धो भीमं सायकवृष्टिभिः}


\twolineshloka
{तत्र भारत भीमस्य बलं वीर्यं पराक्रमम्}
{व्यवसायं च पुत्रास्ते ददृशुः सहसैनिकाः}


\twolineshloka
{तां समुद्रमिवोद्भूतां शरवृष्टिं समुत्थिताम्}
{अचिन्तयित्वा भीमस्तु क्रुद्धः कर्णमुपाद्रवत्}


\threelineshloka
{रुक्मपृष्ठं महच्चापं भीमस्यासीद्विशाम्पते}
{आकर्षान्मण्डलीभूतं शक्रचापमिवापरम्}
{तस्माच्छराः प्रादुरासन्पूरयन्त इवाम्बरम्}


\twolineshloka
{सुवर्णपुङ्खैर्भीमेन सायकैर्नतपर्वभिः}
{गगने रचिता माला काञ्चनीव व्यरोचतं}


\twolineshloka
{ततो व्योम्नि विषक्तानि शरजालानि भागशः}
{आहतानि व्यशीर्यन्त भीमसेनस्य पत्रिभिः}


\threelineshloka
{कर्णस्य शरजालौघैर्भीमसेनस्य चोभयोः}
{अग्निस्फुलिङ्गसंस्पर्शैरञ्जोगतिभिराहवे}
{तैस्तैः कनकपुङ्खानां द्यौरासीत्संवृता व्रजैः}


\twolineshloka
{न स्म सूर्यस्तदा भाति न स्म वाति समीरणः}
{शरजालावृते व्योम्नि न प्राज्ञायत किञ्चन}


\twolineshloka
{स भिमं छादयन्बाणैः सूतपुत्रः पृथग्विधैः}
{उपारोहदनादृत्य तस्य वीर्यं महात्मनः}


\twolineshloka
{तयोर्विसृजतोस्तत्र शरजालानि मारिष}
{वायुभूतान्यदृश्यन्त संसक्तानीतरेतरम्}


\twolineshloka
{अन्योन्यशरसंसम्पर्शात्तयोर्मनुजसिंहयोः}
{आकाशे भरतश्रेष्ठ पावकः समजायत}


\twolineshloka
{तथा कर्णः शितान्बाणान्कर्मारपरिमार्जितान्}
{सुवर्णविकृतान्क्रुद्धः प्राहिणोद्वधकाङ्क्षया}


\twolineshloka
{तानन्तरिक्षे विशिखैस्त्रिधैकैकमशातयत्}
{विशेषयन्सूतपुत्रं भीमस्तिष्ठेति चाब्रवीत्}


\twolineshloka
{पुनश्चासृजदुग्राणि शरवर्षाणि पाण्डवः}
{अमर्षी बलवान्क्रुद्धौ दिधक्षन्निव पावकः}


% Check verse!
ततश्चटचटाशब्दो गोधाघातादभूत्तयोः
\twolineshloka
{तलशब्दश्च सुभहान्सिंहनादश्च भैरवः}
{रथनेमिनिनादश्च ज्याशब्दश्चैव दारुणः}


\twolineshloka
{योधा व्युपारमन्युद्धाद्दिदृक्षन्तः पराक्रमम्}
{कर्णपाण्डवयो राजन्परस्परवधैषिणोः}


\twolineshloka
{देवर्षिसिद्धगन्धर्वाः साधुसाध्वित्यपूजयन्}
{मुमुचुः पुष्पवर्षं च विद्याधरगणास्तथा}


\twolineshloka
{ततो भीमो महाबाहुः संरम्भी दृढविक्रमः}
{अस्त्रैरस्त्राणि संवार्य शरैर्विव्याध सूतजम्}


\twolineshloka
{कर्णोऽपि भीमसेनस्य निवार्येषून्महाबलः}
{प्राहिणोन्नव नाराचानाशीविषसमान्रणे}


\twolineshloka
{तावद्भिरथ तान्भीमो व्योम्नि चिच्छेद पत्रिभिः}
{नाराचान्सूत पुत्रस्य तिष्ठतिष्ठेति चाब्रवीत्}


\twolineshloka
{ततो भीमो महाबाहुः शरं क्रुद्धान्तकोपमम्}
{मुमोचाधिरथेर्वीरो यमदण्डमिवापरम्}


\twolineshloka
{तमापतन्तं चिच्छेद राधेयः प्रहसन्निव}
{त्रिभिः शरैः शरं राजन्पाण्डवस्य प्रतापवान्}


\threelineshloka
{पुनश्चासृजदुग्राणि शरवर्षाणि पाण्डवः}
{तस्य तान्याददे कर्णः सर्वाण्यस्त्राण्यभीतवत्}
{युध्यमानस्य भीमस्य सूतपुत्रोऽस्त्रमायया}


\threelineshloka
{तस्येषुधी धनुर्ज्यां च बाणैः सन्नतपर्वभिः}
{रश्मीन्योक्त्राणि चाश्वानां क्रुद्धः कर्णोऽच्छिनन्मृधे}
{}


\twolineshloka
{तस्याश्वांश्च पुनर्हत्वा सूतं विव्याध पञ्चभिः}
{सोपसृत्य द्रुतं सूतो युधामन्यो रथं ययौ}


\twolineshloka
{विहसन्निव भीमस्य क्रुद्धः कालानलद्युतिः}
{ध्वजं चिच्छेद राधेयः पताकां च व्यपातयत्}


\twolineshloka
{स विधन्वा महाबाहुरथ शक्तिं परामृशत्}
{तां व्यवासृजदाविध्य क्रुद्धः कर्णरथं प्रति}


\twolineshloka
{तामाधिरथिरायस्तः शक्तिं काञ्चनभूषणाम्}
{आपतन्तीं महोल्काभां चिच्छेद दशभिः शरैः}


\twolineshloka
{साऽपतद्दृशधा छिन्ना कर्णस्य निशितैः शरैः}
{अस्यतः सूतपुत्रस्य मित्रार्थे चित्रयोधिनः}


\twolineshloka
{स चर्मादत्त कौन्तेयो जातरूपपरिष्कृतम्}
{खङ्गं चान्यतरप्रेप्सुर्जयं मरणमेव वा}


\twolineshloka
{तदस्य तरसा क्रुद्धो व्यधमच्चर्म सुप्रभम्}
{शरैर्बहुभिरत्युग्रैः प्रहसन्निव भारत}


\twolineshloka
{स विचर्मा महाराज विरथः क्रोधमूर्च्छितः}
{असिं प्रासृजदाविध्य त्वरन्कर्णरथं प्रति}


\twolineshloka
{स धनुः सूतपुत्रस्य सज्यं छित्त्वा महानसिः}
{पपात भुवि राजेन्द्र क्रुद्धः सर्प इवाम्बरात्}


\twolineshloka
{ततः प्रहस्याधिरथिरन्यदादाय कार्मुकम्}
{शत्रुघ्नं समरे क्रुद्धो दृढज्यं वेगवत्तरम्}


\twolineshloka
{व्यायच्छत्स शरान्कर्णः कुन्तीपुत्रजिघांसया}
{सहस्रशो महाराज रुक्मपुङ्खान्सुतेजनान्}


\twolineshloka
{स वध्यमानो बलवान्कर्णचापच्युतैः शरैः}
{वैहायसं प्राक्रमद्धै कर्णस्य रथमाविशत्}


\twolineshloka
{स तस्य चरितं दृष्ट्वा सङ्ग्रामे विजयैषिणः}
{ध्वजालयस्थो राधेयो भीमसेनमवञ्चयत्}


\twolineshloka
{तमदृष्ट्वा रथोपस्थे निलीनं व्यथितेन्द्रियम्}
{ध्वजमस्य समारुह्य तस्मै भीमो महीतले}


\twolineshloka
{तदस्य कुरवः सर्वे चाणाश्चाभ्यपूजयन्}
{यदियेष रथात्कर्णं हर्तुं तार्क्ष्य इवोरगम्}


\twolineshloka
{स च्छिन्नधन्वा विरथः स्वधर्ममनुपालयन्}
{स्वरथं पृष्ठतः कृत्वा युद्धायैव व्यवस्थितः}


\twolineshloka
{तद्विचिन्त्य स राधेयस्तत एनं समभ्ययात्}
{संरम्भात्पाण्डवं सङ्ख्ये युद्धाय समुपस्थितम्}


\twolineshloka
{तौ समेतौ महाराज स्पर्धमानौ महाबलौ}
{जीमूताविव घर्मान्ते गर्जमानौ नरर्षभौ}


\twolineshloka
{तयोरासीत्संप्रहारः क्रुद्धयोर्नरसिंहयोः}
{अमृष्यमाणयोः सङ्ख्ये देवदानवयोरिव}


\threelineshloka
{क्षीणशस्त्रस्तु कौन्तेयः कर्णेन समभिद्रुतः}
{दृष्ट्वाऽर्जुनहतान्नागान्पतितान्पर्वतोपमान्}
{रथमार्गविघातार्थं व्यायुधः प्रविवेश ह}


\twolineshloka
{हस्तिनां व्रजमासाद्य रथदुर्गं प्रविश्य च}
{पाण्डवो जीविताकाङ्क्षी राधेयं नाभ्यवर्तत}


\twolineshloka
{व्यवस्थानमथाकाङ्क्षन्धनञ्जयशरैर्हतम्}
{उद्यम्य कुञ्जरं पार्थस्तस्थौ परपुरञ्जयः}


\twolineshloka
{महौषधिसमायुक्तं हनूमानिव पर्वतम्}
{तमस्य विशिखैः कर्णो व्यधमत्कुञ्जरं पुनः}


\twolineshloka
{हस्त्यङ्गान्यथ कर्णाय प्राहिणोत्पाण्डुनन्दनः}
{चक्राण्यश्चांस्तथा चान्यद्यद्यत्पश्यति भूतले}


\twolineshloka
{तत्तदादाय चिक्षेप क्रुद्धः कर्णाय पाण्डवः}
{तदस्य सर्वं चिच्छेद क्षिप्तंक्षिप्तं शितैः शरैः}


\twolineshloka
{भीमोऽपि मुष्टिमुद्यम्य वज्रगर्भं सुदारुणम्}
{हन्तुमैच्छत्सूतपुत्रं संस्परन्नर्जुनं क्षणात्}


\twolineshloka
{शक्तोऽपि नावधीत्कर्णं समर्थः पाण्डुनन्दनः}
{रक्षमाणः प्रतिज्ञां तां या कृता सव्यसाचिना}


\twolineshloka
{तमेवं व्याकुलं भीमं भूयोभूयः शितैः शरैः}
{मूर्च्छयाभिपरीताङ्गमकरोत्सूतनन्दनः}


\twolineshloka
{व्यायुधं नावधीच्चैनं कर्णः कुन्त्या वचः स्मरन्}
{धनुषोग्रेण तं कर्णः सोऽभिद्रुत्य परामृशत्}


\twolineshloka
{धनुषा स्पृष्टमात्रेण क्रुद्धः सर्प इव श्वसन्}
{आच्छिद्य स धनुस्तस्य कर्णं मूर्धन्यताडयत्}


\twolineshloka
{ताडितो भीमसेनेन क्रोधादारक्तलोचनः}
{विहसन्निव राधेयो वाक्यमेतदुवाच ह}


\twolineshloka
{पुनः पुनस्तूबरक मूढ औदरिकेति च}
{अकृतास्त्रक मा योत्सीर्बाल सङ्ग्रामकातर}


\twolineshloka
{यत्र भोज्यं बहुविधं भक्ष्यं पेयं च पाण्डव}
{तत्र त्वं दुर्मते योग्यो न युद्धेषु कदाचन}


\twolineshloka
{मूलपुष्पफलाहारो व्रतेषु नियमेषु च}
{उचितस्त्वं वने भीम न त्वं युद्धविशारदः}


\twolineshloka
{क्व युद्वं क्व मुनित्वं च वनं गच्छ वृकोदर}
{न त्वं युद्धोचितस्तात वनवासरतिर्भवान्}


\twolineshloka
{सूदान्भृत्यजनान्दासांस्त्वं गृहे त्वरयन्भृशम्}
{योग्यस्ताडयितुं क्रोधाद्भोजनार्थं वृकोदर}


\twolineshloka
{मुनिर्भूत्वाऽथवा भीम फलान्यादत्स्व दुर्मते}
{वनाय व्रज कौन्तेय न त्वं युद्धविशारदः}


\threelineshloka
{फलमूलाशने शक्तस्त्वं तथाऽतिथिपूजने}
{न त्वां शस्त्रसमुद्योगे योग्यं मन्ये वृकोदर ॥`सञ्जय उवाच}
{}


% Check verse!
एवं तं विरथं दृष्ट्वा स्मृत्वा कर्णोऽब्रवीद्वचः'
\twolineshloka
{कौमारे यानि वृत्तानि विप्रियाणि विशाम्पते}
{तानि सर्वाणि चाप्येव रूक्षाण्यश्रावयद्भृशम्}


\twolineshloka
{अथैनं तत्र संलीनमस्पृशद्धनुषा पुनः}
{प्रहसंश्च पुनर्वाक्यं भीममाह वृषस्तदा}


\twolineshloka
{योद्धव्यं मारिषान्यत्र न योद्धव्यं च मादृशैः}
{मादृशैर्युध्यमानानामेतच्चान्यच्च विद्यते}


\threelineshloka
{गच्छ वा यत्र तौ कृष्णौ तौ त्वां रक्षिष्यतो रणे}
{गृहं वा गच्छ कौन्तेय किं ते युद्धेन बालक ॥सञ्जय उवाच}
{}


\twolineshloka
{कर्णस्य वचनं श्रुत्वा भीमसेनोऽतिदारुणम्}
{उवाच कर्णं प्रहसन्सर्वेषां शृण्वतां वचः}


\twolineshloka
{जितस्त्वमसकृद्दुष्ट कत्थसे किं वृथात्मना}
{जयाजयौ महेन्द्रस्य लोके दृष्टौ पुरातनैः}


\threelineshloka
{मल्लयुद्धं मया सार्धं कुरु दुष्कुलसम्भव}
{महाबलो महाभोगी कीचको निहतो यथा}
{तथा त्वां घातयिष्यामि पश्यत्सु सर्वराजसु}


\twolineshloka
{भीमस्य मतमाज्ञाय कर्णो बुद्धिमतां वरः}
{विरराम रणात्तस्मात्पश्यतां सर्वधन्विनाम्}


\twolineshloka
{एवं तं विरथं कृत्वा कर्णो राजन्व्यकत्थयत्}
{प्रसुखे वृष्णिसिंहस्य पार्थस्य च महात्मनः}


\twolineshloka
{तं ब्रुवाणं तु भीमः स काङ्क्षन्भीमपराक्रमः}
{न चाकार्षीन्मृतं स्मृत्वा ह्यर्जुनस्य महाबलम्}


% Check verse!
तस्य कार्मुकमारुज्य बभञ्जाशुपराक्रमी
\threelineshloka
{ततो दृष्ट्वा महाराज वासुदेवो महाद्युतिः}
{अर्जुनार्जुन पश्येमं भीमं कर्णेन बाधितम् ॥सञ्जय उवाच}
{}


\threelineshloka
{एवमुक्तस्तदा पार्थः केशवेन महात्मना}
{भीमसेनं तथाभूतं क्रोधसंरक्तलोचनः}
{अमर्षवशमापन्नो निर्दहन्निव चक्षुषा'}


\twolineshloka
{ततो राजञ्शिलाधौताञ्शराञ्शाखामृगध्वजः}
{प्राहिणोत्सूतपुत्राय केशवेन प्रचोदितः}


\twolineshloka
{ततः पार्थभुजोत्सृष्टाः शराः कनकभूषणाः}
{गाण्डीवप्रभवाः कर्णं हंसाः क्रौञ्चमिवाविशन्}


\twolineshloka
{स भुजङ्गैरिवाविष्टैर्गाण्डीवप्रेषितैः शरैः}
{भीमसेनादपासेधत्सूतपुत्रं धनञ्जयः}


\twolineshloka
{स च्छिन्नधन्वा भीमेन धनञ्जयशराहतः}
{कर्णो भीमादपायासीद्रथेन महता द्रुतम्}


\twolineshloka
{भीमोऽपि सात्यकेर्वाहं समारुह्य नरर्षभः}
{अन्वयाद्धातरं सङ्ख्ये पाण्डवं सव्यसाचिनम्}


\twolineshloka
{ततः कर्णं समुद्दिश्य त्वरमाणो धनञ्चयः}
{नाराचं क्रोधताम्राक्षः प्रैषीन्मृत्युमिवान्तकः}


\twolineshloka
{स गरुत्मानिवाकशे प्रार्थयन्भुजगोत्तमम्}
{नाराचोऽभ्यपतत्कर्णं तूर्णं गाण्डीवचोदितः}


\twolineshloka
{तमन्तरिक्षे नाराचं द्रौणिश्चिच्छेद पत्रिणा}
{धनञ्जयभयात्कर्णमुज्जिहीर्षन्महारथः}


\twolineshloka
{ततो द्रौणिं चतुःषष्ट्या विव्याध कुपितोऽर्जुनः}
{शिलीमुखैर्महाराज मागास्तिष्ठेति चाब्रवीत्}


\twolineshloka
{स तु मत्तगजाकीर्णमनीकं रथसङ्कुलम्}
{तूर्णमभ्याविशद्द्रौणिर्धनञ्जयशरार्दितः}


\twolineshloka
{ततः सुवर्णपृष्ठानां चापानां कूजतां रणे}
{शब्दं गाण्डीवघोषेण कौन्तेयोऽभ्यभवद्बली}


\twolineshloka
{धनञ्जयस्तथाऽऽयान्तं पृष्ठतो द्रौणिमभ्यगात्}
{नातिदीर्घमिवाध्वानं शरैः सन्त्रासयन्बलम्}


\twolineshloka
{विदार्य देहान्नाराचैर्नरवारणवाजिनाम्}
{कङ्कबर्हिणवासोभिर्बलं व्यधमदर्जुनः}


\twolineshloka
{तद्बलं भरतश्रेष्ठ सवाजिद्विपमानवम्}
{पाकशासनिरायत्तः पार्थः स निजघान ह}


\chapter{अध्यायः १४०}
\twolineshloka
{धृतराष्ट्र उवाच}
{}


\twolineshloka
{`दातव्यमद्य मन्येऽहं पाण्डवानां स्वकं पुनः}
{न विग्रहो हि बलिना श्रेयसे स्याद्यथातथा}


\twolineshloka
{पादयोः प्रणतेनापि भुक्त्वाऽप्युच्छिष्टमप्यरेः}
{अतोऽन्यद्वापि कृत्वैव जीव्यं लोके नरेण वै}


\twolineshloka
{जीवतैव परो लोकः साध्यते चैव सर्वथा}
{अजीवतस्तथैवासीन्न सुखं न परा गतिः}


\twolineshloka
{विनाशे सर्वथोत्पन्ने न बालो बुध्यते क्रियाम्}
{मिथ्याभिमानदग्धो हि न बुद्ध्येत कृताकृते'}


\twolineshloka
{अहन्यहनि मे दीप्तं यशः पतति सञ्जय}
{हता मे बहवो योधा मन्ये कालस्य पर्ययम्}


\twolineshloka
{धनञ्जयः सुसङ्क्रद्धः प्रविष्टो मामकं बलम्}
{रक्षितं द्रौणिकर्णाभ्यामप्रवेश्यं सुरैरपि}


\twolineshloka
{ताभ्यामूर्जितवीर्याभ्यामाप्यायितपराक्रमः}
{सहितः कृष्णभीमाभ्यां शिनीनामृषभेण च}


\twolineshloka
{तदाप्रभृति मां शोको दहत्यग्निरिवाशयम्}
{ग्रस्तानिव प्रपश्यामि भूमिपालान्ससैन्धवान्}


\threelineshloka
{अप्रियं सुमहत्कृत्वा सिन्धुराजः किरीटिनः}
{`क्रुद्धस्य देवराजस्य शक्रस्येव महाद्युतेः'}
{चक्षुर्विषयमापन्नः कथं जीवितमाप्नुयात्}


\twolineshloka
{अनुमानाच्च पश्यामि नास्ति सञ्जय सैन्धवः}
{युद्धं तु तद्यथावृत्तं तन्ममाचक्ष्व तत्त्वतः}


\twolineshloka
{यच्च विक्षोभ्य महतीं सेनामालोड्य चासकृत्}
{एकः प्रविष्टः सङ्क्रुद्धो नलिनीमिव कुञ्जरः}


\threelineshloka
{तस्य मे वृष्णिवीरस्य ब्रूहि युद्धं यथातथम्}
{धनञ्जयार्थे यत्तस्य कुशलो ह्यसि सञ्जय ॥सञ्जय उवाच}
{}


\twolineshloka
{तथा तु वैकर्त नपीडितं तंभीमं प्रयान्तं पुरुषप्रवीरम्}
{समीक्ष्य राजन्नरवीरमध्येशिनिप्रवीरोऽनुययौ रथेन}


\twolineshloka
{नदन्यथा वज्रधरस्तपान्तेज्वलन्यथा जलदान्ते च सूर्यः}
{निघ्नन्नमित्रान्धनुषा दृढेनस कम्पयंस्तव पुत्रस्य सेनाम्}


\twolineshloka
{तं यान्तमश्वै रजतप्रकाशै--रायोधने वीरतरं नदन्तम्}
{नाशक्नुवन्वारयितुं त्वदीयाःसर्वे रथा भारत माधवाग्र्यम्}


\twolineshloka
{अमर्षपूर्णस्त्वनिवृत्तयोधीशरासनी काञ्चनवर्मधारी}
{अलम्बुसः सात्यकिं माधवाग्र्य--मवारयद्राजवरोऽभिपत्य}


\twolineshloka
{तयोरभूद्भारत सम्प्रहारोयथाविधो नैव बभूव कश्चित्}
{प्रेक्षन्त एवाहवशोभिनौ तौयोधास्त्वदीयाश्च परे च सर्वे}


\twolineshloka
{आविध्यदेनं दशभिः पृषत्कै--रलम्बुसो राजवरः प्रसह्य}
{अनागतानेव तु तान्पृषत्कां--श्चिच्छेद बाणैः शिनिपुङ्गवोऽपि}


\twolineshloka
{पुनः स बाणैस्त्रिभिरग्निकल्पै--राकर्णपूर्णैर्निशितैः सपुङ्खैः}
{विव्याध देहावरणं विदार्यते सात्यकेराविविशुः शरीरम्}


\twolineshloka
{तैः कायमस्याग्न्यनिलप्रभावै--र्विदार्य बाणैर्निशितैर्ज्वलद्भिः}
{आजघ्निवांस्तान्रजतप्रकाशा--नश्वांश्चतुर्भिश्चतुरः प्रसह्य}


\twolineshloka
{तथा तु तेनाभिहतस्तरस्वीनप्ता शिनेश्चक्रधरप्रभावः}
{अलम्बुसस्योत्तमवेगवद्भि--रश्वांशअचतुर्भिर्निजघान बाणैः}


\twolineshloka
{अथास्य सूतस्य शिरो निकृत्यभल्लेन कालानलसन्निभेन}
{सकुण्डलं पूर्मशशिप्रकाशंभ्राजिष्णु वक्त्रं निचकर्त देहात्}


\twolineshloka
{निहत्य तं पार्थिवपुत्रपौत्रंसङ्ख्ये यदूनामृषभः प्रमाथी}
{ततोऽन्वयादर्जुनमेव वीरःसैन्यानि राजंस्तव सन्निवार्य}


\twolineshloka
{अन्वागतं वृष्णिवीरं समीक्ष्यतथारिमध्ये परिवर्तमानम्}
{घ्नन्तं कुरूणामिषुभिर्बलानिपुनःपुनर्वायुमिवाभ्रपूगान्}


\twolineshloka
{ततोऽवहन्सैन्धवाः साधुदान्तागोक्षीरकुन्देन्दुहिमप्रकाशाः}
{सुवर्णजालावतताः सदश्वायतोयतः कामयते नृसिंहः}


\twolineshloka
{अथात्मजास्ते सहिताऽभिपेतु--रन्ये च योधास्त्वरितास्त्वदीयाः}
{कृत्वा मुखं भारतयोधमुख्यंदुःशासनं त्वत्सुतमाजमीढ}


\twolineshloka
{ते सर्वतः सम्परिवार्य सङ्ख्येशैनेयमाजघ्नुरनीकसाहाः}
{स चापि तान्प्रवरः सात्वतानांन्यवारयद्बाणजालेन वीरः}


\twolineshloka
{निवार्य तांस्तूर्णममित्रघातीनप्ता शिनेः पत्रिभिरग्निकल्पैः}
{दुःशासनस्याभिजघान वाहा--ञ्जवीयसस्तन्मनसोऽपि बाणैः}


% Check verse!
ततोऽर्जुनो हर्षमवाप सङ्ख्येकृष्णश्च दृष्ट्वा पुरुषप्रवीरम्
\chapter{अध्यायः १४१}
\twolineshloka
{सञ्जय उवाच}
{}


\twolineshloka
{तमुद्यतं महाबाहुं दुःशासनरथं प्रति}
{त्वरितं त्वरणीयेषु धनञ्जयजयैषिणम्}


\twolineshloka
{त्रिगर्तानां महेष्वासाः सुवर्णविकृतध्वजाः}
{सेनासमुद्रमाविष्टमनन्तं पर्यवारयन्}


\twolineshloka
{अथैनं रथवंशेन सर्वतः सन्निवार्य ते}
{अवाकिरञ्छरव्रातैः क्रुद्धाः परमधन्विनः}


\twolineshloka
{अजयद्राजपुत्रांस्तान्भ्राजमानान्महारणे}
{एकः पञ्चाशतं शत्रून्सात्यकिः सत्यविक्रमः}


\twolineshloka
{सम्प्राप्य भारतीमध्यं तलघोषसमाकुलम्}
{असिशक्तिगदापूर्णमप्लुवं सागरं यथा}


\threelineshloka
{`अथैनमनुवृत्तास्तु त्रिगर्ताः सहिताः पुनः}
{तीव्रेण रथवंशेन महता पर्यवारयन्}
{विकर्षन्तोऽतिमात्राणि चापानि भरतर्षभ'}


\twolineshloka
{तत्राद्भुतमपश्याम शैनेयचरितं रणे}
{प्रतीच्यां दिशि तं दृष्ट्वा प्राच्यां पश्यामि लाघवात्}


\twolineshloka
{उदीचीं दक्षिणां प्राचीं प्रतीचीं विदिशस्तथा}
{पुनर्मध्यगतो वीर आहवे युद्धदुर्मदः}


\twolineshloka
{एकः पर्यचरन्रङ्गे बहुधा स महाबलः}
{नृत्यन्निवाचरच्छूरो यथा रथशतं तथा}


\twolineshloka
{तद्दृष्ट्वा चरितं तस्य सिंहविक्रान्तगामिनः}
{त्रिगर्ताः सन्न्यवर्तन्त सन्तप्ताः स्वजनं प्रति}


\twolineshloka
{तमन्ये शूरसेनानां शूराः सङ्ख्ये न्यवारयन्}
{नियच्छन्तः शरव्रातैर्मत्तं द्विपमिवाङ्कुशैः}


\twolineshloka
{तैर्व्यवाहरदार्यात्मा मुहूर्तादेव सात्यकिः}
{ततः कलिङ्गैर्युयुधे सोऽचिन्त्यबलविक्रमः}


\twolineshloka
{तां च सेनामतिक्रम्य कलिङ्गानां दुरत्ययाम्}
{अथ पार्थं महाबाहुर्धनञ्जयमुपासदत्}


\twolineshloka
{तरन्निव जले श्रान्तो यथा स्थलमुपेयिवान्}
{तं दृष्ट्वा पुरुषव्याघ्रं युयुधानः समाश्वसत्}


\twolineshloka
{तमायान्तमभिप्रेक्ष्य केशवः पार्थमब्रवीत्}
{असावायाति शैनेयस्तव पार्थ पदानुगः}


\twolineshloka
{एष शिष्यः सखा चैव तव सत्यपराक्रमः}
{सर्वान्योधांस्तृणीकृत्य विजिग्ये पुरुषर्षभः}


\twolineshloka
{एष कौरवयोधानां कृत्वा घोरमुपद्रवम्}
{तव प्राणैः प्रियतमः किरीटिन्नेति सात्यकिः}


\twolineshloka
{एष द्रोणं तथा भोजं कृतवर्माणमेव च}
{कदर्थीकृत्य विशिखैः फल्गुनाभ्येति सात्यकिः}


\twolineshloka
{धर्मराजप्रियान्वेषी हत्वा योधान्वरान्वरान्}
{शूरश्चैव कृतास्त्रश्च फल्गुनाभ्येति सात्यकिः}


\twolineshloka
{कृत्वा सुदुष्करं कर्म सैन्यमध्ये महाबलः}
{तव दर्शनमन्विच्छन्पाण्डवाभ्येति सात्यकिः}


\twolineshloka
{बहूनेकरथेनाजौ योधयित्वा महारथान्}
{आचार्यप्रमुखान्पार्थ प्रयात्येष स सात्यकिः}


\twolineshloka
{स्वबाहुबलमाश्रित्य विदार्य च वरूथिनीम्}
{प्रेषितो धर्मराजेन पार्थैषोऽभ्येति सात्यकिः}


\twolineshloka
{प्रियः शिष्यश्च ते पार्थ त्वया तुल्यपराक्रमः}
{विद्राव्यं महतीं सेनामेष ह्यायाति सात्यकिः}


\twolineshloka
{यस्य नास्ति समो योधः कौरवेषु कथञ्चन}
{सोऽयमायाति कौन्तेय सात्यकिर्युद्धदुर्मदः}


\twolineshloka
{कुरुसैन्याद्विमुक्तो वै सिंहो मध्याद्गवामिव}
{निहत्य बहुलाः सेनाः पार्थैषोऽब्येति सात्यकिः}


\twolineshloka
{एष राजसहस्राणां वक्त्रैः पङ्कजसन्निभैः}
{आस्तीर्य वसुधां पार्थ क्षिप्रमायाति सात्यकिः}


\twolineshloka
{एष दुर्योधनं जित्वा भ्रातृभिः सहितं रणे}
{निहत्य जलसन्धं च क्षिप्रमायाति सात्यकिः}


\threelineshloka
{रुधिरौघवतीं कृत्वा नदीं शोणितकर्दमाम्}
{तृणवद्व्यस्य कौरव्यानेष ह्यायाति सात्यकिः ॥सञ्जय उवाच}
{}


\twolineshloka
{ततः प्रहृष्टः कौन्तेयः केशवं वाक्यमब्रवीत्}
{न मे प्रियं महाबाहो यन्मामभ्येति सात्यकिः}


\twolineshloka
{न हि जानामि वृत्तान्तं धर्मराजस्य केशव}
{सात्वतेन विहीनः स यदि जीवति वा न वा}


\twolineshloka
{एतेन हि महाबाहो रक्षितव्यः स पार्थिवः}
{तमेष कथमुत्सृज्य मम कृष्ण पदानुगः}


\twolineshloka
{राजा द्रोणाय चोत्सृष्टः सैन्धवश्चानिपातितः}
{प्रत्युद्याति च शैनेयमेष भूरिश्रवा रणे}


\twolineshloka
{सोऽयं गुरुतरो भारः सैन्धवार्थे समाहितः}
{ज्ञातव्यश्च हि मे राजा रक्षितव्यश्च सात्यकिः}


\twolineshloka
{जयद्रथश्च हन्तव्यो लम्बते च दिवाकरः}
{श्रान्तश्चैष महाबाहुरल्पप्राणश्च साम्प्रतम्}


\twolineshloka
{परिश्रान्ता हयाश्चास्य हययन्ता च माधव}
{न च भूरिश्रवाः श्रान्तः ससहायश्च केशव}


\threelineshloka
{अपीदानीं भवेदस्य क्षेममस्मिन्समागमे}
{कच्चिन्न सागरं तीर्त्वा सात्यकिः सत्यविक्रमः}
{गोष्पदं प्राप्य सीदेत महौजाः शिनिपुङ्गवः}


\twolineshloka
{अपि कौरवमुख्येन कृतास्त्रेण महात्मना}
{समेत्य भूरिश्रवसा स्वस्तिमान्सात्यकिर्भवेत्}


\twolineshloka
{व्यतिक्रममिमं मन्ये धर्मराजस्य केशव}
{आचार्याद्भयमुत्सृज्य यः प्रैषयत सात्यकिम्}


\twolineshloka
{ग्रहणं धर्मराजस्य खगः श्येन इवामिषम्}
{नित्यमाशंसते द्रोणः कच्चित्स्यात्कुशली नृपः}


\chapter{अध्यायः १४२}
\twolineshloka
{सञ्जय उवाच}
{}


\twolineshloka
{तमापतन्तं सम्प्रेक्ष्य सात्वतं युद्धदुर्मदम्}
{क्रोधाद्भूरिश्रवा राजन्सहसा समुपाद्रवत्}


\twolineshloka
{तमब्रवीन्महाराज कौरव्यः शिनिपुङ्गवम्}
{अद्य प्राप्तोऽपि दिष्ट्या मे चक्षुर्विषयमित्युत}


\twolineshloka
{चिराभिलषितं काममहं प्राप्स्यामि संयुगे}
{न हि मे मोक्ष्यसे जीवन्यदि नोत्सृजसे रणम्}


\twolineshloka
{अद्य त्वां समरे हत्वा नित्यं शूराभिमानिनम्}
{नन्दयिष्यामि दाशार्ह कुरुराजं सुयोधनम्}


\twolineshloka
{अद्य मद्बाणनिर्दग्धं पतितं धरणीतले}
{द्रक्ष्यतस्त्वां रणे वीरौ सहितौ केशवार्जुनौ}


\twolineshloka
{अद्य धर्मसुतो राजा श्रुत्वा त्वां निहतं मया}
{सव्रीडो भविता सद्यो येनासीह प्रवेशितः}


\twolineshloka
{अद्य मे विक्रमं पार्थो विज्ञास्यति धनञ्जयः}
{त्वयि भूमौ विनिहते शयाने रुधिरोक्षिते}


\twolineshloka
{चिराभिलषितो ह्येष त्वया सह समागमः}
{पुरा देवासुरे युद्धे शक्रस्य बलिना यथा}


\twolineshloka
{अद्य युद्धं महाघोरं तव दास्यामि सात्वत}
{ततो ज्ञास्यसि तत्त्वेन मद्वीर्यबलपौरुषम्}


\twolineshloka
{अद्य संयमनीं याता मया त्वं निहतो रणे}
{यथा रामानुजेनाजौ रावणिर्लक्ष्मणेन ह}


\twolineshloka
{अद्य कृष्णश्च पार्थश्च धर्मराजश्च माधव}
{हते त्वयि निरुत्साहारणं त्यक्ष्यन्त्यसंशयम्}


\twolineshloka
{अद्य तेऽपचितिं कृत्वा शितैर्माधव सायकैः}
{तत्स्त्रियो नन्दयिष्यामि ये त्वया निहता रणे}


\threelineshloka
{मच्चक्षुर्विषये प्राप्तो न त्वं माधव मोक्ष्यसे}
{सिंहस्य विषयं प्राप्तो यथा क्षुद्रमृगस्तथा ॥सञ्जय उवाच}
{}


\twolineshloka
{युयुधानस्तु तं राजन्प्रत्युवाच हसन्निव}
{कौरवेय न सन्त्रासो विद्यते मम संयुगे}


\twolineshloka
{नाहं भीषयितुं शक्यो वाङ्मात्रेण तु केवलम्}
{स मां निहन्यात्सङ्ग्रामे यो मां कुर्यान्निरायुधम्}


\twolineshloka
{समास्तु शाश्वतीर्हन्याद्यो मां हन्याद्धि संयुगे}
{किं वृथोक्तेन बहुना कर्मणा तत्समाचर}


\twolineshloka
{शारदस्येव मेघस्य गर्जितं निष्फलं हिते}
{श्रुत्वा त्वद्गर्जितं वीर हास्यं हि मम जायते}


\fourlineindentedshloka
{चिरकालेप्सितं लोके युद्धमद्यास्तु कौरव}
{त्वरते मे मतिस्तात तव युद्धाभिकाङ्क्षिणी}
{नाहत्वाऽहं निवर्तिष्ये त्वामद्य पुरुषाधम ॥सञ्जय उवाच}
{}


\twolineshloka
{अन्योन्यं तौ तथा वाग्भिस्तक्षन्तौ नरपुङ्गवौ}
{जिघांसू परमक्रुद्धभिजघ्नतुराहवे}


\twolineshloka
{समेतौ तौ महेष्वासौ शुष्मिणौ स्पर्धिनौ रणे}
{द्विरदाविव सङ्क्रुद्धौ वासितार्थे मदोत्कटौ}


\twolineshloka
{भूरिश्रवाः सात्यकिश्च ववर्षतुररिन्दमौ}
{शरवर्षाणि घोराणि मेघाविव परस्परम्}


\twolineshloka
{सौमदत्तिस्तु शैनेयं प्रच्छाद्येषुभिराशुगैः}
{जिघांसुर्भरतश्रेष्ठ विव्याध निशितैः शरैः}


\twolineshloka
{दशभिः सात्यकिं विद्ध्वा सौमदत्तिरथापरान्}
{मुमोच निशितान्बाणाञ्जिघांसुः शिनिपुङ्गवं}


\twolineshloka
{तानस्य विशिखांस्तीक्ष्णानन्तरिक्षे विशाम्पते}
{अप्राप्तानस्त्रमायाभिरग्रसत्सात्यकिः प्रभो}


\twolineshloka
{तौ पृथक्शस्त्रवर्षाभ्यामवर्षेतां परस्परम्}
{उत्तमाभिजनौ वीरौ कुरुवृष्मियशस्करौ}


\twolineshloka
{तौ नखैरिव शार्दूलौ दन्तैरिव महाद्विपौ}
{रथशक्तिभिरन्योन्यं विशिखैश्चाप्यकृन्तताम्}


\twolineshloka
{निर्भिदन्तौ हि गात्राणि विक्षरन्तौ च शोणितम्}
{व्यष्टम्भयेतामन्योन्यं प्राणद्यूताभिदेविनौ}


\twolineshloka
{एवमुत्तमकर्माणौ कुरुवृष्णियशस्करौ}
{परस्परमयुध्येतां वारणाविव यूथपौ}


\twolineshloka
{तावदीर्घेण कालेन ब्रह्मलोकपुरस्कृतौ}
{यियासन्तौ परं स्थानमन्योन्यं सञ्जगर्जतुः}


\twolineshloka
{सात्यकिः सौमदत्तिश्च शरवृष्ट्या परस्परम्}
{हृष्टवद्धार्तराष्ट्राणां पश्यतामभ्यवर्षताम्}


\twolineshloka
{सम्प्रैक्षन्त जनास्तौ तु युध्यमानौ युधाम्पती}
{यूथपौ वासिताहेतोः प्रयुद्धाविव कुञ्जरौ}


\twolineshloka
{`भूयोभूयः शरै राजंस्तक्षन्तौ क्रोधमूर्च्छितौ}
{अयुध्येतां महारङ्गे वने केसरिणाविव}


\twolineshloka
{मर्मज्ञाविव सङ्क्रुद्धौ जिघांसन्तौ जगर्जतुः}
{विमर्दन्तावथान्योन्यं बलवज्रधराविव}


\twolineshloka
{अथान्योन्यं पताकाश्च रथोपकरणानि च}
{सञ्चिच्छिदतुरायान्तौ बाणैः सन्नतपर्वभिः}


\twolineshloka
{पुनश्च शरवर्षाभ्यामन्योन्यमभिवर्षताम्}
{उभौ तु जघ्नतुस्तूर्णमितरेतरसारथी'}


\twolineshloka
{अन्योन्यस्य हयान्हत्वा धनुषी विनिकृत्य च}
{विरथावसियुद्धाय समेयातां महारणे}


\twolineshloka
{आर्षभे चर्मणी चित्रे प्रगृह्य विपुले शुभे}
{विकोशौ चाप्यसी कृत्वा समरे तौ विचेरतुः}


\threelineshloka
{चरन्तौ विविधान्मार्गान्मण्डलानि च भागशः}
{मुहुराजघ्नतुः क्रुद्धावन्योन्यमरिमर्दनौ}
{सखङ्गौ चित्रवर्माणौ सनिष्काङ्गदभूषणौ}


\twolineshloka
{भ्रान्तमुद्धान्तमाविद्धमाप्लुतं विप्लुतं सृतम्}
{सम्पातं समुदीर्णं च दर्शयन्तौ यशस्विनौ}


\twolineshloka
{असिभ्यां सम्प्रजहाते परस्परमरिन्दमौ}
{उभौ छिद्रैषिणौ वीरावुभौ चित्रं ववल्गतुः}


\twolineshloka
{दर्शयन्तावुभौ शिक्षां लाघवं सौष्ठवं तथा}
{रणे रणकृतां श्रेष्ठावन्योन्यं पर्यकर्षताम्}


\twolineshloka
{मुहूर्तमिव राजेन्द्र समाहत्य परस्परम्}
{पश्यतां सर्वसैन्यानां वीरावाश्वसतां पुनः}


\twolineshloka
{असिभ्यां चर्मणी चित्रे शतचन्द्रे नराधिप}
{निकृत्य पुरुषव्याघ्रौ बाहुयुद्धं प्रचक्रतुः}


\twolineshloka
{व्यूढोरस्कौ दीर्घभुजौ नियुद्धकुशलावुभौ}
{बाहुभिः समसज्जेतामायसैः परिधैरिव}


\twolineshloka
{तयो राजन्भुजाघातनिग्रहप्रग्रहास्तथा}
{शिक्षाबलसमुद्भूताः सर्वयोधप्रहर्षणाः}


\twolineshloka
{तयोर्नृवरयो राजन्समरे युध्यमानयोः}
{भीमोऽभवन्महाशब्दो वज्रपर्वतयोरिव}


\twolineshloka
{द्विपाविव विषाणाग्रैः शृङ्गैरिव महर्षभौ}
{भुजयोक्रावबन्धैश्च शिरोभ्यां चावघातनैः}


\twolineshloka
{पादावकर्षसन्धानैस्तोमराङ्कशलासनैः}
{पादोदरविबन्धैश्च भूमावुद्धमणैस्तथा}


\twolineshloka
{गतप्रत्यागताक्षेपैः पातनोत्थानसम्पुतैः}
{युयुधाते महात्मानौ कुरुसात्वतपुङ्गवौ}


\twolineshloka
{द्वात्रिंशत्कारणानि स्युर्यानि युद्धानि भारत}
{तान्यदर्शयतां तत्र युध्यमानौ महाबलौ}


\twolineshloka
{क्षीणायुधे सात्वते युध्यमानेततोऽब्रवीदर्जुनं वासुदेवः}
{पश्यस्वैनं विरथं युध्यमानंरणे वरं सर्वधनुर्धराणाम्}


\twolineshloka
{`सिन्धुराजवधे सक्तं पार्थं कृष्णोऽब्रवीत्पुनः}
{सीदन्तं सात्यकिं पश्य पार्थैनं परिरक्ष च'}


\twolineshloka
{प्रविष्टो भारतीं भित्त्वा तव पाण्डव पृष्ठतः}
{योधितश्च महावीर्यैः सर्वैर्भारत भारतैः}


\twolineshloka
{धार्तराष्ट्राश्च ये मुख्या ये च मुख्या महारथाः}
{निहता वृष्णिवीरेण शतशोऽथ सहस्रशः}


\twolineshloka
{परिश्रान्तं युधां श्रेष्ठं सम्प्राप्तो भूरिदक्षिणः}
{युद्धाकाङ्क्षी समायान्तं नैतत्सममिवार्जुन}


\twolineshloka
{ततो भूरिश्रवाः क्रुद्धः सात्यकिं युद्धदुर्मदः}
{उद्यम्याभ्याहनद्राजन्मत्तो मत्तमिव द्विपम्}


\threelineshloka
{ततो जलदनिर्घोषः समीपे नृपसत्तम}
{हाहाकारो महानासीत्सैन्यानां भरतर्षभ}
{यदुद्यम्य महाबाहुः सात्यकिं न्यहनद्भुवि}


\twolineshloka
{रथस्थयोर्द्वयोर्युद्धे क्रुद्धयोर्योधमुख्ययोः}
{केशवार्युनयो राजन्समरे प्रेक्षमाणयोः}


\twolineshloka
{अथ कृष्णो महाबाहुरर्जुनं प्रत्यभाषत}
{पश्य वृष्ण्यन्धकव्याघ्रं सौमदत्तिवशं गतम्}


\twolineshloka
{परिश्रान्तं गतं भूमौ कृत्वा कर्म सुदुष्करम्}
{तवान्तेवासिनं वीरं पालयार्जुन सात्यकिम्}


\threelineshloka
{न वशं यज्ञशीलस्य गच्छेदेष वरोऽर्जुन}
{त्वत्कृते पुरुषव्याघ्र तदाऽऽशु क्रियतां विभो ॥सञ्जय उवाच}
{}


\twolineshloka
{स सिंह इव मातङ्गं विकर्षन्भूरिदक्षिणः}
{व्यरोचत कुरुश्रेष्ठः सात्वतप्रवरं युधि}


\threelineshloka
{अथाब्रवीद्धृष्टमना वासुदेवं धनञ्जयः}
{पश्य वृष्णिप्रवीरेण क्रीडन्तं कुरुपुङ्गवम्}
{महाद्विपेनेव वने मत्तेन हरियूथपम्}


\twolineshloka
{`मामेव च महाबाहो परियान्ति महारथाः}
{यथाशक्ति यतन्तो मां योधयन्तो जनार्दन}


\twolineshloka
{ध्रुवं च योधयाम्येताञ्छिद्रान्वेषणतत्परान्}
{रक्षामि सात्यकिं चैव सौमदत्तिवशं गतम्}


\twolineshloka
{अप्राप्तोऽयं मया कृष्म हन्तुं भूरिश्रवा रणे}
{अन्येन तु समासक्तं मम नोत्सहते मनः}


\twolineshloka
{अवश्यं च मया कृष्ण वृष्णिवीरस्य रक्षणम्}
{मदर्थं युध्यमानस्य कार्यं प्राणैरपि प्रभो}


\threelineshloka
{अधर्मो वाऽस्तु धर्मो वा मम माधव माधवः}
{परेण निहतो मा स्म प्राणान्हासीन्महारथः ॥सञ्जय उवाच}
{}


\twolineshloka
{एवमुक्त्वाऽर्जुनः कृष्णं परानाशु शितैः शरैः}
{छादयामास सङ्क्रुद्धः परे चापि धनञ्जयम्}


\twolineshloka
{एवं स्म युध्यते वीरः सात्यकिं च मुहुर्मुहुः}
{प्रेक्षते स्म नरव्याघ्रो भूरिश्रवसमेव च'}


\twolineshloka
{तथा तु कृष्यमाणं तं दृष्ट्वा सात्यकिमाहवे}
{वासुदेवस्तदा वाक्यं भूयोऽप्यर्जुनमब्रवीत्}


\twolineshloka
{पश्य वृष्ण्यन्धकव्याघ्रं सौमदत्तिवंशं गतम्}
{तव शिष्यं महाबाहो धनुष्यनवमं त्वया}


\twolineshloka
{अनित्यो विक्रमः पार्थ यत्र भूरिश्रवा रणे}
{विशेषयति वार्ष्णेयं सात्यकिं सत्यविक्रमम्}


\threelineshloka
{`बहुभिर्महारथैरेष पराक्रान्तैर्युयुत्सुभिः}
{युद्ध्वा भृशं परिश्रान्तः क्षीणायुधरपरिच्छदः' ॥सञ्जय उवाच}
{}


\twolineshloka
{एवमुक्तो महाराज वासुदेवेन पाण्डवः}
{मनसा पूजयामास भूरिश्रवसमाहवे}


\twolineshloka
{विकर्षन्सात्वतां श्रेष्ठं क्रीडमान इवाहवे}
{स हर्षयति मां भूयः कुरूणां नन्दिवर्धनः}


\twolineshloka
{प्रवरं वृष्णिवीराणां यन्निहन्त्येष सात्यकिम्}
{महाद्विपमिवारण्ये कर्षन्निव हरिर्भृशम्}


\twolineshloka
{एवं तु मनसा राजन्पार्थः सम्पूज्य कौरवम्}
{`अयुध्यतारिभिर्वीरस्तं च सम्प्रेक्षते मुहुः'}


\threelineshloka
{अथ कोशाद्विनिष्कृष्य खङ्गं भूरिश्रवाः शितम्}
{मूर्धजेषु च जग्राह पदा चोरस्याताडयत्}
{`आक्रम्य चाप्यथोद्यम्य सहासिं सुभुजो भुजम्}


\twolineshloka
{शुशुभे स भुजस्तस्य तपनीयविभूषणः}
{मध्ये रथसमूहस्य इन्द्रध्वज इवोच्छ्रितः}


\twolineshloka
{हाहाकृतमभूत्सर्वं पाण्डवानां महद्बलम्}
{तावकाश्च मुदा युक्ताः सिंहनादं विचुक्रुशुः}


\twolineshloka
{निमीलिताक्षास्त्वभवञ्जनाः सङ्ग्रामभीरवः}
{तथा भूरिश्रवोग्रस्ते सात्वते नष्टविक्रमे}


\twolineshloka
{वासुदेवं महाबाहुरर्जुनः प्रत्यभाषत}
{सैन्धवे सक्तदृष्टित्वान्न तं पश्यामि माधवम्}


\fourlineindentedshloka
{एष त्वसुकरं कर्म यादवार्थे करोम्यहम्}
{मम शिष्यो ममार्थाय युध्यते मम शत्रुभिः}
{तं कृष्ण मोक्षयिष्यामि दावात्सिंहशिशुं यथा' ॥सञ्जय उवाच}
{}


\twolineshloka
{इत्युक्त्वा वचनं कुर्वन्वासुदेवस्य पाण्डवः}
{ततः क्षुरप्रं निशितं गाण्डीवे समयोजयत्}


\twolineshloka
{पार्थबाहुविसृष्टः स महोल्केव नभश्च्युता}
{सखङ्गं यज्ञशीलस्य बाहुं दक्षिणमच्छिनत्}


\chapter{अध्यायः १४३}
\twolineshloka
{सञ्जय उवाच}
{}


\threelineshloka
{स बाहुर्न्यपतद्भूमौ सखङ्गः सशुभाङ्गदः}
{आदधज्जीवलोकस्य दुःखमद्भुतमुत्तमः}
{`यन्त्रमुक्तो महेन्द्रस्य ध्वजो वृत्तोत्सवो यथा'}


\twolineshloka
{प्रहरिष्यन्हृतो बाहुरदृश्यते किरीटिना}
{वेगेन न्यपतद्भूमौ पञ्चास्य इव पन्नगः}


\twolineshloka
{स मोघं कृतमात्मानं दृष्ट्वा पार्थेन कौरवः}
{उत्सृज्य सात्यकिं क्रोधाद्ग्रर्हयामास पाण्डवम्}


\fourlineindentedshloka
{`स विबाहुर्महाराज एकपक्ष इवाण्डजः}
{एकचक्रो रथो यद्वद्धरणीमास्थितो नृपः}
{उवाच पाण्डवं चैव सर्वक्षत्रस्य पश्यतः ॥भूरिश्रवा उवाच}
{}


\twolineshloka
{नृशंसं बत कौन्तेय कर्मेदं कृतवानसि}
{अपश्यतो विषक्तस्य यन्मे त्वं बाहुमच्छिनः}


\twolineshloka
{येषु येषु नरः पार्थ वर्तते सुसमाहितः}
{आशु तच्छीलतामेति तदिदं दृश्यते त्वयि}


% Check verse!
किं नु वक्ष्यसि राजानं धर्मपुत्रं युधिष्ठिरम्किं कुर्वाणो मया सङ्ख्ये हतो भूरिश्रवा इति
\twolineshloka
{इदमिन्द्रेण ते साक्षादुपदिष्टं महात्मना}
{अस्त्रं रुद्रेण वा पार्थ द्रोणेनाथ कृपेण वा}


\twolineshloka
{ननु नामास्त्रधर्मज्ञस्त्वं लोकेऽभ्यधिकः परैः}
{सोऽयुध्यमानस्य कथं रणे प्रहृतवानसि}


\twolineshloka
{न प्रमत्ताय भीताय विरथाय प्रयाचते}
{व्यसने वर्तमानाय प्रहरन्ति मनीषिणः}


\twolineshloka
{इदं तु नीचाचरितमसत्पुरुषसेवितम्}
{कथमाचरितं पार्थ पापकर्म सुदुष्करम्}


\twolineshloka
{आर्येण सुकरं त्वाहुरार्यकर्म धनञ्जय}
{अनार्यकर्म त्वार्येण सुदुष्करतमं भुवि}


\twolineshloka
{कथं हि राजवंश्यस्त्वं कौरवेयो विशेषतः}
{क्षत्रधर्मादपक्रान्तः सुवृत्तश्चारितव्रतः}


\twolineshloka
{`अल्पस्तवापराधोऽत्र न त्वां तात विगर्हये}
{वार्ष्णेयापशदं प्राप्य क्षुद्रं कृतमिदं त्वया'}


\twolineshloka
{इदं तु यदतिक्षुद्रं वार्ष्णेयार्थे कृतं त्वया}
{वासुदेवमतं नूनं नैतत्त्वय्युपपद्यते}


\twolineshloka
{को हि नाम प्रमत्ताय परेण सह युध्यते}
{ईदृशं व्यसनं दद्याद्यो न कृष्णसखो भवेत्}


\threelineshloka
{व्रात्याः सङ्क्लिष्टकर्माणः प्रकृत्यैव च गर्हिताः}
{वृष्ण्यन्धकाः कथं पार्थ प्रमाणं भवता कृताः ॥सञ्जय उवाच}
{}


\threelineshloka
{एवमुक्तो रणे पार्थो भूरिश्रवसमब्रवीत्}
{व्यक्तं हि जीर्यमाणोऽपि बुद्धिं जरयते नरः}
{अनर्थकमिदं सर्वं यत्त्वया व्याहृतं प्रभो}


\twolineshloka
{जानन्नेव हृषीकेशं गर्हसे मां च पाण्डवम्}
{सङ्गामाणां हि धर्मज्ञः सर्वशास्त्रार्थपारगः}


% Check verse!
न चाधर्ममहं कुर्यां जानंश्चैव हि मुह्यसे
\threelineshloka
{युध्यन्ते क्षत्रियाः शत्रून्स्वैःस्वैः परिवृता नराः}
{भ्रातृभिः पितृभिः पुत्रैस्तथा सम्बन्धिबान्धवैः}
{वयस्यैरथ मित्रैश्च स्वबाहुबलमाश्रिताः}


\twolineshloka
{स कथं सात्यकिं शिष्यं सुखसम्बन्धिमेव च}
{अस्मदर्थे च युध्यन्तं त्यक्त्वा प्राणान्सुदुस्त्यजान्}


\twolineshloka
{मम बाहुं रणे राजन्दक्षिणं युद्धदुर्मदम्}
{*त्वया निकृष्यमाणं च दृष्टवानस्मि निष्क्रियम्}


\threelineshloka
{न च त्वं रक्षितव्यो हि एको रणगतेन हि}
{यो यस्य युध्यतेऽर्थाय संरक्ष्यो नराधिप}
{तै रक्ष्यमाणैः स नृपो रक्षितव्यो महामृधे}


\twolineshloka
{यद्यहं सात्यकिं दृष्ट्वा तूष्णीमासिष्य आहवे}
{ततस्तेन वियोगश्च प्राप्यं नरकमेव च}


\twolineshloka
{रक्षितव्यो मया यस्मात्तस्माल्लब्धो मया स च}
{यशश्चैव स्वपक्षेभ्यः फलं मित्रस्य रक्षणात्}


\twolineshloka
{यच्च मां गर्हसे राजन्कृष्णेन सह सङ्गतम्}
{कस्तेन सङ्गमं नेच्छेत्तत्र ते बुद्धिविभ्रमः}


\twolineshloka
{आबद्धकवचस्येह रथमारुह्य तिष्ठतः}
{सर्वायुधैरुपेतस्य प्रतियोद्धृप्रतीक्षिणः}


\twolineshloka
{अस्मिन्रथगजानीके हयपत्तिसमाकुले}
{सिंहनादोद्धतरवे गम्भीरे सैन्यसागरे}


\twolineshloka
{स्वैश्चापि समुपेतस्य विक्रान्तस्य तथा रणे}
{सात्यकेन कथं योग्यः सङ्ग्रामस्ते भविष्यति}


\threelineshloka
{बहुभिः सह सङ्गम्य निर्जित्य च महारथान्}
{श्रान्तश्च श्रान्तवाहश्च क्षीणसर्वायुधस्त्वया}
{समेतः सात्यकिः सङ्ख्ये निर्जितश्च महारथः}


\twolineshloka
{ईदृशं सात्यकिं सङ्ख्ये निर्जित्य च महारथम्}
{अधिकत्वं विजानीपे स्ववीर्यवशमागतम्}


\threelineshloka
{इच्छसि त्वं शिरस्तस्य असिना हर्तुमाहवे}
{तथा कृच्छ्रगतं दृष्ट्वा सात्यकिं कः क्षमिष्यति}
{एकस्यैकेन हि कथं सङ्ग्रामः सम्भविष्यति}


\twolineshloka
{त्वं तु गर्हय चात्मानं स्वधर्मं यो न रक्षसि}
{कथं रक्षिष्यसे वीर ये वै त्वां संश्रिता जनाः}


\twolineshloka
{आत्तशस्त्रस्य हि रणे वृष्णिपुत्रं जिघांसतः}
{छिन्नवान्यदहं बाहुं नैतल्लोकविगर्हितम्}


\threelineshloka
{न्यस्तशस्त्रस्य हि पुनर्विकलस्य विवर्मणः}
{अभिमन्योर्वधं तात धार्मिकः को नु पूजयेत् ॥सञ्जय उवाच}
{}


\twolineshloka
{एवमुक्तो महार्बाहुर्यूपकेतुर्महायशाः}
{युयुधानं समुत्सृज्य रणे प्रायमुपाविशत्}


\twolineshloka
{शरानास्तीर्य सव्येन पाणिना पुण्यलक्षणः}
{यियासुर्ब्रह्मलोकाय प्राणान्प्राणेष्वथाजुहोत्}


\twolineshloka
{सूर्ये चक्षुः समाधाय प्रसन्नं सलिले मनः}
{ध्यायन्महोपनिषदं योगयुक्तोऽभवन्मुनिः}


\twolineshloka
{ततस्ते सर्वसेनासु जनाः कृष्णधनञ्जयौ}
{गर्हयामासुरप्येतौ शशंसुर्भूरिदक्षिणम्}


\twolineshloka
{निन्द्यमानौ तथा कृष्णौ नोचतुः किञ्चिदप्रियम्}
{ततः प्रशस्यमानश्च नाहृष्यद्यूपकेतनः}


\twolineshloka
{तांस्तथावादिनो राजन्पुत्रांस्तव धनञ्जयः}
{अमूष्यमाणो मनसा तेषां तस्य च भाषितम्}


\twolineshloka
{असङ्क्रुद्धमना वाचः स्मारयन्निव भारत}
{उवाच पाण्डुतनयः साक्षेपमिव फल्गुनः}


\twolineshloka
{मम सर्वेऽपि राजानो जानन्त्येतन्महाव्रतम्}
{न शक्यो मामको हन्तुं यो मे स्याद्बाणगोचरे}


\twolineshloka
{यूपकेतुं समीक्ष्यैतन्न मां गर्हितुमर्हथ}
{न हि धर्ममविज्ञाय युक्तं गर्हयितुं परम्}


\twolineshloka
{न्यस्तशस्त्रस्य बालस्य विरथस्य विवर्मणः}
{`नाभिमन्योर्वधं यूयं गर्हयध्वं कुतस्तदा}


\twolineshloka
{दुर्योधनस्य क्षुद्रस्य अप्रमाणे च तिष्ठतः}
{सौमदत्तेरथं साधुः सर्वसाहाय्यकारिणः}


\twolineshloka
{अस्मदीया मया रक्ष्याः प्राणबाध उपस्थिते}
{ये मे प्रत्यक्षतो वीरा हन्येरन्निति मे मतिः}


\threelineshloka
{सात्यकश्च वशं नीतः कौरवेण महात्मना}
{ततो मयैतच्चरितं प्रतिज्ञारक्षणं प्रति ॥सञ्जय उवाच}
{}


\twolineshloka
{पुनश्च कृपयाऽऽविष्टो बहु तत्तद्विचिन्तयन्}
{उवाच चैनं कौरव्यमर्जुनः शोकपीडितः}


\twolineshloka
{धिगस्तु क्षत्रधर्मं तु यत्र त्वं पुरुषेश्वरः}
{अवस्थामीदृशीं प्राप्तः शरण्यः शरणप्रदः}


\twolineshloka
{नातिभारः कृतान्तस्य विद्यते कुरुनन्दन}
{यत्र त्वं पुरुषव्याघ्रः प्राप्तः पापामिमां दशाम्}


\twolineshloka
{नात्मनः सुकृतस्यास्य फलं वै नृपसत्तम}
{यत्र त्वं कुरुशार्दूल प्राप्तः पापामिमां दशाम्}


\twolineshloka
{रौरवं नरकं भीमं गमिष्यति सुयोधनः}
{यत्कृते नरशार्दूलः प्राप्तः पापामिमां दशाम्}


\twolineshloka
{को हि नाम पुमाँल्लोके मादृशः पुरुषोत्तम}
{प्रहरेत्त्वद्विधे त्वद्य प्रतिज्ञा यदि नो भवेत्'}


\twolineshloka
{एवमुक्तः स पार्थेन शिरसा भूमिमस्पृशत्}
{पाणिना चैव सव्येन प्राहिणोदस्य दक्षिणम्}


\threelineshloka
{एतत्पार्थस्य तु वचस्ततः श्रुत्वा महाद्युतिः}
{युपकेतुर्महाराज तूष्णीमासीदवाङ्मुखः ॥अर्जुन उवाच}
{}


\twolineshloka
{या प्रीतिर्धर्मराजे मे भीमे च बलिनां वरे}
{नकुले सहदेवे च सा मे त्वयि शलाग्रज}


\threelineshloka
{मया त्वं समनुज्ञातः कृष्णेन च महात्मना}
{गच्छ पुण्यकृतां लोकाञ्छिबिरौशीनरो यथा ॥वासुदेव उवाच}
{}


\threelineshloka
{ये लोका मम विमलाः सकृद्विभाताब्रह्माद्यैः सुरवृषभैरपीष्यमाणाः}
{तान्क्षिप्रं व्रज सतताग्निहोत्रयाजि--न्मत्तुल्यो भव गरुडोत्तमाङ्गयानः ॥सञ्जय उवाच}
{}


\twolineshloka
{`धनञ्जये ब्रुवत्येवं घृणया च परिप्लुते}
{अवाङ्मुखा बभूवुश्च सैनिकाः सर्व एव ते}


\twolineshloka
{मुहूर्तादिव विश्रम्य सात्यकिः क्रोधमूर्च्छितः}
{अमर्षवशमापन्नः सौमदत्तिनिराकृतः'}


\twolineshloka
{उत्थितः स तु शैनेयो विमुक्तः सौमदत्तिना}
{सङ्गमादाय चिच्छित्सुः शिरस्तस्य महात्मनः}


\twolineshloka
{निहतं पाण्डुपुत्रेण प्रसक्तं भूरिदक्षिणम्}
{इयेष सात्यकिर्हन्तुं शलाग्रजमकल्मषम्}


\twolineshloka
{निकृत्तभुजमासीनं छिन्नहस्तमिव द्विपम्}
{क्रोशतां सर्वसैन्यानां निन्द्यमानः सुदुर्मनाः}


\twolineshloka
{वार्यमाणः स कृष्णेन पार्थेन च महात्मना}
{भीमेन चक्रक्षाभ्यामश्वत्थाम्ना कृपेण च}


\twolineshloka
{कर्णेन वृषसेनेन सैन्धवेन तथैव च}
{विक्रोशतां च सैन्यानामवधीत्तं धृतव्रतम्}


\twolineshloka
{प्रायोपविष्टस्य रणे पार्थेन च्छिन्नबाहुनः}
{सात्यकिः कौरवेयस्य खङ्गेनापाहरच्छिरः}


\twolineshloka
{नाभ्यनन्दन्त तं सैन्याः सात्यकिं तेन कर्मणा}
{अर्जुनेन हतं पूर्वं यज्जघान कुरूद्वहम्}


\twolineshloka
{सहस्राक्षसमं चैव सिद्धचारणमानवाः}
{भूरिश्रवसमालोक्य युद्धे प्रायगतं हतम्}


\twolineshloka
{अपूजयन्त तं देवा विस्मितास्तेऽस्य कर्मभिः}
{पक्षवादांश्च सुबहून्प्रावदंस्तव सैनिकाः}


\twolineshloka
{न वार्ष्णेयस्यापराधो भवितव्यं हि तत्तथा}
{तस्मान्मन्युर्न वः कार्यः क्रोधो दुःखतरो नृणाम्}


\twolineshloka
{हन्तव्यश्चैष वीरेण नात्र कार्या विचारणा}
{विहितो ह्यस्य धात्रैव मृत्युः सात्यकिराहवे}


\twolineshloka
{`मर्तव्यमेव सर्वेण चरमं पूर्वमेव वा}
{मन्यध्वं मृत इत्येष माभूद्वो बुद्धिलाघवम्}


\twolineshloka
{तस्मिन्हते महाबाहौ यूपकेतौ महात्मनि}
{धिगेनमिति चाक्रन्दन्क्षत्रियाः क्रोधमूर्च्छिताः}


\threelineshloka
{अन्ये न युक्तमित्येव भवितव्यं तथेति च}
{केचिदासन्विमनसः केचिद्दुःखसमन्विताः' ॥सात्यकिरुवाच}
{}


\twolineshloka
{न हन्तव्यो न हन्तव्य इति मन्दाः प्रभाषत}
{धर्मवादैरधर्मिष्ठा धर्मकञ्चुकमास्थिताः}


\twolineshloka
{यदा बालः सुभद्रायाः सुतः शस्त्रनिनाकृतः}
{युष्माभिर्निहतो युद्धे तदा धर्मः क्व वो गतः}


\threelineshloka
{मया त्वेतत्प्रतिज्ञातं क्षेपे कस्मिंश्चिदेव हि}
{`श्रुत्वा तत्सर्वभावेन गर्हयध्वं न चार्जुनम्}
{शृणुध्वं सर्वमेवेह श्रुत्वा गर्हथ मानवाः'}


\twolineshloka
{यो मां निष्पिष्य सङ्ग्रमे जीवन्हन्यात्पदा रुषा}
{स मे वध्यो भवेच्छत्रुर्यद्यपि स्यान्मुनिव्रतः}


\twolineshloka
{चेष्टमानं प्रतीघाते सभुजं मां सचक्षुषः}
{मन्यध्वं मृत इत्येवमेतद्वो बुद्धिलाघवम्}


% Check verse!
युक्तो ह्यस्य प्रतीघातः कृतो मे कुरुपुङ्गवाः
\twolineshloka
{यत्तु पार्थेन मां दृष्ट्वा प्रतिज्ञामभिरक्षता}
{सखङ्गोऽस्य हृतो बाहुरेतेनैवास्मि वञ्चितः}


\twolineshloka
{भवितव्यं हि यद्भावि दैवं चेष्टयते हि तत्}
{सोयं हतो विमर्देऽस्मिन्किमत्राधर्मचेष्टितम्}


\twolineshloka
{अपि चापं पुरा गीतः श्लोको वाल्मीकिना भुवि}
{न हन्तव्याः स्त्रिय इति यद्ब्रवीषि प्लवङ्गम}


\fourlineindentedshloka
{सर्वकालं मनुष्येण व्यवसायवता सदा}
{पीडाकरममित्राणां यत्स्यात्कर्तव्यमेव तत्}
{अनुष्ठितं मया तच्च कस्माद्गर्हथ मूढवत् ॥सञ्चय उवाच}
{}


\twolineshloka
{एवमुक्ते महाराज सर्वे कौरवपुङ्गवाः}
{न स्म किंचिदभाषन्त मनसा समपूजयन्}


\twolineshloka
{मन्त्रभिपूतस्य महाध्वरेषुयशस्विनो भूरिसहस्रदस्य}
{मुनेरिवारण्यगतस्य तस्यन तत्र कश्चिद्वधमभ्यनन्दत्}


\twolineshloka
{सुनीलकेशं वरदस्य तस्यसूरस्य पारावतलोहिताक्षम्}
{अश्वस्य मेध्यस्य शिरो निकृत्तंन्यस्तं हविर्धानमिवान्तरेण}


\twolineshloka
{स तेजसा शस्त्रकृतेन पूतोमहाहवे देहवरं विसृज्य}
{आक्रामदूर्ध्वं वरदो वरार्होव्यावृत्त्य धर्मेण परेम रोदसी}


\chapter{अध्यायः १४४}
\twolineshloka
{धृतराष्ट्र उवाच}
{}


\threelineshloka
{अजितो द्रोणराधेयविकर्णकृतवर्मभिः}
{`यश्चैवोत्सहते वेतुं समस्तं मामकं बलम'}
{तीर्णः सैन्यार्णवं वीरः प्रतिश्रुत्य युधिष्ठिरे}


\threelineshloka
{स कथं कौरवेयेण समरेष्वनिवारितः}
{निगृह्य भूरिश्रवसा बलाद्भुवि निपातितः ॥सञ्जय उवाच}
{}


\twolineshloka
{शृणु राजन्निहोत्पत्तिं शैनेयस्य यथा पुरा}
{यथा च भूरिश्रवसो यत्र ते संशयो नृप}


\twolineshloka
{`ब्रह्मणस्त्वभवत्पुत्रो मानसोऽत्रिर्महातपाः'}
{अत्रेः पुत्रोऽभवत्सोमः सोमस्य तु बुधः स्मृतः}


\twolineshloka
{बुधस्यैलो महाबाहुः पुत्र आसीत्पुरूरवाः}
{पुरूरवस आयुस्तु आयुषो नहुषः सुतः}


\twolineshloka
{नहुषस्य ययातिस्तु राजा देवर्षिसम्मतः}
{ययातेर्देवयान्यां तु यदुर्ज्येष्ठोऽभवत्सुतः}


\twolineshloka
{यदोरभूदन्ववाये आजमीढ इति स्मृतः}
{यादवस्तस्य तु सुतः शूरस्त्रैलोक्यसम्मतः}


\fourlineindentedshloka
{शूरस्य शौरिर्नृवरो वसुदेवो महायशाः}
{धनुष्यनवरः शूरः कार्तवीर्यसमो युधि}
{धनुष्यनवरः शूरः कार्तवीर्यसमो युधि}
{तद्वार्यश्चापि तत्रैव कुले शिनिरभून्नृप}


\twolineshloka
{एतस्मिन्नेव काले तु देवकस्य महात्मनः}
{दुहितः स्वयंवरे राजन्सर्वक्षत्रसमागमे}


\twolineshloka
{तत्र वै देवकीं देवीं वसुदेवार्थमाशु वै}
{निर्जित्व पार्थिवान्सर्वान्रथमारोपयच्छिनिः}


\twolineshloka
{तां दृष्ट्वा देवकीं शूरो रथस्थां पुरुषर्षभः}
{नामृष्यत महातेजाः सोमदत्तः शिनेर्नृप}


\twolineshloka
{तयोर्युद्धमभूद्राजन्दिनार्धं चित्रमुद्भतम्}
{बाहुयुद्धं सुबलिनोः प्रसक्तं पुरुषर्षभः}


\twolineshloka
{शिनिना सोमदत्तस्तु प्रसह्य भुवि पातितः}
{असिमुद्यम्य केशेषु प्रगृह्य च पदा हतः}


\twolineshloka
{मध्ये राजसहस्राणां प्रेक्षकाणां समन्ततः}
{कृपया च पुनस्तेन स जीवेति विसर्जितः}


\twolineshloka
{तदवस्थः कृतस्तेन सोमदत्तोऽथ मारिष}
{प्रासादयन्महादेवममर्षवशमास्थितः}


\twolineshloka
{तस्य तुष्टो महादेवो वराणां वरदः प्रभुः}
{वरेण च्छन्दयामास स तु वव्रे वरं नृपः}


\twolineshloka
{पुत्रमिच्छामि भगवन्यो निपात्य शिनेः सुतम्}
{मध्ये राजसहस्राणां पदा हन्याच्च संयुगे}


\threelineshloka
{तस्य तद्वचनं श्रुत्वा सोमदत्तस्य पार्थिव}
{एवमस्त्विति तत्रोक्त्वा स देवोऽन्तरधीयत}
{स तेन वरदानेन लब्धवान्भूरिदक्षिणम्}


\twolineshloka
{अपातयच्च समरे सौमदत्तिः शिनेः सुतम्}
{पश्यतां सर्वसैन्यानां पदा चैनमताडयत्}


\twolineshloka
{एतत्ते कथितं राजन्यन्मां त्वं परिपृच्छसि}
{न हि शक्यो रणे जेतुं सात्वतो मनुजर्षभैः}


\threelineshloka
{लब्धलक्ष्याश्च सङ्ग्रामे बहुशश्चित्रयोधिनः}
{देवदानवगन्धर्वान्विजेतारो ह्यविस्मिताः}
{स्ववीर्यविजये युक्ता नैते परपरिग्रहाः}


\twolineshloka
{न तुल्यं वृष्णिभिरिह दृश्यते किञ्चन प्रभो}
{भूतं भव्यं भविष्यच्च बलेन भरतर्षभ}


% Check verse!
न ज्ञातिमवमन्यन्ते वृद्धानां शासने रताः
\twolineshloka
{न देवासुरगन्धर्वा न यक्षोरगराक्षसाः}
{जेतारो वृष्णिवीराणां किं पुनर्मानवा रणे}


\twolineshloka
{ब्रह्मद्रव्ये गुरुद्रव्ये ज्ञातिस्वे चाप्यहिंसकाः}
{एतेषां रक्षितारश्च ये स्युः कस्याञ्चिदापदि}


\twolineshloka
{अर्थवन्तो न चोत्सिक्ता ब्रह्मण्याः सत्यवादिनः}
{समर्थान्नावमन्यन्ते दीनानभ्युद्वरन्ति च}


\twolineshloka
{नित्यं देवपरा दान्तास्त्रातारश्चाविकत्थनाः}
{तेन वृष्णिप्रवीराणां चक्रं न प्रतिहन्यते}


\twolineshloka
{अपि मेरुं वहेत्कश्चित्तरेद्वा मकरालयम्}
{न तु वृष्णिप्रवीराणां समेत्यान्तं व्रजेन्नृप}


\twolineshloka
{एतत्ते सर्वमाख्यातं यत्र ते संशयः प्रभो}
{कुरुराज नरश्रेष्ठ तव व्यपनयो महान्}


\chapter{अध्यायः १४५}
\twolineshloka
{धृतराष्ट्र उवाच}
{}


\threelineshloka
{तदवस्थे हे तस्मिन्भूरिश्रवसि कौरवे}
{यथा भूयोऽभवद्युद्धं तन्ममाचक्ष्व सञ्जय ॥सञ्जय उवाच}
{}


\threelineshloka
{भूरिश्रवसि सङ्क्रान्ते परलोकाय भारत}
{वासुदेवं महाबाहुरर्जुनः समचूचुदत् ॥अर्जुन उवाच}
{}


\threelineshloka
{चोदयाश्वान्भृशं कृष्ण यतो राजा जयद्रथः}
{[श्रूयते पुण्डरीकाक्ष त्रिषु धर्मेषु वर्तते}
{]प्रतिज्ञां सफलां चापि कर्तुमर्हसि मेऽनघ}


% Check verse!
अस्तमेति महाबाहो त्वरमाणो दिवाकरः
\twolineshloka
{एतद्धि पुरुषव्याघ्र महदब्युद्यतं मया}
{कार्यं संरक्ष्यते चैष कुरुसेनामहारथैः}


\twolineshloka
{यथा नाभ्येति सूर्योऽस्तं यथा सत्यं भवेद्वचः}
{चोदयाश्वांस्तथा कृष्ण यथा हन्यां जयद्रथम्}


\twolineshloka
{`ततः कृष्णो महाबाहुरश्वान्रजतसन्निभान्}
{हयज्ञश्चोदयामास जयद्रथवधं प्रति}


\twolineshloka
{तं प्रयान्तममोघेषुमुत्पतद्भिरिवाशुगैः}
{त्वरमाणा महाराज सेनामुख्याः समाद्रवन्}


\twolineshloka
{दुर्योधनश्च कर्णश्च वृषसेनोऽथ मद्रराट्}
{अश्वत्थामा कृपश्चैव स्वयमेव च सैन्धवः}


\twolineshloka
{समासाद्य च बीभत्सुः सैन्धवं समुपस्थितम्}
{नेत्राभ्यां क्रोधदीप्ताभ्यां सम्प्रैक्षन्निर्दहन्निव'}


\twolineshloka
{यथाग्निरिन्धनेद्धौ वै क्रोधेन्धनसमीरितः}
{सैन्धवस्य मुखं त्यक्त्वा कर्णः सात्वतमभ्ययात्}


\twolineshloka
{उपायान्तं* तु राधेयं दृष्ट्वा पार्थो महारथः}
{प्रहसन्देवकीपुत्रमिदं वचनमब्रवीत्}


\twolineshloka
{एष प्रयात्याधिरथिः सात्यकेः स्यन्दनं प्रति}
{न मृष्यति हतं नूनं भूरिश्रवसमाहवे}


\threelineshloka
{यत्र यात्येष तत्र त्वं चोदयाश्वाञ्जनार्दन}
{न सौमदत्तिपदवीं गमयेत्सात्यकिं वृषः ॥सञ्जय उवाच}
{}


\twolineshloka
{एवमुक्तो महाबाहुः केशवः सव्यसाचिना}
{प्रत्युवाच महातेजाः कालयुक्तमिदं वचः}


\twolineshloka
{अलमेष महाबाहुः कर्णायैकोऽपि पाण्डव}
{किं पुनर्द्रौपदेयाभ्यां सहितः सात्वतर्षभः}


\threelineshloka
{न च तावत्क्षमः पार्थ तव कर्णेन सङ्गरः}
{प्रज्वलन्ती महोल्केव तिष्ठत्यस्मिन्हि वासवी}
{त्वदर्थं पूज्यमानैषा रक्ष्यते परवीरहन्}


\twolineshloka
{`न कर्णं प्राकृतं मन्ये तेन योद्धुं न साम्प्रतम्}
{यः कर्णो बलवानेष शक्तोऽस्माञ्जेतुमोजसा}


\twolineshloka
{कर्णस्यैष महान्दोषो यद्दूयेन पदेपदे}
{प्रमादाच्च घृणित्वाच्च तेन शोचिति मे मनः'}


\twolineshloka
{अतः कर्णः प्रयात्वत्र सात्वतस्य यथा तथा}
{अहं ज्ञास्यामि कौन्तेय कालमस्य दुरात्मनः}


\threelineshloka
{यत्रैनं विशिखैस्तीक्ष्णैः पातयिष्यसि भूतले}
{तदा गन्तासि पार्थ त्वं तेन योद्धुं दुरात्मना ॥धृतराष्ट्र उवाच}
{}


\twolineshloka
{योऽसौ कर्णेन वीरस्य वार्ष्णेयस्य समागमः}
{हते तु भूरिश्रवसि तद्युद्धमभवत्कथम्}


\threelineshloka
{सात्यकिश्चापि विरथः कं समारूढवान्रथम्}
{चक्ररक्षौ च पाञ्चाल्यौ तन्ममाचक्ष्व सञ्जय ॥सञ्जय उवाच}
{}


\threelineshloka
{हन्त ते कथयिष्यामि यथावृत्तं महारणे}
{पश्यतां सर्वसैन्यानां केशवार्युनयोरपि}
{शुश्रूषस्व स्थिरो भूत्वा दुराचरितमात्मनः}


\twolineshloka
{xxमेव हि कृष्णस्य मनोगतमिदं प्रभो}
{विजेतव्यो यथा वीरः सात्यकिः सौमदत्तिना}


\threelineshloka
{अतीतानागते राजन्स हि वेत्ति जनार्दनः}
{ततः सूतं समाहूय दारुकं संदिदेश ह}
{}


\threelineshloka
{रथो मे युज्यतां कल्यमिति राजन्महाबालः ॥न हि देवा न गन्धर्वा न यक्षोरगराक्षसाः}
{मानवा वाऽपि जेतारः कृष्णयोः सन्ति केचन}
{}


\twolineshloka
{पितामहपुरोगाश्च देवाः सिद्धाश्च तं विदुः}
{तयोः प्रभावमतुलं शृणु युद्धं तु तत्तथा}


\twolineshloka
{सात्यकिं विरथं दृष्ट्वा कर्णं चाभ्युद्यतं रणे}
{दध्मौ शङ्खं महानादमार्षभेणाथ माधवः}


\twolineshloka
{दारुकोऽवेत्य सन्देशं श्रुत्वा शङ्खस्य च स्वनम्}
{रथमन्वानयत्तस्मै सुपर्णोच्छ्रितकेतनम्}


\twolineshloka
{स केशवस्यानुमते रथं दारुकसंयुतम्}
{आरुरोह शिनेः पौत्रो ज्वलनादित्यसन्निभम्}


\twolineshloka
{कामगैः शैव्यसुग्रीवमेघपुष्पवलाहकैः}
{हयोदग्रैर्महावेगैर्हेमभाण्डविभूषितैः}


\twolineshloka
{युक्तं समारुह्य च तं विमानप्रतिमं रथम्}
{अभ्यद्रवत राधेयं प्रवपन्सायकान्बहून्}


\twolineshloka
{चक्ररक्षावपि तदा युधामन्यूत्तमौजसौ}
{धनञ्जयरथं हित्वा राधेयं प्रत्युदीयतुः}


\twolineshloka
{राधेयोऽपि महाराज शरवर्षं समुत्सृजन्}
{अभ्यद्रवत्सुसङ्क्रुद्धो रणे शैनेयमच्युतम्}


\threelineshloka
{नैव दैवं न गान्धर्वं नासुरं न च राक्षसम्}
{तादृशं भुवि नो युद्धं दिवि वा श्रुतमित्युत}
{उपारसत तत्सैन्यं सरथाश्वनरद्विपम्}


\twolineshloka
{तयोर्दृष्ट्वा महाराज कर्म सम्मूढचेतसः}
{सर्वे च समपश्यन्त तद्युद्धमतिमानुषम्}


% Check verse!
तयोर्नृवरयो राजन्सारथ्यं दारुकस्य च
\twolineshloka
{गतप्रत्यागतावृत्तैर्मण्डलैः सन्निवर्तनैः}
{सारथेस्तु रथस्थस्य काश्यपेयस्य विस्मिताः}


\twolineshloka
{नभस्तलगताश्चैव देवगान्धर्वदानवाः}
{अतीवावहिता द्रष्टुं कर्णशैनेययो रणम्}


% Check verse!
मित्रार्थे तौ पराक्रान्तौ शुष्मिणौ स्पर्धिनौ रमे
\twolineshloka
{कर्णश्चामरसङ्काशो युयुधानश्च सात्यकिः}
{अन्योन्यं तौ महाराज शरवर्षैरवर्षताम्}


\twolineshloka
{प्रममाथ शिनेः पौत्रं कर्णः सायकवृष्टिभिः}
{अमृष्यमाणो निधनं कौरव्यजलसन्धयोः}


\threelineshloka
{कर्णः शोकसमाविष्टो महोरग इव श्वसन्}
{स शैनेयं रणे क्रुद्धः प्रदहन्निव चक्षुषा}
{अभ्यधावत वेगेन पुनःपुनररिन्दम}


\twolineshloka
{तं तु सक्रोधमालोक्य सात्यकिः प्रत्ययुध्यत}
{महता शरवर्षेण गजं प्रतिगजो यथा}


\twolineshloka
{तौ समेतौ नरव्याघ्रौ व्याघ्राविव तरस्विनौ}
{अन्योन्यं सन्ततक्षाते रणेऽनुपमविक्रमौ}


\twolineshloka
{ततः कर्णं शिनेः पौत्रः सर्वपारसवैः शरैः}
{बिभेद सर्वगात्रेषु पुनःपुनररिन्दम}


\twolineshloka
{सारथिं चास्य भल्लेन रथनीडादपातयत्}
{अश्वांश्च चतुरः श्वेतान्निजघान शितैः शरैः}


\twolineshloka
{छित्त्वा ध्वजं रथं चैव शतधा पुनर्षर्षभ}
{चकार विरथं कर्णं तव पुत्रस्य पश्यतः}


\threelineshloka
{ततो विमनसो राजंस्तावकास्ते महारथाः}
{वृषसेनः कर्णसुतः शल्यो मद्राधिपस्तथा}
{}


% Check verse!
द्रोणपुत्रश्च शैनेयं सर्वतः पर्यवारयन् ॥ततः पर्याकुलं सर्वं न प्राज्ञायत किंचन
\twolineshloka
{तथा सात्यकिना वीरे विरथे सूतजे कृते}
{हाहाकारस्ततो राजन्सर्वसैन्येष्वभून्महान्}


\twolineshloka
{कर्णोऽपि विरथो राजन्सात्वतेन कृतः शरैः}
{दुर्योधनरथं तूर्णमारुरोह विनिःश्वसन्}


\twolineshloka
{मानयंस्तव पुत्रस्य बाल्यात्प्रभृति सौहृदम्}
{कृतां राज्यप्रदानेन प्रतिज्ञां परिपालयन्}


% Check verse!
तथा तु विरथं कर्णं पुत्रांश्च परिपालयन् ॥दुःशासनमुखान्वीरान्नावधीत्सात्यकिर्वशी
\twolineshloka
{रक्षन्प्रतिज्ञां भीमेन पार्थेन च पुराकृताम्}
{विरथान्विह्वलांश्चक्रे न तु प्राणैर्व्ययोजयत्}


\twolineshloka
{भीमसेनेन तु वदः पुत्राणां ते प्रतिश्रुतः}
{अनुद्यूते च पार्थेन वधः कर्णस्य संश्रुतः}


\twolineshloka
{वधे त्वकुर्वन्यत्नं ते तस्य कर्णमुखास्तदा}
{नाशक्नुवंस्ततो हन्तुं सात्यकिं प्रवरा रथाः}


\threelineshloka
{द्रौणिश्च कृतवर्मा च तथैवान्ये महारथाः}
{निर्जिता धनुषैकेन शतशः क्षत्रियर्षभाः}
{काङ्क्षता परलोकं च धर्मराजस्य च प्रियम्}


\twolineshloka
{कृष्णयोः सदृशो वीर्ये सात्यकिः शत्रुतापनः}
{जितवान्सर्वसैन्यानि तावकानि हसन्निव}


\threelineshloka
{कृष्णो वाऽपि भवेल्लोके पार्थो बाऽपि धनुर्धरः}
{शैनेयो वा नरव्याघ्र चतुर्थस्तु न विद्यते ॥धृतराष्ट्र उवाच}
{}


\threelineshloka
{अजय्यं वासुदेवस्य रथमास्थाय सात्यकिः}
{विरथं कृतवान्कर्णं वासुदेवसमो युधि}
{दारुकेण समायुक्तः स्वबाहुबलदर्पितः}


\fourlineindentedshloka
{कच्चिदन्यं समारूढः सात्यकिः शत्रुतापनः}
{एतदिच्छाम्यहं श्रोतुं कुशलो ह्यसि भाषितुम्}
{असह्यं तमहं मन्ये तन्ममाचक्ष्व सञ्जय ॥सञ्जय उवाच}
{}


\twolineshloka
{शृणु राजन्यथावृत्तं रथमन्यं महामतिः}
{दारुकस्यानुजस्तूर्णं कल्पनाविदिकल्पितम्}


\twolineshloka
{आयसैः काञ्चनैश्चापि पट्टैः सन्नद्धकूबरम्}
{तारासहस्रखचितं सिंहध्वजपताकिनम्}


\twolineshloka
{अश्वैर्वातजवैर्युक्तं हेमभाण्डपरिच्छदैः}
{सैन्धवैरिन्दुसङ्काशैः सर्वशब्दातिगैर्दृढैः}


\twolineshloka
{चित्रकाञ्चनसन्नाहैर्वाजिमुख्यैर्विशाम्पते}
{घण्टाजालाकुलरवं शक्तितोमरविद्युतम्}


\twolineshloka
{युक्तं साङ्ग्रामिकैर्द्रव्यैर्बहुशस्त्रपरिच्छदैः}
{रथं सम्पादयामास मेघगम्भीरनिःस्वनम्}


\twolineshloka
{तं समारुह्य शैनेयस्तव सैन्यमुपाद्रवत्}
{दारुकोऽपि यथाकामं प्रययौ केशवान्तिकम्}


\twolineshloka
{कर्णस्यापि रथं राजञ्शङ्खगोक्षीरपाण्डुरैः}
{चित्रकाञ्चनसन्नाहैः सदश्वैर्वेगवत्तरैः}


\threelineshloka
{हेमकक्ष्याध्वजोपेतं क्लृप्तयन्त्रपताकिनम्}
{अग्र्यं रथं सुयन्तारं बहुशस्त्रपरिच्छदम्}
{उपाजह्रुस्तमास्थाय कर्णोऽप्यभ्यद्रवद्रिपून्}


\twolineshloka
{एतत्ते सर्वमाख्यातं यन्मां त्वं परिपृच्छसि}
{भूयश्चापि निबोधेमं तवापन्नयजं क्षयम्}


\twolineshloka
{एकत्रिंशत्तव सुता भीमसेनेन पातिताः}
{दुर्मुखं प्रमुखे कृत्वा सततं चित्रयोधिनम्}


\threelineshloka
{शतशो निहताः शूराः सात्वतेनार्जुनेन च}
{भीष्मं प्रमुखतः कृत्वा भगदत्तं च भारत}
{एवमेष क्षयो वृत्तो राजन्दुर्मन्त्रिते तव}


\chapter{अध्यायः १४६}
\twolineshloka
{धृतराष्ट्र उवाच}
{}


\threelineshloka
{तथा गतेषु शूरेषु तेषां मम च सञ्जय}
{किंस्विद्भीमार्जुनौ चैव सात्यकिश्चाकरोत्तदा ॥सञ्जय उवाच}
{}


\twolineshloka
{विरथो भीमसेनो वै कर्णवाक्शल्यपीडितः}
{अमर्षवशमापन्नः फल्गुनं वाक्यमब्रवीत्}


\twolineshloka
{पुनःपुनस्तूबरक मूढ औदरिकेति च}
{अकृतास्त्रक मा योत्सीर्बाल सङ्ग्रामकातर}


\twolineshloka
{इति मामब्रवीत्कर्णः पश्यतस्ते धनञ्जय}
{एवं वक्ता च मे वध्यस्तेन चोक्तस्तथा ह्यहम्}


\twolineshloka
{एतद्व्रतं महाबाहो त्वया सह कृतं मया}
{तथैतन्मम कौन्तेय यथा तव न संशयः}


\twolineshloka
{तद्वधाय नरश्रेष्ठ स्मरैतद्वचनं मम}
{यथा भवति तत्सत्यं तथा कुरु धनञ्जय}


\twolineshloka
{तच्छ्रुत्वा वचनं तस्य भीमस्यामितविक्रमः}
{ततोऽर्जुनोऽब्रवीत्कर्णं किञ्चिदभ्येत्य संयुगे}


\twolineshloka
{कर्णकर्ण वृथादृष्टे सूतपुत्रात्मसंस्तुत}
{अधर्मबुद्धे शृणु मे यत्त्वां वक्ष्यामि साम्प्रतम्}


\twolineshloka
{द्वावेवं कर्ण शूराणां रणे दृष्टौ जयाजयौ}
{तौ चाप्यनित्यौ राधेय वासवस्यापि युध्यतः}


\threelineshloka
{`रणमुत्सृज्य निर्लज्ज गच्छसे च पुनः पुनः}
{माहात्म्यं पश्य भीमस्य कर्म जन्म कुले तथा}
{नोक्तवान्परुषं यत्त्वां पलायनपरायणम्}


\twolineshloka
{भूयस्त्वमपि सङ्गम्यं सकृदेव यदृच्छया}
{विरथं कृतवान्वीरं पाण्डवं सूतदायद}


\twolineshloka
{कुलस्य सदृशं चापि राधेय कृतवानसि}
{नैकान्तसिद्धिः सङ्ग्रामे वासवस्यापि विद्यते'}


\twolineshloka
{मुमूर्षुर्युयुधानेन विरथो विकलेन्द्रियः}
{मद्वध्यस्त्वमिति ज्ञात्वा जित्वा जीवन्विसर्जितः}


\threelineshloka
{यदृच्छया रणे भीमं युध्यमानं महाबलम्}
{कथञ्चिद्विरथं कृत्वा यत्त्वं रूक्षमभाषथाः}
{अधर्मस्त्वेष सुमहाननार्यचरितं च तत्}


\twolineshloka
{नारिं जित्वाऽतिकत्थन्ते न च जल्पन्ति दुर्वचः}
{न च कञ्चन निन्दन्ति सन्तः शूरा नरर्षभाः}


\twolineshloka
{त्वं तु प्राकृतविज्ञानस्तत्तद्वदसि सूतज}
{बह्वबद्धमकर्ण्यं च चापलादपरीक्षितम्}


\threelineshloka
{युध्यमानं पराक्रान्तं शूरमार्यव्रते रतम्}
{यदवोचोऽप्रियं भीमं नैतत्सत्यं वचस्तव}
{पश्यतां सर्वसैन्यानां केशवस्य ममैव च}


\twolineshloka
{विरथो भीमसेनेन कृतोऽसि बहुशो रणे}
{न च त्वां परुषं किञ्चिदुक्तवान्पाण्डुनन्दनः}


\threelineshloka
{यस्मात्तु बहु रूक्षं च श्रावितस्ते वृकोदरः}
{परोक्षं यच्च सौभद्रो युष्माभिर्बहुभिर्हतः}
{तस्मादस्यावलेपस्य सद्यः फलमवाप्नुहि}


\twolineshloka
{त्वया तस्य धनुश्छिन्नमात्मनाशाय दुर्मते}
{तस्माद्वध्योऽसि मे मूढ सभृत्यसुतबान्धवः}


\twolineshloka
{कुरु त्वं सर्वकृत्यानि महत्ते भयमागतम्}
{हन्ताऽस्मि वृषसेनं ते प्रेक्षमाणस्य संयुगे}


\twolineshloka
{ये चान्येऽप्युपयास्यन्ति बुद्धिमोहेन मां नृपाः}
{तांश्च सर्वान्हनिष्यामि सत्येनायुधमालभे}


\threelineshloka
{त्वां च मूढाकृतप्रज्ञमतिमानिनमाहवे}
{दृष्ट्वा दुर्योधनो मन्दो भृशं तप्स्यति पातितम् ॥सञ्जय उवाच}
{}


\twolineshloka
{अर्जुनेन प्रतिज्ञाते वधे कर्णसुतस्य तु}
{महान्सुतुमुलः शब्दो बभूव रथिनां तदा}


\twolineshloka
{तस्मिन्नाकुलसङ्ग्रामे वर्तमाने महाभये}
{मन्दरश्मिः सहस्रांशुरस्तङ्गिरिमुपाद्रवत्}


\twolineshloka
{ततो दुर्योधनो राजा राधेयं त्वरितोऽब्रवीत्}
{अर्जुनं प्रेक्ष्य संयातं जयद्रथवधं प्रति}


\twolineshloka
{`अमानुषाणि कर्माणि कुर्वन्तौ भरतर्षभ}
{सात्यकिं भीमसेनं च यत्तौ तौ दर्शयन्निव}


\twolineshloka
{अयं स वैकर्तन युद्धकालोविदर्शयस्वात्मबलं महात्मन्}
{यथा न वध्येत रणेऽर्जुनेनजयद्रथः कर्ण तथा कुरुष्व}


\twolineshloka
{अल्पावशेषो दिवसो हि साम्प्रतंनिवारयेहाद्य रिपुं शरौघैः}
{दिनक्षयं प्राप्य नरप्रवीरध्रुवो हि नः कर्ण जयो भविष्यति}


\twolineshloka
{सैन्धवे रक्ष्यमाणे तु सूर्यस्यास्तमनं प्रति}
{मिथ्याप्रतिज्ञः कौन्तेयः प्रवेक्ष्यति हुताशनम्}


\twolineshloka
{अनर्जुनायां च भुवि मुहूर्तमपि मानद}
{जीवितुं नोत्सहेरन्वै भ्रातरोऽस्य सहानुगाः}


\twolineshloka
{पाण्डवेषु विनष्टेषु सशैलवनकाननाम्}
{वसुन्धरामिमां कर्ण भोक्ष्यामो महतकण्टकाम्}


\twolineshloka
{दैवेनोपहतः पार्थो विपरीतश्च मानद}
{कार्याकार्यमजानानः प्रतिज्ञां कृतवान्रणे}


\twolineshloka
{नूनमात्मविनाशाय पाण्डवेन किरीटिना}
{मिथ्याप्रतिज्ञैव कृता जयद्रथवधं प्रति}


\twolineshloka
{कथं जीवति दुर्धर्षे त्वपि राधेय फल्गुनः}
{अनस्तङ्गत आदित्ये हन्यात्सैन्धवकं नृपम्}


\twolineshloka
{रक्षितं मद्रराजेन कृपेण च महात्मना}
{जयद्रथं रणमुखे कथं हन्याद्धनञ्जयः}


\twolineshloka
{द्रौणिना रक्ष्यमाणं च मया दुःशासनेन च}
{कथं प्राप्स्यति बीभत्सुः सैन्धवं कालचोदितः}


\twolineshloka
{युध्यन्ते बहवः शूरा लम्बते च दिवाकरः}
{शङ्के जयद्रथं पार्थो नैव प्राप्स्यति मानद}


\threelineshloka
{स त्वं कर्ण मया सार्धं शूरैश्चान्यैर्महारथैः}
{द्रौणिना त्वं हि सहितो मद्रेशेन कृपेण च}
{युध्यस्व यत्नमास्थाय परं पार्थेन संयुगे}


\twolineshloka
{एवमुक्तस्तु राधेयस्तव पुत्रेण मारिष}
{दुर्योधनमिदं वाक्यं प्रत्युवाच कुरूत्तमम्}


\twolineshloka
{दृढलक्ष्येण वीरेण भीमसेनेन धन्विना}
{भृशं भिन्नतनुः सङ्ख्ये शरजालैरनेकशः}


\threelineshloka
{स्थातव्यमिति तिष्ठामि रणे सम्प्रति मानद}
{नाङ्गमिङ्गति किञ्चिन्मे सन्तप्तस्य महेषुभिः}
{योत्स्यामि तु यथाशक्त्या त्वदर्थं जीवितं मम}


\twolineshloka
{`तत्तथा प्रयतिष्येऽहं परं शक्त्याxxxxसुयोधन'}
{यथा पाण्डवमुख्योऽसौ न हनिष्याति सैन्धवम्}


\twolineshloka
{न हि मे युध्यमानस्य सायकानस्यतः शितान्}
{सैन्धवं प्राप्स्यते वीरः सव्यसाची धनञ्जयः}


\twolineshloka
{यत्तु भक्तिमता कार्यं सततं हितकाङ्क्षिणा}
{तत्करिष्यामि कौरव्य जयो दैवे प्रतिष्ठितः}


\twolineshloka
{सैन्धवार्थे परं यत्नं करिष्याम्यद्य संयुगे}
{त्वत्प्रियार्थं महाराज जयो दैवे प्रतिष्ठितः}


\twolineshloka
{अद्य योत्स्येऽर्जुमहं पौरुषं स्वं व्यपाश्रितः}
{त्वदर्थे पुरुषव्याघ्र जयो दैवे प्रतिष्ठितः}


\threelineshloka
{अद्य युद्धं कुरुश्रेष्ठ मम पार्थस्य चोभयोः}
{पश्यन्तु सर्वसैन्यानि दारुणं रोमहर्षणम् ॥सञ्जय उवाच}
{}


\twolineshloka
{कर्णकौरवयोरेवं रणे संभाषमाणयोः}
{अर्जुनो निशितैर्बाणैर्जघान तव वाहिनीम्}


% Check verse!
चिच्छेद निशितैर्बाणैः शूराणामनिवर्तिनाम्
\twolineshloka
{भुजान्परिघसङ्काशान्हस्तिहस्तोपमान्रणे}
{शिरांसि च महाबाहुश्चिच्छेद निशितैः शरैः}


\twolineshloka
{हस्तिहस्तान्द्विपस्कन्धान्रथाक्षांश्च समन्ततः}
{शोणिताक्तान्हयारोहान्गृहीतप्रासतोमरान्}


\threelineshloka
{क्षुरैश्चिच्छेद बीभत्सुर्द्विधैकैकं त्रिधैव च}
{हयान्वारणमुख्यांश्च पदातींश्च समन्ततः}
{ध्वजांश्छत्राणि चापानि चामराणि शिरांसि च}


\twolineshloka
{कक्षमग्निरिवोद्धूतः प्रादहत्तव वाहिनीम्}
{अचिरेण महीं पार्थश्चकार रुधिरोत्तराम्}


\twolineshloka
{हतभूयिष्ठयोधं तत्कृत्वा तव बलं बली}
{आससाद दुराधर्षः सैन्धवं सत्यविक्रमः}


\twolineshloka
{बीभत्सुर्भीमसेनेन सात्वतेन च रक्षितः}
{प्रबभौ भरतश्रेष्ठ ज्वलन्निव हुताशनः}


\twolineshloka
{तं तथाऽवस्थितं दृष्ट्वा त्वदीया वीरसंमताः}
{नामृष्यन्त महेष्वासाः पाण्डवं पुरुषर्षभाः}


\twolineshloka
{दुर्योधनश्च कर्णश्च वृषसेनोऽथ मद्रराट्}
{अश्वत्थामा कृपश्चैव स्वयमेव च सैन्धवः}


\twolineshloka
{सन्नद्धाः सैन्धवस्यार्थे समावृण्वन्किरीटिनम्}
{नृत्यन्तं रथमार्गेषु धनुर्ज्यातलनिःस्वनैः}


\twolineshloka
{सङ्ग्रामकोविदं पार्थं सर्वे युद्धविशारदाः}
{अभीताः पर्यवर्तन्त व्यादितास्यमिवान्तकम्}


\twolineshloka
{सैन्धवं पृष्ठतः कृत्वा जिघांसन्तोऽच्युतार्जुनौ}
{सूर्यास्तमनमिच्छन्तो लोहितायति भास्करे}


\twolineshloka
{ते भुजैर्भोगिभोगाभैर्धनूंष्यानम्य सायकान्}
{मुमुचुःसूर्यरश्म्याभाञ्छतशः फल्गुनं प्रति}


\twolineshloka
{ततस्तानस्यमानांश्च किरीटी युद्धदुर्मदः}
{द्विधा त्रिधाऽष्टधैकैकं छित्त्वा विव्याध तान्रथान्}


\twolineshloka
{सिंहलाङ्गूलकेतुस्तु दर्शयन्वीर्यमात्मनः}
{शारद्वतीसुतो राजन्नर्जुनं प्रत्यवारयन्}


\twolineshloka
{स विद््वा दशभिः पार्थं वासुदेवं च सप्तभिः}
{अतिष्ठद्रथमार्गेषु सैन्धवं प्रतिपालयन्}


\twolineshloka
{अथैनं कौरवश्रेष्ठाः सर्व एव महारथाः}
{महता रथवंशेन सर्वतः प्रत्यवारयन्}


\twolineshloka
{विष्फारयन्तश्चापानि विसृजन्तश्च सायकान्}
{सैन्धवं पर्यरक्षन्त शासनात्तनयस्य ते}


\twolineshloka
{ततः पार्थस्य शूरस्य बाह्वोर्बलमदृश्यत}
{इषूणामक्षयत्वं च धनुषो गाण्डिवस्य च}


\twolineshloka
{अस्त्रैरस्त्राणि संवार्य द्रौणेः शारद्वतस्य च}
{एकैकं दशभिर्बाणैः सर्वानेव समार्पयत्}


\twolineshloka
{तं द्रौणिः पञ्चविंशत्या वृषसेनश्च सप्तभिः}
{दुर्योधनश्च विंशत्या कर्णशल्यौ त्रिभिस्त्रिभिः}


\twolineshloka
{त एनमभिगर्जन्तो विध्यन्तश्च पुनःपुनः}
{विधुन्वन्तश्च चापानि सर्वतः प्रत्यवारयन्}


\twolineshloka
{श्लिष्टं च सर्वतश्चक्रू रथमण्डलमाशु ते}
{सूर्यास्तमनमिच्छन्तस्त्वरमाणा महारथाः}


\twolineshloka
{त एनमभिनर्दन्तौ विधुन्वाना धनूंषि च}
{सिषिचुर्मार्गणैस्तीक्ष्णैर्गिरिं मेघा इवाम्बुभिः}


\twolineshloka
{ते महास्त्राणि दिव्यानि तत्र राजन्व्यदर्शयन्}
{धनञ्जयस्य गात्रे तु शूराः परिघबाहवः}


\threelineshloka
{`निवार्य ताञ्शरव्रातैर्दिव्यान्यस्त्राणि दर्शयन्'}
{हतभूयिष्ठयोधं तत्कृत्वा तव बलं बली}
{आससाद दुराधर्षः सैन्धवं सत्यविक्रमः}


\twolineshloka
{तं कर्णः संयुगे राजन्प्रत्यवारयदाशुगैः}
{मिषतो भीमसेनस्य सात्वतस्य च भारत}


\twolineshloka
{तं पार्थो दशभिर्बाणैः प्रत्यविध्यद्राणाजिरे}
{सूतपूत्रं महाबाहुः सर्वसैन्यस्य पश्यतः}


\twolineshloka
{सात्वतश्च त्रिभिर्बाणैः कर्णं विव्याध मारिष}
{भीमसेनस्त्रिभिश्चैव पुनः पार्थश्च सप्तभिः}


\twolineshloka
{तान्कर्णः प्रतिविव्याध षष्ट्याषष्ट्या महारथः}
{तद्युद्धमभवद्राजन्कर्णस्य बहुभिः सह}


\twolineshloka
{तत्राद्भूतमपश्याम सूतपुत्रस्य मारिष}
{यदेकः समरे क्रुद्धस्त्रीन्रथान्पर्थवारयत्}


\twolineshloka
{फल्गुनस्तु महाबाहुः कर्णं वैकर्तनं रणे}
{सायकानां शतेनैव सर्वमर्मस्वताडयत्}


\twolineshloka
{रुधिरोक्षितसर्वाङ्गः सूतपुत्रः प्रतापवान्}
{शरैः पञ्चाशता वीरः फल्गुनं प्रत्यविध्यत}


\threelineshloka
{तस्य तल्लाघवं दृष्ट्वा नामृष्यत रणेऽर्जुनः}
{तस्य पार्थो धनुश्छित्त्वा विव्याधैनं स्तनान्तरे}
{सायकैर्नवभिर्वीरस्त्वरमाणो धनञ्जयः}


\twolineshloka
{अथान्यद्धनुरादाय सूतपुत्रः प्रतापवान्}
{सायकैरष्टसाहस्रैश्छादयामास पाण्डवम्}


\twolineshloka
{तां बाणवृष्टिमतुलां कर्णचापसमुत्थिताम्}
{व्यधमत्सायकैः पार्थः शलभानिव मारुतः}


\twolineshloka
{छादयामास च तदा सायकैरर्जुनो रणे}
{पश्यतां सर्पयोधानां दर्शयन्पाणिलाघवम्}


\twolineshloka
{वधार्थं चास्य समरे सायकं सूर्यवर्चसम्}
{चिक्षेप त्वरया युक्तस्त्वराकाले धनञ्जयः}


\twolineshloka
{तमापतन्तं वेगेन द्रौणिश्चिच्छेद सायकम्}
{अर्धचन्द्रेण तीक्ष्णेन स च्छिन्नः प्रापतद्भूवि}


\twolineshloka
{कर्णोऽपि द्विषतां हन्ता छादयामास फल्गुनम्}
{सायकैर्बहुसाहस्रैः कृतप्रतिकृतेप्सया}


\twolineshloka
{तौ वृषाविव नर्दन्तौ नरसिंहौ महारथौ}
{सायकैस्तु प्रतिच्छन्नं चक्रतुः खमजिह्मगैः}


\twolineshloka
{अदृश्यौ च शरौथैस्तौ निघ्न्तावितरेतरम्}
{कर्ण पार्थोऽस्ति तिष्ठत्वं कर्णोऽहं तिष्ठ फल्गुन्न}


\twolineshloka
{इत्येवं तर्जयन्तौ तौ वाक्शल्यैस्तु परस्परम्}
{विध्येतां समरे वीरौ चित्रं लघु च सुष्ठु च}


\twolineshloka
{प्रेक्षणीयौ चाभवतां सर्वयोधसमागमे}
{प्रशस्यामानौ समरे सिद्धचारणपन्नगैः}


\twolineshloka
{अयुध्येतां महाराज परस्परवधैषिणौ}
{ततो दुर्योधनो राजंस्तावकानभ्यभाषत}


\twolineshloka
{यत्नाद्रक्षत राधेयं नाहत्वा समरेऽर्जुनम्}
{निवर्तिष्यति राधेयो न वाऽजित्वाऽद्य फल्गुनं}


\threelineshloka
{एतस्मिन्न्तरे राजन्दृष्ट्वा कर्णस्य विक्रमम्}
{आकर्णमुक्तैरिषुभिः कर्णस्य चतुरो हयान्}
{अनयत्प्रेतलोकाय चतुर्भिः श्वेतवाहनः}


\twolineshloka
{सारथिं चास्य भल्लेन रथनीडादपातयत्}
{छादयामास स शरैस्तव पुत्रस्य पश्यतः}


\twolineshloka
{सञ्छाद्यमानः समरे हताश्वो हतसारथिः}
{मोहितः शरजालेन कर्तव्यं नाभ्यपद्यत}


\twolineshloka
{तं तथा विरथं दृष्ट्वा रथमारोप्य तं तदा}
{अश्वत्थामा महाराज भूयोऽर्जुनमयोधयत्}


\twolineshloka
{मद्रराजश्च कौन्तेयमविध्यत्त्रिंशता शरैः}
{`आवव्रेऽर्जुनमार्गं च शरजालेन भारत'}


\threelineshloka
{शारद्वतस्तु विंशत्यं वासुदेवं समार्पयत्}
{धनञ्जयं द्वादशभिराजघान शिलीमुखैः ॥चतुर्भिः सिन्धुराजश्च वृषसेनश्च सप्तभिः}
{पृथक्पृथङ््महाराज विव्यधुः कृष्णपाण्डवौ}


\threelineshloka
{तथैव तान्प्रत्यविध्यत्कुन्तीपुत्रो धनञ्जयः}
{द्रोणपुत्रं चतुःषष्ट्या मद्रराजं शतेन च ॥सैन्धवं दशभिर्बाणैर्वृषसेनं त्रिभिः शरैः}
{शारद्वतं च विंशत्या विद्व्वा पार्थो ननाद ह}


\twolineshloka
{ते प्रतिज्ञाप्रतीघातमिच्छन्तः सव्यसाचिनः}
{सहितास्तावकास्तूर्णमभिपेतुर्धनञ्जयम्}


\twolineshloka
{अथार्जुनः सर्वतो वारुणास्त्रंप्रादुश्चक्रे त्रासयन्धार्तराष्ट्रान्}
{तं प्रत्युदीयः कुरुवः पाण्डुपुत्रंरथैर्महार्हैः शरवर्षं सृजन्तः}


\twolineshloka
{ततस्तु तस्मिंस्तुमले समुत्थितेसुदारुणे भारतमोहनीये}
{नो मुह्यत प्राप्य स राजपुत्रःकिरीटमाली व्यसृजच्छरौघान्}


\twolineshloka
{वधप्रेप्सुः सव्यसाची कुरूणांस्मरन्क्लेशान्द्वादशवर्षवृत्तान्}
{गाण्डीवमुक्तैरिषुभिर्महात्मासर्वा दिशो व्यावृणोदप्रमेयः}


\twolineshloka
{प्रदीप्तोल्कमभवच्चान्तरिक्षंमृतेषु देहेष्वपतन्वयांसि}
{यत्पिङ्गलज्येन किरीटमालीक्रुद्धो रिपूनाजगवेन हन्ति}


\twolineshloka
{ततः किरीटि महता महायशाःशरासनेनास्य शराननीकजित्}
{हयप्रवेकोत्तमनागघूर्णिता--न्कुरुप्रवीरानिषुभिर्व्यापातयत्}


\twolineshloka
{गदाश्च गुर्वीः परिघानयस्मया--नसींश्च शक्तीश्च रणे नराधिपाः}
{महान्ति शस्त्राणि च भीमदर्शनाःप्रगृह्य पार्थं सहसाऽभिदुद्रुवुः}


\twolineshloka
{ततो युगान्ताभ्रसमस्वनं मह--न्महेन्द्रचापप्रतिमं च गाण्डिवम्}
{चकर्ष दोर्भ्यों विहसन्भृशं ययौदहंस्त्वहीयान्यमराष्ट्रवर्धनः}


\twolineshloka
{स तानुदीर्णान्सरथान्सवारणा--न्पदातिसङ्घांश्च महाधनुर्धरः}
{विपन्नसर्वायुधजीवितान्रणेचकार वीरो यमराष्ट्रवर्धनान्}


\chapter{अध्यायः १४७}
\twolineshloka
{सञ्जय उवाच}
{}


\twolineshloka
{श्रुत्वा निनादं धनुषश्च तस्यविस्पषृमुत्कष्टमिवान्तकस्य}
{शक्राशनिस्फोटसमं सुघोरंविकृष्यमाणस्य धनञ्जयेन}


\threelineshloka
{त्रासोद्विग्नं तथोद्धान्तं त्वदीयं तद्बलं नृप}
{युगान्तवातसङ्क्षुब्धं चलद्वीचितरङ्गितम्}
{प्रलीनमीनमकरं सागराम्भ इवाभवत्}


\twolineshloka
{`मध्यन्दिनगतं सूर्यमापतन्तमिवाम्बरे}
{न शेकुः सर्वभूतानि पाण्डवं प्रतिवीक्षितुम्'}


\twolineshloka
{स रणे व्यचरत्पार्थः प्रेक्ष्यमाणो धनञ्जयः}
{युगपद्दिक्षु सर्वासु सर्वाण्यस्त्राणि दर्शयन्}


\twolineshloka
{आददानं महाराज सन्दधानं च पाण्डवम्}
{उत्कर्षन्तं सृजन्तं च न स्म पश्याम लाघवात्}


\twolineshloka
{[ततः क्रुद्धो महाबाहुरैन्द्रमस्त्रं दुरासदम्}
{प्रादुश्चक्रे महाराज त्रासयन्सर्वभारतान्}


\twolineshloka
{ततः शराः प्रादुरासन्दिव्यास्त्रप्रतिमन्त्रिताः}
{प्रदीपाश्च शिखिमुखाः शतशोऽथ सहस्रशः}


\twolineshloka
{आकर्णपूर्णनिर्मुक्तैरग्न्यर्कांशुनिभैः शरैः}
{नभोऽभवत्तद्दुष्प्रेक्ष्यमुल्काभिरिव संवृतम्}


\twolineshloka
{ततः शस्त्रान्धकारं तत्कौरवैः समुदीरितम्}
{अशक्यं मनसाऽप्यन्यैः पाण्डवः सम्भ्रमन्निव}


\twolineshloka
{नाशयामास विक्रम्य शरैर्दिव्यास्त्रमन्त्रितैः}
{नैशं तमोंशुभिः क्षिप्रं दिनादाविव भास्करः}


\twolineshloka
{ततस्तु तावकं सैन्यं दीप्तैः शरगभस्तिभिः}
{आक्षिपत्पल्वलाम्बूनि निदाघार्क इव प्रभुः}


\twolineshloka
{ततो दिव्यास्त्रविदुषा प्रहिताः सायकांशवः}
{समाप्लवन्द्विषत्सैन्यं लोकं भागोरिवांशवः}


\twolineshloka
{अथापरे समुत्सृष्टा विशिखास्तिग्मतेजसः}
{हृदयान्याशु वीराणां विविशुः प्रियबन्धुवत्}


\twolineshloka
{य एनमीयुः समरे त्वद्योधाः शूरमानिनः}
{शलभा इव ते दीप्तमग्निं प्राप्य ययुः क्षयम्}


\twolineshloka
{एवं स मृद्रञ्शत्रूणां जीवितानि यशांसि च}
{पार्थश्चचार सङ्ग्रामे मृत्युर्विग्रहवानिव}


\twolineshloka
{स किरीटानि वस्त्राणि साङ्गदान्विपुलान्भुजान्}
{सकुण्डलयुगान्कर्णान्केषाञ्चिदहरच्छरैः}


\threelineshloka
{सतोमरान्गजस्थानां सप्रासान्हयसादिनाम्}
{सचर्मणः पदातीनां रथिनां च सधन्वनः}
{सप्रतोदान्नियन्तॄणां बाहूंश्चिच्छेद पाण्डवः}


\twolineshloka
{प्रदीप्तोग्रशरार्जिष्मान्बभौ तत्र धनञ्जयः}
{सविस्फुलिङ्गाग्रशिखो ज्वलन्निव हुताशनः}


\twolineshloka
{तं देवराजप्रतिमं सर्वशस्त्रभृतां वरम्}
{युगपद्दिक्षु सर्वासु रथस्थं पुरुषर्षभम्}


\twolineshloka
{निक्षिपन्तं महास्त्राणि प्रेक्षणीयं धनञ्जयम्}
{नृत्यन्तं रथमार्गेषु धनुर्ज्यातलनादिनम्}


\twolineshloka
{निरीक्षितुं न शेकुस्ते यत्नवन्तोऽपि पार्थिवाः}
{मध्यन्दिनगतं सूर्यं प्रतपन्तमिवाम्बरे}


\twolineshloka
{दीप्तोग्रसम्भृतशरः किरीटी विरराज ह}
{वर्षास्विवोदीर्णजलः सेन्द्रधन्वाम्बुदो महान्}


\twolineshloka
{महास्त्रसंप्लुवे तस्मिञ्जिष्णुना सम्प्रवर्तिते}
{सुदुस्तरे महाघोरे ममज्जुर्योधपुङ्गवाः}


\twolineshloka
{उत्कृत्तवदनैर्देहैः शरीरैः कृत्तबाहुभिः}
{भुजैश्च पाणिनिर्मुक्तैः पाणिभिर्व्यङ्गुलीकृतैः}


\twolineshloka
{कृत्ताग्रहस्तैः करिभिः कृत्तदन्तैर्मदोत्कटैः}
{हयैश्च विधुरग्रीवै रथैश्च शकलीकृतैः}


\twolineshloka
{निकृत्तान्त्रैः कृत्तपादैस्तथाऽन्यैः कृत्तसन्धिभिः}
{निश्चेष्टैर्विस्फुरद्भिश्च शतशोऽथ सहस्रशः}


\twolineshloka
{मृत्योराघातललितं तत्पार्थायोधनं महत्}
{अपश्याम महीपाल भीरूणां भयवर्धनम्}


\twolineshloka
{आक्रीडमिव रुद्रस्य पुराभ्यर्दयतः पशून्}
{गजानां क्षुरनिर्मुक्तैः करैः सभुजगेव भूः}


\twolineshloka
{क्वजिद्बभौ स्रग्विणीव वक्त्रपद्मैः समाचिता}
{विचित्रोष्णीषमुकुटैः केयूराङ्गदकुण्डलैः}


\threelineshloka
{स्वर्णचित्रतनुत्रैश्च भाण्डैश्च गजवाजिनाम्}
{किरीटशतसङ्गीर्णा तत्र तत्र समाचिता}
{विराज भृशं चित्रा मही नववधूरिव}


\twolineshloka
{मज्जामेदःकर्दमिनीं शोणितौघतरङ्गिणीम्}
{मर्मास्थिभिरगाधां च केशशैवलशाद्वलाम्}


\twolineshloka
{शिरोबाहूपलतटां रुग्णक्रोडास्थिसङ्कटाम्}
{चित्रध्वजपताकाढ्यां छत्राचापोर्मिमालिनीम्}


\twolineshloka
{विगतासुमहाकायां गजदेहाभिसङ्कुलाम्}
{रथोडुपशताकीर्णां हयसङ्घातरोधसम्}


\twolineshloka
{रथचक्रयुगेषाक्षकूबरैरतिदुर्गमाम्}
{प्रासासिशक्तिपरशुविशिखाहिदुरासदाम्}


\twolineshloka
{बलकङ्कुमहानक्रां गोमायुमकरोत्कटाम्}
{गृघ्रोदग्रमहाग्राहां शिवाविरुतभैरवाम्}


\twolineshloka
{नृत्यत्प्रेतपिशाचाद्यैर्भूताकीर्णां सहस्रशः}
{गतासुयोधनिश्चेष्टशरीरशतवाहिनीम्}


\twolineshloka
{महाप्रतिभयां रौद्रां घोरां वैतरणीमिव}
{नदीं प्रवर्तयामास भीरूणां भयवर्धिनीम्}


\twolineshloka
{तं दृष्ट्वा तस्य विक्रान्तमन्तकस्येव रूपिणः}
{अभूतपूर्वं कुरुषु भयमागाद्रणाजिरे}


\twolineshloka
{तत आदाय वीराणामस्त्रैरस्त्राणि पाण्डवः}
{आत्मानं रौद्रमाचष्ट रौद्रकर्मण्यधिष्ठितः}


% Check verse!
ततो रथवरान्राजन्नत्यतिक्रामदर्जुनः ॥]
\twolineshloka
{मध्यन्दिनगतं सूर्यं प्रतपन्तमिवाम्बरे}
{न शेकुः सर्वभूतानि पाण्डवं प्रतिवीक्षितुम्}


\twolineshloka
{प्रसृतांस्तस्य गाण्डीवाच्छरव्रातान्महात्मनः}
{सङ्ग्रामे सम्प्रपश्यामो हंसपङ्क्तिरिवाम्बरे}


\twolineshloka
{विनिवार्य स वीराणामस्त्रैरस्त्राणि सर्वतः}
{दर्शयन्रौद्रमात्मानमुग्रे कर्मणि धिष्ठितः}


\twolineshloka
{स तान्रथवरान्राजन्नत्याक्रामत्तदाऽर्जुनः}
{मोहयन्निव नाराचैर्जयद्रथवधेप्सया}


\twolineshloka
{विसृजन्दिक्षु सर्वासु शरानच्युतसारथिः}
{सरथो व्यचरत्तूर्णं प्रेक्षणीयो धनञ्जयः}


\twolineshloka
{भ्रमन्त इव शूरस्य शरव्राताः महात्मनः}
{अदृश्यन्तान्तरिक्षस्थाः शतशोऽथ सहस्रशः}


\twolineshloka
{आददानं महेष्वासं सन्दधानं च सायकम्}
{विसृजन्तं च कौन्तेयं नानुपश्याम वै तदा}


\threelineshloka
{तथा सर्वा दिशो राजन्सर्वांश्च रथिनो रणे}
{आकुलीकृत्य कौन्तेयो जयद्रथमुपाद्रवत्}
{विव्याध च चतुःषष्ट्या शराणां नतपर्वणाम्}


\twolineshloka
{[सैन्धवाभिमुखं यान्तं योधाः सम्प्रेक्ष्य पाण्डवम्}
{न्यवर्तन्त रणाद्वीरा निराशास्तस्य जीविते}


\twolineshloka
{यो योऽभ्यधावदाक्रन्दे तावकः पाण्डवं रणे}
{तस्यतस्यान्तगा बाणाः शरीरे न्यपतन्प्रभो}


\twolineshloka
{कबन्धसङ्कुलं चक्रे तव सैन्यं महारथः}
{अर्जुनो जयतां श्रेष्ठः शरैरग्न्यंशुसन्निभैः}


\twolineshloka
{एवं तत्तव राजेन्द्र चतुरङ्गबलं तदा}
{व्याकुलीकृत्य कौन्तेयो जयद्रथमुपाद्रवत्}


\twolineshloka
{द्रौणिं पञ्चाशताऽविध्यद्वृषसेनं त्रिभिः शरैः}
{कृपायमाणः कौन्तेयः कृपं नवभिरार्दयत्}


\twolineshloka
{शल्यं षोडशभिर्बाणैः कर्णं द्वात्रिंशता शरैः}
{सैन्धवं तु चतुःषष्ट्या विद्व्वा सिंह इवानदत् ॥]}


\twolineshloka
{सैन्धवस्तु तथा विद्धः शरैर्गाण्डीवधन्वना}
{न चक्षमे सुसङ्क्रुद्धस्तोत्रार्दित इव द्विपः}


\twolineshloka
{स वराहध्वजस्तूर्णं गार्ध्रपत्रानजिह्मगान्}
{क्रुद्धाशीविषसङ्काशान्कर्मारपरिमार्जितान्}


\twolineshloka
{आकर्णपूर्णांश्चिक्षेप फल्गुनस्य रथं प्रति}
{त्रिभिस्तु विद््वा गोविन्दं नाराचैः षड्भिरर्जुनम्}


\twolineshloka
{अष्टभिर्वाजिनोऽविध्यद्ध्वजं चैकेन पत्रिणा}
{`भूयश्चैवार्जुनं सङ्ख्ये शरवर्षैरवाकिरत्'}


\threelineshloka
{स विक्षिप्यार्जुनस्तूर्णं सैन्धवप्रहिताञ्शरान्}
{युगपत्तस्य चिच्छेद शराभ्यां सैन्धवस्य ह}
{सारथेश्च शिरः कायाद्व्वजं च समलङ्कृतम्}


\twolineshloka
{स च्छिन्नयष्टिः सुमहान्धनञ्जयशराहतः}
{वराहः सिन्धुराजस्य पपातेन्दुसमप्रभः}


\twolineshloka
{आदित्यं प्रेक्षमाणस्तु बीभत्सुः सृक्विणी लिहन्}
{अपश्यन्नान्तरं तस्य रक्षिभिः संवृतस्य वै}


\twolineshloka
{अभवक्त्रोधरक्ताक्षो व्यात्तानन इवान्तकः}
{अथाब्रवीद्वासुदेवः कुन्तीपुत्रं धनञ्जयम्}


\threelineshloka
{नैव शक्यस्त्वया हन्तुं निर्व्याजं भरतर्षभ}
{स्रक्ष्याम्यहमुपायं तमादित्यस्यापवारणे}
{ततोऽस्तं गतमादित्यं मंस्यते सिन्धुराडिह}


\twolineshloka
{ततोऽस्य विस्मयः पार्थ हर्षश्चैव भविष्यति}
{आत्मजीवितलाभाच्च प्रतिज्ञायाश्च नाशनात्}


\twolineshloka
{अस्तङ्गतमिवादित्यं दृष्ट्वा मोहेन बालिशः}
{न हि शक्ष्यत्यथाऽऽत्मानं रक्षितुं हर्षसम्भवात्}


\threelineshloka
{एतस्मिन्नेव काले तु प्रहर्तव्यं धनञ्जय}
{जयद्रथस्य क्षुद्रस्य सविदुर्दर्शनार्थिनः ॥सञ्जय उवाच}
{}


\twolineshloka
{एवमुक्त्वा ततः पार्थं क्षिप्रमेवाहरत्प्रभाम्}
{जनार्दनेन सृष्टं वै तम आदित्यनाशनम्}


\twolineshloka
{पार्थस्य बलमज्ञानात्तमो दृष्ट्वा सुदुःखितम्}
{अभवंस्तावका योधा हर्षसङ्कुलचेतसः}


\twolineshloka
{ते प्रहृष्टा रणे राजन्नपश्यन्सैनिका रविम्}
{उन्नाम्य वक्त्राणि तदा स च राजा जयद्रथः}


\threelineshloka
{वीक्षमाणे ततस्तस्मिन्सिन्धुराजे दिवाकरम्}
{उन्नमय्य शिरोग्रीवां वीक्षते सूर्यमण्डलम्}
{तस्य शीघ्रं पृषत्केन कायाच्छीर्षमपाहर}


\twolineshloka
{केशवेनैवमुक्तः सन्नमर्षाद्रक्तलोचनः}
{उद्बबर्ह शरं तीक्ष्णममरैरपि दुःसहम्}


\threelineshloka
{एवं तूणीशयं घोरमिन्द्राशनिसमप्रभम्}
{वर्मभेदिनमत्यर्थं गन्धमाल्यार्चितं सदा}
{विससर्ज ततस्तूर्णं सैन्धवस्य रथं प्रति}


\twolineshloka
{स तु गाण्डीवनिर्मुक्तः शरः श्येन इवाशुनः}
{शकुन्तमिव वृक्षाग्रात्सैन्धवस्याहरच्छिरः}


\threelineshloka
{प्रपतिष्यति शीर्षे तु कृष्णोऽर्जुनमथाब्रवीत्}
{पार्थ पार्थ शिरो ह्येतद्यथा नेयान्महीतलम्}
{तथा कुरु कुरुश्रेष्ठ वक्ष्ये तस्यापि कारणम्}


\threelineshloka
{श्रुत्वा तु वचनं तस्य त्वरमाणोऽस्त्रमायया}
{तथाऽहरच्छिरस्तस्य शरैरूर्ध्वं धनञ्जयः}
{दुर्हृदामपहर्षाय सुहृदां हर्षणाय च}


\twolineshloka
{तिर्यगूर्ध्वमधश्चैव पुनरूर्ध्वमथापि च}
{दीर्घकालमवाक्वैव सम्प्रक्रीडन्निवार्जुनः}


\twolineshloka
{तत्सैन्यं सर्वतोपश्यन्महदाश्चर्यमद्भुतम्}
{प्रापयत्स शिरो यस्माद्योधयन्नेव पार्थिवान्}


\twolineshloka
{शरैः कन्दुकवत्कृत्वा तस्यालोकाय पाण्डवः}
{स्यमन्तपञ्चकाद्बाह्ये शिरस्तदहरच्छरैः}


\twolineshloka
{जयद्रथस्यार्जुनबाणनालंमुखारविन्दं रुधिराम्बुसिक्तम्}
{दृष्टं नरैश्चोपरिवर्तमानंविद्याधरोत्सृष्टमिवैकपद्मम्}


\twolineshloka
{स देवशत्रूनिव देवराजःकिरीटमाली व्यधमत्सपत्नान्}
{यथा तमांस्यभ्युदितस्तमोघ्नःपूर्णां प्रतिज्ञां स समाप्य वीरः}


\threelineshloka
{अथाब्रवीत्केशवं पाण्डवेयःकियन्तमध्यवानमिदं हरामि}
{किमर्थमेतन्न निपात्यमुर्व्यांक्व च प्रयातव्यमिदं च शंस}
{शक्नोम्यहं यत्र भवान्ब्रवीतितं भूमिदेशं च शिरो विनेतुम्}


\chapter{अध्यायः १४८}
\twolineshloka
{सञ्जय उवाच}
{}


\twolineshloka
{एवस्मिन्नेव काले तु त्वरमाणे दिवाकरे}
{अब्रवीत्पाण्डवं तत्र त्वरमाणो जनार्दनः}


\twolineshloka
{पार्थ पार्थ शिरो ह्येनद्यत्कृते न पतेद्भुवि}
{श्रूयतां तद्यथावृत्तं कारणं सैन्धवं *प्रति}


\twolineshloka
{वृद्धक्षत्रः सैन्धवस्य पिता जगति विश्रुतः}
{स च घोरेण तपसा दैन्धवं प्राप्तवान्सुतम्}


\twolineshloka
{जयद्रथममित्रघ्नं वागुवाचाशरीरिणी}
{नृपमन्तर्हिता वाणी मेघदुन्दुभिनिःस्वना}


\threelineshloka
{तवात्मजो मनुष्येन्द्र कुलशीलदमादिभिः}
{गुणैर्भविष्यति विभो सदृशो वंशयोर्द्वयोः}
{क्षत्रियप्रवरो लोके नित्यं शूराभिसत्कृतः}


% Check verse!
शिरश्छेत्स्यति सङ्क्रुद्धः शत्रुश्चालक्षितो भुवि
\twolineshloka
{एतच्छ्रुत्वा सिन्धुराजो ध्यात्वा चिरमरिन्दमः}
{ज्ञातीन्सर्वानुवाचेदं पुत्रस्नेहाभिचोदितः}


\twolineshloka
{सङ्ग्रामे युध्यमानस्य वहतो महतीं धुरम्}
{धरण्यां मम पुत्रस्य पातयिष्यति यः शिरः}


\twolineshloka
{तस्यापि शतधा मूर्धा फलिष्यति न संशयः}
{`यदि चेदस्ति मे लाभस्तपसो वा दमस्य वा'}


\twolineshloka
{एवमुक्त्वा ततो राज्ये स्थापयित्वा जयद्रथम्}
{वृद्धक्षत्रो वनं यातस्तपश्चोग्रं समास्थितः}


\twolineshloka
{सोऽयं तप्यति तेजस्वी तपो घोरं दुरासदम्}
{स्यमन्तपञ्चकादस्माद्बहिर्वानरकेतन}


\twolineshloka
{तस्माज्जयद्रथस्य त्वं शिरश्छित्त्वा महामृधे}
{दिव्येनास्त्रेण रिपुहन्घोरेणाद्भुतकर्मणा}


\twolineshloka
{सकुण्डलं सिन्धुपतेः प्रभञ्जनसुतानुज}
{उत्सङ्गे पातयस्वास्य वृद्धक्षत्रस्य भारत}


\twolineshloka
{अथ त्वमस्य मूर्धानं पातयिष्यसि भूतले}
{तवापि शतधा मूर्धा फलिष्यति न संशयः}


\twolineshloka
{यथा चेदं न जानीयात्स राजा तपसि स्थितः}
{तथा कुरु कुरुश्रेष्ठ दिव्यमस्त्रमुपाश्रितः}


\twolineshloka
{न ह्यसाध्यमकार्यं वा विद्यते तव किञ्चन}
{समस्तेष्वपि लोकेषु त्रिषु वासवनन्दन}


\threelineshloka
{`यथैतत्सैन्धवशिरः शरैरेव धनञ्जय}
{वृद्धक्षत्रे पतत्येव तथा नीतिर्विधीयताम्' ॥सञ्जय उवाच}
{}


\twolineshloka
{ततः सुमहदाश्चर्यं तत्रापश्याम भारत}
{स्यमन्तपञ्चकाद्बाह्यं शिरो यद्व्यहरत्ततः}


\twolineshloka
{एतस्मिन्नेव काले तु वृद्धक्षत्रो महीपतिः}
{सन्ध्यामुपास्ते तेजस्वी सम्बन्धी तव मारिष}


\twolineshloka
{उपासीनस्य तस्याथ कृष्णकेशं सकुण्डलम्}
{सिन्धुराजस्य मूर्धानमुत्सङ्गे समपातयत्}


\twolineshloka
{तस्योत्सङ्गे निपतितं शिरस्तच्चारुकुण्डलम्}
{वृद्धक्षत्रस्य नृपतेरलक्षितमरिन्दम}


\twolineshloka
{कृतजप्यस्य तस्याथ वृद्धक्षत्रस्य भारत}
{प्रोत्तिष्ठतस्तत्सहसा शिरोऽगच्छद्वरातलम्}


\twolineshloka
{ततस्तस्य नरेन्द्रस्य पुत्रमूर्धनि भूतले}
{गते तस्यापि शतधा मूर्धाऽगच्छदरिन्दम}


\twolineshloka
{ततः सर्वाणि सैन्यानि विस्मयं जग्मुरुत्तमम्}
{वासुदेवश्च बीभत्सुं प्रशशंस महारथम्}


\twolineshloka
{`ततो दृष्ट्वा विनिहतं सिन्धुराजं जयद्रथम्}
{कर्मणा तेन पार्थस्य विस्मिताः सर्वदेवताः}


\twolineshloka
{सर्वदा समरे यस्य गोप्ता नित्यं जनार्दनः}
{कथं तस्य जयो न स्यादिति भूतानि मेनिरे}


\twolineshloka
{एतदर्थं शिरस्तस्य व्यापयामास पाण्डवः}
{स्यमन्तपञ्चकाद्बाह्यं शरैरेव यथाक्रमम्}


\twolineshloka
{कृत्वा तच्च महत्कर्म निहत्य च जयद्रथम्}
{अस्त्रं पाशुपतं पार्थः संहर्तुमुपचक्रमे}


\twolineshloka
{संहरत्यपि कौन्तेये तदस्त्रं तत्र भारत}
{ववौ शीतः सुगन्धश्च पवनो ह्लादयन्निव}


\twolineshloka
{संहारं च प्रमोक्षं च दृष्ट्वा तत्र दिवौकसः}
{विस्मयं परमं जग्मुः प्रशशंसुश्च पाण्डवम्}


\twolineshloka
{एवमस्त्रेण तान्वीरो योधयित्वा धनञ्जयः}
{जयद्रथशिरः पश्चाद्व्यापयामास पाण्डवः}


\twolineshloka
{तच्छिरश्च्यावमानं तु ददृशुस्तावका युधि}
{शल्यकर्णकृपा राजन्मोहिताः सव्यसाचिना}


\twolineshloka
{शिरसि च्याविते तस्य शरैराशीविषोपमैः}
{पश्चात्कायोऽपतद्भूमिं शोचयन्सर्वपार्थिवान्}


\twolineshloka
{दृष्ट्वा तु निहतं सङ्ख्ये सिन्धुराजं महारथम्}
{पुत्राणां तव नेत्रेभ्यो दुःखादास्रं प्रवर्तत'}


\twolineshloka
{ततो विनिहते राजन्सिन्धुराजे किरीटिना}
{तमस्तद्वासुदेवेन संहृतं भरतर्षभ}


\twolineshloka
{पश्चाज्ज्ञातं महीपाल तव पुत्रैः सहानुगैः}
{वासुदेवप्रयुक्तेयं मायेति नृपसत्तम}


\twolineshloka
{एवं स निहतो राजन्पार्थेनामिततेजसा}
{अक्षौहिणीरष्ट हत्वा जामाता तव सैन्धवः}


\twolineshloka
{हतं जयद्रथं दृष्ट्वा तव पुत्रा नराधिप}
{दुःखादश्रूणि मुमुचुर्निराशाश्चाभवञ्जये}


\twolineshloka
{ततो जयद्रथे राजन्हते पार्थेन केशवः}
{दध्मौ शङ्खं महाबाहुरर्जुनश्च परन्तपः}


\twolineshloka
{भीमश्च वृष्णिसिंहश्च युधामन्युश्च भारत}
{उत्तमौजाश्च विक्रान्तः शङ्खान्दध्मुः पृथक्पृथक्}


\twolineshloka
{श्रुत्वा महान्तं तं शब्दं धर्मराजो युधिष्ठिरः}
{सैन्धनं निहतं मेने फल्गुनेन महात्मना}


\twolineshloka
{ततो वादित्रघोषेण स्वान्योधान्पर्यहर्षयत्}
{अभ्यवर्तत सङ्ग्रामे भारद्वाजं युयुत्सया}


\twolineshloka
{ततः प्रववृते राजन्नस्तं गच्छति भास्करे}
{द्रोणस्य सोमकैः सार्धं सङ्ग्रामो रोमहर्षणः}


\twolineshloka
{ते तु सर्वे प्रयत्नेन भारद्वाजं जिघांसवः}
{सैन्धवे निहते राजन्नयुध्यन्त महारथाः}


\twolineshloka
{पाण्डवास्तु जयं लब्ध्वा सैन्धवं विनिहत्य च}
{अयोधयंस्तु ते द्रोणं जयोऽन्मत्तास्ततस्ततः}


\twolineshloka
{अर्जुनोऽपि ततो योधांस्तावकान्रथसत्तमान्}
{अयोधयन्महाबाहुर्हत्वा सैन्धवकं नृपम्}


\chapter{अध्यायः १४९}
\twolineshloka
{धृतराष्ट्र उवाच}
{}


\twolineshloka
{तस्मिन्विनिहते वीरे सैन्धवे सव्यसाचिना}
{मामका यदकुर्वन्त तन्ममाचक्ष्व सञ्जय}


\twolineshloka
{`पश्यतां सर्वसैन्यानां मामकानां महारणे}
{अहन्यत कथं युद्धे सैन्धवः सव्यसाचिना}


\twolineshloka
{कथं द्रौणिकृपैर्गुप्तः कर्णेन च महारणे}
{फल्गुनाग्निमुखं घोरं प्रविष्टः साधु सैन्धवः}


\threelineshloka
{तस्मिन्हते महेष्वासे मन्दात्मा स सुयोधनः}
{भ्रातृबिः सहितः सूत किमकार्षीदनन्तरम्' ॥सञ्जय उवाच}
{}


\twolineshloka
{सैन्धवं निहतं दृष्ट्वा रणे पार्थेन भारत}
{अमर्षवशमापन्नः कृपः शारद्वतस्ततः}


\twolineshloka
{महता शरवर्षेण पाण्डवं समवाकिरत्}
{द्रौणिश्चाभ्यद्रवद्राजन्रथमास्थाय फल्गुनम्}


\twolineshloka
{तावेतौ रथिनां श्रेष्ठौ रथाभ्यां रथसत्तमौ}
{उभावुभयतस्तीक्ष्णैर्विशिखैरभ्यवर्षताम्}


\twolineshloka
{स तथा शरवर्षाभ्यां सुमहद्ध्यां महाभुजः}
{पीड्यमानः परामार्तिमगमद्रथिनां वरः}


\twolineshloka
{सोऽजिघांसुर्गुरुं सङ्ख्ये गुरोस्तनयमेव च}
{चकाराचार्यकं तत्र कुन्तीपुत्रो धनञ्जयः}


\twolineshloka
{अस्त्रैरस्त्राणि संवार्य द्रौणेः शारद्वतस्य च}
{मन्दवेगानिषूंस्ताभ्यामजिघांसुरवासृजत्}


\twolineshloka
{तेन नातिभृशं त्रस्तौ विशिखैर्भृशपीडितौ}
{बहुत्वात्तु परामार्तिं शराणां तावगच्छताम्}


\twolineshloka
{अथ शारद्वतो राजन्कौन्तेयशरपीडितः}
{अवासीदद्रथोपस्थे मूर्च्छामभिजगाम ह}


\twolineshloka
{विह्वलं तमभिज्ञाय भर्तारं शरपीडितम्}
{हतोऽयमिति च ज्ञात्वा सारथिस्तमपावहत्}


\twolineshloka
{तस्मिन्भग्ने महाराज कृपे शारद्वते युधि}
{अश्वत्थामाप्यपायासीत्पाण्डवेयाद्रथान्तरम्}


\threelineshloka
{दृष्ट्वा शारद्वतं पार्थो मूर्च्छितं शरपीडितम्}
{धिग्धिङ्मामिति चैवोक्त्वा कृपणं पर्यदेवयत्}
{अश्रुपूर्णमुखो दीनो वचनं चेदमब्रवीत्}


\twolineshloka
{पश्यन्निदं महाप्राज्ञः क्षत्ता राजानमुक्तवान्}
{कुलान्तकरणे पापे जातमात्रे सुयोधने}


\twolineshloka
{नीयतां परलोकाय साध्वयं कुलपांसनः}
{अस्माद्धि कुरुमुख्यानां महदुत्पत्स्यते भयम्}


\twolineshloka
{तदिदं समनुप्राप्तं वचनं सत्यवादिनः}
{तत्कृते ह्यद्य पश्यामि शरतल्पगतं गुरुम्}


\twolineshloka
{धिगस्तु क्षात्रमाचारं धिगस्तु बलपौरुषम्}
{को हि ब्राह्मणमाचार्यमभिद्रुह्येत मादृशः}


\twolineshloka
{ऋषिपुत्रो ममाचार्यो द्रोणस्य परमः सखा}
{एष शेते रथोपस्थे कृपो मद्बाणपीडितः}


\twolineshloka
{अकामयानेन मया विशिखैरर्दितो भृशम्}
{अवसीदन्रथोपस्थे प्राणान्पीडयतीव मे}


\twolineshloka
{पुत्रशोकाभितप्तेन शरैरभ्यर्दितेन च}
{अभ्यस्तो बहुभिर्बाणैर्दशधर्मगतेन वै}


\threelineshloka
{`शरार्दितः स हि मया प्रेक्षमाणो महाद्युतिः'}
{शोचयत्येष नियतं भूयः पुत्रवधाद्धि माम्}
{कृपणं स्वरथे सन्नं पश्य कृष्ण यथागतम्}


\twolineshloka
{उपाकृत्य तु वै विद्यामाचार्येभ्यो नरर्षभाः}
{प्रयच्छन्तीह ये कामान्देवत्वमुपयान्ति ते}


\twolineshloka
{ये च विद्यामुपादाय गुरुभ्यः पुरुषाधमाः}
{घ्नन्ति तानेव दुर्वृत्तास्ते वै निरयगामिनः}


\twolineshloka
{तदिदं नरकायाद्य कृतं कर्म मया ध्रुवम्}
{आचार्यं शरवर्षेण रथे सादयता कृपम्}


\twolineshloka
{यत्तत्पूर्वमुपाकुर्वन्नस्त्रं मामब्रवीत्कृपः}
{न कथञ्चन कौरव्य प्रहर्तव्यं गुराविति}


\twolineshloka
{तदिदं वचनं साधोराचार्यस्य महात्मनः}
{नानुष्ठितं तमेवाजौ विशिखैरभिवर्षता}


\threelineshloka
{नमस्तस्मै सुपूज्याय गौतमायापलायिने}
{धिगस्तु मम वार्ष्णेय यदस्मै प्रहराम्यहम् ॥सञ्जय उवाच}
{}


\threelineshloka
{तथा विलपमाने तु सव्यसाचिनि तं प्रति}
{*ततो राजन्हृष्टीकेशः संङ्गामशिरसि स्थितम्}
{तीर्णप्रतिज्ञं बीभत्सुं परिष्वज्यैनमब्रवीत्}


\twolineshloka
{दिष्ठ्या सम्पादिता जिष्णो प्रतिज्ञा महती त्वया}
{दिष्ट्या विनिहतः पापो वृद्धक्षत्रः सहात्मजः}


\twolineshloka
{धार्तराष्ट्रबलं प्राप्य देवसेनापि भारत}
{सीदेत समरे जिष्णो नात्र कार्या विचारणा}


\twolineshloka
{न तं पश्यामि लोकेषु चिन्तयन्पुरुषं क्वचित्}
{त्वदृते पुरुषव्याघ्र य एतद्योधयेद्बलम्}


\twolineshloka
{महाप्रभावा बहवस्त्वया तुल्याधिकापि वा}
{समेताः पृथिवीपाला धार्तराष्ट्रस्य कारणात्}


\twolineshloka
{ते त्वां प्राप्य रणे क्रुद्धा नाभ्यवर्तन्त दंशिताः}
{तव वीर्यं बलं चैव रुद्रशक्रान्तकोपमम्}


\twolineshloka
{नेदृशं शक्नुयात्कश्चिद्रमे कर्तुं पराक्रमम्}
{यादृशं कृतवानद्य त्वमेकः शत्रुतापनः}


\twolineshloka
{एवमेव हते कर्णे सानुबन्धे दुरात्मनि}
{वर्धयिष्यामि भूयस्त्वां विजितारिं हतद्विषम्}


\twolineshloka
{तमर्जुनः प्रत्युवाच प्रसादात्तव माधव}
{प्रतिज्ञेयं मया तीर्णा विबुधैरपि दुस्तरा}


\twolineshloka
{अनाश्चार्यो जयस्तेषां येषां नाथोऽसि केशव}
{त्वत्प्रसादान्महीं कृत्स्नां संप्राप्स्यति युधिष्ठिरः}


\twolineshloka
{तवैष भारो धार्ष्णेय तवैव विजयः प्रभो}
{वर्धनीयास्तव वयं प्रेष्याश्च मधुसूदन}


\threelineshloka
{एवमुक्तस्ततः कृष्णः शनकैर्वाहयन्हयान्}
{दर्शयामास पार्थाय क्रूरमायोधनं महत् ॥श्रीकृष्ण उवाच}
{}


\twolineshloka
{प्रार्थयन्तो जये युद्धे प्रथितं च महद्यशः}
{पृथिव्यां शेरते शूराः पार्थिवास्त्वच्छरैर्हताः}


\twolineshloka
{विकीर्णशस्त्राभरणा विपन्नाश्वरथद्विपाः}
{सञ्छिन्नभिन्नमर्माणो वैक्लव्यं परमं गताः}


\twolineshloka
{ससत्वा गतसत्वाश्च प्रभया परया युताः}
{सजीवा इव लक्ष्यन्ते गतसत्वा नराधिपाः}


\twolineshloka
{तेषां शरैः स्वर्णपुङ्खैः शस्त्रैश्च विविधैः शितैः}
{वाहनैरायुधैश्चैव सम्पूर्णां पश्य मेदिनीम्}


\twolineshloka
{वर्मभिश्चर्मभिर्हारैः शिरोभिश्च सकुण्डलैः}
{उष्णीषैर्मकुटैः स्रग्भिश्चूडामणिभिरम्बरैः}


\twolineshloka
{कण्ठसूत्रैरङ्गदैश्च निष्कैरपि च सप्रभैः}
{अन्यैश्चाभरणैश्चित्रैर्भाति भारत मेदिनी}


\twolineshloka
{[अनुकर्षैरुपासङ्गैः पताकाभिर्ध्वजैस्तथा}
{उपस्करैरधिष्ठानैरीषादण्डकबन्धुरैः}


\twolineshloka
{चक्रैः प्रमथितैश्चित्रैरक्षैश्च बहुधा रणे}
{युगैर्योक्त्रैः कलापैश्च धनुर्भिः सायकैस्तथा}


\twolineshloka
{परिस्तोमैः कुथाभिश्च परिघैरङ्कुशैस्तथा}
{शक्तिभिर्भिण्डिपालैश्च तूणैः शूलैः परश्वथैः}


\twolineshloka
{प्रासैश्च तोमरैश्चैव कुन्तैर्यष्टिभिरेव च}
{शतघ्नीभिर्भुशुण्डीभिः खङ्गैः परशुभिस्तथा}


\twolineshloka
{मुसलैर्मुद्गरैश्चैव गदाभिः कुणपैस्तथा}
{सुवर्णविकृताभिश्च कशाभिर्भरतर्षभ}


\threelineshloka
{घण्टाभिश्च गजेन्द्राणां भाण्डैश्च विविधैरपि}
{स्रग्भिश्च नानाभरणैर्बस्त्रैश्चैव महाधनैः}
{अपविद्धैर्बभौ भूमिर्ग्रहैर्द्यौरिव शारदी}


\twolineshloka
{पृथिव्यां पृथिवीहेतोः पृथिवीपतयो हताः}
{पृथिवीमुपगुह्याङ्गैः सुप्ताः कान्तामिव प्रियाम्}


\threelineshloka
{इमांश्च गिरिकूटाभान्नागानैरावतोपमान्}
{क्षरतः शोणितं भूरि शस्त्रच्छेददरीमुखैः}
{दरीमुखैरिव गिरीन्गैरिकाम्बुपरिस्रवान्}


\twolineshloka
{तांश्च बाणहतान्वीर पश्य निष्टनतः क्षितौ}
{हयांश्च पतितान्पश्य स्वर्णभाण्डविभूषितान्}


\twolineshloka
{गन्धर्वनगराकारान्रथांश्च निहतेश्वरान्}
{छिन्नध्वजपताकाक्षान्विचक्रान्हतसारथीन्}


\twolineshloka
{निकृत्तकूबरयुगान्भग्नेषान्बन्धुरान्प्रभो}
{पश्य पार्थ हयान्भूमौ विमानोपमदर्शनान्}


\twolineshloka
{पत्तींश्च निहतान्वीर शतशोऽथ सहस्रशः}
{धनुर्भृतश्चर्मभृतः शयानान्रुधिरोक्षितान्}


\twolineshloka
{महीमालिङ्ग्य सर्वाङ्गैः पांसुध्वस्तशिरोरुहान्}
{पश्य योधान्महाबाहो त्वच्छरैर्भिन्नविग्रहान्}


\twolineshloka
{निपातितद्विपरथाजिसङ्कुल--मसृग्वसापिशितसमृद्धकर्दमम्}
{निशाचरश्ववृकपिशाचमोदनंमहीतलं नरवर पश्य दुर्दृशम्}


\threelineshloka
{इदंमहत्त्वय्युपपद्यते प्रभोरणाजिरे कर्म यशोऽभिवर्धनम्}
{शतक्रतौ चापि च देवसत्तमेमहाहवे जघ्नुषि दैत्यदानवान् ॥]सञ्जय उवाच}
{}


\twolineshloka
{एवं सन्दर्शयन्कृष्णो रणभूमिं किरीटिने}
{स्वैः समेतः समुदितैः पाञ्चजन्यं व्यनादयत्}


\twolineshloka
{`सात्यकिः पार्थमभ्यायाद्भीमसेनश्च पाण्डवः}
{युधामन्यूत्तमौजौ च पाञ्चालस्यात्मजावुभौ}


\twolineshloka
{ते निवार्य शरैर्द्रौणिं कर्णं च सह भूमिपैः}
{आगच्छन्रथिनश्चैव यत्र राजा युधिष्ठिरः'}


\twolineshloka
{स दर्शयन्नेव किरीटिनेऽरिहाजनार्दनस्तामरिभूमिमञ्जसा}
{अजातशत्रुं समुपेत्य पाण्डवंनिवेदयामास हतं जयद्रथम्}


\chapter{अध्यायः १५०}
\twolineshloka
{सञ्जय उवाच}
{}


\twolineshloka
{ततो राजानमभ्येत्य धर्मपुत्रं युधिष्ठिरम्}
{ववन्दे सम्प्रहृष्टात्मा हते पार्थेन सैन्धवे}


\twolineshloka
{दिष्ट्या वर्धसि राजेन्द्र हतशत्रुर्नरोत्तम}
{दिष्ट्या निस्तीर्णवांशैव प्रतिज्ञामनुजस्तव}


\threelineshloka
{स त्वेवमुक्तः कृष्णेन हृष्टः परपुरञ्जयः}
{ततो युधिष्ठिरो राजा रथादाप्लुत्य भारत}
{पर्यष्वजत्तदा कृष्णावानन्दाश्रुपरिप्लुतः}


\twolineshloka
{प्रमृज्य वदनं शुभ्रं पुण्डरीकसमप्रभम्}
{अब्रवीद्वासुदेवं च पाण्डवं च धनञ्जयम्}


\twolineshloka
{प्रियमेतदुपश्रुत्य त्वत्तः पुष्करलोचन}
{नान्तं गच्छामि हर्षस्य तितीर्षुरुदधेरिव}


\twolineshloka
{अत्यद्भुतमिदं कृष्ण कृतं पार्थेन धीमता}
{दिष्ट्या पश्यामि सङ्ग्रामे तीर्णभारौ महारथौ}


\twolineshloka
{दिष्ट्या च निहतः पापः सैन्धवः पुरुषाधमः}
{`दिष्ट्या शत्रुगणाश्चैव निमग्नाः शोकसागरे'}


\twolineshloka
{कृष्ण दिष्ट्या मम प्रीतिर्महती प्रतिपादिता}
{त्वया गुप्तेन गोविन्द घ्नता पापं जयद्रथम्}


\threelineshloka
{किन्तु नात्यद्भुतं तेषां येषां नस्त्वं समाश्रयः}
{न तेषां दुष्कृतं किञ्चित्त्रिषु लोकेषु विद्यते}
{सर्वलोकगुरुर्येषां त्वं नाथो मधुसूदन}


\twolineshloka
{त्वत्प्रसादाद्धि गोविन्द वयं जेष्यामहे रिपून्}
{`यथा पुरा प्रसादात्ते दानवान्पाकशासनः}


\twolineshloka
{पृथिव्या विजयो वापि त्रैलोक्यविजयोऽपि वा}
{ध्रुवो हि तेषां वार्ष्णेय येषां तुष्टोऽसि मानद}


\twolineshloka
{न तेषां विद्यते पापं सङ्ग्रामे वा पराजयः}
{त्रिदशेश्वर नाथस्त्वं येषां तुष्टोऽसि माधव}


\twolineshloka
{त्वत्प्रसादाद्धृषीकेश शक्रः सुरगणेश्वरः}
{त्रैलोक्यविजयं श्रीमान्प्राप्तवान्रणमूर्धनि}


\twolineshloka
{तव चैव प्रसादेन त्रिदशास्त्रिदशेश्वराः}
{अमरत्वं गताः कृष्ण लोकांश्चाश्नुवतेऽक्षयान्'}


\threelineshloka
{स्थितः सर्वात्मना नित्यं प्रियेषु च हितेषु च}
{त्वां चैवास्माभिराश्रित्य कृतः शस्त्रसमुद्यमः}
{सुरैरिवासुरवधे शक्रं शक्रानुजाहवे}


\twolineshloka
{असंभाव्यमिदं कर्म देवैरपि जनार्दन}
{त्वद्बुद्धिबलवीर्येण कृतवानेष फल्गुनः}


\threelineshloka
{बाल्यात्प्रभृति ते कृष्ण कर्माणि श्रुतवानहम्}
{अमानुषाणि दिव्यानि महान्ति च बहूनि च}
{तदैवाज्ञासिषं शत्रून्हतान्प्राप्तां च मेदिनीम्}


\twolineshloka
{त्वत्प्रसादसमुत्थेन विक्रमेणारिसूदन}
{सुरेशत्वं गतः शक्रो हत्वा दैत्यान्सहस्रशः}


\twolineshloka
{त्वत्प्रसादाद्वृषीकेश जगत्स्थावरजङ्गमम्}
{स्ववर्त्मनि स्थितं वीर जपरहोमेषु वर्तते}


\twolineshloka
{एकार्णवमिदं पूर्वं सर्वमासीत्तमोमयम्}
{त्वत्प्रसादान्माहाबाहो जगत्प्राप्तं नरोत्तम}


\twolineshloka
{स्रष्टारं सर्वलोकानां परमात्मानमव्ययम्}
{ये पश्यन्ति हृषीकेश न ते मुह्यन्ति कर्हिचित्}


\twolineshloka
{पुराणं परमं देवं देवदेवं सनातनम्}
{ये प्रपन्नाः सुरगुरुं न ते मुह्यन्ति कर्हिचित्}


\twolineshloka
{अनादिनिधनं देवं लोककर्तारमव्ययम्}
{ये भक्तास्त्वां हृषीकेश दुर्गाण्यतितरन्ति ते}


\twolineshloka
{परं पुराणं पुरुषं पराणां परमं च यत्}
{प्रपद्यतस्तत्परमं परा भूतिर्विधीयते}


\twolineshloka
{गायन्ति चतुरो वेदा यश्च देवेषु गीयते}
{तं प्रपद्य महात्मानं भूतिमश्नाम्यनुत्तमाम्}


\twolineshloka
{परमेश परेशेश तिर्यगीश नरेश्वर}
{सर्वेश्वरेश्वरेशेन नमस्ते पुरुषोत्तम}


\twolineshloka
{त्वमीशेशेश्वरेशान प्रभो वर्धस्व माधव}
{प्रभवाप्यय सर्वस्य सर्वात्मन्पृथुलोचन}


\twolineshloka
{धनञ्जयसखा यश्च धनञ्जयहितश्च यः}
{धनञ्जयस्य गोप्ता तं प्रपद्य सुखमेधते}


\twolineshloka
{[मार्कण्डेयः पुराणर्षिश्चरितज्ञस्तवानघ}
{माहात्म्यमनुभावं पुरा फीर्तितवान्मुनिः}


\twolineshloka
{असितो देवलश्चैव नारदश्च महातपाः}
{पितामहश्च मे व्यासस्त्वामाहुर्विधिमुत्तमम्}


\twolineshloka
{त्वं तेजस्त्वं परं ब्रह्म त्वं सत्यं त्वं महत्तपः}
{त्वं श्रेयस्त्वं यशश्चाग्र््यं कारणं जगतस्तथा}


\twolineshloka
{त्वया सृष्टमिदं सर्वं जगत्स्थावरजङ्गमम्}
{प्रलये समनुप्राप्ते त्वां वै निविशते पुनः}


\twolineshloka
{अनादिनिधनं देवं विश्वस्येशं जगत्पते}
{धातारमजमव्यक्तमाहुर्वेदविदो जनाः}


\threelineshloka
{भूतात्मानं महात्मानमनन्तं विश्वतोमुखम्}
{अपि देवा न जानन्ति गुह्यमाद्यं जगत्पतिम्}
{नारायणं परं देवं परमात्मानमीश्वरम्}


\twolineshloka
{ज्ञानयोनिं हरिं विष्णुं मुमुक्षूणां परायणम्}
{परं पुराणं पुरुषं पुराणानां परं च यत्}


\twolineshloka
{एवमादिगुणानां ते कर्मणां दिवि चेह च}
{अतीतभूतभव्यानां सङ्ख्याताऽत्र न विद्यते}


\twolineshloka
{सर्वतो रक्षणीयाः स्म शक्रेणेव दिवौकसः}
{यैस्त्वं सर्वगुणोपेतः सुहृन्न उपपादितः}


\twolineshloka
{इत्येवं धर्मराजेन हरिरुक्तो महायशाः}
{अनुरूपमिदं वाक्यं प्रत्युवाच जनार्दनः}


\twolineshloka
{भवता तपसोग्रेण धर्मेण परमेण च}
{साधुत्वादार्जवाच्चैव हतः पापो जयद्रथः}


\twolineshloka
{अयं च पुरुषव्याघ्र त्वदनुध्यानसंवृतः}
{हत्वा योधसहस्राणि न्यहञ्जिष्णुर्जयद्रथम्}


\twolineshloka
{कृतित्वे बाहुवीर्ये च तथैवासम्भ्रमेऽपि च}
{शीघ्रतामोधबुद्धित्वे नास्ति पार्थसमः क्वचित्}


\twolineshloka
{तदयं भरतश्रेष्ठ भ्राता तेऽद्य यदर्जुनः}
{सैन्यक्षयं रणे कृत्वा सिन्धुराजशिरोऽहरत्}


\twolineshloka
{ततो धर्मसुतो जिष्णुं परिष्वज्य विशाम्पते}
{प्रमृज्य वदनं तस्य पर्याश्वासयत प्रभुः}


\twolineshloka
{अतीव सुमहत्कर्म कृतवानसि फल्गुन}
{असह्यं चाविषह्यं च देवैरपि सवासवैः}


\twolineshloka
{दिष्टा निस्तीर्णभारोऽसि हतारिश्चासि शत्रुहन्}
{दिष्ट्या सत्या प्रतिज्ञेयं कृता हत्वा जयद्रथम्}


\threelineshloka
{एवमुक्त्वा गुडाकेशं धर्मराजो महायशाः}
{पस्पर्श पुण्यगन्धेन पृष्ठे हस्तेन पार्थिवाः ॥]सञ्जय उवाच}
{}


\twolineshloka
{एवमुक्तौ महात्मानावुभौ केशवपाण्डवौ}
{तावब्रूतां तदा कृष्णौ राजानं पृथिवीपतिम्}


\threelineshloka
{तव कोपाग्निना दग्धः पापो राजञ्जयद्रथः}
{उत्तीर्णं चापि सुमहद्धार्तराष्ट्रबलं रणे ॥युधिष्ठिर उवाच}
{}


\twolineshloka
{हन्यन्ते निहताश्चैव विनङ्क्ष्यन्ति च केशव}
{तव क्रोधहता ह्येते कौरवाः शत्रुसूदन}


\twolineshloka
{त्वां हि चक्षुर्हणं वीर कोपयित्वा सुयोधनः}
{समित्रबन्धुः समरे प्राणांस्त्यक्ष्यति दुर्मतिः}


\twolineshloka
{तव क्रोधहतः पूर्वं देवैरपि सुदुर्जयः}
{शरतल्पगतः शेते भीष्मः कुरुपितामहः}


\twolineshloka
{दुर्लभो विजयस्तेषां सम्प्रामे रिपुघातिनाम्}
{याता मृत्युवशं ते वै येषां क्रुद्धोऽपि माधव}


\threelineshloka
{राज्यं प्राणाः श्रियः पुत्राः सौख्यानि विविधानि च}
{अचिरात्तस्य नश्यन्ति येषां क्रुद्धोऽसि मानद ॥भगवानुवाच}
{}


\threelineshloka
{विनष्टान्कौरवान्मन्ये सपुत्रपशुबान्धवान्}
{राजधर्मपरे नित्यं त्वयि क्रुद्धे परन्तप ॥सञ्जय उवाच}
{}


\threelineshloka
{ततो भीमो महाबाहुः सात्यकिश्च महारथः}
{अभिवाद्य गुरुं ज्येष्ठं मार्गणैः क्षतविक्षतौ}
{क्षितावास्तां महेष्वासौ पाञ्चाल्यपरिवारितौ}


\twolineshloka
{तौ दृष्ट्वा मुदितो वीरौ प्राञ्जली चाग्रतः स्थितौ}
{अभ्यनन्दत कौन्तेयस्तावुभौ भीमसात्यकी}


\twolineshloka
{दिष्ट्या पश्यामि वा शूरौ विमुक्तौ सैन्यसागरात्}
{द्रोणग्राहदुराधर्षाद्धार्दिक्यमकरालयात्}


\twolineshloka
{दिष्ट्या विनिर्जिताः सङ्ख्ये पृथिव्यां सर्वपार्थिवाः}
{युवां विजयिनौ चापि दिष्ट्या पश्यामि संयुगे}


\twolineshloka
{दिष्ट्या द्रोणो जितः सङ्ख्ये हार्दिक्यश्च महाबलः}
{दिष्ट्या विकर्णिभिः कर्णोरणे नीतः पराभवम्}


\threelineshloka
{विमुखश्च कृतः शल्यो युवाभ्यां पुरुषर्षभौ}
{दिष्ट्या युवां कुशलिनौ सङ्ग्रामात्पुनरागतौ}
{पश्यामि रथिनां श्रेष्ठावुभौ युद्धविशारदौ}


\twolineshloka
{मम वाक्यकरौ वीरौ मम गौरवयन्त्रितौ}
{सैन्यार्णवं समुत्तीर्णौ दिष्ट्या पश्यामि वामहम्}


\twolineshloka
{समरश्लाघिनौ वीरौ समरेष्वपराजितौ}
{मम वाक्यसमौ चैव दिष्ट्या पश्यामि वामहम्}


\twolineshloka
{इत्युक्त्वा पाण्डवो राजन्युयुधानवृकोदरौ}
{सस्वजे पुरुषव्याघ्रौ हर्षाद्बाष्पं मुपोच ह}


\twolineshloka
{ततः प्रमुदितं सर्वं बलमासीद्विशाम्पते}
{पाण्डवानां रणे हृष्टं युद्धाय तु मनो दधे}


\chapter{अध्यायः १५१}
\twolineshloka
{सञ्जय उवाच}
{}


\twolineshloka
{सैन्धवे निहते राजन्पुत्रस्तव सुयोधनः}
{[अश्रुपूर्णमुखो दीनो निरुत्साहो द्विषज्जये}


\twolineshloka
{दुर्मना निःश्वसन्दुष्टो भग्नदंष्ट्र इवोरगः}
{आगस्कृत्सर्वलोकस्य पुत्रस्तेऽर्तिं परामगात्}


\twolineshloka
{दृष्ट्वा तत्कदनं घोरं स्वबलस्य कृतं महत्}
{जिष्णुना भीमसेनेन सात्वतेन च संयुगे}


\twolineshloka
{स विवर्णः कृशो दीनो बाष्पविप्लुतलोचनः}
{]अमन्यतार्जुनसमो न योद्धा भुवि विद्यते}


\twolineshloka
{न द्रोणो न च राधेयो नाश्वत्थामा कृपो न च}
{पार्थस्य सम्मखे स्थातुं पर्याप्ता इति मारिष}


\twolineshloka
{निर्जित्य हि रणे पार्थः सर्वान्मम महारथान्}
{अवधीत्सैन्धवं सङ्ख्ये न च कश्चिदवारयत््}


\twolineshloka
{सर्वथा हतमेवेदं कौरवाणां महद्बलम्}
{न ह्यस्य विद्यते त्राता साक्षादपि पुरन्दरः}


\twolineshloka
{यमुपाश्रित्य सङ्ग्रामे कृतः शस्त्रसमुद्यमः}
{स कर्णो निर्जितः सङ्ख्ये हतश्चैव जयद्रथः}


\twolineshloka
{`परुषाणि सभामध्ये प्रोक्तवान्यो हि पाण्डवान्}
{स कर्णो निर्जितः सङ्ख्ये सैन्धवश्च निपातितः}


\twolineshloka
{यस्य वीर्यं समाश्रित्य शमं याचन्तमच्युतम्}
{तृणवत्तमहं मन्ये स कर्णो निर्जितो युधि}


\twolineshloka
{एवं क्लान्तमना राजन्नुपायाद्रोणमीक्षितुम्}
{आगस्कृत्सर्वलोकस्य पुत्रस्ते भरतर्षभ}


\threelineshloka
{ततस्तत्सर्वमाचख्यौ कुरूणां वैशसं महत्}
{परान्विजयतश्चापि धार्तराष्ट्रान्निमज्जतः ॥दुर्योधन उवाच}
{}


\twolineshloka
{पश्य मूर्धाभिषिक्तानामाचार्य कदनं महत्}
{कृत्वा प्रमुखतः शूरं भीष्मं मम पितामहम्}


\twolineshloka
{तं निहत्य प्रलुब्धोऽयं शिखण्डी पूर्णमानसः}
{पाञ्चाल्यैः सहितः सर्वैः सेनाग्रमभिवर्तते}


\twolineshloka
{अपरश्चापि दुर्धर्षः शिष्यस्ते सव्यसाचिना}
{अक्षौहिणीः सप्त हत्वा हतो राजा जयद्रथः}


\twolineshloka
{अस्मद्विजयकामानां सुहृदामुपकारिणाम्}
{गन्ताऽस्मि कथमानृण्यं गतानां यमसादनम्}


\twolineshloka
{ये मदर्थं परीप्सन्ते वसुधां वसुधाधिपाः}
{ते हित्वा वसुधैश्वर्यं वसुधामधिशेरते}


\twolineshloka
{सोऽहं कापुरुषः क-त्वा मित्राणां क्षयमीदृशम्}
{अश्वमेधसहस्रेण पावितुं न समुत्सहे}


\twolineshloka
{मम लुब्धस्य पापस्य तथा धर्मापचायिनः}
{व्यायामेन जिगीषन्तः प्राप्ता वैवस्वतक्षयम्}


\twolineshloka
{कथं पतितवृत्तस्य पृथिवी सुहृदां द्रुहः}
{विवरं नाशकद्दातुं मम पार्थिवसंसदि}


\twolineshloka
{योऽहं रुधिरसिक्ताङ्गं राज्ञां मध्ये पितामहम्}
{शयानं नाशकं त्रातुं भीष्ममायोधने हतम्}


\twolineshloka
{तं मामनार्यपुरुषं मित्रद्रुहमधार्मिकम्}
{किं वक्ष्यति हि दुर्धर्षः समेत्य परलोकजित्}


\twolineshloka
{जलसन्धं महेष्वासं पश्य सात्यकिना हतम्}
{मदर्थमुद्यतं शूरं प्राणांस्त्यक्त्वा महारथम्}


\twolineshloka
{काम्भोजं निहतं दृष्ट्वा तथालम्बुसमेव च}
{अन्यान्बहूंश्च सुहृदो जीवितार्थोऽद्य को मम}


\twolineshloka
{व्यायच्छन्तो हताः शूरा मदर्थे ये पराङ्मुखाः}
{यतमानाः परं शक्त्या विजेतुमहितान्मम}


\twolineshloka
{तेषां गत्वाऽहमानृण्यमद्य शक्त्या परन्तप}
{तर्पयिष्यामि तानेव जलेन यमुनामनु}


\twolineshloka
{सत्यं ते प्रतिजानामि सर्वशस्त्रभृतां वर}
{इष्टापूर्तेन च शपे वीर्येण च सुतैरपि}


\twolineshloka
{निहत्य तान्रमे सर्वान्पाञ्चालान्पाण्डवैः सह}
{सान्तिं लब्धाऽस्मि तेषां वा रणे गन्ता सलोकताम्}


\threelineshloka
{सोऽहं तत्र गमिष्यामि यत्र ते पुरुषर्षभाः}
{हता मदर्थे सङ्ग्रामे युध्यमानाः किरीटिना ॥ 5-151-30 न हीदानींसहाया मे परीप्सन्त्यनुपस्कृताः}
{श्रेयो हि पाण्डून्मन्यन्ते न तथास्मान्महाभुज}


\twolineshloka
{स्वयं हि मृत्युर्विहितः सत्यसन्धेन संयुगे}
{भवानुपेक्षां कुरुते शिष्यत्वादर्जुनस्य हि}


\twolineshloka
{अतो विनिहताः सर्वे येऽस्मज्जयचिकीर्षवः}
{कर्णमेव तु पश्यामि सम्प्रत्यस्मज्जयैषिणम्}


\twolineshloka
{यो हि मित्रमविज्ञाय याथातथ्येन मन्दधीः}
{मित्रार्थे योजयत्येनं तस्य सोऽर्थोऽवसीदति}


\twolineshloka
{तादृग्रूपं कृतमिदं मम कार्यं सुहृत्तमैः}
{मोहाल्लुब्धस्य पापस्य जिह्मस्य धनमीहतः}


\threelineshloka
{यौ मे युद्धस्य कृत्यस्य जिह्माचारैः समीयतुः}
{हतो जयद्रथश्चैव सौमदत्तिश्च विर्यवान्}
{अभीषाहाः शूरसेनाः शिबयोऽथ वसातयः}


\twolineshloka
{सोऽहमद्य गमिष्यामि यत्र ते पुरुषर्षभाः}
{हता मदर्थे सङ्ग्रामे युध्यमानाः किरीटिना}


\twolineshloka
{न हि मे जीवितेनार्थस्तानृते पुरुषर्षभान्}
{आचार्यः पाण्डुपुत्राणामनुजानातु नो भवान्}


\chapter{अध्यायः १५२}
\twolineshloka
{धृतराष्ट्र उवाच}
{}


\twolineshloka
{सिन्धुराजे हते तात समरे सव्यसाचिना}
{तथैव भूरिश्रवसि किमासीद्वो मनस्तदा}


\threelineshloka
{दुर्योधनेन च द्रोणस्तथोक्तः कुरसंसदि}
{किमुक्तवान्परं तस्मात्तन्ममाचक्ष्व सञ्जय ॥सञ्जय उवाच}
{}


\twolineshloka
{निष्टानको महानासीत्सैन्यानां तव भारत}
{सैन्धवं निहतं दृष्ट्वा भूरिश्रवसमेव च}


\twolineshloka
{मन्त्रितं तव पुत्रस्य ते सर्वमवमेनिरे}
{येन मन्त्रैण निहताः शतशः क्षत्रियर्षभाः}


\threelineshloka
{द्रोणस्तु तद्वचः श्रुत्वा पुत्रस्य तव दुर्मनाः}
{मुहूर्तमिव च ध्यात्वा भृशमार्तोऽभ्यभाषत ॥द्रोण उवाच}
{}


\twolineshloka
{दुर्योधन किमेवं मां वाक्शरैरुपकृन्तसि}
{अजय्यं सततं सङ्ख्ये ब्रुवाणं सव्यसाचिनम्}


\twolineshloka
{एतेनवार्जुनं ज्ञातुमलं कौरव संयुगे}
{यच्छिखण्ड्यवधीद्भीष्मं पाल्यमानः किरीटिना}


\twolineshloka
{अवध्यं निहतं दृष्ट्वा संयुगे देवदानवैः}
{तदैवाज्ञासिषमहं नेयमस्तीति भारती}


\twolineshloka
{यं पुंसां त्रिषु लोकेषु सर्वशूरममंस्म हि}
{तस्मिन्निपतिते भीष्मे कं शेषं पर्युपास्महे}


\threelineshloka
{यान्स्म तान्ग्लहते तात शकुनिः कुरुसंसदि}
{अक्षान्न तेऽक्षा निशिता बाणास्ते शत्रुतापनाः}
{}


\twolineshloka
{`अक्षांस्तु मन्यसे बाणाञ्छोभमानाञ्छिताञ्शरान्'त एते घ्नन्ति नस्तात विशिखाः पार्थचोदिताः}
{तांस्तदाऽऽख्यायमानस्त्वं विदुरेण न बुद्धवान्}


\twolineshloka
{यास्ता विलपतश्चापि विदुरस्य महात्मनः}
{धीरस्य वाचो नाश्रौषीः क्षेमाय वदतः शिवाः}


\twolineshloka
{तदिदं वर्तते घोरमागतं वैशसं महत्}
{तस्यावमानाद्वाक्यस्य दुःशासनकृतेन च}


\twolineshloka
{योऽवमन्य वचः पथ्यं सुहृदामाप्तकारिणाम्}
{स्वमतं कुरुते मूढः स शोच्यो नचिरादिव}


\twolineshloka
{यच्च नः प्रेक्षमाणानां कृष्णामानाय्य तत्सभाम्}
{अनर्हन्तीं कुले जातां सर्वधर्मानुचारिणीम्}


\twolineshloka
{तस्याधर्मस्य गान्धारे फलं प्राप्तमिदं महत्}
{नोचेत्पापं परे लोके त्वमृच्छेथास्ततोऽधिकम्}


\twolineshloka
{यच्च तान्पाण्डवान्द्यूते विषमेण विजित्य ह}
{प्राव्राजयस्तदाऽरण्ये रौरवाजिनवाससः}


\twolineshloka
{पुत्राणामिव चैतेषां धर्ममाचरतां सदा}
{द्रुह्येत्को नु नरो लोके मदन्यो ब्राह्मणब्रुवः}


\twolineshloka
{पाण्डवानामयं कोपस्त्वया शकुनिना सह}
{आहृतो धृतराष्ट्रस्य सम्मते कुरु संसदि}


\twolineshloka
{दुःशासनेन संयुक्तः कर्णेन परिवर्धितः}
{क्षत्तुर्वाक्यमनादृत्य त्वयाऽब्यस्तः पुनः पुनः}


\twolineshloka
{`स हि नः क्रोधवृक्षश्च कृतमूलो महात्मनाम्}
{तस्य पुष्पफले राजन्नुपभुङ्क्ष्व महाबल'}


\twolineshloka
{यत्ताः सर्वे पराभूताः पर्यवारयतार्जुनम्}
{सिन्धुराजानमाश्रित्य स वो मध्ये कथं हतः}


\twolineshloka
{कथं त्वयि च कर्णे च कृपे शल्ये च जीवति}
{अश्वत्थाम्नि च कौरव्य निधनं सैन्धवोऽगमत्}


\twolineshloka
{युध्यन्तः सर्वराजानस्तेजस्विन उपासते}
{सिन्धुराजं परित्रातुं वो मध्ये कथं हतः}


\twolineshloka
{मय्येव हि विशेषेण तथा दुर्योधन त्वयि}
{आशंसत परित्राणमर्जुनात्स महीपतिः}


\twolineshloka
{ततस्तस्मिन्परित्राणमलब्धवति फल्गुनात्}
{न किञ्चिदनुपश्यामि जीवितस्थानमात्मनः}


\twolineshloka
{मज्जन्तमिव चात्मानं धृष्टद्युम्नस्य संयुगे}
{पश्याम्यहत्वा पाञ्चालान्सह तेन शिखण्डिना}


\twolineshloka
{तन्मां किमभितप्यन्तं वाक्शरैरेव कृन्तसि}
{अशक्तः सिन्धुराजस्य भूत्वा त्राणाय भारत}


\twolineshloka
{सौवर्णं सत्यसन्धस्य ध्वजमक्लिष्टकर्मणः}
{अपश्यन्युधि भीष्मस्य कथमाशंससे जयम्}


\twolineshloka
{मध्ये महारथानां किं शेषं तत्र मन्यसे}
{हतो भूरिश्रवाश्चैव किं शेषं तत्र मन्यसे}


\twolineshloka
{कृप एव च दुर्धर्षो यदि जीवति पार्थिव}
{यो नागात्सिन्धुराजस्य वर्त्म तं पूजयाम्यहम्}


\twolineshloka
{यत्रापश्यं हतं भीष्मं पश्यतस्तेऽनुजस्य वै}
{दुःशासनस्य कौरव्य कुर्वाणं कर्म दुष्करम्}


\twolineshloka
{अवध्यकल्पं सङ्ग्रामे देवैरपि सवासवैः}
{न ते वसुन्धराऽस्तीति तदाहं चिन्तये नृप}


\twolineshloka
{इमानि पाण्डवानां च सृज्जयानां च भारत}
{अनीकान्याद्रवन्ते मां सहितान्यद्य भारत}


\twolineshloka
{नाहत्वा सर्वपाञ्चलान्कवचस्य विमोक्षणम्}
{कर्ताऽस्मि समरे कर्म धार्तराष्ट्र हितं तव}


\twolineshloka
{राजन्ब्रूयाः सुतं मे त्वमश्वत्थामानमाहवे}
{न सोमकाः प्रमोक्तव्या जीवितं परिरक्षता}


\twolineshloka
{यच्च पित्राऽनुशिष्टोऽसि तद्वचः परिपालय}
{आनृशंस्ये दमे सत्ये चार्जवे च स्थिरो भव}


\twolineshloka
{धर्मार्थकामकुशलो धर्मार्थावप्यपीडयन्}
{धर्मप्रधानकार्याणि कुर्याश्चेति पुनः पुनः}


\twolineshloka
{चक्षुर्मनोभ्यां सन्तोष्या विप्राः पूज्याश्च शक्तितः}
{न चैषां विप्रियं कार्यं ते हि वह्निशिखोपमाः}


\twolineshloka
{एष त्वहमनीकानि प्रविशाम्यरिसूदन}
{रणाय महते राजंस्त्वया वाक्शरपीडितः}


\twolineshloka
{त्वं च दुर्योधन बलं यदि शक्तोऽसि पालय}
{रात्रावपि च योद्धव्याः संरब्धाः कुरुसृञ्जयाः}


\twolineshloka
{एवमुक्त्वा ततः प्रायाद्द्रोणः पाण्वसृञ्जयान्}
{मुष्णन्क्षत्रियतेजांसि नक्षत्राणामिवांशुमान्}


\chapter{अध्यायः १५३}
\twolineshloka
{सञ्जय उवाच}
{}


\threelineshloka
{ततो दुर्योधनो राजा द्रोणेनैवं प्रचोदितः}
{अमर्षवशमापन्नो युद्धायैव मनो दधे}
{अब्रवीच्च तदा कर्णं पुत्रो दुर्योधनस्तव}


\twolineshloka
{पश्य कृष्णसहायेन पाण्डवेन किरीटिना}
{आचार्यविहितं व्यूहं भित्त्वा देवैः सुदुर्भिदम्}


\twolineshloka
{तव व्यायच्छमानस्य द्रोणस्य च महात्मनः}
{मिषतां योधमुख्यानां सैन्धवो विनिपातितः}


\twolineshloka
{पश्य राधेय पृथ्वीशाः पृथिव्यां प्रवरा युधि}
{पार्थेनैकेन निहताः सिंहेनेवेतरे मृगाः}


\twolineshloka
{मम व्यायच्छमानस्य समरे शत्रुसूदन}
{अल्पावशेषं सैन्यं मे कृतं शक्रात्मजेन ह}


\twolineshloka
{कथं वै युध्यमानस्य द्रोणस्य युधि फल्गुनः}
{भिन्द्यात्सुदुर्भिदं व्यूहं यतमानोऽपि संयुगे}


% Check verse!
प्रतिज्ञाया गतः पारं हत्वा सैन्धवमर्जुनः
\twolineshloka
{पश्य राधेय पृथ्वीशान्पृथिव्यां पातितान्बहून्}
{पार्थेन निहतान्सङ्ख्ये महेन्द्रोपमविक्रमान्}


\twolineshloka
{अनिच्छतः कथं वीर द्रोणस्य युधि पाण्डवः}
{भिन्द्यात्सुदुर्भिदं व्यृहं यतमानस्य शुप्मिणः}


\twolineshloka
{दयितः फल्गुनो नित्यमाचार्यस्य महात्मनः}
{ततोऽस्य दत्तवान्द्वारमयुद्धेनैव शत्रुहन्}


\twolineshloka
{अभयं सिन्धुराजाय दत्त्वा द्रोणः परन्तपः}
{प्रादात्किरीटिने द्वारं पश्यन्निर्गुणतां मयि}


\twolineshloka
{यद्यदास्यदनुज्ञां वै पूर्वमेव गृहान्प्रति}
{प्रम्थातुं सिन्धुराजस्य न भवेज्जीवितक्षयः}


\twolineshloka
{जयद्रथो जीवितार्थी गच्छमानो गृहान्प्रति}
{मयाऽनार्येण संरुद्धो द्रोणात्प्राप्याभयं रणे}


\twolineshloka
{`रक्षामि सैन्धवं युद्धे नैनं प्राप्स्यति फल्गुनः}
{मम सैन्यविनाशाय रुद्धो विप्रेण सैन्धवः}


\twolineshloka
{तस्य मे मन्दभाग्यस्य यतमानस्य संयुगे}
{हतानि सर्वसैन्यानि हतो राजा जयद्रथः}


\twolineshloka
{पश्य योधवरान्कर्ण शतशोऽथ सहस्रशः}
{पार्थनामाङ्कितैर्वाणैः सर्वे नीता यमक्षयम्}


\twolineshloka
{कथमेकरथेनाजौ बहूनां नः प्रपश्यताम्}
{विपन्नः सैन्धवो राजा योधाश्चैव सहस्रशः'}


\threelineshloka
{अद्य मे भ्रातरः क्षीणाश्चित्रसेनादयो रणे}
{भामसनं समासाद्य पश्यतां नोदुरात्मनाम् ॥कर्ण उवाच}
{}


\twolineshloka
{आचार्यं मा विगर्हस्व शक्त्याऽसौ युध्यते द्विजः}
{यथाबलं यथोत्साहं त्यक्त्वा जीवितमात्मनः}


\twolineshloka
{यद्येनं समतिक्रम्य प्रविष्टः श्वेतवाहनः}
{नात्र सूक्ष्मोऽपि दोषःस्यादाचार्यस्य कथञ्चन}


\twolineshloka
{कृती दक्षो युवा शूरः कृतास्त्रो लघुविक्रमः}
{दिव्यास्त्रयुक्तमास्थाय रथं वानरलक्षणम्}


\twolineshloka
{कृष्णेन च गृहीताश्वमभेद्यकचावृतः}
{गाण्डीवमजरं दिव्यं धनुरादाय वीर्यवान्}


\twolineshloka
{प्रवर्षन्निशितान्बाणान्बाहुद्रविणदर्पितः}
{यदर्जुनोऽभ्ययाद्द्रोणमुपपन्नं हि तस्य तत्}


\twolineshloka
{आचार्यः स्थविरो राजञ्शीघ्रयाने तथाऽक्षमः}
{बाहुव्यायामचेष्टायामशक्तस्तु नराधिप}


\twolineshloka
{तेनैवमभ्यतिक्रान्तः श्वेताश्वः कृष्णसारथिः}
{तस्य दोषं न पश्यामि द्रोणस्यानेन हेतुना}


\twolineshloka
{अजय्यान्पाण्डवान्मन्ये द्रोणोनास्त्रविदा मृधे}
{तथा ह्येनमतिक्रम्य प्रविष्टः श्वेतवाहनः}


\twolineshloka
{सततं चेष्टमानानां निकृत्या विक्रमेण च}
{दैवादिष्टोऽन्यथाभावो न मन्ये विद्यते क्वचित्}


\twolineshloka
{यतो नो युध्यमानानां परं शक्त्या सुयोधन}
{सैन्धवो निहतो युद्धे दैवमत्र परं स्मृतम्}


\twolineshloka
{परं यत्नं कुर्वतां च त्वया सार्धं रणाजिरे}
{हत्वाऽस्माकं पौरुषं वै दैवं पश्चात्करोति नः}


\twolineshloka
{दैवोपसृष्टपुरुषो यत्कर्म कुरुते क्वचित्}
{कृतं कृतं हि तत्तस्यट दैवेन विनिपात्यते}


\twolineshloka
{यत्कर्तव्यं मनुष्येण व्यवसायवता सदा}
{तत्कार्यमविशङ्केन सिद्धिर्दैवे प्रतिष्ठिता}


\twolineshloka
{निकृत्या वञ्चिताः पार्था विषयोगैश्च भारत}
{दग्धा जतुगृहे चापि द्यूतेन च पराजिताः}


\twolineshloka
{राजनीतिं समाश्रित्य प्रहिताश्चैव काननम्}
{यत्नेन च कृतं तत्तद्दैवेन विनिपातितम्}


\twolineshloka
{युध्यस्व यत्नमास्थय दैवं कृत्वा निरर्थकम्}
{यततस्तव तेषां च दैवं मार्गेण यास्यति}


\twolineshloka
{न तेषां मतिपूर्वं हि सुकृतं दृश्यते क्वचित्}
{दुष्कृतं तव वा वीर बुद्ध्या हीनं कुरूद्वह}


\threelineshloka
{दैवं प्रमाणं सर्वस्य सुकृतस्येतरस्य वा}
{अनन्यकर्म दैवं हि जागर्ति स्वपतामपि}
{`तेन युक्तो हि पुरुषः कार्याकार्ये नियुज्यते'}


\twolineshloka
{बहूनि तव सैन्यानि योधाश्च बहवस्तव}
{न तथा पाण्डुपुत्राणामेवं युद्धमवर्तत}


\threelineshloka
{तैरल्पैर्बहवो यूयं क्षयं नीताः प्रहारिणः}
{शङ्के दैवस्य तत्कर्म पौरुषं येन नाशितम् ॥सञ्जय उवाच}
{}


\twolineshloka
{एवं संभाषमाणानां बहु तत्तज्जनाधिप}
{पाण्डवानामनीकानि समदृश्यन्त संयुगे}


\twolineshloka
{ततः प्रववृते युद्धं व्यतिषक्तरथद्विपम्}
{तावकानां परैः सार्धं राजन्दुर्मन्त्रिते तव}


\chapter{अध्यायः १५४}
\twolineshloka
{सञ्जय उवाच}
{}


\twolineshloka
{तदुदीर्णं रथाश्वौघं बलं तव जनाधिप}
{पाण्डुसेनामतिक्रम्य योधयामास सर्वतः}


\twolineshloka
{पाञ्चालाः कुरवश्चैव योधयन्तः परस्परम्}
{यमराष्ट्राय महते परलोकाय दीक्षिताः}


\twolineshloka
{शूराः शूरैः समागम्य शरतोमरशक्तिभिः}
{विव्यधुः समरेऽन्योन्यं निन्युश्चैव यमक्षयम्}


\twolineshloka
{रथिनां रथिभिः सार्धं रुधिरस्रावदारुणम्}
{प्रावर्तत महद्युद्धं निघ्नतामितरेतरम्}


\twolineshloka
{वारणाश्च महाराज समासाद्य परस्परम्}
{विषाणैर्दारयामासुः सुसङ्क्रुद्धा मदोत्कटाः}


\twolineshloka
{हयारोहान्हयारोहाः प्रासशक्तिपरश्वथैः}
{बिभिदुस्तुमुले युद्धे प्रार्थयन्तो महद्यशः}


\twolineshloka
{पत्तयश्च महाबाहो शतशः शस्त्रपाणयः}
{अन्योन्यमार्दयन्राजन्नित्यं यत्ताः पराक्रमे}


\twolineshloka
{गोत्राणां नामधेयानां कुलानां चैव मारिष}
{श्रवणाद्धि विजानीमः पाञ्चालान्कुरुभिः सह}


\twolineshloka
{तेऽन्योन्यं समरे योधाः शरशक्तिपरश्वथैः}
{प्रैषयन्परलोकाय विचरन्तो ह्यभीतवत्}


\twolineshloka
{शरैर्दश दिशो राजंस्तेषां मुक्तैः सहस्रशः}
{न भ्राजन्ते यथा पूर्वं भास्करस्य च नाशनात्}


\twolineshloka
{तथा प्रयुध्यमानेषु पाण्डवेयेषु भारत}
{दुर्योधनो महाराज व्यवागाहत तद्बलम्}


\twolineshloka
{सैन्धवस्य वधेनैव भृशं दुःखसमन्वितः}
{मर्तव्यमिति संचिन्त्य प्राविशच्च द्विषद्बलम्}


\twolineshloka
{नादयन्रथघोषेण कम्पयन्निव मेदिनीम्}
{अभ्यवर्तत पुत्रस्ते पाण्डवानामनीकिनीम्}


\threelineshloka
{स सन्निपातस्तुमुलस्तस्य तेषां च भारत}
{अभवत्सर्वसैन्यानामभावकरमो महान् ॥`धृतराष्ट्र उवाच}
{}


\twolineshloka
{तथा हतेषु सैन्येषु तथा कृच्छ्रगतः स्वयम्}
{कच्चिद्दुर्योधनः सूत नाकार्षीत्पृष्ठतो रणम्}


\twolineshloka
{एकस्य च बहुनां च सन्निपातो महानभूत्}
{विशेषतो हि नृपतेर्विषमं प्रतिभाति मे}


\twolineshloka
{सोऽत्यन्तसुखसंवृद्धो रक्ष्यो लोकस्य चेश्वरः}
{एको बहून्समासाद्य कच्चिन्नासीत्पराङ्मुखः}


\twolineshloka
{द्रोणः कर्णः कृपश्चैव कृतवर्मा च सात्वतः}
{नावारयन्कथं युद्धे राजानं राज्यकाङ्क्षिणः}


\twolineshloka
{सर्वोपायैर्हि युद्धेषु रक्षितव्यो महीपतिः}
{एषा नीतिः परा युद्धे दृष्टा तत्र मनीषिभिः}


\threelineshloka
{प्रविष्टे वै मम सुते परेषां वै महद्बलम्}
{मामका रथिनां श्रेष्ठाः किमकुर्वत स़ञ्जय ॥सञ्जय उवाच}
{}


\twolineshloka
{राजन्सङ्ग्राममाश्चर्यं पुत्रस्य तव भारत}
{एकस्य च बहूनां च शृणु मे ब्रुवतो ध्रुवम्}


\twolineshloka
{द्रोणेन वार्यमाणोऽसौ कर्णेन च कृपेण च}
{प्राविशत्पाण्डवीं सेनां मकरः सागरं यथा}


\twolineshloka
{किरन्निषुसहस्राणि तत्रतत्र तथातथा}
{पाञ्चालान्पाण्डवांश्चैव विव्याध निशितैः शरैः}


\twolineshloka
{यथोदयगतः सूर्यो रश्मिभिर्नाशयेत्तमः}
{तथा पुत्रस्तव बलं नाशयत्तन्महाबलः'}


\threelineshloka
{यथा मध्यन्दिने सूर्यं प्रतपन्तं गभस्तिभिः}
{तथा तव सुतं मध्ये प्रतपन्तं शरार्चिभिः}
{न शेकुर्भारतं युद्धे पाण्डवाः समुदीक्षितुम्}


\twolineshloka
{पलायनकृतोत्साहा निरुत्साहा द्विषज्जये}
{पर्यधावन्त पाञ्चाला वध्यमाना महात्मना}


\twolineshloka
{रुक्मपुङ्खैः प्रसन्नाग्रैस्तव पुत्रेण धन्विना}
{अर्द्यमानाः शरैस्तूर्णं न्यपतन्पाण्डुसैनिकाः}


\twolineshloka
{`व्यद्रवंश्च भयाद्योधा दृष्ट्वा तं परमाहवे}
{व्यात्ताननमिव प्राप्तमन्तकं प्राणहारिणम्'}


\twolineshloka
{न तादृशं रमे कर्म कृतवन्तस्तु तावकाः}
{यादृशं कृतवान्राजा पुत्रस्तव विशाम्पते}


\twolineshloka
{पुत्रेण तव सा सेना पाण्डवी मथिता रणे}
{नलिनी द्विरदेनेव समन्तात्फुल्लपङ्कजा}


\twolineshloka
{क्षीणतोयानिलार्काभ्यां हतत्विडिव पद्मिनी}
{बभूव पाण्डवी सेना तव पुत्रस्य तेजसा}


\twolineshloka
{पाण्डुसेनां हतां दृष्ट्वा तव पुत्रेण भारत}
{भीमसेनपुरोगास्तु पाञ्चालाः समुपाद्रवन्}


\twolineshloka
{स भीमसेनं दशभिर्माद्रीपुत्रौ त्रिभिस्त्रिभिः}
{विराटद्रुपदौ षड्भिः शतेन च शिखण्डिनम्}


\twolineshloka
{धृष्टद्युम्नं च सप्तत्या धर्मपुत्रं च सप्तभिः}
{केकयांश्चैव चेदींश्च बहुभिर्निशितैः शरैः}


\twolineshloka
{सात्वतं पञ्चभिर्विद्धा द्रौपदेयांस्त्रिभिस्त्रिभिः}
{घटोत्कचं च समरे विद्ध्वा सिंह इवानदत्}


\twolineshloka
{शतशश्चापरान्योधान्सद्विपांश्च महारणे}
{शरैरवचकर्तोग्रैः क्रुद्धोऽन्तक इव प्रजाः}


\twolineshloka
{सा तेन पाण्डवी सेना वध्यमाना शिलीमुखैः}
{तव पुत्रेण सङ्ग्रामे विदुद्राव नराधिप}


\twolineshloka
{तं तपन्तमिवादित्यं कुरुराजं महाहवे}
{नाशकन्वीक्षितुं राजन्पाण्डुपुत्रस्य सैनिकाः}


\twolineshloka
{ततो युधिष्ठिरो राजा कुपितो राजसत्तम}
{अभ्यधावत्कुरुपतिं तव पुत्रं जिघांसया}


\twolineshloka
{तावुभौ युधि कौरव्यौ समीयतुररिन्दमौ}
{स्वार्थहेतोः पराक्रान्तौ दुर्योधनयुधिष्ठिरौ}


\twolineshloka
{ततो दुर्योधनः क्रुद्धः शरैः सन्नतपर्वभिः}
{विव्याध दशभिस्तूर्णं ध्वजं चिच्छेद चेषुणा}


\twolineshloka
{इन्द्रसेनं त्रिभिश्चैव ललाटे जघ्निवान्नृप}
{सारथिं दयितं राज्ञः पाण्डवस्य महात्मनः}


\twolineshloka
{धनुश्च पुनरन्येन चकर्तास्य महारथः}
{चतुर्भिश्चतुरश्चैव बाणैर्विव्याध वाजिनाः}


\twolineshloka
{ततो युधिष्ठिरः क्रुद्धो निमेषादिव कार्मुकम्}
{अन्यदादाय वेगेन कौरवं प्रत्यवारयत्}


\twolineshloka
{तस्य तान्निघ्नतः शत्रून्रुक्मपृष्ठं महद्धनुः}
{भल्लाभ्यां पण्डवो ज्येष्ठस्त्रिधा चिच्छेद मारिष}


\twolineshloka
{विव्याध चैनं दशभिः सम्यगस्तैः शितैः शरैः}
{मर्म भित्त्वा तु ते सर्वे संलग्नाः क्षितिमाविशन्}


\twolineshloka
{ततः परिवृता योधाः परिवव्रुर्युधिष्ठिरम्}
{वृत्रेण सह युध्यन्तं यद्वद्देवाः शतक्रतुम्}


\threelineshloka
{ततो युधिष्ठिरो राजा तव पुत्रस्य मारिष}
{शरं च सूर्यरश्म्याभमत्युग्रमनिवारणम्}
{हा हतोऽसीति राजानमुक्त्वाऽमुञ्चद्युधिष्ठिरः}


\twolineshloka
{स तेनाकर्णमुक्तेन विद्धो बाणेन कौरवः}
{निषसाद रथोपस्थे भृशं सम्मूढचेतनः}


\twolineshloka
{ततः पाञ्चाल्यसेनानां भृशमासीद्रवो महान्}
{हतो राजेति राजेन्द्र मुदितानां समन्ततः}


\twolineshloka
{बाणशब्दरवश्चोग्रः शुश्रुवे तत्र मारिष}
{अथ द्रोणो द्रुतं तत्र प्रत्यदृश्यत संयुगे}


\twolineshloka
{हृष्टो दुर्योधनश्चापि दृढमादाय कार्मुकम्}
{तिष्ठतिष्ठेति राजानं ब्रुवन्पाण्डवमभ्ययात्}


\threelineshloka
{प्रत्युद्ययुस्तं त्वरिताः पाञ्चाला जयगृद्धिनः}
{तान्द्रोणः प्रतिजग्राह परीप्सन्कुरुसत्तमम्}
{चण्डवातः समुद्वूतो मेघानम्बुमुचो यथा}


\twolineshloka
{ततो राजन्महानासीत्सङ्ग्रामो भूरिवर्धनः}
{तावकानां परेषां च समेतानां युयुत्सया}


\chapter{अध्यायः १५५}
\twolineshloka
{धृतराष्ट्र उवाच}
{}


\twolineshloka
{यत्तदा प्राविशत्पाण्डूनाचार्यः कुपितो बली}
{उक्ता दुर्योधनं मन्दं मम शास्त्रातिगं सुतम्}


\twolineshloka
{प्रविश्य विचरन्तं च रथे शूरमवस्थितम्}
{कथं द्रोणं महेष्वासं पाण्डवाः पर्यवारयन्}


\twolineshloka
{केऽरक्षन्दक्षिणं चक्रमाचार्यस्य महाहवे}
{के चोत्तरमरक्षन्त निघ्नतः शास्त्रवान्बहून्}


\twolineshloka
{के चास्य पृष्ठतोऽन्वासन्वीरा वीरस्य योधिनः}
{के पुरस्तादवर्तन्त रथिनस्तस्य शत्रवः}


\twolineshloka
{मन्ये तानस्पृशच्छीतमतिवेलमनार्तवम्}
{मन्ये ते समवेपन्त गावो वै शिशिरे यथा}


\twolineshloka
{यत्प्राविशन्महेष्वासः पाञ्चालानपराजितः}
{नृत्यन्स रथमार्गेषु सर्वशस्त्रभृतां वरः}


\threelineshloka
{निर्दहन्सर्वसैन्यानि पाञ्चालानां रथर्षभः}
{धूमकेतुरिव क्रुद्धः कथं मृत्युमुपेयिवान् ॥सञ्जय उवाच}
{}


\twolineshloka
{सायाह्ने सैन्धवं हत्वा राज्ञा पार्थः समेत्य च}
{सात्यकिश्च महेष्वासो द्रोणमेवाभ्यधावताम्}


\twolineshloka
{तथा युधिष्ठिरस्तूर्णं भीमसेनश्च पाण्डवः}
{पृथक्चमूभ्यां संयत्तौ द्रोणमेवाभ्यधावताम्}


\threelineshloka
{तथैव नकुलो धीमान्सहदेवश्च दुर्जयः}
{धृष्टद्युम्नः शतानीको विराटश्च सकेकयः}
{मात्स्याः साल्वाः ससेनाश्च द्रोणमेव ययुर्युधि}


\twolineshloka
{द्रुपदश्च तथा राजा पाञ्चालैरभिरक्षितः}
{धृष्टद्युम्नपिता राजन्द्रोणमेवाभ्यवर्तत}


\twolineshloka
{द्रौपदेया महेष्वासा राक्षसश्च घटोत्कचः}
{ससैन्यास्ते न्यवर्तन्त द्रोणमेव महाद्युतिम्}


\twolineshloka
{प्रभद्रकाश्च पाञ्चालाः षट््सहस्राः प्रहारिणः}
{द्रोणमेवाभ्यवर्तन्त पुरस्कृत्य शिखण्डिनम्}


\twolineshloka
{तथेतरे नरव्याघ्राः पाण्डवानां महारथाः}
{सहिताः सन्न्यवर्तन्त द्रोणमेव द्विजर्षभम्}


\twolineshloka
{तेषु शूरेषु युद्धाय गतेषु भरतर्षभ}
{बभूव रजनी घोरा भीरूणां भयवर्धिनी}


\twolineshloka
{योधानामशिवा रौद्रा राजन्नन्तकगामिनी}
{कुञ्जराश्वमनुष्याणां प्राणान्तकरणी तदा}


\twolineshloka
{तस्यां रजन्यां घोरायां नदन्त्यः सर्वतः शिवाः}
{न्यवेदयन्भयं घोरं सज्वालकबलैर्मुखैः}


\twolineshloka
{उलूकाश्चाप्यदृश्यन्त शंसन्तो विपुलं भयम्}
{विशेषतः कौरवाणां ध्वजिन्यामतिदारुणाः}


\twolineshloka
{ततः सैन्येषु राजेन्द्र शब्दः समभवन्महान्}
{भेरीशब्देन महता मृदङ्गानां स्वनेन च}


\twolineshloka
{गजानां बृंहितैश्चापि तुरङ्गाणां च हेषितैः}
{खुरशब्दनिपातैश्च तुमुलः सर्वतोऽभवत्}


\twolineshloka
{ततः समभवद्युद्धं सन्ध्यायामतिदारुणम्}
{द्रोणस्य च महाराज सृञ्जयानां च सर्वशः}


\twolineshloka
{तमसा चावृते लोके न प्राज्ञायत किञ्चन}
{सैन्येन रजसा चैव समन्तादुत्थितेन ह}


\twolineshloka
{नरस्याश्वस्य नागस्य समसज्जत शोणितम्}
{नापश्याम रजो भौमं कश्मलेनाभिसंवृताः}


\twolineshloka
{रात्रौ वंशवनस्येव दह्यमानस्य पर्वते}
{घोरश्चटचटाशब्दः शस्त्राणां पततामभूत्}


\twolineshloka
{मृदङ्गानकनिर्ह्रादैर्झर्झरैः पटहैस्तथा}
{फेत्कारैर्हेषितैः शब्दैः सर्वमैवाकुलं बभौ}


\twolineshloka
{नैव स्वे न परे राजन्प्राज्ञायन्त तमोवृते}
{उन्मत्तमिव तत्सर्वं बभूव रजनीमुखे}


\twolineshloka
{भौमं रजोऽथ राजेन्द्र शोणितेन प्रणाशितम्}
{शातकौम्भैश्च कवचैर्भूषणैश्च तमोऽनशत्}


\twolineshloka
{ततः सा भारती सेना मणिहेमविभूषिता}
{द्यौरिवासीत्सनक्षत्रा रजन्यां भरतर्षभ}


\twolineshloka
{गोमायुबलसङ्घुष्टा शक्तिध्वजसमाकुला}
{वारणाभिरुता घोरा क्ष्वेडितोत्क्रुष्टनादिता}


\twolineshloka
{तत्राभवन्महाशब्दस्तुमुलो रोमहर्षणः}
{समावृण्वन्दिशः सर्वा महेन्द्राशनिनिःस्वनः}


\twolineshloka
{सा निशीथे महाराज सेनाऽदृश्यत भारती}
{अङ्गदैः कुण्डलैर्निष्कैः शस्त्रैश्चैवावभासिता}


\twolineshloka
{तत्र नागा रथाश्चैव जाम्बूनदविभूषिताः}
{निशायां प्रत्यदृश्यन्त मेघा इव सविद्युतः}


\twolineshloka
{ऋष्टिशक्तिगदाबाणमुसलप्रासपट्टसाः}
{सम्पतन्तो व्यदृश्यन्त भ्राजमाना इवाग्नयः}


\twolineshloka
{दुर्योधनपुरोवातां रथनागवलाहकाम्}
{वादित्रघोषस्तनितां चापविद्युद्व्वजैर्वृताम्}


\twolineshloka
{द्रोणपाण्डवपर्जन्यां खङ्गशक्तिगदाशनिम्}
{शरधारास्त्रपवनां शस्त्रपातोष्मसङ्कुलाम्}


\twolineshloka
{घोरां विस्मापनीमुग्रां जीवितच्छिदमप्लुवाम्}
{तां प्राविशन्नतिभयां नदीं युद्धचिकीर्षवः}


\twolineshloka
{तस्मिन्रात्रिमुखे घोरे महाशब्दनिनादिते}
{भीरूणां त्रासजनने शूराणां हर्षवर्धने}


\twolineshloka
{रात्रियुद्धे महाघोरे वर्तमाने सुदारुणे}
{द्रोणमभ्यद्रुवन्क्रुद्धाः सहिताः पाण्डुसृञ्जयाः}


\twolineshloka
{ये ये प्रमुखतो राजन्नावर्तन्त महारथाः}
{तान्सर्वान्विमुखांश्चक्रे कांश्चिन्निन्ये यमक्षयम्}


\threelineshloka
{तानि नागसहस्राणि रथानामयुतानि च}
{पदातिहयसङ्घानां प्रयुतान्यर्बुदानि च}
{द्रोणेनैकेन नाराचैर्निर्भिन्नानि निशामुखे}


\chapter{अध्यायः १५६}
\twolineshloka
{धृतराष्ट्र उवाच}
{}


\twolineshloka
{तस्मिन्प्रविष्टे दुर्धर्षे सृञ्जयानमितौजसि}
{अमृष्यमाणे संरब्धे का वोऽभूद्वै मतिस्तदा}


\twolineshloka
{दुर्योधनं तथा पुत्रमुक्त्वा शास्त्रातिगं मम}
{यत्प्राविशदमेयात्मा किं पार्थः प्रत्यपद्यत}


\twolineshloka
{निहते सैन्धवे वीरे भूरिश्रवसि चैव ह}
{यदाऽभ्यगान्महातेजाः पाञ्चालानपराजितः}


\twolineshloka
{किममन्यत दुर्धर्षे प्रविष्टे शत्रुतापने}
{दुर्योधनस्तु किं कृत्यं प्राप्तकालममन्यत}


\threelineshloka
{के च तं वरदं वीरमन्वयुर्द्विजसत्तमम्}
{के चास्य पृष्ठतोऽगच्छन्वीराः शूरस्य युध्यतः}
{के पुरस्तादवर्तन्त निघ्नन्तः शास्त्रवान्रणे}


\twolineshloka
{मन्येऽहं पाण्डवान्सर्वान्भारद्वाजशरार्दितान्}
{शिशिरे कम्पमाना वै कृशा गाव इव प्रभो}


\twolineshloka
{प्रविश्य स महेष्वासः पाञ्चालानरिमर्दनः}
{कथं नु पुरुषव्याघ्रः पञ्चत्वमुपजग्मिवान्}


\twolineshloka
{सर्वेषु योधेषु च सङ्गतेषुरात्रौ समेतेषु महारथेषु}
{संलोड्यमानेषु पृथग्बलेषुके वस्तदानीं मतिमन्त आसन्}


\threelineshloka
{हतांश्चैव विषक्तांश्च पराभूतांश्च शंससि}
{रथिनो विरथांश्चैव कृतान्युद्धेषु मामकान् ॥ 5-156-10 तेषांसंलोड्यमानानां पाण्डवैर्हतचेतसाम्}
{अन्धे तमसि मग्नानामभवत्का मतिस्तदा}


\twolineshloka
{प्रहृष्टांश्चाप्युदग्रांश्च सन्तुष्टांश्चैव पाण्डवान्}
{शंससीहाप्रहृष्टांश्च विभ्रष्टांश्चैव मामकान्}


\threelineshloka
{कथमेषां तदा तत्र पार्थानामपलायिनाम्}
{प्रकाशमभवद्रात्रौ कथं कुरुषु सञ्जय ॥सञ्जय उवाच}
{}


\twolineshloka
{रात्रियुद्धे तदा राजन्वर्तमाने सुदारुणे}
{द्रोणमभ्यद्रवन्सर्वे पाण्डवाः सह सोमकैः}


\twolineshloka
{ततो द्रोणः केकयांश्च धृष्टद्युम्नस्य चात्मजान्}
{सम्प्रैषयत्प्रेतलोकं सर्वानिषुभिराशुगैः}


\twolineshloka
{तस्य प्रमुखतो राजन्येऽवर्तन्त महारथाः}
{तान्सर्वान्प्रेषयामास पितृलोकं स भारत}


\twolineshloka
{प्रमथ्नन्तं तदा वीरान्भारद्वाजं महारथम्}
{अभ्यवर्तत सङ्क्रुद्धः शिबी राजा प्रतापवान्}


\twolineshloka
{तमापतन्तं सम्प्रेक्ष्य पाण्डवानां महारथम्}
{विव्याध दशभिर्द्रोणः सर्वपारशवैः शितैः}


\twolineshloka
{तं शिबिः प्रतिविव्याध त्रिंशता निशितैः शरैः}
{सारथिं चास्य भुल्लेन स्मयमानो न्यपातयत्}


\twolineshloka
{तस्य द्रोणो हयान्हत्वा सारथिं च महात्मनः}
{अथाऽस्य सशिरस्त्राणं शिरः कायादपाहरत्}


\twolineshloka
{ततोऽस्य सारथिं क्षिप्रमन्यं दुर्योधनोऽदिशत्}
{स तेन सङ्गृहीताश्वः पुनरभ्यद्रवद्रिपून्}


\twolineshloka
{कलिङ्गानामनीकेन कालिङ्गस्य सुतो रणे}
{पूर्वं पितृवधात्क्रुद्धौ भीमसेनमुपाद्रवत्}


\twolineshloka
{स भीमं पञ्चभिर्विद्ध्वा पुनर्विव्याध सप्तभिः}
{विशोकं त्रिभिरानर्च्छद्वजमेकेन पत्रिणा}


\twolineshloka
{कलिङ्गानां तु तं शूरं क्रुद्धं क्रुद्धो वृकोदरः}
{रथाद्रथमभिद्रुत्य मुष्टिनाऽभिजघान ह}


\twolineshloka
{तस्य मुष्टिहतस्याजौ पाण्डवेन बलीयसा}
{सर्वाण्यस्थीनि सहसा प्रापतन्वै पृथक्पृथक्}


\twolineshloka
{`तस्मिन्हते तदा तेन सिंहनादो महानभूत्}
{पाञ्चालानां महाराज साधुसाध्विति पाण्डवम्'}


\twolineshloka
{तं कर्णो भ्रातरश्चास्य नामृष्यन्त परन्तप}
{ते भीमसेनं नाराचैर्जघ्नुराशीविषोपमैः}


\twolineshloka
{ततः शत्रुरथं त्यक्त्वा भीमो द्रुमरथं गतः}
{द्रुमं चास्यन्तमनिशं मुष्टिना समपोथयत्}


\twolineshloka
{`यथा काचमणिर्न्यस्तो मुष्टिनैकेन लीलया}
{स हतः सर्वथा चूर्णो रक्तमेवोदपद्यत}


\twolineshloka
{तथा चूर्णमभूत्तत्र कर्णभ्राता द्रुमस्तथा'}
{स तदा पाण़्डुपुत्रेण बलिनाऽभिहतोऽपतत्}


\twolineshloka
{तं निहत्य महाराज भीमसेनो महाबलः}
{जयरातरथं प्राप्य मुहुः सिंह इवानदत्}


\twolineshloka
{जयरातं तु चरणे गृह्य भीमो महाबलः}
{सूतं चास्य महाबाहुर्गृह्य राजंस्तथैव च}


\twolineshloka
{पादयोर्गृह्य तौ वीरौ भीमः कर्णस्य पश्यतः}
{भूमावाविद्ध्य जघ्ने स तौ च प्राणैर्व्ययुज्यताम्}


\twolineshloka
{कर्णस्तु पाण्डवे शक्तिं काञ्चनीं समवासृजत्}
{ततस्तामेव जग्राह प्रहसन्पाण्डुनन्दनः}


\twolineshloka
{कर्णायैव च दुर्धर्षश्चिक्षेपाजौ वृकोदरः}
{तामापतन्तीं चिच्छेद शकुनिस्तैलपायिना}


\twolineshloka
{एतत्कृत्वा महत्कर्म रणेऽद्भुतपराक्रमः}
{पुनः स्वरथमास्थाय दुद्राव तव वाहिनीम्}


\threelineshloka
{तमायान्तं जिघांसन्तं भीमं क्रुद्धमिवान्तकम्}
{न्यवारयन्महाबाहुं तव पुत्रा विशाम्पते}
{महता शरवर्षेण च्छादयन्तो महारथाः}


\twolineshloka
{दुर्मदस्य ततो भीमः प्रहसन्निव संयुगे}
{सारथिं च हयांश्चैव शरैर्निन्ये यमक्षयम्}


\threelineshloka
{दुर्मदस्तु ततो यानं दुष्कर्णस्यावचक्रमे}
{तावेकरथमारूढौ भ्रातरौ परतापनौ}
{}


\twolineshloka
{सङ्ग्रामशिरसो मध्ये भीमं द्वावप्यधावताम्}
{यथाम्बुपतिमित्रौ हि तारकं दैत्यसत्तमम्}


\twolineshloka
{ततस्तु दुर्मदश्चैव दुष्कर्णश्च तवात्मजौ}
{रथमेकं समारुह्य भीमं बाणैरविध्यताम्}


\twolineshloka
{ततः कर्णस्य मिषतो द्रौणेर्दुर्योधनस्य च}
{कृपस्य सोमदत्तस्य बाह्लीकस्य च पाण्डवः}


\twolineshloka
{दुर्मदस्य च वीरस्य दुष्कर्णस्य च तं रथम्}
{पादप्रहारेण धरां प्रावेशयदरिन्दमः}


\twolineshloka
{ततः सुतौ ते बलिनौ शूरौ दुष्कर्णदुर्मदौ}
{मुष्टिनाऽऽहत्य सङ्क्रुद्धौ ममर्द चरणेन ह}


\twolineshloka
{ततो हाहाकृते सैन्ये दृष्ट्वा भीमं नृपाऽब्रुवन्}
{रुद्रोऽयं भीमरूपेण धार्तराष्ट्रेषु युध्यति}


\twolineshloka
{एवमुक्त्वा पलायन्ते सर्वे भारत पार्थिवाः}
{विसंज्ञा वाहयन्वाहान्न च द्वौ सह धावतः}


\twolineshloka
{ततो बले भृशलुलिते निशामुखेसुपूजितो नृपवृषभैर्वृकोदरः}
{महाबलः कमलविबुद्धलोचनोयुधिष्ठिरं नृपतिमपूजयद्बली}


\twolineshloka
{ततो यमौ द्रुपदविराटकेकयायुधिष्ठिरश्चापि परां मुदं ययुः}
{वृकोदरं भृशमनुपूजयंश्च तेयथान्धके प्रतिनिहते हरं सुराः}


\twolineshloka
{ततः सुतास्ते वरुणात्मजोपमारुपान्विताः सह गुरुणा महात्मना}
{वृकोदरं सरथपदातिकुञ्जरायुयुत्सवो भृशमभिपर्यवारयन्}


\twolineshloka
{`ततो यमौ द्रुपदसुताः ससैनिकायुधिष्ठिरद्रुपदविराटसात्वताः}
{घटोत्कचो जयविजयौ द्रुमो वृकःससृञ्जयास्तव तनयानवारयन्'}


\twolineshloka
{ततोऽभवत्तिमिरघनैरिवावृतेमहाभये भयदमतीव दारुणम्}
{निशामुखे वृकबलगृध्रमोदनंमहात्मना नृपवर युद्धमद्भुतम्}


\chapter{अध्यायः १५७}
\twolineshloka
{सञ्जय उवाच}
{}


\twolineshloka
{प्रायोपविष्टे तु हते पुत्रे सात्यकिना तदा}
{सोमदत्तो भृशं क्रुद्धः सात्यकिं वाक्यमब्रवीत्}


\twolineshloka
{क्षत्रधर्मः परो दृष्टो यस्तु देवैर्महात्मभिः}
{तं त्वं सात्वत सन्तज्य दस्युधर्मे कथं रतः}


\twolineshloka
{पराङ्मुखाय दीनाय न्यस्तशस्त्राय याचते}
{क्षत्रधर्मरतः प्राज्ञः कथं नु प्रहरेद्रणे}


\twolineshloka
{द्वावेव किल वृष्णीनां तत्र ख्यातौ महारथौ}
{प्रद्युम्नश्च महाबाहुस्त्वं चैव युधि सात्वत}


\twolineshloka
{कथं प्रायोपविष्टाय पार्थेन च्छिन्नबाहवे}
{नृशंसं बत दीनं च तादृशं कृतवानसि}


\twolineshloka
{कर्ममस्तस्य दुर्वृत्त फलं प्राप्नुहि संयुगे}
{अद्य छेत्स्यामि ते मूढ शिरो विक्रम्य पत्रिणा}


\twolineshloka
{शपे सात्वत पुत्राभ्यामिष्टेन सुकृतेन च}
{अनतीतामिमां रात्रिं यदि त्वां वीरमानिनम्}


\twolineshloka
{अरक्ष्यमाणं पार्थेन जिष्णुना ससुतानुजम्}
{न हन्यां नरके घोरे पतेयं वृष्णिपांसन}


\twolineshloka
{एवमुक्त्वा सुसङ्क्रुद्धः सोमदत्तो महाबलः}
{दध्मौ शङ्खं च तारेण सिंहनादं ननाद च}


\twolineshloka
{ततः कमलपत्राक्षः सिंहदंष्ट्रो दुरासदः}
{सात्यकिर्भृशसङ्क्रुद्धः सोमदत्तमथाब्रवीत्}


\twolineshloka
{कौरवेय न मे त्रासः कथञ्चिदपि विद्यते}
{त्वया सार्धमथान्यैश्च युध्यतो हृदि कश्चन}


\twolineshloka
{यदि सर्वेण सैन्येन गुप्तो मां योधयिष्यसि}
{तथापि न व्यथा काचित्त्वयि स्यान्मम कौरव}


\twolineshloka
{युद्धसारेण वाक्येन असतां सम्मतेन च}
{नाहं भीषयितुं शक्यः क्षत्रवृत्ते स्थितस्त्वया}


\twolineshloka
{यदि तेऽस्ति युयुत्साऽद्य मया सह नराधिप}
{निर्दयो निशितैर्बाणैः प्रहर प्रहरामि ते}


\twolineshloka
{हतो भूरिश्रवा वीरस्तव पुत्रो महारथः}
{शलश्चैव महाराज भ्रातृव्यसनकर्षितः}


\twolineshloka
{त्वां चाप्यद्य वधिष्यामि सहपुत्रं सबान्धवम्}
{तिष्ठेदानीं रणे यत्तः कौरवोऽसि महारथः}


\twolineshloka
{यस्मिन्दानं दमः शौचमहिंसा हीर्धृतिः क्षमा}
{अनपायानि सर्वाणि नित्यं राज्ञि युधिष्ठिरे}


\twolineshloka
{मृदङ्गकेतोस्तस्य त्वं तेजसा निहतः पुरा}
{सकर्णसौबलः सङ्ख्ये विनाशमुपयास्यसि}


\twolineshloka
{शपेऽहं कृष्णचरणैरिष्टापूर्तेन चैव ह}
{यदि त्वां ससुतं पापं न हन्यां युधि रोषितः}


% Check verse!
अपयास्यसि चोत्त्यक्त्वा रणं मुक्तो भविष्यसि
\twolineshloka
{एवमाभाष्य चान्योन्यं क्रोधसंरक्तलोचनौ}
{प्रवृत्तौ शरसम्पातं कर्तुं पुरुषसत्तमौ}


\twolineshloka
{ततो रथसहस्रेण नागानामयुतेन च}
{दुर्योधनः सोमदत्तं परिवार्य व्यवस्थितः}


\threelineshloka
{शकुनिश्च सुसङ्क्रुद्धः सर्वशस्त्रभृतां वरः}
{पुत्रपौत्रैः परिवृतो भ्रातृभिश्चेन्द्रविक्रमैः}
{श्यालस्तव महाबाहुर्वज्रसंहननो युवा}


\twolineshloka
{साग्रं शतसहस्रं तु हयानां तस्य धीमतः}
{सोमदत्तं महेष्वासं समन्तात्पर्यरक्षत}


\twolineshloka
{रक्ष्यमाणश्च बलिभिश्छादयामास सात्यकिम्}
{तं छाद्यमानं विशिखैर्दृष्ट्वा सन्नतपर्वभिः}


\twolineshloka
{धृष्टद्युम्नोऽभ्ययात्क्रुद्धः प्रगृह्य महतीं चमूम्}
{`अभ्यरक्षन्महाबाहुः सात्वतं सत्यविक्रमम्'}


\twolineshloka
{चण्डवाताभिसृष्टानामुदधीनामिव स्वनः}
{आसीद्राजन्बलौघानामन्योन्यमभिनिघ्नताम्}


\twolineshloka
{विव्याध सोमदत्तस्तु सात्वतं नवभिः शरैः}
{सात्यकिर्नवभिश्चैनमवधीत्कुरुपुङ्गवम्}


\twolineshloka
{सोऽतिविद्धो बलवता समरे दृढधन्विना}
{रथोपस्थं समासाद्य मुमोह गतचेतनः}


\twolineshloka
{तं विमूढं समालक्ष्य सारथिस्त्वरयाऽन्वितः}
{अपोवाह रणाद्वीरं सोमदत्तं महारथम्}


\twolineshloka
{तं विसंज्ञं समालक्ष्य युयुधानशरार्दितम्}
{अभ्यद्रवत्ततो द्रोणो यदुवीरजिघांसया}


\twolineshloka
{तमायान्तमभिप्रेक्ष्य युधिष्ठिरपुरोगमाः}
{परिवव्रुर्महात्मानं परीप्सन्तो यदूत्तमम्}


\twolineshloka
{ततः प्रववृते युद्धं द्रोणस्य सह पाण्डवैः}
{बलेरिव सुरैः पूर्वं त्रैलोक्यजयकाङ्क्षया}


\twolineshloka
{ततः सायकजालेन पाण्डवानीकमावृणोत्}
{भारद्वाजो महातेजा विव्याध च युधिष्ठिरम्}


\twolineshloka
{सात्यकिं दशभिर्बाणैर्विंशत्या पार्षतं शरैः}
{भीमसेनं च नवभिर्नकुलं पञ्चभिस्तथा}


\twolineshloka
{सहदेवं तथाऽष्टाभिः शतेन च शिखण्डिनम्}
{द्रौपदेयान्महाबाहुः पञ्चभिः पञ्जभिः शरैः}


\threelineshloka
{विराटं मत्स्यमष्टाभिर्द्रुपदं दशभिः शरैः}
{युधामन्युं त्रिभिः षड्भिरुत्तमौजसमाहवे}
{अन्यांश्च सैनिकान्विद्व्वा युधिष्ठिरमुपाद्रवत्}


\twolineshloka
{ते वध्यमाना द्रोणेन पाण्डुपुत्रस्य सैनिकाः}
{प्राद्रवन्वै भयाद्राजन्सार्तनादा दिशो दश}


\twolineshloka
{काल्यमानं तु तत्सैन्यं दृष्ट्वा द्रोणेन फल्गुनः}
{किञ्चिदागतसंरम्भो गुरुं पार्थोऽभ्ययाद्द्रुतम्}


\twolineshloka
{दृष्ट्वा द्रोणं तु बीभत्सुमभिधावन्तमाहवे}
{सन्न्यवर्तत तत्सैन्यं पुनर्यौधिष्ठिरं बलम्}


% Check verse!
ततो युद्धमभूद्भूयो भारद्वाजस्य पाण्डवैः
\twolineshloka
{द्रोणस्तव सुतै राजन्सर्वतः परिवारितः}
{व्यधमत्पाण्डुसैन्यानि तूलराशिमिवानलः}


\twolineshloka
{तं ज्वलन्तमिवादित्यं दीप्तानलसमद्युतिम्}
{राजन्ननिशमत्यन्तं दृष्ट्वा द्रोणं शरार्चिषम्}


\twolineshloka
{मण्डलीकृतधन्वानं तपन्तमिव भास्करम्}
{दहन्तमहितान्सैन्ये नैनं कश्चिदवारयत्}


\twolineshloka
{योयो हि प्रमुखे तस्य तस्थौ द्रोणस्य पुरूषः}
{तस्यतस्य शिरश्छित्त्वा ययुर्द्रोणशराः क्षितिम्}


\twolineshloka
{एवं सा पाण्डवी सेना वध्यमाना महात्मना}
{प्रदुद्राव पुनर्भीता पश्यतः सव्यसाचिनः}


\twolineshloka
{सम्प्रभग्नं बलं दृष्ट्वा द्रोणेन निशि भारत}
{गोविन्दमब्रवीज्जिष्णुर्गच्छ द्रोणरथं प्रति}


\twolineshloka
{ततो रजतगोक्षीरकुन्देन्दुसदृशप्रभान्}
{चोदयामास दाशार्हो हयान्द्रोणरथं प्रति}


\twolineshloka
{भीमसेनोऽपि तं दृष्ट्वा यान्तं द्रोणाय फल्गुनम्}
{स्वसारथिमुवाचेदं द्रोणानीकाय मां वह}


\twolineshloka
{सोऽपि तस्य वचः श्रुत्वा विशोकोऽवहयद्वयान्}
{पृष्ठतः सत्यसन्धस्य जिष्णोर्भरतसत्तम}


\threelineshloka
{तौ दृष्ट्वा भ्रातरौ यत्तौ द्रोणानीकमभिद्रुतौ}
{पाञ्चालाः सृञ्जया मात्स्याश्चेदिकारूशकोसलाः}
{अन्वगच्छन्महाराज केकयाश्च महारथाः}


% Check verse!
ततो राजन्नभूद्वोरः सङ्ग्रामो रोमहर्षणः
\twolineshloka
{बीभत्सुर्दक्षिणं पार्श्वमुत्तरं च वृकोदरः}
{महद्भ्यां रथबृन्दाभ्यां बलं जगृहतुस्तव}


\twolineshloka
{तौ दृष्ट्वा पुरुषव्याघ्रौ भीमसेनधनञ्जयौ}
{धृष्टद्युम्नोऽभ्ययाद्राजन्सात्यकिश्च महाबलः}


\twolineshloka
{चण्डवाताभिपन्नानामुदधीनामिव स्वनः}
{आसीद्राजन्बलौघानां तदाऽन्योन्यमभिघ्नतां}


\twolineshloka
{सौमदत्तिवधात्क्रुद्धो दृष्ट्वा सात्यकिमाहवे}
{द्रौणिरभ्यद्रवद्राजन्वधाय कृतनिश्चयः}


\twolineshloka
{तमापतन्तं सम्प्रेक्ष्य शैनेयस्य रथं प्रति}
{भैमसेनिः सुसङ्क्रुद्धः प्रत्यमित्रमवारयत्}


\twolineshloka
{कार्ष्णायसं महाघोरमृक्षचर्मपरिच्छदम्}
{महान्तं रथमास्थाय त्रिंशन्नल्वान्तरान्तरम्}


\twolineshloka
{विक्षिप्तयन्त्रसन्नाहं महामेघौघनिःस्वनम्}
{युक्तं गजनिभैर्वाहैर्न हयैर्नापि वारणैः}


\twolineshloka
{विक्षिप्तपक्षचरणविवृताक्षेण कूजता}
{ध्वजेनोच्छ्रितदण्डेन गृध्रराजेन राजितम्}


\twolineshloka
{लोहितार्द्रपताकं तु आन्त्रमालाविभूषितम्}
{अष्टचक्रसमायुक्तमास्थाय विपुलं रथम्}


\twolineshloka
{शूलमुद्गरधारिण्या शैलपादपहस्तया}
{रक्षसां घोररूपाणामक्षौहिण्या समावृतः}


\twolineshloka
{तमुद्यतमहाचापं निशाम्य व्यथिता नृपाः}
{युगान्तकालसमये दण्डहस्तमिवान्तकम्}


\twolineshloka
{ततस्तं गिरिशृङ्गाभं भीमरूपं भयावहम्}
{दंष्ट्राकरालोग्रमुखं शङ्कुकर्णं महाहनुम्}


\twolineshloka
{ऊर्ध्वकेशं विरूपाक्षं दीप्तास्यं निम्नितोदरम्}
{महाश्वभ्रगलद्वारं किरीटच्छन्नमूर्धजम्}


\twolineshloka
{त्रासनं सर्वभूतानां व्यात्ताननभिवान्तकम्}
{वीक्ष्य दीप्तमिवायान्तं रिपुविक्षोभकारिणम्}


\threelineshloka
{तमुद्यतमहाचापं राक्षसेन्द्रं घटोत्कचम्}
{भयार्दिता प्रचुक्षोभ पुत्रस्य तव वाहिनी}
{वायुना क्षोभितावर्ता गङ्गेवोर्ध्वतरङ्गिणी}


\twolineshloka
{घटोत्कचप्रयुक्तेन सिंहनादेन भीषिताः}
{प्रसुस्रुवुर्गजा मूत्रं विव्यथुश्च नरा भृशम्}


\twolineshloka
{ततोऽश्मवृष्टिरत्यर्थमासीत्तत्र समन्ततः}
{सन्ध्याकालाधिकबलैः प्रयुक्ता राक्षसैः क्षितौ}


\twolineshloka
{आयसानि च चक्राणि भुशुण्ड्यः प्रासतोमराः}
{पतन्त्यविरताः शूलाः शतघ्न्यः पट्टसास्तथा}


\twolineshloka
{तदुग्रमतिरौद्रं च दृष्ट्वा युद्धं नराधिपाः}
{तनयास्तव कर्णश्च व्यथिताः प्राद्रवन्दिशः}


\twolineshloka
{तत्रैकोऽस्त्रबलश्लाघी द्रौणिर्मानी न विव्यथे}
{व्यधमच्च शरैर्मायां घटोत्कचविनिर्भिताम्}


\threelineshloka
{विहतायां तु मायायाममर्षी स घटोत्कचः}
{विससर्ज शरान्घोरांस्तेऽश्वत्थामानमाविशन्}
{भुजङ्गा इव वेगेन वल्मीकं क्रोधमूर्च्छिताः}


\twolineshloka
{ते शरा रुधिराक्ताङ्गा भित्त्वा शारद्वतीसुतम्}
{विविशुर्धरणीं शीघ्रा रुक्मपुङ्खाः शिलाशिताः}


\twolineshloka
{अश्वत्थामा तु सङ्क्रुद्धो लघुहस्तः प्रतापवान्}
{घटोत्कचमभिक्रुद्धं बिभेद दशभिः शरैः}


\twolineshloka
{घटोत्कचोऽतिविद्धस्तु द्रोणपुत्रेण मर्मसु}
{चक्रं शतसहस्रारमगृह्णाद्व्यथितो भृशम्}


\twolineshloka
{क्षुरान्तं बालसूर्याभं मणिवज्रविभूषितम्}
{अश्वत्थाम्नि स चिक्षेप भैमसेनिर्जिघांसया}


\twolineshloka
{वेगेन महता गच्छद्विक्षिप्तं द्रौणिना शरैः}
{अभाग्यस्येव सङ्कल्पस्तन्मोघमपतद्भुवि}


\twolineshloka
{घटोत्कचस्ततस्तूर्णं दृष्ट्वा चक्रं निपातितम्}
{द्रौणिं प्राच्छादयद्बाणैः स्वर्भानुरिव भास्करम्}


\twolineshloka
{घटोत्कचसुतः श्रीमान्भिन्नाञ्जनचयोपमः}
{रुरोध द्रौणिमायान्तं प्रभञ्जनमिवाद्रिराट्}


\twolineshloka
{पौत्रेण भीमसेनस्य शरैरञ्जनपर्वणा}
{बभौ मेघेन धाराभिर्गिरिर्मेरुरिवावृतः}


\twolineshloka
{अश्वत्थामा त्वसम्भ्रान्तो रुद्रोपेन्द्रेन्द्रविक्रमः}
{ध्वजमेकेन बाणेन चिच्छेदाञ्जनपर्वणः}


\twolineshloka
{द्वाभ्यां तु रथयन्तारौ त्रिभिश्चास्य त्रिवेणुकम्}
{धनुरेकेन चिच्छेद चतुर्भिश्चतुरो हयान्}


\twolineshloka
{विरथस्योद्यतं हस्ताद्धेमबिन्दुभिराचितम्}
{विशिखेन सतीक्ष्णेन खङ्गमस्य द्विधाऽकरोत्}


\twolineshloka
{गदा हेमाङ्गदा राजंस्तूर्णं हैडिम्बिसूनुना}
{भ्राम्योत्क्षिप्ताशरैःसापि द्रौणिनाभ्याहताऽपतत्}


\twolineshloka
{ततोऽन्तरिक्षमुत्प्लुत्य कालमेघ इवोन्नदन्}
{ववर्षाञ्जनपर्वा स द्रुमवर्षं नभस्तलात्}


\twolineshloka
{ततो मायाधरं द्रौणिर्घटोत्कचसुतं दिवि}
{मार्गणैरभिविव्याध धनं सूर्य इवांशुभिः}


\twolineshloka
{सोऽवतीर्य पुरस्तस्थौ रथे हेमविभूषिते}
{महीगत इवात्युग्रः श्रीमानञ्जनपर्वतः}


\twolineshloka
{तमयस्मयवर्माणं द्रौणिर्भीमात्मजात्मजम्}
{जघानाञ्जनपर्वाणं महेश्वर इवान्धकम्}


\twolineshloka
{अथ दृष्ट्वा हतं पुत्रमश्वत्थाम्ना महाबलम्}
{द्रौणेः सकाशमभ्येत्य रोषात्प्रस्फुरिताधरः}


\threelineshloka
{प्राह वाक्यमसम्भ्रान्तो वीरं शारद्वतीसुतम्}
{दहन्तं पाण्डवानीकं वनमग्निमिवोच्छ्रितम् ॥घटोत्कच उवाच}
{}


\threelineshloka
{तिष्ठतिष्ठ न मे जीवन्द्रोणपुत्र गमिष्यसि}
{त्वामद्य निहनिष्यामि क्रौञ्चमग्निसुतो यथा ॥अश्वत्थामोवाच}
{}


\twolineshloka
{गच्छ वत्स सहान्यैस्त्वं युध्यस्वामरविक्रम}
{न हि पुत्रेण हैडिम्ब पिता न्याय्यः प्रबाधितुम्}


\threelineshloka
{कामं खलु न रोषो मे हैडिम्बे विद्यते त्वयि}
{किं तु रोषान्वितो जन्तुर्हन्यादात्मानमप्युत ॥सञ्जय उवाच}
{}


\twolineshloka
{श्रुत्वैतत्क्रोधताम्राक्षः पुत्रशोकसमन्वितः}
{अश्वत्थामानमायस्तो भैमसेनिरभाषत}


% Check verse!
किमहं कातरो द्रौणे पृथग्जन इवाहवे ॥यन्मां भीषयसे वाग्भिरसदेतद्वचस्तव
\twolineshloka
{भीमात्खलु समुत्पन्नः कुरूणां विपुले कुले}
{पाण्डवानामहं पुत्रः समरेष्वनिवर्तिनाम्}


\twolineshloka
{रक्षसामधिराजोऽहं दशग्रीवसमो बले}
{तिष्ठतिष्ठ न मे जीवन्द्रोणपुत्र गमिष्यसि}


% Check verse!
युद्धश्रद्धामहं तेऽद्य विनेष्यामि रणाजिरे
\twolineshloka
{इत्युक्त्वा क्रोधताम्राक्षो राक्षसः सुमहाबलः}
{द्रौणिमभ्यद्रवत्क्रुद्धो गजेन्द्रमिव केसरी}


\twolineshloka
{रथाक्षमात्रैरिषुभिरभ्यवर्षद्धटोत्कचः}
{रथिनामृषभं द्रौणिं धाराभिरिव तोयदः}


\twolineshloka
{शरवृष्टिं शरैर्द्रौणिरप्राप्तां तां व्यशातयत्}
{ततोन्तरिक्षे बाणानां सङ्ग्रामोऽन्य इवाभवत्}


\twolineshloka
{अथास्त्रसम्मर्दकृतैर्विस्फुलिङ्गैस्तदा बभौ}
{विभावरीमुखे व्योम खद्योतैरिव चित्रितम्}


\twolineshloka
{निशाम्य निहतां मायां द्रौणिना रणमानिना}
{घटोत्कचस्ततो मायां ससर्जान्तर्हितः पुनः}


\twolineshloka
{सोऽभवद्गिरिरत्युच्चः शिखरैस्तरुसङ्कटैः}
{शूलप्रासासिमुसलजलप्रस्रवणो महान्}


\twolineshloka
{तमञ्जनगिरिप्रख्यं द्रौणिर्दृष्ट्वा महीधरम्}
{प्रपतद्भिश्च बहुभिः शस्त्रसङ्घैर्न विव्यथे}


\twolineshloka
{ततो हसन्निव द्रौणिर्वज्रमस्त्रमुदैरयत्}
{स ते नास्त्रैण शैलेन्द्रः क्षिप्तः क्षिप्रं व्यनश्यत}


\twolineshloka
{ततः स तोयदो भूत्वा नीलः सेन्द्रायुधो दिवि}
{अश्मवृष्टिभिरत्युग्रो द्रौणिमाच्छादयद्रणे}


\twolineshloka
{अथ सन्धाय वायव्यमस्त्रमस्त्रविदां वरः}
{व्यधमद्द्रोणतनयो नीलमेघं समुत्थितम्}


\twolineshloka
{स मार्गणगणैर्द्रौणिर्दिशः प्रच्छाद्य सर्वशः}
{शतं रथसहस्राणां जघान द्विपदां वरः}


\twolineshloka
{स दृष्ट्वा पुनरायान्तं रथेनायतकार्मुकम्}
{घटोत्कचमसम्भ्रान्तं राक्षसैर्बहुभिर्वृतम्}


\twolineshloka
{सिंहशार्दूलसदृशैर्मत्तद्विरदविक्रमैः}
{गजस्थैश्च रथस्थैश्च वाजिपृष्ठगतैरपि}


\twolineshloka
{विकृतास्यशिरोग्रीवैर्हिडिम्बानुचरैः सह}
{पौलस्त्यैर्यातुधानैश्च तामसैश्चेन्द्रविक्रमैः}


\twolineshloka
{नानाशस्त्रधरैर्वीरैर्नानाकवचभूषणैः}
{महाबलैर्भीमरवैः संरम्भोद्वृत्तलोचनैः}


\twolineshloka
{उपस्थितैस्ततो युद्धे राक्षसैर्युद्धदुर्मदैः}
{विषण्णमभिसम्प्रेक्ष्य पुत्रं ते द्रौणिरब्रवीत्}


\twolineshloka
{तिष्ठ दुर्योधनाद्य त्वं न कार्यः सम्भ्रमस्त्वया}
{सहैभिर्भ्रातृभिर्वीरैः पार्थिवैश्चेन्द्रविक्रमैः}


\twolineshloka
{निहनिष्याम्यमित्रांस्त न तवास्ति पराजयः}
{सत्यं ते प्रतिजानामि पर्याश्वासय वाहिनीम् ॥दुर्योधन उवाच}


\threelineshloka
{न त्वेतदद्भुतं मन्ये यत्ते महदिदं मनः}
{अस्मासु च परा भक्तिस्तव गौतमिनन्दन ॥सञ्जय उवाच}
{}


\twolineshloka
{अश्वत्थामानमुक्त्वैवं ततः सौबलमब्रवीत्}
{वृतं रथसहस्रेण हयानां रणशोभिनाम्}


\twolineshloka
{षष्ट्या रथसहस्रैश्च प्रयाहि त्वं धनञ्जयम्}
{कर्णश्च वृषसेनश्च कृपो नीलस्तथैव च}


\twolineshloka
{उदीच्याः कृतवर्मा च पुरुमित्रः सुतापनः}
{दुःशासनो निकुम्भश्च कुण्डभेदी पराक्रमः}


\twolineshloka
{पूरञ्जयो दृढरथः पताकी हेमकम्पनः}
{शल्यारुणीन्द्रसेनाश्च सञ्जयो विजयो जयः}


\twolineshloka
{कमलाक्षः परक्राथी जयवर्मा सुदर्शनः}
{एते त्वामनुयास्यन्ति पत्तीनामयुतानि षट्}


\twolineshloka
{जहि भीमं यमौ चोभौ धर्मराजं च मातुल}
{असुरानिव देवेन्द्रो जयाशा मे त्वयि स्थिता}


\threelineshloka
{दारितान्द्रौणिना बाणैर्भृशं विक्षतविग्रहान्}
{जहि मातुल कौन्तेयानसुरानिव पावकिः ॥सञ्जय उवाच}
{}


\twolineshloka
{एवमुक्तो ययौ शीघ्रं पुत्रेण तव सौबलः}
{पिप्रीषुस्ते सुतान्राजन्दिधक्षुश्चैव पाण्डवान्}


\twolineshloka
{अथ प्रववृते युद्धं द्रौणिराक्षसयोर्मृधे}
{विभावर्यां सुतुमुलं शक्रप्रह्लादयोरिव}


\twolineshloka
{ततो घटोत्कचो बाणैर्दशभिर्गौतमीसुतम्}
{जघानोरसि सङ्क्रुद्धौ विषाग्निप्रतिमैर्दृढैः}


\twolineshloka
{स तैरभ्याहतो गाढं शरैर्भीमसुतेरितैः}
{चचाल रथमध्यस्थो वातोद्धूत इव द्रुमः}


\twolineshloka
{भूयश्चाञ्जलिकेनाथ मार्गणेन महाप्रभम्}
{द्रौणिहस्तस्थितं चापं चिच्छेदाशु घटोत्कचः}


\twolineshloka
{ततोऽन्यद्द्रौणिरादय धनुर्भारसहं महत्}
{ववर्ष विशिखांस्तीक्ष्णान्वारिधारा इवाम्बुदः}


\twolineshloka
{ततः शारद्वतीपुत्रः प्रेषयामास भारत}
{सुवर्णपुङ्खाञ्छत्रुघ्नान्खचरान्खचरं प्रति}


\twolineshloka
{तद्बाणैरर्दितं यूथं रक्षसां पीनवक्षसाम्}
{सिंहैरिव बभौ मत्तं गजानामाकुलं कुलम्}


\twolineshloka
{विधम्य राक्षसान्बाणैः साश्वसूतरथद्विपान्}
{ददाह भगवान्वह्निर्भूतानीव युगक्षये}


\twolineshloka
{स दग्ध्वाऽक्षौहिणीं बामैर्नैर्ऋतीं रुरुचे नृप}
{पुरेव त्रिपुरं दग्ध्वा दिवि देवो महेश्वरः}


\twolineshloka
{युगान्ते सर्वभूतानि दग्ध्वेव वसुरुल्बणः}
{रराज जयतां श्रेष्ठो द्रोणपुत्रस्तवाहितान्}


\twolineshloka
{ततो घटोत्कचः क्रुद्धो रक्षसां भीमकर्मणाम्}
{द्रौणिं हतेति महतीं चोदयामास तां चमूम्}


\twolineshloka
{घटोत्कचस्य तामाज्ञां प्रतिगृह्याथ राक्षसाः}
{दंष्ट्रोज्ज्वला महावक्त्रा घोररूपा भयानकाः}


\twolineshloka
{व्यात्तानना घोरजिह्वाः क्रोधताम्रेक्षणा भृशम्}
{सिंहनादेन महता नादयन्तो वसुन्धराम्}


\twolineshloka
{हन्तुमभ्यद्रवन्द्रौणिं नानाप्रहरणायुधाः}
{शक्तीः शतघ्नीः परिघानशनीः शूलपट्टशान्}


\twolineshloka
{खङ्गान्गदाभिण्डिपालान्मुसलानि परश्वथान्}
{प्रासानसींस्तोमरांश्च कणपान्कम्पनाञ्छितान्}


\twolineshloka
{स्थूलान्भुशुण्ड्यश्मगदास्थूणान्कार्ष्ण्यायसांस्तथा}
{मुद्गराँश्च महाघोरान्समरे शत्रुदारणान्}


\twolineshloka
{द्रौणिमूर्धन्यसन्त्रस्ता राक्षसा भीमविक्रमाः}
{चिक्षिपुः क्रोधताम्राक्षाः शतशोऽथ सहस्रशः}


\twolineshloka
{तच्छस्त्रवर्षं सुमहद्द्रोणपुत्रस्य मूर्धनि}
{पतमानं समीक्ष्याऽथ योधास्ते व्यथिताऽभवन्}


\twolineshloka
{द्रोणपुत्रस्तु विक्रान्तस्तद्वर्षं घोरमुच्छ्रितम्}
{शरैर्विध्वंसयामास वज्रकल्पैः शिलाशितैः}


\twolineshloka
{ततोऽन्यैर्विशिखैस्तूर्णं स्वर्णपुङ्खैर्महामनाः}
{निजघ्ने राक्षसान्द्रौणिर्दिव्यास्त्रप्रतिमन्त्रितैः}


\twolineshloka
{तद्बाणैरर्दितं यूथं रक्षसां पीनवक्षसाम्}
{सिंहैरिव बभौ मत्तं गजानामाकुलं कुलम्}


\twolineshloka
{ते राक्षसाः सुसङ्क्रुद्धा द्रोणपुत्रेण ताडिताः}
{क्रुद्धाः स्म प्राद्रवन्द्रौणिं जिघांसन्तो महाबलः}


\twolineshloka
{तत्राद्भुतमिमं द्रौणिर्दर्शयामास विक्रमम्}
{अशक्यं कर्तुमन्येन सर्वभूतेषु भारत}


\twolineshloka
{यदेको राक्षसीं सेनां क्षणाद्द्रौणिर्महास्त्रवित्}
{ददाह ज्वलितैर्बाणै राक्षसेन्द्रस्य पश्यतः}


\twolineshloka
{स हत्वा राक्षसानीकं रराज द्रौणिराहवे}
{युगान्ते सर्वभूतानि संवर्तक इवानलः}


\twolineshloka
{तं दहन्तमनीकानि शरैराशीविषोपमैः}
{तेषु राजसहस्रेषु पाण्डवेयेषु भारत}


\twolineshloka
{नैनं निरीक्षितुं कश्चिदशक्नोद्द्रौणिमाहवे}
{ऋते घटोत्कचाद्वीराद्राक्षसेन्द्रान्महाबलात्}


\twolineshloka
{स पुनर्भरतश्रेष्ठ क्रोधादुद्धान्तलोचनः}
{तलं तलेन संहत्य सन्दश्य दशनच्छदम्}


\twolineshloka
{स्व सूतमब्रवीत्क्रुद्धो द्रोणपुत्राय मां वह}
{स ययौ घोररूपेण सुपताकेन भास्वता}


\twolineshloka
{द्वैरथं द्रोणपुत्रेण पुनरप्यरिसूदनः}
{स विनद्य महानादं सिंहवद्भीमविक्रमः}


\twolineshloka
{चिक्षेपाविद्ध्य सङ्ग्रामे द्रोणपुत्राय राक्षसः}
{अष्टघण्टां महाघोरामशनिं देवनिर्मिताम्}


\twolineshloka
{तामवप्लुत्य जग्राह द्रौणिर्न्यस्य रथे धनुः}
{चिक्षेप चैनां तस्यैव स्यन्दनात्सोऽवप्लुप्लुवे}


\twolineshloka
{साश्वसूतध्वजं यानं भस्म कृत्वा महाप्रभा}
{विवेश वसुधां भित्त्वा साऽशनिर्भृशदारुणा}


\twolineshloka
{द्रौणेस्तत्कर्म दृष्ट्वा तु सर्वभूतान्यपूजयन्}
{यदप्लुत्य जग्राह घोरां शङ्करनिर्मिताम्}


\threelineshloka
{धृष्टद्युम्नरथं गत्वा भैमसेनिस्ततो नृप}
{धनुर्घोरं समादाय महदिन्द्रायुधोपमम्}
{मुमोच निशितान्बाणान्पुनर्द्रौणेर्महोरसि}


\twolineshloka
{धृष्टद्युम्नस्त्वसम्भ्रान्तो मुमोचाशीविषोपमान्}
{सुवर्णपुङ्खान्विशिखान्द्रोणपुत्रस्य वक्षसि}


\twolineshloka
{ततो मुमोच नाराचान्द्रौणिस्तांश्च सहस्रशः}
{तावप्यग्निशिखप्रख्यैर्जुघ्नतुस्तस्य मार्गणान्}


\twolineshloka
{अतितीव्रं महद्युद्धं तयोः पुरुषसिंहयोः}
{योधानं प्रीतिजननं द्रौणेश्च भरतर्षभ}


\twolineshloka
{ततो रथसहस्रेण द्विरदानां शतैस्त्रिभिः}
{षड्भिर्वाजिसहस्रैश्च भीमस्तं देशमागमत्}


\twolineshloka
{ततो भीमात्मजं रक्षो धृष्टद्युम्नं च सानुगम्}
{अयोधयत धर्मात्मा द्रौणिरक्लिष्टविक्रमः}


\twolineshloka
{तत्राद्भुततमं द्रौणिर्दर्शयामास विक्रमम्}
{अशक्यं कर्तुमन्येन सर्वभूतेषु भारत}


\twolineshloka
{निमेषान्तरमात्रेण साश्वसूतरथद्विपाम्}
{अक्षौहिणीं राक्षसानां शितैर्बाणैरशातयत्}


\twolineshloka
{मिषतो भीमसेनस्य हैडिम्बेः पार्षतस्य च}
{यमयोर्धर्मपुत्रस्य विजयस्याच्युतस्य च}


\twolineshloka
{प्रगाढमञ्जोगतिभिर्नाराचैरभिताडिताः}
{निपेतुर्द्विरदा भूमौ सशृङ्गा इव पर्वताः}


\twolineshloka
{निकृत्तैर्हस्तिहस्तैश्च विचलद्भिरितस्ततः}
{रराज वसुधा कीर्णा विसर्पद्भिरिवोरगैः}


\twolineshloka
{क्षिप्तैः काञ्चनदण्डैश्च नृपच्छत्रैः क्षितिर्बभौ}
{द्यौरिवोदितचन्द्रार्का ग्रहाकीर्णा युगक्षये}


% Check verse!
प्रवृद्धध्वजमण्डूकां भेरीविस्तीर्णकच्छपाम् ॥छत्रहंसावलीजुष्टां फेनचामरमालिनीम्
\twolineshloka
{कङ्कगृघ्रमहाग्राहां नैकायुधझषाकुलाम्}
{विस्तीर्णगजपाषाणां हताश्वमकराकुलाम्}


\twolineshloka
{रथक्षिप्तमहावप्रां पताकारुचिरद्रुमाम्}
{शरमीनां महारौद्रां प्रासशक्त्यृष्टिडुण्डुभाम्}


\twolineshloka
{मज्जामांसमहापङ्कां कबन्धावर्जितोडुपाम्}
{केशशैवलकल्माषां भीरूणां कश्मलावहाम्}


\twolineshloka
{नागेन्द्रहययोधानां शरीरव्ययसम्भवाम्}
{शोणितौघमहाघोरां द्रौणिः प्रावर्तयन्नदीम्}


\twolineshloka
{योधार्तरवनिर्घोषां क्षतजोर्मिसमाकुलाम्}
{प्रयान्तीं सुमहाघोरां यमराष्ट्रमहोदधिम्}


\twolineshloka
{निहत्य राक्षसान्बाणैर्द्रौणिर्हैडिम्बिमार्दयत्}
{पुनरप्यतिसङ्क्रुद्धः सवृकोदरपार्षतान्}


\twolineshloka
{स नाराचगणैः पार्थान्द्रौणिर्विद्धा महाबलः}
{जघान सुरथं नाम द्रुपदस्य सुतं विभुः}


\twolineshloka
{पुनः शत्रुञ्जयं नाम द्रुपदस्यात्मजं रणे}
{बलानीकं जयानीकं जयाश्वं चाभिजघ्निवान्}


\twolineshloka
{श्रुताह्वयं च राजानं द्रौणिर्निन्ये यमक्षयम्}
{त्रिभिश्चान्यैः शरैस्तीक्ष्णैः सुपुङ्खैर्हेममालिनम्}


\twolineshloka
{जघान स पृषध्रं च चन्द्रसेनं च मारिष}
{कुन्तिभोजसुतांश्चासौ दशभिर्दश जघ्निवान्}


\threelineshloka
{अस्वत्थामा सुसङ्क्रुद्धः सन्धायोग्रमजिह्मगम्}
{मुमोचाकर्णपूर्णेन धनुषा शरमुत्तमम्}
{यमदण्डोपमं घोरमुद्दिश्याशु घटोत्कचम्}


\twolineshloka
{स भित्त्वा हृदयं तस्य राक्षसस्य महाशरः}
{विवेश वसुधां शीघ्रं सपुङ्खः पृथिवीपते}


\twolineshloka
{तं हतं पतितं ज्ञात्वा धृष्टद्युम्नो महारथः}
{द्रौणेः सकाशाद्राजेन्द्र व्यपनिन्ये रथोत्तमम्}


\threelineshloka
{ततः पराङ्मुखनृपं सैन्यं यौधिष्ठिरं नृप}
{पराजित्य रणे वीरो द्रोणपुत्रो ननाद ह}
{पूजितः सर्वभूतेषु तव पुत्रैश्च भारत}


\twolineshloka
{अथ शरशतभिन्नकृत्तदेहै--र्हतपतितैः क्षणदाचरैः समन्तात्}
{निधनमुपगतैर्मही कृताऽभू--द्गिरिशिखरैरिव दुर्गमाऽतिरौद्रा}


\twolineshloka
{तं सिद्धगन्धर्वपिशाचसङ्घानागाः सुपर्णाः पितरो वयांसि}
{रक्षोगणा भूतगणाश्च द्रौणि--मपूजयन्नप्सरसः सुराश्च}


\chapter{अध्यायः १५८}
\twolineshloka
{सञ्जय उवाच}
{}


\twolineshloka
{द्रुपदस्यात्मजान्दृष्ट्वा कुन्तिभोजसुतांस्तथा}
{द्रोणपुत्रेण निहतान्राक्षसांश्च सहस्रशः}


\twolineshloka
{युधिष्ठिरो भीमसेनो धृष्टद्युम्नश्च पार्षतः}
{युयुधानश्च संयत्ता युद्धायैव मनो दधुः}


\twolineshloka
{सोमदत्तः पुनः क्रुद्धो दृष्ट्वा सात्यकिमाहवे}
{महता शरवर्षेण च्छादयामास भारत}


\twolineshloka
{ततः समभवद्युद्धमतीव भयवर्धनम्}
{त्वदीयानां परेषां च घोरं विजयकाङ्क्षिणाम्}


\twolineshloka
{तं दृष्ट्वा समुपायान्तं रुक्मपुङ्खैः शिलाशितैः}
{दशभिः सात्वतस्यार्थे भीमो विव्याध सायकैः}


\twolineshloka
{सोमदत्तोऽपि तं वीरं शतेन प्रत्यविध्यत}
{सात्वतस्त्वभिसंक्रुद्धः पुत्राधिभिरभिप्लुतम्}


\threelineshloka
{वृद्धं वृद्धगुणैर्युक्तं ययातिमिव नाहुषम्}
{विव्याध दशभिस्तीक्ष्णैः शरैर्वज्रनिपातनैः}
{शक्त्या चैनं विनिर्भिद्य पुनर्विव्याध सप्तभिः}


\twolineshloka
{ततस्तु सात्यकेरर्थे भीमसेनो नवं दृढम्}
{मुमोच परिघं घोरं सोमदत्तस्य मूर्धनि}


\twolineshloka
{सात्वतोऽप्यग्निसङ्काशं मुमोच शरमुत्तमम्}
{सोमदत्तोरसि क्रुद्धः सुपत्रं निशितं युधि}


\twolineshloka
{युगपत्पेततुर्वीर घोरौ परिघमार्गणौ}
{शरीरे सोमदत्तस्य स पपात महारथः}


\twolineshloka
{व्यामोहिते तुनये बाह्लीकस्तमुपाद्रवत्}
{विसृजञ्छरवर्षाणि कालवर्षीव तोयदः}


\twolineshloka
{भीमोऽथ सात्वतस्यार्थे बाह्लीकं नवभिः शरैः}
{प्रातिपेयं महात्मानं विव्याध रणमूर्धनि}


\twolineshloka
{प्रातिपेयस्तु सङ्क्रुद्धः शक्तिं भीमस्य वक्षसि}
{निचखान महाबाहुः पुरन्दर इवाशनिम्}


\twolineshloka
{स तथाऽभिहतो भीमश्चकम्पे च मुमोह च}
{प्राप्य चेतश्च बलवान्गदामस्मै ससर्ज ह}


\twolineshloka
{सा पाण्डवेन प्रहिता बाह्लीकस्य शिरोऽहरत्}
{स पपात हतः पृथ्व्यां वज्राहत इवाद्रिराट्}


\twolineshloka
{तस्मिन्विनिहते वीरे बाह्लीके पुरुषर्षभ}
{पुत्रास्तेऽभ्यर्दयन्भीमं दश दाशरथेः समाः}


\twolineshloka
{नागदत्तो दृढरथो महाबाहुरयोभुजः}
{दृढः सुहस्तो विरजाः प्रमाथ्युग्रोऽनुयाय्यपि}


\twolineshloka
{तान्दृष्ट्वा चुक्रुधे भीमो जगृहे भारसाधनान्}
{एकमेकं समुद्दिश्य पातयामास मर्मसु}


\twolineshloka
{ते विद्धा व्यसवः पेतुः स्यन्दनेभ्यो हतौजसः}
{चण्डवातप्रभग्नास्तु पर्वताग्रान्महीरुहाः}


\twolineshloka
{नाराचैर्दशभिर्भीमस्तान्निहत्य तवात्मजान्}
{कर्णस्य दयितं पुत्रं वृषसेनमवाकिरत्}


\twolineshloka
{ततो वृकरथो नाम भ्राता कर्णस्य विश्रुतः}
{जघान भीमं नाराचैस्तमप्यभ्यद्रवद्बली}


\twolineshloka
{ततः सप्तरथान्वीरः श्यालानां तव भारत}
{निहत्य भीमो नाराचैः शतचन्द्रमपोथयत्}


\twolineshloka
{अमर्षयन्तो निहतं शतचन्द्रं महारथम्}
{शकुनेर्भ्रातरो वीरा गवाक्षः शरभो विभुः}


\twolineshloka
{सुभगो भानुदत्तश्च शूराः पञ्च महारथाः}
{अभिद्रुत्य शरैस्तीक्ष्णैर्भीमसेनमताडयन्}


\twolineshloka
{स ताड्यमानो नाराचैर्वृष्टिवेगैरिवाचलः}
{जघान पञ्चभिर्बाणैः पञ्चैवातिरथान्बली}


% Check verse!
तान्दृष्ट्वा निहतान्वीरान्विचेलुर्नृपसत्तमाः
% Check verse!
ततो युधिष्ठिरः क्रुद्धस्तवानीकमशातयत् ॥मिषतः कुम्भयोनेस्तु पुत्राणां तव चानघ
\twolineshloka
{अम्बष्ठान्मालवाञ्छूरांस्त्रिगर्तान्सशिबीनपि}
{प्राहिणोन्मृत्युलोकाय क्रुद्धो युद्धे युधिष्ठिरः}


\twolineshloka
{अभीषाहाञ्छूरसेनान्बाह्लीकान्सवसातिकान्}
{निहत्य पृथिवीं राजा चक्रे शोणितकर्दमाम्}


\twolineshloka
{यौधेयान्मालवान्राजन्मद्रकाणां गणान्युधि}
{प्राहिणोन्मृत्युलोकाय शूरान्बाणैर्युधिष्ठिरः}


\twolineshloka
{हताहरत गृह्णीत विध्यत व्यवकृन्तत}
{इत्यासीत्तुमुलः शब्दो युधिष्ठिररथं प्रति}


\twolineshloka
{सैन्यानि द्रावयन्तं तं द्रोणो दृष्ट्वा युधिष्ठिरम्}
{चोदितस्तव पुत्रेण सायकैरभ्यवाकिरत्}


\twolineshloka
{द्रोणस्तु परमक्रुद्धो वायव्यास्त्रेण पार्थिवम्}
{विव्याध सोऽपि तद्दिव्यमस्त्रमस्त्रेण जघ्निवान्}


\threelineshloka
{तस्मिन्विनिहते चास्त्रे भारद्वाजो युधिष्ठिरे}
{वारुणं याम्यमाग्नेयं त्वाष्ट्रं सावित्रमेव च}
{चिक्षेप परमक्रुद्धो जिघांसुः पाण्डुनन्दनम्}


\twolineshloka
{क्षिप्तानि क्षिप्यमाणानि तानि चास्त्राणि धर्मजः}
{जघानास्त्रैर्महाबाहुः कुम्भयोनेरवित्रसन्}


% Check verse!
सत्यां चिकीर्षमाणस्तु प्रतिज्ञां कुम्भसम्भवः
\twolineshloka
{प्रादुश्चक्रेऽस्त्रमैन्द्रं वै प्राजापत्यं च भारत}
{जिघांसुर्धर्मतनयं तव पुत्रहिते रतः}


\twolineshloka
{पतिः कुरूणां गजसिंहगामीविशालवक्षाः पृथुलोहिताक्षः}
{प्रादुश्चकारास्त्रमहीनतेजामाहेन्द्रमस्त्रं स जघान तेन}


\twolineshloka
{विहन्यमानेष्वस्त्रेषु द्रोणः क्रोधसमन्वितः}
{युधिष्ठिरवधं प्रेप्सुर्ब्राह्ममस्त्रमुदैरयत्}


\twolineshloka
{ततो नाज्ञासिषं किञ्चिद्धोरेण तमसाऽऽवृते}
{सर्वभूतानि च परं त्रासं जग्मुर्महीपते}


\twolineshloka
{ब्रह्मास्त्रमुद्यतं दृष्ट्वा कुन्तीपुत्रो युधिष्ठिरः}
{ब्रह्मास्त्रेणैव राजेन्द्र तदस्त्रं प्रत्यवारयत्}


\twolineshloka
{ततः सैनिकमुख्यास्ते प्रशशंसुर्नरर्षभौ}
{द्रोणपार्थौ महेष्वासौ सर्वयुद्धविशारदौ}


\twolineshloka
{ततः प्रमुच्य कौन्तेयं द्रोणो द्रुपदवाहिनीम्}
{व्यधमत्क्रोधताम्राक्षो वायव्यास्त्रेण भारत}


\twolineshloka
{ते हन्यमाना द्रोणेन पाञ्चालाः प्राद्रवन्भयात्}
{पश्यतो भीमसेनस्य पार्थस्य च महात्मनः}


\twolineshloka
{ततः किरीटी भीमश्च सहसा सन्न्यवर्तताम्}
{महद्ध्यां रथवंशाभ्यां परिगृह्य बलं तदा}


\twolineshloka
{बीभत्सुर्दक्षिणं पार्श्वमुत्तरं च वृकोदरः}
{भारद्वाजं शरौघाभ्यां महद्भ्यामभ्यवर्षताम्}


\twolineshloka
{केकयाः सृञ्जयाश्चैव पाञ्चालाश्च महौजसः}
{अन्वगच्छन्महाराज मात्स्याश्च सह सात्वतैः}


\twolineshloka
{ततः सा भारती सेना वध्यमाना किरीटिना}
{तमसा निद्रया चैव पुनरेव व्यदीर्यत}


\twolineshloka
{द्रोणेन वार्यमाणास्ते स्वयं तव सुतेन च}
{नाशक्यन्त महाराज योधा वारयितुं तदा}


\chapter{अध्यायः १५९}
\twolineshloka
{सञ्जय उवाच}
{}


\twolineshloka
{उदीर्यमाणं तद्दृष्ट्वा पाण्डवानां महद्बलम्}
{अविषह्यं च मन्वानः कर्णं दुर्योधनोऽब्रवीत्}


\twolineshloka
{अयं स कालः सम्प्राप्तो मित्राणां मित्रवत्सल}
{त्रायस्व समरे कर्ण सर्वान्योधान्महारथान्}


\twolineshloka
{पाञ्चालैर्मत्स्यकैकेयैः पाण्डवैश्च महारथैः}
{वृतान्समन्तात्सह्क्रुद्धैर्निः श्वसद्भिरिवोरगैः}


\threelineshloka
{एते नदन्ति संहृष्टाः पाण्डवा जितकाशिनः}
{शक्रोपमाश्च बहवः पाञ्चालानां रथव्रजाः ॥कर्ण उवाच}
{}


\twolineshloka
{परित्रातुमिह प्राप्तो यदि पार्थं पुरन्दरः}
{तमप्याशु पराजित्य ततो हन्ताऽस्मि पाण्डवम्}


\twolineshloka
{सत्यं ते प्रतिजानामि समाश्वसिहि भारत}
{हन्तास्मि पाण्डुतनयान्पाञ्चालांश्च समागतान्}


\twolineshloka
{जयं ते प्रतिदास्यामि वासवस्येव पावकिः}
{प्रियं तव मया कार्यमिति जीवामि पार्थिव}


\twolineshloka
{सर्वेषामेव पार्थानां फल्गुनो बलवत्तरः}
{तस्यामोघां विमोक्ष्यामिशक्तिं शक्रविनिर्मितां}


\twolineshloka
{तस्मिन्हते महेष्वासे भ्रातरस्तस्य मानद}
{तव वश्या भविष्यन्ति वनं यास्यन्ति वा पुनः}


\twolineshloka
{मयि जीवति कौरव्य विषादं मा कृथाः क्वचित्}
{अहं जेष्यामि समरे सहितान्सर्वपाण्डवान्}


\threelineshloka
{पाञ्चालान्केकयांश्चैव वृष्णींश्चापि समागतान्}
{बाणौघैः शकलीकृत्य तव दास्यामि मेदिनीम् ॥सञ्जय उवाच}
{}


\twolineshloka
{एवं ब्रुवाणं कर्णं तु कृपः शारद्वतोऽब्रवीत्}
{स्मयन्निव महाबाहुः सूतपुत्रमिदं वचः}


\twolineshloka
{शोभनं शोभनं कर्ण सनाथः कुरुपुङ्गवः}
{त्वया नाथेन राधेय वचसा यदि सिध्यति}


\twolineshloka
{बहुशः कत्थसे कर्ण कौरवस्य समीपतः}
{न तु ते विक्रमः कश्चिद्दृश्यते फलमेव वा}


\twolineshloka
{समागमः पाण्डुसुतैर्दृष्टस्ते बहुशो युधि}
{सर्वत्र निर्जितश्चासि पाण्डवैः सूतनन्दन}


\twolineshloka
{हियमाणे तदा कर्ण गन्धर्वैर्धृतराष्ट्रजे}
{तदाऽयुध्यन्त सैन्यानि त्वमेकोऽग्रेऽपलायिथाः}


\twolineshloka
{विराटनगरे चापि समेताः सर्वकौरवाः}
{पार्थेन निर्जिता युद्धे त्वं च कर्ण सहानुजः}


\twolineshloka
{एकस्याप्यसमर्थस्त्वं फल्गुनस्य रणाजिरे}
{कथमुत्सहसे जेतुं सकृष्णान्सर्वपाण्डवान्}


\twolineshloka
{अब्रुवन्कर्ण युध्यस्व कत्थसे बहु सूतज}
{अनुक्त्वा विक्रमेद्यस्तु तद्वै सत्पुरुषव्रतम्}


\twolineshloka
{गर्जित्वा सूतपुत्र त्वं शारदाभ्र इवाजलः}
{निष्फलो दृश्यसे कर्ण तच्च राजा न बुध्यते}


\twolineshloka
{तावद्गर्जस्व राधेय यावत्पार्थं न पश्यसि}
{आरात्पार्थं हि ते दृष्ट्वा दुर्लभं गर्जितं पुनः}


\twolineshloka
{त्वमनासाद्य तान्बाणान्फल्गुनस्य विगर्जसि}
{पार्थसायकविंद्धस्य दुर्लभं गर्जितं तव}


\threelineshloka
{बाहुभिः क्षत्रियाः शूरा वाग्भिः शूरा द्विजातयः}
{धनुषा फल्गुनः शूरः कर्णः शूरो मनोरथैः}
{तोषितो येन रुद्रोऽपि कः पार्थं प्रतिघातयेत्}


\twolineshloka
{एवं संरुषितस्तेन तदा शारद्वतेन ह}
{कर्णः प्रहरतां श्रेष्ठः कृपं वाक्यमथाब्रवीत्}


\twolineshloka
{शूरा गर्जन्ति सततं प्रावृषीव वलाहकाः}
{फलं चाशु प्रयच्छन्ति बीजमुप्तमनूषरे}


\twolineshloka
{दोषमत्र न पश्यामि शूराणां रणमूर्धनि}
{तत्तद्विकत्थमानानां भारं चोद्वहतां मृधे}


\twolineshloka
{यं भारं पुरुषो वोढुं मनसा हि व्यवस्यति}
{दैवमस्य ध्रुवं तत्र साहाय्यायोपपद्यते}


\threelineshloka
{व्यवसायद्वितीयोऽहं मनसा भारमुद्वहन्}
{हत्वा पाण्डुसुतानाजौ सकृष्णान्सहसात्वतान्}
{गर्जामि यद्यहं विप्र तव किं तत्र नश्यति}


\twolineshloka
{वृथा शूरा न गर्जन्ति सजला इव तोयदाः}
{सामर्थ्यमात्मनो ज्ञात्वा ततो गर्जन्ति पण्डिताः}


\twolineshloka
{सोऽहमद्य रणे यत्तौ सहितौ कृष्मपाण्डवौ}
{उत्सहामि रणे जेतुं ततो गर्जामि गौतम}


\fourlineindentedshloka
{पश्य त्वं गर्जितस्यास्य फलं मे विप्र सानुगान्}
{हत्वा पाण्डुसुतानाजौ सहकृष्णान्ससात्वतान्}
{दुर्योधनाय दास्यामि पृथिवीं हतकण्टकाम् ॥कृप उवाच}
{}


% Check verse!
मनोरथप्रलापा मे न ग्राह्यास्तव सूतज
\twolineshloka
{सदा क्षिपसि वै कृष्णौ धर्मराजं च पाण्डवम्}
{ध्रुवस्तत्र जयः कर्ण यत्र युद्धविशारदौ}


\twolineshloka
{देवगन्धर्वयक्षाणां मनुष्योरगरक्षसाम्}
{दंशितानामपि रणे अजेयौ कृष्णपाण्डवौ}


\twolineshloka
{ब्रह्मण्यः सत्यवाग्दान्तो गुरुदैवतपूजकः}
{नित्यं धर्मरतश्चैव कृतास्त्रश्च विशेषतः}


\twolineshloka
{धृतिमांश्च कृतज्ञश्च धर्मपुत्रो युधिष्ठिरः}
{भ्रातरश्चास्य बलिनः सर्वास्त्रेषु कृतश्रमाः}


\twolineshloka
{गुरुवृत्तिरताः प्राज्ञा धर्मनित्या यशस्विनः}
{सम्बन्धिनश्चेन्द्रवीर्याः स्वनुर्ताः प्रहारिणः}


\twolineshloka
{धृष्टद्युम्नः शिखण्डी च दौर्मुखिर्जनमेजयः}
{चन्द्रसेनो रुद्रसेनः कीर्तिधर्मा ध्रुवो धरः}


\twolineshloka
{वसुचन्द्रो दामचन्द्रः सिंहचन्द्रः सुतेजनः}
{द्रुपदस्य तथा पुत्रा द्रुपदश्च महास्त्रवित्}


\twolineshloka
{येषामर्थाय संयत्तो मत्स्यराजः सहानुजः}
{शतानीकः सूर्यदत्तः श्रुतानीकः श्रुतध्वजः}


\twolineshloka
{बलानीको जयानीको जयाश्वो रथवाहनः}
{चन्द्रोदयः समरथो विराटभ्रातरः शुभाः}


\twolineshloka
{यमौ च द्रौपदेयाश्च राक्षसश्च घटोत्कचः}
{येषामर्थाय युध्यन्ते न तेषां विद्यते क्षयः}


% Check verse!
एते चान्ये च बहवो गणाः पाण्डुसुतस्य वै
\threelineshloka
{कामं खलु जगत्सर्वं सदेवासुरमानुषम्}
{सयक्षराक्षसगणं सभूतभुजगद्विपम्}
{निःशेषमस्त्रवीर्येण कुर्वाते भीमफल्गुनौ}


% Check verse!
युधिष्ठिरश्च पृथिवीं निर्दहेद्धोरचक्षुषा
\twolineshloka
{अप्रमेयबलः शौरिर्येषामर्थे च दंशितः}
{कथं तान्संयुगे कर्ण जेतुमुत्सहसे परान्}


\threelineshloka
{महानपनयस्त्वेष नित्यं हि तव सूतज}
{यस्त्वमुत्सहसे योद्धुं समरे शौरिणा सह ॥सञ्जय उवाच}
{}


\twolineshloka
{एवमुक्तस्तु राधेयः प्रहसन्भरतर्षभ}
{अब्रवीच्च तदा कर्णो गुरं शारद्वतं कृपम्}


\twolineshloka
{सत्यमुक्तं त्वया ब्रह्मन्पाण्डवान्प्रति यद्वचः}
{एते चान्ये च बहवो गुणाः पाण्डुसुतेषु वै}


\twolineshloka
{अजय्याश्च रणे पार्था देवैरपि सवासवैः}
{सदैत्ययक्षगन्धर्वैः पिशाचोरगराक्षसैः}


\threelineshloka
{तथापि पार्थाञ्जेष्यामि शक्त्या वासवदत्तया}
{मम ह्यमोघा दत्तेयं शक्तिः शक्रेण वै द्विज}
{एतया निहनिष्यामि सव्यसाचिनमाहवे}


\twolineshloka
{हते तु पाण्डवे कृष्णे भ्रातरश्चास्य सोदराः}
{अनर्जुना न शक्ष्यन्ति महीं भोक्तुं कथञ्चन}


\twolineshloka
{तेषु नष्टेषु सर्वेषु पृथिवीयं ससागरा}
{अयत्नात्कौरवेन्द्रस्य वशे स्थास्यति गौतम्}


\twolineshloka
{सुनीतैरिह सर्वार्थाः सिध्यन्ते नात्र संशयः}
{एतमर्थमहं ज्ञात्वा ततो गर्जामि गौतम}


\twolineshloka
{त्वं तु विप्रश्च वृद्धश्च अशक्तश्चापि संयुगे}
{कृतस्नेहश्च पार्थेषु मोहान्मामवमन्यसे}


\twolineshloka
{यद्येवं वक्ष्यसे भूयो ममाप्रियमिह द्विज}
{ततस्ते खङ्गमुद्यम्य जिह्वां छेत्स्यामि दुर्मते}


\twolineshloka
{यच्चापि पाण्डवान्विप्र स्तोतुमिच्छसि संयुगे}
{भीषयन्सर्वसैन्यानि कौरवेयांश्च दुर्मते}


\twolineshloka
{अत्रापि शृणु मे वाक्यं यथावद्ब्रुवतो द्विज}
{दुर्योधनश्च द्रोणश्च शकुनिर्दुर्मुखो जयः}


\twolineshloka
{दुःशासनो वृषसेनो मद्राजस्त्वमेव च}
{सोमदत्तश्च भूरिश्च तथा द्रौणिर्विविंशतिः}


\twolineshloka
{तिष्ठेयुर्दंशिता यत्रा सर्वे युद्धविशारदाः}
{जयेदेतान्नरः को नु शक्रतुल्यबलोऽप्यरिः}


\twolineshloka
{शूराश्च हि कृतास्त्राश्च बलिनः स्वर्गलिप्सवः}
{धर्मज्ञा युद्धकुशला हन्युर्युद्धे सुरानपि}


\twolineshloka
{एते स्थास्यन्ति सङ्ग्रामे पाण्डवानां वधार्थिनः}
{जयमाकाङ्क्षमाणा हि कौरवेयस्य दंशिताः}


\twolineshloka
{दैवायत्तमहं मन्ये जयं सुबलिनामपि}
{यत्र भीष्मो महाबाहुः शेते शरशताचितः}


\twolineshloka
{विकर्णश्चित्रसेनश्च बाह्लीकोऽथ जयद्रथः}
{भूरिश्रवा जयश्चैव जलसन्धः सुदक्षिणः}


\twolineshloka
{शलश्च रथिनां श्रेष्ठो भगदत्तश्च वीर्यवान्}
{एते चान्ये च राजानो देवैरपि सुदुर्जयाः}


\twolineshloka
{निहताः समरे शूराः पाण्डवैर्बलवत्तराः}
{किमन्यद्दैवसंयोगान्मन्यसे पुरुषाधम}


\twolineshloka
{यांश्च तांस्तौषि सततं दुर्योधनरिपून्द्विज}
{तेषामपि हताः शूराः शतशोऽथ सहस्रशः}


\twolineshloka
{क्षीयन्ते सर्वसैन्यानि कुरूणां पाण्डवैः सह}
{प्रभावं नात्र पाश्यामि पाण्डवानां कथञ्चन}


\threelineshloka
{यस्तान्बलवतो नित्यं मन्यसे त्वं द्विजाधम}
{यतिष्येऽहं यथाशक्ति योद्धुं तैः सह संयुगे}
{दुर्योधनहितार्थाय जयो दैवे प्रतिष्ठितः}


\chapter{अध्यायः १६०}
\twolineshloka
{सञ्जय उवाच}
{}


\twolineshloka
{तथा परुषितं दृष्ट्वा सूतपुत्रेण मातुलम्}
{खङ्गमुद्यम्य वेगेन द्रौणिरभ्यपतद्द्रुतम्}


\threelineshloka
{ततः परमसङ्क्रुद्धः सिंहो मत्तमिव द्विपम्}
{प्रेक्षतः कुरुराजस्य द्रौणिः कर्णं समभ्ययात् ॥अश्वत्थामोवाच}
{}


\twolineshloka
{यदर्युनगुणांस्तथ्यान्कीर्तयानं नराधम}
{शूरं द्वेषात्सुदुर्बुद्धे त्वं भर्त्सयसि मातुलम्}


% Check verse!
विकत्थमानः शौर्येण सर्वलोकधनुर्धरम् ॥दर्पोत्सेधगृहीतोऽद्य न कञ्चिद्गणयन्मृधे
\twolineshloka
{क्व ते वीर्यं क्व चास्त्राणि यं त्वां निर्जित्य संयुगे}
{गाण्डीवधन्वा हतवान्प्रेक्षतस्ते जयद्रथम्}


\twolineshloka
{येन साक्षान्महादेवो योधितः समरे पुरा}
{तमिच्छसि वृथा जेतुं सूताधम मनोरथैः}


\twolineshloka
{यं हि कृष्णेन सहितं सर्वशस्त्रभृतां वरम्}
{जेतुं न शक्ताः सहिताः सेन्द्रा अपि सुरासुराः}


\twolineshloka
{लोकैकवीरमजितमर्जुनं सूत संयुगे}
{किं पुनस्त्वं सुदुर्बुद्धे सहैभिर्वसुधाधिपैः}


\threelineshloka
{कर्ण पश्य सुदुर्बुद्धे तिष्ठेदानीं नराधम}
{एष तेऽद्य शिरः कायादुद्धरामि सुदुर्मते ॥सञ्जय उवाच}
{}


\threelineshloka
{तमुद्यतं तु वेगेन राजा दुर्योधनः स्वयम्}
{न्यवारयन्महातेजाः कृपश्च द्विपदां वरः ॥कर्ण उवाच}
{}


\threelineshloka
{शूरोऽयं समरश्लाघी दुर्मतिश्च द्विजाधमः}
{आसादयतु मद्वीर्यं मुञ्चेमं कुरुसत्तम ॥अश्वत्थामोवाच}
{}


\threelineshloka
{तैवतत्क्षम्यतेऽस्माभिः सूतात्मज सुदुर्मते}
{दर्पमुत्सिक्तमेतत्ते फल्गुनो नाशयिष्यति ॥दुर्योधन उवाच}
{}


\twolineshloka
{अश्वत्थामन्प्रसीदस्व क्षन्तुमर्हसि मानद}
{कोपः खलु न कर्तव्यः सूतपुत्रं कथञ्चन}


\twolineshloka
{त्वयि कर्णे कृपे द्रोणे मद्रराजेऽथ सौबले}
{महत्कार्यं समासक्तं प्रसीद द्विजसत्तम}


\threelineshloka
{एते ह्यभिमुखाः सर्वे राधेयेन युयुत्सवः}
{आयान्ति पाण्डवा ब्रह्मन्नाह्वयन्तः समन्ततः ॥सञ्जय उवाच}
{}


\twolineshloka
{प्रसाद्यमानस्तु ततो राज्ञा द्रौणिर्महामनाः}
{प्रससाद महाराज क्रोधवेगसमन्वितः}


\threelineshloka
{ततः कृपोऽप्युवाचेदमाचार्यः सुमहामनाः}
{सौम्यस्वभावाद्राजनेन्द्र क्षिप्रमागतमार्दवः ॥कृप उवाच}
{}


\threelineshloka
{तवैतत्क्षम्यतेऽस्माभिः सूतात्मज सुदुर्मते}
{दर्पमुत्सिक्तमेतत्ते फल्गुनो नाशयिष्यति ॥सञ्जय उवाच}
{}


\twolineshloka
{ततस्ते पाण्डवा राजन्पाञ्चालाश्च यशस्विनः}
{आजग्मुः सहिताः कर्णं तर्जयन्तः समन्ततः}


\threelineshloka
{कर्णोऽपि रथिनां श्रेष्ठश्चापमुद्यम्य वीर्यवान्}
{कौरवाग्र्यैः परिवृतः शक्रो देवगणैरिव}
{पर्यतिष्ठत तेजस्वी स्वबाहुबलमाश्रितः}


\twolineshloka
{ततः प्रववृते युद्धं कर्णस्य सह पाण्डवैः}
{भीषणं सुमहाराज सिंहनादविराजितम्}


\twolineshloka
{ततस्ते पाण्डवा राजन्पाञ्चालाश्च यशस्विनः}
{दृष्ट्वा कर्णं महाबाहुमुच्चैः शब्दमथानदन्}


\twolineshloka
{अयं कर्णः कुतः कर्णस्तिष्ठ कर्ण महारणे}
{युध्यस्व सहितोऽस्माभिर्दुरात्मन्पुरुषाधम}


\fourlineindentedshloka
{अन्ये तु दृष्ट्वा राधेयं क्रोधरक्रेक्षणाऽब्रुवन्}
{हन्यतामयमुत्सिक्तः सूतपुत्रोऽल्पचेतनः}
{सर्वैः पार्थिवशार्दूलैर्नानेनार्थोऽस्ति जीवता}
{}


\twolineshloka
{अत्यन्तवैरी पार्थानां सततं पापपूरुषः}
{एष मूलमनर्थानां दुर्योधनमते स्थितः}


\threelineshloka
{घ्नतैनमिति जल्पन्तः क्षत्रियाः समुपाद्रवन्}
{महता शरवर्षेण च्छादयन्तो महारथाः}
{वधार्थं सूतपुत्रस्य पाण्डवेयेन चोदिताः}


\twolineshloka
{तांस्तु सर्वांस्तथा दृष्ट्वा धावमानान्महारथान्}
{न विव्यथे सूतपुत्रो न च त्रासमगच्छत}


\twolineshloka
{दृष्ट्वा सागस्कल्पं तमुद्धूतं सैन्यसागरम्}
{पिप्रीषुस्तव पुत्राणां सङ्ग्रामेष्वपराजितः}


\twolineshloka
{सायकौधेन बलवान्क्षिप्रकारी महाबलः}
{वारयामास तत्सैन्यं समन्ताद्भरतर्पभ}


\threelineshloka
{ततस्तु शरवर्षेण पार्थिवास्तमवारयन्}
{धनुम्पि ते विधुन्वानाः शतशोऽथ सहस्रशः}
{अयोधयन्त राधेयं शक्रं दैत्यगणा इव}


\twolineshloka
{शरवर्पं तु तत्कर्णः पार्थिवैः समुदीरितम्}
{शरवर्षेण महता समन्ताद्व्यकिरत्प्रभो}


\twolineshloka
{तद्युद्धमभवत्तेषां कृतप्रतिकृतैषिणाम्}
{यथा देवासुरे युद्धे शक्रस्य सह दानैवः}


\twolineshloka
{तत्राद्भुतमपश्याम सूतपुत्रस्य लाघवम्}
{यदेनं सर्वतो यत्ता नाप्नुवन्ति परे युधि}


\threelineshloka
{निवार्य च शरौघांस्तान्पार्थिवानां महारथः}
{युगेष्वीषासु च्छत्रेषु ध्वजेषु च हयेषु च}
{आत्मनामाङ्कितान्घोरान्राधेयः प्राहिणोच्छरान्}


\twolineshloka
{ततस्ते व्याकुलीभूता राजानः कर्णपीडिताः}
{बभ्रुमुस्तत्र तत्रैव गावः शीतार्दिता इव}


\twolineshloka
{हयानां वध्यमानानां गजानां रथिनां तथा}
{तत्रतत्राभ्यवेक्षाम सङ्घान्कर्णेन ताडितान्}


\twolineshloka
{शिरोभिः पतितै राजन्बाहुभिश्च समन्ततः}
{आस्तीर्णा वसुधा सर्वा शूराणामनिवर्तिनाम्}


\twolineshloka
{हतैश्च हन्यमानैश्च निष्टनद्भिश्च सर्वशः}
{बभूवायोधनं रौद्रं वैवस्वतपुरोपमम्}


\twolineshloka
{ततो दुर्योधनो राजा दृष्ट्वा कर्णस्य विक्रमम्}
{अश्वत्थामानमासाद्य वाक्यमेतदुवाच ह}


% Check verse!
युध्यतेऽसौ रणे कर्णो दंशितः सर्वपार्थिवैः
\twolineshloka
{पश्यैतां द्रवतीं सेनां कर्णसायकपीडिताम्}
{कार्तिकेयेन विध्वस्तामासुरीं पृतनामिव}


\twolineshloka
{दृष्ट्वैतां निर्जितां सेनां रणे कर्णेन धीमता}
{अभियात्येष बीभत्सुः सूतपुत्रजिघांसया}


\twolineshloka
{तद्यथा प्रेक्षमाणानां सूतपुत्रं महारथम्}
{न हन्यात्पाण्डवः सङ्ख्ये तथा नीतिर्विधीयतां}


\threelineshloka
{ततो द्रौणिः कृपः शल्यो हार्दिक्यश्च महारथः}
{प्रत्युद्ययुस्तदा पार्थं सूतपुत्रपरीप्सया}
{आयान्तं वीक्ष्य कौन्तेयं शक्रं दैत्यचमूमिव}


\threelineshloka
{बीभत्सुरपि राजेन्द्र पाञ्चालैरभिसंवृतः}
{प्रत्युद्ययौ तदा कर्णं यथा वृत्रं शतक्रतुः ॥धृतराष्ट्र उवाच}
{}


\twolineshloka
{संरब्धं फल्गुनं दृष्ट्वा कालान्तकयमोपमम्}
{कर्णो वैकर्तनः सूत प्रत्यपद्यत्किमुत्तरम्}


\twolineshloka
{यो ह्यस्पर्धत पार्थेन नित्यमेव महारथः}
{आशंसते च बीभत्सुं युद्धे जेतुं सुदारुणम्}


\threelineshloka
{स तु तं सहसा प्राप्तं नित्यमत्यन्तवैरिणम्}
{कर्णो वैकर्तनः सूत किमुत्तरमपद्यत ॥सञ्जय उवाच}
{}


\twolineshloka
{आयान्तं पाण्डव दृष्ट्वा गजं प्रतिगजो यथा}
{असम्भ्रान्तो रणे कर्णः प्रत्युदीयाद्धनञ्जयम्}


\twolineshloka
{तमापतन्तं वेगेन वैकर्तनमजिह्मगैः}
{छादयामास पार्थोऽथ कर्णस्तु विजयं शरैः}


\twolineshloka
{स कर्णं शरजालेन च्छादयामास पाण़्डवः}
{ततः कर्णः सुसंरब्धः शरैस्त्रिभिरविध्यत}


% Check verse!
तस्य तल्लाघवं पार्थो नामृष्यत महाबलः
\twolineshloka
{तस्मै बाणाञ्शिलाधौतान्प्रसन्नाग्रानजिह्मगान्}
{प्राहिणोत्सूतपुत्राय त्रिशतं शत्रुतापनः}


\twolineshloka
{विव्याध चैनं संरब्धो बाणेनैकेन वीर्यवान्}
{सव्ये भुजाग्रे बलवान्नाराचेन हसन्निव}


% Check verse!
तस्य विद्धस्य बाणेन कराच्चापं पपात ह
\twolineshloka
{पुनरादाय तच्चापं निमेषार्धान्महाबलः}
{छादयामास बाणौघैः फल्गुनं कृतहस्तवत्}


\twolineshloka
{शरवृष्टिं तु तां मुक्तां सूतपुत्रेण भारत}
{व्यधमच्छरवर्षेण स्मयन्निव धनञ्जयः}


\twolineshloka
{तौ परस्परमासाद्य शरवर्षेण पार्थिव}
{छादयेतां महेष्वासौ कृतप्रतिकृतैषिणौ}


\twolineshloka
{तदद्भुतं महद्युद्धं कर्णपाण्डवयोर्मृधे}
{क्रुद्धयोर्वासिताहेतोर्वन्ययोर्गजयोरिव}


\twolineshloka
{ततः पार्थो महेष्वासो दृष्ट्वा कर्णस्य विक्रमम्}
{मुष्टिदेशे धनुस्तस्य चिच्छेद त्वरयाऽन्वितः}


\twolineshloka
{अश्वांश्च चतुरो भल्लैरनयद्यमसादनम्}
{सारथेश्च शिसः कायादहरच्छत्रुतापनः}


\twolineshloka
{अथैनं छिन्नधन्वानं हताश्वं हतसारथिम्}
{विव्याध सायकैः पार्थश्चतुर्भिः पाण्डुनन्दनः}


\twolineshloka
{हताश्वात्तु रथात्तूर्णमवप्लुत्य नरर्षभः}
{आरुरोह रथं तूर्णं कृपस्य शरपीडितः}


\twolineshloka
{स नुन्नोऽर्जुनबाणौघैराचितः शल्यको यथा}
{जीवितार्थमभिप्रेप्सुः कृपस्य रथमारुहत्}


\twolineshloka
{राधेयं निर्जितं दृष्ट्वा तावका भरतर्षभ}
{धनञ्जयशरैर्नुन्नाः प्राद्रवन्त दिशो दश}


\twolineshloka
{द्रवतस्तान्समालोक्य राजा दुर्योधनो नृप}
{निवर्तयामास तदा वाक्यमेतदुवाच ह}


\threelineshloka
{अलं द्रुतेन वः शूरास्तिष्ठध्वं क्षत्रियर्षभाः}
{एष पार्थवधायाहं स्वयं गच्छामि संयुगे}
{अहं पार्थान्हनिष्यामि सपाञ्चालान्ससोमकान्}


\twolineshloka
{अद्य मे युध्यमानस्य सह गाण्डीवधन्वना}
{द्रक्ष्यन्ति विक्रमं पार्थाः कालस्येव युगक्षये}


\twolineshloka
{अद्य मद्बाणजालानि विमुक्तानि सहस्रशः}
{द्रक्ष्यन्ति समरे योधाः शलभानामिवायतीः}


\twolineshloka
{अद्य बाणमयं वर्षं सृजतो मम धन्विनः}
{जीमूतस्येव घर्मान्ते द्रक्ष्यन्ति युधि सैनिकाः}


\twolineshloka
{जेष्याम्यद्य रणे पार्थं सायकैर्नतपर्वभिः}
{तिष्ठध्वं समरे शूरा भयं त्यजत फल्गुनात्}


\twolineshloka
{न हि मद्वीर्यमासाद्य फल्गुनः प्रसहिष्यति}
{यथा वेलां समासाद्य सागरो मकरालयः}


\twolineshloka
{इत्युक्त्वा प्रययौ राजा सैन्येन महता वृतः}
{फल्गुनं प्रतिदुर्धर्षः क्रोधात्संरक्तलोचनः}


\twolineshloka
{तं प्रयान्तं महाबाहुं दृष्ट्वा शारद्वतस्तदा}
{अश्वत्थामानमासाद्य वाक्यमेतदुवाच ह}


\twolineshloka
{एष राजा महाबाहुरमर्षी क्रोधमूर्च्छितः}
{पतङ्गवृत्तिमास्थाय फल्गुनं योद्धुमिच्छति}


\twolineshloka
{यावन्नः पश्यमानानां प्राणान्पार्थेन संगतः}
{न जह्यात्पुरुषव्याघ्रस्तावद्वारय कौरवम्}


\twolineshloka
{यावत्फल्गुनबाणानां गोचरं नाद्य गच्छति}
{कौरवः पार्थिवो वीरस्तावद्वारय संयुगे}


\twolineshloka
{यावत्पार्थशरैर्घोरैर्निर्मुक्तोरगसन्निभैः}
{न भस्मीक्रियते राजा तावद्युद्धान्निवार्यताम्}


\twolineshloka
{अयुक्तमिव पश्यामि तिष्ठत्स्वस्मासु मानद}
{स्वयं युद्धाय यद्राजा पार्थं यात्यसहायवान्}


\twolineshloka
{दुर्लभं जीवितं मन्ये कौरव्यस्य किरीटिना}
{युध्यमानस्य पार्थेन शार्दूलेनेव हस्तिनः}


\twolineshloka
{मातुलेनैवमुक्तस्तु द्रौणिः शस्त्रभृतां वरः}
{दुर्योधनमिदं वाक्यं त्वरितः समभाषत}


\twolineshloka
{मयि जीवति गान्धारे न युद्धं गन्तुमर्हसि}
{मामनादृत्य कौरव्य तव नित्यं हितैषिणम्}


\threelineshloka
{न हि सम्भ्रमः कार्यः पार्थस्य विजयं प्रति}
{अहमावारयिष्यामि पार्थं तिष्ठ सुयोधन ॥दुर्योधन उवाच}
{}


\twolineshloka
{आचार्यः पाण्डुपुत्रान्वै पुत्रवत्परिरक्षति}
{त्वमप्युपेक्षां कुरुषे तेषु नित्यं द्विजोत्तम्}


\twolineshloka
{मम वा मन्दभाग्यत्वान्मन्दस्ते विक्रमो युधि}
{धर्मराजप्नियार्थं वा द्रौपद्या वा न विद्म तत्}


\twolineshloka
{धिगस्तु मम लुब्धस्य यत्कृते सर्वबान्धवाः}
{सुखार्हाः परमं दुःखं प्राप्नुवन्त्यपराजिताः}


\twolineshloka
{को हि शस्त्रविदां मुख्यो महेश्वरसमो युधि}
{शत्रुं न क्षपयेच्छक्तो यो न स्याद्गौतमीसुतः}


\twolineshloka
{अश्वत्थामन्प्रसीदस्व नाशयैतान्ममाहितान्}
{तवास्त्रगोचरे शक्ताः स्थातुं देवा न दानवाः}


\twolineshloka
{पाञ्चालान्सोमकांश्चैव जहि द्रौणे सहानुगान्}
{वयं शेषान्हनिष्यामस्त्वयैव परिरक्षिताः}


\twolineshloka
{एते हि सोमका विप्र पाञ्चालाश्च यशस्विनः}
{मम सैन्येषु सङ्क्रुद्धा विचरन्ति दवाग्निवत्}


\twolineshloka
{तान्वारय महाबाहो केकयांश्च नरोत्तम}
{पुरा कुर्वन्ति निःशेषं रक्ष्यमाणाः किरीटिना}


\twolineshloka
{अश्वत्थामंस्त्वरायुक्तो याहि शीघ्रमरिन्दम}
{आदौ वा यदि वा पश्चात्तवेदं कर्म मारिष}


\twolineshloka
{त्वमुत्पन्नो महाबाहो पाञ्चालानां वधं प्रति}
{करिष्यसि जगत्सर्वमपाञ्चालं किलोद्यतः}


\twolineshloka
{एवं सिद्धाऽब्रुवन्वाचो भविष्यति च तत्तथा}
{तस्मात्त्वं पुरुषव्याघ्र पाञ्चालाञ्जहि सानुगान्}


\twolineshloka
{न तेऽस्त्रगोचरे शक्ताः स्थातुं देवाः सवासवाः}
{किमु पार्थाः सपाञ्चालाः सत्यमेतद्ब्रवीमि ते}


\twolineshloka
{न त्वां समर्थाः सङ्ग्रामे पाण्डवाः सह सोमकैः}
{बलाद्योधयितुं वीर सत्यमेतद्ब्रवीमि ते}


\twolineshloka
{गच्छगच्छ महाबाहो न नः कालात्ययो भवेत्}
{इयं हि द्रवते सेना पार्तसायकपीडिता}


\twolineshloka
{शक्तो ह्यसि महाबाहो दिव्येन स्वेन तेजसा}
{निग्रहे पाण्डुपुत्राणां पाञ्चालानां च मानद}


\chapter{अध्यायः १६१}
\twolineshloka
{सञ्जय उवाच}
{}


\threelineshloka
{दुर्योधनेनैवमुक्तो द्रौणिराहवदुर्मदः}
{चकारारिवधे यत्नमिन्द्रो दैत्यवधे यथा}
{प्रत्युवाच महाबाहुस्तव पुत्रमिदं वचः}


\twolineshloka
{सत्यमेतन्महाबाहो यथा वदसि कौरव}
{प्रिया हि पाण्डवा नित्यं मम चापि पितुश्च मे}


\twolineshloka
{तथैवावां प्रियौ तेषां न तु युद्धे कुरूद्वह}
{शक्तितस्तात युध्यामस्त्यक्त्वा प्राणानभीतवत्}


\twolineshloka
{अहं कर्णश्च शल्यश्च कृपो हार्दिक्य एव च}
{निमेषात्पाण्डवीं सेनां क्षपयेम नृपोत्तम}


\twolineshloka
{ते चापि कौरवीं सेनां निमेषार्धात्कुरूद्वह}
{क्षपयेयुर्महाबाहो न स्याम यदि संयुगे}


\twolineshloka
{युध्याम पाण्डवाच्छक्त्या ते चाप्यस्मान्यथाबलम्}
{तेजस्तेजः समासाद्य प्रशमं यातु भारत}


\twolineshloka
{अशक्त्या तरसा जेतुं पाण्डवानामनीकिनी}
{जीवत्सु पाण्डुपुत्रेषु तद्धि सत्यं ब्रवीमि ते}


\twolineshloka
{आत्मार्थं युध्यमानास्ते समर्थाः पाण्डुनन्दनाः}
{किमर्थं तव सैन्यानि न हनिष्यन्ति भारत}


\twolineshloka
{त्वं तु लुब्धतमो राजन्निकृतिज्ञश्च कौरव}
{सर्वाभिशङ्की मानी च ततोऽस्मानभिशङ्कसे}


\twolineshloka
{मन्ये त्वं कुत्सितो राजन्पापात्मा पापपूरुषः}
{अन्यानपि स नः क्षुद्र शङ्कसे पापभावितः}


\threelineshloka
{अहं तु यत्नमास्थाय त्वदर्थे त्यक्तजीवितः}
{एष गच्छामि सङ्ग्रामं त्वत्कृते कुरुनन्दन}
{योत्स्येऽहं शत्रुभिः सार्धं जेष्यामि च वरान्वरान्}


\twolineshloka
{पाञ्चालैः सह योत्स्यामि सोमकैः केकयैस्तथा}
{पाण्डवेयैश्च सङ्ग्रामे त्वत्प्रियार्थमरिन्दम}


\twolineshloka
{अद्य मद्बाणनिर्दग्धाः पाञ्चालाः सोमकास्तथा}
{सिंहेनेवार्दिता गावो विद्रविष्यन्ति सर्वशः}


\twolineshloka
{अद्य धर्मसुतो राजा दृष्ट्वा मम पराक्रमम्}
{अश्वत्थाममयं लोकं मंस्यते सह सोमकैः}


\twolineshloka
{आगमिष्यति निर्वेदं धर्मपुत्रो युधिष्ठिरः}
{दृष्ट्वा विनिहतान्सङ्ख्ये पाञ्चालान्सोमकैः सह}


\twolineshloka
{ये मां युद्धेऽभियोत्स्यन्ति तान्हनिष्यामि भारत}
{न हि ते वीर मोक्ष्यन्ते मद्बाह्वन्तरमागताः}


\threelineshloka
{एवमुक्त्वा महाबाहुः पुत्रं दुर्योधनं तव}
{अभ्यवर्तत युद्धाय त्रासयन्सर्वधन्विनः}
{चिकीर्षुस्तव पुत्राणां प्रियं प्राणभृतां वरः}


% Check verse!
ततोऽब्रवीत्सकैकेयान्पाञ्चालान्गौतमीसुतः
\twolineshloka
{प्रहरध्वमितः सर्वे मम गात्रे महारथाः}
{स्थिरीभूताश्च युध्यध्वं दर्शयन्तोऽस्त्रलाघवम्}


\twolineshloka
{एवमुक्तास्तु ते सर्वे शस्त्रवृषीरपातयन्}
{द्रौणिं प्रति महाराज जलं जलधरा इव}


\twolineshloka
{तान्निहत्य शरान्द्रौणिर्दश वीरानपोथयत्}
{प्रमुखे पाण्डुपुत्राणां धृष्टद्युम्नस्य च प्रभो}


\twolineshloka
{ते हन्यमानाः समरे पाञ्चालाः सोमकास्तथा}
{परित्यज्य रणे द्रौणिं व्यद्रवन्त दिशो दश}


\twolineshloka
{तान्दृष्ट्वा द्रवतः शूरान्पाञ्चालान्सहसोमकान्}
{धृष्टद्युम्नो महाराज द्रौणिमभ्यद्रवद्रणे}


\twolineshloka
{ततः काञ्चनचित्राणां सजलाम्बुदनादिनाम्}
{वृतः शतेन शूराणां रथानामनिवर्तिनाम्}


\twolineshloka
{पुत्रः पाञ्चालराजस्य धृष्टद्युम्नो महारथः}
{द्रौणिमित्यब्रवीद्वाक्यं दृष्ट्वा योधान्निपातितान्}


\threelineshloka
{आचार्यपुत्र भद्रं ते किमन्यैर्निहतैस्तव}
{समागच्छ मया सार्धं यदि शूरोऽसि संयुगे}
{अहं त्वां निहनिष्यामि तिष्ठेदानीं ममाग्रतः}


\twolineshloka
{ततस्तमाचार्यसुतं धृष्टद्युम्नः प्रतापवान्}
{मर्मभिद्भिः शरैस्तीक्ष्णैर्जघान भरतर्षभ}


\threelineshloka
{ते तु पङ्क्तीकृता द्रौणिं शरा विविशुराशुगाः}
{रुक्मपुङ्खाः प्रसन्नाग्राः सर्वकायावदारणाः}
{मध्वर्थिन इवोद्दामा भ्रमराः पुष्पितं द्रुमम्}


\twolineshloka
{सोऽतिविद्धो भृशं क्रुद्धः पदाक्रान्त इवोरगः}
{मानी द्रौणिरसम्भ्रान्तो बाणपाणिरभाषत}


\twolineshloka
{धृष्टद्युम्न स्थिरो भूत्वा मुहूर्तं प्रतिपालय}
{यावत्त्वां निशितैर्बाणैः प्रेषयामि यमक्षयम्}


\twolineshloka
{द्रौणिरेवमथाभाष्य पार्षतं परवीरहा}
{छादयामास बाणौघैः समन्ताल्लुघुहस्तवत्}


\twolineshloka
{स बाध्यमानः समरे द्रौणिना युद्धदुर्मदः}
{द्रौणिं पाञ्चालतनयो वाग्भिरातर्जयत्तदा}


\twolineshloka
{न जानीषे प्रतिज्ञां मे विप्रोत्पत्तिं तथैव च}
{द्रोणं हत्वा किल मया हन्तव्यस्त्वं सुदुर्मते}


% Check verse!
ततस्त्वाऽहं न हन्म्यद्य द्रोणे जीवति संयुगे
\threelineshloka
{इमां तु रजनीं प्राप्तामप्रभातां सुदुर्मते}
{निहत्य पितरं तेऽद्य ततस्त्वामपि संयुगे}
{नेष्यामि प्रेतलोकाय ह्येतन्मे मनसि स्थितम्}


\twolineshloka
{यस्ते पार्थेषु विद्वेषो या भक्तिः कौरवेषु च}
{तां दर्शय स्थिरो भूत्वा न मे जीवन्विमोक्ष्यसे}


\twolineshloka
{यो हि ब्राह्मण्यमुत्सृज्य क्षत्रधर्मरतो द्विजः}
{स वध्यः सर्वलोकस्य यथा त्वं पुरुषाधमः}


\twolineshloka
{इत्युक्तः परुषं वाक्यं पार्षतेन द्विजोत्तमः}
{क्रोधमाहारयत्तीव्रं तिष्ठतिष्ठेति चाब्रवीत्}


\twolineshloka
{निर्दहन्निव चक्षुर्भ्यां पार्षतं सोऽभ्यवैक्षत}
{छादयामास च शरैर्निःश्वसन्पन्नगो यथा}


\twolineshloka
{स च्छाद्यमानः समरे द्रौणिना राजसत्तम}
{सर्वपाञ्चालसेनाभिः संवृतो रथसत्तमः}


\twolineshloka
{नाकम्पत महाबाहुः स्ववीर्यं समुपाश्रितः}
{सायकांश्चैव विविधानश्वत्थाम्नि मुमोच ह}


\threelineshloka
{तौ पुनः सन्न्यवर्तेतां प्राणद्यूतपणे रणे}
{निपीडयन्तौ बाणौघैः परस्परममर्षिणौ}
{उत्सृजन्तौ महेष्वासौ शरवृष्टीः समन्ततः}


\twolineshloka
{द्रौणिपार्षतयोर्युद्धं घोररूपं भयानकम्}
{दृष्ट्वा सम्पूजयामासुः सिद्धचारणवादिकाः}


\twolineshloka
{शरौघैः पूरयन्तौ तावाकाशं च दिशस्तथा}
{अलक्ष्यौ समयुध्येतां महत्कृत्वा शरैस्तमः}


\twolineshloka
{नृत्यमानाविव रणे मण्डलीकृतकार्मुकौ}
{परस्परवधे यत्तौ सर्वभूतभयङ्करौ}


\twolineshloka
{अयुध्येतां महाबाहू चित्रं लघु च सुष्ठु च}
{सम्पूज्यमानौ समरे योधमुख्यैः सहस्रशः}


\twolineshloka
{तौ प्रबुद्धौ रणे दृष्ट्वा वने वन्यौ गजाविव}
{उभयोः सेनयोर्हर्षस्तुमुलः समपद्यत}


\twolineshloka
{सिंहनादरवाश्चासन्दध्मुः शङ्खांश्च सैनिकाः}
{वादित्राण्यभ्यवाद्यन्त शतशोऽथ सहस्रशः}


\twolineshloka
{तस्मिंस्तु तुमुले युद्धे भीरुणां भयवर्धने}
{मुहूर्तमपि तद्युद्धं समरूपं तदाऽभवत्}


\threelineshloka
{ततो द्रौणिर्महाराज पार्षतस्य महात्मनः}
{ध्वजं धनुस्तथा छत्रमुभौ च पार्ष्णिसारथी}
{सूतमश्वांश्च चतुरो निहत्याभ्यद्रवद्रणे}


\twolineshloka
{पाञ्चालांश्चैव तान्सर्वान्बाणैः सन्नतपर्वभिः}
{व्यद्रावयदमेयात्मा शतशोऽथ सहस्रशः}


\twolineshloka
{ततस्तु विव्यथे सेना पाण्डवी भरतर्षभ}
{दृष्ट्वा द्रौणेर्महत्कर्म वासवस्येव संयुगे}


\twolineshloka
{शतेन च शतं हत्वा पाञ्चालानां महारथः}
{त्रिभिश्च निशितैर्बाणैर्हत्वा त्रीन्वै महारथान्}


\twolineshloka
{द्रौणिर्द्रुपदपुत्रस्य फल्गुनस्य च पश्यतः}
{नाशयामास पाञ्चालान्भूयिष्ठं ये व्यवस्थिताः}


\twolineshloka
{ते वध्यमानाः पाञ्चालाः समरे सह सृञ्जयैः}
{अगच्छन्द्रौणिमुत्सृज्य विप्रकीर्णरथध्वजाः}


\twolineshloka
{स जित्वा समरे शत्रून्द्रोणपुत्रो महारथः}
{ननाद सुमहानादं तपान्ते जलदो यथा}


\twolineshloka
{स निहत्य बहूञ्छूरानश्वत्थामा व्यरोचत}
{युगान्ते सर्वभूतानि भस्म कृत्वेव पावकः}


\twolineshloka
{सम्पूज्यमानो युधि कौरवेयै--र्निर्जित्य सङ्ख्येऽरिगणान्सहस्रशः}
{व्यरोचत द्रोणसुतः प्रतापवा--न्यथा सुरेन्द्रोऽरिगणान्निहत्य वै}


\chapter{अध्यायः १६२}
\twolineshloka
{सञ्जय उवाच}
{}


\twolineshloka
{ततो युधिष्ठिरश्चैव भीमसेनश्च पाण्डवः}
{द्रोणपुत्रं महाराज समन्तात्पर्यवारयन्}


\threelineshloka
{ततो दुर्योधनो राजा भारद्वाजेन संवृतः}
{अभ्ययात्पाण्डवान्सङ्ख्ये ततो युद्धमवर्तत}
{घोररूपं महाराज भीरूणां भयवर्धनम्}


\twolineshloka
{अम्बष्ठान्मालवान्वङ्गाञ्छिबींस्त्रैगर्तकानपि}
{प्राहिणोन्मृत्युलोकाय गणान्क्रुद्धो वृकोदरः}


\twolineshloka
{अभीषाहाञ्छूरसेनान्क्षत्रियान्युद्धदुर्मदान्}
{निकृत्त्य पृथिवीं चक्रे भीमः शोणितकर्दमाम्}


\twolineshloka
{यौधेयानद्रिजान्राजन्मद्रकान्मालवानपि}
{प्राहिणोन्मृत्युलोकाय किरीटी निशितैः शरैः}


\twolineshloka
{प्रगाढमञ्चोगतिभिर्नाराचैरभिताडिताः}
{निपेतुर्द्विरदा भूमौ द्विशृङ्गा इव पर्वताः}


\twolineshloka
{निकृत्तैर्हस्तिहस्तैश्च चेष्टमानैरितस्ततः}
{रराज वसुधाऽऽकीर्णा विसर्पद्भिरिवोरगैः}


\twolineshloka
{क्षिप्तैः कनकचित्रैश्च नृपच्छत्रैः क्षितिर्बभौ}
{द्यौरिवादित्यचन्द्राद्यैर्ग्रहैः कीर्णा युगक्षये}


\twolineshloka
{हत प्रहरताभीता विध्यत व्यवकृन्तत}
{इत्यासीत्तुमुलः शब्दः शोणाश्वस्य रथं प्रति}


\twolineshloka
{द्रोणस्तु परमक्रुद्धो वायव्यास्त्रेण संयुगे}
{व्यधमत्तान्महवायुर्मेघानिव दुरत्ययः}


\twolineshloka
{ते हन्यमाना द्रोणेन पाञ्चालाः प्राद्रवन्भयात्}
{पश्यतो भीमसेनस्य पार्थस्य च महात्मनः}


\twolineshloka
{ततः किरीटि भीमश्च सहसा सन्न्यवर्तताम्}
{महता रथवंशेन परिगृह्य बलं महत्}


\twolineshloka
{बीभत्सुर्दक्षिणं वार्श्वमुत्तरं तु वृकोदरः}
{भारद्वाजं शरौघाभ्यां महद्भामभ्यवर्षताम्}


\twolineshloka
{तौ तथा सृञ्जयाश्चैव पाञ्चालाश्च महौजसः}
{अन्वगच्छन्महाराज मात्स्यैश्च सह सोमकैः}


\twolineshloka
{तथैव तव पुत्रस्य रथोदाराः प्रहारिणः}
{महत्या सेनया राजञ्जग्मुर्द्रोणरथं प्रति}


\twolineshloka
{ततः सा भारती सेना हन्यमाना किरीटिना}
{तमसा निद्रया चैव पुनरेव व्यदीर्यत}


\twolineshloka
{द्रोणेन वार्यमाणास्ते स्वयं तव सुतेन च}
{नाशक्यन्त महाराज योधा वारयितुं तदा}


\twolineshloka
{सा पाण्डुपुत्रस्य शरैर्दीर्यमाणा महाचमूः}
{तमसा संवृते लोके व्यद्रवत्सर्वतोमुखी}


\twolineshloka
{उत्सृज्य शतशो वाहांस्तत्र केचिन्नराधिपाः}
{प्राद्रवन्त महाराज भयाविष्टाः समन्ततः}


\chapter{अध्यायः १६३}
\twolineshloka
{सञ्जय उवाच}
{}


\twolineshloka
{सोमदत्तं तु सम्प्रेक्ष्य विधुन्वानं महद्धनुः}
{सात्यकिः प्राह यन्तारं सोमदत्ताय मां वह}


\twolineshloka
{न ह्यहत्वा रणे शत्रुं सोमदत्तं महाबलम्}
{निवर्तिष्ये रणात्सूत सत्यमेतद्वचो मम}


\twolineshloka
{ततः सम्प्रैषयद्यन्ता सैन्धवांस्तान्मनोजवान्}
{तुरङ्गामाञ्छङ्खवर्णान्सर्वशब्दातिगान्रणे}


\twolineshloka
{तेऽवहन्युयुधानं तु मनोमारुतरंहसः}
{यथेन्द्रं हरयो राजन्पुरा दैत्यवधोद्यतम्}


\twolineshloka
{तमापतन्तं सम्प्रेक्ष्य सात्वतं रभसं रणे}
{सोमदत्तो महाबाहुरसम्भ्रान्तो न्यवर्तत}


\twolineshloka
{विमुञ्चञ्छरवर्षाणि पर्जन्य इव वृष्टिमान्}
{छादयामास शैनेयं जलदो भास्करं यथा}


\twolineshloka
{असम्भ्रान्तश्च समरे सात्यकिः कुरुपुङ्गवम्}
{छादयामास बाणौघैः समन्ताद्भरतर्षभ}


\twolineshloka
{सोमदत्तस्तुतं षष्ट्या विव्याधोरसि माधवम्}
{सात्यकिश्चापि तं राजन्नविध्यत्सायकैः शितैः}


\twolineshloka
{तावन्योन्यं शरैः कृत्तौ व्यराजेतां नरर्षभौ}
{सुपुष्पौ पुष्पसमये पुष्पिताविव किंशकौ}


\twolineshloka
{रुधिरोक्षितसर्वाङ्गौ कुरुवृष्णियशस्करौ}
{परस्परमवेक्षेतां दहन्ताविव लोचनैः}


\twolineshloka
{रथमण्डलमार्गेषु चरन्तावरिमर्दनौ}
{घोररूपौ हितावास्तां वृष्टिमन्ताविवाम्बुदौ}


\twolineshloka
{शरसम्भिन्नगात्रौ तु सर्वतः शखलीकृतौ}
{श्वाविधाविव राजेन्द्र दृश्येतां शरविक्षतौ}


\twolineshloka
{सुवर्णपुङ्खैरिषुभिराचितौ तौ व्यराजताम्}
{खद्योतैरावृतौ राजन्प्रावृषीव वनस्पती}


\twolineshloka
{सम्प्रदीपितसर्वाङ्गौ सायकैस्तैर्महारथौ}
{अदृश्येतां रणे क्रुद्धावुल्काभिरिव कुञ्जरौ}


\twolineshloka
{ततो युधि महारज सोमदत्तो महारथः}
{अर्धचन्द्रेम चिच्छेद माधवस्य महद्धनुः}


\twolineshloka
{अथैनं पञ्चविंशत्या सायकानां समार्पयत्}
{त्वरमाणस्त्वराकाले पुनश्च दशभिः शरैः}


\twolineshloka
{अथान्यद्धनुरादाय सात्यकिर्वेगवत्तरम्}
{पञ्चभिः सायकैस्तूर्णं सोमदत्तमविध्यत}


\twolineshloka
{ततोऽपरेण भल्लेन ध्वजं चिच्छेद काञ्चनम्}
{बाह्लीकस्य रणे राजन्सात्यकिः प्रहसन्निव}


\twolineshloka
{सोमदत्तस्त्वसम्भ्रान्तो दृष्ट्वा केतुं निपातितम्}
{शैनेयं पञ्चविंशत्या सायकानां समाचिनोत्}


\twolineshloka
{सात्वतोऽपि रणे क्रुद्धः सोमदत्तस्य धन्विनः}
{धनुश्छिच्छेद भल्लेन क्षुरप्रेण शितेन ह}


\twolineshloka
{अथैनं रुक्मपुङ्खानां शतेन नतपर्वणाम्}
{आचिनोद्बहुधा राजन्भग्नदंष्ट्रमिव द्विपम्}


\twolineshloka
{अथान्यद्धनुरादाय सोमदत्तो महारथः}
{सात्यकिं छादयामास शरवृष्ट्या महाबलः}


\twolineshloka
{सोमदत्तं तु सङ्क्रुद्धो रणे विव्याध सात्यकिः}
{सात्यकिं शरजालेन सोमदत्तोऽप्यपीडयत्}


\twolineshloka
{दशभिः सात्वतस्यार्थे भीमोऽहन्बाह्लिकात्मजम्}
{सोमदत्तोप्यसम्भ्रान्तो भीममार्च्छच्छितैः शरैः}


\twolineshloka
{ततस्तु सात्वतस्यार्थे भीमसेनो नवं दृढम्}
{मुमोच परिघं घोरं सोमदत्तस्य वक्षसि}


\twolineshloka
{तमापतन्तं वेगेन परिघं घोरदर्शनम्}
{द्विधा चिच्छेद समरे प्रहसन्निव कौरवः}


\twolineshloka
{स पपात द्विधा च्छिन्न आयसः परिघो महान्}
{महीधरस्येव महच्छिखरं वज्रदारितम्}


\twolineshloka
{ततस्तु सात्यकी राजन्सोमदत्तस्य संयुगे}
{धनुश्चिच्छेद भल्लेन हस्तावापं च पञ्चभिः}


\twolineshloka
{ततश्चतुर्भिश्च शरैस्तूर्णं तांस्तुरगोत्तमान्}
{समीपं प्रेषयामास प्रेतराजस्य भारत}


\twolineshloka
{सारथेश्च शिरः कायाद्भल्लेन नतपर्वणा}
{जहार नरशार्दूलः प्रहसञ्छिनिपुङ्गवः}


\twolineshloka
{ततः शरं महाघोरं ज्वलन्तमिव पावकम्}
{मुमोच सात्वतो राजन्स्वर्णपुङ्गं शिलाशितम्}


\twolineshloka
{स विमुक्तो बलवता शैनेयेन शरोत्तमः}
{घोरस्तस्योरसि विभो निपपाताशु भारत}


\twolineshloka
{सोऽतिविद्धो महाराज सात्वतेन महारथः}
{सोमदत्तो महाबाहुर्निपपात ममार च}


\twolineshloka
{तं दृष्ट्वा निहतं तत्र सोमदत्तं महारथाः}
{महता शरवर्षेण युयुधानमुपाद्रवन्}


\threelineshloka
{छाद्यमानं शरैर्दृष्ट्वा युयुधानं युधिष्ठिरः}
{पाण्डवाश्च महाराज सह सर्वैः प्रभद्रकैः}
{महत्या सेनया सार्धं द्रोणानीकमुपाद्रवन्}


\twolineshloka
{ततो युधिष्ठिरः क्रुद्धस्तावकानां महाबलम्}
{शरैर्विद्रावयामास भारद्वाजस्य पश्यतः}


\twolineshloka
{सैन्यानि द्रावयन्तं तु द्रोणो दृष्ट्वा युधिष्ठिरम्}
{अभिदुद्राव वेगेन क्रोधसंरक्तलोचनः}


\twolineshloka
{ततः सुनिशितैर्बाणैः पार्थं विव्याध सप्तभिः}
{युधिष्ठिरोऽपि सङ्क्रुद्धः प्रतिविव्याध पञ्चभिः}


\twolineshloka
{सोऽतिविद्धो महाबाहुः सृक्विणी परिसंलिहन्}
{युधिष्ठिरस्य चिच्छेद ध्वजं कार्मुकमेव च}


\twolineshloka
{स च्छिन्नधन्वा त्वरितस्त्वराकाले नृपोत्तमः}
{अन्यदादत्त वेगेन कार्मुकं समरे दृढम्}


\twolineshloka
{ततः शरसहस्रेण द्रोणं विव्याध पार्थिवः}
{साश्वसूतध्वजरथं तदद्भुतमिवाभवत्}


\twolineshloka
{ततो मुहूर्तं व्यथितः शरपातप्रपीडितः}
{निषसाद रथोपस्थे द्रोणो भरतसत्तम}


\twolineshloka
{प्रतिलभ्य ततः संज्ञां मुहूर्ताद्द्विजसत्तमः}
{क्रोधेन महताऽऽविष्टो वायव्यास्त्रमवासृजत्}


\twolineshloka
{असम्भ्रान्तस्ततः पार्थो वायव्येनैव वीर्यवान्}
{तदस्त्रमस्त्रेण रणे स्तम्भयामास भारत}


\threelineshloka
{चिच्छेद च धनुर्दीर्घं ब्राह्मणस्य च पाण्डवः}
{ततोऽन्यद्धनुरादत्त द्रोणः क्षत्रियमर्दनः}
{तदप्यस्य शितैर्भल्लैश्चिच्छेद कुरुपुङ्गवः}


% Check verse!
ततोऽब्रवीद्वासुदेवः कुन्तीपुत्रं युधिष्ठिरम्
\twolineshloka
{युधिष्ठिर महाबाहो यत्त्वां वक्ष्यामि तच्छ्रणु}
{उपारमस्व युद्धे त्वं द्रोणाद्भरतसत्तम}


\twolineshloka
{यतते हि सदा द्रोणो ग्रहणे तव संयुगे}
{नानुरूपमहं मन्ये युद्धमस्य त्वया सह}


\twolineshloka
{योऽस्य सृष्टो विनाशाय स एवैनं हनिष्यति}
{परिवर्ज्य गुरुं याहि यत्र राजा सुयोधनः}


\twolineshloka
{राजा राज्ञा हि योद्धव्यो नाराज्ञा युद्धमिष्यते}
{तत्र त्वं गच्छ कौन्तेय हस्त्यश्वरथसंवृतः}


\twolineshloka
{यावन्मात्रेण च मया सहायेन धनञ्जयः}
{भीमश्च रथशार्दूलो युध्यते कौरवैः सह}


\twolineshloka
{वासुदेववचः श्रुत्वा धर्मराजो युधिष्ठिरः}
{मुहूर्तं चिन्तयित्वा तु ततो दारुणमाहवम्}


\twolineshloka
{प्रायाद्द्रुतममित्रघ्नो यत्र भीमो व्यवस्थितः}
{विनिघ्नंस्तावकान्योधान्व्यादितास्य इवान्तकः}


\threelineshloka
{रथघोषेण महता नादयन्वसुधातलम्}
{पर्जन्य इव धर्मान्ते नादयन्वै दिशो दश}
{भीमस्य निघ्नतः शत्रून्पार्ष्णिं जाग्रह पाण्डवः}


\twolineshloka
{द्रोणोऽपि पाण्डुपाञ्चालान्व्यधमद्रजनीमुखे}
{`नादयंस्तलघोषेण तव सैन्यानि मारिष'}


\chapter{अध्यायः १६४}
\twolineshloka
{सञ्जय उवाच}
{}


\twolineshloka
{वर्तमाने यथा यद्धे घोररूपे भयावहे}
{तमसा संवृते लोके रजसा च महीपते}


\twolineshloka
{नापश्यन्त रणे योधाः परस्परमवस्थिताः}
{अनुमानेन संज्ञाभिर्युद्धं तद्ववृधे महत्}


\threelineshloka
{नरनागाश्वमथनं परमं रोमहर्षणम्}
{द्रोणकर्णकृपा वीरा भीमपार्षतसात्यकाः}
{अन्योन्यं क्षोभयामासुः सैन्यानि नृपसत्तम}


\twolineshloka
{वध्यमानानि सैन्यानि समन्तात्तैर्महारथैः}
{तमसा संवृते चैव समन्ताद्विप्रदुद्रुवुः}


\twolineshloka
{ते सर्वतो विद्रवन्तो योधा विध्वस्तचेतनाः}
{अहन्यन्त महाराज धावमानाश्च संयुगे}


\twolineshloka
{महारथसहस्राणि जघ्नुरन्योन्यमाहवे}
{अन्ये तमसि मूढानि पुत्रस्य तव मन्त्रिते}


\threelineshloka
{ततः सर्वाणि सैन्यानि सेनागोपाश्च भारत}
{व्यमुह्यन्त रणे तत्र तमसा संवृते सति ॥धृतराष्ट्र उवाच}
{}


\twolineshloka
{तेषां संलोड्यमानानां पाण्डवैर्विहतौजसाम्}
{अन्धे तमसि मग्नानामासीत्किं वो मनस्तदा}


\threelineshloka
{कथं प्रकाशस्तेषां वा मम सैन्यस्य वा पुनः}
{बभूव लोके तमसा तथा सञ्जय संवृते ॥सञ्जय उवाच}
{}


\twolineshloka
{ततः सर्वाणि सैन्यानि हतशिष्टानि यानि वै}
{सेनागोप्तॄनथादिश्य पुनर्व्यूहमकल्पयत्}


\twolineshloka
{द्रोणः पुरस्ताज्जघने तु शल्य--स्तथा द्रौणिः कृतवर्मा सौबलश्च}
{स्वयं तु सर्वाणि बलानि राज--न्राजाऽभ्ययाद्गोपयन्वै निशायाम्}


\twolineshloka
{उवाच सर्वांश्च पदातिसङ्घा--न्दुर्योधनः पार्थिवसान्त्वपूर्वम्}
{उत्सृज्य सर्वे परमायुधानिगृह्णीत हस्तैर्ज्वलितान्प्रदीपान्}


% Check verse!
ते चोदिताः पार्थिवसत्तमेनततः प्रहृष्टा जगृहुः प्रदीपान्
\twolineshloka
{देवर्षिगन्धर्वसुरर्षिसङ्घाविद्याधराश्चाप्सरसां गणाश्च}
{नागाः सयक्षोरगकिन्नराश्चहृष्टा दिविस्था जगृहुः प्रदीपान्}


\twolineshloka
{दिग्धैर्वतेभ्यश्च समापतन्तो--ऽदृश्यन्त दीपाः ससुगन्धितैलाः}
{विशेषतो नारदपर्वताभ्यांसम्बोध्यमानाः कुरुपाण्डवार्थम्}


\twolineshloka
{सा भूय एव ध्वजिनी विभक्ताव्यरोचताग्निप्रभया निशायाम्}
{महाधनैराभरणैश्च दिव्यैःशस्त्रैश्च दीप्तैरपि सम्पतद्भिः}


\twolineshloka
{रथेरथे पञ्च विदीपकास्तुप्रदीपकास्तत्र गजे त्रयश्च}
{प्रत्यश्वमेकश्च महाप्रदीपःकृतास्तु तैः पाण्डवैः कौरवेयैः}


% Check verse!
क्षणेन सर्वे विहिताः प्रदीपाव्यादीपयन्तो ध्वजिनीं तवासु
\twolineshloka
{सर्वास्तु सेना व्यतिसेव्यमानाःपदातिभिः पावकतैलहस्तैः}
{प्रकाश्यमाना ददृशुर्निशायांयथान्तरिक्षे जलदास्तडिद्भिः}


\twolineshloka
{प्रकाशितायां तु ततो ध्वजिन्यांद्रोणोऽग्निकल्पः प्रतपन्सपत्नान्}
{रराज राजेन्दर सुवर्णवर्मामध्यं गतः सूर्य इवांशुमाली}


\twolineshloka
{जाम्बूनदेष्वाभरणेषु चैवनिष्केषु शुद्धेषु शरासनेषु}
{पीतेषु शस्त्रेषु च पावकस्यप्रतिप्रभास्तत्र तदा बभूवुः}


\twolineshloka
{गदाश्च शैक्याः परिघाश्च शुभ्रारथेषु शक्त्यश्च विवर्तमानाः}
{प्रतिप्रभारश्मिभिराजमीढपुनःपुनः सञ्जनयन्ति दीपान्}


\twolineshloka
{छत्राणि वालव्यजनानि खङ्गादीप्ता महोल्काश्च तथैव राजन्}
{व्याघूर्णमानाश्च सुवर्णमालाव्यायच्छतां तत्र तदा विरेजुः}


\twolineshloka
{शस्त्रप्रभाभिश्च विराजमानंदीपप्रभाभिश्च तदा बलं तत्}
{प्रकाशितं चाभरणप्रभाभि--र्भृशं प्रकाशं नृपते बभूव}


\twolineshloka
{पीतानि शस्त्राण्यसृगुक्षितानिवीरावधूतानि तनुच्छदानि}
{दीप्तां प्रभां प्राजनयन्त तत्रतपात्यये विद्युदिवान्तरिक्षे}


\twolineshloka
{प्रकम्पितानामभिधातवेगै--रभिघ्नतां चापततां जवेन}
{वक्राण्यकाशन्त तदा नराणांवाय्वीरितानीव महाम्बुजानि}


\twolineshloka
{महावने दारुमये प्रदीप्तेयथा प्रभा भास्करस्यापि नश्येत्}
{तथा तदाऽऽसीद्वजिनी प्रदीप्तामहाभया भारत भीमरूपा}


\twolineshloka
{तत्संप्रदीप्तां बलमस्मदीयंनिशाम्य पार्थास्त्वरितास्तथैव}
{सर्वेषु सैन्येषु पदातिसङ्घा--नचोदयंस्तेऽपि चक्रुः प्रदीपान्}


\twolineshloka
{गजेगजे सप्त कृपाः प्रदीपारथेरथे चैव दश प्रदीपाः}
{द्वावश्वपृष्ठे परिपार्श्वतोऽन्येध्वजेषु चान्ये जघनेषु चान्ये}


\twolineshloka
{सेनासु सर्वासु च पार्श्वतोऽन्येपश्चात्पुरस्ताच्च समन्ततश्च}
{मध्ये तथाऽन्ये ज्वलिताग्निहस्ताव्यदीपयन्पाण्डुसुतस्य सेनाम्}


% Check verse!
मध्ये तथाऽन्ये ज्वलिताग्निहस्ताःसेनाद्वयेऽपि स्म नरा विचेरुः
\twolineshloka
{सर्वेषु सैन्येषु पदातिसङ्घाविमिश्रिता हस्तिरथाश्वबृन्दैः}
{व्यदीपयंस्ते ध्वजिनीं प्रदीप्ता--स्तथा बलं पाण्डवेयाभिगुप्तम्}


\twolineshloka
{तेन प्रदीप्तेन तथा प्रदीप्तं--बलं तवासीद्बलवद्बलेन}
{भाः कुर्वता भानुमता ग्रहेणदिवाकरेणाग्निरिवाभिगुप्तः}


\twolineshloka
{तयोः प्रभाः पृथिवीमन्तरिक्षंसर्वा व्यतिक्रम्य दिशश्च वृद्धाः}
{तेन प्रकाशेन भृशं प्रकाशंबभूव तेषां तव चैव सैन्यम्}


\twolineshloka
{तेन प्रकाशेन दिवं गतेनसम्बोधिता देवगणाश्च राजन्}
{गन्धर्वयक्षासुरसिद्धसङ्घाःसमागमन्नप्सरसश्च सर्वाः}


\twolineshloka
{तद्देवगन्धर्वसमाकुलं चयक्षासुरेन्द्राप्सरसां गणैश्च}
{हतैश्च शूरैर्दिवमारुहद्भि--रायोधनं दिव्यकल्पं बभूव}


\twolineshloka
{रथाश्वनागाकुलदीपदीप्तंसंरब्धयोधं हतविद्रुताश्वम्}
{महद्बलं व्यूढरथाश्वनागंसुरासुरव्यूहसमं बभूव}


\twolineshloka
{तच्छक्तिसङ्घाकुलचण़्डवातंमहारथाभ्रं गजवाजिघोषम्}
{शस्त्रौगवर्षं रुधिराम्बुधारंनिशि प्रवृत्तं रथदुर्दिनं तत्}


\twolineshloka
{तस्मिन्महाग्निप्रतिमो महात्मासन्तापंयन्पाण्डवान्विप्रमुख्यः}
{गभस्तिभिर्मध्यगतो यथाऽर्कोवर्षात्यये तद्वदभून्नरेन्द्र}


\chapter{अध्यायः १६५}
\twolineshloka
{सञ्जय उवाच}
{}


\twolineshloka
{प्रकाशिते तदा लोके रजसा तमसावृते}
{समाजग्मुरथो वीराः परस्परवधैषिणः}


\twolineshloka
{ते समेत्य रणे राजञ्शस्त्रप्रासासिधारिणः}
{परस्परमुदैक्षन्त परस्परकृतागसः}


\twolineshloka
{प्रदीपानां सहस्रैश्च दीप्यमानैः समन्ततः}
{रत्नाचितैः स्वर्णदण्डैर्गन्धतैलावसिञ्चितैः}


\twolineshloka
{देवगन्धर्वदीपाद्यैः प्रभाभिरधिकोज्ज्वलैः}
{विरराज तदा भूमिर्ग्रहैर्द्यौरिव भारत}


\twolineshloka
{उल्काशतैः प्रज्वलितै रणभूमिर्व्यराजत}
{दह्यमानेव लोकानामभावे च वसुन्धरा}


\twolineshloka
{व्यदीप्यन्त दिशः सर्वाः प्रदीपैस्तैः समन्ततः}
{वर्षाप्रदोषे खद्योतैर्वृता वृक्षा इवाबभुः}


\threelineshloka
{असज्जन्त ततो वीरा विरेष्वेव पृथक्पृथक्}
{नागा नागैः समाजग्मुस्तुरगा हयसादिभिः}
{रथा रथवरैरेव समजग्मुर्मुदा युताः}


\twolineshloka
{तस्मिन्रात्रिमुखे घोरे तव पुत्रस्य शासनात्}
{चतुरङ्गस्य सैन्यस्य सम्पातश्च महारनभूत्}


\threelineshloka
{ततोऽर्जुनो महाराज कौरवाणामनीकिनीम्}
{व्यधमत्त्वरया युक्तः क्षपयन्सर्वपार्थिवान् ॥धृतराष्ट्र उवाच}
{}


\twolineshloka
{तस्मिन्प्रविष्टे संरब्धे मम पुत्रस्य वाहिनीम्}
{अमृष्यमाणे दुर्धर्षे कथमासीन्मनो हि वः}


\twolineshloka
{किमकुर्वत सैन्यानि प्रविष्टे परषीडने}
{दुर्योधनश्च किं कृत्यं प्राप्तकालममन्यत}


\twolineshloka
{के चैनं समरे वीरं प्रत्युद्ययुररिन्दमाः}
{द्रोणं च के व्यरक्षन्त प्रविष्टे श्वेतवाहने}


\threelineshloka
{केऽरक्षन्दक्षिणं चक्रं के च द्रोणस्य सव्यतः}
{के पृष्ठतश्चाप्यभवन्वीरा वीरान्विनिघ्नतः}
{के पुरस्तादगच्छन्त निघ्नन्तः शास्त्रवान्रणे}


\twolineshloka
{यत्प्राविशन्महेष्वासः पाञ्चालनापराजिताः}
{नृत्यन्निव नरव्याघ्रो रथमार्गेषु वीर्यवान्}


\twolineshloka
{यो ददाह शरैर्द्रोणः पाञ्चालानां रथव्रजान्}
{धूमकेतुरिव क्रुद्धः कथं मृत्युमुपेयिवान्}


\twolineshloka
{अव्यग्रानेव हि परान्कथयस्यपराजितान्}
{हृष्टानुदीर्णान्सङ्ग्रामे न तथा सूत मामकान्}


\threelineshloka
{हतांश्चैव विदीर्णांश्च विप्रकीर्णांश्च शंससि}
{रथिनो विरथांश्चैव कृतान्युद्वेषु मामकान् ॥सञ्जय उवाच}
{}


\twolineshloka
{द्रोणस्य मतमाज्ञाय योद्वुकामस्य तां निशाम्}
{दुर्योधनो महाराज वश्यान्भ्रातॄनुवाच ह}


\twolineshloka
{कर्णं च वृषसेनं च मद्रराजं च कौरव}
{दुर्धर्षं दीर्घबाहुं च ये च तेषां पदानुगाः}


\twolineshloka
{द्रोणं यत्ताः पराक्रान्ताः सर्वे रक्षन्तु पृष्ठतः}
{हार्दिक्यो दक्षिणं चक्रं शल्यश्चैवोत्तरं तथा}


\threelineshloka
{त्रिगर्तानां च ये शूरा हतशिष्टा महारथाः}
{तांश्चैव पुरतः सर्वान्पुत्रस्ते समचोदयत् ॥दुर्योधन उवाच}
{}


\twolineshloka
{आचार्यो हि सुसंयत्तो भृशं यत्ताश्च पाण्डवाः}
{तं रक्षत सुसंयत्ता निघ्नन्तं शास्त्रवान्रणे}


\twolineshloka
{द्रोणो हि बलवान्युद्धे क्षिप्रहस्तः प्रतापवान्}
{निर्जयेत्त्रिदशान्युद्धे किमु पार्थान्ससोमकान्}


\twolineshloka
{ते यूयं सहिताः सर्वे भृशं यत्ता महारथाः}
{द्रोणं रक्षत पाञ्चालाद्धृष्टद्युम्नान्महारथात्}


\twolineshloka
{पाण़्डवीयेषु सैन्येषु न तं पश्याम कञ्चन}
{यो योधयेद्रणे द्रोणं धृष्टद्युम्नादृते पुमान्}


\twolineshloka
{तस्मात्सर्वात्मना मन्ये भारद्वाजस्य रक्षणम्}
{सुगुप्तः पाण्डवान्हन्यात्सृञ्जयांश्च ससोमकान्}


\twolineshloka
{सृञ्जयेष्वथ सर्वेषु निहतेषु चमूमुखे}
{धृष्टद्युम्नं रमे द्रौणिर्हनिष्यति न संशयः}


\threelineshloka
{तथाऽर्जुनं च राधेयो हनिष्यति महारथः}
{भीमसेनमहं चापि युद्धेजेष्यामि दीक्षितः}
{शेषांश्च पाण्डवान्योधाः प्रसभं हीनतेजसः}


\twolineshloka
{सोयं मम जयो व्यक्तो दीर्घकालं भविष्यति}
{तस्माद्रक्षत सङ्ग्रामे द्रोणमेव महारथम्}


\twolineshloka
{इत्युक्त्वा भरतश्रेष्ठ पुत्रो दुर्याधनस्तव}
{व्यादिदेश तथा सैन्यं तस्मिंस्तमसि दारुणे}


\twolineshloka
{ततः प्रववृते युद्धं रात्रौ भरतसत्तम}
{उभयोः सेनयोर्घोरं परस्परजिगीषया}


\twolineshloka
{अर्जुनः कौरवं सैन्यमर्जुनं चापि कौरवाः}
{नानाशस्त्रसमावायैरन्योन्यं समपीडयन्}


\twolineshloka
{द्रौणिः पाञ्चालराजं च भारद्वाजश्च सृञ्जयान्}
{छादयांचक्रिरे सङ्ख्ये शरैः सन्नतपर्वभिः}


\twolineshloka
{पाण्डुपाञ्चालसैन्यानां कौरवाणां च भारत}
{आसीन्निष्टानको घोरो निघ्नतामितरेतरम्}


\twolineshloka
{नैवास्माभिस्तथा पूर्वैर्दृष्टपूर्वं तथाविधम्}
{युद्धं यादृशमेवासीत्तां रात्रि सुभयावहम्}


\chapter{अध्यायः १६६}
\twolineshloka
{सञ्जय उवाच}
{}


\twolineshloka
{वर्तमाने तदा रौद्रे रात्रियुद्धे विशाम्पते}
{सर्वभूतक्षयकरे धर्मपुत्रो युधिष्ठिरः}


\twolineshloka
{अब्रवीत्पाण्डवांश्चैव पाञ्चालांश्चैव सोमकान्}
{अभिद्रवत संयात द्रोणमेव जिघांसया}


\twolineshloka
{राज्ञस्ते वचनाद्राजन्पाञ्चालाः सृञ्जयास्तथा}
{द्रोणमेवाभ्यवर्तन्त नदन्तो भैरवान्रवान्}


\twolineshloka
{तान्वयं प्रतिगर्जन्तः प्रत्युद्याताः स्म हर्षिताः}
{यथाशक्ति यथोत्साहं यथासत्वं च संयुगे}


\twolineshloka
{कृतवर्मा तु हार्दिक्यो युधिष्ठिरमुपाद्रवत्}
{द्रोणं प्रति समायान्तं मत्तो मत्तमिव द्विपम्}


\twolineshloka
{शैनेयं शरवर्षाणि विकिरन्तं समन्ततः}
{अभ्ययात्कौरवो राजन्भूरिः सङ्ग्राममूर्धनि}


\twolineshloka
{सहदेवमथायान्तं द्रोणप्रेप्सुं महारथम्}
{कर्णो वैकर्तनो राजन्वारयामास पाण़्डवम्}


\twolineshloka
{भीमसेनमथायान्तं व्यादितास्यामिवान्तकम्}
{स्वयं दुर्योधनो राजा प्रतीपं मृत्युमाव्रजत्}


\twolineshloka
{नकुलं च युधां श्रेष्ठं सर्वयुद्धविशारदम्}
{शकुनिः सौबलो राजन्वारयामास सत्वरः}


\twolineshloka
{शिखण्डिनमथायान्तं रथेन रथिनां वरम्}
{कृपः शारद्वतो राजन्वारयामास संयुगे}


\twolineshloka
{प्रतिविन्ध्यमथायान्तं मयूरसदृशैर्हयैः}
{दुःशासनो महाराज यत्तो यत्तमवारयत्}


\twolineshloka
{भैमसेनिमथायान्तं मायाशतविशारदम्}
{अश्वत्थामा महाराज राक्षसं प्रत्यषेधयत्}


\twolineshloka
{द्रुपदं वृषसेनस्तु ससैन्यं सपदानुगम्}
{वारयामास समरे द्रोणप्रेप्सुं महारथम्}


\twolineshloka
{विराटं द्रुतमायान्तं द्रोणस्य निधनं प्रति}
{मद्रराजः सुसङ्क्रुद्धो वारयामास भारत}


\twolineshloka
{शतानीकमथायान्तं नाकुलिं रभसं रणे}
{चित्रसेनो रुरोधाशु शरैर्द्रोणपरीप्सया}


\twolineshloka
{अर्जुनं च युधांश्रेष्ठं प्राद्रवन्तं महारथम्}
{अलायुधो महाराज राक्षसेन्द्रो न्यवारयत्}


\twolineshloka
{तथा द्रोणं महेष्वासं निघ्नन्तं शात्रवान्रणे}
{धृष्टद्युम्नोऽथ पाञ्चाल्यो हृष्टरूपमवारयत्}


\twolineshloka
{तथाऽन्यान्पाण्डुपुत्राणां समायातान्महारथान्}
{तावका रथिनो राजन्वारयामासुरोजसा}


\twolineshloka
{गजारोहा गजैस्तूर्णं सन्निपत्य महामृधे}
{योधयन्तः स्म दृश्यन्ते शतशोऽथ सहस्रशः}


\twolineshloka
{निशीथे तुरगा राजन्द्रावयन्तः परस्परम्}
{समदृश्यन्त वेगेन पक्षवन्तो यथाऽद्रयः}


\twolineshloka
{सादिनः सादिभिः सार्धं प्रासशक्त्यृष्टिपाणयः}
{समागच्छन्महाराज विनदन्तः पृथक्पृथक्}


\twolineshloka
{नरास्तु बहवस्तत्र समाजग्मुः परस्परम्}
{गादाभिर्मुसलैश्चैव नानाशस्त्रैश्च संयुगे}


\twolineshloka
{कृतवर्मा तु हार्दिक्यो धर्मपुत्रं युधिष्ठिरम्}
{वारयामास सङ्क्रुद्धो वेलेवोद्वृत्तमर्णवम्}


\twolineshloka
{युधिष्ठिरस्तु हार्दिक्यं विद्व्वा पञ्चभिराशुगैः}
{पुनर्विव्याध विंशत्या तिष्ठतिष्ठेति चाब्रवीत्}


\twolineshloka
{कृतवर्मा तु सङ्क्रुद्धो धर्मपुत्रस्य मारिष}
{धनुश्चिच्छेद भल्लेन तं च विव्याध सप्तभिः}


\twolineshloka
{अथान्यद्धनुरादाय धर्मपुत्रो महारथः}
{हार्दिक्यं दशभिर्बाणैर्बाह्वोरुरसि चार्पयत्}


\twolineshloka
{माधवस्तु रणे विद्धो धर्मपुत्रेण मारिष}
{प्राकम्पत च रोषेण सप्तभिश्चार्दयच्छरैः}


\twolineshloka
{तस्य पार्थो धनुश्छित्त्वा हस्तावापं निकृत्य च}
{प्राहिणोन्निशितान्बाणान्पञ्च राजञ्छिलाशितान्}


\twolineshloka
{ते तस्य कवचं भित्त्वा हेमचित्रं महाधनम्}
{प्राविशन्धरणीं भित्त्वा वल्मीकमिव पन्नगाः}


\twolineshloka
{अक्ष्णोर्निमेषमात्रेण सोऽन्यदादाय कार्मुकम्}
{विव्याध पाण्डवं षष्ट्या सूतं च नवभिः शरैः}


\twolineshloka
{तस्य शक्तिममेयात्मा पाण्डवो भुजगोपमाम्}
{चिक्षेप भरश्रेष्ठ रथे न्यस्य महद्धनुः}


\twolineshloka
{सा हेमचित्रा महती पाण्डवेयेन प्रेरिता}
{निर्भिद्य दक्षिणं बाहुं प्राविशद्धरणीतलम्}


\twolineshloka
{एतस्मिन्नेव काले तु गृह्य पार्थः पुनर्धनुः}
{हार्दिक्यं छादयामास शरैः सन्नतपर्वभिः}


\twolineshloka
{ततस्तु समरे शूरो वृष्णीनां प्रवरो रथी}
{व्यश्चसूतरथं चक्रे निमेषार्धाद्युधिष्ठिरम्}


\twolineshloka
{ततस्तु पाण्डवो ज्येष्ठः खङ्गं चर्म समाददे}
{तदस्य निशितैर्बाणैर्व्यधमन्माधवो रणे}


\twolineshloka
{तोमरं तु ततो गृह्य स्वर्णदण्डं दुरासदम्}
{प्रैषयत्समरे तूर्णं हार्दिक्यस्य युधिष्ठिरः}


\twolineshloka
{तमापतन्तं सहसा धर्मराजभुजच्युतम्}
{द्विधा चिच्छेद हार्दिक्यः कृतहस्तः स्मयन्निव}


\twolineshloka
{ततः शरशतेनाजौ धर्मपुत्रमवाकिरत्}
{कवचं चास्य सङ्क्रुद्धः शरैस्तीक्ष्णैरदारयत्}


\twolineshloka
{हार्दिक्यशरसञ्छन्नं कवचं तन्महाधनम्}
{व्यशीर्यत रणे राजंस्ताराजालमिवाम्बरात्}


\twolineshloka
{स च्छिन्नधन्वा विरथः शीर्णवर्मा शरार्दितः}
{अपायासीद्रणात्तूर्णं धर्मपुत्रो युधिष्ठिरः}


\twolineshloka
{कृतवर्मा तु निर्जित्य धर्मात्मानं युधिष्ठिरम्}
{पुनर्द्रोणस्य जुगुपे चक्रमेव महात्मनः}


\chapter{अध्यायः १६७}
\twolineshloka
{सञ्जय उवाच}
{}


\twolineshloka
{भूरिस्तु समरे राजञ्शैनेयं रथिनां वरम्}
{आपतन्तमपासेधत्प्रयाणादिव कुञ्जरम्}


\twolineshloka
{अथैनं सात्यकिः क्रुद्धः पञ्चभिर्निशितैः शरैः}
{विव्याध हृदये तस्य प्रास्रवत्तस्य शोणितम्}


\twolineshloka
{तथैव कौरवो युद्धे शैनेयं युद्धदुर्मदम्}
{दशिभिर्निशितैस्तीक्ष्णैरविध्यत भुजान्तरे}


\twolineshloka
{तावन्योन्यं महाराज ततक्षाते शरैर्भृशम्}
{क्रोधसंरक्तनयनौ क्रोधाद्विष्फार्य कार्मुके}


\twolineshloka
{तयोरासीन्महाराज शस्त्रवृष्टिः सुदारुणा}
{क्रुद्धयोः सायकमुचोर्यमान्तकनिकाशयोः}


\twolineshloka
{तावन्योन्यं शरैः राजन्सञ्छाद्य समवस्थितौ}
{मुहूर्तं चैव तद्युद्धं समरूपमिवाभावत्}


\twolineshloka
{ततः क्रुद्धो महाराज शैनेयः प्रहसन्निव}
{धनुश्चिच्छेद समरे कौरव्यस्य महात्मनः}


\twolineshloka
{अथैनं छिन्नधन्वानं नवभिर्निशितैः शरैः}
{विव्याध हृदये तूर्णं तिष्ठतिष्ठेति चाब्रवीत्}


\twolineshloka
{सोऽतिविद्धो बलवता शत्रुणा शत्रुतापनः}
{धनुरन्यत्समादाय सात्वतं प्रत्यविध्यत}


\twolineshloka
{स विद्ध्वा सात्वतं बाणैस्त्रिभिरेव विशाम्पते}
{धनुश्चिच्छेद भल्लेन सुतीक्ष्णेन हसन्निव}


\twolineshloka
{छिन्नधन्वा महाराज सात्यकिः क्रोधमूर्च्छितः}
{प्रजहार महावेगां शक्तिं तस्य महोरसि}


\twolineshloka
{स तु शक्त्या विभिन्नाङ्गो निपपात रथोत्तमात्}
{लोहिताङ्ग इवाकाशाद्दीप्तरश्मिर्यदृच्छया}


\twolineshloka
{तं तु दृष्ट्वा हतं शूरमश्वत्थामा महारथः}
{अभ्यधावत वेगेन शैनेयं प्रति संयुगे}


\twolineshloka
{तिष्ठतिष्ठेति चाभाष्य शैनेयं स नराधिप}
{अभ्यवर्षच्छरौघेण मेरुं वृष्ट्या यथाम्बुदः}


\twolineshloka
{तमापतन्तं संरब्धं शैनेयस्य रथं प्रति}
{घटोत्कचोऽब्रवीद्राजन्नाहं मुक्त्वा महारथः}


\twolineshloka
{तिष्ठतिष्ठ न मे जीवन्द्रोणपुत्र गमिष्यसि}
{एष त्वां निहनिष्यामि महिषं मृगराडिव}


\threelineshloka
{युद्धश्रद्धामहं तेऽद्य विनेष्यामि रणाजिरे}
{इत्युक्त्वा क्रोधताम्राक्षो राक्षसः परवीरहा}
{द्रोणिमभ्यद्रवत्क्रुद्धो गजेन्द्रमिव केसरी}


\twolineshloka
{रथाक्षमात्रैरिषुभिरभ्यवर्षद्धटोत्कचः}
{रथिनामृषभं द्रौणिं धाराभिरिव तोयदः}


\twolineshloka
{शरवृष्टिं तु तां प्राप्तां शरैराशीविषोपमैः}
{शातयामास समरे तरसा द्रौणिरुत्स्मयन्}


\twolineshloka
{ततः शरशतैस्तीक्ष्णैर्मर्मभेदिभिराशुगैः}
{समाचिनोद्राक्षसेन्द्रं घटोत्कचमरिन्दमम्}


\twolineshloka
{स शरैराचितस्तेन राक्षसो रणमूर्धनि}
{व्यकाशत महाराज श्वाविच्छललतो यथा}


\twolineshloka
{ततः क्रोधसमाविष्टो भैमसेनिः प्रतापवान्}
{शरैरवचकर्तोग्रैर्द्रौणिं वज्राशनिप्रभैः}


\twolineshloka
{क्षुरप्रैरर्धचन्द्रैश्च नाराचैः सशिलीमुखैः}
{वराहकर्णैर्नालीकैर्विकर्णैश्चाभ्यवीवृषत्}


\twolineshloka
{तां शस्त्रवृष्टिमतुलां वज्राशनिसमस्वनाम्}
{पतन्तीमुपरि क्रुद्धो द्रौणिरव्यथितेन्द्रियः}


\twolineshloka
{सुदुःसहां शरैर्घोरैर्दिव्यास्त्रप्रतिमन्त्रितैः}
{व्यधमत्सुमहातेजा महाभ्राणीव मारुतः}


\twolineshloka
{ततोऽन्तरिक्षे बाणानां सङ्ग्रामोऽन्य इवाभवत्}
{घोररूपो महाराज योधानां हर्षवर्धनः}


\twolineshloka
{ततोऽस्त्रसन्धर्षकृतैर्विस्फुलिङ्गैः समन्ततः}
{वमौ निशामुखे व्योम खद्योतैरिव संवृतम्}


% Check verse!
स मार्गणगणैर्द्रौणिर्दिशः प्रच्छाद्य सर्वतः.प्रियार्थं तव पुत्राणां राक्षसं समवाकिरत्
\twolineshloka
{ततः प्रववृते युद्धं द्रौणिराक्षसयोर्मृधे}
{विगाढे रजनीमध्ये शक्रप्रह्लादयोरिव}


\twolineshloka
{ततो घटोत्कचो बाणैर्दशभिर्द्रौणिमाहवे}
{जघानोरसि सङ्क्रुद्धः कालज्वलनसन्निभैः}


\twolineshloka
{स तैरभ्यायतैर्विद्धो राक्षसेन महाबलः}
{चचाल समरे द्रौणिर्वातनुन्न इव द्रुमः}


% Check verse!
स मोहं समरे प्राप्तो ध्वजयष्टिं समाश्रितः
\twolineshloka
{ततो हाहाकृतं सैन्यं तव सर्वं जनाधिप}
{हतं स्म मेनिरे सर्वे द्रोणसूनुं जनाधिपाः}


\twolineshloka
{तं तु दृष्ट्वा तथावस्थमश्वत्थामानमाहवे}
{पाञ्चालाः सृञ्जयाश्चैव सिंहनादं प्रचक्रिरे}


\twolineshloka
{प्रतिलभ्य ततः संज्ञामश्वत्थामा महाबलः}
{धनुः प्रपीड्य वामेन करेणामित्रकर्शनः}


\twolineshloka
{मुमोचाकर्णपूर्णेन धनुषा शरमुत्तमम्}
{यमदण़्डोपमं घोरमुद्दिश्याशु घटोत्कचम्}


\twolineshloka
{स भित्त्वा हृदयं तस्य राक्षसस्य शरोत्तमः}
{विवेश वसुधामुग्रः सपुङ्खः पृथिवीपते}


\twolineshloka
{सोऽतिविद्धो महाराज रथोपस्थ उपाविशत्}
{राक्षसेन्द्रः पुबलवान्द्रौणिना रणशालिना}


\twolineshloka
{दृष्ट्वा विमूढं हैडिम्बं सारथिस्तु रणाजिरात्}
{द्रौणेः सकाशात्सम्भ्रान्तस्त्वपनिन्येत्वरान्वितः}


\twolineshloka
{तथा तु समरे विद्ध्वा राक्षसेन्द्रं घटोत्कचम्}
{ननाद सुमहानादं द्रोणपुत्रो महारथः}


\twolineshloka
{पूजितस्तव पुत्रैश्च सर्वयोधैश्च भारत}
{वपुषाऽतिप्रजज्वाल मध्याह्न इव भास्करः}


\twolineshloka
{भीमसेनं तु युध्यन्तं भारद्वाजरथं प्रति}
{स्वयं दुर्योधनो राजा प्रत्यविध्यच्छितैः शरैः}


\twolineshloka
{तं भीमसेनो दशभिः शरैर्विव्याध मारिष}
{दुर्योधनोऽपि विंशत्या शराणां प्रत्यविध्यत}


\twolineshloka
{तौ सायकैरवच्छिन्नावदृश्येतां रणाजिरे}
{मेघजालसमाच्छन्नौ नभसीवेन्दुभास्करौ}


\twolineshloka
{अथ दुर्योधनो राजा भीमं विव्याध पत्रिभिः}
{पञ्चभिर्भरतश्रेष्ठ तिष्ठतिष्ठेति चाब्रवीत्}


\twolineshloka
{तस्य भीमो धनुश्छित्त्वा ध्वजं च दशभिः शरैः}
{विव्याध कौरवश्रेष्ठं नवत्या नतपर्वणाम्}


\threelineshloka
{ततो दुर्योधनः क्रुद्धो भनुरन्यन्महत्तरम्}
{गृहीत्वा भरतश्रेष्ठो भीमसेनं शितैः शरैः}
{अपीडयद्रणमुखे पश्यतां सर्वधन्विनाम्}


\twolineshloka
{तान्निहत्य शरान्भीमो दुर्योधनधनुश्च्युतान्}
{कौरवं पञ्चविंशत्या क्षुद्रकाणां समार्पयत्}


\twolineshloka
{दुर्योधनस्तु सङ्क्रुद्धो भीमसेनस्य मारिष}
{क्षुरप्रेण धनुश्छित्त्वा दशभिः प्रत्यविध्यत}


\twolineshloka
{अथान्यद्धनुरादाय भीमसेनो महाबलः}
{विव्याध नृपतिं तूर्णं सप्तभिर्निशितैः शरैः}


\twolineshloka
{तदप्यस्य धनुः क्षिप्रं चिच्छेद लघुहस्तवत्}
{द्वितीयं च तृतीयं च चतुर्थं पञ्चमं तथा}


\twolineshloka
{आत्तमात्तं महाराज भीमस्य धनुराच्छिनत्}
{तव पुत्रो महाराज जितकाशी मदोत्कटः}


\twolineshloka
{स तथा भिद्यमानेषु कार्मुकेषु पुनः पुनः}
{शक्तिं चिक्षेप समरे सर्वपारशवीं शुभाम्}


\twolineshloka
{मृत्योरिव स्वसारं हि दीप्तां वह्निशिखामिव}
{सीमन्तमिव कुर्वन्तीं नभसोऽग्निसमप्रभाम्}


\twolineshloka
{अप्राप्तामेव तां शक्तिं त्रिधा चिच्छेद कौरवः}
{पश्यतः सर्वलोकस्य भीमस्य च महात्मनः}


\twolineshloka
{ततो भीमो महाराज गदां गुर्वीं महाप्रभाम्}
{चिक्षेपाविध्य वेगेन दुर्योधनरथं प्रति}


\twolineshloka
{ततः सा सहसा वाहांस्तव पुत्रस्य संयुगे}
{सारथिं च गदा गुर्वी ममर्दास्य रथं पुनः}


\twolineshloka
{पुत्रस्तु तव राजेन्द्र भीमाद्भीतः प्रणश्य च}
{आरुरोह रथं चान्यं नन्दकस्य महात्मनः}


\twolineshloka
{ततो भीमो हतं मत्वा तव पुत्रं महारथम्}
{सिंहनादं महच्चक्रे तर्जयन्निशि कौरवान्}


\twolineshloka
{तावकाः सैनिकाश्चापि मेनिरे निहतं नृपम्}
{ततोऽतिचुक्रुशुः सर्वे ते हाहेति समन्ततः}


\twolineshloka
{तेषां तु निनदं श्रुत्वा त्रस्तानां सर्वयोधिनाम्}
{भीमसेनस्य नादं च श्रुत्वा राजन्महात्मनः}


\twolineshloka
{ततो युधिष्ठिरो राजा हतं मत्वा सुयोधनम्}
{अभ्यवर्तत वेगेन यत्र पार्थो वृकोदरः}


\twolineshloka
{पाञ्चलाः केकया मात्स्याः सृञ्जयाश्च विशाम्पते}
{सर्वोद्योगेनाभिजग्मुद्रोणमेव युयुत्सया}


\twolineshloka
{तत्रासीत्सुमहद्युद्धं द्रोणस्याथ परैः सह}
{घोरे तमसि मग्नानां निघ्नतामितरेतरम्}


\chapter{अध्यायः १६८}
\twolineshloka
{सञ्जय उवाच}
{}


\twolineshloka
{सहदेवमथायान्तं द्रोणप्रेप्तुं विशाम्पते}
{कर्णो वैकर्तनो युद्धे वारयामास भारत}


\twolineshloka
{सहदेवस्तु राधेयं विद्ध्वा नवभिराशुगैः}
{पुनर्विव्याध दशभिर्विशिखैर्नतपर्वभिः}


\twolineshloka
{तं कर्णः प्रतिविव्याध शतेन नतपर्वणाम्}
{सज्यं चास्य धनुः शीघ्रं चिच्छेद लघुहस्तवत्}


\twolineshloka
{ततोऽन्यद्धनुरादाय माद्रीपुत्रः प्रतापवान}
{कर्णं विव्याध विंशत्या तदद्भुतमिवाभवत्}


\twolineshloka
{तस्य कर्णो हयान्हत्वा शरैः सन्नतपर्वभिः}
{सारथिं चास्य भल्लेन द्रुतं निन्ये यमक्षयम्}


\twolineshloka
{विरथः सहदेवस्तु खङ्गं चर्म समाददे}
{तदप्यस्य शरैः कर्णो व्यधमत्प्रहसन्निव}


\twolineshloka
{अथ गुर्वीं महाघोरां हेमचित्रां महागदाम्}
{प्रेषयामास सङ्क्रुद्धो वैकर्तनरथं प्रति}


\twolineshloka
{तामापतन्तीं सहसा सहदेवप्रचोदिताम्}
{व्यष्टम्भयच्छरैः कर्णो भूमौ चैनामपातयत्}


\twolineshloka
{गदां विनिहतां दृष्ट्वा सहदेवस्त्वरान्वितः}
{शक्तिं चिक्षेप कर्णाय तामप्यस्याच्छिनच्छरैः}


\twolineshloka
{ससम्भ्रमं ततस्तूर्णमवप्लुत्य रथोत्तमात्}
{सहदेवो महाराज दृष्ट्वा कर्णं व्यवस्थितम्}


\threelineshloka
{रथचक्रं प्रगृह्याजौ मुमोचाधिरथं प्रति}
{तदापतद्वै सहसा कालचक्रमिवोद्यतम्}
{शरैरनेकसाहस्रैराच्छिनत्सूतनन्दनः}


\twolineshloka
{तस्मिंस्तु निहते चक्रे सूतजेन महात्मना}
{ईषादण्डकयोक्रांश्च युगानि विविधानि च}


\twolineshloka
{हस्त्यङ्गानि तथाश्वांश्च मृतांश्च पुरुषान्बहून्}
{चिक्षेप कर्णमुद्दिश्य कर्णस्तान्व्यधमच्छरैः}


\twolineshloka
{स निरायुधमात्मानं ज्ञात्वा माद्रवतीसुतः}
{वार्यमाणस्तु विशिखैः सहदेवो रणं जहौ}


\twolineshloka
{तमभिद्रुत्य राधेयो मुहूर्ताद्भरतर्षभ}
{अब्रवीत्प्रहसन्वाक्यं सहदेवं विशाम्पते}


\twolineshloka
{मा युध्यस्व रणेऽधीर विशिष्टै रथिभिः सह}
{}


% Check verse!
सदृशैर्युध्य माद्रेय वचो मे मा विशङ्किथाः ॥अथैनं धनुषोग्रेण तुदन्भूयोऽब्रवीद्वचः
\twolineshloka
{एषोऽर्जुनो रणे तूर्णं युध्यते कुरुभिः सह}
{तत्र गच्छस्व माद्रेय गृहं वा यदि मन्यसे}


\twolineshloka
{एवमुक्त्वा तु तं कर्णो रथेन रथिनां वरः}
{प्रायात्पाञ्चालपाण्डूनां सैन्यानि प्रदहन्निव}


\twolineshloka
{वधं प्राप्तं तु माद्रेयं नावधीत्समरेऽरिहा}
{कुन्त्याःस्मृत्वा वचो राजन्सत्यसन्धो महायशाः}


\twolineshloka
{सहदेवस्ततो राजन्विमनाः शरपीडितः}
{कर्णवाक्शरतप्तश्च जीवितान्निरविद्यत}


\twolineshloka
{आरुरोह रथं चापि पाञ्चाल्यस्य महात्मनः}
{जनमेजयस्य समरे त्वरायुक्तो महारथः}


\twolineshloka
{विराटं सहसेनं तु द्रोणार्थे द्रुतमागतम्}
{मद्रराजः शरौघेण च्छादयामास धन्विनम्}


\twolineshloka
{तयोः समभवद्युद्धं समरे दृढधन्विनोः}
{यादृशं ह्यभवद्राजञ्जम्भवासवयोः पुरा}


\twolineshloka
{मद्रराजो महाराज विराटं वाहिनीपतिम्}
{आजघ्ने त्वरितस्तूर्णं शतेन नतपर्वणाम्}


\twolineshloka
{प्रतिविव्याध तं राजन्नवभिर्निशितै शरैः}
{पुनश्चैनं त्रिसप्तत्या भूयश्चैव शतेन तु}


\twolineshloka
{तस्य मद्राधिपो हत्वा चतुरो रथवाजिनः}
{सूतं ध्वजं च समरे शराभ्यां सन्न्यपातयत्}


\twolineshloka
{हताश्वात्तुं रथात्तूर्णमवप्लुत्य महारथः}
{तस्थौ विष्फारयंश्चापं विमुञ्चंश्च शिताञ्छरान्}


\twolineshloka
{शतानीकस्ततो दृष्ट्वा भ्रातरं हतवाहनम्}
{रथेनाभ्यपतत्तूर्णं सर्वलोकस्य पश्यतः}


\twolineshloka
{शतानीकमथायान्तं मद्राराजो महामृधे}
{विशिखैर्बहुभिर्विद्ध्वा ततो निन्ये यमक्षयम्}


\twolineshloka
{तस्मिंस्तु निहते वीरे विराटो रथसत्तमः}
{आरुरोह रथं तूर्णं तमेव ध्वजमालिनम्}


\twolineshloka
{ततो विष्फार्य नयने क्रोधाद्द्विगुणविक्रमः}
{मद्रराजरथं तूर्णं छादयामास पत्रिभिः}


\twolineshloka
{ततो मद्राधिपः क्रुद्धः शरेणानतपर्वणा}
{आजघानोरसि दृढं विराटं वाहिनीपतिम्}


\threelineshloka
{सोऽतिविद्धो महाराज रथोपस्थ उपाविशत्}
{कश्मलं चाविशत्तीव्रं विराटो भरतर्षभ}
{सारथिस्तमपोवाह समरे शरविक्षतम्}


\twolineshloka
{ततः सा महती सेना प्राद्रवन्निशि भारत}
{वध्यमाना शरशतैः शल्येनाहवशोभिना}


\twolineshloka
{तां दृष्ट्वा विद्रुतां सेनां वासुदेवधनञ्जयौ}
{प्रयातौ तत्र राजेन्द्र यत्र शल्यो व्यवस्थितः}


\twolineshloka
{तौ तु प्रत्युद्ययौ राजन्राक्षसेन्द्रो ह्यलायुधः}
{अष्टचक्रसमायुक्तमास्थाय प्रवरं रथम्}


\threelineshloka
{तुरङ्गममुखैर्युक्तं पिशाचैर्घोरदर्शनैः}
{लोहितार्द्रपताकं तं रक्तमाल्यविभूषितम्}
{कार्ष्णायसमयं घोरमृक्षचर्मसमावृतम्}


\threelineshloka
{रौद्रेण चित्रपक्षेण विवृताक्षेण कूजता}
{ध्वजेनोच्छ्रितदण्डेन गृध्रराजेन राजता}
{स बभौ राक्षसो राजन्भिन्नाञ्जनचयोपमः}


\twolineshloka
{रुरोधार्जुनमायान्तं प्रभञ्जनमिवाद्रिराट्}
{किरन्बाणगणान्राजञ्शतशोऽर्जुनमूर्धनि}


\threelineshloka
{अतितीव्रं महाद्युद्धं नरराक्षसयोस्तदा}
{द्रष्टॄणां प्रीतिजननं सर्वेषां तत्र भारत}
{गृध्रकाकबलोलूककङ्कगोमायुहर्षणम्}


\twolineshloka
{तमर्जुनः शतेनैव पत्रिणां समताडयत्}
{नवभिश्च शितैर्बाणैर्ध्वजं चिच्छेद भारत}


\twolineshloka
{सारथिं च त्रिभिर्बाणैस्त्रिभिरेव त्रिवेणुकम्}
{धनुरेकेन चिच्छेद चतुर्भिश्चतुरो हयान्}


\twolineshloka
{पुनः सज्यं कृतं चापं तदप्यस्य द्विधाऽच्छिनत्}
{विरथस्योद्यतं खङ्गं शरेणास्य द्विधाऽकरोत्}


\twolineshloka
{अथैनं निशितैर्बाणैश्चतुर्भिर्भरतर्षभ}
{पार्थोऽविध्यद्राक्षसेन्द्रं स विद्वः प्राद्रवद्भयात्}


\twolineshloka
{तं विजित्यार्जुनस्तूर्णं द्रोणान्तिकमुपाययौ}
{किऱञ्शरगणान्राजन्नरवारणवाजिषु}


\twolineshloka
{वध्यमाना महाराज पाण्डवेन यशस्विना}
{सैनिका न्यपतन्नुर्व्यां वातनुन्ना इव द्रुमाः}


\twolineshloka
{तेषु तूत्साद्यमानेषु फल्गुनेन महात्मना}
{सम्प्राद्रवद्बलं सर्वं पुत्राणां ते विशाम्पते}


\chapter{अध्यायः १६९}
\twolineshloka
{सञ्जय उवाच}
{}


\twolineshloka
{शतानीकं शरैस्तूर्णं निर्दहन्तं चमूं तव}
{चित्रसेनस्तव सुतो वारयामास भारत}


\twolineshloka
{नाकुलिश्चित्रसेनं तु विद्ध्वा पञ्चभिराशुगैः}
{स तु तं प्रतिविव्याध दशभिर्निशितैः शरैः}


\twolineshloka
{चित्रसेनो महाराज शतानीकं पुनर्युधि}
{नवभिर्निशितैर्बाणैराजघान स्तनान्तरे}


\twolineshloka
{नाकुलिस्तस्य विशिखैर्वर्म सन्नतपर्वभिः}
{गात्रात्सञ्चावयामास तदद्भुतमिवाभवत्}


\twolineshloka
{सोपेतवर्मा पुत्रस्ते विरराज भृशं नृप}
{उत्सृज्य काले राजेन्द्र निर्मोकमिव पन्नगः}


\twolineshloka
{ततोऽस्य निशितैर्बाणैर्ध्वजं चिच्छेद नाकुलिः}
{धनुश्चैव महाराज यतमानस्य संयुगे}


\twolineshloka
{स च्छिन्नधन्वा समरे विवर्मा च महारथः}
{धनुरन्यन्महाराज जग्राहारिविदारणम्}


\twolineshloka
{ततस्तूर्णं चित्रसेनो नाकुलिं नवभिः शरैः}
{विव्याध समरे क्रुद्धो भरतानां महारथः}


\twolineshloka
{शतानीकोऽथ सङ्क्रुद्धश्चित्रसेनस्य मारिष}
{जघान चतुरो वाहान्सारथिं च नरोत्तमः}


\twolineshloka
{अवप्लुत्य रथात्तस्माच्चित्रसेनो महारथः}
{नाकुलिं पञ्चविंशत्या शराणामार्दयद्बली}


\twolineshloka
{तस्य तत्कुर्वतः कर्म नकुलस्य सुतो रणे}
{अर्धचन्द्रेण चिच्छेद चापं रत्नविभूषितम्}


\twolineshloka
{स च्छिन्नधन्वा विरथो हताश्वो हतसारथिः}
{आरुरोह रथं तूर्णं हार्दिक्यस्य महात्मनः}


\twolineshloka
{द्रुपदं तु सहानीकं द्रोणप्रेप्सुं महारथम्}
{वृषसेनोऽभ्ययात्तूर्णं किरञ्शरशतैस्तदा}


\twolineshloka
{यज्ञसेनस्तु समरे कर्णपुत्रं महारथम्}
{षष्ट्या शराणां विव्याध बाह्वोरुरसि चानघ}


\twolineshloka
{वृषसेनस्तु सङ्क्रुद्धो यज्ञसेनं रथे स्थितम्}
{बहुभिः सायकैस्तीक्ष्णैराजघान स्तनान्तरे}


\twolineshloka
{तावुभौ शरनुन्नाङ्गौ शरकण्टकितौ रणे}
{व्यभ्राजेतां महाराज श्वाविधौ शललैरिव}


\twolineshloka
{रुक्मपुङ्खैः प्रसन्नाग्रैः शरैश्छिन्नतनुच्छदौ}
{रुधिरौघपरिक्लिन्नौ व्यभ्राजेतां महामृधे}


\twolineshloka
{तपनीयनिभौ चित्रौ कल्पवृक्षाविवाद्भुतौ}
{किंशुकाविव चोत्फुल्लौ व्यकाशेतां रणाजिरे}


\twolineshloka
{वृषसेनस्ततो राजन्द्रुपदं नवभिः शरैः}
{विद्ध्वा विव्याध सप्तत्या पुनरन्यैस्त्रिभिस्त्रिभिः}


\twolineshloka
{ततः शरसहस्राणि विमुञ्चन्विबभौ तदा}
{कर्णपुत्रो महाराज वर्षमाण इवाम्बुदः}


\twolineshloka
{[द्रुपदस्तु ततः क्रुद्धो वृषसेनस्य कार्मुकम्}
{द्विधा चिच्छेद भल्लेन पीतेन निशितेन च}


\twolineshloka
{सोऽन्यत्कार्मुकमादाय रुक्मबद्धं नवं दृढम्}
{तूणादाकृष्य विमलं भल्लं पीतं शितं दृढम्}


\twolineshloka
{कार्मुके योजयित्वा तं द्रुपदं सन्निरीक्ष्य च}
{आकर्णपूर्णं मुमुचे त्रासयन्सर्वसोमकान्}


\twolineshloka
{हृदयं तस्य भित्त्वा च जगाम वसुधातलम्}
{कश्मलं प्राविशद्राजा वृषसेनशराहतः}


\twolineshloka
{सारथिस्तमपोवाह स्मरन्सारथिचेष्टितम्}
{तस्मिन्प्रभग्ने राजेन्द्र पाञ्चालानां महारथे ॥]}


\twolineshloka
{ततस्तु द्रुपदानीकं शरैश्छिन्नतनुच्छदम्}
{सम्प्राद्रवद्रणाद्राजन्निशीथे भैरवे सति}


\twolineshloka
{प्रदीपैरपरित्यक्तैर्ज्वलद्भिस्तैः समन्ततः}
{व्यराजद्वसुधा तत्र वीताभ्रा द्यौरिव ग्रहैः}


\twolineshloka
{तथाङ्गदैर्निपतितैर्व्यराजत वसुन्धरा}
{प्रावृट््काले महाराज विद्युद्भिरिव तोयदः}


\twolineshloka
{ततः कर्णसुतात्त्रस्ताः सोमका विप्रदुद्रुवुः}
{यथेन्द्रभयवित्रस्ता दानवास्तारकामये}


\twolineshloka
{तेनार्द्यमानाः समरे द्रवमाणाश्च सोमकाः}
{व्यराजन्त महाराज प्रदीपैरवभासिताः}


\twolineshloka
{तांस्तु निर्जित्य समरे कर्णपुत्रो व्यरोचत}
{मध्यन्दिनमनुप्राप्तो घर्मांशुरिव भारत}


\threelineshloka
{तेषु राजसहस्रेषु तावकेषु परेषु च}
{एक एव ज्वलंस्तस्थौ वृषसेनः प्रतापवान्}
{}


\twolineshloka
{स विजित्य रणे शूरान्सोमकानां महारथान्}
{जगाम त्वरितस्तत्र यत्र राजा युधिष्ठिरः}


\twolineshloka
{प्रतिविन्ध्यमथ क्रुद्धं प्रदहन्तं रणे रिपून्}
{दुःशासनस्तव सुतः प्रत्यगच्छन्महारथः}


\twolineshloka
{तयोः समागमो राजंश्चित्ररूपो बभूव ह}
{व्यपेतजलदे व्योम्नि बुधभास्करयोरिव}


\threelineshloka
{प्रतिविन्ध्यं तु समरे कुर्वाणां कर्म दुष्करम्}
{`अपूजयन्महाराज तव सैन्ये महारथाः'}
{दुःशासनस्त्रिभिर्बाणैर्ललाटे समविध्यत}


\twolineshloka
{सोऽतिविद्धो बलवता तव पुत्रेण धन्विना}
{विरराज महाबाहुः सशृङ्ग इव पर्वतः}


\twolineshloka
{दुःशासनं तु समरे प्रतिविन्ध्यो महारथः}
{नवभिः सायकैर्विद्धा पुनर्विव्याध सप्तभिः}


\twolineshloka
{तत्र भारत पुत्रस्ते कृतवान्कर्म दुष्करम्}
{प्रतिविन्ध्यहयानुग्रैः पातयामास सायकैः}


\twolineshloka
{सारथिं चास्य भल्लेन ध्वजं च समपातयत्}
{रथं च तिलशो राजन्व्यधमत्तस्य धन्विनः}


\twolineshloka
{पताकाश्च सतूणीरा रश्मीन्योक्राणि च प्रभो}
{चिच्छेद तिलशः क्रुद्धःशरैः सन्नतपर्वभिः}


\twolineshloka
{विरथः स तु धर्मात्मा धनुष्पाणिरवस्थितः}
{अयोधयत्तव सुतं किरञ्शरशतान्बहून्}


\twolineshloka
{क्षुरप्रेण धनुस्तस्य चिच्छेद तनयस्तव}
{अथैनं दशभिर्बाणेश्छिन्नधन्वानमार्दयत्}


\twolineshloka
{तं दृष्ट्वा विरथं तत्र भ्रातरोऽस्य महारथाः}
{अन्ववर्तन्त सङ्ग्रामे महत्या सेनया सह}


\twolineshloka
{आप्लुतः स ततो यानं सुतसोमस्य भास्वरम्}
{धनुर्गृह्य महाराज विव्याध तनयं तव}


\twolineshloka
{ततस्तु तावकाः सर्वे परिवार्य सुतं तव}
{अभ्यवर्तन्त वेगेन महत्या सेनया वृताः}


\twolineshloka
{ततः प्रववृते युद्धं तव तेषां च भारत}
{निशीथे दारुणे काले यमराष्ट्रविवर्धनम्}


\chapter{अध्यायः १७०}
\twolineshloka
{सञ्चय उवाच}
{}


\twolineshloka
{नकुलं रभसं युद्धे निघ्नन्तं वाहिनीं तव}
{अभ्ययात्सौबलः क्रुद्धस्तिष्ठतिष्ठेति चाब्रवीत्}


\twolineshloka
{कृतवैरौ तु तौ वीरावन्योन्यवधकाङ्क्षिणौ}
{शरैः पूर्णायतोत्सृष्टैरन्योन्यमभिजघ्नतुः}


\twolineshloka
{यथैव नकुलो राजञ्शरवर्षाण्यमुञ्चत}
{तथैव सौबलश्चापि शिक्षां सन्दर्शयन्युधि}


\twolineshloka
{तावुभौ समरे शूरौ शरकण्टकिनौ तदा}
{व्यराजेतां महाराज श्वाविधौ शललैरिव}


\twolineshloka
{रुक्मपुङ्खैरजिह्माग्रैः शरैश्छिन्नतनुच्छदौ}
{रुधिरौघपरिक्लिन्नौ व्यभ्राजेतां महामृधे}


\twolineshloka
{तपनीयनिभौ चित्रो कल्पवृक्षाविव द्रुमौ}
{किंशुकाविव चोत्फुल्लौ प्रकाशेते रणाजिरे}


\twolineshloka
{तावुभौ समरे शूरौ शरकण्टकिनौ तदा}
{व्यराजेतां महारान्कण्टकैरिव शाल्मली}


\twolineshloka
{सुजिह्मं प्रेक्षमाणौ च राजन्विवृतलोचनौ}
{क्रोधसंरक्तनयनौ निर्दहन्तौ परस्परम्}


\twolineshloka
{श्यालस्तु तव सङ्क्रुद्धो माद्रीपुत्रं हसन्निव}
{कर्णिनैकेन विव्याध हृदये निशितेन ह}


\twolineshloka
{नकुलस्तु भृशं विद्वः श्यालेन तव धन्विना}
{निषसाद रथोपस्थे कश्मलं चाविशन्महत्}


\twolineshloka
{अत्यन्तवैरिणां दृप्तं दृष्ट्वा शत्रुं तथाऽऽगतम्}
{ननाद शकुनी राजंस्तपान्ते जलदो यथा}


\twolineshloka
{प्रतिलभ्य ततः संज्ञां नकुलः पाण्डुनन्दनः}
{अभ्ययात्सौबलं भूयो व्यात्तानन इवान्तकः}


\twolineshloka
{सङ्क्रुद्धः शकुनिं षष्ट्या विव्याध भरतर्षभ}
{पुनश्चैनं शतेनैव नाराचानां स्तनान्तरे}


\twolineshloka
{अथास्य सशरं चापं मृष्टिदेशेऽच्छिनत्तदा}
{ध्वजं च त्वरितं छित्त्वा रथाद्भूमावपातयत्}


\threelineshloka
{विशिखेन च तीक्ष्णेन पीतेन निशितेन च}
{ऊरू निर्भिद्य चैकेन नकुलः पाण्डुनन्दनः}
{श्येनं सपक्षं व्याधेन पातयामास तं तदा}


\twolineshloka
{सोऽतिविद्धो महाराज रथोपस्थ उपाविशत्}
{ध्वजयष्टिं परिक्लिश्य कामुकः कामिनीं यथा}


\twolineshloka
{तं विसंज्ञं निपतितं दृष्ट्वा श्यालं तवानघ}
{अपोवाह रथेनाशु सारथिर्ध्वजिनीमुखात्}


% Check verse!
ततः सञ्चुक्रुशुः पार्था ये च तेषां पदानुगाः
\twolineshloka
{निर्जित्य च रणे शत्रुं नकुलः शत्रुतापनः}
{अब्रवीत्सारथिं क्रुद्धो द्रोणानीकायं मां वह}


\twolineshloka
{तस्य तद्वचनं श्रुत्वा माद्रीपुत्रस्य सारथिः}
{प्रायात्तेन तदा राजन्यत्र द्रोणो व्यवस्थितः}


\twolineshloka
{शिखण्डिनं तु समरे द्रोणप्रेप्सुं विशाम्पते}
{कृपः शारद्वतो यत्तः प्रत्यगच्छत्स वेगितः}


\twolineshloka
{गौतमं द्रुतमायान्तं द्रोणानीकमरिन्दमम्}
{विव्याध नवभिर्भल्लैः शिखण्डी प्रहसन्निव}


% Check verse!
तमाचार्यो महाराज विद्धा पञ्चभिराशुगैः ॥पुनर्विव्याध विंशत्या पुत्राणां प्रियकृत्तव
% Check verse!
महद्युद्धं तयोरासीद्धोररूपं भयानकम् ॥यथा देवासुरे युद्धे शम्बरामरराजयोः
\threelineshloka
{शरजालावृतं व्योम चक्रतुस्तौ महारथौ}
{मेघाविव तपापाये वीरौ समरदुर्मदौ}
{प्रकृत्या घोररूपं तदासीद्धोरतरं पुनः}


\twolineshloka
{रात्रिश्च भरतश्रेष्ठ योधानां युद्धशालिनाम्}
{कालरात्रिनिभा ह्यासीद्धोररूपा भयानका}


\twolineshloka
{शिखण्डी तु महाराज गौतमस्य महद्धनुः}
{अर्धचन्द्रेण चिच्छेद सज्यं सविशिखं तदा}


\twolineshloka
{तस्य क्रुद्धः कृपो राजञ्शक्तिं चिक्षेप दारुणाम्}
{स्वर्णदण्डामकुण्ठाग्रां कर्मारपरिमार्जिताम्}


\twolineshloka
{तामापतन्तीं चिच्छेद शिखण्डी बहुभिः शरैः}
{सापतन्मेदिनीं कृत्ता भासयन्ती महाप्रभा}


\twolineshloka
{अथान्यद्धनुरादाय गौतमो रथिनां वारः}
{प्राच्छादयच्छितैर्बाणैर्महाराज शिखण्डिनम्}


\twolineshloka
{स च्छाद्यमानः समरे गौतमेन यशस्विना}
{न्यषीदत रथोपस्थे शिखण्डी रथिनां वरः}


\twolineshloka
{सीदन्तं चैनमालोक्य कृपः शारद्वतो युधि}
{आजघ्ने बहुभिर्बाणैर्जिघांसन्निव भारत}


\twolineshloka
{विमुखं तु रणे दृष्ट्वा याज्ञसेनिं महारथम्}
{पाञ्चालाः सोमकाश्चैव परिवव्रुः समन्ततः}


\twolineshloka
{तथैव तव पुत्राश्च परिवव्रुर्द्विजोत्तमम्}
{महत्या सेनया सार्धं ततो युद्धमवर्तत}


\twolineshloka
{रथानां च रणे राजन्नन्योन्यमभिधावताम्}
{बभूव तुमुलः शब्दो मेघानां गर्जतामिव}


\twolineshloka
{द्रवतां सादिनां चैव गजानां च विशाम्पते}
{अन्योन्यमभितो राजन्क्रूरमायोधनं बभौ}


\twolineshloka
{पत्तीनां द्रवतां चैव पादशब्देन मेदिनी}
{अकम्पत महाराज भयत्रस्तेव चाङ्गना}


\twolineshloka
{रथिनो रथमारुह्य प्रद्रुता वेगवत्तरम्}
{अगृह्णन्बहवो राजञ्शलभान्वायसा इव}


\twolineshloka
{तथा गजान्प्रभिन्नांश्च सम्प्रभिन्ना महागजाः}
{तस्मिन्नेव पदे यत्ता निगृह्णन्ति स्म भारत}


\twolineshloka
{सादी सादिनमासाद्य पत्तयश्च पदातिनम्}
{समासाद्य रणेऽन्योन्यं संरब्धा नातिचक्रमुः}


\twolineshloka
{धावतां द्रवतां चैव पुनरावर्ततामपि}
{बभूव तत्र सैन्यानां शब्दः सुविपुलो निशि}


\twolineshloka
{दीप्यमानाः प्रदीपाश्च रथवारणवाजिषु}
{अदृश्यन्त महाराज महोल्का इव खाच्च्युताः}


\twolineshloka
{सा निशा भरतश्रेष्ठ प्रदीपैरवभासिता}
{दिवसप्रतिमा राजन्बभूव रणमूर्धनि}


\twolineshloka
{आदित्येन यथा व्याप्तं तमो लोके प्रणश्यति}
{तथा नष्टं तमो घोरं दीपैर्दीप्तैरितस्ततः}


\twolineshloka
{दिवं च पृथिवीं चैव दिशश्च प्रदिशस्तथा}
{रजसा तमसा व्याप्ता द्योतिताः प्रभया पुनः}


\twolineshloka
{अस्त्राणां कवचनानां च मणीनां च महात्मनाम्}
{अन्तर्दधुः प्रभाः सर्वा दीपैस्तैरवभासिताः}


\twolineshloka
{तस्मिन्कोलाहले युद्धे वर्तमाने निशामुखे}
{न किञ्चिद्विदुरात्मानमयमस्मीति भारत}


\threelineshloka
{अवधीत्समरे पुत्रं पिता भरतसत्तम}
{पुत्रश्च पितरं मोहात्सखायं च सखा तथा}
{स्वस्रीयं मातुलश्चापि स्वस्रीयश्चापि मातुलम्}


\twolineshloka
{स्वे स्वान्परे परांश्चापि निजघ्नुरितरेतरम्}
{निर्मर्यादमभूद्युद्धं रात्रौ भीरुभयानकम्}


\chapter{अध्यायः १७१}
\twolineshloka
{सञ्जय उवाच}
{}


\twolineshloka
{तस्मिन्सुतुमुले युद्धे वर्तमाने भयावहे}
{धृष्टद्युम्नो महाराज द्रोणमेवाभ्यवर्तत}


\twolineshloka
{सन्दधानो धनुःश्रेष्ठं ज्यां विकुर्षन्पुनः पुनः}
{अभ्यद्रवत द्रोणस्य रथं रुक्मविभूषितम्}


\twolineshloka
{धृष्टद्युम्नमथायान्तं द्रोणस्यान्तचिकीर्षया}
{परिवव्रुर्महाराज पाञ्चालाः पाण्डवैः सह}


\twolineshloka
{तथा परिवृतं दृष्ट्वा द्रोणमाचार्यसत्तमम्}
{पुत्रास्ते सर्वतो यत्ता ररक्षुर्द्रोणमाहवे}


\twolineshloka
{बलार्णवौ ततस्तौ तु समेयातां निशामुखे}
{गातोद्धूतौ क्षुब्धसत्वौ भैरवौ सागराविव}


\twolineshloka
{ततो द्रोणं महाराज पाञ्चाल्यः पञ्चभिः शरैः}
{विव्याध हृदये तूर्णं सिंहनादं ननाद च}


\twolineshloka
{तं द्रोणः पञ्चविंशत्या विद्धा भारत संयुगे}
{चिच्छेदान्येन भल्लेन धनुरस्य महास्वनम्}


\twolineshloka
{धृष्टद्युम्नस्तु निर्विद्धो द्रोणेन भरतर्षभ}
{उत्ससर्ज धनुस्तूर्णं सन्दश्य दशनच्छदम्}


\twolineshloka
{ततः क्रुद्धो महाराज धृष्टद्युम्नः प्रतापवान्}
{आददेऽन्यद्धनुःश्रेष्ठं द्रोणस्यान्तचिकीर्षया}


\twolineshloka
{विकृष्य च धनुश्चित्रमाकर्णात्परवीरहा}
{द्रोणस्यानत्करं घोरं व्यसृजत्सायकं ततः}


\twolineshloka
{स विसृष्टो बलवता शरो घोरो महामृधे}
{भासयामास तत्सैन्यं दिवाकर इवोदितः}


\twolineshloka
{तं तु दृष्ट्वा शरं घोरं देवगन्धर्वमानवाः}
{स्वस्त्यस्तु समरे राजन्द्रोणायेत्यब्रुवन्वचः}


\twolineshloka
{तं तु सायकमायान्तमाचार्यस्य रथं प्रति}
{कर्णो द्वादशधआ राजंश्चिच्छेद कृतहस्तवत्}


\twolineshloka
{स च्छिन्नो बहुधा राजन्सूतपुत्रेण धन्विना}
{निपपात शरस्तूर्णं निर्विषो भुजगो यथा}


\twolineshloka
{`छित्त्वा तु समरे बाणं शरैः सन्नतपर्वभिः'}
{धृष्टद्युम्नं ततः कर्णो विव्याध दशभिः शरैः}


\twolineshloka
{पञ्चभिर्द्रोणपुत्रस्तु स्वयं द्रोणस्तु सप्तभिः}
{शल्यश्च दशभिर्बाणैस्त्रिभिर्दुःशासनस्तथा}


\twolineshloka
{दुर्योधनस्तु विंशत्या शकुनिश्चापि पञ्चभिः}
{पाञ्चाल्यं त्वरयाऽविध्यन्सर्व एव महारथाः}


\threelineshloka
{स विद्धः सप्तभिर्घोरैर्द्रोणस्यार्थे महाहवे}
{सर्वानसम्भ्रमाद्राज प्रत्यविद्ध्यत्त्रिभिस्त्रिभिः}
{द्रोणं द्रौणिं च कर्णं च विव्याध च तवात्मजम्}


\twolineshloka
{ते भिन्ना धन्विना तेन धृष्टद्युम्नं पुनर्मृधे}
{विव्यधुः पञ्चभिस्तूर्णमेकैको रथिनां वरः}


\twolineshloka
{द्रुमसेनस्तु सङ्क्रुद्धो राजन्विव्याध पत्रिणा}
{त्रिभिश्चान्यैः शरैस्तूर्णं तिष्ठ तिष्ठेति चाब्रवीत्}


\twolineshloka
{स तु तं प्रतिविव्याध त्रिभिस्तीक्ष्णैरजिह्मगैः}
{स्वर्णपुङ्खैः शिलाधौतैः प्राणान्तकरमैर्युधि}


\twolineshloka
{भल्लेनान्येन तु पुनः सुवर्णोज्ज्वलकुण्डलम्}
{निचकर्त शिरः कायाद्द्रुमसेनस्य वीर्यवान्}


\twolineshloka
{तच्छिरो न्यपतद्भूमौ सन्दष्टौष्ठपुटं रणे}
{महावातसमुद्धूतं पक्वं तालफलं यथा}


\twolineshloka
{तान्स विद्धा पुनर्योधान्वीरः सुनिशितैः शरैः}
{राधेयस्याच्छिनद्भल्लैः कार्मुकं चित्रयोधिनः}


\twolineshloka
{न तु तन्ममृषे कर्णो धनुषश्छेदनं तथा}
{निकर्तनमिवात्युग्रं लाङ्गूलस्य महाहरिः}


\twolineshloka
{सोऽन्यद्धनुः समादाय क्रोधरक्तेक्षणः श्वसन्}
{अभ्यद्रवच्छरौघैस्तं धृष्टद्युम्नं महाबलम्}


\twolineshloka
{दृष्ट्वा कर्णं तु संरब्धं ते वीराः षड्रथर्षभाः}
{पाञ्चाल्यपुत्रं त्वरिताः परिवव्रुर्जिघांसया}


\twolineshloka
{षण्णां योधप्रवीराणां तावकानां पुरस्कृतम्}
{मृत्योरास्यमनुप्राप्तं धृष्टद्युम्नमममंस्महि}


\twolineshloka
{पतस्मिन्नेव काले तु दाशार्हो विकिरञ्छरान्}
{धृष्टद्युम्नं पराक्रान्तं सात्यकिः प्रत्यपद्यत}


\twolineshloka
{तमायान्तं महेष्वासं सात्यकिं युद्धदुर्मदम्}
{राधेयो दशभिर्बाणैः प्रत्यविध्यदजिह्मगैः}


\twolineshloka
{तं सात्यकिर्महाराज विव्याध दशभिः शरैः}
{पश्यतां सर्ववीराणां मा गास्तिष्ठेति चाब्रवीत्}


\twolineshloka
{स सात्यकेस्तु बलिनः कर्णस्य च महात्मनः}
{आसीत्समागमो राजन्बलिवासवयोरिव}


\twolineshloka
{त्रासयन्रथघोषेण क्षत्रियान्क्षत्रियर्षभः}
{राजीवलोचनं कर्णं सात्यकिः प्रत्यविध्यत}


\twolineshloka
{कम्पयन्निव घोषेण धनुषो वसुधां बली}
{सूतपुत्रो महाराज सात्यकिं प्रत्ययोधयत्}


\twolineshloka
{विपाठकर्णिनाराचैर्वत्सदन्तैः क्षुरैरपि}
{कर्णः शरशतैश्चापि शैनेयं प्रत्यनिद्व्यत}


\twolineshloka
{तथैव युध्यमानोऽपि वृष्णीनां प्रवरो युधि}
{अभ्यवर्षच्छरैः कर्णं तद्युद्धमभवत्समम्}


\twolineshloka
{तावकाश्च महाराज कर्णपुत्रश्च दंशितः}
{सात्यकिं विव्यधुस्तूर्ण समन्तान्निशितैः शरैः}


\twolineshloka
{अस्त्रैरस्त्राणि संवार्य तेषां कर्णस्य वा विभो}
{अविद्व्यत्सात्यकिः क्रुद्धो वृषसेनं स्तनान्तरे}


\twolineshloka
{तेन बाणेन निर्विद्धो वृषसेनो विशाम्पते}
{न्यपतत्स रथे मूढो धनुरुत्सृज्य वीर्यवान्}


\twolineshloka
{ततः कर्णो हतं मत्वा वृषसेनं महारथम्}
{पुत्रशोकाभिसन्तप्तः सात्यकिं प्रत्यपीडयत्}


\twolineshloka
{पीड्यमानस्तु कर्णेन युयुधानो महारथः}
{विव्याध बहुभिः कर्णं त्वरमाणः पुनःपुनः}


\twolineshloka
{स कर्णं दशभिर्विद्ध्वा वृषसेनं च सप्तभिः}
{स हस्तावापधनुषी तयोश्चिच्छेद सात्वतः}


% Check verse!
तावन्ये धनुषी सज्जे कृत्वा शत्रुभयङ्करे ॥युयुधानमाविध्येतां समन्तान्निशितैः शरैः
\twolineshloka
{वर्तमाने तु सङ्ग्रमे तस्मिन्वीरवरक्षये}
{अतीव शुश्रुवे राजन्गाण्डीवस्य महास्वनः}


\twolineshloka
{श्रुत्वा तु रथनिर्घोषं गाण्डीवस्य च निःस्वनम्}
{सूतपुत्रोऽब्रवीद्राजन्दुर्योधनमिदं वचः}


\twolineshloka
{एष सर्वां चमूं हत्वा मुख्यांश्चैव नरर्षभान्}
{पौरवांश्च महेष्वासो विक्षिपन्नुत्तमं धनुः}


\twolineshloka
{पार्थो विजयते तत्र गाण्डीवनिनदो महान्}
{श्रूयते रथघोषश्च वासवस्येव नर्दतः}


\twolineshloka
{करोति पाण्डवो व्यक्तं कर्मौपयिकमात्मनः}
{एषा विदार्यते राजन्बहुधा भारती चमूः}


\twolineshloka
{विप्रकीर्णान्यनीकानि न हि तिष्ठन्ति कर्हिचित्}
{वातेनेव समुद्धूतमभ्रजालं विदीर्यते}


% Check verse!
सव्यसाचिनमासाद्य भिन्ना नौरिव सागरे
\twolineshloka
{द्रवतां योधमुख्यानां गाण्डीवप्रेषितैः शरैः}
{विद्धानां शतशो राजञ्श्रूयते निःस्वनो महान्}


\twolineshloka
{शृणु दुन्दुभिनिर्घोषमर्जुनस्य रथं प्रति}
{निशीथे राजशार्दूल स्तनयित्नोरिवाम्बरे}


\twolineshloka
{हाहाकाररवांश्चैव सिंहनादांश्च पुष्कलान्}
{शृणु शब्दान्बहुविधानर्जुनस्य रथं प्रति}


\twolineshloka
{अयं मध्ये स्थितोऽस्माकं सात्यकिः सात्वतां वरः}
{इह चेल्लभ्यते लक्ष्यं कृत्स्नाञ्जेष्यामहे परान्}


\twolineshloka
{एष पाञ्चालराजस्य पुत्रो द्रोणेन सङ्गतः}
{सर्वतः संवृतो योधैः शूरैश्च रथसत्तमैः}


\twolineshloka
{सात्यकिं यदि हन्याम धृष्टद्युम्नं च पार्षतम्}
{असंशयं महाराज ध्रुवो नो विजयो भवेत्}


\twolineshloka
{सौभद्रवदिमौ वीरौ परिवार्य महारथौ}
{प्रयतायो महाराज निहन्तुं वृष्णिपार्षतौ}


\twolineshloka
{सव्यसाची पुरोऽभ्येति द्रोणानीकाय भारत}
{संसक्तं सात्यकिं ज्ञात्वा बहुभिः कुरुपुङ्गवैः}


\twolineshloka
{तत्र गच्छन्तु बहवः प्रवरा रथसत्तमाः}
{यावत्पार्थो न जानाति सात्यकिं बहुभिर्वृतम्}


\threelineshloka
{ते त्वरध्वं तथा शूराः शराणां मोक्षणे भृशम्}
{यथा त्विह व्रजत्येष परलोकाय माधवः}
{तथा कुरु महाराज सुनीत्या सुप्रयुक्तया}


\twolineshloka
{कर्णस्य मतमास्थाय पुत्रस्ते प्राह सौबलम्}
{यथेन्द्रः समरे राजन्प्राह विष्णुं यशस्विनम्}


\threelineshloka
{वृतः सहस्रैर्दशभिर्गजानामनिवर्तिनाम्}
{रथैश्च दशसाहस्रैस्तूर्णं याहि धनञ्जयम्}
{}


\twolineshloka
{दुःशासनो दुर्विषहः सुबाहुर्दुः प्रधर्षणः}
{एते त्वामनुयास्यन्ति पत्तिभिर्बहुभिर्वृताः}


\twolineshloka
{जहि कृष्णौ महाबाहो धर्मराजं च मातुल}
{नकुलं सहदेवं च भीमसेनं तथैव च}


\twolineshloka
{देवानामिव देवेन्द्रे जयाशा त्वयि मे स्थिता}
{जहि मातुल कौन्तेयानसुरानिव पावकिः}


\twolineshloka
{एवमुक्तो ययौ पार्थान्पुत्रेण तव सौबलः}
{महत्या सेनया सार्धं सह पुत्रैश्च ते विभो}


\twolineshloka
{प्रियार्थं तव पुत्राणां दिधक्षुः पाण्डुनन्दनान्}
{ततः प्रववृते युद्धं तावकानां परैः सह}


\twolineshloka
{प्रयाते सौबले राजन्पाण्डवानामनीकिनीम्}
{बलेन महता युक्ताः सूतपुत्रस्तु सात्वतम्}


\twolineshloka
{अभ्ययात्त्वरितो युद्धे किरञ्शरशतान्बहून्}
{तथैव पार्थिवाः सर्वे सात्यकिं पर्यवारयन्}


\threelineshloka
{भारद्वाजस्ततो गत्वा धृष्टद्युम्नरथं प्रति}
{महद्युद्धं तदासीत्तुं द्रोणस्य निशि भारत}
{धृष्टद्युम्नेन वीरेण पाञ्चालैश्च सहाद्भुतम्}


\chapter{अध्यायः १७२}
\twolineshloka
{सञ्जय उवाच}
{}


\twolineshloka
{ततस्ते प्राद्रवन्सर्वे त्वरिता युद्धदुर्मदाः}
{अमृष्यमाणाः संरब्धा युयुधानरथं प्रति}


\twolineshloka
{ते रथैः कल्पितै राजन्हेमरूप्यविभूषितैः}
{सादिभिश्च गजैश्चैव परिवव्रुः समन्ततः}


\twolineshloka
{अथैनं कोष्ठकीकृत्य सर्वतस्ते महारथाः}
{सिंहनादांस्ततश्चक्रुस्तर्जयन्ति स्म सात्यकिम्}


\twolineshloka
{तेऽभ्यवर्षञ्छरैस्तीक्ष्णैः सात्यकिं सत्यविक्रमम्}
{त्वरमाणा महावीरा माधवस्य वधैषिणः}


\twolineshloka
{तान्दृष्ट्वा पततस्तूर्णं शैनेयः परवीरहा}
{प्रत्यगृह्णान्महाबाहुः प्रमुञ्चन्विशिखान्बहून्}


\twolineshloka
{तत्र वीरो महेष्वासः सात्यकिर्युद्धदुर्मदः}
{निचकर्त शिरांस्युग्रैः शरैः सन्नतपर्वभिः}


\twolineshloka
{हस्तिहस्तान्हयग्रीवा बाहूनपि च सायुधान्}
{क्षुरप्रैः शातयामास तावकानां स माधवः}


\twolineshloka
{पतितैश्चामरैश्चैव श्वेतच्छत्रैश्च भारत}
{बभूव धरणी पूर्णा नक्षत्रैर्द्यौरिव प्रभो}


\twolineshloka
{एतेषां युयुधानेन युध्यतां युधि भारत}
{बभूव तुमुलः शब्दः प्रेतानां क्रन्दतामिव}


\twolineshloka
{तेन शब्देन महता पूरिताऽभूद्वसुन्धरा}
{रात्रिः समभवच्चैव तीव्ररूपा भयावहा}


\twolineshloka
{दीर्यमाणं बलं दृष्ट्वा युयुधानशराहतम्}
{श्रुत्वा च विपुलं नादं निशीथे रोमहर्षणे}


\twolineshloka
{सुतस्तवाब्रवीद्राजन्सारथिंरथिनां वरः}
{यत्रैष शब्दस्तत्राश्वांश्चोदयेति पुनःपुनः}


\twolineshloka
{तेन सञ्चोद्यमानस्तु ततस्तांस्तुरगोत्तमान्}
{सूतः सञ्चोदयामास युयुधानरथं प्रति}


\twolineshloka
{ततो दुर्योधनः क्रुद्धो दृढधन्वा जितक्लमः}
{शीघ्रहस्तश्चित्रयोधी युयुधानमुपाद्रवत्}


\twolineshloka
{ततः पूर्णायतोत्सृष्टैः शरैः शोणितभोजनैः}
{दुर्योधनं द्वादशभिर्माधवः प्रत्यविध्यत}


\twolineshloka
{दुर्योधनस्तेन तथा पूर्वमेवार्दितः शरैः}
{शैनेयं दशभिर्बाणैः प्रत्यविध्यदमर्षितः}


\twolineshloka
{ततः समभवद्युद्धं तुमुलं भरतर्षभ}
{पाञ्चालानां च सर्वेषां भरतानां च दारुणम्}


\twolineshloka
{शैनेयस्तु रणे क्रुद्धस्तव पुत्रं महारथम्}
{सायकानामशीत्या तु विव्याधोरसि भारत}


\twolineshloka
{ततोऽस्य वाहान्समरे शरैर्निन्ये यमक्षयम्}
{सारथिं च रथात्तूर्णं पातयामास पत्रिणा}


\twolineshloka
{हताश्वे तु रथेऽतिष्ठन्पुत्रस्तव विशाम्पते}
{मुमोच निशितान्बाणाञ्शैनेयस्य रथं प्रति}


\twolineshloka
{शरान्पञ्छसतांस्तस्मै शैनेयः कृतहस्तवत्}
{चिच्छेद समरे राजन्प्रेषितांस्तनयेन ते}


\twolineshloka
{अथापरेण भल्लेन मुष्टिदेशे महद्धनुः}
{चिच्छेद तरसा युद्धे तव पुत्रस्य माधव}


\twolineshloka
{विरथो विधनुष्कश्च सर्वलोकेश्वरः प्रभुः}
{आरुरोह रथं तूर्णं भास्वरं कृतवर्मणः}


\twolineshloka
{दुर्योधने परावृत्ते शैनेयस्तव वाहिनीम्}
{द्रावयामास विशिखैर्निशामध्ये विशाम्पते}


\threelineshloka
{शकुनिश्चार्जुनं राजन्परिवार्य समन्ततः}
{रथैरनेकसाहस्रैर्गजैश्चापि सहस्रशः}
{तथा हयसहस्रैश्च नानाशस्त्रैरवाकिरत्}


\twolineshloka
{ते महास्त्राणि सर्वाणि विकिरन्तोऽर्जुनं प्रति}
{अर्जुनं योधयन्ति स्म क्षत्रियाः कालचोदिताः}


\twolineshloka
{तान्यर्जुनः सहस्राणि रथवारणवाजिनाम्}
{प्रत्यवारयदायस्तः प्रकुर्वन्विपुलं क्षयम्}


\twolineshloka
{ततस्तु समरे शूरः शकुनिः सौबलस्तदा}
{विव्याध निशितैर्बाणैरर्जुनं प्रहसन्निव}


% Check verse!
पुनश्चैव शतेनास्य संरुरोध महारथम्
\twolineshloka
{तमर्जुनस्तु विंशत्या विव्याध युधि भारत}
{अथेतरान्महेष्वासांस्त्रिभिस्त्रिभिरविध्यत}


\twolineshloka
{निवार्य तान्बाणगणैर्युधि राजन्धनञ्जयः}
{जघान तावकान्योधान्वज्रपाणिरिवासुरान्}


\twolineshloka
{भुजैश्छिन्नैर्महीपाल हस्तिहस्तोपमैर्मृधे}
{समाकीर्णा मही भाति पञ्चास्यैरिव पन्नगैः}


\twolineshloka
{शिरोभिः सकिरीटैश्च सुनसैश्चारुकुण्डलैः}
{सन्दष्टौष्ठपुटैः क्रुद्धैस्तथैवोद्वृत्तलोचनैः}


\twolineshloka
{निष्कचूडामणिधरैः क्षत्रियाणां पियंवदैः}
{पङ्कजैरिव विन्यस्तैः पर्वतैर्विबभौ मही}


\twolineshloka
{कृत्वा तत्कर्म बीभत्सुरुग्रमुग्रपराक्रमः}
{विव्याध शकुनिं भूयः पञ्चभिर्नतपर्वभिः}


\twolineshloka
{अताडयदुलूकं च त्रिभिरेव तथा शरैः}
{उलूकं पुनराजघ्ने त्रिभिरेव महायशाः}


\twolineshloka
{स तु तेन समाविद्ध क्रोधाद्द्विगुणविक्रमः}
{शरैरनेकसाहस्रैः सोऽर्जुनं प्रत्यविध्यत}


\twolineshloka
{तमुलूकस्तथा विद्धा वासुदेवमताडयत्}
{ननाद च महानादं पूरयन्निव मेदिनीम्}


\twolineshloka
{अर्जुनः शकुनेश्चापं सायकैरच्छिनद्रणे}
{निन्ये च चतुरो वाहान्यमस्य सदनं प्रति}


\twolineshloka
{ततो रथादवप्लुत्य सौबलो भरतर्षभ}
{उलूकस्य रथं तूर्णमारुरोह विशाम्पते}


\twolineshloka
{तावेकरथमारूढौ पितापुत्रौ महारथौ}
{पार्थं सिषिचतुर्बाणैर्गिरिं मेघाविवाम्बुभिः}


\twolineshloka
{तौ तु विद्धा महाराज पाण्डवो निशितैः शरैः}
{विद्रावयंस्तव चमूं शतशो व्यधमच्छरैः}


\twolineshloka
{अनिलेन यथाऽभ्राणि विच्छिन्नानि समन्ततः}
{विच्छिन्नानि तथा राजन्बलान्यासन्विशाम्पते}


\twolineshloka
{तद्बलं भरतश्चेष्ठ वध्यमानं तदा निशि}
{प्रदुद्राव दिशः सर्वा वीक्षमाणं भयार्दितम्}


\twolineshloka
{उत्सृज्य वाहान्समरे चोदयन्तस्तथाऽपरे}
{सम्भ्रान्ताः पर्यधावन्त तस्मिंस्तमसि दारुणे}


\twolineshloka
{विजित्य समरे योधांस्तावकान्भरतर्षभ}
{दध्मतुर्मुदितौ शङ्खौ वासुदेवधनञ्जयौ}


\twolineshloka
{धृष्टद्युम्नो महाराज द्रोणं विद्धा त्रिभिः शरैः}
{चिच्छेद धनुषस्तूर्णं ज्यां शरेण शितेन ह}


\twolineshloka
{तन्निधाय धनुर्भूमौ द्रोणः क्षत्रियमर्दनः}
{आददेऽन्यद्धनुः शूरो वेगवत्सारवत्तरम्}


\twolineshloka
{धृष्टद्युम्नं ततो द्रोणो विद्धा सप्तभिराशुगैः}
{सारथिं पञ्चभिर्बाणै राजन्विव्याध संयुगे}


\twolineshloka
{तं निवार्य शरैस्तूर्णं धृष्टद्युम्नो महारथः}
{व्यधमत्कौरवीं सेनामासुरीं मघवानिव}


\twolineshloka
{वध्यमाने बले तस्मिंस्तव पुत्रस्य मारिष}
{प्रावर्तत नदी घोरां शोणितौघतरङ्गिणी}


\twolineshloka
{उभयोः सेनयोर्मध्ये नराधद्विपवाहिनी}
{तथा वैतरणी राजन्यतराजुपुरं प्रति}


\twolineshloka
{द्रावयित्वा तु तत्सैन्यं धृष्टद्युम्नः प्रतापवान्}
{अभ्यराजत तेजस्वी शक्रो देवगणेष्विव}


\twolineshloka
{अथ दध्मुर्महाशङ्खान्धृष्टद्युम्नशिखण्डिनौ}
{यमौ च युयुधानश्च पाण्डक्य वृकोदरः}


\twolineshloka
{जित्वा रथसहस्राणि तावकानां महारथाः}
{सिंहनादरवांश्चक्रुः पाण़्डवा जितकाशिनः}


\twolineshloka
{पश्यतस्तव पुत्रस्य कर्णस्य च रणोत्कटाः}
{तथा द्रोणस्य शूरस्य द्रौणेश्चैव विशाम्पते}


\chapter{अध्यायः १७३}
\twolineshloka
{सञ्जय उवाच}
{}


\twolineshloka
{विद्रुतं स्वबलं दृष्ट्वा वध्यमानं महात्मभिः}
{क्रोधेन महताऽविष्टः पुत्रस्तव विशाम्पते}


\twolineshloka
{अभ्येत्य सहसा कर्णं द्रोणं च जयतां वरम्}
{अमर्षवशमापन्नो वाक्यज्ञो वाक्यमब्रवीत्}


\twolineshloka
{भवद्ध्भामिह सङ्ग्रामः क्रुद्धाभ्यां सम्प्रवर्तितः}
{आहवे निहतं दृष्ट्वा सैन्धवं सव्यसाचिना}


\twolineshloka
{निहन्यमानां पाण़्डूनां बलेन मम वाहिनीम्}
{भूत्वा तद्विजये शक्तावशक्ताविव पश्यतः}


\twolineshloka
{यद्यहं भवतोस्त्याज्यो न वाच्योऽस्मि तदैव हि}
{आवां पाण्डुसुतान्सङ्ख्ये जेष्याव इति मानदौ}


\twolineshloka
{तदैवाहं वचः श्रुत्वा भवद्ध्यामनुसम्पतम्}
{नाकरिष्यमिदं पार्थैर्वैरं योधविनाशनम्}


% Check verse!
यदि नाहं परित्याज्यो भवद्भ्यां पुरुषर्षभौ ॥युध्यतामनुरूपेण विक्रमेण सुविक्रमौ
\twolineshloka
{वाक्प्रतोदेन तौ वीरौ प्रणुन्नौ तनयेन ते}
{प्रावर्तयेतां सङ्ग्रामं घट्टिताविव पन्नगौ}


\twolineshloka
{ततस्तौ रथिनां श्रेष्ठौ सर्वलोकधनुर्धरौ}
{शैनेयप्रमुखान्पार्थानभिदुद्रुवत् रणे}


\twolineshloka
{तथैव सहिताः पार्थाः सर्वसैन्येन संवृताः}
{अभ्यवर्तन्त तौ वीरौ नर्दमानौ मुहुर्मुहुः}


\twolineshloka
{अथ द्रोणो महेष्वासो दशभिः शिनिपुङ्गवम्}
{अविध्यत्त्वरितं क्रुद्धः सर्वशस्त्रभृतां वरः}


\threelineshloka
{कर्णश्च दशभिर्बाणैः पुत्रश्च तव सप्तभिः}
{दशभिर्वृषसेनश्च सौबलश्चापि सप्तभिः}
{एते कौरव सङ्क्रन्दे शैनेयं पर्यवाकिरन्}


\twolineshloka
{दृष्ट्वा च समरे द्रोणं निघ्नन्तं पाण्डवीं चमूम्}
{विव्यधुः सोमकास्तूर्णं समन्ताच्छरवृष्टिभिः}


\twolineshloka
{तत्र द्रोणोऽहरत्प्राणान्क्षत्रियाणां विशाम्पते}
{रश्मिभिर्भास्करो राजंस्तमांसीव समन्ततः}


\twolineshloka
{द्रोणेन वध्यमानानां पाञ्चालानां विशाम्पते}
{शुश्रुवे तुमुलः शब्दः क्रोशतामितरेतरम्}


\threelineshloka
{पुत्रानन्ये पितॄनन्ये भ्रातॄनन्ये च मातुलान्}
{भागिनेयान्वयस्यांश्च तथा सम्बन्धिबान्धवान्}
{उत्सृज्योत्सृज्य गच्छन्ति त्वरिता जीवितेप्सवः}


\twolineshloka
{अपरे मोहिता मोहात्तमेवाभिमुखा ययुः}
{पाण्डवानां रणे योधाः परलोकं गताः परे}


\twolineshloka
{सा तथा पाण़्वी सेना पीड्यमाना महात्मना}
{निशि सम्प्राद्रवद्राजन्नुत्सृज्योल्काः सहस्रशः}


\twolineshloka
{पश्यतो भीमसेनस्य विजयस्याच्युतस्य च}
{ययमोर्धर्मपुत्रस्य पार्षतस्य च पश्यतः}


\twolineshloka
{तमसा संवृते लोके न प्राज्ञायत किञ्चन}
{कौरवाणां प्रकाशेन दृश्यन्ते विद्रुताः परे}


\twolineshloka
{द्रवमाणं तु तत्सैन्यं द्रोणकर्णौ महारथौ}
{जघ्नतुः पृष्ठतो राजन्किरन्तौ सायकान्बहून्}


\twolineshloka
{पाञ्चालेषु प्रभग्नेषु क्षीयमाणेषु सर्वतः}
{जनार्दनो दीनमानाः प्रत्यभाषत फल्गुनम्}


\twolineshloka
{द्रोणकर्णौ महेष्वासावेतौ पार्षतसात्यकी}
{पाञ्चालांश्चैव सहितौ जघ्नतुः सायकैर्भृशम्}


\twolineshloka
{एतयोः शरवर्षेण प्रभग्ना नो महारथाः}
{वार्यमाणापि कौन्तेय पृतना नावतिष्ठते}


\twolineshloka
{तां तु विद्रवतीं दृष्ट्वा ऊचतुः केशवार्जुनौ}
{मा विद्रवत वित्रस्ता भयं त्यजत पाण्डवाः}


\twolineshloka
{तावावां सर्वसैन्यैश्च व्यूहैः सम्यगुदायुधैः}
{द्रोणं च सूतपुत्रं च प्रयतावः प्रबाधितुम्}


\twolineshloka
{एतौ हि बलिनौ शूरौ कृतास्त्रौ जितकाशिनौ}
{उपेक्षितौ बलं तूर्णं नाशयेतां निशामिमाम्}


\threelineshloka
{सञ्जय उवाच}
{तयोः संवदतोरेवं भीमकर्मा महाबलः}
{}


\twolineshloka
{आयाद्वृकोदरः शीघ्रं पुनरावर्त्य वाहिनीम् ॥वृकोदरमथायान्तं दृष्ट्वा तत्र जनार्दनः}
{}


\twolineshloka
{पुनरेवाब्रवीद्राजन्हर्षयन्निव पाण्डवम् ॥एष भीमो रणश्लाघी वृतः सोमकपाण्डवैः}
{}


\twolineshloka
{अभ्यवर्तत वेगेन द्रोणकर्णौ महारथौ ॥एतेन सहितो युद्ध्य पाञ्चालैश्च महारथैः}
{}


\twolineshloka
{आश्वासनार्थं सैन्यानां सर्वेषां पाण़्डुनन्दन ॥ततस्तौ पुरुषव्याघ्रावुभौ माधवपाण्डवौ}
{}


\threelineshloka
{द्रोणकर्णौ समासाद्य धिष्ठितौ रणमूर्धनि ॥सञ्जय उवाच}
{ततस्तत्पुनरावृत्तं युधिष्ठिरबलं महत्}
{ततो द्रोणश्च कर्णश्च परान्ममृदतुर्युधि}


\twolineshloka
{स सम्प्रहारस्तुमुलो निशि प्रत्यभवन्महान्}
{यथा सागरयो राजंश्चन्द्रोदयविरुद्धयोः}


\twolineshloka
{तत उत्सृज्य पाणिभ्यां प्रदीपांस्तव वाहिनी}
{युयुधे पाण्डवैः सार्धमुन्मत्तवदसङ्कुला}


\twolineshloka
{रजसा तमसा चैव संवृते भृशदारुणे}
{केवलं नामगोत्रेण प्रायुध्यन्त जयैषिणः}


\twolineshloka
{अश्रूयन्त हि नामानि श्राव्यमाणानि पार्थिवैः}
{प्रहरद्भिर्महाराज स्वयंवर इवाहवे}


\twolineshloka
{निःशब्दमासीत्सहसा पुनः शब्दो महानभूत्}
{क्रुद्धानां युध्यमानानां जीयतां जयतामपि}


\twolineshloka
{यत्रयत्र स्म दृश्यन्ते प्रदीपाः कुरुसत्तम}
{तत्रतत्र स्म शूरास्ते निपतन्ति पतङ्गवत्}


\twolineshloka
{तथा संयुध्यमानानां विगाढाऽऽसीन्महानिशा}
{पाण्डवानां च राजेन्द्र कौरवाणां च सर्वशः}


\chapter{अध्यायः १७४}
\twolineshloka
{सञ्जय उवाच}
{}


\twolineshloka
{ततः कर्णो रणे दृष्ट्वा पार्षतं परवीरहा}
{आजघानरोसि शरैर्दशभिर्मर्मभेदिभिः}


\twolineshloka
{प्रतिविव्याध तं तूर्णं धृष्टद्युम्नोपि मारिष}
{दशभिः सायकैर्हृष्टस्तिष्ठतिष्ठेति चाब्रवीत्}


\twolineshloka
{तावन्योन्यं शरैः सङ्ख्ये सञ्छाद्य सुमहारथैः}
{पुनः पूर्णायतोत्सृष्टैर्विव्यधाते परस्परम्}


\twolineshloka
{ततः पाञ्चालमुख्यस्य धृष्टद्युम्नस्य संयुगे}
{सारथिं चतुरश्चाश्वान्कर्णो विव्याध सायकैः}


\twolineshloka
{कार्मुकप्रवरं चापि प्रचिच्छेद शितैः शरैः}
{सारथिं चास्य भल्लेन रथनीडादपातयत्}


\twolineshloka
{धृष्टद्युम्नस्तु विरथो हताश्वो हतसारथिः}
{गृहीत्वा परिघं घोरं कर्णस्याश्वानपीडयत्}


\twolineshloka
{विद्धश्च बहुभिस्तेन शरैराशीविषोपमैः}
{ततो युधिष्ठिरानीकं पद्ध्यामेवान्वपद्यत}


\twolineshloka
{आरुरोह रथं चापि सहदेवस्य मारिष}
{प्रयातुकामः कर्णाय वारितो धर्मसूनुना}


\twolineshloka
{कर्णस्तु सुमहातेजाः सिंहनादविमिश्रितम्}
{धनुःशब्दं महच्चक्रे दध्मौ तारेण चाम्बुजम्}


\twolineshloka
{दृष्ट्वा विनिर्जितं युद्धे पार्षतं ते महारथाः}
{अमर्षवशमापन्नाः पाञ्चालाः सहसोमकाः}


\twolineshloka
{सूतपुत्रवधार्थाय शस्त्राण्यादाय सर्वशः}
{प्रययुः कर्णमुद्दिश्य मृत्युं कृत्वा निवर्तनम्}


\twolineshloka
{कर्णस्यापि रथे वाहानन्यान्सूतोऽभ्ययोजयत्}
{शङ्खवर्णान्महावेगान्सैन्धवान्साधुवाहिनः}


\twolineshloka
{लब्धलक्षस्तु राधेयः पञ्चालानां महारथान्}
{अभ्यपीडयदायस्तः शरैर्मेघ इवाचलम्}


\twolineshloka
{सा पीड्यमाना कर्णेन पाञ्चालानां महाचमूः}
{सम्प्राद्रवत्सुसन्त्रस्ता सिंहेनेवार्दिता मृगी}


\twolineshloka
{पतितास्तुरगेभ्यश्च गजेभ्यश्च महीतले}
{रथेभ्यश्च नरास्तूर्णमदृश्यन्त ततस्ततः}


\twolineshloka
{धावमानस्य योधस्य क्षुरप्रैः स महामृधे}
{बाहू चिच्छेद वै कर्णः शिरश्चैव सकुण्डलम्}


\twolineshloka
{ऊरू चिच्छेद चान्यस्य गजस्थस्य विशाम्पते}
{वाजिपृष्ठगतस्यापि भूमिष्ठस्य च मारिष}


\twolineshloka
{नाज्ञासिषुर्धावमाना बहवश्च महारथाः}
{सञ्छिन्नान्यात्मगात्राणि वाहनानि च संयुगे}


\twolineshloka
{ते वध्यमानाः समरे पाञ्चालाः सृञ्जयैः सह}
{तृणप्रस्पन्दनाच्चापि सूतपुत्रं स्म मेनिरे}


\twolineshloka
{अपि स्वं समरे योधं धावमानं विचेतसम्}
{कर्णमेवाभ्यमन्यन्त ततो भीता द्रवन्ति ते}


\twolineshloka
{तान्यनीकानि भग्नानि द्रवमाणानि भारत}
{अभ्यद्रवद्द्रुतं कर्णः पृष्ठतो विकिरञ्छरान्}


\twolineshloka
{अवेक्षमाणास्त्वन्योन्यं सुसम्मूढा विचेतसः}
{नाशक्नुवन्नवस्थातुं काल्यमाना महात्मना}


\twolineshloka
{कर्णेनाभ्याहता राजन्पाञ्चालाः परमेषुभिः}
{द्रोणेन च दिशः सर्वा वीक्षमाणाः प्रदुद्रुवुः}


\twolineshloka
{ततो युधिष्ठिरो राजा स्वसैन्यं प्रेक्ष्य विद्रुतम्}
{अपयाने मनः कृत्वा फल्गुनं वाक्यमब्रवीत्}


\twolineshloka
{पश्य कर्णं महेष्वासं धनुष्पाणिमवस्थितम्}
{निशीथे दारुणे काले तपन्तमिव भास्करम्}


\twolineshloka
{कर्णसायकनुन्नानां क्रोशतामेष निःस्वनः}
{अनिशं श्रूयते पार्थ त्वद्बन्धूनामनाथवत्}


\twolineshloka
{यथा विसृजतश्चास्य सन्दधानस्य चाशुगान्}
{पश्यामि नान्तरं पार्थ क्षपयिष्यति नो ध्रुवम्}


\twolineshloka
{यदत्रानन्तरं कार्यं प्राप्तकाल च पश्यसि}
{कर्णस्य वधसंयुक्तं तत्कुरुष्व धनञ्जय}


\twolineshloka
{एवमुक्तो महाराज पार्थः कृष्णमथाब्रवीत्}
{भीतः कुन्तीसुतो राजा राधेयस्याद्य विक्रमात्}


\twolineshloka
{एवं गते प्राप्तकालं कर्णानीके पुनः पुनः}
{भवान्व्यवस्यतु क्षिप्रं द्रवते हि वरूथिनी}


\twolineshloka
{द्रोणसायकनुन्नानां भग्नानां मधुसूदन}
{कर्णेन त्रास्यमानानामवस्थानं न विद्यते}


\twolineshloka
{पश्यामि च तथा कर्णं विचरन्तमभीतवत्}
{द्रवमाणान्रथोदारान्किरन्तं निशितैः शरैः}


\twolineshloka
{नैनं शक्ष्यामि संसोढुं चरन्तं रणमूर्धनि}
{प्रत्यक्षं वृष्णिशार्दूल पादस्पर्शमिवोरगः}


\threelineshloka
{स सवांस्तत्र यात्वाशु यत्र कर्णो महारथः}
{अहमेनं हनिष्यामि मां वैष मधुसूदन ॥श्रीवासुदेव उवाच}
{}


\twolineshloka
{पश्यामि कर्णं कौन्तेय देवराजमिवाहवे}
{विचरन्तं नरव्याघ्रमतिमानुषविक्रमम्}


\twolineshloka
{नैतस्यान्योऽस्ति सङ्ग्रामे प्रत्युद्याता धनञ्जय}
{ऋते त्वां पुरुषव्याघ्र राक्षसाद्वा घटोत्कचात्}


\twolineshloka
{न तु तावदहं मन्ये प्राप्तकालं तवानघ}
{समागमं महाबाहो सूतपुत्रेण संयुगे}


\twolineshloka
{दीप्यमाना महोल्केव तिष्ठत्यस्य हि वासवी}
{त्वदर्थं हि महाबाहो सूतपुत्रेण संयुगे}


\twolineshloka
{रक्ष्यते शक्तिरेषा हि रौद्रां रूपं बिभर्ति च}
{घटोत्कचस्तु राधेयं प्रत्युद्यातु महाबलः}


\twolineshloka
{स हि भीमेन बलिना जातः सुरपराक्रमः}
{तस्मिन्नस्त्राणि दिव्यानि राक्षसान्यासुराणि च}


\threelineshloka
{सततं चानुरक्तो वो हितैषी च घटोत्कचः}
{विजेष्यति रणे कर्णमिति मे नात्र संशयः ॥सञ्जय उवाच}
{}


\twolineshloka
{एवमुक्तो महाबाहुः पार्थः पुष्करलोचनः}
{आजुहावाथ तद्रक्षस्तच्चासीत्प्रादुरप्रतः}


\threelineshloka
{कवची सशरः खङ्गी सधन्वा च विशाम्पते}
{अभिवाद्य ततः कृष्णं पाण्डवं च धनञ्जयम्}
{अब्रवीच्च तदा कृष्णमयमस्म्यनुशाधि माम्}


\threelineshloka
{ततस्तं मेघसङ्काशं दीप्तास्यं दीप्तकुण्डलम्}
{अभ्यभाषत हैडिम्बिं दाशार्हः प्रहसन्निव ॥वासुदेव उवाच}
{}


\threelineshloka
{घटोत्कच विजानीहि यत्त्वां वक्ष्यामि पुत्रक}
{प्राप्तो विक्रमकालोऽयं तव नान्यस्य कस्याचित्}
{स त्वं निमज्जतां युद्धे बन्धूनां बान्धवो भव}


\twolineshloka
{यावन्ति च तवास्त्राणि सन्ति मायाश्च राक्षसीः}
{`अद्यैव समयस्तेषां मायानां चैव राक्षस'}


\twolineshloka
{पश्य कर्णेन हैडिम्बे पाण्डवानामनीकिनीम्}
{काल्यमानां रणे तात सिंहनेवेतरान्मृगान्}


\twolineshloka
{एष कर्णो महेष्वासो मतिमान्दृढविक्रमः}
{पाण्वानामनीकेषु निहन्ति क्षत्रियर्षभान्}


\twolineshloka
{किरन्तः शरवर्षाणि महान्ति दृढधन्विनः}
{न शक्नुवन्त्यवस्थातुं पीड्यमानाः शरार्चिषा}


\twolineshloka
{निशीथे सूतपुत्रेण शरवर्षेण पीडिताः}
{एते द्रवन्ति पाञ्चालाः सिंहेनेवार्दिता मृशाः}


\twolineshloka
{एतस्यैवं प्रवृद्धस्य सूतपुत्रस्य संयुगे}
{निषेद्धा विद्यते नान्यस्त्वामृते भीमविक्रम}


\twolineshloka
{स त्वं कुरु महाबाहो कर्म युक्तमिहात्मनः}
{मातुलानां पितॄणां च तेसजोऽस्त्रबलस्य च}


\twolineshloka
{एतदर्थे हि हैडिम्बे पुत्रानिच्छन्ति मानवाः}
{कथं नस्तारयेद्दुःखात्स त्वं तारय बान्धवान्}


\twolineshloka
{इच्छन्ति पितरः पुत्रान्स्वार्थहेतोर्घटोत्कच}
{हहलोकात्परे लोके तारयिष्यन्ति ये हिताः}


\twolineshloka
{तव ह्यत्र बलं भीमं मायाश्च तव दुस्तराः}
{सङ्क्रामे युध्यमानस्य सततं भीमनन्दन}


\twolineshloka
{पाण्डवानां प्रभग्नानां कर्णेन निशि सायकैः}
{मज्जतां धार्तराष्ट्रेषु भव पारं परन्तप}


\twolineshloka
{रात्रौ हि राक्षसा भूयो भवन्त्यमितविक्रमाः}
{बलवन्तः सुदुर्धर्षाः शूरा विक्रान्तचारिणः}


\threelineshloka
{जहि कर्णं हमेष्वासं निशीथे मायया रणे}
{पार्था द्रोणं वधिष्यन्ति धृष्टद्युम्नपुरोगमाः ॥सञ्जय उवाच}
{}


\twolineshloka
{केशवस्य वचः श्रुत्वा बीभत्सुरपि राक्षसम्}
{अभ्यभाषत कौरव्य घटोत्कचमरिन्दमम्}


\twolineshloka
{घटोत्कच भवांश्चैव दीर्घबाहुश्च सात्यकिः}
{मतौ मे सर्वसैन्येषु भीमसेनश्च पाण्डवः}


\twolineshloka
{तद्भवान्यातु कर्णेन द्वैरथं युध्यतां निशि}
{सात्यकिः पृष्ठगोपस्ते भविष्यति महारथः}


\threelineshloka
{जहि कर्णं रणे शूरं सात्वतेन सहायवान्}
{यथेन्द्रस्तारकं पूर्वं स्कन्देन सह जघ्निवान् ॥घटोत्कच उवाच}
{}


\twolineshloka
{`एवमेव महाबाहो यथा वदसि मां प्रभो}
{त्वया नियुक्तो गच्छामि कर्णस्य बधकाङ्क्षया'}


\twolineshloka
{अलमेवास्मि कर्णाय द्रोणायालं च भारत}
{अन्येषां क्षत्रियाणां च कृतास्त्राणां महात्मनाम्}


\twolineshloka
{अद्य दास्यामि सङ्ग्रामं सूतपुत्राय तं निशि}
{यं जनाः सम्प्रवक्ष्यन्ति यावद्भूमिर्धरिष्यति}


\threelineshloka
{न चात्र शूरान्मोक्ष्यामि न भीतान्न कृताञ्जलीन्}
{सर्वानेव वधिष्यामि राक्षसं धर्ममास्थितः ॥सञ्जय उवाच}
{}


\twolineshloka
{एवमुक्त्वा महाबाहुर्हैडिम्बिः परवीरहा}
{अभ्ययात्तुमुले कर्णं तव सैन्यं विभीषयन्}


\twolineshloka
{तमापतन्तं सङ्क्रुद्धं दीप्तास्यं दीप्तमूर्धजम्}
{प्रहसन्पुरुषव्याघ्रः प्रतिजग्राह सूतजः}


\twolineshloka
{तयोः समभवद्युद्धं कर्णराक्षसयोर्निशि}
{गर्जतो राजशार्दूल शक्रप्रह्लादयोरिव}


\chapter{अध्यायः १७५}
\twolineshloka
{सञ्जय उवाच}
{}


\threelineshloka
{दृष्ट्वा घटोत्कचं राजन्सूतपुत्ररथं प्रति}
{आयान्तं तु तथायुक्तं जिघांसुं कर्णमाहवे}
{अब्रवीत्तत्र पुत्रस्ते दुःशासनमिदं वचः}


\twolineshloka
{एतद्रक्षो रणे तूर्णं दृष्ट्वा कर्णस्य विक्रमम्}
{अभियाति द्रुतं कर्णं तद्वारय महारथम्}


\twolineshloka
{वृतः सैन्येन महता याहि यत्र महाबलः}
{कर्णो वैकर्तनो युद्धे राक्षसेन युयुत्सति}


\twolineshloka
{रक्ष कर्णं रणे यत्तो वृतः सैन्येन मानद}
{मा कर्णं राक्षसो घोरः प्रमादान्नाशयिष्यति}


\twolineshloka
{एतस्मिन्नन्तरे राजञ्जटासुरसुतो बली}
{दुर्योधनमुपागम्य प्राह प्रहरतां वरः}


\twolineshloka
{दुर्योधन तवामित्रान्प्रख्यातान्युद्धदुर्मदान्}
{पाण्डवान्हन्तुमिच्छामि त्वयाऽऽज्ञप्तः सहानुगान्}


\twolineshloka
{जटासुरो मम पिता रक्षसां ग्रामणीः पुरा}
{प्रयुज्य कर्म रक्षोघ्नं क्षुद्रैः पार्थैर्निपातितः}


\twolineshloka
{तस्यापचितिमिच्छामि शत्रुशोणितपूजया}
{शत्रुमांसैश्च राजेन्द्र मामनुज्ञातुमर्हसि}


\twolineshloka
{तमब्रवीत्ततो राजा प्रीयमाणः पुनःपुनः}
{द्रोणकर्मादिभिः सार्धं पर्याप्तोऽहं द्विषद्वधे}


\twolineshloka
{त्वं तु गच्छ मयाऽऽज्ञप्तो जहि युद्धे घटोत्कचम्}
{राक्षसं क्रूरकर्माणं रक्षोमानुषसम्भवम्}


\twolineshloka
{पाण्डवानां हितं नित्यं हस्त्यश्वरथघातिनम्}
{वैहायसगतं युद्धे प्रेषयेर्यमसादनम्}


\twolineshloka
{तथेत्युक्त्वा महाकायः समाहूय घटोत्कचम्}
{जाटासुरिर्भैमसेनिं नानाशस्त्रैरवाकिरत्}


\twolineshloka
{अलम्बलं च कर्णं च कुरुसैन्यं च दुस्तरम्}
{हैडिम्बिः प्रममाथैको महाबातोम्बुदानिव}


\twolineshloka
{ततो मायाबलं दृष्ट्वा रक्षस्तूर्णमलम्बलः}
{घटोत्कचं शरव्रातैर्नानालिङ्गैः समार्पयत्}


\twolineshloka
{विद्धा च बहुभिर्बाणैर्भैमसेनिं महाबलः}
{व्यद्रावयच्छरव्रातैः पाण्डवानामनीकिनीम्}


\twolineshloka
{तेन विद्राव्यमाणानि पाण्डुसैन्यानि भारत}
{निशीथे विप्रकीर्यन्ते वातनुन्ना घना इव}


\twolineshloka
{घटोत्कचशरैर्नुन्ना तथैव तव वाहिनी}
{निशीथे प्राद्रवद्राजन्नुत्सृज्योल्काः सहस्रशः}


\twolineshloka
{अलम्बलस्ततः क्रुद्धो भैमसेनिं महामृधे}
{आजघ्ने दशभिर्बाणैस्तोत्रैरिव महाद्विपम्}


\twolineshloka
{तिxशस्तस्य संवाहं सूतं सर्वायुधानि च}
{घटोत्कचः प्रचिच्छेद प्रणदंश्चातिदारुणम्}


\twolineshloka
{ततः कर्णं शरव्रातैः कुरूनन्यान्सहस्रशः}
{अलम्बलं चाभ्यवर्षन्मेघो मेरुमिवाचलम्}


\twolineshloka
{ततः सञ्चुक्षुभे सैन्यं कुरूणां राक्षसार्दितम्}
{उपर्युपरि चान्योन्यं चतुरङ्गं ममर्द ह}


\twolineshloka
{जाटासुरिर्महाराज विरथो हतसारथिः}
{घटोत्कचं रणे क्रुद्धो मुष्टिनाऽभ्यहनद्दृढम्}


\twolineshloka
{मुष्टिनाऽभ्याहतस्तेन प्रचचाल घटोत्कचः}
{क्षितिकम्पे यथा शैलः सवृक्षस्तृणगुल्मवान्}


\twolineshloka
{ततः स परिघाभेन द्विट््सङ्घघ्नेन बाहुना}
{जाटासुरिं भैमसेनिरवधीन्मुष्टिना भृशम्}


\twolineshloka
{तं प्रमथ्य ततः क्रुद्धस्तूर्णं हैडिम्बिराक्षिपत्}
{दोर्भ्यामिन्द्रध्वजाभाभ्यां निष्पिपेष च भूतले}


\twolineshloka
{जाटासुरिर्मोक्षयित्वा आत्मानं च घटोत्कचात्}
{पुनरुत्थाय वेगेन घटोत्कचमुपाद्रावत्}


\twolineshloka
{अलम्बलोऽपि विक्षिप्य समुत्क्षिप्य च राक्षसम्}
{घटोत्कचं रणे रोषान्निष्पिपेष च भूतले}


\twolineshloka
{तयो समभवद्युद्धं गर्जतोरतिकाययोः}
{घटोत्कचालम्बलयोस्तुमुलं रोमहर्षणम्}


\twolineshloka
{विशेषयन्तावन्योन्यं मायाभिरतिमायिनौ}
{युयुधाते महावीर्याविन्द्रवैरोचनाविव}


\twolineshloka
{पावकाम्बुनिधी भूत्वा पुनर्गरुडतक्षकौ}
{पुनर्मेघमहाबत्तौ पुनर्वज्रमहाचलौ}


\threelineshloka
{पुनः कुञ्जरशार्दूलौ पुनः स्वर्भानुभास्करौ}
{एवं मायाशतसृजावन्योन्यवधकाङ्क्षिणौ}
{भृशं चित्रमयुध्येतामलम्बलघटोत्कचौ}


\twolineshloka
{परिघैश्च गदाभिश्च प्रासमुद्ररपट्टसैः}
{मुसलैः पर्वताग्रैश्च तावन्योन्यं विजघ्नतुः}


\twolineshloka
{हयाभ्यां च गजाभ्यां च रथाभ्यां च पदातिभिः}
{युयुधाते महामायौ राक्षसप्रवरौ युधि}


\twolineshloka
{ततो घटोत्कचो राजन्नलम्बलवधेप्सया}
{उत्पतात भृशं क्रुद्धः श्येनवन्निपपात च}


\twolineshloka
{गृहीत्वा च महाकायं राक्षसेन्द्रमलम्बलम्}
{उद्यम्य न्यवधीद्भूमौ मयं विष्णुरिवाहवे}


\twolineshloka
{ततो घटोत्कचः खङ्गमुद्वृत्याद्भुतदर्शनम्}
{रौद्रस्य कायाद्वि शिरो भीमं विकृतदर्शनम्}


\twolineshloka
{स्फुरतस्तस्य समरे नदतश्चातिभैरवम्}
{निचकर्त महाराज शत्रोमितविक्रमः}


\twolineshloka
{शिरस्तच्चापि सङ्गृह्य केशेषु रुधिरोक्षितम्}
{ययौ घटोत्कचस्तूर्णं दुर्योधनरथं प्रति}


\twolineshloka
{`द्रोणकर्णकृपान्योधानतीत्य सुमहाबलः'}
{अभ्येत्य च महाबाहुः स्मयमानः स राक्षसः}


\twolineshloka
{शिरो रथेऽस्य निक्षिप्य विकृताननमूर्धजम्}
{प्राणदद्भैरवं नादं प्रावृषीव वलाहकः}


% Check verse!
अब्रवीच्च ततो राजन्दुर्योधनमिदं वचः
\twolineshloka
{एष ते निहतो बन्धुस्त्वया दृष्टोऽस्य विक्रमः}
{पुनर्द्रष्टासि कर्णस्य निष्ठामेतां तथाऽऽत्मनः}


\fourlineindentedshloka
{स्वधर्ममर्थं कामं च त्रितयं योऽभिवाञ्छति}
{रिक्तपाणिर्न पश्येत राजानं ब्राह्मणं स्त्रियम्}
{तिष्ठस्व तावत्सुप्रीतो यावत्कर्णं वधाम्यहम् ॥सञ्जय उवाच}
{}


\twolineshloka
{एवमुक्त्वा ततः प्रायात्कर्णं प्रति नरेश्वरः}
{किरञ्छरगणांस्तीक्ष्णान्रुषितो रणमूर्धनि}


\twolineshloka
{ततः समभवद्युद्धं घोररूपं भयानकम्}
{विस्मापनं महाराज नरराक्षसयोर्मृधे}


\chapter{अध्यायः १७६}
\twolineshloka
{धृतराष्ट्र उवाच}
{}


\twolineshloka
{यत्तद्वैकर्तनः कर्णो राक्षसश्च घटोत्कचः}
{निशीथे समसज्जेतां तद्युद्धमभवत्कथम्}


\twolineshloka
{कीदृशं चाभवद्रूपं तस्य घोरस्य रक्षसः}
{येन वैकर्तनः कर्णः सङ्ग्रामे तेन निर्जितः}


\twolineshloka
{रथश्च कीदृशस्तस्य मायाः सर्वायुधानि च}
{किम्प्रमाणा हयास्तस्य रथकेतुर्धनुस्तथा}


\threelineshloka
{कीदृशं वर्म चैवास्य शिरस्त्राणं च कीदृशम्}
{पृष्टस्त्वमेतदाचक्ष्व कुशलो ह्यसि सञ्जय ॥सञ्जय उवाच}
{}


\twolineshloka
{लोहितोक्षो महाकायस्ताम्रास्यो निम्नितोदरः}
{ऊर्ध्वरोमा हरिश्मश्रुः शङ्कुकर्णो महाहनुः}


\twolineshloka
{आकर्णदारितास्यश्च तीक्ष्णदंष्ट्रः करालवान्}
{सुदीर्घताम्रजिह्वोष्ठो लम्बभ्रूः स्थूलनासिकः}


\twolineshloka
{नीलाङ्गो लोहितग्रीवो गिरिवर्ष्मा भयङ्करः}
{महाकायो महाबाहुर्महाशीर्षो महाबलः}


\twolineshloka
{विकृतः परुषस्पर्शो विकचोद्वृद्धपिण्डकः}
{स्थूलस्फिग्गूढनाभिश्च शिथिलोपचयो महान्}


\twolineshloka
{तथैव हस्ताभरणी महामायोऽङ्गदी तथा}
{उरसाऽधारयन्निष्कमग्निमालां यथाऽचलः}


\twolineshloka
{तस्य हेममयं चित्रं बहुरूपाङ्गशभितम्}
{तोरणप्रतिमं शुभ्रं किरीटं मूर्ध्न्यशोभत}


\threelineshloka
{कुण्डले बालसूर्याभे मालां हेममयीं शुभाम्}
{धारयन्विपुलं कांस्यं कवचं च महाप्रभम्}
{ताराजालनिभं राजन्पूर्णचन्द्रसमप्रभम्}


\twolineshloka
{कण्ठसूत्रं महच्चापि तपनीयमधारयत्}
{स तेन विभ्राजत वै सविद्युदिव तोयदः}


\twolineshloka
{किङ्किणीशतनिर्घोपं रक्तध्वजपताकिनम्}
{ऋक्षचर्मावनद्वाङ्गं नल्वमात्रं महारथम्}


\twolineshloka
{सर्वायुधवरोपेतमास्थितं ध्वजमालिनम्}
{अष्टचक्रसमायुक्तं मेघगम्भीरनिःस्वनम्}


\twolineshloka
{मत्तमातङ्गसङ्काशा लोहिताक्षा विभीपणाः}
{कालवर्णासमायुक्ता बलवन्तः शतं हयाः}


\twolineshloka
{वहन्तो राक्षसं घोरं वालवन्तो जितश्रमाः}
{विपुलाभिः सटाभिस्ते हेषमाणा मुहुर्मुहुः}


\twolineshloka
{मुखैर्नानाविधाकारैर्वगवन्तो हयोत्तमाः}
{रथेऽस्य युक्ता गकर्जन्तोऽवहंस्ते राक्षसाधिपम्}


\threelineshloka
{राक्षसोऽस्य विरूपाक्षः सूतो दीप्तास्यकुण्डलः}
{`घोररूपो महाकायः करालो विकृताननः'}
{रश्मिभिः सूर्यरश्म्याह्वैः सञ्जग्राह हयान्रणे}


\twolineshloka
{स तेन सहितस्तस्थावरुणेन यथा रविः}
{संसक्त इव चाभ्रेण यथाऽद्रिर्महता महान्}


\twolineshloka
{दिवस्पृक्सुमहान्केतुः स्यन्दनेऽस्य समुच्छ्रितः}
{रक्तोत्तमाङ्गः क्रव्यादो गृध्रः परमभीषणः}


\twolineshloka
{वासवाशनिनिर्घोषं दृढज्यमतिविक्षिपन्}
{व्यक्तं किष्कुपरीणाहं द्वादशारत्नि कार्मुकम्}


\twolineshloka
{रथाक्षमात्रैरिपुभिः सर्वाः प्रच्छादयन्दिशः}
{तस्यां वीरापहारिण्यां निशायां कर्णमभ्ययात्}


\twolineshloka
{तस्य विक्षिपतश्चापं रथे विष्टभ्य तिष्ठतः}
{अश्रूयत धनुर्घोषो विस्फूर्जितमिवाशनेः}


\twolineshloka
{तेन वित्रास्यमानानि तव सैन्यानि भारत}
{समकम्पन्त सर्वाणि सिन्धोरिव महोर्मयः}


\twolineshloka
{तमापतन्तं सम्प्रेक्ष्य विरूपाक्षं विभीषणम्}
{उत्स्मयन्निव राधेयस्त्वरमाणोऽभ्यवारयत्}


\twolineshloka
{ततः कर्णोऽभ्ययादेनमस्यन्नस्यन्तमन्तिकात्}
{मातङ्ग इव मातङ्गं यूथर्षभमिवर्षभः}


\twolineshloka
{स सन्निपातस्तुमुलस्तयोरासीद्विशाम्पते}
{कर्णराक्षसयो राजन्निन्द्रशम्बरयोरिव}


\twolineshloka
{तौ प्रगृह्य महावेगे धनुषी भीमनिःस्वने}
{प्राच्छादयेतामन्योन्यं तक्षमाणौ महेषुभिः}


\twolineshloka
{ततः पूर्णायतोत्सृष्टैरिषुभिर्नतपर्वभिः}
{न्यवारयेतामन्योन्यं कांस्ये निर्भिद्य वर्मणी}


\twolineshloka
{तौ नखैरिव शार्दूलौ दन्तैरिव महाद्विपौ}
{रथशक्तिभिरन्योन्यं विशिखैश्च ततक्षतुः}


\twolineshloka
{सञ्छिन्दन्तौ च गात्राणि सन्दधानौ च सायकान्}
{दहन्तौ च शरोल्काभिर्दुष्प्रेक्ष्यौ च बभूवतुः}


\twolineshloka
{तौ तु विक्षतसर्वाङ्गौ रुधिरौघपरिप्लुतौ}
{विभ्राजेतां यथा वारि स्रवन्तौ गैरिकाचलौ}


\twolineshloka
{तौ शराग्रविनुन्नाङ्गौ निर्भिन्दन्तौ परस्परम्}
{नाकम्पयेतामन्योन्यं यतमानौ महाद्युती}


\twolineshloka
{तत्प्रवृत्तं निशायुद्धं चिरं सममिवाभवत्}
{प्राणयोर्दीव्यतो राजन्कर्णराक्षसयोर्मृधे}


\twolineshloka
{तस्य सन्दधतस्तीक्ष्णाञ्छरांश्चासक्तमस्यतः}
{धनुर्घोषेण वित्रस्ताः स्वे परे च तदाऽभवन्}


\twolineshloka
{घटोत्कचं यदा कर्णो विशेषयति नो नृप}
{ततः प्रादुष्करोद्दिव्यमस्त्रमस्त्रविदां वरः}


\twolineshloka
{कर्णेन सन्धितं दृष्ट्वा दिव्यमस्त्रं घटोत्कचः}
{प्रादुश्चक्रे महामायां राक्षसीं पाण्डुनन्दनः}


\twolineshloka
{शूलमुद्गरधारिण्या शैलपादपहस्तया}
{रक्षसां घोररूपाणां महत्या सेनया वृतः}


\twolineshloka
{तमुद्यतमहाचापं दृष्ट्वा ते व्यथिता नृपाः}
{भूतान्तकमिवायान्तं कालदण्डोग्रधारिणम्}


\twolineshloka
{घटोत्कचप्रयुक्तेन सिंहनादेन भीषिताः}
{प्रसुस्रुवुर्गजा मूत्रं विव्यधुश्च नरा भृशम्}


\twolineshloka
{ततोऽश्मवृष्टिरत्युग्रा महत्यासीत्समन्ततः}
{अर्धरात्रेऽधिकबलैर्विमुक्ता रक्षसां बलैः}


\twolineshloka
{आयसानि च चक्राणि भुशुण़्ड्यः शक्तितोमराः}
{पतन्त्यविरलाः शूलाः शतघ्न्यः पट्टसास्तथा}


\twolineshloka
{तदुग्रमतिरौद्रं च दृष्ट्वा युद्धं नराधिप}
{पुत्राश्च तव योधाश्च व्यथिता विप्रदुद्रुवुः}


\twolineshloka
{तत्रैकोऽस्तबलश्लाघी कर्णो मानी न विव्यथे}
{व्यधमच्च शरैर्मायां तां घटोत्कचनिर्मिताम्}


\twolineshloka
{मायायां तु प्रहीणायाममर्षाच्च घटोत्कचः}
{विससर्ज शरान्धोरान्सूत पुत्रं त आविशन्}


\twolineshloka
{ततस्ते रुधिराभ्यक्ता भित्त्वा कर्णं महाहवे}
{विविशुर्धरणीं बाणाः सङ्क्रुद्धा इव पन्नगाः}


\twolineshloka
{सूतपुत्रस्तु सङ्क्रद्धो लघुहस्तः प्रतापवान्}
{घटोत्कचमतिक्रम्य बिभेद दशभिः शरैः}


\twolineshloka
{घटोत्कचो विनिर्भिन्नः सूतपुत्रेण मर्मसु}
{चक्रं दिव्यं सहस्रारमगृह्णाद्व्यथितो भृशम्}


\twolineshloka
{क्षुरान्तं बालसूर्याभं मणिरत्नविभूषितम्}
{चिक्षेपाधिरथेः क्रुद्धो भैमसेनिर्जिघांसया}


\twolineshloka
{प्रविद्वमतिवेगेन विक्षिप्तं कर्णसायकैः}
{अभाग्यस्येव सङ्कल्पस्तन्मोघमपतद्भुवि}


\twolineshloka
{घटोत्कचस्तु सङ्क्रुद्धो दृष्ट्वा चक्रं निपातितम्}
{कर्णं प्राच्छादयद्बाणैः स्वर्भानुरिव भास्करम्}


\twolineshloka
{सूतपुत्रस्त्वसम्भ्रान्तो रुद्रोपेन्द्रेन्द्रविक्रमः}
{घटोत्कचरथं तूर्णं छादयामास पत्रिभिः}


\twolineshloka
{घटोत्कचेन क्रुद्धेन गदा हेमाङ्गदा तदा}
{क्षिप्ता भ्राम्य शरैः साऽपि कर्णेनाभ्याहताऽपतत्}


\twolineshloka
{ततोऽन्तरिक्षमुत्पत्य कालमेघ इवोन्नदन्}
{प्रववर्ष महाकायो द्रुमवर्षं नभस्तलात्}


\twolineshloka
{ततो मायाविनं कर्णो भीमसेनसुतं दिवि}
{मार्गणैरभिविव्याध घनं सूर्य इवांशुभिः}


\twolineshloka
{तस्य सर्वान्हयान्हत्वा सञ्छिद्य शतधा रथम्}
{अभ्यवर्षच्छरैः कर्णः पर्जन्य इव वृष्टिमान्}


\twolineshloka
{न चास्यासीदनिर्भिन्नं गात्रे द्व्यङ्गुलमन्तरम्}
{सोऽदृश्यत मुहूर्तेन श्वाविच्छललितो यथा}


\twolineshloka
{न हयान्न रथं तस्य न ध्वजं न घटोत्कचम्}
{दृष्टवन्तः स्म समरे शरौघैरभिसंवृतम्}


\twolineshloka
{स तु कर्णस्य तद्दिव्यमस्त्रमस्त्रेण शातयन्}
{मायायुद्धेन मायावी सूतपुत्रमयोधयत्}


\twolineshloka
{सोऽयोधयत्तदा कर्णं मायया लाघवेन च}
{अलक्ष्यमाणानि दिवि शरजालानि चापतन्}


\twolineshloka
{भैमसेनिर्महामायो मायया कुरुसत्तम}
{विचचार महाकायो मोहयन्निव भारत}


\twolineshloka
{स तु कृत्वा विरूपाणि वदनान्यशुभानि च}
{अग्रसत्सूतपुत्रस्य दिव्यान्यस्त्राणि मायया}


\twolineshloka
{पुनश्चापि महाकायः सञ्छिन्नः शतधा रणे}
{गतसत्वो निरुत्साहः पतितः खाद्व्यदृश्यत}


\twolineshloka
{तं हतं मन्यमानाः स्म प्राणदन्कुरुपुङ्गवाः}
{अथ देहैर्नवैरन्यैर्दिक्षु सर्वास्वदृश्यत}


\twolineshloka
{पुनश्चापि महाकायः शतशीर्षः शतोदरः}
{व्यदृश्यत महाबाहुर्मैनाक इव पर्वतः}


\twolineshloka
{अङ्गुष्ठमात्रो भूत्वा च पुनरेव स राक्षसः}
{सागरोर्मिरिवोद्वूतस्तिर्यगूर्ध्वमवर्तत}


\twolineshloka
{वसुधां दारयित्वा च पुनरप्यु न्यमज्जत}
{अदृश्यत तदा तत्र पुनरुन्मज्जितोऽन्यतः}


\twolineshloka
{सोऽवतीर्य पुनस्तस्थौ रथे हेमपरिष्कृते}
{क्षितिं खं च दिशश्चैव माययाऽभ्येत्य दंशितः}


\twolineshloka
{गत्वा कर्णरथाभ्याशं व्यचरत्कुण्डलाननः}
{प्राह वाक्यमसम्भ्रान्तः सूतपुत्रं विशाम्पते}


\twolineshloka
{तिष्ठेदानीं क्व मे जीवन्सूतपुत्र गमिष्यसि}
{युद्धश्रद्धामहं तेऽद्य विनेष्यामि रमाजिरे}


\threelineshloka
{इत्युक्त्वा रोषताम्राक्षं रक्षः क्रूरपराक्रमम्}
{उत्पपातान्तरिक्षं च जहास च सुविस्तरम्}
{कर्णमभ्यहनच्चैव गजेन्द्रमिव केसरी}


\twolineshloka
{रथाक्षमात्रेरिषुभिरभ्यवर्षद्धटोत्कचः}
{}


% Check verse!
शरवृष्टिं च तां कर्णो दूरात्प्राप्तामशातयत्
\twolineshloka
{दृष्ट्वा च विहतां मायां कर्णेन भरतर्षभ}
{घटोत्कचस्ततो मायां ससर्जान्तर्हितः पुनः}


\twolineshloka
{सोऽभवद्गिरिरत्युच्चः शिखरैस्तरुसङ्कटैः}
{शूलप्रासासिमुसलजलप्रस्रवणो महान्}


\twolineshloka
{तमञ्जनचयप्रख्यं कर्णो दृष्ट्वा महीधरम्}
{प्रपातैरायुधान्युग्राण्युद्वहन्तं न चुक्षुभे}


\twolineshloka
{स्मयन्निव ततः कर्णो दिव्यमस्त्रमुदैरयत्}
{ततः सोऽस्रेण शैलेन्द्रो विक्षिप्तो वै व्यनश्यत}


\twolineshloka
{ततः स तोयदो भूत्वा नीलः सेन्द्रायुधो दिवि}
{अश्मवृष्टिभिरत्युग्रः सूतपुत्रमवाकिरत्}


\twolineshloka
{अथ सन्धाय वायव्यमस्त्रमस्त्रविदां वरः}
{व्यधमत्कालमेघं तं कर्णो वैकर्तनो वृषः}


\twolineshloka
{स मार्गणगणैः कर्णो दिशः प्रच्छाद्य सर्वशः}
{जघानास्त्रं महाराज घटोत्कचसमीरितम्}


\twolineshloka
{ततः प्रहस्य समरे भैमसेनिर्महाबलः}
{प्रादुश्चक्रे महामायां कर्णं प्रति महारथम्}


\twolineshloka
{स दृष्ट्वा पुनरायान्तं रथेन रथिनां वरम्}
{घटोत्कचमसम्भ्रान्तं राक्षसैर्बहुभिर्वृतम्}


\threelineshloka
{सिंहशार्दूलसदृशैर्मत्तमातङ्गविक्रमैः}
{गजस्थैश्च रथस्थैश्च वाजिपृष्ठगतैस्तथा}
{नानाशस्त्रधरैर्घोरैर्नानाकवचभूषणैः}


\twolineshloka
{वृतं घटोत्कचं क्रूरैर्मरुद्भिरिव वासवम्}
{दृष्ट्वा कर्णो महेष्वासो योधयामास राक्षसम्}


\twolineshloka
{घटोत्कचस्ततः कर्णं विद्धा पञ्चभिराशुगैः}
{ननाद भैरवं नादं भीषयन्सर्वपार्थिवान्}


\twolineshloka
{भूयश्चाञ्जलिकेनाथ स मार्गणगणं महत्}
{कर्णहस्तस्थितं चापं चिच्छेदाशु घटोत्कचः}


\twolineshloka
{अथान्यद्धनुरादाय दृढं भारसहं महत्}
{विचकर्ष बलात्कर्ण इन्द्रायुधमिवोच्छ्रितम्}


\twolineshloka
{ततः कर्णो महाराज प्रेषयामास सायकान्}
{सुवर्णपुङ्खाञ्छत्रुघ्नान्खेचरान्राक्षसान्प्रति}


\twolineshloka
{तद्बाणैरर्दितं यूथं रक्षसां पीनवक्षसाम्}
{सिंहेनेवार्दितं वन्यं गजानामाकुलं कुलम्}


\twolineshloka
{विधम्य राक्षसान्बाणैः साश्वसूतगजान्विभुः}
{ददाह भगवान्वह्निर्भूतानीव युगक्षये}


\twolineshloka
{स हत्वा राक्षसीं सेनां शुशुभे सूतनन्दनः}
{पुरेव त्रिपुरं दग्ध्वा दिवि देवो महेश्वरः}


\twolineshloka
{तेषु राजसहस्रेषु पाण्डवेयेषु मारिष}
{नैनं निरीक्षितुमपि कश्चिच्छक्नोति पार्थिव}


\twolineshloka
{ऋते घटोत्कचाद्राजन्राक्षसेन्द्रान्महाबलात्}
{भीमवीर्यबलोपेतात्क्रुद्ध्द्वैवस्वतादिव}


\twolineshloka
{तस्य क्रुद्धस्य नेत्राभ्यां पावकः समजायत}
{महोल्काभ्यां यथा राजन्सार्जिषः स्नेहबिन्दवः}


\twolineshloka
{तलं तलेन संहत्य सन्दश्य दशनच्छदम्}
{रथमास्थाय च पुनर्मायया निर्मितं तदा}


\twolineshloka
{युक्तं गजनिभैर्वाहैः पिशाचवदनैः खरैः}
{स सूतमब्रवीत्क्रुद्धः सूतपुत्राय मां वह}


\twolineshloka
{स ययौ घोररूपेण रथेन रथिनां वरः}
{द्वैरथं सूतपुत्रेण पुनरेव विशाम्पते}


\twolineshloka
{स चिक्षेप पुनः क्रुद्धः सुतपुत्राय राक्षसः}
{अष्टचक्रां महाघोरामशनिं रुद्रनिर्मिताम्}


\twolineshloka
{द्वियोजनसमुत्सेधां योजनायामविस्तराम्}
{आयसीं निचितां शूलैः कदम्बमिव केशरैः}


\twolineshloka
{तामवप्लुत्य जग्राह कर्णो न्यस्य महद्धनुः}
{चिक्षेप चैनां तस्यैव स्यन्दनात्सोऽवपुप्लुवे}


\twolineshloka
{साश्वसूतध्वजं यानं भस्म कृत्वा महाप्रभा}
{विवेश वसुधां भित्त्वा सुरास्तत्र विसिस्मियुः}


\twolineshloka
{कर्णं तु सर्वभूतानि पूजयामासुरञ्जसा}
{यदप्लुत्य जग्राह देवसृष्टां महाशनिम्}


\twolineshloka
{एवं कृत्वा रणे कर्ण आरुरोह रथं पुनः}
{ततो मुमोच नाराचान्सूतपुत्रः परन्तपः}


\twolineshloka
{अशक्यं कर्तुमन्येन सर्वभूतेषु मानद}
{यदकार्षीत्तदा कर्णः सङ्ग्रामे भीमदर्शने}


\twolineshloka
{स हन्यमानो नाराचैर्धाराभिरिव पर्वतः}
{गन्धर्वनगराकारः पुनरन्तरधीयत}


\twolineshloka
{एवं स वै महाकायो मायया लाघवेन च}
{अस्त्राणि तानि दिव्यानि जघान रिपुसूदनः}


\twolineshloka
{निहन्यमानेष्वस्त्रेषु मायया तेन रक्षसा}
{असम्भ्रान्तस्तदा कर्णस्तद्रक्षः प्रत्ययुध्यत}


\twolineshloka
{ततः क्रुद्धो महाराज भैमसेनिर्महाबलः}
{चकार बहुधाऽऽत्मानं भीषयाणो महारथान्}


\twolineshloka
{ततो दिग्भ्यः समापेतुः सिंहव्याघ्रतरक्षवः}
{अग्निजिह्वाश्च भुजगा विहगाश्चाप्ययोमुखाः}


\twolineshloka
{स कीर्यमाणो विशिखैः कर्णचापच्युतैः शरैः}
{नागराडिव दुष्प्रेक्ष्यस्तत्रैवान्तरधीयत}


\twolineshloka
{राक्षसाश्च पिशाचाश्च यातुधानास्तथैव च}
{शालावृकाश्च बहवो वृकाश्च विकृताननाः}


\twolineshloka
{ते कर्णं क्षपयिष्यन्तः सर्वतः समुपाद्रवन्}
{अथैनं वाग्भिरुग्राभिस्त्रासयाञ्चिक्रिरे तदा}


\twolineshloka
{उद्यतैर्बहुभिर्घोरैरायुधैः शोणितोक्षितैः}
{तेषामनेकैरेकैकं कर्णो विव्याध सायकैः}


\twolineshloka
{प्रतिहत्य तु तां मायां दिव्येनास्त्रेण राक्षसीम्}
{आजघान हयानस्य शरैः सन्नतपर्वभिः}


\twolineshloka
{ते भग्ना विक्षताङ्गाश्च भिन्नपृष्ठाश्च सायकैः}
{वसुधामन्वपद्यन्त पश्यतस्तस्य रक्षसः}


\twolineshloka
{स भग्नमायो हैडिम्बिः कर्णं वैकर्तनं तदा}
{एष ते विदधे मृत्युमित्युक्त्वान्तरधीयत}


\chapter{अध्यायः १७७}
\twolineshloka
{सञ्जय उवाच}
{}


\twolineshloka
{तस्मिंस्तथा वर्तमाने कर्णराक्षसयोर्मृधे}
{अलायुधो राक्षसेन्द्रो वीर्यवानभ्यवर्तत}


\threelineshloka
{महत्या सेनया युक्तो दुर्योधनमुपागमत्}
{राक्षसानां विरूपाणां सहस्रैः परिवारितः}
{नानारूपधरैर्वीरैः पूर्ववैरमनुस्मरन्}


\twolineshloka
{तस्य ज्ञातिर्हि विक्रान्तो ब्राह्मणादो बको हतः}
{किर्णीरश्च महातेजा हिडिम्बश्च सखा तदा}


\twolineshloka
{स दीर्घकालाध्युपितं पूर्ववैरमनुस्मरन्}
{विज्ञायैतन्निशायुद्धं जिघांसुर्भीममाहवे}


\twolineshloka
{स मत्त इव मातङ्गः सङ्क्रुद्व इव चोरगः}
{दुर्योधनमिदं वाक्यमब्रवीद्युद्वलालसः}


\twolineshloka
{विदितं ते महाराज यथा भीमेन राक्षसाः}
{हिडिम्बबककिर्मीरा निहता मम बान्धवाः}


\twolineshloka
{परामर्शश्च कन्याया हिडिम्बायाः कृतः पुरा}
{किमन्यद्राक्षसानन्यानस्मांश्च परिभूय ह}


\twolineshloka
{तमहं सगणं राजन्सवाजिरथकुञ्जरम्}
{हैडिम्बिं च सहामात्यं हन्तुमभ्यागतः स्वयम्}


\twolineshloka
{अद्य कुन्तीसुतान्सर्वान्वासुदेवपुरोगमान्}
{हत्वा सम्भक्षयिष्यामि सर्वैरनुचरैः सह}


% Check verse!
निवारय बलं सर्वं वयं योत्स्याम पाण्डवान्
\twolineshloka
{तस्यैतद्वचनं श्रुत्वा हृष्टो दुर्योधनस्तदा}
{प्रतिपूज्याब्रवीद्वाक्यं भ्रातृभिः परिवारितः}


\twolineshloka
{त्वां पुरस्कृत्य सगणं वयं योत्स्यामहे परान्}
{न हि वैरान्तमनसः स्थास्यन्ति मम सैनिकाः}


\twolineshloka
{एवमस्त्विति राजानमुक्त्वा राक्षसपुङ्गवः}
{अभ्ययात्त्वरितो भैमिं सहितः पुरुषादकैः}


\twolineshloka
{दीप्यमानेन वपुषा रथेनादित्यवर्चसा}
{तादृशेनैव राजेन्द्र यादृशेन घटोत्कचः}


\twolineshloka
{तस्याप्यतुलनिर्घोपो बहुतोरणचित्रितः}
{ऋक्षचर्मावनद्वाङ्गो नल्वमात्रो महारथः}


\twolineshloka
{तस्यापि तुरगाः शीघ्रा हस्तिकायाः खरस्वनाः}
{शतं युक्ता महाकाया मांसशोणितभोजनाः}


\threelineshloka
{तस्यापि रथनिर्घोपो महामेघरवोपमः}
{स्यापि सुमहच्चापं दृढज्यं कनकोज्ज्वलम् ॥तस्याप्यक्षसमा बाणा रुक्म पुङ्खाः शिलाशिताः}
{सोऽपि वीरो महाबाहुर्यथैव स घटोत्कचः}


\twolineshloka
{तस्यापि गोमायुबलाभिगुप्तोबभूव केतुर्ज्वलनार्कतुल्यः}
{स चापि रूपेण घटोत्कचस्यश्रीमत्तमो ह्यन्तकसन्निकाशः}


\twolineshloka
{दीप्ताङ्गदो दीप्तकिरीटमालीबद्धस्रगुष्णीषनिबद्धखङ्गः}
{गदी भुशुण्डी मुसली हली चशरासनी वारणतुल्यवर्ष्मा}


\twolineshloka
{रथेन तेनानलवर्चसा तदाविद्रावयन्पाण्डववाहिनीं ताम्}
{रराज सह्ख्ये परिवर्तमानोविद्युन्माली मेघ इवान्तरिक्षे}


\twolineshloka
{ते चापि सर्वप्रवरा नरेन्द्रामहाबला वर्मिमश्चर्मिणश्च}
{हर्षान्विता युयुधुस्तस्य राजन्समन्ततः पाण्डवयोधवीराः}


\chapter{अध्यायः १७८}
\twolineshloka
{सञ्जय उवाच}
{}


\twolineshloka
{तमागतमभिप्रेक्ष्य भीमकर्माणमाहवे}
{हर्षमाहारयाञ्चक्रुः कुरवः सर्व एव ते}


\twolineshloka
{तथैव तव पुत्रास्ते दुर्योधनपुरोगमाः}
{अप्लुवाः प्लवमासाद्य तर्तुकामा इवार्णवम्}


\twolineshloka
{पुनर्जातमिवात्मानं मन्वानाः पुरुषर्षभाः}
{अलायुधं राक्षसेन्द्रं स्वागतेनाभ्यपूजयन्}


\twolineshloka
{तस्मिंस्त्वमानुषे युद्धे वर्तमाने महाभये}
{कर्णराक्षसयोर्नक्तं दारुणप्रतिदर्शने}


\twolineshloka
{`न द्रौणिर्न कृपद्रोणौ न शल्यो न च माधवः}
{एक एव तु तेनासीद्योद्धा कर्णो रणे वृषा'}


\twolineshloka
{उपप्रैक्षन्त पाञ्चालाः स्मयमानाः सराजकाः}
{व्यभ्रमंस्तावकाश्चापि घूर्णमानास्ततस्ततः}


\twolineshloka
{चुक्रुशुर्नेदमस्तीति द्रोणद्रौणिकृपादयः}
{तत्कर्म दृष्ट्वा सम्भ्रान्ता हैडिम्बस्य रणाजिरे}


\twolineshloka
{सर्वमाविग्नमभवद्धाहाभूतमचेतनम्}
{तव सैन्यं महाराज निराशं कर्णजीविते}


\twolineshloka
{दुर्योधनस्तु सम्प्रेक्ष्य कर्णमार्तिं परां गतम्}
{अलायुधं राक्षसेन्द्रं समाहूयेदमब्रवीत्}


\twolineshloka
{एष वैकर्तनः कर्णो हैडिम्बेन समागतः}
{कुरुते कर्म सुमहद्यदस्योपयिकं मृधे}


\twolineshloka
{पश्यैतान्पार्थिवाञ्शूरान्निहतान्भैमसेनिना}
{नानाशस्त्रैरभिहतान्पादपानिव दन्तिना}


\twolineshloka
{तवैष भारः समरे ज्ञातिमध्ये मया कृतः}
{तवैवानुमते वीर तं विक्रम्य निबर्हय}


\twolineshloka
{पुरा वैकर्तनं कर्णमेष पापो घटोत्कचः}
{मायाबलं समाश्रित्य कर्षयत्यरिकर्शन}


\twolineshloka
{एवमुक्तः स राज्ञा तु राक्षसो भीमविक्रमः}
{तथेत्युक्त्वा महाबाहुर्घटोत्कचमुपाद्रवत्}


\twolineshloka
{ततः कर्णं समुत्सृज्य भैमसेनिरपि प्रभो}
{प्रत्यमित्रमुपायान्तमर्दयामास मार्गणैः}


\twolineshloka
{तयोः समभवद्युद्धं क्रुद्धयो राक्षसेन्द्रयोः}
{मत्तयोर्वासिताहेतोर्दीपयोरिव कानने}


\twolineshloka
{रक्षसा विप्रमुक्तस्तु कर्णोऽपि रथिनां वरः}
{अभ्यद्रवद्भीमसेनं रथेनादित्यवर्चसा}


\twolineshloka
{तमायान्तमनादृत्य दृष्ट्वा ग्रस्तं घटोत्कचम्}
{अलायुधेन समरे सिंहेनेव गवां पतिम्}


\twolineshloka
{रथेनादित्यवपुषा भीमः प्रहरतां वरः}
{किरञ्छरौघान्प्रययावलायुधरथं प्रति}


\twolineshloka
{तमायान्तमभिप्रेक्ष्य स तदाऽलायुधः प्रभो}
{घटोत्कचं समुत्सृज्य भीमसेनं समाह्वयत्}


\twolineshloka
{तं भीमः सहसाऽभ्येत्य राक्षसान्तकरः प्रभो}
{सगणं राक्षसेन्द्रं तं शरवर्षैरवाकिरत्}


\twolineshloka
{तथैवालायुधो राजञ्शिलाधौतैरजिह्मगैः}
{अभ्यवर्षत कौन्तेयं पुनःपुनररिन्दम}


\twolineshloka
{तथा ते राक्षसाः सर्वे भीमसेनमुपाद्रवन्}
{नानाप्रहरणा भीमास्त्वत्सुतानां जयैषिणः}


\twolineshloka
{स ताड्यमानो बहुभिर्भीमसेनो महाबलः}
{पञ्चभिः पञ्चभिः सर्वांस्तानविध्यच्छितैः शरैः}


\twolineshloka
{ते वध्यमाना भीमेन राक्षसाः क्रूरबुद्धयः}
{विनेदुस्तुमुलान्नादान्दुद्रुवुस्ते दिशो दश}


\twolineshloka
{तांस्त्रास्यमानान्भीमेन दृष्ट्वा रक्षो मबहालम्}
{अभिदुद्राव वेगेन शरैश्चैनमवाकिरत्}


% Check verse!
तं भीमसेनः समरे तीक्ष्णाग्रैरक्षिणोच्छरैः
\twolineshloka
{अलायुधस्तु तानस्तान्भीमेन विशिखान्रणे}
{चिच्छेद कांश्चित्समरे त्वरया कांश्चिदग्रहीत्}


\twolineshloka
{स तं दृष्ट्वा राक्षसेन्द्रं भीमो भीमपराक्रमः}
{गदां चिक्षेप वेगेन वज्रपातोपमां तदा}


\twolineshloka
{तामापतन्तीं वेगेन गदां ज्वालाकुलां ततः}
{गदया ताडयामास सा गदा भीममाव्रजत्}


\twolineshloka
{स राक्षसेन्द्रं कौन्तेयः शरवर्षैरवाकिरत्}
{तानप्यस्याकरोन्मोघान्राक्षसो निशितैः शरैः}


\twolineshloka
{ते चापि राक्षसाः सर्वे रजन्यां भीमरूपिणः}
{शासनाद्राक्षसेन्द्रस्य निजघ्नू रथकुञ्जरान्}


\twolineshloka
{पाञ्चालाः सृञ्जयाश्चैव वाजिनः परमद्विपाः}
{न शान्तिं लेभिरे तत्र राक्षसैर्भृशपीडिताः}


\twolineshloka
{तं तु दृष्ट्वा महाघोरं वर्तमानं महाहवम्}
{अब्रवीत्पुण्डरीकाक्षो धनञ्जयमिदं वचः}


\twolineshloka
{पश्य भीमं महाबाहुं राक्षसेन्द्रवशं गतम्}
{पदमस्यानुगच्छ त्वं मा विचारय पाण्डव}


\twolineshloka
{धृष्टद्युम्नः शिखण्डी च युधामन्यूत्तमौजसौ}
{सहितौ द्रौपदेयाश्च कर्णं यान्तु महारथाः}


\twolineshloka
{नकुलः सहदेवश्च युयुधानश्च वीर्यवान्}
{इतरान्राक्षसान्घ्नन्तु शासनात्तव पाण्डव}


\twolineshloka
{त्वमपीमां महाबाहो चमूं द्रोणपुरस्कृताम्}
{मारयस्व नरव्याघ्र महद्धि भयमागतम्}


\twolineshloka
{एवमुक्ते तु कृष्णेन यथोद्दिष्टा महारथाः}
{जग्मुर्वैकर्तनं कर्णं राक्षसांश्चैव तान्रणे}


\twolineshloka
{अथ पूर्णायतोत्सृष्टैः शरैराशीविषोपमैः}
{धनुश्चिच्छेद भीमस्य राक्षसेन्द्रः प्रतापवान्}


\twolineshloka
{हयांश्चास्य शितैर्बाणैः सारथिं च महाबलः}
{जघान मिषतः सङ्ख्ये भीमसेनस्य राक्षसः}


\twolineshloka
{सोऽवतीर्य रथोपस्थाद्धताश्वो हतसारथिः}
{तस्मै गुर्वीं गदां घोरां विनदन्नुत्ससर्ज ह}


\twolineshloka
{ततस्तां भीमनिर्घोषामापतन्तीं महागदाम्}
{गदया राक्षसो घोरो निजघान ननाद च}


\twolineshloka
{तद्दृष्ट्वा राक्षसेन्द्रस्य घोरं कर्म भयावहम्}
{भीमसेनः प्रहृष्टात्मा गदामाशु परामृशत्}


\twolineshloka
{तयोः समभवद्युद्धं तुमुलं नररक्षसोः}
{गदानिपातसंहादैर्भुवं कम्पयतोर्भृशम्}


\twolineshloka
{गदाविमुक्तौ तौ भूयः समासाद्येतरेतरम्}
{मुष्टिभिर्वज्रसंहादैरन्योन्यमभिजघ्नतुः}


\twolineshloka
{रथचक्रैर्युगैरक्षैरधिष्ठानैरुपस्करैः}
{यथासन्नमुपादाय निजघ्नतुरमर्षणौ}


\twolineshloka
{तौ विक्षरन्तौ रुधिरं समासाद्येतरेतरम्}
{मत्ताविव महानागौ चकृपाते पुनः पुनः}


\twolineshloka
{तावपश्यद्वृषीकेशः पाण्डवानां हिते रतः}
{स भीमसेनरक्षार्थं हैडिम्बिं पर्यचोदयत्}


\chapter{अध्यायः १७९}
\twolineshloka
{सञ्जय उवाच}
{}


\twolineshloka
{संदृश्य समरे भीमं रक्षसा ग्रस्तमन्तिकात्}
{वासुदेवोऽव्रवीद्राजन्घटोत्कचमिदं वचः}


\twolineshloka
{पश्य भीमं महाबाहो रक्षसा ग्रस्तमाहवे}
{पश्यतां सर्वसैन्यानां तव चैव महाद्युते}


\twolineshloka
{स कर्णं त्वं समुत्सृज्य राक्षसेन्द्रमलायुधम्}
{जहि क्षिप्रं महाबाहो पश्चात्कर्णं बधिष्यसि}


\twolineshloka
{स वार्ष्णेयवचः श्रुत्वा कर्णमुत्सृज्य वीर्यवान्}
{युयुधे राक्षसेन्द्रेण बकभ्रात्रा घटोत्कचः}


\twolineshloka
{तयोः सुतुमुलं युद्धं बभूव निशि रक्षसोः}
{अलायुधस्य चैवोग्रं हेडिम्बेश्चापि भारत}


\twolineshloka
{अलायुधस्य योधांश्च राक्षसान्भीमदर्शनान्}
{वेगेनापततः शूरान्प्रगृहीतशरासनान्}


\twolineshloka
{आत्तायुधः सुसङ्क्रुद्धो युयुधानो महारथः}
{नकुलः सहदेवश्च चिच्छिदुर्निशितैः शरैः}


\twolineshloka
{सर्वांश्च समरे राजन्किरीटी क्षत्रियर्षभान्}
{परिचिक्षेप बीभत्सुः सर्वतः प्रकिरञ्छरान्}


\twolineshloka
{कर्णश्च समरे राजन्व्यद्रावयत पार्थिवान्}
{धृष्टद्युम्नशिखण्ड्यादीन्पाञ्चालानां पहारथान्}


\twolineshloka
{तान्वध्यमानान्दृष्ट्वाऽथ भीमो भीमपराक्रमः}
{अभ्ययात्त्वरितः कर्णं विशिखान्प्रकिरन्रणे}


\threelineshloka
{ततस्तेऽप्याययुर्हत्वा राक्षसान्यत्र सूतजः}
{नकुलः सहदेवश्च सात्यकिश्च महारथः}
{ते कर्णे योधयामासुः पाञ्चाला द्रोणमेव तु}


\twolineshloka
{अलायुधस्तु सङ्क्रुद्धो घटोत्कचमरिन्दमम्}
{परिघेणातिकायेन ताडयामास मूर्धनि}


\twolineshloka
{स तु तेन प्रहारेण भैमसेनिर्महाबलः}
{ईषन्मूर्छितमात्मानमस्तम्भयत वीर्यवान्}


\twolineshloka
{ततो दीप्ताग्निसङ्काशां शतघण्टामलङ्कृताम्}
{चिक्षेप तस्मै समरे गदां काञ्चनभूषिताम्}


\twolineshloka
{सा हयांश्च रथं चास्य सारथिं च महास्वना}
{चूर्णयामास वेगेन विसृष्टा भीमकर्मणा}


\twolineshloka
{स भग्नहयचक्राक्षाद्विशीर्णध्वजकूबरात्}
{उत्पपात रथात्तूर्णं मायामास्थाय राक्षसीम्}


\twolineshloka
{स समास्थाय मायां तु ववर्ष रुधिरं बहु}
{विद्युद्विभ्राजितं चासीत्तुमुलाभ्राकुलं नभः}


\twolineshloka
{ततो वज्रनिपाताश्च साशनिस्तनयित्नवः}
{महांश्चटचटाशब्दस्तत्रासीच्च महाहवे}


\twolineshloka
{तां प्रेक्ष्य महतीं मायां राक्षसो राक्षसस्य च}
{ऊर्ध्वमुत्पत्य हैडिम्बिस्तां मायां माययाऽवधीत्}


\twolineshloka
{सोऽभिवीक्ष्य हतां मायां मायावी माययैव हि}
{अश्मवर्षं सुतुमुलं विससर्ज घटोत्कचे}


\twolineshloka
{अश्मवर्षं स तं घोरं शरवर्षेण वीर्यवान्}
{दिक्षु विध्वंसयामास तदद्भुतमिवाभवत्}


\twolineshloka
{ततो नानाप्रहरणैरन्योन्यमभिवर्षताम्}
{आयसैः परिघैः शूलैर्गदामुसलमुद्गरैः}


\twolineshloka
{पिनाकैः करवालैश्च तोमरप्रासकम्पनैः}
{नाराचैर्निशितैर्भल्लैः शरैश्चक्रैः परश्वथैः}


\twolineshloka
{अयोगुडैर्भिण्डिपालैर्गोशीर्षोलूखलैरपि}
{उत्पाटितैर्महाशाखैर्विविधैर्जगतीरुहैः}


\twolineshloka
{शमीपीलूकदम्बैश्च चम्पकैश्चैव भारत}
{इङ्गुदैर्बदरीभिश्च कोविदारैश्च पुष्पितैः}


\twolineshloka
{पलाशैश्चारिमेदैश्च प्लुक्षन्यग्रोधपिप्पलैः}
{महद्भिः समरे तस्मिन्न्योन्यमभिजघ्नतुः}


\twolineshloka
{विपुलैः पर्वताग्रैश्च नानाधातुभिराचितैः}
{तेषां शब्दो महानासीद्वज्राणां भिद्यतामिव}


\twolineshloka
{युद्धं समभवद्धोरं भैम्यलायुधयोर्नृप}
{हरीन्द्रयोर्यथा राजन्वालिसुग्रीवयोः पुरा}


\twolineshloka
{तौ युद्ध्वा विविधैर्घोरैरायुधैर्विशिखैस्तथा}
{उत्सृज्य च शितौ खङ्खागवन्योमभिपेततुः}


\twolineshloka
{तावन्योन्यमभिद्रुत्य केशेषु सुमहाबलौ}
{भुजाभ्यां पर्यगृह्णीतां महाकायौ महाबलौ}


\twolineshloka
{तौ स्विन्नगात्रौ प्रस्वेदं सुस्रुवाते जनाधिप}
{रुधिरं च महाकायावतिवृष्टाविवाम्बुदौ}


\twolineshloka
{अथाभिपत्य वेगेन समुद्धाम्य च राक्षसम्}
{बलेनाक्षिप्य हैडिम्बिश्चकर्तास्य शिरो महत्}


\twolineshloka
{सोऽपहृत्य शिरस्तस्य कुण्डलाभ्यां विभूपितम्}
{तदा सुतुमुलं नादं ननाद सुमहाबलः}


\twolineshloka
{हतं दृष्ट्वा महाकायं बकज्ञातिमरिन्दमम्}
{पाञ्चालाः पाण्डवाश्चैव सिंहनादान्विनेदिरे}


\twolineshloka
{ततो भेरीसहस्राणि शङ्खानामयुतानि च}
{अवादयन्पाण्डवेया राक्षसे निहते युधि}


\twolineshloka
{अतीव सा निशा तेषां बभूव विजयावहा}
{विद्योतमाना विबभौ समन्ताद्दीपमालिनी}


\twolineshloka
{अलायुधस्य तु शिरो भैमसेनिर्महाबलः}
{दुर्योधनस्य प्रमुखे चिक्षेप गतचेतसः}


\twolineshloka
{अथ दुर्योधनो राजा दृष्ट्वा हतमलायुधम्}
{बभूव परमोद्विग्नः सह सैन्येन भारत}


\twolineshloka
{तेन ह्यस्य प्रतिज्ञातं भीमसेनमहं युधि}
{हन्तेति स्वयमागम्य स्मरता वैरमुत्तमम्}


\twolineshloka
{ध्रुवं स तेन हन्तव्य इत्यमन्यत पार्थिवः}
{जीवितं चिरकालं भ्रातॄणां चाप्यमन्यत}


\twolineshloka
{स तं दृष्ट्वा विनिहतं भीमसेनात्मजेन वै}
{प्रतिज्ञां भीमसेनस्य पूर्णामेवाभ्यमन्यत}


\chapter{अध्यायः १८०}
\twolineshloka
{सञ्जय उवाच}
{}


\twolineshloka
{निहत्यालायुधं रक्षः प्रहृष्टात्मा घटोत्कचः}
{ननाद विविधान्नादान्वाहिन्या प्रमुखे तव}


\twolineshloka
{तस्य तं तुमुलं शब्दं श्रुत्वा हृदयकम्पनम्}
{तावकानां महाराज भयमासीत्सुदारुणम्}


\twolineshloka
{अलायुधविषक्तं तु भैमसेनिं महाबलम्}
{दृष्ट्वा कर्णो महाबाहुः पाञ्चालान्समुपाद्रवत्}


\twolineshloka
{दशभिर्दशभिर्बाणैर्धृष्टद्युम्नशिखण्डिनौ}
{दृढैः पूर्णायतोत्सृष्टैर्बिभेद नतपर्वभिः}


\twolineshloka
{ततः परमनाराचैर्युधामन्यूत्तमौजसौ}
{सात्यकिं च रथोदारं कम्पयमास मार्गणैः}


\twolineshloka
{तेषामप्यस्यतां सङ्ख्ये सर्वेषां सव्यदक्षिणम्}
{मण्डलान्येव चापानि व्यदृश्यन्त जनाधिप}


\twolineshloka
{तेषां ज्यातलनिर्घोषो रथनेमिस्वनश्च ह}
{मेघानामिव घर्मान्ते बभूव तुमुलो निशि}


\twolineshloka
{ज्यानेमिघोषस्तनयित्नुमान्वैधनुस्तटिन्मण्डलकेतुशृङ्गः}
{शरौघवर्षाकुलवृष्टिमांश्चसङ्क्राममेघः स बभूव राजन्}


\twolineshloka
{तदद्भुतं शैल इवाप्रकम्पोवर्षं महाशैलसमानसारः}
{विध्वंसयामास रणे नरेनद्रवैकर्तनः शत्रुगणावमर्दी}


\twolineshloka
{ततोऽतुलैर्वज्रनिपातकल्पैःशितैः शरैः काञ्चनचित्रपुङ्खैः}
{शत्रून्व्यपोहत्समरे महात्मावैकर्तनः पुत्रहिते रतस्ते}


\twolineshloka
{सञ्छिन्नभिन्नध्वजिनश्च केचि--त्केचिच्छरैरर्दितभिन्नदेहाः}
{केचिद्विसूता विहयाश्च केचि--द्वैकर्तनेनाशु कृता बभूवुः}


\twolineshloka
{अविन्दमानास्त्वथ शर्म सङ्ख्येयौधिष्ठिरं ते बलमभ्यपद्यन्}
{तान्प्रेक्ष्य भग्नान्विमुखीकृतांश्चघटोत्कचो रोषमतीव चक्रे}


\twolineshloka
{आस्थाय तं काञ्चनरत्नचित्रंरथोत्तमं सिंहवत्सन्ननाद}
{वैकर्तनं कर्णमुपेत्य चापिविव्याध वज्रप्रतिमैः पृषत्कैः}


\twolineshloka
{तौ कर्णिनाराचशिलीमुखैश्चनालीकदण्डासनवत्सदन्तैः}
{वराहकर्णैः सविपाठशृङ्गै-क्षुरप्रवर्षैश्च विनेदतुः खम्}


\twolineshloka
{तद्बाणधारावृतमन्तरिक्षंतिर्यग्गताभिः समरे रराज}
{सुवर्णपुङ्खज्वलितप्रभाभि--र्विचित्रपुष्पाभिरिव प्रजाभिः}


\twolineshloka
{समाहितावप्रतिमप्रभावा--वन्योन्यमाजघ्नतुरुत्तमास्त्रैः}
{तयोर्हि वीरोत्तमयोर्न कश्चि--द्ददर्श तस्मिन्समरे विशेषम्}


\threelineshloka
{अतीव तच्चित्रमतुल्यरूपंबभूव युद्धं रविभीमसून्वोः}
{समाकुलं शस्त्रनिपातघोरंदिवीव राह्वंशुमतोः प्रमत्तम् ॥सञ्जय उवाच}
{}


\twolineshloka
{घटोत्कचं यदा कर्णो न विशेषयते नृप}
{ततः प्रादुश्चकारोग्रमत्त्रमस्त्रविदां वरः}


\threelineshloka
{तेनास्त्रेणावधीत्तस्य रथं सहयसारथिम्}
{विरथश्चापि हैडिम्बिः क्षिप्रमन्तरधीयत ॥धृतराष्ट्र उवाच}
{}


\threelineshloka
{तस्मिन्नन्तर्हिते तूर्णं कुटयोधिनि राक्षसे}
{मामकैः प्रतिपन्नं यत्तन्ममाचक्ष्व सञ्जय ॥सञ्जय उवाच}
{}


\twolineshloka
{अन्तर्हितं राक्षसेन्द्रं विदित्वासम्प्राक्रोशन्कुरवः सर्व एव}
{कथं नायं राक्षसः कूटयोधीहन्यात्कर्णं समरेऽदृश्यमानः}


\twolineshloka
{ततः कर्णो लघुचित्रास्त्रयोधीसर्वा दिशः प्रावृणोद्बाणजालैः}
{न वै किञ्चित्प्रापतत्तत्र भूतंतमोभूते सायकैरन्तरिक्षे}


\twolineshloka
{नैवाददानो न च सन्दधानोन चेषुधीस्पृशमानः कारग्रैः}
{अदृश्यद्वै लाघवात्सूतपुत्रःसर्वं बाणैश्छादयानोऽन्तरिक्षम्}


\twolineshloka
{ततो मायां दारुणामन्तरिक्षेघोरां भीमां विहितां राक्षसेन}
{अपश्याम लोहिताभ्रप्रकाशांदेदीप्यन्तीमग्निशिखामिवोग्राम्}


\twolineshloka
{ततस्तस्यां विद्युतः प्रादुरास--न्नुल्काश्चापि ज्वलिताः कौरवेन्द्र}
{घोपश्चास्यः प्रादुरासीत्सुघोरःसहस्रशो नदतां दुन्दुभीनाम्}


\twolineshloka
{ततः शराः प्रापतन्रुक्मपुङ्खाःशक्त्यृष्टिप्रासमुसलान्यायुधानि}
{परश्वथास्तैलधौताश्च खङ्गाःप्रदीप्ताग्रास्तोमराः पट्टसाश्च}


\twolineshloka
{मयूखिनः परिघा लोहबद्धागदाश्चित्राः शितधाराश्च शूलाः}
{गुर्व्यो गदा हेमपट्टावनद्धाःशतघ्न्यश्च प्रादुरासन्समन्तात्}


\twolineshloka
{महाशिलाश्चापतंस्तत्रतत्रसहस्रशः साशनयश्च वज्राः}
{चक्राणि चानेकशतक्षुराणिप्रादुर्बभूवुर्ज्वलनप्रभाणि}


\twolineshloka
{तां शक्तिपाषाणपरश्वथानांप्रासासिवज्राशनिमुद्गराणाम्}
{वृष्टिं विशालां ज्वलितां पतन्तींकर्णः शरौघैर्न शशाक हन्तुम्}


\twolineshloka
{शराहतानां पततां हयानांवज्राहतानां च तथा गजानाम्}
{शिलाहतानां च महारथानांमहान्निनादः पततां बभूव}


\twolineshloka
{सुभीमनानाविधशस्त्रपातै--र्घटोत्कचेनाभिहतं समन्तात्}
{दौर्योधनं वै बलमार्तरूप--मावर्तमानं ददृशे भ्रमत्तत्}


\twolineshloka
{हाहाकृतं सम्परिवर्तमानं संलीयमानं च विषण्णरूपम्}
{ते त्वार्यभावात्पुरुषप्रवीराःपराङ्मुखा नो बभूवुस्तदानीम्}


\twolineshloka
{तां राक्षसीं भीमरूपां सुघोरांवृष्टिं महाशस्त्रमयीं पतन्तीम्}
{दृष्ट्वा बलौघांश्च निपात्यमाना--न्महद्भयं तव पुत्रान्विवेश}


\twolineshloka
{शिवाश्च वैश्वानरदीप्तजिह्वाःसुभीमनादाः शतशो नदन्तीः}
{रक्षोगणान्नर्दतश्चापि वीक्ष्यनरेन्द्र योधा व्यथिता बभूवुः}


\twolineshloka
{ते दीप्तजिह्वानलतीक्ष्णदंष्ट्राविभीषणाः शैलनिकाशकायाः}
{नभोगताः शक्तिविषक्तहस्तामेघा व्यमुञ्चन्निव वृष्टिमुग्राम्}


\twolineshloka
{तैराहतास्ते शरशक्तिशूलै--र्गदाभिरुग्रैः परिघैश्च दीप्तैः}
{वज्रैः पिनाकैरशनिप्रहारैःशतघ्निचक्रैर्भथिताश्च पेतुः}


\twolineshloka
{शूला भुशुण्ड्योऽश्मगुडाः शतघ्न्यःस्थूलाश्च कार्ष्णायसपट्टनद्वाः}
{तेऽवाकिरंस्तव पुत्रस्य सैन्यंततो रौद्रं कश्मलं प्रादुरासीत्}


\twolineshloka
{विकीर्णान्त्रा विहतैरुत्तमाङ्गैःसम्भग्नाङ्गः शिश्यिरे तत्र शूराः}
{छिन्ना हयाः कुञ्जराश्चापि भग्नान मुच्यन्ते याचमानाः सुभीताः}


\twolineshloka
{एवं महच्छस्त्रवर्षं सृजन्त--स्ते यातुधाना भुवि घोररूपाः}
{मायाः सृष्टास्तत्र घटोत्कचेननामुञ्चन्वै याचमानं न भीतम्}


\twolineshloka
{तस्मिन्घोरे कुरुवीरावमर्देकालोत्सृष्टे क्षत्रियाणामभावे}
{ते वै भग्नाः सहसा व्यद्रवन्तप्राक्रोशन्तः कौरवाः सर्व एव}


\twolineshloka
{पलायध्वं कुरवो नैतदस्तिसेन्द्रा देवा घ्नन्ति नः पाण्डवार्थे}
{तथा तेषां मज्जतां भारतानांतस्मिन्द्वीपः सूतपुत्रो बभूव}


\twolineshloka
{तस्मिन्सङ्क्रन्दे तुमुले वर्तमानेसैन्ये भग्ने लीयमाने कुरूणाम्}
{अन्नीकानां प्रविभागे प्रकाशेन ज्ञायन्ते कुरवो नेतरे च}


\twolineshloka
{निर्मर्यादे विद्रवे घोररूपेसर्वा दिशः प्रेक्षमाणाः स्म शून्याः}
{तां शस्त्रवृष्टिमुरसा गाहमानंकर्णं स्मैकं तत्र राजन्नपश्यन्}


\twolineshloka
{ततो बाणैरावृणोदन्तरिक्षंदिव्यां मायां योधयन्राक्षसस्य}
{हीमान्कुर्वन्दुष्करं चार्यकर्मनैवामुह्यत्संयुगे सूतपुत्रः}


\twolineshloka
{ततो भीताः समुदैक्षन्त कर्णंराजन्सर्वे सैन्धवा बाह्लिकाश्च}
{असम्मोहं पूजयन्तोऽस्य सङ्ख्येसम्पश्यन्तो विजयं राक्षसस्य}


\twolineshloka
{तेनोत्सृष्टा चक्रयुक्ता शतघ्नीसमं सर्वांश्चतुरोऽश्वाञ्जघान}
{ते जानुभिर्जगतीमन्वपद्य--न्गतासवो निर्दशनाक्षिजिह्वाः}


\twolineshloka
{ततो हताश्वादवरुह्य याना--दन्तर्मनाः कुरुषु प्राद्रुवत्सु}
{दिव्ये चास्त्रे मायया वध्यमानेनैवामुह्यच्चिन्तयन्त्प्राप्तकालम्}


\twolineshloka
{ततोऽब्रुवन्कुरवः सर्व एवकर्णं दृष्ट्वा घोररूपां च मायाम्}
{शक्त्या रक्षो जहि कर्णाद्य तूर्णंनश्यन्त्येते कुरवो धार्तराष्ट्राः}


\twolineshloka
{करिष्यतः किञ्च नो भीमपार्थौतपन्तमेनं जहि पापं निशीथे}
{यो नः सङ्ग्रामाद्धोररूपाद्विमुच्ये--त्स नः पार्थान्सबालान्योधयेत}


\twolineshloka
{तस्मादेनं राक्षसं घोररूपंशक्त्या जहि त्वं दत्तया वासवेन}
{मा कौरवाः सर्व एवेन्द्रकल्पारात्रियुद्धे कर्ण नेशुः सयोधाः}


\twolineshloka
{स वध्यमानो रक्षसा वै निशीथेदृष्ट्वा राजंस्त्रास्यमानं बलं च}
{महच्छ्रुत्वा निनदं कौरवाणांमतिं दध्रे शक्तिमोक्षाय कर्णः}


\twolineshloka
{स वै क्रुद्धः सिंह इवात्यमर्षीनामर्षयन्प्रतिघातं रणेऽसौ}
{शक्तिं श्रेष्ठां वैजयन्तीमसह्यांसमाददे तस्य वधं चिकीर्षन्}


\twolineshloka
{याऽसौ राजन्निहिता वर्षपूगा--न्वधायाजौ सत्कृता फल्गुनस्य}
{यां वै प्रादात्सूत पुत्राय शक्रःशक्तिं श्रेष्ठां कुण्डलाभ्यां निमाय}


\twolineshloka
{तां वै शक्तिं लेलिहानां प्रदीप्तांपाशैर्युक्तामन्तकस्येव जिह्वाम्}
{मृत्योः स्वसारं ज्वलितामिवोल्कांवैकर्तनः प्राहिणोद्राक्षसाय}


\twolineshloka
{तामुत्तमां परकायावहन्त्रींदृष्ट्वा शक्तिं बाहुसंस्थां ज्वलन्तीम्}
{भीतं रक्षो विप्रदुद्राव राजन्कृत्वाऽऽत्मानं विन्ध्यतुल्यप्रमाणम्}


\twolineshloka
{दृष्ट्वा शक्तिं कर्णबाह्वन्तरस्थांनेदुर्भूतान्यन्तरिक्षे नरेन्द्र}
{ववुर्वातास्तुमुलाश्चापि राज--न्सनिर्घाता चाशनिर्गां जगाम}


\twolineshloka
{सा तां मायां भस्म कृत्वा ज्वलन्तीभित्त्वा गाढं हृदयं राक्षसस्य}
{ऊर्ध्वं ययौ दीप्यमाना निशायांनक्षत्राणामन्तराण्याविवेश}


\twolineshloka
{स निर्भिन्नो विविधैरस्त्रपूगै--र्दिव्यैर्नागैर्मानुषै राक्षसैश्च}
{नदन्नादान्विविधान्भैरवांश्चप्राणानिष्टांस्त्याजितः शक्रशक्त्या}


\twolineshloka
{इदं चान्यच्चित्रमाश्चर्यरूपंचकारासौ कर्म शत्रुक्षयाय}
{तस्मिन्काले शक्तिनिर्भिन्नमर्माबभौ राजञ्शैलमेघप्रकाशः}


\twolineshloka
{ततोऽन्तरिक्षादपतद्गतासुःस राक्षसेन्द्रो भुवि भिन्नदेहः}
{अवाक्शिराः स्तब्धगात्रो विजिह्वोघटोत्कचो महदास्थाय रूपम्}


\twolineshloka
{स तद्रूपं भैरवं भीमकर्माभीमं कृत्वा भैमसेनिः पपात}
{हतोऽप्येवं तव सैन्यैकदेश--मपोथयत्स्वेन देहेन राजन्}


\twolineshloka
{पतद्रक्षः स्वेन कायेन तूर्ण--मतिप्रमाणेन विवर्धता च}
{प्रियं कुर्वन्पाण्डवानां गतासु--रक्षौहिणीं तव तूर्णं जघान}


\twolineshloka
{ततो मिश्राः प्राणदन्सिंहनादै--र्भेर्यः शङ्खा मुरजाश्चानकाश्च}
{दग्धां मायां निहतं राक्षसं चदृष्ट्वा हृष्ट्वाः प्राणदन्कौरवेयाः}


\twolineshloka
{ततः कर्णः कुरुभिः पूज्यमानोयथा शक्रो वृत्रवधे मरुद्भिः}
{अन्वारूढस्तव पुत्रस्य यानंहृष्टश्चापि प्राविशत्तत्स्वसैन्यम्}


\chapter{अध्यायः १८१}
\twolineshloka
{`धृतराष्ट्र उवाच}
{}


\twolineshloka
{तस्मिन्हते महामाये महातेजसि राक्षसे}
{अमर्षिताः पाण्डवेयाः किमकुर्वन्महारणे}


% Check verse!
मर्दिताश्च भृशं युद्धे किमकुर्वन्त सञ्जय
\twolineshloka
{ये च तेऽभ्यद्रवन्द्रोणं व्यूढानीकं प्रहारिणः}
{सृञ्जयाः सह पाञ्चालैस्तेऽरुर्वन्किं महारणे}


\twolineshloka
{सौमदत्तिवधाद्द्रोणमायत्तं सैन्धवस्य च}
{अमर्षाज्जीवितं त्यक्त्वा गाहमानं वरूथिनीम्}


\twolineshloka
{जृम्भमाणमिव व्याघ्रं व्यात्ताननमिवान्तकम्}
{कथं प्रत्युद्ययुर्द्रोणमजय्यं कुरुसृञ्जयाः}


\twolineshloka
{आचार्यं ये च रक्षन्ति दुर्योधनपुरोगमाः}
{द्रौणिकर्णकृपास्तात ते ह्यकुर्वन्किमाहवे}


\twolineshloka
{भारद्वाजं जिघांसन्तौ सव्यसाचिवृकोदरौ}
{आर्च्छतां मामकान्युद्धे कथं सञ्जय शंस मे}


\threelineshloka
{सिन्धुराजवधेनेमे घटोत्कचवधेन ते}
{अमर्षिताश्च सङ्क्रुद्धा रणं चक्रुः कथं युधि' ॥सञ्जय उवाच}
{}


\twolineshloka
{हैडिम्बिं निहतं दृष्ट्वा विशीर्णमिव पर्वतम्}
{बभूवुः पाण्डवाः सर्वे शोकबाष्पाकुलेक्षणाः}


\twolineshloka
{वासुदेवस्तु हर्षेण महताऽभिपरिप्लुतः}
{ननाद सिंहवन्नादं व्यथयन्निव भारत}


\twolineshloka
{हतं घटोत्कचं ज्ञात्वा वासुदेवः प्रतापवान्}
{विनद्य च महानादं पर्यष्वजत फल्गुनम्}


\twolineshloka
{स विनद्य महानादमभीशून्सन्नियम्य च}
{ननर्त हर्षसंवीतो वातोद्धूत इव द्रुमः}


\twolineshloka
{ततः परिष्वज्य पुनः पार्थमास्फोट्य चासकृत्}
{रथोपस्थगतो धीमान्प्राणदत्पुनरच्युतः}


\twolineshloka
{प्रहृष्टमनसं ज्ञात्वा वासुदेवं महाबलः}
{अर्जुनोऽथाब्रवीद्राजन्नातिहृष्टमना इव}


\twolineshloka
{अतिहर्षोऽयमस्थाने तवाद्य मधुसूदन}
{शोकस्थाने तु सम्प्राप्ते हैडिम्बस्य वधेन तु}


\twolineshloka
{विमुखानीह सैन्यानि हतं दृष्ट्वा घटोत्कचम्}
{वयं च भृशमुद्विग्ना हेडिम्बेस्तु निपातनात्}


\twolineshloka
{नैतत्कारणमल्पं हि भविष्यति जनार्दन}
{तदद्य शंस मे पृष्टः सत्यं सत्यवतां वर}


\twolineshloka
{यद्येतन्न रहस्यं ते वक्तुमर्हस्यरिन्दम}
{धैर्यस्य विकृतिं ब्रूहि त्वमद्य मधुसूदन}


\threelineshloka
{समुद्रस्येव संशोषं मेरोरिव विसर्पणम्}
{तथैतदद्य मन्येऽहं तव कर्म जनार्दन ॥वासुदेव उवाच}
{}


\twolineshloka
{अतिहर्षमिमं प्राप्तं शृणु मे त्वं धनञ्जय}
{अतीव मनसः सद्यः प्रसादकरमुत्तमम्}


\twolineshloka
{शक्तिं घटोत्कचेनेमां व्यंसयित्वा महाद्युते}
{कर्णं निहतमेवाजौ विद्धि सद्यो धनञ्जय}


\twolineshloka
{शक्तिहस्तं पुनः कर्णं को लोकेऽस्ति पुमानिह}
{य एनमभितस्तिष्ठेत्कार्तिकेयमिवाहवे}


\twolineshloka
{दिष्ट्याऽपनीतकवचो दिष्ट्याऽपहृतकुण्डलः}
{दिष्ट्या सा व्यंसिता शक्तिरमोघाऽस्य घटोत्कचे}


\twolineshloka
{यदि हि स्यात्सकवचस्तथैव स्यात्सकुण्डलः}
{सामरानपि लोकांस्त्रीनेकः कर्णो जयेद्रणे}


\twolineshloka
{वासवो वा कुबेरो वा वरुणो वा जलेश्वरः}
{यमो वा नोत्सहेत्कर्णं रणे प्रतिसमासितुम्}


\twolineshloka
{गाण्डीवमुद्यम्य भवांश्चक्रं चाहं सुदर्शनम्}
{न शक्तौ स्वो रणे जेतुं तथायुक्तं नरर्षभम्}


\twolineshloka
{त्वद्धितार्थं तु शक्रेण मायापहृतकुण्डलः}
{विहीनकवचश्चायं कृतः परपुरञ्जयः}


\twolineshloka
{उत्कृत्य कवचं यस्मात्कुण्डले विमले च ते}
{प्रादाच्छक्राय वै कर्णस्तस्माद्वैकर्तनः स्मृतः}


\twolineshloka
{आशिविष इव क्रुद्वः स्तम्भितो मन्त्रतेजसा}
{तथाऽद्य भाति कर्णो मे शान्तज्वाल इवानलः}


\twolineshloka
{यदाप्रभृति कर्णाय शक्तिर्दत्ता महात्मना}
{वासवेन महाबाहो क्षिप्ता याऽसौ घटत्कचे}


\twolineshloka
{कुण्डलाभ्यां निमायाथ दिव्येन कवचेन च}
{तां प्राप्यामन्यत वृषा सततं त्वां हतं रणे}


\twolineshloka
{एवं गतोऽपि शक्योऽयं हतुं नान्येन केनचित्}
{ऋते त्वां पुरुषव्याघ्र शपे सत्येन चानघ}


\twolineshloka
{ब्रह्मण्यः सत्यवादी च सपस्वी नियतव्रतः}
{रिपुष्वपि दयावांश्च तस्मात्कर्णो वृषा स्मृतः}


\threelineshloka
{युद्धशौण्डो महाबाहुर्नित्योद्यतशरासनः}
{केसरीव वने नर्दन्मत्तमातङ्गयूथपान्}
{विमर्दन्रथशार्दूलान्राजते रणमूर्धनिxxx}


\twolineshloka
{मध्यंगत इवादित्यो यो न शक्यो निरीक्षितुम्}
{त्वदीयैः पुरुषव्याघ्र योधमुख्यैर्महात्सभिः}


% Check verse!
शरजालसहस्रांशुः शरदीव दिxxxx
\twolineshloka
{तपान्ते जलदो यद्वच्छरधाराः क्षरन्मुहुः}
{दिव्यास्त्रजलदः कर्णः पर्जन्य इव वृष्टिमान्}


\twolineshloka
{त्रिदशैरपि चास्यद्भिः शरवर्षं समन्ततः}
{अशक्यस्तदयं जेतुं स्रवद्भिर्मांसशोणितम्}


\twolineshloka
{कवचेन विहीनश्च कुण्डलाभ्यां च पाण्डव}
{सोऽद्य मानुषतां प्राप्तो विमुक्तः शक्रदत्तया}


\twolineshloka
{एको हि योगोऽस्य भवेद्वधायच्छिद्रे ह्येनं स्वप्रमत्तः प्रमत्तम्}
{कृच्छ्रं प्राप्तं रथचक्रे विमग्नेहन्याः पूर्वं त्वं तु संज्ञां विचार्य}


% Check verse!
न ह्युद्यतास्त्रं युधि हन्यादजय्य--मप्येकवीरो बलभित्सवज्रः
\twolineshloka
{जरासन्धश्चेदिराजो महात्मामहाबाहुश्चैकलव्यो निषादः}
{एकैकशो निहताः सर्व एतेयोगैस्तैस्तैस्त्वद्धितार्थं मयैव}


\twolineshloka
{अथापरे निहता राक्षसेन्द्राहिडिम्बकिर्मीरबकप्रधानाः}
{अलायुधः परचक्रावमर्दीघटोत्कचश्चोग्रकर्मा तरस्वी}


\chapter{अध्यायः १८२}
\twolineshloka
{अर्जुन उवाच}
{}


\threelineshloka
{कथमस्मद्धितार्थं ते कैश्च योगैर्जनार्दन}
{जरासन्धप्रभृतयो घातिताः पृथिवीश्वराः ॥वासुदेव उवाच}
{}


\twolineshloka
{जरासन्धश्चेदिराजो नैषादिश्च महाबलः}
{यदि स्युर्न हताः पूर्वमिदानीं स्युर्भयङ्कराः}


\twolineshloka
{दुर्योधनस्तानवश्यं वृणुयाद्रथसत्तमान्}
{तेऽस्मासु नित्यविद्विष्टाः संश्रयेयुश्च कौरवान्}


\twolineshloka
{ते हि वीरा महेष्वासाः कृतास्त्रा दृढयोधिनः}
{धार्तराष्ट्रचमूं कृत्स्नां रक्षेयुरमरा इव}


\twolineshloka
{सूतपुत्रो जरासन्धश्चेदिराजो निषादजः}
{सुयोधनं समाश्रित्य तपेरन्पृथिवीमिमाम्}


\twolineshloka
{योगैरभिहता यैस्ते तन्मे शृणु धनञ्जय}
{अजय्या हि विना योगैर्मृधे ते दैवतैरपि}


\twolineshloka
{एकैको हि पृथक् तेषां समस्तां रिपुवाहिनीम्}
{योधयेत्समरे पार्थ लोकपालाभिरक्षिताम्}


\twolineshloka
{जरासन्धो हि रुषितो रौहिणेयप्रधर्षितः}
{अस्मद्वधार्थं चिक्षेप गदां वै सर्वघातिनीम्}


\twolineshloka
{सीमन्तमिव कुर्वाणा नभसः पावकप्रभा}
{अदृश्यतापतन्ती सा शक्रुमुक्ता यथाऽशनिः}


\twolineshloka
{तामापतन्तीं दृष्ट्वैव गदां रोहिणिनन्दनः}
{प्रतिघातार्थमस्त्रं वै स्थूणाकर्णमवासृजत्}


\twolineshloka
{अस्त्रवेगप्रतिहता सा गदा प्रापतद्भुवि}
{दारयन्ती धरां देवीं कम्पयन्तीव पर्वतान्}


\twolineshloka
{तत्र सा राक्षसी घोरा जरानाम्नी सुविक्रमा}
{सन्दधे सा हि सञ्जातं जरासन्धमरिन्दमम्}


\twolineshloka
{द्वाभ्यां जातो हि मातृभ्यामर्धदेहः पृथक्पृथक्}
{जरया सन्धितो यस्माज्जरासन्धस्ततोऽभवत्}


\twolineshloka
{सा तु भूमिं गता पार्थ हता ससुतबान्दवा}
{गदया तेन चास्त्रेण स्थूणाकर्णेन राक्षसी}


\twolineshloka
{विनाभूतः स गदया जरासन्धो महामृधे}
{निहतो भीमसेनेन पश्यतस्ते धनञ्जय}


\twolineshloka
{यदि हि स्याद्गदापाणिर्जरासन्धः प्रतापवान्}
{सेन्द्रा देवा न तं हन्तुं रणे शक्ता नरोत्तम}


\twolineshloka
{त्वद्धितार्थं च नैषादिरङ्गुष्ठेन वियोजितः}
{द्रोणेनाचार्यकं कृत्वा छद्मना सत्यविक्रमः}


\twolineshloka
{स तु बद्धाङ्गुलित्राणो नैषादिर्दृढविक्रमः}
{अतिमानी वचनरो बभौ राम इवापरः}


\twolineshloka
{एकलव्यं हि साङ्गुष्ठमशक्ता देवदानवाः}
{सराक्षसोरगाः पार्थ विजेतुं युधि कर्हिचित्}


\twolineshloka
{किमु मानुषमात्रेण शक्यः स्यात्प्रतिवीक्षितुम्}
{दृढमुष्टिः कृती नित्यमस्यमानो दिवानिशम्}


% Check verse!
त्वद्धितार्थं तु स मया हतः सङ्ग्राममूर्धनि
\twolineshloka
{चेदिराजश्च विक्रान्तः प्रत्यक्षं निहतस्तव}
{स चाप्यशक्यः सङ्ग्रामे जेतुं सर्वसुरासुरैः}


\twolineshloka
{वधार्थं तस्य जातोऽहमन्येषां च सुरद्विषाम्}
{त्वत्सहायो नरव्याघ्र लोकानां हितकाम्यया}


\twolineshloka
{हिडिम्बबककिर्मीरा भीमसेनेन पातिताः}
{रावणेन समप्राणा ब्रह्मयज्ञविनाशनाः}


\twolineshloka
{हतस्तथैव मायावी हैडिम्बेनाप्यलायुधः}
{हैडिम्बश्चाप्युपायेन शक्त्या कर्णेन घातितः}


\threelineshloka
{यदि ह्येनं नाहनिष्यत्कर्णः शक्त्या महामृधे}
{मयां वध्योऽभविष्यत्स भैमसेनिर्घटोत्कचः}
{मया न निहतः पूर्वमेष युष्मत्प्रियेप्सया}


\threelineshloka
{एष हि ब्राह्मणद्वेषी यज्ञद्वेषी च राक्षसः}
{धर्मस्य लोप्ता पापात्मा तस्मादेव निपातितः}
{व्यंसिता चाप्यूपायेन शक्रदत्ता मयाऽनघ}


\twolineshloka
{येहि धर्मस्य लोप्तारो वध्यास्ते मम पाण्डव}
{धर्मसंस्थापनार्थं हि प्रतिज्ञैषा ममाव्यया}


\twolineshloka
{ब्रह्म सत्यं दमः शौचं धर्मो हीः श्रीर्धृतिः क्षमा}
{यत्र तत्र रमे नित्यमहं सत्येन ते शपे}


\twolineshloka
{न विषादस्त्वया कार्यः कर्णं वैकर्तनं प्रति}
{उपदेक्ष्याम्युपायं ते येन तं प्रसहिष्यसि}


\twolineshloka
{सुयोधनं चापि रणे हनिष्यति वृकोदरः}
{तस्यापि च वधोपायं वक्ष्यामि तव पाण्डव}


\twolineshloka
{वर्धते तुमुलस्त्वेष शब्दः पचरमूं प्रति}
{विद्रवन्ति च सैन्यानि त्वदीयानि दिशो दश}


\twolineshloka
{लब्धलक्ष्या हि कौरव्या विधमन्ति चमूं तव}
{दहत्येष च वः सैन्यं द्रोणः प्रहरतां वरः}


\chapter{अध्यायः १८३}
\twolineshloka
{धृतराष्ट्र उवाच}
{}


\twolineshloka
{एकवीरवधे मोघा शक्तिः सूतात्मजे यदा}
{कस्मात्सर्वान्समुत्सृज्य स तां पार्थे न मुक्तवान्}


\twolineshloka
{तस्मिन्हते हता हि स्युः सर्वे पाण्डवसृञ्जयाः}
{एकवीरवधे कस्माद्युद्धे न जयमादधे}


\twolineshloka
{आहूतो न निवर्तेयमिति तस्य महाव्रतम्}
{स्वयं मार्गयितव्यः स सूतपुत्रेण फल्गुनः}


\twolineshloka
{ततो द्वैरथमानीय फल्गुनं शक्तदत्तया}
{जघान न वृषा कस्मात्तन्ममाचक्ष्व सञ्जय}


\twolineshloka
{नूनं बुद्धिविहीनश्चाप्यसहायश्च मे सुतः}
{शत्रुभिर्व्यंसितः पापः कथं नु स जयेदरीन्}


\twolineshloka
{या ह्यस्य परमा शक्तिर्जयस्य च परायणम्}
{सा शख्तिर्वासुदेवेन व्यंसिता च घटोत्कचे}


\twolineshloka
{कुणेर्यथा हस्तगतं हियेद्बिल्वं बलीयसा}
{तथा शक्तिरमोघा सा मोघीभूता घटोत्कचे}


\twolineshloka
{यथा वराहस्य शुनश्च युध्यतो--स्तयोरभावे श्वपचस्य लाभः}
{मन्ये विद्वन्वासुदेवस्य तद्व--द्युद्धे लाभः कर्णहैडिम्बयोर्वै}


\twolineshloka
{घटोत्कचो यदि हन्याद्वि कर्णंपरो लाभः स भवेत्पाण्डवानाम्}
{वैकर्तनो वा यदि तं निहन्या--त्तथापि कृत्यं शक्तिनाशात्कृतं स्यात्}


\threelineshloka
{इति प्राज्ञः प्रज्ञयैतद्विचिन्त्यघटोत्कचं सूतपुत्रेण युद्धे}
{अघातयद्वासुदेवो नृसिंहःप्रियं कुर्वन्पाण्डवानां हितं च ॥सञ्जय उवाच}
{}


\twolineshloka
{एतच्चिकीर्षितं ज्ञात्वा कर्णे मधुनिहा नृप}
{नियोजयामास तदा द्वैरथे राक्षसेश्वरम्}


\twolineshloka
{घटोत्कचं महावीर्यं महाबुद्धिर्जनार्दनः}
{अमोघाया विघातार्थं राजन्दुर्मन्त्रिते तव}


\twolineshloka
{तदैव कृतकार्या हि वयं स्याम कुरूद्वह}
{न रक्षेद्यदि कृष्णस्तं पार्थं कर्णान्महारथात्}


\twolineshloka
{साश्वध्वजरथः सङ्क्ये धृतराष्ट्र पतेद्भुवि}
{विना जनार्दनं पार्थो योगानामीश्वरं प्रभुम्}


\twolineshloka
{तैस्तैरुपायैर्बहुभी रक्ष्यमाणः स पार्थिव}
{जयत्यभिमुखः शत्रून्पार्थः कृष्णेन पालितः}


\threelineshloka
{सविशेषं त्वमेयात्मा कृष्णो रक्षेन्न फल्गुनम्}
{हन्यात्क्षिप्रं हि कौन्तेयं शक्तिर्वृक्षमिवाशनिः ॥धृतराष्ट्र उवाच}
{}


\twolineshloka
{विरोधी च कुमन्त्री च प्राज्ञमानी ममात्मजः}
{यस्यैष समतिक्रान्तो वधोपायो जयं प्रति}


\twolineshloka
{स वा कर्णो महाबुद्धिः सर्वशस्त्रभृतां वरः}
{न मुक्तवान्कथं सूत ताममोघां धनञ्जये}


\threelineshloka
{तवापि समतिक्रान्तमेतद्गावल्गणे कथम्}
{एतमर्थं महाबुद्धे यत्त्वया नावबोधितः ॥सञ्जय उवाच}
{}


\twolineshloka
{दुर्योधनस्य शकुनेर्मम दुःशासनस्य च}
{रात्रौ रात्रौ भवत्येषा नित्यमेव विकत्थना}


\twolineshloka
{श्वः सर्वसैन्यानुत्सृज्य जहि कर्ण धनञ्जयम्}
{प्रेष्यवत्पाण्डुपाञ्चालानुपभोक्ष्यामहे ततः}


\twolineshloka
{अथवा निहते पार्थं पाण्डवान्यतमं ततः}
{स्थापयेद्यदि वार्ष्णेयस्तस्मात्कृष्णो हि हन्यताम्}


\twolineshloka
{कृष्णो हि मूलं पाण्डूनां पार्थः स्कन्ध इवोद्गतः}
{शाखा इवेतरे पार्थाः पाञ्चालाः पत्रसंज्ञिताः}


\twolineshloka
{कृष्णाश्रयाः कृष्णबलाः कृष्णनाथाश्च पाण्डवाः}
{कृष्णः परायणं चैषां ज्योतिषामिव चन्द्रमाः}


\twolineshloka
{तस्मात्पर्णानि शाखाश्च स्कन्धं चोत्सृज्य सूतज}
{कृष्णं हि विद्वि पाण्डूनां मूलं सर्वत्र सर्वदा}


\twolineshloka
{हन्याद्यदि हि दाशार्हं कर्णो यादवनन्दनम्}
{कृत्स्ना वसुमती राजन्वशे तस्य न संशयः}


\twolineshloka
{यदि हि स निहतः शयीत भूमौयदुकुलपाण्डवनन्दनो महात्मा}
{ननु तव वसुधा नरेन्द्र सर्वासगिरिसमुद्रवना वशं व्रजेत}


\twolineshloka
{सा तु बुद्धिः कृताऽप्येवं जाग्रति त्रिदशेश्वरे}
{अप्रमेये हृषीकेशे युद्धकाले त्वमुह्यत}


\twolineshloka
{अर्जुनं चापि राधेयात्सदा रक्षति केशवः}
{न ह्येनमैच्छत्प्रमुखे सौतेः स्थापयितुं रणे}


\twolineshloka
{अन्यांश्चास्मै रथोदारानुपास्थापयदच्युतः}
{अमोघां तां कथं शक्तिं मोघां कुर्यामिति प्रभो}


\twolineshloka
{यश्चैवं रक्षते पार्थं कर्णात्कृष्णो महामनाः}
{आत्मानं स कथं राजन्न रक्षेत्पुरुपोत्तमः}


\threelineshloka
{परिचिन्त्य तु पश्यामि चक्रायुधमरिन्दमम्}
{न सोऽस्ति त्रिषु लोकेषु यो जयेत जनार्दनम् ॥सञ्जय उवाच}
{}


\twolineshloka
{ततः कृष्णं महाबाहुं सात्यकिः सत्यविक्रमः}
{पप्रच्छ रथशार्दूलः कर्णं प्रति महारथः}


\threelineshloka
{अयं च प्रत्ययः कर्णे शक्तिश्चामितविक्रमा}
{किमर्थं सूतपुत्रेण न मुक्ता फल्गुने तु सा ॥वासुदेव उवाच}
{}


\threelineshloka
{दुःशासनश्च कर्णश्च शकुनिश्च मसैन्धवः}
{सततं मन्त्रयन्ति स्म दुर्योधनपुरोगमाः}
{}


\twolineshloka
{कर्णकर्ण महेष्वास रणेऽमितपराक्रम}
{नान्यस्य शक्तिरेषा ते मोक्तव्या जयतां वर}


\twolineshloka
{ऋते महारथात्कर्ण कुन्तीपुत्राद्वनञ्जयात्}
{स हि तेषामतियशा देवानामिव वासवः}


\twolineshloka
{तस्मिन्विनिहते पार्थे पाण्डवाः सृञ्जयैः सह}
{भविष्यन्ति गतात्मानः सुरा इव निरग्नयः}


\twolineshloka
{तथेति च प्रतिज्ञातं कर्णेन शिनिपुङ्गव}
{हृदि नित्यं च कर्णस्य वधो गाण्डीवधन्वनः}


\twolineshloka
{अहमेव तु राधेयं मोहयामि युधांवर}
{ततो नावासृजच्छक्तिं पाण्डवे श्वेतवाहने}


\twolineshloka
{फल्गुनस्य हि सा मृत्युरिति चिन्तयतोऽनिशम्}
{न निद्रा न च मे हार्षो मनसोऽस्ति युधांवर}


\twolineshloka
{घटोत्कचे व्यंसितां तु दृष्ट्वा तां शिनिपुङ्गव}
{मृत्योरास्यान्तरान्मुक्तं पस्याम्यद्य धनञ्जयम्}


\twolineshloka
{न पिता न च मे माता न यूयं भ्रातरस्तथा}
{न च प्राणास्तथा रक्ष्या यथा बीभत्सुराहवे}


\twolineshloka
{त्रैलोक्यराज्याद्यत्किञ्चिद्भवेदन्यत्सुदुर्लभम्}
{नेच्छेयं सात्वताहं तद्विना पार्थं धनञ्जयम्}


\twolineshloka
{अतः प्रहर्षः सुमहान्युयुधानाद्य मेऽभवत्}
{मृतं प्रत्यागतमिव दृष्ट्वा पार्थं धनञ्जयम्}


\threelineshloka
{अतश्च प्रहितो युद्धे मया कर्णाय राक्षसः}
{न ह्यन्यः समरे रात्रौ शक्तः कर्णं प्रबाधितुम् ॥सञ्जय उवाच}
{}


\twolineshloka
{इति सात्यकये प्राह तदा देवकिनन्दनः}
{धनञ्जयहिते युक्तस्तत्प्रिये सततं रतः}


\chapter{अध्यायः १८४}
\twolineshloka
{धृतराष्ट्र उवाच}
{}


\twolineshloka
{कर्णदुर्योधनादीनां शकुनेः सौबलस्य च}
{अपनीतं महत्तात तव चैव विशेषतः}


\twolineshloka
{यदि जानीथ तां शक्तिमेकघ्नीं सततं रणे}
{अनिवार्यामसह्यां च देवैरपि सवासवैः}


\threelineshloka
{सा किमर्थं तु कर्णेन प्रवृत्ते समरे पुरा}
{न देवकीसुते मुक्ता फल्गुने वाऽपि सञ्जय ॥सञ्जय उवाच}
{}


\twolineshloka
{सङ्ग्रामाद्विनिवृत्तानां सर्वेषां नो विशाम्पते}
{रात्रौ कुरुकुलश्रेष्ठ मन्त्रोऽयं समजायत}


\twolineshloka
{प्रभातमात्रे श्वोभूते केशवायार्जुनाय वा}
{शक्तिरेषा हि मोक्तव्या कर्णकर्णेनि नित्यशः}


\twolineshloka
{ततः प्रभातसमये राजन्कर्णस्य दैवतैः}
{अन्येषां चैव योधानां सा बुद्धिर्नाश्यते पुनः}


\twolineshloka
{दैवमेव परं मन्ये यत्कर्णो हस्तसंस्थया}
{न जघान रणे पार्थं कृष्णं वा देवकीसुतम्}


\twolineshloka
{तस्य हस्तस्थिता शक्तिः कालरात्रिरिवोद्यता}
{दैवोपहतबुद्धित्वान्न तां कर्णो विमुक्तवान्}


\threelineshloka
{कृष्णे वा देवकीपुत्रे मोहितो देवमायया}
{पार्थे वा शक्रकल्पे वै वधार्थं वासवीं प्रभो ॥धृतराष्ट्र उवाच}
{}


\twolineshloka
{दैवेनोपहता यूयं स्वबुद्ध्या केशवस्य च}
{गता हि वासवी हत्वा तृणभूतं घटोत्कचम्}


\twolineshloka
{कर्णश्च मम पुत्राश्च सर्वे चान्ये च पार्थिवाः}
{तेन वै दुष्प्रणीतेन गता वैवस्वतक्षयम्}


\twolineshloka
{भूय एव तु मे शंस यथा युद्धमवर्तत}
{कृरूणां पाण्डवानां च हैडिम्बौ निहते तदा}


\twolineshloka
{ये च तेऽभ्यद्रवन्द्रोणं व्यूढानीकाः प्रहारिणः}
{सृञ्जयाः सह पाञ्चालैस्तेऽप्यकुर्वन्कथं रणम्}


\twolineshloka
{सौमदत्तेर्वधाद्द्रोणमायान्तं सैन्धवस्य च}
{अमर्षाज्जीवितं त्यक्त्वा गाहमानं वरूथिनीम्}


\twolineshloka
{जृम्भमाणमिव व्याघ्रं व्यात्ताननमिवान्तकम्}
{कथं प्रत्युद्ययुर्द्रोणमस्यन्तं पाण्डुसृञ्जयाः}


\twolineshloka
{आचार्यं ये च तेऽरक्षन्दुर्योधनपुरोगमाः}
{द्रौणिकर्णकृपास्तात ते वाऽकुर्वन्किमाहवे}


\twolineshloka
{भारद्वाजं जिघांसन्तौ सव्यसाचिवृकीदरौ}
{समार्च्छन्मामका युद्धे कथं सञ्जय शंस मे}


\threelineshloka
{सिन्धुराजवधेनेमे घटोत्कचवधेन ते}
{अमर्पिताः सुसङ्क्रुद्धा रणं चक्रुः कथं निशि ॥सञ्जय उवाच}
{}


\twolineshloka
{हते घटोत्कचे राजन्कर्णेन निशि राक्षसे}
{प्रणदत्सु च हृष्टेषु तावकेषु युयुत्सुषु}


\twolineshloka
{आपतत्सु च वेगेन वध्यमाने बलेऽपि च}
{विगाढायां रजन्यां च राजा दैन्यं परं गतः}


\twolineshloka
{अब्रवीच्च महाबाहुर्भीमसेनमिदं वचः}
{आवारय महाबाहो धार्तराष्ट्रस्य वाहिनीम्}


\twolineshloka
{हैडिम्बेश्चैव घातेन मोहो मामाविशन्महान्}
{एवं भीमं समादिश्य स्वरथे समुपाविशत्}


\twolineshloka
{अश्रुपूर्णमुखो राजा निःश्वसंश्च पुनःपुनः}
{कश्मलं प्राविशद्धोरं दृष्ट्वा कर्णस्य विक्रमम्}


% Check verse!
तं तथा व्यथितं दृष्ट्वा कृष्णो वचनमब्रवीत्
\twolineshloka
{मा व्यथां कुरु कौन्तेय नैतत्त्वय्युपपद्यते}
{वैक्लव्यं भरतश्रेष्ठ यथा प्राकृतपूरुषे}


\twolineshloka
{उत्तिष्ठ राजन्युध्वस्व वह गुर्वीं धुरं विभो}
{त्वयि वैक्लव्यमापन्ने संशयो विजये भवेत्}


\twolineshloka
{श्रुत्वा कृष्णस्य वचनं धर्मराजो युधिष्ठिरः}
{विमृज्य नेत्रे पाणिभ्यां कृष्णं वचनमब्रवीत्}


\twolineshloka
{विदिता मे महाबाहो धर्माणां परमा गतिः}
{ब्रह्महत्याफलं तस्य यः कृतं नावबुध्यते}


\twolineshloka
{अस्माकं हि वनस्थानां हैडिम्बेन महात्सना}
{बालेनापि सता तेन कृतं साह्यं जनार्दन}


\twolineshloka
{अस्त्रहेतोर्गतं ज्ञात्वा पाण्डवं स्वेतवाहनम्}
{असौ कृष्ण महेष्वासः काम्यके मामुपस्थितः}


\threelineshloka
{उषितश्च सहास्माभिर्यावन्नासीद्धनञ्जयः}
{गन्धमादनयात्रायां दुर्गेभ्यश्च स्म तारिताः}
{पाञ्चाली च परिश्रान्ता पृष्ठेनोढा महात्मना}


\twolineshloka
{आरम्भांश्चैव युद्धानां यदेष कृतवान्प्रभो}
{मदर्थे दुष्करं कर्म कृतं तेन महाहवे}


\twolineshloka
{स्वभावाद्या च मे प्रीतिः सहदेवे जनार्दन}
{सैव मे द्विगुणा प्रीती राक्षसेन्द्रे घटोत्कचे}


\twolineshloka
{भक्तश्च मेहाबाहुः प्रियोऽस्याहं प्रियश्च मे}
{तेन विन्दामि वार्ष्णेय कश्मलं शोकतापिताः}


\twolineshloka
{पश्य सैन्यानि वार्ष्णेय द्राव्यमाणानि कौरवैः}
{द्रोणकर्णौ तु संयत्तौ पश्य युद्धे महारथौ}


\twolineshloka
{निशीथे पाण्डवं सैन्यमाभ्यां माधव मर्दितम्}
{गजाभ्यामिव मत्ताभ्यां यथा नलवनं महत्}


\twolineshloka
{अनादृत्य बलं बाह्वोर्भीमसेनस्य माधव}
{चित्रास्त्रतां च पार्थस्य विक्रमन्ते च कौरवाः}


\twolineshloka
{एष द्रोणश्च कर्णश्च राजा चैव सुयोधनः}
{निहत्य राक्षसं युद्धे हृष्टा नर्दन्ति संयुगे}


\twolineshloka
{कथं वाऽसमासु जीवत्सु त्वयि चैव जनार्दन}
{हैडिम्बिः प्राप्तवान्मृत्यं सूतपुत्रेण सङ्गतः}


\twolineshloka
{कदर्थीकृत्य नः सर्वान्पश्यतः सव्यसाचिनः}
{निहतो राक्षसः कृष्ण भैमसेनिर्महाबलः}


\twolineshloka
{यदाऽभिमन्युर्निहतो धार्तराष्ट्रैर्दुरात्मभिः}
{नासीत्तत्र रणे कृष्ण सव्यसाची महारथः}


\twolineshloka
{निरुद्वाश्च वयं सर्वे सैन्धवेन दुरात्मना}
{निमित्तमभवद्द्रोणः सपुत्रस्तत्र कर्मणि}


\twolineshloka
{उपदिष्टो वधोपायः कर्णस्य गुरुणा स्वयम्}
{व्यायच्छतश्च खङ्गेन द्विधा खङ्गं चकार ह}


\threelineshloka
{व्यसने वर्तमानस्य कृतवर्मा नृशंसवत्}
{अश्वाञ्जघान सहसा तथोभौ पार्ष्णिसारथी}
{तथेतरे महेष्वासाः सौभद्रं युध्यपातयन्}


\twolineshloka
{अल्पे च कारणे कृष्ण हतो गाण्डीवधन्वना}
{सैन्धवो यादवश्रेष्ठ तच्च नातिप्रियं मम}


\twolineshloka
{यदि शत्रुवधो न्याय्यो भवेत्कर्तुं हि पाण्डवैः}
{कर्णद्रोणौ रणे पूर्वं हन्तव्याविति मे मतिः}


\twolineshloka
{एतौ हि मूलं दुःखानामस्माकं पुरुषर्षभ}
{एतौ रणे समासाद्य समाश्वस्तः सुयोधनः}


\twolineshloka
{यत्र वध्यो भवेद्द्रोणः सूतपुत्रश्च सानुगः}
{तत्रावधीन्महाबाहुः सैन्धवं दूरवासिनम्}


\twolineshloka
{अवश्यं तु मया कार्यः सूतपुत्रस्य निग्रहः}
{ततो यास्याम्यहं वीर स्वयं कर्णजिघांसया}


% Check verse!
भीमसेनो महाबाहुर्द्रोणानीकेन सङ्गतः
\twolineshloka
{एवमुक्त्वा ययौ तूर्णं त्वरमाणो युधिष्ठिरः}
{स विष्फार्य महच्चापं शङ्खं प्रध्माप्य भैरवम्}


\threelineshloka
{ततो रथसहस्रेण गजानां च शतैस्त्रिभिः}
{वाजिभिः पञ्चसाहस्रैः पाञ्चालैः सप्रभद्रकैः}
{वृतः शिखण्डी तवरितो राजानं पृष्ठतोऽन्वयात्}


\threelineshloka
{ततो भेरीः समाजघ्नुः पणवानकगोमुखान्}
{शह्खशब्दरवांश्चैव पाणिशब्दांश्च दंशिताः}
{पाञ्चालाः पाण्डवाश्चैव युधिष्ठिरपुरोगमाः}


% Check verse!
ततोऽब्रवीन्महाबाहुर्वासुदेवो धनञ्जयम्
\twolineshloka
{एष प्रयाति त्वरितः क्रोधाविष्टो युधिष्ठिरः}
{जिघांसुः सूतपुत्रस्य तस्योपेक्षा न युज्यते}


\twolineshloka
{एवमुक्त्वा हृषीकेशः शीघ्रमश्वानचोदयत्}
{दूरं प्रयान्तं राजानमन्वगच्छज्जनार्दनः}


\fourlineindentedshloka
{तं दृष्ट्वा सहसा यान्तं भूतपुत्रजिघांसया}
{शोकोपहतसङ्कल्पं दह्यमानमिवाग्निना}
{अभिगम्याब्रवीद्व्यासो धर्मपुत्रं युधिष्ठिरम् ॥व्यास उवाच}
{}


% Check verse!
कर्णमासाद्य सङ्ग्रामे दिष्ट्या जीवति फल्गुनः
\twolineshloka
{सव्यसाचिवधाकाङ्क्षी शक्तिं रक्षितवान्हि सः}
{न चागाद्द्वैरथं जिष्णुर्दिष्ट्या तेन महारणे}


\twolineshloka
{सृजेतां स्पर्धिनावेतौ दिव्यान्यस्त्राणि सर्वशः}
{वध्मानेषु चास्त्रेषु पीडितः सूतनन्दनः}


\twolineshloka
{वासवीं समरे शक्तिं ध्रुवं मुञ्चेद्युधिष्ठिर}
{ततो भवेत्ते व्यसनं घोरं भरतसत्तम}


\twolineshloka
{दिष्ट्या रक्षो हतं युद्धे सूतपुत्रेण मानद}
{वासवीं कारणं कृत्वा कालेनोपहतो ह्यसौ}


\twolineshloka
{तवैव कारणाद्रक्षो निहतं तात संयुगे}
{मा क्रुधो भरतश्रेष्ठ मा च शोके मनः कृथाः}


\twolineshloka
{प्राणिनामिह सर्वेषामेषा निष्ठा युधिष्ठिर}
{भ्रातृभिः सहितः सर्वैः पार्थिवैश्च महात्मभिः}


\twolineshloka
{कौरवान्समरे राजन्प्रतियुध्यस्व भारत}
{पञ्चमे दिवसे तात पृथिवी ते भविष्यति}


\twolineshloka
{नित्यं च पुरुषव्याघ्र धर्ममेवानुचिन्तय}
{आनृशंस्यं तपो दानं क्षमां सत्यं च पाण्डव}


\twolineshloka
{सेवेथाः परमप्रीतो यतो धर्मस्ततो जयः}
{इत्युक्त्वा पाण्डवं व्यासस्तत्रैवान्तरधीयत}


\chapter{अध्यायः १८५}
\twolineshloka
{सञ्जय उवाच}
{}


\twolineshloka
{व्यासेनैवमथोक्तस्तु धर्मराजो युधिष्ठिरः}
{स्वयं कर्णवधाद्वीरो निवृत्तो भरतर्षभ}


\twolineshloka
{घटोत्कचे तु निहते सूतपुत्रेण ता निशाम्}
{दुःखामर्षवशं प्राप्तो धर्मराजो युधिष्ठिरः}


\twolineshloka
{दृष्ट्वा भीमेन महतीं वार्यमाणां चमूं तव}
{धृष्टद्युम्नमुवाचेदं कुम्भयोनिं निवारय}


\threelineshloka
{त्वं हि द्रोणविनाशाय समुत्पन्नो हुताशनात्}
{सशरः कवची खङ्गी धन्वी च परतापनः}
{अभिद्रव रणे हृष्टो मा च ते भीः कथञ्चन}


\twolineshloka
{जनमेजयः शिखण्डी च दौर्मुखिश्च यशोधरः}
{अभिद्रवन्तु संहृष्टाः कुम्भयोनिं समन्ततः}


\twolineshloka
{नकुलः सहदेवश्च द्रौपदेयाः प्रभद्रकाः}
{द्रुपदश्च विराटश्च पुत्रभातृसमन्वितौ}


\twolineshloka
{सात्यकिः केकयाश्चैव पाण्डवश्च धनञ्जयः}
{अभिद्रवन्तु वेगेन कुम्भयोनिवधेप्सया}


\twolineshloka
{तथैव रथिनः सर्वे हस्त्यश्वं यच्च किञ्चन}
{पदाताश्च रणे द्रोणं पातयन्तु महारथम्}


\twolineshloka
{तथाऽऽज्ञप्तास्तु ते सर्वे पाण्डवेन महात्मना}
{अभ्यद्रवन्त वेगेन कुम्भयोनिवधेप्सया}


\twolineshloka
{आगच्छतस्तान्सहसा सर्वोद्योगेन पाण्डवान्}
{प्रतिजग्राह समरे द्रोणः शस्त्रभृतां वरः}


\twolineshloka
{ततो दुर्योधनो राजा सर्वोद्योगेन पाण्डवान्}
{अभ्यद्रवत्सुसङ्क्रुद्ध इच्छन्द्रोणस्य जीवितम्}


\twolineshloka
{ततः प्रववृते युद्धं श्रान्तवाहनसैनिकम्}
{पाण्डवानां कुरूणां च गर्जतामितरेतरम्}


\twolineshloka
{निद्रान्धास्ते महाराज परिश्रान्ताश्च संयुगे}
{नाभ्यपद्यन्त समरे काञ्चिच्चेष्टां महारथाः}


\twolineshloka
{त्रियामा रजनी चैषा घोररूपा भयानका}
{सहस्रयामप्रतिमा बभूव प्राणहारिणी}


\twolineshloka
{वध्यतां च तथा तेषां क्षतानां च विशेषतः}
{घोरा रात्रिः समाजज्ञे निद्रान्धानां विशेषतः}


\twolineshloka
{सर्वे ह्यासन्निरुत्साहाः क्षत्रिया दीनचेतसः}
{तव चैव परेषां च गतास्त्रा विगतेषवः}


\twolineshloka
{ते तदा पारयन्तश्च भ्रमन्तश्च विशेषतः}
{स्वधर्ममनुपश्यन्तो न जहुः स्वामनीकिनीम्}


\twolineshloka
{अस्त्राण्यन्ये समुत्सृज्य निद्रान्धाः शेरते जनाः}
{रथेष्वन्ये गजेष्वन्ये हयेष्वन्ये च भारत}


\twolineshloka
{निद्रान्धा नो बुबुधिरे काञ्चिच्चेष्टां नराधिप}
{तानन्ये समरे योधाः प्रेषयन्ति यमक्षयम्}


\twolineshloka
{स्वप्नायमानांस्त्वपरे परानतिविचेतसः}
{आत्मानं समरे जघ्नुः स्वानेव च परानपि}


% Check verse!
नानावाचो विमुञ्चन्तो निद्रान्धास्ते महारणे
\twolineshloka
{अस्माकं च महाराज परेभ्यो बहवो जनाः}
{योद्धव्यमिति तिष्ठन्तो निदासंरक्तलोचनाः}


\twolineshloka
{संसर्पन्तो रणे केचिन्निद्रान्धास्ते तथापरान्}
{जघ्रुः शूरा रणे शूरांस्तस्मिंस्तमसि दारुणे}


\twolineshloka
{हन्यमानमथात्मावनं परैश्च बहवो जनाः}
{नाभ्यजानन्त समरे निद्रया मोहिता भृशम्}


\twolineshloka
{तेषामेतादृशीं चेष्टां विज्ञाय पुरुषर्षभः}
{उवाच वाक्यं बीभत्सुरुच्चैः सन्नादयन्दिशः}


\twolineshloka
{श्रान्ता भवन्तो निद्रान्धाः सर्व एव सवाहनाः}
{तमसा च वृते सैन्ये रजसा बहुलेन च}


\twolineshloka
{ते यूयं यदि मन्यध्वमुपारमत सैनिकाः}
{निमीलयत चात्रैव रणभूमौ मुहूर्तकम्}


\twolineshloka
{ततो विनिद्रा विश्रान्ताश्चन्द्रमस्युदिते पुनः}
{संसाधयिष्यथान्योन्यं स्वर्गाय कुरुपाण्डवाः}


\twolineshloka
{तद्वचः सर्वधर्मज्ञा धार्मिकस्य विशाम्पते}
{अरोचयन्त सैन्यानि तथा चान्योन्यमब्रुवन्}


\twolineshloka
{चुक्रुशुः कर्णकर्णेनि तथा दुर्योधनेति च}
{उपारमत पाण्डूनां विरता हि वरूथिनी}


\twolineshloka
{तथा विक्रोशमानस्य फल्गुनस्य ततस्ततः}
{उपारमत पाण्डूनां सेना तव च भारत}


\twolineshloka
{तामस्य वाचं देवाश्च ऋषयश्च महात्मनः}
{सर्वसैन्यानि चाक्षुद्रां प्रहृष्टाः प्रत्यपूजयन्}


\twolineshloka
{तत्सम्पूज्य वचोऽक्रूरं सर्वसैन्यानि भारत}
{मुहूर्तमस्वपन्राजञ्श्रान्तानि भरतर्षभ}


\twolineshloka
{सा तु सम्प्राप्य विश्रामं ध्वजिनी तव भारत}
{सुखमाप्तवती वीरमर्जुनं प्रत्यपूजयत्}


\twolineshloka
{त्वयि वेदास्तथास्त्राणि त्वयि बुद्धिपराक्रमौ}
{धर्मस्त्वयि महाबाहो दया भूतेषु चानघ}


\twolineshloka
{यच्चाश्वस्तास्तवेच्छामः शर्म पार्थ तदस्तु ते}
{मनसश्च प्रियानर्थान्वीर क्षिप्रमवाप्नुहि}


\twolineshloka
{इति ते तं नरव्याघ्रं प्रशंसन्तो महारथाः}
{निद्रया समवाक्षिप्तास्तूष्णीमासन्विशाम्पते}


\twolineshloka
{`वीरा वारणकुम्भेषु सुषुपुर्युद्धकर्शिताः}
{रात्रौ रतिपरिश्रान्ताः कामिनीनां कुचेष्विव'}


\twolineshloka
{अश्वपृष्ठेषु चाप्यन्ये रथनीडेषु चापरे}
{गजस्कन्धगताश्चान्ये शेरते चापरे क्षितौ}


\twolineshloka
{सायुधाः सगदाश्चैव सखङ्गाः सपरश्वथाः}
{सप्रासकवचाश्चान्ये नराः सुप्ताः पृथक्पृथक्}


\twolineshloka
{गजास्ते पन्नगाभोगैर्हस्तैर्भूरेणुगुण्ठितैः}
{निद्रान्धा वसुधां चक्रुर्घ्राणनिः श्वासशीतलाम्}


\twolineshloka
{सुप्ताः शुशुभिरे तत्र निःश्वसन्तो महीतले}
{विकीर्णा गिरयो यद्वन्निःश्वसद्भिर्महोरगैः}


\threelineshloka
{समां च विषमां चक्रुः खुराग्रैर्विकृतां महीम्}
{हयाः काञ्चनयोक्त्रास्ते केसरालम्बिभिर्युगैः}
{सुषुपुस्तत्र राजेन्द्र युक्ता वाहेषु सर्वशः}


\twolineshloka
{एवं हयाश्च नागाश्च योधाश्च भरतर्पभ}
{युद्धाद्विरम्य सुषुपुः श्रमेण महताऽन्विताः}


\twolineshloka
{तत्तथा निद्रया भग्नं तद्बभौ निःस्वनं बलम्}
{कुशलैः शिल्पिभिर्न्यस्तं पटे चित्रमिवाद्भुतम्}


\twolineshloka
{ते क्षत्रियाः कुण्डलिनो युवानःपरस्परं सायकविक्षताङ्गाः}
{कुम्भेषु लीनाः सुषुपुर्गजानांकुचेषु लग्ना इव कामिनीनाम्}


\twolineshloka
{ततः कुमुदनाथेन कामिनीगण्डपाण्डुना}
{नेत्रानन्देन चन्द्रेण माहेन्द्री दिगलङ्कृता}


\twolineshloka
{दशशताक्षककुब्दरिनिःसृतःकिरणकेसरभासुरपिञ्जरः}
{तिमिरवारणवृथविदारणःसमुदियादुदयाचलकेसरी}


\twolineshloka
{हरवृषोत्तमगात्रसमद्युतिःस्मरशरासनपूर्णसमप्रभः}
{नववधृस्मितचारुमनोहरःप्रविसृतः कुमुदाकरबान्धवः}


\twolineshloka
{ततो मुहूर्ताद्भगवान्पुरस्ताच्छशलक्षणः}
{अरुणं दर्शयामास ग्रसञ्ज्योतिः प्रभां प्रभुः}


\twolineshloka
{अरुणस्य तु तस्यानु जातरूपसमप्रभम्}
{रश्मिजालं महच्चन्द्रो मन्दं मन्दमवासृजत्}


\twolineshloka
{उत्सारयन्तः प्रभया तमस्ते चन्द्ररश्मयः}
{पर्यगच्छञ्छनैः सर्वा दिशः खं च क्षितिं तथा}


\twolineshloka
{ततो मुहूर्ताद्भुवनं ज्योतिर्भूतमिवाभवत्}
{अप्रख्यमप्रकाशं च जगामाशु तमस्तथा}


\twolineshloka
{प्रतिप्रकाशिते लोके दिवाभूते निशाकरे}
{विचेरुर्न विचेरुश्च राजन्नक्तञ्चरास्ततः}


\twolineshloka
{बोध्यमानं तु तत्सैन्यं राजंश्चन्द्रस्य रश्मिभिः}
{बुबुधे शतपत्राणां वनं सूर्यांशुभिर्यथा}


\twolineshloka
{यथा चन्द्रोदयोद्धूतः क्षुभितः सागरोऽभवत्}
{तथा चन्द्रोदयोद्धूतः क्षुभितश्च बलार्णवः}


\twolineshloka
{ततः प्रववृते युद्धं पुनरेव विशाम्पते}
{लोके लोकविनाशाय परं लोकमभीप्सताम्}


\chapter{अध्यायः १८६}
\twolineshloka
{सञ्जय उवाच}
{}


\threelineshloka
{ततो दुर्योधनो द्रोममभिगम्याब्रवीदिदम्}
{अमर्षवशमापन्नो जनयन्हर्षतेजसी ॥दुर्योधन उवाच}
{}


\twolineshloka
{न मर्षणीयाः सङ्ग्रामे विश्रमन्तः श्रमान्विताः}
{सपत्ना ग्लानमनसो लब्धलक्षा विशेषतः}


\twolineshloka
{यत्तु मर्षितमस्माभिर्भवतः प्रियकाम्यया}
{त एते परिविश्रान्ताः पाण्डवा बलवत्तराः}


\twolineshloka
{सर्वथा परिहीनाः स्म तेजसा च बलेन च}
{भवता पाल्यमानास्ते विवर्धन्ते पुनःपुनः}


\twolineshloka
{दिव्यान्यस्त्राणि सर्वाणि ब्राह्मादीनि च यानि ह}
{तानि सर्वामि तिष्ठन्ति भवत्येव विशेषतः}


\twolineshloka
{न पाण्डवेया न वयं नान्ये लोके धनुर्धराः}
{युध्यमानस्य ते तुल्याः सत्यमेतद्ब्रवीमि ते}


\twolineshloka
{ससुरासुरगन्धर्वानिमाँलोकान्द्विजोत्तम}
{सर्वास्त्रविद्भवान्हन्याद्दिव्यैरस्त्रैर्न संशयः}


\threelineshloka
{स भवान्मर्षयत्येनांस्तव तुल्यानवेत्य च}
{शिष्यत्वं वा पुरस्कृत्य मम वा मन्दभाग्यताम् ॥सञ्जय उवाच}
{}


\twolineshloka
{एवमुद्धर्षितो द्रोणः कोपितश्च सुतेन ते}
{समन्युरब्रवीद्राजन्दुर्योधनमिदं वचः}


\threelineshloka
{स्थविरः सन्परं शक्त्या घटे दुर्योधनाहवे}
{अतः परं मया कार्यं क्षुद्रं विजयगृद्धिना}
{अनस्त्रविदयं सर्वो हन्तव्योऽस्त्रविदा जनः}


\twolineshloka
{यद्भवान्मन्यते चापि शुभं वा यदि वाऽशुभम्}
{तद्वै कर्ताऽस्मि कौरव्य वचनात्तव नान्यथा}


\twolineshloka
{निहत्य सर्वपाञ्चालान्युद्धे कृत्वा पराक्रमम्}
{विमोक्ष्ये कवचं राजन्सत्येनायुधमालभे}


\twolineshloka
{मन्यसे यच्च कौन्तेयमर्जुनं श्रान्तमाहवे}
{तस्य वीर्यं महाबाहो शृणु सत्येन कौरव}


\twolineshloka
{तं न देवा न गन्धर्वा न यक्षा न च राक्षसाः}
{उत्सहन्ते रणे जेतुं कुपितं सव्यसाचिनम्}


\twolineshloka
{खाण्डवे येन भगवान्प्रत्युद्यातः सुरेश्वरः}
{सायकैर्वारितश्चापि वर्षमाणो महात्मना}


\twolineshloka
{यक्षा नागास्तथा दैत्या ये चान्ये बलगर्विताः}
{निहताः पुरुषेन्द्रेण तच्चापि विदितं तव}


\twolineshloka
{गन्धर्वा घोषयात्रायां चित्रसेनादयो जिताः}
{यूयं तैर्ह्रियमाणाश्च मोक्षिता दृढधन्वना}


\twolineshloka
{निवातकवचाश्चापि देवानां शत्रवस्तथा}
{सुरैरवध्याः सङ्ग्रामे तेन वीरेण निर्जिताः}


\twolineshloka
{दानवानां सहस्राणि हिरण्यपुरवासिनाम्}
{विजिग्ये पुरुषव्याघ्रः स शक्यो मानुषैः कथम्}


\threelineshloka
{प्रत्यक्षं चैव ते सर्वं यथा बलमिदं तव}
{क्षपितं पाण्डुपुत्रेण चेष्टतां नो विशाम्पते ॥सञ्जय उवाच}
{}


\twolineshloka
{तं तदाऽभिप्रशंसन्तमर्जुनं कुपितस्तदा}
{द्रोणं तव सुतो राजन्पुनरेवेदमब्रवीत्}


\threelineshloka
{अहं दुःशासनः कर्णः शकुनिर्मातुलश्च मे}
{हनिष्यामोऽर्जुनं सङ्ख्ये द्विधा कृत्वाऽद्य भारतीम्}
{`तिष्ठ स त्वं महाबाहो नित्यं शिष्यः प्रियस्तव'}


\twolineshloka
{तस्य तद्वचनं श्रुत्वा भारद्वाजो हसन्निव}
{अन्ववर्तत राजानं स्वस्ति तेऽस्त्विति चाब्रवीत्}


\twolineshloka
{को हि गाण्डीवधन्वानं ज्वलन्तमिव तेजसा}
{अक्षयं क्षपयेत्कश्चित्क्षत्रियः क्षत्रियर्षभम्}


\twolineshloka
{तं न वित्तपतिर्नेन्द्रो न यमो न जलेश्वरः}
{नासुरोरगरक्षांसि क्षपयेयुः सहायुधम्}


\twolineshloka
{मूढास्त्वेतानि भाषन्ते यानीमान्यात्थ भारत}
{युद्धे ह्यर्जुनमासाद्य स्वस्तिमान्को व्रजेद्गृहान्}


\twolineshloka
{त्वं तु सर्वाभिशङ्कित्वान्निष्ठुरः पापनिश्चयः}
{श्रेयसस्त्वद्धिते युक्तांस्तत्तद्वक्तुमिहेच्छसि}


\twolineshloka
{गच्छ त्वमपि कौन्तेयमात्मार्थे जहि माचिरम्}
{त्वमप्याशंससे योद्धुं कुलजः क्षत्रियो ह्यसि}


\twolineshloka
{इमान्किं क्षत्रियान्सर्वान्घातयिष्यस्यनागसः}
{त्वस्य मूलं वैरस्य तस्मादासादयार्जुनम्}


\twolineshloka
{एष ते मातुलः प्राज्ञः क्षत्रधर्ममनुव्रतः}
{दुर्द्युतदेवी गान्धारे प्रयात्वर्जुनमाहवे}


\twolineshloka
{एषोऽक्षकुशलो जिह्मो द्यूतकृत्कितवः शठः}
{देविता निकृतिप्रज्ञो युधि जेष्यति पाण्डवान्}


\twolineshloka
{त्वया कथितमत्यर्थं कर्णेन सह हृष्टवत्}
{असकृच्छून्यवन्मोहाद्वृतराष्ट्रस्य शृण्वतः}


\twolineshloka
{अहं च तात कर्णश्च भ्राता दुःशानश्च मे}
{पाण्डुपुत्रान्हनिष्यामः सहिताः समरे त्रयः}


\twolineshloka
{इति ते कत्थमानस्य श्रुतं संसदिसंसदि}
{अनुतिष्ठ प्रतिज्ञां तां सत्यवाग्भव तैः सह}


\twolineshloka
{एष ते पाण्डवः शत्रुरविशङ्कोऽग्रतः स्थितः}
{क्षत्रधर्ममवेक्षस्व श्लाघ्यस्तव वधो जयात्}


\twolineshloka
{दत्तं भुक्तमधीतं च प्राप्तमैश्वर्यमीप्सितम्}
{कृतकृत्योऽनृणश्चासि मा भैर्युध्यस्व पाण्डवम्}


\twolineshloka
{इत्युक्त्वा समरे द्रोणो न्यवर्तत यतः परे}
{द्वैधीकृत्य ततः सेनां युद्धं समभवत्तदा}


\chapter{अध्यायः १८७}
\twolineshloka
{सञ्जय उवाच}
{}


\twolineshloka
{त्रिभागमात्रेषायां रात्र्यां युद्धमवर्तत}
{कुरूणां पाण्डवानां च संहृष्टानां विशाम्पते}


\twolineshloka
{अथ चन्द्रप्रभां मुष्णन्नादित्यस्य पुरःसरः}
{अरुणोऽभ्युदयाञ्चक्रे ताम्रीकृर्वन्निवाम्बरम्}


\threelineshloka
{`प्रकाशमकरोद्व्योम जगत्संरञ्जयन्निव'}
{प्राच्यां दिशि सहस्रांशोररुणेनारुणीकृतम्}
{तापनीयं यथा चक्रं भ्राजते रविमण्डलम्}


\twolineshloka
{ततो रथाश्वांश्च मनुष्ययाना--न्युत्सृज्य सर्वे कुरुपाण्डुयोधाः}
{दिवाकरस्याभिमुखं जपन्तःसन्ध्यागताः प्राञ्जलयो बभूवुः}


\twolineshloka
{ततो द्वैधीकृते सैन्ये द्रोणः सोमकपाण्डवान्}
{अभ्यद्रवत्सपाञ्चालान्दुर्योधनपुरोगमः}


\twolineshloka
{द्वैधीकृतान्कुरून्दृष्ट्वा माधवोऽर्जुनमब्रवीत्}
{सपत्नान्सव्यतः कुर्याः सव्यसाचिन्निमान्कुरून्}


\twolineshloka
{स माधवमनुज्ञाय कुरुष्वेति धनञ्जयः}
{द्रोणकर्णौ महेष्वासौ सव्यतः पर्यवर्तत}


\threelineshloka
{अभिप्रायं तु कृष्णस्य ज्ञात्वा परपुरञ्जयः}
{आजिशीर्षगतं पार्थं भीमसेनोऽभ्युवाच ह ॥भीमसेन उवाच}
{}


\twolineshloka
{अर्जुनार्जुन बीभत्सो शृणुष्वैतद्वचो मम}
{यदर्थं क्षत्रिया सूते तस्य कालोऽयमागतः}


\twolineshloka
{अस्मिंश्चेदागते काले श्रेयो न प्रतिपत्स्यसे}
{असम्भावितरूपस्त्वं सुनृशंसं करिष्यसि}


\threelineshloka
{सत्यश्रीधर्मयशसां वीर्येणानृण्यमाप्नुहि}
{भिन्ध्यनीकं युधांश्रेष्ठ प्रतिज्ञां सफलां कुरु ॥सञ्जय उवाच}
{}


\twolineshloka
{स सव्यसाची भीमेन चोदितः केशवेन च}
{कर्णद्रोणावतिक्रम्य समन्तात्पर्यवारयत्}


\threelineshloka
{तमाजिशीर्षमायान्तं दहन्तं क्षत्रियर्षभान्}
{पराक्रान्तं पराक्रम्य ततः क्षत्रियपुङ्गवाः}
{नाशक्नुवन्वारयितुं वर्धमानमिवानलम्}


\twolineshloka
{अथ दुर्योधनः कर्णः शकुनिश्चापि सौबलः}
{अभ्यवर्षञ्छरव्रातैः कुन्तीपुत्रं धनञ्जयम्}


\twolineshloka
{तेषामस्त्राणि सर्वेषामुत्तमास्त्रविदां वरः}
{कदर्थीकृत्य राजेन्द्र शरवर्षैरवाकिरत्}


\twolineshloka
{अस्त्रैरस्त्राणि संवार्य लघुहस्तो धनञ्जयः}
{सर्वानविध्यन्निशितैर्दशभिर्दशभिः शरैः}


\twolineshloka
{उद्धूता रजसो वृष्टिः शरवृष्टिस्तथैव च}
{तमश्च घोरं शब्दश्च तदा समभवन्महान्}


\twolineshloka
{न द्यौर्न भूमिर्न दिशः प्राज्ञायन्त तथागते}
{सैन्येन रजसा मूढं सर्वमन्घमिवाभवत्}


\threelineshloka
{नैव ते न वयं राजन्प्राज्ञासिष्म परस्परम्}
{`शब्दमात्रेण जानीमो वयं ते च परस्परम्'}
{उद्देशेन हि तेन स्म समयुध्यन्त पार्थिवाः}


\twolineshloka
{विरथा रथिनो राजन्समासाद्य परस्परम्}
{केशेषु समसज्जन्त कवचेषु भुजेषु च}


\twolineshloka
{हताश्वा हतसूताश्च निश्चेष्टा रथिनो हताः}
{जीवन्त इव तत्र स्म व्यदृश्यन्त भयार्दिताः}


\twolineshloka
{हतान्गजान्समाश्लिष्य पर्वतानिव वाजिनः}
{गतसत्वा व्यदृश्यन्त तथैव सह सादिभिः}


\twolineshloka
{ततस्त्वभ्यवसृत्यैव सङ्ग्रामादुत्तरां दिशम्}
{आतिष्ठदाहवे द्रोणो विधूमोऽग्निरिव ज्वलन्}


\twolineshloka
{तमाजिशीर्षादेकान्तमपक्रान्तं निशम्य तु}
{समकम्पन्त सैन्यानि पाण्डवानां विशाम्पते}


\twolineshloka
{भ्राजमानं श्रिया युक्तं ज्वलन्तमिव तेजसा}
{द्रोणं दृष्ट्वा परे त्रेसुश्चेरुर्मम्लुश्च भारत}


\twolineshloka
{आह्वयन्तं परानीकं प्रभिन्नमिव वारणम्}
{नैनमाशंसिरे जेतुं दानवा वासवं यथा}


\twolineshloka
{केचिदासन्निरुत्साहाः केचित्क्रुद्धा मनस्विनः}
{विस्मिताश्चाभवन्केचित्केचिद्वृष्टाऽभवन्युधि}


\twolineshloka
{हस्तैर्हस्ताग्रमपरे प्रत्यपिम्पन्नराधिपाः}
{अपरे दशनैरोष्ठानदशन्क्रोधमूर्च्छिताः}


\twolineshloka
{व्याक्षिपन्नायुधान्यन्ये ममृदुश्चापरे भुजान्}
{अन्ये चान्वपतन्द्रोणं त्यक्तात्मानो महौजसः}


\twolineshloka
{पाञ्चालास्तु विशेषेण द्रोणसायकपीडिताः}
{समसज्जन्त राजेन्द्र समरे भृशवेदनाः}


\twolineshloka
{ततो विराटद्रुपदौ द्रोणं प्रति ययू रणे}
{तथा चरन्तं सङ्ग्रामे भृशं समरदुर्जयम्}


\twolineshloka
{द्रुपदस्य ततः पौत्रास्त्रय एव विशाम्पते}
{चेदयश्च महेष्वासा द्रोणमेवाभ्ययुर्युधि}


\twolineshloka
{तेषां द्रुपदपौत्राणां त्रयाणां निशितैः शरैः}
{त्रिभिर्द्रोणोऽहरत्प्राणांस्ते हता न्यपतन्भुवि}


\twolineshloka
{ततो द्रोणोऽजयद्युद्धे चेदिकैकेयसृञ्जयान्}
{मात्स्यांश्चैवाजयत्कृत्स्नान्भारद्वाजो महारथान्}


\twolineshloka
{ततस्तु द्रुपदः क्रोधाच्छरवर्षमवासृजत्}
{द्रोणं प्रति महाराज विराटश्चैव संयुगे}


\twolineshloka
{तं निहत्येषुवर्षं तु द्रोणः क्षत्रियमर्दनः}
{तौ शरैश्छादयामास विराटद्रुपदावुभौ}


\twolineshloka
{द्रोणेन च्छाद्यमानौ तु क्रुद्धौ सङ्ग्राममूर्धनि}
{द्रोणं शरैर्विव्यधतुः परमं क्रोधमास्थितौ}


\twolineshloka
{तत द्रोणो महाराज क्रोधामर्षसमन्वितः}
{भल्लाभ्यां भृशतीक्ष्णाभ्यां चिच्छेद धनुषी तयोः}


\twolineshloka
{ततो विराटः कुपितः समरे तोमरान्दश}
{दश चिक्षेप च शरान्द्रोणस्य वधकाङ्क्षया}


\twolineshloka
{शक्तिं च द्रुपदो घोरामायसीं स्वर्णभूषिताम्}
{चिक्षेप भुजगेन्द्राभां क्रुद्धो द्रोणरथं प्रति}


\twolineshloka
{ततो भल्लैः सुनिशितैश्छित्त्वा तांस्तोमारान्दश}
{शक्तिं कनकवैदूर्यां द्रोणश्चिच्छेद सायकैः}


\twolineshloka
{ततो द्रोणः सुपीताभ्यां भल्लाभ्यामरिमर्दनः}
{द्रुपदं च विराटं च प्रेषयामास मृत्यवे}


\twolineshloka
{हते विराटे द्रुपदे केकयेषु तथैव च}
{तथैव चेदिमात्स्येषु पाञ्चालेषु तथैव च}


\threelineshloka
{हतेषु त्रिषु वीरेषु द्रुपदस्य च नप्तृषु}
{द्रोणस्य कर्म तद्दृष्ट्वा कोपदुःखसमन्वितः}
{शशाप रथिनां मध्ये धृष्टद्युम्नो महामनाः}


\twolineshloka
{इष्टापूर्तात्तथा क्षात्राद्ब्राह्मण्याच्च स नश्यतु}
{द्रोणो यस्याद्य मुच्येत यं वा द्रोणः पराभवेत्}


\twolineshloka
{इति तेषां प्रतिश्रुत्य मध्ये सर्वधनुष्मताम्}
{आयाद्द्रोणं सहानीकाः पाञ्चाल्यऋ परवीरहा}


\threelineshloka
{पाञ्चालास्त्वेकतो द्रोणभ्यघ्नन्पाण्डवैः सह}
{दुर्योधनश्च कर्णश्च शकुनिश्चापि सौबलः}
{सोदर्याश्च यथा मुख्यास्तेऽरक्षन्द्रोणमाहवे}


\twolineshloka
{रक्ष्यमाणं तथा द्रोणं सर्वैस्तैस्तु महारथैः}
{यतमानास्तु पाञ्चाला न शेकुः प्रतिवीक्षितुम्}


\threelineshloka
{तत्राक्रुध्यद्भीमसेनो धृष्टद्युम्नस्य मारिष}
{स एनं वाग्भिरुग्राभिस्ततक्ष पुरुषर्षभः ॥भीमसेन उवाच}
{}


\twolineshloka
{द्रुपदस्य कुले जातः सर्वास्त्रेष्वस्त्रवित्तमः}
{कः क्षत्रियो मन्यमानः प्रेक्षेतारिमवस्थितम्}


\twolineshloka
{पितृपुत्रवधं प्राप्य पुंस्त्वं को वा प्रहर्षयेत्}
{विशेषतस्तु शपथं शपित्वा राजसंसदि}


\twolineshloka
{एष वैश्वानर इव समिद्धः स्वेन तेजसा}
{शरचापेन्धनो द्रोणः क्षत्रं दहति तेजसा}


\twolineshloka
{पुरा करोति निःशेषां पाण्डवानामनीकिनीम्}
{स्थिताः पश्यत मे कर्म द्रोणमेव व्रजाम्यहम्}


\twolineshloka
{इत्युक्त्वा प्राविशत्क्रुद्धो द्रोणानीकं वृकोदरः}
{शरैः पूर्णायतोत्सृष्टौर्द्रावयंस्तव वाहिनीम्}


\twolineshloka
{धृष्टद्युम्नोऽपि पाञ्चाल्यः प्रविश्य महतीं चमूम्}
{आससाद रणे द्रोणं तदासीत्तुमुलं महत्}


\twolineshloka
{नैव नस्तादृशं युद्धं दृष्टपूर्वं न च श्रुतम्}
{यथा सूर्योदये राजन्समुत्पिञ्जोऽभवन्महान्}


\twolineshloka
{संसक्तान्येव चादृश्यन्रथवृन्दानि मारिष}
{हतानि च विकीर्णानि सरीराणि शरीरिणाम्}


\twolineshloka
{केचिदन्यत्र गच्छन्तः पथि चान्यैरुपद्रुताः}
{विमुखाः पृष्ठतश्चान्ये ताड्यन्ते पार्श्वतः परे}


\twolineshloka
{तथा संसक्तयुद्धं तदभवद्भृशदारुणम्}
{अथ सन्ध्यागतः सूर्यः क्षणेन समपद्यत}


\chapter{अध्यायः १८८}
\twolineshloka
{सञ्जय उवाच}
{}


\twolineshloka
{ते तथैव महाराज दंशिता रणमूर्धनि}
{सन्ध्यागतं सहस्रांशुमादित्यमुपतस्थिरे}


\twolineshloka
{उदिते तु सहस्रांशौ तप्तकाञ्चनसप्रभे}
{प्रकाशितेषु लोकेषु पुनर्युद्धमवर्तत}


\twolineshloka
{द्वन्द्वानि तत्र यान्यासन्संसक्तानि पुरोदयात्}
{तान्येवाभ्युदिते सूर्ये समसज्जन्त भारत}


\twolineshloka
{रथैर्हया हयैर्नागाः पादातैश्चापि कुञ्जराः}
{हयैर्हयाः समाजग्मुः पादाताश्च पदातिभिः}


\twolineshloka
{रथा रथैरिभैर्नागास्तथैव भरतर्षभ}
{संसक्ताश्च वियुक्ताश्च योधाः सन्न्यपतन्रणे}


\twolineshloka
{ते रात्रौ कृतकर्माणः श्रान्ताः सूर्यस्य तेजसा}
{क्षुत्पिपासापरीताङ्गा विसंज्ञा बहवोऽभवन्}


\twolineshloka
{शङ्खबेरीमृदङ्गानां कुञ्जराणां च गर्जताम्}
{विष्फारितविकृष्टानां कार्मुकाणां च कूजताम्}


\twolineshloka
{शब्दः समभवद्राजन्दिविस्पृग्भरतर्षभ}
{द्रवतां च पदातीनां शस्त्राणां पततामपि}


\twolineshloka
{हयानां हेषतां चापि रथानां च निवर्तताम्}
{क्रोशतां गर्जतां चैव तदासीत्तुमुलं महत्}


\twolineshloka
{विवृद्धस्तुमुलः शब्दो द्यामगच्छन्महांस्तदा}
{नानायुधनिकृत्तानां चेष्टतामातुरः स्वनः}


\twolineshloka
{भूमावश्रूयत महांस्तदाऽऽसीत्कृपणं महत्}
{पततां पात्यमानानां पत्त्यश्वरथदन्तिनाम्}


\threelineshloka
{तेषु सर्वेष्वनीकेषु व्यतिषक्तेष्वनेकशः}
{स्वे स्वाञ्जघ्नुः परे स्वांश्च स्वान्परेषां परे परान्}
{}


\twolineshloka
{वीरबाहुविसृष्टाश्च योधेषु च गजेषु च}
{राशयः प्रत्यदृश्यन्त वाससां नेजनेष्विव}


\twolineshloka
{उद्यतप्रतिपिष्टानां खङ्गानां वीरबाहुभिः}
{स एव शब्दस्तद्रूपो वाससां निज्यतामिव}


\twolineshloka
{अर्धासिभिस्तथा खङ्गैस्तोमरैः सपरश्वथैः}
{निकृष्टयुद्धं संसक्तं महदासीत्सुदारुणम्}


\twolineshloka
{गजाश्वकायप्रभवां नरदेहप्रवाहिनीम्}
{शस्त्रमत्स्यसुसम्पूर्णां मांसशोणितकर्दमाम्}


\twolineshloka
{आर्तनादस्वनवतीं पताकाशस्त्रफेनिलाम्}
{नदीं प्रावर्तयन्वीराः परलोकौघगामिनीम्}


\twolineshloka
{शरशक्त्यर्दिता क्लान्ता रात्रिमूढात्पचेतसः}
{विष्टभ्य सर्वगात्राणि व्यतिष्ठन्गजवाजिनः}


\twolineshloka
{संशुष्कवदना वीराः शिरोभिश्चारुकुण्डलैः}
{युद्धोपकरणैश्चान्यैस्तत्रतत्र चकाशिरे}


\twolineshloka
{क्रव्यादसङ्घैराकीर्णं मृतैरर्धमृतैरपि}
{नासीद्रथपथस्तत्र सर्वमायोधनं प्रति}


\threelineshloka
{मञ्जत्सु चक्रेषु रथान्सत्वमास्थाय वाजिनः}
{कथञ्चिदवहन्श्रान्ता वेपमानाः शरार्दिताः}
{कुलसत्वबलोपेता वाजिनो वारणोपमाः}


\twolineshloka
{विह्वलं तूर्णमुद्धान्तं सभयं भारतातुरम्}
{बलमासीत्तदा सर्वमृते द्रोणार्जुनावुभौ}


\twolineshloka
{तावेवास्तां निलयनं तावार्तायनमेव च}
{तावेवान्ये समासाद्य जग्मुर्वैवस्वतक्षयम्}


\twolineshloka
{आविग्नमभवत्सर्वं कौरवाणां महद्बलम्}
{पाञ्चालानां च संसक्तं न प्राज्ञायत किञ्चन}


\twolineshloka
{अन्तकाक्रीडसदृशं भीरूणां भयवर्धनम्}
{पृथिव्यां राजवंश्यानामुत्थिते महति क्षये}


\twolineshloka
{न तत्र कर्णं द्रोणं वा नार्जुनं न युधिष्ठिरम्}
{न भीमसेनं न यमौ न पाञ्चाल्यं न सात्यकिम्}


\twolineshloka
{न च दुःशासनं द्रौणिं न दुर्योधनसौबलौ}
{न कृपं मद्रराजं च कृतवर्माणमेव च}


\twolineshloka
{न चान्यान्नैव चात्मानं न क्षितिं न दिशस्तथा}
{पश्याम राजन्संसक्तान्सैन्येन रजसा धृतान्}


\twolineshloka
{सम्भ्रान्ते तुमुले घोरे रजोमेघे समुत्थिते}
{त्रियामामिव संप्राप्ताममन्यन्त निशाचराः}


\twolineshloka
{न ज्ञायन्ते कौरवेया न पाञ्चाला न पाण्डवाः}
{न दिशो द्यौर्न चोर्वी च न समं विषमं तथा}


\twolineshloka
{हस्तसंस्पर्शमापन्नान्परानप्यथवा स्वकान्}
{न्यपातयंस्तदा युद्धे नराः स्म विजयैषिणः}


\twolineshloka
{उद्धूतत्वात्तु रजसः प्रसेकाच्छोणितस्य च}
{प्राशाम्यत रजो भौमं शीघ्रत्वादनिलस्य च}


\twolineshloka
{तत्र नागा हया योधा रथिनोऽथ पदातयः}
{पारिजातवनानीव व्यरोचन्रुधिरोक्षिताः}


\twolineshloka
{ततो दुर्योधनः कर्णो द्रोणो दुःशासनस्तथा}
{पाण्डवैः समसज्जन्त चतुर्भिश्चतुरो रथाः}


\twolineshloka
{दुर्योधनः सह भ्रात्रा यमाभ्यां समसज्जत}
{वृकोदरेण राधेयो भारद्वाजेन चार्जुनः}


\twolineshloka
{तद्धोरं महदाश्चर्यं युद्धं दैवासुरोपमम्}
{रथर्षभाणामुग्राणां सन्निपातममानुषम्}


\twolineshloka
{रथमार्गैर्विचित्रैस्तैर्विचित्ररथसङ्कुलम्}
{अपश्यन्रथिनो युद्धं विचित्रं चित्रयोधिनाम्}


\twolineshloka
{यतमानाः पराक्रान्ताः परस्परजिगीषवः}
{जीमूता इव घर्मान्ते शरवर्षैरवाकिरन्}


\twolineshloka
{ते रथान्सूर्यसङ्काशानास्थिताः पुरुषर्षभाः}
{अशोभन्त यथा मेघाः शारदाश्चलविद्युतः}


\threelineshloka
{योधास्ते तु महाराज क्रोधामर्षसमन्विताः}
{स्पर्धिनश्च महेष्वासाः कृतयत्ना धनुर्धराः}
{अभ्यगच्छंस्तथाऽन्योन्यं मत्ता गजवृषा इव}


\twolineshloka
{न नूनं देहभेदोऽस्ति काले राजन्ननागते}
{यत्र सर्वे न युगपद्व्यशीर्यन्त महारथाः}


\twolineshloka
{बाहुभिश्चपरणैश्चिन्नैः शिरोभिश्च सकुण्डलैः}
{कार्मुकैर्विशिखैः प्रासैः खङ्गैः परशुपट्टसैः}


\twolineshloka
{नालीकैः क्षुद्रनाराचैर्नखरैः शक्तितोमरैः}
{अन्यैश्च विविधाकारैर्धौतैः प्रहरणोत्तमैः}


\twolineshloka
{विचित्रैर्विविधाकारैः शरीरावरणैरपि}
{विचित्रैश्च रथैर्भग्नैर्हतैश्च गजवाजिभिः}


\twolineshloka
{शून्यैश्चैव नगाकारैर्हतयोधध्वजै रथैः}
{अमनुष्यैर्हयैस्त्रस्तैः कृष्यमाणैस्ततस्ततः}


\twolineshloka
{वातायमानैरसकृद्धतवीरैरलंकृतैः}
{व्यजनैः कङ्कटैश्चैव ध्वजैश्च विनिपातितैः}


\twolineshloka
{छत्रैराभरणैर्वस्त्रैर्माल्यैश्च ससुगन्धिभिः}
{हारैः किरीटैर्मुकुटैरुष्णीषैः किङ्किणीगणैः}


\twolineshloka
{उरस्थैर्मणिभिर्निष्कैश्चूडामणिभिरेव च}
{आसीदायोधनं तत्र नभस्तारागणैरिव}


\twolineshloka
{ततो दुर्योधनस्यासीन्नकुलेन समागमः}
{अमर्षितेन क्रुद्धस्य क्रुद्धेनामर्षितस्य च}


\twolineshloka
{अपसव्यं चकाराथ माद्रीपुत्रस्तवात्मजम्}
{किरञ्छरशतैर्हृष्टस्तत्र नादो महानभूत्}


\threelineshloka
{अपसव्यं कृतं सङ्ख्ये भ्रातृव्येनात्यमर्षिणा}
{नामृष्यत तमप्याजौ प्रतिचक्रे परन्तपः}
{पुत्रस्तव महाराज राजा दुर्योधनो द्रुतम्}


\twolineshloka
{ततः प्रतिचिकीर्षन्तमपसव्यं तु ते सुतम्}
{न्यवारयत तेजस्वी नकुलश्चित्रमार्गवित्}


\twolineshloka
{स सर्वतो निवार्यैनं शरजालेन पीडयन्}
{विमुखं नकुलश्चक्रे तत्सैन्याः समपूजयन्}


\twolineshloka
{तिष्ठतिष्ठेति नकुलो बभाषे तनयं तव}
{संस्मृत्य सर्वदुःखानि तव दुर्मन्त्रितं च तत्}


\chapter{अध्यायः १८९}
\twolineshloka
{सञ्जय उवाच}
{}


\twolineshloka
{ततो दुःशासनः क्रुद्धः सहदेवमुपाद्रवत्}
{रथवेगेन तीव्रेण कम्पयन्निव मेदिनीम्}


\twolineshloka
{तस्यापतत एवाशु भल्लेनामित्रकर्शनः}
{माद्रीपुत्रः शिरो यन्तुः सशिरस्त्राणमच्छिनत्}


\twolineshloka
{नैनं दुःशासनः सूतं नापि कश्चन सैनिकः}
{कृत्तोत्तमाङ्गमाशुत्वात्सहदेवेन बुद्धवान्}


\twolineshloka
{यदा त्वसङ्गृहीतत्वात्प्रयान्त्यश्वा यथासुखम्}
{ततो दुःशासनः सूतं बुबुधे गतचेतसम्}


\twolineshloka
{स हयान्सन्निगृह्याजौ स्वयं हयविशारदः}
{युयुधे रथिनां श्रेष्ठो लघु चित्रं च सुष्ठु च}


\twolineshloka
{तदस्यापूजयन्कर्म स्वे परे चापि संयुगे}
{हतसूतरथेनाजौ व्यचरद्यदभीतवत्}


\twolineshloka
{सहदेवस्तु तानश्वांस्तीक्ष्णैर्बाणैरवाकिरत्}
{पीड्यमानाः शरैश्चाशु प्राद्रवंस्ते ततस्ततः}


\twolineshloka
{स रश्मिषु विषक्तत्वादुत्सर्ज शरासनम्}
{धनुषा कर्म कुर्वंस्तु रश्मींश्च पुनरुत्सृजत्}


\twolineshloka
{छिद्रेष्वेतेषु तं बाणैर्माद्रीपुत्रोऽभ्यवाकिरत्}
{परीप्संस्त्वत्सुतं कर्णस्तदन्तरमवापतत्}


\twolineshloka
{वृकोदरस्ततः कर्णं त्रिभिर्भल्लैः समाहितः}
{आकर्णपूर्णैरभ्यघ्नद्बाह्वोरुरसि चानदत्}


\threelineshloka
{स निवृत्तस्ततः कर्णः सङ्घट्टित इवोरगः}
{भीममावारयामास विकिरन्निशिताञ्छरान्}
{ततोऽभूत्तुमुलं युद्धं भीमराधेययोस्तदा}


\twolineshloka
{तौ वृषाविव नर्दन्तौ विवृत्तनयनावुभौ}
{वेगेन महताऽन्योन्यं संरब्धावभिपेततुः}


\twolineshloka
{अभिसंश्लिष्टयोस्तत्र तयोराहवशौण्डयोः}
{विच्छिन्नशरपातत्वाद्गदायुद्धमवर्तत}


\twolineshloka
{गदया भीमसेनस्तु कर्णस्य रथकूबरम्}
{बिभेद शतधा राजंस्तदद्भुतमिवाभवत्}


\twolineshloka
{ततो भीमस्य राधेयो गदामाविध्य वीर्यवान्}
{अवासृजद्रथे तां तु बिभेद गदया गदाम्}


\twolineshloka
{ततो भीमः पुनर्गुर्वीं चिक्षेपाधिरथेर्गदाम्}
{तां गदां बहुभिः कर्णः सुपुङ्खैः सुप्रवेजितैः}


\twolineshloka
{प्रत्यविध्यत्पुनश्चान्यैः सा भीमं पुनराव्रजत्}
{व्यालीव मन्त्राभिहता कर्णबाणैरभिद्रुता}


\twolineshloka
{तस्याः प्रतिनिपातेन भीमस्य विपुलो ध्वजः}
{पपात सारथिश्चास्य मुमोह च गदाहतः}


\twolineshloka
{स कर्णं सायकानष्टौ व्यसृजत्क्रोधमूर्च्छितः}
{तैस्तस्य निशितैस्तीक्ष्णैर्भीमसेनो महाबलः}


\twolineshloka
{चिच्छेद परवीरघ्नः प्रहसन्निव भारत}
{ध्वजं शरासनं चैव शरावापं च भारत}


\threelineshloka
{कर्णोऽप्यन्यद्धनुर्गृह्य हेमपृष्ठं दुरासदम्}
{ततः पुनस्तु राधेयो हयानस्य रथेषुभिः}
{ऋक्षवर्णाञ्जघानाशु तथोभौ पार्ष्णिसारथी}


\twolineshloka
{स विपन्नरथो भीमो नकुलस्याप्लुतो रथम्}
{हरिर्यथा गिरेः शृङ्गं समाक्रमदरिन्दमः}


\twolineshloka
{तथा द्रोणार्जुनौ चित्रमयुध्येतां महारथौ}
{आचार्यशिष्यौ राजेन्द्र कृतप्रतिकृतौ युधि}


\twolineshloka
{लघुसन्धानयोगाभ्यां रथयोश्च रणेन च}
{मोहयन्तौ मनुष्याणां चक्षूंषि च मनांसि च}


\threelineshloka
{उपारमन्त ते सर्वे योधाऽस्माकं परे तथा}
{विचित्रान्पृतनामध्ये रथमार्गानुदीर्य तौ}
{}


% Check verse!
अन्योन्यमपसव्यं च कर्तुं वीरौ तदेषतुः
% Check verse!
पराक्रमं तयोर्योधा ददृशुस्ते सुविस्मिताः
\twolineshloka
{तयोः समभवद्युद्धं द्रोणपाण्डवयोर्महत्}
{आमिषार्थे महाराज गगने श्येनयोरिव}


\twolineshloka
{यद्यच्चकार द्रोणस्तु कुन्तीपुत्रजिगीषया}
{तत्तत्प्रतिजघानाशु प्रहसंस्तस्य पाण्डवः}


\twolineshloka
{यदा द्रोणो न शक्नोति पाण्डवं स्म विशेषितुम्}
{ततः प्रादुश्चकारास्त्रमस्त्रमार्गविशारदः}


\twolineshloka
{ऐन्द्रं पाशुपतं त्वाष्ट्रं वायव्यमथ वारुणम्}
{मुक्तंमुक्तं द्रोणचापात्तज्जघान धनञ्जयः}


\twolineshloka
{अस्त्राण्यस्त्रैर्यदा तस्य विधिवद्धन्ति पाण्डवः}
{ततोऽस्त्रैः परमैर्दिव्यैर्द्रोणः पार्थमवाकिरत्}


\twolineshloka
{यद्यदस्त्रं स पार्थाय प्रयुङ्क्ते विजिगीषया}
{तस्यतस्य विघाताय तत्तद्धि कुरुतेऽर्जुनः}


\twolineshloka
{स वध्यमानेष्वस्त्रेषु दिव्येष्वपि यथाविधि}
{अर्जुनेनार्जुनं द्रोणो मनसैवाभ्यपूजयत्}


\twolineshloka
{मेने चात्मानमधिकं पृथिव्यामधि भारत}
{तेन शिष्येण सर्वेभ्यः शस्त्रविद्भ्य परन्तपः}


\twolineshloka
{वार्यमाणस्तु पार्थेन तथा मध्ये महात्मनाम्}
{यतमानोऽर्जुनं प्रीत्या प्रीयते स्मार्जुनेन सः}


\twolineshloka
{ततोऽन्तरिक्षे देवाश्च गन्धर्वाश्च सहस्रशः}
{ऋषयः सिद्धसङ्घाश्च व्यतिष्ठन्त दिदृक्षया}


\twolineshloka
{तदप्सरोभिराकीर्णं यक्षगन्धर्वसङ्कुलम्}
{श्रीमदाकाशमभवद्भूयो मेघाकुलं यथा}


\twolineshloka
{तत्रास्मान्तर्हिता वाचो व्यचरन्त पुनः पुनः}
{द्रोणपार्थस्तवोपेता व्यश्रूयन्त नराधिप}


\twolineshloka
{विसृज्यमानेष्वस्त्रेषु ज्वालयस्तु दिशो दश}
{अब्रुवंस्तत्र सिद्धाश्च ऋषयश्च समागताः}


\twolineshloka
{नैवेदं मानुषं युद्धं नासुरं न च राक्षसम्}
{न दैवं न च गान्धर्वं ब्राह्मणं ध्रुवमिदं परम्}


\twolineshloka
{विचित्रमिदमाश्चर्यं न नो दृष्टं न च श्रुतम्}
{अति पाण्डवमाचार्यो द्रोणं चाप्यति पाण्डवः}


% Check verse!
नानयोरन्तरं शक्यं द्रष्टुमन्येन केनचित्
\twolineshloka
{यदि रुद्रो द्विधाकृत्य युध्येतात्मानमात्मना}
{तत्र शक्योपमा कर्तुमन्यत्र तु न विद्यते}


\threelineshloka
{ज्ञानमेकस्थमाचार्ये ज्ञानं योगश्च पाण्डवे}
{शौर्यमेकस्थमाचार्ये बलं शौर्यं च पाण्डवे}
{}


\twolineshloka
{नेमौ शक्यौ महेष्वासौ युद्धे क्षपयितुं परैः}
{इच्छमानौ पुनरिमौ हन्येतां सामरं जगत्}


\twolineshloka
{इत्यब्रुवन्महाराज दृष्ट्वा तौ पुरुषर्षभौ}
{अन्तर्हितानि भूतानि प्रकाशानि च सर्वशः}


\twolineshloka
{`यदा द्रोणं महाराज विशेषयति पाण्डवः'}
{ततो द्रोणो ब्राह्ममस्त्रं प्रादुश्चक्रे महामतिः}


\twolineshloka
{`तदस्त्रं संहितं राजन्धोररूपं महाहवे'}
{सन्तापयद्रणे पार्थं भूतान्यन्तर्हितानि च}


\threelineshloka
{ततश्चचाल पृथिवी सपर्वतवनद्रुमा}
{`सरितश्च प्रतिस्रोतः प्रवहुर्वै क्षणान्तरम्'}
{ववौ च विषमो वायुः सागराश्चापि चुक्षुभुः}


\twolineshloka
{ततस्त्रासो महानासीत्कुरुपाण्डवसेनयोः}
{सर्वेषां चैव भूतानामुद्यतेऽस्त्रे महात्मना}


\twolineshloka
{ततः पार्थोऽप्यसम्भ्रान्तस्तदस्त्रं प्रतिजघ्निवान्}
{ब्रह्मास्त्रेणैव राजेन्द्र ततः सर्वमशीशमत्}


\twolineshloka
{यदा न गम्यते पारं तयोरन्यतरस्य वा}
{ततः सङ्कुलयुद्धेन तद्युद्धं व्याकुलीकृतम्}


\twolineshloka
{नाज्ञायत ततः किञ्चित्पुनरेव विशाम्पते}
{प्रवृत्ते तुमुले युद्धे द्रोणपाण्डवयोर्नृप}


\twolineshloka
{`द्रोणो मुक्त्वा रणे पार्थं पाञ्चालानन्वधावत}
{अर्जुनोपि रणे द्रोणं त्यक्त्वा प्राद्रवयत्कुरून्}


\twolineshloka
{शरौघैरथ ताभ्यां तु छायाभूतं महामृधे}
{तुमलं प्रबभौ राजन्सर्वस्य जगतो भयम्'}


\twolineshloka
{शरजालैः समाकीर्णे मेघजालैरिवाम्बरे}
{नापतच्च ततः कश्चिदन्तरिक्षचरस्तदा}


\chapter{अध्यायः १९०}
\twolineshloka
{सञ्जय उवाच}
{}


\twolineshloka
{तस्मिंस्तथा वर्तमाने गजाश्वनरसंक्षये}
{दुःशासनो महाराज धृष्टद्युम्नमयोधयत्}


\twolineshloka
{स तु रुक्मरथासक्तो दुःशासनशरार्दितः}
{अमर्षात्तव पुत्रस्य शरैर्वाहानवाकिरत्}


\twolineshloka
{क्षणेन स रथस्तस्य सध्वजः सहसारथिः}
{नादृश्यत महाराज पार्षतस्य शरैश्चितः}


\twolineshloka
{दुःशासनस्तु राजेन्द्र पाञ्चाल्यस्य महात्मनः}
{नाशकत्प्रमुखे स्थातुं शरजालप्रपीडितः}


\twolineshloka
{स तु दुःशासनं बाणैर्विमुखीकृत्य पार्षतः}
{किऱञ्छरसहस्राणि द्रोणमेवाभ्ययाद्रणे}


\twolineshloka
{अभ्यपद्यत हार्दिक्यः कृतवर्मा त्वनन्तरम्}
{सोदर्याणां त्रयश्चैव त एनं पर्यवारयन्}


\twolineshloka
{तं यमौ पृष्ठतोऽन्वैतां रक्षन्तौ पुरुषर्षभौ}
{द्रोणायाभिमुखं यान्तं दीप्यमानमिवानलम्}


\twolineshloka
{सम्प्रहारमकुर्वंस्ते सर्वे च सुमहारथाः}
{अमर्षिताः सत्त्ववन्तः कृत्वा मरणमग्रतः}


\twolineshloka
{शुद्धात्मानः शुद्धवृत्ता राजन्स्वर्गपुरस्कृताः}
{आर्यं युद्धमकुर्वन्त परस्परजिगीषवः}


\twolineshloka
{शुक्लाभिजनकर्माणो मतिमन्तो जनाधिप}
{धर्मयुद्धमयुध्यन्त प्रेप्सन्तो गतिमुत्तमाम्}


\twolineshloka
{न तत्रासीदधर्मिष्ठमशस्तं युद्धमेव च}
{नात्र कर्णी न नालीको न लिप्तो न च बस्तिकः}


\threelineshloka
{न सूची कपिशो नैव न गवास्थिर्गजास्थिजः}
{न चूली बलिशस्तत्र न यमी नापि पाचकः}
{इषुरासीन्न संश्लिष्टो न पूतिर्न च जिह्मगः}


\twolineshloka
{ऋजून्येव विशुद्धानि सर्वे शस्त्राण्यधारयन्}
{सुयुद्धेन पराँल्लोकानीप्सन्तः कीर्तिमेव च}


\twolineshloka
{तदासीत्तुमुलं युद्धं सर्वदोषविवर्जितम्}
{चतुर्णां तव योधानां तैस्त्रिभिः पाण्डवैः सह}


\twolineshloka
{धृष्टद्युम्नस्तु तान्दृष्ट्वा तव राजन्रथर्षभान्}
{यमाभ्यां वारितान्वीराञ्छीघ्रास्त्रो द्रोणमभ्ययात्}


\twolineshloka
{निवारितास्तु ते वीरास्तयोः पुरुषसिंहयोः}
{समसज्जन्त चत्वारो वाताः पर्वतयोरिव}


\twolineshloka
{द्वाभ्यां द्वाभ्यां यमौ सार्धं रथाभ्यां रथपुङ्गवौ}
{समासक्तौ ततो द्रोणं धृष्टद्युम्नोऽभ्यवर्तत}


\twolineshloka
{दृष्ट्वा द्रोणाय पाञ्चाल्यं व्रजन्तं युद्धदुर्मदम्}
{यमाभ्यां तांश्च संसक्तांस्तदन्तरमुपाद्रवत्}


\twolineshloka
{दुर्योधनो महाराज किरञ्छोणितभोजनान्}
{तं सात्यकिः शीघ्रतरं पुनरेवाभ्यवर्तत}


\twolineshloka
{तौ परस्परमासाद्य समीपे कुरुमाधवौ}
{हसमानौ नृशार्दूलावभीतौ समसज्जताम्}


\twolineshloka
{बाल्यवृत्तानि सर्वाणि प्रीयमाणौ विचिन्त्य तौ}
{अन्योन्यं प्रेक्षमाणौ च स्मयमानौ पुनःपुनः}


\twolineshloka
{अथ दुर्योधनो राजा सात्यकिं समभाषत}
{प्रियं सखायं सततं गर्हयन्वृत्तमात्मनः}


\twolineshloka
{धिक् क्रोधं धिक्सखे लोभं धिङ्मोहं धिगमर्षितम्}
{धिगस्तु क्षात्रमाचारं धिगस्तु बलमौरसम्}


\twolineshloka
{यत्र मामभिसन्धत्से त्वां चाहं शिनिपुङ्गव}
{त्वं हि प्राणैः प्रियतरो ममाहं च सदा तव}


\threelineshloka
{स्मरामि तानि सर्वाणि बाल्यवृत्तानि यानि नौ}
{तानि सर्वाणि जीर्णानि साम्प्रतं नो रणाजिरे}
{किमन्यत्क्रोधलोभाभ्यां युद्धमेवाद्य सात्वत}


\twolineshloka
{तं तथावादिनं तत्र सात्यकिः प्रत्यभाषत}
{प्रहसन्विशिखांस्तीक्ष्णानुद्यम्य परमास्त्रवित्}


\threelineshloka
{नेयं सभा राजपुत्र नाचार्यस्य निवेशनम्}
{यत्र क्रीडितमस्माभिस्तदा राजन्समागतैः ॥दुर्योधन उवाच}
{}


\twolineshloka
{क्व सा क्रीडा गताऽस्माकं बाल्ये वै शिनिपुङ्गव}
{क्व च युद्धमिदं भूयः कालो हि दुरतिक्रमः}


\threelineshloka
{किन्तु नो विद्यते कृत्यं धनेन धनलिप्सया}
{यत्र युध्यामहे सर्वे धनलोभात्समागताः ॥सञ्जय उवाय}
{}


\twolineshloka
{तं तथावादिनं तत्र राजानं माधवोऽब्रवीत्}
{एवं वृत्तं सदा क्षात्रं युध्यन्तीह गुरूनपि}


\twolineshloka
{यदि तेऽहं प्रियो राजञ्जहि मां मा चिरं कृथाः}
{त्वत्कृते सुकृतांल्लोकान्गच्छेयं भरतर्षभ}


\twolineshloka
{या ते शक्तिर्बलं यच्च तत्क्षिप्रं मयि दर्शय}
{नेच्छामि तदहं द्रष्टुं मित्राणां व्यसनं महत्}


\twolineshloka
{इत्येवं व्यक्तमाभाष्य प्रतिभाष्य च सात्यकिः}
{अभ्ययात्तूर्णमव्यग्रो दयां नाकुरुतात्मनि}


\twolineshloka
{तमायान्तं महाबाहुं प्रत्यगृह्णात्तवात्मजः}
{शरैश्चावाकिरद्राजञ्शैनेयं तनयस्तव}


\twolineshloka
{ततः प्रववृते युद्धं कुरुमाधवसिंहयोः}
{अन्योन्यं क्रुद्धयोर्घोरं यथा द्विरदसिंहयोः}


\twolineshloka
{ततः पूर्णायतोत्सृष्टैः सात्वतं युद्धदुर्मदम्}
{दुर्योधनः प्रत्यविध्यत्कुपितो दशभिः शरैः}


\twolineshloka
{तं सात्यकिः प्रत्यविध्यत्तथैवावाकिरच्छरैः}
{पञ्चाशता पुनश्चाजौ त्रिंशता दशभिश्च ह}


\twolineshloka
{सात्यकिं तु रणे राजन्प्रहसंस्तनयस्तव}
{आकर्णपूर्णैर्निशितैर्विव्याध त्रिंशता शरैः}


% Check verse!
ततोऽस्य सशरं चापं क्षुरप्रेण द्विधाऽच्छिनत्
\twolineshloka
{सोऽन्यत्कार्मुकमादाय लघुहस्तस्ततो दृढम्}
{सात्यकिर्व्यसृजच्चापि शरश्रेणीं सुतस्य ते}


\twolineshloka
{तामापतन्तीं सहसा शरश्रेणीं जिघांसया}
{चिच्छेद बहुधा राजा तत उच्चुक्रुशुर्जनाः}


\twolineshloka
{सात्यकिं च त्रिसप्तत्या पीडयामास वेगितः}
{स्वर्णपुङ्खैः शिलाधौतैराकर्णापूर्णनिःसृतैः}


\twolineshloka
{तस्य सन्दधतश्चेषु संहितेषु च कार्मुकम्}
{आच्छिनत्सात्यकिस्तूर्णं शरैश्चैवाप्यवीविधत्}


\twolineshloka
{स गाढविद्धो व्यथितः प्रत्यपायाद्रथान्तरे}
{दुर्योधनो महाराज दाशार्हशरपीडितः}


\twolineshloka
{समाश्वस्य तु पुत्रस्ते सात्यकिं पुनरभ्ययात्}
{विसृजन्निषुजालानि युयुधानरथं प्रति}


\twolineshloka
{तथैव सात्यकिर्बाणान्दुर्योधनरथं प्रति}
{सततं विसृजन्राजंस्तत्सङ्कुलमवर्तत}


\twolineshloka
{तत्रेषुभिः क्षिप्यमाणैः पतद्भिश्च शरीरिषु}
{अग्नेरिव महाकक्षैः शब्दः समभवन्महान्}


\twolineshloka
{तयोः शरसहस्रैश्च सञ्छन्नं वसुधातलम्}
{अगम्यरूपं च शरैराकाशं समपद्यत}


\twolineshloka
{तत्राप्यधिकमालक्ष्य माधवं रथसत्तमम्}
{क्षिप्रमभ्यपतत्कर्णः परीप्संस्तनयं तव}


\twolineshloka
{न तु तं मर्षयामास भीमसेनो महाबलः}
{सोऽभ्ययात्त्वरितः कर्णं विसृजन्सायकान्बहून्}


\twolineshloka
{तस्य कर्णः शितान्बाणान्प्रतिहन्य हसन्निव}
{धनुः शरांश्च चिच्छेद सूतं चाभ्यहनच्छरैः}


\twolineshloka
{भीमसेनस्तु सङ्क्रुद्धो गदामादाय पाण्डवः}
{ध्वजं धनुश्च सूतं च सम्ममर्दाहवे रिपोः}


\twolineshloka
{रथचक्रं च कर्णस्य बभञ्ज स महाबलः}
{भग्नयक्रे रथेऽतिष्ठदकम्पः शैलराडिव}


\twolineshloka
{एकचक्रं रथं तस्य तमूहुः सुचिरं हयाः}
{एकचक्रमिवार्कस्य रथं सप्त हया यथा}


\twolineshloka
{अमृष्यमाणः कर्णस्तु भीमसेनमयुध्यत}
{विविधैरिषुजालैश्च नानाशस्त्रैश्च संयुगे}


\threelineshloka
{भीमसेनस्तु सङ्क्रुद्धः सूतपुत्रमयोधयत्}
{तस्मिंस्तथा वर्तमाने क्रुद्धो धर्मसुतोऽब्रवीत्}
{पाञ्चालानां नरव्याघ्रान्मात्स्यांश्च पुरुषर्षभान्}


\twolineshloka
{ये नः प्राणाः शिरो ये च ये नो योधा महारथाः}
{त एते धार्तराष्ट्रेषु विषक्ताः पुरुषर्षभान्}


\twolineshloka
{ये नः प्राणाः शिरो ये च ये नो योधा महारथाः}
{त एते धार्तराष्ट्रेषु विषक्ताः पुरुषर्षभाः}


\twolineshloka
{किं तिष्ठत यथा मूढाः सर्वे विगतचेतसः}
{तत्र गच्छत यत्रैते युध्यन्ते मामका रथाः}


\twolineshloka
{क्षत्रधर्मं पुरस्कृत्य सर्व एव गतज्वराः}
{जयन्तो वध्यमानाश्च गतिमिष्टां गमिष्यथ}


\twolineshloka
{ते राज्ञा चोदिता वीरा योत्स्यमाना महारथाः}
{चतुर्धा वाहिनीं कृत्वा त्वरिता द्रोणमभ्ययुः}


\twolineshloka
{पाञ्चालास्त्वेकतो द्रोणमभ्यघ्नन्निशितैः शरैः}
{भीमसेनपुरोगाश्चाप्येकतः पर्यवारयन्}


\twolineshloka
{आसंस्तु पाण्डुपुत्राणां त्रयो जिह्मा महारथाः}
{यमौ च भीमसेनश्च प्राक्रोशंस्ते धनञ्जयम्}


\twolineshloka
{अभिद्रवार्जुन क्षिप्रं कुरून्द्रोणादपानुद}
{तत एनं हनिष्यन्ति पाञ्चाला हतरक्षिणम्}


\threelineshloka
{कौरवेयांस्ततः पार्थः ससा समुपाद्रवत्}
{पाञ्चालानेव तु द्रोणो धृष्टद्युम्नपुरोगमान्}
{ममर्दुस्तरसा वीराः पञ्चमेऽहनि भारत}


\chapter{अध्यायः १९१}
\twolineshloka
{सञ्जय उवाच}
{}


\twolineshloka
{पाञ्चालानां ततो द्रोणोऽप्यकरोत्कदनं महत्}
{यथा क्रुद्धो रणे शक्रो दानवानां क्षयं पुरा}


\twolineshloka
{द्रोणास्त्रेण महाराज वध्यमानाः परे युधि}
{नात्रसन्त रणे द्रोणत्सत्ववन्तो महारथाः}


\twolineshloka
{युध्यमाना महाराज पाञ्चालाः सृञ्जयास्तथा}
{द्रोणमेवाभ्ययुर्युद्धे योधयन्तो महारथाः}


\twolineshloka
{तेषां तु च्छाद्यमानानां पाञ्चालानां समन्ततः}
{अभवद्भैरवो नादो वध्यतां शरवृष्टिभिः}


\twolineshloka
{वध्यमानेषु सङ्ग्रामे पाञ्चालेषु महात्मना}
{उदीर्यमाणे द्रोणास्त्रे पाण़्डवान्भयमाविशत्}


\twolineshloka
{दृष्ट्वाऽश्वनरयोधानां विपुलं च क्षयं युधि}
{पाण़्डवेया महाराज नाशशंसुर्जयं तदा}


\twolineshloka
{कच्चिद्द्रोणो न नः सर्वान्क्षपयेत्परमास्त्रवित्}
{समिद्धः शिशिरापाये दहन्कक्षमिवानलः}


\twolineshloka
{न चैनं संयुगे कश्चित्समर्थः प्रतिवीक्षितुम्}
{न चैनमर्जुनो जातु प्रतियुध्येत धर्मवित्}


\twolineshloka
{त्रस्तान्कुन्तीसुतान्दृष्ट्वा द्रोणसायकपीडितान्}
{मतिमाञ्श्रेयसे युक्तः केशवोऽर्जुनमब्रवीत्}


\twolineshloka
{नैष युद्धेन सङ्ग्रामे जेतुं शक्यः कथञ्चन}
{सधनुर्धन्विनां श्रेष्ठो देवैरपि सवासवैः}


\threelineshloka
{न्यस्तशस्त्रस्तु सङ्गारमे शक्यो हन्तुं भवेन्नृभिः}
{आस्थीयतां जये योगो धर्ममुत्सृज्य पाण्डवाः}
{यथा न संयुगे सर्वान्निहन्याद्रुक्मवाहनः}


\twolineshloka
{अश्वत्थाम्नि हते नैष युध्येदिति मतिर्मम}
{तं हतं संयुगे कश्चिदस्मै शंसतु मानवः}


\twolineshloka
{एतन्नारोचयद्राजन्कुन्तीपुत्रो धनञ्जयः}
{अन्ये त्वरोचयन्सर्वे कृच्छ्रेण तु युधिष्ठिरः}


\threelineshloka
{ततो भीमो महाबाहुरनीकेषु महागजम्}
{जघान गदया राजन्नश्वत्थामानमित्युत}
{परप्रमथनं घोरं मालवस्येन्द्रवर्मणः}


\twolineshloka
{भीमसेनस्तु सव्रीडमुपेत्य द्रोणमाहवे}
{अश्वत्थामा हत इति शब्दमुच्चैश्चकार ह}


\twolineshloka
{अश्वत्थामेति हि गजः ख्यातो नाम्ना हतोऽभवत्}
{कृत्वा मनसि तं भीमो मिथ्या व्याहृतवांस्तदा}


\twolineshloka
{भीमसेनवचः श्रुत्वा द्रोणस्तत्परमाप्रियम्}
{मनसा सन्नागात्रोऽभूद्यथा सैकतमम्भसि}


\twolineshloka
{शङ्कमानः स तन्मिथ्या वीर्यज्ञः स्वसुतस्य वै}
{हतः स इति च श्रुत्वा नैव धैर्यादकम्पत}


\twolineshloka
{स लब्ध्वा चेतनां द्रोणः क्षणेनैव समाश्वसत्}
{अनुचिन्त्यात्मनः पुत्रमविषह्यमरातिभिः}


\twolineshloka
{स पार्षतमभिद्रुत्य जिघांसुर्मृत्युमात्मनः}
{अवाकिरत्सहस्रेण तीक्ष्णानां कङ्कपत्रिणाम्}


\twolineshloka
{तं विंशतिसहस्राणि पाञ्चालानां नरर्षभाः}
{यथा चरन्तं सङ्ग्रामे सर्वतोऽवाकिरञ्छरैः}


\twolineshloka
{शरैस्तैराचितं द्रोणं नापश्याम महारथम्}
{भास्करं जलदै रुद्धं वर्षास्विव विशाम्पते}


\twolineshloka
{विधूय तान्बाणगणान्पाञ्चालानां महारथः}
{प्रादुश्चक्रे ततो द्रोणो ब्राह्ममस्त्रं परन्तपः}


\twolineshloka
{वधाय तेषां शूराणां पाञ्चालानाममर्षितः}
{ततो व्यरोचत द्रोणो विनिघ्नन्सर्वसैनिकान्}


\twolineshloka
{शिरांस्यापातयच्चापि पाञ्चालानां महामृधे}
{तथैव परिघाकारान्बाहून्कनकभूषणान्}


\twolineshloka
{ते वध्यमानाः समरे भारद्वाजेन पार्थिवाः}
{मेदिन्यामन्वकीर्यन्त वातनुन्ना इव द्रुमाः}


\twolineshloka
{कुञ्जराणां च पततां हयौघानां च भारत}
{अगम्यरूपा पृथिवी मांसशोणितकर्दमा}


\twolineshloka
{हत्वा विंशतिसाहस्रान्पाञ्चालानां रथव्रजान्}
{अतिष्ठदाहवे द्रोणो विधूमोऽग्निरिव ज्वलन्}


\twolineshloka
{तथैव च पुनः क्रुद्धो भारद्वाजः प्रतापवान्}
{वसुदानस्य भल्लेन शिरः कायादपाहरत्}


\twolineshloka
{पुनः पञ्चशतान्मात्स्यान्षट््सहस्रांश्च सृञ्जयान्}
{हस्तिनामयुतं हत्वा जघानाश्वायुतं पुनः}


\twolineshloka
{क्षत्रियाणामभावाय दृष्ट्वा द्रोणमवस्थितम्}
{ऋषयोऽभ्यागतास्तूर्णं हव्यवाहपुरोगमाः}


\twolineshloka
{विश्वामित्रो जमदग्निर्भरद्वाजोऽथ गौतमः}
{वसिष्ठः कश्यपोऽत्रिश्च ब्रह्मलोकं निनीषवः}


\twolineshloka
{सिकताः पृश्नयो गर्गा वालखिल्या मरीचिपाः}
{भृगवोऽङ्गिरसश्चैव सूक्ष्माश्चान्ये महर्षयः}


\twolineshloka
{त एनमब्रुवन्सर्वे द्रोणमाहरवशोभिनम्}
{अधर्मतः कृतं युद्धं समयो निधनस्य ते}


\twolineshloka
{न्यस्यायुधं रणे द्रोण समीक्षास्मानवस्थितान्}
{नातः क्रूरतरं कर्म पुनः कर्तुमिहार्हसि}


\twolineshloka
{वेदवेदाङ्गविदुषः सत्यधर्मरतस्य ते}
{ब्राह्मणस्य विशेषेण तवैतन्नोपपद्यते}


\twolineshloka
{त्यजायुधममोधेषो तिष्ठ वर्त्मनि शाश्वते}
{परिपूर्णश्च कालस्ते वस्तुं लोकेऽद्य मानुषे}


\twolineshloka
{ब्रह्मास्त्रेण त्वया दग्धा अनस्त्रज्ञा नरा भुवि}
{यदेतदीदृशं विप्र कृतं कर्म न साधु तत्}


\twolineshloka
{न्यस्यायुधं रणे विप्र द्रोण मा त्वं चिरं कृथाः}
{मा पापिष्ठतरं कर्म करिष्यसि पुनर्द्विज}


\twolineshloka
{इति तेषां वचः श्रुत्वा भीमसेनवचः स्मरन्}
{धृष्टद्युम्नं च सम्प्रेक्ष्य रणे स विमनाऽभवत्}


\twolineshloka
{सन्दिह्यमानो व्यथितः कुन्तीपुत्रं युधिष्ठिरम्}
{अहतं वा हतं वेति पप्रच्छ सुतमात्मनः}


\twolineshloka
{स्थिरा बुद्धिर्हि द्रोणस्य न पार्थो वक्ष्यतेऽनृतम्}
{त्रयाणामपि लोकानामैश्वर्यार्थे कथञ्चन}


\twolineshloka
{तस्मात्तं परिपप्रच्छ नान्यं कञ्छिद्द्विजर्षभः}
{तस्मिंस्तस्य हि सत्याशा बाल्यात्प्रभृति पाण्डवे}


\twolineshloka
{ततो निष्पाण्डवामुर्वीं करिष्यन्तं युधाम्पतिम्}
{द्रोणं ज्ञात्वा धर्मराजं गोविन्दो व्यथितोऽब्रवीत्}


\twolineshloka
{यद्यर्धदिवसं द्रोणो युध्यते मन्युमास्थितः}
{सत्यं ब्रवीमि ते सेना विनाशं समुपैष्यति}


\twolineshloka
{स भवांस्त्रातु नो द्रोणात्सत्याज्ज्यायोऽनृतं वचः}
{अनृतं जीवितस्यार्थे वदन्न स्पृश्यतेऽनृतैः}


% Check verse!
तयोः संवदतोरेवं भीमसेनोऽब्रवीदिदम्
\twolineshloka
{श्रुत्वैवं तु महाराज वधोपायं महात्मनः}
{गाहमानस्य ते सेनां मालवस्येन्द्रवर्मणः}


\twolineshloka
{अश्वत्थामेति विख्यातो गजः शक्रगजोपमः}
{निहतो युधि विक्रम्य ततोऽहं द्रोणमब्रुवम्}


\twolineshloka
{अश्वत्थामा हतो ब्रह्मन्निवर्तस्वाहवादिति}
{नूनं नाश्रद्दधद्वाक्यमेष मे पुरुषर्षभः}


\twolineshloka
{स त्वं गोविन्दवाक्यानि मानयस्व जयैषिणः}
{द्रोणाय निहतं शंस राजञ्शारद्वतीसुतम्}


\twolineshloka
{त्वयोक्ते नैव युध्येत जातु राजन्द्विजर्षभः}
{सत्यवागिति लोकेऽस्मिन्भवान्ख्यातो जनाधिप}


\twolineshloka
{तस्य तद्वचनं श्रुत्वा कृष्णवाक्यप्रचोदितः}
{भावित्वाच्च महाराज वक्तुं समुपचक्रमे}


\threelineshloka
{तमतथ्यभये मग्नो जये सक्तो युधिष्ठिरः}
{`अश्वत्थामा हति शब्दमुच्चैश्चकार ह'}
{अव्यक्तमब्रवीद्राजन्हतः कुञ्जर इत्युत}


\twolineshloka
{तस्य पूर्वं रथः पृथ्व्याश्चतुरङ्गुलमुच्छ्रितः}
{बभूवैवं च तेनोक्ते तस्य वाहाः स्पृशन्महीम्}


\twolineshloka
{युधिष्ठिरात्तु तद्वाक्यं श्रुत्वा द्रोणो महारथः}
{पुत्रव्यसनसन्तप्तो निराशो जीवितेऽभवत्}


\twolineshloka
{आगस्कृतमिवात्मानं पाण्डवानां महात्मनाम्}
{ऋषिवाक्येन मन्वानः श्रुत्वा च निहतं सुतम्}


\twolineshloka
{विचेताः परमोद्विग्नो धृष्टद्युम्नमवेक्ष्य च}
{योद्धुं नाशक्नुवद्राजन्यथापूर्वमरिन्दमः}


\chapter{अध्यायः १९२}
\twolineshloka
{सञ्जय उवाच}
{}


\twolineshloka
{तं दृष्ट्वा परमोद्विग्नं शोकोपहतचेतसम्}
{पाञ्चालराजस्य सुतो धृष्टद्युम्नः समाद्रवत्}


\twolineshloka
{य इष्ट्वा मनुजेन्द्रेण द्रुपदेन महामखे}
{लब्धो द्रोणविनाशाय समिद्धाद्धव्यवाहनात्}


\twolineshloka
{स धनुर्जैत्रमादाय घोरं जलदनिःस्वनम्}
{दृढज्यमजरं दिव्यं शरं चाशीविषोपमम्}


\twolineshloka
{सन्दधे कार्मुके तस्मिंस्ततस्तमनलोपमम्}
{द्रोणं जिघांस्तुः पाञ्चाल्यो महाज्वालमिवानलम्}


\twolineshloka
{तस्य रूपं शरस्यासीद्धनुर्ज्यामण्डलान्तरे}
{द्योततो भास्करस्येव घनान्ते परिवेषिणः}


\twolineshloka
{पार्षतेन परामृष्टं ज्वलन्तमिव तद्धनुः}
{अन्तकालमनुप्राप्तं मेनिरे वीक्ष्य सैनिकाः}


\twolineshloka
{तमिषुं संहतं तेन भारद्वाजः प्रतापवान्}
{दृष्ट्वाऽमन्यत देहस्य कालपर्यायमागतम्}


\twolineshloka
{ततः प्रयत्नमातिष्ठदाचार्यस्तस्य वारमे}
{न चास्यास्त्राणि राजेन्द्र पादुरासन्महात्मनः}


\twolineshloka
{तस्य त्वहानि चत्वारि क्षपा चैकाऽस्यतो गता}
{तस्य चाह्नस्त्रिभागेन क्षयं जग्मुः पतत्रिणः}


\twolineshloka
{स शरक्षयमासाद्य पुत्रशोकेन चार्दितः}
{विविधानां च दिव्यामानमस्त्राणामप्रसादतः}


\twolineshloka
{उत्स्रष्टुकामः शस्त्राणि ऋषिवाक्यप्रचोदितः}
{तेजसा पूर्यमाणश्च युयुधे न यथा पुरा}


\twolineshloka
{भूयश्चान्यत्समादाय दिव्यमाङ्गिरसं धनुः}
{शरांश्च ब्रह्मदण्डाभान्धृष्टद्युम्नमयोधयत्}


\twolineshloka
{ततस्तं शरवर्षेण महता समवाकिरत्}
{व्यशातयच्च सङ्क्रुद्धो धृष्टद्युम्नममर्षणम्}


\twolineshloka
{शरांश्च शतधा तस्य द्रोणश्चिच्छेद सायकैः}
{ध्वजं धनुश्च निशितैः सारथिं चाप्यपातयत्}


\twolineshloka
{धृष्टद्युम्नः प्रहस्यान्यत्पुनरादाय कार्मुकम्}
{शितेन चैनं बाणेन प्रत्यविध्यत्स्तनान्तरे}


\twolineshloka
{सोऽतिविद्धो महेष्वासोऽसम्भ्रान्त इव संयुगे}
{भल्लेन शितधारेण चिच्छेदास्य पुनर्धनुः}


\twolineshloka
{यच्चास्य बाणविकृतं धनूंषि च विशाम्पते}
{सर्वं चिच्छेद दुर्धर्षो गदां खङ्गं च वर्जयन्}


\twolineshloka
{धृष्टद्युम्नं च विव्याध नवभिर्निशितैः शरैः}
{जीवितान्तकरैः क्रुद्धः शिलाधौतैः परन्तपः}


\twolineshloka
{धृष्टद्युम्नोऽथ तस्याश्वान्खरथाश्वैर्महारथः}
{व्यामिश्रयदमेयात्मा ब्राह्ममस्त्रमुदीरयन्}


\twolineshloka
{ते मिश्रा बह्वशोभन्त जवना वातरंहसः}
{पारावतसवर्णाश्च शोणाश्च भरतर्षभ}


\twolineshloka
{यथा सविद्युतो मेघा नदन्तो जलदागमे}
{तथा रेजुर्महाराज मिश्रिता रणमूर्धनि}


\twolineshloka
{ईषाबन्धं चक्रबन्धं रथबन्धं तथैव च}
{प्राणाशयदमेयात्मा धृष्टद्युम्नस्य स द्विजः}


\twolineshloka
{स च्छिन्नधन्वा पाञ्चाल्यो निकृत्तध्वजसारथिः}
{उत्तमामापदं प्राप्य गदां वीरः परामृशत्}


\twolineshloka
{तामस्य विशिखैस्तीक्ष्णैः क्षिप्यमाणां महारथः}
{निजघानः शरैर्द्रोणः क्रुद्धः सत्यपराक्रमः}


\twolineshloka
{तां तु दृष्ट्वा नरव्याघ्रो द्रोणेन निहतां शरैः}
{विमलं खङ्गमादत्त शतचन्द्रं च भानुमत्}


\twolineshloka
{असंशयं तथाभूतः पाञ्चाल्यः साध्वमन्यत}
{वधमाचार्यमुख्यस्य प्राप्तकालं महात्मनः}


\twolineshloka
{ततः स रथनीडस्थं स्वरथस्य रथेपया}
{अगच्छदसिमुद्यम्य शतचन्द्रं च भानुमत्}


\twolineshloka
{चिकीर्षुर्दुष्करं कर्म धृष्टद्युम्नो महारथः}
{इयेष वक्षो भेत्तुं स भारद्वाजस्य संयुगे}


\twolineshloka
{सोऽतिष्ठद्युगमध्ये वै युगसन्नहनेषु च}
{जघनार्धेषु चाश्वानां तत्सैन्याः समपूजयन्}


\twolineshloka
{तिष्ठतो युगपालीपु शोणानप्यधितिष्ठतः}
{नापश्यदन्तरं द्रोणस्तदद्भुतमिवाभवत्}


\twolineshloka
{क्षिप्रं श्येनस्य चरतो यथैवामिषगृद्धिनः}
{तद्वदासीदभीसारो द्रोमपार्षतयो रणे}


\twolineshloka
{तस्य पारावतानश्वान्रथशक्त्या पराभिनत्}
{सर्वानेकैकशो द्रोणो रक्तानश्वान्विवर्जयन्}


\twolineshloka
{ते हता न्यपतन्भूमौ धृष्टद्युम्नस्य वाजिनः}
{शोणास्तु पर्यमुच्यन्त रथबन्धाद्विशाम्पते}


\twolineshloka
{तान्हयान्निहतान्दृष्ट्वा द्विजाग्र्येण स पार्षतः}
{नामृष्यत युधां श्रेष्ठो याज्ञसेनिर्महारथः}


\twolineshloka
{विरथः स गृहीत्वा तु खङ्गं खङ्गभृतां वरः}
{द्रोणमभ्यपतद्राजन्वैनतेय इवोरगम्}


\twolineshloka
{तस्य रूपं बभौ राजन्भारद्वाजं जिघांसतः}
{यथा रूपं पुरा विष्णोर्हिरण्यकशिपोर्वधे}


\twolineshloka
{स तदा विविधान्मर्गान्प्रवरांश्चैकविंशतिम्}
{दर्शयामास कौरव्य पार्षतो विचरन्रणे}


\twolineshloka
{भ्रान्तुमुद्धान्तमाविद्धमाप्लुतं प्रसृतं सृतम्}
{परिवृत्तं निवृत्तं च खङ्गं चर्म च धारयन्}


\threelineshloka
{सम्पातं समुदीर्णं च दर्शयामास पार्षतः}
{भारतं कौशिकं चैव सात्वतं चैव शिक्षया}
{}


\twolineshloka
{दर्शयन्व्यचरद्युद्धे द्रोणस्यान्तचिकीर्षया ॥चरतस्तस्य तान्मर्गान्विचित्रान्खङ्गचर्मिणः}
{व्यस्मयन्त रणे योधा देवताश्च समागताः}


\twolineshloka
{ततः शरसहस्रेण शतचन्द्रमपातयत्}
{चर्म खङ्गं च सम्बाधे धृष्टद्युम्नस्य स द्विजः}


\twolineshloka
{ये तु वैतस्तिका नाम शरा आसन्नयोधिनः}
{निकृष्टयुद्धे द्रोणस्य नान्येषां सन्ति ते शराः}


\twolineshloka
{ऋते शारद्वतात्पर्थाद्द्रौणेर्वैकर्तनात्तथा}
{प्रद्युम्नयुयुधानाभ्यामभिमन्योश्च भारत}


\twolineshloka
{अथास्येषु समाधत्त दृढं परमसम्मतम्}
{अन्तेवासिनमाचार्यो जिघांसुः पुत्रसम्मितम्}


\threelineshloka
{तं शरैर्दशभिस्तीक्ष्णैश्चिच्छेद शिनिपुङ्गवः}
{पश्यतस्तव पुत्रस्य कर्णस्य च महात्मनः}
{ग्रस्तमाचार्यमुख्येन धृष्टद्युम्नममोचयत्}


\threelineshloka
{चरन्तं रथमार्गेषु सात्यकिं सत्यविक्रमम्}
{द्रोणकर्मान्तरगतं कृपस्यापि च भारत}
{अपश्येतां महात्मानौ विष्वक्सेनधनञ्जयौ}


\twolineshloka
{अपूजयेतां वार्ष्णेयं ब्रुवाणौ साधुसाध्विति}
{दिव्यान्यस्त्राणि सर्वेषां युधि निघ्न्तमच्युतम्}


\twolineshloka
{धनञ्जयस्ततः कृष्णमब्रवीत्पश्य केशव}
{आचार्यरथमुख्यानां मध्ये क्रीडन्मधूद्वहः}


\twolineshloka
{आनन्दयति मां भूयः सात्यकिः परवीरहा}
{माद्रीपुत्रौ च भीमं च राजानं च युधिष्ठिरम्}


\twolineshloka
{यच्छिक्षयाऽनुद्धतः सन्रणे चरति सात्यकिः}
{महारथानुपक्रीडन्वृष्णीनां कीर्तिवर्धनः}


\threelineshloka
{तमेते प्रतिनन्दन्ति सिद्धाः सैन्याश्च विस्मिताः}
{अजय्यं समरे दृष्ट्वा साधुसाध्विति सात्यकिम्}
{योधाश्चोभयतः सर्वे कर्मभिः समपूजयन्}


\chapter{अध्यायः १९३}
\twolineshloka
{सञ्चय उवाच}
{}


\twolineshloka
{[सात्वतस्य* तु तत्कर्म दृष्ट्वा दुर्योधनादयः}
{शैनेयं सर्वतः क्रुद्धा वारयामासुरञ्जसा}


% Check verse!
कृपकर्णौ च समरे पुत्राश्च तव मारिष ॥शैनेयं त्वरयाऽभ्येत्य विनिघ्नन्निशितैः शरैः
\twolineshloka
{युधिष्ठिरस्ततो राजा माद्रीपुत्रौ च पाण्डवौ}
{भीमसेनश्च बलवान्सात्यकिं पर्यवारयन्}


\twolineshloka
{कर्णश्च शरवर्षेण गौतमश्च महारथः}
{दुर्योधनादयस्ते च शैनेयं पर्यवारयन्}


\twolineshloka
{तां वृष्टिं सहसा राजन्नुत्थितां घोररूपिणीम्}
{वारयामास शैनेयो योधयंस्तान्महारथान्}


\twolineshloka
{तेषामस्त्राणि दिव्यानि संहितानि महात्मनाम्}
{वारयामास विधिवद्दिव्यैरस्त्रैर्महामृधे}


\twolineshloka
{क्रूरमायोधनं जज्ञे तस्मिन्राजसमागमे}
{रुद्रस्येव हि क्रुद्धस्य निघ्नतस्तान्पशून्पुरा}


\threelineshloka
{हस्तानामुत्तमाङ्गानां कार्मुकाणां च भारत}
{छत्राणां चापविद्धानां चामराणां च सञ्चयैः}
{राशयः स्म व्यदृश्यन्त तत्रतत्र रणाजिरे}


\twolineshloka
{भग्नचक्रै रथैश्चापि पातितैश्च महाध्वजैः}
{सादिभिश्च हतैः शूरैः सङ्कीर्णा वसुधाऽभवत्}


\twolineshloka
{बाणपातनिकृत्तास्तु योधास्ते कुरुसत्तम}
{चेष्टन्तो विविधाश्चेष्टा व्यदृश्यन्त महाहवे}


\twolineshloka
{वर्तमाने तथा युद्धे घोरे देवासुरोपमे}
{अब्रवीत्क्षत्रियांस्तत्र धर्मराजो युधिष्ठिरः}


\twolineshloka
{अभिद्रवत संयत्ताः कुम्भयोनिं महारथाः}
{एषो हि पार्पतो वीरो भारद्वाजेन सङ्गतः}


\twolineshloka
{घटते च यथाशक्ति भारद्वाजस्य नाशने}
{यादृशानि हि रूपाणि दृश्यन्तेऽस्य महारणे}


\threelineshloka
{अद्य द्रोणं रणे क्रुद्धो घातयिष्यति पार्पतः}
{ते यूयं सहिता भूत्वा युध्यध्वं कुम्भसम्भवम्}
{}


\twolineshloka
{युधिष्ठिरसमाज्ञप्ताः सृञ्जयानां महारथाः}
{अभ्यद्रवन्त संयत्ता भारद्वाजजिघांसवः}


\twolineshloka
{तान्समापततः सर्वान्भारद्वाजो महारथः}
{अभ्यवर्तत वेगेन मर्तव्यमिति निश्चितः}


\twolineshloka
{प्रयाते सत्यसन्धे तु समकम्पत मेदिनी}
{ववुर्वाताः सनिर्घातास्त्रासयाना वरूथिनीम्}


\twolineshloka
{पपात महती चोल्का आदित्यान्निश्चरन्त्युत}
{दीपयन्ती उभे सेने शंसन्तीव महद्भयम्}


\twolineshloka
{जज्वलुश्चैव शस्त्राणि भारद्वाजस्य मारिष}
{रथाः स्वनन्ति चात्यर्थं हयाश्चाश्रूण्यवासृजन्}


\twolineshloka
{हतौजा इव चाप्यासीद्भारद्वाजो महारथः}
{प्रास्फुरन्नयनं चास्य वामबाहुस्तथैव च}


% Check verse!
विमनाश्चाभवद्युद्धे दृष्ट्वा पार्षतमग्रतः
\twolineshloka
{ऋषीणां ब्रह्मवादानां स्वर्गस्य गमनं प्रति}
{सुयुद्धेन ततः प्राणानुत्स्रष्टुमुपचक्रमे ॥]}


\twolineshloka
{ततश्चतुर्दिशं सैन्यैर्द्रुपदस्याभिसंवृतः}
{निर्दहन्क्षत्रियव्रातान्द्रोणः पर्यचरद्रणे}


\twolineshloka
{हत्वा विंशतिसाहस्रान्क्षत्रियानरिमर्दनः}
{दशायुतानि करिणामवधीद्विशिखैः शितैः}


\twolineshloka
{सोऽतिष्ठदाहवे यत्तो विधूमोऽग्निरिव ज्वलन्}
{क्षत्रियाणामभावाय ब्राह्ममस्त्रं समास्थितः}


\twolineshloka
{पाञ्चाल्यं विरथं भीमो हतसर्वायुधं बली}
{सुविपण्णं महात्मानं त्वरमाणः समभ्ययात्}


\twolineshloka
{ततः स्वरथमारोप्य पाञ्चाल्यमरिमर्दनः}
{अब्रवीदभिसम्प्रेक्ष्य द्रोणमस्यन्तमन्तिकात्}


\twolineshloka
{न त्वदन्य इहाचार्यं योद्धुमुत्सहते पुमान्}
{त्वरस्व प्राग्वधायैव त्वयि भारः समाहितः}


\twolineshloka
{स तथोक्तो महाबाहुः सर्वभारसहं धनुः}
{अभिपत्याददे क्षिप्रमायुधप्रवरं दृढम्}


\twolineshloka
{संरब्धश्च शरानस्यन्द्रोणं दुर्वारणं रणे}
{विवारयिपुराचार्यं शरवर्षैरवाकिरत्}


\twolineshloka
{तौ न्यवारयतां श्रेष्ठौ संरब्धौ रणशोभिनौ}
{उदीरयेतां ब्राह्माणि दिव्यान्यस्त्राण्यनेकशः}


\twolineshloka
{स महास्त्रैर्महाराज द्रोणमाच्छादयद्रणे}
{निहत्य सर्वाण्यस्त्राणि भारद्वाजस्य पार्षतः}


\twolineshloka
{स वसातीञ्शिवींस्चैव वाह्लीकान्कौरवानपि}
{रक्षिष्यमाणान्सङ्ग्रामे द्रोणं व्यधमदच्युतः}


\twolineshloka
{धृष्टद्युम्नस्तथा राजन्गभस्तिभिरिवांशुमान्}
{वभौ प्रच्छादयन्नाशाः शरजालैः समन्ततः}


\twolineshloka
{तस्य द्रोणो धनुश्छित्त्वा विद्ध्वा चैनं शिलीमुखैः}
{मर्माण्यभ्यहनद्भूयः स व्यथां परमामगात्}


\twolineshloka
{ततो भीमो दृढक्रोधो द्रोणस्याश्लिष्य तं रथम्}
{शनकैरिव राजेन्द्र द्रोणं वचनमब्रवीत्}


\twolineshloka
{यदि नाम न युध्येरञ्शिक्षिता ब्रह्मबन्धवः}
{स्वकर्मभिरसन्तुष्टा न स्म क्षत्रं क्षयं व्रजेत्}


\twolineshloka
{अहिंसां सर्वभूतेषु धर्मं ज्यायस्तरं विदुः}
{तस्य च ब्राह्मणो मूलं भवांश्च ब्रह्मवित्तमः}


\twolineshloka
{श्वपाकवन्म्लेच्छगणान्हत्वा चान्यान्पृथग्विधान्}
{`भरन्ति हि सुतान्दारांस्तद्वदज्ञानमोहिताः'}


\threelineshloka
{अज्ञानान्मूढवद्ब्रह्मन्पुत्रदारधनेप्सया}
{एकस्यार्थे बहून्हत्वा पुत्रस्याधर्मविद्यया}
{स्वकर्मस्थान्विकर्मस्थो न व्यपत्रपसे कथम्}


\twolineshloka
{`आचारहीन निर्लज्ज ब्रह्मवन्धो वरायुध}
{इदानीं तिष्ठ दुर्बुद्धे न मे जीवन्विमोक्ष्यसे}


\twolineshloka
{यस्यार्थे शस्त्रमादाय यमपेक्ष्य च जीवसि}
{स चाद्य पतितः शेते पृष्टेनावेदितस्तवः}


\twolineshloka
{`स वै च निहतः शेते तव पुत्रः सुमन्दधीः'}
{धर्मराजस्य तद्वाक्यं नाभिशङ्कितुमर्हसि}


\threelineshloka
{एवमुक्तस्ततो द्रोणो भीमेनोत्सृज्य तद्वनुः}
{`सन्न्यासाय शरीरस्य योक्ष्यमाणः स वै द्विजः'}
{सर्वाण्यस्त्राणि धर्मात्मा हातुकामोऽभ्यभापत}


\twolineshloka
{`कर्णं दुर्योधनं राजंस्त्वरमाणः पराक्रमम्'}
{कर्णकर्ण महेष्वास कृप दुर्योधनेति च}


\threelineshloka
{सङ्ग्रामे क्रियतां यत्नो ब्रवीम्येष पुनःपुनः}
{पाण्डवेभ्यः शिवं वोस्तु शस्त्रमभ्युत्सृजाम्यहम्}
{इति तत्र महाराज प्राक्रोशद्द्रौणिमेव च}


\twolineshloka
{उत्सृज्य च रणे शस्त्रं रथोपस्थे निविश्य च}
{अभयं सर्वभूतानां प्रददौ योगमीयिवान्}


\threelineshloka
{तस्य तच्छिद्रनाज्ञाय धृष्टद्युम्नः प्रतापवान्}
{सशरं तद्वनुर्घोरं सन्न्यस्याथ रथे ततः}
{खङ्गी रथादवप्लुत्य सहसा द्रोणमभ्ययात्}


\twolineshloka
{`प्रद्रुते त्वथ द्रोणाय धृष्टद्युम्ने महारथे'}
{हाहाकृतानि भूतानि मानुषाणीतराणि च}


\twolineshloka
{द्रोणं तथागतं दृष्ट्वा धृष्टद्युम्नवशं गतम्}
{हाहाकारं भृशं चक्रुरहो धिगिति चाबुवन्}


\twolineshloka
{द्रोणोऽपि शस्त्राण्युत्सृज्य परमं साङ्ख्यमास्थितः}
{तथोक्त्वा योगमास्थाय ज्योतिर्भूतो महातपाः}


\twolineshloka
{पुराणं पुरुषं विष्णुं जगाम मनसा परम्}
{मुखं किञ्चित्समुन्नाम्य विष्टब्योरस्तथाग्रतः}


\twolineshloka
{निमीलिताक्षः सत्वस्थो निक्षिप्य हृदि धारणाम्}
{ओमित्येकाक्षरं ब्रह्म ज्योतिर्भूतो महातपाः}


\twolineshloka
{स्मरित्वा देवदेवेशमक्षरं परमं प्रभुम्}
{दिवमाक्रामदाचार्यः साक्षात्सद्भिर्दुराक्रमाम्}


\twolineshloka
{`मूर्धानं तस्य निर्भिद्य ज्योती राजन्महात्मनः}
{जगाम परमं स्थानं देहं न्यस्य रथोत्तमे'}


\twolineshloka
{द्वौ सूर्याविति नो बुद्धिरासीत्तस्मिंस्तथागते}
{एकरूपमिवाभासीज्योतिर्भिः पूरितं नभः}


\twolineshloka
{समपद्यत चोल्काभं द्रोणस्य निधने तदा}
{निमेषमात्रेण च तज्ज्योतिरन्तरधीयत}


\twolineshloka
{आसीत्किलकिलाशब्दः प्रहृष्टानां दिवौकसाम्}
{ब्रह्मलोकगते द्रोणे धृष्टद्युम्ने च मोहिते}


\twolineshloka
{वयमेव तदाऽद्राक्ष्म पञ्च मानुषयोनयः}
{योगयुक्तं महात्मानं गच्छन्तं परमां गतिम्}


\twolineshloka
{अहं धनञ्जयः पार्थः कृपः शारद्वतो द्विजः}
{वासुदेवश्च वार्ष्णेयो धर्मपुत्रश्च पाण्डवः}


\threelineshloka
{अन्ये तु सर्वे नापश्यन्भारद्वाजस्य धीमतः}
{महिमानं महाराज योगयुक्तस्य गच्छतः}
{ब्रह्मलोकं महद्दिव्यं देवगुह्यं हि तत्परम्}


\threelineshloka
{गतिं परमिकां प्राप्तमजानन्तो नृयोनयः}
{नापश्यन्गच्छमानं हि तं सार्धमृषिपुङ्गवैः}
{आचार्यं योगमास्थाय ब्रह्मलोकमरिन्दमम्}


\threelineshloka
{वितुन्नाङ्गं शरव्रातैरन्यस्तायुधमसृक्क्षरम्}
{विकृष्य पार्षतः खङ्गं क्रोधामर्षवशं गतः}
{दृश्यमानः सर्वभूतैः केशपक्षे परामृशत्}


\twolineshloka
{तस्य मूर्धानमालम्ब्य गतसत्वस्य देहिनः}
{किञ्चिदब्रुवतः कायाद्विचकर्तासिना शिरः}


\twolineshloka
{हर्षेण महता महता युक्तो भारद्वाजे निपातिते}
{सिंहनादरवं चक्रे भ्रामयन्खङ्गमाहवे}


\twolineshloka
{आकर्णपलितः श्यामो वयसाऽशीतिपञ्चकः}
{त्वत्कृते व्यचरत्सङ्ख्ये स तु षोडशवर्षवत्}


\twolineshloka
{उक्तवाश्च महाबाहुः कुन्तीपुत्रो धनञ्जयः}
{जीवन्तमानयाचार्यं मा वधीर्द्रुपदात्मज}


\twolineshloka
{न हन्तव्यो न हन्तव्य इति ते सैनिकाश्च ह}
{उत्क्रोशन्नर्जुनश्चैव सानुक्रोशस्तमाव्रजत्}


\twolineshloka
{क्रोशमानेऽर्जुने चैव पार्थिवेषु च सर्वशः}
{धृष्टद्युम्नोऽवधीद्द्रोणं रथतल्पे नरर्षभम्}


\twolineshloka
{शोणितेन परिक्लिन्नो रथाद्भूमिमथापतत्}
{लोहिताङ्ग इवादित्यो दुर्धर्षः समपद्यत}


\twolineshloka
{एवं तं निहतं सङ्ख्ये ददृशे सैनिको जनः}
{धृष्टद्युम्नस्तु तद्राजन्भारद्वाजशिरोऽहरत्}


\twolineshloka
{तावकानां महेष्वासः प्रमुखे तत्समाक्षिपत्}
{ते तु दृष्ट्वा शिरो राजन्भारद्वाजस्य तावकाः}


\twolineshloka
{पलायनकृतोत्साहा दुद्रुवः सर्वतोदिशम्}
{द्रोणस्तु दिवमास्थाय नक्षत्रपथमाविशत्}


\twolineshloka
{अहमेव तदाऽद्राक्षं द्रोणस्य निधनं नृप}
{ऋषेः प्रसादात्कृष्मस्य सत्यवत्याः सुतस्य च}


\twolineshloka
{विधूमामिह संयान्तीमुल्कां प्रज्वलितामिव}
{अपश्याम दिवं स्तब्ध्वा गच्छन्तं तं महाद्युतिम्}


\twolineshloka
{हते द्रोणे निरुत्साहाः कुरुपाण्डवसृञ्जयाः}
{अभ्यद्रवन्महावेगास्ततः सैन्यं व्यदीर्यत}


\twolineshloka
{निहता हतभूयिष्ठाः सङ्ग्रामे निशितैः शरैः}
{तावका निहते द्रोणे गतासव इवाभवन्}


\twolineshloka
{पराजयमथावाप्य परत्र च महद्भयम्}
{उभयेनैव ते हीना व्यनिन्दन्मतिमात्मनः}


\twolineshloka
{अन्विच्छन्तः शरीरं तु भारद्वाजस्य पार्थिवाः}
{नान्वगच्छन्महाराज कबन्धायुतसङ्कुले}


\twolineshloka
{`पतिते त्वथ संरब्धे सेनायां तत्र भारत}
{उदिष्ठन्कबन्धानां सहस्राण्येकविंशतिः}


\twolineshloka
{शोणितेन परिक्लिन्ना रणभूमिश्च भारत}
{लोहितार्द्र इवादित्यो दुर्दर्शश्चाभवत्तदा'}


\twolineshloka
{पाण्डवास्तु जयं लब्ध्वा परत्र च महद्यशः}
{बाणशङ्खरवांश्चक्रुः सिंहनादांश्च पुष्कलान्}


\twolineshloka
{भीमसेनस्ततो राजन्धृष्टद्युम्नश्च पार्षतः}
{वरूथिन्यामनृत्येतां परिष्वज्य परस्परम्}


\threelineshloka
{अब्रवीच्च तदा भीमः पार्षतं शत्रुतापनम्}
{भूयोऽहं त्वां परिष्वज्य परिवक्ष्यामि पार्षत}
{सूतपुत्रे हते पापे धार्तराष्ट्रे च संयुगे}


\twolineshloka
{एतावदुक्त्वा भीमस्तु हर्षेण महता युतः}
{बाहुशब्देन पृथिवीं कम्पयामास पाण्डवः}


\twolineshloka
{तस्य शब्देन वित्रस्ताः प्राद्रवंस्तावका युधि}
{क्षत्रधर्मं समुत्सृज्य पलायनपरायणाः}


\twolineshloka
{पाण्डवास्तु जयं लब्धा हृष्टा ह्यासन्विशाम्पते}
{अरिक्षयं च सङ्ग्रामे तेन ते सुखमाप्नुवन्}


\chapter{अध्यायः १९४}
\twolineshloka
{सञ्जय उवाच}
{}


\twolineshloka
{ततो द्रोणे हते राजन्कुरवः शस्त्रपीडिताः}
{हतप्रवीरा विध्वस्ता भृशं शोकपरायणाः}


\twolineshloka
{उदीर्णांश्च परान्दृष्ट्वा हर्षमाणान्पुनः पुनः}
{अश्रुपूर्णेक्षणास्त्रस्ता दीनास्त्वासन्विशाम्पते}


\twolineshloka
{विचेतसो हतोत्साहाः कश्मलाभिहतौजसः}
{आर्तस्वरेण महता पुत्रं ते पर्यवारयन्}


\twolineshloka
{वेषण्णवदना दीना वीक्षमाणा दिशो दश}
{अश्रुकण्ठा यथा दैत्या हिरण्याक्षे पुरा हते}


\twolineshloka
{स तैः परिवृतो राजा त्रस्तैः क्षुद्रमृगैरिव}
{अशक्नुवन्नवस्थातुमपायात्तनयस्तव}


\twolineshloka
{क्षुत्पिपासापरिम्लानास्ते योधास्तव भारत}
{आदित्येनेव सन्तप्ता भृशं विमनसोऽभवन्}


\twolineshloka
{भास्करस्येव पतनं समुद्रस्येव शोषणम्}
{विपर्यासं यथा मेरोर्वासवस्येव निर्जयम्}


\twolineshloka
{अप्रेक्षणीयं तद्दृष्ट्वा भारद्वाजस्य पातनम्}
{त्रस्तरूपतरा राजन्कौरवाः प्राद्रवन्भयात्}


\twolineshloka
{गान्धारराजः शकुनिस्त्रस्तस्त्रस्ततरैः सह}
{हतं रुक्मरथं श्रुत्वा प्राद्रवत्सहितो रथैः}


\twolineshloka
{वरूथिनीं वेगवतीं विद्रुतां सपताकिनीम्}
{परिगृह्य महासेनां सूतपुत्रोऽपयाद्भयात्}


\twolineshloka
{रथनागाश्वकलिलां पुरस्कृत्य तु वाहिनीम्}
{मद्राणामीश्वरः शल्यो वीक्षमाणोऽपयाद्भयात्}


\twolineshloka
{हतप्रवीरैर्भूयिष्ठैर्ध्वजैर्बहुपताकिभिः}
{वृतः शारद्वतोऽगच्छत्कष्टङ्कष्टमिति ब्रुवन्}


\twolineshloka
{भोजनीकेन शिष्टेन कलिङ्गारट्टबाह्लिकैः}
{कृतवर्मा वृतो राजन्प्रायात्सुजवनैर्हयैः}


\twolineshloka
{पदातिगणसंयुक्तस्त्रस्तो राजन्भयार्दितः}
{उलूकः प्राद्रवत्तत्र दृष्ट्वा द्रोणं निपातितम्}


\twolineshloka
{दर्शनीयो युवा चैव शौर्येण कृतलक्षणः}
{दुःशासनो भृशोद्विग्नः प्राद्रवद्गजसंवृतः}


\twolineshloka
{रथानामयुतं गृह्य त्रिसाहस्रं च दन्तिनाम्}
{वृषसेनो ययौ तूर्णं दृष्ट्वा द्रोणं निपातितम्}


\twolineshloka
{गजाश्वरथसंयुक्तो वृतश्चैव पदातिभिः}
{दुर्योधनो महाराज प्रायात्तत्र महारथः}


\twolineshloka
{संशप्तकगणाद्गृह्य हतशेषान्किरीटिना}
{सुशर्मा प्राद्रावद्राजन्दृष्ट्वा द्रोणं निपातितम्}


\twolineshloka
{गजान्रथान्समारुह्य व्युदस्य च हयाञ्जनाः}
{प्राद्रवन्सर्वतः सङ्ख्ये दृष्ट्वा रुक्मरथं हतम्}


\twolineshloka
{त्वरयन्तः पितॄनन्ये भ्रातॄनन्येऽथ मातुलान्}
{पुत्रानन्ये वयस्यांश्च प्राद्रवन्कुरवस्तदा}


\twolineshloka
{चोदयन्तश्च सैन्यानि स्वस्त्रीयांश्च तथाऽपरे}
{सम्बन्धिनस्तथाऽन्ये च प्राद्रवन्त दिशो दश}


\twolineshloka
{प्रकीर्णकेशा विध्वस्ता न द्वावेकत्र धावतः}
{नेदमस्तीति मन्वाना हतोत्साहा हतौजसः}


\twolineshloka
{उत्सृज्य कवचानन्ये प्राद्रवंस्तावका विभो}
{अन्योन्यं ते समाक्रोशन्यैनिका भरतर्षभ}


\threelineshloka
{तिष्ठतिष्ठेति न च ते स्वयं तत्रावतस्थिरे}
{धुर्यानुन्मुच्य च रथाद्धतसूतात्स्वलङ्कृतान्}
{अधिरुह्य हयान्योधाः क्षिप्रं पद्भिरचोदयन्}


\twolineshloka
{द्रमाणे तथा सैन्ये त्रस्तरूपे हतौजसि}
{प्रतिस्रोत इव ग्राहो द्रोणपुत्रः परानियात्}


\twolineshloka
{तस्यासीत्सुमहद्युद्धं शिखण्डिप्रमुखैर्गणैः}
{प्रभद्रकैश्च पाञ्चालैश्चेदिभिश्च सकेकयैः}


\twolineshloka
{हत्वा बहुविधाः सेनाः पाण्डूनां युद्धदुर्मदः}
{कथञ्चित्सङ्कटान्मुक्तो मत्तद्विरदविक्रमः}


\twolineshloka
{द्रवमाणं बलं दृष्ट्वा पलायनकृतक्षणम्}
{दुर्योधनं समासाद्य द्रोणपुत्रोऽब्रवीदिदम्}


\twolineshloka
{किमियं द्रवते सेना त्रस्तरूपेव भारत}
{द्रवमाणां च राजेन्द्र नावस्थापयसे रणे}


\twolineshloka
{त्वं चापि न यथा पूर्वं प्रकृतिस्थो नराधिप}
{कर्णप्रभृतयश्चेमे नावतिष्ठन्ति पार्थिव}


\twolineshloka
{अन्येष्वपि च युद्धेषु नैव सेनाऽद्रवत्तदा}
{कच्चित्क्षेमं महाबाहो तव सैन्यस्य भारत}


\twolineshloka
{कस्पिन्निदं हते राजन्रथसिंहे बलं तव}
{एतामवस्थां संप्राप्तं तन्ममाचक्ष्व कौरव}


\twolineshloka
{तत्तु दुर्योधनः श्रुत्वा द्रोणपुत्रस्य भाषितम्}
{घोरमप्रियमाख्यातुं नाशक्नोत्पार्थिवर्षभः}


\twolineshloka
{भिन्ना नौरिव ते पुत्रो मग्नः शोकमहार्णवे}
{बाष्पेणापिहितो दृष्ट्वा द्रोणपुत्रं रथे स्थितम्}


\twolineshloka
{ततः शारद्वतं राजा सव्रीडमिदमब्रवीत्}
{शंसात्र भद्रं ते सर्वं यथा सैन्यमिदं द्रुतम्}


\threelineshloka
{अथ शारद्वतो राजन्नार्तिमार्च्छन्पुनःपुनः}
{शशंस द्रोणपुत्राय यथा द्रोणो निपातितः ॥कृप उवाच}
{}


\twolineshloka
{वयं द्रोणं पुरस्कृत्य पृथिव्यां प्रवरं रथम्}
{प्रावर्तयाम सङ्ग्रामं पाञ्चालैरेव केवलम्}


\twolineshloka
{ततः प्रवृत्ते सङ्ग्रमे विमिश्राः कुरुसोमकाः}
{अन्योन्यमभिगर्जन्तः शस्त्रैर्देहानपातयन्}


\twolineshloka
{वर्तमाने तथा युद्धे क्षीयमाणेषु संयुगे}
{धार्तराष्ट्रेषु सङ्क्रुद्धः पिता तेऽस्त्रमुदैरयत्}


\twolineshloka
{ततो द्रोणो ब्राह्ममस्त्रं विकुर्वाणो नरर्षभः}
{व्यहनच्छात्रवान्भल्लैः शतशोऽथ सहस्रशः}


\twolineshloka
{पाण्डवाः केकया मात्स्याः पाञ्चालाश्च विशेषतः}
{सङ्ख्ये द्रोणरथं प्राप्य व्यनशन्कालचोदिताः}


\twolineshloka
{सहस्रं नरसिंहानां द्विसाहस्रं च दन्तिनाम्}
{द्रोणो ब्रह्मास्त्रयोगेन प्रेषयामास मृत्यवे}


\twolineshloka
{आकर्णपलितश्यामो वयसाऽशीतिपञ्चकः}
{रणे पर्यचरद्द्रोणो वृद्धः षोडशवर्षवत्}


\twolineshloka
{क्लिश्यमानेषु सैन्येषु वध्यमानेषु राजसु}
{अमर्षवशमापन्नाः पाञ्चाला विमुखाऽभवन्}


\twolineshloka
{तेषु किञ्चित्प्रभग्नेषु विमुखेषु सपत्नजित्}
{दिव्यमस्त्रं विकुर्वाणो बभूवार्क इवोदितः}


\twolineshloka
{स मध्यं प्राप्य पाण्डूनां शररश्मिः प्रतापवान्}
{मध्यं गत इवादित्यो दुष्प्रेक्ष्यस्ते पिताऽभवत्}


\twolineshloka
{ते दह्यमाना द्रोणेन सूर्येणेव विराजता}
{दग्धवीर्या निरुत्साहा बभूवुर्गतचेतसाः}


\twolineshloka
{तान्दृष्ट्वा पीडितान्बाणैर्द्रोणेन मधुसूदनः}
{जयैषी पाण्डुपुत्राणामिदं वचनमब्रवीत्}


\twolineshloka
{नैष जातु नरैः शक्यो जेतुं शस्त्रभृतां वरः}
{अपि वृत्रहणा सङ्ख्ये रथयूथपयूथपः}


\twolineshloka
{ते यूयं धर्ममुत्सृज्य जयं रक्षत पाण्डवाः}
{यथा वः संयुगे सर्वान्न हन्याद्रुक्मवाहनः}


\twolineshloka
{अश्वत्थाम्नि हते नैष युध्येदिति मतिर्मम}
{हतं तं संयुगे कश्चिदाख्यात्वस्मै मृषा नरः}


\twolineshloka
{एतन्नारोचयद्वाक्यं कुन्तीपुत्रो धनञ्जयः}
{अरोचयंस्तु सर्वेऽन्ये कृच्छ्रेण तु युधिष्ठिरः}


\twolineshloka
{भीमसेनस्तु सव्रीडमब्रवीत्पितरं तव}
{अश्वत्थामा हत इति तं नाबुध्यत ते पिता}


\twolineshloka
{स शङ्कमानस्तन्मिथ्या धर्मराजमपृच्छत}
{हतं वाऽप्यहतं वाऽऽजौ त्वां पिता पुत्रवत्सलः}


\twolineshloka
{तमतथ्यभये मग्नो जये सक्तो युधिष्ठिरः}
{अश्वत्थामानमायोधे हतं दृष्ट्वा महागजम्}


\twolineshloka
{भीमेन गिरिवर्ष्माणं मालवस्येन्द्रवर्मणः}
{उपसृत्य तदा द्रोणमुच्चैरिदमुवाच ह}


\twolineshloka
{यस्यार्थे शस्त्रमादत्से यमवेक्ष्य च जीवसि}
{पुत्रस्ते दयितो नित्यं सोश्वत्थामा निपातितः}


% Check verse!
शेते विनिहतो भूमौ वने सिंहशिशुर्यथा
\twolineshloka
{जानन्नप्यनृतस्याथ दोषान्स द्विजसत्तमम्}
{अव्यक्तमब्रवीद्राजा हतः कुञ्जर इत्युत}


\twolineshloka
{स त्वां निहतमाक्रन्दे श्रुत्वा सन्तापतापितः}
{नियम्य दिव्यान्यस्त्राणि नायुध्यत यथा पुरा}


\twolineshloka
{तं दृष्ट्वा परमोद्विग्नं शोकातुरमचेतसम्}
{पाञ्चालराजस्य सुतः क्रुरकर्मा समाद्रवत्}


\twolineshloka
{तं दृष्ट्वा परमोद्विग्नं शोकातुरमचेतसम्}
{दिव्यान्यस्त्राण्यथोत्सृज्य रणे प्रायमुपाविशत्}


\twolineshloka
{ततोऽस्य केशान्सव्येन गृहीत्वा पाणिना तदा}
{पार्षतः क्रोशमानानां वीराणामच्छिनच्छिरः}


\twolineshloka
{न हन्तव्यो न हन्तव्य इति ते सर्वतोऽब्रुवन्}
{तथैव चार्जुनो वाहादवरुह्यैनमाद्रवत्}


\twolineshloka
{उद्यम्य त्वरितो बाहुं ब्रुवाणश्च पुनःपुनः}
{जीवन्तमानयाचार्यं मावधीरिति धर्मवित्}


\twolineshloka
{तथा निवार्यमाणेन कौरवैरर्जुनेन च}
{हत एव नृशंसेन पिता तव नरर्षभ}


\threelineshloka
{सैनिकाश्च ततः सर्वे प्राद्रवन्त भयार्दिताः}
{वयं चापि निरुत्साहा हते पितरि तेऽनघ ॥सञ्जय उवाच}
{}


\twolineshloka
{तच्छ्रुत्वा द्रोणपुत्रस्तु निधनं पितुराहवे}
{क्रोधमाहारयत्तीव्रं दण्डाहत इवोरगः}


\twolineshloka
{ततः क्रुद्धो रणे द्रौणिर्भृशं जज्वाल मारिष}
{यथेन्धनं महत्प्राप्य प्राज्वलद्धव्यवाहनः}


\twolineshloka
{तलं तलेन निष्पिष्य दन्तैर्दन्तानुपास्पृशत्}
{निःश्वसन्नुरगो यद्वल्लोहिताक्षोऽभवत्तदा}


\chapter{अध्यायः १९५}
\twolineshloka
{धृतराष्ट्र उवाच}
{}


\twolineshloka
{अधर्मेण हतं श्रुत्वा धृष्टद्युम्नेन सञ्जय}
{ब्राह्मणं पितरं वृद्धमश्वत्थामा किमब्रवीत्}


\twolineshloka
{मानवं वारुणाग्नेयं ब्राह्ममस्त्रं च वीर्यवान्}
{ऐन्द्रं नारायणं चैव यस्मिन्नित्यं प्रतिष्ठितम्}


\twolineshloka
{तमधर्मेण धर्मिष्ठं धृष्टद्युम्नेन संयुगे}
{श्रुत्वा निहतमाचार्यं सोऽश्वत्थामा किमब्रवीत्}


\twolineshloka
{येन रामादवाप्येह धनुर्वेदं महात्मना}
{प्रोक्तान्यस्त्राणि दिव्यानि पुत्राय गुणकाङ्क्षिणा}


\twolineshloka
{एकमेव हि लोकेऽस्मिन्नात्मतो गुणवत्तरम्}
{इच्छन्ति पुरुषाः पुत्रं लोके नान्यं कथञ्चन}


\twolineshloka
{आचार्याणां भवन्त्येव रहस्यानि महात्मनाम्}
{तानि पुत्राय वा दद्युः शिष्यायानुगताय वा}


\twolineshloka
{स शिष्यः प्राप्य तत्सर्वं सविशेषं च सञ्जय}
{शूरः शारद्वतीपुत्रः सङ्ख्ये द्रोणादनन्तरः}


\twolineshloka
{रामस्य तु समः शस्त्रे पुरन्दरसमो युधि}
{कार्तवीर्यसमो वीर्ये बृहस्पतिसमो मतौ}


\twolineshloka
{महीधरसमः स्थैर्ये तेजसाऽग्निसमो युवा}
{समुद्र इव गाम्भीर्ये क्रोधे चाशीविषोपमः}


\twolineshloka
{स रथी प्रथमो लोके दृढधन्वा जितक्लमः}
{शीघ्रोऽनिल इवाक्रन्दे चरन्क्रुद्ध इवान्तकः}


\threelineshloka
{अस्यता येन सङ्ग्रामे धरण्यभिनिपीडिता}
{`मेघस्तनितनिर्घोषं कम्पते भयविह्वलाः'}
{यो न व्यथति सङ्ग्रमे वीरः सत्यपराक्रमः}


\twolineshloka
{वेदस्नातो व्रतस्नातो धनुर्वेदे च पारगः}
{महोदधिरिवाक्षोभ्यो रामो दाशरथिर्यथा}


\twolineshloka
{तमधर्मेण धर्मिष्ठं धृष्टद्युम्नेन संयुगे}
{श्रुत्वा निहतमाचार्यमश्वत्थामा किमब्रवीत्}


\twolineshloka
{धृष्टद्युम्नस्य यो मृत्युः सृष्टस्तेन महात्मना}
{यथा द्रोणस्य पाञ्चाल्यो यज्ञसेनसुतोऽभवत्}


\twolineshloka
{तं नृशंसेन पापेन क्रूरेणादीर्घदर्शिना}
{श्रुत्वा निहतमाचार्यमश्वत्थामा किमब्रवीत्}


\chapter{अध्यायः १९६}
\twolineshloka
{सञ्जय उवाच}
{}


\twolineshloka
{छद्मना निहतं श्रुत्वा पितरं पापकर्मणा}
{बाष्पेणापूर्यत द्रौणी रोषेण च नरर्षभ}


\twolineshloka
{तस्य क्रुद्धस्य राजेन्द्र वपुर्दीप्तमदृश्यत}
{अन्तकस्येव भूतानि जिहीर्षोः कालपर्यये}


\twolineshloka
{अश्रूपुर्णे ततो नेत्रे व्यपमृज्य पुनः पुनः}
{उवाच कोपान्निःश्वस्य दुर्योधनमिदं वचः}


\threelineshloka
{पिता मम यथा क्षुद्रैर्न्यस्तशस्त्रो निपातितः}
{धर्मध्वजवता पापं कृतं तद्विदितं मम}
{अनार्यं सुनृशंसं च धर्मपुत्रस्य मे श्रुतम्}


\twolineshloka
{युद्धेष्वपि प्रवृत्तानां ध्रुवं जयपराजयौ}
{द्वयमेतद्भवेद्राजन्वधस्तत्र प्रशस्यते}


\twolineshloka
{न्यायवृत्तो वधो यस्तु संङ्ग्रामे युध्यतो भवेत्}
{न स दुःखाय भवति तथा दृष्टो हि स द्विजैः}


\twolineshloka
{गतः स वीरलोकाय पिता मम न संशयः}
{न शोच्यः पुरुषव्याघ्र यस्तदा निधनं गतः}


\twolineshloka
{यत्तु धर्मप्रवृतः सन्केशग्रहणमाप्तवान्}
{पश्यतां सर्वसैन्यानां तन्मे मर्माणि कृन्तति}


\twolineshloka
{मयि जीवति यत्तातः केशग्रहमवाप्तवान्}
{कथमन्ये करिष्यन्ति पुत्रेभ्यः पुत्रिणः स्पृहाम्}


\twolineshloka
{कामात्क्रोधादविज्ञानाद्वर्षाद्बाल्येन वा पुनः}
{विधर्मकाणि कुर्वन्ति तथा परिभवन्ति च}


\twolineshloka
{तदिदं पार्षतेनेह महदाधर्मिकं कृतम्}
{अवज्ञाय च मां नूनं नृशंसेन दुरात्मना}


\twolineshloka
{तस्यानुबन्धं द्रष्टाऽसौ धृष्टद्युम्नः सुदारुणम्}
{अकार्यं परमं कृत्वा मिथ्यावादी च पाण्डवः}


\twolineshloka
{यो ह्यसौ छद्मनाऽऽचार्यं शस्त्रं सन्न्यासयत्तदा}
{तस्याद्य धर्मराजस्य भूमिः पास्यति शोणितम्}


\twolineshloka
{शपे सत्येन कौरव्य इष्टापूर्तेन चैव ह}
{अहत्वा सर्वपाञ्चालाञ्जीवेयं न कथञ्चन}


\twolineshloka
{सर्वोपायैर्यतिष्यामि पाञ्चालानामहं वधे}
{धृष्टद्युम्नं च समरे हन्ताऽहं पापकारिणम्}


\twolineshloka
{कर्मणा येन तेनेह मृदुना दारुणेन च}
{पाञ्चालानां वधं कृत्वा शान्तिं लब्धास्मि कौरव}


\twolineshloka
{यदर्थं पुरुषव्याघ्र पुत्रानिच्छन्ति मानवाः}
{प्रेत्य चेह च सम्प्राप्तात्त्रायन्ते महतो भयात्}


\twolineshloka
{पित्रा तु मम साऽवस्था प्राप्ता निर्बन्धुना यथा}
{मयि शैलप्रतीकाशे पुत्रे शिष्ये च जीवति}


\twolineshloka
{धिङ्ममास्त्राणि दिव्यानि धिग्बाहू धिक्पराक्रमम्}
{यं स्म द्रोणः सुतं प्राप्य केशग्रहमवाप्तवान्}


\twolineshloka
{स तथाहं करिष्यासि यथा भरतसत्तम}
{परलोकगतस्यापि भविष्याम्यनृणः पितुः}


\twolineshloka
{आर्येण हि न वक्तव्या कदाचित्स्तुतिरात्मनः}
{पितुर्वधममृष्यंस्तु वक्ष्याम्यद्येह पौरुषम्}


\twolineshloka
{अद्य पश्यन्तु मे वीर्यं पाण्डवाः सजनार्दनाः}
{मृद्गतः सर्वसैन्यानि युगान्तमिव कुर्वतः}


\twolineshloka
{न हि देवा न गन्धर्वा नासुरा न च राक्षसाः}
{अद्य शक्ता रणे जेतुं रथस्थं मां नरर्षभाः}


\threelineshloka
{मदन्यो नास्ति लोकेऽस्मिन्नर्जुनाद्वाऽस्त्रवित्क्वचित्}
{अहं हि ज्वलतां मध्ये मयूखानामिवांशुमान्}
{प्रयोक्ता देवसृष्टानामस्त्राणां पृतनागतः}


\twolineshloka
{भृशमिष्वसनादद्य मत्प्रयुक्ता महाहवे}
{दर्शयन्तः शरा वीर्यं प्रमथिष्यन्ति पाण्डवान्}


\twolineshloka
{अद्य सर्वा दिशो राजन्धाराभिरिव सङ्कुलाः}
{आवृताः पत्रिभिस्तीक्ष्णैर्द्रष्टारो मामकैरिह}


\twolineshloka
{विकिरञ्छरजालानि सर्वतो भैरवस्वनान्}
{शत्रून्निपातयिष्यामि महावात इव द्रुमान्}


\twolineshloka
{न हि जानाति बीभत्सुस्तदस्त्रं न जनार्दनः}
{न भीमसेनो न यमौ न च राजा युधिष्ठिरः}


\twolineshloka
{न पार्षतो दुरात्माऽसौ न शिखण्डी न सात्यकिः}
{यदिदं मयि कौरव्य सकल्यं सनिवर्तनम्}


\twolineshloka
{नारायणाय मे पित्रा प्रणम्य विधिपूर्वकम्}
{उपहारः पुरा दत्तो ब्रह्मरूप उपस्थितः}


\twolineshloka
{तं स्वयं प्रतिगृह्याथ भगवान्स वरं ददौ}
{वव्रे पिता मे परममस्त्रं नारायणं ततः}


\twolineshloka
{अथैनमब्रवीद्राजन्भागवान्देवसत्तमः}
{भविता त्वत्समो नान्यः कश्चिद्युधि नरः क्वचित्}


\twolineshloka
{न त्विदं सहसा ब्रह्मन्प्रयोक्तव्यं कथञ्चन}
{न ह्येतदस्त्रमन्यत्र वधाच्छत्रोर्निवर्तते}


\twolineshloka
{न चैतच्छक्यते जेतुं को न वध्येत वै प्रभो}
{अवध्यमपि हन्याद्वि तस्मान्नैतत्प्रयोजयेत्}


\twolineshloka
{अथ सङ्ख्ये रथस्यैव शस्त्राणां च विसर्जनम्}
{प्रयाचनं च शत्रूणां गमनं शरणस्य च}


\twolineshloka
{एते प्रशमने योगा महास्त्रस्य परन्तप}
{सर्वथा पीडितो हिंस्यादवध्यान्पीडयन्रणे}


\threelineshloka
{तज्जग्राह पिता मह्यमब्रवीच्चैव स प्रभुः}
{त्वं वधिष्यसि सर्वाणि शस्त्रवर्षाण्यनेकशः}
{अनेनास्त्रेण सङ्ग्रामे तेजसा च ज्वलिष्यसि}


\twolineshloka
{एवमुक्त्वा स भगवान्दिवमाचक्रमे प्रभुः}
{एतन्नारायणादस्त्रं तत्प्राप्तं पितृबन्धुना}


\twolineshloka
{तेनाहं पाण्डवांश्चैव पाञ्चालान्मात्स्यकेकयान्}
{विद्रावयिष्यामि रणे शचीपतिरिवासुरान्}


\twolineshloka
{यथायथाऽहमिच्छेयं तथा भूत्वा शरा मम}
{निपतेयुः सपत्नेषु विक्रमत्स्वपि भारत}


\twolineshloka
{यथेष्टमश्मवर्षेण प्रवर्षिष्ये रणे स्थितः}
{अयोमुखैश्च विहगैर्द्रावयिष्ये महारथान्}


% Check verse!
परश्वथांश्च निशितानुत्स्रक्ष्येऽहमसंशयम्
\twolineshloka
{सोऽहं नारायणास्त्रेण महता शत्रुतापनः}
{शत्रून्विध्वंसयिष्यामि कदर्थीकृत्य पाण्डवान्}


\twolineshloka
{मित्रब्रह्मगुरुद्रोही जाल्मकः सुविगर्हितः}
{पाञ्चालापशदश्चाद्य न मे जीवन्विमोक्ष्यते}


\threelineshloka
{तच्छ्रुत्वा द्रोणपुत्रस्य पर्यवर्तत वाहिनी}
{ततः सर्वे महाशङ्खान्दध्मुः पुरुषसत्तमाः}
{भेरीश्चाभ्यहनन्हृष्टा डिण्डिभांश्च सहस्रशः}


\twolineshloka
{तथा ननाद वसुधा खुरनेमिप्रपीडिता}
{स शब्दस्तुमुलः खं द्यां पृथिवीं च व्यनादयत्}


\twolineshloka
{तं शब्दं पाण्डवाः श्रुत्वा पर्जन्यनिनदोपमम्}
{समेत्य रथिनां श्रेष्ठाः सहिताश्चाप्यमन्त्रयन्}


\twolineshloka
{तथोक्त्वा द्रोणपुत्रस्तु वार्युपस्पृश्य भारत}
{प्रादुश्चकार तद्दिव्यमस्त्रं नारायणं तदा}


\chapter{अध्यायः १९७}
\twolineshloka
{सञ्जय उवाच}
{}


\twolineshloka
{प्रादुर्भूते ततस्तस्मिन्नस्त्रे नारायणे प्रभो}
{प्रावात्सपृषतो वायुरनभ्रे स्तनयित्नुमान्}


\twolineshloka
{चचाल पृथिवी चापि चुक्षुभे च महोदधिः}
{प्रतिस्रोतः प्रवृत्ताश्च गन्तुं तत्र समुद्रगाः}


\twolineshloka
{शिखराणि व्यशीर्यन्त गिरीणां तत्र भारत}
{अपसव्यं मृगाश्चैव पाण्डुसेनां प्रचक्रिरे}


\twolineshloka
{तमसा तावकीर्यन्त सूर्यश्च कलुषोऽभवत्}
{सम्पतन्ति च भूतानि क्रव्यादानि प्रहृष्टवत्}


\twolineshloka
{देवदानवगन्धर्वास्त्रस्तास्त्वासन्विशाम्पते}
{कथङ्खथाऽभवत्तीव्रा दृष्ट्वा तद्व्याकुलं महत्}


\threelineshloka
{व्यथिताः सर्वराजानस्त्रस्ताश्चासन्विशाम्पते}
{तद्दृष्ट्वा घोररूपं वै द्रौणेरस्त्रं भयावहम् ॥धृतराष्ट्र उवाच}
{}


\twolineshloka
{निवर्तितेषु सैन्येषु द्रोणपुत्रेण संयुगे}
{भृशं शोकाभितप्तेन पितुर्वधममृष्यता}


\threelineshloka
{कुरूनापततो दृष्ट्वा धृष्टद्मुम्नस्य रक्षणे}
{को मन्त्रः पाण्डवेष्वासीत्तन्ममाचक्ष्व सञ्जय ॥सञ्जय उवाच}
{}


\threelineshloka
{प्रागेव विद्रुतान्दृष्ट्वा धार्तराष्ट्रान्युधिष्ठिरः}
{पुनश्च तुमुलं शब्दं श्रुत्वाऽर्जुनमथाब्रवीत् ॥युधिष्ठिर उवाच}
{}


\twolineshloka
{आचार्ये निहते द्रोणे धृष्टद्युम्नेन संयुगे}
{निहते वज्रहस्तेन यथा वृत्रे महासुरे}


\twolineshloka
{नाशंसन्तो जयं युद्धे दीनात्मानो धनञ्जय}
{आत्मत्राणे मतिं कृत्वा प्राद्रवन्कुरवो रणात्}


\twolineshloka
{केचिद्वान्तै रथैस्तूर्णं निहतैः पाष्णियन्तृभिः}
{विपताकध्वजच्छत्रैः पार्थिवाः शीर्णकूबरैः}


\twolineshloka
{भग्ननीडैराकुलाश्वैः प्रारुह्यान्यान्विचेतसः}
{भीताः पादैर्हयान्केचित्त्वरयन्तः स्वयं रथान्}


\twolineshloka
{भग्नाक्षयुगचक्रैश्च व्याकृष्यन्त समन्ततः}
{रथान्विशीर्णानुत्सृज्य पद्भिः केचिच्च विद्रुताः}


\twolineshloka
{हयपृष्ठगताश्चान्ये कृष्यन्तेऽर्धच्युतासनाः}
{गजस्कन्धेषु संस्यूता नाराचैश्चलितासनाः}


\twolineshloka
{शरार्तैर्विद्रुतैर्नागैर्हृताः केचिद्दिशो दश}
{विशस्त्रकवचनाश्चान्ये वाहनेभ्यः क्षितिं गताः}


\twolineshloka
{सञ्छिन्ना नेमिभिश्चैव मृदिताश्च हयद्विपैः}
{क्रोशन्तस्तात पुत्रेति पलायन्ते परे भयात्}


\twolineshloka
{नाभिजानन्ति चान्योन्यं कश्मलाभिहतौजसः}
{पुत्रान्पितॄन्सखीन्भ्रातॄन्समारोप्य दृढक्षतान्}


\twolineshloka
{जलेन क्लेदयन्त्यन्ये विमुच्य कवचनान्यपि}
{`पलायनपराश्चान्ये योधाः शतसहस्रशः'}


\twolineshloka
{अवस्थां तादृशीं प्राप्य हते द्रोणे द्रुतं बलम्}
{पुनरावर्तितं केन यदि जानासि शंस मे}


\twolineshloka
{हयानां हेषतां शब्दः कुञ्जराणां च बृंहताम्}
{रथनेमिस्वनैश्चात्र विमिश्रः श्रूयते महान्}


\twolineshloka
{एते शब्दा भृशं तीव्राः प्रवृत्ताः कुरुसागरे}
{मुमुर्मुहुरुदीर्यन्ते कम्पयन्त्यपि मामकान्}


\twolineshloka
{य एष तुमुलः शब्दः श्रूयते रोमहर्षणः}
{सेन्द्रानप्येष लोकांस्त्रीन्ग्रसेदिति मतिर्मम}


\twolineshloka
{मन्ये वज्रधरस्यैष निनादो भैरवस्वनः}
{द्रोणे हते कौरवार्यं व्यक्तमभ्येति वासवः}


\twolineshloka
{प्रहृष्टरोमकूपाः स्मः संविग्नरथकुञ्जराः}
{धनञ्जय गुरुं श्रुत्वा तत्र नादं सुभीषणम्}


\threelineshloka
{क एष कौरवान्दीर्णानवस्थाप्य महारथः}
{निवर्तयति युद्धार्थं मृधे देवेश्वरो यथा ॥अर्जुन उवाच}
{}


\twolineshloka
{उद्यम्यात्मानमुग्राय कर्मणे वीर्यमास्थिताः}
{धमन्ति कौरवाः शङ्खान्यस्य वीर्यं समाश्रिताः}


\twolineshloka
{यत्र ते संशयो राजन्न्यस्तशस्त्रे गुरौ हते}
{धार्तराष्ट्रानवस्थाप्य क एष नदतीति हि}


\twolineshloka
{हीमन्तं तं महाबाहुं मत्तद्विरदगामिनम्}
{`इन्द्रविष्मुसमं वीर्ये कोपेऽन्तकमिव स्थितम्}


\twolineshloka
{बृहस्पतिसमं बुद्ध्या नीतिमन्तं महारथम्'}
{आख्यास्याम्युग्रकर्माणं कुरूणामभयङ्करम्}


\twolineshloka
{यस्मिञ्जाते ददौ द्रोणो गवां दशशतं धनम्}
{ब्राह्मणेभ्यो महार्हेभ्यः सोऽश्वत्थामैष गर्जति}


\twolineshloka
{जातमात्रेण वीरेण येनोच्चैः श्रवसा यथा}
{हेषता कम्पिता भूमिर्लोकाश्च सकलास्त्रयः}


\twolineshloka
{तच्छ्रुत्वान्तर्हितं भूतं नाम तस्याकरोत्तदा}
{अश्वत्थामेति सोऽद्यैष शूरो नदति पाण्डव}


\twolineshloka
{यो ह्यनाथ इवाक्रम्य पार्षतेन हतस्तथा}
{कर्मणा सुनृशंसेन तस्य नाथो व्यवस्थितः}


\twolineshloka
{गुरुं मे यत्र पाञ्चाल्यः केशपक्षे परामृशत्}
{तन्न जातुं क्षमेद््द्रौणिर्जानन्पौरुषमात्मनः ॥`स हि तेनैव नः सर्वान्क्षपयेदिति मे मतिः'}


\twolineshloka
{उपचीर्णो गुरुर्मिथ्या भवता राज्यकारणात्}
{धर्मज्ञेन सता नाम सोऽधर्मः सुमहान्कृतः}


\twolineshloka
{चिरं स्थास्यति चाकीर्तिस्त्रैलोक्ये सचराचरे}
{रामे वालिवधाद्यद्वदेवं द्रोणे निपातिते}


\twolineshloka
{सर्वधर्मोपपन्नोऽयं स मे शिष्यश्च पाण्डवः}
{नायं वक्ष्यति मिथ्येति प्रत्ययं कृतवांस्त्वयि}


\twolineshloka
{स सत्यकञ्चुकं नाम प्रविष्टेन ततोऽनृतम्}
{आचार्य उक्तो भवता हतः कुञ्जर इत्युत}


\twolineshloka
{ततः शस्त्रं समुत्सृज्य निर्ममो गतचेतनः}
{आसीत्सुविह्वलो राजन्यथा दृष्टस्त्वया विभुः}


\twolineshloka
{स तु शोकसमाविष्टो विमुखः पुत्रवत्सलः}
{शाश्वतं धर्ममुत्सृज्य गुरुः शिष्येण घातितः}


\twolineshloka
{न्यस्तशस्त्रमधर्मेण घातयित्वा गुरुं भवान्}
{रक्षत्विदानीं सामात्यो यदि शक्तोसि पार्षतम्}


\twolineshloka
{ग्रस्तमाचार्य पुत्रेण क्रुद्धेन हतबन्धुना}
{सर्वे वयं परित्रातुं न शक्ष्यामोऽद्य पार्षतम्}


\twolineshloka
{सौहार्दं सर्वभूतेषु यः करोत्यतिमानुषः}
{सोऽद्य केशग्रहं श्रुत्वा पितुर्धक्ष्यति नो रणे}


\twolineshloka
{विक्रोशमाने हि मयि भृशमाचार्यगृद्विनि}
{अपाकीर्य स्वयं धर्मं शिप्येण निहतो गुरुः}


\twolineshloka
{यदा गतं वयो भूयः शिष्टमल्पतरं च नः}
{तस्येदानीं विरोधोऽयमधर्मोऽयं कृतो महान्}


\twolineshloka
{पितेव नित्यं सौहार्दात्पितेव हि च धर्मतः}
{सोऽल्पकालस्य राज्यस्य कारमाद्वातितो गुरुः}


\twolineshloka
{धृतराष्ट्रेण भीष्माय द्रोणाय च विशाम्पते}
{विसृष्टा पृथिवी सर्वा सह पुत्रैश्च तत्परैः}


\twolineshloka
{सम्प्राप्य तादृशीं वृत्तिं सत्कृतः सततं परैः}
{अब्रवीत्सततं पुत्रान्मामेवाभ्यधिकं गुरुः}


\twolineshloka
{अवेक्षमाणस्त्वां मां च न्यस्तास्त्रश्चाहवे हतः}
{न त्वेनं युध्यमानं वै हन्यादपि शतक्रतुः}


\twolineshloka
{तस्याचार्यस्य वृद्धस्य द्रोहो नित्योपकारिणः}
{कृतो ह्यनार्यैरस्माभीर राज्यार्थे लुब्धबुद्धिभिः}


\twolineshloka
{अहो बत महत्पापं कृतं कर्म सुदारुणम्}
{यद्राज्यसुखलोभेन द्रोणोऽयं साधुघातितः}


\twolineshloka
{पुत्रान्भ्रातॄन्पितॄन्दाराञ्जीवितं चैव वासविः}
{त्यजेत्सर्वं मम प्रेम्णा जानात्येवं हि मे गुरुः}


\twolineshloka
{स मया राज्यकामेन हन्यमानो ह्युपेक्षितः}
{तस्मादर्वाक्शिरा राजन्प्राप्तोऽस्मि नरकं प्रभो}


\twolineshloka
{ब्राह्मणं वृद्धमाचार्यं न्यस्तशस्त्रं महामुनिम्}
{घातयित्वाऽद्य राज्यार्थे मृतं श्रेयो न जीवितम्}


\chapter{अध्यायः १९८}
\twolineshloka
{सञ्जय उवाच}
{}


\twolineshloka
{अर्जुनस्य वचः श्रुत्वा नोचुस्तत्र महारथाः}
{अप्रियं वा प्रियं वाऽपि महाराज धनञ्जयम्}


\twolineshloka
{ततः क्रुद्धो महाबाहुर्भीमसेनोऽभ्यभाषत}
{कुत्सयन्निव कौन्तेयमर्जुनं भरतर्षभ}


\twolineshloka
{मुनिर्यथाऽरण्यगतो भाषते धर्मसंहितम्}
{न्यस्तदण्डो यथा पार्थं ब्राह्मणः संशितव्रतः}


\twolineshloka
{क्षतत्राता क्षताज्जीवन्क्षन्ता स्त्रीष्वपि साधुषु}
{क्षत्रियः क्षितिमाप्नोति क्षिप्रं धर्मं यशः श्रियः}


\twolineshloka
{स भवान्क्षत्रियगुणैर्युक्तः सर्वैः कुलोद्वहः}
{अविपश्चिद्यथा वाचं व्याहरन्नाद्य शोभसे}


\twolineshloka
{पराक्रमस्ते कौन्तेय शक्रस्येव शचीपतेः}
{न चातिवर्तसे धर्मं वेलामिव महोदधिः}


\twolineshloka
{न पूजयेत्त्वां कोन्वद्य यत्त्रयोदशवार्षिकम्}
{अमर्षं पृष्ठतः कृत्वा धर्ममेवाभिकाङ्क्षसे}


\twolineshloka
{दिष्ट्या तात मनस्तेऽद्य स्वधर्ममनुवर्तते}
{आनृशंस्ये च ते दिष्ट्या बुद्धिः सततमच्युत}


\twolineshloka
{यत्तु धर्मप्रवृत्तस्य हृतं राज्यमधर्मतः}
{द्रौपदी च परामृष्टा सभामानीय शत्रुभिः}


\twolineshloka
{वनं प्रव्राजिताश्चास्म वल्कलाजिनवाससः}
{अनर्हमाणास्तं भावं त्रयोदश समाः परैः}


\twolineshloka
{`बहूनि क्षाम्य शत्रूणां सत्यधर्मरता वयम्}
{तथा तन्मर्षयित्वा तु यथा ते उत्पथे स्थिताः'}


\twolineshloka
{एतान्यमर्षस्थानानि मर्षितानि त्वयाऽघन}
{क्षत्रधर्मप्रसक्तेन सर्वमेतदनुष्ठितम्}


\twolineshloka
{तमधर्ममपाकृष्टं स्मृत्वाऽद्य सहितस्त्वया}
{सानुबन्धान्हनिष्यामि क्षुद्रान्राज्यहरानहम्}


\twolineshloka
{त्वया हि कथितं पूर्वं युद्धायाभ्यागता वयम्}
{घटामहे यथाशक्ति त्वं तु नोऽद्य जुगुप्ससे}


\twolineshloka
{स्वधर्मं नेच्छसे ज्ञातुं वृथा वचनमेव ते}
{भयार्दितानामस्माकं वाचा मर्माणि कृन्तसि}


\twolineshloka
{वपन्व्रणे क्षारमिव क्षतानां शत्रुकर्शन}
{विदीर्यते मे हृदयं त्वया वाक्शल्यपीडितम्}


\twolineshloka
{अधर्ममेन विपुलं धार्मिकः सन्न बुध्यसे}
{यत्त्वमात्मानमस्मांश्च प्रशंस्यान्न प्रशंससि}


\twolineshloka
{वासुदेवे स्थिते चापि द्रोणपुत्रं प्रशंससि}
{यः कलां षोडशीं पूर्णां धनञ्जय न तेऽर्हति}


% Check verse!
स्वयमेवात्मनो दोषान्ब्रुवाणः किं न लज्जसे
\twolineshloka
{`मम नागायुतं पार्थ बलं बाह्वोर्विधीयते'}
{दारयेयं महीं क्रोधाद्विकिरेयं च पर्वतान्}


\twolineshloka
{आविध्यैतां गदां गुर्वी भीमां काञ्चनमालिनीम्}
{गिरिप्रकाशान्क्षितिजान्भञ्जेयमनिलो यथा}


\twolineshloka
{द्रावयेयं शरैश्चापि सेन्द्रान्देवान्समागतान्}
{सराक्षसगणान्पार्थ सासुरोरगमानवान्}


\twolineshloka
{स त्वमेवंविधं जानन्भ्रातरं मां नरर्षभ}
{द्रोणपुत्राद्भयं कर्तुं नार्हस्यमितविक्रम}


\twolineshloka
{अथवा तिष्ठ बीभत्सो सह सर्वैः सहोदरैः}
{अहमेनं गदापाणिर्जेष्याम्येको महाहवे}


\threelineshloka
{ततः पाञ्चालराजस्य पुत्रः पार्थमथाब्रवीत्}
{सङ्क्रुद्धमिव नर्दन्तं हिरण्यकशिपुर्हरिम् ॥धृष्टद्युम्न उवाच}
{}


\twolineshloka
{बीभत्सो विप्रकर्माणि विहितानि मनीषिभिः}
{याजनाध्यापने दानं यथा यज्ञप्रतिग्रहौ}


\twolineshloka
{षष्ठमध्ययनं नाम तेषां कस्मिन्प्रतिष्ठितः}
{हतो द्रोणो मया ह्येवं किं मां पार्थ विगर्हसे}


\twolineshloka
{अपक्रान्तः स्वधर्माच्च क्षात्रधर्मं व्यपाश्रितः}
{अधर्मेण हतस्तस्मादस्त्रेण क्षुद्रकर्मकृत्}


\twolineshloka
{तथा मायां प्रयुञ्जानमसह्यं ब्राह्मणब्रुवम्}
{प्रतिमायी निहन्याद्यो न युक्तं पार्थ तत्र किम्}


\twolineshloka
{तस्मिंस्तथा मया शस्ते यदि द्रौणायनी रुषा}
{कुरुते भैरवं नादं तत्र किं मम हीयते}


\twolineshloka
{न चाद्भुतमिदं मन्ये यद्द्रौणिश्चात्र गर्जति}
{घातयिष्यति कौरव्यान्परित्रातुमशक्नुवन्}


\twolineshloka
{यच्च मां धार्मिको भूत्वा ब्रवीषि गुरुघातिनम्}
{तदर्थमहमुत्पन्नः पाञ्चाल्यस्य सुतोऽनलात्}


\twolineshloka
{यस्य कार्यमकार्यं वा युध्यतः स्यात्समं रणे}
{तं कथं ब्राह्मणं ब्रूयाः क्षत्रियं वा धनञ्जय}


\twolineshloka
{यो ह्यनस्त्रविदो हन्याद्ब्रह्मास्त्रैः क्रोधमूर्च्छितः}
{सर्वोपायैर्न स कथं वध्यः पुरुषसत्तम}


\threelineshloka
{विशेषात्पितृहन्ता मे न स वध्यः कथं मया}
{योऽयं पापः सुदुर्मेधा बान्धवान्युधि जघ्निवान्}
{तस्य विप्रब्रुववधे कथं पापं भवेन्मम'}


\twolineshloka
{विधर्मिणं धर्मविद्भिः प्रोक्तं तेषां विषोपमम्}
{जानन्धर्मार्थतत्त्वज्ञ किं मामर्जुन गर्हसे}


\twolineshloka
{नृशंसः स मयाऽऽक्रम्य रथ एव निपातितः}
{तन्मामनिन्द्यं बीभत्सो किमर्थं नाभिनन्दसे}


\twolineshloka
{कालानलसमं पार्थ ज्वलार्कविषोपमम्}
{भीमं द्रोणशिरश्छिन्नं न प्रशंससि मे कथम्}


\twolineshloka
{योऽसौ ममैव नान्यस्य बान्धवान्युधि जघ्निवान्}
{छित्त्वाऽपि तस्य मूर्धानं नैवास्मि विगतज्वरेः}


\twolineshloka
{तच्च मे कृन्तते मर्म यन्न तस्य शिरो मया}
{निषादविषये क्षिप्तं जयद्रथशिरो यथा}


\twolineshloka
{अथावधश्च शत्रूणामधर्मः श्रूयतेऽर्जुन}
{क्षत्रियस्य हि धर्मोऽयं हन्याद्धन्येत वा पुनः}


\twolineshloka
{स शत्रुर्निहतः सङ्ख्ये मया धर्मेण पाण्डव}
{यथा त्वया हतः शूरो भगदत्तः पितुः सखा}


\twolineshloka
{पितामहं रणे हत्वा मन्यसे धर्ममात्मनः}
{मया शत्रौ हते कस्मात्पापे धर्मं न मन्यसे}


\twolineshloka
{सम्बन्धावनतं पार्थ न मां त्वं वक्तुमर्हसि}
{स्वगात्रकृतसोपानं निषण्णमिव दन्तिनम्}


\twolineshloka
{क्षमामि ते सर्वमेव वाग्व्यतिक्रममर्जुन}
{द्रौपद्या द्रौपदेयानां कृते नान्येन हेतुना}


\twolineshloka
{कुलक्रमागतं वैरं ममाचार्येण विश्रुतम्}
{तथा जानात्ययं लोको न यूयं पाण्डुनन्दनाः}


\twolineshloka
{नानृती पाण्डवो ज्येष्ठो नाहं वाऽधार्मिकोऽर्जुन}
{`न क्षत्रिय इति प्राहुर्यो न हन्ति रणाजिरे}


\twolineshloka
{पितरं वा गुरुं वाऽपि जिघांसु पुत्रशिष्ययोः}
{जिह्मेन वाऽप्यजिह्मेन हन्यादेवाविचारयन्}


\twolineshloka
{इत्युक्तं ब्रह्मणा पूर्वं क्षत्रियाणां द्विषद्वधे}
{तस्माच्छिष्येण निहतः शत्रुर्मे ब्राह्मणब्रुवः}


\threelineshloka
{यः क्षत्रियसुतो हन्यात्पितरं वा गुरुं च वा}
{अनिष्टं क्षत्रियो हन्यात्स वै क्षत्रिय उच्यते'}
{शिष्यद्रोही हतः पापो युध्यस्व विजयस्तव}


\chapter{अध्यायः १९९}
\twolineshloka
{धृतराष्ट्र उवाच}
{}


\twolineshloka
{साङ्गा वेदा यथान्यायं येनाधीता महात्मना}
{यस्मिन्साक्षाद्धनुर्वेदो हीनिषेवे प्रतिष्ठितः}


\twolineshloka
{यस्य प्रसादात्कुर्वन्ति कर्माणि पुरुषर्षभाः}
{अमानुषाणि सङ्ग्रामे देवैरसुकराणि च}


\twolineshloka
{तस्मिन्नाक्रुश्यति द्रोणे समक्षं पापकर्मणा}
{नीचं कर्म कृतं तेन क्षुद्रेण गुरुघातिना}


% Check verse!
नामर्षं तत्र कुर्वन्ति धिक्क्षात्रं धिगमर्षिताम्
\threelineshloka
{पार्थाः सर्वे च राजानः पृथिव्यां ये धनुर्धराः}
{श्रुत्वा किमाहुः पाञ्चाल्यं तन्ममाचक्ष्व सञ्जय ॥सञ्जय उवाच}
{}


\twolineshloka
{श्रुत्वा द्रुपदपुत्रस्य ता वाचः क्रूरकर्मणः}
{तूष्णीं बभूवू राजानः सर्व एव विशाम्पते}


\twolineshloka
{अर्जुनस्तु कटाक्षेण जिह्मं विप्रेक्ष्य पार्षतम्}
{सबाष्पमतिनिःश्वस्य धिग्धिगित्येव चाब्रवीत्}


\twolineshloka
{युधिष्ठिरश्च भीमश्च यमौ कृष्णस्तथाऽपरे}
{आसन्सुव्रीडिता राजन्सात्यकिस्त्वब्रवीदिदम्}


\twolineshloka
{नेहास्ति पुरुषः कश्चिद्य इमं पापपूरुषम्}
{भाषमाणमकल्याणं शीघ्रं हन्यान्नराधमम्}


\twolineshloka
{एते त्वां पाण्डवाः सर्वे कुत्सयन्ति विवित्सया}
{कर्मणा तेन पापेन श्वपाकं ब्राह्मणा इव}


\twolineshloka
{एतत्कृत्वा महत्पापं निन्दितः सर्वसाधुभिः}
{न लज्जसे कथं वक्तुं समितिं प्राप्य शोभनाम्}


\twolineshloka
{कथं च शतधा जिह्वा न ते मूर्धा च दीर्यते}
{गुरुमाक्रोशतः क्षुद्र न चाधर्मेण पात्यसे}


\twolineshloka
{वाच्यस्त्वमसि पार्थैश्च सर्वैश्चान्धकवृष्णिभिः}
{यत्कर्म कलुषं कृत्वा श्लाघसे जनसंसदि}


\twolineshloka
{अकार्यं तादृशं कृत्वा पुनरेव गुरुं क्षिपन्}
{वध्यस्त्वं न त्वयार्थोऽस्ति मुहूर्तमपि जीवता}


\twolineshloka
{कस्त्वेतद्व्यवसेत्पायं त्वदन्यः पुरुषाधम}
{निगृह्य केशेषु वधं गुरोर्धर्मात्मनः सतः}


\twolineshloka
{सप्तावरे तथा पूर्वे बान्धवास्ते निमञ्जिताः}
{यशसा च परित्यक्तास्त्वां प्राप्य कुलपांसनम्}


\twolineshloka
{उक्तवांश्चापि यत्पार्थे भीष्मं प्रति नरर्षभ}
{तथान्तो विहितस्तेन स्वयमेव महात्मना}


\twolineshloka
{तस्यापि तव सोदर्यो निहन्ता पापकृत्तमः}
{नान्यः पाञ्चालपुत्रेभ्यो विद्यते भुवि पापकृत्}


\twolineshloka
{स चापि सृष्टः पित्रा ते भीष्मस्यान्तकरः किल}
{शिखण्डी रक्षितस्तेन स च मृत्युर्महात्मनः}


\twolineshloka
{पाञ्चालाश्चलिता धर्मात्क्षुद्रा मित्रगुरुद्रुहः}
{त्वां प्राप्य सहसोदर्यं धिक्कृतं सर्वसाधुभिः}


\twolineshloka
{पुनश्चेदीदृशीं वाचं मत्समीपे वदिष्यसि}
{शिरस्ते पोथयिष्यामि गदया वज्रकल्पया}


\twolineshloka
{त्वां च ब्रह्महणं दृष्ट्वा जनः सूर्यमवेक्षते}
{ब्रह्महत्या हि ते पापं प्रायश्चित्तार्थमात्मनः}


\twolineshloka
{पाञ्चालक सुदुर्वृत्त ममैव गुरुमग्रतः}
{गुरोर्गुरुं च भूयोऽपि क्षिपन्नैव हि लज्जसे}


\twolineshloka
{तिष्ठतिष्ठ सहस्वैकं गदापातमिमं मम}
{तव चापि सहिष्येऽहं गदापाताननेकशः}


\threelineshloka
{सात्वतेनैवमाक्षिप्तः पार्षतः परुषाक्षरम्}
{संरब्धं सात्यकिं प्राह सङ्क्रुद्धः प्रहसन्निव ॥धृष्टद्युम्न उवाच}
{}


\twolineshloka
{श्रूयतेश्रूयते चेति क्षम्यते चेति माधव}
{सदाऽनार्योऽशुभः साधुं पुरुषं क्षेप्तुमिच्छति}


\twolineshloka
{क्षमा प्रशस्यते लोके न तु पापोऽर्हति क्षमाम्}
{क्षमावन्तं हि पापात्मा जितोऽयमिति मन्यते}


\twolineshloka
{स त्वं क्षुद्रसमाचारो नीचात्मा पापनिश्चयः}
{आकेशाग्रान्नखाग्राच्च वक्तव्यो वक्तुमिच्छसि}


\threelineshloka
{`वरं हि ते मृतिः पाप न च ते चित्तमीदृशम्}
{श्रोतुं वक्तव्यतामूलं नीचाचाराश्च मानवाः}
{परान्क्षिपन्ति दोषेण स्वेषु दोषेष्वदृष्टयः'}


\twolineshloka
{यः स भूरिश्रवाश्छिन्नभुजः प्रायगतस्त्वया}
{वार्यमाणेन हि हतस्ततः पापतरं नु किम्}


\twolineshloka
{प्रेषमाणो मया द्रोणो दिव्येनास्त्रेण संयुगे}
{विसृष्टशस्त्रो निहतः किं तत्र कुर दुष्कृतम्}


\twolineshloka
{अयुध्यमानं यस्त्वाजौ तथा प्रायगतं मुनिम्}
{छिन्नबाहुं परैर्हन्यात्सात्यके स कथं वदेत्}


\twolineshloka
{निहत्य त्वां पदा भूमौ स विकर्षति वीर्यवान्}
{किं तदा न निहंस्येनं भूत्वा पुरुषसत्तमः}


\twolineshloka
{त्वया पुनरनार्येण पूर्वं पार्थेन निर्जितः}
{यदा तदा हतः शूरः सौमदत्तिः प्रतापवान्}


\twolineshloka
{यत्रयत्र तु पाण्डूनां द्रोणो द्रावयते चमूम्}
{किरञ्छरसहस्राणि तत्रतत्र प्रयाम्यहम्}


\twolineshloka
{स त्वमेवंविधं कृत्वा कर्म चण्डालवत्स्वयम्}
{वक्तुमर्हसि वक्तव्यः कस्मात्त्वं परुषाण्यथ}


\twolineshloka
{कर्ता त्वं कर्मणो ह्यस्य नाहं वृष्णिकुलाधम}
{पापानां च त्वमावासः कर्मणां मा पुनर्वद}


\twolineshloka
{जोषमास्व न मां भूयो वक्तुमर्हस्यतः परम्}
{अधरोत्तरमेतद्वि यन्मां त्वं वक्तुमर्हसि}


\twolineshloka
{अथ वक्ष्यसि मां मौर्ख्याद्भूयः परुषमीदृशम्}
{गमयिष्यामि बाणैस्त्वां युधि वैवस्वतक्षयम्}


\twolineshloka
{न चैव मूर्ख धर्मेण केवलेनैव शक्यते}
{तेषामपि ह्यधर्मेण चेष्टितं शृणु यादृशम्}


\twolineshloka
{वञ्चितः पाण्डवः पूर्वमधर्मेण युधिष्ठिरः}
{द्रौपदी च परिक्लिष्टा तथाऽधर्मेण सात्यके}


\twolineshloka
{प्रव्राजिता वनं सर्वे पाण्डवाः सह कृष्णया}
{सर्वस्वमपकृष्टं च तथाऽधर्मेण बालिश}


\twolineshloka
{अधर्मेणापकृष्टश्च मद्रराजः परैरितः}
{अधर्मेण तथा बालः सौभद्रो विनिपातितः}


\twolineshloka
{इतोऽप्यधर्मेण हतो भीष्मः परपुरञ्जयः}
{भूरिश्रवा ह्यधर्मेण त्वया धर्मविदा हतः}


\twolineshloka
{एवं परैराचरितं पाण्डवेयैश्च संयुगे}
{रक्षमाणैर्जयं वीरैर्धर्मज्ञैरपि सात्वत}


\threelineshloka
{दुर्ज्ञेयः स परो धर्मस्तथाऽधर्मश्च दुर्विदः}
{युध्यस्व कौरवैः सार्धं मा गाः पितृनिवेशनम् ॥सञ्जय उवाच}
{}


\twolineshloka
{एवमादीनि वाक्यानि क्रूराणि परुषाणि च}
{श्रावितः सात्यकिः श्रीमानाकम्पित इवाभवत्}


\twolineshloka
{तच्छ्रुत्वा क्रोधताम्राक्षः सात्यकिस्त्वाददे गदाम्}
{विनिः श्वस्य यथा सर्पः प्रणिधाय रथे धनुः}


\twolineshloka
{ततोऽभिपत्य पाञ्चाल्यं संरम्भेणेदमब्रवीत्}
{न त्वां वक्ष्यामि परुषं हनिष्ये त्वां वधक्षमम्}


\twolineshloka
{तमापतन्तं सहसा महाबलममर्षणम्}
{पाञ्चाल्यायाभिसङ्क्रुद्धमन्तकायान्तकोपमम्}


\twolineshloka
{चोदितो वासुदेवेन भीमसेनो महाबलः}
{अवप्लुत्य रथात्तूर्णं बाहुभ्यां समवारयत्}


\twolineshloka
{द्रवमाणं तथा क्रुद्धं सात्यकिं पाण्डवो बली}
{प्रस्पन्दमानमादाय जगाम बलिनं बलात्}


\twolineshloka
{स्थित्वा विष्टभ्य चरणौ भीमेन शिनिपुङ्गवः}
{निगृहीतः पदे षष्ठे बलेन बलिनां वरः}


\twolineshloka
{अवरुह्य रथात्तूर्णं ध्रियमाणं बलीयसा}
{उवाच श्लक्ष्णया वाचा सहदेवो विशाम्पते}


\twolineshloka
{अस्माकं पुरुषव्याघ्र मित्रमन्यन्न विद्यते}
{परमन्धकवृष्णिभ्यः पाञ्चलेभ्यश्च मारिष}


\twolineshloka
{तथैवान्धकवृष्णीनां तथैव च विशेषतः}
{कृष्णस्य च तथाऽस्मत्तो मित्रमन्यन्न विद्यते}


\twolineshloka
{पाञ्चालानां च वार्ष्णेय समुद्रान्तां विचिन्वताम्}
{नान्यदस्ति परं मित्रं यथा पाण्डववृष्णयः}


\twolineshloka
{स भवानीदृशं मित्रं मन्यते च यथा भवान्}
{भवन्तश्च यथाऽस्माकं भवतां च तथा वयम्}


\twolineshloka
{स एवं सर्वधर्मज्ञ मित्रधर्ममनुस्मरन्}
{नियच्छ मन्युं पाञ्चाल्यात्प्रशाम्य शिनिपुङ्गवा}


\twolineshloka
{पार्षतस्य क्षम त्वं वै क्षमतां पार्षतश्च ते}
{वयं क्षमयितारश्च किमन्यत्र शमाद्भवेत्}


\twolineshloka
{प्रशाम्यमाने शैनेये सहदेवेन मारिष}
{पाञ्चालराजस्य सुतः प्रहसन्निदमब्रवीत्}


\twolineshloka
{मुञ्चमुञ्च शिनेः पौत्रं भीम युद्धमदान्वितम्}
{आसादयतु मामेष धराधरमिवानिलः}


\twolineshloka
{यावदस्य शितैर्बाणैः संरम्भं विनयाम्यहम्}
{युद्धश्रुद्धां च कौन्तेय जीवितं चास्य संयुगे}


\twolineshloka
{किं नु शक्यं मया कर्तुं यत्कार्यमिदमुच्यताम्}
{सुमहत्पाण्डुपुत्राणामायान्त्येते हि कौरवाः}


\twolineshloka
{अथवा फल्गुनः सर्वान्वारयिष्यति संयुगे}
{अहमप्यस्य मूर्धानं पातयिष्यामि सायकैः}


\twolineshloka
{मन्यते च्छिन्नबाहुं मां भूरिश्रवसमाहवे}
{उत्सृजैनमहं चैनमेष वा मां हनिष्यति}


\twolineshloka
{शृण्वन्पाञ्चालवाक्यानि सात्यकिः सर्पवच्छ्वसन्}
{भीमबाह्वन्तरे सक्तो विस्फुरत्यनिशं बली}


\threelineshloka
{तौ वृषाविव नर्दन्तौ बलिनौ बाहुशालिनौ}
{त्वरया वासुदेवश्च धर्मराजश्च मारिष}
{यत्नेन महता वीरौ वारयामासतुस्ततः}


\twolineshloka
{निवार्य परमेष्वासौ कोपसंरक्तलोचनौ}
{युयुत्सूनपरान्सङ्ख्ये प्रतीयुः क्षत्रियर्षभाः}


\chapter{अध्यायः २००}
\twolineshloka
{सञ्जय उवच}
{}


\twolineshloka
{ततः स कदनं चक्रे रिपूणां द्रोणनन्दनः}
{युगान्ते सर्वभूतानां कालसृष्ट इवान्तकः}


\twolineshloka
{ध्वजद्रुमं शस्त्रशृङ्गं हतनागमहाशिलम्}
{अश्वकिम्पुरुषाकीर्णं शरासनलतावृतम्}


\twolineshloka
{क्रव्यादपक्षिसङ्घुष्टं भूतयक्षणाकुलम्}
{निहत्य शात्रवान्भल्लैः सोऽचिनोद्देहपर्वतम्}


\twolineshloka
{ततो वेगेन महता विनद्य स नरर्षभः}
{प्रतिज्ञां श्रावयामास पुनरेव तवात्मजम्}


\twolineshloka
{यदत्र च्छद्मनाऽऽचार्यं धर्मकञ्चुकमास्थितः}
{मुञ्च शस्त्रमिति प्राह कुन्तीपुत्रो युधिष्ठिरः}


\twolineshloka
{तस्मात्सम्पश्यतस्तस्य द्रावयिष्यानि वाहिनीम्}
{विद्राव्यसर्वान्हन्ताऽस्मि जाल्मं पाञ्चाल्यमेव तु}


\twolineshloka
{सर्वानेतान्हनिष्यामि यदि योत्स्यन्ति मां रणे}
{सत्यं ते प्रतिजानामि परिवर्तय वाहिनीम्}


\twolineshloka
{तच्छ्रुत्वा तव पुत्रस्तु वाहिनीं पर्यवर्तयत्}
{सिंहनादेन महता व्यपोह्य सुमहद्भयम्}


\twolineshloka
{ततः समागमो राजन्कुरुपाण्डवसेनयोः}
{पुनरेवाभवत्तीव्रः पूर्णसागरयोरिव}


\twolineshloka
{संरब्धा हि स्थिरीभूता द्रोण पुत्रेण कौरवाः}
{उदग्राः पाण्डुपाञ्चाला द्रोणस्य निधनेन च}


\twolineshloka
{तेषां परमहृष्टानां जयमात्मनि पश्यताम्}
{संरब्धानां महावेगः प्रादुरासीद्विशाम्पते}


\twolineshloka
{यथा शिलोच्चये शैलः सागरैः सागरो यथा}
{प्रतिहन्येत राजेन्द्र तथाऽसन्कुरुपाण्डवाः}


\threelineshloka
{`चुक्षुभे पृथिवी सर्वा दिशश्च प्रतिसस्वनुः}
{सम्भ्रान्तान्यपि भूतानि जलजान्यपि मारिष}
{ते च सर्वे तदा योधाः सम्प्रहृष्टा युयुत्सवः'}


\twolineshloka
{ततः शङ्खसहस्राणि भेरीणामयुतानि च}
{अवादयन्त संहृष्टाः कुरुपाण्डवसैनिकाः}


\twolineshloka
{यथा निर्मथ्यमानस्य सागरस्य तु निःस्वनः}
{अभवत्तव सैन्यस्य सुमहानद्भुतोपमः}


\twolineshloka
{`वर्तमाने तथा शब्दे रौद्रे तस्मिन्भयानके}
{सम्पतत्सु रथौघेषु तव तेषां च भारत'}


\twolineshloka
{प्रादुश्चक्रे ततो द्रौणिरस्त्रं नारायणं तदा}
{अभिसन्धाय पाण्डूनां पाञ्चालानां च वाहिनीं}


\twolineshloka
{प्रादुरासंस्ततो बाणा दीप्ताग्राः खे सहस्रशः}
{पाण्डवान्क्षपयिष्यन्तो दीप्तास्यः पन्नगा इव}


\twolineshloka
{ते दिशः खं च सैन्यं च समावृण्वन्महाहवे}
{मुहूर्ताद्भास्करस्येव लोके राजन्गभस्तयः}


\twolineshloka
{तथाऽपरे द्योतमाना ज्योतींषीवामलाम्बरे}
{प्रादुरासन्महाराज कार्ष्णायसमया गुडाः}


\twolineshloka
{चतुश्चक्रा द्विचक्राश्च शतघ्नो बहुला गदाः}
{चक्राणि च क्षुरान्तानि मण्डलानीव भास्वतः}


\twolineshloka
{शस्त्राकृतिभिराकीर्णमतीव पुरुषर्षभ}
{दृष्ट्वान्तरिक्षमाविग्नाः पाण्डुपाञ्चालसृञ्जयाः}


\twolineshloka
{यथायथा ह्ययुध्यन्त पाण्डवानां महारथाः}
{तथातथा तदस्त्रं वै व्यवर्धत जनाधिप}


\twolineshloka
{वध्यमानास्तदाऽस्त्रेण तेन नारायणेन वै}
{दह्यमानाऽनलेनेव सर्वतोऽभ्यर्दिता रणे}


\twolineshloka
{यथा हि शिशिरापाये दहेत्कक्षं हुताशनः}
{तथा तदस्त्रं पाण्डूनां ददाह ध्वजिनीं प्रभो}


\twolineshloka
{आपूर्यमाणेनास्त्रेण सैन्ये क्षीयति च प्रभो}
{जगाम परमं त्रासं धर्मपुत्रो युधिष्ठिरः}


\twolineshloka
{द्रवमाणं तु तत्सैन्यं दृष्ट्वा विगतचेतनम्}
{मध्यस्थतां च पार्थस्य धर्मपुत्रोऽब्रवीदिदम्}


\twolineshloka
{धृष्टद्युम्न पलायस्व सह पाञ्चालसेनया}
{सात्यके त्वं च गच्छस्व वृष्ण्यन्धकवृतो महान्}


\twolineshloka
{वासुदेवोऽपि धर्मात्मा करिष्यत्यात्मनः क्षमम्}
{श्रेयो ह्युपदिशत्येष लोकस्य किमुतात्मनः}


\twolineshloka
{सङ्ग्रामस्तु न कर्तव्यः सर्वसैन्यान्ब्रवीमि वः}
{अहं हि सह सोदर्यैः प्रवेक्ष्ये हव्यवाहनम्}


\twolineshloka
{भीष्मद्रोणार्णवं तीर्त्वा सङ्ग्रामे भीरुदुस्तरे}
{विमज्जिष्यामि सलिले सगणो द्रौणिगोष्पदे}


\twolineshloka
{कामः सम्पद्यतामस्य बीभत्सोराशु मां प्रति}
{कल्याणवृत्तिराचार्यो मया युधि निपातितः}


\twolineshloka
{येन बालः स सौभद्रो युद्धानामविशारदः}
{समर्थैर्बहुभिः क्रूरैर्घातितो नाभिपालितः}


\twolineshloka
{येन विब्रुवती प्रश्नं तथा कृष्णा सभां गता}
{उपेक्षिता सपुत्रेण दासीभावं नियच्छती}


\twolineshloka
{`रक्षणे च महान्यत्नः सैन्धवस्य कृतो युधि}
{अर्जुनस्य विघातार्थं प्रतिज्ञा येन रक्षिता}


\twolineshloka
{व्यूहद्वारि वयं चैव धृता येन जिगीषवः}
{वारितं च महत्सैन्यं प्रविशत्तद्यथाबलम्'}


\twolineshloka
{जिघांसुर्धार्तराष्ट्रश्च श्रान्तेष्वन्येषु फल्गुनम्}
{कवचेन तथा गुप्तो रक्षार्थं सैन्धवस्य च}


\twolineshloka
{येन ब्रह्मास्त्रविदुषा पाञ्चालाः सत्यजिन्मुखाः}
{कुर्वाणा मज्जये यत्नं समूला विनिपातिताः}


\twolineshloka
{`ग्रहणे च परो यत्नः कृतस्तेन यथा मम}
{विदितं सर्वमेवैतद्भवतां सर्वयोधिनाम्'}


\twolineshloka
{येन प्रव्राज्यमानाश्च राज्याद्वयमधर्मतः}
{निवार्यमाणेनास्माभिरनुगन्तुं तदेषिताः}


\twolineshloka
{`वनवासान्निवृत्तानां समये च तथा कृते}
{स्नेहश्च दर्शितो नित्यं प्रत्यक्षं वो महारथाः'}


\threelineshloka
{योऽसावत्यन्तमस्मासु कुर्वाणः सौहृदं परम्}
{हतस्तदर्थे मरणं गमिष्यामि सबान्धवः ॥सञ्जय उवाच}
{}


\twolineshloka
{एवं ब्रुवति कौन्तेये दाशार्हस्त्वरितस्ततः}
{निवार्य सैन्यं बाहुभ्यामिदं वचनमब्रवीत्}


\twolineshloka
{शीघ्रं न्यस्यत शस्त्राणि वाहेभ्यश्चावरोहत}
{एष योगोऽत्र विहितः प्रतिघाते महात्मनः}


\twolineshloka
{द्विपाश्वस्यन्दनेभ्यश्च क्षितिं सर्वेऽवरोहत}
{एवमेतन्न वो हन्यादस्त्रं भूमौ निरायुधान्}


\twolineshloka
{यथायथा हि युध्यन्ते योधा ह्यस्त्रमिदं प्रति}
{तथातथा भवन्त्येते कौरवा बलवत्तराः}


\threelineshloka
{निक्षेप्स्यन्ति च शस्त्राणि वाहनेभ्योऽवरुह्य ये}
{येऽञ्जलिं कुर्वते वीर नमस्ति च विवाहनाः}
{तान्नैतदस्त्रं सङ्ग्रामे निहनिष्यति मानवान्}


\twolineshloka
{ये त्वेतत्प्रतियोत्स्यन्ति मनसाऽपीह केचन}
{निहनिष्यति तान्सर्वान्रसातलगतानपि}


\twolineshloka
{ते वचस्तस्य तच्छ्रुत्वा वासुदेवस्य भारत}
{ईषुः सर्वे समुत्स्रष्टुं मनोभिः करणेन च}


\twolineshloka
{तत उत्स्रष्टुकामांस्तानस्त्राण्यालक्ष्य पाण्डवः}
{भीमसेनोऽब्रवीद्राजन्निदं संहर्षयन्वचः}


\twolineshloka
{न कथञ्चन शस्त्राणि मोक्तव्यानीह केनचित्}
{अहमावारयिष्यामि द्रोणपुत्रास्त्रमाशुगैः}


\twolineshloka
{गदयाऽप्यनया गुर्व्या हेमविग्रहया रणे}
{कालवत्प्रहरिष्यामि द्रौणेरस्त्रं विशातयन्}


\twolineshloka
{न हि मे विक्रमे तुल्यः कश्चिदस्ति पुमानिह}
{यथैव सवितुस्तुल्यं ज्योतिरन्यन्न विद्यते}


\twolineshloka
{पश्यतेमौ हि मे बाहू नागराजकरोपमौ}
{समर्थौ पर्वतस्यापि शैशिरस्य निपातने}


\twolineshloka
{नागायुतसमप्राणो ह्यहमेको नरेष्विह}
{शक्रो यथाऽप्रतिद्वन्द्वो दिवि देवेषु विश्रुतः}


\twolineshloka
{अद्य पश्यत मे वीर्यं बाह्वोः पीनांसयोर्युधि}
{ज्वलमानस्य दीप्तस्य द्रौणेरस्त्रस्य वारणे}


\twolineshloka
{यदि नारायणास्त्रस्य प्रतियोद्वा न विद्यते}
{अद्यैतत्प्रतियोत्स्यामि पश्यत्सु कुरुपाण्डुपु}


\threelineshloka
{अर्जुनार्जुन बीभत्सो न न्यस्यं गाण्डिवं त्वया}
{शशाङ्कस्येव ते पङ्को नैर्मल्यं पातयिष्यति ॥अर्जुन उवाच}
{}


\twolineshloka
{भीम नारायणास्त्रे मे गोषु च ब्राह्मणेषु च}
{एतेषु गाण्डिवं न्यस्यमेतद्धि व्रतमुत्तमम्}


\twolineshloka
{एवमुक्तस्ततो भीमो द्रोणपुत्रमरिन्दमम्}
{अभ्ययान्मेघघोषेण रथेनादित्यवर्चसा}


\twolineshloka
{`कम्पयन्मेदिनीं सर्वां त्रासयंश्च चमूं तव}
{शङ्खशब्दं महत्कृत्वा भुजशब्दं च पाण्डवः}


\twolineshloka
{तस्य शङ्खस्वनं श्रुत्वा बाहुशब्दं च तावकाः}
{समन्तात्कोष्ठकीकृत्य शरव्रातैरवाकिरन्'}


\twolineshloka
{स एनमिषुजालेन लघुत्वाच्छीघ्रविक्रमः}
{निमेषमात्रेणासाद्य कुन्तीपुत्रोऽभ्यवाकिरत्}


\twolineshloka
{ततो द्रौणिः प्रहस्यैनं द्रवन्तमभिभाष्य च}
{अवाकिरत्प्रदीप्ताग्रैः शरैस्तैरभिमन्त्रितैः}


\twolineshloka
{पन्नगैरिव दीप्तास्यैर्वमद्भिर्ज्वलनं रणे}
{अवकीर्णोऽभवत्पार्थः स्फुलिङ्गैरिव काञ्चनैः}


\twolineshloka
{तस्य रूपमभूद्राजन्भीमसेनस्य संयुगे}
{खद्योतैरावृतस्येव पर्वतस्य दिनक्षये}


\twolineshloka
{तदस्त्रं द्रोणपुत्रस्य तस्मिन्प्रतिसमस्यति}
{अवर्धत महाराज यथाऽग्निरनिलोद्धतः}


\twolineshloka
{विवर्धमानमालक्ष्य तदस्त्रं भीमविक्रमम्}
{पाण्डुसैन्यमृते भीमं सुमहद्भयमाविशत्}


\twolineshloka
{ततः शस्त्राणि ते सर्वे समुत्सृज्य महीतले}
{अवारोहन्रथेभ्यश्च हस्त्यश्वेभ्यश्च सर्वशः}


\twolineshloka
{तेषु निक्षिप्तशस्त्रेषु वाहनेभ्यश्च्युतेषु च}
{तदस्त्रवीर्यं विपुलं भीममूर्धन्यथापतत्}


\twolineshloka
{हाहाकृतानि भूतानि पाण्डवाश्च विशेषतः}
{भीमसेनमपश्यन्त तेजसा संवृतं तथा}


\chapter{अध्यायः २०१}
\twolineshloka
{सञ्जय उवाय}
{}


\twolineshloka
{भीमसेनं समाकीर्णं दृष्ट्वाऽस्त्रेण धनञ्जयः}
{तेजसः प्रतिघातार्थं वारुणेन समावृणोत्}


\twolineshloka
{नालक्षयत तत्कश्चिद्वारुणास्त्रेण संवृतम्}
{अर्जुनस्य लघुत्वाच्च संवृतत्वाच्च तेजसः}


\twolineshloka
{साश्वसूतरथो भीमो द्रोणपुत्रास्त्रसंवृतः}
{अग्नावग्निरिव न्यस्तो ज्वालामाली सुदुर्दृशः}


\twolineshloka
{यथा रात्रिक्षये राजञ्ज्योतीष्यस्तगिरिं प्रति}
{समापेतुस्तथा बाणा भीमसेनरथं प्रति}


\twolineshloka
{स हि भीमो रथश्चास्य हयाः सूतश्च मारिष}
{संवृता द्रोणपुत्रेण पावकान्तर्गताऽभवन्}


\twolineshloka
{यथा जग्ध्वा जगत्कृत्स्नं समये सचराचरम्}
{गच्छेद्वह्निर्विभोरास्यं तथाऽस्त्रं भीममावृणोत्}


\twolineshloka
{सूर्यमग्निः प्रविष्टः स्याद्यथा चाग्निं दिवाकरः}
{तथा प्रविष्टं तत्तेजो न आज्ञायत पाण्डवः}


\threelineshloka
{`तदस्त्रं भीमहुङ्कारादपयाति पुनःपुनः}
{पुनः पुनस्तमायाति हुंकारात्तं विमुञ्चति}
{ततो देवाः सगन्धर्वा भीमं दृष्ट्वा सुविस्मिताः'}


\twolineshloka
{विकीर्णमस्त्रं तद्दृष्ट्वा तथा भीमरथं प्रति}
{उदीर्यमाणं द्रौणिं च निष्प्रतिद्वन्द्वमाहवे}


\twolineshloka
{सर्वसैन्यं च पाण्डूनां न्यस्तशस्त्रमचेतनम्}
{युधिष्ठिरपुरोगांश्च विमुखांस्तान्महारथनान्}


\twolineshloka
{अर्जुनो वासुदेवश्च त्वरमाणौ महाद्युती}
{अवप्लुत्य रथाद्वीरौ भीममाद्रवतां ततः}


\twolineshloka
{ततस्तद्द्रोणपुत्रस्य तेजोऽस्त्रबलसम्भवम्}
{विगाह्य तौ सुबलिनौ माययाऽविशतां तथा}


\twolineshloka
{न्यस्तशस्त्रौ ततस्तौ तु नादहत्सोऽस्त्रजोऽनलः}
{वारुणास्त्रप्रयोगाच्च वीर्यवत्त्वाच्च कृष्णयोः}


\twolineshloka
{ततश्चकृषतुर्भीमं सर्वशस्त्रायुधानि च}
{नारायणास्त्रशान्त्यर्थं नरनारायणौ बलात्}


\twolineshloka
{आकृष्यमाणः कौन्तेयो नदत्येव महारवम्}
{वर्धते चैव तद्वोरं द्रौणेरस्त्रं सुदुर्जयम्}


\twolineshloka
{तमब्रवीद्वासुदेवः किमिदं पाण्डुनन्दन}
{वार्यमाणोऽपि कौन्तेय यद्युद्वान्न निवर्तसे}


\twolineshloka
{यदि युद्वेन जेयाः स्युरिमे कौरवनन्दनाः}
{वयमप्यत्र युध्येम तथा चेमे नरर्षभाः}


\twolineshloka
{रथेभ्यस्त्ववतीर्णाः स्म सर्व एव हि तावकाः}
{तस्मात्त्वमपि कौन्तेय रथात्तूर्णमपाक्रम}


\twolineshloka
{एवमुक्त्वा तु तं कृष्णो रथाद्भूमिमवर्तयत्}
{निःश्वसन्तं यथा नागं क्रोधसंरक्तलोचनम्}


\threelineshloka
{यदाऽपकृष्टः स रथान्न्यासितश्चायुधं भुवि}
{ततो नारायणास्त्रं तत्प्रशान्तं शत्रुतापनम् ॥सञ्जय उवाच}
{}


\twolineshloka
{तस्मिन्प्रशान्ते विधिना तेन तेजसि दुःसहे}
{बभूवुर्विमलाः सर्वा दिशः प्रदिश एव च}


\twolineshloka
{प्रववुश्च शिवा वाताः प्रशान्ता मृगपक्षिणः}
{वाहनानि च हृष्टानि प्रशान्तेऽस्त्रे सुदुर्जये}


\twolineshloka
{व्यपोढे च ततो घोरे तस्मिंस्तेजसि भारत}
{बभौ भीमो निशापाये धीमान्सूर्य इवोदितः}


\twolineshloka
{हतशेषं बलं तत्तु पाण्डवानामतिष्ठत}
{अस्त्रव्युपरमाद्वृष्टं तव पुत्रजिघांसया}


\twolineshloka
{व्यवस्थिते बले तस्मिन्नेस्त्रे प्रतिहते तथा}
{दुर्योधनो महाराज द्रोणपुत्रमथाब्रवीत्}


\twolineshloka
{अश्वत्थामन्पुनः शीघ्रमस्त्रमेतत्प्रयोजय}
{अवस्थिता हि पाञ्चालाः पुनरेते जयैषिणः}


\twolineshloka
{अश्वत्थामा तथोक्तस्तु तव पुत्रेण मारिष}
{सुदीनमभिनिः श्वस्य राजानमिदमब्रवीत्}


\twolineshloka
{नैतदावर्तते राजन्नस्त्रं द्विर्नोपपद्यते}
{आवृतं हि निवर्तेत प्रयोक्तारं न संशयः}


\twolineshloka
{एष चास्त्रप्रतीघातं वासुदेवः प्रयुक्तवान्}
{`अस्य तु ह्येष वै दाता मानुषेषु न विद्यते}


\twolineshloka
{परावरज्ञो लोकानां न तदस्ति न वेत्ति यत्}
{तदेतदस्त्रं प्रशमं यातं कृष्णस्य मन्त्रितैः'}


\twolineshloka
{द्विविधो विहितः सङ्ख्ये वधः शस्त्रोर्जनाधिप}
{पराजयो वा मृत्युर्वा श्रेयान्मृत्युर्न निर्जयः}


\twolineshloka
{विजिताश्चारयो ह्योते शस्त्रोत्सर्गान्मृतोपमाः ॥दुर्योधन उवाच}
{}


\twolineshloka
{आचार्यपुत्र यद्येतद्द्विरस्त्रं न प्रयुज्यते}
{अन्यैर्गुरुघ्ना वध्यन्तामस्त्रैरस्त्रविदां वर}


\fourlineindentedshloka
{त्वयि शस्त्राणि दिव्यानि त्र्यम्बके चामितौजसि}
{`घ्नतैतान्सुमहावीर्य शात्रवान्युद्वकोविदः'}
{इच्छतो न हि ते मुच्येत्सङ्क्रुद्धो हि पुरन्दरः ॥धृतराष्ट्र उवाच}
{}


\twolineshloka
{तस्मिन्नस्त्रे प्रतिहते द्रोणे चोपधिना हते}
{तथा दुर्योधनेनोक्तो द्रौणिः किमकरोत्पुनः}


\threelineshloka
{दृष्ट्वा पार्थांश्च सङ्ग्रामे युद्धाय समुपस्थितान्}
{नारायणास्त्रनिर्मुक्तांश्चरतः पृतनामुखे ॥सञ्जय उवाच}
{}


\twolineshloka
{जानन्पितुः स निधनं सिंहलाङ्गूलकेतनः}
{सक्रोधो भयमुत्सृज्य सोऽभिदुद्राव पार्षतम्}


\twolineshloka
{अभिद्रुत्य च विंशत्या क्षुद्रकाणां नरर्षभ}
{पञ्चभिश्चातिवेगेन विव्याध पुरुषर्षभः}


\twolineshloka
{धृष्टद्युम्नस्ततो राजञ्ज्वलन्तमिव पावकम्}
{द्रोणपुत्रं त्रिषष्ट्या तु राजन्विव्याध पत्रिणाम्}


\twolineshloka
{सारथिं चास्य विंशत्या स्वर्णपुङ्खैः शिलाशितैः}
{हयांश्च चतुरोऽविध्यच्चतुर्भिर्निशितैः शरैः}


\twolineshloka
{विद्व्वा विद्वाऽनदद्द्रौणिं कम्पयन्निव मेदिनीम्}
{आददे सर्वलोकस्य प्राणानिव महारणे}


\twolineshloka
{पार्षतस्तु बली राजन्कृतास्त्रः कृतनिश्चयः}
{द्रौणिमेवाभिदुद्राव मृत्युं कृत्वा निवर्तनम्}


\twolineshloka
{ततो बाणमयं वर्षं द्रोणपुत्रस्य मूर्धनि}
{अवासृजदमेयात्मा पाञ्चाल्यो रथिनांवरः}


\twolineshloka
{तं द्रौणिः समरे क्रुद्धं छादयामास पत्रिभिः}
{विव्याध चैनं दशभिः पितुर्वधमनुस्मरन्}


\twolineshloka
{द्वाभ्यां च सुविसृष्टाभ्यां क्षुराभ्यां ध्वजकार्मुके}
{छित्त्वा पाञ्चालराजस्य द्रौणिरन्यैः समार्दयत्}


\twolineshloka
{व्यश्वसूतरथं चैनं द्रौणिश्चक्रे महाहावे}
{तस्य चानुचरान्सर्वान्क्रुद्धः प्राद्रावयच्छरैः}


\twolineshloka
{ततः प्रदुद्रुवे सैन्यं पाञ्चालानां विशाम्पते}
{सम्भ्रान्तरूपमार्तं च न परस्परमैक्षत}


\twolineshloka
{दृष्ट्वा तु विमुखान्योधान्धृष्टद्युम्नं च पीडितम्}
{शैनेयोऽचोदयत्तूर्णं रथं द्रौणिरथं प्रति}


\threelineshloka
{अष्टभिर्निशितैर्बाणैरश्वत्थामानमार्दयत्}
{विंशत्या पुनराहत्य नानारूपैरमर्षणः}
{विव्याध च तथा सूतं चतुर्भिश्चतुरो हयान्}


\threelineshloka
{[धनुर्ध्वजं* ; संयत्तश्चिच्छेद कृतहस्तवत्}
{स साश्वं व्यधमच्चापि रथं हेमपरिष्कृतम्}
{हृदि विव्याध समरे त्रिंशता सायकैर्भृशम्}


\twolineshloka
{एवं स पीडितो राजन्नश्वत्थामा महाबालः}
{शरजालैः परिवृतः कर्तव्यं नान्वपद्यत}


\twolineshloka
{एवं गते गुरोः पुत्रे तव पुत्रो महारथः}
{कृपकर्णादिभिः सार्धं शरैः सात्वतमावृणोत्}


\twolineshloka
{दुर्योधनस्तु विंशत्या कृपः शारद्वतस्त्रिभिः}
{कृतवर्माऽथ दशभिः कर्णः पञ्चाशता शरैः}


\twolineshloka
{दुःशासनः शतेनैव वृषसेनश्च सप्तभिः}
{सात्यकिं विव्यधुस्तूर्णं समन्तान्निशितैः शरैः}


\twolineshloka
{ततः स सात्यकी राजन्सर्वानेव महारथान्}
{विरथान्विमुखांश्चैव क्षणेनैवाकरोन्नृप}


\twolineshloka
{अश्वत्थामा तु सम्प्राप्य चेतनां भरतर्षभ}
{चिन्तयामास दुःखार्तो निःश्वसंश्च पुनःपुनः}


\twolineshloka
{अथो रथान्तरं द्रौणिः समारुह्य परन्तपः}
{सात्यकिं वारयामास किरञ्शरशतान्बहून्}


\twolineshloka
{तमापतन्तं सम्प्रेक्ष्य भारद्वाजसुतं रणे}
{विरथं विमुखं चैव पुनश्चक्रे महारथः}


\twolineshloka
{ततस्ते पाण्डवा राजन्दृष्ट्वा सात्यकिविक्रमम्}
{शङ्खशब्दान्भृशं चक्रुः सिंहनादांश्च नेदिरे}


\twolineshloka
{एवं तं विरथं कृत्वा सात्यकिः सत्यविक्रमः}
{जघान वृषसेनस्य त्रिसाहस्रान्महारथान्}


\twolineshloka
{अयुतं दन्तिनां सार्धं कृपस्य निजघान सः}
{पञ्चायुतानि चाश्वानां शकुनेर्निजघान ह}


\threelineshloka
{ततो द्रौणिर्महाराज रथमारुह्य वीर्यवान्}
{सात्यकिं प्रति सङ्क्रुद्धः प्रययौ तद्वधेप्सया}
{}


\twolineshloka
{पुनस्तमागतं दृष्ट्वा शनेयो निशितैः शरैः}
{अदारयत्क्रुरतरैः पुनःपुनररिन्दम ॥]}


\twolineshloka
{सोऽतिविद्धो महेष्वासो नानालिङ्गैरमर्षणः}
{युयुधानेन वै द्रौणिः प्रहसन्वाक्यमब्रवीत्}


\twolineshloka
{शैनेयाब्युपपत्तिं ते जानाम्याचार्यघातिनि}
{न चैनं त्रास्यसि मया ग्रस्तमात्मानमेव च}


\twolineshloka
{शपेत्मनाऽहं शैनेय सत्येन तपसा तथा}
{अहत्वा सर्वपाञ्चालान्यदि शान्तिमहं लभे}


\twolineshloka
{यद्बलं पाण्डवेयानां वृष्णीनामपि यद्बलम्}
{क्रियतां सर्वमेवेह निहनिष्यामि सोमकान्}


\twolineshloka
{एवमुक्त्वाऽर्करम्याभं सुतीक्ष्णं तं शरोत्तमम्}
{व्यसृजत्सात्वते द्रौणिर्वज्रं वृत्रे यथा हरिः}


\twolineshloka
{स तं निर्भिद्य तेनास्तः सायकः शरावरम्}
{विवेश वसुधां भित्त्वा श्वसन्बिलमिवोरगः}


\twolineshloka
{स भिन्नकवचः शूरस्तोत्रार्दित इव द्विपः}
{विमुच्य सशरं चापं भूरिव्रणपरिस्रवः}


\twolineshloka
{सीदन्रुधिरसिक्तश्च रथोपस्थ उपाविशत्}
{सूतेनापहृतस्तूर्णं द्रोणपुत्राद्रथान्तरम्}


\twolineshloka
{अथान्येन सुपुङ्खेन शरेणानतपर्वणा}
{आजघान भ्रवोर्मध्ये धृष्टद्युम्नं परन्तपः}


\twolineshloka
{स पूर्वमतिविद्धश्च भृशं पस्चाच्च पीडितः}
{ससदाथ च पाञ्चाल्यो व्यपाश्रयत च ध्वजम्}


\twolineshloka
{तं नागमिव सिंहेन दृष्ट्वा राजञ्शरार्दितम्}
{जवेनाभ्यद्रवञ्छूराः पञ्च पाण़्डवतो रथाः}


\twolineshloka
{किरीटी भीमसेनश्च वृद्वक्षत्रश्च पौरवः}
{युवराजश्च चेदीनां मालवश्च सुदर्शनः}


\twolineshloka
{एते हाहाकृताः सर्वे प्रगृहीतशरासनाः}
{वीरं द्रौणायनिं वीराः सर्वतः पर्यवारयन्}


\twolineshloka
{ते विंशतिपदे यत्ता गुरुपुत्रममर्षणम्}
{पञ्चभिः पञ्चभिर्बाणैरभ्यघ्नन्सर्वतः समम्}


\twolineshloka
{आशीविषाभैर्विंशत्या पञ्चभिस्तु शितैः शरैः}
{चिच्छेद युगपद्द्रौणिः पञ्चविंशतिसायकान्}


\threelineshloka
{सप्तभिस्तु शितैर्बाणैः पौरवं द्रौणिरार्दयत्}
{मालवं त्रिभिरेकेन पार्थं ष़ड्भिर्वृकोदरम्}
{}


\twolineshloka
{ततस्ते विव्यधुः सर्वे द्रौणिं राजन्महारथाः}
{युगपच्च पृथक्चैव रुक्मपुङ्खैः शिलाशितैः}


\twolineshloka
{युवराजश्च विंशत्या द्रौणिं विव्याध पत्रिभिः}
{पार्थश्च पुनरष्टाभिस्तथा सर्वे त्रिभिस्त्रिभिः}


\twolineshloka
{ततोऽर्जुनं षड्भिरथाजघानद्रौणायनिर्दशभिर्वासुदेवम्}
{भीमं दशार्धैर्युवारजं चतुर्भि-र्द्वाभ्यां द्वाभ्यां मालवं पौरवं च}


\twolineshloka
{सूतं विद्व्वाः भीमसेनस्य षड्भि--र्द्वाभ्यां विद्व्वा कार्मुकं च ध्वजं च}
{पुनः पार्थं शरवर्षेण विद्व्वाद्रौणिर्घोरं सिंहनादं ननाद}


\twolineshloka
{तस्यास्यतस्तान्निशितान्पीतधारा--न्द्रौणेः शरान्पृष्ठतश्चाग्रतश्च}
{धरा वियद्द्यौः प्रदिशो दिशश्चच्छन्ना बाणैरभवन्धोररूपैः}


\twolineshloka
{आसन्नस्य स्वरथं तीव्रतेजाःसुदर्शनस्येन्द्रकेतुप्रकाशौ}
{भुजौ शिरश्चेन्द्रसमानवीर्य--स्त्रिभिः शरैर्युगपत्सञ्चकर्त}


\twolineshloka
{स पौरवं रथशक्त्या निहत्यच्छित्त्वा रथं तिलशश्चास्य बाणैः}
{छित्त्वा च बाहू वरचन्दनाक्तौभल्लेन कायाच्छिर उच्चकर्त}


\twolineshloka
{युवानमिन्दीवरदामवर्णंचेदिप्रभुं युवराजं प्रसह्य}
{बाणैस्त्वरावाञ्ज्वलिताग्निकल्पै--र्विद्व्वा प्रादान्मृत्येव साश्वसूतम्}


\threelineshloka
{[*मालवं पौरवं चैव युवराजं च चेदिपम्}
{दृष्ट्वा समक्षं निहतं द्रोणपुत्रोण पाण्डवः}
{भीमसेनो महाबाहुः क्रोधमाहारयत्परम्}


\twolineshloka
{ततः शरशतैस्तीक्ष्णैः सङ्क्रुद्धाशीविषोपमैः}
{छादयामास समरे द्रोणपुत्रं परन्तपः}


\twolineshloka
{ततो द्रौणिर्महातेजाः शरवर्षं निहत्य तम्}
{विव्याध निशितैर्बाणैर्भीमसेनममर्षणः}


\twolineshloka
{ततो भीमो महाबाहुद्रौणेर्युधि महाबलः}
{क्षुरप्रेण धनुश्छित्त्वा द्रौणिं विव्याध पत्रिणा}


\twolineshloka
{तदपास्य धनुश्छिन्नं द्रोणपुत्रो महामनाः}
{अन्यत्कार्मुकमादाय भीमं विव्याध पत्रिभिः}


\twolineshloka
{तौ द्रौणिभीमौ समरे पराक्रान्तौ महाबलौ}
{अवर्षतां शरवर्षं वृष्टिमन्ताविवाम्बुदौ}


\twolineshloka
{भीमनामाङ्किता बाणाः स्वर्णपुङ्खाः शिलाशिताः}
{द्रौणिं सञ्छादयामासुर्घनौघा इव भास्करम्}


\twolineshloka
{तथैव द्रौणिनिर्मुक्तैर्भीमः सन्नतपर्वभिः}
{अवाकीर्यत स क्षिप्रं शरैः शतसहस्रशः}


\twolineshloka
{स च्छाद्यमानः समरे द्रौणिना रणशालिना}
{न विव्यथे महाराज तदद्भुतमिवाभवत्}


\twolineshloka
{ततो भीमो महाबाहुः कार्तस्वरविभूषितान्}
{नाराचान्दश सम्प्रैषीद्यमदण्डनिभाञ्छितान्}


\twolineshloka
{ते जत्रुदेशमासाद्य द्रोणपुत्रस्य मारिष}
{निर्भिद्य विविशुस्तूर्णं वल्मीकमिव पन्नगाः}


\twolineshloka
{सोऽतिविद्धो भृशं द्रौणिः पाण्डवेन महात्मना}
{ध्वजयष्टिं समासाद्य न्यमीलयत लोचने}


\twolineshloka
{स मुहूर्तात्पुनः संज्ञां लब्ध्वा द्रौणिर्नराधिप}
{क्रोधं परममातस्थौ समरे रुधिरोक्षितः}


\twolineshloka
{दृढं सोऽभिहतस्तेन पाण्डवेन महात्मना}
{वेगं चक्रे महाबाहुर्भीमसेनरथं प्रति}


\twolineshloka
{तत आकर्णपूर्णानां शराणां तिग्मतेजसाम्}
{शतमाशीविषाभानां प्रेषयामास भारत}


\twolineshloka
{भीमोऽपि समरश्लाघी तस्य वीर्यमचिन्तयत्}
{तूर्णं प्रासृजदुग्राणि शरवर्षाणि पाण्डवः}


\twolineshloka
{ततो द्रौणिर्महाराज च्छित्त्वाऽस्य विशिखैर्धनुः}
{आजघानोरसि क्रुद्धः पाण्डवं निशितैः शरैः}


\twolineshloka
{ततोऽन्यद्धनुरादाय भीमसेनो ह्यमर्षणः}
{विव्याध निशितैर्बाणैर्द्रौणिं पञ्चभिराहवे}


\twolineshloka
{जीमूताविव घर्मान्ते तौ शरौघप्रवर्षिणौ}
{अन्योन्यक्रोधताम्राक्षौ छादयामासतुर्युधि}


\twolineshloka
{तलशब्दैस्ततो घोरैस्त्रासयन्तौ परस्परम्}
{अयुध्येतां सुसंरब्धौ कृतप्रतिकृतैषिणौ}


\threelineshloka
{ततो विष्फार्य सुमहच्चापं रुक्मविभूषितम्}
{भीमं प्रैक्षत स द्रौणिः शरानस्यन्तमन्तिकात्}
{शरद्यहर्मध्यगतो द्रीप्तार्चिरिव भास्करः}


\twolineshloka
{आददनास्य विशिखान्सन्दधानस्य चाशुगान्}
{विकर्षतो मुञ्चतश्च नान्तरं ददृशुर्जनाः}


\twolineshloka
{अलातचक्रप्रतिमं तस्य मण्डलमायुधम्}
{द्रौणेरासीन्महाराज बाणान्विसृजतस्तदा}


\twolineshloka
{धनुश्च्युताः शरास्तस्य शतशोऽथ सहस्रशः}
{आकाशे प्रत्यदृश्यन्त शलभानामिवायतीः}


\twolineshloka
{ते तु द्रौणिविनिर्मुक्ताः शरा हेमविभूषिताः}
{अजस्रमन्वकीर्यन्त घोरा भीमरथं प्रति}


\twolineshloka
{तत्राद्भुतकमपश्याम भीमसेनस्य विक्रमम्}
{बलं वीर्यं प्रभावं च व्यवसायं च भारत}


\twolineshloka
{तां स मेघादिवोद्भूतां बाणवृष्टिं समन्ततः}
{जलवृष्टिं महाघोरां तपान्त इव चिन्तयन्}


\twolineshloka
{द्रोणपुत्रवधप्रेप्सुर्भीमो भीमपराक्रमः}
{अमुञ्चच्छरवर्षाणि प्रावृषीव बलाहकः}


\twolineshloka
{तद्रुक्मपृष्टं भीमस्य धनुर्घारं महारणे}
{विकृष्यमाणं विबभौ शक्रचापमिवापरम्}


\twolineshloka
{तस्माच्छराः प्रादुरासञ्छतशोऽथ सहस्रशः}
{सञ्छादयन्तः समरे द्रौणिमाहवशोभिनम्}


\twolineshloka
{तयोर्विसृजतोरेवं शरजालानि मारिष}
{वायुरप्यन्तरा राजन्नाशक्नोत्प्रतिसर्पितुम्}


\twolineshloka
{तथा द्रौणिर्महाराज शरान्हेमविभूषितान्}
{तैलधौतान्प्रसन्नाग्रान्प्राहिणोद्वधकाङ्क्षया}


% Check verse!
तानन्तरिक्षे विशिखैस्त्रिधैकैकमशातयत्विशेषयन्द्रोणसुतं तिष्ठतिष्ठेति चाब्रवीत्
\twolineshloka
{पुनश्च शरवर्षाणि घोराण्युग्राणि पाण्डवः}
{व्यसृजद्बलवानक्रुद्धो द्रोणपुत्रवधेप्सया}


\threelineshloka
{ततोऽस्त्रमायया तूर्णं शरवृष्टिं निवार्य ताम्}
{धनुश्चिच्छेद भीमस्य द्रोणपुत्रो महास्त्रवित्}
{शरैश्चैनं सुबहुभिः क्रुद्धः सङ्ख्ये पराभिनत्}


% Check verse!
स छिन्नधन्वा बलवान्रथशक्तिं सुदारुणाम् ॥वेगेनाविध्य चिक्षेप द्रोणपुत्ररथं प्रति
\twolineshloka
{तामापतन्तीं सहसा महोल्काभां शितैः शरैः}
{चिच्छेद समरे द्रौणिर्दर्शयन्पाणिलाघवम्}


\twolineshloka
{एतस्मिन्नन्तरे भीमो दृढमादाय कार्मुकम्}
{द्रौणिं विव्याध विशिखैः स्मयमानो वृकोदरः}


\twolineshloka
{ततो द्रौणिर्महाराज भीमसेनस्य सारथिम्}
{ललाटे दारयामास शरेणानतपर्वणा}


\twolineshloka
{सोऽतिविद्धो बलवता द्रोणपुत्रेण सारथिः}
{व्यामोहमगमद्राजन्रश्मीनुत्सृज्य वाजिनां}


\twolineshloka
{ततोऽश्वाः प्राद्रवंस्तूर्णं मोहिते रथसारथौ}
{भीमसेनस्य राजेनद्र् पश्यतां सर्वधन्विनाम्}


\twolineshloka
{तं दृष्ट्वा प्रद्रुतैरश्वैरपकृष्टं रणाजिरात्}
{दध्मौ प्रमुदितः शङ्खं बृहन्तमपराजितः}


\twolineshloka
{ततः सर्वे च पाञ्चाला भीमसेनश्च पाण़्डवः}
{धृष्टद्युम्नरथं त्यक्त्वा भीताः सम्प्राद्रवन्दिशः}


\twolineshloka
{तान्प्रभग्नांस्ततो द्रौणिः पृष्ठतो विकिरञ्शरान्}
{अभ्यवर्तत वेगेन कालयन्पाण्डुवाहिनीम्}


\twolineshloka
{ते वध्यमानाः समरे द्रोणपुत्रेण पार्थिवाः}
{द्रोणपुत्रभयाद्राजन्दिशः सर्वाश्च भेजिरे}


\chapter{अध्यायः २०२}
\twolineshloka
{सञ्जय उवाच}
{}


\twolineshloka
{तत्प्रभग्नं बलं दृष्ट्वा कुन्तीपुत्रो धनञ्जयः}
{न्यवारयदमेयात्मा द्रोणपुत्रजयेप्सया}


\twolineshloka
{ततस्ते सैनिका राजन्सर्वे तत्रावतस्थिरे}
{संस्थाप्यमाना यत्नेन गोविन्देनार्जुनेन च}


\twolineshloka
{एक एव च बीभत्सुः सोमकावयवैः सह}
{मात्स्यैरन्यैश्च संधाय कौरवान्सन्न्यवर्तत}


\twolineshloka
{ततो द्रुतमतिक्रम्य सिंहलाङ्गूलकेतनम्}
{सव्यसाची महेष्वासमश्वत्थामानमब्रवीत्}


\twolineshloka
{य शक्तिर्यच्च विज्ञानं यद्वीर्यं यच्च पौरुषम्}
{धार्तराष्ट्रेषु या प्रीतिर्देषोऽस्मासु च यच्च ते}


\twolineshloka
{यच्च भूयोऽस्ति तेजस्ते तत्सर्वं मयि दर्शय}
{स एव द्रोणहन्ता ते दर्पं छेत्स्यति पार्षतः}


\fourlineindentedshloka
{कालानलसमप्रख्यं द्विषामन्तकोपमम्}
{समासादय पाञ्चाल्यं मां चापि सहकेशवम्}
{दर्पं नाशयितास्म्यद्य तवोद्वृत्तस्य संयुगे ॥धृतराष्ट्र उवाच}
{}


\twolineshloka
{आचार्यपुत्रो मानार्हो बलवांश्चापि सञ्जय}
{प्रीतिर्धनञ्जये चास्य प्रियश्चापि महात्मनः}


\twolineshloka
{न भूतपूर्वं बीभत्सोर्वाक्यं परुषमीदृशम् ॥अथ कस्मात्स कौन्तेयः सखायं रूक्षमुक्तवान् ॥सञ्जय उवाच}
{}


\twolineshloka
{युवराजे हते चैव वृद्वक्षत्रे च पौरवे}
{इष्वस्त्रविधिसम्पन्ने मालवे च सुदर्शने}


\threelineshloka
{धृष्टद्युम्ने सात्यकौ च भीमे चापि पारजिते}
{युधिष्ठिरस्य तैर्वाक्यैर्मर्मण्यपि च घट्टिते}
{}


\twolineshloka
{अन्तर्भेदे च सञ्जाते दुःखं संस्मृत्य च प्रभो}
{अभूतपूर्वो बीभत्सोर्दुःखान्मन्युरजायत}


\twolineshloka
{तस्मादनर्हमश्लीलमप्रियं द्रोणिमुक्तवान्}
{मान्यमाचार्यतनयं रूक्षं कापुरुषं यथा}


\twolineshloka
{एवमुक्तः श्वसन्क्रोधान्महेष्वासतमो नृप}
{पार्थेन परुषं वाक्यं सर्वमर्मभिदा गिरा}


\twolineshloka
{द्रौणिश्चुकोप पार्थाय कृष्णाय च विशेषतः}
{स तु यत्तो रथे स्थित्वा वार्युपस्पृश्य वीर्यवान्}


\twolineshloka
{देवैरपि सुदुर्धर्षमस्त्रमाग्नेयमाददे}
{दृश्यादृश्यानरिगणानुद्दिश्याचार्यनन्दनः}


\twolineshloka
{सोऽभिमन्त्र्य शरं दीप्तं विधूममिव पावकम्}
{सर्वतः क्रोधमाविश्य चिक्षेप परवीरहा}


\twolineshloka
{ततस्तुमुलमाकाशे शरवर्षमजायत}
{पावकार्चिपरीतं तत्पार्थमेवाभिपुप्लुवे}


\twolineshloka
{उल्काश्च गगनात्पेतुर्दिशश्च न चकाशिरे}
{तमश्च सहसा रौद्रं चमूमवततार ताम्}


\twolineshloka
{रक्षांसि च पिशाचाश्च विनेदुरतिसङ्गताः}
{ववुश्चाशिशिरा वाताः सूर्यो नैव तताप च}


\twolineshloka
{वायसाश्चापि चाक्रन्दन्दिक्षु सर्वासु भैरवम्}
{रुधिरं चापि वर्षन्तो निनेदुस्तोयदा दिवि}


\twolineshloka
{पक्षिणः पशवो गावो विनेदुश्चापि सुव्रताः}
{परमं प्रयतात्मानो न शान्तिमुपलेभिरे}


\twolineshloka
{भ्रान्तसर्वमहाभूतमावर्तितदिवाकरम्}
{त्रैलोक्यमभिसन्तप्तं ज्वराविष्टमिवाभवत्}


\twolineshloka
{अस्त्रतेजोभिसन्तप्ता नागा भूमिशयास्तथा}
{निःश्वसन्तः समुत्पेतुस्तेजो घोरं मुमुक्षवः}


\twolineshloka
{जलजानि च सत्वानि दह्यमानानि भारत}
{न शान्तिमुपजग्मुर्हि तप्यमानैर्जलाशयैः}


\twolineshloka
{दिग्भ्यः प्रदिग्भ्यः खाद्भूमेः सर्वतः शरवृष्टयः}
{उच्चावचा निपेतुर्वै गरुडानिलरंहसः}


\twolineshloka
{तैः शरैर्द्रोणपुत्रस्य वज्रवेगैः समाहताः}
{प्रदग्धा रिपवः पेतुरग्निदग्धा इव द्रुमाः}


\twolineshloka
{दह्यमाना महानागाः पेतुरुर्व्यां समन्ततः}
{नदन्तो भैरवान्नादाञ्चलदोपमनिस्वनान्}


\twolineshloka
{अपरे प्रद्रुता नागा भयत्रस्ता विशाम्पते}
{त्रेसुर्दिशो यथा पूर्वं वने दावाग्निसंवृताः}


\threelineshloka
{द्रुमाणां शिखराणीव दावदग्धानि मारिष}
{अश्वबृन्दान्यदृश्यन्त रथबृन्दानि भारत}
{अपतन्त रथौघाश्च तत्रतत्र सहस्रशः}


\twolineshloka
{तत्सैन्यं भयसंविग्नं ददाह युधि भारत}
{युगान्ते सर्वभूतानि संवर्तक इवानलः}


\twolineshloka
{दृष्ट्वा तु पाण्डवीं सेनां दह्यमानां महाहवे}
{प्रहृष्टास्तावका राजन्सिंहनादान्विनेदिरे}


\twolineshloka
{ततस्तूर्यसहस्राणि नानालिङ्गानि भारत}
{तूर्णमाजघ्निरे हृष्टास्तावका जितकाशिनः}


\twolineshloka
{कृत्स्ना ह्यक्षौहिणी राजन्सव्यसाची च पाण्डवः}
{तमसा संवृते लोके नादृश्यन्त महाहवे}


\twolineshloka
{नैव नस्तादृशं राजन्दृष्टपूर्वं न च श्रुतम्}
{यादृशं द्रोणपुत्रेण सृष्टमस्त्रममर्षिणा}


\twolineshloka
{अर्जुनस्तु महाराज ब्राह्ममस्त्रमुदैरयत्}
{सर्वास्त्रप्रतिघातार्थं विहितं पद्मयोनिना}


\twolineshloka
{ततो मुहूर्तादिव तत्तमो व्युपशशाम ह}
{प्रववौ चानिलः शीतो दिशश्च विमला बभुः}


\twolineshloka
{तत्राद्भुतमपश्याम कृत्स्नामक्षौहिणीं हताम्}
{अनभिज्ञेयरूपां च प्रदग्धामस्त्रतेजसा}


\threelineshloka
{ततो वीरौ महेष्वासौ विमुक्तौ केशवार्जुनौ}
{सहितौ प्रत्यदृश्येतां नभसीव तमोनुदौ}
{ततो गण्डीवधनन्वा च केशवश्चाक्षतावुभौ}


\twolineshloka
{सपताकध्वजहयः सानुकर्षवरायुधः}
{प्रबभौ सरथो युक्तस्तावकानां भयङ्करः}


\twolineshloka
{ततः किलकिलाशब्दः शङ्खभेरीस्वनैः सह}
{पाण्डवानां प्रहृष्टानां क्षणेन समजायत}


\twolineshloka
{हताविति तयोरासीत्सेनयोरुभयोर्मतिः}
{तरसाऽभ्यागतौ दृष्ट्वा सहितौ कैशवार्जुनौ}


\twolineshloka
{तावक्षतौ प्रमुदितौ दध्मतुर्वारिजोत्तमौ}
{दृष्ट्वा प्रमुदितान्पार्थांस्त्वदीया व्यथिताऽभवन्}


\twolineshloka
{विमुक्तौ च महात्मानौ दृष्ट्वा द्रौणिः सुदुःखितः}
{मुहूर्तं चिन्तयामास किन्त्वेतदिति मारिष}


\threelineshloka
{चिन्तयित्वा तु राजेन्द्र ध्यानशोकपरायणः}
{निःश्वसन्दीर्घमुष्णं च विमनाश्चाभवत्ततः ॥सञ्जय उवाच}
{}


\twolineshloka
{ततो द्रौणिर्धनुस्त्यक्त्वा रथात्प्रस्कन्द्य वेगितः}
{धिग्धिक्सर्वमिदं मिथ्येत्युक्त्वा सम्प्राद्रवद्रणात्}


\twolineshloka
{ततः स्निग्धाम्बुदाभासं वेदावासमकल्मषम्}
{वेदव्यासं सरस्वत्यावासं व्यासं ददर्श ह}


\twolineshloka
{तं द्रौणिरग्रतो दृष्ट्वा स्थितं कुरुकुलोद्वहम्}
{सन्नकण्ठोऽब्रवीद्वाक्यमभिवाद्य सुदीनवत्}


\twolineshloka
{भोभो माया यदृच्छा वा दैवं वा किमिदं भवेत्}
{अस्त्रं त्विदं कथं मिथ्या मम कश्च व्यतिक्रमः}


\twolineshloka
{अधरोत्तरमेतद्वा लोकानां वा पराभवः}
{यदिमौ जीवतः कृष्णौ कालो हि दुरतिक्रमः}


\twolineshloka
{नासुरा न च गन्धर्वा न पिशाचा न राक्षसाः}
{न सर्पा यक्षपतगा न मनुष्याः कथञ्चन}


\twolineshloka
{उत्सहन्तेऽन्यथा कर्तुमेतदस्त्रं मयेरितम्}
{तदिदं केवलं हत्वा शान्तमक्षौहिणीं ज्वलत्}


\twolineshloka
{सर्वघानि मया मुक्तमस्त्रं परमदारुणम्}
{केनेमौ मर्त्यधर्माणौ नाधाक्षीत्केशवार्जुनौ}


\threelineshloka
{एतत्प्रब्रूहि भगवन्मया पृष्टो यथातथम्}
{श्रोतुमिच्छामि तत्त्वेन सर्वमेतन्महामुने ॥व्यास उवाच}
{}


\twolineshloka
{महान्तमेवमर्थं मां यं त्वं पृच्छसि विस्मयात्}
{तं प्रवक्ष्यामि ते सर्वं समाधाय मनः शृणु}


\twolineshloka
{योऽसौ नारायणो नाम पूर्वेषामपि पूर्वजः}
{`आदिदेवो जगन्नाथो लोककर्ता स्वयं प्रभुः}


\twolineshloka
{आद्यः सर्वस्य लोकस्य अनादिनिधनोऽच्युतः}
{व्याकुर्वते यस्य तत्त्वं श्रुतयो मुनयश्च ह}


\twolineshloka
{अतोऽजय्यः सर्वभूतैर्मनसाऽपि जगत्पतिः}
{तस्मादिमं जेतुकाम अज्ञानतमसा वृतः}


\twolineshloka
{मा शुचः पुरुषव्याघ्र विद्वि तद्वदिहार्जुनम्}
{तस्य शक्तिरसौ पार्थास्तस्माच्छोकमिमं त्यज}


\twolineshloka
{विश्वेश्वरोऽयं लोकादिः परमात्मा ह्यधोक्षजः}
{सहस्रसंहितादंशादेकांशोऽयमजायत}


\twolineshloka
{देवानां हितकामार्थं लोकानां चैव सत्तमः'}
{अजायत च कार्यार्थं पुत्रो धर्मस्य विश्वकृत्}


\twolineshloka
{स तपस्तीव्रमातस्थे शिशिरं गिरिमास्थितः}
{ऊर्ध्वबाहुर्महातेजा ज्वलनादित्यसन्निभः}


\twolineshloka
{षष्टिं वर्षसहस्राणि तावन्त्येव शतानि च}
{अशोषयत्तदाऽऽत्मानं वायुभक्षोऽम्बुजेक्षणः}


\twolineshloka
{अथापरं तपस्तप्त्वा द्विस्ततोऽन्यत्पुनर्महत्}
{द्यावापृथिव्योर्विवरं तेजसा समपूरयत्}


\threelineshloka
{स तेन तपसा तात ब्रह्मभूतो यदाऽभवत्}
{ततो विश्वेश्वरं योनिं विश्वस्य जगतः पतिम्}
{ददर्श भृशदुर्धर्षं सर्वदेवैरभिष्टुतम्}


\twolineshloka
{अणीयांसमणुभ्यश्च बृहद्भ्यश्च बृहत्तमम्}
{रुद्रमीशानवृषभं हरं शंभुं कपर्दिनम्}


\threelineshloka
{चेकितानं परां योनिं तिष्ठतो गच्छतश्च ह}
{दुर्वारणं सुदुर्धर्षं दुर्निरीक्ष्यं दुरासदम्}
{अतिमन्युं महात्मानं सर्वभूतप्रचेतसम्}


\twolineshloka
{दिव्यं चापमिषुधी चाददानंहिरण्यवर्माणमनन्तवीर्यम्}
{पिनाकिनं वज्रिणं दीप्तशूलंपरश्वथिं गदिनं चायतासिम्}


\threelineshloka
{बभ्रूजटामण्डलचन्द्रमौलिंव्याघ्राजिनं परिघिणं दण्डपाणिम्}
{शुभाङ्गदं नागयज्ञोपवीतंविश्वैर्गणैः शोभितं भूतसङ्घैः}
{एकीभूतं तपसां सन्निधानंवचोधिकैः सुष्टुतमिष्टवाग्भिः}


\twolineshloka
{जलं दिशं खं क्षितिं चन्द्रसूर्यौहुताशवायुप्रतिमं सुरेशम्}
{नालं द्रष्टुं यं जना भिन्नवृत्ताब्रह्मद्विषघ्नममृतस्य योनिम्}


\twolineshloka
{यं पश्यन्ति ब्राह्मणाः साधुवृत्ताःक्षीणे पापे मनसा वीतशोकाः}
{तन्निष्ठात्मा तपसा धर्ममीड्यंतद्भक्त्या वै विश्वरूपं ददर्श}


% Check verse!
दृष्ट्वा चैनं वाड्मनोबुद्धिदेहैःसंहृष्टात्मा मुमुदे वासुदेवः
\twolineshloka
{अक्षमालापरिक्षिप्तं ज्योतिषां परमं निधिम्}
{रुद्रं नारायणो दृष्ट्वा ववन्दे विश्वसम्भवम्}


\twolineshloka
{वरदं पृथुचार्वङ्ग्या पार्वत्या सहितं प्रभुम्}
{क्रीडमानं महात्मानं भूतसङ्घगणैर्वृतम्}


\fourlineindentedshloka
{अजमीशानमव्यक्तं कारणात्मानमच्युतम्}
{अभिवाद्याथ रुद्राय सद्योऽन्धकनिपातिने}
{पद्माक्षस्तं विरुपाक्षमभितुष्टाव भक्तिमान् ॥श्रीनारायण उवाच}
{}


\threelineshloka
{त्वत्सम्भूता भूतकृतो वरेण्यगोसारोऽस्य भुवनस्यादिदेव}
{आविश्येमां धरणीं येऽभ्यरक्षन्पुरा पुराणीं तव देव सृष्टिम् ॥सुरासुरान्नागरक्षःपिशाचा--न्नरान्सुपर्णानथ गन्धर्वयक्षान्}
{}


% Check verse!
पृथग्विधान्भूतसङ्घांश्च विश्वां--स्त्वत्सम्भूतान्विद्म सर्वांस्तथैव
% Check verse!
ऐन्द्रं याम्यं वारुणं वैत्तपाल्यंपैत्रं त्वाष्ट्रं कर्म सौम्यं च तुभ्यम्
\twolineshloka
{रूपं ज्योतिः शब्द आकाशवायुःस्पर्शः स्वाद्यं सलिलं गन्ध उर्वी}
{कालो ब्रह्मा ब्रह्म च ब्राह्मणाश्चत्वत्सम्भूताः स्थास्नु चरिष्णु चेदम्}


\twolineshloka
{अद्भ्यः स्तोका यान्ति यथा पृथक्त्वंताथिश्चैक्यं संक्षये यान्ति भूयः}
{एवं विद्वान्प्रभवं चाप्ययं चत्वसम्भूतं तव सायुज्यमेति}


\threelineshloka
{दिव्यामृतौ मानसौ द्वौ सुपर्णा--ववाक्शाखः पिप्पलः सप्तगोप्तः}
{वाचश्चार्था देवताः लोकपालालोकानन्ये ये पुरा धारयित्वा}
{गता हि तेभ्यः परमं तत्परं चत्वत्सम्भूतास्ते च तेभ्यः परस्त्वम्}


\twolineshloka
{भूतं भव्यं भविता चाप्रधृष्यंत्वत्सम्भूता भुवनानीह विश्वा}
{भक्तं च मां भजमानं भजस्वप्रीत्या मां वै लोकपितामहेश}


\fourlineindentedshloka
{आत्मानं त्वमात्मनो न व्यबोधंविद्वानेवं गच्छि ब्रह्म शुक्रम्}
{अस्तौषं त्वां तव संमानमिच्छ--निचिन्वन्वै सदृशं देववर्य}
{सुदुर्लभान्देहि परान्ममेष्टा--नभिष्टुतः प्रविकार्षीश्च मायाम् ॥व्यास उवाच}
{}


\threelineshloka
{तस्मै वरानचिन्त्यात्मा देवदेवः पिनाकधृत्}
{विष्णवे देवमुख्याय प्रायच्छदृषिसंस्तुतः ॥श्रीभगवानुवाच}
{}


\twolineshloka
{मत्प्रसादान्मनुष्येषु देवगन्धर्वयोनिषु}
{अप्रमेयबलात्मा त्वं नारायण भविष्यसि}


\twolineshloka
{न च त्वां प्रसहिष्यन्ति देवासुरमहोरगाः}
{न पिशाचा न गन्धर्वा न यक्षा न च राक्षसाः}


\twolineshloka
{न सुपर्णास्तथा नागा न च विश्वे वियोनिजाः}
{न कश्चित्त्वां च देवोऽपि समरेषु विजेष्यति}


\twolineshloka
{न शस्त्रेण न वज्रेण नाग्निना न च वारिणा}
{न चार्द्रेण न शुष्केण त्रसेन स्थावरेण च}


\threelineshloka
{न हस्तेन न पादेन न काष्ठेन न लोष्टुना}
{कश्चित्तव रुजां कर्ता मत्प्रसादात्कथञ्चन}
{अपि वै समरं गत्वा भविष्यसि ममाधिकः}


\twolineshloka
{एवमादीन्वराँल्लब्ध्वा प्रसादादथ शूलिनः}
{स एष देवश्चरति मायया मोहयञ्जगत्}


\twolineshloka
{तस्यैव तपसा जातो नरो नाम महामुनिः}
{तुल्यमेतेन देवेन तं जानीह्यर्जुनं सदा}


\threelineshloka
{तावेतौ पूर्वदेवानां परमोपचितादृषी}
{लोकयात्राविधानार्थं दानवानां वधाय च}
{धर्मसंस्थापनार्थाय सञ्जायेते युगेयुगे}


\twolineshloka
{तथैव कर्मणा कृत्स्नं महतस्तपरसोऽपि च}
{तेजो वाग्भिश्च विद्वंस्त्वं जातो रुद्रान्महामते}


\twolineshloka
{स भवान्देववत्प्राज्ञो ज्ञात्वा भवमयं जगत्}
{अवाकर्षस्त्वमात्मानं नियमैस्तत्प्रियेप्सया}


\twolineshloka
{शुभ्रमत्र हविः कृत्वा महापुरुषविग्रहम्}
{ईजिवांस्त्वं जपैर्होमैरुपहारैश्च मानद}


\twolineshloka
{स तथा पूज्यमानस्ते पूर्वदेवोऽप्यतूतुषत्}
{पुष्कलांश्च वरान्प्रादात्तव विद्वन्हृदि स्थितान्}


\threelineshloka
{जन्मकर्मतपोयोगास्तयोस्तव च पुष्कलाः}
{ताभ्यां लिङ्गेऽर्चितो देवस्त्वयाऽर्चायां युगेयुगे}
{देवदेवस्त्वचिन्त्यात्मा अजेयो विष्णुसम्भवः}


\twolineshloka
{सर्वरूपं भवं ज्ञात्वा लिङ्गे योऽर्चयति प्रभुम्}
{आत्मयोगाश्च तस्मिन्वै शास्त्रयोगाश्च शाश्वताः}


\twolineshloka
{एवं देवा यजन्तो हि सिद्धाश्च परमर्षयः}
{प्रार्थयन्ते परं लोके स्थाणुमेव च शाश्वतम्}


\twolineshloka
{[स एष* रुद्रभक्तश्च केशवो रुद्रसम्भवः}
{]कृष्ण एव हि यष्टव्यो यज्ञैश्चैव सनातनः}


\threelineshloka
{सर्वभूतभवं ज्ञात्वा लिङ्गमर्चति यः प्रभोः}
{तस्मिन्नभ्यधिकां प्रीतिं करोति वृषभध्वजः ॥सञ्जय उवाच}
{}


\twolineshloka
{तस्य तद्वचनं श्रुत्वा द्रोणपुत्रो महारथः}
{नमश्चकार रुद्राय बहुमेने च केशवम्}


\twolineshloka
{हष्टरोमा च वश्यात्मा सोऽभिवाद्य महर्षये}
{वरूथिनीमभिप्रेक्ष्य ह्यवहारमकारयत्}


\twolineshloka
{ततः प्रत्यवहारोऽभूत्पाण्डवानां विशाम्पते}
{कौरवाणां च दीनानां द्रोणे युधि निपातिते}


\twolineshloka
{युद्धं कृत्वा दिनान्पञ्च द्रोणो हत्वा वरूथिनीम्}
{ब्रह्मलोकं गतो राजन्ब्राह्मणो वेदपारगः}


\chapter{अध्यायः २०३}
\twolineshloka
{धृतराष्ट्र उवाच}
{}


\threelineshloka
{तस्मिन्नतिरथे द्रोणे निहते तत्र सञ्जय}
{मामकाः पाण्डवाश्चैव किमकुर्वन्नतः परम् ॥सञ्जय उवाच}
{}


\twolineshloka
{तस्मिन्नतिरथे द्रोणे निहते पार्षतेन वै}
{कौरवेषु च भग्नेषु कुन्तीपुत्रो धनञ्जयः}


\fourlineindentedshloka
{दृष्ट्वा सुमहदाश्चर्यमात्मनो विजयावहम्}
{मुनिं स्निग्धाम्बुदाभासं वेदव्यासमकल्मषम्}
{यदृच्छयाऽऽगतं व्यासं पप्रच्छ भरतर्षभ ॥अर्जुन उवाच}
{}


\twolineshloka
{सङ्ग्रामे न्यहनं शत्रूञ्शरौघैर्विमलैरहम्}
{अग्रतो लक्षये यान्तं पुरुषं पावकप्रभम्}


\twolineshloka
{ज्वलन्तं शूलमुद्यम्य यां दिशं प्रतिपद्यते}
{तस्यां दिशि विदीर्यन्ते शत्रवो मे महामुने}


\twolineshloka
{तेन भग्नानरीन्सर्वान्मद्भग्नान्मन्यते जनः}
{तेन भग्नानि सैन्यानि पृष्ठतोऽनुव्रजाम्यहम्}


\twolineshloka
{भगवंस्तन्ममाचक्ष्व को वै स पुरुषोत्तमः}
{शूलपाणिर्मया दृष्टस्तेजसा सूर्यसन्निभः}


\threelineshloka
{न पद्भ्यां स्पृशते भूमिं न च शूलं विमुञ्चति}
{शूलाच्छूलसहस्राणि निष्पेतुस्तस्य तेजसा ॥व्यास उवाच}
{}


\twolineshloka
{प्रजापतीनां प्रथमं तैजसं पुरुषं प्रभुम्}
{भुवनं भूर्भुवं देवं तेजसां प्रवरं प्रभुम्}


\twolineshloka
{ईशानं वरदं पार्थ दृष्टवानसि शङ्करम्}
{तं गच्छ शरणं देवं वरदं भुवनेश्वरम्}


\twolineshloka
{महादेवं महात्मानमीशानं जटिलं विभुम्}
{त्र्यक्षं महाभुजं रुद्रं शिखिनं चीवराससम्}


\twolineshloka
{महादेवं हरं स्थाणुं वरदं भुवनेश्वरम्}
{जगत्प्रधानमजितं जगत्प्रीतिमधीश्वरम्}


\twolineshloka
{जगद्योनिं जगद्बीजं जयिनं जगतो गतिम्}
{विश्वात्मानं विश्वसृजं विश्वमूर्तिं यशस्विनम्}


\twolineshloka
{विश्वेश्वरं विश्वनरं कर्मणामीश्वरं प्रभुम्}
{शंभुं स्वयंभुं भूतेशं भूतभव्यभवोद्भवम्}


\twolineshloka
{योगं योगेश्वरं सर्वं सर्वलोकेश्वरेश्वरम्}
{सर्वश्रेष्ठं जगच्छ्रेष्ठं वरिष्ठं परमेष्ठिनम्}


\twolineshloka
{लोकत्रयविधातारमेकं लोकत्रयाश्रयम्}
{शुद्धात्मानं भवं भीमं शशाङ्ककृतशेखरम्}


\twolineshloka
{शाश्वतं भूधरं देवं सर्ववागीश्वरेश्वरम्}
{सुदुर्जयं जगन्नाथं जन्ममृत्युजरातिगम्}


\twolineshloka
{ज्ञानात्मानं ज्ञानगम्यं ज्ञानश्रेष्ठं सुदुर्विदम्}
{दातारं चैव भक्तानां प्रसादविहितान्वरान्}


\twolineshloka
{तस्य पारिषदा दिव्या रूपैर्नानाविधैर्विभोः}
{वामना जटिला मुण्डा हस्वग्रीवा महोदराः}


\twolineshloka
{महाकाया महोत्साहा महाकर्णास्तथाऽपरे}
{आननैर्विकृतैः पादैः पार्थ वेषैश्च वैकृतैः}


\twolineshloka
{ईदृशैः स महादेवः पूज्यमानो महेश्वरः}
{स शिवस्तात तेजस्वी प्रसादाद्याति तेऽग्रतः}


\twolineshloka
{तस्मिन्घोरे सदा पार्थ सङ्ग्रामे रोमहर्षणे}
{द्रौणिकर्णकृपैर्गुप्तां महेष्वासैः प्रहारिभिः}


\twolineshloka
{कस्तां सेनां तदा पार्थ मनसाऽपि प्रधर्षयेत्}
{ऋते देवान्महेष्वासाद्बहुरूपान्महेश्वरात्}


\twolineshloka
{स्थातुमुत्सहते कश्चिन्न तस्मिन्नग्रतः स्थिते}
{न हि भूतं समं तेन त्रिषु लोकेषु विद्यते}


\twolineshloka
{गन्धेनापि हि सङ्ग्रामे तस्य क्रुद्धस्य शत्रवः}
{विसंज्ञा हतभूयिष्ठा वेपन्ति च पतन्ति च}


\twolineshloka
{तस्मै नमस्तु कुर्वन्तो देवास्तिष्ठन्ति वै दिवि}
{ये चान्ये मानवा लोके ये च स्वर्गजितो नराः}


\twolineshloka
{ये भक्ता वरदं देव शिवं रुद्रमुमापतिम्}
{अनन्यभावेन सदा सर्वेशं समुपासते}


\twolineshloka
{`सङ्ग्रामेषु जयं प्राप्य पालयन्ति महीमिमाम्'}
{इह लोके सुखं प्राप्य ते यान्ति परमां गतिम्}


\twolineshloka
{नमस्कुरुष्व कौन्तेय तस्मै शान्ताय वै सदा}
{रुद्राय शितिकण्ठाय कनिष्ठाय सुवर्चसे}


\twolineshloka
{कपर्दिने करालाय हर्यक्षवरदाय च}
{याम्यायाव्यक्तकेशाय सद्वृत्ते शङ्कराय च}


% Check verse!
काम्याय हरिनेत्राय स्थाणवे पुरुषाय च
\threelineshloka
{`नमो वृक्षाय सेनान्ये मध्यमाय नमोनमः'}
{हरिकेशाय मुण्डाय कृशायोत्तारणाय च}
{भास्कारय सुतीर्थाय देवदेवाय रंहसे}


\twolineshloka
{बहुरूपाय सर्वाय प्रियाय प्रियवाससे}
{उष्णीषिणे सुवक्त्राय सहस्राक्षाय मीढुषे}


\twolineshloka
{गिरिशाय प्रशान्ताय पतये चीरवाससे}
{हिरण्यबाहवे राजन्नुग्राय पतये दिशाम्}


\twolineshloka
{पर्जन्यपतये चैव भूतानां पतये नमः}
{वृक्षाणां पतये चैव गवां च पतये नमः}


\twolineshloka
{वृक्षैरावृतकायाय सेनान्ये मध्यमाय च}
{स्रुवहस्ताय देवाय धन्विने भार्गवाय च}


\threelineshloka
{बहुरूपाय विश्वस्य पतये मुञ्चवाससे}
{सहस्रशिरसे चैव सहस्रचरणाय च}
{सहस्रबाहवे चैव सहस्रवदनाय च}


\twolineshloka
{शरणं गच्छ कौन्तेय वरदं भुवनेश्वरम्}
{उमापतिं विरूपाक्षं दक्षयज्ञनिबर्हणम्}


\twolineshloka
{प्रजानां पतिमव्यग्रं भूतानां पतिमव्ययम्}
{कपर्दिनं वृषावर्तं वृषनाभं वृषध्वजम्}


\twolineshloka
{वृषदर्पं वृषपतिं वृषशृङ्गं वृषर्षभम्}
{वृषाङ्गं वृषभोदारं वृषभं वृषभेक्षणम्}


\twolineshloka
{वृषायुधं वृषशरं वृषभूतं वृषेश्वरम्}
{महोदरं महाकायं द्वीपिचर्मनिवासिनम्}


\twolineshloka
{लोकेशं वरदं पुण्यं ब्रह्मण्यं ब्राह्मणप्रियम्}
{त्रिशूलपाणिं वरदं खङ्गचर्मधरं प्रभुम्}


\twolineshloka
{पिनाकिनं खङ्घधरं लोकानां पतिमीश्वरम्}
{प्रपद्ये शरणं देवं शरण्यं चीरवाससम्}


\twolineshloka
{नमस्तस्मै सुरेशाय गणानां पतये नमः}
{सुवाससे नमस्तुभ्यं सुव्रताय सुधन्विने}


\twolineshloka
{धनुर्धराय देवाय प्रियधन्वाय धन्विने}
{धन्वन्तराय धनुषे धन्वाचार्याय ते नमः}


\twolineshloka
{उग्रायुधाय देवाय नमः सुरवराय च}
{नमोस्तु बहुरूपाय नमोस्तु बहुधन्विने}


\threelineshloka
{नमोस्तु स्थाणवे नित्यं नमस्तस्मै तपस्विने}
{नमोस्तु त्रिपुरघ्नाय भगघ्नाय च वै नमः}
{}


\twolineshloka
{वनस्पतीनां पतये नराणां पतये नमः}
{मातॄणां पतये चैव गणानां पतये नमः}


\twolineshloka
{गवां च पतये नित्यं यज्ञानां पतये नमः}
{अपां च पतये नित्यं देवानां पतये नमः}


\twolineshloka
{पूष्णो दन्तविनाशाय त्र्यक्षाय वरदाय च}
{नीलकण्ठाय पिङ्गाय स्वर्णकेशाय वै नमः}


\twolineshloka
{कर्माणि यानि दिव्यानि महादेवस्य धीमतः}
{तानि ते कीर्तयिष्यामि तथाप्रज्ञं यथाश्रुतम्}


\twolineshloka
{न सुरा नासुरा लोके न गन्धर्वा न राक्षसाः}
{सुखमेधन्ति कुपिते तस्मिन्नपि गुरागताः}


\threelineshloka
{दक्षस्य यजमानस्य विधिवत्संभृतं पुरा}
{विव्याध कुपितो यज्ञं निर्दयस्त्वभवत्तदा}
{धनुषा बाणमुत्सृज्य सघोषं विननाद च}


\twolineshloka
{तेन शर्म कुतः शान्तिं लेभिरे स्म सुरास्तदा}
{विद्रुते सहसा यज्ञे कुपिते च महेश्वरे}


\twolineshloka
{तेन ज्यातलघोषेण सर्वे लोकाः समाकुलाः}
{बभूवुर्वशगाः पार्थ निपेतुश्च सुरासुराः}


\twolineshloka
{आपश्चुक्षुभिरे सर्वाश्चकम्पे च वसुन्धरा}
{पर्वताश्च व्यशीर्यन्त दिशो नागाश्च मोहिताः}


\twolineshloka
{अन्धेन तमसा लोका न प्राकाशन्त संवृताः}
{जघ्निवान्सह सूर्येण सर्वेषां ज्योतिषां प्रभाः}


\twolineshloka
{चुक्षुभुर्भयभीताश्च शान्तिं चक्रुस्तथैव च}
{ऋषयः सर्वभूतानामात्मनश्च सुखैषिणः}


\twolineshloka
{पूषाणमभ्यद्रवत शङ्करः प्रहसन्निव}
{पुरोडाशं भक्षयतो दशनान्वै व्यशातयत्}


% Check verse!
ततो निश्चक्रमुर्देवा वेपमाना भयार्दिताः
\twolineshloka
{पुनश्च सन्दधे दीप्तान्देवानां निशिताञ्शरान्}
{सधूमान्सस्फुलिङ्गांश्च विद्युत्तोयदसन्निभान्}


\twolineshloka
{तं दृष्ट्वा तु सुराः सर्वे प्रणिपत्य महेश्वरम्}
{रुद्रस्य यज्ञभागं च विशिष्टं ते त्वकल्पयन्}


\threelineshloka
{भयेन त्रिदशा राजञ्छरणं च प्रपेदिरे}
{तेन चैवातिकोपेन स यज्ञः सन्धितस्तदा}
{भग्नाश्चापि सुरा आसन्भीताश्चाद्यापि तं प्रति}


\twolineshloka
{असुराणां पुराण्यासंस्त्रीणि वीर्यवतां दिवि}
{आयसं राजतं चैव सौवर्णं परमं महत्}


\threelineshloka
{सौवर्णं कमलाक्षस्य तारकाक्षस्य राजतम्}
{तृतीयं तु पुरं तेषां विद्युन्मालिन आयसम्}
{न शक्तस्तानि मघवान्भेत्तुं सर्वायुधैरपि}


\twolineshloka
{अथ सर्वे सुरा रुद्रं जग्मुः शरणमर्दिताः}
{ते तमूचुर्महात्मानं सर्वे देवाः सवासवाः}


\twolineshloka
{ब्रह्मदत्तवरा ह्येते घोरास्त्रिपुरवासिनः}
{पीडयन्त्यधिकं लोकं यस्मात्ते वरदर्पिताः}


\twolineshloka
{त्वदृते देवदेवेश नान्यः शक्तः कथञ्चन}
{हन्तुं दैत्यान्महादेव जहि तांस्त्वं सुरद्विपः}


\twolineshloka
{रुद्र रौद्रा भविष्यन्ति पशवः सर्वकर्मसु}
{निपातयिष्यसे चैतानसुरान्भुवनेश्वर}


\twolineshloka
{स तथोक्तस्तथेत्युक्त्वा देवानां हितकाम्यया}
{गन्धमादनविन्ध्यौ च कृत्वा वंशध्वजौ हरः}


\twolineshloka
{पृथ्वीं ससागरवनां रथं कृत्वा तु शङ्करः}
{अक्षं कृत्वा तु नागेन्द्रं शेषं नाम त्रिलोचनः}


\twolineshloka
{चक्रे कृत्वा तु चन्द्रार्कौ देवदेवः पिनाकधृत्}
{अणी कृत्वैलपत्रं च पुष्पदन्तं च त्र्यम्बकः}


% Check verse!
यूपं कृत्वा तु मलयमवनाहं च तक्षकम्
\twolineshloka
{योक्त्राङ्गानि च सत्वानि कृत्वा शर्वः प्रतापवान्}
{वेदान्कृत्वाऽथ चतुरश्चतुरश्वान्महेश्वरः}


\twolineshloka
{उपदेवान्खलीनांश्च कृत्वा लोकत्रयेश्वरः}
{गायत्रीं प्रग्रहं कृत्वा सावित्रीं च महेश्वरः}


\twolineshloka
{कृत्वोंकारं प्रतोदं च ब्रह्माणं चैव सारथिम्}
{गाण्डीवं मन्दरं कृत्वा गुणं कृत्वा तु वासुकिम्}


\twolineshloka
{विष्णुं शरोत्तमं कृत्वा शल्यमग्निं तथैव च}
{वायुं कृत्वाऽथ वाजाभ्यां पुङ्खे वैवस्वतं यमम्}


\twolineshloka
{विद्युत्कृत्वाऽथ निश्राणं मेरुं कृत्वाऽथ वै ध्वजम्}
{आरुह्य स रथं दिव्यं सर्वदेवमयं शिवः}


\twolineshloka
{त्रिपुरस्य वधार्थाय स्थाणुः प्रहरतां वरः}
{असुराणामन्तकरः श्रीमानतुलविक्रमः}


\threelineshloka
{स्तूयमानः सुरैः पार्थ ऋषिभिश्च तपोधनैः}
{स्थानं माहेश्वरं कृत्वा दिव्यमप्रतिमं प्रभुः}
{अतिष्ठत्स्थाणुभूतः स सहस्रं परिवत्सरान्}


\twolineshloka
{यदा त्रीणि समेतानि अन्तरिक्षे पुराणि च}
{त्रिपर्वणा त्रिशल्येन तदा तानि बिभेद सः}


\twolineshloka
{पुराणि न च तं शेकुर्दानवाः प्रतिवीक्षितुम्}
{शरं कालाग्निसंयुक्तं विष्णुसोमसमायुतम्}


\twolineshloka
{पुराणि दग्धवन्तं तं देवी याता प्रवीक्षितुम्}
{`देव्याः स्वयंवरे वृत्तं शृणुष्वान्यद्धनञ्जय'}


\twolineshloka
{बालमङ्कगतं कृत्वा स्वयं पञ्चशिखं पुनः}
{उमा जिज्ञासमाना वै कोयमित्यब्रवीत्सुरान्}


\threelineshloka
{असूयतश्च शक्रस्य वज्रेण प्रहरिष्यतः}
{बाहुं सवज्रं तं तस्य क्रुद्धस्यास्तम्भयत्प्रभुः}
{[प्रहस्य* भगवांस्तूर्णं सर्वलोकेश्वरो विभुः}


\twolineshloka
{ततः स स्तम्भितभुजः शक्रो देवगणैर्वृतः}
{जगाम ससुरस्तूर्णं ब्रह्माणं प्रभुमव्ययम्}


% Check verse!
ते तं प्रणम्य शिरसा प्रोचुः प्राञ्जलयस्तदा
\twolineshloka
{किमप्यङ्कगतं ब्रह्मन्पार्वत्या भूतमद्भुतम्}
{बालरूपधरं दृष्ट्वा नास्माभिरभिलक्षितः}


\twolineshloka
{तस्मात्त्वां प्रष्टुमिच्छामो निर्जिता येन वै वयम्}
{अयुध्यता हि बालेन लीलया सपुरंदराः}


\threelineshloka
{तेषां तद्वचनं श्रुत्वा ब्रह्मा ब्रह्मविदां वरः}
{ध्यात्वा स शंभुं भगवान्बालं चामिततेजसम्}
{उवाच भगवान्ब्रह्मा शक्रादींश्च सुरोत्तमान्}


\twolineshloka
{चराचरस्य जगतः प्रभुः स भगवान्हरः}
{तस्मात्परतरं नान्यत्किञ्चिदस्ति महेश्वरात्}


\twolineshloka
{यो दृष्टो ह्युमया सार्धं युष्माभिरमितद्युतिः}
{स पार्वत्याः कृते शर्वः कृतवान्बालरूपताम्}


\twolineshloka
{ते मया सहिता यूयं प्रापद्यध्वं तमेव हि]}
{स एष भगवान्देवः सर्वलोकेश्वरः प्रभुः}


\twolineshloka
{न सम्बबुधिरे चैनं देवास्तं भुवनेश्वरम्}
{सप्रजापतयः सर्वे बालार्कसदृशप्रभम्}


\threelineshloka
{अथाभ्येत्य ततो ब्रह्मा दृष्ट्वा स च महेश्वरम्}
{अयं श्रेष्ठ इति ज्ञात्वा ववन्दे तं पितामहः ॥[ब्रह्मोवाच}
{}


\twolineshloka
{त्वं यज्ञो भुनस्यास्य त्वं गतिस्त्वं परायणम्}
{त्वं भवस्त्वं महादेवस्त्वं धाम परमं पदम्}


% Check verse!
त्वया सर्वमिदं व्याप्तं जगत्स्थावरजङ्गमम्
\threelineshloka
{भगवन्भूतभव्येश लोकनाथ जगत्पते}
{प्रसादं कुरु शक्रस्य त्वया क्रोधार्दितस्य वै ॥व्यास उवाच}
{}


% Check verse!
पद्मयोनिवचः श्रुत्वा ततः प्रीतो महेश्वरःप्रसादाभिमुखो भूत्वा अट्टहासमथाकरोत् ॥]
\twolineshloka
{ततः प्रसादयामासुरुमां रुद्रं च ते सुराः}
{अभवच्च पुनर्बाहुर्यथाप्रकृति वज्रिणः}


\twolineshloka
{तेषां प्रसन्नो भगवान्सपत्नीको वृषध्वजः}
{देवानां त्रिदशश्रेष्ठो दक्षयज्ञविनाशनः}


\twolineshloka
{स वै रुद्रः स च शिवः सोऽग्निः सर्वश्च सर्ववित्}
{स चेन्द्रश्चैव वायुश्च सोऽश्विनौ च स विद्युतः}


\twolineshloka
{स भवः स च पर्जन्यो महादेवः सनातनः}
{स चन्द्रमाः स चेशानः स सूर्यो वरुणश्च सः}


\twolineshloka
{स कालः सोऽन्तको मृत्युः स यमो रात्र्यहानि तु}
{मासार्धमासा ऋतवः सन्ध्ये संवत्सरश्च सः}


\twolineshloka
{धाता च स विधाता च विश्वात्मा विश्वकर्मकृत्}
{सर्वामां देवतानां च धारयत्यवपुर्वपुः}


\twolineshloka
{सर्वदेवैः स्तुतो देवः सैकधा बहुधा च सः}
{शतधा सहस्रधा चैव भूयः शतसहस्रधाः}


\twolineshloka
{द्वे तनू तस्य देवस्य वेदज्ञा ब्राह्मणा विदुः}
{घोरा चान्या शिवा चान्या ते तनू बहुधा पुनः}


\twolineshloka
{घोरा तु यातुधानस्य सोऽग्निर्विष्णुः स भास्करः}
{सौम्यातु पुनरेवास्य आपो ज्योतींषि चन्द्रमाः}


\twolineshloka
{वेदाः साङ्गोपनिषदः पुराणाध्यात्मनिश्चयाः}
{यदत्र परमं गुह्यं स वै देवो महेश्वरः}


\twolineshloka
{ईदृशश्च महादेवो भूयांश्च भगवानजः}
{न हि सर्वे मया शक्या वक्तुं भगवतो गुणाः}


\twolineshloka
{Numbering changed here, check in edition.अपि वर्षसहस्रेण सततं पाण़्डुनन्दन}
{सर्वैर्ग्रहैर्गृहीतान्वै सर्वपापसमन्वितान्}


\twolineshloka
{स मोचयति सुप्रीतः शरण्यः शरणागतान्}
{आयुरारोग्यमैश्वर्यं वित्तं कामांश्च पुष्कलान्}


\twolineshloka
{स ददाति मनुष्येभ्यः स चैवाक्षिपते पुनः}
{सेन्द्रादिषु च देवेषु तस्य चैश्वर्यमुच्यते}


\twolineshloka
{स चैव वेत्ति लोकेषु मनुष्याणां शुभाशुभे}
{ऐश्वर्याच्चैव कामानामीश्वरश्च स उच्यते}


\twolineshloka
{महेश्वरश्च महतां भूतानामीश्वरश्च सः}
{बहुभिर्बहुधा रूपैर्विश्वं व्याप्नोति वै जगत्}


\twolineshloka
{तस्य देवस्य यद्वक्त्रं समुद्रे तदधिष्ठितम्}
{ब़डबामुखेति विख्यातं पिबत्तोयमयं हविः}


\twolineshloka
{एष चैव श्मशानेषु देवो वसति नित्यशः}
{यजन्त्येनं जनास्तत्र वीरस्थान इतीश्वरम्}


\twolineshloka
{अस्य दीप्तानि रूपाणि घोराणि च बहूनि च}
{लोके यान्यस्य पूज्यन्ते मनुष्याः प्रवदन्ति च}


\twolineshloka
{नामधेयानि लोकेषु बहून्यस्य यथार्थवत्}
{निरुच्यन्ते महत्त्वाच्च विभुत्वात्कर्मणस्तथा}


\twolineshloka
{वेदे चास्य समाम्नातं शतरुद्रियमुत्तमम्}
{नाम्ना चानन्तरुद्रेति ह्युपस्थानं महात्मनः}


\twolineshloka
{स कामानां प्रभुर्देवो ये दिव्या ये च मानुषाः}
{स विभुः स प्रभुर्देवोविश्वं व्याप्नोति वै महत्}


\twolineshloka
{ज्येष्ठं भूतं वदन्त्येनं ब्राह्मणा मनुयस्तथा}
{प्रथमो ह्येष देवानां मुखादस्यानलोऽभवत्}


\twolineshloka
{सर्वथा यत्पशून्पाति तैश्च यद्रमते पुनः}
{तेषामधिपतिर्यच्च तस्मात्पशुपतिः स्मृतः}


\twolineshloka
{दिव्यं च ब्रह्मचर्येण लिङ्गमस्य यथास्थितम्}
{महयत्येष लोकांश्च महेश्वर इति स्मृतः}


\twolineshloka
{ऋषयश्चैव देवाश्च गन्धर्वाप्यरसस्तथा}
{लिङ्गमस्यार्चयन्ति स्म तच्चाप्यूर्ध्वं समास्थितम्}


\twolineshloka
{पूज्यमाने ततस्तस्मिन्मोदते स महेश्वरः}
{सुखी प्रीतश्च भवति प्रहृष्टश्चैव शङ्करः}


\twolineshloka
{यदस्य बहुधा रूपं भूतभव्यभवस्थितम्}
{स्थावरं जङ्गमं चैव बहुरूपस्ततः स्मृतः}


\twolineshloka
{एकाक्षो जाज्वलन्नास्ते सर्वतोक्षिमयोऽपि वा}
{क्रोधाद्यश्चाविशल्लोकांस्तस्मात्सर्व इति स्मृतः}


\twolineshloka
{धूम्ररूपं च यत्तस्य धूर्जटिस्तेन चोच्यते}
{विश्वे देवाश्च यत्तस्मिन्विश्वरूपस्ततः स्मृतः}


\twolineshloka
{तिस्रो देवीर्यदा चैव भजते भुवनेश्वरः}
{द्यामपः पृथिवीं चैव त्र्यम्बकश्च ततः स्मृतः}


\twolineshloka
{समेधयति यन्नित्यं सर्वार्थान्सर्वकर्मसु}
{शिवमिच्छन्मनुष्याणां तस्मादेष शिवः स्मृतः}


\twolineshloka
{सहस्राक्षोऽयुताक्षो वा सर्वतोक्षिमयोऽपि वा}
{यच्च विश्वं महत्पाति महादेवस्ततः स्मृतः}


\twolineshloka
{महत्पूर्वं स्थितो यच्च प्राणोत्पत्तिस्थितश्च यत्}
{स्थितलिङ्गश्च यन्नित्यं तस्मात्स्थाणुरिति स्मृतः}


\twolineshloka
{[सूर्याचन्द्रमसोर्लोके प्रकाशन्ते रुचश्च याः}
{ताः केशसंज्ञितास्त्र्यक्षे व्योमकेशस्ततः स्मृतः}


\twolineshloka
{भूतं भव्यं भविष्यं च सर्वं जगदशेषतः}
{भव एव ततो यस्माद्भूतभव्यभवोद्भवः}


\twolineshloka
{कपिः श्रेष्ठ इति प्रोक्तो धर्मश्च वृष उच्यते}
{स देवदेवो भगवान्कीर्त्यतेऽतो वृषाकपिः}


\twolineshloka
{ब्रह्माणमिन्द्रं वरुणं यमं धनदमेव च}
{निगृह्य हरते यस्मात्तस्माद्वर इति स्मृतः}


\twolineshloka
{निमीलिताभ्यां नेत्राभ्यां बलाद्देवो महेश्वरः}
{ललाटे नेत्रमसृजत्तेन त्र्यक्षः स उच्यते ॥]}


\twolineshloka
{विषमस्थः शरीरेषु समश्च प्राणिनामिह}
{स वायुर्विषमस्थेषु प्राणोऽपानः शरीरिषु}


\twolineshloka
{पूजयेद्विग्रहं यस्तु लिङ्गं चापि महात्मनः}
{लिङ्गं पूजयिता नित्यं महतीं श्रियमश्नुते}


\twolineshloka
{ऊरुभ्यामर्धमाग्नेयं सोमोऽर्धं च शिवा तनुः}
{आत्मनोऽर्धं तथा चाग्निः सोमोर्धं पुनरुच्यते}


\twolineshloka
{तैजसी महती दीप्ता देवेभ्योऽस्य शिवा तनुः}
{भास्वती मानुषेष्वस्य तनुर्घोराऽग्निरुच्यते}


\twolineshloka
{ब्रह्मचर्यं चरत्येष शिवा याऽस्य तनुस्तया}
{याऽस्य घोरतरा मूर्तिः सर्वानत्ति तयेश्वरः}


\twolineshloka
{यन्निर्दहति यत्तीक्ष्णो यदुग्रो यत्प्रतापवान्}
{मांसशोणितमज्जादो यत्ततो रुद्र उच्यते}


\twolineshloka
{एष देवो महादेवो योऽसौ पार्थ तवाग्रतः}
{सङ्ग्रामे शास्त्रवान्निघ्नंस्त्वया दृष्टः पिनाकधृत}


\twolineshloka
{सिन्धुराजवधार्थाय प्रतिज्ञाते त्वयाऽनघ}
{कृष्णेन दर्शितः स्वप्ने यस्तु शैलेन्द्रमूर्धनि}


\twolineshloka
{एष वै भगवान्देवः सङ्ग्रामे याति तेऽग्रतः}
{येन दत्तानि तेऽस्त्राणि यैस्त्वया दानवा हताः}


\threelineshloka
{धन्यं यशस्यमायुष्यं पुण्यं वेदैश्च सम्मितम्}
{देवदेवस्य ते पार्थ व्याख्यातं शतरुद्रियम्}
{}


\twolineshloka
{सर्वार्थसाधनं पुण्यं सर्वकिल्बिषनाशनम्}
{सर्वपापप्रशमनं सर्वदुःखभयापहम्}


\twolineshloka
{चतुर्विधमिदं स्तोत्रं यः शृणोति नरः सदा}
{विजित्य शत्रून्सर्वान्स रुद्रलोके महीयते}


\twolineshloka
{चरितं महात्मनो नित्यं साङ्ग्रामिकमिदं स्मृतम्}
{पठन्वै शतरुद्रीयं शृण्वंश्च सततोत्थितः}


\twolineshloka
{भक्तो विश्वेश्वरं देवं मानुषेषु च यः सदा}
{वरान्कामान्स लभते प्रसन्ने त्र्यम्बके नरः}


\threelineshloka
{गच्छ युध्यस्व कौन्तेय न तवास्ति पराजयः}
{गस्य मन्त्री च गोप्ता च पार्श्वस्थो हि जनार्दनः ॥सञ्जय उवाच}
{}


\threelineshloka
{एवमुक्त्वाऽर्जुनं सङ्ख्ये पराशरसुतस्तदा}
{जगाम भरतश्रेष्ठ यथागतमरिन्दम ॥`वेशंपायन उवाच}
{}


\twolineshloka
{एतदाख्याय वै सूतो राज्ञः सर्वं तु सञ्जयः}
{प्रयातः शिबिरायैव द्रष्टुं कर्णस्य वैशसम्}


\twolineshloka
{युद्धं कृत्वा महद्धोरं पञ्चाहानि महाबलः}
{ब्राह्मणो निहतो राजन्ब्रह्मलोकमवाप्तवान्}


\twolineshloka
{स्वधीते यत्फलं वेदे तदस्मिन्नपि पर्वणि}
{क्षत्रियाणामभीरूणां युक्तमत्र महद्यशः}


\twolineshloka
{य इदं पठते पर्व शृणुयाद्वाऽपि नित्यशः}
{स मुच्यते महापापैः कृतैर्घोरैश्च कर्मभिः}


\twolineshloka
{यज्ञावाप्तिर्ब्राह्मणस्येह नित्यंघोरे युद्धे क्षत्रियाणां यशश्च}
{शेषौ वर्णौ काममिष्टं लभेतेपुत्रान्पौत्रान्नित्यमिष्टांस्तथैव}


