\part{मौसलपर्व}
\chapter{अध्यायः १}
\threelineshloka
{श्रीवेदव्यासाय नमः}
{नारायणं नमस्कृत्य नरं चैव नरोत्तमम्}
{देवीं सरस्वतीं व्यासं ततो जयमुदीरयेत्}


\threelineshloka
{वैशम्पायन उवाच}
{षट्त्रिंशे त्वथ् संप्राप्ते वर्षे कौरवन्दनः}
{ददर्श विपरीतानि निमित्तानि युधिष्टिरः}


\twolineshloka
{ववुर्वाताश्च निर्घाता रूक्षाः शर्करवर्षिणः}
{अपसव्यानि शकुना मण्डलानि प्रचक्रिरे}


\twolineshloka
{प्रत्यगूहुर्महानद्यो दिशो नीहारसंवृताः}
{उल्काश्चाङ्गारवर्षिण्यः प्रापतन्गगनाद्भुवि}


% Check verse!
आदित्यो रजसा राजन्समवच्छन्नगनाद्भुवि ॥विरश्मिरुदये नित्यं कबन्धैः समलङ्कृतः
\twolineshloka
{परिवेषाश्च दृश्यन्ते दारुणाश्च द्रसूर्ययोः}
{त्रिवर्णाः श्यामरुक्षान्तास्तथा भस्मारुणप्रभाः}


\twolineshloka
{एते चान्ये च बहव उत्पाता भयसंसिनः}
{दृश्यन्ते जगतो नाथ दिवमारोढुमिच्छति}


\twolineshloka
{`यस्य प्रसादाद्धर्मोऽयं कृते यद्वत्कलावपि}
{पाण्डवाश्च महाभाग युक्तास्तु यशसाऽनघ ॥'}


\twolineshloka
{कस्य चित्त्वथ कालस्य कुरुराजो विशङ्कितः}
{शुश्राव वृष्णिचक्रस्य मुसलोत्सादनं कृतम्}


\twolineshloka
{विमुक्तं वासुदेवं च श्रुत्वा रामं च पाण्डवः}
{समानीयाब्रवीद्भ्रातृन्कि करिष्याम इत्युत}


\twolineshloka
{परस्परं समासाद्य ब्रह्मदण्डबलात्कृतान्}
{वृष्णिन्विनष्टास्ते श्रुत्वा व्यथिताः पाण्डवा भृशम्}


\twolineshloka
{निधनं वासुदेवस्य समुद्रस्येव शोषणम्}
{वीरा न श्रद्दधुस्तस्य विनाशं शार्ङ्गधन्वनः}


\twolineshloka
{मौसलं ते समाश्रित्य दुःखशोकसमन्विताः}
{विष्ण्णा हतसंकल्पाःकि पाण्डवाः समुपाविशन्}


\chapter{अध्यायः २}
\threelineshloka
{कथं विनष्टा भगवन्नन्धका वृष्णिभिः सह}
{पश्यतो वासुदेवस्य भोजश्चैवक महारथाः ॥वैशम्पायन उवाच}
{}


\threelineshloka
{षट्त्रिंशेऽथ ततो वर्षे वृष्णीनामनयो महान्}
{अन्योन्यं मुसलैस्ते तु निजघ्नुः कालचोदिताः ॥जनमेजय उवाच}
{}


\threelineshloka
{केनानुशप्तास्ते वीराः क्षयं वृष्ण्यन्धका गताः}
{भोजाश्च द्विजवर्य त्वं विस्तरेण वदस्व मे ॥वैशम्पायन उवाच}
{}


\twolineshloka
{विश्वामित्रं च कण्वं च नारदं च तपोधनम्}
{सारणप्रमुखा वीरा ददृशुर्द्वारकां गतान्}


\twolineshloka
{ते वै सांबं पुरस्कृत्य भूषयित्वा स्त्रियं यथा}
{अब्रुवन्नुपसङ्गम्य् दैवदण्डिनिपीडिताः}


\twolineshloka
{इयं स्त्री पुत्रकामस्य बभ्रोरमिततेजसः}
{ऋषयः साधु जानीत किमियं जनयिष्यति}


\twolineshloka
{इत्युक्तास्ते तदा राजन्विप्रलम्भप्रधर्षिताः}
{प्रत्यबुवंस्तान्मुनयो यत्तच्छृणु नराधिप}


\twolineshloka
{वृष्ण्यन्धकविनाशाय मुसलं घोरमायसम्}
{वासुदेवस्य दायादः सांबोऽयं जनयिष्यति}


\twolineshloka
{येन यूयं सुदुर्वृत्ता नृशंसा जातमन्यवः}
{उच्छेत्तारः कुलं कृत्स्नमृते रामजनार्दनौ}


\twolineshloka
{समुद्रं यास्यति श्रीमांस्त्यक्त्वा देहं हलायुधः}
{जरःकृष्णं महात्मानं शयानं भुवि भेत्स्यति}


\threelineshloka
{इत्यब्रुवंस्ततो राजन्प्रलब्धास्तैर्दुरात्मभिः}
{मुनयः क्रोधरक्ताक्षाः समीक्ष्याथ परस्परम्}
{तथोक्त्वा मुनयस्ते तु यथागतमथो ययुः}


\twolineshloka
{अथाब्रवीत्तदा वृष्णीञ्श्रुत्वैतन्मधुसूदनः}
{अन्तज्ञो मतिमांस्तस्य भवितव्यं तथेति तान्}


\twolineshloka
{एवमुक्त्वा हृषीकेशः प्रविवेश पुरं तदा}
{कृतान्तमन्यथा नैच्छत्कर्तुं स जगतः प्रभुः}


\twolineshloka
{श्वोभूतेऽथ ततः सांबो मुसलं तदसूकत वै}
{येन वृष्णन्धककुले पुरुषा भस्मसात्कृताः}


\twolineshloka
{वृष्ण्यन्धकविनाशाय किंकरप्रतिमं महत्}
{असूत सापजं घोरं तच्च राज्ञे न्यवेदयन्}


\twolineshloka
{विषण्णरूपस्तद्राजा सूक्ष्मं चूर्णमकारयत्}
{तच्चूर्णं सागरे चापि प्राक्षिपन्पुरुषा नृप}


\twolineshloka
{अघोषयंश्च नगरे वचनादाहुकस्य ते}
{जनार्दनस्य रामस्य बभ्रोश्चैव महात्मनः}


\twolineshloka
{अद्यप्रभृति सर्वेषु वृष्ण्यन्धककुलेष्विह}
{सुरासवो न कर्तव्यः, सर्वैर्नगरवासिभिः}


\twolineshloka
{यश्च नो विदितं कुर्यात्पेयं कश्चिन्नरः क्वचित्}
{जीवन्स शूलमारोहेत्स्वयं कृत्वा सबान्धवः}


\twolineshloka
{ततो राजभयात्सर्वे नियमं चक्रिरे तदा}
{नराः सासनमाज्ञाय तस्य राज्ञो महात्मनः}


\chapter{अध्यायः ३}
\twolineshloka
{एवं प्रयतमानानां वृष्णीनामन्धकैः सह}
{कालो गृहाणि सर्वेषां परिचक्राम नित्यशः}


\twolineshloka
{करालो विकटो मुण्डः पुरुषः कृष्णपिङ्गलः}
{गृहाण्यावेक्ष्य वृष्णीनां नादृश्यतं पुनः क्वचित्}


\twolineshloka
{तमघ्नन्त महेष्वासाः शरैः शतसहस्रशः}
{न चाशक्यत वेद्धुं स सर्वभूतात्ययस्तदा}


\twolineshloka
{उत्पेदिरे महावाता दारुणाश्च दिनेदिने}
{वृष्ण्यन्धकविनाशाय बहवो रोमहर्षणाः}


\twolineshloka
{विवृद्धमूषिका रथ्या विभिन्नमणिकास्तथा}
{केशा नखाश्च सुप्तानामद्यन्ते मूषिकैर्निशि}


\twolineshloka
{चीचीकूचीति वाशन्ति शारिका वृष्णिवेश्मसु}
{नोपशाम्यति शब्दस्य सदिवारात्रमेव हि}


\twolineshloka
{अन्वकुर्वन्नुलूकानां सारसा विरुतं तथा}
{अजाः शिवानां विरुतमन्वकुर्वत भारत}


\twolineshloka
{पाण्डुरा रक्तपादाश्च विहगाः कालचोदिताः}
{वृष्ण्यन्धकानां गेहेषु कपोता व्यचरंस्तदा}


\twolineshloka
{व्यजायन्त खरा गोषु करभाश्वतरीषु च}
{शुनीष्वपि बिडालाश्च मूषिका नकुलीषु च}


\twolineshloka
{नापत्रपन्त पापानि कुर्वन्तो वृष्णयस्तदा}
{प्रार्दयन्ब्राह्मणांश्चापि पितॄन्देवांस्तथैव च}


\twolineshloka
{गुरूंश्चाप्यवमन्यन्ते न तु रामजनार्दनौ}
{पत्न्यः पतीनुच्चरन्ति पत्नीश्चक पतयस्तथा}


\twolineshloka
{विभावसुः प्रज्वलितो वामं विपरिवर्तते}
{नीललोहितमञ्जिष्ठा विसृजन्नर्चिषः पृथक्}


\twolineshloka
{उदयास्तमये नित्यं पुर्यां तस्यां दिवाकरः}
{व्यदृश्यतासकृत्पुंभिः कबन्धैः परिवारितः}


\twolineshloka
{महानसेषु सिद्धेषु संस्कृतेऽतीवि भारत}
{आहार्यमाणे कृमयो व्यदृश्यन्त सहस्रशः}


\twolineshloka
{पुण्याहे वाच्यमाने तु जपत्सु च महात्मसु}
{अभिधावन्तः श्रूयन्ते न चादृश्यत कश्चन}


% Check verse!
परस्परं च नक्षत्रं हन्यमानं पुनःपुनः ॥ग्रहैरपश्यन्सर्वे ते नात्मनस्तु कथञ्चन
\twolineshloka
{नदन्तं पाञ्चिजन्यं च वृष्णन्धकनिवेशने}
{समन्तात्पर्यवाशन्त रासभा दारुणस्वराः}


\twolineshloka
{एवं पश्यन्हृषिकेशः संप्राप्तं कालपर्ययम्}
{त्रयोदश्याममावास्यां तान्दृष्ट्वा प्राब्रवीदिदम्}


\twolineshloka
{चतुर्दशी पञ्चदशी कृतेयं राहुणा पुनः}
{प्राप्ते वै भारते युद्धे प्राप्ता चाद्य क्षंयाय नः}


\twolineshloka
{विमृशन्नेव कालं तं परिचिन्त्य जनार्दनः}
{मेने प्राप्तं स षट्त्रिंशं वर्षं वै केशिसूदनः}


\twolineshloka
{पुत्रशोकाभिसंतप्ता गान्धारी हतबान्धवा}
{यदनुव्याजहारार्ता तदिदं समुपागमत्}


\twolineshloka
{इदं च तदनुप्राप्तमब्रवीद्यद्युधिष्ठिरः}
{पुरा व्यूढेष्वनीकेषु दृष्ट्वेत्पातान्सुदारुणान्}


\twolineshloka
{इत्युक्त्वा वासुदेवस्तु चिकीर्षुः सत्यमेव तत्}
{आज्ञापयामास तदा तीर्थयात्रामरिंदमः}


\twolineshloka
{अघोषयन्त पुरुषास्तत्र केशवशासनात्}
{तीर्थयात्रा समुद्रे वः कार्येति पुरुषर्षभाः}


\chapter{अध्यायः ४}
\twolineshloka
{काली स्त्री पाण्डुरैर्दन्तैः प्रविश्य हसती निशि}
{स्त्रियः स्वप्नेषु मुष्णन्ती द्वारकां परिधावति}


\twolineshloka
{अग्निहोत्रनिकेतेषु वास्तुमध्येषु वेश्मसु}
{वृष्ण्यन्धकानखादन्त स्वप्ने गृध्रा भयानकाः}


\twolineshloka
{अलङ्काराश्च च्छत्रं च ध्वजाश्च कवचानि च}
{ह्रियमाणान्यदृश्यन्त रक्षोभिः सुभयानकैः}


\twolineshloka
{तच्चाग्निदत्तं कृष्णस्य वज्रनाभमयोमयम्}
{दिवमाचक्रमे चक्रं वृष्णीनां पश्यतां तदा}


\twolineshloka
{युक्तं रथं दिव्यमादित्यवर्णंहयाऽहरन्पश्यतो दारुकस्य}
{ते सागरस्योपरिष्टादगच्छ-न्मनोजवाश्चतुरो वाजिमुख्याः}


\twolineshloka
{तालः सुपर्णश्च महाध्वजौ तौसुपूजितौ रामजनार्धनाभ्याम्}
{उच्चैर्जह्रुरप्सरसश्च दिव्यादेवा ऊचुर्गम्यतां तीर्थयात्राः}


\twolineshloka
{`भगवद्दर्शनस्पर्शस्नेहैर्विततभोगिनः}
{दिवः प्रवेशानार्थं ते विमानैर्दिवमाययुः ॥'}


\twolineshloka
{ततो जिगमिषन्तस्ते वृष्ण्यन्धकमहारथाः}
{सान्तःपुरास्तदा तीर्थयात्रामैच्छन्नरर्षभाः}


\twolineshloka
{ततो मनोहरा हृष्टाःक पेयं मांसं मधूनि च}
{बहु नानाविधं चक्रुस्त्वरिताः कालचोदिताः}


\twolineshloka
{ततः सीधुषु सक्ताश्च निर्ययुर्नगराद्बहिः}
{यानैरश्वैर्गजैश्चैव श्रीमन्तस्तिग्मतेजसः}


\twolineshloka
{ततः प्रभासे न्यवसन्यथोद्दिष्टं यथागृहम्}
{प्रभूतभक्ष्यपेयास्ते सदारा यादवास्तदा}


\twolineshloka
{निविष्टांस्तान्निशाम्याथ समुद्रान्ते स योगवित्}
{जगामामन्त्र्य तान्वीरानुद्धवोऽर्थविशारदः}


\twolineshloka
{तं प्रस्थितं महात्मानमभिवाद्य कृताञ्जलिम्}
{जानन्विनाशं वृष्णीनां नैच्छद्वारयितुं हरिः}


\twolineshloka
{ततः कालपरीतास्ते वृष्ण्यन्धकमहारथाः}
{अपश्यन्नुद्धवं यान्तं तेजसाऽऽवृत्य रोदसी}


\twolineshloka
{ब्राह्मणार्थेषु यत्सिद्धमन्नं तेषां महात्मनाम्}
{तद्वानरेभ्यः प्रददुः सुरागन्धसमन्वितम्}


\twolineshloka
{ततस्तूर्यशताकीर्णं नटनर्तकसंकुलम्}
{अवर्तत महापानं प्रभासे तिग्मतेजसाम्}


\twolineshloka
{कृष्णस्य सन्निधौ रामः सहितः कृतवर्मणा}
{अपिबद्युयुधानश्च गदो बभ्रुस्तथैव च}


\twolineshloka
{ततः परिषदो मध्ये युयुधानो मदोत्कटः}
{अब्रवीत्कृतवर्माणमवहास्यावमत्य च}


\twolineshloka
{कः क्षत्रियो हन्यमानः सुप्तान्हन्यान्मृतानिव}
{तन्न मृष्यन्ति हार्दिक्य यादवा यत्त्वया कृतम्}


\twolineshloka
{इत्युक्ते युयुधानेन पूजयामास तद्वचः}
{प्रद्युम्नो रथिनांश्रेष्ठो हार्दिक्यमवमत्य व}


\twolineshloka
{ततः परमसंक्रुद्धः कृतवर्मा तमब्रवीत्}
{निर्दिशन्निव सावज्ञं तदा सव्येन पाणिना}


\twolineshloka
{भूरिश्रवाश्छिन्नबाहुर्युर्द्धे प्रायगतस्त्वया}
{वधेन सुनृशंसेन कथं वीरेण पातितः}


\twolineshloka
{इति तस्य वचः श्रुत्वा केशवः परवीरहा}
{तिर्यक्सरोषया दृष्ट्या वीक्षांचक्रे स मन्युमान्}


\twolineshloka
{मणिः स्यमन्तकश्चैव यः स सत्राजितोऽभवत्}
{तां कथां श्रावयामास सात्यकिर्मधुसूदनम्}


\twolineshloka
{तच्छ्रुत्वा केशवस्याङ्कमगमद्रुदती तदा}
{सत्यभामा प्रकुपिता कोपयन्ती जनार्दनम्}


\twolineshloka
{तत उत्थाय सक्रोधः सात्यकिर्वाक्यमब्रवीत्}
{}


\twolineshloka
{पञ्चानां द्रौपदेयानां धृष्टद्युम्नशिखण्डिनोः}
{एष गच्छामि पदवीं सत्येन च तथा शपे}


\twolineshloka
{सौप्तिके ये च निहताः सुप्ता येन दुरात्मना}
{द्रोणपुत्रसहायेन पापेन कृतवर्मणा}


\threelineshloka
{समाप्तमायुरस्याद्य यशश्चैव सुमध्यते}
{इत्येवमुक्त्वा खङ्गेन केशवस्य समीपतः}
{अभिद्रुत्य शिरः क्रुद्धश्चिच्छेद कृतवर्मणः}


\twolineshloka
{तथाऽन्यानपि निघ्नन्तं युयुधानं समन्ततः}
{अभ्यधावद्धृषीकेशो विनिवारयितुं तदा}


\twolineshloka
{एकीभूतास्ततः सर्वे कालपर्यायचोदिताः}
{भोजान्धका महाराज शैनेयं पर्यवारयन्}


\twolineshloka
{तान्दृष्ट्वा पततस्तूर्णमभिक्रुद्धाञ्जनार्दनः}
{न चुक्रोध महातेजा जानन्कालस्य पर्ययम्}


\twolineshloka
{ते तु पानमदाविष्टाश्चोदिताः कालधर्मणा}
{युयुधानमथाभ्यघ्नन्नुच्छिष्टैर्भाजनैस्तदा}


\twolineshloka
{हन्यमाने तु शैनेये क्रुद्धो रुक्मिणिनन्दनः}
{तदनन्तरमागच्छन्मोचयिष्यञ्शिनेः सुतम्}


\threelineshloka
{स भोजैः सह संयुक्तः सात्यकिश्चान्धकैः सह}
{व्यायच्छमानौ तौ वीरौ बाहुद्रविणशालिनौ}
{बहुत्वान्निहतौ तत्र उभौ कुष्णस्य पश्यतः}


\twolineshloka
{हतं दृष्ट्वा च शैनेयं पुत्रं च यदुनन्दनः}
{एरकाणां ततो मुष्टिं कोपाज्जग्राह केशवः}


\twolineshloka
{तदभून्मुसलं घोरं वज्रकल्पमयोमयम्}
{जघान कुष्णस्तांस्तेनि येये प्रमुखतोऽभवन्}


\twolineshloka
{ततोऽन्धकाश्च भोजाश्च शैनेया वृष्णयस्तथा}
{जघ्नुरन्योन्यमाक्रन्दे मुसलैः कालचोदिताः}


\twolineshloka
{यस्तेषामेरकां कश्चिज्जग्राह कुपितो नृपः}
{वज्रभूतेव सा राजन्नदृश्यत तदा विभो}


\twolineshloka
{तृणं च मुसलीभूतमपि तत्र व्यदृश्यत}
{ब्रह्मदण्डकृतं सर्वमिति तद्विद्वि पार्थिव}


\twolineshloka
{आविध्याविध्य ते राजन्प्रहरन्ति स्म तैस्तृणैः}
{तद्वज्रभूतं मुसलं व्यदृश्यत तदा दृढम्}


\twolineshloka
{अवधीत्पितरं पुत्रः पिता पुत्रं च भारत}
{मत्ताः परिपतन्ति स्म योधयन्तः परस्परम्}


\twolineshloka
{पतङ्गा इव चाग्नौ ते निपेतुः कुकुरान्धकाः}
{नासीत्पलायने बुद्धिर्वधअयमानस्य कस्यचित्}


\twolineshloka
{तत्रापश्यन्महाबाहुर्जानन्कालस्य पर्ययम्}
{मुसलं समवष्टभ्य तस्थौ स मधुसूदनः}


\twolineshloka
{सांबं च निहतं दृष्ट्वा चारुदेष्णं च माधवः}
{प्रद्युम्नं चानिरुद्धं च ततश्चुक्रोध भारत}


\twolineshloka
{गदं वीक्ष्य शयानं च भृशं कोपसमन्वितः}
{स निःशेषं तदा चक्रे शार्ङ्गचक्रगंदाधरः}


\twolineshloka
{तं निघ्नन्तं महातेजा बभ्रुः परपुरंजयः}
{दारुकश्चैव दाशार्हमूवतुर्यन्निबोध तत्}


\twolineshloka
{भगवन्निहताः सर्वे त्वया भूयिष्ठशो नराः}
{रामस्य पदमन्विच्छ तत्र गच्छाम यत्र सः}


\chapter{अध्यायः ५}
\threelineshloka
{ततो ययुर्दारुकः केशवश्चबभ्रुश्च रामस्य पदं पतन्तः}
{अथापश्यन्राममनन्तवीर्यंवृक्षाश्रितं चिन्तयानं विविक्ते}
{}


\twolineshloka
{ततः समासाद्य महानुभावःकृष्णस्तदा दारुकमन्वशासत्}
{गत्वा कुरून्सर्वमिमं महान्तंपार्थाय शंसस्व वधं यदूनाम्}


\twolineshloka
{ततोऽर्जुनः क्षिप्रमिहोपयातुश्रुत्वा मृतान्यादवान्ब्रह्मशापात्}
{इत्येवमुक्तः स ययौ रथेनकुरूंस्तदा दारुको नष्टचेताः}


\twolineshloka
{ततो गते दारुके केशवोथदृष्ट्वान्तिके बभ्रुमुवाच वाक्यम्}
{स्त्रियो भवान्रक्षितुं यातु शीघ्रंनैता हिंस्युर्दस्यवो वित्तलोभात्}


\threelineshloka
{स प्रस्थितः केशवेनानुशिष्टोमदातुरो ज्ञातिवधार्दितश्च}
{तं वै यातं सन्निधौ केशवस्यत्वरन्तमेकं सहसैव बभ्रूम्}
{ब्रह्मानुशप्तमवधीन्महद्वैकूटे युक्तं मुसलं लुब्धकस्य}


\twolineshloka
{ततो दृष्ट्वा निहतं बभ्रुमाहकृष्णेऽग्रजं भ्रातरमुग्रतेजाः}
{हहैव त्वं मां प्रतीक्षस्व रामयावत्स्त्रियो ज्ञातिवशाः करोमि}


\twolineshloka
{ततः पुरीं द्वारवतीं पविश्यजनार्दनः पितरं प्राह वाक्यम्}
{सर्वं भवान्रक्षतु नः समग्रंधनंजयस्यागमनं प्रतीक्षन्}


\twolineshloka
{रामो वनान्ते प्रतिपालयन्मा-मास्तेऽद्याहं तेन समानमिष्ये}
{दृष्टं मयेदं निधनं यदूनांराज्ञां च पूर्वं कुरुपुङ्गवानाम्}


\twolineshloka
{नाहं विना यदुभिर्यादवानांपुरीमिमामशकं द्रष्टुमद्य}
{तपश्चरिष्यामि निबोध तन्मेरामेणि सार्धं वनमभ्युपेत्य}


\twolineshloka
{इतीदमुक्त्वा शिरसा च पादौसंस्पृश्य कृष्णस्त्वरितो जगाम}
{ततो महान्निनदः प्रादुरासी-त्सस्त्रीकुमारस्य पुरस्य तस्य}


\twolineshloka
{अथाब्रवीत्केशवः सन्निवर्त्यशब्दं श्रुत्वा योषितां क्रोशतीनाम्}
{पुरीमिमामेष्यति सव्यसाचीस वो दुःखान्मोचयिता नराग्र्यः}


% Check verse!
ततो गत्वा केशवस्तं ददर्शरामं वने स्थितमेकं विविक्ते
\threelineshloka
{अथापश्यद्योगयुक्तस्य तस्यनागं मुखान्निस्सरन्तं महान्तम्}
{श्वेतं ययौ सततः प्रेक्षमाणोमहार्णवो येन महानुभावः}
{सहस्रशीर्षः पर्वताभोगवर्ष्मारक्ताननः स्वां तनुं तां विमुच्य}


\twolineshloka
{संदृश्यन्तं सागरान्तं विशन्तंसम्यक्तदा सागरः प्रत्यगृह्णात्}
{नागा दिव्याः सरितश्चैव पुण्याःकर्कोटको वासुकिस्तक्षकश्च}


\twolineshloka
{पृथुश्रवा अरुणः कुञ्जरश्चमिश्री शङ्खः कुमुदः पुण्डरीकः}
{तथा नागो धृतराष्ट्रो महात्माह्रादः क्राथः शितिपृष्ठो निकेतुः}


\threelineshloka
{तथा नागौ चक्रमन्दातिषण्डौनागश्रेष्ठो दुर्मखश्चांबरीषः}
{स्वयं राजा वरुणश्चापि राज-न्प्रत्युद्गम्य स्वागतेनाभ्यनन्दन्}
{तेऽपूजयंश्चार्घ्यपाद्यक्रियाभिःसङ्कर्षणं भूधरं शुद्धबुद्धिम्}


\twolineshloka
{ततो गते भ्रातरि वासुदेवोजानन्सर्वा गतयो दिव्यदृष्टिः}
{वने शून्ये विचरंश्चिन्तयानोभूमौ चाथ संविवेशाग्र्यतेजाः}


\twolineshloka
{सर्वं तेन प्राक्तदा वित्तमासी-द्गान्धार्या यद्वाक्यमुक्तं स्वधर्मात्}
{दुर्वाससा पारसोच्छिष्टलिप्तेयच्चाप्युक्तं तच्च सस्मार कृष्णः}


\twolineshloka
{`कुलस्त्रीणां संवृतानां गवां चद्विजेन्द्राणां बलमादाय धर्मे}
{योनिर्देवानां वरदो ब्रह्मदेवोगोविन्दाख्यो वासुदेवोऽथ नित्यः}


\twolineshloka
{संचिन्तयन्नन्धकवृष्णिनाशंकुरुक्षयं चैव महानुभावः}
{मेने ततः संक्रमणस्य कालंततश्चकारेन्द्रियसन्निरोधम्}


\twolineshloka
{यथा च लोकत्रयपालनार्थंदुर्वासवाक्यप्रतिपालनाय}
{देवोपि सन्देहविमोक्षहेतो-र्निमित्तमैच्छत्सकलार्थतत्त्ववित्}


\twolineshloka
{स सन्निरुद्धेन्द्रियवाङ्मनास्तुशिश्ये महायोगमुपेत्य कृष्णः}
{जरोथ तं देशमुपाजगामलुब्धस्तदानीं मृगलिप्सुरुग्रः}


\twolineshloka
{स केशवं योगयुक्तं शयानंमृगासक्तो लुब्धकः सायकेन}
{जरोऽविध्यत्पादतले त्वरावां-स्तं चाभितस्तज्जिघृक्षुर्जगाम}


\twolineshloka
{अथापश्यत्पुरुषं योगयुक्तंपीतांबरं लुब्धकोऽनेकबाहुम्}
{मत्वाऽऽत्मानं त्वपराद्धं स तस्यपादौ जरो जगृहे शङ्कितात्मा}


\twolineshloka
{आश्वासितः पुण्यफलेन भक्तयातथाऽनुतापात्कर्मणो जन्मनश्च}
{दृष्ट्वा तथा देवमनन्तवीर्यंदेवैस्स्वर्गं प्रातितस्त्यक्तदेहः}


% Check verse!
गणैर्मुनीनां पूजितस्यत्र कृष्णोगच्छन्नूर्द्ध्वं व्याप्य लोकान्स लक्ष्म्या
\twolineshloka
{दिवं प्राप्तं वासवोऽथाश्विनौ चरुद्रादित्या वसवश्चाथ विश्वे}
{प्रत्युद्ययुर्मुनयश्चापि सिद्धागन्धर्वमुख्याश्च सहाप्सरोभिः}


\twolineshloka
{ततो राजन्भगवानुग्रतेजानारायणः प्रभवश्चाव्ययश्च}
{योगाचार्यो रोदसी व्याप्य लक्ष्म्यास्थानं प्राप स्वं महात्माऽप्रमेयम्}


\twolineshloka
{ततो देवैर्ऋषिभिश्चापि कृष्णःसमागतश्चारणैश्चैव राजन्}
{गन्धर्वाग्र्यैरप्सरोभिर्वराभिःसिद्धैः साध्यैश्चानतैः पूज्यमानः}


\twolineshloka
{तं वै देवाः प्रत्यनन्दन्त राज-न्मुनिश्रेष्ठा ऋग्भिरानर्चुरीशम्}
{तं गन्धर्वाश्चापि तस्थुः स्तुवन्तःप्रीत्य चैनं पुरुहूतोऽभ्यनन्दत्}


\twolineshloka
{`नमोनमस्ते भगवञ्शार्ङ्गधन्व-न्धर्मस्थित्या प्रादुरासीर्धरायाम्}
{कंसाख्यादीन्देवशत्रूंश्च सर्वा-न्हत्वा भूमिः स्थापिता भारतप्ता}


\twolineshloka
{दिव्यं स्थानमजरं चाप्रमेयंदुर्विज्ञेयं चागमैर्गम्यमग्र्यम्}
{गच्छ प्रभो रक्ष चार्तिं प्रपन्ना-न्कल्पेकल्पे जायमानांश्च मर्त्यान्}


\twolineshloka
{इत्येवमुक्त्वा देवसंघोऽनुगम्यश्रिया युक्तं पुष्पवृष्ट्या ववर्ष}
{वाणी चासीत्संश्रिता रूपिणी साभानोर्मध्ये प्रविश त्वं तु राजन्}


\twolineshloka
{भुजैश्चतुर्भिः समुपेतं ममेदंरूपं विशिष्टं जीवितं संस्थितं च}
{भूमौ गतं पूजयताप्रमेयंसदा हि तस्मिन्निवसामीति देवाः}


\twolineshloka
{देवा निवृत्तास्तत्पदं नाप्नुवन्तोबुद्ध्या देवं संस्मरन्तः प्रतीताः}
{ब्रह्माद्यास्ते तद्गुणान्कीर्तयन्तःशुभाँल्लोकान्स्वान्प्रपेदुः सुरेन्द्राः}


\chapter{अध्यायः ६}
\twolineshloka
{दारुकोपि कुरून्गत्वा दृष्ट्वा पार्थान्महारथान्}
{आचष्ट मौसले वृष्णीनन्योन्येनोपसंहृतान्}


\twolineshloka
{श्रुत्वा विनष्टान्वार्ष्णेयान्सभोजान्धककौकुरान्}
{पाण्डवाः शोकसंतप्ता वित्रस्तमनसोऽभवन्}


\twolineshloka
{ततोऽर्जुनस्तानामन्त्र्य केशवस्य प्रियः सखा}
{प्रययौ मातुलं द्रष्टुं नेदमस्तीति चाब्रवीत्}


\twolineshloka
{स वृष्णिनिलयं गत्वा दारुकेण सह प्रभो}
{ददर्श द्वारकां वीरो मृतनाथामिव स्त्रियम्}


\twolineshloka
{याः स्म ता लोकनाथेन नाथवत्यः पुराऽभवन्}
{तास्त्वनाथास्तदा नाथं पार्थं दृष्ट्वा विचुक्रुशुः}


\twolineshloka
{षोडश स्त्रीसहस्राणि वासुदेवपरिग्रहाः}
{तासामासीन्महान्नादो दृष्ट्वैवार्जुनमागतम्}


\twolineshloka
{तास्तु दृष्ट्वैव कौरव्यो बाष्पेणापिहितेक्षणः}
{हीनाः कृष्णेन पुत्रैश्च नाशकत्सोभिवीक्षितुम्}


\twolineshloka
{स तां वृष्ण्यन्धकजलां हयमीनां रथोडुपाम्}
{वादित्ररथघोषौघां वेश्मतीर्थमहाग्रहाम्}


\twolineshloka
{रत्नशैवलसंघातां वज्रप्राकारमालिनीम्}
{रथ्यास्रोतोजलावर्तां चत्वरस्तिमितह्रदाम्}


\twolineshloka
{रामकृष्णमहाग्राहां द्वारकां सरितं तदा}
{कालपाशग्रहां भीमां नदीं वैतरणीमिव}


\threelineshloka
{ददर्श वासविर्धीमान्विहीनां वृष्णिपुङ्गवैः}
{गतश्रियं विरानन्दां पद्मिनीं शिशिरे यथा}
{}


\twolineshloka
{तां दृष्ट्वा द्वारकां पार्थस्ताश्च कृष्णस्य योषितः}
{सस्वनं बाष्पमुत्सृज्य निपात महीतले}


\twolineshloka
{सात्राजिती ततः सत्या रुक्मिणी च विशांपते}
{अभिपत्य प्ररुरुदुः परिवार्य धनंजयम्}


\twolineshloka
{ततस्तं काञ्चने पीठे समुत्ताप्योपवेश्य च}
{अभिनन्द्य महात्मानं परिवार्योपतस्थिरे}


\twolineshloka
{ततः संस्तूय गोविन्दं कथयित्वा च पाण्डवः}
{आश्वास्य ताः स्त्रियश्चापि मातुलं द्रष्टुमभ्यगात्}


\chapter{अध्यायः ७}
\twolineshloka
{तं प्रविष्टं महात्मानं वीरमानकदुन्दुभिः}
{पुत्रशोकेन संतप्तो ददर्श कुरुपुङ्गवम्}


\twolineshloka
{तस्याश्रुपरिपूर्णाक्षो व्यूढोरस्को महाभुजः}
{आर्तस्यार्ततरः पार्थः पादौ जग्राह भारत}


\twolineshloka
{तस्य मूर्धानमाघ्रातुमियेषानकदुन्दुभिः}
{स्वस्रीयस्य महाबाहुर्न शशाक च शत्रुहन्}


\fourlineindentedshloka
{समालिङ्ग्यार्जुनं वृद्धः स भुजाभ्यां महाभुजः}
{रुरोदाथ स्मरञ्शौरिं विललाप सुविह्वलः}
{भ्रातॄन्पुत्रांश्च पौत्रांश्च दौहित्रान्स सखीनपि ॥वसुदेव उवाच}
{}


\twolineshloka
{यैर्जिता भूमिपालाश्च दैत्याश्च शतशोऽर्जुन}
{तान्सर्वान्नेह पश्यामि जीवाम्यर्जुन दुर्मरः}


\twolineshloka
{यो तावर्जुनशिष्यौ ते प्रियौ बहुमतौ सदा}
{तयोरपनयात्पार्थ वृष्णयो निधनं गताः}


\twolineshloka
{यौ तौ वृष्णिप्रवीराणां द्वावेवातिरथौ मतौ}
{प्रद्युम्नो युयुधानश्चि कथनन्कत्थसे च यौ}


\twolineshloka
{तौ सदा कुरुशार्दूल कृष्णस्य प्रियभाजनौ}
{तावुभौ वृष्णिनाशस्य मुखमास्तां धनंजय}


\twolineshloka
{न तु गर्हामि शैनेयं हार्दिक्यं चाहमर्जुन}
{अक्रूरं रौक्मिणेयं च शापो ह्येवात्र कारणम्}


\twolineshloka
{केशिनं यस्तु कंसं च विक्रम्यि हतवान्प्रभुः}
{विदेहमकरोत्पार्थं चैद्यं च बलगर्वितम्}


\twolineshloka
{नैषादिमेकलव्यं च चक्रे कालिङ्गमागधान्}
{गान्धारान्काशिराजं च मरुभूमौ च पार्थिवान्}


\twolineshloka
{प्राच्यांश्च दाक्षिणात्यांश्च पार्वतीयांस्तथा नृपान्}
{सोभ्युपेक्षितवानेवमनयान्मधुसूदनः}


\twolineshloka
{त्वं हि तं नारदश्चैव मुनयश्च सनातनम्}
{गोविन्दमनघं देवमभिजानीध्वमच्युतम्}


\twolineshloka
{प्रत्यपश्यच्च स विभुर्ज्ञातिक्षयमधोक्षजः}
{समुपेक्षितवान्नित्यं स्वयं स मम पुत्रकः}


\twolineshloka
{गान्धार्या वचनं यत्तदृषीणां च परंतप}
{तन्नूनमन्यथा कर्तुं नैच्छत्स जगतः प्रभुः}


\threelineshloka
{प्रत्यक्षं भवतश्चापि तव पौत्रः परंतप}
{अश्वत्थाम्ना हतश्चापि जीवितस्तस्य तेजसा}
{इमांस्तु नैच्छत्स्वान्ज्ञातीन्रक्षितुं च सखा तव}


\twolineshloka
{ततः पुत्रांश्च पौत्रांश्च भ्रातॄनथ सखींस्तथा}
{शयानान्निहतान्दृष्ट्वा कृष्णो मामब्रवीदिदम्}


\threelineshloka
{सम्प्राप्तोऽद्यायमस्यान्तः कुलस्य भरतर्षभ}
{आगमिष्यति बीभत्सुरिमां द्वारवतीं पुरीम्}
{आख्येयं तस्य यद्वृत्तं वृष्णीनां वैशसं महत्}


\twolineshloka
{स तु श्रुत्वा महातेजा यदूनां निधनं प्रभो}
{आगन्ता क्षिप्रमेवेह न मेऽत्रास्ति विचारणा}


\twolineshloka
{योऽहं तमर्जुनं विद्धि योऽर्जुनः सोहमेव तु}
{यद्ब्रूयात्तत्तथा कार्यमिति बुद्ध्यस्व भारत}


\twolineshloka
{स स्त्रीषु प्राप्तकालासु पाण्डवो बालकेषु च}
{प्रतिपत्स्यति बीभत्सुर्भवतश्चौर्ध्वद्गेहिकम्}


\twolineshloka
{इमां च नगरीं सद्यः प्रतियाते धनंजये}
{प्राकारट्टालकोपेतां समुद्रः प्लावयिष्यति}


\twolineshloka
{अहं देशे तु कस्मिंश्चित्पुण्ये नियममास्थितः}
{कालं काङ्क्षे सद्य एव रामेण सहितो वने}


\twolineshloka
{एवमुक्त्वा हृषीकेशो मामचिन्त्यपराक्रमः}
{हित्वा मां बालकैः सार्धं दिशं कामप्यगात्प्रभुः}


\twolineshloka
{सोहं तौ च महात्मानौ चिन्तयन्रामकेशवौ}
{घोरं ज्ञातिवधं चैव न भुञ्जे शोककर्शितः}


\twolineshloka
{न मोक्ष्येन च जीविष्ये दिष्ट्या प्राप्तोसि पाण्डव}
{यदुक्तं पार्थं कृष्णेन तत्सर्वमखिलं कुरु}


\twolineshloka
{एतत्ते पार्थ राज्यं च स्त्रियो रत्नानि चैव हि}
{इष्टान्प्राणानहं हीमांस्त्यक्ष्यामि रिपुसूदन}


\chapter{अध्यायः ८}
\twolineshloka
{एवमुक्तः स बीभत्सुर्मातुलेनि परंतप}
{दुर्मना दीनवदनो वसुदेवमुवाच ह}


\twolineshloka
{नाहं वृष्णिप्रवीरेण बन्धुभिश्चैव मातुल}
{विहीनां पृथिवीं द्रष्टुं शक्तश्चिरमिह प्रभो}


\twolineshloka
{राजा च भीमसेनश्च सहदेवश्च पाण्डवः}
{नकुलो याज्ञसेनी च षडेकमनसो वयम्}


\twolineshloka
{राज्ञः संक्रमणे चापि कामोऽयं नित्यमेव हि}
{तमिमं विद्धि सम्प्राप्तं कालं कालविदांवर}


\twolineshloka
{सर्वथा वृष्णिदारांस्तु बालं वृद्धं तथैव च}
{नयिष्ये परिगृह्याहमिन्द्रप्रस्थमरिंदम}


\twolineshloka
{इत्युक्त्वा दारुकमिदं वाक्यमाह धनंजयः}
{अमात्यान्वृष्णिवीराणां द्रष्टुमिच्छामि माचिरम्}


\twolineshloka
{इत्येवमुक्त्वा वचनं सुधर्मां यादवीं सभाम्}
{प्रविवेशार्जुनः शूरः शोचमानो महाभुजः}


\twolineshloka
{तमासनगतं तत्र सर्वाः प्रकृतयस्तथा}
{ब्राह्मणा नागरास्तत्र परिवार्योपतस्थिरे}


\twolineshloka
{तान्दीनमनसः सर्वान्विमूढान्गतचेतसः}
{उवाचेदं वचः काले पार्थो दीनतरस्तथा}


\twolineshloka
{शक्रप्रस्थमहं नेष्ये पृष्णन्धकजनं स्वयम्}
{इदं तु नगरं सर्वं समुद्रः प्लावयिष्यति}


\twolineshloka
{सज्जीकुरुत यानानि रत्नानि विविधानि च}
{वज्रोऽयं भवतां राजा शक्रप्रस्थे भविष्यति}


\twolineshloka
{सप्तमे दिवसे चैव रवौ विमल उद्गते}
{बहिर्वत्स्यामहे सर्वे सज्जीभावत माचिरम्}


\twolineshloka
{इत्युक्तास्तेन ते सर्वे पार्थेनाक्लिष्टकर्मणा}
{सज्जमाशु ततश्चक्रुः स्वसिद्ध्यर्थं समुत्सुकाः}


\twolineshloka
{तां रात्रिमवसत्पार्थः केशवस्य निवेशने}
{महता शोकमोहेन सहसाऽभिपरिप्लुतः}


\twolineshloka
{श्वोभूतेऽथ ततः शौरिर्वसुदेवः प्रतापवान्}
{युक्त्वाऽऽत्मानं महातेजा जगाम गतिमुत्तमां}


\twolineshloka
{ततः शब्दो महानासीद्वसुदेवनिवेशने}
{दारुणः क्रोशतीनां च रुदतीनां च योषिताम्}


\threelineshloka
{प्रकीर्णमूर्धजाः सर्वा विमुक्ताभरणस्रजः}
{उरांसि पाणिभिर्घ्नन्त्यो व्यलपन्तकरुणं स्त्रियः}
{}


\twolineshloka
{तं देवकी च भद्रा च रोहिणी मदिरा तथा}
{अन्वारोढुं व्यवसिता भर्तारं योषितां वराः}


\twolineshloka
{ततः शौरिं नृयुक्तेन बहुमूल्येन भारत}
{यानेन महता पार्थो बहिर्निष्क्रामयत्तदा}


\twolineshloka
{तमन्वयुस्तत्रतत्रं दुःखसोकसमन्विताः}
{द्वारकावासिनः सर्वे पौरजानपदा हिताः}


\twolineshloka
{तस्याश्वमेधिकं छत्रं दीप्यमानाश्च पावकाः}
{पुरस्तात्तस्य यानस्य याजकाश्च ततो ययुः}


% Check verse!
अनुजग्मुश्च तं वीरं देव्यस्ता वै स्वलङ्कृताः ॥स्त्रीसहस्रैः परिवृता वधूभिश्च सहस्रशः
\twolineshloka
{यस्तु देशः प्रियस्तस्य जीवतोऽभून्महात्मनः}
{तत्रैनमुपसंकल्प्य वितृमेधं प्रचक्रिरे}


\twolineshloka
{तं चिताग्निगतं वीरं शूरपुत्रं वराङ्गनाः}
{ततोन्वारुरुहुः पत्न्यश्चतस्रः पतिलोकगाः}


\twolineshloka
{तं वै चतसृभिः स्त्रीभिरन्वितं पाण्डुनन्दनः}
{अदाहयच्चन्दनैश्च गन्धैरुच्चावचैरपि}


\twolineshloka
{ततः प्रादुरभूच्छब्दः समिद्धस्य विभावसोः}
{सामगानां च निर्घोषो नराणां रुदतामपि}


\threelineshloka
{ततो वज्रप्रधानास्ते वृष्ण्यन्धककुमारकाः}
{सर्वे चैवोदकं चक्रुः स्त्रियश्चैव महात्मनः}
{}


\twolineshloka
{अलुप्तधर्मस्तं धर्मं कारयित्वा स फल्गुनः}
{जगाम वृष्णयो यत्र विनष्टा भरतर्षभ}


\twolineshloka
{स तान्दृष्ट्वा निपतितान्कदने भृशदुःखितः}
{बभूवातीव कौरव्यः प्राप्तकालं चकार ह}


\twolineshloka
{यथाप्रधानतश्चैव चक्रे सर्वास्तथा क्रियाः}
{ये हता ब्रह्मशापेन मुसलैरेरकोद्भवैः}


\twolineshloka
{`ततो भगवतो देहं दृष्ट्वा शिष्यः प्रलप्य च}
{स्मृत्वा तद्वचनं सर्वं मोहशोकोपबृंहितः ॥'}


\twolineshloka
{ततः शरीरे रामस्य वासुदेवस्य चोभयोः}
{अन्वीक्ष्य दाहयामास पुरुषैराप्तकारिभिः}


\twolineshloka
{स तेषां विधिवत्कृत्वा प्रेतकार्याणि पाण्डवः}
{सप्तमे दिवसे प्रायाद्रथमारुह्य सत्वरः}


\twolineshloka
{अश्वयुक्तै रथैश्चापि गोखरोष्ट्रयुतैरपि}
{स्त्रियस्ता वृष्णिवीराणां रुदन्त्यः शोककर्शिताः}


\twolineshloka
{अनुजग्मुर्महात्मानं पाण्डुपुत्रं धनंजयम्}
{भृत्याश्चान्धकवृष्णीनां सादिनो रथिनश्च ये}


\twolineshloka
{वीरहीना वृद्धबालाः पौरजानपदास्तथा}
{ययुस्ते परिवार्याथ कलत्रं पार्थशासनात्}


\twolineshloka
{कुञ्जरैश्च गजारोहा ययुः शैलनिर्भस्तथा}
{सपादरक्षैः संयुक्ताः सोत्तरायुधिका ययुः}


\twolineshloka
{पुत्राश्चान्धकवृष्णीनां सर्वे पार्थमनुव्रताः}
{ब्राह्मणाः क्षत्रिया वैश्याः शूद्राश्चैव महाधनाः}


\twolineshloka
{दश षट् च सहस्राणि वासुदेवावरोधनम्}
{पुरस्कृत्य ययुर्वज्रं पौत्रं कृष्णस्य धीमतः}


\twolineshloka
{बहूनि च सहस्राणि प्रयुतान्यर्बुदानि च}
{भोजवृष्ण्यन्धकस्त्रीणां हतनाथानि निर्ययुः}


\twolineshloka
{तत्सागरसमप्रख्यं वृष्णिचक्रं महर्द्धिमत्}
{उवाह रथिनांश्रेष्ठः पार्थः परपुरंजयः}


\twolineshloka
{निर्याते तु जने तस्मिन्सागरो मकरालयः}
{द्वारकां रत्नसंपूर्णां जलेनाप्लावयत्तदा}


\twolineshloka
{यद्यद्धि पुरुषव्याघ्रो भूमेस्तस्या व्यमुञ्चत}
{तत्तत्संप्लावयामास सलिलेन स सागरः}


\twolineshloka
{तदद्भुतमभिप्रेक्ष्य द्वारकावासिनो जनाः}
{तूर्णात्तूर्णतरं जग्मुरहो दैवमिति ब्रुवन्}


\twolineshloka
{काननेषु च रम्येषु पर्वतेषु नदीषु च}
{निवसन्नानयामास वृष्णिदारान्धनंजयः}


\twolineshloka
{स पञ्चनदमासाद्य धीमानतिसमृद्धिमत्}
{देशे गोपशुधान्याढ्ये निवासमकरोत्प्रभुः}


\twolineshloka
{ततो लोभः समभवद्दस्यूनां निहतेश्वराः}
{दृष्ट्वा स्त्रियो नीयमानाः पार्थेनैकेन भारत}


\twolineshloka
{ततस्ते पापकर्माणो लोभोपहतचेतसः}
{आभीरा मन्त्रयामासुः समेत्याशुभदर्शनाः}


\twolineshloka
{अयमेकोऽर्जुनो धन्वी वृद्धबालं हतेश्वरम्}
{नयत्यस्मानतिक्रम्य् योधाश्चेमे हतौजसः}


\twolineshloka
{ततो यष्टिप्रहरणा दस्यवस्ते सहस्रशः}
{अभ्यधावन्त वृष्णीनां तं जनं लोप्त्रहारिणः}


\twolineshloka
{महता सिंहनादेन त्रासयन्तः पृथग्जनम्}
{अभिपेतुर्धनार्थं ते कालपर्यायचोदिताः}


\twolineshloka
{ततो निवृत्तः कौन्तेयः सहसा सपदानुगः}
{उवाच तान्महाबाहुरर्जुनः प्रहसन्निव}


\twolineshloka
{निवर्तध्वमधर्मज्ञा यदि जीवितुमिच्छथ}
{इदानीं शरनिर्भिन्नाः शोचध्वं निहता मया}


\twolineshloka
{तथोक्तास्तेन वीरेणि कदर्तीकृत्य तद्वचः}
{अभिपेतुर्जनं मूढा वार्यमाणाः पुनःपुनः}


\twolineshloka
{ततोऽर्जुनो धनुर्दिव्यं गाण्डीवमजरं महत्}
{कआरोपयितुमारेभे यत्नादिव कथञ्चन}


\twolineshloka
{चकार सज्यं कृच्छ्रेणि सम्भ्रमे तुमुले सति}
{चिन्तयामास शस्त्राणि न च सस्मार तान्यपि}


\twolineshloka
{वैकृतं तन्महद्दृष्ट्वा भुजवीर्य तथा युधि}
{दिव्यानां च महास्त्राणां विनाशाद्व्रीडितोऽभवत्}


\twolineshloka
{वृष्णियोधाश्च ते सर्वे गजाश्वरथयोधिनः}
{न शेकुरावर्तयितुं ह्रियमाणं स्वकं धनम्}


\twolineshloka
{कलत्रस्य बहुत्वाद्धि सम्पतन्सु ततस्ततः}
{प्रयत्नमकरोत्पार्थो जनस्य परिरक्षणे}


\twolineshloka
{मिषतां सर्वयोधानां ततस्ताः प्रमदोत्तमाः}
{समन्ततोवकृष्यन्त कामाच्चान्याः प्रवव्रजुः}


\twolineshloka
{ततो गाण्डीवनिर्मुक्तैः शरैः पार्थो धनंजयः}
{जघान दस्यून्सोद्वेगो वृष्णिभृत्यैः सहस्रशः}


\twolineshloka
{क्षणेन तस्य ते राजन्क्षयं जग्मुरजिह्मगाः}
{अक्षया हि पुरा भूत्वा क्षीणाः क्षतजभोजनाः}


\twolineshloka
{स शरक्षयमासाद्य दुःखशोकसमाहतः}
{धनुष्कोट्या तदा दस्यूनवधीत्पाकशासनिः}


\twolineshloka
{[प्रेक्षतस्त्वेव पार्थस्य वृष्ण्यन्धकवरस्त्रियः}
{जग्मुरादाय ते म्लेच्छाः समन्ताज्जनमेजय ॥]}


\twolineshloka
{धनंजयस्तु दैवं तन्मनसाऽचिन्तयत्प्रभुः}
{दुःखसोकसमाविष्टो निःश्वासपरमोऽभवत्}


\twolineshloka
{अस्त्राणां च प्रणाशेन बाहुवीर्यस्य च क्षयात्}
{धनुषश्चाविधेयत्वाच्छराणां संक्षयेण च}


\twolineshloka
{बभूव विमनाः पार्थो दैवमित्युनुचिन्तयन्}
{न्यवर्तत ततो राजन्नेदमस्तीति चिन्तयन्}


\twolineshloka
{ततः शेषं समादाय कलत्रस्य महामतिः}
{हृतभूयिष्ठरत्नस्य कुरुक्षेत्रमवातरत्}


\twolineshloka
{युधिष्ठिरस्यानुमते वंशकर्तॄन्कुमारकान्}
{न्यवेशयत कौरव्यस्तत्रतत्र धनंजयः}


\threelineshloka
{हार्दिक्यतनयं पार्थो नगरे मृत्तिकावते}
{अश्वपतिं खाण्डवारण्ये राज्ये तत्र न्येवशयत्}
{भोजराजकलत्रं च हृतशेषं नरोत्तमः}


\twolineshloka
{तथा वृद्धांश्च बालांश्च स्त्रियश्चादाय पाण्डवः}
{वीरैर्विहीनान्सर्वांस्ताञ्शक्रप्रस्थे न्यवेशयत्}


\twolineshloka
{यौयुधानिं सरस्वत्याः पुत्रं सात्यकिनः प्रियम्}
{न्यवेशयत धर्मात्मा वृद्धबालपुरस्कृतम्}


\twolineshloka
{इन्द्रप्रस्थे ददौ राज्यं व्रज्राय परवीरहा}
{वज्रेणाक्रूरदारास्तु वार्यमाणाः प्रवव्रजुः}


\twolineshloka
{रुक्मिणी त्वथ गान्धारी शैब्दा हैमवतीत्यपि}
{देवी जींबवती चैव विविशुर्जातवेदसम्}


\twolineshloka
{सत्यभामा तथैवान्या देव्यः कृष्णस्य सम्मताः}
{वनं प्रविविशू राजंस्तापस्ये कृतनिश्चयाः}


\twolineshloka
{द्वारकावासिनो ये पुरुषाः पार्थमभ्ययुः}
{यथार्हं संविभज्यैनान्वज्रे पर्यददज्जयः}


\twolineshloka
{स तत्कृत्वा प्राप्तकालं बाष्णेणापिहितोऽर्जुनः}
{कृष्णद्वैपायनं व्यासं ददर्शासीनमाश्रमे}


\chapter{अध्यायः ९}
\twolineshloka
{प्रविश्य त्वर्जुनो राजन्नाश्रमं सत्यवादिनः}
{ददर्शासीनमेकान्तो मुनिं सत्यवतीसुतम्}


\twolineshloka
{स तमासाद्य धर्मज्ञमुपतस्थे महाव्रतम्}
{अर्जुनोस्मीति नामास्तै निवेद्याभ्यवदत्ततः}


\twolineshloka
{स्वागतं तेऽस्त्विति प्राह मुनिः सत्यवतीसुतः}
{आस्यतामिति होवाच प्रसन्नात्मा महामुनिः}


\twolineshloka
{तमप्रतीतमनसं निःश्वसन्तं पुनःपुनः}
{निर्विष्णमनसं दृष्ट्वा पार्थं व्यासोऽब्रवीदिदम्}


\twolineshloka
{नखकेशदशाकुंभवारिणा किं समुक्षितः}
{आवीरजानुगमनं ब्राह्मणो वा हतस्त्वया}


\fourlineindentedshloka
{युद्धे पराजितो वाऽसि दतश्रीरिव लक्ष्यसे}
{न त्वां प्रभिन्नं जानामि किमिदं भरतर्षभ}
{श्रोतव्यं चेन्मया पार्थ क्षश्रिप्रमाख्यातुमर्हसि ॥अर्जुन उवाच}
{}


\twolineshloka
{यः स मेघवपुः श्रीमान्बृहत्पङ्कजलोचनः}
{स कृष्णः सह रामेण त्यक्त्वा देहं दिवं गतः}


\twolineshloka
{`तदनुस्मृत्य संमोहं तदा शोकं महामते}
{प्रयामि सर्वदा मह्यं मुमूर्षा चोपजायते}


\twolineshloka
{तद्वाक्यस्पर्शनालोकसुखं चामृतसन्निभम्}
{संस्मृत्य देवदेवस्य प्रमुह्यित्यमृतात्मनः ॥'}


\twolineshloka
{मौसले वृष्णिवीराणां विनाशो ब्रह्मशापजः}
{बभूव वीरान्तकरः प्रभासे रोमहर्षणः}


\twolineshloka
{एते शूरा महात्मानः सिंहदर्पा महाबलाः}
{भोजवृष्ण्यन्धका ब्रह्मन्नन्योन्यं तैर्हतं युधि}


\twolineshloka
{गदापरिघशक्तीनां सहाः परिघबाहवः}
{त एरकाभिर्निहताः पश्य कालस्य पर्ययम्}


\twolineshloka
{हतं पञ्चशतं तेषां सहस्रं बाहुशालिनाम्}
{निधनं समनुप्राप्तं समासाद्येतरेतरम्}


\twolineshloka
{पुनःपुनर्न मृष्यामि विनाशममितौजसाम्}
{चिन्तयानो यदूनां च कृष्णस्य च यशस्विनः}


\twolineshloka
{शोषणं सागरस्येव मन्दरस्येव चालनम्}
{नभसः पतनं चैव शैत्यमग्नेस्तथैव च}


\twolineshloka
{अश्रद्धेयमहं मन्ये विनाशं शार्ङ्गधन्वनः}
{न चेह स्थातुमिच्छामि लोके कृष्णविनाकृतः}


\twolineshloka
{इतः कष्टतरं चान्यच्छृणु तद्वै तपोधन}
{मनो मे दीर्यते येन चिन्तयानस्य वै मुहुः}


\twolineshloka
{पश्यतो वृष्णिदाराश्च मम ब्रह्मन्सहस्रशः}
{आभीरैरभिभूयाजौ हृताः पञ्चनदालयैः}


\twolineshloka
{धनुरादाय तत्राहं नाशकं तस्य पूरणे}
{यथापुरा च मे वीर्यं भुजयोर्न तथाऽभवत्}


\twolineshloka
{अस्त्राणि मे प्रनष्टानि विविधानि महामुने}
{शराश्च क्षयमापन्नाः क्षणेनैव समन्ततः}


\twolineshloka
{पुरुषस्चाप्रमेयात्मा शङ्खचक्रगदाधरः}
{चतुर्भुजः पीतवासाः श्यामः पद्मदलेक्षणः}


\twolineshloka
{यश्च याति पुरस्तान्मे रथस्य सुमहाद्युतिः}
{प्रदहन्रिपुसैन्यानि न पश्याम्यहमद्य तम्}


\twolineshloka
{येन पूर्वं प्रदग्धानि शत्रुसैन्यानि तेजसा}
{शरैर्गाण्डीवनिर्मुक्तैरहं पश्चादशातयम्}


\twolineshloka
{तमपश्यन्विषीदामि घूर्णामीव च सत्तम}
{परिनिर्विण्णचेताश्च शान्तिं नोपलभेऽपि च}


\twolineshloka
{`देवकीनन्दनं देवं वासुदेवमजं प्रभुम्}
{'विना जनार्दनं वीरं नाहं जीवितुमुत्सहे}


\twolineshloka
{श्रुत्वैव हि गतं विष्णुं ममापि मुमुहुर्दिशः}
{प्रनष्टज्ञातिवीर्यस्य शून्यस्य परिधावतः}


\twolineshloka
{उपदेष्टुं मम श्रेयो भवानर्हति सत्तम ॥व्यास उवाच}
{}


\twolineshloka
{`देवांशा देवभूतेन सम्भूतास्ते गतास्सह}
{धर्मव्यवस्थारक्षार्थं देवेन समुपेक्षिताः'}


\twolineshloka
{ब्रह्मशापविनिर्दग्धा वृष्ण्यन्धकसमहारथाः}
{विनष्टाः कुरुशार्दूल न ताञ्शोचितुमर्हसि}


\twolineshloka
{भवितव्यं तथा तच्च दिष्टमेतन्महात्मनाम्}
{उपेक्षितं च कृष्णेन शक्तेनापि व्यपोहितुम्}


\twolineshloka
{त्रैलोक्यमपि गोविन्दः कृत्स्नं स्थावरजङ्गमम्}
{प्रसहेदन्यथा कर्तुं कुतः शापं महात्मनाम्}


\twolineshloka
{`स्त्रियश्च ताः पुरा शप्ताः प्रभासे कुपितेन वै}
{अष्टावक्रेण मुनिना तदर्थं त्वद्बलक्षयम् ॥'}


\twolineshloka
{रथस्य पुरतो याति यः स चक्रगदाधरः}
{तव स्नेहात्पुराणर्षिर्वासुदेवश्चतुर्भुजः}


\twolineshloka
{कृत्वा भारावतरणं पृथिव्याः पृथुलोचनः}
{मोक्षयित्वा तनुं प्राप्तः कृष्णः स्वस्थानमुत्तमम्}


\twolineshloka
{त्वयाऽपीह महत्कर्म देवानां पुरुषर्षभ}
{कृतं भीमसहायेन यमाभ्यां च महाभुज}


\twolineshloka
{कृतकृत्यांश्च वो मन्ये संसिद्धान्कुरुपुङ्गव}
{गमनं प्राप्तकालं व इदं श्रेयस्करं विभो}


\twolineshloka
{बलं बुद्धिश्च तेजश्च प्रतिपत्तिश्च भारत}
{भवन्ति भवकालेषु विपद्यन्ते विपर्यये}


\twolineshloka
{कालमूलमिदं सर्वं जगद्बीजं धनंजय}
{काल एव समादत्ते पुनरेव यदृच्छया}


\twolineshloka
{स एव बलवान्भूत्वा पुनर्भवति दुर्बलः}
{स एवेशश्च भूत्वेह परैराज्ञाप्यते पुनः}


\twolineshloka
{कृतकृत्यानि चास्त्राणि गतान्यद्य यथागतम्}
{पुनरेष्यन्ति ते हस्तं यदा कालो भविष्यति}


\threelineshloka
{कालो गन्तुं गतिं मुख्यां भवतामपि भारत}
{एतच्छ्रेयो हि वो मन्ये परमं भरतर्षभ ॥वैशम्पायन उवाच}
{}


\twolineshloka
{एतद्वचनमाज्ञाय व्यासस्यामिततेजसः}
{अनुज्ञातो ययौ पार्थो नगरं नागसाह्वयम्}


\twolineshloka
{प्रविश्य च पुरीं वीरः समासाद्य युधिष्ठिरम्}
{आचष्ट तद्यथावृत्तं वृष्ण्यन्धककुलं प्रति}


