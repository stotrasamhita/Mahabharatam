\part{कर्णपर्व}
\chapter{अध्यायः १}
\twolineshloka
{श्रीवेदव्यासाय नमः}
{}


\fourlineindentedshloka
{नारायणं नमस्कृत्य नरं चैव नरोत्तमम्}
{देवीं सरस्वतीं व्यासं ततो जयमुदीरयेत् ॥`वैशम्पायन उवाच}
{शिबिराद्धास्तिनपुरं प्राप्य भारत सञ्जयः}
{प्रविवेश महाबाहुर्धृतराष्ट्रनिवेशनम्}


\twolineshloka
{शोकेनापहतः सूतो विलपन्भृशदुःखितः}
{चिन्तयन्निधनं घोरं सूतपुत्रस्य पाण्डवैः}


\twolineshloka
{अन्तःपुरं प्रविश्यैव सञ्जयो राजसत्तमम्}
{अश्रुपूर्णो भृशं त्रस्तो राजानमुपजग्मिवान्}


\threelineshloka
{उपस्थाय च राजानं विनिश्वस्य च सूतजः}
{नातिहृष्टमना राजन्निदं वचनमब्रवीत् ॥सञ्जय उवाच}
{}


\twolineshloka
{सञ्चयोऽहं महाराज नमस्ते भरतर्षभ}
{हतो वैकर्तनः कर्णः कृत्वा कर्म सुदुष्करम्}


\twolineshloka
{चेदिकाशिकरूशानां मत्स्यानां सोमकैः सह}
{कृत्वाऽसौ कदनं शेते वातनुन्न इव द्रुमः}


\twolineshloka
{गोष्ठमध्येव ऋषभो गोव्रजैः परिवारितः}
{व्यालेन निहतो यद्वत्तथाऽसौ निहतः परैः}


\twolineshloka
{निराशान्पाण्डवान्कृत्वा दृढं राजन्ससात्यकान्}
{पाञ्चालानां रथांश्चैव विनिहत्य सहस्रशः}


\twolineshloka
{जित्वा शूरान्महेष्वासान्विद्राव्य च दिशोदश}
{हतो वैकर्तनः कर्णः पाण्डवेन किरीटिना}


\twolineshloka
{वैरस्य गत आनृण्यं दुर्गमस्य दुरात्मभिः}
{हत्वा कर्णं महाराज विशल्यः पाण्डवोऽभवत्}


\twolineshloka
{शोषणं सागराणां वा पतनं वा विवस्वतः}
{विशीर्णत्वं यथा मेरोस्तथा कर्णस्य पातनम्}


\twolineshloka
{योधाश्च बहवो राजन्हतास्तत्र जयैषिणः}
{राजानो राजपुत्राश्च शूराः परिघबाहवः}


\twolineshloka
{रथौघाश्च नरौघाश्च हता राजन्सहस्रशः}
{वारणा निहतास्तत्र वाजिनश्च महाहवे}


\twolineshloka
{क्षत्रियाश्च महाराज सेनयोरुभयोर्हताः}
{परस्परं वै वीक्ष्यात्र परस्परकृतागसः}


\twolineshloka
{किञ्चिच्छेषान्परान्कृत्वा तीर्त्वा पाण्डववाहिनीम्}
{पार्थवेलां समासाद्य हतो वैकर्तनो वृषा}


\twolineshloka
{जयाशा धार्तराष्ट्राणां वैरस्य च मुखं नृप}
{तीर्णं तत्पाण्डवै राजन्यत्पुरा नावबुध्यसे}


\twolineshloka
{प्रोच्यमानं महाराज बन्धुभिर्हितबुद्धिभिः}
{तदिदं समनुप्राप्तं व्यसनं त्वां महाभयम्}


\twolineshloka
{पुत्राणां राज्यकामेन त्वया राजन्हितैषिणा}
{चरितान्यहितान्येव तेषां ते फलमागतम्}


\twolineshloka
{हतो दुःशासनो राजन्यथोक्तं पाण्डवेन तु}
{प्रतिज्ञा भीमसेनेन निस्तीर्णा सा चमूमुखे}


\threelineshloka
{पीतं च क्षतजं तस्य धार्तराष्ट्रस्य संयुगे}
{पाण्डवेन महाराज कर्म कृत्वा सुदुष्करम्*' ॥वैशम्पायन उवाच}
{}


\threelineshloka
{एतच्छ्रुत्वा महाराज धृतराष्ट्रोऽम्बिकासुतः}
{शोकस्यान्तमपश्यन्वै हतं मत्वा सुयोधनम्}
{विह्वलः पतितो भूमौ नष्टचेता इव द्विपः}


\twolineshloka
{तस्मिन्निपतिते भूमौ विह्वले राजसत्तमे}
{आर्तनादो महानासीत्स्त्रीणां भरतसत्तम}


% Check verse!
श शब्दः पृथिवीं कृत्स्नां पूरयामास सर्वशः
\twolineshloka
{शोकार्णवे महाघोरे निमग्ना भरतस्त्रियः}
{रुरुदुर्दुःखशोकार्ता भृशमुद्विग्नचेतसः}


\twolineshloka
{राजानं च समासाद्य गान्धारी भरतर्षभ}
{निःसंज्ञा पतिता भूमौ सर्वाण्यन्तः पुराणि च}


\twolineshloka
{ततस्ताः स़ञ्जयो राजन्समाश्वासयदातुराः}
{मुह्यामानाः सुबहुशो मुञ्चन्तीर्वारि नेत्रजम्}


\twolineshloka
{समाश्वस्ताः स्त्रियस्तास्तु वेपमाना मुहुर्मुहुः}
{कदल्य इव वातेन धूयमानाः समन्ततः}


\twolineshloka
{राजानं विदुरश्चापि प्रज्ञाचक्षुषमीश्वरम्}
{आश्वासयामास तदा सिञ्चंस्तोयेन कौरवम्}


\twolineshloka
{स लब्ध्वा सनकैः संज्ञां ताश्च दृष्ट्वा स्त्रियो नृपः}
{उन्मत्त इव राजेन्द्र स्थितस्तूष्णीं विशाम्पते}


\twolineshloka
{ततो ध्यात्वा चिरं कालं निःश्वस्य च पुनःपुनः}
{स्वान्पुत्रान्गर्हयामास बहुमेने च पाण्डवान्}


\twolineshloka
{गर्हयंश्चात्मनो बुद्धिं शकुनेः सौबलस्य च}
{ध्यात्वा तु सुचिरं कालं वेपमानो मुहुर्मुहुः}


\twolineshloka
{संस्तभ्य च मनो भूयो राजा धैर्यसमन्वितः}
{पुनर्गावल्गणिं सूतं पर्यपृच्छत सञ्जयम्}


% Check verse!
न त्वया कथितं वाक्यं श्रुतं सञ्जय तन्मया
\fourlineindentedshloka
{कच्चिद्दुर्योधनः सूत न गतो वै यमक्षयम्}
{जये निराशः पुत्रो मे सततं जयकामुकः}
{ब्रूहि सञ्जय तत्त्वेन पुनरुक्तां कथामिमाम् ॥वैशम्पायन उवाच}
{}


\threelineshloka
{एवमुक्तोऽब्रवीत्सूतो राजानं जनमेजय}
{हतो वैकर्तनो राजन्सहपुत्रैर्महारथः}
{भ्रातृभिश्च महेष्वासैः सूतपुत्रैस्तनुत्यजैः}


\threelineshloka
{दुःशासनश्च निहतः पाण्डवेन यशस्विना}
{पीतं च रुधिरं कोपाद्भीमसेनेन संयुगे ॥`वैशम्पायन उवाच}
{}


\threelineshloka
{एतच्छ्रुत्वा महाराज धृतराष्ट्रोऽम्बिकासुतः}
{दह्यमानोऽब्रवीत्सूतं मुहूर्तं तिष्ठ सञ्जय}
{व्याकुलं मे मनस्तात मा तावत्किंचिदुच्यताम्}


\twolineshloka
{राजापि नाब्रवीत्किञ्चित्सञ्जयो विदुरस्तथा}
{तूष्णीम्भूतस्तदा सोऽथ बभूव जगतीपतिः'}


% Check verse!
श्रीकृष्णाय नमः
\threelineshloka
{[नारायणं नमस्कृत्य नरं चैव नरोत्तमम्}
{देवीं सरस्वतीं चैव ततो जयमुदीरयेत् ॥वैशम्पायन उवाच}
{}


\twolineshloka
{ततो द्रोणे हते राजन्दुर्योधनमुखा नृपाः}
{भृशमुद्विग्नमनसो द्रोणपुत्रमुपागमन्}


\twolineshloka
{ते द्रोणमनुशोचन्तः कश्मलाभिहतौजसः}
{पर्युपासन्त शोकार्तास्ततः शारद्वतीसुतम्}


\twolineshloka
{ते मुहूर्तं समाश्वस्य हेतुभिः शास्त्रसम्मितैः}
{रात्र्यागमे महीपालाः स्वानि वेश्मानि भेजिरे}


\twolineshloka
{ते वेश्मस्वपि कौरव्य पृथ्वीशा नाप्नुवन्सुखम्}
{चिन्तयन्तः क्षयं तीव्रं दुःखशोकसमन्विताः}


\twolineshloka
{विशेषतः सूतपुत्रो राजा चैव सुयोधनः}
{दुःशासनश्च शकुनिः सौबलश्च महाबलः}


\twolineshloka
{उषितास्ते निशां तां तु दुर्योधननिवेशने}
{चिन्तयन्तः परिक्लेशान्पाण्डवानां महात्मनाम्}


\twolineshloka
{यत्तद्द्यूते परिक्लिष्टा कृष्णा चानायिता सभाम्}
{तत्स्मरन्तोऽनुशोचन्तो भृशमुद्विग्नचतेतसः}


\twolineshloka
{तथा तु सञ्चिन्तयतां तान्क्लेशान्द्यूतकारितान्}
{दुःखेन क्षणदा राजञ्जगामाब्दशतोपमा}


\twolineshloka
{ततः प्रभाते विमले स्थिता दिष्टस्य शासने}
{चक्रुरावश्यकं सर्वे विधिदृष्टेन कर्मणा}


\twolineshloka
{ते कृत्वाऽवश्यकार्याणि समाश्वस्य च भारत}
{योगमाज्ञापयामासुर्युद्वाय च विनिर्ययुः}


\twolineshloka
{कर्णं सेनापतिं कृत्वा कृतकौतुकमङ्गलाः}
{पूजयित्वा द्विजश्रेष्ठान्दधिपात्रघृताक्षतैः}


\twolineshloka
{गोभिरश्वैश्च निष्कैश्च वासोभिश्च महाधनैः}
{वन्द्यमाना जयाशीर्भिः सूतमागधबन्दिभिः}


\twolineshloka
{तथैव पाण्डवा राजन्कृतपूर्वाह्णिकक्रियाः}
{शिबिरान्निर्ययुस्तूर्णं युद्वाय कृतनिश्चयाः}


\twolineshloka
{ततः प्रववृते युद्वं तुमुलं रोमहर्षणम्}
{कुरूणां पाण्डवानां च परस्परजयैषिणाम्}


\twolineshloka
{तयोर्द्वौ दिवसौ युद्वं कुरुपाण्डवसेनयोः}
{कर्णे सेनापतौ राजन्बभूवाद्भुतदर्शनम्}


\twolineshloka
{ततः शत्रुक्षयं कृत्वा सुमहान्तं रणे वृषः}
{पश्यतां धार्तराष्ट्राणां फल्गुनेन निपातितः}


\threelineshloka
{ततस्तु सञ्जयः सर्वं गत्वा नागपुरं द्रुतम्}
{आचष्ट धृतराष्ट्राय यद्वृत्तं कुरुजाङ्गले ॥जनमयेजय उवाच}
{}


\twolineshloka
{आपगेयं हतं श्रुत्वा द्रोणं चापि महारथम्}
{आजगाम परामार्तिं वृद्वो राजाऽम्बिकासुतः}


\twolineshloka
{स श्रुत्वा निहतं कर्णं दुर्योधनहितैषिणम्}
{कथं द्विजवर प्राणानधारयत दुःखितः}


\twolineshloka
{यस्मिञ्जयाशां पुत्राणां सममन्यत पार्थिवः}
{तस्मिन्हते स कौरव्यः कथं प्राणानधारयत्}


\twolineshloka
{दुर्मरं तदहं मन्ये नृणां कृच्छ्रेऽपि वर्तताम्}
{यत्र कर्णं हतं श्रुत्वा नात्यजज्जीवितं नृपः}


\twolineshloka
{तथा शान्तनवं वृद्वं ब्रह्मन्बाह्लीकमेव च}
{द्रोणं च सोमदत्तं च भूरिश्रवसमेव च}


\twolineshloka
{तथैव चान्यान्सुहृदः पुत्रान्पौत्रांश्च पातितान्}
{श्रुत्वा यन्नाजहात्प्राणांस्तन्मन्ये दुष्करं द्विज}


\twolineshloka
{एतन्मे सर्वमाचक्ष्व विस्तरेण महामुने}
{न हि तृप्यामि पूर्वेषां शृण्वानश्चरितं महत्}


\twolineshloka
{वैशम्पायन उवाच}
{}


\twolineshloka
{हते कर्णे महाराज निशि गावल्गणिस्तदा}
{दीनो ययौ नागपुरमश्वैर्वातसमैर्जवे}


\twolineshloka
{स हास्तिनपुरं गत्वा भृशमुद्विग्नचेतनः}
{जगाम धृतराष्ट्रस्य क्षयं प्रक्षीणबान्धवम्}


\twolineshloka
{स तमुद्वीक्ष्य राजानं कश्मलाभिहतौजसम्}
{ववन्दे प्राञ्जलिर्भूत्वा मूर्ध्ना पादौ नृपस्य ह}


\twolineshloka
{सम्पूज्य च यथान्यायं धृतराष्ट्रं महीपतिम्}
{हा कष्टमिति चोक्त्वा स ततो वचनमाददे}


\twolineshloka
{सञ्जयोऽहं क्षितिपते कच्चिदास्ते सुखं भवान्}
{स्वदोषैरापदं प्राप्य कच्चिन्नाद्य विमुह्यति}


\twolineshloka
{हितान्युक्तानि विदुरद्रोणगाङ्गेयकेशवैः}
{अगृहीतान्यनुस्मृत्य कच्चिन्न कुरुषे व्यथाम्}


\twolineshloka
{रामनारदकण्वाद्यैर्हितमुक्तं सभातले}
{न गृहीतमनुस्मृत्य कच्चिन्न कुरुषे व्यथाम्}


\twolineshloka
{सुहृदस्त्वद्विते युक्तान्भीष्मद्रोणमुखान्परैः}
{निहतान्युधि संस्मृत्य कच्चिन्न कुरुषे व्यथाम्}


\threelineshloka
{तमेवंवादिनं राजा सूतपुत्रं कृताजलिम्}
{सुदीर्घमथ निःश्वस्य दुःखार्त इदमब्रवीत् ॥धृतराष्ट्र उवाच}
{}


\twolineshloka
{आपगेये हते शूरे दिव्यास्त्रवति सञ्जय}
{द्रोणे च परमेष्वासे भृशं मे व्यथितं मनः}


\twolineshloka
{यो रथानां सहस्राणि दंशितानां दशैव तु}
{अहन्यहनि तेजस्वी निजघ्ने वसुसम्भवः}


\twolineshloka
{तं हतं यज्ञसेनस्य पुत्रेणेह शिखण्डिना}
{पाण्डवेयाभिगुप्तेन श्रुत्वा मे व्यथितं मनः}


\twolineshloka
{भार्गवः प्रददौ यस्मै परमास्त्रं महाहवे}
{साक्षाद्रामणे यो बाल्ये धनुर्वेद उपाकृतः}


\twolineshloka
{यस्य प्रसादात्कौन्तेया राजपुत्रा महारथाः}
{महारथत्वं सम्प्राप्तास्तथाऽन्ये वसुधाधिपाः}


\twolineshloka
{तं द्रोणं निहतं श्रुत्वा धृष्टद्युम्नेन संयुगे}
{सत्यसन्धं महेष्वासं भृशं मे व्यथितं मनः}


\twolineshloka
{ययोर्लोके पुमानस्त्रे न समोऽस्ति चतुर्विधे}
{तौ द्रोणभीष्मौ श्रुत्वा तु हतौ मे व्यथितं मनः}


\twolineshloka
{त्रैलोक्ये यस्य चास्त्रेषु न पुमान्विद्यते समः}
{तं द्रोणं निहतं श्रुत्वा किमकुर्वत मामकाः}


\twolineshloka
{संशप्तकानां च बले पाण्डवेन महात्मना}
{धनञ्जयेन विक्रम्य गमिते यमसादनम्}


\threelineshloka
{नारायणास्त्रे च हते द्रोणपुत्रस्य धीमतः}
{विप्रद्रुतानहं मन्येनिमग्नाञ्शोकसागरे}
{}


\twolineshloka
{प्लवमानान्हते द्रोणे सन्ननौकानिवार्णवे ॥दुर्योधनस्य कर्णस्य भोजस्य कृतवर्मणः}
{}


\twolineshloka
{मद्रराजस्य शल्यस्य द्रौणेश्चैव कृपस्य च ॥मत्पुत्रस्य च शेषस्य तथाऽन्येषां च सञ्जय}
{}


\twolineshloka
{विप्रद्रुतेष्वनीकेषु मुखवर्णोऽभवत्कथम् ॥एतत्सर्वं यथावृत्तं तथा गावल्गणे मम}
{}


\twolineshloka
{आचक्ष्व पाण्डवेयानां मामकानां च विक्रमम् ॥सञ्जय उवाच}
{}


\twolineshloka
{तवापराधाद्यद्वृत्तं कौरवेयेषु मारिष}
{तच्छ्रुत्वा मा व्यथां कार्षीर्दिष्टे न व्यथते बुधः}


\threelineshloka
{यस्मादभावी भावी वा भवेदर्थो नरं प्रति}
{अप्राप्तौ तस्य वा प्राप्तौ न कश्चिद्यथते बुधः ॥धृतराष्ट्र उवाच}
{}


\twolineshloka
{न व्यथाऽभ्यधिका काचिद्विद्यते मम सञ्जय}
{दिष्टमेतत्पुरा मन्ये कथयस्य यथेच्छकम्}


\twolineshloka
{सञ्जय उवाच}
{}


\twolineshloka
{हते द्रोणे महेष्वासे तव पुत्रा महारथाः}
{बभूवुरस्वस्थमुखा विषण्णा गतचेतसः}


\twolineshloka
{अवाङ्मुखाः शस्त्रभृतः सर्व एव विशाम्पते}
{अवेक्षमाणाः शोकार्ता नाभ्यभाषन्परस्परम्}


\twolineshloka
{तान्दृष्ट्वा व्यथिताकारान्सैन्यानि तव भारत}
{ऊर्ध्वमेव निरैक्षन्त दुःखत्रस्तान्यनेकशः}


\twolineshloka
{शस्त्राण्येषां तु राजेन्द्र शोणिताक्तानि सर्वशः}
{प्राभ्रश्यन्त कराग्रेभ्यो दृष्ट्वा द्रोणं हतं युधि}


\twolineshloka
{तानि बद्वान्यरिष्टानि लम्बमानानि भारत}
{अदृश्यन्त महाराज नक्षत्राणि यथा दिवि}


\twolineshloka
{तथा तु स्तिमितं दृष्ट्वा गतसत्वमवस्थितम्}
{बलं तव महाराज राजा दुर्योधनोऽब्रवीत्}


\twolineshloka
{भवतां बाहुवीर्यं हि समाश्रित्य मया युधि}
{पाण्डवेयाः समाहूता युद्वं चेदं प्रवर्तितम्}


\twolineshloka
{तदिदं निहते द्रोणे विषण्णमिव लक्ष्यते}
{युध्यमानाश्च समरे योधा वध्यन्ति सर्वशः}


\twolineshloka
{जयो वाऽपि वधो वाऽपि युध्यमानस्य संयुगे}
{भवेत्किमत्र चित्रं वै युध्यध्वं सर्वतोमुखाः}


\twolineshloka
{पश्यध्वं च महात्मानं कर्णं वैकर्तनं युधि}
{प्रचरन्तं महेष्वासं दिव्यैरस्त्रैर्महाबलम्}


\twolineshloka
{यस्य वै युधि सन्त्रासात्कुन्तीपुत्रो धनञ्जयः}
{निवर्तते सदा मन्दः सिंहात्क्षुद्रमृगो यथा}


\twolineshloka
{येन नागायुतप्राणो भीमसेनो महाबलः}
{मानुषेणैव युद्धेन तामवस्थां प्रवेशितः}


\twolineshloka
{येन दिव्यास्त्रविच्छूरो मायावी स घटोत्कचः}
{अमोघया रणे शक्त्या निहतो भैरवं नदन्}


\twolineshloka
{तस्य दुर्वारवीर्यस्य सत्यसन्धस्य धीमतः}
{बाह्वोर्द्रविणमक्षय्यमद्य द्रक्ष्यथ संयुगे}


\twolineshloka
{द्रोणपुत्रस्य विक्रान्तं राधेयस्यैव चोभयोः}
{पश्यन्तु पाण्डुपुत्रास्ते विष्णुवासवयोरिव}


\fourlineindentedshloka
{सर्व एव भवन्तश्च शक्ताः प्रत्येकशोऽपि वा}
{पाण्डुपुत्रान्रणे हन्तुं ससैन्यान्किसु संहताः}
{वीर्यवन्तः कृतास्त्राश्च द्रक्ष्यथाद्य परस्परम् ॥सञ्जय उवाच}
{}


\twolineshloka
{एवमुक्त्वा ततः कर्णं चक्रे सेनापतिं तदा}
{तव पुत्रो महावीर्यो भ्रातृभिः सहितोऽनघ}


\twolineshloka
{सैनापत्यमथावाप्य कर्णो राजन्महारथः}
{सिंहनादं विनद्योच्चैः प्रायुध्यत रणोत्कटः}


\twolineshloka
{ससृञ्जयानां सर्वेषां पाञ्चालानां च मारिष}
{केकयानां विदेहानां चकार कदनं महत्}


\twolineshloka
{तस्येषुधाराः शतशः प्रादुरासञ्छरासनात्}
{अग्रे पुङ्खेषु संसक्ता यथा भ्रमरपङ्क्तयः}


\twolineshloka
{स पीडयित्वा पाञ्चालान्पाण्डवांश्च तरस्विनः}
{हत्वा सहस्रशो योधानर्जुनेन निपातितः}


\chapter{अध्यायः २}
% Check verse!
श्रीकृष्णाय नमः
\threelineshloka
{[नारायणं नमस्कृत्य नरं चैव नरोत्तमम्}
{देवीं सरस्वतीं चैव ततो जयमुदीरयेत् ॥वैशम्पायन उवाच}
{}


\twolineshloka
{ततो द्रोणे हते राजन्दुर्योधनमुखा नृपाः}
{भृशमुद्विग्नमनसो द्रोणपुत्रमुपागमन्}


\twolineshloka
{ते द्रोणमनुशोचन्तः कश्मलाभिहतौजसः}
{पर्युपासन्त शोकार्तास्ततः शारद्वतीसुतम्}


\twolineshloka
{ते मुहूर्तं समाश्वस्य हेतुभिः शास्त्रसम्मितैः}
{रात्र्यागमे महीपालाः स्वानि वेश्मानि भेजिरे}


\twolineshloka
{ते वेश्मस्वपि कौरव्य पृथ्वीशा नाप्नुवन्सुखम्}
{चिन्तयन्तः क्षयं तीव्रं दुःखशोकसमन्विताः}


\twolineshloka
{विशेषतः सूतपुत्रो राजा चैव सुयोधनः}
{दुःशासनश्च शकुनिः सौबलश्च महाबलः}


\twolineshloka
{उषितास्ते निशां तां तु दुर्योधननिवेशने}
{चिन्तयन्तः परिक्लेशान्पाण्डवानां महात्मनाम्}


\twolineshloka
{यत्तद्द्यूते परिक्लिष्टा कृष्णा चानायिता सभाम्}
{तत्स्मरन्तोऽनुशोचन्तो भृशमुद्विग्नचतेतसः}


\twolineshloka
{तथा तु सञ्चिन्तयतां तान्क्लेशान्द्यूतकारितान्}
{दुःखेन क्षणदा राजञ्जगामाब्दशतोपमा}


\twolineshloka
{ततः प्रभाते विमले स्थिता दिष्टस्य शासने}
{चक्रुरावश्यकं सर्वे विधिदृष्टेन कर्मणा}


\twolineshloka
{ते कृत्वाऽवश्यकार्याणि समाश्वस्य च भारत}
{योगमाज्ञापयामासुर्युद्वाय च विनिर्ययुः}


\twolineshloka
{कर्णं सेनापतिं कृत्वा कृतकौतुकमङ्गलाः}
{पूजयित्वा द्विजश्रेष्ठान्दधिपात्रघृताक्षतैः}


\twolineshloka
{गोभिरश्वैश्च निष्कैश्च वासोभिश्च महाधनैः}
{वन्द्यमाना जयाशीर्भिः सूतमागधबन्दिभिः}


\twolineshloka
{तथैव पाण्डवा राजन्कृतपूर्वाह्णिकक्रियाः}
{शिबिरान्निर्ययुस्तूर्णं युद्वाय कृतनिश्चयाः}


\twolineshloka
{ततः प्रववृते युद्वं तुमुलं रोमहर्षणम्}
{कुरूणां पाण्डवानां च परस्परजयैषिणाम्}


\twolineshloka
{तयोर्द्वौ दिवसौ युद्वं कुरुपाण्डवसेनयोः}
{कर्णे सेनापतौ राजन्बभूवाद्भुतदर्शनम्}


\twolineshloka
{ततः शत्रुक्षयं कृत्वा सुमहान्तं रणे वृषः}
{पश्यतां धार्तराष्ट्राणां फल्गुनेन निपातितः}


\threelineshloka
{ततस्तु सञ्जयः सर्वं गत्वा नागपुरं द्रुतम्}
{आचष्ट धृतराष्ट्राय यद्वृत्तं कुरुजाङ्गले ॥जनमयेजय उवाच}
{}


\twolineshloka
{आपगेयं हतं श्रुत्वा द्रोणं चापि महारथम्}
{आजगाम परामार्तिं वृद्वो राजाऽम्बिकासुतः}


\twolineshloka
{स श्रुत्वा निहतं कर्णं दुर्योधनहितैषिणम्}
{कथं द्विजवर प्राणानधारयत दुःखितः}


\twolineshloka
{यस्मिञ्जयाशां पुत्राणां सममन्यत पार्थिवः}
{तस्मिन्हते स कौरव्यः कथं प्राणानधारयत्}


\twolineshloka
{दुर्मरं तदहं मन्ये नृणां कृच्छ्रेऽपि वर्तताम्}
{यत्र कर्णं हतं श्रुत्वा नात्यजज्जीवितं नृपः}


\twolineshloka
{तथा शान्तनवं वृद्वं ब्रह्मन्बाह्लीकमेव च}
{द्रोणं च सोमदत्तं च भूरिश्रवसमेव च}


\twolineshloka
{तथैव चान्यान्सुहृदः पुत्रान्पौत्रांश्च पातितान्}
{श्रुत्वा यन्नाजहात्प्राणांस्तन्मन्ये दुष्करं द्विज}


\twolineshloka
{एतन्मे सर्वमाचक्ष्व विस्तरेण महामुने}
{न हि तृप्यामि पूर्वेषां शृण्वानश्चरितं महत्}


\twolineshloka
{वैशम्पायन उवाच}
{}


\twolineshloka
{हते कर्णे महाराज निशि गावल्गणिस्तदा}
{दीनो ययौ नागपुरमश्वैर्वातसमैर्जवे}


\twolineshloka
{स हास्तिनपुरं गत्वा भृशमुद्विग्नचेतनः}
{जगाम धृतराष्ट्रस्य क्षयं प्रक्षीणबान्धवम्}


\twolineshloka
{स तमुद्वीक्ष्य राजानं कश्मलाभिहतौजसम्}
{ववन्दे प्राञ्जलिर्भूत्वा मूर्ध्ना पादौ नृपस्य ह}


\twolineshloka
{सम्पूज्य च यथान्यायं धृतराष्ट्रं महीपतिम्}
{हा कष्टमिति चोक्त्वा स ततो वचनमाददे}


\twolineshloka
{सञ्जयोऽहं क्षितिपते कच्चिदास्ते सुखं भवान्}
{स्वदोषैरापदं प्राप्य कच्चिन्नाद्य विमुह्यति}


\twolineshloka
{हितान्युक्तानि विदुरद्रोणगाङ्गेयकेशवैः}
{अगृहीतान्यनुस्मृत्य कच्चिन्न कुरुषे व्यथाम्}


\twolineshloka
{रामनारदकण्वाद्यैर्हितमुक्तं सभातले}
{न गृहीतमनुस्मृत्य कच्चिन्न कुरुषे व्यथाम्}


\twolineshloka
{सुहृदस्त्वद्विते युक्तान्भीष्मद्रोणमुखान्परैः}
{निहतान्युधि संस्मृत्य कच्चिन्न कुरुषे व्यथाम्}


\threelineshloka
{तमेवंवादिनं राजा सूतपुत्रं कृताजलिम्}
{सुदीर्घमथ निःश्वस्य दुःखार्त इदमब्रवीत् ॥धृतराष्ट्र उवाच}
{}


\twolineshloka
{आपगेये हते शूरे दिव्यास्त्रवति सञ्जय}
{द्रोणे च परमेष्वासे भृशं मे व्यथितं मनः}


\twolineshloka
{यो रथानां सहस्राणि दंशितानां दशैव तु}
{अहन्यहनि तेजस्वी निजघ्ने वसुसम्भवः}


\twolineshloka
{तं हतं यज्ञसेनस्य पुत्रेणेह शिखण्डिना}
{पाण्डवेयाभिगुप्तेन श्रुत्वा मे व्यथितं मनः}


\twolineshloka
{भार्गवः प्रददौ यस्मै परमास्त्रं महाहवे}
{साक्षाद्रामणे यो बाल्ये धनुर्वेद उपाकृतः}


\twolineshloka
{यस्य प्रसादात्कौन्तेया राजपुत्रा महारथाः}
{महारथत्वं सम्प्राप्तास्तथाऽन्ये वसुधाधिपाः}


\twolineshloka
{तं द्रोणं निहतं श्रुत्वा धृष्टद्युम्नेन संयुगे}
{सत्यसन्धं महेष्वासं भृशं मे व्यथितं मनः}


\twolineshloka
{ययोर्लोके पुमानस्त्रे न समोऽस्ति चतुर्विधे}
{तौ द्रोणभीष्मौ श्रुत्वा तु हतौ मे व्यथितं मनः}


\twolineshloka
{त्रैलोक्ये यस्य चास्त्रेषु न पुमान्विद्यते समः}
{तं द्रोणं निहतं श्रुत्वा किमकुर्वत मामकाः}


\twolineshloka
{संशप्तकानां च बले पाण्डवेन महात्मना}
{धनञ्जयेन विक्रम्य गमिते यमसादनम्}


\threelineshloka
{नारायणास्त्रे च हते द्रोणपुत्रस्य धीमतः}
{विप्रद्रुतानहं मन्येनिमग्नाञ्शोकसागरे}
{}


\twolineshloka
{प्लवमानान्हते द्रोणे सन्ननौकानिवार्णवे ॥दुर्योधनस्य कर्णस्य भोजस्य कृतवर्मणः}
{}


\twolineshloka
{मद्रराजस्य शल्यस्य द्रौणेश्चैव कृपस्य च ॥मत्पुत्रस्य च शेषस्य तथाऽन्येषां च सञ्जय}
{}


\twolineshloka
{विप्रद्रुतेष्वनीकेषु मुखवर्णोऽभवत्कथम् ॥एतत्सर्वं यथावृत्तं तथा गावल्गणे मम}
{}


\twolineshloka
{आचक्ष्व पाण्डवेयानां मामकानां च विक्रमम् ॥सञ्जय उवाच}
{}


\twolineshloka
{तवापराधाद्यद्वृत्तं कौरवेयेषु मारिष}
{तच्छ्रुत्वा मा व्यथां कार्षीर्दिष्टे न व्यथते बुधः}


\threelineshloka
{यस्मादभावी भावी वा भवेदर्थो नरं प्रति}
{अप्राप्तौ तस्य वा प्राप्तौ न कश्चिद्यथते बुधः ॥धृतराष्ट्र उवाच}
{}


\twolineshloka
{न व्यथाऽभ्यधिका काचिद्विद्यते मम सञ्जय}
{दिष्टमेतत्पुरा मन्ये कथयस्य यथेच्छकम्}


\twolineshloka
{सञ्जय उवाच}
{}


\twolineshloka
{हते द्रोणे महेष्वासे तव पुत्रा महारथाः}
{बभूवुरस्वस्थमुखा विषण्णा गतचेतसः}


\twolineshloka
{अवाङ्मुखाः शस्त्रभृतः सर्व एव विशाम्पते}
{अवेक्षमाणाः शोकार्ता नाभ्यभाषन्परस्परम्}


\twolineshloka
{तान्दृष्ट्वा व्यथिताकारान्सैन्यानि तव भारत}
{ऊर्ध्वमेव निरैक्षन्त दुःखत्रस्तान्यनेकशः}


\twolineshloka
{शस्त्राण्येषां तु राजेन्द्र शोणिताक्तानि सर्वशः}
{प्राभ्रश्यन्त कराग्रेभ्यो दृष्ट्वा द्रोणं हतं युधि}


\twolineshloka
{तानि बद्वान्यरिष्टानि लम्बमानानि भारत}
{अदृश्यन्त महाराज नक्षत्राणि यथा दिवि}


\twolineshloka
{तथा तु स्तिमितं दृष्ट्वा गतसत्वमवस्थितम्}
{बलं तव महाराज राजा दुर्योधनोऽब्रवीत्}


\twolineshloka
{भवतां बाहुवीर्यं हि समाश्रित्य मया युधि}
{पाण्डवेयाः समाहूता युद्वं चेदं प्रवर्तितम्}


\twolineshloka
{तदिदं निहते द्रोणे विषण्णमिव लक्ष्यते}
{युध्यमानाश्च समरे योधा वध्यन्ति सर्वशः}


\twolineshloka
{जयो वाऽपि वधो वाऽपि युध्यमानस्य संयुगे}
{भवेत्किमत्र चित्रं वै युध्यध्वं सर्वतोमुखाः}


\twolineshloka
{पश्यध्वं च महात्मानं कर्णं वैकर्तनं युधि}
{प्रचरन्तं महेष्वासं दिव्यैरस्त्रैर्महाबलम्}


\twolineshloka
{यस्य वै युधि सन्त्रासात्कुन्तीपुत्रो धनञ्जयः}
{निवर्तते सदा मन्दः सिंहात्क्षुद्रमृगो यथा}


\twolineshloka
{येन नागायुतप्राणो भीमसेनो महाबलः}
{मानुषेणैव युद्धेन तामवस्थां प्रवेशितः}


\twolineshloka
{येन दिव्यास्त्रविच्छूरो मायावी स घटोत्कचः}
{अमोघया रणे शक्त्या निहतो भैरवं नदन्}


\twolineshloka
{तस्य दुर्वारवीर्यस्य सत्यसन्धस्य धीमतः}
{बाह्वोर्द्रविणमक्षय्यमद्य द्रक्ष्यथ संयुगे}


\twolineshloka
{द्रोणपुत्रस्य विक्रान्तं राधेयस्यैव चोभयोः}
{पश्यन्तु पाण्डुपुत्रास्ते विष्णुवासवयोरिव}


\fourlineindentedshloka
{सर्व एव भवन्तश्च शक्ताः प्रत्येकशोऽपि वा}
{पाण्डुपुत्रान्रणे हन्तुं ससैन्यान्किसु संहताः}
{वीर्यवन्तः कृतास्त्राश्च द्रक्ष्यथाद्य परस्परम् ॥सञ्जय उवाच}
{}


\twolineshloka
{एवमुक्त्वा ततः कर्णं चक्रे सेनापतिं तदा}
{तव पुत्रो महावीर्यो भ्रातृभिः सहितोऽनघ}


\twolineshloka
{सैनापत्यमथावाप्य कर्णो राजन्महारथः}
{सिंहनादं विनद्योच्चैः प्रायुध्यत रणोत्कटः}


\twolineshloka
{ससृञ्जयानां सर्वेषां पाञ्चालानां च मारिष}
{केकयानां विदेहानां चकार कदनं महत्}


\twolineshloka
{तस्येषुधाराः शतशः प्रादुरासञ्छरासनात्}
{अग्रे पुङ्खेषु संसक्ता यथा भ्रमरपङ्क्तयः}


\twolineshloka
{स पीडयित्वा पाञ्चालान्पाण्डवांश्च तरस्विनः}
{हत्वा सहस्रशो योधानर्जुनेन निपातितः}


\chapter{अध्यायः ३}
% Check verse!
श्रीकृष्णाय नमः
\threelineshloka
{[नारायणं नमस्कृत्य नरं चैव नरोत्तमम्}
{देवीं सरस्वतीं चैव ततो जयमुदीरयेत् ॥वैशम्पायन उवाच}
{}


\twolineshloka
{ततो द्रोणे हते राजन्दुर्योधनमुखा नृपाः}
{भृशमुद्विग्नमनसो द्रोणपुत्रमुपागमन्}


\twolineshloka
{ते द्रोणमनुशोचन्तः कश्मलाभिहतौजसः}
{पर्युपासन्त शोकार्तास्ततः शारद्वतीसुतम्}


\twolineshloka
{ते मुहूर्तं समाश्वस्य हेतुभिः शास्त्रसम्मितैः}
{रात्र्यागमे महीपालाः स्वानि वेश्मानि भेजिरे}


\twolineshloka
{ते वेश्मस्वपि कौरव्य पृथ्वीशा नाप्नुवन्सुखम्}
{चिन्तयन्तः क्षयं तीव्रं दुःखशोकसमन्विताः}


\twolineshloka
{विशेषतः सूतपुत्रो राजा चैव सुयोधनः}
{दुःशासनश्च शकुनिः सौबलश्च महाबलः}


\twolineshloka
{उषितास्ते निशां तां तु दुर्योधननिवेशने}
{चिन्तयन्तः परिक्लेशान्पाण्डवानां महात्मनाम्}


\twolineshloka
{यत्तद्द्यूते परिक्लिष्टा कृष्णा चानायिता सभाम्}
{तत्स्मरन्तोऽनुशोचन्तो भृशमुद्विग्नचतेतसः}


\twolineshloka
{तथा तु सञ्चिन्तयतां तान्क्लेशान्द्यूतकारितान्}
{दुःखेन क्षणदा राजञ्जगामाब्दशतोपमा}


\twolineshloka
{ततः प्रभाते विमले स्थिता दिष्टस्य शासने}
{चक्रुरावश्यकं सर्वे विधिदृष्टेन कर्मणा}


\twolineshloka
{ते कृत्वाऽवश्यकार्याणि समाश्वस्य च भारत}
{योगमाज्ञापयामासुर्युद्वाय च विनिर्ययुः}


\twolineshloka
{कर्णं सेनापतिं कृत्वा कृतकौतुकमङ्गलाः}
{पूजयित्वा द्विजश्रेष्ठान्दधिपात्रघृताक्षतैः}


\twolineshloka
{गोभिरश्वैश्च निष्कैश्च वासोभिश्च महाधनैः}
{वन्द्यमाना जयाशीर्भिः सूतमागधबन्दिभिः}


\twolineshloka
{तथैव पाण्डवा राजन्कृतपूर्वाह्णिकक्रियाः}
{शिबिरान्निर्ययुस्तूर्णं युद्वाय कृतनिश्चयाः}


\twolineshloka
{ततः प्रववृते युद्वं तुमुलं रोमहर्षणम्}
{कुरूणां पाण्डवानां च परस्परजयैषिणाम्}


\twolineshloka
{तयोर्द्वौ दिवसौ युद्वं कुरुपाण्डवसेनयोः}
{कर्णे सेनापतौ राजन्बभूवाद्भुतदर्शनम्}


\twolineshloka
{ततः शत्रुक्षयं कृत्वा सुमहान्तं रणे वृषः}
{पश्यतां धार्तराष्ट्राणां फल्गुनेन निपातितः}


\threelineshloka
{ततस्तु सञ्जयः सर्वं गत्वा नागपुरं द्रुतम्}
{आचष्ट धृतराष्ट्राय यद्वृत्तं कुरुजाङ्गले ॥जनमयेजय उवाच}
{}


\twolineshloka
{आपगेयं हतं श्रुत्वा द्रोणं चापि महारथम्}
{आजगाम परामार्तिं वृद्वो राजाऽम्बिकासुतः}


\twolineshloka
{स श्रुत्वा निहतं कर्णं दुर्योधनहितैषिणम्}
{कथं द्विजवर प्राणानधारयत दुःखितः}


\twolineshloka
{यस्मिञ्जयाशां पुत्राणां सममन्यत पार्थिवः}
{तस्मिन्हते स कौरव्यः कथं प्राणानधारयत्}


\twolineshloka
{दुर्मरं तदहं मन्ये नृणां कृच्छ्रेऽपि वर्तताम्}
{यत्र कर्णं हतं श्रुत्वा नात्यजज्जीवितं नृपः}


\twolineshloka
{तथा शान्तनवं वृद्वं ब्रह्मन्बाह्लीकमेव च}
{द्रोणं च सोमदत्तं च भूरिश्रवसमेव च}


\twolineshloka
{तथैव चान्यान्सुहृदः पुत्रान्पौत्रांश्च पातितान्}
{श्रुत्वा यन्नाजहात्प्राणांस्तन्मन्ये दुष्करं द्विज}


\twolineshloka
{एतन्मे सर्वमाचक्ष्व विस्तरेण महामुने}
{न हि तृप्यामि पूर्वेषां शृण्वानश्चरितं महत्}


\chapter{अध्यायः ४}
\twolineshloka
{वैशम्पायन उवाच}
{}


\twolineshloka
{हते कर्णे महाराज निशि गावल्गणिस्तदा}
{दीनो ययौ नागपुरमश्वैर्वातसमैर्जवे}


\twolineshloka
{स हास्तिनपुरं गत्वा भृशमुद्विग्नचेतनः}
{जगाम धृतराष्ट्रस्य क्षयं प्रक्षीणबान्धवम्}


\twolineshloka
{स तमुद्वीक्ष्य राजानं कश्मलाभिहतौजसम्}
{ववन्दे प्राञ्जलिर्भूत्वा मूर्ध्ना पादौ नृपस्य ह}


\twolineshloka
{सम्पूज्य च यथान्यायं धृतराष्ट्रं महीपतिम्}
{हा कष्टमिति चोक्त्वा स ततो वचनमाददे}


\twolineshloka
{सञ्जयोऽहं क्षितिपते कच्चिदास्ते सुखं भवान्}
{स्वदोषैरापदं प्राप्य कच्चिन्नाद्य विमुह्यति}


\twolineshloka
{हितान्युक्तानि विदुरद्रोणगाङ्गेयकेशवैः}
{अगृहीतान्यनुस्मृत्य कच्चिन्न कुरुषे व्यथाम्}


\twolineshloka
{रामनारदकण्वाद्यैर्हितमुक्तं सभातले}
{न गृहीतमनुस्मृत्य कच्चिन्न कुरुषे व्यथाम्}


\twolineshloka
{सुहृदस्त्वद्विते युक्तान्भीष्मद्रोणमुखान्परैः}
{निहतान्युधि संस्मृत्य कच्चिन्न कुरुषे व्यथाम्}


\threelineshloka
{तमेवंवादिनं राजा सूतपुत्रं कृताजलिम्}
{सुदीर्घमथ निःश्वस्य दुःखार्त इदमब्रवीत् ॥धृतराष्ट्र उवाच}
{}


\twolineshloka
{आपगेये हते शूरे दिव्यास्त्रवति सञ्जय}
{द्रोणे च परमेष्वासे भृशं मे व्यथितं मनः}


\twolineshloka
{यो रथानां सहस्राणि दंशितानां दशैव तु}
{अहन्यहनि तेजस्वी निजघ्ने वसुसम्भवः}


\twolineshloka
{तं हतं यज्ञसेनस्य पुत्रेणेह शिखण्डिना}
{पाण्डवेयाभिगुप्तेन श्रुत्वा मे व्यथितं मनः}


\twolineshloka
{भार्गवः प्रददौ यस्मै परमास्त्रं महाहवे}
{साक्षाद्रामणे यो बाल्ये धनुर्वेद उपाकृतः}


\twolineshloka
{यस्य प्रसादात्कौन्तेया राजपुत्रा महारथाः}
{महारथत्वं सम्प्राप्तास्तथाऽन्ये वसुधाधिपाः}


\twolineshloka
{तं द्रोणं निहतं श्रुत्वा धृष्टद्युम्नेन संयुगे}
{सत्यसन्धं महेष्वासं भृशं मे व्यथितं मनः}


\twolineshloka
{ययोर्लोके पुमानस्त्रे न समोऽस्ति चतुर्विधे}
{तौ द्रोणभीष्मौ श्रुत्वा तु हतौ मे व्यथितं मनः}


\twolineshloka
{त्रैलोक्ये यस्य चास्त्रेषु न पुमान्विद्यते समः}
{तं द्रोणं निहतं श्रुत्वा किमकुर्वत मामकाः}


\twolineshloka
{संशप्तकानां च बले पाण्डवेन महात्मना}
{धनञ्जयेन विक्रम्य गमिते यमसादनम्}


\threelineshloka
{नारायणास्त्रे च हते द्रोणपुत्रस्य धीमतः}
{विप्रद्रुतानहं मन्येनिमग्नाञ्शोकसागरे}
{}


\twolineshloka
{प्लवमानान्हते द्रोणे सन्ननौकानिवार्णवे ॥दुर्योधनस्य कर्णस्य भोजस्य कृतवर्मणः}
{}


\twolineshloka
{मद्रराजस्य शल्यस्य द्रौणेश्चैव कृपस्य च ॥मत्पुत्रस्य च शेषस्य तथाऽन्येषां च सञ्जय}
{}


\twolineshloka
{विप्रद्रुतेष्वनीकेषु मुखवर्णोऽभवत्कथम् ॥एतत्सर्वं यथावृत्तं तथा गावल्गणे मम}
{}


\twolineshloka
{आचक्ष्व पाण्डवेयानां मामकानां च विक्रमम् ॥सञ्जय उवाच}
{}


\twolineshloka
{तवापराधाद्यद्वृत्तं कौरवेयेषु मारिष}
{तच्छ्रुत्वा मा व्यथां कार्षीर्दिष्टे न व्यथते बुधः}


\threelineshloka
{यस्मादभावी भावी वा भवेदर्थो नरं प्रति}
{अप्राप्तौ तस्य वा प्राप्तौ न कश्चिद्यथते बुधः ॥धृतराष्ट्र उवाच}
{}


\twolineshloka
{न व्यथाऽभ्यधिका काचिद्विद्यते मम सञ्जय}
{दिष्टमेतत्पुरा मन्ये कथयस्य यथेच्छकम्}


\chapter{अध्यायः ५}
\twolineshloka
{सञ्जय उवाच}
{}


\twolineshloka
{हते द्रोणे महेष्वासे तव पुत्रा महारथाः}
{बभूवुरस्वस्थमुखा विषण्णा गतचेतसः}


\twolineshloka
{अवाङ्मुखाः शस्त्रभृतः सर्व एव विशाम्पते}
{अवेक्षमाणाः शोकार्ता नाभ्यभाषन्परस्परम्}


\twolineshloka
{तान्दृष्ट्वा व्यथिताकारान्सैन्यानि तव भारत}
{ऊर्ध्वमेव निरैक्षन्त दुःखत्रस्तान्यनेकशः}


\twolineshloka
{शस्त्राण्येषां तु राजेन्द्र शोणिताक्तानि सर्वशः}
{प्राभ्रश्यन्त कराग्रेभ्यो दृष्ट्वा द्रोणं हतं युधि}


\twolineshloka
{तानि बद्वान्यरिष्टानि लम्बमानानि भारत}
{अदृश्यन्त महाराज नक्षत्राणि यथा दिवि}


\twolineshloka
{तथा तु स्तिमितं दृष्ट्वा गतसत्वमवस्थितम्}
{बलं तव महाराज राजा दुर्योधनोऽब्रवीत्}


\twolineshloka
{भवतां बाहुवीर्यं हि समाश्रित्य मया युधि}
{पाण्डवेयाः समाहूता युद्वं चेदं प्रवर्तितम्}


\twolineshloka
{तदिदं निहते द्रोणे विषण्णमिव लक्ष्यते}
{युध्यमानाश्च समरे योधा वध्यन्ति सर्वशः}


\twolineshloka
{जयो वाऽपि वधो वाऽपि युध्यमानस्य संयुगे}
{भवेत्किमत्र चित्रं वै युध्यध्वं सर्वतोमुखाः}


\twolineshloka
{पश्यध्वं च महात्मानं कर्णं वैकर्तनं युधि}
{प्रचरन्तं महेष्वासं दिव्यैरस्त्रैर्महाबलम्}


\twolineshloka
{यस्य वै युधि सन्त्रासात्कुन्तीपुत्रो धनञ्जयः}
{निवर्तते सदा मन्दः सिंहात्क्षुद्रमृगो यथा}


\twolineshloka
{येन नागायुतप्राणो भीमसेनो महाबलः}
{मानुषेणैव युद्धेन तामवस्थां प्रवेशितः}


\twolineshloka
{येन दिव्यास्त्रविच्छूरो मायावी स घटोत्कचः}
{अमोघया रणे शक्त्या निहतो भैरवं नदन्}


\twolineshloka
{तस्य दुर्वारवीर्यस्य सत्यसन्धस्य धीमतः}
{बाह्वोर्द्रविणमक्षय्यमद्य द्रक्ष्यथ संयुगे}


\twolineshloka
{द्रोणपुत्रस्य विक्रान्तं राधेयस्यैव चोभयोः}
{पश्यन्तु पाण्डुपुत्रास्ते विष्णुवासवयोरिव}


\fourlineindentedshloka
{सर्व एव भवन्तश्च शक्ताः प्रत्येकशोऽपि वा}
{पाण्डुपुत्रान्रणे हन्तुं ससैन्यान्किसु संहताः}
{वीर्यवन्तः कृतास्त्राश्च द्रक्ष्यथाद्य परस्परम् ॥सञ्जय उवाच}
{}


\twolineshloka
{एवमुक्त्वा ततः कर्णं चक्रे सेनापतिं तदा}
{तव पुत्रो महावीर्यो भ्रातृभिः सहितोऽनघ}


\twolineshloka
{सैनापत्यमथावाप्य कर्णो राजन्महारथः}
{सिंहनादं विनद्योच्चैः प्रायुध्यत रणोत्कटः}


\twolineshloka
{ससृञ्जयानां सर्वेषां पाञ्चालानां च मारिष}
{केकयानां विदेहानां चकार कदनं महत्}


\twolineshloka
{तस्येषुधाराः शतशः प्रादुरासञ्छरासनात्}
{अग्रे पुङ्खेषु संसक्ता यथा भ्रमरपङ्क्तयः}


\twolineshloka
{स पीडयित्वा पाञ्चालान्पाण्डवांश्च तरस्विनः}
{हत्वा सहस्रशो योधानर्जुनेन निपातितः}


\chapter{अध्यायः ६}
\twolineshloka
{`जनमेजय उवाच}
{}


\twolineshloka
{तच्छ्रुत्वा कर्णहननं पुत्रांश्चैव पलायितान्}
{धृतराष्ट्रो नृपश्चैव द्विजश्रेष्ठ किमब्रवीत्}


\threelineshloka
{प्राप्तवान्व्यसनं घोरं पुत्रव्यसनजं महत्}
{काले यदुक्तवांस्तस्मिंस्तन्ममाचक्ष्व तत्त्वतः' ॥वैशम्पायन उवाच}
{}


\twolineshloka
{एतच्छ्रुत्वा महाराज धृतराष्ट्रोऽम्बिकासुतः}
{अब्रवीत्सञ्जयं सूतं शोकसंविग्नमानसः}


\twolineshloka
{दुष्प्रणीतेन मे तात पुत्रस्यादीर्घजीविनः}
{हतं वैकर्तनं श्रुत्वा शोको मर्माणि कृन्तति}


\threelineshloka
{तस्य मे संशयं छिन्धि दुःखपारं तितीर्षतः}
{कुरूणां सृञ्जयानां च के जीवन्ति के मृताः ॥सञ्जय उवाच}
{}


\twolineshloka
{हतः शान्तनवो राजन्दुराधर्षः प्रतापवान्}
{हत्वा पाण्डवयोधानामर्बुदं दशभिर्दिनैः}


\twolineshloka
{तथा द्रोणो महेष्वासः पाञ्चालानां रथव्रजान्}
{निहत्य युधि दुर्धर्षः पञ्चाद्रुक्मरथो हतः}


\twolineshloka
{हतशेषस्य भीष्मेण द्रोणेन च महात्मना}
{अर्धं निहत्य सैन्यस्य कर्णो वैकर्तनो हतः}


\twolineshloka
{विविंशतिर्महाराज राजपुत्रो महाबलः}
{आनर्तयोधाञ्शतशो निहत्य निहतो रणे}


\twolineshloka
{तथा पुत्रो विकर्णस्ते क्षत्रव्रतमनुस्मरन्}
{क्षीणबाणो हतः शूरः स्थितो ह्यभिमुखः परैः}


\twolineshloka
{धोररूपान्परिक्लेशान्दुर्योधनकृतान्बहून्}
{प्रतिज्ञां स्मरता तेन भीमसेनेन पातितः}


\twolineshloka
{विन्दानुविन्दावावन्त्यौ राजपुत्रौ महारथौ}
{कृत्वा स्वसुकरं कर्म गतौ वैवस्वतक्षयम्}


\twolineshloka
{सिन्धुराष्ट्रमुखानीह दश राष्ट्राणि यानिह}
{यस्य तिष्ठन्ति वचने यः स्थितस्तव शासने}


\twolineshloka
{अक्षौहिणीर्दशैकां च विनिर्जित्य शितैः शरैः}
{सोऽर्जुनेन हतो राजन्महावीर्यो जयद्रथः}


\twolineshloka
{तथा दुर्योधनसुतस्तरस्वी युद्धदुर्मदः}
{वर्तमानः पितुः शास्त्रे सौभद्रेण निपातितः}


\twolineshloka
{तथा दौःशासनिः सूरो बाहुशाली रणोत्कटः}
{द्रौपदेयेन सङ्गम्य गमितो यमसादनम्}


\twolineshloka
{किरातानामधिपतिः सागरानूपवासिनाम्}
{देवराजस्य धर्मात्मा प्रियो बहुमतः सखा}


\twolineshloka
{भगदत्तो महीपालः क्षत्रधर्मरतः सदा}
{धनञ्जयेन विक्रम्य गमितो यमसादनम्}


\twolineshloka
{तथा कौरवदायादो न्यस्तशस्त्रो महायशाः}
{हतो भूरिश्रवा राजञ्शूरः सात्यकिना युधि}


\twolineshloka
{श्रुतायुरपि चाम्बष्ठः क्षत्रियाणां धुरन्धरः}
{चरन्नभीतवत्सङ्ख्ये निहतः सव्यसाचिना}


\twolineshloka
{तव पुत्रः सदामर्षी कृतास्त्रो युद्धदुर्मदः}
{दुःशासनो महाराज भीमसेनेन पातितः}


\twolineshloka
{यस्य राजन्गजानीकं बहुसाहस्रमद्भुतम्}
{सुदक्षिणः स सङ्ग्रामे निहतः सव्यसाचिना}


\twolineshloka
{कोसलानामधिपतिर्हत्वा बहुमतान्परान्}
{सौभद्रेणेह विक्रम्य गमितो यमसादनम्}


\twolineshloka
{बहुशो योधयित्वा तु भीमसेनं महारथम्}
{चित्रसेनस्तव सुतो भीमसेनेन पातितः}


\twolineshloka
{मद्रराजात्मजः शूरः परेषां भयवर्धनः}
{असिचर्मधरः श्रीमान्सौभद्रेण निपातितः}


\twolineshloka
{समः कर्णस्य समरे यः स कर्णस्य पश्यतः}
{वृषसेनो महातेजाः शीघ्रास्त्रो दृढविक्रमः}


\twolineshloka
{अभिमन्योर्वधं स्मृत्वा प्रतिज्ञामपि चात्मनः}
{धनञ्जयेन विक्रम्य गमितो यमसादनम्}


\twolineshloka
{नित्यं प्रसक्तवैरो यः पाण्डवैः पृथिवीपतिः}
{विश्राव्य वैरं पार्थेन श्रुतायुः स निपातितः}


\twolineshloka
{शल्यपुत्रस्तु विक्रान्तः सहदेवेन मारिष}
{हतो रुक्मरथो राजञ्श्यालो मातुलजो युधि}


\twolineshloka
{राजा भागीरथो वृद्धो बृहत्क्षत्रश्च केकयः}
{पराक्रमन्तौ विक्रान्तौ निहतौ वीर्यवत्तरौ}


\twolineshloka
{भगदत्तसुतो राजन्कृतप्रज्ञो मंहाबलः}
{श्येनवच्चरता सङ्ख्ये नकुलेन निपातितः}


\twolineshloka
{पितामहस्तव तथा बाह्लीकः सह बाह्लिकैः}
{निहतो भीमसेनेन महाबालपराक्रमः}


\twolineshloka
{जयत्सेनस्तथा राजञ्जारासन्धिर्महाबलः}
{मागधो निहतः सङ्ख्ये सौभद्रेण महात्मना}


\twolineshloka
{पुत्रस्ते दुर्मुखो राजन्दुःसहश्च महारथः}
{गदया भीमसेनेन निहतौ शूरमानिनौ}


\twolineshloka
{दुर्मर्षणो दुर्विषहो दुर्जयश्च महारथः}
{कृत्वा त्वसुकरं कर्म गता वैवस्वतक्षयम्}


\twolineshloka
{उभौ कलिङ्गवृषकौ भ्रातरौ युद्धदुर्मदौ}
{कृत्वा चासुकरं कर्म गतौ वैवस्वतक्षयम्}


\twolineshloka
{सचिवो वृषपर्वा ते शूरः परमवीर्यवान्}
{भीमसेनेन विक्रम्य गमितो यमसादनम्}


\twolineshloka
{तथैव पौरवो राजा नागायुतबलो महान्}
{समरे पाण्डुपुत्रेण निहतः सव्यसाचिना}


\twolineshloka
{वसातयो महाराज द्विसाहस्राः प्रहारिणः}
{शूरसेनाश्च विक्रान्ताः सर्वे युधि निपातिताः}


\twolineshloka
{अभीषाहाः कवचिनः प्रहरन्तो रणोत्कटाः}
{शिबयश्च रथोदाराः कालिङ्गसहिता हताः}


\twolineshloka
{गोकुले नित्यसंवृद्धा युद्धे परमकोपनाः}
{गोपालाः कृतवीरास्ते निहताः सव्यसाचिना}


\twolineshloka
{श्रेणयो बहुसाहस्राः संशप्तकगणाश्च ये}
{ते सर्वे पार्थमासाद्य गदा वैवस्वतक्षयम्}


\twolineshloka
{श्यालौ तव महाराज राजानौ वृषकाचलौ}
{त्वदर्थमतिविक्रान्तौ निहतौ सव्यसाचिना}


\twolineshloka
{उग्रकर्मा महेष्वासो नामतः कर्मतस्तथा}
{साल्वराजो महाबाहुर्भीमसेनेन पातितः}


\twolineshloka
{ओघवांश्च महाराज बृहन्तः सहितौ रणे}
{पराक्रमन्तौ मित्रार्थे गतौ वैवस्वतक्षयम्}


\twolineshloka
{तथैव रथिनां श्रेष्ठः क्षेमधूर्तिर्विशाम्पते}
{निहतो गदया राजन्भीमसेनेन संयुगे}


\twolineshloka
{तथा राजन्महेष्वासो जलसन्धौ महाबलः}
{सुमहत्कदनं कृत्वा हतः सात्यकिना रणे}


\twolineshloka
{अलम्बुसो राक्षसेन्द्रः खरबन्धुरयानवान्}
{घटोत्कचेन विक्रम्य गमितो यमसादनम्}


\twolineshloka
{राधेयः सूतपुत्रश्च भ्रातरश्च महारथाः}
{केकयाः सर्वशश्चापि निहताः स्व्यसाचिना}


\twolineshloka
{मालवा मद्रकाश्चैव द्राविडाश्चोग्रकर्मिणः}
{यौधेयाश्च ललित्थाश्च क्षुद्रकाश्चाप्युशीनराः}


\twolineshloka
{मावेल्लकास्तुण्डिकेराः सावित्रीपुत्रकाश्च ये}
{प्राच्योदीच्याः प्रतीच्याश्च दाक्षिणात्याश्च मारिष}


\twolineshloka
{पत्तीनां निहताः सङ्घा हयानां प्रयुतानि च}
{रथव्रजाश्च निहता हताश्च वरवारणाः}


\twolineshloka
{सध्वजाः सायुधाः शूराः सवर्माम्बरभूषणाः}
{कालेन महताऽऽयस्ताः कुशलैर्ये च वर्धिताः}


\twolineshloka
{ते हताः समरे राजन्पार्थेनाक्लिष्टकर्मणा}
{अन्ये तथाऽमितबलाः परस्परवधैषिणः}


\twolineshloka
{एते चान्ये च बहवो राजानः सगणा रणे}
{हताः सहस्रशो राजन्यन्मां त्वं परिपृच्छसि}


\twolineshloka
{एवमेष क्षयो वृत्तः कर्णार्जुनसमागमे}
{महेन्द्रेण यथा वृत्रो यथा रामेण रावणः}


\twolineshloka
{यथा कृष्णेन नरको मुरुश्च निहतो रणे}
{कार्तवीर्यश्च रामेण भार्गवेण यथा हतः}


\twolineshloka
{सज्ञातिबान्धवः शूरः समरे युद्धदुर्मदः}
{रणे कृत्वा महद्युद्धं घोरं त्रैलोक्यमोहनम्}


\twolineshloka
{यथा स्कन्देन महिषो यथा रुद्रेण चान्धकः}
{तथाऽर्जुनेन स हतो द्वैरथे युद्धदुर्मदः}


\twolineshloka
{सामात्यबान्धवो राजन्कर्णः प्रहरतां वरः}
{जयाशा धार्तराष्ट्राणां वैरस्य च मुखं नृपः}


\twolineshloka
{तीर्णं तत्पाण्डवै राजन्यत्पुरा नावबुध्यसे}
{उच्यमानो महाराज बन्धुभिर्हितकाङ्क्षिभिः}


\twolineshloka
{`न कृतं च त्वया पूर्वं दैवेन विधिना बलात्'}
{तदिदं समनुप्राप्तं व्यसनं सुमहात्ययम्}


\twolineshloka
{गाणां राज्यकामानां त्वया राजन्हितैषिणा}
{अहितान्येव चीर्णानि तेषां ते फलमागतम्}


\chapter{अध्यायः ७}
\twolineshloka
{धृतराष्ट्र उवाच}
{}


\threelineshloka
{आख्याता मामकास्तात निहता युधि पाण्डवैः}
{हतांश्च पाण्डवेयानां मामकैर्ब्रूहि सञ्जय ॥सञ्जय उवाच}
{}


\twolineshloka
{कृतिनो युधि विक्रान्ता महासत्वा महाबलाः}
{सानुबन्धाः सहामात्या गाङ्गेयेन निपातिताः}


\twolineshloka
{नारायणा वल्लवाश्च रामाश्च शतशो परे}
{अनुरक्ताश्च वीरेण भीष्मेण युधि पातिताः}


\twolineshloka
{समः किरीटिना सङ्ख्ये वीर्येण च बलेन च}
{सत्यजित्सत्यसन्धेन द्रोणेन निहतो युधि}


\twolineshloka
{पाञ्चालानां महेष्वासाः सर्वे युद्वविशारदाः}
{द्रोणेन सह सङ्गम्य गता वैवस्वतक्षयम्}


\twolineshloka
{तथा विराटद्रुपदौ वृद्वौ सहसुतौ नृपौ}
{पराक्रमन्तौ मित्रार्थे द्रोणेन निहतौ रणे}


\twolineshloka
{यो बाल एव समरे सम्मितः सव्यसाचिना}
{केशवेन च दुर्धर्षो बलदेवेन चाभिभूः}


\threelineshloka
{परेषां कदनं कृत्वा महद्रणविशारदः}
{परिवार्य मदामात्रैः षड्भिः परमकै रथैः}
{अशक्नुवद्भिर्बीभत्सुमभिमन्युर्निपातितः}


\twolineshloka
{कृतं तं विरथं वीरं क्षत्रधर्मे व्यवस्थितम्}
{दौःशासनिर्महाराज सौभद्रं हतवान्रणे}


\twolineshloka
{पटच्चरनिहन्ता च महत्या सेनया वृतः}
{अम्बष्ठस्य सुतः श्रीमान्मित्रहेतोः पराक्रमन्}


\twolineshloka
{आसाद्य लक्ष्मणं वीरं दुर्योधनसुतं रणे}
{सुमहत्कदनं कृत्वा गतो वैवस्वतक्षयम्}


\twolineshloka
{बृहन्तः सुमहेष्वासः कृतास्त्रो युद्वदुर्मदः}
{दुःशासनेन विक्रम्य गमितो यमसादनम्}


\twolineshloka
{मणिमान्दण्डधारश्च राजानौ युद्धदुर्मदौ}
{पराक्रमन्तौ मित्रार्थे द्रोणेन युधि पातितौ}


\twolineshloka
{अंशुमान्भोजराजस्तु सहसैन्यो महारथः}
{भारद्वाजेन विक्रम्य गमितो यमसादनम्}


\twolineshloka
{सामुद्रश्चित्रसेनश्च सह पुत्रेण भारत}
{समुद्रसेनेन बलाद्गमितो यमसादनम्}


\twolineshloka
{अनूपवासी नीलश्च व्याध्रदत्तश्च वीर्यवान्}
{अश्वत्थाम्ना विकर्णेन गमितौ यमसादनम्}


\threelineshloka
{चित्रायुधश्चित्रयोधी कृत्वा च कदनं महत्}
{चित्रमार्गेण विक्रम्य विकर्णेन हतो मृधे}
{}


\twolineshloka
{वृकोदरसमो युद्धे वृतः केकयजो युधि}
{केकयेन च विक्रम्य भ्राता भ्रात्रा निपातितः}


\twolineshloka
{जनमेजयो गदायोधी पार्वतीयः प्रतापवान्}
{दुर्मुखेन महाराज तव पुत्रेण पातितः}


\twolineshloka
{रोचमानौ नरव्याघ्रौ रोचमानौ ग्रहाविव}
{द्रौणेन युगपद्राजन्दिवं सम्प्रातितौ शरैः}


\twolineshloka
{नृपाश्च प्रतियुध्यन्तः पराक्रान्ता विशाम्पते}
{कृत्वा नसुकरं कर्म गता वैवस्वतक्षयम्}


\twolineshloka
{पुरुजित्कृन्तिभोजश्च मातुलौ सव्यसाचिनः}
{संग्रामनिर्जिताँल्लोकान्गमितौ द्रोणसायकैः}


\twolineshloka
{अभिभूः काशिराजश्च काशिकैर्बहुभिर्वृतः}
{वसुदानस्य पुत्रेण त्याजितो देहमाहवे}


\twolineshloka
{अमितौजा युधामन्युरुत्तमौजाश्च वीर्यवान्}
{निहत्य शतशः सूरानस्मदीयैर्निपातिताः}


\twolineshloka
{मित्रवर्मा च पाञ्चाल्यः क्षत्रधर्मा च भारत}
{द्रोणेन परमेष्वासौ गमितौ यमसादनम्}


\twolineshloka
{शिखण्डितनयो युद्धे क्षत्रदेवो युधाम्पतिः}
{लक्ष्मणेन हतो राजंस्तव पौत्रेण भारत}


\twolineshloka
{सुचित्रश्चित्रवर्मा च पितापुत्रौ महारथौ}
{प्रचरन्तौ महावीरौ द्रोणेन निहतौ रणे}


\twolineshloka
{वार्धक्षेमिर्महाराज समुद्र इव पर्वणि}
{आयुधक्षयमासाद्य प्रशान्तिं परमां गतः}


\twolineshloka
{सेनाबिन्दुसुतः श्रेष्ठः शास्त्रवान्प्रहरन्युधि}
{बाह्लिकेन महाराज कौरवेन्द्रेण पातितः}


\twolineshloka
{धृष्टकेतुर्महाराज चेदीनां प्रवरो रथः}
{कृत्वा नसुकरं कर्म गतो वैवस्वतक्षयम्}


\twolineshloka
{तथा सत्यधृतिर्वीरः कृत्वा कदनमाहवे}
{पाण्डवार्थे पराक्रान्तो गमितो यमसादनम्}


\twolineshloka
{पुत्रस्तु शिशुपालस्य सुकेतुः पृथिवीपतिः}
{निहत्य शात्रवान्सङ्ख्ये द्रोणेन निहतो युधि}


\twolineshloka
{तथा सत्यधृतिर्वीरो मदिराश्वश्च वीर्यवान्}
{सूर्यदत्तश्च विक्रान्तो निहतो द्रोणसायकैः}


\threelineshloka
{`मात्स्यादवरजः श्रीमाञ्शतानीको निपातितः}
{श्रेणिमांश्च महाराज युध्यमानः पराक्रमी}
{कृत्वा नसुकरं कर्म गतो वैवस्वतक्षयम्}


\twolineshloka
{तथैव युधि विक्रान्तो मागधः परमास्त्रवित्}
{भीष्मेण निहतो राजञ्शेतेऽद्य परवीरहा}


\twolineshloka
{विराटपुत्रः शङ्खस्तु उत्तरश्च महारथः}
{कुर्वन्तौ सुमहत्कर्म गतौ वैवस्वतक्षयम्}


\twolineshloka
{वसुदानश्च कदनं कुर्वाणोऽतीव संयुगे}
{भारद्वाजेन विक्रम्य गमितो यमसादनम्}


\twolineshloka
{एते चान्ये च बहवः पाण्डवानां महारथाः}
{हता द्रोणेन विक्रम्य यन्मां त्वं परिपृच्छसि}


\chapter{अध्यायः ८}
\twolineshloka
{धृतराष्ट्र उवाच}
{}


\twolineshloka
{मामकस्यास्य सैन्यस्य हृतोत्सेकस्य स़ञ्जय}
{अवशेषं न पश्यामि ककुदे मृदिते सति}


\twolineshloka
{तौ हि वीरौ महेष्वासौ मदर्थे कुरुसत्तमौ}
{भीष्मद्रोणौ हतौ श्रुत्वा कोन्वर्थो जीवितेन मे}


\twolineshloka
{न च मृष्यामि राधेयं हतमाहवशोभिनम्}
{यस्य बाह्वोर्बलं तुल्यं कुञ्जराणां शतं शतम्}


\twolineshloka
{हतप्रवरसैन्यं मे यथा शंससि सञ्जय}
{अहतानपि मे शंस येऽत्र जीवन्ति केचन}


\threelineshloka
{एतेषु हि मृतेष्वद्य ये त्वया परिकीर्तिताः}
{येऽपि जीवन्ति ते सर्वे मृता इति मतिर्मम ॥सञ्जय उवाच}
{}


\twolineshloka
{यस्मिन्महास्त्राणि समर्पितानिचित्राणि शुभ्राणि चतुर्विधानि}
{दिव्यानि राजन्विहितानि चैवद्रोणेन वीरे द्विजसत्तमेन}


\twolineshloka
{महारथः कृतिमान्क्षिप्रहस्तोदृढायुधो दृढमुष्टिर्दृढेषुः}
{स वीर्यवान्द्रोणपुत्रस्तरस्वीव्यवस्थितो योद्वुकामस्त्वदर्थे}


\twolineshloka
{आनर्तवासी हृदिकात्मजोऽसौमहारथः सात्वतानां वरिष्ठः}
{स्वयं भोजः कृतवर्मा कृतास्त्रोव्यवस्थितो योद्वुकामस्त्वदर्थे}


\twolineshloka
{आर्तायनिः समरे दुष्प्रकम्प्यःसेनाग्रणीः प्रथमस्तावकानाम्}
{यः स्वस्रीयान्पाण्डवेयान्विसृज्यसत्यां वाचं स्वां चिकीर्षुस्तरस्वी}


\twolineshloka
{तजोवधं सूतपुत्रस्य सङ्ख्येप्रतिश्रुत्याजातशत्रोः पुरस्तात्}
{दुराधर्षः शक्रसमानवीर्यःशल्यः स्थितो योद्वुकामस्त्वदर्थे}


\twolineshloka
{शारद्वतो गौतमश्चापि राज--न्महाबाहुचित्रास्त्रयोधी}
{धनुश्चित्रं सुमहद्भावरसाहंव्यवस्थितो योद्वुकामः प्रगृह्य}


\twolineshloka
{गान्धारराजः ससुतश्च राज--न्दुर्द्यूतदेवी कलहप्रियश्च}
{गान्धारमुख्यैर्यवनैश्च राज--न्व्यवस्थितो योद्वुकामस्त्वदर्थे}


\twolineshloka
{महारथः केकयराजपुत्रःसदश्वयुक्तं च पताकिनं च}
{रथं समारुह्य कुरुप्रवीरव्यवस्थितो योद्वुकामस्त्वदर्थे}


\twolineshloka
{तथा सुतस्ते ज्वलनार्कवर्णंरथं समास्थाय कुरुप्रवीरः}
{व्यवस्थितः पुरुमित्रो नरेन्द्रव्यभ्रे सूर्यो भ्राजमानो यथा खे}


\twolineshloka
{दुर्योधनो नागकुलस्य मध्येव्यवस्थितः सिंह इवाबभासे}
{रथेन जाम्बूनदभूषणेनव्यवस्थितः समरे योत्स्यमानः}


\twolineshloka
{स राजमध्ये पुरुषप्रवीरोरराज जाम्बूनदचित्रवर्मा}
{पद्मप्रभो वह्निरिवाल्पधूमोमेघान्तरे सूर्य इव प्रकाशः}


\twolineshloka
{तथा सुषेणोऽप्यसिचर्मपाणि--स्तवात्मजः सत्यसेनश्च वीरः}
{व्यवस्थितौ चित्रसेनेन सार्धंहृष्टात्मानौ समरे योद्धकामौ}


\twolineshloka
{हीनिषेवो भारतराजपुत्रउग्रायुधः श्रुतवर्मा जयश्च}
{शलश्च सत्यव्रतदुःशलौ चव्यवस्थिताः सहसैन्या नराग्र्याः}


\twolineshloka
{कैतव्यानामधिपः शूरमानीरणेरणे शत्रुहा राजपुत्रः}
{रथी हयी नागपत्तिप्रयायीव्यवस्थितो योद्वुकामस्त्वदर्थे}


\twolineshloka
{वीरः श्रुतायुश्च धृतायुधश्चचित्राङ्गदश्चित्रसेनश्च वीरः}
{व्यवस्थिता योद्वुकामा नराग्र्याःप्रहारिणो मानिनः सत्यसन्धाः}


% Check verse!
कर्णात्मजः सत्यसन्धो महात्माव्यवस्थितः समरे योद्वुकामः
\twolineshloka
{अथापरौ कर्णसुतौ वरास्त्रौव्यवस्थितौ लघुहस्तौ नरेन्द्र}
{बले महत्युल्बणसत्ववीर्यौसमास्थितौ योद्वुकामौ त्वदर्थे}


\threelineshloka
{एतैश्च मुख्यैरपरैश्च राज--न्योधप्रवीरैरमितप्रभावैः}
{व्यवस्थितो नागकुलस्य मध्येयथा महेन्द्रः कुरुराजो जयाय ॥धृतराष्ट्र उवाच}
{}


\threelineshloka
{आख्याता जीवमाना ये परे सैन्या यथायथम्}
{इतीदमवगच्छामि व्यक्तमर्थाभिपत्तितः ॥वैशम्पायन उवाच}
{}


\threelineshloka
{एवं ब्रुवन्नेव तदा धृतराष्ट्रोऽम्बिकासुतः}
{हतप्रवीरं भूयिष्ठं किञ्चिच्छेषं स्वकं बलम्}
{श्रुत्वा व्यामोहमागच्छच्छोकव्याकुलितेन्द्रियः}


\threelineshloka
{व्याकुलं मे मनस्तात श्रुत्वा सुमहदप्रियम् ॥मनो मुह्यति चाङ्गानि न च शक्नोमि धावितुम्}
{इत्येवमुक्त्वा वचनं धृतराष्ट्रोऽम्बिकासुतः}
{भ्रान्तचित्तस्ततः सोऽथ बभूव जगतीपतिः}


\chapter{अध्यायः ९}
\twolineshloka
{जनमेजय उवाच}
{}


\twolineshloka
{श्रुत्वा कर्णं हतं युद्वे पुत्रांश्चैव पलायिनः}
{नरेन्द्रः किञ्चिदाश्वस्तो द्विजश्रेष्ठ किमब्रवीत्}


\threelineshloka
{प्राप्तवान्परमं दुःखं पुत्रव्यसनजं महत्}
{तस्मिन्यदुक्तवान्काले तन्ममाचक्ष्व पृच्छतः ॥वैशम्पायन उवाच}
{}


\twolineshloka
{श्रुत्वा कर्णस्य निधनमश्रद्वेयमिवाद्भुतम्}
{भूतसम्मोहनं भीमं मेरोः संसर्पणं यथा}


\twolineshloka
{चित्तमोहमिवायुक्तं भार्गवस्य महामतेः}
{पराजयमिवेन्द्रस्य द्विषद्ध्यो भीमकर्मणः}


\twolineshloka
{दिवः प्रपतनं भानोरुरव्यामिव महाद्युतेः}
{संशोषणमिवाचिन्त्यं समुद्रस्याक्षयाम्भसः}


\twolineshloka
{महीवियद्दिगम्बूनां सर्वनाशमिवाद्भुतम्}
{कर्मणोरिव वैफल्यमुभयोः पुण्यपापयोः}


\twolineshloka
{सञ्चिन्त्य निपुणं बुद्ध्या धृतराष्ट्रो जनेश्वरः}
{नेदमस्तीति सञ्चिन्त्य कर्णस्य समरे वधम्}


\twolineshloka
{प्राणिनामेवमात्मत्वात्स्यादपीति विचार्य च}
{शोकाग्निना दह्यमानो धम्यमान इवाशये}


\threelineshloka
{विस्रस्ताङ्गः श्वसन्दीनो हाहेत्युक्त्वा सुदुःखितः}
{विललाप महाराज धृतराष्ट्रोऽम्बिकासुतः ॥धृतराष्ट्र उवाच}
{}


\twolineshloka
{सञ्जयाधिरथिर्वीरः सिंहद्विरदविक्रमः}
{वृषभप्रतिमस्कन्धो वृषभाक्षगतिस्वरः}


\twolineshloka
{वृषभो वृषभस्येव यो युद्धे न निवर्तते}
{शत्रोरपि महेन्द्रस्य वज्रसंहननो युवा}


\twolineshloka
{यस्य ज्यातलशब्देन शरवृष्टिरवेण च}
{रथाश्वनरमातङ्गा नावतिष्ठन्ति संयुगे}


\twolineshloka
{यमाश्रित्य महाबाहुं द्विषत्सङ्घघ्नमच्युतम्}
{दुर्योधनोऽकरोद्वैरं पाण्डुपुत्रैर्महारथैः}


\twolineshloka
{स कथं रथिनां श्रेष्ठः कर्णः पार्थेन संयुगे}
{निहतः पुरुषव्याघ्रः प्रसह्यासह्यविक्रमः}


\twolineshloka
{यो नामन्यत वै नित्यमच्युतं च धनञ्जयम्}
{न वृष्मीन्सहितानन्यान्स्वबाहुबलदर्पितः}


\twolineshloka
{शार्ङ्गगाण्डीवधन्वानौ सहितावपराजितौ}
{अहं दिव्याद्रथादेकः पातयिष्यामि संयुगे}


\twolineshloka
{इति यः सततं मन्दमवोचल्लोभमोहितम्}
{दुर्योधनमवाचीनं राज्यकामुकमातुरम्}


\threelineshloka
{योऽजयत्सर्वकाम्भोजानावन्त्यान्केकयैः सह}
{गान्धारान्मद्रकान्मात्स्यांस्त्रिगर्तांस्तङ्कणाञ्शकान्}
{}


% Check verse!
पाञ्चालांश्च विदेहांश्च कुलिन्दान्काशिकोसलान्सूह्मानङ्गांश्च वङ्गांश्च निषादान्पुण्ड्रकीकटान्
\twolineshloka
{वत्सान्कलिङ्गान्दरदानश्मकानृषिकानपि}
{`शबरान्हारहूणांश्च प्रघूणान्सरलानपि}


\twolineshloka
{म्लेच्छराष्ट्राधिपांश्चैव दुर्गानाटविकांस्तथा'}
{जित्वैतान्समरे वीरः प्रदीप्तैः कङ्कपत्रिभिः}


\threelineshloka
{करमाहारयामास जित्वा सर्वानरींस्तथा}
{दुर्योधनस्य वृद्ध्यर्थं राधेयो रथिनां वरः}
{दिव्यास्त्रविमन्हातेजाः कर्णो वैकर्तनो वृषः}


\twolineshloka
{सेनागोपश्च स कथं शत्रुभिः परमास्त्रवित्}
{घातितः पाण्डवैः शूरैः समरे वीर्यशालिभिः}


\twolineshloka
{वृषो महेन्द्रो देवेषु वृषः कर्णो नरेष्वपि}
{तृतीयमन्यं लोकेषु वृषं नैवानुशुश्रुम}


\twolineshloka
{उच्चैः श्रवा वरोऽश्वानां राज्ञां वैश्रवणो वरः}
{वरो महेन्द्रो देवानां कर्णः प्रहरतां वरः}


\twolineshloka
{योऽजितः पार्थिवैः शूरैः समर्थैर्वीर्यशालिभिः}
{दुर्योधनस्य वृद्ध्यर्थं कृत्स्नामुर्वीमथाजयत्}


\twolineshloka
{यं लब्ध्वा मागधो राजा सांत्वमानोऽथ सौहृदैः}
{अरौत्सीत्पार्थिवं क्षत्रमृते यादवकौरवान्}


\twolineshloka
{तं श्रुत्वा निहतं कर्णं द्वैरथे सव्यसाचिना}
{शोकार्णवे निमग्नोऽहं भिन्ना नौरिव सागरे}


\twolineshloka
{तं वृषं निहतं श्रुत्वा द्वैरथे रथिनां वरम्}
{शोकार्णवे निमग्नोऽहमप्लुवः सागरे यथा}


\twolineshloka
{ईदृशैर्यद्यहं दुःखैर्न विनश्यामि सञ्जय}
{वज्राद्दृढतरं मन्ये हृदयं मम दुर्भिदम्}


\twolineshloka
{ज्ञातिसंबन्धिमित्राणामिमं श्रुत्वा पराभवम्}
{को मदन्यः पुमाँल्लोके न जह्यात्सूत जीवितम्}


\threelineshloka
{विषमग्निं प्रपातं च पर्वताग्रादहं वृणे}
{`महाप्रस्थानगमनं जलं प्रायोपवेशनम्'}
{न हि शक्ष्यामि दुःखानि सोढुं कष्टानि सञ्जय}


\chapter{अध्यायः १०}
\twolineshloka
{सञ्जय उवाच}
{}


\twolineshloka
{श्रिया कुलेन यशसा तपसा च श्रुतेन च}
{त्वामद्य सन्तो मन्यन्ते ययातिमिव नाहुषम्}


\threelineshloka
{श्रुते महर्षिप्रतिमः कृतकृत्योऽसि पार्थिव}
{पर्यवस्थापयात्मानं मा विषादे मनः कृथाः ॥धृतराष्ट्र उवाच}
{}


\twolineshloka
{दैवमेव परं मन्ये धिक्पौरुषमनर्थकम्}
{यत्र सालप्रतीकाशः कर्णोऽहन्यत संयुगे}


\twolineshloka
{हत्वा युधिष्ठिरानीकं पाञ्चालानां रथव्रजान्}
{प्रताप्य शरवर्षेण दिशः सर्वा महारथः}


\twolineshloka
{मोहियित्वा रणे पार्थान्वज्नहस्त इवासुरान्}
{स कथं निहतः शेते वायुरुग्ण इव द्रुमः}


\twolineshloka
{शोकस्यान्तं न पश्यामि पारं जलनिधेरिव}
{चिन्ता मे वर्धतेऽतीव मुमूर्षा चापि जायते}


\twolineshloka
{कर्णस्य निधनं श्रुत्वा विजयं फल्गुनस्य च}
{अश्रद्धेयमहं मन्ये वधं कर्णस्य सञ्जय}


\twolineshloka
{वज्रसारमिदं नूनं हृदयं दुर्भिदं मम}
{यच्छ्रुत्वा पुरुषव्याघ्रं हतं कर्णं न दीर्यते}


\twolineshloka
{आर्युर्नूनं सुदीर्घं मे विहितं दैवतैः पुरा}
{यत्र कर्णं हतं श्रुत्वा जीवामीह सुदुःखितः}


\threelineshloka
{धिग्जीवितमिदं चैव सुहृद्वीनस्य सञ्जय}
{अद्य चाहं दशामेतां गतः सञ्जय गर्हिताम्}
{कृपणं वर्तयिष्यामि शोच्यः सर्वस्य मन्दधीः}


\twolineshloka
{अहमेव पुरा भूत्वा सर्वलोकस्य सत्कृतः}
{परिभूतः कथं सूत परैः शक्ष्यामि जीवितुम्}


\twolineshloka
{दुःखाद्दुःखतरं भन्ये प्राप्तवानस्मि सञ्जय}
{भीष्मद्रोणवधेनैव कर्णस्य च महात्मनः}


\twolineshloka
{नावशेषं प्रपश्यामि सूतपुत्रे हते युधि}
{सहि पारं महानासीत्पुत्राणां मम सञ्जय}


\twolineshloka
{युद्धे हि निहतः शूरो विसृजन्सायकान्बहून्}
{को हि मे जीवितेनार्थस्तमृते पुरुषर्षभम्}


\twolineshloka
{रथादाधिरथिर्नूनं न्यपतत्सायकार्दितः}
{पर्वतस्येव शिखरं व ज्रपाताद्विदारितम्}


\twolineshloka
{स शेते पृथिवीं नूनं शोभयन्रुधिरोक्षितः}
{मातङ्ग इव मत्तेन मताङ्गेन निपातितः}


\twolineshloka
{यो बलं धार्तराष्ट्राणां पाण्डवानां यतो भयम्}
{सोऽर्जुनेन हतः कर्णः प्रतिमानं धनुष्मताम्}


\twolineshloka
{स हि वीरो महेष्वासो मित्राणामभयङ्करः}
{शेते विनिहतो वीरः शक्रेणेव पुरा बलः}


\twolineshloka
{पङ्गोरिवाध्वगमनं दरिद्रस्येव कामितम्}
{दुर्योधनस्य लोभश्च समान्येतानि सञ्जय}


\twolineshloka
{अन्यथा चिन्तितं कार्यमन्यथा तत्तु जायते}
{अहो नु बलवद्दैवं कालश्च दुरतिक्रमः}


\twolineshloka
{पलायमानः कृपणो दीनात्मा दीनपौरुषः}
{कच्चिद्विनिहतः सूत पुत्रो दुःशासनो मम}


\twolineshloka
{कच्चिन्न दीनाचरितं कृतवांस्तात संयुगे}
{कच्चिन्न निहतः शूरा यथाऽन्ये क्षत्रियर्षभाः}


\twolineshloka
{युधिष्ठिरस्य वचनं मा युद्धमिति सर्वदा}
{दुर्योधनो नाभ्यगृह्णान्मूढः पथ्यमिवौषधम्}


\twolineshloka
{शतल्पे शयानेन भीष्मेण सुमहात्मना}
{पानीयं याचितः पार्थः सोविध्यन्मेदिनीतलम्}


\twolineshloka
{जलस्य धारां जनितां दृष्ट्वा पाण्डुसुतेन च}
{अब्रवीत्स महाबाहुस्तात संशाम्य पाण्डवैः}


\twolineshloka
{प्रशमाद्वि भवेच्छान्तिर्मदन्तं युद्धमस्तु वः}
{भ्रातृभावेन पृथिवीं भुङ्क्ष्व पाण्डुसुतैः सह}


\twolineshloka
{अकुर्वन्वचनं तस्य नूनं शोचति पुत्रकः}
{तदिदं समनुप्राप्त वचनं दीर्घदर्शिनः}


\twolineshloka
{अहं तु निहतामात्यो हतपुत्रश्च सञ्जय}
{द्यूततः कृच्छ्रमापन्नो लूनपक्ष इव द्विजः}


\twolineshloka
{यथा हि शकुनिं गृह्य छित्त्वा पक्षौ च सञ्जय}
{विसर्जयन्ति संहृष्टाः क्रीडमानाः कुमारकाः}


\twolineshloka
{लूनपक्षतया तस्य गमनं नोपपद्यते}
{तथाहमपि सम्प्राप्तो लूनपक्ष इव द्विजः}


\threelineshloka
{क्षीणः सर्वार्थहीनश्च निर्ज्ञातिर्बन्धुवर्जितः}
{कां दिशं प्रतिपत्स्यामि दीनः शत्रुवशं गतः ॥वैशम्पायन उवाच}
{}


\threelineshloka
{इत्येवं धृतराष्ट्रोऽथ विलप्य बहुदुःखितः}
{प्रोवाच सञ्जयं भूयः सोकव्याकुलमानसः ॥धृतराष्ट्र उवाच}
{}


\twolineshloka
{योऽजयत्सर्वकाम्भोजानम्बष्ठान्कैकयैः सह}
{गान्धारांश्च विदेहांश्च जित्वा कार्यार्थमाहवे}


\twolineshloka
{दुर्योधनस्य वृद्ध्यर्थं योऽजयत्पृथिवीं प्रभुः}
{स जितः पाण्डवैः शूरैः समरे वाहुशालिभिः}


\twolineshloka
{तस्मिन्हते महेष्वासे कर्णे युधि किरीटिना}
{के वीराः पर्यतिष्ठन्त तन्ममाचक्ष्व सञ्जय}


\twolineshloka
{कच्चिन्नैकः परित्यक्तः पाण्डवैर्निहतो रणे}
{उक्तं त्वया पुरा तात यथा वीरो निपातितः}


\twolineshloka
{भीष्ममप्रतियुध्यन्तं शिखम्डी सायकोत्तमैः}
{पातयामास समरे सर्वशस्त्रभृतां वरम्}


\twolineshloka
{तथा द्रौपदिना द्रोणो न्यस्तसर्वायुधो युधि}
{`धर्मराजवचः श्रुत्वा अश्वत्थामा हतस्त्विति'}


\twolineshloka
{युक्तयोगो महेष्वासः शरैर्बहुभिराचितः}
{निहतः खङ्गमुद्यम्य धृष्टद्युम्नेन सञ्जय}


\twolineshloka
{अन्तरेण हतावेतौ छलेन च विशेषतः}
{अश्रौषमहमेतद्वै भीष्मद्रोणौ निपातितौ}


\twolineshloka
{भीष्मद्रोणौ हि समरे न हन्याद्वज्रभृत्स्वयम्}
{न्यायेन युध्यन्समरे तद्वै सत्यं ब्रवीमि ते}


\twolineshloka
{कर्णं त्वस्यन्तमस्त्राणि दिव्यानि च बहूनि च}
{कथमिन्द्रोपमं वीरं मृत्युर्युद्वे समस्पृशत्}


\twolineshloka
{यस्य विद्युत्प्रभां शक्तिं दिव्यां कनकभूषणाम्}
{प्रायच्छद्द्विषतां हन्त्रीं कुण्डलाभ्यां पुरन्दरः}


\twolineshloka
{यस्य सर्पमुखो दिव्यः शरः काञ्चनभूषणः}
{अशेत निशितः पत्री समरेष्वरिसूदनः}


\twolineshloka
{भीष्मद्रोणमुखान्वीरान्योऽवमत्य महारथान्}
{जामदग्न्यान्महाघोरं ब्राह्ममस्त्रमशिक्षत}


\twolineshloka
{यश्च द्रोणमुखान्दृष्ट्वा विमुखानर्दिताञ्शरैः}
{सौभद्रस्य महाबाहुर्व्यधमत्कार्मुकं शितैः}


\twolineshloka
{यश्च नागायुतप्राणां वज्ररंहसमच्युतम्}
{विरथं सहसा कृत्वा भीमसेनमपाहसत्}


\twolineshloka
{सहदेवं च निर्जित्य शरैः सन्नतपर्वभिः}
{कृपया विरथं कृत्वा नाहनद्धर्मचिन्तया}


\twolineshloka
{यश्च मायासहस्राणि विकुर्वाणं जयैषिणम्}
{घटोत्कचं राक्षसेन्द्रं शक्रशक्त्या निजघ्निवान्}


\twolineshloka
{एवांश्च दिवसान्यस्य युद्धे भीतो धनञ्जयः}
{नागमद्द्वैरथं वीरः स कथं निहतो रणे}


\twolineshloka
{[संशप्तकानां योधा ये आह्वयन्त सदाऽन्यतः}
{एतान्हत्वा हनिष्यामि पश्चाद्वैकर्तनं रणे}


\twolineshloka
{इति व्यपदिशन्पार्थो वर्जयन्सूतजं रणे}
{स कथं निहतो वीरः पार्थेन परवीरहा ॥]}


\twolineshloka
{रथभह्गो न चेत्तस्य धनुर्वा न व्यशीर्यत}
{न चेदस्त्राणि नष्टानि स कथं निहतः परैः}


\threelineshloka
{को हि शक्तो रणे कर्णं विधुन्वानं महद्धनुः}
{विमुञ्चन्तं शरान्घोरान्दिव्यान्यस्त्राणि चाहवे}
{जेतुं पुरुषशार्दूलं शार्दूलमिव वेगिनम्}


\threelineshloka
{ध्रवं तस्य धनुश्छिन्नं रथो वाऽपि महीं गतः}
{अस्त्राणि वा प्रनष्टानि यथा शंससि मे हतम्}
{न ह्यन्यदपि पश्यामि कारणं तस्य नाशने}


\twolineshloka
{न हिन्मि फल्गुनं यावत्तावत्पादौ न धावये}
{इति यस्य महाघोरं व्रतमासीन्महात्मनः}


\twolineshloka
{यस्य भीतो रमे निद्रां धर्मराजो युधिष्ठिरः}
{त्रयोदशसमा नित्यं नाभजत्पुरुषर्षभः}


\twolineshloka
{यस्य वीर्यवतो वीर्यमुपाश्रित्य महात्मनः}
{मम पुत्रः सभां भार्यां पाण्डूनां नीतवान्बलात्}


\twolineshloka
{तत्रापि च सभामध्ये पाण्डवानां च पश्यताम्}
{दासभार्येति पाञ्चालीमब्रवीत्कुरुसन्निधौ}


\twolineshloka
{[न सन्ति तपयः कृष्णे सर्वे षण्डतिलैः समाः}
{उपतिष्ठस्व भर्तारमन्यं वा वरवर्णिनि}


\twolineshloka
{इत्येवं यः पुरा वाचो रूक्षाः संश्रावयन्रुषा}
{सभायां सूतजः कृष्णां स कथं निहतः परैः}


\threelineshloka
{यदि भीष्मो रणश्लाघी द्रोणो वा युधि दुर्मदः}
{न हनिष्यति कौन्तेयान्पक्षपातात्सुयोधन}
{सर्वानेव हनिष्यामि व्येतु ते मानसो ज्वरः}


\threelineshloka
{किं करिष्यति गाण्डीवमक्षय्यौ च महेषुधी}
{स्निग्धचन्दनदिग्धस्य मच्छरस्याभिधावतः}
{स नूनमृषभस्कन्धो ह्यर्जुनेन कथं हतः ॥]}


\twolineshloka
{यश्च गाण्डीवमुक्तानां स्पर्शमुग्रमचिन्तयन्}
{अपतिर्ह्यसि कृष्णेति ब्रुवन्पार्थानवैक्षत}


\twolineshloka
{यस्य नासीद्भयं पार्थात्सपुत्रात्सजनार्दनात्}
{स्वबाहुबलमाश्रित्य मुहूर्तमपि सञ्जय}


\twolineshloka
{तस्य नाऽहं वधं मन्ये देवैरपि सवासवैः}
{प्रतीपमभिघावद्भिः किं पुनस्तात पाण्डवैः}


\twolineshloka
{न हि ज्यां संसम्पृशानस्य तलत्रे वाऽपि गृह्णतः}
{पुमानाधिरथेः स्थातुं कश्चित्प्रमुखतोऽर्हति}


\twolineshloka
{अपि स्यान्मेदिनी हीना सोमसूर्यप्रभांशुभिः}
{न वध- पुरुषेन्द्रस्य संयुगेष्वपलायिनः}


\twolineshloka
{येन मन्दः सहायेन भ्रात्रा दुःशासनेन च}
{वासुदेवस्य दुर्बुद्धिः प्रत्याख्यानमरोचत}


\twolineshloka
{स नूनं वृषभस्कन्धं कर्णं दृष्ट्वा निपातितम्}
{दुःशासनं च निहतं मन्ये शोचति पुत्रकः}


\twolineshloka
{हतं वैकर्तनं दृष्ट्वा द्वैरथे सव्यसाचिना}
{जयतः पाण्डवान्दृष्ट्वा किंस्विद्दुर्योधनोऽब्रवीत्}


\twolineshloka
{[दुर्मर्षणं हतं दृष्ट्वा वृषसेनं च संयुगे}
{प्रभग्रं च बलं दृष्ट्वा वध्यमानं महारथैः}


\twolineshloka
{पराङ्मुखांश्च राज्ञस्तु पलायनपरायणान्}
{विद्रुतान्रथिनो दृष्ट्वा मन्ये शोचति पुत्रकः}


\twolineshloka
{अनेयश्चाभिमानी च दुर्बुद्धिरजितेन्द्रियः}
{हतोत्साहं बलं दृष्ट्वा किंस्विदुर्योधनोऽब्रवीत्}


\twolineshloka
{स्वयं वैरं महत्कृत्वा वार्यमाणः सुहृद्गणैः}
{प्रधने हतभूयिष्ठैः किंस्विद्दुर्योधनोऽब्रवीत्}


\twolineshloka
{भ्रातरं निहतं दृष्ट्वा भीमसेनेन संयुगे}
{रुधिरे पीयमाने च किंस्विद्द्रुर्योधनोऽब्रवीत् ॥]}


\twolineshloka
{सह गान्धारराजेन सभायां यदभाषत}
{कर्णोऽर्जुनं रणे हन्ता हते तस्मिन्किमब्रवीत्}


\twolineshloka
{द्यूतं कृत्वा पुरा हृष्ठो वञ्चयित्वा च पाण़्डवान्}
{शकुनिः सौबलस्तात हते कर्णे किमब्रवीत्}


\twolineshloka
{कृतवर्मा महेष्वासः सात्वतानां महारथः}
{हतं वैकर्तनं दृष्ट्वा हार्दिक्यः किमभाषत}


\twolineshloka
{ब्राह्मणाः क्षत्रिया वैश्या यस्य शिक्षामुपासते}
{धनुर्वेदं चिकीर्षन्तो द्रोणपुत्रस्य धीमतः}


\twolineshloka
{युवा रूपेण सम्पन्नो दर्शनीयो महायशाः}
{अश्वत्थामा हते कर्णे किमभाषत सञ्जय}


\twolineshloka
{आचार्यो यो धनुर्वेदे गौतमो रथसत्तमः}
{कृपः शारद्वतस्तात हते कर्णे किमब्रवीत्}


\threelineshloka
{मद्रराजो महेष्वासः शल्यः समितिशोभनः}
{दृष्ट्वा विनिहतं कर्णं सारथ्ये रथिनां वरः}
{किमभाषत सौवीरो मद्राणामधिपो बली}


\threelineshloka
{दृष्ट्वा विनिहतं सर्वे योधा वारणदुर्जयाः}
{ये च केन राजानः पृथिव्यां योद्वुमागताः}
{वैकर्तनं हतं दृष्टवा कान्यभाषन्त सञ्जय}


\twolineshloka
{द्रोणे तु निहते वीरे रथव्याघ्रे नरर्षभे}
{के वा मुखमनीकानामासन्सञ्जय भागशः}


\twolineshloka
{मद्रराजः कथं शल्यो नियुक्तो रथिनां वरः}
{वैकर्तनस्य सारथ्ये तन्ममाचक्ष्व सञ्जय}


\twolineshloka
{केऽरक्षन्दक्षिणं चक्रं सूतपुत्रस्य युध्यतः}
{वामं चक्रं ररक्षुर्वा के वा वीरस्य पृष्ठतः}


\twolineshloka
{के कर्णं न जहुः शूराः के क्षुद्राः प्राद्रवंस्ततः}
{कथं च वः समेतानां हतः कर्णो महारथः}


\twolineshloka
{पाण्डवाश्च स्वयं शूराः प्रत्युदीयुर्महारथाः}
{सृजन्तः शरवर्षाणि वारिधारा इवाम्बुदाः}


\twolineshloka
{स च सर्पमुखो दिव्यो महेषुप्रवरस्तदा}
{व्यर्थः कथं समभवत्तन्ममाचक्ष्व स़ञ्जय}


\twolineshloka
{मामकस्यास्य सैन्यस्य हतोत्सेधस्य सञ्जय}
{अवशेषं न पश्यामि ककुदे मृदिते सति}


\twolineshloka
{तौ हि वीरौ महेष्वासौ मदर्थे त्यक्तजीवीतौ}
{भीष्मद्रोणौ हतौ श्रुत्वा को न्वर्थो जीवितेन मे}


\twolineshloka
{पुनः पुनर्न मृष्यामि वधं कर्णस्य पाण्डवैः}
{यस्य बाह्वोर्बलं तुल्यं कुञ्जराणां शतं शतम्}


\twolineshloka
{द्रोणे हते च यद्वृत्तं कौरवाणां परैः सह}
{सङ्घामे नरवीराणां तन्ममाचक्ष्व सञ्जय}


\twolineshloka
{यथा कर्णश्च कौन्तेयैः सह युद्धमयोजयत्}
{तथा च द्विषतां हन्ता रणे शान्तस्तदुच्यताम्}


\chapter{अध्यायः ११}
\twolineshloka
{सञ्जय उवाच}
{}


\twolineshloka
{हते द्रोणे महेष्वासे तस्मिन्नहनि भारत}
{कृते च मोघसंकल्पे द्रोणपुत्रे महारथे}


\twolineshloka
{द्रवमाणे महाराज कौरवाणां बलार्णवे}
{व्यूह्य पार्थः स्वकं सैन्यमतिष्ठद्वातृभिर्वृतः}


\twolineshloka
{तमवस्थितमाज्ञाय पुत्रस्ते भरतर्षभ}
{विद्रुतं स्वबलं दृष्ट्वा पौरुषेण न्यवारयत्}


\twolineshloka
{स्वमनीकमवस्थाप्य बाहुवीर्यमुपाश्रितः}
{युद्धा च सुचिरं कालं पाण्डवैः सह भारत}


\twolineshloka
{लब्धलक्षैः परैर्हष्टैर्व्यायच्छद्भिश्चिरं तदा}
{सन्ध्याकालं समासाद्य प्रत्याहारमकारयत्}


\twolineshloka
{`निवेश्य बलवद्वोरं क्षुत्पिपासाबलैर्युतम्}
{श्रमेण महता युक्तं तथा द्रोणवधेन च}


\twolineshloka
{दीनरूपा रणे कर्म कृत्वा घोरं च शर्वरीम्}
{विशं प्राप्य सा सेना विश्रम्य मुदिताऽभवत्'}


\twolineshloka
{कृत्वाऽवहारं सैन्यानां प्रविश्य शिबिरं स्वकम्}
{कुरवः सहिता मन्त्रं मन्त्रयाञ्चक्रिरे मिथः}


\twolineshloka
{पर्यङ्केषु परार्ध्येषु स्पर्ध्यास्तरणवत्सु च}
{वरासनेषूपविष्टाः सुखशय्यास्विवामराः}


\twolineshloka
{ततो दुर्योधनो राजा साम्ना परमवल्गुना}
{तानाभाष्य महेष्वासान्प्राप्तकालमभाषत}


\threelineshloka
{मतं मतिमतां श्रेष्ठाः सर्वे प्रब्रूत मा चिरम्}
{एवङ्गतेन यत्कार्यं भवेत्कार्यतरं नृपाः ॥सञ्जय उवाच}
{}


\twolineshloka
{एवमुक्ते नरेन्द्रेण नरसिंहा युयुत्सवः}
{चक्रुर्नानाविधाश्चेष्टाः सिंहासनगतास्तदा}


\threelineshloka
{तेषां निशाम्येङ्गितानि युद्धे प्राणाञ्जुहूषताम्}
{समुद्वीक्ष्य मुखं राज्ञो बालार्कसमवर्चसम्}
{आचार्यपुत्रो मेधावी वाक्यज्ञो वाक्यमाददे}


\twolineshloka
{रागो योगस्तथा दाक्ष्यां नयश्चेत्यर्थसाधकाः}
{उपायाः पण्डितैः प्रोक्तास्ते तु दैवमुपाश्रिताः}


\twolineshloka
{लोकप्रवीरा येऽस्माकं देवकल्पा महारथाः}
{नीतिमन्तस्तथा युक्ता दक्षा रक्ताश्च ते हताः}


\twolineshloka
{न त्वेव कार्यं नैराश्यमस्माभिर्विजयं प्रति}
{सुनीतैरिह सर्वार्थैर्दैवमप्यनुलोम्यते}


\twolineshloka
{ते वयं प्रवरं नॄणां सर्वैर्योधगुणैर्युतम्}
{कर्णमेवाभिषेभ्यामः सैनापत्येन भारत}


% Check verse!
कर्णं सेनापतिं कृत्वा प्रमथिष्यामहे रिपून्
\twolineshloka
{एष ह्यतिबलः शूरः कृतास्त्रो युद्धदुर्मदः}
{वैवस्वत इवासह्यः शक्तो जेतुं रणे निपून्}


\twolineshloka
{एतदाचार्यतनयाच्छ्रुत्वा राजंस्तवात्मजः}
{`दुर्योधनो महाराज भृशं प्रीतमनास्तदा'}


\twolineshloka
{आशां बहुमतीं चक्रे वर्णं प्रति स वै तदा}
{हते भीष्मे च द्रोणे च कर्णो जेष्यति पाण्डवान्}


\threelineshloka
{तामाशां हृदये कृत्वा समाश्वस्य च भारत}
{[ततो दुर्योधनः प्रीतः प्रियं श्रुत्वाऽस्य तद्वचः}
{प्रीतिसत्कारसंयुक्तं तथ्यमात्महितं शूभम् ॥]}


% Check verse!
स्वं मनः समवस्थाप्य बाहुवीर्यमुपाश्रितः
\twolineshloka
{`प्रियसत्कारसंयुक्तं तथ्यमात्महिते रतम्'}
{दुर्योधनो महाराज राधेयमिदमब्रवीत्}


\twolineshloka
{कर्ण जानामि ते वीर्यं सौहृदं परमं मयि}
{तथाऽपि त्वां महाबाहो प्रवक्ष्यामि हितं वचः}


\twolineshloka
{श्रुत्वा यथेष्टं च कुरु वीर यत्तव रोचते}
{भवान्प्राज्ञतमो नित्यं मम चैव परा गतिः}


\twolineshloka
{भीष्मद्रोणावतिरथौ हतौ सेनापती मम}
{सेनापतिर्भवानस्तु ताभ्यां द्रविणवत्तरः}


\twolineshloka
{वृद्धो च तौ महेष्वासौ सापेक्षो च धनञ्जये}
{मानितौ च मया वीरौ राधेय वचनात्तव}


\twolineshloka
{पितामहत्वं सम्प्रेक्ष्य पाण्डुपुत्रा महारणे}
{रक्षितास्तात भीष्मेण दिवसानि दशैव तु}


\twolineshloka
{न्यस्तशस्त्रे तु भवति हतो भीष्मः पितामहः}
{शिखण्डिनं पुरुस्कृत्य फल्गुनेन महाहवे}


\twolineshloka
{हते तस्मिन्महेष्वासे शरतल्पगते तथा}
{त्वयोक्ते पुरुषव्याघ्र द्रोणो ह्यासीत्पुरःसरः}


\twolineshloka
{तेनापि रक्षिताः पार्थाः शिष्यत्वादिति मे मतिः}
{स चापि निहतो वृद्धो धृष्टद्युम्नेन सत्वरम्}


\twolineshloka
{निहताभ्यां प्रधानाभ्यां ताभ्यामतुलविक्रम}
{त्वत्समं समरे योधं नान्यं पश्यामि चिन्तयन्}


\twolineshloka
{भवानेव तु नः शक्तो विजयाय न संशयः}
{पूर्वं मध्ये च पश्चाश्च तथैव विहितं हितम्}


\twolineshloka
{स भवान्धुर्यवत्सङ्ख्ये धुरमुद्वोऽढुमर्हति}
{अभिषेचय सैनान्ये स्वयमात्मानमात्मना}


\twolineshloka
{देवतानां यथा स्कन्दः सेनानीः प्रभुरव्ययः}
{तथा भवानिमां सेनां धार्तराष्ट्रीं बिभर्तु वै}


\threelineshloka
{जहि शत्रुगणान्सर्वान्महेन्द्रो दानवानिव}
{अवस्थितं रमे दृष्ट्वा पाण्डवास्त्वां महारथाः}
{द्रविष्यन्ति च पाञ्चाला विष्णुं दृष्ट्वेव दानवाः}


% Check verse!
तस्मात्त्वं पुरुषव्याघ्र प्रकर्षैतां महाचमूम्
\threelineshloka
{भवत्यवस्थिते यत्ते पाण्डवा मन्दचेतसः}
{द्रविष्यन्ति सहामात्याः पाञ्चालाः सृञ्जयाश्च ह}
{}


\threelineshloka
{यथा ह्यभ्युदितः सूर्यः प्रतपन्स्वेन तेजसा}
{व्यपोहति तमस्तीव्रं तथा शत्रून्प्रतापय ॥सञ्जय उवाच}
{}


\fourlineindentedshloka
{एवमुक्तस्तु राधेयो राज्ञा दुर्योधनेन ह}
{राज्ञां मध्ये महाबाहुः प्रीतात्मा स महाबलः}
{हर्षयन्नब्रवीत्कर्णो दुर्योधनमिदं वचः ॥कर्ण उवाच}
{}


\twolineshloka
{उक्तमेतन्मया पूर्वं गान्धारे तव सन्निधौ}
{जेष्यामि पाण्डवान्सर्वान्सपुत्रान्सजनार्दनान्}


\threelineshloka
{सेनापतिर्भविष्यामि तवाहं नात्र संशयः}
{स्थिरो भव महाराज जितान्विद्वि च पाण्डवान् ॥सञ्जय उवाच}
{}


\twolineshloka
{एवमुक्तो महाराज ततो दुर्याधनो नृपः}
{उत्तस्थौ राजभिः सार्धं देवैरिव शतक्रतुः}


\twolineshloka
{सैनापत्येन सत्कर्तुं कर्णं स्कन्दमिवामराः}
{ततोऽभिषिषिचुः कर्णं विधिदृष्टेन कर्मणा}


\twolineshloka
{दुर्योधनमुक्ता राजन्राजानो विजयैषिणः}
{शातकुम्भमयैः कुम्भेर्माहेयैश्चाभिमन्त्रितैः}


\twolineshloka
{तोयपूर्णैर्विषाणैश्च द्विपखङ्गमहर्षभैः}
{मणिमुक्तायुतैश्चान्यैः पुण्यगन्धैस्तथौषधैः}


\twolineshloka
{औदुम्बरे सुखासीनमासने क्षौमसंवृते}
{शास्त्रदृष्टेन विधिना सम्भारैश्च सुसम्भृतैः}


\twolineshloka
{ब्राह्मणः क्षत्रिया वैश्यास्तथा शूद्राश्च सम्मताः}
{तुष्टुवुस्तं महात्मानमभिषिक्तं वरासने}


\twolineshloka
{ततोऽभिषिक्ते राजेन्द्र निष्कैर्गोभिर्धनेन च}
{वाचयामास विप्राग्र्यान्राधेयः परवीरहा}


\twolineshloka
{`स व्यरोचत राधेयः सूतमागधबन्दिभिः}
{स्तूयमानो यथा भानुरुदये ब्रह्मवादिभिः}


\twolineshloka
{ततः पुण्याहघोषेण वादित्रनिनदेन च}
{जयशब्देन शूराणां तुमुलः सर्वतोऽभवत्}


% Check verse!
जयेत्यूचुर्नृपाः सर्वे राधेयं तत्र सङ्गताः'
\twolineshloka
{जय पार्थान्सगोविन्दान्सानुगांस्तान्महामृधे}
{इति तं बन्दिनः प्राहुर्द्विजाश्च पुरुषर्षभम्}


\twolineshloka
{जहि पार्थान्सपाञ्चालान्राधेय विजयाय नः}
{उद्यन्निव सदा भानुस्तमांस्युग्रैर्गभस्तिभिः}


\twolineshloka
{न ह्यलं त्वद्विसृष्टानां शराणां वै सकेशवाः}
{उलूकाः सूर्यरश्मीनां ज्वलतामिव दर्शने}


\threelineshloka
{न हि पार्थाः सपाञ्चालाः स्थातुं शक्तास्तवाग्रतः}
{आत्तवज्रस्य समरे महेन्द्रस्येव दानवाः ॥`सञ्जय उवाच}
{}


\twolineshloka
{स सत्कृतः स्तूयमानः सुहृद्गणवृतो रुषा}
{कर्णो दुर्योधनं वाक्यमब्रवीत्प्रहसन्प्रियम्}


\twolineshloka
{दुर्योधनाद्य सगणं पाण्डूनां प्रवरैः सह}
{फल्गुनं सूदयिष्यामि त्वत्प्रियार्थं सबान्धवम्}


\twolineshloka
{सपर्वतार्णवद्वीपां शाधि गां गतपाण्डवाम्}
{पुत्रपौत्रप्रपौत्रेषु प्रतिष्ठां गमयिष्यमि}


\twolineshloka
{नासह्यं विद्यते मह्यं त्वत्प्रियार्थमरिन्दम}
{सत्यधर्मानुरक्तस्य सिद्धिरात्मवतो यथा}


\twolineshloka
{अभिषिक्तस्तु राघेयः प्रभया सोऽमितप्रभः}
{अत्यरिच्यत रूपेण दिवाकर इवापरः}


\twolineshloka
{सैनापत्ये तु राधेयमभिषिच्य सुतस्तव}
{अमन्यत तदात्मानं कृतार्थं कालचोदितः}


\twolineshloka
{कर्णोऽपि राज्ञः संप्राप्य सैनापत्यमरिन्दमः}
{योगमाज्ञापयामास सूर्यस्योदयनं प्रति}


\twolineshloka
{तव पुत्रैर्वृतः कर्णः शुशुभे तत्र भारत}
{देवैरिव यथा स्कन्दः सङ्ग्रामे तारकामये}


\chapter{अध्यायः १२}
\twolineshloka
{धृतराष्ट्र उवाच}
{}


\twolineshloka
{सैनापत्यं तु सम्प्राप्य कर्णो वैकर्तनस्तदा}
{तथोक्तश्च स्वयं राज्ञा स्निग्धं भ्रातृसमं वचः}


\threelineshloka
{हितश्च प्रियकामश्च मम पुत्रस्य नित्यशः}
{अकरोत्किं महाप्राज्ञस्तन्ममाचक्ष्व स़ञ्जय ॥सञ्जय उवाच}
{}


\twolineshloka
{कर्णस्य मतमाज्ञाय पुत्रास्ते भरतर्षभ}
{योगमाज्ञापयामासुर्नन्दितूर्यपुरःसरम्}


\twolineshloka
{महत्यपररात्रे च तव सैन्यस्य मारिष}
{योगो योग इति ह्याशु प्रादुरासीन्महास्वनः}


\twolineshloka
{कल्पतां नागमुख्यानां रथानां च वरुथिनाम्}
{सन्नह्यतां नराणां च वाजिनां च विशाम्पते}


\twolineshloka
{क्रोशतां चैव योधानां त्वरितानां परस्परम््}
{बभूव तुमुलः शब्दो दिवस्पृक्सुमहांस्ततः}


\twolineshloka
{ततः श्वेतपताकेन बलाकावर्णवाजिना}
{हेमपृष्ठेन धनुषा नागकक्षेण केतुना}


\twolineshloka
{तूणीरशतपूर्णेन सगदेन वरूथिना}
{शतघ्नीकिङ्किणीशक्तिशूलतोमरधारिणा}


\twolineshloka
{कार्मुकैरुपपन्नेन विमलादित्यवर्चसा}
{रथेनाभिपताकेन सुतपुत्रो ह्यदृश्यत}


\twolineshloka
{ध्मापयन्वारिजं राजन्हेमजालविभूषिन्तम्}
{विध्रुन्वानो महच्चापं कार्तस्वरविभूषितम्}


\twolineshloka
{दृष्ट्वा कर्णं महेष्वासं रथस्थं रथिनां वरम्}
{भानुमन्तमिवोद्यन्तं तमो घ्नन्तं दुरासदम्}


\twolineshloka
{न भीष्मव्यसनं केचिन्नापि द्रोणस्य मारिष}
{नान्येषां पुरुषव्याघ्र मेनिरे तत्र कौरवाः}


\twolineshloka
{ततस्तु त्वरयन्योधाञ्शङ्खशब्देन मारिष}
{कर्णो निष्कर्षयामास कौरवाणां महद्बलम्}


\twolineshloka
{व्यूहं व्यूह्य महेष्वासो मकरं शत्रुतापनः}
{प्रत्युद्ययौ तथा कर्णः पाण्डवान्विजिगीषया}


\twolineshloka
{मकरस्य तु तुण्डे वै कर्णो राजन्व्यवस्थितः}
{नेत्राभ्यां शकुनिः शूर उलूकश्च महारथः}


\twolineshloka
{द्रोणपुत्रस्तु शिरसि ग्रीवायां सर्वसोदराः}
{मध्येदुर्योधनो राजा बलेन महता वृतः}


\twolineshloka
{वामपादे तु राजेन्द्र कृतवर्मा व्यवस्थितः}
{नारायणबलैर्युक्तो गोपालैर्युद्धदुर्मदैः}


\twolineshloka
{पादे तु दक्षिणे राजन्गौतमः सत्यविक्रमः}
{त्रिगर्तैः सुमहेष्वासैर्दाक्षिणात्यैश्च संवृतः}


\twolineshloka
{अनुपादे तु यो वामस्तत्र शल्यो व्यवस्थितः}
{महत्या सेनया सार्धं मद्रदेशसमुत्थया}


\twolineshloka
{दक्षिणे तु महाराज सुषेणः सत्यसङ्गरः}
{वृतो रथमहस्रेण दन्तिनां च त्रिभिः शतैः}


\twolineshloka
{पुच्छे ह्यास्तां महावीर्यौ भ्रातरौ पार्थिवौ तदा}
{चित्रश्च चित्रसेनश्च महत्या सेनया वृतौ}


\twolineshloka
{तथा प्रयाते राजेन्द्र कर्णे नरवरोत्तमे}
{धनञ्जयमभिपेक्ष्य धर्मराजोऽब्रवीदिदम्}


\twolineshloka
{पश्य पार्थ यथा सेना धार्तराष्ट्रीह संयुगे}
{कर्णेन विहिता वीर गुप्ता वीरैर्महारथैः}


\twolineshloka
{हतवीरतमा ह्येषा धार्तराष्ट्री महाचमूः}
{फल्गुशेषा महाबाहो तृणैस्तुल्या मता मम}


\threelineshloka
{एको ह्यत्र महेष्वासः सूतपुत्रो विराजते}
{सदेवासुरगन्धर्वैः सकिन्नरमहोरगैः}
{चराचरैस्त्रिभिर्लोकैरजेयो यो महारथः}


\twolineshloka
{तं हत्वाऽद्य महाबाहो विजयस्तव फल्गुन}
{उद्धृतश्च भवेच्छल्यो मम द्वादशवार्षिकः}


\twolineshloka
{एवं ज्ञात्वा महाबाहो व्यूहं व्यूह यथेच्छसि ॥सञ्जय उवाच}
{}


\twolineshloka
{भ्रातुरेतद्वचः श्रुत्वा पाण्डवः श्वेतवाहनः}
{अर्धचन्द्रेण व्यूहेन प्रत्यव्यूहत तां चमूम्}


\twolineshloka
{वामपार्श्वे तु तस्याथ भीमसेनो व्यवस्थितः}
{दक्षिणे च महेष्वासो धृष्टद्युम्नो व्यवस्थितः}


\twolineshloka
{मध्ये व्यूहस्य राजा तु पाण्डवश्च धनञ्जयः}
{नकुलः सहदेवश्च धर्मराजस्य पृष्ठतः}


\twolineshloka
{चक्ररक्षौ तु पाञ्चाल्यौ युधामन्यूत्तमौजसौ}
{पार्थं न जहतुर्युद्वे पाल्यमानौ किरीटिना}


\twolineshloka
{शेषां नृपतयो वीराः स्थिता व्यूहस्य दंशिताः}
{यथाभागं यथोत्साहं यथायत्नं च भारत}


\twolineshloka
{एवमेतन्महाव्यूहं व्यूह्य भारत पाण्डवाः}
{तावकाश्च महेष्वासा युद्धायैव मनो दधुः}


\twolineshloka
{दृष्ट्वा व्यूढां तव चमूं सूतपुत्रेण संयुगे}
{निहतान्पाण्डवान्मेने धार्तराष्ट्रः सबान्धवः}


\twolineshloka
{तथैव पाण्डवीं सेनां व्यूढां दृष्ट्वा युधिष्ठिरः}
{धार्तराष्ट्रान्हतान्मेने सकर्णान्वै जनाधिपः}


\twolineshloka
{ततः शङ्खाश्च भेर्यश्च पणवानकगोमुखाः}
{डिण्डिमाश्चाप्यहन्यन्त झर्झराश्च समन्ततः}


\twolineshloka
{सेनयोरुभयो राजन्प्रावाद्यन्त महास्वनाः}
{सिंहनादश्च सञ्जज्ञे शूराणां जयगृद्धिनाम्}


\twolineshloka
{हयहेषितशब्दाश्च वारणानां च बृंहिताः}
{रथनेमिस्वनाश्चोग्राः सम्बभूवुर्जनाधिप}


\twolineshloka
{न द्रोणव्यसनं कश्चिज्जानीते तत्र भारत}
{दृष्ट्वा कर्णं महेष्वासं मुखे व्यूहस्य दंशितम्}


\twolineshloka
{उभे सैन्ये महाराज प्रहृष्टनरसङ्कुले}
{योद्वुकामे स्थिते राजन्हन्तुमन्योन्यमोजसा}


\twolineshloka
{विजये जातसंरम्भे दृष्ट्वाऽन्योन्यं व्यवस्थिते}
{अनीकमध्ये राजेन्द्र चेरतुः कर्णपाण्डवौ}


\twolineshloka
{नृत्यन्त्याविव ते सेने समेयातां परस्परम्}
{तयोः पक्षप्रपक्षेभ्यो निर्ययुर्युद्वलिप्सवः}


\twolineshloka
{ततः प्रववृते युद्धं नरवारणवाजिनाम्}
{रथानां च महाराज अन्योन्यमभिनिघ्नताम्}


\chapter{अध्यायः १३}
\twolineshloka
{सञ्जय उवाच}
{}


\twolineshloka
{ते सेनेऽन्योन्यमासाद्य प्रहृष्टाश्वनरद्विपे}
{बृहत्यौ सम्प्रजहाते देवासुरचमूसमे}


\twolineshloka
{ततो नानारथाश्वेभाः पत्तयश्चोग्रविक्रमाः}
{सम्प्रहारान्भृशं चक्रुर्देहपाप्मविनाशनान्}


\twolineshloka
{पूर्णचन्द्रार्कपद्मानां कान्तित्विङ्गन्धतः समैः}
{उत्तमाङ्गैर्नृसिंहानां नृसिंहास्तस्तरुर्महीम्}


\twolineshloka
{अर्धचन्द्रैस्तथा भल्लैः क्षुरप्रैरसिपट्टसैः}
{परश्वथैश्चापहृतान्युत्तमाङ्गानि युध्यताम्}


\twolineshloka
{व्यायतायतबाहूनां व्यायतायतबाहुभिः}
{वाहवः पातिता रेजुर्धरण्यां सायुधाङ्गदैः}


\twolineshloka
{तैः स्फुरद्भिर्मही भाति रक्ताङ्गुलितलैस्तथा}
{गरुडप्रहितैरुग्रैः पञ्चास्यैरुरगैरिव}


\twolineshloka
{द्विरदस्यन्दनाश्वेभ्यः पेतुर्वीरा द्विषद्धताः}
{विमानेभ्यो यथा क्षीणे पुण्ये स्वर्गदस्तथा}


\twolineshloka
{गदाभिरन्ये गुर्वीभिः परिधैर्मुसलैरपि}
{पोथिताः शतशः पेतुर्वीरा वीरतरै रणे}


\twolineshloka
{रथा रथैर्विमथिता मत्ता मत्तैर्द्विपा द्विपैः}
{सादिनः सादिभिश्चैव तस्मिन्परमसङ्कुले}


\twolineshloka
{रथैर्नरा रथा नागैरश्वारोहाश्च पत्तिभिः}
{अश्वारोहैः पदाताश्च निहता युधि शेरते}


\twolineshloka
{रथाश्वपत्तयो नागै रथाश्वेभाश्च पत्तिभिः}
{रथपत्तिद्विपाश्चाश्चै रथैश्चापि नरद्विपाः}


\twolineshloka
{रथाश्वेभनराणां तु नराश्वेभरथैः कृतम्}
{पाणिपादैश्च शस्त्रैश्च रथैश्च कदनं महत्}


\twolineshloka
{तथा तस्मिन्बले शूरैर्वध्यमाने हतेऽपि च}
{अस्मानभ्याययुः पार्था वृकोदरपुरोगमाः}


\twolineshloka
{वृष्टद्युम्नः शिखण्डी च द्रौपदेयाः प्रभद्रकाः}
{सात्यकिश्चेकितानश्च द्राविडैः सैनिकैः सह}


\twolineshloka
{वृता व्यूहेन महता पाण्ड्याश्चोलाः सकेरलाः}
{व्यूढोरस्का दीर्घभुजाः प्रांशवः पृथुलोचनाः}


\twolineshloka
{आपीडिनो रक्तदन्ता मत्तमातङ्गविक्रमाः}
{नानाविरागवसाना गन्धचूर्णावचूर्णिताः}


\twolineshloka
{बद्धासयः पाशहस्ता वारणप्रतिवारणाः}
{समानमृत्यवो राजन्नात्यजन्त परस्परम्}


\twolineshloka
{कलापिनश्चापहस्ता दीर्घकेशाः प्रियंवदाः}
{पत्तयः सायकैर्विद्धा घोररूपपराक्रमाः}


\twolineshloka
{अथापरे पुनः शूराश्चेदिपाञ्चालकेकयाः}
{कारूशाः कोसलाः काञ्च्या मागधाश्चापि दुद्रुवुः}


\twolineshloka
{तेषां रथाश्वनागाश्च प्रवराश्चोग्रपत्तयः}
{नानाबाणरवैर्हृष्टा नृत्यन्ति च हसन्ति च}


\twolineshloka
{तस्य सैन्यस्य महतो महामात्रवरैर्वृतः}
{मध्ये वृकोदरोऽभ्यायात्त्वदीयान्नागधूर्गतः}


\twolineshloka
{स नागप्रवरोऽत्युग्रो विधिवत्कल्पितो बभौ}
{उदयाग्राद्रिभवनं यथाऽभ्युदितभास्करम्}


\twolineshloka
{तस्यायसं वर्मवरं वररत्नविभूषितम्}
{ताराव्याप्तस्य नभसः शारदस्य समं त्विषा}


\twolineshloka
{स तोमरव्यग्रकरश्चारुमौलिः स्वलङ्कृतः}
{शरन्मध्यन्दिनार्काभस्तेजया प्रदाहद्रिपून्}


\twolineshloka
{तं दृष्ट्वा द्विरदं दूरात्क्षेमधूर्तिर्द्विपस्थित}
{आह्वयन्नभिदुद्राव प्रहसन्पृतनामुखे}


\twolineshloka
{तयोः समभवद्युद्धं द्विपयोरुग्ररूपयोः}
{यदृच्छया द्रुतवतोर्महापर्वतयोरिव}


\twolineshloka
{संसक्तनागौ तौ वीरौ तोमरैरितरेतरम्}
{बलवत्सूर्यरश्म्याभैर्भित्त्वाऽन्योन्यं विनेदतुः}


\twolineshloka
{व्यपसृत्य तु नागाभ्यां मण्डलानि विचेरतुः}
{प्रगृह्य चोभौ धनुषी जघ्नतुर्वै परस्परम्}


\twolineshloka
{क्ष्वेडितास्फोटितरवैर्बाणशब्दैस्तु सर्वतः}
{तौ जनं हर्षयन्तौ च सिंहनादं प्रचक्रतुः}


\twolineshloka
{समुद्यतकराभ्यां तौ द्विपाभ्यां कृतिनावुभौ}
{वातोद्वूतपताकाभ्यां युयुधाते महाबलौ}


\twolineshloka
{तावन्योन्यस्य धनुषी छित्त्वाऽन्योन्यं विनेदतुः}
{शक्तितोमरवर्षेण प्रावृण्मेघाविवाम्बुभिः}


\twolineshloka
{क्षेमधूर्तिस्तदा भीमं तोमरेण स्तनान्तरे}
{निर्बिभेदातिवेगेन षड्भिश्चाप्यपरैर्नदन्}


\twolineshloka
{स भीमसेनः शुशुभे तोमरैरङ्गमाश्रितैः}
{क्रोधदीप्तवपुर्मेधैः सप्तसप्तिरिवांशमान्}


\twolineshloka
{ततो भास्करवर्णाभमञ्चोगतिमयस्मायम्}
{ससर्ज तोमरं भीमः प्रत्यमित्राय यत्नवान्}


\twolineshloka
{ततः करूशाधिपतिश्चापमानम्य सायकैः}
{दशभिस्तोमरं भित्त्वा षष्ट्या विव्याध पाण्डवम्}


\twolineshloka
{अथ कार्मुकमादाय भीमो जलदनिः स्वनम्}
{रिपोरभ्यर्दयन्नागमुन्नदन्पाण्डवाः शरैः}


\twolineshloka
{स शरौघार्दितो नागो भीमसेनेन संयुगे}
{गृह्यमाणोऽपि नातिष्ठद्वातोद्धूत इवाम्बुदः}


\twolineshloka
{तमभ्यधावद्द्विरदं भीमो भीमस्य नागराट्}
{महावातेरितं मेघं वातोद्धूत इवाम्बुदः}


\twolineshloka
{सन्निवार्यात्मनो नागं क्षेमधूर्तिः प्रतापवान्}
{विव्याधाभिद्रुतं बाणैर्भीमसेनस्य कुञ्जरम्}


\twolineshloka
{ततः साधुविसृष्टेन क्षुरेणानतपर्वणा}
{छित्त्वा शरासनं शत्रोर्नागं चापि प्रमार्दयत्}


\twolineshloka
{ततः क्रुद्धो रणे भीमं क्षेमधूर्तिः पराभिनत्}
{जघान चास्य द्विरदं नाराचैः सर्वमर्मसु}


\twolineshloka
{स पपात महानागो भीमसेनस्य भारत}
{पुरा नागस्य पतनादवप्लुत्य स्थितो महीम्}


\twolineshloka
{भीमसेनोऽपि तन्नागं गदया समपोथयत्}
{तस्मात्प्रमथितान्नागात्क्षेमधूर्तिरवप्लुतः}


\twolineshloka
{`उद्धृत्य स्वङ्गं निशितमभ्यधावत्स पाण्डवम्'}
{उद्यतायुधमायान्तं सदयाऽहन्वृकोदरः}


\twolineshloka
{स पपात हतः सासिर्व्यसुस्तमभितो द्विपम्}
{वज्रप्रभग्नमचलं सिंहो वज्रहतो यथा}


\twolineshloka
{तं हतं नृपतिं दृष्ट्वा करूशानां यशस्करम्}
{प्राद्रावद्व्यथिता सेना त्वदीया भरतर्षभ}


\chapter{अध्यायः १४}
\twolineshloka
{सञ्चय उवाच}
{}


\twolineshloka
{ततः कर्णो महेष्वासः पाण्डवानामनीकिनीम्}
{जघान समरे शूरः सन्नतपर्वभिः}


\twolineshloka
{तथैव पाण्डवा राजंस्तव पुत्रस्य वाहिनीम्}
{कर्णस्य प्रमुखे क्रुद्धा निजघ्नुस्ते महारथाः}


\twolineshloka
{कर्णोऽपि राजन्समरे व्यहनत्पाण्डवीं चमूम्}
{नाराचैरर्करश्म्याभैः कर्मारपरिमार्जितैः}


\twolineshloka
{तत्र भारत कर्णेन नाराचैस्ताडिता गजाः}
{नेदुः सेदुश्च मम्लुश्च बभ्रुमुश्च दिशो दश}


\twolineshloka
{वध्यमाने बले तस्मिन्सूतपुत्रेण मारिष}
{नकुलोऽभ्यद्रवत्तूर्णं सूतपुत्रं महारणे}


\twolineshloka
{भीमसेनस्तथा द्रौणिं कुर्वाणं कर्म दुष्करम्}
{विन्दानुविन्दौ कैकेयौ सात्यकिः समवारयत्}


\twolineshloka
{श्रुतकर्माणमायान्तं चित्रसेनो महीपतिः}
{प्रतिविन्ध्यस्तथा चित्रं चित्रकेतनकार्मुकम्}


\twolineshloka
{दुर्योधनस्तु राजानं धर्मपुत्रं युधिष्ठिरम्}
{संशप्तकगणा हृष्टा ह्यभ्यधावन्धनञ्जयम्}


\twolineshloka
{धृष्टद्युम्नः कृपं चापि तस्मिन्वीरवरक्षये}
{शिखण्डी कृतवर्माणं समासादयदच्युतम्}


\twolineshloka
{श्रुतकीर्तिस्तथा शल्यं माद्रीपुत्रः सुतं तव}
{दुःशासनं महाराज सहदेवः प्रतापवान्}


\twolineshloka
{कैकेयौ सात्यकिं युद्धे शरवर्षेण भास्वता}
{सात्यकिः केकयौ चापि च्छादयामास भारत}


\twolineshloka
{तावेनं भ्रातरौ वीरौ जघ्नतुर्हृदये भृशम्}
{विषाणाभ्यां यथा नागौ प्रतिनागं महावने}


\twolineshloka
{शरसम्बिन्नवर्माणौ तावुभौ भ्रातरौ रणे}
{सात्यकिं सत्यकर्माणं राजन्विव्यधतुः शरैः}


\twolineshloka
{तौ सात्यकिर्महाराज प्रहसन्सर्वतोदिशः}
{छादयञ्छरवर्षेण वारयामास भारत}


\twolineshloka
{वार्यमाणौ ततस्तौ हि शैनेयशरवृष्टिभिः}
{शैनेयस्य रथं तूर्णं छादयामासतुः शरैः}


\twolineshloka
{तयोस्तु धनुषी चित्रे छित्त्वा शौरिर्महायशाः}
{अथ तौ सायकैस्तीक्ष्णैर्वारयामास सात्यकिः}


\twolineshloka
{अथान्ये धनुषी चित्रे प्रगृह्य च महाशरान्}
{सात्यकिं छादयन्तौ तौ चेरतुर्लघु सुष्ठु च}


\twolineshloka
{ताभ्यां मुक्ता महाबाणाः कङ्कबर्हिणवाससः}
{द्योतयन्तो दिशः सर्वाः सम्पेतुः स्वर्णभूषणाः}


\twolineshloka
{बाणान्धकारमभवत्तयो राजन्महामृधे}
{अन्योन्यस्य धनुश्चैव चिच्छिदुस्ते महारथाः}


\twolineshloka
{ततः क्रुद्धो महाराज सात्वतो युद्धदुर्मदः}
{धनुरन्यत्समादाय सज्यं कृत्वा च संयुगे}


\twolineshloka
{क्षुरप्रेण सुतीक्ष्णेन ह्यनुविन्दशिरोऽहरत्}
{अपतत्तच्छिरो राजन्कुण़्डलोपचितं महत्}


\twolineshloka
{शम्बरस्य शिरो यद्वन्निहतस्य महारणे}
{शोचयन्केकयान्सर्वाञ्जगामाशु वसुन्धराम्}


\twolineshloka
{तं दृष्ट्वा निहतं शूरं भ्राता तस्य महारथः}
{सज्यमन्यद्धनुः कृत्वा शैनेयं पर्यवारयत्}


\twolineshloka
{स षष्ट्या सात्यकिं विद्व्वा स्वर्णपुङ्खैः शिलाशितैः}
{ननाद बलवन्नादं तिष्ठतिष्ठेति चाब्रवीत्}


\twolineshloka
{सात्यकिं च ततस्तूर्णं केकयानां महारथः}
{शरैरनेकसाहस्रैर्बाह्वोरुरसि चार्पयत्}


\twolineshloka
{स शरैः क्षतसर्वाङ्गः सात्यकिः सत्यविक्रमः}
{रराज समरे राजन्सपुष्प इव किंशुकः}


\twolineshloka
{सात्यकिः समरे विद्वः कैकेयेन महात्मना}
{कैकेयं पञ्चविंशत्या विव्याध प्रहसन्निव}


\twolineshloka
{तावन्योन्यस्य समरे सञ्छिद्य धनुषी शुभे}
{हत्वा च सारथी तूर्णं हयांश्च रथिनां वरौ}


\twolineshloka
{विरथावसियुद्धाय समाजग्मतुराहवे}
{शतचन्द्रचिते गृह्य चर्मणी सुभुजौ तथा}


\twolineshloka
{विरोचेतां महारङ्गे निस्त्रिंशबरधारिणौ}
{यथा देवासुरे युद्धे जम्भशक्रौ महाबलौ}


\twolineshloka
{मण्डलानि ततस्तौ तु विचन्तौ महारणे}
{अन्योन्यमभितस्तूर्णं समाजग्मतुराहवे}


\twolineshloka
{अन्योन्यस्य वधे चैव चक्रतुर्यत्नमुत्तमम्}
{कैकेयस्य द्विधा चर्म ततश्चिच्छेद सात्वतः}


\threelineshloka
{सात्यकेस्तु तथैवासौ चर्म चिच्छेद पार्थिवः}
{चर्म च्छित्त्वा तु कैकेयस्तारागणशतैर्वृतम्}
{चचार मण्डलान्येव गतप्रत्यागतानि च}


\twolineshloka
{तं चरन्तं महारङ्गे निस्त्रिंशवरधारिणम्}
{अपहस्तेन चिच्छेद शैनेयस्त्वरयाऽन्वितः}


\twolineshloka
{सवर्मा केकयो राजन्द्विधा च्छिन्नो महारणे}
{निपपात महेष्वासो वज्राहत इवात्वलः}


\twolineshloka
{तं निहत्य रणे शूरः शैनेयो रथसत्तमः}
{युधामन्युरथं तूर्णमारुरोह परन्तपः}


\twolineshloka
{ततोऽन्यं रथमास्थाय विधिवत्कल्पितं पुनः}
{केकयानां महत्सैन्यं व्यधमत्सात्यकिः शरैः}


\twolineshloka
{सा वध्यमाना समरे केकयानां महाचमूः}
{तमुत्सृज्य रणे शत्रुं प्रदुद्राव दिशो दश}


\chapter{अध्यायः १५}
\twolineshloka
{सञ्जय उवाच}
{}


\twolineshloka
{श्रुतकर्मा ततो राजंश्चित्रसेनं महीपतिम्}
{आजघ्ने समरे क्रुद्धः पञ्चाशद्भिः शिलीमुखैः}


\twolineshloka
{चित्रसेनस्तु तं राजन्नवभिर्नतपर्वभिः}
{श्रुतकर्माणमाहत्य सूतं विव्याध पञ्चभिः}


\twolineshloka
{श्रुतकर्मा ततः क्रुद्धश्चित्रसेनं चमूमुखे}
{नाराचेन सुतीक्ष्णेन मर्मदेशे समार्पयत्}


\twolineshloka
{सोऽतिविद्धो महाराज नाराचेन महात्मना}
{मूर्च्छामभिययौ वीरः कश्मलं चाविवेश ह}


\twolineshloka
{एतस्मिन्नन्तरे चैनं श्रुतकीर्तिर्महायशाः}
{नवत्या जगतीपालं छादयामास पत्रिभिः}


\twolineshloka
{प्रतिलभ्य ततः संज्ञां चित्रसेनो महारथः}
{धनुश्चिच्छेद भल्लेन तं च विव्याध सप्तभिः}


\twolineshloka
{सोऽन्यत्कार्मुकमादाय वेगघ्नं रुक्मभूषितम्}
{चित्ररूपधरं चक्रे चित्रसेनं शरोर्मिभिः}


\twolineshloka
{स शरैश्चित्रितो राजा चित्रमाल्यधरो युवा}
{अशोभत महारङ्गे श्वाविच्छललो यथा}


\twolineshloka
{श्रुतकर्माणमथ वै नाराचेन स्तनान्तरे}
{बिभेद तरसा शूरस्तिष्ठतिष्ठेति चाब्रवीत्}


\twolineshloka
{श्रुतकर्माणि समरे नाराचेन समर्पितः}
{सुस्राव रुधिरं तत्र गैरिकाम्बु यथाऽचलः}


\twolineshloka
{ततः स रुधिराक्ताङ्गो रुधिरेण कृतच्छविः}
{रराज समरे वीरः सपुष्प इव किंशुकः}


\twolineshloka
{श्रुतकर्मा ततो राजञ्शत्रुणा समभिद्रुतः}
{शत्रुसंवारणं क्रुद्धो द्विधा चिच्छेद कार्मुकम्}


\twolineshloka
{अथैनं छिन्नधन्वानं नाराचानां शतैस्त्रिभिः}
{छादयन्समरे राजन्विव्याध च सुपत्रिभिः}


\twolineshloka
{ततोऽपरेण भल्लेन तीक्ष्णेन निशितेन च}
{जहार सशिरस्त्राणं शिरस्तस्य महात्मनः}


\twolineshloka
{तच्छिरो न्यपतद्भूमौ चित्रसेनस्य दीप्तिमत्}
{यदृच्छया यथा चन्द्रश्च्युतः स्वर्गान्महीतलम्}


\twolineshloka
{राजानं निहतं दृष्ट्वा तेऽभिसारं तु मारिष}
{अभ्यद्रवन्त वेगेन चित्रसेनस्य सैनिकाः}


\twolineshloka
{ततः क्रुद्धो महेष्वासस्तत्सैन्यं प्राद्रवच्छरैः}
{अन्तकाले यथा क्रुद्धः सर्वभूतानि प्रेतराट्}


\twolineshloka
{ते वध्यमानाः समरे तव पौत्रेण धन्विना}
{व्यद्रवन्त दिशस्तूर्णं दावदग्धा इव द्विपाः}


\twolineshloka
{तांस्तु विद्रवतो दृष्ट्वा निरुत्साहान्द्विषज्जये}
{द्रावयन्निषुभिस्तीक्ष्णैः श्रुतकर्मा व्यरोचत}


\twolineshloka
{प्रतिविन्ध्यस्ततश्चित्रं भित्त्वा पञ्चभिराशुगैः}
{सारथिं च त्रिभिर्विद्व्वा ध्वजमेकेषुणापि च}


\twolineshloka
{तं चित्रो नवभिर्भल्लैर्बाह्वोरुरसि चार्पयत्}
{स्वर्णपुङ्खैः प्रसन्नाग्नैः कङ्कबर्हिणवाजितैः}


\twolineshloka
{प्रतिविन्ध्यो धनुश्छित्त्वा तस्य बारत सायकेः}
{पञ्चभिर्निशितैर्बाणैरथैनं स हि जघ्निवान्}


\twolineshloka
{ततः शक्तिं महाराज स्वर्णघण्टां दुरासदाम्}
{प्राहिणोत्प्रतिविन्ध्याय विचिन्वन्तीमसूनिव}


\twolineshloka
{तामापतन्ती सहसा शक्तिमुल्कामिवाम्बरे}
{द्विधा चिच्छेद समरे प्रतिविन्ध्यो हसन्निव}


\twolineshloka
{सा पपात द्विधा छिन्ना प्रतिविन्ध्यशरैः शितैः}
{युगान्ते सर्वभूतानि त्रासयन्ती यथाऽशनिः}


\twolineshloka
{शक्तिं तां प्रहतां दृष्ट्वा चित्रो गृह्य महागदाम्}
{प्रतिविन्ध्याय चिक्षेप रुक्मजालविभूषिताम्}


\twolineshloka
{सा जघान हयांस्तस्य सारथिं च महारणे}
{रथं प्रमृद्य वेगेन धरमीमन्वपद्यत}


\twolineshloka
{एतस्मिन्नेव काले तु रथादाप्लुत्य भारत}
{शक्तिं चिक्षेप चित्राय स्वर्णदण्डामलङ्कृताम्}


\twolineshloka
{तामापतन्तीं जग्राह चित्रो राजन्महामनाः}
{ततस्तामेव चिक्षेप प्रतिविन्ध्याय पार्थिवः}


\threelineshloka
{समासाद्य रणे शूरं प्रतिविन्ध्यं महाप्रभा}
{निर्भिद्य दक्षिणं बाहुं निपपात महीतले}
{पतिताऽभासयश्चैव तं देशमशनिर्यथा}


\twolineshloka
{प्रतिविन्ध्यस्ततो राजंस्तोमरं हेमभूषितम्}
{प्रेषयामास सङ्क्रुद्धश्चित्रस्य वधकाङ्क्षया}


\twolineshloka
{स तस्य गात्रावरणं भित्त्वा हृदयमेव च}
{जगाम धरणीं तूर्णं महोरग इवाशयम्}


\twolineshloka
{स पपात तदा राजा तोमरेण समाहतः}
{प्रसार्य विपुलौ बाहू पीनौ पिघसन्निभौ}


\twolineshloka
{चित्रं सम्प्रेक्ष्य निहतं तावका रणशोभिनः}
{अभ्यद्रवन्त वेगेन प्रतिविन्ध्यं समन्ततः}


\twolineshloka
{सृजन्तो विविधान्बाणाञ्शतघ्नीश्च सकिङ्किणीः}
{तमवच्छादयामासुः सूर्यमभ्रगणा इव}


\twolineshloka
{तान्विधम्य महाबाहुः शरजालेन संयुगे}
{व्यद्रावयत्तव चमूं वज्रहस्त इवासुरीम्}


\twolineshloka
{ते वध्यमानाः समरे तावकाः पाण्डवैर्नृप}
{विप्रकीर्यन्त सहसा वातनुन्ना घना इव}


\twolineshloka
{विप्रद्रुते बले तस्मिन्वध्यमाने समन्ततः}
{द्रौणिरेकोऽभ्ययात्तूर्णं भीमसेनं महाबलम्}


\twolineshloka
{तः समागमो घोरो बभूव सहसा तयोः}
{यथा दैवासुरे युद्धे वृत्रवासवयोरिव}


\chapter{अध्यायः १६}
\twolineshloka
{सञ्जय उवाच}
{}


\twolineshloka
{भीमसेनं ततो द्रौणी राजन्विव्याध पत्रिणा}
{परया त्वरया युक्तो दर्शयन्नस्त्रलाघवम्}


\twolineshloka
{अथैनं पुनराजघ्ने नवत्या निशितैः शरैः}
{सर्वमर्माणि सम्प्रेक्ष्य मर्मज्ञो लघुहस्तवत्}


\twolineshloka
{भीमसेनः समाकीर्णो द्रौणिना निशितैः शरैः}
{रराज समरे राजन्रश्मिवानिव भास्करः}


\twolineshloka
{ततः शरसहस्रेण सुप्रयुक्तेन पाण्डवः}
{द्रोणपुत्रमवच्छाद्य सिंहनादममुञ्चत}


\twolineshloka
{शरैः सरांस्ततो द्रौणिः संवार्य युधि पाण्डवम्}
{ललाटेऽभ्याहनद्राजन्नाराचेन स्मयन्निव}


\twolineshloka
{ललाटस्थं ततो बाणं धारयामास पाण्डवः}
{यथा शृङ्गं वने दृप्तः खङ्गो धारयते नृप}


\twolineshloka
{ततो द्रौणिं रणे भीमो यतमानं पराक्रमी}
{त्रिभिर्विव्याध नाराचैर्ललाटे विस्मयन्निव}


\twolineshloka
{ललाटस्थैस्ततो बाणैर्ब्राह्मणोऽसौ व्यशोभत}
{प्रावृषीव यथा सिक्तस्त्रिशृङ्गः पर्वतोत्तमः}


\twolineshloka
{ततः शरशतैर्द्रौणिरर्दयामास पाण्डवम्}
{न चैनं कम्पयामास मातरिश्वेव पर्वतम्}


\twolineshloka
{तथैव पांण्डवो युद्धे द्रौणिं शरशतैः शितैः}
{नाकम्पयत संहृष्टो वार्योध इव पर्वतम्}


\twolineshloka
{तावन्योन्यं शरैर्घोरैश्छादयानौ महारथौ}
{रथवर्यगतौ वीरौ शुशुभाते बलोत्कटौ}


\twolineshloka
{आदित्याविव सन्दीप्तौ लोकक्षयकरावुभौ}
{स्वरश्मिभिरिवान्योन्यं तापयन्तौ शरोत्तमैः}


\twolineshloka
{ततः प्रविकृते यत्नं कुर्वाणौ तौ महारणे}
{कृतप्रतिकृते यत्तौ शरसङ्घैरभीतवत्}


\twolineshloka
{व्याघ्राविव च सङ्ग्रामे चेरतुस्तौ नरोत्तमौ}
{शरदंष्ट्रौ दुराधर्षौ चापवक्रौ भयङ्करौ}


\twolineshloka
{अभूतां तावदृश्यौ च शरजालैः समन्ततः}
{मेघजालैरिव च्छनौ गगने चन्द्रभास्करौ}


\twolineshloka
{चकाशेते मुहूर्तेन ततस्तावप्यरिन्दमौ}
{विमुक्तावभ्रजालेन अङ्गारकबुधाविव}


\twolineshloka
{अथ तत्रैव सङ्ग्रमे वर्तमाने सुदारुणे}
{अपसव्यं ततश्चक्रे द्रौणिस्तत्र वृकोदरम्}


\twolineshloka
{किरञ्छरशतैरुग्रैर्धाराभिरिव पर्वतम्}
{न तु तन्ममृषे भीमः शत्रोर्विजलक्षणम्}


\twolineshloka
{प्रतिचक्रे ततो राजन्पाण्डवोऽप्यपसव्यतः}
{मण्डलानां विभागेन गतप्रत्यागतेन च}


\twolineshloka
{बभूव तुमुलं युद्धं तयोः पुरुषसिंहयोः}
{चरित्वा विविधान्मार्गान्मण्डलस्थानमेव च}


\twolineshloka
{शरैः पूर्णायतोत्सृष्टैरन्योन्यमभिजघ्नतुः}
{अन्योन्यस्य वधे चैव चक्रतुर्यत्नमुत्तमम्}


\twolineshloka
{ईषतुर्विरथं चैव कर्तुमन्योन्यमाहवे}
{ततो द्रौणिर्महास्त्राणि प्रादुश्चक्रे महारथः}


\twolineshloka
{तान्यस्त्रैरेव समरे प्रतिजघ्नेऽथ पाण्डवः}
{ततो घोरं महाराज अस्त्रयुद्धमवर्तत}


\twolineshloka
{ग्रहयुद्वं यथा घोरं प्रजासंहरणे ह्यभूत्}
{ते वाणाः समसज्जन्त मुक्तास्ताभ्यां तु भारत}


\twolineshloka
{द्योतग्रन्तो दिशः सर्वास्तव सैन्यं समन्ततः}
{बाणसङ्घैर्वृतं घोरमाकाशं समपद्यत}


\twolineshloka
{उल्कापातावृतं युद्धं प्रजानां संक्षये नृप}
{बाणाभिघातात्सञ्जज्ञे तत्र भारत पावकः}


\twolineshloka
{सविस्फुलिङ्गो दीप्तार्चिर्योऽदहद्वाहिनीद्वयम्}
{तत्र सिद्धा महाराज सम्पतन्तोऽब्रुवन्वचः}


\twolineshloka
{युद्धानामति सर्वेषां युद्धमेतदिति प्रभो}
{सर्वयुद्धानि चैतस्य कलां नार्हन्ति षोडशीम्}


\twolineshloka
{नेदृशं च पुनर्युद्धं भविष्यति कदाचन}
{अहो ज्ञानेन सम्पन्नावुभ ब्राह्मणक्षत्रियौ}


\twolineshloka
{अहो शौर्येण सम्पन्नावुभौ चोग्रपराक्रमौ}
{अहो भीमबलो भीम एतस्य च कृतास्त्रता}


\twolineshloka
{अहो वीर्यस्य सारत्वमहो सौष्ठवमेतयोः}
{स्थितावेतौ हि समरे कालान्तकयमोपमौ}


\twolineshloka
{रुद्रौ द्वावि सम्भूतौ यथा द्वाविव भास्करौ}
{यमौ वा पुरुषव्याघ्रौ घोररूपावुभौ रणे}


\twolineshloka
{इति वाचः स्म श्रूयन्ते सिद्धानां वै मुहुर्मुहुः}
{सिंहनादश्च सञ्जज्ञे समेतानां दिवौकसाम्}


\twolineshloka
{अद्भुतं चाप्यचिन्त्यं च दृष्ट्वा कर्म तयो रणे}
{सिद्वचारणसङ्घानां विस्मयः समपद्यत}


\twolineshloka
{प्रशंसन्ति तदा देवाः सिद्वाश्च परमर्षयः}
{साधु द्रौणे महाबाहो साधु भीमेति चाब्रुवन्}


\twolineshloka
{तौ शूरौ समरे राजन्परस्परकृतागसौ}
{परस्परमुदीक्षेतां क्रोधादुद्वृत्य चक्षुषी}


\twolineshloka
{क्रोधरक्तेक्षणौ तौ तु क्रोधात्प्रस्फुरिताधरौ}
{क्रोधात्संदष्टदशनौ तथैव दशनच्छदौ}


\twolineshloka
{अन्योन्यं छादयन्तौ स्म शरवृष्ट्या महारथौ}
{शराम्बुधारो समरे शस्त्रविद्युत्प्रकाशिनौ}


\twolineshloka
{तावन्यन्यं ध्वजं विद्व्वा सारथिं च महारणे}
{अन्योन्यस्य हयान्विद्वा बिभिदाते परस्परम्}


\twolineshloka
{ततः क्रुद्धौ महाराज बाणौ गृह्य महाहवे}
{उभौ चिक्षिपतुस्तूर्ममन्योन्यस्य वधैषिणौ}


\twolineshloka
{तौ सायकौ महाराज द्योतमानौ चमूमुखे}
{आजघ्नतुः समासाद्य वज्रवेगौ दुरासदौ}


\twolineshloka
{तौ परस्परवेगाच्च शराभ्यां च भृशाहतौ}
{निपेततुर्महावीर्यौ रथोपस्थे तयोस्तदा}


\twolineshloka
{ततस्तु सारथिर्ज्ञात्वा द्रोणपुत्रमचेतनम्}
{अपोवाह रणाद्राजन्सर्वसैन्यस्य पश्यतः}


\threelineshloka
{तथैव पाण्डवं राजन्विह्वलन्तं मुहुर्मुहुः}
{अपोवाह रथेनाजौ विशोकः शत्रुतापनम् ॥ ॥ *इति श्रीमन्महाभारतेकर्णपर्वणि षोडशदिवसयुद्धे द्वादशोऽध्यायः ॥ 12 ॥ *.झ. पन्तकेएतदनन्तरं स्थिताः सप्ताध्याया एतत्पाठे क्रमेण 52-53-54-55-56-59-60तमाध्यायतया वर्तन्ते}
{}


\chapter{अध्यायः १७}
\twolineshloka
{`सञ्जय उवाच}
{}


\twolineshloka
{श्रुतकीर्तिमथायान्तं किरन्तं निशिताञ्शरान्}
{मद्रराजो महाराज वारयामास हृष्टवत्}


\twolineshloka
{मद्रराजं समासाद्य श्रुतकीर्तिर्महारथः}
{विव्याध भल्लैर्विशत्या कार्तस्वरविभूषितैः}


\twolineshloka
{प्रतिविव्याध तं शल्यं त्रिभिस्तूर्णमजिह्मगैः}
{सारथिं चास्य भल्लेन भृशं विव्याध भारत}


\twolineshloka
{स शल्यं शरवर्षेण च्छादयामास संयुगे}
{मुमोच निशितान्बाणान्मद्रराजरथं प्रति}


\twolineshloka
{ततः शल्यो महाराज श्रुतकीर्तिभुजच्युतान्}
{चिच्छेद समरे बाणासन्बणैः सन्नतपर्वभिः}


\twolineshloka
{श्रुतकीर्तिस्ततः श्ल्यं भित्त्वा नवभिरायसैः}
{सारथिं त्रिभिरानर्च्छत्पुनः शल्यं च प़ञ्चभिः}


\twolineshloka
{तस्य शल्यो धनुश्छित्त्वा हस्तावापं निकृत्य च}
{विव्याध समरे तूर्णं सप्तभिस्तं शरोत्तमैः}


\twolineshloka
{अथान्यद्धनुरादाय श्रुतकीर्तिर्महारथः}
{भद्रेश्वरं चतुःषष्ट्या बाह्वोरुरसि चार्पयत्}


\twolineshloka
{ततस्तु समरे राजंस्तेन विद्वः शिलीमुखैः}
{xxविव्याध तं चापि नवत्या निशितैः शरैः}


\twolineshloka
{तस्य मद्रेश्वरश्चापं पुनश्चिच्छेद मारिष}
{स च्छिन्नधन्वा समरे गदां चिक्षेप सत्वरः}


\twolineshloka
{पट्टैर्जाम्बूनदैर्बद्धां रुप्यपट्टैश्च भारत}
{भ्राजमानां यथा नारीं दिव्यवस्त्रविभूषिताम्}


\twolineshloka
{तामापतन्तीं सहसा दीप्यमानाशनिप्रभाम्}
{शरैरनेकसाहस्रैर्व्यष्टम्भयत मद्रराट्}


\twolineshloka
{विष्टभ्य च गदां वीरः पातयित्वा च भूतले}
{श्रुकीर्तिमथायत्तो राजन्विव्याध पञ्चभिः}


\twolineshloka
{तस्य शक्तिं रमे भूयश्चिक्षेप भुजोपमाम्}
{तां द्विधा चाछिनच्छल्यो मेदिन्यां सा त्वशीर्यत}


\twolineshloka
{तस्य शल्यः क्षुरप्रेण यन्तुः कायाच्छिरोऽहरत्}
{बालहस्ताद्यथा श्येन आमिषं वै नरोत्तम}


\twolineshloka
{स पपात रथोपस्थात्सारथिस्तस्य भारत}
{ततस्ते प्राद्रवन्सङ्ख्ये हय्नास्तस्य महात्मनः}


\twolineshloka
{पलायमानैस्तैरश्वैः सोपनीतो रमाजिरात्}
{श्रुतकीर्तिर्महाराज पश्यतां सर्वयोधिनाम्}


\twolineshloka
{ततो मद्रेश्वरो राजा पाण्डवानामनीकिनीम्}
{व्यगाहत मुदा युक्तो नलिनीं द्विरदो यथा}


\twolineshloka
{लोलयामास स बलं सिंहः पशुगणानिव}
{शल्यस्तत्र महारङ्गे पाण्डवानां महात्मनाम्}


\twolineshloka
{निपात्य पाण्डुपाञ्चालान्पृतनासु व्यवस्थितः}
{अशोभत रणे शल्यो विधूमोऽग्निरिव ज्वलन्}


\twolineshloka
{सेनाकक्षं महद्दग्ध्वा कक्षमग्निरिवोत्थितः}
{स्थितो रराज समरे पुरं दग्ध्वेव शङ्करः}


\chapter{अध्यायः १८}
\twolineshloka
{सञ्जय उवाच}
{}


\twolineshloka
{सहदेवं तथा क्रुद्धं निघ्नन्तं तव वाहिनीम्}
{दुःशासनो महाराज भ्राता भ्रातरमभ्ययात्}


\twolineshloka
{तौ समेतौ महाराज दृष्ट्वा तत्र महारथाः}
{सिंहनादरवांश्चक्रुर्वासांस्यादुधुवुश्च ह}


\twolineshloka
{ततो भारत रुष्टेन तव पुत्रेण धन्विना}
{पाण्डुपुत्रस्त्रिभिर्बाणैर्वक्षस्यभिहतो बली}


\twolineshloka
{सहदेवस्ततो राजन्नाराचेन तवात्मजम्}
{विद्व्वा विव्याध सप्तत्या सारथिं च त्रिभिः शरैः}


\twolineshloka
{दुःशासनस्ततश्चापं छित्त्वा राजन्महाहवे}
{सहदेवं त्रिसप्तत्या बाहोरुरसि चार्पयत्}


\twolineshloka
{सहदेवस्तु सङ्क्रुद्धः खङ्गं गृह्य महाहवे}
{आविध्य प्रासृजत्तूर्णं तव पुत्रथं प्रति}


\twolineshloka
{समार्गणगुणं चापं छित्त्वा तस्य महानसिः}
{निपपात ततो भूमौ च्युतः सर्प इवाम्बरात्}


\twolineshloka
{अथान्यद्धनुरादाय सहदेवः प्रतापवान्}
{दुःशासनाय चिक्षेप बाणमन्तकरं ततः}


\twolineshloka
{तमापतन्तं विशिखं यमदण्डोपमत्विषम्}
{खङ्गेन शितधारेण द्विधा चिच्छेद कौरवः}


\twolineshloka
{ततस्तं निशितं खङ्गमाविध्य युधि सत्वरः}
{धनुश्चान्यत्समादाय शरं जग्राह वीर्यवान्}


\twolineshloka
{तमापतन्तं सहसा निस्त्रिंशं निशितैः शरैः}
{पातयामास समरे सहदेवो हसन्निव}


\twolineshloka
{ततो बाणांश्चतुः षष्टिं तव पुत्रो महारणे}
{सहदेवरथं तूर्णं प्रेषयामास भारत}


\twolineshloka
{ताच्छरान्समरे राजन्वेगेनापपतो बहून्}
{एकैकं पञ्चभिर्बाणैः सहदेवो न्यकृतन्तत}


\twolineshloka
{सन्निवार्य महाबाणान्सहदेवः प्रतापवान्}
{अथास्मै सुबहून्बाणान्प्रेषयामास संयुगे}


\twolineshloka
{तान्वाणांस्तव पुत्रोऽपि छित्त्वैकैकं त्रिभिः शरैः}
{ननाद सुमहानादं नादयानो वसुंधराम्}


\twolineshloka
{ततो दुःशासनो राजन्द्वाभ्यां पाण्डुसुतं रणे}
{सारथिं नवभिर्बाणैर्माद्रोयस्य समाचिनोत्}


\twolineshloka
{ततः क्रुद्धो महाराज सदहेवः प्रतापवान्}
{समाधत्त शरं घोरं मृत्युकालान्तकोपमम्}


\twolineshloka
{विकृष्य बलवच्चापं तव पुत्राय सोऽसृजत्}
{स तं निर्भिद्य वेगेन भित्त्वा च कवचं महत्}


\twolineshloka
{प्राविशद्वरणीं राजन्वल्मीकमिव पन्नगः}
{ततः सम्मुमुहे राजंस्तव पुत्रो महारथः}


\twolineshloka
{मूढं चैनं समालोक्य सारथिस्त्वरितो रथम्}
{अपोवाह भृशं त्रस्तो वध्यमानं शितैः शरैः}


\twolineshloka
{पराजित्य रणे तं तु कौरव्यं पाण्डुनन्दनः}
{दुर्योधनबलं दृष्ट्वा प्रममाथ समन्ततः}


\twolineshloka
{पिपीलिकपुटं राजन्यथा मृद्गन्नरो रुषा}
{तथा सा कौरवी सेना मृदिता तेन भारत}


\chapter{अध्यायः १९}
\twolineshloka
{सञ्जय उवाच}
{}


\twolineshloka
{नकुलं रभसं युद्धे द्रावयन्तं वरूथिनीम्}
{कर्णो वैकर्तनो राजन्वारयामास वै रुषा}


\twolineshloka
{नकुलस्तु ततः कर्णं प्रहसन्निदमब्रवीत् ॥चिरस्य बत दृष्टोऽहं दैवतैः सौम्यचक्षुषा}
{}


\twolineshloka
{यस्य मे त्वं रणे पाप चक्षुर्विषयमागतः ॥त्वं हि मूलमनर्थानां वैरस्य कलहस्य च}
{}


\twolineshloka
{त्वद्दोषात्कुरवः क्षीणाः समासाद्य परस्परम्}
{त्वामद्य समरे हत्वा कृतकृत्योऽस्मि विज्वरः}


\twolineshloka
{एवमुक्तः प्रत्युवाच नकुलं सूतनन्दनः}
{सदृशं राजपुत्रस्य धन्विनश्च विशेषतः}


\twolineshloka
{प्रहरस्व च मे वीर पश्यामस्तव पौरुषम्}
{कर्म कृत्वा रमे शूर ततः कत्थितुमर्हसि}


\twolineshloka
{अनुक्त्वा समरे तात शूरा युध्यन्ति शक्तितः}
{प्रयुध्यस्व मया शक्त्या हनिष्ये दर्पमेव ते}


\twolineshloka
{इत्युक्त्वा प्राहरत्तूर्णं पाण्डुपुत्राय सूतजः}
{विव्याध चैनं समरे त्रिसप्तत्या शिलीमुखैः}


\twolineshloka
{नकुलस्तु ततो विद्वः सूतपुत्रेण भारत}
{अशीत्याऽऽशीविषप्रख्यैः सूतपुत्रमविध्यत}


\twolineshloka
{तस्य कर्णो धनुश्छित्त्वा स्वर्णपुङ्खैः शिलाशितैः}
{त्रिंशता परमेष्वासः शरैः पाण्डवमर्दयत्}


\twolineshloka
{ते तस्य कवचं भित्त्वा पपुः शोणितमाहवे}
{आशीविषा यथा नागा भित्त्वा गां सलिलं पपुः}


\twolineshloka
{अथान्यद्वनुरादाय हेमपृष्ठं दुरासदम्}
{कर्णं विव्याध सप्तत्या सारथिञ्च त्रिभिः शरैः}


\twolineshloka
{ततः क्रुद्धो महाराज नकुलः परवीहा}
{क्षुरप्रेण सुतीक्ष्णेन कर्णस्य धनुराच्छिनत्}


\twolineshloka
{अथैनं छिन्नधन्वानं सायकानां शतैस्त्रिभिः}
{आजघ्ने प्रहसन्वीरः सर्वलोकमहारथम्}


\twolineshloka
{कर्णमभ्यर्दितं दृष्ट्वा पाण्डुपुत्रेण मारिष}
{विस्मयं परमं जग्मुर्ऋषयः सह दैवतैः}


\twolineshloka
{अथान्यद्वनुरादाय कर्णो वैकर्तनस्तदा}
{नकुलं पञ्छभिर्बाणैर्जत्रुदेशे समार्पयत्}


\twolineshloka
{उरस्थैरथ तैर्बाणैर्माद्रीपुत्रो व्यरोचत}
{स्वरश्मिभिरिवादित्यो भुवने विसृजन्प्रभाम्}


\twolineshloka
{नकुलस्तु ततः कर्णं विद्व्वा सप्तभिराशुगैः}
{अथास्य धनुषः कोटिं पुनश्चिच्छेद मारिष}


\twolineshloka
{सोऽन्यत्कार्मुकमादाय समरे वेगवत्तरम्}
{नकुलस्य ततो बैणैः समन्ताच्छादयद्दिशः}


\twolineshloka
{सञ्छाद्यमानः सहसा कर्णचाप्युतैः सरैः}
{चिच्छेद स शरांस्तूर्णं शरैरेव महारथः}


\twolineshloka
{ततो बाणमयं जालं विततं व्योम्न्यदृश्यत}
{खद्योतानामिव व्रातैः सम्पतद्भिर्यथा नभः}


\twolineshloka
{तैर्विमुक्तैः शरशतैश्छादितं गगनं तदा}
{शलभानां यथा व्रातैस्तद्वदासीद्विशाम्पते}


\twolineshloka
{ते शरा हेमविकृताः सम्पन्तौ मुहुर्मुहुः}
{श्रेणीकृता व्यकाशान्त क्रौञ्चाः श्रेणीकृता इव}


\twolineshloka
{बाणजालवृते व्योम्नि च्छादिते च दिवाकरे}
{न स्म सम्पतते भूतं किञ्चिदप्यन्तरिक्षगम्}


\twolineshloka
{निरुद्वे तत्र मार्गे च शरसङ्घैः समन्ततः}
{व्यरोचेतां महात्मानौ कालसूर्याविवोदितौ}


\twolineshloka
{कर्णचापच्युतैर्बाणैर्वध्यमानास्तु सोमकाः}
{अवालीयन्त राजेन्द्र वेदनार्ता भूशार्दिताः}


\twolineshloka
{नकुलस्य तथा बाणैर्हन्यमाना चमूस्तव}
{व्यशीर्यत दिशो राजन्वातनुन्ना इवाम्बुदाः}


\twolineshloka
{ते सेने हन्यमाने तु ताभ्यां दिव्यैर्महाशरैः}
{शरपातमपाक्रम्य तस्थतुः प्रेक्षिके तदा}


\twolineshloka
{प्रोत्सारितजने तस्मिन्कर्णपाण्डवयोः शरैः}
{अविध्येतां महात्मानावन्योन्यं शरवृष्टिभिः}


\twolineshloka
{विदर्शयन्तौ दिव्यानि शस्त्राणि रणमूर्धनि}
{छादयन्तौ च सहसा परस्परवधैषिणौ}


\twolineshloka
{नकुलेन शरा मुक्ता कङ्कबर्हिणवाससः}
{सूतपुत्रमवच्छाद्य व्यतिष्ठन्त यथाऽम्बरे}


\twolineshloka
{तथैव सूतपुत्रेण प्रेषिताः परमाहवे}
{पाण्डुपुत्रमवच्छाद्य व्यतिष्ठन्ताम्बरे शराः}


\twolineshloka
{शरवेश्मप्रविष्टौ तौ ददृशाते न कैश्चन}
{सूर्याचन्द्रमसौ राजञ्छाद्यमानौ घनैरिव}


\twolineshloka
{ततः क्रुद्धो रणे कर्णः कृत्वा घोरतरं वपुः}
{पाण्डवं छादयामास समन्ताच्छरवृष्टिभिः}


\twolineshloka
{सोऽतिच्छन्नो महाराज सूतपुत्रेण पाण्डवः}
{न चकार व्यथां राजन्भास्करो जलदैर्यथा}


\twolineshloka
{ततः प्रहस्याधिरथिः शरजालानि मारिष}
{प्रेषयामास समरे शतशोऽथ सहस्रशः}


% Check verse!
एकच्छायमभूत्सर्वं तस्य बाणैर्महात्मनः ॥अभ्रच्छायेव सञ्जज्ञे सम्पतद्भिः शरोत्तमैः
\twolineshloka
{ततः कर्णो महाराज धनुश्छित्त्वा महात्मनः}
{सारथिं पातयामास रथनीडाद्वसन्निव}


\twolineshloka
{ततोऽश्वांश्छतुरश्चास्य चतुर्भिर्निशितैः शरैः}
{यमस्य भवनं तूर्णं प्रेषयामास भारत}


\twolineshloka
{अथास्य तं रथं दिव्यं तिलशो व्यधमच्छरैः}
{पताकां चक्ररक्षांश्च गदां खङ्गं च मारिष}


\twolineshloka
{शतचन्द्रं च तच्चर्म सर्वापकरणानि च}
{`सुवर्णविकृतं तच्च धनुः शरमाहवे'}


\twolineshloka
{हताश्वो विरथश्चैव विवर्मा च विशाम्पते}
{अवतीर्य रथात्तूर्णं परिघं गृह्य धिष्ठितः}


\twolineshloka
{तमुद्यतं महारघोरं परिघं तस्य सूतजः}
{व्यहनत्सायकै राजन्सुतीक्ष्णैर्भारसाधनैः}


\twolineshloka
{व्यायुधं चैनमालक्ष्य शरैः सन्नतपर्वभिः}
{आर्पयद्बहुभिः कर्णो न चैनं समपीडयत्}


\twolineshloka
{स हन्यमानः समरे कृतास्त्रेण बलीयसा}
{प्राद्रवत्सहसा राजन्नकुलो व्याकुलेन्द्रियः}


\twolineshloka
{तमभिद्रुत्य राधेयः प्रहसन्वै पुनःपुनः}
{सज्यमस्य धनुः कण्ठे व्यवासृजत भारत}


\twolineshloka
{ततः स शुशुभे राजन्कण्ठासक्तमहाधनुः}
{परिवेषमनुप्राप्तो यथा स्याद्व्योम्नि चन्द्रमाः}


\twolineshloka
{यथैव चासितो मेघः शक्रचापेन शोभितः}
{`अशोभत महाराज पाण्डुपुत्रस्तथा रणे'}


\twolineshloka
{तमब्रवीत्ततः कर्णो व्यर्थं व्याहृतवानसि}
{वदेदानीं पुनर्हृष्टो वध्यमानः पुनःपुनः}


\twolineshloka
{मा योत्सीः कुरुभिः सार्धं बलवद्भिश्च पाण्डव}
{सदृशैस्तात युध्यस्व व्रीडां मा कुरु पाण्डव}


\twolineshloka
{गृहं वा गच्छ माद्रेय यत्र वा कृष्णफल्गुनौ}
{एवमुक्त्वा महाराज विससर्ज स तं तदा}


\twolineshloka
{वधप्राप्तं तु तं शूरो नाहनद्वर्मवित्तदा}
{स्मृत्वा कुन्त्या वचो राजंस्तत एनं व्यसर्जयत्}


\twolineshloka
{विसृष्टः पाण्डवो राजन्सूतपुत्रेण धन्विना}
{व्रीडन्निव जगामाथ युधिष्ठिररथं प्रति}


\twolineshloka
{आरुरोह रथं चापि सूतपुत्रप्रतापितः}
{निःश्वसन्दुःखसन्तप्तः कुम्भक्षिप्त इवोरगः}


\twolineshloka
{तं विजित्याथ कर्णोऽपि पाञ्चालांस्त्वरितो ययौ}
{रथेनातिपताकेन चन्द्रवर्णहयेन च}


\twolineshloka
{तत्राक्रन्दो महानासीत्पाण्डवानां विशाम्पते}
{दृष्ट्वा सेनापतिं यान्तं पाञ्चालानां रथव्रजान्}


\twolineshloka
{तत्राकरोन्महाराज कदनं सूतनन्दनः}
{मध्यं प्राप्ते दिनकरे चक्रवद्विचरन्प्रभुः}


\threelineshloka
{भग्नचक्रै रथैः कैश्चिच्छिन्नध्यवजपताकिभिः}
{हताश्वैर्हतसूतैश्च भग्नाक्षैश्चैव मारिष}
{हियमाणानपश्याम पाञ्चालानां रथव्रजान्}


\twolineshloka
{तत्रतत्र च सम्भ्रान्ता विचेरुरथ कुञ्जराः}
{दावाग्न्यभिपरीताङ्गा यथैव स्युर्महावने}


\threelineshloka
{भिन्नकुम्भाः सरुधिराश्छिन्नहस्ताश्च वारणाः}
{छिन्नगात्रावराश्चैव च्छिन्नवालधयोऽपरे}
{छिन्नाभ्राणीव सम्पेतुर्हन्यमाना महात्मना}


\twolineshloka
{अपरे त्रासिता नागा नाराचशरतोमरैः}
{तमेवाभिमुखं जग्मुः शलभा इव पावकम्}


\twolineshloka
{अपरे निष्टनन्तश्च व्यदृश्यन्त महाद्विपाः}
{क्षरन्तः शोणितं गात्रैर्नगा इव जलस्रवाः}


\twolineshloka
{उरश्छदैर्वियुक्तांश्च वालबन्धैश्च वाजिनः}
{राजतैश्च तथा कांस्यैः सौवर्णैश्चैव भूषणैः}


\twolineshloka
{हीनांश्चाभरणैश्चैव खलीनैश्च विवर्जितान्}
{चामरैश्च कुथाभिश्च तूणीरैः पतितैरपि}


\twolineshloka
{निहतैः सादिभिश्चैव शूरैराहवशोभितैः}
{अपश्याम रणे तत्र भ्राम्यमाणान्हयोत्तमान्}


\twolineshloka
{प्रासैः खङ्गैश्च रहितानृष्टिभिश्चापि भारत}
{हयसादीनपश्याम कञ्चुकोष्णीषधारिणः}


\twolineshloka
{निहतान्वध्यमानांश्च वेपमानांश्च भारत}
{नागाङ्गावयवैर्हीनांस्तत्रतत्रैव भारत}


\twolineshloka
{रथान्हेमपरिष्कारान्संयुक्ताञ्जवनैर्हयैः}
{भ्राम्यमाणानपश्याम हतेषु रथिषु द्रुतम्}


\twolineshloka
{भग्नाक्षकूबरान्कांश्चिद्भग्नचक्रांश्च भारत}
{विपताकध्वजांश्चान्याञ्छिन्नेषादण्डबन्धुरान्}


\twolineshloka
{विहीनान्रथिनस्तत्र धावमानांस्ततस्ततः}
{सूतपुत्रशरैस्तीक्ष्णैर्हन्यमानान्विशाम्पते}


\twolineshloka
{विशस्त्रांश्च तथैवान्यान्सशस्त्रांश्च हतान्बहून्}
{तारकाजालसञ्छन्नान्वरघण्टाविशोभितान्}


\twolineshloka
{नानावर्णविचित्राभिः पताकाभिरलङ्कृतान्}
{वारमआननुपश्याम धावमानान्समन्ततः}


\twolineshloka
{शिरांसि बाहूनूरूंश्च च्छिन्नानन्यांस्तथैव च}
{कर्णचापच्युतैर्बाणैरपश्याम समन्ततः}


\twolineshloka
{महान्व्यतिकरो रौद्रो योधानामन्वपद्यत}
{कर्णसायकनुन्नानां युध्यतां च शितैः शरैः}


\twolineshloka
{ते वध्यमानाः समरे सूतपुत्रेण सृञ्जयाः}
{तमेवाभिमुखं यान्ति पतङ्गा इव पावकम्}


\twolineshloka
{तं दहन्तमनीकानि तत्रतत्र महारथम्}
{क्षत्रिया वर्जयामासुर्युगान्ताग्निमिवोल्बणम्}


\threelineshloka
{हतशेषास्तु ये वीराः पाञ्चालानां महारथाः}
{तान्प्रभग्रान्दुतान्वीरः पृष्ठो विकिऱञ्छरैः}
{अभ्यधावत तेजस्वी विशीर्णवचनध्वजान्}


\twolineshloka
{तापयामास तान्बाणैः सूतपुत्रो महाबलः}
{मध्यन्दिनमनुप्राप्तो भूतानीव तमोनुदः}


\chapter{अध्यायः २०}
\twolineshloka
{सञ्जय उवाच}
{}


\twolineshloka
{युयुत्सं तव पुत्रस्य द्रावयन्तं बलं महत्}
{उलूको न्यपतत्तूर्णं तिष्ठतिष्ठेति चाब्रवीत्}


\twolineshloka
{युयुत्सुश्च ततो राजञ्शितधारेण पत्रिणा}
{उलूकं ताडयामास वज्रेणेव महाबलम्}


\twolineshloka
{उलूकस्तु ततः क्रुद्धस्तव पुत्रस्य संयुगे}
{क्षुरप्रेण धनुश्छित्त्वा ताडयामास कर्णिना}


\twolineshloka
{तदपास्य धनुश्छिन्नं युयुत्सर्वेगवत्तरम्}
{अन्यदादत्त सुमहच्चापं संरक्तलोचनः}


\twolineshloka
{शाकुनिं तु ततः षष्ट्या विव्याध भरतर्षभ}
{सारथिं त्रिभिरानर्च्छत्तं च भूयो व्यविध्यत}


\twolineshloka
{उलूकस्तं तु विंशत्या विद्व्वा स्वर्णविभूषितैः}
{अथास्य समरे क्रुद्धो ध्वजं चिच्छेद काञ्चनम्}


\twolineshloka
{स च्छिन्नयष्टिः सुमहाञ्शीर्यमाणो महाध्वजः}
{पपात प्रमुखे राजन्युयुत्सोः काञ्चनध्वजः}


\twolineshloka
{ध्वजमुन्मथितं दृष्ट्वा युयुत्सुः क्रोधमूर्च्छितः}
{उलूकं पञ्चभिर्बाणैराजघान स्तनान्तरे}


\twolineshloka
{उलूकस्तस्य समरे तैलधौतेन मारिष}
{शिरश्चिच्छेद भल्लेन यन्तुर्भरतसत्तम}


\twolineshloka
{तच्छिन्नमपतद्भूमौ युयुत्सोः सारथेस्तदा}
{तारारूपं यथा चित्रं निपपात महीतले}


\twolineshloka
{जघान चतुरोऽश्वांश्च तं च विव्याध पञ्चभिः}
{सोऽतिविद्वो बलवता प्रत्यपायाद्रथान्तरम्}


\twolineshloka
{तं निर्जित्य रणे राजन्नुलूकस्त्वरितो ययौ}
{पाञ्चालान्सृञ्जयांश्चैव विनिघ्नन्निशितैः शरैः}


\twolineshloka
{शतानीकं महाराज श्रुतकर्मा सुतस्तव}
{व्यश्वसूतरथं चक्रे निमेषार्धादसम्भ्रमः}


\twolineshloka
{हताश्वे तु रथे तिष्ठञ्शतानीको महारथः}
{गदां चिक्षेप सङ्क्रुद्धस्तव पुत्रस्य मारिष}


\twolineshloka
{सा कृत्वा स्यन्दनं भस्म हयांश्चैव ससारथीन्}
{पपात धरणीं तूर्णं दारयन्तीव भारत}


\twolineshloka
{तावुभौ विरथौ वीरौ कुरूणां कीर्तिवर्धनौ}
{व्यपाक्रमेतां युद्वात्तु प्रेक्षमाणौ परस्परम्}


\twolineshloka
{पुत्रस्तु तव सम्भ्रान्तो विविंशो रथमारुहत्}
{शतानीकोऽपि त्वरितः प्रतिविन्ध्यरथं गतः}


\twolineshloka
{सुतमोमं तु शकुनिर्विद्व्वा तु निशितैः शरैः}
{नाकम्पयत सङ्क्रुद्वो वार्योघ इव पर्वतम्}


\twolineshloka
{सुतसोमस्तु तं दृष्ट्वा पितुरत्यन्तवैरिणम्}
{शरैरनेकसाहस्रैश्छादयामास भारत}


\twolineshloka
{ताञ्शराञ्शकुनिस्तूर्णं चिच्छेदान्यैः पतत्रिभिः}
{लघ्वस्त्रश्चित्रयोधी च जितकाशी च संयुगे}


\twolineshloka
{निवार्य समरे चापि शरांस्तान्निशितैः शरैः}
{आजघान सुसङ्क्रुद्धः सुतसोमं त्रिभिः शरैः}


\threelineshloka
{`नाकम्पयत सङ्क्रुद्धो वार्योघ इव पर्वतम्'}
{तस्याश्वान्केतनं सूतं तिलशो व्यधमच्छरैः}
{श्यालस्तव महाराज तत उच्चुक्रुशुर्जनाः}


\twolineshloka
{हताश्वो विरथश्चैव छिन्नकेतुश्च मारिष}
{धन्वी धनुर्वरं गृह्य रथाद्भूमावतिष्ठत}


\threelineshloka
{व्यसृजत्सायकांश्चैव स्वर्णपुङ्खाञ्शिलाशितान्}
{छादयामासुरथ ते तव श्यालस्य तं रथम्}
{शळभानामिव व्राताः शरव्राता महारथम्}


\twolineshloka
{सञ्छाद्यमानोऽपि दृढं विव्यथे नैव सौबलः}
{प्रममाथ शरांस्तस्य शरव्रातैर्महायशाः}


\threelineshloka
{तत्रातुष्यन्त योधाश्च सिद्वाश्चापि दिवि स्थिताः}
{सुतसोमस्य तत्कर्म दृष्ट्वाऽश्रद्वेयमद्भुतम्}
{रथस्थं शकुनिं यत्तु पदातिः समयोधयत्}


\twolineshloka
{तस्य तीक्ष्णैर्महावेगैर्भल्लैः सन्नतपर्वभिः}
{व्यहनत्कारामुकं राजंस्तूणीरांश्चैव सर्वशः}


\twolineshloka
{स च्छिन्नधन्वा विरथः खङ्गमुद्यम्य चानदत्}
{वैदूर्योत्पलवर्णाभं दन्तिदन्तमयत्सरुम्}


\twolineshloka
{भ्राम्यमाणं ततस्तं तु विमलाम्बरवर्चसम्}
{कालदण्डोपमं मेने सुतसोमस्य धीमतः}


\twolineshloka
{सोऽचरत्सहसा खङ्गी मण्डलानि सहस्रशः}
{चतुर्दश महाराज शिक्षाबलसमन्वितः}


\twolineshloka
{भ्रान्तमुद्धान्तमाविद्धमाप्लुतं विप्लुतं सृतम्}
{सम्पातसमुदीर्णे च दर्शयामास संयुगे}


\twolineshloka
{सौबलस्तु ततस्तस्य शरांश्चिक्षेप वीर्यवान्}
{तानापतत एवाशु चिच्छेद परमासिना}


\twolineshloka
{ततः क्रुद्धो महाराज सौबलः परवीरहा}
{प्राहिणोत्सुतसोमाय शरानाशीविषोपमान्}


\twolineshloka
{चिच्छेद तांस्तु खङ्गेन शिक्षया च बलेन च}
{दर्शयँल्लाघवं युद्धे तार्क्ष्यतुल्यपराक्रमः}


\twolineshloka
{तस्य सञ्चरतो राजन्मण्डलावर्तने तदा}
{क्षुरप्रेण सुतीक्ष्णेन खङ्गं चिच्छेद सुप्रभम्}


\twolineshloka
{स च्छिन्नः सहसा भूमौ निपपात महानसिः}
{अधर्मस्य स्थितं हस्ते सुत्सरोस्तत्र भारत}


\twolineshloka
{छिन्नमाज्ञाय निस्त्रिंशमवप्लुत्य पदानि षट्}
{प्राविध्यत ततः शेषं सुतसोमो महारथः}


\twolineshloka
{तच्छित्त्वा सगुणं चापं रणे तस्य महात्मनः}
{पपात धरणीं तूर्णं स्वर्णवज्रविभूषितम्}


\twolineshloka
{सुतसोमस्ततोऽगच्छच्छ्रुतकीर्तेर्महारथम्}
{सौबलोऽपि धनुर्गृह्य घोरमन्यत्सुदुर्जयम्}


\twolineshloka
{अभ्ययात्पाण्डवानीकं निघ्नञ्शत्रुगणान्बहून्}
{तत्र नादो महानासीत्पाण्डवानां विशाम्पते}


\threelineshloka
{सौबलं समरे दृष्ट्वा विचरन्तमभीतवत्}
{तान्यनीकानि दृप्तानि शस्त्रवन्ति महान्ति च}
{द्राव्यमाणान्यदृश्यन्त सौबलेन महात्मना}


\twolineshloka
{यथा दैत्यचमूं राजन्देवराजो ममर्द ह}
{तथैव पाण़्वीं सेनां सौबलेयो व्यनाशयत्}


\chapter{अध्यायः २१}
\twolineshloka
{सञ्जय उवाच}
{}


\twolineshloka
{धृष्टद्युम्नं कृपो राजन्वारयामास संयुगे}
{यथा दृष्ट्वा वने सिंहं शरभो वारयेद्युधि}


\twolineshloka
{निरुद्वः पार्षतस्तेन गौतमेन बलीयसा}
{पदात्पदं विचलितुं नाशकत्तत्र भारत}


\twolineshloka
{गौतमस्य रथं दृष्ट्वा धृष्टद्युम्नरथं प्रति}
{वित्रेसुः सर्वभूतानि क्षयं प्राप्तं च मेनिरे}


\twolineshloka
{तदा जल्पन्ति पाण्डूनां रथिनः सादिनस्तथा}
{द्रोणस्य निधनान्नूनं सङ्क्रुद्वो द्विपदां वरः}


\twolineshloka
{शारद्वतो महातेजा दिव्यास्त्रविदुदारधीः}
{अपि स्वस्ति भवेदद्य धृष्टद्युम्नस्य गौतमात्}


\twolineshloka
{अपीयं वाहिनी कृत्स्ना मुच्येत महतो भयात्}
{अप्ययं ब्राह्मणः सर्वान्न नो हन्यात्समागतान्}


\twolineshloka
{यादृशं दृश्यते रूपमन्तकप्रतिमं रणे}
{गमिष्यत्यद्य पदवीं भारद्वाजस्य गौतमः}


\twolineshloka
{आचार्यः क्षिप्रहस्तश्च विजयी च सदा युधि}
{अस्त्रवान्वीर्यसम्पन्नः क्रोधेन च समन्वितः}


\threelineshloka
{पार्षतश्च महायुद्वे विमुखोऽद्याभिलक्ष्यते}
{इत्येवं विविधा वाचस्तावकानां परैः सह}
{व्यश्रूयन्त महाराज तयोस्तत्र समागमे}


\twolineshloka
{विनिःश्वस्य ततः क्रोधात्कृपः शारद्वतो नृप}
{पार्षतं चार्दयामास निश्चेष्टं सर्वमर्मसु}


\twolineshloka
{स हन्यमानः समरे गौतमेन महात्मना}
{कर्तव्यं न स्म जानाति मोहेन महता वृतः}


\twolineshloka
{तमब्रवीत्ततो यन्ता कञ्चित्क्षेमं तु पार्षत}
{ईदृशं व्यसनं युद्धे न ते दृष्टं मया क्वचित्}


\twolineshloka
{दैवयोगात्तु ते बाणा नापतन्मर्मभेदिनः}
{प्रेषिता द्विजमुख्येन मर्माण्युद्दिश्य सर्वतः}


\twolineshloka
{व्यावर्तये रथं तूर्णं नदीवेगमिवार्णवात्}
{अवध्यं ब्राह्मणं मन्ये येन ते विक्रमो हतः}


\twolineshloka
{धृष्टद्युम्नस्ततो राजञ्शनकैरब्रवीद्वचः}
{मुह्यते मे मनस्तात गात्रस्वेदश्च जायते}


\twolineshloka
{वेपथुश्च शरीरे मे रोमहर्षश्च सारथे}
{वर्जयन्ब्राह्मणं युद्धे शनैर्याहि यतोऽर्जुनः}


\twolineshloka
{अर्जुनं भीमसेनं वा समरे प्राप्य सारथे}
{क्षेममद्य भवेन्मह्यमिति मे नैष्ठिकी मतिः}


\twolineshloka
{ततः प्रायान्महाराज सारथिस्त्वरयन्हयान्}
{यतो भीमो महेष्वासो युयुधे तव सैनिकैः}


\twolineshloka
{प्रद्रुतं च रथं दृष्ट्वा धृष्टद्युम्नस्य मारिष}
{किरञ्शरशतान्येव गौतमोऽनुययौ तदा}


\twolineshloka
{शङ्खं च पूरयामास मुहुर्मुहुररिन्दमः}
{पार्षतं त्रासयामास महेन्द्रो नमुचिं यथा}


\twolineshloka
{शिखण्डिनं तु समरे भीष्ममृत्युं दुरासदम्}
{हार्दिक्यो वारयामास स्मयन्निव मुहुर्मुहुः}


\twolineshloka
{शिखण्डी तु समासाद्य हृदिकानां महारथम्}
{पञ्चभिर्निशितैर्भल्लैर्जत्रुदेशे समाहनत्}


\twolineshloka
{कृतवर्मा तु सङ्क्रद्धो भित्त्वा षष्ट्या पतत्रिभिः}
{धनुरेकेन चिच्छेद हसन्राजन्महारथः}


\twolineshloka
{अथान्यद्वनुरादाय द्रुपदस्यात्मजो बली}
{तिष्ठतिष्ठेति सङ्क्रुद्धो हार्दिक्यं प्रत्यभाषत}


\twolineshloka
{ततोऽस्य नवतिं बाणान्रुक्मपुङ्खान्सुतेजनान्}
{प्रेषयामास राजेन्द्र तेऽस्याभ्रश्यन्त वर्मणः}


\twolineshloka
{वितथांस्तान्समालक्ष्य पतितांश्च महीतले}
{क्षुरप्रेण सुतीक्ष्णेन कार्मुकं चिच्छिदे भृशम्}


\twolineshloka
{अथैनं छिन्नधन्वानं भग्नशृङ्गमिवर्षभम्}
{अशीत्या मार्गणैः क्रुद्धो बाह्वोरुरसि चार्पयत्}


\twolineshloka
{कृतवर्मा तु सङ्क्रुद्धो मार्गणैः क्षतविक्षतः}
{ववाम रुधिरं गात्रैः कुम्भवक्रादिवोदकम्}


\twolineshloka
{रुधिरेण परिक्लिन्नः कृतवर्मा त्वराजत}
{वर्षेण क्लेदितो राजन्यथा गैरिकपर्वतः}


\twolineshloka
{अथान्यद्वनुरादाय समार्गणगुणं प्रभुः}
{शिखण्डिनं बाणगणैः स्कन्धदेशे व्यताडयत्}


\twolineshloka
{स्कन्धदेशस्थितैर्बाणैः शिखण्डी तु व्यराजत}
{शाखाप्रशाखाविपुलः सुमहान्पादपो यथा}


\threelineshloka
{तावन्योन्यं भृशं विद्व्वा रुधिरेण समुक्षितौ}
{`पोप्लुयमानौ हि यथा महान्तौ शोणितहदे}
{तद्वद्विरेजतुर्वीरौ शोणितेन परिप्लुतौ}


\threelineshloka
{यथा च किंशुकौ फुल्लौ पुष्पमासे समागते}
{रुधिरोक्षितसर्वाङ्गौ रक्तचन्दनरूषितौ}
{भुजगाविव सङ्क्रुद्धौ रेजतुस्तौ नरोत्तमौ}


\twolineshloka
{तावुभौ शरवनुन्नाङ्गौ परस्परशरक्षतौ'}
{अन्योन्यशृङ्गाभिहतौ रेजतुर्वृषभाविव}


\twolineshloka
{अन्योन्यस्य वधे यत्नं कुर्वाणौ तौ महारथौ}
{रथाभ्यां चेरतुस्तत्र मण्डलानि सहस्रशः}


\twolineshloka
{कृतवर्मा महाराज पार्षतं निशितैः शरैः}
{रणे विव्याध सप्तत्या स्वर्णपुङ्गैः शिलाशितैः}


\twolineshloka
{ततोऽस्य समरे बाणं भोजः प्रहरतां वरः}
{जीवितान्तकरं घोरं व्यसृजत्त्वरयाऽन्वितः}


\twolineshloka
{स तेनाभिहतो राजन्मूर्च्छामाशु समाविशत्}
{ध्वजयष्टिं च सहसा शिश्रिये कश्मलावृतः}


\twolineshloka
{अपोवाह रणात्तूर्णं सारथी रथिनां वरम्}
{हार्दिक्यशरसन्तप्तं निःश्वसन्तं पुनःपुनः}


\twolineshloka
{पराजिते ततः शूरे द्रुपदस्यात्मजे प्रभो}
{व्यद्रवत्पाण्डवी सेना व्यधमाना समन्ततः}


\chapter{अध्यायः २२}
\twolineshloka
{सञ्जय उवाच}
{}


\twolineshloka
{श्वेताश्वोऽपि महाराज व्यधमत्तावकं बलम्}
{यथा वायुः समासाद्य तूलराशिं समन्ततः}


\twolineshloka
{प्रत्युद्ययुस्त्रिगर्तास्तं शिबयः कौरवैः सह}
{वसातयोऽथ साल्वाश्च गोपालाश्च यशस्विनः}


\twolineshloka
{सत्यदेवः सत्यकीर्तिर्मित्रदेवः श्रुतञ्जयः}
{सौश्रुतिश्चन्द्रदेवश्च मित्रवर्मा च भारत}


\twolineshloka
{त्रिगर्तराजः समरे भ्रातृभिः परिवारितः}
{पुत्रैश्चैव महेष्वासैर्नानाशस्त्रविशारदैः}


\twolineshloka
{व्यसृजन्त शस्त्रातान्किरन्तोऽर्जुनमाहवे}
{अभ्यवर्तन्त सहसा वार्योघा इव सागरम्}


\twolineshloka
{ते त्वर्जुनं समासाद्य योधाः शतसहस्रशः}
{अगच्छन्विलयं सर्वे तार्क्ष्यं दृष्ट्वेव पन्नगाः}


\twolineshloka
{ते हन्यमानाः समरे न जहुः पाण्डवं रणे}
{हन्यमाना महाराज शलभा इव पावकम्}


\twolineshloka
{सत्यदेवस्त्रिभिर्बाणैर्विव्याध युधि पाण्डवम्}
{मित्रदेवस्त्रिषष्ट्या तु चन्द्रदेवस्तु सप्तभिः}


\twolineshloka
{मित्रवर्मा त्रिसप्तत्या सौश्रुतिश्चापि सप्तभिः}
{श्रुतंजयस्तु विंशत्या सुशर्मा नवभिः शरैः}


\twolineshloka
{स विद्वो बहुभिः सङ्ख्ये प्रतिविव्याध तान्नृपान्}
{सौश्रुतिं सप्तभिर्विद्व्वा चन्द्रदेवं त्रिभिः शरैः}


\twolineshloka
{श्रुतंजयं च विंशत्या चन्द्रदेवं तथाऽष्टभिः}
{मित्रदेवं शतेनैव सत्यदेवं त्रिभिः शरैः}


\twolineshloka
{नवभिर्मित्रवर्माणं सुशर्माणं तथाऽष्टभिः}
{श्रुतञ्जयं च राजानं हत्वा तत्र शिलाशितैः}


\twolineshloka
{सौश्रुतेः सशिरस्त्राणं शिरः कायादपाहरत्}
{त्वरितश्चन्द्रदेवं च शरैर्निन्ये यमक्षयम्}


\twolineshloka
{तथेतरान्महाराज यतमानान्महारथान्}
{पञ्चभिः पञ्चभिर्बाणैरेकैकं प्रत्यवारयत्}


\twolineshloka
{सत्यदेवस्तु सङ्क्रुद्वस्तोमरं व्यसृजन्महत्}
{समुद्दिश्य रणे कृष्णं सिंहनादं ननाद च}


\twolineshloka
{स निर्भिद्य भुजं सव्यं माधवस्य महात्मनः}
{अयस्मयः स्वर्णदण्डो जगाम धरणीं तदा}


\twolineshloka
{माधवस्य तु विद्धस्य तोमरेण महारणे}
{प्रतोदः प्रापतद्वस्ताद्रश्मयश्च विशाम्पते}


\twolineshloka
{वासुदेवं विभिन्नाङ्गं दृष्ट्वा पार्थो धनञ्जयः}
{क्रोधमाहारयत्तीव्रं कृष्णं चेदमुवाच ह}


\twolineshloka
{प्रापयाश्वान्महाबाहो सत्यदेवं प्रति प्रभो}
{यावदेनं शरैस्तीक्ष्णैर्नयामि यमसादनम्}


\twolineshloka
{प्रतोदं गृह्य सोऽन्यत्तु रश्मीन्सङ्गृह्य च द्रुतम्}
{वाहयामास तानश्वान्सत्यदेवरथं प्रति}


\twolineshloka
{विष्वक्सेनं सुनिर्विद्वं दृष्ट्वा पार्थो धनञ्जयः}
{सत्यदेवं शरैस्तीक्ष्णैर्वारयित्वा महारथः}


\twolineshloka
{ततः सुनिशितैर्भल्लै राज्ञस्तस्य महच्छिरः}
{कुम्डलोपचितं कायाच्चकर्त पृतनान्तरे}


\twolineshloka
{तन्निकृत्य शितैर्बाणैर्भित्रवर्माणमाक्षिपत्}
{वत्सदन्तेन तीक्ष्णेन सारथिं चास्य मारिष}


\twolineshloka
{ततः शरशतैर्भूयः संशप्तकगणान्बली}
{पातयामास सङ्क्रुद्वः शतशोऽथ सहस्रशः}


\threelineshloka
{ततो रजतपुङ्खेन राजञ्शीर्षं महात्मनः}
{मित्रदेवस्य चिच्छेद क्षुरप्रेण महारथः}
{सुशर्माणं सुसङ्क्रुद्धो जत्रुदेशे समाहनत्}


\twolineshloka
{ततः संशप्तकाः सर्वे परिवार्य धनञ्जयम्}
{शस्त्रौघैर्ममृदुः क्रुद्वा नादयन्तो दिशो दश}


\twolineshloka
{अभ्यर्दितस्तु तज्जिष्णुः शक्रतुल्यपराक्रमः}
{ऐन्द्रमस्त्रममेयात्मा प्रादुश्चक्रे महारथः}


\twolineshloka
{ततः शरसहस्राणि प्रादुरासन्विशाम्पते}
{`कार्मुकात्पाण्डुपुत्रस्य पार्थस्यामिततेजसः'}


\twolineshloka
{ततस्तु च्छिद्यमानानां ध्वजानां धनुषां तथा}
{रथानां सपताकानां तूणीराणां युगैः सह}


\twolineshloka
{अक्षाणामथ चक्राणां योक्त्राणां रश्मिभिः सह}
{कूबराणां वरूथानां पृषत्कानां च संयुगे}


\twolineshloka
{अश्वानां पततां चापि प्रासानामृष्टिभिः सह}
{गदानां परिघानां च शक्तितोमरपट्टसैः}


\twolineshloka
{शतघ्नीनां सचक्राणां भुजानां चोरुभिः सह}
{कण्ठसूत्राङ्गदानां च केयूराणां च मारिष}


\threelineshloka
{हाराणामथ निष्काणां तनुत्राणां च भारत}
{छत्राणां व्यजनानां च शिरसां कुमुटैः सह}
{राशयश्चात्र दृश्यन्ते पतितानां महीतले}


\twolineshloka
{सकुण्डलानि स्वक्षीमि पूर्णचन्द्रनिभानि च}
{शिरांस्युर्व्यामदृश्यन्त ताराजालमिवाम्बरे}


\twolineshloka
{सुस्रग्वीणि सुवासांसि चन्दनेनोक्षितानि च}
{शरीराणि व्यदृश्यन्त निहतानां महीतले}


% Check verse!
रुद्रस्याक्रीडसदृशं घोरमायोधनं तदा
\twolineshloka
{निहतै राजपुत्रैश्च क्षत्रियैश्च महाबलैः}
{`अर्जुनेन महाराज तत्रतत्र महारणे'}


\twolineshloka
{हस्तिभिः पतितैश्चैव तुरङ्गैश्चाभवन्मही}
{अगम्यरूपा समरे विशीर्णैरिव पर्वतैः}


\twolineshloka
{नासीद्रथपथस्तस्य पाण्डवस्य महात्मनः}
{निघ्नतः शात्रवान्भल्लैर्हस्त्यश्वं चास्यतो महत्}


\twolineshloka
{स्वानुगा इव सीदन्ति रथचक्राणि मारिष}
{चरतस्तस्य सङ्ग्रामे तस्मिंल्लोहितकर्दमे}


\twolineshloka
{सीदमानानि चक्राणि समुहूस्तुरगा भृशम्}
{श्रमेण महता युक्ता मनोमारुतरंहसः}


\twolineshloka
{वध्यमानं तु तत्सैन्यं पाण्डुपुत्रेण धन्विना}
{प्रायशो विमुखं सर्वं नावतिष्ठत भारत}


\twolineshloka
{ताञ्जित्वा समरे जिष्णुः संशप्तकगणान्बहून्}
{विरराज तदा पार्थो विधूमोऽग्निरिव ज्वलन्}


\chapter{अध्यायः २३}
\twolineshloka
{सञ्जय उवाच}
{}


\twolineshloka
{युधिष्ठिरं महाराज विसृजन्तं शरान्बहून्}
{स्वयं दुर्योधनो राजा प्रत्यगृह्णादभीतवत्}


\twolineshloka
{तमापतन्तं सहसा तव पुत्रं महारथम्}
{धर्मराजो द्रुतं विद््ध्वा तिष्ठतिष्ठेति चाब्रवीत्}


\twolineshloka
{स तु तं प्रतिविव्याध नवभिर्निशितैः शरैः}
{सारथिं चास्य भल्लेन भृशं क्रुद्धोऽभ्यताडयत्}


\twolineshloka
{ततो युधिष्ठिरो राजन्स्वर्णपुङ्खाञ्शिलीमुखान्}
{दुर्योधनाय चिक्षेप त्रयोदश शिलाशितान्}


\twolineshloka
{चतुर्भिश्चतुरो वाहांस्तस्य हत्वा महारथः}
{पञ्चमेन शिरः कायात्सारथेश्च समाक्षिपत्}


\threelineshloka
{षष्ठेन तु ध्वजं राज्ञः सप्तमेन तु कार्मुकम्}
{अष्टमेन तथा खङ्गं पातयामास भूतले}
{पञ्चभिर्नृपतिं चापि धर्मराजोऽर्दयद्भृशम्}


\twolineshloka
{हताश्वात्तु रथात्तस्मादवप्लुत्य सुतस्तव}
{उत्तमं व्यसनं प्राप्तो भूमावेवावतिष्ठत}


\twolineshloka
{तं तु कृच्छ्रगतं दृष्ट्वा कर्णद्रौणिकृपादयः}
{अभ्यवर्तन्त सहसा परीप्सन्तो नराधिपम्}


\twolineshloka
{अथ पाण्डुसुताः सर्वे परिवार्य युधिष्ठिरम्}
{अन्वयुः समरे राजंस्ततो युद्धमवर्तत}


% Check verse!
ततस्तूर्यसहस्राणि प्रावाद्यन्त महामृधे
\twolineshloka
{क्ष्वेलाः किलकिलाशब्दाः प्रादुरासन्महीपते}
{यत्राभ्यगच्छन्समरे पाञ्चालाः कौरवैः सह}


\twolineshloka
{नरा नरैः समाजग्मुर्वारणा वरवारणैः}
{रथाश्च रथिभिः सार्धं हयाश्च हयसादिभिः}


\twolineshloka
{द्वन्द्वान्यासन्महाराज प्रेक्षणीयानि संयुगे}
{विविधान्यप्यचिन्त्यानि शस्त्रवन्त्युत्तमानि च}


\twolineshloka
{ते शूराः समरे सर्वे चित्रं लघु च सुष्ठु च}
{अयुध्यन्त महावेगाः परस्परवधैषिणः}


\twolineshloka
{अन्योन्यं समरे जघ्नुर्योधव्रतमनुष्ठिताः}
{न हि ते समरं चक्रुः पृष्ठतो वै कथञ्चन}


\twolineshloka
{मुहूर्तमेव तद्युद्वमासीन्मधुरदर्शनम्}
{तत उन्मत्तवद्राजन्निर्मर्यादमवर्तत}


\twolineshloka
{रथी नागं समासाद्य दारयन्निशितैः शरैः}
{प्रेषयामास कालाय शरैः सन्नतपर्वभिः}


\twolineshloka
{नागा हयान्समासाद्य विक्षिपन्तो बहून्रणे}
{दारयामासुरत्युग्रं तत्रतत्र तदातदा}


\twolineshloka
{हयारोहाश्च बहवः परिवार्य हयोत्तमान्}
{तलशब्दरवांश्चक्रुः सम्पतन्तस्ततस्ततः}


\twolineshloka
{धावमानांस्ततस्तांस्तु द्रवमाणान्महागजान्}
{पार्श्वतः पृष्ठतश्चैव निजघ्नुर्हयसादिनः}


\twolineshloka
{विद्राव्य च बहूनश्वान्नागा राजन्मदोत्कटाः}
{विषाणैश्चापरे जघ्नुर्ममृदुश्चापरे भृशम्}


\twolineshloka
{साश्वारोहांश्च तुरगान्विषामैर्विव्यधू रुषा}
{अपरे चिक्षिपुर्वेगात्प्रगृह्यातिबलास्तदा}


\twolineshloka
{पादातैराहता नागा विवरेषु समन्ततः}
{चक्रुरार्तस्वरं घोरं दुद्रुवुश्च दिशो दश}


\twolineshloka
{पदातीनां तु सहसा प्रद्रुतानां महाहवे}
{उत्सृज्याभरणं तूर्णमववव्रू रणाजिरे}


\twolineshloka
{निमित्तं मन्यमानास्तु परिणाम्य महागजाः}
{जगृहुर्बिभिदुश्चैव चित्राण्याभरणानि च}


\twolineshloka
{तांस्तु तत्र प्रसक्तान्वै परिन्वार्य पदातयः}
{हस्त्यारोहान्निजघ्नुस्ते महावेगा बलोत्कटाः}


\twolineshloka
{अपरे हस्तिभिर्हस्तैः खं विक्षिप्ता महाहवे}
{निपतन्तो विषाणाग्रैर्भृशं विद्वाः सुशिक्षितैः}


\twolineshloka
{अपरे सहसा गृह्य विषाणैरेव सूदिताः}
{सेनान्तरं समासाद्य केचित्तत्र महागजैः}


\twolineshloka
{क्षुण्णगात्रा महाराज विक्षिप्य च पुनःपुनः}
{अपरे व्यजनानीव विभ्राम्य निहता मृधे}


\twolineshloka
{पुरःसराश्च नागानामपरेषां विशाम्पते}
{शरीराण्यतिविद्वानि तत्रतत्र रणाजिरे}


\twolineshloka
{प्रतिमानेषु कुम्भेषु दन्तवेष्टेषु चापरे}
{निगृहीता भृशं नागाः प्रासतोमरशक्तिभिः}


\twolineshloka
{निगृह्य च गजाः केचित् पार्श्वस्थैर्भृशदारुणैः}
{रथाश्वसादिभिस्तत्र सम्भिन्ना न्यपतन्भुवि}


\twolineshloka
{सहयाः सादिनस्तत्र तोमरेण महामृधे}
{भूमावमृद्रन्वेगेन सचर्माणं पदातिनम्}


\threelineshloka
{चछा सावरणान्कांश्चित्तत्रतत्र विशाम्पते}
{रथान्नागाः समासाद्य परिगृह्य च मारिष}
{व्याक्षिपन्सहसा तत्र घोररूपे भयानके}


\twolineshloka
{नाराचैर्निहताश्चापि गजाः पेतुर्महाबलाः}
{पर्वतस्येव शिखरं वज्ररुग्णं महीतले}


\twolineshloka
{योधा योधान्समासाद्य मुष्टिभिर्व्यहनन्युधि}
{केशेष्वन्योन्यमाक्षिप्य चिक्षिपुर्बिभिदुश्च ह}


\twolineshloka
{उद्यम्य च भुजानन्ये निक्षिप्य च महीतले}
{पदा चोरः समाक्रम्य स्फुरतोऽपाहरच्छिरः}


\twolineshloka
{पततश्चापरो राजन्विजहारासिना शिरः}
{`मृतमन्यो महाराज पदा ताडितवांस्तदा'}


\threelineshloka
{जीवतश्च तथैवान्यः शस्त्रं काये न्यमज्जयत्}
{मुष्टियुद्धं महच्चासीद्योधानां तत्र भारत}
{तथा केशग्रहश्चोग्रो बाहुयुद्धं च भैरवम्}


\twolineshloka
{समासक्तस्य चान्येन अविज्ञातस्तथाऽपरः}
{जहार समरे प्राणान्नानाशस्तैरनेकधा}


\twolineshloka
{संसक्तेषु च योधेषु वर्तमाने च सङ्कुले}
{कबन्धान्युत्थितानि स्युः शतशोऽथ सहस्रशः}


\twolineshloka
{शोणितैः सिच्यमानानि शस्त्राणि कवचानि च}
{महारागानुरक्तानि वस्त्राणीव चकाशिरे}


\twolineshloka
{एवमेतन्महद्युद्वं दारुणं शस्त्रसङ्कुलम्}
{उन्मत्तगङ्गाप्रतिमं शब्देनापूरयज्जगत्}


\twolineshloka
{नैव स्वे न परे राजन्विज्ञायन्ते शरातुराः}
{योद्वव्यमिति युध्यन्ते राजानो जयगृद्विनः}


\twolineshloka
{स्वान्स्वे जघ्नुर्महराज परांश्चैव समागतान्}
{उभयोः सेनयोर्वीरैर्व्याकुलं समपद्यत}


\twolineshloka
{रथैर्भग्नैर्महाराज वारणैश्च निपातितैः}
{हयैश्च पतितैस्तत्र नरैश्च विनिपातितैः}


\twolineshloka
{अगम्यरूपा पृथिवी क्षणेन समपद्यत}
{क्षणेनासीन्महीपाल क्षतजौघप्रवर्तिनी}


\twolineshloka
{पाञ्चालानहतत्कर्णस्त्रिगर्तांश्च धनञ्जयः}
{भीमसेनः कुरून्राजन्हस्त्यनीकं च सर्वशः}


\twolineshloka
{एवमेष क्षयो वृत्तः कुरुपाण्डवसेनयोः}
{अपराह्णे गते सूर्ये काङ्क्षतां विपुलं यशः}


\chapter{अध्यायः २४}
\twolineshloka
{धृतराष्ट्र उवाच}
{}


\twolineshloka
{अतितीव्राणि दुःखानि दुःसहानि बहूनि च}
{त्वत्तोऽहं सञ्जयाश्रौषं पुत्राणां चैव सङ्क्षयम्}


\twolineshloka
{यथा त्वं मे कथयसे यथा युद्वमवर्तत}
{न सन्ति सूत कौरव्या इति मे निश्चिता मतिः}


\twolineshloka
{दुर्योधनश्च विरथः कृतस्तत्र महारथः}
{धर्मपुत्रः कथं चक्रे तस्य वा नृपतिः कथम्}


\threelineshloka
{अपराह्णे कथं युद्वमभवद्रोमहर्षणम्}
{तन्ममाचक्ष्व तत्त्वेन कुशलो ह्यसि सञ्जय ॥सञ्जय उवाच}
{}


\twolineshloka
{संसक्तेषु तु सैन्येषु वध्यमानेषु भागशः}
{रथमन्यं समास्थाय पुत्रस्तव विशाम्पते}


\threelineshloka
{क्रोधेन महता युक्तः सविषो भुजगो यथा}
{दुर्योधनः समालक्ष्य धर्मराजं युधिष्ठिरम्}
{प्रोवाच सूतं त्वरितो याहियाहीति भारत}


\twolineshloka
{तत्र मां प्रापय क्षिप्रं सारथे यत्र पाण्डवः}
{ध्रियमाणातप्रतेण राजा राजति दंशितः}


\twolineshloka
{स सूतश्चोदितो राज्ञा राज्ञः स्यन्दनमुत्तमम्}
{युधिष्ठिरस्याभिमुखं प्रेषयामास संयुगे}


\twolineshloka
{ततो युधिष्ठिरः क्रुद्वः प्रभिन्न इव कुञ्जरः}
{सारथिं चोदयामास याहि यत्र सुयोधनः}


% Check verse!
तौ समाजग्मतुर्वीरौ भ्रातरौ रथसत्तमौ
\twolineshloka
{समेत्य च महावीरौ संरब्धौ युद्वदुर्मदौ}
{ववर्षतुर्महेष्वासौ शरैरन्योन्यमाहवे}


\twolineshloka
{ततो दुर्योधनो राजा धर्मशीलस्य मारिष}
{शिलाशितेन भल्लेन धनुश्चिच्छेद संयुगे}


\twolineshloka
{तं नामृष्यत सङ्क्रुद्धो ह्यवमानं युधिष्ठिरः}
{अपविध्य धनुश्छिन्नं क्रोधसंरक्तलोचनः}


\twolineshloka
{अन्यत्कार्मुकमादाय धर्मपुत्रश्चमूमुखे}
{दुर्योधनस्य चिच्छेद ध्वजं कार्मुकमेव च}


\threelineshloka
{अथान्यद्वनुरादाय पुत्रस्ते भरतर्षभ}
{`युधिष्ठिरस्य चिक्षेप शरान्कनकभूषणान्}
{रुक्मपुङ्खान्प्रसन्नाग्रान्सविषानिव पन्नगान्'}


\twolineshloka
{तावन्योन्यं सुसङ्क्रुद्धौ शस्त्रवर्षाण्यमुञ्चताम्}
{सिंहाविव सुसंरब्धौ परस्परजिगीषया}


\twolineshloka
{जघ्नतुस्तौ रणेऽन्योन्यं नर्दमानौ वृषाविव}
{अन्तरं मार्गमाणौ च चेरतुस्तौ महारथौ}


\twolineshloka
{ततः पूर्णायतोत्सृष्टैः शरैस्तौ तु कृतव्रणौ}
{विरेजतुर्महाराज किंशुकाविव पुष्पितौ}


\twolineshloka
{ततो राजन्विमुञ्चन्तौ सिंहनादान्मुहुर्मुहुः}
{तलयोश्च तथा शब्दान्धनुषश्च महाहवे}


\twolineshloka
{शङ्खशब्दवरांश्चैव चक्रतुस्तौ नरेश्वरौ}
{अन्योन्यं तौ महाराज पीडयांचक्रतुर्भृशम्}


\twolineshloka
{ततो युधिष्ठिरो राजा पुत्रं तव शरैस्त्रिभिः}
{आजघानोरसि क्रुद्धो वज्रवेगैर्दुरासदैः}


\twolineshloka
{प्रतिविव्याध तं तूर्णं तव पुत्रो महीपतिः}
{पञ्चभिर्निशितैर्बाणैः स्वर्णपुङ्खैः शिलाशितैः}


\twolineshloka
{ततो दुर्योधनो राजा शक्तिं चिक्षेप भारत}
{सर्वपारशवीं तीक्ष्णां महोल्काप्रतिमां तदा}


\twolineshloka
{तामापतन्तीं सहसा धर्मराजः शितैः शरैः}
{त्रिभिश्चिच्छेद सहसा तं च विव्याध पञ्चभिः}


\twolineshloka
{निपपात ततः साऽथ स्वर्णदण्डा महास्वना}
{निपतन्ती महोल्केव व्यराजच्छिखिसन्निभा}


\twolineshloka
{शक्तिं विनिहतां दृष्ट्वा पुत्रस्तव विशाम्पते}
{नवभिर्निशितैर्भल्लैर्निजघान युधिष्ठिरम्}


\twolineshloka
{सोऽतिविद्वो बलवता शत्रुणआ शत्रुतापनः}
{दुर्योधनं समुद्दिश्य बाणं जग्राह सत्वरः}


\twolineshloka
{समाधत्त च तं बाणं धनुर्मध्ये महाबलः}
{चिक्षेप च महाराज ततः क्रुद्वः पराक्रमी}


\twolineshloka
{स तु बाणः समासाद्य तव पुत्रं महारथम्}
{व्यामोहयत राजानं धरणीं च ददार ह}


\twolineshloka
{ततो दुर्योधनः क्रुद्धो गदामुद्यम्य वेगितः}
{विधित्सुः कलहस्यान्तं धर्मराजमुपाद्रवत्}


\twolineshloka
{तमुद्यतगदं दृष्ट्वा दण्डहस्तमिवान्तकम्}
{धर्मराजो महाशक्तिं प्राहिणोत्तव सूनवे}


\twolineshloka
{दीप्यमानां महावेगां महोल्कां ज्वलितामिव}
{यमदण्डनिभां घोरां कालरात्रिमिवापराम्}


\twolineshloka
{रथस्थः स तया विद्वो वर्म भित्त्वा स्तनान्तरे}
{भृशं संविग्नहृदयः पपात च मुमोह च}


\twolineshloka
{नभस्तमाह च ततः प्रतिज्ञामनुपालय}
{नायं वध्यस्तव नृप इत्युक्तः स न्यवर्तत}


\twolineshloka
{ततस्त्वरितमागम्य कृतवर्मा तवात्मजम्}
{प्रत्यपद्यत राजानं निम्नं व्यसनार्णवे}


\twolineshloka
{गदामादाय भीमोऽपि हेमपट्टपरिष्कृताम्}
{अभिदुद्राव वेगेन कृतवर्माणमाहवे}


\twolineshloka
{एवं तदभवद्युद्धं त्वदीयानां परैः सह}
{अपराह्णे महाराज काङ्क्षतां विजयं युधि}


\chapter{अध्यायः २५}
\twolineshloka
{सञ्जय उवाच}
{}


\twolineshloka
{ततः कर्णं पुरस्कृत्य त्वदीया युद्वदुर्मदाः}
{पुनरावृत्य सङ्ग्रामं चक्रुर्देवासुरोपमम्}


\twolineshloka
{द्विरदनररथाश्वशङ्खशब्दैःपरिहृषिता विविधैश्च शस्त्रपातैः}
{द्विरदरथपदातिसादिसङ्घाःपरिकुपिताभिमुखाः प्रजघ्निरे ते}


\twolineshloka
{शितपरश्वथसासिपट्टसै--रिषुभिरनेकविधैश्च सूदिताः}
{द्विरदरथहया महाहवेवरपुरुषैः पुरुषाश्च वाहनैः}


\twolineshloka
{कमलदिनकरेन्दुसन्निभैःसितदशनैः सुमुखाक्षिनासिकैः}
{रुचिरमकुटकुण्डलैर्महीपुरुषशिरोभिरुपस्तृता बभौ}


\twolineshloka
{परिघमुसलशक्तितौमरै--र्नखरभुशुण्डिगदाशतैर्हताः}
{द्विरदनरहयाः सहस्रशोरुधिरनदीप्रवहास्तदाऽभवन्}


\twolineshloka
{प्रहतरथनराश्वकुञ्जरंप्रतिभयदर्शनमुल्बणव्रणम्}
{तदहितहतमाबभौ बलंपितृपतिराष्ट्रमिव प्रजाक्षये}


\twolineshloka
{अथ तव नरदेव सैनिका--स्तव च सुताः सुरसूनुसन्निभाः}
{अमितबलपुरः सरा रणेकुरुवृषभाः शिनिपुत्रमभ्ययुः}


\twolineshloka
{तदतिरुधिरभीममाबभौपुरुषवराश्वरथद्विपाकुलम्}
{लवणजलसमुद्वतस्वनंबलमसुरामरसैन्यसप्रभम्}


\twolineshloka
{सुरपतिसमविक्रमस्तत--स्त्रिदशवरावरजोपमं युधि}
{दिनकरकिरणप्रभैः पृषत्कैरवितनयोऽभ्यहनच्छिनिप्रवीरम्}


\twolineshloka
{तमपि सरथवाजिसारथिंशिनिवृषभो विविधैः शरैस्त्वरन्}
{भुजगविषसमप्रभै रणेपुरुषवरं समवास्तृणोत्तदा}


\twolineshloka
{शिनिवृषभशरैर्निपीडितंतव सुहृदो वसुषेणमभ्ययुः}
{त्वरितमतिरथा रथर्षभंद्विरदरथाश्वपदातिभिः सह}


\twolineshloka
{तदुदधिनिभमाद्रवद्बलंत्वरिततरैः समभिद्रुतं परैः}
{द्रुपदसुतमुखैस्तदाऽभवत्पुरुषरथाश्वगजक्षयो महान्}


\twolineshloka
{अथ पुरुषवरौ कृताह्निकौभवमभिपूज्य यथाविधि प्रभुम्}
{अरिवधकृतनिश्चयौ द्रुतंतव बलमर्जुनकेशवौ सृतौ}


\twolineshloka
{जलदनिनदनिःस्वनं रथंपवनविधूतपताककेतनम्}
{सितहयमुपयान्तमन्तिकंकृतमनसो ददृशुस्तदाऽरयः}


\twolineshloka
{अथ विष्फार्य गाण्डीवं रथे नृत्यन्निवार्जुनः}
{शरसम्बाधमकरोत्खं दिशः प्रदिशस्तथा}


\twolineshloka
{रथान्विमानप्रतिमान्मज्जयन्सायुधध्वजान्}
{ससारथींस्तदा बाणैरभ्राणीवानिलोऽवधीत्}


\twolineshloka
{गजान्गजप्रयन्तॄंश्च वैजयन्त्यायुधध्वजान्}
{सादिनोऽश्वांश्च पत्तींश्च शरैर्निन्ये यमक्षयम्}


\twolineshloka
{तमन्तकमिव क्रुद्धमनिवार्यं महारथम्}
{दुर्योधनोऽभ्ययादेको निघ्नन्बाणैरजिह्मगैः}


\twolineshloka
{तस्यार्जुनो धनुः सूतमश्वान्केतुं च सायकैः}
{हत्वा सप्तभिरेकेन च्छत्रं चिच्छेद पत्रिणा}


\twolineshloka
{नवमं च समाधाय व्यसृजत्प्राणघातिनम्}
{दुर्योधनायेषुवरं तं द्रौणिः सप्तधाऽच्छिनत्}


\twolineshloka
{ततो द्रौणेर्धनुश्छित्वा हत्वा चाश्वरथाञ्शरैः}
{कृपस्यापि तदत्युग्रं धनुश्चिच्छेद पाण्डवः}


\twolineshloka
{हार्दिक्यस्य धनुश्छित्त्वा ध्वजं चाश्वांस्तदाऽवधीत्}
{दुःशासनस्येष्वसनं छित्त्वा राधेयमभ्ययात्}


\twolineshloka
{अथ सात्यकिमुत्सृज्य त्वरन्कर्णोऽर्जुनं त्रिभिः}
{विद्व्वा विव्याध विंशत्या कृष्णं पार्थं पुनःपुनः}


\twolineshloka
{न ग्लानिरासीत्कर्णस्य क्षिपतः सायकान्बहून्}
{रणे विनिघ्नतः शत्रून्क्रुद्धस्येव शतक्रतोः}


\twolineshloka
{अथ सात्यकिरागत्य कर्णं विद्द्वा शितैः शरैः}
{नवत्या नवभिश्चोग्रैः शतेन पुनरार्पयत्}


\twolineshloka
{ततः प्रवीराः पार्थानां सर्वे कर्णमपीडयन्}
{युधामन्युः शिखण्डी च द्रौपदेयाः प्रभद्रकाः}


\twolineshloka
{उत्तमौजा युयुत्सुश्च यमौ पार्षत एव च}
{चेदिकारूशमात्स्यानां केकयानां च यद्बलं}


\twolineshloka
{चेकितानश्च बलवान्धर्मराजश्च सुव्रतः}
{एते रथाश्वद्विरदैः पत्तिभिश्चोग्रविक्रमैः}


\twolineshloka
{परिवार्य रणे कर्णं नानाशस्त्रैरवाकिरन्}
{भाषन्तो वाग्भिरुग्राभिः सर्वे कर्णवधे धृताः}


\twolineshloka
{तां शस्त्रवृष्टिं बहुधा कर्णश्छित्त्वा शितैः शरैः}
{अपोवाहास्त्रवीर्येण द्रुमं भङ्क्त्वेव मारुतः}


\twolineshloka
{रथिनः समहामात्रान्गजानश्वान्ससादिनः}
{पत्तिव्रातांश्च सङ्क्रुद्वो निघ्नन्कर्णो व्यदृश्यत}


\twolineshloka
{तद्वध्यमानं पाण्डूनां बलं कर्णास्त्रतेजसा}
{विशस्त्रपत्रदेहासु प्राय आसीत्पराङ्मुखम्}


\twolineshloka
{अथ कर्णास्त्रमस्त्रेण प्रतिहत्यार्जुनः स्मयन्}
{दिशं खं चैव भूमिं च प्रावृणोच्छरवृष्टिभिः}


\twolineshloka
{मुसलानीव सम्पेतुः परिघा इव चेषवः}
{शतघ्न्य इव चाप्यन्ये वज्राण्युग्राणि चापरे}


\twolineshloka
{तैर्वध्यमानं तत्सैन्यं सपत्त्यश्वरथद्विपम्}
{निमीलिताक्षमत्यर्थं बभ्राम च ननाद च}


\twolineshloka
{निष्कैवल्यं तदा युद्वं प्रापुरश्वनरद्विपाः}
{हन्यमानाः शरैरार्तास्तदा भीताः प्रदुद्रुवुः}


\twolineshloka
{त्वदीयानां तदा युद्वे संसक्तानां जयैषिणाम्}
{गिरिमस्तं समासाद्य प्रत्यपद्यत भानुमान्}


\twolineshloka
{तमसा च महाराज रजसा च विशेषतः}
{न किञ्चित्प्रत्यपश्याम शुभं वा यदि वाऽशुभम्}


\twolineshloka
{ते त्रस्यन्तो महेष्वासा रात्रियुद्वस्य भारत}
{अपयानं ततश्चक्रुः सहिताः सर्वयोधिभिः}


\twolineshloka
{कौरवेष्वपयातेषु तदा राजन्दिनक्षये}
{जयं सुमनसः प्राप्य पार्थाः स्वशिबिरं ययुः}


\twolineshloka
{वादित्रशब्दैर्विविधैः सिंहनादैः सगर्जितैः}
{परानुवहन्तश्च स्तुवन्तश्चाच्युतार्जुनौ}


\twolineshloka
{कृतेऽवहारे तैर्वीरैः सैनिकाः सर्व एव ते}
{आशीर्वाचः पाण्डवेषु प्रायुञ्जन्त नरेश्वराः}


\twolineshloka
{ततः कृतेऽवहारे च प्रहृष्टास्तत्र पाण्डवाः}
{निशायां शिबिरं गत्वा न्यवसन्त नरेश्वराः}


\twolineshloka
{ततो रक्षः पिशाचाश्च श्वापदाश्चैव सङ्घशः}
{जग्मुरायोधनं घोरं रुद्रस्याक्रीडसन्निभम्}


\chapter{अध्यायः २६}
\twolineshloka
{धृतराष्ट्र उवाच}
{}


\twolineshloka
{स्वेन च्छन्देन नः सर्वानवधीद्व्यक्तमर्जुनः}
{न ह्यस्य समरे मुच्येदन्तकोऽप्याततायिनः}


\twolineshloka
{पार्थो ह्येकोऽहरद्भद्रामेकश्चाग्निमतर्पयत्}
{एकश्चेमां महीं जित्वा चक्रे बलिभृतो नृपान्}


\twolineshloka
{एको निवातकवचनानहनद्दिव्यकार्मुकः}
{एकः किरातरूपेण स्थितं शर्वमयोधयत्}


\twolineshloka
{एको ह्यरक्षद्भरतानेको भवमतोषयत्}
{तेनैकेन जिताः सर्वे मदीया ह्युग्रतेजसः}


\threelineshloka
{न ते निन्द्याः प्रशस्यास्ते यत्ते चक्रुर्ब्रवीहि तत्}
{ततो दुर्योधनः सूत पश्चात्किमकरोत्तदा ॥सञ्जय उवाच}
{}


\twolineshloka
{हतप्रहतविध्वस्ता विवर्मायुधवाहनाः}
{दीनस्वरा दूयमाना मानिनः शत्रुनिर्जिताः}


\twolineshloka
{शिबिरस्थाः पुनर्मन्त्रं मन्त्रयन्ति स्म कौरवाः}
{भग्नदंष्ट्रा हतविषाः पादाक्रान्ता इवोरगाः}


\twolineshloka
{तानब्रवीत्ततः कर्णः क्रुद्वः सर्प इव श्वसन्}
{करं करेण निष्पीड्य प्रेक्षमाणस्तवात्मजम्}


\twolineshloka
{यत्तो दृढश्च दक्षश्च धृतिमानर्जुनः सदा}
{सम्बोधयति चाप्येनं यथाकालमधोक्षजः}


\twolineshloka
{सहसाऽस्त्रविसर्गेण वयं तेनाद्य वञ्चिताः}
{श्वस्त्वहं तस्य सङ्कल्पं सर्वं हन्ता महीपते}


\twolineshloka
{एवमुक्तस्तथेत्युक्त्वा सोऽनुजज्ञे नृपोत्तमान्}
{तेऽनुज्ञाता नृपाः सर्वे स्वानि वेश्मानि भेजिरे}


\threelineshloka
{सुखोषितास्तां रजनीं हृष्टा युद्वाय निर्ययुः}
{तेऽपश्यन्विहितं व्यूहं धर्मराजेन दुर्जयम्}
{प्रयत्नात्कुरुमुख्येन बृहस्पत्युशनोमतात्}


\twolineshloka
{अथ प्रतीपकर्तारं प्रवीरं परवीरहा}
{सस्मार वृषभस्कन्धं कर्णं दुर्योधनस्तदा}


\twolineshloka
{पुरन्दरसमं युद्वे मरुद्गणसमं बले}
{कार्तवीर्यसमं वीर्ये कर्णं राज्ञोऽगमन्मनः}


\threelineshloka
{सर्वेषां चैव सैन्यानां कर्णमेवागमन्मनः}
{सूतपुत्रं महेष्वासं बन्धुमात्ययिकेष्विव ॥धृतराष्ट्र उवाच}
{}


\twolineshloka
{ततो दुर्योधनः सूत पश्चात्किमकरोत्तदा}
{यद्वोऽगमन्मनो मन्दाः कर्णं वैकर्तनं प्रति}


\twolineshloka
{अप्यपश्यत राधेयं शीतार्ता इव भास्करम्}
{कृतेऽवहारे सैन्यानां प्रवृत्ते च रणे पुनः}


\twolineshloka
{`दुर्योधनं च तत्राजौ पाण्डवेन भृशार्दितम्}
{पराक्रान्तान्पाण्डुसुतान्दृष्ट्वाऽपि भृशार्दितम्'}


\twolineshloka
{कथं वैकर्तनः कर्णस्तत्रायुध्यत सञ्जय}
{कथं च पाण्डवाः सर्वे युयुधुस्तत्र सूतजम्}


\twolineshloka
{कर्णो ह्येको महाबाहुर्हन्यात्पार्थान्ससृञ्जयान्}
{कर्णस्य भुजयोर्वीर्यं शक्रविष्णुसमं युधि}


\twolineshloka
{तस्य शस्त्राणि घोराणि विक्रमश्च महात्मनः}
{कर्णमाश्रित्य सङ्ग्रामे मत्तो दुर्योधनो नृपः}


\twolineshloka
{दुर्योधनं ततो दृष्ट्वा पाण्डवेन भृशार्दितम्}
{पराक्रान्तान्पाण्डुसुतान्दृष्ट्वा चापि महारथः}


\twolineshloka
{कर्णमाश्रित्य सङ्ग्रामे मन्दो दुर्योधनः पुनः}
{जेतुमुत्सहते पार्थान्सपुत्रान्सहकेशवान्}


\twolineshloka
{`यः सौबलं तथा तात नीतिमानिति मन्यते}
{कर्णं चाग्रतिमं युद्वे देवैरपि दुरुत्सहम्}


\threelineshloka
{मन्यतेऽभ्यधिकं पार्थादेवं चास्य हृदि स्थितम्}
{विजेष्यति रणे कर्ण एकः पर्थान्ससोमकान्}
{मम चैव सदा मन्दः शंसते नित्यमग्रतः'}


\twolineshloka
{अहो बत महद्दुःखं यत्र पाण्डुसुतान्रणे}
{नातरद्रभसः कर्णो दैवं नूनं परायणम्}


% Check verse!
अहो द्यूतस्य निष्ठेयं घोरा सम्प्रति वर्तते
\twolineshloka
{अहो तीव्राणि दुःखानि दुर्योधनकृतान्यहम्}
{सोढा घोराणि बहुशः शल्यभूतानि सञ्जय}


\twolineshloka
{सौबलं च तदा तात नीतिमानिति मन्यते}
{कर्णश्च रभसो नित्यं राजा त चाप्यनुव्रतः}


\twolineshloka
{यदेवं वर्तमानेषु महायुद्वेषु सञ्जय}
{अश्रौषं निहतान्पुत्रान्नित्यमेव विनिर्जितान्}


\threelineshloka
{न पाण्डवानां समरे कश्चिदस्ति हतः किल}
{स्त्रीमध्यमिव गाहन्ते दैवं तु बलवत्तरम् ॥सञ्जय उवाच}
{}


\twolineshloka
{[राजन्पूर्वनिमित्तानि धर्मिष्ठानि विचिन्तय]}
{}


\twolineshloka
{अतिक्रान्तं हि यत्कार्यं पश्चाच्चिन्तयते नरः}
{तच्चास्य न भवेत्कार्यं चिन्तया च विनश्यति}


\twolineshloka
{तदिदं तव कार्यं तु दूरप्राप्तं विजानता}
{न कृतं यत्त्वया पूर्वं प्राप्ताप्राप्तविचारणम्}


\twolineshloka
{उक्तोऽसि बहुधा राज्ञन्मा युध्यस्वेति पाण्डवैः}
{न च तत्त्वग्रहीर्द्वेषात्पाण्डवेषु विशाम्पते}


\twolineshloka
{त्वया पापानि घोराणि समाचीर्णानि पाण्डुषु}
{त्वत्कृते वर्तते घोरः पार्थिवानां जनक्षयः}


\twolineshloka
{तत्त्विदानीमतिक्रान्तं मा शुचो भरतर्षभ}
{शृणु सर्वं यथावृत्तं घोरं वैशसमुच्यते}


\threelineshloka
{प्रभातायां रजन्यां तु कर्णो राजानमभ्ययात्}
{समेत्य च महाबाहुर्दुर्योधनमथाब्रवीत् ॥कर्ण उवाच}
{}


\twolineshloka
{अद्य राजन्समेष्यामि पाण्डवेन यशस्विना}
{निहनिष्यामि तं वीरं स वा मां निहनिष्यति}


\twolineshloka
{बहुत्वान्मम कार्याणां तथा पार्थस्य भारत}
{नाभूत्समागमो राजन्मम चैवार्जुनस्य च}


\twolineshloka
{इदं तु मे यथाप्रज्ञं शृणु वाक्यं विशाम्पते}
{अनिहत्य रणे पार्थं नाहमेष्यामि भारत}


\twolineshloka
{हतप्रवीरे सैन्येऽस्मिन्मयि चावस्थिते युधि}
{अभियास्यति मां पार्थः शक्रशक्तिविनाकृतम्}


\twolineshloka
{ततः श्रेयस्करं यच्च तन्निबोध जनेश्वर}
{आयुधानां च मे वीर्यं दिव्यानामर्जुनस्य च}


\twolineshloka
{कार्यस्य महतो भेदे लाघवे दूरपातने}
{सौष्ठवे चास्त्रपाते च सव्यसाची न मत्समः}


\twolineshloka
{प्राणे शौर्येऽथ विज्ञाने विक्रमे चापि भारत}
{निमित्तज्ञानयोगे च सव्यसाची न मत्समः}


\twolineshloka
{सर्वायुधमहामात्रं विजयं नाम तद्वनुः}
{इन्द्रार्थं प्रियकामेन निर्मितं विश्वकर्मणा}


\twolineshloka
{येन दैत्यगणान्राजञ्जितवान्वै शतक्रतुः}
{यस्य घोषेण दैत्यानां व्यामुह्यन्त दिशो दश}


\twolineshloka
{तद्भार्गवाय प्रायच्छच्छक्रः परमसम्मतम्}
{तद्दिव्यं भार्गवो मह्यमददद्वनुरत्तमम्}


\threelineshloka
{तेन योत्स्ये महाबाहुमर्जुनं जयतां वरम्}
{यथेन्द्रः समरे सर्वान्दैतेयान्वै समागतान्}
{`निजघान तथा सर्वाञ्जेष्यामि युधि पाण्डवान्'}


\twolineshloka
{धनुर्घोरं रामदत्तं गाण्डीवात्तद्विशिष्यते}
{त्रिस्सप्तकृत्वः पृथिवी धनुषा येन निर्जिता}


\twolineshloka
{धनुषो ह्यस्य कर्माणि दिव्यानि प्राह भार्गवः}
{तद्रामो ह्यददन्मह्यं तेन योत्स्यामि पाण्डवम्}


\twolineshloka
{अद्य दुर्योधनाहं त्वां नन्दयिष्ये सबान्धवम्}
{निहत्य समरे वीरमर्जुनं जयतां वरम्}


\twolineshloka
{सपर्वतवनद्वीपा हतवीरा ससागरा}
{पुत्रपौत्रप्रतिष्ठा ते भविष्यत्यद्य पार्थिव}


\twolineshloka
{नाशक्यं विद्यते मेऽद्य त्वत्प्रियार्थं विशेषतः}
{सम्यग्धर्मानुरक्तस्य सिद्विरात्मवतो यथा}


\twolineshloka
{न हि मे विक्रमं सोढुं स शक्तोऽग्निं तरुर्यथा}
{अवश्यं तु मया वाच्यं येन हीनोऽस्मि फल्गुनात्}


\twolineshloka
{ज्या तस्य धनुषो दिव्या तथाऽक्षय्ये महेषुधी}
{सारथिस्तस्य गोविन्दो मम तादृङ्ग विद्यते}


\twolineshloka
{तस्य दिव्यं धनुः श्रेष्ठं गाण्डीवमजितं युधि}
{विजयं च महद्दिव्यं ममापि धनुरुत्तमम्}


\twolineshloka
{तत्राहमधिकः पार्थाद्वनुषा तेन पार्थिव}
{येन चाप्यधिको वीरः पाण्डवस्तन्निबोध मे}


\twolineshloka
{रश्मिग्राहश्च दाशार्हः सर्वलोकनमस्कृतः}
{अग्निदत्तश्च वै दिव्यो रथः काञ्चनभूषणः}


\twolineshloka
{अच्छेद्याः सर्वतो वीर वाजिनश्च मनोजवाः}
{ध्वजश्च दिव्यो द्युतिमान्वानरोपि भयङ्करः}


\twolineshloka
{कृष्णश्च जगतः स्रष्टा रथं तमभिरक्षति}
{एभिस्त्रिभिरहं हीनो युद्वुमिच्छामि पाण्डवम्}


\twolineshloka
{अयं तु सदृशः शौरेः शल्यः समितिशोभनः}
{सारथ्यं यदि मे कुर्याद्द्रुवस्ते विजयो भवेत्}


\twolineshloka
{तस्य मे सारथिः शल्यो भवत्वसुकरः परैः}
{नाराचान्गार्ध्रपत्रांश्च शकटानि वहन्तु मे}


\twolineshloka
{रथाश्च मुख्या राजेन्द्र युक्ता वाजिभिरुत्तमैः}
{आयान्तु पश्चात्सततं मामेव भरतर्षभ}


\twolineshloka
{एवमभ्यधिकः पार्थाद्भविष्यामि गुणैरहम्}
{शल्योप्यभ्यधिकः कृष्णादर्जुनादपिचाप्यहम्}


\twolineshloka
{यथाऽश्वहृदयं वेद दाशार्हः परवीरहा}
{तथा शल्यो विजानीते हयज्ञानं महारथः}


\twolineshloka
{बाहुवीर्ये समो नास्ति मद्रराजस्य कश्चन}
{तथाऽस्त्रे मत्समो नास्ति कश्चिदेव धनुर्धरः}


\twolineshloka
{तथा शल्यसमो नास्ति हयज्ञाने हि कश्चन}
{सोऽयमभ्यधिकः कृष्णाद्भविष्यति रथो मम}


\twolineshloka
{एवं कृते रथस्थोऽहं गुणैरभ्यधिकोऽर्जुनात्}
{विजयेयमहं सङ्ख्ये फल्गुनं कुरुसत्तम}


\twolineshloka
{समुद्यातुं न शक्ष्यन्ति देवा अपि सवासवाः}
{एतत्कृतं महाराज त्वयेच्छामि परन्तप}


\twolineshloka
{क्रियतामेष कामो मे मा वः कालोऽत्यगादयम्}
{एवं कृते कृतं मह्यं त्वया सर्वं भविष्यति}


\twolineshloka
{ततो द्रक्ष्यसि सङ्ग्रामे यत्करिष्यामि भारत}
{सर्वथा पाण्डवान्सङ्ख्ये विजेष्ये वै समागतान्}


\threelineshloka
{न हि मे समरे शक्ताः समुद्यातुं सुरासुराः}
{किमु पाण्डुसुता राजन्रणे मानुषयोनयः ॥सञ्जय उवाच}
{}


\threelineshloka
{एवमुक्तस्तव सुतः कर्णेनाहवशोभिना}
{सम्पूज्य सम्प्रहृष्टात्मा ततो राधेयमब्रवीत् ॥दुर्योधन उवाच}
{}


\twolineshloka
{एवमेतत्करिष्यामि यथा त्वं कर्ण मन्यसे}
{सोपासङ्गा रथाः साश्वाः स्वनुयास्यन्ति संयुगे}


\threelineshloka
{नाराचान्गार्ध्रपत्रांश्च शकटानि वहन्तु ते}
{अनुयास्याम कर्ण त्वां वयं सर्वे च पार्थिवाः ॥सञ्जय उवाच}
{}


\twolineshloka
{एवमुक्त्वा महाराज तव पुत्रः प्रतापवान्}
{अभिगम्याब्रवीद्राजा मद्रराजमिदं वचः}


\chapter{अध्यायः २७}
\twolineshloka
{सञ्जय उवाच}
{}


\threelineshloka
{एवमुक्तो महाराज तव पुत्रः प्रतापवान्}
{मद्रेश्वरं प्रति रणे सुरसैन्यभयङ्करम्}
{सर्वं तथाऽकरोत्तूर्णं राधेयस्तद्यथाऽब्रवीत्}


\twolineshloka
{मद्राजं च समरे प्रणम्य च महारथम्}
{विनयेनोपसङ्गम्य प्रणयाद्वाक्यमब्रवीत्}


\twolineshloka
{सत्यव्रत महाभाग द्विषतां तापवर्धन}
{मद्रेश्वर रणेशूर परसैन्यभयङ्कर}


\threelineshloka
{`मत्प्रियार्थं हितार्थं च तथा पार्थवधाय च'}
{श्रुतवानसि कर्णस्य ब्रुवतो वदतां वर}
{यथा नृपतिसिंहानां मध्ये त्वां वरये स्वयम्}


\twolineshloka
{तत्त्वामप्रतिवीर्याद्य शत्रुपक्षक्षयावह}
{मद्रेश्वर प्रयाचेऽहं शिरसा विनयेन च}


\twolineshloka
{तस्मात्पार्थविनाशार्थं हितार्थं मम चैव हि}
{त्वं पाहि सर्वतः कर्णं भवं ब्रह्मेव सुव्रत}


\twolineshloka
{सारथ्यं रथिनां श्रेष्ठ प्रणयात्कर्तुमर्हसि}
{त्वयि यन्तरि राधेयो विद्विषो मे विजेष्यते}


\twolineshloka
{अभीषूणां हि कर्णस्य ग्रहीतान्यो न विद्यते}
{ऋते हि त्वां महाभाग वासुदेवसमं युधि}


\threelineshloka
{स पाहि सर्वथा कर्णं यथा ब्रह्मा महेश्वरम्}
{यथा च सर्वथाऽऽपत्सु वार्ष्णेयः पाति पाण्डवम्}
{तथा मद्रेश्वराद्य त्वं राधेयं प्रतिपालय}


\twolineshloka
{भीष्मो द्रोणः कृपः कर्णो भवान्भोजश्च वीर्यवान्}
{शकुनिः सौबलो द्रौणिरहमेव च नो बलम्}


\twolineshloka
{एवमेष कृतो भागो नवधा पृथिवीपते}
{न च भागोऽत्र भीष्मस्य द्रोणस्य च महात्मनः}


\twolineshloka
{ताभ्यामतीत्य तौ भागौ निहता मम शत्रवः}
{वृद्वौ हि तौ महेष्वासौ छलेन निहतौ युधि}


\twolineshloka
{कृत्वा चासुकरं कर्म गतौ स्वर्गमितोऽनघ}
{तथान्ये पुरुषव्याघ्राः परैर्विनिहता युधि}


\threelineshloka
{अस्मदीयाश्च बहवः स्वर्गाय गमिताः परैः}
{त्यक्त्वा प्राणानय्थाशक्ति चेष्टां कृत्वा च पुष्कलाम्}
{}


\twolineshloka
{तदिदं हतभूयिष्ठं बलं मम नराधिप}
{पूर्वमप्यल्पकैः पार्थैर्हतं किमुत साम्प्रतम्}


\twolineshloka
{बलवन्तो महात्मानः कौन्तेयाः सत्यविक्रमाः}
{बलं शेषं न हन्युर्मे यथा तत्कुरु पार्थिव}


\twolineshloka
{हतवीरमिदं सैन्यं पाण्डवैः समरे विभो}
{कर्णो ह्येको महाबाहुरस्मत्प्रियहिते रतः}


\twolineshloka
{भवांश्च पुरुषव्याघ्र सर्वलोकमहारथः}
{शल्य कर्णोऽर्जुनेनाद्य योद्वुमिच्छति संयुगे}


\twolineshloka
{तस्मिञ्जयाशा विपुला मद्रराज नराधिप}
{तस्याभीषुग्रहवरो नान्योऽस्ति भुवि कश्चन}


\twolineshloka
{पार्थस्य समरे कृष्णो यथाऽभीषुग्रहो वरः}
{तथा त्वमपि कर्णस्य रथेऽभीषुग्रहो भव}


\twolineshloka
{तेन युक्तो रणे पार्थो रक्ष्यमाणश्च पार्थिव}
{यानि कर्माणि कुरुते प्रत्यक्षाणि तथैव तत्}


\twolineshloka
{पूर्वं नः समरे ह्येवमधीदर्जुनो रिपून्}
{इदानीं विक्रमो ह्यस्य कृष्णेन सहितस्य च}


\twolineshloka
{कृष्णेन सहितः पार्थो धार्तराष्ट्रीं महाचमूम्}
{अहन्यहनि मद्रेश द्रावयन्दृश्यते युधि}


\twolineshloka
{भागोऽवशिष्टः कर्णस्य तव चैव महाद्युते}
{तं भागं सह कर्णेन युगपन्नाशयाद्य हि}


\twolineshloka
{अरुणेन यथा सार्धं तमः सूर्यो व्यपोहति}
{तथा कर्णेन सहितो जहि पार्थं महाहवे}


\twolineshloka
{उद्यन्तौ च यथा सूर्यौ बालसूर्यसमप्रभौ}
{कर्णशल्यौ रणे दृष्ट्वा विद्रवन्तु महारथाः}


\twolineshloka
{सूर्यारुषौ यथा दृष्ट्वा तमो नश्यति मारिष}
{तथा नश्यन्तु कौन्तेयाः सपाञ्चालाः ससृञ्जयाः}


\twolineshloka
{रथिनां प्रवरः कर्णो यन्तॄणां प्रवरो भवान्}
{संयोगो युवयोर्लोके नाभून्न च भविष्यति}


\threelineshloka
{यथा सर्वास्ववस्थासु वार्ष्मेयः पाति पाण्डवम्}
{तथा भवान्परित्रातु कर्णं वैकर्तनं रणे}
{`सारथ्यं क्रियतां तस्य युध्यमानस्य संयुगे}


\twolineshloka
{यथा च समरे कृष्णो रक्षते सर्वतोऽर्जुनम्}
{तथा त्वमपि राधेयं रक्षस्व च महारणे'}


\fourlineindentedshloka
{त्वया सारथिना ह्येष अप्रधृष्यो भविष्यति}
{देवतानामपि रणे सशक्राणां महीपते}
{किं पुनः पाण्डवेयानां मा विशङ्कीर्वचो मम ॥सञ्जय उवाच}
{}


\twolineshloka
{दुर्योधनवचः श्रुत्वा शल्यः क्रोधसमन्वितः}
{`धार्तराष्ट्रमथोवाच सृक्विणी परिलेलिहन्'}


\fourlineindentedshloka
{त्रिशिखां भ्रुकुटिं कृत्वा धुन्वन्हस्तौ पुनःपुनः}
{क्रोधरक्ते महानेत्रे परिवृत्य महाभुजः}
{कुलैश्वर्यश्रुतबलैर्दृप्तः शल्योऽब्रवीदिदम् ॥शल्य उवाच}
{}


\twolineshloka
{न मामर्हसि राजेद्र नियोक्तुं कर्मणीदृशे}
{न हि पापीयसः श्रेयान्प्रेष्यत्वं कर्तुमर्हति}


\twolineshloka
{यो ह्यभ्युपगतं मित्रं गरीयांसं विशेषतः}
{वशे पापीयसो धत्ते तत्कार्यमधरोत्तरम्}


\twolineshloka
{ब्रह्मणा ब्राह्मणाः सृष्टा मुखात्क्षत्रं च बाहुतः}
{ऊरुभ्यामथ वैश्या वै शूद्राः पद्भ्यामिति श्रुतिः}


\twolineshloka
{तेभ्यो वर्णविशेषाश्च प्रतिलोमानुलोमजाः}
{अथान्योन्यस्य संयोगाच्चातुर्वर्ण्यस्य भारत}


% Check verse!
गोप्तारः सङ्गृहीतारो दातारः क्षत्रियाः स्मृताः
\twolineshloka
{याजनाध्यापनैर्विप्रा विशुद्वैश्च प्रतिग्रहैः}
{लोकस्यानुग्रहार्थाय स्थापिता ब्राह्मणा भुवि}


\twolineshloka
{कृषिश्च पाशुपाल्यं च विशां दानं च धर्मतः}
{ब्रह्मक्षत्रविशां शूद्रा विहिताः परिचारकाः}


\twolineshloka
{ब्रह्मक्षत्रस्य विहिताः सूता वै परिचारकाः}
{न क्षत्रियो वै सूतानां शृणुयाच्च कथञ्चन}


\twolineshloka
{अहं मूर्धाभिषिक्तो हि राजर्षिकुलजो नृपः}
{महारथसमाख्यातः सेव्यः स्तुत्यश्च वन्दिनां}


\twolineshloka
{सोऽमहेतादृशो भूत्वा नेहारिबलसूदनः}
{सूतपुत्रस्य सङ्ग्रामे सारथ्यं कर्तुमुत्सहे}


\twolineshloka
{अवमन्यसे मां गान्धारे ध्रुवं चाप्यतिशङ्कसे}
{यस्माद्ब्रवीषि विस्रब्धः सारथ्यं क्रियतामिति}


\twolineshloka
{अस्मत्तोऽभ्यधिकं कर्णं मन्वानस्तं प्रशंससि}
{न चाहं युधि राधेयं गणये तुल्यमात्मनः}


\twolineshloka
{आदिश्यतामभ्यधिको ममांशः पृथिवीपते}
{तमहं समरे हत्वा गमिष्यामि यथागतम्}


\twolineshloka
{अथवाप्येक एवाहं योत्स्यामि तव शत्रुभिः}
{पश्य वीर्यं ममाद्य त्वं सङ्ग्रामे दहतो रिपून्}


\twolineshloka
{न चापि कामान्कौरव्य निधाय हृदये पुमान्}
{अस्मद्विधः प्रवर्तेत मा मा त्वमतिशङ्किथाः}


\twolineshloka
{अपि चाप्यवमानो मे न कर्तव्यः कथञ्चन}
{पश्य भीमौ मम भुजौ वज्रसंहननौ दृढौ}


\twolineshloka
{धनुः पश्य च मे चित्रं शरांश्चाशीविषपमान्}
{रथं च पश्य मे चित्रं सदश्वैर्वातवेगिभिः}


\threelineshloka
{गदां च पश्य गान्धारे हेमपट्टविभूषिताम्}
{दारयेयं महीं क्रुद्धो विकिरेयं च पर्वतान्}
{शोषयेयं समुद्रांश्च पातयेयं च भास्करम्}


\twolineshloka
{तं मामेवंविधं राजन्समर्थमरिनिग्रहे}
{कस्माद्युनङ्क्षि सारथ्ये सूतस्याधिरथेस्तु माम्}


% Check verse!
न चापि मम राधेयः कलामर्हति षोडशीम्
\twolineshloka
{सकर्णा ये त्रयो लोकाः सार्जुनाः सजनार्दनाः}
{निबद्वांस्तन्त्रणे राजन्न गणेयं कथञ्चन}


\twolineshloka
{न चाहं प्राकृतः कश्चिन्न चास्म्यधिगतः स्वयम्}
{अयं चाप्यवमानो मे न कर्तव्यः कथञ्चन}


\twolineshloka
{आपृच्छे त्वाद्य गान्धारे यास्यामि विषयं प्रति}
{न चाहं सूतपुत्रस्य सारथ्यमुपजग्मिवान्}


\threelineshloka
{अवमानमहं प्राप्य न योत्स्यामि कथञ्चन}
{आपृच्छे त्वाऽद्य गान्धारे गमिष्यामि गृहाय वै ॥सञ्जय उवाच}
{}


\twolineshloka
{एवमुक्त्वा महाराज शल्यः समितिशोभनः}
{उत्थाय प्रययौ तूर्णं राजमध्यादमर्षितः}


\twolineshloka
{प्रणयाद्बहुमानाच्च तं निगृह्य सुतस्तव}
{अब्रवीन्मधुरं वाक्यं साम्ना सर्वार्थसाधकम्}


\twolineshloka
{यथा शल्य त्वमात्थेदमेवमेतदसंशयम्}
{अभिप्रायस्तु मे कश्चित्तं निबोध जनेश्वर}


\twolineshloka
{न कर्णोऽभ्यधिकस्त्वत्तो न चान्ये चैव पार्थिवाः}
{न च मद्रेश्वर त्वां वै कृष्णः सोढुं च शक्ष्यति}


\twolineshloka
{ऋतमेव हि पूर्वास्ते वदन्ति पुरुषोत्तमाः}
{तस्मादार्तायनिः प्रोक्तो भवानिति मतिर्मम}


\twolineshloka
{शल्यबूतस्तु शत्रूणां यस्मात्त्वं युधि मानद}
{तस्माच्छल्येति ते नाम कथ्यते पृथिवीतले}


\twolineshloka
{यदेतद्व्याहृतं पूर्वं भवता भूरिदक्षिण}
{तदेव कुरु धर्मज्ञ मदर्थं यद्यदुच्यते}


\twolineshloka
{न च त्वत्तो हि राधेयो न चाहमपि वीर्यवान्}
{वृणेऽहं त्वां हयाग्र्याणां यन्तारमिह संयुगे}


\twolineshloka
{मन्ये चाभ्यधिकं शल्य गुणैः कर्णं धनञ्जयात्}
{भवन्तं वासुदेवाच्च लोकोऽयमतिमन्यते}


\twolineshloka
{कर्णो ह्यभ्यधिकः पार्थादस्त्रैरेव नरर्षभ}
{भवानभ्यधिकः कृष्णादश्वज्ञाने बले तथा}


\threelineshloka
{यथाऽश्वहृदयं वेद वासुदेवो महामनाः}
{द्विगुणं त्वं तथा वेत्सि मद्रराजेश्वरात्मज ॥शल्य उवाच}
{}


\twolineshloka
{यन्मां ब्रवीषि गान्धारे मध्ये सैन्यस्य कौरव}
{विशिष्टं देवकीपुत्रात्प्रीतिमानस्म्यहं त्वयि}


\twolineshloka
{एष सारथ्यमातिष्ठे राधेयस्य यशस्विनः}
{युध्यतः पाण्डवाग्र्येण यथा त्वं वीर मन्यसे}


\twolineshloka
{समयश्च हि मे वीर कश्चिद्वैकर्तनं प्रति}
{उत्सृजेयं यथाश्रद्वमहं वाचोऽस्य सन्निधौ}


\threelineshloka
{सञ्जय उवाच}
{तथेति राजन्पुत्रस्ते सह कर्णेन भारत}
{अब्रवीन्मद्रराजस्य मतं भरतसत्तम्}


\chapter{अध्यायः २८}
\twolineshloka
{दुर्योधन उवाच}
{}


\twolineshloka
{भूय एव तु मद्रेश यत्त्वा वक्ष्यामि तच्छृणु}
{यथा पुरावृत्तमिदं युद्धे देवासुरे प्रभो}


\threelineshloka
{यदुक्तवान्पितुर्मह्यं मार्कण्डेयो महातपाः}
{ब्रुवतस्तदशेषेण मम राजर्षिसत्तम}
{निबोध मनसा चात्र न ते कार्या विचारणा}


\twolineshloka
{`समुत्पन्नो हि राजानः प्रमोह इति निश्चयम्}
{कृत्वा चैव व्यवस्यन्ति सर्वे धर्मार्थनिश्चयान्}


\threelineshloka
{देवानामसुराणां च महानासीत्समुच्छ्रयः}
{सैंहिकेयास्तदोद्वृत्ता विबुधानवसूदयन्}
{ते निरस्तः कृता देवैर्दानवा बलगर्विताः'}


\twolineshloka
{तत्रासीत्प्रथमो राजन्सङ्ग्रामस्तारकामयः}
{निर्जिताश्च ततो दैत्या दैवतैरिति नः श्रुतिः}


% Check verse!
भग्नदर्पा निरुत्साहाः पातालं विविशुस्तदा
\twolineshloka
{निर्जितेषु च दैत्येषु तारकस्य सुतास्त्रयः}
{ताराक्षः कमलाक्षश्च विद्युन्माली च पार्थिव}


\twolineshloka
{तप उग्रं समास्थाय नियमे परमे स्थिताः}
{तपसा कर्णयामासुर्देहांस्ताञ्शत्रुकर्शनाः}


\twolineshloka
{दमेन तपसा चैव नियमेन समाधिना}
{तेषां पितामहः प्रीतो वरदः प्रददौ वरम्}


\twolineshloka
{अवध्यत्वं च ते सर्वे सर्वभूतेषु सर्वदा}
{सहिता वरयामासः सर्वलोकपितामहम्}


\threelineshloka
{तानब्रवीत्तदा देवः सर्वलोकगुरुः प्रभुः}
{नास्ति सर्वामरत्वं वै निवर्तध्वमितोऽसुराः}
{अन्यं वरं वृणीध्वं वै रोचते यादृशो हि वः}


\twolineshloka
{ततस्ते सहिता राजन्सम्प्रधार्यासकृद्बहुः}
{सर्वलोकेश्वरं वाक्यं प्रणम्यैनमथाऽब्रुवन्}


\twolineshloka
{वस्तुमिच्छाम नगरं कर्तुं कामगमं शुभम्}
{सर्वकामसमृद्वार्थमवध्यं देवदानवैः}


\threelineshloka
{यक्षरक्षोरगगणैर्नानाजातिभिरेव च}
{न कृत्याभिर्न शस्त्रैश्च न शापैर्ब्रह्मवादिनाम्}
{वध्येत त्रिपुरं देव प्रयच्छेः प्रपितामह}


\twolineshloka
{वयं पुराणि त्रीण्येव समास्थाय महीमिमाम्}
{विचरिष्याम लोकेऽस्मिंस्त्वत्प्रसादपुरस्कृताः}


\twolineshloka
{ततो वर्षसहस्रेषु समेष्यामः परस्परम्}
{एकीभावं गभिष्यन्ति पुराण्येतानि चानघ}


\threelineshloka
{समागतानि चैतानि यो हन्याद्भगवांस्तदा}
{एकेषुणा देववरः स नो मृत्युर्भविष्यति ॥दुर्योधन उवाच}
{}


\twolineshloka
{तेषां तद्वचनं श्रुत्वा दानवानां पितामहः}
{एवमस्त्विति तान्देवः प्रत्युक्त्वा प्राविशद्दिवम्}


\threelineshloka
{ते तु लब्धवराः प्रीताः सम्प्रधार्य परस्परम्}
{पुरत्रयविसृष्ट्यर्थं मयं वव्रुर्महारथाः}
{विश्वकार्माणमजरं दैत्यदानवपूजितम्}


\twolineshloka
{ततो मयः स्वतपसा चक्रे धीमान्पुराणि च}
{त्रीणि काञ्चनमेकं वै रौप्यं कार्ष्णायसं तथा}


\twolineshloka
{काञ्चनं दिवि तत्रासीदन्तरिक्षे च राजतम्}
{आयसं चाभवद्भौमं तदा तेषां परन्तप}


\twolineshloka
{एकैकं योजनशतं विस्तृतं तावदायतम्}
{दृढं चाट्टालकयुतं बृहत्प्राकारतोरणम्}


\twolineshloka
{गृहप्रवरसम्बाधमसम्बाधमहापथम्}
{प्रासादैर्विविधैश्चापि द्वारैश्चैवोपशोभितम्}


\twolineshloka
{त्रिपुरं तेषु चाप्यासन्राजानो वै पृथक्पृथक्}
{दिव्यमाल्याम्बरधरा दैतेया राजसत्तम}


\twolineshloka
{काञ्चनं तारकाक्षस्य दिव्यमासीन्महात्मनः}
{राजतं कमलाक्षस्य विद्युन्मालिन आयसम्}


\twolineshloka
{त्रयस्ते दैत्यराजानस्त्रीँल्लोकानाशु तेजसा}
{आक्रम्य तस्थुरूचुश्च कश्च नाम प्रजापतिः}


\twolineshloka
{तेषां दानवमुख्यानां प्रयुतान्यर्बुदानि च}
{कोट्याश्चाप्रतिवीराणां समाजग्मुस्ततस्ततः}


\twolineshloka
{मांसाशिनः सुदृप्ताश्च सुरैर्विनिकृताः पुरा}
{महदैश्वर्यमिच्छन्तस्त्रिपुरं दुर्गमाश्रिताः}


\twolineshloka
{सर्वेषां च पुनश्चैषां सर्वयोगवहो मयः}
{तमाश्रित्य हि ते सर्वेऽवर्तयन्नकुतोभयाः}


\twolineshloka
{यो हि यं मनसा कामं दध्यौ त्रिपुरसंश्रयः}
{तस्मै तस्मै मयस्तं तं विदधे मायया तदा}


\twolineshloka
{तारकाक्षसुतश्चासीद्वरिर्नाम महाबलः}
{तपस्तेपे परमकं येनातुष्यत्पितामहः}


\twolineshloka
{सन्तुष्टमवृणोद्देवं वापी भवतुः नः पुरे}
{शस्त्रैर्विनिहता यत्र क्षिप्ताः स्युर्बलवत्तराः}


\twolineshloka
{स तु लब्ध्वा वरं वीरस्तारकाक्षसुतो हरिः}
{ससृजे तत्र वापीं तां मृतसञ्जीविनीं प्रभो}


\threelineshloka
{येन रूपेण यो दैत्यो येन वेषेण चाप्यथ}
{क्षिप्यते निहतो वाप्यां तादृशेनैव जायते}
{सम्पूर्णबलवीर्यस्तु राजञ्छौर्यसमन्वितः}


\twolineshloka
{एवं वीर्येण संयुक्तां कृतां तेन महात्मना}
{तां प्राप्य त्रैपुरा वापीं लोकान्सर्वान्बबाधिरे}


\threelineshloka
{महता तपसा सिद्धाः सुराणां भयवर्धनाः}
{एकस्मिन्निहते दैत्ये सृजन्ति स्म दशासुरान्}
{न तेषां विद्यते युद्वे क्षयो राजन्कथञ्चन}


\twolineshloka
{ततस्ते लोभमोहाभ्यामभिभूता विचेतसः}
{निर्भीकाः सहिताः सर्वे स्थापिताः समलोलुपाः}


\twolineshloka
{विद्राव्य सगणान्देवांस्तत्रतत्र तदातदा}
{विचेरुः स्वेन कामेन वरदानेन दर्पिताः}


\threelineshloka
{देवोद्यानानि सर्वाणि स्थानानि च दिवौकसाम्}
{ऋषीणामाश्रमान्पुण्यान्रम्याञ्चनपदांस्तथा}
{उत्सादयन्त मर्यादां दानवा दुष्टचारिणः}


\twolineshloka
{निःस्थानाश्च कृता देवा ऋषयः पितृभिः सह}
{दैत्यैस्त्रिभिस्त्रयो लोका ह्याक्रान्तास्तैः सुरेतरैः}


\twolineshloka
{पीड्यमानेषु लोकेषु ततः शक्रो मरुद्वृतः}
{पुराण्यायोधयाञ्चक्रे वज्रहस्तः समन्ततः}


\twolineshloka
{नाशकत्तान्यभेद्यानि यदा भेत्तुं पुरन्दरः}
{पुराणि वरदत्तानि धात्रा तेन नराधिप}


\threelineshloka
{तदा भीतः सुरपतिर्मुक्त्वा तानि पुराण्यथ}
{तैरेव विबुधैः सार्धं पितामहमरिन्दम}
{जगामाथ तदाख्यातुं विप्रकारं सुरेतरैः}


\twolineshloka
{ते तत्त्वं सर्वमाख्याय शिरोभिः सम्प्रणम्य च}
{तद्वधोपायमाचक्ष्व भगवन्निति चाब्रुवन्}


\twolineshloka
{श्रुत्वा तद्भगवान्देवो देवानिदमुवाच ह}
{श्रूयतां त्रिदशाः सर्वे यथेदं वाक्यगौरवम्}


\twolineshloka
{दुरात्मानोऽसुरा नित्यं ते चापि विबुधा मम}
{न शक्नुवन्ति ते (ये) सर्वे युष्मान्वै पीडयन्ति ते}


\twolineshloka
{अहं समस्तु सर्वेषां भूतानां नात्र संशयः}
{अविनीता निहन्तव्या इत्येवं प्रब्रवीमि वः}


\twolineshloka
{एकेषुणा (हि) विभेद्यानि तानि दुर्गाणि नान्यथा}
{शक्तस्तु तानि बाणेन भेत्तुं कामं त्रिलोचनः}


\twolineshloka
{ते यूयं स्थाणुमीशानं जिष्मुमक्लिष्टकारिणम्}
{योद्वारं वृणुत क्षिप्रं स तान्हन्ता सुरेतरान्}


\threelineshloka
{ते देवास्तेन वाक्येन चोदिताः प्रणताः स्थिताः}
{दिव्यं वर्षसहस्रं वै तपस्तप्त्वा सुरर्षभाः}
{शुभात्मानो महात्मानो जग्मुर्वै वृषभध्वजम्}


\twolineshloka
{ब्रह्माणमग्रतः कृत्वा शरण्यं शरणागताः}
{तपः परममाजग्मुर्गृणन्तो ब्रह्म शाश्वतम्}


\twolineshloka
{अनङ्गमथनं सर्वे भवं सर्वात्मना गताः}
{देवदेवं परं स्थाणुं वरदं त्र्यम्बकं शिवम्}


\threelineshloka
{शर्वमीड्यमजं रुद्रं शशाङ्काङ्कितमूर्धजम्}
{तुष्टुवुर्वाग्भिरुग्राभिर्भयेष्वभयमच्युतम्}
{सर्वात्मानं महात्मानं येनाप्तं विश्वमात्मना}


\twolineshloka
{तपोविशेषैर्विविधैर्योगं यो वेद चात्मनः}
{यः साङ्ख्यमात्मना वेत्ति यस्य चात्मा वेशे सदा}


\threelineshloka
{तं ते ददृशुरीशानं तेजोराशिमुमापतिम्}
{`परेण यत्नेन भवं त्रिदशाः शर्वमीश्वरम्'}
{अनन्यसदृशं लोके प्रतपन्तमकल्मषम्}


\threelineshloka
{एकश्च भगवांस्तत्र नानारूपमकल्पयत्}
{आत्मनः प्रतिरूपाणि रूपाण्यथ महात्मनि}
{परस्परस्य चापश्यन्सर्वे परमविस्मिताः}


\twolineshloka
{सर्वभूतमयं दृष्ट्वा तमजं जगतः परिम्}
{देवा ब्रह्मर्षयश्चैव शिरोभिर्धरणीं गताः}


\twolineshloka
{तान्स्वस्तिवाच्य चाभ्यर्च्य समुत्थाप्य च शङ्करः}
{ब्रूतब्रूतेति भगवान्स्मयमानोऽभ्यभाषत}


\twolineshloka
{त्र्यम्बकेणाभ्यनुज्ञातास्ततस्ते स्वस्थचेतसः}
{नमो नमो नमस्तेऽस्तु प्रभो इत्यब्रुवन्भवम्}


\twolineshloka
{नमो देवाधिदेवाय प्रियाधाघ्नेऽतिमन्यवे}
{प्रजापतिमखघ्नाय प्रजापतिभिरीड्यते}


\twolineshloka
{नमः स्तुताय स्तुत्याय स्तूयमानाय शम्भवे}
{विलोहिताय रुद्राय नीलग्रीवाय शूलिने}


\twolineshloka
{अमोघाय मृगाक्षाय प्रवरायुधयोधिने}
{अर्हाय चैव शुद्वाय क्षयाय क्रथनाय च}


\twolineshloka
{दुर्वारणाय शुक्राय ब्रह्मणे ब्रह्मचारिणे}
{ईशानायाप्रमेयाय निहन्त्रे चर्मवाससे}


\twolineshloka
{तपोरताय पिङ्गाय व्रतिने कृत्तिवाससे}
{कुमारपित्रे त्र्यक्षाय प्रवरायुधधारिणे}


\twolineshloka
{प्रपन्नार्तिविनाशाय ब्रह्मद्विट््सङ्घघातिने}
{वनस्पतीनां पतये वनानां पतये नमः}


\threelineshloka
{गवां च पतये नित्यं यज्ञानां पतये नमः}
{नमोस्तु ते ससैन्याय त्र्यम्बकायामितौजसे}
{मनोवाक्कर्मभिर्देव त्वां प्रपन्नान्भजस्व नः}


\twolineshloka
{ततः प्रसन्नो भगवान्स्वागतेनाभिनन्द्य च}
{प्रोवाच व्येतु वस्त्रासो ब्रूत किं करवाणि वः}


% Check verse!
`देवाः शर्वस्य वचनं श्रुत्वा हर्षमुपागताः'
\chapter{अध्यायः २९}
\twolineshloka
{दुर्योधन उवाच}
{}


\twolineshloka
{पितृदेवर्षिसङ्घेभ्योऽभये दत्ते महात्मना}
{सत्कृत्य शंकरं प्राह ब्रह्मा लोकपितामहः}


\twolineshloka
{इमान्यसुरदुर्गाणि लोकांस्त्रीनाक्रमन्ति हि}
{कश्च प्रजापतिर्घोरांस्तान्स्रर्वाञ्जहि मा चिरम्}


\twolineshloka
{वरातिसर्गाद्देवेश प्राजापत्यमिदं परम्}
{मयाऽधितिष्ठता दत्तो दानवेभ्यो महान्वरः}


\threelineshloka
{नास्त्यन्यो युधि तेषां वै निहन्ता इति नः श्रुतम्}
{तानतिक्रान्तमर्यादान्नान्यः संहर्तुमर्हति}
{त्वामृते सर्वभूतेश त्वं ह्येषां प्रत्यरिर्वधे}


\twolineshloka
{स त्वं देव प्रपन्नानां याचतां च दिवोकसाम्}
{कुरु प्रसादं वरद दानवाञ्जहि संयुगे}


\threelineshloka
{त्वत्प्रसादाज्जगत्सर्वं सुखमैधत मानद}
{शरण्यस्त्वं हि देवानां वयं त्वां शरणं गताः ॥ईश्वर उवाच}
{}


\twolineshloka
{हन्तव्याः शत्रवः सर्वे युष्माकमिति मे मतिः}
{न त्वेकश्चोत्सहे हन्तुं बलिनः सुरविद्विषः}


\threelineshloka
{ते यूयं सहिताः सर्वे मदीयेनापि तेजसा}
{जयध्वं युधि तान्सर्वान्सङ्घातेन महाबलान् ॥देवा ऊचुः}
{}


\threelineshloka
{अस्मत्तेजोबलं यावत्तावद्द्विगुणमेव वा}
{तत्तेषामिति मन्यामो दृष्टतेजोबला हि ते ॥ईश्वर उवाच}
{}


\threelineshloka
{वध्यास्ते सर्वथा पापा ये युष्मास्वपराधिनः}
{मम तेजोबलार्धेन सर्वे मृद्गथ शात्रवान् ॥देवा ऊचुः}
{}


\twolineshloka
{बिभर्तुं तव तेजोर्धं न शक्ष्याम सुरेश्वर}
{सर्वेषां नो बलार्धेन त्वमेव जहि शात्रवान्}


\threelineshloka
{वयं च सर्वथा देव रक्षणीयास्तथैव च}
{स नो रक्ष महादेव त्वमेव जहि शात्रवान् ॥ईश्वर उवाच}
{}


\twolineshloka
{मम तेजो न शक्ता हि सर्वे धारयितुं यदि}
{अहमेनान्वधिष्यामि युष्मत्तेजोर्धसंयुतः}


\twolineshloka
{बलार्धं यदि मे देवा न धारयितुमाहवे}
{शक्ताः सर्वे हि सङ्गम्य यूयं तत्प्रब्रवीमि वः}


\twolineshloka
{समा भवन्ति मे सर्वे दानवाश्चामराश्च ये}
{शिवोऽस्मि सर्वभूतानां शिवत्वं तेन मे सुराः}


\twolineshloka
{किन्त्वधर्मेण वर्तन्ते यस्मात्ते सुरशत्रवः}
{तस्माद्वध्या मयाप्येते युष्माकं च हितेप्सया}


\twolineshloka
{शरणं वः प्रपन्नानां धर्मेण च जिगीषताम्}
{साहाय्यं वः करिष्यामि निहनिष्यामि वो रिपून्}


\twolineshloka
{दीयतां च बलार्धं मे सर्वैरपि पृथक्पृथक्}
{पशुत्वं चैव मे लोकाः सर्वे कल्पन्तु पीडिताः}


\twolineshloka
{पशूनां तु पतित्वं मे भवत्वद्य दिवौकसः}
{एवं न पापं प्राप्स्यामि पशून्हत्वा सुरद्विपः}


\twolineshloka
{कल्पयध्वं रथं दिव्यं रथाश्वांश्चैव पारगान्}
{धनुः शरं सारथिं च ततो जेष्यामि वो रिपून्}


\twolineshloka
{इति श्रुत्वा वचो देवा देवदेवस्य भूपतेः}
{विषादमगमन्सर्वे पशुत्वं प्रति शङ्किताः}


\twolineshloka
{तेषां भावं भवो ज्ञात्वा देवस्तानब्रवीदिदम्}
{मा वोस्तु पशुभावेऽस्मिन्भयं विबुधसत्तमाः}


% Check verse!
श्रूयतां पशुभावस्य विमोक्षः क्रियतां च सः
\twolineshloka
{यो वः पशुपतेश्चर्यां चरिष्यति स मोक्ष्यते}
{पशुत्वादिति सत्यं वः प्रतिजाने समागमे}


% Check verse!
ये चाप्यन्ये चरिष्यन्ति व्रतं मोक्ष्यन्ति तेऽपि च
\twolineshloka
{नै(त्य)ष्ठिकं द्वादशाब्दं वा अर्धमब्दमृतुत्रयम्}
{मासं द्वादशरात्रं गुह्यं व्रतं दिव्यं चरिष्यथ}


\twolineshloka
{तं तथेत्यब्रुवन्देवा देवदेवनमस्कृतम्}
{ऊचुश्चेदं गृहाणेदं तेजसोऽर्धमिति प्रभुम्}


\twolineshloka
{प्रत्युवाच तथेत्येव शूलधृद्राजसत्तम}
{ततस्ते प्रददुः सर्वे तेजसोऽर्धं महात्मने}


\twolineshloka
{सर्वमादाय सर्वेषां तेजसोऽर्धं दिवौकसाम्}
{तेजसाप्यधिको भूत्वा भूयोऽप्यतिबलोऽभवत्}


\twolineshloka
{ततः प्रभृति देवानां देवदेवोऽभवद्भवः}
{पतिश्च सर्वभूतानां पशूनां चाभवत्तदा}


\twolineshloka
{तस्मात्पशुपतिश्चोक्तो भवत्वाच्च भवेति वै ॥अर्धमादाय सर्वेषां तेजसा प्रज्वलन्निव}
{}


\twolineshloka
{भासयामास तान्सर्वान्देवदेवो महाद्युतिः ॥ततोऽभिषिषिच्युः सर्वे सुरा रुद्रं पुरारिणम्}
{}


% Check verse!
महादेव इति ह्यासीद्देवदेवो महेश्वरः
\chapter{अध्यायः ३०}
\twolineshloka
{दुर्योधन उवाच}
{}


\twolineshloka
{तेजसोऽर्धं सुरा दत्त्वा शङ्कराय महात्मने}
{पशुत्वमपि चोपेत्य विश्वकर्माणमव्ययम्}


\twolineshloka
{ऊचुः सर्वे समाभाष्य रथः सङ्कल्प्यतामिति}
{विश्वकर्माऽपि सञ्चिन्त्य रथं दिव्यमकल्पयत्}


\twolineshloka
{समेतां पृथिवीं देवी विशालां पुरमालिनीम्}
{सपर्वतवनद्वीपां चक्रे भूतधरां रथम्}


\twolineshloka
{ईषां नक्षत्रवंशं च छत्रं मेरुमहागिरिम्}
{अनेकद्रुसञ्छन्नं रत्नाकरमनुत्तमम्}


\twolineshloka
{हिमवन्तं च विन्ध्यं च नानाद्रुमलताकुलम्}
{अवस्करं प्रतिष्ठानं कल्पयामास वै तदा}


\twolineshloka
{अस्तंगिरिमधिष्ठानं नानाद्विजगणायुतम्}
{चकार भगवांस्त्वष्टा उदयं रथकूबरम्}


\twolineshloka
{मीननक्रझषावासं दानवालयमुत्तमम्}
{समुद्रमक्षं विदधे पत्तनाकरशोभितम्}


\twolineshloka
{चक्रं चक्रे चन्द्रमसं तारकागणमण्डितम्}
{दिवाकरं चाप्यपरं चक्रं चक्रेंऽशुमालिनम्}


\twolineshloka
{गङ्गां सरस्वतीं तूणीं चक्रे विश्वकृदव्ययः}
{अलङ्कारा रथस्यासन्नापगाः सरितस्तथा}


\twolineshloka
{त्रीनग्नीन्मन्त्रवच्चक्रे रथस्याथ त्रिवेणुकम्}
{अनुकर्षान्रथे दीप्तान्वरूथांश्चापि तारकाः}


\twolineshloka
{ओषधीर्वीरुधश्चैव घण्टाजालं च भानुमत्}
{अलञ्चकार च रथं मासपक्षर्तुभिर्विभुः}


\twolineshloka
{अहोरात्रैः कलाभिश्च काष्ठाभिरयनैस्तथा}
{द्यां युगं युगपर्वाणि संवर्तकबलाहरकान्}


\twolineshloka
{शम्यां धृतिं च मेधां च स्थितिं सन्नतिमेव च}
{ऋघ्वेदं सामवेदं च धुर्यावश्वावकल्पयत्}


\twolineshloka
{पृष्ठाश्वौ तु यजुर्वेदः कल्पितोऽथर्वणस्तथा}
{अश्वानां चाप्यलङ्कारं विदधे पदसञ्चयम्}


\threelineshloka
{सिनीवालीमनुमतिं कुहूं राकां च सुप्रभाम्}
{योक्ताणि चक्रे चाश्वानां कूश्माण्डांश्चापि पन्नगान्}
{}


\twolineshloka
{तालपृष्ठोऽथ नहुषः कार्कोटकधनञ्जयौ}
{इतरे चाभवन्नागा हयानां वाहबन्धनम्}


\twolineshloka
{अभीशवः षडङ्गानि कल्पितानि महीपते}
{ओङ्कारः कल्पितस्तस्य प्रतोदो विश्वकर्मणा}


\twolineshloka
{यज्ञाः सर्वे पृथक्लृप्ता रथाङ्गानि च भागशः}
{अधिष्ठानं मनश्चासीत्परिरथ्या सरस्वती}


\twolineshloka
{नानावर्णानि शस्त्राणि पताकाः पवनेरिताः}
{विद्युदिन्द्रधनुर्युक्तं रथं दीप्त्या व्यदीपयत्}


\twolineshloka
{वर्म योद्धुं च विहितं नमो ग्रहगणाकुलम्}
{अभेद्यं भानुमच्चित्रं कालचक्रपरिक्षतम्}


\twolineshloka
{एवमस्मिन्महाराज कल्पिते रथसत्तमे}
{त्वष्ट्रा मनुजशार्दूल द्विषतां भयवर्धने}


\twolineshloka
{स्वान्यायुधानि दिव्यानि न्यदधाच्छङ्करो रथे}
{ध्वजयष्टिं वियत्कृत्वा स्थापयामास गोवृषम्}


\twolineshloka
{ब्रह्मदण़्डः कालदम्डो रुद्रदण्डश्च ते ज्वराः}
{परिष्कारा र्थस्यासन्समन्ताद्दिशमुद्धताः}


\twolineshloka
{विचित्रमृतुभिः षड्भिः कृत्वा संवत्सरं धनुः}
{छायामेवात्मनश्चक्रे धनुर्ज्यामक्षयां ध्रुवाम्}


\twolineshloka
{कालो हि भगवान्रुद्रस्तच्च संवत्सरं धनुः}
{तस्माद्रौद्री कालरात्री ज्या कृता धनुषो जरा}


\twolineshloka
{ततो रथे रथाश्वांस्तानृषयः समयोजयन्}
{एकैकशः सुसंहृष्टानादाय सुधृतव्रताः}


\twolineshloka
{दक्षिणस्यां धुरि कृत ऋग्वेदो मन्त्रपारगैः}
{सव्यतः सामवेदश्च युक्तो राजन्महर्षिभिः}


\twolineshloka
{पार्ष्ठिदक्षिणतो युक्तो यजुर्वेदः सुरद्विजैः}
{इतरस्यां तथा पार्ष्ठ्यां युक्तो राजन्नथर्वणः}


\twolineshloka
{एवं ते वाजिनो युक्ता यज्ञविद्भिस्तथा रथे}
{अशोभन्त तथा युक्ता यथैवाध्वरमध्यगाः}


\twolineshloka
{कल्पयित्वा रथं दिव्यं ततो बाणमकल्पयत्}
{चिन्तयित्वा हरिं विष्णुमव्ययं यज्ञवाहनम्}


\twolineshloka
{शरं सङ्कल्पयाञ्चक्रे विश्वकर्मा महामनाः}
{तस्य वाजांश्च पुङ्खं च कल्पयामास वै तदा}


\twolineshloka
{पुण्यगन्धवहं राजञ्श्वसनं राजसत्तम}
{अग्नीषोमौ शरमुखे कल्पयामास वै तदा}


\twolineshloka
{अग्नीषोमात्मकं कृत्स्नमुच्यते वैष्णवं जगत्}
{विष्णुरात्मा भगवतो भवस्यामिततेजसः}


\twolineshloka
{तस्माद्धनुर्ज्यासंस्पर्शं स विषेहे शरस्य वै}
{तस्मिञ्शरे तीक्ष्णमन्युममुञ्चद्दुःसहं प्रभुः}


\twolineshloka
{भृग्वङ्गिरोमन्युभवः क्रोधाग्निरतिदुःसहः}
{स नीललोहितो धूम्रः कृत्तिवासा भयानकः}


\twolineshloka
{आदित्यायुतसङ्काशस्तेजोज्वालावृतो भवः}
{दुश्चर्यच्यावको जेता हन्ता ब्रह्मद्विषां वरः}


\twolineshloka
{तस्याङ्गानि समाश्रित्य स्थितं विश्वमिदं जगत्}
{जङ्गमाजङ्गमं राजञ्छुशुभेऽद्भुतदर्शनम्}


\twolineshloka
{दृष्टा तु तं रथं दिव्यं कवची स शरासनी}
{आददे स शरं दिव्यं सोमविष्ण्वग्निवायुजम्}


\twolineshloka
{तमादाय महादेवस्त्रासयन्दैत्यदानवान्}
{आरुरोह तदा यत्तः कम्पयन्निव रोदसी}


\twolineshloka
{महर्षिभिः स्तूयमानो वन्द्यमानश्च वन्दिभिः}
{उपनृत्तश्चाप्सरसां गणैर्नृत्तविशारदैः}


\twolineshloka
{स शोभमानो वरदः खङ्गी बाणी शरासनी}
{हसन्निवाब्रवीद्देवः सारथिः को भवेदिति}


\chapter{अध्यायः ३१}
\twolineshloka
{दुर्योधन उवाच}
{}


\twolineshloka
{तमब्रुवन्देवगणा यं भवान्सन्नियोक्ष्यते}
{स भविष्यति देवेश सारथिस्ते न संशयः}


\twolineshloka
{तानब्रवीन्महादेवो मत्तः श्रेष्ठतरो हि यः}
{तं सारथिं कुरुध्वं वै स्वयं सञ्चिन्त्य मा चिरम्}


\twolineshloka
{एतच्छ्रुत्वा वचो देवाः सर्वे गत्वा पितामहम्}
{प्रणिपत्योचुरेकाग्राः प्रसाद्यैनं महर्षिभिः}


\twolineshloka
{त्वया यत्कथितं देव त्रिदशारिनिबर्हणे}
{तथा तत्कृतमस्माभिः प्रसन्नश्च वृषध्वजः}


\twolineshloka
{रथश्च विहितोऽस्माभिर्विचित्रायुधसंवृतः}
{सारथिं च न जानीमः कः स्यात्तस्मिन्रथोत्तमे}


\threelineshloka
{तस्माद्विधीयतां कश्चित्सारथिर्देवसत्तम}
{सफलां तां गिरं देव कर्तुमर्हसि नो विभो}
{एवमस्मासु हि पुरा भगवन्नुक्तवानसि}


\twolineshloka
{सदैव युक्तो रथसत्तमो वैदुराधर्षो द्रावणः शात्रवाणाम्}
{पिनाकधन्वा विहितोऽत्र योद्धाविभीषयन्दानवानुद्यतोऽसौ}


\twolineshloka
{तथैव वेदाश्च हया रथाग्र्यधरा सशैला च रथो महात्मनः}
{नक्षत्रवंशानुगतो वरूथीयस्मिन्योद्धा सारथिनाऽभिरक्ष्यः}


% Check verse!
तत्र सारथिरेष्टव्यः सर्वैरेतैर्विशेषवान्
\twolineshloka
{तं प्रविष्टा रथं देवा रथयोद्वारमेव च}
{कवचानि च शस्त्राणि कार्मुकं च पितामह}


\twolineshloka
{त्वामृते सारथिं तत्र नान्यं पश्यामहे वयम्}
{त्वं हि सर्वैर्गणैर्युक्तोदेवताभ्योऽधिकः प्रभो}


\twolineshloka
{त्वं देव शक्तो लोकेश नियन्तुं प्रद्रुतानिमान्}
{वेदांश्च सोपनिषदः सारथिर्भव नः स्वयम्}


\twolineshloka
{योद्वुं बलेन वीर्येण सत्वेन विनयेन च}
{अधिकःसारथिःकार्यो नास्ति चान्योऽधिको भवात्}


\twolineshloka
{स भवांस्तारयत्वस्मान्कुरु सारथ्यमव्यय}
{भवानभ्यधिकस्त्वत्तो नान्योस्ति भविता त्विह}


\twolineshloka
{त्वं हि देवेश सर्वैस्तु विशिष्टो वदतां वर}
{तं रथं त्वं समारुह्य संयच्छ परमान्हयान्}


\twolineshloka
{तव प्रसादाद्वध्येयुर्देव दैवतकण्टकाः}
{स नो रक्ष महाबाहो दैत्येभ्यो महतो भयात्}


\twolineshloka
{त्वं हि नो गतिरव्यग्र तवं न्त्रे गोप्ता महाव्रत}
{त्वत्प्रसादात्सुराः सर्वे पूज्यन्ते त्रिदिवे प्रभो}


\threelineshloka
{इति ते शिरसाऽगच्छंस्त्रिलोकेशं पितामहम्}
{देवाः प्रसादयामासुः सारथ्यायेति नः श्रुतम् ॥ब्रह्मोवाच}
{}


\twolineshloka
{एवमेतत्सुरास्तथ्यं नान्यस्त्वभ्यधिको भवात्}
{सारथित्वं करिष्यामि शङ्करस्य महात्मनः}


\fourlineindentedshloka
{सर्वथा रथिनः श्रेयान्कर्तव्यो रथसारथिः}
{तस्मादेतद्यथातत्त्वं ज्ञात्वा युष्मांश्च सङ्गतान्}
{संयच्छामि हयानेष विबुधाय कपर्दिने ॥दुर्योधन उवाच}
{}


\threelineshloka
{एवमुक्त्वा जटाभारं संयम्य प्रपितामहः}
{परिधायाजिनं गाढं सन्न्यस्य च कमण्डलुम्}
{प्रतोदपाणिर्भगवानारुरोह रथं तदा}


\threelineshloka
{सारथौ कल्पिते देवैरीशानस्य महात्मनः}
{तस्मिन्नारोहति रथं कल्पितं लोकसम्भृतम्}
{शिरोभिः पतिता भूमौ तुरगा वेदसम्भृताः}


\twolineshloka
{उभाभ्यां लोकनाथाभ्यामास्थितं रथसत्तमम्}
{ओढुं न शक्ता वेदाश्वा जानुभ्यामपतन्महीम्}


\threelineshloka
{अभीशुभिस्तु भगवानुद्यम्य च हयान्विभुः}
{अस्तु वीर्यं च शौर्यं च वेदाश्वानामिति प्रभुः}
{रथं सञ्चोदयामास देवानां प्रभुरव्ययः}


\twolineshloka
{ततोऽधिरूढे वरदे रथं पशुपतिस्तदा}
{साधुसाध्विति देवेशं स्मयमानोऽभ्यभाषत}


\twolineshloka
{याहि देव यतो दैत्याश्चोदयाश्वानरिन्दम}
{पश्य बाह्वोर्बलं मेऽद्य निघ्नतः शात्रवान्रणे}


\twolineshloka
{ततोऽश्वांश्चोदयामास मनोमारुतरंहसः}
{पुराण्युद्दिश्य खस्थानि दानवानां तरस्विनाम्}


\twolineshloka
{ततस्ते सहसोत्पत्य वेदाख्या रथवाजिनः}
{क्षणेन तेन दैत्यानां पुराणि प्रापयन्हरम्}


\threelineshloka
{अथर्वाङ्गिरसौ चास्तां चक्ररक्षौ महात्मनः}
{अथाधिज्यं धनुः कृत्वा शर्वः सन्धाय तं शरम्}
{युक्त्वा पाशुपतास्त्रेण त्रिपुरं समचिन्तयत्}


\twolineshloka
{तस्मिन्स्थिते ततो राजन्रुद्रे सज्जितकार्मुके}
{पुराणि तेन कालेन जग्मुरेकत्वमाशु वै}


\twolineshloka
{एकीभावं गते चैव त्रिपुरत्वमुपागते}
{बभूव तुमुलो हर्षो देवतानां महात्मनाम्}


\twolineshloka
{ततो देवगुणाः सर्वे सिद्धाश्च परमर्षयः}
{जयेति वाचो मुमुचुः संस्तुवन्तो महेश्वरम्}


\twolineshloka
{ततोऽग्रतः प्रादुरभूत्त्रिपुरं जघ्नुषोऽसुरान्}
{अनिर्देश्याग्र्यवपुषो देवस्यासह्यतेजसः}


\threelineshloka
{त्रीणि दृष्ट्वैव संस्थानि पुराण्यथ पिनाकधृत्}
{स तद्विकृष्य भगवान्दिव्यं लोकेश्वरो धनुः}
{त्रैलोक्यसारं तमिषुं मुमोच त्रिपुरं प्रति}


\twolineshloka
{एकबाणेन तं देवस्त्रिपुरं परमेश्वरः}
{निजघ्ने सासुरगणं देवदेवो महेश्वरः}


\threelineshloka
{बाणतेजोग्निदग्धं तद्विप्रकीर्णं सहस्रधा}
{महदार्तस्वरं कृत्वा नावशेषमुपागतम्}
{मद्रेश सासुरगणं प्रापतत्पश्चिमार्णवे}


\twolineshloka
{एवं हि त्रिपुरं दग्धं दानवाश्चाप्यशेषतः}
{महेश्वरेण क्रुद्धेन त्रैलोक्यस्य हितैषिणा}


\twolineshloka
{स चात्मक्रोधजो विह्निर्दहेत्युक्तो निवारितः}
{त्रैलोक्यमविशेषेण पुनर्दग्धुं प्रचक्रमे}


\twolineshloka
{कालाग्निमिव निर्दग्धुमुत्थितं तं पुनः पुनः}
{माकार्षीर्भस्मासाल्लोकानिति त्र्यक्षोऽब्रवीद्वचः}


\twolineshloka
{ततः प्रकृतिमापन्ना देलोकास्तथर्षयः}
{तुष्टुवुर्वाग्भिरग्र्याभिः स्थाणुं त्रिपुरवैरिणम्}


\twolineshloka
{तेऽनुज्ञाता भगवता सर्वे जग्मुर्यथागतम्}
{कृतकामाः प्रसन्नेन प्रजापतिमुखाः सुराः}


\twolineshloka
{एवं रुद्रस्य कृतवान्सारथ्यं तु पितामहः}
{संयच्छ तुरगानस्य राधेयस्य महात्मनः}


\twolineshloka
{त्वं हि कृष्णाच्च कर्णाच्च फल्गुनाच्च गुणाधिकः}
{बलतो रूपतो योगादस्त्रसम्पद एव च}


\threelineshloka
{समासक्तं महीपाल कुरु मे हितमीप्सितम्}
{युद्वे ह्ययं रुद्रकल्पस्त्वं च ब्रह्मसमोऽनघ}
{तस्माच्छक्तौ युवां जेतुं मच्छत्रून्दिवि वा सुरान्}


\twolineshloka
{स यथा शल्य कर्णोऽयं श्वंताश्वं कृष्णसारथिम्}
{प्रमथ्य हन्यात्कौन्तेयं तथा नीतिर्विधीयताम्}


\twolineshloka
{त्वयि राज्यं सुखं चैव जीवितं जयमेव च}
{समासक्तं महीपाल कुरु मे हितमीप्सितम्}


% Check verse!
संयच्छास्य हयान्त्राजन्मत्प्रियार्थं परन्तप
\chapter{अध्यायः ३२}
\twolineshloka
{दुर्योधन उवाच}
{}


\twolineshloka
{इमं चाप्यपरं भूय इति हासं निबोध मे}
{पितुर्मम सकाशे वै ब्राह्मणः प्राह धर्मवित्}


\twolineshloka
{श्रुत्वा चैतद्वचश्चित्रं हेतुकार्यार्थतत्त्ववित्}
{कुरु शल्य विनिश्चित्य मा भूदत्र विचारणा}


\twolineshloka
{भार्गवाणां कुले जातो जमदग्निर्महातपाः}
{तस्य रामेति विख्यातः पुत्रस्तेजोगुणान्वितः}


\twolineshloka
{स तीव्रं तप आस्थाय सम्प्रसादितवान्भवम्}
{अस्त्रहेतोः प्रसन्नात्मा नियतः संयतेन्द्रियः}


\fourlineindentedshloka
{तस्मै तुष्टो महादेवो भक्त्या च प्रशमेन च}
{हृद्गतं चास्य विज्ञाय दर्शयामास शङ्करः}
{प्रत्यक्षेण महादेवः स्वां तनुं सर्वशङ्करः ॥शङ्कर उवाच}
{}


\twolineshloka
{राम तुष्टोऽस्मि भद्रं ते विदितं तव चेप्सितम्}
{कुरुष्व पूतमात्मानं सर्वमेतदवाप्स्यसि}


\twolineshloka
{दास्यामि ते तदाऽस्त्राणि यदा पूतो भविष्यसि}
{अपात्रमसमर्थं च दहन्त्यस्त्राणि भार्गव}


\twolineshloka
{इत्युक्तो जामदग्न्यस्तु देवदेवेन शूलिना}
{प्रत्युवाच महात्मानं शिरसाऽवनतः प्रभुम्}


\twolineshloka
{यदा जानासि देवेश पात्रं मामस्त्रधारणे}
{तदा शुश्रूषतेऽस्त्राणि भवान्मे दातुमर्हति}


\twolineshloka
{ततः स तपसा चैव व्रतेन नियमेन च}
{पूजोपहारबलिभिर्होममन्त्रपुरस्कृतैः}


\twolineshloka
{समाराधितवाञ्शर्वं बहुवर्षगणांस्तदा}
{प्रसन्नश्च महादेवो भार्गवस्य महात्मनः}


\twolineshloka
{असकृच्चाब्रवीत्तस्य गुणान्देव्याः सकाशतः}
{भक्तिमानेष सततं मयि रामो दृढव्रतः}


\twolineshloka
{एवमस्य गुणान्प्रीतो बहुशो कथयद्विभुः}
{देवतानां पितॄणां च समक्षमरिसूदन}


\twolineshloka
{एतस्मिन्नेव काले तु दैत्या ह्यासन्महाबलाः}
{तैस्तदा दर्पमोहाद्वै अबाध्यन्त दिवौकसः}


\twolineshloka
{ततः सम्भूय विबुधास्तान्हन्तुं कृतनिश्चयाः}
{चक्रुः शत्रुवधे यत्नं दैत्याञ्चेतुमशक्नुवन्}


\twolineshloka
{अभिगम्य ततो देवा महेश्वरमुमापतिम्}
{प्रासादयंस्तदा भक्त्या जहि शत्रुगणानिति}


\twolineshloka
{प्रतिज्ञाय ततो देवो देवतानां रिपुक्षयम्}
{रामं भार्गवमाहूय सोऽभ्यभाषत शङ्करः}


\threelineshloka
{रिपून्भार्गव देवानां जहि सर्वान्समागतान्}
{लोकानां हितकामार्थं मत्प्रियार्थं तथैव च ॥परशुराम उवाच}
{}


\threelineshloka
{का शक्तिर्मम देवेश अकृतास्त्रस्य संयुगे}
{निहन्तुं दानवान्सर्वान्कृतास्त्रान्युद्धदुर्मदान् ॥ईश्वर उवाच}
{}


\threelineshloka
{गच्छ त्वं मदनुज्ञानान्निहनिष्यसि शात्रवान्}
{विजित्य च रिपून्सर्वान्गुणान्प्राप्स्यसि पुष्कलान् ॥दुर्योधन उवाच}
{}


\twolineshloka
{एतच्छ्रुत्वा तु वचनं प्रतिगृह्य च सर्वशः}
{रामः कृतस्वस्त्ययनः प्रययौ दानवान्प्रति}


\twolineshloka
{ततोऽजयद्देवशत्रून्महादर्पबलान्वितान्}
{वज्राशनिसमस्पर्शैः प्रहारैरेव भार्गवः}


\twolineshloka
{स दानवैः क्षततनुर्जामदग्न्यो नृपोत्तम}
{संस्पृष्टः स्थाणुना सद्यो निर्व्रणः समपद्यत}


\twolineshloka
{प्रीतश्च भगवान्देवः कर्मणा तेन तस्य वै}
{वरान्प्रादाद्बहुविधान्भार्गवाय महात्मने}


% Check verse!
उक्तश्च प्रीतियुक्तेन देवदेवेन शूलिना
\threelineshloka
{निपातात्तव शस्त्राणां शरीरे याऽभवद्रुजा}
{तया ते मानुषं कर्म व्यपोहं भृगुनन्दन}
{गृहाणास्त्राणि दिव्यानि मत्सकाशाद्यथेप्सितं}


\threelineshloka
{ततोऽस्त्राणि समग्राणि वरांश्च मनसेप्सितान्}
{लब्ध्वा बहुविधान्रामः प्रणम्यशिरसा शिवम्}
{अनुज्ञां प्राप्य देवेशाज्जगाम स महातपाः}


% Check verse!
एवमेतत्पुरावृत्तं तथा कथितवानृषिः
\twolineshloka
{भार्गवोऽपि ददौ सर्वं धनुर्वेदं महात्मने}
{कर्णाय पुरुषव्याघ्रः सुप्रीतेनान्तरात्मना}


\twolineshloka
{वृजिनं न भवेत्किञ्चिदस्य कर्णस्य पार्थिव}
{सूतेन वर्धितो नित्यं न सूतो नृप एव सः}


\threelineshloka
{विशुद्धयोनिं विज्ञाय दिव्यान्यस्त्राण्यदाद्भृगुः}
{नापि सूतकुले जातं मन्ये कर्णं कथञ्चन}
{देवपुत्रमदं मन्ये क्षत्रियाणां कुलोद्भवम्}


\twolineshloka
{विसृष्टमविबोधार्थं कुलस्येंति मतिर्मम}
{सर्वथा न ह्यसो शल्य कर्णः सूतकुलोद्भवः}


\twolineshloka
{सकुण्डलं सकवचं दीर्घबाहुमरिन्दमम्}
{कथमादित्यसङ्काशं मृगी सिंहं प्रसूयते}


\twolineshloka
{पश्य ह्यस्य भुजौ पीनौ नागराजकरोपमौ}
{वक्षः पश्य विशालं च सर्वशस्त्रसहं रणे}


\twolineshloka
{न त्वेव प्राकृतः कश्चित्कर्णो वैकर्तनो वृषा}
{महात्माह्येष राजेन्द्र रामशिष्यः प्रतापवान्}


\chapter{अध्यायः ३३}
\twolineshloka
{दुर्योधन उवाच}
{}


\twolineshloka
{एवं स भगवान्देवः सर्वलोकपितामहः}
{सारथ्यमकरोत्तत्र ब्रह्मा रुद्रोऽभवद्रथी}


\twolineshloka
{रथिनोऽभ्यधिको वीर कर्तव्यो रथसारथिः}
{तस्मात्त्वं पुरुषव्याघ्र नियच्छ तुरगान्युधि}


\twolineshloka
{यथा देवगणैस्तत्र वृतो यत्नात्पितामहः}
{तथाऽस्माभिर्भवान्यत्नात्कर्णादभ्यधिको वृतः}


\fourlineindentedshloka
{यथा देवैर्महाराज ईश्वरादधिको वृतः}
{तथा देवैर्महाराज क्षिप्रं रुद्रस्यव पितामहः}
{नियच्छ तुरगान्युद्वे राधेयस्य महाद्युते ॥शल्य उवाच}
{}


\twolineshloka
{मयाप्येतन्नरश्रेष्ठ बहुशो नरसिंहयोः}
{कथ्यमानं श्रुतं दिव्यमाख्यानमतिमानुषम्}


\twolineshloka
{यथा च चक्रे सारथ्यं भवस्य प्रपितामहः}
{यथाऽसुराश्च निहता इषुणैकेन भारत}


% Check verse!
कृष्णस्य चापि विदितं सर्वमेतत्पुरा ह्यभूत्
\twolineshloka
{यथा पितामहो जज्ञे भगवान्सारथिस्तदा}
{अनागतमतिक्रान्तं वेद कृष्णोऽपि तत्त्वतः}


\twolineshloka
{एतदर्थं विदित्वाऽपि सारथ्यमुपजग्मिवान्}
{स्वयम्भूरिव रुद्रस्य कृष्णः पार्थस्य भारत}


\twolineshloka
{यदि हन्याच्च कौन्तेयं सूतपुत्रः कथञ्चनः}
{दृष्ट्वा पार्थं हि निहतं स्वयं योत्स्यति केशवः}


\threelineshloka
{शङ्खचक्रगदापाणिर्धक्ष्यते तव वाहिनीम्}
{न चापि तस्य क्रुद्धस्य वार्ष्णेयस्य महात्मनः}
{स्थास्यते प्रत्यनीकेषु कश्चिदत्र नृपस्तव}


\threelineshloka
{सञ्जय उवाच}
{तं तथा भाषमाणं तु मद्रराजमरिन्दमः}
{प्रत्युवाच महाबाहुरदीनात्मा सुतस्तव}


\twolineshloka
{मावमंस्था महाबाहो कर्णं वैकर्तनं रणे}
{सर्वशस्त्रभृतां श्रेष्ठं सर्वशास्त्रार्थपरागम्}


\twolineshloka
{यस्य ज्यातलनिर्घोषं श्रुत्वा भयकरं महत्}
{पाण्डवेयानि सैन्यानि विद्रवन्ति दिशो दश}


\twolineshloka
{प्रत्यक्ष ते महाबाहो यथा रात्रौ घटोत्कचः}
{मायाशतानि कुर्वाणो हतो मायापुरस्कृतः}


\twolineshloka
{न चातिष्ठत बीभत्सुः प्रत्यनीके कथञ्चन}
{एतांश्च दिवसान्सर्वान्भयेन महता वृतः}


\twolineshloka
{भीमसेनश्च बलवान्धनुष्कोट्याभिचोदितः}
{उक्तश्च संज्ञया राजन्मूढ औदरिको यथा}


\twolineshloka
{माद्रीपुत्रौ तथा शूरौ येन जित्वा महारणे}
{कमप्यर्थं पुरस्कृत्य न हतौ युधि मारिष}


\twolineshloka
{येन वृष्णिप्रवीरस्तु सात्यकिः सात्वतां वरः}
{निर्जित्य समरे शूरो विस्थश्च बलात्कृतः}


\twolineshloka
{सृञ्जयाश्चेतरे सर्वे धृष्टद्युम्नपुरोगमाः}
{असकृन्निर्जिताः सङ्ख्ये स्मयमानेन संयुगे}


\twolineshloka
{तं कथं पाण्डवा युद्वे विजेष्यन्ति महारथम्}
{यो हन्यात्समरे क्रुद्वो वज्रहस्तं पुरन्दरम्}


\twolineshloka
{त्वं च सर्वास्त्रविद्वीरः सर्वविद्यास्त्रपारगः}
{बाहुवीर्येण ते तुल्यः पृथिव्यां नास्ति कश्चन}


\twolineshloka
{त्वं शल्यभूतः शत्रूणामविषह्यः पराक्रमे}
{ततस्त्वमुच्यसे राजञ्शल्य इत्यरिसूदन}


\twolineshloka
{तव बाहुबलं प्राप्य न शेकुः सर्वसात्वताः}
{तव बाहुबलाद्राजन्किं नु कृष्णो बलाधिकः}


\twolineshloka
{यथा हि कृष्णेन बलं धार्यं वै फल्गुने हते}
{तथा कर्णात्ययीभावे त्वया धार्यं महद्बलम्}


\twolineshloka
{किमर्थं समरे सैन्यं वासुदेवो न्यवारयत्}
{किमर्थं च भवान्सैन्यं न हनिष्यति मारिष}


\threelineshloka
{त्वत्कृते पदवीं गन्तुमिच्छेयं युधि मारिष}
{सोदराणां च वीराणां सर्वेषां च महीक्षिताम् ॥शल्य उवाच}
{}


\twolineshloka
{यन्मां ब्रवीषि गान्धारे अग्रे सैन्यस्य मानद}
{विशिष्टं देवकीपुत्रात्प्रीतिमानस्म्यहं त्वयि}


\twolineshloka
{एष सारथ्यमातिष्ठे राधेयस्य यशस्विनः}
{युध्यतः पाण्डवाग्र्येण यथा त्वं वीर मन्यसे}


\threelineshloka
{समयश्च हि मे वीर कश्चिद्वैकर्तनं प्रति}
{उत्सृजेयं यथाश्रद्वमहं वाचोऽस्य सन्निधौ ॥सञ्जय उवाच}
{}


\twolineshloka
{तथेति राजन्पुत्रस्ते सह कर्णेन मारिष}
{अब्रवीत्मद्रराजानं सर्वक्षत्रस्य सन्निधौ}


\twolineshloka
{सारथ्यस्याभ्युपगमाच्छल्येनाश्वासितस्तदा}
{दुर्योधनस्तदा हृष्टः कर्णं तमभिषस्वजे}


\twolineshloka
{अब्रवीच्च पुनः कर्णं स्तूयमानः सुतस्तव}
{जहि पार्थान्रणे सर्वान्महेन्द्रो दानवानिव}


\twolineshloka
{स शल्येनाभ्युपगते हयानां सन्नियच्छने}
{कर्णो हृष्टमना भूयो दुर्योधनभाषत}


\twolineshloka
{नातिहृष्टमना ह्येष मद्रराजोऽभिभाषते}
{राजन्मधुरया वाचा पुनरेनं ब्रवीहि वै}


\twolineshloka
{ततो राजा महाप्राज्ञः सर्वास्त्रकुशलो बली}
{दुर्योधनोऽब्रवीच्छल्यं मद्रराजं महीपतिम्}


\twolineshloka
{पूरयन्निव घोषेण मेघगम्भीरया गिरा}
{शल्य कर्णोऽर्जुनेनाद्य योद्वव्यमिति मन्यते}


\threelineshloka
{तस्य त्वं पुरुषव्याघ्र नियच्छ तुरगान्युधि}
{कर्णो हत्वेतरान्सर्वान्फल्गुनं हन्तुमिच्छति}
{तस्याभीषुग्रहे राजन्प्रयाचे त्वां पुनः पुनः}


\threelineshloka
{पार्थस्य सचिवः कृष्णो यथाऽभीषुग्रहो वरः}
{तथा त्वमपि राधेयं सर्वतः परिपालय ॥सञ्जय उवाच}
{}


\threelineshloka
{ततः शल्यः परिष्वज्य सुतं ते वाक्यमब्रवीत्}
{दुर्योधनममित्रघ्नं प्रीतो मद्राधिपस्तदा ॥शल्य उवाच}
{}


\twolineshloka
{एवं चेन्मन्यसे राजन्गान्धारे प्रियदर्शन}
{तस्मात्ते यत्प्रियं किञ्चित्तत्सर्वं करवाण्यहम्}


\twolineshloka
{यत्रास्मि भरतश्रेष्ठ योग्यः कर्मणि कर्हिचित्}
{तत्र सर्वात्मना युक्तो वक्ष्ये कार्यं परन्तप}


\threelineshloka
{यत्तु कर्णमहं ब्रूयां हितकामः प्रियाप्रिये}
{मम तत्क्षमतां सर्वं भवान्कर्णश्च सर्वशः ॥कर्ण उवाच}
{}


\threelineshloka
{ईशानस्य यथा ब्रह्मा यथा पार्थस्य केशवः}
{तथा नित्यं हिते युक्तो मद्रराज भवस्व नः ॥शल्य उवाच}
{}


\twolineshloka
{आत्मनिन्दात्मपूजा च परनिन्दा परस्तवः}
{अनाचरितमार्याणां वृत्तमेतच्चतुर्विधम्}


\twolineshloka
{यत्तु विद्वन्प्रवक्ष्यामि प्रxxxxयार्थमहं तव}
{आत्मनः स्तवसंयुक्तं तन्निबोध यथातथम्}


\twolineshloka
{अहं शक्रस्य सारथ्ये योग्यो मातलिवत्प्रभो}
{अप्रमादात्प्रयोगाच्च ज्ञानविद्याचिकित्सनैः}


\twolineshloka
{ततः पार्थेन सङ्ग्रामे युध्यमानस्य तेऽनघ}
{वाहयिष्यामि तुरगान्विज्वरो भव सूतज}


\chapter{अध्यायः ३४}
\twolineshloka
{दुर्योधन उवाच}
{}


\twolineshloka
{अयं ते कर्ण सारथ्यं मद्रराजः करिष्यति}
{कृष्णादभ्यधिको यन्ता देवेशस्येव मातलिः}


\twolineshloka
{यथा हरिहयैर्युक्तं सङ्गृह्णाति स मातलिः}
{शल्यस्तथा तवाद्यायं संयन्ता रथवाजिनाम्}


\threelineshloka
{योधे त्वयि रथस्थे च मद्रराजे च सारथौ}
{रथश्रेष्ठो ध्रुवं सङ्ख्ये पार्थानभिभविष्यति ॥सञ्जय उवाच}
{}


\twolineshloka
{ततो दुर्योधनो भूयो मद्रराजं तरस्विनम्}
{उवाच राजन्सङ्ग्रामेऽध्युषिते पर्युपस्थिते}


\twolineshloka
{कर्णस्य यच्छ सङ्ग्रामे मद्रराज हयोत्तमान्}
{त्वयाऽभिगुप्तो राधेयो विजेष्यति धनञ्जयम्}


\twolineshloka
{इत्युक्तो रथमास्थाय तथेति प्राह भारत}
{शल्येऽभ्युपगते कर्णः सारथिं सुमनाऽब्रवीत्}


% Check verse!
त्वं सूत स्यन्दनं मह्यं कल्पयेत्यसकृत्त्वरन्
\twolineshloka
{ततो जैत्रं रथवरं गन्धर्वनगरोपमम्}
{विधिवत्कल्पितं भद्रं जयेत्युक्त्वा न्यवेदयत्}


\twolineshloka
{तं रथं रथिनां श्रेष्ठः कर्णोऽभ्यर्च्य यथाविधि}
{सम्पादितं ब्रह्मविदा पूर्वमेव पुरोधसा}


\twolineshloka
{कृत्वा प्रदक्षिणं यत्नादुपस्थाय च भास्करम्}
{समीपस्थं मद्रराजमारोह त्वमथाब्रवीत्}


\twolineshloka
{ततः कर्णस्य दुर्धर्षं स्यन्दनप्रवरं महत्}
{आरुरोह महातेजाः शल्यः सिंह इवाचलम्}


\twolineshloka
{ततः शल्याश्रितं दृष्ट्वा कर्णः स्वं रथमुत्तमम्}
{अध्यतिष्ठद्यथाऽम्भोदं विद्युत्वन्तं दिवाकरः}


\twolineshloka
{तावेकरथमारूढावादित्याग्निसमत्विषौ}
{अभ्राजेतां यथा मेघं सूर्याग्नी सहितौ दिवि}


\twolineshloka
{संस्तूयमानौ तौ वीरौ तदास्तां द्युतिमत्तमौ}
{ऋत्विक्सदस्यैरिन्द्राग्नी स्तूयमानाविवाध्यरे}


\twolineshloka
{स शल्यसङ्गृहीताश्वे रथे कर्णः स्थितो बभौ}
{धनुर्विष्फारयन्घोरं परिवेषीव भास्करः}


\twolineshloka
{आस्थितः स रथश्रेष्ठं कर्णः शरगभस्तिमान्}
{प्रबभौ पुरुषव्याघ्रो मन्दरस्थ इवांशुमान्}


\twolineshloka
{तं रथस्थं महाबाहुं युद्वायामिततेजसम्}
{दुर्योधनस्तु राधेयमिदं वचनमब्रवीत्}


\twolineshloka
{अकृतं द्रोणभीष्माभ्यां दुष्करं कर्म संयुगे}
{कुरुष्वाधिरथे वीर मिषतां सर्वधन्विनाम्}


\twolineshloka
{मनोगतं मम ह्यासीद्भीष्मद्रोणौ महारथौ}
{अर्जुनं भीमसेनं च निहन्ताराविति ध्रुवम्}


\twolineshloka
{ताभ्यां यदकृतं वीर वीरकर्म महामृधे}
{तत्कर्म कुरु राधेय वज्रपाणिरिवापरः}


\twolineshloka
{गृहाण धर्मराजं वा जहि वा त्वं धनञ्जयम्}
{भीमसेनं च राधेय माद्रीपुत्रौ यमावपि}


\twolineshloka
{जयश्च तेऽस्तु भद्रं ते प्रयाहि पुरुषर्षभ}
{पाण्डुपुत्रस्य सैन्यानि कुरु सर्वाणि भस्मसात्}


\twolineshloka
{ततस्तूर्यसहस्राणि भेरीणामयुतानि च}
{वाद्यमानान्यरोचन्त मेघशब्दो यथा दिवि}


\twolineshloka
{प्रतिगृह्य तु तद्वाक्यं रथस्थो रथरुत्तमः}
{अभ्यभाषत राधेयः शल्यं युद्धविशारदम्}


\twolineshloka
{चोदयाश्वान्महाबाहो यावद्वन्मि धनञ्जयम्}
{भीमसेनं यमौ चोभौ राजानं च युधिष्ठिरम्}


\twolineshloka
{अद्य पश्यतु मे शल्य बाहुवीर्यं धनञ्जयः}
{अस्यतः कङ्कपत्राणां सहस्राणि शतानि च}


\threelineshloka
{अद्य क्षेप्स्याम्यहं शल्य शरान्परमतेजनान्}
{पाण्डवानां विनाशाय दुर्योधनजयाय च ॥सञ्जय उवाच}
{}


\twolineshloka
{एतच्छ्रुत्वा वचस्तस्य शल्यः कर्णं वचोऽब्रवीत्}
{कथं नु तान्महावीर्यान्पाण्डवानवमन्यसे}


\threelineshloka
{सर्वास्त्रज्ञान्महेष्वासान्सर्वानेन महाबलान्}
{अनिवर्तिनो महाभागानजय्यान्सत्यविक्रमान्}
{अपि सञ्जनयेयुर्ये भयं साक्षाच्छतक्रतोः}


\twolineshloka
{यदा श्रोष्यसि निर्घोषं विस्फूर्जितमिवाशनेः}
{राधेय पार्थधनुषस्तदा नैवं वदिष्यसि}


\twolineshloka
{यदा द्रक्ष्यसि भीमेन कुञ्जरानीकमाहवे}
{विशीर्णदन्तं निहतं तदा नैवं वदिष्यसि}


\twolineshloka
{यदा द्रक्ष्यसि सङ्ग्रामे धर्मपुत्रं यमौ तथा}
{शितैः पृषत्कैः कुर्वाणानभ्रच्छायामिवाम्बरे}


\threelineshloka
{अस्यतः क्षिण्वतश्चारीँल्लघुहस्तान्दुरासदान्}
{पार्थिवागपि चान्यांस्त्वं तदा नैवं वदिष्यसि ॥सञ्जय उवाच}
{}


\twolineshloka
{अनादृत्य तु तद्वाक्यं मद्रराजे भाषितम्}
{द्रक्ष्यस्यद्येत्यवोचत्तं शल्यं कर्णो जनेश्वर}


\chapter{अध्यायः ३५}
\twolineshloka
{सञ्जय उवाच}
{}


\twolineshloka
{दृष्ट्वा कर्णं महावीर्यं युयुत्सं समवस्थितम्}
{चुक्रुशुः कुरवः सर्वे हृष्टरूपाः समन्ततः}


\threelineshloka
{ततो दुन्दुभिनिर्घोषैर्भेरीणां निनदेन च}
{कोणशब्दैश्च विविधैर्गर्जितैश्च तरस्विनाम्}
{निर्ययुस्तावकाः सर्वे मृत्युं कृत्वा निवर्तनम्}


\twolineshloka
{प्रयाते तु ततः कर्णे रथेषु मदितेषु च}
{चचाल च मही सर्वा ररास च सुविस्वरम्}


% Check verse!
निःसरन्तो व्यदृश्यन्त सूर्यात्सप्त महाग्रहाः
\twolineshloka
{उल्कापाताश्च सञ्जज्ञुर्दिग्दाहाश्चैव दारुणाः}
{तथाऽशन्यश्च सम्पेतुर्ववुर्वाताश्च भैरवाः}


\twolineshloka
{मृगपक्षिगणाश्चैव पृतनां बहुशस्तव}
{अपसव्यं तदा चक्रुर्वेदयन्तो महाभयम्}


\twolineshloka
{प्रस्थितस्य च कर्णस्य निपेतुस्तुरगा भुवि}
{अस्थिवर्षं च पतितमन्तरिक्षाद्भयानकम्}


\twolineshloka
{जज्वलुश्चैव शस्त्राणि ध्वजाश्चैव चकम्पिरे}
{अश्रूणि च व्यमुञ्चन्त वाहनानि विशाम्पते}


\twolineshloka
{एते चान्ये च बहव उत्पातास्तत्र मारिष}
{समुत्पेतुर्विनाशाय कौरवाणां सुदारुणाः}


\twolineshloka
{न च तान्गणयामासुः सर्वे दैवेन मोहिताः}
{प्रस्थितं सूतपुत्रं च जयेत्सूचुर्नराधिपाः}


\twolineshloka
{`शल्येन सहितं दृष्ट्वा कर्णमाहवशोभिनम्'}
{निर्जितान्पाण्डवांश्चैव मेनिरे तत्र कौरवाः}


\twolineshloka
{ततो रथस्थः परवीरहन्ताभीष्मद्रोणवतिवीर्यौ समीक्ष्य}
{समुज्ज्वलन्भास्करपावकाभोवैकर्तनोऽसौ रथकुञ्जरो वृषा}


% Check verse!

% Check verse!

% Check verse!

% Check verse!

% Check verse!

% Check verse!

% Check verse!

% Check verse!
अलं मनुष्यस्य सुखाय वर्तितुंतथाहि युद्वे निहतः परैर्गुरुः
% Check verse!
हुताशनादित्यसमानतेजसंपराक्रमे विष्णुपुरन्दरोपमम् ॥नये ब-हस्पत्युशनःसमं सदान चैनमस्त्रं तदुपास्त दुःसहम्
\twolineshloka
{सम्प्राक्रुष्टे रुदितस्त्रीकुमारेपराभूते पौरुषे धार्तराष्ट्रे}
{कृत्यं मया नाद्य जानामि शल्यप्रयाहि तस्माद्द्विषतामनीकम्}


\twolineshloka
{स यत्र राजा पाण्डवः सत्यसन्धोव्यवस्थितो भीमसेनार्जुनौ च}
{स वासुदेवः सात्यकिः सृञ्जयाश्चयमौ च कस्तान्विषहेन्मदन्यः}


\twolineshloka
{तस्मात्क्षिप्रं मद्रपते प्रयाहिरणे पाञ्चालान्पाण्डवान्सृञ्जयांश्च}
{तान्वा हनिष्यामि समेत्य सङ्ख्येयास्यामि वा द्रोणमुखो यमाय}


\twolineshloka
{न त्वेवाहं न गमिष्यामि मध्येतेषां शूराणामिति शल्याद्य विद्वि}
{मित्रद्रोहो मर्षणीयो न मेऽयंत्यक्त्वा प्राणाननुयास्यामि द्रोणम्}


\twolineshloka
{प्राज्ञस्य मूढस्य च जीवितान्तेनास्ति प्रमोक्षोऽन्तकसन्निभस्य}
{अतो विद्वन्नभियोत्स्यामि पार्थंदिष्टं न शक्यं व्यतिवर्तितुं वै}


\twolineshloka
{कल्याणवृत्तः सततं हि राजावैचित्रवीर्यस्य सुतो ममासीत्}
{तस्यार्थसिद्व्यर्थमहं त्यजामिप्रियान्भोगान्दुस्त्यजं जीवितं च}


\twolineshloka
{वैयाघ्रचर्माणमकूजनाक्षंहैमं त्रिकोशं रजतत्रिवेणुम्}
{रथप्रबर्हं तुरगप्रबार्है--र्युक्तं प्रादान्मह्यमिमं हि रामः}


\twolineshloka
{धनूंषि चित्राणि निरीक्ष्य शल्यध्वजान्गदाः सायकांश्चोग्ररूपान्}
{असिं च दीप्तं परमायुधं चशङ्खं च शुभ्रं स्वनवन्तमुग्रम्}


\twolineshloka
{पताकिनं वज्रनिपातनिः स्वनंसिताश्वयुक्तं शुभतूणशोभितम्}
{इमं समास्थाय रथर्षभं रणेदृढं हनिष्याम्यहमर्जुनं बलात्}


\twolineshloka
{तं चेन्मृत्युः सर्वहरोऽभिरक्षे--त्सदाऽप्रमत्तः समरे पाण्डुपुत्रम्}
{तं वा हनिष्यामि रणे समेत्ययास्यामि वा द्रोणमुखो यमाय}


\threelineshloka
{यमवरुणकुबेरवासवा वायदि युगपत्सगणा महाहवे}
{जुगुपुरिह समेत्य फल्गुनंकिमु बहुना सह तैर्जयामि तम् ॥सञ्जय उवाच}
{}


\threelineshloka
{इति रणरभसस्य कत्थत--स्तदुपनिशम्य वचः स मद्रराट्}
{अवहसदवमन्य वीर्यवान्प्रतिषिषिधे च जगाद चोत्तरम् ॥शल्य उवाच}
{}


\twolineshloka
{विरम विरम कर्ण कत्थना--दतिरभसोऽप्यतिवाचमुक्तवान्}
{क्व च हि नरवरो धनञ्जयःक्व पुनरिह त्वमहो नराधमः}


\twolineshloka
{यदुसदनमुपेन्द्रपालितंत्रिदिवमिवामरराजरक्षितम्}
{प्रसभमतिविलोड्य को हरे--त्पुरुषवरावरजामृतेऽर्जुनात्}


\twolineshloka
{त्रिभुवनसृजमीश्वरेश्वरंक इह पुमान्भवमाह्वयेद्युधि}
{मृगवधकलहे ऋतेऽर्जुनात्सुरपतिवीर्यसमप्रभावतः}


\twolineshloka
{असुरसुरमहोरगान्नरान्गरुडपिशाचसयक्षराक्षसान्}
{इषुभिरजयदग्निगौरवात्स्वभिलषितं च हविर्ददौ जयः}


\twolineshloka
{स्मरसि ननु यदा परैर्हृतःसहधृतराष्ट्रसुतो विमोक्षितः}
{अधिरथज नरोत्तमैर्युतान्कुरुषु बहून्विनिहत्य तानरीन्}


\twolineshloka
{प्रथममपि पलायिते त्वयिप्रियकलहा धृतराष्ट्रसूनवः}
{स्मरसि ननु यदा प्रमोचिताःखचरगणानवजित्य पाण्डवैः}


\twolineshloka
{समुदितबलवाहनाः पुनःपुरुषवरेण जिताः स्थ गोग्रहे}
{सगुरुगुरुसुताः सभीष्मकाःकिमु न जितः स तदा त्वयाऽर्जुनः}


\threelineshloka
{इदमपरमुपस्थितं पुन--स्तव निधनाय सुयुद्वमद्य वै}
{यदि न रिपुभयात्पलायसेसमरगतोऽद्य हतोऽसि सूतज ॥सञ्जय उवाच}
{}


\threelineshloka
{इति बहुपरुषं प्रभाषतिप्रमनसि मद्रपतौ रिपुस्तवम्}
{भृशमभिरुषितः सरन्तपःकुरुपृतनापतिराह मद्रपम् ॥कर्ण उवाच}
{}


\threelineshloka
{भवतु भवतु किं विकत्थसेननु मर्म तस्य हि युद्धमुद्यतम्}
{यदि स जयति मामिहाहवेतत इदमस्तु सुकत्थितं तव ॥सञ्जय उवाच}
{}


\twolineshloka
{एवमस्त्विति मद्रेश उक्त्वा नोत्तरमुक्तवान्}
{याहि शल्येति चाप्येनं कंर्णः प्राह युयुत्सया}


\twolineshloka
{स रथः प्रययौ शत्रून्निघ्नतः शल्यसारथिः}
{निघ्नन्नमित्रान्समरे तमोघ्नः सविता यथा}


\twolineshloka
{ततः प्रायात्प्रीतिमान्वै रथेनवैयाघ्रेण श्वेतयुजाऽथ कर्णः}
{स चालोक्य ध्वजिनीं पाण्डवानांधनञ्जयं त्वरया पर्यपृच्छत्}


\chapter{अध्यायः ३६}
\twolineshloka
{सञ्जय उवाच}
{}


\twolineshloka
{प्रयत्नेन तदा कर्णो हर्षयन्वाहिनीं तव}
{एकैकं समरे दृष्ट्वा पाण्डवं पर्यपृच्छत}


\twolineshloka
{यो ममाद्य महात्मानं दर्शयेच्छ्वेतवाहनम्}
{तस्मै दद्यामभिप्रेतं धनं यन्मनसेच्छति}


\twolineshloka
{न चेत्तदभिमन्येत तस्मै दद्यामहं पुनः}
{शकटं रत्नसम्पूर्णं यो मे ब्रूयाद्वनञ्जयम्}


\twolineshloka
{न चेत्तदभिमन्येत पुरुषोऽर्जुनदर्शिवान्}
{शतं दद्यां गवां तस्मै नैत्यकं कांस्यदोहनम्}


\threelineshloka
{शतं ग्रामवरांश्चैव दद्यामर्जुनदर्शिने}
{तथा तस्मै पुनर्दद्यां श्वेतमश्वतरीरथम्}
{युक्तमञ्जनकेशीभिर्यो मे ब्रूयाद्वनञ्जयम्}


\twolineshloka
{न चेत्तदभिमन्येत पुरुषोऽर्जुनदर्शिवान्}
{अन्यं वाऽस्मै पुनर्दद्यां सौवर्णं हस्तिषङ्गवम्}


\twolineshloka
{तथाप्यस्मै पुनर्दद्यां स्त्रीणां शतमलङ्कृतम्}
{श्यामानां निष्ककण्ठीनां गीतवाद्यविपश्चिताम्}


\twolineshloka
{न चेत्तदभिमन्येत पुरुषोऽर्जुनदर्शिवान्}
{तस्मै दद्यां शतं नागाञ्शतं ग्रामाञ्शतं रथान्}


\twolineshloka
{सुवर्णस्य च मुख्यस्य हयाग्र्याणां शतं शतान्}
{ऋद्व्या गुणैः सुदान्तांश्च धुर्यवाहान्सुशिक्षितान्}


\twolineshloka
{तथा सुवर्णशृङ्गीणां गोधेनूनां चतुःशतम्}
{दद्यां तस्मै सवत्सानां यो मे ब्रूयाद्वनञ्जयम्}


\twolineshloka
{न चेत्तदभिमन्येत पुरुषोऽर्जुनदर्शिवान्}
{अन्यदस्मै वरं दद्यां श्वेतान्पञ्चशतान्हयान्}


\twolineshloka
{हेमभाण्डपरिच्छन्नान्सुमृष्टमणिभूषणान्}
{सुदान्तानपि चैवाहं दद्यामष्टादशापरान्}


\twolineshloka
{रथं च शुभ्रं सौवर्णं दद्यां तस्मै स्वलङ्कृतम्}
{युक्तं परमकाम्भोजैर्यो मे ब्रूयाद्वनञ्जयम्}


\twolineshloka
{न चेत्तदभिमन्येत पुरुषोऽर्जुनदर्शिवान्}
{अन्यदस्मै वरं दद्यां कुञ्जराणां शतानि षट्}


\twolineshloka
{काञ्चनैर्विविधैर्भाण्डैराच्छन्नान्हेममालिनः}
{उत्पन्नानपरान्तेषु विनीतान्हस्तिशिक्षकैः}


\twolineshloka
{न चेत्तदभिमन्येत पुरुषोऽर्जुनदर्शिवान्}
{अन्यदस्मै वरं दद्यां वैश्यग्रामांश्चतुर्दश}


\twolineshloka
{सुस्फीतान्धनसंयुक्तान्प्रत्यासन्नवनोदकान्}
{अकुतोभयान्सुसम्पन्नान्राजभोज्यांश्चतुर्दश}


\twolineshloka
{दासीनां निष्ककण्ठीनां मागधीनां शतं तथा}
{प्रत्यग्रवयसां दद्यां यो मे ब्रूयाद्धनञ्जयम्}


\twolineshloka
{न चेत्तदभिमन्येत पुरुषोऽर्जुनदर्शिवान्}
{अन्यं तस्मै वरं दद्यां यमसौ कामयेत्स्वयम्}


\twolineshloka
{पुत्रदारान्विहायैव यदन्यद्वित्तमस्ति मे}
{तच्च तस्मै पुनर्दद्यां यद्यच्च मनसेच्छति}


\threelineshloka
{हत्वा च सहितौ कृष्णौ तयोर्वित्तानि सर्वशः}
{तस्मै दद्यामहं यो मे प्रब्रूयात्केशवार्जुनौ ॥सञ्जय उवाच}
{}


\twolineshloka
{एता वाचः सुबहुशः कर्ण उच्चारयन्युधि}
{दध्मौ सागरसम्भूतं सुस्वरं शङ्खमुत्तमम्}


\twolineshloka
{ता वाचः सूतपुत्रस्य तथा युक्ता निशम्य तु}
{दुर्योधनो महाराज संहृष्टः सानुजोऽभवत्}


\twolineshloka
{ततो दुन्दुभिनिर्घोषो मृदङ्गानां च सर्वशा}
{सिंहनादः सवादित्रः कुञ्चराणां च निःस्वनः}


\twolineshloka
{प्रादुरासीत्तदा राजंस्त्वत्सैन्ये पुरुषर्षभ}
{योधानां सम्प्रहृष्टानां तथा समभवत्स्वनः}


\threelineshloka
{तथा प्रहृष्टे सैन्ये तं प्लुवमानं महारथम्}
{विकत्थमानं च तदा राधेयमरिकर्शनम्}
{मद्रराजः प्रहस्येदं वचनं प्रत्यभाषत}


\chapter{अध्यायः ३७}
\twolineshloka
{शल्य उवाच}
{}


\twolineshloka
{मा सूतपुत्र मन्येत(थाः) सौवर्णं हस्तिषङ्गवम्}
{प्रयच्छसि मुधैव त्वं द्रक्ष्यस्यद्य धनञ्जयम्}


\twolineshloka
{मा सूतपुत्र दानेन सौवर्णं हस्तिषङ्गवम्}
{प्रयच्छ पुरुषायाद्य द्रक्ष्यसि त्वं धनञ्जयम्}


\twolineshloka
{बलेन मत्तस्त्यजसि वसु वैश्रवणो यथा}
{अयत्नेनैव राधेय द्रष्टास्यद्य धनञ्जयम्}


\twolineshloka
{पुरा सृजसि यच्चापि वित्तं बहु च मूढवत्}
{अपात्रदानाद्ये दोषास्तान्मोहान्नावबुध्यसे}


\twolineshloka
{यत्प्रवेदयसे वित्तं बहु तेन खलु त्वया}
{शक्यं बहुविधैर्यज्ञैर्यष्टुं सूत यजस्व तैः}


\twolineshloka
{यच्च प्रार्थयसे हन्तुं कृष्णौ मोहाद्वृथैव तत्}
{न हि शुश्रुम सम्मर्दे क्रोष्ट्रा सिंहौ निपातितौ}


\twolineshloka
{अप्रार्थितं प्रार्थयसे सुहृदो न हि सन्ति ते}
{ये त्वां निवारयन्त्याशु प्रपतन्तं हुताशने}


\twolineshloka
{कार्याकार्यं न जानीषे कालपक्कोऽस्यसंशयम्}
{बह्वबद्धमकर्णीयं को हि ब्रूयाज्जिजीविषुः}


\twolineshloka
{समुद्रतरणं दोर्भ्यां कण्ठे बद्ध्वा यथा शिलाम्}
{गिर्यग्राद्वा निपतनं तादृक्त्व चिकीर्षितम्}


\twolineshloka
{सहितः सर्वयोधैस्त्वं व्यूढानीकैः सुरक्षितः}
{धनञ्जयेन युध्यस्व श्रेयश्चेत्प्राप्तुमिच्छसि}


\threelineshloka
{हितार्थं धार्तराष्ट्रस्य ब्रवीमि त्वां न हिंसया}
{श्रद्धस्वेदं मया प्रोक्तं यदि तेऽस्ति जिजीविषा ॥कर्ण उवाच}
{}


\twolineshloka
{स्वबाहुवीर्यमाश्रित्य प्रार्थयाम्यर्जुनं रणे}
{त्वं तु मित्रमुखः शत्रुर्मां भीषयितुमिच्छसि}


\threelineshloka
{न मामस्मादभिप्रायात्कश्चिदद्य निवर्तयेत्}
{अपीन्द्रो वज्रमुद्यम्य किमु मर्त्यः कथञ्चन ॥सञ्जय उवाच}
{}


\twolineshloka
{इति कर्णस्य वाक्यान्ते शल्यः प्राहोत्तरं वचः}
{चुकोपयिषुरत्यर्थं कर्णं मद्रेश्वरः पुनः}


\twolineshloka
{यदा वै त्वां फल्गुनबाहुवेगा--ज्ज्याचोदिता वेगवन्तोऽग्निकल्पाः}
{अन्वेतारः कङ्कपत्राः शिताग्रा--स्त्यक्ष्यत्येषा स्वां तदा कर्ण बुद्धिः}


\twolineshloka
{यदा दिर्व्य धनुरादाय पार्थःप्रतापयन्पृतनां सव्यसाची}
{त्वां मर्दयिष्यत्यसुखैः पृषत्कै--स्तदापृच्छां त्यक्ष्यसे पाण्डवस्य}


\twolineshloka
{बालश्चन्द्रं मातुरङ्के शयानोयथा कश्चित्प्रार्थयतेऽपहर्तुम्}
{तद्वन्मोहाद्दयोतामानं रथस्थंजेतुं पार्थं काङ्क्षसे सूतपुत्र}


\twolineshloka
{हरादस्त्रं तीक्ष्णधारं यथाऽद्यसर्वाणि गात्राणि निकर्षसि त्वम्}
{सुतीक्ष्णशस्त्रोपमकर्मणेहयुयुत्ससे फल्गुनेनाद्य कर्ण}


\twolineshloka
{हन्यादसिं तीक्ष्णधारं यथाऽतःसुतेजनं निहितं वै पृथिव्याम्}
{तथा खनस्यद्य शितान्पृषत्का--न्यथार्थयस्यर्जुनेनेह युद्धम्}


\twolineshloka
{त्रिशूलमाश्लिष्य सुतीक्ष्णधारंसर्वाणि गात्राणि विघर्षसि त्वम्}
{सुतीक्ष्णधारोपमकर्मणा त्वंयुयुत्ससे योऽर्जुनेनाद्य कर्ण}


\twolineshloka
{क्रुद्धं सिंहं केसरिणं बृहन्तंबालो मूढः क्षुद्रमृगस्तरस्वी}
{समाह्वयेद्वृष्टमुपेत्य योद्धुंतथा त्वमद्याह्वयसे हि पार्थम्}


\twolineshloka
{मा सूतपुत्राह्वय राजपुत्रंमहावीर्यं केसरिणं यथैव}
{वने शृगालः पिशितेन तृप्तःपार्थं समासाद्य विनङ्क्ष्यसि त्वम्}


\twolineshloka
{ईषादन्तं महानागं प्रभिन्नकरटामुखम्}
{शशको हयसे युद्वे कर्ण पार्थं धनञ्जयम्}


\twolineshloka
{विलस्थं कृष्णसर्पं त्वं जाल्यात्काष्ठेन विध्यसि}
{महाविषं पूर्णकोपं यत्पार्थं योद्वुमिच्छसि}


\twolineshloka
{सिंहं केसरिणं क्रुद्धमभिक्रम्याभिनर्दसे}
{शृगाल इव मूढस्त्वं नृसिंहं कर्ण पाण्डवम्}


\twolineshloka
{सुपर्णं पतगश्रेष्ठं वैनतेयं तरस्विनम्}
{स्वगेवाह्वयसे पार्थं तथा कर्णं धनञ्जयम्}


\twolineshloka
{सर्वाम्भसां निधिं भीमं मूर्तिमन्तं झषाकुलम्}
{चन्द्रोदये विवर्धन्तमप्लुवस्त्वं तितीर्षसि}


\twolineshloka
{ऋषभं दुन्दुभिग्रीवं तीक्ष्णशृङ्गं प्रहारिणम्}
{वत्स आह्वयसे युद्वे कर्ण पार्थं धनञ्जयम्}


\twolineshloka
{महामेघं महाघोषं दर्दुरः प्रतिनर्दसि}
{बाणतोयप्रदं काले नरपर्जन्यमर्जुनम्}


\twolineshloka
{यथा च श्वा गृहस्थस्तु व्याघ्रं वनगतं भषेत्}
{तथा त्वं भषसे कर्ण नरव्याघ्रं धनञ्जयम्}


\twolineshloka
{शृगालो हि वने कर्ण शशैः परिवृतो वसन्}
{मन्यते सिंहमात्मानं यावत्सिंहं न पश्यति}


\twolineshloka
{तथा त्वमपि राधेय सिंहमात्मानमिच्छसि}
{अपश्यञ्शत्रुदमनं नरसिंहं रणेऽर्जुनम्}


\twolineshloka
{व्याघ्रं त्वं मन्यसेऽऽत्मानं यावत्कृष्णौ न पश्यसि}
{समास्थितावेकरथे सूर्याचन्द्रमसाविव}


\twolineshloka
{यावद्ग्राण्डीवघोषं त्वं न शृणोषि महाहवे}
{तावदेव त्वया कर्ण शक्यं वक्तुं यथेच्छसि}


\twolineshloka
{रथशङ्खधनुःशब्दैर्नादयन्तं दिशो दश}
{नर्दन्तमिव शार्दूलं दृष्ट्वा क्रोष्टा भविष्यसि}


\twolineshloka
{नित्यमेव शृगालस्त्वं नित्यं सिंहो धनञ्जयः}
{वीरप्रद्वेषणान्मूढ तस्मात्क्रोष्टेव लक्ष्यसे}


\twolineshloka
{यथाऽऽखुः स्याद्विडालश्च श्वा व्याघ्रश्च बलाबले}
{यथा शृगालः सिंहश्च यथा च शशकुञ्जरौ}


\twolineshloka
{यथाऽनृतं च सत्यं च यथा चापि विषामृते}
{तथा त्वमपि पार्थश्च प्रख्यातावात्मकर्मभिः}


\chapter{अध्यायः ३८}
\twolineshloka
{सञ्जय उवाच}
{}


\threelineshloka
{अधिक्षिप्तस्तु राधेयः शल्येनामिततेजसा}
{शल्यमाह सुसङ्क्रुद्धो वाक्शल्यमवधारयन् ॥कर्ण उवाच}
{}


\twolineshloka
{गुणान्गुणवतां शल्य गुणवान्वेत्ति नागुणः}
{त्वं तु नित्यं गुणैर्हीनः किं ज्ञास्यसि गुणागुणान्}


\twolineshloka
{अर्जुनस्य महास्त्राणि क्रोधं वीर्यं धनुः शरान्}
{अहं शल्याभिजानामि न त्वं जानासि तत्तथा}


\twolineshloka
{तथा कृष्णस्य माहात्म्यमृषभस्य महीक्षिताम्}
{यथाहं शल्य जानामि न त्वं जानासि तत्तथा}


\twolineshloka
{एवमेवात्मनो वीर्यमहं वीर्यं च पाण्डवे}
{जानंस्तावाह्वये युद्धे शल्य नाग्निं पतङ्गवत्}


\twolineshloka
{अस्ति चाऽयमिषुः शल्य सुपुङ्खो रक्तभोजनः}
{एकस्तूणीशयः पत्री सुधौतः समलङ्कृतः}


\twolineshloka
{शेते चन्दनचूर्णेन पूजितो बहुलाः समाः}
{आहेयो विषवानुग्रो नराश्वद्विपसङ्घहा}


\twolineshloka
{घोररूपो महारौद्रस्तनुत्रास्थिविदारणः}
{निर्भिन्द्यां येन रुष्टोऽहमपि मेरुं महागिरिम्}


\twolineshloka
{तमहं जातु नास्येयमन्यस्मिन्फल्गुनादृते}
{कृष्णाद्वा देवकीपुत्रात्सत्यं चापि शृणुष्व मे}


\twolineshloka
{तेनाहमिषुणा शल्य वासुदेवधनञ्जयौ}
{योत्स्ये परमसंरब्धौ तत्कर्म सदृशं मम}


\threelineshloka
{सर्वेषां वृष्णिवीराणां कृष्णे लक्ष्मीः प्रतिष्ठिता}
{सर्वेषां पाण्डुपुत्राणां जयः पार्थे प्रतिष्ठितः}
{उभयं तु समासाद्य को निवर्तितुमर्हति}


\twolineshloka
{तावेतौ पुरुषव्याघ्रौ समेतौ स्यन्दने स्थितौ}
{मामेकमभिसंयुद्धौ सुजातं पश्य शल्य मे}


\twolineshloka
{पितृष्वसामातुलजौ भ्रातरावपराजितौ}
{मणी सूत्र इव प्रोतौ द्रष्टासि निहतौ मया}


\twolineshloka
{अर्जुने गाण्डिवं कृष्णे चक्रं तार्क्ष्यकपिध्वजौ}
{भीरूणां त्रासजननं शल्य हर्षकरं मम}


\twolineshloka
{त्वं तु दुष्प्रकृतिर्मूढो महायुद्वेष्वकोविदः}
{भयावदीर्णः सन्त्रासादबद्धं बहुः भाषसे}


\twolineshloka
{संस्तौषि तौ तु केनापि हेतुना त्वं कुदेशज}
{तौ हत्वा समरे हन्ता त्वामद्य सह बान्धवम्}


\twolineshloka
{पापदेशज दुर्बुद्धे क्षुद्र क्षत्रियपांसन}
{सुहृद्रूपो रिपुः किं मां कृष्णाभ्यां भीषयिष्यसि}


\twolineshloka
{तौ वा मामद्य हन्तारौ हनिष्ये वाऽपि तावहम्}
{नाहं बिभेमि कृष्णाभ्यां विजानन्नात्मनो बलम्}


\twolineshloka
{वासुदेवसहस्रं वा फल्गुनानां शतानि वा}
{अहमेको हनिष्यामि जोषमास्स्व कुदेशज}


% Check verse!
स्त्रियो बालाश्च वृद्वाश्च प्रायः क्रीडागता जनाः
\twolineshloka
{या गाथाः सम्प्रगायन्ति कुर्वतोऽध्ययनं यथा}
{ता गाथाः शृणु मे शल्य मद्रकेषु दुरात्मसु}


\twolineshloka
{ब्राह्मणैः कथिताः पूर्वं यथावद्राजसन्निधौ}
{श्रुत्वा चैकमना मूढ मम वा ब्रूहि चोत्तरम्}


\twolineshloka
{मित्रध्रुङ्मद्रको नित्यं यो नो द्वेष्टि स मद्रकः}
{मद्रके सङ्गतं नास्ति क्षुद्रवाक्ये नराधमे}


\twolineshloka
{दुरात्मा मद्रको नित्यं नित्यमानृतिकोऽनृजुः}
{यच्चान्यदपि दौरात्म्यं मद्रकेष्विति नः श्रुतम्}


\twolineshloka
{पिता पुत्रश्च माता च श्वश्रूश्वशुरमातुलाः}
{भगिनी दुहिता भ्राता नप्ताऽन्ये ते च बान्धवाः}


\twolineshloka
{वयस्याभ्यागताश्चान्ये दासीदासाश्च सङ्गताः}
{पुम्भिर्विमिश्रा नार्यश्च ज्ञाताज्ञाताः स्वयेच्छया}


\twolineshloka
{येषां गृहेष्वशिष्टानां सक्तुमद्याशिनां सदा}
{पीत्वा शीधु सगोमांसं क्रन्दन्ति च हसन्ति च}


\twolineshloka
{गायन्ति चाप्यबद्धानि प्रवर्तन्ते च कामतः}
{कामप्रलापिनोऽन्योन्यं तेषु धर्मःकथं भवेत्}


\twolineshloka
{मद्रकेष्ववलिप्तेषु प्रख्याताशुभकर्मसु}
{नापि वैरं न सौहार्दं मद्रकेण समाचरेत्}


\twolineshloka
{मद्रके सङ्गतं नास्ति मद्रको हि सदा मलः}
{मद्रकेषु च संसृष्टं शौचं गान्धारकेषु च}


\twolineshloka
{राजयाजकयाज्ये हि यथा दत्तं हविर्नशेत्}
{शूद्रसंस्कारको विप्रो यथा याति पराभवम्}


\twolineshloka
{यथा ब्रह्मद्विषो नित्यं गच्छन्तीह पराभवम्}
{यथैव संगतं कृत्वा नरः पतति मद्रकैः}


\twolineshloka
{बालेष्वपि सदा न स्म धनं वृश्चिक ते विषम्}
{आथर्वणेन मन्त्रेण सर्वशान्तिः कृता मया}


\twolineshloka
{इति वृश्चिकदष्टस्य विषवेगहतस्य च}
{कुर्वन्ति भेषजं प्राज्ञाः सत्यं तच्चापि दृश्यते}


\twolineshloka
{एवं विद्वञ्जोषमास्स्व ब्रूहि चात्रोत्तरं वचः}
{वासांस्युत्सृज्य नृत्यन्ति स्त्रियो मद्यविमोहिताः}


\twolineshloka
{मैथुनेऽसङ्गताश्चापि यथाकामवराश्च ताः}
{तासां पुत्र कथं धर्मं मद्रको वक्तुमर्हति}


\threelineshloka
{यास्तिष्ठन्त्यः प्रमेहन्ति यथैवोष्ट्रदशेरकाः}
{तासां विभ्रष्टधर्माणां निर्लज्जानां ततस्ततः}
{त्वं पुत्रस्तादृशीनां हि धर्मं वक्तुमिहेच्छसि}


\twolineshloka
{सौवीरकं याच्यमाना मद्रिका कर्षति स्फिचौ}
{अदातुकामा वचनमिदं वदति दारुणम्}


\twolineshloka
{मा मां सौवीरकं कश्चिद्याचतां दयितं मम}
{पुत्रं दद्यां पतिं दद्यां न तु दद्यां सुवीरकम्}


\twolineshloka
{गौर्यो बृहत्यो निर्हीका मद्रिकाः कम्पलावृताः}
{घस्मरा नष्टशौचाश्च प्राय इत्यनुशुश्रुम}


\twolineshloka
{पमेहित्वा स्फिचौ भूमौ घर्षन्त्यो हीनशोधनाः}
{शुद्धा नाद्भिर्न मृद्भिश्च नित्योच्छिष्टा भवन्ति हि}


\twolineshloka
{एवमादि मयाऽन्यैर्वा शक्यं वक्तुं भवेद्बहु}
{आकेशान्तान्नखाग्राच्च वक्तव्येषु कुकर्मसु}


\twolineshloka
{मद्रकाः सिन्धुसौवीरा धर्मं विद्युः कथं त्विह}
{पापदेशोद्भवा म्लेच्छा धर्माणामविचक्षणाः}


\twolineshloka
{एष मुख्यतमो धर्मः क्षत्रियस्येति नः श्रुतम्}
{यदाजौ निहतः शेते सद्भिः समभिपूजितः}


\twolineshloka
{आयुधानां साम्पराये यन्मुञ्चेयमहं ततः}
{ममैष प्रथमः कल्पो निधने स्वर्गमिच्छतः}


\twolineshloka
{सोयं प्रियसखा चास्मि धार्तराष्ट्रस्य धीमतः}
{तदर्थे हि मम प्राणा यच्च मे विद्यते वसु}


\twolineshloka
{व्यक्तं त्वमप्युपहितः पाण्डवैः पापदेशज}
{यथा चामित्रवत्सर्वं त्वमस्मासु प्रवर्तसे}


\twolineshloka
{कामं न खलु शक्योऽहं त्वद्विधानां शतैरपि}
{सङ्ग्रामाद्विमुखः कर्तुं धर्मज्ञ इव नास्तिकैः}


\twolineshloka
{सारङ्ग इव घर्मार्तः कामं विलप शुष्य च}
{नाहं भीषयितुं शक्यः क्षत्रधर्मे व्यवस्थितः}


\twolineshloka
{तनुत्यजां नृसिंहानामाहवेष्वनिवर्तिनाम्}
{या गतिर्गुरुणा प्रोक्ता प्रेम्णा रामेण तां स्मरे}


\twolineshloka
{स्वेषां त्राणार्थमुद्यन्तं वधार्थं द्विषतामपि}
{विद्धि मामास्थितं वृत्तं पौरूरवसमुत्तमम्}


\twolineshloka
{न तद्भूतं प्रपश्यामि त्रिषु लोकेषु मद्रप}
{यो मामस्मादभिप्रायाद्वारयेदिति मे मतिः}


\twolineshloka
{एवं विद्वञ्जोषमास्स्व त्रासात्किं बहु भाषसे}
{न त्वां हत्वा प्रदास्यामि क्रव्याद्भ्यो मद्रकाधम्}


\twolineshloka
{मित्रप्रतीक्षया शल्य धृतराष्ट्रस्य चोभयोः}
{अपवादतितिक्षाभिस्त्रिभिरेतैर्हि जीवसि}


\twolineshloka
{पुनश्चेदीदृशं वाक्यं मद्रराज वदिष्यसि}
{शिरस्ते पातयिष्यासि गदया वज्रकल्पया}


\twolineshloka
{श्रोतारस्त्विदमद्येह द्रष्टारो व्रा कुदेशज}
{कर्णं वा जघ्नतुः कृष्णौ कर्णो वा निजघान तौ}


\twolineshloka
{एवमुक्त्वा तु राधेयः पुनरेव विशाम्पते}
{अब्रवीन्मद्रराजानं याहियाहीत्यसम्भ्रमम्}


\chapter{अध्यायः ३९}
\twolineshloka
{सञ्जय उवाच}
{}


\twolineshloka
{मारिषाधिरथेः श्रुत्वा वाचो युद्धाभिनन्दिनः}
{शल्योऽब्रवीत्पुनः कर्णं निदर्शनमुदाहरन्}


\twolineshloka
{जातोऽहं यज्वनां वंशे सङ्ग्रामेष्वनिवर्तिनाम्}
{राज्ञां मूर्धाभिषिक्तानां स्वयं धर्मपरायणः}


\twolineshloka
{यथैव मत्तो मद्येन त्वं तथा लक्ष्यसे वृष}
{तथाऽद्य त्वां प्रमाद्यन्तं चिकित्सेयं सुहृत्तया}


\twolineshloka
{इमां काकोपमां कर्ण प्रोच्यमानां निबोध मे}
{श्रुत्वा यथेष्टं कुर्यास्त्वं निहीन कुलपांसन}


\twolineshloka
{नाहमात्मनि किञ्चिद्वै किल्बिषं कर्ण संस्मरे}
{येन मां त्वं महाबाहो हन्तुमिच्चस्यनागसम्}


\twolineshloka
{अवश्यं तु मया वाच्यं बुद्ध्वा तव हिताहितम्}
{विशेषतो रथस्थेन राज्ञश्चैव हितैषिणा}


\twolineshloka
{समं च विषमं चैव रथिनश्च बलाबलम्}
{श्रमः खेदश्च सततं हयानां रथिना सह}


\twolineshloka
{आयुधस्य परिज्ञानं रुतं च मृगपक्षिणाम्}
{सारं चैवाप्यासरं च शल्यानां च प्रतिक्रिया}


\threelineshloka
{अस्त्रयोगं च युद्वं च निमित्तानि तथैव च}
{सर्वमेतन्मया ज्ञेयं रथस्यास्य कुटुम्बिना}
{अतस्त्वां कथये कर्ण निदर्शनमिदं पुनः}


\twolineshloka
{वैश्यः किल समुद्रान्ते प्रभूतधनधान्यवान्}
{यज्वा दानपतिः क्षान्तः स्वकर्मस्थोऽभवच्छुचिः}


\twolineshloka
{बहुपुत्रः प्रियापत्यः सर्वभूतानुकम्पकः}
{राज्ञो धर्मप्रधानस्य राष्ट्रे वसति निर्भयः}


\twolineshloka
{पुत्राणां तस्य बालानां कुमाराणां यशस्विनाम्}
{काको बहूनामभवदुच्छिष्टकृतभोजनः}


\twolineshloka
{तस्मैसदा प्रयच्छन्ति वैश्यपुत्राः कुमारकाः}
{मांसोदनं दधि क्षीरं पायसं मधुसर्पिषी}


\twolineshloka
{स चोच्छिष्टभृतः काको वैश्यपुत्रैः कुमारकैः}
{सदृशान्पक्षिणो दृप्तः श्रेयससश्चावमन्यते}


\twolineshloka
{अथ हंसाः समुद्रान्ते प्रजग्मुरतिपातिनः}
{गरुडस्य गतौ तुल्याश्चक्राङ्गा हृष्टचेतसः}


\twolineshloka
{कुमरारकास्तदा हंसान्दृष्ट्वा काकमथाब्रुवन्}
{भवानेव विशिष्टोऽसि पतत्रिभ्यो विहङ्गम}


\twolineshloka
{एतेऽतिपातिनः पश्य विहङ्गान्वियदाश्रितान्}
{एहि त्वमपि शक्तो हि कस्मान्न पतितं त्वया}


\twolineshloka
{प्रतार्यमाणस्तैः सर्वैरल्पबुद्धिभइरण्डजः}
{तद्वचः सत्यमित्येव मौर्ख्याद्दर्पाच्च जज्ञिवान्}


\twolineshloka
{तान्सोनुपत्य जिज्ञासुः क एषां श्रेष्ठभागिति}
{उच्छिष्टदर्पितः काको बहूनामेकपातिनाम्}


\threelineshloka
{तेषां यं प्रवरं मेने हंसानां दूरपातिनाम्}
{स तमाह्वत दुर्बुद्धिः पताव इति पक्षिणम्}
{तच्छ्रुत्वा प्राहसन्हंसा ये तत्रासन्समागताः}


\threelineshloka
{भाषतो बहु काकस्य बालिश्यात्पततां वराः}
{इदमूचुः स्म चक्राङ्गावचः काकं विङ्गमाः ॥हंसा ऊचुः}
{}


\twolineshloka
{वयं हंसाश्चरामेमां पृथिवीं मानसौकसः}
{पक्षिणां च वयं नित्यं दूरपातेन पूजिताः}


\threelineshloka
{कथं हंसं नु बलिनं चक्राङ्गं दूरपातिनम्}
{काको भूत्वा निपतने समाह्वयसि दुर्मते}
{कथं त्वं पतिता काक सहास्माभिर्ब्रवीहि तत्}


\threelineshloka
{अथ हंसवचो मूढः कुत्सयित्वा पुनः पुनः}
{प्रजगादोत्तरं काकः कत्थनो जातिलाघवात् ॥काक उवाच}
{}


\twolineshloka
{शतमेकं च पातानां पतिताऽस्मि न संशयः}
{शतयोजनमेकैकं विचित्रं विविधं तथा}


\twolineshloka
{उड्डीनमवडीनं च प्रडीनं डीनमेव च}
{निडीनमथ संडीनं तिर्यक््डीनगतानि च}


\twolineshloka
{विडीनं परिडीनं च पराडीनं सुडीनकम्}
{अभिडीनं महाडीनं निर्डीनमतिडीनकम्}


\twolineshloka
{अवडीनं प्रडीनं च संडीनं डीनडीनकम्}
{संडीनोड्डीनडीनं च पुनर्डीनविडीनकम्}


\threelineshloka
{संपातं समुदीषं च ततोऽन्यद्व्यतिरिक्तकम्}
{गतागतं प्रतिगतं बह्वीश्च निकुलीनकाः}
{कर्ताऽस्मि मिषतां वोऽद्य ततो द्रक्षयथ मे बलम्}


\twolineshloka
{तेषामन्यतमेनाहं पतिष्यामि विहायसम्}
{प्रदिशध्वं यथान्यायं केन हंसाः पताम्यहम्}


\twolineshloka
{ते वै ध्रुवं विनिश्चित्य सम्पतध्वं मया सह}
{पातैरेभिः खलु खगाः पतितुं खे निराश्रये}


\threelineshloka
{एवमुक्ते तु काकेन प्रहस्यैको विहंगमः}
{उवाच काकं राधेय वचनं तन्निबोध मे ॥हंस उवाच}
{}


\fourlineindentedshloka
{शतमेकं च पातानां त्वं काक पतिता ध्रुवम्}
{एकमेव तु यं पातं विदुः सर्वे विहङ्गमाः}
{तमहं पतिता काक नान्यं जानामि कञ्चन ॥काक उवाच}
{}


% Check verse!
पत त्वमपि तत्राशु येन पातेन मन्यसे
\threelineshloka
{अथ काकं प्रजहसुर्ये तत्रासन्समागताः}
{कथमेकेन पातेन हंसः पातशतं जयेत्}
{एकेन वायस त्वैनं पातेनाभिभविष्यति}


\threelineshloka
{हंसश्चोत्पतितः काको बलवानाशुविक्रमः}
{प्रपेततुः स्पर्धया च ततस्तौ हंसवायसौ}
{उपर्युपरि वेगेन समुद्रं मकरालयम्}


\twolineshloka
{हंसस्त्वेकेन पातेन काकः पातशतेन च}
{स्पर्धिनौ सहितौ तूर्णं खमास्थाय तरस्विनौ}


\twolineshloka
{हंसस्तु मृदुनैकेन विक्रान्तुमुपचक्रमे}
{पूर्वमेव तु वै काकः स तं तूर्णं प्रचक्रमे}


% Check verse!
नात्यहीयत काकश्च मुहूर्तमिव सूतज
\twolineshloka
{काकोपि हंसं चापल्याच्छीघ्रतां प्रतिदर्शयन्}
{वेगेनातीत्य तरसा पुनरेति मुहुर्मुहुः}


\threelineshloka
{तुण्डेनाभ्यहनच्चैनं कुर्वन्नामापसव्यतः}
{रोरूयन्निव चाप्येनं समाह्वयति वै मुहुः}
{विसिस्मापयिषुः पातैर्दर्शयन्नात्मनः क्रियाम्}


\twolineshloka
{अथ तानि विचित्राणि पतितानीतराणि च}
{दृष्ट्वा प्रमुदिताः काका विनेदुरधिकैः स्वरैः}


\twolineshloka
{हंसाश्चापहसन्ति स्म प्रवदन्ति प्रियाणि च}
{कुर्वाणा विविधान्रावानिच्छन्तो जयमात्मनः}


\twolineshloka
{उत्पत्योत्पत्य च प्राहुर्मुहूर्तमिति चेति च}
{वृक्षाग्रेभ्यः स्थलेभ्यश्च निपतन्त्युत्पतन्ति च}


\twolineshloka
{अवमत्य च हंसं तमिदं काकोऽब्रवीद्वचः}
{योऽसावुत्पतितो हंसः सोऽसावेव प्रहीयते}


\twolineshloka
{अथ हंसस्तु तच्छ्रुत्वा भाषितं पततांवरः}
{विगाह्य हंसो विक्षोभ्य प्रापतत्पश्चिमां दिशम्}


\twolineshloka
{उपर्युपरि वेगेन सागरं मकरालयम्}
{बहुसत्वसमाकीर्णं वीचीभिर्भीमदर्शनम्}


\twolineshloka
{अथाचिरेण राधेय काको वेगादहीयत}
{ततोऽतीव परिश्रान्तः कथञ्चिद्वंसमन्वगात्}


\threelineshloka
{भीश्चैनमाविशत्तीव्रं काकं कर्ण विचेतसम्}
{द्वीपान्द्रुमान्वितान्पश्यन्विश्रमार्थं श्रमातुरः}
{निपतेयं क्कनु श्रान्त इति तस्मिञ्जलार्णवे}


\twolineshloka
{अवगाह्य समुद्रोऽपि बहुसत्वगणालयः}
{महान्भूतशतोद्भूसी नभसस्तु विशिष्यते}


\threelineshloka
{गाम्भीर्यार्द्वि समुद्रस्य न विशेषं कुलाधम}
{दिशश्च नाम्भसां कर्ण समुद्रस्था विदुर्जनाः}
{विपश्चितोप्यपारत्वात्किं पुनः कर्ण वायसः}


\twolineshloka
{अथ हंसोप्यतिक्रामन्मुहूर्तमिव दूरतः}
{अतिक्रम्य च चक्राङ्गः काकं तं समुदैक्षत}


\twolineshloka
{शनकैः परिहीनं तं परिश्रान्तमचेतसम्}
{अवेक्षमाणस्तं काकं प्रत्यागम्य हसन्निव}


\threelineshloka
{तं प्रहस्य च चक्राङ्गः काकं मन्दगतिं तदा}
{हीयमानमथो दृष्ट्वा हंसः प्राह यथार्थवत्}
{उज्जिहीर्षुर्निमज्जन्तं स्मरन्सत्पुरुषव्रतम्}


\twolineshloka
{बहूनि पतितानि त्वमाचक्षाणो मुहुर्मुहुः}
{पतस्यव्याहरन्खेदं ततो गर्ह्यं प्रभाषसे}


\twolineshloka
{जलं नाम पतितं काक यत्त्वं पतसि साम्प्रतम्}
{जलं स्पृशसि पक्षाभ्यां तुण्डेन च पुनः पुनः}


\twolineshloka
{प्रब्रूहि कतमोऽयं ते पातो वर्तति वायस}
{एह्येहि काक शीघ्रं त्वमेष त्वां परिपालये}


\twolineshloka
{स पक्षाभ्यां स्पृशन्नार्तस्तुण्डेन च जलं तदा}
{काको दृढं परिश्रान्तः सहसा निपपात ह}


\twolineshloka
{सागराम्भसि तं दृष्ट्वा पततन्तं दीनचेतसम्}
{म्रियमाणमिदं वाक्यं हंसः काकमुवाच ह}


\twolineshloka
{शतमेकं च पातानां पताम्यहमिति स्म ह}
{श्लाघन्नात्मानमद्य त्वं काक भाषितवानसि}


\twolineshloka
{स त्वमेकशतं पातं पतन्नभ्यधिको मया}
{कथमेवं परिश्रान्तः पतितोऽसि महार्णवे}


\twolineshloka
{प्रत्युवाच ततः काकः सीदमान इदं वचः}
{उपरीतं तदा हंसं प्रसमीक्ष्य प्रसादयन्}


\twolineshloka
{उच्छिष्टान्नेन पुष्टोऽहं दर्पोत्सिक्तः कुसत्वतः}
{सुपर्ण इव चात्मानं ज्ञातवान्पुत्रलाघवात्}


\twolineshloka
{शरणं त्वां प्रपन्नोऽहमुदकान्तमवाप्नुयाम्}
{हंसेदानीं परिश्रान्तमापदो मां समुद्वर}


\twolineshloka
{यदि नाम पुनर्जीवन्गच्छेयं स्वं निवेशनम्}
{नैनं पुनर्हीनमपि क्षेप्ताऽहं खे विचारिणम्}


\threelineshloka
{तमेवं करुणं दीनं विलपन्तमचेतसम्}
{काककाकेति वाशन्तं निमज्जन्तं महार्णवे}
{कृपयाऽभिपरीतो वै कृपां चक्रे दुरात्मनि}


\twolineshloka
{अथ तं पततं दीनं जलक्लिन्नमचेतनम्}
{पद्भ्यामुत्क्षिप्य वेपन्तं पृष्ठमारो पयत्तदा}


\threelineshloka
{स हंसः पृष्ठमारोप्य काकं कर्ण विचेतसम्}
{अविस्मयंस्तदा कर्ण पुनरेव समानयत्}
{तमेव देशं तरसा स्पर्धया पेततुर्यतः}


\threelineshloka
{उत्सृज्य त्वब्रवीन्मैवं पुनः कार्यं त्वयेति ह}
{स्थाप्य चैनं पुनर्द्वीपे समाश्वस्य च विक्लवम्}
{गतो यथेप्सितं देशं हंसो मन इवाशुगः}


\twolineshloka
{एवमुच्छिष्टपुष्टः स काकोऽहंसपराजितः}
{बलं वीर्यं महत्कर्ण त्यक्त्वा क्षान्तिमुपागतः}


\twolineshloka
{उच्छिष्टभोजनः काको यथा वैश्यकुले पुरा}
{एवं त्वमुच्छिष्टभृतो धार्तराष्ट्रैर्न संशयः}


\twolineshloka
{सदृशाञ्श्रेयसश्चापि सर्वान्कर्णावमन्यसे}
{द्रोणद्रौणिकृपैर्गुप्तो भीष्मेणान्यैश्च कौरवैः}


\twolineshloka
{विराटनगरे पार्थमेकं किं नावधीस्तदा}
{यत्र व्यस्ताः समस्ताश्च निर्जिताः स्थ किरीटिनासृगाला इव सिंहेन क्व ते वीर्यं तदा गतम्}


\twolineshloka
{भ्रातरं निहतं दृष्ट्वा समरे सव्यसाचिना}
{पश्यतां कुरवीराणां प्रथमं त्वं पलायितः}


\twolineshloka
{तथा द्वैतवने कर्ण गन्धर्वैः समभिद्रुतान्}
{कुरून्सदारानुत्सृज्य त्वमेवाग्रे पलायितः}


\twolineshloka
{हत्वा जित्वा च गन्धर्वांश्चित्रसेनमुखान्रणे}
{कर्ण दुर्योधनं पार्थः सभ्रातृकममोक्षयत्}


\twolineshloka
{पुनः प्रभावः पार्थस्य पुराणः केशवस्य च}
{कथितः कर्ण रामेण सभायां कुरुसंसदि}


\twolineshloka
{सततं च त्वमश्रौषीर्वचनं द्रोणभीष्मयोः}
{अवध्यौ वदतोः कृष्णौ सन्निधौ च महीक्षिताम्}


\twolineshloka
{कियन्तं तत्र वक्ष्यामि येनयेन धनञ्जयः}
{त्वत्तोऽतिरिक्तः सर्वेभ्यो भूतेभ्यो ब्राह्मणो यथा}


\twolineshloka
{इदानीमेव द्रष्टासि प्रधाने स्यन्दने स्थितौ}
{पुत्रं च वसुदेवस्य कुन्तीपुत्रं च पाण्डवम्}


\twolineshloka
{यथाऽश्रयत चक्राङ्गं वायसो बुद्धिमास्थितः}
{तथाश्रयस्व वार्ष्णेयं पाण्डवं च धनञ्जयम्}


\twolineshloka
{यदा त्वं युधि विक्रान्तौ वासुदेवधनञ्जयौ}
{द्रष्टास्येकरथे कर्ण तदा नैवं वदिष्यसि}


\twolineshloka
{यदा शरशतैः पार्थो दर्पं तव वधिष्यति}
{तदा त्वमन्तरं द्रष्टा आत्मनश्चार्जुनस्य च}


\twolineshloka
{देवासुरमनुष्येषु प्रख्यातौ यौ नरोत्तमौ}
{तौ मावमंस्था मौर्ख्यात्त्वं स्वद्योत इव रोचनौ}


\twolineshloka
{सूर्याचन्द्रमसौ यद्वत्तद्वदजुनकेशवौ}
{प्रकाश्येनाभिविख्यातौ त्वं तु खद्योतवन्नृषु}


\twolineshloka
{एवं विद्वन्मावमंस्थाः सूतपुत्राच्युतार्जुनौ}
{नरसिंहौ नरश्वा त्वं जोषमास्स्व विकत्थसे}


\chapter{अध्यायः ४०}
\twolineshloka
{सञ्जय उवाच}
{}


\twolineshloka
{मद्राधिपस्याधिरथिर्महात्मावचो निशम्याप्रियमप्रतीतः}
{उवाच शल्यं विदितं ममैत--द्यथाविधावर्जुनवासुदेवौ}


\twolineshloka
{शौरे रथं वाहयतोऽर्जुनस्यबलं महास्त्राणि च पाण्डवस्य}
{अहं विजानामि यथावदद्यपरोक्षभूतं तव तत्तु शल्य}


\twolineshloka
{तौ चाप्यहं शस्त्रभृतां वरिष्ठौव्यपेतभीर्योधयिष्यामि कृष्णौ}
{सन्तापयत्यभ्यधिकं हि रामा--च्छापोऽद्य मां ब्राह्मणसत्तमाच्च}


\twolineshloka
{`पुरा महेन्द्राद्रिवरे समुद्रेतपस्विनं राममुपेत्य शल्य}
{अस्त्रार्थिनं माऽद्य शिष्यं गृहाणे--त्यथाऽब्रुवं ब्राह्मणच्छद्मना च}


\twolineshloka
{तत्रावसं ब्राह्मण इत्यविप्रोब्रह्मास्त्रलोभादनृतेन चाहम्}
{तज्जामदग्न्येन परं महास्त्रंसमन्त्रयुक्तं विहितं ममासीत्}


\twolineshloka
{अस्त्रं यदा तद्विदितं ममासी--त्तदाऽब्रवीद्ब्राह्मणो मां महर्षिः}
{आपद्गतेनास्त्रमिदं प्रयोज्यंत्वया रणे गच्छता साधयेति'}


\twolineshloka
{तत्रापि मे देवराजेन विघ्नोहितार्थिना फल्गुनस्यैव शल्य}
{कृतो विभेदेन ममोरुमेत्यप्रविश्य कीटस्य तनुं विरूपाम्}


\twolineshloka
{ममोरुमेत्य प्रबिभेद कीटःसुप्ते गुरौ तत्र शिरो निधाय}
{ऊरुप्रभेदाच्च महान्बभूवशरीरतो मे घनशोणितौघः}


\twolineshloka
{गुरोर्भयाच्चापि न चेलिवानहंततो विबुद्धो ददृशे स विप्रः}
{स धैर्ययुक्तं प्रसमीक्ष्य मां वैन त्वं विप्रः कोऽसि सत्यं वदेति}


\twolineshloka
{तस्मै तदाऽऽत्मानमहं यथाव--दाख्यातवान्सूत इत्येव शल्य}
{स मां निशाम्याथ महातपस्वीसंशप्तवान्रोषपरीतचेताः}


\twolineshloka
{सूत त्वया ह्याप्तमिदं तवास्त्रंन कर्मकाले प्रतिभास्यतीति}
{अन्यत्र तु स्यात्तव मृत्युकाला--दब्राह्मणे ब्रह्म न हि ध्रुवं स्यात्}


% Check verse!
तदद्य पर्याप्तमतीव मेऽस्त्र--मुपस्थितेऽस्मिंस्तुमुले विमर्दे
\twolineshloka
{शल्योग्रधन्वानमहं वरिष्ठंतरस्विनं भीममसह्यवीर्यम्}
{सत्यप्रतिज्ञं युधि पाण्डवेयंधनञ्जयं मृत्युमुखं नयिष्ये}


\threelineshloka
{अस्त्रं ततोऽन्यत्प्रतिपन्नमद्ययेन क्षेप्स्ये समरे शत्रुपूगान्}
{प्रतापिनं बलवन्तं कृतास्त्रंतमुग्रधन्वानममितौजसं च}
{क्रूरं शूरं रौद्रममित्रसाहंधनञ्जयं संयुगेऽहं हनिष्ये}


\twolineshloka
{अपांपतिर्वेगवानप्रमेयोनिमज्जयिष्यन्बहुलाः प्रजा इव}
{महारवं यः कुरुते समुद्रोवेलेव तं वारयाम्यप्रमेयम्}


\twolineshloka
{प्रमुञ्चन्तं बाणसङ्घानमेया--न्मर्मच्छिदो वीरहणः सुपत्रान्}
{कुन्तीपुत्रं प्रतियोत्स्यामि युद्धेज्यां कर्षतामुत्तमं मर्त्यलोके}


\twolineshloka
{एवं बलेनातिबलं महास्त्रंसमुद्रकल्पं सुदुरापमुग्रम्}
{शरौघिणं पार्थिवान्मज्जयन्तंवेलेव पार्थमिषुभिर्वारयिष्ये}


\twolineshloka
{कृती कृतास्त्रो दृढहस्तयोधीदिव्यास्त्रविच्छ्वेतहयः प्रमाथी}
{सुरासुरान्युधि वै यो जयेततेनाद्य मे पश्य युद्वं सुघोरम्}


\twolineshloka
{अभीर्मानी पाण्डवो युद्वकामोह्यमानुषैरस्यति मां महास्त्रैः}
{तस्यास्त्रमस्त्रैरभिभूय सङ्ख्येबाणोत्तमैः पातयिष्यामि पार्थम्}


\twolineshloka
{सहस्ररश्मिप्रतिमं ज्वलन्तंदिशश्च सर्वाः प्रतपन्तमुग्रम्}
{तमोनुदं मेघ इवातिमात्रंधनञ्जयं छादयिष्यामि बाणैः}


\twolineshloka
{वैश्वानरं धूमकेतुं ज्वलन्तंतेजस्विनं नरवीरान्दहन्तम्}
{पर्जन्यभूतः शरवर्षैर्यथाऽग्निंतथा पार्थं शमयिष्यामि युद्धे}


\twolineshloka
{आशीविषं दृष्टिहणं सुघोरंसुतीक्ष्णदंष्ट्रं ज्वलनप्रभवाम्}
{क्रोधात्प्रदीप्तानलवद्दहन्तंकुन्तीपुत्रं शमयिष्यामि भल्लैः}


\threelineshloka
{प्रवाहिणं बलवन्तं महौजसंप्रभञ्जनं मातरिश्वानमुग्रम्}
{युद्धे सहिष्ये हिमवानिवाचलोधनञ्जयं क्रुद्धममृष्यमाणम्}
{}


\twolineshloka
{विशारदं रथमार्गेषु शक्तंधुर्यं नित्यं समरेषु प्रवीरम्}
{लोके वरं सर्वधनुर्धराणांधनञ्जयं संयुगेऽहं हनिष्ये}


\twolineshloka
{अद्याहवे यस्य न तुल्यमन्यंमन्ये मनुष्यं धनुराददानम्}
{सर्वामिमां यः पृथिवीं सहेततथापि तेनाद्य रणे समेष्ये}


% Check verse!
यः सर्वभूतानि सदैवतानिप्रस्थेऽजयत्खाण्डवे सव्यसाचीको जीवितं रक्षमाणो हि तेनयुयुत्सतेऽस्त्रैर्मानुषो मामृतेऽन्यः
\threelineshloka
{मानी कृतास्त्रः कृतहस्तयोधीदिव्यास्त्रविच्छ्वेतहयः प्रमाथी}
{तस्याहमद्यातिरथस्य काया--च्छिरो हरिष्यामि शितैः पृषत्कैः}
{}


\twolineshloka
{योत्स्याम्येनं शल्य धनञ्जयं वैमृत्युं पुरस्कृत्य रणे जयं वा}
{अन्यो हि न ह्येकरथेन मर्त्योयुध्येत यः पाण्डवमिन्द्रकल्पम्}


% Check verse!
तस्याहवे पौरुषं पाण्डवस्यब्रूयां पृष्टः समितौ क्षत्रियाणाम्किं त्वं मूर्खः प्रहसन्मूढचेताआख्यासि मे पौरुषं फल्गुनस्य
\twolineshloka
{अत्यप्रियो यः पुरुषो निष्ठुरो हिक्षुद्रः क्षेप्ता क्षमिणश्चाक्षमावान्}
{हन्यामहं त्वादृशानां शतानिक्षमाम्यहं क्षमिणां काल एषः}


\twolineshloka
{अवोचस्त्वं पाण्डवार्ये प्रियाणिप्रधर्षयन्मां मूढवत्पापकर्मन्}
{मय्यार्जवे जिह्ममतिर्यतस्त्वंमित्रद्रोही साप्तपदं हि मित्रम्}


\twolineshloka
{कालस्त्वयं प्रत्युपयाति दारुणोदुर्योधनो युद्धमुपागमद्यत्}
{तस्यार्थसिद्धिं त्वभिकाङ्क्षमाण--स्तमन्वेष्ये यत्र चैकान्तमस्ति}


\twolineshloka
{`तथाप्यहं पाण्डववासुदेवौयोत्स्ये रणे मद्विधस्यैव कर्म}
{न प्राकृतः सज्जते वै कदाचि--द्यः प्रत्युदीयात्कृष्णधनञ्जयौ तौ'}


\twolineshloka
{मित्रं मिन्देर्नन्दतेः प्रीयतेर्वासन्त्रायतेर्मिनुतेर्मोदतेर्वा}
{दुर्योधने सर्वमिदं ममास्तितच्चापि सर्वं मम वेत्ति राजा}


\twolineshloka
{शत्रुः शदेः शासतेर्वा श्यतेर्वाशृणातेर्वा श्वसतेः सीदतेर्वा}
{श्रमेः शुचो बहुशः सूदतेश्चप्रायेण सर्वं त्वयि तच्च मह्यम्}


\twolineshloka
{दुर्योधनार्थं तव च प्रियार्थंयशोर्थमात्मार्थमपीश्वरार्थम्}
{तस्मादहं पाण्डववासुदेवौयोत्स्ये यत्नात्कर्म च पश्य मेऽद्य}


\twolineshloka
{`शल्याद्याहं सङ्गतः पाण्डवेनमुच्येयं चेज्जीवमानः कथञ्चित्}
{शश्वन्मृत्योः स्यामनाधृष्यरूपोव्यक्तं तस्मात्संयुगाद्विप्रमुक्ताः'}


\twolineshloka
{अस्त्रं ब्राह्मं मनसा सञ्जपन्वैयदाऽस्यते क्रोधितः सव्यसाची}
{तदापि मे नैव मुच्येत पार्थोन चेत्पतेद्विषमे मेऽद्य चक्रम्}


\twolineshloka
{अस्त्राणि विद्ध्वा समरे गतानिब्राह्माणि दिव्यान्यथ मानुषाणि}
{आसादयिष्याम्यहमुग्रवीर्यंनागोत्तमं नाग इव प्रभिन्नः}


\twolineshloka
{वैवस्वताद्दण्डहस्ताद्वरुणाद्वापि पाशिनः}
{सगदाद्वा धनपतेः सवज्राद्वा सुराधिपात्}


\twolineshloka
{अथान्यस्मादपि सुरादमित्रादाततायिनः}
{इति शल्य विजानीहि यथा नाहं बिभेम्युत}


\twolineshloka
{तस्मान्न मे भयं पार्थान्नापि चैव जनार्दनात्}
{सह युद्धं हि मे ताभ्यां साम्पराये भविष्यति}


\twolineshloka
{श्वभ्रे ते पततां चक्रमिति मां ब्राह्मणोऽब्रवीत्}
{युध्यमानस्य सङ्ग्रामे प्राप्तस्यैकायनं भयम्}


\twolineshloka
{तस्माद्बिभेमि बलवद्ब्राह्मणव्याहृतादहम्}
{एते हि सोमराजान ईश्वराः सुखदुःखयोः}


\twolineshloka
{कदाचिब्राह्मणस्याथ योग्यहेतेरहं नृप}
{अजानन्नक्षिपं बाणं घोररूपं भयावहम्}


\twolineshloka
{होमधेन्वास्ततो वत्सः प्रमत्त इषुणा हतः}
{चरन्वै विजने शल्य ततोऽनु व्याजहार सः}


\twolineshloka
{यस्माद्वत्सस्त्वया चात्र होमधेन्वा हतो नृप}
{तस्मात्त्वमपि राधेय वाक्शल्यं महदाप्नुहि}


\twolineshloka
{श्वभ्रे ते पतिता चक्रं युध्यमानस्य शत्रुणा}
{प्राप्त एकायने काले मृत्युसाधारणे त्वयि}


\twolineshloka
{स्पर्धसे येन सङ्ग्रामे यदर्थं घटसेऽनिशम्}
{तत एव ध्रुवं मृत्युं सूत प्राप्स्यसि संयुगे}


% Check verse!
अहं प्रसादयाञ्चक्रे ब्राह्मणं संशितव्रतम्
\twolineshloka
{गवां दशशतं वित्तं बलीवर्दांश्च षट््शतम्}
{प्रच्छन्नं काञ्चनैः कामं ब्राह्मणार्थमहं तदा}


\twolineshloka
{दासीशतं निष्ककण्ठं शतमश्वतरीरथान्}
{कन्यानां निष्ककण्ठीनां सहस्रं समलंकृतम्}


\twolineshloka
{ईषादन्तान्नागशतान्दासीदासशतानि च}
{दद्मि तैर्द्विजमुख्यो मे प्रसादं न चकार सः}


\twolineshloka
{कृष्णानां श्वेतवत्सानां गोशतानि चतुर्दश}
{ददन्हि न लभे तस्मात्प्रसादं द्विजसत्तमात्}


\threelineshloka
{यत्किञ्चिन्मामकं वित्तं त्वदधीनं करोमि तत्}
{इति मां याचमानं वै ब्राह्मणः प्रत्यवारयत्}
{क्रोधदीप्तेक्षणः शळ्य निर्दहन्निव चक्षुषा}


\twolineshloka
{व्याहृतं यन्मया सूत तत्तथा न तदन्यथा}
{अनृतोक्तं प्रजां हन्यात्ततः पापमवाप्नुयाम्}


\twolineshloka
{तस्माद्धर्माभिरक्षार्थं नानृतं वक्तुमुत्सहे}
{मा त्वं ब्रह्मगतिं हिंस्याः प्रायश्चित्तं कृतं त्वया}


\twolineshloka
{मद्वाक्यं नानृतं लोके कश्चित्कुर्यात्समाप्नुहि}
{`यन्मयोक्तं सरोषेण गच्छ सूतज माचिरम्}


\twolineshloka
{इति मामसकृत्क्रुद्धः स उवाच द्बिजोत्तमः}
{एते हि सोमराजान ईश्वराः सुखदुःखयोः}


\twolineshloka
{नाहं बिभेमि बीभत्सोर्न शल्य मधुसूदनाम्}
{तस्माद्बिभेम्यहं शापात्तेन सत्येन ते शपे}


\twolineshloka
{सह युद्धं समेताभ्यामद्येदं समुपस्थितम्}
{युद्वेऽस्मिञ्जीवितं मेऽद्य शल्य संशयमागतम्}


\twolineshloka
{शक्रोऽप्यमरराट् ताभ्यामुपगम्याहवं सह}
{संशयं परमं गच्छेत्कथं वा मन्यते भवान्}


\twolineshloka
{इत्येवं ते मयाऽऽख्यातं क्षिप्तेन न सुहृत्तया}
{जानामि त्वाऽधिक्षिपन्तं दोषमात्मगतं शृणु'}


\chapter{अध्यायः ४१}
\twolineshloka
{सञ्जय उवाच}
{}


\twolineshloka
{ततः पुनर्महाराज मद्रराजमरिन्दमः}
{अभ्यभाषत राधेयः सन्निवार्योत्तरं वचः}


\twolineshloka
{निर्भर्त्सनार्थं शल्य त्वं यत्तु जल्पितवानसि}
{नाहं शक्यस्त्वया वाचा विभीषयितुमाहवे}


\twolineshloka
{यदि मां देवताः सर्वा योधयेयुः सवासवाः}
{तथापि मे भयं न स्यात्किमु पार्थात्सकेशवात्}


\twolineshloka
{नाहं भीषयितुं शक्यः शुद्धकर्णा महाहवे}
{अभिजानीहि शक्तस्त्वमन्यं भीषयितुं रणे}


\twolineshloka
{नीचस्य बलमेतावत्पारुष्यं यतत्त्वमात्थ माम्}
{अशक्तो हि गुणान्वक्तुं वल्गसे बहु दुर्मते}


\twolineshloka
{न हि कर्णः समुद्भूतो भयार्थमिह मद्रक}
{विक्रमार्थमहं जातो यशोर्थं च तथाऽऽत्मनः}


\twolineshloka
{सखिभावेन सौहार्दान्मितत्रभावेन चैव हि}
{कारणैस्त्रिभिरेतैस्त्वं शल्य जीवसि साम्प्रतम्}


\twolineshloka
{राज्ञश्च धार्तराष्ट्रस्य कार्यं सुमहदुद्यतम्}
{मयि तच्चाहितं शल्य तेन जीवसि मे क्षणम्}


\twolineshloka
{कृतश्च समयः पूर्वं क्षन्तव्यं विप्रियं तव}
{`समयः परिपाल्यो मे तेन जीवसि साम्प्रतम्'}


\threelineshloka
{ऋते शल्यसहस्रेण विजयेयमहं परान्}
{मित्रद्रोहस्तु पापीयानिति जीवसि साम्प्रतम् ॥शल्य उवाच}
{}


\threelineshloka
{आर्तप्रलापांस्त्वं कर्ण यान्ब्रवीषि परान्प्रति}
{न ते कर्णसहस्रेण शक्या जेतुं परे युधि ॥सञ्जय उवाच}
{}


\twolineshloka
{तथा ब्रुवन्तं परुषं कर्णो मद्राधिपं तदा}
{परुषं द्विगुणं भूयः प्रोवाचाप्रियदर्शनम्}


\twolineshloka
{इदं त्वमेकाग्रमनाः शृणु मद्रजनाधिप}
{सन्निधौ धृतराष्ट्रस्य प्रोच्यमानं मया श्रुतम्}


\twolineshloka
{देशांश्च विविधांश्चित्रान्पूर्ववृत्तांश्च पार्थिवान्}
{ब्राह्मणाः कथयन्ति स्म धृतराष्ट्रनिवेशने}


\twolineshloka
{तत्र वृद्धः पुरावृत्ताः कथाः कश्चिद्द्विजोत्तमः}
{बाह्लीकदेशं मद्रांश्च कुत्सयन्वाक्यमब्रवीत्}


\twolineshloka
{बहिष्कृता हिमवतता गङ्गया च तिरस्कृताः}
{सरस्वत्या यमुनया कुरुक्षेत्रेण चापि ये}


\twolineshloka
{पञ्चानां सिन्धुषष्ठानां नदीनां येऽन्तराश्रिताः}
{तान्धर्मबाह्यानशुचीन्बाह्लीकानपि वर्जयेत्}


\twolineshloka
{गोवर्धनो नाम वटः सुभाण्डं नाम पत्तनम्}
{एतद्राजन्कलिद्वारमाकुमारात्स्मराम्यहम्}


\twolineshloka
{कार्येणात्यर्थगूढेन बाह्लीकेषूषितं मया}
{तत एषां समाचारः संवासाद्विदितो मया}


\twolineshloka
{शाकलं नाम नगरमापगानामनिम्नगा}
{चण्डाला नाम बाह्लीकास्तेषां वृत्तं सुनिन्दितम्}


\twolineshloka
{पानं गुडासवं पीत्वा गोमांसं लशुनैः सह}
{अपूपसक्तुमद्यानामाशिताः शीलवर्जिताः}


\twolineshloka
{गायन्त्यथ च नृत्यन्ति स्त्रियो मत्ता विवाससः}
{नगरापणवेशेषु बहिर्माल्यानुलेपनाः}


\twolineshloka
{मत्ताः प्रगीतविरुतैः खरोष्ट्रनिनदोपमैः}
{अनावृता मैथुने ताः कामचाराश्च सर्वशः}


\twolineshloka
{आहुरन्योन्यसूक्तानि प्रब्रुवाणा मदोत्कxxxः}
{हेहतेहेहतेत्येवं स्वामिभर्तृहतेति च}


\twolineshloka
{आक्रोशन्त्यः प्रनृत्यन्ति व्रात्याः पर्वस्वसंयताः}
{तासां किलावलिप्तानां निवसन्कुरुजाङ्गले}


\twolineshloka
{कश्चिद्बाह्लीकदुष्टानां नातिहृष्टमना जगौ}
{सा नूनं बृहती गौरी सूक्ष्मकम्बलवासिनी}


\threelineshloka
{मामनुस्मरती शेते बाह्लीकं कुरुवासिनम्}
{शतद्रुं नु कदा तीर्त्वा तां च रम्यामिरावतीम्}
{गत्वा स्वदेशं द्रक्ष्यामि स्थूलजङ्घाः शुभाः स्त्रियः}


\twolineshloka
{मनःशिलोज्ज्वलापाङ्ग्यो गौर्यस्ताः काकुकूजिताः}
{कम्बलाजिनसंवीता रुदन्त्यः प्रियदर्शनाः}


\twolineshloka
{मृदङ्गानकशङ्खानां मर्दलानां च निःस्वनैः}
{खरोष्ट्राश्वतरैश्चैव मत्ता यास्यामहे सुखम्}


\twolineshloka
{शमीपीलुकरीराणां वनेषु सुखवर्त्मसु}
{अपूपान्सक्तुपिण्डांश्च प्राश्नन्तो मथितान्वितान्}


\twolineshloka
{पथिषु प्रबलो भूत्वा तथा सम्पततोऽध्वगान्}
{चेलापहारं कुर्वाणास्ताडयिष्याम भूयसः}


\threelineshloka
{एवंशीलेषु व्रात्येषु बाह्लीकेषु दुरात्मसु}
{कश्चेतयानो निवसेन्मुहूर्तमपि मानवः ॥कर्ण उवाच}
{}


\twolineshloka
{ईदृशा ब्राह्मणेनोक्ता बाह्लीका मोघचारिणः}
{येषां षड्भागहर्ता त्वमुभयोः पुण्यपापयोः}


\twolineshloka
{इत्युक्त्वा ब्राह्मणः साधुरुत्तरं पुनरुक्तवान्}
{बाह्लीकेष्वविनीतेषु प्रोच्यमानं निबोध तत्}


\twolineshloka
{ततत्र स्म राक्षसी गाति कृष्णचतुर्दशीम्}
{नगरे शाकले स्फीते आहत्य निशि दुन्दुभिम्}


\twolineshloka
{कथं वस्तादृशो गाथाः पुनर्गास्यामि शाकले}
{गव्यस्य तृप्ता मांसस्य पीत्वा गौडं सुरासवम्}


\twolineshloka
{गौरीभिः सह नारीभिर्बृहतीभिः स्वलङ्कृता}
{पलाण्डुगण्डूषयुताः खादन्तश्चेशिकान्बहून्}


\twolineshloka
{वाराहं कौक्कुटं मांसं गव्यं गार्दभमौष्ट्रकम्}
{धानाश्च ये न खादन्ति तेषां जन्म निरर्थकम्}


\twolineshloka
{एवं गायन्ति ये मत्ताः शीधुना पीलुकावने}
{सबालवृद्धाः क्रन्दन्ति तेषां धर्मः कथं भवेत्}


\twolineshloka
{यत्र लोकेश्वरः कृष्णो देवदेवो जनार्दनः}
{विस्मृतः पुरुषैरुग्रैस्तेषां धर्मः कथं भवेत्}


\twolineshloka
{इति शल्य विजानीहि हन्त भूयो ब्रवीमि ते}
{यदन्योप्युक्तवानस्मान्ब्राह्मणः कुरुसंसदि}


\twolineshloka
{पञ्चनद्यो वहन्त्येता यत्र पीलुवनान्युत}
{शतद्रुश्च विपाशा च तृतीयैरावती तथा}


\twolineshloka
{चन्द्रभागा वितस्ता च सिन्धुषष्ठा महानदी}
{आरट्टा नाम बह्लीका एतेष्वार्यो हि नो वसेत्}


\twolineshloka
{व्रात्यानां दासमीयानां विदेहानामयज्वनाम्}
{न देवाः प्रतिगृह्णन्ति पितरो ब्राह्मणास्तथा}


\twolineshloka
{तेषां प्रनष्टधर्माणां बाह्लीकानामिति श्रुतिः}
{ब्राह्मणेन यथा प्रोक्तं विदुषा साधुसंसदि}


\twolineshloka
{काष्ठकुण्डेषु बाह्लीका मृण्मयेषु च भुञ्जते}
{सक्तुमद्यावलिप्तेषु श्वावलीढेषु निर्घृणाः}


\twolineshloka
{आविकं चौष्ट्रिकं चैव क्षीरं गार्दभमेव च}
{तद्विकारांश्च बाह्लीकाः खादन्ति च पिबन्ति च}


\twolineshloka
{पात्रसङ्करिणो जाल्माः सर्वान्नक्षीरभोजनाः}
{आरट्टा नाम बाह्लीका वर्जनीया विपश्चिता}


\twolineshloka
{इति शल्य विजानीहि हन्त भूयो ब्रवीमि ते}
{यदन्योऽप्युक्तवान्मह्यं ब्राह्मणः कुरुसंसदि}


\twolineshloka
{युगन्धरे पयः पीत्वा प्रोष्य चाप्यच्युतस्थले}
{तद्वद्भूतिलये स्नात्वा कथं स्वर्गं गमिष्यति}


\twolineshloka
{पञ्चनद्यो वहन्त्येता यत्र निःसृत्य पर्वतात्}
{आरट्टा नाम बाह्लीका न तेष्वार्यो द्व्यहं वसेत्}


\twolineshloka
{बाह्लीका नाम हीकश्च विपाशायां पिशाचकौ}
{तयोरपत्यं बाह्लीका नैषा सृष्टिः प्रजापतेः}


% Check verse!
ते कथं विविधान्धर्माञ्ज्ञास्यन्ते हीनयोनयः
\twolineshloka
{कारस्करान्माहिषकान्करम्भान्कटकालिकान्}
{कर्करान्वीरकान्वीरा उन्मत्तांश्च विवर्जयेत्}


\twolineshloka
{इति तीर्थानुसञ्चारी राक्षसी काचिदब्रवीत्}
{एकरात्रमुषित्वेह महोलूखलमेखला}


\twolineshloka
{आरट्टा नाम ते देशा बाह्लीकं नाम तद्वनम्}
{वसातिसिन्धुसौवीरा इति प्रायोऽतिकुत्सिताः}


\chapter{अध्यायः ४२}
\twolineshloka
{कर्ण उवाच}
{}


\twolineshloka
{इति शल्य विजानीहि हन्त भूयो ब्रवीमि ते}
{उन्यमानं मया सम्यक्त्वमेकाग्रमनाः शृणु}


\twolineshloka
{ब्राह्मणः किल नो गेहमध्यगच्छत्पुराऽतिथिः}
{आचारं तत्र सम्प्रेक्ष्य प्रीतो वचनमब्रवीत्}


\twolineshloka
{मया हिमवतः शृङ्गमेकेनाध्युषितं चिरम्}
{दृष्टाश्च बहवो देशा नानाधर्मसमावृताः}


\twolineshloka
{न च केन च धर्मेण विरुध्यन्ते प्रजा इमाः}
{सर्वाश्चाप्याचरन्धर्मं यदुक्तं वेदपारगैः}


\twolineshloka
{अटता तु मया देशान्नानाधर्मसमाकुलान्}
{आगच्छता महाराज बाह्लीकेषु निशामितम्}


\twolineshloka
{तत्र वै ब्राह्मणो भूत्वा पुनर्भवति क्षत्रियः}
{वैश्यः शूद्रश्च बाह्लीकस्ततो भवति नापितः}


\twolineshloka
{नापितश्च ततो भूत्वा पुनर्भवति ब्राह्मणः}
{द्विजो भूत्वा च तत्रैव पुनर्दासोऽभिजायते}


\twolineshloka
{भवन्त्येककुले विप्राः शिष्टा ये कामचारिणः}
{गान्धारा मद्रकाश्चैव बाह्लीकाश्चाप्यतेजसः}


\twolineshloka
{एतन्मया श्रुतं तत्र धर्मसङ्करकारकम्}
{कृत्स्नामटित्वा पृथिवीं बाह्लीकेषु विपर्ययः}


\twolineshloka
{इति शल्य विजानीहि हन्त भूयो ब्रवीमि ते}
{यदप्यन्योऽब्रवीद्वाक्यं बाह्लीकानां विकुत्सितं}


\twolineshloka
{आनीयेहाक्षता काचिदारट्टात्किल दस्युभिः}
{अधर्मतश्चोपयाता सा तानभ्यशपत्ततः}


\twolineshloka
{बालां बन्धुमतीं यन्मामधर्मेणोपगच्छथ}
{तस्मात्कुमार्यः स्वैरिण्यो भविष्यन्ति कुलेषु वः}


\twolineshloka
{न चैवास्मात्प्रमोक्षध्वं घोराच्छापान्नराधमाः}
{तस्मात्तेषां भागहरः कथं धर्मान्वदिष्यसि}


\twolineshloka
{कुरवः सहपाञ्चालाः साल्वा मात्स्याः सनैमिशाः}
{कोसलाः काशपौण्ड्राश्च कालिङ्गा मागधास्तथा}


\twolineshloka
{चेदयश्च महाभागा धर्मं जानन्ति शाश्वतम्}
{नानादेशेष्वसन्तश्च प्रायो बाह्यालयानृते}


\twolineshloka
{आमत्स्येभ्यः कुरुपाञ्चालसाल्वाआनैमिशाच्चेदयो ये विशिष्टाः}
{धर्मं पुराणमुपजीवन्ति सन्तोमद्रानृते पाञ्चनदांश्च जिह्मान्}


\twolineshloka
{एवं विद्वान्धर्मकथासु राजं--स्तूष्णींभूतः शल्य भवेः सदा त्वम्}
{त्वं तस्य गोप्ता च जनस्य राजाषड्भागहर्ता शुभदुष्कृतस्य}


\twolineshloka
{अथवा दुष्कृतस्य त्वं हर्ता तेषामरक्षिता}
{रक्षिता पुण्यभाग्राजा प्रजानां त्वं ह्यपुण्यभाक्}


\twolineshloka
{पूज्यमाने पुरा धर्मे सर्वदेशेषु शाश्वते}
{धर्मं पाञ्चनदं दृष्ट्वा धिगित्याह पितामहः}


\twolineshloka
{व्रात्यानां दाशकीयानां कृतेऽप्यशुभकर्मणाम्}
{ब्रह्मणा निन्दितान्धर्मान्कश्चित्सिद्धात्मकोऽब्रवीत्}


\twolineshloka
{इति पाञ्चनदं धर्मवमेने पितामहः}
{स्वधर्मस्थेषु वर्णेषु सोऽप्येतान्नाभ्यपूजयत्}


\twolineshloka
{इति शल्य विजानीहि हन्त भूयो ब्रवीमि ते}
{कल्माषपादः सरसि निम़ज्जन्राक्षसोऽब्रवीत्}


\twolineshloka
{क्षत्रियस्य मलं भैक्ष्यं ब्राह्मणस्याश्रुतं मलम्}
{मलं पृथिव्या बाह्लीकाः स्त्रीणां कौतूहलं मलम्}


\twolineshloka
{निमज्जमानमुद्धृत्य कश्चिद्राजा निशाचरम्}
{अपृच्छत्तेन चाख्यातं तच्छृणुष्व नराधिप}


\twolineshloka
{मानुषाणां मलं म्लेच्छा म्लेच्छानां मुष्टिका मलम्}
{मुष्टिकानां मलं षण्ढाः षण्ढानां राजयाजकाः}


\twolineshloka
{राजयाजकयाज्यानां मद्रकाणां च यन्मलम्}
{तद्भवेद्वै तव मलं यद्यस्मान्न विमुञ्चसि}


\twolineshloka
{इति रक्षोपसृष्टेषु विषवीर्यहतेषु च}
{विद्वद्भिर्भेषजं दृष्टं संसिद्धवचनोत्तरम्}


\twolineshloka
{ब्राहयाः पञ्चालाः कौरवेयास्तु धर्म्याःसत्या मत्स्याः शूरसेनाश्च याज्याः}
{प्राच्या दासा वृषला दाक्षिणात्याःस्तेना बाह्लीकाः सङ्कार वै सुराष्ट्राः}


\twolineshloka
{कृतघ्नता परवित्तापहारोमद्यपानं गुरुदारावमर्दः}
{वाक्पारुष्यं गोवधो रात्रिचर्याबहिर्गेहं परवस्त्रोपभोगः}


\threelineshloka
{येषां धर्मस्तान्प्रति नास्त्यधर्मोह्यारट्टजान्पाञ्चनदान्धिगस्तु}
{आपाञ्चालाः कुरवो नैमिशाश्चमत्स्याश्चैवाप्यनुजानन्ति धर्मम्}
{अथोदीच्याश्चोदयो भागधाश्चशिष्टान्धर्मानुपजीवन्ति वृद्धाः}


\twolineshloka
{प्राचीं दिशं श्रिता देवा जातवेदःपुरोगमाः}
{दक्षिणा पितृभिर्गुप्ता यमेन शुभकर्मणा}


\twolineshloka
{प्रतीचीं वरुणः पाति पालयन्नसुरान्बहून्}
{उदीचीं भगवान्सोम ब्रह्मा च ब्राह्मणैः सह}


\twolineshloka
{रक्षःपिशाचान्हिमवान्गुह्यकान्गन्धमादनः}
{ध्रुवं सर्वाणि भूतानि विष्णुः पाति सुरोत्तमः}


\twolineshloka
{इङ्गित5आश्च मगधाः प्रेक्षितज्ञाश्च कोसलाः}
{अर्धोक्ताः कुरुपाञ्चालाः सर्वोक्ता दाक्षिणापथाः}


\threelineshloka
{पार्वतीयाश्च विषया यथैव गिरयस्तथा}
{सर्वज्ञा यवना राजञ्शूराश्चैव विशेषतः}
{म्लेच्छाः स्वसंज्ञानियता नानुक्तमितरे जनाः}


\twolineshloka
{प्रतिरब्धास्तु बाह्लीका न च केचन मद्रकाः}
{स त्वमेतादृशः शल्य नोत्तरं वक्तुमर्हसि}


\threelineshloka
{एवं ज्ञात्वा जोषमास्स्व प्रतीपंमा स्म क्रुद्धः पापभृतां वरिष्ठ}
{पूर्वं हत्वा त्वां सपुत्रं बलैश्चपश्चाद्धंस्ये वासुदेवार्जुनौ च ॥शल्य उवाच}
{}


\twolineshloka
{आतुराणां परित्यागः स्वदारसुतविक्रयः}
{अङ्गे प्रवर्तते कर्ण येषामधिपतिर्भवान्}


\twolineshloka
{रथातिरथसङ्ख्यायां यत्त्वां भीष्मस्तदाब्रवीत्}
{तान्विदित्वाऽत्मनो दोषान्निर्मन्युर्भव मा क्रुधः}


\twolineshloka
{सर्वत्र ब्राह्मणाः सन्ति सन्ति सर्वत्र क्षत्रियाः}
{वैश्याः शूद्रास्तथा कर्ण स्त्रियः साध्व्यश्च सुव्रताः}


\twolineshloka
{रमन्ते चोपहासेन पुरुषाः पुरुषैः सह}
{अन्योन्येन रताः क्षीबा देशेदेशे च मैथुनम्}


\twolineshloka
{परवाच्येषु निपुणः सर्वो भवति सर्वदा}
{आत्मवाच्यं न जानीते जानन्नपि च मुह्यति}


\twolineshloka
{सर्वत्र सन्ति राजनः स्वंस्वं धर्ममनुव्रताः}
{दुर्मनुष्यान्निगृह्णन्ति सन्ति सर्वत्र धार्मिकाः}


\threelineshloka
{न कर्ण देशसामान्यात्सर्वः पापं निषेवते}
{यादृशाः स्वस्वभावेन देशांस्तांस्तादृशान्विदुः ॥सञ्जय उवाच}
{}


\twolineshloka
{तततो दुर्योधनो राजा कर्णशल्याववारयत्}
{सखिभावेन राधेयं शल्यं सौजन्यकेन च}


\twolineshloka
{ततो निवारितः कर्णो धार्तराष्ट्रेण मारिष}
{प्हहस्य तत्र राजेन्द्रं यादि शल्येत्यचोदयत्}


\chapter{अध्यायः ४३}
\twolineshloka
{सञ्जय उवाच}
{}


\twolineshloka
{ततः परानीकभिदं व्यूहमप्रतिमं कृतम्}
{समीक्ष्य कर्णः पार्थानां धृष्टधुम्नाभिरक्षितम्}


\threelineshloka
{प्रययौ रथघोषेण सिंहनादरवेण च}
{वादित्राणां च निनदैः कम्पयन्निव मेदिनीम् ॥सञ्जय उवाच}
{}


\twolineshloka
{ततः प्रयान्तं राधेयं मद्रराजः स्मयन्निव}
{अवधूय इदं वाक्यमब्रवीत्कुरुसन्निधौ}


\twolineshloka
{चमूं तवेमां विपुलां समृद्वा--मसङ्खेयामश्वनराकुलां च}
{तथा प्रवेष्टा समरे धनञ्जयःकक्षं दहन्दीप्त इवाश्रयाशी}


\twolineshloka
{रथे स्तितौ वीरवरौ वरेण्यौसिंहस्कन्धौ लोहितपद्मनेत्रौ}
{द्रष्टा भवानद्य विना प्रयत्ना--त्तथाहि मे शकुना वेदयन्ति}


\twolineshloka
{अद्य द्रष्टाऽसि तं वीरं श्वेताश्वं कृष्णसारथिम्}
{निघ्नन्तं शात्रवान्सङ्ख्ये यं कर्ण परिपृच्छसि}


\twolineshloka
{अद्य तौ पुरुषव्याघ्रौ लोहिताक्षौ परन्तपौ}
{वासुदेवार्जुनौ कर्ण द्रष्टाऽस्यमितविक्रमौ}


\twolineshloka
{सारथिर्यस्य गोविन्दो गाण्डीवं यस्य कार्मुकम्}
{तं चेद्वन्ताऽसि राधेय त्वं नो राजा भविष्यसि}


\twolineshloka
{प्रवात्येष महावायुरभितस्तव वाहिनीम्}
{क्रव्यादा व्याहरन्त्येते मृगाः कुर्वन्ति भैरवम्}


\twolineshloka
{पश्य कर्ण महाघोरं भयं वै रोमहर्षणम्}
{अर्कं जीमूतसङ्काशः कबन्धो वार्य तिष्ठति}


\twolineshloka
{पश्य यूथं मृगशतं मृगाणां सर्वतोदिशम्}
{ररास दीप्तलाङ्गूलमादित्यामभि संस्थितम्}


\twolineshloka
{पश्य काकांश्च गृध्रांश्च समवेतान्सहस्रशः}
{आदित्यमभिवीक्षन्ते ह्यशिवाः कर्ण सस्वराः}


\twolineshloka
{श्वेतवाजिसमायुक्ते तव कर्ण महारथे}
{पताकाः प्रज्वलन्त्येता ध्वजश्चातीव कम्पते}


\twolineshloka
{उदीर्यतो हयान्पश्य महाकायान्महाजवान्}
{प्लवमानान्महावीर्यानाकाशे गरुडानिव}


\twolineshloka
{ध्रुवमेषु निमित्तेषु भूमिमावृत्य पार्थिवाः}
{स्वप्स्यन्ति निहताः कर्ण शतशोऽथ सहस्रशः}


\twolineshloka
{शङ्खानां तुमुलः शब्दः श्रूयते रोमहर्षणः}
{आनकानां च राधेय मृदङ्गानां च हन्यताम्}


\twolineshloka
{शृणु शब्दान्बहुविधान्नराश्वरथवाजिनाम्}
{ज्यातलत्रेषुशब्दांश्च शृणु कर्ण महात्मनाम्}


\twolineshloka
{हेमरूप्यप्रमृष्टानां वाससां शिल्पिनिर्मिताः}
{सहेमचन्द्रताराभाः पताकाः किङ्गिणीयुताः}


\twolineshloka
{नानावर्णा रणे भान्ति श्वसनेन प्रकम्पिताः}
{पश्य कर्णार्जुनस्यैताः सौदामिन्य इवाम्बुदे}


\twolineshloka
{ध्वजाः कणकणायन्ते वातेनातिसमीरिताः}
{सपताकरथाश्चापि पाञ्चालानां महात्मनाम्}


\twolineshloka
{एष रेणुः समुद्भूतो दिवमावृत्य तिष्ठति}
{गजवाजिप्रणुन्ना च कम्पते कर्ण मेदिनी}


\threelineshloka
{श्रूयते मेघसङ्काशो रथनेमिस्वनस्तथा}
{पश्य कुन्तीसुतं वीरं बीभत्सुमपराजितम्}
{प्रहरिष्यन्तमायान्तं कपिप्रवरकेतनम्}


\twolineshloka
{गजाश्वरथपत्त्योघांस्तावकान्युधि निघ्नतः}
{ध्वजाग्रं दृश्यते तस्य ज्याशब्दश्चैव श्रूयते}


\threelineshloka
{एष संशप्तकान्भूयस्तानेवाभिमुखो गतः}
{करोति कदनं चैष सङ्गामे द्विषतां बली ॥सञ्जय उवाच}
{}


% Check verse!
इति ब्रुवाणं मद्रेशं कर्णः प्रोवाच मन्युमान्
\fourlineindentedshloka
{पश्य संशप्तकैः क्रुद्धैः समन्तात्समभिद्रुतः}
{एष सूर्य इवाम्भोदैश्छन्नः पार्थो न दृश्यते}
{एष दान्तोऽर्जुनः शल्य निमग्नः शोकसागरे ॥शल्य उवाच}
{}


\twolineshloka
{वरुणं कोऽम्भसा हन्यादिन्धनेन च कोऽनलम्}
{कोवाऽनिलं निगृह्णीयात्पिबेद्वा को महार्णवम्}


\twolineshloka
{ईदृग्रूपमहं मन्ये पार्थस्य युधि निर्जयम्}
{न हि शक्योऽर्जुनो जेतुं युधि देवासुरैरपि}


\twolineshloka
{अथैघं परितोषस्ते वाग्भिस्त्वं सुमाना भव}
{न हि शक्यो युधा जेतुमन्यं कुरु मनोरथम्}


\twolineshloka
{बाहुभ्यामुद्धरेद्भूमिं दहेत्क्रुद्ध इमाः प्रजाः}
{पातयेत्त्रिदिवाद्देवान्नार्जुनं समरे जयेत्}


\twolineshloka
{पश्य कुन्तीसुतं भीमं वीरमक्लिष्टकारिणम्}
{प्रहरन्तं महाबाहुं स्थितं मेरुमिवाचलम्}


\twolineshloka
{अमर्षी पुरुषव्याघ्रः सदा वैरमनुस्मरन्}
{एष भीमो जयप्रेप्सुर्युधि तिष्ठति वीर्यवान्}


\twolineshloka
{एष धर्मभृतां श्रेष्ठो धर्मराजो युधिष्ठिरः}
{तिष्ठत्यसुकरः सङ्ख्ये परैः परपुरञ्जयः}


\twolineshloka
{एतौ च पुरुषव्याघ्रावाश्विनेयौ महारथौ}
{नकुलः सहदेवश्च तिष्ठतो युधि दुर्जयौ}


\twolineshloka
{एते तिष्ठन्ति कार्ष्णेयाः पञ्च पञ्चाचला इव}
{योत्स्यमाना महावीर्या भीमार्जुनसमा युधि}


\twolineshloka
{एते द्रुपदपुत्राश्च धृष्टद्युम्नपुरोगमाः}
{हीनाः सत्यजिता वीरास्तिष्ठन्ति परमौजसः}


\twolineshloka
{यत्र कृष्णार्जुनौ वीरौ यत्र राजा युधिष्ठिरः}
{तत्र धर्मश्च सत्यं च यतो धर्मस्ततो जयः}


\twolineshloka
{इति संवदतोरेव तयोः पुरुषसिंहयोः}
{ते सेने समसज्जेतां ङ्गायमुनवद्भृशम्}


\chapter{अध्यायः ४४}
\twolineshloka
{सञ्जय उवाच}
{}


\twolineshloka
{दुर्योधनस्ततः कर्णमुपेत्य भरतर्षभ}
{अब्रवीन्मद्रराजं च तथैवान्यान्महारथान्}


\twolineshloka
{यदृच्छयाऽयमव्यग्रो धर्मः परमकः सखा}
{सुखिनः क्षत्रियाः कर्ण लभन्ते युद्धमीदृशम्}


\twolineshloka
{यादृशं क्षत्रियैः शूरैः शूराणां दीव्यतां युधि}
{इष्टं भवति राधेय तदिदं समुपस्थितम्}


\twolineshloka
{हत्वा तु पाण्डवान्युद्धे स्थिरामुर्वीं प्रशासथ}
{निहता वा परैर्युद्धे वीरलोकानवाप्स्यथ}


\twolineshloka
{दुर्योधनस्य वचनं श्रुत्वा तत्क्षत्रियर्षभाः}
{सिंहनादानुदक्रोशन्वादित्राणि च जघ्निरे}


\twolineshloka
{तस्मिन्प्रमुदिते सैन्ये त्वदीये भरतर्षभ}
{हर्षयंस्तावकान्योधान्द्रौणिर्विचनमब्रवीत्}


\twolineshloka
{प्रत्यक्षं सर्वसैन्यानां भवतां चापि पश्यताम्}
{न्यस्तशस्त्रो मम पिता धृष्टद्युम्नेन पातितः}


\twolineshloka
{स तेनाहममर्षेण पित्रर्थे चापि भारत}
{धृष्टद्युम्नमहत्वाऽहं न विमोक्ष्यामि दंशनम्}


\fourlineindentedshloka
{कृत्वाऽनृतां प्रतिज्ञां वो नास्मि प्राप्ता महत्फलम्}
{अर्जुनं भीमसेनं च यश्च मां प्रतियोत्स्यति}
{सर्वांस्तान्प्रमथिष्यामि इति मे निश्चिता मतिः ॥सञ्जय उवाच}
{}


\twolineshloka
{एवमुक्ते ततः सर्वा हर्षिता भारती चमूः}
{अभ्यवर्तत कौन्तेयांस्तथा तां चापि पाण्डवाः}


\twolineshloka
{स सन्निपातो रथयूथपानांमहाहवे भारत लोभनीये}
{जनक्षयः कालयुगान्तकल्पःप्रावर्तताग्रे कुरुसृञ्जयानाम्}


\twolineshloka
{ततः प्रवृत्ते युधि सम्प्रहारेभूतानि सर्वाणि सदैवतानि}
{आसन्समेतानि सहाप्सरोभि--निरीक्षतीभिर्युधि वीरसङ्घान्}


\twolineshloka
{दिव्यैश्च गन्धैः परमैश्च पुष्पै--रन्यैश्च रत्नैर्विविधैर्नराग्र्याः}
{रणेषु कर्मोद्वहताः प्रहृष्टा-ननन्दयन्नप्सरसः प्रहृष्टाः}


\twolineshloka
{समीरणस्तांस्तु निषेव्य गन्धा--न्निषेवते तानपि योधमुख्यान्}
{निषेव्यमाणास्त्वनिलेन योधाःपरस्परं चुक्रुशुराजिमध्ये}


\twolineshloka
{तथा तु तस्मिंस्तुमुले प्रवृत्तेदुर्योधनः क्रोधममृष्यमाणः}
{`अवेभ्य भीमं बलमध्यसंस्थःसमार्पयत्क्षुद्रकाणां शतेन}


\twolineshloka
{दुःशासनश्चित्रसेनश्च वीर--स्तथोलूकः कितवः सौबलश्च}
{गजानीकैः सर्वतो भीमसेनंतथाविषक्तं सहसैवाभ्यगच्छन्}


\twolineshloka
{तमापतन्तं सम्प्रेक्ष्य गजानीकं वृकोदरः}
{दुर्योधनं महाबाहुः शरवर्षैरवाकिरत्}


\twolineshloka
{दुर्योधनं ततो भीमः सायकैर्वज्रसन्निभैः}
{पाण्डवो विमुखीकृत्य गजानभ्यद्रवद्बली}


\twolineshloka
{ततः पावकसङ्काशैर्भीमबाणैरवक्रगैः}
{शलभैरिव नागांस्तानर्दयामास पाण्डवः}


\twolineshloka
{ततः कुञ्जरयूथानि भीमसेनो महाबलः}
{व्यधमन्निशितैर्बाणैर्महावातो घनानिव}


\twolineshloka
{नित्यमत्ताश्च मातङ्गाः शूरैर्मत्तैरधिष्ठिताः}
{आरोहकैर्महामात्रैस्तोमराङ्कुरापाणिभिः}


\twolineshloka
{सुवर्णजालैः प्रच्छन्ना मणिजालैश्च कुञ्जराः}
{रूप्यजाम्बूनदाभासः क्षुरमालाभ्यलङ्कृताः}


\twolineshloka
{वध्यमानाः शरै राजन्भीमसेनेन ते गजाः}
{विभिन्नहृदयाः केचित्तत्रैवाभ्यपतन्भुवि}


\twolineshloka
{निपतद्भिर्महावेगैर्हेमभाण्डविभूषितैः}
{अशोभत महाराज धातुचित्रैरिवाचलैः}


\twolineshloka
{दीप्ताभरणवद्भिश्च गजपृष्ठनिपातितैः}
{सङ्गरः शुशुभे राजन्क्षीणपुण्यैरिवामरैः}


\twolineshloka
{महापरिघसङ्काशौ चन्दनागरुरूषितौ}
{अपश्यं भीमसेनस्य धनुर्विक्षिपतो भुजौ}


\twolineshloka
{तस्य ज्यातलनिर्घोषमस्यतः सव्यदक्षिणम्}
{तं श्रुत्वाऽभ्यद्रवन्नागा भीमसेनभयार्दिताः}


\twolineshloka
{तस्य भीमस्य तत्कर्म राजन्नेकस्य धीमतः}
{अपश्याम महाराज तद्भूतमिवाभवत्'}


\chapter{अध्यायः ४५}
\twolineshloka
{सञ्जय उवाच}
{}


\twolineshloka
{ततः परानीकभिदं व्यूहमप्रतिमं महत्}
{समीक्ष्य कर्णः पार्थानां धृष्टद्युम्नाभिरक्षितम्}


\twolineshloka
{प्रययौ रथघोषेण सिंहनादरवेण च}
{वादित्राणां च निनदैः कम्पयन्निव मेदिनीम्}


\twolineshloka
{वेपमान इव क्रोधाद्युद्वशौण्डः परन्तपः}
{प्रतिव्यूह्य महातेजा यथावद्भरतर्षभ}


\twolineshloka
{व्यधमत्पाण्डवीं सेनामासुरीं मघवानिव}
{युधिष्ठिरं चाभ्यहनदपसव्यं चकारह}


\twolineshloka
{कर्णस्य रथघोषेण मौर्वीनिष्पेषणेन च}
{सुसङ्ग्राहेण रश्मीनां समकम्पन्त सृञ्जयाः}


\twolineshloka
{तानि सर्वाणि सैन्यानि कर्ण दृष्ट्वा विशाम्पते}
{बभूवुः सम्प्रहृष्टानि तावकानि युयुत्सया}


\twolineshloka
{अश्रूयन्त ततो वाचस्तावकानां विशाम्पते}
{कर्णार्जुनमहायुद्धमेतदद्य भविष्यति}


\twolineshloka
{अद्य दुर्योधनो राजा हतामित्रो भविष्यति}
{अद्य कर्णं रणे दृष्ट्वा फल्गुनो विद्रविष्यति}


\twolineshloka
{अद्य तावद्वयं युद्धे कर्णस्यैवानुगामिनः}
{कर्णबाणमयं भीमं युद्धं द्रक्ष्याम संयुगे}


\twolineshloka
{चिरकालेप्सितमिदमद्येदानीं भविष्यति}
{अद्य द्रक्ष्याम सङ्ग्रामं घोरं देवासुरोपमम्}


\twolineshloka
{अद्येदानीं महद्युद्धं भविष्यति भयानकम्}
{अद्येदानीं जयो नित्यमेकस्यैकस्य वा रणे}


\twolineshloka
{अर्जुनं किल राधेयो वधिष्यति महारणे}
{अथवा कं नरं लोके न स्पृशन्ति मनोरथाः}


\twolineshloka
{इत्युक्त्वा विविधा वाचः कुरवः कुरुनन्दन}
{आजघ्नुः पटहांश्चैव तूर्यांश्चैव सहस्रशः}


\twolineshloka
{भेरीनादांश्च विविधान्सिंहनादांश्च पुष्कलान््}
{मुरजानां महाशब्दानानकानां महारवान्}


\twolineshloka
{नृत्यमानाश्च बहुशस्तर्जमानाश्च मारिष}
{अन्योन्यमभ्ययुर्युद्धे युद्धरङ्गगता नराः}


\twolineshloka
{तेषां पदाता नागानां पादरक्षाः समन्ततः}
{पट्टसासिधराः शूराश्चापबाणमुसुण्ठिनः}


\threelineshloka
{भिण्डिपालधराश्चैव शूरलहस्ताः सुचक्रिणः}
{तेषां समागमो घोरो देवासुररणोपमः ॥धृतराष्ट्रं उवाच}
{}


\threelineshloka
{कथं सञ्जय राधेयः प्रत्यव्यूहत पाण्डवान्}
{धृष्टद्युम्नमुखान्सङ्ख्ये भीमसेनाभिरक्षितान्}
{सर्वानेव महेष्वासानजय्यानमरैरपि}


\twolineshloka
{के च प्रपक्षौ पक्षौ वा मम सैन्यस्य सञ्जय}
{प्रविभज्य यथान्यायमवतस्थुः सुदंशिताः}


\twolineshloka
{कथं पाण्डुसुताश्चापि प्रत्यव्यूहन्त मामकान्}
{कथं चैतन्महद्युद्धं प्रावर्तत सुदारुणम्}


\twolineshloka
{क्व च बीभत्सुरभवद्यत्कर्णोऽयाद्युधिष्ठिरम्}
{को ह्यर्जुनस्य सान्निध्ये शक्तो जेतुं युधिष्ठिरम्}


\threelineshloka
{सर्वभूतानि यो ह्येकः खाण्डवे जितबान्पुरा}
{कस्तमन्यस्तु राधेयात्प्रतियुध्येज्जिजीविषुः ॥सञ्जय उवाच}
{}


\twolineshloka
{शृणु व्यूहस्य रचनामर्जुनश्च यथा गतः}
{परिवार्य नृपान्सर्वान्सङ्ग्रामश्चाभवद्यथा}


\twolineshloka
{कृपः शारद्वतो राजन्मागधाश्च तरस्विनः}
{सात्वतः कृतवर्मा च दक्षिणं पक्षमाश्रिताः}


\twolineshloka
{तेषां प्रपक्षे शकुनिरुलूकश्च महारथः}
{सादिभिर्विमलप्रासैस्तदानीमभ्यरक्षताम्}


\twolineshloka
{गान्धारैश्चाप्यसम्भ्रान्तैः पार्वतीयैश्च दुर्जयैः}
{शलभानामिव व्रातैः पिशाचैरिव दुर्दृशैः}


\twolineshloka
{चतुर्विंशत्सहस्राणि रथानामनिवर्तिनाम्}
{संशप्तका युद्वशौण्डा वामपाश्वमपालयन्}


\twolineshloka
{समन्वितास्तव सुतैः कृष्णार्जुनजिघांसवः}
{तेषां प्रपक्षाः काम्भोजाः शकाश्च यवनैः सह}


\twolineshloka
{निदेशात्सूतपुत्रस्य सरथाश्वेभपत्तयः}
{आह्वयन्तोऽर्जुनं तस्थुः केशवं च महाबलम्}


\twolineshloka
{ततः सेनामुखे कर्णोऽप्यवातिष्ठत दंशितः}
{चित्रवर्माङ्गदः स्रग्वी पालयन्वाहिनीमुखम्}


\twolineshloka
{रक्षमाणैः सुसंरब्धैः पुत्रैः शस्त्रभृतां वरः}
{वाहिनीं प्रमुखे वीरः सम्प्रकर्षन्नशोभत}


\threelineshloka
{अशोकस्तु महाबाहुः सूर्यवैश्वानरप्रभः}
{महाद्विपस्कन्धगतः पिङ्गाक्षः प्रियदर्शनः}
{दुःशासनो वृतः सैन्यैः स्थितो व्यूहस्य पृष्ठतः}


\twolineshloka
{तन्मध्ये स महाबाहुः स्वयं दुर्योधनो नृपः}
{चित्राश्वैश्चित्रसन्नाहैः सोदर्यैरभिरक्षितः}


\twolineshloka
{रक्ष्यमाणो महावीर्यैः सहितैर्मद्रकेकयैः}
{अशोभत महाराज देवैरिव शतक्रतुः}


\threelineshloka
{अश्वत्थामा कुरूणां च ये प्रवीरा महारथाः}
{नित्यमत्ताश्च मातङ्गाः शूरैर्म्लेच्छैः समन्विताः}
{अन्वयुस्तद्रथानीकं क्षरन्त इव तोयदाः}


\twolineshloka
{ते ध्वजैर्वैजयन्तीभिर्ज्वलद्भिः परमायुधैः}
{सादिभिश्चास्थिता रेजुर्द्रुमवन्त इवाचलाः}


\twolineshloka
{तेषां पदातिनागानां पादरक्षाः सहस्रशः}
{पट्टसासिधराः शूरा बभूवुरनिवर्तिनः}


\twolineshloka
{सादिभिः स्यन्दनैर्नागैरधिकं समलङ्कृतैः}
{स व्यूहराजो विबभौ देवासुरचमूपमः}


\twolineshloka
{बार्हस्पत्यः सुविहितो नायकेन विपश्चिता}
{नृत्यतीव महाव्यूहः परेषां भयमादधत्}


\twolineshloka
{तस्य पक्षप्रपक्षेभ्यो निष्पतन्ति युयुत्सवः}
{हस्त्यश्वरथमातङ्गाः प्रावृषीव बलाहकाः}


\twolineshloka
{ततः सेनामुखे कर्णं दृष्ट्वा राजा युधिष्ठिरः}
{धनञ्जयममित्रघ्नमेकवीरमुवाच ह}


\twolineshloka
{पश्यार्जुन महाव्यूहं कर्णेन विहितं रणे}
{युक्तं पक्षैः प्रपक्षैश्च परानीकं प्रकाशते}


\threelineshloka
{तत्समीपगतं ह्येतत्प्रत्यमित्रं महद्बूलम्}
{यथा नाभिभवत्यस्मांस्तथा नीतिर्विधीयताम् ॥सञ्जय उवाच}
{}


% Check verse!
एवमुक्तोऽर्जुनो राज्ञा प्राञ्जलिर्नृपमब्रवीत्
\threelineshloka
{यथा यथा भवानाह तत्तथा न तदन्यथा}
{यत्तस्य विहितं कार्यं तत्करिष्यामि सुव्रत}
{प्रधानमथ चैवास्य विनाशं च करोम्यहम्}


\twolineshloka
{तस्मात्त्वं जहि राधेयं भीमसेनः सुयोधनम्}
{वृषसेनं च नकुलः सहदेवोऽपि सौबलम्}


\twolineshloka
{दुःशानं शतानीको हार्दिक्यं शिनिपुङ्गवः}
{धृष्टद्युम्नो द्रोणसुतं स्वयं योत्स्याम्यहं कृपम्}


\threelineshloka
{द्रौपदेया धार्तराष्ट्राञ्शिष्टान्सह शिखण्डिना}
{ते ते च तांस्तानहितानस्माकं घ्न्तु मामकाः ॥सञ्जय उवाच}
{}


\twolineshloka
{इत्युक्तो धर्मराजेन तथेत्युक्त्वा धनञ्जयः}
{व्यादिदेश स्वसैन्यानि स्वयं चागाच्चमूमुखम्}


\twolineshloka
{धनञ्जयो महाराज दक्षिणं पक्षमास्थितः}
{भीमसेनो महाबाहुर्वामं पक्षमुपाश्रितः}


\twolineshloka
{सात्यकिर्द्रौपदेयाश्च स्वयं राजा च पाण्डवः}
{व्यूहस्य प्रमुखे तस्थुः स्वेनानीकेन संवृताः}


\twolineshloka
{स्वबलेनारिसैन्यं तत्प्रत्यवस्थाप्य पाण्डवः}
{प्रत्यव्यूहत्पुरस्कृत्य धृष्टद्युम्नशिखण्डिनौ}


\twolineshloka
{तत्सादिनागकलिलं पदातिरथसङ्कुलम्}
{धृष्टद्युम्नमुखं घोरमशोभत महद्बलम्}


\chapter{अध्यायः ४६}
\twolineshloka
{धृतराष्ट्र उवाच}
{}


\twolineshloka
{तथा व्यूढेष्वनीकेषु संसक्तेषु च सञ्जय}
{संशप्तकाः कथं पार्थं कथं कर्णं च पाण्डवाः}


\threelineshloka
{एतद्विस्तरशो युद्धं प्रब्रूहि कुशलो ह्यसि}
{न हि तृप्यामि वीराणां शृण्वानो युधि विक्रमम् ॥सञ्जय उवाच}
{}


\twolineshloka
{तत्संस्थिमथो दृष्ट्वा प्रत्यमित्रं महद्बलम्}
{प्रत्यव्यूहत्ततः कर्णो हितार्थं तनयस्य ते}


\twolineshloka
{तत्सादिनागकलिलं पदातिरथसङ्कुलम्}
{धृष्टद्युम्नमुखं घोरमशोभत महद्बलम्}


\twolineshloka
{पारावतसवर्णाश्वश्चन्द्रसूर्यसमद्युतिः}
{स पार्षतो बभौ धन्वी कालो विग्रहवानिव}


\twolineshloka
{पार्षतं त्वभितस्तस्थुर्द्रौपदेया युयुत्सवः}
{सानुगा दीप्तवपुषश्चन्द्रं तारागणा इव}


\twolineshloka
{अथ व्यूढेष्वनीकेषु प्रेक्ष्य संशप्तकान्रणे}
{क्रुद्धोऽर्जुनोऽभिदुद्राव व्याक्षिपन्गाण्डिवं धनुः}


\twolineshloka
{अथ संशप्तकाः पार्थमभ्यधावन्वधैषिणः}
{विजये धृतसङ्कल्पा मृत्युं कृत्वानिवर्तनम्}


\twolineshloka
{तदश्वसङ्घबहुलं पत्तिनागरथाकुलम्}
{उदीर्यमाणं संरब्धं धनञ्जयमभिद्रवत्}


\twolineshloka
{तदुद्यतं महत्सैन्यं सागरौघसमं जवे}
{पार्थवेलां समासाद्य विष्ठितं समदृश्यत}


% Check verse!
मृद्गन्तं शवर्षौघैर्द्रुतमर्जुनमार्दयत्
\twolineshloka
{स सम्प्रहारस्तुमुलस्तेषामासीत्किरीटिना}
{तस्यैव तु यथा राजन्निवातकवचैः सह}


\twolineshloka
{रथानश्वान्ध्वजान्नागान्पत्तीन्गजगतानपि}
{इषून्धनूंषि खङ्गांश्च चक्राणि च परश्वथान्}


\twolineshloka
{सायुधानुद्यतान्बाहून्विविधान्यायुधानि च}
{चिच्छेद द्विषतां पार्थः शिरांसि च सहस्रशः}


\twolineshloka
{तस्मिन्सैन्यमहावर्ते पातालतलसन्निभे}
{निमग्नं तं रथं मत्वा नेदुः संशप्तका मुदा}


\twolineshloka
{स पुरस्तादरीन्हत्वा पुनरुत्तरतोऽवधीत्}
{दक्षिणेन च पश्चाच्च क्रुद्धो रुद्रः पशूनिव}


\twolineshloka
{अथ पाञ्चालचेदीनां सृढ्जयानां च मारिष}
{त्वदीयैः सह सङ्ग्राम आसीत्परमदारुणः}


\twolineshloka
{कृपश्च कृतवर्मा च शकुनिश्चापि सौबलः}
{हृष्टसेनाः सुसंरब्धा रथानीकप्रहारिणः}


\twolineshloka
{कोसलैः काश्यमत्स्यैश्च कारूशैः केकयैरपि}
{शूरसेनैः शूरवरैर्युयुधुर्युद्धदुर्मदाः}


\twolineshloka
{तेषामन्तकरं युद्धं देहपाप्मविनाशनम्}
{क्षत्रविट््शूद्रवीराणां धर्म्यं स्वर्ग्यं यशस्करम्}


\twolineshloka
{दुर्योधनोऽथ सहितो भ्रातृभिर्भरतर्षभ}
{गुप्तः कुरुप्रवीरैश्च मद्राणां च महारथैः}


\twolineshloka
{पाण्डवैः सह पाञ्चालैश्चेदिभिः सात्यकेन च}
{युध्यमानं रणे कर्णं कुरुवीरोऽभ्यपालयत्}


\twolineshloka
{कर्णोऽपि निशितैर्बाणैर्विनिहत्य महाचमूम्}
{प्रमृद्य च रथश्रेष्ठान्युधिष्ठिरमपीडयत्}


\twolineshloka
{विपत्रायुधदेहासून्कृत्वा शत्रून्सहस्रशः}
{युक्त्वा स्वर्गयशोभ्यां च स्वेभ्यो मुदमुदावहत्}


\twolineshloka
{तद्विगाह्य महानीकं सूतपुत्रो महारथः}
{नदीं प्रवर्तयामास शोणितौघतरङ्गिणीम्}


\twolineshloka
{शोणितोदां क्षुण्णमत्स्थां नागनक्रदुरत्ययाम्}
{मांलमज्जाकर्दमिनीं चक्रकूर्ममहोडुपाम्}


\twolineshloka
{पातितैर्मेघसङ्काशैस्तत्रतत्र महाद्विपैः}
{अशनीभिरिव ध्वस्ता नदी राजन्विराजते}


\twolineshloka
{तां शरोर्मिमहावर्तां छत्रहंससमाकुलाम्}
{तनुत्रोष्णीषसङ्घाटामस्थिपाषाणसङ्कुलाम्}


\twolineshloka
{अपारामनपारां च शङ्खदुन्दुभिघोषिणीम्}
{रौद्रां नदीं महाराज रजसा सर्वतोवृताम्}


\twolineshloka
{अतितीक्ष्णां नराकीर्णां नदीमन्तकगामिनीम्}
{समं च विषमं चैव समायान्तीं महाभयाम्}


\twolineshloka
{आकुलाश्चात्र सीदन्ति नराः शोणितकर्दमे}
{नरैरभिपरिक्षिप्ता यथा राजन्महाद्रुमाः}


\twolineshloka
{ततस्ते तत्रतत्रैव प्रचरन्तो महानदीम्}
{विचेरुः सर्वतो योधा नौवारणमहारथैः}


\twolineshloka
{शोणितेन समं राजन्कृतमासीत्समन्ततः}
{नदीवेगैर्यथा भूमिस्तद्वदासीद्विशापते}


\twolineshloka
{एवं भारत सङ्ग्रामो नरवाजिगजक्षयः}
{कुरूणां सृञ्जयानां च देवासुरसमोऽभवत्}


\chapter{अध्यायः ४७}
\twolineshloka
{धृतराष्ट्र उवाच}
{}


\twolineshloka
{युक्तं प्रविश्य पार्थानां सैन्यं कुर्वञ्जनक्षयम्}
{कर्णो राजानमभ्येत्य तन्ममाचक्ष्व स़ञ्जय}


\threelineshloka
{के च प्रवीराः पार्थानां युधि कर्णमवारयन्}
{कांश्च प्रमथ्याधिरथिर्युधिष्ठिरमपीडयत् ॥सञ्जय उवाच}
{}


\twolineshloka
{धृष्टद्युम्नमुखान्पार्थान्दृष्ट्वा कर्णो व्यवस्थितान्}
{समभ्यधावत्त्वरितः पाञ्चालाञ्छत्रुकर्शनः}


\twolineshloka
{तं तूर्णमभिधावन्तं पाञ्चाला जितकाशिनः}
{प्रत्युद्ययुर्महात्मानं हंसाः सर इवोष्णगे}


\twolineshloka
{ततः शङ्खसहस्राणां निःस्वनो हृदयङ्गमः}
{प्रादुरासीदुभयतो भेरीशब्दश्च दारुणः}


\twolineshloka
{नानाबाणनिनादश्च द्विपाश्वरथनिःस्वनः}
{सिंहनादश्च वीराणामभवद्दारुणस्तदा}


\twolineshloka
{साद्रिद्रुमार्णवा भूमिः सवाताम्बुदमम्बरम्}
{सार्केन्दुग्रहनक्षत्रा द्यौश्च व्यक्तं विघूर्णिता}


\twolineshloka
{इति भूतानि तं शब्दं मेनिरे ते च विव्यथुः}
{यानि चाप्यल्पसत्वानि प्रायस्तानि मृतानि वै}


\twolineshloka
{अथ कर्णो भृशं क्रुद्धः शीघ्रमस्त्रमुदीरयन्}
{जघान पाण्डवीं सेनामासुरीं मघवानिव}


\twolineshloka
{स पाण्डवबलं कर्णः प्रविश्य विसृजञ्छरान्}
{प्रभद्रकाणां प्रवरानहनत्सप्तसप्ततिम्}


\twolineshloka
{ततः सुपुङ्खैर्निशितैरथ श्रष्ठो रथेषुभिः}
{अवधीत्पञ्चविंशत्या पाञ्चालान्पञ्चविंशतिम्}


\twolineshloka
{सुवर्णपुङ्खैर्नाराचैः परकायविदारणैः}
{चेदिकानवधीद्वीरः शतशोऽथ सहस्रशः}


\twolineshloka
{तं तथा समरे कर्म कुर्वाणमतिमानुषम्}
{परिवव्रुर्महाराज पाञ्चालानां रथव्रजाः}


\twolineshloka
{ततः सन्धाय विशिखान्पञ्च भारत दुःसहान्}
{पाञ्चालानवधीत्पञ्च कर्णो वैकर्तनो वृषः}


\twolineshloka
{भानुदेवं चित्रसेनं सेनाबिन्दुं च भारत}
{तपनं शूरसेनं च पाञ्चालानहनद्रणे}


\twolineshloka
{पाञ्चालेषु च शूरेषु वध्यमानेषु सायकैः}
{हाहाकारो महानासीत्पाञ्चालानां महाहवे}


\twolineshloka
{तेषां सह्क्रीडमानानां हाहाकारं प्रकुर्वताम्}
{पुनरेव च तान्कर्णो जघानाशु पतत्त्रिभिः}


\twolineshloka
{चक्ररक्षौ तु कर्णस्य पुत्रौ मारिष दुर्जयौ}
{सुषेणः सत्यसेनश्च त्यक्त्वा प्राणानयुध्यताम्}


\twolineshloka
{पृष्ठगोप्ता तु कर्णस्य ज्येष्ठः पुत्रो महारथः}
{वृषसेनोऽन्वयात्कर्णं पृष्ठतः परिपालयन्}


\twolineshloka
{धृष्टद्युम्नः सात्यकिश्च द्रौपदेया वृकोदरः}
{जनमेजयः शिखण्डी च प्रवीराश्च प्रभद्रकाः}


\twolineshloka
{चेदिकेकयपाञ्चाला यमौ मात्स्याश्च दंशिताः}
{समभ्यधावन्राधेयं जिघांसन्तः प्रहारिणम्}


\twolineshloka
{त एनं विविधैः शस्त्रैः शरधाराभिरेव च}
{अभ्यवर्षन्विमर्दन्तं प्रावृषीवाम्बुदा गिरिम्}


\twolineshloka
{पितरं तु परीप्सन्तः कर्णपुत्राः प्रहारिणः}
{त्वदीयाश्चापरे राजन्वीरा वीरानवारयन्}


\twolineshloka
{सुषेणो भीमसेनस्य च्छित्त्वा भल्लेन कार्मुकम्}
{नाराचैः सप्तबिर्विद्ध्वा हृदि भीमं ननाद ह}


\twolineshloka
{अथान्यद्वनुरादाय सुदृढं भीमविक्रमः}
{सज्यं वृकोदरः कृत्वा सुषेणस्याच्छिनद्वनुः}


\twolineshloka
{विव्याध चैनं दशभिः क्रुद्धो नृत्यन्निवेषुभिः}
{कर्णं च तूर्णं विव्याध त्रिसप्तत्या शितैः शरैः}


\twolineshloka
{सत्यसेनं च दशभिः साश्वसूतायुधध्वजम्}
{पश्यतां सुहृदां मध्ये कर्णपुत्रमपातयत्}


\twolineshloka
{क्षुरप्रणुन्नं तत्तस्य शिरश्चन्द्रनिभाननम्}
{शुभदर्शनमेवासीन्नालभ्रष्टमिवाम्बुजम्}


\twolineshloka
{हत्वा कर्णसुतं भीमस्तावकान्पुनरार्दयत्}
{कृपहार्दिक्ययोश्छित्त्वा चापौ तावप्यथार्दयत्}


\twolineshloka
{दुःशासनं त्रिभिर्विद्ध्वा शकुनिं षड्भिरायसैः}
{उलूकं च पतत्रिं च चकार विरथावुभौ}


\twolineshloka
{हे सुषेण हतोऽसीति ब्रुवन्नादत्त सायकम्}
{तमस्य कर्णश्चिच्छेद त्रिभिश्चैनमताडयत्}


\twolineshloka
{अथान्यं परिजग्राह सुपर्वाणं सुतेजनम्}
{सुषेणायासृजद्भीमस्तमप्यस्याच्छिनद्वृषः}


\twolineshloka
{पुनः कर्णस्त्रिसप्तत्या भीमसेनमथेषुभिः}
{पुत्रं परीप्सन्विव्याध क्रुद्धः शत्रुजिघांसया}


\twolineshloka
{सुषेणस्तु धनुर्गृह्य भारसाधनमुत्तमम्}
{नकुलंपञ्चभिर्बाणैर्बाह्वोरुरसि चार्पयत्}


\twolineshloka
{नकुलस्तं तु विंशत्या विद्व्वा भारसहैर्दृढैः}
{ननाद बलवन्नादं कर्णस्य भयमादधत्}


\twolineshloka
{तं सुषेणो महाराज विद्ध्वा दशभिराशुगैः}
{चिच्छेद च धनुः शीघ्रं क्षुरप्रेण महारथः}


\twolineshloka
{अथान्यद्वनुरादाय नकुलः क्रोधमूर्च्छितः}
{सुषेणं नवभिर्बाणैर्वारयामास संयुगे}


\twolineshloka
{स तु बाणैर्दिशो राजन्नाच्छाद्य परवीरहा ॥आजघ्ने सारथिं चास्य सुषेणं च ततस्त्रिभिः}
{चिच्छेद चास्य सुदृढं धनुर्भल्लैस्त्रिभिस्त्रिधा}


\twolineshloka
{अथान्यद्वनुरादाय सुषेणः क्रोधमूर्च्छितः}
{आविध्यन्नकुलं षष्ट्या सहदेवं च सप्तभिः}


\twolineshloka
{तद्युद्धं सुमहद्धोरमासीद्देवासुरोपमम्}
{निघ्नतां सायकैस्तूर्णमन्योन्यस्य वधं प्रति}


\twolineshloka
{सात्यकिर्वृषसेनं तु विद्ध्वा सप्तभिरायसैः}
{`पुनर्विव्याध सप्तत्या सारथिं च त्रिभिः शितैः}


\twolineshloka
{वृषसेनस्तु शैनेयं शरेणानतपर्वणा}
{आजघान महाराज शङ्खदेशे महारथम्}


\twolineshloka
{शैनेयो वृषसेनेन पत्रिणा परिपीडितः}
{कोपं चक्रे महाराज क्रुद्धो वेगं च दारुणम्}


\twolineshloka
{जग्राहेषुवरान्वीरः शीघ्रं वै दश पञ्च च}
{सात्यकिर्वृषसेनस्य सूतं हत्वा त्रिभिः शरैः'}


\twolineshloka
{धनुश्चिच्छेद भल्लेन जघानाश्वांश्च सप्तभिः}
{ध्वजमेकेषुणोन्मथ्य त्रिभिश्तं हृद्यताडयत्}


\threelineshloka
{अथावसन्नः स्वरथे मुहूर्तात्पुनरुत्थितः}
{स रणे युयुधानेन विसूताश्वरथध्वजः}
{कृतो जिघांसुः शैनेयं खङ्गचर्मधृगभ्ययात्}


\twolineshloka
{तस्य चापततः शीघ्रं वृषसेनस्य सात्यकिः}
{वाराहकर्णैर्दशभिरविध्यदसिचर्मणी}


\twolineshloka
{दुःशासनस्तु तं दृष्ट्वा विरथं व्यायुधं कृतम्}
{आरोप्य स्वरथं तूर्णमपोवाह रथान्तरम्}


\twolineshloka
{अथान्यं रथमास्थाय वृषसेनो महारथः}
{द्रौपदेयांस्त्रिसप्तत्या युयुधानं च पञ्चभिः}


\twolineshloka
{भीमसेनं चतुःषष्ट्या सहदेवं च पञ्चभिः}
{नकुलं त्रिंशता बाणैः शतानीकं च सप्तभिः}


\twolineshloka
{शिखण्डिनं च दशभिर्धर्मराजं शतेन च}
{एतांश्चान्यांश्च राजेन्द्र प्रवीराञ्चयगृद्धिनः}


\twolineshloka
{अभ्यर्दयन्महेष्वासः कर्णपुत्रो विशाम्पते}
{कर्णस्य युधि दुर्धर्षस्ततः पृष्ठमपालयत्}


\twolineshloka
{युयुधानं च राधेयो नवैर्नवभिरायसैः}
{विसूताश्वरथं कृत्वा ललाटे त्रिभिरार्पयत्}


\twolineshloka
{स त्वन्यं रथमास्थाय विधिवत्कल्पितं पुनः}
{युयुधे पाण्डुभिः सार्धं कर्णस्य व्यधमद्बलम्}


\twolineshloka
{धृष्टद्युम्नस्ततः कर्णमविध्यद्दशभिः शरैः}
{द्रौपदेयास्त्रिसप्तत्या युयुधानस्तु सप्तभिः}


\twolineshloka
{भीमसेनश्चतुःषष्ट्या सहदेवश्च सप्तिभिः}
{नकुलस्त्रिंशता बाणैः शतानीकस्तु सप्तभिः}


\threelineshloka
{शिखण्डी दशभिर्वीरो धर्मराजः शतेन तु}
{एते चान्ये च राजेन्द्र प्रवीरा जयगृद्विनः}
{अभ्यर्दयन्महेष्वासं सूतपुत्रं महामृधे}


\threelineshloka
{तान्सूतपुत्रो विशिखैर्दशभिर्दशभिः शरैः}
{रथेनानुचरन्वीरः प्रत्यविध्यदरिन्दमः}
{}


\twolineshloka
{`सात्यकिं भीमसेनं च धृष्टद्युम्नं शिखण्डिनम्}
{द्रौपदेयांश्च संरब्धान्यतमानान्महारथान्}


\twolineshloka
{पश्यतां सर्वसैन्यानामथैनान्सूतनन्दनः}
{विरथान्रथिनः श्रेष्ठान्निमेषार्धाच्चकार ह}


% Check verse!
अमोघत्वाच्च बाणानां भूतसङ्घा विसिप्मयुः'
\twolineshloka
{तत्रास्त्रवीर्यं कर्णस्य लाघवं च महात्मनः}
{अपश्याम महाभाग तदद्भुतमिवाभवत्}


\twolineshloka
{न ह्याददानं ददृशुः सन्दधानं च सायकान्}
{विमुञ्चन्तं च संरम्भाद पश्यन्नैव तं जनाः}


\twolineshloka
{प्रतीच्यां दिशि तं दृष्ट्वा प्राच्यां पश्याम लाघवात्}
{न च पश्याम राजेन्द्र क्व नु कर्णो व्यतिष्ठत}


\twolineshloka
{इषूनेवास्य पश्यामो विनिकीर्णाऽन्समन्ततः}
{छादयानान्दिशो राजञ्शलभानामिव व्रजान्}


\twolineshloka
{तस्य तैरिषुभिस्त्रीक्ष्णैः सम्पतद्भिः सहस्रशः}
{मरीचिभिरिवोष्णांशोः शरैः सन्नतपर्वभिः}


\twolineshloka
{व्याप्ताः सर्वा दिशो राजन्योधाश्च ददृशुस्तदा}
{शरैः संवृतमाकाशं तत्राभ्रैरिव चाभवत्}


\twolineshloka
{द्यौर्वियद्भूर्दिशश्चैव प्रच्छन्ना निशितैः शरैः}
{अरुणाभ्रावृताकारं तस्मिन्देशे बभौ वियत्}


\twolineshloka
{ततः पुनरमेयात्मा कर्णो राजा महारथः}
{न्यहनत्समरे योधान्योधवृत्तमनुष्ठितः}


\twolineshloka
{नृत्यन्निव हि राधेयश्चापहस्तो रणाजिरे}
{यैर्विद्वः प्रत्यविद्व्यत्तानेकैकं त्रिगुणैः शरैः}


\twolineshloka
{शतैश्च दशभिश्चैतान्पुनर्विद्ध्वा ननाद च}
{साश्वसूतध्वजच्छत्रास्ततस्ते विवरं ददुः}


\twolineshloka
{`ते हन्यमानाः कर्णेन पलायन्त दिशो दश}
{नादयन्तो दिशः सर्वाः कर्मत्रस्ता विचेतसः'}


\twolineshloka
{तान्प्रमथ्य महेष्वासान्राधेयः शरवृष्टिभिः}
{राजानीकमसम्बाधं प्राविशच्छत्रुकर्शनः}


\twolineshloka
{स रथांस्त्रिशतं हत्वा चेदीनामनिवर्तिनाम्}
{राधेयो निशितैर्बाणैस्ततोऽभ्यागाद्युधिष्ठिरम्}


\twolineshloka
{`ततस्ते विरथाः शूरा रथानन्यान्समास्थिताः}
{परिवव्रुर्महाराज धर्मपुत्रं युधिष्ठिरम्'}


\threelineshloka
{ततस्ते तु परे राजञ्छिखण्डी च ससात्यकिः}
{राधेयात्परिरक्षन्तो राजानं पर्यवारयन्}
{मुञ्चन्तो विविधान्बाणान्स्वर्णपुङ्खाञ्शिलाशितान्}


\twolineshloka
{तथैव तावकाः सर्वे कर्णं दुर्वारणं रणे}
{यत्ताः शूरा महेष्वासाः पर्यरक्षन्त सर्वशः}


\twolineshloka
{नानावादित्रघोषाश्च प्रादुरासन्विशाम्पते}
{सिंहनादश्च सञ्जज्ञे शूराणामभिगर्जताम्}


\twolineshloka
{ततः पुनः समाजग्मुरभीताः कुरुपाण्डवाः}
{युधिष्ठिरमुखाः पार्थाः सूतपुत्रमुखा वयम्}


\chapter{अध्यायः ४८}
\twolineshloka
{सञ्जय उवाच}
{}


\twolineshloka
{विदार्य कर्णस्तां सेनां युधिष्ठिरमथाद्रवत्}
{रथहस्त्यश्वपत्तीनां सहस्रैः परिवारितः}


\twolineshloka
{नानायुधसहस्राणि प्रेरितान्यरिभिर्वृषः}
{छित्त्वा बाणशतैरुग्रैस्तानविध्यदसम्भ्रमात्}


% Check verse!
निचकर्त शिरांस्येषां बाहूनूरूंश्च सूतजः ॥ते हता भूतले पेतुर्भग्नाश्चान्ये विदुद्रुवुः
\twolineshloka
{द्रविडान्ध्रनिषादास्तु पुनः सात्यकिचोदिताः}
{प्रभ्यद्रवञ्जिघांसन्तः पत्तयः कर्णमाहवे}


\twolineshloka
{ते विबाहुशिरस्त्राणाः प्रहताः कर्णसायकैः}
{पेतुः पृथिव्यां युगपच्छिन्नं सालवनं यथा}


\twolineshloka
{एवं योधशतान्याजौ सहस्राण्ययुतानि च}
{हतानीयुर्महीं देहैर्यशसाऽपूरयन्दिशः}


\twolineshloka
{अथ वैकर्तनं कर्णं रणे क्रुद्धमिवान्तकम्}
{रुरुधुः पाण्डुपाञ्चाला व्याधिं मन्त्रौषधा इव}


\twolineshloka
{स तान्प्रमृद्याब्यपतत्पुनरेव युधिष्ठिरम्}
{मन्त्रौषधिक्रियातीतो व्याधिरत्युल्बणो यथा}


\twolineshloka
{स राजगृद्धिभी रुद्धः पाण्डुपाञ्चालकेकयैः}
{नाशकत्तानतिक्रान्तुं मृत्युर्ब्राह्मविदो यथा}


\twolineshloka
{ततो युधिष्ठिरः कर्णमदूरस्थं निवारितम्}
{`तैर्योधप्रमुखैर्वीरं दृष्ट्वा विव्याध सायकैः}


\twolineshloka
{कर्णः पार्थशराविद्धस्तोत्रार्दित इव द्विपः}
{प्रमथ्य सहितान्वीरान्युधिष्ठिरमपीडयत्}


\twolineshloka
{ततो युधिष्ठिरः कर्णमासाद्य जयतां वरम्'}
{अब्रवीत्परवीरघ्नं क्रोधसंरक्तलोचनः}


\twolineshloka
{कर्णकर्ण वृथादृष्टे सूतपुत्र वचः शृणु}
{सदा स्पर्धसि सङ्ग्रामे फल्गुनेन तरस्विना}


\twolineshloka
{यथाऽस्मान्बाधसे नित्यं धार्तराष्ट्रमते स्थितः}
{यद्बलं यच्च ते वीर्यं प्रद्वेषो यस्तु पाण्डुषु}


\twolineshloka
{तत्सर्वं दर्शयस्वाद्य पौरुषं महदास्थितः}
{युद्धश्रद्वां च तेऽद्याहं विनेष्यामि महाहवे}


\twolineshloka
{एवमुक्त्वा महाराज कर्णं पाण्डुसुतस्तदा}
{सुवर्णपुङ्खैर्दशभिर्विव्यादायस्मयैः शरैः}


\twolineshloka
{तं सूतपुत्रो दशभिः प्रत्यविद्ध्यदरिन्दमः}
{वत्सदन्तैर्महेष्वासः प्रहसन्निव भारत}


\twolineshloka
{सोऽवज्ञाय तु निर्विद्धः सूतपुत्रेण पाण्डवः}
{प्रजज्वाल ततः क्रोधाद्धविषेव हुताशनः}


\twolineshloka
{ज्वालामालापरिक्षिप्तो राज्ञो देहो व्यदृश्यत}
{युगान्ते दग्धुकामस्य संवर्ताग्नेरिवापरः}


\twolineshloka
{ततो विस्फार्य सुमहच्चापं हेमपरिष्कृतम्}
{समाधत्त शितं बाणं गिरीणामपि दारणम्}


\twolineshloka
{ततः पूर्णायतोत्कृष्टं यमदण्डनिभं शरम्}
{मुमोच त्वरितो राजा सूतपुत्रजिघांसया}


\twolineshloka
{स तु वेगवता मुक्तो बाणो वज्राशनिस्वनः}
{विवेश सहसा कर्णं सव्ये पार्श्वे महारथम्}


\twolineshloka
{स तु तेन प्रहारेण पीडितः प्रमुमोह वै}
{स्रस्तगात्रो महाबाहुर्धनुरुत्सृज्य कम्पितः}


\twolineshloka
{गतासुरिव निश्चेताः शल्यस्याभिमुखोऽपतत्}
{राजाऽपि भूयो नाजघ्ने कर्णं पार्थहितेप्सया}


\twolineshloka
{ततो हाहाकृतं सर्वं धार्तराष्ट्रबलं महत्}
{विवर्णमुखमार्तं च कर्णं दृष्ट्वा तथागतम्}


\twolineshloka
{सिंहनादाश्च सञ्जज्ञुः क्ष्वेलाः किलकिलास्तथा}
{पाण्डवानां महाराज दृष्ट्वा राज्ञः पराक्रमम्}


\twolineshloka
{प्रतिलभ्य तु राधेयः संज्ञां नातिचिरादिव}
{दध्रे राजविनाशाय मनः क्रूरपराक्रमः}


\twolineshloka
{स हेमविकृतं चापं विस्फार्य विजयं महत्}
{अवाकिरदमेयात्मा पाण्डवं निशितैः शरैः}


\twolineshloka
{ततः क्षुराभ्यां पाञ्चाल्यौ चक्ररक्षौ महात्मनः}
{जघान चन्द्रदेवं च दण्डधारं च संयुगे}


\twolineshloka
{तावुभौ धर्मराजस्य प्रवीरौ युधि भारत}
{रथाभ्याशे चकाशेते चन्द्रस्येव पुनर्वसू}


\twolineshloka
{युधिष्ठिरः पुनः कर्णमविद्व्यत्त्रिंशता शरैः}
{सुषेणं सत्यसेनं च त्रिभिस्त्रिभिरताडयत्}


\twolineshloka
{शल्यं नवत्या विव्याध त्रिसप्तत्या च सूतजम्}
{तांस्तस्य गोप्तॄन्विव्याध त्रिभिस्त्रिभिरजिह्मगैः}


\twolineshloka
{ततः प्रहस्याधिरथिः षष्ठ्या राजञ्शितैः शरैः}
{विद्ध्वा युधिष्ठिरं सङ्ख्ये ननादातिरथस्तदा}


\twolineshloka
{ततः प्रवीराः पाण्डूनामभ्यधावन्नमर्षिताः}
{युधिष्ठिरं परीप्सन्तः कर्णमभ्यर्दयञ्छरैः}


\twolineshloka
{सात्यकिश्चेकितानश्च युयुत्सुः पाण्ड्य एव च}
{धृष्टद्युम्नः शिखण्डी च द्रौपदेयाः प्रभद्रकाः}


\twolineshloka
{यमौ च भीमसेनश्च शिशुपालस्य चात्मजः}
{कारूशा मात्स्यशेषाश्च केकयाः काशिकोसलाः}


\twolineshloka
{एते च त्वरिता वीरा वसुषेणमताडयन्}
{जनमेजयश्च पाञ्चाल्यः कर्णं विव्याध सायकैः}


\twolineshloka
{वाराहकर्णनाराचैर्नालीकैर्निशितैः शरैः}
{वत्सदन्तैर्विपाठैश्च क्षुरप्रैश्चटकामुखैः}


\twolineshloka
{नानाप्रहरणैश्चोग्रै रथहस्त्यश्वसादिभिः}
{सर्वतोऽभ्यद्रवत्कर्णं परिवार्य जिघांसया}


\twolineshloka
{स पाण्डवानां प्रवरैः सर्वतः समभिद्रुतः}
{छाद्यमानः शिततैर्घोरैः स्वस्वनामाङ्कितैः शरैः}


\twolineshloka
{न चचाल रणे कर्णो महेन्द्रो दानवैरिव}
{`निजघान महेष्वासान्पाञ्चालानेकविंशतिम्}


\twolineshloka
{ततः पुनरमेयात्मा चेदीनां प्रवरान्दश}
{न्यहनद्भरतश्रेष्ठ कर्णो वैकर्तनस्तदा}


\twolineshloka
{तस्य बाणसहस्राणि सम्प्रपन्नानि मारिष}
{दृश्यन्ते दिक्षु सर्वासु शलभानामिव व्रजाः}


\twolineshloka
{कर्णनामाङ्किता बाणाः स्वर्णपुङ्खाः सुतेजनाः}
{नराश्वकायान्निर्भिद्य पेतुरुर्व्यां समन्ततः}


\twolineshloka
{कर्णेनैकेन समरे चेदीनां प्रवरास्तदा}
{सृज्जयानां च सर्वेषां शतशो निहता रणे}


\twolineshloka
{कर्णस्य शरसञ्छन्नं बभूव तुमुलं तमः}
{नाज्ञायत ततः किञ्चित्परेषामात्मनोऽपि वा}


\twolineshloka
{तस्मिंस्तमसि भूते च क्षत्रियाणां भयङ्करे}
{विचचार महबिहुर्निर्दहन्क्षत्रियान्बहून्'}


\twolineshloka
{ततः शरमहाज्वालो वीर्योष्मा कर्णपावकः}
{निर्दहन्पाण्डववनं चारु पर्यचरद्रणे}


\threelineshloka
{`ततस्तेषां महाराज पाण्डवानां महारथाः}
{सृञ्जयानां च सर्वेषां शतशोऽथ सहस्रशः}
{अस्त्रैः कर्णं महेष्वासं समन्तात्पर्यवारयन्}


\twolineshloka
{स संवार्य महास्त्राणि महेष्वासो महात्मनाम्}
{प्रहस्य पुरुषेन्द्रस्य कर्णश्चिच्छेद कार्मुकम्'}


\twolineshloka
{ततः सन्धाय नवतिं निमेषान्नतपर्वणाम्}
{बिभेद कवचं राज्ञो रणे कर्णः शितैः शरैः}


\twolineshloka
{तद्वर्म हेमविकृतं रत्नचित्रं बभौ पतत्}
{सविद्युदभ्रं सवितुः श्लिष्टं वातहतं यथा}


\twolineshloka
{तदङ्गात्पुरुषेन्द्रस्य भ्रष्टं वर्म व्यरोचत}
{रत्नैरलंवृतं चित्रेर्व्यभ्नं निशि यथा नभः}


\twolineshloka
{भिन्नवर्मा शरैः पार्थो रुधिरेण समुक्षितः}
{बभासे पुरुषश्रेष्ठ उद्यन्निव दिवाकरः}


\twolineshloka
{स शराचितसर्वाङ्गश्छिन्नवर्माऽथ संयुगे}
{क्षात्रं धर्मं समास्थाय सिंहनादमकुर्वत}


\twolineshloka
{नर्दित्वा च चिरं कालं पाण्डवो रभसो रणे}
{शक्तिं चिक्षेप वेगेन प्रदीप्तामशनीमिव}


\twolineshloka
{तां ज्वलन्तीमिवाकाशे शक्तिं शप्तभिराशुगैः}
{भल्लैश्चिच्छेद राधेयः सा त्वशीर्यत वै रणे}


\twolineshloka
{हृदि बाह्वोर्ललाटे च क्षिप्रकारी युधिष्ठिरः}
{चतुर्भिस्तोमरैः कर्णं ताडयित्वाऽनदन्मुदा}


\twolineshloka
{उद्भिन्नरुधिरः कर्णः क्रुद्वः सर्प इव श्वसन्}
{जघान सूतं पार्थस्य पार्ष्ठिं च नवभिः शरैः}


\twolineshloka
{ध्वजं चिच्छेद नृपतेस्त्रिभिर्विव्याध चैव तम्}
{इषुधी चास्य चिच्छेद रथं च तिलशोऽच्छिनत्}


\twolineshloka
{एतस्मिन्नन्तरे शूराः पाण्डवानां महारथाः}
{ववृषुः शरवर्षाणि राधेयं प्रति भारत}


\twolineshloka
{सात्यकिः पञ्चविंशत्या शिखण्डी नवभिः शरैः}
{अवर्षतां महाराज राधेयं शत्रुकर्शनम्}


\twolineshloka
{शैनेयं तु ततुः क्रुद्धः कर्णः पञ्चभिरायसैः}
{विव्याध समरे राजंस्त्रिभिश्चान्यैः शिलीमुखैः}


\twolineshloka
{दक्षिणं तु भुजं तस्य त्रिभिः कर्णोऽभ्यविध्यत}
{सव्यं षोडशभिर्बाणैर्यन्तारं चास्य सप्तभिः}


\twolineshloka
{अथास्य चतुरो वाहांश्चतुर्भिर्निशितैः शरैः}
{सूतपुत्रोऽनयत्क्षिप्रं यमस्य सदनं प्रति}


\twolineshloka
{अथापरेण भल्लेन धनुश्छित्त्वा महारथः}
{सारथेः सशिरस्त्राणं शिरः कायादपाहरत्}


\twolineshloka
{हताश्वसूते तु रथे स्थितः स शिनिपुङ्गवः}
{शक्तिं चिक्षेप कर्णाय वैदूर्यमणिभूषिताम्}


\twolineshloka
{तामापतन्तीं सहसा द्विधा चिच्छेद भारत}
{कर्णो वै धन्विनां श्रेष्ठस्तांश्च सर्वानवारयत्}


\twolineshloka
{ततस्तान्निशितैर्बाणैः पाण्डवानां महारथान्}
{न्यवारयदमेयात्मा शिक्षया च बलेन च}


\threelineshloka
{अर्दयित्वा शरैस्तांस्तु सिंहः क्षुद्रमृगानिव}
{पीडयन्धर्मराजानं शरैः सन्नतपर्वमिः}
{अभ्यद्रवत राधेयो धर्मपुत्रं शितैः शरैः}


\twolineshloka
{ततः पार्षो ह्यपासासीद्वताश्वो हतसारथिः}
{अशक्नुवन्मुखे स्थातुं ततः कर्णस्य दुर्मनाः}


\twolineshloka
{तमभिद्रुत्य राधेयः पाण्डुपुत्रं युधिष्ठिरम्}
{अब्रवीत्प्रहसन्राजन्कुत्सयन्निव पाण्डवम्}


\twolineshloka
{कथं नाम कुले जातः क्षत्रधर्मे व्यवस्थितः}
{प्रजह्यात्समरे शत्रून्प्राणान्रक्ष महाहवे}


\twolineshloka
{न भवान्क्षत्रधर्मेषु कुशलोऽसीति मे मतिः}
{ब्राह्मे बले भवान्युक्तः स्वाध्याये यज्ञकर्मणि}


\twolineshloka
{मा स्म युध्यस्व कौन्तेय मा स्म वीरान्समासदः}
{मा चैवं विप्रियं ब्रूहि मा च त्वं भज संयुगम्}


\twolineshloka
{वक्तव्या मारिषाऽन्ये तु न वक्तव्यास्तु मादृशाः}
{मादृशान्हि ब्रुवन्युद्धे एतदन्यच्च लप्स्यसे}


\threelineshloka
{स्वगृहं गच्छ कौन्तेय यत्र वा केशवार्जुनौ}
{न हि त्वां समरे राजन्हन्यात्कर्णः कथञ्चन ॥सञ्जय उवाच}
{}


\twolineshloka
{वरप्रदानं कुन्त्यास्तु कर्णः स्मृत्वा महारथः}
{वधप्राप्तं तु कौन्तेयं नावधीत्पुरुषोत्तमः}


\twolineshloka
{एवमुक्त्वा ततः पार्थं विसृज्य च महाबलः}
{न्यहनत्पाण्डवीं सेनां वज्रहस्त इवासुरीम्}


% Check verse!
ततोऽपायाद्द्रुतं राजन्व्रीडन्निव नरेश्वरः
\threelineshloka
{अथापयातं राजानं मत्वाऽन्वीयुस्तमच्युतम्}
{चेदिपाण्डवपाञ्चालाः सात्यकिश्च महारथः}
{द्रौपदेयास्तथा शूरा माद्रीपुत्रौ च पाण्डवौ}


\twolineshloka
{ततो युधिष्ठिरानीकं दृष्ट्वा कर्णः पराङ्मुखम्}
{कुरुभिः सहितो वीरः प्रहृष्टः पृष्ठतोऽन्वगात्}


\twolineshloka
{भेरीशङ्खमृदङ्गानां कार्मुकाणां च निःस्वनः}
{बभूव धार्तराष्ट्राणां सिंहनादरवस्तथा}


\twolineshloka
{युधिष्ठिरस्तु कौरव्य रथमारुह्य सत्वरम्}
{श्रुतकीर्तेर्महाराज दृष्ट्वा तत्कर्णविक्रमम्}


\twolineshloka
{काल्यमानं बलं दृष्ट्वा सूतपुत्रेण मारिष}
{स्वान्योधानब्रवीत्क्रुद्वो निघ्नतैतान्किमासत}


\twolineshloka
{ततो राज्ञाऽभ्यनुज्ञाताः पाण्डवानां महारथाः}
{भीमसेनमुखाः सर्वे पुत्रांस्ते प्रत्युपाद्रवन्}


\threelineshloka
{अभवत्तुमुलः शब्दो योधानां तत्र भारत}
{रथहस्त्यश्वपत्तीनां द्रवतां निनदो महान्}
{उद्यतप्रतिविष्टानां शस्त्राणां च ततस्ततः}


\twolineshloka
{आगच्छत प्रहरत क्षिप्रं विपरिधावत}
{इति ब्रुवाणा ह्यन्योन्यं जघ्नुर्योधा महारणे}


\twolineshloka
{अभ्रच्छायेव तत्रासीच्छरवृष्टिभिरम्बरे}
{समावृतैर्नरवरैर्निघ्नद्भिरितरेतरम्}


\threelineshloka
{विपताकध्वजच्छत्रा व्यश्वसूतरथायुधाः}
{व्यङ्गाङ्गावयवाः पेतुः क्षितौ क्षीणाः क्षितीश्वराः}
{शिखराणीव शेलानां वज्रभिन्नानि मानिष}


\twolineshloka
{सारोहा निहताः पेतुर्द्विपा भिन्ना महीतले}
{छिन्नभिन्नविपर्यस्तवर्मालङ्कारविग्रहाः}


\threelineshloka
{सारोहास्तुरगाः पेतुर्हतवीराः सहस्रशः}
{विप्रविद्वायुधाङ्गाश्च द्विरदा रथिभिर्हताः}
{प्रतिवीरैश्च सम्मर्दे पत्तिसङ्घाः सहस्रशः}


\twolineshloka
{विशालायतताम्राक्षैः पद्मेन्दुसदृशाननैः}
{शिरोभिर्युद्धशौण्डानां सर्वतः संवृता मही}


\twolineshloka
{यथा भुवि तथा व्योम्नि निःस्वनं शुश्रुवुर्जनाः}
{विमानेऽप्सरसां सङ्घैर्गीतवादित्रनिःस्वनैः}


\twolineshloka
{हतानभिमुखान्वीरान्वीरैः शतसहस्रशः}
{आरोप्यारोप्य गच्छन्ति विमानेष्वप्सरोगणाः}


\twolineshloka
{तद्दृष्ट्वा महदाश्चर्यं प्रत्यक्षं स्वर्गलिप्सया}
{प्रहृष्टमनसः शूराः क्षिप्रं जघ्नुः परस्परम्}


\twolineshloka
{रथिनो रथिभिः सार्धं चित्रं युयुधुराहवे}
{पत्तयः पत्तिभिर्नागाः सह नागैर्हयैर्हयोः}


\twolineshloka
{एवं प्रवृत्ते सङ्ग्रामे गजवाजिनरक्षये}
{सैन्येन रजसा वृत्तते स्वे स्वाञ्जघ्नुः परे परान्}


\twolineshloka
{कचाकचं युद्धमासीद्दन्तादन्ति नखानखि}
{मुष्टियुद्धं नियुद्धं च देहपाप्मविनाशनम्}


\twolineshloka
{तथा वर्ततति सङ्ग्रमे गजवाजिनरक्षये}
{नराश्वनागदेहेभ्यः प्रसृता लोहितापगा}


\twolineshloka
{गजाश्वनरदेहान्सा व्युवाह पतितान्बहून्}
{नराश्वगजसम्बाधे नराश्वगजसादिनाम्}


\twolineshloka
{लोहितोदा महाघोरा मांसशोणितकर्दमा}
{नराश्वगजदेहान्वै वहन्ती भीरुभीषणा}


\twolineshloka
{तस्या नद्याः परं पारं व्रजन्ति विजयैषिणः}
{नागेन च प्लवन्तो वै निमज्ज्योन्मज्ज्य चापरे}


\twolineshloka
{ते तु लोहितदिग्धाङ्गा रक्तवर्मायुधाम्बराः}
{सस्नुस्ततस्यां पपुश्चास्रं मम्लुश्च भरतर्षभ}


\threelineshloka
{रथानश्वान्नारान्नागानायुधाभरणानि च}
{वर्माणि चाप्यपश्याम पतितानि सहस्रशः}
{खं द्यां भूमिं दिशश्चैव प्रायः पश्याम लोहिताः}


\threelineshloka
{लोहितस्य तु गन्धेन स्पर्देन च रसेन च}
{रूपेण चातिरक्ततेन शब्देन च विसर्पता}
{विषादः सुमहानासीत्प्रायः सैन्यस्य भारत}


\twolineshloka
{तत्तु विप्रहतं सैन्यं भीमसेनमुखैस्तव}
{भूयः समाद्रवन्वीराः सातत्यकिप्रमुखास्तदा}


\twolineshloka
{तेषामापतततां वेगमविषह्यं निरीक्ष्य च}
{पुत्राणां ते महासैन्यमासीद्राजन्पराङ्मुखम्}


\twolineshloka
{तत्प्रकीर्णरथाश्वेभं नरवाजिसमाकुलम्}
{विध्वस्तवर्मकवचं प्रविद्वायुधकार्मुकम्}


\twolineshloka
{व्यद्रवत्तावकं सैन्यं लोड्यमानं समन्ततः}
{सिंहार्दितमिवारण्ये यथा गजकुलं तथा}


\chapter{अध्यायः ४९}
\twolineshloka
{सञ्जय उवाच}
{}


\twolineshloka
{तानभिद्रवतो दृष्ट्वा पाण्डवांस्तावकं बलम्}
{}


% Check verse!

% Check verse!

% Check verse!

% Check verse!

% Check verse!

% Check verse!

% Check verse!

% Check verse!

% Check verse!
अन्तमद्य गमिष्यामि तस्य दुःखस्य पार्षत
\twolineshloka
{हन्तास्म्यद्य रणे कर्णं स वा मां निहनिष्यति}
{सङ्ग्रामेऽद्य सुघोरेऽस्मिन्सत्यमेतद्ब्रवीमि वः}


\twolineshloka
{राजानमद्य भवतां न्यासभूतं ददानि वै}
{तस्य संरक्षणे सर्वे यतध्वं विगतज्वराः}


\twolineshloka
{एवमुक्त्वा महाबाहुः प्रायादाधिरथिं प्रति}
{सिंहनादेन महता सर्वाः सन्नादयन्दिशः}


\threelineshloka
{दृष्ट्वा त्वरितमायान्तं भीमं युद्धाभिनन्दिनम्}
{सूतपुत्रमथोवाच मद्राणामीश्वरो विभुः ॥शल्य उवाच}
{}


\twolineshloka
{पश्य कर्ण महाबाहुं सङ्क्रुद्धं पाण्डुनन्दनम्}
{दीर्घकालार्जितं क्रोधं मोक्तुकामं त्वयि ध्रुवम्}


\twolineshloka
{ईदृशं नास्य रूपं मे दृष्टपूर्वं कदाचन}
{अभिमन्यौ हते कर्ण राक्षसे च घटोत्कचे}


\threelineshloka
{त्रैलोक्यस्य समस्तस्य शक्तः क्रुद्धो निवारणे}
{बिभर्ति सदृशं रूपं युगान्ताग्निसमप्रभम् ॥सञ्जय उवाच}
{}


\twolineshloka
{इति ब्रुवति राधेयं मद्राणामीश्वरे नृप}
{अभ्यवर्तत वै कर्णं क्रोधदीप्ततो वृकोदरः}


\twolineshloka
{अथागतं तु सम्प्रेक्ष्य भीमं युद्धाभिनन्दिनम्}
{अब्रवीद्वचनं शल्यं राधेयः प्रहसन्निव}


\twolineshloka
{यदुक्तं वचनं मेऽद्य त्वया मद्रजनेश्वर}
{भीमसेनं प्रति विभो तत्सत्यं नात्र संशयः}


\twolineshloka
{एष शूरश्च वीरश्च क्रोधनश्च वृकोदरः}
{निरपेक्षः शरीरे च प्राणतश्च बलाधिकः}


\threelineshloka
{अज्ञातवासं वसता विराटनगरे तदा}
{द्रौपद्याः प्रियकामेन केवलं बाहुसंश्रयात्}
{गूढभावं समाश्रित्य कीचकः सगणो हतः}


\twolineshloka
{सोऽद्य सङ्ग्रामशिरसि सन्नद्वः क्रोधमूर्च्छितः}
{किं करोद्यतदण्डेन मृत्युनापि व्रजेद्रणम्}


\twolineshloka
{चिरकालाभिलषितो ममायं तु मनोरथः}
{अर्जुनं समरे हन्यां मां वा हन्याद्वनञ्जयः}


% Check verse!
स मे कदाचिदद्यैव भवेद्भीमसमागमात्
\fourlineindentedshloka
{निहते भीमसेने वा यदि वा विरथीकृते}
{अभियास्यति मां पार्थस्तन्मे साधु भविष्यति}
{अत्र यन्मन्यसे प्राप्तं तच्छीघ्रं सम्प्रधारय ॥सञ्जय उवाच}
{}


\twolineshloka
{एतच्छ्रुत्वा तु वचनं राधेयस्यामितौजसः}
{उवाच वचनं शल्यः सूतपुत्रं तथागतम्}


\twolineshloka
{अभियाहि महाबाहो भीमसेनं महाबलम्}
{निरपेक्षश्च युध्यस्व शक्तिं स्वां सम्प्रदर्शयन्}


\twolineshloka
{यस्ते कामोऽभिलषितश्चिरात्प्रभृति हृद्गततः}
{स वै सम्पत्स्यते कर्ण सत्यमेतद्ब्रवीमि ते}


\twolineshloka
{एवमुक्ते ततः कर्णः शल्यं पुनरभाषत}
{हन्ताऽहमेनं संरब्धं मां वा हन्ता वृकोदरः}


\twolineshloka
{एवमुक्त्वा महाराज राधेयो रथिनां वरः}
{युद्धे मनः समाधाय याहि याहीत्यचोदयत्}


\twolineshloka
{ततः प्रायाद्रथेनाशु शल्यस्तत्र विशाम्पते}
{यत्र भीमो महेष्वासो व्यद्रावयत वाहिनीम्}


\twolineshloka
{ततस्तूर्यनिनादश्च भेरीणां च महास्वनः}
{उदतिष्ठच्च राजेन्द्र कर्णभीमसमागमे}


\twolineshloka
{भीमसेनोऽथथ सङ्क्रुद्धस्तस्य सैन्यं दुरासदम्}
{नाराचैर्विमलैस्तीक्ष्णैर्दिशः प्राद्रावयद्बली}


\twolineshloka
{स सन्निपातस्ततुमुलो घोररूपो विशाम्पते}
{आसीद्रौद्रो महाराज कर्णपाण्डवयोर्मृधे}


\twolineshloka
{ततो मुहूर्ताद्राजेन्द्र नातिकृच्छ्राद्धसन्निव}
{भीमसेनो महाबाहुः कर्णं प्रेप्सुरभिद्रवत्}


\threelineshloka
{समापतन्तं सम्प्रेक्ष्य कर्णो वैकर्तनो वृषा}
{आजघान सुसङ्क्रुद्धो नाराचेन स्तनान्तरे}
{पुनश्चैनममेयात्मा शरवर्षैरवाकिरत्}


\twolineshloka
{स विद्वः सूतपुत्रेण च्छादयामास पत्रिभिः}
{विव्याध निशितैः कर्णं नवभिर्नतपर्वभिः}


% Check verse!
तस्य कर्णो धनुर्मध्ये द्विधा चिच्छेद पत्रिभिः
\twolineshloka
{अथैनं छिन्नधन्वानं प्रतत्यविध्यत्स्तनान्तरे}
{नाराचेन सुतीक्ष्णेन सर्वावरणभेदिना}


\threelineshloka
{सोऽन्यत्कार्मुकमादाय सूततपुत्रं वृकोदरः}
{राजन्मर्मसु मर्मज्ञो विव्याध निशितैः शरैः}
{ननाद बलवन्नादं कम्पयन्निव रोदसी}


\twolineshloka
{ततं कर्णः पञ्चविंशत्या नाराचानां समार्पयत्}
{मदोत्कटं वने दृप्ततमुल्काभिरिव कुञ्जरम्}


\twolineshloka
{ततः सायकभिन्नाङ्गः पाण्डवः क्रोधमूर्च्छितः}
{संरम्भामर्षताम्राक्षः सूतपुत्रवधेप्सया}


\twolineshloka
{स कार्मुके महावेगं भारसाधनमुत्तमम्}
{गिरीणामपि भेत्तारं सायकं समयोजयत्}


\twolineshloka
{विकृष्य बलवच्चापमाकर्णादतिमारुतिः}
{तं मुमोच महेष्वासः क्रुद्धः कर्णजिघांसया}


\twolineshloka
{स विसृष्टो बलवता बाणो वज्राशनिस्वनः}
{अदारयद्रणे कर्णं वज्रवेगो यथाऽचलम्}


\threelineshloka
{स भीमसेनाभिहतः सूतपुत्रः कुरूद्वह}
{निषसाद रथोपस्थे विसंज्ञः पृतनापतिः}
{`रुधिरेणावसिक्ताङ्गो गतासुवदरिन्दमः}


\twolineshloka
{एतस्मिन्नन्तरे दृष्ट्वा मद्रराजो वृकोदरम्}
{जिघांसुं छेत्तुमायान्तं साधयन्निदमब्रवीत्}


\twolineshloka
{भीमसेन महाबाहो यतत्तत्वां वक्ष्यामि तच्छृणु}
{वचनं हेतुसम्पन्नं श्रुत्वा चैतत्तथा कुरु}


\threelineshloka
{अर्जुनेन प्रतिज्ञातो वधः कर्णस्य शुष्मिणः}
{तां तथा कुरु भद्रं ते प्रतिज्ञां सव्यसाचिनः ॥भीमसेन उवाच}
{}


\twolineshloka
{दृढव्रतत्वं पार्थस्य जानामि नृपसत्तम}
{राज्ञस्तु धर्षणं पापः कृतवान्मम सन्निधौ}


\threelineshloka
{ततः कोपाभिभूतेन शेषं न गणितं मया}
{पतिते चापि राधेये न मे मन्युः शमं गतः)}
{जिह्वोद्वरणमेवास्य प्राप्तकालं मतं मम}


\threelineshloka
{अनेन सुनृशंसेन समवेतेषु राजसु}
{अस्माकं शृण्वतां शल्य यानि वाक्यानि मातुल}
{असहेयानि नीचानि बहूनि श्रावितानि भो}


\threelineshloka
{नूनं चैतत्प्रतिज्ञातं दूरस्थस्यापि पार्थिव}
{छेदनं चास्य जिह्वायास्तदैवाकाङ्क्षितं मया}
{राज्ञस्तु प्रियकामेन कालोऽयं परिपालितः}


\twolineshloka
{भवता तु यदुक्तोऽस्मि वाक्यं हेत्वर्थसंहितम्}
{तद्गृहीतं महारात कटुकस्थमिवौषधम्}


\twolineshloka
{हीनप्रतिज्ञो बीभत्सुर्न हि जीवेत कर्हिचित्}
{अस्मिन्विनष्टे नष्टाः स्मः सर्व एव सकेशवाः}


\twolineshloka
{अद्य चैव नृशंसात्मा पापः पापकृतां वरः}
{गमिष्यतति परीभावं दृष्टमात्रः किरीटिना}


\threelineshloka
{युधिष्ठिरस्य कोपेन पूर्वं दग्धो नृशंसकृत्}
{त्वया संरक्षिततस्तत्वद्य मत्समीपादुपागतः ॥सञ्जय उवाच}
{'}


\twolineshloka
{एवं मद्राधिपः श्रुत्वा विसंज्ञं सूतनन्दनम्}
{अपोवाह रथेनाजौ कर्णमाहवशोभिनम्}


\twolineshloka
{पराजिते ततः कर्णे धार्तराष्ट्री महाचमूः}
{व्यपायात्सर्वतो भग्ना हाहाभूता समन्ततः}


\chapter{अध्यायः ५०}
\twolineshloka
{धृतराष्ट्र उवाच}
{}


\twolineshloka
{सुदुष्करमिदं कर्म कृतं भीमेन सञ्जय}
{येन कर्णो महाबाहू रथोपस्थे निपातितः}


\twolineshloka
{कर्णो ह्येको रणे हन्ता पाण्डवान्सृञ्जयैः सह}
{इति दुर्योधनः सूत प्राब्रवीन्मां मुहुर्मुहुः}


\threelineshloka
{पराजितं तु राधेयं दृष्ट्वा भीमेन संयुगे}
{ततः परं किमकरोत्पुत्रो दुर्योधनो मम ॥सञ्जय उवाच}
{}


\twolineshloka
{विमुखं प्रेक्ष्य राधेयं सूतपुत्रं महाहवे}
{पुत्रस्तव महाराज सोदर्यानिदमब्रवीत्}


\twolineshloka
{शीघ्रं गच्छत भद्रं वो राधेयं परिरक्षत}
{भीमसेनभयेऽगाधे म़ज्जन्तं व्यसनार्णवे}


\twolineshloka
{ते तु राज्ञा समादिष्टा भीमसेनं जिघांसवः}
{अभ्यवर्तन्त सङ्क्रुद्धाः पतङ्गाः पावकं यथा}


\twolineshloka
{श्रुतायुर्दुर्धरः क्राथो विवित्सुर्विकटः समः}
{निषङ्गी कवची पाशी तथा नन्दोपनन्दकौ}


\twolineshloka
{दुष्प्रधर्षः सुबाहुश्च वातवेगसुवर्चसौ}
{धनुर्ग्राहो दुर्मदश्च जलसन्धः शलः सहः}


\twolineshloka
{एते रथैः परिवृता वीर्यवन्तो महाबलाः}
{भीमसेनं समासाद्य समन्तात्पर्यवारयन्}


\twolineshloka
{ते व्यमुञ्चञ्छरव्रातान्नानालिङ्गान्समन्ततः}
{स तैरभ्यर्द्यमानस्तु भीमसेनो महाबलः}


\twolineshloka
{तेषामापततां क्षिप्रं सुतानां ते जनाधिप}
{रथैः पञ्चशतैः सार्धं पञ्चाशदहनद्रथान्}


\twolineshloka
{विवित्सोस्तु ततः क्रुद्धो भल्लेनापाहरच्छिरः}
{भीमसेनो महाराज तत्पपात हतं भुवि}


\twolineshloka
{सकुण्डलशिरस्त्राणं पूर्णचन्द्रोपमं तथा}
{`अशोभत महाराज पूर्णचन्द्र इवाम्बरे'}


\twolineshloka
{तं दृष्ट्वा निहतं शूरं भ्रातरः सर्वतस्तदा}
{अभ्यद्रवन्त समरे भीमं भीमपराक्रमम्}


\twolineshloka
{ततोऽपराभ्यां भल्लाभ्यां पुत्रयोस्ते महाहवे}
{जहार समरे प्राणान्भीमो भीमपराक्रमः}


\twolineshloka
{तौ धरामन्वपद्येतां वातरुग्णाविव द्रुमौ}
{विकटश्च सहश्चोमौ देवपुत्रोपमौ नृप}


\twolineshloka
{ततस्तु त्वरितो भीमः क्राथं निन्ये यमक्षयम्}
{नाराचेन सुतीक्ष्णेन स हतो न्यपतद्भुवि}


\twolineshloka
{हाहाकारस्ततस्तीव्रः सम्बभूव जनेश्वर}
{वध्यमानेषु वीरेषु तव पुत्रेषु धन्विषु}


\twolineshloka
{तेषां सुलुलिते सैन्ये पुनर्भीमो महाबलः}
{नन्दोपनन्दौ समरे प्रैषयद्यमसादनम्}


\twolineshloka
{ततस्ते प्राद्रवन्भीताः पुत्रास्ते विह्वलीकृताः}
{भीमसेनं रणे दृष्ट्वा कालान्तकयमोपमम्}


\twolineshloka
{पुत्रांस्ते निहतान्दृष्ट्वा सूतपुत्रः सुदुर्मनाः}
{हंसवर्णान्हयान्भूयः प्रैषयद्यत्र पाण्डवः}


\twolineshloka
{ते प्रेषिता महाराज मद्रराजेन वाजिनः}
{भीमसेनरथं प्राप्य समसज्जन्त वेगिताः}


\twolineshloka
{स सन्निपातस्तुमुलो घोररूपो विशाम्पते}
{आसीद्रौद्रो महाराज कर्णपाण्डवयोर्मृधे}


\twolineshloka
{दृष्ट्वा मम महाराज तौ समेतौ महारथौ}
{आसीद्बुद्विः कथं युद्वमेतदद्य भविष्यति}


\twolineshloka
{ततो भीमो रणश्लाघी छादयामास पत्रिभिः}
{कर्णं रणे महाराज पुत्राणां तव पश्यताम्}


\twolineshloka
{ततः कर्णो भृशं क्रुद्धो भीमं नवभिरायसैः}
{विव्याध परमास्त्रज्ञो भल्लैः सन्नतपर्वभिः}


\twolineshloka
{तान्निहत्य महाबाहुर्भीमो भीमपराक्रमः}
{आकर्णपूर्णैर्विशिखैः कर्णं विव्याध सप्तभिः}


\twolineshloka
{ततः कर्णो महाराज आशीविष इव श्वसन्}
{शरवर्षेण महता छादयामास पाण्डवम्}


\twolineshloka
{भीमोऽपि ततं शरव्रातैश्छादयित्वा महारथम्}
{पश्यतां कौरवेयाणां विननर्द महाबलः}


\threelineshloka
{ततः कर्णो भृशं क्रुद्धो दृढमादाय कार्मुकम्}
{भीमं विव्याध दशभिः कङ्कपत्रैः शिलाशितैः}
{कार्मुकं चास्य चिच्छेद भल्लेन निशितेन च}


\threelineshloka
{ततो भीमो महाबाहुर्हेमपट्टविभूषितम्}
{परिघं घोरमादाय मृतत्युदण्डमिवापरम्}
{कर्णस्य निधनाकाङ्क्षी चिक्षेपातिबलो नदन्}


\twolineshloka
{तमापतन्तं परिघं वज्राशनिसमस्वनम्}
{चिच्छेद बहुधा कर्णः शरैराशीविषोपमैः}


\twolineshloka
{ततः कार्मुकमादाय भीमो दृढतरं तदा}
{छादयामास विशिखैः कर्णं परबलार्दनम्}


\twolineshloka
{ततो युद्धमभूद्धोरं कर्णपाण्डवयोर्मृधे}
{बलीन्द्रयोरिव मुहुः परस्परवधैषिणोः}


\twolineshloka
{ततः कर्णो महाराज भीमसेनं त्रिभिः शरैः}
{आकर्णपूर्णैर्विव्याध दृढमानम्य कार्मुकम्}


\twolineshloka
{सोऽतिविद्धो महेष्वासः कर्णेन बलिनां वरः}
{घोरमादत्त विशिखं कर्णकायावदारणम्}


\twolineshloka
{तस्य भित्त्वा तनुत्राणं भित्त्वा कायं च सायकः}
{प्राविशद्वरणीं राजन्वल्मीकमिव पन्नगः}


% Check verse!
स तेनातिप्रहारेण व्यथितो विह्वलन्निव ॥सञ्चचाल रथे कर्णः क्षितिकम्पे यथाऽचलः
\twolineshloka
{ततः कर्णो महाराज रोषामर्षसमन्वितः}
{भीमं तं पञ्चविंशत्या नाराचानां समार्पयत्}


% Check verse!
चिच्छेद कार्मुकं तूर्णं पाण्डवस्याशु पत्रिणा
\twolineshloka
{तततो मुहूर्ताद्राजेन्द्र नातिकृच्छ्राद्वसन्निव}
{विरथं भीमकर्माणं भीमं कर्णश्चकार ह}


\twolineshloka
{विरथो भरतश्रेष्ठ प्रहसन्ननिलोपमः}
{गदां गृह्य महाबाहुरपतत्स्यन्दनोत्तमात्}


\twolineshloka
{गदया च महाराज कर्णस्य रथकूबरम्}
{पोथयामास सङ्क्रुद्धः समरे शत्रुतापनः}


\twolineshloka
{स क्रोधवशमापन्नः पाण्डुपुत्रः प्रतापवान्}
{विद्राव्य गदया वीरस्तव पुत्रान्महाहवे}


\twolineshloka
{नागान्सप्तशतान्राजन्नीषादन्तान्प्रहारिणः}
{व्यधमत्सहसा भीमः क्रुद्धरूपः परन्तपः}


\twolineshloka
{दन्तवेष्टेषु नेत्रेषु कुम्भेषु च कटेषु च}
{मर्मस्वपि च मर्मज्ञस्तान्नागानवधीद्बली}


\threelineshloka
{`अर्दिता भीमसेनेन विनदन्तो भृशातुराः'}
{ततस्ते प्राद्रवन्भीताः प्रहताश्च पुनःपुनः}
{महामात्रास्तमावव्रुर्मेघा इव दिवाकरम्}


\twolineshloka
{तान्स सप्तशतान्नागान्सारोहायुधकेतनान्}
{भूमिष्ठो गदया जघ्ने वज्रेणेन्द्र इवाचलान्}


\twolineshloka
{ततः सुबलपुत्रस्य नागानतिबलान्पुनः}
{पोथयामास कौन्तेयो द्विपञ्चाशदरिन्दमः}


\twolineshloka
{तथा रथशततं साग्रं पत्तींश्च शतशोऽपरान्}
{न्यहनत्पाण्डवो युद्धे तापयंस्तव वाहिनीम्}


\twolineshloka
{प्रताप्यमानं सूर्येण भीमेन च महात्मना}
{तव सैन्यं सञ्चुकोच चर्माग्नावाहितं यथा}


\twolineshloka
{ते भीमभयसन्त्रस्तास्तावका भरतर्षभ}
{विहाय समरे कर्णं दुद्रुवुर्वै दिशो दश}


\twolineshloka
{रथाः पञ्चशताश्चान्ये हादिनः शरवर्षिणः}
{भीममभ्यद्रवन्घ्नन्तः शरपूगैः समन्ततः}


\twolineshloka
{तान्स पञ्चशतान्वीरान्सपताकध्वजायुधान्}
{पोथयामास गदया भीमो विष्णुरिवासुरान्}


\twolineshloka
{ततः शकुनिनिर्दिष्टाः सादिनः शूरसम्मताः}
{त्रिसाहस्राण्यभिययुः शरशक्त्यृष्टिपाणयः}


\twolineshloka
{तान्प्रत्युद्गम्य यवनान्साश्वारोहांस्तदाऽरिहा}
{विविधान्विचरन्मार्गान्गदया समपोथयत्}


\twolineshloka
{तेषामासीन्महाञ्छब्दस्ताडितानां च सर्वशः}
{अग्निभिर्दह्यमानानां नलानामिव भारत}


\twolineshloka
{एवं सुबलपुत्रस्य त्रिसाहस्रान्हयोत्तमान्}
{हत्वाऽन्यं रथमास्थाय क्रुद्धो राधेयमभ्ययात्}


\twolineshloka
{कर्णोऽपि समरे राजन्धर्मपुत्रमरिन्दमम्}
{स शरैश्छादयामास सारथिं चाप्यपातयत्}


\twolineshloka
{ततः स प्रद्रुतं सैन्यं दृष्ट्वा कर्णो महारथः}
{अन्वधावत्किरन्बाणैः कङ्कपत्रैरजिह्मगैः}


\twolineshloka
{राजानमभिधावन्तं शरैरावृत्य रोदसी}
{क्रुद्धः प्रच्छादयामास शरजालेन मारुतिः}


\twolineshloka
{सन्निवृत्तस्ततस्तूर्णं राधेयः शत्रुकर्शनः}
{भीमं प्रच्छादयामास समन्तान्निशितैः शरैः}


\twolineshloka
{भीमसेनरथप्रेप्सुं कर्णं भारत सात्यकिः}
{अभ्यर्दयदमेयात्मा पार्ष्णिग्रहणकारणात्}


\twolineshloka
{भीमसेनरथप्रेप्सुं कर्णो भारत सात्यकिम्}
{अभ्यवर्तत शैनेयमर्दयञ्छरवृष्टिभिः}


\twolineshloka
{तावन्योन्यं समासाद्य वृषभौ सर्वधन्विनाम्}
{विसृजन्तौ शरान्दीप्तान्विभ्राजेतां मनस्विनौ}


\twolineshloka
{ताभ्यां वियति राजेन्द्र विततततं भीमदर्शनम्}
{कौञ्चपृष्ठारुणं रौद्रं बाणजालं व्यदृश्यत}


\twolineshloka
{नैव सूर्यप्रभा राजन्न दिशः प्रदिशस्तथा}
{प्राज्ञासिष्म वयं ते वा शरैर्मुक्तैः सहस्रशः}


\twolineshloka
{मध्याह्ने तपतो राजन्भास्करस्य महाप्रभाः}
{हृताः सर्वाः शरौघैस्तैः कर्णमाधवयोस्तदा}


\twolineshloka
{सौबलं कृतवर्माणं द्रौणिमाधिरथिं कृपम्}
{संसक्तान्पाण्डवैर्दृष्ट्वा निवृत्ताः कुरवः पुनः}


\twolineshloka
{तेषामापततां शब्दस्तीत्र आसीद्विशाम्पते}
{उद्वृत्तानां यथा वृष्ट्या सागराणां भयावहः}


\twolineshloka
{ते सेने भृशसंसक्ते दृष्ट्वाऽन्योन्यं महाहवे}
{हर्षेण महता युक्ते परिगृह्य परस्परम्}


\twolineshloka
{ततः प्रववृते युद्धं मध्यं प्राप्ते दिवाकरे}
{यादृशं नैवमस्माभिर्दृष्टपूर्वं न च श्रुतम्}


\twolineshloka
{बलौघस्तु समासाद्य बलौघं सहसा रणे}
{उपसर्पत वेगेन वार्योघ इव सागरम्}


\twolineshloka
{आसीन्निनादः सुमहान्बाणौघानां परस्परम्}
{गर्जतां सागरौघाणां यथा स्यान्निःस्वनो महान्}


\twolineshloka
{ते तु सेने समासाद्य वेगवत्यौ परस्परम्}
{एकीभावमनुप्राप्ते नद्याविव समागमे}


\twolineshloka
{ततः प्रववृते युद्धं घोररूपं विशाम्पते}
{कुरूणां पाण्डवानां च लिप्सतां सुमहद्यशः}


\twolineshloka
{शूराणां गर्जतां तत्र ह्यविच्छेदकृता गिरः}
{श्रुयन्ते विविधा राजन्नामान्युद्दिश्य भारत}


\twolineshloka
{यस्य यद्धि रणे व्यङ्गं पितृतो मातृतोऽपि वा}
{कर्मतः शीलतो वाऽपि स तच्छ्रावयते युधि}


\twolineshloka
{तान्दृष्ट्वा समरे शूशांस्तर्जमानान्परस्परम्}
{अभवन्मे मती राजन्नैषामस्तीति जीवितम्}


\twolineshloka
{तेषां दृष्ट्वा तु क्रुद्धानां वपूंष्यमिततेजसाम्}
{अभवन्मे भयं तीव्रं कथमेतद्भविष्यति}


\twolineshloka
{ततस्ते पाण्डवा राजन्कौरवाश्च महारथाः}
{ततक्षुः सायकैस्तूर्णमन्योन्यं विजयैषिणः}


\chapter{अध्यायः ५१}
\twolineshloka
{सञ्जय उवाच}
{}


\twolineshloka
{क्षत्रियास्ते महाराज परस्परवधैषिणः}
{अन्योन्यं समरे जघ्नुः कृतवैरा महौजसः}


\twolineshloka
{रथौघाश्च हयौघाश्च नरौघाश्च समन्ततः}
{गजौघाश्च महाराज संसक्ताश्च परस्परम्}


\twolineshloka
{गदानां परिघाणां च शक्तीनां च परस्परम्}
{प्रासानां भिण्डिपालानां मुसुण्ठीनां च सर्वशः}


\twolineshloka
{सम्पातं चानुपश्याम सङ्ग्रामे भृशदारुणे}
{शलभा इव सम्पेतुः शरवृष्ट्यः समन्ततः}


\twolineshloka
{नागान्नागाः समासाद्य व्यधमन्त परस्परम्}
{हया हयांश्च समरे रथिनो रथिनस्तथा}


\threelineshloka
{पत्तयः पत्तिसङ्घांश्च हयसङ्घांश्च पत्तयः}
{पत्तयो रथमातङ्गान्रथा हस्त्यश्वमेव च}
{नागाश्च समरे त्र्यङ्गं ममृदुः शीघ्रगा नृप}


\twolineshloka
{वध्यतां तत्र शूराणां क्रोशतां च परस्परम्}
{घोरमायोधनं जज्ञे पशूनां घातने यथा}


\twolineshloka
{रुधिरेण समास्तीर्णा बभौ भारत मेदिनी}
{शक्रगोपगणाकीर्णा प्रावृषीव वसुन्धरा}


\twolineshloka
{यथा वा वाससी शुक्ले महारजनरञ्जिते}
{बिभ्रती युवती श्यामा तद्वदासीद्वसुन्धरा}


\twolineshloka
{`बद्वचूडामणिधरैः शिरोभिश्चारुकुण्डलैः}
{उज्झितैर्वृषभाक्षाणां भ्राजते स्म वसुन्धरा'}


\twolineshloka
{शशशोणितदिग्धेव शातकुम्भमयीव च}
{`भूर्बभौ भरतश्रेष्ठ शान्तार्चिर्भिरिवानलैः ॥'}


\twolineshloka
{छिन्नानां चोत्तमाङ्गानां बाहूनां चोरुभिः सह}
{कुण्डलानां पिनद्धानां भूषणानां च भारत}


\threelineshloka
{निष्काणां हेमसूत्राणां शरीराणां च धन्विनाम्}
{चर्मणां सपताकानां सङ्घास्तत्रापतन्भुवि}
{}


\fourlineindentedshloka
{`अगम्यकल्पा पृथिवी सर्वतो भृशकर्दमा' ॥गजा गजान्समासाद्य विषाणैरार्दयन्नृप}
{विषाणाग्रहतास्तत्र भ्राजन्ते द्विरदोत्तमाः}
{रुधिरेणावसिक्ताङ्गा गैरिकप्रस्रवा इव}
{`प्रावृट््काले महाराज मेघा इव सविद्युतः'}


\twolineshloka
{स्यन्दमाना यथा भान्ति पर्वता धातुमण्डिताः}
{`तथा रेजू रणे नागा रुधिरेण समाप्लुताः'}


\twolineshloka
{तोमरान्सादिभिर्मुक्तान्प्रतिमानस्थितान्बहून्}
{हस्तैरुद्धृत्य तान्नागा बभञ्जुश्चापरे रणे}


\twolineshloka
{नाराचैश्छिन्नवर्माणो भ्राजन्ते स्म गजोत्तमाः}
{हिमागमे यथा राजन्व्यभ्रा इव महीधराः}


\twolineshloka
{शरैः कनकपुङ्खैश्च चिता रेर्जुर्गजोत्तमाः}
{उल्काभिः सम्प्रदीप्ताग्राः पर्वता इव भारत}


\twolineshloka
{केचिदभ्याहता नागैर्नागा नगवरोपमाः}
{न चेलुः समरात्तस्माच्छिन्नपक्षा इवाद्रयः}


\twolineshloka
{अपरे प्राद्रवन्नागाः शरार्ता व्रणपीडिताः}
{प्रतिमानैश्च कुम्भैश्च पेतुरुर्व्यां महाहवे}


\twolineshloka
{निषेदुः सिंहवच्चान्ये नदन्तो भैरवान्रवान्}
{वेमुश्च बहवो राजंश्चुक्रुशुश्चापरे गजाः}


\twolineshloka
{हयाश्च निहता बाणैर्हेमभाण्डविभूषिताः}
{निषेदुश्चैव मम्लुश्च बभ्रमुश्च दिशो दश}


\twolineshloka
{अपरे क्लिश्यमानाश्च व्यवेष्टन्त महीतले}
{भावान्बहुविधांश्चक्रुस्ताडिताः शरतोमरैः}


\twolineshloka
{नरास्तु निहता भूमौ कूजन्तस्तत्र मारिप}
{दृष्ट्वा च बान्धवानन्ये पितॄनन्ये पितामहान्}


\twolineshloka
{धावमानान्परांश्चान्ये दृष्ट्वान्ये तत्र भारत}
{ख्यातानि गोत्रनामानि शशंसुरितरेतरम्}


\twolineshloka
{तेषां छिन्ना महाराज भुजाः कनकभूषणाः}
{उद्वेष्टन्ते विवेष्टन्ते भ्रमन्ति ह्युत्पतन्ति च}


\twolineshloka
{निपतन्ति तथैवान्ये स्फुरन्ति च सहस्रशः}
{वेगांश्चान्ये रमे चक्रुः पञ्चास्या इव पन्नगाः}


\twolineshloka
{ते भुजा भोगिभोगाभाश्चन्दनाक्ता विशाम्पते}
{लोहितार्द्रा भृशं रेजुस्तपनीयध्वजा इव}


\twolineshloka
{वर्तमाने तथा घोरे सङ्कुले सर्वतो दिशम्}
{अविज्ञाताः स्म युध्यन्ते विनिघ्नन्तः परस्परम्}


\twolineshloka
{भौमेन रजसाकीर्णे शस्त्रसम्पातसङ्कुले}
{नैव स्वे न परे राजन्व्यज्ञायन्त तमोवृताः}


\twolineshloka
{तथा तदभवद्युद्धं घोररूपं भयानकम्}
{लोहितोदा महानद्यः प्रसस्रुस्तत्र चासकृत्}


\twolineshloka
{शीर्षपाषाणसञ्छन्नाः केशशैवलशाद्वलाः}
{अस्थिमीनसमाकीर्णा धनुःशरगदोडुपाः}


\twolineshloka
{मांसशोणितपङ्किन्यो घोररूपाः सुदारुणाः}
{नदीः प्रवर्तयामासुः सोणितौघविवर्धिनीः}


\twolineshloka
{भीरुवित्रासकारिण्यः शूराणां हर्षवर्धनाः}
{ता नद्यो घोररूपास्तु नयन्त्यो यमसादनम्}


\twolineshloka
{अवगाढान्मज्जयन्त्यः क्षत्रस्याजनयन्भयम्}
{क्रव्यादानां नरव्याघ्र नर्दतां तत्रतत्र ह}


\twolineshloka
{घोरमायोधनं जज्ञे प्रेतराजपुरोपमम्}
{उत्थितान्यगणेयानि कबन्धानि समन्ततः}


\twolineshloka
{नृत्यन्ति वै भूतगणाः सुतृप्ता मांसशोणितैः}
{पीत्वा च शोणितं तत्र वसां भुक्त्वा च भारत}


\twolineshloka
{मेदोमज्जावसामत्तास्तृप्ता मांसस्य चैव ह}
{धावमानाः स्म दृश्यन्ते काकगृघ्रबकास्तथा}


\twolineshloka
{शूरास्तु समरे राजन्भयं त्यक्त्वा सुदुस्त्यजम्}
{योधव्रतसमाख्याताश्चक्रुः कर्माण्यभीतवत्}


\twolineshloka
{शरशक्तिसमाकीर्णे क्रव्यादगणसङ्कुले}
{व्यचरन्त रणे शूराः ख्यापयन्तः स्वपौरुषम्}


\twolineshloka
{अन्योन्यं श्रावयन्ति स्म नामगोत्राणि भारत}
{पितृनामानि च रणे गोत्रनामानि वा विभो}


\twolineshloka
{श्रावयाणाश्च बहवस्तत्र योधा विशाम्पते}
{अन्योन्यमवमृद्रन्तः शक्तितोमरपट्टसैः}


\twolineshloka
{वर्तमाने तथा युद्धे घोररूपे सुदारुणे}
{व्यषीदत्कौरवी सेना भिन्ना नौरिव सागरे}


\chapter{अध्यायः ५२}
\twolineshloka
{सञ्जय उवाच}
{}


\twolineshloka
{वर्तमाने तथा युद्धे क्षत्रियाणां निमज्जने}
{गाण्डीवस्य महाघोषः श्रूयते युधि मारिष}


\twolineshloka
{संशप्तकानामकरोत्कदनं यत्र पाण्डवः}
{कोसलानां तथा राजन्नारायणबलस्य च}


\twolineshloka
{संशप्तकास्तु समरे शरवृष्टिः समन्ततः}
{अपातयन्पार्थमूर्ध्नि जयगृद्धाः प्रमन्यवः}


\twolineshloka
{तां शस्त्रवृष्टिमायान्तीमुरसा धारयन्प्रभुः}
{व्यगाहत परान्पार्थो विनिघ्नन्रथिनां वरान्}


\twolineshloka
{विक्षोभ्य तु रथानीकं कङ्कपत्रैः शिलाशितैः}
{आससाद ततः पार्थः सुशर्माणं महारथम्}


\twolineshloka
{स तस्य शरवर्षाणि ववर्ष रथिनां वरः}
{तथा संशप्तकाश्चैव पार्थस्य समरे स्थिताः}


\twolineshloka
{सुशर्मा तु ततः पार्थं विद्ध्वा दशभिराशुगैः}
{जनार्दनं त्रिभिर्बाणैरहनद्दक्षिणे भुजे}


\twolineshloka
{ततोऽपरेण भल्लेन केतुं पार्थस्य मारिष}
{विव्याध समरे राजन्सुशर्मा क्रोधमूर्च्छितः}


\twolineshloka
{स वानरवरो राजन्विश्वकर्मकृतो महान्}
{ननाद सुमहानादं नृत्यन्निव विभीषयन्}


\twolineshloka
{कपेस्तु निनदं श्रुत्वा सन्त्रस्ता तव वाहिनी}
{भयं विपुलमासाद्य निश्चेष्टा समपद्यत}


\twolineshloka
{ततः सा शुशुभे सेना निश्चेष्टाऽवस्थिता नृप}
{नानापुष्पसमाकीर्णं यथा चित्रीकृतं वनम्}


\twolineshloka
{प्रतिलभ्य ततः संज्ञां योधास्ते कुरुसत्तम}
{अर्जुनं सिषिचुर्बाणैः पर्वतं जलदा इव}


\threelineshloka
{सञ्छाद्य समरे पार्थं परिवव्रुः समन्ततः}
{ते हयान्रथचक्रे तु रथेषां चापि मारिष}
{ग्रहीतुं प्रचक्रमुश्चैव क्रोधाविष्टाः समन्ततः}


\twolineshloka
{निगृह्य तु रथं तस्य योधास्ते तु सहस्रशः}
{रथबन्धं प्रचक्रुर्हि पाण्डवस्यामितौजसः}


\twolineshloka
{रथमारुरुहुः केचित्कृष्णपार्थौ जिघृक्षवः}
{संशप्तकानां योधास्ते सिंहनादांश्च नेदिरे}


\twolineshloka
{अपरे जगृहुश्चैव केशवस्य महाभुजौ}
{पार्थं चैके महाराज रथस्थं जगृहुर्मुदा}


\twolineshloka
{अच्युतः स महाबाहुर्विधुन्वन्रणमूर्धनि}
{पातयामास तान्सर्वान्दुष्टहस्तीव हस्तिपान्}


\twolineshloka
{स रथस्तैर्गृहीतस्तु पाण्डवस्य महात्मनः}
{स्पन्दितुं नाशकद्राजंस्तदद्भुतमिवाभवत्}


\twolineshloka
{ततः पार्थो महाबाहुः संवृतः स्तैर्महारथैः}
{निगृहीतं रथं दृष्ट्वा तांश्चाप्याद्रवतो बहून्}


\twolineshloka
{रथारूढांश्च सुबहून्पदाऽऽक्षिप्य न्यपातयत्}
{अपातयदसम्भ्रान्तः शरैरासन्नयोधिभिः}


\twolineshloka
{तांस्तापयित्वा समरे पार्थः परपुरञ्जयः}
{स्मयन्निव महाबाहुः केशवं वाक्यमब्रवीत्}


\twolineshloka
{पश्य कृष्ण महाबाहो संशप्तकगणान्बहून्}
{कुर्वतोऽसुकरं कर्म मुमूर्षून्कालचोदितान्}


\twolineshloka
{रथबन्धमिमं प्राप्य पृथिव्यां नास्ति कश्चन}
{यः सहेत पुमाँल्लोके मदन्यः क्षत्रियर्षभः}


\twolineshloka
{पश्यैततानद्य समरे मत्प्रयुक्तैः सुतेजनैः}
{पात्यमानान्रणे कृष्ण शरैराशीविषोपमैः}


\twolineshloka
{इत्येवमुक्त्वा बीभत्सुः शङ्खप्रवरमुत्तमम्}
{व्यनादयदमेयात्मा देवदत्तं महामृधे}


\twolineshloka
{देवदत्तस्वनं श्रुत्वा केशवोऽपि महायशाः}
{पाञ्चजन्यस्वनं चक्रे पूरयन्निव रोदसी}


\twolineshloka
{तयोः शङ्खस्वनं श्रुत्वा संशप्तकवरूथिनी}
{सञ्चचाल महाराज वित्रस्ता चाद्रवद्भृशम्}


\twolineshloka
{नागमस्त्रं ततः पार्थः प्रादुश्चक्रे हसन्निव}
{पादबन्धं स तेषां वै चक्रे तेन महास्त्रवित्}


\twolineshloka
{यानुद्दिश्य रणे पार्थः पादबन्धं चकार ह}
{ते बद्धाः पादबन्धेन योधाः संशप्तकास्तदा}


\threelineshloka
{निर्विचेष्टास्तदाऽभूवन्पाण्डवस्यास्त्रतेजसा}
{निर्विचेष्टांस्ततो योधानवधीत्पाण्डुनन्दनः}
{यथेन्द्रः समरे दैत्यांस्तारकस्य वधे पुरा}


\twolineshloka
{ते वध्यमानाः समरे मुमुचुस्तं रथोत्तमम्}
{आयुधानि च सर्वाणि विस्रष्टुमुपचक्रमुः}


\twolineshloka
{ततः सुशर्मा राजेन्द्र गृहीतां वीक्ष्य वाहिनीम्}
{सौपर्णमस्त्रं त्वरितः प्रादुश्चक्रे महारथः}


\twolineshloka
{ततः सुपर्णाः सम्पेतुर्भक्षयन्तो भुजङ्गमान्}
{ततो विदुद्रुवुर्नागास्तान्दृष्ट्वा खेचरान्नृप}


\twolineshloka
{तद्विमुक्तं बलं रेजे पादबन्धाद्विशाम्पते}
{मेघबन्धाद्यथा मुक्तो भास्करस्तापयन्प्रजाः}


\twolineshloka
{विप्रमुक्ताः स्वका योधाः फल्गुनस्य रथं प्रति}
{ससृजुः शरसङ्घांश्च शस्त्रसङ्घांश्च मारिष}


\twolineshloka
{तां महास्त्रमयीं वृष्टिं शरैः सञ्छिद्य भारत}
{अवधीत्सततो योधान्वासविः परवीरहा}


\twolineshloka
{सुशर्मा तु ततो राजन्बाणेन नतपर्वणा}
{अर्जुनं हृदये विद्ध्वा विव्याधान्यैस्त्रिसप्तभिः}


\twolineshloka
{स गाढविद्धो व्यथितो रथोपस्थ उपाविशतत्}
{तत उच्चुक्रुशुः सर्वे हतः पार्थ इति ब्रुवन्}


\twolineshloka
{ततः शङ्खनिनादाश्च भेरीशब्दाश्च पुष्कलाः}
{नानावादित्रनिनदाः सिंहनादाश्च जज्ञिरे}


\threelineshloka
{प्रतिलभ्य ततः संज्ञां श्वेताश्वः कृष्णसारथिः}
{सोऽतिविद्धो महेष्वासः शरैराशीविपोपमैः}
{सुशर्माणं महाराज क्रोधाविष्टो महारथः}


\twolineshloka
{ततः शरशतैः पार्थः सञ्छाद्यैनं क्षणाद्रणे}
{दिशस्तु वारयामास बाणैस्तत्र महास्त्रवित्}


\twolineshloka
{विमुखीकृत्य समरे सुशर्माणं धनञ्जयः}
{ऐन्द्रमस्त्रममेयात्मा प्रादुश्चक्रे हसन्निव}


\twolineshloka
{ततो बाणसहस्राणि तदुत्सृष्टानि मारिष}
{सर्वदिक्षु व्यदृश्यन्त सूदयन्ति रथद्विपान्}


\twolineshloka
{हयान्पत्तींश्च समरे शस्त्रैः शतसहस्रशः}
{रराज समरे राजञ्शक्रो निघ्नन्निवासुरान्}


\twolineshloka
{वध्यमाने ततः सैन्ये भयं सुमहदाविशत्}
{संशप्तकगणानां च गोपालानां च भारत}


\twolineshloka
{न हि तत्र पुमान्कश्चिद्योऽर्जुनं प्रतिबुध्यते}
{पश्यतां तत्र वीराणामहन्यत बलं तव}


% Check verse!
हन्यमानं च तदभून्निश्चेष्टं च पराक्रमे
\twolineshloka
{अयुतं तत्र योधानां हत्वा पाण्डुसुतो रणे}
{व्यभ्राजत महाराज विधूमोऽग्निरिव ज्वलन्}


\twolineshloka
{चतुर्दशसहस्राणि यानि दृष्टनि भारत}
{रथानामयुतं चैव त्रिसहस्राश्च दन्तिनः}


\twolineshloka
{ततः संशप्तका भूयः परिवव्रुर्धनञ्जयम्}
{मर्तव्यमिति निश्चित्य जयं वाप्यनिवर्तनम्}


\threelineshloka
{तत्र युद्धं महच्चासीत्तावकानां विशाम्पते}
{शूरेण बलिना सार्धं पाण्डवेन किरीटिना}
{जित्वा तान्न्यहनत्पार्थः शत्रूञ्शक्र इवासुरान्}


\chapter{अध्यायः ५३}
\twolineshloka
{सञ्जय उवाच}
{}


\twolineshloka
{कृतवर्मा कृपो द्रौणिः सूतपुत्रश्च मारिष}
{उलूकः सौबलश्चैव राजा च सह सोदरैः}


\twolineshloka
{सीदमानां चमूं दृष्ट्वा पाण्डुपुत्रभयार्दिताम्}
{समुज्जह्रुः स्म वेगेन भिन्नां नावमिवार्णवे}


\twolineshloka
{ततो युद्धमतीवासीन्मुहूर्तमिव भारत}
{भीरूणां त्रासजननं शूराणां हर्षवर्धनम्}


\twolineshloka
{कृपेण शरवर्षाणि प्रतिमुक्तानि संयुगे}
{सृञ्जयांश्छादयामासुः शलभानां व्रजा इव}


\twolineshloka
{शिखण्डी च ततः क्रुद्धो गौतमं त्वरितो ययौ}
{ववर्ष शरवर्षाणि समन्ताद्द्विजपुङ्गवम्}


\twolineshloka
{कृपस्तु शरवर्षं तद्विनिहत्य महास्त्रवित्}
{शिखण्डिनं रणे क्रुद्धो विव्याध दशभिः शरैः}


\twolineshloka
{ततः शिखण्डी कुपितः शरैः सप्तभिराहवे}
{कृपं विव्याध कुपितं कङ्कपत्रैरजिह्मगैः}


\threelineshloka
{ततः कृपः शरैस्तीक्ष्णैः सोऽतिविद्धो महारथः}
{`ततः सुनिशितैस्तीक्ष्णैः क्षुरप्रैर्हेमभूषितैः'}
{व्यश्वसूतरथं चक्रे शिखण्डिनमथो द्विजः}


\twolineshloka
{हताश्वात्तु ततो यानादवप्लुत्य महारथः}
{खङ्गं चर्म तथा गृह्य सत्वरं ब्राह्मणं ययौ}


\twolineshloka
{तमापतन्तं सहसा शरैः सन्नतपर्वभिः}
{वारयामास समरे तदद्भुतमिवाभवत्}


\twolineshloka
{तत्राद्भुतमपश्याम शिलानां प्लवनं यथा}
{निश्चेष्टो यद्रणे राजञ्छिखण्डी समतिष्ठत}


\twolineshloka
{कृपेण वारितं दृष्ट्वा शिखण्डिनमथो नृप}
{प्रत्युद्ययौ कृपं तूर्णं धृष्टद्युम्नो महारथः}


\twolineshloka
{धृष्टद्युम्नं ततो यान्तं शारद्वतरथं प्रति}
{प्रतिजग्राह वेगेन कृतवर्मा महारथः}


\twolineshloka
{युधिष्ठिरमथायान्तं शारद्वतरथं प्रति}
{सपुत्रं सहसैन्यं च द्रोणपुत्रो न्यवारयत्}


\twolineshloka
{नकुलं सहदेवं च त्वरमाणौ महारथौ}
{प्रतिजग्राह ते पुत्रः शरवर्षेण वारयन्}


\twolineshloka
{भीमसेनं करूशांश्च केकयान्सह सृञ्जयैः}
{कर्णो वैकर्तनो युद्धे वारयामास भारत}


\twolineshloka
{शिखण्डिने ततो बाणान्कृपः शारद्वतो युधि}
{प्राहिणोत्तवरया युक्तो बीभत्सोरथ सन्निधौ}


\twolineshloka
{ताञ्छरान्प्रेषितांस्तेन समन्तात्स्वर्णभूषितान्}
{चिच्छेद खङ्गमाविध्य भ्रामयंश्च पुनः पुनः}


\twolineshloka
{शतचन्द्रं च तच्चर्म गौतमस्तस्य भारत}
{व्यधमत्सायकैस्तूर्णं तत उच्चुक्रुशुर्जनाः}


\twolineshloka
{स विचर्मा महाराज खङ्गपाणिरुपाद्रवत्}
{कृपस्तं शरसङ्घातैराद्रवन्तमपीडयत्}


\twolineshloka
{शारद्वतशरैर्ग्रस्तं क्लिश्यमानं महाबलः}
{चित्रकेतुसुतो राजन्सुकेतुस्त्वरितो ययौ}


\twolineshloka
{विकिरन्ब्राह्मणं युद्धे बहुभिर्निशितैः शरैः}
{अभ्यापतदमेयात्मा गौतमस्य रथं प्रति}


\twolineshloka
{दृष्ट्वा विषक्तं तं चैव ब्राह्मणं च निवारितम्}
{अपयातस्ततस्तूर्णं शिखण्डी राजसत्तम}


\twolineshloka
{सुकेतुस्तु ततो राजन्गौतमं नवभिः शरैः}
{विद्ध्वा विव्याध सप्तत्या पुनश्चैनं त्रिभिः शरैः}


\twolineshloka
{अथास्य सशरं चापं पुनश्चिच्छेद मारिष}
{सारथिं च शरेणास्य मृशं मर्मस्वताडयत्}


\twolineshloka
{गौतमस्तु ततः क्रुद्धो धनुर्गृह्य नवं दृढम्}
{सुकेतुं त्रिंशता बाणैः सर्वमर्मस्वताडयत्}


\twolineshloka
{स विह्वलितसर्वाङ्गः प्रचचाल रथोत्तमे}
{भूमिकम्पे यथा वृक्षश्चचालाकम्पितो भृशम्}


\twolineshloka
{चलतस्तस्य कायात्तुं शिरो ज्वलितकुण्डलम्}
{सोष्णीषं सशिरस्त्राणं क्षुरप्रेण त्वपातयत्}


\twolineshloka
{तच्छिरः प्रापतद्भूमौ श्येनाहृतमिवामिषम्}
{ततोऽस्य कायो वसुधां पश्चात्प्रापतदच्युत}


\twolineshloka
{तस्मिन्हते महाराज पुत्रास्तस्य पदानुगाः}
{गौतमं समरे त्यक्त्वा दुद्रुवुस्ते दिशो दश}


\twolineshloka
{धृष्टद्युम्नं तु समरे सन्निवार्य महारथः}
{कृतवर्माऽब्रवीद्वृष्टस्तिष्ठतिष्ठेति भारत}


\twolineshloka
{तदभूत्तुमुलं युद्वं वृष्णिपार्षतयो रणे}
{आमिषार्थे यथा युद्धं श्येनयोः क्रुद्धयोर्नृप}


\twolineshloka
{धृष्टद्युम्नस्तु समरे हार्दिक्यं नवभिः शरैः}
{आजघानोरसि क्रुद्वः पीडयन्हृदिकात्मजम्}


\twolineshloka
{कृतवर्मा तु समरे पार्षतेन दृढाहतः}
{पार्षतं सरथं साश्वं छादयामास सायकैः}


\twolineshloka
{सरथश्छादितो राजन्धृष्टद्युम्नो न दृश्यते}
{मेघैरिव परिच्छन्नो भास्करो जलधारिभिः}


\twolineshloka
{विधूय तं बाणगणं शरैः कनकभूषणैः}
{व्यरोचत रणे राजन्धृष्टद्युम्नः कृतव्रणः}


\twolineshloka
{ततस्तु पार्षतः क्रुद्धः शस्त्रवृष्टिं सुदारुणाम्}
{कृतवर्माणमासाद्य व्यसृजत्पृतनापतिः}


\twolineshloka
{तामापतन्तीं सहसा शस्त्रवृष्टिं सुदारुणाम्}
{शरैरनेकसाहस्रैर्हार्दिक्योऽवारयद्युधि}


\twolineshloka
{दृष्ट्वा तु वारितां युद्धे शस्त्रवृष्टिं दुरासदाम्}
{कृतवर्माणमासाद्य वारयामास पार्षतः}


\twolineshloka
{`यथा युग्मरथेनाजौ वाहान्वाहेरमिश्रयत्}
{गृहीत्वा चर्म खङ्गं च रथं तस्यावपुप्लुवे}


\twolineshloka
{मिलितेष्वथ वाहेषु प्रत्यासन्ने च पार्षते}
{दृष्ट्वाऽपदानं तस्याशु गदां जग्राह सात्वतः}


\twolineshloka
{गदापाणिस्ततो राजन्रथात्तूर्णमवप्लुतः}
{तमदृष्ट्वा रथोपस्थे सारथिं समताडयत्}


\twolineshloka
{खङ्गेन शितधारेण स हतः प्रापतद्रथात्}
{कृतवर्मा ततो हृष्टस्तलशब्दं चकार ह}


\twolineshloka
{पार्षतं चाब्रवीद्राजन्नेह्येहीति पुनःपुनः}
{स तं न ममृषे युद्वे तलशब्दं समीरितम्}


\twolineshloka
{अवप्लुत्य रथात्तस्मात्स्वरथं पुनरास्थितः}
{अभ्ययात्स तु तत्तूर्णं तिष्ठतिष्ठेति चाब्रवीत्}


\twolineshloka
{ततो राजन्महेष्वासं कृतवर्माणमाशु वै}
{गदां गृह्य पुनर्वेगात्कृतवर्माणमाहनत्}


\twolineshloka
{सोऽतिविद्वो बलवता न्यपतन्मूर्च्छया हतः}
{श्रुतर्वा रथमारोप्य अपोवाह रणाजिरात्'}


\twolineshloka
{धृष्टद्युम्नस्तु समरे दृष्ट्वा शत्रुं महारथः}
{कौरवान्समरे तूर्णं वारयामास सायकैः}


\twolineshloka
{ततस्ते तावका योधा धृष्टद्युम्नमुपाद्रवन्}
{सिंहनादरवांश्चक्रुस्ततो युद्धमवर्तत}


\chapter{अध्यायः ५४}
\twolineshloka
{सञ्जय उवाच}
{}


\twolineshloka
{द्रौणिर्युधिष्ठिरं दृष्ट्वा शैनेयेनाभिरक्षितम्}
{द्रौपदेयैस्तथा शूरैरभ्यवर्तत हृष्टवत्}


\twolineshloka
{किरन्निषुगणान्घोरान्हेमपुङ्खाञ्शिलाशितान्}
{दर्शयन्विविधान्मार्गाञ्शिक्षया लघुहस्तवान्}


\twolineshloka
{खं पुरयञ्शरैस्तीक्ष्णैर्वेगवद्भिः समन्ततः}
{युधिष्ठिरस्य समरे दिशः सर्वाः समावृणोत्}


\twolineshloka
{भारद्वाजशरैश्छन्नं न प्राज्ञायत किञ्चन}
{बाणभूतमभूत्सर्वमायोधनशिरो महत्}


\twolineshloka
{बाणजालं दिविष्ठं तत्स्वर्णपुङ्खविभूषितम्}
{शुशुभे भरतश्रेष्ठ विमानमिव विष्ठितम्}


\twolineshloka
{तेन च्छन्ने रणे राजन्बाणजालेन भास्वता}
{अभ्रच्छायेव सजज्ञे हृतरश्मिस्तमोनुदः}


\twolineshloka
{तदद्भुतमपश्याम बाणभूते नभस्थले}
{न स्म सम्पतते भूतं द्रौणेर्दृष्ट्वा पराक्रमम्}


\twolineshloka
{सात्यकिर्यतमानस्तु धर्मराजस्तु पाण़्डवः}
{तथेतराणि सैन्यानि न स्म चक्रुः पराक्रमम्}


\twolineshloka
{लाघवं द्रोणपुत्रस्य दृष्ट्वा तत्र महारथाः}
{व्यस्मयन्त महाराज न चैनं प्रत्युदीक्षितुम्}


\twolineshloka
{शेकुस्ते सर्वराजानस्तपन्तमिव भास्करम्}
{वध्यमाने ततः सैन्ये द्रौपदेया महारथाः}


\twolineshloka
{सात्यकिर्धर्मराजश्च पाञ्चालाश्चापि सङ्गताः}
{त्यक्त्वा मृत्युभयं घोरं द्रौणायनिभुपाद्रवन्}


\twolineshloka
{सात्यकिः सप्तविंशत्या द्रौणिं विद्ध्वा शिलीमुखैः}
{पुनर्विव्याध नाराचैः सप्तभिः स्वर्णभूषितैः}


\threelineshloka
{युधिष्ठिरस्त्रिसप्तत्या प्रतिविन्ध्यश्च सप्तभिः}
{श्रुतकर्मा त्रिभिर्वाणैः श्रुतकीर्तिश्च सप्तभिः}
{}


\twolineshloka
{सुतसोमस्तु नवभिः शतानीकश्च सप्तभिः}
{अन्ये च वहवः शूरा विव्यधुस्तं समन्ततः}


\twolineshloka
{स तु क्रुद्धस्ततो राजन्नाशीविप इव श्वसन्}
{सात्यकिं पञ्चविंशत्या प्राविध्यत शिलीमुखैः}


\twolineshloka
{श्रुतकीर्तिं च नवभिः सुतसोमं च पञ्चभिः}
{अष्टभिः श्रुतकर्माणं प्रतिबिन्ध्यं त्रिभिः शरैः}


\threelineshloka
{शतानीकं च नवभिर्धर्मपुत्रं च पञ्चभिः}
{तथेतरांस्ततः शूरान्द्वाभ्यां द्वाभ्यामताडयत्}
{श्रुतकीर्तेस्तथा चापं चिच्छेद निशितैः शरैः}


\twolineshloka
{अथान्यद्वनुरादाय श्रुतकीर्तिर्महारथः}
{द्रौणायनिं त्रिभिर्विद्व्वा विव्याधान्यैः शितैः शरैः}


\twolineshloka
{ततो द्रौणिर्महाराज शरवर्षेण भारिष}
{छादयामास तत्सैन्यं समन्ताद्भरतर्षभ}


\twolineshloka
{ततः पुनरमेयात्मा धर्मराजस्य कार्मुकम्}
{द्रौणिश्चिच्छेद विहसन्विव्याध च शरैस्त्रिभिः}


\twolineshloka
{ततो धर्मसुतो राजन्प्रगृह्यान्यन्महद्धनुः}
{द्रोणपुत्रं त्रिपष्टxx ततु बाह्वोरुरसि चार्पययत्}


\twolineshloka
{सात्यकिस्तु ततः क्रुद्धो द्रौणेः प्रहरतो रणे}
{अर्धचन्द्रेणतीक्ष्णेन धनुश्छित्त्वाऽनदद्भृशम्}


\twolineshloka
{छिन्नधन्वा ततो द्रौणिः शक्त्या शक्तिमतां वरः}
{सारथिं पातयामास शैनेयस्य रथाद्रुतम्}


\twolineshloka
{अथान्यद्वनुरादाय द्रोणपुत्रः प्रतापवान्}
{शैनेयं शरवर्षेण च्छादयामास भारत}


\twolineshloka
{तस्याश्वाः प्रद्रुताः सङ्ख्ये पतिते रथसारथौ}
{तत्रतत्रैव धावन्तः समदृश्यन्त भारत}


\twolineshloka
{युधिष्ठिरपुरोगास्तु द्रौणिं शस्त्रभृतां वरम्}
{अभ्यवर्षन्त वेगेन विसृजन्तः शिताञ्छरान्}


\twolineshloka
{सहसा पततस्तान्वै क्रुद्वरूपान्परन्तपः}
{प्रहसन्प्रतिजग्राह द्रोणपुत्रो महारणे}


\twolineshloka
{ततः शरशतज्वालः सेनाकक्षं महारणे}
{द्रौणिर्ददाह समरे कक्षमग्निरिवोत्थितः}


\twolineshloka
{तद्बलं पाण्डुपुत्रस्य द्रोणपुत्रप्रतापितम्}
{चुक्षुभे भरतश्रेष्ठ तस्मिन्नेव चमूमुखे}


\twolineshloka
{तद्बलं पाण्डवेयस्य द्रोणपुत्रः शरार्चिषा}
{तापयन्भरतश्रेष्ठ गभस्तिभिरिवांशुमान्}


\twolineshloka
{तत्पच्यमानं मर्षेण ब्राह्मणस्य च सायकैः}
{क्षुभ्यते पाण्डवं सैन्यं तिमिनेव नदीमुखम्}


\twolineshloka
{दृष्ट्वा चैव महाराज द्रोणपुत्रपराक्रमम्}
{सर्वान्दुर्योधनः पार्थान्हतान्युद्वेऽभ्यमन्यत}


\twolineshloka
{युधिष्ठिरस्तु त्वरितो द्रौणिं पीड्य महारथम्}
{अब्रवीद्द्रोणपुत्रं स रोषामर्षसमन्वितः}


\twolineshloka
{`जानामि त्वां युधि श्रेष्ठ वीर्यवन्तं बलान्वितम्}
{कृतास्त्रं कृतिनं चैव तथा लघुपराक्रमम्}


\twolineshloka
{बलमेतद्भवान्सर्वं पार्षते यदि दर्सयेत्}
{ततस्त्वां बलवन्तं च कृतविद्यं च विद्महे}


\twolineshloka
{न हि वै पार्पतं दृष्ट्वा समरे शत्रुसूदनम्}
{भवेत्तव बलं किञ्चिद्ब्रवीमि त्वां न तु द्विजम्'}


\twolineshloka
{नैव नाम तव प्रीतिर्नैव नाम कृतज्ञता}
{यतस्त्वं पुरुषव्याघ्र मामेवाद्य जिघांससि}


\twolineshloka
{ब्राह्मणेन तपः कार्यं शमश्च दमएव च}
{क्षत्रियेण धनुर्नाभ्यं स भवान्ब्राह्मणब्रुवः}


\twolineshloka
{मिषतस्ते महावाहो युधि जेष्यामि कौरवान्}
{कुरुष्व समरे कर्म ब्रह्मबन्धे यथेष्टतः}


\twolineshloka
{एवमुक्तो महाराज द्रोणपुत्रः सयन्निव}
{युक्त तत्त्वं च सञ्चिन्त्य नोत्तरं किञ्चिदब्रवीत्}


\twolineshloka
{अनुक्त्वा च ततः किञ्चिच्छरवर्षेण पाण्डवम्}
{छादयामास समरे क्रुद्धोऽन्तक इव प्रजाः}


\twolineshloka
{स च्छाद्यमानस्तु तदा द्रोणपुत्रेण मारिष}
{पार्थोऽपयातः शीघ्रं वै प्रगृह्य महतीं चमूम्}


\twolineshloka
{अपयाते ततस्तस्मिन्धर्मपुत्रे युधिष्ठिरे}
{द्रोणपुत्रः स्थितो राजन्प्रत्यादेशान्महात्मनः}


\twolineshloka
{ततो युधिष्ठिरो राजंस्त्यक्त्वा द्रौणिं महाहवे}
{प्रययौ तावकं सैन्यं युक्तः क्रुराय कर्मणे}


\chapter{अध्यायः ५५}
\twolineshloka
{सञ्जय उवाच}
{}


\twolineshloka
{भीमसेनं सपाञ्चालं चेदिकेकयसंवृतम्}
{सेनापतिः स्वयं क्रुद्धो वारयामास सायकैः}


\twolineshloka
{ततस्तु चेदिकारूशान्सृढ्जयांश्च महारथान्}
{कर्णो जघान समरे भीमसेनस्य पश्यतः}


\twolineshloka
{भीमसेनस्ततः कर्णे विहाय रथसत्तमम्}
{प्रययौ कौरवं सैन्यं कक्षमग्निरिव ज्वलन्}


\twolineshloka
{सूतपुत्रोऽपि समरे पाञ्चालान्केकयांस्तथा}
{सृञ्जयांश्च महेष्वासान्निजघानं सहस्रशः}


\twolineshloka
{संशप्तकेषु पार्थश्च कौरवेषु वृकोदरः}
{पाञ्चालेषु तथा कर्णः क्षयं चक्रुर्महारथाः}


\twolineshloka
{ते क्षत्रिया दह्यामानास्त्रिभिस्तैः पावकोपमैः}
{जग्मुर्विनाशं समरे राजन्दुर्मन्त्रिते तव}


\twolineshloka
{ततो दुर्योधनः क्रुद्दो नकुलं नवभिः शरैः}
{विव्याध भरतश्रेष्ठ चतुरश्चास्य वाजिनः}


\twolineshloka
{ततः परममेयात्मा तव पुत्रो जनाधिप}
{क्षुरेण सहदेवस्य ध्वजं चिच्छेद काञ्चनम्}


\twolineshloka
{नकुलस्तु ततः क्रुद्धस्तव पुत्रं च सप्तभिः}
{जघान समरे राजन्सहदेवश्च पञ्चभिः}


\twolineshloka
{तावुभौ भरतश्रेष्ठौ ज्येष्ठौ सर्वधनुष्मताम्}
{विव्याधोरसि सङ्क्रुद्धः पञ्चभिः पञ्चभिः शरैः}


\twolineshloka
{ततोऽपराभ्यां भल्लाब्यां धनुषी समकृन्तत}
{यमयोः प्रसभं वीरो विव्याधाशु त्रिभिस्त्रिभिः}


\twolineshloka
{तावन्ये धनुषी श्रेष्ठे शक्रचापनिभे शुभे}
{प्रगृह्य रेजतुः शूरौ देवपुत्रसमौ युधि}


\twolineshloka
{ततस्तौ रभसौ युद्धे भ्रातरौ भ्रातरं युधि}
{शरैर्ववर्षतुर्घोरैर्महामेघौ यथाऽचलम्}


\twolineshloka
{ततः क्रुद्धो महाराज तव पुत्रो महारथः}
{पाण़्डुपुत्रौ महेष्वासौ वारयामास पत्रिभिः}


\twolineshloka
{धनुर्मण्डलमेवास्य ददृशे युधि भारत}
{सायकाश्चाप्यदृश्यन्त निश्चरन्तः समन्ततः}


\twolineshloka
{`तस्य सायकसञ्छन्नौ माद्रेयौ न विरेजतुः}
{मेघच्छन्नौ यथा व्योम्नि चन्द्रसूर्यौ गतप्रभौ}


\twolineshloka
{ते तु बाणा महाराज स्वर्णपुङ्खाः शिलाशिताः'}
{आच्छादयन्दिशः सर्वाः सूर्यस्येव मरीचयः}


\twolineshloka
{वाणभूते ततस्तस्मिन्सञ्छन्ने च नभस्तले}
{राज्ञस्तु ददृशे रूपं कालान्तकयमोपमम्}


\twolineshloka
{पराक्रमं तु तं दृष्ट्वा तव सूनोर्महारथाः}
{मृत्योरुपान्तिकं प्राप्तौ माद्रीपुत्रौ स्म मेनिरे}


\twolineshloka
{ततः सेनापती राजन्पाण्डवानां महारथः}
{पार्षतः प्रययौ तत्र यत्र राजा सुयोधनः}


\twolineshloka
{माद्रीपुत्रौ ततः शूरौ व्यतिक्रम्य महारथौ}
{धृष्टद्युम्नस्तव सुतं पीडयामास सायकैः}


\twolineshloka
{तमविध्यदमेयात्मा तव पुत्रो ह्यमर्पणः}
{पाञ्चाल्यं पञ्चविंशत्या प्रहस्य पुरुषर्षभः}


\twolineshloka
{ततः पुनरमेयात्मा तव पुत्रो ह्यमर्षणः}
{विद्ध्वा ननाद पाञ्चाल्यं षष्ट्या पञ्चभिरेव च}


\twolineshloka
{तथास्य सशरं चापं मुष्टिदेशे विशाम्पते}
{क्षुरप्रेण सुतीक्ष्णेन राजा चिच्छेद संयुगे}


\twolineshloka
{तदपास्य धनुश्छिन्नं पाञ्चाल्यः शत्रुकर्शनः}
{अन्यदादत्त वेगेन धनुर्भारसहं नवम्}


\twolineshloka
{क्रोधाद्रुधिररक्ताक्षः संरम्भात्प्रज्वलन्निव}
{अशोभत महेष्वासो धृष्टद्युम्नः कृतव्रणः}


\twolineshloka
{स पञ्चदश नाराचाञ्श्वसतः पन्नगानिव}
{जिघांसुर्भरतश्रेष्ठं धृष्टद्युम्नो व्यपासृजत्}


\twolineshloka
{ते वर्म हेमविकृतं भित्त्वा राज्ञः शिलीमुखाः}
{विविशुर्वसुधां वेगात्कङ्कबर्हिणवाससः}


\twolineshloka
{सोऽतिविद्धो महाराज पुत्रस्तेऽतिव्यराजत}
{वसन्तकाले सुमहान्प्रफुल्ल इव किंशुकः}


\twolineshloka
{स च्छिन्नवर्मा नाराचप्रहारैर्झर्झरीकृतः}
{धृष्टद्युम्नस्य भल्लेन क्रुद्धश्चिच्छेद कार्मुकम्}


\twolineshloka
{अथैनं छिन्नधन्वानं त्वरमाणो महीपतिः}
{सायकैर्दशभी राजन्भ्रुवोर्मध्ये समार्पयत्}


\twolineshloka
{तस्य तेऽशोभयन्वक्त्रं कर्मारपरिमार्जिताः}
{प्रफुल्लं पङ्कजं यद्वद्धमरा मधुलिप्सवः}


\twolineshloka
{तदपास्य धनुश्छिन्नं दृष्टद्युम्नो महामनाः}
{अन्यदादत्त वेगेन धनुर्भल्लांश्च षोडश}


\twolineshloka
{ततो दुर्योधनस्याश्वान्हत्वा सूतं च पञ्चभिः}
{धनुश्चिच्छेद भल्लेन जातरूपपरिष्कृतम्}


\twolineshloka
{रथं सोपस्करं छत्रं शक्तिं खङ्गं गदां ध्वजम्}
{भल्लैश्चिच्छेद दशभिः पुत्रस्य तव पार्षतः}


\twolineshloka
{तपनीयाङ्गदं छत्रं नागं मणिमयं शुभम्}
{ध्वजं कुरुपतेश्छिन्नं ददृशुः सर्वपार्थिवाः}


\twolineshloka
{दुर्योधनं तु विरथं छिन्नवर्मायुधध्वजम्}
{भ्रातरं समुदैक्षन्त सोदरा भरतर्षभ}


\twolineshloka
{तमारोप्य रथे राजन्कुण्डधारो महारथम्}
{अपाहरदसम्भ्रान्तो धृष्टद्युम्नस्य पश्यतः}


\twolineshloka
{कर्णस्तु सात्यकिं हित्वा राजगृघ्नुरमर्षणः}
{द्रोमहन्तारमुग्रेषुं ससाराभिमुखो रणे}


\twolineshloka
{तं पृष्ठतोऽन्वयात्तूर्णं शैनेयोऽभिहतः शरैः}
{वारणं जघनोपान्ते विषाणाभ्यामिव द्विपः}


\twolineshloka
{स भारतपुरोगानां राज्ञां च सुमहात्मनाम्}
{कर्णपार्षतयोर्युद्वे सङ्क्रद्धानां महारणे}


\twolineshloka
{न पाण्डवानां नास्माकं कश्चिदासीत्पराङ्मुखः}
{प्रत्यदृश्यत यत्कर्णः पाञ्चालांस्त्वरितो ययौ}


\twolineshloka
{तस्मिन्क्षणे नरश्रेष्ठ गजवाजिरथक्षयः}
{प्रादुरासीदुभयतो राजन्मध्यगतेऽहनि}


\twolineshloka
{पाञ्चालास्तु महाराज त्वरिता विजिगीषवः}
{सर्वतोऽभ्यद्रवन्कर्णं पतत्त्रिण इव द्रुमम्}


\twolineshloka
{तेषामाधिरथिः क्रुद्धः प्रधानान्वै तरस्विनः}
{विचिन्वन्निव बाणौघैः समासादयदग्रतः}


\threelineshloka
{व्याघ्रकेतुः सुशर्मा च शुक्रश्चित्रायुधः क्रतुः}
{दुर्जयो रोचमानश्च सिंहसेनस्तथाऽष्टमः}
{महता रथवंशेन परिवव्रुर्नरोत्तमम्}


\threelineshloka
{सृजन्तः सायकांस्तूर्णं कर्णमाहवशोभिनम्}
{यतमानांस्तु ताञ्शूरान्मनुजेन्द्रः शितैः शरैः}
{अष्टाभिरष्टौ राधेयः पाञ्चालान्न्यहनद्रणे}


\twolineshloka
{अथापरान्महाराज सूतपुत्रः प्रतापवान्}
{जघान बहुसाहस्रान्योधान्युद्वविशारदान्}


\twolineshloka
{विष्णुं च विष्णुवर्माणं देवापिं भद्रमेव च}
{दण्डधारं च समरे चित्रं चिंत्रायुधं हरिम्}


\twolineshloka
{सिंहकेतुं रोचमानं शलभं च महारथम्}
{निजघान सुसङ्क्रुद्वश्चेदीनां च महारथान्}


\twolineshloka
{तेषामाददतः प्राणानासीदाधिरथेर्वपुः}
{शोणिताभ्युक्षिताङ्गस्य रुद्रस्यैवातिभैरवम्}


\twolineshloka
{तत्र भारत कर्णेन मातङ्गास्ताडिताः शरैः}
{सर्वतोऽभ्यद्रवन्भीताः कुर्वन्तो महदाकुलम्}


\twolineshloka
{निपेतुरुर्व्यां समरे कर्णसायकताडिताः}
{कुर्वन्तो विविधान्नादान्वज्रनुन्ना इवाचलाः}


\twolineshloka
{गजवाजिमनुष्यैश्च निपतद्भिः समन्ततः}
{रथैर्भग्नैर्ध्वजैश्चैव समास्तीर्यत मेदिनी}


\twolineshloka
{नैवं भाष्मो न च द्रोणो नान्ये युधि च तावकाः}
{चक्रुः स्म तादृशं कर्म यादृशं कृतवान्रणे}


\twolineshloka
{मृगमध्ये यथा सिंहो दृश्यते निर्भयश्चरन्}
{पाञ्चालानां तथा मध्ये कर्णोऽचरदभीतवत्}


\twolineshloka
{सूतपुत्रोऽथ नागेषु हयेषु च रथेषु च}
{नरेषु च नरव्याघ्रश्चकार कदनं महत्}


\twolineshloka
{यथा मृगगणांस्त्रस्तान्सिंहो द्रावयते दिशः}
{पाञ्चालानां रथव्रातान्कर्णो व्यद्रावयत्तथा}


\twolineshloka
{सिंहं स्म हि यथा प्राप्य न जीवन्ति मृगाः क्वचित्}
{तथा कर्णमनुप्राप्ता न जीवन्ति स्म सृञ्जयाः}


\twolineshloka
{वैश्वानरमुखं प्राप्य दह्यन्ते शलभा यथा}
{कर्णाग्रिं समरे प्राप्य दग्धा भारत सृञ्जयः}


\twolineshloka
{चेदिष्वेकेन कर्णेन पाञ्चालेषु च भारत}
{वेश्राव्य नाम निहता बहवः शूरसम्पताः}


\twolineshloka
{मम चासीन्मती राजन्दृष्ट्वा कर्णस्य विक्रमम्}
{नैकोऽप्याधिरथेर्जीवन्पाञ्चाल्यो मोक्ष्यते युधि}


\twolineshloka
{सङ्ख्ये विमर्द्य पाञ्चालान्सूतपुत्रः पुनः पुनः}
{अभ्यधावत्सुसङ्क्रुद्धः कुन्तीपुत्रं युधिष्ठिरम्}


\twolineshloka
{धृष्टद्युम्नश्च राजानं द्रौपदेयाश्च मारिष}
{परिवव्रुरमित्रघ्नं शतशश्चापरे जनाः}


\twolineshloka
{शिखण्डी सहदेवश्च नकुलो नाकुलिस्तथा}
{जनमेजयः शिनेर्नप्ता बहवश्च प्रभद्रकाः}


\twolineshloka
{एते पुरोगमा भूत्वा धृष्टद्युम्नश्च संयुगे}
{कर्णमस्यन्तमिष्वस्त्रैस्ततक्षुरमितौजसम्}


\twolineshloka
{तांस्तत्राधिरथिः सङ्ख्ये चेदिपाञ्चालपाण्डवान्}
{एको बहूनभ्यपतद्गरुत्मानिव पन्नगान्}


\twolineshloka
{तैः कर्णस्याभवद्युद्धं घोररूपं विशाम्पते}
{तादृग्यादृक्पुरावृत्तं देवानां दानवैः सह}


\twolineshloka
{तान्समेतान्महेष्वासाञ्शरवर्षौधवर्षिणः}
{एको व्यधमदव्यग्रस्तमांसीव दिवाकरः}


\twolineshloka
{भीमसेनस्तु संसक्ते राधेये पाण्डवैः सह}
{सर्वतोऽभ्यहनत्क्रुद्धो यमदण्डनिभैः शरैः}


\twolineshloka
{बाह्लीकान्केकयान्मात्स्यान्वासात्यान्मद्रसैन्धवान्}
{एकः सङ्ख्ये महेष्वासो योधयन्बह्वशोभत}


\twolineshloka
{तत्र मर्मसु भीमेन नाराचैस्ताडिता गजाः}
{प्रपतन्तो हतारोहाः कम्पयन्ति स्म मेदिनीम्}


\twolineshloka
{वाजिनश्च हतारोहाः पत्तयश्च गतासवः}
{अशेरत महाराज वमन्तो रुधिरं बहु}


\twolineshloka
{ताडिताः सहसा नागा भीमसेनेन मारिष}
{निपतन्ति महावेगा वज्ररुग्णा इवाचलाः}


\twolineshloka
{पतितैस्तैर्महाराज वेगवद्भिर्महारथैः}
{शुशुभे वसुधा राजन्विकीर्णैरिव पर्वतैः}


\twolineshloka
{सहस्रशश्च रथिनः पत्तयः पतितायुधाः}
{अक्षताः समदृश्यन्त भीमाद्भीता गतासवः}


\threelineshloka
{रथिभिः सादिभिः सूतैः पादातैर्वाजिभिर्गजैः}
{भीमसेनशरैश्छिन्नैराच्छन्ना सुधाभवत् ॥तत्स्तम्भितमिवातिष्ठद्भीमसेनभयार्दितम्}
{}


\twolineshloka
{दुर्योधनबलं सर्वं निरुत्साहं कृतं रणे}
{निश्चेष्टं तुमुलं दीनं बभौ तस्मिन्महारणे}


\twolineshloka
{प्रसन्नसलिले काले यथा स्यात्सागरो नृप}
{तद्वत्तव बलं तद्वै निश्चलं समवस्थितम्}


\twolineshloka
{मन्युवीर्यबलोपेतं दर्पात्प्रत्यवरोपितम्}
{अभवत्तव पुत्रस्य तत्सैन्यं निष्प्रभं तदा}


\twolineshloka
{ते बले भरतश्रेष्ठ वध्यमाने परस्परम्}
{रुधिरौघपरिक्लिन्ने रुधिरार्द्रे बभूवतुः}


\twolineshloka
{सूतपुत्रोऽवधीत्क्रुद्वः पाण्डवानामनीकिनीम्}
{भीमसेनः कुरूणां च त्रिगर्तानां धनञ्जयः}


\threelineshloka
{वर्तमाने तथा रोद्रे सङ्ग्रामेऽद्भुतदर्शने}
{निहत्य पृतनामध्यं संशप्तकगणान्बहून्}
{अर्जुनो जयतां श्रेष्ठो वासुदेवमथाब्रवीत्}


\twolineshloka
{प्रभग्नं बलमेतद्धि योत्स्यमानं मया सह}
{एते धावन्ति सगणाः संशप्तकमहारथाः}


\twolineshloka
{`दुर्जया ह्येव समरे देवैरपि सवासवैः'}
{अपारयन्तो मद्बाणान्सिंहशब्दं मृगा इव}


\twolineshloka
{दीर्यते च महत्सैन्यं सृञ्जयानां महारणे}
{`कुरवश्चाभिधावन्ति भीमसेनभयार्दिताः'}


\twolineshloka
{हस्तिकक्ष्यो ह्यसौ कृष्ण केतुः कर्णस्य धीमतः}
{दृश्यते राजसैन्यस्य मध्ये विचरतो मुहुः}


\twolineshloka
{न च कर्णं रणे शक्ता जेतुमेते महारथाः}
{जानीते हि भवान्कर्णं वीर्यवन्तं पराक्रमे}


\threelineshloka
{तत्र याहि यतः कर्णो द्रावयत्येष नो बलम्}
{त्रिगर्तान्वर्जयन्याहि सूतपुत्रं महारथम्}
{एतन्मे रोचते कृष्ण यथा वा तव रोचते}


\twolineshloka
{एतच्छ्रुत्वा वचस्तस्य गोविन्दः प्रहसन्निव}
{अब्रवीदर्जुनं तूर्णं कौरवान्याहि पाण्डव}


\twolineshloka
{ततस्तव महासैन्यं गोविन्दप्रेरिता हयाः}
{हंसवर्णा विशंस्तूर्णं वहन्तोऽर्जुनमाहवे}


\twolineshloka
{केशवप्रेरितैरश्वैः श्वेतैः काञ्चनभूषणैः}
{प्रविशद्भिस्तव बलं चतुर्दिशमदीर्यत}


\twolineshloka
{मेघस्तनितनिर्हादः स रथो वानरध्वजः}
{चलत्पताकस्तां सेनां विमानं द्यामिवाविशत्}


\twolineshloka
{तौ विदार्य महासेनां प्रविष्टौ केशवार्जुनो}
{क्रुद्धौ संरम्भरक्ताक्षौ विभ्राजेतां महाद्युती}


\twolineshloka
{युद्धशौण्डौ समाहूतावागतौ तौ रणाध्वरम्}
{यज्वभिर्विधिना हूतौ मखं देवाविवाश्विनौ}


\twolineshloka
{`क्रोधताम्रेक्षणौ शूरौ शुशुभाते महाबलौ}
{मदोत्कटौ यथा नागौ दृष्टिसञ्चारचारिणौ'}


\twolineshloka
{क्रुद्धौ तौ तु नरव्याघ्रो योगवन्तौ बभूवतुः}
{तलशब्देन रुषितौ महानागाविवोत्कटौ}


\twolineshloka
{विगाह्य तु रथानीकमश्वसङ्घांश्च फल्गुनः}
{व्यचरत्पृतनामध्ये पाशहस्त इवानत्कः}


\twolineshloka
{तं दृष्ट्वा युधि विक्रान्तं सेनायां तव भारत}
{संशप्तकगणान्भूयः पुत्रस्ते समचूचुदत्}


\threelineshloka
{ततो रथसहस्रेण द्विरदानां त्रिभिः शतैः}
{चतुर्दशसहस्रैस्तु तुरगाणां महाहवे}
{द्वाभ्यां शतसहस्राभ्यां पदातीनां च धन्विनाम्}


\twolineshloka
{शूराणां लब्धलक्षाणां विदितानां समन्ततः}
{अभ्यवर्तन्त कौन्तेयं छादयन्तो महारथाः}


\twolineshloka
{शरवर्षैर्महाराज सर्वतः पाण्डुनन्दनम्}
{स च्छाद्यमानः समरे शरैः परबलार्दनः}


\twolineshloka
{दर्शयन्रौद्रमात्मानं पाशहस्त इवान्तकः}
{निघ्नन्संशप्तकान्पार्थः प्रेक्षणीयतरोऽभवत्}


\twolineshloka
{ततो विद्युत्प्रभैर्बाणैः कार्तस्वरविभूषितैः}
{निरन्तरमिवाकाशमासीच्छन्नं किरीटिना}


\twolineshloka
{किरीटिचापनिर्मुक्तैः सम्पतद्भिर्महाशरैः}
{समाच्छन्नं बभौ सर्वं काद्रवेयैरिव प्रभो}


\twolineshloka
{रुक्मपुङ्खाञ्शरान्घोरान्प्रसन्नान्नतपर्वणः}
{अवासृजदमेयात्मा दिक्षु सर्वासु पाण्डवः}


\twolineshloka
{मही वियद्दिशः सर्वाः समुद्रा गिरयोऽपि वा}
{स्फुटन्तीति जना जज्ञुः पार्थस्य तलनिःस्वनात्}


\twolineshloka
{हत्वा दशसहस्राणि पार्थिवानां महारथः}
{संशप्तकानां कौन्तेयः प्रपक्षं त्वरितोभ्ययात्}


\twolineshloka
{प्रपक्षं च समासाद्य पार्थः काम्भोजरक्षितम्}
{प्रममाथ बलं भल्लैर्दानवानिव वासवः}


\twolineshloka
{प्रचिच्छेदाशु भल्लेन द्विषतामाततायिनाम्}
{शस्त्रपाणींस्तथा बाहूंस्तथापि च शिरांस्युत}


\twolineshloka
{अङ्गांङ्गावयवैश्छिन्नैर्व्यायुधास्तेऽपतन्भुवि}
{विष्वग्वाताभिसम्भग्ना बहुशाखा इव द्रुमाः}


\twolineshloka
{हस्त्यश्वरथपत्तीनां व्रातान्निघ्नन्तमर्जुनम्}
{सुदक्षिणादवरजः शरवृष्ट्याभ्यवीवृषत्}


\twolineshloka
{तस्यास्यतोऽर्धचन्द्राभ्यां स बाहू परिघोपमौ}
{पूर्णचन्द्राभवक्त्रं च क्षुरेणापाहरच्छिरः}


\twolineshloka
{तत्पपात ततो वेगात्स्वलोहितपरिष्कृतम्}
{मनः शिलागिरेः शृङ्गं व्ज्रेणेवावदारितम्}


\threelineshloka
{सुदक्षिणादवरजं काम्भोजा ददृशुर्हतम्}
{प्रांशुं कमलपत्राक्षमत्यर्थं प्रियदर्शनम्}
{काञ्चनस्तम्भसदृशं भिन्नं हेमगिरिं यथा}


\twolineshloka
{ततोऽभवत्पुनर्युद्धं घोरमत्यर्थमद्भुतम्}
{नानावस्थाश्च योधानां बभूवुस्तत्र युध्यताम्}


\twolineshloka
{पार्थेषुनिहतैरश्वैः काम्भोजैर्यवनैः शकैः}
{शोणिताक्तैस्तदा रक्तं सर्वामासीद्विशां पते}


\twolineshloka
{रथैर्हताश्वसूतैश्च हतारोहैश्च वाजिभिः}
{द्विरदैश्च हतारोहैर्महामात्रैर्हतद्विपैः}


\twolineshloka
{अन्योन्येन महाराज विनाशः पृथिवीक्षिताम्}
{आसीत्क्रुद्वेऽर्जुने कर्णे भीमसेने च दारुणम्}


\twolineshloka
{तस्पिन्प्रपक्षे पक्षे च वध्यमाने मदोत्कटः}
{अर्जुनं जयतां श्रेष्ठं त्वरितो द्रौणिरभ्ययात्}


\twolineshloka
{विधुन्वानो महच्चापं कार्तस्वरविभूषितम्}
{आददानः शरान्घोरान्स्वरश्मीनिव भास्करः}


\twolineshloka
{तयोरासीन्महद्युद्धं धर्मभ्रात्रोरनैष्ठिकम्}
{विस्मापयिषतोर्लोकं यशश्चोत्तममिच्छतोः}


\twolineshloka
{संशप्तकांस्तु कौन्तेयः कुरूंश्चापि वृकोदरः}
{सूतपुत्रस्तु पाञ्चालांस्त्रयोऽघ्नंस्त्वरिताः शरैः}


\twolineshloka
{एवमेष महाराज विनाशः पृथिवीक्षिताम्}
{युद्वं घोरं तथा त्वासीत्त्रिधाभूते चमूमुखे}


\chapter{अध्यायः ५६}
\twolineshloka
{धृतराष्ट्र उवाच}
{}


\twolineshloka
{कथं संशप्तकैः सार्धमर्जुनस्याभवद्रणः}
{सूतपुत्रस्य पाञ्चालैः कथं युद्धं प्रवर्तितम्}


\threelineshloka
{अश्वत्थाम्नस्तु यद्युद्धमर्जुनस्य च सञ्जय}
{अन्येषां च मदीयानां पाण्डवैस्तद्ब्रवीहि मे ॥सञ्जय उवाच}
{}


\twolineshloka
{शृणु राजन्यथावृत्तं सङ्ग्रामं ब्रुवतो मम}
{वीराणां शत्रुभिः सार्धं देहपाप्मविनाशनम्}


\twolineshloka
{पार्थः संशप्तकबलं प्रविश्यार्णवसन्निभम्}
{व्यक्षोभयदमित्रघ्नो महावात इवार्णवम्}


\threelineshloka
{पूर्णचन्द्राभवक्राणि स्वक्षिभ्रूदशनानि च}
{शिरांस्युन्मथ्य वीराणां शितैर्भल्लैर्धनञ्जयः}
{सन्तस्तार क्षितिं क्षिप्रं विनालैर्नलिनैरिव}


\threelineshloka
{सुवृत्तानायतान्पुष्टांश्चन्दनागुरुभूषितान्}
{सायुधान्सतलत्रांश्च पञ्चास्योरगसन्निभान्}
{बाहून्क्षुरैरमित्राणां चिच्छेद समरेऽर्जुनः}


\twolineshloka
{धुर्यान्धुर्योतरान्सूतान्ध्वजांश्चापानि सायकान्}
{पाणीन्नितान्तनिशितैर्भल्लैश्चिच्छेद पाण्डवः}


\twolineshloka
{रथान्द्विपान्हयांश्चैव सारोहानर्जुनो युधि}
{शरैरनेकसाहस्रैर्निन्ये राजन्यमक्षयम्}


\threelineshloka
{तं प्रवीराः सुसंरब्धा नर्दमाना इवर्षभाः}
{वासितार्थमिव क्रुद्धमभिद्रुत्य महोत्कटाः}
{निघ्न्तमभिजघ्नुस्ते शरैः शृङ्गैरिवर्षभाः}


\twolineshloka
{तस्य तेषां च तद्युद्वमभवद्रोमहर्षणम्}
{त्रैलोक्यविजये यद्वद्दैत्यानां सह वज्रिणा}


\twolineshloka
{अस्त्रैस्त्राणि संवार्य द्विषतां सर्वतोऽर्जुनः}
{इषुभिर्बहुभिस्तूर्णं विद्व्वा प्राणाञ्जहार सः}


\twolineshloka
{छिन्नत्रिवेणुचक्राक्षान्हतयोधान्ससारथीन्}
{विध्वस्तायुधतूणीरान्समुन्मथितकेतनान्}


\twolineshloka
{सञ्छिन्नयोक्ररश्मीषान्वित्रिवेणून्विकूबरान्}
{विस्रस्तबन्धुरयुगान्विस्रस्ताक्षप्रमण्डलान्}


% Check verse!
रथान्विशकलीकुर्वन्महाभ्राणीव मारुतः
\twolineshloka
{विस्मापकं प्रेक्षकाणां द्विषतां भयवर्धनम्}
{महारथसहस्रस्य समं कर्माकरोज्जयः}


% Check verse!
सिद्धदेवर्षिसङ्घाश्च चारणाश्चापि तुष्टुवुः
\twolineshloka
{देवदुन्दुभयो नेदुः पुष्पवर्षाणि चापतन्}
{केशवार्जुनयोर्मूर्ध्नि प्राह वाक्वाशरीरिणी}


\twolineshloka
{चन्द्राग्न्यनिलसूर्याणां कान्तिदीप्तिबलद्युतीः}
{यौ सदा बिभ्रतुर्वीराविमौ तौ केशवार्जुनौ}


\twolineshloka
{ब्रह्मेशानाविव पुरा वीरावेकरथे स्थितौ}
{सर्वभूतवरौ वीरौ नरनारायणाविमौ}


\twolineshloka
{इत्येतन्महदाश्चर्यं दृष्ट्वा श्रुत्वा च भारत}
{अश्वत्थामा सुसङ्क्रुद्धः कृष्णावभ्यद्रवद्रणे}


\twolineshloka
{अथ पाण्डवमस्यन्तममित्रान्तकराञ्छरान्}
{सेषुणा पाणिनाऽऽहूय प्रहसन्द्रौणिरब्रवीत्}


\twolineshloka
{यदि मां मन्यसे वीर प्राप्तमर्हमिहातिथिम्}
{ततः सर्वात्मना त्वद्य युद्वातिथ्यं प्रयच्छ मे}


\twolineshloka
{एवमाचार्यपुत्रेण समाहूतो युयुत्सया}
{बहुमेनेऽर्जुनोऽऽत्मानमिति चाह जनार्दनम्}


\twolineshloka
{संशप्तकाश्च मे वध्या द्रौणिराह्वयते च माम्}
{यदत्रानन्तरं प्राप्तं शंस मे तद्वि माधव}


\twolineshloka
{आतिथ्यकर्माभ्युत्थाय दीयतां यदि मन्यसे}
{एवमुक्तोऽवहत्पार्थं कृष्णोद्रोणात्मजान्तिके}


\twolineshloka
{शैक्ष्येण विधिनाऽऽहूतं वायुरिन्द्रमिवाध्वरे}
{तमामन्त्र्यैकमनसं केशवो द्रौणिमब्रवीत्}


\twolineshloka
{अश्वत्थामन्स्थिरो भूत्वा प्रहराशु सहस्व च}
{निर्वेष्टुं भर्तृपिण्डं हि कालोऽयमुपजीविनाम्}


% Check verse!
सूक्ष्मो विवादो विप्राणां सूक्ष्मौ क्षास्त्रौ जयाजयौ
\twolineshloka
{नहि संक्षमसे मोहाद्दिव्यां पार्थस्य सत्क्रियाम्}
{समाप्तिमिच्छन्युध्यस्व स्यिरो भूत्वाऽद्य पाण्डवं}


\twolineshloka
{इत्युक्तो वासुदेवेन तथेत्युक्त्वा द्विजोत्तमः}
{विव्याध केशवं षष्ट्या नाराचैरर्जुनं त्रिभिः}


\twolineshloka
{तस्यार्जुनः सुसङ्क्रुद्धस्त्रिभिर्बाणैः शरासनम्}
{चिच्छेदाथान्यदादत्त द्रौणिर्घोरतरं धनुः}


\twolineshloka
{सज्यं कृत्वा निमेषाच्च विव्याधार्जुनकेशवौ}
{त्रिभिः शतैर्वासुदेवं सहस्रेण च पाण्डवम्}


\twolineshloka
{ततः शरसहस्राणि प्रयुतान्यर्बुदानि च}
{ससृजे द्रौणिरायस्तः स्तम्भयामास चार्जुनम्}


\twolineshloka
{इषुधेर्धनुषो ज्यायास्त्वङ्गुलिभ्यश्च मारिष}
{बाह्वोः कराभ्यामुरसो वदनाद्व्राणनेत्रतः}


\twolineshloka
{कर्णाभ्यां शिरसोऽङ्गेभ्यो लोमवर्मभ्य एव च}
{रथध्वजाभ्यां च शरा निष्पेतुर्ब्रह्मवादिनः}


\twolineshloka
{शरजालेन महता बद्ध्वा माधवपाण्डवौ}
{ननाद सुदितो द्रौणिर्महामेघौघनिःस्वनम्}


\twolineshloka
{`तैः पतद्भिर्महाराज द्रौणिमुक्तैः समन्ततः}
{सञ्छादितौ रथस्थौ तावुभौ कृष्णधनञ्जयौ}


\twolineshloka
{दतः शरxxxxxxक्ष्णैर्भारद्वाजः प्रतापवान्}
{निxxxxxचक्रे रथे माधवपाण्डवौ}


\twolineshloka
{xxxxजङ्गमं स्थावरं तथा}
{चराचरसा मोप्तारौ दृष्ट्वा सञ्छादितौ शरैः}


\twolineshloka
{सिद्धचारxxxxxश्च संपेतुर्शै समन्ततः}
{अपि xxxxxxxx लोकानामिति चाब्रुवन्}


\twolineshloka
{न मया xxxxxxxxxx राजन्दृष्टपृर्वः पराकमः}
{xxxxxxxxxxxx कृष्णो छादयतो रणे}


\twolineshloka
{xxxxxxxxxxxशब्दं रथानां त्रासनं रणे}
{xxxxxxxxxx राजन्सिंहस्य नदतो यथा}


\twolineshloka
{चरतो युद्धे सव्यं दक्षिणमस्यतः}
{विद्युदम्भोधरस्येव भ्राजमाना व्यदृश्यत}


\twolineshloka
{स तदा क्षिप्रकारी च दृढहस्तश्च पाण्डवः}
{प्रमोहं परमं गत्वा प्रेक्षन्नास्ते धनञ्जयः}


\twolineshloka
{विक्रमं चरतो युद्धे सव्यं दक्षिणमस्यतः}
{विक्रमं च हृतं मेने आत्मनस्तेन संयुगे}


\twolineshloka
{अथास्य समरे राजन्वपुरासीत्सुदुर्दृशम्}
{द्रौणेस्तत्कुर्वतः कर्म यादृग्रूपं पिनाकिनः}


\twolineshloka
{वर्धमाने ततस्तत्र द्रोणपुत्रे विशाम्पते}
{हीयमाने च कौन्तेये कृष्णं रोषः समाविशत्}


\twolineshloka
{स रोषान्निश्वसन्राजन्निर्दहन्निव चक्षुषा}
{द्रौणिं ददर्श सङ्ग्रामे फल्गुनं च मुहुर्मुहुः}


% Check verse!
ततः कृष्णोऽब्रवीत्क्रुद्धः पार्थं सप्रणयं वचः
\twolineshloka
{अत्यद्भुततमिदं पार्थ त्वयि पश्यामि संयुगे}
{यत्त्वां विशेषयत्याजौ द्रोणपुत्रोऽद्य भारत}


\twolineshloka
{कच्चित्ते गाण्डिवं हस्ते मुष्टिर्वा न व्यशीर्यत}
{कच्चिद्वीर्यं यथापूर्वं भुजयोर्वा बलं तव}


\fourlineindentedshloka
{उदीर्यमाणं हि रणे पश्यामि द्रौणिमाहवे}
{गुरुपुत्र इति ह्येनं मानयन्पाण्डवर्षभ}
{उपेक्षां मा कृथाः पार्थ नायं काल उपेक्षितुम्' ॥अर्जुन उवाच}
{}


\threelineshloka
{पश्य माधव दौरात्म्यं गुरुपुत्रस्य मां प्रति}
{वधं प्राप्तौ मन्यते नौ प्रावेश्य शवरेश्मनि}
{एषोस्मि हन्मि सङ्कल्पं शिक्षया च बलेन च}


\twolineshloka
{`एवमुक्त्वाऽस्य चिच्छेद भल्लैः कर्मारमार्जितैः}
{धनुश्छत्रं पताकां च रथशक्तिं गदां वराम्'}


\twolineshloka
{अश्वत्थाम्नः शरानस्ताञ्छित्त्वैकैकं त्रिधा त्रिधा}
{व्यधमद्भरतश्रेष्ठो नीहारमिव मारुतः}


\twolineshloka
{ततः संशप्तकान्भूयः साश्वसूतरथद्विपान्}
{ध्वजपत्तिगणानुर्ग्रैर्बाणैर्विव्याध पाण्डवः}


\twolineshloka
{ये ये ददृशिरे तत्र यद्यद्रूपास्तदा जनाः}
{ते ते तत्र शरैर्व्याप्तं मेनिरेऽऽत्मानमात्मना}


\twolineshloka
{ते गाण्डीवप्रमुक्तास्तु नानारूपाः पतत्रिणः}
{क्रोशे साग्रे स्थिताञ्जघ्नुर्द्विपांश्च पुरुषान्रणे}


\twolineshloka
{भल्लैश्छिन्नाः कराः पेतुः करिणां मदवर्षिणाम्}
{यथा वने परशुभिर्निकृत्ताः शाल्मलिद्रुमाः}


\twolineshloka
{पश्चात्तु शैलवत्पतुस्ते गजाः सह सादिभिः}
{वज्रिवज्रावमथिता यथैवाद्रिचयास्तथा}


\twolineshloka
{गन्धर्वनगराकारान्रथांश्चैव सुकल्पितान्}
{विनीतैर्जवनैर्युक्तानास्थितान्युद्वदुर्मदैः}


\twolineshloka
{शरैर्विशकलीकुर्वन्नमित्रानभ्यवीवृषत्}
{स्वलङ्कृतानश्वसादीन्पत्तींश्चाहन्धनञ्जयः}


\twolineshloka
{धनञ्जययुगान्तार्कः संशप्तकमहार्णवम्}
{व्यशोषयत दुःशोषं तीक्ष्णैः शरगभस्तिभिः}


\twolineshloka
{पुनर्द्रौणिमहाशैलं नाराचैर्वज्रसन्निभैः}
{निर्बिभेद महावेगैस्त्वरन्वज्रीव पर्वतम्}


\twolineshloka
{तमाचार्यसुतः क्रुद्धः साश्वयन्तारमाशुगैः}
{युयुत्सुरागमद्योद्धुं पार्थस्तानच्छिनच्छरान्}


\twolineshloka
{ततः परमसङ्क्रुद्धः काण्डकोशमवासृजत्}
{अश्वत्थामाभिरूपाय गृहानतिथये यथा}


\twolineshloka
{अथ संशप्तकांस्त्यक्त्वा पाण्डवो द्रौणिमभ्ययात्}
{अपाङ्क्तेयानिव त्यक्त्वा दाता पाङ्क्तेयमर्थिनम्}


% Check verse!
`स्थिताः संशप्तका राजन्दृष्ट्वा युद्वं महात्मनोः'
\chapter{अध्यायः ५७}
\twolineshloka
{सञ्जय उवाच}
{}


\twolineshloka
{ततः समभवद्युद्धं शुक्राङ्गिरसवर्चसोः}
{नक्षत्रमभितो व्योम्नि शुक्राङ्गिरसयोरिव}


\twolineshloka
{सन्तापयन्तावन्योन्यं दीप्तैः शरगभस्तिभिः}
{लोकत्रासकरावास्तां विमार्गस्थौ ग्रहाविव}


\twolineshloka
{ततोऽविध्यद्धुवोर्मध्ये नाराचैरर्जुनो भृशम्}
{स तेन विबभौ द्रौणिरूर्ध्वरश्मिर्यथा रविः}


\twolineshloka
{अथ कृष्णौ शरशतैरश्वत्थाम्नाऽर्दितौ भृशम्}
{स्वरश्मिजालविकचौ युगान्तार्काविवासुतः}


\twolineshloka
{ततोऽर्जुनः सर्वतोधारमस्त्र--मवासृजद्वासुदेवेऽभिभूते}
{द्रौणायनिश्चाभ्यहनत्पृषत्कै--र्वज्राग्निवैवस्वतदण्डकल्पैः}


\twolineshloka
{स केशवं चार्जुनं चातितेजाविव्याध मर्मस्वतिरौद्रकर्मा}
{बाणैः सुमुक्तैरतितीव्रवेगै--र्यैराहतो मृत्युरपि व्यथेत}


\twolineshloka
{द्रौणेरिषूनिषुभिः सन्निवार्यव्यायच्छतस्तद्द्विगुणैः सुपुङ्खैः}
{तं साश्वसूतध्वजमेकवीर--मावृत्य संशप्तकसैन्यमार्च्छत्}


\twolineshloka
{धनूंषि बाणानिषुधीर्धनुर्ज्याःपाणीन्भुजान्पाणिगतं च शस्त्रम्}
{छत्राणि केतूंस्तुरगान्रथेषांवस्त्राणि माल्यान्यथा भूषणानि}


\twolineshloka
{चर्माणि वर्माणि मनोरमाणिप्रसह्य चैषां स शिरांसि चैव}
{स्रिच्छेद पार्थो द्विषतां सुमुक्तै--र्बाणैः स्थितानामपराङ्मुखानाम्}


\twolineshloka
{सुकल्पिताः स्मन्दनवाजिनागाःसमास्थिताः कृतयत्नैर्नृवीरैः}
{पार्थेरितैर्बाणशतैर्निरस्ता--स्तैरेव सार्धं नृवरैर्विनेशुः}


\twolineshloka
{पद्मार्कपूर्णेन्दुनिभाननानिकिरीटमाल्याभरणोज्ज्वलानि}
{भल्लार्धचन्द्रक्षुरसन्निकृत्ता--न्यपातयच्छत्रुशिरांस्यजस्रम्}


\twolineshloka
{अथ द्विपैर्दैत्यरिपुद्विपाभै--र्देवारिकल्पा बलमन्युकल्पैः}
{कलिङ्गवङ्गाङ्गनिषादवीराजिघांसवः पाण्डवमभ्यधावन्}


\twolineshloka
{तेषां द्विपानां निचकर्त पार्थोवर्माणि चर्माणि करान्नियन्तॄन्}
{ध्वजाः पताकाश्च ततः प्रपेतु--र्वज्राहतानीव गिरेः शिरांसि}


\twolineshloka
{तेषु प्रभग्नेषु गुरोस्तनूजंबाणैः किरीटि नवसूर्यवर्णैः}
{प्रच्छादयामास महाभ्रजालै--र्वायुः समुद्यन्तमिवांशुमन्तम्}


\twolineshloka
{ततोऽर्जुनेषूनिषुभिर्निरस्यद्रौणिः शितैरर्जुनवासुदेवौ}
{प्रच्छादयित्वा दिवि चन्द्रसूर्यौननाद सोऽम्भोद इवातपान्ते}


\twolineshloka
{तमर्जुनस्तांश्च पुनस्त्वदीया--नभ्यर्दितस्तैरभिसृत्य शस्त्रैः}
{वाणान्धकारं सहसैव कृत्त्वाविव्याध पार्थो वरहेमपुङ्खैः}


\twolineshloka
{नाप्याददत्सन्दधन्नैव मुञ्च--न्बाणान्रथेऽदृश्यत सव्यसाची}
{रथांश्च नागांस्तुरगान्पदातीन्संस्यूतदेहान्ददृशुर्हतांश्च}


\twolineshloka
{सन्धाय नाराचवरान्दशाशुद्रौणिस्त्वरन्नेकमिवोत्ससर्ज}
{तेषां च पञ्चार्जुनमभ्यविध्य--न्पञ्चाच्युतं निर्बिभुदुः सुपुङ्खाः}


\twolineshloka
{तैराहतौ सर्वमनुष्यमुख्या--वसृक्स्रवन्तौ धनदेन्द्रकल्पौ}
{समाप्तविद्येन तथाऽभिभूतौहतौ रणे ताविति मेनिरेऽन्ये}


\twolineshloka
{अथार्जुनं प्राह दशार्हनाथःप्रमाद्यसे किं जहि योधमेतम्}
{कुर्याद्धि दोषं समुपेक्षितोऽयंकष्टो भवेद्व्याधिरिवाक्रियावान्}


\twolineshloka
{तथेति चोक्त्वाऽच्युतमप्रमादीद्रौणेः प्रहस्याशु किरीटमाली}
{भुजौ वरौ चन्दनसारदिग्धौवक्षः शिरोऽथाप्रतिमौ तथोरू}


\twolineshloka
{गाण्डीवमुक्तैः कुपिताहिकल्पै--द्रौणिं शरैः संयति निर्बिभेद}
{छित्त्वा तु रश्मींस्तुरगानविध्य--त्ते तं रणादूहुरतीव दूरम्}


\threelineshloka
{स तैर्हृतो वातजवैस्तुरङ्गै-र्द्रौणिर्दृढं पार्थशराभिभूतः}
{आवृत्य नैव व्यषहत्स योद्धुंपार्थेन सार्धं मतिमान्विमृश्य}
{}


% Check verse!
जानञ्जयं नियतं वृष्णिवीरेधनञ्जये चाङ्गिरसां वरिष्ठः ॥विवेश कर्णस्य बलं तरस्वीभग्नोत्साहः क्षीणवाणास्त्रयोगः
\twolineshloka
{प्रतीपकामे तु रणादश्वत्थाम्नि हृते हयैः}
{मन्त्रौषधिक्रियायोगैर्व्याधौ देहादिवाहृते}


\twolineshloka
{संशप्तकानभिमुखौ प्रयातौ केशवार्जुनौ}
{वातोद्वूतपताकेन स्यन्दनेनौघनादिना}


\chapter{अध्यायः ५८}
\twolineshloka
{सञ्जय उवाच}
{}


\twolineshloka
{अथोत्तरेण पाण्डूनां सेनायां ध्वनिरुत्थितः}
{रथनागाश्वपत्तीनां दण्डधारेण वध्यताम्}


\twolineshloka
{निवर्तयित्वा तु रथं केशवोऽर्जुनमब्रवीत्}
{वाहयन्नेव तुरगान्गरुडानिलरंहसः}


\twolineshloka
{मागधोऽसावतिक्रान्तो द्विरदेन प्रमाथिना}
{भगदत्तादनवमः शिक्षया च बलेन च}


\twolineshloka
{एनं हत्वा निहन्ताऽसि पुनः संशप्तकानिति}
{वाक्यान्ते प्रापयत्पार्थं दण्डधारगजं प्रति}


\twolineshloka
{स मागधानां प्रवरो महाबलोऽ--शुभग्रहो योधगणैः समन्वितः}
{सपत्नसेनां प्रममाथ दारुणोभीमं समग्रां बलवानिव ग्रहम्}


\twolineshloka
{सुकल्पितं दानवनागसन्निभंमहाभ्रनिर्हादसमस्वनं रणे}
{समास्थितो नागवरं नरेश्वरोरथाश्वमातङ्गनरप्रमाथिनम्}


\twolineshloka
{स नागयन्तॄन्समरे महारथा--न्सपत्तिसङ्घांस्तुरगान्ससादिनः}
{द्विपांश्च बाणैर्निजघान वीर्यवा--न्समन्ततो घ्नन्निव कालचक्रवत्}


\twolineshloka
{नरांस्तु कांस्यायसवर्मभूषणा--न्निपात्य साश्वानपि पत्तिभिः सह}
{व्यपोथयद्दन्तिवरेण शुष्मिणासशब्दवत्स्थूलनलं यथा तथा}


\twolineshloka
{अथार्जुनो ज्यातलनेमिनिःस्वनेमृदङ्गभेरीबहुशङ्खनादिते}
{रथाश्वमातङ्गसहस्रनादितेरथोत्तमेनाभ्यपतद्द्विपोत्तमम्}


\twolineshloka
{ततोऽर्जुनं द्वादशभिः शरोत्तमै--र्जनार्दनं षोडशभिः समार्पयत्}
{स दण़्डधारस्तुरगांस्त्रिभिस्त्रिभि---स्ततो ननाद प्रजहास चासकृत्}


\twolineshloka
{ततोऽस्य पार्थः सगुणेषुकार्मुकंचकर्त भल्लैर्ध्वजमप्यलङ्कृतम्}
{पुनर्नियन्तॄन्सहपादगोप्तृभि--स्ततः स चुक्रोध गिरिव्रजेश्वरः}


\twolineshloka
{ततोऽर्जुनं भिन्नकटेन दन्तिनाघनाघनेनानिलतुल्यरंहसा}
{अतीव चुक्रोधयिषुर्जनार्दनंघनञ्जयं चाभिजघान तोमरैः}


\twolineshloka
{अथास्य बाहू द्विपहस्तसन्निभौशिरश्च पूर्णेन्दुनिभाननं त्रिभिः}
{क्षुरैः प्रचिच्छेद सहैव पाण्डव--स्ततो द्विपं बाणशतैः समर्पयत्}


\twolineshloka
{स पार्थबाणैस्तपनीयभूषणैःसमावृतः काञ्चनवर्मभृद्द्विपः}
{भृशं चकाशे निशि पर्वतो यथादावाग्निना प्रज्वलितौषधिद्रुमः}


\twolineshloka
{स वेदनार्तोऽम्बुदनिस्वनो नदं--श्चरन्भ्रमन्प्रस्स्वलितान्तरोऽद्रवत्}
{पपात रुग्णः सनियन्तृकस्तथा ॥यथा गिरिर्वज्रविदारितस्तथा}


\twolineshloka
{हिमावदानेन सुवर्णमालिनाहिमाद्रिकूटप्रतिमेन दन्तिना}
{हते रणे भ्रातरि दण्ड आव्रज--ज्जिघांसुरिन्द्रावरजं धनञ्जयम्}


\twolineshloka
{सतोमरैरर्करप्रभैस्त्रिभि--र्जनार्दनं पञ्चभिरर्जुनं शितैः}
{समर्पयित्वा विननाद चार्दयं--स्ततोऽस्य बाहू निचकर्त पाण्डवः}


\twolineshloka
{क्षुरप्रकृत्तौ विपुलौ सतोमरौशुभाङ्गदौ चन्दनरूषितौ भुजौ}
{गजात्पतन्तौ युगपद्विरेजतु--र्यथोरगौ पर्वतशृङ्गवन्महीम्}


\twolineshloka
{अथाऽर्धचन्द्रेण हतं किरीटिनापपात दण्डस्य शिरः क्षितिं द्विपात्}
{स शोणितार्द्रो निपतन्विरेजेदिवाकरोऽस्तादिव पश्चिमां दिशम्}


\twolineshloka
{अथ द्विपं श्वेतनगाग्रसन्निभंदिवाकरांशुप्रतिमैः शरोत्तमैः}
{बिभेद पार्थः स पपात नादयन्हिमाद्रिकूटं कुलिशाहतं यथा}


\twolineshloka
{ततोऽपरं तत्प्रतिमा गजोत्तमाजिगीषवः संयति सव्यसाचिना}
{तथा कृतास्तेऽपि यथैव तौ द्विपौततः प्रभग्नं सुमहद्रिपोर्बलम्}


\twolineshloka
{गजा रथाश्वाः पुरुषाश्च सङ्घशःपरस्परघ्नाः परिपेतुराहवे}
{परस्परं प्रस्खलिताः समाहताभृशं च तत्तद्बहुभाषिणो हताः}


\twolineshloka
{अथार्जुनं स्वे परिवार्य सैनिकाःपुरन्दरं देवगणा इवाब्रुवन्}
{अभैष्म यस्मान्मरणादिव प्रजाःस वीर दिष्ट्या निहतस्त्वया रिपुः}


\twolineshloka
{न चेत्परित्रास्यदिमाञ्जनान्भया--द्द्विषद्भिरेवं बलिभिः प्रपीडितान्}
{तथाऽभविष्यद्द्विषतां प्रमोदनंयथा हतेष्वेष्विह नोऽरिसूदन}


\twolineshloka
{इतीव भूयश्च सुहृद्भिरीडितानिशम्य वाचः सुमनास्तदाऽर्जुनः}
{यथाऽनुरूपं प्रतिपूज्य तं जनंजगाम संशप्तकसङ्घहा पुनः}


\chapter{अध्यायः ५९}
\twolineshloka
{सञ्जय उवाच}
{}


\twolineshloka
{प्रत्येत्याथ पुनर्जिष्णुर्जित्वा संशप्तकान्बभौ}
{वक्रानुवक्रगमनादङ्गारक इव ग्रहः}


\twolineshloka
{पार्थबाणहता राजन्नराश्वरथकुञ्जराः}
{विचेलुर्बभ्रमुर्नेशुः पेतुर्मम्लुश्च भारत}


\twolineshloka
{धुर्यान्धुर्येतरान्सूतान्ध्वजांश्चपासिसायकान्}
{पाणीन्पाणिगताञ्शस्त्रान्बाहूनपि शिरांसि च}


\twolineshloka
{भल्लैः क्षुरैरर्धचन्द्रैर्वत्सदन्तैश्च पाण्डवः}
{चिच्छेदामित्रवीराणां समरे प्रतियुध्यताम्}


\threelineshloka
{तं प्रवीरास्त्वदीयानां नर्दमानाऽभिदुद्रुवुः}
{वासितार्थे युयुत्सन्तो वृषभा वृषभं यथा}
{निपतन्त्यर्जुनं शूराः शतशोऽथ सहस्रशः}


\twolineshloka
{तेषां तस्य च तद्युद्धमभवद्रोमहर्षणम्}
{त्रैलोक्यविजये यादृग्दैत्यानां सह वज्रिणा}


\twolineshloka
{तमविध्यत्त्रिभिर्बाणैर्दन्दशूकैरिवाहिभिः}
{उग्रायुधस्य तस्याशु शिरः कायादपाहरत्}


\twolineshloka
{तेऽर्जुनं सर्वतः क्रुद्धा नानाशस्त्रैरवीवृषन्}
{मरुद्भिः प्रेरिता मेघा हिमवन्तमिवोष्णगे}


\twolineshloka
{अस्त्रैरस्त्राणिं संवार्य द्विषतां सर्वतोऽर्जुनः}
{सम्यगस्तैः शरैः सर्वानहितानहनद्बहून्}


\twolineshloka
{छिन्नत्रिवेणुजङ्घेषान्हताश्वान्हतसारथीन्}
{विस्रस्तहस्ततूणीरान्विचक्रथकेतनान्}


\twolineshloka
{सञ्छिन्नरश्मियोक्त्राक्षान्व्यनुकर्षयुगान्रथान्}
{विध्वस्तसर्वसन्नाहान्बाणैश्चक्रेऽर्जुनस्त्वरन्}


\twolineshloka
{ते रथास्तत्र विध्वस्ताः परार्द्व्या भान्त्यनेकशः}
{धनिनामिव वेश्मानि हतान्यग्न्यनिलाम्बुभिः}


\twolineshloka
{द्विपाः सम्भिन्नमर्माणो वज्राशनिसमैः शरैः}
{पेतुर्गिर्यग्रवेश्मानि वज्रवाताग्निभिर्यथा}


\twolineshloka
{अनेकैश्च शिलाधौतैर्वज्रानलविषोपमैः}
{शरैर्निजघ्निवान्पार्थो महेन्द्र इव दानवान्}


\twolineshloka
{सारोहास्तुरगाः पेतुर्बहवोऽर्जुनताडिताः}
{निर्जिह्वान्त्राः क्षितौ क्षीणा रुधिरार्द्राः सुदुर्दृशः}


\twolineshloka
{नराश्वनागा नाराचैः संस्यूताः सव्यसाचिना}
{बभ्रमुश्चस्खलुः पेतुर्नेदुर्मम्लुश्च मारिष}


\twolineshloka
{महार्हवर्माभरणा नानारूपाम्बरायुधाः}
{सरथाश्वद्विपा वीरा हताः पार्थेन शेरते}


\twolineshloka
{निर्भयाः पुण्यकर्माणो विशिष्टाभिजनश्रुताः}
{गताः शरीरैर्वसुधामूर्जितैः कर्मभिर्दिवम्}


\twolineshloka
{अथार्जुनरथं वीरास्त्वदीयाः समभिद्रवन्}
{नानाजनपदाध्यक्षाः सगणा जातमन्यवः}


\twolineshloka
{उह्यमना रथाश्वेभैः पत्तयश्च जिघांसवः}
{समभ्यधावन्नस्यन्तो विविधं क्षिप्रमायुधम्}


\twolineshloka
{तदायुधमहावर्षं क्षिप्तं योधमहाम्बुदैः}
{व्यधमन्निशितैर्बाणैः क्षिप्रमर्जुनमारुतः}


% Check verse!
साश्वपत्तिद्विपरथं महाशस्त्रौघसम्प्लवम् ॥सहसा सन्तितीर्षन्तं पार्थं शस्त्रास्त्रसेतुना
\twolineshloka
{अथाब्रवीद्वासुदेवः पार्थ किं क्रीडसेऽनघ}
{संशप्तकान्प्रमथ्यैनांस्ततः कर्णवधे त्वर}


\twolineshloka
{तथेत्युक्त्वाऽर्जुनः कृष्णं शिष्टान्संशप्तकांस्तदा}
{क्षपयिष्यंस्तदा बाणैर्दैत्यानिन्द्र इवावधीत्}


\twolineshloka
{आददत्सन्दधन्नेषृन्दृष्टः कैश्चिद्रणेऽर्जुनः}
{विमुञ्चन्वा शराञ्शीघ्रं दृश्यन्ते वै नरा हताः}


\twolineshloka
{आश्चर्यमिति गोविन्दो ब्रुवन्नश्वानचोदयत्}
{हंसांशुगौरास्ते सेनां हंसाः सर इवाविशन्}


\twolineshloka
{ततः सङ्ग्रामभूमिं तां वर्तमाने जनक्षये}
{अवेक्षमाणो गोविन्दः सव्यसाचिनमब्रवीत्}


\twolineshloka
{एष पार्थ महारौद्रो वर्तते भरतक्षयः}
{पृथिव्यां पार्थिवानां वै दुर्योधनकृते महान्}


\twolineshloka
{पश्य भारत चापानि रुक्मपृष्ठानि धन्विनाम्}
{हतानामपविद्धानि कलापानिषुधींस्तथा}


\twolineshloka
{जातरूपमयैः पुङ्खैः शरांश्च नतपर्वणः}
{तैलधौतांश्च नाराचान्निर्मुक्तानिव पन्नगान्}


\twolineshloka
{आकीर्णांस्तोमरान्वाहांश्छत्रान्हेमविभूषितान्}
{चर्माणि चापविद्धानि रुक्मपृष्ठानि भारत}


\twolineshloka
{सुवर्णविकृतान्प्रासाञ्शक्तीः कनकभूषिताः}
{जाम्बूनदमयैः पट्टैर्बद्धाश्च विपुला गदाः}


\twolineshloka
{जातरूपमयीश्चर्ष्टीः पट्टसान्हेमभूषितान्}
{दण्डैः कनकचित्रैश्च विप्रविद्धान्परश्वथान्}


\twolineshloka
{परिघान्भिण्डिपालांश्च भुशुण्डीः कणपानपि}
{अयस्कुन्तांश्च पतितान्मुसलानि गुरूणि च}


\twolineshloka
{नानाविधानि शस्त्राणि प्रगृह्य जयगृद्धिनः}
{जीवन्त इव दृश्यन्ते गतसत्वास्तरस्विनः}


\twolineshloka
{गदाविमथितैर्गात्रैर्मुसलैर्भिन्नमस्तकान्}
{गजवाजिरथैः क्षुण्णान्पश्य योधान्सहस्रशः}


\twolineshloka
{मनुष्यगजवाजीनां शरशक्त्यृष्टितोमरैः}
{निस्त्रिंशैः पट्टसैः प्रासैर्नखरैर्लगुडैरपि}


\twolineshloka
{शरीरैर्बहुघा छिन्नैः शोणितौघपरिप्लुतैः}
{गतासुभिरमित्रघ्न संवृता रणभूमयः}


\twolineshloka
{बाहुभिश्चन्दनादिग्धैः साङ्गदैः शुभभूषणैः}
{सतलत्रैः सकेयूरैर्भाति भारत मेदिनी}


\twolineshloka
{साङ्गुलित्रैर्भुजाग्रैश्च विप्रविद्वैरलङ्कृतैः}
{हस्तिहस्तोपमैश्छिन्नैरूरुभिश्च तरस्विनाम्}


\twolineshloka
{बद्धचूडामणिवरैः शिरोभिश्च सकुण्डलैः}
{`निकृत्तैर्वृषभाक्षाणां शरीरैश्चापि सङ्घशः}


\threelineshloka
{गजवाजिमनुष्याणां शरशक्त्यृष्टितोमरैः}
{कबन्धैः शोणितादिग्धैश्छिन्नगात्रशिरोधरैः}
{भूर्भाति भरतश्रेष्ठ शान्ताग्निभिरिवाध्वरे'}


\twolineshloka
{रथांश्च बहुधा भग्नान्हेमकिङ्किणिनः शुभान्}
{अश्वांश्च बुहधा पश्य शोणितेन परिप्लुतान्}


\threelineshloka
{अनुकर्षानुपासङ्गान्पताका विविधान्ध्वजान्}
{योधानां च महाशङ्खान्पाण्डुरांश्च प्रकीर्णकान्}
{निरस्तजिह्वान्मातङ्गाञ्शयानान्पर्वतोपमान्}


\twolineshloka
{वैजयन्तीर्विचित्राश्च हतांश्च गजयोधिनः}
{वारणानां परिस्तोमान्संयुक्तानेककम्बलान्}


\twolineshloka
{विपाटितविचित्राश्च रूपचित्राः कुथास्तथा}
{भिन्नाश्च बहुधा घण्टाः पतद्भिश्चूर्णिता गजैः}


\twolineshloka
{वैदूर्यमणिदण्डांश्च पतितांश्चाङ्कुशान्भुवि}
{अश्वानां च युगापीडान्रत्नचित्रानुरश्छदान्}


\twolineshloka
{विद्धाः सादिध्वजाग्रेषु सुवर्णविकृताः कुथाः}
{विचित्रान्मणिचित्रांश्च जातरूपपरिष्कृतान्}


\threelineshloka
{अश्वास्तरपरिस्तोमान्राङ्कवान्पतितान्भुवि}
{चूडामणीन्नरेन्द्राणां विचित्राः काञ्चनस्रजः}
{छत्राणि चापविद्धानि चामरव्यजनानि च}


\threelineshloka
{चन्द्रनक्षत्रभासैश्च वदनैश्चारुकुण्डलैः}
{क्लृप्तश्मश्रुभिरत्यर्थं वीराणां समलङ्कृतैः}
{वदनैः पश्य सञ्छन्नां महीं शोणितकर्दमाम्}


\threelineshloka
{सजीवांश्च नरान्पश्य कूजमानान्समन्ततः}
{उपास्यमानांश्च बहून्न्यस्तशस्त्रैश्च भूपते}
{ज्ञातिभिश्च जलक्लिन्नान्रोदमानैर्मुहुर्मुहुः}


\twolineshloka
{उत्क्रान्तानपरान्योधाञ्छादयित्वा तरस्विनः}
{पुनर्युद्धाय गच्छन्ति जये लुब्धाः प्रमन्यवः}


\twolineshloka
{अपरे तत्रतत्रैव परिधावन्ति मानवाः}
{सन्निवृत्ताश्च ते शूरा दृष्टा चैतान्विचेतसः}


\twolineshloka
{जलं त्यक्त्वा प्रधावन्ति क्रोशमानाः परस्परम्}
{जलं पीत्वा मृतान्पश्य पिबतोऽन्यांश्च भारत}


\twolineshloka
{परिष्वज्य प्रियानन्ये बान्धवान्बन्धुवत्सलाः}
{विसंज्ञान्समरे योधांस्तत्रतत्र महारणे}


\twolineshloka
{पश्यापरान्नरश्रेष्ठ सन्दष्टोष्ठपुटान्पुनः}
{भृकुटीकुटिलैर्वक्त्रैः प्रेक्षमाणान्महारणम्}


\threelineshloka
{एतत्तवैवानुरूपं कर्मार्जुन महाहवे}
{दिवि वा देवराजस्य त्वया यत्कृतमाहवे ॥सञ्जय उवाच}
{}


\twolineshloka
{एवं तां दर्शयन्कृष्णो युद्धभूमिं किरीटिने}
{गच्छन्नेवाशृणोच्छब्दं दुर्योधनबले महत्}


\twolineshloka
{शङ्खदुन्दुभिनिर्घोषं भेरीपणवनिःस्वनम्}
{रथाश्वनरनागानां पादशब्दांश्च दारुणान्}


\twolineshloka
{प्रविश्य तद्बलं कृष्णस्तुरगैर्वातवेगितैः}
{पाण्ड्येनाभ्यर्दितं सैन्यं त्वदीयं वीक्ष्य विस्मितः}


\twolineshloka
{स हि नानाविधैर्बाणैरिष्वस्त्रप्रवरो युधि}
{न्यहनद्द्विषतां पूगान्गतासूनन्तको यथा}


\twolineshloka
{गजवाजिमनुष्याणां शरीराणि शितैः शरैः}
{भित्त्वा प्रहरतां श्रेष्ठो विदेहासूनपातयत्}


\twolineshloka
{शत्रुप्रवीरैरस्त्राणि नानाशस्त्राणि सायकैः}
{छित्त्वा तानवधीच्छत्रून्पाण्ड्यः शक्र इवासुरान्}


\chapter{अध्यायः ६०}
\twolineshloka
{धृतराष्ट्र उवाच}
{}


\twolineshloka
{प्रोक्तस्त्वया पूर्वमेव प्रवीरो लोकविश्रुतः}
{न त्वस्य कर्म सङ्ग्रामे त्वया सञ्जय कीर्तितम्}


\threelineshloka
{तस्य विस्तरशो ब्रूहि प्रवीरस्याद्य विक्रमम्}
{शिक्षां प्रभावं वीर्यं च प्रमाणं दर्पमेव च ॥सञ्जय उवाच}
{}


\twolineshloka
{भीष्मद्रोणकृपद्रौणिकर्णार्जुनजनार्दनान्}
{समाप्तविद्यान्धनुषि श्रेष्ठान्यान्सप्त मन्यसे}


\twolineshloka
{यो ह्याक्षिपति वीर्येण सर्वानेतान्महारथान्}
{न मेने चात्मना तुल्यं कञ्चिदेव नरेश्वरम्}


\twolineshloka
{तुल्यतांद्रोणभीष्माभ्यामात्मनो यो न मृष्यते}
{वासुदेवार्जुनाभ्यां च न्यूनतां नैच्छतात्मनि}


\twolineshloka
{स पाण्ड्योऽर्थपतिश्रेष्ठः सर्वशस्त्रभृतां वरः}
{कर्णस्यानीकमहनत्पाशहस्त इवान्तकः}


\twolineshloka
{तदुदीर्णरथाश्वेभं पत्तिप्रवरसङ्कुलम्}
{कुलालचक्रवद्धान्तं पाण्ड्येनाभ्याहतं बलात्}


\twolineshloka
{व्यश्वसूतध्वजरथान्विप्रयुक्तयुगान्रथान्}
{सम्यगस्तैः शरैः पाण्ड्यो वायुर्मेघानिवाक्षिपत्}


\twolineshloka
{द्विरदान्प्रवरारोहान्विपताकायुधध्वजान्}
{स पादरक्षानहनद्वज्रेणाद्रीनिवाद्रिहा}


\twolineshloka
{स शक्तिप्रासतूणीरानश्वारोहान्हयानपि}
{पुलिन्दखसबाह्लीकनिषादान्ध्रककुन्तलान्}


\twolineshloka
{दाक्षिणात्यांश्च भोजांश्च शूरान्सङ्ग्रामकर्कशान्}
{विशस्त्रकवचान्बाणैः कृत्वा पाण्ड्योऽकरोद्व्यसून्}


\twolineshloka
{चतुरङ्गं बलं बाणैर्निघ्नन्तं पाण्डयमाहवे}
{दृष्ट्वा द्रौणिरसम्भ्रान्तमसम्भ्रान्तस्ततोऽभ्ययात्}


\twolineshloka
{आभाष्य चैनं मधुरमभीतं तमभीतवत्}
{प्राह प्रहरतां श्रेष्ठः स्मितपूर्वं समाह्वयत्}


\twolineshloka
{राजन्कमलपत्राक्ष प्रधानायुधवाहन}
{वज्रसंहननप्रख्य प्रख्यातबलपौरुष}


\twolineshloka
{मुष्टिक्लिष्टाङ्गुलिभ्यां च व्यायताभ्यां महद्धनुः}
{दोर्भ्यां विस्फारयन्भासि महाजलदवद्भृशम्}


\twolineshloka
{शरवर्षैर्महावेगैरमित्रानभिवर्षतः}
{मदन्यं नानुपश्यामि प्रतिवीरं तवाहवे}


\twolineshloka
{रथद्विरदपत्त्यश्वानेकः प्रमथसे बहून्}
{मृगसङ्घानिवारण्ये विभीर्भीमबलो हरिः}


\twolineshloka
{महता रथघोषेण दिवं भूमिं च नादयन्}
{वर्षान्ते सस्यां सूर्यो भाभिरादीपयन्निव}


\twolineshloka
{संस्पृशानः शरैः पूर्णौ तूणी चाशीविषोपमैः}
{मयैवैकेन युध्यस्व त्र्यम्बकेनान्धको यथा}


\twolineshloka
{एवमुक्तस्तथेत्युक्त्वा प्रमथ्यैनं स पार्थिवः}
{कर्णिना द्रोणतनयं विव्याध मलयध्वजः}


\twolineshloka
{मर्मभेदिभिरत्युग्रैर्बाणैरग्निशिखोपमैः}
{मर्मस्वभ्यहनद्द्रौणिः पाण्ड्यमाचार्यनन्दनः}


\twolineshloka
{ततोऽपरान्नवांस्तूर्णं नाराचान्कङ्कवाससः}
{गत्या दशम्या संयुक्तानश्वत्थामाऽप्यवासृजत्}


\twolineshloka
{तानच्छिनत्तदा पाण्ड्यश्चतुर्भिरपरैः शरैः}
{चतुरोऽभ्याहनच्चाश्वानाशु ते व्यसवोऽभवन्}


\twolineshloka
{अथ द्रोणसुतस्येषूंस्ताञ्छित्त्वा निशितैः शरैः}
{धनुर्ज्यां विततां पाण्ड्यश्चिच्छेदादित्यतेजसः}


\twolineshloka
{दिव्यं धनुरथाधिज्यं कृत्वा द्रौणिरमित्रहा}
{प्रेक्ष्य चाशु रथे युक्तान्नरैरन्यान्हयोत्तमान्}


\twolineshloka
{ततः शरसहस्राणि प्रेषयामास वै द्विजः}
{इषुसम्बाधमाकाशमकरोद्दिश एव च}


\twolineshloka
{ततस्तानस्यतः सर्वान्द्रौणेर्बाणान्महात्मनः}
{जानानोप्यक्षयान्पाण्ड्यो शातयत्पुरुषर्षभः}


\twolineshloka
{प्रयुक्तांस्तान्प्रयत्नेन छित्त्वा द्रौणेरिषूनरिः}
{चक्ररक्षौ रणे तस्य प्राणुदन्निशितैः शरैः}


\twolineshloka
{अथ तल्लाघवं दृष्ट्वा मण्डलीकृतकार्मुकः}
{प्रास्यद्द्रोणसुतो बाणान्वृष्टिं पूषानुजो यथा}


\twolineshloka
{अष्टावष्टगवान्यूहुः शकटानि यदायुधम्}
{अह्नस्तदष्टभागेन द्रौणिश्चिक्षेप मारिष}


\twolineshloka
{तमन्तकमिव क्रुद्धमन्तकालान्तकोपमम्}
{ये ये ददृशिरे रूपं विसंज्ञाः प्रायशोऽभवन्}


\twolineshloka
{आचार्यपुत्रस्तां सेनां बाणवृष्ट्या व्यवीवृषत्}
{पर्जन्य इव घर्मान्ते वृष्ट्या साद्रिद्रुमां महिम्}


\twolineshloka
{द्रौणिपर्जन्यमुक्तां तां बाणवृष्टिं सुदुःसहाम्}
{वायव्यास्त्रेण स क्षिप्रं विद्ध्वा पाण्ड्यातिलोऽनदत्}


\twolineshloka
{तस्य नानदतः केतुं चन्दनागुरुरूषितम्}
{मलयप्रतिमं द्रौणिश्छित्त्वाश्वांश्चतुरोऽहनत्}


\twolineshloka
{सूतमेकेषुणा हत्वा महाजलदनिःस्वनम्}
{धनुश्छित्त्वाऽर्धचन्द्रेण तिलशो व्यधमद्रथम्}


\twolineshloka
{अस्त्रैरस्त्राणि संवार्य छित्त्वा सर्वायुधानि च}
{प्राप्तमप्यहितं द्रौणिर्नाहनद्युद्धतृष्णया}


\twolineshloka
{हतेश्वरो दन्तिवरः सुकल्पित--सत्वराभिसृष्टः प्रतिशब्दगो बली}
{तमाद्रवद्द्रौणिशराहतस्त्वरन्जवेन कृत्वा प्रतिहस्तिगर्जितम्}


\twolineshloka
{तं वारणं वारणयुद्धकोविदोद्विपोत्तमं पर्वतसानुसन्निभम्}
{सम्भ्यतिष्ठन्मलयध्वजस्त्वर--न्यथाऽद्रिशृङ्गं हरिरुन्नदंस्तथा}


\twolineshloka
{स तोमरं भास्कररश्मिवर्चसंबलास्त्रसर्गोत्तमयत्नमन्युभिः}
{ससर्ज शीघ्रं परिपीडयन्गजंगुरोः सुताय द्रविडेश्वरो नदन्}


\twolineshloka
{मणिप्रवेकोत्तमवज्रहाटकै--रलङ्कृतं चांशुकमाल्यमौक्तिकैः}
{हतो मयासीत्यसकृन्मुदा नद--न्पराभिनद्द्रौणिवराङ्गभूषणम्}


\twolineshloka
{तदर्कचन्द्रग्रहपावकत्विषंभृशाभिघातात्पतितं विघूर्णितम्}
{महेन्द्रवज्राभिहतं महास्वनंयथाऽद्रिशृङ्गं भरणीतले तथा}


\twolineshloka
{ततः प्रजज्वाल परेण मन्युना'पादाहतो नागपतिर्यथा तथा}
{समाददे चान्तकदण्डसन्निभा--निषूनमित्रान्तकरांश्चतुर्दश}


\twolineshloka
{द्विपस्य पादाग्रकरान्स पञ्चभि--र्नृपस्य बाहू च शिरोऽथ च त्रिभिः}
{जघान पड्भिः पडृतूपमत्विषःस पाण्ड्यराजानुचरान्महारथान्}


\twolineshloka
{सुदीर्घवृत्तौ वरचन्दनोक्षितौसुवर्णमुक्तामणिवज्रभूषणौ}
{भुजौ धरायां पतितौ नृपस्य तौविचेष्टतुस्तार्क्ष्यहताविवोरगौ}


\twolineshloka
{शिरश्च तत्पूर्णशशिप्रभाननंसरोषताम्रायतनेत्रमुन्नसम्}
{क्षितावपि भ्राजति तत्सकुण्डलंविशाखयोर्मध्यगतः शशी यथा}


\twolineshloka
{[स तु द्विपः पञ्चभिरुत्तमेषुभिःकृतः षडंशश्चतुरो नृपस्त्रिभिः}
{कृतो दशांशः कुशलेन युध्यतायथा हविस्तद्दश दैवतं तथा}


\twolineshloka
{स पादशो राक्षसभोजनान्बहून्प्रदाय पाण्ड्योऽश्वमनुष्यकुञ्जरान्}
{स्वधामिवाप्य ज्वलनः पितृप्रिय--स्ततः प्रशान्तः सलिलप्रवाहतः]}


\twolineshloka
{समाप्तविद्यं तु गुरोः सुतं नृपःसमाप्तकर्माणमुपेत्य ते सुतः}
{सुहृद्वृतोऽत्यर्थमपूजयन्मुदाजिते बलौ विष्णुमिवामरेश्वरः}


\chapter{अध्यायः ६१}
\twolineshloka
{दृतराष्ट्र उवाच}
{}


\twolineshloka
{पाण्ड्ये हते किमकरोदर्जुनो जयतांवरः}
{एकवीरेण कर्णेन द्राविते च युधिष्ठिरे}


\twolineshloka
{समाप्तविद्यो बलवान्युक्तो वीरश्च पाण्डवः}
{सर्वभूतेष्वनुज्ञातः शङ्करेण महात्मना}


\threelineshloka
{तस्मान्महद्भयं तीव्रममित्रघ्नाद्धनञ्जयात्}
{स यत्तत्राकरोत्पार्थस्तन्ममाचक्ष्व सञ्जय ॥सञ्जय उवाच}
{}


\twolineshloka
{हते पाण्ड्येऽर्जुनं कृष्णस्त्वरन्नाह वचो हितम्}
{हतं पश्य स्वराजानमपयातांश्च पाण्डवान्}


\twolineshloka
{कर्णं पश्य महारङ्गे ज्वलन्तमिव पावकम्}
{असौ भीमो महेष्वासः प्रतिवृत्तो रणं प्रति}


\twolineshloka
{त इमे नानुवर्तन्ते धृष्टद्युम्नपुरोगमाः}
{पाञ्चालाः सृञ्जयाश्चैव पाण्डवानां चमूमुखम्}


\twolineshloka
{`निवृत्तैश्च पुनः पार्थैर्भग्नं शत्रुबलं महत्'}
{कौरवान्द्रवतो ह्येष कर्णो वारयते भृशम्}


\twolineshloka
{अन्तकप्रतिमो वेगे शक्रतुल्यपराक्रमः}
{असौ गच्छति कौन्तेय द्रौणिः शस्त्रभृतां वरः}


\twolineshloka
{तमेष प्रद्रुतः सङ्ख्ये धृष्टद्युम्नो महारथः}
{अश्वत्थाम्ना हताः सङ्ख्ये सर्वे कौन्तेय सृञ्जयाः}


\twolineshloka
{रथाश्वनरनागानां कृतं च कदनं महत्}
{इत्येतत्सर्वमाचष्ट वासुदेवः किरीटिने}


\twolineshloka
{एतच्छ्रुत्वा च दृष्ट्वा च भ्रातृणां व्यसनं महत्}
{वाहयाश्वान्हृषीकेश क्षिप्रमित्याह पाण्डवः}


\twolineshloka
{ततः प्रायाद्धृषीकेशो रथेनाप्रतिमो युधि}
{`ततो रेणुः समभवत्पुनस्तत्र महारणे}


\threelineshloka
{ततो राजन्महानासीत्सङ्गामो रोमहर्षणः}
{सिंहनादरवाश्चात्र प्रादुरासन्समागमे}
{उभयोः सेनयो राजन्मृत्युं कृत्वा निवर्तनम्}


\twolineshloka
{ततः हुनः समाजग्मुरभीताः कुरुसृञ्जयाः}
{युधिषरमुखाः पार्था वैकर्तनमुखा वयम्}


\twolineshloka
{ततः प्रववृते यूद्धं घोररूपं विशाम्पते}
{कर्णस्य पाण्डवानां च यमराष्ट्रविवर्धनम्}


\twolineshloka
{तस्मिन्प्रवृत्ते सङ्ग्रामे तुमुले रोमहर्षणे}
{संशप्तकेषु वीरेषु किञ्चिच्छेषेषु भारत}


\twolineshloka
{धृष्टद्युम्नो महाराज सहितः सर्वराजभिः}
{कर्णमेवाभिदुद्राव पाण्डवाश्च महारथाः}


\twolineshloka
{आगच्छमानांस्तान्दृष्ट्वा सङ्ग्रमे विजयैषिणः}
{दधारैको रणे कार्णो जलौघानिव पर्वतः}


\twolineshloka
{समासाद्य तु ते कर्णं व्यशीर्यन्त महारथाः}
{यथाऽचलं समासाद्य वार्योधाः सर्वतोदिशम्}


\twolineshloka
{तयोरासीन्महाराज सङ्ग्रामो घोरदर्शनः}
{प्रवृद्धयोर्महारङ्गे बलिनोर्विजिगीषतोः}


\twolineshloka
{धृष्टद्युम्नस्तु राधेयं शरेणानतपर्वणा}
{छादयामास सङ्क्रुद्धस्तिष्ठतिष्ठेति चाब्रवीत्}


\twolineshloka
{विजयं तु धनुः श्रेष्ठं विधृन्वानो महारथः}
{पार्षतस्य धनुश्छित्त्वा शरांश्चाशीविषोपमान्}


\twolineshloka
{ववर्ष शरवर्षाणि तोयवर्षानिवाम्बुदः}
{छादयामास सङ्क्रुद्धः पार्षतं नवभिः शरैः}


\twolineshloka
{ते हेमविकृतं वर्म भित्त्वा तस्य महात्मनः}
{शोणिताक्ता व्यराजन्त शक्रगोपा इवानघ}


\threelineshloka
{तदपास्य धनुश्छिन्नं धृष्टद्युम्नः प्रतापवान्}
{ततोऽन्यद्धनुरादाय सारवद्भारसाधनम्}
{कर्णं विव्याध सप्तत्या शरैः सन्नतपर्वभिः}


\twolineshloka
{तथैव राजन्कर्णोऽपि पार्षतं शत्रुतापनम्}
{द्रोणशत्रुं महेष्वासो विव्याध निशितैः शरैः}


\twolineshloka
{पुनरन्यं महाराज शरं कनकभूषणम्}
{प्रेषयामास समरे मृत्युदण्डमिवापरम्}


\twolineshloka
{तमापतन्तं सहसा घोररूपं विशाम्पते}
{चिच्छेद शतधा राजन्पार्षतः कृतहस्तवत्}


% Check verse!
दृष्टवान्पतितं भूमौ शरं कर्णो विशाम्पते
\twolineshloka
{पार्षतः शरवर्षेण समन्तात्पर्यवारयत्}
{विव्याध चैनं त्वरितो नाराचैः सप्तभिस्तदा}


\twolineshloka
{तं प्रत्यविध्यद्दशभिः शरैर्हेमविभूषितैः}
{तयोर्युद्धं समभवद्दृष्टिश्रोत्रमनोहरम्}


% Check verse!
आसीद्धोरं च चित्रं च प्रेक्षणीयं च सर्वशः
\twolineshloka
{सर्वेषां तत्र भूतानां रोमहर्षो व्यजायत}
{दृष्ट्वा तत्समरे कर्म कर्णपार्षतयोर्नृप}


\twolineshloka
{एतस्मिन्नन्तरे द्रौणिरभ्ययात्तं महारथम्}
{पार्षतं शत्रुदमनं शत्रुवीर्यासुनाशनम्}


\twolineshloka
{स दृष्ट्वा समरे यान्तमभीतं च महारथम्}
{अभ्यभाषत सङ्क्रुद्धः पार्षतं शत्रुतापनम्}


% Check verse!
रथं रथेन सम्पीड्य पार्षतस्य तु ब्राह्मणः
\threelineshloka
{तिष्ठतिष्ठेति ब्रह्मघ्न न मे जीवन्विमोक्ष्यसे}
{इत्युक्त्वा सुभृशं वीरः शीघ्रकृन्निशितैः शरैः}
{छादयामास समरे यतमानो महारथः}


\twolineshloka
{यत्नतः परया शक्त्या धृष्टद्युम्नं महारथम्}
{योधयामास समरे क्रुद्धरूपो विशाम्पते}


% Check verse!
तयोस्तु सन्निपाते हि घोररूपो विशाम्पते
\twolineshloka
{यथा हि समरे द्रौणिः पार्षतं दृश्यत मारिष}
{नातिहृष्टमना ह्यासीन्मन्वानो मृत्युमात्मनः}


\threelineshloka
{स ज्ञात्वा पितुरत्यन्तरैरिणं तु महाहवे}
{आत्मानं समरे ज्ञात्वाऽशस्त्रवध्यं महाबलः}
{जवेनाभिययौ द्रौणिं कालः कालक्षये यथा}


\twolineshloka
{द्रौणिस्तु दृष्ट्वा राजेन्द्र धृष्टद्युम्नमवस्थितम्}
{क्रोधेन निःश्वसन्वीरः पार्षदं समभिद्रवत्}


\twolineshloka
{तावन्योन्यं तु दृष्ट्वैव संरम्भं जग्मतुः परम्}
{प्रगृह्य महती चापे शरासनसमन्विते}


\twolineshloka
{अथाब्रवीन्महाराज द्रोणपुत्रः प्रतापवान्}
{धृष्टद्युम्नं समीपस्थं त्वरमाणो विशाम्पते}


\twolineshloka
{पाञ्चालापशदाद्य त्वां प्रेषयामि यमक्षयम्}
{पापं हि यत्त्वया कर्म कृतं तातं घ्नता रणे}


\threelineshloka
{अद्य त्वं प्राप्स्यसे तद्वै यथा ह्यकुशलस्तथा}
{अरक्ष्यमाणः पार्थेन यदि तिष्ठसि संयुगे}
{नापक्रामसि वा मोहात्सत्यमेतद्ब्रवीमि ते}


\chapter{अध्यायः ६२}
\twolineshloka
{सञ्जय उवाच}
{}


\twolineshloka
{एवमुक्तः प्रत्युवाच धृष्टद्युम्नः प्रतापवान्}
{आशिषं तां प्रवक्तव्यां मामको दास्यते तव}


\threelineshloka
{येनैव ते पितुर्दत्ता यतमानस्य संयुगे}
{एष ते प्रतिवाक्यं वै असिर्दास्यति मामकः}
{येन कृत्तं तव पितुर्यतमानस्य तच्छिरः}


\twolineshloka
{यदि तावन्मया द्रोणो निहतो ब्राह्मणब्रुवः}
{त्वामिदानीं कथं युद्धे न हनिष्यामि किल्बिषम्}


\twolineshloka
{एवमुक्त्वा महातेजाः सेनापतिररिन्दमः}
{सुतीक्ष्णेनाथ भल्लेन द्रौणिं विव्याध पार्षतः}


\twolineshloka
{ततो द्रौणिः सुसङ्क्रुद्धः शरैः सन्नतपर्वभिः}
{आच्छादयद्दिशो राजन्धृष्टद्युम्नस्य संयुगे}


\threelineshloka
{नैवान्तरिक्षं न दिशो नापि योधाः सहस्रशः}
{दृश्यन्ते वै महाराज शरैश्छन्नाः समन्ततः}
{एकः सन्वारयामास प्रेक्षणीयः समन्ततः}


\twolineshloka
{तथैव पार्षतो राजन्द्रौणिमाहवशोभिनम्}
{शरैः सञ्छादयामास सूतपुत्रस्य पश्यतः}


\twolineshloka
{राधेयोऽपि महाराज पाञ्चालान्सह पाण्डवैः}
{द्रौपदेया युधामन्युमुत्तमौजसमेव च}


\twolineshloka
{सात्यकिश्च महाराज योधांश्चान्यान्सहस्रशः}
{एकस्तान्वारयामास प्रेक्षणीयः समन्ततः}


\twolineshloka
{धृष्टद्युम्नस्तु समरे द्रौणेश्चिच्छेद कार्मुकम्}
{क्षुरप्रेण सुतीक्ष्णेन पश्यतां सर्वयोधिनाम्}


\twolineshloka
{तदपास्य धनुश्छिन्नमन्यदादत्त कार्मुकम्}
{वेगवत्समरे घोरं शरांश्चाशीविषोपमान्}


\twolineshloka
{स पार्षतस्य राजेन्द्र धनुः शक्तिं गदां ध्वजम्}
{हयान्सूतं रथं चैव निमेषाद्व्यधमच्छरैः}


\twolineshloka
{स च्छिन्नधन्वा विरथो हताश्वो हतसारथिः}
{खङ्गमादत्त विपुलं शतचन्द्रं च भानुमत्}


\twolineshloka
{द्रौणिस्तदपि राजेन्द्र भल्लैः क्षिप्रं महारथः}
{चिच्छेद समरे वीरः क्षिप्रहस्तो दृढायुधः}


\twolineshloka
{रथादनवरूढस्य धन्विनो बाहुशालिनः}
{पश्यतां सर्वसैन्यानां तदद्भुतमिवाभवत्}


\threelineshloka
{धृष्टद्युम्नं तु विरथं हताश्वरथसारथिम्}
{शस्त्रैश्च बहुधा विद्धमस्त्रैश्च शकलीकृतम्}
{नातरद्भरतश्रेष्ठ यदमानो महारथः}


\threelineshloka
{तस्यान्तमिषुभी राजन्यदा द्रौणिर्न गच्छति}
{अथ त्यक्त्वा रथं वीरः पार्षतं त्वरितोऽन्वगात्}
{प्रगृह्य विपुलं खह्गं जातरूपपरिष्कृतम्}


\twolineshloka
{तस्य सम्पततो राजन्वपुरासीन्महात्मनः}
{गरुडस्येव ततो जिघृक्षोः पन्नगोत्तमम्}


\twolineshloka
{एतस्मिन्नेव काले तु केशवः परवीरहा}
{अब्रवीद्भरतश्रेष्ठमर्जुनं जयतां वरम्}


\twolineshloka
{पश्य द्रौणिं पार्षतस्य यतमानं वधं प्रति}
{यत्नं करोति विपुलं हन्याच्चैनं न संशयः}


\twolineshloka
{त्रायस्वैनं महाबाहो पार्षतं युद्धदुर्मदम्}
{द्रौणेरास्यगतं वीर मृत्योरास्यगतं यथा}


\twolineshloka
{एवमुक्त्वा महाराज वासुदेवः प्रतापवान्}
{प्रैषयत्तुरगान्यत्र द्रौणिर्वीरो व्यवस्थितः}


\twolineshloka
{ते हयाश्चन्द्रसङ्काशाः केशवेन प्रचोदिताः}
{पिबन्त इव चाकाशं जग्मुद्रौणेर्महारथम्}


\twolineshloka
{दृष्ट्वा द्रौणिर्महाराज वासुदेवधनंजयौ}
{धृष्टद्युम्नवधे राजंश्चक्रे यत्नं महाबलः}


\twolineshloka
{विकृष्यमाणं दृष्ट्वै धृष्टद्युम्नं जनेश्वर}
{शरांश्चिक्षेप वै पार्थो द्रौणिं प्रति महारथः}


\twolineshloka
{ते शरा हेमविकृता गाण्डीवप्रेषिता भृशम्}
{द्रौणिमासाद्य विविशुर्वल्मीकमिव पन्नगाः}


\twolineshloka
{विद्धस्तु तैः शरैर्घोरैद्रोणपुत्रः प्रतापवान्}
{उत्सृज्य समरे राजन्पाञ्चालममितौजसम्}


\twolineshloka
{आरुरोह रथं वीरो धनञ्जयशरार्दितः}
{प्रगृह्य च धनुः श्रेष्ठं पार्थं विव्याध सायकैः}


\twolineshloka
{एतस्मिन्नन्तरे शूरः सहदेवो जनाधिप}
{अपोवाह रथेनाजौ पार्षतं शत्रुतापनम्}


\twolineshloka
{अर्जुनोऽपि महाराज द्रौणिं विव्याध पत्रिभिः}
{तं द्रोणपुत्रः संरब्धो बाह्वोरुरसि चार्पयत्}


\twolineshloka
{क्रोधितस्तु रणे पार्थो नाराचं कालसन्निभम्}
{द्रोणपुत्राय चिक्षेप यमदण्डमिवापरम्}


\threelineshloka
{स ब्राह्मणस्यांसदेशे निपपात महाद्युतेः}
{स विह्वलो महाराज शरवेगेन संयुगे}
{निषसाद रथोपस्थे वैक्लव्यं परमं ययौ}


\twolineshloka
{तं तु मत्वा हतं वीरं सारथिः शत्रुकर्शनम्}
{अपोवाह रणेनाजौ त्वरमाणो रणाजिरात्}


\twolineshloka
{अथोद्धृष्टं महाराज पाञ्चालैर्जितकाशिभिः}
{मोक्षितं पार्षतं दृष्ट्वा द्रोणपुत्रं च पीडितम्}


\twolineshloka
{वादित्राणि च दिव्यानि प्रावाद्यन्त सहस्रशः}
{सिंहनादश्च सञ्जज्ञे दृष्ट्वा युद्धं तदद्भुतम्}


\threelineshloka
{ततः कर्णो महाराज व्याक्षिपद्विजयं धनुः}
{दृष्ट्वार्जुनं रणे क्रुद्धः प्रेक्षते च मुहुर्मुहुः}
{द्वैरथं चापि पार्थेन गन्तुकामो महाद्युतिः}


\twolineshloka
{एवं कृत्वाऽब्रवीत्पार्थो वासुदेवं धनञ्जयः}
{याहि संशप्तकान्कृष्ण कार्यमेतत्परं मम}


\twolineshloka
{ततः प्रयातो गोविन्दः श्रुत्वा पाण्डवभाषितम्}
{रथेनातिपताकेन मनोमारुतरंहसा}


\twolineshloka
{एवमेष क्षयो वृत्तः पृथिव्यां पृथिवीपते}
{तावकानां परेषां च राजन्दुर्मन्त्रिते तव}


\chapter{अध्यायः ६३}
\twolineshloka
{सञ्जय उवाच}
{}


\twolineshloka
{ततः प्रववृते भूयः सङ्ग्रामो राजसत्तम}
{कर्णस्य पाण्डवानां च यमराष्ट्रविवर्धनः}


\twolineshloka
{धनूंषि बाणान्परिघानसिपट्टसतोमरान्}
{मुसलानि भुशुण्डीश्च शक्तिरिष्टिपरश्वथान्}


\twolineshloka
{गदाः प्रासाञ्शितान्कुन्तान्भिंडिपालान्महाङ्कुशान्}
{प्रगृह्य क्षिप्रमापेतुः परस्परजिघांसया}


\twolineshloka
{बाणज्यातलशब्देन द्यां दिशः प्रदिशो वियत्}
{पृथिवीं नेमिघोषेण नादयन्तोऽभ्ययुः परान्}


\twolineshloka
{तेन शब्देन महता संहृष्टाश्चक्रुराहवम्}
{वीरा वीरैर्महाप्तोरकलहान्तं तितीर्षवः}


\twolineshloka
{ज्यातलत्रधनुःशब्दाञ्शस्त्राणां च निपात्यताम्}
{ताडितानां च पतता निनादांश्च पदातिनाम्}


\twolineshloka
{बाणशब्दांश्च विविधाञ्शूराणां चाभिगर्जताम्}
{श्रुत्वा गजा भृशं त्रेसुः पेतुर्मम्लुश्च सैनिकाः}


\twolineshloka
{तेषां निनदतां चैव शस्त्रवर्षं च मुञ्चताम्}
{बहूनाधिरथिर्वीरः प्रममाथेषुभिः परान्}


\twolineshloka
{पञ्च पाञ्चालवीराणां रथान्दश च पञ्च च}
{साश्वसूतध्वजान्कर्णः शरैर्निन्ये यमक्षयम्}


\twolineshloka
{योधमुख्या महावीर्याः पाण्डूनां कर्णमाहवे}
{शिक्षितास्तमभिद्रुत्य परिवव्रुः समन्ततः}


\twolineshloka
{ततः कर्णो द्विषत्सेनां शरवर्षैर्विलोडयन्}
{विजगाहाण्डजाकीर्णां पद्मिनीमिव यूथपः}


\twolineshloka
{द्विषत्सैन्यमवस्कन्द्य राधेयो धनुरुत्तमम्}
{विधुन्वानः शितैर्बाणैः शिरांस्युन्मथ्य पातयत्}


\twolineshloka
{`हस्तिनः समहामात्रान्साश्वारोहान्हयानपि}
{रथिनोऽप्येकबाणेन भ्रमतश्चावपातयत्'}


\twolineshloka
{वर्म वा चर्म वाऽश्रित्य निपातमपि देहिनः}
{विषेहुर्नास्य संस्पर्शं द्वितीयस्य पतत्त्रिणः}


\twolineshloka
{वर्मयुक्तेषु देहेषु धनुषोऽस्यञ्शिताञ्शरान्}
{मौर्व्या तलत्रे न्यहनत्कशया वाजिनो यथा}


\twolineshloka
{पाण्डुसृञ्जयपाञ्चालाञ्शरगोचरमागतान्}
{ममर्द तरसा कर्णः सिंहो मृगगणानिव}


\twolineshloka
{ततः पाञ्चालपुत्रश्च द्रौपदेयाश्च मारिष}
{यमौ च युयुधानश्च त्वरिताः कर्णमभ्ययुः}


\twolineshloka
{तेषु व्यायच्छमानेषु कुरुपाञ्चालपाण्डुषु}
{प्रियानसून्रणे त्यक्त्वा योधा जघ्नुः परस्परम्}


\twolineshloka
{सुसन्नद्धाः कवचिनः सशिरस्त्राणभूषणाः}
{गदाभिर्मुसलैश्चान्ये परिघैश्च महाबालाः}


\twolineshloka
{समभ्यधावन्त भृशं कालदण्डैरिवोद्यतैः}
{नर्दन्तश्चाह्वयन्तश्च प्रवल्गन्तश्च मारिष}


\twolineshloka
{ततो निजघ्नुरन्योन्यं पेतुश्चान्योन्यताडिताः}
{वमन्तो रुधिरं गात्रैर्विमस्तिष्केक्षमायुधाः}


\twolineshloka
{दन्तपूर्णैः सरुधिरैर्वक्रैर्दाडिमसन्निभैः}
{जीवन्त इव चाप्येके तस्थुः शस्त्रौघबृंहिताः}


\twolineshloka
{परश्वथैश्चाप्यपरे पट्टसैरसिभिस्तथा}
{शक्तिभिर्भिण्डिपालैश्च नखरप्रासतोमरैः}


\twolineshloka
{ततक्षुश्चिच्छिदुश्चान्ये बिभिदुश्चिक्षिपुस्तथा}
{सञ्चकर्तुश्च जघ्नुश्च राजन्योधा महारणे}


\twolineshloka
{पेतुरन्योन्यनिहता व्यसवो रुधिरोक्षिताः}
{क्षरन्तः स्वरसं रक्तं प्रकृत्ताश्चन्दना इव}


\twolineshloka
{रथै रथा विनिहता हस्तिभिश्चापि हस्तिनः}
{नरैर्नरा हताः पेतुरश्वाश्चाश्वैः सहस्रशः}


\twolineshloka
{ध्वजाः शिरांसि च्छिन्नाः पेतुर्महीतले ॥`वध्यतां दारुणाः शब्दाः पततां स्तनतामपि}
{}


% Check verse!
क्षुरैर्भल्लार्धचन्द्रैश्च च्छिन्नाः पेतुर्महीतले ॥नराश्वेभरथानां हिनराश्वेभरथैस्तथा'
% Check verse!
नरांश्च नागांश्च रथान्हयान्ममृदुराहवे
\twolineshloka
{अश्वारोहैर्हताः शूराश्छिन्नहस्ताश्च दन्तिनः}
{सपताकाध्वजाः पेतुर्विशीर्णा इव पर्वताः}


\twolineshloka
{पत्तिभिश्च समाप्लुत्य द्विरदाः स्यन्दनास्तथा}
{हताश्च हन्यमानाश्च पतिताश्चैव सर्वशः}


\twolineshloka
{अश्वारोहाः समासाद्य त्वरिताः पत्तिभिर्हताः}
{सादिभिः पत्तिसङ्घाश्च निहता युधि शेरते}


\twolineshloka
{मृदितानीव पद्मानि प्रम्लाना इव च स्रजः}
{हतानां वदनान्यासन्गात्राणि च महाहवे}


\twolineshloka
{रूपाण्यत्यर्थकान्तानि द्विरदाश्वनृणां नृप}
{समुन्नानीव वस्त्राणि ययुर्दुर्दुर्शतां पराम्}


\chapter{अध्यायः ६४}
\twolineshloka
{सञ्जय उवाच}
{}


\twolineshloka
{हस्तिभिस्तु महामात्रास्तव पुत्रेण चोदिताः}
{धृष्टद्युम्नं घ्नतेत्याशु क्रुद्धाः पार्षतमभ्ययुः}


\twolineshloka
{प्राच्याश्च दाक्षिणात्याश्च प्रवरा गजयोधिनः}
{अङ्गा वङ्गाश्च पुण्ड्राश्च मागधास्ताम्रलिप्तकाः}


\twolineshloka
{मेकलाः कोशला मद्रा दशार्णा निषधास्तथा}
{गजयुद्धेषु कुशलाः कलिङ्गैः सह भारत}


\twolineshloka
{शरतोमरनाराचैर्वृष्टिमन्त इवाम्बुदाः}
{सिषिचुस्ते ततः सर्वे पाञ्चालबलमाहवे}


\twolineshloka
{तानभिद्रवतो नागान्मदवेगसमुद्धतान्}
{विपाठक्षुरनाराचैर्धृष्टद्युम्नो ह्यवीवृषत्}


\twolineshloka
{एकैकं दशभिः ष़ड्भिरष्टाभिरपि पार्षतः}
{द्विरदानभिविव्याध क्षिप्रं गिरिनिभाञ्शरैः}


\twolineshloka
{प्रच्छाद्यमानं द्विरदैर्मेघैरिव दिवाकरम्}
{पर्ययुः पाण्डुपाञ्चाला नदन्तो ह्यात्तकार्मुकाः}


\threelineshloka
{तान्नागानभिवर्षन्तो ज्यातलत्रविनादिनः}
{वीरनृत्यं प्रनृत्यन्तः शूरतालप्रचोदितैः}
{नकुलः सहदेवश्च द्रौपदेयाः प्रभद्रकाः}


\twolineshloka
{सात्यकिश्च शिखण्डी च चेकितानश्च वीर्यवान्}
{समन्तात्सिषिचुर्वीरा मेघास्तोयैरिवाचलान्}


\twolineshloka
{ते म्लेच्छैः प्रेषिता नागा नरानश्वान्रथानपि}
{हस्तैराक्षिप्य ममृदुः पद्भिश्चाप्यतिमन्यवः}


\twolineshloka
{बिभिदुश्च विषाणाग्रैः समाक्षिप्य च चिक्षिपुः}
{विषाणलग्नाश्चाप्यन्ये परिपेतुर्विभीषणाः}


\twolineshloka
{प्रमुखे वर्तमानं तु द्विपमङ्गस्य सात्यकिः}
{नाराचेनोग्रवेगेन भित्त्वा मर्माण्यताडयत्}


\twolineshloka
{तस्यावर्जितकायस्य द्विरदादुत्पतिष्यतः}
{नाराचेनाहनद्वक्षः सात्यकिः सोऽपतद्भुवि}


\twolineshloka
{पुण्ड्रस्यापततो नागं चलन्तमिव पर्वतम्}
{सहदेवः प्रसन्नाग्रैर्नाराचैरहनत्त्रिभिः}


\twolineshloka
{विपताकं वियन्तारं विवर्मध्वजजीवितम्}
{तं कृत्वा द्विरदं भूयः सहदेवोऽङ्गमभ्ययात्}


\twolineshloka
{सहदेवं तु नकुलो वारयित्वाङ्गमार्दयत्}
{नाराचैर्यमदण्डाभैस्त्रिभिर्नागं शतेन तम्}


\twolineshloka
{दिवाकरकप्रख्यानङ्गश्चिक्षेप तोमरान्}
{नकुलाय शतान्यष्टौ त्रिधैकैकं तु सोऽच्छिनत्}


\twolineshloka
{तथाऽर्धचन्द्रेण शिरस्तस्य चिच्छेद पाण्डवः}
{स पपात हतो म्लेच्छस्तेनैव सह दन्तिना}


\twolineshloka
{अथाङ्गपुत्रे निहते हस्तिशिक्षाविशारदे}
{अङ्गाः क्रुद्धा महामात्रा नागैर्नकुलमभ्ययुः}


\twolineshloka
{चलत्पताकैः सुमुखैर्हेमकक्ष्यातनुच्छदैः}
{मिमर्दिषन्तस्त्वरिताः प्रदीप्तैरिव पर्वतैः}


\twolineshloka
{मेकलोत्कलकालिङ्गा निषधास्ताम्रलिप्तकाः}
{शरतोमरवर्षाणि विमुञ्चन्तो जिघांसवः}


\twolineshloka
{तैश्छाद्यमानं नकुलं निशाकरमिवाम्बुदैः}
{परिपेतुः सुसंरब्धाः पाण्डुपाञ्चालसात्यकाः}


\twolineshloka
{ततस्तदभवद्युद्धं रथिनां हस्तिभिः सह}
{सृजतां शरवर्षाणि तोमरांश्च सहस्रशः}


\twolineshloka
{नागानां प्रास्फुटन्कुम्भा मर्माणि च नखानि च}
{दन्ताश्चैवातिविद्धानां नाराचैर्हेमभूषणैः}


\twolineshloka
{तेषामष्टौ महानागांश्चतुःषष्ट्या सुतेजनैः}
{नाराचैः सहदेवस्तान्पातयामास सादिभिः}


\twolineshloka
{अञ्जोगतिभिरायम्य प्रयत्नाद्धनुरुत्तमम्}
{नाराचैरहनन्नागान्नकुलः कुलनन्दनः}


\twolineshloka
{ततः पाञ्चालशैनेयौ द्रौपदेयाः प्रभद्रकाः}
{शिखण्डी च महानागान्सिषिचुः शरवृष्टिभिः}


\twolineshloka
{ते पाण्डुयोधाम्बुधरैः शत्रुद्विरदपर्वताः}
{बाणवर्षैर्हताः पेतुर्वज्रवर्षैरिवाचलाः}


\twolineshloka
{एवं हत्वा तव गजांस्ते पाण्डुरथकुञ्जराः}
{द्रुतं सेनां च वर्षन्तो भिन्नकूलामिवापगाम्}


\twolineshloka
{तां ते सेनां समालोड्य पाण्डुपुत्रस्य सैनिकाः}
{विक्षोभयित्वा च पुनः कर्णं समभिदुद्रुवुः}


\chapter{अध्यायः ६५}
\twolineshloka
{सञ्जय उवाच}
{}


\twolineshloka
{तेषां प्रवृत्ते सङ्ग्रामे विपुले शोणितोदके}
{रराज लोहितेनोर्वी संसिक्ता बहुधा भृशम्}


\threelineshloka
{ततो रजसि संशान्ते प्रकाशः सर्वतोऽभवत्}
{एतस्मिन्नन्तरे पार्थं कृष्णो वचनमब्रवीत्}
{दर्शयन्निव कौन्तेयं धर्मराजं युधिष्ठिरम्}


\twolineshloka
{एष पाण्डव ते भ्राता धार्तराष्ट्रैर्महाबलैः}
{जिघांसुभिर्महेष्वासैः शीघ्रं पार्थोऽनुसार्यते}


\twolineshloka
{तमन्वगेव पाञ्चालाश्चेदिमात्स्याश्च भारत}
{अनुयान्ति महात्मानं परीप्सन्तो महाजवाः}


\twolineshloka
{एष दुर्योधनः पार्थ गजानीकेन दंशितः}
{राजा सर्वस्य लोकस्य राजानमनुधावति}


\twolineshloka
{जिघांसुः पुरुषव्याघ्रं भ्रातृभिः सहितो बली}
{आशीविषसमस्पर्शैः सर्वायुधविशारदैः}


\twolineshloka
{नदद्भिः सिंहनादांश्च धमद्भिश्चापि वारिजान्}
{बलवद्भिर्महेष्वासैर्विधून्वानैर्धनूंषि च}


\twolineshloka
{एते जिघृक्षवो यान्ति द्विपाश्वरथपत्तयः}
{युधिष्ठिरं धार्तराष्ट्रो रत्नोत्तममिवार्थिनः}


\twolineshloka
{पश्य सात्वतभीमाभ्यां निरुद्धा विष्ठिताः पुनः}
{जिहीर्षवोऽमृतं दैत्याः शक्राग्निभ्यामिवाहवे}


\twolineshloka
{एते बहुत्वात्त्वरिताः पुनर्गच्छन्ति पाण्डवम्}
{समुद्रमिव वार्योघाः प्रावृट््काले महारथाः}


\twolineshloka
{मृत्योर्मुखगतं मन्ये कुन्तीपुत्रं युधिष्ठिरम्}
{हुतमग्नौ च कौन्तेयं दुर्योधनवशं गतम्}


\twolineshloka
{यथायुक्तमनीकं हि धार्तराष्ट्रस्य पाण्डव}
{नास्य शक्रोऽपि मुच्येत सम्प्राप्तो बाणगोचरम्}


\twolineshloka
{दुर्योधनस्य वीरस्य शरौघाञ्शीघ्रमस्यतः}
{सङ्क्रुद्धस्यान्तकस्येव को वेगं संसहेद्रणे}


\twolineshloka
{रसतस्तस्य वीरस्य द्रौणेः शारद्वतस्य च}
{कर्णस्य चेषुवेगो वै पर्वतानपि शातयेत्}


% Check verse!
कर्णेन च कृतो राजा विमुखोऽद्य तु दृश्यते
\threelineshloka
{बलवाँल्लुघुहस्तश्च कृती युद्धविशारदः}
{राधेयः पाण्डवश्रेष्ठं शक्तः पीडयितुं रणे}
{सहितो धृतराष्ट्रस्य पुत्रैः शूरैर्महाबलैः}


\twolineshloka
{तस्यैभिर्युध्यमानस्य सङ्ग्रामे शंसितात्मनः}
{अन्यैरपि च पार्थस्य कृतं कर्म महारथैः}


\twolineshloka
{उपवासकृशो राजा भृशं भरतसत्तमः}
{ब्राह्मो बले स्थितो ह्येष न क्षात्रे हि बले विभुः}


\twolineshloka
{कर्णेन चाभियुक्तोऽयं भूपतिः शत्रुतापनः}
{संशयं समनुप्राप्तः पाण्डवो वै युधिष्ठिरः}


\twolineshloka
{न जीवति महाराजो मन्ये पार्थ युधिष्ठिरः}
{यद्भीमसेनः सहते सिंहनादममर्षणः}


\twolineshloka
{नर्दतां धार्तराष्ट्राणां पुनःपुनररिन्दमः}
{धमतां च महाशङ्खान्सङ्ग्रामे जितकाशिनाम्}


\twolineshloka
{युधिष्ठिरं पाण्डवेयं हतेति भरतर्षभ}
{सञ्चोदयत्यसौ कर्णो धार्तराष्ट्रान्महाबलान्}


\twolineshloka
{स्थूणाकर्णास्त्रजालेन पार्थ पाशुपतेन च}
{प्रच्छादयन्ति राजानमनुयान्ति महारथाः}


\chapter{अध्यायः ६६}
\twolineshloka
{श्रीभगवानुवाच}
{}


\twolineshloka
{आतुरो मे मतो राजा अविपह्यश्च भारत}
{यथैनमनुवर्तन्ते पञ्चालाः सह पाण्डवैः}


\twolineshloka
{त्वरमाणास्त्वराकाले सर्वशस्त्रभृतां वराः}
{मज्जन्तमिव पाताले बलिनो ह्युज्जिहीर्षवः}


\twolineshloka
{न केतुर्दृश्यते राज्ञः कर्णस्य पिहितः शरैः}
{पश्यतोर्यमयोः पार्थ सात्यकेश्च शिखण्डिनः}


\twolineshloka
{धृष्टद्युम्नस्य भीमस्य शतानीकस्य वा विभो}
{पाञ्चालानां च सर्वेषां चेदीनां चैव भारत}


\twolineshloka
{एष कर्णो रणे पार्थ पाण्डवानामनीकिनीम्}
{शरैर्विध्वंसयामास नलिनीमिव कुञ्जरः}


\twolineshloka
{एते द्रवन्ति रथिनस्त्वदीयाः पाण्डुनन्दन}
{पश्य पश्य यथा पार्थ गच्छन्त्येते महारथाः}


\twolineshloka
{एते भारत मातङ्गाः कर्णेनाभिहताः शरैः}
{आर्तनादान्विकुर्वाणा विद्रवन्ति दिशो दश}


\twolineshloka
{रथानां द्रवते वृन्दमेतच्चैव समन्ततः}
{द्राव्यमाणं रणे पार्थ कर्णेनामिततेजसा}


\twolineshloka
{हस्तिकक्ष्यां रणे पश्य चरन्तीं तत्रतत्र ह}
{रथस्थं सूतपुत्रस्य केतुं केतुमतां वर}


\twolineshloka
{असौ धावति राधेयो भीमसेनरथं प्रति}
{किरञ्शरशतान्येव विनिघ्नंस्तव वाहिनीम्}


\twolineshloka
{एते नश्यन्ति पाञ्चाला द्राव्यमाणा महात्मना}
{शक्रेणेव यथा दैत्या द्राव्यमाणा महात्मना}


\twolineshloka
{एष कर्णो रणे जित्वा पाञ्चालान्पाण्डुसृञ्जयान्}
{दिशो विप्रेक्षते सर्वास्त्वदर्थमिति मे मतिः}


\twolineshloka
{एष कर्णो धनुःश्रेष्ठं विधून्वन्बहुशोभते}
{शत्रुं जित्वा यथा शक्रो देवसङ्खैः समावृतः}


\twolineshloka
{एते नर्दन्ति कौरव्या दृष्ट्वा कर्णस्य विक्रमम्}
{त्रासयन्तो रणे पाण्डून्सृंजयांश्च समन्ततः}


\twolineshloka
{एष सर्वात्मना पाण्डूंस्त्रासयित्वा महारणे}
{अभिभाषति राधेयः सर्वसैन्यानि मानद}


\twolineshloka
{अभिद्रवत भद्रं वो द्रुतं द्रवत कौरवाः}
{यथा न जीववान्कश्चिन्मुच्येत युधि सृञ्जयः}


\twolineshloka
{तथा कुरुत संयत्ता वयं यास्याम पृष्ठतः}
{एवमुक्त्वा गतो ह्येष पृष्ठतो विकिरञ्छरान्}


\twolineshloka
{पश्य कर्णं रणे पार्थ श्वेतच्छत्रविराजितम्}
{उदयं पर्वतं यद्वच्छशाङ्केनाभिशोभितम्}


\twolineshloka
{पूर्णचन्द्रनिकाशेन मूर्ध्नि च्छत्रेण भारत}
{ध्रियमाणेन समरे श्रीमच्छतशलाकिना}


\twolineshloka
{एष त्वां प्रेक्षते कर्णः सकटाक्षं धनञ्जय}
{उत्तमं यत्नमास्थाय ध्रुवमेष्यति संयुगे}


\twolineshloka
{पश्य ह्येनं महाबाहो विधुन्वानं महद्धनुः}
{शरांश्चाशीविषाकारान्विसृजन्तं महारणे}


\threelineshloka
{असौ निवृत्तो राधेयो दृष्ट्वा ते वानरध्वजम्}
{प्रार्थयन्समरं पार्थ त्वया सह परन्तप}
{वधाय ह्यात्मनोऽभ्येति पावकं शलभो यथा}


\twolineshloka
{कर्णमेकाकिनं दृष्ट्वा धार्तराष्ट्रो रणाजिरे}
{त्वां च पार्थाभिसंरब्धं कर्णं प्रति महारथम्}


\twolineshloka
{कृतागसं च राधेयं धर्मात्मनि युधिष्ठिरे}
{अत्मानं च कृतार्थं च समीक्ष्य भरतर्षभ}


\twolineshloka
{असौ दुर्योधनः क्रुद्धो रथानीकेन भारत}
{रिरक्षिषुः सुसंवृत्तो धार्तराष्ट्रो निवर्तते}


\twolineshloka
{सर्वैः सहैभिर्दुष्टात्मा बध्यतां च प्रयत्नतः}
{त्वया यशश्च राज्यं च सुखं चोत्तममिच्छता}


\threelineshloka
{अदीनयोर्विश्रुतयोर्युवयोर्योत्स्यमानयोः}
{देवासुरे पार्थ मृधे देवदानवयोरिव}
{पश्यन्तु कौरवाः सर्वे तव पार्थ पराक्रमम्}


\twolineshloka
{त्वां च दृष्ट्वातिसंरब्धं कर्णं च भरतर्षम्}
{असौ दुर्योधनः क्रुद्धो नोत्तरं प्रतिपद्यते}


\twolineshloka
{आत्मानं च कृतात्मानं समीक्ष्य भरतर्षभ}
{कृतागसं च राधेयं धर्मात्मनि युधिष्ठिरे}


\twolineshloka
{प्रतिपद्यस्व कौन्तेय प्राप्तकालमनन्तरम्}
{आर्यां युद्धे मतिं कृत्वा प्रत्येहि रथयूथपम्}


\twolineshloka
{पञ्च ह्येतानि मुख्यानि रथानां रथसत्तम}
{शतान्यxxxxxxजलिनां तिन्मतेजसाम्}


\twolineshloka
{पञ्च नाघxxxxx द्विगुणा वाजिनस्तथा}
{अभिसंहत्य कौन्तेय पदाताः प्रयुतानि च}


\twolineshloka
{xxxxxxxx वीर बलं तामभिवर्तते}
{द्रोमपुत्रं पुरस्कृत्य तच्छीघ्रं सन्निषूदय}


\twolineshloka
{निकृत्त्यैतद्रथानीकं बलिनं लोकविश्रुतम्}
{सूतपुत्रे महेष्वासे दर्शयात्मानमात्मना}


\threelineshloka
{xxxxxxxxxxxxxxxभरतर्षभ}
{xxxxxxxxxxxxx पाञ्चालानभिघावति}
{}


\twolineshloka
{xxxxxx हि पश्यामि धृष्टद्युम्नरर्थ प्रति}
{xxxxxx पाञ्चालानिति मन्ये परन्तप}


\twolineshloka
{आचचक्षे प्रियं पार्थ तवेदं भरतर्षभ}
{राजासौ कुशली श्रीमान्धर्मपुत्रो युधिष्ठिरः}


\twolineshloka
{असौ भीमो महाबाहुः सन्निवृत्तश्चमूयुखे}
{वृतः सृञ्जय सैन्येन शैनेयेन च भप्तत}


\twolineshloka
{वध्यन्त एते समरे कौरवा निशितैः शरैः}
{भीमसेनेन कौन्तेय पाञ्चालैश्च महात्मभिः}


\twolineshloka
{सेना हि धार्तराष्ट्रस्य विमुखा व्यद्रवद्रणात्}
{वेगेन भीमसेनस्य विहता विविधैः शरैः}


\twolineshloka
{विपन्नसस्येव मही रुधिरेण समुक्षिता}
{भारती भरतश्रेष्ठ सेना कृपणदर्शना}


\twolineshloka
{निवृत्तं पश्य कौन्तेय भीमसेनं युधाम्पतिम्}
{आशीविषमिव क्रुद्धं तस्माद्द्रवति भारति}


\twolineshloka
{पीतरक्तासितसितास्ताराचन्द्रार्कमण्डिताः}
{पताका विप्रकीर्यन्ते छत्राण्येतानि चार्जुन}


\twolineshloka
{सौवर्णा राजताश्चैव तैजसाश्च पृथग्विधाः}
{केतवोऽभिनिपात्यन्ते हस्त्यश्वं च प्रकीर्यते}


\twolineshloka
{रथेभ्यः प्रपतन्त्येते रथिनो विगतासवः}
{नानावर्णैर्हता बाणैः पाञ्चालैरपलायिभिः}


\twolineshloka
{निर्मनुष्यान्गजानश्वान्रथांश्चैव धनञ्जय}
{समाद्रवन्ति पाञ्चाला धार्तराष्ट्रांस्तरस्विनः}


\twolineshloka
{विमृद्रन्ति नरव्याघ्रा भीमसेनबलाश्रयात्}
{बलं परेषां दुर्धर्षं त्यक्त्वा प्राणानरिन्दम्}


\twolineshloka
{एते नर्दिन्ति पाञ्चाला ध्मापयन्ति च वारिजान्}
{अभिद्रवन्ति च रणे मृद्रन्तः सायकैः परान्}


\twolineshloka
{पश्य स्वर्गस्य माहात्म्यं पाञ्चाला हि पराक्रमात्}
{धार्तराष्ट्रान्विनिघ्नन्तो विशन्त्येते रथोत्तमान्}


\twolineshloka
{[शस्त्रमाच्छिद्य शत्रूणां सायुधानां निरायुधाः}
{तेनैवैतानमोघास्त्रा निघ्नन्ति च नदन्ति च}


\twolineshloka
{शिरांस्येतानि पात्यन्ते शत्रूणां बाहवोऽपि च}
{रथनागहया वीरा यशस्याः सर्व एव च]}


\twolineshloka
{सर्वतश्चाभिपन्नैषा धार्तराष्ट्री महाचमूः}
{त्यक्त्वा प्राणान्महेष्वासैः पाञ्चालैः परिपात्यते}


\twolineshloka
{सुहृदश्च पराक्रान्ताः कृपकर्णादयो विभो}
{निवारणे महेष्वासाः पाञ्चालानां परन्तप}


\twolineshloka
{अनिवृत्तांश्च भीतांस्तान्धार्तराष्ट्रान्परन्तप}
{धृष्टद्युम्नमुखा वीरा घ्नन्ति शत्रून्सहस्रशः}


\twolineshloka
{रथाश्च विविधाः सर्वे निवृत्ते भरतर्षभे}
{विवर्णमुखभूयिष्ठा धार्तराष्ट्री महाचमूः}


\twolineshloka
{पश्य भीमेन नाराचैर्भिन्ना नागाः पतन्त्यमी}
{वज्रिवज्रहतानीव शिखराणि धराभृताम्}


\twolineshloka
{भीमसेनस्य निर्विद्धा बाणैः सन्नतपर्वभिः}
{स्वान्यनीकानि मृद्गन्तो द्रवन्त्येते महागजाः}


\twolineshloka
{`एते द्रवन्ति कुरवो भीमसेनभयार्दिताः}
{त्यक्त्वा रथान्गजांश्चैव हयांश्चैव सहस्रशः}


\twolineshloka
{हस्त्यश्वरथपत्तीनां द्रवतां निःस्वनं शृणु}
{भीमसेनस्य निनदं द्रावयानस्य कौरवान्}


\twolineshloka
{अभिजानामि भीमस्य सिंहनादं पुनःपुनः'}
{नदतः पाण्डवेयस्य सङ्ग्रामे जितकाशिनः}


\twolineshloka
{एष नैषादिरभ्येति द्विपमुख्येन पाण्डवम्}
{विसृजंस्तोमरान्क्रुद्धो दण्डपाणिरिवान्तकः}


\twolineshloka
{तस्य चैव भुजौ छिन्नौ भीमसेनेन गर्जतः}
{नागश्च क्रकरप्रख्यैर्नाराचैर्दशभिर्हतः}


\twolineshloka
{हन्तैते पुनरायान्ति नागा ह्यन्ये प्रहारिणः}
{नीलाञ्जनचयप्रख्या महामात्रैरधिष्ठिताः}


\threelineshloka
{शक्तितोमरसङ्घतैर्विनिघ्नन्तो वृकोदरम्}
{सप्तसप्त च नागस्था वैजयन्त्यश्च सध्वजाः}
{नवत्या निशितैर्बाणैश्छिन्नाः पार्थाग्रजेन ते}


% Check verse!
दशभिर्दशभिश्चैको नाराचैर्निहतो गजः
\twolineshloka
{न चासौ धार्तराष्ट्राणां श्रूयते निनदस्तथा}
{पुरन्दरसमे क्रुद्धे निवृत्ते भरतर्षभे}


\twolineshloka
{अक्षौहिण्यस्तथा तिस्रो धार्तराष्ट्रस्य सङ्गताः}
{क्रुद्धेन भीमसेनेन नरसिंहेन वारिताः}


\twolineshloka
{न शक्नुवन्ति वै पार्थं पार्थिवाः समुदीक्षितुम्}
{मध्यन्दिनगतं सूर्यं यथा दुर्बलचक्षुषः}


\twolineshloka
{एते भीमस्य सन्त्रस्ताः सिंहस्येवेतरे मृगाः}
{शरैः सन्त्रासिताः सह्ख्ये न लभन्ते सुखं क्वचित्}


\fourlineindentedshloka
{राधेयो बहुभिः सार्धमसौ गच्छति वेगितः}
{वर्धयित्वा तु भीरुं तं पार्श्वतो ह्यानयद्धनुः}
{तं पालयन्महाराजं धार्तराष्ट्रं बलान्वितः ॥सञ्जय उवाच}
{}


\threelineshloka
{एतच्छ्रुत्वा महाबाहुर्वासुदेवाद्धनञ्जयः}
{भीमसेनेन तत्कर्म कृतं दृष्ट्वा सुदुष्करम्}
{अर्जुनो व्यधमच्छिष्टान्संशप्तकगणान्बहून्}


\twolineshloka
{ते वध्यमानाः पार्थेन संशप्तकगणाः प्रभो}
{शक्रस्यातिथितां गत्वा विशोका ह्यभवंस्तदा}


\twolineshloka
{नारायणांस्तु गोपालान्व्यधमत्पाण्डुनन्दनः}
{उत्तमं वेxxxxxप्तास्थाय चण्डवायुर्धनानिव}


\twolineshloka
{अन्वकीर्यन्त भीतास्ते तत्रतत्रैव भारत}
{लुलितांश्च ततः शूरानहनत्पुरुषोत्तमः}


\twolineshloka
{पुनश्च पुरुषव्याघ्रः शरैः सन्नतपर्वभिः}
{जघान धार्तराष्ट्रस्य चतुर्विधबलां चमूम्}


\chapter{अध्यायः ६७}
\twolineshloka
{धृतराष्ट्र उवाच}
{}


\twolineshloka
{निवृत्ते भीमसेने च पाण्डवे च युधिष्ठिरे}
{वध्यमाने बले चापि मामके पाण्डुसृञ्जयैः}


\twolineshloka
{द्रवमाणे बलौघे च निराक्रन्दे मुहुर्मुहुः}
{`अवशेषं न पश्यामि मम सैन्येषु सञ्जय}


\twolineshloka
{अहो बत दशां प्राप्तो न हि शक्ष्यामि जीवितुम्}
{जयकाङ्क्षी कथं सूत पुत्राणामनिवर्तिनाम्}


\twolineshloka
{कथं जीवामि निहताञ्श्रुत्वा च मम सैनिकान्}
{बहुनाऽद्य किमुक्तेन दैवं तेषां परायणम्'}


\twolineshloka
{मामकाः किमकुर्वन्त तन्ममाचक्ष्व सञ्जय ॥सञ्जय उवाच}
{}


\twolineshloka
{दृष्ट्वा भीमं महाबाहुं तव पुत्रः प्रतापवान्}
{क्रोधरक्तेक्षणो राजन्भीमसेनमभिद्रवत्}


\threelineshloka
{`तिष्ठतिष्ठ पृथापुत्र पश्य मेऽद्य पराक्रमम्}
{अद्य त्वां प्रेषायष्यामि यमस्य सदनं प्राते}
{इत्युक्त्वा प्रययौ कर्णो यत्र भीमो व्यवस्थितः'}


\twolineshloka
{तावकं तु बलं दृष्ट्वा भीमसेनात्पराङ्मुखम्}
{यत्नेन महता कर्णः पर्यवस्थापयद्बली}


\twolineshloka
{व्यवस्थाप्य महाबाहुस्तव पुत्रस्य वाहिनीम्}
{प्रत्युद्ययौ तदा कर्णः पाण्डवान्युद्धदुर्मदान्}


\twolineshloka
{प्रत्युद्ययुश्च राधेयं पाण्डवानां महारथाः}
{धुन्वानाः कार्मुकाण्याजौ विक्षिपन्तश्च सायकान्}


\twolineshloka
{भीमसेनः शिनेर्नप्ता शिखण्डी जनमेजयः}
{धृष्टद्युम्नश्च बलवान्सर्वे चापि प्रभद्रकाः}


\twolineshloka
{पाञ्चालानां नरव्याघ्राः समन्तात्त्व वाहिनीम्}
{अभ्यद्रवन्त सङ्क्रुद्धाः समरे जितकाशिनः}


\twolineshloka
{तथैव तावका राजन्पाण्डवानामनीकिनीम्}
{अभ्यद्रवन्त त्वरिता जिघांसन्तो महारथान्}


\twolineshloka
{रथनागाश्वकलिलं पत्तिध्वजसमाकुलम्}
{बभूव पुरुषव्याघ्र सैन्यमद्भुतदर्शनम्}


\twolineshloka
{शिखण्डी तु ययौ कर्णं धृष्टद्युम्नः सुतं तव}
{दुःशासनं महाराज महासेनः समभ्ययात्}


\twolineshloka
{नकुलो वृषसेनं तु चित्रसेनं युधिष्ठिरः}
{उलूकं समरे राजन्सहदेवः समभ्ययात्}


\twolineshloka
{सात्यकिः शकुनिं चापि द्रौपदेयाश्च कौरवान्}
{अर्जुनं च रणे यत्तो द्रोणपुत्रो महारथः}


\twolineshloka
{युधामन्युं महेष्वासं गौतमोऽभ्यपतद्रणे}
{कृतवर्मा च बलवानुत्तमौजसमाद्रवत्}


\twolineshloka
{भीमसेनः कुरून्सर्वान्पुत्रांश्च तव मारिष}
{सहानीकान्महाबाहुरेक एव न्यवारयत्}


\twolineshloka
{शिखण्डी तु ततः कर्णं विचरन्तमभीतवत्}
{भीष्महन्ता महाराज वारयामास पत्रिभिः}


\twolineshloka
{प्रतिरुद्धस्ततः कर्णो रोपात्प्रस्फुरिताधरः}
{शिखण्डिनं त्रिभिर्बाणैर्भ्रुवोर्मध्येऽभ्यताडयत्}


\twolineshloka
{धारयंस्तु स तान्वाणाञ्शिखण्डी बह्वशोभत}
{राजतः पर्वतो यद्वत्त्रिभिः शृङ्गैः समन्वितः}


\twolineshloka
{सोऽतिविद्धो महेष्वासः सूतपुत्रेण संयुगे}
{कर्णं विव्याध समरे नवत्या निशितैः शरैः}


\twolineshloka
{तस्य कर्णो हयान्हत्वा सारथिं च शरैः}
{उन्ममाथ ध्वजं चास्य क्षुरप्रेण महारथः}


\twolineshloka
{हताश्वात्तु ततो यानादवप्लुत्य महारथः}
{शक्तिं चिक्षेप कर्णाय सङ्क्रुद्धः शत्रुतापनः}


\twolineshloka
{तां छित्त्वा समरे कर्णस्त्रिभिर्भारत सायकैः}
{शिखण्डिनमथाविध्यन्नवभिर्निशितैः शरैः}


\twolineshloka
{कर्णचापच्युतान्बाणान्वर्जयंस्तु नरोत्तमः}
{अपयातस्ततस्तूर्णं शिखण्डी भृशविक्षतः}


\twolineshloka
{ततः कर्णो महाराज पाण्डुसैन्यान्यशातयत्}
{तूलराशिं समासाद्य यथा वायुर्महाबलः}


\twolineshloka
{धृष्टद्युम्नो महाराज तव पुत्रेण पीडितः}
{दुःशासनं त्रिभिर्बाणैः प्रत्यविध्यत्स्तनान्तरे}


\twolineshloka
{तस्य दुःशासनो बाहुं सव्यं विव्याध मारिष}
{स तेन रुक्मपुङ्खेन भल्लेन नतपर्वणा}


\twolineshloka
{धृष्टद्युम्नस्तु निर्विद्धः शरं घोरममर्षणः}
{दुःशासनाय सङ्क्रुद्धः प्रेषयामास भारत}


\twolineshloka
{आपतन्तं महावेगं धृष्टद्युम्नसमीरितम्}
{शरैश्चिच्छेद पुत्रस्ते त्रिभिरेव विशाम्पते}


\twolineshloka
{अथापरैः षोडशभिर्भल्लैः कनकभूषणैः}
{धृष्टद्युम्नं समासाद्य बाह्वोरुरसि चार्पयत्}


\twolineshloka
{ततः स पार्षतः क्रुद्धो धनुश्चिच्छेद मारिष}
{क्षुरप्रेण सुतीक्ष्णेन तत उच्चुक्रुशुर्जनाः}


\twolineshloka
{अथान्यद्धनुरादाय पुत्रस्ते प्रहसन्निव}
{धृष्टद्युम्नं शरव्रातैः समन्तात्पर्यवारयत्}


\twolineshloka
{तव पुत्रस्य ते दृष्ट्वा विक्रमं सुमहात्मनः}
{व्यस्मयन्त रणे योधाः सिद्धाश्चाप्सरसस्तथा}


\twolineshloka
{धृष्टद्युम्नं तु पश्याम घटमानं महाबलम्}
{दुःशासनेन संरुद्धं सिंहेनेव महागजम्}


\chapter{अध्यायः ६८}
\twolineshloka
{सञ्जय उवाच}
{}


\twolineshloka
{ततः सरथनागाश्वाः पाञ्चालाः पाण्डुसृञ्जयाः}
{सेनापतिं परीप्सन्तो रुरुधुस्तनयं तव}


\twolineshloka
{ततः प्रववृते युद्वं तावकानां परैः सह}
{घोरं प्राणभृतां काले भीमरूपं परन्तप}


\twolineshloka
{नकुलो वृषसेनं तु विद्ध्वा पञ्चभिरायसैः}
{पितुः समीपे तिष्ठन्तं त्रिभिरन्त्यैरविध्यत}


\twolineshloka
{नकुलं तु ततः शूरो वृषसेनो हसन्निव}
{नाराचेन सुतीक्ष्णेन विव्याध हृदये भृशम्'}


\twolineshloka
{सोऽतिविद्धो बलवता शत्रुणा शत्रुकर्शन}
{शत्रुं विव्याध विंशत्या सूतं चैवास्य सप्तभिः}


\twolineshloka
{ततः शरसहस्रेण तावुभौ पुरुषर्षभौ}
{अन्योन्यमाच्छादयतां तवाभज्यत वाहिनी}


\twolineshloka
{स दृष्ट्वा प्रद्रुतां सेनां धार्तराष्ट्रस्य सूतजः}
{निवारयामास बलादनुपत्य विशाम्पते}


\twolineshloka
{`एतस्मिन्नन्तरे युद्धं कष्टमासीद्विशाम्पते'}
{निवृत्ते तु ततः सैन्ये नकुलः कौरवान्ययौ}


\twolineshloka
{कर्णपुत्रस्तु समरे हित्वा नकुलमेव तु}
{जुगोप चक्रं त्वरितो राधेयस्यैव मारिष}


\twolineshloka
{उलूकस्तु रणे क्रुद्धः सहदेवेन मारिष}
{`निवारितः शरशतैः क्रुद्धेन रणमूर्धनि}


\twolineshloka
{तस्याश्वांश्चतुरो हत्वा सहदेवः प्रतापवान्}
{सारथिं प्रेषयामास यमस्य सदनं प्रति}


\twolineshloka
{उलूकस्तु ततो यानादवप्लुत्य विशाम्पते}
{त्रिगर्तानां बलं तूर्णं जगाम पितृनन्दनः}


\twolineshloka
{सात्यकिः शकुनिं विद्ध्वा विंशत्या निशितैः शरैः}
{ध्वजं चिच्छेद भल्लेन सौबलस्य हसन्निव}


\twolineshloka
{सौबलस्तस्य समरे क्रुद्धो राजन्प्रतापवान्}
{विदार्य कवचं भूयो ध्वजं चिच्छेद काञ्चनम्}


\twolineshloka
{तथैनं निशितैर्बाणैः सात्यकिः प्रत्यविध्यत}
{सारथिं च महाराज त्रिभिरेव समार्पयत्}


\twolineshloka
{अथास्य वाहांस्त्वरितः शरैर्नित्ये यमक्षयम्}
{ततोऽवप्लुत्य सहसा शकुनिर्भरतर्षभ}


\twolineshloka
{आरुरोह रथं तूर्णमुलूकस्य महात्मनः}
{अपोवाहाथ शीघ्रं स शैनेयाद्युद्धशालिनः}


\twolineshloka
{सात्यकिस्तु रण राजंस्तावकानामनीकिनीम्}
{अभिदुद्राव वेगेन ततोऽनीकमभज्वत}


\twolineshloka
{शैनेयशरसञ्छन्नं तव सैन्यं विशाम्पते}
{भेजे दशा दिशस्तूर्णं न्यपतच्च गतासुवत्}


% Check verse!
भीमसेनं तव सुतो वारयामास संयुगे
\twolineshloka
{तं तु भीमो मुहूर्तेन व्यश्वसूतरथध्वजम्}
{चक्रे लोकेश्वरं तत्र तेनातुष्यन्त वै जनाः}


\threelineshloka
{ततोऽपायान्नृपस्तस्माद्भीमसेनभयार्दितः}
{कुरुसैन्यं ततः सर्वं भीमसेनमुपाद्रवत्}
{तत्र नादो महानासीद्धीमसेनं जिघांसताम्}


\threelineshloka
{युधामन्युः कृपं विद्ध्वा धनुरस्याशु चिच्छदे}
{अथान्यद्धनुरादाय कृपः शस्त्रभृतां वरः}
{युधामन्योर्ध्वजं सूतं छत्रं चापातयत्क्षितौ}


% Check verse!
ततोऽपायाद्रथेनैव युधामन्युर्महारथः
\twolineshloka
{उत्तमौजाश्च हार्दिक्यं भीमं भीमपराक्रमम्}
{छादयामास सहसा मेघो वृष्ट्येव पर्वतम्}


\twolineshloka
{तद्युद्धमासीत्सुमहद्धोररूपं परन्तप}
{तादृशं न मया युद्धं दृष्टपूर्वं विशाम्पते}


\twolineshloka
{कृतवर्मा ततो राजन्नुत्तमौजसमाहवे}
{हृदि विव्याध सहसा रथोपस्थ उपाविशत्}


\twolineshloka
{सारथिस्तमपोवाह रथेन रथिनां वरम्}
{ततस्तु सात्वतो राजन्पाण्डुसैन्यमभिद्रवत्}


\twolineshloka
{[दुःशासनः सौबलश्च गजानीकेन पाण्डवम्}
{महता परिवार्यैव क्षुद्रकैरभ्यताडयत्}


\twolineshloka
{ततो भीमः शरशतैर्दुर्योधनममर्षणम्}
{विमुखीकृत्य तरसा गजानीकमुपाद्रवत्}


\twolineshloka
{तमापतन्तं सहसा गजानीकं वृकोदरः}
{दृष्ट्वैव सुभृशं क्रुद्धो दिव्यमस्त्रमुदैरयत्}


% Check verse!
गजैर्गजानभ्यहनद्वज्रेणेन्द्र इवासुरान्
\twolineshloka
{ततोऽन्तरिक्षं बाणौघैः शलभैरिव पादपम्}
{छादयामास समरे गजान्निघ्नन्वृकोदरः}


\twolineshloka
{ततः कुञ्जरयूथानि समेतानि सहस्रशः}
{व्यधमत्तरसा भीमो मेघसङ्घानिवानिलः}


\twolineshloka
{सुवर्णजालापिहिता मणिजालैश्च कुञ्जराः}
{रेजुरभ्यधिकं संख्ये विद्युत्वन्त इवाम्बुदाः}


\twolineshloka
{ते वध्यमाना भीमेन गजा राजन्विदुद्रुवुः}
{केचिद्विभिन्नहृदयाः कुञ्जरा न्यपतन्भुवि}


\twolineshloka
{पतितैर्निपतद्भिश्च गजैर्हेमविभूषितैः}
{अशोभत मही तत्र विशीर्णैरिव पर्वतैः}


\twolineshloka
{दीप्ताभै रत्नवद्भिश्च पतितैर्गजयोधिभिः}
{रराज भूमिः पतितैः क्षीणपुण्यैरिव ग्रहैः}


\twolineshloka
{ततो भिन्नकटा नागा भिन्नकुम्भकरास्तथा}
{दुद्रुवुः शतशः संख्ये भीमसेनशराहताः}


\twolineshloka
{केचिद्वमन्तो रुधिरं भयार्ताः पर्वतोपमाः}
{व्यद्रवञ्छरविद्धाङ्गा धातुचित्रा इवाचलाः}


\twolineshloka
{महाभुजगसङ्काशौ चन्दनागुरुरूषितौ}
{अपश्यं भीमसेनस्य धनुर्विक्षिपतो भुजौ}


\twolineshloka
{तस्य ज्यातलनिर्घोषं श्रुत्वाशनिसमस्वनम्}
{विमुञ्चन्तः शकृन्मूत्रं गजाः प्रादुद्रुवुर्भृशम्}


\twolineshloka
{भीमसेनस्य तत्कर्म राजन्नेकस्य धीमतः}
{निघ्नतः सर्वभूतानि रुद्रस्येव च निर्बभौ ॥]}


\chapter{अध्यायः ६९}
\twolineshloka
{सञ्जय उवाच}
{}


\twolineshloka
{ततः श्वेताश्वसंयुक्ते नारायणसमाहिते}
{तिष्ठन्रथवरे श्रीमानर्जुनः समपद्यत}


\twolineshloka
{तद्बलं नृपतिश्रेष्ठ तावकं विजयो रणे}
{व्यक्षोभयदुदीर्णाश्वं महोदधिमिवानिलः}


\twolineshloka
{दुर्योधनस्तव सुतः प्रमत्ते श्वेतवाहने}
{अभ्येत्य सहसा क्रुद्धः सैन्यार्धेनाभिसंवृतः}


\twolineshloka
{पर्यवारयदायान्तं युधिष्ठिरममर्षणम्}
{क्षुरप्राणां त्रिसप्तत्या ततोऽविध्यत पाण्डवम्}


\twolineshloka
{अक्रुध्यत भृशं तत्र कुन्तीपुत्रो युधिष्ठिरः}
{स भल्लांस्त्रिंशतस्तूर्णं तव पुत्रे न्यवेशयत्}


% Check verse!
ततो धावन्त कौरव्या जिघृक्षन्तो युधिष्ठिरम्
\threelineshloka
{दुष्टभावान्पराञ्ज्ञात्वा समवेता महारथाः}
{आजग्मुस्तं परीप्सन्तः कुन्तीपुत्रं युधिष्ठिरम् ॥`सञ्जय उवाच}
{}


\twolineshloka
{युधिष्ठिरश्चित्रसेनं शरवर्षैरवाकिरत्}
{चित्रसेनस्तु कौन्तेयं सङ्क्रुद्धः समवारयत्}


\twolineshloka
{मुहूर्ताद्विमुखीकृत्य चित्रसेनं स धर्मराट्}
{तावकं सैन्यमभ्यघ्नत्समन्तान्निशितैः शरैः}


\twolineshloka
{तावका हि माबाहो दुर्योधनपुरोगमाः}
{युधिष्ठिरं जिघृक्षन्तः सर्वसैन्यं समाक्षिपन्}


\twolineshloka
{दृष्टप्रभावांस्तान्मत्वा समवेतान्महारथान्}
{आजग्मुः सम्परीप्सन्तः पाण्डवेयं युधिष्ठिरम्'}


\twolineshloka
{नकुलः सहदेवश्च धृष्टद्युम्नश्च पार्षतः}
{अक्षौहिण्या परिवृतास्तेऽभ्यधावन्युधिष्टिरम्}


\twolineshloka
{भीमसेनश्च नाराचैर्मृद्रंस्तव महारथान्}
{अभ्यधावदभिप्रेप्सू राजानं शत्रुभिर्वृतम्}


\twolineshloka
{तांस्तु सर्वान्महेष्वासान्कर्णो वैकर्तनो वृषा}
{शरवर्षेण महता प्रत्यवारयदागतान्}


\twolineshloka
{शरौघान्विसृजन्तस्ते प्रेयन्तश्च कुञ्जरान्}
{न शेकुर्यत्नवन्तोऽपि राधेयं प्रतिवीक्षितुम्}


\twolineshloka
{तांश्च सर्वान्महेष्वासान्सर्वशस्त्रविशारदान्}
{महता शरवर्षेण राधेयः प्रत्यवारयत्}


\twolineshloka
{दुर्योधनं च विंशत्या नाराचानां कृती बली}
{अविध्यत्तूर्णमभ्येत्य सहदेवो महीपतिम्}


\twolineshloka
{स विद्धः सहदेवेन रराज बलसन्निधौ}
{प्रभिन्न इव मातङ्गो रुधिरेण परिप्लुतः}


\twolineshloka
{दृष्ट्वा तव सुतं तत्र गाढविद्धं सुतेजनैः}
{अभ्यधावद्दृढं क्रुद्धो राधेयो रथिनां वरः}


\twolineshloka
{दुर्योधनं तथा दृष्ट्वा शीघ्रमस्त्रमुदीर्य सः}
{अवधीत्पाण्डवानीकं पाञ्चालांश्चैव मारिष}


\twolineshloka
{ततो यौधिष्ठिरं सैन्यं वध्यमानं महात्मना}
{सहसा प्राद्रावद्राजन्मूतपुत्रशरार्दितम्}


\twolineshloka
{विविधा विशिखास्तत्र सम्पतन्ति परस्परम्}
{भल्लाः पुङ्खसमाश्लिष्टाः सूतपुत्रधनुश्च्युताः}


\twolineshloka
{अन्तरिक्षे शरौघाणां पततां च परस्परम्}
{सङ्घर्पेण महाराज पावकः समजायत}


\twolineshloka
{तकतो दशदिशः कर्णः शलभैरिव यायिभिः}
{अभ्यघ्नंस्तरसा राजञ्शरैः परशरीरगैः}


\twolineshloka
{रक्तचन्दनसन्दिग्धौ मणिहेमविभूषितौ}
{बाहू व्यत्यक्षिपत्कर्णः परमास्त्रं विदर्शयन्}


\twolineshloka
{ततः सर्वा दिशो राजन्सायकैर्विप्रमोहयन्}
{अपीडयद्भृशं कर्णो धर्मराजं युधिष्ठिरम्}


\twolineshloka
{ततः क्रुद्धो महाराज धर्मपुत्रो युधिष्ठिरः}
{निशितैरिषुभिः कर्णं पञ्चाशद्भिः समार्पयत्}


\twolineshloka
{[बाणान्धकारमभवत्तद्युद्धं घोरदर्शनम्}
{हाहाकारो महानासीत्तावकानां विशाम्पते}


\threelineshloka
{वध्यमाने तदा सैन्ये धर्मपुत्रेण मारिष}
{सायकैर्विविधैस्तीक्ष्णैः कङ्कपत्रैः शिलाशितैः}
{भल्लैरनेकैर्विविधैः शक्त्यृष्टिमुसलैरपि}


\twolineshloka
{यत्र यत्र स धर्मात्मा दुष्टां दृष्टिं व्यसर्जयत्}
{तत्र तत्र व्यशीर्यन्त तावका भरतर्षभ}


\twolineshloka
{कर्णोऽपि भृशसङ्कुद्रो धर्मराजं युधिष्ठिरम्}
{नाराचैरर्धचन्द्रैश्च वत्सदन्तैश्च संयुगे}


\twolineshloka
{अमर्षी क्रोधनश्चैव रोषप्रस्फुरिताननः}
{सायकैरप्रमेयात्मा युधिष्ठिरमभिद्रवत्}


% Check verse!
युधिष्ठिरश्चापि स तं स्वर्णपुङ्खैः शितैः शरैः ॥]
\twolineshloka
{प्रहसन्निव तं कर्णः कङ्कपत्रैः शिलाशितैः}
{उरस्यविध्यद्राजानं कुन्तीपुत्रं युधिष्ठिरम्}


\twolineshloka
{स पीडितो भृशं तेन धर्मराजो युधिष्ठिरः}
{उपविश्य रथोपस्थे सूतं याहीत्यनोदयत्}


\twolineshloka
{अक्रोशन्त ततः सर्वे धार्तराष्ट्राः सराजकाः}
{गृह्णीध्वमिति राजानमभ्यधावन्त सर्वशः}


\twolineshloka
{ततः शताः सप्तदश केकयानां प्रहारिणाम्}
{पञ्चालैः सहिता राजन्धार्तराष्ट्रान्न्यवारयन्}


\twolineshloka
{तस्मिन्सुतुमुले युद्धे वर्तमाने जनक्षये}
{दुर्योधनश्च भीमश्च समेयातां महाबलौ}


\chapter{अध्यायः ७०}
\twolineshloka
{सञ्जय उवाच}
{}


\twolineshloka
{कर्णोऽपि शरजालेन केकयानां महारथान्}
{व्यधमत्परमेष्वासानग्रानीके व्यवस्थितान्}


\twolineshloka
{तेषां प्रयतमानानां राधेयस्य निवारणे}
{रथान्पञ्चशतान्कर्णः प्राहिणोद्यमसादनम्}


\twolineshloka
{अविषह्यतमं दृष्ट्वा राधेयं युधि योधिनः}
{भीमसेनमुपागच्छन्कर्णबाणप्रपीडिताः}


\twolineshloka
{रथानीकं विदार्यैव शरै राजन्ननेकधा}
{कर्ण एकरथेनैव युधिष्ठिरमुपाद्रवत्}


\twolineshloka
{सेनानिवेशं गच्छन्तं मार्गणैर्भृशपीडितम्}
{यमयोर्मध्यगं वीरं शनैर्यान्तं विचेतसम्}


\twolineshloka
{स समासाद्य राजानं दुर्योधनहितेप्सया}
{सूतपुत्रस्त्रिभिस्तीक्ष्णैर्विव्याध परमेषुभिः}


\twolineshloka
{तथैव राजा राधेयं प्रत्यविध्यत्स्तनान्तरे}
{शरैस्त्रिभिश्च यन्तारं चतुर्भिश्चतुरो हयान्}


\twolineshloka
{चक्ररक्षौ नृपसुतौ माद्रीपुत्रौ परन्तपौ}
{तावप्यधावतां कर्णं राजानं मा वधीदिति}


\twolineshloka
{तौ पृथक्शरवर्षाभ्यां राधेयं समवर्षताम्}
{नकुलः सहदेवश्च परमं यत्नमास्थितौ}


\twolineshloka
{तथैव तौ प्रत्यविध्यत्सूतपुत्रः प्रतापवान्}
{भल्लाभ्यां शितधाराभ्यां महात्मानावरिन्दमौ}


\twolineshloka
{दन्तवर्णांस्तु राधेयो निजघान मनोजवान्}
{युधिष्ठिरस्य सङ्ग्रामे कालवालान्हयोत्तमान्}


\twolineshloka
{ततोऽपरेण भल्लेन शिरस्त्राणमपातयत्}
{कौन्तेयस्य महेष्वासः प्रहसन्निव सूतजः}


\threelineshloka
{तथैव नकुलस्यापि हयान्हत्वा महारथः}
{ततोऽपरेण भल्लेन भृशं तीक्ष्णेन भारत}
{धनुश्चिच्छेद वीरस्य माद्रीपुत्रस्य धीमतः}


\twolineshloka
{तौ हताश्वौ हतरथौ पाण्डवौ भृशविक्षतौ}
{भ्रातरावारुरुहतुः सहदेवरथं तदा}


\twolineshloka
{तौ दृष्ट्वा मातुलस्तत्र विरथौ परवीरहा}
{अभ्यभाषत राधेयं मद्रराजोऽनुकम्पया}


\twolineshloka
{मार्गितव्यस्त्वया कर्ण कुन्तीपुत्रो धनञ्जयः}
{अतस्त्वं धर्मराजेन किमर्थमिह युध्यसे}


\threelineshloka
{[क्षीणशस्त्रास्त्राकवचः क्षीणबाणो विबाणधिः}
{श्रान्तसारथिवाहश्च च्छन्नोऽस्त्रैररिभिस्तथा}
{पार्थमासाद्य राधेय उपहास्यो भविष्यसि ॥]}


\twolineshloka
{तथापि कर्णः संरब्धो युधिष्ठिरमपीडयत्}
{शरैस्तीक्ष्णैर्महावीर्यैर्माद्रीपुत्रौ च पाण्डवौ}


\twolineshloka
{ततः शल्यः प्रहस्येदं कर्णं पुनरुवाच ह}
{रथस्थमतिसंरब्धं युधिष्ठिरवधे स्थितम्}


\threelineshloka
{यदर्थं धार्तराष्ट्रेण सततं मानितो भवान्}
{तं पार्थ जहि राधेय किं ते हत्वा युधिष्ठिरम्}
{`हते यस्मिन्ध्रुवं पार्थः सर्वाञ्जेष्यति नो रथान्}


\threelineshloka
{तस्मिन्हि धार्तराष्ट्रस्य निहते तु ध्रुवो जयः}
{ध्वजोऽसौ दृश्यते तस्य रोचमानोंऽशुमानिव}
{सारो ह्येष महाबाहो किं ते हत्वा युधिष्ठिरम् ॥'}


\twolineshloka
{शङ्खयोर्ध्मातयोः शब्दः सुमहानेष कृष्णयोः}
{श्रूयते चापघोषश्च प्रावृषीवाम्बुदस्य ह}


\twolineshloka
{असौ निघ्नन्रथोदारानर्जुनः शरवृष्टिभिः}
{सर्वां ग्रसति नः सेनां कर्ण पश्यैनमाहवे}


% Check verse!
पृष्ठरक्षौ च शूरस्य युधामन्यूत्तमौजसौ
\twolineshloka
{उत्तरं चास्य वै शूरश्चक्रं रक्षति सांत्यकिः}
{धृष्टद्युम्नस्तथा चास्य चक्रं रक्षति दक्षिणम्}


\threelineshloka
{भीमसेनश्च वै राज्ञा धार्तराष्ट्रेण युध्यते}
{यथा न हन्यात्तं भीमः सर्वेषां नोऽद्य पश्यताम्}
{कुरु राधेय वै राजा यथा मुच्येत तत्तथा}


\twolineshloka
{xxxx भीमसेनेन ग्रस्तमाहवशोभिना}
{यदि नामाद्य मुच्येत विस्मयः सुमहान्भवेत्}


\twolineshloka
{परित्राह्येनमभ्येत्य संशयं परमं गतम्}
{किं नु माद्रीसुतौ हत्वा राजानं वा युधिष्ठिरम्}


\twolineshloka
{इति शल्यवचः श्रुत्वा राधेयः पृथिवीपते}
{दृष्ट्वा दुर्योधनं चैव भीमग्रस्तं महाहवे}


\twolineshloka
{राजगृध्नुर्भृशं चैव शल्यवाक्यप्रचोदितः}
{अजातशत्रुमुत्सृज्य माद्रीपुत्रौ च पाण्डवौ}


\twolineshloka
{तव पुत्रं परित्रातुमभ्यधावत वीर्यवान्}
{मद्रराजप्रणुदितैरश्वैराकाशगैरिव}


\twolineshloka
{गते कर्णे तु कौन्तेयः पाण्डुपुत्रो युधिष्ठिरः}
{अपायाज्जवनैरश्वैः सहदेवस्य मारिष}


\twolineshloka
{[ताभ्यां स सहितस्तूर्णं व्रीडन्निव नरेश्वरः]}
{प्राप्य सेनानिवेशं स्वं मार्गणैर्भृशविक्षतः}


\twolineshloka
{अवतीर्य रथात्तूर्णमाविशच्छयनं शुभम्}
{अपेतशल्यो राजा तु हृच्छल्येनाभिपीडितः}


\twolineshloka
{सोऽब्रवीद्धातरौ राजा माद्रीपुत्रौ महारथौ}
{गच्छन्त त्वरितौ वीरौ यत्र भीमो व्यवस्थितः}


\threelineshloka
{`ततस्ते पाण्डवाः सर्वे समाभाष्य परस्परम्}
{अनीकं भीमसेनस्य पाण्डवावभ्यगच्छताम्'}
{जीमूत इव नर्दंस्तु युध्यते स वृकोदरः}


\twolineshloka
{ततोऽन्यं रथमास्थाय नकुलो रथपुङ्गवः}
{सहदेवश्च तेजस्वी भ्रातरौ शत्रुकर्शनौ}


\twolineshloka
{तुरगैरग्र्यरंहोभिर्यत्र भीमस्तरस्विनौ}
{अनीकैः सहितौ तत्र भ्रातरौ समवस्थितौ}


\chapter{अध्यायः ७१}
\twolineshloka
{सञ्जय उवाच}
{}


\twolineshloka
{द्रौणिस्तु रथवंशेन महता परिवारितः}
{अपतत्सहसा राजन्यत्र पार्थो व्यवस्थितः}


\twolineshloka
{तमापतन्तं सहसा शूरः शौरिसहायवान्}
{दधार समरे पार्थो वेलेव मकरालयम्}


\twolineshloka
{ततः क्रुद्धो महाराज द्रोणपुत्रः प्रतापवान्}
{अर्जुनं वासुदेवं च च्छादयामास सायकैः}


\twolineshloka
{अवच्छन्नौ ततः कृष्णौ दृष्ट्वा तत्र महारथाः}
{विस्मयं परमं गत्वा प्रैक्षन्त कुरवस्तदा}


\twolineshloka
{अर्जुनस्तु ततो दिव्यमस्त्रं चक्रे हसन्निव}
{तदस्त्रं वारयामास ब्राह्मणो युधि भारत}


\twolineshloka
{यद्यद्धि व्याक्षिपद्युद्धे फल्गुनोऽस्त्रं जिघांसया}
{तत्तदस्त्रैर्महेष्वासो द्रोणपुत्रो जघान ह}


\twolineshloka
{अस्त्रयुद्धे महाराज वर्तमाने महाभये}
{अपश्याम रणे द्रौणिं व्यात्ताननमिवान्तकम्}


\twolineshloka
{स दिशः प्रदिशश्चैव च्छादयित्वा ह्यजिह्मगैः}
{वासुदेवं त्रिभिर्बाणैरविध्यद्दक्षिणे भुजे}


\threelineshloka
{ततोऽर्जुनो रथान्सर्वान्हत्वा तस्य पदानुगान्}
{चकार समरे भूमिं शोणितौघपरिप्लुताम्}
{[सर्बलोकवहां रौद्रां परलोकवहां नदीम्]}


% Check verse!
सरथा रथिनः पेतुः पार्थचापच्युतैः शरैः
\twolineshloka
{[द्रौणेरपहतान्सङ्खे ददृशु स च तां तथा}
{प्रावर्तयन्महाघोरां नदीं परवहां तदा}


\twolineshloka
{तयोस्तु व्याकुले युद्धे द्रौणेः पार्थस्य दारुणे}
{अमर्यादं योधयन्तः पर्यधावन्त पृष्ठतः}


\twolineshloka
{रथैर्हताश्वसूतैश्च हतारोहैश्च वाजिभिः}
{द्विरदैश्च हतारोहैर्महामात्रैर्हतद्विपैः}


\twolineshloka
{पार्थेन समरे राजन्कृतो घोरो जनक्षयः}
{निहता रथिनः पेतुः पार्थचापच्युतैः शरैः ॥]}


\twolineshloka
{हयाश्च पर्यधावन्त मुक्तयोक्त्रास्ततस्ततः}
{तद्दृष्ट्वा कर्म पार्थस्य द्रौणिराहवशोभिनः}


\twolineshloka
{अर्जुनं जयतां श्रेष्ठं त्वरितोऽभ्येत्य वीर्यवान्}
{विधुन्वानो महच्चापं कार्तस्वरविभूषितम्}


\twolineshloka
{अवाकिरत्ततो द्रौमिः समन्तान्निशितैः शरैः}
{भूयोऽर्जुनं महाराज द्रौणिरायम्य पत्रिणा}


\twolineshloka
{वक्षोदेशे भृशं पार्थं ताडयामास निर्दयम्}
{सोऽतिविद्धो रणे तेन द्रोणपुत्रेण भारत}


\twolineshloka
{गाण्डीवधन्वा प्रसभं शरवर्षैरुदारधीः}
{सञ्छाद्य समरे द्रौणिं चिच्छेदास्य च कार्मुकम्}


\twolineshloka
{स च्छिन्नधन्वा परिधं वज्रस्पर्शसमं युधि}
{आदाय चिक्षेप तदा द्रोणपुत्रः किरीटिने}


\twolineshloka
{तमापन्ततं परिघं जाम्बूनदपरिष्कृतम्}
{चिच्छेद सहसा राजन्प्रहसन्निव पाण्डवः}


\twolineshloka
{स पपात तदा भूमौ निकृत्तः पार्थसायकैः}
{विकीर्णः पर्वतो राजन्यथा वज्रेण ताडितः}


\twolineshloka
{ततः क्रुद्धो महाराज द्रोणपुत्रो महारथः}
{ऐन्द्रेण चास्त्रवेगेन बीभत्सुं समवाकिरत्}


\twolineshloka
{तस्येन्द्रजालावततं समीक्ष्यपार्थो राजन्गाण्डिवमाददे सः}
{ऐन्द्रं जालं प्रत्यहरत्तरस्वीवरास्त्रमादाय महेन्द्रसृष्टम्}


\twolineshloka
{विदार्य तज्जालमथेन्द्रमुक्तंपार्थस्ततो द्रौणिरथं क्षणेन}
{प्रच्छादयामास ततोऽभ्युपेत्यद्रौणिस्तदा पार्थशराभिभूतः}


\twolineshloka
{विगाह्य तां पाण्डवबाणवृष्टिंशरैः परं नाम ततः प्रकाश्य}
{शतेन कृष्णं सहसाभ्यविद्ध्या--त्त्रिभिः शतैरर्जुनं क्षुद्रकाणाम्}


\twolineshloka
{ततोऽर्जुनः सायकानां शतेनगुरोः सुतं मर्मसु निर्बिभेद}
{अश्वांश्च सूतं च तथा धनुर्ज्या--मवाकिरत्पश्यतां तावकानाम्}


\twolineshloka
{स विद्ध्वा मर्मसु द्रौणिं पाण्डवः परवीरहा}
{सारथिं चास्य भल्लेन रथनीडादपातयत्}


\twolineshloka
{स सङ्गृह्य स्वयं वाहान्कृष्णौ प्राच्छदयच्छरैः}
{तत्राद्भुतमपरश्याम द्रौणेराशु पराक्रमम्}


\twolineshloka
{संयच्छंस्तुरगान्यच्च फल्गुनं चाप्ययोधयत्}
{तदस्य समरे राजन्सर्वभूतान्यपूजयन्}


\twolineshloka
{`तेनास्य समरे योधास्तुष्टुवुः सर्व एव ते}
{यदा निजगृहे पार्थो द्रोणपुत्रेण धन्विना}


\twolineshloka
{तदा प्रहस्य बीभस्तुर्द्रोणपुत्रस्य संयुगे}
{क्षिप्रं रश्मीनथाश्वानां क्षुरप्रैश्चिच्छिदे जयः}


\twolineshloka
{प्राद्रवंस्तुरगास्तस्य शरवेगप्रपीडिताः}
{ततोऽभून्निनदो घोरस्तव सैन्यस्य भारत}


\twolineshloka
{पाण्डवास्तु जयं लब्ध्वा तव सैन्यं समाद्रवन्}
{समन्तान्निशितान्बाणान्विमुञ्चन्तो जयैषिणः}


\twolineshloka
{तावकी चतुरङ्गैषा पाण्डवैस्तु महाचमूः}
{पुनःपुनरथो वीरैः प्रभग्ना जितकाशिभिः}


\twolineshloka
{पश्यतां ते महाराज पुत्राणां चित्रयोधिनाम्}
{शकुनेः सौबलेयस्य कर्णस्य च विशाम्पते}


\twolineshloka
{वार्यमाणा महासेना पुत्रैस्तव जनेश्वर}
{न चातिष्ठत सङ्ग्रामे पीड्यमाना समन्ततः}


\twolineshloka
{ततो योधैर्महाराज पलायद्भिः समन्ततः}
{अभवद्य्वाकुलं भीतं पुत्राणां ते महद्बलम्}


\twolineshloka
{तिष्ठतिष्ठेति सततं सूतपुत्रस्य जल्पतः}
{नावतिष्ठति सा सेना वध्यमाना महात्मभिः}


\twolineshloka
{अथोत्क्रुष्टं महाराज पाण्डवैर्जितकाशिभिः}
{धार्तराष्ट्रबलं दृष्ट्वा विद्रुतं वै समन्ततः}


\twolineshloka
{ततो दुर्योधनः कर्णमब्रवीत्प्रणयादिव}
{पश्य कर्ण महासेनां पाण्डवैरर्दितां भृशम्}


\twolineshloka
{त्वयि तिष्ठति सन्त्रासात्समन्तात्प्रपलायते}
{एतज्ज्ञत्वा महाबाहो कुरु प्राप्तमरिन्दम}


\threelineshloka
{सहस्राणि च योधानां त्वामेव पुरुषोत्तम}
{क्रोशन्ते समरे वीर द्राव्यमाणानि पाण्डवैः ॥सञ्जय उवाच}
{}


\twolineshloka
{एतच्छ्रुत्वा तु राधेयो दुर्योधनवचो महत्}
{मद्रराजमिदं वाक्यमब्रवीत्प्रहसन्निव}


\twolineshloka
{पश्य मे भुजयोर्वीर्यमस्त्राणां च जनेश्वर}
{अद्य हन्मि रणे सर्वान्पाञ्चालान्पाण्डुभिः सह}


\twolineshloka
{वाहयाश्वान्नरव्याघ्र मद्राधिप न संशयः}
{एवमुक्त्वा महाराज सूतपुत्रः प्रतापवान्}


\twolineshloka
{प्रगृह्य विजयं वीरो धनुःश्रेष्ठं पुरातनम्}
{सज्यं कृत्वा महाराज सम्मृज्य च पुनः पुनः}


\twolineshloka
{सन्निवार्य च तान्योधान्सत्येन शपथेन च}
{प्रायोजयदमेयात्मा भार्गवास्त्रं महाबलः}


\twolineshloka
{तस्माद्राजन्सहस्राणि प्रयुतान्यर्बुदानि च}
{कोटिशश्च शरास्तीक्ष्णा निरगच्छन्महामृधे}


\twolineshloka
{ज्वलितैस्तैः शरैर्घोरैः कङ्कबर्हिणवाजितैः}
{सञ्चन्ना पाण्डवी सेना न प्राज्ञायत किञ्चन}


\twolineshloka
{हाहाकारो महानासीत्पाण्डवानां विशाम्पते}
{पीडितानां बलवता भार्गवास्त्रेण संयुगे}


\twolineshloka
{निपतद्भिर्गजै राजन्नश्वैश्चापि सहस्रशः}
{रथैश्चापि नरव्याघ्र नरैश्चैव समन्ततः}


\twolineshloka
{प्राकम्पत मही राजन्निहतैस्तैः समन्ततः}
{व्याकुलं सर्वमभवत्पाण्डवानां महद्बलम्}


\twolineshloka
{कर्णस्त्वेको युधां श्रेष्ठो विधूम इव पावकः}
{दहञ्शत्रून्नरव्याघ्रः शुशुभे स परन्तपः}


\twolineshloka
{ते वध्यमानाः कर्णेन पाञ्चालाश्चेदिभिः सह}
{तत्र तत्र व्यमुह्यन्त वनदाहे यथा द्विपाः}


\twolineshloka
{चुक्रुशुश्च नरव्याघ्र यथाप्राणं नरोत्तमाः}
{तेषां तु क्रोशतां दीनं भीतानां रणमूर्धनि}


\twolineshloka
{धावतां च ततो राजंस्त्रस्तानां च समन्ततः}
{आर्तनादो बभूवात्र भूतानामिव सम्प्लुवे}


% Check verse!
वध्यमानांस्तु तान्दृष्ट्वा सूतपुत्रेण मारिष ॥वित्रेसुः सर्वभूतानि तिर्यग्योनिगतान्यपि
\threelineshloka
{ते वध्यमाताः समरे सूतपुत्रेण सृञ्जयाः}
{अर्जुनं वासुदेवं च क्रोशन्ति च मुहुर्मुहुः}
{प्रेतराजपुरे यद्वत्प्रेतराजं विचेतसः}


\threelineshloka
{श्रुत्वा तु निनदं तेषां वध्यतां कर्णसायकैः}
{अथाब्रवीद्वासुदेवं कुन्तीपुत्रो धऩञ्जयः}
{भार्गवास्त्रं महाघोरं दृष्ट्वा तत्र समीरितम्}


\twolineshloka
{पश्य कृष्ण महाबाहो भार्गवास्त्रस्य विक्रमम्}
{नैतदस्त्रं हि समरे शक्यं हन्तुं कथञ्चन}


\twolineshloka
{सूतपुत्रं च संरब्धं पश्य कृष्ण महारणे}
{अन्तकप्रतिमं वीर्ये कुर्वाणं कर्म दारुणम्}


\twolineshloka
{अभीक्ष्णं चोदयन्नश्वान्प्रेक्षते मां मुहुर्मुहुः}
{नच शक्ष्यामि कर्णस्य संयुगेऽहं पलायितुम्}


\twolineshloka
{जीवन्प्राप्नोति पुरुषः सङ्ख्ये जयपराजयौ}
{मृतस्य तु हृपीकेश भङ्ग एव कुतो जयः}


\twolineshloka
{एष कर्णो रणे भाति मध्याह्न इव भास्करः}
{निवर्तय रथं कृष्ण जीवन्भद्राणि पश्यति}


\chapter{अध्यायः ७२}
\twolineshloka
{सञ्जय उवाच}
{}


\twolineshloka
{ततो जनार्दनः श्रुत्वा तस्य वाक्यं विशाम्पते}
{रथेनापययौ क्षिप्रं सङ्ग्रामादिति भारत}


\twolineshloka
{प्रत्यनीकमवस्थाप्य भीमं भीमपराक्रमम्}
{अलं विषहितुं ह्येष कुरूणां सम्प्रपश्यताम्}


\threelineshloka
{कर्णं च समरे राजन्ग्राहयिष्यञ्श्रमं प्रति}
{विश्वमार्थं च कौरव्य पार्थस्य च महात्मनः}
{अपयातो रणाद्वीरो राजानं द्रष्टुमेव च}


\twolineshloka
{अर्जुनं चाब्रवीत्कृष्णो भृशं राजा परिक्षतः}
{तमाश्वास्य कुरुश्रेष्ठ ततः कर्णं हनिष्यसि}


\threelineshloka
{ततो धनञ्जयः श्रुत्वा राजानं शल्यपीडितम्}
{याहियाहिति बहुशो वासुदेवमचोदयत्}
{राजानं प्रति वार्ष्णेय दूयते मे दृढं मनः}


\twolineshloka
{स चोद्यमानः पार्थेन केशिघ्नो वृष्णिनन्दनः}
{रथेनापययौ क्षिप्रं सङ्ग्रामाद्धोरदर्शनात्}


\twolineshloka
{गच्छन्नेव तु कौन्तेयो धर्मराजदिदृक्षया}
{सैन्यमालोकयामास नापश्यत्तत्र चाग्रजम्}


\twolineshloka
{युद्धं कृत्वा तु कौन्तेयो द्रोणपुत्रेण भारत}
{दुःसहं वज्रिणा सङ्ख्ये पराजित्य गुरोः सुतम्}


\twolineshloka
{द्रौणिं पराजित्य ततोग्रधन्वाकृत्वा महद्दुष्करं शूरकर्म}
{आलोकयामास ततः स्वसैन्यंधनञ्जयः शत्रुभिरप्रधृष्यः}


\twolineshloka
{संयुध्यमानान्पृतनामुखस्था--ञ्शूरः शूरान्हर्षयन्सव्यसाची}
{पूर्वापदानैः प्रथितैः प्रशंस--न्स्थितांश्चकारात्मरथाननीके}


\threelineshloka
{अपश्यमानस्तु किरीटमालीयुधिष्ठिरं भ्रातरमाजमीढम्}
{उवाच भीमं तरसाभ्युपेत्यराज्ञः प्रवृत्तिस्त्विह कुत्र चेति ॥भीमसेन उवाच}
{}


\threelineshloka
{अपयात इतो राजा धर्मपुत्रो युधिष्ठिरः}
{कर्णबाणाभितप्ताङ्गो यदि जीवेत्कथञ्चन ॥अर्जुन उवाच}
{}


\twolineshloka
{तस्माद्भवाञ्शीघ्रमितः प्रयातुज्ञातुं प्रवृत्तिं पुरुषर्षभस्य}
{नूनं स विद्धोऽतिभृशं पृषत्कैःकर्णेन राजा शिबिरं गतोऽसौ}


\twolineshloka
{यः सम्प्रहारैर्युधि सम्प्रवृत्तेद्रोणेन विद्धोऽतिभृशं तरस्वी}
{तस्थौ स तत्रापि जयप्रतीक्षोद्रोणेन यावन्न हतः किलासीत्}


\threelineshloka
{स संशयं गमितः पाण्डवाग्र्यःसङ्ख्येऽद्य कर्णेन महानुभावः}
{ज्ञातुं प्रयाह्याशु तमद्य भीमस्थास्याम्यहं शत्रुगणान्निरुध्य ॥भीमसेन उवाच}
{}


\twolineshloka
{त्वमेव जानीहि महानुभावराज्ञः प्रवृत्तिं भरतर्षभस्य}
{अहं हि यद्यर्जुन याम्यमित्रावदन्ति मां भीत इति प्रवीराः}


\twolineshloka
{ततोऽब्रवीदर्जुनो भीमसेनंसंशप्तकाः प्रत्यनीकस्थिता मे}
{एतानहत्वाद्य मया न शक्य--मितोपयातुं रिपुसङ्खमध्यात्}


\twolineshloka
{अथाब्रवीदर्जुनं भीमसेनःस्ववीर्यमासाद्य कुरुप्रवीर}
{संशप्तकान्प्रतियोत्स्यामि सङ्ख्येसर्वानहं याहि धनञ्जय त्वम्}


\threelineshloka
{तद्भीमसेनस्य वचो निशम्यधनञ्जयो भ्रातुरमित्रमध्ये}
{द्रष्टुं कुरुश्रेष्ठमभिप्रयास्य--न्प्रोवाच वृष्णिप्रवरं तदानीम् ॥अर्जुन उवाच}
{}


\fourlineindentedshloka
{`राजानं प्रति वार्ष्णेय दूयते मे दृढं मनः'}
{चोदयाश्वान्हृषीकेश विहायैतद्बलार्णवम्}
{अजातशत्रुं राजानं द्रष्टुमिच्छामि केशव ॥सञ्जय उवाच}
{}


% Check verse!
`तं रथ चोदयामास बीभत्सोर्वचनाद्धरिः'
\twolineshloka
{ततो हयान्सर्वदाशार्हमुख्यःप्रचोदयन्भीममुवाच चेदम्}
{नैतच्चित्रं तव कर्माद्य भीमयास्यावहे जहि पार्थारिसङ्घान्}


\twolineshloka
{ततो ययौ हृषीकेशो यत्र राजा युधिष्ठिरः}
{शीघ्राच्छीघ्रतरो राजन्वाजिभिर्गरुडोपमैः}


\twolineshloka
{प्रत्यनीके त्ववस्थाप्य भीमसेनमरिन्दमम्}
{सन्दिश्य चैनं राजेन्द्र युद्धं प्रति वृकोदरम्}


\twolineshloka
{ततस्तु गत्वा पुरुषप्रवीरौराजानमासाद्य शयानमेकम्}
{रथादुभौ प्रत्यवरुह्य तस्मा--द्ववन्दतुर्धर्मराजस्य पादौ}


\threelineshloka
{`इत्येवमुपसङ्गृह्य उभौ तु पाञ्जली स्थितौ}
{शस्त्रक्षतौ महाराज रुधिरेण समुक्षितौ}
{निहत्य वाहिनीं तुभ्यमपयातौ रणाजिरात्'}


\twolineshloka
{तं दृष्ट्वा पुरुषव्याघ्रं क्षेमिणं पुरुषर्षभम्}
{मुदाभ्यनन्दतां कृष्णावश्विनाविव वासवम्}


\twolineshloka
{तावभ्यनन्दद्राजापि विवस्वानश्विनाविव}
{हते महासुरे जम्भे शक्रविष्णू यथा गुरुः}


\threelineshloka
{मन्यमानो हतं कर्णं प्रीतः परपुरञ्जयः}
{`स भूत्वा पुरुषव्याघ्रो राजा तत्र युधिष्ठिरः'}
{हर्षगद्गदया वाचा प्रीतः प्राह परन्तपः}


\chapter{अध्यायः ७३}
\twolineshloka
{सञ्जय उवाच}
{}


\twolineshloka
{[अथोपयातौ पृथुलोहिताक्षौशराचिताङ्गौ रुधिरप्रदिग्धौ}
{समीक्ष्य सेनाग्रनरप्रवीरौयुधिष्ठिरो वाक्यमिदं बभाषे ॥]}


\twolineshloka
{ततो युधिष्ठिरो दृष्ट्वा सहितौ केशवार्जुनौ}
{हतमाधिरथिं मेने सङ्ख्ये गाण्डीवधन्वना}


\threelineshloka
{तावभ्यनन्दत्कौन्तेयः साम्ना परमवल्गुना}
{स्मितपूर्वममित्रघ्न पूजयन्भरतर्पभ ॥युधिष्ठिर उवाच}
{}


\twolineshloka
{स्वागतं देवकीपुत्र स्वागतं ते धनञ्जय}
{प्रियं मे दर्शनं वाढं युवयोर्नरसिंहयोः}


\twolineshloka
{अरिष्टयोरक्षतयोः कर्णं हत्वा महारथम्}
{आशीविपसमं युद्धे सर्वशस्त्रविशारदम्}


\twolineshloka
{अग्रगं धार्तराष्ट्राणां सर्वेषां शर्म वर्म च}
{रक्षितं वृषसेनेन सुपेणेन च धन्विना}


\twolineshloka
{अनुज्ञातं महावीर्यं रामेणास्त्रे सुदुर्जयम्}
{अग्र्यं सर्वस्य लोकस्य रथिनं लोकविश्रुतम्}


\twolineshloka
{त्रातारं धार्तराष्ट्राणां गन्तारं वाहिनीमुखे}
{हन्तारं परसैन्यानाममित्रगणमर्दनम्}


\twolineshloka
{दुर्योधनहिते युक्तमस्मद्दुःखाय चोद्यतम्}
{अप्रधृष्यं महायुद्धे देवैरपि सवासवैः}


\twolineshloka
{अनलानिलयोस्तुल्यं तेजसा च बलेन च}
{पातालमिव गम्भीरं सुहृदां नन्दिवर्धनम्}


\twolineshloka
{अन्तकं मम मित्राणां हत्वा कर्णं महामृधे}
{दिष्ट्या युवामनुप्राप्तौ हत्वाऽसुरमिवामरौ}


\threelineshloka
{तेन युद्धमभीतेन मयाऽद्याप्यच्युतार्जुनौ}
{कृतं घोरं महाबाहू धृष्टद्युम्नस्य पश्यतः}
{अन्तकेनेव क्रुद्धेन प्रजाः सर्वा जिघांसता}


\twolineshloka
{तेन केतुश्च मे च्छिन्नो हतौ च पार्ष्णिसारथी}
{हतवाहः कृतश्चास्मि युयुधानस्य पश्यतः}


\twolineshloka
{धृष्टद्युम्नस्य यमयोर्वीरस्य च शिखण्डिनः}
{पश्यतां द्रौपदेयानां शतद्रोश्च महात्मनः}


\twolineshloka
{एताञ्जित्वा महावीर्यः कर्णः शत्रुगणान्बहून्}
{जितवान्मां महाबाहो यतमानो महारणे}


\twolineshloka
{अनुसृत्य च मां युद्धे परुषाण्युक्तवान्बहु}
{तत्र तत्र युधां श्रेष्ठ परिभूय न संशयः}


\twolineshloka
{भीमसेनप्रभावात्तु यज्जीवामि धनञ्जय}
{बहुनात्र किमुक्तेन हते सोढुं समुत्सहे}


\twolineshloka
{त्रयोदशाहं वर्षाणि यस्माद्भीतो धनञ्जय}
{न स्म निद्रां लभे रात्रौ न चाहनि सुखं क्वचित्}


\twolineshloka
{तस्य द्वेषेण संयुक्तः परिदह्ये धनञ्जय}
{आत्मनो मरणं जानन्वाध्रीणस इव द्विजः}


\twolineshloka
{तस्यायमगमत्कालश्चिन्तयानस्य मे चिरम्}
{कथं कर्णो मया शक्यो युद्धे क्षपयितुं भवेत्}


\twolineshloka
{जाग्रत्स्वपंश्च कौन्तेय कर्णमेव स्मराह्यहम्}
{पश्यामि तत्रतत्रैव कर्णभूतमिदं जगत्}


\twolineshloka
{यत्र यत्र हि गच्छामि कर्णाद्भीतो धनञ्जय}
{तत्र तत्र हि पश्यामि कर्णमेवाग्रतः स्थितम्}


\twolineshloka
{सोऽहं तेनैव वीरेण समरेष्वपलायिना}
{सहयः सरथः पार्थ जित्वा जीवन्विसर्जितः}


\twolineshloka
{को नु मे जीवितेनार्थो राज्येनार्थो भवेत्पुनः}
{ममैवं विक्षतस्याद्य कर्णेनाहवशोभिना}


\twolineshloka
{न प्राप्तपूर्वं यद्भीष्मात्कृपाद्द्रोणाच्च संयुगे}
{तत्प्राप्तमद्य मे युद्धे सूतपुत्रान्महारथात्}


\twolineshloka
{स त्वां पृच्छामि कौन्तेय यथा न कुशलस्तथा}
{तन्ममाचक्ष्व कार्त्स्न्येन यथा कर्णो हतस्त्वया}


\twolineshloka
{शक्रतुल्यबलो युद्धे यमतुल्यः पराक्रमे}
{रामतुल्यस्तथास्त्रेषु स कथं ते निषूदितः}


\twolineshloka
{महारथसमाख्यातः सर्वयुद्धविशारदः}
{धनुर्धराणां प्रवरः सर्वेषां वै सदार्जुन}


\twolineshloka
{पूजितो धृतराष्ट्रेण सपुत्रेण विशाम्पते}
{त्वदर्थमेव राधेयः स कथं निहतस्त्वया}


\twolineshloka
{धार्तराष्ट्रस्य योद्धारः सर्व एव सदार्जुन}
{तव मृत्युं रणे कर्णं मन्यन्ते पुरुषर्षभ}


\twolineshloka
{स त्वया पुरुषव्याघ्र कथं युद्धे निषूदितः}
{तन्ममाचक्ष्व कौन्तेय यथा कर्णो हतस्त्वया}


\twolineshloka
{सोत्सेधमस्य च शिरः पश्यतां सुहृदां हृतम्}
{त्वया पुरुषशार्दूल सिंहेनेव यथा रुरोः}


\twolineshloka
{सर्वा दिशः पर्यपतत्त्वदर्थंत्वां सूतपुत्रः समरे परीप्सन्}
{दित्सन्नृणां शकटं रत्नपूर्णंकथं त्वयासौ निहतोऽद्य कर्णः}


% Check verse!
त्वया रणे निहतः सूतपुत्रः कच्चित्कृतं मे परमं त्वयाद्यप्रियं रणे सूतपुत्रं निहत्य
\twolineshloka
{सर्वा दिशः पर्यपतत्त्वदर्थंमुदान्वितो गर्वितः सूतपुत्रः}
{स शूरमानी समरे समेत्यकच्चित्त्वया निहतः सूतपुत्रः}


\twolineshloka
{राज्यं परं हस्तिगवाश्वयुक्तंरथं दित्सुर्यः परेभ्यस्त्वर्दर्थे}
{त्वया रणे स्पर्धते यः स पापःकच्चित्त्वया निहतः पापबुद्धिः}


\twolineshloka
{योऽसौ नित्यं शौर्यमदेन मत्तोविकत्थते संसदि कौरवाणाम्}
{प्रीत्यर्थं वै तात सुयोधनस्यकच्चित्स पापो निहतस्त्वयाद्य}


\twolineshloka
{कच्चित्समागम्य धनुःप्रमुक्तै--स्त्वत्प्रेषितैर्लोहमयैर्विहंगैः}
{शेते भिन्नः पांसुषु सूतपुत्रःकच्चिद्भग्रौ धार्तराष्ट्रस्य बाहू}


\twolineshloka
{योऽसौ सदा श्लाघते राजमध्येदुर्योधनं हर्षयन्दर्पयुक्तः}
{हन्ताऽस्मि सर्वानिति पाण्डुपुत्रा--नहं हन्ता फल्गुनस्येति मोहात्}


\twolineshloka
{कच्चिद्वचोऽस्य वितथं त्वया कृतंयत्तत्प्रियामवदत्तात कर्णः}
{सभामथ्ये रूक्षमनेकरूपंधिक्पाण्डवानपतिस्त्वं हि कृष्णे}


\twolineshloka
{कच्चिद्भुवं शत्रुरयं महात्माह्यधारयद्द्वादश यः समास्तु}
{कर्णो व्रतं घोरममित्रसाहोदुर्योधनस्यार्थनिविष्टबुद्धिः}


\twolineshloka
{पादौ न धावे यावदहं न हन्मिधनञ्जयं समरेषूग्रवेगम्}
{कच्चिद्रणे फल्गुन तं निहत्यकच्चिद्व्रतं तस्य भग्नं त्वयाऽद्य}


\twolineshloka
{योऽसौ कृष्णामब्रवीद्दुष्टबुद्धिःकर्णः सभायां कुरुवीरमध्ये}
{किं पाण्डवांस्त्वं न जहासि कृष्णेसुदुर्बलान्पतितान्हीनसत्वान्}


\twolineshloka
{योऽसौ कर्णः प्रत्यजानात्त्वदर्थेनाहं हत्वा सह कृष्णेन पार्थम्}
{इहोपयातेति स पापबुद्धिःकच्चिच्छेते शरसम्भिन्नगात्रः}


\twolineshloka
{कच्चित्सङ्ग्रामो विदितो वै तवायंसमागमे सृञ्जय कौरवाणाम्}
{येनावस्थामीदृशीं प्रापितोऽहंकच्चित्त्वया सोऽद्य हतो दुरात्मा}


\twolineshloka
{कच्चित्त्वया तस्य सुमन्दबुद्धे--र्गाण्डीवमुक्तैर्विशिखैर्ज्वलद्भिः}
{सकुण्डलं भानुमदुत्तमाङ्गंकायात्प्रकृत्तं युधि सव्यसाचिन्}


\twolineshloka
{यत्तन्मया बाणसमर्पितेनध्यातोऽसि कर्णस्य वधाय वीर}
{तन्मे त्वया कच्चिदमोघमद्यध्यानं कृतं कर्णनिपातनेन}


\twolineshloka
{यद्दर्पपूर्णः स सुयोधनोऽस्मा--न्दिधक्षते कर्णसमाश्रयेण}
{कच्चित्त्वया सोऽद्य समाश्रयोऽस्यभग्नः पराक्रम्य सुयोधनस्य}


\twolineshloka
{यो नः पुरा षण्ढतिलानवोच--त्सभामध्ये कौरवाणां समक्षम्}
{स दुर्मतिः कच्चिदुपेत्य सङ्ख्येत्वया हतः सूतपुत्रोऽत्यमर्षी}


\twolineshloka
{यः प्राहिणोत्सूतपुत्रो दुरात्माकृष्णां जितां सौबलेहानयेति}
{स मन्दबुद्धिर्निहतः प्रसह्यवैकर्तनस्त्वद्य कच्चिन्महात्मन्}


\twolineshloka
{यः शस्त्रभृच्छ्रेष्ठतमं पृथिव्यांपितामहं व्याक्षिपदल्पचेताः}
{सङ्ख्यायमानोऽर्धरथः स कच्चि--त्त्वयां हतोऽद्याधिरथिर्महात्मन्}


\twolineshloka
{अमर्षजं निकृतिसमीरणेरितंहृदि स्थितं ज्वलनमिमं सदा मम}
{हतो मया सोऽद्य समेत्य कर्णइति ब्रुवन्प्रशमयसेऽद्य फल्गुन}


\chapter{अध्यायः ७४}
\twolineshloka
{सञ्जय उवाच}
{}


\twolineshloka
{तद्धर्मशीलस्य वचो निशम्यराज्ञः क्रुद्धस्यातिरथो महात्मा}
{उवाच दुर्मर्षमदीनसत्वोयुधिष्ठिरं जिष्णुरनन्तवीर्यः}


\twolineshloka
{द्रोणं हतं पार्थ कर्णो विदित्वाभिन्नां नावविमावत्यगाधे कुरूणाम्}
{सम्मुह्यमानान्धार्तराष्ट्रान्विदित्वानिरुत्साहांश्च विजये परेषाम्}


\twolineshloka
{सोदर्यवत्त्वरितोऽमितौजाउत्तारयिष्यन्धृतराष्ट्रस्य पुत्रान्}
{रणे रथेनाधिरथिर्महात्माततो हि मां त्वरितः सोऽभ्यधावत्}


\twolineshloka
{संशप्तकैर्युध्यमानस्य मेऽद्यसेनाग्रयायी कुरुसैन्यस्य राजन्}
{आशीविषाभान्विकिरञ्शरौघा--न्द्रौणिः पुरस्तात्सहसाध्यतिष्ठत्}


\twolineshloka
{स मे दृष्ट्वा शूरतमो ध्वजाग्रंसमादिशद्रथसङ्घाननेकान्}
{तेषामहं पञ्चशतं निहत्यआसादयं द्रोणपुत्रं नदन्तम्}


\twolineshloka
{स द्रोणपुत्रः सदृशं महात्मामामप्यरौत्सीत्तदनीकमध्ये}
{किरञ्शरौघान्बहुरूपान्विचित्रा--न्स्वातीगतः शुक्र इवातिवर्षन्}


\twolineshloka
{स मे रान्सर्वतः कङ्कपत्रा--नवासृजद्वै पृथिवीप्रकाशान्}
{निवार्य तूर्णं परमाजिमध्येततोऽपि मां बाणगणैः समार्पयत्}


\twolineshloka
{आकर्षणं वापि विकृष्य मुक्तंन दृश्यते तस्य महारथस्य}
{न सन्दधानः कुत आददानोन व्याक्षिपन्न विकर्षन्विमुञ्चन्}


\twolineshloka
{सव्येन वा यदि वा दक्षिणेनन ज्ञायते कतरेणास्यतीति}
{आचार्यवत्समरे पर्यवर्त--न्महच्चित्रं दर्शयन्सर्वतः स्म}


\twolineshloka
{दृष्ट्वीविषं चासुहरं परेषांसर्वा दिशः पूरयानं शरौधैः}
{अलातचक्रप्रतिमं महात्मनःसदा नतं कार्मुकं ब्रह्मबन्धोः}


\twolineshloka
{ततोऽपरान्बाणगणाननेका--नाकर्णपूर्णायतविप्रमुक्तान्}
{ससर्ज शीघ्रास्त्रबलेन वीर--स्तोयं यथा प्रावृषि कालमेघः}


\twolineshloka
{आविध्यन्मां पञ्चभिद्रोणपुत्रःशितैः शरैः पञ्चभिर्वासुदेवम्}
{अभ्यघ्नं बाणैस्तमहं सुधारै--र्निमेषमात्रेण सुवर्णपुङ्खैः}


\twolineshloka
{स निर्विद्धो विरथो द्रोणपुत्रोरथानीकं चाधिरथेर्विवेश}
{मयाभिभूतान्स्वरथप्रबर्हा--नस्त्रं च पश्यन्रुधिरप्रदिग्धम्}


\twolineshloka
{ततोऽभिभूतं युधि वीक्ष्य सैन्यंविध्वस्तयोधं द्रुतनागयूथम्}
{पञ्चाशता रथमुख्यैः समेतःकर्णस्त्वरन्नभ्यपतत्प्रमाथी}


\twolineshloka
{तान्सूदयित्वाऽहमपास्य कर्णंद्रष्टुं भवन्तं त्वरयोपयातः}
{सर्वे पाञ्चाला ह्युद्विजन्ते स्म कर्णंदृष्ट्वा गावः केसरिणं यथैव}


\twolineshloka
{मृत्योरास्यं व्यात्तमिवाभिपद्यप्रभद्रकाः कर्णमासाद्य सर्वे}
{यास्यामि तांस्तारयिष्यन्बलौघा--द्दिष्ट्या भवान्स्वस्तिमान्पार्थ दृष्टः}


\twolineshloka
{हनिष्येऽहं भारत सूतपुत्र--मस्मिन्सङ्ग्रामे यदि दृश्यतेऽद्य}
{आयाहि पश्याद्य युयुत्समानौमां सूतपुत्रं च धृतौ रणाय}


\twolineshloka
{महाझषस्येव मुखं प्रपन्नाःप्रभद्रकाः कर्णमुखं प्रपन्नाः}
{षट््साहस्रा भारत राजपुत्राःस्वर्गाय लोकाय रणे निमग्नाः}


% Check verse!
तानद्य यास्यामि रणाद्विमोक्तुंसर्वात्मना सूतपुत्रं च हन्तुम्
\twolineshloka
{अथ प्रवीरेण महानुभावद्विषत्सैन्यं निर्दहन्विस्तरेण}
{समेत्याहं सूतपुत्रेण सङ्ख्ये वज्रीव वृत्रेण नरेन्द्रमुख्य}


\twolineshloka
{एवं गते किञ्च मयाऽद्य शक्यंकार्यं कर्तुं निग्रहे सूतजस्य}
{तथैव राज्ञश्च सुयोधनस्यये चापि मां योद्धुकामाः समेताः}


\twolineshloka
{कर्णं न चेदद्य निहन्मि राज--न्सबान्धवं युध्यमानं प्रसह्य}
{प्रतिश्रुत्याऽकुर्वतां या गतिर्वैकष्टां यायां तामहं राजसिंह}


\twolineshloka
{आमन्त्रये त्वां ब्रूहि रणे जयं मेपुरा भीमं धार्तराष्ट्रा ग्रसन्ति}
{सौतिं हनिष्यामि नरेन्द्रसिंहसैन्यं तथा शत्रुगणांश्च सर्वान्}


\chapter{अध्यायः ७५}
\twolineshloka
{सञ्जय उवाच}
{}


\twolineshloka
{वैकर्तनं कुशलिनं निशम्यक्रुद्धः पार्थः फल्गूनस्यामितौजाः}
{धनञ्जयं वाक्यमुवाच राजायुधिष्ठिरः कर्णशराभितप्तः}


\twolineshloka
{इमां च वसतिं हित्वा भयात्कर्णेन फल्गुन}
{अहं च भीमसेनश्च माद्रीपुत्रौ च पाण्डवौ}


\threelineshloka
{वासुदेवेन सहिता वयं कर्णेन निर्जिताः}
{पुनरेव वनं गत्वा तपश्चर्यां च कुर्महे}
{अथवा धार्तराष्ट्राणां परिचर्यां चरामहे}


\twolineshloka
{इत्येवमुक्त्वा बीभत्सुं रोषात्संरक्तलोचनः}
{अब्रवीत्पुनरेवात्र धर्मराजो युधिष्ठिरः}


\twolineshloka
{इदं यदि द्वैतवने तु सूक्तंकर्णं नियोद्धुं न सहे नृपेति}
{वयं ततः प्राप्तकालं यथाव--त्कृत्वाऽभ्युपेक्षाम तथैव राज्यम्}


\twolineshloka
{मयि प्रतिश्रुत्य वधं हि तस्यबलस्य चार्धस्य तथैव युद्धे}
{आनीय मां शत्रुमध्ये स कस्मा--त्समुन्नाम्य स्थण्डिले प्रत्यपिंष्ठाः}


\twolineshloka
{अन्वास्य सत्येन यदात्थ पार्थसत्यं शपन्वासुदेवेन सार्धम्}
{तन्नः सत्यमफलं ह्यकार्षीःफलस्य काले चाच्छिन्नः पुष्पवृक्षम्}


\twolineshloka
{प्रच्छादितस्त्वं बालिश दुर्यशोभि--रनर्थवाक्योऽस्यर्जुन नैव साधुः}
{त्यक्त्वा भीमं सर्वभीमेषु भीमंसंयोजितस्त्वं सूतपुत्रं निहन्तुम्}


\twolineshloka
{यत्तद्वृथा वागुवाचान्तरिक्षेसप्ताह्नि जाते त्वयि मन्दबुद्धौ}
{अपापीयान्वासवात्कुन्तिजातोबहून्सङ्ग्रामानयमेव जेता}


\twolineshloka
{अयं जेता पाण्डवो देवसङ्घान्सर्वाणि भूतान्यपि चोत्तमौजाः}
{अयं जेता मद्रकलिङ्गजाता--न्दैत्यांश्च रक्षांसि समागतानि}


\twolineshloka
{भूमिं च सर्वां निखिलेन जेताकुरूंश्च जेता स्वगणांश्च जेता}
{अस्मात्परो नो भविता धनुर्धरोनैष्यन्न भूतः कश्चिदेनं विजेता}


\twolineshloka
{इच्छन्नयं सर्वभूतानि कुर्या--द्वशी वशे सर्वसमाप्तविद्यः}
{कान्त्या शशाङ्कस्य जवेन वायोःस्यैर्येण मेरोः क्षमया पृथिव्याः}


\twolineshloka
{अग्नेश्च तापे धनदस्य लक्ष्म्याशौर्येण शक्रस्य जयेन विष्णोः}
{तुल्यो महात्मा तव कुन्ति पुत्रोजातोऽदितेर्विष्णुरिवामितौजाः}


\twolineshloka
{स्वेषां जयाय द्विषतां वधायजातो महात्मा तव नन्दिकर्ता}
{इत्यन्तरिक्षे शतशृङ्गमूर्ध्नितपस्विनां शृण्वतां वागुवाच}


% Check verse!
एवंविधस्त्वं न च भूतस्तथाद्यदेवाश्च नूनमनृतं वदन्ति
\twolineshloka
{तथा परेषामृषिसत्तमानांश्रुत्वा गिरः पूजयतां सदा त्वाम्}
{न सन्नतिं यामि सुयोधनस्यन त्वां जानाम्याधिरथेर्भयार्तम्}


\threelineshloka
{त्वष्ट्रा कृतं वाहमकूजनाक्षंशुभं समास्थाय कपिध्वजं तम्}
{खङ्गं गृहीत्वा हेमचित्रावनद्वंधनुर्वरं गाण्डिवं तालमात्रम्}
{त्वं वासुदेवेनोह्यमानः कथं नुकर्णाद्भीतो व्यपयातोऽसि पार्थ}


\twolineshloka
{पूर्वं यदुक्तं हि सुयोधनेनन फल्गुनः प्रमुखे स्थास्यतीति}
{कर्णस्य युद्धे हि महाबलस्यमौर्ख्यात्तु तन्नावबुद्धं मयाऽसीत्}


\twolineshloka
{तेनाद्य तप्स्ये भृशमप्रमेयंयन्मित्रवर्गो नरकं प्रविष्टः}
{तदैव वाच्योऽस्मि ननु त्वयाऽहंन योत्स्येऽहं सूतपुत्रं कथञ्चित्}


% Check verse!
ततो नाहं सृञ्जयान्केकयांश्चसमानयेयं सुहृदो रणाय
\twolineshloka
{एवं गते किं च मयाऽद्य शक्यंकार्यं कर्तुं निग्रहे सूतजस्य}
{तथैव राज्ञश्च सुयोधनस्यये वापि मां योद्धुकामाः समेताः}


\twolineshloka
{धिगस्तु मज्जीवितमद्य कृष्णयोऽहं वशं सूतपुत्रस्य यातः}
{मध्ये कुरूणां समुयोधनानांये चाप्यन्ये योद्धुकामाः समेताः}


\twolineshloka
{एकः स मे भीमसेनोऽद्यं नाथोयेनाभिपन्नोऽस्मि रणे महाभये}
{विमोच्य मां चापि रुपान्वितस्ततःशरेण तीक्ष्णेन बिभेद कर्णम्}


\twolineshloka
{त्यक्त्वा प्राणान्समरे भीमसेन--श्चक्रे युद्धं कुरुमुख्यैः समेतैः}
{गदाग्रहस्तो रुधिरोक्षिताङ्ग--श्चरन्रणे काल इवान्तकाले}


\twolineshloka
{असौ हि भीमस्य महानिनादोमुहुर्महुः श्रूयते धार्तराष्ट्रैः}
{यदि स्म जीवेत्समरे निहन्तामहारथानां प्रवरो नरोत्तमः}


\twolineshloka
{तवाभिमन्युस्तनयोऽद्य पार्थन चास्मि गन्ता समरे पराभवम्}
{अथापि जीवेत्समरे घटोत्कच--स्तथाऽपि नाहं समरे पराङ्युखः}


\twolineshloka
{भीमस्य पुत्रः समराग्रयायीमहास्त्रविन्नापि तवानुरूपः}
{यं तं xxxसाद्य रिपोर्वलं नोनिमीलिताक्षं भयविप्लुतं भवेत्}


% Check verse!
xxxxxनिशि युद्धमेक--स्त्यक्सा रणं यत्व भयाद्रवन्ते
\twolineshloka
{स चेत्समासाद्य महानुभावःकर्णं रणे बाणगणैः प्रमोह्यः}
{धैर्ये स्थितेनापि च सूतजेनशक्या हतो वासवदत्तया तया}


\threelineshloka
{ममैव भाग्यानि पुरा कृतानिपापानि नूनं फलवन्ति युद्धे}
{तृणं च कृत्वा समरे भवन्तंततोऽहमेवं निकृतो दुरात्मना}
{वैकर्तनेनैव तथा कृतोऽहंयथा ह्यशक्तः क्रियते त्वबान्धवः}


\twolineshloka
{आपद्ग्रतं यश्च नरो विमोक्षये--त्स बान्धवः स्नेहयुतः सुहृच्च}
{एवं पुराणा ऋषयो वदन्तिधर्मः सदा सद्भिरनुष्ठितश्च}


\twolineshloka
{खङ्गं विभ्रज्जातरूपत्सरुं चधनुर्वेदं गाण्डिवं तालमात्रम्}
{स केशवेनोह्यमानः कथं नुकर्णात्पार्थस्त्वमपैतुं समैषीः}


\twolineshloka
{स गाण्डीवं केशवाय प्रदाययन्ता भवेस्त्वं यदि केशवस्य}
{ततस्तरेत्केशवः कर्णमुग्रंमरुत्सस्वो वृत्रमिवात्तवज्रः}


\twolineshloka
{मासेऽपतिष्यो यदि पञ्चमे त्वंन वा गर्भो यद्यभवः पृथायाः}
{मत्तः श्रेयान्राजपुत्रोऽभविष्य--न्न ते यशः फल्गुन इत्यपेयात्}


\chapter{अध्यायः ७६}
\twolineshloka
{सञ्जय उवाच}
{}


\twolineshloka
{युधिष्ठिरेणैxxxxक्तः कौन्तेयः श्वेतवाहनः}
{असिं जग्राह सङ्क्रुद्धो जिघांसुर्भरतर्षभम्}


\twolineshloka
{तस्य कोपं समाज्ञाय चित्तज्ञः केशवस्तदा}
{उवाच किमिदं खङ्गं गृह्णास्यर्जुन शंम मे}


\twolineshloka
{नेह पश्यामि योद्धव्यं त्वया किञ्चिद्वनञ्जय}
{नेहागता धार्तराष्ट्राः सर्वे भीमेन वारिताः}


\twolineshloka
{उपागतोऽसि कौन्तेय राजा द्रष्टव्य इत्यसौ}
{स राजा भवता दृष्टः कुशली च युधिष्ठिरः}


\twolineshloka
{तं दृष्ट्वा नृपशार्दूलं शार्दूलसमविक्रमम्}
{हर्षकाले च सम्प्राप्ते कस्मात्त्वां मन्युराविशत्}


\twolineshloka
{न तं पश्यामि कौन्तेय यस्ते वध्यो भवेदिह}
{प्रहर्तुमिच्छसे कस्मात्किं वा ते चित्तविभ्रमः}


\twolineshloka
{कस्माद्भवान्महाखङ्गं परिगृह्णात्यकारणात्}
{}


\twolineshloka
{तत्त्वां पृच्छामि कौन्तेय किमिदं ते चिकीर्षितम्}
{परामृशसि यत्क्रुद्धः खङ्गमद्भुतविक्रम}


\twolineshloka
{एवमुक्तः कटाक्षेण प्रेक्षमाणो युधिष्ठिरम्}
{अर्जुनः प्राह गोविन्दं क्रुद्धः सर्प इव श्वसन्}


\threelineshloka
{अन्यस्मै देहि गाण्डीवमिति मां यः प्रचोदयेत्}
{भिन्द्यामहं तस्य शिर इत्युपांशु व्रतं मम}
{युधिष्ठिरेण तेनाहमुक्तश्चास्मि जनार्दन}


\twolineshloka
{यदुक्तोऽहमदीनार्थं राज्ञाऽनेन यशस्विना}
{समक्षं तव गोविन्द न तत्क्षन्तुमिहोत्सहे}


\twolineshloka
{तस्मादेनं विधिष्यामि राजानं धर्मचारिणाम्}
{प्रतिज्ञां पालयिष्यामि हत्वैनं नरसत्तमम्}


\threelineshloka
{एतदर्थं मया खह्गो गृहीतो यदुनन्दन}
{सोऽहं युधिष्ठिरं हत्वा सत्यस्यानृण्यतां गतः}
{विशोको विज्वरश्चापि भविष्यामि जनार्दन}


\twolineshloka
{किं वा त्वं मन्यसे प्राप्तमस्मिन्कार्य उपस्थिते}
{त्वमस्य जगतस्तात वेत्थ सर्वं गतागतम्}


\threelineshloka
{`जातस्त्वत्तो हि धर्मश्चाधर्मश्चेति परा श्रुतिः'}
{तत्तथा प्रकरिष्यामि यथा मां वक्ष्यते भवान् ॥सञ्जय उवाच}
{}


\twolineshloka
{धिग्धिगित्येव गोविन्दः पार्थमुक्त्वाब्रवीत्पुनः ॥कृष्ण उवाच}
{}


\twolineshloka
{इदानीं पार्थ जानामि न वृद्धाः सेवितास्त्वया}
{अकाले पुरुषव्याघ्र संरम्भं यद्भवानगात्}


\twolineshloka
{न हि धर्मविगज्ञैरेवं कार्यं धनञ्जय}
{यथा त्वं पाण्डवाद्येह धर्मभीरुरण्डितः}


\twolineshloka
{योऽकार्याणां क्रियायाश्च संयोगं नावबुध्यते}
{कार्याणामक्रियायाश्च स पार्थ पुरुषाधमः}


\twolineshloka
{अनुसृत्य तु ये धर्मं कवयः समुपस्थिताः}
{समासविस्तरविदां न तेषां वेत्सि निश्चयम्}


\twolineshloka
{अनिश्चयज्ञो हि नरः कार्याकार्यविनिश्चये}
{एवं स मुह्यत्यवशो यथा त्वं पार्थ मुह्यसि}


\twolineshloka
{न हि कार्यमकार्यं वा सुखं ज्ञातुं कथञ्चन}
{श्रुतेन ज्ञायते सर्वं तच्च त्वं नावहुध्यसे}


\twolineshloka
{अविज्ञानादिह भवान्सत्यं रक्षति धर्मवित्}
{प्राणिनां त्वं वधं पार्थ धार्मिको नावबुध्यसे}


\twolineshloka
{प्राणिनां हि वधात्तात वृथा धर्मो मतो मम}
{अनृतं तु भवेद्वाच्यं न तु हिंसा कदाचन}


\twolineshloka
{स कथं भ्रातरं ज्येष्ठं राजानं धर्मकोविदम्}
{हन्याद्भवान्नरश्रेष्ठ प्राकृतोऽन्यः पुमानिव}


\twolineshloka
{अयुध्यमानस्य गुरोस्तथाऽशत्रोश्च मानद}
{पराङ्मुखस्य द्रवतः शरणं चापि गच्छतः}


\twolineshloka
{कृताञ्जलेः प्रपन्नस्य प्रमत्तस्य तथैव च}
{न वधः पूज्यते सद्भिस्तच्च सर्वं गुरौ तव}


\twolineshloka
{त्वया चैतद्व्रतं पार्थ बालेनेव कृतं पुरा}
{तस्मादधर्मसंयुक्तं मौर्ख्यादेव व्यवस्यसि}


\twolineshloka
{स्वगुरुं पार्थ कस्मात्त्वं हन्तुकामोऽभिधावसि}
{असम्प्रधार्य धर्माणां गतिं सूक्ष्मां दुरत्ययाम्}


\twolineshloka
{इदं धर्मरहस्यं च तव वक्ष्यामि पाण्डव}
{यद्ब्रूयात्तव भीष्मो वा राजा वापि युधिष्ठिरः}


\threelineshloka
{विदुरो वा पुनः क्षत्ता गान्धारी वा यशस्विनी}
{कुन्ती वा भरतश्रेष्ठ द्रौपदी वा यशस्विनी}
{तत्ते वक्ष्यामि तत्त्वेन निबोधैतद्धनञ्जय}


\twolineshloka
{सत्यस्यं वचनं साधु न सत्याद्विद्यते परम्}
{सुदुर्विदं हि तत्त्वेन तत्सत्यमिति मे सतिः}


\twolineshloka
{भवेत्सत्यमवक्तव्यं वक्तव्यमनृतं भवेत्}
{यत्रानृतं भवेत्सत्यं सत्यं चाप्यनृतं भवेत्}


\twolineshloka
{विवाहकाले रतिसम्प्रयोगेप्राणात्यये सर्वत्रनापहारे}
{विप्रस्य चार्थे ह्यनृतं वदेतपञ्चानृतान्याहुरपातकानि}


% Check verse!
सर्वस्वस्यापहारे तु वक्तव्यमनृतं भवेत् ॥तत्रानृतं भवेत्सत्यं सत्यं चाप्यनृतं भवेत्
\twolineshloka
{तादृशो हन्यते बालो यस्य सत्यमनिश्चितम्}
{सत्यानृते विनिश्चित्य ततो भवति धर्मवित्}


\twolineshloka
{किमाश्चर्यं कृतप्रज्ञः पुरुषोऽपि सुदारुणः}
{सुमहत्प्राप्नुयात्पुण्यं बलाकोऽन्धवधादिव}


\threelineshloka
{किमाश्चर्यं पुनर्मूढो धर्मकामो ह्यपण्डितः}
{सुमहत्प्राप्नुयात्पापमापगास्विव कौशिकः ॥अर्जुन उवाच}
{}


\threelineshloka
{आचक्ष्व भगवन्नेतद्यथा विन्दाम्यहं तथा}
{वलाकस्यानुसम्बद्धं नदीनां कौशिकस्य च ॥वासुदेव उवाच}
{}


\twolineshloka
{मृगव्याधोऽभवत्कश्चिद्बलाको नाम भारत}
{यात्रार्थं पुत्रदाराणां मृगान्हन्ति न कामतः}


\twolineshloka
{वृद्धौ च मातापितरौ बिभर्त्यन्यांश्च संश्रितान्}
{स्वधर्मनिरतो नित्यं संविभज्यानसूयकः}


\twolineshloka
{स कदाचिन्मृगप्रेप्सुर्नान्वविन्दन्मृगं क्वचित्}
{अथापश्यत्स पीवानं श्वापदं घ्राणचक्षुषम्}


\twolineshloka
{अदृष्टपूर्वमपि तत्सत्वं तेन हतं तदा}
{अन्धे हते ततो व्योम्नः पुष्पवर्षं पपात च}


\twolineshloka
{अप्सरोगीतवादित्रैर्नादितं च मनोरमम्}
{विमानमगमत्स्वर्गान्मृगव्याधनिनीषया}


\twolineshloka
{तद्भूतं सर्वभूतानामभावाय किलार्जुन}
{तपस्तप्त्वा वरं प्राप्तं कृतमन्धं स्वयम्भुवा}


\twolineshloka
{तद्धत्वा सर्वभूतानामभावकृतनिश्चयम्}
{बलाकोऽगात्स्वर्गलोकमेवं धर्मः सुदुर्विदः}


\twolineshloka
{कौशिकोऽप्यभवद्विप्रस्तपस्वीनो बहुश्रुतः}
{नदीनां सङ्गमे ग्रामाददूरे स किलावसत्}


\twolineshloka
{सत्यं मया सदा वाच्यमिति तस्याभवद्द्वतम्}
{सत्यवादीति विख्यातः स तदासीद्धनञ्जय}


\twolineshloka
{अथ दस्युभयात्केचित्तदा तद्वनमाविशन्}
{तत्रापि दस्यवः क्रुद्धास्तानमार्गन्त यत्नतः}


\fourlineindentedshloka
{अथ कौशिकमभ्येत्य प्रोचुस्ते सत्यवादिनम्}
{कतरेण पथा याता भगवन्निति वै जनाः}
{सत्येन पृष्टः प्रब्रूहि यदि तद्वेत्थ शंस नः ॥श्रीकृष्ण उवाच}
{}


\threelineshloka
{सत्यस्य त्वविभागज्ञः सत्यं तेभ्यः शशंस ह}
{बहुवृक्षलतागुल्ममेतद्गहनमाश्रिताः}
{इति तान्ख्यापयामास तेभ्यस्तत्त्वं स कौशिकः}


\twolineshloka
{ततस्ते तान्समासाद्य क्रूरा जघ्नुरिति श्रुतिः}
{ततोऽधर्मेण महता वाग्दुरुक्तेन कौशिकः}


\twolineshloka
{गतः सुकष्टं निरयं धर्मसूक्ष्मेष्वतत्त्ववित्}
{दृष्टपूर्वश्रुतो मूढो धर्माणामविशारदः}


% Check verse!
वृद्धानपृच्छन्सन्देहानन्धः श्वभ्रमिवर्च्छति
\twolineshloka
{तत्र ते लक्षणोद्देशः कश्चिदेव भविष्यति}
{दुष्करं प्रतिसंख्यानं कार्त्स्न्येनात्र व्यवस्थितिः}


\twolineshloka
{सत्यं धर्म इति ह्येके वदन्ति बहवो जनाः}
{न च पार्थाभ्यसूयामि नैतत्सर्वत्र शिष्यते}


\twolineshloka
{श्रुतिस्तु धार्या इत्येके वदन्ति बहवो जनाः}
{न त्वेतत्प्रत्यसूयामि तत्र सर्वं विधीयते}


\threelineshloka
{यत्स्यादहिंसासंयुक्तं स धर्म इति निश्चयः}
{अहिंसार्थाय हिंस्राणां धर्मप्रवचनं कृतम्}
{धारणाद्धर्ममित्याहुर्धर्मो धारयते प्रजाः}


\twolineshloka
{प्रभवार्थाय भूतानां धर्मप्रवचनं कृतम्}
{यस्मात्प्रभवसंयुक्तः स धर्म इति निश्चयः}


\twolineshloka
{येऽन्यायेन जिगीषन्तो धर्मं पृच्छन्ति मानवाः}
{अकूजनेन चेन्मोक्षो नात्र कूजेत्कथञ्चन}


\threelineshloka
{अवश्यं कूजितव्ये ह शङ्केरन्वाप्यकूजनात्}
{येऽन्यायेन जिहीर्षन्तो धर्मं पृच्छन्ति कस्यचित्}
{श्रेयस्तत्रानृतं वक्तुं सत्यादिति विनिश्चित्म्}


\twolineshloka
{प्राणात्यये विवाहे वा सर्वजात्या महाभये}
{सर्वस्वस्य च लोपे वा वक्तव्यमनृतं भवेत्}


\twolineshloka
{अधर्मं हि न पश्यन्ति मृषोद्यं तत्र पण्डिताः}
{सर्वथाऽभिवदेत्तत्तु नानृतं स्याद्विचक्षणः}


\twolineshloka
{यः स्तेनैः सह सम्बन्धो मुच्यते शपथादपि}
{भवेत्तत्रानृतं श्रेयः सत्यादिति विचारितम्}


\threelineshloka
{न च तेभ्यो धनं देयं सत्यादिति कथञ्चन}
{पापेभ्योऽपि धनं दत्तं दातारमपि पीडयेत्}
{तस्माद्धर्मार्थमनृतमुक्त््वा नानृतवाग्भवेत्}


\threelineshloka
{एष ते लक्षणोद्देशो मयोद्दिष्टो यथाविधि}
{एतज्ज्ञात्वा ब्रूहि पार्थ यदि वध्यो युधिष्ठिरः ॥अर्जुन उवाच}
{}


% Check verse!
यथा ब्रूयान्महाप्राज्ञो यथा ब्रूयान्महायशाः
\threelineshloka
{सुहृद्ब्रूयाद्यथाऽस्माकं तथोक्तं वचनं त्वया}
{भवान्मातृसमोऽस्माकं भवान्पितृसमोऽपि च}
{गतिश्च परमा कृष्ण त्वमेव च परायणम्}


\twolineshloka
{न हि ते त्रिषु लोकेषु विद्यतेऽविदितं क्वचित्}
{तस्माद्भवान्परं धर्मं वेद सर्वं यथातथम्}


\twolineshloka
{अवध्यं पाण्डवं मन्ये धर्मराजं युधिष्ठिरम्}
{अधर्मयुक्ते संयोगे ब्रूहि किञ्चिदनुग्रहम्}


% Check verse!
इदं चापरमत्रैव ब्रूहि तत्त्वं विवक्षितम्
\twolineshloka
{जानासि दाशार्ह मम व्रतं तुयो मां ब्रूयात्कश्चन मानुषेषु}
{अन्यस्मै त्वं गाण्डिवं देहि पार्थयो मत्तोऽस्त्रे वीर्यतो वा विशिष्टः}


\twolineshloka
{हन्यामहं केशव तं प्रसह्यभीमो हन्यात्तूवरकेति चोक्तः}
{वन्मां राजा ह्युक्तवांस्ते समक्षंधनुर्देहीत्यसकृद्वृष्णिवीरे}


\twolineshloka
{तं हन्यां चेत्केशव जीवलोकेस्थाता नाहं कालमप्यल्पमात्रम्}
{ध्यात्वा नूनं ह्येनसा चापि मुक्तोवधं राज्ञो भ्रष्टवीर्यो विचेताः}


\threelineshloka
{यथा प्रतिज्ञा मम लोकबुद्धौभवेत्सत्या धर्मभृतां वरिष्ठ}
{यथा जीवत्पाण्डवोऽहं च कृष्णतथा बुद्धिं दातुमप्यर्हसि त्वम् ॥वासुदेव उवाच}
{}


\twolineshloka
{राजा श्रान्तो विक्षतो दुःखितश्चकर्णेन सह्ख्ये निशितैर्बाणसङ्घैः}
{यश्चानिशं सूतपुत्रेण वीरशरैर्भृशं ताडितो युध्यमानः}


\twolineshloka
{अतस्त्वमेतेन सरोषमुक्तोदुःखान्वितेनेदमयुक्तरूपम्}
{अकोपितो ह्येष यदि स्म सङ्ख्येकर्णं न हन्यादिति चाब्रवीत्सः}


\twolineshloka
{जानाति तं पाण्डव एष चापिपापं लोके कर्णमसह्यमन्यैः}
{ततस्त्वमुक्तो भृशरोषितेनराज्ञा समक्षं परुषाणि पार्थ}


\twolineshloka
{नित्योद्युक्ते सततं चाप्रसह्येकर्णे द्यूतं ह्यद्य रणे निबद्धम्}
{तस्मिन्हते कुरवो निर्जिताः स्यु--रेवं बुद्धिः पार्थिवे धर्मपुत्रे}


\twolineshloka
{ततो बधं नार्हति धर्मपुत्र--स्त्वया प्रतिज्ञाऽर्जुन पालनीया}
{जीवन्नयं येन मृतो भवेद्धितन्मे निबोधेह तवानुरूपम्}


\twolineshloka
{यदा मानं लभते माननार्ह--स्तदा स वै जीवति जीवलोके}
{यदाऽवमानं लभते महान्तंतदा जीवन्मृत इत्युच्यते सः}


\twolineshloka
{सम्मानितः पार्थिवोऽयं सदैवत्वया च भीमेन तथा यमाभ्याम्}
{वृद्धैश्च लोके पुरुषैश्च शूरै--स्तस्यापमानं कलया प्रयुङ्क्ष्व}


\twolineshloka
{त्वमित्यत्रभवन्तं हि ब्रूहि पार्थ युधिष्ठिरम्}
{त्वमित्युक्तो हि निहतो गुरुर्भवति भारत}


\twolineshloka
{एवमाचर कौन्तेय धर्मराजे युधिष्ठिरे}
{अधर्मयुक्तं संयोगं कुरुष्वैनं कुरूद्वह}


\twolineshloka
{अथर्वाङ्गिरसी ह्येषा श्रुतीनामुत्तमा श्रुतिः}
{अविचार्यैव कार्यैषा श्रेयस्कामैर्नरैः सदा}


\twolineshloka
{अवधेन वधः प्रोक्तो यद्गुरुस्त्वमिति प्रभुः}
{तद्ब्रूहि त्वं यन्मयोक्तं धर्मराजस्य धर्मवित्}


\twolineshloka
{यदा ह्यं पाण्डव धर्मराज--सत्वत्तोऽयुक्तं लप्स्यते चैव साधु}
{ततोऽस्य पादावभिवाद्य पश्चा--च्छ्रेयो ब्रूयात्सान्त्वयुक्तं हितं च}


\twolineshloka
{भ्राता प्राज्ञस्तव कोपं न जातुकुर्याद्राजा धर्ममार्गानुसारी}
{मुक्तोऽनृताद्वातृवधाच्च पापा--द्धृष्टः कर्णं त्वं जहि पार्थ पश्चात्}


\chapter{अध्यायः ७७}
\twolineshloka
{सञ्जय उवाच}
{}


\twolineshloka
{इत्येवमुक्तस्तु जनार्दनेनपार्थः प्रशस्याथ सुहृद्वचस्तत्}
{ततोऽब्रवीदर्जुनो धर्मराज--मनुक्तपूर्वं परुषं प्रसह्य}


\twolineshloka
{मा त्वं राजन्व्याहरस्वाद्य पापंत्वं तिष्ठसि क्रोशमात्रे व्यपेत्य}
{भीमस्तु मामर्हति गर्हणाययो युध्यते सर्वलोकप्रवीरैः}


\twolineshloka
{काले हि शत्रून्परिपीड्य वीरोहत्वा प्रवीरान्वसुधाधिपानाम्}
{रथप्रधानोत्तमनागमुख्यायुधिप्रवीरा निहताश्च शूराः}


\twolineshloka
{सुदुष्करं कर्म करोति भीमःकर्तुं यथा नार्हति कश्चिदन्यः}
{रथादवस्कन्द्य गदां परामृशं--स्तथा रणे हन्ति तथैव वारणान्}


\twolineshloka
{स कुञ्जराणामधिकं सहस्रंहत्वा नदंस्तुमुलं सिंहनादम्}
{काम्भोजवानायुजपार्वतीया--नीहामृगाभान्विनिहत्य वाजिनः}


\twolineshloka
{महारथान्दविरदाञ्शैलकल्पा--न्सहेत यः कुञ्जरान्वध्यमानान्}
{असौ भीमो धार्तराष्ट्रेषु मग्नःस मामुपालब्धुमरिन्दमोऽर्हति}


\twolineshloka
{वरासिना चाथ नराश्वकुञ्जरां--स्तथा रथाङ्गैर्धनुषा च हन्त्यरीन्}
{प्रमृद्य पद्ध्यामहितांस्तु हन्तिपुनश्च दोर्भ्यां शतमन्युविक्रमः}


\twolineshloka
{महाबलो वैश्रवणान्तकोपमःप्रसह्य हर्ता द्विषतां यशांसि}
{स भीमसेनोऽर्हति गर्हणायन त्वं नित्यं रक्ष्यमाणः सुहृद्भिः}


\twolineshloka
{कलिङ्गवङ्गाङ्गनिपादमागधा--न्सदा महाशैलवलाहकोपमान्}
{निहन्ति यः शत्रुगणाननेकशःस मां हि वक्तुं प्रभवत्यमानगसम्}


\twolineshloka
{संयुक्तमास्थाय रथं हि कालेधनुर्विधून्वञ्छरपूर्णमुष्टिः}
{सृजेच्च यो बाणसङ्घान्परेषुमहाबलो मेघ इवाम्बुधाराः}


\twolineshloka
{शतान्यष्टौ वारणानामदर्शय--द्विशातितैः कुम्भघटाग्रहस्तैः}
{यो भीमसेनो निहतारिसङ्घःस मामुपालब्धुमरिन्दमोऽर्हति}


\twolineshloka
{रथाश्च नागाश्च हयाश्च राज--न्भीमेनाजौ निहताः सङ्घशोऽद्य}
{राजानश्च बहवो महाबलाःस मामुपालब्धुमरिन्दमोऽर्हति}


\twolineshloka
{धृतराष्ट्रपुत्रा बलिनश्च येनमहाबला निहताः प्रायशो वै}
{शूरो युद्धे ह्यप्रतिवार्यवीर्यःस मामुपालब्धुमरिन्दमोऽर्हति}


\twolineshloka
{प्रतापयंस्तद्बलमुग्ररूपंयोऽसौ रमे धार्तराष्ट्रस्य वीरः}
{एकः सहेताप्रतिसह्यपौरुष--स्तेनास्मि वाच्यो न त्वया वै कदाचित्}


\twolineshloka
{महारथा यत्र यत्रैव युद्धेभिन्दन्ति सैन्यं तव कामतोऽद्य}
{तत्रैव तत्रैव रणे महात्मादृढं भीमः परसङ्घानमृद्रात्}


\twolineshloka
{तेनास्मि वाच्यो न त्वया हं कदाचि--न्मा मा वोचः क्रूरमिहाद्य पार्थ}
{नास्मद्विधो वै भवता तु वाच्योयथा भवान्सर्वलोकस्य वाच्यः}


\twolineshloka
{एवं हि मा ते ब्रुवतो नरेन्द्रकथं न दीर्येच्छतधाऽद्य जिह्वा}
{अहो बतेदं सुनृशंसरूपंकामादवोचस्त्वमिहाद्य यद्वै}


\twolineshloka
{बलं न वाधिष्ठितं सत्तमानांयत्क्षत्रियाणां बहुलं वदन्ति}
{त्वं चाबलो भारत निष्ठुरोऽसित्वमेव मां वेत्सि यथाविधोऽहम्}


\twolineshloka
{नकुलेन राजन्गजवाजियोधाहताश्च वीराः सहसा समेत्य}
{त्यक्त्वा प्राणान्समरे युद्धकाङ्क्षीस मामुपालब्धुमरिन्दमोऽर्हति}


\twolineshloka
{कृतं कर्म सहदेवेन दुष्करंयो युध्यते परसैन्यावमर्दी}
{न चाब्रवीत्किञ्चिदिहागतो बलीपश्यान्तरं तस्य चेवात्मनश्च}


\twolineshloka
{धृष्टद्युम्नः सात्यकिर्द्रौपदेयायुधामन्युश्चोत्तमौजाः शिखण्डी}
{एतेऽद्य युधि सम्प्रपीडिता--स्ते मामुलपालब्धुमर्हन्ति न त्वम्}


\twolineshloka
{त्वन्मूलमस्माभिरिदं हि वैरंप्राप्तं तथा व्यसनं चातिघोरम्}
{द्यूप्रमत्तेन कृतं त्वयाऽसकृ--त्कस्मादुपालब्धुमिहार्हसि त्वम्}


\twolineshloka
{त्वमेव राजन्सततं प्रमत्त--स्त्वमेव मूढो भारतानामसाधुः}
{त्वां प्राप्य राज्यं च विनष्टमेत--त्प्राप्ता महत्पाण्डवाश्चापि दास्यम्}


\twolineshloka
{त्वत्तः कृतोऽस्मकद्वनवासदुःखंराज्यस्य नाशो ह्यभिमन्योश्च घोरः}
{आत्मानमेवं सुनृशंसरूपंज्ञात्वा किमर्थं गर्हसे माद्य वीर}


\twolineshloka
{लज्जस्व राजन्यदि तेऽस्ति लज्जातूष्णीम्भूतः पश्य सर्वं कृतघ्नः}
{भीमो नित्यं समरस्य कर्तादर्पस्य भेत्ता पुनरेव नित्यम्}


\twolineshloka
{स्वयं ह्यशक्तेन नरेन्द्र युद्धेनरेण कार्या सततं क्षमैव}
{बलं हि वाचि द्विजसत्तमानांक्षात्रं द्विजा बाहुबलं वदन्ति}


\twolineshloka
{त्वं वाग्बलो भारत निष्ठुरोऽसित्वमेव मां वेत्सि यथाविधोऽहम्}
{घटामि नित्यं तव कर्तुमिष्टंदारैः सुतैर्जीवितेनात्मना च}


\twolineshloka
{एवञ्च मां वाक्छलाकैर्हिनत्सित्वत्तः सुखं न वयं विद्म किञ्चित्}
{मा मामवंस्था द्रौपदीतल्पकसंस्थोमहारथान्प्रतिहन्मि त्वदर्थे}


\twolineshloka
{सर्वातिशङ्की भवसि प्रमत्त--स्त्वत्तः सुखं नाभिजानामि किञ्चित्}
{प्रोक्तः स्वयं सत्यसन्धेन मृत्यु--स्तव प्रियार्थं नरदेव युद्धे}


\twolineshloka
{शिखण्डिनाम्ना प्रधने तवार्थेमयाभिगुप्तेन हतश्च भीष्मः}
{द्रोणो हतो यः सततोपकारीधृष्टद्युम्नेन स्यन्दनाद्विप्रकृष्टः}


\twolineshloka
{द्रौणिश्च रुद्धः सगणो महात्मातथापि ते वै वचनं नृशंसवत्}
{दुःखं प्रियं ते नरदेव कर्तुंयस्य प्रियं ते न करोम्यहं वै}


\twolineshloka
{न युच्यते वै दिवि चेह यः पुमा--न्यस्ते मदन्योऽप्रियमारभेत}
{न चाभिनन्दामि तथाहि राज्यंयतस्त्वमक्षेषु दृढं प्रसक्तः}


% Check verse!
स्वयं कृतं पापमनार्यजुष्ट--मस्माभिराजौ व्यसनं तितीर्षसि ॥अक्षेषु दोषा बहवो विधर्म्याःश्रुतास्त्वया सहदेवोऽब्रवीद्यान्
\twolineshloka
{तान्नेच्छसि त्यक्तुमनार्यजुष्टा--न्घोरे स्म सर्वे निरये त्वयाऽस्ताः}
{त्वं देविता त्वत्कृते राज्यनाशं--स्त्वत्सम्भवं व्यसनं नो नरेन्द्र}


\twolineshloka
{मास्मान्क्रूरैर्षाक्प्रतोदैस्तुदस्त्वंभूयो राजन्कोपयस्यल्पबुद्ध्या ॥सञ्जय उवाच}
{}


\twolineshloka
{एता वाचः परुषाः सव्यसाचीस्थिरप्रतिज्ञः श्रावयित्वा नरेन्द्रम्}
{विनिः श्वसञ्ज्येष्ठमनिष्टमुक्त्वाततस्तु कोशादसिमुद्वबर्ह}


\twolineshloka
{तमाह कृष्णः किमिदं पुनर्भवा--न्विकोशमाकाशनिभं करोत्यसिम्}
{प्रब्रूहि सत्यं पुनरुत्तरं सखेवचः प्रवक्ष्याभि तवार्थसिद्धये}


\twolineshloka
{इतीव पृष्टः पुरुषोत्तमेनसुदुःखितः केशवमाह पार्थः}
{अहं हनिष्ये स्वशरीरमेत--त्प्रसह्य येनाहितमुक्तवान्गुरुम्}


\twolineshloka
{निशम्य तत्पार्थवचोऽब्रवीदिदंजनार्दनो धर्मभृतां वरिष्ठः}
{प्रब्रूहि पार्थ स्वगुणानिहात्मन--स्ततो हतात्मा भवसीति निश्चयः}


\threelineshloka
{तथा तु कृष्णस्य वचो निशम्यततोऽर्जुनः प्राह धनुः प्रगृह्य}
{युधिष्ठिरं धर्मभृतां वरिष्ठंशृणुष्व राजन्निति दुर्वचः स्वयम् ॥अर्जुन उवाच}
{}


\twolineshloka
{न मादृशोऽन्यो नरदेव विद्यतेधनुर्धरो देवमृते पिनाकिनम्}
{अहं हि तेनानुमतो महात्मना क्षणेन हन्यां सचराचरं जगत्}


\twolineshloka
{मया हि राजन्सदिगीश्वरा दिशोविजित्य सर्वा भवतः कृता वशे}
{स राजसूयश्च समाप्तदक्षिणःसभा च दिव्या भवतो ममौजसा}


\twolineshloka
{पाणी पृषत्कालिखिताविमौ पुन--र्धनुश्च सव्ये विततं सबाणम्}
{पादौ च मे लक्षणतः प्रशस्तौन मादृशं युद्धगतं जयन्ति}


\twolineshloka
{हता उदीच्या निहताः प्रतीच्याःप्राच्या निरस्ता दाक्षिणात्या विशस्ताः}
{संशप्तकानां किञ्चिदेवावशिष्टंसर्वस्य लोकस्य हतं मयाऽर्धम्}


\twolineshloka
{शेते मया निहता शत्रुसेनाछिन्नैर्गात्रैर्भूमितले स्खलन्ती}
{अनस्त्रज्ञान्नैव निहन्मि चास्त्रै--स्तस्मान्न भस्मैव करोमि लोकान्}


\twolineshloka
{जैत्रं रथं भीममास्थाय कृष्णयावच्छीघ्रं सूतपुत्रं निहन्तुम्}
{राजा भवत्वद्य सुनिर्वृतोऽयंकर्णं रणे नाशयितास्मि बाणैः}


\fourlineindentedshloka
{इत्येवमुक्त्वा पुनराह पार्थोयुधिष्ठिरं धर्मभृतां वरिष्ठम्}
{अद्यापुत्रा सूतमाता भवित्रीकुन्ती वाथो वा मया तेन वापि}
{सत्यं वदाम्यद्य न कर्णमाजौशरैरहत्वा कवचं विमोक्ष्ये ॥सञ्जय उवाच}
{}


\twolineshloka
{इत्येवमुक्त्वा पुनरेव पार्थोयुधिष्ठिरं धर्मभृतां वरिष्ठम्}
{विमुच्य शस्त्राणि धनुर्विसृज्यकोशे च खङ्गं विनिधाय तूर्णम्}


\twolineshloka
{स व्रीडया नम्रशिराः किरीटीयुधिष्ठिरं प्राञ्जलिरभ्युवाच}
{प्रसीद राजन्क्षम यन्मयोक्तंकाले भवान्वेत्स्यति तन्नमस्ते}


\twolineshloka
{ततस्तु पादावुपगृह्य पार्थःसमुत्थितो दीप्ततेजाः किरीटि}
{प्रसाद्य राजानममित्रसाहंस्थितोऽब्रवीच्चैनमभिप्रतप्तम्}


\twolineshloka
{याम्येष भीमं समराद्विमोक्तुंसर्वात्मना सूतपुत्रं च हन्तुम्}
{भवत्प्रियार्थं मम जीवितं हिब्रवीमि सत्यं तदवेहि राजन्}


\twolineshloka
{नेदं चिरात्क्षिप्रमिदं भविष्य--दावर्ततेऽसावभियामि चैनम्}
{अद्याप्यपुत्रा तेन हतेन राधाकुन्ती मया वा तदिदं विद्धि राजन्}


\chapter{अध्यायः ७८}
\twolineshloka
{सञ्जय उवाच}
{}


\twolineshloka
{एतच्छ्रुत्वा पाण्डवो धर्मराजोभ्रातुर्वाक्यं परुषं फल्गुनस्य}
{उत्थाय तस्माच्छयनादुवाचपार्थं ततो दुःखपरीतचेताः}


\twolineshloka
{कृतं मया पार्थं न साधुकर्मयेन प्राप्तं व्यसनं वः सुघोरम्}
{तस्माच्छिरश्छिन्धि ममैतदद्यकुलान्तकस्याधमपूरुषस्य}


\twolineshloka
{पापस्य पाप्मोहतस्य वीरविमूढबुद्धेरलसस्य भीरोः}
{वृद्धावमन्सुः परुषत्य चैवकिं ते चिरं मामनुसृज्य रूक्षम्}


\twolineshloka
{गच्छाम्यहं वनमद्यैव पापःसुखं भवान्वर्ततां मद्विहीनः}
{योग्यो राजा भीमसेनो महात्माक्लीबस्य किं वा मम राज्यकृत्यम्}


\twolineshloka
{न चापि शक्तः परुषाणि सोढुंपुनस्तवेमानि रुषान्वितस्य}
{भीमोऽस्तु राजा मम जीवितेनन कार्यमद्यावमतस्य वीर}


\twolineshloka
{इत्येवमुक्त्वा सहसोत्पपातरुषान्वितस्तच्छयनं विहाय}
{इयेष निर्गन्तुमथो वनायतं वासुदेवः प्रणतोऽभ्युवाच}


\twolineshloka
{न राजन्विदितं तत्ते यथा गाण्डीवधन्वनः}
{प्रतिज्ञा सत्यसन्धस्य गाण्डीवं प्रति विश्रुता}


\twolineshloka
{ब्रूयाद्य एनं गाण्डीवं देह्यन्यस्मै त्वमित्युत}
{वध्योऽस्य स पुमाँल्लोके त्वया चोक्तोयमीदृशम्}


\twolineshloka
{ततः सत्यां प्रतिज्ञां तां पार्थेन प्रतिरक्षता}
{मच्छन्दादवमानोऽयं कृतस्तव महीपते}


\twolineshloka
{गुरूणामवमानो हि वध इत्यभिधीयते}
{तस्मात्क्षम महाबाहो मम पार्थस्य चोभयोः}


\twolineshloka
{व्यतिक्रममिमं राजन्सत्यसंरक्षणं प्रति}
{शरणं त्वां महाराज प्रतिपन्नावुभावपि}


\twolineshloka
{क्षन्तुमर्हसि मे राजन्प्रणतस्याभियाचतः}
{राधेयस्याद्य पापस्य भूमिः पास्यति शोणितम्}


\twolineshloka
{सत्यं ते प्रतिजानामि हतं विद्व्यद्य सूतजम्}
{यस्येच्छसि वधं तस्य गतमेवाद्य जिवीतम्}


\threelineshloka
{इति कृष्णवचः श्रुत्वा धर्मराजो युधिष्ठिरः}
{ससम्भ्रमं हृषीकेशमुत्थाप्य प्रणतं तदा}
{कृताञ्जलिमुवाचेदं वाक्यं यत्समनन्तरम्}


\twolineshloka
{एवमेव यथात्थ त्वमस्त्येषोऽतिक्रमो मम}
{अनुनीतोऽस्मि गोविन्द तारितश्चास्मि माधव}


\threelineshloka
{मोचिता व्यसनाद्धोराद्वयमद्य त्वयाच्युत}
{भवन्तं नावमासाद्य ह्यावां व्यसनसागरात्}
{घोरादद्य समुत्तीर्णावुभावज्ञानमोहितौ}


\twolineshloka
{त्वद्बुद्धिप्लवमासाद्य दुःखशोकार्णवाद्वयम्}
{समुत्तीर्णाः सहामात्याः सनाथाः स्म त्वयाच्युत}


\chapter{अध्यायः ७९}
\twolineshloka
{सञ्जय उवाच}
{}


\twolineshloka
{इति स्म कृष्णवचनात्प्रत्युच्चार्य युधिष्ठिरम्}
{बभूव विमनाः पार्थः किञ्चित्कृत्वेव पातकम्}


\twolineshloka
{ततोऽब्रवीद्वासुदेवः प्रहसन्निव पाण्डवम्}
{कथं नाम भवेदेतद्यदि त्वं पार्थ धर्मजम्}


\twolineshloka
{असिना तीक्ष्णधारणे हन्या धर्मे व्यस्थितम्}
{त्वमित्युक्त्वाथ राजानमेवं कश्मलमाविशः}


\twolineshloka
{हत्वा तु नृपतिं पार्थ करिष्यसि किमुत्तरम्}
{एवं हि दुर्विदो धर्मो मन्दप्रज्ञैर्विशेषतः}


\twolineshloka
{स भवान्धर्मभीरुत्वाद्धवं यायान्महत्तमः}
{नरकं घोररूपं च भ्रातुर्ज्येतुस्य वै वथात्}


\twolineshloka
{स त्वं धर्मभृतां श्रेष्ठं राजानं धर्मसंहितम्}
{प्रसादय कुरुश्रेष्ठमेतदत्र मतं मम}


\twolineshloka
{प्रसाद्य भक्त्या राजानं प्रीते चैव युधिष्ठिरे}
{प्रयावस्त्वरितौ योद्धुं सूतपुत्ररथं प्रति}


\twolineshloka
{हत्वा तु समरे कर्णं त्वमद्य निशितैः शरैः}
{विपुलां प्रीतिमाधत्स्व धर्मपुत्रस्य मानद}


\threelineshloka
{एतदत्र महाबाहो प्राप्तकालं मतं मम}
{एवं कृते कृतं चैव तव कार्यं भविष्यति ॥सञ्जय उवाच}
{}


\twolineshloka
{ततोऽर्जुनो महाराज लज्जपा वै समन्वितः}
{धर्तराजस्य चरणौ प्रपद्य शिरसा नतः}


\twolineshloka
{उवाच भरतश्रेष्ठं प्रसीदेति पुनः पुनः}
{क्षमस्व राजन्यत्प्रोक्तस्त्वं मया धर्मभीरुणा}


\twolineshloka
{पादयोः पतितं दृष्ट्वा धर्मराजो युधिष्ठिरः}
{धनञ्जयममित्रघ्नं रुदन्तं भरतर्षभम्}


\twolineshloka
{उत्थाप्य भ्रातरं राजा धर्मराजो धनञ्जयम्}
{समाश्लिष्य च सस्नेहं प्ररुरोद महीपतिः}


\twolineshloka
{रुदित्वा सुचिरं कालं भ्रातरौ सुमहाद्युती}
{कृतशौचौ महाराज प्रीतिमन्तौ बभूवतुः}


\threelineshloka
{तत आश्लिष्य तं प्रेम्णा मूर्ध्नि चाघ्राय पाण्डवः}
{प्रीत्या परमया युक्तो विस्मयंश्च पुनःपुनः}
{अब्रवीत्तं महेष्वासं धर्मराजो धनञ्जयम्}


\threelineshloka
{कर्णेन मे महाबाहो सर्वसैन्यस्य पश्यतः}
{कवचं च ध्वजं चैव धनुः शक्तिर्हयाः शराः}
{शरैः कृत्ता महेष्वास यतमानस्य संयुगे}


\twolineshloka
{सोऽहं दृष्ट्वा रणे तस्य कर्ण कर्णस्य फल्गुन}
{व्यवसीदामि दुःखेन न तु मे जीवितं प्रियम्}


\twolineshloka
{न चेदद्य हि तं वीरं निहनिष्यसि संयुगे}
{प्राणानेव परित्यक्ष्ये जीवितार्थो हि को मम}


% Check verse!
एवमुक्तः प्रत्युवाच विजयो भरतर्षभ
\twolineshloka
{सत्येन ते शपे राजंस्त्वत्पादेन तथैव च}
{भीमेन च नरश्रेष्ठ यमाभ्यां च महीपते}


\threelineshloka
{`अहमेनं नरश्रेष्ठ सामात्यं च महीपते'}
{यथाद्य समरे कर्णं हनिष्यामि हतोपि वा}
{महीतले पतिष्यामि सत्येनायुधमालभे}


\threelineshloka
{एवमाभाष्य राजानमब्रवीन्माधवं वचः}
{अद्य कर्णं रणे कृष्ण सूदयिष्ये न संशयः}
{त्वमनुध्याहि भद्रं ते वधं तस्य दुरात्मनः}


\twolineshloka
{एवमुक्तोऽब्रवीत्पार्थं केशवो राजसत्तम}
{शक्तोऽसि भरतश्रेष्ठ यत्नं कर्तुं यदात्थ माम्}


\twolineshloka
{एष चापि हि मे कामो नित्यमेव महारथ}
{कथं भवान्रणे कर्णं निहन्यादिति सत्तम}


\twolineshloka
{एवमुक्तस्ततो राजन्पार्थो वचनमब्रवीत्}
{हन्यते द्वैरथे भूयो युज्यन्तां वै रथोत्तमे}


\twolineshloka
{उपावृत्ताश्च तुरगाः शिक्षिताश्चाश्वसादिभिः}
{रथोपकरणैः सर्वैः सत्वरं यातु मे रथः}


\threelineshloka
{एवमुक्तो महाराज फल्गुनेन महात्मना}
{उवाच दारुकं कृष्णः कुरु सर्वं यदब्रवीत्}
{अर्जुनो भारतश्रेष्ठः श्रेष्ठः सर्वधनुष्मताम्}


\threelineshloka
{आज्ञप्तस्त्वथ कृष्णेन दारुको राजसत्तम}
{योजयामास च रथं वैयाघ्नं शत्रुतापनम्}
{सज्जं निवेदयामास पाण्डवस्य महात्मनः}


\threelineshloka
{युक्तं तु रथमास्थाय दारुकेण महात्मना}
{उपस्थितं रथं दृष्ट्वा पद्मनाभो रणान्तकृत्}
{भूयश्चोवाच मतिमान्माधवो धर्मनन्दनम्}


\twolineshloka
{युधिष्ठिरेमं बीभत्सुं त्वं सान्त्वयितुमर्हसि}
{अनुज्ञातुं च कर्णस्य वधायाद्य दुरात्मनः}


\twolineshloka
{श्रुत्वा ह्यावां महासंख्ये त्वां कर्णशरपीडितम्}
{प्रवृत्तिं ज्ञातुमायाताविहावां पाण्डुनन्दन}


\threelineshloka
{दिष्ट्यासि राजन्विरुजो दिष्ट्या न ग्रहणं गतः}
{परिसान्त्वय बीभत्सुं जयमाशाधि चानघ ॥युधिष्ठिर उवाच}
{}


\twolineshloka
{एह्येहि पार्थ बीभत्सो मां परिष्वज पाण्डव}
{वक्तव्यमुक्तोस्म्यहितं त्वया क्षान्तं च तन्मया}


\threelineshloka
{अहं त्वामनुजानामि जहि कर्णं धनञ्जय}
{मन्युं च मा कृथाः पार्थ यन्मयोक्तोऽसि दारुणं ॥सञ्जय उवाच}
{}


\twolineshloka
{ततो धनञ्जयो राजञ्शिरसा प्रणतस्तदा}
{पादौ जग्राह पाणिभ्यां भ्रातुर्ज्येष्ठस्य मारिष}


\twolineshloka
{तमुत्थाप्य ततो राजा परिष्वज्य च पीडितम्}
{मूर्ध्नुपाघ्राय चैवैनमिदं पुनरुवाच ह}


\threelineshloka
{धनञ्जय महाबाहो मानितोऽस्मि दृढं त्वया}
{माहात्म्यं विजयं चैवं भूयः प्राप्नुहि शाश्वतम् ॥अर्जुन उवाच}
{}


\twolineshloka
{अद्य तं पापकर्माणं सानुबन्धं रणे शरैः}
{नयाम्यन्तं समासाद्य राधेयं बलगर्वितम्}


\twolineshloka
{येन त्वं पीडितो बाणैर्दृढमायम्य कार्मुकम्}
{तस्याद्य कर्मणः कर्णः फलमाप्स्यति दारुणम्}


\twolineshloka
{अद्य त्वामुपयास्यामि कर्णं हत्वा महीपते}
{सभाजये त्वामाक्रन्दादिति सत्यं ब्रवीमि ते}


\threelineshloka
{नाहत्वा विनिषर्तिष्ये कर्णमद्य रणाजिरात्}
{इति सत्येन ते पादौ स्पृशामि जगतीपते ॥सञ्जय उवाच}
{}


\twolineshloka
{इति ब्रुवाणं सुमनाः किरीटिनंयुधिष्ठिरः प्राह वचो बृहत्तरम्}
{यशोऽक्षयं जीवितमीप्सितं तेजयं सदा वीर्यमरिक्षयं तदा}


\twolineshloka
{प्रयाहि वृद्धिं च दिशन्तु देवतायथाहमिच्छामि तवास्तु तत्तथा}
{प्रयाहि शीघ्रं जहि कर्णमाहवेपुरन्दरो वृत्रमिवात्मवृद्वये}


\chapter{अध्यायः ८०}
\twolineshloka
{सञ्जय उवाच}
{}


\threelineshloka
{प्रसाद्य धर्मराजानं प्रहृष्टेनान्तरात्मना}
{सम्पूज्य देवताः सर्व ब्राह्मणान्स्वस्ति वाच्य च}
{सुमङ्गलं स्वस्त्ययनमारुरोह रथोत्तमम्}


\twolineshloka
{तस्य राजा महाप्राज्ञो धर्मराजो युधिष्ठिरः}
{आशिषोऽयुङ्क्त परमाः युक्तं कर्णरथं प्रति}


\twolineshloka
{तमायान्तं महेष्वासं दृष्ट्वा भूतानि भारत}
{निहतं मेनिरे कर्णं पाण्डवेन महात्मना}


\threelineshloka
{बभूवुर्विमलाः सर्वा दिशो राजन्समन्ततः}
{चाषाश्च शतपत्राश्च क्रौञ्चाश्चैव जनेश्वर}
{प्रदक्षिणमकुर्वन्त तदा वै पाण्डुनन्दनम्}


\twolineshloka
{बहवः पक्षिणो राजन्पुन्नामानः शुभाः शिवाः}
{त्वरयन्तोऽर्जुनं युद्धे हृष्टरूपा ववाशिरे}


\twolineshloka
{कङ्का गृध्रा बकाः श्येना वायसाश्च विशाम्पते}
{अग्रतस्तस्य गच्छन्ति भक्ष्यहेतोर्भयानकाः}


\twolineshloka
{निमित्तानि च धन्यानि पाण्डवस्य शशंसिरे}
{विनाशमरिसैन्यानां कर्णस्य च वधं तथा}


\twolineshloka
{प्रयातस्याथ पार्थस्य महान्स्वेदो व्यजायत}
{चिन्ता च विपुला जज्ञे कथं चेदं भविष्यति}


\twolineshloka
{विषण्णं तु ततो ज्ञात्वा सव्यसाचिनमच्युतः}
{सञ्चोदयति तेजस्वी मधुहा वानरध्वजम्}


\twolineshloka
{ततो गाण्डीवधन्वानमब्रवीन्मधुसूदनः}
{दृष्ट्वा पार्थं तथायान्तं चिन्तापरिगतं तदा}


\twolineshloka
{गाण्डीवधन्वन्सङ्ग्रामे ये त्वया धनुषा जिताः}
{न तेषां मानुषो जेता त्वदन्य इह विद्यते}


\twolineshloka
{एते हि बहवः शूराः शक्रतुल्यपराक्रमाः}
{त्वां प्राप्य समरे शूरं प्रयाताः परमां गतिम्}


\twolineshloka
{को हि द्रोणं च भीष्मं च भगदत्तं च मारिष}
{विन्दानुविन्दावावन्त्यौ काम्भोजं च सुदक्षिणम्}


\twolineshloka
{श्रुतायुं चाश्रुतायुं च शतायुं च महारथम्}
{प्रत्युद्गम्य भवेत्क्षेमी यो न स्यात्त्वद्विधः प्रभुः}


\threelineshloka
{तव ह्यस्त्राणि दिव्यानि लाघवं बलमेव च}
{असम्मोहश्च युद्धेषु विज्ञानस्य च सन्नतिः}
{वेधः पातश्च लक्षेषु योगश्चैव तथार्जुन}


\twolineshloka
{भवान्देवान्सगन्धर्वान्हन्यात्सर्वांश्च राक्षसान्}
{पृथिव्यां तु रणे पार्थ न योद्धा त्वत्समः पुमान्}


\twolineshloka
{धनुर्गृह्मन्ति ये केचित्क्षत्रिया युद्धदुर्मदाः}
{आत्मनस्तु समं तेषां न पश्यामि शृणोमि च}


\twolineshloka
{ब्रह्मणा हि प्रजाः सृष्टा गाण्डीवं च महद्धनुः}
{येन त्वं युध्यसे पार्थ तस्मान्नास्ति त्वया समः}


\twolineshloka
{अवश्यं तु मया वाच्यं यत्पथ्यं तव पाण्डव}
{मावमंस्था महाबाहो कर्णमाहवशोभिनम्}


\twolineshloka
{कर्णो हि बलवान्दृप्तः कृतास्त्रश्च महारथः}
{कृती च चित्रयोधी च देशकालस्य कोविदः}


\threelineshloka
{बहुनात्र किमुक्तेन संक्षेपाच्छृणु पाण्डव}
{त्वत्समं त्वद्विशिष्टं वा कर्णं मन्ये महारथम्}
{परमं यत्नमास्थाय त्वया वध्यो महाहवे}


\twolineshloka
{तेजसा वह्निसदृशो वायुवेगसमो जवे}
{अन्तकप्रतिमः क्रोधे सिंहसंहननो बली}


\twolineshloka
{अष्टरत्निर्महाबाहुर्व्यूढोरस्कः सुदुर्जयः}
{अभिमानी च शूरश्च प्रवीरः प्रियदर्शनः}


\twolineshloka
{सर्वयोधगुणैर्युक्तो मित्राणामभयङ्करः}
{सततं पाण्डवद्वेषी धार्तराष्ट्रहिते रतः}


\threelineshloka
{सर्वैरवध्यो राधेयो देवैरपि सवासवैः}
{ऋते त्वामिति मे बुद्धिस्तदद्य जहि सूतजम् ॥देवैरपि हि संयत्तैर्बिभ्रद्भिर्मांसशोणितम्}
{अशक्यः स रथो जेतुं सर्वैरपि युयुत्सुभिः}


\twolineshloka
{दुरात्मानं पापवृत्तं नृशंसंदुष्टप्रज्ञं पाण्डवेयेषु नित्यम्}
{हीनस्वार्थं पाण्डवेयैर्विरोधेहत्वा कर्णं निश्चितार्थो भवाद्य}


\twolineshloka
{तं सूतपुत्रं रथिनां वरिष्ठंनिष्कालिकं कालवशं नयाद्य}
{तं सूतपुत्रं रथिनां वरिष्ठंहत्वा प्रीतिं धर्मराजे कुरुष्व}


\twolineshloka
{जानामि ते पार्थं वीर्यं यथाव--द्दुर्वारणीयं च सुरासुरैश्च}
{सदावजानाति हि पाण्डुपुत्रा--नसौ दर्पात्सूतपुत्रो दुरात्मा}


\twolineshloka
{आत्मानं मन्यते वीरं येन पापः सुयोधनः}
{तमद्य मूलं पापानां जहि सौतिं धनञ्जय}


\twolineshloka
{खङ्गजिह्वं धनुरास्यं शरदंष्ट्रं तरस्विनम्}
{दृप्तं पुरुषशार्दूलं जहि कर्णं धनञ्जय}


\twolineshloka
{अहं त्वामनुजानामि वीर्येण च बलेन च}
{जहि कर्णं रणे शूरं मातङ्गमिव केसरी}


\twolineshloka
{यस्य वीर्येण वीर्यं ते धार्तराष्ट्रोऽवमन्यते}
{तमद्य पार्थ सङ्ग्रामे कर्णं वैकर्तनं जहि}


\chapter{अध्यायः ८१}
\twolineshloka
{सञ्जय उवाच}
{}


\twolineshloka
{ततः पुनरमेयात्मा केशवोऽर्जुनमब्रवीत्}
{कृतसङ्कल्पमायान्तं वधे कर्णस्य भारत}


\twolineshloka
{अद्य सप्तदशाहानि वर्तमानस्य नित्यशः}
{विनाशस्यातिघोरस्य नरवारणवाजिनाम्}


\twolineshloka
{भूत्वा हि विपुला सेना तावकानां परैः सह}
{अन्योन्यं समरं प्राप्य किञ्चिच्छेषा विशाम्पते}


\twolineshloka
{भूत्वा वै कौरवाः पार्थ प्रभूतगजवाजिनः}
{त्वां वै शत्रुं समासाद्य विनष्टा रणमूर्धनि}


\twolineshloka
{एते ते पृथिवीपालाः सृञ्जयाश्च समागताः}
{त्वां समासाद्य दुर्धर्षं पाण्डवाश्च व्यवस्थिताः}


\threelineshloka
{पाञ्चालैः पाण्डवैर्मात्स्यैः कारूशैश्चेदिभिः सह}
{`मगधैः पारिजातैश्च दाक्षिणात्यैः सकेरलैः'}
{त्वया गुप्तैरमित्रघ्नैः कृतः शत्रुगणक्षयः}


\twolineshloka
{को हि शक्तो रणे जेतुं कौरवांस्तात संयुगे}
{अन्यत्र पाण्डवाद्युद्धे श्वेताश्वाद्वानरध्वजात्}


\twolineshloka
{शक्तस्त्वं हि रणे जेतुं ससुरासुरमानुषान्}
{त्रींल्लोकान्समरे युक्तान्किं पुनः कौरवं बलम्}


\twolineshloka
{भगदत्तं च राजानं कोऽन्यः शक्तस्त्वया विना}
{जेतुं पुरुषशार्दूल योऽपि स्याद्वासवोपमः}


\twolineshloka
{तथेमां विपुलां सेनां गुप्तां पार्थ त्वयाऽनघ}
{न शेकुः पार्थिवाः सर्वे चक्षुर्भिरपि वीक्षितुम्}


\twolineshloka
{तथैव सततं पार्थ रक्षिताभ्यां त्वया रणे}
{धृष्टद्युम्नशिखण़्डिभ्यां द्रोणभीष्मौ निपातितौ}


\twolineshloka
{को हि शक्तो रणे पार्थ भारतानां महारथौ}
{भीष्मद्रोणौ युधा जेतुं शक्रतुल्यपराक्रमौ}


\twolineshloka
{को हि शान्तनवं भीष्मं द्रोणं वैकर्तनं कृपम्}
{द्रौणिं च सौमदत्तिं च कृतवर्माणमेव च}


\twolineshloka
{सैन्धवं मद्रराजं च राजानं च सुयोधनम्}
{वीरान्कृतास्त्रान्समरे सर्वानेवानिवर्तिनः}


\twolineshloka
{अक्षौहिणीपतीनुग्रान्संहतान्युद्धदुर्मदान्}
{त्वामृते पुरुषव्याघ्र जेतुं शक्तः पुमानिह}


\twolineshloka
{श्रेण्यश्च बहुलाः क्षीणाः प्रदीर्णाश्वरथद्विपाः}
{नानाजनपदाश्चोग्राः क्षत्रियाणाममर्षिणाम्}


\threelineshloka
{गणाश्च दासमीयानां वसातीनां च भारत}
{प्राच्यानां वाटधानानां भोजानां चाभिमानिनाम्}
{}


\twolineshloka
{उदीर्णाश्वगजा सेना सर्वक्षत्रस्य भारत}
{त्वां समासाद्य निधनं गता भीमं च भारत}


\twolineshloka
{उग्राश्च भीमकर्माणस्तुषारा यवनाः खशाः}
{दार्वाभिसारा दरदाः शका माठरतङ्कणाः}


\twolineshloka
{आन्ध्रकाश्च पुलिन्दाश्च किराताश्चोग्रविक्रमाः}
{म्लेच्छाश्च पर्वतीयाश्च सागरानूपवासिनः}


% Check verse!
संरम्भिणो युद्धशौण्डा बलिनो दण्डपाणयः
\twolineshloka
{एते सुयोधनस्यार्थे संरब्धाः कुरुभिः सह}
{न शक्या युधि निर्जेतुं त्वदन्येन परन्तप}


\twolineshloka
{धार्तराष्ट्रमुदग्रं हि व्यूढं दृष्ट्वा महद्बलम्}
{यदि त्वं न भवेस्त्राता प्रतीयात्को नु मानवः}


\twolineshloka
{तत्सागरमिवोद्वूतं रजसा संवृतं बलम्}
{विदार्य पाण्डवैः क्रुद्धैस्त्वया गुप्तैर्हतं विभो}


\twolineshloka
{मागधानामधिपतिर्जयत्सेनो महाबलः}
{अद्य सप्तैव चाहानि हतः सङ्ख्येऽभिमन्युना}


\twolineshloka
{ततो दशसहस्राणि गजानां भीमकर्मणाम्}
{जघान गदया भीमस्तस्य राज्ञः परिच्छदम्}


\twolineshloka
{तथान्येऽभिहता नागा रथाश्च शतशो बलात्}
{तदेवं समरे पार्थ वर्तमाने महाभये}


\twolineshloka
{भीमसेनं समासाद्य त्वां च पाण्डव कौरवाः}
{सवाजिरथमातङ्गा मृत्युलोकमितो गताः}


\twolineshloka
{तथा सेनामुखे तत्र निहते पार्थ पाण्डवैः}
{भीष्मः प्रासृजदुग्राणि शरतालानि मारिष}


\twolineshloka
{सचेदिकाशिपाञ्चालान्करूशान्मात्स्यकेकयान्}
{शरैः प्रच्छाद्य निधनमनयत्परमास्त्रवित्}


\twolineshloka
{तस्य चापच्युतैर्बाणैः परदेहविदारणैः}
{पूर्णमाकाशमभवद्रुक्मपुह्खैरजिह्मगैः}


\twolineshloka
{हन्याद्रथसहस्राणि एकैकेनैव मुष्टिना}
{लक्षं नरद्विपान्हत्वा समेतान्समहाबलान्}


\twolineshloka
{गत्या दशम्या ते गत्वा जघ्नुर्वाजिरथद्विपान्}
{हित्वा नवगतीर्दुष्टाः स बाणानाहवेऽत्यजत्}


\twolineshloka
{दिनानि दश भीष्मेण निघ्नता तावकं बलम्}
{शून्याः कृता रथोपस्था हताश्च गजवाजिनः}


\twolineshloka
{`दशमेऽहनि सम्प्राप्ते कृत्वा घोरं पराक्रमम्'}
{दर्शयित्वाऽऽत्मनो रूपं रुद्रोपेन्द्रसमं युधि}


\twolineshloka
{पाण्डवानामनीकानि प्रविगाह्य विशाम्पते}
{विनिघ्नन्पृथिवीपालांश्चेदिपाञ्चालकेकयान्}


\twolineshloka
{अहनत्पाण्डवीं सेनां रथाश्वगजसङ्कुलाम्}
{मज्जन्तमप्लुवे मन्दमुज्जिहीर्षुः सुयोधनम्}


\twolineshloka
{तथा चरन्तं समरे तपन्तमिव भास्करम्}
{पदातिकोटिसाहस्राः प्रवरायुधपाणयः}


\twolineshloka
{न शेकुः सृञ्जया द्रष्टुं तथैवान्ये महीक्षितः}
{विचरन्तं तथा तं तु सङ्ग्रमे जितकाशिनम्}


\threelineshloka
{सर्वोद्यमेन महता पाण्डवान्समभिद्रवत्}
{स तु विद्राव्य समरे पाण्डवान्सृञ्जयानपि}
{एक एव रणे भीष्म एकवीरत्वमागतः}


\twolineshloka
{तं शिखण्डी समासाद्य त्वया गुप्तो महाव्रतम्}
{जघान पुरुषव्याघ्रं शरैः सन्नतपर्वभिः}


\twolineshloka
{स एष पतितः शेते शरतल्पे पितामहः}
{त्वां प्राप्य पुरुषव्याघ्रं वृत्रः प्राप्येव वासवम्}


\twolineshloka
{द्रोणः पञ्चदिनान्युग्रो विधम्य रिपुवाहिनीम्}
{कृत्वा व्यूहमभेद्यं च पातयित्वा महारथान्}


\twolineshloka
{जयद्रथस्य समरे कृत्वा रक्षां महारथः}
{अन्तकप्रतिमश्चोग्रो रात्रियुद्धेऽदहत्प्रजाः}


\twolineshloka
{दग्ध्वा योधाञ्छरैर्वीरो भारद्वाजः प्रतापवान्}
{धृष्टद्युम्नं समासाद्य स गतः परमां गतिम्}


\twolineshloka
{यदि वाऽद्य भवान्युद्धे सूतपुत्रमुखान्रथान्}
{नावारयिष्यः सङ्ग्रामे न स्म द्रोणो व्यनङ्क्ष्यत}


\twolineshloka
{भवता तु बलं सर्वं धार्तराष्ट्रस्य वारितम्}
{ततो द्रोणो हतो युद्धे पार्षतेन धनञ्जय}


\twolineshloka
{कश्च शक्तो रणे कर्तुं त्वदन्यः पुरुषब्रुवः}
{यादृशं ते कृतं पार्थ जयद्रथवधं प्रति}


\twolineshloka
{निवार्य सेनां महतीं हत्वा शूरांश्च पार्थिवान्}
{निहतः सैन्धवो राजा त्वयाऽस्त्रबलतेजसा}


\twolineshloka
{आश्चर्यं सिन्धुराजस्य वधं जानन्ति पार्थिवाः}
{अनाश्चार्यं हि तत्त्वत्तस्त्वं हि पार्थ महारथः}


\twolineshloka
{त्वां हि प्राप्य रणे क्षत्रमेकाहादिति भारत}
{नश्यमानमहं युक्तं मन्येयमिति मे मतिः}


\twolineshloka
{सेयं पार्थ चमूर्घोरा धार्तराष्ट्रस्य संयुगे}
{हतसर्वस्वभूयिष्ठा भीष्मद्रोणौ हतौ यथा}


\twolineshloka
{शीर्णप्रवरयोधाढ्या हतवाजिरथद्विपा}
{हीना सूर्येन्दुनक्षत्रैर्द्यौरिवाभाति भारती}


\twolineshloka
{विध्वस्ता हि रणे पार्थ सेनेयं भीमविक्रम}
{आसुरीव महासेना देवराजपराक्रमैः}


\twolineshloka
{तेषां हतावशिष्टास्तु सन्ति पञ्च महारथाः}
{द्रौणिश्च कृतवर्मा च कर्णो मद्राधिपः कृपः}


\twolineshloka
{तांस्त्वमद्य नरव्याघ्र हत्वा पञ्च महारथान्}
{हतामित्रः प्रयच्छोर्वी राज्ञे सद्वीपपत्तनाम्}


\twolineshloka
{साकाशजलपातालां सपर्वतमहावनाम्}
{प्रयच्छामितवीर्याय पार्थायाद्य वसुन्धराम्}


\twolineshloka
{एतां पुरा विष्णुरिव हत्वा दैतेयदानवान्}
{प्रयच्छ मेदिनीं राज्ञे शक्रायैव हरिर्यथा}


\twolineshloka
{अद्य मोदन्तु पाञ्चाला निहतेष्वरिषु त्वया}
{विष्णुना निहतेष्वेव दानवेयेषु देवताः}


\twolineshloka
{यदि वा द्विपदां श्रेष्ठं द्रोणं मानयतो गुरुम्}
{अश्वत्थाम्नि कृपा तेऽस्ति कृपे वाचार्यगौरवात्}


\twolineshloka
{अत्यन्तापचितान्बन्धून्मानयन्मातृबान्धवान्}
{कृतवर्माणमासाद्य न नेष्यासि यमक्षयम्}


\twolineshloka
{भ्रातरं मातुरासाद्य शल्यं मद्रजनाधिपम्}
{यदि त्वमरविन्दाक्ष दयावान्न जिघाससि}


\twolineshloka
{एतत्ते सुकृतं कर्म नात्र किञ्चन विद्यते}
{वयमप्यनुजानीमो नात्र दोषोऽस्ति कश्चन}


\twolineshloka
{इमं पापमतिं क्षुद्रमत्यन्तं पाण्डवान्प्रति}
{कर्णमद्य नरश्रेष्ठ जहि पार्थ शितैः शरैः}


\threelineshloka
{दहने यत्सपुत्राया निशि मातुस्तवानघ}
{द्यूतार्थे यच्च युष्मासु प्रावर्तत सुयोधनः}
{तस्य सर्वस्य दुष्टात्मा कर्णो वै मूलमित्युत}


\twolineshloka
{प्रोत्साहयति दुष्टात्मा धार्तराष्ट्रं सुदुर्मतिम्}
{समितौ गदते कर्णस्तमद्य जहि भारत}


\twolineshloka
{यश्च युष्मासु पापं वै धार्तराष्ट्रः प्रयुक्तवान्}
{तस्य सर्वस्य दुर्बुद्धिः कर्णो मूलमिहार्जुन}


\twolineshloka
{कर्णं हि मन्यते त्राणं नित्यमेव सुयोधनः}
{ततो मामपि संरब्धो निग्रहीतुं पराक्रमात्}


\twolineshloka
{स्थिता बुद्धिर्नरेन्द्राणां धार्तराष्ट्रस्य चोभयोः}
{कर्णः पार्थान्रणे सर्वान्नाशयिष्यति सायकैः}


\twolineshloka
{कर्णमाश्रित्य कौन्तेय धार्तराष्ट्रस्य विग्रहः}
{रुचितो भवता सार्धं जानतोऽपि बलं तव}


\threelineshloka
{कर्णो जल्पति वै नित्यमहं पार्थान्समागतान्}
{वासुदेवं च दाशार्हं विजेष्यामि महारणे}
{समितौ वल्गते कर्णस्तमद्य जहि फल्गुन}


\twolineshloka
{यच्च युष्मासु पापं वै धार्तराष्ट्रः प्रतापवान्}
{सभायां कृतवान्नित्यं कर्णमाश्रित्य वै पुरा}


\twolineshloka
{यच्च तं धार्तराष्ट्राणां षड्भिः शूरैर्महारथैः}
{पश्यतां संवृतं शूरं सौभद्रमपराजितम्}


\twolineshloka
{द्रोणद्रौणिकृपान्वीरान्कम्पयानं महेषुभिः}
{विधमन्तमनीकानि प्रमथन्तं महारथान्}


\twolineshloka
{मनुष्यवाजिमातङ्गान्प्रेषयन्तं यमक्षयम्}
{शरैः सौभद्रमायान्तं दहन्तमरिवाहिनीम्}


\twolineshloka
{निर्मनुष्याश्च मातङ्गा विरथाश्च महारथाः}
{प्रद्रवन्ति स्म समरे दिशो भीताऽभिमन्यवे}


\threelineshloka
{विगतासूंश्च तुरगान्पत्तीन्व्यायुधजीवितान्}
{कुर्वन्तमृषभस्कन्धं कुरुवृष्णियशस्करम्}
{तन्मे दहति गात्राणि सखे सत्येन ते शपे}


\twolineshloka
{यत्तदासीत्सुदुष्टात्मा कर्णो विनिहतः प्रभुः}
{न शक्तो ह्यभिमन्योस्तु कर्णः स्थातुं रणाग्रतः}


\twolineshloka
{सौभद्रशरनिर्भिन्नो विसंज्ञः शोणितोक्षितः}
{निश्वसन्क्रोधसन्दीप्तो विमुखः सायकार्दितः}


\twolineshloka
{तस्थौ स विह्वलः सङ्ख्ये प्रहारजनितच्छविः}
{अपयानकृतोत्साहो निरुत्साहश्च भारत}


\twolineshloka
{दुर्योधनं रणे दृष्ट्वा लज्जमानो मुहुर्मुहुः}
{नापयासीत्तततः पार्थ सोऽभिमन्योर्महारणे}


\twolineshloka
{दृष्ट्वा द्रोणं वधोपायमभिमन्योश्च पृष्टवान्}
{श्रुत्वा द्रोणवचः क्रूरं ततश्चिच्छेद कार्मुकम्}


\twolineshloka
{ततश्छिन्नायुधं तेन दृष्ट्वा पञ्च महारथाः}
{स चैव निकृतिप्राज्ञः प्राहिणोच्छरवृष्टिभिः}


\threelineshloka
{प्रहसन्स तु दुष्टात्मा कर्णो राजा च कौरवः}
{यच्च कर्णोऽब्रवीत्कृष्णां सभायां परुषं वचः}
{प्रमुखे पाण्डवेयानां कुरूणां चैव पश्यताम्}


\twolineshloka
{विनष्टाः पाण्डवाः कृष्णे शाश्वतं नरकं गताः}
{पतिमन्यं पृथुश्रोणि वृणीष्व मृदुभाषिणि}


\twolineshloka
{एषा त्वं धृतराष्ट्रस्य दासीभूता निवेशनम्}
{प्रविशारालपक्ष्माक्षि न सन्ति पतयस्तव}


\twolineshloka
{न पाण्डवाः प्रभवन्ति तव कृष्णे कथञ्चन}
{दासभार्या च पाञ्चालि स्वयं दासी च शोभने}


\twolineshloka
{अद्य दुर्योधनो राजा पृथिव्यां नृपतिः स्मृतः}
{सर्वे चास्य महिपाला योगक्षेममुपासते}


\twolineshloka
{पश्येदानीं यदा भद्रे निविष्टाः पाण्डवाः समम्}
{अन्योन्यं समुदीक्षन्ते धार्तराष्ट्रस्य तेजसा}


\twolineshloka
{व्यक्तं षण्डतिला ह्येते नरके च निमज्जिताः}
{प्रेष्यवच्चापि राजानमुपस्थास्यन्ति कौरवम्}


\twolineshloka
{उक्तवान्स च पातात्मा तथा परमदुर्मतिः}
{पापः पापवचः कर्णः पश्यतस्ते धनञ्जय}


\twolineshloka
{अस्य पापस्य तद्वाक्यं सुवर्णविकृताः शराः}
{शमयन्ति शिलाधौता नाशयन्तोऽस्य जीवितम्}


\twolineshloka
{अद्य कर्णं रणे ग्रस्तं पश्यन्तु कुरवस्त्वया}
{स्वर्गावतरणे यत्नं स्वर्गद्वारगतं यथा}


\twolineshloka
{अद्य ते समरे वीर्यं पश्यन्तु कुरुयोधिनः}
{सूतपुत्रे हते पार्थ जानन्तु त्वां महारथम्}


\twolineshloka
{अद्य काकवला गृध्रा वायसा जम्बुकास्तथा}
{विप्रकर्षन्तु गात्राणि सूतपुत्रस्य मारिष}


\twolineshloka
{अद्याधिरथिराक्षिप्तो निहतश्च त्वया रणे}
{कुरूणां शोकमाधत्तां पाण्वानां मुदं तदा}


\twolineshloka
{अद्य त्वां प्रतिमर्दन्तु पाञ्चालाः पाण्डवैः सह}
{यथा वृत्रवधे वृत्ते देवाः सर्वे शतक्रतुम्}


\twolineshloka
{अद्य कर्णं रणे हत्वा प्राप्य चैवोत्तमं यशः}
{विशोको विज्वरः पार्थ भव बन्धुपुरस्कृतः}


\twolineshloka
{नरसिंहवपुः कृत्वा यथा शस्तो महासुरः}
{हिरण्यकशिपुर्दैत्यो विष्णुना प्रभविष्णुना}


\twolineshloka
{तथा त्वमपि राधेयं घोरां कृत्वा महातनुम्}
{जहि युद्धे महाबाहो त्रायस्व च भयात्स्वकान्}


\twolineshloka
{कर्णं हाहाकृतं दीनं विषण्णं त्वच्छरार्दितम्}
{प्रपतन्तं महीं कर्णं पश्यन्तु वसुधाधिपाः}


\twolineshloka
{तं च स्वशोणिते मग्नं शयानं पतितं भुवि}
{अपविद्धायुकधं कर्णमद्य पश्यन्तु बान्धवाः}


\twolineshloka
{तच्चैवाद्य महत्कर्म गाण्डीवप्रेषितैः शरैः}
{रथोपस्थे विशीर्येत ताराराज इवाम्बरात्}


\twolineshloka
{आशु चाद्य शरास्तस्य सम्पतन्तो महाजवैः}
{त्वच्छरैः सन्निकृत्ताग्रा विशीर्यन्ते महीतले}


\twolineshloka
{त्वया चाद्य हते तस्य विक्रमे भरतर्षभ}
{विमुखाः सर्वराजानो भवन्तु गतजीविताः}


\twolineshloka
{तथा चाधिरथौ याते प्रयान्तु कुरवो दिशः}
{मन्वानास्तं रथश्रेष्ठं सर्वलोकेषु धन्विनाम्}


\twolineshloka
{स वै चाद्य भयात्त्यक्त्वा धार्तराष्ट्रो महाचमूम्}
{दुर्योधनो भयोद्विग्नो द्रवतु स्वं निवेशनम्}


\twolineshloka
{तथा चाद्य हतं श्रुत्वा धृतराष्ट्रो जनेश्वरः}
{क्षणेन निपतेद्भूमौ विसंज्ञो वै महीपतिः}


\twolineshloka
{अद्य जानन्तु ते पार्थ विक्रमं सर्वयोधिनः}
{यदुवाच सभामध्ये परुषं भारत त्वयि}


\twolineshloka
{यानि चान्यानि दुष्टात्मा पापानि कृतवांस्त्वयि}
{तान्यद्य भरतश्रेष्ठ नाशयन्तु शरास्तव}


\twolineshloka
{शान्तिं कुरु परिक्लेशा कृष्णायाः शत्रुपातन}
{हत्वा शत्रुं रणे श्लाघ्यं गर्जन्तमतिपौरुषम्}


\twolineshloka
{अद्य चाधिरथिर्वेद्धस्तव बाणैः समन्ततः}
{मन्यतां त्वां नरव्याघ्र प्रवरं सर्वधन्विनाम्}


\twolineshloka
{गाण्डीवप्रसृतान्वाणानद्य गात्रस्पृशः शरान्}
{यातु कर्णो रणे पार्थ श्वाविच्छललतो यथा}


\threelineshloka
{तं कथं कर्णमासाद्य विद्रवेयुर्महारथाः}
{यस्त्वेकः सर्वपाञ्चालानहन्यहनि नाशयन्}
{कालवच्चरते वीर पाञ्चालानां रथव्रजे}


\threelineshloka
{तमप्यासाद्य समरे मित्रार्थे मित्रवत्सल}
{तथा ज्वलन्तमस्त्रैश्च शूरं सर्वधनुष्मताम्}
{निर्दहन्तं समारूढं दुर्धर्षं द्रोणमञ्जसा}


\twolineshloka
{ते नित्यमुदिता जेतुं युधि शत्रुमरिन्दमाः}
{न चेदाधिरथेर्भीताः पाञ्चालाः स्युः पराङ्मुखाः}


\twolineshloka
{तेषामापततां शूरः पाञ्चालानां तरस्विनाम्}
{आदत्तासूञ्शरैः कर्णः पतङ्गानामिवानलः}


\twolineshloka
{एते द्रवन्ति पाञ्चाला द्राव्यन्ते योधिभिर्ध्रुवम्}
{कर्णेन भरतश्रेष्ठ पश्यपश्य तथाकृतान्}


\twolineshloka
{तान्समारोहतः शूरान्मित्रार्थे त्यक्तजीवितान्}
{निस्तारय महाबाहो कर्णास्त्रात्पावकोपमात्}


\twolineshloka
{अस्त्रं हिरामात्कर्णेन भार्गवादृषिसत्तमात्}
{यदवाप्तं तदा घोरं तस्य रूपमुदीर्यते}


\twolineshloka
{तापनं सर्वसैन्यस्य घोररूपं भयानकम्}
{यमाश्रित्य महासेना ज्वलते स्वेन तेजसा}


\twolineshloka
{एते चरन्ति सङ्ग्रामे कर्णचापच्युताः शराः}
{प्रभया इह शत्रूणां घातयन्तो जनान्प्रभो}


\twolineshloka
{एते भ्रमन्ति पाञ्चाला उत्क्रयन्ति च मारिष}
{कर्णास्त्रं समरे प्राप्य दुर्निवार्यं महात्मभिः}


\twolineshloka
{एष भीमो दृढक्रोधो वृतः पार्थ समन्ततः}
{सृञ्जयैर्योऽजयत्कर्णं पीड्यते निशितैः शरैः}


\twolineshloka
{पाञ्चालान्सृञ्जयांश्चैव पाण्डवांश्चैव भारत}
{उपेक्षितो दहेत्कर्णो रोगो देहमिवान्तकः}


\twolineshloka
{नान्यं त्वत्तो हि पश्यामि योधं यौधिष्ठिरे बले}
{यः समासाद्य राधेयं स्वस्तिमानाव्रजेद्गृहान्}


\twolineshloka
{तमद्य निशितैर्बाणैर्निहत्य भरतर्षभ}
{यथा प्रतिज्ञां त्व पार्थ तीर्त्वा कीर्तिमवाप्स्यसि}


\twolineshloka
{त्वं हि शक्तो रणे कर्णं विजेतुं सह पार्थिवैः}
{नान्यो युधि युधां श्रेष्ठ सत्यमेतद्ब्रवीमि ते}


\twolineshloka
{एतत्कृत्वा महत्कार्म हत्वा कर्णं महारथम्}
{कृतार्थः सफलः पार्थ सुखी भव नरोत्तम}


\chapter{अध्यायः ८२}
\twolineshloka
{सञ्जय उवाच}
{}


\twolineshloka
{स केशवस्य बीभत्सुः श्रुत्वा भारत भाषितम्}
{विशोकः सम्प्रहृष्टश्च क्षणेन समपद्यत}


\twolineshloka
{ततो ज्यामभिमृज्याशु व्याक्षिपद्गाण्डिवं धनुः}
{दध्रे कर्णविनाशाय केशवं चाभ्यभाषत}


\twolineshloka
{त्वया नाथेन गोविन्द ध्रुव एव जयो मम}
{प्रसन्नो यस्य भगवान्भूतभव्यभविष्यकृत्}


\twolineshloka
{त्वत्सहायो ह्यहं कृष्ण त्रींल्लोकान्वै समागतान्}
{प्रापयेयं परं लोकं किमु कर्णं महाहवे}


\twolineshloka
{पश्यामि द्रवतीं सेनां पाञ्चालानां जनार्दन}
{पश्यामि कर्णं समरे विचरन्तमभीतवत्}


\twolineshloka
{भार्गवास्त्रं च पश्यामि ज्वलन्तं कृष्ण सर्वशः}
{सृष्टं कर्णेन वार्ष्णेय शक्रेणेव महाशनिम्}


\twolineshloka
{अनेन खलु सङ्ग्रामे यत्तु कृष्ण मया कृतम्}
{कथयिष्यन्ति भूतानि यावद्भूमिर्धरिष्यति}


\twolineshloka
{अद्य कृष्ण विकर्णा मे कर्णं नेष्यन्ति मृत्यवे}
{गाण्डीवमुक्ताः क्षिण्वन्तो मम हस्तप्रचोदिताः}


\twolineshloka
{अद्य राजा धृतराष्ट्रः स्वां बुद्धिमवमंस्यते}
{दुर्योधनमराज्यार्हं यया राज्येऽभ्यषेचयत्}


\twolineshloka
{अद्य राज्यात्सुखाच्चैव श्रियो राष्ट्रात्तथा पुरात्}
{पुत्रेभ्यश्च महाबाहो धृतराष्ट्रो विमोक्ष्यति}


\twolineshloka
{गुणवन्तं हि यो हित्वा निर्गुणं कुरुते प्रभुम्}
{स शोचति चिरं कृष्ण क्षिप्रमेवागते क्षये}


\twolineshloka
{यथा च पुरुषः कश्चिच्छित्त्वा चाम्रवणं महत्}
{पलाशसेचने बुद्धिं कृत्वा शोचति मन्दधीः}


\threelineshloka
{दृष्ट्वा पुष्पं पले गृध्नुः फलं दृष्ट्वाऽनुशोचति}
{तथेदं धृतराष्ट्रस्य पुष्पलुप्धस्य मानद}
{फलं दृष्ट्वा भृशं दुःखं भविष्यति जनार्दन}


% Check verse!
सूतपुत्रे हते त्वद्य निराशो भविता प्रभुः
\twolineshloka
{अद्य दुर्योधनो राज्याज्जीविताच्च निराशकः}
{भविष्यति हते कर्णे कृष्ण सत्यं ब्रवीमि ते}


\twolineshloka
{अद्य दृष्ट्वा मया कर्णं शरैर्विशकलीकृतम्}
{स्मरतां तव वाक्यानि शमं प्रति जनेश्वरः}


\twolineshloka
{अद्यासौ सौबलः कृष्ण ग्लहाञ्जानातु वै शरान्}
{दुरोदरं च गाण्डीवं गण़्डवं च रथं मम}


\twolineshloka
{अद्य कुन्तीसुतस्याहं दृढं राज्ञः प्रजागरम्}
{व्यपनेष्यामि गोविन्द हत्वा कर्णं शितैः शरैः}


\twolineshloka
{अद्य कुन्तीसुतो राजा हते सूतसुते मया}
{सुप्रहृष्टमनाः प्रीतश्चिरं सुखमवाप्स्यति}


\twolineshloka
{अद्य वाहमनाधृष्यं केशवाप्रतिमं शरम्}
{उत्स्रक्ष्यामीहयः कर्णं जीविताद्धंशयिष्यति}


\twolineshloka
{यस्य चैतद्व्रतं मह्यं वधे किल दुरात्मनः}
{पादौन मxxxxxxन्यां न फल्गुनम्}


\twolineshloka
{xxxxxxxz पापस्य मधुसूदन}
{xxxxxxxxxxxxxx सन्नतपर्वभिः}


\twolineshloka
{xxxxxxxxxxxxxx नान्यं पृथिव्यामनुमन्यते}
{xxxxxxxxx सूतपुत्रस्य भूमिः पास्यति शोणितम्}


\twolineshloka
{xxxxxxx सूतपुत्रो यदव्रवीत्}
{xxxxxxxxxxx श्लाघमानः स्वकान्गुणान्}


\twolineshloka
{अनृतं तत्करिष्यन्ति मामका निशिताः शराः}
{आशीविषा इव क्रुद्धास्तस्यपास्यन्ति शोणितम्}


\twolineshloka
{मया हस्तवता मुक्ता नाराचा वैद्युतत्विषः}
{गाण्डीवसृष्टा दास्यन्ति कर्णस्य परमां गतिम्}


\twolineshloka
{अद्य तप्स्यति राधेयः पाञ्चालीं यत्तदब्रवीत्}
{सभामध्ये वचः क्रूरं कुत्सयन्पाण्डवान्प्रति}


\threelineshloka
{एते षण्डतिलाः कृष्णे निर्वीर्या हतविक्रमाः}
{अहं वः पाण्डवेयेभ्यो भयात्त्रास्येति चाब्रवीत्}
{हन्ताऽहं पाण्डवान्सर्वान्सपुत्रानिह भारत}


\twolineshloka
{अनृतं तत्करिष्यन्ति मामका निशिताः शराः}
{हते वैकर्तने कर्णे सूतपुत्रे दुरात्मनि}


\twolineshloka
{यस्य वीर्यं समाश्रित्य धार्तराष्ट्रो बृहन्मनाः}
{तमद्य कर्णं हन्तास्मि समरे मधुसूदन}


\twolineshloka
{अद्य कर्णे हते कृष्ण धार्तराष्ट्राः सराजकाः}
{विद्रवन्ति दिशो भीताः सिंहं दृष्ट्वा मृगा इव}


\twolineshloka
{अद्य दुर्योधनो राजा पृथिवीं नान्ववेक्षते}
{हते कर्णे मया सङ्ख्ये सपुत्रे ससुहृज्जने}


\twolineshloka
{अद्य कर्णे हतं दृष्ट्वा धार्तराष्ट्रोऽत्यमर्षणः}
{जानातु मां रणे कृष्ण प्रवरं सर्वधन्विनाम्}


\twolineshloka
{सपुत्रपौत्रः सामात्यः ससुहृच्च निराशिषः}
{पित्र्ये राज्ये निराशश्च धार्तराष्ट्रो निराश्रयः}


\twolineshloka
{अद्य राजा धर्मपुत्रो हतामित्रो भविष्यति}
{अद्य दुर्योधनो दीप्तां श्रियं राज्यं च हास्यति}


\twolineshloka
{हते वैकर्तने कर्णे भीष्मे द्रोणे च संयुगे}
{कातरं तद्बलं कृष्ण प्रविष्टं भोक्ष्यते तु यत्}


\twolineshloka
{अद्यप्रभृति राजानं धर्मपुत्रं युधिष्ठिरम्}
{अनुमोदन्तु सुहृदो ज्ञातपूर्वाश्च ब्राह्मणाः}


\twolineshloka
{अद्य तं निहतं श्रुत्वा कर्णं वैकर्तनं मया}
{करोतु पटहोन्मिश्रं देवतास्थानपूजनम्}


\twolineshloka
{अद्य कृष्ण हते कर्णे कुरुतां चिरसम्भृतम्}
{याजनं वै महाबाहो देवतानां यथाविधि}


\twolineshloka
{अद्य त्वम्बा च कृष्णा च त्वरमाणे परस्परम्}
{सस्वजेतां हृषीकेश सम्पूर्णेऽस्मिन्मनोरथे}


\twolineshloka
{अद्य त्वं पाण्डवो ज्येष्ठस्तथाऽऽर्यश्च वृकोदरः}
{उदीक्षेतां हते कर्णे कृष्ण सौम्येन चक्षुषा}


\twolineshloka
{अभिवाद्य गुरूनद्य कनिष्ठैश्चाभिवादितः}
{सस्वजानो ह्यहं दोर्भ्यां प्राप्स्यामि विपूलं यशः}


\twolineshloka
{अद्य कर्णे हते कृष्ण प्रशंसन्तोऽर्जुनं सुराः}
{त्रिदिवं यान्तु संहृष्टाः सङ्गताश्च तपोधनाः}


\twolineshloka
{अद्य लोकास्त्रयः कृष्ण जानन्तु मम पौरुषम्}
{दृष्ट्वा कर्णं हतं युद्धे द्वैरथे सव्यसाचिना}


\twolineshloka
{अद्याहमनृणः कृष्ण भविष्यामि धनुष्मताम्}
{रथस्य च शराणां च धनुषो गाण्डिवस्य च}


\twolineshloka
{मोक्ष्याम्यद्य महद्दुःखं त्रयोदशसमार्जितम्}
{हत्वा कर्णं रणे कृष्ण शम्बरं मघवानिव}


\twolineshloka
{अद्य कर्णे हते युद्धे सोमकानां महारथाः}
{कृतकार्याः प्रमोदन्तां मित्रकार्येप्सवो युधि}


\threelineshloka
{न जाने च कथं प्रीतिः शैनेयस्याद्य माधव}
{अहं हत्वा रणे कर्णं पुत्रांश्चास्य जयाधिकान्}
{प्रीतिं दास्यामि भीमस्य सात्यकेर्यमयोस्तथा}


\threelineshloka
{धृष्टद्युम्नस्य वीरस्य तथैव च शिखण्डिनः}
{अद्यानृण्यं गमिष्यामि हत्वा कर्णं महाहवे}
{धर्मराजस्य संश्रुत्य वार्ष्णेयशपथं मिथः}


\twolineshloka
{अद्य पश्यन्तु सङ्ग्रामे धनञ्जयममर्षणम्}
{युध्यन्तं कौरवैः सार्धं घातयन्तं च सूतजम्}


\twolineshloka
{भवत्समक्षमं वक्ष्यामि पुनरेवात्मसंस्तवम्}
{इत्यप्यमित्रप्रवरमद्याहं हन्मि सूतजम्}


\twolineshloka
{धनुर्वेदे मत्समः को नु लोकेपराक्रमे वा मम कोऽस्ति तुल्यः}
{को वाऽप्यन्यो मत्समोऽस्ति क्षमायांममानृशंस्ये सदृशोऽस्ति कोऽन्यः}


\twolineshloka
{अहं धनुष्मान्ससुरासुरांश्चसर्वाणि भूतानि च सङ्गतानि}
{स्वबाहुवीर्याद्गमये पराभवंमत्पौरुषं विद्वि परं परेभ्यः}


\twolineshloka
{शरार्चिषा गाण्डिवेनाहमेकःसर्वान्कुरून्बाह्लिकांश्चाभिपन्नः}
{हिमात्यये कक्षगतो यथाऽग्निःस निर्दहेयं सहसा प्रगृह्य}


\twolineshloka
{पाणी पृषत्कालिखितौ ममैतौधनुश्च सङ्ख्ये विततं सबाणम्}
{पादौ चेमौ सुरथौ सध्वजौ चन मादृशं युद्धगतं भजन्ति}


\chapter{अध्यायः ८३}
\twolineshloka
{[धृतराष्ट्र उवाच}
{}


\threelineshloka
{समागमे पाण्डवसृञ्जयानांमहाभये मामकानामगाधे}
{धनञ्जये तात रणाय यातेकर्णेन तद्युद्वमथोऽत्र कीदृक् ॥]सञ्जय उवाच}
{}


\twolineshloka
{तेषामनीकानि महाध्वजानिरणे समृद्धानि सुसङ्गतानि}
{भेरीनिरादोन्मुखराण्यंगर्ज--न्मेघा इव प्रावृषि मारुतास्ताः}


\twolineshloka
{महागजाभ्राकुलमस्त्रतोयंवादित्रनेमीतलशब्दवच्च}
{हिरण्यचित्रायुधविद्युतं चशरासिनाराचमहास्त्रधारम्}


\twolineshloka
{तद्भीमवेगं रुधिरौघवाहिखङ्गाकुलं क्षत्रियजीवघाति}
{अनार्तवं क्रूरमनिष्टवर्षंबभूव संरम्भकरं प्रजानाम्}


\twolineshloka
{एकं रथं सम्परिवार्य मृत्युंनयन्त्यनेके च रथाः समेताः}
{एकस्तथैकं रथिनं रथाग्र्यां--स्तथा रथश्चापि रथाननेकान्}


\twolineshloka
{रथं ससूतं सहयं च कञ्चि--त्कश्चिद्रती मृत्युवशं निनाय}
{निनाय चाप्येकगजेन कश्चि--द्रथान्बहून्मृत्युवशे तथाश्वान्}


\twolineshloka
{रथान्ससूतान्सहयान्गजांश्चसर्वानरीन्मृत्युवशं शरौघैः}
{निन्ये हयांश्चैव तथा ससादी--न्पदातिसङ्घांश्च तथैव पार्थः}


\twolineshloka
{कृपः शिखण्डी च रणे समेतौदुर्योधनं सात्यकिरध्यगच्छत्}
{श्रुतश्रवा द्रोणपुत्रेण सार्धंयुधामन्युश्चित्रसेनेन सार्धम्}


\twolineshloka
{कर्णस्य पुत्रं तु रथी सुषेणंसमागतं सृञ्जयश्चोत्तमौजाः}
{गान्धारराजं सहदेवोऽक्षधूर्तंमहर्षभं सिंह इवाभ्यधावत्}


\twolineshloka
{शतानीको नाकुलिः कर्णपुत्रंयुवा युवानं वृषसेनं शरौघैः}
{समार्पयत्कर्णपुत्रश्च शूरःपाञ्चालेयं शरवर्षैरनेकैः}


\twolineshloka
{रथर्षभः कृतवर्माणभार्च्छ--न्माद्रीपुत्रो नकुलश्चित्रयोधी}
{पाञ्चालानामधिपो याज्ञसेनिःसेनापतिः कर्णमार्च्छत्ससैन्यम्}


\twolineshloka
{दुःशासनो भारत भारतं तुव्यात्ताननं क्रूरमिवान्तकाभम्}
{भीमं रणे शस्त्रभृतां वरिष्ठंभीमं समार्च्छत्तमसह्यवेगम्}


\twolineshloka
{कर्णात्मजं तत्र जघान वीर--स्तथाच्छिनच्चोत्तमौजाः प्रसह्य}
{तस्योत्तमाङ्गं निपपात भूमौनिनादयद्गां निनदेन खं च}


\twolineshloka
{सुषेणशीर्षं पतितं पृथिव्यांविलोक्य कर्णोऽथ तदार्तरूपः}
{क्रोधाद्धयांस्तस्य रथं ध्वजं चबाणैः सुधारैर्निशितैरकृन्तत्}


\twolineshloka
{स तूत्तमौजा निशितैः पृषत्कै--र्विव्याध खङ्गेन च भास्वरेण}
{पार्ष्णिग्रहांश्चैव कृपस्य हत्वाशिखण्डिवाहं स ततोऽध्यरोहत्}


\twolineshloka
{कृपं तु दृष्ट्वा विरथं रथस्थोनैच्छच्छरैस्ताडयितुं शिखण्डी}
{तं द्रौणिरावार्य रथं कृपस्यसमुज्जहे पङ्कगतां यथा गाम्}


\twolineshloka
{हिरण्यवर्मा निशितैः पृषत्कै--स्तवात्मजानामनिलात्मजो वै}
{अतापयत्सैन्यमतीव भीमःकाले शुचौ मध्यगतो यथाऽर्कः}


\chapter{अध्यायः ८४}
\twolineshloka
{सञ्जय उवाच}
{}


\threelineshloka
{अथ त्विदानीं तुमुले विमर्देद्विषद्भिरेको बहुभिः समावृतः}
{महारणे सारथिमित्युवाचभीमश्चमूं वाहय धार्तराष्ट्रीम्}
{त्वं सारथे याहि जवेन वाहै--र्नयाम्येतान्धार्तराष्ट्रान्यमाय}


\twolineshloka
{सञ्चोदितो भीमसेनेन चैवंस सारथिः पुत्रबलं त्वदीयम्}
{प्रायात्ततः सत्वरमुग्रवेगोयतो भीमस्तद्बलं गन्तुमैच्छत्}


\twolineshloka
{ततोऽपरे नागरथाश्वपत्तिभिः प्रत्युद्ययुस्तं कुरवः समन्तात्}
{भीमस्य वाहाग्र्यमुदारवेगंसमन्ततो बाणगणैर्निजघ्नुः}


\twolineshloka
{ततः शरानापततो महात्माचिच्छेद बाणैस्तपीनयपुङ्खैः}
{ते वै निपेतुस्तपनीयपुङ्खाद्विधा त्रिधा भीमशरैर्निकृत्ताः}


\twolineshloka
{ततो राजन्नागरथाश्वयूनांभीमाहतानां वरराजमध्ये}
{घोरो निनादः सुमहानभूत्तदावज्राहतानामिव पर्वतानाम्}


\twolineshloka
{ते वध्यमानाश्च नरेन्द्रमुख्यानिर्भिद्यन्तो भीमशरप्रवेकैः}
{भीमं समन्तात्समरेऽभ्यरोहन्वृक्षं शकुन्ता इव पुष्पहेतोः}


\twolineshloka
{ततोऽभियाते तव सैन्ये स भीमःप्रादुश्चक्रे वेगमनन्तवेगः}
{यथाऽन्तकाले क्षपयन्दिधक्षु--र्भूतान्तकृत्काल इवात्तदण्डः}


\twolineshloka
{तस्यातिवेगस्य रणेऽतिवेगंनाशक्नुवन्धारयितुं त्वदीयाः}
{व्यात्ताननस्यापततो यथैवकालस्य काले हरतः प्रजा वै}


\twolineshloka
{ततो बलं भारत भारतानांप्रदह्यमानं समरे महात्मना}
{भीतं दिशोऽकीर्यत भीमनुन्नंमहानिलेनाभ्रगणा यथैव}


% Check verse!
ततो धीमान्सारथिमब्रवीद्बलीस भीमसेनः पुनरेव हृष्टः
\twolineshloka
{सूताभिजानीहि परान्स्वकान्वारथान्ध्वजांश्चापततः समेतान्}
{युध्यन्ह्यहं नाभिजानामि किञ्चि--न्मा सैन्यं स्वं छादयिष्ये प्रमत्तः}


\twolineshloka
{अरीन्विशोकान्हि निरीक्ष्य सर्वतोमहांश्च मन्युः पुनरेति मां भृशम्}
{राजातुरो नागमद्यत्किरीटीबहूनि दुःखान्यभिचिन्तयामि}


\twolineshloka
{एतद्दुःखं धारये धर्मराजोयन्मां हित्वा यातवाञ्शत्रुमध्ये}
{नैनं जीवं नाभिजानाम्यजीवंबीभत्सुं वा तन्ममाद्यातिदुःखम्}


\twolineshloka
{सोऽहं द्विषत्सैन्यमुदग्रकल्पंविनाशयिष्ये परमप्रतीतः}
{एतन्निहत्याजिमध्ये समेतंप्रीतो भविष्यामि सह त्वयाद्य}


\fourlineindentedshloka
{सर्वांस्तूणान्सायकानामवेक्ष्यकिं शिष्टं स्यात्सायकानां रथे मे}
{का वा जातिः किं प्रमाणं च तेषांज्ञात्वा व्यक्तं तत्समाचक्ष्व सूत}
{`कति वा सहस्राणि कति वा शतानिह्याचक्ष्व मे सारथे क्षिप्रमेव ॥विशोक उवाच}
{}


\threelineshloka
{सर्वं विदित्वैवमहं वदामितवार्थसिद्धिप्रदमद्य वीर}
{कैकेयकाम्भोजसुराष्ट्रबाह्लिकाम्लेच्छाश्च सुह्माः परतङ्कणाश्च}
{}


\twolineshloka
{मद्राश्च वङ्गा मगधाः कुणिन्दाआनर्तकावर्तकाः पार्वतीयाः}
{सर्वे गृहीतप्रवरायुधास्त्वांसंवेष्ट्य संवेष्ट्य ततो विवेदुः}


\twolineshloka
{रथे तवास्मिन्निशिताः सुपीता--स्ततो भल्ला द्वादश वै सहस्राः'}
{षण्मार्गणानामयुतानि वीरक्षुराश्च भल्लाश्च तथायुताख्याः}


\twolineshloka
{नाराचानां द्वे सहस्रे च वीरत्रीण्येप च प्रदराणां स्म पार्थ}
{अस्त्यायुधं पाण्डवेयावशिष्टंन यद्वहेच्छकटं षङ्गवीयम्}


\threelineshloka
{एतद्विद्वन्मुञ्च सहस्रशोऽपिगदासिबाहुद्रविणं च तेऽस्ति}
{प्रासाश्च मुद्रराः शक्तयस्तोमराश्चमाभैषीस्त्वं संक्षयादायुधानाम् ॥भीमसेन उवाच}
{}


\threelineshloka
{`अद्यैव नूनं कथयन्तु सिद्धांमम प्रतिज्ञां सर्वलोके विशोक}
{न मोक्ष्यते समरे भीमसेनएकः शत्रून्समरे वाप्यजैषीत्}
{आशंसितानामिदमेकमस्तुतन्मे देवाः सकलं साधयन्तु ॥'}


\twolineshloka
{सूताद्य मद्बाहुयुतैः समस्ता--न्समाहनद्भिः पार्थिवानाशुवेगैः}
{च्छन्नं बाणैराहवं घोररूपंनष्टादित्यं मृत्युलोकेन तुल्यम्}


\twolineshloka
{अद्यैतद्यै विदितं पार्थिवानांभविष्यति ह्याकुमारं च सूत}
{निमग्नो वा समरे भीमसेनएकः कुरून्वा समरे व्यजैषीत्}


\twolineshloka
{सर्वे सङ्ख्ये कुरवो निष्पतन्तुमां वा लोकाः कीर्तयन्त्वाकुमारम्}
{सर्वानेकस्तानहं पातयिष्येते वा सर्वे भीमसेनं तुदन्तु}


\twolineshloka
{आशास्तारः कर्म चाप्युत्तमं येतन्मे देवाः सफलं साधयन्तु}
{आयातीह केशवसारथी रथीशक्रस्तूर्णं यज्ञ इवोपहूतः}


\twolineshloka
{ईक्षस्वैतां भारतीं दीर्यमाणा--मेते कस्माद्विद्रवन्ते नरेन्द्राः}
{व्यक्तं धीमान्सव्यसाची नराग्र्यःसैन्यं ह्येतच्छादयत्याशु बाणैः}


\twolineshloka
{पश्य ध्वजांश्च द्रवतो विशोकनागान्हयान्पत्तिसङ्घांश्च सङ्ख्ये}
{रथान्विकीर्णाञ्शरशक्तिताडिता--न्पश्यस्वैतान्रथिनश्चैव सूत}


\twolineshloka
{आपूर्यते कौरवी चाप्यभीक्ष्णंसेना ह्यसौ सुभृशं हन्यमाना}
{धनञ्जयस्याशनितुल्यवेगै--र्ग्रस्ता शरैः काञ्चनबर्हजालैः}


\twolineshloka
{एते द्रवन्ति स्म रथाश्वनागाःपदातिसङ्घानतिमर्दयन्तः}
{सम्मुह्यमानाः कौरवाः सर्व एवद्रवन्ति नागा इव दाहभीताः}


\twolineshloka
{हाहाकृताश्चैव रणे विशोकमुञ्चन्ति नादान्विपुलान्गजेन्द्राः ॥विशोक उवाच}
{}


\twolineshloka
{किं भीम नैनं त्वमिहाशृणोषिविस्फारितं गाण्डिवस्यातिघोरम्}
{क्रुद्धेन पार्थेन विकृष्यतोऽद्यकच्चिन्नेमौ तव कर्णौ विनष्टौ}


\twolineshloka
{सर्वे कामाः पाण्डव ते समृद्धाःकपिर्ह्यसौ दृश्यते हस्तिसैन्ये}
{नीलाद्धनाद्विद्युतमुच्चरन्तींतथा पश्य विस्फुरन्तीं धनुर्ज्याम्}


\twolineshloka
{कपिर्ह्यसौ वीक्षते सर्वतो वैध्वजाग्रमारुह्य धनञ्जयस्य}
{वित्रासयन्रि पुसङ्घान्विमर्देबिभेम्यस्मादात्मनैवाभिवीक्ष्य}


\twolineshloka
{विभ्राजते चातिमात्रं किरीटंविचित्रमेतच्च धनञ्जयस्य}
{दिवाकराभो मणिरेष दिव्योविभ्राजते चैव किरीटसंस्थः}


\twolineshloka
{पार्श्वे भीमं पाण्डुराभ्रप्रकाशंपश्यस्व सङ्खं देवदत्तं सुघोषम्}
{अभीषुहस्तस्य जनार्दनस्यविगाहमानस्य चमूं परेषाम्}


\twolineshloka
{रविप्रभं वज्रनाभं क्षुरान्तंपार्श्वे स्थितं पश्य जनार्दनस्य}
{चक्रं यशोवर्धनं केशवस्यसदार्चितं यदिभिः पश्य वीर}


\twolineshloka
{महाद्विपानां सरलद्रुमोपमाःकरा निकृत्ताः प्रपतन्त्यमी क्षुरैः}
{किरीटिना तेन पुनः ससादिनःशरैर्निकृत्ताः कुलिशैरिवाद्रयः}


\twolineshloka
{तथैव कृष्णस्य च पाञ्चजन्यंमहार्हमेतं द्विजराजवर्णम्}
{कौन्तेय पश्योरसि कौस्तुभं चजाज्वल्यमानं विजयां स्रजं च}


\twolineshloka
{ध्रवं रथाग्र्यः समुपैति पार्थोविद्रावयन्सैन्यमिदं परेषाम्}
{सिताभ्रवर्णैरसितप्रयुक्तै--र्हयैर्महार्है रथिनां वरिष्ठः}


\twolineshloka
{रथान्हयान्पत्तिगणांश्च सायकै--र्विदारितान्पश्य पतन्त्यमी यथा}
{तवानुजेनामरराजतेजसामहावनानीव सुपर्णवायुना}


\twolineshloka
{चतुः शतान्पश्य रथानिमान्हतान्सवाजिसूतान्समरे किरीटिना}
{महेषुभिः सप्तशतानि दन्तिनां पदातिसादींश्च रथाननेकशः}


\threelineshloka
{अयं समभ्येति तवान्तिकं बलीनिघ्नन्कुरूंश्चित्र इव ग्रहोऽर्जुनः}
{समृद्धकामोऽसि हतास्तवाहिताबलं तवायुश्च चिराय वर्धताम् ॥भीमसेन उवाच}
{}


\twolineshloka
{ददानि ते ग्रामवरांश्चतुर्दशप्रियाख्याने सारथे सुप्रसन्नः}
{दासीशतं चापि रथांश्च विंशतिंयदर्जुनं वेदयसे विशोक}


\chapter{अध्यायः ८५}
\twolineshloka
{सञ्जय उवाच}
{}


\twolineshloka
{श्रुत्वा तु रथनिर्घोषं सिंहनादं च संयुगे}
{अर्जुनः प्राह गोविन्दं शीघ्रं वाहय वाजिनः}


\twolineshloka
{अर्जुनस्य वचः श्रुत्वा गोविन्दोऽर्जुनमब्रवीत्}
{एष गच्छामि सुक्षिप्रं यत्र भीमो व्यवस्थितः}


\twolineshloka
{तं यान्तमश्वा हिमशङ्खवर्णाःसुवर्णमुक्तामणिजालनद्धाः}
{जम्भं जिघांसुं प्रगृहीतवज्रंहया यथा तत्र यथा वहंस्तदा}


\twolineshloka
{रथाश्वमातङ्गपदातिसङ्घाबाणस्वनैर्नेमिखुरस्वनैश्च}
{सन्नादयन्तो वसुधां दिशश्चक्रुद्धा नृसिंहा जयमभ्युदीयः}


\twolineshloka
{तेषां च पार्थस्य च मारिषासी--द्देहासुपापक्षपणं सुयुद्धम्}
{त्रैलोक्यहेतोरसुरैर्यथासी--द्देवस्य विष्णोर्जयतां वरस्य}


\twolineshloka
{तैरस्तमुच्चावचमायुधं त--देकः प्रचिच्छेद किरीटमाली}
{क्षुरार्धचन्द्रैर्निशितैश्च भल्लैःशिरांसि तेषां बहुधा च बाहून्}


\twolineshloka
{छत्राणि वालव्यजनानि केतू--नश्वान्रथान्पत्तिगणान्द्विपांश्च}
{ते पेतुरुर्व्यां बहुधा विकृत्तावातप्रणुन्नानि यथा वनानि}


\twolineshloka
{सुवर्णजालावनता महागजाःसवैजयन्तीध्वजयोधकल्पिताः}
{सुवर्णपुङ्घैरिषुभिः समाचिता--शकाशिरे प्रज्वलिता यथाचलाः}


\twolineshloka
{विदार्य नागाश्वरथान्धञ्जयःशरोत्तमैर्वासववज्रसन्निभैः}
{द्रुतं ययौ कर्णजिघांसया तथायथा मरुत्वान्बलभेदने पुरा}


\twolineshloka
{ततः स पुरुषव्याघ्रस्तव सैन्यमरिन्दमः}
{प्रविवेश महाबाहुर्मकरः सागरं यथा}


\twolineshloka
{तं हृष्टास्तावका राजन्रथपत्तिसमन्विताः}
{गजाश्वसादिबहुलाः पाण्डवं समुपाद्रवन्}


\twolineshloka
{तेषामापततां पार्थमारावः सुमहानभूत्}
{सागरस्येव क्षुब्धस्य यथा स्यात्सलिलस्वनः}


\twolineshloka
{ते तु तं पुरुषव्याघ्रं व्याघ्रा इव महारथाः}
{अभ्यद्रवन्त सङ्ग्रामे त्यक्त्वा प्राणकृतं भयम्}


\twolineshloka
{तेषामापततां तत्र शरवर्षाणि मुञ्चताम्}
{अर्जुनो व्यधमत्सैन्यं महावातो घनानिव}


\twolineshloka
{तेऽर्जुनं सहिता भूत्वा रथवंशैः ग्रहारिणः}
{अभियाय महेष्वासा विव्यधुर्निशितैः शरैः}


\twolineshloka
{ततोऽर्जुनः सहस्राणि रथवारणवाजिनाम्}
{प्रेषयामास विशिखैर्यमस्य सदनं प्रति}


\twolineshloka
{ते वध्यमानाः समरे पार्थचापच्युतैः शरैः}
{तत्रतत्र स्म लीयन्ते भये जाते महारथाः}


\twolineshloka
{तेषां चतुःशतान्वीरान्यतमानान्महारथान्}
{अर्जुनो निशितैर्बाणैरनयद्यमसादनम्}


\twolineshloka
{ते वध्यमानाः समरे नानालिङ्गैः शितैः शरैः}
{अर्जुनं समभित्यज्य दुद्रुवुर्वै दिशो दश}


\twolineshloka
{तेषां शब्दो महानासीद्द्रवतां वाहिनीमुखे}
{मेघौघस्येव भद्रं ते गिरिमासाद्य दीर्यतः}


\twolineshloka
{तां तु सेनां भृशं विद्ध्वा द्रावयित्वाऽर्जुनः शरैः}
{प्रायादभिमुखः पार्थः सूतानीकं हि मारिष}


\twolineshloka
{तस्य शब्दो महानासीत्परानभिमुखस्य वै}
{गरुडस्येव पततः पन्नगार्थे यथा पुरा}


\twolineshloka
{तं तु शब्दमभिश्रुत्य भीमसेनो महाबलः}
{बभूव परमप्रीतः पार्थदर्शनलालसः}


\twolineshloka
{श्रुत्वैव पार्थमायान्तं भीमसेनः प्रतापवान्}
{त्यक्त्वा प्राणान्महाराज सेनां तव ममर्द ह}


\twolineshloka
{स वायुवीर्यप्रतिमो वायुवेगसमो जवे}
{वायुवद्व्यचरद्भीमो वायुपुत्रः प्रतापवान्}


\twolineshloka
{तेनाद्यमाना राजेन्द्र सेना तव विशाम्पते}
{व्यभ्रश्यत महाराज भिन्ना नौरिव सागरे}


\twolineshloka
{तां तु सेनां तदा भीमो दर्शयन्पाणिलाघवम्}
{शरैरवचकर्तोग्रैः प्रेपयिष्यन्यमक्षयम्}


\twolineshloka
{तत्र भारत भीमस्य बलं दृष्ट्वातिमानुषम्}
{व्यत्रस्यन्त रणे योधाः कालस्येव युगक्षये}


\twolineshloka
{तथाऽर्दितान्भीमबलान्भीमसेनेन भारत}
{दृष्ट्वा दुर्योधनो राजा इदं वचनमब्रवीत्}


\twolineshloka
{सैनिकांश्च महेष्वासान्योधांश्च भरतर्षभ}
{समादिशन्रणे सर्वान्हत भीममिति स्म ह}


% Check verse!
तस्मिन्हते हतं मन्ये पाण्डुसैन्यमशेषतः
\twolineshloka
{प्रतिगृह्य च तामाज्ञां तव पुत्रस्य पार्थिवाः}
{भीमं प्रच्छादयामासुः शरवर्षैः समन्ततः}


\twolineshloka
{गजाश्च बहुला राजन्नराश्च जयगृद्धिनः}
{रथे स्थिताश्च राजेन्द्र परिवव्रुर्वृकोदरम्}


\twolineshloka
{स तैः परिवृतः शूरैः शूरो राजन्समन्ततः}
{शुशुभे भरतश्रेष्ठो नक्षत्रैरिव चन्द्रमाः}


\threelineshloka
{परिवेषी यथा सोमः परिपूर्णो विराजते}
{स रराज तथा सङ्ख्ये दर्शनीयो नरोत्तमः}
{निर्विशेषो महाराज विजयेनैव संयुगे}


\twolineshloka
{तस्य ते पार्थिवाः सर्वे शरवृष्टिं समासृजन्}
{क्रोधरक्तेक्षणाः शूरा हन्तुकामा वृकोदरम्}


\twolineshloka
{तां विदार्य महासेनां शरैः सन्नतपर्वभिः}
{निश्चक्राम रणाद्भीमो मत्स्यो जालादिवाम्भसि}


\twolineshloka
{हत्वा दशसहस्राणि गजानामनिवर्तिनाम्}
{नृणां शतसहस्रे द्वे द्वे शते चैव भारत}


\twolineshloka
{पञ्च चाश्वसहस्राणि रथानां शतमेव च}
{हत्वा प्रास्यन्दयद्भीमो नदीं शोणितवाहिनीम्}


\twolineshloka
{शोणितोदां रथावर्तां हस्तिग्राहसमाकुलाम्}
{नरमीनाश्वनक्रान्तां केशशैवलशाद्वलाम्}


\twolineshloka
{सञ्छिन्नभुजनागेन्द्रां बहुरत्नापहारिणीम्}
{ऊरुग्राहां मज्जपङ्कां शीर्षोपलसमावृताम्}


\twolineshloka
{धनुःकाशां शरावापां गदापरिघकेतनाम्}
{हंसछत्रध्वजोपेतामुष्णीषवरफेनिलाम्}


\twolineshloka
{हारपद्माकरां चैव भूमिरेणूर्मिमालिनीम्}
{आर्यवृत्तवतीं सङ्ख्ये सुतरां भीरुदुस्तराम्}


\twolineshloka
{योधग्राहवतीं सङ्ख्ये वहन्तीं पितृसादनम्}
{क्षणेन पुरुषव्याघ्रः प्रावर्तयत निम्नगाम्}


\twolineshloka
{यथा वैतरणीमुग्रां दुस्तरामकृतात्मभिः}
{तथा दुस्तरणीं घोरां भीरूणां भयवर्धनीम्}


\twolineshloka
{यतोयतः पाण्डवेयः प्रविष्टो रथसत्तमः}
{ततस्ततः प्रापतन्वै योधाः शतसहस्रशः}


\twolineshloka
{एवं दृष्ट्वा कृतं कर्म भीमसेनेन संयुगे}
{दुर्योधनो महाराज शकुनिं वाक्यमब्रवीत्}


\twolineshloka
{जहि मातुल सङ्ग्रामे भीमसेनं महाबलम्}
{अस्मिञ्जिते जितं मन्ये पाण्डवेयं महाबलम्}


\twolineshloka
{ततः प्रायान्महाराज सौबलेयः प्रतापवान्}
{रणाय महते युक्तो भ्रातृभिः परिवारितः}


\twolineshloka
{स समासाद्य सङ्ग्रामे भीमं भीमपराक्रमम्}
{वारयामास तं वीरो वेलेव मकरालयम्}


\threelineshloka
{सन्न्यवर्तत तं भीमो वार्यमाणः शितैः शरैः}
{शकुनिस्तस्य राजेन्द्र वामपार्श्वे स्तनान्तरे}
{प्रेषयामास नाराचान्रुक्मपुङ्खाञ्शिलाशितान्}


\twolineshloka
{वर्म भित्त्वा तु ते घोराः पाण्डवस्य महात्मनः}
{न्यम़ञ्जन्त महाराज कङ्कबर्हिणवाससः}


\twolineshloka
{सोऽतिविद्धो रणे भीमः शरं रुक्मविभूषितम्}
{प्रेषयामास स रुषा सौबलं प्रति भारत}


\twolineshloka
{तमायान्तं शरं घोरं शकुनिः शत्रुतापनः}
{चिच्छेद सप्तधा राजन्कृतहस्तो महाबलः}


\twolineshloka
{तस्मिन्निपतिते भूमौ भीमः क्रुद्धो विशाम्पते}
{धनुश्छिच्छेद भल्लेन सौबलस्य हसन्निव}


\threelineshloka
{तदपास्य धनुश्छिन्नं सौबलेयः प्रतापवान्}
{अन्यदादाय वेगेन धनुर्भल्लांश्च षोडश}
{तैस्तस्य तु महाराज भल्लैः सन्नतपर्वभिः}


\threelineshloka
{द्वाभ्यां स सारथिं ह्यार्च्छद्भीमं सप्तभिरेव च}
{ध्वजमेकेन चिच्छेद द्वाभ्यां छत्रं विशाम्पते}
{चतुर्भिश्चतुरो वाहान्विव्याध सुबलात्मजः}


\twolineshloka
{ततः क्रुद्धो महाराज भीमसेनः प्रतापवान्}
{शक्तिं चिक्षेप समरे रुक्मदण्डामयस्मयीम्}


\twolineshloka
{सा भीमभुजनिर्मुक्ता नागजिह्वेन चञ्चला}
{निपपात रणे तूर्णं सौबलस्य महात्मनः}


\twolineshloka
{ततस्तामेव संगृह्य शक्तिं कनकभूषणाम्}
{भीमसेनाय चिक्षेप क्रुद्धरूपो विशाम्पते}


\twolineshloka
{सा निर्भिद्य भुजं सव्यं पाण्डवस्य महात्मनः}
{निपपात तदा भूमौ यथा विद्युन्नभश्च्युता}


\twolineshloka
{अथोत्क्रुष्टं महाराज धार्तराष्ट्रैः समन्ततः}
{न तु तं ममृषे भीमः सिंहनादं तरस्विनाम्}


\threelineshloka
{अन्यद्गृह्य धनुः सज्यं त्वरमाणो महाबलः}
{मुहूर्तादिव राजेन्द्र च्छादयामास सायकैः}
{सौबलस्य बलं सङ्ख्ये त्यक्त्वा नादं महाबलः}


\twolineshloka
{तस्याश्वांश्चतुरो हत्वा सूतं चैव विशाम्पते}
{ध्वजं चिच्छेद भल्लेन त्वरमाणः पराक्रमी}


\twolineshloka
{हताश्वं रथमुत्सृज्य त्वरमाणो नरोत्तमः}
{तस्थौ विस्फारयंश्चापं क्रोधरक्तेक्षणः श्वसन्}


% Check verse!
शरैश्च बहुधा राजन्भीममार्च्छत्समन्ततः
\twolineshloka
{प्रतिहत्य तु वेगेन भीमसेनः प्रतापवान्}
{धनुश्चिच्छेद सङ्क्रुद्धो विव्याध च शितैः शरैः}


\twolineshloka
{सोऽतिविद्धो बलवता शत्रुणा शत्रुकर्शनः}
{निपपात तदा भूमौ किञ्चित्प्राणो नराधिपः}


\twolineshloka
{ततस्तं विह्वलं ज्ञात्वा पुत्रस्तव विशाम्पते}
{अपोवाह रथेनाजौ भीमसेनस्य पश्यतः}


\twolineshloka
{शकुनिं विह्वलं दृष्ट्वा धार्तराष्ट्राः पराङ्मुखाः}
{प्रदुद्रुवुर्दिशो भीता भीमसेनभयात्प्रभो}


\threelineshloka
{सौबले निर्जिते राजन्भीमसेनेन धन्विना}
{भयेन महताऽऽविष्टः पुत्रो दुर्योधनस्तव}
{अपायाज्जवनैरश्वैः सापेक्षो मातुलं प्रति}


\twolineshloka
{पराङ्मुखं तु राजानं दृष्ट्वा सैन्यानि भारत}
{विप्रजग्मुः समुत्सृज्य द्वैरथानि समन्ततः}


\twolineshloka
{तान्दृष्ट्वा विद्रुतान्सर्वान्धार्तराष्ट्रान्पराङ्मुखान्}
{जवेनाभ्यापतद्भीमः किरञ्शरशतान्बहून्}


\twolineshloka
{ते वध्यमाना भीमेन धार्तराष्ट्राः पराङ्मुखाः}
{कर्णमासाद्य समरे स्थिता राजन्समन्ततः}


\twolineshloka
{स हि तेषां महावीर्यो द्वीपोऽभूत्सुमहाबलः}
{भिन्ननौका यथा राजन्द्वीपमासाद्य निर्वृताः}


\twolineshloka
{भवन्ति पुरुषव्याघ्र नाविकाः कालपर्यये}
{तथा कर्णं समासाद्य तावकाः पुरुषर्षभ}


\twolineshloka
{समाश्वस्ताः स्थिता राजन्सम्प्रहृष्टाः परस्परम्}
{समाजग्मुश्च युद्धाय मृत्युं कृत्वा निवर्तनम्}


\chapter{अध्यायः ८६}
\twolineshloka
{धृतराष्ट्र उवाच}
{}


\twolineshloka
{ततो भग्नेषु सैन्येषु भीमसेनेन संयुगे}
{दुर्योधनोऽब्रवीत्किं नु सौबलो वाऽपि सञ्जय}


\twolineshloka
{कर्णो वा जयतां श्रेष्ठो योधा वा मामका युधि}
{कृपो वा कृतवर्मा वा द्रौणिर्दुःशासनोऽपि वा}


\twolineshloka
{अत्यद्भुतमहं मन्ये पाण्डवेयस्य विक्रमम्}
{यदेकः समरे सर्वान्योधयामास मामकान्}


\threelineshloka
{यथाप्रतिज्ञं योधानां राधेयः कृतवानपि}
{कुरूणामथ सर्वेषां कर्णः शत्रुनिषूदनः}
{शर्म वर्म प्रतिष्ठा च जीविताशा च सञ्जय}


\twolineshloka
{तद्भग्रं स्वबलं दृष्ट्वा कौन्तेयेनामितौजसा}
{धनुर्दराणां प्रवरः कर्णः किमकरोद्युधि}


\threelineshloka
{पुत्रा वा मम दुर्धर्षा राजानो वा महारथाः}
{एतन्मे सर्वमाचक्ष्व कुशलो ह्यसि सञ्जय ॥सञ्जय उवाच}
{}


\twolineshloka
{अपराह्णे महाराज सूतपुत्रः प्रतापवान्}
{जघान सोमकान्सर्वान्भीमसेनस्य पश्यतः}


% Check verse!
भीमोऽप्यतिबलं सैन्यं धार्तराष्ट्रं व्यपोथयत्
\twolineshloka
{द्राव्यमाणं बलं दृष्ट्वा भीमसेनेन धीमता}
{यन्तारमब्रवीत्कर्णः पाञ्चालानेव मां वह}


\twolineshloka
{मद्रराजस्ततः शल्यः श्वेतानश्वान्महाजवान्}
{प्राहिणोच्चेदिपाञ्चालान्करूशांश्च महाबलः}


\twolineshloka
{प्रविश्य च महत्सैन्यं शल्यः परबलार्दनः}
{न्ययच्छत्तुरगान्हृष्टो यत्रयत्र च ते रथाः}


\twolineshloka
{तं रथं मेघसङ्काशं वैयाघ्रपरिवारणम्}
{संदृश्य पाण्डुपञ्चालास्त्रस्ता ह्यासन्विशाम्पते}


\twolineshloka
{ततो रथस्य निनदः प्रादुरासीन्महारणे}
{पर्जन्यसमनिर्घोषः पर्वतस्येव दीर्यतः}


\twolineshloka
{ततः शरशतैस्तीक्षणैः कर्ण आकर्णनिःसृतैः}
{जघान पाण्डवबलं शतशोऽथ सहस्रशः}


\twolineshloka
{तं तथा समरे कर्म कुर्वाणमपराजितम्}
{परिवव्रुर्महेष्वासाः पाण्डवानां महारथाः}


\twolineshloka
{तं शिखण्डी च भीमश्च धृष्टद्युम्नश्च पार्षतः}
{नकुलः सहदेवश्च द्रौपदेयाश्च सात्यकिः}


\twolineshloka
{परिवव्रुर्जिघांसन्तो राधेयं शरवृष्टिभिः}
{सात्यकिस्तु तदा कर्णं विंशत्या निशितैः शरैः}


\twolineshloka
{अताडयद्रणे शूरो जत्रुदेशे नरोत्तमः}
{शिखण्डी पञ्चविंशत्या धृष्टद्युम्नश्च सप्तभिः}


\twolineshloka
{द्रौपदेयाश्चतुःषष्ट्या सहदेवश्च सप्तभिः}
{नकुलश्च शतेनाजौ कर्णं विव्याध सायकैः}


\twolineshloka
{भीमसेनस्तु राधेयं नवत्या नतपर्वणाम्}
{विव्याध समरे क्रुद्धो जत्रुदेशे महाबलः}


\twolineshloka
{अथ प्रहस्याधिरथिर्व्याक्षिपद्धनुरुत्तमम्}
{मुमोच निशितान्बाणान्पीडयन्सुमहाबलः}


\twolineshloka
{तान्प्रत्यविध्यद्राधेयः पञ्चभिः पञ्चभिः शरैः}
{सात्यकेस्तु धनुश्छित्त्वा ध्वजं च भरतर्षभ}


\twolineshloka
{तं तथा नवभिर्बाणैराजघान स्तनान्तरे}
{भीमसेनं ततः क्रुद्धो विव्याध त्रिंशता शरैः}


\twolineshloka
{सहदेवस्य भल्लेन ध्वजं चिच्छेद मारिष}
{सारथिं च त्रिभिर्बाणैराजघान परन्तपः}


\twolineshloka
{विरथान्द्रौपदेयांश्च चकार भरतर्षभ}
{अक्ष्णोर्निमेषमात्रेण तदद्भुतमिवाभवत्}


\twolineshloka
{विमुखीकृत्य तान्सर्वाञ्शरैः सन्नतपर्वभिः}
{पाञ्चालानहनञ्छूरांश्चेदीनां च महारथान्}


\twolineshloka
{ते वध्यमानाः समरे चेदिमत्स्या विशाम्पते}
{कर्णमेकमभिद्रुत्य शरसङ्खैः समार्पयन्}


\threelineshloka
{ताञ्जघान शितैर्बाणैः सूतपुत्रो महारथः}
{ते वध्यमानाः समरे चेदिमात्स्या विशाम्पते}
{प्राद्रवन्त रणे भीताः सिंहत्रस्ता मृगा इव}


\twolineshloka
{एतदत्युद्भतं कर्म दृष्ट्वानस्मि भारत}
{यदेकः समरे शूरान्सूतपुत्रः प्रतापवान्}


\twolineshloka
{यतमानान्परं शक्त्या योधयानांश्च धन्विनः}
{पाण्डवेयान्महाराज शरैर्वारितवान्रणे}


\twolineshloka
{तत्र भारत कर्णस्य लाघवेन महात्मनः}
{तुतुषुर्देवताः सर्वाः सिद्धाश्च सह चारणैः}


\twolineshloka
{अपूजयन्महेष्वासा धार्तराष्ट्रा नरोत्तमम्}
{कर्णं रथवरश्रेष्ठं श्रेष्ठं सर्वधनुष्मताम्}


\twolineshloka
{ततः कर्णो महाराज ददाह रिपुवाहिनीम्}
{कक्षमिद्धो यथा वह्निर्निदाधे ज्वलितो महान्}


\twolineshloka
{ते वध्यमानाः कर्णेन पाण्डवेयास्ततस्ततः}
{प्राद्रवन्त रणे भीताः कर्णं दृष्ट्वा महारथम्}


\twolineshloka
{तत्राक्रन्दो महानासीत्पाञ्चालानां महारणे}
{वध्यतां सायकैस्तीक्ष्णैः कर्णचापवरच्युतैः}


\twolineshloka
{तेन शब्देन वित्रस्ता पाण्डवानां महाचमूः}
{कर्णमेकं रणे योधं मेनिरे तत्र शात्रवाः}


\twolineshloka
{तत्राद्भुतं पुनश्चक्रे राधेयः शत्रुकर्शनः}
{यदेनं पाण्डवाः सर्वे न शेकुरभिवीक्षितुम्}


\twolineshloka
{जलौघः पर्वतश्रेष्ठं यथासाद्य प्रभिद्यते}
{तथा तत्पाण्डवं सैन्यं कर्णमासाद्य दीर्यते}


\twolineshloka
{कर्णोऽपि समरे राजन्विधूमोऽग्निरिव ज्वलत्}
{दहंस्तस्थौ महाबाहुः पाण्डवानां महाचमूम्}


\twolineshloka
{शिरांसि च महाराज कर्णां श्चैव सकुण्डलान्}
{बाहूंश्च वीरो वीराणां चिच्छेद लघु चेषुभिः}


\twolineshloka
{हस्तिदन्तत्सरून्खङ्गान्ध्वजाञ्शक्तीर्हयान्गजान्}
{रथांश्च विविधान्राजन्पताका व्यजनानि च}


\twolineshloka
{अक्षं च युगयोक्त्राणि चक्राणि विविधानि च}
{चिच्छेद बहुधा कर्णो योधव्रतमनुष्ठितः}


\twolineshloka
{तत्र भारत कर्णेन निहतैर्गजवाजिभिः}
{अगम्यरूपा पृथिवी मांसशोणितकर्दमा}


\twolineshloka
{विषमं च समं चैव हतैरश्वपदातिभिः}
{रथैश्च कुञ्जरैश्चैव न प्राज्ञायत किञ्चन}


\twolineshloka
{नापि स्वे न परे योधाः प्राज्ञायन्त परस्परम्}
{घोरे शरान्धकारे तु कर्णास्त्रे च विजृम्भिते}


\twolineshloka
{राधेयचापनिर्मुक्तैः शरैः काञ्चनभूषणैः}
{सञ्छादिता महाराज पाण्डवानां महारथाः}


\twolineshloka
{ते पाण्डवेयाः समरे राधेयेन पुनः पुनः}
{अभज्यन्त महाराज यतमाना महारथाः}


\twolineshloka
{मृगसङ्घान्यथा क्रुद्धः सिंहो द्रावयते वने}
{पञ्चालानां रथश्रेष्ठान्द्रावयञ्शात्रवांस्तथा}


\twolineshloka
{कर्णस्तु समरे योधांस्त्रासयुन्सुमहायशाः}
{कालयामास तत्सैन्यं यथा पशुगणान्वृकः}


\twolineshloka
{दृष्ट्वा तु पाण्डवीं मेनां धार्तराष्ट्राः पराङ्युखीम्}
{तत्राजग्मुर्महेष्वासा रुदन्तो भैरवात्रवान्}


\twolineshloka
{दुर्योधनो हि राजेन्द्र मुदा परमया युतः}
{वादयामास संहृष्टो नानावाद्यानि सर्वशः}


\twolineshloka
{पाञ्चालाश्च महेष्वासा भग्नास्तत्र नरोत्तमाः}
{न्यवर्तन्त तदा शूरा मृत्युं कृत्वा निवर्तनम्}


\twolineshloka
{तान्निवृत्तान्रणे शूरान्राधेयः शत्रुतापनः}
{अनेकशो महाराज बभञ्ज पुरुषर्षभः}


\twolineshloka
{तत्र भारत कर्णेन पाञ्चाला विंशती रथाः}
{निहताः सायकैः क्रोधाच्चेदयश्च परंशताः}


\twolineshloka
{कृत्वा शून्यान्रथोपस्थान्वाजिपृष्ठांश्च भारत}
{निर्मनुष्यान्गजस्कन्धान्पादातांश्चैव विद्रुतान्}


\twolineshloka
{आदित्य इव मध्याह्ने दुर्निरीक्ष्यः परन्तपः}
{कालान्तकवपुः शूरः सूतपुत्रोऽभ्यराजत}


\twolineshloka
{एवमेतन्महाराज नरवाजिरथद्विपान्}
{हत्वा तस्थौ महेष्वासः कर्णोऽरिगणसूदनः}


\twolineshloka
{यथा भूतगणान्हत्वा कालस्तिष्ठेन्महाबलः}
{तथा स सोमकान्हत्वा तस्थावेको महारथः}


\twolineshloka
{तत्राद्भुतमपश्याम पाञ्चालानां पराक्रमम्}
{वध्यमानाऽपि यत्कर्णं नाजहू रणमूर्धनि}


\threelineshloka
{राजा दुःशासनश्चैव कृपः शारद्वतस्तथा}
{अश्वत्थामा कृतवर्मा शकुनिश्च महाबलः}
{व्यहनन्पाण्डवीं सेनां शतशोऽथ सहस्रशः}


\twolineshloka
{कर्मपुत्रौ तु राजेन्द्र भ्रातरौ सत्यविक्रमौ}
{निजघ्नाते बलं क्रुद्धौ पाण्डवानामितस्ततः}


% Check verse!
तत्र युद्धं महच्चासीत्क्रूरं विशसनं महत्
\twolineshloka
{तथैव पाण्डवाः शूरा धृष्टद्युम्नशिखण्डिनौ}
{द्रौपदेयाश्च सङ्क्रुद्धा अभ्यघ्नंस्तावकं बलम्}


\twolineshloka
{एवमेष क्षयो वृत्तः पाण्डवानां ततस्ततः}
{तावकानामपि रणे भीमं प्राप्य महाबलम्}


\chapter{अध्यायः ८७}
\twolineshloka
{सञ्जय उवाच}
{}


\twolineshloka
{हत्वा तु फल्गुनः सेनां कौरवाणां पृथक्पृथक्}
{सूतपुत्रस्य संरम्भं दृष्ट्वा चैव महारणे}


\twolineshloka
{शोणितोदां नदीं कृत्वा मांसमज्जास्थिपङ्किलाम्}
{मनुष्यशीर्षपाषाणां हस्त्यश्वकृतरोधसम्}


\twolineshloka
{शूरास्थित्तयसङ्कीर्णां काकगृध्रानुनादिताम्}
{छत्रहंसप्लुवोपेतां वीरवृक्षापहारिणीम्}


\twolineshloka
{हारपद्माकरवतीमुष्णीषवरफेनिलाम्}
{धनुःशरध्वजोपेतां नरक्षुद्रकपालिनीम्}


\twolineshloka
{चर्मवर्मभ्रमोपेतां रथोडुपसमाकुलाम्}
{जयैषिणां च सुतरां भीरूणां च सुदुस्तराम्}


\threelineshloka
{नदीं प्रावर्तयित्वा च बीभत्सुः परवीरहा}
{वासुदेवमिदं वाक्यमब्रवीत्पुरुषर्षभः ॥अर्जुन उवाच}
{}


\twolineshloka
{एष केतू रणे कृष्ण सूतपुत्रस्य दृश्यते}
{भीमसेनादयश्चैते योधयन्ति महारथम्}


\twolineshloka
{एते द्रवन्ति पाञ्चालाः कर्मत्रस्ता जनार्दन}
{एष दुर्योधनो राजा श्वेतच्छत्रेण धार्यते}


\twolineshloka
{कर्णेन भग्नान्पाञ्चालान्द्रावयन्बहु शोभते}
{कृपश्च कृतवर्मा च द्रौणिश्चैव महारथः}


\twolineshloka
{एते रक्षन्ति राजानं सूतपुत्रेण रक्षिताः}
{अवध्यमानास्तेऽस्माभिर्योधयिष्यन्ति सोमकान्}


\twolineshloka
{एष शल्यो रथोपस्थे रश्मिसङ्ग्राहकोविदः}
{सूतपुत्ररथं कृष्ण वाहयन्बहु शोभते}


\twolineshloka
{तत्र मे बुद्धिरुत्पन्ना वाहयात्र रथं मम}
{नाहत्वा समरे कर्णं निवर्तिष्ये कथञ्चन}


\threelineshloka
{मा स्म कर्णो रणे पार्थान्सृञ्जयांश्च महारथान्}
{निःशेषान्समरे कुर्यात्पश्यतां नो जनार्दन ॥सञ्जय उवाच}
{}


\twolineshloka
{ततः प्रायाज्जवेनाशु केशवस्तव वाहिनीम्}
{कर्णं प्रति महेष्वासं द्वैरथे सव्यसाचिनः}


\twolineshloka
{स प्रयातो रथेनाशु कृष्णो राजन्महाहवे}
{आश्वासयन्रणे चाशु पाण्डुसैन्यानि सर्वशः}


\twolineshloka
{रथघोषस्ततस्तस्य पाण्डवस्य बभूव ह}
{वासवास्त्रनिपातेन पर्वतेष्विव मारिष}


\twolineshloka
{महता रथघोषेण पाण्डवः सत्यविक्रमः}
{अभ्ययादप्रमेयात्मा निर्जयंस्तव वाहिनीम्}


\twolineshloka
{तमायान्तं समीक्ष्यैव श्वेताश्वं कृष्णसारथिम्}
{मद्रराजोऽब्रवीत्कर्णं केतुं दृष्ट्वा महात्मनः}


\twolineshloka
{अयं स रथ आयाति श्वेताश्वः कृष्णसारथिः}
{निघ्नन्नमित्रान्समरे यं कर्ण परिपृच्छसि}


\twolineshloka
{एष तिष्ठति कौन्तेयः संस्पृशन्गाण्डिवं धनुः}
{तं हनिष्यसि चेदद्य तन्नः श्रेयो भविष्यति}


\twolineshloka
{धनुर्ज्या चन्द्रताराङ्का पताकाकिङ्किणीयुता}
{पश्य कर्णार्जुनस्यैषा सौदामन्यम्बरे यथा}


\twolineshloka
{एष ध्वजाग्रे पार्थस्य प्रेक्षमाणः समन्ततः}
{दृश्यते वानरो भीमो वीक्षतां भयवर्धनः}


\twolineshloka
{एतच्चक्रं गदा शङ्खः शार्ङ्गं कृष्णस्य च प्रभो}
{दृश्यते पाण्डवरथे वाहयानस्य वाजिनः}


\twolineshloka
{एतत्कूजति गाण्डीवं विसृष्टं सव्यसाचिना}
{एते हस्तवता मुक्ता घ्नन्त्यमित्राञ्शिताः शराः}


\twolineshloka
{विशालायतताम्राक्षैः पूर्णचन्द्रनिभाननैः}
{एषा भूः कीर्यते राज्ञां शिरोभिरपलायिनाम्}


\twolineshloka
{एते परिघसङ्काशाः पुण्यगन्धानुलेपनाः}
{उद्धता रणशूराणां पात्यन्ते सायुधा भुजाः}


\twolineshloka
{निरस्तजिह्वा नेत्रान्ता वाजिनः सह सादिभिः}
{पतिताः पात्यमानाश्च क्षितौ क्षीणा विशेरते}


\twolineshloka
{एते पर्वतशृङ्गाणां तुल्या हैमवता गजाः}
{सञ्छिन्नकुम्भाः पार्थेन प्रपतन्त्यद्रयो यथा}


\twolineshloka
{गन्धर्वनगराकारा रथा हतनरेश्वराः}
{विमानादिव पुण्यान्ते स्वर्गिणो निपतन्त्यमी}


\twolineshloka
{व्याकुलीकृतमत्यर्थं पश्य सैन्यं किरीटिना}
{नानामृगसहस्राणां यूथं केसरिणा यथा}


\twolineshloka
{त्वामभिप्रेप्सुरायाति कर्ण निघ्नन्वरान्रथान्}
{असज्जमानो राधेय तं याहि प्रति भारत}


\twolineshloka
{एषा विदीर्यते सेना धार्तराष्ट्री समन्ततः}
{अर्जुनस्य भयात्तूर्णं निघ्नतः शात्रवान्बहून्}


\twolineshloka
{वर्जयन्सर्वसैन्यानि त्वरते हि धनञ्जयः}
{त्वदर्थमिति मन्येऽहं यथास्योदीर्यते वपुः}


\twolineshloka
{न ह्यवस्थास्यते पार्थो युयुत्सुः केनचित्सह}
{त्वामृते क्रोधदीप्तो हि पीड्यमाने वृकोदरे}


\twolineshloka
{विरथं धर्मराजं तु दृष्ट्वा सुदृढविक्षतम्}
{शिखण्डिनं सात्यकिं च धृष्टद्युम्नं च पार्षतम्}


\twolineshloka
{द्रौपदेयान्युधामन्युमुत्तमौजसमेव च}
{नकुलं सहदेवं च वशगांस्ते समीक्ष्य तु}


\twolineshloka
{सहसैकरथः पार्थस्त्वामभ्येति परन्तपः}
{क्रोधरक्तेक्षणः क्रुद्धो जिघांसुः सर्वपार्थिवान्}


\twolineshloka
{त्वरितोऽभिपतत्यस्मांस्त्यक्त्वा सैन्यान्यसंशयम्}
{त्वं कर्ण प्रतियाह्येनं नास्त्यन्यो हि धनुर्धरः}


\twolineshloka
{न तं पश्यामि लोकेऽस्मिंस्त्वत्तो ह्यन्यं धनुर्धरम्}
{अर्जुनं समरे क्रुद्धं यो वेलामिव वारयेत्}


\twolineshloka
{न चास्य रक्षां पश्यामि पार्श्वतो न च पृष्ठतः}
{एक एवाभियाति त्वां पश्य साफल्यमात्मनः}


\twolineshloka
{त्वं हि कृष्णौ रणे शक्तो योद्धुमेतौ परन्तपौ}
{तवैव भारो राधेय प्रत्युद्याहि धनञ्जयम्}


\twolineshloka
{समानो ह्यसि भीष्मेण द्रोमद्रौणिकृपेण च}
{सव्यसाचिनमायान्तं निवारय महारणे}


\twolineshloka
{लेलिहानं यथा सर्पं गर्जन्तमृषभं यथा}
{वनस्थितं यथा व्याघ्रं जहि कर्ण धनञ्जयम्}


\twolineshloka
{एते द्रवन्ति समरान्निरपेक्षा नराधिपाः}
{अर्जुनस्य भयत्रस्ता धार्तराष्ट्रा महाबलाः}


\twolineshloka
{द्रवतामथ तेषां तु नान्योऽस्ति युधि मानवः}
{भयहा यो भवेद्वीरस्त्वामृते सूतनन्दन}


\twolineshloka
{एते त्वां कुरवः सर्वे द्वीपमासाद्य संयुगे}
{धिष्ठिताः पुरुषव्याघ्र त्वत्तः शरणकाङ्क्षिणः}


\threelineshloka
{वैदेहाम्बष्ठकाम्भोजास्तथा नग्नजितस्त्वया}
{गान्धाराश्च यया धृत्या जिताः सङ्ख्ये सुदुर्जयाः}
{तां धृतिं कुरुराधेय ततः प्रत्येहि पाण्डवम्}


\threelineshloka
{वासुदेवं च वार्ष्णेयं प्रीयमाणं किरीटिना}
{प्रत्युद्याहि महाबाहो पौरुषे महति स्थितः ॥कर्ण उवाच}
{}


\twolineshloka
{प्रकृतिस्थोऽसि मे शल्य इदानीं सम्मतस्तथा}
{प्रतिभासि महाबाहो अभीतश्च धनञ्जयात्}


\twolineshloka
{पश्य बाह्वोर्बलं मेऽद्य शिक्षितस्य च पश्य मे}
{एकोऽद्य निहनिष्यामि पाण्डवानां महाचमूम्}


\twolineshloka
{कृष्णौ च पुरुषव्याघ्र ततः सत्यं ब्रवीमि ते}
{नाहत्वा युधि तौ वीरौ व्यपयास्ये कथञ्चन}


\threelineshloka
{शिश्ये वा निहतस्ताभ्यामनित्यो हि रणे जयः}
{कृतार्थोऽद्य भविष्यामि हत्वा वाप्यथ वा हतः ॥शल्य उवाच}
{}


\threelineshloka
{अजय्यमेनं प्रवदन्ति युद्धेमहारथाः कर्ण रथप्रवीरम्}
{एकाकिन किमु कृष्णाभिगुप्तंविजेतुमेनं क इहोत्सहेत ॥कर्ण उवाच}
{}


\twolineshloka
{नैतादृशो जातु बभूव लोकेरथोत्तमो यावदुपश्रुतं नः}
{तमीदृशं प्रतियोत्स्यामि पार्थंमहाहवे पश्य च पौरुषं मे}


\twolineshloka
{रणे चरत्येष रथप्रवीरःसितैर्हयैः कौरवराजपुत्रः}
{स वाऽद्य मां नेष्यति कृच्छ्रमेत--त्कर्णोऽस्यान्तेऽप्यत्र भवेत्समर्थः}


\twolineshloka
{अस्वेदिनौ राजपुत्रस्य हस्ता--ववेपमानौ जातकिणौ बृहन्तौ}
{दृढायुधः कृतिमान्क्षिप्रहस्तोन पाण्डवेयेन समोऽस्ति योधः}


\twolineshloka
{गृह्णात्यनेकानपि कङ्कपत्रा--नेकं यथा तान्प्रतियोज्य चाशु}
{ते क्रोशमात्रे निपतन्त्यमोधाःकस्तेन योधोऽस्ति समः पृथिव्याम्}


\twolineshloka
{अतोषयत्स्वाण्डवे यो हुताशंकृष्णद्वितीयोऽतिरथस्तरस्वी}
{लेभे चक्रं यत्र कृष्णो महात्माधनुर्गाण्डीवं पाण्डवः सव्यसाची}


\twolineshloka
{श्वेताश्वयुक्तं च सुघोषमुग्रंरथं महाबाहुरदीनसत्वः}
{महेषुधी चाक्षये दिव्यरूपेशस्त्राणि दिव्यानि च हव्यवाहात्}


\twolineshloka
{यस्त्विन्द्रलोके निजघान दैत्या--नसङ्ख्येयान्कालकेयांश्च सर्वान्}
{लेभे शङ्खं देवदत्तं स्म तत्रको नाम तेनाभ्यधिकः पृथिव्याम्}


\twolineshloka
{महादेवं तोषयामास योऽस्त्रैःसाक्षात्सुयुद्धेन महानुभावः}
{लेभे ततः पाशुपतं सुघोरंत्रैलोक्यसंहारकरं महास्त्रम्}


\twolineshloka
{पृथक्पृथग्लोकपालाः समेताददुर्महास्त्राण्यप्रमेयाणि सङ्ख्ये}
{यैस्ताञ्जघानाशु रणे नृसिंहःसकालकेयानसुरान्समेतान्}


\twolineshloka
{तथा विराटस्य पुरे समेता--न्सर्वानस्मानेकरथेन जित्वा}
{जहार तद्रोधनमाजिमध्येवस्त्राणि चादत्त महारथेभ्यः}


\twolineshloka
{तमीदृशं वीर्यगुणोपपन्नंकृष्णद्वितीयं परमं नृपाणाम्}
{तमाह्वयन्साहसमुत्तमं वैजाने स्वयं सर्वलोकस्य शल्य}


\twolineshloka
{अनन्तवीर्येण च केशवेननारायणेनाप्रतिमेन गुप्तः}
{वर्षायुतैर्यस्य गुणा न शक्यावक्तुं समेतैरपि सर्वलोकैः}


\twolineshloka
{महात्मनः शङ्खचक्रासिपाणे--र्विष्णोर्जिष्णोर्वसुदेवात्मजस्य}
{भयं न मे जायते साध्वसं चदृष्ट्वा कृष्णावेकरथे समेतौ}


\twolineshloka
{अतीवायं धनुषि राजपुत्रे--ष्वतीवान्यान्केशवश्चक्रयुद्धे}
{एवंविधौ पाण्डववासुदेवौचलेत्स्वदेशाद्विमवान्न कृष्णौ}


\twolineshloka
{उभौ हि शूरौ बलिनौ दृढायुधौमहारथौ संहननोपपन्नौ}
{एतौ वीरौ नरवीरौ समेतौस्थानाच्च्युतौ देवकुमाररूपौ}


\threelineshloka
{अग्र्यादित्याविन्द्रवृहस्पती वायमान्तकौ वा शत्रिपूषणौ वा}
{भगांशमित्रावरुणावश्विनौ वामरुद्गणौ वा वसवः समेताः}
{व्यस्ताः समस्ताश्च युधा न शक्ताजेतुं प्रसह्यार्जुनं चाच्युतं च}


% Check verse!
एतौ हि तावर्जुनवासुदेवौकोऽन्यः प्रतीयान्मदृते तु शल्य
\twolineshloka
{सर्वेषां वृष्णिवीराणां कृष्णे लक्ष्मीः प्रतिष्ठिता}
{सर्वेषां पाण्डुपुत्राणां जयः पार्थे प्रतिष्ठितः}


\twolineshloka
{तावुभौ पुरुषव्याघ्रौ समाने स्यन्दने स्थितौ}
{मामेकमभियोद्धारौ सुजातं बत शल्य मे}


\twolineshloka
{नैतच्चिरं क्षिप्रमिमं रथं मेप्रवर्तयैतावभियामि चैवम्}
{अस्मिन्मुहूर्ते निहतौ पश्य कृष्णौताभ्यां हतं वा युधि मां रिपुभ्याम्}


% Check verse!
एवं ब्रुवाणः सहसा महारथ--स्त्वभ्यद्रवत्पाण्डवं सूतपुत्रः
\threelineshloka
{अभ्येत्य पुत्रेण तवाभिनन्दिताःसमेत्य चोक्ताः कुरुवीरसत्तमाः}
{कृपश्च भोजश्च महारथावुभौतथैव गान्धारपतिः सहानुजः}
{गुरोः सुतस्तस्य तवात्मजास्तथापदातिसङ्घा द्विरदास्तथा तदा}


\twolineshloka
{निरुध्यताभिद्रवताच्युतार्जुनौश्रमेण संयोजयताशु सर्वशः}
{यथा भवद्भिर्भृशविक्षतावुभौसुखेन हन्यान्मम वाहिनीपतिः}


\threelineshloka
{तथेति चोक्त्वा त्वरिताः स्म तेऽर्जुनंजिघांसवो वीरतराः समाययुः}
{शरैश्च जघ्नुर्युधि तं महारथाधनञ्जयं कर्णनिदेशकारिणः}
{नदीनदं भूरिजलो महार्णवोयथा तथा तान्समरेऽर्जुनोऽग्रसत्}


\twolineshloka
{न सन्दधानो न तथा शरोत्तमान्प्रमुञ्चमानो रिपुभिः प्रदृश्यते}
{धनञ्जयास्तैः स्म शरैर्विदारिताहता निपेतुर्नरवाजिकुञ्जराः}


\twolineshloka
{शरार्चिषं गाण्डिवचारुमण्डलंयुगान्तसूर्यप्रतिमानतेजसम्}
{न कौरवाः शेकुरुदीक्षितुं जयंयथा रविं व्याधितचक्षुषो जनाः}


\twolineshloka
{शरोत्तमान्सम्प्रहितान्महारथैश्चिच्छेद पार्थः प्रहसञ्छरौधैः}
{भूयश्च तानहनद्बाणसङ्घान्गाण्डीवधन्वायतपूर्णमण्डलम्}


\twolineshloka
{यथोग्ररश्मिः शुचिशुक्रमध्यगःसुखं विवस्वान्हरते जलौघान्}
{तथार्जुनो बाणगणान्निरस्यददाह सेनां तव पार्थिवेन्द्र}


\twolineshloka
{तमभ्यधावद्विसृजन्कृपः शरां--स्तथैव भोजस्तव चात्मजः स्वयम्}
{महारथो द्रोणसुतश्च सायकै--रवाकिरंस्तोयधरा यथाऽचलम्}


\twolineshloka
{जिघांसुभिस्तान्कुशलैः शरोत्तमान्महात्मभिः सम्प्रहितान्प्रयत्नतः}
{शरैः प्रतिच्छेद स पाण्डवस्त्वरन्परान्विनिर्भिद्य च तांस्त्रिभिस्त्रिभिः}


\twolineshloka
{स गाण्डिवव्यायतपूर्णमण्डल--स्तपन्रिपूनर्जुनभास्करो वभौ}
{शरोग्रश्मिः शुचिशुक्रमध्यगोयथैव सूर्यः परिवेयवांस्तथा}


\twolineshloka
{अथाग्र्यबाणैर्दशभिर्धनञ्जयंपराभिनद्दोणसुतोऽच्युतं त्रिभिः}
{चतुर्भिरश्वांश्चतुरः कपिं ततःशरैश्च नाराचवरैरवाकिरत्}


\twolineshloka
{तथा ततः प्रस्फुरदस्य कार्मुकंत्रिभिः शरैर्यन्तृशिरश्चकर्त ह}
{हयांश्चतुर्भिश्चतुरस्त्रिभिर्ध्वजंधनञ्जयो द्रौणिरथादपातयत्}


\twolineshloka
{स रोषपूर्णो ह्मतिवज्रहाटकै---रलङ्कृतं तक्षकभोगवर्चसम्}
{स तद्वधे कार्मुकमन्यदाददेयथा महाहिप्रवरं तथैव च}


\twolineshloka
{स्वमायुधं चापि विकीर्य भूतलेधनुश्च कृत्वा सगुणं गुणाधिकः}
{समानयंसल्तावजितौ नरोत्तमौशरोत्तमैर्द्रौणिरविध्यदन्तिकात्}


\twolineshloka
{कृपश्च भोजश्च तवात्मजश्च तेशरैरनेकैर्युधि पाण्डवर्षभम्}
{महारथाः संयुगमूर्धनि स्थिता--स्तमोनुदं वारिधरा इवापतन्}


\twolineshloka
{कृपस्य पार्थः शरं शरासनंहयान्ध्वजान्सारधिमेव पत्रिभिः}
{[समार्पयद्वाहुसहस्रविक्रम--स्तथा यथा वज्रधरः पुरा बलेः}


\twolineshloka
{स पार्थबाणैर्विनिपातितायुधोध्वजावमर्दे च कृते महाहवे}
{कृतः कृपो बाणसहस्रयनितोयथापगेयः प्रथमं किरीटिना ॥]}


\twolineshloka
{शरैः प्रचिच्छेद तवात्मजस्यध्वजं धनुश्च प्रचकर्त नर्दतः}
{जघान चाश्वान्कृतवर्मणः शुभान्ध्वजं च चिच्छेद तवात्मजस्य ह}


\twolineshloka
{सवाजिसूतेष्वसनान्सकेतना--ञ्जयान नागाश्वरथांश्च स त्वरन्}
{ततः प्रकीर्णं सुमहद्वलं तवप्रपीडितं सवितुरिवौजसा भृशम्}


\threelineshloka
{ततोऽर्जुनस्वाशु रथेन केशव--श्चकार शत्रूनपसव्यमातुरान्}
{ततः प्रयातं त्वरितं धनञ्जयंशतक्रतुं वृत्रनिजघ्रुषं यथा}
{समन्वधावन्पुनरुत्थितैर्ध्वजैरथैः सुयुक्तैरपरे युयुत्सवः}


\twolineshloka
{अथाभिसृत्य प्रतिवार्य चाहितान्धनञ्जयस्यानुचरान्महारथाः}
{शिखण्डिशैनेययमाः शितैः शरै---र्विदारयन्तो व्यनदन्सुभैरवम्}


\twolineshloka
{ततोऽभिजघ्नुः कुपिताः परस्परंशरैस्तदाञ्चोगतिभिः सुतेजनैः}
{कुरुप्रवीराः सह पृञ्जयैर्यथासुरारयो देवपतिं यथा तथा}


\twolineshloka
{जयेप्सवः स्वर्गमनाय चोत्सुकाःपतन्ति नागाश्वरथाः परन्तप}
{तथैव सर्वे बहवश्च विव्यधुःशरैः सुमुक्तैरितरेतरं पृथक्}


\twolineshloka
{शरान्धकारे तु महात्मभिः कृतेमहामृधे योधवरैः परस्परम्}
{चतुर्दिशो वै विदिशश्च पार्थिवप्रभा च सूर्यस्य तमोवृताऽभवत्}


\chapter{अध्यायः ८८}
\twolineshloka
{सञ्जय उवाच}
{}


\twolineshloka
{xxxxxxxxxxxप्रबरैर्वलैर्भीममभिद्रुतम्}
{xxxxxxxxxx कौन्तेयमुस्थिहीर्षुर्धनञ्जयः}


\twolineshloka
{xxxxxxxxx सेनां मारत सायकैः}
{xxxxxxxxxx परवीरान्धनञ्जयः}


\twolineshloka
{xxxxxxxxxxx शरजालानि भागशः}
{अदृश्यन्त तथान्ये च निजघ्नुस्तव वाहिनीम्}


\twolineshloka
{स पक्षिसङ्घाचरितमाकाशं पूरयञ्शरैः}
{धनञ्जयो महाबाहुः कुरूणामन्तकोऽमयत्}


\twolineshloka
{ततो भल्लैः क्षुरप्रैश्च नाराचैर्विमलैरपि}
{गात्राणि प्राच्छिनत्पार्थः शिरांसि च चकर्त ह}


\twolineshloka
{छिन्नगात्रैर्विकवचैर्विशिरस्कैः समन्ततः}
{पातितैश्च पतद्भिश्च योधैरासीत्समावृता}


\threelineshloka
{धनञ्जयशराभ्यस्तैः स्यन्दनाश्वरथद्विपैः}
{सञ्छिन्नभिन्नविध्वस्तैर्व्यङ्गाङ्गावयवैः स्तृता}
{गतसत्वैः ससत्वैश्च संवृताऽऽसीद्वसुन्धरा}


\twolineshloka
{सुदुर्गमा सुविषमा घोराऽत्यर्थं सुदुर्दृशा}
{रणभूमिरभूद्राजन्महावैतरणी यथा}


\twolineshloka
{ईषाचक्राक्षभल्लैश्च व्यश्वैः साश्वैश्च युध्यताम्}
{ससूतैर्हतसूतैश्च रथैः स्तीर्णाऽभवन्मही}


\twolineshloka
{सुवर्णवर्मसन्नाहैर्योधैः कनकमालिभिः}
{आस्थिताः क्लृप्तवर्माणो भद्रा नित्यमदा द्विपाः}


\twolineshloka
{क्रुद्धाः क्रुरैर्महामात्रैः प्रेषितार्जुनमभ्ययुः}
{चतुः शता रथवरा हताः पेतुः किरीटिना}


\twolineshloka
{पर्यस्तानीव शृङ्गाणि ससत्वानि महागिरेः}
{धनञ्जयशराभ्यस्तैः स्तीर्णा भूर्वरवारणैः}


\twolineshloka
{समन्ताज्जलदप्रख्यान्वारणान्मदवर्षिणः}
{अभिपेदेऽर्जुनरथो घनान्भिन्दन्निवांशुमान्}


\threelineshloka
{हतैर्गजमनुष्याश्वैर्भिन्नैश्च बहुधा रथैः}
{विशस्त्रयन्त्रकवचैर्युद्धशौण्डैर्गतासुभिः}
{उपविद्धायुधैर्मार्गः स्तीर्णोऽभूत्फल्गुनेन वै}


% Check verse!
विस्फारितं च गाण्डीवमत्यासीद्भैरवस्वनम्सुघोषं वज्रनिष्पेषं स्तनयन्निव तोयदः
\twolineshloka
{ततः प्रादीर्यत चमूर्धनञ्जयशराहता}
{महावातसमाविद्धा महानौरिव सागरे}


\twolineshloka
{नानारूपाः प्राणहराः शरा गाण्डीवचोदिताः}
{अलातोल्काशनिप्रख्यास्तव सैन्यं विनिर्दहन्}


\twolineshloka
{महागिरौ वेणुवनं निशि प्रज्वलितं यथा}
{तथा तव महासैन्यं प्रास्फुरच्छरपीडितम्}


\twolineshloka
{तद्विनष्टं सुविध्वस्तं तव सैन्यं किरीटिना}
{हतप्रविहतं बाणैः सर्वतः प्रद्रुतं दिशः}


\twolineshloka
{सश्वापदा मृगगणा दावाग्नित्रासिता यथा}
{कुरवः पर्यवर्तन्त निर्दग्धाः सव्यसाचिना}


\twolineshloka
{उत्सृज्य च महाबाहुं भीमसेनं तथा रणे}
{बलं कुरूणामुद्विग्नं सर्वमासीत्पराङ्मुखम्}


\twolineshloka
{ततः कुरुषु भग्नेषु बीभत्सुरपराजितः}
{भीमसेनं समासाद्य मुहूर्तं सोऽभ्यवर्तत}


\twolineshloka
{समागम्य च भीमेन मन्त्रयित्वा च फल्गुनः}
{विशल्यमरुजं चास्मै कथयित्वा युधिष्ठिरम्}


\twolineshloka
{भीमसेनाभ्यनुज्ञातस्ततः प्रायाद्धनञ्जयः}
{नादन्रथघोषेण पृथिवीं द्यां च भारत}


\twolineshloka
{ततः परिवृतो वीरैर्दशभिः शत्रुतापनः}
{दुःशासनादवरजैस्तव पुत्रैर्धनञ्जयः}


\twolineshloka
{ते तमभ्यर्दयन्बाणैरुल्काभिरिव कुञ्जरम्}
{आततेष्वसना वीरा नृत्यन्त इव भारत}


\twolineshloka
{अपसव्यांस्तु तांश्चक्रे रथेन मधुसूदनः}
{नियुक्तान्हि स तान्मेने यमायाशु किरीटिना}


% Check verse!
तथान्ये प्राणदन्मूढाः पराङ्मुखमिवार्जुनम्
\twolineshloka
{तेषां नानदतां केतूनश्वांश्चापानि सारथीन्}
{नाराचैरर्धचन्द्रैश्च क्षिप्रं पार्थो न्यकृन्तत}


\twolineshloka
{*अथान्यैर्दशभिर्भल्लैः शिरांस्येषामपातयत्}
{रोषसंरक्तनेत्राणि सन्दष्टौष्ठानि भूतले}


% Check verse!
तेषां वक्राणि विबभुर्व्योम्नि तारागणा इव
\twolineshloka
{तांस्तु भल्लैर्महावेगैर्दशभिर्दश कौरवान्}
{रुक्माङ्गदाव्रुक्मपुङ्खैर्हत्वा प्रायादमित्रहा}


\chapter{अध्यायः ८९}
\twolineshloka
{सञ्जय उवाच}
{}


\twolineshloka
{तमायान्तं महावेगैरश्वैः कपिवरध्वजम्}
{युद्धायाभ्यद्रवन्दीराः कुरूणां नवतीं रथाः}


\twolineshloka
{कृत्वा संशप्तका घोरं शपथं पारलौकिकम्}
{परिवव्रुर्नरव्याघ्रा नरव्याघ्रं रणेऽर्जुनम्}


\twolineshloka
{कृष्णः श्वेतान्महावेगानश्वान्काञ्चनभूषणान्}
{मुक्ताजालप्रतिच्छन्नान्प्रैषीत्कर्णरथं प्रति}


\twolineshloka
{प्रेक्ष्य कर्णरथं यान्तमरिघ्नं तं धनञ्जयम्}
{बाणवर्षैरभिघ्नन्तः संशप्तकगणा ययुः}


\twolineshloka
{त्वरमाणांस्तु तान्सर्वान्ससूतेष्वसनध्वजान्}
{जघान नवतिं वीरानर्जुनो निशितैः शरैः}


\twolineshloka
{तेऽपतन्त हता बाणैर्नानारूपैः किरीटिना}
{विमानेभ्यः सुकृतिनः स्वर्गात्पुण्यक्षये यथा}


\twolineshloka
{ततः सरथनागाश्वाः कुरवः कुरुसत्तमम्}
{निर्भया भरतश्रेष्ठमभ्यवर्तन्त फल्गुनम्}


\twolineshloka
{तदायस्तमनुष्याश्वमुदीर्णवरवारणम्}
{पुत्राणां ते महासैन्यं समरौत्सीद्धनञ्जयम्}


\twolineshloka
{शक्त्यृष्टितोमरप्रासैर्गदानिस्त्रिंशसायकैः}
{प्राच्छादयन्महेष्वासाः कुरवः कुरुनन्दनम्}


\twolineshloka
{तामन्तरिक्षे विततां शस्त्रवृष्टिं समन्ततः}
{व्यधमत्पाण्डवो बाणैस्तमः सूर्य इवांशुभिः}


\twolineshloka
{ततो म्लेच्छाः स्थिता मत्तैस्त्रयोदशशतैर्गजैः}
{पार्श्वतो व्यहनन्पार्थं तव पुत्रस्य शासनात्}


\twolineshloka
{कर्णिनालीकनाराचैस्तोमरप्रासशक्तिभिः}
{कर्पणैर्भिण्डिपालैश्च रथस्थं पार्थमार्दयन्}


\twolineshloka
{तां शस्त्रवृष्टिमतुलां द्विपस्थैः प्रेषितां प्रभुः}
{चिच्छेद निशितैर्भल्लैरर्धचन्द्रैश्च फल्गुनः}


\twolineshloka
{अथ तान्द्विरदान्सर्वान्नानालिङ्गैः शरोत्तमैः}
{सपताकध्वजारोहान्गिरीन्वज्रैरिवाभिनत्}


\twolineshloka
{ते हेमपुङ्खैरिषुभिरर्दिता हेममालिनः}
{हताः पेतुर्महानागाः साग्रिज्वाला इवाद्रयः}


\twolineshloka
{ततो गाण्डीवनिर्घोषो महानासीद्विशाम्पते}
{स्तनतां कूजतां चैव मनुष्यगजवाजिनाम्}


\twolineshloka
{कुञ्चराश्च हता राजन्दद्रुवुस्ते समन्ततः}
{अश्वाश्च पर्यधावन्त हतारोहा दिशो दश}


\twolineshloka
{रथा हीना नहाराज रथिभिर्वाजिभिस्तथा}
{गन्धर्वनगराकारा दृश्यन्ते स्म सहस्रशः}


\twolineshloka
{अश्वारोहा महाराज धावमाना इतस्ततः}
{तत्रतत्रैव दृश्यन्ते निहताः पार्थसायकैः}


\twolineshloka
{तस्मिन्क्षणे पाण्डवस्य बाह्वोर्बलमदृश्यत}
{यत्सादिनो वारणांश्च रथांश्चैकोऽजयद्युधि}


\twolineshloka
{असंयुक्ताश्च ते राजन्परिवृत्ता रणं प्रति}
{नरा नागा रथाश्चैव नदन्तोऽर्जुनमभ्ययुः}


\twolineshloka
{ततस्त्र्येङ्गेण महता बलेन भरतर्षभ}
{दृष्ट्वा परिवृतं राजन्भीमसेनः किरीटिनम्}


\twolineshloka
{हतावशेषानुत्सृज्य त्वदीयान्कतिचिद्रथान्}
{जवेनाभ्यद्रवद्भीमो धनञ्जयरथं प्रति}


\twolineshloka
{ततस्तत्प्राद्रवत्सैन्यं हतभूयिष्ठमातुरम्}
{दृष्ट्वाऽर्जुनं तदा भीममागतं भ्रातरं प्रति}


\twolineshloka
{हतावशिष्टांस्तुरगानर्जुनेन महाबलः}
{भीमो व्यधमदश्रान्तो गदापाणिर्महाहवे}


\twolineshloka
{`गदापाणिं तदा दृष्ट्वा भीमं भारत भारताः}
{मेनिरे तमनुप्राप्तं दण्डपाणिमिवान्तकम्'}


\twolineshloka
{कालरात्रिमिवात्युग्रां नरनागाश्वभोजनाम्}
{प्राकाराट्टपुरद्वारदारणीमतिदारुणाम्}


\twolineshloka
{गदां तुरगनाशाय त्वरन्भीमो व्यवासृजत्}
{सा जघान बहूनश्वानश्वारोहांश्च मारिष}


\twolineshloka
{कांस्यायसतनुत्रांश्च नरानश्वांश्च पाण्डवः}
{पोथयामास गदया सशब्दं तेऽपतन्हताः}


\twolineshloka
{दन्तैर्दशन्तो वसुधां शेरते क्षतजोक्षिताः}
{भग्नमूर्धास्थिचरणाः क्रव्यादगणभोजनाः}


\twolineshloka
{असृङ्मांसवसाभिश्च तृप्तिमभ्यागता गदा}
{अस्थीन्यप्यश्नती तस्थौ कालरात्रीव दुर्दृशा}


\twolineshloka
{सहस्राणि दशाश्वानां हत्वा पत्तींश्च भूयसः}
{भीमोऽभ्यधावत्सङ्क्रुद्धो गदापाणिरितस्ततः}


\twolineshloka
{गदापाणिं ततो भीमं दृष्ट्वा भारत तावकाः}
{मेनिरे समनुप्राप्तं कालदण़्डोद्यतं यमम्}


\twolineshloka
{स मत्त इव मातङ्गः सङ्क्रुद्धः पाण्डुनन्दनः}
{प्रविवेश गजानीकं मकरः सागरं यथा}


\twolineshloka
{विगाह्य च गजानीकं प्रगृह्य महतीं गदाम्}
{क्षणेन भीमः सङ्क्रुद्धस्तन्निन्ये यमसादनम्}


\twolineshloka
{गजान्सकङ्कटान्मत्तान्सारोहान्सपताकिनः}
{पततः समपश्याम सपक्षान्पर्वतानिव}


\twolineshloka
{हत्वा तु तद्गजानीकं भीमसेनो महाबलः}
{पुनः स्वरथमास्थाय पृष्ठतोऽर्जुनमभ्ययात्}


\twolineshloka
{ततः पराङ्मुखीभूतं निरुत्साहं बलं तव}
{तदानीं तु महाराज प्रायशः शस्त्रवेष्टितम्}


\twolineshloka
{विलम्बमानं तत्सैन्यभप्रगल्भमवस्थितम्}
{दृष्ट्वा प्राच्छादयद्बाणैरर्जुनः शस्त्रवृष्टिभिः}


\twolineshloka
{नराश्वरथमातङ्गा युधि गाण्डीवधन्वना}
{शरव्रातैश्चिता रेजुः कदम्बा इव केसरैः}


\twolineshloka
{ततः कुरुणामभवदार्तनादो महान्नृप}
{नराश्वनागासुहनान्दृष्ट्वा बाणान्किरीटिनः}


\twolineshloka
{हाहाकृतं भृशं त्रस्तं लीयमानं परस्परम्}
{अलातचक्रवत्सैन्यं तदा बभ्राम तावकम्}


\twolineshloka
{ततोऽर्जुनशरध्वस्तं कुरूणां सुमहद्बलम्}
{न ह्यत्रासीदनिर्भिन्नो रथः सादी हयो गजः}


\twolineshloka
{आदीप्तमिव तत्सैन्यं शरैश्छिन्नतनुच्छदम्}
{आसीत्स्वशोणितक्लिन्नं रौद्रं नष्टं विशाम्पते}


\twolineshloka
{तत्सैन्यं हतभूयिष्ठमाहतं निशितैः शरैः}
{न जहौ समरं प्राप्य फल्गुनं शत्रुतापनम्}


\twolineshloka
{तत्राद्भुतमपश्याम कौरवाणां पराक्रमम्}
{वध्यमानोऽपि यत्पार्थं न जहौ भरतर्षभ}


\twolineshloka
{तं दृष्ट्वा विक्रमं तस्य कुरवः सव्यसाचिनः}
{निराशाः समपद्यन्त सर्वे कर्णस्य जीविते}


\twolineshloka
{अविषह्यं तु पार्थस्य शरसंस्पर्शमाहवे}
{मत्वा न्यवर्तन्कुरवो जिता गाण्डीवधन्वना}


\twolineshloka
{ते हित्वा समरे पार्थं वध्यमानाश्च सायकैः}
{प्रदुद्रुवुर्दिशो भीताश्चुक्रुशुश्चापि सूतजम्}


\twolineshloka
{तान्विद्रावयते पार्थः किरञ्शरशतान्बहून्}
{हर्षयन्पाण्डवीं सेनां राजानं च युधिष्ठिरम्}


\twolineshloka
{पुत्रास्तव महाराज जग्मुः कर्णरथं प्रति}
{अगाधे मज्जतां तेषां द्वीपः कर्णोऽभवत्तदा}


\twolineshloka
{कुरवो हि महाराज निर्विषाः पन्नगा इव}
{कर्णमेवोपलीयन्त भयाद्गाण्डीवघन्वनः}


\twolineshloka
{यथा सर्वाणि भूतानि मृत्योर्भीतानि मारिष}
{उपलीयन्ति सन्त्रासाद्धर्मं लोकपरायणम्}


\twolineshloka
{तथा कर्णं महेष्वासं पुत्रास्तव नराधिप}
{उपालीयन्त सन्त्रासात्पाण्डवस्य महात्मनः}


\twolineshloka
{ताञ्शोणितपरिक्लिन्नान्विषमस्थाञ्शरातुरान्}
{मा भैष्टेत्यब्रवीत्कर्णो भीतानाविष्टचेतसः}


\twolineshloka
{सम्भग्नं हि बलं दृष्ट्वा बलात्पार्थेन तावकम्}
{धनुर्विस्फारयन्कर्णस्तस्थौ शत्रुजिघांसया}


\twolineshloka
{तान्प्रद्रुतान्कुरून्दृष्ट्वा कर्णः शस्त्रभृतां वरः}
{सञ्चिन्तयित्वा पार्थस्य वधे दध्रे मनः श्वसन्}


\twolineshloka
{विस्फार्य सुमहच्चापं ततश्चाधिरथिर्वृषः}
{पाञ्चालान्पुनराधावत्पश्यतः सव्यसाचिनः}


\twolineshloka
{ततः क्षणेन क्षितिपाः क्षतजप्रतिमेक्षणाः}
{शरौघैश्छादयामासुर्महामेघा इवाचलम्}


\twolineshloka
{ततः शरसहस्राणि कर्णमुक्तानि मारिष}
{पाञ्चालानां हरत्प्राणांस्तमांसीव तमोनुदः}


\twolineshloka
{महदासीत्तदा युद्धं पाञ्चालानां महामते}
{वध्यतां सूतपुत्रेण मित्रार्थे मित्रगृद्धिना}


\chapter{अध्यायः ९०}
\twolineshloka
{सञ्जय उवाच}
{}


\twolineshloka
{ततः कर्णः कुरुषु प्रद्रुतेषुवरूथिना श्वेतहयेन राजन्}
{पाञ्चालपुत्रान्व्यधमत्सूतपुत्रोमहेषुमिर्वात इवाभ्रसङ्घान्}


\twolineshloka
{सूतं रथादञ्जलिकैर्निपात्यजघान चाश्वाञ्जनमेजयस्य}
{शतानीकं सुतसोमं च भल्लै--रवाकिरद्धनुषी चाप्यकृन्तत्}


\twolineshloka
{धृष्टद्युम्नं निर्बिभेदाथ षड्भि--र्जघानाश्वान्दक्षिणांस्तस्य सङ्ख्ये}
{हत्वा चाश्वान्सात्यकेः सूतपुत्रःकैकेयपुत्रं न्यवधीद्विशोकम्}


\twolineshloka
{तमभ्यधावन्निहते कुमारेसेनापतिः कैकयो मित्रवर्मा}
{शरैर्विधुन्वन्भृशमुग्रवेगैःकर्णात्मजं चाप्यहनत्सुदेवम्}


\twolineshloka
{तस्यार्धचन्द्रैस्त्रिभिरुच्चकर्तप्रहस्य बाहू च शिरश्च कर्णः}
{स स्यन्दनाद्गामगमद्गतासुःपरश्वथैः साल इवावरुग्णः}


\twolineshloka
{हताश्वमञ्जोगमिभिः सुषेणःशिनिप्रवीरं निशितैः पृषत्कैः}
{प्रच्छाद्य नृत्यन्निव कर्णपुत्रःशैनेयबाणाभिहतः पपात}


\twolineshloka
{पुत्रे हते क्रोधपरीतयेताःकर्णः शिनीनां प्रवरं जिघांसुः}
{हतोऽसि हे सात्यक इत्युदीर्यव्यवासृजद्बाणममित्रसाहम्}


\twolineshloka
{तमस्य चिच्छेद शरं शिखण्डीत्रिभिस्त्रिभिश्च प्रतुतोद कर्णम्}
{शिखण्डिनः कार्मुकं च ध्वजं चच्छित्त्वा क्षुराभ्यां न्यहनत्सुजातः}


\twolineshloka
{शिखण्डिनं षड्भिरविध्यदुग्रोधार्ष्टद्युम्नेः स शिरश्चोच्चकर्त}
{ततोऽभिनत्सुतसोमं क्षुरेणसुसञ्झितेनाधिरथिर्महात्मा}


\twolineshloka
{अथाक्रन्दे तुमुले वर्तमानेधार्ष्टद्युम्ने निहते तत्र कृष्णः}
{अपाञ्चाल्यं क्रियते याहि पार्थकर्णं जहीलाब्रवीद्राजसिंह}


\twolineshloka
{ततः प्रहस्याशु नरप्रवीरोरथं रथेनाधिरथेर्जगाम}
{भये तेषां त्राणमिच्छन्सुबाहु--रभ्याहतानां रथयूथपेन}


\twolineshloka
{ततोऽपरे भारत दुष्प्रकम्प्याःपाञ्चालानां रथसङ्घाः समेताः}
{प्रतिश्रिता ह्यन्तरिक्षे ग्रहाभाधनुःप्रवीरास्तु रथप्रवीराः}


\twolineshloka
{विस्फार्य गाण्डीवमथोग्रघोषंज्यया समाहत्य तले भृशं च}
{बाणान्धकारं सहसैव कृत्वाजघान नागाश्वरथध्वजांश्च}


\twolineshloka
{[प्रतिश्रुतः प्राचरदन्तरिक्षेगुहा गिरीणामपतन्वयांसि}
{यन्मण्डलज्येन विजृम्भमाणोरौद्रे मुहूर्तेऽभ्यपतत्किरीटी ॥]}


\twolineshloka
{तं भीमसेनोऽनुययौ रथेनपृष्ठे रक्षन्पाण्डवमेकवीरः}
{तौ राजपुत्रौ त्वरितौ रथाभ्यांकर्णाय यातावरिभिर्विषक्तौ}


\twolineshloka
{तत्रान्तरे सुमहान्सूतपुत्र--श्चक्रे युद्धं सोमकान्सम्प्रमथ्य}
{रथाश्वमातङ्गणाञ्जघानप्रच्छादयामास शरैर्दिशश्च}


\twolineshloka
{तमुत्तमौजा जनमेजयश्चक्रुद्धौ युधामन्युशिखण्डिनौ च}
{कर्णं बिभेदुः सहिताः पृषत्कैःसन्नर्दमानाः सह पार्षतेन}


\twolineshloka
{ते पञ्च पाञ्चालरथाः समेतावैकर्तनं कर्णमभिद्रवन्तः}
{तस्माद्रथाच्च्यावयितुं न शेकु--र्धैर्यात्कृतात्मानमिवेन्द्रियार्थाः}


\twolineshloka
{तेषां धनूंषि ध्वजवाजिसूतां--स्तूर्णं पताकाश्च निकृत्य बाणैः}
{तान्पाञ्चालानभ्यहनत्पृवत्कैःकर्णस्ततः सिंह इवोन्ननाद}


\twolineshloka
{तस्यास्यतस्तानभिनिघ्नतश्चज्याबाणहस्तस्य धनुःस्वेनन}
{साद्रिद्रुमा स्यात्पृथिवी विशीर्णे--त्यतीव मत्वा जनता व्यपीदत्}


\twolineshloka
{स शक्रचापप्रतिमेन धन्वनाभृशायतेनाधिरथिः शरान्सृजन्}
{बभौ रणे दीप्तमरीचिमण्डलोयथांशुमाली परिवेषवांस्तथा}


\twolineshloka
{शिखण्डिनं द्वादशभिः पराभिन--च्छितैः शरैः षड्भिरथोत्तमौजसम्}
{त्रिभिर्युधामन्युमविध्यदाशुगै--स्त्रिभिस्त्रिभिः सोमकपार्षतात्मजौ}


\twolineshloka
{पराजिताः पञ्च महारथास्तु तेमहाहवे सूतसुतेन मारिष}
{निरुद्यमास्तस्थुरमित्रनन्दनायथेन्द्रियार्थात्मवता पराजिताः}


\twolineshloka
{निमज्जतस्तानथ कर्णसागरेविपन्ननावो वणिजो यथार्णवे}
{उद्दध्रिरे नौभिरिवार्णवाद्रथैःसुकल्पितैर्द्रौपदिजाः स्वमातुलान्}


\twolineshloka
{ततः शिनीनामृषभः शितैः शरै--र्निकृत्य कर्णप्रहितानिषून्वहून्}
{विदार्य कर्णं निशितैरयस्मयै--स्तवात्मजं ज्येष्ठमविध्यदष्टमिः}


\twolineshloka
{कृपोऽथ भोजश्च तवात्मजस्तथास्वयं च कर्णो निशितैरताडयत्}
{स तैश्चतुर्भिर्युयुधे यदूत्तमोदिगीश्वरैर्दैत्यपतिर्यथा तथा}


\twolineshloka
{समाततेनेष्वसनेन कूजतामहास्वनेनाशनिपातदीधितिः}
{बभूव दुर्धर्षतरः स सात्यकिःशरन्नभोमध्यगतो यथा रविः}


\twolineshloka
{पुनः समास्थाय रथान्सुदंशिताःशिनिप्रवीरं जुगुपुः परन्तपाः}
{समेत्य पाञ्चालमहारथा रणेमरुद्रणाः शक्रमिवारिनिग्रहे}


\twolineshloka
{ततोऽभवद्युद्धमतीव दारुणंतवाहितानां तव सैनिकैः सह}
{रथाश्वमातङ्गविनाशनं तथायथा सुराणामसुरैः पुराऽभवत्}


\twolineshloka
{रथा द्विपा वाजिपदातयस्तथाभ्रमन्ति नानाविधशस्त्रवेष्टिताः}
{परस्परेणाभिहताश्च चस्खलु--र्विनेदुरार्ता व्यसवोऽपतंस्तथा}


\twolineshloka
{तथागतं भीममभीस्तवात्मजःससार राजावरजः किरञ्शरैः}
{तमभ्यधावत्त्वरितो वृकोदरोमहारुरुं सिंह इवाभिपेदिवान्}


\twolineshloka
{ततस्तयोर्युद्धमतीव दारुणंप्रदीव्यतोः प्राणदुरोदरं द्वयोः}
{परस्परेणाभिनिविष्टरोषयो--रुदग्रयोः शम्बरशक्रयोर्यथा}


\twolineshloka
{शरैः शरीरार्तिकरैः सुतेजनै--र्निजघ्नतुस्तावितरेतरं भृशम्}
{सकृत्प्रभिन्नाविव वासितान्तरेमहागजौ दोर्भिरदीनघातिनौ}


\twolineshloka
{आलोक्य तौ चैव परस्परं ततःसमं च शूरौ च ससारथी तदा}
{भीमोऽब्रवीद्याहि दुःशासनायदुःशासनो याहि वृकोदराय}


\threelineshloka
{तयो रथौ सारथिसम्प्रचोदितौसमं रथौ तौ महसा समीयतुः}
{नानायुधच्छत्रपताकिकाध्वजौदिवीव पूर्वं बलशक्रयो रणे ॥भीम उवाच}
{}


\twolineshloka
{दिष्ट्याऽसि दुःशासन अद्य दृष्टःक्षणं प्रतीच्छे सहवृद्धि मूलम्}
{चिरोदितं यन्मया ते सभायांकृष्णाभिमर्शेन गृहाण मत्तः}


\twolineshloka
{स एवमुक्तस्तु ततो महात्मादुःशासनो वाक्यमुवाच वीरः}
{सर्वं स्मरन्नेव विसंस्मरामिउदीर्यमाणं शृणु भीमसेन}


\twolineshloka
{स्मरामि चात्मप्रभवं चिराययज्जातुषे वेश्मनि रात्र्यहानि}
{विश्वासहीना मृगयां चरन्तोवसन्ति सर्वत्र निराकृतास्तु}


\twolineshloka
{महाभयं रात्र्यहानि स्मरन्त--स्तथोपभोगाच्च सुखाच्च हीनाः}
{वनेष्वटन्तो गिरिगह्वराणिपाञ्चालराजस्य पुरं प्रविष्टाः}


\twolineshloka
{मायां यूयं कामपि सम्प्रविष्टायतो वृतः कृष्णया फल्गुनो वः}
{संभूय पापैस्तदनार्यवृत्तंकृतं तदा मातृकृतानुरूपम्}


\fourlineindentedshloka
{एका वृता पञ्चभिः साभिपन्ना}
{ह्यलज्जमानेश्च परस्परस्य}
{स्मरे सभायां सुबलात्मजेनदासीकृता यत्सह कृष्णया च ॥सञ्जय उवाच}
{}


\twolineshloka
{इत्येवमुक्तस्तु तवात्मजेनपाण्डोः सुतः कोपवशं भृशंधनुः क्षुराभ्यां ध्वजमेव चाच्छिनत्}
{ललाटमप्यस्य बिभेद पत्रिणा}


\twolineshloka
{शिरश्च कायात्प्रजहार सारथेः ॥स राजपुत्रोऽन्यदवाप्य कार्मुकंवृकोदरं द्वादशभिः पराभिनत्}
{स्वयं नियच्छंस्तुरगानजिह्मगैः}


% Check verse!
शरैश्च भीमं पुनरप्यवीवृषत्
\chapter{अध्यायः ९१}
\twolineshloka
{सञ्जय उवाच}
{}


% Check verse!
सर राजपुत्रेण समार्च्छदुग्र--दुःशासनेन निकृतो निकृत्या
\twolineshloka
{तत्राकरोद्दुष्करं राजपुत्रोदुःशासनः कुरुवीरो महात्मा}
{यद्भीमसेनं प्रतियोधयद्रणेजम्भो यथा शक्रमुदारवीर्यम्}


\twolineshloka
{धनुच्छित्त्वा भीमसेनस्य सङ्ख्येषड्भिः शरैः सारथिमभ्यविध्यत्}
{ततोऽविध्यत्त्रिंशता भीमसेनंवरेषुभिर्वन्यमिव द्विपेन्द्रम्}


\twolineshloka
{स कार्मुकं गृह्य तु भारसाधनंभीमस्तदा राजपुत्रं ह्यविध्यत्}
{पञ्चाशतैर्बाणगणैः स्तनानन्तरेतोत्रैर्यथा तीव्रवेगं द्विपेन्द्रम्}


\twolineshloka
{ततस्तु राजन्विरथं महात्मादुःशासनो भीमसेनं चकार}
{निहत्य सङ्ख्ये चतुरोऽस्य बाहा--ञ्छित्त्वा रथेषां पुनरेव चाक्षिपत्}


\twolineshloka
{ततः क्षितिस्वो ह्यबरुह्य याना--द्वृकोदरो मदया तस्य वाहान्}
{यमक्षयं प्रेषयित्वा महात्मारथं समाकर्पत राजसूनोः}


\twolineshloka
{तस्मादवप्लुत्य रथात्ससर्जदुःशासनस्तोमरमुग्रवेगम्}
{स तेन विद्धो ह्युरसि ह्यप्रमेयोगदां तस्मै विससर्जाप्रमेयाम्}


\twolineshloka
{ततः क्रोधाद्भीमसेनः कृतानिसर्वाणि दुःखान्यनुसंस्मरन्वै}
{संस्मृत्य संस्मृत्य तथा प्रतिज्ञा--मुग्रामसौ राजपुत्रौ व्यषीदत्}


\twolineshloka
{सञ्चिन्तितं रोषमतीव वेगा-त्त्रयोदशाब्दं पुरुषप्रवीरः}
{प्रगृह्य वज्राशनितुल्यवेगांगदां करेणाथ वृकोदरो रुषा}


\twolineshloka
{निपातयित्वा पृथिवीतले भृशंस ताजयामा वृकोदरो बली}
{अतीव सन्ताडितभिन्नगात्रोदुःशासनो वै निपपात भूमौ}


\twolineshloka
{आक्रम्य कण्ठे युधि राजपुत्रंसंरक्तनेत्रो ह्यब्रवीद्धार्तराष्ट्रम्}
{तद्ब्रूहि किं त्वं परिमार्गमाणोह्यस्मान्पराभूय इहागतः पुनः}


\twolineshloka
{दिदीपयंस्तद्भृशदीपितं मेचिरार्जितं रोषमतिप्रदीप्तम्}
{मधु प्रपास्ये तव कोष्टभाजना--दित्यब्रवीद्भीमसेनस्तरस्वी}


\twolineshloka
{दुःशासनं कण्ठदेशे प्रमृद्रं--स्ततः क्रूरं भीमसेनश्चकार}
{कतं व्यंसयित्वा सहसा ससारबलादसौ धार्तराष्ट्रस्तरस्वी}


\twolineshloka
{भीमोऽभिदुद्राव सुतं त्वदीयंसपत्रतां दर्शयन्धार्तराष्ट्रे}
{मृगं मुहुः सिंहशिशुर्यथा वनेतथाश्वभिद्रुत्य महाबलं बली}


\twolineshloka
{निगृह्य चैनं परमेण कर्मणाउत्क्षिप्य चोत्क्षिप्य च तूर्णमेनम्}
{भूमौ तदा निष्पिपेषाथ वीरअसिं विकोशं विमलं चकार}


\twolineshloka
{तं पातयित्वा तु वृकोदरोऽथजगर्ज हर्षेण विनादयन्दिशः}
{नादेन तेनाखिलपार्श्ववर्तिनोमूर्च्छाकुलाः पतितास्त्वाजमीढ}


\twolineshloka
{दुःशासनं तत्र समीक्ष्य राज--न्भीमो महाबाहुरचिन्त्यकर्मा}
{स्मृत्वा च केशग्रहणं च देव्यावस्त्रापहारं च रजस्वलायाः}


\twolineshloka
{अनागसो भर्तृपराङ्मुखायादुःखानि दत्तान्यपि विप्रचिन्त्या}
{जज्वाल कोपादथ भीमसेनआज्यप्रसिक्तो हि यथा हुताशः}


\twolineshloka
{तत्राह कर्णं च सुयोधनं चकृपं द्रौणिं कृतवर्माणमेव}
{निहन्मि दुःशासनमद्य पापंसंरक्ष्यतामद्य समस्तयोधाः}


\threelineshloka
{इत्येवमुक्त्वा सहसाभ्यधाव--न्निहन्तुकामोऽतिबलस्तरस्वी}
{तथा तु विक्रम्य रमे वृकोदरोमहागजं केसरिको यथैव}
{निगृह्य दुःशासनमेकवीरःसुयोधनस्याधिरथेः समक्षम्}


\twolineshloka
{असिं समुद्यम्य सितं सुधारंकण्ठे पदाक्रम्य च वेपमानम्}
{उवाच तद्गौरिति यद्ब्रुवाणोहृष्टोऽवदः कर्णसुयोधनाभ्याम्}


\twolineshloka
{ये राजसूयावभृथे पवित्राजाताः कछा याज्ञसेन्या दुरात्मन्}
{ते पाणिना कतरेणावकृष्टा--स्तद्ब्रूहि त्वां पृच्छति भीमसेनः}


\threelineshloka
{श्रुत्वा तु तद्भीमवचः सुघोरंदुःशासनो भीमसेनं निरीक्ष्य}
{जज्वाल बीमं स तदा स्मयेनसंशृण्वतां कौरवसोमकानाम्}
{उक्तस्तदाजौ स तथा सरोषंजगाद भीमं परिवृत्तनेत्रः}


\twolineshloka
{अयं करिकराकारः पीनस्तनविमर्दनः}
{गोसहस्रप्रदाता च क्षत्रियान्तकरः करः}


\twolineshloka
{अनेन याज्ञसेन्या मे भीम केशा विकर्षिताः}
{पश्यतां कुरुमुख्यानां युष्माकं च सभासदाम्}


\twolineshloka
{एवं त्वसौ राजसुतं निशम्यब्रुवन्तमाजौ विनिपीड्य वक्षः}
{भीमो बलात्तं प्रतिगृह्य दोर्भ्या--मुच्चैर्ननादाथ समस्तयोधान्}


\twolineshloka
{उवाच यस्यास्ति बलं स रक्ष--त्वसौ भवेदद्य निरस्तबाहुः}
{दुःशानं जीवितं प्रोत्सृजन्त--माक्षिप्य योधांस्तरसा महाबलः}


\twolineshloka
{एवं क्रुद्धो भीमसेनः करेणउत्पाटयामास भुजं महात्मा}
{दुःशासनं तेन स वीरमध्येजघान वज्राशनिसन्निभेन}


\twolineshloka
{कण्ठे समाक्रम्य च वेपमानंकृत्वाऽनु रूपं परमं सुघोरम्}
{कालान्तकाभ्यां सदृशं तदानींविदार्य वक्षश्च महारथस्य}


\twolineshloka
{दुःशासनस्य निपुशासनस्यउद्धृत्य वक्षः पतितस्य भूमौ}
{ततोऽपिबच्छोणितमस्य कोष्ण--मास्वाद्य चास्वाद्य च वीक्षमाणः}


\twolineshloka
{क्रुद्धः प्रहृष्टो निजगाद वाक्यंस्तन्यस्य मातुः पयसोऽमृतस्य}
{माध्वीकजस्येव रसस्य तस्यमधोश्च पानाद्यवकस्य पानात्}


\twolineshloka
{पयोदधिभ्यां मथिताच्च मुख्या--त्तथेक्षुसारस्य मनोहरस्य}
{सर्वेभ्य एवाभ्यधिको रसोऽस्यमतो ममाद्याहितलोहितस्य}


\twolineshloka
{एवं ब्रुवाणं पुनरुत्थितं त--मास्फोट्य वल्गन्तमतिप्रहृष्टम्}
{ये भीमसेनं ददृशुस्तदानींप्रायेण तेऽपि व्यथिता निपेतुः}


\twolineshloka
{ये चापि नासन्पतिता मनुष्या--स्तेषां करेभ्यः पतितं तु शस्त्रम्}
{भयाच्च सञ्चुक्रुशुरस्वरैस्तदानिमीलिताक्षा मुमुहुश्च तत्र}


\twolineshloka
{ये तत्र भीमं रुधिरं पिबन्तंदुःशासनस्य ददृशुः प्रपन्नाः}
{नायं मनुष्यस्त्विति भाषमाणाःसर्वेऽपलायन्त भयाभिपन्नाः}


\chapter{अध्यायः ९२}
\twolineshloka
{सञ्जय उवाच}
{}


\twolineshloka
{स पीत्वा रुधिरं तस्य चरणौ गृह्य भारत}
{इत्युच्चैर्वचनं प्राह प्रतिनृत्यन्वृकोदरः}


\twolineshloka
{एष ते रुधिरं तीव्रं पिबामि पुरुषादवत्}
{वदेदानीं सुसंरब्धः पुनर्गौरिति गौरिति}


\twolineshloka
{ये चास्मान्प्रतिनृत्यन्ति तदा गौरिति गौरिति}
{तान्वयं प्रतिनृत्यामः पुनर्गौरिति गौरिति}


\twolineshloka
{प्रमाणकोट्यां शयने कालकूटकभोजनम्}
{दंशनं चाहिभिस्तीक्ष्णैर्दाहं जतुगृहे च यत्}


\twolineshloka
{द्यूते च दोषभूयस्त्वमरण्ये वसतिश्च या}
{इष्वस्त्राणि च सङ्ग्रामे अनिलानिलवेश्मसु}


\twolineshloka
{दुःखान्येवाभिजानीमो न सुखानि कदाचन}
{धृतराष्ट्रस्य दौरात्म्यात्सपुत्रस्य वयं सदा}


\twolineshloka
{इत्युच्चैर्वचनं प्रोच्य जयं प्राप्य वृकोदरः}
{पुनरेव महाराज तव सैन्यमभिद्रवत्}


\twolineshloka
{रक्तार्द्रपाणिस्तु ततो महात्मागदापाणिः काल इवान्तकाले}
{विभीषयंस्तव पुत्रस्य सैन्य--मितस्ततो धावति वाहिनीं ते}


\twolineshloka
{ततः क्षणाद्भारत शून्यमासी--दायोधनं घोरतरं कुरूणाम्}
{यत्राजिमध्ये प्रापिबद्भीमसेनोदुःशासनस्य रुधिरं क्रोधदीप्तः}


\twolineshloka
{स हत्वा समरे राजन्राजपुत्रं महाबलम्}
{पूर्णकामो मदोदग्रः सिंहो रुरुमिवोत्कटः}


\twolineshloka
{रुधिरार्द्रो महाराज व्यशोभत परन्तपः}
{सपुष्पः किंशुक इव रक्तरक्ततरो बभौ}


\twolineshloka
{रुधिराक्तो घोरवेषः क्रुद्धो राजन्वचोऽब्रवीत्}
{ब्रूहीदानीं पापमते नृशंस पतितो ह्यसि}


\twolineshloka
{दुःशासने यादृशं संश्रुतं न--स्तदवाप्तं पाण्डवैः सर्वमेव}
{अत्रैवमाप्स्याम्यपरं द्वितीयंदुर्योधनं यज्ञपशुं विशस्य}


\twolineshloka
{शिरो मृदित्वाऽस्य पुनश्च शान्तिंयास्याम्यहं कौरवाणां समक्षम्}
{या साऽपतिः सा सपतिर्हि जातायास्ताः सपत्योऽपतयस्तु जाताः}


\twolineshloka
{पश्यन्तु चित्रं विविधं हि लोकेये वै तिलाः षण्डतिला बभूवुः}
{ते चेत्संसिद्धा निधनं गताः परेकिं चित्ररूपं बत जीवलोके}


\twolineshloka
{एतावदुक्त्वा वचनं प्रहृष्टःप्राक्रोशदुच्चै रुधिरार्द्रवक्रः}
{ननर्त चैवातिबलो महात्मावृत्रं निहत्येव सहस्रनेत्रः}


\twolineshloka
{दृष्ट्वा तु नृत्यन्तमुदग्रवीर्यंकालं यथा त्वन्तकाले प्रजानाम्}
{महद्भयं चाधिरथिं विवेशजये निराशाश्च सुतास्त्वदीयाः}


\threelineshloka
{दुःशासने तु निहते पुत्रास्तव महारथाः}
{महत्क्रोधविषं वीरा धारयन्तो महाबलाः}
{ते तु राजन्महावीर्या भीमं प्राच्छादयञ्छरैः}


\twolineshloka
{निषङ्गी कवची खङ्गी दण्डधारो धनुर्धरः}
{अलम्बुर्जलसन्धश्च वातवेगसुवर्चसौ}


\twolineshloka
{एते समेत्य सहिता भ्रातृव्यसनकर्शिताः}
{भीमसेनं महाबाहुं पीडयामासुरञ्जसा}


\twolineshloka
{स वध्यमानो विशिखैः समन्तात्तैर्महारथैः}
{भीमः क्रोधाभिरक्ताक्षः क्रुद्धः काल इवाबभौ}


\twolineshloka
{ततः परिवृतो राजन्नवभिः शत्रुतापनैः}
{दुःशासनादवरजैः पुत्रैस्तव वृकोदरः}


\twolineshloka
{तांस्तु भल्लैर्महावेगैर्नवभिर्नव भारत}
{रुक्माङ्गदान्रुक्मपुङ्खैः पार्थो निन्ये यमक्षयम्}


\twolineshloka
{हतेषु तव पुत्रेषु बलं तद्विप्रदुद्रुवे}
{पश्यतः सूतपुत्रस्य पाण्डवस्य भयार्दितम्}


\twolineshloka
{ततः कर्णं महाराज प्रविवेश महद्भयम्}
{दृष्ट्वा भीमस्य विक्रान्तमन्तकस्य प्रजास्विव}


\twolineshloka
{तस्य त्वाकारभावज्ञः शल्यः समितिशोभनः}
{}


% Check verse!
उवाच वचनं कर्णं प्राप्तकालं हितं तदा
\twolineshloka
{एते द्रवन्ति राजानो भीमसेनभयार्दिताः}
{दुर्योधनश्च सम्मूढो भ्रातृव्यसनकर्शितः}


\twolineshloka
{दुःशासनस्य रुधिरे पीयमाने महात्मना}
{व्यापन्नचेताः सहसा शोकोपहतचेतनः}


\twolineshloka
{उपासते त्वामेते हि परिवार्य महारथाः}
{कृपप्रभृतयः कर्ण हतशेषाः सहोदराः}


\twolineshloka
{पाण्डवा लब्धलक्षाश्च धनञ्जयपुरोगमाः}
{त्वामेवाभिमुखाः शूरा युद्धाय समुपस्थिताः}


\twolineshloka
{स त्वं पुरुषशार्दूल पौरुषे महति स्थितः}
{क्षत्रधर्मं पुरस्कृत्य प्रत्युद्याहि धनञ्जयम्}


\twolineshloka
{भारो हि धार्तराष्ट्रेण त्वयि सर्वः समाहितः}
{तमुद्वह महाबाहो यथाशक्ति यथाबलम्}


\fourlineindentedshloka
{जये स्याद्विपुला कीर्तिर्ध्रुवः स्वर्गः पराजये}
{वृषसेनश्च राधेय सङ्क्रुद्धस्तनयस्तव}
{त्वयि मोहं समापन्ने पाण़्डवानभिधावति ॥सञ्जय उवाच}
{}


\twolineshloka
{एतच्छ्रुत्वा तु वचनं शल्यस्यामिततोजसः}
{हृदि मानुष्यकं भावं कृत्वा युद्वाय सुस्थिरम्}


\twolineshloka
{ततः क्रुद्धो वृषसेनोऽभ्यधाव--द्भीमं समायान्तममित्रसाहम्}
{बाणैः किरन्तं प्रतियाति चोग्रंव्यात्ताननं कालमिवापतन्तम्}


\twolineshloka
{तमभ्यधावन्नकुलः प्रवीर--मारादमित्रं प्रतुदन्पृषत्कैः}
{कर्णस्य पुत्रं समरे प्रहृष्टंजिष्णुर्जिघांसुर्मघवेव जम्भम्}


\twolineshloka
{ततो ध्वजं स्फाटिकहेमचित्रंचिच्छेद धैर्यान्नकुलः क्षुरेण}
{कर्णात्मजस्येष्वसनं च चित्रंभल्लेन जाम्बूनदचित्रनद्धम्}


\twolineshloka
{अथान्यदादाय धनुः स शीघ्रंकर्णात्मजः पाण्डवमभ्यविध्यत्}
{दिव्यैरस्त्रैरभ्यवर्षत्तदैनंकर्णस्य पुत्रो नकुलं कृतास्त्रम्}


\twolineshloka
{ततः क्रुद्धो नकुलः कर्णपुत्रंशरैर्महोल्काभिरिवाभ्यपीडयत्}
{स कर्णपुत्रो नकुलस्य राज--न्सर्वानस्त्रान्वारयदुत्तमास्त्रैः}


% Check verse!
आसीत्सुघोरं भरतप्रवीरयुद्धं तदा कर्णजपाण्डवाभ्याम्
\twolineshloka
{वनायुजान्सुकुमारान्सुशुभ्रा--नलङ्कृताञ्जातरूपेण चित्रान्}
{जघान चाश्वान्नकुलस्य वीरोरणाजिरे सूतपुत्रस्य पुत्रः}


\twolineshloka
{ततो हताश्वादवरुह्य याना--च्चर्माऽऽददे रुचिरं बर्हिचित्रम्}
{आकाशसङ्काशमसिं प्रगृह्यप्रकाशमानः खगवच्चचार}


\twolineshloka
{ततोऽस्य पक्षाननयद्यमायद्विसाहस्रान्नकुलः क्षिप्रकारी}
{ते प्रापतन्नसिना वै विशस्तायथाऽश्वमेधे पशवः शमित्रा}


\twolineshloka
{ततस्ततो विजिता युद्धशौण्डानानादेश्याः सुहृदः सत्यसन्धाः}
{एकेन शीघ्रं नकुलेन नुन्नाःसारेप्सुनेवोत्तमचन्दनस्य}


\twolineshloka
{भूमौ चरन्तं नकुलं रथौघाःसमन्ततः सायकैः प्रत्यगृह्णन्}
{स तुद्यमानो नकुलो विहङ्गै--श्चचार सङ्ख्ये द्विषतो विनिघ्नन्}


\twolineshloka
{तं कर्णपुत्रोऽपि चरन्तमाजौनराश्वमातङ्गरथप्रवीरान्}
{निघ्नन्तमष्टादशभिः पृषत्कै--र्विव्याध वीरः स चुकोप विद्धः}


\twolineshloka
{ततोऽभ्यधावत्समरे जिघांसुःकर्णात्मजं पाण्डुसुतो नृवीरम्}
{तस्येषुभिर्वधमत्सूतपुत्रोमहारणे वर्म सहस्रभारम्}


\twolineshloka
{तस्याथ कांस्यं सुशितं सुपीत--मसिप्रवीरं गुरुभारासाहम्}
{द्विषच्छरीरापहरं सुघोर--माधून्वतः सर्पमिवात्तकोशम्}


\twolineshloka
{क्षिप्रं शरैः षङ्भिरमित्रसाह--श्चकर्त खङ्गं निशितैः सुधारैः}
{पुनश्च पार्थं निशितैः पृषत्कैःस्तनान्तरे क्षिप्रमिवात्यविध्यत्}


% Check verse!
स भीमसेनस्य रथं च गत्वाववर्ष वै शरवर्षं सुघोरम्
\twolineshloka
{नकुलमथ विदित्वा छिन्नबाणासनासिंविरथमरिभिरार्तं कर्णपुत्रास्त्रमग्नम्}
{पवनचलपताका ह्लादिनो वल्गिताश्वाःपरपुरुषनियुक्ताः स्वै रथैः शीघ्रमीयुः}


\twolineshloka
{द्रुपदसुतवरिष्ठाः पञ्च शैनेयषष्ठाद्रुपददुहितृपुत्राः पञ्च चामित्रसाहाः}
{द्विरदरथनराश्वान्सूदयन्तस्त्वदीयान्भगवत इव रुद्राः सङ्ख्यया हेतिमन्तः}


\twolineshloka
{अथ तव रथमुख्यास्तान्प्रतीयुस्त्वरन्तःकृपहृदिकसुतौ च द्रौणिदुर्योधनौ च}
{शकुनिशुकवृकाश्च क्राथदेवावृधौ चद्विरदजलदघौषैः स्यन्दनैः कार्मुकैश्च}


\twolineshloka
{अथ तव रथवर्यास्तान्दशैकप्रवीरा--निषुभिरशनिकल्पैस्ताडयन्तो न्यरुन्धन्}
{नवजलदसवर्णैर्हस्तिभिस्तान्स्म वव्रु--र्गिरिशिखरनिकाशैर्भीमवेगैः कुणिन्दाः}


\chapter{अध्यायः ९३}
\twolineshloka
{सञ्जय उवाच}
{}


\twolineshloka
{सुकल्पिता हैमवता महोत्कटारणाभिकामैः कृतिभिः समास्थिताः}
{सुवर्णजालैर्वितता बभुर्गजा--स्तथा यथा खे जलदाः सविद्युतः}


\twolineshloka
{कुणिन्दपुत्रो दशभिर्महायसैःकृपं ससूताश्वमपीडयद्भृशम्}
{ततः शरद्वत्सुतसायकैर्हतःसहैव नागेन पपात भूतले}


\twolineshloka
{कुणिन्दपुत्रावरजस्तु तोमरै--र्दिवाकरांशुप्रतिमैरयस्मयैः}
{जघान भोजस्य हयानथापतन्क्षणाद्विशस्ताः कृतवर्मणे हयाः}


\twolineshloka
{अथापरे द्रौणिहता महाद्विपा--स्त्रयः ससर्वायुधयोधकेतनाः}
{निपेतुरुर्व्यां व्यसवो विचेतना--स्तथा यथा वज्रहता महाचलाः}


\twolineshloka
{कुणिन्दराजावरजादनन्तरःस्तनान्तरे पत्रिवरैरताडयत्}
{तवात्मजं तस्य तवात्मजः शरैःशितैः शरीरं व्यहनद्द्विपं च तम्}


\twolineshloka
{स नागराजः सह राजसूनुनापपात रक्तं बहु सर्वतः क्षरन्}
{महेन्द्रवज्रप्रहतोऽम्बुदागमेयथा जलं गेरिकपर्वतस्तथा}


\twolineshloka
{कुणिन्दपुत्रप्रहितोऽपरो द्विपःक्राथं ससूताश्वरथं व्यपोथयत्}
{ततोऽपतत्क्राथसराभिघातितःसहेश्वरो वज्रहतो यथा गिरिः}


\twolineshloka
{रथी द्विपस्थेन हतोऽपतच्छरैःक्राथाधिपः पर्वतजेन दुर्जयः}
{सवाजिसूतेष्वसनध्वजस्तथायथा महावातहतो महाद्रुमः}


\twolineshloka
{वृको द्विपस्थं गिरिराजवासिनंभृशं शरैर्द्वादशभिः पराभिनत्}
{ततो वृकं साश्वरथं महाद्विपोद्रुतं चतुर्भिरणैर्व्यपोथयत्}


\twolineshloka
{स पोथितो नागरवेण वीर्यवा--न्पराभिनद्द्वादशभिः शिलीमुखैः}
{वृकेण बाणाभिहतोऽपतत्क्षितौसवारणो बभ्रुसुतेन सार्धम्}


\twolineshloka
{कुणिन्दराजस्य सुतोऽपरस्तदास चापि शूरः सहसा समर्पितः}
{पपात बाणैः सुबलस्य सूनुनाविषाणपुच्छापरगात्रपातिना}


\twolineshloka
{गजेन वाहाञ्शकुनेः कुणिन्दजोनिनाय वैवस्वतमन्दिरं रणे}
{रथं च संक्षुभ्य ननाद नर्दत--स्ततोऽस्य गान्धारपतिः शिरोऽहरत्}


\twolineshloka
{ततः कुणिन्देषु गतेषु तेषुप्रहृष्टरूपास्तव ते महारथाः}
{भृशं प्रदरध्मुर्लवणाम्बुसम्भवा--न्वरांश्च बाणासनपाणयोऽभ्ययुः}


\twolineshloka
{तथाऽभवद्युद्धमतीव दारुणंपुनः कुरूणां सह पाण्डुसृञ्जयैः}
{शरासिशक्त्यृष्टिगदापरश्वथै--र्नराश्वनागासुहरं भृशाकुलम्}


\twolineshloka
{रथाश्वमातङ्गपदातयस्ततःपरस्परं विप्रहताः क्षितौ पतन्}
{यथा सविद्युत्तटितो जलप्रदाःसमुत्थितैर्दिग्भ्य इवोग्रमारुतेः}


\twolineshloka
{ततः शतानीकहता महागजाहया रथाः पत्तिगणाश्च तावकाः}
{सुपर्णवातप्रहता यथोरगा--स्तथा गता गां विवशा विचूर्णिताः}


\twolineshloka
{ततोऽभ्यविद्ध्यद्बहुभिः शितैः शरैःस विन्दपुत्रो नकुलात्मजं स्मयन्}
{ततोऽस्य कोपाद्विचकर्त नाकुलिःशिरः क्षुरेणाम्बुजसन्निभाननम्}


\twolineshloka
{ततः शतानीकमविध्यदायसै--स्त्रिभिः शरैः कर्णसुतोऽर्जुनं त्रिभिः}
{त्रिभिश्च भीमं नकुलं च सप्तभि--र्जनार्दनं द्वादशभिश्च सायकैः}


\twolineshloka
{तदस्य कर्मातिमनुष्यकर्मणःसमीक्ष्य हृष्टाः कुरवोऽभ्यपूजयन्}
{पराक्रमज्ञास्तु धनञ्जयस्य येहुतोऽयमग्नाविति ते तु मेनिरे}


\chapter{अध्यायः ९४}
\twolineshloka
{सञ्जय उवाच}
{}


\threelineshloka
{ततः किरीटि परवीरघातीहताश्वमालोक्य नरप्रवीरः}
{माद्रीसुतं नकुलं लोकमध्येसमीक्ष्य कृष्णं भृशविक्षतं च}
{समभ्यधावद्वृषसेनमाहवेस सूतजस्य प्रमुखे स्थितस्तदा}


\twolineshloka
{तमापतन्तं नरवीरमुग्रंमहाहवे बाणसहस्रधारिणम्}
{अभ्यापतत्कर्णसुतो महारथंयथा महेन्द्रं नमुचिः पुरा तथा}


\twolineshloka
{तौ तत्र शूरौ रथकुञ्जरौ रणेपरस्परस्याभिमुखौ महारथौ}
{ससर्जतुः शरसङ्घाननेका--न्सम्भ्रान्तरूपौ सुभृशं तदानीम्}


\twolineshloka
{ततो द्रुतं चैकशरेण पार्थंशितेन विद्ध्वा युधि कर्णपुत्रः}
{ननाद नादं सुमहानुभावोविद्ध्वेव शक्रं नमुचिः स वीरः}


\twolineshloka
{पुनः स पार्थं वृषसेन उग्रै--र्बाणैरविद्ध्यद्भुजमूले तु सव्ये}
{तथैव कृष्णं नवभिः समार्दय--त्पुनश्च पार्थं दशभिर्जघान}


\twolineshloka
{पूर्वं यथा वृषसेनप्रयुक्तै--रभ्याहतः श्वेतहयः शरैस्तैः}
{संरम्भमीषद्गमितो वधायकर्णात्मजस्याथ मनः प्रदध्रे}


\threelineshloka
{ततः किरीटि रणमूर्ध्नि कोपात्कृत्वा त्रिशाखां भ्रुकुटिं ललाटे}
{मुमोच तूर्णं विशिखान्महात्मावधे धृतः कर्णसुतस्य सङ्ख्ये}
{आरक्तनेत्रोऽन्तकशत्रुहन्ताउवाच कर्णं भृशमुत्स्मयंस्तदा}


\twolineshloka
{दुर्योधनं द्रौणिमुखांश्च सर्वा--नहं रणे वृषसेनं तमुग्रम्}
{सम्पश्यतः कर्ण तवाद्य सङ्ख्येनयामि लोकं निशितैः पृषत्कैः}


\twolineshloka
{ऊनं च तावद्धि जना वदन्तिसर्वैर्भवद्भिर्मम सूनुर्हतोऽसौ}
{एको रथो मद्विहीनस्तरस्वीअहं हनिष्ये भवतां समक्षम्}


% Check verse!
संरक्ष्यतां रथसंस्थाः सुतोऽय--महं हनिष्ये भवतां समक्षम् ॥पश्चाद्वधिष्ये त्वामपि सम्प्रमूढ--महं हनिष्येऽर्जुन आजिमध्ये
\threelineshloka
{तमद्य मूलं कलहस्य सङ्ख्येदुर्योधनापाश्रय जातदर्पम्}
{त्वामद्य हन्ताऽस्मि रणे प्रसह्यअस्यैव हन्ता युधि भीमसेनः}
{दुर्योधनस्याधमपूरुषस्ययस्यानयादेष महान्क्षयोऽभवत्}


\twolineshloka
{स एवमुक्त्वा विनिमृज्य चापंलक्ष्यं हि कृत्वा वृषसेनमाजौ}
{ससर्ज बाणान्विशिखान्महात्मावधाय राजन्कर्णसुतस्य सङ्ख्ये}


\twolineshloka
{विव्याध चैनं दशभिः पृषत्कै--र्मर्मस्वशङ्कं प्रहसन्किरीटी}
{चिच्छेद चास्येष्वसनं भुजौ चक्षुरैश्चतुर्भिर्निशितैः शिरश्च}


\twolineshloka
{स पार्थबाणाभिहतः पपातरथाद्विबाहुर्विशिरा धरायाम्}
{सुपुष्पितो वज्रहतोऽतिमात्रोभग्नो यथा साल इवावकृत्तः}


% Check verse!
तं पार्थबाणाभिहतं पतन्तंसम्प्रेक्ष्य कर्णः सुतमाशुकारी

रथं रथेनाशु जगाम रोषा--त्किरीटिनः पुत्रवधाभितप्तः ॥ 8-90-1x1 सञ्जय उवाच

8-90-1a1तमायान्तमभिप्रेक्ष्य वेलोद्वृत्तमिवार्णवम्

8-90-1b1 गर्जन्तंसुमहाकायं दुर्निवारं सुरैरपि

8-90-1c1 अर्जुनं प्राह दाशार्हःप्रहस्य पुरुषर्षभः ॥ 8-90-2a2 अर्य सरथ आयासि श्वेताश्वः शल्यसारथिः

8-90-2b2 येन ते सह योद्धव्यं स्थिरो भव धनञ्जय ॥ 8-90-3a3 पश्य चैनंसमायुक्तं रथं कर्णस्य पाण्डव

8-90-3b3 श्वेतवाजिसमायुक्तं युक्तंराधासुतेन च ॥ 8-90-4a4 नानापताकाकलिलं किङ्किणीजालमालिनम्

8-90-4b4उह्यमानमिवाकाशो विमानं पाण्डुरैर्हयैः ॥ 8-90-5a5 ध्वजं च पश्यकर्णस्य नागकक्षं महात्मनः

8-90-5b5आखण्डलधनुःप्रख्यमुल्लिखन्तमिवाम्बरम् ॥ 8-90-6a6 पश्य कर्णं समायान्तंधार्तराष्ट्रप्रियैषिणम्

8-90-6b6 शरधारा विमुञ्चन्तंधारासारमिवाम्बुदम् ॥ 8-90-7a7 एष मद्रेश्वरो राजा रथाग्रेपर्यवस्थितः

8-90-7b7 नियच्छति हयानस्य राधेयस्यामितौजसः ॥ 8-90-8a8शृणु दुन्दुभिनिर्घोषं शङ्खशब्दं च दारुणम्

8-90-8b8 सिंहनादांश्चविविधाञ्शृणु पाण़्डव सर्वतः ॥ 8-90-9a9 अन्तर्धायमहाशब्दान्कर्णेनामिततेजसा

8-90-9b9 दोधूयमानस्य भृशं धनुषः शृणुनिःस्वनम् ॥ 8-90-10a10 एते दीर्यन्ति सगणाः पाञ्चालानां महारथाः

8-90-10b10 दृष्ट्वा केसरिणं क्रुद्धं मृगा इव महावने ॥ 8-90-11a11सर्वयत्नेन कौन्तेय हन्तुमर्हसि सूतजम्

8-90-11b11 न हि कर्णशरानन्यःसोढुमुत्सहते नरः ॥ 8-90-12a12सदेवासुरगन्धर्वांस्त्रीँल्लोकान्सचराचरान्

8-90-12b12 त्वं हि जेतुंरणे शक्तस्तथैव विदितं मम ॥ 8-90-13a13 भीममुग्रं महात्मानं त्र्यक्षंशर्वं कपर्दिनम्

8-90-13b13 न शक्ता द्रष्टुमीशानं किं पुनर्योधितुंप्रभुम् ॥ 8-90-14a14 त्वया साक्षान्महादेवः सर्वभूतशिवः शिवः

8-90-14b14 युद्धेनाराधितः स्थाणुर्देवाश्च वरदास्तव ॥ 8-90-15a15 तस्यपार्थ प्रसादेन देवदेवस्य शूलिनः

8-90-15b15 जहि कर्णं महाबाहो नमुचिंवृत्रहा यथा ॥ 8-90-16a16 श्रेयस्तेऽस्तु सदा पार्थ युद्धेजयमवाप्नुहि ॥ 8-90-16b17x अर्जुन उवाच

8-90-17a17 ध्रुव एव जयः कृष्णमम नास्त्यत्र संशयः

8-90-17b17 सर्वलोकगुरुर्यस्त्वं तुष्टोऽसिमधुसूदन ॥ 8-90-18a18 चोदयाश्वान्हृषीकेश रथं मम महारथ

8-90-18b18नाहत्वा समरे कर्णं निवर्तिष्यति फल्गुनः ॥ 8-90-19a19 अद्य कर्णं हतंपश्य मच्छरैः शकलीकृतम्

8-90-19b19 मां वा द्रक्ष्यसि गोविन्द कर्णेननिहतं शरैः ॥ 8-90-20a20 उपस्थितमिदं घोरं युद्धं त्रैलोक्यमोहनम्

8-90-20b20 यज्जनाः कथयिष्यन्ति यावद्भूमिर्धरिष्यति ॥ 8-90-21a21 एवंब्रुवंस्तदा पार्थः कृष्णमक्लिष्टकारिणम्

8-90-21b21 प्रत्युद्ययौरथेनाशु गजं प्रतिगजो यथा ॥ 8-90-22a22 पुनरप्याह तेजस्वी पार्थःकृष्णमरिन्दम

8-90-22b22 चोदयाश्वान्हृषीकेश कालोऽयमतिवर्तते ॥ 8-90-23a23 एवमुक्तस्तदा तेन पाण़्डवेन महात्मना

8-90-23b23 जयेनसम्पूज्य स पाण्डवं तदा 8-90-23c23 प्रचोदयामास हयान्मनोजवान् ॥ 8-90-24a24 स पाण्डुपुत्रस्य रथो मनोजवः 8-90-24b24 क्षणेन कर्णस्यरथाग्रगोऽभवत्
\chapter{अध्यायः ९५}
\twolineshloka
{सञ्जय उवाच}
{}


\twolineshloka
{वृषसेनं हतं दृष्ट्वा क्रोधामर्षसमन्वितः}
{पुत्रशोकोद्भवं वारि नेत्राभ्यां समवासृजत्}


\twolineshloka
{रथेन कर्णस्तेजस्वी जगामाभिमुखो रिपुम्}
{युद्धायामर्पताम्राक्षः समाहूय धनञ्जयम्}


\twolineshloka
{तौ रथौ सूर्यसङ्काशौ वैयाघ्रपरिवारितौ}
{समेतौ ददृशुस्तत्र द्वाविवार्कौ समुद्गतौ}


\threelineshloka
{श्वेताश्वौ पुरुषादित्यावास्थितावरिमर्दनौ}
{शुशुभाते महात्मानौ चन्द्रादित्यौ यथा दिवि}
{`रथौ चतुर्भिर्जलदैर्भगमित्राविवाम्बरे'}


\twolineshloka
{तौ दृष्ट्वा विस्मयं जग्मुः सर्वसैन्यानि मारिष}
{त्रैलोक्यविजये यत्ताविन्द्रवैरोचनाविव}


\twolineshloka
{रथज्यातलनिर्हादैर्बाणशङ्खरवैस्तथा}
{तौ रथावभ्यधावन्त क्षत्रियाः सर्व एव हि}


\twolineshloka
{ध्वजावालोक्य वीराणां विस्मयः समपद्यत}
{हस्तिकक्ष्यां च कर्णस्य वानरं च किरीटिनः}


\twolineshloka
{तौ रथौ सम्प्रसक्तौ तु दृष्ट्वा भारत पार्थिवाः}
{सिंहनादरवांश्चक्रुः साधुवादांश्च पुष्कलान्}


\twolineshloka
{श्रुत्वा तयोर्द्वैरथं च तत्र योधाः सहस्रशः}
{चक्रुर्बाहुस्वनांश्चैव तथा बाणरवं महत्}


\twolineshloka
{आजघ्नुः कुरवस्तत्र वादित्राणि समन्ततः}
{राधेयमभितो दध्मुः शङ्खाञ्शतसहस्रशः}


\twolineshloka
{तथैव पाण्डवाः सर्वे हर्षयन्तो धनञ्जयम्}
{तूर्यशङ्खनिनादेन दिशः सर्वा व्यनादयन्}


\twolineshloka
{क्ष्वेलितास्फोटितोत्क्रुष्टैस्तुमुलं सर्वतोऽभवत्}
{बाहुशब्दैश्च शूराणां कर्णार्जुनसमागमे}


\twolineshloka
{तौ दृष्ट्वा पुरुषव्याघ्रौ रथस्थौ रथिनां वरौ}
{प्रगृहीतमहाचापौ शरशक्तिध्वजायुतौ}


\twolineshloka
{वर्मिणौ बद्धनिस्त्रिंशौ श्वेताश्वौ शङ्खशोभितौ}
{तूणीरवसम्पन्नौ द्वावप्येतौ सुदर्शनौ}


\twolineshloka
{रक्तचन्दनदिग्धाङ्गौ समदौ गोवृषाविव}
{चापविद्युद्ध्वजोपेतौ शस्त्रसम्पत्तियोधिनौ}


\twolineshloka
{चामरव्यजनोपैतौ श्वेतच्छत्रोपशोभितौ}
{कृष्णशल्यरथोपेतौ तुल्यरूपौ महारथौ}


\threelineshloka
{सिंहस्कन्धौ दीर्घभुजौ रक्ताक्षौ हेममालिनौ}
{सिंहस्कन्धप्रतीकाशौ व्यूढोरस्कौ महाबलौ}
{अन्योन्यवधमिच्छन्तावन्योन्यजयकाङ्क्षिणौ}


\twolineshloka
{अन्योन्यमभिधावन्तौ गोष्ठे गोवृषभाविव}
{प्रभिन्नाविव मातङ्गौ सुसंरब्धाविवाचलौ}


\twolineshloka
{आशीविषशिशुप्रख्यौ यमकालान्तकोपमौ}
{इन्द्रवृत्राविव क्रुद्धौ सूर्यचन्द्रसमप्रभौ}


\twolineshloka
{महाग्रहाविव क्रुद्धौ युगान्ताय समुत्थितौ}
{देवगर्भौ देवसमौ देवतुल्यौ च रूपतः}


\twolineshloka
{यदृच्छया समायातौ सूर्याचन्द्रमसौ यथा}
{बलिनौ समरे दृप्तौ नानाशस्त्रधरौ युधि}


\twolineshloka
{तौ दृष्ट्वा पुरुषव्याघ्रौ शार्दूलाविव धिष्ठितौ}
{बभूव परमो हर्षस्तावकानां विशाम्पते}


\twolineshloka
{संशयः सर्वभूतानां विजये समपद्यत}
{समेतौ पुरुषव्याघ्रौ प्रेक्ष्य कर्णधनञ्जयौ}


\twolineshloka
{उभौ वरायुधधरावुभौ रणकृतश्रमौ}
{उभौ च बाहुशब्देन नादयन्तौ नभस्तलम्}


\twolineshloka
{उभौ विश्रुतकर्माणौ पौरुषेण बलेन च}
{उभौ च सदृशौ युद्धे शम्बरामरराजयोः}


\twolineshloka
{कार्तवीर्यसमौ चोभावुभौ दाशरथेः समौ}
{विष्णुवीर्यसमौ चोभावुभौ भवसमौ युधि}


\twolineshloka
{उभौ श्वेतहयौ राजन्रथप्रवरवाहिनौ}
{सारथिप्रवरौ चापि उभौ मद्रजनार्दनौ}


\twolineshloka
{ततो दृष्ट्वा महाराज राजमानौ महारथौ}
{सिद्धचारणसङ्घानां विस्मयः समपद्यत}


\twolineshloka
{धार्तराष्ट्रास्ततस्तूर्णं सबला भरतर्षभ}
{परिवव्रुर्महात्मानं कर्णमाहवशोभिनम्}


\twolineshloka
{तथैव पाण्डवा दृष्ट्वा धृष्टद्युम्नपुरोगमाः}
{यमौ च चेकितानश्च प्रहृष्टाश्च प्रभद्रकाः}


\twolineshloka
{नानादेश्याश्च ये शूराः शिष्टा युद्धाभिनन्दिनः}
{ते सर्वे सहिता हृष्टाः परिवव्रुर्धनञ्जयम्}


\twolineshloka
{रिरक्षिषन्तः शथ्रुघ्नाः पत्त्यश्वरथकुञ्जराः}
{धनञ्जयस्य विजये धृताः कर्णवधेऽपि च}


\twolineshloka
{तथैव तावकाः सर्वे यत्ताः सेनाप्रहारिणः}
{दुर्योधनमुखा राजन्कर्णं जुगुपुराहवे}


\twolineshloka
{तावकानां रणे कर्णो ग्लहो ह्यासीद्विशाम्पते}
{तथैव पाण्डवेयानां ग्लहः पार्थोऽभवत्तदा}


\twolineshloka
{तयोस्तु सभ्यास्तत्रासन्प्रेक्षकाश्चाभवन्युधि}
{तत्रैषां ग्लहमानानां ध्रुवौ जयपराजयौ}


\twolineshloka
{ताभ्यां द्यतं समासक्तं विजयायेतराय वा}
{अस्माकं पाण्डवानां च स्थितानां रणमूर्धनि}


\twolineshloka
{तौ तु स्थितौ महाराज समरे युद्धशालिनौ}
{अन्योन्यं प्रतिसंरब्धावन्योन्यवधकाङ्क्षिणौ}


\twolineshloka
{तावुमौ प्रजिहीर्षन्ताविन्द्रवृत्राविव प्रभो}
{भीमरूपधरावास्तां महाधूमाविव ग्रहौ}


\twolineshloka
{ततोऽन्तरिक्षे सञ्जज्ञे विवादो भरतर्षभ}
{मिथो भेदाश्च भूतानामासन्कर्णार्जुनान्तरे}


\threelineshloka
{व्याश्रयन्त द्विधा भिन्नाः सर्वे लोकास्तु मारिष}
{देवदानवगन्धर्वाः पिशाचोरगराक्षसाः}
{प्रतिपक्षग्रहं चक्रुः कर्णार्जुनसमागमे}


\twolineshloka
{द्यौरासीत्कर्णपक्षेऽत्र सनक्षत्रा विशाम्पते}
{भूर्विशाला पार्थमाता पुत्रस्य जयकाङ्क्षिणी}


\twolineshloka
{सागराश्चैव गिरयः सरितश्च नरोत्तम}
{महीजा जलजाश्चैव व्याश्रयन्त किरीटिनम्}


\twolineshloka
{असुरा यातुधानाश्च गुह्यकाश्च परन्तप}
{कर्णः समभवद्यत्र खेचराणि वयांसि च}


\twolineshloka
{रत्नानि निधयः सर्वे वेदाश्चाख्यानपञ्चमाः}
{सोपवेदोपनिषदो व्याश्रयन्त किरीटिनम्}


\threelineshloka
{वासुकिश्चित्रसेनश्च तक्षकश्चोपतक्षकः}
{महीवियज्जलचराः काद्रवेयाश्च सान्वयाः}
{विषवन्तो महानागा वेगिनश्चार्जुनेऽभवन्}


\twolineshloka
{ऐरावताः सौरभेया वैशालेयाश्च भोगिनः}
{एतेऽभवन्नर्जुनस्य पापाः सर्पाश्च कर्णतः}


\threelineshloka
{ईहामृगा व्यालमृगा मङ्गला मृगपक्षिणः}
{मङ्गलाः पशवश्चैव सिंहव्याघ्रास्तथैव च}
{पार्थस्य विजये राजन्सर्व एव समाश्रिताः}


\threelineshloka
{वसवो मरुतः साध्या रुद्रा देवाश्विनावपि}
{अग्नी रुद्रश्च सोमश्च पन्नगाश्च दिशो दश}
{कर्णतः समपद्यन्त श्वसृगालवयांसि च}


\twolineshloka
{वसवश्च महेन्द्रेण मरुतश्च सहाग्निना}
{धनञ्जयस्य ते वर्गा आदित्याः कर्णतोऽभवन्}


\threelineshloka
{देवताः पितृभिः सार्धमृषिभिश्च परन्तप}
{तुम्बुरुप्रमुखाः सर्वे गन्धर्वा भरतर्षभ}
{यमौ वैश्रवणश्चैव वरुणश्च यतोऽर्जुनः}


% Check verse!
देवर्षिब्रह्मर्षिगणाः सर्वे च खचराश्च येप्रालेयाः सहमौनेयाः शुभाश्चाप्सरसां गणाः
\twolineshloka
{सहाप्सरोभिः शुभ्राभिर्देवदूताश्च गुह्यकाः}
{किरीटिनं संश्रिताः स्म पुण्यगन्धा मनोरमाः}


\twolineshloka
{अमनोज्ञाश्च ये गन्धास्ते सर्वे कर्णमाश्रिताः}
{विपरीतान्यनिष्टानि भवन्ति विनशिष्यताम्}


\threelineshloka
{ये त्वन्तकाले पुरुषं विपरीतमुपाश्रितम्}
{प्रविशन्ति नरं क्षिप्रं मृत्युकालेऽभ्युपागते}
{ते भावाः सहिताः कर्णं प्रविष्टाः सूतनन्दनम्}


\twolineshloka
{ओजस्तेजश्च सिद्धिश्च प्रहर्षः सत्यविक्रमौ}
{मनस्तुष्टिर्जयश्चापि तथाऽऽनन्दो नृपोत्तम}


\twolineshloka
{ईदृशार्नि नरव्याघ् तस्मिन्सङ्ग्रामसागरे}
{निमित्तानि च शुभ्राणि विविशुर्जिष्णुमाहवे}


% Check verse!
ऋषयो ब्राह्मणैः सार्धमभजन्त किरीटिनम्
\twolineshloka
{ततो देवगणैः सार्धं सिद्धाश्च सह चारणैः}
{द्विधा भूता महाराज व्याश्रयन्त नरोत्तमौ}


\threelineshloka
{विमानानि विचित्राणि गुणवन्ति च सर्वशः}
{समारुह्य समाजग्मुर्द्वैरथं कर्णपार्थयोः}
{अन्तरिक्षे महाराज देवगन्धर्वराक्षसाः}


\threelineshloka
{एवं सर्वेषु भूतेषु द्विधा भूतेषु भारत}
{आशंसमानेषु जयं राधेयस्यार्जुनस्य च}
{विमानायुतसम्बाधमाकाशमभवत्तदा}


\threelineshloka
{ईहामृगव्यालमृगैर्दिपाश्वरथपङ्क्तिभिः}
{ऊह्यमानाः परे मेघैर्वायुना च मनीषिणः}
{दिदृक्षवः समाजग्मुः कर्णार्जुनसमागमम्}


\twolineshloka
{देवदानवगन्धर्वा नागयक्षपतत्रिणः}
{महर्षयो देवगणाः पितरश्च स्वधाभुजः}


\twolineshloka
{तपोविद्यौषधीसिदधा नानारूपाम्बरत्विषः}
{अन्तरिक्षे महाराज विवदन्तोऽवतस्थिरे}


\twolineshloka
{ब्रह्मा ब्रह्मर्षिभिः सार्धं प्रजापतिभिरेव च}
{आस्थितो यानमाकाशे दिव्यं तेजः समागताः}


\twolineshloka
{ततः प्रजापतिस्तूर्णमाजगाम महामते}
{द्वैरथं युधि तं द्रुष्टं कर्णपाण्डवयोस्तदा}


\twolineshloka
{विजित्य कर्णः स्विदिमां वसुन्धरा--मथार्जुनः स्वित्प्रतिपद्यतेऽखिलाम्}
{इतीश्वरस्यापि बभूव संशयःप्रजापतेः प्रेक्ष्य तयोर्महद्बलम्}


\chapter{अध्यायः ९६}
\twolineshloka
{सञ्जय उवाच}
{}


\twolineshloka
{प्रजापतिस्तु तं दृष्ट्वा देवभागं समागतम्}
{अब्रवीत्तु ततो राजन्पश्यतो वै स्वयंभुवः}


\twolineshloka
{उभावतिरथौ शूरावुभौ दृढपराक्रमौ}
{उभौ सदृशकर्माणौ वज्रिचक्रायुधौपमौ}


\twolineshloka
{अहो बत महद्युद्धं कर्णार्जुनसमागमे}
{भविष्यति महाघोरं वृत्रवासवयोरिव}


\twolineshloka
{प्रजापतिरथोक्त्वैवं स्वयम्भुवमचोदयत्}
{समोऽस्तु विजयः काममुभयोर्नरसिंहयोः}


\twolineshloka
{कर्णार्जुनविवादेन मा नश्येदखिलं जगत्}
{स्वयंभो ब्रूहि तद्वाक्यं समोऽस्तु विजयोऽनयोः}


\twolineshloka
{एवमुक्तस्तु भगवाञ्जये ताभ्यामनिश्चिते}
{इत्यब्रवीन्महाराज महाब्रह्मा प्रजापतिम्}


\twolineshloka
{द्वावप्येतौ हि कृतिनौ द्वावप्यतिबलोत्कटौ}
{भविष्यत्यनयोर्युद्धं त्रैलोक्यस्य भयावहम्}


\twolineshloka
{ततः प्रजापतिं तत्र सहस्राक्षोऽभ्यचोदयत्}
{विजयो ध्रुव एवास्तु पाण्डवस्य महात्मनः}


\twolineshloka
{मनस्वी बलवाञ्शूरः कृतविद्यस्तपोधनः}
{विभर्ति च महातेजा धनुर्वेदमशेषतः}


\twolineshloka
{पार्थः सर्वगुणोपेतो देवकार्यमिदं यतः}
{क्लिश्यन्ते पाण्डवा नित्यं वनवासादिभिर्भृशं}


\twolineshloka
{सम्पन्नस्तपसा चैव पर्याप्तः पुरुषर्षभः}
{अतिक्रामेच्च माहात्म्याद्दिष्टमप्यविचारयन्}


\twolineshloka
{अतिक्रमे च लोकानामभावो नियतो भवेत्}
{नावस्थानं च पश्यामि क्रुद्धयोः कृष्णयोः क्वचित्}


\twolineshloka
{स्रष्टारौ जगतश्चैतौ ततश्च पुरुषर्षभौ}
{नरनारायणावेतौ पुराणावृषिसत्तमौ}


\twolineshloka
{अनियम्यौ नियन्तारौ जगतः पुरुषर्षभौ}
{[नैतयोस्तु समः कश्चिद्दिवि वा मानुषेषु वा}


\threelineshloka
{अनुगम्यास्त्रयो लोकाः सह देवर्षिचारणैः}
{सर्वदेवगणाश्चापि सर्वभूतानि यानि च}
{अनयोस्तु प्रभावेन वर्तते निखिलं जगत् ॥]}


\twolineshloka
{कामं तु सुकृताँल्लोकानाप्नोतु पुरुषर्षभः}
{कर्णो वैकर्तनः शूरो विजयस्त्वस्तु कृष्णयोः}


\twolineshloka
{वसूनां समलोकत्वं मरुतां वासमाप्नुयात्}
{सहितो द्रोणभीष्माभ्यां नाकपृष्ठे महीयताम्}


\twolineshloka
{`क्लेशितो हि वने पार्थो दिर्घकालं पितामह}
{तस्मादेष जयेद्युद्धे तपसाऽभ्यधिकोऽर्जुनः}


\twolineshloka
{पूर्वं भगवता प्रोक्तः कृष्णयोर्विजयो ध्रुवः}
{तत्तथास्तु नमस्तेऽस्तु प्रमो ब्रूहि पितामह}


\twolineshloka
{तत्सहस्राक्षवचनं निशम्य भगवान्प्रभुः}
{नोवाच तज्जयं तुल्यं तयोः कर्णकिरीटिनोः}


\twolineshloka
{तस्मादाशां गतः शक्रस्तूष्णीम्भूते पितामहे}
{विजयः पाण्डवेयस्य कर्णस्य च वधो भवेत्}


\twolineshloka
{ब्रह्मेशानौ ततो वाक्यमूचतुर्भुवनेश्वरम्}
{विजयो ध्रुव एवास्तु पाण्डवस्य महात्मनः}


\twolineshloka
{मनस्वी जयतां शूरः कृतविद्यस्तपोधनः}
{बिभर्ति च महातेजा धनुर्वेदमशेषतः}


\twolineshloka
{अतिक्रामेच्च माहात्म्याद्दिष्टमप्यविचारयन्}
{अतिक्रमे च लोकानामभावो नियतो भवेत्}


\twolineshloka
{न च विद्म ह्यवस्थानं क्रुद्धयोः कृष्णयोः क्वचित्}
{स्रष्टारौ जगतश्चैव सतश्च पुरुषर्षभौ}


\twolineshloka
{कामं तु सुकृताँल्लोकान्प्राप्नोत्वेष परन्तपः}
{कर्णो वैकर्तनः शूरो विजयस्तु नरे ध्रुवः}


\twolineshloka
{तथोक्ते देवदेवाभ्यां सहस्राक्षोऽब्रवीद्वचः}
{आमन्त्र्य सर्वभूतानि ब्रह्मेशानानुशासनात्}


\twolineshloka
{श्रुतं भवद्भिर्यत्प्रोक्तं भगवद्भ्यां जगद्धितम्}
{तत्तथा व्येतु ते रोगः शमाप्नुत विमन्यवः}


\threelineshloka
{इति श्रुत्वेन्द्रवचनं सर्वभूतानि मानिष}
{विस्मितान्यभवन्राजन्पूजयाञ्चक्रिरे तदा}
{नोचुस्तदा जयं तुल्यं तयोः पुरुषसिंहयोः}


\twolineshloka
{व्यसृजंश्च सुगन्धीनि पुष्पवर्षाणि हर्षिताः}
{नानारूपाणि विबुधा देवतूर्याण्यवादयन्}


\twolineshloka
{दिदृक्षवश्चाप्रतिमं द्वैरथं नरसिंहयोः}
{विस्मयोत्फुल्लनयना नान्या बुबुधिरे क्रियाः}


\chapter{अध्यायः ९७}
\twolineshloka
{सञ्जय उवाच}
{}


\threelineshloka
{रथौ तयोः श्वेतहयौ दिव्यौ युक्तौ महास्वनौ}
{समास्थितौ लोकवीरौ शङ्खान्दध्मुः पृथक्पृथकम्}
{वासुदेवार्जुनौ वीरौ कर्णशल्यौ च भारत}


\twolineshloka
{तद्भीरुसन्त्रासकरं युद्धं समभवत्तदा}
{अन्योन्यस्पांधेनोरुग्रं शक्रशम्बरयोरिव}


\twolineshloka
{तयोर्ध्वजौ हि विमलौ शुशुभाते रथे स्थितौ}
{राहुकेतू यथाऽऽकाशे उदितौ जगतः क्षये}


\twolineshloka
{कर्णस्याशीविषनिभा रत्नसारमयी दृढ़ा}
{पुरन्दरधनुः प्रख्या हस्तिकक्ष्या विराजते}


\twolineshloka
{कपिश्रेष्ठस्तु पार्थस्य व्यादितास्य इवान्तकः}
{दंष्ट्राभिर्भीषयन्भाभिर्दुर्निरीक्ष्यो रविर्यथा}


\twolineshloka
{युद्धाभिलाषुको भूत्वा ध्वजो गाण्डीवधन्वनः}
{कर्णध्वजमुपातिष्ठत्स्वस्थानाद्वेगवान्कपिः}


\twolineshloka
{उत्पत्य च महावेगः कक्ष्यामभ्याहनत्तदा}
{नखैश्च दशनैश्चैव गरुडः पन्नगं यथा}


\twolineshloka
{सा किङ्किणीकाभरणा कालपाशोपमा तदा}
{अभ्यद्रवत्सुसंरब्धा हस्तिकक्ष्याऽथ तं कपिम्}


\twolineshloka
{तयोर्घोरतरे युद्धे द्वैरथे द्यूत आहिते}
{प्राकुर्वतां ध्वजौ युद्धं पूर्वं पूर्वतरं तदा}


\threelineshloka
{हया हयानभ्यहेषन्स्पर्धमानाः परस्परम्}
{अविध्यत्पुण्डरीकाक्षः शल्यं नयनसायकैः}
{शल्यश्च पुण्डरीकाक्षं तथैवाभिसमैक्षत}


\twolineshloka
{तत्राजयद्वासुदेवः शल्यं नयनसायकैः}
{कर्णं चाप्यजयदृष्ट्या कुन्तीपुत्रो धनञ्जयः}


% Check verse!
अयाब्रवीत्सूतपुत्रः शल्यमाभाष्य सस्मितम्
\threelineshloka
{यदि पार्थो रणे हन्यादद्य मामिह कर्हिचित्}
{किं करिष्यसि सङ्घामे शल्य सत्यमथोच्यताम् ॥शल्य उवाच}
{}


\threelineshloka
{यदि कर्ण रणे हन्यादद्य त्वां श्वेतवाहनः}
{उभावेकरयेनाहं हन्यां माधवपाण्डवौ ॥सञ्जय उवाच}
{}


\twolineshloka
{एवमेव तु गोविन्दमर्जुनः प्रत्यभाषत}
{तं प्रहस्याब्रवीत्कृष्णः पार्थं परमिदं वचः}


\twolineshloka
{पतेद्दिवाकरः स्थानाच्छीर्येद्भूमिरनेकधा}
{शैल्यमग्निरियान्न त्वां हन्यात्कार्णो धनञ्जय}


\twolineshloka
{यदि चैतत्कथञ्चित्स्याल्लोकपर्यसनं भवेत्}
{हन्यां कर्णं तथा शल्यं बाहुभ्यामेव संयुगे}


\twolineshloka
{इति कृष्णवचः श्रुत्वा प्रहसन्कपिकेतनः}
{अर्जुनः प्रत्युवाचेदं कृष्णमक्लिष्टकारिणम्}


\twolineshloka
{मम तावदपर्याप्तौ कर्णशल्यौ जनार्दन}
{सपताकध्वजं कर्णं सशल्यरथवाजिनम्}


\twolineshloka
{सच्छत्रकवचं चैव सशक्तिशरकार्मुकम्}
{द्रष्टाऽस्यद्य रणे कृष्ण शरैश्छिन्नमनेकधा}


\twolineshloka
{अद्यैव सरथं साश्वं सशक्तिकवचायुधम्}
{सञ्चूर्णितमिवारण्ये वृक्षं पश्याद्य दन्तिना}


\twolineshloka
{अद्य राधेयभार्याणां वैधव्यं समुपस्थितम्}
{ध्रुवं स्वप्नेष्वनिष्टानि ताभिर्दृष्टानि माधव}


\twolineshloka
{द्रष्टासि ध्रुवमद्यैव विधवाः कर्णयोषितः}
{न हि मे शाम्यते मन्युर्यदनेन पुरा कृतम्}


\twolineshloka
{कृष्णां सभागतां दृष्ट्वा मूढेनादीर्घदर्शिना}
{अस्मांस्तथाऽवहसता क्षिपता च पुनःपुनः}


\twolineshloka
{अद्य द्रष्टासि गोविन्द कर्णमुन्मथितं मया}
{वारणेनेव मत्तेन पुष्पितं गजतीरुहम्}


\twolineshloka
{अद्य ता मधुरा वाचः श्रोतासि मधुसूदन}
{दिष्ट्या जयसि वार्ष्णेय इति कर्णे निपातिते}


\twolineshloka
{अद्याभिमन्युजननीं प्रहृष्टः सांत्वयिष्यसि}
{कुन्तीं पितृष्वसारं च प्रहृष्टः सन् जनार्दन}


\twolineshloka
{अद्य बाष्पमुखीं कृष्णां सान्त्वयिष्यसि माधव}
{वाग्भिश्चामृतकल्पाभिर्धर्मराजं च पाण्डवम्}


\chapter{अध्यायः ९८}
\twolineshloka
{सञ्जय उवाच}
{}


\twolineshloka
{तद्देतनागासुरसिद्धयक्षै--र्गन्धर्वरक्षोप्सरसां च सङ्घैः}
{ब्रह्मर्षिराजर्षिसुपर्णजुष्टंबभौ वियद्विस्मयनीयरूपम्}


\twolineshloka
{नानद्यमानं निनदैर्मनोज्ञै--र्वादित्रगीतस्तुतिनृत्यहासैः}
{सर्वेऽन्तरिक्षं ददृशुर्मनुष्याःखस्थाश्च तद्विस्मयनीयरूपम्}


\twolineshloka
{ततः प्रहृष्टाः कुरुपाण्डुयोधावादित्रशङ्खस्वनसिंहनादैः}
{विनादयन्तो वसुधां दिशश्चस्वनेन सर्वान्द्विषतो निजघ्नुः}


\twolineshloka
{नराश्वमातङ्गरथैः समाकुलंशरासिशक्त्यृष्टिनिपातदुःसहम्}
{अभीरुजुष्टं हतदेहसङ्कुलंरणाजिरं लोहितमाबभौ तदा}


\threelineshloka
{[बभूव युद्धं कुरुपाण्डवानांयथा सुराणामसुरैः सहाभवत्}
{]तथा प्रवृते तुमुले सुदारुणेधनञ्जयश्चाधिरथिश्च सायकैः}
{दिशश्च सैन्यं च शितैरजिह्मगैःपरस्परं प्रावृणुतां सुदंशितौ}


\twolineshloka
{ततस्त्वदीयाश्च परे च सायकैःकृतेऽन्धकारे ददृशुर्न किञ्चन}
{भयातुरा एकरथौ समाश्रयं--स्ततोऽभवत्त्वद्भुतमेव सर्वतः}


\twolineshloka
{ततोऽस्त्रमस्त्रेण परस्परं तौविधूय वाताविव पूर्वपश्चिमौ}
{घनान्धकारे वितते तमोनुदौयथोदितौ तद्वदतीव रेजतुः}


\twolineshloka
{न चाभिसर्तव्यमिति प्रचोदिताःपरे त्वदीयाश्च तथाऽवतस्थिरे}
{महारथौ तौ परिवार्य सर्वतःसुरासुराः शम्बरवासवाविव}


\twolineshloka
{मृदङ्गभेरीपणवानकस्वनै--र्निनादिते वाहनशङ्खनिस्वनैः}
{तौ सिंहनादैर्बभतुर्नरोत्तमौपतङ्गचन्द्राविव दुःसहावुभौ}


\twolineshloka
{महाधनुर्मण्डुलमध्यगावुभौसुवर्चसौ बाणसहस्रदीधिती}
{दिधक्षमाणौ सचराचरं जग--द्युगान्तसूर्याविव दुःसहौ रणे}


\twolineshloka
{उभावजेयावहितान्तकावुभा--वुभौ जिघांसू कृतिनौ परस्परम्}
{महाहवे वीतभयौ समीयतु--र्महेन्द्रजम्भाविव कर्णपाण्डवौ}


\twolineshloka
{ततो महास्त्राणि महाधनुर्धरौविमुञ्चमानाविषुभिर्भयानकैः}
{नराश्वनागानमितान्निजघ्नतुःपरस्परं चापि महारथौ नृप}


\twolineshloka
{ततो विसस्रुः पुनरर्दिता नरानरोत्तमाभ्यां कुरुपाण्डवाश्रयाः}
{सनागपत्त्यश्वरथा दिशो द्रुता--स्तथा यथा सिंहहता वनौकसः}


\twolineshloka
{ततस्तु दुर्योधनभोजसौबलाःकृपो गुरोश्चापि सुतो महाहवे}
{महारथाः पञ्च धनञ्जयाच्युतौशरैः शरीरार्तिकरैरताडयन्}


\twolineshloka
{धनूंषि तेषामिषुधीन्ध्वजान्हया--न्रथांश्च सूतांश्च धनञ्जयः शरैः}
{समं प्रमध्याशु परान्समन्ततःशरोत्तमैर्द्वादशभिन्न सूतजम्}


\twolineshloka
{अथाभ्यधावंस्त्वरिताः शतं रथाःशतं गजाश्चार्जुनमाततायिनः}
{शकास्तुषारा यवनाश्च सादिनःसहैव काम्भोजवरैर्जिघांसवः}


\twolineshloka
{वरायुधान्पाणिगतैः शरैः सहक्षुरैर्न्यकृन्तत्प्रपतञ्शिरांसि च}
{हयांश्च नागांश्च रथांश्च युध्यतोधनञ्जयऋ शत्रुगणान्क्षितौ क्षिणौत्}


\twolineshloka
{ततोऽन्तरिक्षे सुरतूर्यनिःस्वनाःससाधुवादा हृषितैः समीरिताः}
{निपेतुरप्युत्तमपुष्पवृष्टयःसुरूपगन्धाः पवनेरिताः शुभाः}


\twolineshloka
{तदद्भुतं देवमनुष्यसाक्षिकंसमीक्ष्य भूतानि विसिस्मियुस्तदा}
{तवात्मजः सूतसुतश्च न व्यथांन विस्मयं जग्मतुरेकनिश्चयौ}


\twolineshloka
{अथाब्रवीद्द्रोणसुतस्तवात्मजंकरं करेण प्रतिपीड्य सान्त्वयन्}
{प्रसीद दुर्योधन शाम्य पाण्डवै--रलं विरोधेन धिगस्तु विग्रहम्}


\twolineshloka
{हतो गुरुर्ब्रह्मसमो महास्त्रवि--त्तथैव भीष्मप्रमुखा महारथाः}
{अहं त्ववध्यो मम चापि मातुलःप्रशाधि राज्यं सह पाण्डवैश्चिरम्}


\twolineshloka
{धनञ्जयः स्यास्यति वारितो मयाजनार्दनो नैव विरोधमिच्छति}
{युधिष्ठिरो भूतहिते रतः सदावृकोदरस्तद्वशगस्तथा यमौ}


\threelineshloka
{त्वया तु पार्थैश्च कृते च संविदेप्रजाः शिवं प्राप्नुयुरिच्छया तव}
{व्रजन्तु शेषाः स्वपुराणि पार्थिवानिवृत्तवैराश्च भवन्तु सैनिकाः}
{न चेद्वचः श्रोष्यसि मे नराधिपध्रुवं हि तप्तोऽसि हतोऽरिभिर्युधि}


\twolineshloka
{वृद्धं पितरमालोक्य गान्धारीं च यशस्विनीम्}
{कृपालुर्धर्मराजो हि याचितः शममेष्यति}


\twolineshloka
{यथोचितं च वै राज्यमनुज्ञास्यति ते प्रभुः}
{विपश्चित्सुमतिर्वीरः सर्वशास्त्रार्थतत्त्ववित्}


\twolineshloka
{वैरं नेष्यति धर्मात्मा स्वजने नास्त्यतिक्रमः}
{न विग्रहमतिः कृष्णा स्वजने प्रतिनन्दति}


\threelineshloka
{भीमसेनार्जुनौ चोभौ माद्रीपुत्रौ च पाण्डवौ}
{वासुदेवमते चैव पाण्डवस्य च धीमतः}
{स्थास्यन्ति पुरुषव्याघ्रास्तयोर्वचनगौरवात्}


\twolineshloka
{रक्ष दुर्योधनात्मानमात्मा सर्वस्य भाजनम्}
{जीवने यत्नमातिष्ठ जीवन्भद्राणि पश्यति}


\twolineshloka
{राज्यं श्रीश्चैव भद्रं ते जीवमाने च कल्पते}
{मृतस्य तु खलु कौरव्य नैव राज्यं कुतः सुखम्}


\twolineshloka
{लोकवृत्तमिदं वृत्तं प्रवृत्तं पश्य भारत}
{शाम्य त्वं पाण्डवैः सार्धं शेषं रक्ष कुलस्य च}


\twolineshloka
{मा भूत्स कालः कौरव्य यदाऽहमहितं वचः}
{ब्रूयां कामं महाबाहो माऽवमंस्था वचो मम}


\threelineshloka
{धर्मिष्ठमिदमत्यर्थं राज्ञश्चैव कुलस्य च}
{एतद्धि परमं श्रेयः कुरुवंशस्य वृद्धये}
{प्रयाहि तं च गान्धारे कुलस्य च सुखावहम्}


\twolineshloka
{पथ्यमायतिसंयुक्तं कर्णोऽप्यर्जुनमाहवे}
{न जेष्यति नरव्याघ्र इति मे निश्चिता मतिः}


\twolineshloka
{रोचतां ते नरश्रेष्ठ ममैतद्वचनं शुभम्}
{अतोऽन्यथा हि राजेन्द्र विनाशः सुमहान्भवेत्}


\twolineshloka
{इदं च दृष्टं जगता सह त्वयाकृतं यदेकेन किरीटमालिना}
{यथा न कुर्याद्बलभिन्न चान्तकोन च प्रचेता भगवान्न यक्षराट्}


\twolineshloka
{अतोऽपि भूयान्स्वगुणैर्धनञ्जयोन चातिवर्तिष्यति मे वचोऽखिलम्}
{तवानुयात्रां च सदा करिष्यतिप्रसीद राजेन्द्र शमं त्वमाप्नुहि}


\twolineshloka
{ममाभिमानः परमः सदा त्वयिब्रवीम्यतस्त्वां परमाच्च सौहृदात्}
{निवारयिष्यामि च कर्णमाहवेमहान्हि मेऽस्ति प्रणयोऽथ सूतजे}


\twolineshloka
{वदन्ति मैत्रीं सहजां विचक्षणा--स्तथैव साम्ना च धनेन चार्जिताम्}
{प्रतापतश्चोपनतां चतुर्विधांतदस्ति सर्वं तव पाण्डवेषु}


\twolineshloka
{निसर्गतस्ते तव वीर बान्धवाःपुनश्च साम्ना हितमाप्नुयुः प्रभो}
{त्वयि प्रसन्ने यदि मित्रतां गतेहितं कृतं स्याज्जगतस्त्वयाऽतुलम्}


\twolineshloka
{इतीदमुक्तः सुहृदा वचो हितंविचिन्त्य निःश्वस्य च दुर्मनाऽब्रवीत्}
{यथा भवानाह सखे तथैव त--न्ममापि विज्ञापयतो वचः शृणु}


\twolineshloka
{निहत्य दुःशासनमुक्तवान्बहुप्रहस्य शार्दूलवदेष दुर्मतिः}
{वृकोदरस्तद्वृदये मम स्थितंन तत्परोक्षं भवतः कुतः शमः}


\twolineshloka
{न चापि कर्णं प्रसहेद्रणेऽर्जुनोमहागिरिं मेरुमिवोग्रमारुतः}
{न चाश्वसिष्यन्ति पृथात्मजा मयिप्रसह्य वैरं बहुशो विचिन्त्य}


\twolineshloka
{न चापि कर्णं गुरुपुत्र संयुगा--दुपारमेत्यर्हसि वीर भाषितुम्}
{श्रमेण युक्तो महताद्य फल्गुन--स्तमेष कर्णः प्रसभं हनिष्यति}


\twolineshloka
{`वसुन्धरायाः परिवर्तनं भवे--द्व्रजेच्च शोषं मकरालयोऽर्णवः}
{प्लवेयुरप्यद्रिवरा महाम्बुधौन चार्जुनो जेष्यति कर्णमाहवे}


\twolineshloka
{अपां पृथिव्या नभसो नभस्वतःसुतिग्मदीप्तेश्च हिरण्यरेतसः}
{अभाव एषामपि सर्वतो भवे--न्न चार्जुनो जेष्यति कर्णमाहवे'}


\twolineshloka
{तमेवमुक्त्वाऽप्यनुनीय चासकृ--त्तवात्मजः स्वाननुशास्ति सैनिकान्}
{द्रुतं घ्नताभिद्रवताहितानिमा--नहीनसत्वाः किमु जोषमासते}


\chapter{अध्यायः ९९}
\twolineshloka
{सञ्जय उवाच}
{}


\twolineshloka
{तौ शङ्खभेरीनिनदे समृद्धेसमीयतुः श्वेतहयौ नराग्र्यौ}
{वैकर्तनः सूतपुत्रोऽर्जुनश्चदुर्मन्त्रिते ते ससुतस्य राजन्}


\twolineshloka
{`आशीविषावग्निमिवोत्सृजन्तौतथा मुखाभ्यामभिनिः श्वसन्तौ}
{यशस्विनौ जज्वलतुर्मृधे तदाघृतावसिक्ताविव हव्यवाहौ'}


\twolineshloka
{यथा गजौ हैमवतौ प्रभिन्नौप्रवृद्धदन्ताविव वासितार्थे}
{तथा समाजग्मतुरुग्रवीर्यौधनञ्जयश्चाधिरथिश्च वीरौ}


\twolineshloka
{बलाहकेनेव महाबलाहकोयदृच्छया वा गिरिणा यथा गिरिः}
{तथा धनुर्ज्यातलनेमिनिःस्वनौसमीयतुस्ताविषुवर्षवर्षिणौ}


\twolineshloka
{शरास्त्रशक्त्यृष्टिगदासिसर्पौरोषानिलोद्वूतमहोर्मिमालौ}
{यथाऽचलौ द्वौ चलतस्तथा तौयथाऽर्णवौ चाशु चतुर्युगान्ते'}


\twolineshloka
{प्रवृद्धशृङ्गद्रुमवीरुदोषधीप्रवृद्वनानाविधनिर्झरौघौ}
{यथाऽचलौ वा चलितौ महाजलै--स्तथा महास्त्रैरितरेतरं हतः}


\twolineshloka
{स सन्निपातो रथयोर्महानभू--त्सुरेशवैरोचनयोर्यथा पुरा}
{शरैर्विनुन्नाश्वनियन्तृदेहयोःसुदुःसहास्त्रैः परिभिन्नदेहयोः}


\twolineshloka
{प्रभूतपद्मोत्पलमत्स्यकच्छपौमहाहदौ पक्षिगणानुनादितौ}
{सुसन्निकृष्टावनिलोद्धतौ यथातथा रथौ तौ ध्वजिनौ समीयतुः}


\twolineshloka
{उभौ महेन्द्रस्य समानविक्रमा--वुभौ महेन्द्रप्रतिमौ महारथौ}
{महेन्द्रवज्रप्रतिमैश्च सायकै--र्महेन्द्रवृत्राविव सम्प्रजघ्नतुः}


\twolineshloka
{सनागपत्त्यश्वरथे उभे बलेविचित्रवर्माभरणाम्बरायुधे}
{चकम्पतुश्चोन्नदतुश्च विस्मया--द्धरा वियच्चार्जुनकर्णसङ्गमे}


\twolineshloka
{भुजाः सवस्त्राङ्गुलयः समुच्छ्रिताःससिंहनादैर्हृषितैर्दिदृक्षुभिः}
{यदाऽर्जुनं मत्त इव द्विपो द्विपंसमभ्ययादाधिरथिर्जिघांसया}


\twolineshloka
{उदक्रोशन्सोमकास्तत्र पार्थंत्वरस्व याह्यर्जुन भिन्धि कर्णम्}
{छिन्ध्यस्य मूर्धानमलं चिरेणश्रद्धां च राज्याद्वृतराष्ट्रसूनोः}


\twolineshloka
{तथाऽस्माकं बहवस्तत्र योधाःकर्णं तथा याहि याहीत्यवोचन्}
{जह्यर्जुनं कर्ण ततः सुदीनाःपुनर्वनं यान्त्वचिराय पार्थाः}


\twolineshloka
{कर्णोऽथ पूर्वं दशभिः पृषत्कै--र्गाण्डीवधन्वानमविध्यदाशु}
{जघान तं चापि ततः किरीटीशरैस्तदाष्टादशभिः सुमुक्तैः}


\twolineshloka
{पुनश्च कर्णस्त्वरितोऽपि पार्थंरथेषुभिस्तं दशभिर्जघान}
{तं चापि पार्थो दशभिः शिताग्रैःकक्ष्यान्तरे तीक्ष्णमुखैरविध्यत्}


\twolineshloka
{कर्णस्ततो भारत साम्परायेघोरेऽतिवेलं रणसंविमर्दी}
{जघान पार्थं नवभिः शिताग्रैःकक्ष्यान्ते नागमिव प्रभिन्नम्}


\twolineshloka
{ततोऽपराभ्यां युधि सूतपुत्रोद्वाभ्यां क्षुराभ्यां हरिमाशुकारी}
{समाजघान त्वरया महात्मायथा सुरेन्द्रं नमुचिः प्रसह्य}


\twolineshloka
{तं पाण़्डवः पञ्चभिरायसाग्रै--राकर्णपूर्णैर्निजघान कर्णम्}
{ते शोणितं तस्य पपुस्तदानींकालस्य दूता इव पार्थबाणाः}


\twolineshloka
{कर्णोऽपि पार्थं सह वासुदेवंसमाचिनोद्भारत वत्सदन्तैः}
{परस्परं तौ विशिखैः प्रमुक्तै--स्ततक्षतुः सूतपुत्रोऽर्जुनश्च}


% Check verse!
परस्परं छिद्रदिदृक्षया चसुभीममभ्याययतुः प्ररुष्टौ
\twolineshloka
{ततोऽस्त्रमाग्नेयममित्रतापनंमुमोच कर्णाय सुरेश्वरात्मजः}
{धनञ्जयात्संयुगमूर्ध्नि निःसृतंतदा प्रजज्वाल तदस्त्रमुत्तमम्}


\twolineshloka
{समीक्ष्य कर्णो ज्वलनास्त्रमुद्यतंस वारुणं तत्प्रशमार्थमाहवे}
{समुत्सृजत्सूतपुत्रः प्रतापवान्स तेन वह्निं शमयाञ्चकार}


\twolineshloka
{वलाहकास्त्रेण दिशस्तरस्वीचकार सर्वास्तिमिरेण संवृताः}
{अपावहन्मेघगणांस्ततस्तान्समीरणास्त्रेण समीरितेन}


\twolineshloka
{ततः सोऽस्त्रं दयितं देवराज्ञःप्रादुश्चक्रे वज्रममित्रतापनः}
{गाण्डीवज्या विमृशंश्चातिमन्यु--र्धनञ्जयः शत्रुसङ्घप्रमाथी}


\twolineshloka
{नाराचनालीकवराहकर्णागाण्डीवतः प्रादुरासन्सुतीक्ष्णाः}
{सहस्रशो वज्रसमानवेगा--स्ते सर्वतः पर्यधावन्त घोराः}


\twolineshloka
{पार्थेषवः कर्णरथं विलग्नाअधोमुखाः पक्षिगणा दिनान्ते}
{निशानिकेतार्थमिवाशु वृक्षंजग्राह तान्सूतपुत्रः पृषत्कैः}


\twolineshloka
{क्षिप्तांस्तथा पाण्डवबाणसङ्घा--नमृष्यमाणस्य धनञ्जयस्य}
{रणाजिरे त्वन्तकतुल्यकर्मावैकर्तनो रोषपरीतचेताः}


\twolineshloka
{ज्योतिष्प्रभां यद्वदुपागतः स--न्दिवाकरो नाशयते क्षणेन}
{पार्थस्य तान्बाणगणान्समग्रा--न्व्यनाशयद्युध्यत एव कर्णः}


\twolineshloka
{रोषात्प्रदीप्तः सुमहाविमर्देभीमस्ततोऽक्रुध्यददीनसत्वः}
{पाणिं स्वपाणौ स विनिष्पिष्य रोषा--दमर्षितो वाक्यमुवाच पार्थम्}


\twolineshloka
{त्वां सूतपुत्रो नु कथं किरीटि--न्रथेषुभिर्हन्ति शिताग्रधारैः}
{धृत्या हि भूतानि ययाऽजयस्त्वंग्रासं ददत्खाण्डवे पावकाय}


\twolineshloka
{धृत्या तया सूतपुत्रं जहि त्व--महं वैनं गदया पोथयिष्ये}
{समेत्य पार्थं सुनृशंसवादीजीवन्नायं यास्यति कालपक्वः}


\chapter{अध्यायः १००}
\twolineshloka
{सञ्जय उवाच}
{}


\twolineshloka
{अथाब्रवीच्चक्रधरोऽपि पार्थंदृष्ट्वा रथेषून्प्रतिहन्यमानान्}
{अमूमुषत्पश्यत एव तेऽद्यह्यस्त्राणि कर्णोऽस्त्रगणैः किरीटिन्}


\twolineshloka
{स पार्थ किं मुह्यसि वेत्सि चैव ता--न्दृष्ट्वा समेतान्नदतः कुरूस्त्वम्}
{कर्णं पुरस्कृत्य विदुर्हि सर्वेतवास्त्रमस्त्रैर्विनिहन्यमानम्}


\threelineshloka
{यया धृत्या तामसं जघ्निवांस्त्वंतत्संयुगे तामसांश्चातिघोरान्}
{दम्भोद्भवं चासुरमाहवे त्वंदर्पोत्सिक्तं वीर्यवन्तं किरीटिन्}
{तया धृत्या त्वं जहि कर्णमद्यपार्थाहवे त्यक्तुमस्त्रं समर्थः}


\twolineshloka
{अनेन चाशु क्षुरनेमिनाद्यमया विसृष्टेन सुदर्शनेन}
{छिन्ध्यस्य मूर्धानमरेः प्रसह्यवज्रेण शक्रो नमुचेरिवारेः}


\twolineshloka
{किरातरूपी भगवान्यया चत्वया महात्मा परितोषितोऽभूत्}
{स तां पुनर्वीर धृतिं गृहीत्वासहानुबन्धं जहि सूतपुत्रम्}


\twolineshloka
{ततो महीं सागरमेखलां त्वंसपत्तगा ग्रामवतीं समृद्धाम्}
{प्रयच्छ राज्ञे निहतारिसङ्घांयशश्च पार्थातुलमाप्नुहि त्वम्}


\fourlineindentedshloka
{कर्णं पुरस्कृत्य नदन्तिं सर्वेतवास्त्रमग्र्यं प्रतिहत्य वीराः}
{कुरु प्रयत्नं भरतप्रवीरद्रवन्त्यमी सृञ्जयाः सोमकाश्च}
{दृष्ट्वाथ कर्णं समरे प्रहृष्टंत्वां चापि दृष्ट्वा परिहीयमानम् ॥सञ्जय उवाच}
{}


\twolineshloka
{सञ्चोदितो भीमजनार्दनाभ्यांस्मृत्वा तदात्मानमवेत्य सर्वम्}
{इहात्मनश्चागमने विदित्वाप्रयोजनं केशवमित्युवाच}


\twolineshloka
{प्रादुष्करोभ्येष महास्त्रमुग्रंशिवाय लोकस्य वधाय सौतेः}
{तन्मेऽनुजानातु भवान्सुराश्चब्रह्मा शिवो ब्रह्मविदश्च सर्वे}


\twolineshloka
{इति स्मोक्त्वा पाण्डवः सव्यसाचीनमस्कृत्वा ब्रह्मणे सोऽमितात्मा}
{अनुत्तमं ब्राह्ममसह्यमस्त्रंप्रादुश्चक्रे मनसा यद्विधेयम्}


\twolineshloka
{ततो दिशश्च प्रदिशश्च सर्वाःसमावृणोत्सायकैर्भूरितेजाः}
{गाण्डीवमुक्तैर्भुजगैरिवोग्रै--र्दिवाकरांशुप्रतिमैर्ज्वलद्भिः}


\twolineshloka
{सृष्टास्तु बाणा भरतर्षभेणशतं शतं भीममुखाः सुतीक्ष्णाः}
{प्राच्छादयन्कर्णरथं क्षणेनयुगान्तकालार्ककरप्रकाशाः}


\twolineshloka
{वैकर्तनेनाशु तदाजिमध्येसहस्रशो बाणगणा विसृष्टाः}
{ते चाक्षयाः पाण्डवमभ्युपेयुःपर्जन्यसृष्टा इव वारिधाराः}


\twolineshloka
{ततः स कृष्णं च किरीटिनं चवृकोदरं चाप्रतिमप्रभावः}
{त्रिभिस्त्रिभिर्भीमबलो निहत्यननाद घोरं महता स्वनेन}


\twolineshloka
{ततः स बाणाभिहतः किरीटीभीमं तथा प्रेक्ष्य जनार्दनं च}
{अमृष्यमाणः पुनरेव पार्थःशरान्दशाष्टौ सममुद्ववर्ष}


\twolineshloka
{स केतुमेकेन शरेण विद्ध्वाशल्यं चतुर्भिस्त्रिभिरेव कर्णम्}
{ततः सुमुक्तैर्दशभिर्जघानसेनापतिं काञ्चनवर्मनद्वम्}


\twolineshloka
{स राजपुत्रो विशिरा विबाहु--र्विवाजिसूतो विधनुर्विकेतुः}
{अथो रथाग्रादपतत्प्ररुग्णःपरश्वथैः साल इवावकृत्तः}


\threelineshloka
{पुनश्च कर्णं त्रिभिरष्टभिश्चद्वाभ्यां चतुर्भिर्दशभिश्च विद्धा}
{चतुश्शतं द्विरदानां निपात्यरथाञ्जघानाष्टरथान्किरीटि}
{सहस्रमश्वांश्च पुनश्च सादी--नष्टौ सहस्राणि च पत्तिवीरान्}


\twolineshloka
{दृष्ट्वा तु मुख्यावतिविध्यमानौवरेषुभिः शूरवरावरिघ्नौ}
{कर्णं च पार्थं च नियम्य वाहा-न्सर्वे वरिष्ठाश्च ततोऽवतस्थुः}


\twolineshloka
{प्रच्छादयामास ततः पृषत्कैःसघोषमाच्छिद्य च गाण्डिवज्याम्}
{अस्मिन्क्षणे फल्गुनं सूतपुत्रःसमाचिनोत्क्षुद्रकाणां शतेन}


\twolineshloka
{निर्मुक्तसर्पप्रतिमैश्च तीक्ष्णै--स्तैलप्रघौतैः खगपत्रवाजैः}
{षष्ट्या बिभेदाशु च वासुदेवंतदन्तरे प्रात्वरन्सोमकाश्च}


\twolineshloka
{ततो नवज्यां सुदृढां किरीटिस्वबाहुविक्षेपसहां प्रगृह्य}
{समादधे गाण्डिवे क्षिप्रकारीनिमेषमात्रेण महाधनुष्मान्}


\twolineshloka
{ज्याछेदनं ज्याविधानं च तस्यनैवावबुध्यत्सूतपुत्रो लघुत्वात्}
{पार्थस्य सङ्ख्ये द्विषतां निहन्तु--स्तदद्भुतं तत्र बभूव राजन्}


\twolineshloka
{पार्थोऽपि तां ज्यामवधाय तूर्णंशरासनज्यामाधिरथेर्विहत्य}
{सुसंरब्धः कर्णशरैः क्षताङ्गोरणे योधांस्तावकान्प्रत्यगृह्णात्}


% Check verse!
नैवापत्पक्षिगणोऽन्तरिक्षेपार्थेन चास्त्रेण कृतेऽन्धकारे
\twolineshloka
{शल्यं तु पार्थो दशभिर्निमेषा--द्भृशं तनुत्रे प्रहसन्नविध्यत्}
{ततः कर्णं द्वादशभिः पृषत्कै--र्विद्ध्वा पुनः सप्तभिरप्यविध्यत्}


\twolineshloka
{स पार्थबाणासनवेगनुन्नै--र्दृढाहतः पत्रिभिरुग्रवेगैः}
{विभिन्नगात्रः क्षतजोक्षिताङ्गःकर्णो बभौ रुद्र इवान्तकाले}


\twolineshloka
{ततस्त्रिभिस्तं त्रिदशाधिपोपमंशरैर्बिभेदाधिरथिर्धनञ्जयम्}
{शरांश्च पञ्च ज्वलनानिवोरगा--न्प्रवेशयामास जिघांसुरच्युते}


\twolineshloka
{सुवर्णचित्रं पुरुषोत्तमस्यवर्माथ भित्त्वाभ्यपतन्सुपुङ्खाः}
{वेगेन गां ते विविशुश्च राज--न्स्नात्वा कर्णाऽभिमुखाः प्रतीयुः}


\twolineshloka
{तान्पञ्चभल्लैर्दशभिः सुमुक्तै--स्त्रिधात्रिधैकैकमथोच्चकर्त}
{धनञ्जयस्ते न्यपतन्पृथिव्यांयथाऽहयस्तार्क्ष्यमुखेन कृत्ताः}


\threelineshloka
{ततः प्रजज्वाल किरीटमालीक्रोधेन कक्षं प्रदहन्निवाग्निः}
{स कर्णमाकर्णविकृष्टसृष्टैःशरैः शरीरान्तकरैर्ज्वलद्भिः}
{मर्मस्वविध्यत्स चचाल दुःखा--द्धैर्यात्तुं तस्थावतिमात्रधैर्यः}


\twolineshloka
{प्रादुश्चकाराथ शरान्महात्मादेहं विचिन्वन्निव सूतजस्य}
{शरास्तु ते काञ्चनचित्रपुङ्खाःसम्पेतुरुर्व्यां शतशो महान्तः}


\twolineshloka
{ततः शरौघैः प्रदिशो दिशश्चरविप्रभाः कर्णरथश्च राजन्}
{अदृश्य आसीत्कुपिते धनञ्जयेतुषारनीहारवृतो गिरिर्यथा}


\twolineshloka
{सचक्ररक्षा अपि पृष्ठगोपाःकर्णस्य ये चापि पुरःसराश्च}
{भीता द्रवन्ति स्म निहन्यमानामहेषुभिः पार्थकरप्रणुन्नैः}


\twolineshloka
{ततोऽर्जुनो वै भरतप्रवीरोमहानुभावः समरे निहन्ता}
{सुयोधनेनानुमतान्विनिघ्न--न्समुच्छ्रितान्सरथान्सारभूतान्}


\twolineshloka
{गाण्डीवधन्वा द्विगुणं सहस्रंकुरुप्रवीरानृषभः कुरूणाम्}
{क्षणेन सर्वान्सरथान्ससूता--न्निनाय राजन्क्षयमेकवीरः}


\twolineshloka
{अथो पलायन्त विहाय कर्णंतवात्मजा ये कुरवश्च शिष्टाः}
{एतानवाकीर्य शरक्षतांश्चविलप्यमानांस्तनयान्विमृद्गन्}


\twolineshloka
{सर्वे प्रणेशुः कुरवो विभग्नाःपार्थेषुभिः सम्परितप्यमानाः}
{सुयोधनेनाथ पुनर्वरिष्ठाःप्रचोदिताः कर्णरथानुयाने}


\twolineshloka
{भोः क्षत्रियाः शूरतमास्तु सर्वेक्षत्रे च धर्मे निरताः स्थ यूयम्}
{न युक्तरूपं भवतां समीपा--त्पलायनं कर्णमतिप्रहाय}


\twolineshloka
{तवात्मजेनापि तथोच्यमानानैवावतिष्ठन्त भयाद्विवर्णाः}
{क्षणेन नष्टाः प्रदिशो दिशश्चसर्वे ततः प्रेक्ष्य दिशो विशून्याः}


\twolineshloka
{भयावदीर्णः कुरुभिर्विहीनःपार्थेषुभिः सम्परितप्यमानः}
{न विव्यधे भारत तत्र कर्णः प्रतीपमेवार्जुनमभ्यधावत्}


\chapter{अध्यायः १०१}
\twolineshloka
{सञ्जय उवाच}
{}


\twolineshloka
{ततोऽपयाताः शरपातमात्र--मवस्थिता वै कुरवो नरेन्द्र}
{विद्युत्प्रकाशं ददृशुः समन्ता--द्वनञ्जयास्त्रं समुदीर्यमाणम्}


\twolineshloka
{ततोऽग्रसत्सूतपुत्रोऽर्जुनस्यवियद्गतं घोरतरं शरैस्तत्}
{क्रुद्धेन पार्थेन शरं विसृष्टंवधाय कर्णस्य महाविमर्दे}


\twolineshloka
{[उदीर्यमाणं स्म कुरून्दहन्तंसुवर्णपुङ्खैर्विशिखैर्ममर्द}
{कर्णस्त्वमोघेष्वसनं दृढज्यंविस्फारयित्वा विसृजञ्छरोघान् ॥]}


\twolineshloka
{रामादुपात्तेन महामहिम्नाह्याथर्वणेनारिविनाशनेन}
{तदर्जुनास्त्रं व्यधमद्दहन्तंपार्थं च बाणैर्निशितैरविध्यत्}


\twolineshloka
{ततो विमर्दः सुमहान्बभूवतत्रार्जुनस्याधिरथेश्च राजन्}
{अन्योन्यमासादयतोः पृषत्कै--र्विषाणघातैर्द्विपयोरिवाजौ}


\twolineshloka
{ततोऽस्त्रसङ्घातसमावृतं तदाबभूव राजंस्तुमुलं रणाजिरम्}
{यत्कर्णपार्थौ शरवृष्टिसङ्घै--र्निरन्तरं चक्रतुरम्बरं तदा}


\twolineshloka
{[ततो जालं बाणमयं महान्तंसर्वेऽद्राक्षुः कुरवः सोमकाश्च}
{नान्यं च भूतं ददृशुस्तदा तेबाणान्धकारे तुमुलेऽथ किञ्चित्}


\threelineshloka
{तौ सन्दधानावनिशं च राजन्}
{समस्यन्तौ चापि शराननेकान्}
{सन्दर्शयेतां युधि मार्गान्विचित्रा--न्धनुर्धरौ तौ विविधैः कृतास्त्रैः}


\twolineshloka
{तयोरेवं युध्यतोराजिमध्येसूतात्मजोऽभूदधिकः कदाचित्}
{पार्थः कदाचित्त्वधिकः किरीटिवीर्यास्त्रमायाबलपौरुषेण}


\twolineshloka
{दृष्ट्वा तयोस्तं युधि सम्प्रहारंपरस्परस्यान्तरमीक्षमाणयोः}
{घोरं तयोर्दुर्विषहं रणेऽन्यै--र्योधाः सर्वे विस्मयमभ्यगच्छन्}


\twolineshloka
{ततो भूतान्यन्तरिक्षस्थितानितौ कर्णपार्थौ प्रशशंसुर्नरेन्द्र}
{भोः कर्ण साध्वर्जुन साधु चेतिवियत्सु वाणी श्रूयते सर्वतोपि}


\twolineshloka
{तस्मिन्विमर्दे रथवाजिनागै--स्तदाऽभिघातैर्दलिते हि भूतले}
{ततस्तु पातालतले शयानोनागोऽश्वसेनः कृतवैरोऽर्जुनेन}


\twolineshloka
{राजंस्तदा खाण्डवदाहमुक्तोविवेश कोपाद्वसुधातले यः}
{अथोत्पपातोर्ध्वगतिर्जवेनसन्दृश्य कर्णार्जुनयोर्विमर्दम्}


\twolineshloka
{अयं हि कालोस्य दुरात्मनो वैपार्थस्य वैरप्रतियातनाय}
{सञ्चिन्त्य तूर्णं प्रविवेश चैवकर्णस्य राजन्शररूपधारी}


\twolineshloka
{ततोस्त्रसङ्घातसमाकुलं तदाबभूव जन्यं विततांशुजालम्}
{तत्कर्णपार्थो शरसङ्घवृष्टिभि--र्निरन्तरं चक्रतुरम्बरं तदा}


\twolineshloka
{तद्बाणजालैकमयं महान्तंसर्वेऽत्रसन्कुरवः सोमकाश्च}
{नान्यत्किञ्चिद्ददृशुः सम्पतद्वैबाणान्धकारे तुमुलेऽतिमात्रम्}


\twolineshloka
{ततस्तौ पुरुषव्याघ्रौ सर्वलोकधनुर्धरौ}
{त्यक्तप्राणौ रणे वीरो युद्धश्रममुपागतौ}


\threelineshloka
{समुत्क्षेपैर्वीक्षमाणौ सिक्तौ चन्दनवारिणा}
{सवालव्यजनैर्दिव्यैर्दिविस्थैरप्सरोगणैः}
{शक्रसूर्यकराब्जाभ्यां प्रमार्जितमुखावुभौ}


\twolineshloka
{कर्णोऽथ पार्थं न विशेषयद्यदाभृशं च पार्थेन शराभितप्तः}
{ततस्तु वीरः शरविक्षताङ्गोदध्रे मनो ह्येकशयस्य तस्य ॥]}


\twolineshloka
{ततो रिपुघ्नं समधत्त कर्णःसुसञ्चितं सर्पमुखं ज्वलन्तम्}
{रौद्रं शरं सन्नतमुग्रधौतंपार्थार्थमित्येव चिराभिगुप्तम्}


\twolineshloka
{सदार्चितं चन्दनचूर्णशायितंसुवर्णतूणीरशयं महार्चिषम्}
{आकर्णपूर्णं विचकर्ष कर्णोविमोक्तुकामः शरमुग्रवेगम्}


\twolineshloka
{प्रदीप्तमैरावतवंशसम्भवंशिरो जिहीर्षुर्युधि सव्यसाचिनः}
{ततः प्रजज्वाल दिशो नभश्चउल्काश्च घोराः शतशः प्रपेतुः}


% Check verse!
तस्मिंस्तु नागे धनुषि प्रयुक्तेहाहाकृता लोकपालाः सशक्राः
\twolineshloka
{स सूतपुत्रस्तमपाङ्गदेशेअवाङ्मुखं सन्धयति स्म रोषात्}
{न तं स्म जानाति महानुभाव--मपाङ्गदेशेऽभिनिविष्टमाजौ}


\threelineshloka
{[न चापि तं बुबुधे सूतपुत्रोबाणे प्रविष्टं योगबलेन नागम्}
{दशशतनयनोऽहिं दृश्य बाणे प्रविष्टंनिहत इति सुतो मे स्रस्तगात्रो बभूव}
{जलजकुसुमयोनिः श्रेष्ठभावो जितात्मात्रिदशपतिमवोचन्मा व्यथिष्ठा जये श्रीः}


\twolineshloka
{ततोऽब्रवीन्मद्रराजो महात्मादृष्ट्वा कर्णं प्रहितेषु तमुग्रम्}
{ग्रीवायतः कर्ण न संहितोऽयंसमीक्ष्य सन्धत्स्व शरं परघ्नम्}


\twolineshloka
{अथाब्रवीत्क्रोधसंरक्तनेत्रोमद्राधिपं सूतपुत्रो मनस्वी}
{न सन्धत्ते द्विः शरं शल्य कर्णोन मादृशा जिह्मयुद्वा भवन्ति}


\twolineshloka
{इतीदमुक्त्वा विससर्ज तं शरंप्रयत्नतो वर्षगणाभिपूजितम्}
{हतोसि रे फल्गुन इत्यधिक्षिप--न्नुवाच चोच्चैर्निरमूर्जितां वृषः}


\twolineshloka
{स सायकः कर्णभुजप्रसृष्टोहुताशनार्कप्रतिमः सुघोरः}
{गुणच्युतः कर्णधनुःप्रमुक्तोवियद्गतः प्राज्वलदन्तरिक्षे}


\twolineshloka
{तमापतन्तं ज्वलितं निरीक्ष्यवियद्गतं वृष्णिकुलप्रवीरः}
{रथस्य चक्रं सहसा निपीड्यपञ्चाङ्गुलं मज्जयति स्म वीरः}


\threelineshloka
{ततोऽन्तरिक्षे सुमहान्निनादःसम्पूजनार्थं मधुसूदनस्य}
{[दिव्याश्च वाचः सहसा बभूवु--र्दिव्यानि पुष्पाण्यथ सिंहनादाः}
{तस्मिंस्तथा वै धरणीं निमग्नेरथे प्रयत्नान्मधुसूदनस्य ॥]}


\twolineshloka
{ततः किरीटं बहुरत्नचित्रंजघान नागोऽर्जुनमूर्धतो बलात्}
{गिरेः सुजाताङ्कुरपुष्पितद्रुमंमहेन्द्रवज्रं शिखरं यथोत्तमम्}


\twolineshloka
{ततः किरीटं तपनीयचित्रंपार्थोत्तमाङ्गादहरत्तरस्वी}
{तद्धेमजालावनतं सुघोरंसम्प्रज्वलत्तन्निपपात भूमौ}


\twolineshloka
{ततोऽर्जुनस्योत्तमगात्रभूषणंसुवर्णमुक्तामणिवज्रचित्रितम्}
{धरावियद्द्योसलिलेषु विश्रुतंबलं निसर्गोत्तममन्युधिः सदा}


\twolineshloka
{शरेण मूर्ध्नि प्रजहार सूतजोदिवाकरेन्दुज्वलनग्रहत्विषम्}
{सुवर्णमुक्तामणिवज्रभूषणंपुरन्दरार्थं तपसा प्रयत्नतः}


\twolineshloka
{स्वयं कृतं यद्विधिना स्वयम्भुवामहार्हरूपं द्विषतां भयङ्करम्}
{निजघ्नते देवरिपून्सुरेश्वरःस्वयं ददौ यद्धि मुदाऽर्जुनाय}


\twolineshloka
{हरिप्रचेतोहरिवित्तगोप्तृभिःपिनाकपाशाशनिदण्डधारिभिः}
{सुरोत्तमैरप्यविषह्यमर्दितुंप्रसह्य नागेन जहार तद्वृषा}


\twolineshloka
{तदुत्तमेषून्मथितं विषाहिनाप्रदीप्तमर्चिष्मदतीव सुन्दरम्}
{पपात पार्थस्य किरीटमुत्तमंदिवाकरोस्तादिव पर्वताज्ज्वलन्}


\twolineshloka
{महीवियद्द्योसलिलानि वायुनाप्रसह्य रुग्णानि विघूर्णितानि वा}
{इतीव शब्दं भुवनेषु तत्तथाजना व्यवस्यन्ति दिशश्च विह्वलाः}


% Check verse!
विना किरीटं* शुशुभे स पार्थःश्यामो युवा शैल त्रिशृङ्गैः
\twolineshloka
{ततः समुद्ग्रथ्य सितेन वाससास्वमूर्धजनाव्यथितस्तदाऽर्जुनः}
{बभौ सुसम्पूर्णमरीचिनेन्दुनाशिरोगतेनोदयपर्वतो यथा}


\twolineshloka
{स चापि राधेयभुजप्रमुक्तोहुताशनार्कप्रतिमद्युतिर्महान्}
{महोरगः कृतवैरोऽर्जुनेनकिरीटमाहृत्य समुत्पपात}


\twolineshloka
{तमुत्पतन्तं द्विपदां वरिष्ठोदृष्ट्वा वचः पार्थमुवाच कृष्णः}
{महोरगं पाण्डव पश्यपश्यप्रयोजितं त्वन्निधनार्थमुग्रम्}


\threelineshloka
{स एवमुक्तो मधुसूदनेनगाण्डीवधन्वा हरिमुग्रवाचा}
{उवाच को न्वेष ममाद्य नागःक्षयाय नागाद्गरुडस्य वक्रम् ॥कृष्ण उवाच}
{}


\twolineshloka
{योऽसौ त्वया खाण्डवे चित्रभानुंसन्तर्पयाणेन धनुर्धरेण}
{वियद्गतो जननीगुप्तदेहोमत्वैकरूपं निहताऽस्य माता}


\threelineshloka
{स एष तद्वैरमनुस्मरन्वैत्वां प्रार्थयत्यात्मवधाय नूनम्}
{नभश्चुतां प्रज्वलितामिवोल्कांपश्यैनमायान्तममित्रसाह ॥सञ्जय उवाच}
{}


\twolineshloka
{ततः स जिष्णुः परिवृत्य रोषा--च्चिच्छेद षड्भिर्निशितैः सुधारैः}
{नागं वियत्तिर्यगिवोत्पततंस च्छिन्नगात्रो निपपात भूमौ}


\twolineshloka
{[गते च तस्मिन्भुजगे किरीटिनास्वयं विभुः पार्थिव भूतलादथ}
{समुज्जहाराशु पुनः पतन्तंरथं भुजाभ्यां पुरुषोत्तमस्ततः ॥]}


\twolineshloka
{तस्मिन्मुहूर्ते दशभिः पृषत्कैःसिलाशितैर्बर्हिणबर्हवाजितैः}
{विव्याध कर्णः पुरुषप्रवीरोधनञ्जयं तिर्यगवेक्षमाणः}


\twolineshloka
{ततोऽर्जुनो द्वादशभिः सुमुक्तै--र्वराहकर्णैर्जिशितैः समर्प्य}
{नाराचमाशीविषतुल्यवेग--माकर्णपूर्णायतमुत्ससर्ज}


\twolineshloka
{स चित्रवर्मेषुवरो विदार्यप्राणान्निरस्यन्निव साधुमुक्तः}
{कर्णस्य पीत्वा रुधिरं विवेशवसुन्धरां शोणितदिग्धवाजः}


\twolineshloka
{ततो वृषो बाणनिपातकोपितोमहोरगो दण्डविघट्टितो यथा}
{तदाऽऽशुकारी व्यसृजच्छरोत्तमान्महाविषः सर्प इवोत्तमं विषम्}


\twolineshloka
{जनार्दनं द्वादशभिः पराभिन--न्नवैर्नवत्या च शरैस्तथाऽर्जुनम्}
{शरेण घोरेण पुनश्च पाण्डवंविदार्य कर्णो व्यनदज्जहास च}


\twolineshloka
{तमस्य हर्षं ममृषे न पाण्डवोबिभेद मर्माणि ततोऽस्य मर्मवित्}
{परःशतैः पत्रिभिरिन्द्रविक्रम--स्तथा यथेन्द्रो बलमोजसा रणे}


\twolineshloka
{ततः शराणां नवतिं तदाऽर्जुनःससर्ज कर्णेऽन्तकदण्डसन्निभाम्}
{तैः पत्रिभिर्विद्धतनुः स विव्यथेतथा यथा वज्रविदारितोऽचलः}


\twolineshloka
{मणिप्रवेकोत्तमवज्रहाटकै--रलङ्कृतं चास्य वराङ्गभूषणम्}
{प्रविद्धमुर्व्यां निपपात पत्रिभि--र्धनञ्जयेनोत्तमकुण्डलेपि च}


\twolineshloka
{महाधनं शिल्पिवरैः प्रयत्नतःकृतं यदस्योत्तमवर्म भास्वरम्}
{सुदीर्घकालेन ततोऽस्य पाण्डवःक्षणेन बाणैर्बहुधा व्यशातयत्}


\twolineshloka
{`तस्येषुभिः खण्डितकुण्डलोऽन्तःपरिक्षतश्चाभ्यधिकं तदानीम्}
{स लोहिताङ्गश्रवणश्चकाशेसलोहिताङ्गश्रवणो यथा दिवि'}


\twolineshloka
{स तं विवर्माणमथोत्तमेषुभिःशितैश्चतुर्भिः कुपितः पराभिनत्}
{स विव्यथेऽत्यर्थमरिप्रताडितोयथाऽऽतुरः पित्तकफानिलज्वरैः}


\twolineshloka
{महाधनुर्मण्डलनिःसृतैः शितैःक्रियाप्रयत्नप्रहितैर्बलेन च}
{ततक्ष कर्णं बहुभिः शरोत्तमै--र्बिभेद मर्मस्वपि चार्जुनस्त्वरन्}


\twolineshloka
{दृढाहतः पत्रिभिरुग्रवेगैःपार्थेन कर्णो विविधैः शिताग्रैः}
{बभौ गिरिर्गैरिकधातुरक्तःक्षरन्प्रपातैरिव रक्तमम्भः}


\twolineshloka
{[ततोऽर्जुनः कर्णमवक्रगैर्नवैःसुवर्णपुङ्खैः सुदृढैरयस्मयैः}
{यमाग्निदण्डप्रतिमैः स्तनान्तरेपराभिनत्क्रौञ्चमिवाद्रिमग्निजः}


\threelineshloka
{ततः शरावापमपास्य सूतजोधनुश्च तच्छक्रशरासनोपमम्}
{ततो रथस्थः स मुमोह च स्खलन्}
{प्रशीर्णमुष्टिः सुभृशाहतः प्रभो}


\twolineshloka
{न चार्जुनस्तं व्यसने तदेषिवा--न्निहन्तुमार्यः पुरुषव्रते स्थितः}
{ततस्तमिन्द्रावरजः सुसम्भ्रमा--दुवाच किं पाण़्डव हे प्रमाद्यसे}


\twolineshloka
{नैवाहितानां सततं विपश्चितःक्षणं प्रतीक्षन्त्यपि दुर्बलीयसाम्}
{विशेषतोऽरीन्व्यसनेषु पण्डितोनिहत्य धर्मं च यशश्च विन्दते}


\twolineshloka
{तदेकवीरं तव चाहितं सदात्वरस्व कर्णं सहसाऽभिमर्दितुम्}
{पुरा समर्थः समुपैति सूतजोभिन्धि त्वमेनं नमुचिं यथा हरिः}


\twolineshloka
{ततस्तदेवेत्यभिपूज्य सत्वरंजनार्दनं कर्णमविध्यदर्जुनः}
{शरोत्तमैः सर्वकुरूत्तमस्त्वरं--स्तथा यथा शम्बरहा पुरा बलिम् ॥]}


\twolineshloka
{साश्वं तु कर्णं सरथं किरीटिसमाचिनोद्भारत वत्सदन्तैः}
{प्रच्छादयामास दिशश्च बाणैःसर्वप्रयत्नात्तपनीयपुङ्खैः}


\twolineshloka
{सवत्सदन्तैः पृथुपीनवक्षाःसमाचितः सोऽधिरथिर्विभाति}
{सुपुष्पिताशोकपलाशशाल्मलि--र्यथाऽचलश्चन्दनकाननायुतः}


\twolineshloka
{शरैः शरीरे बहुभिः समर्पितै--र्विभाति कर्णः समरे विशाम्पते}
{महीरुहैराचितसानुकन्दरोयथा गिरीन्द्रः स्फुटकर्णिकारवान्}


\twolineshloka
{स वाणसङ्घान्बहुधा व्यवासृजन्विभाति कर्णः शरजालरश्मिवान्}
{स लोहितो रक्तगभस्तिमण्डलोदिवाकरोऽस्ताभिमुखो यथा तथा}


\twolineshloka
{बाह्वन्तरादाधिरथेर्विमुक्तान्बाणान्महाहीनिव दीप्यमानान्}
{व्यध्वंसयन्नर्जुनबाहुमुक्ताःशराः समासाद्य दिशः शिताग्राः}


\twolineshloka
{[ततः स कर्णः समवाप्य धैर्यंबाणान्विमुञ्चन्कुपिताहिकल्पान्}
{विव्याध पार्थं दशभिः पृषत्कैःकृष्णं च ष़ड्भिः कुपिताहिकल्पैः}


\twolineshloka
{ततः किरीटि भृशमुग्रनिःस्वनंमहाशरं सर्पविषानलोपमम्}
{अयस्मायं रौद्रमहास्त्रसम्भृतंमहाहवे क्षेप्तुमना महामतिः ॥]}


\twolineshloka
{कालो ह्यदृश्यो नृप विप्रशापा--न्निदर्शयन्कर्णवधं ब्रुवाणः}
{भूमिस्तु चक्रं ग्रसतीत्यवोच--त्कर्णस्य तस्मिन्वधकाल आगते}


\twolineshloka
{न चास्य घोरं प्रतिभाति चास्त्रंयद्भार्गवोऽस्मै प्रददौ महात्मा}
{चक्रं च वामं ग्रसते भूमिरस्यप्राप्ते तस्मिन्वधकाले नृवीर}


\twolineshloka
{ततो रथो घूर्णितवान्नरेन्द्रशापात्तदा ब्राह्मणसत्तमस्य}
{प्राप्तं वधं शंसति चाप्यथास्त्रंप्रणश्यमानं द्विजमुख्यशापात्}


\twolineshloka
{[ततश्चक्रमपतत्तस्य भूमौस विह्वलः समरे सूतपुत्रः}
{सवेदिकश्चैत्य इवातिमात्रःसुपुष्पितो भूमितले निमग्नः]}


\twolineshloka
{मग्ने रथे ब्राह्मणशापमूढोह्यस्त्रं च तं मोघमिषु च सर्पम्}
{अमृष्यमाणो व्यसनानि तानिहस्तौ विधुन्वन्विजगर्ह धर्मम्}


\threelineshloka
{धर्मप्रधानं किल पाति धर्मइत्यब्रुवन्धर्मविदः सदैव}
{वयं च धर्मे प्रयताम नित्यंचर्तुं यथाशक्ति यथाश्रुतं च}
{स चापि निघ्नाति न पाति भक्तान्मन्ये न नित्यं परिपाति धर्मः}


\twolineshloka
{एवं ब्रुवन्प्रस्खलिताश्वसूतोविचाल्यमानोऽर्जुनबाणपातैः}
{मर्माभिघाताच्छिथिलः क्रियासुपुनः पुनर्धर्ममसौ जगर्ह}


\twolineshloka
{ततः शरैर्भीमतरैरविध्यत्त्रिभिराहवे}
{हस्ते कृष्णं तथा पार्थमभ्यविध्यच्च सप्तभिः}


\twolineshloka
{ततोऽर्जुनः सप्तदश तिग्मवेगानजिह्मगान्}
{इन्द्राशनिसमान्घोरानसृजत्पावकोपमान्}


\twolineshloka
{निर्भिद्य ते भीमवेगा ह्यपतन्पृथिवीतले}
{कम्पितात्मा ततः कर्णः शक्त्या चेष्टामदर्शयत्}


\twolineshloka
{बलेनाथ स संस्तभ्य ब्रह्मास्त्रं समुदैरयत्}
{ऐन्द्रं ततोऽर्जुनश्चापि तं दृष्ट्वाऽभ्युपमन्त्रयत्}


\twolineshloka
{गाण्डीवं ज्यां च बाणांश्च सोऽनुमन्त्र्य परन्तपः}
{व्यसृजच्छरवर्षाणि वर्षाणीव पुरन्दरः}


\threelineshloka
{ततस्तेजोमया बाणा रथात्पार्थस्य निःसृताः}
{प्रादुरासन्महावीर्याः कर्णस्य रथमन्तिकात्}
{}


\twolineshloka
{स कर्णोऽग्रसदस्यास्त्रं कुर्वन्मोक्षं महारथः ॥ततोऽब्रवीद्वृष्णिवीरस्तस्मिन्नस्त्रे विनाशिते}
{विसृजास्त्रं परं पार्थ राधेयो ग्रसते शरान्}


\twolineshloka
{ततो ब्रह्मास्त्रमत्युग्रं सम्मन्त्र्य समयोजयत्}
{छादयित्वा ततो बाणैः कर्णं प्रत्यस्यदर्जुनः}


\twolineshloka
{ततः कर्णः शितैर्बाणैर्ज्यां चिच्छेद सुतेजनैः}
{[द्वितीयां च तृतीयां च चतुर्थी पञ्चमीं तथा}


\threelineshloka
{षष्ठीमथास्य चिच्छेद सप्तमीं च तथाऽष्टमीम्}
{नवमीं दशमीं चास्य तथा चैकादशीं वृषः}
{ज्याशतं शतसन्धानः स कर्णो नावबुध्यते ॥]}


\twolineshloka
{ततो ज्यां विनिधायान्यामभिमन्त्र्य च पाण्डवः}
{शरैरवाकिरत्कर्णं दीप्यमानैरिवाहिभिः}


\twolineshloka
{तस्य ज्याच्छेदनं कर्णो ज्यावधानं च संयुगे}
{नान्वबुध्यत शीघ्रत्वात्तदद्भुतमिवाभवत्}


\twolineshloka
{अस्त्रैरस्त्राणि संवार्य प्रनिघ्नन्सव्यसाचिनः}
{चक्रे चाप्यधिकं पार्थात्स्ववीर्यमतिदर्शयन्}


\twolineshloka
{ततः कर्णोऽर्जुनं दृष्ट्वा स्वस्यास्त्रेण च पीडितम्}
{अभ्यसेत्यब्रवीत्पार्थमातिष्ठास्त्रं व्रजेति च}


\threelineshloka
{ततोऽग्निसदृशं घोरं शरं सर्पविषोपमम्}
{अश्मसारमं दिव्यमभिमन्त्र्यं परन्तपः}
{रौद्रमस्त्रं समाधाय क्षेप्तुकामः किरीटिने}


% Check verse!
ततोऽग्रसन्मही चक्रं राधेयस्य तदा नृप
\twolineshloka
{[ततोऽवतीर्य राधेयो रथादाशु समुद्यतः}
{चक्रं भुजाभ्यामालम्ब्य समुत्क्षेप्तुमियेष सः}


\twolineshloka
{सप्तद्वीपा वसुमती सशैलवनकानना}
{गीर्णचक्रा समुत्क्षिप्ता कर्णेन चतुरङ्गुलम्}


\twolineshloka
{ग्रस्तचक्रस्तु राधेयः क्रोधादश्रूण्यवर्तयत्}
{अर्जुनं वीक्ष्य संरब्धमिदं वचनमब्रवीत्}


\twolineshloka
{भोभो पार्थ महेष्वास मुहुर्तं परिपालय}
{यावच्चक्रमिदं ग्रस्तमुद्धरामि महीतलात्}


\twolineshloka
{सव्यं चक्रं महीग्रस्तं दृष्ट्वा दैवादिदं मम}
{पार्थ कापुरुषाचीर्णमभिसन्धिं विसर्जय}


\threelineshloka
{न त्वं कापुरुषाचीर्णं मार्गमास्थातुमर्हसि}
{ख्यातस्त्वमसि कौन्तेय विशिष्टो रणकर्मसु}
{विशिष्टतरमेव त्वं कर्तुमर्हसि पाण्डव}


\twolineshloka
{प्रकीर्णकेशे विमुखे ब्राह्मणेऽथ कृताञ्जलौ}
{शरणागते न्यस्तशस्त्रेयाचमाने तथाऽर्जुन}


\twolineshloka
{अबाणे भ्रष्टकवचे भ्रष्टभग्नायुधे तथा}
{न विमुञ्चन्ति शस्त्राणि शूराः साधुव्रते स्थिताः}


\twolineshloka
{त्वं च शूरतमो लोके साधुवृत्तश्च पाण्डव}
{अभिज्ञो युद्धधर्माणां तस्मात्क्षम मुहूर्तकम्}


\twolineshloka
{दिव्यास्त्रविदमेयात्मा कार्तवीर्यसमो युधि}
{यावच्चक्रमिदं ग्रस्तमुद्वरामि महाभुज}


\twolineshloka
{न मां रथस्थो भूमिष्ठं विकलं हन्तुमर्हसि}
{न वासुदेवात्त्वत्तो वा पाण्डवेय बिभेम्यहम्}


\twolineshloka
{त्वं हि क्षत्रियदायादो महालकुलविवर्धनः}
{अतस्त्वां प्रबवीम्येष मुहूर्तं क्षम पाण़्डव}


\fourlineindentedshloka
{[विना किरीटं शंशुभे स पार्थः}
{श्यामो युवा नील इवोच्चशृङ्गः}
{ततः समुद्ग्रथ्य सितेन वाससास्वमूर्धजानव्यथितस्तदाऽर्जुनः}
{विभासितः सूर्यमरीचिना दृढंशिरोगतेनोदयपर्वतो यथा}


% Check verse!
गोकर्णा सुमुखी कृतेन इषुणा गोपुत्रसम्प्रेषितागोशब्दात्मजभूषणं सुविहितं सुव्यक्तगोऽसुप्रभंदृष्ट्वा गोगतकं जहार मुकुटं गोशब्दगोपूरि वैगोकर्णासनमर्दनश्च न ययावप्राप्य मृत्योर्वशं
\twolineshloka
{स सायकः कर्णभुजप्रसृष्टोहुताशनार्कप्रतिमो महार्हः}
{महोरगः कृतवैरोऽर्जुनेनकिरीटमाहत्य ततो व्यतीयात्}


\twolineshloka
{तं चापि दग्ध्वा तपनीयचित्रंकिरीटमाकृष्य तदर्जुनस्य}
{इयेष गन्तुं पुनरेव तूणंदृष्टश्च कर्णेन ततोऽब्रवीत्तम्}


\twolineshloka
{मुक्तस्त्वयाऽहं त्वसमीक्ष्य कर्णशिरो हृतं यन्न मयाऽर्जुनस्य}
{समीक्ष्य मां मुञ्च रणे त्वमाशुहन्ताऽस्मि शत्रुं तव चात्मनश्च}


\twolineshloka
{स एवमुक्तो युधि सूतपुत्र--स्तम्ब्रवीत्को भवानुग्ररूपः}
{नागोऽब्रवीद्विद्वि कृतागसं मांपार्थेन मातुर्वधजातवैरम्}


\threelineshloka
{यदि स्वयं वज्रधरोऽस्य गोप्तातथापि याता पितृराजवेश्मनि}
{कर्ण उवाच}
{न नाग कर्णोऽद्य रणे परस्यबलं समास्थाय जयं बुभूषेत्}


\twolineshloka
{न संदध्यां द्विः शरं चैव नागयद्यर्जुनानां शतमेव हन्याम्}
{तमाह कर्णः पुनरेव नागंतदाऽजिमध्ये रविसूनुसत्तमः}


\twolineshloka
{व्यालास्त्रसर्गोत्तमयत्नमन्युभि--र्हन्ताऽस्मि पार्थं सुसुखी व्रज त्वम्}
{इत्येवमुक्तो युधि नागराजःकर्णेन रोषादसहंस्तस्य वाक्यम्}


\twolineshloka
{स्वयं प्रायात्पार्थवधाय राजन्कृत्वा स्वरूपं विजिघांसुरुग्रः}
{ततः कृष्णः पार्थमुवाच सङ्ख्येमहोरगं कृतवैरं जहि त्वम्}


\twolineshloka
{स एवमुक्तो मधुसूदनेनगाण्डीवधन्वा रिपुवीर्यसाहः}
{उवाच को ह्येष ममाद्य नागःस्वयं य आयाद्गरुडस्य वक्त्रम् ॥]}


\chapter{अध्यायः १०२}
\fourlineindentedshloka
{[विना किरीटं शंशुभे स पार्थः}
{श्यामो युवा नील इवोच्चशृङ्गः}
{ततः समुद्ग्रथ्य सितेन वाससास्वमूर्धजानव्यथितस्तदाऽर्जुनः}
{विभासितः सूर्यमरीचिना दृढंशिरोगतेनोदयपर्वतो यथा}


% Check verse!
गोकर्णा सुमुखी कृतेन इषुणा गोपुत्रसम्प्रेषितागोशब्दात्मजभूषणं सुविहितं सुव्यक्तगोऽसुप्रभंदृष्ट्वा गोगतकं जहार मुकुटं गोशब्दगोपूरि वैगोकर्णासनमर्दनश्च न ययावप्राप्य मृत्योर्वशं
\twolineshloka
{स सायकः कर्णभुजप्रसृष्टोहुताशनार्कप्रतिमो महार्हः}
{महोरगः कृतवैरोऽर्जुनेनकिरीटमाहत्य ततो व्यतीयात्}


\twolineshloka
{तं चापि दग्ध्वा तपनीयचित्रंकिरीटमाकृष्य तदर्जुनस्य}
{इयेष गन्तुं पुनरेव तूणंदृष्टश्च कर्णेन ततोऽब्रवीत्तम्}


\twolineshloka
{मुक्तस्त्वयाऽहं त्वसमीक्ष्य कर्णशिरो हृतं यन्न मयाऽर्जुनस्य}
{समीक्ष्य मां मुञ्च रणे त्वमाशुहन्ताऽस्मि शत्रुं तव चात्मनश्च}


\twolineshloka
{स एवमुक्तो युधि सूतपुत्र--स्तम्ब्रवीत्को भवानुग्ररूपः}
{नागोऽब्रवीद्विद्वि कृतागसं मांपार्थेन मातुर्वधजातवैरम्}


\threelineshloka
{यदि स्वयं वज्रधरोऽस्य गोप्तातथापि याता पितृराजवेश्मनि}
{कर्ण उवाच}
{न नाग कर्णोऽद्य रणे परस्यबलं समास्थाय जयं बुभूषेत्}


\twolineshloka
{न संदध्यां द्विः शरं चैव नागयद्यर्जुनानां शतमेव हन्याम्}
{तमाह कर्णः पुनरेव नागंतदाऽजिमध्ये रविसूनुसत्तमः}


\twolineshloka
{व्यालास्त्रसर्गोत्तमयत्नमन्युभि--र्हन्ताऽस्मि पार्थं सुसुखी व्रज त्वम्}
{इत्येवमुक्तो युधि नागराजःकर्णेन रोषादसहंस्तस्य वाक्यम्}


\twolineshloka
{स्वयं प्रायात्पार्थवधाय राजन्कृत्वा स्वरूपं विजिघांसुरुग्रः}
{ततः कृष्णः पार्थमुवाच सङ्ख्येमहोरगं कृतवैरं जहि त्वम्}


\twolineshloka
{स एवमुक्तो मधुसूदनेनगाण्डीवधन्वा रिपुवीर्यसाहः}
{उवाच को ह्येष ममाद्य नागःस्वयं य आयाद्गरुडस्य वक्त्रम् ॥]}


\chapter{अध्यायः १०३}
\twolineshloka
{सञ्जय उवाच}
{}


\twolineshloka
{अथाब्रवीद्वासुदेवो महात्माराधेय दिष्ट्या स्मरसीह धर्मम्}
{`धर्मे हि बद्धा सततं हि पार्था--स्तेभ्यस्ततो वृद्धिसौ ददाति}


\twolineshloka
{धर्मादपेताः परिपन्थिनस्तेतस्माद्गता वै कुरवो विनाशम्'}
{प्रायेण नीचा व्यसनेषु मग्नानिन्दन्ति दैवं न तु दुष्कृतं स्वम्}


\threelineshloka
{कृष्णां सभां कर्ण यदेकवस्त्रा--मानीतवांस्त्वं च सुयोधनश्च}
{दुःशासनः शकुनिः सौबलश्च}
{धर्मस्तदा ते रुचितो न कस्मात्}


\twolineshloka
{यदा सभायां राजानमनक्षज्ञं युधिष्ठिरम्}
{आनीय जितवन्तो वै क्व ते धर्मस्तदा गतः}


\twolineshloka
{वनवासे व्यतीते च कर्ण वर्षे त्रयोदशे}
{न प्रयच्छसि यद्राज्यं क्व ते धर्मस्तदा गतः}


\twolineshloka
{यद्भीमसेनं सर्पैश्च विषयुक्तैश्च भोजनैः}
{आचरत्त्वन्मते राजा क्व ते धर्मस्तदा गतः}


\twolineshloka
{यद्वारणावते पार्थान्सुप्ताञ्जतुगृहे तदा}
{हन्तुकामास्तदा यूयं क्व ते धर्मस्तदा गतः}


\twolineshloka
{यदा रजस्वलां कृष्णां दुःशासनवशे स्थिताम्}
{सभायां प्राहसः कर्ण क्व ते धर्मस्तदा गतः}


\twolineshloka
{यदनार्यैः पुरा कृष्णां क्लिश्यमानामनागसम्}
{उपप्रेक्षसि राधेय क्व ते धर्मस्तदा गतः}


\threelineshloka
{विनष्टाः पाण्डवाः कृष्णे शाश्वतं नरकं गताः}
{पतिमन्यं वृणीष्वेति वदंस्त्वं गजगामिनीम्}
{उपप्रेक्षसि राधेय क्व ते धर्मस्तदा गतः}


\twolineshloka
{राज्यलुब्धः पुनः कर्ण समाह्वयसि पाण्डवान्}
{यदा शकुनिमाश्रित्य क्व ते धर्मस्तदा गतः}


\twolineshloka
{यदाऽभिमन्युं बहवो युद्धे जघ्नुर्महारथाः}
{परिवार्य रणे बालं क्व ते धर्मस्तदा गतः}


\twolineshloka
{[यद्येष धर्मस्तत्र न विद्यते हिकिं सर्वथा तालुविशोषणेन}
{अद्येह धर्म्याणि विधत्स्व सूततथापि जीवन्न विमोक्ष्यसे हि}


\twolineshloka
{नलो ह्यक्षैर्निर्जितः पुष्करेणपुनर्यशो राज्यमवाप वीर्यात्}
{प्राप्तास्तथा पाण्डवा बाहुवीर्या--त्सर्वैः समेताः परिवृत्तलोभाः}


\threelineshloka
{निहत्य शत्रून्समरे प्रवृद्धा--न्ससोमका राज्यमवाप्नुयुस्ते}
{तथा गता धार्तराष्ट्रा विनाशंधर्माभिगुप्तैः सततं नृसिंहैः] ॥सञ्जय उवाच}
{}


\twolineshloka
{एवमुक्तस्तदा कर्णो वासुदेवेन भारत}
{लज्जयावनतो भूत्वा नोत्तरं किञ्चिदुक्तवान्}


\twolineshloka
{क्रोधात्प्रस्फुरमाणौष्ठो धनुरुद्यम्य भारत}
{योधयामास वै पार्थं महावेगपराक्रमः}


\twolineshloka
{ततोऽब्रवीद्वासुदेवः फल्गुनं पुरुषर्षभम्}
{दिव्यास्त्रेणैव निर्भिद्य पातयस्व महाबल}


\twolineshloka
{एवमुक्तस्तु देवेन क्रोधमागात्तदाऽर्जुनः}
{मन्युमभ्याविशद्धोरं स्मृत्वा तत्तु धनञ्जयः}


\twolineshloka
{तस्य क्रुद्धस्य सर्वेभ्यः स्रोतोभ्यस्तेजसोऽर्चिषः}
{प्रादुरासंस्तदा राजंस्तदद्भुतमिवाभवत्}


\twolineshloka
{तत्समीक्ष्य ततः कर्णो ब्रह्मास्त्रेण धनञ्जयम्}
{अभ्यवर्षत्पुनर्यत्नमकरोद्रथसर्जने}


\twolineshloka
{ब्रह्मास्त्रेणव तं पार्थो कवर्ष शरवृष्टिभिः}
{तदस्त्रमस्त्रेणावार्य प्रजहार च पाण्डवः}


\twolineshloka
{ततोऽन्यदस्त्रं कौन्तेयो दयितं जातवेदसः}
{मुमोच कर्णमुद्दिश्य तत्प्रजज्वाल तेजसा}


\twolineshloka
{वारुणेन ततः कर्णः शमयामास पावक्रम्}
{जीमूतैश्च दिशः सर्वाश्चक्रे तिमिरदुर्दिनाः}


\twolineshloka
{पाण्डवेयस्त्वसम्भ्रान्तो वायव्यास्त्रेण वीर्यवान्}
{अपोवाह तदाऽभ्राणि राधेयस्य प्रपश्यतः}


\twolineshloka
{[ततः शरं महाघोरं ज्वलन्तिमिव पावकम्}
{आददे पाण्डुपुत्रस्य सूतपुत्रो जिघांसया}


\twolineshloka
{योज्यमाने ततस्तस्मिन्बाणे धनुषि पूजिते}
{चचाल पृथिवी राजन्सशैलवनकानना}


\twolineshloka
{ववौ सशर्करो वायुर्दिशश्च रजसा वृताः}
{हाहाकारश्च संवज्ञे सुराणां दिवि भारत}


\twolineshloka
{तमिगुं सन्धितं दृष्ट्वा सूतपुत्रेण मारिष}
{विषादं परमं जग्मुः पाण्डवा दीनचेतसः}


\twolineshloka
{स सायकः कर्णभुजप्रमुक्तःशक्राशनिग्रख्यरुचिः शिताग्रः}
{मुलान्तरं प्राप्य धनञ्जयस्यविवेश वल्मीकमिवोरमोत्तमः}


\twolineshloka
{स राढविद्धः समरे महात्माविधूर्णमानः श्लथहस्तगाण्डिवः}
{चचाल वीभत्सुरमित्रमर्दनःक्षितेः प्रकम्पे च यथाऽचलोत्तमः}


\twolineshloka
{तदन्तरं प्राप्य वृषो महारथोरथाङ्गसुर्वीगतमुज्जिहीर्षुः}
{रथादवप्लुत्य निगृह्य दोर्भ्यांशशाक दैवान्न महाबलोऽपि}


\twolineshloka
{ततः किरीटि प्रतिलभ्य संज्ञांजग्राह बाणं यमदण्डकल्पम्}
{ततोऽर्जुनः प्राञ्जलिकं महात्माततोऽब्रवीद्वासुदेवोऽपि पार्थम्}


\threelineshloka
{छिन्ध्यस्य मूर्धानमरेः शरेणन यावदारोहति वै रथं वृषः}
{तथैव सम्पूज्य स तद्वचः प्रभो--स्ततः शरं प्रज्वलितं प्रगृह्य}
{जघान कक्षाममलार्कवर्णां महारथे रथचक्रे विमग्ने}


\twolineshloka
{तं हस्तिकक्षाप्रवरं सुकेतुंसुवर्णमुक्तामणिवज्रजुष्टम्}
{कलाप्रकृष्टोत्तमशिल्पियत्नैःकृतं सुरूपं तपनीयचित्रम्}


\twolineshloka
{जयास्पदं तव सैन्यस्य नित्य---ममित्रवित्रासनमीड्यरूपम्}
{विख्यातमादित्यसुतस्य लोकेत्विषा समं पावकभानुचन्द्रैः}


\twolineshloka
{ततः क्षुरप्रेण सुसंशितेनसुवर्णपुङ्खेन हुताग्निवर्चसा}
{श्रिया ज्वलन्तं ध्वजमुन्ममाथमहारथस्याधिरथेः किरीटी}


\twolineshloka
{यशश्च दर्पश्च तथा प्रियाणिसर्वाणि कार्याणि च तेन केतुना}
{साकं कुरूणां हृदयानि चापतन्बभूव हाहेति च निःस्वनो महान्}


\twolineshloka
{द्वष्ट्वा ध्वजं पातितमाशुकारिणाकुरुप्रवीरेण निकृत्तमाहवे}
{नाशंसिरे सूतपुत्रस्य सर्वेजयं तदा भारत ये त्वदीयाः}


\twolineshloka
{अथ त्वरन्कर्णवधाय पार्थोमहेद्रवज्रानलदण्डसन्निभम्}
{आदत्त चाथाञ्जलिकं निषङ्गा--त्सहस्ररश्मेरिव रश्मिमुत्तमम्}


\twolineshloka
{मर्मच्छिदं शोणितमांसदिग्धंवैश्वानरार्कप्रतिमं महार्हम्}
{नराश्वनागासुहरं त्र्यरत्निंषड्वाजमञ्जोगतिमुग्रवेगम्}


\twolineshloka
{सहस्रनेत्राशनितुल्यवीर्यंकालानलं व्यात्तमिवातिघोरम्}
{पिनाकनारायणचक्रसन्निभंभयङ्करं प्राणभृतां विनाशनम्}


\twolineshloka
{जग्राह पार्थः स शरं प्रहृष्टोयो देवसङ्घैरपि दुर्निवार्यः}
{सम्पूजितो यः सततं महात्मादेवासुरान्यो विजयेन्महेषुः}


\twolineshloka
{तं वै प्रमृष्टं प्रसमीक्ष्य युद्धेचचाल सर्वं सचराचरं जगत्}
{कृत्स्नं जगत्स्वस्त्यृषयोऽभिचक्रु--स्तमुद्यतं प्रेक्ष्य महाहवेषुम्}


\twolineshloka
{ततस्तु तं वै शरमप्रमेयंगाण्डीवधन्वा धनुषि व्ययोजयत्}
{युक्त्वा महास्त्रेण परेण चापंविकृष्य गाण्डीवमुवाच सत्वरम्}


\twolineshloka
{अयं हास्त्रप्रहितो महाशरःशरीरहृच्चासुहरश्च दुर्हृदः}
{तपोऽस्ति तप्तं गुरवश्च तोषितामया यदीष्टं सुहुतं यदि श्रुतम्}


\twolineshloka
{अनेन सत्येन निहन्त्वयं शरःसुसंहितः कर्णमरिं ममोर्जितम्}
{इत्यूचिवांस्तं प्रमुमोच बाणंधनञ्जयः कर्णवधाय घोरम्}


\twolineshloka
{कृत्या ह्यथर्वाङ्गिरसी प्रचोदितायथा तथा त्वं जहि शात्रवं मम}
{ब्रुवन्किरीटी तमतिप्रहृष्टो--ऽसृजद्देवानां जयहेतुं महेषुम्}


\twolineshloka
{जिघांसुरर्केन्दुसमप्रभावःकर्णं वशी पाण्डवः क्षिप्रकारी}
{ततो विमुक्तो बलिना महेषुःप्रज्वालयामास नभो दिशश्च}


\twolineshloka
{सैन्यान्यनेकानि च विप्रमोह्यगाण्डीवमुक्तेन ततो महात्मा}
{तेनार्जुनस्तन्महनीयमस्यशिरोऽहरत्सूतपुत्रस्य राजन्}


\threelineshloka
{शरोत्तमेनाञ्जलिकेन राजं--स्तदा महास्त्रप्रतिमन्त्रितेन}
{पार्थोऽपराह्णे शिर उच्चकर्तवैकर्तनस्याथ महेन्द्रसूनुः}
{छिन्नं पपाताञ्जलिकेन तूर्णंकायश्च पश्चाद्धरणीं जगाम}


\twolineshloka
{तदुद्यतादित्यसमानतेजसंशरन्नभोमध्यगभास्करोपमम्}
{वराङ्गमुर्व्यामपतच्चमुमूखेदिवाकरोऽस्तादिव रक्तमण्डलः}


\twolineshloka
{ततोऽस्य देहं सततं सुखैधितंसुरूपमत्यर्थसुखं सुगन्धि च}
{परेण कृच्छ्रेण शिरः समत्यज--द्गृहं महर्धीन्निवसन्निवेश्वरः}


\twolineshloka
{शरैर्विभिन्नं व्यसु तत्सुवर्चसःपपात कर्णस्य शरीमुच्छ्रितम्}
{स्रवद्व्रणं गैरिकतोयविस्रवंगिरेर्यथा वज्रहतं महाशिरः}


\twolineshloka
{देहाच्च कर्णस्य निपातितस्यतेजः सूर्यं खं वितत्याविवेश}
{तदद्भुतं सर्वमनुष्ययोधाःसंदृष्टवन्तो निहते स्म कर्णे}


\twolineshloka
{ततः शङ्खान्पाण्डवा दध्मुरुच्चै--र्दृष्ट्वा कर्णं पातितं फल्गुनेन}
{तथैव कृष्णश्च धनञ्जयश्चहष्टौ यमौ दध्मतुर्वारिजातौ}


\twolineshloka
{तं सोमकाः प्रेक्ष्य हतं शयानंसैन्यैः सार्धं सिंहनादान्प्रचक्रुः}
{तूर्याणि सञ्जघ्नुरतीव हृष्टावासांसि चैवादुधुवुर्भुजांश्च}


\twolineshloka
{संवर्धयन्तश्च नरेन्द्र योधाःपार्थं समाजग्मुरतीव हृष्टाः}
{बलान्विताश्चापरे ह्यप्यनृत्य--न्नन्योन्यमाश्लिष्य नदन्त ऊचुः}


\twolineshloka
{दृष्ट्वा तु कर्णं भुवि वा विपन्नंकृत्तं रथात्सायकैरर्जुनस्य}
{महानिलेनाद्रिमिवापविद्वंयज्ञावसानेऽग्निमिव प्रशान्तम्}


\twolineshloka
{तदाननं सूर्यसुतस्य राज--न्विभ्राजते पद्मिवावनालम्}
{रराज कर्णस्य शिरो निकृत्त--मस्तंगतं भास्करस्येव बिम्बम्}


\twolineshloka
{शरैराचितसर्वाङ्गः शोणितौघपरिप्लुतः}
{रराज देहः कर्णस्य स्वरश्मिभिरिवांशुमान्}


\twolineshloka
{प्रताप्य सेनामामित्रीं दीप्तैः शरगभस्तिभिः}
{बलिनार्जुनकालेन नीतोऽस्तं कर्णभास्करः}


\twolineshloka
{अस्तं गच्छन्यथाऽदित्यः प्रभामादाय गच्छति}
{तथा जीवितमाहाय कर्णस्येषुर्जगाम सः}


\twolineshloka
{अपराह्णेऽपराह्णस्य सूतपुत्रस्य मारिष}
{छिन्नमञ्जलिकेनाजौ सोत्सेधमपतच्छिरः}


\twolineshloka
{उपर्युपरि सैन्यानां विनिघ्नन्नितराञ्जनान्}
{शिरः कर्णस्य सोऽत्सेधमिषुः सोप्यहरद्द्रुतम्}


\twolineshloka
{कर्णं तु शूरं पतितं पृथिव्यांशराचितं शोणितदिग्धगात्रम्}
{दृष्ट्वा शयानं भुवि मद्रराज--श्छिन्नध्वजेनाथ ययौ रथेन}


\twolineshloka
{हते कर्णे कुरवः प्राद्रवन्तभयार्दिता गाढविद्धाश्च सङ्ख्ये}
{अवेक्षमाणा मुहुरर्जुनस्यध्वजं महान्तं वपुषा ज्वलन्तम्}


\twolineshloka
{तच्छिरो भरतश्रेष्ठ शोभयामास मेदिनीम्}
{यदृच्छयेव पतितं मण्डलं चाण्डदीधितेः}


\twolineshloka
{तं दृष्ट्वा समरविमर्द (बद्ध) लब्धनिद्रंदष्टोष्ठं रुधिरपीरतकातराक्षम्}
{राधेयं रथवरपृष्ठसन्निषण्णंहीनांशुर्दिवसकरो मुहूर्तमासीत्}


\twolineshloka
{निःशब्दतूर्यं हतयोधमुख्यंप्रशान्तदर्पं धृतराष्ट्रसैन्यम्}
{न शोभते सूर्यसुतेन हीनंवृन्दं ग्रहाणामिव चन्द्रहीनम्}


\twolineshloka
{सहस्रनेत्रप्रतिमानकर्मणःसहस्रपत्रप्रतिमाननं शुभम्}
{सहस्ररश्मिर्दिवसक्षये यथातथाऽपतत्कर्णशिरो वसुन्धराम्}


\twolineshloka
{व्यूढोरस्कं कमलवदनं तप्तहेमावभासंकर्णं दृष्ट्वा भुवि निपतितं पार्थबाणाभितप्तम्}
{पांसुग्रस्तं मलिनमसकृत्पुत्रमन्वीक्षमाणोमन्दम्मन्दं व्रजति सविता मन्दिरं मन्दरश्मिः}


\chapter{अध्यायः १०४}
\twolineshloka
{सञ्जय उवाच}
{}


\twolineshloka
{कर्णं तु शूरं पतितं पृथिव्यांशराचितं शोणितदिग्धगात्रम्}
{यदृच्छया सूर्यमिवावनिं गतंदिदृक्षवः सम्परिवार्य तस्थुः}


\twolineshloka
{प्रहृष्टवित्रस्तविषण्णविस्मिता--स्तथाऽपरे शोकसमन्विताऽभवन्}
{परे त्वदीयाश्च रणे विशाम्पतेयथायथेष्टं पृतना तथा गता}


\twolineshloka
{प्रविद्धवर्माभरणायुधाम्बरंधनञ्जयेन प्रहतं महारथम्}
{निशाम्य कर्णं कुरवः प्रदुद्रुवु--र्हतर्षभं केसरिणेव गोकुलम्}


\twolineshloka
{कृत्वा दिमर्दं भृशमर्जुनेनकर्णं हतं केसरिणेव नागम्}
{दृष्ट्वा शयानं युधि मद्रराज--श्छिन्नध्वजेनापययौ रथेन}


\twolineshloka
{कर्णे हते पार्थभयात्प्रदुद्रुवु--र्वैकर्तने धार्तराष्ट्राः सशल्याः}
{अवेक्षमाणा मुहुरर्जुनस्यकेतुं महान्तं यशसा ज्वलन्तम्}


\twolineshloka
{शल्यस्तु कर्णार्जुनयोर्विमर्देबलानि दृष्ट्वा मृदितध्वजानि}
{ययौ स तेनैव रथेन तूर्णंहेलीकृतः सृञ्जयसोमकैश्च}


\threelineshloka
{निपातितस्यन्दनवाजिनागंबलं च दृष्ट्वा हतसूतपुत्रम्}
{भीमस्तु भीमेन तदा स्वरेणसमुन्नदद्रोदसी कम्पयंश्च}
{}


\twolineshloka
{आस्फोटयन्नृत्यति वल्गते चहते कर्णे त्रासयन्धार्तराष्ट्रान्}
{तदा राजन्सृञ्जयाः सोमकाश्चशङ्खान्दध्मुः सस्वजुश्चापि सर्वे}


\twolineshloka
{परस्परं क्षत्रिया हृष्टरूपाःसूतात्मजे वै निहते तदानीम्}
{दुर्योधनोऽश्रुप्रतिपूर्णनेत्रोदीनो मुहुर्निश्वसन्नार्तरूपः}


\twolineshloka
{मद्राधिपश्चापि विमूढचेता--स्तूर्णं ध्वजेनापहृतेन तेन}
{रथेन दुर्योधनमेत्य तूर्णंपश्यन्सुदुःखात्तमुवाच राजन्}


\twolineshloka
{विशीर्णनागाश्वरथप्रवीरंबलं त्वदीयं यमराष्ट्रकल्पम्}
{अन्योन्यमासाद्य हतैः शयानै--र्नराश्वनागैर्गिरिकूटकल्पैः}


\twolineshloka
{नैतादृशं भारत युद्धमासी--द्यथा हि कर्णार्जुनयोर्बभूव}
{ग्रस्तौ हि कर्णेन समेत्य कृष्णा--वन्ये हि सर्वे तव शात्रवेयाः}


\twolineshloka
{दैवं तु यत्तत्स्ववशे प्रवृत्तंतत्पाण्डवान्पाति निहन्ति चास्मान्}
{तवार्थसिद्ध्यर्थकरा हि सर्वेप्रसह्य वीरा निहता द्विषद्भिः}


\twolineshloka
{कुबेरवैवस्वतवासवानांतुल्यप्रभावा वरुणस्य चापि}
{वीर्येण शौर्येण पराक्रमेणतैश्चापि युक्ता विमलैर्गुणौघैः}


\twolineshloka
{अवध्यकल्पा निहता नरेन्द्रा--स्तवार्थकामा युधि पाण्डवेयैः}
{तन्मा शुचो भारत दिष्टमेत--त्पर्यायसिद्धिर्न तु नित्यसिद्धिः}


\twolineshloka
{एतद्वचो मद्रपतेर्निशम्यस्वं चाविनीतं मनसा विचिन्त्य}
{दुर्योधनोऽश्रुप्रतिपूर्णनेत्रोमुहुर्मुहुर्निश्वसन्नार्तरूपः}


\twolineshloka
{*तद्व्यानमूकं कृपणं भृशार्त--मार्तायनिर्दीनमुवाच राजन्}
{पश्येदमुग्रैर्नरवाजिनागै--रायोधनं वीर हतैः सुपूर्णम्}


\twolineshloka
{महीधराभैः पतितैश्च नागैःसकृत्प्रयत्नैः शरभिन्नगात्रैः}
{प्रविह्वलद्भिश्च गतासुभिश्चप्रध्वस्तवर्मायुधसर्वयोधैः}


\twolineshloka
{वज्रापविद्वैरिव चाचलौघै--र्विभिन्नपाषाणमृगद्रुमौघैः}
{प्रविद्धघण्टाङ्कुशतोमरध्वजैःसहेमजालै रुधिरौघसम्प्लवैः}


\twolineshloka
{शरावनुन्नैः पतितैस्तुरङ्गै--स्तनद्भिरार्तैः क्षतजं वमद्भिः}
{दीनैस्त्रसद्भिः परिवृत्तनेत्रै--र्मही पतद्भिः कृपणा वभूव}


\twolineshloka
{तथाऽपविद्वैर्गजवाजियूथै--र्मन्दासुभिश्चापि गतासुभिश्च}
{रथैश्च नागैस्तुरगैश्च नुन्नै--र्मही महावैतरणीव भाति}


\twolineshloka
{गजैर्निकृत्तापरहस्तगात्रै--रुद्वेपमानैः पतितैः पृथिव्याम्}
{रथर्षभैर्नागवराश्वयोधैःपदातिभिश्चाभिमुखैः परैर्हतैः}


\twolineshloka
{विशीर्णवर्माभरणाम्बरायुधै--र्युता प्रशान्तैरिव पावकैर्मही}
{शरप्रहाराभिहतैर्महाबलै--रवेक्षमाणैः पतितैः समन्ततः}


\twolineshloka
{प्रनष्टसंज्ञैः पुनराश्वसद्भि--र्मही बभौ कुम्भगतैरिवाग्निभिः}
{रत्नैश्च्युतैर्भूरतिदीप्तिमद्भि--र्नक्तं ग्रहैर्द्यौर्विमलैर्विभाति}


\twolineshloka
{शरास्तु कर्णार्जुनबाहुमुक्ताविदार्य नागांश्च मनुष्यदेहान्}
{प्राणान्निरस्याशु महीं प्रतीयु--र्महोरगा वासमिवाभिनम्राः}


\twolineshloka
{हतैर्मनुष्यैश्च गजैश्च सङ्ख्येशरावभिन्नै रुधिरप्रदिग्धैः}
{धऩञ्जयस्याधिरथेश्च मार्गेहतैरगम्या वसुधाऽतिदुर्गा}


% Check verse!
रथैर्महेषून्मथितैः सुकल्पितैःसनागयोधाश्ववरायुधध्वजैःविशीर्णयोक्त्रैर्विनिकीर्णबन्धनै--र्निकृत्तचक्राक्षयुगत्रिवेणुभिः
\twolineshloka
{विमुक्तयोक्त्रैश्च रथैरुपस्करैःकृताभिषङ्गैर्विनिबद्वकूबरैः}
{प्रभग्ननीडैर्मणिहेमभूषितै--र्वृता मही द्यौरिव शारदैर्घनैः}


\twolineshloka
{निकृष्यमाणा जवनैस्तुरङ्गमै--र्हतेश्वरा राजरथाः सुकल्पिताः}
{मनुष्यमातङ्करथाश्वसादिषुद्रुतं व्रजन्तो बहुधा विघूर्णिताः}


\twolineshloka
{सहेमपृष्ठाः सपरश्वथायुधाःशिताश्च शूला मुसलाः समुद्गराः}
{चित्राश्च खङ्गा विमलाश्च कोशागजाश्च जाम्बूनदपट्टनद्धाः}


\twolineshloka
{चापानि जाम्बूनदभूषितानिशराश्च कार्तस्वरचित्रपुङ्खाः}
{ऋष्ट्यश्च पीता विमलाश्च कोशाःप्रासाश्च दण्डाः कनकप्रभासाः}


\twolineshloka
{शुभ्राणि वालव्यजनानि शङ्खा--श्छिन्नावरुद्धा विरुचः स्रजश्च}
{कुथाः पताकाम्बरवेष्टनानिकिरीटमाला मकुटाश्च शुभ्राः}


\twolineshloka
{प्रकीर्णका विप्रतिकीर्णकाश्चप्रधानमुख्यास्तरलाश्च हाराः}
{आपीडकेयूरवराङ्गदानिग्रैवेयनिष्काः ससुवर्णसूत्राः}


\twolineshloka
{मण्युत्तमा वज्रसुवर्णयुक्तारत्नानि चोच्चावचमङ्गलानि}
{गात्राणि चाभान्ति सुखोचितानिशिरांसि चेन्दुप्रतिमाननानि}


\twolineshloka
{देहांश्च भोगांश्च परिच्छदांश्चत्यक्त्वा मनोज्ञा निहताः सुखानि}
{स्वधर्मनिष्ठां महतीमवाप्यप्राप्ताश्च लोकान्यशसा गतास्ते}


\threelineshloka
{निवर्त दुर्योधन यान्तु सैनिकाव्रजस्व राजञ्शिबिराणि मानद}
{दिवाकरोऽप्येष विलम्बते प्रभोपुनस्त्वमेवात्र नरेन्द्र कारणम् ॥सञ्जय उवाच}
{}


\twolineshloka
{एतावदुक्त्वा विरराम शल्योदुर्योधनं चैव निवर्तयित्वा}
{युद्धाय राजन्विनिविष्टबुद्धि--र्हा कर्ण हा कर्ण इति ब्रुवाणम्}


\twolineshloka
{तं द्रोणपुत्रप्रमुखा नरेन्द्राःसर्वे समाश्वास्य ततः प्रजग्मुः}
{निरीक्षमाणा ध्वजमर्जुनस्यरथं च दिव्यं यशसा ज्वलन्तम्}


\twolineshloka
{नराश्वमातङ्गशरीरजेनरक्तेन सिक्तां रुधिरेण भूमिम्}
{रक्ताम्बरस्रक्तपनीययोगा--न्नारीं प्रकाशामिव सर्वरम्याम्}


\twolineshloka
{प्रच्छन्नरूपां रुधिरेण राजन्रौद्रे मुहूर्तेऽतिविराजमाने}
{नैवावतस्थुः कुरवः समीक्ष्यप्रव्राजिता नरदेवा यथाऽक्षैः}


\twolineshloka
{वधेन कर्णस्य सुदुःखितास्तेहा कर्ण हा कर्ण इति ब्रुवाणाः}
{द्रुतं प्रजग्मुः शिबिराणि राज-- न्दिवाकरं रक्तमवेक्षमाणाः}


\twolineshloka
{गाण्डीवमुक्तैस्तु सुवर्णपुङ्खैःशिलाशितैः शोणितदिग्धवाजैः}
{शरैश्चिताङ्गोऽपि रराज कर्णोहतोऽपि दीप्तांशुरिवांशुजालैः}


\twolineshloka
{कर्णस्य देहं रुधिराम्बुसिक्तंपुत्रानकुम्पी भगवान्विवस्वान्}
{स्पृष्ट्वा कराग्रैः क्षतजानुरूपैःस्नातुं तदाऽगादपरं समुद्रम्}


\twolineshloka
{इतीव सञ्चिन्त्य सुरर्षभाश्चसम्प्रस्थितास्ते तु यथानिकेतम्}
{सञ्चिन्त्य सर्वा जनता विसस्रु--र्यथासुखं खे च महीतले च}


\twolineshloka
{तदद्भुतं प्राणभृतां भयङ्करंनिशाम्य युद्धं कुरुवीरमुख्ययोः}
{समाचितौ कर्णशरैऋ परन्तपौतथैव संरेजतुरच्युतार्जुनौ}


\twolineshloka
{तमो निहत्याभ्युदितौ यथाऽमलौशशाङ्कसूर्याविव रश्मिमालिनौ}
{विहाय तान्बाणगणान्महाबलौसुहृद्वृतावप्रतिमानविक्रमौ}


\threelineshloka
{सुखप्रविष्टौ सुखदौ विरेजतुःसदस्यचिन्त्याविव विष्णुवासवौ}
{सदेवगन्धर्वमनुष्यचारणै--र्महर्षिभिर्यक्षमहोरगैरपि}
{जयाभिवृद्ध्या परयाऽभिपूजितौनिहत्य कर्णं परमाहवे तदा}


\twolineshloka
{यथानुरूपं प्रतिपूज्य तानथप्रशस्यमानावतुलैश्च कर्मभिः}
{ननन्दतुश्चैव सुहृद्भिरावृतौबलं निहत्येव सुरेशकेशवौ}


\twolineshloka
{[धृष्टराष्ट्र उवाच}
{}


\threelineshloka
{तस्मिंस्तु कर्णार्जुनयोर्विमर्देदग्धस्य रौद्रेऽहनि विद्रुतस्य}
{बभूव कुरुसृञ्जयानांबलस्य बाणोन्मथितस्य कीदृक् ॥सञ्जय उवाच}
{}


\twolineshloka
{शृणु राजन्नवहितो यथा वृत्तो महाक्षयः}
{घोरो मनुष्यदेहानामाजौ च गजवाजिनाम्}


\twolineshloka
{यत्र कर्णे हते पार्थः सिंहनादमथाकरोत्}
{तदा तव सुतान्राजन्नाविवेश महद्भयम्}


\twolineshloka
{न सन्धातुमनीकानि न चैवाशु पराक्रमे}
{आसीद्वुद्विर्हते कर्णे तव योधस्य कर्हिचित्}


\twolineshloka
{वणिजो नावि भिन्नायामगाधे विप्लुवे यथा}
{अपारे पारमिच्छन्तो हते द्वीपे किरीटिना}


\twolineshloka
{सूतपुत्रे हते राजन्वित्रस्ताः शस्त्रविक्षताः}
{अनाथा नाथमिच्छन्तो मृगाः सिंहैरिवार्दिताः}


\twolineshloka
{भग्नशृङ्गा वृषा यद्वद्भग्नदंष्ट्रा इवोरगाः}
{प्रत्यपायाम सायाह्ने निर्जिताः सव्यसाचिना}


\twolineshloka
{हतप्रवीरा विध्वस्ता निकृत्ता निशितैः शरैः}
{सूतपुत्रे हते राजन्पुत्रास्ते दुद्रुवुर्भयात्}


\twolineshloka
{विस्रस्तयन्त्रकवचाः कान्दिग्भूता विचेतसः}
{अन्योन्यमवमृद्गन्तो वीक्षमाणा भयार्दिताः}


\twolineshloka
{मामेव नूनं बीभत्सुर्मामेव च वृकोदरः}
{अभियातीति मन्वानाः पेतुर्मम्लुश्च सम्भ्रमात्}


\twolineshloka
{हयानन्ये गजानन्ये रथानन्ये महारथाः}
{आरुह्य जवसम्पन्नाः पदातीन्प्रजहुर्भयात्}


\twolineshloka
{कुञ्जरैः स्यन्दनाः क्षुण्णाः सादिनश्च महारथैः}
{पदातिसङ्घाश्चाश्वौधैः पलायद्भिर्भयार्दितैः}


\twolineshloka
{व्यालतस्करसङ्कीर्णे सार्थहीना यथा वने}
{सूतपुत्रे हते राजंस्तव योधास्तथाऽभवन्}


\twolineshloka
{हतारोहा यथा नागाश्छिन्नहस्ता यथा नराः}
{सर्वे पार्थमयं लोकं सम्पश्यन्तो भयार्दिताः}


\twolineshloka
{सम्प्रेक्ष्य द्रवतः सर्वान्भीमसेनभयार्दितान्}
{दुर्योधनोऽथ स्वं सूतं हाहाकृत्वेदमब्रवीत्}


\twolineshloka
{नातिक्रमेच्च मां पार्थो धनुष्पाणिमवस्थितम्}
{जघने सर्वसैन्यानां शनैरश्वान्प्रचोदय}


\twolineshloka
{युध्यमानं हि कौन्तेयं हनिष्यामि न संशयः}
{नोत्सहेन्मामतिक्रान्तुं वेलामिव महोदधिः}


\twolineshloka
{अद्यार्जुनं सगोविन्दं मानिनं च वृकोदरम्}
{अन्याञ्शिष्टांस्तथा शत्रून्कर्णस्यानृण्यमाप्नुयाम्}


\twolineshloka
{तच्छ्रुत्वा कुरुराजस्य शूरार्यसदृशं वचः}
{सूतो हेमपरिच्छन्नाञ्शनैरश्वानचोदयत्}


\twolineshloka
{रथाश्वनागहीनास्तु पादातास्तव मारिष}
{पञ्चविंशतिसाहस्रा युद्वायैव व्यवस्थिताः}


\twolineshloka
{तान्भीमसेनः सङ्क्रुद्धो धृष्ट्यद्युम्नश्च पार्षतः}
{बलेन चतुरङ्गेण संवृत्याजघ्नतुः शरैः}


\twolineshloka
{प्रत्ययुध्यन्त समरे भीमसेनं सपार्षतम्}
{पार्थपार्षतयोश्चान्ये जगृहुस्तत्र नामनी}


\twolineshloka
{अक्रुध्यत रणे भीमस्तैस्तदा पर्यवस्थितैः}
{सोऽवतीर्य रथात्तूर्णं गदापाणिरयुध्यत}


\twolineshloka
{न तान्रथस्थो भूमिष्ठान्धर्मापेक्षी वृकोदरः}
{योधयामास कौन्तेयो भुजवीर्यव्यपाश्रयः}


\twolineshloka
{जातरूपपरिच्छन्नां प्रगृह्य महतीं गदाम्}
{अवधीत्तावकान्सर्वान्दण्डपाणिरिवान्तकः}


\twolineshloka
{पदातिनोऽपि सन्त्यक्त्वा प्रियं जीवितमात्मनः}
{भीममभ्यद्रवन्सङ्ख्ये पतङ्गा ज्वलनं यथा}


\twolineshloka
{आसाद्य भीमसेनं तु संरब्धा युद्धदुर्मदाः}
{विनेशुः सहसा दृष्ट्वा भूतग्रामा इवान्तकम्}


\twolineshloka
{श्येनवद्विचरन्भीमो गदाहस्तो महाबलः}
{पञ्चविंशतिसाहस्रांस्तावकानवपोथयत्}


\twolineshloka
{हत्वा तत्पुरुषानीकं भीमः सत्यपराक्रमः}
{धृष्टद्युम्नं पुरस्कृत्य तस्थौ तत्र महाबलः}


\threelineshloka
{धनञ्जयो रथानीकमभ्यवर्तत वीर्यवान्}
{माद्रीपुत्रौ तु शकुनिं सात्यकिश्च महारथः}
{जवेनाभ्यपतन्हृष्टा घ्नन्तो दौर्योधनं बलम्}


\twolineshloka
{तस्याश्वसादीन्सुबहूंस्ते निहत्य शितैः शरैः}
{समभ्यधावंस्त्वरितास्तत्र युद्धमभून्महत्}


\twolineshloka
{धनञ्जयोऽपि चाभ्येत्य रथानीकं तव प्रभो}
{विश्रुतं त्रिषु लोकेषु गाण्डीवं विक्षिपन्धनुः}


\twolineshloka
{कृष्णसारथिमायान्तं दृष्ट्वा श्वेतहयं रथम्}
{अर्जुनं चापि योद्वारं त्वदीयाः प्राद्रवन्भयात्}


\twolineshloka
{विप्रहीणरथाश्चैव शरैश्च परिकर्शिताः}
{पञ्चविंशतिसाहस्राः कालमार्छन्पदातयः}


\twolineshloka
{हत्वा तान्पुरुषव्याघ्रः पाञ्चालानां महारथः}
{पुत्राः पाञ्चालराजस्य धृष्टद्युम्नो महामनाः}


\twolineshloka
{भीमसेनं पुरस्कृत्य नचिरात्प्रत्यदृश्यत}
{महाधनुर्धरः श्रीमानमित्रगणतापनः}


\twolineshloka
{पारावतसवर्णाश्वं कोविदारमयध्वजम्}
{धृष्टद्युम्नं रणे दृष्ट्वा त्वदीयाः प्राद्रवन्भयात्}


\twolineshloka
{गान्धारराजं शीघ्रास्त्रमनुसृत्य यशस्विनौ}
{नचिरात्प्रत्यदृश्येतां माद्रीपुत्रौ ससात्यकी}


\twolineshloka
{वेकितानः शिखण्डी च द्रौपदेयाश्च मारिष}
{हत्वा त्वदीयं सुमहत्सैन्यं शङ्खांस्तथाऽधमन्}


\twolineshloka
{ते सर्वे तावकान्प्रेक्ष्य द्रवतोऽपि पराङ्मुखान्}
{अभ्यवर्तन्त संरब्धान्वृषाञ्जित्वा यथा वृषाः}


\twolineshloka
{सेनावशेषं तं दृष्ट्वा तव सैन्यस्य पाण्डवः}
{व्यवस्थितः सव्यसाची चुक्रोध बलवान्नृप}


\twolineshloka
{धनञ्जयो रथानीकमभ्यवर्तत वीर्यवान्}
{विश्रुतं त्रिषु लोकेषु व्याक्षिपद्गाण्डिवं धनुः}


\twolineshloka
{तत एनाञ्शरव्रातैः सहसा समवाकिरत्}
{तमसा संवृतेनाथ न स्म किञ्चिद्वदृश्यत}


\twolineshloka
{अन्धकारीकृते लोके रजोभूते महीतले}
{योधाः सर्वे महाराज तावकाः प्राद्रावन्भयात्}


\twolineshloka
{सम्भज्यमाने सैन्ये तु कुरुराजो विशाम्पते}
{परानभिमुखांश्चैव सुतस्ते समुपाद्रवत्}


\twolineshloka
{ततो दुर्योधनः सर्वानाजुहावाथ पाण्डवान्}
{युद्धाय भरतश्रेष्ठ देवानिव पुरा बलिः}


\twolineshloka
{त एनमभिगर्जन्तः सहिताः समुपाद्रवन्}
{नानाशस्त्रभृतः क्रुद्धा भर्त्सयन्तो मुहुर्मुहुः}


\twolineshloka
{दुर्योधनोऽप्यसम्भ्रान्तस्तान्रणे निशितैः शरैः}
{तत्रावधीत्ततः क्रुद्वः शतशोऽथ सहस्रशः}


% Check verse!
तत्सैन्यं पाण्डवेयानां योधयामास सर्वतः
\threelineshloka
{तत्राद्भुतमपश्याम तव पुत्रस्य पौरुषम्}
{यदेकः सहितान्सर्वान्रणेऽयुध्यत पाण्डवान्}
{ततोऽपश्यन्महात्मा स स्वसैन्यं भृशदुःखितम्}


\twolineshloka
{ततोऽवस्थाप्य राजेन्द्र कृतबुद्धिस्तवात्मजः}
{हर्षयन्निव तान्योधानिदं वचनमब्रवीत्}


\twolineshloka
{न तं देशं प्रपश्यामि यत्र याता भयार्दिताः}
{गतानां यत्र वै मोक्षः पाण्डवात्किं गतेन वः}


\twolineshloka
{अल्पं च बलमेतेषां कृष्णौ च भृशविक्षतौ}
{अद्य सर्वान्हनिष्यामिध्रुवो हि विजयो भवेत्}


\twolineshloka
{विप्रयातांस्तु वो भिन्नान्पाण्डवाः कृतकिल्विषान्}
{अनुसृत्य वधिष्यन्ति श्रेयान्नः समरे वधः}


\twolineshloka
{सुखं साङ्ग्रामिको मृत्युः क्षत्रधर्मेण युध्यताम्}
{मृतो दुःखं न जानीते प्रेत्य चानन्त्यमश्नुते}


\twolineshloka
{शृणुध्वं क्षत्रियाः सर्वे यावन्तः स्थ समागताः}
{यदा शूरं च भीरुं च मारयत्यन्तको मयः}


\twolineshloka
{को न मूढो न युध्येत मादृशः क्षत्रियव्रतः}
{द्विषतो भीमसेनस्य क्रद्धस्य वशमेष्यथा}


\twolineshloka
{पितामहैराचरितं न धर्मं हातुमर्हथ}
{न ह्यधर्मोऽस्ति पापीयान्क्षत्रियस्य पलायनात्}


\threelineshloka
{न युद्धधर्माच्छ्रेयो हि पन्थाः स्वर्गस्य कौरवाः}
{अचिरेण हता लोकान्सद्यो योधाः समश्नुत ॥सञ्जय उवाच}
{}


\threelineshloka
{एवं ब्रुवति पुत्रे ते सैनिका भृशविक्षताः}
{अनवेक्ष्यैव तद्वाक्यं प्राद्रवन्सर्वतो दिशः ॥सञ्जय उवाच}
{}


\twolineshloka
{दृष्ट्वा तु सैन्यं परिवृत्यमानंपुत्रेण ते मद्रपतिस्तदानीम्}
{सन्त्रस्तरूपः परिमूढचेतादुर्योधनं वाक्यमिदं बभाषे]}


\chapter{अध्यायः १०५}
\twolineshloka
{[धृष्टराष्ट्र उवाच}
{}


\threelineshloka
{तस्मिंस्तु कर्णार्जुनयोर्विमर्देदग्धस्य रौद्रेऽहनि विद्रुतस्य}
{बभूव कुरुसृञ्जयानांबलस्य बाणोन्मथितस्य कीदृक् ॥सञ्जय उवाच}
{}


\twolineshloka
{शृणु राजन्नवहितो यथा वृत्तो महाक्षयः}
{घोरो मनुष्यदेहानामाजौ च गजवाजिनाम्}


\twolineshloka
{यत्र कर्णे हते पार्थः सिंहनादमथाकरोत्}
{तदा तव सुतान्राजन्नाविवेश महद्भयम्}


\twolineshloka
{न सन्धातुमनीकानि न चैवाशु पराक्रमे}
{आसीद्वुद्विर्हते कर्णे तव योधस्य कर्हिचित्}


\twolineshloka
{वणिजो नावि भिन्नायामगाधे विप्लुवे यथा}
{अपारे पारमिच्छन्तो हते द्वीपे किरीटिना}


\twolineshloka
{सूतपुत्रे हते राजन्वित्रस्ताः शस्त्रविक्षताः}
{अनाथा नाथमिच्छन्तो मृगाः सिंहैरिवार्दिताः}


\twolineshloka
{भग्नशृङ्गा वृषा यद्वद्भग्नदंष्ट्रा इवोरगाः}
{प्रत्यपायाम सायाह्ने निर्जिताः सव्यसाचिना}


\twolineshloka
{हतप्रवीरा विध्वस्ता निकृत्ता निशितैः शरैः}
{सूतपुत्रे हते राजन्पुत्रास्ते दुद्रुवुर्भयात्}


\twolineshloka
{विस्रस्तयन्त्रकवचाः कान्दिग्भूता विचेतसः}
{अन्योन्यमवमृद्गन्तो वीक्षमाणा भयार्दिताः}


\twolineshloka
{मामेव नूनं बीभत्सुर्मामेव च वृकोदरः}
{अभियातीति मन्वानाः पेतुर्मम्लुश्च सम्भ्रमात्}


\twolineshloka
{हयानन्ये गजानन्ये रथानन्ये महारथाः}
{आरुह्य जवसम्पन्नाः पदातीन्प्रजहुर्भयात्}


\twolineshloka
{कुञ्जरैः स्यन्दनाः क्षुण्णाः सादिनश्च महारथैः}
{पदातिसङ्घाश्चाश्वौधैः पलायद्भिर्भयार्दितैः}


\twolineshloka
{व्यालतस्करसङ्कीर्णे सार्थहीना यथा वने}
{सूतपुत्रे हते राजंस्तव योधास्तथाऽभवन्}


\twolineshloka
{हतारोहा यथा नागाश्छिन्नहस्ता यथा नराः}
{सर्वे पार्थमयं लोकं सम्पश्यन्तो भयार्दिताः}


\twolineshloka
{सम्प्रेक्ष्य द्रवतः सर्वान्भीमसेनभयार्दितान्}
{दुर्योधनोऽथ स्वं सूतं हाहाकृत्वेदमब्रवीत्}


\twolineshloka
{नातिक्रमेच्च मां पार्थो धनुष्पाणिमवस्थितम्}
{जघने सर्वसैन्यानां शनैरश्वान्प्रचोदय}


\twolineshloka
{युध्यमानं हि कौन्तेयं हनिष्यामि न संशयः}
{नोत्सहेन्मामतिक्रान्तुं वेलामिव महोदधिः}


\twolineshloka
{अद्यार्जुनं सगोविन्दं मानिनं च वृकोदरम्}
{अन्याञ्शिष्टांस्तथा शत्रून्कर्णस्यानृण्यमाप्नुयाम्}


\twolineshloka
{तच्छ्रुत्वा कुरुराजस्य शूरार्यसदृशं वचः}
{सूतो हेमपरिच्छन्नाञ्शनैरश्वानचोदयत्}


\twolineshloka
{रथाश्वनागहीनास्तु पादातास्तव मारिष}
{पञ्चविंशतिसाहस्रा युद्वायैव व्यवस्थिताः}


\twolineshloka
{तान्भीमसेनः सङ्क्रुद्धो धृष्ट्यद्युम्नश्च पार्षतः}
{बलेन चतुरङ्गेण संवृत्याजघ्नतुः शरैः}


\twolineshloka
{प्रत्ययुध्यन्त समरे भीमसेनं सपार्षतम्}
{पार्थपार्षतयोश्चान्ये जगृहुस्तत्र नामनी}


\twolineshloka
{अक्रुध्यत रणे भीमस्तैस्तदा पर्यवस्थितैः}
{सोऽवतीर्य रथात्तूर्णं गदापाणिरयुध्यत}


\twolineshloka
{न तान्रथस्थो भूमिष्ठान्धर्मापेक्षी वृकोदरः}
{योधयामास कौन्तेयो भुजवीर्यव्यपाश्रयः}


\twolineshloka
{जातरूपपरिच्छन्नां प्रगृह्य महतीं गदाम्}
{अवधीत्तावकान्सर्वान्दण्डपाणिरिवान्तकः}


\twolineshloka
{पदातिनोऽपि सन्त्यक्त्वा प्रियं जीवितमात्मनः}
{भीममभ्यद्रवन्सङ्ख्ये पतङ्गा ज्वलनं यथा}


\twolineshloka
{आसाद्य भीमसेनं तु संरब्धा युद्धदुर्मदाः}
{विनेशुः सहसा दृष्ट्वा भूतग्रामा इवान्तकम्}


\twolineshloka
{श्येनवद्विचरन्भीमो गदाहस्तो महाबलः}
{पञ्चविंशतिसाहस्रांस्तावकानवपोथयत्}


\twolineshloka
{हत्वा तत्पुरुषानीकं भीमः सत्यपराक्रमः}
{धृष्टद्युम्नं पुरस्कृत्य तस्थौ तत्र महाबलः}


\threelineshloka
{धनञ्जयो रथानीकमभ्यवर्तत वीर्यवान्}
{माद्रीपुत्रौ तु शकुनिं सात्यकिश्च महारथः}
{जवेनाभ्यपतन्हृष्टा घ्नन्तो दौर्योधनं बलम्}


\twolineshloka
{तस्याश्वसादीन्सुबहूंस्ते निहत्य शितैः शरैः}
{समभ्यधावंस्त्वरितास्तत्र युद्धमभून्महत्}


\twolineshloka
{धनञ्जयोऽपि चाभ्येत्य रथानीकं तव प्रभो}
{विश्रुतं त्रिषु लोकेषु गाण्डीवं विक्षिपन्धनुः}


\twolineshloka
{कृष्णसारथिमायान्तं दृष्ट्वा श्वेतहयं रथम्}
{अर्जुनं चापि योद्वारं त्वदीयाः प्राद्रवन्भयात्}


\twolineshloka
{विप्रहीणरथाश्चैव शरैश्च परिकर्शिताः}
{पञ्चविंशतिसाहस्राः कालमार्छन्पदातयः}


\twolineshloka
{हत्वा तान्पुरुषव्याघ्रः पाञ्चालानां महारथः}
{पुत्राः पाञ्चालराजस्य धृष्टद्युम्नो महामनाः}


\twolineshloka
{भीमसेनं पुरस्कृत्य नचिरात्प्रत्यदृश्यत}
{महाधनुर्धरः श्रीमानमित्रगणतापनः}


\twolineshloka
{पारावतसवर्णाश्वं कोविदारमयध्वजम्}
{धृष्टद्युम्नं रणे दृष्ट्वा त्वदीयाः प्राद्रवन्भयात्}


\twolineshloka
{गान्धारराजं शीघ्रास्त्रमनुसृत्य यशस्विनौ}
{नचिरात्प्रत्यदृश्येतां माद्रीपुत्रौ ससात्यकी}


\twolineshloka
{वेकितानः शिखण्डी च द्रौपदेयाश्च मारिष}
{हत्वा त्वदीयं सुमहत्सैन्यं शङ्खांस्तथाऽधमन्}


\twolineshloka
{ते सर्वे तावकान्प्रेक्ष्य द्रवतोऽपि पराङ्मुखान्}
{अभ्यवर्तन्त संरब्धान्वृषाञ्जित्वा यथा वृषाः}


\twolineshloka
{सेनावशेषं तं दृष्ट्वा तव सैन्यस्य पाण्डवः}
{व्यवस्थितः सव्यसाची चुक्रोध बलवान्नृप}


\twolineshloka
{धनञ्जयो रथानीकमभ्यवर्तत वीर्यवान्}
{विश्रुतं त्रिषु लोकेषु व्याक्षिपद्गाण्डिवं धनुः}


\twolineshloka
{तत एनाञ्शरव्रातैः सहसा समवाकिरत्}
{तमसा संवृतेनाथ न स्म किञ्चिद्वदृश्यत}


\twolineshloka
{अन्धकारीकृते लोके रजोभूते महीतले}
{योधाः सर्वे महाराज तावकाः प्राद्रावन्भयात्}


\twolineshloka
{सम्भज्यमाने सैन्ये तु कुरुराजो विशाम्पते}
{परानभिमुखांश्चैव सुतस्ते समुपाद्रवत्}


\twolineshloka
{ततो दुर्योधनः सर्वानाजुहावाथ पाण्डवान्}
{युद्धाय भरतश्रेष्ठ देवानिव पुरा बलिः}


\twolineshloka
{त एनमभिगर्जन्तः सहिताः समुपाद्रवन्}
{नानाशस्त्रभृतः क्रुद्धा भर्त्सयन्तो मुहुर्मुहुः}


\twolineshloka
{दुर्योधनोऽप्यसम्भ्रान्तस्तान्रणे निशितैः शरैः}
{तत्रावधीत्ततः क्रुद्वः शतशोऽथ सहस्रशः}


% Check verse!
तत्सैन्यं पाण्डवेयानां योधयामास सर्वतः
\threelineshloka
{तत्राद्भुतमपश्याम तव पुत्रस्य पौरुषम्}
{यदेकः सहितान्सर्वान्रणेऽयुध्यत पाण्डवान्}
{ततोऽपश्यन्महात्मा स स्वसैन्यं भृशदुःखितम्}


\twolineshloka
{ततोऽवस्थाप्य राजेन्द्र कृतबुद्धिस्तवात्मजः}
{हर्षयन्निव तान्योधानिदं वचनमब्रवीत्}


\twolineshloka
{न तं देशं प्रपश्यामि यत्र याता भयार्दिताः}
{गतानां यत्र वै मोक्षः पाण्डवात्किं गतेन वः}


\twolineshloka
{अल्पं च बलमेतेषां कृष्णौ च भृशविक्षतौ}
{अद्य सर्वान्हनिष्यामिध्रुवो हि विजयो भवेत्}


\twolineshloka
{विप्रयातांस्तु वो भिन्नान्पाण्डवाः कृतकिल्विषान्}
{अनुसृत्य वधिष्यन्ति श्रेयान्नः समरे वधः}


\twolineshloka
{सुखं साङ्ग्रामिको मृत्युः क्षत्रधर्मेण युध्यताम्}
{मृतो दुःखं न जानीते प्रेत्य चानन्त्यमश्नुते}


\twolineshloka
{शृणुध्वं क्षत्रियाः सर्वे यावन्तः स्थ समागताः}
{यदा शूरं च भीरुं च मारयत्यन्तको मयः}


\twolineshloka
{को न मूढो न युध्येत मादृशः क्षत्रियव्रतः}
{द्विषतो भीमसेनस्य क्रद्धस्य वशमेष्यथा}


\twolineshloka
{पितामहैराचरितं न धर्मं हातुमर्हथ}
{न ह्यधर्मोऽस्ति पापीयान्क्षत्रियस्य पलायनात्}


\threelineshloka
{न युद्धधर्माच्छ्रेयो हि पन्थाः स्वर्गस्य कौरवाः}
{अचिरेण हता लोकान्सद्यो योधाः समश्नुत ॥सञ्जय उवाच}
{}


\threelineshloka
{एवं ब्रुवति पुत्रे ते सैनिका भृशविक्षताः}
{अनवेक्ष्यैव तद्वाक्यं प्राद्रवन्सर्वतो दिशः ॥सञ्जय उवाच}
{}


\twolineshloka
{दृष्ट्वा तु सैन्यं परिवृत्यमानंपुत्रेण ते मद्रपतिस्तदानीम्}
{सन्त्रस्तरूपः परिमूढचेतादुर्योधनं वाक्यमिदं बभाषे]}


\chapter{अध्यायः १०६}
\twolineshloka
{सञ्जय उवाच}
{}


\twolineshloka
{शरसङ्कृत्तवर्माणं रुधिरोक्षितवक्षसम्}
{गतासुमपि राधेयं नैव लक्ष्मीर्विमुञ्चति}


\twolineshloka
{तप्तजाम्बूनदनिभं बालार्कसदृशद्युतिम्}
{जीवन्तमिव तं शूरं सर्वभूतानि मेनिरे}


\twolineshloka
{हतस्यापि महाराज सूतपुत्रस्य संयुगे}
{वित्रेसुः सर्वतो योधाः सिंहस्येवेतरे मृगाः}


\twolineshloka
{हतोऽपि पुरुषव्याघ्रो व्याहरन्निव लक्ष्यते}
{नाभवद्विकृतिः काचिन्मृतस्यापि महात्मनः}


\twolineshloka
{चारुवेषधरं राजंश्चारुमौलिशिरोधरम्}
{तन्मुखं सूतपुत्रस्य पूर्णचन्द्रसमद्युति}


\twolineshloka
{कनकोत्तमसह्काशो ज्वलन्निव विभावसुः}
{स शान्तः पुरुषव्याघ्रः पार्थसायकवारिणा}


\twolineshloka
{यथाऽग्निर्ज्वलनो दीप्तो जलमासाद्य शाम्यति}
{कर्णाग्निः समरे तद्वत्पार्थमेघेन शामितः}


\twolineshloka
{आहृत्य स यशो दीप्तं सुयुद्धेनात्मनो भुवि}
{सपुत्रः समरे कर्णः प्रशान्तः पार्थतेजसा}


\twolineshloka
{प्रताप्य पाण्डवान्सर्वान्पाञ्चालानस्त्रतेजसा}
{नानाभरणवान्रजंस्तप्तजाम्बूनदप्रभः}


\threelineshloka
{वर्षित्वा शरवर्षाणि प्रताप्य रिपुवाहिनीम्}
{श्रीमानिव सहस्रांशुर्ज्वलन्सर्वान्प्रताप्य च}
{हतो वैकर्तनः कर्णः पादपोऽङ्कुरवानिव}


\twolineshloka
{ददानीत्येव यः प्रादान्न नास्तीत्यर्थितोऽवदत्}
{सद्ध्यः सदा सत्पुरुषः स हतो द्वैरथे वृषा}


\twolineshloka
{यस्य ब्राह्मणसात्सर्वं वित्तमासीन्महात्मनः}
{नादेयं ब्राह्मणेष्वासीद्यस्य स्वमपि जीवितम्}


\twolineshloka
{स दातॄणां प्रियो राजन्दाता चैव मनोरथान्}
{स पार्थास्त्रविनिर्दग्धो गतः परमिकां गतिम्}


\twolineshloka
{यमाश्रित्याकरोद्वैरं सुतस्ते स गतो दिवम्}
{आदाय तव पुत्राणां जयाशां सर्म वर्म च}


\twolineshloka
{हते च कर्णे सरितो न सस्रु--र्जगाम चास्तं कलुषो दिवाकरः}
{श्वेतो ग्रहश्च ज्वलितार्कवर्णोयमस्य पुत्रोऽभ्युदितः स तिर्यक्}


\twolineshloka
{नभश्चचालाथ ननाद चोर्वीववुश्च वाताः परुषाश्च घोराः}
{दिशो बभूवुर्ज्वलिताः सधूमामहार्णवाः सस्वनुश्चुक्षुभुश्च}


\twolineshloka
{सकाननाश्चाद्रिवराश्चकम्पिरेप्रविव्यथुर्भूतगणाश्च मारिष}
{बृहस्पतिः सम्परिवार्य रोहिणींबभूव चन्द्रार्कसमो विशाम्पते}


\twolineshloka
{हते कर्णे न दिशो विप्रचाराःसचन्द्रार्का द्यौर्विचचाल भूमिः}
{पपात चोल्का ज्वलनप्रभा चनिशाचरा हृष्टतरा बभूवुः}


\twolineshloka
{शशिप्रकाशाननमर्जुनो यदाजहार कर्णस्य शिरः शरेण}
{ततोऽन्तरिक्षे सहसैव शब्दोबभूव हाहेति सुरैर्विमुक्तः}


\twolineshloka
{सदेवगन्धर्वमनुष्यपूजितंनिहत्य कर्णं रिपुमाहवेऽर्जुनः}
{रराज राजन्परमेण तेजसायथा पुरा वृत्रवधे शतक्रतुः}


\twolineshloka
{पताकिना भीमनिनादकेतुनारथेन शङ्खस्फटिकावभासिना}
{महेन्द्रवाहप्रतिमेन तावुभौमहेन्द्रवीर्यप्रतिमानपौरुषौ}


\threelineshloka
{सुवर्णमुक्तामणिवज्रविद्रुमै--रलङ्कृतेनाप्रतिमेन रंहसा}
{नरोत्तमौ यादवपाण्डुनन्दनौदिवाकरौ दीप्तहुताशनाविव}
{रणाजिरे वीतभयौ विरेजतुःसमानयोगाविव विष्णुवासवैः}


\threelineshloka
{ततो धनुर्ज्यातलनेमिनिस्वनैःप्रसह्य कृत्वा च रिपून्गतप्रभान्}
{स साधयित्वा च रिपूञ्शरौघैःकपिध्वजः पक्षिवरध्वजश्च}
{प्रदध्मतुः शङ्खवरौ सुघोषौमनांस्यरीणामुपतापयन्तौ}


\twolineshloka
{सुवर्णजालावततौ महास्वनौहिमावदातौ परिगृह्य पाणिमिः}
{चुचुम्बतुः शङ्खवरौ नृणां वरौविघोषयन्तौ विजयं जगत्त्रये}


\twolineshloka
{पाञ्चजन्यस्य निर्घोषो देवदत्तस्य चोभयोः}
{पृथिवीमन्तरिक्षं च दिशश्च समपूरयत्}


\twolineshloka
{वित्रस्ताश्चाभवन्सर्वे कुरवो राजसत्तम}
{शूरयोः शङ्खशब्देन माधवस्यार्जुनस्य च}


\twolineshloka
{तौ शङ्खशब्देन निनादयन्तौवनानि शैलान्सरितो गुहाश्च}
{वित्रासयन्तौ तव पुत्रसेनांयुधिष्ठिरं नन्दयितुं प्रयातौ}


\twolineshloka
{ततः प्रजग्मुः कुरवो जवेनश्रुत्वैव शङ्खस्वनमीर्यमाणम्}
{विहाय मद्राधिपतिं पतिं चदुर्योधनं भारत भारतानाम्}


\twolineshloka
{ततो रथेनाम्बुदबृन्दनादिनाशरद्वहर्मध्यदिवाकरत्विषा}
{धनञ्जयस्याधिरथेश्च विस्मिताःप्रशंसमानाः प्रययुस्तदाऽर्जुनम्}


\twolineshloka
{महाहवे तं बहुशोभमानंधनञ्जयं योधगणैः समेताः}
{तावन्वमोदंश्च जनाः प्रकामंप्रभाकरावभ्युदितौ यथैव}


\chapter{अध्यायः १०७}
\fourlineindentedshloka
{अतः परं शल्यपर्व भविष्यति}
{तस्यायमाद्यः श्लोकः ॥जनमेजय उवाच}
{एवं निपातिते कर्णे समरे सव्यसाचिना}
{अल्पावशिष्टाः कुरवः किमकुर्वत वै द्विज}


